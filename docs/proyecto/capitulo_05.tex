\chapter{Conclusiones y Recomendaciones}

Una vez concluido el desarrollo del sistema, este paso a la etapa de
implantación y la posterior evaluación, inicialmente el sistema habiendo sido
desplegado en un servidor web provisto por el centro MEMI, fue evaluado a través
de la funcionalidad construida para este propósito, habiéndose arrojando
resultados para el control de los factores críticos del proyecto estos son
analizados.

Este capitulo compendia el tiempo que se ha establecido para una adecuada
evaluación, nos mostrará el conjunto de gráficas que han sido generadas, además
de los datos reales tabulados; posteriormente pasaremos a la conclusión misma
del proyecto, para finalizar presentando las recomendaciones finales que se han
planteado, a partir de los resultados obtenidos.

\section{Resultados}
Como se describió en la parte final del capitulo anterior, se ha desarrollado
una funcionalidad especifica capaz de generar gráficas estadísticas, además de
presentar datos tabulados, de forma que puedan apreciarse diversos indicadores
sobre variables del sistema específicos.

Una vez concluido el desarrollo del sistema, este fue implantado en un servidor
del Centro MEMI, donde con la colaboración de algunos docentes pudo ser puesta
en practica en algunos grupos del Departamento de Informática y Sistemas, los
resultados aquí presentados, se extrajeron de ese periodo  de evaluación.

Inicialmente veremos algunas variables de contexto, necesarias para la
comprensión de los datos extraídos; posteriormente pasaremos a mostrar los
resultados obtenidos tanto como datos tabulados, como gráficas generadas.

\subsection{Contexto de despliegue}
En el cuadro \ref{contexto}, puede apreciarse un resumen con los datos mas
relevantes respecto al periodo de evaluación; desde el inicio del periodo de
evaluación, hasta el final de este, pasaron casi dos periodos académicos, pero
únicamente uno de ellos fue completo.

\begin{table}
\centering
\begin{tabular}{|l|l|}
\hline
Sitio web                          & http://yachay.memi.umss.edu.bo \\
Periodo académico                  & I/2011 \\
Tiempo de evaluación               & 325 días \\
Fecha de inicio                    & 23 de Septiembre del 2010 \\
Fecha de fin                       & 14 de Agosto del 2011 \\
Lugar de evaluación                & Carrera de Informática y Sistemas \\
Caídas del servidor                & 4 \\
Tiempo del servidor fuera de linea & 2 semanas acumuladas \\
Docentes participantes             & 4 \\
Materias participantes             & 4 \\
Grupos participantes               & 8 \\
Usuarios participantes             & 542 (estudiantes de primeros semestres) \\
Espacios virtuales creados         & 33 \\
Recursos publicados                & 68 \\
\hline
\end{tabular}
\caption{Variables de contexto para el periodo de evaluación}
\label{contexto}
\end{table}

Durante el periodo II/2010, si bien el sistema estaba en funcionamiento, no se
pudo incluir a estudiantes, es así como no existió mucha producción de recursos,
ni mucha afluencia de personas al sitio web habilitado (pueden apreciarse estos
detalles en la linea de tiempo).

Durante el periodo I/2011, si existió un registro apropiado de estudiantes, y es
realmente este periodo de tiempo el que mas atención se pusieron a los datos
generados.

\subsection{Usuarios}
El primer objeto a analizar fue el grado de uso de los usuarios con respecto del
sistema, puede apreciarse en la cuadro \ref{usuarios_tabla_1} un conteo de
usuarios clasificados por rol, con respecto a ciertas actividades que se consideran claves como un reflejo del interés que se muestra hacia el sistema, las
columnas presentadas son:

\begin{table}
\centering
\begin{tabular}{l|c c c c c}
$Rol$ & $Registrados$ & $Logeados$ & $Correo$ & $Usuario$ & $Fotografia$ \\
\hline
$Invitado$      & $ 10$ & $  9$ & $10$ & $10$ & $ 1$ \\
$Estudiante$    & $517$ & $103$ & $47$ & $15$ & $11$ \\
$Auxiliar$      & $  7$ & $  6$ & $ 7$ & $ 4$ & $ 3$ \\
$Docente$       & $  3$ & $  3$ & $ 3$ & $ 3$ & $ 1$ \\
$Moderador$     & $  1$ & $  1$ & $ 1$ & $ 1$ & $ 0$ \\
$Desarrollador$ & $  3$ & $  3$ & $ 3$ & $ 3$ & $ 2$ \\
$Administrador$ & $  1$ & $  1$ & $ 1$ & $ 1$ & $ 1$ \\
\end{tabular}
\caption{Niveles de intención de los usuarios clasificados por rol}
\label{usuarios_tabla_1}
\end{table}

\begin{description}
\item [Registrados] Representa el numero real de registros de usuario que el
sistema poseen de el rol especificado.
\item [Logeados] Siendo que el sistema no posee un registro de usuarios libre,
los usuarios deben ser registrados por sus docentes respectivos, o ser invitados
por un usuario que ya posea una cuenta, este valor representa el numero de
usuarios que ingreso a su cuenta de usuario en el sistema al menos una vez.
\item [Correo] Cuando un estudiante es automáticamente registrado por su
docente, este no cuenta con un registro de correo electrónico (además que el
sistema notifica automáticamente al usuario de este detalle), este valor
representa el numero de usuarios que establecieron un correo electrónico valido
en el sistema.
\item [Usuario] Cuando un estudiante es automáticamente registrado por su
docente, este tendrá el mismo nombre de usuario que su código SIS, este valor
representa el numero de usuarios que establecieron un nombre de usuario propio.
\item [Fotografía] El sistema provee automáticamente a cada usuario de una
imagen anónima, este valor representa el numero de usuarios que establecieron
y utilizan una imagen de usuario propia.
\end{description}

Puede verse cada actividad descrita anteriormente, como un grado de afinidad
hacia el sistema, que parte inicialmente de ingresar al sistema como primer
paso, y termina con un grado de afinidad alto cuando se establece una imagen
de usuario propia.

En la figura \ref{usuarios_bars_1}, se puede apreciar el diagrama de barras de
los datos presentados en el cuadro \ref{usuarios_tabla_1}, destaca en esta la
disparidad entre el rol de estudiante y los demás roles, del conjunto de
usuarios registrados, únicamente el 20\% ingreso alguna vez al sistema.

\begin{figure}
\centering
%LaTeX with PSTricks extensions
%%Creator: inkscape 0.48.5
%%Please note this file requires PSTricks extensions
\psset{xunit=.5pt,yunit=.5pt,runit=.5pt}
\begin{pspicture}(866,429)
{
\newrgbcolor{curcolor}{0 0 0}
\pscustom[linestyle=none,fillstyle=solid,fillcolor=curcolor]
{
\newpath
\moveto(382.45475342,12.82530273)
\curveto(382.48474374,12.69530247)(382.44474378,12.59530257)(382.33475342,12.52530273)
\curveto(382.28474394,12.49530267)(382.21974401,12.47530269)(382.13975342,12.46530273)
\lineto(381.89975342,12.46530273)
\lineto(381.41975342,12.46530273)
\curveto(381.25974497,12.4653027)(381.14474508,12.50030267)(381.07475342,12.57030273)
\curveto(381.00474522,12.62030255)(380.96474526,12.69530247)(380.95475342,12.79530273)
\lineto(380.95475342,13.12530273)
\lineto(380.95475342,13.23030273)
\curveto(380.96474526,13.2703019)(380.97474525,13.30530186)(380.98475342,13.33530273)
\curveto(380.97474525,13.38530178)(380.97974525,13.43030174)(380.99975342,13.47030273)
\curveto(381.01974521,13.51030166)(381.0247452,13.55030162)(381.01475342,13.59030273)
\lineto(381.04475342,13.77030273)
\lineto(381.07475342,13.95030273)
\lineto(381.16475342,14.62530273)
\curveto(381.16474506,14.69530047)(381.16974506,14.7653004)(381.17975342,14.83530273)
\curveto(381.18974504,14.90530026)(381.19474503,14.98030019)(381.19475342,15.06030273)
\curveto(381.18474504,15.24029993)(381.18474504,15.42029975)(381.19475342,15.60030273)
\curveto(381.20474502,15.78029939)(381.18974504,15.95029922)(381.14975342,16.11030273)
\curveto(381.03974519,16.53029864)(380.77974545,16.81029836)(380.36975342,16.95030273)
\curveto(380.24974598,17.00029817)(380.10974612,17.02529814)(379.94975342,17.02530273)
\curveto(379.79974643,17.03529813)(379.63974659,17.04029813)(379.46975342,17.04030273)
\lineto(376.70975342,17.04030273)
\curveto(376.63974959,17.02029815)(376.57474965,17.00029817)(376.51475342,16.98030273)
\curveto(376.45474977,16.9702982)(376.39974983,16.94029823)(376.34975342,16.89030273)
\curveto(376.25974997,16.79029838)(376.19475003,16.62529854)(376.15475342,16.39530273)
\curveto(376.11475011,16.17529899)(376.07975015,15.98029919)(376.04975342,15.81030273)
\lineto(375.61475342,13.63530273)
\curveto(375.58475064,13.49530167)(375.55475067,13.32030185)(375.52475342,13.11030273)
\curveto(375.49475073,12.91030226)(375.44475078,12.76030241)(375.37475342,12.66030273)
\curveto(375.34475088,12.59030258)(375.29475093,12.54530262)(375.22475342,12.52530273)
\curveto(375.18475104,12.50530266)(375.14475108,12.49530267)(375.10475342,12.49530273)
\curveto(375.07475115,12.49530267)(375.0297512,12.48530268)(374.96975342,12.46530273)
\curveto(374.9297513,12.45530271)(374.88475134,12.45030272)(374.83475342,12.45030273)
\curveto(374.78475144,12.46030271)(374.73475149,12.4653027)(374.68475342,12.46530273)
\lineto(374.36975342,12.46530273)
\curveto(374.26975196,12.47530269)(374.18975204,12.50530266)(374.12975342,12.55530273)
\curveto(374.05975217,12.60530256)(374.03475219,12.69530247)(374.05475342,12.82530273)
\curveto(374.08475214,12.9653022)(374.11475211,13.10030207)(374.14475342,13.23030273)
\lineto(375.95975342,22.35030273)
\curveto(375.97975025,22.46029271)(375.99975023,22.57529259)(376.01975342,22.69530273)
\curveto(376.03975019,22.81529235)(376.08475014,22.91029226)(376.15475342,22.98030273)
\curveto(376.20475002,23.04029213)(376.28974994,23.09029208)(376.40975342,23.13030273)
\curveto(376.4297498,23.14029203)(376.44974978,23.14029203)(376.46975342,23.13030273)
\curveto(376.48974974,23.13029204)(376.50974972,23.13529203)(376.52975342,23.14530273)
\lineto(380.87975342,23.14530273)
\curveto(380.94974528,23.14529202)(381.0247452,23.14529202)(381.10475342,23.14530273)
\curveto(381.18474504,23.15529201)(381.25474497,23.15529201)(381.31475342,23.14530273)
\lineto(381.47975342,23.14530273)
\curveto(381.53974469,23.13529203)(381.59474463,23.12529204)(381.64475342,23.11530273)
\curveto(381.70474452,23.11529205)(381.76974446,23.11029206)(381.83975342,23.10030273)
\curveto(381.91974431,23.08029209)(381.99974423,23.0652921)(382.07975342,23.05530273)
\curveto(382.15974407,23.04529212)(382.23974399,23.03029214)(382.31975342,23.01030273)
\curveto(382.49974373,22.95029222)(382.65974357,22.88529228)(382.79975342,22.81530273)
\curveto(382.94974328,22.74529242)(383.08474314,22.66029251)(383.20475342,22.56030273)
\curveto(383.4247428,22.39029278)(383.58474264,22.18029299)(383.68475342,21.93030273)
\curveto(383.79474243,21.69029348)(383.86474236,21.40529376)(383.89475342,21.07530273)
\curveto(383.90474232,20.99529417)(383.89974233,20.91029426)(383.87975342,20.82030273)
\curveto(383.86974236,20.74029443)(383.86974236,20.66029451)(383.87975342,20.58030273)
\lineto(383.84975342,20.43030273)
\curveto(383.84974238,20.38029479)(383.83974239,20.32029485)(383.81975342,20.25030273)
\curveto(383.79974243,20.19029498)(383.77974245,20.13529503)(383.75975342,20.08530273)
\lineto(383.72975342,19.92030273)
\curveto(383.68974254,19.84029533)(383.65974257,19.7652954)(383.63975342,19.69530273)
\curveto(383.61974261,19.62529554)(383.59474263,19.55529561)(383.56475342,19.48530273)
\curveto(383.48474274,19.33529583)(383.40974282,19.19029598)(383.33975342,19.05030273)
\curveto(383.26974296,18.92029625)(383.17974305,18.79529637)(383.06975342,18.67530273)
\curveto(383.0297432,18.62529654)(382.98974324,18.58029659)(382.94975342,18.54030273)
\curveto(382.90974332,18.50029667)(382.86974336,18.45529671)(382.82975342,18.40530273)
\curveto(382.80974342,18.39529677)(382.79474343,18.38529678)(382.78475342,18.37530273)
\curveto(382.77474345,18.37529679)(382.76474346,18.3702968)(382.75475342,18.36030273)
\curveto(382.73474349,18.34029683)(382.70474352,18.31529685)(382.66475342,18.28530273)
\lineto(382.58975342,18.21030273)
\curveto(382.49974373,18.15029702)(382.41474381,18.09029708)(382.33475342,18.03030273)
\curveto(382.25474397,17.98029719)(382.16974406,17.93029724)(382.07975342,17.88030273)
\curveto(382.01974421,17.85029732)(381.96474426,17.81529735)(381.91475342,17.77530273)
\curveto(381.87474435,17.74529742)(381.83974439,17.70029747)(381.80975342,17.64030273)
\curveto(381.78974444,17.58029759)(381.80474442,17.53029764)(381.85475342,17.49030273)
\curveto(381.90474432,17.45029772)(381.94474428,17.42029775)(381.97475342,17.40030273)
\curveto(382.07474415,17.33029784)(382.16474406,17.25529791)(382.24475342,17.17530273)
\curveto(382.3247439,17.09529807)(382.38474384,17.00029817)(382.42475342,16.89030273)
\curveto(382.51474371,16.75029842)(382.56474366,16.59029858)(382.57475342,16.41030273)
\curveto(382.59474363,16.24029893)(382.60974362,16.05529911)(382.61975342,15.85530273)
\lineto(382.58975342,15.61530273)
\curveto(382.58974364,15.54529962)(382.58474364,15.4702997)(382.57475342,15.39030273)
\curveto(382.58474364,15.32029985)(382.57474365,15.25029992)(382.54475342,15.18030273)
\curveto(382.5247437,15.11030006)(382.51974371,15.04030013)(382.52975342,14.97030273)
\lineto(382.49975342,14.83530273)
\curveto(382.50974372,14.7653004)(382.49974373,14.69030048)(382.46975342,14.61030273)
\curveto(382.43974379,14.53030064)(382.4297438,14.45030072)(382.43975342,14.37030273)
\curveto(382.43974379,14.33030084)(382.43474379,14.29030088)(382.42475342,14.25030273)
\curveto(382.41474381,14.22030095)(382.40974382,14.18030099)(382.40975342,14.13030273)
\curveto(382.40974382,14.03030114)(382.39974383,13.92530124)(382.37975342,13.81530273)
\curveto(382.36974386,13.71530145)(382.37474385,13.62030155)(382.39475342,13.53030273)
\curveto(382.39474383,13.4703017)(382.38974384,13.41030176)(382.37975342,13.35030273)
\curveto(382.37974385,13.30030187)(382.38474384,13.24530192)(382.39475342,13.18530273)
\lineto(382.45475342,12.82530273)
\moveto(381.88475342,19.05030273)
\curveto(381.97474425,19.16029601)(382.04974418,19.27529589)(382.10975342,19.39530273)
\curveto(382.16974406,19.51529565)(382.229744,19.64529552)(382.28975342,19.78530273)
\lineto(382.31975342,19.92030273)
\curveto(382.37974385,20.06029511)(382.41474381,20.21029496)(382.42475342,20.37030273)
\curveto(382.43474379,20.54029463)(382.4297438,20.68029449)(382.40975342,20.79030273)
\curveto(382.34974388,21.29029388)(382.10474412,21.63529353)(381.67475342,21.82530273)
\curveto(381.49474473,21.90529326)(381.26974496,21.95029322)(380.99975342,21.96030273)
\curveto(380.73974549,21.9702932)(380.46974576,21.97529319)(380.18975342,21.97530273)
\lineto(377.71475342,21.97530273)
\curveto(377.69474853,21.9652932)(377.66974856,21.96029321)(377.63975342,21.96030273)
\curveto(377.61974861,21.96029321)(377.59474863,21.95529321)(377.56475342,21.94530273)
\curveto(377.43474879,21.91529325)(377.33974889,21.85029332)(377.27975342,21.75030273)
\curveto(377.21974901,21.66029351)(377.17474905,21.53529363)(377.14475342,21.37530273)
\curveto(377.1247491,21.21529395)(377.09974913,21.0702941)(377.06975342,20.94030273)
\lineto(376.72475342,19.21530273)
\curveto(376.69474953,19.0652961)(376.65974957,18.90529626)(376.61975342,18.73530273)
\curveto(376.58974964,18.57529659)(376.59974963,18.45029672)(376.64975342,18.36030273)
\curveto(376.68974954,18.29029688)(376.75474947,18.24529692)(376.84475342,18.22530273)
\curveto(376.94474928,18.21529695)(377.05474917,18.21029696)(377.17475342,18.21030273)
\lineto(378.10475342,18.21030273)
\curveto(378.49474773,18.21029696)(378.87474735,18.20529696)(379.24475342,18.19530273)
\curveto(379.61474661,18.19529697)(379.95974627,18.21529695)(380.27975342,18.25530273)
\curveto(380.60974562,18.30529686)(380.90974532,18.39029678)(381.17975342,18.51030273)
\curveto(381.44974478,18.63029654)(381.68474454,18.81029636)(381.88475342,19.05030273)
}
}
{
\newrgbcolor{curcolor}{0 0 0}
\pscustom[linestyle=none,fillstyle=solid,fillcolor=curcolor]
{
\newpath
\moveto(391.97295654,16.65030273)
\curveto(391.98294765,16.59029858)(391.97294766,16.49529867)(391.94295654,16.36530273)
\curveto(391.92294771,16.24529892)(391.90294773,16.16029901)(391.88295654,16.11030273)
\lineto(391.85295654,15.96030273)
\curveto(391.82294781,15.88029929)(391.79794784,15.80529936)(391.77795654,15.73530273)
\curveto(391.76794787,15.67529949)(391.74794789,15.60529956)(391.71795654,15.52530273)
\curveto(391.68794795,15.4652997)(391.66294797,15.40529976)(391.64295654,15.34530273)
\curveto(391.632948,15.28529988)(391.60794803,15.22529994)(391.56795654,15.16530273)
\lineto(391.38795654,14.77530273)
\curveto(391.3379483,14.64530052)(391.27294836,14.52530064)(391.19295654,14.41530273)
\curveto(390.89294874,13.93530123)(390.5329491,13.53030164)(390.11295654,13.20030273)
\curveto(389.70294993,12.88030229)(389.22295041,12.63530253)(388.67295654,12.46530273)
\curveto(388.56295107,12.42530274)(388.44295119,12.39530277)(388.31295654,12.37530273)
\curveto(388.18295145,12.35530281)(388.04795159,12.33530283)(387.90795654,12.31530273)
\curveto(387.84795179,12.30530286)(387.78295185,12.30030287)(387.71295654,12.30030273)
\curveto(387.65295198,12.29030288)(387.59295204,12.28530288)(387.53295654,12.28530273)
\curveto(387.49295214,12.27530289)(387.4329522,12.2703029)(387.35295654,12.27030273)
\curveto(387.28295235,12.2703029)(387.2329524,12.27530289)(387.20295654,12.28530273)
\curveto(387.16295247,12.29530287)(387.12295251,12.30030287)(387.08295654,12.30030273)
\curveto(387.04295259,12.29030288)(387.00795263,12.29030288)(386.97795654,12.30030273)
\lineto(386.88795654,12.30030273)
\lineto(386.54295654,12.34530273)
\lineto(386.15295654,12.46530273)
\curveto(386.0329536,12.50530266)(385.91795372,12.55030262)(385.80795654,12.60030273)
\curveto(385.39795424,12.80030237)(385.07795456,13.06030211)(384.84795654,13.38030273)
\curveto(384.62795501,13.70030147)(384.46795517,14.09030108)(384.36795654,14.55030273)
\curveto(384.3379553,14.65030052)(384.31795532,14.75030042)(384.30795654,14.85030273)
\lineto(384.30795654,15.16530273)
\curveto(384.29795534,15.20529996)(384.29795534,15.23529993)(384.30795654,15.25530273)
\curveto(384.31795532,15.28529988)(384.32295531,15.32029985)(384.32295654,15.36030273)
\curveto(384.32295531,15.44029973)(384.32795531,15.52029965)(384.33795654,15.60030273)
\curveto(384.34795529,15.69029948)(384.35295528,15.77529939)(384.35295654,15.85530273)
\curveto(384.36295527,15.90529926)(384.36795527,15.94529922)(384.36795654,15.97530273)
\curveto(384.37795526,16.01529915)(384.38295525,16.06029911)(384.38295654,16.11030273)
\curveto(384.38295525,16.16029901)(384.39295524,16.24529892)(384.41295654,16.36530273)
\curveto(384.44295519,16.49529867)(384.47295516,16.59029858)(384.50295654,16.65030273)
\curveto(384.54295509,16.72029845)(384.56295507,16.79029838)(384.56295654,16.86030273)
\curveto(384.56295507,16.93029824)(384.58295505,17.00029817)(384.62295654,17.07030273)
\curveto(384.64295499,17.12029805)(384.65795498,17.16029801)(384.66795654,17.19030273)
\curveto(384.67795496,17.23029794)(384.69295494,17.27529789)(384.71295654,17.32530273)
\curveto(384.77295486,17.44529772)(384.82295481,17.5652976)(384.86295654,17.68530273)
\curveto(384.91295472,17.80529736)(384.97795466,17.92029725)(385.05795654,18.03030273)
\curveto(385.27795436,18.40029677)(385.52295411,18.73029644)(385.79295654,19.02030273)
\curveto(386.07295356,19.32029585)(386.38795325,19.5702956)(386.73795654,19.77030273)
\curveto(386.86795277,19.85029532)(387.00295263,19.91529525)(387.14295654,19.96530273)
\lineto(387.59295654,20.14530273)
\curveto(387.72295191,20.19529497)(387.85795178,20.22529494)(387.99795654,20.23530273)
\curveto(388.1379515,20.25529491)(388.28295135,20.28529488)(388.43295654,20.32530273)
\lineto(388.62795654,20.32530273)
\lineto(388.83795654,20.35530273)
\curveto(389.72794991,20.3652948)(390.42794921,20.18029499)(390.93795654,19.80030273)
\curveto(391.45794818,19.42029575)(391.78294785,18.92529624)(391.91295654,18.31530273)
\curveto(391.94294769,18.21529695)(391.96294767,18.11529705)(391.97295654,18.01530273)
\curveto(391.98294765,17.91529725)(391.99794764,17.81029736)(392.01795654,17.70030273)
\curveto(392.02794761,17.59029758)(392.02794761,17.4702977)(392.01795654,17.34030273)
\lineto(392.01795654,16.96530273)
\curveto(392.01794762,16.91529825)(392.00794763,16.86029831)(391.98795654,16.80030273)
\curveto(391.97794766,16.75029842)(391.97294766,16.70029847)(391.97295654,16.65030273)
\moveto(390.47295654,15.79530273)
\curveto(390.50294913,15.8652993)(390.52294911,15.94529922)(390.53295654,16.03530273)
\curveto(390.55294908,16.12529904)(390.56794907,16.21029896)(390.57795654,16.29030273)
\curveto(390.65794898,16.68029849)(390.69294894,17.01029816)(390.68295654,17.28030273)
\curveto(390.66294897,17.36029781)(390.64794899,17.44029773)(390.63795654,17.52030273)
\curveto(390.637949,17.60029757)(390.632949,17.67529749)(390.62295654,17.74530273)
\curveto(390.47294916,18.39529677)(390.11794952,18.84529632)(389.55795654,19.09530273)
\curveto(389.48795015,19.12529604)(389.41295022,19.14529602)(389.33295654,19.15530273)
\curveto(389.26295037,19.17529599)(389.18795045,19.19529597)(389.10795654,19.21530273)
\curveto(389.0379506,19.23529593)(388.95795068,19.24529592)(388.86795654,19.24530273)
\lineto(388.59795654,19.24530273)
\lineto(388.31295654,19.20030273)
\curveto(388.21295142,19.18029599)(388.11795152,19.15529601)(388.02795654,19.12530273)
\curveto(387.9379517,19.10529606)(387.84795179,19.07529609)(387.75795654,19.03530273)
\curveto(387.68795195,19.01529615)(387.61795202,18.98529618)(387.54795654,18.94530273)
\curveto(387.47795216,18.90529626)(387.41295222,18.8652963)(387.35295654,18.82530273)
\curveto(387.08295255,18.65529651)(386.84795279,18.45029672)(386.64795654,18.21030273)
\curveto(386.44795319,17.9702972)(386.26295337,17.69029748)(386.09295654,17.37030273)
\curveto(386.04295359,17.2702979)(386.00295363,17.165298)(385.97295654,17.05530273)
\curveto(385.94295369,16.95529821)(385.90295373,16.85029832)(385.85295654,16.74030273)
\curveto(385.84295379,16.70029847)(385.82795381,16.63529853)(385.80795654,16.54530273)
\curveto(385.78795385,16.51529865)(385.77795386,16.48029869)(385.77795654,16.44030273)
\curveto(385.77795386,16.40029877)(385.77295386,16.35529881)(385.76295654,16.30530273)
\lineto(385.70295654,16.00530273)
\curveto(385.68295395,15.90529926)(385.67295396,15.81529935)(385.67295654,15.73530273)
\lineto(385.67295654,15.55530273)
\curveto(385.67295396,15.45529971)(385.66795397,15.35529981)(385.65795654,15.25530273)
\curveto(385.65795398,15.1653)(385.66795397,15.08030009)(385.68795654,15.00030273)
\curveto(385.7379539,14.76030041)(385.80795383,14.53530063)(385.89795654,14.32530273)
\curveto(385.99795364,14.11530105)(386.1329535,13.94030123)(386.30295654,13.80030273)
\curveto(386.35295328,13.7703014)(386.39295324,13.74530142)(386.42295654,13.72530273)
\curveto(386.46295317,13.70530146)(386.50295313,13.68030149)(386.54295654,13.65030273)
\curveto(386.61295302,13.60030157)(386.69295294,13.55530161)(386.78295654,13.51530273)
\curveto(386.87295276,13.48530168)(386.96795267,13.45530171)(387.06795654,13.42530273)
\curveto(387.11795252,13.40530176)(387.16295247,13.39530177)(387.20295654,13.39530273)
\curveto(387.25295238,13.40530176)(387.30295233,13.40530176)(387.35295654,13.39530273)
\curveto(387.38295225,13.38530178)(387.44295219,13.37530179)(387.53295654,13.36530273)
\curveto(387.62295201,13.35530181)(387.69795194,13.36030181)(387.75795654,13.38030273)
\curveto(387.79795184,13.39030178)(387.8379518,13.39030178)(387.87795654,13.38030273)
\curveto(387.91795172,13.38030179)(387.95795168,13.39030178)(387.99795654,13.41030273)
\curveto(388.07795156,13.43030174)(388.15795148,13.44530172)(388.23795654,13.45530273)
\curveto(388.32795131,13.47530169)(388.41295122,13.50030167)(388.49295654,13.53030273)
\curveto(388.85295078,13.6703015)(389.16295047,13.8653013)(389.42295654,14.11530273)
\curveto(389.68294995,14.3653008)(389.91794972,14.66030051)(390.12795654,15.00030273)
\curveto(390.20794943,15.12030005)(390.26794937,15.24529992)(390.30795654,15.37530273)
\curveto(390.34794929,15.51529965)(390.40294923,15.65529951)(390.47295654,15.79530273)
}
}
{
\newrgbcolor{curcolor}{0 0 0}
\pscustom[linestyle=none,fillstyle=solid,fillcolor=curcolor]
{
\newpath
\moveto(395.33623779,23.13030273)
\curveto(395.46623403,23.13029204)(395.6012339,23.13029204)(395.74123779,23.13030273)
\curveto(395.89123361,23.13029204)(395.99123351,23.09529207)(396.04123779,23.02530273)
\curveto(396.08123342,22.95529221)(396.09123341,22.86029231)(396.07123779,22.74030273)
\curveto(396.05123345,22.63029254)(396.03123347,22.51529265)(396.01123779,22.39530273)
\lineto(395.74123779,21.06030273)
\lineto(394.52623779,14.98530273)
\lineto(394.19623779,13.30530273)
\curveto(394.16623533,13.18530198)(394.13623536,13.05530211)(394.10623779,12.91530273)
\curveto(394.08623541,12.77530239)(394.04123546,12.6653025)(393.97123779,12.58530273)
\curveto(393.93123557,12.53530263)(393.88123562,12.50530266)(393.82123779,12.49530273)
\curveto(393.77123573,12.48530268)(393.7012358,12.4703027)(393.61123779,12.45030273)
\lineto(393.40123779,12.45030273)
\lineto(393.08623779,12.45030273)
\curveto(392.98623651,12.46030271)(392.92123658,12.49530267)(392.89123779,12.55530273)
\curveto(392.85123665,12.63530253)(392.84123666,12.73530243)(392.86123779,12.85530273)
\curveto(392.88123662,12.97530219)(392.90623659,13.10030207)(392.93623779,13.23030273)
\lineto(393.20623779,14.61030273)
\lineto(394.45123779,20.85030273)
\lineto(394.75123779,22.32030273)
\curveto(394.77123473,22.43029274)(394.79123471,22.54529262)(394.81123779,22.66530273)
\curveto(394.83123467,22.79529237)(394.87123463,22.89529227)(394.93123779,22.96530273)
\curveto(394.99123451,23.02529214)(395.07623442,23.07529209)(395.18623779,23.11530273)
\curveto(395.21623428,23.12529204)(395.24123426,23.12529204)(395.26123779,23.11530273)
\curveto(395.28123422,23.11529205)(395.30623419,23.12029205)(395.33623779,23.13030273)
}
}
{
\newrgbcolor{curcolor}{0 0 0}
\pscustom[linestyle=none,fillstyle=solid,fillcolor=curcolor]
{
\newpath
\moveto(403.55108154,16.62030273)
\curveto(403.55107304,16.52029865)(403.53107306,16.40529876)(403.49108154,16.27530273)
\curveto(403.45107314,16.15529901)(403.40107319,16.0702991)(403.34108154,16.02030273)
\curveto(403.28107331,15.98029919)(403.20107339,15.95029922)(403.10108154,15.93030273)
\curveto(403.00107359,15.92029925)(402.8910737,15.91529925)(402.77108154,15.91530273)
\lineto(402.41108154,15.91530273)
\curveto(402.30107429,15.92529924)(402.20107439,15.93029924)(402.11108154,15.93030273)
\lineto(398.27108154,15.93030273)
\curveto(398.1910784,15.93029924)(398.10607848,15.92529924)(398.01608154,15.91530273)
\curveto(397.93607865,15.91529925)(397.87107872,15.90029927)(397.82108154,15.87030273)
\curveto(397.77107882,15.85029932)(397.72107887,15.81029936)(397.67108154,15.75030273)
\lineto(397.58108154,15.61530273)
\curveto(397.55107904,15.5652996)(397.54107905,15.51529965)(397.55108154,15.46530273)
\curveto(397.55107904,15.41529975)(397.54607904,15.3702998)(397.53608154,15.33030273)
\lineto(397.53608154,15.21030273)
\lineto(397.53608154,14.95530273)
\curveto(397.54607904,14.87530029)(397.56107903,14.79530037)(397.58108154,14.71530273)
\curveto(397.71107888,14.17530099)(398.01607857,13.79030138)(398.49608154,13.56030273)
\curveto(398.54607804,13.53030164)(398.60607798,13.50530166)(398.67608154,13.48530273)
\curveto(398.74607784,13.4653017)(398.81107778,13.44530172)(398.87108154,13.42530273)
\curveto(398.90107769,13.41530175)(398.95107764,13.41030176)(399.02108154,13.41030273)
\curveto(399.15107744,13.3703018)(399.33107726,13.35030182)(399.56108154,13.35030273)
\curveto(399.7910768,13.35030182)(399.98107661,13.3703018)(400.13108154,13.41030273)
\curveto(400.28107631,13.45030172)(400.41607617,13.49030168)(400.53608154,13.53030273)
\curveto(400.66607592,13.58030159)(400.7860758,13.64030153)(400.89608154,13.71030273)
\curveto(401.01607557,13.78030139)(401.12607546,13.86030131)(401.22608154,13.95030273)
\curveto(401.32607526,14.05030112)(401.41607517,14.15530101)(401.49608154,14.26530273)
\curveto(401.57607501,14.3653008)(401.65107494,14.4703007)(401.72108154,14.58030273)
\curveto(401.7910748,14.69030048)(401.8860747,14.7703004)(402.00608154,14.82030273)
\curveto(402.04607454,14.84030033)(402.11107448,14.85530031)(402.20108154,14.86530273)
\curveto(402.30107429,14.87530029)(402.3910742,14.87530029)(402.47108154,14.86530273)
\curveto(402.56107403,14.8653003)(402.64607394,14.86030031)(402.72608154,14.85030273)
\curveto(402.80607378,14.84030033)(402.85607373,14.82030035)(402.87608154,14.79030273)
\curveto(402.96607362,14.72030045)(402.97107362,14.60530056)(402.89108154,14.44530273)
\curveto(402.75107384,14.17530099)(402.59607399,13.93530123)(402.42608154,13.72530273)
\curveto(402.16607442,13.40530176)(401.8860747,13.14030203)(401.58608154,12.93030273)
\curveto(401.29607529,12.73030244)(400.94107565,12.5653026)(400.52108154,12.43530273)
\curveto(400.41107618,12.39530277)(400.30607628,12.3703028)(400.20608154,12.36030273)
\curveto(400.10607648,12.34030283)(399.99607659,12.32030285)(399.87608154,12.30030273)
\curveto(399.82607676,12.29030288)(399.77607681,12.28530288)(399.72608154,12.28530273)
\curveto(399.6860769,12.28530288)(399.64107695,12.28030289)(399.59108154,12.27030273)
\lineto(399.44108154,12.27030273)
\curveto(399.3910772,12.26030291)(399.33107726,12.25530291)(399.26108154,12.25530273)
\curveto(399.20107739,12.25530291)(399.15107744,12.26030291)(399.11108154,12.27030273)
\lineto(398.97608154,12.27030273)
\curveto(398.92607766,12.28030289)(398.88107771,12.28530288)(398.84108154,12.28530273)
\curveto(398.80107779,12.28530288)(398.76107783,12.29030288)(398.72108154,12.30030273)
\curveto(398.67107792,12.31030286)(398.61607797,12.32030285)(398.55608154,12.33030273)
\curveto(398.50607808,12.33030284)(398.45607813,12.33530283)(398.40608154,12.34530273)
\curveto(398.31607827,12.3653028)(398.22607836,12.39030278)(398.13608154,12.42030273)
\curveto(398.05607853,12.44030273)(397.98107861,12.4653027)(397.91108154,12.49530273)
\curveto(397.87107872,12.51530265)(397.83607875,12.52530264)(397.80608154,12.52530273)
\curveto(397.77607881,12.53530263)(397.74607884,12.55030262)(397.71608154,12.57030273)
\curveto(397.57607901,12.64030253)(397.43107916,12.72530244)(397.28108154,12.82530273)
\curveto(397.03107956,13.01530215)(396.83107976,13.24530192)(396.68108154,13.51530273)
\curveto(396.53108006,13.79530137)(396.42108017,14.10530106)(396.35108154,14.44530273)
\curveto(396.32108027,14.55530061)(396.30608028,14.6703005)(396.30608154,14.79030273)
\curveto(396.30608028,14.91030026)(396.29608029,15.03030014)(396.27608154,15.15030273)
\lineto(396.27608154,15.25530273)
\curveto(396.2860803,15.28529988)(396.2910803,15.32529984)(396.29108154,15.37530273)
\lineto(396.29108154,15.63030273)
\curveto(396.30108029,15.72029945)(396.30608028,15.81029936)(396.30608154,15.90030273)
\lineto(396.35108154,16.11030273)
\curveto(396.35108024,16.15029902)(396.35608023,16.20529896)(396.36608154,16.27530273)
\curveto(396.37608021,16.35529881)(396.3910802,16.42029875)(396.41108154,16.47030273)
\lineto(396.44108154,16.63530273)
\curveto(396.47108012,16.68529848)(396.4860801,16.73529843)(396.48608154,16.78530273)
\curveto(396.49608009,16.84529832)(396.51108008,16.90029827)(396.53108154,16.95030273)
\curveto(396.60107999,17.11029806)(396.66607992,17.2702979)(396.72608154,17.43030273)
\curveto(396.7860798,17.59029758)(396.86107973,17.74029743)(396.95108154,17.88030273)
\curveto(397.02107957,17.99029718)(397.0860795,18.10029707)(397.14608154,18.21030273)
\curveto(397.21607937,18.33029684)(397.29607929,18.44529672)(397.38608154,18.55530273)
\curveto(397.67607891,18.90529626)(397.9860786,19.20529596)(398.31608154,19.45530273)
\curveto(398.64607794,19.71529545)(399.03107756,19.93029524)(399.47108154,20.10030273)
\curveto(399.60107699,20.15029502)(399.73107686,20.18529498)(399.86108154,20.20530273)
\curveto(399.9910766,20.23529493)(400.13107646,20.2652949)(400.28108154,20.29530273)
\curveto(400.33107626,20.30529486)(400.37607621,20.31029486)(400.41608154,20.31030273)
\curveto(400.45607613,20.32029485)(400.50107609,20.32529484)(400.55108154,20.32530273)
\curveto(400.57107602,20.33529483)(400.59607599,20.33529483)(400.62608154,20.32530273)
\curveto(400.65607593,20.31529485)(400.68107591,20.32029485)(400.70108154,20.34030273)
\curveto(401.13107546,20.35029482)(401.4910751,20.30529486)(401.78108154,20.20530273)
\curveto(402.07107452,20.11529505)(402.32607426,19.99029518)(402.54608154,19.83030273)
\curveto(402.586074,19.81029536)(402.61607397,19.78029539)(402.63608154,19.74030273)
\curveto(402.66607392,19.71029546)(402.69607389,19.68529548)(402.72608154,19.66530273)
\curveto(402.79607379,19.60529556)(402.86607372,19.53529563)(402.93608154,19.45530273)
\curveto(403.00607358,19.37529579)(403.06107353,19.29529587)(403.10108154,19.21530273)
\curveto(403.22107337,19.00529616)(403.31607327,18.80529636)(403.38608154,18.61530273)
\curveto(403.43607315,18.50529666)(403.46607312,18.38529678)(403.47608154,18.25530273)
\lineto(403.53608154,17.86530273)
\curveto(403.56607302,17.73529743)(403.57607301,17.60029757)(403.56608154,17.46030273)
\curveto(403.56607302,17.32029785)(403.57107302,17.18029799)(403.58108154,17.04030273)
\curveto(403.58107301,16.9702982)(403.57607301,16.90029827)(403.56608154,16.83030273)
\curveto(403.55607303,16.76029841)(403.55107304,16.69029848)(403.55108154,16.62030273)
\moveto(402.20108154,17.13030273)
\curveto(402.23107436,17.170298)(402.26107433,17.22029795)(402.29108154,17.28030273)
\curveto(402.33107426,17.35029782)(402.34607424,17.42029775)(402.33608154,17.49030273)
\curveto(402.32607426,17.71029746)(402.2860743,17.91529725)(402.21608154,18.10530273)
\curveto(402.11607447,18.33529683)(401.99607459,18.53029664)(401.85608154,18.69030273)
\curveto(401.72607486,18.85029632)(401.53607505,18.98529618)(401.28608154,19.09530273)
\curveto(401.21607537,19.11529605)(401.14607544,19.13029604)(401.07608154,19.14030273)
\curveto(401.01607557,19.16029601)(400.94607564,19.18029599)(400.86608154,19.20030273)
\curveto(400.79607579,19.22029595)(400.71607587,19.23029594)(400.62608154,19.23030273)
\lineto(400.37108154,19.23030273)
\curveto(400.33107626,19.21029596)(400.2910763,19.20029597)(400.25108154,19.20030273)
\curveto(400.21107638,19.21029596)(400.17607641,19.21029596)(400.14608154,19.20030273)
\lineto(399.90608154,19.14030273)
\curveto(399.82607676,19.13029604)(399.75107684,19.11529605)(399.68108154,19.09530273)
\curveto(399.36107723,18.97529619)(399.09607749,18.82529634)(398.88608154,18.64530273)
\curveto(398.67607791,18.4652967)(398.47607811,18.24029693)(398.28608154,17.97030273)
\curveto(398.24607834,17.92029725)(398.20107839,17.85529731)(398.15108154,17.77530273)
\curveto(398.11107848,17.70529746)(398.07107852,17.62529754)(398.03108154,17.53530273)
\curveto(397.9910786,17.44529772)(397.96607862,17.36029781)(397.95608154,17.28030273)
\curveto(397.95607863,17.20029797)(397.98107861,17.14029803)(398.03108154,17.10030273)
\curveto(398.10107849,17.04029813)(398.23107836,17.01029816)(398.42108154,17.01030273)
\curveto(398.62107797,17.02029815)(398.7910778,17.02529814)(398.93108154,17.02530273)
\lineto(401.21108154,17.02530273)
\curveto(401.36107523,17.02529814)(401.54107505,17.02029815)(401.75108154,17.01030273)
\curveto(401.96107463,17.01029816)(402.11107448,17.05029812)(402.20108154,17.13030273)
}
}
{
\newrgbcolor{curcolor}{0 0 0}
\pscustom[linestyle=none,fillstyle=solid,fillcolor=curcolor]
{
\newpath
\moveto(408.03772217,20.35530273)
\curveto(408.75771651,20.3652948)(409.34271593,20.28029489)(409.79272217,20.10030273)
\curveto(410.25271502,19.93029524)(410.5727147,19.62529554)(410.75272217,19.18530273)
\curveto(410.80271447,19.07529609)(410.83271444,18.96029621)(410.84272217,18.84030273)
\curveto(410.86271441,18.73029644)(410.87771439,18.60529656)(410.88772217,18.46530273)
\curveto(410.89771437,18.39529677)(410.88771438,18.32029685)(410.85772217,18.24030273)
\curveto(410.83771443,18.170297)(410.81271446,18.11529705)(410.78272217,18.07530273)
\curveto(410.76271451,18.05529711)(410.73271454,18.03529713)(410.69272217,18.01530273)
\curveto(410.66271461,18.00529716)(410.63771463,17.99029718)(410.61772217,17.97030273)
\curveto(410.55771471,17.95029722)(410.50271477,17.94529722)(410.45272217,17.95530273)
\curveto(410.41271486,17.9652972)(410.3677149,17.9652972)(410.31772217,17.95530273)
\curveto(410.22771504,17.93529723)(410.11771515,17.93029724)(409.98772217,17.94030273)
\curveto(409.8677154,17.96029721)(409.78271549,17.98529718)(409.73272217,18.01530273)
\curveto(409.66271561,18.0652971)(409.62271565,18.13029704)(409.61272217,18.21030273)
\curveto(409.61271566,18.30029687)(409.59271568,18.38529678)(409.55272217,18.46530273)
\curveto(409.50271577,18.62529654)(409.40771586,18.7702964)(409.26772217,18.90030273)
\curveto(409.17771609,18.98029619)(409.0677162,19.04029613)(408.93772217,19.08030273)
\curveto(408.81771645,19.12029605)(408.68771658,19.16029601)(408.54772217,19.20030273)
\curveto(408.50771676,19.22029595)(408.45771681,19.22529594)(408.39772217,19.21530273)
\curveto(408.34771692,19.21529595)(408.30271697,19.22029595)(408.26272217,19.23030273)
\curveto(408.20271707,19.25029592)(408.12771714,19.26029591)(408.03772217,19.26030273)
\curveto(407.94771732,19.26029591)(407.8727174,19.25029592)(407.81272217,19.23030273)
\lineto(407.72272217,19.23030273)
\curveto(407.66271761,19.22029595)(407.60771766,19.21029596)(407.55772217,19.20030273)
\curveto(407.50771776,19.20029597)(407.45771781,19.19529597)(407.40772217,19.18530273)
\curveto(407.13771813,19.12529604)(406.90271837,19.04029613)(406.70272217,18.93030273)
\curveto(406.51271876,18.82029635)(406.36271891,18.63529653)(406.25272217,18.37530273)
\curveto(406.22271905,18.30529686)(406.20771906,18.23529693)(406.20772217,18.16530273)
\curveto(406.20771906,18.09529707)(406.21271906,18.03529713)(406.22272217,17.98530273)
\curveto(406.25271902,17.83529733)(406.30271897,17.72529744)(406.37272217,17.65530273)
\curveto(406.44271883,17.59529757)(406.53771873,17.52529764)(406.65772217,17.44530273)
\curveto(406.79771847,17.34529782)(406.96271831,17.2702979)(407.15272217,17.22030273)
\curveto(407.34271793,17.18029799)(407.53271774,17.13029804)(407.72272217,17.07030273)
\curveto(407.84271743,17.03029814)(407.96271731,17.00029817)(408.08272217,16.98030273)
\curveto(408.21271706,16.96029821)(408.33771693,16.93029824)(408.45772217,16.89030273)
\curveto(408.65771661,16.83029834)(408.85271642,16.7702984)(409.04272217,16.71030273)
\curveto(409.23271604,16.66029851)(409.41771585,16.59529857)(409.59772217,16.51530273)
\curveto(409.64771562,16.49529867)(409.69271558,16.47529869)(409.73272217,16.45530273)
\curveto(409.78271549,16.43529873)(409.83271544,16.41029876)(409.88272217,16.38030273)
\curveto(410.05271522,16.26029891)(410.19771507,16.12529904)(410.31772217,15.97530273)
\curveto(410.43771483,15.82529934)(410.52771474,15.63529953)(410.58772217,15.40530273)
\lineto(410.58772217,15.12030273)
\curveto(410.58771468,15.05030012)(410.58271469,14.97530019)(410.57272217,14.89530273)
\curveto(410.56271471,14.82530034)(410.55271472,14.74530042)(410.54272217,14.65530273)
\lineto(410.51272217,14.50530273)
\curveto(410.4727148,14.43530073)(410.44271483,14.3653008)(410.42272217,14.29530273)
\curveto(410.41271486,14.22530094)(410.39271488,14.15530101)(410.36272217,14.08530273)
\curveto(410.31271496,13.97530119)(410.25771501,13.8703013)(410.19772217,13.77030273)
\curveto(410.13771513,13.6703015)(410.0727152,13.58030159)(410.00272217,13.50030273)
\curveto(409.79271548,13.24030193)(409.54771572,13.03030214)(409.26772217,12.87030273)
\curveto(408.98771628,12.72030245)(408.68271659,12.59030258)(408.35272217,12.48030273)
\curveto(408.25271702,12.45030272)(408.15271712,12.43030274)(408.05272217,12.42030273)
\curveto(407.95271732,12.40030277)(407.85771741,12.37530279)(407.76772217,12.34530273)
\curveto(407.65771761,12.32530284)(407.55271772,12.31530285)(407.45272217,12.31530273)
\curveto(407.35271792,12.31530285)(407.25271802,12.30530286)(407.15272217,12.28530273)
\lineto(407.00272217,12.28530273)
\curveto(406.95271832,12.27530289)(406.88271839,12.2703029)(406.79272217,12.27030273)
\curveto(406.70271857,12.2703029)(406.63271864,12.27530289)(406.58272217,12.28530273)
\lineto(406.41772217,12.28530273)
\curveto(406.35771891,12.30530286)(406.29271898,12.31530285)(406.22272217,12.31530273)
\curveto(406.15271912,12.30530286)(406.09771917,12.31030286)(406.05772217,12.33030273)
\curveto(406.00771926,12.34030283)(405.94271933,12.34530282)(405.86272217,12.34530273)
\curveto(405.78271949,12.3653028)(405.70771956,12.38530278)(405.63772217,12.40530273)
\curveto(405.5677197,12.41530275)(405.49271978,12.43530273)(405.41272217,12.46530273)
\curveto(405.12272015,12.5653026)(404.87772039,12.69030248)(404.67772217,12.84030273)
\curveto(404.47772079,12.99030218)(404.31772095,13.18530198)(404.19772217,13.42530273)
\curveto(404.13772113,13.55530161)(404.08772118,13.69030148)(404.04772217,13.83030273)
\curveto(404.01772125,13.9703012)(403.99772127,14.12530104)(403.98772217,14.29530273)
\curveto(403.97772129,14.35530081)(403.98272129,14.42530074)(404.00272217,14.50530273)
\curveto(404.02272125,14.59530057)(404.04772122,14.6653005)(404.07772217,14.71530273)
\curveto(404.11772115,14.75530041)(404.17772109,14.79530037)(404.25772217,14.83530273)
\curveto(404.30772096,14.85530031)(404.37772089,14.8653003)(404.46772217,14.86530273)
\curveto(404.5677207,14.87530029)(404.65772061,14.87530029)(404.73772217,14.86530273)
\curveto(404.82772044,14.85530031)(404.91272036,14.84030033)(404.99272217,14.82030273)
\curveto(405.08272019,14.81030036)(405.13772013,14.79530037)(405.15772217,14.77530273)
\curveto(405.21772005,14.72530044)(405.24772002,14.65030052)(405.24772217,14.55030273)
\curveto(405.25772001,14.46030071)(405.27771999,14.37530079)(405.30772217,14.29530273)
\curveto(405.35771991,14.07530109)(405.45771981,13.90530126)(405.60772217,13.78530273)
\curveto(405.70771956,13.69530147)(405.82771944,13.62530154)(405.96772217,13.57530273)
\curveto(406.10771916,13.52530164)(406.25771901,13.47530169)(406.41772217,13.42530273)
\lineto(406.73272217,13.38030273)
\lineto(406.82272217,13.38030273)
\curveto(406.88271839,13.36030181)(406.9677183,13.35030182)(407.07772217,13.35030273)
\curveto(407.19771807,13.35030182)(407.30271797,13.36030181)(407.39272217,13.38030273)
\curveto(407.46271781,13.38030179)(407.51771775,13.38530178)(407.55772217,13.39530273)
\curveto(407.61771765,13.40530176)(407.67771759,13.41030176)(407.73772217,13.41030273)
\curveto(407.79771747,13.42030175)(407.85271742,13.43030174)(407.90272217,13.44030273)
\curveto(408.21271706,13.52030165)(408.46271681,13.62530154)(408.65272217,13.75530273)
\curveto(408.85271642,13.88530128)(409.01771625,14.10530106)(409.14772217,14.41530273)
\curveto(409.17771609,14.4653007)(409.19271608,14.52030065)(409.19272217,14.58030273)
\curveto(409.20271607,14.64030053)(409.20271607,14.68530048)(409.19272217,14.71530273)
\curveto(409.18271609,14.90530026)(409.14271613,15.04530012)(409.07272217,15.13530273)
\curveto(409.00271627,15.23529993)(408.90771636,15.32529984)(408.78772217,15.40530273)
\curveto(408.70771656,15.4652997)(408.61271666,15.51529965)(408.50272217,15.55530273)
\lineto(408.20272217,15.67530273)
\curveto(408.1727171,15.68529948)(408.14271713,15.69029948)(408.11272217,15.69030273)
\curveto(408.09271718,15.69029948)(408.0727172,15.70029947)(408.05272217,15.72030273)
\curveto(407.73271754,15.83029934)(407.39271788,15.91029926)(407.03272217,15.96030273)
\curveto(406.68271859,16.02029915)(406.36271891,16.11529905)(406.07272217,16.24530273)
\curveto(405.98271929,16.28529888)(405.89271938,16.32029885)(405.80272217,16.35030273)
\curveto(405.72271955,16.38029879)(405.64771962,16.42029875)(405.57772217,16.47030273)
\curveto(405.40771986,16.58029859)(405.25772001,16.70529846)(405.12772217,16.84530273)
\curveto(404.99772027,16.98529818)(404.90772036,17.16029801)(404.85772217,17.37030273)
\curveto(404.83772043,17.44029773)(404.82772044,17.51029766)(404.82772217,17.58030273)
\lineto(404.82772217,17.80530273)
\curveto(404.81772045,17.92529724)(404.83272044,18.06029711)(404.87272217,18.21030273)
\curveto(404.91272036,18.3702968)(404.95272032,18.50529666)(404.99272217,18.61530273)
\curveto(405.02272025,18.6652965)(405.04272023,18.70529646)(405.05272217,18.73530273)
\curveto(405.0727202,18.77529639)(405.09772017,18.81529635)(405.12772217,18.85530273)
\curveto(405.25772001,19.08529608)(405.41771985,19.28529588)(405.60772217,19.45530273)
\curveto(405.79771947,19.62529554)(406.00771926,19.77529539)(406.23772217,19.90530273)
\curveto(406.39771887,19.99529517)(406.5727187,20.0652951)(406.76272217,20.11530273)
\curveto(406.96271831,20.17529499)(407.1677181,20.23029494)(407.37772217,20.28030273)
\curveto(407.44771782,20.29029488)(407.51271776,20.30029487)(407.57272217,20.31030273)
\curveto(407.64271763,20.32029485)(407.71771755,20.33029484)(407.79772217,20.34030273)
\curveto(407.83771743,20.35029482)(407.87771739,20.35029482)(407.91772217,20.34030273)
\curveto(407.9677173,20.33029484)(408.00771726,20.33529483)(408.03772217,20.35530273)
}
}
{
\newrgbcolor{curcolor}{0 0 0}
\pscustom[linestyle=none,fillstyle=solid,fillcolor=curcolor]
{
}
}
{
\newrgbcolor{curcolor}{0 0 0}
\pscustom[linestyle=none,fillstyle=solid,fillcolor=curcolor]
{
\newpath
\moveto(422.80287842,13.26030273)
\lineto(422.71287842,12.87030273)
\curveto(422.69287049,12.75030242)(422.65287053,12.65030252)(422.59287842,12.57030273)
\curveto(422.52287066,12.50030267)(422.42787075,12.46030271)(422.30787842,12.45030273)
\lineto(421.96287842,12.45030273)
\curveto(421.90287128,12.45030272)(421.84287134,12.44530272)(421.78287842,12.43530273)
\curveto(421.73287145,12.43530273)(421.68787149,12.44530272)(421.64787842,12.46530273)
\curveto(421.56787161,12.48530268)(421.51787166,12.52530264)(421.49787842,12.58530273)
\curveto(421.46787171,12.63530253)(421.45787172,12.69530247)(421.46787842,12.76530273)
\curveto(421.4778717,12.83530233)(421.47287171,12.90530226)(421.45287842,12.97530273)
\curveto(421.45287173,12.99530217)(421.44287174,13.01030216)(421.42287842,13.02030273)
\lineto(421.39287842,13.08030273)
\curveto(421.29287189,13.09030208)(421.20787197,13.0703021)(421.13787842,13.02030273)
\curveto(421.0778721,12.9703022)(421.01287217,12.92030225)(420.94287842,12.87030273)
\curveto(420.71287247,12.72030245)(420.48787269,12.60530256)(420.26787842,12.52530273)
\curveto(420.0778731,12.44530272)(419.85787332,12.38530278)(419.60787842,12.34530273)
\curveto(419.36787381,12.30530286)(419.12287406,12.28530288)(418.87287842,12.28530273)
\curveto(418.63287455,12.27530289)(418.39287479,12.29030288)(418.15287842,12.33030273)
\curveto(417.92287526,12.36030281)(417.72787545,12.41530275)(417.56787842,12.49530273)
\curveto(417.08787609,12.71530245)(416.72287646,13.01030216)(416.47287842,13.38030273)
\curveto(416.23287695,13.76030141)(416.0778771,14.23030094)(416.00787842,14.79030273)
\curveto(415.98787719,14.88030029)(415.9778772,14.9703002)(415.97787842,15.06030273)
\curveto(415.98787719,15.16030001)(415.98787719,15.26029991)(415.97787842,15.36030273)
\curveto(415.9778772,15.41029976)(415.9828772,15.46029971)(415.99287842,15.51030273)
\curveto(416.00287718,15.56029961)(416.00787717,15.61029956)(416.00787842,15.66030273)
\curveto(415.99787718,15.71029946)(415.99787718,15.76029941)(416.00787842,15.81030273)
\curveto(416.02787715,15.8702993)(416.03787714,15.92529924)(416.03787842,15.97530273)
\lineto(416.06787842,16.12530273)
\curveto(416.05787712,16.17529899)(416.05787712,16.24029893)(416.06787842,16.32030273)
\curveto(416.08787709,16.40029877)(416.11287707,16.4652987)(416.14287842,16.51530273)
\lineto(416.18787842,16.68030273)
\curveto(416.21787696,16.75029842)(416.23787694,16.82029835)(416.24787842,16.89030273)
\curveto(416.25787692,16.9702982)(416.2778769,17.04529812)(416.30787842,17.11530273)
\curveto(416.32787685,17.165298)(416.34287684,17.21029796)(416.35287842,17.25030273)
\curveto(416.36287682,17.29029788)(416.3778768,17.33529783)(416.39787842,17.38530273)
\curveto(416.44787673,17.48529768)(416.49287669,17.58029759)(416.53287842,17.67030273)
\curveto(416.57287661,17.7702974)(416.61787656,17.8652973)(416.66787842,17.95530273)
\curveto(416.86787631,18.33529683)(417.09787608,18.67529649)(417.35787842,18.97530273)
\curveto(417.62787555,19.28529588)(417.92787525,19.54029563)(418.25787842,19.74030273)
\curveto(418.45787472,19.86029531)(418.65787452,19.96029521)(418.85787842,20.04030273)
\curveto(419.05787412,20.12029505)(419.27287391,20.19029498)(419.50287842,20.25030273)
\lineto(419.71287842,20.28030273)
\curveto(419.7828734,20.29029488)(419.85287333,20.30529486)(419.92287842,20.32530273)
\lineto(420.07287842,20.32530273)
\curveto(420.16287302,20.34529482)(420.2828729,20.35529481)(420.43287842,20.35530273)
\curveto(420.59287259,20.35529481)(420.70787247,20.34529482)(420.77787842,20.32530273)
\curveto(420.81787236,20.31529485)(420.87287231,20.31029486)(420.94287842,20.31030273)
\curveto(421.04287214,20.28029489)(421.14787203,20.25529491)(421.25787842,20.23530273)
\curveto(421.36787181,20.22529494)(421.46787171,20.19529497)(421.55787842,20.14530273)
\curveto(421.69787148,20.08529508)(421.82787135,20.02029515)(421.94787842,19.95030273)
\curveto(422.06787111,19.88029529)(422.177871,19.80029537)(422.27787842,19.71030273)
\curveto(422.32787085,19.66029551)(422.3778708,19.60529556)(422.42787842,19.54530273)
\curveto(422.48787069,19.49529567)(422.57287061,19.48029569)(422.68287842,19.50030273)
\lineto(422.75787842,19.57530273)
\curveto(422.7778704,19.59529557)(422.79287039,19.62529554)(422.80287842,19.66530273)
\curveto(422.85287033,19.75529541)(422.88787029,19.8702953)(422.90787842,20.01030273)
\curveto(422.93787024,20.15029502)(422.96287022,20.27529489)(422.98287842,20.38530273)
\lineto(423.32787842,22.11030273)
\curveto(423.35786982,22.25029292)(423.38786979,22.40529276)(423.41787842,22.57530273)
\curveto(423.45786972,22.75529241)(423.50786967,22.88529228)(423.56787842,22.96530273)
\curveto(423.62786955,23.03529213)(423.69786948,23.08029209)(423.77787842,23.10030273)
\curveto(423.79786938,23.10029207)(423.82286936,23.10029207)(423.85287842,23.10030273)
\curveto(423.8828693,23.11029206)(423.90786927,23.11529205)(423.92787842,23.11530273)
\curveto(424.0778691,23.12529204)(424.22786895,23.12529204)(424.37787842,23.11530273)
\curveto(424.52786865,23.11529205)(424.62786855,23.07529209)(424.67787842,22.99530273)
\curveto(424.70786847,22.91529225)(424.70786847,22.81529235)(424.67787842,22.69530273)
\curveto(424.65786852,22.57529259)(424.63786854,22.45029272)(424.61787842,22.32030273)
\lineto(422.80287842,13.26030273)
\moveto(422.15787842,16.09530273)
\curveto(422.18787099,16.14529902)(422.20787097,16.21029896)(422.21787842,16.29030273)
\curveto(422.23787094,16.38029879)(422.24287094,16.45029872)(422.23287842,16.50030273)
\lineto(422.27787842,16.72530273)
\curveto(422.2778709,16.81529835)(422.2828709,16.90529826)(422.29287842,16.99530273)
\curveto(422.30287088,17.09529807)(422.29787088,17.18529798)(422.27787842,17.26530273)
\lineto(422.27787842,17.49030273)
\curveto(422.2778709,17.56029761)(422.26787091,17.63029754)(422.24787842,17.70030273)
\curveto(422.18787099,18.00029717)(422.0828711,18.2652969)(421.93287842,18.49530273)
\curveto(421.79287139,18.72529644)(421.59287159,18.90529626)(421.33287842,19.03530273)
\curveto(421.24287194,19.08529608)(421.14787203,19.12029605)(421.04787842,19.14030273)
\curveto(420.94787223,19.170296)(420.83787234,19.19529597)(420.71787842,19.21530273)
\curveto(420.64787253,19.23529593)(420.56287262,19.24529592)(420.46287842,19.24530273)
\lineto(420.19287842,19.24530273)
\lineto(420.04287842,19.21530273)
\lineto(419.90787842,19.21530273)
\curveto(419.82787335,19.19529597)(419.74287344,19.17529599)(419.65287842,19.15530273)
\curveto(419.56287362,19.13529603)(419.4778737,19.11029606)(419.39787842,19.08030273)
\curveto(419.04787413,18.94029623)(418.74787443,18.73529643)(418.49787842,18.46530273)
\curveto(418.24787493,18.20529696)(418.02787515,17.90029727)(417.83787842,17.55030273)
\curveto(417.7778754,17.44029773)(417.72787545,17.32529784)(417.68787842,17.20530273)
\lineto(417.56787842,16.87530273)
\lineto(417.53787842,16.75530273)
\curveto(417.52787565,16.72529844)(417.51787566,16.69029848)(417.50787842,16.65030273)
\curveto(417.4778757,16.60029857)(417.45787572,16.54529862)(417.44787842,16.48530273)
\curveto(417.44787573,16.42529874)(417.44287574,16.3702988)(417.43287842,16.32030273)
\curveto(417.41287577,16.21029896)(417.38787579,16.10029907)(417.35787842,15.99030273)
\curveto(417.33787584,15.89029928)(417.33287585,15.79529937)(417.34287842,15.70530273)
\curveto(417.34287584,15.67529949)(417.33787584,15.62529954)(417.32787842,15.55530273)
\lineto(417.32787842,15.34530273)
\curveto(417.32787585,15.27529989)(417.33287585,15.20529996)(417.34287842,15.13530273)
\curveto(417.3828758,14.78530038)(417.47287571,14.48530068)(417.61287842,14.23530273)
\curveto(417.75287543,13.98530118)(417.95287523,13.78030139)(418.21287842,13.62030273)
\curveto(418.29287489,13.5703016)(418.37287481,13.53030164)(418.45287842,13.50030273)
\curveto(418.54287464,13.4703017)(418.63787454,13.44030173)(418.73787842,13.41030273)
\curveto(418.78787439,13.39030178)(418.83787434,13.38530178)(418.88787842,13.39530273)
\curveto(418.94787423,13.40530176)(419.00287418,13.40030177)(419.05287842,13.38030273)
\curveto(419.0828741,13.3703018)(419.11787406,13.3653018)(419.15787842,13.36530273)
\lineto(419.29287842,13.36530273)
\lineto(419.42787842,13.36530273)
\curveto(419.46787371,13.37530179)(419.52287366,13.38030179)(419.59287842,13.38030273)
\curveto(419.67287351,13.40030177)(419.75287343,13.41530175)(419.83287842,13.42530273)
\curveto(419.92287326,13.44530172)(420.00287318,13.4703017)(420.07287842,13.50030273)
\curveto(420.43287275,13.64030153)(420.73787244,13.81530135)(420.98787842,14.02530273)
\curveto(421.23787194,14.24530092)(421.46287172,14.52030065)(421.66287842,14.85030273)
\curveto(421.73287145,14.96030021)(421.78787139,15.0703001)(421.82787842,15.18030273)
\lineto(421.97787842,15.51030273)
\curveto(422.00787117,15.55029962)(422.02287116,15.58529958)(422.02287842,15.61530273)
\curveto(422.03287115,15.65529951)(422.04787113,15.69529947)(422.06787842,15.73530273)
\curveto(422.08787109,15.79529937)(422.10287108,15.85529931)(422.11287842,15.91530273)
\curveto(422.12287106,15.97529919)(422.13787104,16.03529913)(422.15787842,16.09530273)
}
}
{
\newrgbcolor{curcolor}{0 0 0}
\pscustom[linestyle=none,fillstyle=solid,fillcolor=curcolor]
{
\newpath
\moveto(432.17412842,16.62030273)
\curveto(432.17411991,16.52029865)(432.15411993,16.40529876)(432.11412842,16.27530273)
\curveto(432.07412001,16.15529901)(432.02412006,16.0702991)(431.96412842,16.02030273)
\curveto(431.90412018,15.98029919)(431.82412026,15.95029922)(431.72412842,15.93030273)
\curveto(431.62412046,15.92029925)(431.51412057,15.91529925)(431.39412842,15.91530273)
\lineto(431.03412842,15.91530273)
\curveto(430.92412116,15.92529924)(430.82412126,15.93029924)(430.73412842,15.93030273)
\lineto(426.89412842,15.93030273)
\curveto(426.81412527,15.93029924)(426.72912536,15.92529924)(426.63912842,15.91530273)
\curveto(426.55912553,15.91529925)(426.49412559,15.90029927)(426.44412842,15.87030273)
\curveto(426.39412569,15.85029932)(426.34412574,15.81029936)(426.29412842,15.75030273)
\lineto(426.20412842,15.61530273)
\curveto(426.17412591,15.5652996)(426.16412592,15.51529965)(426.17412842,15.46530273)
\curveto(426.17412591,15.41529975)(426.16912592,15.3702998)(426.15912842,15.33030273)
\lineto(426.15912842,15.21030273)
\lineto(426.15912842,14.95530273)
\curveto(426.16912592,14.87530029)(426.1841259,14.79530037)(426.20412842,14.71530273)
\curveto(426.33412575,14.17530099)(426.63912545,13.79030138)(427.11912842,13.56030273)
\curveto(427.16912492,13.53030164)(427.22912486,13.50530166)(427.29912842,13.48530273)
\curveto(427.36912472,13.4653017)(427.43412465,13.44530172)(427.49412842,13.42530273)
\curveto(427.52412456,13.41530175)(427.57412451,13.41030176)(427.64412842,13.41030273)
\curveto(427.77412431,13.3703018)(427.95412413,13.35030182)(428.18412842,13.35030273)
\curveto(428.41412367,13.35030182)(428.60412348,13.3703018)(428.75412842,13.41030273)
\curveto(428.90412318,13.45030172)(429.03912305,13.49030168)(429.15912842,13.53030273)
\curveto(429.2891228,13.58030159)(429.40912268,13.64030153)(429.51912842,13.71030273)
\curveto(429.63912245,13.78030139)(429.74912234,13.86030131)(429.84912842,13.95030273)
\curveto(429.94912214,14.05030112)(430.03912205,14.15530101)(430.11912842,14.26530273)
\curveto(430.19912189,14.3653008)(430.27412181,14.4703007)(430.34412842,14.58030273)
\curveto(430.41412167,14.69030048)(430.50912158,14.7703004)(430.62912842,14.82030273)
\curveto(430.66912142,14.84030033)(430.73412135,14.85530031)(430.82412842,14.86530273)
\curveto(430.92412116,14.87530029)(431.01412107,14.87530029)(431.09412842,14.86530273)
\curveto(431.1841209,14.8653003)(431.26912082,14.86030031)(431.34912842,14.85030273)
\curveto(431.42912066,14.84030033)(431.47912061,14.82030035)(431.49912842,14.79030273)
\curveto(431.5891205,14.72030045)(431.59412049,14.60530056)(431.51412842,14.44530273)
\curveto(431.37412071,14.17530099)(431.21912087,13.93530123)(431.04912842,13.72530273)
\curveto(430.7891213,13.40530176)(430.50912158,13.14030203)(430.20912842,12.93030273)
\curveto(429.91912217,12.73030244)(429.56412252,12.5653026)(429.14412842,12.43530273)
\curveto(429.03412305,12.39530277)(428.92912316,12.3703028)(428.82912842,12.36030273)
\curveto(428.72912336,12.34030283)(428.61912347,12.32030285)(428.49912842,12.30030273)
\curveto(428.44912364,12.29030288)(428.39912369,12.28530288)(428.34912842,12.28530273)
\curveto(428.30912378,12.28530288)(428.26412382,12.28030289)(428.21412842,12.27030273)
\lineto(428.06412842,12.27030273)
\curveto(428.01412407,12.26030291)(427.95412413,12.25530291)(427.88412842,12.25530273)
\curveto(427.82412426,12.25530291)(427.77412431,12.26030291)(427.73412842,12.27030273)
\lineto(427.59912842,12.27030273)
\curveto(427.54912454,12.28030289)(427.50412458,12.28530288)(427.46412842,12.28530273)
\curveto(427.42412466,12.28530288)(427.3841247,12.29030288)(427.34412842,12.30030273)
\curveto(427.29412479,12.31030286)(427.23912485,12.32030285)(427.17912842,12.33030273)
\curveto(427.12912496,12.33030284)(427.07912501,12.33530283)(427.02912842,12.34530273)
\curveto(426.93912515,12.3653028)(426.84912524,12.39030278)(426.75912842,12.42030273)
\curveto(426.67912541,12.44030273)(426.60412548,12.4653027)(426.53412842,12.49530273)
\curveto(426.49412559,12.51530265)(426.45912563,12.52530264)(426.42912842,12.52530273)
\curveto(426.39912569,12.53530263)(426.36912572,12.55030262)(426.33912842,12.57030273)
\curveto(426.19912589,12.64030253)(426.05412603,12.72530244)(425.90412842,12.82530273)
\curveto(425.65412643,13.01530215)(425.45412663,13.24530192)(425.30412842,13.51530273)
\curveto(425.15412693,13.79530137)(425.04412704,14.10530106)(424.97412842,14.44530273)
\curveto(424.94412714,14.55530061)(424.92912716,14.6703005)(424.92912842,14.79030273)
\curveto(424.92912716,14.91030026)(424.91912717,15.03030014)(424.89912842,15.15030273)
\lineto(424.89912842,15.25530273)
\curveto(424.90912718,15.28529988)(424.91412717,15.32529984)(424.91412842,15.37530273)
\lineto(424.91412842,15.63030273)
\curveto(424.92412716,15.72029945)(424.92912716,15.81029936)(424.92912842,15.90030273)
\lineto(424.97412842,16.11030273)
\curveto(424.97412711,16.15029902)(424.97912711,16.20529896)(424.98912842,16.27530273)
\curveto(424.99912709,16.35529881)(425.01412707,16.42029875)(425.03412842,16.47030273)
\lineto(425.06412842,16.63530273)
\curveto(425.09412699,16.68529848)(425.10912698,16.73529843)(425.10912842,16.78530273)
\curveto(425.11912697,16.84529832)(425.13412695,16.90029827)(425.15412842,16.95030273)
\curveto(425.22412686,17.11029806)(425.2891268,17.2702979)(425.34912842,17.43030273)
\curveto(425.40912668,17.59029758)(425.4841266,17.74029743)(425.57412842,17.88030273)
\curveto(425.64412644,17.99029718)(425.70912638,18.10029707)(425.76912842,18.21030273)
\curveto(425.83912625,18.33029684)(425.91912617,18.44529672)(426.00912842,18.55530273)
\curveto(426.29912579,18.90529626)(426.60912548,19.20529596)(426.93912842,19.45530273)
\curveto(427.26912482,19.71529545)(427.65412443,19.93029524)(428.09412842,20.10030273)
\curveto(428.22412386,20.15029502)(428.35412373,20.18529498)(428.48412842,20.20530273)
\curveto(428.61412347,20.23529493)(428.75412333,20.2652949)(428.90412842,20.29530273)
\curveto(428.95412313,20.30529486)(428.99912309,20.31029486)(429.03912842,20.31030273)
\curveto(429.07912301,20.32029485)(429.12412296,20.32529484)(429.17412842,20.32530273)
\curveto(429.19412289,20.33529483)(429.21912287,20.33529483)(429.24912842,20.32530273)
\curveto(429.27912281,20.31529485)(429.30412278,20.32029485)(429.32412842,20.34030273)
\curveto(429.75412233,20.35029482)(430.11412197,20.30529486)(430.40412842,20.20530273)
\curveto(430.69412139,20.11529505)(430.94912114,19.99029518)(431.16912842,19.83030273)
\curveto(431.20912088,19.81029536)(431.23912085,19.78029539)(431.25912842,19.74030273)
\curveto(431.2891208,19.71029546)(431.31912077,19.68529548)(431.34912842,19.66530273)
\curveto(431.41912067,19.60529556)(431.4891206,19.53529563)(431.55912842,19.45530273)
\curveto(431.62912046,19.37529579)(431.6841204,19.29529587)(431.72412842,19.21530273)
\curveto(431.84412024,19.00529616)(431.93912015,18.80529636)(432.00912842,18.61530273)
\curveto(432.05912003,18.50529666)(432.08912,18.38529678)(432.09912842,18.25530273)
\lineto(432.15912842,17.86530273)
\curveto(432.1891199,17.73529743)(432.19911989,17.60029757)(432.18912842,17.46030273)
\curveto(432.1891199,17.32029785)(432.19411989,17.18029799)(432.20412842,17.04030273)
\curveto(432.20411988,16.9702982)(432.19911989,16.90029827)(432.18912842,16.83030273)
\curveto(432.17911991,16.76029841)(432.17411991,16.69029848)(432.17412842,16.62030273)
\moveto(430.82412842,17.13030273)
\curveto(430.85412123,17.170298)(430.8841212,17.22029795)(430.91412842,17.28030273)
\curveto(430.95412113,17.35029782)(430.96912112,17.42029775)(430.95912842,17.49030273)
\curveto(430.94912114,17.71029746)(430.90912118,17.91529725)(430.83912842,18.10530273)
\curveto(430.73912135,18.33529683)(430.61912147,18.53029664)(430.47912842,18.69030273)
\curveto(430.34912174,18.85029632)(430.15912193,18.98529618)(429.90912842,19.09530273)
\curveto(429.83912225,19.11529605)(429.76912232,19.13029604)(429.69912842,19.14030273)
\curveto(429.63912245,19.16029601)(429.56912252,19.18029599)(429.48912842,19.20030273)
\curveto(429.41912267,19.22029595)(429.33912275,19.23029594)(429.24912842,19.23030273)
\lineto(428.99412842,19.23030273)
\curveto(428.95412313,19.21029596)(428.91412317,19.20029597)(428.87412842,19.20030273)
\curveto(428.83412325,19.21029596)(428.79912329,19.21029596)(428.76912842,19.20030273)
\lineto(428.52912842,19.14030273)
\curveto(428.44912364,19.13029604)(428.37412371,19.11529605)(428.30412842,19.09530273)
\curveto(427.9841241,18.97529619)(427.71912437,18.82529634)(427.50912842,18.64530273)
\curveto(427.29912479,18.4652967)(427.09912499,18.24029693)(426.90912842,17.97030273)
\curveto(426.86912522,17.92029725)(426.82412526,17.85529731)(426.77412842,17.77530273)
\curveto(426.73412535,17.70529746)(426.69412539,17.62529754)(426.65412842,17.53530273)
\curveto(426.61412547,17.44529772)(426.5891255,17.36029781)(426.57912842,17.28030273)
\curveto(426.57912551,17.20029797)(426.60412548,17.14029803)(426.65412842,17.10030273)
\curveto(426.72412536,17.04029813)(426.85412523,17.01029816)(427.04412842,17.01030273)
\curveto(427.24412484,17.02029815)(427.41412467,17.02529814)(427.55412842,17.02530273)
\lineto(429.83412842,17.02530273)
\curveto(429.9841221,17.02529814)(430.16412192,17.02029815)(430.37412842,17.01030273)
\curveto(430.5841215,17.01029816)(430.73412135,17.05029812)(430.82412842,17.13030273)
}
}
{
\newrgbcolor{curcolor}{0 0 0}
\pscustom[linestyle=none,fillstyle=solid,fillcolor=curcolor]
{
\newpath
\moveto(437.24576904,23.25030273)
\curveto(437.42576334,23.26029191)(437.61576315,23.26029191)(437.81576904,23.25030273)
\curveto(438.01576275,23.24029193)(438.14576262,23.18029199)(438.20576904,23.07030273)
\curveto(438.23576253,23.01029216)(438.24576252,22.93529223)(438.23576904,22.84530273)
\curveto(438.22576254,22.7652924)(438.21076256,22.67529249)(438.19076904,22.57530273)
\curveto(438.1707626,22.44529272)(438.12576264,22.34029283)(438.05576904,22.26030273)
\curveto(438.00576276,22.21029296)(437.94076283,22.17529299)(437.86076904,22.15530273)
\curveto(437.78076299,22.14529302)(437.69576307,22.14029303)(437.60576904,22.14030273)
\lineto(437.33576904,22.14030273)
\curveto(437.24576352,22.15029302)(437.16076361,22.15029302)(437.08076904,22.14030273)
\curveto(436.79076398,22.06029311)(436.58576418,21.93029324)(436.46576904,21.75030273)
\curveto(436.34576442,21.58029359)(436.25076452,21.32029385)(436.18076904,20.97030273)
\curveto(436.16076461,20.90029427)(436.13576463,20.82529434)(436.10576904,20.74530273)
\curveto(436.08576468,20.67529449)(436.08076469,20.61029456)(436.09076904,20.55030273)
\curveto(436.09076468,20.40029477)(436.13576463,20.29529487)(436.22576904,20.23530273)
\curveto(436.29576447,20.20529496)(436.39076438,20.19029498)(436.51076904,20.19030273)
\lineto(436.87076904,20.19030273)
\lineto(437.09576904,20.19030273)
\curveto(437.12576364,20.170295)(437.15576361,20.165295)(437.18576904,20.17530273)
\curveto(437.21576355,20.18529498)(437.24576352,20.18029499)(437.27576904,20.16030273)
\curveto(437.3657634,20.13029504)(437.41576335,20.0702951)(437.42576904,19.98030273)
\curveto(437.44576332,19.90029527)(437.44076333,19.79529537)(437.41076904,19.66530273)
\lineto(437.38076904,19.54530273)
\lineto(437.35076904,19.42530273)
\curveto(437.29076348,19.27529589)(437.20576356,19.17529599)(437.09576904,19.12530273)
\curveto(436.95576381,19.07529609)(436.78576398,19.06029611)(436.58576904,19.08030273)
\curveto(436.38576438,19.11029606)(436.21076456,19.10529606)(436.06076904,19.06530273)
\curveto(435.98076479,19.04529612)(435.91576485,19.00529616)(435.86576904,18.94530273)
\curveto(435.81576495,18.89529627)(435.770765,18.82529634)(435.73076904,18.73530273)
\curveto(435.70076507,18.6652965)(435.68076509,18.58529658)(435.67076904,18.49530273)
\curveto(435.66076511,18.40529676)(435.64576512,18.32029685)(435.62576904,18.24030273)
\lineto(435.43076904,17.25030273)
\lineto(434.80076904,14.07030273)
\lineto(434.65076904,13.32030273)
\curveto(434.64076613,13.26030191)(434.63076614,13.19530197)(434.62076904,13.12530273)
\curveto(434.61076616,13.05530211)(434.59076618,12.99530217)(434.56076904,12.94530273)
\lineto(434.53076904,12.82530273)
\lineto(434.47076904,12.70530273)
\curveto(434.46076631,12.6653025)(434.44076633,12.63030254)(434.41076904,12.60030273)
\curveto(434.35076642,12.53030264)(434.2657665,12.49030268)(434.15576904,12.48030273)
\curveto(434.05576671,12.4703027)(433.94576682,12.4653027)(433.82576904,12.46530273)
\lineto(433.54076904,12.46530273)
\curveto(433.50076727,12.48530268)(433.45576731,12.50030267)(433.40576904,12.51030273)
\curveto(433.3657674,12.53030264)(433.33576743,12.5653026)(433.31576904,12.61530273)
\curveto(433.30576746,12.64530252)(433.30076747,12.71030246)(433.30076904,12.81030273)
\lineto(433.31576904,12.91530273)
\curveto(433.30576746,12.9653022)(433.31076746,13.01530215)(433.33076904,13.06530273)
\curveto(433.35076742,13.12530204)(433.3657674,13.18030199)(433.37576904,13.23030273)
\lineto(433.49576904,13.83030273)
\lineto(434.30576904,17.92530273)
\curveto(434.32576644,18.03529713)(434.35076642,18.15029702)(434.38076904,18.27030273)
\curveto(434.41076636,18.39029678)(434.43076634,18.50029667)(434.44076904,18.60030273)
\curveto(434.46076631,18.71029646)(434.46076631,18.80529636)(434.44076904,18.88530273)
\curveto(434.43076634,18.9652962)(434.38576638,19.02029615)(434.30576904,19.05030273)
\curveto(434.25576651,19.08029609)(434.19076658,19.09529607)(434.11076904,19.09530273)
\lineto(433.88576904,19.09530273)
\lineto(433.64576904,19.09530273)
\curveto(433.57576719,19.09529607)(433.51076726,19.10529606)(433.45076904,19.12530273)
\curveto(433.3707674,19.165296)(433.32576744,19.25029592)(433.31576904,19.38030273)
\lineto(433.31576904,19.51530273)
\curveto(433.32576744,19.55529561)(433.33576743,19.60029557)(433.34576904,19.65030273)
\curveto(433.37576739,19.79029538)(433.41076736,19.90029527)(433.45076904,19.98030273)
\curveto(433.50076727,20.0702951)(433.58076719,20.13029504)(433.69076904,20.16030273)
\curveto(433.770767,20.19029498)(433.85576691,20.20029497)(433.94576904,20.19030273)
\lineto(434.21576904,20.19030273)
\curveto(434.31576645,20.19029498)(434.40576636,20.20029497)(434.48576904,20.22030273)
\curveto(434.5657662,20.24029493)(434.63576613,20.28029489)(434.69576904,20.34030273)
\curveto(434.78576598,20.42029475)(434.84576592,20.54529462)(434.87576904,20.71530273)
\curveto(434.90576586,20.88529428)(434.93576583,21.04529412)(434.96576904,21.19530273)
\curveto(435.00576576,21.39529377)(435.05576571,21.58029359)(435.11576904,21.75030273)
\curveto(435.17576559,21.93029324)(435.25076552,22.09029308)(435.34076904,22.23030273)
\curveto(435.49076528,22.4702927)(435.6707651,22.6652925)(435.88076904,22.81530273)
\curveto(436.10076467,22.9652922)(436.35076442,23.08029209)(436.63076904,23.16030273)
\curveto(436.69076408,23.18029199)(436.75576401,23.19029198)(436.82576904,23.19030273)
\curveto(436.89576387,23.20029197)(436.9657638,23.21529195)(437.03576904,23.23530273)
\curveto(437.05576371,23.24529192)(437.09076368,23.24529192)(437.14076904,23.23530273)
\curveto(437.19076358,23.23529193)(437.22576354,23.24029193)(437.24576904,23.25030273)
\moveto(439.19576904,21.67530273)
\curveto(439.25576151,21.62529354)(439.33576143,21.60029357)(439.43576904,21.60030273)
\lineto(439.75076904,21.60030273)
\lineto(439.91576904,21.60030273)
\curveto(439.97576079,21.60029357)(440.03576073,21.61029356)(440.09576904,21.63030273)
\curveto(440.23576053,21.68029349)(440.32076045,21.78529338)(440.35076904,21.94530273)
\curveto(440.39076038,22.10529306)(440.43076034,22.27529289)(440.47076904,22.45530273)
\curveto(440.48076029,22.54529262)(440.49576027,22.63029254)(440.51576904,22.71030273)
\curveto(440.53576023,22.80029237)(440.53576023,22.87529229)(440.51576904,22.93530273)
\curveto(440.48576028,23.04529212)(440.39576037,23.10529206)(440.24576904,23.11530273)
\curveto(440.10576066,23.12529204)(439.95076082,23.13029204)(439.78076904,23.13030273)
\curveto(439.75076102,23.12029205)(439.72576104,23.11529205)(439.70576904,23.11530273)
\curveto(439.68576108,23.12529204)(439.66076111,23.12529204)(439.63076904,23.11530273)
\curveto(439.51076126,23.07529209)(439.42076135,23.01529215)(439.36076904,22.93530273)
\curveto(439.32076145,22.87529229)(439.29076148,22.80029237)(439.27076904,22.71030273)
\curveto(439.25076152,22.62029255)(439.23576153,22.53529263)(439.22576904,22.45530273)
\curveto(439.19576157,22.30529286)(439.1657616,22.15029302)(439.13576904,21.99030273)
\curveto(439.10576166,21.84029333)(439.12576164,21.73529343)(439.19576904,21.67530273)
\moveto(439.87076904,19.51530273)
\curveto(439.89076088,19.61529555)(439.91076086,19.71029546)(439.93076904,19.80030273)
\curveto(439.95076082,19.90029527)(439.94076083,19.98029519)(439.90076904,20.04030273)
\curveto(439.8707609,20.12029505)(439.78576098,20.16029501)(439.64576904,20.16030273)
\curveto(439.51576125,20.170295)(439.38576138,20.17529499)(439.25576904,20.17530273)
\curveto(439.23576153,20.165295)(439.21076156,20.16029501)(439.18076904,20.16030273)
\curveto(439.16076161,20.170295)(439.14076163,20.17529499)(439.12076904,20.17530273)
\curveto(439.06076171,20.15529501)(439.00076177,20.14029503)(438.94076904,20.13030273)
\curveto(438.89076188,20.12029505)(438.84576192,20.09029508)(438.80576904,20.04030273)
\curveto(438.74576202,19.98029519)(438.70576206,19.89529527)(438.68576904,19.78530273)
\curveto(438.6657621,19.68529548)(438.64576212,19.58029559)(438.62576904,19.47030273)
\lineto(437.35076904,13.12530273)
\curveto(437.33076344,13.03530213)(437.31076346,12.94030223)(437.29076904,12.84030273)
\curveto(437.28076349,12.75030242)(437.28576348,12.67530249)(437.30576904,12.61530273)
\curveto(437.34576342,12.53530263)(437.41076336,12.48530268)(437.50076904,12.46530273)
\curveto(437.59076318,12.45530271)(437.70076307,12.45030272)(437.83076904,12.45030273)
\lineto(438.05576904,12.45030273)
\curveto(438.14576262,12.4703027)(438.22076255,12.48530268)(438.28076904,12.49530273)
\curveto(438.34076243,12.51530265)(438.39076238,12.55530261)(438.43076904,12.61530273)
\curveto(438.50076227,12.67530249)(438.54076223,12.75530241)(438.55076904,12.85530273)
\curveto(438.5707622,12.9653022)(438.59076218,13.0703021)(438.61076904,13.17030273)
\lineto(439.87076904,19.51530273)
}
}
{
\newrgbcolor{curcolor}{0 0 0}
\pscustom[linestyle=none,fillstyle=solid,fillcolor=curcolor]
{
\newpath
\moveto(445.66944092,20.32530273)
\curveto(446.3094341,20.34529482)(446.79943361,20.26029491)(447.13944092,20.07030273)
\curveto(447.47943293,19.88029529)(447.72443268,19.59529557)(447.87444092,19.21530273)
\curveto(447.91443249,19.11529605)(447.93943247,19.00529616)(447.94944092,18.88530273)
\curveto(447.96943244,18.77529639)(447.97943243,18.66029651)(447.97944092,18.54030273)
\curveto(447.99943241,18.35029682)(447.98943242,18.14529702)(447.94944092,17.92530273)
\curveto(447.91943249,17.70529746)(447.87943253,17.48029769)(447.82944092,17.25030273)
\lineto(447.51444092,15.64530273)
\lineto(447.04944092,13.30530273)
\lineto(446.92944092,12.79530273)
\curveto(446.88943352,12.62530254)(446.79943361,12.51530265)(446.65944092,12.46530273)
\curveto(446.6094338,12.44530272)(446.55443385,12.43530273)(446.49444092,12.43530273)
\curveto(446.44443396,12.42530274)(446.38943402,12.42030275)(446.32944092,12.42030273)
\curveto(446.19943421,12.42030275)(446.07443433,12.42530274)(445.95444092,12.43530273)
\curveto(445.83443457,12.43530273)(445.75943465,12.47530269)(445.72944092,12.55530273)
\curveto(445.68943472,12.62530254)(445.67943473,12.71530245)(445.69944092,12.82530273)
\curveto(445.71943469,12.93530223)(445.74443466,13.04530212)(445.77444092,13.15530273)
\lineto(446.02944092,14.44530273)
\lineto(446.50944092,16.89030273)
\curveto(446.56943384,17.16029801)(446.61943379,17.42529774)(446.65944092,17.68530273)
\curveto(446.69943371,17.95529721)(446.69943371,18.18529698)(446.65944092,18.37530273)
\curveto(446.61943379,18.57529659)(446.52943388,18.73529643)(446.38944092,18.85530273)
\curveto(446.25943415,18.98529618)(446.09943431,19.08529608)(445.90944092,19.15530273)
\curveto(445.84943456,19.17529599)(445.78443462,19.18529598)(445.71444092,19.18530273)
\curveto(445.65443475,19.19529597)(445.59943481,19.21029596)(445.54944092,19.23030273)
\curveto(445.49943491,19.24029593)(445.41943499,19.24029593)(445.30944092,19.23030273)
\curveto(445.2094352,19.23029594)(445.13443527,19.22529594)(445.08444092,19.21530273)
\curveto(445.04443536,19.19529597)(445.0094354,19.18529598)(444.97944092,19.18530273)
\curveto(444.94943546,19.19529597)(444.91443549,19.19529597)(444.87444092,19.18530273)
\curveto(444.73443567,19.15529601)(444.6044358,19.12029605)(444.48444092,19.08030273)
\curveto(444.36443604,19.05029612)(444.24943616,19.00529616)(444.13944092,18.94530273)
\curveto(444.08943632,18.92529624)(444.04943636,18.90529626)(444.01944092,18.88530273)
\curveto(443.98943642,18.8652963)(443.94943646,18.84529632)(443.89944092,18.82530273)
\curveto(443.49943691,18.57529659)(443.16943724,18.20029697)(442.90944092,17.70030273)
\curveto(442.86943754,17.62029755)(442.83443757,17.53529763)(442.80444092,17.44530273)
\lineto(442.71444092,17.20530273)
\curveto(442.68443772,17.15529801)(442.66943774,17.10529806)(442.66944092,17.05530273)
\curveto(442.66943774,17.01529815)(442.65443775,16.97529819)(442.62444092,16.93530273)
\lineto(442.56444092,16.62030273)
\curveto(442.54443786,16.59029858)(442.53443787,16.55529861)(442.53444092,16.51530273)
\curveto(442.53443787,16.47529869)(442.52943788,16.43029874)(442.51944092,16.38030273)
\lineto(442.42944092,15.93030273)
\lineto(442.12944092,14.49030273)
\lineto(441.87444092,13.17030273)
\curveto(441.85443855,13.06030211)(441.82943858,12.94530222)(441.79944092,12.82530273)
\curveto(441.77943863,12.71530245)(441.73943867,12.62530254)(441.67944092,12.55530273)
\curveto(441.6094388,12.47530269)(441.5094389,12.43530273)(441.37944092,12.43530273)
\curveto(441.25943915,12.42530274)(441.13443927,12.42030275)(441.00444092,12.42030273)
\curveto(440.92443948,12.42030275)(440.84943956,12.42530274)(440.77944092,12.43530273)
\curveto(440.7094397,12.44530272)(440.65443975,12.4703027)(440.61444092,12.51030273)
\curveto(440.54443986,12.56030261)(440.52443988,12.65530251)(440.55444092,12.79530273)
\curveto(440.58443982,12.93530223)(440.6094398,13.0703021)(440.62944092,13.20030273)
\lineto(440.98944092,14.97030273)
\lineto(441.70944092,18.60030273)
\lineto(441.88944092,19.51530273)
\lineto(441.94944092,19.78530273)
\curveto(441.96943844,19.87529529)(442.0044384,19.94529522)(442.05444092,19.99530273)
\curveto(442.09443831,20.05529511)(442.14943826,20.09529507)(442.21944092,20.11530273)
\curveto(442.26943814,20.12529504)(442.32943808,20.13529503)(442.39944092,20.14530273)
\curveto(442.47943793,20.15529501)(442.55943785,20.16029501)(442.63944092,20.16030273)
\curveto(442.71943769,20.16029501)(442.79443761,20.15529501)(442.86444092,20.14530273)
\curveto(442.94443746,20.13529503)(442.99443741,20.12029505)(443.01444092,20.10030273)
\curveto(443.11443729,20.03029514)(443.14943726,19.94029523)(443.11944092,19.83030273)
\curveto(443.08943732,19.73029544)(443.07943733,19.61529555)(443.08944092,19.48530273)
\curveto(443.09943731,19.42529574)(443.12943728,19.37529579)(443.17944092,19.33530273)
\curveto(443.29943711,19.32529584)(443.404437,19.3702958)(443.49444092,19.47030273)
\curveto(443.59443681,19.5702956)(443.68943672,19.65029552)(443.77944092,19.71030273)
\curveto(443.93943647,19.81029536)(444.09943631,19.90029527)(444.25944092,19.98030273)
\curveto(444.41943599,20.0702951)(444.6044358,20.14529502)(444.81444092,20.20530273)
\curveto(444.89443551,20.23529493)(444.98443542,20.25529491)(445.08444092,20.26530273)
\curveto(445.18443522,20.27529489)(445.27943513,20.29029488)(445.36944092,20.31030273)
\curveto(445.41943499,20.32029485)(445.46943494,20.32529484)(445.51944092,20.32530273)
\lineto(445.66944092,20.32530273)
}
}
{
\newrgbcolor{curcolor}{0 0 0}
\pscustom[linestyle=none,fillstyle=solid,fillcolor=curcolor]
{
\newpath
\moveto(450.88405029,21.67530273)
\curveto(450.81404732,21.73529343)(450.79404734,21.84029333)(450.82405029,21.99030273)
\curveto(450.85404728,22.15029302)(450.88404725,22.30529286)(450.91405029,22.45530273)
\curveto(450.92404721,22.53529263)(450.93904719,22.62029255)(450.95905029,22.71030273)
\curveto(450.97904715,22.80029237)(451.00904712,22.87529229)(451.04905029,22.93530273)
\curveto(451.10904702,23.01529215)(451.19904693,23.07529209)(451.31905029,23.11530273)
\curveto(451.34904678,23.12529204)(451.37404676,23.12529204)(451.39405029,23.11530273)
\curveto(451.41404672,23.11529205)(451.43904669,23.12029205)(451.46905029,23.13030273)
\curveto(451.63904649,23.13029204)(451.79404634,23.12529204)(451.93405029,23.11530273)
\curveto(452.08404605,23.10529206)(452.17404596,23.04529212)(452.20405029,22.93530273)
\curveto(452.22404591,22.87529229)(452.22404591,22.80029237)(452.20405029,22.71030273)
\curveto(452.18404595,22.63029254)(452.16904596,22.54529262)(452.15905029,22.45530273)
\curveto(452.11904601,22.27529289)(452.07904605,22.10529306)(452.03905029,21.94530273)
\curveto(452.00904612,21.78529338)(451.92404621,21.68029349)(451.78405029,21.63030273)
\curveto(451.72404641,21.61029356)(451.66404647,21.60029357)(451.60405029,21.60030273)
\lineto(451.43905029,21.60030273)
\lineto(451.12405029,21.60030273)
\curveto(451.02404711,21.60029357)(450.94404719,21.62529354)(450.88405029,21.67530273)
\moveto(450.29905029,13.17030273)
\curveto(450.27904785,13.0703021)(450.25904787,12.9653022)(450.23905029,12.85530273)
\curveto(450.2290479,12.75530241)(450.18904794,12.67530249)(450.11905029,12.61530273)
\curveto(450.07904805,12.55530261)(450.0290481,12.51530265)(449.96905029,12.49530273)
\curveto(449.90904822,12.48530268)(449.8340483,12.4703027)(449.74405029,12.45030273)
\lineto(449.51905029,12.45030273)
\curveto(449.38904874,12.45030272)(449.27904885,12.45530271)(449.18905029,12.46530273)
\curveto(449.09904903,12.48530268)(449.0340491,12.53530263)(448.99405029,12.61530273)
\curveto(448.97404916,12.67530249)(448.96904916,12.75030242)(448.97905029,12.84030273)
\curveto(448.99904913,12.94030223)(449.01904911,13.03530213)(449.03905029,13.12530273)
\lineto(450.31405029,19.47030273)
\curveto(450.3340478,19.58029559)(450.35404778,19.68529548)(450.37405029,19.78530273)
\curveto(450.39404774,19.89529527)(450.4340477,19.98029519)(450.49405029,20.04030273)
\curveto(450.5340476,20.09029508)(450.57904755,20.12029505)(450.62905029,20.13030273)
\curveto(450.68904744,20.14029503)(450.74904738,20.15529501)(450.80905029,20.17530273)
\curveto(450.8290473,20.17529499)(450.84904728,20.170295)(450.86905029,20.16030273)
\curveto(450.89904723,20.16029501)(450.92404721,20.165295)(450.94405029,20.17530273)
\curveto(451.07404706,20.17529499)(451.20404693,20.170295)(451.33405029,20.16030273)
\curveto(451.47404666,20.16029501)(451.55904657,20.12029505)(451.58905029,20.04030273)
\curveto(451.6290465,19.98029519)(451.63904649,19.90029527)(451.61905029,19.80030273)
\curveto(451.59904653,19.71029546)(451.57904655,19.61529555)(451.55905029,19.51530273)
\lineto(450.29905029,13.17030273)
}
}
{
\newrgbcolor{curcolor}{0 0 0}
\pscustom[linestyle=none,fillstyle=solid,fillcolor=curcolor]
{
\newpath
\moveto(459.21889404,13.26030273)
\lineto(459.12889404,12.87030273)
\curveto(459.10888611,12.75030242)(459.06888615,12.65030252)(459.00889404,12.57030273)
\curveto(458.93888628,12.50030267)(458.84388638,12.46030271)(458.72389404,12.45030273)
\lineto(458.37889404,12.45030273)
\curveto(458.3188869,12.45030272)(458.25888696,12.44530272)(458.19889404,12.43530273)
\curveto(458.14888707,12.43530273)(458.10388712,12.44530272)(458.06389404,12.46530273)
\curveto(457.98388724,12.48530268)(457.93388729,12.52530264)(457.91389404,12.58530273)
\curveto(457.88388734,12.63530253)(457.87388735,12.69530247)(457.88389404,12.76530273)
\curveto(457.89388733,12.83530233)(457.88888733,12.90530226)(457.86889404,12.97530273)
\curveto(457.86888735,12.99530217)(457.85888736,13.01030216)(457.83889404,13.02030273)
\lineto(457.80889404,13.08030273)
\curveto(457.70888751,13.09030208)(457.6238876,13.0703021)(457.55389404,13.02030273)
\curveto(457.49388773,12.9703022)(457.42888779,12.92030225)(457.35889404,12.87030273)
\curveto(457.12888809,12.72030245)(456.90388832,12.60530256)(456.68389404,12.52530273)
\curveto(456.49388873,12.44530272)(456.27388895,12.38530278)(456.02389404,12.34530273)
\curveto(455.78388944,12.30530286)(455.53888968,12.28530288)(455.28889404,12.28530273)
\curveto(455.04889017,12.27530289)(454.80889041,12.29030288)(454.56889404,12.33030273)
\curveto(454.33889088,12.36030281)(454.14389108,12.41530275)(453.98389404,12.49530273)
\curveto(453.50389172,12.71530245)(453.13889208,13.01030216)(452.88889404,13.38030273)
\curveto(452.64889257,13.76030141)(452.49389273,14.23030094)(452.42389404,14.79030273)
\curveto(452.40389282,14.88030029)(452.39389283,14.9703002)(452.39389404,15.06030273)
\curveto(452.40389282,15.16030001)(452.40389282,15.26029991)(452.39389404,15.36030273)
\curveto(452.39389283,15.41029976)(452.39889282,15.46029971)(452.40889404,15.51030273)
\curveto(452.4188928,15.56029961)(452.4238928,15.61029956)(452.42389404,15.66030273)
\curveto(452.41389281,15.71029946)(452.41389281,15.76029941)(452.42389404,15.81030273)
\curveto(452.44389278,15.8702993)(452.45389277,15.92529924)(452.45389404,15.97530273)
\lineto(452.48389404,16.12530273)
\curveto(452.47389275,16.17529899)(452.47389275,16.24029893)(452.48389404,16.32030273)
\curveto(452.50389272,16.40029877)(452.52889269,16.4652987)(452.55889404,16.51530273)
\lineto(452.60389404,16.68030273)
\curveto(452.63389259,16.75029842)(452.65389257,16.82029835)(452.66389404,16.89030273)
\curveto(452.67389255,16.9702982)(452.69389253,17.04529812)(452.72389404,17.11530273)
\curveto(452.74389248,17.165298)(452.75889246,17.21029796)(452.76889404,17.25030273)
\curveto(452.77889244,17.29029788)(452.79389243,17.33529783)(452.81389404,17.38530273)
\curveto(452.86389236,17.48529768)(452.90889231,17.58029759)(452.94889404,17.67030273)
\curveto(452.98889223,17.7702974)(453.03389219,17.8652973)(453.08389404,17.95530273)
\curveto(453.28389194,18.33529683)(453.51389171,18.67529649)(453.77389404,18.97530273)
\curveto(454.04389118,19.28529588)(454.34389088,19.54029563)(454.67389404,19.74030273)
\curveto(454.87389035,19.86029531)(455.07389015,19.96029521)(455.27389404,20.04030273)
\curveto(455.47388975,20.12029505)(455.68888953,20.19029498)(455.91889404,20.25030273)
\lineto(456.12889404,20.28030273)
\curveto(456.19888902,20.29029488)(456.26888895,20.30529486)(456.33889404,20.32530273)
\lineto(456.48889404,20.32530273)
\curveto(456.57888864,20.34529482)(456.69888852,20.35529481)(456.84889404,20.35530273)
\curveto(457.00888821,20.35529481)(457.1238881,20.34529482)(457.19389404,20.32530273)
\curveto(457.23388799,20.31529485)(457.28888793,20.31029486)(457.35889404,20.31030273)
\curveto(457.45888776,20.28029489)(457.56388766,20.25529491)(457.67389404,20.23530273)
\curveto(457.78388744,20.22529494)(457.88388734,20.19529497)(457.97389404,20.14530273)
\curveto(458.11388711,20.08529508)(458.24388698,20.02029515)(458.36389404,19.95030273)
\curveto(458.48388674,19.88029529)(458.59388663,19.80029537)(458.69389404,19.71030273)
\curveto(458.74388648,19.66029551)(458.79388643,19.60529556)(458.84389404,19.54530273)
\curveto(458.90388632,19.49529567)(458.98888623,19.48029569)(459.09889404,19.50030273)
\lineto(459.17389404,19.57530273)
\curveto(459.19388603,19.59529557)(459.20888601,19.62529554)(459.21889404,19.66530273)
\curveto(459.26888595,19.75529541)(459.30388592,19.8702953)(459.32389404,20.01030273)
\curveto(459.35388587,20.15029502)(459.37888584,20.27529489)(459.39889404,20.38530273)
\lineto(459.74389404,22.11030273)
\curveto(459.77388545,22.25029292)(459.80388542,22.40529276)(459.83389404,22.57530273)
\curveto(459.87388535,22.75529241)(459.9238853,22.88529228)(459.98389404,22.96530273)
\curveto(460.04388518,23.03529213)(460.11388511,23.08029209)(460.19389404,23.10030273)
\curveto(460.21388501,23.10029207)(460.23888498,23.10029207)(460.26889404,23.10030273)
\curveto(460.29888492,23.11029206)(460.3238849,23.11529205)(460.34389404,23.11530273)
\curveto(460.49388473,23.12529204)(460.64388458,23.12529204)(460.79389404,23.11530273)
\curveto(460.94388428,23.11529205)(461.04388418,23.07529209)(461.09389404,22.99530273)
\curveto(461.1238841,22.91529225)(461.1238841,22.81529235)(461.09389404,22.69530273)
\curveto(461.07388415,22.57529259)(461.05388417,22.45029272)(461.03389404,22.32030273)
\lineto(459.21889404,13.26030273)
\moveto(458.57389404,16.09530273)
\curveto(458.60388662,16.14529902)(458.6238866,16.21029896)(458.63389404,16.29030273)
\curveto(458.65388657,16.38029879)(458.65888656,16.45029872)(458.64889404,16.50030273)
\lineto(458.69389404,16.72530273)
\curveto(458.69388653,16.81529835)(458.69888652,16.90529826)(458.70889404,16.99530273)
\curveto(458.7188865,17.09529807)(458.71388651,17.18529798)(458.69389404,17.26530273)
\lineto(458.69389404,17.49030273)
\curveto(458.69388653,17.56029761)(458.68388654,17.63029754)(458.66389404,17.70030273)
\curveto(458.60388662,18.00029717)(458.49888672,18.2652969)(458.34889404,18.49530273)
\curveto(458.20888701,18.72529644)(458.00888721,18.90529626)(457.74889404,19.03530273)
\curveto(457.65888756,19.08529608)(457.56388766,19.12029605)(457.46389404,19.14030273)
\curveto(457.36388786,19.170296)(457.25388797,19.19529597)(457.13389404,19.21530273)
\curveto(457.06388816,19.23529593)(456.97888824,19.24529592)(456.87889404,19.24530273)
\lineto(456.60889404,19.24530273)
\lineto(456.45889404,19.21530273)
\lineto(456.32389404,19.21530273)
\curveto(456.24388898,19.19529597)(456.15888906,19.17529599)(456.06889404,19.15530273)
\curveto(455.97888924,19.13529603)(455.89388933,19.11029606)(455.81389404,19.08030273)
\curveto(455.46388976,18.94029623)(455.16389006,18.73529643)(454.91389404,18.46530273)
\curveto(454.66389056,18.20529696)(454.44389078,17.90029727)(454.25389404,17.55030273)
\curveto(454.19389103,17.44029773)(454.14389108,17.32529784)(454.10389404,17.20530273)
\lineto(453.98389404,16.87530273)
\lineto(453.95389404,16.75530273)
\curveto(453.94389128,16.72529844)(453.93389129,16.69029848)(453.92389404,16.65030273)
\curveto(453.89389133,16.60029857)(453.87389135,16.54529862)(453.86389404,16.48530273)
\curveto(453.86389136,16.42529874)(453.85889136,16.3702988)(453.84889404,16.32030273)
\curveto(453.82889139,16.21029896)(453.80389142,16.10029907)(453.77389404,15.99030273)
\curveto(453.75389147,15.89029928)(453.74889147,15.79529937)(453.75889404,15.70530273)
\curveto(453.75889146,15.67529949)(453.75389147,15.62529954)(453.74389404,15.55530273)
\lineto(453.74389404,15.34530273)
\curveto(453.74389148,15.27529989)(453.74889147,15.20529996)(453.75889404,15.13530273)
\curveto(453.79889142,14.78530038)(453.88889133,14.48530068)(454.02889404,14.23530273)
\curveto(454.16889105,13.98530118)(454.36889085,13.78030139)(454.62889404,13.62030273)
\curveto(454.70889051,13.5703016)(454.78889043,13.53030164)(454.86889404,13.50030273)
\curveto(454.95889026,13.4703017)(455.05389017,13.44030173)(455.15389404,13.41030273)
\curveto(455.20389002,13.39030178)(455.25388997,13.38530178)(455.30389404,13.39530273)
\curveto(455.36388986,13.40530176)(455.4188898,13.40030177)(455.46889404,13.38030273)
\curveto(455.49888972,13.3703018)(455.53388969,13.3653018)(455.57389404,13.36530273)
\lineto(455.70889404,13.36530273)
\lineto(455.84389404,13.36530273)
\curveto(455.88388934,13.37530179)(455.93888928,13.38030179)(456.00889404,13.38030273)
\curveto(456.08888913,13.40030177)(456.16888905,13.41530175)(456.24889404,13.42530273)
\curveto(456.33888888,13.44530172)(456.4188888,13.4703017)(456.48889404,13.50030273)
\curveto(456.84888837,13.64030153)(457.15388807,13.81530135)(457.40389404,14.02530273)
\curveto(457.65388757,14.24530092)(457.87888734,14.52030065)(458.07889404,14.85030273)
\curveto(458.14888707,14.96030021)(458.20388702,15.0703001)(458.24389404,15.18030273)
\lineto(458.39389404,15.51030273)
\curveto(458.4238868,15.55029962)(458.43888678,15.58529958)(458.43889404,15.61530273)
\curveto(458.44888677,15.65529951)(458.46388676,15.69529947)(458.48389404,15.73530273)
\curveto(458.50388672,15.79529937)(458.5188867,15.85529931)(458.52889404,15.91530273)
\curveto(458.53888668,15.97529919)(458.55388667,16.03529913)(458.57389404,16.09530273)
}
}
{
\newrgbcolor{curcolor}{0 0 0}
\pscustom[linestyle=none,fillstyle=solid,fillcolor=curcolor]
{
\newpath
\moveto(468.96514404,16.65030273)
\curveto(468.97513515,16.59029858)(468.96513516,16.49529867)(468.93514404,16.36530273)
\curveto(468.91513521,16.24529892)(468.89513523,16.16029901)(468.87514404,16.11030273)
\lineto(468.84514404,15.96030273)
\curveto(468.81513531,15.88029929)(468.79013534,15.80529936)(468.77014404,15.73530273)
\curveto(468.76013537,15.67529949)(468.74013539,15.60529956)(468.71014404,15.52530273)
\curveto(468.68013545,15.4652997)(468.65513547,15.40529976)(468.63514404,15.34530273)
\curveto(468.6251355,15.28529988)(468.60013553,15.22529994)(468.56014404,15.16530273)
\lineto(468.38014404,14.77530273)
\curveto(468.3301358,14.64530052)(468.26513586,14.52530064)(468.18514404,14.41530273)
\curveto(467.88513624,13.93530123)(467.5251366,13.53030164)(467.10514404,13.20030273)
\curveto(466.69513743,12.88030229)(466.21513791,12.63530253)(465.66514404,12.46530273)
\curveto(465.55513857,12.42530274)(465.43513869,12.39530277)(465.30514404,12.37530273)
\curveto(465.17513895,12.35530281)(465.04013909,12.33530283)(464.90014404,12.31530273)
\curveto(464.84013929,12.30530286)(464.77513935,12.30030287)(464.70514404,12.30030273)
\curveto(464.64513948,12.29030288)(464.58513954,12.28530288)(464.52514404,12.28530273)
\curveto(464.48513964,12.27530289)(464.4251397,12.2703029)(464.34514404,12.27030273)
\curveto(464.27513985,12.2703029)(464.2251399,12.27530289)(464.19514404,12.28530273)
\curveto(464.15513997,12.29530287)(464.11514001,12.30030287)(464.07514404,12.30030273)
\curveto(464.03514009,12.29030288)(464.00014013,12.29030288)(463.97014404,12.30030273)
\lineto(463.88014404,12.30030273)
\lineto(463.53514404,12.34530273)
\lineto(463.14514404,12.46530273)
\curveto(463.0251411,12.50530266)(462.91014122,12.55030262)(462.80014404,12.60030273)
\curveto(462.39014174,12.80030237)(462.07014206,13.06030211)(461.84014404,13.38030273)
\curveto(461.62014251,13.70030147)(461.46014267,14.09030108)(461.36014404,14.55030273)
\curveto(461.3301428,14.65030052)(461.31014282,14.75030042)(461.30014404,14.85030273)
\lineto(461.30014404,15.16530273)
\curveto(461.29014284,15.20529996)(461.29014284,15.23529993)(461.30014404,15.25530273)
\curveto(461.31014282,15.28529988)(461.31514281,15.32029985)(461.31514404,15.36030273)
\curveto(461.31514281,15.44029973)(461.32014281,15.52029965)(461.33014404,15.60030273)
\curveto(461.34014279,15.69029948)(461.34514278,15.77529939)(461.34514404,15.85530273)
\curveto(461.35514277,15.90529926)(461.36014277,15.94529922)(461.36014404,15.97530273)
\curveto(461.37014276,16.01529915)(461.37514275,16.06029911)(461.37514404,16.11030273)
\curveto(461.37514275,16.16029901)(461.38514274,16.24529892)(461.40514404,16.36530273)
\curveto(461.43514269,16.49529867)(461.46514266,16.59029858)(461.49514404,16.65030273)
\curveto(461.53514259,16.72029845)(461.55514257,16.79029838)(461.55514404,16.86030273)
\curveto(461.55514257,16.93029824)(461.57514255,17.00029817)(461.61514404,17.07030273)
\curveto(461.63514249,17.12029805)(461.65014248,17.16029801)(461.66014404,17.19030273)
\curveto(461.67014246,17.23029794)(461.68514244,17.27529789)(461.70514404,17.32530273)
\curveto(461.76514236,17.44529772)(461.81514231,17.5652976)(461.85514404,17.68530273)
\curveto(461.90514222,17.80529736)(461.97014216,17.92029725)(462.05014404,18.03030273)
\curveto(462.27014186,18.40029677)(462.51514161,18.73029644)(462.78514404,19.02030273)
\curveto(463.06514106,19.32029585)(463.38014075,19.5702956)(463.73014404,19.77030273)
\curveto(463.86014027,19.85029532)(463.99514013,19.91529525)(464.13514404,19.96530273)
\lineto(464.58514404,20.14530273)
\curveto(464.71513941,20.19529497)(464.85013928,20.22529494)(464.99014404,20.23530273)
\curveto(465.130139,20.25529491)(465.27513885,20.28529488)(465.42514404,20.32530273)
\lineto(465.62014404,20.32530273)
\lineto(465.83014404,20.35530273)
\curveto(466.72013741,20.3652948)(467.42013671,20.18029499)(467.93014404,19.80030273)
\curveto(468.45013568,19.42029575)(468.77513535,18.92529624)(468.90514404,18.31530273)
\curveto(468.93513519,18.21529695)(468.95513517,18.11529705)(468.96514404,18.01530273)
\curveto(468.97513515,17.91529725)(468.99013514,17.81029736)(469.01014404,17.70030273)
\curveto(469.02013511,17.59029758)(469.02013511,17.4702977)(469.01014404,17.34030273)
\lineto(469.01014404,16.96530273)
\curveto(469.01013512,16.91529825)(469.00013513,16.86029831)(468.98014404,16.80030273)
\curveto(468.97013516,16.75029842)(468.96513516,16.70029847)(468.96514404,16.65030273)
\moveto(467.46514404,15.79530273)
\curveto(467.49513663,15.8652993)(467.51513661,15.94529922)(467.52514404,16.03530273)
\curveto(467.54513658,16.12529904)(467.56013657,16.21029896)(467.57014404,16.29030273)
\curveto(467.65013648,16.68029849)(467.68513644,17.01029816)(467.67514404,17.28030273)
\curveto(467.65513647,17.36029781)(467.64013649,17.44029773)(467.63014404,17.52030273)
\curveto(467.6301365,17.60029757)(467.6251365,17.67529749)(467.61514404,17.74530273)
\curveto(467.46513666,18.39529677)(467.11013702,18.84529632)(466.55014404,19.09530273)
\curveto(466.48013765,19.12529604)(466.40513772,19.14529602)(466.32514404,19.15530273)
\curveto(466.25513787,19.17529599)(466.18013795,19.19529597)(466.10014404,19.21530273)
\curveto(466.0301381,19.23529593)(465.95013818,19.24529592)(465.86014404,19.24530273)
\lineto(465.59014404,19.24530273)
\lineto(465.30514404,19.20030273)
\curveto(465.20513892,19.18029599)(465.11013902,19.15529601)(465.02014404,19.12530273)
\curveto(464.9301392,19.10529606)(464.84013929,19.07529609)(464.75014404,19.03530273)
\curveto(464.68013945,19.01529615)(464.61013952,18.98529618)(464.54014404,18.94530273)
\curveto(464.47013966,18.90529626)(464.40513972,18.8652963)(464.34514404,18.82530273)
\curveto(464.07514005,18.65529651)(463.84014029,18.45029672)(463.64014404,18.21030273)
\curveto(463.44014069,17.9702972)(463.25514087,17.69029748)(463.08514404,17.37030273)
\curveto(463.03514109,17.2702979)(462.99514113,17.165298)(462.96514404,17.05530273)
\curveto(462.93514119,16.95529821)(462.89514123,16.85029832)(462.84514404,16.74030273)
\curveto(462.83514129,16.70029847)(462.82014131,16.63529853)(462.80014404,16.54530273)
\curveto(462.78014135,16.51529865)(462.77014136,16.48029869)(462.77014404,16.44030273)
\curveto(462.77014136,16.40029877)(462.76514136,16.35529881)(462.75514404,16.30530273)
\lineto(462.69514404,16.00530273)
\curveto(462.67514145,15.90529926)(462.66514146,15.81529935)(462.66514404,15.73530273)
\lineto(462.66514404,15.55530273)
\curveto(462.66514146,15.45529971)(462.66014147,15.35529981)(462.65014404,15.25530273)
\curveto(462.65014148,15.1653)(462.66014147,15.08030009)(462.68014404,15.00030273)
\curveto(462.7301414,14.76030041)(462.80014133,14.53530063)(462.89014404,14.32530273)
\curveto(462.99014114,14.11530105)(463.125141,13.94030123)(463.29514404,13.80030273)
\curveto(463.34514078,13.7703014)(463.38514074,13.74530142)(463.41514404,13.72530273)
\curveto(463.45514067,13.70530146)(463.49514063,13.68030149)(463.53514404,13.65030273)
\curveto(463.60514052,13.60030157)(463.68514044,13.55530161)(463.77514404,13.51530273)
\curveto(463.86514026,13.48530168)(463.96014017,13.45530171)(464.06014404,13.42530273)
\curveto(464.11014002,13.40530176)(464.15513997,13.39530177)(464.19514404,13.39530273)
\curveto(464.24513988,13.40530176)(464.29513983,13.40530176)(464.34514404,13.39530273)
\curveto(464.37513975,13.38530178)(464.43513969,13.37530179)(464.52514404,13.36530273)
\curveto(464.61513951,13.35530181)(464.69013944,13.36030181)(464.75014404,13.38030273)
\curveto(464.79013934,13.39030178)(464.8301393,13.39030178)(464.87014404,13.38030273)
\curveto(464.91013922,13.38030179)(464.95013918,13.39030178)(464.99014404,13.41030273)
\curveto(465.07013906,13.43030174)(465.15013898,13.44530172)(465.23014404,13.45530273)
\curveto(465.32013881,13.47530169)(465.40513872,13.50030167)(465.48514404,13.53030273)
\curveto(465.84513828,13.6703015)(466.15513797,13.8653013)(466.41514404,14.11530273)
\curveto(466.67513745,14.3653008)(466.91013722,14.66030051)(467.12014404,15.00030273)
\curveto(467.20013693,15.12030005)(467.26013687,15.24529992)(467.30014404,15.37530273)
\curveto(467.34013679,15.51529965)(467.39513673,15.65529951)(467.46514404,15.79530273)
}
}
{
\newrgbcolor{curcolor}{0 0 0}
\pscustom[linestyle=none,fillstyle=solid,fillcolor=curcolor]
{
\newpath
\moveto(473.63342529,20.35530273)
\curveto(474.35341964,20.3652948)(474.93841905,20.28029489)(475.38842529,20.10030273)
\curveto(475.84841814,19.93029524)(476.16841782,19.62529554)(476.34842529,19.18530273)
\curveto(476.39841759,19.07529609)(476.42841756,18.96029621)(476.43842529,18.84030273)
\curveto(476.45841753,18.73029644)(476.47341752,18.60529656)(476.48342529,18.46530273)
\curveto(476.4934175,18.39529677)(476.48341751,18.32029685)(476.45342529,18.24030273)
\curveto(476.43341756,18.170297)(476.40841758,18.11529705)(476.37842529,18.07530273)
\curveto(476.35841763,18.05529711)(476.32841766,18.03529713)(476.28842529,18.01530273)
\curveto(476.25841773,18.00529716)(476.23341776,17.99029718)(476.21342529,17.97030273)
\curveto(476.15341784,17.95029722)(476.09841789,17.94529722)(476.04842529,17.95530273)
\curveto(476.00841798,17.9652972)(475.96341803,17.9652972)(475.91342529,17.95530273)
\curveto(475.82341817,17.93529723)(475.71341828,17.93029724)(475.58342529,17.94030273)
\curveto(475.46341853,17.96029721)(475.37841861,17.98529718)(475.32842529,18.01530273)
\curveto(475.25841873,18.0652971)(475.21841877,18.13029704)(475.20842529,18.21030273)
\curveto(475.20841878,18.30029687)(475.1884188,18.38529678)(475.14842529,18.46530273)
\curveto(475.09841889,18.62529654)(475.00341899,18.7702964)(474.86342529,18.90030273)
\curveto(474.77341922,18.98029619)(474.66341933,19.04029613)(474.53342529,19.08030273)
\curveto(474.41341958,19.12029605)(474.28341971,19.16029601)(474.14342529,19.20030273)
\curveto(474.10341989,19.22029595)(474.05341994,19.22529594)(473.99342529,19.21530273)
\curveto(473.94342005,19.21529595)(473.89842009,19.22029595)(473.85842529,19.23030273)
\curveto(473.79842019,19.25029592)(473.72342027,19.26029591)(473.63342529,19.26030273)
\curveto(473.54342045,19.26029591)(473.46842052,19.25029592)(473.40842529,19.23030273)
\lineto(473.31842529,19.23030273)
\curveto(473.25842073,19.22029595)(473.20342079,19.21029596)(473.15342529,19.20030273)
\curveto(473.10342089,19.20029597)(473.05342094,19.19529597)(473.00342529,19.18530273)
\curveto(472.73342126,19.12529604)(472.49842149,19.04029613)(472.29842529,18.93030273)
\curveto(472.10842188,18.82029635)(471.95842203,18.63529653)(471.84842529,18.37530273)
\curveto(471.81842217,18.30529686)(471.80342219,18.23529693)(471.80342529,18.16530273)
\curveto(471.80342219,18.09529707)(471.80842218,18.03529713)(471.81842529,17.98530273)
\curveto(471.84842214,17.83529733)(471.89842209,17.72529744)(471.96842529,17.65530273)
\curveto(472.03842195,17.59529757)(472.13342186,17.52529764)(472.25342529,17.44530273)
\curveto(472.3934216,17.34529782)(472.55842143,17.2702979)(472.74842529,17.22030273)
\curveto(472.93842105,17.18029799)(473.12842086,17.13029804)(473.31842529,17.07030273)
\curveto(473.43842055,17.03029814)(473.55842043,17.00029817)(473.67842529,16.98030273)
\curveto(473.80842018,16.96029821)(473.93342006,16.93029824)(474.05342529,16.89030273)
\curveto(474.25341974,16.83029834)(474.44841954,16.7702984)(474.63842529,16.71030273)
\curveto(474.82841916,16.66029851)(475.01341898,16.59529857)(475.19342529,16.51530273)
\curveto(475.24341875,16.49529867)(475.2884187,16.47529869)(475.32842529,16.45530273)
\curveto(475.37841861,16.43529873)(475.42841856,16.41029876)(475.47842529,16.38030273)
\curveto(475.64841834,16.26029891)(475.7934182,16.12529904)(475.91342529,15.97530273)
\curveto(476.03341796,15.82529934)(476.12341787,15.63529953)(476.18342529,15.40530273)
\lineto(476.18342529,15.12030273)
\curveto(476.18341781,15.05030012)(476.17841781,14.97530019)(476.16842529,14.89530273)
\curveto(476.15841783,14.82530034)(476.14841784,14.74530042)(476.13842529,14.65530273)
\lineto(476.10842529,14.50530273)
\curveto(476.06841792,14.43530073)(476.03841795,14.3653008)(476.01842529,14.29530273)
\curveto(476.00841798,14.22530094)(475.988418,14.15530101)(475.95842529,14.08530273)
\curveto(475.90841808,13.97530119)(475.85341814,13.8703013)(475.79342529,13.77030273)
\curveto(475.73341826,13.6703015)(475.66841832,13.58030159)(475.59842529,13.50030273)
\curveto(475.3884186,13.24030193)(475.14341885,13.03030214)(474.86342529,12.87030273)
\curveto(474.58341941,12.72030245)(474.27841971,12.59030258)(473.94842529,12.48030273)
\curveto(473.84842014,12.45030272)(473.74842024,12.43030274)(473.64842529,12.42030273)
\curveto(473.54842044,12.40030277)(473.45342054,12.37530279)(473.36342529,12.34530273)
\curveto(473.25342074,12.32530284)(473.14842084,12.31530285)(473.04842529,12.31530273)
\curveto(472.94842104,12.31530285)(472.84842114,12.30530286)(472.74842529,12.28530273)
\lineto(472.59842529,12.28530273)
\curveto(472.54842144,12.27530289)(472.47842151,12.2703029)(472.38842529,12.27030273)
\curveto(472.29842169,12.2703029)(472.22842176,12.27530289)(472.17842529,12.28530273)
\lineto(472.01342529,12.28530273)
\curveto(471.95342204,12.30530286)(471.8884221,12.31530285)(471.81842529,12.31530273)
\curveto(471.74842224,12.30530286)(471.6934223,12.31030286)(471.65342529,12.33030273)
\curveto(471.60342239,12.34030283)(471.53842245,12.34530282)(471.45842529,12.34530273)
\curveto(471.37842261,12.3653028)(471.30342269,12.38530278)(471.23342529,12.40530273)
\curveto(471.16342283,12.41530275)(471.0884229,12.43530273)(471.00842529,12.46530273)
\curveto(470.71842327,12.5653026)(470.47342352,12.69030248)(470.27342529,12.84030273)
\curveto(470.07342392,12.99030218)(469.91342408,13.18530198)(469.79342529,13.42530273)
\curveto(469.73342426,13.55530161)(469.68342431,13.69030148)(469.64342529,13.83030273)
\curveto(469.61342438,13.9703012)(469.5934244,14.12530104)(469.58342529,14.29530273)
\curveto(469.57342442,14.35530081)(469.57842441,14.42530074)(469.59842529,14.50530273)
\curveto(469.61842437,14.59530057)(469.64342435,14.6653005)(469.67342529,14.71530273)
\curveto(469.71342428,14.75530041)(469.77342422,14.79530037)(469.85342529,14.83530273)
\curveto(469.90342409,14.85530031)(469.97342402,14.8653003)(470.06342529,14.86530273)
\curveto(470.16342383,14.87530029)(470.25342374,14.87530029)(470.33342529,14.86530273)
\curveto(470.42342357,14.85530031)(470.50842348,14.84030033)(470.58842529,14.82030273)
\curveto(470.67842331,14.81030036)(470.73342326,14.79530037)(470.75342529,14.77530273)
\curveto(470.81342318,14.72530044)(470.84342315,14.65030052)(470.84342529,14.55030273)
\curveto(470.85342314,14.46030071)(470.87342312,14.37530079)(470.90342529,14.29530273)
\curveto(470.95342304,14.07530109)(471.05342294,13.90530126)(471.20342529,13.78530273)
\curveto(471.30342269,13.69530147)(471.42342257,13.62530154)(471.56342529,13.57530273)
\curveto(471.70342229,13.52530164)(471.85342214,13.47530169)(472.01342529,13.42530273)
\lineto(472.32842529,13.38030273)
\lineto(472.41842529,13.38030273)
\curveto(472.47842151,13.36030181)(472.56342143,13.35030182)(472.67342529,13.35030273)
\curveto(472.7934212,13.35030182)(472.89842109,13.36030181)(472.98842529,13.38030273)
\curveto(473.05842093,13.38030179)(473.11342088,13.38530178)(473.15342529,13.39530273)
\curveto(473.21342078,13.40530176)(473.27342072,13.41030176)(473.33342529,13.41030273)
\curveto(473.3934206,13.42030175)(473.44842054,13.43030174)(473.49842529,13.44030273)
\curveto(473.80842018,13.52030165)(474.05841993,13.62530154)(474.24842529,13.75530273)
\curveto(474.44841954,13.88530128)(474.61341938,14.10530106)(474.74342529,14.41530273)
\curveto(474.77341922,14.4653007)(474.7884192,14.52030065)(474.78842529,14.58030273)
\curveto(474.79841919,14.64030053)(474.79841919,14.68530048)(474.78842529,14.71530273)
\curveto(474.77841921,14.90530026)(474.73841925,15.04530012)(474.66842529,15.13530273)
\curveto(474.59841939,15.23529993)(474.50341949,15.32529984)(474.38342529,15.40530273)
\curveto(474.30341969,15.4652997)(474.20841978,15.51529965)(474.09842529,15.55530273)
\lineto(473.79842529,15.67530273)
\curveto(473.76842022,15.68529948)(473.73842025,15.69029948)(473.70842529,15.69030273)
\curveto(473.6884203,15.69029948)(473.66842032,15.70029947)(473.64842529,15.72030273)
\curveto(473.32842066,15.83029934)(472.988421,15.91029926)(472.62842529,15.96030273)
\curveto(472.27842171,16.02029915)(471.95842203,16.11529905)(471.66842529,16.24530273)
\curveto(471.57842241,16.28529888)(471.4884225,16.32029885)(471.39842529,16.35030273)
\curveto(471.31842267,16.38029879)(471.24342275,16.42029875)(471.17342529,16.47030273)
\curveto(471.00342299,16.58029859)(470.85342314,16.70529846)(470.72342529,16.84530273)
\curveto(470.5934234,16.98529818)(470.50342349,17.16029801)(470.45342529,17.37030273)
\curveto(470.43342356,17.44029773)(470.42342357,17.51029766)(470.42342529,17.58030273)
\lineto(470.42342529,17.80530273)
\curveto(470.41342358,17.92529724)(470.42842356,18.06029711)(470.46842529,18.21030273)
\curveto(470.50842348,18.3702968)(470.54842344,18.50529666)(470.58842529,18.61530273)
\curveto(470.61842337,18.6652965)(470.63842335,18.70529646)(470.64842529,18.73530273)
\curveto(470.66842332,18.77529639)(470.6934233,18.81529635)(470.72342529,18.85530273)
\curveto(470.85342314,19.08529608)(471.01342298,19.28529588)(471.20342529,19.45530273)
\curveto(471.3934226,19.62529554)(471.60342239,19.77529539)(471.83342529,19.90530273)
\curveto(471.993422,19.99529517)(472.16842182,20.0652951)(472.35842529,20.11530273)
\curveto(472.55842143,20.17529499)(472.76342123,20.23029494)(472.97342529,20.28030273)
\curveto(473.04342095,20.29029488)(473.10842088,20.30029487)(473.16842529,20.31030273)
\curveto(473.23842075,20.32029485)(473.31342068,20.33029484)(473.39342529,20.34030273)
\curveto(473.43342056,20.35029482)(473.47342052,20.35029482)(473.51342529,20.34030273)
\curveto(473.56342043,20.33029484)(473.60342039,20.33529483)(473.63342529,20.35530273)
}
}
{
\newrgbcolor{curcolor}{0 0 0}
\pscustom[linewidth=1,linecolor=curcolor]
{
\newpath
\moveto(103.01786,82.52252)
\lineto(740.99776,82.52252)
}
}
{
\newrgbcolor{curcolor}{0 0 0}
\pscustom[linewidth=1,linecolor=curcolor]
{
\newpath
\moveto(103.01786,156.51623)
\lineto(740.99776,156.51623)
}
}
{
\newrgbcolor{curcolor}{0 0 0}
\pscustom[linewidth=1,linecolor=curcolor]
{
\newpath
\moveto(103.01786,230.55822)
\lineto(740.99776,230.55822)
}
}
{
\newrgbcolor{curcolor}{0 0 0}
\pscustom[linewidth=1,linecolor=curcolor]
{
\newpath
\moveto(103.01786,305.62335)
\lineto(740.99776,305.62335)
}
}
{
\newrgbcolor{curcolor}{0 0 0}
\pscustom[linewidth=1,linecolor=curcolor]
{
\newpath
\moveto(103.01786,379.589017)
\lineto(740.99776,379.589017)
}
}
{
\newrgbcolor{curcolor}{0 0 0}
\pscustom[linestyle=none,fillstyle=solid,fillcolor=curcolor]
{
\newpath
\moveto(104.44285278,415.81210297)
\lineto(105.73285278,415.81210297)
\curveto(105.84284996,415.81209229)(105.94784985,415.80709229)(106.04785278,415.79710297)
\curveto(106.14784965,415.7970923)(106.22284958,415.76209234)(106.27285278,415.69210297)
\curveto(106.32284948,415.62209248)(106.34784945,415.53209257)(106.34785278,415.42210297)
\curveto(106.35784944,415.31209279)(106.36284944,415.19209291)(106.36285278,415.06210297)
\lineto(106.36285278,413.75710297)
\lineto(106.36285278,408.55210297)
\lineto(106.36285278,406.09210297)
\lineto(106.36285278,405.65710297)
\curveto(106.37284943,405.4971026)(106.35284945,405.37710272)(106.30285278,405.29710297)
\curveto(106.26284954,405.22710287)(106.17284963,405.17210293)(106.03285278,405.13210297)
\curveto(105.96284984,405.11210299)(105.88784991,405.10710299)(105.80785278,405.11710297)
\curveto(105.72785007,405.12710297)(105.64785015,405.13210297)(105.56785278,405.13210297)
\lineto(104.68285278,405.13210297)
\curveto(104.57285123,405.13210297)(104.46785133,405.13710296)(104.36785278,405.14710297)
\curveto(104.27785152,405.15710294)(104.2028516,405.18710291)(104.14285278,405.23710297)
\curveto(104.09285171,405.28710281)(104.06285174,405.36210274)(104.05285278,405.46210297)
\curveto(104.04285176,405.56210254)(104.03785176,405.66710243)(104.03785278,405.77710297)
\lineto(104.03785278,407.08210297)
\lineto(104.03785278,412.55710297)
\lineto(104.03785278,414.74710297)
\curveto(104.03785176,414.88709321)(104.03285177,415.05209305)(104.02285278,415.24210297)
\curveto(104.02285178,415.43209267)(104.04785175,415.56709253)(104.09785278,415.64710297)
\curveto(104.13785166,415.70709239)(104.2028516,415.75709234)(104.29285278,415.79710297)
\curveto(104.32285148,415.7970923)(104.34785145,415.7970923)(104.36785278,415.79710297)
\curveto(104.3978514,415.80709229)(104.42285138,415.81209229)(104.44285278,415.81210297)
}
}
{
\newrgbcolor{curcolor}{0 0 0}
\pscustom[linestyle=none,fillstyle=solid,fillcolor=curcolor]
{
\newpath
\moveto(112.64668091,413.06710297)
\curveto(113.2466751,413.08709501)(113.7466746,413.0020951)(114.14668091,412.81210297)
\curveto(114.5466738,412.62209548)(114.86167349,412.34209576)(115.09168091,411.97210297)
\curveto(115.16167319,411.86209624)(115.21667313,411.74209636)(115.25668091,411.61210297)
\curveto(115.29667305,411.49209661)(115.33667301,411.36709673)(115.37668091,411.23710297)
\curveto(115.39667295,411.15709694)(115.40667294,411.08209702)(115.40668091,411.01210297)
\curveto(115.41667293,410.94209716)(115.43167292,410.87209723)(115.45168091,410.80210297)
\curveto(115.4516729,410.74209736)(115.45667289,410.7020974)(115.46668091,410.68210297)
\curveto(115.48667286,410.54209756)(115.49667285,410.3970977)(115.49668091,410.24710297)
\lineto(115.49668091,409.81210297)
\lineto(115.49668091,408.47710297)
\lineto(115.49668091,406.04710297)
\curveto(115.49667285,405.85710224)(115.49167286,405.67210243)(115.48168091,405.49210297)
\curveto(115.48167287,405.32210278)(115.41167294,405.21210289)(115.27168091,405.16210297)
\curveto(115.21167314,405.14210296)(115.14167321,405.13210297)(115.06168091,405.13210297)
\lineto(114.82168091,405.13210297)
\lineto(114.01168091,405.13210297)
\curveto(113.89167446,405.13210297)(113.78167457,405.13710296)(113.68168091,405.14710297)
\curveto(113.59167476,405.16710293)(113.52167483,405.21210289)(113.47168091,405.28210297)
\curveto(113.43167492,405.34210276)(113.40667494,405.41710268)(113.39668091,405.50710297)
\lineto(113.39668091,405.82210297)
\lineto(113.39668091,406.87210297)
\lineto(113.39668091,409.10710297)
\curveto(113.39667495,409.47709862)(113.38167497,409.81709828)(113.35168091,410.12710297)
\curveto(113.32167503,410.44709765)(113.23167512,410.71709738)(113.08168091,410.93710297)
\curveto(112.94167541,411.13709696)(112.73667561,411.27709682)(112.46668091,411.35710297)
\curveto(112.41667593,411.37709672)(112.36167599,411.38709671)(112.30168091,411.38710297)
\curveto(112.2516761,411.38709671)(112.19667615,411.3970967)(112.13668091,411.41710297)
\curveto(112.08667626,411.42709667)(112.02167633,411.42709667)(111.94168091,411.41710297)
\curveto(111.87167648,411.41709668)(111.81667653,411.41209669)(111.77668091,411.40210297)
\curveto(111.73667661,411.39209671)(111.70167665,411.38709671)(111.67168091,411.38710297)
\curveto(111.64167671,411.38709671)(111.61167674,411.38209672)(111.58168091,411.37210297)
\curveto(111.351677,411.31209679)(111.16667718,411.23209687)(111.02668091,411.13210297)
\curveto(110.70667764,410.9020972)(110.51667783,410.56709753)(110.45668091,410.12710297)
\curveto(110.39667795,409.68709841)(110.36667798,409.19209891)(110.36668091,408.64210297)
\lineto(110.36668091,406.76710297)
\lineto(110.36668091,405.85210297)
\lineto(110.36668091,405.58210297)
\curveto(110.36667798,405.49210261)(110.351678,405.41710268)(110.32168091,405.35710297)
\curveto(110.27167808,405.24710285)(110.19167816,405.18210292)(110.08168091,405.16210297)
\curveto(109.97167838,405.14210296)(109.83667851,405.13210297)(109.67668091,405.13210297)
\lineto(108.92668091,405.13210297)
\curveto(108.81667953,405.13210297)(108.70667964,405.13710296)(108.59668091,405.14710297)
\curveto(108.48667986,405.15710294)(108.40667994,405.19210291)(108.35668091,405.25210297)
\curveto(108.28668006,405.34210276)(108.2516801,405.47210263)(108.25168091,405.64210297)
\curveto(108.26168009,405.81210229)(108.26668008,405.97210213)(108.26668091,406.12210297)
\lineto(108.26668091,408.16210297)
\lineto(108.26668091,411.46210297)
\lineto(108.26668091,412.22710297)
\lineto(108.26668091,412.52710297)
\curveto(108.27668007,412.61709548)(108.30668004,412.69209541)(108.35668091,412.75210297)
\curveto(108.37667997,412.78209532)(108.40667994,412.8020953)(108.44668091,412.81210297)
\curveto(108.49667985,412.83209527)(108.5466798,412.84709525)(108.59668091,412.85710297)
\lineto(108.67168091,412.85710297)
\curveto(108.72167963,412.86709523)(108.77167958,412.87209523)(108.82168091,412.87210297)
\lineto(108.98668091,412.87210297)
\lineto(109.61668091,412.87210297)
\curveto(109.69667865,412.87209523)(109.77167858,412.86709523)(109.84168091,412.85710297)
\curveto(109.92167843,412.85709524)(109.99167836,412.84709525)(110.05168091,412.82710297)
\curveto(110.12167823,412.7970953)(110.16667818,412.75209535)(110.18668091,412.69210297)
\curveto(110.21667813,412.63209547)(110.24167811,412.56209554)(110.26168091,412.48210297)
\curveto(110.27167808,412.44209566)(110.27167808,412.40709569)(110.26168091,412.37710297)
\curveto(110.26167809,412.34709575)(110.27167808,412.31709578)(110.29168091,412.28710297)
\curveto(110.31167804,412.23709586)(110.32667802,412.20709589)(110.33668091,412.19710297)
\curveto(110.35667799,412.18709591)(110.38167797,412.17209593)(110.41168091,412.15210297)
\curveto(110.52167783,412.14209596)(110.61167774,412.17709592)(110.68168091,412.25710297)
\curveto(110.7516776,412.34709575)(110.82667752,412.41709568)(110.90668091,412.46710297)
\curveto(111.17667717,412.66709543)(111.47667687,412.82709527)(111.80668091,412.94710297)
\curveto(111.89667645,412.97709512)(111.98667636,412.9970951)(112.07668091,413.00710297)
\curveto(112.17667617,413.01709508)(112.28167607,413.03209507)(112.39168091,413.05210297)
\curveto(112.42167593,413.06209504)(112.46667588,413.06209504)(112.52668091,413.05210297)
\curveto(112.58667576,413.05209505)(112.62667572,413.05709504)(112.64668091,413.06710297)
}
}
{
\newrgbcolor{curcolor}{0 0 0}
\pscustom[linestyle=none,fillstyle=solid,fillcolor=curcolor]
{
\newpath
\moveto(118.17793091,415.18210297)
\lineto(119.18293091,415.18210297)
\curveto(119.33292792,415.18209292)(119.46292779,415.17209293)(119.57293091,415.15210297)
\curveto(119.69292756,415.14209296)(119.77792748,415.08209302)(119.82793091,414.97210297)
\curveto(119.84792741,414.92209318)(119.8579274,414.86209324)(119.85793091,414.79210297)
\lineto(119.85793091,414.58210297)
\lineto(119.85793091,413.90710297)
\curveto(119.8579274,413.85709424)(119.8529274,413.7970943)(119.84293091,413.72710297)
\curveto(119.84292741,413.66709443)(119.84792741,413.61209449)(119.85793091,413.56210297)
\lineto(119.85793091,413.39710297)
\curveto(119.8579274,413.31709478)(119.86292739,413.24209486)(119.87293091,413.17210297)
\curveto(119.88292737,413.11209499)(119.90792735,413.05709504)(119.94793091,413.00710297)
\curveto(120.01792724,412.91709518)(120.14292711,412.86709523)(120.32293091,412.85710297)
\lineto(120.86293091,412.85710297)
\lineto(121.04293091,412.85710297)
\curveto(121.10292615,412.85709524)(121.1579261,412.84709525)(121.20793091,412.82710297)
\curveto(121.31792594,412.77709532)(121.37792588,412.68709541)(121.38793091,412.55710297)
\curveto(121.40792585,412.42709567)(121.41792584,412.28209582)(121.41793091,412.12210297)
\lineto(121.41793091,411.91210297)
\curveto(121.42792583,411.84209626)(121.42292583,411.78209632)(121.40293091,411.73210297)
\curveto(121.3529259,411.57209653)(121.24792601,411.48709661)(121.08793091,411.47710297)
\curveto(120.92792633,411.46709663)(120.74792651,411.46209664)(120.54793091,411.46210297)
\lineto(120.41293091,411.46210297)
\curveto(120.37292688,411.47209663)(120.33792692,411.47209663)(120.30793091,411.46210297)
\curveto(120.26792699,411.45209665)(120.23292702,411.44709665)(120.20293091,411.44710297)
\curveto(120.17292708,411.45709664)(120.14292711,411.45209665)(120.11293091,411.43210297)
\curveto(120.03292722,411.41209669)(119.97292728,411.36709673)(119.93293091,411.29710297)
\curveto(119.90292735,411.23709686)(119.87792738,411.16209694)(119.85793091,411.07210297)
\curveto(119.84792741,411.02209708)(119.84792741,410.96709713)(119.85793091,410.90710297)
\curveto(119.86792739,410.84709725)(119.86792739,410.79209731)(119.85793091,410.74210297)
\lineto(119.85793091,409.81210297)
\lineto(119.85793091,408.05710297)
\curveto(119.8579274,407.80710029)(119.86292739,407.58710051)(119.87293091,407.39710297)
\curveto(119.89292736,407.21710088)(119.9579273,407.05710104)(120.06793091,406.91710297)
\curveto(120.11792714,406.85710124)(120.18292707,406.81210129)(120.26293091,406.78210297)
\lineto(120.53293091,406.72210297)
\curveto(120.56292669,406.71210139)(120.59292666,406.70710139)(120.62293091,406.70710297)
\curveto(120.66292659,406.71710138)(120.69292656,406.71710138)(120.71293091,406.70710297)
\lineto(120.87793091,406.70710297)
\curveto(120.98792627,406.70710139)(121.08292617,406.7021014)(121.16293091,406.69210297)
\curveto(121.24292601,406.68210142)(121.30792595,406.64210146)(121.35793091,406.57210297)
\curveto(121.39792586,406.51210159)(121.41792584,406.43210167)(121.41793091,406.33210297)
\lineto(121.41793091,406.04710297)
\curveto(121.41792584,405.83710226)(121.41292584,405.64210246)(121.40293091,405.46210297)
\curveto(121.40292585,405.29210281)(121.32292593,405.17710292)(121.16293091,405.11710297)
\curveto(121.11292614,405.097103)(121.06792619,405.09210301)(121.02793091,405.10210297)
\curveto(120.98792627,405.102103)(120.94292631,405.09210301)(120.89293091,405.07210297)
\lineto(120.74293091,405.07210297)
\curveto(120.72292653,405.07210303)(120.69292656,405.07710302)(120.65293091,405.08710297)
\curveto(120.61292664,405.08710301)(120.57792668,405.08210302)(120.54793091,405.07210297)
\curveto(120.49792676,405.06210304)(120.44292681,405.06210304)(120.38293091,405.07210297)
\lineto(120.23293091,405.07210297)
\lineto(120.08293091,405.07210297)
\curveto(120.03292722,405.06210304)(119.98792727,405.06210304)(119.94793091,405.07210297)
\lineto(119.78293091,405.07210297)
\curveto(119.73292752,405.08210302)(119.67792758,405.08710301)(119.61793091,405.08710297)
\curveto(119.5579277,405.08710301)(119.50292775,405.09210301)(119.45293091,405.10210297)
\curveto(119.38292787,405.11210299)(119.31792794,405.12210298)(119.25793091,405.13210297)
\lineto(119.07793091,405.16210297)
\curveto(118.96792829,405.19210291)(118.86292839,405.22710287)(118.76293091,405.26710297)
\curveto(118.66292859,405.30710279)(118.56792869,405.35210275)(118.47793091,405.40210297)
\lineto(118.38793091,405.46210297)
\curveto(118.3579289,405.49210261)(118.32292893,405.52210258)(118.28293091,405.55210297)
\curveto(118.26292899,405.57210253)(118.23792902,405.59210251)(118.20793091,405.61210297)
\lineto(118.13293091,405.68710297)
\curveto(117.99292926,405.87710222)(117.88792937,406.08710201)(117.81793091,406.31710297)
\curveto(117.79792946,406.35710174)(117.78792947,406.39210171)(117.78793091,406.42210297)
\curveto(117.79792946,406.46210164)(117.79792946,406.50710159)(117.78793091,406.55710297)
\curveto(117.77792948,406.57710152)(117.77292948,406.6021015)(117.77293091,406.63210297)
\curveto(117.77292948,406.66210144)(117.76792949,406.68710141)(117.75793091,406.70710297)
\lineto(117.75793091,406.85710297)
\curveto(117.74792951,406.8971012)(117.74292951,406.94210116)(117.74293091,406.99210297)
\curveto(117.7529295,407.04210106)(117.7579295,407.09210101)(117.75793091,407.14210297)
\lineto(117.75793091,407.71210297)
\lineto(117.75793091,409.94710297)
\lineto(117.75793091,410.74210297)
\lineto(117.75793091,410.95210297)
\curveto(117.76792949,411.02209708)(117.76292949,411.08709701)(117.74293091,411.14710297)
\curveto(117.70292955,411.28709681)(117.63292962,411.37709672)(117.53293091,411.41710297)
\curveto(117.42292983,411.46709663)(117.28292997,411.48209662)(117.11293091,411.46210297)
\curveto(116.94293031,411.44209666)(116.79793046,411.45709664)(116.67793091,411.50710297)
\curveto(116.59793066,411.53709656)(116.54793071,411.58209652)(116.52793091,411.64210297)
\curveto(116.50793075,411.7020964)(116.48793077,411.77709632)(116.46793091,411.86710297)
\lineto(116.46793091,412.18210297)
\curveto(116.46793079,412.36209574)(116.47793078,412.50709559)(116.49793091,412.61710297)
\curveto(116.51793074,412.72709537)(116.60293065,412.8020953)(116.75293091,412.84210297)
\curveto(116.79293046,412.86209524)(116.83293042,412.86709523)(116.87293091,412.85710297)
\lineto(117.00793091,412.85710297)
\curveto(117.1579301,412.85709524)(117.29792996,412.86209524)(117.42793091,412.87210297)
\curveto(117.5579297,412.89209521)(117.64792961,412.95209515)(117.69793091,413.05210297)
\curveto(117.72792953,413.12209498)(117.74292951,413.2020949)(117.74293091,413.29210297)
\curveto(117.7529295,413.38209472)(117.7579295,413.47209463)(117.75793091,413.56210297)
\lineto(117.75793091,414.49210297)
\lineto(117.75793091,414.74710297)
\curveto(117.7579295,414.83709326)(117.76792949,414.91209319)(117.78793091,414.97210297)
\curveto(117.83792942,415.07209303)(117.91292934,415.13709296)(118.01293091,415.16710297)
\curveto(118.03292922,415.17709292)(118.0579292,415.17709292)(118.08793091,415.16710297)
\curveto(118.12792913,415.16709293)(118.1579291,415.17209293)(118.17793091,415.18210297)
}
}
{
\newrgbcolor{curcolor}{0 0 0}
\pscustom[linestyle=none,fillstyle=solid,fillcolor=curcolor]
{
\newpath
\moveto(129.76636841,409.07710297)
\curveto(129.78636024,408.9970991)(129.78636024,408.90709919)(129.76636841,408.80710297)
\curveto(129.74636028,408.70709939)(129.71136032,408.64209946)(129.66136841,408.61210297)
\curveto(129.61136042,408.57209953)(129.53636049,408.54209956)(129.43636841,408.52210297)
\curveto(129.34636068,408.51209959)(129.24136079,408.5020996)(129.12136841,408.49210297)
\lineto(128.77636841,408.49210297)
\curveto(128.66636136,408.5020996)(128.56636146,408.50709959)(128.47636841,408.50710297)
\lineto(124.81636841,408.50710297)
\lineto(124.60636841,408.50710297)
\curveto(124.54636548,408.50709959)(124.49136554,408.4970996)(124.44136841,408.47710297)
\curveto(124.36136567,408.43709966)(124.31136572,408.3970997)(124.29136841,408.35710297)
\curveto(124.27136576,408.33709976)(124.25136578,408.2970998)(124.23136841,408.23710297)
\curveto(124.21136582,408.18709991)(124.20636582,408.13709996)(124.21636841,408.08710297)
\curveto(124.23636579,408.02710007)(124.24636578,407.96710013)(124.24636841,407.90710297)
\curveto(124.25636577,407.85710024)(124.27136576,407.8021003)(124.29136841,407.74210297)
\curveto(124.37136566,407.5021006)(124.46636556,407.3021008)(124.57636841,407.14210297)
\curveto(124.69636533,406.99210111)(124.85636517,406.85710124)(125.05636841,406.73710297)
\curveto(125.13636489,406.68710141)(125.21636481,406.65210145)(125.29636841,406.63210297)
\curveto(125.38636464,406.62210148)(125.47636455,406.6021015)(125.56636841,406.57210297)
\curveto(125.64636438,406.55210155)(125.75636427,406.53710156)(125.89636841,406.52710297)
\curveto(126.03636399,406.51710158)(126.15636387,406.52210158)(126.25636841,406.54210297)
\lineto(126.39136841,406.54210297)
\curveto(126.49136354,406.56210154)(126.58136345,406.58210152)(126.66136841,406.60210297)
\curveto(126.75136328,406.63210147)(126.83636319,406.66210144)(126.91636841,406.69210297)
\curveto(127.01636301,406.74210136)(127.1263629,406.80710129)(127.24636841,406.88710297)
\curveto(127.37636265,406.96710113)(127.47136256,407.04710105)(127.53136841,407.12710297)
\curveto(127.58136245,407.1971009)(127.6313624,407.26210084)(127.68136841,407.32210297)
\curveto(127.74136229,407.39210071)(127.81136222,407.44210066)(127.89136841,407.47210297)
\curveto(127.99136204,407.52210058)(128.11636191,407.54210056)(128.26636841,407.53210297)
\lineto(128.70136841,407.53210297)
\lineto(128.88136841,407.53210297)
\curveto(128.95136108,407.54210056)(129.01136102,407.53710056)(129.06136841,407.51710297)
\lineto(129.21136841,407.51710297)
\curveto(129.31136072,407.4971006)(129.38136065,407.47210063)(129.42136841,407.44210297)
\curveto(129.46136057,407.42210068)(129.48136055,407.37710072)(129.48136841,407.30710297)
\curveto(129.49136054,407.23710086)(129.48636054,407.17710092)(129.46636841,407.12710297)
\curveto(129.41636061,406.98710111)(129.36136067,406.86210124)(129.30136841,406.75210297)
\curveto(129.24136079,406.64210146)(129.17136086,406.53210157)(129.09136841,406.42210297)
\curveto(128.87136116,406.09210201)(128.62136141,405.82710227)(128.34136841,405.62710297)
\curveto(128.06136197,405.42710267)(127.71136232,405.25710284)(127.29136841,405.11710297)
\curveto(127.18136285,405.07710302)(127.07136296,405.05210305)(126.96136841,405.04210297)
\curveto(126.85136318,405.03210307)(126.73636329,405.01210309)(126.61636841,404.98210297)
\curveto(126.57636345,404.97210313)(126.5313635,404.97210313)(126.48136841,404.98210297)
\curveto(126.44136359,404.98210312)(126.40136363,404.97710312)(126.36136841,404.96710297)
\lineto(126.19636841,404.96710297)
\curveto(126.14636388,404.94710315)(126.08636394,404.94210316)(126.01636841,404.95210297)
\curveto(125.95636407,404.95210315)(125.90136413,404.95710314)(125.85136841,404.96710297)
\curveto(125.77136426,404.97710312)(125.70136433,404.97710312)(125.64136841,404.96710297)
\curveto(125.58136445,404.95710314)(125.51636451,404.96210314)(125.44636841,404.98210297)
\curveto(125.39636463,405.0021031)(125.34136469,405.01210309)(125.28136841,405.01210297)
\curveto(125.22136481,405.01210309)(125.16636486,405.02210308)(125.11636841,405.04210297)
\curveto(125.00636502,405.06210304)(124.89636513,405.08710301)(124.78636841,405.11710297)
\curveto(124.67636535,405.13710296)(124.57636545,405.17210293)(124.48636841,405.22210297)
\curveto(124.37636565,405.26210284)(124.27136576,405.2971028)(124.17136841,405.32710297)
\curveto(124.08136595,405.36710273)(123.99636603,405.41210269)(123.91636841,405.46210297)
\curveto(123.59636643,405.66210244)(123.31136672,405.89210221)(123.06136841,406.15210297)
\curveto(122.81136722,406.42210168)(122.60636742,406.73210137)(122.44636841,407.08210297)
\curveto(122.39636763,407.19210091)(122.35636767,407.3021008)(122.32636841,407.41210297)
\curveto(122.29636773,407.53210057)(122.25636777,407.65210045)(122.20636841,407.77210297)
\curveto(122.19636783,407.81210029)(122.19136784,407.84710025)(122.19136841,407.87710297)
\curveto(122.19136784,407.91710018)(122.18636784,407.95710014)(122.17636841,407.99710297)
\curveto(122.13636789,408.11709998)(122.11136792,408.24709985)(122.10136841,408.38710297)
\lineto(122.07136841,408.80710297)
\curveto(122.07136796,408.85709924)(122.06636796,408.91209919)(122.05636841,408.97210297)
\curveto(122.05636797,409.03209907)(122.06136797,409.08709901)(122.07136841,409.13710297)
\lineto(122.07136841,409.31710297)
\lineto(122.11636841,409.67710297)
\curveto(122.15636787,409.84709825)(122.19136784,410.01209809)(122.22136841,410.17210297)
\curveto(122.25136778,410.33209777)(122.29636773,410.48209762)(122.35636841,410.62210297)
\curveto(122.78636724,411.66209644)(123.51636651,412.3970957)(124.54636841,412.82710297)
\curveto(124.68636534,412.88709521)(124.8263652,412.92709517)(124.96636841,412.94710297)
\curveto(125.11636491,412.97709512)(125.27136476,413.01209509)(125.43136841,413.05210297)
\curveto(125.51136452,413.06209504)(125.58636444,413.06709503)(125.65636841,413.06710297)
\curveto(125.7263643,413.06709503)(125.80136423,413.07209503)(125.88136841,413.08210297)
\curveto(126.39136364,413.09209501)(126.8263632,413.03209507)(127.18636841,412.90210297)
\curveto(127.55636247,412.78209532)(127.88636214,412.62209548)(128.17636841,412.42210297)
\curveto(128.26636176,412.36209574)(128.35636167,412.29209581)(128.44636841,412.21210297)
\curveto(128.53636149,412.14209596)(128.61636141,412.06709603)(128.68636841,411.98710297)
\curveto(128.71636131,411.93709616)(128.75636127,411.8970962)(128.80636841,411.86710297)
\curveto(128.88636114,411.75709634)(128.96136107,411.64209646)(129.03136841,411.52210297)
\curveto(129.10136093,411.41209669)(129.17636085,411.2970968)(129.25636841,411.17710297)
\curveto(129.30636072,411.08709701)(129.34636068,410.99209711)(129.37636841,410.89210297)
\curveto(129.41636061,410.8020973)(129.45636057,410.7020974)(129.49636841,410.59210297)
\curveto(129.54636048,410.46209764)(129.58636044,410.32709777)(129.61636841,410.18710297)
\curveto(129.64636038,410.04709805)(129.68136035,409.90709819)(129.72136841,409.76710297)
\curveto(129.74136029,409.68709841)(129.74636028,409.5970985)(129.73636841,409.49710297)
\curveto(129.73636029,409.40709869)(129.74636028,409.32209878)(129.76636841,409.24210297)
\lineto(129.76636841,409.07710297)
\moveto(127.51636841,409.96210297)
\curveto(127.58636244,410.06209804)(127.59136244,410.18209792)(127.53136841,410.32210297)
\curveto(127.48136255,410.47209763)(127.44136259,410.58209752)(127.41136841,410.65210297)
\curveto(127.27136276,410.92209718)(127.08636294,411.12709697)(126.85636841,411.26710297)
\curveto(126.6263634,411.41709668)(126.30636372,411.4970966)(125.89636841,411.50710297)
\curveto(125.86636416,411.48709661)(125.8313642,411.48209662)(125.79136841,411.49210297)
\curveto(125.75136428,411.5020966)(125.71636431,411.5020966)(125.68636841,411.49210297)
\curveto(125.63636439,411.47209663)(125.58136445,411.45709664)(125.52136841,411.44710297)
\curveto(125.46136457,411.44709665)(125.40636462,411.43709666)(125.35636841,411.41710297)
\curveto(124.91636511,411.27709682)(124.59136544,411.0020971)(124.38136841,410.59210297)
\curveto(124.36136567,410.55209755)(124.33636569,410.4970976)(124.30636841,410.42710297)
\curveto(124.28636574,410.36709773)(124.27136576,410.3020978)(124.26136841,410.23210297)
\curveto(124.25136578,410.17209793)(124.25136578,410.11209799)(124.26136841,410.05210297)
\curveto(124.28136575,409.99209811)(124.31636571,409.94209816)(124.36636841,409.90210297)
\curveto(124.44636558,409.85209825)(124.55636547,409.82709827)(124.69636841,409.82710297)
\lineto(125.10136841,409.82710297)
\lineto(126.76636841,409.82710297)
\lineto(127.20136841,409.82710297)
\curveto(127.36136267,409.83709826)(127.46636256,409.88209822)(127.51636841,409.96210297)
}
}
{
\newrgbcolor{curcolor}{0 0 0}
\pscustom[linestyle=none,fillstyle=solid,fillcolor=curcolor]
{
\newpath
\moveto(135.43964966,413.06710297)
\curveto(136.03964385,413.08709501)(136.53964335,413.0020951)(136.93964966,412.81210297)
\curveto(137.33964255,412.62209548)(137.65464224,412.34209576)(137.88464966,411.97210297)
\curveto(137.95464194,411.86209624)(138.00964188,411.74209636)(138.04964966,411.61210297)
\curveto(138.0896418,411.49209661)(138.12964176,411.36709673)(138.16964966,411.23710297)
\curveto(138.1896417,411.15709694)(138.19964169,411.08209702)(138.19964966,411.01210297)
\curveto(138.20964168,410.94209716)(138.22464167,410.87209723)(138.24464966,410.80210297)
\curveto(138.24464165,410.74209736)(138.24964164,410.7020974)(138.25964966,410.68210297)
\curveto(138.27964161,410.54209756)(138.2896416,410.3970977)(138.28964966,410.24710297)
\lineto(138.28964966,409.81210297)
\lineto(138.28964966,408.47710297)
\lineto(138.28964966,406.04710297)
\curveto(138.2896416,405.85710224)(138.28464161,405.67210243)(138.27464966,405.49210297)
\curveto(138.27464162,405.32210278)(138.20464169,405.21210289)(138.06464966,405.16210297)
\curveto(138.00464189,405.14210296)(137.93464196,405.13210297)(137.85464966,405.13210297)
\lineto(137.61464966,405.13210297)
\lineto(136.80464966,405.13210297)
\curveto(136.68464321,405.13210297)(136.57464332,405.13710296)(136.47464966,405.14710297)
\curveto(136.38464351,405.16710293)(136.31464358,405.21210289)(136.26464966,405.28210297)
\curveto(136.22464367,405.34210276)(136.19964369,405.41710268)(136.18964966,405.50710297)
\lineto(136.18964966,405.82210297)
\lineto(136.18964966,406.87210297)
\lineto(136.18964966,409.10710297)
\curveto(136.1896437,409.47709862)(136.17464372,409.81709828)(136.14464966,410.12710297)
\curveto(136.11464378,410.44709765)(136.02464387,410.71709738)(135.87464966,410.93710297)
\curveto(135.73464416,411.13709696)(135.52964436,411.27709682)(135.25964966,411.35710297)
\curveto(135.20964468,411.37709672)(135.15464474,411.38709671)(135.09464966,411.38710297)
\curveto(135.04464485,411.38709671)(134.9896449,411.3970967)(134.92964966,411.41710297)
\curveto(134.87964501,411.42709667)(134.81464508,411.42709667)(134.73464966,411.41710297)
\curveto(134.66464523,411.41709668)(134.60964528,411.41209669)(134.56964966,411.40210297)
\curveto(134.52964536,411.39209671)(134.4946454,411.38709671)(134.46464966,411.38710297)
\curveto(134.43464546,411.38709671)(134.40464549,411.38209672)(134.37464966,411.37210297)
\curveto(134.14464575,411.31209679)(133.95964593,411.23209687)(133.81964966,411.13210297)
\curveto(133.49964639,410.9020972)(133.30964658,410.56709753)(133.24964966,410.12710297)
\curveto(133.1896467,409.68709841)(133.15964673,409.19209891)(133.15964966,408.64210297)
\lineto(133.15964966,406.76710297)
\lineto(133.15964966,405.85210297)
\lineto(133.15964966,405.58210297)
\curveto(133.15964673,405.49210261)(133.14464675,405.41710268)(133.11464966,405.35710297)
\curveto(133.06464683,405.24710285)(132.98464691,405.18210292)(132.87464966,405.16210297)
\curveto(132.76464713,405.14210296)(132.62964726,405.13210297)(132.46964966,405.13210297)
\lineto(131.71964966,405.13210297)
\curveto(131.60964828,405.13210297)(131.49964839,405.13710296)(131.38964966,405.14710297)
\curveto(131.27964861,405.15710294)(131.19964869,405.19210291)(131.14964966,405.25210297)
\curveto(131.07964881,405.34210276)(131.04464885,405.47210263)(131.04464966,405.64210297)
\curveto(131.05464884,405.81210229)(131.05964883,405.97210213)(131.05964966,406.12210297)
\lineto(131.05964966,408.16210297)
\lineto(131.05964966,411.46210297)
\lineto(131.05964966,412.22710297)
\lineto(131.05964966,412.52710297)
\curveto(131.06964882,412.61709548)(131.09964879,412.69209541)(131.14964966,412.75210297)
\curveto(131.16964872,412.78209532)(131.19964869,412.8020953)(131.23964966,412.81210297)
\curveto(131.2896486,412.83209527)(131.33964855,412.84709525)(131.38964966,412.85710297)
\lineto(131.46464966,412.85710297)
\curveto(131.51464838,412.86709523)(131.56464833,412.87209523)(131.61464966,412.87210297)
\lineto(131.77964966,412.87210297)
\lineto(132.40964966,412.87210297)
\curveto(132.4896474,412.87209523)(132.56464733,412.86709523)(132.63464966,412.85710297)
\curveto(132.71464718,412.85709524)(132.78464711,412.84709525)(132.84464966,412.82710297)
\curveto(132.91464698,412.7970953)(132.95964693,412.75209535)(132.97964966,412.69210297)
\curveto(133.00964688,412.63209547)(133.03464686,412.56209554)(133.05464966,412.48210297)
\curveto(133.06464683,412.44209566)(133.06464683,412.40709569)(133.05464966,412.37710297)
\curveto(133.05464684,412.34709575)(133.06464683,412.31709578)(133.08464966,412.28710297)
\curveto(133.10464679,412.23709586)(133.11964677,412.20709589)(133.12964966,412.19710297)
\curveto(133.14964674,412.18709591)(133.17464672,412.17209593)(133.20464966,412.15210297)
\curveto(133.31464658,412.14209596)(133.40464649,412.17709592)(133.47464966,412.25710297)
\curveto(133.54464635,412.34709575)(133.61964627,412.41709568)(133.69964966,412.46710297)
\curveto(133.96964592,412.66709543)(134.26964562,412.82709527)(134.59964966,412.94710297)
\curveto(134.6896452,412.97709512)(134.77964511,412.9970951)(134.86964966,413.00710297)
\curveto(134.96964492,413.01709508)(135.07464482,413.03209507)(135.18464966,413.05210297)
\curveto(135.21464468,413.06209504)(135.25964463,413.06209504)(135.31964966,413.05210297)
\curveto(135.37964451,413.05209505)(135.41964447,413.05709504)(135.43964966,413.06710297)
}
}
{
\newrgbcolor{curcolor}{0 0 0}
\pscustom[linestyle=none,fillstyle=solid,fillcolor=curcolor]
{
\newpath
\moveto(143.49089966,413.08210297)
\curveto(144.3008945,413.102095)(144.97589382,412.98209512)(145.51589966,412.72210297)
\curveto(146.06589273,412.46209564)(146.5008923,412.09209601)(146.82089966,411.61210297)
\curveto(146.98089182,411.37209673)(147.1008917,411.097097)(147.18089966,410.78710297)
\curveto(147.2008916,410.73709736)(147.21589158,410.67209743)(147.22589966,410.59210297)
\curveto(147.24589155,410.51209759)(147.24589155,410.44209766)(147.22589966,410.38210297)
\curveto(147.18589161,410.27209783)(147.11589168,410.20709789)(147.01589966,410.18710297)
\curveto(146.91589188,410.17709792)(146.795892,410.17209793)(146.65589966,410.17210297)
\lineto(145.87589966,410.17210297)
\lineto(145.59089966,410.17210297)
\curveto(145.5008933,410.17209793)(145.42589337,410.19209791)(145.36589966,410.23210297)
\curveto(145.28589351,410.27209783)(145.23089357,410.33209777)(145.20089966,410.41210297)
\curveto(145.17089363,410.5020976)(145.13089367,410.59209751)(145.08089966,410.68210297)
\curveto(145.02089378,410.79209731)(144.95589384,410.89209721)(144.88589966,410.98210297)
\curveto(144.81589398,411.07209703)(144.73589406,411.15209695)(144.64589966,411.22210297)
\curveto(144.50589429,411.31209679)(144.35089445,411.38209672)(144.18089966,411.43210297)
\curveto(144.12089468,411.45209665)(144.06089474,411.46209664)(144.00089966,411.46210297)
\curveto(143.94089486,411.46209664)(143.88589491,411.47209663)(143.83589966,411.49210297)
\lineto(143.68589966,411.49210297)
\curveto(143.48589531,411.49209661)(143.32589547,411.47209663)(143.20589966,411.43210297)
\curveto(142.91589588,411.34209676)(142.68089612,411.2020969)(142.50089966,411.01210297)
\curveto(142.32089648,410.83209727)(142.17589662,410.61209749)(142.06589966,410.35210297)
\curveto(142.01589678,410.24209786)(141.97589682,410.12209798)(141.94589966,409.99210297)
\curveto(141.92589687,409.87209823)(141.9008969,409.74209836)(141.87089966,409.60210297)
\curveto(141.86089694,409.56209854)(141.85589694,409.52209858)(141.85589966,409.48210297)
\curveto(141.85589694,409.44209866)(141.85089695,409.4020987)(141.84089966,409.36210297)
\curveto(141.82089698,409.26209884)(141.81089699,409.12209898)(141.81089966,408.94210297)
\curveto(141.82089698,408.76209934)(141.83589696,408.62209948)(141.85589966,408.52210297)
\curveto(141.85589694,408.44209966)(141.86089694,408.38709971)(141.87089966,408.35710297)
\curveto(141.89089691,408.28709981)(141.9008969,408.21709988)(141.90089966,408.14710297)
\curveto(141.91089689,408.07710002)(141.92589687,408.00710009)(141.94589966,407.93710297)
\curveto(142.02589677,407.70710039)(142.12089668,407.4971006)(142.23089966,407.30710297)
\curveto(142.34089646,407.11710098)(142.48089632,406.95710114)(142.65089966,406.82710297)
\curveto(142.69089611,406.7971013)(142.75089605,406.76210134)(142.83089966,406.72210297)
\curveto(142.94089586,406.65210145)(143.05089575,406.60710149)(143.16089966,406.58710297)
\curveto(143.28089552,406.56710153)(143.42589537,406.54710155)(143.59589966,406.52710297)
\lineto(143.68589966,406.52710297)
\curveto(143.72589507,406.52710157)(143.75589504,406.53210157)(143.77589966,406.54210297)
\lineto(143.91089966,406.54210297)
\curveto(143.98089482,406.56210154)(144.04589475,406.57710152)(144.10589966,406.58710297)
\curveto(144.17589462,406.60710149)(144.24089456,406.62710147)(144.30089966,406.64710297)
\curveto(144.6008942,406.77710132)(144.83089397,406.96710113)(144.99089966,407.21710297)
\curveto(145.03089377,407.26710083)(145.06589373,407.32210078)(145.09589966,407.38210297)
\curveto(145.12589367,407.45210065)(145.15089365,407.51210059)(145.17089966,407.56210297)
\curveto(145.21089359,407.67210043)(145.24589355,407.76710033)(145.27589966,407.84710297)
\curveto(145.30589349,407.93710016)(145.37589342,408.00710009)(145.48589966,408.05710297)
\curveto(145.57589322,408.0971)(145.72089308,408.11209999)(145.92089966,408.10210297)
\lineto(146.41589966,408.10210297)
\lineto(146.62589966,408.10210297)
\curveto(146.70589209,408.11209999)(146.77089203,408.10709999)(146.82089966,408.08710297)
\lineto(146.94089966,408.08710297)
\lineto(147.06089966,408.05710297)
\curveto(147.1008917,408.05710004)(147.13089167,408.04710005)(147.15089966,408.02710297)
\curveto(147.2008916,407.98710011)(147.23089157,407.92710017)(147.24089966,407.84710297)
\curveto(147.26089154,407.77710032)(147.26089154,407.7021004)(147.24089966,407.62210297)
\curveto(147.15089165,407.29210081)(147.04089176,406.9971011)(146.91089966,406.73710297)
\curveto(146.5008923,405.96710213)(145.84589295,405.43210267)(144.94589966,405.13210297)
\curveto(144.84589395,405.102103)(144.74089406,405.08210302)(144.63089966,405.07210297)
\curveto(144.52089428,405.05210305)(144.41089439,405.02710307)(144.30089966,404.99710297)
\curveto(144.24089456,404.98710311)(144.18089462,404.98210312)(144.12089966,404.98210297)
\curveto(144.06089474,404.98210312)(144.0008948,404.97710312)(143.94089966,404.96710297)
\lineto(143.77589966,404.96710297)
\curveto(143.72589507,404.94710315)(143.65089515,404.94210316)(143.55089966,404.95210297)
\curveto(143.45089535,404.95210315)(143.37589542,404.95710314)(143.32589966,404.96710297)
\curveto(143.24589555,404.98710311)(143.17089563,404.9971031)(143.10089966,404.99710297)
\curveto(143.04089576,404.98710311)(142.97589582,404.99210311)(142.90589966,405.01210297)
\lineto(142.75589966,405.04210297)
\curveto(142.70589609,405.04210306)(142.65589614,405.04710305)(142.60589966,405.05710297)
\curveto(142.4958963,405.08710301)(142.39089641,405.11710298)(142.29089966,405.14710297)
\curveto(142.19089661,405.17710292)(142.0958967,405.21210289)(142.00589966,405.25210297)
\curveto(141.53589726,405.45210265)(141.14089766,405.70710239)(140.82089966,406.01710297)
\curveto(140.5008983,406.33710176)(140.24089856,406.73210137)(140.04089966,407.20210297)
\curveto(139.99089881,407.29210081)(139.95089885,407.38710071)(139.92089966,407.48710297)
\lineto(139.83089966,407.81710297)
\curveto(139.82089898,407.85710024)(139.81589898,407.89210021)(139.81589966,407.92210297)
\curveto(139.81589898,407.96210014)(139.80589899,408.00710009)(139.78589966,408.05710297)
\curveto(139.76589903,408.12709997)(139.75589904,408.1970999)(139.75589966,408.26710297)
\curveto(139.75589904,408.34709975)(139.74589905,408.42209968)(139.72589966,408.49210297)
\lineto(139.72589966,408.74710297)
\curveto(139.70589909,408.7970993)(139.6958991,408.85209925)(139.69589966,408.91210297)
\curveto(139.6958991,408.98209912)(139.70589909,409.04209906)(139.72589966,409.09210297)
\curveto(139.73589906,409.14209896)(139.73589906,409.18709891)(139.72589966,409.22710297)
\curveto(139.71589908,409.26709883)(139.71589908,409.30709879)(139.72589966,409.34710297)
\curveto(139.74589905,409.41709868)(139.75089905,409.48209862)(139.74089966,409.54210297)
\curveto(139.74089906,409.6020985)(139.75089905,409.66209844)(139.77089966,409.72210297)
\curveto(139.82089898,409.9020982)(139.86089894,410.07209803)(139.89089966,410.23210297)
\curveto(139.92089888,410.4020977)(139.96589883,410.56709753)(140.02589966,410.72710297)
\curveto(140.24589855,411.23709686)(140.52089828,411.66209644)(140.85089966,412.00210297)
\curveto(141.19089761,412.34209576)(141.62089718,412.61709548)(142.14089966,412.82710297)
\curveto(142.28089652,412.88709521)(142.42589637,412.92709517)(142.57589966,412.94710297)
\curveto(142.72589607,412.97709512)(142.88089592,413.01209509)(143.04089966,413.05210297)
\curveto(143.12089568,413.06209504)(143.1958956,413.06709503)(143.26589966,413.06710297)
\curveto(143.33589546,413.06709503)(143.41089539,413.07209503)(143.49089966,413.08210297)
}
}
{
\newrgbcolor{curcolor}{0 0 0}
\pscustom[linestyle=none,fillstyle=solid,fillcolor=curcolor]
{
\newpath
\moveto(150.63418091,415.72210297)
\curveto(150.70417796,415.64209246)(150.73917792,415.52209258)(150.73918091,415.36210297)
\lineto(150.73918091,414.89710297)
\lineto(150.73918091,414.49210297)
\curveto(150.73917792,414.35209375)(150.70417796,414.25709384)(150.63418091,414.20710297)
\curveto(150.57417809,414.15709394)(150.49417817,414.12709397)(150.39418091,414.11710297)
\curveto(150.30417836,414.10709399)(150.20417846,414.102094)(150.09418091,414.10210297)
\lineto(149.25418091,414.10210297)
\curveto(149.14417952,414.102094)(149.04417962,414.10709399)(148.95418091,414.11710297)
\curveto(148.87417979,414.12709397)(148.80417986,414.15709394)(148.74418091,414.20710297)
\curveto(148.70417996,414.23709386)(148.67417999,414.29209381)(148.65418091,414.37210297)
\curveto(148.64418002,414.46209364)(148.63418003,414.55709354)(148.62418091,414.65710297)
\lineto(148.62418091,414.98710297)
\curveto(148.63418003,415.097093)(148.63918002,415.19209291)(148.63918091,415.27210297)
\lineto(148.63918091,415.48210297)
\curveto(148.64918001,415.55209255)(148.66917999,415.61209249)(148.69918091,415.66210297)
\curveto(148.71917994,415.7020924)(148.74417992,415.73209237)(148.77418091,415.75210297)
\lineto(148.89418091,415.81210297)
\curveto(148.91417975,415.81209229)(148.93917972,415.81209229)(148.96918091,415.81210297)
\curveto(148.99917966,415.82209228)(149.02417964,415.82709227)(149.04418091,415.82710297)
\lineto(150.13918091,415.82710297)
\curveto(150.23917842,415.82709227)(150.33417833,415.82209228)(150.42418091,415.81210297)
\curveto(150.51417815,415.8020923)(150.58417808,415.77209233)(150.63418091,415.72210297)
\moveto(150.73918091,405.95710297)
\curveto(150.73917792,405.75710234)(150.73417793,405.58710251)(150.72418091,405.44710297)
\curveto(150.71417795,405.30710279)(150.62417804,405.21210289)(150.45418091,405.16210297)
\curveto(150.39417827,405.14210296)(150.32917833,405.13210297)(150.25918091,405.13210297)
\curveto(150.18917847,405.14210296)(150.11417855,405.14710295)(150.03418091,405.14710297)
\lineto(149.19418091,405.14710297)
\curveto(149.10417956,405.14710295)(149.01417965,405.15210295)(148.92418091,405.16210297)
\curveto(148.84417982,405.17210293)(148.78417988,405.2021029)(148.74418091,405.25210297)
\curveto(148.68417998,405.32210278)(148.64918001,405.40710269)(148.63918091,405.50710297)
\lineto(148.63918091,405.85210297)
\lineto(148.63918091,412.18210297)
\lineto(148.63918091,412.48210297)
\curveto(148.63918002,412.58209552)(148.65918,412.66209544)(148.69918091,412.72210297)
\curveto(148.7591799,412.79209531)(148.84417982,412.83709526)(148.95418091,412.85710297)
\curveto(148.97417969,412.86709523)(148.99917966,412.86709523)(149.02918091,412.85710297)
\curveto(149.06917959,412.85709524)(149.09917956,412.86209524)(149.11918091,412.87210297)
\lineto(149.86918091,412.87210297)
\lineto(150.06418091,412.87210297)
\curveto(150.14417852,412.88209522)(150.20917845,412.88209522)(150.25918091,412.87210297)
\lineto(150.37918091,412.87210297)
\curveto(150.43917822,412.85209525)(150.49417817,412.83709526)(150.54418091,412.82710297)
\curveto(150.59417807,412.81709528)(150.63417803,412.78709531)(150.66418091,412.73710297)
\curveto(150.70417796,412.68709541)(150.72417794,412.61709548)(150.72418091,412.52710297)
\curveto(150.73417793,412.43709566)(150.73917792,412.34209576)(150.73918091,412.24210297)
\lineto(150.73918091,405.95710297)
}
}
{
\newrgbcolor{curcolor}{0 0 0}
\pscustom[linestyle=none,fillstyle=solid,fillcolor=curcolor]
{
\newpath
\moveto(160.17136841,409.31710297)
\curveto(160.15135988,409.36709873)(160.14635988,409.42209868)(160.15636841,409.48210297)
\curveto(160.16635986,409.54209856)(160.16135987,409.5970985)(160.14136841,409.64710297)
\curveto(160.1313599,409.68709841)(160.1263599,409.72709837)(160.12636841,409.76710297)
\curveto(160.1263599,409.80709829)(160.12135991,409.84709825)(160.11136841,409.88710297)
\lineto(160.05136841,410.15710297)
\curveto(160.03136,410.24709785)(160.00636002,410.33209777)(159.97636841,410.41210297)
\curveto(159.9263601,410.55209755)(159.88136015,410.68209742)(159.84136841,410.80210297)
\curveto(159.80136023,410.93209717)(159.74636028,411.05209705)(159.67636841,411.16210297)
\curveto(159.60636042,411.27209683)(159.53636049,411.37709672)(159.46636841,411.47710297)
\curveto(159.40636062,411.57709652)(159.33636069,411.67709642)(159.25636841,411.77710297)
\curveto(159.17636085,411.88709621)(159.07636095,411.98709611)(158.95636841,412.07710297)
\curveto(158.84636118,412.17709592)(158.73636129,412.26709583)(158.62636841,412.34710297)
\curveto(158.29636173,412.57709552)(157.91636211,412.75709534)(157.48636841,412.88710297)
\curveto(157.06636296,413.01709508)(156.56636346,413.07709502)(155.98636841,413.06710297)
\curveto(155.91636411,413.05709504)(155.84636418,413.05209505)(155.77636841,413.05210297)
\curveto(155.70636432,413.05209505)(155.6313644,413.04709505)(155.55136841,413.03710297)
\curveto(155.40136463,412.9970951)(155.25636477,412.96709513)(155.11636841,412.94710297)
\curveto(154.97636505,412.92709517)(154.84136519,412.89209521)(154.71136841,412.84210297)
\curveto(154.60136543,412.79209531)(154.49136554,412.74709535)(154.38136841,412.70710297)
\curveto(154.27136576,412.66709543)(154.16636586,412.62209548)(154.06636841,412.57210297)
\curveto(153.70636632,412.34209576)(153.40136663,412.08709601)(153.15136841,411.80710297)
\curveto(152.90136713,411.53709656)(152.68636734,411.1970969)(152.50636841,410.78710297)
\curveto(152.45636757,410.66709743)(152.41636761,410.54209756)(152.38636841,410.41210297)
\curveto(152.35636767,410.29209781)(152.32136771,410.16709793)(152.28136841,410.03710297)
\curveto(152.26136777,409.98709811)(152.25136778,409.93709816)(152.25136841,409.88710297)
\curveto(152.25136778,409.84709825)(152.24636778,409.8020983)(152.23636841,409.75210297)
\curveto(152.21636781,409.7020984)(152.20636782,409.64709845)(152.20636841,409.58710297)
\curveto(152.21636781,409.53709856)(152.21636781,409.48709861)(152.20636841,409.43710297)
\lineto(152.20636841,409.33210297)
\curveto(152.18636784,409.27209883)(152.17136786,409.18709891)(152.16136841,409.07710297)
\curveto(152.16136787,408.96709913)(152.17136786,408.88209922)(152.19136841,408.82210297)
\lineto(152.19136841,408.68710297)
\curveto(152.19136784,408.64709945)(152.19636783,408.6020995)(152.20636841,408.55210297)
\curveto(152.2263678,408.47209963)(152.23636779,408.38709971)(152.23636841,408.29710297)
\curveto(152.23636779,408.21709988)(152.24636778,408.13709996)(152.26636841,408.05710297)
\curveto(152.28636774,408.00710009)(152.29636773,407.96210014)(152.29636841,407.92210297)
\curveto(152.29636773,407.88210022)(152.30636772,407.83710026)(152.32636841,407.78710297)
\curveto(152.35636767,407.67710042)(152.38136765,407.57210053)(152.40136841,407.47210297)
\curveto(152.4313676,407.37210073)(152.47136756,407.27710082)(152.52136841,407.18710297)
\curveto(152.69136734,406.7971013)(152.90136713,406.46210164)(153.15136841,406.18210297)
\curveto(153.40136663,405.9021022)(153.70136633,405.65710244)(154.05136841,405.44710297)
\curveto(154.16136587,405.38710271)(154.26636576,405.33710276)(154.36636841,405.29710297)
\curveto(154.47636555,405.25710284)(154.59136544,405.21710288)(154.71136841,405.17710297)
\curveto(154.80136523,405.13710296)(154.89636513,405.10710299)(154.99636841,405.08710297)
\curveto(155.09636493,405.06710303)(155.19636483,405.04210306)(155.29636841,405.01210297)
\curveto(155.34636468,405.0021031)(155.38636464,404.9971031)(155.41636841,404.99710297)
\curveto(155.45636457,404.9971031)(155.49636453,404.99210311)(155.53636841,404.98210297)
\curveto(155.58636444,404.96210314)(155.63636439,404.95710314)(155.68636841,404.96710297)
\curveto(155.74636428,404.96710313)(155.80136423,404.96210314)(155.85136841,404.95210297)
\lineto(156.00136841,404.95210297)
\curveto(156.06136397,404.93210317)(156.14636388,404.92710317)(156.25636841,404.93710297)
\curveto(156.36636366,404.93710316)(156.44636358,404.94210316)(156.49636841,404.95210297)
\curveto(156.5263635,404.95210315)(156.55636347,404.95710314)(156.58636841,404.96710297)
\lineto(156.69136841,404.96710297)
\curveto(156.74136329,404.97710312)(156.79636323,404.98210312)(156.85636841,404.98210297)
\curveto(156.91636311,404.98210312)(156.97136306,404.99210311)(157.02136841,405.01210297)
\curveto(157.15136288,405.04210306)(157.27636275,405.07210303)(157.39636841,405.10210297)
\curveto(157.5263625,405.12210298)(157.65136238,405.15710294)(157.77136841,405.20710297)
\curveto(158.25136178,405.40710269)(158.66136137,405.65710244)(159.00136841,405.95710297)
\curveto(159.34136069,406.25710184)(159.61636041,406.64710145)(159.82636841,407.12710297)
\curveto(159.87636015,407.22710087)(159.91636011,407.33210077)(159.94636841,407.44210297)
\curveto(159.97636005,407.56210054)(160.01136002,407.67710042)(160.05136841,407.78710297)
\curveto(160.06135997,407.85710024)(160.07135996,407.92210018)(160.08136841,407.98210297)
\curveto(160.09135994,408.04210006)(160.10635992,408.10709999)(160.12636841,408.17710297)
\curveto(160.14635988,408.25709984)(160.15135988,408.33709976)(160.14136841,408.41710297)
\curveto(160.14135989,408.4970996)(160.15135988,408.57709952)(160.17136841,408.65710297)
\lineto(160.17136841,408.80710297)
\curveto(160.19135984,408.86709923)(160.20135983,408.95209915)(160.20136841,409.06210297)
\curveto(160.20135983,409.17209893)(160.19135984,409.25709884)(160.17136841,409.31710297)
\moveto(158.07136841,408.77710297)
\curveto(158.06136197,408.72709937)(158.05636197,408.67709942)(158.05636841,408.62710297)
\lineto(158.05636841,408.49210297)
\curveto(158.04636198,408.45209965)(158.04136199,408.41209969)(158.04136841,408.37210297)
\curveto(158.04136199,408.34209976)(158.03636199,408.30709979)(158.02636841,408.26710297)
\curveto(157.99636203,408.15709994)(157.97136206,408.05210005)(157.95136841,407.95210297)
\curveto(157.9313621,407.85210025)(157.90136213,407.75210035)(157.86136841,407.65210297)
\curveto(157.75136228,407.4021007)(157.61636241,407.19210091)(157.45636841,407.02210297)
\curveto(157.29636273,406.85210125)(157.08636294,406.71710138)(156.82636841,406.61710297)
\curveto(156.75636327,406.58710151)(156.68136335,406.56710153)(156.60136841,406.55710297)
\curveto(156.52136351,406.54710155)(156.44136359,406.53210157)(156.36136841,406.51210297)
\lineto(156.24136841,406.51210297)
\curveto(156.20136383,406.5021016)(156.15636387,406.4971016)(156.10636841,406.49710297)
\lineto(155.98636841,406.52710297)
\curveto(155.94636408,406.53710156)(155.91136412,406.53710156)(155.88136841,406.52710297)
\curveto(155.85136418,406.52710157)(155.81636421,406.53210157)(155.77636841,406.54210297)
\curveto(155.68636434,406.56210154)(155.59636443,406.58710151)(155.50636841,406.61710297)
\curveto(155.4263646,406.64710145)(155.35136468,406.68710141)(155.28136841,406.73710297)
\curveto(155.031365,406.88710121)(154.84636518,407.05210105)(154.72636841,407.23210297)
\curveto(154.61636541,407.42210068)(154.51136552,407.66710043)(154.41136841,407.96710297)
\curveto(154.39136564,408.04710005)(154.37636565,408.12209998)(154.36636841,408.19210297)
\curveto(154.35636567,408.27209983)(154.34136569,408.35209975)(154.32136841,408.43210297)
\lineto(154.32136841,408.56710297)
\curveto(154.30136573,408.63709946)(154.28636574,408.74209936)(154.27636841,408.88210297)
\curveto(154.27636575,409.02209908)(154.28636574,409.12709897)(154.30636841,409.19710297)
\lineto(154.30636841,409.34710297)
\curveto(154.30636572,409.3970987)(154.31136572,409.44709865)(154.32136841,409.49710297)
\curveto(154.34136569,409.60709849)(154.35636567,409.71709838)(154.36636841,409.82710297)
\curveto(154.38636564,409.93709816)(154.41136562,410.04209806)(154.44136841,410.14210297)
\curveto(154.5313655,410.41209769)(154.65136538,410.64709745)(154.80136841,410.84710297)
\curveto(154.96136507,411.05709704)(155.16636486,411.21709688)(155.41636841,411.32710297)
\curveto(155.46636456,411.35709674)(155.52136451,411.37709672)(155.58136841,411.38710297)
\lineto(155.79136841,411.44710297)
\curveto(155.82136421,411.45709664)(155.85636417,411.45709664)(155.89636841,411.44710297)
\curveto(155.93636409,411.44709665)(155.97136406,411.45709664)(156.00136841,411.47710297)
\lineto(156.27136841,411.47710297)
\curveto(156.36136367,411.48709661)(156.44636358,411.48209662)(156.52636841,411.46210297)
\curveto(156.59636343,411.44209666)(156.66136337,411.42209668)(156.72136841,411.40210297)
\curveto(156.78136325,411.39209671)(156.84136319,411.37709672)(156.90136841,411.35710297)
\curveto(157.15136288,411.24709685)(157.35136268,411.097097)(157.50136841,410.90710297)
\curveto(157.65136238,410.72709737)(157.78136225,410.50709759)(157.89136841,410.24710297)
\curveto(157.92136211,410.16709793)(157.94136209,410.08209802)(157.95136841,409.99210297)
\lineto(158.01136841,409.75210297)
\curveto(158.02136201,409.73209837)(158.026362,409.7020984)(158.02636841,409.66210297)
\curveto(158.03636199,409.61209849)(158.04136199,409.55709854)(158.04136841,409.49710297)
\curveto(158.04136199,409.43709866)(158.05136198,409.38209872)(158.07136841,409.33210297)
\lineto(158.07136841,409.21210297)
\curveto(158.08136195,409.16209894)(158.08636194,409.08709901)(158.08636841,408.98710297)
\curveto(158.08636194,408.8970992)(158.08136195,408.82709927)(158.07136841,408.77710297)
\moveto(156.84136841,415.94710297)
\lineto(157.90636841,415.94710297)
\curveto(157.98636204,415.94709215)(158.08136195,415.94709215)(158.19136841,415.94710297)
\curveto(158.30136173,415.94709215)(158.38136165,415.93209217)(158.43136841,415.90210297)
\curveto(158.45136158,415.89209221)(158.46136157,415.87709222)(158.46136841,415.85710297)
\curveto(158.47136156,415.84709225)(158.48636154,415.83709226)(158.50636841,415.82710297)
\curveto(158.51636151,415.70709239)(158.46636156,415.6020925)(158.35636841,415.51210297)
\curveto(158.25636177,415.42209268)(158.17136186,415.34209276)(158.10136841,415.27210297)
\curveto(158.02136201,415.2020929)(157.94136209,415.12709297)(157.86136841,415.04710297)
\curveto(157.79136224,414.97709312)(157.71636231,414.91209319)(157.63636841,414.85210297)
\curveto(157.59636243,414.82209328)(157.56136247,414.78709331)(157.53136841,414.74710297)
\curveto(157.51136252,414.71709338)(157.48136255,414.69209341)(157.44136841,414.67210297)
\curveto(157.42136261,414.64209346)(157.39636263,414.61709348)(157.36636841,414.59710297)
\lineto(157.21636841,414.44710297)
\lineto(157.06636841,414.32710297)
\lineto(157.02136841,414.28210297)
\curveto(157.02136301,414.27209383)(157.01136302,414.25709384)(156.99136841,414.23710297)
\curveto(156.91136312,414.17709392)(156.8313632,414.11209399)(156.75136841,414.04210297)
\curveto(156.68136335,413.97209413)(156.59136344,413.91709418)(156.48136841,413.87710297)
\curveto(156.44136359,413.86709423)(156.40136363,413.86209424)(156.36136841,413.86210297)
\curveto(156.3313637,413.86209424)(156.29136374,413.85709424)(156.24136841,413.84710297)
\curveto(156.21136382,413.83709426)(156.17136386,413.83209427)(156.12136841,413.83210297)
\curveto(156.07136396,413.84209426)(156.026364,413.84709425)(155.98636841,413.84710297)
\lineto(155.64136841,413.84710297)
\curveto(155.52136451,413.84709425)(155.4313646,413.87209423)(155.37136841,413.92210297)
\curveto(155.31136472,413.96209414)(155.29636473,414.03209407)(155.32636841,414.13210297)
\curveto(155.34636468,414.21209389)(155.38136465,414.28209382)(155.43136841,414.34210297)
\curveto(155.48136455,414.41209369)(155.5263645,414.48209362)(155.56636841,414.55210297)
\curveto(155.66636436,414.69209341)(155.76136427,414.82709327)(155.85136841,414.95710297)
\curveto(155.94136409,415.08709301)(156.031364,415.22209288)(156.12136841,415.36210297)
\curveto(156.17136386,415.44209266)(156.22136381,415.52709257)(156.27136841,415.61710297)
\curveto(156.3313637,415.70709239)(156.39636363,415.77709232)(156.46636841,415.82710297)
\curveto(156.50636352,415.85709224)(156.57636345,415.89209221)(156.67636841,415.93210297)
\curveto(156.69636333,415.94209216)(156.72136331,415.94209216)(156.75136841,415.93210297)
\curveto(156.79136324,415.93209217)(156.82136321,415.93709216)(156.84136841,415.94710297)
}
}
{
\newrgbcolor{curcolor}{0 0 0}
\pscustom[linestyle=none,fillstyle=solid,fillcolor=curcolor]
{
\newpath
\moveto(165.99629028,413.06710297)
\curveto(166.59628448,413.08709501)(167.09628398,413.0020951)(167.49629028,412.81210297)
\curveto(167.89628318,412.62209548)(168.21128286,412.34209576)(168.44129028,411.97210297)
\curveto(168.51128256,411.86209624)(168.56628251,411.74209636)(168.60629028,411.61210297)
\curveto(168.64628243,411.49209661)(168.68628239,411.36709673)(168.72629028,411.23710297)
\curveto(168.74628233,411.15709694)(168.75628232,411.08209702)(168.75629028,411.01210297)
\curveto(168.76628231,410.94209716)(168.78128229,410.87209723)(168.80129028,410.80210297)
\curveto(168.80128227,410.74209736)(168.80628227,410.7020974)(168.81629028,410.68210297)
\curveto(168.83628224,410.54209756)(168.84628223,410.3970977)(168.84629028,410.24710297)
\lineto(168.84629028,409.81210297)
\lineto(168.84629028,408.47710297)
\lineto(168.84629028,406.04710297)
\curveto(168.84628223,405.85710224)(168.84128223,405.67210243)(168.83129028,405.49210297)
\curveto(168.83128224,405.32210278)(168.76128231,405.21210289)(168.62129028,405.16210297)
\curveto(168.56128251,405.14210296)(168.49128258,405.13210297)(168.41129028,405.13210297)
\lineto(168.17129028,405.13210297)
\lineto(167.36129028,405.13210297)
\curveto(167.24128383,405.13210297)(167.13128394,405.13710296)(167.03129028,405.14710297)
\curveto(166.94128413,405.16710293)(166.8712842,405.21210289)(166.82129028,405.28210297)
\curveto(166.78128429,405.34210276)(166.75628432,405.41710268)(166.74629028,405.50710297)
\lineto(166.74629028,405.82210297)
\lineto(166.74629028,406.87210297)
\lineto(166.74629028,409.10710297)
\curveto(166.74628433,409.47709862)(166.73128434,409.81709828)(166.70129028,410.12710297)
\curveto(166.6712844,410.44709765)(166.58128449,410.71709738)(166.43129028,410.93710297)
\curveto(166.29128478,411.13709696)(166.08628499,411.27709682)(165.81629028,411.35710297)
\curveto(165.76628531,411.37709672)(165.71128536,411.38709671)(165.65129028,411.38710297)
\curveto(165.60128547,411.38709671)(165.54628553,411.3970967)(165.48629028,411.41710297)
\curveto(165.43628564,411.42709667)(165.3712857,411.42709667)(165.29129028,411.41710297)
\curveto(165.22128585,411.41709668)(165.16628591,411.41209669)(165.12629028,411.40210297)
\curveto(165.08628599,411.39209671)(165.05128602,411.38709671)(165.02129028,411.38710297)
\curveto(164.99128608,411.38709671)(164.96128611,411.38209672)(164.93129028,411.37210297)
\curveto(164.70128637,411.31209679)(164.51628656,411.23209687)(164.37629028,411.13210297)
\curveto(164.05628702,410.9020972)(163.86628721,410.56709753)(163.80629028,410.12710297)
\curveto(163.74628733,409.68709841)(163.71628736,409.19209891)(163.71629028,408.64210297)
\lineto(163.71629028,406.76710297)
\lineto(163.71629028,405.85210297)
\lineto(163.71629028,405.58210297)
\curveto(163.71628736,405.49210261)(163.70128737,405.41710268)(163.67129028,405.35710297)
\curveto(163.62128745,405.24710285)(163.54128753,405.18210292)(163.43129028,405.16210297)
\curveto(163.32128775,405.14210296)(163.18628789,405.13210297)(163.02629028,405.13210297)
\lineto(162.27629028,405.13210297)
\curveto(162.16628891,405.13210297)(162.05628902,405.13710296)(161.94629028,405.14710297)
\curveto(161.83628924,405.15710294)(161.75628932,405.19210291)(161.70629028,405.25210297)
\curveto(161.63628944,405.34210276)(161.60128947,405.47210263)(161.60129028,405.64210297)
\curveto(161.61128946,405.81210229)(161.61628946,405.97210213)(161.61629028,406.12210297)
\lineto(161.61629028,408.16210297)
\lineto(161.61629028,411.46210297)
\lineto(161.61629028,412.22710297)
\lineto(161.61629028,412.52710297)
\curveto(161.62628945,412.61709548)(161.65628942,412.69209541)(161.70629028,412.75210297)
\curveto(161.72628935,412.78209532)(161.75628932,412.8020953)(161.79629028,412.81210297)
\curveto(161.84628923,412.83209527)(161.89628918,412.84709525)(161.94629028,412.85710297)
\lineto(162.02129028,412.85710297)
\curveto(162.071289,412.86709523)(162.12128895,412.87209523)(162.17129028,412.87210297)
\lineto(162.33629028,412.87210297)
\lineto(162.96629028,412.87210297)
\curveto(163.04628803,412.87209523)(163.12128795,412.86709523)(163.19129028,412.85710297)
\curveto(163.2712878,412.85709524)(163.34128773,412.84709525)(163.40129028,412.82710297)
\curveto(163.4712876,412.7970953)(163.51628756,412.75209535)(163.53629028,412.69210297)
\curveto(163.56628751,412.63209547)(163.59128748,412.56209554)(163.61129028,412.48210297)
\curveto(163.62128745,412.44209566)(163.62128745,412.40709569)(163.61129028,412.37710297)
\curveto(163.61128746,412.34709575)(163.62128745,412.31709578)(163.64129028,412.28710297)
\curveto(163.66128741,412.23709586)(163.6762874,412.20709589)(163.68629028,412.19710297)
\curveto(163.70628737,412.18709591)(163.73128734,412.17209593)(163.76129028,412.15210297)
\curveto(163.8712872,412.14209596)(163.96128711,412.17709592)(164.03129028,412.25710297)
\curveto(164.10128697,412.34709575)(164.1762869,412.41709568)(164.25629028,412.46710297)
\curveto(164.52628655,412.66709543)(164.82628625,412.82709527)(165.15629028,412.94710297)
\curveto(165.24628583,412.97709512)(165.33628574,412.9970951)(165.42629028,413.00710297)
\curveto(165.52628555,413.01709508)(165.63128544,413.03209507)(165.74129028,413.05210297)
\curveto(165.7712853,413.06209504)(165.81628526,413.06209504)(165.87629028,413.05210297)
\curveto(165.93628514,413.05209505)(165.9762851,413.05709504)(165.99629028,413.06710297)
}
}
{
\newrgbcolor{curcolor}{0 0 0}
\pscustom[linestyle=none,fillstyle=solid,fillcolor=curcolor]
{
}
}
{
\newrgbcolor{curcolor}{0 0 0}
\pscustom[linestyle=none,fillstyle=solid,fillcolor=curcolor]
{
\newpath
\moveto(182.22769653,405.98710297)
\lineto(182.22769653,405.56710297)
\curveto(182.22768816,405.43710266)(182.19768819,405.33210277)(182.13769653,405.25210297)
\curveto(182.0876883,405.2021029)(182.02268837,405.16710293)(181.94269653,405.14710297)
\curveto(181.86268853,405.13710296)(181.77268862,405.13210297)(181.67269653,405.13210297)
\lineto(180.84769653,405.13210297)
\lineto(180.56269653,405.13210297)
\curveto(180.48268991,405.14210296)(180.41768997,405.16710293)(180.36769653,405.20710297)
\curveto(180.29769009,405.25710284)(180.25769013,405.32210278)(180.24769653,405.40210297)
\curveto(180.23769015,405.48210262)(180.21769017,405.56210254)(180.18769653,405.64210297)
\curveto(180.16769022,405.66210244)(180.14769024,405.67710242)(180.12769653,405.68710297)
\curveto(180.11769027,405.70710239)(180.10269029,405.72710237)(180.08269653,405.74710297)
\curveto(179.97269042,405.74710235)(179.8926905,405.72210238)(179.84269653,405.67210297)
\lineto(179.69269653,405.52210297)
\curveto(179.62269077,405.47210263)(179.55769083,405.42710267)(179.49769653,405.38710297)
\curveto(179.43769095,405.35710274)(179.37269102,405.31710278)(179.30269653,405.26710297)
\curveto(179.26269113,405.24710285)(179.21769117,405.22710287)(179.16769653,405.20710297)
\curveto(179.12769126,405.18710291)(179.08269131,405.16710293)(179.03269653,405.14710297)
\curveto(178.8926915,405.097103)(178.74269165,405.05210305)(178.58269653,405.01210297)
\curveto(178.53269186,404.99210311)(178.4876919,404.98210312)(178.44769653,404.98210297)
\curveto(178.40769198,404.98210312)(178.36769202,404.97710312)(178.32769653,404.96710297)
\lineto(178.19269653,404.96710297)
\curveto(178.16269223,404.95710314)(178.12269227,404.95210315)(178.07269653,404.95210297)
\lineto(177.93769653,404.95210297)
\curveto(177.87769251,404.93210317)(177.7876926,404.92710317)(177.66769653,404.93710297)
\curveto(177.54769284,404.93710316)(177.46269293,404.94710315)(177.41269653,404.96710297)
\curveto(177.34269305,404.98710311)(177.27769311,404.9971031)(177.21769653,404.99710297)
\curveto(177.16769322,404.98710311)(177.11269328,404.99210311)(177.05269653,405.01210297)
\lineto(176.69269653,405.13210297)
\curveto(176.58269381,405.16210294)(176.47269392,405.2021029)(176.36269653,405.25210297)
\curveto(176.01269438,405.4021027)(175.69769469,405.63210247)(175.41769653,405.94210297)
\curveto(175.14769524,406.26210184)(174.93269546,406.5971015)(174.77269653,406.94710297)
\curveto(174.72269567,407.05710104)(174.68269571,407.16210094)(174.65269653,407.26210297)
\curveto(174.62269577,407.37210073)(174.5876958,407.48210062)(174.54769653,407.59210297)
\curveto(174.53769585,407.63210047)(174.53269586,407.66710043)(174.53269653,407.69710297)
\curveto(174.53269586,407.73710036)(174.52269587,407.78210032)(174.50269653,407.83210297)
\curveto(174.48269591,407.91210019)(174.46269593,407.9971001)(174.44269653,408.08710297)
\curveto(174.43269596,408.18709991)(174.41769597,408.28709981)(174.39769653,408.38710297)
\curveto(174.387696,408.41709968)(174.38269601,408.45209965)(174.38269653,408.49210297)
\curveto(174.392696,408.53209957)(174.392696,408.56709953)(174.38269653,408.59710297)
\lineto(174.38269653,408.73210297)
\curveto(174.38269601,408.78209932)(174.37769601,408.83209927)(174.36769653,408.88210297)
\curveto(174.35769603,408.93209917)(174.35269604,408.98709911)(174.35269653,409.04710297)
\curveto(174.35269604,409.11709898)(174.35769603,409.17209893)(174.36769653,409.21210297)
\curveto(174.37769601,409.26209884)(174.38269601,409.30709879)(174.38269653,409.34710297)
\lineto(174.38269653,409.49710297)
\curveto(174.392696,409.54709855)(174.392696,409.59209851)(174.38269653,409.63210297)
\curveto(174.38269601,409.68209842)(174.392696,409.73209837)(174.41269653,409.78210297)
\curveto(174.43269596,409.89209821)(174.44769594,409.9970981)(174.45769653,410.09710297)
\curveto(174.47769591,410.1970979)(174.50269589,410.2970978)(174.53269653,410.39710297)
\curveto(174.57269582,410.51709758)(174.60769578,410.63209747)(174.63769653,410.74210297)
\curveto(174.66769572,410.85209725)(174.70769568,410.96209714)(174.75769653,411.07210297)
\curveto(174.89769549,411.37209673)(175.07269532,411.65709644)(175.28269653,411.92710297)
\curveto(175.30269509,411.95709614)(175.32769506,411.98209612)(175.35769653,412.00210297)
\curveto(175.39769499,412.03209607)(175.42769496,412.06209604)(175.44769653,412.09210297)
\curveto(175.4876949,412.14209596)(175.52769486,412.18709591)(175.56769653,412.22710297)
\curveto(175.60769478,412.26709583)(175.65269474,412.30709579)(175.70269653,412.34710297)
\curveto(175.74269465,412.36709573)(175.77769461,412.39209571)(175.80769653,412.42210297)
\curveto(175.83769455,412.46209564)(175.87269452,412.49209561)(175.91269653,412.51210297)
\curveto(176.16269423,412.68209542)(176.45269394,412.82209528)(176.78269653,412.93210297)
\curveto(176.85269354,412.95209515)(176.92269347,412.96709513)(176.99269653,412.97710297)
\curveto(177.07269332,412.98709511)(177.15269324,413.0020951)(177.23269653,413.02210297)
\curveto(177.30269309,413.04209506)(177.392693,413.05209505)(177.50269653,413.05210297)
\curveto(177.61269278,413.06209504)(177.72269267,413.06709503)(177.83269653,413.06710297)
\curveto(177.94269245,413.06709503)(178.04769234,413.06209504)(178.14769653,413.05210297)
\curveto(178.25769213,413.04209506)(178.34769204,413.02709507)(178.41769653,413.00710297)
\curveto(178.56769182,412.95709514)(178.71269168,412.91209519)(178.85269653,412.87210297)
\curveto(178.9926914,412.83209527)(179.12269127,412.77709532)(179.24269653,412.70710297)
\curveto(179.31269108,412.65709544)(179.37769101,412.60709549)(179.43769653,412.55710297)
\curveto(179.49769089,412.51709558)(179.56269083,412.47209563)(179.63269653,412.42210297)
\curveto(179.67269072,412.39209571)(179.72769066,412.35209575)(179.79769653,412.30210297)
\curveto(179.87769051,412.25209585)(179.95269044,412.25209585)(180.02269653,412.30210297)
\curveto(180.06269033,412.32209578)(180.08269031,412.35709574)(180.08269653,412.40710297)
\curveto(180.08269031,412.45709564)(180.0926903,412.50709559)(180.11269653,412.55710297)
\lineto(180.11269653,412.70710297)
\curveto(180.12269027,412.73709536)(180.12769026,412.77209533)(180.12769653,412.81210297)
\lineto(180.12769653,412.93210297)
\lineto(180.12769653,414.97210297)
\curveto(180.12769026,415.08209302)(180.12269027,415.2020929)(180.11269653,415.33210297)
\curveto(180.11269028,415.47209263)(180.13769025,415.57709252)(180.18769653,415.64710297)
\curveto(180.22769016,415.72709237)(180.30269009,415.77709232)(180.41269653,415.79710297)
\curveto(180.43268996,415.80709229)(180.45268994,415.80709229)(180.47269653,415.79710297)
\curveto(180.4926899,415.7970923)(180.51268988,415.8020923)(180.53269653,415.81210297)
\lineto(181.59769653,415.81210297)
\curveto(181.71768867,415.81209229)(181.82768856,415.80709229)(181.92769653,415.79710297)
\curveto(182.02768836,415.78709231)(182.10268829,415.74709235)(182.15269653,415.67710297)
\curveto(182.20268819,415.5970925)(182.22768816,415.49209261)(182.22769653,415.36210297)
\lineto(182.22769653,415.00210297)
\lineto(182.22769653,405.98710297)
\moveto(180.18769653,408.92710297)
\curveto(180.19769019,408.96709913)(180.19769019,409.00709909)(180.18769653,409.04710297)
\lineto(180.18769653,409.18210297)
\curveto(180.1876902,409.28209882)(180.18269021,409.38209872)(180.17269653,409.48210297)
\curveto(180.16269023,409.58209852)(180.14769024,409.67209843)(180.12769653,409.75210297)
\curveto(180.10769028,409.86209824)(180.0876903,409.96209814)(180.06769653,410.05210297)
\curveto(180.05769033,410.14209796)(180.03269036,410.22709787)(179.99269653,410.30710297)
\curveto(179.85269054,410.66709743)(179.64769074,410.95209715)(179.37769653,411.16210297)
\curveto(179.11769127,411.37209673)(178.73769165,411.47709662)(178.23769653,411.47710297)
\curveto(178.17769221,411.47709662)(178.09769229,411.46709663)(177.99769653,411.44710297)
\curveto(177.91769247,411.42709667)(177.84269255,411.40709669)(177.77269653,411.38710297)
\curveto(177.71269268,411.37709672)(177.65269274,411.35709674)(177.59269653,411.32710297)
\curveto(177.32269307,411.21709688)(177.11269328,411.04709705)(176.96269653,410.81710297)
\curveto(176.81269358,410.58709751)(176.6926937,410.32709777)(176.60269653,410.03710297)
\curveto(176.57269382,409.93709816)(176.55269384,409.83709826)(176.54269653,409.73710297)
\curveto(176.53269386,409.63709846)(176.51269388,409.53209857)(176.48269653,409.42210297)
\lineto(176.48269653,409.21210297)
\curveto(176.46269393,409.12209898)(176.45769393,408.9970991)(176.46769653,408.83710297)
\curveto(176.47769391,408.68709941)(176.4926939,408.57709952)(176.51269653,408.50710297)
\lineto(176.51269653,408.41710297)
\curveto(176.52269387,408.3970997)(176.52769386,408.37709972)(176.52769653,408.35710297)
\curveto(176.54769384,408.27709982)(176.56269383,408.2020999)(176.57269653,408.13210297)
\curveto(176.5926938,408.06210004)(176.61269378,407.98710011)(176.63269653,407.90710297)
\curveto(176.80269359,407.38710071)(177.0926933,407.0021011)(177.50269653,406.75210297)
\curveto(177.63269276,406.66210144)(177.81269258,406.59210151)(178.04269653,406.54210297)
\curveto(178.08269231,406.53210157)(178.14269225,406.52710157)(178.22269653,406.52710297)
\curveto(178.25269214,406.51710158)(178.29769209,406.50710159)(178.35769653,406.49710297)
\curveto(178.42769196,406.4971016)(178.48269191,406.5021016)(178.52269653,406.51210297)
\curveto(178.60269179,406.53210157)(178.68269171,406.54710155)(178.76269653,406.55710297)
\curveto(178.84269155,406.56710153)(178.92269147,406.58710151)(179.00269653,406.61710297)
\curveto(179.25269114,406.72710137)(179.45269094,406.86710123)(179.60269653,407.03710297)
\curveto(179.75269064,407.20710089)(179.88269051,407.42210068)(179.99269653,407.68210297)
\curveto(180.03269036,407.77210033)(180.06269033,407.86210024)(180.08269653,407.95210297)
\curveto(180.10269029,408.05210005)(180.12269027,408.15709994)(180.14269653,408.26710297)
\curveto(180.15269024,408.31709978)(180.15269024,408.36209974)(180.14269653,408.40210297)
\curveto(180.14269025,408.45209965)(180.15269024,408.5020996)(180.17269653,408.55210297)
\curveto(180.18269021,408.58209952)(180.1876902,408.61709948)(180.18769653,408.65710297)
\lineto(180.18769653,408.79210297)
\lineto(180.18769653,408.92710297)
}
}
{
\newrgbcolor{curcolor}{0 0 0}
\pscustom[linestyle=none,fillstyle=solid,fillcolor=curcolor]
{
\newpath
\moveto(191.17261841,409.07710297)
\curveto(191.19261024,408.9970991)(191.19261024,408.90709919)(191.17261841,408.80710297)
\curveto(191.15261028,408.70709939)(191.11761032,408.64209946)(191.06761841,408.61210297)
\curveto(191.01761042,408.57209953)(190.94261049,408.54209956)(190.84261841,408.52210297)
\curveto(190.75261068,408.51209959)(190.64761079,408.5020996)(190.52761841,408.49210297)
\lineto(190.18261841,408.49210297)
\curveto(190.07261136,408.5020996)(189.97261146,408.50709959)(189.88261841,408.50710297)
\lineto(186.22261841,408.50710297)
\lineto(186.01261841,408.50710297)
\curveto(185.95261548,408.50709959)(185.89761554,408.4970996)(185.84761841,408.47710297)
\curveto(185.76761567,408.43709966)(185.71761572,408.3970997)(185.69761841,408.35710297)
\curveto(185.67761576,408.33709976)(185.65761578,408.2970998)(185.63761841,408.23710297)
\curveto(185.61761582,408.18709991)(185.61261582,408.13709996)(185.62261841,408.08710297)
\curveto(185.64261579,408.02710007)(185.65261578,407.96710013)(185.65261841,407.90710297)
\curveto(185.66261577,407.85710024)(185.67761576,407.8021003)(185.69761841,407.74210297)
\curveto(185.77761566,407.5021006)(185.87261556,407.3021008)(185.98261841,407.14210297)
\curveto(186.10261533,406.99210111)(186.26261517,406.85710124)(186.46261841,406.73710297)
\curveto(186.54261489,406.68710141)(186.62261481,406.65210145)(186.70261841,406.63210297)
\curveto(186.79261464,406.62210148)(186.88261455,406.6021015)(186.97261841,406.57210297)
\curveto(187.05261438,406.55210155)(187.16261427,406.53710156)(187.30261841,406.52710297)
\curveto(187.44261399,406.51710158)(187.56261387,406.52210158)(187.66261841,406.54210297)
\lineto(187.79761841,406.54210297)
\curveto(187.89761354,406.56210154)(187.98761345,406.58210152)(188.06761841,406.60210297)
\curveto(188.15761328,406.63210147)(188.24261319,406.66210144)(188.32261841,406.69210297)
\curveto(188.42261301,406.74210136)(188.5326129,406.80710129)(188.65261841,406.88710297)
\curveto(188.78261265,406.96710113)(188.87761256,407.04710105)(188.93761841,407.12710297)
\curveto(188.98761245,407.1971009)(189.0376124,407.26210084)(189.08761841,407.32210297)
\curveto(189.14761229,407.39210071)(189.21761222,407.44210066)(189.29761841,407.47210297)
\curveto(189.39761204,407.52210058)(189.52261191,407.54210056)(189.67261841,407.53210297)
\lineto(190.10761841,407.53210297)
\lineto(190.28761841,407.53210297)
\curveto(190.35761108,407.54210056)(190.41761102,407.53710056)(190.46761841,407.51710297)
\lineto(190.61761841,407.51710297)
\curveto(190.71761072,407.4971006)(190.78761065,407.47210063)(190.82761841,407.44210297)
\curveto(190.86761057,407.42210068)(190.88761055,407.37710072)(190.88761841,407.30710297)
\curveto(190.89761054,407.23710086)(190.89261054,407.17710092)(190.87261841,407.12710297)
\curveto(190.82261061,406.98710111)(190.76761067,406.86210124)(190.70761841,406.75210297)
\curveto(190.64761079,406.64210146)(190.57761086,406.53210157)(190.49761841,406.42210297)
\curveto(190.27761116,406.09210201)(190.02761141,405.82710227)(189.74761841,405.62710297)
\curveto(189.46761197,405.42710267)(189.11761232,405.25710284)(188.69761841,405.11710297)
\curveto(188.58761285,405.07710302)(188.47761296,405.05210305)(188.36761841,405.04210297)
\curveto(188.25761318,405.03210307)(188.14261329,405.01210309)(188.02261841,404.98210297)
\curveto(187.98261345,404.97210313)(187.9376135,404.97210313)(187.88761841,404.98210297)
\curveto(187.84761359,404.98210312)(187.80761363,404.97710312)(187.76761841,404.96710297)
\lineto(187.60261841,404.96710297)
\curveto(187.55261388,404.94710315)(187.49261394,404.94210316)(187.42261841,404.95210297)
\curveto(187.36261407,404.95210315)(187.30761413,404.95710314)(187.25761841,404.96710297)
\curveto(187.17761426,404.97710312)(187.10761433,404.97710312)(187.04761841,404.96710297)
\curveto(186.98761445,404.95710314)(186.92261451,404.96210314)(186.85261841,404.98210297)
\curveto(186.80261463,405.0021031)(186.74761469,405.01210309)(186.68761841,405.01210297)
\curveto(186.62761481,405.01210309)(186.57261486,405.02210308)(186.52261841,405.04210297)
\curveto(186.41261502,405.06210304)(186.30261513,405.08710301)(186.19261841,405.11710297)
\curveto(186.08261535,405.13710296)(185.98261545,405.17210293)(185.89261841,405.22210297)
\curveto(185.78261565,405.26210284)(185.67761576,405.2971028)(185.57761841,405.32710297)
\curveto(185.48761595,405.36710273)(185.40261603,405.41210269)(185.32261841,405.46210297)
\curveto(185.00261643,405.66210244)(184.71761672,405.89210221)(184.46761841,406.15210297)
\curveto(184.21761722,406.42210168)(184.01261742,406.73210137)(183.85261841,407.08210297)
\curveto(183.80261763,407.19210091)(183.76261767,407.3021008)(183.73261841,407.41210297)
\curveto(183.70261773,407.53210057)(183.66261777,407.65210045)(183.61261841,407.77210297)
\curveto(183.60261783,407.81210029)(183.59761784,407.84710025)(183.59761841,407.87710297)
\curveto(183.59761784,407.91710018)(183.59261784,407.95710014)(183.58261841,407.99710297)
\curveto(183.54261789,408.11709998)(183.51761792,408.24709985)(183.50761841,408.38710297)
\lineto(183.47761841,408.80710297)
\curveto(183.47761796,408.85709924)(183.47261796,408.91209919)(183.46261841,408.97210297)
\curveto(183.46261797,409.03209907)(183.46761797,409.08709901)(183.47761841,409.13710297)
\lineto(183.47761841,409.31710297)
\lineto(183.52261841,409.67710297)
\curveto(183.56261787,409.84709825)(183.59761784,410.01209809)(183.62761841,410.17210297)
\curveto(183.65761778,410.33209777)(183.70261773,410.48209762)(183.76261841,410.62210297)
\curveto(184.19261724,411.66209644)(184.92261651,412.3970957)(185.95261841,412.82710297)
\curveto(186.09261534,412.88709521)(186.2326152,412.92709517)(186.37261841,412.94710297)
\curveto(186.52261491,412.97709512)(186.67761476,413.01209509)(186.83761841,413.05210297)
\curveto(186.91761452,413.06209504)(186.99261444,413.06709503)(187.06261841,413.06710297)
\curveto(187.1326143,413.06709503)(187.20761423,413.07209503)(187.28761841,413.08210297)
\curveto(187.79761364,413.09209501)(188.2326132,413.03209507)(188.59261841,412.90210297)
\curveto(188.96261247,412.78209532)(189.29261214,412.62209548)(189.58261841,412.42210297)
\curveto(189.67261176,412.36209574)(189.76261167,412.29209581)(189.85261841,412.21210297)
\curveto(189.94261149,412.14209596)(190.02261141,412.06709603)(190.09261841,411.98710297)
\curveto(190.12261131,411.93709616)(190.16261127,411.8970962)(190.21261841,411.86710297)
\curveto(190.29261114,411.75709634)(190.36761107,411.64209646)(190.43761841,411.52210297)
\curveto(190.50761093,411.41209669)(190.58261085,411.2970968)(190.66261841,411.17710297)
\curveto(190.71261072,411.08709701)(190.75261068,410.99209711)(190.78261841,410.89210297)
\curveto(190.82261061,410.8020973)(190.86261057,410.7020974)(190.90261841,410.59210297)
\curveto(190.95261048,410.46209764)(190.99261044,410.32709777)(191.02261841,410.18710297)
\curveto(191.05261038,410.04709805)(191.08761035,409.90709819)(191.12761841,409.76710297)
\curveto(191.14761029,409.68709841)(191.15261028,409.5970985)(191.14261841,409.49710297)
\curveto(191.14261029,409.40709869)(191.15261028,409.32209878)(191.17261841,409.24210297)
\lineto(191.17261841,409.07710297)
\moveto(188.92261841,409.96210297)
\curveto(188.99261244,410.06209804)(188.99761244,410.18209792)(188.93761841,410.32210297)
\curveto(188.88761255,410.47209763)(188.84761259,410.58209752)(188.81761841,410.65210297)
\curveto(188.67761276,410.92209718)(188.49261294,411.12709697)(188.26261841,411.26710297)
\curveto(188.0326134,411.41709668)(187.71261372,411.4970966)(187.30261841,411.50710297)
\curveto(187.27261416,411.48709661)(187.2376142,411.48209662)(187.19761841,411.49210297)
\curveto(187.15761428,411.5020966)(187.12261431,411.5020966)(187.09261841,411.49210297)
\curveto(187.04261439,411.47209663)(186.98761445,411.45709664)(186.92761841,411.44710297)
\curveto(186.86761457,411.44709665)(186.81261462,411.43709666)(186.76261841,411.41710297)
\curveto(186.32261511,411.27709682)(185.99761544,411.0020971)(185.78761841,410.59210297)
\curveto(185.76761567,410.55209755)(185.74261569,410.4970976)(185.71261841,410.42710297)
\curveto(185.69261574,410.36709773)(185.67761576,410.3020978)(185.66761841,410.23210297)
\curveto(185.65761578,410.17209793)(185.65761578,410.11209799)(185.66761841,410.05210297)
\curveto(185.68761575,409.99209811)(185.72261571,409.94209816)(185.77261841,409.90210297)
\curveto(185.85261558,409.85209825)(185.96261547,409.82709827)(186.10261841,409.82710297)
\lineto(186.50761841,409.82710297)
\lineto(188.17261841,409.82710297)
\lineto(188.60761841,409.82710297)
\curveto(188.76761267,409.83709826)(188.87261256,409.88209822)(188.92261841,409.96210297)
}
}
{
\newrgbcolor{curcolor}{0 0 0}
\pscustom[linestyle=none,fillstyle=solid,fillcolor=curcolor]
{
}
}
{
\newrgbcolor{curcolor}{0 0 0}
\pscustom[linestyle=none,fillstyle=solid,fillcolor=curcolor]
{
\newpath
\moveto(197.09105591,415.82710297)
\lineto(198.18605591,415.82710297)
\curveto(198.28605342,415.82709227)(198.38105333,415.82209228)(198.47105591,415.81210297)
\curveto(198.56105315,415.8020923)(198.63105308,415.77209233)(198.68105591,415.72210297)
\curveto(198.74105297,415.65209245)(198.77105294,415.55709254)(198.77105591,415.43710297)
\curveto(198.78105293,415.32709277)(198.78605292,415.21209289)(198.78605591,415.09210297)
\lineto(198.78605591,413.75710297)
\lineto(198.78605591,408.37210297)
\lineto(198.78605591,406.07710297)
\lineto(198.78605591,405.65710297)
\curveto(198.79605291,405.50710259)(198.77605293,405.39210271)(198.72605591,405.31210297)
\curveto(198.67605303,405.23210287)(198.58605312,405.17710292)(198.45605591,405.14710297)
\curveto(198.39605331,405.12710297)(198.32605338,405.12210298)(198.24605591,405.13210297)
\curveto(198.17605353,405.14210296)(198.1060536,405.14710295)(198.03605591,405.14710297)
\lineto(197.31605591,405.14710297)
\curveto(197.2060545,405.14710295)(197.1060546,405.15210295)(197.01605591,405.16210297)
\curveto(196.92605478,405.17210293)(196.85105486,405.2021029)(196.79105591,405.25210297)
\curveto(196.73105498,405.3021028)(196.69605501,405.37710272)(196.68605591,405.47710297)
\lineto(196.68605591,405.80710297)
\lineto(196.68605591,407.14210297)
\lineto(196.68605591,412.76710297)
\lineto(196.68605591,414.80710297)
\curveto(196.68605502,414.93709316)(196.68105503,415.09209301)(196.67105591,415.27210297)
\curveto(196.67105504,415.45209265)(196.69605501,415.58209252)(196.74605591,415.66210297)
\curveto(196.76605494,415.7020924)(196.79105492,415.73209237)(196.82105591,415.75210297)
\lineto(196.94105591,415.81210297)
\curveto(196.96105475,415.81209229)(196.98605472,415.81209229)(197.01605591,415.81210297)
\curveto(197.04605466,415.82209228)(197.07105464,415.82709227)(197.09105591,415.82710297)
}
}
{
\newrgbcolor{curcolor}{0 0 0}
\pscustom[linestyle=none,fillstyle=solid,fillcolor=curcolor]
{
\newpath
\moveto(208.21824341,409.31710297)
\curveto(208.23823484,409.25709884)(208.24823483,409.17209893)(208.24824341,409.06210297)
\curveto(208.24823483,408.95209915)(208.23823484,408.86709923)(208.21824341,408.80710297)
\lineto(208.21824341,408.65710297)
\curveto(208.19823488,408.57709952)(208.18823489,408.4970996)(208.18824341,408.41710297)
\curveto(208.19823488,408.33709976)(208.19323488,408.25709984)(208.17324341,408.17710297)
\curveto(208.15323492,408.10709999)(208.13823494,408.04210006)(208.12824341,407.98210297)
\curveto(208.11823496,407.92210018)(208.10823497,407.85710024)(208.09824341,407.78710297)
\curveto(208.05823502,407.67710042)(208.02323505,407.56210054)(207.99324341,407.44210297)
\curveto(207.96323511,407.33210077)(207.92323515,407.22710087)(207.87324341,407.12710297)
\curveto(207.66323541,406.64710145)(207.38823569,406.25710184)(207.04824341,405.95710297)
\curveto(206.70823637,405.65710244)(206.29823678,405.40710269)(205.81824341,405.20710297)
\curveto(205.69823738,405.15710294)(205.5732375,405.12210298)(205.44324341,405.10210297)
\curveto(205.32323775,405.07210303)(205.19823788,405.04210306)(205.06824341,405.01210297)
\curveto(205.01823806,404.99210311)(204.96323811,404.98210312)(204.90324341,404.98210297)
\curveto(204.84323823,404.98210312)(204.78823829,404.97710312)(204.73824341,404.96710297)
\lineto(204.63324341,404.96710297)
\curveto(204.60323847,404.95710314)(204.5732385,404.95210315)(204.54324341,404.95210297)
\curveto(204.49323858,404.94210316)(204.41323866,404.93710316)(204.30324341,404.93710297)
\curveto(204.19323888,404.92710317)(204.10823897,404.93210317)(204.04824341,404.95210297)
\lineto(203.89824341,404.95210297)
\curveto(203.84823923,404.96210314)(203.79323928,404.96710313)(203.73324341,404.96710297)
\curveto(203.68323939,404.95710314)(203.63323944,404.96210314)(203.58324341,404.98210297)
\curveto(203.54323953,404.99210311)(203.50323957,404.9971031)(203.46324341,404.99710297)
\curveto(203.43323964,404.9971031)(203.39323968,405.0021031)(203.34324341,405.01210297)
\curveto(203.24323983,405.04210306)(203.14323993,405.06710303)(203.04324341,405.08710297)
\curveto(202.94324013,405.10710299)(202.84824023,405.13710296)(202.75824341,405.17710297)
\curveto(202.63824044,405.21710288)(202.52324055,405.25710284)(202.41324341,405.29710297)
\curveto(202.31324076,405.33710276)(202.20824087,405.38710271)(202.09824341,405.44710297)
\curveto(201.74824133,405.65710244)(201.44824163,405.9021022)(201.19824341,406.18210297)
\curveto(200.94824213,406.46210164)(200.73824234,406.7971013)(200.56824341,407.18710297)
\curveto(200.51824256,407.27710082)(200.4782426,407.37210073)(200.44824341,407.47210297)
\curveto(200.42824265,407.57210053)(200.40324267,407.67710042)(200.37324341,407.78710297)
\curveto(200.35324272,407.83710026)(200.34324273,407.88210022)(200.34324341,407.92210297)
\curveto(200.34324273,407.96210014)(200.33324274,408.00710009)(200.31324341,408.05710297)
\curveto(200.29324278,408.13709996)(200.28324279,408.21709988)(200.28324341,408.29710297)
\curveto(200.28324279,408.38709971)(200.2732428,408.47209963)(200.25324341,408.55210297)
\curveto(200.24324283,408.6020995)(200.23824284,408.64709945)(200.23824341,408.68710297)
\lineto(200.23824341,408.82210297)
\curveto(200.21824286,408.88209922)(200.20824287,408.96709913)(200.20824341,409.07710297)
\curveto(200.21824286,409.18709891)(200.23324284,409.27209883)(200.25324341,409.33210297)
\lineto(200.25324341,409.43710297)
\curveto(200.26324281,409.48709861)(200.26324281,409.53709856)(200.25324341,409.58710297)
\curveto(200.25324282,409.64709845)(200.26324281,409.7020984)(200.28324341,409.75210297)
\curveto(200.29324278,409.8020983)(200.29824278,409.84709825)(200.29824341,409.88710297)
\curveto(200.29824278,409.93709816)(200.30824277,409.98709811)(200.32824341,410.03710297)
\curveto(200.36824271,410.16709793)(200.40324267,410.29209781)(200.43324341,410.41210297)
\curveto(200.46324261,410.54209756)(200.50324257,410.66709743)(200.55324341,410.78710297)
\curveto(200.73324234,411.1970969)(200.94824213,411.53709656)(201.19824341,411.80710297)
\curveto(201.44824163,412.08709601)(201.75324132,412.34209576)(202.11324341,412.57210297)
\curveto(202.21324086,412.62209548)(202.31824076,412.66709543)(202.42824341,412.70710297)
\curveto(202.53824054,412.74709535)(202.64824043,412.79209531)(202.75824341,412.84210297)
\curveto(202.88824019,412.89209521)(203.02324005,412.92709517)(203.16324341,412.94710297)
\curveto(203.30323977,412.96709513)(203.44823963,412.9970951)(203.59824341,413.03710297)
\curveto(203.6782394,413.04709505)(203.75323932,413.05209505)(203.82324341,413.05210297)
\curveto(203.89323918,413.05209505)(203.96323911,413.05709504)(204.03324341,413.06710297)
\curveto(204.61323846,413.07709502)(205.11323796,413.01709508)(205.53324341,412.88710297)
\curveto(205.96323711,412.75709534)(206.34323673,412.57709552)(206.67324341,412.34710297)
\curveto(206.78323629,412.26709583)(206.89323618,412.17709592)(207.00324341,412.07710297)
\curveto(207.12323595,411.98709611)(207.22323585,411.88709621)(207.30324341,411.77710297)
\curveto(207.38323569,411.67709642)(207.45323562,411.57709652)(207.51324341,411.47710297)
\curveto(207.58323549,411.37709672)(207.65323542,411.27209683)(207.72324341,411.16210297)
\curveto(207.79323528,411.05209705)(207.84823523,410.93209717)(207.88824341,410.80210297)
\curveto(207.92823515,410.68209742)(207.9732351,410.55209755)(208.02324341,410.41210297)
\curveto(208.05323502,410.33209777)(208.078235,410.24709785)(208.09824341,410.15710297)
\lineto(208.15824341,409.88710297)
\curveto(208.16823491,409.84709825)(208.1732349,409.80709829)(208.17324341,409.76710297)
\curveto(208.1732349,409.72709837)(208.1782349,409.68709841)(208.18824341,409.64710297)
\curveto(208.20823487,409.5970985)(208.21323486,409.54209856)(208.20324341,409.48210297)
\curveto(208.19323488,409.42209868)(208.19823488,409.36709873)(208.21824341,409.31710297)
\moveto(206.11824341,408.77710297)
\curveto(206.12823695,408.82709927)(206.13323694,408.8970992)(206.13324341,408.98710297)
\curveto(206.13323694,409.08709901)(206.12823695,409.16209894)(206.11824341,409.21210297)
\lineto(206.11824341,409.33210297)
\curveto(206.09823698,409.38209872)(206.08823699,409.43709866)(206.08824341,409.49710297)
\curveto(206.08823699,409.55709854)(206.08323699,409.61209849)(206.07324341,409.66210297)
\curveto(206.073237,409.7020984)(206.06823701,409.73209837)(206.05824341,409.75210297)
\lineto(205.99824341,409.99210297)
\curveto(205.98823709,410.08209802)(205.96823711,410.16709793)(205.93824341,410.24710297)
\curveto(205.82823725,410.50709759)(205.69823738,410.72709737)(205.54824341,410.90710297)
\curveto(205.39823768,411.097097)(205.19823788,411.24709685)(204.94824341,411.35710297)
\curveto(204.88823819,411.37709672)(204.82823825,411.39209671)(204.76824341,411.40210297)
\curveto(204.70823837,411.42209668)(204.64323843,411.44209666)(204.57324341,411.46210297)
\curveto(204.49323858,411.48209662)(204.40823867,411.48709661)(204.31824341,411.47710297)
\lineto(204.04824341,411.47710297)
\curveto(204.01823906,411.45709664)(203.98323909,411.44709665)(203.94324341,411.44710297)
\curveto(203.90323917,411.45709664)(203.86823921,411.45709664)(203.83824341,411.44710297)
\lineto(203.62824341,411.38710297)
\curveto(203.56823951,411.37709672)(203.51323956,411.35709674)(203.46324341,411.32710297)
\curveto(203.21323986,411.21709688)(203.00824007,411.05709704)(202.84824341,410.84710297)
\curveto(202.69824038,410.64709745)(202.5782405,410.41209769)(202.48824341,410.14210297)
\curveto(202.45824062,410.04209806)(202.43324064,409.93709816)(202.41324341,409.82710297)
\curveto(202.40324067,409.71709838)(202.38824069,409.60709849)(202.36824341,409.49710297)
\curveto(202.35824072,409.44709865)(202.35324072,409.3970987)(202.35324341,409.34710297)
\lineto(202.35324341,409.19710297)
\curveto(202.33324074,409.12709897)(202.32324075,409.02209908)(202.32324341,408.88210297)
\curveto(202.33324074,408.74209936)(202.34824073,408.63709946)(202.36824341,408.56710297)
\lineto(202.36824341,408.43210297)
\curveto(202.38824069,408.35209975)(202.40324067,408.27209983)(202.41324341,408.19210297)
\curveto(202.42324065,408.12209998)(202.43824064,408.04710005)(202.45824341,407.96710297)
\curveto(202.55824052,407.66710043)(202.66324041,407.42210068)(202.77324341,407.23210297)
\curveto(202.89324018,407.05210105)(203.07824,406.88710121)(203.32824341,406.73710297)
\curveto(203.39823968,406.68710141)(203.4732396,406.64710145)(203.55324341,406.61710297)
\curveto(203.64323943,406.58710151)(203.73323934,406.56210154)(203.82324341,406.54210297)
\curveto(203.86323921,406.53210157)(203.89823918,406.52710157)(203.92824341,406.52710297)
\curveto(203.95823912,406.53710156)(203.99323908,406.53710156)(204.03324341,406.52710297)
\lineto(204.15324341,406.49710297)
\curveto(204.20323887,406.4971016)(204.24823883,406.5021016)(204.28824341,406.51210297)
\lineto(204.40824341,406.51210297)
\curveto(204.48823859,406.53210157)(204.56823851,406.54710155)(204.64824341,406.55710297)
\curveto(204.72823835,406.56710153)(204.80323827,406.58710151)(204.87324341,406.61710297)
\curveto(205.13323794,406.71710138)(205.34323773,406.85210125)(205.50324341,407.02210297)
\curveto(205.66323741,407.19210091)(205.79823728,407.4021007)(205.90824341,407.65210297)
\curveto(205.94823713,407.75210035)(205.9782371,407.85210025)(205.99824341,407.95210297)
\curveto(206.01823706,408.05210005)(206.04323703,408.15709994)(206.07324341,408.26710297)
\curveto(206.08323699,408.30709979)(206.08823699,408.34209976)(206.08824341,408.37210297)
\curveto(206.08823699,408.41209969)(206.09323698,408.45209965)(206.10324341,408.49210297)
\lineto(206.10324341,408.62710297)
\curveto(206.10323697,408.67709942)(206.10823697,408.72709937)(206.11824341,408.77710297)
}
}
{
\newrgbcolor{curcolor}{0 0 0}
\pscustom[linestyle=none,fillstyle=solid,fillcolor=curcolor]
{
\newpath
\moveto(212.58816528,413.08210297)
\curveto(213.33816078,413.102095)(213.98816013,413.01709508)(214.53816528,412.82710297)
\curveto(215.09815902,412.64709545)(215.5231586,412.33209577)(215.81316528,411.88210297)
\curveto(215.88315824,411.77209633)(215.94315818,411.65709644)(215.99316528,411.53710297)
\curveto(216.05315807,411.42709667)(216.10315802,411.3020968)(216.14316528,411.16210297)
\curveto(216.16315796,411.102097)(216.17315795,411.03709706)(216.17316528,410.96710297)
\curveto(216.17315795,410.8970972)(216.16315796,410.83709726)(216.14316528,410.78710297)
\curveto(216.10315802,410.72709737)(216.04815807,410.68709741)(215.97816528,410.66710297)
\curveto(215.92815819,410.64709745)(215.86815825,410.63709746)(215.79816528,410.63710297)
\lineto(215.58816528,410.63710297)
\lineto(214.92816528,410.63710297)
\curveto(214.85815926,410.63709746)(214.78815933,410.63209747)(214.71816528,410.62210297)
\curveto(214.64815947,410.62209748)(214.58315954,410.63209747)(214.52316528,410.65210297)
\curveto(214.4231597,410.67209743)(214.34815977,410.71209739)(214.29816528,410.77210297)
\curveto(214.24815987,410.83209727)(214.20315992,410.89209721)(214.16316528,410.95210297)
\lineto(214.04316528,411.16210297)
\curveto(214.01316011,411.24209686)(213.96316016,411.30709679)(213.89316528,411.35710297)
\curveto(213.79316033,411.43709666)(213.69316043,411.4970966)(213.59316528,411.53710297)
\curveto(213.50316062,411.57709652)(213.38816073,411.61209649)(213.24816528,411.64210297)
\curveto(213.17816094,411.66209644)(213.07316105,411.67709642)(212.93316528,411.68710297)
\curveto(212.80316132,411.6970964)(212.70316142,411.69209641)(212.63316528,411.67210297)
\lineto(212.52816528,411.67210297)
\lineto(212.37816528,411.64210297)
\curveto(212.33816178,411.64209646)(212.29316183,411.63709646)(212.24316528,411.62710297)
\curveto(212.07316205,411.57709652)(211.93316219,411.50709659)(211.82316528,411.41710297)
\curveto(211.7231624,411.33709676)(211.65316247,411.21209689)(211.61316528,411.04210297)
\curveto(211.59316253,410.97209713)(211.59316253,410.90709719)(211.61316528,410.84710297)
\curveto(211.63316249,410.78709731)(211.65316247,410.73709736)(211.67316528,410.69710297)
\curveto(211.74316238,410.57709752)(211.8231623,410.48209762)(211.91316528,410.41210297)
\curveto(212.01316211,410.34209776)(212.12816199,410.28209782)(212.25816528,410.23210297)
\curveto(212.44816167,410.15209795)(212.65316147,410.08209802)(212.87316528,410.02210297)
\lineto(213.56316528,409.87210297)
\curveto(213.80316032,409.83209827)(214.03316009,409.78209832)(214.25316528,409.72210297)
\curveto(214.48315964,409.67209843)(214.69815942,409.60709849)(214.89816528,409.52710297)
\curveto(214.98815913,409.48709861)(215.07315905,409.45209865)(215.15316528,409.42210297)
\curveto(215.24315888,409.4020987)(215.32815879,409.36709873)(215.40816528,409.31710297)
\curveto(215.59815852,409.1970989)(215.76815835,409.06709903)(215.91816528,408.92710297)
\curveto(216.07815804,408.78709931)(216.20315792,408.61209949)(216.29316528,408.40210297)
\curveto(216.3231578,408.33209977)(216.34815777,408.26209984)(216.36816528,408.19210297)
\curveto(216.38815773,408.12209998)(216.40815771,408.04710005)(216.42816528,407.96710297)
\curveto(216.43815768,407.90710019)(216.44315768,407.81210029)(216.44316528,407.68210297)
\curveto(216.45315767,407.56210054)(216.45315767,407.46710063)(216.44316528,407.39710297)
\lineto(216.44316528,407.32210297)
\curveto(216.4231577,407.26210084)(216.40815771,407.2021009)(216.39816528,407.14210297)
\curveto(216.39815772,407.09210101)(216.39315773,407.04210106)(216.38316528,406.99210297)
\curveto(216.31315781,406.69210141)(216.20315792,406.42710167)(216.05316528,406.19710297)
\curveto(215.89315823,405.95710214)(215.69815842,405.76210234)(215.46816528,405.61210297)
\curveto(215.23815888,405.46210264)(214.97815914,405.33210277)(214.68816528,405.22210297)
\curveto(214.57815954,405.17210293)(214.45815966,405.13710296)(214.32816528,405.11710297)
\curveto(214.20815991,405.097103)(214.08816003,405.07210303)(213.96816528,405.04210297)
\curveto(213.87816024,405.02210308)(213.78316034,405.01210309)(213.68316528,405.01210297)
\curveto(213.59316053,405.0021031)(213.50316062,404.98710311)(213.41316528,404.96710297)
\lineto(213.14316528,404.96710297)
\curveto(213.08316104,404.94710315)(212.97816114,404.93710316)(212.82816528,404.93710297)
\curveto(212.68816143,404.93710316)(212.58816153,404.94710315)(212.52816528,404.96710297)
\curveto(212.49816162,404.96710313)(212.46316166,404.97210313)(212.42316528,404.98210297)
\lineto(212.31816528,404.98210297)
\curveto(212.19816192,405.0021031)(212.07816204,405.01710308)(211.95816528,405.02710297)
\curveto(211.83816228,405.03710306)(211.7231624,405.05710304)(211.61316528,405.08710297)
\curveto(211.2231629,405.1971029)(210.87816324,405.32210278)(210.57816528,405.46210297)
\curveto(210.27816384,405.61210249)(210.0231641,405.83210227)(209.81316528,406.12210297)
\curveto(209.67316445,406.31210179)(209.55316457,406.53210157)(209.45316528,406.78210297)
\curveto(209.43316469,406.84210126)(209.41316471,406.92210118)(209.39316528,407.02210297)
\curveto(209.37316475,407.07210103)(209.35816476,407.14210096)(209.34816528,407.23210297)
\curveto(209.33816478,407.32210078)(209.34316478,407.3971007)(209.36316528,407.45710297)
\curveto(209.39316473,407.52710057)(209.44316468,407.57710052)(209.51316528,407.60710297)
\curveto(209.56316456,407.62710047)(209.6231645,407.63710046)(209.69316528,407.63710297)
\lineto(209.91816528,407.63710297)
\lineto(210.62316528,407.63710297)
\lineto(210.86316528,407.63710297)
\curveto(210.94316318,407.63710046)(211.01316311,407.62710047)(211.07316528,407.60710297)
\curveto(211.18316294,407.56710053)(211.25316287,407.5021006)(211.28316528,407.41210297)
\curveto(211.3231628,407.32210078)(211.36816275,407.22710087)(211.41816528,407.12710297)
\curveto(211.43816268,407.07710102)(211.47316265,407.01210109)(211.52316528,406.93210297)
\curveto(211.58316254,406.85210125)(211.63316249,406.8021013)(211.67316528,406.78210297)
\curveto(211.79316233,406.68210142)(211.90816221,406.6021015)(212.01816528,406.54210297)
\curveto(212.12816199,406.49210161)(212.26816185,406.44210166)(212.43816528,406.39210297)
\curveto(212.48816163,406.37210173)(212.53816158,406.36210174)(212.58816528,406.36210297)
\curveto(212.63816148,406.37210173)(212.68816143,406.37210173)(212.73816528,406.36210297)
\curveto(212.8181613,406.34210176)(212.90316122,406.33210177)(212.99316528,406.33210297)
\curveto(213.09316103,406.34210176)(213.17816094,406.35710174)(213.24816528,406.37710297)
\curveto(213.29816082,406.38710171)(213.34316078,406.39210171)(213.38316528,406.39210297)
\curveto(213.43316069,406.39210171)(213.48316064,406.4021017)(213.53316528,406.42210297)
\curveto(213.67316045,406.47210163)(213.79816032,406.53210157)(213.90816528,406.60210297)
\curveto(214.02816009,406.67210143)(214.12316,406.76210134)(214.19316528,406.87210297)
\curveto(214.24315988,406.95210115)(214.28315984,407.07710102)(214.31316528,407.24710297)
\curveto(214.33315979,407.31710078)(214.33315979,407.38210072)(214.31316528,407.44210297)
\curveto(214.29315983,407.5021006)(214.27315985,407.55210055)(214.25316528,407.59210297)
\curveto(214.18315994,407.73210037)(214.09316003,407.83710026)(213.98316528,407.90710297)
\curveto(213.88316024,407.97710012)(213.76316036,408.04210006)(213.62316528,408.10210297)
\curveto(213.43316069,408.18209992)(213.23316089,408.24709985)(213.02316528,408.29710297)
\curveto(212.81316131,408.34709975)(212.60316152,408.4020997)(212.39316528,408.46210297)
\curveto(212.31316181,408.48209962)(212.22816189,408.4970996)(212.13816528,408.50710297)
\curveto(212.05816206,408.51709958)(211.97816214,408.53209957)(211.89816528,408.55210297)
\curveto(211.57816254,408.64209946)(211.27316285,408.72709937)(210.98316528,408.80710297)
\curveto(210.69316343,408.8970992)(210.42816369,409.02709907)(210.18816528,409.19710297)
\curveto(209.90816421,409.3970987)(209.70316442,409.66709843)(209.57316528,410.00710297)
\curveto(209.55316457,410.07709802)(209.53316459,410.17209793)(209.51316528,410.29210297)
\curveto(209.49316463,410.36209774)(209.47816464,410.44709765)(209.46816528,410.54710297)
\curveto(209.45816466,410.64709745)(209.46316466,410.73709736)(209.48316528,410.81710297)
\curveto(209.50316462,410.86709723)(209.50816461,410.90709719)(209.49816528,410.93710297)
\curveto(209.48816463,410.97709712)(209.49316463,411.02209708)(209.51316528,411.07210297)
\curveto(209.53316459,411.18209692)(209.55316457,411.28209682)(209.57316528,411.37210297)
\curveto(209.60316452,411.47209663)(209.63816448,411.56709653)(209.67816528,411.65710297)
\curveto(209.80816431,411.94709615)(209.98816413,412.18209592)(210.21816528,412.36210297)
\curveto(210.44816367,412.54209556)(210.70816341,412.68709541)(210.99816528,412.79710297)
\curveto(211.10816301,412.84709525)(211.2231629,412.88209522)(211.34316528,412.90210297)
\curveto(211.46316266,412.93209517)(211.58816253,412.96209514)(211.71816528,412.99210297)
\curveto(211.77816234,413.01209509)(211.83816228,413.02209508)(211.89816528,413.02210297)
\lineto(212.07816528,413.05210297)
\curveto(212.15816196,413.06209504)(212.24316188,413.06709503)(212.33316528,413.06710297)
\curveto(212.4231617,413.06709503)(212.50816161,413.07209503)(212.58816528,413.08210297)
}
}
{
\newrgbcolor{curcolor}{0 0 0}
\pscustom[linestyle=none,fillstyle=solid,fillcolor=curcolor]
{
}
}
{
\newrgbcolor{curcolor}{0 0 0}
\pscustom[linestyle=none,fillstyle=solid,fillcolor=curcolor]
{
\newpath
\moveto(222.25496216,412.85710297)
\lineto(223.37996216,412.85710297)
\curveto(223.48995972,412.85709524)(223.58995962,412.85209525)(223.67996216,412.84210297)
\curveto(223.76995944,412.83209527)(223.83495938,412.7970953)(223.87496216,412.73710297)
\curveto(223.92495929,412.67709542)(223.95495926,412.59209551)(223.96496216,412.48210297)
\curveto(223.97495924,412.38209572)(223.97995923,412.27709582)(223.97996216,412.16710297)
\lineto(223.97996216,411.11710297)
\lineto(223.97996216,408.88210297)
\curveto(223.97995923,408.52209958)(223.99495922,408.18209992)(224.02496216,407.86210297)
\curveto(224.05495916,407.54210056)(224.14495907,407.27710082)(224.29496216,407.06710297)
\curveto(224.43495878,406.85710124)(224.65995855,406.70710139)(224.96996216,406.61710297)
\curveto(225.01995819,406.60710149)(225.05995815,406.6021015)(225.08996216,406.60210297)
\curveto(225.12995808,406.6021015)(225.17495804,406.5971015)(225.22496216,406.58710297)
\curveto(225.27495794,406.57710152)(225.32995788,406.57210153)(225.38996216,406.57210297)
\curveto(225.44995776,406.57210153)(225.49495772,406.57710152)(225.52496216,406.58710297)
\curveto(225.57495764,406.60710149)(225.6149576,406.61210149)(225.64496216,406.60210297)
\curveto(225.68495753,406.59210151)(225.72495749,406.5971015)(225.76496216,406.61710297)
\curveto(225.97495724,406.66710143)(226.13995707,406.73210137)(226.25996216,406.81210297)
\curveto(226.43995677,406.92210118)(226.57995663,407.06210104)(226.67996216,407.23210297)
\curveto(226.78995642,407.41210069)(226.86495635,407.60710049)(226.90496216,407.81710297)
\curveto(226.95495626,408.03710006)(226.98495623,408.27709982)(226.99496216,408.53710297)
\curveto(227.00495621,408.80709929)(227.0099562,409.08709901)(227.00996216,409.37710297)
\lineto(227.00996216,411.19210297)
\lineto(227.00996216,412.16710297)
\lineto(227.00996216,412.43710297)
\curveto(227.0099562,412.53709556)(227.02995618,412.61709548)(227.06996216,412.67710297)
\curveto(227.11995609,412.76709533)(227.19495602,412.81709528)(227.29496216,412.82710297)
\curveto(227.39495582,412.84709525)(227.5149557,412.85709524)(227.65496216,412.85710297)
\lineto(228.44996216,412.85710297)
\lineto(228.73496216,412.85710297)
\curveto(228.82495439,412.85709524)(228.89995431,412.83709526)(228.95996216,412.79710297)
\curveto(229.03995417,412.74709535)(229.08495413,412.67209543)(229.09496216,412.57210297)
\curveto(229.10495411,412.47209563)(229.1099541,412.35709574)(229.10996216,412.22710297)
\lineto(229.10996216,411.08710297)
\lineto(229.10996216,406.87210297)
\lineto(229.10996216,405.80710297)
\lineto(229.10996216,405.50710297)
\curveto(229.1099541,405.40710269)(229.08995412,405.33210277)(229.04996216,405.28210297)
\curveto(228.99995421,405.2021029)(228.92495429,405.15710294)(228.82496216,405.14710297)
\curveto(228.72495449,405.13710296)(228.61995459,405.13210297)(228.50996216,405.13210297)
\lineto(227.69996216,405.13210297)
\curveto(227.58995562,405.13210297)(227.48995572,405.13710296)(227.39996216,405.14710297)
\curveto(227.31995589,405.15710294)(227.25495596,405.1971029)(227.20496216,405.26710297)
\curveto(227.18495603,405.2971028)(227.16495605,405.34210276)(227.14496216,405.40210297)
\curveto(227.13495608,405.46210264)(227.11995609,405.52210258)(227.09996216,405.58210297)
\curveto(227.08995612,405.64210246)(227.07495614,405.6971024)(227.05496216,405.74710297)
\curveto(227.03495618,405.7971023)(227.00495621,405.82710227)(226.96496216,405.83710297)
\curveto(226.94495627,405.85710224)(226.91995629,405.86210224)(226.88996216,405.85210297)
\curveto(226.85995635,405.84210226)(226.83495638,405.83210227)(226.81496216,405.82210297)
\curveto(226.74495647,405.78210232)(226.68495653,405.73710236)(226.63496216,405.68710297)
\curveto(226.58495663,405.63710246)(226.52995668,405.59210251)(226.46996216,405.55210297)
\curveto(226.42995678,405.52210258)(226.38995682,405.48710261)(226.34996216,405.44710297)
\curveto(226.31995689,405.41710268)(226.27995693,405.38710271)(226.22996216,405.35710297)
\curveto(225.99995721,405.21710288)(225.72995748,405.10710299)(225.41996216,405.02710297)
\curveto(225.34995786,405.00710309)(225.27995793,404.9971031)(225.20996216,404.99710297)
\curveto(225.13995807,404.98710311)(225.06495815,404.97210313)(224.98496216,404.95210297)
\curveto(224.94495827,404.94210316)(224.89995831,404.94210316)(224.84996216,404.95210297)
\curveto(224.8099584,404.95210315)(224.76995844,404.94710315)(224.72996216,404.93710297)
\curveto(224.69995851,404.92710317)(224.63495858,404.92710317)(224.53496216,404.93710297)
\curveto(224.44495877,404.93710316)(224.38495883,404.94210316)(224.35496216,404.95210297)
\curveto(224.30495891,404.95210315)(224.25495896,404.95710314)(224.20496216,404.96710297)
\lineto(224.05496216,404.96710297)
\curveto(223.93495928,404.9971031)(223.81995939,405.02210308)(223.70996216,405.04210297)
\curveto(223.59995961,405.06210304)(223.48995972,405.09210301)(223.37996216,405.13210297)
\curveto(223.32995988,405.15210295)(223.28495993,405.16710293)(223.24496216,405.17710297)
\curveto(223.21496,405.1971029)(223.17496004,405.21710288)(223.12496216,405.23710297)
\curveto(222.77496044,405.42710267)(222.49496072,405.69210241)(222.28496216,406.03210297)
\curveto(222.15496106,406.24210186)(222.05996115,406.49210161)(221.99996216,406.78210297)
\curveto(221.93996127,407.08210102)(221.89996131,407.3971007)(221.87996216,407.72710297)
\curveto(221.86996134,408.06710003)(221.86496135,408.41209969)(221.86496216,408.76210297)
\curveto(221.87496134,409.12209898)(221.87996133,409.47709862)(221.87996216,409.82710297)
\lineto(221.87996216,411.86710297)
\curveto(221.87996133,411.9970961)(221.87496134,412.14709595)(221.86496216,412.31710297)
\curveto(221.86496135,412.4970956)(221.88996132,412.62709547)(221.93996216,412.70710297)
\curveto(221.96996124,412.75709534)(222.02996118,412.8020953)(222.11996216,412.84210297)
\curveto(222.17996103,412.84209526)(222.22496099,412.84709525)(222.25496216,412.85710297)
}
}
{
\newrgbcolor{curcolor}{0 0 0}
\pscustom[linestyle=none,fillstyle=solid,fillcolor=curcolor]
{
\newpath
\moveto(233.71121216,413.08210297)
\curveto(234.46120766,413.102095)(235.11120701,413.01709508)(235.66121216,412.82710297)
\curveto(236.2212059,412.64709545)(236.64620547,412.33209577)(236.93621216,411.88210297)
\curveto(237.00620511,411.77209633)(237.06620505,411.65709644)(237.11621216,411.53710297)
\curveto(237.17620494,411.42709667)(237.22620489,411.3020968)(237.26621216,411.16210297)
\curveto(237.28620483,411.102097)(237.29620482,411.03709706)(237.29621216,410.96710297)
\curveto(237.29620482,410.8970972)(237.28620483,410.83709726)(237.26621216,410.78710297)
\curveto(237.22620489,410.72709737)(237.17120495,410.68709741)(237.10121216,410.66710297)
\curveto(237.05120507,410.64709745)(236.99120513,410.63709746)(236.92121216,410.63710297)
\lineto(236.71121216,410.63710297)
\lineto(236.05121216,410.63710297)
\curveto(235.98120614,410.63709746)(235.91120621,410.63209747)(235.84121216,410.62210297)
\curveto(235.77120635,410.62209748)(235.70620641,410.63209747)(235.64621216,410.65210297)
\curveto(235.54620657,410.67209743)(235.47120665,410.71209739)(235.42121216,410.77210297)
\curveto(235.37120675,410.83209727)(235.32620679,410.89209721)(235.28621216,410.95210297)
\lineto(235.16621216,411.16210297)
\curveto(235.13620698,411.24209686)(235.08620703,411.30709679)(235.01621216,411.35710297)
\curveto(234.9162072,411.43709666)(234.8162073,411.4970966)(234.71621216,411.53710297)
\curveto(234.62620749,411.57709652)(234.51120761,411.61209649)(234.37121216,411.64210297)
\curveto(234.30120782,411.66209644)(234.19620792,411.67709642)(234.05621216,411.68710297)
\curveto(233.92620819,411.6970964)(233.82620829,411.69209641)(233.75621216,411.67210297)
\lineto(233.65121216,411.67210297)
\lineto(233.50121216,411.64210297)
\curveto(233.46120866,411.64209646)(233.4162087,411.63709646)(233.36621216,411.62710297)
\curveto(233.19620892,411.57709652)(233.05620906,411.50709659)(232.94621216,411.41710297)
\curveto(232.84620927,411.33709676)(232.77620934,411.21209689)(232.73621216,411.04210297)
\curveto(232.7162094,410.97209713)(232.7162094,410.90709719)(232.73621216,410.84710297)
\curveto(232.75620936,410.78709731)(232.77620934,410.73709736)(232.79621216,410.69710297)
\curveto(232.86620925,410.57709752)(232.94620917,410.48209762)(233.03621216,410.41210297)
\curveto(233.13620898,410.34209776)(233.25120887,410.28209782)(233.38121216,410.23210297)
\curveto(233.57120855,410.15209795)(233.77620834,410.08209802)(233.99621216,410.02210297)
\lineto(234.68621216,409.87210297)
\curveto(234.92620719,409.83209827)(235.15620696,409.78209832)(235.37621216,409.72210297)
\curveto(235.60620651,409.67209843)(235.8212063,409.60709849)(236.02121216,409.52710297)
\curveto(236.11120601,409.48709861)(236.19620592,409.45209865)(236.27621216,409.42210297)
\curveto(236.36620575,409.4020987)(236.45120567,409.36709873)(236.53121216,409.31710297)
\curveto(236.7212054,409.1970989)(236.89120523,409.06709903)(237.04121216,408.92710297)
\curveto(237.20120492,408.78709931)(237.32620479,408.61209949)(237.41621216,408.40210297)
\curveto(237.44620467,408.33209977)(237.47120465,408.26209984)(237.49121216,408.19210297)
\curveto(237.51120461,408.12209998)(237.53120459,408.04710005)(237.55121216,407.96710297)
\curveto(237.56120456,407.90710019)(237.56620455,407.81210029)(237.56621216,407.68210297)
\curveto(237.57620454,407.56210054)(237.57620454,407.46710063)(237.56621216,407.39710297)
\lineto(237.56621216,407.32210297)
\curveto(237.54620457,407.26210084)(237.53120459,407.2021009)(237.52121216,407.14210297)
\curveto(237.5212046,407.09210101)(237.5162046,407.04210106)(237.50621216,406.99210297)
\curveto(237.43620468,406.69210141)(237.32620479,406.42710167)(237.17621216,406.19710297)
\curveto(237.0162051,405.95710214)(236.8212053,405.76210234)(236.59121216,405.61210297)
\curveto(236.36120576,405.46210264)(236.10120602,405.33210277)(235.81121216,405.22210297)
\curveto(235.70120642,405.17210293)(235.58120654,405.13710296)(235.45121216,405.11710297)
\curveto(235.33120679,405.097103)(235.21120691,405.07210303)(235.09121216,405.04210297)
\curveto(235.00120712,405.02210308)(234.90620721,405.01210309)(234.80621216,405.01210297)
\curveto(234.7162074,405.0021031)(234.62620749,404.98710311)(234.53621216,404.96710297)
\lineto(234.26621216,404.96710297)
\curveto(234.20620791,404.94710315)(234.10120802,404.93710316)(233.95121216,404.93710297)
\curveto(233.81120831,404.93710316)(233.71120841,404.94710315)(233.65121216,404.96710297)
\curveto(233.6212085,404.96710313)(233.58620853,404.97210313)(233.54621216,404.98210297)
\lineto(233.44121216,404.98210297)
\curveto(233.3212088,405.0021031)(233.20120892,405.01710308)(233.08121216,405.02710297)
\curveto(232.96120916,405.03710306)(232.84620927,405.05710304)(232.73621216,405.08710297)
\curveto(232.34620977,405.1971029)(232.00121012,405.32210278)(231.70121216,405.46210297)
\curveto(231.40121072,405.61210249)(231.14621097,405.83210227)(230.93621216,406.12210297)
\curveto(230.79621132,406.31210179)(230.67621144,406.53210157)(230.57621216,406.78210297)
\curveto(230.55621156,406.84210126)(230.53621158,406.92210118)(230.51621216,407.02210297)
\curveto(230.49621162,407.07210103)(230.48121164,407.14210096)(230.47121216,407.23210297)
\curveto(230.46121166,407.32210078)(230.46621165,407.3971007)(230.48621216,407.45710297)
\curveto(230.5162116,407.52710057)(230.56621155,407.57710052)(230.63621216,407.60710297)
\curveto(230.68621143,407.62710047)(230.74621137,407.63710046)(230.81621216,407.63710297)
\lineto(231.04121216,407.63710297)
\lineto(231.74621216,407.63710297)
\lineto(231.98621216,407.63710297)
\curveto(232.06621005,407.63710046)(232.13620998,407.62710047)(232.19621216,407.60710297)
\curveto(232.30620981,407.56710053)(232.37620974,407.5021006)(232.40621216,407.41210297)
\curveto(232.44620967,407.32210078)(232.49120963,407.22710087)(232.54121216,407.12710297)
\curveto(232.56120956,407.07710102)(232.59620952,407.01210109)(232.64621216,406.93210297)
\curveto(232.70620941,406.85210125)(232.75620936,406.8021013)(232.79621216,406.78210297)
\curveto(232.9162092,406.68210142)(233.03120909,406.6021015)(233.14121216,406.54210297)
\curveto(233.25120887,406.49210161)(233.39120873,406.44210166)(233.56121216,406.39210297)
\curveto(233.61120851,406.37210173)(233.66120846,406.36210174)(233.71121216,406.36210297)
\curveto(233.76120836,406.37210173)(233.81120831,406.37210173)(233.86121216,406.36210297)
\curveto(233.94120818,406.34210176)(234.02620809,406.33210177)(234.11621216,406.33210297)
\curveto(234.2162079,406.34210176)(234.30120782,406.35710174)(234.37121216,406.37710297)
\curveto(234.4212077,406.38710171)(234.46620765,406.39210171)(234.50621216,406.39210297)
\curveto(234.55620756,406.39210171)(234.60620751,406.4021017)(234.65621216,406.42210297)
\curveto(234.79620732,406.47210163)(234.9212072,406.53210157)(235.03121216,406.60210297)
\curveto(235.15120697,406.67210143)(235.24620687,406.76210134)(235.31621216,406.87210297)
\curveto(235.36620675,406.95210115)(235.40620671,407.07710102)(235.43621216,407.24710297)
\curveto(235.45620666,407.31710078)(235.45620666,407.38210072)(235.43621216,407.44210297)
\curveto(235.4162067,407.5021006)(235.39620672,407.55210055)(235.37621216,407.59210297)
\curveto(235.30620681,407.73210037)(235.2162069,407.83710026)(235.10621216,407.90710297)
\curveto(235.00620711,407.97710012)(234.88620723,408.04210006)(234.74621216,408.10210297)
\curveto(234.55620756,408.18209992)(234.35620776,408.24709985)(234.14621216,408.29710297)
\curveto(233.93620818,408.34709975)(233.72620839,408.4020997)(233.51621216,408.46210297)
\curveto(233.43620868,408.48209962)(233.35120877,408.4970996)(233.26121216,408.50710297)
\curveto(233.18120894,408.51709958)(233.10120902,408.53209957)(233.02121216,408.55210297)
\curveto(232.70120942,408.64209946)(232.39620972,408.72709937)(232.10621216,408.80710297)
\curveto(231.8162103,408.8970992)(231.55121057,409.02709907)(231.31121216,409.19710297)
\curveto(231.03121109,409.3970987)(230.82621129,409.66709843)(230.69621216,410.00710297)
\curveto(230.67621144,410.07709802)(230.65621146,410.17209793)(230.63621216,410.29210297)
\curveto(230.6162115,410.36209774)(230.60121152,410.44709765)(230.59121216,410.54710297)
\curveto(230.58121154,410.64709745)(230.58621153,410.73709736)(230.60621216,410.81710297)
\curveto(230.62621149,410.86709723)(230.63121149,410.90709719)(230.62121216,410.93710297)
\curveto(230.61121151,410.97709712)(230.6162115,411.02209708)(230.63621216,411.07210297)
\curveto(230.65621146,411.18209692)(230.67621144,411.28209682)(230.69621216,411.37210297)
\curveto(230.72621139,411.47209663)(230.76121136,411.56709653)(230.80121216,411.65710297)
\curveto(230.93121119,411.94709615)(231.11121101,412.18209592)(231.34121216,412.36210297)
\curveto(231.57121055,412.54209556)(231.83121029,412.68709541)(232.12121216,412.79710297)
\curveto(232.23120989,412.84709525)(232.34620977,412.88209522)(232.46621216,412.90210297)
\curveto(232.58620953,412.93209517)(232.71120941,412.96209514)(232.84121216,412.99210297)
\curveto(232.90120922,413.01209509)(232.96120916,413.02209508)(233.02121216,413.02210297)
\lineto(233.20121216,413.05210297)
\curveto(233.28120884,413.06209504)(233.36620875,413.06709503)(233.45621216,413.06710297)
\curveto(233.54620857,413.06709503)(233.63120849,413.07209503)(233.71121216,413.08210297)
}
}
{
\newrgbcolor{curcolor}{0 0 0}
\pscustom[linestyle=none,fillstyle=solid,fillcolor=curcolor]
{
\newpath
\moveto(239.21785278,412.85710297)
\lineto(240.34285278,412.85710297)
\curveto(240.45285035,412.85709524)(240.55285025,412.85209525)(240.64285278,412.84210297)
\curveto(240.73285007,412.83209527)(240.79785,412.7970953)(240.83785278,412.73710297)
\curveto(240.88784991,412.67709542)(240.91784988,412.59209551)(240.92785278,412.48210297)
\curveto(240.93784986,412.38209572)(240.94284986,412.27709582)(240.94285278,412.16710297)
\lineto(240.94285278,411.11710297)
\lineto(240.94285278,408.88210297)
\curveto(240.94284986,408.52209958)(240.95784984,408.18209992)(240.98785278,407.86210297)
\curveto(241.01784978,407.54210056)(241.10784969,407.27710082)(241.25785278,407.06710297)
\curveto(241.3978494,406.85710124)(241.62284918,406.70710139)(241.93285278,406.61710297)
\curveto(241.98284882,406.60710149)(242.02284878,406.6021015)(242.05285278,406.60210297)
\curveto(242.09284871,406.6021015)(242.13784866,406.5971015)(242.18785278,406.58710297)
\curveto(242.23784856,406.57710152)(242.29284851,406.57210153)(242.35285278,406.57210297)
\curveto(242.41284839,406.57210153)(242.45784834,406.57710152)(242.48785278,406.58710297)
\curveto(242.53784826,406.60710149)(242.57784822,406.61210149)(242.60785278,406.60210297)
\curveto(242.64784815,406.59210151)(242.68784811,406.5971015)(242.72785278,406.61710297)
\curveto(242.93784786,406.66710143)(243.1028477,406.73210137)(243.22285278,406.81210297)
\curveto(243.4028474,406.92210118)(243.54284726,407.06210104)(243.64285278,407.23210297)
\curveto(243.75284705,407.41210069)(243.82784697,407.60710049)(243.86785278,407.81710297)
\curveto(243.91784688,408.03710006)(243.94784685,408.27709982)(243.95785278,408.53710297)
\curveto(243.96784683,408.80709929)(243.97284683,409.08709901)(243.97285278,409.37710297)
\lineto(243.97285278,411.19210297)
\lineto(243.97285278,412.16710297)
\lineto(243.97285278,412.43710297)
\curveto(243.97284683,412.53709556)(243.99284681,412.61709548)(244.03285278,412.67710297)
\curveto(244.08284672,412.76709533)(244.15784664,412.81709528)(244.25785278,412.82710297)
\curveto(244.35784644,412.84709525)(244.47784632,412.85709524)(244.61785278,412.85710297)
\lineto(245.41285278,412.85710297)
\lineto(245.69785278,412.85710297)
\curveto(245.78784501,412.85709524)(245.86284494,412.83709526)(245.92285278,412.79710297)
\curveto(246.0028448,412.74709535)(246.04784475,412.67209543)(246.05785278,412.57210297)
\curveto(246.06784473,412.47209563)(246.07284473,412.35709574)(246.07285278,412.22710297)
\lineto(246.07285278,411.08710297)
\lineto(246.07285278,406.87210297)
\lineto(246.07285278,405.80710297)
\lineto(246.07285278,405.50710297)
\curveto(246.07284473,405.40710269)(246.05284475,405.33210277)(246.01285278,405.28210297)
\curveto(245.96284484,405.2021029)(245.88784491,405.15710294)(245.78785278,405.14710297)
\curveto(245.68784511,405.13710296)(245.58284522,405.13210297)(245.47285278,405.13210297)
\lineto(244.66285278,405.13210297)
\curveto(244.55284625,405.13210297)(244.45284635,405.13710296)(244.36285278,405.14710297)
\curveto(244.28284652,405.15710294)(244.21784658,405.1971029)(244.16785278,405.26710297)
\curveto(244.14784665,405.2971028)(244.12784667,405.34210276)(244.10785278,405.40210297)
\curveto(244.0978467,405.46210264)(244.08284672,405.52210258)(244.06285278,405.58210297)
\curveto(244.05284675,405.64210246)(244.03784676,405.6971024)(244.01785278,405.74710297)
\curveto(243.9978468,405.7971023)(243.96784683,405.82710227)(243.92785278,405.83710297)
\curveto(243.90784689,405.85710224)(243.88284692,405.86210224)(243.85285278,405.85210297)
\curveto(243.82284698,405.84210226)(243.797847,405.83210227)(243.77785278,405.82210297)
\curveto(243.70784709,405.78210232)(243.64784715,405.73710236)(243.59785278,405.68710297)
\curveto(243.54784725,405.63710246)(243.49284731,405.59210251)(243.43285278,405.55210297)
\curveto(243.39284741,405.52210258)(243.35284745,405.48710261)(243.31285278,405.44710297)
\curveto(243.28284752,405.41710268)(243.24284756,405.38710271)(243.19285278,405.35710297)
\curveto(242.96284784,405.21710288)(242.69284811,405.10710299)(242.38285278,405.02710297)
\curveto(242.31284849,405.00710309)(242.24284856,404.9971031)(242.17285278,404.99710297)
\curveto(242.1028487,404.98710311)(242.02784877,404.97210313)(241.94785278,404.95210297)
\curveto(241.90784889,404.94210316)(241.86284894,404.94210316)(241.81285278,404.95210297)
\curveto(241.77284903,404.95210315)(241.73284907,404.94710315)(241.69285278,404.93710297)
\curveto(241.66284914,404.92710317)(241.5978492,404.92710317)(241.49785278,404.93710297)
\curveto(241.40784939,404.93710316)(241.34784945,404.94210316)(241.31785278,404.95210297)
\curveto(241.26784953,404.95210315)(241.21784958,404.95710314)(241.16785278,404.96710297)
\lineto(241.01785278,404.96710297)
\curveto(240.8978499,404.9971031)(240.78285002,405.02210308)(240.67285278,405.04210297)
\curveto(240.56285024,405.06210304)(240.45285035,405.09210301)(240.34285278,405.13210297)
\curveto(240.29285051,405.15210295)(240.24785055,405.16710293)(240.20785278,405.17710297)
\curveto(240.17785062,405.1971029)(240.13785066,405.21710288)(240.08785278,405.23710297)
\curveto(239.73785106,405.42710267)(239.45785134,405.69210241)(239.24785278,406.03210297)
\curveto(239.11785168,406.24210186)(239.02285178,406.49210161)(238.96285278,406.78210297)
\curveto(238.9028519,407.08210102)(238.86285194,407.3971007)(238.84285278,407.72710297)
\curveto(238.83285197,408.06710003)(238.82785197,408.41209969)(238.82785278,408.76210297)
\curveto(238.83785196,409.12209898)(238.84285196,409.47709862)(238.84285278,409.82710297)
\lineto(238.84285278,411.86710297)
\curveto(238.84285196,411.9970961)(238.83785196,412.14709595)(238.82785278,412.31710297)
\curveto(238.82785197,412.4970956)(238.85285195,412.62709547)(238.90285278,412.70710297)
\curveto(238.93285187,412.75709534)(238.99285181,412.8020953)(239.08285278,412.84210297)
\curveto(239.14285166,412.84209526)(239.18785161,412.84709525)(239.21785278,412.85710297)
}
}
{
\newrgbcolor{curcolor}{0 0 0}
\pscustom[linestyle=none,fillstyle=solid,fillcolor=curcolor]
{
\newpath
\moveto(254.75410278,405.73210297)
\curveto(254.77409493,405.62210248)(254.78409492,405.51210259)(254.78410278,405.40210297)
\curveto(254.79409491,405.29210281)(254.74409496,405.21710288)(254.63410278,405.17710297)
\curveto(254.57409513,405.14710295)(254.5040952,405.13210297)(254.42410278,405.13210297)
\lineto(254.18410278,405.13210297)
\lineto(253.37410278,405.13210297)
\lineto(253.10410278,405.13210297)
\curveto(253.02409668,405.14210296)(252.95909675,405.16710293)(252.90910278,405.20710297)
\curveto(252.83909687,405.24710285)(252.78409692,405.3021028)(252.74410278,405.37210297)
\curveto(252.71409699,405.45210265)(252.66909704,405.51710258)(252.60910278,405.56710297)
\curveto(252.58909712,405.58710251)(252.56409714,405.6021025)(252.53410278,405.61210297)
\curveto(252.5040972,405.63210247)(252.46409724,405.63710246)(252.41410278,405.62710297)
\curveto(252.36409734,405.60710249)(252.31409739,405.58210252)(252.26410278,405.55210297)
\curveto(252.22409748,405.52210258)(252.17909753,405.4971026)(252.12910278,405.47710297)
\curveto(252.07909763,405.43710266)(252.02409768,405.4021027)(251.96410278,405.37210297)
\lineto(251.78410278,405.28210297)
\curveto(251.65409805,405.22210288)(251.51909819,405.17210293)(251.37910278,405.13210297)
\curveto(251.23909847,405.102103)(251.09409861,405.06710303)(250.94410278,405.02710297)
\curveto(250.87409883,405.00710309)(250.8040989,404.9971031)(250.73410278,404.99710297)
\curveto(250.67409903,404.98710311)(250.6090991,404.97710312)(250.53910278,404.96710297)
\lineto(250.44910278,404.96710297)
\curveto(250.41909929,404.95710314)(250.38909932,404.95210315)(250.35910278,404.95210297)
\lineto(250.19410278,404.95210297)
\curveto(250.09409961,404.93210317)(249.99409971,404.93210317)(249.89410278,404.95210297)
\lineto(249.75910278,404.95210297)
\curveto(249.68910002,404.97210313)(249.61910009,404.98210312)(249.54910278,404.98210297)
\curveto(249.48910022,404.97210313)(249.42910028,404.97710312)(249.36910278,404.99710297)
\curveto(249.26910044,405.01710308)(249.17410053,405.03710306)(249.08410278,405.05710297)
\curveto(248.99410071,405.06710303)(248.9091008,405.09210301)(248.82910278,405.13210297)
\curveto(248.53910117,405.24210286)(248.28910142,405.38210272)(248.07910278,405.55210297)
\curveto(247.87910183,405.73210237)(247.71910199,405.96710213)(247.59910278,406.25710297)
\curveto(247.56910214,406.32710177)(247.53910217,406.4021017)(247.50910278,406.48210297)
\curveto(247.48910222,406.56210154)(247.46910224,406.64710145)(247.44910278,406.73710297)
\curveto(247.42910228,406.78710131)(247.41910229,406.83710126)(247.41910278,406.88710297)
\curveto(247.42910228,406.93710116)(247.42910228,406.98710111)(247.41910278,407.03710297)
\curveto(247.4091023,407.06710103)(247.39910231,407.12710097)(247.38910278,407.21710297)
\curveto(247.38910232,407.31710078)(247.39410231,407.38710071)(247.40410278,407.42710297)
\curveto(247.42410228,407.52710057)(247.43410227,407.61210049)(247.43410278,407.68210297)
\lineto(247.52410278,408.01210297)
\curveto(247.55410215,408.13209997)(247.59410211,408.23709986)(247.64410278,408.32710297)
\curveto(247.81410189,408.61709948)(248.0091017,408.83709926)(248.22910278,408.98710297)
\curveto(248.44910126,409.13709896)(248.72910098,409.26709883)(249.06910278,409.37710297)
\curveto(249.19910051,409.42709867)(249.33410037,409.46209864)(249.47410278,409.48210297)
\curveto(249.61410009,409.5020986)(249.75409995,409.52709857)(249.89410278,409.55710297)
\curveto(249.97409973,409.57709852)(250.05909965,409.58709851)(250.14910278,409.58710297)
\curveto(250.23909947,409.5970985)(250.32909938,409.61209849)(250.41910278,409.63210297)
\curveto(250.48909922,409.65209845)(250.55909915,409.65709844)(250.62910278,409.64710297)
\curveto(250.69909901,409.64709845)(250.77409893,409.65709844)(250.85410278,409.67710297)
\curveto(250.92409878,409.6970984)(250.99409871,409.70709839)(251.06410278,409.70710297)
\curveto(251.13409857,409.70709839)(251.2090985,409.71709838)(251.28910278,409.73710297)
\curveto(251.49909821,409.78709831)(251.68909802,409.82709827)(251.85910278,409.85710297)
\curveto(252.03909767,409.8970982)(252.19909751,409.98709811)(252.33910278,410.12710297)
\curveto(252.42909728,410.21709788)(252.48909722,410.31709778)(252.51910278,410.42710297)
\curveto(252.52909718,410.45709764)(252.52909718,410.48209762)(252.51910278,410.50210297)
\curveto(252.51909719,410.52209758)(252.52409718,410.54209756)(252.53410278,410.56210297)
\curveto(252.54409716,410.58209752)(252.54909716,410.61209749)(252.54910278,410.65210297)
\lineto(252.54910278,410.74210297)
\lineto(252.51910278,410.86210297)
\curveto(252.51909719,410.9020972)(252.51409719,410.93709716)(252.50410278,410.96710297)
\curveto(252.4040973,411.26709683)(252.19409751,411.47209663)(251.87410278,411.58210297)
\curveto(251.78409792,411.61209649)(251.67409803,411.63209647)(251.54410278,411.64210297)
\curveto(251.42409828,411.66209644)(251.29909841,411.66709643)(251.16910278,411.65710297)
\curveto(251.03909867,411.65709644)(250.91409879,411.64709645)(250.79410278,411.62710297)
\curveto(250.67409903,411.60709649)(250.56909914,411.58209652)(250.47910278,411.55210297)
\curveto(250.41909929,411.53209657)(250.35909935,411.5020966)(250.29910278,411.46210297)
\curveto(250.24909946,411.43209667)(250.19909951,411.3970967)(250.14910278,411.35710297)
\curveto(250.09909961,411.31709678)(250.04409966,411.26209684)(249.98410278,411.19210297)
\curveto(249.93409977,411.12209698)(249.89909981,411.05709704)(249.87910278,410.99710297)
\curveto(249.82909988,410.8970972)(249.78409992,410.8020973)(249.74410278,410.71210297)
\curveto(249.71409999,410.62209748)(249.64410006,410.56209754)(249.53410278,410.53210297)
\curveto(249.45410025,410.51209759)(249.36910034,410.5020976)(249.27910278,410.50210297)
\lineto(249.00910278,410.50210297)
\lineto(248.43910278,410.50210297)
\curveto(248.38910132,410.5020976)(248.33910137,410.4970976)(248.28910278,410.48710297)
\curveto(248.23910147,410.48709761)(248.19410151,410.49209761)(248.15410278,410.50210297)
\lineto(248.01910278,410.50210297)
\curveto(247.99910171,410.51209759)(247.97410173,410.51709758)(247.94410278,410.51710297)
\curveto(247.91410179,410.51709758)(247.88910182,410.52709757)(247.86910278,410.54710297)
\curveto(247.78910192,410.56709753)(247.73410197,410.63209747)(247.70410278,410.74210297)
\curveto(247.69410201,410.79209731)(247.69410201,410.84209726)(247.70410278,410.89210297)
\curveto(247.71410199,410.94209716)(247.72410198,410.98709711)(247.73410278,411.02710297)
\curveto(247.76410194,411.13709696)(247.79410191,411.23709686)(247.82410278,411.32710297)
\curveto(247.86410184,411.42709667)(247.9091018,411.51709658)(247.95910278,411.59710297)
\lineto(248.04910278,411.74710297)
\lineto(248.13910278,411.89710297)
\curveto(248.21910149,412.00709609)(248.31910139,412.11209599)(248.43910278,412.21210297)
\curveto(248.45910125,412.22209588)(248.48910122,412.24709585)(248.52910278,412.28710297)
\curveto(248.57910113,412.32709577)(248.62410108,412.36209574)(248.66410278,412.39210297)
\curveto(248.704101,412.42209568)(248.74910096,412.45209565)(248.79910278,412.48210297)
\curveto(248.96910074,412.59209551)(249.14910056,412.67709542)(249.33910278,412.73710297)
\curveto(249.52910018,412.80709529)(249.72409998,412.87209523)(249.92410278,412.93210297)
\curveto(250.04409966,412.96209514)(250.16909954,412.98209512)(250.29910278,412.99210297)
\curveto(250.42909928,413.0020951)(250.55909915,413.02209508)(250.68910278,413.05210297)
\curveto(250.72909898,413.06209504)(250.78909892,413.06209504)(250.86910278,413.05210297)
\curveto(250.95909875,413.04209506)(251.01409869,413.04709505)(251.03410278,413.06710297)
\curveto(251.44409826,413.07709502)(251.83409787,413.06209504)(252.20410278,413.02210297)
\curveto(252.58409712,412.98209512)(252.92409678,412.90709519)(253.22410278,412.79710297)
\curveto(253.53409617,412.68709541)(253.79909591,412.53709556)(254.01910278,412.34710297)
\curveto(254.23909547,412.16709593)(254.4090953,411.93209617)(254.52910278,411.64210297)
\curveto(254.59909511,411.47209663)(254.63909507,411.27709682)(254.64910278,411.05710297)
\curveto(254.65909505,410.83709726)(254.66409504,410.61209749)(254.66410278,410.38210297)
\lineto(254.66410278,407.03710297)
\lineto(254.66410278,406.45210297)
\curveto(254.66409504,406.26210184)(254.68409502,406.08710201)(254.72410278,405.92710297)
\curveto(254.73409497,405.8971022)(254.73909497,405.86210224)(254.73910278,405.82210297)
\curveto(254.73909497,405.79210231)(254.74409496,405.76210234)(254.75410278,405.73210297)
\moveto(252.54910278,408.04210297)
\curveto(252.55909715,408.09210001)(252.56409714,408.14709995)(252.56410278,408.20710297)
\curveto(252.56409714,408.27709982)(252.55909715,408.33709976)(252.54910278,408.38710297)
\curveto(252.52909718,408.44709965)(252.51909719,408.5020996)(252.51910278,408.55210297)
\curveto(252.51909719,408.6020995)(252.49909721,408.64209946)(252.45910278,408.67210297)
\curveto(252.4090973,408.71209939)(252.33409737,408.73209937)(252.23410278,408.73210297)
\curveto(252.19409751,408.72209938)(252.15909755,408.71209939)(252.12910278,408.70210297)
\curveto(252.09909761,408.7020994)(252.06409764,408.6970994)(252.02410278,408.68710297)
\curveto(251.95409775,408.66709943)(251.87909783,408.65209945)(251.79910278,408.64210297)
\curveto(251.71909799,408.63209947)(251.63909807,408.61709948)(251.55910278,408.59710297)
\curveto(251.52909818,408.58709951)(251.48409822,408.58209952)(251.42410278,408.58210297)
\curveto(251.29409841,408.55209955)(251.16409854,408.53209957)(251.03410278,408.52210297)
\curveto(250.9040988,408.51209959)(250.77909893,408.48709961)(250.65910278,408.44710297)
\curveto(250.57909913,408.42709967)(250.5040992,408.40709969)(250.43410278,408.38710297)
\curveto(250.36409934,408.37709972)(250.29409941,408.35709974)(250.22410278,408.32710297)
\curveto(250.01409969,408.23709986)(249.83409987,408.1021)(249.68410278,407.92210297)
\curveto(249.54410016,407.74210036)(249.49410021,407.49210061)(249.53410278,407.17210297)
\curveto(249.55410015,407.0021011)(249.6091001,406.86210124)(249.69910278,406.75210297)
\curveto(249.76909994,406.64210146)(249.87409983,406.55210155)(250.01410278,406.48210297)
\curveto(250.15409955,406.42210168)(250.3040994,406.37710172)(250.46410278,406.34710297)
\curveto(250.63409907,406.31710178)(250.8090989,406.30710179)(250.98910278,406.31710297)
\curveto(251.17909853,406.33710176)(251.35409835,406.37210173)(251.51410278,406.42210297)
\curveto(251.77409793,406.5021016)(251.97909773,406.62710147)(252.12910278,406.79710297)
\curveto(252.27909743,406.97710112)(252.39409731,407.1971009)(252.47410278,407.45710297)
\curveto(252.49409721,407.52710057)(252.5040972,407.5971005)(252.50410278,407.66710297)
\curveto(252.51409719,407.74710035)(252.52909718,407.82710027)(252.54910278,407.90710297)
\lineto(252.54910278,408.04210297)
}
}
{
\newrgbcolor{curcolor}{0 0 0}
\pscustom[linestyle=none,fillstyle=solid,fillcolor=curcolor]
{
\newpath
\moveto(260.74238403,413.06710297)
\curveto(260.85237872,413.06709503)(260.94737862,413.05709504)(261.02738403,413.03710297)
\curveto(261.11737845,413.01709508)(261.18737838,412.97209513)(261.23738403,412.90210297)
\curveto(261.29737827,412.82209528)(261.32737824,412.68209542)(261.32738403,412.48210297)
\lineto(261.32738403,411.97210297)
\lineto(261.32738403,411.59710297)
\curveto(261.33737823,411.45709664)(261.32237825,411.34709675)(261.28238403,411.26710297)
\curveto(261.24237833,411.1970969)(261.18237839,411.15209695)(261.10238403,411.13210297)
\curveto(261.03237854,411.11209699)(260.94737862,411.102097)(260.84738403,411.10210297)
\curveto(260.75737881,411.102097)(260.65737891,411.10709699)(260.54738403,411.11710297)
\curveto(260.44737912,411.12709697)(260.35237922,411.12209698)(260.26238403,411.10210297)
\curveto(260.19237938,411.08209702)(260.12237945,411.06709703)(260.05238403,411.05710297)
\curveto(259.98237959,411.05709704)(259.91737965,411.04709705)(259.85738403,411.02710297)
\curveto(259.69737987,410.97709712)(259.53738003,410.9020972)(259.37738403,410.80210297)
\curveto(259.21738035,410.71209739)(259.09238048,410.60709749)(259.00238403,410.48710297)
\curveto(258.95238062,410.40709769)(258.89738067,410.32209778)(258.83738403,410.23210297)
\curveto(258.78738078,410.15209795)(258.73738083,410.06709803)(258.68738403,409.97710297)
\curveto(258.65738091,409.8970982)(258.62738094,409.81209829)(258.59738403,409.72210297)
\lineto(258.53738403,409.48210297)
\curveto(258.51738105,409.41209869)(258.50738106,409.33709876)(258.50738403,409.25710297)
\curveto(258.50738106,409.18709891)(258.49738107,409.11709898)(258.47738403,409.04710297)
\curveto(258.4673811,409.00709909)(258.46238111,408.96709913)(258.46238403,408.92710297)
\curveto(258.4723811,408.8970992)(258.4723811,408.86709923)(258.46238403,408.83710297)
\lineto(258.46238403,408.59710297)
\curveto(258.44238113,408.52709957)(258.43738113,408.44709965)(258.44738403,408.35710297)
\curveto(258.45738111,408.27709982)(258.46238111,408.1970999)(258.46238403,408.11710297)
\lineto(258.46238403,407.15710297)
\lineto(258.46238403,405.88210297)
\curveto(258.46238111,405.75210235)(258.45738111,405.63210247)(258.44738403,405.52210297)
\curveto(258.43738113,405.41210269)(258.40738116,405.32210278)(258.35738403,405.25210297)
\curveto(258.33738123,405.22210288)(258.30238127,405.1971029)(258.25238403,405.17710297)
\curveto(258.21238136,405.16710293)(258.1673814,405.15710294)(258.11738403,405.14710297)
\lineto(258.04238403,405.14710297)
\curveto(257.99238158,405.13710296)(257.93738163,405.13210297)(257.87738403,405.13210297)
\lineto(257.71238403,405.13210297)
\lineto(257.06738403,405.13210297)
\curveto(257.00738256,405.14210296)(256.94238263,405.14710295)(256.87238403,405.14710297)
\lineto(256.67738403,405.14710297)
\curveto(256.62738294,405.16710293)(256.57738299,405.18210292)(256.52738403,405.19210297)
\curveto(256.47738309,405.21210289)(256.44238313,405.24710285)(256.42238403,405.29710297)
\curveto(256.38238319,405.34710275)(256.35738321,405.41710268)(256.34738403,405.50710297)
\lineto(256.34738403,405.80710297)
\lineto(256.34738403,406.82710297)
\lineto(256.34738403,411.05710297)
\lineto(256.34738403,412.16710297)
\lineto(256.34738403,412.45210297)
\curveto(256.34738322,412.55209555)(256.3673832,412.63209547)(256.40738403,412.69210297)
\curveto(256.45738311,412.77209533)(256.53238304,412.82209528)(256.63238403,412.84210297)
\curveto(256.73238284,412.86209524)(256.85238272,412.87209523)(256.99238403,412.87210297)
\lineto(257.75738403,412.87210297)
\curveto(257.87738169,412.87209523)(257.98238159,412.86209524)(258.07238403,412.84210297)
\curveto(258.16238141,412.83209527)(258.23238134,412.78709531)(258.28238403,412.70710297)
\curveto(258.31238126,412.65709544)(258.32738124,412.58709551)(258.32738403,412.49710297)
\lineto(258.35738403,412.22710297)
\curveto(258.3673812,412.14709595)(258.38238119,412.07209603)(258.40238403,412.00210297)
\curveto(258.43238114,411.93209617)(258.48238109,411.8970962)(258.55238403,411.89710297)
\curveto(258.572381,411.91709618)(258.59238098,411.92709617)(258.61238403,411.92710297)
\curveto(258.63238094,411.92709617)(258.65238092,411.93709616)(258.67238403,411.95710297)
\curveto(258.73238084,412.00709609)(258.78238079,412.06209604)(258.82238403,412.12210297)
\curveto(258.8723807,412.19209591)(258.93238064,412.25209585)(259.00238403,412.30210297)
\curveto(259.04238053,412.33209577)(259.07738049,412.36209574)(259.10738403,412.39210297)
\curveto(259.13738043,412.43209567)(259.1723804,412.46709563)(259.21238403,412.49710297)
\lineto(259.48238403,412.67710297)
\curveto(259.58237999,412.73709536)(259.68237989,412.79209531)(259.78238403,412.84210297)
\curveto(259.88237969,412.88209522)(259.98237959,412.91709518)(260.08238403,412.94710297)
\lineto(260.41238403,413.03710297)
\curveto(260.44237913,413.04709505)(260.49737907,413.04709505)(260.57738403,413.03710297)
\curveto(260.6673789,413.03709506)(260.72237885,413.04709505)(260.74238403,413.06710297)
}
}
{
\newrgbcolor{curcolor}{0 0 0}
\pscustom[linestyle=none,fillstyle=solid,fillcolor=curcolor]
{
\newpath
\moveto(264.24746216,415.72210297)
\curveto(264.31745921,415.64209246)(264.35245917,415.52209258)(264.35246216,415.36210297)
\lineto(264.35246216,414.89710297)
\lineto(264.35246216,414.49210297)
\curveto(264.35245917,414.35209375)(264.31745921,414.25709384)(264.24746216,414.20710297)
\curveto(264.18745934,414.15709394)(264.10745942,414.12709397)(264.00746216,414.11710297)
\curveto(263.91745961,414.10709399)(263.81745971,414.102094)(263.70746216,414.10210297)
\lineto(262.86746216,414.10210297)
\curveto(262.75746077,414.102094)(262.65746087,414.10709399)(262.56746216,414.11710297)
\curveto(262.48746104,414.12709397)(262.41746111,414.15709394)(262.35746216,414.20710297)
\curveto(262.31746121,414.23709386)(262.28746124,414.29209381)(262.26746216,414.37210297)
\curveto(262.25746127,414.46209364)(262.24746128,414.55709354)(262.23746216,414.65710297)
\lineto(262.23746216,414.98710297)
\curveto(262.24746128,415.097093)(262.25246127,415.19209291)(262.25246216,415.27210297)
\lineto(262.25246216,415.48210297)
\curveto(262.26246126,415.55209255)(262.28246124,415.61209249)(262.31246216,415.66210297)
\curveto(262.33246119,415.7020924)(262.35746117,415.73209237)(262.38746216,415.75210297)
\lineto(262.50746216,415.81210297)
\curveto(262.527461,415.81209229)(262.55246097,415.81209229)(262.58246216,415.81210297)
\curveto(262.61246091,415.82209228)(262.63746089,415.82709227)(262.65746216,415.82710297)
\lineto(263.75246216,415.82710297)
\curveto(263.85245967,415.82709227)(263.94745958,415.82209228)(264.03746216,415.81210297)
\curveto(264.1274594,415.8020923)(264.19745933,415.77209233)(264.24746216,415.72210297)
\moveto(264.35246216,405.95710297)
\curveto(264.35245917,405.75710234)(264.34745918,405.58710251)(264.33746216,405.44710297)
\curveto(264.3274592,405.30710279)(264.23745929,405.21210289)(264.06746216,405.16210297)
\curveto(264.00745952,405.14210296)(263.94245958,405.13210297)(263.87246216,405.13210297)
\curveto(263.80245972,405.14210296)(263.7274598,405.14710295)(263.64746216,405.14710297)
\lineto(262.80746216,405.14710297)
\curveto(262.71746081,405.14710295)(262.6274609,405.15210295)(262.53746216,405.16210297)
\curveto(262.45746107,405.17210293)(262.39746113,405.2021029)(262.35746216,405.25210297)
\curveto(262.29746123,405.32210278)(262.26246126,405.40710269)(262.25246216,405.50710297)
\lineto(262.25246216,405.85210297)
\lineto(262.25246216,412.18210297)
\lineto(262.25246216,412.48210297)
\curveto(262.25246127,412.58209552)(262.27246125,412.66209544)(262.31246216,412.72210297)
\curveto(262.37246115,412.79209531)(262.45746107,412.83709526)(262.56746216,412.85710297)
\curveto(262.58746094,412.86709523)(262.61246091,412.86709523)(262.64246216,412.85710297)
\curveto(262.68246084,412.85709524)(262.71246081,412.86209524)(262.73246216,412.87210297)
\lineto(263.48246216,412.87210297)
\lineto(263.67746216,412.87210297)
\curveto(263.75745977,412.88209522)(263.8224597,412.88209522)(263.87246216,412.87210297)
\lineto(263.99246216,412.87210297)
\curveto(264.05245947,412.85209525)(264.10745942,412.83709526)(264.15746216,412.82710297)
\curveto(264.20745932,412.81709528)(264.24745928,412.78709531)(264.27746216,412.73710297)
\curveto(264.31745921,412.68709541)(264.33745919,412.61709548)(264.33746216,412.52710297)
\curveto(264.34745918,412.43709566)(264.35245917,412.34209576)(264.35246216,412.24210297)
\lineto(264.35246216,405.95710297)
}
}
{
\newrgbcolor{curcolor}{0 0 0}
\pscustom[linestyle=none,fillstyle=solid,fillcolor=curcolor]
{
\newpath
\moveto(273.78464966,409.31710297)
\curveto(273.80464109,409.25709884)(273.81464108,409.17209893)(273.81464966,409.06210297)
\curveto(273.81464108,408.95209915)(273.80464109,408.86709923)(273.78464966,408.80710297)
\lineto(273.78464966,408.65710297)
\curveto(273.76464113,408.57709952)(273.75464114,408.4970996)(273.75464966,408.41710297)
\curveto(273.76464113,408.33709976)(273.75964113,408.25709984)(273.73964966,408.17710297)
\curveto(273.71964117,408.10709999)(273.70464119,408.04210006)(273.69464966,407.98210297)
\curveto(273.68464121,407.92210018)(273.67464122,407.85710024)(273.66464966,407.78710297)
\curveto(273.62464127,407.67710042)(273.5896413,407.56210054)(273.55964966,407.44210297)
\curveto(273.52964136,407.33210077)(273.4896414,407.22710087)(273.43964966,407.12710297)
\curveto(273.22964166,406.64710145)(272.95464194,406.25710184)(272.61464966,405.95710297)
\curveto(272.27464262,405.65710244)(271.86464303,405.40710269)(271.38464966,405.20710297)
\curveto(271.26464363,405.15710294)(271.13964375,405.12210298)(271.00964966,405.10210297)
\curveto(270.889644,405.07210303)(270.76464413,405.04210306)(270.63464966,405.01210297)
\curveto(270.58464431,404.99210311)(270.52964436,404.98210312)(270.46964966,404.98210297)
\curveto(270.40964448,404.98210312)(270.35464454,404.97710312)(270.30464966,404.96710297)
\lineto(270.19964966,404.96710297)
\curveto(270.16964472,404.95710314)(270.13964475,404.95210315)(270.10964966,404.95210297)
\curveto(270.05964483,404.94210316)(269.97964491,404.93710316)(269.86964966,404.93710297)
\curveto(269.75964513,404.92710317)(269.67464522,404.93210317)(269.61464966,404.95210297)
\lineto(269.46464966,404.95210297)
\curveto(269.41464548,404.96210314)(269.35964553,404.96710313)(269.29964966,404.96710297)
\curveto(269.24964564,404.95710314)(269.19964569,404.96210314)(269.14964966,404.98210297)
\curveto(269.10964578,404.99210311)(269.06964582,404.9971031)(269.02964966,404.99710297)
\curveto(268.99964589,404.9971031)(268.95964593,405.0021031)(268.90964966,405.01210297)
\curveto(268.80964608,405.04210306)(268.70964618,405.06710303)(268.60964966,405.08710297)
\curveto(268.50964638,405.10710299)(268.41464648,405.13710296)(268.32464966,405.17710297)
\curveto(268.20464669,405.21710288)(268.0896468,405.25710284)(267.97964966,405.29710297)
\curveto(267.87964701,405.33710276)(267.77464712,405.38710271)(267.66464966,405.44710297)
\curveto(267.31464758,405.65710244)(267.01464788,405.9021022)(266.76464966,406.18210297)
\curveto(266.51464838,406.46210164)(266.30464859,406.7971013)(266.13464966,407.18710297)
\curveto(266.08464881,407.27710082)(266.04464885,407.37210073)(266.01464966,407.47210297)
\curveto(265.9946489,407.57210053)(265.96964892,407.67710042)(265.93964966,407.78710297)
\curveto(265.91964897,407.83710026)(265.90964898,407.88210022)(265.90964966,407.92210297)
\curveto(265.90964898,407.96210014)(265.89964899,408.00710009)(265.87964966,408.05710297)
\curveto(265.85964903,408.13709996)(265.84964904,408.21709988)(265.84964966,408.29710297)
\curveto(265.84964904,408.38709971)(265.83964905,408.47209963)(265.81964966,408.55210297)
\curveto(265.80964908,408.6020995)(265.80464909,408.64709945)(265.80464966,408.68710297)
\lineto(265.80464966,408.82210297)
\curveto(265.78464911,408.88209922)(265.77464912,408.96709913)(265.77464966,409.07710297)
\curveto(265.78464911,409.18709891)(265.79964909,409.27209883)(265.81964966,409.33210297)
\lineto(265.81964966,409.43710297)
\curveto(265.82964906,409.48709861)(265.82964906,409.53709856)(265.81964966,409.58710297)
\curveto(265.81964907,409.64709845)(265.82964906,409.7020984)(265.84964966,409.75210297)
\curveto(265.85964903,409.8020983)(265.86464903,409.84709825)(265.86464966,409.88710297)
\curveto(265.86464903,409.93709816)(265.87464902,409.98709811)(265.89464966,410.03710297)
\curveto(265.93464896,410.16709793)(265.96964892,410.29209781)(265.99964966,410.41210297)
\curveto(266.02964886,410.54209756)(266.06964882,410.66709743)(266.11964966,410.78710297)
\curveto(266.29964859,411.1970969)(266.51464838,411.53709656)(266.76464966,411.80710297)
\curveto(267.01464788,412.08709601)(267.31964757,412.34209576)(267.67964966,412.57210297)
\curveto(267.77964711,412.62209548)(267.88464701,412.66709543)(267.99464966,412.70710297)
\curveto(268.10464679,412.74709535)(268.21464668,412.79209531)(268.32464966,412.84210297)
\curveto(268.45464644,412.89209521)(268.5896463,412.92709517)(268.72964966,412.94710297)
\curveto(268.86964602,412.96709513)(269.01464588,412.9970951)(269.16464966,413.03710297)
\curveto(269.24464565,413.04709505)(269.31964557,413.05209505)(269.38964966,413.05210297)
\curveto(269.45964543,413.05209505)(269.52964536,413.05709504)(269.59964966,413.06710297)
\curveto(270.17964471,413.07709502)(270.67964421,413.01709508)(271.09964966,412.88710297)
\curveto(271.52964336,412.75709534)(271.90964298,412.57709552)(272.23964966,412.34710297)
\curveto(272.34964254,412.26709583)(272.45964243,412.17709592)(272.56964966,412.07710297)
\curveto(272.6896422,411.98709611)(272.7896421,411.88709621)(272.86964966,411.77710297)
\curveto(272.94964194,411.67709642)(273.01964187,411.57709652)(273.07964966,411.47710297)
\curveto(273.14964174,411.37709672)(273.21964167,411.27209683)(273.28964966,411.16210297)
\curveto(273.35964153,411.05209705)(273.41464148,410.93209717)(273.45464966,410.80210297)
\curveto(273.4946414,410.68209742)(273.53964135,410.55209755)(273.58964966,410.41210297)
\curveto(273.61964127,410.33209777)(273.64464125,410.24709785)(273.66464966,410.15710297)
\lineto(273.72464966,409.88710297)
\curveto(273.73464116,409.84709825)(273.73964115,409.80709829)(273.73964966,409.76710297)
\curveto(273.73964115,409.72709837)(273.74464115,409.68709841)(273.75464966,409.64710297)
\curveto(273.77464112,409.5970985)(273.77964111,409.54209856)(273.76964966,409.48210297)
\curveto(273.75964113,409.42209868)(273.76464113,409.36709873)(273.78464966,409.31710297)
\moveto(271.68464966,408.77710297)
\curveto(271.6946432,408.82709927)(271.69964319,408.8970992)(271.69964966,408.98710297)
\curveto(271.69964319,409.08709901)(271.6946432,409.16209894)(271.68464966,409.21210297)
\lineto(271.68464966,409.33210297)
\curveto(271.66464323,409.38209872)(271.65464324,409.43709866)(271.65464966,409.49710297)
\curveto(271.65464324,409.55709854)(271.64964324,409.61209849)(271.63964966,409.66210297)
\curveto(271.63964325,409.7020984)(271.63464326,409.73209837)(271.62464966,409.75210297)
\lineto(271.56464966,409.99210297)
\curveto(271.55464334,410.08209802)(271.53464336,410.16709793)(271.50464966,410.24710297)
\curveto(271.3946435,410.50709759)(271.26464363,410.72709737)(271.11464966,410.90710297)
\curveto(270.96464393,411.097097)(270.76464413,411.24709685)(270.51464966,411.35710297)
\curveto(270.45464444,411.37709672)(270.3946445,411.39209671)(270.33464966,411.40210297)
\curveto(270.27464462,411.42209668)(270.20964468,411.44209666)(270.13964966,411.46210297)
\curveto(270.05964483,411.48209662)(269.97464492,411.48709661)(269.88464966,411.47710297)
\lineto(269.61464966,411.47710297)
\curveto(269.58464531,411.45709664)(269.54964534,411.44709665)(269.50964966,411.44710297)
\curveto(269.46964542,411.45709664)(269.43464546,411.45709664)(269.40464966,411.44710297)
\lineto(269.19464966,411.38710297)
\curveto(269.13464576,411.37709672)(269.07964581,411.35709674)(269.02964966,411.32710297)
\curveto(268.77964611,411.21709688)(268.57464632,411.05709704)(268.41464966,410.84710297)
\curveto(268.26464663,410.64709745)(268.14464675,410.41209769)(268.05464966,410.14210297)
\curveto(268.02464687,410.04209806)(267.99964689,409.93709816)(267.97964966,409.82710297)
\curveto(267.96964692,409.71709838)(267.95464694,409.60709849)(267.93464966,409.49710297)
\curveto(267.92464697,409.44709865)(267.91964697,409.3970987)(267.91964966,409.34710297)
\lineto(267.91964966,409.19710297)
\curveto(267.89964699,409.12709897)(267.889647,409.02209908)(267.88964966,408.88210297)
\curveto(267.89964699,408.74209936)(267.91464698,408.63709946)(267.93464966,408.56710297)
\lineto(267.93464966,408.43210297)
\curveto(267.95464694,408.35209975)(267.96964692,408.27209983)(267.97964966,408.19210297)
\curveto(267.9896469,408.12209998)(268.00464689,408.04710005)(268.02464966,407.96710297)
\curveto(268.12464677,407.66710043)(268.22964666,407.42210068)(268.33964966,407.23210297)
\curveto(268.45964643,407.05210105)(268.64464625,406.88710121)(268.89464966,406.73710297)
\curveto(268.96464593,406.68710141)(269.03964585,406.64710145)(269.11964966,406.61710297)
\curveto(269.20964568,406.58710151)(269.29964559,406.56210154)(269.38964966,406.54210297)
\curveto(269.42964546,406.53210157)(269.46464543,406.52710157)(269.49464966,406.52710297)
\curveto(269.52464537,406.53710156)(269.55964533,406.53710156)(269.59964966,406.52710297)
\lineto(269.71964966,406.49710297)
\curveto(269.76964512,406.4971016)(269.81464508,406.5021016)(269.85464966,406.51210297)
\lineto(269.97464966,406.51210297)
\curveto(270.05464484,406.53210157)(270.13464476,406.54710155)(270.21464966,406.55710297)
\curveto(270.2946446,406.56710153)(270.36964452,406.58710151)(270.43964966,406.61710297)
\curveto(270.69964419,406.71710138)(270.90964398,406.85210125)(271.06964966,407.02210297)
\curveto(271.22964366,407.19210091)(271.36464353,407.4021007)(271.47464966,407.65210297)
\curveto(271.51464338,407.75210035)(271.54464335,407.85210025)(271.56464966,407.95210297)
\curveto(271.58464331,408.05210005)(271.60964328,408.15709994)(271.63964966,408.26710297)
\curveto(271.64964324,408.30709979)(271.65464324,408.34209976)(271.65464966,408.37210297)
\curveto(271.65464324,408.41209969)(271.65964323,408.45209965)(271.66964966,408.49210297)
\lineto(271.66964966,408.62710297)
\curveto(271.66964322,408.67709942)(271.67464322,408.72709937)(271.68464966,408.77710297)
}
}
{
\newrgbcolor{curcolor}{0 0 0}
\pscustom[linestyle=none,fillstyle=solid,fillcolor=curcolor]
{
\newpath
\moveto(278.15457153,413.08210297)
\curveto(278.90456703,413.102095)(279.55456638,413.01709508)(280.10457153,412.82710297)
\curveto(280.66456527,412.64709545)(281.08956485,412.33209577)(281.37957153,411.88210297)
\curveto(281.44956449,411.77209633)(281.50956443,411.65709644)(281.55957153,411.53710297)
\curveto(281.61956432,411.42709667)(281.66956427,411.3020968)(281.70957153,411.16210297)
\curveto(281.72956421,411.102097)(281.7395642,411.03709706)(281.73957153,410.96710297)
\curveto(281.7395642,410.8970972)(281.72956421,410.83709726)(281.70957153,410.78710297)
\curveto(281.66956427,410.72709737)(281.61456432,410.68709741)(281.54457153,410.66710297)
\curveto(281.49456444,410.64709745)(281.4345645,410.63709746)(281.36457153,410.63710297)
\lineto(281.15457153,410.63710297)
\lineto(280.49457153,410.63710297)
\curveto(280.42456551,410.63709746)(280.35456558,410.63209747)(280.28457153,410.62210297)
\curveto(280.21456572,410.62209748)(280.14956579,410.63209747)(280.08957153,410.65210297)
\curveto(279.98956595,410.67209743)(279.91456602,410.71209739)(279.86457153,410.77210297)
\curveto(279.81456612,410.83209727)(279.76956617,410.89209721)(279.72957153,410.95210297)
\lineto(279.60957153,411.16210297)
\curveto(279.57956636,411.24209686)(279.52956641,411.30709679)(279.45957153,411.35710297)
\curveto(279.35956658,411.43709666)(279.25956668,411.4970966)(279.15957153,411.53710297)
\curveto(279.06956687,411.57709652)(278.95456698,411.61209649)(278.81457153,411.64210297)
\curveto(278.74456719,411.66209644)(278.6395673,411.67709642)(278.49957153,411.68710297)
\curveto(278.36956757,411.6970964)(278.26956767,411.69209641)(278.19957153,411.67210297)
\lineto(278.09457153,411.67210297)
\lineto(277.94457153,411.64210297)
\curveto(277.90456803,411.64209646)(277.85956808,411.63709646)(277.80957153,411.62710297)
\curveto(277.6395683,411.57709652)(277.49956844,411.50709659)(277.38957153,411.41710297)
\curveto(277.28956865,411.33709676)(277.21956872,411.21209689)(277.17957153,411.04210297)
\curveto(277.15956878,410.97209713)(277.15956878,410.90709719)(277.17957153,410.84710297)
\curveto(277.19956874,410.78709731)(277.21956872,410.73709736)(277.23957153,410.69710297)
\curveto(277.30956863,410.57709752)(277.38956855,410.48209762)(277.47957153,410.41210297)
\curveto(277.57956836,410.34209776)(277.69456824,410.28209782)(277.82457153,410.23210297)
\curveto(278.01456792,410.15209795)(278.21956772,410.08209802)(278.43957153,410.02210297)
\lineto(279.12957153,409.87210297)
\curveto(279.36956657,409.83209827)(279.59956634,409.78209832)(279.81957153,409.72210297)
\curveto(280.04956589,409.67209843)(280.26456567,409.60709849)(280.46457153,409.52710297)
\curveto(280.55456538,409.48709861)(280.6395653,409.45209865)(280.71957153,409.42210297)
\curveto(280.80956513,409.4020987)(280.89456504,409.36709873)(280.97457153,409.31710297)
\curveto(281.16456477,409.1970989)(281.3345646,409.06709903)(281.48457153,408.92710297)
\curveto(281.64456429,408.78709931)(281.76956417,408.61209949)(281.85957153,408.40210297)
\curveto(281.88956405,408.33209977)(281.91456402,408.26209984)(281.93457153,408.19210297)
\curveto(281.95456398,408.12209998)(281.97456396,408.04710005)(281.99457153,407.96710297)
\curveto(282.00456393,407.90710019)(282.00956393,407.81210029)(282.00957153,407.68210297)
\curveto(282.01956392,407.56210054)(282.01956392,407.46710063)(282.00957153,407.39710297)
\lineto(282.00957153,407.32210297)
\curveto(281.98956395,407.26210084)(281.97456396,407.2021009)(281.96457153,407.14210297)
\curveto(281.96456397,407.09210101)(281.95956398,407.04210106)(281.94957153,406.99210297)
\curveto(281.87956406,406.69210141)(281.76956417,406.42710167)(281.61957153,406.19710297)
\curveto(281.45956448,405.95710214)(281.26456467,405.76210234)(281.03457153,405.61210297)
\curveto(280.80456513,405.46210264)(280.54456539,405.33210277)(280.25457153,405.22210297)
\curveto(280.14456579,405.17210293)(280.02456591,405.13710296)(279.89457153,405.11710297)
\curveto(279.77456616,405.097103)(279.65456628,405.07210303)(279.53457153,405.04210297)
\curveto(279.44456649,405.02210308)(279.34956659,405.01210309)(279.24957153,405.01210297)
\curveto(279.15956678,405.0021031)(279.06956687,404.98710311)(278.97957153,404.96710297)
\lineto(278.70957153,404.96710297)
\curveto(278.64956729,404.94710315)(278.54456739,404.93710316)(278.39457153,404.93710297)
\curveto(278.25456768,404.93710316)(278.15456778,404.94710315)(278.09457153,404.96710297)
\curveto(278.06456787,404.96710313)(278.02956791,404.97210313)(277.98957153,404.98210297)
\lineto(277.88457153,404.98210297)
\curveto(277.76456817,405.0021031)(277.64456829,405.01710308)(277.52457153,405.02710297)
\curveto(277.40456853,405.03710306)(277.28956865,405.05710304)(277.17957153,405.08710297)
\curveto(276.78956915,405.1971029)(276.44456949,405.32210278)(276.14457153,405.46210297)
\curveto(275.84457009,405.61210249)(275.58957035,405.83210227)(275.37957153,406.12210297)
\curveto(275.2395707,406.31210179)(275.11957082,406.53210157)(275.01957153,406.78210297)
\curveto(274.99957094,406.84210126)(274.97957096,406.92210118)(274.95957153,407.02210297)
\curveto(274.939571,407.07210103)(274.92457101,407.14210096)(274.91457153,407.23210297)
\curveto(274.90457103,407.32210078)(274.90957103,407.3971007)(274.92957153,407.45710297)
\curveto(274.95957098,407.52710057)(275.00957093,407.57710052)(275.07957153,407.60710297)
\curveto(275.12957081,407.62710047)(275.18957075,407.63710046)(275.25957153,407.63710297)
\lineto(275.48457153,407.63710297)
\lineto(276.18957153,407.63710297)
\lineto(276.42957153,407.63710297)
\curveto(276.50956943,407.63710046)(276.57956936,407.62710047)(276.63957153,407.60710297)
\curveto(276.74956919,407.56710053)(276.81956912,407.5021006)(276.84957153,407.41210297)
\curveto(276.88956905,407.32210078)(276.934569,407.22710087)(276.98457153,407.12710297)
\curveto(277.00456893,407.07710102)(277.0395689,407.01210109)(277.08957153,406.93210297)
\curveto(277.14956879,406.85210125)(277.19956874,406.8021013)(277.23957153,406.78210297)
\curveto(277.35956858,406.68210142)(277.47456846,406.6021015)(277.58457153,406.54210297)
\curveto(277.69456824,406.49210161)(277.8345681,406.44210166)(278.00457153,406.39210297)
\curveto(278.05456788,406.37210173)(278.10456783,406.36210174)(278.15457153,406.36210297)
\curveto(278.20456773,406.37210173)(278.25456768,406.37210173)(278.30457153,406.36210297)
\curveto(278.38456755,406.34210176)(278.46956747,406.33210177)(278.55957153,406.33210297)
\curveto(278.65956728,406.34210176)(278.74456719,406.35710174)(278.81457153,406.37710297)
\curveto(278.86456707,406.38710171)(278.90956703,406.39210171)(278.94957153,406.39210297)
\curveto(278.99956694,406.39210171)(279.04956689,406.4021017)(279.09957153,406.42210297)
\curveto(279.2395667,406.47210163)(279.36456657,406.53210157)(279.47457153,406.60210297)
\curveto(279.59456634,406.67210143)(279.68956625,406.76210134)(279.75957153,406.87210297)
\curveto(279.80956613,406.95210115)(279.84956609,407.07710102)(279.87957153,407.24710297)
\curveto(279.89956604,407.31710078)(279.89956604,407.38210072)(279.87957153,407.44210297)
\curveto(279.85956608,407.5021006)(279.8395661,407.55210055)(279.81957153,407.59210297)
\curveto(279.74956619,407.73210037)(279.65956628,407.83710026)(279.54957153,407.90710297)
\curveto(279.44956649,407.97710012)(279.32956661,408.04210006)(279.18957153,408.10210297)
\curveto(278.99956694,408.18209992)(278.79956714,408.24709985)(278.58957153,408.29710297)
\curveto(278.37956756,408.34709975)(278.16956777,408.4020997)(277.95957153,408.46210297)
\curveto(277.87956806,408.48209962)(277.79456814,408.4970996)(277.70457153,408.50710297)
\curveto(277.62456831,408.51709958)(277.54456839,408.53209957)(277.46457153,408.55210297)
\curveto(277.14456879,408.64209946)(276.8395691,408.72709937)(276.54957153,408.80710297)
\curveto(276.25956968,408.8970992)(275.99456994,409.02709907)(275.75457153,409.19710297)
\curveto(275.47457046,409.3970987)(275.26957067,409.66709843)(275.13957153,410.00710297)
\curveto(275.11957082,410.07709802)(275.09957084,410.17209793)(275.07957153,410.29210297)
\curveto(275.05957088,410.36209774)(275.04457089,410.44709765)(275.03457153,410.54710297)
\curveto(275.02457091,410.64709745)(275.02957091,410.73709736)(275.04957153,410.81710297)
\curveto(275.06957087,410.86709723)(275.07457086,410.90709719)(275.06457153,410.93710297)
\curveto(275.05457088,410.97709712)(275.05957088,411.02209708)(275.07957153,411.07210297)
\curveto(275.09957084,411.18209692)(275.11957082,411.28209682)(275.13957153,411.37210297)
\curveto(275.16957077,411.47209663)(275.20457073,411.56709653)(275.24457153,411.65710297)
\curveto(275.37457056,411.94709615)(275.55457038,412.18209592)(275.78457153,412.36210297)
\curveto(276.01456992,412.54209556)(276.27456966,412.68709541)(276.56457153,412.79710297)
\curveto(276.67456926,412.84709525)(276.78956915,412.88209522)(276.90957153,412.90210297)
\curveto(277.02956891,412.93209517)(277.15456878,412.96209514)(277.28457153,412.99210297)
\curveto(277.34456859,413.01209509)(277.40456853,413.02209508)(277.46457153,413.02210297)
\lineto(277.64457153,413.05210297)
\curveto(277.72456821,413.06209504)(277.80956813,413.06709503)(277.89957153,413.06710297)
\curveto(277.98956795,413.06709503)(278.07456786,413.07209503)(278.15457153,413.08210297)
}
}
{
\newrgbcolor{curcolor}{0 0 0}
\pscustom[linestyle=none,fillstyle=solid,fillcolor=curcolor]
{
}
}
{
\newrgbcolor{curcolor}{0 0 0}
\pscustom[linestyle=none,fillstyle=solid,fillcolor=curcolor]
{
\newpath
\moveto(295.29136841,409.09210297)
\curveto(295.30135973,409.03209907)(295.30635972,408.94209916)(295.30636841,408.82210297)
\curveto(295.30635972,408.7020994)(295.29635973,408.61709948)(295.27636841,408.56710297)
\lineto(295.27636841,408.37210297)
\curveto(295.24635978,408.26209984)(295.2263598,408.15709994)(295.21636841,408.05710297)
\curveto(295.21635981,407.95710014)(295.20135983,407.85710024)(295.17136841,407.75710297)
\curveto(295.15135988,407.66710043)(295.1313599,407.57210053)(295.11136841,407.47210297)
\curveto(295.09135994,407.38210072)(295.06135997,407.29210081)(295.02136841,407.20210297)
\curveto(294.95136008,407.03210107)(294.88136015,406.87210123)(294.81136841,406.72210297)
\curveto(294.74136029,406.58210152)(294.66136037,406.44210166)(294.57136841,406.30210297)
\curveto(294.51136052,406.21210189)(294.44636058,406.12710197)(294.37636841,406.04710297)
\curveto(294.31636071,405.97710212)(294.24636078,405.9021022)(294.16636841,405.82210297)
\lineto(294.06136841,405.71710297)
\curveto(294.01136102,405.66710243)(293.95636107,405.62210248)(293.89636841,405.58210297)
\lineto(293.74636841,405.46210297)
\curveto(293.66636136,405.4021027)(293.57636145,405.34710275)(293.47636841,405.29710297)
\curveto(293.38636164,405.25710284)(293.29136174,405.21210289)(293.19136841,405.16210297)
\curveto(293.09136194,405.11210299)(292.98636204,405.07710302)(292.87636841,405.05710297)
\curveto(292.77636225,405.03710306)(292.67136236,405.01710308)(292.56136841,404.99710297)
\curveto(292.50136253,404.97710312)(292.43636259,404.96710313)(292.36636841,404.96710297)
\curveto(292.30636272,404.96710313)(292.24136279,404.95710314)(292.17136841,404.93710297)
\lineto(292.03636841,404.93710297)
\curveto(291.95636307,404.91710318)(291.88136315,404.91710318)(291.81136841,404.93710297)
\lineto(291.66136841,404.93710297)
\curveto(291.60136343,404.95710314)(291.53636349,404.96710313)(291.46636841,404.96710297)
\curveto(291.40636362,404.95710314)(291.34636368,404.96210314)(291.28636841,404.98210297)
\curveto(291.1263639,405.03210307)(290.97136406,405.07710302)(290.82136841,405.11710297)
\curveto(290.68136435,405.15710294)(290.55136448,405.21710288)(290.43136841,405.29710297)
\curveto(290.36136467,405.33710276)(290.29636473,405.37710272)(290.23636841,405.41710297)
\curveto(290.17636485,405.46710263)(290.11136492,405.51710258)(290.04136841,405.56710297)
\lineto(289.86136841,405.70210297)
\curveto(289.78136525,405.76210234)(289.71136532,405.76710233)(289.65136841,405.71710297)
\curveto(289.60136543,405.68710241)(289.57636545,405.64710245)(289.57636841,405.59710297)
\curveto(289.57636545,405.55710254)(289.56636546,405.50710259)(289.54636841,405.44710297)
\curveto(289.5263655,405.34710275)(289.51636551,405.23210287)(289.51636841,405.10210297)
\curveto(289.5263655,404.97210313)(289.5313655,404.85210325)(289.53136841,404.74210297)
\lineto(289.53136841,403.21210297)
\curveto(289.5313655,403.08210502)(289.5263655,402.95710514)(289.51636841,402.83710297)
\curveto(289.51636551,402.70710539)(289.49136554,402.6021055)(289.44136841,402.52210297)
\curveto(289.41136562,402.48210562)(289.35636567,402.45210565)(289.27636841,402.43210297)
\curveto(289.19636583,402.41210569)(289.10636592,402.4021057)(289.00636841,402.40210297)
\curveto(288.90636612,402.39210571)(288.80636622,402.39210571)(288.70636841,402.40210297)
\lineto(288.45136841,402.40210297)
\lineto(288.04636841,402.40210297)
\lineto(287.94136841,402.40210297)
\curveto(287.90136713,402.4021057)(287.86636716,402.40710569)(287.83636841,402.41710297)
\lineto(287.71636841,402.41710297)
\curveto(287.54636748,402.46710563)(287.45636757,402.56710553)(287.44636841,402.71710297)
\curveto(287.43636759,402.85710524)(287.4313676,403.02710507)(287.43136841,403.22710297)
\lineto(287.43136841,412.03210297)
\curveto(287.4313676,412.14209596)(287.4263676,412.25709584)(287.41636841,412.37710297)
\curveto(287.41636761,412.50709559)(287.44136759,412.60709549)(287.49136841,412.67710297)
\curveto(287.5313675,412.74709535)(287.58636744,412.79209531)(287.65636841,412.81210297)
\curveto(287.70636732,412.83209527)(287.76636726,412.84209526)(287.83636841,412.84210297)
\lineto(288.06136841,412.84210297)
\lineto(288.78136841,412.84210297)
\lineto(289.06636841,412.84210297)
\curveto(289.15636587,412.84209526)(289.2313658,412.81709528)(289.29136841,412.76710297)
\curveto(289.36136567,412.71709538)(289.39636563,412.65209545)(289.39636841,412.57210297)
\curveto(289.40636562,412.5020956)(289.4313656,412.42709567)(289.47136841,412.34710297)
\curveto(289.48136555,412.31709578)(289.49136554,412.29209581)(289.50136841,412.27210297)
\curveto(289.52136551,412.26209584)(289.54136549,412.24709585)(289.56136841,412.22710297)
\curveto(289.67136536,412.21709588)(289.76136527,412.24709585)(289.83136841,412.31710297)
\curveto(289.90136513,412.38709571)(289.97136506,412.44709565)(290.04136841,412.49710297)
\curveto(290.17136486,412.58709551)(290.30636472,412.66709543)(290.44636841,412.73710297)
\curveto(290.58636444,412.81709528)(290.74136429,412.88209522)(290.91136841,412.93210297)
\curveto(290.99136404,412.96209514)(291.07636395,412.98209512)(291.16636841,412.99210297)
\curveto(291.26636376,413.0020951)(291.36136367,413.01709508)(291.45136841,413.03710297)
\curveto(291.49136354,413.04709505)(291.5313635,413.04709505)(291.57136841,413.03710297)
\curveto(291.62136341,413.02709507)(291.66136337,413.03209507)(291.69136841,413.05210297)
\curveto(292.26136277,413.07209503)(292.74136229,412.99209511)(293.13136841,412.81210297)
\curveto(293.5313615,412.64209546)(293.87136116,412.41709568)(294.15136841,412.13710297)
\curveto(294.20136083,412.08709601)(294.24636078,412.03709606)(294.28636841,411.98710297)
\curveto(294.3263607,411.94709615)(294.36636066,411.9020962)(294.40636841,411.85210297)
\curveto(294.47636055,411.76209634)(294.53636049,411.67209643)(294.58636841,411.58210297)
\curveto(294.64636038,411.49209661)(294.70136033,411.4020967)(294.75136841,411.31210297)
\curveto(294.77136026,411.29209681)(294.78136025,411.26709683)(294.78136841,411.23710297)
\curveto(294.79136024,411.20709689)(294.80636022,411.17209693)(294.82636841,411.13210297)
\curveto(294.88636014,411.03209707)(294.94136009,410.91209719)(294.99136841,410.77210297)
\curveto(295.01136002,410.71209739)(295.03136,410.64709745)(295.05136841,410.57710297)
\curveto(295.07135996,410.51709758)(295.09135994,410.45209765)(295.11136841,410.38210297)
\curveto(295.15135988,410.26209784)(295.17635985,410.13709796)(295.18636841,410.00710297)
\curveto(295.20635982,409.87709822)(295.2313598,409.74209836)(295.26136841,409.60210297)
\lineto(295.26136841,409.43710297)
\lineto(295.29136841,409.25710297)
\lineto(295.29136841,409.09210297)
\moveto(293.17636841,408.74710297)
\curveto(293.18636184,408.7970993)(293.19136184,408.86209924)(293.19136841,408.94210297)
\curveto(293.19136184,409.03209907)(293.18636184,409.102099)(293.17636841,409.15210297)
\lineto(293.17636841,409.28710297)
\curveto(293.15636187,409.34709875)(293.14636188,409.41209869)(293.14636841,409.48210297)
\curveto(293.14636188,409.55209855)(293.13636189,409.62209848)(293.11636841,409.69210297)
\curveto(293.09636193,409.79209831)(293.07636195,409.88709821)(293.05636841,409.97710297)
\curveto(293.03636199,410.07709802)(293.00636202,410.16709793)(292.96636841,410.24710297)
\curveto(292.84636218,410.56709753)(292.69136234,410.82209728)(292.50136841,411.01210297)
\curveto(292.31136272,411.2020969)(292.04136299,411.34209676)(291.69136841,411.43210297)
\curveto(291.61136342,411.45209665)(291.52136351,411.46209664)(291.42136841,411.46210297)
\lineto(291.15136841,411.46210297)
\curveto(291.11136392,411.45209665)(291.07636395,411.44709665)(291.04636841,411.44710297)
\curveto(291.01636401,411.44709665)(290.98136405,411.44209666)(290.94136841,411.43210297)
\lineto(290.73136841,411.37210297)
\curveto(290.67136436,411.36209674)(290.61136442,411.34209676)(290.55136841,411.31210297)
\curveto(290.29136474,411.2020969)(290.08636494,411.03209707)(289.93636841,410.80210297)
\curveto(289.79636523,410.57209753)(289.68136535,410.31709778)(289.59136841,410.03710297)
\curveto(289.57136546,409.95709814)(289.55636547,409.87209823)(289.54636841,409.78210297)
\curveto(289.53636549,409.7020984)(289.52136551,409.62209848)(289.50136841,409.54210297)
\curveto(289.49136554,409.5020986)(289.48636554,409.43709866)(289.48636841,409.34710297)
\curveto(289.46636556,409.30709879)(289.46136557,409.25709884)(289.47136841,409.19710297)
\curveto(289.48136555,409.14709895)(289.48136555,409.097099)(289.47136841,409.04710297)
\curveto(289.45136558,408.98709911)(289.45136558,408.93209917)(289.47136841,408.88210297)
\lineto(289.47136841,408.70210297)
\lineto(289.47136841,408.56710297)
\curveto(289.47136556,408.52709957)(289.48136555,408.48709961)(289.50136841,408.44710297)
\curveto(289.50136553,408.37709972)(289.50636552,408.32209978)(289.51636841,408.28210297)
\lineto(289.54636841,408.10210297)
\curveto(289.55636547,408.04210006)(289.57136546,407.98210012)(289.59136841,407.92210297)
\curveto(289.68136535,407.63210047)(289.78636524,407.39210071)(289.90636841,407.20210297)
\curveto(290.03636499,407.02210108)(290.21636481,406.86210124)(290.44636841,406.72210297)
\curveto(290.58636444,406.64210146)(290.75136428,406.57710152)(290.94136841,406.52710297)
\curveto(290.98136405,406.51710158)(291.01636401,406.51210159)(291.04636841,406.51210297)
\curveto(291.07636395,406.52210158)(291.11136392,406.52210158)(291.15136841,406.51210297)
\curveto(291.19136384,406.5021016)(291.25136378,406.49210161)(291.33136841,406.48210297)
\curveto(291.41136362,406.48210162)(291.47636355,406.48710161)(291.52636841,406.49710297)
\curveto(291.60636342,406.51710158)(291.68636334,406.53210157)(291.76636841,406.54210297)
\curveto(291.85636317,406.56210154)(291.94136309,406.58710151)(292.02136841,406.61710297)
\curveto(292.26136277,406.71710138)(292.45636257,406.85710124)(292.60636841,407.03710297)
\curveto(292.75636227,407.21710088)(292.88136215,407.42710067)(292.98136841,407.66710297)
\curveto(293.031362,407.78710031)(293.06636196,407.91210019)(293.08636841,408.04210297)
\curveto(293.10636192,408.17209993)(293.1313619,408.30709979)(293.16136841,408.44710297)
\lineto(293.16136841,408.59710297)
\curveto(293.17136186,408.64709945)(293.17636185,408.6970994)(293.17636841,408.74710297)
}
}
{
\newrgbcolor{curcolor}{0 0 0}
\pscustom[linestyle=none,fillstyle=solid,fillcolor=curcolor]
{
\newpath
\moveto(304.34129028,409.31710297)
\curveto(304.36128171,409.25709884)(304.3712817,409.17209893)(304.37129028,409.06210297)
\curveto(304.3712817,408.95209915)(304.36128171,408.86709923)(304.34129028,408.80710297)
\lineto(304.34129028,408.65710297)
\curveto(304.32128175,408.57709952)(304.31128176,408.4970996)(304.31129028,408.41710297)
\curveto(304.32128175,408.33709976)(304.31628176,408.25709984)(304.29629028,408.17710297)
\curveto(304.2762818,408.10709999)(304.26128181,408.04210006)(304.25129028,407.98210297)
\curveto(304.24128183,407.92210018)(304.23128184,407.85710024)(304.22129028,407.78710297)
\curveto(304.18128189,407.67710042)(304.14628193,407.56210054)(304.11629028,407.44210297)
\curveto(304.08628199,407.33210077)(304.04628203,407.22710087)(303.99629028,407.12710297)
\curveto(303.78628229,406.64710145)(303.51128256,406.25710184)(303.17129028,405.95710297)
\curveto(302.83128324,405.65710244)(302.42128365,405.40710269)(301.94129028,405.20710297)
\curveto(301.82128425,405.15710294)(301.69628438,405.12210298)(301.56629028,405.10210297)
\curveto(301.44628463,405.07210303)(301.32128475,405.04210306)(301.19129028,405.01210297)
\curveto(301.14128493,404.99210311)(301.08628499,404.98210312)(301.02629028,404.98210297)
\curveto(300.96628511,404.98210312)(300.91128516,404.97710312)(300.86129028,404.96710297)
\lineto(300.75629028,404.96710297)
\curveto(300.72628535,404.95710314)(300.69628538,404.95210315)(300.66629028,404.95210297)
\curveto(300.61628546,404.94210316)(300.53628554,404.93710316)(300.42629028,404.93710297)
\curveto(300.31628576,404.92710317)(300.23128584,404.93210317)(300.17129028,404.95210297)
\lineto(300.02129028,404.95210297)
\curveto(299.9712861,404.96210314)(299.91628616,404.96710313)(299.85629028,404.96710297)
\curveto(299.80628627,404.95710314)(299.75628632,404.96210314)(299.70629028,404.98210297)
\curveto(299.66628641,404.99210311)(299.62628645,404.9971031)(299.58629028,404.99710297)
\curveto(299.55628652,404.9971031)(299.51628656,405.0021031)(299.46629028,405.01210297)
\curveto(299.36628671,405.04210306)(299.26628681,405.06710303)(299.16629028,405.08710297)
\curveto(299.06628701,405.10710299)(298.9712871,405.13710296)(298.88129028,405.17710297)
\curveto(298.76128731,405.21710288)(298.64628743,405.25710284)(298.53629028,405.29710297)
\curveto(298.43628764,405.33710276)(298.33128774,405.38710271)(298.22129028,405.44710297)
\curveto(297.8712882,405.65710244)(297.5712885,405.9021022)(297.32129028,406.18210297)
\curveto(297.071289,406.46210164)(296.86128921,406.7971013)(296.69129028,407.18710297)
\curveto(296.64128943,407.27710082)(296.60128947,407.37210073)(296.57129028,407.47210297)
\curveto(296.55128952,407.57210053)(296.52628955,407.67710042)(296.49629028,407.78710297)
\curveto(296.4762896,407.83710026)(296.46628961,407.88210022)(296.46629028,407.92210297)
\curveto(296.46628961,407.96210014)(296.45628962,408.00710009)(296.43629028,408.05710297)
\curveto(296.41628966,408.13709996)(296.40628967,408.21709988)(296.40629028,408.29710297)
\curveto(296.40628967,408.38709971)(296.39628968,408.47209963)(296.37629028,408.55210297)
\curveto(296.36628971,408.6020995)(296.36128971,408.64709945)(296.36129028,408.68710297)
\lineto(296.36129028,408.82210297)
\curveto(296.34128973,408.88209922)(296.33128974,408.96709913)(296.33129028,409.07710297)
\curveto(296.34128973,409.18709891)(296.35628972,409.27209883)(296.37629028,409.33210297)
\lineto(296.37629028,409.43710297)
\curveto(296.38628969,409.48709861)(296.38628969,409.53709856)(296.37629028,409.58710297)
\curveto(296.3762897,409.64709845)(296.38628969,409.7020984)(296.40629028,409.75210297)
\curveto(296.41628966,409.8020983)(296.42128965,409.84709825)(296.42129028,409.88710297)
\curveto(296.42128965,409.93709816)(296.43128964,409.98709811)(296.45129028,410.03710297)
\curveto(296.49128958,410.16709793)(296.52628955,410.29209781)(296.55629028,410.41210297)
\curveto(296.58628949,410.54209756)(296.62628945,410.66709743)(296.67629028,410.78710297)
\curveto(296.85628922,411.1970969)(297.071289,411.53709656)(297.32129028,411.80710297)
\curveto(297.5712885,412.08709601)(297.8762882,412.34209576)(298.23629028,412.57210297)
\curveto(298.33628774,412.62209548)(298.44128763,412.66709543)(298.55129028,412.70710297)
\curveto(298.66128741,412.74709535)(298.7712873,412.79209531)(298.88129028,412.84210297)
\curveto(299.01128706,412.89209521)(299.14628693,412.92709517)(299.28629028,412.94710297)
\curveto(299.42628665,412.96709513)(299.5712865,412.9970951)(299.72129028,413.03710297)
\curveto(299.80128627,413.04709505)(299.8762862,413.05209505)(299.94629028,413.05210297)
\curveto(300.01628606,413.05209505)(300.08628599,413.05709504)(300.15629028,413.06710297)
\curveto(300.73628534,413.07709502)(301.23628484,413.01709508)(301.65629028,412.88710297)
\curveto(302.08628399,412.75709534)(302.46628361,412.57709552)(302.79629028,412.34710297)
\curveto(302.90628317,412.26709583)(303.01628306,412.17709592)(303.12629028,412.07710297)
\curveto(303.24628283,411.98709611)(303.34628273,411.88709621)(303.42629028,411.77710297)
\curveto(303.50628257,411.67709642)(303.5762825,411.57709652)(303.63629028,411.47710297)
\curveto(303.70628237,411.37709672)(303.7762823,411.27209683)(303.84629028,411.16210297)
\curveto(303.91628216,411.05209705)(303.9712821,410.93209717)(304.01129028,410.80210297)
\curveto(304.05128202,410.68209742)(304.09628198,410.55209755)(304.14629028,410.41210297)
\curveto(304.1762819,410.33209777)(304.20128187,410.24709785)(304.22129028,410.15710297)
\lineto(304.28129028,409.88710297)
\curveto(304.29128178,409.84709825)(304.29628178,409.80709829)(304.29629028,409.76710297)
\curveto(304.29628178,409.72709837)(304.30128177,409.68709841)(304.31129028,409.64710297)
\curveto(304.33128174,409.5970985)(304.33628174,409.54209856)(304.32629028,409.48210297)
\curveto(304.31628176,409.42209868)(304.32128175,409.36709873)(304.34129028,409.31710297)
\moveto(302.24129028,408.77710297)
\curveto(302.25128382,408.82709927)(302.25628382,408.8970992)(302.25629028,408.98710297)
\curveto(302.25628382,409.08709901)(302.25128382,409.16209894)(302.24129028,409.21210297)
\lineto(302.24129028,409.33210297)
\curveto(302.22128385,409.38209872)(302.21128386,409.43709866)(302.21129028,409.49710297)
\curveto(302.21128386,409.55709854)(302.20628387,409.61209849)(302.19629028,409.66210297)
\curveto(302.19628388,409.7020984)(302.19128388,409.73209837)(302.18129028,409.75210297)
\lineto(302.12129028,409.99210297)
\curveto(302.11128396,410.08209802)(302.09128398,410.16709793)(302.06129028,410.24710297)
\curveto(301.95128412,410.50709759)(301.82128425,410.72709737)(301.67129028,410.90710297)
\curveto(301.52128455,411.097097)(301.32128475,411.24709685)(301.07129028,411.35710297)
\curveto(301.01128506,411.37709672)(300.95128512,411.39209671)(300.89129028,411.40210297)
\curveto(300.83128524,411.42209668)(300.76628531,411.44209666)(300.69629028,411.46210297)
\curveto(300.61628546,411.48209662)(300.53128554,411.48709661)(300.44129028,411.47710297)
\lineto(300.17129028,411.47710297)
\curveto(300.14128593,411.45709664)(300.10628597,411.44709665)(300.06629028,411.44710297)
\curveto(300.02628605,411.45709664)(299.99128608,411.45709664)(299.96129028,411.44710297)
\lineto(299.75129028,411.38710297)
\curveto(299.69128638,411.37709672)(299.63628644,411.35709674)(299.58629028,411.32710297)
\curveto(299.33628674,411.21709688)(299.13128694,411.05709704)(298.97129028,410.84710297)
\curveto(298.82128725,410.64709745)(298.70128737,410.41209769)(298.61129028,410.14210297)
\curveto(298.58128749,410.04209806)(298.55628752,409.93709816)(298.53629028,409.82710297)
\curveto(298.52628755,409.71709838)(298.51128756,409.60709849)(298.49129028,409.49710297)
\curveto(298.48128759,409.44709865)(298.4762876,409.3970987)(298.47629028,409.34710297)
\lineto(298.47629028,409.19710297)
\curveto(298.45628762,409.12709897)(298.44628763,409.02209908)(298.44629028,408.88210297)
\curveto(298.45628762,408.74209936)(298.4712876,408.63709946)(298.49129028,408.56710297)
\lineto(298.49129028,408.43210297)
\curveto(298.51128756,408.35209975)(298.52628755,408.27209983)(298.53629028,408.19210297)
\curveto(298.54628753,408.12209998)(298.56128751,408.04710005)(298.58129028,407.96710297)
\curveto(298.68128739,407.66710043)(298.78628729,407.42210068)(298.89629028,407.23210297)
\curveto(299.01628706,407.05210105)(299.20128687,406.88710121)(299.45129028,406.73710297)
\curveto(299.52128655,406.68710141)(299.59628648,406.64710145)(299.67629028,406.61710297)
\curveto(299.76628631,406.58710151)(299.85628622,406.56210154)(299.94629028,406.54210297)
\curveto(299.98628609,406.53210157)(300.02128605,406.52710157)(300.05129028,406.52710297)
\curveto(300.08128599,406.53710156)(300.11628596,406.53710156)(300.15629028,406.52710297)
\lineto(300.27629028,406.49710297)
\curveto(300.32628575,406.4971016)(300.3712857,406.5021016)(300.41129028,406.51210297)
\lineto(300.53129028,406.51210297)
\curveto(300.61128546,406.53210157)(300.69128538,406.54710155)(300.77129028,406.55710297)
\curveto(300.85128522,406.56710153)(300.92628515,406.58710151)(300.99629028,406.61710297)
\curveto(301.25628482,406.71710138)(301.46628461,406.85210125)(301.62629028,407.02210297)
\curveto(301.78628429,407.19210091)(301.92128415,407.4021007)(302.03129028,407.65210297)
\curveto(302.071284,407.75210035)(302.10128397,407.85210025)(302.12129028,407.95210297)
\curveto(302.14128393,408.05210005)(302.16628391,408.15709994)(302.19629028,408.26710297)
\curveto(302.20628387,408.30709979)(302.21128386,408.34209976)(302.21129028,408.37210297)
\curveto(302.21128386,408.41209969)(302.21628386,408.45209965)(302.22629028,408.49210297)
\lineto(302.22629028,408.62710297)
\curveto(302.22628385,408.67709942)(302.23128384,408.72709937)(302.24129028,408.77710297)
}
}
{
\newrgbcolor{curcolor}{0 0 0}
\pscustom[linestyle=none,fillstyle=solid,fillcolor=curcolor]
{
\newpath
\moveto(310.16621216,413.06710297)
\curveto(310.27620684,413.06709503)(310.37120675,413.05709504)(310.45121216,413.03710297)
\curveto(310.54120658,413.01709508)(310.61120651,412.97209513)(310.66121216,412.90210297)
\curveto(310.7212064,412.82209528)(310.75120637,412.68209542)(310.75121216,412.48210297)
\lineto(310.75121216,411.97210297)
\lineto(310.75121216,411.59710297)
\curveto(310.76120636,411.45709664)(310.74620637,411.34709675)(310.70621216,411.26710297)
\curveto(310.66620645,411.1970969)(310.60620651,411.15209695)(310.52621216,411.13210297)
\curveto(310.45620666,411.11209699)(310.37120675,411.102097)(310.27121216,411.10210297)
\curveto(310.18120694,411.102097)(310.08120704,411.10709699)(309.97121216,411.11710297)
\curveto(309.87120725,411.12709697)(309.77620734,411.12209698)(309.68621216,411.10210297)
\curveto(309.6162075,411.08209702)(309.54620757,411.06709703)(309.47621216,411.05710297)
\curveto(309.40620771,411.05709704)(309.34120778,411.04709705)(309.28121216,411.02710297)
\curveto(309.121208,410.97709712)(308.96120816,410.9020972)(308.80121216,410.80210297)
\curveto(308.64120848,410.71209739)(308.5162086,410.60709749)(308.42621216,410.48710297)
\curveto(308.37620874,410.40709769)(308.3212088,410.32209778)(308.26121216,410.23210297)
\curveto(308.21120891,410.15209795)(308.16120896,410.06709803)(308.11121216,409.97710297)
\curveto(308.08120904,409.8970982)(308.05120907,409.81209829)(308.02121216,409.72210297)
\lineto(307.96121216,409.48210297)
\curveto(307.94120918,409.41209869)(307.93120919,409.33709876)(307.93121216,409.25710297)
\curveto(307.93120919,409.18709891)(307.9212092,409.11709898)(307.90121216,409.04710297)
\curveto(307.89120923,409.00709909)(307.88620923,408.96709913)(307.88621216,408.92710297)
\curveto(307.89620922,408.8970992)(307.89620922,408.86709923)(307.88621216,408.83710297)
\lineto(307.88621216,408.59710297)
\curveto(307.86620925,408.52709957)(307.86120926,408.44709965)(307.87121216,408.35710297)
\curveto(307.88120924,408.27709982)(307.88620923,408.1970999)(307.88621216,408.11710297)
\lineto(307.88621216,407.15710297)
\lineto(307.88621216,405.88210297)
\curveto(307.88620923,405.75210235)(307.88120924,405.63210247)(307.87121216,405.52210297)
\curveto(307.86120926,405.41210269)(307.83120929,405.32210278)(307.78121216,405.25210297)
\curveto(307.76120936,405.22210288)(307.72620939,405.1971029)(307.67621216,405.17710297)
\curveto(307.63620948,405.16710293)(307.59120953,405.15710294)(307.54121216,405.14710297)
\lineto(307.46621216,405.14710297)
\curveto(307.4162097,405.13710296)(307.36120976,405.13210297)(307.30121216,405.13210297)
\lineto(307.13621216,405.13210297)
\lineto(306.49121216,405.13210297)
\curveto(306.43121069,405.14210296)(306.36621075,405.14710295)(306.29621216,405.14710297)
\lineto(306.10121216,405.14710297)
\curveto(306.05121107,405.16710293)(306.00121112,405.18210292)(305.95121216,405.19210297)
\curveto(305.90121122,405.21210289)(305.86621125,405.24710285)(305.84621216,405.29710297)
\curveto(305.80621131,405.34710275)(305.78121134,405.41710268)(305.77121216,405.50710297)
\lineto(305.77121216,405.80710297)
\lineto(305.77121216,406.82710297)
\lineto(305.77121216,411.05710297)
\lineto(305.77121216,412.16710297)
\lineto(305.77121216,412.45210297)
\curveto(305.77121135,412.55209555)(305.79121133,412.63209547)(305.83121216,412.69210297)
\curveto(305.88121124,412.77209533)(305.95621116,412.82209528)(306.05621216,412.84210297)
\curveto(306.15621096,412.86209524)(306.27621084,412.87209523)(306.41621216,412.87210297)
\lineto(307.18121216,412.87210297)
\curveto(307.30120982,412.87209523)(307.40620971,412.86209524)(307.49621216,412.84210297)
\curveto(307.58620953,412.83209527)(307.65620946,412.78709531)(307.70621216,412.70710297)
\curveto(307.73620938,412.65709544)(307.75120937,412.58709551)(307.75121216,412.49710297)
\lineto(307.78121216,412.22710297)
\curveto(307.79120933,412.14709595)(307.80620931,412.07209603)(307.82621216,412.00210297)
\curveto(307.85620926,411.93209617)(307.90620921,411.8970962)(307.97621216,411.89710297)
\curveto(307.99620912,411.91709618)(308.0162091,411.92709617)(308.03621216,411.92710297)
\curveto(308.05620906,411.92709617)(308.07620904,411.93709616)(308.09621216,411.95710297)
\curveto(308.15620896,412.00709609)(308.20620891,412.06209604)(308.24621216,412.12210297)
\curveto(308.29620882,412.19209591)(308.35620876,412.25209585)(308.42621216,412.30210297)
\curveto(308.46620865,412.33209577)(308.50120862,412.36209574)(308.53121216,412.39210297)
\curveto(308.56120856,412.43209567)(308.59620852,412.46709563)(308.63621216,412.49710297)
\lineto(308.90621216,412.67710297)
\curveto(309.00620811,412.73709536)(309.10620801,412.79209531)(309.20621216,412.84210297)
\curveto(309.30620781,412.88209522)(309.40620771,412.91709518)(309.50621216,412.94710297)
\lineto(309.83621216,413.03710297)
\curveto(309.86620725,413.04709505)(309.9212072,413.04709505)(310.00121216,413.03710297)
\curveto(310.09120703,413.03709506)(310.14620697,413.04709505)(310.16621216,413.06710297)
}
}
{
\newrgbcolor{curcolor}{0 0 0}
\pscustom[linestyle=none,fillstyle=solid,fillcolor=curcolor]
{
}
}
{
\newrgbcolor{curcolor}{0 0 0}
\pscustom[linestyle=none,fillstyle=solid,fillcolor=curcolor]
{
\newpath
\moveto(320.15644653,413.06710297)
\curveto(320.26644122,413.06709503)(320.36144112,413.05709504)(320.44144653,413.03710297)
\curveto(320.53144095,413.01709508)(320.60144088,412.97209513)(320.65144653,412.90210297)
\curveto(320.71144077,412.82209528)(320.74144074,412.68209542)(320.74144653,412.48210297)
\lineto(320.74144653,411.97210297)
\lineto(320.74144653,411.59710297)
\curveto(320.75144073,411.45709664)(320.73644075,411.34709675)(320.69644653,411.26710297)
\curveto(320.65644083,411.1970969)(320.59644089,411.15209695)(320.51644653,411.13210297)
\curveto(320.44644104,411.11209699)(320.36144112,411.102097)(320.26144653,411.10210297)
\curveto(320.17144131,411.102097)(320.07144141,411.10709699)(319.96144653,411.11710297)
\curveto(319.86144162,411.12709697)(319.76644172,411.12209698)(319.67644653,411.10210297)
\curveto(319.60644188,411.08209702)(319.53644195,411.06709703)(319.46644653,411.05710297)
\curveto(319.39644209,411.05709704)(319.33144215,411.04709705)(319.27144653,411.02710297)
\curveto(319.11144237,410.97709712)(318.95144253,410.9020972)(318.79144653,410.80210297)
\curveto(318.63144285,410.71209739)(318.50644298,410.60709749)(318.41644653,410.48710297)
\curveto(318.36644312,410.40709769)(318.31144317,410.32209778)(318.25144653,410.23210297)
\curveto(318.20144328,410.15209795)(318.15144333,410.06709803)(318.10144653,409.97710297)
\curveto(318.07144341,409.8970982)(318.04144344,409.81209829)(318.01144653,409.72210297)
\lineto(317.95144653,409.48210297)
\curveto(317.93144355,409.41209869)(317.92144356,409.33709876)(317.92144653,409.25710297)
\curveto(317.92144356,409.18709891)(317.91144357,409.11709898)(317.89144653,409.04710297)
\curveto(317.8814436,409.00709909)(317.87644361,408.96709913)(317.87644653,408.92710297)
\curveto(317.8864436,408.8970992)(317.8864436,408.86709923)(317.87644653,408.83710297)
\lineto(317.87644653,408.59710297)
\curveto(317.85644363,408.52709957)(317.85144363,408.44709965)(317.86144653,408.35710297)
\curveto(317.87144361,408.27709982)(317.87644361,408.1970999)(317.87644653,408.11710297)
\lineto(317.87644653,407.15710297)
\lineto(317.87644653,405.88210297)
\curveto(317.87644361,405.75210235)(317.87144361,405.63210247)(317.86144653,405.52210297)
\curveto(317.85144363,405.41210269)(317.82144366,405.32210278)(317.77144653,405.25210297)
\curveto(317.75144373,405.22210288)(317.71644377,405.1971029)(317.66644653,405.17710297)
\curveto(317.62644386,405.16710293)(317.5814439,405.15710294)(317.53144653,405.14710297)
\lineto(317.45644653,405.14710297)
\curveto(317.40644408,405.13710296)(317.35144413,405.13210297)(317.29144653,405.13210297)
\lineto(317.12644653,405.13210297)
\lineto(316.48144653,405.13210297)
\curveto(316.42144506,405.14210296)(316.35644513,405.14710295)(316.28644653,405.14710297)
\lineto(316.09144653,405.14710297)
\curveto(316.04144544,405.16710293)(315.99144549,405.18210292)(315.94144653,405.19210297)
\curveto(315.89144559,405.21210289)(315.85644563,405.24710285)(315.83644653,405.29710297)
\curveto(315.79644569,405.34710275)(315.77144571,405.41710268)(315.76144653,405.50710297)
\lineto(315.76144653,405.80710297)
\lineto(315.76144653,406.82710297)
\lineto(315.76144653,411.05710297)
\lineto(315.76144653,412.16710297)
\lineto(315.76144653,412.45210297)
\curveto(315.76144572,412.55209555)(315.7814457,412.63209547)(315.82144653,412.69210297)
\curveto(315.87144561,412.77209533)(315.94644554,412.82209528)(316.04644653,412.84210297)
\curveto(316.14644534,412.86209524)(316.26644522,412.87209523)(316.40644653,412.87210297)
\lineto(317.17144653,412.87210297)
\curveto(317.29144419,412.87209523)(317.39644409,412.86209524)(317.48644653,412.84210297)
\curveto(317.57644391,412.83209527)(317.64644384,412.78709531)(317.69644653,412.70710297)
\curveto(317.72644376,412.65709544)(317.74144374,412.58709551)(317.74144653,412.49710297)
\lineto(317.77144653,412.22710297)
\curveto(317.7814437,412.14709595)(317.79644369,412.07209603)(317.81644653,412.00210297)
\curveto(317.84644364,411.93209617)(317.89644359,411.8970962)(317.96644653,411.89710297)
\curveto(317.9864435,411.91709618)(318.00644348,411.92709617)(318.02644653,411.92710297)
\curveto(318.04644344,411.92709617)(318.06644342,411.93709616)(318.08644653,411.95710297)
\curveto(318.14644334,412.00709609)(318.19644329,412.06209604)(318.23644653,412.12210297)
\curveto(318.2864432,412.19209591)(318.34644314,412.25209585)(318.41644653,412.30210297)
\curveto(318.45644303,412.33209577)(318.49144299,412.36209574)(318.52144653,412.39210297)
\curveto(318.55144293,412.43209567)(318.5864429,412.46709563)(318.62644653,412.49710297)
\lineto(318.89644653,412.67710297)
\curveto(318.99644249,412.73709536)(319.09644239,412.79209531)(319.19644653,412.84210297)
\curveto(319.29644219,412.88209522)(319.39644209,412.91709518)(319.49644653,412.94710297)
\lineto(319.82644653,413.03710297)
\curveto(319.85644163,413.04709505)(319.91144157,413.04709505)(319.99144653,413.03710297)
\curveto(320.0814414,413.03709506)(320.13644135,413.04709505)(320.15644653,413.06710297)
}
}
{
\newrgbcolor{curcolor}{0 0 0}
\pscustom[linestyle=none,fillstyle=solid,fillcolor=curcolor]
{
\newpath
\moveto(329.06785278,409.31710297)
\curveto(329.08784421,409.25709884)(329.0978442,409.17209893)(329.09785278,409.06210297)
\curveto(329.0978442,408.95209915)(329.08784421,408.86709923)(329.06785278,408.80710297)
\lineto(329.06785278,408.65710297)
\curveto(329.04784425,408.57709952)(329.03784426,408.4970996)(329.03785278,408.41710297)
\curveto(329.04784425,408.33709976)(329.04284426,408.25709984)(329.02285278,408.17710297)
\curveto(329.0028443,408.10709999)(328.98784431,408.04210006)(328.97785278,407.98210297)
\curveto(328.96784433,407.92210018)(328.95784434,407.85710024)(328.94785278,407.78710297)
\curveto(328.90784439,407.67710042)(328.87284443,407.56210054)(328.84285278,407.44210297)
\curveto(328.81284449,407.33210077)(328.77284453,407.22710087)(328.72285278,407.12710297)
\curveto(328.51284479,406.64710145)(328.23784506,406.25710184)(327.89785278,405.95710297)
\curveto(327.55784574,405.65710244)(327.14784615,405.40710269)(326.66785278,405.20710297)
\curveto(326.54784675,405.15710294)(326.42284688,405.12210298)(326.29285278,405.10210297)
\curveto(326.17284713,405.07210303)(326.04784725,405.04210306)(325.91785278,405.01210297)
\curveto(325.86784743,404.99210311)(325.81284749,404.98210312)(325.75285278,404.98210297)
\curveto(325.69284761,404.98210312)(325.63784766,404.97710312)(325.58785278,404.96710297)
\lineto(325.48285278,404.96710297)
\curveto(325.45284785,404.95710314)(325.42284788,404.95210315)(325.39285278,404.95210297)
\curveto(325.34284796,404.94210316)(325.26284804,404.93710316)(325.15285278,404.93710297)
\curveto(325.04284826,404.92710317)(324.95784834,404.93210317)(324.89785278,404.95210297)
\lineto(324.74785278,404.95210297)
\curveto(324.6978486,404.96210314)(324.64284866,404.96710313)(324.58285278,404.96710297)
\curveto(324.53284877,404.95710314)(324.48284882,404.96210314)(324.43285278,404.98210297)
\curveto(324.39284891,404.99210311)(324.35284895,404.9971031)(324.31285278,404.99710297)
\curveto(324.28284902,404.9971031)(324.24284906,405.0021031)(324.19285278,405.01210297)
\curveto(324.09284921,405.04210306)(323.99284931,405.06710303)(323.89285278,405.08710297)
\curveto(323.79284951,405.10710299)(323.6978496,405.13710296)(323.60785278,405.17710297)
\curveto(323.48784981,405.21710288)(323.37284993,405.25710284)(323.26285278,405.29710297)
\curveto(323.16285014,405.33710276)(323.05785024,405.38710271)(322.94785278,405.44710297)
\curveto(322.5978507,405.65710244)(322.297851,405.9021022)(322.04785278,406.18210297)
\curveto(321.7978515,406.46210164)(321.58785171,406.7971013)(321.41785278,407.18710297)
\curveto(321.36785193,407.27710082)(321.32785197,407.37210073)(321.29785278,407.47210297)
\curveto(321.27785202,407.57210053)(321.25285205,407.67710042)(321.22285278,407.78710297)
\curveto(321.2028521,407.83710026)(321.19285211,407.88210022)(321.19285278,407.92210297)
\curveto(321.19285211,407.96210014)(321.18285212,408.00710009)(321.16285278,408.05710297)
\curveto(321.14285216,408.13709996)(321.13285217,408.21709988)(321.13285278,408.29710297)
\curveto(321.13285217,408.38709971)(321.12285218,408.47209963)(321.10285278,408.55210297)
\curveto(321.09285221,408.6020995)(321.08785221,408.64709945)(321.08785278,408.68710297)
\lineto(321.08785278,408.82210297)
\curveto(321.06785223,408.88209922)(321.05785224,408.96709913)(321.05785278,409.07710297)
\curveto(321.06785223,409.18709891)(321.08285222,409.27209883)(321.10285278,409.33210297)
\lineto(321.10285278,409.43710297)
\curveto(321.11285219,409.48709861)(321.11285219,409.53709856)(321.10285278,409.58710297)
\curveto(321.1028522,409.64709845)(321.11285219,409.7020984)(321.13285278,409.75210297)
\curveto(321.14285216,409.8020983)(321.14785215,409.84709825)(321.14785278,409.88710297)
\curveto(321.14785215,409.93709816)(321.15785214,409.98709811)(321.17785278,410.03710297)
\curveto(321.21785208,410.16709793)(321.25285205,410.29209781)(321.28285278,410.41210297)
\curveto(321.31285199,410.54209756)(321.35285195,410.66709743)(321.40285278,410.78710297)
\curveto(321.58285172,411.1970969)(321.7978515,411.53709656)(322.04785278,411.80710297)
\curveto(322.297851,412.08709601)(322.6028507,412.34209576)(322.96285278,412.57210297)
\curveto(323.06285024,412.62209548)(323.16785013,412.66709543)(323.27785278,412.70710297)
\curveto(323.38784991,412.74709535)(323.4978498,412.79209531)(323.60785278,412.84210297)
\curveto(323.73784956,412.89209521)(323.87284943,412.92709517)(324.01285278,412.94710297)
\curveto(324.15284915,412.96709513)(324.297849,412.9970951)(324.44785278,413.03710297)
\curveto(324.52784877,413.04709505)(324.6028487,413.05209505)(324.67285278,413.05210297)
\curveto(324.74284856,413.05209505)(324.81284849,413.05709504)(324.88285278,413.06710297)
\curveto(325.46284784,413.07709502)(325.96284734,413.01709508)(326.38285278,412.88710297)
\curveto(326.81284649,412.75709534)(327.19284611,412.57709552)(327.52285278,412.34710297)
\curveto(327.63284567,412.26709583)(327.74284556,412.17709592)(327.85285278,412.07710297)
\curveto(327.97284533,411.98709611)(328.07284523,411.88709621)(328.15285278,411.77710297)
\curveto(328.23284507,411.67709642)(328.302845,411.57709652)(328.36285278,411.47710297)
\curveto(328.43284487,411.37709672)(328.5028448,411.27209683)(328.57285278,411.16210297)
\curveto(328.64284466,411.05209705)(328.6978446,410.93209717)(328.73785278,410.80210297)
\curveto(328.77784452,410.68209742)(328.82284448,410.55209755)(328.87285278,410.41210297)
\curveto(328.9028444,410.33209777)(328.92784437,410.24709785)(328.94785278,410.15710297)
\lineto(329.00785278,409.88710297)
\curveto(329.01784428,409.84709825)(329.02284428,409.80709829)(329.02285278,409.76710297)
\curveto(329.02284428,409.72709837)(329.02784427,409.68709841)(329.03785278,409.64710297)
\curveto(329.05784424,409.5970985)(329.06284424,409.54209856)(329.05285278,409.48210297)
\curveto(329.04284426,409.42209868)(329.04784425,409.36709873)(329.06785278,409.31710297)
\moveto(326.96785278,408.77710297)
\curveto(326.97784632,408.82709927)(326.98284632,408.8970992)(326.98285278,408.98710297)
\curveto(326.98284632,409.08709901)(326.97784632,409.16209894)(326.96785278,409.21210297)
\lineto(326.96785278,409.33210297)
\curveto(326.94784635,409.38209872)(326.93784636,409.43709866)(326.93785278,409.49710297)
\curveto(326.93784636,409.55709854)(326.93284637,409.61209849)(326.92285278,409.66210297)
\curveto(326.92284638,409.7020984)(326.91784638,409.73209837)(326.90785278,409.75210297)
\lineto(326.84785278,409.99210297)
\curveto(326.83784646,410.08209802)(326.81784648,410.16709793)(326.78785278,410.24710297)
\curveto(326.67784662,410.50709759)(326.54784675,410.72709737)(326.39785278,410.90710297)
\curveto(326.24784705,411.097097)(326.04784725,411.24709685)(325.79785278,411.35710297)
\curveto(325.73784756,411.37709672)(325.67784762,411.39209671)(325.61785278,411.40210297)
\curveto(325.55784774,411.42209668)(325.49284781,411.44209666)(325.42285278,411.46210297)
\curveto(325.34284796,411.48209662)(325.25784804,411.48709661)(325.16785278,411.47710297)
\lineto(324.89785278,411.47710297)
\curveto(324.86784843,411.45709664)(324.83284847,411.44709665)(324.79285278,411.44710297)
\curveto(324.75284855,411.45709664)(324.71784858,411.45709664)(324.68785278,411.44710297)
\lineto(324.47785278,411.38710297)
\curveto(324.41784888,411.37709672)(324.36284894,411.35709674)(324.31285278,411.32710297)
\curveto(324.06284924,411.21709688)(323.85784944,411.05709704)(323.69785278,410.84710297)
\curveto(323.54784975,410.64709745)(323.42784987,410.41209769)(323.33785278,410.14210297)
\curveto(323.30784999,410.04209806)(323.28285002,409.93709816)(323.26285278,409.82710297)
\curveto(323.25285005,409.71709838)(323.23785006,409.60709849)(323.21785278,409.49710297)
\curveto(323.20785009,409.44709865)(323.2028501,409.3970987)(323.20285278,409.34710297)
\lineto(323.20285278,409.19710297)
\curveto(323.18285012,409.12709897)(323.17285013,409.02209908)(323.17285278,408.88210297)
\curveto(323.18285012,408.74209936)(323.1978501,408.63709946)(323.21785278,408.56710297)
\lineto(323.21785278,408.43210297)
\curveto(323.23785006,408.35209975)(323.25285005,408.27209983)(323.26285278,408.19210297)
\curveto(323.27285003,408.12209998)(323.28785001,408.04710005)(323.30785278,407.96710297)
\curveto(323.40784989,407.66710043)(323.51284979,407.42210068)(323.62285278,407.23210297)
\curveto(323.74284956,407.05210105)(323.92784937,406.88710121)(324.17785278,406.73710297)
\curveto(324.24784905,406.68710141)(324.32284898,406.64710145)(324.40285278,406.61710297)
\curveto(324.49284881,406.58710151)(324.58284872,406.56210154)(324.67285278,406.54210297)
\curveto(324.71284859,406.53210157)(324.74784855,406.52710157)(324.77785278,406.52710297)
\curveto(324.80784849,406.53710156)(324.84284846,406.53710156)(324.88285278,406.52710297)
\lineto(325.00285278,406.49710297)
\curveto(325.05284825,406.4971016)(325.0978482,406.5021016)(325.13785278,406.51210297)
\lineto(325.25785278,406.51210297)
\curveto(325.33784796,406.53210157)(325.41784788,406.54710155)(325.49785278,406.55710297)
\curveto(325.57784772,406.56710153)(325.65284765,406.58710151)(325.72285278,406.61710297)
\curveto(325.98284732,406.71710138)(326.19284711,406.85210125)(326.35285278,407.02210297)
\curveto(326.51284679,407.19210091)(326.64784665,407.4021007)(326.75785278,407.65210297)
\curveto(326.7978465,407.75210035)(326.82784647,407.85210025)(326.84785278,407.95210297)
\curveto(326.86784643,408.05210005)(326.89284641,408.15709994)(326.92285278,408.26710297)
\curveto(326.93284637,408.30709979)(326.93784636,408.34209976)(326.93785278,408.37210297)
\curveto(326.93784636,408.41209969)(326.94284636,408.45209965)(326.95285278,408.49210297)
\lineto(326.95285278,408.62710297)
\curveto(326.95284635,408.67709942)(326.95784634,408.72709937)(326.96785278,408.77710297)
}
}
{
\newrgbcolor{curcolor}{0 0 0}
\pscustom[linestyle=none,fillstyle=solid,fillcolor=curcolor]
{
\newpath
\moveto(330.97777466,415.82710297)
\lineto(332.07277466,415.82710297)
\curveto(332.17277217,415.82709227)(332.26777208,415.82209228)(332.35777466,415.81210297)
\curveto(332.4477719,415.8020923)(332.51777183,415.77209233)(332.56777466,415.72210297)
\curveto(332.62777172,415.65209245)(332.65777169,415.55709254)(332.65777466,415.43710297)
\curveto(332.66777168,415.32709277)(332.67277167,415.21209289)(332.67277466,415.09210297)
\lineto(332.67277466,413.75710297)
\lineto(332.67277466,408.37210297)
\lineto(332.67277466,406.07710297)
\lineto(332.67277466,405.65710297)
\curveto(332.68277166,405.50710259)(332.66277168,405.39210271)(332.61277466,405.31210297)
\curveto(332.56277178,405.23210287)(332.47277187,405.17710292)(332.34277466,405.14710297)
\curveto(332.28277206,405.12710297)(332.21277213,405.12210298)(332.13277466,405.13210297)
\curveto(332.06277228,405.14210296)(331.99277235,405.14710295)(331.92277466,405.14710297)
\lineto(331.20277466,405.14710297)
\curveto(331.09277325,405.14710295)(330.99277335,405.15210295)(330.90277466,405.16210297)
\curveto(330.81277353,405.17210293)(330.73777361,405.2021029)(330.67777466,405.25210297)
\curveto(330.61777373,405.3021028)(330.58277376,405.37710272)(330.57277466,405.47710297)
\lineto(330.57277466,405.80710297)
\lineto(330.57277466,407.14210297)
\lineto(330.57277466,412.76710297)
\lineto(330.57277466,414.80710297)
\curveto(330.57277377,414.93709316)(330.56777378,415.09209301)(330.55777466,415.27210297)
\curveto(330.55777379,415.45209265)(330.58277376,415.58209252)(330.63277466,415.66210297)
\curveto(330.65277369,415.7020924)(330.67777367,415.73209237)(330.70777466,415.75210297)
\lineto(330.82777466,415.81210297)
\curveto(330.8477735,415.81209229)(330.87277347,415.81209229)(330.90277466,415.81210297)
\curveto(330.93277341,415.82209228)(330.95777339,415.82709227)(330.97777466,415.82710297)
}
}
{
\newrgbcolor{curcolor}{0 0 0}
\pscustom[linestyle=none,fillstyle=solid,fillcolor=curcolor]
{
}
}
{
\newrgbcolor{curcolor}{0 0 0}
\pscustom[linestyle=none,fillstyle=solid,fillcolor=curcolor]
{
\newpath
\moveto(346.08511841,405.98710297)
\lineto(346.08511841,405.56710297)
\curveto(346.08511004,405.43710266)(346.05511007,405.33210277)(345.99511841,405.25210297)
\curveto(345.94511018,405.2021029)(345.88011024,405.16710293)(345.80011841,405.14710297)
\curveto(345.7201104,405.13710296)(345.63011049,405.13210297)(345.53011841,405.13210297)
\lineto(344.70511841,405.13210297)
\lineto(344.42011841,405.13210297)
\curveto(344.34011178,405.14210296)(344.27511185,405.16710293)(344.22511841,405.20710297)
\curveto(344.15511197,405.25710284)(344.11511201,405.32210278)(344.10511841,405.40210297)
\curveto(344.09511203,405.48210262)(344.07511205,405.56210254)(344.04511841,405.64210297)
\curveto(344.0251121,405.66210244)(344.00511212,405.67710242)(343.98511841,405.68710297)
\curveto(343.97511215,405.70710239)(343.96011216,405.72710237)(343.94011841,405.74710297)
\curveto(343.83011229,405.74710235)(343.75011237,405.72210238)(343.70011841,405.67210297)
\lineto(343.55011841,405.52210297)
\curveto(343.48011264,405.47210263)(343.41511271,405.42710267)(343.35511841,405.38710297)
\curveto(343.29511283,405.35710274)(343.23011289,405.31710278)(343.16011841,405.26710297)
\curveto(343.120113,405.24710285)(343.07511305,405.22710287)(343.02511841,405.20710297)
\curveto(342.98511314,405.18710291)(342.94011318,405.16710293)(342.89011841,405.14710297)
\curveto(342.75011337,405.097103)(342.60011352,405.05210305)(342.44011841,405.01210297)
\curveto(342.39011373,404.99210311)(342.34511378,404.98210312)(342.30511841,404.98210297)
\curveto(342.26511386,404.98210312)(342.2251139,404.97710312)(342.18511841,404.96710297)
\lineto(342.05011841,404.96710297)
\curveto(342.0201141,404.95710314)(341.98011414,404.95210315)(341.93011841,404.95210297)
\lineto(341.79511841,404.95210297)
\curveto(341.73511439,404.93210317)(341.64511448,404.92710317)(341.52511841,404.93710297)
\curveto(341.40511472,404.93710316)(341.3201148,404.94710315)(341.27011841,404.96710297)
\curveto(341.20011492,404.98710311)(341.13511499,404.9971031)(341.07511841,404.99710297)
\curveto(341.0251151,404.98710311)(340.97011515,404.99210311)(340.91011841,405.01210297)
\lineto(340.55011841,405.13210297)
\curveto(340.44011568,405.16210294)(340.33011579,405.2021029)(340.22011841,405.25210297)
\curveto(339.87011625,405.4021027)(339.55511657,405.63210247)(339.27511841,405.94210297)
\curveto(339.00511712,406.26210184)(338.79011733,406.5971015)(338.63011841,406.94710297)
\curveto(338.58011754,407.05710104)(338.54011758,407.16210094)(338.51011841,407.26210297)
\curveto(338.48011764,407.37210073)(338.44511768,407.48210062)(338.40511841,407.59210297)
\curveto(338.39511773,407.63210047)(338.39011773,407.66710043)(338.39011841,407.69710297)
\curveto(338.39011773,407.73710036)(338.38011774,407.78210032)(338.36011841,407.83210297)
\curveto(338.34011778,407.91210019)(338.3201178,407.9971001)(338.30011841,408.08710297)
\curveto(338.29011783,408.18709991)(338.27511785,408.28709981)(338.25511841,408.38710297)
\curveto(338.24511788,408.41709968)(338.24011788,408.45209965)(338.24011841,408.49210297)
\curveto(338.25011787,408.53209957)(338.25011787,408.56709953)(338.24011841,408.59710297)
\lineto(338.24011841,408.73210297)
\curveto(338.24011788,408.78209932)(338.23511789,408.83209927)(338.22511841,408.88210297)
\curveto(338.21511791,408.93209917)(338.21011791,408.98709911)(338.21011841,409.04710297)
\curveto(338.21011791,409.11709898)(338.21511791,409.17209893)(338.22511841,409.21210297)
\curveto(338.23511789,409.26209884)(338.24011788,409.30709879)(338.24011841,409.34710297)
\lineto(338.24011841,409.49710297)
\curveto(338.25011787,409.54709855)(338.25011787,409.59209851)(338.24011841,409.63210297)
\curveto(338.24011788,409.68209842)(338.25011787,409.73209837)(338.27011841,409.78210297)
\curveto(338.29011783,409.89209821)(338.30511782,409.9970981)(338.31511841,410.09710297)
\curveto(338.33511779,410.1970979)(338.36011776,410.2970978)(338.39011841,410.39710297)
\curveto(338.43011769,410.51709758)(338.46511766,410.63209747)(338.49511841,410.74210297)
\curveto(338.5251176,410.85209725)(338.56511756,410.96209714)(338.61511841,411.07210297)
\curveto(338.75511737,411.37209673)(338.93011719,411.65709644)(339.14011841,411.92710297)
\curveto(339.16011696,411.95709614)(339.18511694,411.98209612)(339.21511841,412.00210297)
\curveto(339.25511687,412.03209607)(339.28511684,412.06209604)(339.30511841,412.09210297)
\curveto(339.34511678,412.14209596)(339.38511674,412.18709591)(339.42511841,412.22710297)
\curveto(339.46511666,412.26709583)(339.51011661,412.30709579)(339.56011841,412.34710297)
\curveto(339.60011652,412.36709573)(339.63511649,412.39209571)(339.66511841,412.42210297)
\curveto(339.69511643,412.46209564)(339.73011639,412.49209561)(339.77011841,412.51210297)
\curveto(340.0201161,412.68209542)(340.31011581,412.82209528)(340.64011841,412.93210297)
\curveto(340.71011541,412.95209515)(340.78011534,412.96709513)(340.85011841,412.97710297)
\curveto(340.93011519,412.98709511)(341.01011511,413.0020951)(341.09011841,413.02210297)
\curveto(341.16011496,413.04209506)(341.25011487,413.05209505)(341.36011841,413.05210297)
\curveto(341.47011465,413.06209504)(341.58011454,413.06709503)(341.69011841,413.06710297)
\curveto(341.80011432,413.06709503)(341.90511422,413.06209504)(342.00511841,413.05210297)
\curveto(342.11511401,413.04209506)(342.20511392,413.02709507)(342.27511841,413.00710297)
\curveto(342.4251137,412.95709514)(342.57011355,412.91209519)(342.71011841,412.87210297)
\curveto(342.85011327,412.83209527)(342.98011314,412.77709532)(343.10011841,412.70710297)
\curveto(343.17011295,412.65709544)(343.23511289,412.60709549)(343.29511841,412.55710297)
\curveto(343.35511277,412.51709558)(343.4201127,412.47209563)(343.49011841,412.42210297)
\curveto(343.53011259,412.39209571)(343.58511254,412.35209575)(343.65511841,412.30210297)
\curveto(343.73511239,412.25209585)(343.81011231,412.25209585)(343.88011841,412.30210297)
\curveto(343.9201122,412.32209578)(343.94011218,412.35709574)(343.94011841,412.40710297)
\curveto(343.94011218,412.45709564)(343.95011217,412.50709559)(343.97011841,412.55710297)
\lineto(343.97011841,412.70710297)
\curveto(343.98011214,412.73709536)(343.98511214,412.77209533)(343.98511841,412.81210297)
\lineto(343.98511841,412.93210297)
\lineto(343.98511841,414.97210297)
\curveto(343.98511214,415.08209302)(343.98011214,415.2020929)(343.97011841,415.33210297)
\curveto(343.97011215,415.47209263)(343.99511213,415.57709252)(344.04511841,415.64710297)
\curveto(344.08511204,415.72709237)(344.16011196,415.77709232)(344.27011841,415.79710297)
\curveto(344.29011183,415.80709229)(344.31011181,415.80709229)(344.33011841,415.79710297)
\curveto(344.35011177,415.7970923)(344.37011175,415.8020923)(344.39011841,415.81210297)
\lineto(345.45511841,415.81210297)
\curveto(345.57511055,415.81209229)(345.68511044,415.80709229)(345.78511841,415.79710297)
\curveto(345.88511024,415.78709231)(345.96011016,415.74709235)(346.01011841,415.67710297)
\curveto(346.06011006,415.5970925)(346.08511004,415.49209261)(346.08511841,415.36210297)
\lineto(346.08511841,415.00210297)
\lineto(346.08511841,405.98710297)
\moveto(344.04511841,408.92710297)
\curveto(344.05511207,408.96709913)(344.05511207,409.00709909)(344.04511841,409.04710297)
\lineto(344.04511841,409.18210297)
\curveto(344.04511208,409.28209882)(344.04011208,409.38209872)(344.03011841,409.48210297)
\curveto(344.0201121,409.58209852)(344.00511212,409.67209843)(343.98511841,409.75210297)
\curveto(343.96511216,409.86209824)(343.94511218,409.96209814)(343.92511841,410.05210297)
\curveto(343.91511221,410.14209796)(343.89011223,410.22709787)(343.85011841,410.30710297)
\curveto(343.71011241,410.66709743)(343.50511262,410.95209715)(343.23511841,411.16210297)
\curveto(342.97511315,411.37209673)(342.59511353,411.47709662)(342.09511841,411.47710297)
\curveto(342.03511409,411.47709662)(341.95511417,411.46709663)(341.85511841,411.44710297)
\curveto(341.77511435,411.42709667)(341.70011442,411.40709669)(341.63011841,411.38710297)
\curveto(341.57011455,411.37709672)(341.51011461,411.35709674)(341.45011841,411.32710297)
\curveto(341.18011494,411.21709688)(340.97011515,411.04709705)(340.82011841,410.81710297)
\curveto(340.67011545,410.58709751)(340.55011557,410.32709777)(340.46011841,410.03710297)
\curveto(340.43011569,409.93709816)(340.41011571,409.83709826)(340.40011841,409.73710297)
\curveto(340.39011573,409.63709846)(340.37011575,409.53209857)(340.34011841,409.42210297)
\lineto(340.34011841,409.21210297)
\curveto(340.3201158,409.12209898)(340.31511581,408.9970991)(340.32511841,408.83710297)
\curveto(340.33511579,408.68709941)(340.35011577,408.57709952)(340.37011841,408.50710297)
\lineto(340.37011841,408.41710297)
\curveto(340.38011574,408.3970997)(340.38511574,408.37709972)(340.38511841,408.35710297)
\curveto(340.40511572,408.27709982)(340.4201157,408.2020999)(340.43011841,408.13210297)
\curveto(340.45011567,408.06210004)(340.47011565,407.98710011)(340.49011841,407.90710297)
\curveto(340.66011546,407.38710071)(340.95011517,407.0021011)(341.36011841,406.75210297)
\curveto(341.49011463,406.66210144)(341.67011445,406.59210151)(341.90011841,406.54210297)
\curveto(341.94011418,406.53210157)(342.00011412,406.52710157)(342.08011841,406.52710297)
\curveto(342.11011401,406.51710158)(342.15511397,406.50710159)(342.21511841,406.49710297)
\curveto(342.28511384,406.4971016)(342.34011378,406.5021016)(342.38011841,406.51210297)
\curveto(342.46011366,406.53210157)(342.54011358,406.54710155)(342.62011841,406.55710297)
\curveto(342.70011342,406.56710153)(342.78011334,406.58710151)(342.86011841,406.61710297)
\curveto(343.11011301,406.72710137)(343.31011281,406.86710123)(343.46011841,407.03710297)
\curveto(343.61011251,407.20710089)(343.74011238,407.42210068)(343.85011841,407.68210297)
\curveto(343.89011223,407.77210033)(343.9201122,407.86210024)(343.94011841,407.95210297)
\curveto(343.96011216,408.05210005)(343.98011214,408.15709994)(344.00011841,408.26710297)
\curveto(344.01011211,408.31709978)(344.01011211,408.36209974)(344.00011841,408.40210297)
\curveto(344.00011212,408.45209965)(344.01011211,408.5020996)(344.03011841,408.55210297)
\curveto(344.04011208,408.58209952)(344.04511208,408.61709948)(344.04511841,408.65710297)
\lineto(344.04511841,408.79210297)
\lineto(344.04511841,408.92710297)
}
}
{
\newrgbcolor{curcolor}{0 0 0}
\pscustom[linestyle=none,fillstyle=solid,fillcolor=curcolor]
{
\newpath
\moveto(355.03004028,409.07710297)
\curveto(355.05003212,408.9970991)(355.05003212,408.90709919)(355.03004028,408.80710297)
\curveto(355.01003216,408.70709939)(354.97503219,408.64209946)(354.92504028,408.61210297)
\curveto(354.87503229,408.57209953)(354.80003237,408.54209956)(354.70004028,408.52210297)
\curveto(354.61003256,408.51209959)(354.50503266,408.5020996)(354.38504028,408.49210297)
\lineto(354.04004028,408.49210297)
\curveto(353.93003324,408.5020996)(353.83003334,408.50709959)(353.74004028,408.50710297)
\lineto(350.08004028,408.50710297)
\lineto(349.87004028,408.50710297)
\curveto(349.81003736,408.50709959)(349.75503741,408.4970996)(349.70504028,408.47710297)
\curveto(349.62503754,408.43709966)(349.57503759,408.3970997)(349.55504028,408.35710297)
\curveto(349.53503763,408.33709976)(349.51503765,408.2970998)(349.49504028,408.23710297)
\curveto(349.47503769,408.18709991)(349.4700377,408.13709996)(349.48004028,408.08710297)
\curveto(349.50003767,408.02710007)(349.51003766,407.96710013)(349.51004028,407.90710297)
\curveto(349.52003765,407.85710024)(349.53503763,407.8021003)(349.55504028,407.74210297)
\curveto(349.63503753,407.5021006)(349.73003744,407.3021008)(349.84004028,407.14210297)
\curveto(349.96003721,406.99210111)(350.12003705,406.85710124)(350.32004028,406.73710297)
\curveto(350.40003677,406.68710141)(350.48003669,406.65210145)(350.56004028,406.63210297)
\curveto(350.65003652,406.62210148)(350.74003643,406.6021015)(350.83004028,406.57210297)
\curveto(350.91003626,406.55210155)(351.02003615,406.53710156)(351.16004028,406.52710297)
\curveto(351.30003587,406.51710158)(351.42003575,406.52210158)(351.52004028,406.54210297)
\lineto(351.65504028,406.54210297)
\curveto(351.75503541,406.56210154)(351.84503532,406.58210152)(351.92504028,406.60210297)
\curveto(352.01503515,406.63210147)(352.10003507,406.66210144)(352.18004028,406.69210297)
\curveto(352.28003489,406.74210136)(352.39003478,406.80710129)(352.51004028,406.88710297)
\curveto(352.64003453,406.96710113)(352.73503443,407.04710105)(352.79504028,407.12710297)
\curveto(352.84503432,407.1971009)(352.89503427,407.26210084)(352.94504028,407.32210297)
\curveto(353.00503416,407.39210071)(353.07503409,407.44210066)(353.15504028,407.47210297)
\curveto(353.25503391,407.52210058)(353.38003379,407.54210056)(353.53004028,407.53210297)
\lineto(353.96504028,407.53210297)
\lineto(354.14504028,407.53210297)
\curveto(354.21503295,407.54210056)(354.27503289,407.53710056)(354.32504028,407.51710297)
\lineto(354.47504028,407.51710297)
\curveto(354.57503259,407.4971006)(354.64503252,407.47210063)(354.68504028,407.44210297)
\curveto(354.72503244,407.42210068)(354.74503242,407.37710072)(354.74504028,407.30710297)
\curveto(354.75503241,407.23710086)(354.75003242,407.17710092)(354.73004028,407.12710297)
\curveto(354.68003249,406.98710111)(354.62503254,406.86210124)(354.56504028,406.75210297)
\curveto(354.50503266,406.64210146)(354.43503273,406.53210157)(354.35504028,406.42210297)
\curveto(354.13503303,406.09210201)(353.88503328,405.82710227)(353.60504028,405.62710297)
\curveto(353.32503384,405.42710267)(352.97503419,405.25710284)(352.55504028,405.11710297)
\curveto(352.44503472,405.07710302)(352.33503483,405.05210305)(352.22504028,405.04210297)
\curveto(352.11503505,405.03210307)(352.00003517,405.01210309)(351.88004028,404.98210297)
\curveto(351.84003533,404.97210313)(351.79503537,404.97210313)(351.74504028,404.98210297)
\curveto(351.70503546,404.98210312)(351.6650355,404.97710312)(351.62504028,404.96710297)
\lineto(351.46004028,404.96710297)
\curveto(351.41003576,404.94710315)(351.35003582,404.94210316)(351.28004028,404.95210297)
\curveto(351.22003595,404.95210315)(351.165036,404.95710314)(351.11504028,404.96710297)
\curveto(351.03503613,404.97710312)(350.9650362,404.97710312)(350.90504028,404.96710297)
\curveto(350.84503632,404.95710314)(350.78003639,404.96210314)(350.71004028,404.98210297)
\curveto(350.66003651,405.0021031)(350.60503656,405.01210309)(350.54504028,405.01210297)
\curveto(350.48503668,405.01210309)(350.43003674,405.02210308)(350.38004028,405.04210297)
\curveto(350.2700369,405.06210304)(350.16003701,405.08710301)(350.05004028,405.11710297)
\curveto(349.94003723,405.13710296)(349.84003733,405.17210293)(349.75004028,405.22210297)
\curveto(349.64003753,405.26210284)(349.53503763,405.2971028)(349.43504028,405.32710297)
\curveto(349.34503782,405.36710273)(349.26003791,405.41210269)(349.18004028,405.46210297)
\curveto(348.86003831,405.66210244)(348.57503859,405.89210221)(348.32504028,406.15210297)
\curveto(348.07503909,406.42210168)(347.8700393,406.73210137)(347.71004028,407.08210297)
\curveto(347.66003951,407.19210091)(347.62003955,407.3021008)(347.59004028,407.41210297)
\curveto(347.56003961,407.53210057)(347.52003965,407.65210045)(347.47004028,407.77210297)
\curveto(347.46003971,407.81210029)(347.45503971,407.84710025)(347.45504028,407.87710297)
\curveto(347.45503971,407.91710018)(347.45003972,407.95710014)(347.44004028,407.99710297)
\curveto(347.40003977,408.11709998)(347.37503979,408.24709985)(347.36504028,408.38710297)
\lineto(347.33504028,408.80710297)
\curveto(347.33503983,408.85709924)(347.33003984,408.91209919)(347.32004028,408.97210297)
\curveto(347.32003985,409.03209907)(347.32503984,409.08709901)(347.33504028,409.13710297)
\lineto(347.33504028,409.31710297)
\lineto(347.38004028,409.67710297)
\curveto(347.42003975,409.84709825)(347.45503971,410.01209809)(347.48504028,410.17210297)
\curveto(347.51503965,410.33209777)(347.56003961,410.48209762)(347.62004028,410.62210297)
\curveto(348.05003912,411.66209644)(348.78003839,412.3970957)(349.81004028,412.82710297)
\curveto(349.95003722,412.88709521)(350.09003708,412.92709517)(350.23004028,412.94710297)
\curveto(350.38003679,412.97709512)(350.53503663,413.01209509)(350.69504028,413.05210297)
\curveto(350.77503639,413.06209504)(350.85003632,413.06709503)(350.92004028,413.06710297)
\curveto(350.99003618,413.06709503)(351.0650361,413.07209503)(351.14504028,413.08210297)
\curveto(351.65503551,413.09209501)(352.09003508,413.03209507)(352.45004028,412.90210297)
\curveto(352.82003435,412.78209532)(353.15003402,412.62209548)(353.44004028,412.42210297)
\curveto(353.53003364,412.36209574)(353.62003355,412.29209581)(353.71004028,412.21210297)
\curveto(353.80003337,412.14209596)(353.88003329,412.06709603)(353.95004028,411.98710297)
\curveto(353.98003319,411.93709616)(354.02003315,411.8970962)(354.07004028,411.86710297)
\curveto(354.15003302,411.75709634)(354.22503294,411.64209646)(354.29504028,411.52210297)
\curveto(354.3650328,411.41209669)(354.44003273,411.2970968)(354.52004028,411.17710297)
\curveto(354.5700326,411.08709701)(354.61003256,410.99209711)(354.64004028,410.89210297)
\curveto(354.68003249,410.8020973)(354.72003245,410.7020974)(354.76004028,410.59210297)
\curveto(354.81003236,410.46209764)(354.85003232,410.32709777)(354.88004028,410.18710297)
\curveto(354.91003226,410.04709805)(354.94503222,409.90709819)(354.98504028,409.76710297)
\curveto(355.00503216,409.68709841)(355.01003216,409.5970985)(355.00004028,409.49710297)
\curveto(355.00003217,409.40709869)(355.01003216,409.32209878)(355.03004028,409.24210297)
\lineto(355.03004028,409.07710297)
\moveto(352.78004028,409.96210297)
\curveto(352.85003432,410.06209804)(352.85503431,410.18209792)(352.79504028,410.32210297)
\curveto(352.74503442,410.47209763)(352.70503446,410.58209752)(352.67504028,410.65210297)
\curveto(352.53503463,410.92209718)(352.35003482,411.12709697)(352.12004028,411.26710297)
\curveto(351.89003528,411.41709668)(351.5700356,411.4970966)(351.16004028,411.50710297)
\curveto(351.13003604,411.48709661)(351.09503607,411.48209662)(351.05504028,411.49210297)
\curveto(351.01503615,411.5020966)(350.98003619,411.5020966)(350.95004028,411.49210297)
\curveto(350.90003627,411.47209663)(350.84503632,411.45709664)(350.78504028,411.44710297)
\curveto(350.72503644,411.44709665)(350.6700365,411.43709666)(350.62004028,411.41710297)
\curveto(350.18003699,411.27709682)(349.85503731,411.0020971)(349.64504028,410.59210297)
\curveto(349.62503754,410.55209755)(349.60003757,410.4970976)(349.57004028,410.42710297)
\curveto(349.55003762,410.36709773)(349.53503763,410.3020978)(349.52504028,410.23210297)
\curveto(349.51503765,410.17209793)(349.51503765,410.11209799)(349.52504028,410.05210297)
\curveto(349.54503762,409.99209811)(349.58003759,409.94209816)(349.63004028,409.90210297)
\curveto(349.71003746,409.85209825)(349.82003735,409.82709827)(349.96004028,409.82710297)
\lineto(350.36504028,409.82710297)
\lineto(352.03004028,409.82710297)
\lineto(352.46504028,409.82710297)
\curveto(352.62503454,409.83709826)(352.73003444,409.88209822)(352.78004028,409.96210297)
}
}
{
\newrgbcolor{curcolor}{0 0 0}
\pscustom[linestyle=none,fillstyle=solid,fillcolor=curcolor]
{
\newpath
\moveto(359.24832153,415.82710297)
\curveto(359.33831769,415.82709227)(359.43831759,415.82709227)(359.54832153,415.82710297)
\curveto(359.66831736,415.82709227)(359.78331725,415.82209228)(359.89332153,415.81210297)
\curveto(360.01331702,415.8020923)(360.11831691,415.78209232)(360.20832153,415.75210297)
\curveto(360.29831673,415.73209237)(360.35831667,415.6970924)(360.38832153,415.64710297)
\curveto(360.44831658,415.56709253)(360.47831655,415.45209265)(360.47832153,415.30210297)
\lineto(360.47832153,414.89710297)
\curveto(360.47831655,414.7970933)(360.47331656,414.6970934)(360.46332153,414.59710297)
\curveto(360.46331657,414.4970936)(360.44331659,414.42209368)(360.40332153,414.37210297)
\curveto(360.36331667,414.31209379)(360.31331672,414.27209383)(360.25332153,414.25210297)
\curveto(360.19331684,414.24209386)(360.12331691,414.23709386)(360.04332153,414.23710297)
\lineto(359.81832153,414.23710297)
\curveto(359.74831728,414.24709385)(359.67831735,414.24709385)(359.60832153,414.23710297)
\curveto(359.4283176,414.1970939)(359.28831774,414.14709395)(359.18832153,414.08710297)
\curveto(359.08831794,414.03709406)(359.00831802,413.92709417)(358.94832153,413.75710297)
\curveto(358.9283181,413.72709437)(358.91831811,413.6970944)(358.91832153,413.66710297)
\curveto(358.9283181,413.64709445)(358.9283181,413.62209448)(358.91832153,413.59210297)
\curveto(358.90831812,413.55209455)(358.89831813,413.49209461)(358.88832153,413.41210297)
\curveto(358.87831815,413.33209477)(358.87831815,413.26709483)(358.88832153,413.21710297)
\curveto(358.90831812,413.14709495)(358.9333181,413.08709501)(358.96332153,413.03710297)
\curveto(358.99331804,412.98709511)(359.03831799,412.94709515)(359.09832153,412.91710297)
\curveto(359.19831783,412.86709523)(359.31831771,412.85209525)(359.45832153,412.87210297)
\curveto(359.59831743,412.89209521)(359.7283173,412.89209521)(359.84832153,412.87210297)
\curveto(359.89831713,412.86209524)(359.93831709,412.85709524)(359.96832153,412.85710297)
\curveto(360.00831702,412.86709523)(360.04831698,412.86709523)(360.08832153,412.85710297)
\curveto(360.17831685,412.81709528)(360.24331679,412.77209533)(360.28332153,412.72210297)
\curveto(360.30331673,412.69209541)(360.31831671,412.64209546)(360.32832153,412.57210297)
\curveto(360.33831669,412.51209559)(360.34831668,412.44209566)(360.35832153,412.36210297)
\curveto(360.36831666,412.29209581)(360.36831666,412.21709588)(360.35832153,412.13710297)
\curveto(360.35831667,412.06709603)(360.35331668,412.01209609)(360.34332153,411.97210297)
\curveto(360.3333167,411.93209617)(360.3333167,411.89209621)(360.34332153,411.85210297)
\curveto(360.35331668,411.82209628)(360.34831668,411.78709631)(360.32832153,411.74710297)
\curveto(360.30831672,411.62709647)(360.24831678,411.55209655)(360.14832153,411.52210297)
\curveto(360.06831696,411.48209662)(359.97331706,411.46209664)(359.86332153,411.46210297)
\curveto(359.75331728,411.47209663)(359.64331739,411.47709662)(359.53332153,411.47710297)
\lineto(359.42832153,411.47710297)
\curveto(359.38831764,411.47709662)(359.35331768,411.47209663)(359.32332153,411.46210297)
\lineto(359.20332153,411.46210297)
\curveto(359.033318,411.42209668)(358.9283181,411.31209679)(358.88832153,411.13210297)
\curveto(358.86831816,411.07209703)(358.86331817,411.0020971)(358.87332153,410.92210297)
\curveto(358.88331815,410.84209726)(358.88831814,410.76209734)(358.88832153,410.68210297)
\lineto(358.88832153,409.76710297)
\lineto(358.88832153,406.84210297)
\lineto(358.88832153,406.13710297)
\lineto(358.88832153,405.94210297)
\curveto(358.89831813,405.88210222)(358.89331814,405.82710227)(358.87332153,405.77710297)
\lineto(358.87332153,405.61210297)
\curveto(358.87331816,405.45210265)(358.84831818,405.33710276)(358.79832153,405.26710297)
\curveto(358.77831825,405.23710286)(358.74331829,405.21210289)(358.69332153,405.19210297)
\curveto(358.64331839,405.18210292)(358.59331844,405.16710293)(358.54332153,405.14710297)
\lineto(358.46832153,405.14710297)
\curveto(358.41831861,405.13710296)(358.36331867,405.13210297)(358.30332153,405.13210297)
\curveto(358.24331879,405.14210296)(358.18831884,405.14710295)(358.13832153,405.14710297)
\lineto(357.47832153,405.14710297)
\curveto(357.40831962,405.14710295)(357.3333197,405.14210296)(357.25332153,405.13210297)
\curveto(357.18331985,405.13210297)(357.12331991,405.14210296)(357.07332153,405.16210297)
\curveto(356.95332008,405.19210291)(356.87332016,405.24210286)(356.83332153,405.31210297)
\curveto(356.80332023,405.36210274)(356.78332025,405.42710267)(356.77332153,405.50710297)
\lineto(356.77332153,405.74710297)
\lineto(356.77332153,406.52710297)
\lineto(356.77332153,410.72710297)
\curveto(356.77332026,410.8970972)(356.76332027,411.04209706)(356.74332153,411.16210297)
\curveto(356.72332031,411.29209681)(356.65332038,411.38209672)(356.53332153,411.43210297)
\curveto(356.42332061,411.48209662)(356.28832074,411.49209661)(356.12832153,411.46210297)
\curveto(355.96832106,411.44209666)(355.8333212,411.45709664)(355.72332153,411.50710297)
\curveto(355.61332142,411.55709654)(355.54332149,411.64209646)(355.51332153,411.76210297)
\curveto(355.49332154,411.81209629)(355.48832154,411.87209623)(355.49832153,411.94210297)
\lineto(355.49832153,412.15210297)
\curveto(355.49832153,412.33209577)(355.50832152,412.48209562)(355.52832153,412.60210297)
\curveto(355.54832148,412.72209538)(355.6333214,412.80709529)(355.78332153,412.85710297)
\curveto(355.86332117,412.87709522)(355.94832108,412.88709521)(356.03832153,412.88710297)
\lineto(356.29332153,412.88710297)
\curveto(356.38332065,412.88709521)(356.46332057,412.89209521)(356.53332153,412.90210297)
\curveto(356.60332043,412.92209518)(356.65832037,412.96209514)(356.69832153,413.02210297)
\curveto(356.76832026,413.12209498)(356.79332024,413.24709485)(356.77332153,413.39710297)
\curveto(356.76332027,413.55709454)(356.77332026,413.70709439)(356.80332153,413.84710297)
\curveto(356.81332022,413.88709421)(356.81832021,413.92709417)(356.81832153,413.96710297)
\curveto(356.8283202,414.00709409)(356.83832019,414.05209405)(356.84832153,414.10210297)
\curveto(356.88832014,414.24209386)(356.9283201,414.36709373)(356.96832153,414.47710297)
\curveto(357.00832002,414.5970935)(357.06331997,414.70709339)(357.13332153,414.80710297)
\curveto(357.27331976,415.04709305)(357.45831957,415.23709286)(357.68832153,415.37710297)
\curveto(357.91831911,415.52709257)(358.17831885,415.64209246)(358.46832153,415.72210297)
\curveto(358.54831848,415.75209235)(358.6333184,415.76709233)(358.72332153,415.76710297)
\curveto(358.81331822,415.77709232)(358.90331813,415.79209231)(358.99332153,415.81210297)
\curveto(359.02331801,415.82209228)(359.06831796,415.82209228)(359.12832153,415.81210297)
\curveto(359.18831784,415.8020923)(359.2283178,415.80709229)(359.24832153,415.82710297)
\moveto(363.67332153,415.72210297)
\curveto(363.62331341,415.77209233)(363.55331348,415.8020923)(363.46332153,415.81210297)
\curveto(363.37331366,415.82209228)(363.27831375,415.82709227)(363.17832153,415.82710297)
\lineto(362.08332153,415.82710297)
\curveto(362.06331497,415.82709227)(362.03831499,415.82209228)(362.00832153,415.81210297)
\curveto(361.97831505,415.81209229)(361.95331508,415.81209229)(361.93332153,415.81210297)
\lineto(361.81332153,415.75210297)
\curveto(361.78331525,415.73209237)(361.75831527,415.7020924)(361.73832153,415.66210297)
\curveto(361.70831532,415.61209249)(361.68831534,415.55209255)(361.67832153,415.48210297)
\lineto(361.67832153,415.27210297)
\curveto(361.67831535,415.19209291)(361.67331536,415.097093)(361.66332153,414.98710297)
\lineto(361.66332153,414.65710297)
\curveto(361.67331536,414.55709354)(361.68331535,414.46209364)(361.69332153,414.37210297)
\curveto(361.71331532,414.29209381)(361.74331529,414.23709386)(361.78332153,414.20710297)
\curveto(361.84331519,414.15709394)(361.91331512,414.12709397)(361.99332153,414.11710297)
\curveto(362.08331495,414.10709399)(362.18331485,414.102094)(362.29332153,414.10210297)
\lineto(363.13332153,414.10210297)
\curveto(363.24331379,414.102094)(363.34331369,414.10709399)(363.43332153,414.11710297)
\curveto(363.5333135,414.12709397)(363.61331342,414.15709394)(363.67332153,414.20710297)
\curveto(363.74331329,414.25709384)(363.77831325,414.35209375)(363.77832153,414.49210297)
\lineto(363.77832153,414.89710297)
\lineto(363.77832153,415.36210297)
\curveto(363.77831325,415.52209258)(363.74331329,415.64209246)(363.67332153,415.72210297)
\moveto(363.77832153,412.24210297)
\curveto(363.77831325,412.34209576)(363.77331326,412.43709566)(363.76332153,412.52710297)
\curveto(363.76331327,412.61709548)(363.74331329,412.68709541)(363.70332153,412.73710297)
\curveto(363.67331336,412.78709531)(363.6333134,412.81709528)(363.58332153,412.82710297)
\curveto(363.5333135,412.83709526)(363.47831355,412.85209525)(363.41832153,412.87210297)
\lineto(363.29832153,412.87210297)
\curveto(363.24831378,412.88209522)(363.18331385,412.88209522)(363.10332153,412.87210297)
\lineto(362.90832153,412.87210297)
\lineto(362.15832153,412.87210297)
\curveto(362.13831489,412.86209524)(362.10831492,412.85709524)(362.06832153,412.85710297)
\curveto(362.03831499,412.86709523)(362.01331502,412.86709523)(361.99332153,412.85710297)
\curveto(361.88331515,412.83709526)(361.79831523,412.79209531)(361.73832153,412.72210297)
\curveto(361.69831533,412.66209544)(361.67831535,412.58209552)(361.67832153,412.48210297)
\lineto(361.67832153,412.18210297)
\lineto(361.67832153,405.85210297)
\lineto(361.67832153,405.50710297)
\curveto(361.68831534,405.40710269)(361.72331531,405.32210278)(361.78332153,405.25210297)
\curveto(361.82331521,405.2021029)(361.88331515,405.17210293)(361.96332153,405.16210297)
\curveto(362.05331498,405.15210295)(362.14331489,405.14710295)(362.23332153,405.14710297)
\lineto(363.07332153,405.14710297)
\curveto(363.15331388,405.14710295)(363.2283138,405.14210296)(363.29832153,405.13210297)
\curveto(363.36831366,405.13210297)(363.4333136,405.14210296)(363.49332153,405.16210297)
\curveto(363.66331337,405.21210289)(363.75331328,405.30710279)(363.76332153,405.44710297)
\curveto(363.77331326,405.58710251)(363.77831325,405.75710234)(363.77832153,405.95710297)
\lineto(363.77832153,412.24210297)
}
}
{
\newrgbcolor{curcolor}{0 0 0}
\pscustom[linestyle=none,fillstyle=solid,fillcolor=curcolor]
{
\newpath
\moveto(369.87324341,413.06710297)
\curveto(370.4732376,413.08709501)(370.9732371,413.0020951)(371.37324341,412.81210297)
\curveto(371.7732363,412.62209548)(372.08823599,412.34209576)(372.31824341,411.97210297)
\curveto(372.38823569,411.86209624)(372.44323563,411.74209636)(372.48324341,411.61210297)
\curveto(372.52323555,411.49209661)(372.56323551,411.36709673)(372.60324341,411.23710297)
\curveto(372.62323545,411.15709694)(372.63323544,411.08209702)(372.63324341,411.01210297)
\curveto(372.64323543,410.94209716)(372.65823542,410.87209723)(372.67824341,410.80210297)
\curveto(372.6782354,410.74209736)(372.68323539,410.7020974)(372.69324341,410.68210297)
\curveto(372.71323536,410.54209756)(372.72323535,410.3970977)(372.72324341,410.24710297)
\lineto(372.72324341,409.81210297)
\lineto(372.72324341,408.47710297)
\lineto(372.72324341,406.04710297)
\curveto(372.72323535,405.85710224)(372.71823536,405.67210243)(372.70824341,405.49210297)
\curveto(372.70823537,405.32210278)(372.63823544,405.21210289)(372.49824341,405.16210297)
\curveto(372.43823564,405.14210296)(372.36823571,405.13210297)(372.28824341,405.13210297)
\lineto(372.04824341,405.13210297)
\lineto(371.23824341,405.13210297)
\curveto(371.11823696,405.13210297)(371.00823707,405.13710296)(370.90824341,405.14710297)
\curveto(370.81823726,405.16710293)(370.74823733,405.21210289)(370.69824341,405.28210297)
\curveto(370.65823742,405.34210276)(370.63323744,405.41710268)(370.62324341,405.50710297)
\lineto(370.62324341,405.82210297)
\lineto(370.62324341,406.87210297)
\lineto(370.62324341,409.10710297)
\curveto(370.62323745,409.47709862)(370.60823747,409.81709828)(370.57824341,410.12710297)
\curveto(370.54823753,410.44709765)(370.45823762,410.71709738)(370.30824341,410.93710297)
\curveto(370.16823791,411.13709696)(369.96323811,411.27709682)(369.69324341,411.35710297)
\curveto(369.64323843,411.37709672)(369.58823849,411.38709671)(369.52824341,411.38710297)
\curveto(369.4782386,411.38709671)(369.42323865,411.3970967)(369.36324341,411.41710297)
\curveto(369.31323876,411.42709667)(369.24823883,411.42709667)(369.16824341,411.41710297)
\curveto(369.09823898,411.41709668)(369.04323903,411.41209669)(369.00324341,411.40210297)
\curveto(368.96323911,411.39209671)(368.92823915,411.38709671)(368.89824341,411.38710297)
\curveto(368.86823921,411.38709671)(368.83823924,411.38209672)(368.80824341,411.37210297)
\curveto(368.5782395,411.31209679)(368.39323968,411.23209687)(368.25324341,411.13210297)
\curveto(367.93324014,410.9020972)(367.74324033,410.56709753)(367.68324341,410.12710297)
\curveto(367.62324045,409.68709841)(367.59324048,409.19209891)(367.59324341,408.64210297)
\lineto(367.59324341,406.76710297)
\lineto(367.59324341,405.85210297)
\lineto(367.59324341,405.58210297)
\curveto(367.59324048,405.49210261)(367.5782405,405.41710268)(367.54824341,405.35710297)
\curveto(367.49824058,405.24710285)(367.41824066,405.18210292)(367.30824341,405.16210297)
\curveto(367.19824088,405.14210296)(367.06324101,405.13210297)(366.90324341,405.13210297)
\lineto(366.15324341,405.13210297)
\curveto(366.04324203,405.13210297)(365.93324214,405.13710296)(365.82324341,405.14710297)
\curveto(365.71324236,405.15710294)(365.63324244,405.19210291)(365.58324341,405.25210297)
\curveto(365.51324256,405.34210276)(365.4782426,405.47210263)(365.47824341,405.64210297)
\curveto(365.48824259,405.81210229)(365.49324258,405.97210213)(365.49324341,406.12210297)
\lineto(365.49324341,408.16210297)
\lineto(365.49324341,411.46210297)
\lineto(365.49324341,412.22710297)
\lineto(365.49324341,412.52710297)
\curveto(365.50324257,412.61709548)(365.53324254,412.69209541)(365.58324341,412.75210297)
\curveto(365.60324247,412.78209532)(365.63324244,412.8020953)(365.67324341,412.81210297)
\curveto(365.72324235,412.83209527)(365.7732423,412.84709525)(365.82324341,412.85710297)
\lineto(365.89824341,412.85710297)
\curveto(365.94824213,412.86709523)(365.99824208,412.87209523)(366.04824341,412.87210297)
\lineto(366.21324341,412.87210297)
\lineto(366.84324341,412.87210297)
\curveto(366.92324115,412.87209523)(366.99824108,412.86709523)(367.06824341,412.85710297)
\curveto(367.14824093,412.85709524)(367.21824086,412.84709525)(367.27824341,412.82710297)
\curveto(367.34824073,412.7970953)(367.39324068,412.75209535)(367.41324341,412.69210297)
\curveto(367.44324063,412.63209547)(367.46824061,412.56209554)(367.48824341,412.48210297)
\curveto(367.49824058,412.44209566)(367.49824058,412.40709569)(367.48824341,412.37710297)
\curveto(367.48824059,412.34709575)(367.49824058,412.31709578)(367.51824341,412.28710297)
\curveto(367.53824054,412.23709586)(367.55324052,412.20709589)(367.56324341,412.19710297)
\curveto(367.58324049,412.18709591)(367.60824047,412.17209593)(367.63824341,412.15210297)
\curveto(367.74824033,412.14209596)(367.83824024,412.17709592)(367.90824341,412.25710297)
\curveto(367.9782401,412.34709575)(368.05324002,412.41709568)(368.13324341,412.46710297)
\curveto(368.40323967,412.66709543)(368.70323937,412.82709527)(369.03324341,412.94710297)
\curveto(369.12323895,412.97709512)(369.21323886,412.9970951)(369.30324341,413.00710297)
\curveto(369.40323867,413.01709508)(369.50823857,413.03209507)(369.61824341,413.05210297)
\curveto(369.64823843,413.06209504)(369.69323838,413.06209504)(369.75324341,413.05210297)
\curveto(369.81323826,413.05209505)(369.85323822,413.05709504)(369.87324341,413.06710297)
}
}
{
\newrgbcolor{curcolor}{0 0 0}
\pscustom[linestyle=none,fillstyle=solid,fillcolor=curcolor]
{
\newpath
\moveto(376.45449341,415.72210297)
\curveto(376.52449046,415.64209246)(376.55949042,415.52209258)(376.55949341,415.36210297)
\lineto(376.55949341,414.89710297)
\lineto(376.55949341,414.49210297)
\curveto(376.55949042,414.35209375)(376.52449046,414.25709384)(376.45449341,414.20710297)
\curveto(376.39449059,414.15709394)(376.31449067,414.12709397)(376.21449341,414.11710297)
\curveto(376.12449086,414.10709399)(376.02449096,414.102094)(375.91449341,414.10210297)
\lineto(375.07449341,414.10210297)
\curveto(374.96449202,414.102094)(374.86449212,414.10709399)(374.77449341,414.11710297)
\curveto(374.69449229,414.12709397)(374.62449236,414.15709394)(374.56449341,414.20710297)
\curveto(374.52449246,414.23709386)(374.49449249,414.29209381)(374.47449341,414.37210297)
\curveto(374.46449252,414.46209364)(374.45449253,414.55709354)(374.44449341,414.65710297)
\lineto(374.44449341,414.98710297)
\curveto(374.45449253,415.097093)(374.45949252,415.19209291)(374.45949341,415.27210297)
\lineto(374.45949341,415.48210297)
\curveto(374.46949251,415.55209255)(374.48949249,415.61209249)(374.51949341,415.66210297)
\curveto(374.53949244,415.7020924)(374.56449242,415.73209237)(374.59449341,415.75210297)
\lineto(374.71449341,415.81210297)
\curveto(374.73449225,415.81209229)(374.75949222,415.81209229)(374.78949341,415.81210297)
\curveto(374.81949216,415.82209228)(374.84449214,415.82709227)(374.86449341,415.82710297)
\lineto(375.95949341,415.82710297)
\curveto(376.05949092,415.82709227)(376.15449083,415.82209228)(376.24449341,415.81210297)
\curveto(376.33449065,415.8020923)(376.40449058,415.77209233)(376.45449341,415.72210297)
\moveto(376.55949341,405.95710297)
\curveto(376.55949042,405.75710234)(376.55449043,405.58710251)(376.54449341,405.44710297)
\curveto(376.53449045,405.30710279)(376.44449054,405.21210289)(376.27449341,405.16210297)
\curveto(376.21449077,405.14210296)(376.14949083,405.13210297)(376.07949341,405.13210297)
\curveto(376.00949097,405.14210296)(375.93449105,405.14710295)(375.85449341,405.14710297)
\lineto(375.01449341,405.14710297)
\curveto(374.92449206,405.14710295)(374.83449215,405.15210295)(374.74449341,405.16210297)
\curveto(374.66449232,405.17210293)(374.60449238,405.2021029)(374.56449341,405.25210297)
\curveto(374.50449248,405.32210278)(374.46949251,405.40710269)(374.45949341,405.50710297)
\lineto(374.45949341,405.85210297)
\lineto(374.45949341,412.18210297)
\lineto(374.45949341,412.48210297)
\curveto(374.45949252,412.58209552)(374.4794925,412.66209544)(374.51949341,412.72210297)
\curveto(374.5794924,412.79209531)(374.66449232,412.83709526)(374.77449341,412.85710297)
\curveto(374.79449219,412.86709523)(374.81949216,412.86709523)(374.84949341,412.85710297)
\curveto(374.88949209,412.85709524)(374.91949206,412.86209524)(374.93949341,412.87210297)
\lineto(375.68949341,412.87210297)
\lineto(375.88449341,412.87210297)
\curveto(375.96449102,412.88209522)(376.02949095,412.88209522)(376.07949341,412.87210297)
\lineto(376.19949341,412.87210297)
\curveto(376.25949072,412.85209525)(376.31449067,412.83709526)(376.36449341,412.82710297)
\curveto(376.41449057,412.81709528)(376.45449053,412.78709531)(376.48449341,412.73710297)
\curveto(376.52449046,412.68709541)(376.54449044,412.61709548)(376.54449341,412.52710297)
\curveto(376.55449043,412.43709566)(376.55949042,412.34209576)(376.55949341,412.24210297)
\lineto(376.55949341,405.95710297)
}
}
{
\newrgbcolor{curcolor}{0 0 0}
\pscustom[linestyle=none,fillstyle=solid,fillcolor=curcolor]
{
\newpath
\moveto(385.81168091,405.98710297)
\lineto(385.81168091,405.56710297)
\curveto(385.81167254,405.43710266)(385.78167257,405.33210277)(385.72168091,405.25210297)
\curveto(385.67167268,405.2021029)(385.60667274,405.16710293)(385.52668091,405.14710297)
\curveto(385.4466729,405.13710296)(385.35667299,405.13210297)(385.25668091,405.13210297)
\lineto(384.43168091,405.13210297)
\lineto(384.14668091,405.13210297)
\curveto(384.06667428,405.14210296)(384.00167435,405.16710293)(383.95168091,405.20710297)
\curveto(383.88167447,405.25710284)(383.84167451,405.32210278)(383.83168091,405.40210297)
\curveto(383.82167453,405.48210262)(383.80167455,405.56210254)(383.77168091,405.64210297)
\curveto(383.7516746,405.66210244)(383.73167462,405.67710242)(383.71168091,405.68710297)
\curveto(383.70167465,405.70710239)(383.68667466,405.72710237)(383.66668091,405.74710297)
\curveto(383.55667479,405.74710235)(383.47667487,405.72210238)(383.42668091,405.67210297)
\lineto(383.27668091,405.52210297)
\curveto(383.20667514,405.47210263)(383.14167521,405.42710267)(383.08168091,405.38710297)
\curveto(383.02167533,405.35710274)(382.95667539,405.31710278)(382.88668091,405.26710297)
\curveto(382.8466755,405.24710285)(382.80167555,405.22710287)(382.75168091,405.20710297)
\curveto(382.71167564,405.18710291)(382.66667568,405.16710293)(382.61668091,405.14710297)
\curveto(382.47667587,405.097103)(382.32667602,405.05210305)(382.16668091,405.01210297)
\curveto(382.11667623,404.99210311)(382.07167628,404.98210312)(382.03168091,404.98210297)
\curveto(381.99167636,404.98210312)(381.9516764,404.97710312)(381.91168091,404.96710297)
\lineto(381.77668091,404.96710297)
\curveto(381.7466766,404.95710314)(381.70667664,404.95210315)(381.65668091,404.95210297)
\lineto(381.52168091,404.95210297)
\curveto(381.46167689,404.93210317)(381.37167698,404.92710317)(381.25168091,404.93710297)
\curveto(381.13167722,404.93710316)(381.0466773,404.94710315)(380.99668091,404.96710297)
\curveto(380.92667742,404.98710311)(380.86167749,404.9971031)(380.80168091,404.99710297)
\curveto(380.7516776,404.98710311)(380.69667765,404.99210311)(380.63668091,405.01210297)
\lineto(380.27668091,405.13210297)
\curveto(380.16667818,405.16210294)(380.05667829,405.2021029)(379.94668091,405.25210297)
\curveto(379.59667875,405.4021027)(379.28167907,405.63210247)(379.00168091,405.94210297)
\curveto(378.73167962,406.26210184)(378.51667983,406.5971015)(378.35668091,406.94710297)
\curveto(378.30668004,407.05710104)(378.26668008,407.16210094)(378.23668091,407.26210297)
\curveto(378.20668014,407.37210073)(378.17168018,407.48210062)(378.13168091,407.59210297)
\curveto(378.12168023,407.63210047)(378.11668023,407.66710043)(378.11668091,407.69710297)
\curveto(378.11668023,407.73710036)(378.10668024,407.78210032)(378.08668091,407.83210297)
\curveto(378.06668028,407.91210019)(378.0466803,407.9971001)(378.02668091,408.08710297)
\curveto(378.01668033,408.18709991)(378.00168035,408.28709981)(377.98168091,408.38710297)
\curveto(377.97168038,408.41709968)(377.96668038,408.45209965)(377.96668091,408.49210297)
\curveto(377.97668037,408.53209957)(377.97668037,408.56709953)(377.96668091,408.59710297)
\lineto(377.96668091,408.73210297)
\curveto(377.96668038,408.78209932)(377.96168039,408.83209927)(377.95168091,408.88210297)
\curveto(377.94168041,408.93209917)(377.93668041,408.98709911)(377.93668091,409.04710297)
\curveto(377.93668041,409.11709898)(377.94168041,409.17209893)(377.95168091,409.21210297)
\curveto(377.96168039,409.26209884)(377.96668038,409.30709879)(377.96668091,409.34710297)
\lineto(377.96668091,409.49710297)
\curveto(377.97668037,409.54709855)(377.97668037,409.59209851)(377.96668091,409.63210297)
\curveto(377.96668038,409.68209842)(377.97668037,409.73209837)(377.99668091,409.78210297)
\curveto(378.01668033,409.89209821)(378.03168032,409.9970981)(378.04168091,410.09710297)
\curveto(378.06168029,410.1970979)(378.08668026,410.2970978)(378.11668091,410.39710297)
\curveto(378.15668019,410.51709758)(378.19168016,410.63209747)(378.22168091,410.74210297)
\curveto(378.2516801,410.85209725)(378.29168006,410.96209714)(378.34168091,411.07210297)
\curveto(378.48167987,411.37209673)(378.65667969,411.65709644)(378.86668091,411.92710297)
\curveto(378.88667946,411.95709614)(378.91167944,411.98209612)(378.94168091,412.00210297)
\curveto(378.98167937,412.03209607)(379.01167934,412.06209604)(379.03168091,412.09210297)
\curveto(379.07167928,412.14209596)(379.11167924,412.18709591)(379.15168091,412.22710297)
\curveto(379.19167916,412.26709583)(379.23667911,412.30709579)(379.28668091,412.34710297)
\curveto(379.32667902,412.36709573)(379.36167899,412.39209571)(379.39168091,412.42210297)
\curveto(379.42167893,412.46209564)(379.45667889,412.49209561)(379.49668091,412.51210297)
\curveto(379.7466786,412.68209542)(380.03667831,412.82209528)(380.36668091,412.93210297)
\curveto(380.43667791,412.95209515)(380.50667784,412.96709513)(380.57668091,412.97710297)
\curveto(380.65667769,412.98709511)(380.73667761,413.0020951)(380.81668091,413.02210297)
\curveto(380.88667746,413.04209506)(380.97667737,413.05209505)(381.08668091,413.05210297)
\curveto(381.19667715,413.06209504)(381.30667704,413.06709503)(381.41668091,413.06710297)
\curveto(381.52667682,413.06709503)(381.63167672,413.06209504)(381.73168091,413.05210297)
\curveto(381.84167651,413.04209506)(381.93167642,413.02709507)(382.00168091,413.00710297)
\curveto(382.1516762,412.95709514)(382.29667605,412.91209519)(382.43668091,412.87210297)
\curveto(382.57667577,412.83209527)(382.70667564,412.77709532)(382.82668091,412.70710297)
\curveto(382.89667545,412.65709544)(382.96167539,412.60709549)(383.02168091,412.55710297)
\curveto(383.08167527,412.51709558)(383.1466752,412.47209563)(383.21668091,412.42210297)
\curveto(383.25667509,412.39209571)(383.31167504,412.35209575)(383.38168091,412.30210297)
\curveto(383.46167489,412.25209585)(383.53667481,412.25209585)(383.60668091,412.30210297)
\curveto(383.6466747,412.32209578)(383.66667468,412.35709574)(383.66668091,412.40710297)
\curveto(383.66667468,412.45709564)(383.67667467,412.50709559)(383.69668091,412.55710297)
\lineto(383.69668091,412.70710297)
\curveto(383.70667464,412.73709536)(383.71167464,412.77209533)(383.71168091,412.81210297)
\lineto(383.71168091,412.93210297)
\lineto(383.71168091,414.97210297)
\curveto(383.71167464,415.08209302)(383.70667464,415.2020929)(383.69668091,415.33210297)
\curveto(383.69667465,415.47209263)(383.72167463,415.57709252)(383.77168091,415.64710297)
\curveto(383.81167454,415.72709237)(383.88667446,415.77709232)(383.99668091,415.79710297)
\curveto(384.01667433,415.80709229)(384.03667431,415.80709229)(384.05668091,415.79710297)
\curveto(384.07667427,415.7970923)(384.09667425,415.8020923)(384.11668091,415.81210297)
\lineto(385.18168091,415.81210297)
\curveto(385.30167305,415.81209229)(385.41167294,415.80709229)(385.51168091,415.79710297)
\curveto(385.61167274,415.78709231)(385.68667266,415.74709235)(385.73668091,415.67710297)
\curveto(385.78667256,415.5970925)(385.81167254,415.49209261)(385.81168091,415.36210297)
\lineto(385.81168091,415.00210297)
\lineto(385.81168091,405.98710297)
\moveto(383.77168091,408.92710297)
\curveto(383.78167457,408.96709913)(383.78167457,409.00709909)(383.77168091,409.04710297)
\lineto(383.77168091,409.18210297)
\curveto(383.77167458,409.28209882)(383.76667458,409.38209872)(383.75668091,409.48210297)
\curveto(383.7466746,409.58209852)(383.73167462,409.67209843)(383.71168091,409.75210297)
\curveto(383.69167466,409.86209824)(383.67167468,409.96209814)(383.65168091,410.05210297)
\curveto(383.64167471,410.14209796)(383.61667473,410.22709787)(383.57668091,410.30710297)
\curveto(383.43667491,410.66709743)(383.23167512,410.95209715)(382.96168091,411.16210297)
\curveto(382.70167565,411.37209673)(382.32167603,411.47709662)(381.82168091,411.47710297)
\curveto(381.76167659,411.47709662)(381.68167667,411.46709663)(381.58168091,411.44710297)
\curveto(381.50167685,411.42709667)(381.42667692,411.40709669)(381.35668091,411.38710297)
\curveto(381.29667705,411.37709672)(381.23667711,411.35709674)(381.17668091,411.32710297)
\curveto(380.90667744,411.21709688)(380.69667765,411.04709705)(380.54668091,410.81710297)
\curveto(380.39667795,410.58709751)(380.27667807,410.32709777)(380.18668091,410.03710297)
\curveto(380.15667819,409.93709816)(380.13667821,409.83709826)(380.12668091,409.73710297)
\curveto(380.11667823,409.63709846)(380.09667825,409.53209857)(380.06668091,409.42210297)
\lineto(380.06668091,409.21210297)
\curveto(380.0466783,409.12209898)(380.04167831,408.9970991)(380.05168091,408.83710297)
\curveto(380.06167829,408.68709941)(380.07667827,408.57709952)(380.09668091,408.50710297)
\lineto(380.09668091,408.41710297)
\curveto(380.10667824,408.3970997)(380.11167824,408.37709972)(380.11168091,408.35710297)
\curveto(380.13167822,408.27709982)(380.1466782,408.2020999)(380.15668091,408.13210297)
\curveto(380.17667817,408.06210004)(380.19667815,407.98710011)(380.21668091,407.90710297)
\curveto(380.38667796,407.38710071)(380.67667767,407.0021011)(381.08668091,406.75210297)
\curveto(381.21667713,406.66210144)(381.39667695,406.59210151)(381.62668091,406.54210297)
\curveto(381.66667668,406.53210157)(381.72667662,406.52710157)(381.80668091,406.52710297)
\curveto(381.83667651,406.51710158)(381.88167647,406.50710159)(381.94168091,406.49710297)
\curveto(382.01167634,406.4971016)(382.06667628,406.5021016)(382.10668091,406.51210297)
\curveto(382.18667616,406.53210157)(382.26667608,406.54710155)(382.34668091,406.55710297)
\curveto(382.42667592,406.56710153)(382.50667584,406.58710151)(382.58668091,406.61710297)
\curveto(382.83667551,406.72710137)(383.03667531,406.86710123)(383.18668091,407.03710297)
\curveto(383.33667501,407.20710089)(383.46667488,407.42210068)(383.57668091,407.68210297)
\curveto(383.61667473,407.77210033)(383.6466747,407.86210024)(383.66668091,407.95210297)
\curveto(383.68667466,408.05210005)(383.70667464,408.15709994)(383.72668091,408.26710297)
\curveto(383.73667461,408.31709978)(383.73667461,408.36209974)(383.72668091,408.40210297)
\curveto(383.72667462,408.45209965)(383.73667461,408.5020996)(383.75668091,408.55210297)
\curveto(383.76667458,408.58209952)(383.77167458,408.61709948)(383.77168091,408.65710297)
\lineto(383.77168091,408.79210297)
\lineto(383.77168091,408.92710297)
}
}
{
\newrgbcolor{curcolor}{0 0 0}
\pscustom[linestyle=none,fillstyle=solid,fillcolor=curcolor]
{
\newpath
\moveto(395.16160278,409.31710297)
\curveto(395.18159421,409.25709884)(395.1915942,409.17209893)(395.19160278,409.06210297)
\curveto(395.1915942,408.95209915)(395.18159421,408.86709923)(395.16160278,408.80710297)
\lineto(395.16160278,408.65710297)
\curveto(395.14159425,408.57709952)(395.13159426,408.4970996)(395.13160278,408.41710297)
\curveto(395.14159425,408.33709976)(395.13659426,408.25709984)(395.11660278,408.17710297)
\curveto(395.0965943,408.10709999)(395.08159431,408.04210006)(395.07160278,407.98210297)
\curveto(395.06159433,407.92210018)(395.05159434,407.85710024)(395.04160278,407.78710297)
\curveto(395.00159439,407.67710042)(394.96659443,407.56210054)(394.93660278,407.44210297)
\curveto(394.90659449,407.33210077)(394.86659453,407.22710087)(394.81660278,407.12710297)
\curveto(394.60659479,406.64710145)(394.33159506,406.25710184)(393.99160278,405.95710297)
\curveto(393.65159574,405.65710244)(393.24159615,405.40710269)(392.76160278,405.20710297)
\curveto(392.64159675,405.15710294)(392.51659688,405.12210298)(392.38660278,405.10210297)
\curveto(392.26659713,405.07210303)(392.14159725,405.04210306)(392.01160278,405.01210297)
\curveto(391.96159743,404.99210311)(391.90659749,404.98210312)(391.84660278,404.98210297)
\curveto(391.78659761,404.98210312)(391.73159766,404.97710312)(391.68160278,404.96710297)
\lineto(391.57660278,404.96710297)
\curveto(391.54659785,404.95710314)(391.51659788,404.95210315)(391.48660278,404.95210297)
\curveto(391.43659796,404.94210316)(391.35659804,404.93710316)(391.24660278,404.93710297)
\curveto(391.13659826,404.92710317)(391.05159834,404.93210317)(390.99160278,404.95210297)
\lineto(390.84160278,404.95210297)
\curveto(390.7915986,404.96210314)(390.73659866,404.96710313)(390.67660278,404.96710297)
\curveto(390.62659877,404.95710314)(390.57659882,404.96210314)(390.52660278,404.98210297)
\curveto(390.48659891,404.99210311)(390.44659895,404.9971031)(390.40660278,404.99710297)
\curveto(390.37659902,404.9971031)(390.33659906,405.0021031)(390.28660278,405.01210297)
\curveto(390.18659921,405.04210306)(390.08659931,405.06710303)(389.98660278,405.08710297)
\curveto(389.88659951,405.10710299)(389.7915996,405.13710296)(389.70160278,405.17710297)
\curveto(389.58159981,405.21710288)(389.46659993,405.25710284)(389.35660278,405.29710297)
\curveto(389.25660014,405.33710276)(389.15160024,405.38710271)(389.04160278,405.44710297)
\curveto(388.6916007,405.65710244)(388.391601,405.9021022)(388.14160278,406.18210297)
\curveto(387.8916015,406.46210164)(387.68160171,406.7971013)(387.51160278,407.18710297)
\curveto(387.46160193,407.27710082)(387.42160197,407.37210073)(387.39160278,407.47210297)
\curveto(387.37160202,407.57210053)(387.34660205,407.67710042)(387.31660278,407.78710297)
\curveto(387.2966021,407.83710026)(387.28660211,407.88210022)(387.28660278,407.92210297)
\curveto(387.28660211,407.96210014)(387.27660212,408.00710009)(387.25660278,408.05710297)
\curveto(387.23660216,408.13709996)(387.22660217,408.21709988)(387.22660278,408.29710297)
\curveto(387.22660217,408.38709971)(387.21660218,408.47209963)(387.19660278,408.55210297)
\curveto(387.18660221,408.6020995)(387.18160221,408.64709945)(387.18160278,408.68710297)
\lineto(387.18160278,408.82210297)
\curveto(387.16160223,408.88209922)(387.15160224,408.96709913)(387.15160278,409.07710297)
\curveto(387.16160223,409.18709891)(387.17660222,409.27209883)(387.19660278,409.33210297)
\lineto(387.19660278,409.43710297)
\curveto(387.20660219,409.48709861)(387.20660219,409.53709856)(387.19660278,409.58710297)
\curveto(387.1966022,409.64709845)(387.20660219,409.7020984)(387.22660278,409.75210297)
\curveto(387.23660216,409.8020983)(387.24160215,409.84709825)(387.24160278,409.88710297)
\curveto(387.24160215,409.93709816)(387.25160214,409.98709811)(387.27160278,410.03710297)
\curveto(387.31160208,410.16709793)(387.34660205,410.29209781)(387.37660278,410.41210297)
\curveto(387.40660199,410.54209756)(387.44660195,410.66709743)(387.49660278,410.78710297)
\curveto(387.67660172,411.1970969)(387.8916015,411.53709656)(388.14160278,411.80710297)
\curveto(388.391601,412.08709601)(388.6966007,412.34209576)(389.05660278,412.57210297)
\curveto(389.15660024,412.62209548)(389.26160013,412.66709543)(389.37160278,412.70710297)
\curveto(389.48159991,412.74709535)(389.5915998,412.79209531)(389.70160278,412.84210297)
\curveto(389.83159956,412.89209521)(389.96659943,412.92709517)(390.10660278,412.94710297)
\curveto(390.24659915,412.96709513)(390.391599,412.9970951)(390.54160278,413.03710297)
\curveto(390.62159877,413.04709505)(390.6965987,413.05209505)(390.76660278,413.05210297)
\curveto(390.83659856,413.05209505)(390.90659849,413.05709504)(390.97660278,413.06710297)
\curveto(391.55659784,413.07709502)(392.05659734,413.01709508)(392.47660278,412.88710297)
\curveto(392.90659649,412.75709534)(393.28659611,412.57709552)(393.61660278,412.34710297)
\curveto(393.72659567,412.26709583)(393.83659556,412.17709592)(393.94660278,412.07710297)
\curveto(394.06659533,411.98709611)(394.16659523,411.88709621)(394.24660278,411.77710297)
\curveto(394.32659507,411.67709642)(394.396595,411.57709652)(394.45660278,411.47710297)
\curveto(394.52659487,411.37709672)(394.5965948,411.27209683)(394.66660278,411.16210297)
\curveto(394.73659466,411.05209705)(394.7915946,410.93209717)(394.83160278,410.80210297)
\curveto(394.87159452,410.68209742)(394.91659448,410.55209755)(394.96660278,410.41210297)
\curveto(394.9965944,410.33209777)(395.02159437,410.24709785)(395.04160278,410.15710297)
\lineto(395.10160278,409.88710297)
\curveto(395.11159428,409.84709825)(395.11659428,409.80709829)(395.11660278,409.76710297)
\curveto(395.11659428,409.72709837)(395.12159427,409.68709841)(395.13160278,409.64710297)
\curveto(395.15159424,409.5970985)(395.15659424,409.54209856)(395.14660278,409.48210297)
\curveto(395.13659426,409.42209868)(395.14159425,409.36709873)(395.16160278,409.31710297)
\moveto(393.06160278,408.77710297)
\curveto(393.07159632,408.82709927)(393.07659632,408.8970992)(393.07660278,408.98710297)
\curveto(393.07659632,409.08709901)(393.07159632,409.16209894)(393.06160278,409.21210297)
\lineto(393.06160278,409.33210297)
\curveto(393.04159635,409.38209872)(393.03159636,409.43709866)(393.03160278,409.49710297)
\curveto(393.03159636,409.55709854)(393.02659637,409.61209849)(393.01660278,409.66210297)
\curveto(393.01659638,409.7020984)(393.01159638,409.73209837)(393.00160278,409.75210297)
\lineto(392.94160278,409.99210297)
\curveto(392.93159646,410.08209802)(392.91159648,410.16709793)(392.88160278,410.24710297)
\curveto(392.77159662,410.50709759)(392.64159675,410.72709737)(392.49160278,410.90710297)
\curveto(392.34159705,411.097097)(392.14159725,411.24709685)(391.89160278,411.35710297)
\curveto(391.83159756,411.37709672)(391.77159762,411.39209671)(391.71160278,411.40210297)
\curveto(391.65159774,411.42209668)(391.58659781,411.44209666)(391.51660278,411.46210297)
\curveto(391.43659796,411.48209662)(391.35159804,411.48709661)(391.26160278,411.47710297)
\lineto(390.99160278,411.47710297)
\curveto(390.96159843,411.45709664)(390.92659847,411.44709665)(390.88660278,411.44710297)
\curveto(390.84659855,411.45709664)(390.81159858,411.45709664)(390.78160278,411.44710297)
\lineto(390.57160278,411.38710297)
\curveto(390.51159888,411.37709672)(390.45659894,411.35709674)(390.40660278,411.32710297)
\curveto(390.15659924,411.21709688)(389.95159944,411.05709704)(389.79160278,410.84710297)
\curveto(389.64159975,410.64709745)(389.52159987,410.41209769)(389.43160278,410.14210297)
\curveto(389.40159999,410.04209806)(389.37660002,409.93709816)(389.35660278,409.82710297)
\curveto(389.34660005,409.71709838)(389.33160006,409.60709849)(389.31160278,409.49710297)
\curveto(389.30160009,409.44709865)(389.2966001,409.3970987)(389.29660278,409.34710297)
\lineto(389.29660278,409.19710297)
\curveto(389.27660012,409.12709897)(389.26660013,409.02209908)(389.26660278,408.88210297)
\curveto(389.27660012,408.74209936)(389.2916001,408.63709946)(389.31160278,408.56710297)
\lineto(389.31160278,408.43210297)
\curveto(389.33160006,408.35209975)(389.34660005,408.27209983)(389.35660278,408.19210297)
\curveto(389.36660003,408.12209998)(389.38160001,408.04710005)(389.40160278,407.96710297)
\curveto(389.50159989,407.66710043)(389.60659979,407.42210068)(389.71660278,407.23210297)
\curveto(389.83659956,407.05210105)(390.02159937,406.88710121)(390.27160278,406.73710297)
\curveto(390.34159905,406.68710141)(390.41659898,406.64710145)(390.49660278,406.61710297)
\curveto(390.58659881,406.58710151)(390.67659872,406.56210154)(390.76660278,406.54210297)
\curveto(390.80659859,406.53210157)(390.84159855,406.52710157)(390.87160278,406.52710297)
\curveto(390.90159849,406.53710156)(390.93659846,406.53710156)(390.97660278,406.52710297)
\lineto(391.09660278,406.49710297)
\curveto(391.14659825,406.4971016)(391.1915982,406.5021016)(391.23160278,406.51210297)
\lineto(391.35160278,406.51210297)
\curveto(391.43159796,406.53210157)(391.51159788,406.54710155)(391.59160278,406.55710297)
\curveto(391.67159772,406.56710153)(391.74659765,406.58710151)(391.81660278,406.61710297)
\curveto(392.07659732,406.71710138)(392.28659711,406.85210125)(392.44660278,407.02210297)
\curveto(392.60659679,407.19210091)(392.74159665,407.4021007)(392.85160278,407.65210297)
\curveto(392.8915965,407.75210035)(392.92159647,407.85210025)(392.94160278,407.95210297)
\curveto(392.96159643,408.05210005)(392.98659641,408.15709994)(393.01660278,408.26710297)
\curveto(393.02659637,408.30709979)(393.03159636,408.34209976)(393.03160278,408.37210297)
\curveto(393.03159636,408.41209969)(393.03659636,408.45209965)(393.04660278,408.49210297)
\lineto(393.04660278,408.62710297)
\curveto(393.04659635,408.67709942)(393.05159634,408.72709937)(393.06160278,408.77710297)
}
}
{
\newrgbcolor{curcolor}{0 0 0}
\pscustom[linestyle=none,fillstyle=solid,fillcolor=curcolor]
{
\newpath
\moveto(12.8408255,170.88068481)
\curveto(12.82083646,171.86067612)(12.9808363,172.65067533)(13.3208255,173.25068481)
\curveto(13.65083563,173.85067413)(14.11083517,174.27567371)(14.7008255,174.52568481)
\curveto(14.86083442,174.59567339)(15.01583426,174.64567334)(15.1658255,174.67568481)
\curveto(15.30583397,174.71567327)(15.4758338,174.74567324)(15.6758255,174.76568481)
\curveto(15.72583355,174.77567321)(15.79583348,174.7806732)(15.8858255,174.78068481)
\curveto(15.96583331,174.79067319)(16.04083324,174.77067321)(16.1108255,174.72068481)
\curveto(16.17083311,174.69067329)(16.21083307,174.64567334)(16.2308255,174.58568481)
\curveto(16.24083304,174.52567346)(16.25583302,174.46567352)(16.2758255,174.40568481)
\lineto(16.2758255,174.25568481)
\curveto(16.28583299,174.22567376)(16.29083299,174.1856738)(16.2908255,174.13568481)
\lineto(16.2908255,174.01568481)
\curveto(16.29083299,173.87567411)(16.28583299,173.74567424)(16.2758255,173.62568481)
\curveto(16.25583302,173.51567447)(16.20583307,173.44567454)(16.1258255,173.41568481)
\curveto(16.02583325,173.36567462)(15.91083337,173.33067465)(15.7808255,173.31068481)
\curveto(15.65083363,173.30067468)(15.53083375,173.27067471)(15.4208255,173.22068481)
\curveto(15.20083408,173.14067484)(15.01083427,173.03067495)(14.8508255,172.89068481)
\curveto(14.69083459,172.76067522)(14.54083474,172.60067538)(14.4008255,172.41068481)
\curveto(14.32083496,172.31067567)(14.26083502,172.19067579)(14.2208255,172.05068481)
\curveto(14.1808351,171.91067607)(14.14083514,171.77067621)(14.1008255,171.63068481)
\curveto(14.07083521,171.53067645)(14.05083523,171.41067657)(14.0408255,171.27068481)
\curveto(14.02083526,171.13067685)(14.01083527,170.980677)(14.0108255,170.82068481)
\curveto(14.01083527,170.67067731)(14.02083526,170.52067746)(14.0408255,170.37068481)
\curveto(14.05083523,170.23067775)(14.07083521,170.10567788)(14.1008255,169.99568481)
\lineto(14.1608255,169.69568481)
\curveto(14.1808351,169.60567838)(14.21083507,169.51067847)(14.2508255,169.41068481)
\curveto(14.60083468,168.50067948)(15.20083408,167.77568021)(16.0508255,167.23568481)
\curveto(16.19083309,167.13568085)(16.34083294,167.04568094)(16.5008255,166.96568481)
\curveto(16.65083263,166.89568109)(16.80583247,166.82068116)(16.9658255,166.74068481)
\curveto(17.02583225,166.71068127)(17.09083219,166.6856813)(17.1608255,166.66568481)
\curveto(17.22083206,166.65568133)(17.280832,166.63568135)(17.3408255,166.60568481)
\curveto(17.3808319,166.5856814)(17.41583186,166.57068141)(17.4458255,166.56068481)
\curveto(17.4758318,166.56068142)(17.51083177,166.55068143)(17.5508255,166.53068481)
\curveto(17.66083162,166.49068149)(17.7758315,166.45568153)(17.8958255,166.42568481)
\curveto(18.00583127,166.39568159)(18.12083116,166.37068161)(18.2408255,166.35068481)
\curveto(18.31083097,166.33068165)(18.38583089,166.31068167)(18.4658255,166.29068481)
\curveto(18.53583074,166.27068171)(18.60083068,166.26068172)(18.6608255,166.26068481)
\lineto(18.8108255,166.23068481)
\curveto(18.8808304,166.24068174)(18.95083033,166.23568175)(19.0208255,166.21568481)
\curveto(19.09083019,166.19568179)(19.16083012,166.19068179)(19.2308255,166.20068481)
\curveto(19.29082999,166.20068178)(19.35082993,166.19068179)(19.4108255,166.17068481)
\curveto(19.47082981,166.16068182)(19.52582975,166.16068182)(19.5758255,166.17068481)
\lineto(19.9658255,166.17068481)
\curveto(20.08582919,166.17068181)(20.20582907,166.1806818)(20.3258255,166.20068481)
\curveto(20.82582845,166.27068171)(21.25582802,166.40568158)(21.6158255,166.60568481)
\curveto(21.96582731,166.80568118)(22.25582702,167.10068088)(22.4858255,167.49068481)
\curveto(22.55582672,167.62068036)(22.61082667,167.75068023)(22.6508255,167.88068481)
\curveto(22.69082659,168.02067996)(22.73582654,168.17067981)(22.7858255,168.33068481)
\curveto(22.80582647,168.40067958)(22.81582646,168.46567952)(22.8158255,168.52568481)
\curveto(22.80582647,168.5856794)(22.81082647,168.65567933)(22.8308255,168.73568481)
\curveto(22.84082644,168.77567921)(22.85082643,168.86067912)(22.8608255,168.99068481)
\curveto(22.86082642,169.12067886)(22.85082643,169.22067876)(22.8308255,169.29068481)
\lineto(22.8308255,169.39568481)
\curveto(22.82082646,169.43567855)(22.82082646,169.47567851)(22.8308255,169.51568481)
\curveto(22.83082645,169.55567843)(22.82082646,169.59567839)(22.8008255,169.63568481)
\curveto(22.7808265,169.73567825)(22.76582651,169.83567815)(22.7558255,169.93568481)
\curveto(22.73582654,170.03567795)(22.70582657,170.13567785)(22.6658255,170.23568481)
\curveto(22.34582693,171.07567691)(21.82082746,171.73067625)(21.0908255,172.20068481)
\curveto(20.91082837,172.31067567)(20.69582858,172.42567556)(20.4458255,172.54568481)
\curveto(20.35582892,172.5856754)(20.26582901,172.62067536)(20.1758255,172.65068481)
\curveto(20.0758292,172.69067529)(19.98582929,172.74567524)(19.9058255,172.81568481)
\curveto(19.82582945,172.87567511)(19.7808295,172.96067502)(19.7708255,173.07068481)
\curveto(19.76082952,173.19067479)(19.75582952,173.31567467)(19.7558255,173.44568481)
\lineto(19.7558255,173.59568481)
\curveto(19.75582952,173.64567434)(19.76082952,173.6856743)(19.7708255,173.71568481)
\lineto(19.7708255,173.82068481)
\lineto(19.8008255,173.91068481)
\curveto(19.80082948,173.95067403)(19.81082947,173.980674)(19.8308255,174.00068481)
\curveto(19.87082941,174.06067392)(19.94582933,174.0856739)(20.0558255,174.07568481)
\curveto(20.15582912,174.06567392)(20.25582902,174.03567395)(20.3558255,173.98568481)
\curveto(20.58582869,173.87567411)(20.80582847,173.77567421)(21.0158255,173.68568481)
\curveto(21.22582805,173.59567439)(21.42582785,173.4806745)(21.6158255,173.34068481)
\curveto(21.74582753,173.23067475)(21.87082741,173.13067485)(21.9908255,173.04068481)
\curveto(22.11082717,172.95067503)(22.23082705,172.85567513)(22.3508255,172.75568481)
\curveto(22.61082667,172.52567546)(22.85082643,172.25567573)(23.0708255,171.94568481)
\curveto(23.280826,171.63567635)(23.45582582,171.31067667)(23.5958255,170.97068481)
\curveto(23.64582563,170.85067713)(23.68582559,170.73567725)(23.7158255,170.62568481)
\curveto(23.74582553,170.51567747)(23.7808255,170.40067758)(23.8208255,170.28068481)
\curveto(23.86082542,170.17067781)(23.88582539,170.05567793)(23.8958255,169.93568481)
\lineto(23.9558255,169.57568481)
\curveto(23.9758253,169.51567847)(23.9808253,169.46567852)(23.9708255,169.42568481)
\curveto(23.97082531,169.3856786)(23.9758253,169.34567864)(23.9858255,169.30568481)
\curveto(23.99582528,169.24567874)(24.00082528,169.1856788)(24.0008255,169.12568481)
\curveto(24.00082528,169.06567892)(24.00582527,169.00067898)(24.0158255,168.93068481)
\curveto(24.02582525,168.90067908)(24.02582525,168.83067915)(24.0158255,168.72068481)
\curveto(24.01582526,168.62067936)(24.01082527,168.55567943)(24.0008255,168.52568481)
\curveto(23.99082529,168.47567951)(23.98582529,168.42567956)(23.9858255,168.37568481)
\curveto(23.99582528,168.33567965)(23.99582528,168.29067969)(23.9858255,168.24068481)
\lineto(23.9858255,168.09068481)
\curveto(23.96582531,168.02067996)(23.95082533,167.95068003)(23.9408255,167.88068481)
\curveto(23.94082534,167.81068017)(23.93082535,167.73568025)(23.9108255,167.65568481)
\curveto(23.89082539,167.57568041)(23.87082541,167.49068049)(23.8508255,167.40068481)
\curveto(23.84082544,167.31068067)(23.82082546,167.23068075)(23.7908255,167.16068481)
\curveto(23.73082555,166.96068102)(23.65582562,166.7856812)(23.5658255,166.63568481)
\curveto(23.29582598,166.05568193)(22.91082637,165.62068236)(22.4108255,165.33068481)
\curveto(21.91082737,165.04068294)(21.32082796,164.85068313)(20.6408255,164.76068481)
\curveto(20.52082876,164.74068324)(20.39582888,164.73068325)(20.2658255,164.73068481)
\lineto(19.8608255,164.73068481)
\curveto(19.81082947,164.72068326)(19.76582951,164.72068326)(19.7258255,164.73068481)
\curveto(19.68582959,164.75068323)(19.64082964,164.76068322)(19.5908255,164.76068481)
\curveto(19.50082978,164.76068322)(19.40582987,164.76568322)(19.3058255,164.77568481)
\curveto(19.20583007,164.79568319)(19.11083017,164.80068318)(19.0208255,164.79068481)
\lineto(18.7358255,164.85068481)
\curveto(18.68583059,164.84068314)(18.60083068,164.84568314)(18.4808255,164.86568481)
\curveto(18.36083092,164.89568309)(18.275831,164.92568306)(18.2258255,164.95568481)
\curveto(18.19583108,164.97568301)(18.16583111,164.980683)(18.1358255,164.97068481)
\curveto(18.09583118,164.97068301)(18.06583121,164.97568301)(18.0458255,164.98568481)
\lineto(17.9108255,165.01568481)
\curveto(17.83083145,165.04568294)(17.75083153,165.07068291)(17.6708255,165.09068481)
\curveto(17.5808317,165.11068287)(17.49583178,165.14068284)(17.4158255,165.18068481)
\curveto(17.35583192,165.21068277)(17.29583198,165.23068275)(17.2358255,165.24068481)
\curveto(17.16583211,165.26068272)(17.09583218,165.2856827)(17.0258255,165.31568481)
\curveto(16.85583242,165.39568259)(16.69083259,165.46568252)(16.5308255,165.52568481)
\curveto(16.37083291,165.59568239)(16.22083306,165.67568231)(16.0808255,165.76568481)
\curveto(15.95083333,165.83568215)(15.82083346,165.91068207)(15.6908255,165.99068481)
\curveto(15.55083373,166.0806819)(15.42583385,166.17068181)(15.3158255,166.26068481)
\curveto(15.05583422,166.46068152)(14.82083446,166.65568133)(14.6108255,166.84568481)
\curveto(14.56083472,166.8856811)(14.52083476,166.93068105)(14.4908255,166.98068481)
\curveto(14.45083483,167.04068094)(14.40583487,167.09068089)(14.3558255,167.13068481)
\curveto(14.275835,167.20068078)(14.20583507,167.27568071)(14.1458255,167.35568481)
\lineto(13.9658255,167.59568481)
\curveto(13.91583536,167.66568032)(13.87083541,167.73068025)(13.8308255,167.79068481)
\curveto(13.7808355,167.86068012)(13.73083555,167.93568005)(13.6808255,168.01568481)
\curveto(13.57083571,168.1856798)(13.4758358,168.36067962)(13.3958255,168.54068481)
\curveto(13.31583596,168.72067926)(13.23583604,168.91067907)(13.1558255,169.11068481)
\curveto(13.10583617,169.23067875)(13.07083621,169.35567863)(13.0508255,169.48568481)
\curveto(13.02083626,169.61567837)(12.98583629,169.74067824)(12.9458255,169.86068481)
\curveto(12.93583634,169.94067804)(12.92583635,170.01067797)(12.9158255,170.07068481)
\lineto(12.8858255,170.25068481)
\curveto(12.8758364,170.33067765)(12.87083641,170.41067757)(12.8708255,170.49068481)
\lineto(12.8408255,170.73068481)
\curveto(12.83083645,170.75067723)(12.83083645,170.77567721)(12.8408255,170.80568481)
\curveto(12.85083643,170.83567715)(12.85083643,170.86067712)(12.8408255,170.88068481)
}
}
{
\newrgbcolor{curcolor}{0 0 0}
\pscustom[linestyle=none,fillstyle=solid,fillcolor=curcolor]
{
\newpath
\moveto(23.2058255,181.93052856)
\curveto(23.36582591,181.92052065)(23.50082578,181.8755207)(23.6108255,181.79552856)
\curveto(23.71082557,181.71552086)(23.78582549,181.62052095)(23.8358255,181.51052856)
\curveto(23.85582542,181.46052111)(23.86582541,181.40552117)(23.8658255,181.34552856)
\curveto(23.86582541,181.29552128)(23.8758254,181.23552134)(23.8958255,181.16552856)
\curveto(23.94582533,180.93552164)(23.93082535,180.72052185)(23.8508255,180.52052856)
\curveto(23.7808255,180.32052225)(23.69082559,180.19552238)(23.5808255,180.14552856)
\curveto(23.51082577,180.10552247)(23.43082585,180.0755225)(23.3408255,180.05552856)
\curveto(23.24082604,180.03552254)(23.16082612,180.00052257)(23.1008255,179.95052856)
\lineto(23.0408255,179.89052856)
\curveto(23.02082626,179.8705227)(23.01582626,179.84052273)(23.0258255,179.80052856)
\curveto(23.05582622,179.68052289)(23.11082617,179.56552301)(23.1908255,179.45552856)
\curveto(23.27082601,179.34552323)(23.34082594,179.24052333)(23.4008255,179.14052856)
\curveto(23.4808258,178.99052358)(23.55582572,178.83552374)(23.6258255,178.67552856)
\curveto(23.68582559,178.51552406)(23.74082554,178.34552423)(23.7908255,178.16552856)
\curveto(23.82082546,178.05552452)(23.84082544,177.94052463)(23.8508255,177.82052856)
\curveto(23.86082542,177.71052486)(23.8758254,177.59552498)(23.8958255,177.47552856)
\curveto(23.90582537,177.42552515)(23.91082537,177.38052519)(23.9108255,177.34052856)
\lineto(23.9108255,177.23552856)
\curveto(23.93082535,177.12552545)(23.93082535,177.02052555)(23.9108255,176.92052856)
\lineto(23.9108255,176.78552856)
\curveto(23.90082538,176.73552584)(23.89582538,176.68552589)(23.8958255,176.63552856)
\curveto(23.89582538,176.58552599)(23.88582539,176.54552603)(23.8658255,176.51552856)
\curveto(23.85582542,176.4755261)(23.85082543,176.44052613)(23.8508255,176.41052856)
\curveto(23.86082542,176.39052618)(23.86082542,176.36552621)(23.8508255,176.33552856)
\lineto(23.7908255,176.09552856)
\curveto(23.7808255,176.02552655)(23.76082552,175.96052661)(23.7308255,175.90052856)
\curveto(23.60082568,175.62052695)(23.45582582,175.40552717)(23.2958255,175.25552856)
\curveto(23.12582615,175.10552747)(22.89082639,175.00052757)(22.5908255,174.94052856)
\curveto(22.37082691,174.89052768)(22.10582717,174.89552768)(21.7958255,174.95552856)
\lineto(21.4808255,175.03052856)
\curveto(21.43082785,175.05052752)(21.3808279,175.06552751)(21.3308255,175.07552856)
\lineto(21.1508255,175.13552856)
\lineto(20.8208255,175.31552856)
\curveto(20.71082857,175.38552719)(20.61082867,175.45552712)(20.5208255,175.52552856)
\curveto(20.23082905,175.76552681)(20.01582926,176.05552652)(19.8758255,176.39552856)
\curveto(19.73582954,176.73552584)(19.61082967,177.10052547)(19.5008255,177.49052856)
\curveto(19.46082982,177.64052493)(19.43082985,177.79052478)(19.4108255,177.94052856)
\curveto(19.39082989,178.10052447)(19.36582991,178.25552432)(19.3358255,178.40552856)
\curveto(19.31582996,178.48552409)(19.30582997,178.55552402)(19.3058255,178.61552856)
\curveto(19.30582997,178.68552389)(19.29582998,178.76052381)(19.2758255,178.84052856)
\curveto(19.25583002,178.91052366)(19.24583003,178.98052359)(19.2458255,179.05052856)
\curveto(19.23583004,179.13052344)(19.22083006,179.21052336)(19.2008255,179.29052856)
\curveto(19.14083014,179.55052302)(19.09083019,179.79552278)(19.0508255,180.02552856)
\curveto(19.00083028,180.25552232)(18.88583039,180.45552212)(18.7058255,180.62552856)
\curveto(18.62583065,180.69552188)(18.52583075,180.76052181)(18.4058255,180.82052856)
\curveto(18.275831,180.89052168)(18.13583114,180.92052165)(17.9858255,180.91052856)
\curveto(17.74583153,180.90052167)(17.55583172,180.85052172)(17.4158255,180.76052856)
\curveto(17.275832,180.68052189)(17.16583211,180.54052203)(17.0858255,180.34052856)
\curveto(17.03583224,180.23052234)(17.00083228,180.09552248)(16.9808255,179.93552856)
\curveto(16.96083232,179.7755228)(16.95083233,179.60552297)(16.9508255,179.42552856)
\curveto(16.95083233,179.24552333)(16.96083232,179.06552351)(16.9808255,178.88552856)
\curveto(17.00083228,178.71552386)(17.03083225,178.56552401)(17.0708255,178.43552856)
\curveto(17.13083215,178.25552432)(17.21583206,178.0755245)(17.3258255,177.89552856)
\curveto(17.38583189,177.80552477)(17.46583181,177.71552486)(17.5658255,177.62552856)
\curveto(17.65583162,177.54552503)(17.75583152,177.4705251)(17.8658255,177.40052856)
\curveto(17.94583133,177.35052522)(18.03083125,177.30552527)(18.1208255,177.26552856)
\curveto(18.21083107,177.22552535)(18.280831,177.16552541)(18.3308255,177.08552856)
\curveto(18.36083092,177.03552554)(18.38583089,176.96052561)(18.4058255,176.86052856)
\curveto(18.41583086,176.76052581)(18.42083086,176.66052591)(18.4208255,176.56052856)
\curveto(18.42083086,176.46052611)(18.41583086,176.36552621)(18.4058255,176.27552856)
\curveto(18.38583089,176.18552639)(18.36083092,176.12552645)(18.3308255,176.09552856)
\curveto(18.30083098,176.05552652)(18.25083103,176.03052654)(18.1808255,176.02052856)
\curveto(18.11083117,176.02052655)(18.03583124,176.04052653)(17.9558255,176.08052856)
\curveto(17.82583145,176.13052644)(17.70583157,176.18552639)(17.5958255,176.24552856)
\curveto(17.4758318,176.30552627)(17.36083192,176.3705262)(17.2508255,176.44052856)
\curveto(16.90083238,176.70052587)(16.63083265,176.99552558)(16.4408255,177.32552856)
\curveto(16.24083304,177.65552492)(16.0808332,178.04552453)(15.9608255,178.49552856)
\curveto(15.94083334,178.60552397)(15.92583335,178.71052386)(15.9158255,178.81052856)
\curveto(15.90583337,178.92052365)(15.89083339,179.03052354)(15.8708255,179.14052856)
\curveto(15.86083342,179.19052338)(15.86083342,179.25552332)(15.8708255,179.33552856)
\curveto(15.87083341,179.42552315)(15.86083342,179.48552309)(15.8408255,179.51552856)
\curveto(15.83083345,180.21552236)(15.91083337,180.80552177)(16.0808255,181.28552856)
\curveto(16.25083303,181.7755208)(16.5758327,182.08052049)(17.0558255,182.20052856)
\curveto(17.25583202,182.25052032)(17.49083179,182.25552032)(17.7608255,182.21552856)
\curveto(18.02083126,182.1755204)(18.29583098,182.12552045)(18.5858255,182.06552856)
\lineto(21.9008255,181.40552856)
\curveto(22.04082724,181.3755212)(22.1758271,181.35052122)(22.3058255,181.33052856)
\curveto(22.43582684,181.32052125)(22.54082674,181.33052124)(22.6208255,181.36052856)
\curveto(22.69082659,181.40052117)(22.74082654,181.45552112)(22.7708255,181.52552856)
\curveto(22.81082647,181.61552096)(22.84082644,181.69552088)(22.8608255,181.76552856)
\curveto(22.87082641,181.84552073)(22.91582636,181.89552068)(22.9958255,181.91552856)
\curveto(23.02582625,181.93552064)(23.05582622,181.94052063)(23.0858255,181.93052856)
\lineto(23.2058255,181.93052856)
\moveto(21.5408255,180.11552856)
\curveto(21.40082788,180.20552237)(21.24082804,180.2705223)(21.0608255,180.31052856)
\curveto(20.87082841,180.35052222)(20.6758286,180.39052218)(20.4758255,180.43052856)
\curveto(20.36582891,180.45052212)(20.26582901,180.46552211)(20.1758255,180.47552856)
\curveto(20.08582919,180.48552209)(20.01582926,180.46052211)(19.9658255,180.40052856)
\curveto(19.94582933,180.3705222)(19.93582934,180.30052227)(19.9358255,180.19052856)
\curveto(19.95582932,180.1705224)(19.96582931,180.13552244)(19.9658255,180.08552856)
\curveto(19.96582931,180.03552254)(19.9758293,179.98552259)(19.9958255,179.93552856)
\curveto(20.01582926,179.85552272)(20.03582924,179.76052281)(20.0558255,179.65052856)
\lineto(20.1158255,179.35052856)
\curveto(20.11582916,179.32052325)(20.12082916,179.28552329)(20.1308255,179.24552856)
\lineto(20.1308255,179.14052856)
\curveto(20.17082911,178.98052359)(20.19582908,178.81052376)(20.2058255,178.63052856)
\curveto(20.20582907,178.46052411)(20.22582905,178.29552428)(20.2658255,178.13552856)
\curveto(20.28582899,178.04552453)(20.30582897,177.96552461)(20.3258255,177.89552856)
\curveto(20.33582894,177.83552474)(20.35082893,177.76052481)(20.3708255,177.67052856)
\curveto(20.42082886,177.50052507)(20.48582879,177.33552524)(20.5658255,177.17552856)
\curveto(20.63582864,177.02552555)(20.72582855,176.89052568)(20.8358255,176.77052856)
\curveto(20.94582833,176.65052592)(21.0808282,176.55052602)(21.2408255,176.47052856)
\curveto(21.39082789,176.39052618)(21.5758277,176.33052624)(21.7958255,176.29052856)
\curveto(21.89582738,176.2705263)(21.99082729,176.2705263)(22.0808255,176.29052856)
\curveto(22.16082712,176.31052626)(22.23582704,176.34052623)(22.3058255,176.38052856)
\curveto(22.41582686,176.43052614)(22.51082677,176.51052606)(22.5908255,176.62052856)
\curveto(22.66082662,176.74052583)(22.72082656,176.8705257)(22.7708255,177.01052856)
\curveto(22.7808265,177.06052551)(22.78582649,177.11052546)(22.7858255,177.16052856)
\curveto(22.78582649,177.21052536)(22.79082649,177.26052531)(22.8008255,177.31052856)
\curveto(22.82082646,177.38052519)(22.83582644,177.46552511)(22.8458255,177.56552856)
\curveto(22.84582643,177.66552491)(22.83582644,177.75552482)(22.8158255,177.83552856)
\curveto(22.79582648,177.89552468)(22.79082649,177.95552462)(22.8008255,178.01552856)
\curveto(22.80082648,178.0755245)(22.79082649,178.13552444)(22.7708255,178.19552856)
\curveto(22.75082653,178.28552429)(22.73582654,178.36552421)(22.7258255,178.43552856)
\curveto(22.71582656,178.51552406)(22.69582658,178.59552398)(22.6658255,178.67552856)
\curveto(22.54582673,178.98552359)(22.40082688,179.26052331)(22.2308255,179.50052856)
\curveto(22.06082722,179.74052283)(21.83082745,179.94552263)(21.5408255,180.11552856)
}
}
{
\newrgbcolor{curcolor}{0 0 0}
\pscustom[linestyle=none,fillstyle=solid,fillcolor=curcolor]
{
\newpath
\moveto(15.8858255,188.24716919)
\curveto(15.86583341,188.88716237)(15.95083333,189.37716188)(16.1408255,189.71716919)
\curveto(16.33083295,190.0571612)(16.61583266,190.30216095)(16.9958255,190.45216919)
\curveto(17.09583218,190.49216076)(17.20583207,190.51716074)(17.3258255,190.52716919)
\curveto(17.43583184,190.54716071)(17.55083173,190.5571607)(17.6708255,190.55716919)
\curveto(17.86083142,190.57716068)(18.06583121,190.56716069)(18.2858255,190.52716919)
\curveto(18.50583077,190.49716076)(18.73083055,190.4571608)(18.9608255,190.40716919)
\lineto(20.5658255,190.09216919)
\lineto(22.9058255,189.62716919)
\lineto(23.4158255,189.50716919)
\curveto(23.58582569,189.46716179)(23.69582558,189.37716188)(23.7458255,189.23716919)
\curveto(23.76582551,189.18716207)(23.7758255,189.13216212)(23.7758255,189.07216919)
\curveto(23.78582549,189.02216223)(23.79082549,188.96716229)(23.7908255,188.90716919)
\curveto(23.79082549,188.77716248)(23.78582549,188.6521626)(23.7758255,188.53216919)
\curveto(23.7758255,188.41216284)(23.73582554,188.33716292)(23.6558255,188.30716919)
\curveto(23.58582569,188.26716299)(23.49582578,188.257163)(23.3858255,188.27716919)
\curveto(23.275826,188.29716296)(23.16582611,188.32216293)(23.0558255,188.35216919)
\lineto(21.7658255,188.60716919)
\lineto(19.3208255,189.08716919)
\curveto(19.05083023,189.14716211)(18.78583049,189.19716206)(18.5258255,189.23716919)
\curveto(18.25583102,189.27716198)(18.02583125,189.27716198)(17.8358255,189.23716919)
\curveto(17.63583164,189.19716206)(17.4758318,189.10716215)(17.3558255,188.96716919)
\curveto(17.22583205,188.83716242)(17.12583215,188.67716258)(17.0558255,188.48716919)
\curveto(17.03583224,188.42716283)(17.02583225,188.36216289)(17.0258255,188.29216919)
\curveto(17.01583226,188.23216302)(17.00083228,188.17716308)(16.9808255,188.12716919)
\curveto(16.97083231,188.07716318)(16.97083231,187.99716326)(16.9808255,187.88716919)
\curveto(16.9808323,187.78716347)(16.98583229,187.71216354)(16.9958255,187.66216919)
\curveto(17.01583226,187.62216363)(17.02583225,187.58716367)(17.0258255,187.55716919)
\curveto(17.01583226,187.52716373)(17.01583226,187.49216376)(17.0258255,187.45216919)
\curveto(17.05583222,187.31216394)(17.09083219,187.18216407)(17.1308255,187.06216919)
\curveto(17.16083212,186.94216431)(17.20583207,186.82716443)(17.2658255,186.71716919)
\curveto(17.28583199,186.66716459)(17.30583197,186.62716463)(17.3258255,186.59716919)
\curveto(17.34583193,186.56716469)(17.36583191,186.52716473)(17.3858255,186.47716919)
\curveto(17.63583164,186.07716518)(18.01083127,185.74716551)(18.5108255,185.48716919)
\curveto(18.59083069,185.44716581)(18.6758306,185.41216584)(18.7658255,185.38216919)
\lineto(19.0058255,185.29216919)
\curveto(19.05583022,185.26216599)(19.10583017,185.24716601)(19.1558255,185.24716919)
\curveto(19.19583008,185.24716601)(19.23583004,185.23216602)(19.2758255,185.20216919)
\lineto(19.5908255,185.14216919)
\curveto(19.62082966,185.12216613)(19.65582962,185.11216614)(19.6958255,185.11216919)
\curveto(19.73582954,185.11216614)(19.7808295,185.10716615)(19.8308255,185.09716919)
\lineto(20.2808255,185.00716919)
\lineto(21.7208255,184.70716919)
\lineto(23.0408255,184.45216919)
\curveto(23.15082613,184.43216682)(23.26582601,184.40716685)(23.3858255,184.37716919)
\curveto(23.49582578,184.3571669)(23.58582569,184.31716694)(23.6558255,184.25716919)
\curveto(23.73582554,184.18716707)(23.7758255,184.08716717)(23.7758255,183.95716919)
\curveto(23.78582549,183.83716742)(23.79082549,183.71216754)(23.7908255,183.58216919)
\curveto(23.79082549,183.50216775)(23.78582549,183.42716783)(23.7758255,183.35716919)
\curveto(23.76582551,183.28716797)(23.74082554,183.23216802)(23.7008255,183.19216919)
\curveto(23.65082563,183.12216813)(23.55582572,183.10216815)(23.4158255,183.13216919)
\curveto(23.275826,183.16216809)(23.14082614,183.18716807)(23.0108255,183.20716919)
\lineto(21.2408255,183.56716919)
\lineto(17.6108255,184.28716919)
\lineto(16.6958255,184.46716919)
\lineto(16.4258255,184.52716919)
\curveto(16.33583294,184.54716671)(16.26583301,184.58216667)(16.2158255,184.63216919)
\curveto(16.15583312,184.67216658)(16.11583316,184.72716653)(16.0958255,184.79716919)
\curveto(16.08583319,184.84716641)(16.0758332,184.90716635)(16.0658255,184.97716919)
\curveto(16.05583322,185.0571662)(16.05083323,185.13716612)(16.0508255,185.21716919)
\curveto(16.05083323,185.29716596)(16.05583322,185.37216588)(16.0658255,185.44216919)
\curveto(16.0758332,185.52216573)(16.09083319,185.57216568)(16.1108255,185.59216919)
\curveto(16.1808331,185.69216556)(16.27083301,185.72716553)(16.3808255,185.69716919)
\curveto(16.4808328,185.66716559)(16.59583268,185.6571656)(16.7258255,185.66716919)
\curveto(16.78583249,185.67716558)(16.83583244,185.70716555)(16.8758255,185.75716919)
\curveto(16.88583239,185.87716538)(16.84083244,185.98216527)(16.7408255,186.07216919)
\curveto(16.64083264,186.17216508)(16.56083272,186.26716499)(16.5008255,186.35716919)
\curveto(16.40083288,186.51716474)(16.31083297,186.67716458)(16.2308255,186.83716919)
\curveto(16.14083314,186.99716426)(16.06583321,187.18216407)(16.0058255,187.39216919)
\curveto(15.9758333,187.47216378)(15.95583332,187.56216369)(15.9458255,187.66216919)
\curveto(15.93583334,187.76216349)(15.92083336,187.8571634)(15.9008255,187.94716919)
\curveto(15.89083339,187.99716326)(15.88583339,188.04716321)(15.8858255,188.09716919)
\lineto(15.8858255,188.24716919)
}
}
{
\newrgbcolor{curcolor}{0 0 0}
\pscustom[linestyle=none,fillstyle=solid,fillcolor=curcolor]
{
\newpath
\moveto(13.6958255,194.28677856)
\curveto(13.69583558,194.43677454)(13.70083558,194.58677439)(13.7108255,194.73677856)
\curveto(13.71083557,194.88677409)(13.75083553,194.98677399)(13.8308255,195.03677856)
\curveto(13.89083539,195.06677391)(13.9758353,195.07177391)(14.0858255,195.05177856)
\curveto(14.18583509,195.04177394)(14.29083499,195.02677395)(14.4008255,195.00677856)
\lineto(15.2708255,194.82677856)
\curveto(15.35083393,194.81677416)(15.43583384,194.79677418)(15.5258255,194.76677856)
\curveto(15.60583367,194.74677423)(15.6758336,194.74177424)(15.7358255,194.75177856)
\curveto(15.8758334,194.76177422)(15.96583331,194.83677414)(16.0058255,194.97677856)
\curveto(16.01583326,195.01677396)(16.02083326,195.05677392)(16.0208255,195.09677856)
\lineto(16.0208255,195.24677856)
\lineto(16.0208255,195.65177856)
\curveto(16.01083327,195.82177316)(16.02083326,195.93677304)(16.0508255,195.99677856)
\curveto(16.11083317,196.0767729)(16.17083311,196.12677285)(16.2308255,196.14677856)
\curveto(16.27083301,196.15677282)(16.31583296,196.15677282)(16.3658255,196.14677856)
\lineto(16.5158255,196.11677856)
\curveto(16.62583265,196.09677288)(16.73083255,196.07177291)(16.8308255,196.04177856)
\curveto(16.92083236,196.01177297)(16.99083229,195.96177302)(17.0408255,195.89177856)
\curveto(17.09083219,195.82177316)(17.12083216,195.73177325)(17.1308255,195.62177856)
\lineto(17.1308255,195.29177856)
\curveto(17.12083216,195.1817738)(17.11583216,195.07177391)(17.1158255,194.96177856)
\curveto(17.11583216,194.85177413)(17.13083215,194.75177423)(17.1608255,194.66177856)
\curveto(17.19083209,194.59177439)(17.24083204,194.53177445)(17.3108255,194.48177856)
\curveto(17.3808319,194.44177454)(17.46583181,194.40677457)(17.5658255,194.37677856)
\curveto(17.65583162,194.34677463)(17.75583152,194.32177466)(17.8658255,194.30177856)
\curveto(17.96583131,194.29177469)(18.06583121,194.2767747)(18.1658255,194.25677856)
\lineto(21.1358255,193.65677856)
\curveto(21.35582792,193.61677536)(21.59082769,193.56677541)(21.8408255,193.50677856)
\curveto(22.0808272,193.45677552)(22.26582701,193.46177552)(22.3958255,193.52177856)
\curveto(22.4758268,193.56177542)(22.53082675,193.61677536)(22.5608255,193.68677856)
\curveto(22.59082669,193.76677521)(22.61582666,193.85677512)(22.6358255,193.95677856)
\curveto(22.64582663,193.98677499)(22.65082663,194.01677496)(22.6508255,194.04677856)
\curveto(22.64082664,194.08677489)(22.64082664,194.12177486)(22.6508255,194.15177856)
\lineto(22.6508255,194.34677856)
\curveto(22.65082663,194.44677453)(22.66082662,194.53677444)(22.6808255,194.61677856)
\curveto(22.69082659,194.69677428)(22.72582655,194.75177423)(22.7858255,194.78177856)
\curveto(22.81582646,194.80177418)(22.87082641,194.81177417)(22.9508255,194.81177856)
\curveto(23.02082626,194.81177417)(23.09582618,194.80177418)(23.1758255,194.78177856)
\curveto(23.25582602,194.77177421)(23.33582594,194.75177423)(23.4158255,194.72177856)
\curveto(23.48582579,194.70177428)(23.54082574,194.6767743)(23.5808255,194.64677856)
\curveto(23.65082563,194.58677439)(23.70082558,194.50177448)(23.7308255,194.39177856)
\curveto(23.75082553,194.30177468)(23.76082552,194.20677477)(23.7608255,194.10677856)
\curveto(23.75082553,194.00677497)(23.74582553,193.91677506)(23.7458255,193.83677856)
\curveto(23.74582553,193.7767752)(23.75082553,193.71677526)(23.7608255,193.65677856)
\curveto(23.76082552,193.59677538)(23.75582552,193.54177544)(23.7458255,193.49177856)
\lineto(23.7458255,193.31177856)
\curveto(23.73582554,193.27177571)(23.73082555,193.22677575)(23.7308255,193.17677856)
\curveto(23.72082556,193.13677584)(23.71582556,193.09177589)(23.7158255,193.04177856)
\curveto(23.66582561,192.85177613)(23.61082567,192.68677629)(23.5508255,192.54677856)
\curveto(23.49082579,192.41677656)(23.38582589,192.31677666)(23.2358255,192.24677856)
\curveto(23.03582624,192.14677683)(22.78582649,192.11677686)(22.4858255,192.15677856)
\curveto(22.1758271,192.19677678)(21.84582743,192.25177673)(21.4958255,192.32177856)
\lineto(17.5658255,193.11677856)
\curveto(17.43583184,193.10677587)(17.34083194,193.09677588)(17.2808255,193.08677856)
\curveto(17.22083206,193.0767759)(17.17083211,193.01677596)(17.1308255,192.90677856)
\curveto(17.12083216,192.86677611)(17.12083216,192.82177616)(17.1308255,192.77177856)
\curveto(17.14083214,192.73177625)(17.13583214,192.69677628)(17.1158255,192.66677856)
\lineto(17.1158255,192.42677856)
\curveto(17.11583216,192.29677668)(17.10583217,192.19177679)(17.0858255,192.11177856)
\curveto(17.05583222,192.03177695)(16.99583228,191.98677699)(16.9058255,191.97677856)
\curveto(16.86583241,191.95677702)(16.82083246,191.94677703)(16.7708255,191.94677856)
\lineto(16.6208255,191.97677856)
\curveto(16.4808328,192.00677697)(16.36583291,192.04177694)(16.2758255,192.08177856)
\curveto(16.1758331,192.12177686)(16.10083318,192.19677678)(16.0508255,192.30677856)
\curveto(16.01083327,192.42677655)(16.00083328,192.57177641)(16.0208255,192.74177856)
\curveto(16.04083324,192.91177607)(16.03083325,193.06177592)(15.9908255,193.19177856)
\curveto(15.94083334,193.29177569)(15.87083341,193.3767756)(15.7808255,193.44677856)
\curveto(15.72083356,193.4767755)(15.64583363,193.49677548)(15.5558255,193.50677856)
\curveto(15.46583381,193.52677545)(15.3808339,193.54677543)(15.3008255,193.56677856)
\lineto(14.3708255,193.74677856)
\curveto(14.29083499,193.76677521)(14.21083507,193.7817752)(14.1308255,193.79177856)
\curveto(14.04083524,193.81177517)(13.96583531,193.84177514)(13.9058255,193.88177856)
\curveto(13.82583545,193.93177505)(13.76083552,194.01677496)(13.7108255,194.13677856)
\curveto(13.71083557,194.16677481)(13.71083557,194.19177479)(13.7108255,194.21177856)
\curveto(13.70083558,194.24177474)(13.69583558,194.26677471)(13.6958255,194.28677856)
}
}
{
\newrgbcolor{curcolor}{0 0 0}
\pscustom[linestyle=none,fillstyle=solid,fillcolor=curcolor]
{
\newpath
\moveto(14.5358255,198.17857544)
\curveto(14.4758348,198.10857246)(14.37083491,198.08857248)(14.2208255,198.11857544)
\curveto(14.06083522,198.14857242)(13.90583537,198.17857239)(13.7558255,198.20857544)
\curveto(13.6758356,198.21857235)(13.59083569,198.23357234)(13.5008255,198.25357544)
\curveto(13.41083587,198.2735723)(13.33583594,198.30357227)(13.2758255,198.34357544)
\curveto(13.19583608,198.40357217)(13.13583614,198.49357208)(13.0958255,198.61357544)
\curveto(13.08583619,198.64357193)(13.08583619,198.6685719)(13.0958255,198.68857544)
\curveto(13.09583618,198.70857186)(13.09083619,198.73357184)(13.0808255,198.76357544)
\curveto(13.0808362,198.93357164)(13.08583619,199.08857148)(13.0958255,199.22857544)
\curveto(13.10583617,199.37857119)(13.16583611,199.4685711)(13.2758255,199.49857544)
\curveto(13.33583594,199.51857105)(13.41083587,199.51857105)(13.5008255,199.49857544)
\curveto(13.5808357,199.47857109)(13.66583561,199.46357111)(13.7558255,199.45357544)
\curveto(13.93583534,199.41357116)(14.10583517,199.3735712)(14.2658255,199.33357544)
\curveto(14.42583485,199.30357127)(14.53083475,199.21857135)(14.5808255,199.07857544)
\curveto(14.60083468,199.01857155)(14.61083467,198.95857161)(14.6108255,198.89857544)
\lineto(14.6108255,198.73357544)
\lineto(14.6108255,198.41857544)
\curveto(14.61083467,198.31857225)(14.58583469,198.23857233)(14.5358255,198.17857544)
\moveto(23.0408255,197.59357544)
\curveto(23.14082614,197.573573)(23.24582603,197.55357302)(23.3558255,197.53357544)
\curveto(23.45582582,197.52357305)(23.53582574,197.48357309)(23.5958255,197.41357544)
\curveto(23.65582562,197.3735732)(23.69582558,197.32357325)(23.7158255,197.26357544)
\curveto(23.72582555,197.20357337)(23.74082554,197.12857344)(23.7608255,197.03857544)
\lineto(23.7608255,196.81357544)
\curveto(23.76082552,196.68357389)(23.75582552,196.573574)(23.7458255,196.48357544)
\curveto(23.72582555,196.39357418)(23.6758256,196.32857424)(23.5958255,196.28857544)
\curveto(23.53582574,196.2685743)(23.46082582,196.26357431)(23.3708255,196.27357544)
\curveto(23.27082601,196.29357428)(23.1758261,196.31357426)(23.0858255,196.33357544)
\lineto(16.7408255,197.60857544)
\curveto(16.63083265,197.62857294)(16.52583275,197.64857292)(16.4258255,197.66857544)
\curveto(16.31583296,197.68857288)(16.23083305,197.72857284)(16.1708255,197.78857544)
\curveto(16.12083316,197.82857274)(16.09083319,197.8735727)(16.0808255,197.92357544)
\curveto(16.07083321,197.98357259)(16.05583322,198.04357253)(16.0358255,198.10357544)
\curveto(16.03583324,198.12357245)(16.04083324,198.14357243)(16.0508255,198.16357544)
\curveto(16.05083323,198.19357238)(16.04583323,198.21857235)(16.0358255,198.23857544)
\curveto(16.03583324,198.3685722)(16.04083324,198.49857207)(16.0508255,198.62857544)
\curveto(16.05083323,198.7685718)(16.09083319,198.85357172)(16.1708255,198.88357544)
\curveto(16.23083305,198.92357165)(16.31083297,198.93357164)(16.4108255,198.91357544)
\curveto(16.50083278,198.89357168)(16.59583268,198.8735717)(16.6958255,198.85357544)
\lineto(23.0408255,197.59357544)
}
}
{
\newrgbcolor{curcolor}{0 0 0}
\pscustom[linestyle=none,fillstyle=solid,fillcolor=curcolor]
{
\newpath
\moveto(22.9508255,206.51341919)
\lineto(23.3408255,206.42341919)
\curveto(23.46082582,206.40341126)(23.56082572,206.3634113)(23.6408255,206.30341919)
\curveto(23.71082557,206.23341143)(23.75082553,206.13841152)(23.7608255,206.01841919)
\lineto(23.7608255,205.67341919)
\curveto(23.76082552,205.61341205)(23.76582551,205.55341211)(23.7758255,205.49341919)
\curveto(23.7758255,205.44341222)(23.76582551,205.39841226)(23.7458255,205.35841919)
\curveto(23.72582555,205.27841238)(23.68582559,205.22841243)(23.6258255,205.20841919)
\curveto(23.5758257,205.17841248)(23.51582576,205.16841249)(23.4458255,205.17841919)
\curveto(23.3758259,205.18841247)(23.30582597,205.18341248)(23.2358255,205.16341919)
\curveto(23.21582606,205.1634125)(23.20082608,205.15341251)(23.1908255,205.13341919)
\lineto(23.1308255,205.10341919)
\curveto(23.12082616,205.00341266)(23.14082614,204.91841274)(23.1908255,204.84841919)
\curveto(23.24082604,204.78841287)(23.29082599,204.72341294)(23.3408255,204.65341919)
\curveto(23.49082579,204.42341324)(23.60582567,204.19841346)(23.6858255,203.97841919)
\curveto(23.76582551,203.78841387)(23.82582545,203.56841409)(23.8658255,203.31841919)
\curveto(23.90582537,203.07841458)(23.92582535,202.83341483)(23.9258255,202.58341919)
\curveto(23.93582534,202.34341532)(23.92082536,202.10341556)(23.8808255,201.86341919)
\curveto(23.85082543,201.63341603)(23.79582548,201.43841622)(23.7158255,201.27841919)
\curveto(23.49582578,200.79841686)(23.20082608,200.43341723)(22.8308255,200.18341919)
\curveto(22.45082683,199.94341772)(21.9808273,199.78841787)(21.4208255,199.71841919)
\curveto(21.33082795,199.69841796)(21.24082804,199.68841797)(21.1508255,199.68841919)
\curveto(21.05082823,199.69841796)(20.95082833,199.69841796)(20.8508255,199.68841919)
\curveto(20.80082848,199.68841797)(20.75082853,199.69341797)(20.7008255,199.70341919)
\curveto(20.65082863,199.71341795)(20.60082868,199.71841794)(20.5508255,199.71841919)
\curveto(20.50082878,199.70841795)(20.45082883,199.70841795)(20.4008255,199.71841919)
\curveto(20.34082894,199.73841792)(20.28582899,199.74841791)(20.2358255,199.74841919)
\lineto(20.0858255,199.77841919)
\curveto(20.03582924,199.76841789)(19.97082931,199.76841789)(19.8908255,199.77841919)
\curveto(19.81082947,199.79841786)(19.74582953,199.82341784)(19.6958255,199.85341919)
\lineto(19.5308255,199.89841919)
\curveto(19.46082982,199.92841773)(19.39082989,199.94841771)(19.3208255,199.95841919)
\curveto(19.24083004,199.96841769)(19.16583011,199.98841767)(19.0958255,200.01841919)
\curveto(19.04583023,200.03841762)(19.00083028,200.05341761)(18.9608255,200.06341919)
\curveto(18.92083036,200.07341759)(18.8758304,200.08841757)(18.8258255,200.10841919)
\curveto(18.72583055,200.1584175)(18.63083065,200.20341746)(18.5408255,200.24341919)
\curveto(18.44083084,200.28341738)(18.34583093,200.32841733)(18.2558255,200.37841919)
\curveto(17.8758314,200.57841708)(17.53583174,200.80841685)(17.2358255,201.06841919)
\curveto(16.92583235,201.33841632)(16.67083261,201.63841602)(16.4708255,201.96841919)
\curveto(16.35083293,202.16841549)(16.25083303,202.36841529)(16.1708255,202.56841919)
\curveto(16.09083319,202.76841489)(16.02083326,202.98341468)(15.9608255,203.21341919)
\lineto(15.9308255,203.42341919)
\curveto(15.92083336,203.49341417)(15.90583337,203.5634141)(15.8858255,203.63341919)
\lineto(15.8858255,203.78341919)
\curveto(15.86583341,203.87341379)(15.85583342,203.99341367)(15.8558255,204.14341919)
\curveto(15.85583342,204.30341336)(15.86583341,204.41841324)(15.8858255,204.48841919)
\curveto(15.89583338,204.52841313)(15.90083338,204.58341308)(15.9008255,204.65341919)
\curveto(15.93083335,204.75341291)(15.95583332,204.8584128)(15.9758255,204.96841919)
\curveto(15.98583329,205.07841258)(16.01583326,205.17841248)(16.0658255,205.26841919)
\curveto(16.12583315,205.40841225)(16.19083309,205.53841212)(16.2608255,205.65841919)
\curveto(16.33083295,205.77841188)(16.41083287,205.88841177)(16.5008255,205.98841919)
\curveto(16.55083273,206.03841162)(16.60583267,206.08841157)(16.6658255,206.13841919)
\curveto(16.71583256,206.19841146)(16.73083255,206.28341138)(16.7108255,206.39341919)
\lineto(16.6358255,206.46841919)
\curveto(16.61583266,206.48841117)(16.58583269,206.50341116)(16.5458255,206.51341919)
\curveto(16.45583282,206.5634111)(16.34083294,206.59841106)(16.2008255,206.61841919)
\curveto(16.06083322,206.64841101)(15.93583334,206.67341099)(15.8258255,206.69341919)
\lineto(14.1008255,207.03841919)
\curveto(13.96083532,207.06841059)(13.80583547,207.09841056)(13.6358255,207.12841919)
\curveto(13.45583582,207.16841049)(13.32583595,207.21841044)(13.2458255,207.27841919)
\curveto(13.1758361,207.33841032)(13.13083615,207.40841025)(13.1108255,207.48841919)
\curveto(13.11083617,207.50841015)(13.11083617,207.53341013)(13.1108255,207.56341919)
\curveto(13.10083618,207.59341007)(13.09583618,207.61841004)(13.0958255,207.63841919)
\curveto(13.08583619,207.78840987)(13.08583619,207.93840972)(13.0958255,208.08841919)
\curveto(13.09583618,208.23840942)(13.13583614,208.33840932)(13.2158255,208.38841919)
\curveto(13.29583598,208.41840924)(13.39583588,208.41840924)(13.5158255,208.38841919)
\curveto(13.63583564,208.36840929)(13.76083552,208.34840931)(13.8908255,208.32841919)
\lineto(22.9508255,206.51341919)
\moveto(20.1158255,205.86841919)
\curveto(20.06582921,205.89841176)(20.00082928,205.91841174)(19.9208255,205.92841919)
\curveto(19.83082945,205.94841171)(19.76082952,205.95341171)(19.7108255,205.94341919)
\lineto(19.4858255,205.98841919)
\curveto(19.39582988,205.98841167)(19.30582997,205.99341167)(19.2158255,206.00341919)
\curveto(19.11583016,206.01341165)(19.02583025,206.00841165)(18.9458255,205.98841919)
\lineto(18.7208255,205.98841919)
\curveto(18.65083063,205.98841167)(18.5808307,205.97841168)(18.5108255,205.95841919)
\curveto(18.21083107,205.89841176)(17.94583133,205.79341187)(17.7158255,205.64341919)
\curveto(17.48583179,205.50341216)(17.30583197,205.30341236)(17.1758255,205.04341919)
\curveto(17.12583215,204.95341271)(17.09083219,204.8584128)(17.0708255,204.75841919)
\curveto(17.04083224,204.658413)(17.01583226,204.54841311)(16.9958255,204.42841919)
\curveto(16.9758323,204.3584133)(16.96583231,204.27341339)(16.9658255,204.17341919)
\lineto(16.9658255,203.90341919)
\lineto(16.9958255,203.75341919)
\lineto(16.9958255,203.61841919)
\curveto(17.01583226,203.53841412)(17.03583224,203.45341421)(17.0558255,203.36341919)
\curveto(17.0758322,203.27341439)(17.10083218,203.18841447)(17.1308255,203.10841919)
\curveto(17.27083201,202.7584149)(17.4758318,202.4584152)(17.7458255,202.20841919)
\curveto(18.00583127,201.9584157)(18.31083097,201.73841592)(18.6608255,201.54841919)
\curveto(18.77083051,201.48841617)(18.88583039,201.43841622)(19.0058255,201.39841919)
\lineto(19.3358255,201.27841919)
\lineto(19.4558255,201.24841919)
\curveto(19.48582979,201.23841642)(19.52082976,201.22841643)(19.5608255,201.21841919)
\curveto(19.61082967,201.18841647)(19.66582961,201.16841649)(19.7258255,201.15841919)
\curveto(19.78582949,201.1584165)(19.84082944,201.15341651)(19.8908255,201.14341919)
\curveto(20.00082928,201.12341654)(20.11082917,201.09841656)(20.2208255,201.06841919)
\curveto(20.32082896,201.04841661)(20.41582886,201.04341662)(20.5058255,201.05341919)
\curveto(20.53582874,201.05341661)(20.58582869,201.04841661)(20.6558255,201.03841919)
\lineto(20.8658255,201.03841919)
\curveto(20.93582834,201.03841662)(21.00582827,201.04341662)(21.0758255,201.05341919)
\curveto(21.42582785,201.09341657)(21.72582755,201.18341648)(21.9758255,201.32341919)
\curveto(22.22582705,201.4634162)(22.43082685,201.663416)(22.5908255,201.92341919)
\curveto(22.64082664,202.00341566)(22.6808266,202.08341558)(22.7108255,202.16341919)
\curveto(22.74082654,202.25341541)(22.77082651,202.34841531)(22.8008255,202.44841919)
\curveto(22.82082646,202.49841516)(22.82582645,202.54841511)(22.8158255,202.59841919)
\curveto(22.80582647,202.658415)(22.81082647,202.71341495)(22.8308255,202.76341919)
\curveto(22.84082644,202.79341487)(22.84582643,202.82841483)(22.8458255,202.86841919)
\lineto(22.8458255,203.00341919)
\lineto(22.8458255,203.13841919)
\curveto(22.83582644,203.17841448)(22.83082645,203.23341443)(22.8308255,203.30341919)
\curveto(22.81082647,203.38341428)(22.79582648,203.4634142)(22.7858255,203.54341919)
\curveto(22.76582651,203.63341403)(22.74082654,203.71341395)(22.7108255,203.78341919)
\curveto(22.57082671,204.14341352)(22.39582688,204.44841321)(22.1858255,204.69841919)
\curveto(21.96582731,204.94841271)(21.69082759,205.17341249)(21.3608255,205.37341919)
\curveto(21.25082803,205.44341222)(21.14082814,205.49841216)(21.0308255,205.53841919)
\lineto(20.7008255,205.68841919)
\curveto(20.66082862,205.71841194)(20.62582865,205.73341193)(20.5958255,205.73341919)
\curveto(20.55582872,205.74341192)(20.51582876,205.7584119)(20.4758255,205.77841919)
\curveto(20.41582886,205.79841186)(20.35582892,205.81341185)(20.2958255,205.82341919)
\curveto(20.23582904,205.83341183)(20.1758291,205.84841181)(20.1158255,205.86841919)
}
}
{
\newrgbcolor{curcolor}{0 0 0}
\pscustom[linestyle=none,fillstyle=solid,fillcolor=curcolor]
{
\newpath
\moveto(23.2058255,215.29966919)
\curveto(23.36582591,215.28966128)(23.50082578,215.24466132)(23.6108255,215.16466919)
\curveto(23.71082557,215.08466148)(23.78582549,214.98966158)(23.8358255,214.87966919)
\curveto(23.85582542,214.82966174)(23.86582541,214.77466179)(23.8658255,214.71466919)
\curveto(23.86582541,214.6646619)(23.8758254,214.60466196)(23.8958255,214.53466919)
\curveto(23.94582533,214.30466226)(23.93082535,214.08966248)(23.8508255,213.88966919)
\curveto(23.7808255,213.68966288)(23.69082559,213.564663)(23.5808255,213.51466919)
\curveto(23.51082577,213.47466309)(23.43082585,213.44466312)(23.3408255,213.42466919)
\curveto(23.24082604,213.40466316)(23.16082612,213.3696632)(23.1008255,213.31966919)
\lineto(23.0408255,213.25966919)
\curveto(23.02082626,213.23966333)(23.01582626,213.20966336)(23.0258255,213.16966919)
\curveto(23.05582622,213.04966352)(23.11082617,212.93466363)(23.1908255,212.82466919)
\curveto(23.27082601,212.71466385)(23.34082594,212.60966396)(23.4008255,212.50966919)
\curveto(23.4808258,212.35966421)(23.55582572,212.20466436)(23.6258255,212.04466919)
\curveto(23.68582559,211.88466468)(23.74082554,211.71466485)(23.7908255,211.53466919)
\curveto(23.82082546,211.42466514)(23.84082544,211.30966526)(23.8508255,211.18966919)
\curveto(23.86082542,211.07966549)(23.8758254,210.9646656)(23.8958255,210.84466919)
\curveto(23.90582537,210.79466577)(23.91082537,210.74966582)(23.9108255,210.70966919)
\lineto(23.9108255,210.60466919)
\curveto(23.93082535,210.49466607)(23.93082535,210.38966618)(23.9108255,210.28966919)
\lineto(23.9108255,210.15466919)
\curveto(23.90082538,210.10466646)(23.89582538,210.05466651)(23.8958255,210.00466919)
\curveto(23.89582538,209.95466661)(23.88582539,209.91466665)(23.8658255,209.88466919)
\curveto(23.85582542,209.84466672)(23.85082543,209.80966676)(23.8508255,209.77966919)
\curveto(23.86082542,209.75966681)(23.86082542,209.73466683)(23.8508255,209.70466919)
\lineto(23.7908255,209.46466919)
\curveto(23.7808255,209.39466717)(23.76082552,209.32966724)(23.7308255,209.26966919)
\curveto(23.60082568,208.98966758)(23.45582582,208.77466779)(23.2958255,208.62466919)
\curveto(23.12582615,208.47466809)(22.89082639,208.3696682)(22.5908255,208.30966919)
\curveto(22.37082691,208.25966831)(22.10582717,208.2646683)(21.7958255,208.32466919)
\lineto(21.4808255,208.39966919)
\curveto(21.43082785,208.41966815)(21.3808279,208.43466813)(21.3308255,208.44466919)
\lineto(21.1508255,208.50466919)
\lineto(20.8208255,208.68466919)
\curveto(20.71082857,208.75466781)(20.61082867,208.82466774)(20.5208255,208.89466919)
\curveto(20.23082905,209.13466743)(20.01582926,209.42466714)(19.8758255,209.76466919)
\curveto(19.73582954,210.10466646)(19.61082967,210.4696661)(19.5008255,210.85966919)
\curveto(19.46082982,211.00966556)(19.43082985,211.15966541)(19.4108255,211.30966919)
\curveto(19.39082989,211.4696651)(19.36582991,211.62466494)(19.3358255,211.77466919)
\curveto(19.31582996,211.85466471)(19.30582997,211.92466464)(19.3058255,211.98466919)
\curveto(19.30582997,212.05466451)(19.29582998,212.12966444)(19.2758255,212.20966919)
\curveto(19.25583002,212.27966429)(19.24583003,212.34966422)(19.2458255,212.41966919)
\curveto(19.23583004,212.49966407)(19.22083006,212.57966399)(19.2008255,212.65966919)
\curveto(19.14083014,212.91966365)(19.09083019,213.1646634)(19.0508255,213.39466919)
\curveto(19.00083028,213.62466294)(18.88583039,213.82466274)(18.7058255,213.99466919)
\curveto(18.62583065,214.0646625)(18.52583075,214.12966244)(18.4058255,214.18966919)
\curveto(18.275831,214.25966231)(18.13583114,214.28966228)(17.9858255,214.27966919)
\curveto(17.74583153,214.2696623)(17.55583172,214.21966235)(17.4158255,214.12966919)
\curveto(17.275832,214.04966252)(17.16583211,213.90966266)(17.0858255,213.70966919)
\curveto(17.03583224,213.59966297)(17.00083228,213.4646631)(16.9808255,213.30466919)
\curveto(16.96083232,213.14466342)(16.95083233,212.97466359)(16.9508255,212.79466919)
\curveto(16.95083233,212.61466395)(16.96083232,212.43466413)(16.9808255,212.25466919)
\curveto(17.00083228,212.08466448)(17.03083225,211.93466463)(17.0708255,211.80466919)
\curveto(17.13083215,211.62466494)(17.21583206,211.44466512)(17.3258255,211.26466919)
\curveto(17.38583189,211.17466539)(17.46583181,211.08466548)(17.5658255,210.99466919)
\curveto(17.65583162,210.91466565)(17.75583152,210.83966573)(17.8658255,210.76966919)
\curveto(17.94583133,210.71966585)(18.03083125,210.67466589)(18.1208255,210.63466919)
\curveto(18.21083107,210.59466597)(18.280831,210.53466603)(18.3308255,210.45466919)
\curveto(18.36083092,210.40466616)(18.38583089,210.32966624)(18.4058255,210.22966919)
\curveto(18.41583086,210.12966644)(18.42083086,210.02966654)(18.4208255,209.92966919)
\curveto(18.42083086,209.82966674)(18.41583086,209.73466683)(18.4058255,209.64466919)
\curveto(18.38583089,209.55466701)(18.36083092,209.49466707)(18.3308255,209.46466919)
\curveto(18.30083098,209.42466714)(18.25083103,209.39966717)(18.1808255,209.38966919)
\curveto(18.11083117,209.38966718)(18.03583124,209.40966716)(17.9558255,209.44966919)
\curveto(17.82583145,209.49966707)(17.70583157,209.55466701)(17.5958255,209.61466919)
\curveto(17.4758318,209.67466689)(17.36083192,209.73966683)(17.2508255,209.80966919)
\curveto(16.90083238,210.0696665)(16.63083265,210.3646662)(16.4408255,210.69466919)
\curveto(16.24083304,211.02466554)(16.0808332,211.41466515)(15.9608255,211.86466919)
\curveto(15.94083334,211.97466459)(15.92583335,212.07966449)(15.9158255,212.17966919)
\curveto(15.90583337,212.28966428)(15.89083339,212.39966417)(15.8708255,212.50966919)
\curveto(15.86083342,212.55966401)(15.86083342,212.62466394)(15.8708255,212.70466919)
\curveto(15.87083341,212.79466377)(15.86083342,212.85466371)(15.8408255,212.88466919)
\curveto(15.83083345,213.58466298)(15.91083337,214.17466239)(16.0808255,214.65466919)
\curveto(16.25083303,215.14466142)(16.5758327,215.44966112)(17.0558255,215.56966919)
\curveto(17.25583202,215.61966095)(17.49083179,215.62466094)(17.7608255,215.58466919)
\curveto(18.02083126,215.54466102)(18.29583098,215.49466107)(18.5858255,215.43466919)
\lineto(21.9008255,214.77466919)
\curveto(22.04082724,214.74466182)(22.1758271,214.71966185)(22.3058255,214.69966919)
\curveto(22.43582684,214.68966188)(22.54082674,214.69966187)(22.6208255,214.72966919)
\curveto(22.69082659,214.7696618)(22.74082654,214.82466174)(22.7708255,214.89466919)
\curveto(22.81082647,214.98466158)(22.84082644,215.0646615)(22.8608255,215.13466919)
\curveto(22.87082641,215.21466135)(22.91582636,215.2646613)(22.9958255,215.28466919)
\curveto(23.02582625,215.30466126)(23.05582622,215.30966126)(23.0858255,215.29966919)
\lineto(23.2058255,215.29966919)
\moveto(21.5408255,213.48466919)
\curveto(21.40082788,213.57466299)(21.24082804,213.63966293)(21.0608255,213.67966919)
\curveto(20.87082841,213.71966285)(20.6758286,213.75966281)(20.4758255,213.79966919)
\curveto(20.36582891,213.81966275)(20.26582901,213.83466273)(20.1758255,213.84466919)
\curveto(20.08582919,213.85466271)(20.01582926,213.82966274)(19.9658255,213.76966919)
\curveto(19.94582933,213.73966283)(19.93582934,213.6696629)(19.9358255,213.55966919)
\curveto(19.95582932,213.53966303)(19.96582931,213.50466306)(19.9658255,213.45466919)
\curveto(19.96582931,213.40466316)(19.9758293,213.35466321)(19.9958255,213.30466919)
\curveto(20.01582926,213.22466334)(20.03582924,213.12966344)(20.0558255,213.01966919)
\lineto(20.1158255,212.71966919)
\curveto(20.11582916,212.68966388)(20.12082916,212.65466391)(20.1308255,212.61466919)
\lineto(20.1308255,212.50966919)
\curveto(20.17082911,212.34966422)(20.19582908,212.17966439)(20.2058255,211.99966919)
\curveto(20.20582907,211.82966474)(20.22582905,211.6646649)(20.2658255,211.50466919)
\curveto(20.28582899,211.41466515)(20.30582897,211.33466523)(20.3258255,211.26466919)
\curveto(20.33582894,211.20466536)(20.35082893,211.12966544)(20.3708255,211.03966919)
\curveto(20.42082886,210.8696657)(20.48582879,210.70466586)(20.5658255,210.54466919)
\curveto(20.63582864,210.39466617)(20.72582855,210.25966631)(20.8358255,210.13966919)
\curveto(20.94582833,210.01966655)(21.0808282,209.91966665)(21.2408255,209.83966919)
\curveto(21.39082789,209.75966681)(21.5758277,209.69966687)(21.7958255,209.65966919)
\curveto(21.89582738,209.63966693)(21.99082729,209.63966693)(22.0808255,209.65966919)
\curveto(22.16082712,209.67966689)(22.23582704,209.70966686)(22.3058255,209.74966919)
\curveto(22.41582686,209.79966677)(22.51082677,209.87966669)(22.5908255,209.98966919)
\curveto(22.66082662,210.10966646)(22.72082656,210.23966633)(22.7708255,210.37966919)
\curveto(22.7808265,210.42966614)(22.78582649,210.47966609)(22.7858255,210.52966919)
\curveto(22.78582649,210.57966599)(22.79082649,210.62966594)(22.8008255,210.67966919)
\curveto(22.82082646,210.74966582)(22.83582644,210.83466573)(22.8458255,210.93466919)
\curveto(22.84582643,211.03466553)(22.83582644,211.12466544)(22.8158255,211.20466919)
\curveto(22.79582648,211.2646653)(22.79082649,211.32466524)(22.8008255,211.38466919)
\curveto(22.80082648,211.44466512)(22.79082649,211.50466506)(22.7708255,211.56466919)
\curveto(22.75082653,211.65466491)(22.73582654,211.73466483)(22.7258255,211.80466919)
\curveto(22.71582656,211.88466468)(22.69582658,211.9646646)(22.6658255,212.04466919)
\curveto(22.54582673,212.35466421)(22.40082688,212.62966394)(22.2308255,212.86966919)
\curveto(22.06082722,213.10966346)(21.83082745,213.31466325)(21.5408255,213.48466919)
}
}
{
\newrgbcolor{curcolor}{0 0 0}
\pscustom[linestyle=none,fillstyle=solid,fillcolor=curcolor]
{
\newpath
\moveto(22.9508255,223.47630981)
\lineto(23.3408255,223.38630981)
\curveto(23.46082582,223.36630188)(23.56082572,223.32630192)(23.6408255,223.26630981)
\curveto(23.71082557,223.19630205)(23.75082553,223.10130215)(23.7608255,222.98130981)
\lineto(23.7608255,222.63630981)
\curveto(23.76082552,222.57630267)(23.76582551,222.51630273)(23.7758255,222.45630981)
\curveto(23.7758255,222.40630284)(23.76582551,222.36130289)(23.7458255,222.32130981)
\curveto(23.72582555,222.24130301)(23.68582559,222.19130306)(23.6258255,222.17130981)
\curveto(23.5758257,222.14130311)(23.51582576,222.13130312)(23.4458255,222.14130981)
\curveto(23.3758259,222.1513031)(23.30582597,222.1463031)(23.2358255,222.12630981)
\curveto(23.21582606,222.12630312)(23.20082608,222.11630313)(23.1908255,222.09630981)
\lineto(23.1308255,222.06630981)
\curveto(23.12082616,221.96630328)(23.14082614,221.88130337)(23.1908255,221.81130981)
\curveto(23.24082604,221.7513035)(23.29082599,221.68630356)(23.3408255,221.61630981)
\curveto(23.49082579,221.38630386)(23.60582567,221.16130409)(23.6858255,220.94130981)
\curveto(23.76582551,220.7513045)(23.82582545,220.53130472)(23.8658255,220.28130981)
\curveto(23.90582537,220.04130521)(23.92582535,219.79630545)(23.9258255,219.54630981)
\curveto(23.93582534,219.30630594)(23.92082536,219.06630618)(23.8808255,218.82630981)
\curveto(23.85082543,218.59630665)(23.79582548,218.40130685)(23.7158255,218.24130981)
\curveto(23.49582578,217.76130749)(23.20082608,217.39630785)(22.8308255,217.14630981)
\curveto(22.45082683,216.90630834)(21.9808273,216.7513085)(21.4208255,216.68130981)
\curveto(21.33082795,216.66130859)(21.24082804,216.6513086)(21.1508255,216.65130981)
\curveto(21.05082823,216.66130859)(20.95082833,216.66130859)(20.8508255,216.65130981)
\curveto(20.80082848,216.6513086)(20.75082853,216.65630859)(20.7008255,216.66630981)
\curveto(20.65082863,216.67630857)(20.60082868,216.68130857)(20.5508255,216.68130981)
\curveto(20.50082878,216.67130858)(20.45082883,216.67130858)(20.4008255,216.68130981)
\curveto(20.34082894,216.70130855)(20.28582899,216.71130854)(20.2358255,216.71130981)
\lineto(20.0858255,216.74130981)
\curveto(20.03582924,216.73130852)(19.97082931,216.73130852)(19.8908255,216.74130981)
\curveto(19.81082947,216.76130849)(19.74582953,216.78630846)(19.6958255,216.81630981)
\lineto(19.5308255,216.86130981)
\curveto(19.46082982,216.89130836)(19.39082989,216.91130834)(19.3208255,216.92130981)
\curveto(19.24083004,216.93130832)(19.16583011,216.9513083)(19.0958255,216.98130981)
\curveto(19.04583023,217.00130825)(19.00083028,217.01630823)(18.9608255,217.02630981)
\curveto(18.92083036,217.03630821)(18.8758304,217.0513082)(18.8258255,217.07130981)
\curveto(18.72583055,217.12130813)(18.63083065,217.16630808)(18.5408255,217.20630981)
\curveto(18.44083084,217.246308)(18.34583093,217.29130796)(18.2558255,217.34130981)
\curveto(17.8758314,217.54130771)(17.53583174,217.77130748)(17.2358255,218.03130981)
\curveto(16.92583235,218.30130695)(16.67083261,218.60130665)(16.4708255,218.93130981)
\curveto(16.35083293,219.13130612)(16.25083303,219.33130592)(16.1708255,219.53130981)
\curveto(16.09083319,219.73130552)(16.02083326,219.9463053)(15.9608255,220.17630981)
\lineto(15.9308255,220.38630981)
\curveto(15.92083336,220.45630479)(15.90583337,220.52630472)(15.8858255,220.59630981)
\lineto(15.8858255,220.74630981)
\curveto(15.86583341,220.83630441)(15.85583342,220.95630429)(15.8558255,221.10630981)
\curveto(15.85583342,221.26630398)(15.86583341,221.38130387)(15.8858255,221.45130981)
\curveto(15.89583338,221.49130376)(15.90083338,221.5463037)(15.9008255,221.61630981)
\curveto(15.93083335,221.71630353)(15.95583332,221.82130343)(15.9758255,221.93130981)
\curveto(15.98583329,222.04130321)(16.01583326,222.14130311)(16.0658255,222.23130981)
\curveto(16.12583315,222.37130288)(16.19083309,222.50130275)(16.2608255,222.62130981)
\curveto(16.33083295,222.74130251)(16.41083287,222.8513024)(16.5008255,222.95130981)
\curveto(16.55083273,223.00130225)(16.60583267,223.0513022)(16.6658255,223.10130981)
\curveto(16.71583256,223.16130209)(16.73083255,223.246302)(16.7108255,223.35630981)
\lineto(16.6358255,223.43130981)
\curveto(16.61583266,223.4513018)(16.58583269,223.46630178)(16.5458255,223.47630981)
\curveto(16.45583282,223.52630172)(16.34083294,223.56130169)(16.2008255,223.58130981)
\curveto(16.06083322,223.61130164)(15.93583334,223.63630161)(15.8258255,223.65630981)
\lineto(14.1008255,224.00130981)
\curveto(13.96083532,224.03130122)(13.80583547,224.06130119)(13.6358255,224.09130981)
\curveto(13.45583582,224.13130112)(13.32583595,224.18130107)(13.2458255,224.24130981)
\curveto(13.1758361,224.30130095)(13.13083615,224.37130088)(13.1108255,224.45130981)
\curveto(13.11083617,224.47130078)(13.11083617,224.49630075)(13.1108255,224.52630981)
\curveto(13.10083618,224.55630069)(13.09583618,224.58130067)(13.0958255,224.60130981)
\curveto(13.08583619,224.7513005)(13.08583619,224.90130035)(13.0958255,225.05130981)
\curveto(13.09583618,225.20130005)(13.13583614,225.30129995)(13.2158255,225.35130981)
\curveto(13.29583598,225.38129987)(13.39583588,225.38129987)(13.5158255,225.35130981)
\curveto(13.63583564,225.33129992)(13.76083552,225.31129994)(13.8908255,225.29130981)
\lineto(22.9508255,223.47630981)
\moveto(20.1158255,222.83130981)
\curveto(20.06582921,222.86130239)(20.00082928,222.88130237)(19.9208255,222.89130981)
\curveto(19.83082945,222.91130234)(19.76082952,222.91630233)(19.7108255,222.90630981)
\lineto(19.4858255,222.95130981)
\curveto(19.39582988,222.9513023)(19.30582997,222.95630229)(19.2158255,222.96630981)
\curveto(19.11583016,222.97630227)(19.02583025,222.97130228)(18.9458255,222.95130981)
\lineto(18.7208255,222.95130981)
\curveto(18.65083063,222.9513023)(18.5808307,222.94130231)(18.5108255,222.92130981)
\curveto(18.21083107,222.86130239)(17.94583133,222.75630249)(17.7158255,222.60630981)
\curveto(17.48583179,222.46630278)(17.30583197,222.26630298)(17.1758255,222.00630981)
\curveto(17.12583215,221.91630333)(17.09083219,221.82130343)(17.0708255,221.72130981)
\curveto(17.04083224,221.62130363)(17.01583226,221.51130374)(16.9958255,221.39130981)
\curveto(16.9758323,221.32130393)(16.96583231,221.23630401)(16.9658255,221.13630981)
\lineto(16.9658255,220.86630981)
\lineto(16.9958255,220.71630981)
\lineto(16.9958255,220.58130981)
\curveto(17.01583226,220.50130475)(17.03583224,220.41630483)(17.0558255,220.32630981)
\curveto(17.0758322,220.23630501)(17.10083218,220.1513051)(17.1308255,220.07130981)
\curveto(17.27083201,219.72130553)(17.4758318,219.42130583)(17.7458255,219.17130981)
\curveto(18.00583127,218.92130633)(18.31083097,218.70130655)(18.6608255,218.51130981)
\curveto(18.77083051,218.4513068)(18.88583039,218.40130685)(19.0058255,218.36130981)
\lineto(19.3358255,218.24130981)
\lineto(19.4558255,218.21130981)
\curveto(19.48582979,218.20130705)(19.52082976,218.19130706)(19.5608255,218.18130981)
\curveto(19.61082967,218.1513071)(19.66582961,218.13130712)(19.7258255,218.12130981)
\curveto(19.78582949,218.12130713)(19.84082944,218.11630713)(19.8908255,218.10630981)
\curveto(20.00082928,218.08630716)(20.11082917,218.06130719)(20.2208255,218.03130981)
\curveto(20.32082896,218.01130724)(20.41582886,218.00630724)(20.5058255,218.01630981)
\curveto(20.53582874,218.01630723)(20.58582869,218.01130724)(20.6558255,218.00130981)
\lineto(20.8658255,218.00130981)
\curveto(20.93582834,218.00130725)(21.00582827,218.00630724)(21.0758255,218.01630981)
\curveto(21.42582785,218.05630719)(21.72582755,218.1463071)(21.9758255,218.28630981)
\curveto(22.22582705,218.42630682)(22.43082685,218.62630662)(22.5908255,218.88630981)
\curveto(22.64082664,218.96630628)(22.6808266,219.0463062)(22.7108255,219.12630981)
\curveto(22.74082654,219.21630603)(22.77082651,219.31130594)(22.8008255,219.41130981)
\curveto(22.82082646,219.46130579)(22.82582645,219.51130574)(22.8158255,219.56130981)
\curveto(22.80582647,219.62130563)(22.81082647,219.67630557)(22.8308255,219.72630981)
\curveto(22.84082644,219.75630549)(22.84582643,219.79130546)(22.8458255,219.83130981)
\lineto(22.8458255,219.96630981)
\lineto(22.8458255,220.10130981)
\curveto(22.83582644,220.14130511)(22.83082645,220.19630505)(22.8308255,220.26630981)
\curveto(22.81082647,220.3463049)(22.79582648,220.42630482)(22.7858255,220.50630981)
\curveto(22.76582651,220.59630465)(22.74082654,220.67630457)(22.7108255,220.74630981)
\curveto(22.57082671,221.10630414)(22.39582688,221.41130384)(22.1858255,221.66130981)
\curveto(21.96582731,221.91130334)(21.69082759,222.13630311)(21.3608255,222.33630981)
\curveto(21.25082803,222.40630284)(21.14082814,222.46130279)(21.0308255,222.50130981)
\lineto(20.7008255,222.65130981)
\curveto(20.66082862,222.68130257)(20.62582865,222.69630255)(20.5958255,222.69630981)
\curveto(20.55582872,222.70630254)(20.51582876,222.72130253)(20.4758255,222.74130981)
\curveto(20.41582886,222.76130249)(20.35582892,222.77630247)(20.2958255,222.78630981)
\curveto(20.23582904,222.79630245)(20.1758291,222.81130244)(20.1158255,222.83130981)
}
}
{
\newrgbcolor{curcolor}{0 0 0}
\pscustom[linestyle=none,fillstyle=solid,fillcolor=curcolor]
{
}
}
{
\newrgbcolor{curcolor}{0 0 0}
\pscustom[linestyle=none,fillstyle=solid,fillcolor=curcolor]
{
\newpath
\moveto(22.9508255,236.54271606)
\lineto(23.3408255,236.45271606)
\curveto(23.46082582,236.43270813)(23.56082572,236.39270817)(23.6408255,236.33271606)
\curveto(23.71082557,236.2627083)(23.75082553,236.1677084)(23.7608255,236.04771606)
\lineto(23.7608255,235.70271606)
\curveto(23.76082552,235.64270892)(23.76582551,235.58270898)(23.7758255,235.52271606)
\curveto(23.7758255,235.47270909)(23.76582551,235.42770914)(23.7458255,235.38771606)
\curveto(23.72582555,235.30770926)(23.68582559,235.25770931)(23.6258255,235.23771606)
\curveto(23.5758257,235.20770936)(23.51582576,235.19770937)(23.4458255,235.20771606)
\curveto(23.3758259,235.21770935)(23.30582597,235.21270935)(23.2358255,235.19271606)
\curveto(23.21582606,235.19270937)(23.20082608,235.18270938)(23.1908255,235.16271606)
\lineto(23.1308255,235.13271606)
\curveto(23.12082616,235.03270953)(23.14082614,234.94770962)(23.1908255,234.87771606)
\curveto(23.24082604,234.81770975)(23.29082599,234.75270981)(23.3408255,234.68271606)
\curveto(23.49082579,234.45271011)(23.60582567,234.22771034)(23.6858255,234.00771606)
\curveto(23.76582551,233.81771075)(23.82582545,233.59771097)(23.8658255,233.34771606)
\curveto(23.90582537,233.10771146)(23.92582535,232.8627117)(23.9258255,232.61271606)
\curveto(23.93582534,232.37271219)(23.92082536,232.13271243)(23.8808255,231.89271606)
\curveto(23.85082543,231.6627129)(23.79582548,231.4677131)(23.7158255,231.30771606)
\curveto(23.49582578,230.82771374)(23.20082608,230.4627141)(22.8308255,230.21271606)
\curveto(22.45082683,229.97271459)(21.9808273,229.81771475)(21.4208255,229.74771606)
\curveto(21.33082795,229.72771484)(21.24082804,229.71771485)(21.1508255,229.71771606)
\curveto(21.05082823,229.72771484)(20.95082833,229.72771484)(20.8508255,229.71771606)
\curveto(20.80082848,229.71771485)(20.75082853,229.72271484)(20.7008255,229.73271606)
\curveto(20.65082863,229.74271482)(20.60082868,229.74771482)(20.5508255,229.74771606)
\curveto(20.50082878,229.73771483)(20.45082883,229.73771483)(20.4008255,229.74771606)
\curveto(20.34082894,229.7677148)(20.28582899,229.77771479)(20.2358255,229.77771606)
\lineto(20.0858255,229.80771606)
\curveto(20.03582924,229.79771477)(19.97082931,229.79771477)(19.8908255,229.80771606)
\curveto(19.81082947,229.82771474)(19.74582953,229.85271471)(19.6958255,229.88271606)
\lineto(19.5308255,229.92771606)
\curveto(19.46082982,229.95771461)(19.39082989,229.97771459)(19.3208255,229.98771606)
\curveto(19.24083004,229.99771457)(19.16583011,230.01771455)(19.0958255,230.04771606)
\curveto(19.04583023,230.0677145)(19.00083028,230.08271448)(18.9608255,230.09271606)
\curveto(18.92083036,230.10271446)(18.8758304,230.11771445)(18.8258255,230.13771606)
\curveto(18.72583055,230.18771438)(18.63083065,230.23271433)(18.5408255,230.27271606)
\curveto(18.44083084,230.31271425)(18.34583093,230.35771421)(18.2558255,230.40771606)
\curveto(17.8758314,230.60771396)(17.53583174,230.83771373)(17.2358255,231.09771606)
\curveto(16.92583235,231.3677132)(16.67083261,231.6677129)(16.4708255,231.99771606)
\curveto(16.35083293,232.19771237)(16.25083303,232.39771217)(16.1708255,232.59771606)
\curveto(16.09083319,232.79771177)(16.02083326,233.01271155)(15.9608255,233.24271606)
\lineto(15.9308255,233.45271606)
\curveto(15.92083336,233.52271104)(15.90583337,233.59271097)(15.8858255,233.66271606)
\lineto(15.8858255,233.81271606)
\curveto(15.86583341,233.90271066)(15.85583342,234.02271054)(15.8558255,234.17271606)
\curveto(15.85583342,234.33271023)(15.86583341,234.44771012)(15.8858255,234.51771606)
\curveto(15.89583338,234.55771001)(15.90083338,234.61270995)(15.9008255,234.68271606)
\curveto(15.93083335,234.78270978)(15.95583332,234.88770968)(15.9758255,234.99771606)
\curveto(15.98583329,235.10770946)(16.01583326,235.20770936)(16.0658255,235.29771606)
\curveto(16.12583315,235.43770913)(16.19083309,235.567709)(16.2608255,235.68771606)
\curveto(16.33083295,235.80770876)(16.41083287,235.91770865)(16.5008255,236.01771606)
\curveto(16.55083273,236.0677085)(16.60583267,236.11770845)(16.6658255,236.16771606)
\curveto(16.71583256,236.22770834)(16.73083255,236.31270825)(16.7108255,236.42271606)
\lineto(16.6358255,236.49771606)
\curveto(16.61583266,236.51770805)(16.58583269,236.53270803)(16.5458255,236.54271606)
\curveto(16.45583282,236.59270797)(16.34083294,236.62770794)(16.2008255,236.64771606)
\curveto(16.06083322,236.67770789)(15.93583334,236.70270786)(15.8258255,236.72271606)
\lineto(14.1008255,237.06771606)
\curveto(13.96083532,237.09770747)(13.80583547,237.12770744)(13.6358255,237.15771606)
\curveto(13.45583582,237.19770737)(13.32583595,237.24770732)(13.2458255,237.30771606)
\curveto(13.1758361,237.3677072)(13.13083615,237.43770713)(13.1108255,237.51771606)
\curveto(13.11083617,237.53770703)(13.11083617,237.562707)(13.1108255,237.59271606)
\curveto(13.10083618,237.62270694)(13.09583618,237.64770692)(13.0958255,237.66771606)
\curveto(13.08583619,237.81770675)(13.08583619,237.9677066)(13.0958255,238.11771606)
\curveto(13.09583618,238.2677063)(13.13583614,238.3677062)(13.2158255,238.41771606)
\curveto(13.29583598,238.44770612)(13.39583588,238.44770612)(13.5158255,238.41771606)
\curveto(13.63583564,238.39770617)(13.76083552,238.37770619)(13.8908255,238.35771606)
\lineto(22.9508255,236.54271606)
\moveto(20.1158255,235.89771606)
\curveto(20.06582921,235.92770864)(20.00082928,235.94770862)(19.9208255,235.95771606)
\curveto(19.83082945,235.97770859)(19.76082952,235.98270858)(19.7108255,235.97271606)
\lineto(19.4858255,236.01771606)
\curveto(19.39582988,236.01770855)(19.30582997,236.02270854)(19.2158255,236.03271606)
\curveto(19.11583016,236.04270852)(19.02583025,236.03770853)(18.9458255,236.01771606)
\lineto(18.7208255,236.01771606)
\curveto(18.65083063,236.01770855)(18.5808307,236.00770856)(18.5108255,235.98771606)
\curveto(18.21083107,235.92770864)(17.94583133,235.82270874)(17.7158255,235.67271606)
\curveto(17.48583179,235.53270903)(17.30583197,235.33270923)(17.1758255,235.07271606)
\curveto(17.12583215,234.98270958)(17.09083219,234.88770968)(17.0708255,234.78771606)
\curveto(17.04083224,234.68770988)(17.01583226,234.57770999)(16.9958255,234.45771606)
\curveto(16.9758323,234.38771018)(16.96583231,234.30271026)(16.9658255,234.20271606)
\lineto(16.9658255,233.93271606)
\lineto(16.9958255,233.78271606)
\lineto(16.9958255,233.64771606)
\curveto(17.01583226,233.567711)(17.03583224,233.48271108)(17.0558255,233.39271606)
\curveto(17.0758322,233.30271126)(17.10083218,233.21771135)(17.1308255,233.13771606)
\curveto(17.27083201,232.78771178)(17.4758318,232.48771208)(17.7458255,232.23771606)
\curveto(18.00583127,231.98771258)(18.31083097,231.7677128)(18.6608255,231.57771606)
\curveto(18.77083051,231.51771305)(18.88583039,231.4677131)(19.0058255,231.42771606)
\lineto(19.3358255,231.30771606)
\lineto(19.4558255,231.27771606)
\curveto(19.48582979,231.2677133)(19.52082976,231.25771331)(19.5608255,231.24771606)
\curveto(19.61082967,231.21771335)(19.66582961,231.19771337)(19.7258255,231.18771606)
\curveto(19.78582949,231.18771338)(19.84082944,231.18271338)(19.8908255,231.17271606)
\curveto(20.00082928,231.15271341)(20.11082917,231.12771344)(20.2208255,231.09771606)
\curveto(20.32082896,231.07771349)(20.41582886,231.07271349)(20.5058255,231.08271606)
\curveto(20.53582874,231.08271348)(20.58582869,231.07771349)(20.6558255,231.06771606)
\lineto(20.8658255,231.06771606)
\curveto(20.93582834,231.0677135)(21.00582827,231.07271349)(21.0758255,231.08271606)
\curveto(21.42582785,231.12271344)(21.72582755,231.21271335)(21.9758255,231.35271606)
\curveto(22.22582705,231.49271307)(22.43082685,231.69271287)(22.5908255,231.95271606)
\curveto(22.64082664,232.03271253)(22.6808266,232.11271245)(22.7108255,232.19271606)
\curveto(22.74082654,232.28271228)(22.77082651,232.37771219)(22.8008255,232.47771606)
\curveto(22.82082646,232.52771204)(22.82582645,232.57771199)(22.8158255,232.62771606)
\curveto(22.80582647,232.68771188)(22.81082647,232.74271182)(22.8308255,232.79271606)
\curveto(22.84082644,232.82271174)(22.84582643,232.85771171)(22.8458255,232.89771606)
\lineto(22.8458255,233.03271606)
\lineto(22.8458255,233.16771606)
\curveto(22.83582644,233.20771136)(22.83082645,233.2627113)(22.8308255,233.33271606)
\curveto(22.81082647,233.41271115)(22.79582648,233.49271107)(22.7858255,233.57271606)
\curveto(22.76582651,233.6627109)(22.74082654,233.74271082)(22.7108255,233.81271606)
\curveto(22.57082671,234.17271039)(22.39582688,234.47771009)(22.1858255,234.72771606)
\curveto(21.96582731,234.97770959)(21.69082759,235.20270936)(21.3608255,235.40271606)
\curveto(21.25082803,235.47270909)(21.14082814,235.52770904)(21.0308255,235.56771606)
\lineto(20.7008255,235.71771606)
\curveto(20.66082862,235.74770882)(20.62582865,235.7627088)(20.5958255,235.76271606)
\curveto(20.55582872,235.77270879)(20.51582876,235.78770878)(20.4758255,235.80771606)
\curveto(20.41582886,235.82770874)(20.35582892,235.84270872)(20.2958255,235.85271606)
\curveto(20.23582904,235.8627087)(20.1758291,235.87770869)(20.1158255,235.89771606)
}
}
{
\newrgbcolor{curcolor}{0 0 0}
\pscustom[linestyle=none,fillstyle=solid,fillcolor=curcolor]
{
\newpath
\moveto(19.5908255,245.91396606)
\curveto(19.69082959,245.91395756)(19.80582947,245.89395758)(19.9358255,245.85396606)
\curveto(20.05582922,245.81395766)(20.14082914,245.76395771)(20.1908255,245.70396606)
\curveto(20.23082905,245.64395783)(20.26082902,245.56395791)(20.2808255,245.46396606)
\curveto(20.29082899,245.36395811)(20.29582898,245.25395822)(20.2958255,245.13396606)
\lineto(20.2958255,244.77396606)
\curveto(20.28582899,244.66395881)(20.280829,244.56395891)(20.2808255,244.47396606)
\lineto(20.2808255,240.63396606)
\curveto(20.280829,240.55396292)(20.28582899,240.468963)(20.2958255,240.37896606)
\curveto(20.29582898,240.29896317)(20.31082897,240.23396324)(20.3408255,240.18396606)
\curveto(20.36082892,240.13396334)(20.40082888,240.08396339)(20.4608255,240.03396606)
\lineto(20.5958255,239.94396606)
\curveto(20.64582863,239.91396356)(20.69582858,239.90396357)(20.7458255,239.91396606)
\curveto(20.79582848,239.91396356)(20.84082844,239.90896356)(20.8808255,239.89896606)
\lineto(21.0008255,239.89896606)
\lineto(21.2558255,239.89896606)
\curveto(21.33582794,239.90896356)(21.41582786,239.92396355)(21.4958255,239.94396606)
\curveto(22.03582724,240.0739634)(22.42082686,240.37896309)(22.6508255,240.85896606)
\curveto(22.6808266,240.90896256)(22.70582657,240.9689625)(22.7258255,241.03896606)
\curveto(22.74582653,241.10896236)(22.76582651,241.1739623)(22.7858255,241.23396606)
\curveto(22.79582648,241.26396221)(22.80082648,241.31396216)(22.8008255,241.38396606)
\curveto(22.84082644,241.51396196)(22.86082642,241.69396178)(22.8608255,241.92396606)
\curveto(22.86082642,242.15396132)(22.84082644,242.34396113)(22.8008255,242.49396606)
\curveto(22.76082652,242.64396083)(22.72082656,242.77896069)(22.6808255,242.89896606)
\curveto(22.63082665,243.02896044)(22.57082671,243.14896032)(22.5008255,243.25896606)
\curveto(22.43082685,243.37896009)(22.35082693,243.48895998)(22.2608255,243.58896606)
\curveto(22.16082712,243.68895978)(22.05582722,243.77895969)(21.9458255,243.85896606)
\curveto(21.84582743,243.93895953)(21.74082754,244.01395946)(21.6308255,244.08396606)
\curveto(21.52082776,244.15395932)(21.44082784,244.24895922)(21.3908255,244.36896606)
\curveto(21.37082791,244.40895906)(21.35582792,244.473959)(21.3458255,244.56396606)
\curveto(21.33582794,244.66395881)(21.33582794,244.75395872)(21.3458255,244.83396606)
\curveto(21.34582793,244.92395855)(21.35082793,245.00895846)(21.3608255,245.08896606)
\curveto(21.37082791,245.1689583)(21.39082789,245.21895825)(21.4208255,245.23896606)
\curveto(21.49082779,245.32895814)(21.60582767,245.33395814)(21.7658255,245.25396606)
\curveto(22.03582724,245.11395836)(22.275827,244.95895851)(22.4858255,244.78896606)
\curveto(22.80582647,244.52895894)(23.07082621,244.24895922)(23.2808255,243.94896606)
\curveto(23.4808258,243.65895981)(23.64582563,243.30396017)(23.7758255,242.88396606)
\curveto(23.81582546,242.7739607)(23.84082544,242.6689608)(23.8508255,242.56896606)
\curveto(23.87082541,242.468961)(23.89082539,242.35896111)(23.9108255,242.23896606)
\curveto(23.92082536,242.18896128)(23.92582535,242.13896133)(23.9258255,242.08896606)
\curveto(23.92582535,242.04896142)(23.93082535,242.00396147)(23.9408255,241.95396606)
\lineto(23.9408255,241.80396606)
\curveto(23.95082533,241.75396172)(23.95582532,241.69396178)(23.9558255,241.62396606)
\curveto(23.95582532,241.56396191)(23.95082533,241.51396196)(23.9408255,241.47396606)
\lineto(23.9408255,241.33896606)
\curveto(23.93082535,241.28896218)(23.92582535,241.24396223)(23.9258255,241.20396606)
\curveto(23.92582535,241.16396231)(23.92082536,241.12396235)(23.9108255,241.08396606)
\curveto(23.90082538,241.03396244)(23.89082539,240.97896249)(23.8808255,240.91896606)
\curveto(23.8808254,240.8689626)(23.8758254,240.81896265)(23.8658255,240.76896606)
\curveto(23.84582543,240.67896279)(23.82082546,240.58896288)(23.7908255,240.49896606)
\curveto(23.77082551,240.41896305)(23.74582553,240.34396313)(23.7158255,240.27396606)
\curveto(23.69582558,240.23396324)(23.68582559,240.19896327)(23.6858255,240.16896606)
\curveto(23.6758256,240.13896333)(23.66082562,240.10896336)(23.6408255,240.07896606)
\curveto(23.57082571,239.93896353)(23.48582579,239.79396368)(23.3858255,239.64396606)
\curveto(23.19582608,239.39396408)(22.96582631,239.19396428)(22.6958255,239.04396606)
\curveto(22.41582686,238.89396458)(22.10582717,238.78396469)(21.7658255,238.71396606)
\curveto(21.65582762,238.68396479)(21.54082774,238.6689648)(21.4208255,238.66896606)
\curveto(21.30082798,238.6689648)(21.1808281,238.65896481)(21.0608255,238.63896606)
\lineto(20.9558255,238.63896606)
\curveto(20.92582835,238.64896482)(20.88582839,238.65396482)(20.8358255,238.65396606)
\lineto(20.5808255,238.65396606)
\curveto(20.49082879,238.66396481)(20.40082888,238.6689648)(20.3108255,238.66896606)
\lineto(20.1008255,238.71396606)
\curveto(20.06082922,238.71396476)(20.00582927,238.71896475)(19.9358255,238.72896606)
\curveto(19.85582942,238.73896473)(19.79082949,238.75396472)(19.7408255,238.77396606)
\lineto(19.5758255,238.80396606)
\curveto(19.52582975,238.83396464)(19.4758298,238.84896462)(19.4258255,238.84896606)
\curveto(19.36582991,238.85896461)(19.31082997,238.8739646)(19.2608255,238.89396606)
\curveto(19.10083018,238.96396451)(18.94083034,239.02896444)(18.7808255,239.08896606)
\curveto(18.62083066,239.14896432)(18.47083081,239.22396425)(18.3308255,239.31396606)
\curveto(18.22083106,239.38396409)(18.11083117,239.44896402)(18.0008255,239.50896606)
\curveto(17.8808314,239.57896389)(17.76583151,239.65896381)(17.6558255,239.74896606)
\curveto(17.30583197,240.03896343)(17.00583227,240.34896312)(16.7558255,240.67896606)
\curveto(16.49583278,241.00896246)(16.280833,241.39396208)(16.1108255,241.83396606)
\curveto(16.06083322,241.96396151)(16.02583325,242.09396138)(16.0058255,242.22396606)
\curveto(15.9758333,242.35396112)(15.94583333,242.49396098)(15.9158255,242.64396606)
\curveto(15.90583337,242.69396078)(15.90083338,242.73896073)(15.9008255,242.77896606)
\curveto(15.89083339,242.81896065)(15.88583339,242.86396061)(15.8858255,242.91396606)
\curveto(15.8758334,242.93396054)(15.8758334,242.95896051)(15.8858255,242.98896606)
\curveto(15.89583338,243.01896045)(15.89083339,243.04396043)(15.8708255,243.06396606)
\curveto(15.86083342,243.49395998)(15.90583337,243.85395962)(16.0058255,244.14396606)
\curveto(16.09583318,244.43395904)(16.22083306,244.68895878)(16.3808255,244.90896606)
\curveto(16.40083288,244.94895852)(16.43083285,244.97895849)(16.4708255,244.99896606)
\curveto(16.50083278,245.02895844)(16.52583275,245.05895841)(16.5458255,245.08896606)
\curveto(16.60583267,245.15895831)(16.6758326,245.22895824)(16.7558255,245.29896606)
\curveto(16.83583244,245.3689581)(16.91583236,245.42395805)(16.9958255,245.46396606)
\curveto(17.20583207,245.58395789)(17.40583187,245.67895779)(17.5958255,245.74896606)
\curveto(17.70583157,245.79895767)(17.82583145,245.82895764)(17.9558255,245.83896606)
\lineto(18.3458255,245.89896606)
\curveto(18.4758308,245.92895754)(18.61083067,245.93895753)(18.7508255,245.92896606)
\curveto(18.89083039,245.92895754)(19.03083025,245.93395754)(19.1708255,245.94396606)
\curveto(19.24083004,245.94395753)(19.31082997,245.93895753)(19.3808255,245.92896606)
\curveto(19.45082983,245.91895755)(19.52082976,245.91395756)(19.5908255,245.91396606)
\moveto(19.0808255,244.56396606)
\curveto(19.04083024,244.59395888)(18.99083029,244.62395885)(18.9308255,244.65396606)
\curveto(18.86083042,244.69395878)(18.79083049,244.70895876)(18.7208255,244.69896606)
\curveto(18.50083078,244.68895878)(18.29583098,244.64895882)(18.1058255,244.57896606)
\curveto(17.8758314,244.47895899)(17.6808316,244.35895911)(17.5208255,244.21896606)
\curveto(17.36083192,244.08895938)(17.22583205,243.89895957)(17.1158255,243.64896606)
\curveto(17.09583218,243.57895989)(17.0808322,243.50895996)(17.0708255,243.43896606)
\curveto(17.05083223,243.37896009)(17.03083225,243.30896016)(17.0108255,243.22896606)
\curveto(16.99083229,243.15896031)(16.9808323,243.07896039)(16.9808255,242.98896606)
\lineto(16.9808255,242.73396606)
\curveto(17.00083228,242.69396078)(17.01083227,242.65396082)(17.0108255,242.61396606)
\curveto(17.00083228,242.5739609)(17.00083228,242.53896093)(17.0108255,242.50896606)
\lineto(17.0708255,242.26896606)
\curveto(17.0808322,242.18896128)(17.09583218,242.11396136)(17.1158255,242.04396606)
\curveto(17.23583204,241.72396175)(17.38583189,241.45896201)(17.5658255,241.24896606)
\curveto(17.74583153,241.03896243)(17.97083131,240.83896263)(18.2408255,240.64896606)
\curveto(18.29083099,240.60896286)(18.35583092,240.56396291)(18.4358255,240.51396606)
\curveto(18.50583077,240.473963)(18.58583069,240.43396304)(18.6758255,240.39396606)
\curveto(18.76583051,240.35396312)(18.85083043,240.32896314)(18.9308255,240.31896606)
\curveto(19.01083027,240.31896315)(19.07083021,240.34396313)(19.1108255,240.39396606)
\curveto(19.17083011,240.46396301)(19.20083008,240.59396288)(19.2008255,240.78396606)
\curveto(19.19083009,240.98396249)(19.18583009,241.15396232)(19.1858255,241.29396606)
\lineto(19.1858255,243.57396606)
\curveto(19.18583009,243.72395975)(19.19083009,243.90395957)(19.2008255,244.11396606)
\curveto(19.20083008,244.32395915)(19.16083012,244.473959)(19.0808255,244.56396606)
}
}
{
\newrgbcolor{curcolor}{0 0 0}
\pscustom[linestyle=none,fillstyle=solid,fillcolor=curcolor]
{
}
}
{
\newrgbcolor{curcolor}{0 0 0}
\pscustom[linestyle=none,fillstyle=solid,fillcolor=curcolor]
{
\newpath
\moveto(16.0358255,252.55076294)
\lineto(16.0358255,252.98576294)
\curveto(16.03583324,253.13575943)(16.0758332,253.23575933)(16.1558255,253.28576294)
\curveto(16.23583304,253.31575925)(16.33583294,253.32075924)(16.4558255,253.30076294)
\lineto(16.8158255,253.24076294)
\lineto(18.2408255,252.95576294)
\lineto(20.5058255,252.50576294)
\curveto(20.72582855,252.45576011)(20.95582832,252.40576016)(21.1958255,252.35576294)
\curveto(21.42582785,252.31576025)(21.62582765,252.30076026)(21.7958255,252.31076294)
\curveto(22.24582703,252.38076018)(22.56082672,252.62075994)(22.7408255,253.03076294)
\curveto(22.83082645,253.23075933)(22.86582641,253.48575908)(22.8458255,253.79576294)
\curveto(22.81582646,254.11575845)(22.76082652,254.38075818)(22.6808255,254.59076294)
\curveto(22.54082674,254.94075762)(22.36582691,255.23575733)(22.1558255,255.47576294)
\curveto(21.93582734,255.71575685)(21.65082763,255.92575664)(21.3008255,256.10576294)
\curveto(21.22082806,256.15575641)(21.14082814,256.19075637)(21.0608255,256.21076294)
\curveto(20.9808283,256.24075632)(20.89582838,256.27575629)(20.8058255,256.31576294)
\curveto(20.75582852,256.33575623)(20.71082857,256.34575622)(20.6708255,256.34576294)
\curveto(20.63082865,256.34575622)(20.58582869,256.3607562)(20.5358255,256.39076294)
\lineto(20.2208255,256.45076294)
\curveto(20.14082914,256.49075607)(20.05082923,256.51575605)(19.9508255,256.52576294)
\curveto(19.84082944,256.53575603)(19.74082954,256.55075601)(19.6508255,256.57076294)
\lineto(18.4808255,256.81076294)
\lineto(16.8908255,257.12576294)
\curveto(16.77083251,257.14575542)(16.64583263,257.1657554)(16.5158255,257.18576294)
\curveto(16.3758329,257.21575535)(16.26583301,257.2607553)(16.1858255,257.32076294)
\curveto(16.13583314,257.37075519)(16.10583317,257.42575514)(16.0958255,257.48576294)
\curveto(16.0758332,257.54575502)(16.05583322,257.61575495)(16.0358255,257.69576294)
\lineto(16.0358255,257.92076294)
\curveto(16.03583324,258.04075452)(16.04083324,258.14575442)(16.0508255,258.23576294)
\curveto(16.06083322,258.33575423)(16.10583317,258.40075416)(16.1858255,258.43076294)
\curveto(16.23583304,258.47075409)(16.31083297,258.48075408)(16.4108255,258.46076294)
\curveto(16.50083278,258.44075412)(16.59583268,258.42075414)(16.6958255,258.40076294)
\lineto(17.7158255,258.20576294)
\lineto(21.7508255,257.39576294)
\lineto(23.1008255,257.12576294)
\curveto(23.22082606,257.10575546)(23.33582594,257.08075548)(23.4458255,257.05076294)
\curveto(23.54582573,257.02075554)(23.62082566,256.9657556)(23.6708255,256.88576294)
\curveto(23.70082558,256.84575572)(23.72582555,256.78075578)(23.7458255,256.69076294)
\curveto(23.75582552,256.61075595)(23.76582551,256.52075604)(23.7758255,256.42076294)
\curveto(23.7758255,256.33075623)(23.77082551,256.24075632)(23.7608255,256.15076294)
\curveto(23.75082553,256.07075649)(23.73082555,256.01575655)(23.7008255,255.98576294)
\curveto(23.66082562,255.94575662)(23.59582568,255.91575665)(23.5058255,255.89576294)
\curveto(23.46582581,255.88575668)(23.41082587,255.88575668)(23.3408255,255.89576294)
\curveto(23.27082601,255.90575666)(23.20582607,255.91075665)(23.1458255,255.91076294)
\curveto(23.0758262,255.92075664)(23.02082626,255.91075665)(22.9808255,255.88076294)
\curveto(22.94082634,255.8607567)(22.92582635,255.82075674)(22.9358255,255.76076294)
\curveto(22.95582632,255.68075688)(23.01582626,255.59075697)(23.1158255,255.49076294)
\curveto(23.20582607,255.39075717)(23.275826,255.30075726)(23.3258255,255.22076294)
\curveto(23.48582579,254.97075759)(23.62582565,254.69075787)(23.7458255,254.38076294)
\curveto(23.79582548,254.2607583)(23.82582545,254.14075842)(23.8358255,254.02076294)
\curveto(23.85582542,253.91075865)(23.8808254,253.79075877)(23.9108255,253.66076294)
\curveto(23.92082536,253.61075895)(23.92082536,253.55575901)(23.9108255,253.49576294)
\curveto(23.90082538,253.44575912)(23.90582537,253.39575917)(23.9258255,253.34576294)
\curveto(23.94582533,253.24575932)(23.94582533,253.15575941)(23.9258255,253.07576294)
\lineto(23.9258255,252.92576294)
\curveto(23.90582537,252.87575969)(23.89582538,252.81575975)(23.8958255,252.74576294)
\curveto(23.89582538,252.68575988)(23.89082539,252.63575993)(23.8808255,252.59576294)
\curveto(23.86082542,252.55576001)(23.85082543,252.51576005)(23.8508255,252.47576294)
\curveto(23.86082542,252.44576012)(23.85582542,252.40576016)(23.8358255,252.35576294)
\curveto(23.81582546,252.28576028)(23.79582548,252.21076035)(23.7758255,252.13076294)
\curveto(23.75582552,252.0607605)(23.72582555,251.99076057)(23.6858255,251.92076294)
\curveto(23.5758257,251.68076088)(23.43082585,251.49076107)(23.2508255,251.35076294)
\curveto(23.06082622,251.22076134)(22.83582644,251.12576144)(22.5758255,251.06576294)
\curveto(22.48582679,251.04576152)(22.39582688,251.03576153)(22.3058255,251.03576294)
\lineto(22.0058255,251.03576294)
\curveto(21.94582733,251.02576154)(21.89082739,251.02576154)(21.8408255,251.03576294)
\curveto(21.7808275,251.05576151)(21.71582756,251.0607615)(21.6458255,251.05076294)
\lineto(21.5708255,251.05076294)
\curveto(21.53082775,251.0607615)(21.49582778,251.0657615)(21.4658255,251.06576294)
\lineto(21.3158255,251.09576294)
\curveto(21.275828,251.09576147)(21.23082805,251.10076146)(21.1808255,251.11076294)
\curveto(21.12082816,251.13076143)(21.06582821,251.14576142)(21.0158255,251.15576294)
\lineto(20.4158255,251.27576294)
\lineto(17.6558255,251.83076294)
\lineto(16.6958255,252.01076294)
\lineto(16.4258255,252.07076294)
\curveto(16.33583294,252.09076047)(16.26083302,252.12576044)(16.2008255,252.17576294)
\curveto(16.13083315,252.22576034)(16.0808332,252.31076025)(16.0508255,252.43076294)
\curveto(16.04083324,252.45076011)(16.04083324,252.47076009)(16.0508255,252.49076294)
\curveto(16.05083323,252.51076005)(16.04583323,252.53076003)(16.0358255,252.55076294)
}
}
{
\newrgbcolor{curcolor}{0 0 0}
\pscustom[linestyle=none,fillstyle=solid,fillcolor=curcolor]
{
\newpath
\moveto(15.8558255,262.91037231)
\curveto(15.84583343,263.63036666)(15.93083335,264.21536607)(16.1108255,264.66537231)
\curveto(16.280833,265.12536516)(16.58583269,265.44536484)(17.0258255,265.62537231)
\curveto(17.13583214,265.67536461)(17.25083203,265.70536458)(17.3708255,265.71537231)
\curveto(17.4808318,265.73536455)(17.60583167,265.75036454)(17.7458255,265.76037231)
\curveto(17.81583146,265.77036452)(17.89083139,265.76036453)(17.9708255,265.73037231)
\curveto(18.04083124,265.71036458)(18.09583118,265.6853646)(18.1358255,265.65537231)
\curveto(18.15583112,265.63536465)(18.1758311,265.60536468)(18.1958255,265.56537231)
\curveto(18.20583107,265.53536475)(18.22083106,265.51036478)(18.2408255,265.49037231)
\curveto(18.26083102,265.43036486)(18.26583101,265.37536491)(18.2558255,265.32537231)
\curveto(18.24583103,265.285365)(18.24583103,265.24036505)(18.2558255,265.19037231)
\curveto(18.275831,265.10036519)(18.280831,264.9903653)(18.2708255,264.86037231)
\curveto(18.25083103,264.74036555)(18.22583105,264.65536563)(18.1958255,264.60537231)
\curveto(18.14583113,264.53536575)(18.0808312,264.49536579)(18.0008255,264.48537231)
\curveto(17.91083137,264.4853658)(17.82583145,264.46536582)(17.7458255,264.42537231)
\curveto(17.58583169,264.37536591)(17.44083184,264.28036601)(17.3108255,264.14037231)
\curveto(17.23083205,264.05036624)(17.17083211,263.94036635)(17.1308255,263.81037231)
\curveto(17.09083219,263.6903666)(17.05083223,263.56036673)(17.0108255,263.42037231)
\curveto(16.99083229,263.38036691)(16.98583229,263.33036696)(16.9958255,263.27037231)
\curveto(16.99583228,263.22036707)(16.99083229,263.17536711)(16.9808255,263.13537231)
\curveto(16.96083232,263.07536721)(16.95083233,263.00036729)(16.9508255,262.91037231)
\curveto(16.95083233,262.82036747)(16.96083232,262.74536754)(16.9808255,262.68537231)
\lineto(16.9808255,262.59537231)
\curveto(16.99083229,262.53536775)(17.00083228,262.48036781)(17.0108255,262.43037231)
\curveto(17.01083227,262.38036791)(17.01583226,262.33036796)(17.0258255,262.28037231)
\curveto(17.08583219,262.01036828)(17.17083211,261.77536851)(17.2808255,261.57537231)
\curveto(17.39083189,261.3853689)(17.5758317,261.23536905)(17.8358255,261.12537231)
\curveto(17.90583137,261.09536919)(17.9758313,261.08036921)(18.0458255,261.08037231)
\curveto(18.11583116,261.08036921)(18.1758311,261.0853692)(18.2258255,261.09537231)
\curveto(18.3758309,261.12536916)(18.48583079,261.17536911)(18.5558255,261.24537231)
\curveto(18.61583066,261.31536897)(18.68583059,261.41036888)(18.7658255,261.53037231)
\curveto(18.86583041,261.67036862)(18.94083034,261.83536845)(18.9908255,262.02537231)
\curveto(19.03083025,262.21536807)(19.0808302,262.40536788)(19.1408255,262.59537231)
\curveto(19.1808301,262.71536757)(19.21083007,262.83536745)(19.2308255,262.95537231)
\curveto(19.25083003,263.0853672)(19.28083,263.21036708)(19.3208255,263.33037231)
\curveto(19.3808299,263.53036676)(19.44082984,263.72536656)(19.5008255,263.91537231)
\curveto(19.55082973,264.10536618)(19.61582966,264.290366)(19.6958255,264.47037231)
\curveto(19.71582956,264.52036577)(19.73582954,264.56536572)(19.7558255,264.60537231)
\curveto(19.7758295,264.65536563)(19.80082948,264.70536558)(19.8308255,264.75537231)
\curveto(19.95082933,264.92536536)(20.08582919,265.07036522)(20.2358255,265.19037231)
\curveto(20.38582889,265.31036498)(20.5758287,265.40036489)(20.8058255,265.46037231)
\lineto(21.0908255,265.46037231)
\curveto(21.16082812,265.46036483)(21.23582804,265.45536483)(21.3158255,265.44537231)
\curveto(21.38582789,265.43536485)(21.46582781,265.42536486)(21.5558255,265.41537231)
\lineto(21.7058255,265.38537231)
\curveto(21.7758275,265.34536494)(21.84582743,265.31536497)(21.9158255,265.29537231)
\curveto(21.98582729,265.285365)(22.05582722,265.26536502)(22.1258255,265.23537231)
\curveto(22.23582704,265.1853651)(22.34082694,265.13036516)(22.4408255,265.07037231)
\curveto(22.54082674,265.01036528)(22.63082665,264.94536534)(22.7108255,264.87537231)
\curveto(22.97082631,264.66536562)(23.1808261,264.42036587)(23.3408255,264.14037231)
\curveto(23.49082579,263.86036643)(23.62082566,263.55536673)(23.7308255,263.22537231)
\curveto(23.76082552,263.12536716)(23.7808255,263.02536726)(23.7908255,262.92537231)
\curveto(23.81082547,262.82536746)(23.83582544,262.73036756)(23.8658255,262.64037231)
\curveto(23.88582539,262.53036776)(23.89582538,262.42536786)(23.8958255,262.32537231)
\curveto(23.89582538,262.22536806)(23.90582537,262.12536816)(23.9258255,262.02537231)
\lineto(23.9258255,261.87537231)
\curveto(23.93582534,261.82536846)(23.94082534,261.75536853)(23.9408255,261.66537231)
\curveto(23.94082534,261.57536871)(23.93582534,261.50536878)(23.9258255,261.45537231)
\lineto(23.9258255,261.29037231)
\curveto(23.90582537,261.23036906)(23.89582538,261.16536912)(23.8958255,261.09537231)
\curveto(23.90582537,261.02536926)(23.90082538,260.97036932)(23.8808255,260.93037231)
\curveto(23.87082541,260.88036941)(23.86582541,260.81536947)(23.8658255,260.73537231)
\curveto(23.84582543,260.65536963)(23.82582545,260.58036971)(23.8058255,260.51037231)
\curveto(23.79582548,260.44036985)(23.7758255,260.36536992)(23.7458255,260.28537231)
\curveto(23.64582563,259.99537029)(23.52082576,259.75037054)(23.3708255,259.55037231)
\curveto(23.22082606,259.35037094)(23.02582625,259.1903711)(22.7858255,259.07037231)
\curveto(22.65582662,259.01037128)(22.52082676,258.96037133)(22.3808255,258.92037231)
\curveto(22.24082704,258.8903714)(22.08582719,258.87037142)(21.9158255,258.86037231)
\curveto(21.85582742,258.85037144)(21.78582749,258.85537143)(21.7058255,258.87537231)
\curveto(21.61582766,258.89537139)(21.54582773,258.92037137)(21.4958255,258.95037231)
\curveto(21.45582782,258.9903713)(21.41582786,259.05037124)(21.3758255,259.13037231)
\curveto(21.35582792,259.18037111)(21.34582793,259.25037104)(21.3458255,259.34037231)
\curveto(21.33582794,259.44037085)(21.33582794,259.53037076)(21.3458255,259.61037231)
\curveto(21.35582792,259.70037059)(21.37082791,259.7853705)(21.3908255,259.86537231)
\curveto(21.40082788,259.95537033)(21.41582786,260.01037028)(21.4358255,260.03037231)
\curveto(21.48582779,260.0903702)(21.56082772,260.12037017)(21.6608255,260.12037231)
\curveto(21.75082753,260.13037016)(21.83582744,260.15037014)(21.9158255,260.18037231)
\curveto(22.13582714,260.23037006)(22.30582697,260.33036996)(22.4258255,260.48037231)
\curveto(22.51582676,260.58036971)(22.58582669,260.70036959)(22.6358255,260.84037231)
\curveto(22.68582659,260.98036931)(22.73582654,261.13036916)(22.7858255,261.29037231)
\lineto(22.8308255,261.60537231)
\lineto(22.8308255,261.69537231)
\curveto(22.85082643,261.75536853)(22.86082642,261.84036845)(22.8608255,261.95037231)
\curveto(22.86082642,262.07036822)(22.85082643,262.17536811)(22.8308255,262.26537231)
\curveto(22.83082645,262.33536795)(22.82582645,262.3903679)(22.8158255,262.43037231)
\curveto(22.80582647,262.4903678)(22.80082648,262.55036774)(22.8008255,262.61037231)
\curveto(22.79082649,262.67036762)(22.7808265,262.72536756)(22.7708255,262.77537231)
\curveto(22.69082659,263.0853672)(22.58582669,263.33536695)(22.4558255,263.52537231)
\curveto(22.32582695,263.72536656)(22.10582717,263.8903664)(21.7958255,264.02037231)
\curveto(21.74582753,264.05036624)(21.69082759,264.06536622)(21.6308255,264.06537231)
\curveto(21.57082771,264.07536621)(21.52582775,264.07536621)(21.4958255,264.06537231)
\curveto(21.30582797,264.05536623)(21.16582811,264.01536627)(21.0758255,263.94537231)
\curveto(20.9758283,263.87536641)(20.88582839,263.78036651)(20.8058255,263.66037231)
\curveto(20.74582853,263.58036671)(20.69582858,263.4853668)(20.6558255,263.37537231)
\lineto(20.5358255,263.07537231)
\curveto(20.52582875,263.04536724)(20.52082876,263.01536727)(20.5208255,262.98537231)
\curveto(20.52082876,262.96536732)(20.51082877,262.94536734)(20.4908255,262.92537231)
\curveto(20.3808289,262.60536768)(20.30082898,262.26536802)(20.2508255,261.90537231)
\curveto(20.19082909,261.55536873)(20.09582918,261.23536905)(19.9658255,260.94537231)
\curveto(19.92582935,260.85536943)(19.89082939,260.76536952)(19.8608255,260.67537231)
\curveto(19.83082945,260.59536969)(19.79082949,260.52036977)(19.7408255,260.45037231)
\curveto(19.63082965,260.28037001)(19.50582977,260.13037016)(19.3658255,260.00037231)
\curveto(19.22583005,259.87037042)(19.05083023,259.78037051)(18.8408255,259.73037231)
\curveto(18.77083051,259.71037058)(18.70083058,259.70037059)(18.6308255,259.70037231)
\lineto(18.4058255,259.70037231)
\curveto(18.28583099,259.6903706)(18.15083113,259.70537058)(18.0008255,259.74537231)
\curveto(17.84083144,259.7853705)(17.70583157,259.82537046)(17.5958255,259.86537231)
\curveto(17.54583173,259.89537039)(17.50583177,259.91537037)(17.4758255,259.92537231)
\curveto(17.43583184,259.94537034)(17.39583188,259.97037032)(17.3558255,260.00037231)
\curveto(17.12583215,260.13037016)(16.92583235,260.29037)(16.7558255,260.48037231)
\curveto(16.58583269,260.67036962)(16.43583284,260.88036941)(16.3058255,261.11037231)
\curveto(16.21583306,261.27036902)(16.14583313,261.44536884)(16.0958255,261.63537231)
\curveto(16.03583324,261.83536845)(15.9808333,262.04036825)(15.9308255,262.25037231)
\curveto(15.92083336,262.32036797)(15.91083337,262.3853679)(15.9008255,262.44537231)
\curveto(15.89083339,262.51536777)(15.8808334,262.5903677)(15.8708255,262.67037231)
\curveto(15.86083342,262.71036758)(15.86083342,262.75036754)(15.8708255,262.79037231)
\curveto(15.8808334,262.84036745)(15.8758334,262.88036741)(15.8558255,262.91037231)
}
}
{
\newrgbcolor{curcolor}{0 0 0}
\pscustom[linestyle=none,fillstyle=solid,fillcolor=curcolor]
{
\newpath
\moveto(16.0358255,268.40037231)
\lineto(16.0358255,268.83537231)
\curveto(16.03583324,268.9853688)(16.0758332,269.0853687)(16.1558255,269.13537231)
\curveto(16.23583304,269.16536862)(16.33583294,269.17036862)(16.4558255,269.15037231)
\lineto(16.8158255,269.09037231)
\lineto(18.2408255,268.80537231)
\lineto(20.5058255,268.35537231)
\curveto(20.72582855,268.30536948)(20.95582832,268.25536953)(21.1958255,268.20537231)
\curveto(21.42582785,268.16536962)(21.62582765,268.15036964)(21.7958255,268.16037231)
\curveto(22.24582703,268.23036956)(22.56082672,268.47036932)(22.7408255,268.88037231)
\curveto(22.83082645,269.08036871)(22.86582641,269.33536845)(22.8458255,269.64537231)
\curveto(22.81582646,269.96536782)(22.76082652,270.23036756)(22.6808255,270.44037231)
\curveto(22.54082674,270.790367)(22.36582691,271.0853667)(22.1558255,271.32537231)
\curveto(21.93582734,271.56536622)(21.65082763,271.77536601)(21.3008255,271.95537231)
\curveto(21.22082806,272.00536578)(21.14082814,272.04036575)(21.0608255,272.06037231)
\curveto(20.9808283,272.0903657)(20.89582838,272.12536566)(20.8058255,272.16537231)
\curveto(20.75582852,272.1853656)(20.71082857,272.19536559)(20.6708255,272.19537231)
\curveto(20.63082865,272.19536559)(20.58582869,272.21036558)(20.5358255,272.24037231)
\lineto(20.2208255,272.30037231)
\curveto(20.14082914,272.34036545)(20.05082923,272.36536542)(19.9508255,272.37537231)
\curveto(19.84082944,272.3853654)(19.74082954,272.40036539)(19.6508255,272.42037231)
\lineto(18.4808255,272.66037231)
\lineto(16.8908255,272.97537231)
\curveto(16.77083251,272.99536479)(16.64583263,273.01536477)(16.5158255,273.03537231)
\curveto(16.3758329,273.06536472)(16.26583301,273.11036468)(16.1858255,273.17037231)
\curveto(16.13583314,273.22036457)(16.10583317,273.27536451)(16.0958255,273.33537231)
\curveto(16.0758332,273.39536439)(16.05583322,273.46536432)(16.0358255,273.54537231)
\lineto(16.0358255,273.77037231)
\curveto(16.03583324,273.8903639)(16.04083324,273.99536379)(16.0508255,274.08537231)
\curveto(16.06083322,274.1853636)(16.10583317,274.25036354)(16.1858255,274.28037231)
\curveto(16.23583304,274.32036347)(16.31083297,274.33036346)(16.4108255,274.31037231)
\curveto(16.50083278,274.2903635)(16.59583268,274.27036352)(16.6958255,274.25037231)
\lineto(17.7158255,274.05537231)
\lineto(21.7508255,273.24537231)
\lineto(23.1008255,272.97537231)
\curveto(23.22082606,272.95536483)(23.33582594,272.93036486)(23.4458255,272.90037231)
\curveto(23.54582573,272.87036492)(23.62082566,272.81536497)(23.6708255,272.73537231)
\curveto(23.70082558,272.69536509)(23.72582555,272.63036516)(23.7458255,272.54037231)
\curveto(23.75582552,272.46036533)(23.76582551,272.37036542)(23.7758255,272.27037231)
\curveto(23.7758255,272.18036561)(23.77082551,272.0903657)(23.7608255,272.00037231)
\curveto(23.75082553,271.92036587)(23.73082555,271.86536592)(23.7008255,271.83537231)
\curveto(23.66082562,271.79536599)(23.59582568,271.76536602)(23.5058255,271.74537231)
\curveto(23.46582581,271.73536605)(23.41082587,271.73536605)(23.3408255,271.74537231)
\curveto(23.27082601,271.75536603)(23.20582607,271.76036603)(23.1458255,271.76037231)
\curveto(23.0758262,271.77036602)(23.02082626,271.76036603)(22.9808255,271.73037231)
\curveto(22.94082634,271.71036608)(22.92582635,271.67036612)(22.9358255,271.61037231)
\curveto(22.95582632,271.53036626)(23.01582626,271.44036635)(23.1158255,271.34037231)
\curveto(23.20582607,271.24036655)(23.275826,271.15036664)(23.3258255,271.07037231)
\curveto(23.48582579,270.82036697)(23.62582565,270.54036725)(23.7458255,270.23037231)
\curveto(23.79582548,270.11036768)(23.82582545,269.9903678)(23.8358255,269.87037231)
\curveto(23.85582542,269.76036803)(23.8808254,269.64036815)(23.9108255,269.51037231)
\curveto(23.92082536,269.46036833)(23.92082536,269.40536838)(23.9108255,269.34537231)
\curveto(23.90082538,269.29536849)(23.90582537,269.24536854)(23.9258255,269.19537231)
\curveto(23.94582533,269.09536869)(23.94582533,269.00536878)(23.9258255,268.92537231)
\lineto(23.9258255,268.77537231)
\curveto(23.90582537,268.72536906)(23.89582538,268.66536912)(23.8958255,268.59537231)
\curveto(23.89582538,268.53536925)(23.89082539,268.4853693)(23.8808255,268.44537231)
\curveto(23.86082542,268.40536938)(23.85082543,268.36536942)(23.8508255,268.32537231)
\curveto(23.86082542,268.29536949)(23.85582542,268.25536953)(23.8358255,268.20537231)
\curveto(23.81582546,268.13536965)(23.79582548,268.06036973)(23.7758255,267.98037231)
\curveto(23.75582552,267.91036988)(23.72582555,267.84036995)(23.6858255,267.77037231)
\curveto(23.5758257,267.53037026)(23.43082585,267.34037045)(23.2508255,267.20037231)
\curveto(23.06082622,267.07037072)(22.83582644,266.97537081)(22.5758255,266.91537231)
\curveto(22.48582679,266.89537089)(22.39582688,266.8853709)(22.3058255,266.88537231)
\lineto(22.0058255,266.88537231)
\curveto(21.94582733,266.87537091)(21.89082739,266.87537091)(21.8408255,266.88537231)
\curveto(21.7808275,266.90537088)(21.71582756,266.91037088)(21.6458255,266.90037231)
\lineto(21.5708255,266.90037231)
\curveto(21.53082775,266.91037088)(21.49582778,266.91537087)(21.4658255,266.91537231)
\lineto(21.3158255,266.94537231)
\curveto(21.275828,266.94537084)(21.23082805,266.95037084)(21.1808255,266.96037231)
\curveto(21.12082816,266.98037081)(21.06582821,266.99537079)(21.0158255,267.00537231)
\lineto(20.4158255,267.12537231)
\lineto(17.6558255,267.68037231)
\lineto(16.6958255,267.86037231)
\lineto(16.4258255,267.92037231)
\curveto(16.33583294,267.94036985)(16.26083302,267.97536981)(16.2008255,268.02537231)
\curveto(16.13083315,268.07536971)(16.0808332,268.16036963)(16.0508255,268.28037231)
\curveto(16.04083324,268.30036949)(16.04083324,268.32036947)(16.0508255,268.34037231)
\curveto(16.05083323,268.36036943)(16.04583323,268.38036941)(16.0358255,268.40037231)
}
}
{
\newrgbcolor{curcolor}{0 0 0}
\pscustom[linestyle=none,fillstyle=solid,fillcolor=curcolor]
{
\newpath
\moveto(23.2058255,281.74498169)
\curveto(23.36582591,281.73497378)(23.50082578,281.68997382)(23.6108255,281.60998169)
\curveto(23.71082557,281.52997398)(23.78582549,281.43497408)(23.8358255,281.32498169)
\curveto(23.85582542,281.27497424)(23.86582541,281.21997429)(23.8658255,281.15998169)
\curveto(23.86582541,281.1099744)(23.8758254,281.04997446)(23.8958255,280.97998169)
\curveto(23.94582533,280.74997476)(23.93082535,280.53497498)(23.8508255,280.33498169)
\curveto(23.7808255,280.13497538)(23.69082559,280.0099755)(23.5808255,279.95998169)
\curveto(23.51082577,279.91997559)(23.43082585,279.88997562)(23.3408255,279.86998169)
\curveto(23.24082604,279.84997566)(23.16082612,279.8149757)(23.1008255,279.76498169)
\lineto(23.0408255,279.70498169)
\curveto(23.02082626,279.68497583)(23.01582626,279.65497586)(23.0258255,279.61498169)
\curveto(23.05582622,279.49497602)(23.11082617,279.37997613)(23.1908255,279.26998169)
\curveto(23.27082601,279.15997635)(23.34082594,279.05497646)(23.4008255,278.95498169)
\curveto(23.4808258,278.80497671)(23.55582572,278.64997686)(23.6258255,278.48998169)
\curveto(23.68582559,278.32997718)(23.74082554,278.15997735)(23.7908255,277.97998169)
\curveto(23.82082546,277.86997764)(23.84082544,277.75497776)(23.8508255,277.63498169)
\curveto(23.86082542,277.52497799)(23.8758254,277.4099781)(23.8958255,277.28998169)
\curveto(23.90582537,277.23997827)(23.91082537,277.19497832)(23.9108255,277.15498169)
\lineto(23.9108255,277.04998169)
\curveto(23.93082535,276.93997857)(23.93082535,276.83497868)(23.9108255,276.73498169)
\lineto(23.9108255,276.59998169)
\curveto(23.90082538,276.54997896)(23.89582538,276.49997901)(23.8958255,276.44998169)
\curveto(23.89582538,276.39997911)(23.88582539,276.35997915)(23.8658255,276.32998169)
\curveto(23.85582542,276.28997922)(23.85082543,276.25497926)(23.8508255,276.22498169)
\curveto(23.86082542,276.20497931)(23.86082542,276.17997933)(23.8508255,276.14998169)
\lineto(23.7908255,275.90998169)
\curveto(23.7808255,275.83997967)(23.76082552,275.77497974)(23.7308255,275.71498169)
\curveto(23.60082568,275.43498008)(23.45582582,275.21998029)(23.2958255,275.06998169)
\curveto(23.12582615,274.91998059)(22.89082639,274.8149807)(22.5908255,274.75498169)
\curveto(22.37082691,274.70498081)(22.10582717,274.7099808)(21.7958255,274.76998169)
\lineto(21.4808255,274.84498169)
\curveto(21.43082785,274.86498065)(21.3808279,274.87998063)(21.3308255,274.88998169)
\lineto(21.1508255,274.94998169)
\lineto(20.8208255,275.12998169)
\curveto(20.71082857,275.19998031)(20.61082867,275.26998024)(20.5208255,275.33998169)
\curveto(20.23082905,275.57997993)(20.01582926,275.86997964)(19.8758255,276.20998169)
\curveto(19.73582954,276.54997896)(19.61082967,276.9149786)(19.5008255,277.30498169)
\curveto(19.46082982,277.45497806)(19.43082985,277.60497791)(19.4108255,277.75498169)
\curveto(19.39082989,277.9149776)(19.36582991,278.06997744)(19.3358255,278.21998169)
\curveto(19.31582996,278.29997721)(19.30582997,278.36997714)(19.3058255,278.42998169)
\curveto(19.30582997,278.49997701)(19.29582998,278.57497694)(19.2758255,278.65498169)
\curveto(19.25583002,278.72497679)(19.24583003,278.79497672)(19.2458255,278.86498169)
\curveto(19.23583004,278.94497657)(19.22083006,279.02497649)(19.2008255,279.10498169)
\curveto(19.14083014,279.36497615)(19.09083019,279.6099759)(19.0508255,279.83998169)
\curveto(19.00083028,280.06997544)(18.88583039,280.26997524)(18.7058255,280.43998169)
\curveto(18.62583065,280.509975)(18.52583075,280.57497494)(18.4058255,280.63498169)
\curveto(18.275831,280.70497481)(18.13583114,280.73497478)(17.9858255,280.72498169)
\curveto(17.74583153,280.7149748)(17.55583172,280.66497485)(17.4158255,280.57498169)
\curveto(17.275832,280.49497502)(17.16583211,280.35497516)(17.0858255,280.15498169)
\curveto(17.03583224,280.04497547)(17.00083228,279.9099756)(16.9808255,279.74998169)
\curveto(16.96083232,279.58997592)(16.95083233,279.41997609)(16.9508255,279.23998169)
\curveto(16.95083233,279.05997645)(16.96083232,278.87997663)(16.9808255,278.69998169)
\curveto(17.00083228,278.52997698)(17.03083225,278.37997713)(17.0708255,278.24998169)
\curveto(17.13083215,278.06997744)(17.21583206,277.88997762)(17.3258255,277.70998169)
\curveto(17.38583189,277.61997789)(17.46583181,277.52997798)(17.5658255,277.43998169)
\curveto(17.65583162,277.35997815)(17.75583152,277.28497823)(17.8658255,277.21498169)
\curveto(17.94583133,277.16497835)(18.03083125,277.11997839)(18.1208255,277.07998169)
\curveto(18.21083107,277.03997847)(18.280831,276.97997853)(18.3308255,276.89998169)
\curveto(18.36083092,276.84997866)(18.38583089,276.77497874)(18.4058255,276.67498169)
\curveto(18.41583086,276.57497894)(18.42083086,276.47497904)(18.4208255,276.37498169)
\curveto(18.42083086,276.27497924)(18.41583086,276.17997933)(18.4058255,276.08998169)
\curveto(18.38583089,275.99997951)(18.36083092,275.93997957)(18.3308255,275.90998169)
\curveto(18.30083098,275.86997964)(18.25083103,275.84497967)(18.1808255,275.83498169)
\curveto(18.11083117,275.83497968)(18.03583124,275.85497966)(17.9558255,275.89498169)
\curveto(17.82583145,275.94497957)(17.70583157,275.99997951)(17.5958255,276.05998169)
\curveto(17.4758318,276.11997939)(17.36083192,276.18497933)(17.2508255,276.25498169)
\curveto(16.90083238,276.514979)(16.63083265,276.8099787)(16.4408255,277.13998169)
\curveto(16.24083304,277.46997804)(16.0808332,277.85997765)(15.9608255,278.30998169)
\curveto(15.94083334,278.41997709)(15.92583335,278.52497699)(15.9158255,278.62498169)
\curveto(15.90583337,278.73497678)(15.89083339,278.84497667)(15.8708255,278.95498169)
\curveto(15.86083342,279.00497651)(15.86083342,279.06997644)(15.8708255,279.14998169)
\curveto(15.87083341,279.23997627)(15.86083342,279.29997621)(15.8408255,279.32998169)
\curveto(15.83083345,280.02997548)(15.91083337,280.61997489)(16.0808255,281.09998169)
\curveto(16.25083303,281.58997392)(16.5758327,281.89497362)(17.0558255,282.01498169)
\curveto(17.25583202,282.06497345)(17.49083179,282.06997344)(17.7608255,282.02998169)
\curveto(18.02083126,281.98997352)(18.29583098,281.93997357)(18.5858255,281.87998169)
\lineto(21.9008255,281.21998169)
\curveto(22.04082724,281.18997432)(22.1758271,281.16497435)(22.3058255,281.14498169)
\curveto(22.43582684,281.13497438)(22.54082674,281.14497437)(22.6208255,281.17498169)
\curveto(22.69082659,281.2149743)(22.74082654,281.26997424)(22.7708255,281.33998169)
\curveto(22.81082647,281.42997408)(22.84082644,281.509974)(22.8608255,281.57998169)
\curveto(22.87082641,281.65997385)(22.91582636,281.7099738)(22.9958255,281.72998169)
\curveto(23.02582625,281.74997376)(23.05582622,281.75497376)(23.0858255,281.74498169)
\lineto(23.2058255,281.74498169)
\moveto(21.5408255,279.92998169)
\curveto(21.40082788,280.01997549)(21.24082804,280.08497543)(21.0608255,280.12498169)
\curveto(20.87082841,280.16497535)(20.6758286,280.20497531)(20.4758255,280.24498169)
\curveto(20.36582891,280.26497525)(20.26582901,280.27997523)(20.1758255,280.28998169)
\curveto(20.08582919,280.29997521)(20.01582926,280.27497524)(19.9658255,280.21498169)
\curveto(19.94582933,280.18497533)(19.93582934,280.1149754)(19.9358255,280.00498169)
\curveto(19.95582932,279.98497553)(19.96582931,279.94997556)(19.9658255,279.89998169)
\curveto(19.96582931,279.84997566)(19.9758293,279.79997571)(19.9958255,279.74998169)
\curveto(20.01582926,279.66997584)(20.03582924,279.57497594)(20.0558255,279.46498169)
\lineto(20.1158255,279.16498169)
\curveto(20.11582916,279.13497638)(20.12082916,279.09997641)(20.1308255,279.05998169)
\lineto(20.1308255,278.95498169)
\curveto(20.17082911,278.79497672)(20.19582908,278.62497689)(20.2058255,278.44498169)
\curveto(20.20582907,278.27497724)(20.22582905,278.1099774)(20.2658255,277.94998169)
\curveto(20.28582899,277.85997765)(20.30582897,277.77997773)(20.3258255,277.70998169)
\curveto(20.33582894,277.64997786)(20.35082893,277.57497794)(20.3708255,277.48498169)
\curveto(20.42082886,277.3149782)(20.48582879,277.14997836)(20.5658255,276.98998169)
\curveto(20.63582864,276.83997867)(20.72582855,276.70497881)(20.8358255,276.58498169)
\curveto(20.94582833,276.46497905)(21.0808282,276.36497915)(21.2408255,276.28498169)
\curveto(21.39082789,276.20497931)(21.5758277,276.14497937)(21.7958255,276.10498169)
\curveto(21.89582738,276.08497943)(21.99082729,276.08497943)(22.0808255,276.10498169)
\curveto(22.16082712,276.12497939)(22.23582704,276.15497936)(22.3058255,276.19498169)
\curveto(22.41582686,276.24497927)(22.51082677,276.32497919)(22.5908255,276.43498169)
\curveto(22.66082662,276.55497896)(22.72082656,276.68497883)(22.7708255,276.82498169)
\curveto(22.7808265,276.87497864)(22.78582649,276.92497859)(22.7858255,276.97498169)
\curveto(22.78582649,277.02497849)(22.79082649,277.07497844)(22.8008255,277.12498169)
\curveto(22.82082646,277.19497832)(22.83582644,277.27997823)(22.8458255,277.37998169)
\curveto(22.84582643,277.47997803)(22.83582644,277.56997794)(22.8158255,277.64998169)
\curveto(22.79582648,277.7099778)(22.79082649,277.76997774)(22.8008255,277.82998169)
\curveto(22.80082648,277.88997762)(22.79082649,277.94997756)(22.7708255,278.00998169)
\curveto(22.75082653,278.09997741)(22.73582654,278.17997733)(22.7258255,278.24998169)
\curveto(22.71582656,278.32997718)(22.69582658,278.4099771)(22.6658255,278.48998169)
\curveto(22.54582673,278.79997671)(22.40082688,279.07497644)(22.2308255,279.31498169)
\curveto(22.06082722,279.55497596)(21.83082745,279.75997575)(21.5408255,279.92998169)
}
}
{
\newrgbcolor{curcolor}{0 0 0}
\pscustom[linestyle=none,fillstyle=solid,fillcolor=curcolor]
{
\newpath
\moveto(15.8558255,288.03162231)
\curveto(15.85583342,288.26161593)(15.91583336,288.37661582)(16.0358255,288.37662231)
\curveto(16.14583313,288.38661581)(16.31083297,288.37161582)(16.5308255,288.33162231)
\curveto(16.63083265,288.31161588)(16.72583255,288.2916159)(16.8158255,288.27162231)
\curveto(16.90583237,288.25161594)(16.9808323,288.21161598)(17.0408255,288.15162231)
\curveto(17.12083216,288.07161612)(17.16583211,287.97661622)(17.1758255,287.86662231)
\curveto(17.1758321,287.75661644)(17.19083209,287.64161655)(17.2208255,287.52162231)
\curveto(17.25083203,287.38161681)(17.280832,287.24661695)(17.3108255,287.11662231)
\curveto(17.34083194,286.98661721)(17.3808319,286.86161733)(17.4308255,286.74162231)
\curveto(17.56083172,286.42161777)(17.74083154,286.14661805)(17.9708255,285.91662231)
\curveto(18.19083109,285.6966185)(18.44583083,285.4916187)(18.7358255,285.30162231)
\curveto(18.84583043,285.24161895)(18.96083032,285.18661901)(19.0808255,285.13662231)
\curveto(19.19083009,285.0966191)(19.30582997,285.05161914)(19.4258255,285.00162231)
\curveto(19.4758298,284.98161921)(19.52582975,284.96661923)(19.5758255,284.95662231)
\curveto(19.62582965,284.95661924)(19.6758296,284.94661925)(19.7258255,284.92662231)
\curveto(19.84582943,284.87661932)(19.98582929,284.83661936)(20.1458255,284.80662231)
\curveto(20.29582898,284.77661942)(20.44082884,284.75161944)(20.5808255,284.73162231)
\lineto(22.4258255,284.35662231)
\curveto(22.53582674,284.33661986)(22.65082663,284.31161988)(22.7708255,284.28162231)
\curveto(22.89082639,284.26161993)(23.00582627,284.23661996)(23.1158255,284.20662231)
\curveto(23.22582605,284.17662002)(23.32082596,284.15162004)(23.4008255,284.13162231)
\curveto(23.4808258,284.12162007)(23.55082573,284.08662011)(23.6108255,284.02662231)
\curveto(23.6808256,283.96662023)(23.72082556,283.88162031)(23.7308255,283.77162231)
\curveto(23.74082554,283.66162053)(23.74582553,283.54662065)(23.7458255,283.42662231)
\lineto(23.7458255,283.15662231)
\curveto(23.72582555,283.11662108)(23.71082557,283.07162112)(23.7008255,283.02162231)
\curveto(23.6808256,282.98162121)(23.65582562,282.95662124)(23.6258255,282.94662231)
\curveto(23.55582572,282.90662129)(23.47082581,282.8966213)(23.3708255,282.91662231)
\lineto(23.0408255,282.97662231)
\lineto(21.8858255,283.20162231)
\lineto(17.7308255,284.04162231)
\lineto(16.6958255,284.23662231)
\curveto(16.58583269,284.26661993)(16.48583279,284.2916199)(16.3958255,284.31162231)
\curveto(16.29583298,284.33161986)(16.21083307,284.37661982)(16.1408255,284.44662231)
\curveto(16.10083318,284.48661971)(16.07083321,284.54161965)(16.0508255,284.61162231)
\curveto(16.03083325,284.6916195)(16.02083326,284.77661942)(16.0208255,284.86662231)
\curveto(16.01083327,284.96661923)(16.01083327,285.05661914)(16.0208255,285.13662231)
\curveto(16.03083325,285.22661897)(16.04583323,285.2966189)(16.0658255,285.34662231)
\curveto(16.09583318,285.41661878)(16.15583312,285.45661874)(16.2458255,285.46662231)
\curveto(16.32583295,285.47661872)(16.41583286,285.47161872)(16.5158255,285.45162231)
\curveto(16.60583267,285.43161876)(16.70583257,285.41161878)(16.8158255,285.39162231)
\curveto(16.91583236,285.37161882)(17.00583227,285.37161882)(17.0858255,285.39162231)
\curveto(17.10583217,285.40161879)(17.12083216,285.40661879)(17.1308255,285.40662231)
\lineto(17.1758255,285.45162231)
\curveto(17.1758321,285.56161863)(17.13083215,285.66161853)(17.0408255,285.75162231)
\curveto(16.94083234,285.84161835)(16.86083242,285.91661828)(16.8008255,285.97662231)
\lineto(16.7108255,286.08162231)
\curveto(16.60083268,286.191618)(16.48583279,286.33661786)(16.3658255,286.51662231)
\curveto(16.24583303,286.70661749)(16.15583312,286.87661732)(16.0958255,287.02662231)
\curveto(16.04583323,287.12661707)(16.01083327,287.22661697)(15.9908255,287.32662231)
\curveto(15.96083332,287.43661676)(15.93083335,287.55161664)(15.9008255,287.67162231)
\curveto(15.89083339,287.73161646)(15.88583339,287.7916164)(15.8858255,287.85162231)
\lineto(15.8558255,288.03162231)
}
}
{
\newrgbcolor{curcolor}{0 0 0}
\pscustom[linestyle=none,fillstyle=solid,fillcolor=curcolor]
{
\newpath
\moveto(14.5358255,289.93638794)
\curveto(14.4758348,289.86638496)(14.37083491,289.84638498)(14.2208255,289.87638794)
\curveto(14.06083522,289.90638492)(13.90583537,289.93638489)(13.7558255,289.96638794)
\curveto(13.6758356,289.97638485)(13.59083569,289.99138484)(13.5008255,290.01138794)
\curveto(13.41083587,290.0313848)(13.33583594,290.06138477)(13.2758255,290.10138794)
\curveto(13.19583608,290.16138467)(13.13583614,290.25138458)(13.0958255,290.37138794)
\curveto(13.08583619,290.40138443)(13.08583619,290.4263844)(13.0958255,290.44638794)
\curveto(13.09583618,290.46638436)(13.09083619,290.49138434)(13.0808255,290.52138794)
\curveto(13.0808362,290.69138414)(13.08583619,290.84638398)(13.0958255,290.98638794)
\curveto(13.10583617,291.13638369)(13.16583611,291.2263836)(13.2758255,291.25638794)
\curveto(13.33583594,291.27638355)(13.41083587,291.27638355)(13.5008255,291.25638794)
\curveto(13.5808357,291.23638359)(13.66583561,291.22138361)(13.7558255,291.21138794)
\curveto(13.93583534,291.17138366)(14.10583517,291.1313837)(14.2658255,291.09138794)
\curveto(14.42583485,291.06138377)(14.53083475,290.97638385)(14.5808255,290.83638794)
\curveto(14.60083468,290.77638405)(14.61083467,290.71638411)(14.6108255,290.65638794)
\lineto(14.6108255,290.49138794)
\lineto(14.6108255,290.17638794)
\curveto(14.61083467,290.07638475)(14.58583469,289.99638483)(14.5358255,289.93638794)
\moveto(23.0408255,289.35138794)
\curveto(23.14082614,289.3313855)(23.24582603,289.31138552)(23.3558255,289.29138794)
\curveto(23.45582582,289.28138555)(23.53582574,289.24138559)(23.5958255,289.17138794)
\curveto(23.65582562,289.1313857)(23.69582558,289.08138575)(23.7158255,289.02138794)
\curveto(23.72582555,288.96138587)(23.74082554,288.88638594)(23.7608255,288.79638794)
\lineto(23.7608255,288.57138794)
\curveto(23.76082552,288.44138639)(23.75582552,288.3313865)(23.7458255,288.24138794)
\curveto(23.72582555,288.15138668)(23.6758256,288.08638674)(23.5958255,288.04638794)
\curveto(23.53582574,288.0263868)(23.46082582,288.02138681)(23.3708255,288.03138794)
\curveto(23.27082601,288.05138678)(23.1758261,288.07138676)(23.0858255,288.09138794)
\lineto(16.7408255,289.36638794)
\curveto(16.63083265,289.38638544)(16.52583275,289.40638542)(16.4258255,289.42638794)
\curveto(16.31583296,289.44638538)(16.23083305,289.48638534)(16.1708255,289.54638794)
\curveto(16.12083316,289.58638524)(16.09083319,289.6313852)(16.0808255,289.68138794)
\curveto(16.07083321,289.74138509)(16.05583322,289.80138503)(16.0358255,289.86138794)
\curveto(16.03583324,289.88138495)(16.04083324,289.90138493)(16.0508255,289.92138794)
\curveto(16.05083323,289.95138488)(16.04583323,289.97638485)(16.0358255,289.99638794)
\curveto(16.03583324,290.1263847)(16.04083324,290.25638457)(16.0508255,290.38638794)
\curveto(16.05083323,290.5263843)(16.09083319,290.61138422)(16.1708255,290.64138794)
\curveto(16.23083305,290.68138415)(16.31083297,290.69138414)(16.4108255,290.67138794)
\curveto(16.50083278,290.65138418)(16.59583268,290.6313842)(16.6958255,290.61138794)
\lineto(23.0408255,289.35138794)
}
}
{
\newrgbcolor{curcolor}{0 0 0}
\pscustom[linestyle=none,fillstyle=solid,fillcolor=curcolor]
{
\newpath
\moveto(19.5608255,299.11123169)
\curveto(19.62082966,299.1212228)(19.71582956,299.11122281)(19.8458255,299.08123169)
\curveto(19.96582931,299.06122286)(20.05082923,299.04122288)(20.1008255,299.02123169)
\lineto(20.2508255,298.99123169)
\curveto(20.33082895,298.96122296)(20.40582887,298.93622298)(20.4758255,298.91623169)
\curveto(20.53582874,298.90622301)(20.60582867,298.88622303)(20.6858255,298.85623169)
\curveto(20.74582853,298.82622309)(20.80582847,298.80122312)(20.8658255,298.78123169)
\curveto(20.92582835,298.77122315)(20.98582829,298.74622317)(21.0458255,298.70623169)
\lineto(21.4358255,298.52623169)
\curveto(21.56582771,298.47622344)(21.68582759,298.41122351)(21.7958255,298.33123169)
\curveto(22.275827,298.03122389)(22.6808266,297.67122425)(23.0108255,297.25123169)
\curveto(23.33082595,296.84122508)(23.5758257,296.36122556)(23.7458255,295.81123169)
\curveto(23.78582549,295.70122622)(23.81582546,295.58122634)(23.8358255,295.45123169)
\curveto(23.85582542,295.3212266)(23.8758254,295.18622673)(23.8958255,295.04623169)
\curveto(23.90582537,294.98622693)(23.91082537,294.921227)(23.9108255,294.85123169)
\curveto(23.92082536,294.79122713)(23.92582535,294.73122719)(23.9258255,294.67123169)
\curveto(23.93582534,294.63122729)(23.94082534,294.57122735)(23.9408255,294.49123169)
\curveto(23.94082534,294.4212275)(23.93582534,294.37122755)(23.9258255,294.34123169)
\curveto(23.91582536,294.30122762)(23.91082537,294.26122766)(23.9108255,294.22123169)
\curveto(23.92082536,294.18122774)(23.92082536,294.14622777)(23.9108255,294.11623169)
\lineto(23.9108255,294.02623169)
\lineto(23.8658255,293.68123169)
\lineto(23.7458255,293.29123169)
\curveto(23.70582557,293.17122875)(23.66082562,293.05622886)(23.6108255,292.94623169)
\curveto(23.41082587,292.53622938)(23.15082613,292.2162297)(22.8308255,291.98623169)
\curveto(22.51082677,291.76623015)(22.12082716,291.60623031)(21.6608255,291.50623169)
\curveto(21.56082772,291.47623044)(21.46082782,291.45623046)(21.3608255,291.44623169)
\lineto(21.0458255,291.44623169)
\curveto(21.00582827,291.43623048)(20.9758283,291.43623048)(20.9558255,291.44623169)
\curveto(20.92582835,291.45623046)(20.89082839,291.46123046)(20.8508255,291.46123169)
\curveto(20.77082851,291.46123046)(20.69082859,291.46623045)(20.6108255,291.47623169)
\curveto(20.52082876,291.48623043)(20.43582884,291.49123043)(20.3558255,291.49123169)
\curveto(20.30582897,291.50123042)(20.26582901,291.50623041)(20.2358255,291.50623169)
\curveto(20.19582908,291.5162304)(20.15082913,291.5212304)(20.1008255,291.52123169)
\curveto(20.05082923,291.5212304)(19.96582931,291.53123039)(19.8458255,291.55123169)
\curveto(19.71582956,291.58123034)(19.62082966,291.61123031)(19.5608255,291.64123169)
\curveto(19.49082979,291.68123024)(19.42082986,291.70123022)(19.3508255,291.70123169)
\curveto(19.28083,291.70123022)(19.21083007,291.7212302)(19.1408255,291.76123169)
\curveto(19.09083019,291.78123014)(19.05083023,291.79623012)(19.0208255,291.80623169)
\curveto(18.9808303,291.8162301)(18.93583034,291.83123009)(18.8858255,291.85123169)
\curveto(18.76583051,291.91123001)(18.64583063,291.96122996)(18.5258255,292.00123169)
\curveto(18.40583087,292.05122987)(18.29083099,292.1162298)(18.1808255,292.19623169)
\curveto(17.81083147,292.4162295)(17.4808318,292.66122926)(17.1908255,292.93123169)
\curveto(16.89083239,293.21122871)(16.64083264,293.52622839)(16.4408255,293.87623169)
\curveto(16.36083292,294.00622791)(16.29583298,294.14122778)(16.2458255,294.28123169)
\lineto(16.0658255,294.73123169)
\curveto(16.01583326,294.86122706)(15.98583329,294.99622692)(15.9758255,295.13623169)
\curveto(15.95583332,295.27622664)(15.92583335,295.4212265)(15.8858255,295.57123169)
\lineto(15.8858255,295.76623169)
\lineto(15.8558255,295.97623169)
\curveto(15.84583343,296.86622505)(16.03083325,297.56622435)(16.4108255,298.07623169)
\curveto(16.79083249,298.59622332)(17.28583199,298.921223)(17.8958255,299.05123169)
\curveto(17.99583128,299.08122284)(18.09583118,299.10122282)(18.1958255,299.11123169)
\curveto(18.29583098,299.1212228)(18.40083088,299.13622278)(18.5108255,299.15623169)
\curveto(18.62083066,299.16622275)(18.74083054,299.16622275)(18.8708255,299.15623169)
\lineto(19.2458255,299.15623169)
\curveto(19.29582998,299.15622276)(19.35082993,299.14622277)(19.4108255,299.12623169)
\curveto(19.46082982,299.1162228)(19.51082977,299.11122281)(19.5608255,299.11123169)
\moveto(20.4158255,297.61123169)
\curveto(20.34582893,297.64122428)(20.26582901,297.66122426)(20.1758255,297.67123169)
\curveto(20.08582919,297.69122423)(20.00082928,297.70622421)(19.9208255,297.71623169)
\curveto(19.53082975,297.79622412)(19.20083008,297.83122409)(18.9308255,297.82123169)
\curveto(18.85083043,297.80122412)(18.77083051,297.78622413)(18.6908255,297.77623169)
\curveto(18.61083067,297.77622414)(18.53583074,297.77122415)(18.4658255,297.76123169)
\curveto(17.81583146,297.61122431)(17.36583191,297.25622466)(17.1158255,296.69623169)
\curveto(17.08583219,296.62622529)(17.06583221,296.55122537)(17.0558255,296.47123169)
\curveto(17.03583224,296.40122552)(17.01583226,296.32622559)(16.9958255,296.24623169)
\curveto(16.9758323,296.17622574)(16.96583231,296.09622582)(16.9658255,296.00623169)
\lineto(16.9658255,295.73623169)
\lineto(17.0108255,295.45123169)
\curveto(17.03083225,295.35122657)(17.05583222,295.25622666)(17.0858255,295.16623169)
\curveto(17.10583217,295.07622684)(17.13583214,294.98622693)(17.1758255,294.89623169)
\curveto(17.19583208,294.82622709)(17.22583205,294.75622716)(17.2658255,294.68623169)
\curveto(17.30583197,294.6162273)(17.34583193,294.55122737)(17.3858255,294.49123169)
\curveto(17.55583172,294.2212277)(17.76083152,293.98622793)(18.0008255,293.78623169)
\curveto(18.24083104,293.58622833)(18.52083076,293.40122852)(18.8408255,293.23123169)
\curveto(18.94083034,293.18122874)(19.04583023,293.14122878)(19.1558255,293.11123169)
\curveto(19.25583002,293.08122884)(19.36082992,293.04122888)(19.4708255,292.99123169)
\curveto(19.51082977,292.98122894)(19.5758297,292.96622895)(19.6658255,292.94623169)
\curveto(19.69582958,292.92622899)(19.73082955,292.916229)(19.7708255,292.91623169)
\curveto(19.81082947,292.916229)(19.85582942,292.91122901)(19.9058255,292.90123169)
\lineto(20.2058255,292.84123169)
\curveto(20.30582897,292.8212291)(20.39582888,292.81122911)(20.4758255,292.81123169)
\lineto(20.6558255,292.81123169)
\curveto(20.75582852,292.81122911)(20.85582842,292.80622911)(20.9558255,292.79623169)
\curveto(21.04582823,292.79622912)(21.13082815,292.80622911)(21.2108255,292.82623169)
\curveto(21.45082783,292.87622904)(21.6758276,292.94622897)(21.8858255,293.03623169)
\curveto(22.09582718,293.13622878)(22.27082701,293.27122865)(22.4108255,293.44123169)
\curveto(22.44082684,293.49122843)(22.46582681,293.53122839)(22.4858255,293.56123169)
\curveto(22.50582677,293.60122832)(22.53082675,293.64122828)(22.5608255,293.68123169)
\curveto(22.61082667,293.75122817)(22.65582662,293.83122809)(22.6958255,293.92123169)
\curveto(22.72582655,294.01122791)(22.75582652,294.10622781)(22.7858255,294.20623169)
\curveto(22.80582647,294.25622766)(22.81582646,294.30122762)(22.8158255,294.34123169)
\curveto(22.80582647,294.39122753)(22.80582647,294.44122748)(22.8158255,294.49123169)
\curveto(22.82582645,294.5212274)(22.83582644,294.58122734)(22.8458255,294.67123169)
\curveto(22.85582642,294.76122716)(22.85082643,294.83622708)(22.8308255,294.89623169)
\curveto(22.82082646,294.93622698)(22.82082646,294.97622694)(22.8308255,295.01623169)
\curveto(22.83082645,295.05622686)(22.82082646,295.09622682)(22.8008255,295.13623169)
\curveto(22.7808265,295.2162267)(22.76582651,295.29622662)(22.7558255,295.37623169)
\curveto(22.73582654,295.46622645)(22.71082657,295.55122637)(22.6808255,295.63123169)
\curveto(22.54082674,295.99122593)(22.34582693,296.30122562)(22.0958255,296.56123169)
\curveto(21.84582743,296.8212251)(21.55082773,297.05622486)(21.2108255,297.26623169)
\curveto(21.09082819,297.34622457)(20.96582831,297.40622451)(20.8358255,297.44623169)
\curveto(20.69582858,297.48622443)(20.55582872,297.54122438)(20.4158255,297.61123169)
}
}
{
\newrgbcolor{curcolor}{0 0 0}
\pscustom[linestyle=none,fillstyle=solid,fillcolor=curcolor]
{
\newpath
\moveto(15.8558255,303.77951294)
\curveto(15.84583343,304.49950728)(15.93083335,305.0845067)(16.1108255,305.53451294)
\curveto(16.280833,305.99450579)(16.58583269,306.31450547)(17.0258255,306.49451294)
\curveto(17.13583214,306.54450524)(17.25083203,306.57450521)(17.3708255,306.58451294)
\curveto(17.4808318,306.60450518)(17.60583167,306.61950516)(17.7458255,306.62951294)
\curveto(17.81583146,306.63950514)(17.89083139,306.62950515)(17.9708255,306.59951294)
\curveto(18.04083124,306.5795052)(18.09583118,306.55450523)(18.1358255,306.52451294)
\curveto(18.15583112,306.50450528)(18.1758311,306.47450531)(18.1958255,306.43451294)
\curveto(18.20583107,306.40450538)(18.22083106,306.3795054)(18.2408255,306.35951294)
\curveto(18.26083102,306.29950548)(18.26583101,306.24450554)(18.2558255,306.19451294)
\curveto(18.24583103,306.15450563)(18.24583103,306.10950567)(18.2558255,306.05951294)
\curveto(18.275831,305.96950581)(18.280831,305.85950592)(18.2708255,305.72951294)
\curveto(18.25083103,305.60950617)(18.22583105,305.52450626)(18.1958255,305.47451294)
\curveto(18.14583113,305.40450638)(18.0808312,305.36450642)(18.0008255,305.35451294)
\curveto(17.91083137,305.35450643)(17.82583145,305.33450645)(17.7458255,305.29451294)
\curveto(17.58583169,305.24450654)(17.44083184,305.14950663)(17.3108255,305.00951294)
\curveto(17.23083205,304.91950686)(17.17083211,304.80950697)(17.1308255,304.67951294)
\curveto(17.09083219,304.55950722)(17.05083223,304.42950735)(17.0108255,304.28951294)
\curveto(16.99083229,304.24950753)(16.98583229,304.19950758)(16.9958255,304.13951294)
\curveto(16.99583228,304.08950769)(16.99083229,304.04450774)(16.9808255,304.00451294)
\curveto(16.96083232,303.94450784)(16.95083233,303.86950791)(16.9508255,303.77951294)
\curveto(16.95083233,303.68950809)(16.96083232,303.61450817)(16.9808255,303.55451294)
\lineto(16.9808255,303.46451294)
\curveto(16.99083229,303.40450838)(17.00083228,303.34950843)(17.0108255,303.29951294)
\curveto(17.01083227,303.24950853)(17.01583226,303.19950858)(17.0258255,303.14951294)
\curveto(17.08583219,302.8795089)(17.17083211,302.64450914)(17.2808255,302.44451294)
\curveto(17.39083189,302.25450953)(17.5758317,302.10450968)(17.8358255,301.99451294)
\curveto(17.90583137,301.96450982)(17.9758313,301.94950983)(18.0458255,301.94951294)
\curveto(18.11583116,301.94950983)(18.1758311,301.95450983)(18.2258255,301.96451294)
\curveto(18.3758309,301.99450979)(18.48583079,302.04450974)(18.5558255,302.11451294)
\curveto(18.61583066,302.1845096)(18.68583059,302.2795095)(18.7658255,302.39951294)
\curveto(18.86583041,302.53950924)(18.94083034,302.70450908)(18.9908255,302.89451294)
\curveto(19.03083025,303.0845087)(19.0808302,303.27450851)(19.1408255,303.46451294)
\curveto(19.1808301,303.5845082)(19.21083007,303.70450808)(19.2308255,303.82451294)
\curveto(19.25083003,303.95450783)(19.28083,304.0795077)(19.3208255,304.19951294)
\curveto(19.3808299,304.39950738)(19.44082984,304.59450719)(19.5008255,304.78451294)
\curveto(19.55082973,304.97450681)(19.61582966,305.15950662)(19.6958255,305.33951294)
\curveto(19.71582956,305.38950639)(19.73582954,305.43450635)(19.7558255,305.47451294)
\curveto(19.7758295,305.52450626)(19.80082948,305.57450621)(19.8308255,305.62451294)
\curveto(19.95082933,305.79450599)(20.08582919,305.93950584)(20.2358255,306.05951294)
\curveto(20.38582889,306.1795056)(20.5758287,306.26950551)(20.8058255,306.32951294)
\lineto(21.0908255,306.32951294)
\curveto(21.16082812,306.32950545)(21.23582804,306.32450546)(21.3158255,306.31451294)
\curveto(21.38582789,306.30450548)(21.46582781,306.29450549)(21.5558255,306.28451294)
\lineto(21.7058255,306.25451294)
\curveto(21.7758275,306.21450557)(21.84582743,306.1845056)(21.9158255,306.16451294)
\curveto(21.98582729,306.15450563)(22.05582722,306.13450565)(22.1258255,306.10451294)
\curveto(22.23582704,306.05450573)(22.34082694,305.99950578)(22.4408255,305.93951294)
\curveto(22.54082674,305.8795059)(22.63082665,305.81450597)(22.7108255,305.74451294)
\curveto(22.97082631,305.53450625)(23.1808261,305.28950649)(23.3408255,305.00951294)
\curveto(23.49082579,304.72950705)(23.62082566,304.42450736)(23.7308255,304.09451294)
\curveto(23.76082552,303.99450779)(23.7808255,303.89450789)(23.7908255,303.79451294)
\curveto(23.81082547,303.69450809)(23.83582544,303.59950818)(23.8658255,303.50951294)
\curveto(23.88582539,303.39950838)(23.89582538,303.29450849)(23.8958255,303.19451294)
\curveto(23.89582538,303.09450869)(23.90582537,302.99450879)(23.9258255,302.89451294)
\lineto(23.9258255,302.74451294)
\curveto(23.93582534,302.69450909)(23.94082534,302.62450916)(23.9408255,302.53451294)
\curveto(23.94082534,302.44450934)(23.93582534,302.37450941)(23.9258255,302.32451294)
\lineto(23.9258255,302.15951294)
\curveto(23.90582537,302.09950968)(23.89582538,302.03450975)(23.8958255,301.96451294)
\curveto(23.90582537,301.89450989)(23.90082538,301.83950994)(23.8808255,301.79951294)
\curveto(23.87082541,301.74951003)(23.86582541,301.6845101)(23.8658255,301.60451294)
\curveto(23.84582543,301.52451026)(23.82582545,301.44951033)(23.8058255,301.37951294)
\curveto(23.79582548,301.30951047)(23.7758255,301.23451055)(23.7458255,301.15451294)
\curveto(23.64582563,300.86451092)(23.52082576,300.61951116)(23.3708255,300.41951294)
\curveto(23.22082606,300.21951156)(23.02582625,300.05951172)(22.7858255,299.93951294)
\curveto(22.65582662,299.8795119)(22.52082676,299.82951195)(22.3808255,299.78951294)
\curveto(22.24082704,299.75951202)(22.08582719,299.73951204)(21.9158255,299.72951294)
\curveto(21.85582742,299.71951206)(21.78582749,299.72451206)(21.7058255,299.74451294)
\curveto(21.61582766,299.76451202)(21.54582773,299.78951199)(21.4958255,299.81951294)
\curveto(21.45582782,299.85951192)(21.41582786,299.91951186)(21.3758255,299.99951294)
\curveto(21.35582792,300.04951173)(21.34582793,300.11951166)(21.3458255,300.20951294)
\curveto(21.33582794,300.30951147)(21.33582794,300.39951138)(21.3458255,300.47951294)
\curveto(21.35582792,300.56951121)(21.37082791,300.65451113)(21.3908255,300.73451294)
\curveto(21.40082788,300.82451096)(21.41582786,300.8795109)(21.4358255,300.89951294)
\curveto(21.48582779,300.95951082)(21.56082772,300.98951079)(21.6608255,300.98951294)
\curveto(21.75082753,300.99951078)(21.83582744,301.01951076)(21.9158255,301.04951294)
\curveto(22.13582714,301.09951068)(22.30582697,301.19951058)(22.4258255,301.34951294)
\curveto(22.51582676,301.44951033)(22.58582669,301.56951021)(22.6358255,301.70951294)
\curveto(22.68582659,301.84950993)(22.73582654,301.99950978)(22.7858255,302.15951294)
\lineto(22.8308255,302.47451294)
\lineto(22.8308255,302.56451294)
\curveto(22.85082643,302.62450916)(22.86082642,302.70950907)(22.8608255,302.81951294)
\curveto(22.86082642,302.93950884)(22.85082643,303.04450874)(22.8308255,303.13451294)
\curveto(22.83082645,303.20450858)(22.82582645,303.25950852)(22.8158255,303.29951294)
\curveto(22.80582647,303.35950842)(22.80082648,303.41950836)(22.8008255,303.47951294)
\curveto(22.79082649,303.53950824)(22.7808265,303.59450819)(22.7708255,303.64451294)
\curveto(22.69082659,303.95450783)(22.58582669,304.20450758)(22.4558255,304.39451294)
\curveto(22.32582695,304.59450719)(22.10582717,304.75950702)(21.7958255,304.88951294)
\curveto(21.74582753,304.91950686)(21.69082759,304.93450685)(21.6308255,304.93451294)
\curveto(21.57082771,304.94450684)(21.52582775,304.94450684)(21.4958255,304.93451294)
\curveto(21.30582797,304.92450686)(21.16582811,304.8845069)(21.0758255,304.81451294)
\curveto(20.9758283,304.74450704)(20.88582839,304.64950713)(20.8058255,304.52951294)
\curveto(20.74582853,304.44950733)(20.69582858,304.35450743)(20.6558255,304.24451294)
\lineto(20.5358255,303.94451294)
\curveto(20.52582875,303.91450787)(20.52082876,303.8845079)(20.5208255,303.85451294)
\curveto(20.52082876,303.83450795)(20.51082877,303.81450797)(20.4908255,303.79451294)
\curveto(20.3808289,303.47450831)(20.30082898,303.13450865)(20.2508255,302.77451294)
\curveto(20.19082909,302.42450936)(20.09582918,302.10450968)(19.9658255,301.81451294)
\curveto(19.92582935,301.72451006)(19.89082939,301.63451015)(19.8608255,301.54451294)
\curveto(19.83082945,301.46451032)(19.79082949,301.38951039)(19.7408255,301.31951294)
\curveto(19.63082965,301.14951063)(19.50582977,300.99951078)(19.3658255,300.86951294)
\curveto(19.22583005,300.73951104)(19.05083023,300.64951113)(18.8408255,300.59951294)
\curveto(18.77083051,300.5795112)(18.70083058,300.56951121)(18.6308255,300.56951294)
\lineto(18.4058255,300.56951294)
\curveto(18.28583099,300.55951122)(18.15083113,300.57451121)(18.0008255,300.61451294)
\curveto(17.84083144,300.65451113)(17.70583157,300.69451109)(17.5958255,300.73451294)
\curveto(17.54583173,300.76451102)(17.50583177,300.784511)(17.4758255,300.79451294)
\curveto(17.43583184,300.81451097)(17.39583188,300.83951094)(17.3558255,300.86951294)
\curveto(17.12583215,300.99951078)(16.92583235,301.15951062)(16.7558255,301.34951294)
\curveto(16.58583269,301.53951024)(16.43583284,301.74951003)(16.3058255,301.97951294)
\curveto(16.21583306,302.13950964)(16.14583313,302.31450947)(16.0958255,302.50451294)
\curveto(16.03583324,302.70450908)(15.9808333,302.90950887)(15.9308255,303.11951294)
\curveto(15.92083336,303.18950859)(15.91083337,303.25450853)(15.9008255,303.31451294)
\curveto(15.89083339,303.3845084)(15.8808334,303.45950832)(15.8708255,303.53951294)
\curveto(15.86083342,303.5795082)(15.86083342,303.61950816)(15.8708255,303.65951294)
\curveto(15.8808334,303.70950807)(15.8758334,303.74950803)(15.8558255,303.77951294)
}
}
{
\newrgbcolor{curcolor}{0 0 0}
\pscustom[linestyle=none,fillstyle=solid,fillcolor=curcolor]
{
\newpath
\moveto(125.45952515,72.65651611)
\curveto(125.52952341,72.65650545)(125.60952333,72.65650545)(125.69952515,72.65651611)
\curveto(125.78952315,72.66650544)(125.87452307,72.66650544)(125.95452515,72.65651611)
\curveto(126.0445229,72.65650545)(126.12452282,72.64650546)(126.19452515,72.62651611)
\curveto(126.26452268,72.6065055)(126.31452263,72.57650553)(126.34452515,72.53651611)
\curveto(126.40452254,72.46650564)(126.43452251,72.36650574)(126.43452515,72.23651611)
\curveto(126.4445225,72.11650599)(126.44952249,71.99150611)(126.44952515,71.86151611)
\lineto(126.44952515,70.40651611)
\lineto(126.44952515,64.61651611)
\lineto(126.44952515,62.86151611)
\lineto(126.44952515,62.44151611)
\curveto(126.44952249,62.3015158)(126.42452252,62.19151591)(126.37452515,62.11151611)
\curveto(126.33452261,62.06151604)(126.28452266,62.03151607)(126.22452515,62.02151611)
\curveto(126.17452277,62.01151609)(126.10952283,61.99651611)(126.02952515,61.97651611)
\lineto(125.74452515,61.97651611)
\curveto(125.60452334,61.97651613)(125.47452347,61.98151612)(125.35452515,61.99151611)
\curveto(125.23452371,62.0015161)(125.14952379,62.05151605)(125.09952515,62.14151611)
\curveto(125.05952388,62.2015159)(125.0395239,62.28151582)(125.03952515,62.38151611)
\lineto(125.03952515,62.71151611)
\lineto(125.03952515,63.91151611)
\lineto(125.03952515,70.18151611)
\lineto(125.03952515,71.80151611)
\curveto(125.0395239,71.91150619)(125.03452391,72.03150607)(125.02452515,72.16151611)
\curveto(125.02452392,72.3015058)(125.04952389,72.41150569)(125.09952515,72.49151611)
\curveto(125.1395238,72.56150554)(125.21952372,72.61150549)(125.33952515,72.64151611)
\curveto(125.35952358,72.65150545)(125.37952356,72.65150545)(125.39952515,72.64151611)
\curveto(125.41952352,72.64150546)(125.4395235,72.64650546)(125.45952515,72.65651611)
}
}
{
\newrgbcolor{curcolor}{0 0 0}
\pscustom[linestyle=none,fillstyle=solid,fillcolor=curcolor]
{
\newpath
\moveto(132.29600952,69.88151611)
\curveto(132.92600429,69.9015082)(133.43100378,69.81650829)(133.81100952,69.62651611)
\curveto(134.19100302,69.43650867)(134.49600272,69.15150895)(134.72600952,68.77151611)
\curveto(134.78600243,68.67150943)(134.83100238,68.56150954)(134.86100952,68.44151611)
\curveto(134.90100231,68.33150977)(134.93600228,68.21650989)(134.96600952,68.09651611)
\curveto(135.0160022,67.9065102)(135.04600217,67.7015104)(135.05600952,67.48151611)
\curveto(135.06600215,67.26151084)(135.07100214,67.03651107)(135.07100952,66.80651611)
\lineto(135.07100952,65.20151611)
\lineto(135.07100952,62.86151611)
\curveto(135.07100214,62.69151541)(135.06600215,62.52151558)(135.05600952,62.35151611)
\curveto(135.05600216,62.18151592)(134.99100222,62.07151603)(134.86100952,62.02151611)
\curveto(134.8110024,62.0015161)(134.75600246,61.99151611)(134.69600952,61.99151611)
\curveto(134.64600257,61.98151612)(134.59100262,61.97651613)(134.53100952,61.97651611)
\curveto(134.40100281,61.97651613)(134.27600294,61.98151612)(134.15600952,61.99151611)
\curveto(134.03600318,61.99151611)(133.95100326,62.03151607)(133.90100952,62.11151611)
\curveto(133.85100336,62.18151592)(133.82600339,62.27151583)(133.82600952,62.38151611)
\lineto(133.82600952,62.71151611)
\lineto(133.82600952,64.00151611)
\lineto(133.82600952,66.44651611)
\curveto(133.82600339,66.71651139)(133.82100339,66.98151112)(133.81100952,67.24151611)
\curveto(133.80100341,67.51151059)(133.75600346,67.74151036)(133.67600952,67.93151611)
\curveto(133.59600362,68.13150997)(133.47600374,68.29150981)(133.31600952,68.41151611)
\curveto(133.15600406,68.54150956)(132.97100424,68.64150946)(132.76100952,68.71151611)
\curveto(132.70100451,68.73150937)(132.63600458,68.74150936)(132.56600952,68.74151611)
\curveto(132.50600471,68.75150935)(132.44600477,68.76650934)(132.38600952,68.78651611)
\curveto(132.33600488,68.79650931)(132.25600496,68.79650931)(132.14600952,68.78651611)
\curveto(132.04600517,68.78650932)(131.97600524,68.78150932)(131.93600952,68.77151611)
\curveto(131.89600532,68.75150935)(131.86100535,68.74150936)(131.83100952,68.74151611)
\curveto(131.80100541,68.75150935)(131.76600545,68.75150935)(131.72600952,68.74151611)
\curveto(131.59600562,68.71150939)(131.47100574,68.67650943)(131.35100952,68.63651611)
\curveto(131.24100597,68.6065095)(131.13600608,68.56150954)(131.03600952,68.50151611)
\curveto(130.99600622,68.48150962)(130.96100625,68.46150964)(130.93100952,68.44151611)
\curveto(130.90100631,68.42150968)(130.86600635,68.4015097)(130.82600952,68.38151611)
\curveto(130.47600674,68.13150997)(130.22100699,67.75651035)(130.06100952,67.25651611)
\curveto(130.03100718,67.17651093)(130.0110072,67.09151101)(130.00100952,67.00151611)
\curveto(129.99100722,66.92151118)(129.97600724,66.84151126)(129.95600952,66.76151611)
\curveto(129.93600728,66.71151139)(129.93100728,66.66151144)(129.94100952,66.61151611)
\curveto(129.95100726,66.57151153)(129.94600727,66.53151157)(129.92600952,66.49151611)
\lineto(129.92600952,66.17651611)
\curveto(129.9160073,66.14651196)(129.9110073,66.11151199)(129.91100952,66.07151611)
\curveto(129.92100729,66.03151207)(129.92600729,65.98651212)(129.92600952,65.93651611)
\lineto(129.92600952,65.48651611)
\lineto(129.92600952,64.04651611)
\lineto(129.92600952,62.72651611)
\lineto(129.92600952,62.38151611)
\curveto(129.92600729,62.27151583)(129.90100731,62.18151592)(129.85100952,62.11151611)
\curveto(129.80100741,62.03151607)(129.7110075,61.99151611)(129.58100952,61.99151611)
\curveto(129.46100775,61.98151612)(129.33600788,61.97651613)(129.20600952,61.97651611)
\curveto(129.12600809,61.97651613)(129.05100816,61.98151612)(128.98100952,61.99151611)
\curveto(128.9110083,62.0015161)(128.85100836,62.02651608)(128.80100952,62.06651611)
\curveto(128.72100849,62.11651599)(128.68100853,62.21151589)(128.68100952,62.35151611)
\lineto(128.68100952,62.75651611)
\lineto(128.68100952,64.52651611)
\lineto(128.68100952,68.15651611)
\lineto(128.68100952,69.07151611)
\lineto(128.68100952,69.34151611)
\curveto(128.68100853,69.43150867)(128.70100851,69.5015086)(128.74100952,69.55151611)
\curveto(128.77100844,69.61150849)(128.82100839,69.65150845)(128.89100952,69.67151611)
\curveto(128.93100828,69.68150842)(128.98600823,69.69150841)(129.05600952,69.70151611)
\curveto(129.13600808,69.71150839)(129.216008,69.71650839)(129.29600952,69.71651611)
\curveto(129.37600784,69.71650839)(129.45100776,69.71150839)(129.52100952,69.70151611)
\curveto(129.60100761,69.69150841)(129.65600756,69.67650843)(129.68600952,69.65651611)
\curveto(129.79600742,69.58650852)(129.84600737,69.49650861)(129.83600952,69.38651611)
\curveto(129.82600739,69.28650882)(129.84100737,69.17150893)(129.88100952,69.04151611)
\curveto(129.90100731,68.98150912)(129.94100727,68.93150917)(130.00100952,68.89151611)
\curveto(130.12100709,68.88150922)(130.216007,68.92650918)(130.28600952,69.02651611)
\curveto(130.36600685,69.12650898)(130.44600677,69.2065089)(130.52600952,69.26651611)
\curveto(130.66600655,69.36650874)(130.80600641,69.45650865)(130.94600952,69.53651611)
\curveto(131.09600612,69.62650848)(131.26600595,69.7015084)(131.45600952,69.76151611)
\curveto(131.53600568,69.79150831)(131.62100559,69.81150829)(131.71100952,69.82151611)
\curveto(131.8110054,69.83150827)(131.90600531,69.84650826)(131.99600952,69.86651611)
\curveto(132.04600517,69.87650823)(132.09600512,69.88150822)(132.14600952,69.88151611)
\lineto(132.29600952,69.88151611)
}
}
{
\newrgbcolor{curcolor}{0 0 0}
\pscustom[linestyle=none,fillstyle=solid,fillcolor=curcolor]
{
\newpath
\moveto(136.7306189,69.73151611)
\lineto(137.2106189,69.73151611)
\curveto(137.38061756,69.73150837)(137.51061743,69.7015084)(137.6006189,69.64151611)
\curveto(137.67061727,69.59150851)(137.71561722,69.52650858)(137.7356189,69.44651611)
\curveto(137.76561717,69.37650873)(137.79561714,69.3015088)(137.8256189,69.22151611)
\curveto(137.88561705,69.08150902)(137.935617,68.94150916)(137.9756189,68.80151611)
\curveto(138.01561692,68.66150944)(138.06061688,68.52150958)(138.1106189,68.38151611)
\curveto(138.31061663,67.84151026)(138.49561644,67.29651081)(138.6656189,66.74651611)
\curveto(138.8356161,66.2065119)(139.02061592,65.66651244)(139.2206189,65.12651611)
\curveto(139.29061565,64.94651316)(139.35061559,64.76151334)(139.4006189,64.57151611)
\curveto(139.45061549,64.39151371)(139.51561542,64.21151389)(139.5956189,64.03151611)
\curveto(139.61561532,63.96151414)(139.6406153,63.88651422)(139.6706189,63.80651611)
\curveto(139.70061524,63.72651438)(139.75061519,63.67651443)(139.8206189,63.65651611)
\curveto(139.90061504,63.63651447)(139.96061498,63.67151443)(140.0006189,63.76151611)
\curveto(140.05061489,63.85151425)(140.08561485,63.92151418)(140.1056189,63.97151611)
\curveto(140.18561475,64.16151394)(140.25061469,64.35151375)(140.3006189,64.54151611)
\curveto(140.36061458,64.74151336)(140.42561451,64.94151316)(140.4956189,65.14151611)
\curveto(140.62561431,65.52151258)(140.75061419,65.89651221)(140.8706189,66.26651611)
\curveto(140.99061395,66.64651146)(141.11561382,67.02651108)(141.2456189,67.40651611)
\curveto(141.29561364,67.57651053)(141.34561359,67.74151036)(141.3956189,67.90151611)
\curveto(141.44561349,68.07151003)(141.50561343,68.23650987)(141.5756189,68.39651611)
\curveto(141.62561331,68.53650957)(141.67061327,68.67650943)(141.7106189,68.81651611)
\curveto(141.75061319,68.95650915)(141.79561314,69.09650901)(141.8456189,69.23651611)
\curveto(141.86561307,69.3065088)(141.89061305,69.37650873)(141.9206189,69.44651611)
\curveto(141.95061299,69.51650859)(141.99061295,69.57650853)(142.0406189,69.62651611)
\curveto(142.12061282,69.67650843)(142.21061273,69.7065084)(142.3106189,69.71651611)
\curveto(142.41061253,69.72650838)(142.53061241,69.73150837)(142.6706189,69.73151611)
\curveto(142.7406122,69.73150837)(142.80561213,69.72650838)(142.8656189,69.71651611)
\curveto(142.92561201,69.71650839)(142.98061196,69.7065084)(143.0306189,69.68651611)
\curveto(143.12061182,69.64650846)(143.16561177,69.58150852)(143.1656189,69.49151611)
\curveto(143.17561176,69.4015087)(143.16061178,69.31150879)(143.1206189,69.22151611)
\curveto(143.06061188,69.05150905)(143.00061194,68.87650923)(142.9406189,68.69651611)
\curveto(142.88061206,68.51650959)(142.81061213,68.34150976)(142.7306189,68.17151611)
\curveto(142.71061223,68.12150998)(142.69561224,68.07151003)(142.6856189,68.02151611)
\curveto(142.67561226,67.98151012)(142.66061228,67.93651017)(142.6406189,67.88651611)
\curveto(142.56061238,67.71651039)(142.49561244,67.54151056)(142.4456189,67.36151611)
\curveto(142.39561254,67.18151092)(142.33061261,67.0015111)(142.2506189,66.82151611)
\curveto(142.20061274,66.69151141)(142.15061279,66.55651155)(142.1006189,66.41651611)
\curveto(142.06061288,66.28651182)(142.01061293,66.15651195)(141.9506189,66.02651611)
\curveto(141.78061316,65.61651249)(141.62561331,65.2015129)(141.4856189,64.78151611)
\curveto(141.35561358,64.36151374)(141.20561373,63.94651416)(141.0356189,63.53651611)
\curveto(140.97561396,63.37651473)(140.92061402,63.21651489)(140.8706189,63.05651611)
\curveto(140.82061412,62.89651521)(140.76061418,62.73651537)(140.6906189,62.57651611)
\curveto(140.6406143,62.46651564)(140.59561434,62.36151574)(140.5556189,62.26151611)
\curveto(140.52561441,62.17151593)(140.45561448,62.101516)(140.3456189,62.05151611)
\curveto(140.28561465,62.02151608)(140.21561472,62.0065161)(140.1356189,62.00651611)
\lineto(139.9106189,62.00651611)
\lineto(139.4456189,62.00651611)
\curveto(139.29561564,62.01651609)(139.18561575,62.06651604)(139.1156189,62.15651611)
\curveto(139.04561589,62.23651587)(138.99561594,62.33151577)(138.9656189,62.44151611)
\curveto(138.935616,62.56151554)(138.89561604,62.67651543)(138.8456189,62.78651611)
\curveto(138.78561615,62.92651518)(138.72561621,63.07151503)(138.6656189,63.22151611)
\curveto(138.61561632,63.38151472)(138.56561637,63.53151457)(138.5156189,63.67151611)
\curveto(138.49561644,63.72151438)(138.48061646,63.76151434)(138.4706189,63.79151611)
\curveto(138.46061648,63.83151427)(138.44561649,63.87651423)(138.4256189,63.92651611)
\curveto(138.22561671,64.4065137)(138.0406169,64.89151321)(137.8706189,65.38151611)
\curveto(137.71061723,65.87151223)(137.53061741,66.35651175)(137.3306189,66.83651611)
\curveto(137.27061767,66.99651111)(137.21061773,67.15151095)(137.1506189,67.30151611)
\curveto(137.10061784,67.46151064)(137.04561789,67.62151048)(136.9856189,67.78151611)
\lineto(136.9256189,67.93151611)
\curveto(136.91561802,67.99151011)(136.90061804,68.04651006)(136.8806189,68.09651611)
\curveto(136.80061814,68.26650984)(136.73061821,68.43650967)(136.6706189,68.60651611)
\curveto(136.62061832,68.77650933)(136.56061838,68.94650916)(136.4906189,69.11651611)
\curveto(136.47061847,69.17650893)(136.44561849,69.25650885)(136.4156189,69.35651611)
\curveto(136.38561855,69.45650865)(136.39061855,69.54150856)(136.4306189,69.61151611)
\curveto(136.48061846,69.66150844)(136.5406184,69.69650841)(136.6106189,69.71651611)
\curveto(136.68061826,69.71650839)(136.72061822,69.72150838)(136.7306189,69.73151611)
}
}
{
\newrgbcolor{curcolor}{0 0 0}
\pscustom[linestyle=none,fillstyle=solid,fillcolor=curcolor]
{
\newpath
\moveto(144.7406189,71.23151611)
\curveto(144.66061778,71.29150681)(144.61561782,71.39650671)(144.6056189,71.54651611)
\lineto(144.6056189,72.01151611)
\lineto(144.6056189,72.26651611)
\curveto(144.60561783,72.35650575)(144.62061782,72.43150567)(144.6506189,72.49151611)
\curveto(144.69061775,72.57150553)(144.77061767,72.63150547)(144.8906189,72.67151611)
\curveto(144.91061753,72.68150542)(144.93061751,72.68150542)(144.9506189,72.67151611)
\curveto(144.98061746,72.67150543)(145.00561743,72.67650543)(145.0256189,72.68651611)
\curveto(145.19561724,72.68650542)(145.35561708,72.68150542)(145.5056189,72.67151611)
\curveto(145.65561678,72.66150544)(145.75561668,72.6015055)(145.8056189,72.49151611)
\curveto(145.8356166,72.43150567)(145.85061659,72.35650575)(145.8506189,72.26651611)
\lineto(145.8506189,72.01151611)
\curveto(145.85061659,71.83150627)(145.84561659,71.66150644)(145.8356189,71.50151611)
\curveto(145.8356166,71.34150676)(145.77061667,71.23650687)(145.6406189,71.18651611)
\curveto(145.59061685,71.16650694)(145.5356169,71.15650695)(145.4756189,71.15651611)
\lineto(145.3106189,71.15651611)
\lineto(144.9956189,71.15651611)
\curveto(144.89561754,71.15650695)(144.81061763,71.18150692)(144.7406189,71.23151611)
\moveto(145.8506189,62.72651611)
\lineto(145.8506189,62.41151611)
\curveto(145.86061658,62.31151579)(145.8406166,62.23151587)(145.7906189,62.17151611)
\curveto(145.76061668,62.11151599)(145.71561672,62.07151603)(145.6556189,62.05151611)
\curveto(145.59561684,62.04151606)(145.52561691,62.02651608)(145.4456189,62.00651611)
\lineto(145.2206189,62.00651611)
\curveto(145.09061735,62.0065161)(144.97561746,62.01151609)(144.8756189,62.02151611)
\curveto(144.78561765,62.04151606)(144.71561772,62.09151601)(144.6656189,62.17151611)
\curveto(144.62561781,62.23151587)(144.60561783,62.3065158)(144.6056189,62.39651611)
\lineto(144.6056189,62.68151611)
\lineto(144.6056189,69.02651611)
\lineto(144.6056189,69.34151611)
\curveto(144.60561783,69.45150865)(144.63061781,69.53650857)(144.6806189,69.59651611)
\curveto(144.71061773,69.64650846)(144.75061769,69.67650843)(144.8006189,69.68651611)
\curveto(144.85061759,69.69650841)(144.90561753,69.71150839)(144.9656189,69.73151611)
\curveto(144.98561745,69.73150837)(145.00561743,69.72650838)(145.0256189,69.71651611)
\curveto(145.05561738,69.71650839)(145.08061736,69.72150838)(145.1006189,69.73151611)
\curveto(145.23061721,69.73150837)(145.36061708,69.72650838)(145.4906189,69.71651611)
\curveto(145.63061681,69.71650839)(145.72561671,69.67650843)(145.7756189,69.59651611)
\curveto(145.82561661,69.53650857)(145.85061659,69.45650865)(145.8506189,69.35651611)
\lineto(145.8506189,69.07151611)
\lineto(145.8506189,62.72651611)
}
}
{
\newrgbcolor{curcolor}{0 0 0}
\pscustom[linestyle=none,fillstyle=solid,fillcolor=curcolor]
{
\newpath
\moveto(148.74046265,72.07151611)
\curveto(148.89046064,72.07150603)(149.04046049,72.06650604)(149.19046265,72.05651611)
\curveto(149.34046019,72.05650605)(149.44546008,72.01650609)(149.50546265,71.93651611)
\curveto(149.55545997,71.87650623)(149.58045995,71.79150631)(149.58046265,71.68151611)
\curveto(149.59045994,71.58150652)(149.59545993,71.47650663)(149.59546265,71.36651611)
\lineto(149.59546265,70.49651611)
\curveto(149.59545993,70.41650769)(149.59045994,70.33150777)(149.58046265,70.24151611)
\curveto(149.58045995,70.16150794)(149.59045994,70.09150801)(149.61046265,70.03151611)
\curveto(149.65045988,69.89150821)(149.74045979,69.8015083)(149.88046265,69.76151611)
\curveto(149.9304596,69.75150835)(149.97545955,69.74650836)(150.01546265,69.74651611)
\lineto(150.16546265,69.74651611)
\lineto(150.57046265,69.74651611)
\curveto(150.7304588,69.75650835)(150.84545868,69.74650836)(150.91546265,69.71651611)
\curveto(151.00545852,69.65650845)(151.06545846,69.59650851)(151.09546265,69.53651611)
\curveto(151.11545841,69.49650861)(151.1254584,69.45150865)(151.12546265,69.40151611)
\lineto(151.12546265,69.25151611)
\curveto(151.1254584,69.14150896)(151.12045841,69.03650907)(151.11046265,68.93651611)
\curveto(151.10045843,68.84650926)(151.06545846,68.77650933)(151.00546265,68.72651611)
\curveto(150.94545858,68.67650943)(150.86045867,68.64650946)(150.75046265,68.63651611)
\lineto(150.42046265,68.63651611)
\curveto(150.31045922,68.64650946)(150.20045933,68.65150945)(150.09046265,68.65151611)
\curveto(149.98045955,68.65150945)(149.88545964,68.63650947)(149.80546265,68.60651611)
\curveto(149.73545979,68.57650953)(149.68545984,68.52650958)(149.65546265,68.45651611)
\curveto(149.6254599,68.38650972)(149.60545992,68.3015098)(149.59546265,68.20151611)
\curveto(149.58545994,68.11150999)(149.58045995,68.01151009)(149.58046265,67.90151611)
\curveto(149.59045994,67.8015103)(149.59545993,67.7015104)(149.59546265,67.60151611)
\lineto(149.59546265,64.63151611)
\curveto(149.59545993,64.41151369)(149.59045994,64.17651393)(149.58046265,63.92651611)
\curveto(149.58045995,63.68651442)(149.6254599,63.5015146)(149.71546265,63.37151611)
\curveto(149.76545976,63.29151481)(149.8304597,63.23651487)(149.91046265,63.20651611)
\curveto(149.99045954,63.17651493)(150.08545944,63.15151495)(150.19546265,63.13151611)
\curveto(150.2254593,63.12151498)(150.25545927,63.11651499)(150.28546265,63.11651611)
\curveto(150.3254592,63.12651498)(150.36045917,63.12651498)(150.39046265,63.11651611)
\lineto(150.58546265,63.11651611)
\curveto(150.68545884,63.11651499)(150.77545875,63.106515)(150.85546265,63.08651611)
\curveto(150.94545858,63.07651503)(151.01045852,63.04151506)(151.05046265,62.98151611)
\curveto(151.07045846,62.95151515)(151.08545844,62.89651521)(151.09546265,62.81651611)
\curveto(151.11545841,62.74651536)(151.1254584,62.67151543)(151.12546265,62.59151611)
\curveto(151.13545839,62.51151559)(151.13545839,62.43151567)(151.12546265,62.35151611)
\curveto(151.11545841,62.28151582)(151.09545843,62.22651588)(151.06546265,62.18651611)
\curveto(151.0254585,62.11651599)(150.95045858,62.06651604)(150.84046265,62.03651611)
\curveto(150.76045877,62.01651609)(150.67045886,62.0065161)(150.57046265,62.00651611)
\curveto(150.47045906,62.01651609)(150.38045915,62.02151608)(150.30046265,62.02151611)
\curveto(150.24045929,62.02151608)(150.18045935,62.01651609)(150.12046265,62.00651611)
\curveto(150.06045947,62.0065161)(150.00545952,62.01151609)(149.95546265,62.02151611)
\lineto(149.77546265,62.02151611)
\curveto(149.7254598,62.03151607)(149.67545985,62.03651607)(149.62546265,62.03651611)
\curveto(149.58545994,62.04651606)(149.54045999,62.05151605)(149.49046265,62.05151611)
\curveto(149.29046024,62.101516)(149.11546041,62.15651595)(148.96546265,62.21651611)
\curveto(148.8254607,62.27651583)(148.70546082,62.38151572)(148.60546265,62.53151611)
\curveto(148.46546106,62.73151537)(148.38546114,62.98151512)(148.36546265,63.28151611)
\curveto(148.34546118,63.59151451)(148.33546119,63.92151418)(148.33546265,64.27151611)
\lineto(148.33546265,68.20151611)
\curveto(148.30546122,68.33150977)(148.27546125,68.42650968)(148.24546265,68.48651611)
\curveto(148.2254613,68.54650956)(148.15546137,68.59650951)(148.03546265,68.63651611)
\curveto(147.99546153,68.64650946)(147.95546157,68.64650946)(147.91546265,68.63651611)
\curveto(147.87546165,68.62650948)(147.83546169,68.63150947)(147.79546265,68.65151611)
\lineto(147.55546265,68.65151611)
\curveto(147.4254621,68.65150945)(147.31546221,68.66150944)(147.22546265,68.68151611)
\curveto(147.14546238,68.71150939)(147.09046244,68.77150933)(147.06046265,68.86151611)
\curveto(147.04046249,68.9015092)(147.0254625,68.94650916)(147.01546265,68.99651611)
\lineto(147.01546265,69.14651611)
\curveto(147.01546251,69.28650882)(147.0254625,69.4015087)(147.04546265,69.49151611)
\curveto(147.06546246,69.59150851)(147.1254624,69.66650844)(147.22546265,69.71651611)
\curveto(147.33546219,69.75650835)(147.47546205,69.76650834)(147.64546265,69.74651611)
\curveto(147.8254617,69.72650838)(147.97546155,69.73650837)(148.09546265,69.77651611)
\curveto(148.18546134,69.82650828)(148.25546127,69.89650821)(148.30546265,69.98651611)
\curveto(148.3254612,70.04650806)(148.33546119,70.12150798)(148.33546265,70.21151611)
\lineto(148.33546265,70.46651611)
\lineto(148.33546265,71.39651611)
\lineto(148.33546265,71.63651611)
\curveto(148.33546119,71.72650638)(148.34546118,71.8015063)(148.36546265,71.86151611)
\curveto(148.40546112,71.94150616)(148.48046105,72.0065061)(148.59046265,72.05651611)
\curveto(148.62046091,72.05650605)(148.64546088,72.05650605)(148.66546265,72.05651611)
\curveto(148.69546083,72.06650604)(148.72046081,72.07150603)(148.74046265,72.07151611)
}
}
{
\newrgbcolor{curcolor}{0 0 0}
\pscustom[linestyle=none,fillstyle=solid,fillcolor=curcolor]
{
\newpath
\moveto(159.39725952,62.56151611)
\curveto(159.42725169,62.4015157)(159.41225171,62.26651584)(159.35225952,62.15651611)
\curveto(159.29225183,62.05651605)(159.21225191,61.98151612)(159.11225952,61.93151611)
\curveto(159.06225206,61.91151619)(159.00725211,61.9015162)(158.94725952,61.90151611)
\curveto(158.89725222,61.9015162)(158.84225228,61.89151621)(158.78225952,61.87151611)
\curveto(158.56225256,61.82151628)(158.34225278,61.83651627)(158.12225952,61.91651611)
\curveto(157.91225321,61.98651612)(157.76725335,62.07651603)(157.68725952,62.18651611)
\curveto(157.63725348,62.25651585)(157.59225353,62.33651577)(157.55225952,62.42651611)
\curveto(157.51225361,62.52651558)(157.46225366,62.6065155)(157.40225952,62.66651611)
\curveto(157.38225374,62.68651542)(157.35725376,62.7065154)(157.32725952,62.72651611)
\curveto(157.30725381,62.74651536)(157.27725384,62.75151535)(157.23725952,62.74151611)
\curveto(157.12725399,62.71151539)(157.0222541,62.65651545)(156.92225952,62.57651611)
\curveto(156.83225429,62.49651561)(156.74225438,62.42651568)(156.65225952,62.36651611)
\curveto(156.5222546,62.28651582)(156.38225474,62.21151589)(156.23225952,62.14151611)
\curveto(156.08225504,62.08151602)(155.9222552,62.02651608)(155.75225952,61.97651611)
\curveto(155.65225547,61.94651616)(155.54225558,61.92651618)(155.42225952,61.91651611)
\curveto(155.31225581,61.9065162)(155.20225592,61.89151621)(155.09225952,61.87151611)
\curveto(155.04225608,61.86151624)(154.99725612,61.85651625)(154.95725952,61.85651611)
\lineto(154.85225952,61.85651611)
\curveto(154.74225638,61.83651627)(154.63725648,61.83651627)(154.53725952,61.85651611)
\lineto(154.40225952,61.85651611)
\curveto(154.35225677,61.86651624)(154.30225682,61.87151623)(154.25225952,61.87151611)
\curveto(154.20225692,61.87151623)(154.15725696,61.88151622)(154.11725952,61.90151611)
\curveto(154.07725704,61.91151619)(154.04225708,61.91651619)(154.01225952,61.91651611)
\curveto(153.99225713,61.9065162)(153.96725715,61.9065162)(153.93725952,61.91651611)
\lineto(153.69725952,61.97651611)
\curveto(153.6172575,61.98651612)(153.54225758,62.0065161)(153.47225952,62.03651611)
\curveto(153.17225795,62.16651594)(152.92725819,62.31151579)(152.73725952,62.47151611)
\curveto(152.55725856,62.64151546)(152.40725871,62.87651523)(152.28725952,63.17651611)
\curveto(152.19725892,63.39651471)(152.15225897,63.66151444)(152.15225952,63.97151611)
\lineto(152.15225952,64.28651611)
\curveto(152.16225896,64.33651377)(152.16725895,64.38651372)(152.16725952,64.43651611)
\lineto(152.19725952,64.61651611)
\lineto(152.31725952,64.94651611)
\curveto(152.35725876,65.05651305)(152.40725871,65.15651295)(152.46725952,65.24651611)
\curveto(152.64725847,65.53651257)(152.89225823,65.75151235)(153.20225952,65.89151611)
\curveto(153.51225761,66.03151207)(153.85225727,66.15651195)(154.22225952,66.26651611)
\curveto(154.36225676,66.3065118)(154.50725661,66.33651177)(154.65725952,66.35651611)
\curveto(154.80725631,66.37651173)(154.95725616,66.4015117)(155.10725952,66.43151611)
\curveto(155.17725594,66.45151165)(155.24225588,66.46151164)(155.30225952,66.46151611)
\curveto(155.37225575,66.46151164)(155.44725567,66.47151163)(155.52725952,66.49151611)
\curveto(155.59725552,66.51151159)(155.66725545,66.52151158)(155.73725952,66.52151611)
\curveto(155.80725531,66.53151157)(155.88225524,66.54651156)(155.96225952,66.56651611)
\curveto(156.21225491,66.62651148)(156.44725467,66.67651143)(156.66725952,66.71651611)
\curveto(156.88725423,66.76651134)(157.06225406,66.88151122)(157.19225952,67.06151611)
\curveto(157.25225387,67.14151096)(157.30225382,67.24151086)(157.34225952,67.36151611)
\curveto(157.38225374,67.49151061)(157.38225374,67.63151047)(157.34225952,67.78151611)
\curveto(157.28225384,68.02151008)(157.19225393,68.21150989)(157.07225952,68.35151611)
\curveto(156.96225416,68.49150961)(156.80225432,68.6015095)(156.59225952,68.68151611)
\curveto(156.47225465,68.73150937)(156.32725479,68.76650934)(156.15725952,68.78651611)
\curveto(155.99725512,68.8065093)(155.82725529,68.81650929)(155.64725952,68.81651611)
\curveto(155.46725565,68.81650929)(155.29225583,68.8065093)(155.12225952,68.78651611)
\curveto(154.95225617,68.76650934)(154.80725631,68.73650937)(154.68725952,68.69651611)
\curveto(154.5172566,68.63650947)(154.35225677,68.55150955)(154.19225952,68.44151611)
\curveto(154.11225701,68.38150972)(154.03725708,68.3015098)(153.96725952,68.20151611)
\curveto(153.90725721,68.11150999)(153.85225727,68.01151009)(153.80225952,67.90151611)
\curveto(153.77225735,67.82151028)(153.74225738,67.73651037)(153.71225952,67.64651611)
\curveto(153.69225743,67.55651055)(153.64725747,67.48651062)(153.57725952,67.43651611)
\curveto(153.53725758,67.4065107)(153.46725765,67.38151072)(153.36725952,67.36151611)
\curveto(153.27725784,67.35151075)(153.18225794,67.34651076)(153.08225952,67.34651611)
\curveto(152.98225814,67.34651076)(152.88225824,67.35151075)(152.78225952,67.36151611)
\curveto(152.69225843,67.38151072)(152.62725849,67.4065107)(152.58725952,67.43651611)
\curveto(152.54725857,67.46651064)(152.5172586,67.51651059)(152.49725952,67.58651611)
\curveto(152.47725864,67.65651045)(152.47725864,67.73151037)(152.49725952,67.81151611)
\curveto(152.52725859,67.94151016)(152.55725856,68.06151004)(152.58725952,68.17151611)
\curveto(152.62725849,68.29150981)(152.67225845,68.4065097)(152.72225952,68.51651611)
\curveto(152.91225821,68.86650924)(153.15225797,69.13650897)(153.44225952,69.32651611)
\curveto(153.73225739,69.52650858)(154.09225703,69.68650842)(154.52225952,69.80651611)
\curveto(154.6222565,69.82650828)(154.7222564,69.84150826)(154.82225952,69.85151611)
\curveto(154.93225619,69.86150824)(155.04225608,69.87650823)(155.15225952,69.89651611)
\curveto(155.19225593,69.9065082)(155.25725586,69.9065082)(155.34725952,69.89651611)
\curveto(155.43725568,69.89650821)(155.49225563,69.9065082)(155.51225952,69.92651611)
\curveto(156.21225491,69.93650817)(156.8222543,69.85650825)(157.34225952,69.68651611)
\curveto(157.86225326,69.51650859)(158.22725289,69.19150891)(158.43725952,68.71151611)
\curveto(158.52725259,68.51150959)(158.57725254,68.27650983)(158.58725952,68.00651611)
\curveto(158.60725251,67.74651036)(158.6172525,67.47151063)(158.61725952,67.18151611)
\lineto(158.61725952,63.86651611)
\curveto(158.6172525,63.72651438)(158.6222525,63.59151451)(158.63225952,63.46151611)
\curveto(158.64225248,63.33151477)(158.67225245,63.22651488)(158.72225952,63.14651611)
\curveto(158.77225235,63.07651503)(158.83725228,63.02651508)(158.91725952,62.99651611)
\curveto(159.00725211,62.95651515)(159.09225203,62.92651518)(159.17225952,62.90651611)
\curveto(159.25225187,62.89651521)(159.31225181,62.85151525)(159.35225952,62.77151611)
\curveto(159.37225175,62.74151536)(159.38225174,62.71151539)(159.38225952,62.68151611)
\curveto(159.38225174,62.65151545)(159.38725173,62.61151549)(159.39725952,62.56151611)
\moveto(157.25225952,64.22651611)
\curveto(157.31225381,64.36651374)(157.34225378,64.52651358)(157.34225952,64.70651611)
\curveto(157.35225377,64.89651321)(157.35725376,65.09151301)(157.35725952,65.29151611)
\curveto(157.35725376,65.4015127)(157.35225377,65.5015126)(157.34225952,65.59151611)
\curveto(157.33225379,65.68151242)(157.29225383,65.75151235)(157.22225952,65.80151611)
\curveto(157.19225393,65.82151228)(157.122254,65.83151227)(157.01225952,65.83151611)
\curveto(156.99225413,65.81151229)(156.95725416,65.8015123)(156.90725952,65.80151611)
\curveto(156.85725426,65.8015123)(156.81225431,65.79151231)(156.77225952,65.77151611)
\curveto(156.69225443,65.75151235)(156.60225452,65.73151237)(156.50225952,65.71151611)
\lineto(156.20225952,65.65151611)
\curveto(156.17225495,65.65151245)(156.13725498,65.64651246)(156.09725952,65.63651611)
\lineto(155.99225952,65.63651611)
\curveto(155.84225528,65.59651251)(155.67725544,65.57151253)(155.49725952,65.56151611)
\curveto(155.32725579,65.56151254)(155.16725595,65.54151256)(155.01725952,65.50151611)
\curveto(154.93725618,65.48151262)(154.86225626,65.46151264)(154.79225952,65.44151611)
\curveto(154.73225639,65.43151267)(154.66225646,65.41651269)(154.58225952,65.39651611)
\curveto(154.4222567,65.34651276)(154.27225685,65.28151282)(154.13225952,65.20151611)
\curveto(153.99225713,65.13151297)(153.87225725,65.04151306)(153.77225952,64.93151611)
\curveto(153.67225745,64.82151328)(153.59725752,64.68651342)(153.54725952,64.52651611)
\curveto(153.49725762,64.37651373)(153.47725764,64.19151391)(153.48725952,63.97151611)
\curveto(153.48725763,63.87151423)(153.50225762,63.77651433)(153.53225952,63.68651611)
\curveto(153.57225755,63.6065145)(153.6172575,63.53151457)(153.66725952,63.46151611)
\curveto(153.74725737,63.35151475)(153.85225727,63.25651485)(153.98225952,63.17651611)
\curveto(154.11225701,63.106515)(154.25225687,63.04651506)(154.40225952,62.99651611)
\curveto(154.45225667,62.98651512)(154.50225662,62.98151512)(154.55225952,62.98151611)
\curveto(154.60225652,62.98151512)(154.65225647,62.97651513)(154.70225952,62.96651611)
\curveto(154.77225635,62.94651516)(154.85725626,62.93151517)(154.95725952,62.92151611)
\curveto(155.06725605,62.92151518)(155.15725596,62.93151517)(155.22725952,62.95151611)
\curveto(155.28725583,62.97151513)(155.34725577,62.97651513)(155.40725952,62.96651611)
\curveto(155.46725565,62.96651514)(155.52725559,62.97651513)(155.58725952,62.99651611)
\curveto(155.66725545,63.01651509)(155.74225538,63.03151507)(155.81225952,63.04151611)
\curveto(155.89225523,63.05151505)(155.96725515,63.07151503)(156.03725952,63.10151611)
\curveto(156.32725479,63.22151488)(156.57225455,63.36651474)(156.77225952,63.53651611)
\curveto(156.98225414,63.7065144)(157.14225398,63.93651417)(157.25225952,64.22651611)
}
}
{
\newrgbcolor{curcolor}{0 0 0}
\pscustom[linestyle=none,fillstyle=solid,fillcolor=curcolor]
{
\newpath
\moveto(167.52890015,62.81651611)
\lineto(167.52890015,62.42651611)
\curveto(167.52889227,62.3065158)(167.5038923,62.2065159)(167.45390015,62.12651611)
\curveto(167.4038924,62.05651605)(167.31889248,62.01651609)(167.19890015,62.00651611)
\lineto(166.85390015,62.00651611)
\curveto(166.79389301,62.0065161)(166.73389307,62.0015161)(166.67390015,61.99151611)
\curveto(166.62389318,61.99151611)(166.57889322,62.0015161)(166.53890015,62.02151611)
\curveto(166.44889335,62.04151606)(166.38889341,62.08151602)(166.35890015,62.14151611)
\curveto(166.31889348,62.19151591)(166.29389351,62.25151585)(166.28390015,62.32151611)
\curveto(166.28389352,62.39151571)(166.26889353,62.46151564)(166.23890015,62.53151611)
\curveto(166.22889357,62.55151555)(166.21389359,62.56651554)(166.19390015,62.57651611)
\curveto(166.18389362,62.59651551)(166.16889363,62.61651549)(166.14890015,62.63651611)
\curveto(166.04889375,62.64651546)(165.96889383,62.62651548)(165.90890015,62.57651611)
\curveto(165.85889394,62.52651558)(165.803894,62.47651563)(165.74390015,62.42651611)
\curveto(165.54389426,62.27651583)(165.34389446,62.16151594)(165.14390015,62.08151611)
\curveto(164.96389484,62.0015161)(164.75389505,61.94151616)(164.51390015,61.90151611)
\curveto(164.28389552,61.86151624)(164.04389576,61.84151626)(163.79390015,61.84151611)
\curveto(163.55389625,61.83151627)(163.31389649,61.84651626)(163.07390015,61.88651611)
\curveto(162.83389697,61.91651619)(162.62389718,61.97151613)(162.44390015,62.05151611)
\curveto(161.92389788,62.27151583)(161.5038983,62.56651554)(161.18390015,62.93651611)
\curveto(160.86389894,63.31651479)(160.61389919,63.78651432)(160.43390015,64.34651611)
\curveto(160.39389941,64.43651367)(160.36389944,64.52651358)(160.34390015,64.61651611)
\curveto(160.33389947,64.71651339)(160.31389949,64.81651329)(160.28390015,64.91651611)
\curveto(160.27389953,64.96651314)(160.26889953,65.01651309)(160.26890015,65.06651611)
\curveto(160.26889953,65.11651299)(160.26389954,65.16651294)(160.25390015,65.21651611)
\curveto(160.23389957,65.26651284)(160.22389958,65.31651279)(160.22390015,65.36651611)
\curveto(160.23389957,65.42651268)(160.23389957,65.48151262)(160.22390015,65.53151611)
\lineto(160.22390015,65.68151611)
\curveto(160.2038996,65.73151237)(160.19389961,65.79651231)(160.19390015,65.87651611)
\curveto(160.19389961,65.95651215)(160.2038996,66.02151208)(160.22390015,66.07151611)
\lineto(160.22390015,66.23651611)
\curveto(160.24389956,66.3065118)(160.24889955,66.37651173)(160.23890015,66.44651611)
\curveto(160.23889956,66.52651158)(160.24889955,66.6015115)(160.26890015,66.67151611)
\curveto(160.27889952,66.72151138)(160.28389952,66.76651134)(160.28390015,66.80651611)
\curveto(160.28389952,66.84651126)(160.28889951,66.89151121)(160.29890015,66.94151611)
\curveto(160.32889947,67.04151106)(160.35389945,67.13651097)(160.37390015,67.22651611)
\curveto(160.39389941,67.32651078)(160.41889938,67.42151068)(160.44890015,67.51151611)
\curveto(160.57889922,67.89151021)(160.74389906,68.23150987)(160.94390015,68.53151611)
\curveto(161.15389865,68.84150926)(161.4038984,69.09650901)(161.69390015,69.29651611)
\curveto(161.86389794,69.41650869)(162.03889776,69.51650859)(162.21890015,69.59651611)
\curveto(162.40889739,69.67650843)(162.61389719,69.74650836)(162.83390015,69.80651611)
\curveto(162.9038969,69.81650829)(162.96889683,69.82650828)(163.02890015,69.83651611)
\curveto(163.0988967,69.84650826)(163.16889663,69.86150824)(163.23890015,69.88151611)
\lineto(163.38890015,69.88151611)
\curveto(163.46889633,69.9015082)(163.58389622,69.91150819)(163.73390015,69.91151611)
\curveto(163.89389591,69.91150819)(164.01389579,69.9015082)(164.09390015,69.88151611)
\curveto(164.13389567,69.87150823)(164.18889561,69.86650824)(164.25890015,69.86651611)
\curveto(164.36889543,69.83650827)(164.47889532,69.81150829)(164.58890015,69.79151611)
\curveto(164.6988951,69.78150832)(164.803895,69.75150835)(164.90390015,69.70151611)
\curveto(165.05389475,69.64150846)(165.19389461,69.57650853)(165.32390015,69.50651611)
\curveto(165.46389434,69.43650867)(165.59389421,69.35650875)(165.71390015,69.26651611)
\curveto(165.77389403,69.21650889)(165.83389397,69.16150894)(165.89390015,69.10151611)
\curveto(165.96389384,69.05150905)(166.05389375,69.03650907)(166.16390015,69.05651611)
\curveto(166.18389362,69.08650902)(166.1988936,69.11150899)(166.20890015,69.13151611)
\curveto(166.22889357,69.15150895)(166.24389356,69.18150892)(166.25390015,69.22151611)
\curveto(166.28389352,69.31150879)(166.29389351,69.42650868)(166.28390015,69.56651611)
\lineto(166.28390015,69.94151611)
\lineto(166.28390015,71.66651611)
\lineto(166.28390015,72.13151611)
\curveto(166.28389352,72.31150579)(166.30889349,72.44150566)(166.35890015,72.52151611)
\curveto(166.3988934,72.59150551)(166.45889334,72.63650547)(166.53890015,72.65651611)
\curveto(166.55889324,72.65650545)(166.58389322,72.65650545)(166.61390015,72.65651611)
\curveto(166.64389316,72.66650544)(166.66889313,72.67150543)(166.68890015,72.67151611)
\curveto(166.82889297,72.68150542)(166.97389283,72.68150542)(167.12390015,72.67151611)
\curveto(167.28389252,72.67150543)(167.39389241,72.63150547)(167.45390015,72.55151611)
\curveto(167.5038923,72.47150563)(167.52889227,72.37150573)(167.52890015,72.25151611)
\lineto(167.52890015,71.87651611)
\lineto(167.52890015,62.81651611)
\moveto(166.31390015,65.65151611)
\curveto(166.33389347,65.7015124)(166.34389346,65.76651234)(166.34390015,65.84651611)
\curveto(166.34389346,65.93651217)(166.33389347,66.0065121)(166.31390015,66.05651611)
\lineto(166.31390015,66.28151611)
\curveto(166.29389351,66.37151173)(166.27889352,66.46151164)(166.26890015,66.55151611)
\curveto(166.25889354,66.65151145)(166.23889356,66.74151136)(166.20890015,66.82151611)
\curveto(166.18889361,66.9015112)(166.16889363,66.97651113)(166.14890015,67.04651611)
\curveto(166.13889366,67.11651099)(166.11889368,67.18651092)(166.08890015,67.25651611)
\curveto(165.96889383,67.55651055)(165.81389399,67.82151028)(165.62390015,68.05151611)
\curveto(165.43389437,68.28150982)(165.19389461,68.46150964)(164.90390015,68.59151611)
\curveto(164.803895,68.64150946)(164.6988951,68.67650943)(164.58890015,68.69651611)
\curveto(164.48889531,68.72650938)(164.37889542,68.75150935)(164.25890015,68.77151611)
\curveto(164.17889562,68.79150931)(164.08889571,68.8015093)(163.98890015,68.80151611)
\lineto(163.71890015,68.80151611)
\curveto(163.66889613,68.79150931)(163.62389618,68.78150932)(163.58390015,68.77151611)
\lineto(163.44890015,68.77151611)
\curveto(163.36889643,68.75150935)(163.28389652,68.73150937)(163.19390015,68.71151611)
\curveto(163.11389669,68.69150941)(163.03389677,68.66650944)(162.95390015,68.63651611)
\curveto(162.63389717,68.49650961)(162.37389743,68.29150981)(162.17390015,68.02151611)
\curveto(161.98389782,67.76151034)(161.82889797,67.45651065)(161.70890015,67.10651611)
\curveto(161.66889813,66.99651111)(161.63889816,66.88151122)(161.61890015,66.76151611)
\curveto(161.60889819,66.65151145)(161.59389821,66.54151156)(161.57390015,66.43151611)
\curveto(161.57389823,66.39151171)(161.56889823,66.35151175)(161.55890015,66.31151611)
\lineto(161.55890015,66.20651611)
\curveto(161.53889826,66.15651195)(161.52889827,66.101512)(161.52890015,66.04151611)
\curveto(161.53889826,65.98151212)(161.54389826,65.92651218)(161.54390015,65.87651611)
\lineto(161.54390015,65.54651611)
\curveto(161.54389826,65.44651266)(161.55389825,65.35151275)(161.57390015,65.26151611)
\curveto(161.58389822,65.23151287)(161.58889821,65.18151292)(161.58890015,65.11151611)
\curveto(161.60889819,65.04151306)(161.62389818,64.97151313)(161.63390015,64.90151611)
\lineto(161.69390015,64.69151611)
\curveto(161.803898,64.34151376)(161.95389785,64.04151406)(162.14390015,63.79151611)
\curveto(162.33389747,63.54151456)(162.57389723,63.33651477)(162.86390015,63.17651611)
\curveto(162.95389685,63.12651498)(163.04389676,63.08651502)(163.13390015,63.05651611)
\curveto(163.22389658,63.02651508)(163.32389648,62.99651511)(163.43390015,62.96651611)
\curveto(163.48389632,62.94651516)(163.53389627,62.94151516)(163.58390015,62.95151611)
\curveto(163.64389616,62.96151514)(163.6988961,62.95651515)(163.74890015,62.93651611)
\curveto(163.78889601,62.92651518)(163.82889597,62.92151518)(163.86890015,62.92151611)
\lineto(164.00390015,62.92151611)
\lineto(164.13890015,62.92151611)
\curveto(164.16889563,62.93151517)(164.21889558,62.93651517)(164.28890015,62.93651611)
\curveto(164.36889543,62.95651515)(164.44889535,62.97151513)(164.52890015,62.98151611)
\curveto(164.60889519,63.0015151)(164.68389512,63.02651508)(164.75390015,63.05651611)
\curveto(165.08389472,63.19651491)(165.34889445,63.37151473)(165.54890015,63.58151611)
\curveto(165.75889404,63.8015143)(165.93389387,64.07651403)(166.07390015,64.40651611)
\curveto(166.12389368,64.51651359)(166.15889364,64.62651348)(166.17890015,64.73651611)
\curveto(166.1988936,64.84651326)(166.22389358,64.95651315)(166.25390015,65.06651611)
\curveto(166.27389353,65.106513)(166.28389352,65.14151296)(166.28390015,65.17151611)
\curveto(166.28389352,65.21151289)(166.28889351,65.25151285)(166.29890015,65.29151611)
\curveto(166.30889349,65.35151275)(166.30889349,65.41151269)(166.29890015,65.47151611)
\curveto(166.2988935,65.53151257)(166.3038935,65.59151251)(166.31390015,65.65151611)
}
}
{
\newrgbcolor{curcolor}{0 0 0}
\pscustom[linestyle=none,fillstyle=solid,fillcolor=curcolor]
{
\newpath
\moveto(176.60015015,66.20651611)
\curveto(176.62014209,66.14651196)(176.63014208,66.05151205)(176.63015015,65.92151611)
\curveto(176.63014208,65.8015123)(176.62514208,65.71651239)(176.61515015,65.66651611)
\lineto(176.61515015,65.51651611)
\curveto(176.6051421,65.43651267)(176.59514211,65.36151274)(176.58515015,65.29151611)
\curveto(176.58514212,65.23151287)(176.58014213,65.16151294)(176.57015015,65.08151611)
\curveto(176.55014216,65.02151308)(176.53514217,64.96151314)(176.52515015,64.90151611)
\curveto(176.52514218,64.84151326)(176.51514219,64.78151332)(176.49515015,64.72151611)
\curveto(176.45514225,64.59151351)(176.42014229,64.46151364)(176.39015015,64.33151611)
\curveto(176.36014235,64.2015139)(176.32014239,64.08151402)(176.27015015,63.97151611)
\curveto(176.06014265,63.49151461)(175.78014293,63.08651502)(175.43015015,62.75651611)
\curveto(175.08014363,62.43651567)(174.65014406,62.19151591)(174.14015015,62.02151611)
\curveto(174.03014468,61.98151612)(173.9101448,61.95151615)(173.78015015,61.93151611)
\curveto(173.66014505,61.91151619)(173.53514517,61.89151621)(173.40515015,61.87151611)
\curveto(173.34514536,61.86151624)(173.28014543,61.85651625)(173.21015015,61.85651611)
\curveto(173.15014556,61.84651626)(173.09014562,61.84151626)(173.03015015,61.84151611)
\curveto(172.99014572,61.83151627)(172.93014578,61.82651628)(172.85015015,61.82651611)
\curveto(172.78014593,61.82651628)(172.73014598,61.83151627)(172.70015015,61.84151611)
\curveto(172.66014605,61.85151625)(172.62014609,61.85651625)(172.58015015,61.85651611)
\curveto(172.54014617,61.84651626)(172.5051462,61.84651626)(172.47515015,61.85651611)
\lineto(172.38515015,61.85651611)
\lineto(172.02515015,61.90151611)
\curveto(171.88514682,61.94151616)(171.75014696,61.98151612)(171.62015015,62.02151611)
\curveto(171.49014722,62.06151604)(171.36514734,62.106516)(171.24515015,62.15651611)
\curveto(170.79514791,62.35651575)(170.42514828,62.61651549)(170.13515015,62.93651611)
\curveto(169.84514886,63.25651485)(169.6051491,63.64651446)(169.41515015,64.10651611)
\curveto(169.36514934,64.2065139)(169.32514938,64.3065138)(169.29515015,64.40651611)
\curveto(169.27514943,64.5065136)(169.25514945,64.61151349)(169.23515015,64.72151611)
\curveto(169.21514949,64.76151334)(169.2051495,64.79151331)(169.20515015,64.81151611)
\curveto(169.21514949,64.84151326)(169.21514949,64.87651323)(169.20515015,64.91651611)
\curveto(169.18514952,64.99651311)(169.17014954,65.07651303)(169.16015015,65.15651611)
\curveto(169.16014955,65.24651286)(169.15014956,65.33151277)(169.13015015,65.41151611)
\lineto(169.13015015,65.53151611)
\curveto(169.13014958,65.57151253)(169.12514958,65.61651249)(169.11515015,65.66651611)
\curveto(169.1051496,65.71651239)(169.10014961,65.8015123)(169.10015015,65.92151611)
\curveto(169.10014961,66.05151205)(169.1101496,66.14651196)(169.13015015,66.20651611)
\curveto(169.15014956,66.27651183)(169.15514955,66.34651176)(169.14515015,66.41651611)
\curveto(169.13514957,66.48651162)(169.14014957,66.55651155)(169.16015015,66.62651611)
\curveto(169.17014954,66.67651143)(169.17514953,66.71651139)(169.17515015,66.74651611)
\curveto(169.18514952,66.78651132)(169.19514951,66.83151127)(169.20515015,66.88151611)
\curveto(169.23514947,67.0015111)(169.26014945,67.12151098)(169.28015015,67.24151611)
\curveto(169.3101494,67.36151074)(169.35014936,67.47651063)(169.40015015,67.58651611)
\curveto(169.55014916,67.95651015)(169.73014898,68.28650982)(169.94015015,68.57651611)
\curveto(170.16014855,68.87650923)(170.42514828,69.12650898)(170.73515015,69.32651611)
\curveto(170.85514785,69.4065087)(170.98014773,69.47150863)(171.11015015,69.52151611)
\curveto(171.24014747,69.58150852)(171.37514733,69.64150846)(171.51515015,69.70151611)
\curveto(171.63514707,69.75150835)(171.76514694,69.78150832)(171.90515015,69.79151611)
\curveto(172.04514666,69.81150829)(172.18514652,69.84150826)(172.32515015,69.88151611)
\lineto(172.52015015,69.88151611)
\curveto(172.59014612,69.89150821)(172.65514605,69.9015082)(172.71515015,69.91151611)
\curveto(173.6051451,69.92150818)(174.34514436,69.73650837)(174.93515015,69.35651611)
\curveto(175.52514318,68.97650913)(175.95014276,68.48150962)(176.21015015,67.87151611)
\curveto(176.26014245,67.77151033)(176.30014241,67.67151043)(176.33015015,67.57151611)
\curveto(176.36014235,67.47151063)(176.39514231,67.36651074)(176.43515015,67.25651611)
\curveto(176.46514224,67.14651096)(176.49014222,67.02651108)(176.51015015,66.89651611)
\curveto(176.53014218,66.77651133)(176.55514215,66.65151145)(176.58515015,66.52151611)
\curveto(176.59514211,66.47151163)(176.59514211,66.41651169)(176.58515015,66.35651611)
\curveto(176.58514212,66.3065118)(176.59014212,66.25651185)(176.60015015,66.20651611)
\moveto(175.26515015,65.35151611)
\curveto(175.28514342,65.42151268)(175.29014342,65.5015126)(175.28015015,65.59151611)
\lineto(175.28015015,65.84651611)
\curveto(175.28014343,66.23651187)(175.24514346,66.56651154)(175.17515015,66.83651611)
\curveto(175.14514356,66.91651119)(175.12014359,66.99651111)(175.10015015,67.07651611)
\curveto(175.08014363,67.15651095)(175.05514365,67.23151087)(175.02515015,67.30151611)
\curveto(174.74514396,67.95151015)(174.30014441,68.4015097)(173.69015015,68.65151611)
\curveto(173.62014509,68.68150942)(173.54514516,68.7015094)(173.46515015,68.71151611)
\lineto(173.22515015,68.77151611)
\curveto(173.14514556,68.79150931)(173.06014565,68.8015093)(172.97015015,68.80151611)
\lineto(172.70015015,68.80151611)
\lineto(172.43015015,68.75651611)
\curveto(172.33014638,68.73650937)(172.23514647,68.71150939)(172.14515015,68.68151611)
\curveto(172.06514664,68.66150944)(171.98514672,68.63150947)(171.90515015,68.59151611)
\curveto(171.83514687,68.57150953)(171.77014694,68.54150956)(171.71015015,68.50151611)
\curveto(171.65014706,68.46150964)(171.59514711,68.42150968)(171.54515015,68.38151611)
\curveto(171.3051474,68.21150989)(171.1101476,68.0065101)(170.96015015,67.76651611)
\curveto(170.8101479,67.52651058)(170.68014803,67.24651086)(170.57015015,66.92651611)
\curveto(170.54014817,66.82651128)(170.52014819,66.72151138)(170.51015015,66.61151611)
\curveto(170.50014821,66.51151159)(170.48514822,66.4065117)(170.46515015,66.29651611)
\curveto(170.45514825,66.25651185)(170.45014826,66.19151191)(170.45015015,66.10151611)
\curveto(170.44014827,66.07151203)(170.43514827,66.03651207)(170.43515015,65.99651611)
\curveto(170.44514826,65.95651215)(170.45014826,65.91151219)(170.45015015,65.86151611)
\lineto(170.45015015,65.56151611)
\curveto(170.45014826,65.46151264)(170.46014825,65.37151273)(170.48015015,65.29151611)
\lineto(170.51015015,65.11151611)
\curveto(170.53014818,65.01151309)(170.54514816,64.91151319)(170.55515015,64.81151611)
\curveto(170.57514813,64.72151338)(170.6051481,64.63651347)(170.64515015,64.55651611)
\curveto(170.74514796,64.31651379)(170.86014785,64.09151401)(170.99015015,63.88151611)
\curveto(171.13014758,63.67151443)(171.30014741,63.49651461)(171.50015015,63.35651611)
\curveto(171.55014716,63.32651478)(171.59514711,63.3015148)(171.63515015,63.28151611)
\curveto(171.67514703,63.26151484)(171.72014699,63.23651487)(171.77015015,63.20651611)
\curveto(171.85014686,63.15651495)(171.93514677,63.11151499)(172.02515015,63.07151611)
\curveto(172.12514658,63.04151506)(172.23014648,63.01151509)(172.34015015,62.98151611)
\curveto(172.39014632,62.96151514)(172.43514627,62.95151515)(172.47515015,62.95151611)
\curveto(172.52514618,62.96151514)(172.57514613,62.96151514)(172.62515015,62.95151611)
\curveto(172.65514605,62.94151516)(172.71514599,62.93151517)(172.80515015,62.92151611)
\curveto(172.9051458,62.91151519)(172.98014573,62.91651519)(173.03015015,62.93651611)
\curveto(173.07014564,62.94651516)(173.1101456,62.94651516)(173.15015015,62.93651611)
\curveto(173.19014552,62.93651517)(173.23014548,62.94651516)(173.27015015,62.96651611)
\curveto(173.35014536,62.98651512)(173.43014528,63.0015151)(173.51015015,63.01151611)
\curveto(173.59014512,63.03151507)(173.66514504,63.05651505)(173.73515015,63.08651611)
\curveto(174.07514463,63.22651488)(174.35014436,63.42151468)(174.56015015,63.67151611)
\curveto(174.77014394,63.92151418)(174.94514376,64.21651389)(175.08515015,64.55651611)
\curveto(175.13514357,64.67651343)(175.16514354,64.8015133)(175.17515015,64.93151611)
\curveto(175.19514351,65.07151303)(175.22514348,65.21151289)(175.26515015,65.35151611)
}
}
{
\newrgbcolor{curcolor}{0 0 0}
\pscustom[linestyle=none,fillstyle=solid,fillcolor=curcolor]
{
\newpath
\moveto(317.29141846,62.78651611)
\curveto(317.31140891,62.73651537)(317.33640889,62.67651543)(317.36641846,62.60651611)
\curveto(317.39640883,62.53651557)(317.41640881,62.46151564)(317.42641846,62.38151611)
\curveto(317.44640878,62.31151579)(317.44640878,62.24151586)(317.42641846,62.17151611)
\curveto(317.41640881,62.11151599)(317.37640885,62.06651604)(317.30641846,62.03651611)
\curveto(317.25640897,62.01651609)(317.19640903,62.0065161)(317.12641846,62.00651611)
\lineto(316.91641846,62.00651611)
\lineto(316.46641846,62.00651611)
\curveto(316.31640991,62.0065161)(316.19641003,62.03151607)(316.10641846,62.08151611)
\curveto(316.00641022,62.14151596)(315.93141029,62.24651586)(315.88141846,62.39651611)
\curveto(315.84141038,62.54651556)(315.79641043,62.68151542)(315.74641846,62.80151611)
\curveto(315.63641059,63.06151504)(315.53641069,63.33151477)(315.44641846,63.61151611)
\curveto(315.35641087,63.89151421)(315.25641097,64.16651394)(315.14641846,64.43651611)
\curveto(315.11641111,64.52651358)(315.08641114,64.61151349)(315.05641846,64.69151611)
\curveto(315.03641119,64.77151333)(315.00641122,64.84651326)(314.96641846,64.91651611)
\curveto(314.93641129,64.98651312)(314.89141133,65.04651306)(314.83141846,65.09651611)
\curveto(314.77141145,65.14651296)(314.69141153,65.18651292)(314.59141846,65.21651611)
\curveto(314.54141168,65.23651287)(314.48141174,65.24151286)(314.41141846,65.23151611)
\lineto(314.21641846,65.23151611)
\lineto(311.38141846,65.23151611)
\lineto(311.08141846,65.23151611)
\curveto(310.97141525,65.24151286)(310.86641536,65.24151286)(310.76641846,65.23151611)
\curveto(310.66641556,65.22151288)(310.57141565,65.2065129)(310.48141846,65.18651611)
\curveto(310.40141582,65.16651294)(310.34141588,65.12651298)(310.30141846,65.06651611)
\curveto(310.221416,64.96651314)(310.16141606,64.85151325)(310.12141846,64.72151611)
\curveto(310.09141613,64.6015135)(310.05141617,64.47651363)(310.00141846,64.34651611)
\curveto(309.90141632,64.11651399)(309.80641642,63.87651423)(309.71641846,63.62651611)
\curveto(309.63641659,63.37651473)(309.54641668,63.13651497)(309.44641846,62.90651611)
\curveto(309.4264168,62.84651526)(309.40141682,62.77651533)(309.37141846,62.69651611)
\curveto(309.35141687,62.62651548)(309.3264169,62.55151555)(309.29641846,62.47151611)
\curveto(309.26641696,62.39151571)(309.23141699,62.31651579)(309.19141846,62.24651611)
\curveto(309.16141706,62.18651592)(309.1264171,62.14151596)(309.08641846,62.11151611)
\curveto(309.00641722,62.05151605)(308.89641733,62.01651609)(308.75641846,62.00651611)
\lineto(308.33641846,62.00651611)
\lineto(308.09641846,62.00651611)
\curveto(308.0264182,62.01651609)(307.96641826,62.04151606)(307.91641846,62.08151611)
\curveto(307.86641836,62.11151599)(307.83641839,62.15651595)(307.82641846,62.21651611)
\curveto(307.8264184,62.27651583)(307.83141839,62.33651577)(307.84141846,62.39651611)
\curveto(307.86141836,62.46651564)(307.88141834,62.53151557)(307.90141846,62.59151611)
\curveto(307.93141829,62.66151544)(307.95641827,62.71151539)(307.97641846,62.74151611)
\curveto(308.11641811,63.06151504)(308.24141798,63.37651473)(308.35141846,63.68651611)
\curveto(308.46141776,64.0065141)(308.58141764,64.32651378)(308.71141846,64.64651611)
\curveto(308.80141742,64.86651324)(308.88641734,65.08151302)(308.96641846,65.29151611)
\curveto(309.04641718,65.51151259)(309.13141709,65.73151237)(309.22141846,65.95151611)
\curveto(309.5214167,66.67151143)(309.80641642,67.39651071)(310.07641846,68.12651611)
\curveto(310.34641588,68.86650924)(310.63141559,69.6015085)(310.93141846,70.33151611)
\curveto(311.04141518,70.59150751)(311.14141508,70.85650725)(311.23141846,71.12651611)
\curveto(311.33141489,71.39650671)(311.43641479,71.66150644)(311.54641846,71.92151611)
\curveto(311.59641463,72.03150607)(311.64141458,72.15150595)(311.68141846,72.28151611)
\curveto(311.73141449,72.42150568)(311.80141442,72.52150558)(311.89141846,72.58151611)
\curveto(311.93141429,72.62150548)(311.99641423,72.65150545)(312.08641846,72.67151611)
\curveto(312.10641412,72.68150542)(312.1264141,72.68150542)(312.14641846,72.67151611)
\curveto(312.17641405,72.67150543)(312.20141402,72.67650543)(312.22141846,72.68651611)
\curveto(312.40141382,72.68650542)(312.61141361,72.68650542)(312.85141846,72.68651611)
\curveto(313.09141313,72.69650541)(313.26641296,72.66150544)(313.37641846,72.58151611)
\curveto(313.45641277,72.52150558)(313.51641271,72.42150568)(313.55641846,72.28151611)
\curveto(313.60641262,72.15150595)(313.65641257,72.03150607)(313.70641846,71.92151611)
\curveto(313.80641242,71.69150641)(313.89641233,71.46150664)(313.97641846,71.23151611)
\curveto(314.05641217,71.0015071)(314.14641208,70.77150733)(314.24641846,70.54151611)
\curveto(314.3264119,70.34150776)(314.40141182,70.13650797)(314.47141846,69.92651611)
\curveto(314.55141167,69.71650839)(314.63641159,69.51150859)(314.72641846,69.31151611)
\curveto(315.0264112,68.58150952)(315.31141091,67.84151026)(315.58141846,67.09151611)
\curveto(315.86141036,66.35151175)(316.15641007,65.61651249)(316.46641846,64.88651611)
\curveto(316.50640972,64.79651331)(316.53640969,64.71151339)(316.55641846,64.63151611)
\curveto(316.58640964,64.55151355)(316.61640961,64.46651364)(316.64641846,64.37651611)
\curveto(316.75640947,64.11651399)(316.86140936,63.85151425)(316.96141846,63.58151611)
\curveto(317.07140915,63.31151479)(317.18140904,63.04651506)(317.29141846,62.78651611)
\moveto(314.08141846,66.43151611)
\curveto(314.17141205,66.46151164)(314.226412,66.51151159)(314.24641846,66.58151611)
\curveto(314.27641195,66.65151145)(314.28141194,66.72651138)(314.26141846,66.80651611)
\curveto(314.25141197,66.89651121)(314.226412,66.98151112)(314.18641846,67.06151611)
\curveto(314.15641207,67.15151095)(314.1264121,67.22651088)(314.09641846,67.28651611)
\curveto(314.07641215,67.32651078)(314.06641216,67.36151074)(314.06641846,67.39151611)
\curveto(314.06641216,67.42151068)(314.05641217,67.45651065)(314.03641846,67.49651611)
\lineto(313.94641846,67.73651611)
\curveto(313.9264123,67.82651028)(313.89641233,67.91651019)(313.85641846,68.00651611)
\curveto(313.70641252,68.36650974)(313.57141265,68.73150937)(313.45141846,69.10151611)
\curveto(313.34141288,69.48150862)(313.21141301,69.85150825)(313.06141846,70.21151611)
\curveto(313.01141321,70.32150778)(312.96641326,70.43150767)(312.92641846,70.54151611)
\curveto(312.89641333,70.65150745)(312.85641337,70.75650735)(312.80641846,70.85651611)
\curveto(312.78641344,70.9065072)(312.76141346,70.95150715)(312.73141846,70.99151611)
\curveto(312.71141351,71.04150706)(312.66141356,71.06650704)(312.58141846,71.06651611)
\curveto(312.56141366,71.04650706)(312.54141368,71.03150707)(312.52141846,71.02151611)
\curveto(312.50141372,71.01150709)(312.48141374,70.99650711)(312.46141846,70.97651611)
\curveto(312.4214138,70.92650718)(312.39141383,70.87150723)(312.37141846,70.81151611)
\curveto(312.35141387,70.76150734)(312.33141389,70.7065074)(312.31141846,70.64651611)
\curveto(312.26141396,70.53650757)(312.221414,70.42650768)(312.19141846,70.31651611)
\curveto(312.16141406,70.2065079)(312.1214141,70.09650801)(312.07141846,69.98651611)
\curveto(311.90141432,69.59650851)(311.75141447,69.2015089)(311.62141846,68.80151611)
\curveto(311.50141472,68.4015097)(311.36141486,68.01151009)(311.20141846,67.63151611)
\lineto(311.14141846,67.48151611)
\curveto(311.13141509,67.43151067)(311.11641511,67.38151072)(311.09641846,67.33151611)
\lineto(311.00641846,67.09151611)
\curveto(310.97641525,67.01151109)(310.95141527,66.93151117)(310.93141846,66.85151611)
\curveto(310.91141531,66.8015113)(310.90141532,66.74651136)(310.90141846,66.68651611)
\curveto(310.91141531,66.62651148)(310.9264153,66.57651153)(310.94641846,66.53651611)
\curveto(310.99641523,66.45651165)(311.10141512,66.41151169)(311.26141846,66.40151611)
\lineto(311.71141846,66.40151611)
\lineto(313.31641846,66.40151611)
\curveto(313.4264128,66.4015117)(313.56141266,66.39651171)(313.72141846,66.38651611)
\curveto(313.88141234,66.38651172)(314.00141222,66.4015117)(314.08141846,66.43151611)
}
}
{
\newrgbcolor{curcolor}{0 0 0}
\pscustom[linestyle=none,fillstyle=solid,fillcolor=curcolor]
{
\newpath
\moveto(318.87298096,69.73151611)
\lineto(319.30798096,69.73151611)
\curveto(319.45797899,69.73150837)(319.56297889,69.69150841)(319.62298096,69.61151611)
\curveto(319.67297878,69.53150857)(319.69797875,69.43150867)(319.69798096,69.31151611)
\curveto(319.70797874,69.19150891)(319.71297874,69.07150903)(319.71298096,68.95151611)
\lineto(319.71298096,67.52651611)
\lineto(319.71298096,65.26151611)
\lineto(319.71298096,64.57151611)
\curveto(319.71297874,64.34151376)(319.73797871,64.14151396)(319.78798096,63.97151611)
\curveto(319.9479785,63.52151458)(320.2479782,63.2065149)(320.68798096,63.02651611)
\curveto(320.90797754,62.93651517)(321.17297728,62.9015152)(321.48298096,62.92151611)
\curveto(321.79297666,62.95151515)(322.04297641,63.0065151)(322.23298096,63.08651611)
\curveto(322.56297589,63.22651488)(322.82297563,63.4015147)(323.01298096,63.61151611)
\curveto(323.21297524,63.83151427)(323.36797508,64.11651399)(323.47798096,64.46651611)
\curveto(323.50797494,64.54651356)(323.52797492,64.62651348)(323.53798096,64.70651611)
\curveto(323.5479749,64.78651332)(323.56297489,64.87151323)(323.58298096,64.96151611)
\curveto(323.59297486,65.01151309)(323.59297486,65.05651305)(323.58298096,65.09651611)
\curveto(323.58297487,65.13651297)(323.59297486,65.18151292)(323.61298096,65.23151611)
\lineto(323.61298096,65.54651611)
\curveto(323.63297482,65.62651248)(323.63797481,65.71651239)(323.62798096,65.81651611)
\curveto(323.61797483,65.92651218)(323.61297484,66.02651208)(323.61298096,66.11651611)
\lineto(323.61298096,67.28651611)
\lineto(323.61298096,68.87651611)
\curveto(323.61297484,68.99650911)(323.60797484,69.12150898)(323.59798096,69.25151611)
\curveto(323.59797485,69.39150871)(323.62297483,69.5015086)(323.67298096,69.58151611)
\curveto(323.71297474,69.63150847)(323.75797469,69.66150844)(323.80798096,69.67151611)
\curveto(323.86797458,69.69150841)(323.93797451,69.71150839)(324.01798096,69.73151611)
\lineto(324.24298096,69.73151611)
\curveto(324.36297409,69.73150837)(324.46797398,69.72650838)(324.55798096,69.71651611)
\curveto(324.65797379,69.7065084)(324.73297372,69.66150844)(324.78298096,69.58151611)
\curveto(324.83297362,69.53150857)(324.85797359,69.45650865)(324.85798096,69.35651611)
\lineto(324.85798096,69.07151611)
\lineto(324.85798096,68.05151611)
\lineto(324.85798096,64.01651611)
\lineto(324.85798096,62.66651611)
\curveto(324.85797359,62.54651556)(324.8529736,62.43151567)(324.84298096,62.32151611)
\curveto(324.84297361,62.22151588)(324.80797364,62.14651596)(324.73798096,62.09651611)
\curveto(324.69797375,62.06651604)(324.63797381,62.04151606)(324.55798096,62.02151611)
\curveto(324.47797397,62.01151609)(324.38797406,62.0015161)(324.28798096,61.99151611)
\curveto(324.19797425,61.99151611)(324.10797434,61.99651611)(324.01798096,62.00651611)
\curveto(323.93797451,62.01651609)(323.87797457,62.03651607)(323.83798096,62.06651611)
\curveto(323.78797466,62.106516)(323.74297471,62.17151593)(323.70298096,62.26151611)
\curveto(323.69297476,62.3015158)(323.68297477,62.35651575)(323.67298096,62.42651611)
\curveto(323.67297478,62.49651561)(323.66797478,62.56151554)(323.65798096,62.62151611)
\curveto(323.6479748,62.69151541)(323.62797482,62.74651536)(323.59798096,62.78651611)
\curveto(323.56797488,62.82651528)(323.52297493,62.84151526)(323.46298096,62.83151611)
\curveto(323.38297507,62.81151529)(323.30297515,62.75151535)(323.22298096,62.65151611)
\curveto(323.14297531,62.56151554)(323.06797538,62.49151561)(322.99798096,62.44151611)
\curveto(322.77797567,62.28151582)(322.52797592,62.14151596)(322.24798096,62.02151611)
\curveto(322.13797631,61.97151613)(322.02297643,61.94151616)(321.90298096,61.93151611)
\curveto(321.79297666,61.91151619)(321.67797677,61.88651622)(321.55798096,61.85651611)
\curveto(321.50797694,61.84651626)(321.452977,61.84651626)(321.39298096,61.85651611)
\curveto(321.34297711,61.86651624)(321.29297716,61.86151624)(321.24298096,61.84151611)
\curveto(321.14297731,61.82151628)(321.0529774,61.82151628)(320.97298096,61.84151611)
\lineto(320.82298096,61.84151611)
\curveto(320.77297768,61.86151624)(320.71297774,61.87151623)(320.64298096,61.87151611)
\curveto(320.58297787,61.87151623)(320.52797792,61.87651623)(320.47798096,61.88651611)
\curveto(320.43797801,61.9065162)(320.39797805,61.91651619)(320.35798096,61.91651611)
\curveto(320.32797812,61.9065162)(320.28797816,61.91151619)(320.23798096,61.93151611)
\lineto(319.99798096,61.99151611)
\curveto(319.92797852,62.01151609)(319.8529786,62.04151606)(319.77298096,62.08151611)
\curveto(319.51297894,62.19151591)(319.29297916,62.33651577)(319.11298096,62.51651611)
\curveto(318.94297951,62.7065154)(318.80297965,62.93151517)(318.69298096,63.19151611)
\curveto(318.6529798,63.28151482)(318.62297983,63.37151473)(318.60298096,63.46151611)
\lineto(318.54298096,63.76151611)
\curveto(318.52297993,63.82151428)(318.51297994,63.87651423)(318.51298096,63.92651611)
\curveto(318.52297993,63.98651412)(318.51797993,64.05151405)(318.49798096,64.12151611)
\curveto(318.48797996,64.14151396)(318.48297997,64.16651394)(318.48298096,64.19651611)
\curveto(318.48297997,64.23651387)(318.47797997,64.27151383)(318.46798096,64.30151611)
\lineto(318.46798096,64.45151611)
\curveto(318.45797999,64.49151361)(318.45298,64.53651357)(318.45298096,64.58651611)
\curveto(318.46297999,64.64651346)(318.46797998,64.7015134)(318.46798096,64.75151611)
\lineto(318.46798096,65.35151611)
\lineto(318.46798096,68.11151611)
\lineto(318.46798096,69.07151611)
\lineto(318.46798096,69.34151611)
\curveto(318.46797998,69.43150867)(318.48797996,69.5065086)(318.52798096,69.56651611)
\curveto(318.56797988,69.63650847)(318.64297981,69.68650842)(318.75298096,69.71651611)
\curveto(318.77297968,69.72650838)(318.79297966,69.72650838)(318.81298096,69.71651611)
\curveto(318.83297962,69.71650839)(318.8529796,69.72150838)(318.87298096,69.73151611)
}
}
{
\newrgbcolor{curcolor}{0 0 0}
\pscustom[linestyle=none,fillstyle=solid,fillcolor=curcolor]
{
\newpath
\moveto(326.77259033,69.73151611)
\lineto(327.29759033,69.73151611)
\curveto(327.49758868,69.74150836)(327.64758853,69.72150838)(327.74759033,69.67151611)
\curveto(327.86758831,69.62150848)(327.96258821,69.54150856)(328.03259033,69.43151611)
\curveto(328.11258806,69.32150878)(328.18758799,69.21150889)(328.25759033,69.10151611)
\curveto(328.38758779,68.9015092)(328.51758766,68.7065094)(328.64759033,68.51651611)
\curveto(328.7775874,68.33650977)(328.91258726,68.14650996)(329.05259033,67.94651611)
\curveto(329.10258707,67.86651024)(329.15258702,67.79151031)(329.20259033,67.72151611)
\curveto(329.26258691,67.65151045)(329.31758686,67.58151052)(329.36759033,67.51151611)
\curveto(329.40758677,67.45151065)(329.44758673,67.39651071)(329.48759033,67.34651611)
\curveto(329.52758665,67.29651081)(329.58758659,67.26151084)(329.66759033,67.24151611)
\curveto(329.71758646,67.22151088)(329.75758642,67.22151088)(329.78759033,67.24151611)
\curveto(329.82758635,67.27151083)(329.85758632,67.29651081)(329.87759033,67.31651611)
\curveto(329.95758622,67.36651074)(330.02258615,67.43651067)(330.07259033,67.52651611)
\curveto(330.13258604,67.61651049)(330.18758599,67.7015104)(330.23759033,67.78151611)
\curveto(330.38758579,67.98151012)(330.53758564,68.18650992)(330.68759033,68.39651611)
\lineto(331.13759033,69.02651611)
\curveto(331.21758496,69.13650897)(331.29758488,69.25150885)(331.37759033,69.37151611)
\curveto(331.45758472,69.49150861)(331.55258462,69.58650852)(331.66259033,69.65651611)
\curveto(331.74258443,69.7065084)(331.83758434,69.73150837)(331.94759033,69.73151611)
\lineto(332.29259033,69.73151611)
\lineto(332.42759033,69.73151611)
\curveto(332.4775837,69.73150837)(332.52758365,69.72650838)(332.57759033,69.71651611)
\lineto(332.65259033,69.71651611)
\curveto(332.7725834,69.69650841)(332.85258332,69.65650845)(332.89259033,69.59651611)
\curveto(332.91258326,69.54650856)(332.90758327,69.49150861)(332.87759033,69.43151611)
\curveto(332.85758332,69.38150872)(332.83758334,69.34150876)(332.81759033,69.31151611)
\lineto(332.60759033,69.01151611)
\curveto(332.53758364,68.92150918)(332.46258371,68.82650928)(332.38259033,68.72651611)
\curveto(332.15258402,68.4065097)(331.91758426,68.09151001)(331.67759033,67.78151611)
\curveto(331.44758473,67.48151062)(331.21758496,67.17151093)(330.98759033,66.85151611)
\curveto(330.93758524,66.77151133)(330.88258529,66.69151141)(330.82259033,66.61151611)
\curveto(330.76258541,66.54151156)(330.70758547,66.46151164)(330.65759033,66.37151611)
\curveto(330.63758554,66.34151176)(330.61758556,66.3015118)(330.59759033,66.25151611)
\curveto(330.5775856,66.21151189)(330.5775856,66.16151194)(330.59759033,66.10151611)
\curveto(330.61758556,66.01151209)(330.64758553,65.93651217)(330.68759033,65.87651611)
\curveto(330.73758544,65.81651229)(330.78758539,65.75151235)(330.83759033,65.68151611)
\lineto(331.01759033,65.41151611)
\curveto(331.08758509,65.32151278)(331.15258502,65.23151287)(331.21259033,65.14151611)
\lineto(331.90259033,64.18151611)
\lineto(332.59259033,63.22151611)
\curveto(332.6725835,63.11151499)(332.75258342,62.99651511)(332.83259033,62.87651611)
\lineto(333.07259033,62.54651611)
\curveto(333.12258305,62.47651563)(333.16258301,62.41151569)(333.19259033,62.35151611)
\curveto(333.23258294,62.3015158)(333.24258293,62.22151588)(333.22259033,62.11151611)
\curveto(333.20258297,62.101516)(333.18258299,62.08651602)(333.16259033,62.06651611)
\curveto(333.15258302,62.05651605)(333.13758304,62.04651606)(333.11759033,62.03651611)
\curveto(333.06758311,62.01651609)(333.00258317,62.0065161)(332.92259033,62.00651611)
\lineto(332.68259033,62.00651611)
\lineto(332.17259033,62.00651611)
\curveto(332.03258414,62.01651609)(331.90758427,62.06151604)(331.79759033,62.14151611)
\curveto(331.74758443,62.17151593)(331.70758447,62.2065159)(331.67759033,62.24651611)
\curveto(331.65758452,62.29651581)(331.63258454,62.34651576)(331.60259033,62.39651611)
\lineto(331.45259033,62.60651611)
\curveto(331.40258477,62.67651543)(331.35258482,62.75151535)(331.30259033,62.83151611)
\lineto(330.35759033,64.22651611)
\curveto(330.30758587,64.3065138)(330.25758592,64.38151372)(330.20759033,64.45151611)
\curveto(330.15758602,64.52151358)(330.10758607,64.59651351)(330.05759033,64.67651611)
\curveto(330.00758617,64.74651336)(329.95758622,64.8065133)(329.90759033,64.85651611)
\curveto(329.86758631,64.91651319)(329.80758637,64.95651315)(329.72759033,64.97651611)
\curveto(329.6775865,64.99651311)(329.62758655,64.98651312)(329.57759033,64.94651611)
\curveto(329.53758664,64.91651319)(329.50758667,64.89151321)(329.48759033,64.87151611)
\curveto(329.40758677,64.79151331)(329.33758684,64.7015134)(329.27759033,64.60151611)
\curveto(329.21758696,64.5015136)(329.15758702,64.4065137)(329.09759033,64.31651611)
\curveto(328.92758725,64.05651405)(328.75258742,63.79651431)(328.57259033,63.53651611)
\curveto(328.40258777,63.28651482)(328.22758795,63.03651507)(328.04759033,62.78651611)
\curveto(327.99758818,62.7065154)(327.94258823,62.62651548)(327.88259033,62.54651611)
\lineto(327.73259033,62.30651611)
\curveto(327.71258846,62.27651583)(327.68758849,62.24151586)(327.65759033,62.20151611)
\curveto(327.63758854,62.17151593)(327.61258856,62.14651596)(327.58259033,62.12651611)
\curveto(327.48258869,62.05651605)(327.36258881,62.01651609)(327.22259033,62.00651611)
\lineto(326.77259033,62.00651611)
\lineto(326.54759033,62.00651611)
\curveto(326.4775897,62.0065161)(326.41758976,62.01651609)(326.36759033,62.03651611)
\curveto(326.33758984,62.05651605)(326.31258986,62.07151603)(326.29259033,62.08151611)
\curveto(326.28258989,62.101516)(326.26758991,62.12151598)(326.24759033,62.14151611)
\curveto(326.23758994,62.25151585)(326.25258992,62.33651577)(326.29259033,62.39651611)
\curveto(326.34258983,62.45651565)(326.39258978,62.52151558)(326.44259033,62.59151611)
\curveto(326.52258965,62.7015154)(326.59758958,62.8015153)(326.66759033,62.89151611)
\curveto(326.73758944,62.99151511)(326.80758937,63.09651501)(326.87759033,63.20651611)
\curveto(327.09758908,63.5065146)(327.31258886,63.8065143)(327.52259033,64.10651611)
\lineto(328.15259033,65.00651611)
\curveto(328.22258795,65.09651301)(328.28758789,65.18651292)(328.34759033,65.27651611)
\curveto(328.41758776,65.36651274)(328.48258769,65.46151264)(328.54259033,65.56151611)
\curveto(328.59258758,65.63151247)(328.64258753,65.69651241)(328.69259033,65.75651611)
\curveto(328.74258743,65.82651228)(328.7775874,65.91651219)(328.79759033,66.02651611)
\curveto(328.81758736,66.07651203)(328.81258736,66.12651198)(328.78259033,66.17651611)
\curveto(328.76258741,66.22651188)(328.74258743,66.26651184)(328.72259033,66.29651611)
\curveto(328.6725875,66.38651172)(328.61758756,66.47151163)(328.55759033,66.55151611)
\lineto(328.37759033,66.79151611)
\curveto(328.14758803,67.11151099)(327.91258826,67.43151067)(327.67259033,67.75151611)
\lineto(326.98259033,68.71151611)
\curveto(326.90258927,68.82150928)(326.82258935,68.92150918)(326.74259033,69.01151611)
\curveto(326.6725895,69.101509)(326.60258957,69.2015089)(326.53259033,69.31151611)
\curveto(326.51258966,69.34150876)(326.49258968,69.38150872)(326.47259033,69.43151611)
\curveto(326.45258972,69.49150861)(326.45258972,69.54150856)(326.47259033,69.58151611)
\curveto(326.49258968,69.63150847)(326.52258965,69.66150844)(326.56259033,69.67151611)
\curveto(326.60258957,69.69150841)(326.64758953,69.7065084)(326.69759033,69.71651611)
\curveto(326.71758946,69.72650838)(326.73258944,69.72650838)(326.74259033,69.71651611)
\curveto(326.75258942,69.71650839)(326.76258941,69.72150838)(326.77259033,69.73151611)
}
}
{
\newrgbcolor{curcolor}{0 0 0}
\pscustom[linestyle=none,fillstyle=solid,fillcolor=curcolor]
{
\newpath
\moveto(334.80626221,71.23151611)
\curveto(334.72626109,71.29150681)(334.68126113,71.39650671)(334.67126221,71.54651611)
\lineto(334.67126221,72.01151611)
\lineto(334.67126221,72.26651611)
\curveto(334.67126114,72.35650575)(334.68626113,72.43150567)(334.71626221,72.49151611)
\curveto(334.75626106,72.57150553)(334.83626098,72.63150547)(334.95626221,72.67151611)
\curveto(334.97626084,72.68150542)(334.99626082,72.68150542)(335.01626221,72.67151611)
\curveto(335.04626077,72.67150543)(335.07126074,72.67650543)(335.09126221,72.68651611)
\curveto(335.26126055,72.68650542)(335.42126039,72.68150542)(335.57126221,72.67151611)
\curveto(335.72126009,72.66150544)(335.82125999,72.6015055)(335.87126221,72.49151611)
\curveto(335.90125991,72.43150567)(335.9162599,72.35650575)(335.91626221,72.26651611)
\lineto(335.91626221,72.01151611)
\curveto(335.9162599,71.83150627)(335.9112599,71.66150644)(335.90126221,71.50151611)
\curveto(335.90125991,71.34150676)(335.83625998,71.23650687)(335.70626221,71.18651611)
\curveto(335.65626016,71.16650694)(335.60126021,71.15650695)(335.54126221,71.15651611)
\lineto(335.37626221,71.15651611)
\lineto(335.06126221,71.15651611)
\curveto(334.96126085,71.15650695)(334.87626094,71.18150692)(334.80626221,71.23151611)
\moveto(335.91626221,62.72651611)
\lineto(335.91626221,62.41151611)
\curveto(335.92625989,62.31151579)(335.90625991,62.23151587)(335.85626221,62.17151611)
\curveto(335.82625999,62.11151599)(335.78126003,62.07151603)(335.72126221,62.05151611)
\curveto(335.66126015,62.04151606)(335.59126022,62.02651608)(335.51126221,62.00651611)
\lineto(335.28626221,62.00651611)
\curveto(335.15626066,62.0065161)(335.04126077,62.01151609)(334.94126221,62.02151611)
\curveto(334.85126096,62.04151606)(334.78126103,62.09151601)(334.73126221,62.17151611)
\curveto(334.69126112,62.23151587)(334.67126114,62.3065158)(334.67126221,62.39651611)
\lineto(334.67126221,62.68151611)
\lineto(334.67126221,69.02651611)
\lineto(334.67126221,69.34151611)
\curveto(334.67126114,69.45150865)(334.69626112,69.53650857)(334.74626221,69.59651611)
\curveto(334.77626104,69.64650846)(334.816261,69.67650843)(334.86626221,69.68651611)
\curveto(334.9162609,69.69650841)(334.97126084,69.71150839)(335.03126221,69.73151611)
\curveto(335.05126076,69.73150837)(335.07126074,69.72650838)(335.09126221,69.71651611)
\curveto(335.12126069,69.71650839)(335.14626067,69.72150838)(335.16626221,69.73151611)
\curveto(335.29626052,69.73150837)(335.42626039,69.72650838)(335.55626221,69.71651611)
\curveto(335.69626012,69.71650839)(335.79126002,69.67650843)(335.84126221,69.59651611)
\curveto(335.89125992,69.53650857)(335.9162599,69.45650865)(335.91626221,69.35651611)
\lineto(335.91626221,69.07151611)
\lineto(335.91626221,62.72651611)
}
}
{
\newrgbcolor{curcolor}{0 0 0}
\pscustom[linestyle=none,fillstyle=solid,fillcolor=curcolor]
{
\newpath
\moveto(338.43110596,72.68651611)
\curveto(338.56110434,72.68650542)(338.69610421,72.68650542)(338.83610596,72.68651611)
\curveto(338.98610392,72.68650542)(339.09610381,72.65150545)(339.16610596,72.58151611)
\curveto(339.21610369,72.51150559)(339.24110366,72.41650569)(339.24110596,72.29651611)
\curveto(339.25110365,72.18650592)(339.25610365,72.07150603)(339.25610596,71.95151611)
\lineto(339.25610596,70.61651611)
\lineto(339.25610596,64.54151611)
\lineto(339.25610596,62.86151611)
\lineto(339.25610596,62.47151611)
\curveto(339.25610365,62.33151577)(339.23110367,62.22151588)(339.18110596,62.14151611)
\curveto(339.15110375,62.09151601)(339.1061038,62.06151604)(339.04610596,62.05151611)
\curveto(338.99610391,62.04151606)(338.93110397,62.02651608)(338.85110596,62.00651611)
\lineto(338.64110596,62.00651611)
\lineto(338.32610596,62.00651611)
\curveto(338.22610468,62.01651609)(338.15110475,62.05151605)(338.10110596,62.11151611)
\curveto(338.05110485,62.19151591)(338.02110488,62.29151581)(338.01110596,62.41151611)
\lineto(338.01110596,62.78651611)
\lineto(338.01110596,64.16651611)
\lineto(338.01110596,70.40651611)
\lineto(338.01110596,71.87651611)
\curveto(338.01110489,71.98650612)(338.0061049,72.101506)(337.99610596,72.22151611)
\curveto(337.99610491,72.35150575)(338.02110488,72.45150565)(338.07110596,72.52151611)
\curveto(338.11110479,72.58150552)(338.18610472,72.63150547)(338.29610596,72.67151611)
\curveto(338.31610459,72.68150542)(338.33610457,72.68150542)(338.35610596,72.67151611)
\curveto(338.38610452,72.67150543)(338.41110449,72.67650543)(338.43110596,72.68651611)
}
}
{
\newrgbcolor{curcolor}{0 0 0}
\pscustom[linestyle=none,fillstyle=solid,fillcolor=curcolor]
{
\newpath
\moveto(341.48594971,71.23151611)
\curveto(341.40594859,71.29150681)(341.36094863,71.39650671)(341.35094971,71.54651611)
\lineto(341.35094971,72.01151611)
\lineto(341.35094971,72.26651611)
\curveto(341.35094864,72.35650575)(341.36594863,72.43150567)(341.39594971,72.49151611)
\curveto(341.43594856,72.57150553)(341.51594848,72.63150547)(341.63594971,72.67151611)
\curveto(341.65594834,72.68150542)(341.67594832,72.68150542)(341.69594971,72.67151611)
\curveto(341.72594827,72.67150543)(341.75094824,72.67650543)(341.77094971,72.68651611)
\curveto(341.94094805,72.68650542)(342.10094789,72.68150542)(342.25094971,72.67151611)
\curveto(342.40094759,72.66150544)(342.50094749,72.6015055)(342.55094971,72.49151611)
\curveto(342.58094741,72.43150567)(342.5959474,72.35650575)(342.59594971,72.26651611)
\lineto(342.59594971,72.01151611)
\curveto(342.5959474,71.83150627)(342.5909474,71.66150644)(342.58094971,71.50151611)
\curveto(342.58094741,71.34150676)(342.51594748,71.23650687)(342.38594971,71.18651611)
\curveto(342.33594766,71.16650694)(342.28094771,71.15650695)(342.22094971,71.15651611)
\lineto(342.05594971,71.15651611)
\lineto(341.74094971,71.15651611)
\curveto(341.64094835,71.15650695)(341.55594844,71.18150692)(341.48594971,71.23151611)
\moveto(342.59594971,62.72651611)
\lineto(342.59594971,62.41151611)
\curveto(342.60594739,62.31151579)(342.58594741,62.23151587)(342.53594971,62.17151611)
\curveto(342.50594749,62.11151599)(342.46094753,62.07151603)(342.40094971,62.05151611)
\curveto(342.34094765,62.04151606)(342.27094772,62.02651608)(342.19094971,62.00651611)
\lineto(341.96594971,62.00651611)
\curveto(341.83594816,62.0065161)(341.72094827,62.01151609)(341.62094971,62.02151611)
\curveto(341.53094846,62.04151606)(341.46094853,62.09151601)(341.41094971,62.17151611)
\curveto(341.37094862,62.23151587)(341.35094864,62.3065158)(341.35094971,62.39651611)
\lineto(341.35094971,62.68151611)
\lineto(341.35094971,69.02651611)
\lineto(341.35094971,69.34151611)
\curveto(341.35094864,69.45150865)(341.37594862,69.53650857)(341.42594971,69.59651611)
\curveto(341.45594854,69.64650846)(341.4959485,69.67650843)(341.54594971,69.68651611)
\curveto(341.5959484,69.69650841)(341.65094834,69.71150839)(341.71094971,69.73151611)
\curveto(341.73094826,69.73150837)(341.75094824,69.72650838)(341.77094971,69.71651611)
\curveto(341.80094819,69.71650839)(341.82594817,69.72150838)(341.84594971,69.73151611)
\curveto(341.97594802,69.73150837)(342.10594789,69.72650838)(342.23594971,69.71651611)
\curveto(342.37594762,69.71650839)(342.47094752,69.67650843)(342.52094971,69.59651611)
\curveto(342.57094742,69.53650857)(342.5959474,69.45650865)(342.59594971,69.35651611)
\lineto(342.59594971,69.07151611)
\lineto(342.59594971,62.72651611)
}
}
{
\newrgbcolor{curcolor}{0 0 0}
\pscustom[linestyle=none,fillstyle=solid,fillcolor=curcolor]
{
\newpath
\moveto(351.42579346,62.56151611)
\curveto(351.45578563,62.4015157)(351.44078564,62.26651584)(351.38079346,62.15651611)
\curveto(351.32078576,62.05651605)(351.24078584,61.98151612)(351.14079346,61.93151611)
\curveto(351.09078599,61.91151619)(351.03578605,61.9015162)(350.97579346,61.90151611)
\curveto(350.92578616,61.9015162)(350.87078621,61.89151621)(350.81079346,61.87151611)
\curveto(350.59078649,61.82151628)(350.37078671,61.83651627)(350.15079346,61.91651611)
\curveto(349.94078714,61.98651612)(349.79578729,62.07651603)(349.71579346,62.18651611)
\curveto(349.66578742,62.25651585)(349.62078746,62.33651577)(349.58079346,62.42651611)
\curveto(349.54078754,62.52651558)(349.49078759,62.6065155)(349.43079346,62.66651611)
\curveto(349.41078767,62.68651542)(349.3857877,62.7065154)(349.35579346,62.72651611)
\curveto(349.33578775,62.74651536)(349.30578778,62.75151535)(349.26579346,62.74151611)
\curveto(349.15578793,62.71151539)(349.05078803,62.65651545)(348.95079346,62.57651611)
\curveto(348.86078822,62.49651561)(348.77078831,62.42651568)(348.68079346,62.36651611)
\curveto(348.55078853,62.28651582)(348.41078867,62.21151589)(348.26079346,62.14151611)
\curveto(348.11078897,62.08151602)(347.95078913,62.02651608)(347.78079346,61.97651611)
\curveto(347.6807894,61.94651616)(347.57078951,61.92651618)(347.45079346,61.91651611)
\curveto(347.34078974,61.9065162)(347.23078985,61.89151621)(347.12079346,61.87151611)
\curveto(347.07079001,61.86151624)(347.02579006,61.85651625)(346.98579346,61.85651611)
\lineto(346.88079346,61.85651611)
\curveto(346.77079031,61.83651627)(346.66579042,61.83651627)(346.56579346,61.85651611)
\lineto(346.43079346,61.85651611)
\curveto(346.3807907,61.86651624)(346.33079075,61.87151623)(346.28079346,61.87151611)
\curveto(346.23079085,61.87151623)(346.1857909,61.88151622)(346.14579346,61.90151611)
\curveto(346.10579098,61.91151619)(346.07079101,61.91651619)(346.04079346,61.91651611)
\curveto(346.02079106,61.9065162)(345.99579109,61.9065162)(345.96579346,61.91651611)
\lineto(345.72579346,61.97651611)
\curveto(345.64579144,61.98651612)(345.57079151,62.0065161)(345.50079346,62.03651611)
\curveto(345.20079188,62.16651594)(344.95579213,62.31151579)(344.76579346,62.47151611)
\curveto(344.5857925,62.64151546)(344.43579265,62.87651523)(344.31579346,63.17651611)
\curveto(344.22579286,63.39651471)(344.1807929,63.66151444)(344.18079346,63.97151611)
\lineto(344.18079346,64.28651611)
\curveto(344.19079289,64.33651377)(344.19579289,64.38651372)(344.19579346,64.43651611)
\lineto(344.22579346,64.61651611)
\lineto(344.34579346,64.94651611)
\curveto(344.3857927,65.05651305)(344.43579265,65.15651295)(344.49579346,65.24651611)
\curveto(344.67579241,65.53651257)(344.92079216,65.75151235)(345.23079346,65.89151611)
\curveto(345.54079154,66.03151207)(345.8807912,66.15651195)(346.25079346,66.26651611)
\curveto(346.39079069,66.3065118)(346.53579055,66.33651177)(346.68579346,66.35651611)
\curveto(346.83579025,66.37651173)(346.9857901,66.4015117)(347.13579346,66.43151611)
\curveto(347.20578988,66.45151165)(347.27078981,66.46151164)(347.33079346,66.46151611)
\curveto(347.40078968,66.46151164)(347.47578961,66.47151163)(347.55579346,66.49151611)
\curveto(347.62578946,66.51151159)(347.69578939,66.52151158)(347.76579346,66.52151611)
\curveto(347.83578925,66.53151157)(347.91078917,66.54651156)(347.99079346,66.56651611)
\curveto(348.24078884,66.62651148)(348.47578861,66.67651143)(348.69579346,66.71651611)
\curveto(348.91578817,66.76651134)(349.09078799,66.88151122)(349.22079346,67.06151611)
\curveto(349.2807878,67.14151096)(349.33078775,67.24151086)(349.37079346,67.36151611)
\curveto(349.41078767,67.49151061)(349.41078767,67.63151047)(349.37079346,67.78151611)
\curveto(349.31078777,68.02151008)(349.22078786,68.21150989)(349.10079346,68.35151611)
\curveto(348.99078809,68.49150961)(348.83078825,68.6015095)(348.62079346,68.68151611)
\curveto(348.50078858,68.73150937)(348.35578873,68.76650934)(348.18579346,68.78651611)
\curveto(348.02578906,68.8065093)(347.85578923,68.81650929)(347.67579346,68.81651611)
\curveto(347.49578959,68.81650929)(347.32078976,68.8065093)(347.15079346,68.78651611)
\curveto(346.9807901,68.76650934)(346.83579025,68.73650937)(346.71579346,68.69651611)
\curveto(346.54579054,68.63650947)(346.3807907,68.55150955)(346.22079346,68.44151611)
\curveto(346.14079094,68.38150972)(346.06579102,68.3015098)(345.99579346,68.20151611)
\curveto(345.93579115,68.11150999)(345.8807912,68.01151009)(345.83079346,67.90151611)
\curveto(345.80079128,67.82151028)(345.77079131,67.73651037)(345.74079346,67.64651611)
\curveto(345.72079136,67.55651055)(345.67579141,67.48651062)(345.60579346,67.43651611)
\curveto(345.56579152,67.4065107)(345.49579159,67.38151072)(345.39579346,67.36151611)
\curveto(345.30579178,67.35151075)(345.21079187,67.34651076)(345.11079346,67.34651611)
\curveto(345.01079207,67.34651076)(344.91079217,67.35151075)(344.81079346,67.36151611)
\curveto(344.72079236,67.38151072)(344.65579243,67.4065107)(344.61579346,67.43651611)
\curveto(344.57579251,67.46651064)(344.54579254,67.51651059)(344.52579346,67.58651611)
\curveto(344.50579258,67.65651045)(344.50579258,67.73151037)(344.52579346,67.81151611)
\curveto(344.55579253,67.94151016)(344.5857925,68.06151004)(344.61579346,68.17151611)
\curveto(344.65579243,68.29150981)(344.70079238,68.4065097)(344.75079346,68.51651611)
\curveto(344.94079214,68.86650924)(345.1807919,69.13650897)(345.47079346,69.32651611)
\curveto(345.76079132,69.52650858)(346.12079096,69.68650842)(346.55079346,69.80651611)
\curveto(346.65079043,69.82650828)(346.75079033,69.84150826)(346.85079346,69.85151611)
\curveto(346.96079012,69.86150824)(347.07079001,69.87650823)(347.18079346,69.89651611)
\curveto(347.22078986,69.9065082)(347.2857898,69.9065082)(347.37579346,69.89651611)
\curveto(347.46578962,69.89650821)(347.52078956,69.9065082)(347.54079346,69.92651611)
\curveto(348.24078884,69.93650817)(348.85078823,69.85650825)(349.37079346,69.68651611)
\curveto(349.89078719,69.51650859)(350.25578683,69.19150891)(350.46579346,68.71151611)
\curveto(350.55578653,68.51150959)(350.60578648,68.27650983)(350.61579346,68.00651611)
\curveto(350.63578645,67.74651036)(350.64578644,67.47151063)(350.64579346,67.18151611)
\lineto(350.64579346,63.86651611)
\curveto(350.64578644,63.72651438)(350.65078643,63.59151451)(350.66079346,63.46151611)
\curveto(350.67078641,63.33151477)(350.70078638,63.22651488)(350.75079346,63.14651611)
\curveto(350.80078628,63.07651503)(350.86578622,63.02651508)(350.94579346,62.99651611)
\curveto(351.03578605,62.95651515)(351.12078596,62.92651518)(351.20079346,62.90651611)
\curveto(351.2807858,62.89651521)(351.34078574,62.85151525)(351.38079346,62.77151611)
\curveto(351.40078568,62.74151536)(351.41078567,62.71151539)(351.41079346,62.68151611)
\curveto(351.41078567,62.65151545)(351.41578567,62.61151549)(351.42579346,62.56151611)
\moveto(349.28079346,64.22651611)
\curveto(349.34078774,64.36651374)(349.37078771,64.52651358)(349.37079346,64.70651611)
\curveto(349.3807877,64.89651321)(349.3857877,65.09151301)(349.38579346,65.29151611)
\curveto(349.3857877,65.4015127)(349.3807877,65.5015126)(349.37079346,65.59151611)
\curveto(349.36078772,65.68151242)(349.32078776,65.75151235)(349.25079346,65.80151611)
\curveto(349.22078786,65.82151228)(349.15078793,65.83151227)(349.04079346,65.83151611)
\curveto(349.02078806,65.81151229)(348.9857881,65.8015123)(348.93579346,65.80151611)
\curveto(348.8857882,65.8015123)(348.84078824,65.79151231)(348.80079346,65.77151611)
\curveto(348.72078836,65.75151235)(348.63078845,65.73151237)(348.53079346,65.71151611)
\lineto(348.23079346,65.65151611)
\curveto(348.20078888,65.65151245)(348.16578892,65.64651246)(348.12579346,65.63651611)
\lineto(348.02079346,65.63651611)
\curveto(347.87078921,65.59651251)(347.70578938,65.57151253)(347.52579346,65.56151611)
\curveto(347.35578973,65.56151254)(347.19578989,65.54151256)(347.04579346,65.50151611)
\curveto(346.96579012,65.48151262)(346.89079019,65.46151264)(346.82079346,65.44151611)
\curveto(346.76079032,65.43151267)(346.69079039,65.41651269)(346.61079346,65.39651611)
\curveto(346.45079063,65.34651276)(346.30079078,65.28151282)(346.16079346,65.20151611)
\curveto(346.02079106,65.13151297)(345.90079118,65.04151306)(345.80079346,64.93151611)
\curveto(345.70079138,64.82151328)(345.62579146,64.68651342)(345.57579346,64.52651611)
\curveto(345.52579156,64.37651373)(345.50579158,64.19151391)(345.51579346,63.97151611)
\curveto(345.51579157,63.87151423)(345.53079155,63.77651433)(345.56079346,63.68651611)
\curveto(345.60079148,63.6065145)(345.64579144,63.53151457)(345.69579346,63.46151611)
\curveto(345.77579131,63.35151475)(345.8807912,63.25651485)(346.01079346,63.17651611)
\curveto(346.14079094,63.106515)(346.2807908,63.04651506)(346.43079346,62.99651611)
\curveto(346.4807906,62.98651512)(346.53079055,62.98151512)(346.58079346,62.98151611)
\curveto(346.63079045,62.98151512)(346.6807904,62.97651513)(346.73079346,62.96651611)
\curveto(346.80079028,62.94651516)(346.8857902,62.93151517)(346.98579346,62.92151611)
\curveto(347.09578999,62.92151518)(347.1857899,62.93151517)(347.25579346,62.95151611)
\curveto(347.31578977,62.97151513)(347.37578971,62.97651513)(347.43579346,62.96651611)
\curveto(347.49578959,62.96651514)(347.55578953,62.97651513)(347.61579346,62.99651611)
\curveto(347.69578939,63.01651509)(347.77078931,63.03151507)(347.84079346,63.04151611)
\curveto(347.92078916,63.05151505)(347.99578909,63.07151503)(348.06579346,63.10151611)
\curveto(348.35578873,63.22151488)(348.60078848,63.36651474)(348.80079346,63.53651611)
\curveto(349.01078807,63.7065144)(349.17078791,63.93651417)(349.28079346,64.22651611)
}
}
{
\newrgbcolor{curcolor}{0 0 0}
\pscustom[linestyle=none,fillstyle=solid,fillcolor=curcolor]
{
\newpath
\moveto(356.24243408,69.91151611)
\curveto(356.47242929,69.91150819)(356.60242916,69.85150825)(356.63243408,69.73151611)
\curveto(356.6624291,69.62150848)(356.67742909,69.45650865)(356.67743408,69.23651611)
\lineto(356.67743408,68.95151611)
\curveto(356.67742909,68.86150924)(356.65242911,68.78650932)(356.60243408,68.72651611)
\curveto(356.54242922,68.64650946)(356.45742931,68.6015095)(356.34743408,68.59151611)
\curveto(356.23742953,68.59150951)(356.12742964,68.57650953)(356.01743408,68.54651611)
\curveto(355.87742989,68.51650959)(355.74243002,68.48650962)(355.61243408,68.45651611)
\curveto(355.49243027,68.42650968)(355.37743039,68.38650972)(355.26743408,68.33651611)
\curveto(354.97743079,68.2065099)(354.74243102,68.02651008)(354.56243408,67.79651611)
\curveto(354.38243138,67.57651053)(354.22743154,67.32151078)(354.09743408,67.03151611)
\curveto(354.05743171,66.92151118)(354.02743174,66.8065113)(354.00743408,66.68651611)
\curveto(353.98743178,66.57651153)(353.9624318,66.46151164)(353.93243408,66.34151611)
\curveto(353.92243184,66.29151181)(353.91743185,66.24151186)(353.91743408,66.19151611)
\curveto(353.92743184,66.14151196)(353.92743184,66.09151201)(353.91743408,66.04151611)
\curveto(353.88743188,65.92151218)(353.87243189,65.78151232)(353.87243408,65.62151611)
\curveto(353.88243188,65.47151263)(353.88743188,65.32651278)(353.88743408,65.18651611)
\lineto(353.88743408,63.34151611)
\lineto(353.88743408,62.99651611)
\curveto(353.88743188,62.87651523)(353.88243188,62.76151534)(353.87243408,62.65151611)
\curveto(353.8624319,62.54151556)(353.85743191,62.44651566)(353.85743408,62.36651611)
\curveto(353.8674319,62.28651582)(353.84743192,62.21651589)(353.79743408,62.15651611)
\curveto(353.74743202,62.08651602)(353.6674321,62.04651606)(353.55743408,62.03651611)
\curveto(353.45743231,62.02651608)(353.34743242,62.02151608)(353.22743408,62.02151611)
\lineto(352.95743408,62.02151611)
\curveto(352.90743286,62.04151606)(352.85743291,62.05651605)(352.80743408,62.06651611)
\curveto(352.767433,62.08651602)(352.73743303,62.11151599)(352.71743408,62.14151611)
\curveto(352.6674331,62.21151589)(352.63743313,62.29651581)(352.62743408,62.39651611)
\lineto(352.62743408,62.72651611)
\lineto(352.62743408,63.88151611)
\lineto(352.62743408,68.03651611)
\lineto(352.62743408,69.07151611)
\lineto(352.62743408,69.37151611)
\curveto(352.63743313,69.47150863)(352.6674331,69.55650855)(352.71743408,69.62651611)
\curveto(352.74743302,69.66650844)(352.79743297,69.69650841)(352.86743408,69.71651611)
\curveto(352.94743282,69.73650837)(353.03243273,69.74650836)(353.12243408,69.74651611)
\curveto(353.21243255,69.75650835)(353.30243246,69.75650835)(353.39243408,69.74651611)
\curveto(353.48243228,69.73650837)(353.55243221,69.72150838)(353.60243408,69.70151611)
\curveto(353.68243208,69.67150843)(353.73243203,69.61150849)(353.75243408,69.52151611)
\curveto(353.78243198,69.44150866)(353.79743197,69.35150875)(353.79743408,69.25151611)
\lineto(353.79743408,68.95151611)
\curveto(353.79743197,68.85150925)(353.81743195,68.76150934)(353.85743408,68.68151611)
\curveto(353.8674319,68.66150944)(353.87743189,68.64650946)(353.88743408,68.63651611)
\lineto(353.93243408,68.59151611)
\curveto(354.04243172,68.59150951)(354.13243163,68.63650947)(354.20243408,68.72651611)
\curveto(354.27243149,68.82650928)(354.33243143,68.9065092)(354.38243408,68.96651611)
\lineto(354.47243408,69.05651611)
\curveto(354.5624312,69.16650894)(354.68743108,69.28150882)(354.84743408,69.40151611)
\curveto(355.00743076,69.52150858)(355.15743061,69.61150849)(355.29743408,69.67151611)
\curveto(355.38743038,69.72150838)(355.48243028,69.75650835)(355.58243408,69.77651611)
\curveto(355.68243008,69.8065083)(355.78742998,69.83650827)(355.89743408,69.86651611)
\curveto(355.95742981,69.87650823)(356.01742975,69.88150822)(356.07743408,69.88151611)
\curveto(356.13742963,69.89150821)(356.19242957,69.9015082)(356.24243408,69.91151611)
}
}
{
\newrgbcolor{curcolor}{0 0 0}
\pscustom[linestyle=none,fillstyle=solid,fillcolor=curcolor]
{
\newpath
\moveto(480.28160645,72.68651611)
\lineto(481.19660645,72.68651611)
\curveto(481.2966038,72.68650542)(481.3916037,72.68650542)(481.48160645,72.68651611)
\curveto(481.57160352,72.68650542)(481.64660345,72.66650544)(481.70660645,72.62651611)
\curveto(481.7966033,72.56650554)(481.85660324,72.48650562)(481.88660645,72.38651611)
\curveto(481.92660317,72.28650582)(481.97160312,72.18150592)(482.02160645,72.07151611)
\curveto(482.10160299,71.88150622)(482.17160292,71.69150641)(482.23160645,71.50151611)
\curveto(482.30160279,71.31150679)(482.37660272,71.12150698)(482.45660645,70.93151611)
\curveto(482.52660257,70.75150735)(482.5916025,70.56650754)(482.65160645,70.37651611)
\curveto(482.71160238,70.19650791)(482.78160231,70.01650809)(482.86160645,69.83651611)
\curveto(482.92160217,69.69650841)(482.97660212,69.55150855)(483.02660645,69.40151611)
\curveto(483.07660202,69.25150885)(483.13160196,69.106509)(483.19160645,68.96651611)
\curveto(483.37160172,68.51650959)(483.54160155,68.06151004)(483.70160645,67.60151611)
\curveto(483.86160123,67.15151095)(484.03160106,66.7015114)(484.21160645,66.25151611)
\curveto(484.23160086,66.2015119)(484.24660085,66.15151195)(484.25660645,66.10151611)
\lineto(484.31660645,65.95151611)
\curveto(484.40660069,65.73151237)(484.4916006,65.5065126)(484.57160645,65.27651611)
\curveto(484.65160044,65.05651305)(484.73660036,64.83651327)(484.82660645,64.61651611)
\curveto(484.86660023,64.52651358)(484.90660019,64.41651369)(484.94660645,64.28651611)
\curveto(484.98660011,64.16651394)(485.05160004,64.09651401)(485.14160645,64.07651611)
\curveto(485.18159991,64.06651404)(485.21159988,64.06651404)(485.23160645,64.07651611)
\lineto(485.29160645,64.13651611)
\curveto(485.34159975,64.18651392)(485.37659972,64.24151386)(485.39660645,64.30151611)
\curveto(485.42659967,64.36151374)(485.45659964,64.42651368)(485.48660645,64.49651611)
\lineto(485.72660645,65.12651611)
\curveto(485.80659929,65.34651276)(485.88659921,65.56151254)(485.96660645,65.77151611)
\lineto(486.02660645,65.92151611)
\lineto(486.08660645,66.10151611)
\curveto(486.16659893,66.29151181)(486.23659886,66.48151162)(486.29660645,66.67151611)
\curveto(486.36659873,66.87151123)(486.44159865,67.07151103)(486.52160645,67.27151611)
\curveto(486.76159833,67.85151025)(486.98159811,68.43650967)(487.18160645,69.02651611)
\curveto(487.3915977,69.61650849)(487.61659748,70.2015079)(487.85660645,70.78151611)
\curveto(487.93659716,70.98150712)(488.01159708,71.18650692)(488.08160645,71.39651611)
\curveto(488.16159693,71.6065065)(488.24159685,71.81150629)(488.32160645,72.01151611)
\curveto(488.36159673,72.09150601)(488.3965967,72.19150591)(488.42660645,72.31151611)
\curveto(488.46659663,72.43150567)(488.52159657,72.51650559)(488.59160645,72.56651611)
\curveto(488.65159644,72.6065055)(488.72659637,72.63650547)(488.81660645,72.65651611)
\curveto(488.91659618,72.67650543)(489.02659607,72.68650542)(489.14660645,72.68651611)
\curveto(489.26659583,72.69650541)(489.38659571,72.69650541)(489.50660645,72.68651611)
\curveto(489.62659547,72.68650542)(489.73659536,72.68650542)(489.83660645,72.68651611)
\curveto(489.92659517,72.68650542)(490.01659508,72.68650542)(490.10660645,72.68651611)
\curveto(490.20659489,72.68650542)(490.28159481,72.66650544)(490.33160645,72.62651611)
\curveto(490.42159467,72.57650553)(490.47159462,72.48650562)(490.48160645,72.35651611)
\curveto(490.4915946,72.22650588)(490.4965946,72.08650602)(490.49660645,71.93651611)
\lineto(490.49660645,70.28651611)
\lineto(490.49660645,64.01651611)
\lineto(490.49660645,62.75651611)
\curveto(490.4965946,62.64651546)(490.4965946,62.53651557)(490.49660645,62.42651611)
\curveto(490.50659459,62.31651579)(490.48659461,62.23151587)(490.43660645,62.17151611)
\curveto(490.40659469,62.11151599)(490.36159473,62.07151603)(490.30160645,62.05151611)
\curveto(490.24159485,62.04151606)(490.17159492,62.02651608)(490.09160645,62.00651611)
\lineto(489.85160645,62.00651611)
\lineto(489.49160645,62.00651611)
\curveto(489.38159571,62.01651609)(489.30159579,62.06151604)(489.25160645,62.14151611)
\curveto(489.23159586,62.17151593)(489.21659588,62.2015159)(489.20660645,62.23151611)
\curveto(489.20659589,62.27151583)(489.1965959,62.31651579)(489.17660645,62.36651611)
\lineto(489.17660645,62.53151611)
\curveto(489.16659593,62.59151551)(489.16159593,62.66151544)(489.16160645,62.74151611)
\curveto(489.17159592,62.82151528)(489.17659592,62.89651521)(489.17660645,62.96651611)
\lineto(489.17660645,63.80651611)
\lineto(489.17660645,68.23151611)
\curveto(489.17659592,68.48150962)(489.17659592,68.73150937)(489.17660645,68.98151611)
\curveto(489.17659592,69.24150886)(489.17159592,69.49150861)(489.16160645,69.73151611)
\curveto(489.16159593,69.83150827)(489.15659594,69.94150816)(489.14660645,70.06151611)
\curveto(489.13659596,70.18150792)(489.08159601,70.24150786)(488.98160645,70.24151611)
\lineto(488.98160645,70.22651611)
\curveto(488.91159618,70.2065079)(488.85159624,70.14150796)(488.80160645,70.03151611)
\curveto(488.76159633,69.92150818)(488.72659637,69.82650828)(488.69660645,69.74651611)
\curveto(488.62659647,69.57650853)(488.56159653,69.4015087)(488.50160645,69.22151611)
\curveto(488.44159665,69.05150905)(488.37159672,68.88150922)(488.29160645,68.71151611)
\curveto(488.27159682,68.66150944)(488.25659684,68.61650949)(488.24660645,68.57651611)
\curveto(488.23659686,68.53650957)(488.22159687,68.49150961)(488.20160645,68.44151611)
\curveto(488.12159697,68.26150984)(488.05159704,68.07651003)(487.99160645,67.88651611)
\curveto(487.94159715,67.7065104)(487.87659722,67.52651058)(487.79660645,67.34651611)
\curveto(487.72659737,67.19651091)(487.66659743,67.04151106)(487.61660645,66.88151611)
\curveto(487.56659753,66.73151137)(487.51159758,66.58151152)(487.45160645,66.43151611)
\curveto(487.25159784,65.96151214)(487.07159802,65.48651262)(486.91160645,65.00651611)
\curveto(486.75159834,64.53651357)(486.57659852,64.07151403)(486.38660645,63.61151611)
\curveto(486.30659879,63.43151467)(486.23659886,63.25151485)(486.17660645,63.07151611)
\curveto(486.11659898,62.89151521)(486.05159904,62.71151539)(485.98160645,62.53151611)
\curveto(485.93159916,62.42151568)(485.88159921,62.31651579)(485.83160645,62.21651611)
\curveto(485.7915993,62.12651598)(485.70659939,62.06151604)(485.57660645,62.02151611)
\curveto(485.55659954,62.01151609)(485.53159956,62.0065161)(485.50160645,62.00651611)
\curveto(485.48159961,62.01651609)(485.45659964,62.01651609)(485.42660645,62.00651611)
\curveto(485.3965997,61.99651611)(485.36159973,61.99151611)(485.32160645,61.99151611)
\curveto(485.28159981,62.0015161)(485.24159985,62.0065161)(485.20160645,62.00651611)
\lineto(484.90160645,62.00651611)
\curveto(484.80160029,62.0065161)(484.72160037,62.03151607)(484.66160645,62.08151611)
\curveto(484.58160051,62.13151597)(484.52160057,62.2015159)(484.48160645,62.29151611)
\curveto(484.45160064,62.39151571)(484.41160068,62.49151561)(484.36160645,62.59151611)
\curveto(484.28160081,62.79151531)(484.20160089,62.99651511)(484.12160645,63.20651611)
\curveto(484.05160104,63.42651468)(483.97660112,63.63651447)(483.89660645,63.83651611)
\curveto(483.81660128,64.01651409)(483.74660135,64.19651391)(483.68660645,64.37651611)
\curveto(483.63660146,64.56651354)(483.57160152,64.75151335)(483.49160645,64.93151611)
\curveto(483.26160183,65.49151261)(483.04660205,66.05651205)(482.84660645,66.62651611)
\curveto(482.64660245,67.19651091)(482.43160266,67.76151034)(482.20160645,68.32151611)
\lineto(481.96160645,68.95151611)
\curveto(481.8916032,69.17150893)(481.81660328,69.38150872)(481.73660645,69.58151611)
\curveto(481.68660341,69.69150841)(481.64160345,69.79650831)(481.60160645,69.89651611)
\curveto(481.57160352,70.0065081)(481.52160357,70.101508)(481.45160645,70.18151611)
\curveto(481.44160365,70.2015079)(481.43160366,70.21150789)(481.42160645,70.21151611)
\lineto(481.39160645,70.24151611)
\lineto(481.31660645,70.24151611)
\lineto(481.28660645,70.21151611)
\curveto(481.27660382,70.21150789)(481.26660383,70.2065079)(481.25660645,70.19651611)
\curveto(481.23660386,70.14650796)(481.22660387,70.09150801)(481.22660645,70.03151611)
\curveto(481.22660387,69.97150813)(481.21660388,69.91150819)(481.19660645,69.85151611)
\lineto(481.19660645,69.68651611)
\curveto(481.17660392,69.62650848)(481.17160392,69.56150854)(481.18160645,69.49151611)
\curveto(481.1916039,69.42150868)(481.1966039,69.35150875)(481.19660645,69.28151611)
\lineto(481.19660645,68.47151611)
\lineto(481.19660645,63.91151611)
\lineto(481.19660645,62.72651611)
\curveto(481.1966039,62.61651549)(481.1916039,62.5065156)(481.18160645,62.39651611)
\curveto(481.18160391,62.28651582)(481.15660394,62.2015159)(481.10660645,62.14151611)
\curveto(481.05660404,62.06151604)(480.96660413,62.01651609)(480.83660645,62.00651611)
\lineto(480.44660645,62.00651611)
\lineto(480.25160645,62.00651611)
\curveto(480.20160489,62.0065161)(480.15160494,62.01651609)(480.10160645,62.03651611)
\curveto(479.97160512,62.07651603)(479.8966052,62.16151594)(479.87660645,62.29151611)
\curveto(479.86660523,62.42151568)(479.86160523,62.57151553)(479.86160645,62.74151611)
\lineto(479.86160645,64.48151611)
\lineto(479.86160645,70.48151611)
\lineto(479.86160645,71.89151611)
\curveto(479.86160523,72.0015061)(479.85660524,72.11650599)(479.84660645,72.23651611)
\curveto(479.84660525,72.35650575)(479.87160522,72.45150565)(479.92160645,72.52151611)
\curveto(479.96160513,72.58150552)(480.03660506,72.63150547)(480.14660645,72.67151611)
\curveto(480.16660493,72.68150542)(480.18660491,72.68150542)(480.20660645,72.67151611)
\curveto(480.23660486,72.67150543)(480.26160483,72.67650543)(480.28160645,72.68651611)
}
}
{
\newrgbcolor{curcolor}{0 0 0}
\pscustom[linestyle=none,fillstyle=solid,fillcolor=curcolor]
{
\newpath
\moveto(499.72371582,66.20651611)
\curveto(499.74370776,66.14651196)(499.75370775,66.05151205)(499.75371582,65.92151611)
\curveto(499.75370775,65.8015123)(499.74870776,65.71651239)(499.73871582,65.66651611)
\lineto(499.73871582,65.51651611)
\curveto(499.72870778,65.43651267)(499.71870779,65.36151274)(499.70871582,65.29151611)
\curveto(499.7087078,65.23151287)(499.7037078,65.16151294)(499.69371582,65.08151611)
\curveto(499.67370783,65.02151308)(499.65870785,64.96151314)(499.64871582,64.90151611)
\curveto(499.64870786,64.84151326)(499.63870787,64.78151332)(499.61871582,64.72151611)
\curveto(499.57870793,64.59151351)(499.54370796,64.46151364)(499.51371582,64.33151611)
\curveto(499.48370802,64.2015139)(499.44370806,64.08151402)(499.39371582,63.97151611)
\curveto(499.18370832,63.49151461)(498.9037086,63.08651502)(498.55371582,62.75651611)
\curveto(498.2037093,62.43651567)(497.77370973,62.19151591)(497.26371582,62.02151611)
\curveto(497.15371035,61.98151612)(497.03371047,61.95151615)(496.90371582,61.93151611)
\curveto(496.78371072,61.91151619)(496.65871085,61.89151621)(496.52871582,61.87151611)
\curveto(496.46871104,61.86151624)(496.4037111,61.85651625)(496.33371582,61.85651611)
\curveto(496.27371123,61.84651626)(496.21371129,61.84151626)(496.15371582,61.84151611)
\curveto(496.11371139,61.83151627)(496.05371145,61.82651628)(495.97371582,61.82651611)
\curveto(495.9037116,61.82651628)(495.85371165,61.83151627)(495.82371582,61.84151611)
\curveto(495.78371172,61.85151625)(495.74371176,61.85651625)(495.70371582,61.85651611)
\curveto(495.66371184,61.84651626)(495.62871188,61.84651626)(495.59871582,61.85651611)
\lineto(495.50871582,61.85651611)
\lineto(495.14871582,61.90151611)
\curveto(495.0087125,61.94151616)(494.87371263,61.98151612)(494.74371582,62.02151611)
\curveto(494.61371289,62.06151604)(494.48871302,62.106516)(494.36871582,62.15651611)
\curveto(493.91871359,62.35651575)(493.54871396,62.61651549)(493.25871582,62.93651611)
\curveto(492.96871454,63.25651485)(492.72871478,63.64651446)(492.53871582,64.10651611)
\curveto(492.48871502,64.2065139)(492.44871506,64.3065138)(492.41871582,64.40651611)
\curveto(492.39871511,64.5065136)(492.37871513,64.61151349)(492.35871582,64.72151611)
\curveto(492.33871517,64.76151334)(492.32871518,64.79151331)(492.32871582,64.81151611)
\curveto(492.33871517,64.84151326)(492.33871517,64.87651323)(492.32871582,64.91651611)
\curveto(492.3087152,64.99651311)(492.29371521,65.07651303)(492.28371582,65.15651611)
\curveto(492.28371522,65.24651286)(492.27371523,65.33151277)(492.25371582,65.41151611)
\lineto(492.25371582,65.53151611)
\curveto(492.25371525,65.57151253)(492.24871526,65.61651249)(492.23871582,65.66651611)
\curveto(492.22871528,65.71651239)(492.22371528,65.8015123)(492.22371582,65.92151611)
\curveto(492.22371528,66.05151205)(492.23371527,66.14651196)(492.25371582,66.20651611)
\curveto(492.27371523,66.27651183)(492.27871523,66.34651176)(492.26871582,66.41651611)
\curveto(492.25871525,66.48651162)(492.26371524,66.55651155)(492.28371582,66.62651611)
\curveto(492.29371521,66.67651143)(492.29871521,66.71651139)(492.29871582,66.74651611)
\curveto(492.3087152,66.78651132)(492.31871519,66.83151127)(492.32871582,66.88151611)
\curveto(492.35871515,67.0015111)(492.38371512,67.12151098)(492.40371582,67.24151611)
\curveto(492.43371507,67.36151074)(492.47371503,67.47651063)(492.52371582,67.58651611)
\curveto(492.67371483,67.95651015)(492.85371465,68.28650982)(493.06371582,68.57651611)
\curveto(493.28371422,68.87650923)(493.54871396,69.12650898)(493.85871582,69.32651611)
\curveto(493.97871353,69.4065087)(494.1037134,69.47150863)(494.23371582,69.52151611)
\curveto(494.36371314,69.58150852)(494.49871301,69.64150846)(494.63871582,69.70151611)
\curveto(494.75871275,69.75150835)(494.88871262,69.78150832)(495.02871582,69.79151611)
\curveto(495.16871234,69.81150829)(495.3087122,69.84150826)(495.44871582,69.88151611)
\lineto(495.64371582,69.88151611)
\curveto(495.71371179,69.89150821)(495.77871173,69.9015082)(495.83871582,69.91151611)
\curveto(496.72871078,69.92150818)(497.46871004,69.73650837)(498.05871582,69.35651611)
\curveto(498.64870886,68.97650913)(499.07370843,68.48150962)(499.33371582,67.87151611)
\curveto(499.38370812,67.77151033)(499.42370808,67.67151043)(499.45371582,67.57151611)
\curveto(499.48370802,67.47151063)(499.51870799,67.36651074)(499.55871582,67.25651611)
\curveto(499.58870792,67.14651096)(499.61370789,67.02651108)(499.63371582,66.89651611)
\curveto(499.65370785,66.77651133)(499.67870783,66.65151145)(499.70871582,66.52151611)
\curveto(499.71870779,66.47151163)(499.71870779,66.41651169)(499.70871582,66.35651611)
\curveto(499.7087078,66.3065118)(499.71370779,66.25651185)(499.72371582,66.20651611)
\moveto(498.38871582,65.35151611)
\curveto(498.4087091,65.42151268)(498.41370909,65.5015126)(498.40371582,65.59151611)
\lineto(498.40371582,65.84651611)
\curveto(498.4037091,66.23651187)(498.36870914,66.56651154)(498.29871582,66.83651611)
\curveto(498.26870924,66.91651119)(498.24370926,66.99651111)(498.22371582,67.07651611)
\curveto(498.2037093,67.15651095)(498.17870933,67.23151087)(498.14871582,67.30151611)
\curveto(497.86870964,67.95151015)(497.42371008,68.4015097)(496.81371582,68.65151611)
\curveto(496.74371076,68.68150942)(496.66871084,68.7015094)(496.58871582,68.71151611)
\lineto(496.34871582,68.77151611)
\curveto(496.26871124,68.79150931)(496.18371132,68.8015093)(496.09371582,68.80151611)
\lineto(495.82371582,68.80151611)
\lineto(495.55371582,68.75651611)
\curveto(495.45371205,68.73650937)(495.35871215,68.71150939)(495.26871582,68.68151611)
\curveto(495.18871232,68.66150944)(495.1087124,68.63150947)(495.02871582,68.59151611)
\curveto(494.95871255,68.57150953)(494.89371261,68.54150956)(494.83371582,68.50151611)
\curveto(494.77371273,68.46150964)(494.71871279,68.42150968)(494.66871582,68.38151611)
\curveto(494.42871308,68.21150989)(494.23371327,68.0065101)(494.08371582,67.76651611)
\curveto(493.93371357,67.52651058)(493.8037137,67.24651086)(493.69371582,66.92651611)
\curveto(493.66371384,66.82651128)(493.64371386,66.72151138)(493.63371582,66.61151611)
\curveto(493.62371388,66.51151159)(493.6087139,66.4065117)(493.58871582,66.29651611)
\curveto(493.57871393,66.25651185)(493.57371393,66.19151191)(493.57371582,66.10151611)
\curveto(493.56371394,66.07151203)(493.55871395,66.03651207)(493.55871582,65.99651611)
\curveto(493.56871394,65.95651215)(493.57371393,65.91151219)(493.57371582,65.86151611)
\lineto(493.57371582,65.56151611)
\curveto(493.57371393,65.46151264)(493.58371392,65.37151273)(493.60371582,65.29151611)
\lineto(493.63371582,65.11151611)
\curveto(493.65371385,65.01151309)(493.66871384,64.91151319)(493.67871582,64.81151611)
\curveto(493.69871381,64.72151338)(493.72871378,64.63651347)(493.76871582,64.55651611)
\curveto(493.86871364,64.31651379)(493.98371352,64.09151401)(494.11371582,63.88151611)
\curveto(494.25371325,63.67151443)(494.42371308,63.49651461)(494.62371582,63.35651611)
\curveto(494.67371283,63.32651478)(494.71871279,63.3015148)(494.75871582,63.28151611)
\curveto(494.79871271,63.26151484)(494.84371266,63.23651487)(494.89371582,63.20651611)
\curveto(494.97371253,63.15651495)(495.05871245,63.11151499)(495.14871582,63.07151611)
\curveto(495.24871226,63.04151506)(495.35371215,63.01151509)(495.46371582,62.98151611)
\curveto(495.51371199,62.96151514)(495.55871195,62.95151515)(495.59871582,62.95151611)
\curveto(495.64871186,62.96151514)(495.69871181,62.96151514)(495.74871582,62.95151611)
\curveto(495.77871173,62.94151516)(495.83871167,62.93151517)(495.92871582,62.92151611)
\curveto(496.02871148,62.91151519)(496.1037114,62.91651519)(496.15371582,62.93651611)
\curveto(496.19371131,62.94651516)(496.23371127,62.94651516)(496.27371582,62.93651611)
\curveto(496.31371119,62.93651517)(496.35371115,62.94651516)(496.39371582,62.96651611)
\curveto(496.47371103,62.98651512)(496.55371095,63.0015151)(496.63371582,63.01151611)
\curveto(496.71371079,63.03151507)(496.78871072,63.05651505)(496.85871582,63.08651611)
\curveto(497.19871031,63.22651488)(497.47371003,63.42151468)(497.68371582,63.67151611)
\curveto(497.89370961,63.92151418)(498.06870944,64.21651389)(498.20871582,64.55651611)
\curveto(498.25870925,64.67651343)(498.28870922,64.8015133)(498.29871582,64.93151611)
\curveto(498.31870919,65.07151303)(498.34870916,65.21151289)(498.38871582,65.35151611)
}
}
{
\newrgbcolor{curcolor}{0 0 0}
\pscustom[linestyle=none,fillstyle=solid,fillcolor=curcolor]
{
\newpath
\moveto(508.17199707,62.81651611)
\lineto(508.17199707,62.42651611)
\curveto(508.1719892,62.3065158)(508.14698922,62.2065159)(508.09699707,62.12651611)
\curveto(508.04698932,62.05651605)(507.96198941,62.01651609)(507.84199707,62.00651611)
\lineto(507.49699707,62.00651611)
\curveto(507.43698993,62.0065161)(507.37698999,62.0015161)(507.31699707,61.99151611)
\curveto(507.2669901,61.99151611)(507.22199015,62.0015161)(507.18199707,62.02151611)
\curveto(507.09199028,62.04151606)(507.03199034,62.08151602)(507.00199707,62.14151611)
\curveto(506.96199041,62.19151591)(506.93699043,62.25151585)(506.92699707,62.32151611)
\curveto(506.92699044,62.39151571)(506.91199046,62.46151564)(506.88199707,62.53151611)
\curveto(506.8719905,62.55151555)(506.85699051,62.56651554)(506.83699707,62.57651611)
\curveto(506.82699054,62.59651551)(506.81199056,62.61651549)(506.79199707,62.63651611)
\curveto(506.69199068,62.64651546)(506.61199076,62.62651548)(506.55199707,62.57651611)
\curveto(506.50199087,62.52651558)(506.44699092,62.47651563)(506.38699707,62.42651611)
\curveto(506.18699118,62.27651583)(505.98699138,62.16151594)(505.78699707,62.08151611)
\curveto(505.60699176,62.0015161)(505.39699197,61.94151616)(505.15699707,61.90151611)
\curveto(504.92699244,61.86151624)(504.68699268,61.84151626)(504.43699707,61.84151611)
\curveto(504.19699317,61.83151627)(503.95699341,61.84651626)(503.71699707,61.88651611)
\curveto(503.47699389,61.91651619)(503.2669941,61.97151613)(503.08699707,62.05151611)
\curveto(502.5669948,62.27151583)(502.14699522,62.56651554)(501.82699707,62.93651611)
\curveto(501.50699586,63.31651479)(501.25699611,63.78651432)(501.07699707,64.34651611)
\curveto(501.03699633,64.43651367)(501.00699636,64.52651358)(500.98699707,64.61651611)
\curveto(500.97699639,64.71651339)(500.95699641,64.81651329)(500.92699707,64.91651611)
\curveto(500.91699645,64.96651314)(500.91199646,65.01651309)(500.91199707,65.06651611)
\curveto(500.91199646,65.11651299)(500.90699646,65.16651294)(500.89699707,65.21651611)
\curveto(500.87699649,65.26651284)(500.8669965,65.31651279)(500.86699707,65.36651611)
\curveto(500.87699649,65.42651268)(500.87699649,65.48151262)(500.86699707,65.53151611)
\lineto(500.86699707,65.68151611)
\curveto(500.84699652,65.73151237)(500.83699653,65.79651231)(500.83699707,65.87651611)
\curveto(500.83699653,65.95651215)(500.84699652,66.02151208)(500.86699707,66.07151611)
\lineto(500.86699707,66.23651611)
\curveto(500.88699648,66.3065118)(500.89199648,66.37651173)(500.88199707,66.44651611)
\curveto(500.88199649,66.52651158)(500.89199648,66.6015115)(500.91199707,66.67151611)
\curveto(500.92199645,66.72151138)(500.92699644,66.76651134)(500.92699707,66.80651611)
\curveto(500.92699644,66.84651126)(500.93199644,66.89151121)(500.94199707,66.94151611)
\curveto(500.9719964,67.04151106)(500.99699637,67.13651097)(501.01699707,67.22651611)
\curveto(501.03699633,67.32651078)(501.06199631,67.42151068)(501.09199707,67.51151611)
\curveto(501.22199615,67.89151021)(501.38699598,68.23150987)(501.58699707,68.53151611)
\curveto(501.79699557,68.84150926)(502.04699532,69.09650901)(502.33699707,69.29651611)
\curveto(502.50699486,69.41650869)(502.68199469,69.51650859)(502.86199707,69.59651611)
\curveto(503.05199432,69.67650843)(503.25699411,69.74650836)(503.47699707,69.80651611)
\curveto(503.54699382,69.81650829)(503.61199376,69.82650828)(503.67199707,69.83651611)
\curveto(503.74199363,69.84650826)(503.81199356,69.86150824)(503.88199707,69.88151611)
\lineto(504.03199707,69.88151611)
\curveto(504.11199326,69.9015082)(504.22699314,69.91150819)(504.37699707,69.91151611)
\curveto(504.53699283,69.91150819)(504.65699271,69.9015082)(504.73699707,69.88151611)
\curveto(504.77699259,69.87150823)(504.83199254,69.86650824)(504.90199707,69.86651611)
\curveto(505.01199236,69.83650827)(505.12199225,69.81150829)(505.23199707,69.79151611)
\curveto(505.34199203,69.78150832)(505.44699192,69.75150835)(505.54699707,69.70151611)
\curveto(505.69699167,69.64150846)(505.83699153,69.57650853)(505.96699707,69.50651611)
\curveto(506.10699126,69.43650867)(506.23699113,69.35650875)(506.35699707,69.26651611)
\curveto(506.41699095,69.21650889)(506.47699089,69.16150894)(506.53699707,69.10151611)
\curveto(506.60699076,69.05150905)(506.69699067,69.03650907)(506.80699707,69.05651611)
\curveto(506.82699054,69.08650902)(506.84199053,69.11150899)(506.85199707,69.13151611)
\curveto(506.8719905,69.15150895)(506.88699048,69.18150892)(506.89699707,69.22151611)
\curveto(506.92699044,69.31150879)(506.93699043,69.42650868)(506.92699707,69.56651611)
\lineto(506.92699707,69.94151611)
\lineto(506.92699707,71.66651611)
\lineto(506.92699707,72.13151611)
\curveto(506.92699044,72.31150579)(506.95199042,72.44150566)(507.00199707,72.52151611)
\curveto(507.04199033,72.59150551)(507.10199027,72.63650547)(507.18199707,72.65651611)
\curveto(507.20199017,72.65650545)(507.22699014,72.65650545)(507.25699707,72.65651611)
\curveto(507.28699008,72.66650544)(507.31199006,72.67150543)(507.33199707,72.67151611)
\curveto(507.4719899,72.68150542)(507.61698975,72.68150542)(507.76699707,72.67151611)
\curveto(507.92698944,72.67150543)(508.03698933,72.63150547)(508.09699707,72.55151611)
\curveto(508.14698922,72.47150563)(508.1719892,72.37150573)(508.17199707,72.25151611)
\lineto(508.17199707,71.87651611)
\lineto(508.17199707,62.81651611)
\moveto(506.95699707,65.65151611)
\curveto(506.97699039,65.7015124)(506.98699038,65.76651234)(506.98699707,65.84651611)
\curveto(506.98699038,65.93651217)(506.97699039,66.0065121)(506.95699707,66.05651611)
\lineto(506.95699707,66.28151611)
\curveto(506.93699043,66.37151173)(506.92199045,66.46151164)(506.91199707,66.55151611)
\curveto(506.90199047,66.65151145)(506.88199049,66.74151136)(506.85199707,66.82151611)
\curveto(506.83199054,66.9015112)(506.81199056,66.97651113)(506.79199707,67.04651611)
\curveto(506.78199059,67.11651099)(506.76199061,67.18651092)(506.73199707,67.25651611)
\curveto(506.61199076,67.55651055)(506.45699091,67.82151028)(506.26699707,68.05151611)
\curveto(506.07699129,68.28150982)(505.83699153,68.46150964)(505.54699707,68.59151611)
\curveto(505.44699192,68.64150946)(505.34199203,68.67650943)(505.23199707,68.69651611)
\curveto(505.13199224,68.72650938)(505.02199235,68.75150935)(504.90199707,68.77151611)
\curveto(504.82199255,68.79150931)(504.73199264,68.8015093)(504.63199707,68.80151611)
\lineto(504.36199707,68.80151611)
\curveto(504.31199306,68.79150931)(504.2669931,68.78150932)(504.22699707,68.77151611)
\lineto(504.09199707,68.77151611)
\curveto(504.01199336,68.75150935)(503.92699344,68.73150937)(503.83699707,68.71151611)
\curveto(503.75699361,68.69150941)(503.67699369,68.66650944)(503.59699707,68.63651611)
\curveto(503.27699409,68.49650961)(503.01699435,68.29150981)(502.81699707,68.02151611)
\curveto(502.62699474,67.76151034)(502.4719949,67.45651065)(502.35199707,67.10651611)
\curveto(502.31199506,66.99651111)(502.28199509,66.88151122)(502.26199707,66.76151611)
\curveto(502.25199512,66.65151145)(502.23699513,66.54151156)(502.21699707,66.43151611)
\curveto(502.21699515,66.39151171)(502.21199516,66.35151175)(502.20199707,66.31151611)
\lineto(502.20199707,66.20651611)
\curveto(502.18199519,66.15651195)(502.1719952,66.101512)(502.17199707,66.04151611)
\curveto(502.18199519,65.98151212)(502.18699518,65.92651218)(502.18699707,65.87651611)
\lineto(502.18699707,65.54651611)
\curveto(502.18699518,65.44651266)(502.19699517,65.35151275)(502.21699707,65.26151611)
\curveto(502.22699514,65.23151287)(502.23199514,65.18151292)(502.23199707,65.11151611)
\curveto(502.25199512,65.04151306)(502.2669951,64.97151313)(502.27699707,64.90151611)
\lineto(502.33699707,64.69151611)
\curveto(502.44699492,64.34151376)(502.59699477,64.04151406)(502.78699707,63.79151611)
\curveto(502.97699439,63.54151456)(503.21699415,63.33651477)(503.50699707,63.17651611)
\curveto(503.59699377,63.12651498)(503.68699368,63.08651502)(503.77699707,63.05651611)
\curveto(503.8669935,63.02651508)(503.9669934,62.99651511)(504.07699707,62.96651611)
\curveto(504.12699324,62.94651516)(504.17699319,62.94151516)(504.22699707,62.95151611)
\curveto(504.28699308,62.96151514)(504.34199303,62.95651515)(504.39199707,62.93651611)
\curveto(504.43199294,62.92651518)(504.4719929,62.92151518)(504.51199707,62.92151611)
\lineto(504.64699707,62.92151611)
\lineto(504.78199707,62.92151611)
\curveto(504.81199256,62.93151517)(504.86199251,62.93651517)(504.93199707,62.93651611)
\curveto(505.01199236,62.95651515)(505.09199228,62.97151513)(505.17199707,62.98151611)
\curveto(505.25199212,63.0015151)(505.32699204,63.02651508)(505.39699707,63.05651611)
\curveto(505.72699164,63.19651491)(505.99199138,63.37151473)(506.19199707,63.58151611)
\curveto(506.40199097,63.8015143)(506.57699079,64.07651403)(506.71699707,64.40651611)
\curveto(506.7669906,64.51651359)(506.80199057,64.62651348)(506.82199707,64.73651611)
\curveto(506.84199053,64.84651326)(506.8669905,64.95651315)(506.89699707,65.06651611)
\curveto(506.91699045,65.106513)(506.92699044,65.14151296)(506.92699707,65.17151611)
\curveto(506.92699044,65.21151289)(506.93199044,65.25151285)(506.94199707,65.29151611)
\curveto(506.95199042,65.35151275)(506.95199042,65.41151269)(506.94199707,65.47151611)
\curveto(506.94199043,65.53151257)(506.94699042,65.59151251)(506.95699707,65.65151611)
}
}
{
\newrgbcolor{curcolor}{0 0 0}
\pscustom[linestyle=none,fillstyle=solid,fillcolor=curcolor]
{
\newpath
\moveto(516.86824707,66.17651611)
\curveto(516.88823939,66.07651203)(516.88823939,65.96151214)(516.86824707,65.83151611)
\curveto(516.85823942,65.71151239)(516.82823945,65.62651248)(516.77824707,65.57651611)
\curveto(516.72823955,65.53651257)(516.65323962,65.5065126)(516.55324707,65.48651611)
\curveto(516.46323981,65.47651263)(516.35823992,65.47151263)(516.23824707,65.47151611)
\lineto(515.87824707,65.47151611)
\curveto(515.75824052,65.48151262)(515.65324062,65.48651262)(515.56324707,65.48651611)
\lineto(511.72324707,65.48651611)
\curveto(511.64324463,65.48651262)(511.56324471,65.48151262)(511.48324707,65.47151611)
\curveto(511.40324487,65.47151263)(511.33824494,65.45651265)(511.28824707,65.42651611)
\curveto(511.24824503,65.4065127)(511.20824507,65.36651274)(511.16824707,65.30651611)
\curveto(511.14824513,65.27651283)(511.12824515,65.23151287)(511.10824707,65.17151611)
\curveto(511.08824519,65.12151298)(511.08824519,65.07151303)(511.10824707,65.02151611)
\curveto(511.11824516,64.97151313)(511.12324515,64.92651318)(511.12324707,64.88651611)
\curveto(511.12324515,64.84651326)(511.12824515,64.8065133)(511.13824707,64.76651611)
\curveto(511.15824512,64.68651342)(511.1782451,64.6015135)(511.19824707,64.51151611)
\curveto(511.21824506,64.43151367)(511.24824503,64.35151375)(511.28824707,64.27151611)
\curveto(511.51824476,63.73151437)(511.89824438,63.34651476)(512.42824707,63.11651611)
\curveto(512.48824379,63.08651502)(512.55324372,63.06151504)(512.62324707,63.04151611)
\lineto(512.83324707,62.98151611)
\curveto(512.86324341,62.97151513)(512.91324336,62.96651514)(512.98324707,62.96651611)
\curveto(513.12324315,62.92651518)(513.30824297,62.9065152)(513.53824707,62.90651611)
\curveto(513.76824251,62.9065152)(513.95324232,62.92651518)(514.09324707,62.96651611)
\curveto(514.23324204,63.0065151)(514.35824192,63.04651506)(514.46824707,63.08651611)
\curveto(514.58824169,63.13651497)(514.69824158,63.19651491)(514.79824707,63.26651611)
\curveto(514.90824137,63.33651477)(515.00324127,63.41651469)(515.08324707,63.50651611)
\curveto(515.16324111,63.6065145)(515.23324104,63.71151439)(515.29324707,63.82151611)
\curveto(515.35324092,63.92151418)(515.40324087,64.02651408)(515.44324707,64.13651611)
\curveto(515.49324078,64.24651386)(515.5732407,64.32651378)(515.68324707,64.37651611)
\curveto(515.72324055,64.39651371)(515.78824049,64.41151369)(515.87824707,64.42151611)
\curveto(515.96824031,64.43151367)(516.05824022,64.43151367)(516.14824707,64.42151611)
\curveto(516.23824004,64.42151368)(516.32323995,64.41651369)(516.40324707,64.40651611)
\curveto(516.48323979,64.39651371)(516.53823974,64.37651373)(516.56824707,64.34651611)
\curveto(516.66823961,64.27651383)(516.69323958,64.16151394)(516.64324707,64.00151611)
\curveto(516.56323971,63.73151437)(516.45823982,63.49151461)(516.32824707,63.28151611)
\curveto(516.12824015,62.96151514)(515.89824038,62.69651541)(515.63824707,62.48651611)
\curveto(515.38824089,62.28651582)(515.06824121,62.12151598)(514.67824707,61.99151611)
\curveto(514.5782417,61.95151615)(514.4782418,61.92651618)(514.37824707,61.91651611)
\curveto(514.278242,61.89651621)(514.1732421,61.87651623)(514.06324707,61.85651611)
\curveto(514.01324226,61.84651626)(513.96324231,61.84151626)(513.91324707,61.84151611)
\curveto(513.8732424,61.84151626)(513.82824245,61.83651627)(513.77824707,61.82651611)
\lineto(513.62824707,61.82651611)
\curveto(513.5782427,61.81651629)(513.51824276,61.81151629)(513.44824707,61.81151611)
\curveto(513.38824289,61.81151629)(513.33824294,61.81651629)(513.29824707,61.82651611)
\lineto(513.16324707,61.82651611)
\curveto(513.11324316,61.83651627)(513.06824321,61.84151626)(513.02824707,61.84151611)
\curveto(512.98824329,61.84151626)(512.94824333,61.84651626)(512.90824707,61.85651611)
\curveto(512.85824342,61.86651624)(512.80324347,61.87651623)(512.74324707,61.88651611)
\curveto(512.68324359,61.88651622)(512.62824365,61.89151621)(512.57824707,61.90151611)
\curveto(512.48824379,61.92151618)(512.39824388,61.94651616)(512.30824707,61.97651611)
\curveto(512.21824406,61.99651611)(512.13324414,62.02151608)(512.05324707,62.05151611)
\curveto(512.01324426,62.07151603)(511.9782443,62.08151602)(511.94824707,62.08151611)
\curveto(511.91824436,62.09151601)(511.88324439,62.106516)(511.84324707,62.12651611)
\curveto(511.69324458,62.19651591)(511.53324474,62.28151582)(511.36324707,62.38151611)
\curveto(511.0732452,62.57151553)(510.82324545,62.8015153)(510.61324707,63.07151611)
\curveto(510.41324586,63.35151475)(510.24324603,63.66151444)(510.10324707,64.00151611)
\curveto(510.05324622,64.11151399)(510.01324626,64.22651388)(509.98324707,64.34651611)
\curveto(509.96324631,64.46651364)(509.93324634,64.58651352)(509.89324707,64.70651611)
\curveto(509.88324639,64.74651336)(509.8782464,64.78151332)(509.87824707,64.81151611)
\curveto(509.8782464,64.84151326)(509.8732464,64.88151322)(509.86324707,64.93151611)
\curveto(509.84324643,65.01151309)(509.82824645,65.09651301)(509.81824707,65.18651611)
\curveto(509.80824647,65.27651283)(509.79324648,65.36651274)(509.77324707,65.45651611)
\lineto(509.77324707,65.66651611)
\curveto(509.76324651,65.7065124)(509.75324652,65.76151234)(509.74324707,65.83151611)
\curveto(509.74324653,65.91151219)(509.74824653,65.97651213)(509.75824707,66.02651611)
\lineto(509.75824707,66.19151611)
\curveto(509.7782465,66.24151186)(509.78324649,66.29151181)(509.77324707,66.34151611)
\curveto(509.7732465,66.4015117)(509.7782465,66.45651165)(509.78824707,66.50651611)
\curveto(509.82824645,66.66651144)(509.85824642,66.82651128)(509.87824707,66.98651611)
\curveto(509.90824637,67.14651096)(509.95324632,67.29651081)(510.01324707,67.43651611)
\curveto(510.06324621,67.54651056)(510.10824617,67.65651045)(510.14824707,67.76651611)
\curveto(510.19824608,67.88651022)(510.25324602,68.0015101)(510.31324707,68.11151611)
\curveto(510.53324574,68.46150964)(510.78324549,68.76150934)(511.06324707,69.01151611)
\curveto(511.34324493,69.27150883)(511.68824459,69.48650862)(512.09824707,69.65651611)
\curveto(512.21824406,69.7065084)(512.33824394,69.74150836)(512.45824707,69.76151611)
\curveto(512.58824369,69.79150831)(512.72324355,69.82150828)(512.86324707,69.85151611)
\curveto(512.91324336,69.86150824)(512.95824332,69.86650824)(512.99824707,69.86651611)
\curveto(513.03824324,69.87650823)(513.08324319,69.88150822)(513.13324707,69.88151611)
\curveto(513.15324312,69.89150821)(513.1782431,69.89150821)(513.20824707,69.88151611)
\curveto(513.23824304,69.87150823)(513.26324301,69.87650823)(513.28324707,69.89651611)
\curveto(513.70324257,69.9065082)(514.06824221,69.86150824)(514.37824707,69.76151611)
\curveto(514.68824159,69.67150843)(514.96824131,69.54650856)(515.21824707,69.38651611)
\curveto(515.26824101,69.36650874)(515.30824097,69.33650877)(515.33824707,69.29651611)
\curveto(515.36824091,69.26650884)(515.40324087,69.24150886)(515.44324707,69.22151611)
\curveto(515.52324075,69.16150894)(515.60324067,69.09150901)(515.68324707,69.01151611)
\curveto(515.7732405,68.93150917)(515.84824043,68.85150925)(515.90824707,68.77151611)
\curveto(516.06824021,68.56150954)(516.20324007,68.36150974)(516.31324707,68.17151611)
\curveto(516.38323989,68.06151004)(516.43823984,67.94151016)(516.47824707,67.81151611)
\curveto(516.51823976,67.68151042)(516.56323971,67.55151055)(516.61324707,67.42151611)
\curveto(516.66323961,67.29151081)(516.69823958,67.15651095)(516.71824707,67.01651611)
\curveto(516.74823953,66.87651123)(516.78323949,66.73651137)(516.82324707,66.59651611)
\curveto(516.83323944,66.52651158)(516.83823944,66.45651165)(516.83824707,66.38651611)
\lineto(516.86824707,66.17651611)
\moveto(515.41324707,66.68651611)
\curveto(515.44324083,66.72651138)(515.46824081,66.77651133)(515.48824707,66.83651611)
\curveto(515.50824077,66.9065112)(515.50824077,66.97651113)(515.48824707,67.04651611)
\curveto(515.42824085,67.26651084)(515.34324093,67.47151063)(515.23324707,67.66151611)
\curveto(515.09324118,67.89151021)(514.93824134,68.08651002)(514.76824707,68.24651611)
\curveto(514.59824168,68.4065097)(514.3782419,68.54150956)(514.10824707,68.65151611)
\curveto(514.03824224,68.67150943)(513.96824231,68.68650942)(513.89824707,68.69651611)
\curveto(513.82824245,68.71650939)(513.75324252,68.73650937)(513.67324707,68.75651611)
\curveto(513.59324268,68.77650933)(513.50824277,68.78650932)(513.41824707,68.78651611)
\lineto(513.16324707,68.78651611)
\curveto(513.13324314,68.76650934)(513.09824318,68.75650935)(513.05824707,68.75651611)
\curveto(513.01824326,68.76650934)(512.98324329,68.76650934)(512.95324707,68.75651611)
\lineto(512.71324707,68.69651611)
\curveto(512.64324363,68.68650942)(512.5732437,68.67150943)(512.50324707,68.65151611)
\curveto(512.21324406,68.53150957)(511.9782443,68.38150972)(511.79824707,68.20151611)
\curveto(511.62824465,68.02151008)(511.4732448,67.79651031)(511.33324707,67.52651611)
\curveto(511.30324497,67.47651063)(511.273245,67.41151069)(511.24324707,67.33151611)
\curveto(511.21324506,67.26151084)(511.18824509,67.18151092)(511.16824707,67.09151611)
\curveto(511.14824513,67.0015111)(511.14324513,66.91651119)(511.15324707,66.83651611)
\curveto(511.16324511,66.75651135)(511.19824508,66.69651141)(511.25824707,66.65651611)
\curveto(511.33824494,66.59651151)(511.4732448,66.56651154)(511.66324707,66.56651611)
\curveto(511.86324441,66.57651153)(512.03324424,66.58151152)(512.17324707,66.58151611)
\lineto(514.45324707,66.58151611)
\curveto(514.60324167,66.58151152)(514.78324149,66.57651153)(514.99324707,66.56651611)
\curveto(515.20324107,66.56651154)(515.34324093,66.6065115)(515.41324707,66.68651611)
}
}
{
\newrgbcolor{curcolor}{0 0 0}
\pscustom[linestyle=none,fillstyle=solid,fillcolor=curcolor]
{
\newpath
\moveto(521.8198877,69.91151611)
\curveto(522.04988291,69.91150819)(522.17988278,69.85150825)(522.2098877,69.73151611)
\curveto(522.23988272,69.62150848)(522.2548827,69.45650865)(522.2548877,69.23651611)
\lineto(522.2548877,68.95151611)
\curveto(522.2548827,68.86150924)(522.22988273,68.78650932)(522.1798877,68.72651611)
\curveto(522.11988284,68.64650946)(522.03488292,68.6015095)(521.9248877,68.59151611)
\curveto(521.81488314,68.59150951)(521.70488325,68.57650953)(521.5948877,68.54651611)
\curveto(521.4548835,68.51650959)(521.31988364,68.48650962)(521.1898877,68.45651611)
\curveto(521.06988389,68.42650968)(520.954884,68.38650972)(520.8448877,68.33651611)
\curveto(520.5548844,68.2065099)(520.31988464,68.02651008)(520.1398877,67.79651611)
\curveto(519.959885,67.57651053)(519.80488515,67.32151078)(519.6748877,67.03151611)
\curveto(519.63488532,66.92151118)(519.60488535,66.8065113)(519.5848877,66.68651611)
\curveto(519.56488539,66.57651153)(519.53988542,66.46151164)(519.5098877,66.34151611)
\curveto(519.49988546,66.29151181)(519.49488546,66.24151186)(519.4948877,66.19151611)
\curveto(519.50488545,66.14151196)(519.50488545,66.09151201)(519.4948877,66.04151611)
\curveto(519.46488549,65.92151218)(519.44988551,65.78151232)(519.4498877,65.62151611)
\curveto(519.4598855,65.47151263)(519.46488549,65.32651278)(519.4648877,65.18651611)
\lineto(519.4648877,63.34151611)
\lineto(519.4648877,62.99651611)
\curveto(519.46488549,62.87651523)(519.4598855,62.76151534)(519.4498877,62.65151611)
\curveto(519.43988552,62.54151556)(519.43488552,62.44651566)(519.4348877,62.36651611)
\curveto(519.44488551,62.28651582)(519.42488553,62.21651589)(519.3748877,62.15651611)
\curveto(519.32488563,62.08651602)(519.24488571,62.04651606)(519.1348877,62.03651611)
\curveto(519.03488592,62.02651608)(518.92488603,62.02151608)(518.8048877,62.02151611)
\lineto(518.5348877,62.02151611)
\curveto(518.48488647,62.04151606)(518.43488652,62.05651605)(518.3848877,62.06651611)
\curveto(518.34488661,62.08651602)(518.31488664,62.11151599)(518.2948877,62.14151611)
\curveto(518.24488671,62.21151589)(518.21488674,62.29651581)(518.2048877,62.39651611)
\lineto(518.2048877,62.72651611)
\lineto(518.2048877,63.88151611)
\lineto(518.2048877,68.03651611)
\lineto(518.2048877,69.07151611)
\lineto(518.2048877,69.37151611)
\curveto(518.21488674,69.47150863)(518.24488671,69.55650855)(518.2948877,69.62651611)
\curveto(518.32488663,69.66650844)(518.37488658,69.69650841)(518.4448877,69.71651611)
\curveto(518.52488643,69.73650837)(518.60988635,69.74650836)(518.6998877,69.74651611)
\curveto(518.78988617,69.75650835)(518.87988608,69.75650835)(518.9698877,69.74651611)
\curveto(519.0598859,69.73650837)(519.12988583,69.72150838)(519.1798877,69.70151611)
\curveto(519.2598857,69.67150843)(519.30988565,69.61150849)(519.3298877,69.52151611)
\curveto(519.3598856,69.44150866)(519.37488558,69.35150875)(519.3748877,69.25151611)
\lineto(519.3748877,68.95151611)
\curveto(519.37488558,68.85150925)(519.39488556,68.76150934)(519.4348877,68.68151611)
\curveto(519.44488551,68.66150944)(519.4548855,68.64650946)(519.4648877,68.63651611)
\lineto(519.5098877,68.59151611)
\curveto(519.61988534,68.59150951)(519.70988525,68.63650947)(519.7798877,68.72651611)
\curveto(519.84988511,68.82650928)(519.90988505,68.9065092)(519.9598877,68.96651611)
\lineto(520.0498877,69.05651611)
\curveto(520.13988482,69.16650894)(520.26488469,69.28150882)(520.4248877,69.40151611)
\curveto(520.58488437,69.52150858)(520.73488422,69.61150849)(520.8748877,69.67151611)
\curveto(520.96488399,69.72150838)(521.0598839,69.75650835)(521.1598877,69.77651611)
\curveto(521.2598837,69.8065083)(521.36488359,69.83650827)(521.4748877,69.86651611)
\curveto(521.53488342,69.87650823)(521.59488336,69.88150822)(521.6548877,69.88151611)
\curveto(521.71488324,69.89150821)(521.76988319,69.9015082)(521.8198877,69.91151611)
}
}
{
\newrgbcolor{curcolor}{0 0 0}
\pscustom[linestyle=none,fillstyle=solid,fillcolor=curcolor]
{
\newpath
\moveto(530.06965332,62.56151611)
\curveto(530.09964549,62.4015157)(530.08464551,62.26651584)(530.02465332,62.15651611)
\curveto(529.96464563,62.05651605)(529.88464571,61.98151612)(529.78465332,61.93151611)
\curveto(529.73464586,61.91151619)(529.67964591,61.9015162)(529.61965332,61.90151611)
\curveto(529.56964602,61.9015162)(529.51464608,61.89151621)(529.45465332,61.87151611)
\curveto(529.23464636,61.82151628)(529.01464658,61.83651627)(528.79465332,61.91651611)
\curveto(528.58464701,61.98651612)(528.43964715,62.07651603)(528.35965332,62.18651611)
\curveto(528.30964728,62.25651585)(528.26464733,62.33651577)(528.22465332,62.42651611)
\curveto(528.18464741,62.52651558)(528.13464746,62.6065155)(528.07465332,62.66651611)
\curveto(528.05464754,62.68651542)(528.02964756,62.7065154)(527.99965332,62.72651611)
\curveto(527.97964761,62.74651536)(527.94964764,62.75151535)(527.90965332,62.74151611)
\curveto(527.79964779,62.71151539)(527.6946479,62.65651545)(527.59465332,62.57651611)
\curveto(527.50464809,62.49651561)(527.41464818,62.42651568)(527.32465332,62.36651611)
\curveto(527.1946484,62.28651582)(527.05464854,62.21151589)(526.90465332,62.14151611)
\curveto(526.75464884,62.08151602)(526.594649,62.02651608)(526.42465332,61.97651611)
\curveto(526.32464927,61.94651616)(526.21464938,61.92651618)(526.09465332,61.91651611)
\curveto(525.98464961,61.9065162)(525.87464972,61.89151621)(525.76465332,61.87151611)
\curveto(525.71464988,61.86151624)(525.66964992,61.85651625)(525.62965332,61.85651611)
\lineto(525.52465332,61.85651611)
\curveto(525.41465018,61.83651627)(525.30965028,61.83651627)(525.20965332,61.85651611)
\lineto(525.07465332,61.85651611)
\curveto(525.02465057,61.86651624)(524.97465062,61.87151623)(524.92465332,61.87151611)
\curveto(524.87465072,61.87151623)(524.82965076,61.88151622)(524.78965332,61.90151611)
\curveto(524.74965084,61.91151619)(524.71465088,61.91651619)(524.68465332,61.91651611)
\curveto(524.66465093,61.9065162)(524.63965095,61.9065162)(524.60965332,61.91651611)
\lineto(524.36965332,61.97651611)
\curveto(524.2896513,61.98651612)(524.21465138,62.0065161)(524.14465332,62.03651611)
\curveto(523.84465175,62.16651594)(523.59965199,62.31151579)(523.40965332,62.47151611)
\curveto(523.22965236,62.64151546)(523.07965251,62.87651523)(522.95965332,63.17651611)
\curveto(522.86965272,63.39651471)(522.82465277,63.66151444)(522.82465332,63.97151611)
\lineto(522.82465332,64.28651611)
\curveto(522.83465276,64.33651377)(522.83965275,64.38651372)(522.83965332,64.43651611)
\lineto(522.86965332,64.61651611)
\lineto(522.98965332,64.94651611)
\curveto(523.02965256,65.05651305)(523.07965251,65.15651295)(523.13965332,65.24651611)
\curveto(523.31965227,65.53651257)(523.56465203,65.75151235)(523.87465332,65.89151611)
\curveto(524.18465141,66.03151207)(524.52465107,66.15651195)(524.89465332,66.26651611)
\curveto(525.03465056,66.3065118)(525.17965041,66.33651177)(525.32965332,66.35651611)
\curveto(525.47965011,66.37651173)(525.62964996,66.4015117)(525.77965332,66.43151611)
\curveto(525.84964974,66.45151165)(525.91464968,66.46151164)(525.97465332,66.46151611)
\curveto(526.04464955,66.46151164)(526.11964947,66.47151163)(526.19965332,66.49151611)
\curveto(526.26964932,66.51151159)(526.33964925,66.52151158)(526.40965332,66.52151611)
\curveto(526.47964911,66.53151157)(526.55464904,66.54651156)(526.63465332,66.56651611)
\curveto(526.88464871,66.62651148)(527.11964847,66.67651143)(527.33965332,66.71651611)
\curveto(527.55964803,66.76651134)(527.73464786,66.88151122)(527.86465332,67.06151611)
\curveto(527.92464767,67.14151096)(527.97464762,67.24151086)(528.01465332,67.36151611)
\curveto(528.05464754,67.49151061)(528.05464754,67.63151047)(528.01465332,67.78151611)
\curveto(527.95464764,68.02151008)(527.86464773,68.21150989)(527.74465332,68.35151611)
\curveto(527.63464796,68.49150961)(527.47464812,68.6015095)(527.26465332,68.68151611)
\curveto(527.14464845,68.73150937)(526.99964859,68.76650934)(526.82965332,68.78651611)
\curveto(526.66964892,68.8065093)(526.49964909,68.81650929)(526.31965332,68.81651611)
\curveto(526.13964945,68.81650929)(525.96464963,68.8065093)(525.79465332,68.78651611)
\curveto(525.62464997,68.76650934)(525.47965011,68.73650937)(525.35965332,68.69651611)
\curveto(525.1896504,68.63650947)(525.02465057,68.55150955)(524.86465332,68.44151611)
\curveto(524.78465081,68.38150972)(524.70965088,68.3015098)(524.63965332,68.20151611)
\curveto(524.57965101,68.11150999)(524.52465107,68.01151009)(524.47465332,67.90151611)
\curveto(524.44465115,67.82151028)(524.41465118,67.73651037)(524.38465332,67.64651611)
\curveto(524.36465123,67.55651055)(524.31965127,67.48651062)(524.24965332,67.43651611)
\curveto(524.20965138,67.4065107)(524.13965145,67.38151072)(524.03965332,67.36151611)
\curveto(523.94965164,67.35151075)(523.85465174,67.34651076)(523.75465332,67.34651611)
\curveto(523.65465194,67.34651076)(523.55465204,67.35151075)(523.45465332,67.36151611)
\curveto(523.36465223,67.38151072)(523.29965229,67.4065107)(523.25965332,67.43651611)
\curveto(523.21965237,67.46651064)(523.1896524,67.51651059)(523.16965332,67.58651611)
\curveto(523.14965244,67.65651045)(523.14965244,67.73151037)(523.16965332,67.81151611)
\curveto(523.19965239,67.94151016)(523.22965236,68.06151004)(523.25965332,68.17151611)
\curveto(523.29965229,68.29150981)(523.34465225,68.4065097)(523.39465332,68.51651611)
\curveto(523.58465201,68.86650924)(523.82465177,69.13650897)(524.11465332,69.32651611)
\curveto(524.40465119,69.52650858)(524.76465083,69.68650842)(525.19465332,69.80651611)
\curveto(525.2946503,69.82650828)(525.3946502,69.84150826)(525.49465332,69.85151611)
\curveto(525.60464999,69.86150824)(525.71464988,69.87650823)(525.82465332,69.89651611)
\curveto(525.86464973,69.9065082)(525.92964966,69.9065082)(526.01965332,69.89651611)
\curveto(526.10964948,69.89650821)(526.16464943,69.9065082)(526.18465332,69.92651611)
\curveto(526.88464871,69.93650817)(527.4946481,69.85650825)(528.01465332,69.68651611)
\curveto(528.53464706,69.51650859)(528.89964669,69.19150891)(529.10965332,68.71151611)
\curveto(529.19964639,68.51150959)(529.24964634,68.27650983)(529.25965332,68.00651611)
\curveto(529.27964631,67.74651036)(529.2896463,67.47151063)(529.28965332,67.18151611)
\lineto(529.28965332,63.86651611)
\curveto(529.2896463,63.72651438)(529.2946463,63.59151451)(529.30465332,63.46151611)
\curveto(529.31464628,63.33151477)(529.34464625,63.22651488)(529.39465332,63.14651611)
\curveto(529.44464615,63.07651503)(529.50964608,63.02651508)(529.58965332,62.99651611)
\curveto(529.67964591,62.95651515)(529.76464583,62.92651518)(529.84465332,62.90651611)
\curveto(529.92464567,62.89651521)(529.98464561,62.85151525)(530.02465332,62.77151611)
\curveto(530.04464555,62.74151536)(530.05464554,62.71151539)(530.05465332,62.68151611)
\curveto(530.05464554,62.65151545)(530.05964553,62.61151549)(530.06965332,62.56151611)
\moveto(527.92465332,64.22651611)
\curveto(527.98464761,64.36651374)(528.01464758,64.52651358)(528.01465332,64.70651611)
\curveto(528.02464757,64.89651321)(528.02964756,65.09151301)(528.02965332,65.29151611)
\curveto(528.02964756,65.4015127)(528.02464757,65.5015126)(528.01465332,65.59151611)
\curveto(528.00464759,65.68151242)(527.96464763,65.75151235)(527.89465332,65.80151611)
\curveto(527.86464773,65.82151228)(527.7946478,65.83151227)(527.68465332,65.83151611)
\curveto(527.66464793,65.81151229)(527.62964796,65.8015123)(527.57965332,65.80151611)
\curveto(527.52964806,65.8015123)(527.48464811,65.79151231)(527.44465332,65.77151611)
\curveto(527.36464823,65.75151235)(527.27464832,65.73151237)(527.17465332,65.71151611)
\lineto(526.87465332,65.65151611)
\curveto(526.84464875,65.65151245)(526.80964878,65.64651246)(526.76965332,65.63651611)
\lineto(526.66465332,65.63651611)
\curveto(526.51464908,65.59651251)(526.34964924,65.57151253)(526.16965332,65.56151611)
\curveto(525.99964959,65.56151254)(525.83964975,65.54151256)(525.68965332,65.50151611)
\curveto(525.60964998,65.48151262)(525.53465006,65.46151264)(525.46465332,65.44151611)
\curveto(525.40465019,65.43151267)(525.33465026,65.41651269)(525.25465332,65.39651611)
\curveto(525.0946505,65.34651276)(524.94465065,65.28151282)(524.80465332,65.20151611)
\curveto(524.66465093,65.13151297)(524.54465105,65.04151306)(524.44465332,64.93151611)
\curveto(524.34465125,64.82151328)(524.26965132,64.68651342)(524.21965332,64.52651611)
\curveto(524.16965142,64.37651373)(524.14965144,64.19151391)(524.15965332,63.97151611)
\curveto(524.15965143,63.87151423)(524.17465142,63.77651433)(524.20465332,63.68651611)
\curveto(524.24465135,63.6065145)(524.2896513,63.53151457)(524.33965332,63.46151611)
\curveto(524.41965117,63.35151475)(524.52465107,63.25651485)(524.65465332,63.17651611)
\curveto(524.78465081,63.106515)(524.92465067,63.04651506)(525.07465332,62.99651611)
\curveto(525.12465047,62.98651512)(525.17465042,62.98151512)(525.22465332,62.98151611)
\curveto(525.27465032,62.98151512)(525.32465027,62.97651513)(525.37465332,62.96651611)
\curveto(525.44465015,62.94651516)(525.52965006,62.93151517)(525.62965332,62.92151611)
\curveto(525.73964985,62.92151518)(525.82964976,62.93151517)(525.89965332,62.95151611)
\curveto(525.95964963,62.97151513)(526.01964957,62.97651513)(526.07965332,62.96651611)
\curveto(526.13964945,62.96651514)(526.19964939,62.97651513)(526.25965332,62.99651611)
\curveto(526.33964925,63.01651509)(526.41464918,63.03151507)(526.48465332,63.04151611)
\curveto(526.56464903,63.05151505)(526.63964895,63.07151503)(526.70965332,63.10151611)
\curveto(526.99964859,63.22151488)(527.24464835,63.36651474)(527.44465332,63.53651611)
\curveto(527.65464794,63.7065144)(527.81464778,63.93651417)(527.92465332,64.22651611)
}
}
{
\newrgbcolor{curcolor}{0 0 0}
\pscustom[linestyle=none,fillstyle=solid,fillcolor=curcolor]
{
\newpath
\moveto(538.20129395,62.81651611)
\lineto(538.20129395,62.42651611)
\curveto(538.20128607,62.3065158)(538.1762861,62.2065159)(538.12629395,62.12651611)
\curveto(538.0762862,62.05651605)(537.99128628,62.01651609)(537.87129395,62.00651611)
\lineto(537.52629395,62.00651611)
\curveto(537.46628681,62.0065161)(537.40628687,62.0015161)(537.34629395,61.99151611)
\curveto(537.29628698,61.99151611)(537.25128702,62.0015161)(537.21129395,62.02151611)
\curveto(537.12128715,62.04151606)(537.06128721,62.08151602)(537.03129395,62.14151611)
\curveto(536.99128728,62.19151591)(536.96628731,62.25151585)(536.95629395,62.32151611)
\curveto(536.95628732,62.39151571)(536.94128733,62.46151564)(536.91129395,62.53151611)
\curveto(536.90128737,62.55151555)(536.88628739,62.56651554)(536.86629395,62.57651611)
\curveto(536.85628742,62.59651551)(536.84128743,62.61651549)(536.82129395,62.63651611)
\curveto(536.72128755,62.64651546)(536.64128763,62.62651548)(536.58129395,62.57651611)
\curveto(536.53128774,62.52651558)(536.4762878,62.47651563)(536.41629395,62.42651611)
\curveto(536.21628806,62.27651583)(536.01628826,62.16151594)(535.81629395,62.08151611)
\curveto(535.63628864,62.0015161)(535.42628885,61.94151616)(535.18629395,61.90151611)
\curveto(534.95628932,61.86151624)(534.71628956,61.84151626)(534.46629395,61.84151611)
\curveto(534.22629005,61.83151627)(533.98629029,61.84651626)(533.74629395,61.88651611)
\curveto(533.50629077,61.91651619)(533.29629098,61.97151613)(533.11629395,62.05151611)
\curveto(532.59629168,62.27151583)(532.1762921,62.56651554)(531.85629395,62.93651611)
\curveto(531.53629274,63.31651479)(531.28629299,63.78651432)(531.10629395,64.34651611)
\curveto(531.06629321,64.43651367)(531.03629324,64.52651358)(531.01629395,64.61651611)
\curveto(531.00629327,64.71651339)(530.98629329,64.81651329)(530.95629395,64.91651611)
\curveto(530.94629333,64.96651314)(530.94129333,65.01651309)(530.94129395,65.06651611)
\curveto(530.94129333,65.11651299)(530.93629334,65.16651294)(530.92629395,65.21651611)
\curveto(530.90629337,65.26651284)(530.89629338,65.31651279)(530.89629395,65.36651611)
\curveto(530.90629337,65.42651268)(530.90629337,65.48151262)(530.89629395,65.53151611)
\lineto(530.89629395,65.68151611)
\curveto(530.8762934,65.73151237)(530.86629341,65.79651231)(530.86629395,65.87651611)
\curveto(530.86629341,65.95651215)(530.8762934,66.02151208)(530.89629395,66.07151611)
\lineto(530.89629395,66.23651611)
\curveto(530.91629336,66.3065118)(530.92129335,66.37651173)(530.91129395,66.44651611)
\curveto(530.91129336,66.52651158)(530.92129335,66.6015115)(530.94129395,66.67151611)
\curveto(530.95129332,66.72151138)(530.95629332,66.76651134)(530.95629395,66.80651611)
\curveto(530.95629332,66.84651126)(530.96129331,66.89151121)(530.97129395,66.94151611)
\curveto(531.00129327,67.04151106)(531.02629325,67.13651097)(531.04629395,67.22651611)
\curveto(531.06629321,67.32651078)(531.09129318,67.42151068)(531.12129395,67.51151611)
\curveto(531.25129302,67.89151021)(531.41629286,68.23150987)(531.61629395,68.53151611)
\curveto(531.82629245,68.84150926)(532.0762922,69.09650901)(532.36629395,69.29651611)
\curveto(532.53629174,69.41650869)(532.71129156,69.51650859)(532.89129395,69.59651611)
\curveto(533.08129119,69.67650843)(533.28629099,69.74650836)(533.50629395,69.80651611)
\curveto(533.5762907,69.81650829)(533.64129063,69.82650828)(533.70129395,69.83651611)
\curveto(533.7712905,69.84650826)(533.84129043,69.86150824)(533.91129395,69.88151611)
\lineto(534.06129395,69.88151611)
\curveto(534.14129013,69.9015082)(534.25629002,69.91150819)(534.40629395,69.91151611)
\curveto(534.56628971,69.91150819)(534.68628959,69.9015082)(534.76629395,69.88151611)
\curveto(534.80628947,69.87150823)(534.86128941,69.86650824)(534.93129395,69.86651611)
\curveto(535.04128923,69.83650827)(535.15128912,69.81150829)(535.26129395,69.79151611)
\curveto(535.3712889,69.78150832)(535.4762888,69.75150835)(535.57629395,69.70151611)
\curveto(535.72628855,69.64150846)(535.86628841,69.57650853)(535.99629395,69.50651611)
\curveto(536.13628814,69.43650867)(536.26628801,69.35650875)(536.38629395,69.26651611)
\curveto(536.44628783,69.21650889)(536.50628777,69.16150894)(536.56629395,69.10151611)
\curveto(536.63628764,69.05150905)(536.72628755,69.03650907)(536.83629395,69.05651611)
\curveto(536.85628742,69.08650902)(536.8712874,69.11150899)(536.88129395,69.13151611)
\curveto(536.90128737,69.15150895)(536.91628736,69.18150892)(536.92629395,69.22151611)
\curveto(536.95628732,69.31150879)(536.96628731,69.42650868)(536.95629395,69.56651611)
\lineto(536.95629395,69.94151611)
\lineto(536.95629395,71.66651611)
\lineto(536.95629395,72.13151611)
\curveto(536.95628732,72.31150579)(536.98128729,72.44150566)(537.03129395,72.52151611)
\curveto(537.0712872,72.59150551)(537.13128714,72.63650547)(537.21129395,72.65651611)
\curveto(537.23128704,72.65650545)(537.25628702,72.65650545)(537.28629395,72.65651611)
\curveto(537.31628696,72.66650544)(537.34128693,72.67150543)(537.36129395,72.67151611)
\curveto(537.50128677,72.68150542)(537.64628663,72.68150542)(537.79629395,72.67151611)
\curveto(537.95628632,72.67150543)(538.06628621,72.63150547)(538.12629395,72.55151611)
\curveto(538.1762861,72.47150563)(538.20128607,72.37150573)(538.20129395,72.25151611)
\lineto(538.20129395,71.87651611)
\lineto(538.20129395,62.81651611)
\moveto(536.98629395,65.65151611)
\curveto(537.00628727,65.7015124)(537.01628726,65.76651234)(537.01629395,65.84651611)
\curveto(537.01628726,65.93651217)(537.00628727,66.0065121)(536.98629395,66.05651611)
\lineto(536.98629395,66.28151611)
\curveto(536.96628731,66.37151173)(536.95128732,66.46151164)(536.94129395,66.55151611)
\curveto(536.93128734,66.65151145)(536.91128736,66.74151136)(536.88129395,66.82151611)
\curveto(536.86128741,66.9015112)(536.84128743,66.97651113)(536.82129395,67.04651611)
\curveto(536.81128746,67.11651099)(536.79128748,67.18651092)(536.76129395,67.25651611)
\curveto(536.64128763,67.55651055)(536.48628779,67.82151028)(536.29629395,68.05151611)
\curveto(536.10628817,68.28150982)(535.86628841,68.46150964)(535.57629395,68.59151611)
\curveto(535.4762888,68.64150946)(535.3712889,68.67650943)(535.26129395,68.69651611)
\curveto(535.16128911,68.72650938)(535.05128922,68.75150935)(534.93129395,68.77151611)
\curveto(534.85128942,68.79150931)(534.76128951,68.8015093)(534.66129395,68.80151611)
\lineto(534.39129395,68.80151611)
\curveto(534.34128993,68.79150931)(534.29628998,68.78150932)(534.25629395,68.77151611)
\lineto(534.12129395,68.77151611)
\curveto(534.04129023,68.75150935)(533.95629032,68.73150937)(533.86629395,68.71151611)
\curveto(533.78629049,68.69150941)(533.70629057,68.66650944)(533.62629395,68.63651611)
\curveto(533.30629097,68.49650961)(533.04629123,68.29150981)(532.84629395,68.02151611)
\curveto(532.65629162,67.76151034)(532.50129177,67.45651065)(532.38129395,67.10651611)
\curveto(532.34129193,66.99651111)(532.31129196,66.88151122)(532.29129395,66.76151611)
\curveto(532.28129199,66.65151145)(532.26629201,66.54151156)(532.24629395,66.43151611)
\curveto(532.24629203,66.39151171)(532.24129203,66.35151175)(532.23129395,66.31151611)
\lineto(532.23129395,66.20651611)
\curveto(532.21129206,66.15651195)(532.20129207,66.101512)(532.20129395,66.04151611)
\curveto(532.21129206,65.98151212)(532.21629206,65.92651218)(532.21629395,65.87651611)
\lineto(532.21629395,65.54651611)
\curveto(532.21629206,65.44651266)(532.22629205,65.35151275)(532.24629395,65.26151611)
\curveto(532.25629202,65.23151287)(532.26129201,65.18151292)(532.26129395,65.11151611)
\curveto(532.28129199,65.04151306)(532.29629198,64.97151313)(532.30629395,64.90151611)
\lineto(532.36629395,64.69151611)
\curveto(532.4762918,64.34151376)(532.62629165,64.04151406)(532.81629395,63.79151611)
\curveto(533.00629127,63.54151456)(533.24629103,63.33651477)(533.53629395,63.17651611)
\curveto(533.62629065,63.12651498)(533.71629056,63.08651502)(533.80629395,63.05651611)
\curveto(533.89629038,63.02651508)(533.99629028,62.99651511)(534.10629395,62.96651611)
\curveto(534.15629012,62.94651516)(534.20629007,62.94151516)(534.25629395,62.95151611)
\curveto(534.31628996,62.96151514)(534.3712899,62.95651515)(534.42129395,62.93651611)
\curveto(534.46128981,62.92651518)(534.50128977,62.92151518)(534.54129395,62.92151611)
\lineto(534.67629395,62.92151611)
\lineto(534.81129395,62.92151611)
\curveto(534.84128943,62.93151517)(534.89128938,62.93651517)(534.96129395,62.93651611)
\curveto(535.04128923,62.95651515)(535.12128915,62.97151513)(535.20129395,62.98151611)
\curveto(535.28128899,63.0015151)(535.35628892,63.02651508)(535.42629395,63.05651611)
\curveto(535.75628852,63.19651491)(536.02128825,63.37151473)(536.22129395,63.58151611)
\curveto(536.43128784,63.8015143)(536.60628767,64.07651403)(536.74629395,64.40651611)
\curveto(536.79628748,64.51651359)(536.83128744,64.62651348)(536.85129395,64.73651611)
\curveto(536.8712874,64.84651326)(536.89628738,64.95651315)(536.92629395,65.06651611)
\curveto(536.94628733,65.106513)(536.95628732,65.14151296)(536.95629395,65.17151611)
\curveto(536.95628732,65.21151289)(536.96128731,65.25151285)(536.97129395,65.29151611)
\curveto(536.98128729,65.35151275)(536.98128729,65.41151269)(536.97129395,65.47151611)
\curveto(536.9712873,65.53151257)(536.9762873,65.59151251)(536.98629395,65.65151611)
}
}
{
\newrgbcolor{curcolor}{0 0 0}
\pscustom[linestyle=none,fillstyle=solid,fillcolor=curcolor]
{
\newpath
\moveto(547.27254395,66.20651611)
\curveto(547.29253589,66.14651196)(547.30253588,66.05151205)(547.30254395,65.92151611)
\curveto(547.30253588,65.8015123)(547.29753588,65.71651239)(547.28754395,65.66651611)
\lineto(547.28754395,65.51651611)
\curveto(547.2775359,65.43651267)(547.26753591,65.36151274)(547.25754395,65.29151611)
\curveto(547.25753592,65.23151287)(547.25253593,65.16151294)(547.24254395,65.08151611)
\curveto(547.22253596,65.02151308)(547.20753597,64.96151314)(547.19754395,64.90151611)
\curveto(547.19753598,64.84151326)(547.18753599,64.78151332)(547.16754395,64.72151611)
\curveto(547.12753605,64.59151351)(547.09253609,64.46151364)(547.06254395,64.33151611)
\curveto(547.03253615,64.2015139)(546.99253619,64.08151402)(546.94254395,63.97151611)
\curveto(546.73253645,63.49151461)(546.45253673,63.08651502)(546.10254395,62.75651611)
\curveto(545.75253743,62.43651567)(545.32253786,62.19151591)(544.81254395,62.02151611)
\curveto(544.70253848,61.98151612)(544.5825386,61.95151615)(544.45254395,61.93151611)
\curveto(544.33253885,61.91151619)(544.20753897,61.89151621)(544.07754395,61.87151611)
\curveto(544.01753916,61.86151624)(543.95253923,61.85651625)(543.88254395,61.85651611)
\curveto(543.82253936,61.84651626)(543.76253942,61.84151626)(543.70254395,61.84151611)
\curveto(543.66253952,61.83151627)(543.60253958,61.82651628)(543.52254395,61.82651611)
\curveto(543.45253973,61.82651628)(543.40253978,61.83151627)(543.37254395,61.84151611)
\curveto(543.33253985,61.85151625)(543.29253989,61.85651625)(543.25254395,61.85651611)
\curveto(543.21253997,61.84651626)(543.17754,61.84651626)(543.14754395,61.85651611)
\lineto(543.05754395,61.85651611)
\lineto(542.69754395,61.90151611)
\curveto(542.55754062,61.94151616)(542.42254076,61.98151612)(542.29254395,62.02151611)
\curveto(542.16254102,62.06151604)(542.03754114,62.106516)(541.91754395,62.15651611)
\curveto(541.46754171,62.35651575)(541.09754208,62.61651549)(540.80754395,62.93651611)
\curveto(540.51754266,63.25651485)(540.2775429,63.64651446)(540.08754395,64.10651611)
\curveto(540.03754314,64.2065139)(539.99754318,64.3065138)(539.96754395,64.40651611)
\curveto(539.94754323,64.5065136)(539.92754325,64.61151349)(539.90754395,64.72151611)
\curveto(539.88754329,64.76151334)(539.8775433,64.79151331)(539.87754395,64.81151611)
\curveto(539.88754329,64.84151326)(539.88754329,64.87651323)(539.87754395,64.91651611)
\curveto(539.85754332,64.99651311)(539.84254334,65.07651303)(539.83254395,65.15651611)
\curveto(539.83254335,65.24651286)(539.82254336,65.33151277)(539.80254395,65.41151611)
\lineto(539.80254395,65.53151611)
\curveto(539.80254338,65.57151253)(539.79754338,65.61651249)(539.78754395,65.66651611)
\curveto(539.7775434,65.71651239)(539.77254341,65.8015123)(539.77254395,65.92151611)
\curveto(539.77254341,66.05151205)(539.7825434,66.14651196)(539.80254395,66.20651611)
\curveto(539.82254336,66.27651183)(539.82754335,66.34651176)(539.81754395,66.41651611)
\curveto(539.80754337,66.48651162)(539.81254337,66.55651155)(539.83254395,66.62651611)
\curveto(539.84254334,66.67651143)(539.84754333,66.71651139)(539.84754395,66.74651611)
\curveto(539.85754332,66.78651132)(539.86754331,66.83151127)(539.87754395,66.88151611)
\curveto(539.90754327,67.0015111)(539.93254325,67.12151098)(539.95254395,67.24151611)
\curveto(539.9825432,67.36151074)(540.02254316,67.47651063)(540.07254395,67.58651611)
\curveto(540.22254296,67.95651015)(540.40254278,68.28650982)(540.61254395,68.57651611)
\curveto(540.83254235,68.87650923)(541.09754208,69.12650898)(541.40754395,69.32651611)
\curveto(541.52754165,69.4065087)(541.65254153,69.47150863)(541.78254395,69.52151611)
\curveto(541.91254127,69.58150852)(542.04754113,69.64150846)(542.18754395,69.70151611)
\curveto(542.30754087,69.75150835)(542.43754074,69.78150832)(542.57754395,69.79151611)
\curveto(542.71754046,69.81150829)(542.85754032,69.84150826)(542.99754395,69.88151611)
\lineto(543.19254395,69.88151611)
\curveto(543.26253992,69.89150821)(543.32753985,69.9015082)(543.38754395,69.91151611)
\curveto(544.2775389,69.92150818)(545.01753816,69.73650837)(545.60754395,69.35651611)
\curveto(546.19753698,68.97650913)(546.62253656,68.48150962)(546.88254395,67.87151611)
\curveto(546.93253625,67.77151033)(546.97253621,67.67151043)(547.00254395,67.57151611)
\curveto(547.03253615,67.47151063)(547.06753611,67.36651074)(547.10754395,67.25651611)
\curveto(547.13753604,67.14651096)(547.16253602,67.02651108)(547.18254395,66.89651611)
\curveto(547.20253598,66.77651133)(547.22753595,66.65151145)(547.25754395,66.52151611)
\curveto(547.26753591,66.47151163)(547.26753591,66.41651169)(547.25754395,66.35651611)
\curveto(547.25753592,66.3065118)(547.26253592,66.25651185)(547.27254395,66.20651611)
\moveto(545.93754395,65.35151611)
\curveto(545.95753722,65.42151268)(545.96253722,65.5015126)(545.95254395,65.59151611)
\lineto(545.95254395,65.84651611)
\curveto(545.95253723,66.23651187)(545.91753726,66.56651154)(545.84754395,66.83651611)
\curveto(545.81753736,66.91651119)(545.79253739,66.99651111)(545.77254395,67.07651611)
\curveto(545.75253743,67.15651095)(545.72753745,67.23151087)(545.69754395,67.30151611)
\curveto(545.41753776,67.95151015)(544.97253821,68.4015097)(544.36254395,68.65151611)
\curveto(544.29253889,68.68150942)(544.21753896,68.7015094)(544.13754395,68.71151611)
\lineto(543.89754395,68.77151611)
\curveto(543.81753936,68.79150931)(543.73253945,68.8015093)(543.64254395,68.80151611)
\lineto(543.37254395,68.80151611)
\lineto(543.10254395,68.75651611)
\curveto(543.00254018,68.73650937)(542.90754027,68.71150939)(542.81754395,68.68151611)
\curveto(542.73754044,68.66150944)(542.65754052,68.63150947)(542.57754395,68.59151611)
\curveto(542.50754067,68.57150953)(542.44254074,68.54150956)(542.38254395,68.50151611)
\curveto(542.32254086,68.46150964)(542.26754091,68.42150968)(542.21754395,68.38151611)
\curveto(541.9775412,68.21150989)(541.7825414,68.0065101)(541.63254395,67.76651611)
\curveto(541.4825417,67.52651058)(541.35254183,67.24651086)(541.24254395,66.92651611)
\curveto(541.21254197,66.82651128)(541.19254199,66.72151138)(541.18254395,66.61151611)
\curveto(541.17254201,66.51151159)(541.15754202,66.4065117)(541.13754395,66.29651611)
\curveto(541.12754205,66.25651185)(541.12254206,66.19151191)(541.12254395,66.10151611)
\curveto(541.11254207,66.07151203)(541.10754207,66.03651207)(541.10754395,65.99651611)
\curveto(541.11754206,65.95651215)(541.12254206,65.91151219)(541.12254395,65.86151611)
\lineto(541.12254395,65.56151611)
\curveto(541.12254206,65.46151264)(541.13254205,65.37151273)(541.15254395,65.29151611)
\lineto(541.18254395,65.11151611)
\curveto(541.20254198,65.01151309)(541.21754196,64.91151319)(541.22754395,64.81151611)
\curveto(541.24754193,64.72151338)(541.2775419,64.63651347)(541.31754395,64.55651611)
\curveto(541.41754176,64.31651379)(541.53254165,64.09151401)(541.66254395,63.88151611)
\curveto(541.80254138,63.67151443)(541.97254121,63.49651461)(542.17254395,63.35651611)
\curveto(542.22254096,63.32651478)(542.26754091,63.3015148)(542.30754395,63.28151611)
\curveto(542.34754083,63.26151484)(542.39254079,63.23651487)(542.44254395,63.20651611)
\curveto(542.52254066,63.15651495)(542.60754057,63.11151499)(542.69754395,63.07151611)
\curveto(542.79754038,63.04151506)(542.90254028,63.01151509)(543.01254395,62.98151611)
\curveto(543.06254012,62.96151514)(543.10754007,62.95151515)(543.14754395,62.95151611)
\curveto(543.19753998,62.96151514)(543.24753993,62.96151514)(543.29754395,62.95151611)
\curveto(543.32753985,62.94151516)(543.38753979,62.93151517)(543.47754395,62.92151611)
\curveto(543.5775396,62.91151519)(543.65253953,62.91651519)(543.70254395,62.93651611)
\curveto(543.74253944,62.94651516)(543.7825394,62.94651516)(543.82254395,62.93651611)
\curveto(543.86253932,62.93651517)(543.90253928,62.94651516)(543.94254395,62.96651611)
\curveto(544.02253916,62.98651512)(544.10253908,63.0015151)(544.18254395,63.01151611)
\curveto(544.26253892,63.03151507)(544.33753884,63.05651505)(544.40754395,63.08651611)
\curveto(544.74753843,63.22651488)(545.02253816,63.42151468)(545.23254395,63.67151611)
\curveto(545.44253774,63.92151418)(545.61753756,64.21651389)(545.75754395,64.55651611)
\curveto(545.80753737,64.67651343)(545.83753734,64.8015133)(545.84754395,64.93151611)
\curveto(545.86753731,65.07151303)(545.89753728,65.21151289)(545.93754395,65.35151611)
}
}
{
\newrgbcolor{curcolor}{0 0 0}
\pscustom[linestyle=none,fillstyle=solid,fillcolor=curcolor]
{
\newpath
\moveto(552.4058252,69.91151611)
\curveto(552.63582041,69.91150819)(552.76582028,69.85150825)(552.7958252,69.73151611)
\curveto(552.82582022,69.62150848)(552.8408202,69.45650865)(552.8408252,69.23651611)
\lineto(552.8408252,68.95151611)
\curveto(552.8408202,68.86150924)(552.81582023,68.78650932)(552.7658252,68.72651611)
\curveto(552.70582034,68.64650946)(552.62082042,68.6015095)(552.5108252,68.59151611)
\curveto(552.40082064,68.59150951)(552.29082075,68.57650953)(552.1808252,68.54651611)
\curveto(552.040821,68.51650959)(551.90582114,68.48650962)(551.7758252,68.45651611)
\curveto(551.65582139,68.42650968)(551.5408215,68.38650972)(551.4308252,68.33651611)
\curveto(551.1408219,68.2065099)(550.90582214,68.02651008)(550.7258252,67.79651611)
\curveto(550.5458225,67.57651053)(550.39082265,67.32151078)(550.2608252,67.03151611)
\curveto(550.22082282,66.92151118)(550.19082285,66.8065113)(550.1708252,66.68651611)
\curveto(550.15082289,66.57651153)(550.12582292,66.46151164)(550.0958252,66.34151611)
\curveto(550.08582296,66.29151181)(550.08082296,66.24151186)(550.0808252,66.19151611)
\curveto(550.09082295,66.14151196)(550.09082295,66.09151201)(550.0808252,66.04151611)
\curveto(550.05082299,65.92151218)(550.03582301,65.78151232)(550.0358252,65.62151611)
\curveto(550.045823,65.47151263)(550.05082299,65.32651278)(550.0508252,65.18651611)
\lineto(550.0508252,63.34151611)
\lineto(550.0508252,62.99651611)
\curveto(550.05082299,62.87651523)(550.045823,62.76151534)(550.0358252,62.65151611)
\curveto(550.02582302,62.54151556)(550.02082302,62.44651566)(550.0208252,62.36651611)
\curveto(550.03082301,62.28651582)(550.01082303,62.21651589)(549.9608252,62.15651611)
\curveto(549.91082313,62.08651602)(549.83082321,62.04651606)(549.7208252,62.03651611)
\curveto(549.62082342,62.02651608)(549.51082353,62.02151608)(549.3908252,62.02151611)
\lineto(549.1208252,62.02151611)
\curveto(549.07082397,62.04151606)(549.02082402,62.05651605)(548.9708252,62.06651611)
\curveto(548.93082411,62.08651602)(548.90082414,62.11151599)(548.8808252,62.14151611)
\curveto(548.83082421,62.21151589)(548.80082424,62.29651581)(548.7908252,62.39651611)
\lineto(548.7908252,62.72651611)
\lineto(548.7908252,63.88151611)
\lineto(548.7908252,68.03651611)
\lineto(548.7908252,69.07151611)
\lineto(548.7908252,69.37151611)
\curveto(548.80082424,69.47150863)(548.83082421,69.55650855)(548.8808252,69.62651611)
\curveto(548.91082413,69.66650844)(548.96082408,69.69650841)(549.0308252,69.71651611)
\curveto(549.11082393,69.73650837)(549.19582385,69.74650836)(549.2858252,69.74651611)
\curveto(549.37582367,69.75650835)(549.46582358,69.75650835)(549.5558252,69.74651611)
\curveto(549.6458234,69.73650837)(549.71582333,69.72150838)(549.7658252,69.70151611)
\curveto(549.8458232,69.67150843)(549.89582315,69.61150849)(549.9158252,69.52151611)
\curveto(549.9458231,69.44150866)(549.96082308,69.35150875)(549.9608252,69.25151611)
\lineto(549.9608252,68.95151611)
\curveto(549.96082308,68.85150925)(549.98082306,68.76150934)(550.0208252,68.68151611)
\curveto(550.03082301,68.66150944)(550.040823,68.64650946)(550.0508252,68.63651611)
\lineto(550.0958252,68.59151611)
\curveto(550.20582284,68.59150951)(550.29582275,68.63650947)(550.3658252,68.72651611)
\curveto(550.43582261,68.82650928)(550.49582255,68.9065092)(550.5458252,68.96651611)
\lineto(550.6358252,69.05651611)
\curveto(550.72582232,69.16650894)(550.85082219,69.28150882)(551.0108252,69.40151611)
\curveto(551.17082187,69.52150858)(551.32082172,69.61150849)(551.4608252,69.67151611)
\curveto(551.55082149,69.72150838)(551.6458214,69.75650835)(551.7458252,69.77651611)
\curveto(551.8458212,69.8065083)(551.95082109,69.83650827)(552.0608252,69.86651611)
\curveto(552.12082092,69.87650823)(552.18082086,69.88150822)(552.2408252,69.88151611)
\curveto(552.30082074,69.89150821)(552.35582069,69.9015082)(552.4058252,69.91151611)
}
}
{
\newrgbcolor{curcolor}{0 0 0}
\pscustom[linestyle=none,fillstyle=solid,fillcolor=curcolor]
{
\newpath
\moveto(662.2480835,62.78651611)
\curveto(662.26807395,62.73651537)(662.29307393,62.67651543)(662.3230835,62.60651611)
\curveto(662.35307387,62.53651557)(662.37307385,62.46151564)(662.3830835,62.38151611)
\curveto(662.40307382,62.31151579)(662.40307382,62.24151586)(662.3830835,62.17151611)
\curveto(662.37307385,62.11151599)(662.33307389,62.06651604)(662.2630835,62.03651611)
\curveto(662.21307401,62.01651609)(662.15307407,62.0065161)(662.0830835,62.00651611)
\lineto(661.8730835,62.00651611)
\lineto(661.4230835,62.00651611)
\curveto(661.27307495,62.0065161)(661.15307507,62.03151607)(661.0630835,62.08151611)
\curveto(660.96307526,62.14151596)(660.88807533,62.24651586)(660.8380835,62.39651611)
\curveto(660.79807542,62.54651556)(660.75307547,62.68151542)(660.7030835,62.80151611)
\curveto(660.59307563,63.06151504)(660.49307573,63.33151477)(660.4030835,63.61151611)
\curveto(660.31307591,63.89151421)(660.21307601,64.16651394)(660.1030835,64.43651611)
\curveto(660.07307615,64.52651358)(660.04307618,64.61151349)(660.0130835,64.69151611)
\curveto(659.99307623,64.77151333)(659.96307626,64.84651326)(659.9230835,64.91651611)
\curveto(659.89307633,64.98651312)(659.84807637,65.04651306)(659.7880835,65.09651611)
\curveto(659.72807649,65.14651296)(659.64807657,65.18651292)(659.5480835,65.21651611)
\curveto(659.49807672,65.23651287)(659.43807678,65.24151286)(659.3680835,65.23151611)
\lineto(659.1730835,65.23151611)
\lineto(656.3380835,65.23151611)
\lineto(656.0380835,65.23151611)
\curveto(655.92808029,65.24151286)(655.8230804,65.24151286)(655.7230835,65.23151611)
\curveto(655.6230806,65.22151288)(655.52808069,65.2065129)(655.4380835,65.18651611)
\curveto(655.35808086,65.16651294)(655.29808092,65.12651298)(655.2580835,65.06651611)
\curveto(655.17808104,64.96651314)(655.1180811,64.85151325)(655.0780835,64.72151611)
\curveto(655.04808117,64.6015135)(655.00808121,64.47651363)(654.9580835,64.34651611)
\curveto(654.85808136,64.11651399)(654.76308146,63.87651423)(654.6730835,63.62651611)
\curveto(654.59308163,63.37651473)(654.50308172,63.13651497)(654.4030835,62.90651611)
\curveto(654.38308184,62.84651526)(654.35808186,62.77651533)(654.3280835,62.69651611)
\curveto(654.30808191,62.62651548)(654.28308194,62.55151555)(654.2530835,62.47151611)
\curveto(654.223082,62.39151571)(654.18808203,62.31651579)(654.1480835,62.24651611)
\curveto(654.1180821,62.18651592)(654.08308214,62.14151596)(654.0430835,62.11151611)
\curveto(653.96308226,62.05151605)(653.85308237,62.01651609)(653.7130835,62.00651611)
\lineto(653.2930835,62.00651611)
\lineto(653.0530835,62.00651611)
\curveto(652.98308324,62.01651609)(652.9230833,62.04151606)(652.8730835,62.08151611)
\curveto(652.8230834,62.11151599)(652.79308343,62.15651595)(652.7830835,62.21651611)
\curveto(652.78308344,62.27651583)(652.78808343,62.33651577)(652.7980835,62.39651611)
\curveto(652.8180834,62.46651564)(652.83808338,62.53151557)(652.8580835,62.59151611)
\curveto(652.88808333,62.66151544)(652.91308331,62.71151539)(652.9330835,62.74151611)
\curveto(653.07308315,63.06151504)(653.19808302,63.37651473)(653.3080835,63.68651611)
\curveto(653.4180828,64.0065141)(653.53808268,64.32651378)(653.6680835,64.64651611)
\curveto(653.75808246,64.86651324)(653.84308238,65.08151302)(653.9230835,65.29151611)
\curveto(654.00308222,65.51151259)(654.08808213,65.73151237)(654.1780835,65.95151611)
\curveto(654.47808174,66.67151143)(654.76308146,67.39651071)(655.0330835,68.12651611)
\curveto(655.30308092,68.86650924)(655.58808063,69.6015085)(655.8880835,70.33151611)
\curveto(655.99808022,70.59150751)(656.09808012,70.85650725)(656.1880835,71.12651611)
\curveto(656.28807993,71.39650671)(656.39307983,71.66150644)(656.5030835,71.92151611)
\curveto(656.55307967,72.03150607)(656.59807962,72.15150595)(656.6380835,72.28151611)
\curveto(656.68807953,72.42150568)(656.75807946,72.52150558)(656.8480835,72.58151611)
\curveto(656.88807933,72.62150548)(656.95307927,72.65150545)(657.0430835,72.67151611)
\curveto(657.06307916,72.68150542)(657.08307914,72.68150542)(657.1030835,72.67151611)
\curveto(657.13307909,72.67150543)(657.15807906,72.67650543)(657.1780835,72.68651611)
\curveto(657.35807886,72.68650542)(657.56807865,72.68650542)(657.8080835,72.68651611)
\curveto(658.04807817,72.69650541)(658.223078,72.66150544)(658.3330835,72.58151611)
\curveto(658.41307781,72.52150558)(658.47307775,72.42150568)(658.5130835,72.28151611)
\curveto(658.56307766,72.15150595)(658.61307761,72.03150607)(658.6630835,71.92151611)
\curveto(658.76307746,71.69150641)(658.85307737,71.46150664)(658.9330835,71.23151611)
\curveto(659.01307721,71.0015071)(659.10307712,70.77150733)(659.2030835,70.54151611)
\curveto(659.28307694,70.34150776)(659.35807686,70.13650797)(659.4280835,69.92651611)
\curveto(659.50807671,69.71650839)(659.59307663,69.51150859)(659.6830835,69.31151611)
\curveto(659.98307624,68.58150952)(660.26807595,67.84151026)(660.5380835,67.09151611)
\curveto(660.8180754,66.35151175)(661.11307511,65.61651249)(661.4230835,64.88651611)
\curveto(661.46307476,64.79651331)(661.49307473,64.71151339)(661.5130835,64.63151611)
\curveto(661.54307468,64.55151355)(661.57307465,64.46651364)(661.6030835,64.37651611)
\curveto(661.71307451,64.11651399)(661.8180744,63.85151425)(661.9180835,63.58151611)
\curveto(662.02807419,63.31151479)(662.13807408,63.04651506)(662.2480835,62.78651611)
\moveto(659.0380835,66.43151611)
\curveto(659.12807709,66.46151164)(659.18307704,66.51151159)(659.2030835,66.58151611)
\curveto(659.23307699,66.65151145)(659.23807698,66.72651138)(659.2180835,66.80651611)
\curveto(659.20807701,66.89651121)(659.18307704,66.98151112)(659.1430835,67.06151611)
\curveto(659.11307711,67.15151095)(659.08307714,67.22651088)(659.0530835,67.28651611)
\curveto(659.03307719,67.32651078)(659.0230772,67.36151074)(659.0230835,67.39151611)
\curveto(659.0230772,67.42151068)(659.01307721,67.45651065)(658.9930835,67.49651611)
\lineto(658.9030835,67.73651611)
\curveto(658.88307734,67.82651028)(658.85307737,67.91651019)(658.8130835,68.00651611)
\curveto(658.66307756,68.36650974)(658.52807769,68.73150937)(658.4080835,69.10151611)
\curveto(658.29807792,69.48150862)(658.16807805,69.85150825)(658.0180835,70.21151611)
\curveto(657.96807825,70.32150778)(657.9230783,70.43150767)(657.8830835,70.54151611)
\curveto(657.85307837,70.65150745)(657.81307841,70.75650735)(657.7630835,70.85651611)
\curveto(657.74307848,70.9065072)(657.7180785,70.95150715)(657.6880835,70.99151611)
\curveto(657.66807855,71.04150706)(657.6180786,71.06650704)(657.5380835,71.06651611)
\curveto(657.5180787,71.04650706)(657.49807872,71.03150707)(657.4780835,71.02151611)
\curveto(657.45807876,71.01150709)(657.43807878,70.99650711)(657.4180835,70.97651611)
\curveto(657.37807884,70.92650718)(657.34807887,70.87150723)(657.3280835,70.81151611)
\curveto(657.30807891,70.76150734)(657.28807893,70.7065074)(657.2680835,70.64651611)
\curveto(657.218079,70.53650757)(657.17807904,70.42650768)(657.1480835,70.31651611)
\curveto(657.1180791,70.2065079)(657.07807914,70.09650801)(657.0280835,69.98651611)
\curveto(656.85807936,69.59650851)(656.70807951,69.2015089)(656.5780835,68.80151611)
\curveto(656.45807976,68.4015097)(656.3180799,68.01151009)(656.1580835,67.63151611)
\lineto(656.0980835,67.48151611)
\curveto(656.08808013,67.43151067)(656.07308015,67.38151072)(656.0530835,67.33151611)
\lineto(655.9630835,67.09151611)
\curveto(655.93308029,67.01151109)(655.90808031,66.93151117)(655.8880835,66.85151611)
\curveto(655.86808035,66.8015113)(655.85808036,66.74651136)(655.8580835,66.68651611)
\curveto(655.86808035,66.62651148)(655.88308034,66.57651153)(655.9030835,66.53651611)
\curveto(655.95308027,66.45651165)(656.05808016,66.41151169)(656.2180835,66.40151611)
\lineto(656.6680835,66.40151611)
\lineto(658.2730835,66.40151611)
\curveto(658.38307784,66.4015117)(658.5180777,66.39651171)(658.6780835,66.38651611)
\curveto(658.83807738,66.38651172)(658.95807726,66.4015117)(659.0380835,66.43151611)
}
}
{
\newrgbcolor{curcolor}{0 0 0}
\pscustom[linestyle=none,fillstyle=solid,fillcolor=curcolor]
{
\newpath
\moveto(670.324646,62.81651611)
\lineto(670.324646,62.42651611)
\curveto(670.32463812,62.3065158)(670.29963815,62.2065159)(670.249646,62.12651611)
\curveto(670.19963825,62.05651605)(670.11463833,62.01651609)(669.994646,62.00651611)
\lineto(669.649646,62.00651611)
\curveto(669.58963886,62.0065161)(669.52963892,62.0015161)(669.469646,61.99151611)
\curveto(669.41963903,61.99151611)(669.37463907,62.0015161)(669.334646,62.02151611)
\curveto(669.2446392,62.04151606)(669.18463926,62.08151602)(669.154646,62.14151611)
\curveto(669.11463933,62.19151591)(669.08963936,62.25151585)(669.079646,62.32151611)
\curveto(669.07963937,62.39151571)(669.06463938,62.46151564)(669.034646,62.53151611)
\curveto(669.02463942,62.55151555)(669.00963944,62.56651554)(668.989646,62.57651611)
\curveto(668.97963947,62.59651551)(668.96463948,62.61651549)(668.944646,62.63651611)
\curveto(668.8446396,62.64651546)(668.76463968,62.62651548)(668.704646,62.57651611)
\curveto(668.65463979,62.52651558)(668.59963985,62.47651563)(668.539646,62.42651611)
\curveto(668.33964011,62.27651583)(668.13964031,62.16151594)(667.939646,62.08151611)
\curveto(667.75964069,62.0015161)(667.5496409,61.94151616)(667.309646,61.90151611)
\curveto(667.07964137,61.86151624)(666.83964161,61.84151626)(666.589646,61.84151611)
\curveto(666.3496421,61.83151627)(666.10964234,61.84651626)(665.869646,61.88651611)
\curveto(665.62964282,61.91651619)(665.41964303,61.97151613)(665.239646,62.05151611)
\curveto(664.71964373,62.27151583)(664.29964415,62.56651554)(663.979646,62.93651611)
\curveto(663.65964479,63.31651479)(663.40964504,63.78651432)(663.229646,64.34651611)
\curveto(663.18964526,64.43651367)(663.15964529,64.52651358)(663.139646,64.61651611)
\curveto(663.12964532,64.71651339)(663.10964534,64.81651329)(663.079646,64.91651611)
\curveto(663.06964538,64.96651314)(663.06464538,65.01651309)(663.064646,65.06651611)
\curveto(663.06464538,65.11651299)(663.05964539,65.16651294)(663.049646,65.21651611)
\curveto(663.02964542,65.26651284)(663.01964543,65.31651279)(663.019646,65.36651611)
\curveto(663.02964542,65.42651268)(663.02964542,65.48151262)(663.019646,65.53151611)
\lineto(663.019646,65.68151611)
\curveto(662.99964545,65.73151237)(662.98964546,65.79651231)(662.989646,65.87651611)
\curveto(662.98964546,65.95651215)(662.99964545,66.02151208)(663.019646,66.07151611)
\lineto(663.019646,66.23651611)
\curveto(663.03964541,66.3065118)(663.0446454,66.37651173)(663.034646,66.44651611)
\curveto(663.03464541,66.52651158)(663.0446454,66.6015115)(663.064646,66.67151611)
\curveto(663.07464537,66.72151138)(663.07964537,66.76651134)(663.079646,66.80651611)
\curveto(663.07964537,66.84651126)(663.08464536,66.89151121)(663.094646,66.94151611)
\curveto(663.12464532,67.04151106)(663.1496453,67.13651097)(663.169646,67.22651611)
\curveto(663.18964526,67.32651078)(663.21464523,67.42151068)(663.244646,67.51151611)
\curveto(663.37464507,67.89151021)(663.53964491,68.23150987)(663.739646,68.53151611)
\curveto(663.9496445,68.84150926)(664.19964425,69.09650901)(664.489646,69.29651611)
\curveto(664.65964379,69.41650869)(664.83464361,69.51650859)(665.014646,69.59651611)
\curveto(665.20464324,69.67650843)(665.40964304,69.74650836)(665.629646,69.80651611)
\curveto(665.69964275,69.81650829)(665.76464268,69.82650828)(665.824646,69.83651611)
\curveto(665.89464255,69.84650826)(665.96464248,69.86150824)(666.034646,69.88151611)
\lineto(666.184646,69.88151611)
\curveto(666.26464218,69.9015082)(666.37964207,69.91150819)(666.529646,69.91151611)
\curveto(666.68964176,69.91150819)(666.80964164,69.9015082)(666.889646,69.88151611)
\curveto(666.92964152,69.87150823)(666.98464146,69.86650824)(667.054646,69.86651611)
\curveto(667.16464128,69.83650827)(667.27464117,69.81150829)(667.384646,69.79151611)
\curveto(667.49464095,69.78150832)(667.59964085,69.75150835)(667.699646,69.70151611)
\curveto(667.8496406,69.64150846)(667.98964046,69.57650853)(668.119646,69.50651611)
\curveto(668.25964019,69.43650867)(668.38964006,69.35650875)(668.509646,69.26651611)
\curveto(668.56963988,69.21650889)(668.62963982,69.16150894)(668.689646,69.10151611)
\curveto(668.75963969,69.05150905)(668.8496396,69.03650907)(668.959646,69.05651611)
\curveto(668.97963947,69.08650902)(668.99463945,69.11150899)(669.004646,69.13151611)
\curveto(669.02463942,69.15150895)(669.03963941,69.18150892)(669.049646,69.22151611)
\curveto(669.07963937,69.31150879)(669.08963936,69.42650868)(669.079646,69.56651611)
\lineto(669.079646,69.94151611)
\lineto(669.079646,71.66651611)
\lineto(669.079646,72.13151611)
\curveto(669.07963937,72.31150579)(669.10463934,72.44150566)(669.154646,72.52151611)
\curveto(669.19463925,72.59150551)(669.25463919,72.63650547)(669.334646,72.65651611)
\curveto(669.35463909,72.65650545)(669.37963907,72.65650545)(669.409646,72.65651611)
\curveto(669.43963901,72.66650544)(669.46463898,72.67150543)(669.484646,72.67151611)
\curveto(669.62463882,72.68150542)(669.76963868,72.68150542)(669.919646,72.67151611)
\curveto(670.07963837,72.67150543)(670.18963826,72.63150547)(670.249646,72.55151611)
\curveto(670.29963815,72.47150563)(670.32463812,72.37150573)(670.324646,72.25151611)
\lineto(670.324646,71.87651611)
\lineto(670.324646,62.81651611)
\moveto(669.109646,65.65151611)
\curveto(669.12963932,65.7015124)(669.13963931,65.76651234)(669.139646,65.84651611)
\curveto(669.13963931,65.93651217)(669.12963932,66.0065121)(669.109646,66.05651611)
\lineto(669.109646,66.28151611)
\curveto(669.08963936,66.37151173)(669.07463937,66.46151164)(669.064646,66.55151611)
\curveto(669.05463939,66.65151145)(669.03463941,66.74151136)(669.004646,66.82151611)
\curveto(668.98463946,66.9015112)(668.96463948,66.97651113)(668.944646,67.04651611)
\curveto(668.93463951,67.11651099)(668.91463953,67.18651092)(668.884646,67.25651611)
\curveto(668.76463968,67.55651055)(668.60963984,67.82151028)(668.419646,68.05151611)
\curveto(668.22964022,68.28150982)(667.98964046,68.46150964)(667.699646,68.59151611)
\curveto(667.59964085,68.64150946)(667.49464095,68.67650943)(667.384646,68.69651611)
\curveto(667.28464116,68.72650938)(667.17464127,68.75150935)(667.054646,68.77151611)
\curveto(666.97464147,68.79150931)(666.88464156,68.8015093)(666.784646,68.80151611)
\lineto(666.514646,68.80151611)
\curveto(666.46464198,68.79150931)(666.41964203,68.78150932)(666.379646,68.77151611)
\lineto(666.244646,68.77151611)
\curveto(666.16464228,68.75150935)(666.07964237,68.73150937)(665.989646,68.71151611)
\curveto(665.90964254,68.69150941)(665.82964262,68.66650944)(665.749646,68.63651611)
\curveto(665.42964302,68.49650961)(665.16964328,68.29150981)(664.969646,68.02151611)
\curveto(664.77964367,67.76151034)(664.62464382,67.45651065)(664.504646,67.10651611)
\curveto(664.46464398,66.99651111)(664.43464401,66.88151122)(664.414646,66.76151611)
\curveto(664.40464404,66.65151145)(664.38964406,66.54151156)(664.369646,66.43151611)
\curveto(664.36964408,66.39151171)(664.36464408,66.35151175)(664.354646,66.31151611)
\lineto(664.354646,66.20651611)
\curveto(664.33464411,66.15651195)(664.32464412,66.101512)(664.324646,66.04151611)
\curveto(664.33464411,65.98151212)(664.33964411,65.92651218)(664.339646,65.87651611)
\lineto(664.339646,65.54651611)
\curveto(664.33964411,65.44651266)(664.3496441,65.35151275)(664.369646,65.26151611)
\curveto(664.37964407,65.23151287)(664.38464406,65.18151292)(664.384646,65.11151611)
\curveto(664.40464404,65.04151306)(664.41964403,64.97151313)(664.429646,64.90151611)
\lineto(664.489646,64.69151611)
\curveto(664.59964385,64.34151376)(664.7496437,64.04151406)(664.939646,63.79151611)
\curveto(665.12964332,63.54151456)(665.36964308,63.33651477)(665.659646,63.17651611)
\curveto(665.7496427,63.12651498)(665.83964261,63.08651502)(665.929646,63.05651611)
\curveto(666.01964243,63.02651508)(666.11964233,62.99651511)(666.229646,62.96651611)
\curveto(666.27964217,62.94651516)(666.32964212,62.94151516)(666.379646,62.95151611)
\curveto(666.43964201,62.96151514)(666.49464195,62.95651515)(666.544646,62.93651611)
\curveto(666.58464186,62.92651518)(666.62464182,62.92151518)(666.664646,62.92151611)
\lineto(666.799646,62.92151611)
\lineto(666.934646,62.92151611)
\curveto(666.96464148,62.93151517)(667.01464143,62.93651517)(667.084646,62.93651611)
\curveto(667.16464128,62.95651515)(667.2446412,62.97151513)(667.324646,62.98151611)
\curveto(667.40464104,63.0015151)(667.47964097,63.02651508)(667.549646,63.05651611)
\curveto(667.87964057,63.19651491)(668.1446403,63.37151473)(668.344646,63.58151611)
\curveto(668.55463989,63.8015143)(668.72963972,64.07651403)(668.869646,64.40651611)
\curveto(668.91963953,64.51651359)(668.95463949,64.62651348)(668.974646,64.73651611)
\curveto(668.99463945,64.84651326)(669.01963943,64.95651315)(669.049646,65.06651611)
\curveto(669.06963938,65.106513)(669.07963937,65.14151296)(669.079646,65.17151611)
\curveto(669.07963937,65.21151289)(669.08463936,65.25151285)(669.094646,65.29151611)
\curveto(669.10463934,65.35151275)(669.10463934,65.41151269)(669.094646,65.47151611)
\curveto(669.09463935,65.53151257)(669.09963935,65.59151251)(669.109646,65.65151611)
}
}
{
\newrgbcolor{curcolor}{0 0 0}
\pscustom[linestyle=none,fillstyle=solid,fillcolor=curcolor]
{
\newpath
\moveto(675.960896,69.91151611)
\curveto(676.34089101,69.92150818)(676.66089069,69.88150822)(676.920896,69.79151611)
\curveto(677.19089016,69.7015084)(677.43588992,69.57150853)(677.655896,69.40151611)
\curveto(677.73588962,69.35150875)(677.80088955,69.28150882)(677.850896,69.19151611)
\curveto(677.91088944,69.11150899)(677.97588938,69.03650907)(678.045896,68.96651611)
\curveto(678.06588929,68.94650916)(678.09588926,68.92150918)(678.135896,68.89151611)
\curveto(678.17588918,68.86150924)(678.22588913,68.85150925)(678.285896,68.86151611)
\curveto(678.38588897,68.89150921)(678.47088888,68.95150915)(678.540896,69.04151611)
\curveto(678.62088873,69.14150896)(678.70088865,69.21650889)(678.780896,69.26651611)
\curveto(678.92088843,69.37650873)(679.06588829,69.47150863)(679.215896,69.55151611)
\curveto(679.36588799,69.64150846)(679.53088782,69.71650839)(679.710896,69.77651611)
\curveto(679.79088756,69.8065083)(679.87588748,69.82650828)(679.965896,69.83651611)
\curveto(680.06588729,69.85650825)(680.16088719,69.87650823)(680.250896,69.89651611)
\curveto(680.30088705,69.9065082)(680.34588701,69.91150819)(680.385896,69.91151611)
\lineto(680.535896,69.91151611)
\curveto(680.58588677,69.93150817)(680.6558867,69.93650817)(680.745896,69.92651611)
\curveto(680.83588652,69.92650818)(680.90088645,69.92150818)(680.940896,69.91151611)
\curveto(680.99088636,69.9015082)(681.06588629,69.89650821)(681.165896,69.89651611)
\curveto(681.2558861,69.87650823)(681.34088601,69.85650825)(681.420896,69.83651611)
\curveto(681.51088584,69.82650828)(681.59588576,69.8065083)(681.675896,69.77651611)
\curveto(681.72588563,69.75650835)(681.77088558,69.74150836)(681.810896,69.73151611)
\curveto(681.86088549,69.73150837)(681.91088544,69.72150838)(681.960896,69.70151611)
\curveto(682.46088489,69.48150862)(682.80588455,69.14150896)(682.995896,68.68151611)
\curveto(683.03588432,68.6015095)(683.06588429,68.51150959)(683.085896,68.41151611)
\curveto(683.10588425,68.32150978)(683.12588423,68.22150988)(683.145896,68.11151611)
\curveto(683.16588419,68.08151002)(683.17088418,68.04651006)(683.160896,68.00651611)
\curveto(683.16088419,67.97651013)(683.16588419,67.94651016)(683.175896,67.91651611)
\lineto(683.175896,67.78151611)
\curveto(683.18588417,67.74151036)(683.18588417,67.69651041)(683.175896,67.64651611)
\curveto(683.17588418,67.59651051)(683.17588418,67.54651056)(683.175896,67.49651611)
\lineto(683.175896,66.91151611)
\lineto(683.175896,65.95151611)
\lineto(683.175896,63.10151611)
\curveto(683.17588418,62.94151516)(683.17588418,62.75151535)(683.175896,62.53151611)
\curveto(683.18588417,62.31151579)(683.14588421,62.16651594)(683.055896,62.09651611)
\curveto(683.01588434,62.06651604)(682.9508844,62.04151606)(682.860896,62.02151611)
\curveto(682.77088458,62.01151609)(682.67588468,62.0065161)(682.575896,62.00651611)
\curveto(682.47588488,62.0065161)(682.37588498,62.01151609)(682.275896,62.02151611)
\curveto(682.18588517,62.03151607)(682.12088523,62.05151605)(682.080896,62.08151611)
\curveto(682.02088533,62.11151599)(681.98088537,62.17151593)(681.960896,62.26151611)
\curveto(681.94088541,62.32151578)(681.93588542,62.38151572)(681.945896,62.44151611)
\curveto(681.9558854,62.51151559)(681.9508854,62.57651553)(681.930896,62.63651611)
\curveto(681.92088543,62.68651542)(681.91588544,62.74151536)(681.915896,62.80151611)
\curveto(681.92588543,62.87151523)(681.93088542,62.93651517)(681.930896,62.99651611)
\lineto(681.930896,63.67151611)
\lineto(681.930896,66.53651611)
\curveto(681.93088542,66.86651124)(681.92088543,67.17651093)(681.900896,67.46651611)
\curveto(681.89088546,67.76651034)(681.82088553,68.01651009)(681.690896,68.21651611)
\curveto(681.54088581,68.45650965)(681.31088604,68.63150947)(681.000896,68.74151611)
\curveto(680.94088641,68.76150934)(680.87588648,68.77150933)(680.805896,68.77151611)
\curveto(680.74588661,68.78150932)(680.68088667,68.79650931)(680.610896,68.81651611)
\curveto(680.57088678,68.82650928)(680.50588685,68.82650928)(680.415896,68.81651611)
\curveto(680.32588703,68.81650929)(680.26588709,68.81150929)(680.235896,68.80151611)
\curveto(680.18588717,68.79150931)(680.13588722,68.78650932)(680.085896,68.78651611)
\curveto(680.03588732,68.79650931)(679.98588737,68.79150931)(679.935896,68.77151611)
\curveto(679.79588756,68.74150936)(679.66088769,68.7015094)(679.530896,68.65151611)
\curveto(679.01088834,68.43150967)(678.66088869,68.04651006)(678.480896,67.49651611)
\curveto(678.43088892,67.32651078)(678.40088895,67.13151097)(678.390896,66.91151611)
\lineto(678.390896,66.23651611)
\lineto(678.390896,64.27151611)
\lineto(678.390896,62.81651611)
\lineto(678.390896,62.44151611)
\curveto(678.39088896,62.32151578)(678.36588899,62.22651588)(678.315896,62.15651611)
\curveto(678.26588909,62.07651603)(678.18088917,62.03151607)(678.060896,62.02151611)
\curveto(677.94088941,62.01151609)(677.81588954,62.0065161)(677.685896,62.00651611)
\curveto(677.51588984,62.0065161)(677.39088996,62.02651608)(677.310896,62.06651611)
\curveto(677.22089013,62.11651599)(677.16589019,62.19651591)(677.145896,62.30651611)
\curveto(677.13589022,62.42651568)(677.13089022,62.55651555)(677.130896,62.69651611)
\lineto(677.130896,64.12151611)
\lineto(677.130896,66.59651611)
\curveto(677.13089022,66.91651119)(677.12089023,67.21151089)(677.100896,67.48151611)
\curveto(677.08089027,67.76151034)(677.01089034,68.0015101)(676.890896,68.20151611)
\curveto(676.78089057,68.38150972)(676.6558907,68.51150959)(676.515896,68.59151611)
\curveto(676.37589098,68.68150942)(676.18589117,68.75150935)(675.945896,68.80151611)
\curveto(675.90589145,68.81150929)(675.86089149,68.81650929)(675.810896,68.81651611)
\lineto(675.675896,68.81651611)
\curveto(675.4558919,68.81650929)(675.26089209,68.79150931)(675.090896,68.74151611)
\curveto(674.93089242,68.69150941)(674.78589257,68.62650948)(674.655896,68.54651611)
\curveto(674.14589321,68.23650987)(673.80589355,67.77151033)(673.635896,67.15151611)
\curveto(673.59589376,67.02151108)(673.57589378,66.87151123)(673.575896,66.70151611)
\curveto(673.58589377,66.54151156)(673.59089376,66.38151172)(673.590896,66.22151611)
\lineto(673.590896,64.52651611)
\lineto(673.590896,62.87651611)
\lineto(673.590896,62.47151611)
\curveto(673.59089376,62.33151577)(673.56089379,62.22151588)(673.500896,62.14151611)
\curveto(673.4508939,62.07151603)(673.37589398,62.03151607)(673.275896,62.02151611)
\curveto(673.17589418,62.01151609)(673.07089428,62.0065161)(672.960896,62.00651611)
\lineto(672.735896,62.00651611)
\curveto(672.67589468,62.02651608)(672.61589474,62.04151606)(672.555896,62.05151611)
\curveto(672.50589485,62.06151604)(672.46089489,62.09151601)(672.420896,62.14151611)
\curveto(672.37089498,62.2015159)(672.34589501,62.27651583)(672.345896,62.36651611)
\lineto(672.345896,62.68151611)
\lineto(672.345896,63.65651611)
\lineto(672.345896,67.94651611)
\lineto(672.345896,69.05651611)
\lineto(672.345896,69.34151611)
\curveto(672.34589501,69.44150866)(672.36589499,69.52150858)(672.405896,69.58151611)
\curveto(672.43589492,69.64150846)(672.48089487,69.68150842)(672.540896,69.70151611)
\curveto(672.62089473,69.73150837)(672.74589461,69.74650836)(672.915896,69.74651611)
\curveto(673.09589426,69.74650836)(673.22589413,69.73150837)(673.305896,69.70151611)
\curveto(673.38589397,69.66150844)(673.44089391,69.61150849)(673.470896,69.55151611)
\curveto(673.49089386,69.5015086)(673.50089385,69.44150866)(673.500896,69.37151611)
\curveto(673.51089384,69.3015088)(673.52089383,69.23650887)(673.530896,69.17651611)
\curveto(673.54089381,69.11650899)(673.56089379,69.06650904)(673.590896,69.02651611)
\curveto(673.62089373,68.98650912)(673.67089368,68.96650914)(673.740896,68.96651611)
\curveto(673.76089359,68.98650912)(673.78089357,68.99650911)(673.800896,68.99651611)
\curveto(673.83089352,68.99650911)(673.8558935,69.0065091)(673.875896,69.02651611)
\curveto(673.93589342,69.07650903)(673.99089336,69.12650898)(674.040896,69.17651611)
\lineto(674.220896,69.32651611)
\curveto(674.44089291,69.48650862)(674.69089266,69.62650848)(674.970896,69.74651611)
\curveto(675.07089228,69.78650832)(675.17089218,69.81150829)(675.270896,69.82151611)
\curveto(675.37089198,69.84150826)(675.47589188,69.86650824)(675.585896,69.89651611)
\lineto(675.765896,69.89651611)
\curveto(675.83589152,69.9065082)(675.90089145,69.91150819)(675.960896,69.91151611)
}
}
{
\newrgbcolor{curcolor}{0 0 0}
\pscustom[linestyle=none,fillstyle=solid,fillcolor=curcolor]
{
\newpath
\moveto(685.35863037,71.23151611)
\curveto(685.27862925,71.29150681)(685.2336293,71.39650671)(685.22363037,71.54651611)
\lineto(685.22363037,72.01151611)
\lineto(685.22363037,72.26651611)
\curveto(685.22362931,72.35650575)(685.23862929,72.43150567)(685.26863037,72.49151611)
\curveto(685.30862922,72.57150553)(685.38862914,72.63150547)(685.50863037,72.67151611)
\curveto(685.528629,72.68150542)(685.54862898,72.68150542)(685.56863037,72.67151611)
\curveto(685.59862893,72.67150543)(685.62362891,72.67650543)(685.64363037,72.68651611)
\curveto(685.81362872,72.68650542)(685.97362856,72.68150542)(686.12363037,72.67151611)
\curveto(686.27362826,72.66150544)(686.37362816,72.6015055)(686.42363037,72.49151611)
\curveto(686.45362808,72.43150567)(686.46862806,72.35650575)(686.46863037,72.26651611)
\lineto(686.46863037,72.01151611)
\curveto(686.46862806,71.83150627)(686.46362807,71.66150644)(686.45363037,71.50151611)
\curveto(686.45362808,71.34150676)(686.38862814,71.23650687)(686.25863037,71.18651611)
\curveto(686.20862832,71.16650694)(686.15362838,71.15650695)(686.09363037,71.15651611)
\lineto(685.92863037,71.15651611)
\lineto(685.61363037,71.15651611)
\curveto(685.51362902,71.15650695)(685.4286291,71.18150692)(685.35863037,71.23151611)
\moveto(686.46863037,62.72651611)
\lineto(686.46863037,62.41151611)
\curveto(686.47862805,62.31151579)(686.45862807,62.23151587)(686.40863037,62.17151611)
\curveto(686.37862815,62.11151599)(686.3336282,62.07151603)(686.27363037,62.05151611)
\curveto(686.21362832,62.04151606)(686.14362839,62.02651608)(686.06363037,62.00651611)
\lineto(685.83863037,62.00651611)
\curveto(685.70862882,62.0065161)(685.59362894,62.01151609)(685.49363037,62.02151611)
\curveto(685.40362913,62.04151606)(685.3336292,62.09151601)(685.28363037,62.17151611)
\curveto(685.24362929,62.23151587)(685.22362931,62.3065158)(685.22363037,62.39651611)
\lineto(685.22363037,62.68151611)
\lineto(685.22363037,69.02651611)
\lineto(685.22363037,69.34151611)
\curveto(685.22362931,69.45150865)(685.24862928,69.53650857)(685.29863037,69.59651611)
\curveto(685.3286292,69.64650846)(685.36862916,69.67650843)(685.41863037,69.68651611)
\curveto(685.46862906,69.69650841)(685.52362901,69.71150839)(685.58363037,69.73151611)
\curveto(685.60362893,69.73150837)(685.62362891,69.72650838)(685.64363037,69.71651611)
\curveto(685.67362886,69.71650839)(685.69862883,69.72150838)(685.71863037,69.73151611)
\curveto(685.84862868,69.73150837)(685.97862855,69.72650838)(686.10863037,69.71651611)
\curveto(686.24862828,69.71650839)(686.34362819,69.67650843)(686.39363037,69.59651611)
\curveto(686.44362809,69.53650857)(686.46862806,69.45650865)(686.46863037,69.35651611)
\lineto(686.46863037,69.07151611)
\lineto(686.46863037,62.72651611)
}
}
{
\newrgbcolor{curcolor}{0 0 0}
\pscustom[linestyle=none,fillstyle=solid,fillcolor=curcolor]
{
\newpath
\moveto(692.10347412,69.88151611)
\curveto(692.73346889,69.9015082)(693.23846838,69.81650829)(693.61847412,69.62651611)
\curveto(693.99846762,69.43650867)(694.30346732,69.15150895)(694.53347412,68.77151611)
\curveto(694.59346703,68.67150943)(694.63846698,68.56150954)(694.66847412,68.44151611)
\curveto(694.70846691,68.33150977)(694.74346688,68.21650989)(694.77347412,68.09651611)
\curveto(694.8234668,67.9065102)(694.85346677,67.7015104)(694.86347412,67.48151611)
\curveto(694.87346675,67.26151084)(694.87846674,67.03651107)(694.87847412,66.80651611)
\lineto(694.87847412,65.20151611)
\lineto(694.87847412,62.86151611)
\curveto(694.87846674,62.69151541)(694.87346675,62.52151558)(694.86347412,62.35151611)
\curveto(694.86346676,62.18151592)(694.79846682,62.07151603)(694.66847412,62.02151611)
\curveto(694.618467,62.0015161)(694.56346706,61.99151611)(694.50347412,61.99151611)
\curveto(694.45346717,61.98151612)(694.39846722,61.97651613)(694.33847412,61.97651611)
\curveto(694.20846741,61.97651613)(694.08346754,61.98151612)(693.96347412,61.99151611)
\curveto(693.84346778,61.99151611)(693.75846786,62.03151607)(693.70847412,62.11151611)
\curveto(693.65846796,62.18151592)(693.63346799,62.27151583)(693.63347412,62.38151611)
\lineto(693.63347412,62.71151611)
\lineto(693.63347412,64.00151611)
\lineto(693.63347412,66.44651611)
\curveto(693.63346799,66.71651139)(693.62846799,66.98151112)(693.61847412,67.24151611)
\curveto(693.60846801,67.51151059)(693.56346806,67.74151036)(693.48347412,67.93151611)
\curveto(693.40346822,68.13150997)(693.28346834,68.29150981)(693.12347412,68.41151611)
\curveto(692.96346866,68.54150956)(692.77846884,68.64150946)(692.56847412,68.71151611)
\curveto(692.50846911,68.73150937)(692.44346918,68.74150936)(692.37347412,68.74151611)
\curveto(692.31346931,68.75150935)(692.25346937,68.76650934)(692.19347412,68.78651611)
\curveto(692.14346948,68.79650931)(692.06346956,68.79650931)(691.95347412,68.78651611)
\curveto(691.85346977,68.78650932)(691.78346984,68.78150932)(691.74347412,68.77151611)
\curveto(691.70346992,68.75150935)(691.66846995,68.74150936)(691.63847412,68.74151611)
\curveto(691.60847001,68.75150935)(691.57347005,68.75150935)(691.53347412,68.74151611)
\curveto(691.40347022,68.71150939)(691.27847034,68.67650943)(691.15847412,68.63651611)
\curveto(691.04847057,68.6065095)(690.94347068,68.56150954)(690.84347412,68.50151611)
\curveto(690.80347082,68.48150962)(690.76847085,68.46150964)(690.73847412,68.44151611)
\curveto(690.70847091,68.42150968)(690.67347095,68.4015097)(690.63347412,68.38151611)
\curveto(690.28347134,68.13150997)(690.02847159,67.75651035)(689.86847412,67.25651611)
\curveto(689.83847178,67.17651093)(689.8184718,67.09151101)(689.80847412,67.00151611)
\curveto(689.79847182,66.92151118)(689.78347184,66.84151126)(689.76347412,66.76151611)
\curveto(689.74347188,66.71151139)(689.73847188,66.66151144)(689.74847412,66.61151611)
\curveto(689.75847186,66.57151153)(689.75347187,66.53151157)(689.73347412,66.49151611)
\lineto(689.73347412,66.17651611)
\curveto(689.7234719,66.14651196)(689.7184719,66.11151199)(689.71847412,66.07151611)
\curveto(689.72847189,66.03151207)(689.73347189,65.98651212)(689.73347412,65.93651611)
\lineto(689.73347412,65.48651611)
\lineto(689.73347412,64.04651611)
\lineto(689.73347412,62.72651611)
\lineto(689.73347412,62.38151611)
\curveto(689.73347189,62.27151583)(689.70847191,62.18151592)(689.65847412,62.11151611)
\curveto(689.60847201,62.03151607)(689.5184721,61.99151611)(689.38847412,61.99151611)
\curveto(689.26847235,61.98151612)(689.14347248,61.97651613)(689.01347412,61.97651611)
\curveto(688.93347269,61.97651613)(688.85847276,61.98151612)(688.78847412,61.99151611)
\curveto(688.7184729,62.0015161)(688.65847296,62.02651608)(688.60847412,62.06651611)
\curveto(688.52847309,62.11651599)(688.48847313,62.21151589)(688.48847412,62.35151611)
\lineto(688.48847412,62.75651611)
\lineto(688.48847412,64.52651611)
\lineto(688.48847412,68.15651611)
\lineto(688.48847412,69.07151611)
\lineto(688.48847412,69.34151611)
\curveto(688.48847313,69.43150867)(688.50847311,69.5015086)(688.54847412,69.55151611)
\curveto(688.57847304,69.61150849)(688.62847299,69.65150845)(688.69847412,69.67151611)
\curveto(688.73847288,69.68150842)(688.79347283,69.69150841)(688.86347412,69.70151611)
\curveto(688.94347268,69.71150839)(689.0234726,69.71650839)(689.10347412,69.71651611)
\curveto(689.18347244,69.71650839)(689.25847236,69.71150839)(689.32847412,69.70151611)
\curveto(689.40847221,69.69150841)(689.46347216,69.67650843)(689.49347412,69.65651611)
\curveto(689.60347202,69.58650852)(689.65347197,69.49650861)(689.64347412,69.38651611)
\curveto(689.63347199,69.28650882)(689.64847197,69.17150893)(689.68847412,69.04151611)
\curveto(689.70847191,68.98150912)(689.74847187,68.93150917)(689.80847412,68.89151611)
\curveto(689.92847169,68.88150922)(690.0234716,68.92650918)(690.09347412,69.02651611)
\curveto(690.17347145,69.12650898)(690.25347137,69.2065089)(690.33347412,69.26651611)
\curveto(690.47347115,69.36650874)(690.61347101,69.45650865)(690.75347412,69.53651611)
\curveto(690.90347072,69.62650848)(691.07347055,69.7015084)(691.26347412,69.76151611)
\curveto(691.34347028,69.79150831)(691.42847019,69.81150829)(691.51847412,69.82151611)
\curveto(691.61847,69.83150827)(691.71346991,69.84650826)(691.80347412,69.86651611)
\curveto(691.85346977,69.87650823)(691.90346972,69.88150822)(691.95347412,69.88151611)
\lineto(692.10347412,69.88151611)
}
}
{
\newrgbcolor{curcolor}{0 0 0}
\pscustom[linestyle=none,fillstyle=solid,fillcolor=curcolor]
{
\newpath
\moveto(697.0480835,71.23151611)
\curveto(696.96808238,71.29150681)(696.92308242,71.39650671)(696.9130835,71.54651611)
\lineto(696.9130835,72.01151611)
\lineto(696.9130835,72.26651611)
\curveto(696.91308243,72.35650575)(696.92808242,72.43150567)(696.9580835,72.49151611)
\curveto(696.99808235,72.57150553)(697.07808227,72.63150547)(697.1980835,72.67151611)
\curveto(697.21808213,72.68150542)(697.23808211,72.68150542)(697.2580835,72.67151611)
\curveto(697.28808206,72.67150543)(697.31308203,72.67650543)(697.3330835,72.68651611)
\curveto(697.50308184,72.68650542)(697.66308168,72.68150542)(697.8130835,72.67151611)
\curveto(697.96308138,72.66150544)(698.06308128,72.6015055)(698.1130835,72.49151611)
\curveto(698.1430812,72.43150567)(698.15808119,72.35650575)(698.1580835,72.26651611)
\lineto(698.1580835,72.01151611)
\curveto(698.15808119,71.83150627)(698.15308119,71.66150644)(698.1430835,71.50151611)
\curveto(698.1430812,71.34150676)(698.07808127,71.23650687)(697.9480835,71.18651611)
\curveto(697.89808145,71.16650694)(697.8430815,71.15650695)(697.7830835,71.15651611)
\lineto(697.6180835,71.15651611)
\lineto(697.3030835,71.15651611)
\curveto(697.20308214,71.15650695)(697.11808223,71.18150692)(697.0480835,71.23151611)
\moveto(698.1580835,62.72651611)
\lineto(698.1580835,62.41151611)
\curveto(698.16808118,62.31151579)(698.1480812,62.23151587)(698.0980835,62.17151611)
\curveto(698.06808128,62.11151599)(698.02308132,62.07151603)(697.9630835,62.05151611)
\curveto(697.90308144,62.04151606)(697.83308151,62.02651608)(697.7530835,62.00651611)
\lineto(697.5280835,62.00651611)
\curveto(697.39808195,62.0065161)(697.28308206,62.01151609)(697.1830835,62.02151611)
\curveto(697.09308225,62.04151606)(697.02308232,62.09151601)(696.9730835,62.17151611)
\curveto(696.93308241,62.23151587)(696.91308243,62.3065158)(696.9130835,62.39651611)
\lineto(696.9130835,62.68151611)
\lineto(696.9130835,69.02651611)
\lineto(696.9130835,69.34151611)
\curveto(696.91308243,69.45150865)(696.93808241,69.53650857)(696.9880835,69.59651611)
\curveto(697.01808233,69.64650846)(697.05808229,69.67650843)(697.1080835,69.68651611)
\curveto(697.15808219,69.69650841)(697.21308213,69.71150839)(697.2730835,69.73151611)
\curveto(697.29308205,69.73150837)(697.31308203,69.72650838)(697.3330835,69.71651611)
\curveto(697.36308198,69.71650839)(697.38808196,69.72150838)(697.4080835,69.73151611)
\curveto(697.53808181,69.73150837)(697.66808168,69.72650838)(697.7980835,69.71651611)
\curveto(697.93808141,69.71650839)(698.03308131,69.67650843)(698.0830835,69.59651611)
\curveto(698.13308121,69.53650857)(698.15808119,69.45650865)(698.1580835,69.35651611)
\lineto(698.1580835,69.07151611)
\lineto(698.1580835,62.72651611)
}
}
{
\newrgbcolor{curcolor}{0 0 0}
\pscustom[linestyle=none,fillstyle=solid,fillcolor=curcolor]
{
\newpath
\moveto(702.53292725,69.91151611)
\curveto(703.25292318,69.92150818)(703.85792258,69.83650827)(704.34792725,69.65651611)
\curveto(704.8379216,69.48650862)(705.21792122,69.18150892)(705.48792725,68.74151611)
\curveto(705.55792088,68.63150947)(705.61292082,68.51650959)(705.65292725,68.39651611)
\curveto(705.69292074,68.28650982)(705.7329207,68.16150994)(705.77292725,68.02151611)
\curveto(705.79292064,67.95151015)(705.79792064,67.87651023)(705.78792725,67.79651611)
\curveto(705.77792066,67.72651038)(705.76292067,67.67151043)(705.74292725,67.63151611)
\curveto(705.72292071,67.61151049)(705.69792074,67.59151051)(705.66792725,67.57151611)
\curveto(705.6379208,67.56151054)(705.61292082,67.54651056)(705.59292725,67.52651611)
\curveto(705.54292089,67.5065106)(705.49292094,67.5015106)(705.44292725,67.51151611)
\curveto(705.39292104,67.52151058)(705.34292109,67.52151058)(705.29292725,67.51151611)
\curveto(705.21292122,67.49151061)(705.10792133,67.48651062)(704.97792725,67.49651611)
\curveto(704.84792159,67.51651059)(704.75792168,67.54151056)(704.70792725,67.57151611)
\curveto(704.62792181,67.62151048)(704.57292186,67.68651042)(704.54292725,67.76651611)
\curveto(704.52292191,67.85651025)(704.48792195,67.94151016)(704.43792725,68.02151611)
\curveto(704.34792209,68.18150992)(704.22292221,68.32650978)(704.06292725,68.45651611)
\curveto(703.95292248,68.53650957)(703.8329226,68.59650951)(703.70292725,68.63651611)
\curveto(703.57292286,68.67650943)(703.432923,68.71650939)(703.28292725,68.75651611)
\curveto(703.2329232,68.77650933)(703.18292325,68.78150932)(703.13292725,68.77151611)
\curveto(703.08292335,68.77150933)(703.0329234,68.77650933)(702.98292725,68.78651611)
\curveto(702.92292351,68.8065093)(702.84792359,68.81650929)(702.75792725,68.81651611)
\curveto(702.66792377,68.81650929)(702.59292384,68.8065093)(702.53292725,68.78651611)
\lineto(702.44292725,68.78651611)
\lineto(702.29292725,68.75651611)
\curveto(702.24292419,68.75650935)(702.19292424,68.75150935)(702.14292725,68.74151611)
\curveto(701.88292455,68.68150942)(701.66792477,68.59650951)(701.49792725,68.48651611)
\curveto(701.32792511,68.37650973)(701.21292522,68.19150991)(701.15292725,67.93151611)
\curveto(701.1329253,67.86151024)(701.12792531,67.79151031)(701.13792725,67.72151611)
\curveto(701.15792528,67.65151045)(701.17792526,67.59151051)(701.19792725,67.54151611)
\curveto(701.25792518,67.39151071)(701.32792511,67.28151082)(701.40792725,67.21151611)
\curveto(701.49792494,67.15151095)(701.60792483,67.08151102)(701.73792725,67.00151611)
\curveto(701.89792454,66.9015112)(702.07792436,66.82651128)(702.27792725,66.77651611)
\curveto(702.47792396,66.73651137)(702.67792376,66.68651142)(702.87792725,66.62651611)
\curveto(703.00792343,66.58651152)(703.1379233,66.55651155)(703.26792725,66.53651611)
\curveto(703.39792304,66.51651159)(703.52792291,66.48651162)(703.65792725,66.44651611)
\curveto(703.86792257,66.38651172)(704.07292236,66.32651178)(704.27292725,66.26651611)
\curveto(704.47292196,66.21651189)(704.67292176,66.15151195)(704.87292725,66.07151611)
\lineto(705.02292725,66.01151611)
\curveto(705.07292136,65.99151211)(705.12292131,65.96651214)(705.17292725,65.93651611)
\curveto(705.37292106,65.81651229)(705.54792089,65.68151242)(705.69792725,65.53151611)
\curveto(705.84792059,65.38151272)(705.97292046,65.19151291)(706.07292725,64.96151611)
\curveto(706.09292034,64.89151321)(706.11292032,64.79651331)(706.13292725,64.67651611)
\curveto(706.15292028,64.6065135)(706.16292027,64.53151357)(706.16292725,64.45151611)
\curveto(706.17292026,64.38151372)(706.17792026,64.3015138)(706.17792725,64.21151611)
\lineto(706.17792725,64.06151611)
\curveto(706.15792028,63.99151411)(706.14792029,63.92151418)(706.14792725,63.85151611)
\curveto(706.14792029,63.78151432)(706.1379203,63.71151439)(706.11792725,63.64151611)
\curveto(706.08792035,63.53151457)(706.05292038,63.42651468)(706.01292725,63.32651611)
\curveto(705.97292046,63.22651488)(705.92792051,63.13651497)(705.87792725,63.05651611)
\curveto(705.71792072,62.79651531)(705.51292092,62.58651552)(705.26292725,62.42651611)
\curveto(705.01292142,62.27651583)(704.7329217,62.14651596)(704.42292725,62.03651611)
\curveto(704.3329221,62.0065161)(704.2379222,61.98651612)(704.13792725,61.97651611)
\curveto(704.04792239,61.95651615)(703.95792248,61.93151617)(703.86792725,61.90151611)
\curveto(703.76792267,61.88151622)(703.66792277,61.87151623)(703.56792725,61.87151611)
\curveto(703.46792297,61.87151623)(703.36792307,61.86151624)(703.26792725,61.84151611)
\lineto(703.11792725,61.84151611)
\curveto(703.06792337,61.83151627)(702.99792344,61.82651628)(702.90792725,61.82651611)
\curveto(702.81792362,61.82651628)(702.74792369,61.83151627)(702.69792725,61.84151611)
\lineto(702.53292725,61.84151611)
\curveto(702.47292396,61.86151624)(702.40792403,61.87151623)(702.33792725,61.87151611)
\curveto(702.26792417,61.86151624)(702.20792423,61.86651624)(702.15792725,61.88651611)
\curveto(702.10792433,61.89651621)(702.04292439,61.9015162)(701.96292725,61.90151611)
\lineto(701.72292725,61.96151611)
\curveto(701.65292478,61.97151613)(701.57792486,61.99151611)(701.49792725,62.02151611)
\curveto(701.18792525,62.12151598)(700.91792552,62.24651586)(700.68792725,62.39651611)
\curveto(700.45792598,62.54651556)(700.25792618,62.74151536)(700.08792725,62.98151611)
\curveto(699.99792644,63.11151499)(699.92292651,63.24651486)(699.86292725,63.38651611)
\curveto(699.80292663,63.52651458)(699.74792669,63.68151442)(699.69792725,63.85151611)
\curveto(699.67792676,63.91151419)(699.66792677,63.98151412)(699.66792725,64.06151611)
\curveto(699.67792676,64.15151395)(699.69292674,64.22151388)(699.71292725,64.27151611)
\curveto(699.74292669,64.31151379)(699.79292664,64.35151375)(699.86292725,64.39151611)
\curveto(699.91292652,64.41151369)(699.98292645,64.42151368)(700.07292725,64.42151611)
\curveto(700.16292627,64.43151367)(700.25292618,64.43151367)(700.34292725,64.42151611)
\curveto(700.432926,64.41151369)(700.51792592,64.39651371)(700.59792725,64.37651611)
\curveto(700.68792575,64.36651374)(700.74792569,64.35151375)(700.77792725,64.33151611)
\curveto(700.84792559,64.28151382)(700.89292554,64.2065139)(700.91292725,64.10651611)
\curveto(700.94292549,64.01651409)(700.97792546,63.93151417)(701.01792725,63.85151611)
\curveto(701.11792532,63.63151447)(701.25292518,63.46151464)(701.42292725,63.34151611)
\curveto(701.54292489,63.25151485)(701.67792476,63.18151492)(701.82792725,63.13151611)
\curveto(701.97792446,63.08151502)(702.1379243,63.03151507)(702.30792725,62.98151611)
\lineto(702.62292725,62.93651611)
\lineto(702.71292725,62.93651611)
\curveto(702.78292365,62.91651519)(702.87292356,62.9065152)(702.98292725,62.90651611)
\curveto(703.10292333,62.9065152)(703.20292323,62.91651519)(703.28292725,62.93651611)
\curveto(703.35292308,62.93651517)(703.40792303,62.94151516)(703.44792725,62.95151611)
\curveto(703.50792293,62.96151514)(703.56792287,62.96651514)(703.62792725,62.96651611)
\curveto(703.68792275,62.97651513)(703.74292269,62.98651512)(703.79292725,62.99651611)
\curveto(704.08292235,63.07651503)(704.31292212,63.18151492)(704.48292725,63.31151611)
\curveto(704.65292178,63.44151466)(704.77292166,63.66151444)(704.84292725,63.97151611)
\curveto(704.86292157,64.02151408)(704.86792157,64.07651403)(704.85792725,64.13651611)
\curveto(704.84792159,64.19651391)(704.8379216,64.24151386)(704.82792725,64.27151611)
\curveto(704.77792166,64.46151364)(704.70792173,64.6015135)(704.61792725,64.69151611)
\curveto(704.52792191,64.79151331)(704.41292202,64.88151322)(704.27292725,64.96151611)
\curveto(704.18292225,65.02151308)(704.08292235,65.07151303)(703.97292725,65.11151611)
\lineto(703.64292725,65.23151611)
\curveto(703.61292282,65.24151286)(703.58292285,65.24651286)(703.55292725,65.24651611)
\curveto(703.5329229,65.24651286)(703.50792293,65.25651285)(703.47792725,65.27651611)
\curveto(703.1379233,65.38651272)(702.78292365,65.46651264)(702.41292725,65.51651611)
\curveto(702.05292438,65.57651253)(701.71292472,65.67151243)(701.39292725,65.80151611)
\curveto(701.29292514,65.84151226)(701.19792524,65.87651223)(701.10792725,65.90651611)
\curveto(701.01792542,65.93651217)(700.9329255,65.97651213)(700.85292725,66.02651611)
\curveto(700.66292577,66.13651197)(700.48792595,66.26151184)(700.32792725,66.40151611)
\curveto(700.16792627,66.54151156)(700.04292639,66.71651139)(699.95292725,66.92651611)
\curveto(699.92292651,66.99651111)(699.89792654,67.06651104)(699.87792725,67.13651611)
\curveto(699.86792657,67.2065109)(699.85292658,67.28151082)(699.83292725,67.36151611)
\curveto(699.80292663,67.48151062)(699.79292664,67.61651049)(699.80292725,67.76651611)
\curveto(699.81292662,67.92651018)(699.82792661,68.06151004)(699.84792725,68.17151611)
\curveto(699.86792657,68.22150988)(699.87792656,68.26150984)(699.87792725,68.29151611)
\curveto(699.88792655,68.33150977)(699.90292653,68.37150973)(699.92292725,68.41151611)
\curveto(700.01292642,68.64150946)(700.1329263,68.84150926)(700.28292725,69.01151611)
\curveto(700.44292599,69.18150892)(700.62292581,69.33150877)(700.82292725,69.46151611)
\curveto(700.97292546,69.55150855)(701.1379253,69.62150848)(701.31792725,69.67151611)
\curveto(701.49792494,69.73150837)(701.68792475,69.78650832)(701.88792725,69.83651611)
\curveto(701.95792448,69.84650826)(702.02292441,69.85650825)(702.08292725,69.86651611)
\curveto(702.15292428,69.87650823)(702.22792421,69.88650822)(702.30792725,69.89651611)
\curveto(702.3379241,69.9065082)(702.37792406,69.9065082)(702.42792725,69.89651611)
\curveto(702.47792396,69.88650822)(702.51292392,69.89150821)(702.53292725,69.91151611)
}
}
{
\newrgbcolor{curcolor}{0 0 0}
\pscustom[linestyle=none,fillstyle=solid,fillcolor=curcolor]
{
\newpath
\moveto(708.54792725,72.07151611)
\curveto(708.69792524,72.07150603)(708.84792509,72.06650604)(708.99792725,72.05651611)
\curveto(709.14792479,72.05650605)(709.25292468,72.01650609)(709.31292725,71.93651611)
\curveto(709.36292457,71.87650623)(709.38792455,71.79150631)(709.38792725,71.68151611)
\curveto(709.39792454,71.58150652)(709.40292453,71.47650663)(709.40292725,71.36651611)
\lineto(709.40292725,70.49651611)
\curveto(709.40292453,70.41650769)(709.39792454,70.33150777)(709.38792725,70.24151611)
\curveto(709.38792455,70.16150794)(709.39792454,70.09150801)(709.41792725,70.03151611)
\curveto(709.45792448,69.89150821)(709.54792439,69.8015083)(709.68792725,69.76151611)
\curveto(709.7379242,69.75150835)(709.78292415,69.74650836)(709.82292725,69.74651611)
\lineto(709.97292725,69.74651611)
\lineto(710.37792725,69.74651611)
\curveto(710.5379234,69.75650835)(710.65292328,69.74650836)(710.72292725,69.71651611)
\curveto(710.81292312,69.65650845)(710.87292306,69.59650851)(710.90292725,69.53651611)
\curveto(710.92292301,69.49650861)(710.932923,69.45150865)(710.93292725,69.40151611)
\lineto(710.93292725,69.25151611)
\curveto(710.932923,69.14150896)(710.92792301,69.03650907)(710.91792725,68.93651611)
\curveto(710.90792303,68.84650926)(710.87292306,68.77650933)(710.81292725,68.72651611)
\curveto(710.75292318,68.67650943)(710.66792327,68.64650946)(710.55792725,68.63651611)
\lineto(710.22792725,68.63651611)
\curveto(710.11792382,68.64650946)(710.00792393,68.65150945)(709.89792725,68.65151611)
\curveto(709.78792415,68.65150945)(709.69292424,68.63650947)(709.61292725,68.60651611)
\curveto(709.54292439,68.57650953)(709.49292444,68.52650958)(709.46292725,68.45651611)
\curveto(709.4329245,68.38650972)(709.41292452,68.3015098)(709.40292725,68.20151611)
\curveto(709.39292454,68.11150999)(709.38792455,68.01151009)(709.38792725,67.90151611)
\curveto(709.39792454,67.8015103)(709.40292453,67.7015104)(709.40292725,67.60151611)
\lineto(709.40292725,64.63151611)
\curveto(709.40292453,64.41151369)(709.39792454,64.17651393)(709.38792725,63.92651611)
\curveto(709.38792455,63.68651442)(709.4329245,63.5015146)(709.52292725,63.37151611)
\curveto(709.57292436,63.29151481)(709.6379243,63.23651487)(709.71792725,63.20651611)
\curveto(709.79792414,63.17651493)(709.89292404,63.15151495)(710.00292725,63.13151611)
\curveto(710.0329239,63.12151498)(710.06292387,63.11651499)(710.09292725,63.11651611)
\curveto(710.1329238,63.12651498)(710.16792377,63.12651498)(710.19792725,63.11651611)
\lineto(710.39292725,63.11651611)
\curveto(710.49292344,63.11651499)(710.58292335,63.106515)(710.66292725,63.08651611)
\curveto(710.75292318,63.07651503)(710.81792312,63.04151506)(710.85792725,62.98151611)
\curveto(710.87792306,62.95151515)(710.89292304,62.89651521)(710.90292725,62.81651611)
\curveto(710.92292301,62.74651536)(710.932923,62.67151543)(710.93292725,62.59151611)
\curveto(710.94292299,62.51151559)(710.94292299,62.43151567)(710.93292725,62.35151611)
\curveto(710.92292301,62.28151582)(710.90292303,62.22651588)(710.87292725,62.18651611)
\curveto(710.8329231,62.11651599)(710.75792318,62.06651604)(710.64792725,62.03651611)
\curveto(710.56792337,62.01651609)(710.47792346,62.0065161)(710.37792725,62.00651611)
\curveto(710.27792366,62.01651609)(710.18792375,62.02151608)(710.10792725,62.02151611)
\curveto(710.04792389,62.02151608)(709.98792395,62.01651609)(709.92792725,62.00651611)
\curveto(709.86792407,62.0065161)(709.81292412,62.01151609)(709.76292725,62.02151611)
\lineto(709.58292725,62.02151611)
\curveto(709.5329244,62.03151607)(709.48292445,62.03651607)(709.43292725,62.03651611)
\curveto(709.39292454,62.04651606)(709.34792459,62.05151605)(709.29792725,62.05151611)
\curveto(709.09792484,62.101516)(708.92292501,62.15651595)(708.77292725,62.21651611)
\curveto(708.6329253,62.27651583)(708.51292542,62.38151572)(708.41292725,62.53151611)
\curveto(708.27292566,62.73151537)(708.19292574,62.98151512)(708.17292725,63.28151611)
\curveto(708.15292578,63.59151451)(708.14292579,63.92151418)(708.14292725,64.27151611)
\lineto(708.14292725,68.20151611)
\curveto(708.11292582,68.33150977)(708.08292585,68.42650968)(708.05292725,68.48651611)
\curveto(708.0329259,68.54650956)(707.96292597,68.59650951)(707.84292725,68.63651611)
\curveto(707.80292613,68.64650946)(707.76292617,68.64650946)(707.72292725,68.63651611)
\curveto(707.68292625,68.62650948)(707.64292629,68.63150947)(707.60292725,68.65151611)
\lineto(707.36292725,68.65151611)
\curveto(707.2329267,68.65150945)(707.12292681,68.66150944)(707.03292725,68.68151611)
\curveto(706.95292698,68.71150939)(706.89792704,68.77150933)(706.86792725,68.86151611)
\curveto(706.84792709,68.9015092)(706.8329271,68.94650916)(706.82292725,68.99651611)
\lineto(706.82292725,69.14651611)
\curveto(706.82292711,69.28650882)(706.8329271,69.4015087)(706.85292725,69.49151611)
\curveto(706.87292706,69.59150851)(706.932927,69.66650844)(707.03292725,69.71651611)
\curveto(707.14292679,69.75650835)(707.28292665,69.76650834)(707.45292725,69.74651611)
\curveto(707.6329263,69.72650838)(707.78292615,69.73650837)(707.90292725,69.77651611)
\curveto(707.99292594,69.82650828)(708.06292587,69.89650821)(708.11292725,69.98651611)
\curveto(708.1329258,70.04650806)(708.14292579,70.12150798)(708.14292725,70.21151611)
\lineto(708.14292725,70.46651611)
\lineto(708.14292725,71.39651611)
\lineto(708.14292725,71.63651611)
\curveto(708.14292579,71.72650638)(708.15292578,71.8015063)(708.17292725,71.86151611)
\curveto(708.21292572,71.94150616)(708.28792565,72.0065061)(708.39792725,72.05651611)
\curveto(708.42792551,72.05650605)(708.45292548,72.05650605)(708.47292725,72.05651611)
\curveto(708.50292543,72.06650604)(708.52792541,72.07150603)(708.54792725,72.07151611)
}
}
{
\newrgbcolor{curcolor}{0 0 0}
\pscustom[linestyle=none,fillstyle=solid,fillcolor=curcolor]
{
\newpath
\moveto(715.96472412,69.91151611)
\curveto(716.19471933,69.91150819)(716.3247192,69.85150825)(716.35472412,69.73151611)
\curveto(716.38471914,69.62150848)(716.39971913,69.45650865)(716.39972412,69.23651611)
\lineto(716.39972412,68.95151611)
\curveto(716.39971913,68.86150924)(716.37471915,68.78650932)(716.32472412,68.72651611)
\curveto(716.26471926,68.64650946)(716.17971935,68.6015095)(716.06972412,68.59151611)
\curveto(715.95971957,68.59150951)(715.84971968,68.57650953)(715.73972412,68.54651611)
\curveto(715.59971993,68.51650959)(715.46472006,68.48650962)(715.33472412,68.45651611)
\curveto(715.21472031,68.42650968)(715.09972043,68.38650972)(714.98972412,68.33651611)
\curveto(714.69972083,68.2065099)(714.46472106,68.02651008)(714.28472412,67.79651611)
\curveto(714.10472142,67.57651053)(713.94972158,67.32151078)(713.81972412,67.03151611)
\curveto(713.77972175,66.92151118)(713.74972178,66.8065113)(713.72972412,66.68651611)
\curveto(713.70972182,66.57651153)(713.68472184,66.46151164)(713.65472412,66.34151611)
\curveto(713.64472188,66.29151181)(713.63972189,66.24151186)(713.63972412,66.19151611)
\curveto(713.64972188,66.14151196)(713.64972188,66.09151201)(713.63972412,66.04151611)
\curveto(713.60972192,65.92151218)(713.59472193,65.78151232)(713.59472412,65.62151611)
\curveto(713.60472192,65.47151263)(713.60972192,65.32651278)(713.60972412,65.18651611)
\lineto(713.60972412,63.34151611)
\lineto(713.60972412,62.99651611)
\curveto(713.60972192,62.87651523)(713.60472192,62.76151534)(713.59472412,62.65151611)
\curveto(713.58472194,62.54151556)(713.57972195,62.44651566)(713.57972412,62.36651611)
\curveto(713.58972194,62.28651582)(713.56972196,62.21651589)(713.51972412,62.15651611)
\curveto(713.46972206,62.08651602)(713.38972214,62.04651606)(713.27972412,62.03651611)
\curveto(713.17972235,62.02651608)(713.06972246,62.02151608)(712.94972412,62.02151611)
\lineto(712.67972412,62.02151611)
\curveto(712.6297229,62.04151606)(712.57972295,62.05651605)(712.52972412,62.06651611)
\curveto(712.48972304,62.08651602)(712.45972307,62.11151599)(712.43972412,62.14151611)
\curveto(712.38972314,62.21151589)(712.35972317,62.29651581)(712.34972412,62.39651611)
\lineto(712.34972412,62.72651611)
\lineto(712.34972412,63.88151611)
\lineto(712.34972412,68.03651611)
\lineto(712.34972412,69.07151611)
\lineto(712.34972412,69.37151611)
\curveto(712.35972317,69.47150863)(712.38972314,69.55650855)(712.43972412,69.62651611)
\curveto(712.46972306,69.66650844)(712.51972301,69.69650841)(712.58972412,69.71651611)
\curveto(712.66972286,69.73650837)(712.75472277,69.74650836)(712.84472412,69.74651611)
\curveto(712.93472259,69.75650835)(713.0247225,69.75650835)(713.11472412,69.74651611)
\curveto(713.20472232,69.73650837)(713.27472225,69.72150838)(713.32472412,69.70151611)
\curveto(713.40472212,69.67150843)(713.45472207,69.61150849)(713.47472412,69.52151611)
\curveto(713.50472202,69.44150866)(713.51972201,69.35150875)(713.51972412,69.25151611)
\lineto(713.51972412,68.95151611)
\curveto(713.51972201,68.85150925)(713.53972199,68.76150934)(713.57972412,68.68151611)
\curveto(713.58972194,68.66150944)(713.59972193,68.64650946)(713.60972412,68.63651611)
\lineto(713.65472412,68.59151611)
\curveto(713.76472176,68.59150951)(713.85472167,68.63650947)(713.92472412,68.72651611)
\curveto(713.99472153,68.82650928)(714.05472147,68.9065092)(714.10472412,68.96651611)
\lineto(714.19472412,69.05651611)
\curveto(714.28472124,69.16650894)(714.40972112,69.28150882)(714.56972412,69.40151611)
\curveto(714.7297208,69.52150858)(714.87972065,69.61150849)(715.01972412,69.67151611)
\curveto(715.10972042,69.72150838)(715.20472032,69.75650835)(715.30472412,69.77651611)
\curveto(715.40472012,69.8065083)(715.50972002,69.83650827)(715.61972412,69.86651611)
\curveto(715.67971985,69.87650823)(715.73971979,69.88150822)(715.79972412,69.88151611)
\curveto(715.85971967,69.89150821)(715.91471961,69.9015082)(715.96472412,69.91151611)
}
}
{
\newrgbcolor{curcolor}{0 0 0}
\pscustom[linestyle=none,fillstyle=solid,fillcolor=curcolor]
{
\newpath
\moveto(724.21448975,62.56151611)
\curveto(724.24448192,62.4015157)(724.22948193,62.26651584)(724.16948975,62.15651611)
\curveto(724.10948205,62.05651605)(724.02948213,61.98151612)(723.92948975,61.93151611)
\curveto(723.87948228,61.91151619)(723.82448234,61.9015162)(723.76448975,61.90151611)
\curveto(723.71448245,61.9015162)(723.6594825,61.89151621)(723.59948975,61.87151611)
\curveto(723.37948278,61.82151628)(723.159483,61.83651627)(722.93948975,61.91651611)
\curveto(722.72948343,61.98651612)(722.58448358,62.07651603)(722.50448975,62.18651611)
\curveto(722.45448371,62.25651585)(722.40948375,62.33651577)(722.36948975,62.42651611)
\curveto(722.32948383,62.52651558)(722.27948388,62.6065155)(722.21948975,62.66651611)
\curveto(722.19948396,62.68651542)(722.17448399,62.7065154)(722.14448975,62.72651611)
\curveto(722.12448404,62.74651536)(722.09448407,62.75151535)(722.05448975,62.74151611)
\curveto(721.94448422,62.71151539)(721.83948432,62.65651545)(721.73948975,62.57651611)
\curveto(721.64948451,62.49651561)(721.5594846,62.42651568)(721.46948975,62.36651611)
\curveto(721.33948482,62.28651582)(721.19948496,62.21151589)(721.04948975,62.14151611)
\curveto(720.89948526,62.08151602)(720.73948542,62.02651608)(720.56948975,61.97651611)
\curveto(720.46948569,61.94651616)(720.3594858,61.92651618)(720.23948975,61.91651611)
\curveto(720.12948603,61.9065162)(720.01948614,61.89151621)(719.90948975,61.87151611)
\curveto(719.8594863,61.86151624)(719.81448635,61.85651625)(719.77448975,61.85651611)
\lineto(719.66948975,61.85651611)
\curveto(719.5594866,61.83651627)(719.45448671,61.83651627)(719.35448975,61.85651611)
\lineto(719.21948975,61.85651611)
\curveto(719.16948699,61.86651624)(719.11948704,61.87151623)(719.06948975,61.87151611)
\curveto(719.01948714,61.87151623)(718.97448719,61.88151622)(718.93448975,61.90151611)
\curveto(718.89448727,61.91151619)(718.8594873,61.91651619)(718.82948975,61.91651611)
\curveto(718.80948735,61.9065162)(718.78448738,61.9065162)(718.75448975,61.91651611)
\lineto(718.51448975,61.97651611)
\curveto(718.43448773,61.98651612)(718.3594878,62.0065161)(718.28948975,62.03651611)
\curveto(717.98948817,62.16651594)(717.74448842,62.31151579)(717.55448975,62.47151611)
\curveto(717.37448879,62.64151546)(717.22448894,62.87651523)(717.10448975,63.17651611)
\curveto(717.01448915,63.39651471)(716.96948919,63.66151444)(716.96948975,63.97151611)
\lineto(716.96948975,64.28651611)
\curveto(716.97948918,64.33651377)(716.98448918,64.38651372)(716.98448975,64.43651611)
\lineto(717.01448975,64.61651611)
\lineto(717.13448975,64.94651611)
\curveto(717.17448899,65.05651305)(717.22448894,65.15651295)(717.28448975,65.24651611)
\curveto(717.4644887,65.53651257)(717.70948845,65.75151235)(718.01948975,65.89151611)
\curveto(718.32948783,66.03151207)(718.66948749,66.15651195)(719.03948975,66.26651611)
\curveto(719.17948698,66.3065118)(719.32448684,66.33651177)(719.47448975,66.35651611)
\curveto(719.62448654,66.37651173)(719.77448639,66.4015117)(719.92448975,66.43151611)
\curveto(719.99448617,66.45151165)(720.0594861,66.46151164)(720.11948975,66.46151611)
\curveto(720.18948597,66.46151164)(720.2644859,66.47151163)(720.34448975,66.49151611)
\curveto(720.41448575,66.51151159)(720.48448568,66.52151158)(720.55448975,66.52151611)
\curveto(720.62448554,66.53151157)(720.69948546,66.54651156)(720.77948975,66.56651611)
\curveto(721.02948513,66.62651148)(721.2644849,66.67651143)(721.48448975,66.71651611)
\curveto(721.70448446,66.76651134)(721.87948428,66.88151122)(722.00948975,67.06151611)
\curveto(722.06948409,67.14151096)(722.11948404,67.24151086)(722.15948975,67.36151611)
\curveto(722.19948396,67.49151061)(722.19948396,67.63151047)(722.15948975,67.78151611)
\curveto(722.09948406,68.02151008)(722.00948415,68.21150989)(721.88948975,68.35151611)
\curveto(721.77948438,68.49150961)(721.61948454,68.6015095)(721.40948975,68.68151611)
\curveto(721.28948487,68.73150937)(721.14448502,68.76650934)(720.97448975,68.78651611)
\curveto(720.81448535,68.8065093)(720.64448552,68.81650929)(720.46448975,68.81651611)
\curveto(720.28448588,68.81650929)(720.10948605,68.8065093)(719.93948975,68.78651611)
\curveto(719.76948639,68.76650934)(719.62448654,68.73650937)(719.50448975,68.69651611)
\curveto(719.33448683,68.63650947)(719.16948699,68.55150955)(719.00948975,68.44151611)
\curveto(718.92948723,68.38150972)(718.85448731,68.3015098)(718.78448975,68.20151611)
\curveto(718.72448744,68.11150999)(718.66948749,68.01151009)(718.61948975,67.90151611)
\curveto(718.58948757,67.82151028)(718.5594876,67.73651037)(718.52948975,67.64651611)
\curveto(718.50948765,67.55651055)(718.4644877,67.48651062)(718.39448975,67.43651611)
\curveto(718.35448781,67.4065107)(718.28448788,67.38151072)(718.18448975,67.36151611)
\curveto(718.09448807,67.35151075)(717.99948816,67.34651076)(717.89948975,67.34651611)
\curveto(717.79948836,67.34651076)(717.69948846,67.35151075)(717.59948975,67.36151611)
\curveto(717.50948865,67.38151072)(717.44448872,67.4065107)(717.40448975,67.43651611)
\curveto(717.3644888,67.46651064)(717.33448883,67.51651059)(717.31448975,67.58651611)
\curveto(717.29448887,67.65651045)(717.29448887,67.73151037)(717.31448975,67.81151611)
\curveto(717.34448882,67.94151016)(717.37448879,68.06151004)(717.40448975,68.17151611)
\curveto(717.44448872,68.29150981)(717.48948867,68.4065097)(717.53948975,68.51651611)
\curveto(717.72948843,68.86650924)(717.96948819,69.13650897)(718.25948975,69.32651611)
\curveto(718.54948761,69.52650858)(718.90948725,69.68650842)(719.33948975,69.80651611)
\curveto(719.43948672,69.82650828)(719.53948662,69.84150826)(719.63948975,69.85151611)
\curveto(719.74948641,69.86150824)(719.8594863,69.87650823)(719.96948975,69.89651611)
\curveto(720.00948615,69.9065082)(720.07448609,69.9065082)(720.16448975,69.89651611)
\curveto(720.25448591,69.89650821)(720.30948585,69.9065082)(720.32948975,69.92651611)
\curveto(721.02948513,69.93650817)(721.63948452,69.85650825)(722.15948975,69.68651611)
\curveto(722.67948348,69.51650859)(723.04448312,69.19150891)(723.25448975,68.71151611)
\curveto(723.34448282,68.51150959)(723.39448277,68.27650983)(723.40448975,68.00651611)
\curveto(723.42448274,67.74651036)(723.43448273,67.47151063)(723.43448975,67.18151611)
\lineto(723.43448975,63.86651611)
\curveto(723.43448273,63.72651438)(723.43948272,63.59151451)(723.44948975,63.46151611)
\curveto(723.4594827,63.33151477)(723.48948267,63.22651488)(723.53948975,63.14651611)
\curveto(723.58948257,63.07651503)(723.65448251,63.02651508)(723.73448975,62.99651611)
\curveto(723.82448234,62.95651515)(723.90948225,62.92651518)(723.98948975,62.90651611)
\curveto(724.06948209,62.89651521)(724.12948203,62.85151525)(724.16948975,62.77151611)
\curveto(724.18948197,62.74151536)(724.19948196,62.71151539)(724.19948975,62.68151611)
\curveto(724.19948196,62.65151545)(724.20448196,62.61151549)(724.21448975,62.56151611)
\moveto(722.06948975,64.22651611)
\curveto(722.12948403,64.36651374)(722.159484,64.52651358)(722.15948975,64.70651611)
\curveto(722.16948399,64.89651321)(722.17448399,65.09151301)(722.17448975,65.29151611)
\curveto(722.17448399,65.4015127)(722.16948399,65.5015126)(722.15948975,65.59151611)
\curveto(722.14948401,65.68151242)(722.10948405,65.75151235)(722.03948975,65.80151611)
\curveto(722.00948415,65.82151228)(721.93948422,65.83151227)(721.82948975,65.83151611)
\curveto(721.80948435,65.81151229)(721.77448439,65.8015123)(721.72448975,65.80151611)
\curveto(721.67448449,65.8015123)(721.62948453,65.79151231)(721.58948975,65.77151611)
\curveto(721.50948465,65.75151235)(721.41948474,65.73151237)(721.31948975,65.71151611)
\lineto(721.01948975,65.65151611)
\curveto(720.98948517,65.65151245)(720.95448521,65.64651246)(720.91448975,65.63651611)
\lineto(720.80948975,65.63651611)
\curveto(720.6594855,65.59651251)(720.49448567,65.57151253)(720.31448975,65.56151611)
\curveto(720.14448602,65.56151254)(719.98448618,65.54151256)(719.83448975,65.50151611)
\curveto(719.75448641,65.48151262)(719.67948648,65.46151264)(719.60948975,65.44151611)
\curveto(719.54948661,65.43151267)(719.47948668,65.41651269)(719.39948975,65.39651611)
\curveto(719.23948692,65.34651276)(719.08948707,65.28151282)(718.94948975,65.20151611)
\curveto(718.80948735,65.13151297)(718.68948747,65.04151306)(718.58948975,64.93151611)
\curveto(718.48948767,64.82151328)(718.41448775,64.68651342)(718.36448975,64.52651611)
\curveto(718.31448785,64.37651373)(718.29448787,64.19151391)(718.30448975,63.97151611)
\curveto(718.30448786,63.87151423)(718.31948784,63.77651433)(718.34948975,63.68651611)
\curveto(718.38948777,63.6065145)(718.43448773,63.53151457)(718.48448975,63.46151611)
\curveto(718.5644876,63.35151475)(718.66948749,63.25651485)(718.79948975,63.17651611)
\curveto(718.92948723,63.106515)(719.06948709,63.04651506)(719.21948975,62.99651611)
\curveto(719.26948689,62.98651512)(719.31948684,62.98151512)(719.36948975,62.98151611)
\curveto(719.41948674,62.98151512)(719.46948669,62.97651513)(719.51948975,62.96651611)
\curveto(719.58948657,62.94651516)(719.67448649,62.93151517)(719.77448975,62.92151611)
\curveto(719.88448628,62.92151518)(719.97448619,62.93151517)(720.04448975,62.95151611)
\curveto(720.10448606,62.97151513)(720.164486,62.97651513)(720.22448975,62.96651611)
\curveto(720.28448588,62.96651514)(720.34448582,62.97651513)(720.40448975,62.99651611)
\curveto(720.48448568,63.01651509)(720.5594856,63.03151507)(720.62948975,63.04151611)
\curveto(720.70948545,63.05151505)(720.78448538,63.07151503)(720.85448975,63.10151611)
\curveto(721.14448502,63.22151488)(721.38948477,63.36651474)(721.58948975,63.53651611)
\curveto(721.79948436,63.7065144)(721.9594842,63.93651417)(722.06948975,64.22651611)
}
}
{
\newrgbcolor{curcolor}{0 0 0}
\pscustom[linestyle=none,fillstyle=solid,fillcolor=curcolor]
{
\newpath
\moveto(732.34613037,62.81651611)
\lineto(732.34613037,62.42651611)
\curveto(732.3461225,62.3065158)(732.32112252,62.2065159)(732.27113037,62.12651611)
\curveto(732.22112262,62.05651605)(732.13612271,62.01651609)(732.01613037,62.00651611)
\lineto(731.67113037,62.00651611)
\curveto(731.61112323,62.0065161)(731.55112329,62.0015161)(731.49113037,61.99151611)
\curveto(731.4411234,61.99151611)(731.39612345,62.0015161)(731.35613037,62.02151611)
\curveto(731.26612358,62.04151606)(731.20612364,62.08151602)(731.17613037,62.14151611)
\curveto(731.13612371,62.19151591)(731.11112373,62.25151585)(731.10113037,62.32151611)
\curveto(731.10112374,62.39151571)(731.08612376,62.46151564)(731.05613037,62.53151611)
\curveto(731.0461238,62.55151555)(731.03112381,62.56651554)(731.01113037,62.57651611)
\curveto(731.00112384,62.59651551)(730.98612386,62.61651549)(730.96613037,62.63651611)
\curveto(730.86612398,62.64651546)(730.78612406,62.62651548)(730.72613037,62.57651611)
\curveto(730.67612417,62.52651558)(730.62112422,62.47651563)(730.56113037,62.42651611)
\curveto(730.36112448,62.27651583)(730.16112468,62.16151594)(729.96113037,62.08151611)
\curveto(729.78112506,62.0015161)(729.57112527,61.94151616)(729.33113037,61.90151611)
\curveto(729.10112574,61.86151624)(728.86112598,61.84151626)(728.61113037,61.84151611)
\curveto(728.37112647,61.83151627)(728.13112671,61.84651626)(727.89113037,61.88651611)
\curveto(727.65112719,61.91651619)(727.4411274,61.97151613)(727.26113037,62.05151611)
\curveto(726.7411281,62.27151583)(726.32112852,62.56651554)(726.00113037,62.93651611)
\curveto(725.68112916,63.31651479)(725.43112941,63.78651432)(725.25113037,64.34651611)
\curveto(725.21112963,64.43651367)(725.18112966,64.52651358)(725.16113037,64.61651611)
\curveto(725.15112969,64.71651339)(725.13112971,64.81651329)(725.10113037,64.91651611)
\curveto(725.09112975,64.96651314)(725.08612976,65.01651309)(725.08613037,65.06651611)
\curveto(725.08612976,65.11651299)(725.08112976,65.16651294)(725.07113037,65.21651611)
\curveto(725.05112979,65.26651284)(725.0411298,65.31651279)(725.04113037,65.36651611)
\curveto(725.05112979,65.42651268)(725.05112979,65.48151262)(725.04113037,65.53151611)
\lineto(725.04113037,65.68151611)
\curveto(725.02112982,65.73151237)(725.01112983,65.79651231)(725.01113037,65.87651611)
\curveto(725.01112983,65.95651215)(725.02112982,66.02151208)(725.04113037,66.07151611)
\lineto(725.04113037,66.23651611)
\curveto(725.06112978,66.3065118)(725.06612978,66.37651173)(725.05613037,66.44651611)
\curveto(725.05612979,66.52651158)(725.06612978,66.6015115)(725.08613037,66.67151611)
\curveto(725.09612975,66.72151138)(725.10112974,66.76651134)(725.10113037,66.80651611)
\curveto(725.10112974,66.84651126)(725.10612974,66.89151121)(725.11613037,66.94151611)
\curveto(725.1461297,67.04151106)(725.17112967,67.13651097)(725.19113037,67.22651611)
\curveto(725.21112963,67.32651078)(725.23612961,67.42151068)(725.26613037,67.51151611)
\curveto(725.39612945,67.89151021)(725.56112928,68.23150987)(725.76113037,68.53151611)
\curveto(725.97112887,68.84150926)(726.22112862,69.09650901)(726.51113037,69.29651611)
\curveto(726.68112816,69.41650869)(726.85612799,69.51650859)(727.03613037,69.59651611)
\curveto(727.22612762,69.67650843)(727.43112741,69.74650836)(727.65113037,69.80651611)
\curveto(727.72112712,69.81650829)(727.78612706,69.82650828)(727.84613037,69.83651611)
\curveto(727.91612693,69.84650826)(727.98612686,69.86150824)(728.05613037,69.88151611)
\lineto(728.20613037,69.88151611)
\curveto(728.28612656,69.9015082)(728.40112644,69.91150819)(728.55113037,69.91151611)
\curveto(728.71112613,69.91150819)(728.83112601,69.9015082)(728.91113037,69.88151611)
\curveto(728.95112589,69.87150823)(729.00612584,69.86650824)(729.07613037,69.86651611)
\curveto(729.18612566,69.83650827)(729.29612555,69.81150829)(729.40613037,69.79151611)
\curveto(729.51612533,69.78150832)(729.62112522,69.75150835)(729.72113037,69.70151611)
\curveto(729.87112497,69.64150846)(730.01112483,69.57650853)(730.14113037,69.50651611)
\curveto(730.28112456,69.43650867)(730.41112443,69.35650875)(730.53113037,69.26651611)
\curveto(730.59112425,69.21650889)(730.65112419,69.16150894)(730.71113037,69.10151611)
\curveto(730.78112406,69.05150905)(730.87112397,69.03650907)(730.98113037,69.05651611)
\curveto(731.00112384,69.08650902)(731.01612383,69.11150899)(731.02613037,69.13151611)
\curveto(731.0461238,69.15150895)(731.06112378,69.18150892)(731.07113037,69.22151611)
\curveto(731.10112374,69.31150879)(731.11112373,69.42650868)(731.10113037,69.56651611)
\lineto(731.10113037,69.94151611)
\lineto(731.10113037,71.66651611)
\lineto(731.10113037,72.13151611)
\curveto(731.10112374,72.31150579)(731.12612372,72.44150566)(731.17613037,72.52151611)
\curveto(731.21612363,72.59150551)(731.27612357,72.63650547)(731.35613037,72.65651611)
\curveto(731.37612347,72.65650545)(731.40112344,72.65650545)(731.43113037,72.65651611)
\curveto(731.46112338,72.66650544)(731.48612336,72.67150543)(731.50613037,72.67151611)
\curveto(731.6461232,72.68150542)(731.79112305,72.68150542)(731.94113037,72.67151611)
\curveto(732.10112274,72.67150543)(732.21112263,72.63150547)(732.27113037,72.55151611)
\curveto(732.32112252,72.47150563)(732.3461225,72.37150573)(732.34613037,72.25151611)
\lineto(732.34613037,71.87651611)
\lineto(732.34613037,62.81651611)
\moveto(731.13113037,65.65151611)
\curveto(731.15112369,65.7015124)(731.16112368,65.76651234)(731.16113037,65.84651611)
\curveto(731.16112368,65.93651217)(731.15112369,66.0065121)(731.13113037,66.05651611)
\lineto(731.13113037,66.28151611)
\curveto(731.11112373,66.37151173)(731.09612375,66.46151164)(731.08613037,66.55151611)
\curveto(731.07612377,66.65151145)(731.05612379,66.74151136)(731.02613037,66.82151611)
\curveto(731.00612384,66.9015112)(730.98612386,66.97651113)(730.96613037,67.04651611)
\curveto(730.95612389,67.11651099)(730.93612391,67.18651092)(730.90613037,67.25651611)
\curveto(730.78612406,67.55651055)(730.63112421,67.82151028)(730.44113037,68.05151611)
\curveto(730.25112459,68.28150982)(730.01112483,68.46150964)(729.72113037,68.59151611)
\curveto(729.62112522,68.64150946)(729.51612533,68.67650943)(729.40613037,68.69651611)
\curveto(729.30612554,68.72650938)(729.19612565,68.75150935)(729.07613037,68.77151611)
\curveto(728.99612585,68.79150931)(728.90612594,68.8015093)(728.80613037,68.80151611)
\lineto(728.53613037,68.80151611)
\curveto(728.48612636,68.79150931)(728.4411264,68.78150932)(728.40113037,68.77151611)
\lineto(728.26613037,68.77151611)
\curveto(728.18612666,68.75150935)(728.10112674,68.73150937)(728.01113037,68.71151611)
\curveto(727.93112691,68.69150941)(727.85112699,68.66650944)(727.77113037,68.63651611)
\curveto(727.45112739,68.49650961)(727.19112765,68.29150981)(726.99113037,68.02151611)
\curveto(726.80112804,67.76151034)(726.6461282,67.45651065)(726.52613037,67.10651611)
\curveto(726.48612836,66.99651111)(726.45612839,66.88151122)(726.43613037,66.76151611)
\curveto(726.42612842,66.65151145)(726.41112843,66.54151156)(726.39113037,66.43151611)
\curveto(726.39112845,66.39151171)(726.38612846,66.35151175)(726.37613037,66.31151611)
\lineto(726.37613037,66.20651611)
\curveto(726.35612849,66.15651195)(726.3461285,66.101512)(726.34613037,66.04151611)
\curveto(726.35612849,65.98151212)(726.36112848,65.92651218)(726.36113037,65.87651611)
\lineto(726.36113037,65.54651611)
\curveto(726.36112848,65.44651266)(726.37112847,65.35151275)(726.39113037,65.26151611)
\curveto(726.40112844,65.23151287)(726.40612844,65.18151292)(726.40613037,65.11151611)
\curveto(726.42612842,65.04151306)(726.4411284,64.97151313)(726.45113037,64.90151611)
\lineto(726.51113037,64.69151611)
\curveto(726.62112822,64.34151376)(726.77112807,64.04151406)(726.96113037,63.79151611)
\curveto(727.15112769,63.54151456)(727.39112745,63.33651477)(727.68113037,63.17651611)
\curveto(727.77112707,63.12651498)(727.86112698,63.08651502)(727.95113037,63.05651611)
\curveto(728.0411268,63.02651508)(728.1411267,62.99651511)(728.25113037,62.96651611)
\curveto(728.30112654,62.94651516)(728.35112649,62.94151516)(728.40113037,62.95151611)
\curveto(728.46112638,62.96151514)(728.51612633,62.95651515)(728.56613037,62.93651611)
\curveto(728.60612624,62.92651518)(728.6461262,62.92151518)(728.68613037,62.92151611)
\lineto(728.82113037,62.92151611)
\lineto(728.95613037,62.92151611)
\curveto(728.98612586,62.93151517)(729.03612581,62.93651517)(729.10613037,62.93651611)
\curveto(729.18612566,62.95651515)(729.26612558,62.97151513)(729.34613037,62.98151611)
\curveto(729.42612542,63.0015151)(729.50112534,63.02651508)(729.57113037,63.05651611)
\curveto(729.90112494,63.19651491)(730.16612468,63.37151473)(730.36613037,63.58151611)
\curveto(730.57612427,63.8015143)(730.75112409,64.07651403)(730.89113037,64.40651611)
\curveto(730.9411239,64.51651359)(730.97612387,64.62651348)(730.99613037,64.73651611)
\curveto(731.01612383,64.84651326)(731.0411238,64.95651315)(731.07113037,65.06651611)
\curveto(731.09112375,65.106513)(731.10112374,65.14151296)(731.10113037,65.17151611)
\curveto(731.10112374,65.21151289)(731.10612374,65.25151285)(731.11613037,65.29151611)
\curveto(731.12612372,65.35151275)(731.12612372,65.41151269)(731.11613037,65.47151611)
\curveto(731.11612373,65.53151257)(731.12112372,65.59151251)(731.13113037,65.65151611)
}
}
{
\newrgbcolor{curcolor}{0 0 0}
\pscustom[linestyle=none,fillstyle=solid,fillcolor=curcolor]
{
\newpath
\moveto(741.41738037,66.20651611)
\curveto(741.43737231,66.14651196)(741.4473723,66.05151205)(741.44738037,65.92151611)
\curveto(741.4473723,65.8015123)(741.44237231,65.71651239)(741.43238037,65.66651611)
\lineto(741.43238037,65.51651611)
\curveto(741.42237233,65.43651267)(741.41237234,65.36151274)(741.40238037,65.29151611)
\curveto(741.40237235,65.23151287)(741.39737235,65.16151294)(741.38738037,65.08151611)
\curveto(741.36737238,65.02151308)(741.3523724,64.96151314)(741.34238037,64.90151611)
\curveto(741.34237241,64.84151326)(741.33237242,64.78151332)(741.31238037,64.72151611)
\curveto(741.27237248,64.59151351)(741.23737251,64.46151364)(741.20738037,64.33151611)
\curveto(741.17737257,64.2015139)(741.13737261,64.08151402)(741.08738037,63.97151611)
\curveto(740.87737287,63.49151461)(740.59737315,63.08651502)(740.24738037,62.75651611)
\curveto(739.89737385,62.43651567)(739.46737428,62.19151591)(738.95738037,62.02151611)
\curveto(738.8473749,61.98151612)(738.72737502,61.95151615)(738.59738037,61.93151611)
\curveto(738.47737527,61.91151619)(738.3523754,61.89151621)(738.22238037,61.87151611)
\curveto(738.16237559,61.86151624)(738.09737565,61.85651625)(738.02738037,61.85651611)
\curveto(737.96737578,61.84651626)(737.90737584,61.84151626)(737.84738037,61.84151611)
\curveto(737.80737594,61.83151627)(737.747376,61.82651628)(737.66738037,61.82651611)
\curveto(737.59737615,61.82651628)(737.5473762,61.83151627)(737.51738037,61.84151611)
\curveto(737.47737627,61.85151625)(737.43737631,61.85651625)(737.39738037,61.85651611)
\curveto(737.35737639,61.84651626)(737.32237643,61.84651626)(737.29238037,61.85651611)
\lineto(737.20238037,61.85651611)
\lineto(736.84238037,61.90151611)
\curveto(736.70237705,61.94151616)(736.56737718,61.98151612)(736.43738037,62.02151611)
\curveto(736.30737744,62.06151604)(736.18237757,62.106516)(736.06238037,62.15651611)
\curveto(735.61237814,62.35651575)(735.24237851,62.61651549)(734.95238037,62.93651611)
\curveto(734.66237909,63.25651485)(734.42237933,63.64651446)(734.23238037,64.10651611)
\curveto(734.18237957,64.2065139)(734.14237961,64.3065138)(734.11238037,64.40651611)
\curveto(734.09237966,64.5065136)(734.07237968,64.61151349)(734.05238037,64.72151611)
\curveto(734.03237972,64.76151334)(734.02237973,64.79151331)(734.02238037,64.81151611)
\curveto(734.03237972,64.84151326)(734.03237972,64.87651323)(734.02238037,64.91651611)
\curveto(734.00237975,64.99651311)(733.98737976,65.07651303)(733.97738037,65.15651611)
\curveto(733.97737977,65.24651286)(733.96737978,65.33151277)(733.94738037,65.41151611)
\lineto(733.94738037,65.53151611)
\curveto(733.9473798,65.57151253)(733.94237981,65.61651249)(733.93238037,65.66651611)
\curveto(733.92237983,65.71651239)(733.91737983,65.8015123)(733.91738037,65.92151611)
\curveto(733.91737983,66.05151205)(733.92737982,66.14651196)(733.94738037,66.20651611)
\curveto(733.96737978,66.27651183)(733.97237978,66.34651176)(733.96238037,66.41651611)
\curveto(733.9523798,66.48651162)(733.95737979,66.55651155)(733.97738037,66.62651611)
\curveto(733.98737976,66.67651143)(733.99237976,66.71651139)(733.99238037,66.74651611)
\curveto(734.00237975,66.78651132)(734.01237974,66.83151127)(734.02238037,66.88151611)
\curveto(734.0523797,67.0015111)(734.07737967,67.12151098)(734.09738037,67.24151611)
\curveto(734.12737962,67.36151074)(734.16737958,67.47651063)(734.21738037,67.58651611)
\curveto(734.36737938,67.95651015)(734.5473792,68.28650982)(734.75738037,68.57651611)
\curveto(734.97737877,68.87650923)(735.24237851,69.12650898)(735.55238037,69.32651611)
\curveto(735.67237808,69.4065087)(735.79737795,69.47150863)(735.92738037,69.52151611)
\curveto(736.05737769,69.58150852)(736.19237756,69.64150846)(736.33238037,69.70151611)
\curveto(736.4523773,69.75150835)(736.58237717,69.78150832)(736.72238037,69.79151611)
\curveto(736.86237689,69.81150829)(737.00237675,69.84150826)(737.14238037,69.88151611)
\lineto(737.33738037,69.88151611)
\curveto(737.40737634,69.89150821)(737.47237628,69.9015082)(737.53238037,69.91151611)
\curveto(738.42237533,69.92150818)(739.16237459,69.73650837)(739.75238037,69.35651611)
\curveto(740.34237341,68.97650913)(740.76737298,68.48150962)(741.02738037,67.87151611)
\curveto(741.07737267,67.77151033)(741.11737263,67.67151043)(741.14738037,67.57151611)
\curveto(741.17737257,67.47151063)(741.21237254,67.36651074)(741.25238037,67.25651611)
\curveto(741.28237247,67.14651096)(741.30737244,67.02651108)(741.32738037,66.89651611)
\curveto(741.3473724,66.77651133)(741.37237238,66.65151145)(741.40238037,66.52151611)
\curveto(741.41237234,66.47151163)(741.41237234,66.41651169)(741.40238037,66.35651611)
\curveto(741.40237235,66.3065118)(741.40737234,66.25651185)(741.41738037,66.20651611)
\moveto(740.08238037,65.35151611)
\curveto(740.10237365,65.42151268)(740.10737364,65.5015126)(740.09738037,65.59151611)
\lineto(740.09738037,65.84651611)
\curveto(740.09737365,66.23651187)(740.06237369,66.56651154)(739.99238037,66.83651611)
\curveto(739.96237379,66.91651119)(739.93737381,66.99651111)(739.91738037,67.07651611)
\curveto(739.89737385,67.15651095)(739.87237388,67.23151087)(739.84238037,67.30151611)
\curveto(739.56237419,67.95151015)(739.11737463,68.4015097)(738.50738037,68.65151611)
\curveto(738.43737531,68.68150942)(738.36237539,68.7015094)(738.28238037,68.71151611)
\lineto(738.04238037,68.77151611)
\curveto(737.96237579,68.79150931)(737.87737587,68.8015093)(737.78738037,68.80151611)
\lineto(737.51738037,68.80151611)
\lineto(737.24738037,68.75651611)
\curveto(737.1473766,68.73650937)(737.0523767,68.71150939)(736.96238037,68.68151611)
\curveto(736.88237687,68.66150944)(736.80237695,68.63150947)(736.72238037,68.59151611)
\curveto(736.6523771,68.57150953)(736.58737716,68.54150956)(736.52738037,68.50151611)
\curveto(736.46737728,68.46150964)(736.41237734,68.42150968)(736.36238037,68.38151611)
\curveto(736.12237763,68.21150989)(735.92737782,68.0065101)(735.77738037,67.76651611)
\curveto(735.62737812,67.52651058)(735.49737825,67.24651086)(735.38738037,66.92651611)
\curveto(735.35737839,66.82651128)(735.33737841,66.72151138)(735.32738037,66.61151611)
\curveto(735.31737843,66.51151159)(735.30237845,66.4065117)(735.28238037,66.29651611)
\curveto(735.27237848,66.25651185)(735.26737848,66.19151191)(735.26738037,66.10151611)
\curveto(735.25737849,66.07151203)(735.2523785,66.03651207)(735.25238037,65.99651611)
\curveto(735.26237849,65.95651215)(735.26737848,65.91151219)(735.26738037,65.86151611)
\lineto(735.26738037,65.56151611)
\curveto(735.26737848,65.46151264)(735.27737847,65.37151273)(735.29738037,65.29151611)
\lineto(735.32738037,65.11151611)
\curveto(735.3473784,65.01151309)(735.36237839,64.91151319)(735.37238037,64.81151611)
\curveto(735.39237836,64.72151338)(735.42237833,64.63651347)(735.46238037,64.55651611)
\curveto(735.56237819,64.31651379)(735.67737807,64.09151401)(735.80738037,63.88151611)
\curveto(735.9473778,63.67151443)(736.11737763,63.49651461)(736.31738037,63.35651611)
\curveto(736.36737738,63.32651478)(736.41237734,63.3015148)(736.45238037,63.28151611)
\curveto(736.49237726,63.26151484)(736.53737721,63.23651487)(736.58738037,63.20651611)
\curveto(736.66737708,63.15651495)(736.752377,63.11151499)(736.84238037,63.07151611)
\curveto(736.94237681,63.04151506)(737.0473767,63.01151509)(737.15738037,62.98151611)
\curveto(737.20737654,62.96151514)(737.2523765,62.95151515)(737.29238037,62.95151611)
\curveto(737.34237641,62.96151514)(737.39237636,62.96151514)(737.44238037,62.95151611)
\curveto(737.47237628,62.94151516)(737.53237622,62.93151517)(737.62238037,62.92151611)
\curveto(737.72237603,62.91151519)(737.79737595,62.91651519)(737.84738037,62.93651611)
\curveto(737.88737586,62.94651516)(737.92737582,62.94651516)(737.96738037,62.93651611)
\curveto(738.00737574,62.93651517)(738.0473757,62.94651516)(738.08738037,62.96651611)
\curveto(738.16737558,62.98651512)(738.2473755,63.0015151)(738.32738037,63.01151611)
\curveto(738.40737534,63.03151507)(738.48237527,63.05651505)(738.55238037,63.08651611)
\curveto(738.89237486,63.22651488)(739.16737458,63.42151468)(739.37738037,63.67151611)
\curveto(739.58737416,63.92151418)(739.76237399,64.21651389)(739.90238037,64.55651611)
\curveto(739.9523738,64.67651343)(739.98237377,64.8015133)(739.99238037,64.93151611)
\curveto(740.01237374,65.07151303)(740.04237371,65.21151289)(740.08238037,65.35151611)
}
}
{
\newrgbcolor{curcolor}{0 0 0}
\pscustom[linestyle=none,fillstyle=solid,fillcolor=curcolor]
{
\newpath
\moveto(746.55066162,69.91151611)
\curveto(746.78065683,69.91150819)(746.9106567,69.85150825)(746.94066162,69.73151611)
\curveto(746.97065664,69.62150848)(746.98565663,69.45650865)(746.98566162,69.23651611)
\lineto(746.98566162,68.95151611)
\curveto(746.98565663,68.86150924)(746.96065665,68.78650932)(746.91066162,68.72651611)
\curveto(746.85065676,68.64650946)(746.76565685,68.6015095)(746.65566162,68.59151611)
\curveto(746.54565707,68.59150951)(746.43565718,68.57650953)(746.32566162,68.54651611)
\curveto(746.18565743,68.51650959)(746.05065756,68.48650962)(745.92066162,68.45651611)
\curveto(745.80065781,68.42650968)(745.68565793,68.38650972)(745.57566162,68.33651611)
\curveto(745.28565833,68.2065099)(745.05065856,68.02651008)(744.87066162,67.79651611)
\curveto(744.69065892,67.57651053)(744.53565908,67.32151078)(744.40566162,67.03151611)
\curveto(744.36565925,66.92151118)(744.33565928,66.8065113)(744.31566162,66.68651611)
\curveto(744.29565932,66.57651153)(744.27065934,66.46151164)(744.24066162,66.34151611)
\curveto(744.23065938,66.29151181)(744.22565939,66.24151186)(744.22566162,66.19151611)
\curveto(744.23565938,66.14151196)(744.23565938,66.09151201)(744.22566162,66.04151611)
\curveto(744.19565942,65.92151218)(744.18065943,65.78151232)(744.18066162,65.62151611)
\curveto(744.19065942,65.47151263)(744.19565942,65.32651278)(744.19566162,65.18651611)
\lineto(744.19566162,63.34151611)
\lineto(744.19566162,62.99651611)
\curveto(744.19565942,62.87651523)(744.19065942,62.76151534)(744.18066162,62.65151611)
\curveto(744.17065944,62.54151556)(744.16565945,62.44651566)(744.16566162,62.36651611)
\curveto(744.17565944,62.28651582)(744.15565946,62.21651589)(744.10566162,62.15651611)
\curveto(744.05565956,62.08651602)(743.97565964,62.04651606)(743.86566162,62.03651611)
\curveto(743.76565985,62.02651608)(743.65565996,62.02151608)(743.53566162,62.02151611)
\lineto(743.26566162,62.02151611)
\curveto(743.2156604,62.04151606)(743.16566045,62.05651605)(743.11566162,62.06651611)
\curveto(743.07566054,62.08651602)(743.04566057,62.11151599)(743.02566162,62.14151611)
\curveto(742.97566064,62.21151589)(742.94566067,62.29651581)(742.93566162,62.39651611)
\lineto(742.93566162,62.72651611)
\lineto(742.93566162,63.88151611)
\lineto(742.93566162,68.03651611)
\lineto(742.93566162,69.07151611)
\lineto(742.93566162,69.37151611)
\curveto(742.94566067,69.47150863)(742.97566064,69.55650855)(743.02566162,69.62651611)
\curveto(743.05566056,69.66650844)(743.10566051,69.69650841)(743.17566162,69.71651611)
\curveto(743.25566036,69.73650837)(743.34066027,69.74650836)(743.43066162,69.74651611)
\curveto(743.52066009,69.75650835)(743.61066,69.75650835)(743.70066162,69.74651611)
\curveto(743.79065982,69.73650837)(743.86065975,69.72150838)(743.91066162,69.70151611)
\curveto(743.99065962,69.67150843)(744.04065957,69.61150849)(744.06066162,69.52151611)
\curveto(744.09065952,69.44150866)(744.10565951,69.35150875)(744.10566162,69.25151611)
\lineto(744.10566162,68.95151611)
\curveto(744.10565951,68.85150925)(744.12565949,68.76150934)(744.16566162,68.68151611)
\curveto(744.17565944,68.66150944)(744.18565943,68.64650946)(744.19566162,68.63651611)
\lineto(744.24066162,68.59151611)
\curveto(744.35065926,68.59150951)(744.44065917,68.63650947)(744.51066162,68.72651611)
\curveto(744.58065903,68.82650928)(744.64065897,68.9065092)(744.69066162,68.96651611)
\lineto(744.78066162,69.05651611)
\curveto(744.87065874,69.16650894)(744.99565862,69.28150882)(745.15566162,69.40151611)
\curveto(745.3156583,69.52150858)(745.46565815,69.61150849)(745.60566162,69.67151611)
\curveto(745.69565792,69.72150838)(745.79065782,69.75650835)(745.89066162,69.77651611)
\curveto(745.99065762,69.8065083)(746.09565752,69.83650827)(746.20566162,69.86651611)
\curveto(746.26565735,69.87650823)(746.32565729,69.88150822)(746.38566162,69.88151611)
\curveto(746.44565717,69.89150821)(746.50065711,69.9015082)(746.55066162,69.91151611)
}
}
{
\newrgbcolor{curcolor}{0 0 0}
\pscustom[linestyle=none,fillstyle=solid,fillcolor=curcolor]
{
\newpath
\moveto(208.47920532,54.82939941)
\lineto(213.38420532,54.82939941)
\lineto(214.67420532,54.82939941)
\curveto(214.78419744,54.82938872)(214.89419733,54.82938872)(215.00420532,54.82939941)
\curveto(215.11419711,54.83938871)(215.20419702,54.81938873)(215.27420532,54.76939941)
\curveto(215.30419692,54.7493888)(215.3291969,54.72438882)(215.34920532,54.69439941)
\curveto(215.36919686,54.66438888)(215.38919684,54.63438891)(215.40920532,54.60439941)
\curveto(215.4291968,54.53438901)(215.43919679,54.41938913)(215.43920532,54.25939941)
\curveto(215.43919679,54.10938944)(215.4291968,53.99438955)(215.40920532,53.91439941)
\curveto(215.36919686,53.77438977)(215.28419694,53.69438985)(215.15420532,53.67439941)
\curveto(215.0241972,53.66438988)(214.86919736,53.65938989)(214.68920532,53.65939941)
\lineto(213.18920532,53.65939941)
\lineto(210.66920532,53.65939941)
\lineto(210.09920532,53.65939941)
\curveto(209.88920234,53.66938988)(209.73420249,53.6443899)(209.63420532,53.58439941)
\curveto(209.53420269,53.52439002)(209.47920275,53.41939013)(209.46920532,53.26939941)
\lineto(209.46920532,52.80439941)
\lineto(209.46920532,51.27439941)
\curveto(209.46920276,51.16439238)(209.46420276,51.03439251)(209.45420532,50.88439941)
\curveto(209.45420277,50.73439281)(209.46420276,50.61439293)(209.48420532,50.52439941)
\curveto(209.51420271,50.40439314)(209.57420265,50.32439322)(209.66420532,50.28439941)
\curveto(209.70420252,50.26439328)(209.77420245,50.2443933)(209.87420532,50.22439941)
\lineto(210.02420532,50.22439941)
\curveto(210.06420216,50.21439333)(210.10420212,50.20939334)(210.14420532,50.20939941)
\curveto(210.19420203,50.21939333)(210.24420198,50.22439332)(210.29420532,50.22439941)
\lineto(210.80420532,50.22439941)
\lineto(213.74420532,50.22439941)
\lineto(214.04420532,50.22439941)
\curveto(214.15419807,50.23439331)(214.26419796,50.23439331)(214.37420532,50.22439941)
\curveto(214.49419773,50.22439332)(214.59919763,50.21439333)(214.68920532,50.19439941)
\curveto(214.78919744,50.18439336)(214.86419736,50.16439338)(214.91420532,50.13439941)
\curveto(214.94419728,50.11439343)(214.96919726,50.06939348)(214.98920532,49.99939941)
\curveto(215.00919722,49.92939362)(215.0241972,49.85439369)(215.03420532,49.77439941)
\curveto(215.04419718,49.69439385)(215.04419718,49.60939394)(215.03420532,49.51939941)
\curveto(215.03419719,49.43939411)(215.0241972,49.36939418)(215.00420532,49.30939941)
\curveto(214.98419724,49.21939433)(214.93919729,49.15439439)(214.86920532,49.11439941)
\curveto(214.84919738,49.09439445)(214.81919741,49.07939447)(214.77920532,49.06939941)
\curveto(214.74919748,49.06939448)(214.71919751,49.06439448)(214.68920532,49.05439941)
\lineto(214.59920532,49.05439941)
\curveto(214.54919768,49.0443945)(214.49919773,49.03939451)(214.44920532,49.03939941)
\curveto(214.39919783,49.0493945)(214.34919788,49.05439449)(214.29920532,49.05439941)
\lineto(213.74420532,49.05439941)
\lineto(210.57920532,49.05439941)
\lineto(210.21920532,49.05439941)
\curveto(210.10920212,49.06439448)(210.00420222,49.05939449)(209.90420532,49.03939941)
\curveto(209.80420242,49.02939452)(209.71420251,49.00439454)(209.63420532,48.96439941)
\curveto(209.56420266,48.92439462)(209.51420271,48.85439469)(209.48420532,48.75439941)
\curveto(209.46420276,48.69439485)(209.45420277,48.62439492)(209.45420532,48.54439941)
\curveto(209.46420276,48.46439508)(209.46920276,48.38439516)(209.46920532,48.30439941)
\lineto(209.46920532,47.46439941)
\lineto(209.46920532,46.03939941)
\curveto(209.46920276,45.89939765)(209.47420275,45.76939778)(209.48420532,45.64939941)
\curveto(209.49420273,45.53939801)(209.53420269,45.45939809)(209.60420532,45.40939941)
\curveto(209.67420255,45.35939819)(209.75420247,45.32939822)(209.84420532,45.31939941)
\lineto(210.14420532,45.31939941)
\lineto(211.10420532,45.31939941)
\lineto(213.87920532,45.31939941)
\lineto(214.73420532,45.31939941)
\lineto(214.97420532,45.31939941)
\curveto(215.05419717,45.32939822)(215.1241971,45.32439822)(215.18420532,45.30439941)
\curveto(215.30419692,45.26439828)(215.38419684,45.20939834)(215.42420532,45.13939941)
\curveto(215.44419678,45.10939844)(215.45919677,45.05939849)(215.46920532,44.98939941)
\curveto(215.47919675,44.91939863)(215.48419674,44.8443987)(215.48420532,44.76439941)
\curveto(215.49419673,44.69439885)(215.49419673,44.61939893)(215.48420532,44.53939941)
\curveto(215.47419675,44.46939908)(215.46419676,44.41439913)(215.45420532,44.37439941)
\curveto(215.41419681,44.29439925)(215.36919686,44.23939931)(215.31920532,44.20939941)
\curveto(215.25919697,44.16939938)(215.17919705,44.1493994)(215.07920532,44.14939941)
\lineto(214.80920532,44.14939941)
\lineto(213.75920532,44.14939941)
\lineto(209.76920532,44.14939941)
\lineto(208.71920532,44.14939941)
\curveto(208.57920365,44.1493994)(208.45920377,44.15439939)(208.35920532,44.16439941)
\curveto(208.25920397,44.18439936)(208.18420404,44.23439931)(208.13420532,44.31439941)
\curveto(208.09420413,44.37439917)(208.07420415,44.4493991)(208.07420532,44.53939941)
\lineto(208.07420532,44.82439941)
\lineto(208.07420532,45.87439941)
\lineto(208.07420532,49.89439941)
\lineto(208.07420532,53.25439941)
\lineto(208.07420532,54.18439941)
\lineto(208.07420532,54.45439941)
\curveto(208.07420415,54.544389)(208.09420413,54.61438893)(208.13420532,54.66439941)
\curveto(208.17420405,54.73438881)(208.24920398,54.78438876)(208.35920532,54.81439941)
\curveto(208.37920385,54.82438872)(208.39920383,54.82438872)(208.41920532,54.81439941)
\curveto(208.43920379,54.81438873)(208.45920377,54.81938873)(208.47920532,54.82939941)
}
}
{
\newrgbcolor{curcolor}{0 0 0}
\pscustom[linestyle=none,fillstyle=solid,fillcolor=curcolor]
{
\newpath
\moveto(219.4191272,52.05439941)
\curveto(220.13912313,52.06439148)(220.74412253,51.97939157)(221.2341272,51.79939941)
\curveto(221.72412155,51.62939192)(222.10412117,51.32439222)(222.3741272,50.88439941)
\curveto(222.44412083,50.77439277)(222.49912077,50.65939289)(222.5391272,50.53939941)
\curveto(222.57912069,50.42939312)(222.61912065,50.30439324)(222.6591272,50.16439941)
\curveto(222.67912059,50.09439345)(222.68412059,50.01939353)(222.6741272,49.93939941)
\curveto(222.66412061,49.86939368)(222.64912062,49.81439373)(222.6291272,49.77439941)
\curveto(222.60912066,49.75439379)(222.58412069,49.73439381)(222.5541272,49.71439941)
\curveto(222.52412075,49.70439384)(222.49912077,49.68939386)(222.4791272,49.66939941)
\curveto(222.42912084,49.6493939)(222.37912089,49.6443939)(222.3291272,49.65439941)
\curveto(222.27912099,49.66439388)(222.22912104,49.66439388)(222.1791272,49.65439941)
\curveto(222.09912117,49.63439391)(221.99412128,49.62939392)(221.8641272,49.63939941)
\curveto(221.73412154,49.65939389)(221.64412163,49.68439386)(221.5941272,49.71439941)
\curveto(221.51412176,49.76439378)(221.45912181,49.82939372)(221.4291272,49.90939941)
\curveto(221.40912186,49.99939355)(221.3741219,50.08439346)(221.3241272,50.16439941)
\curveto(221.23412204,50.32439322)(221.10912216,50.46939308)(220.9491272,50.59939941)
\curveto(220.83912243,50.67939287)(220.71912255,50.73939281)(220.5891272,50.77939941)
\curveto(220.45912281,50.81939273)(220.31912295,50.85939269)(220.1691272,50.89939941)
\curveto(220.11912315,50.91939263)(220.0691232,50.92439262)(220.0191272,50.91439941)
\curveto(219.9691233,50.91439263)(219.91912335,50.91939263)(219.8691272,50.92939941)
\curveto(219.80912346,50.9493926)(219.73412354,50.95939259)(219.6441272,50.95939941)
\curveto(219.55412372,50.95939259)(219.47912379,50.9493926)(219.4191272,50.92939941)
\lineto(219.3291272,50.92939941)
\lineto(219.1791272,50.89939941)
\curveto(219.12912414,50.89939265)(219.07912419,50.89439265)(219.0291272,50.88439941)
\curveto(218.7691245,50.82439272)(218.55412472,50.73939281)(218.3841272,50.62939941)
\curveto(218.21412506,50.51939303)(218.09912517,50.33439321)(218.0391272,50.07439941)
\curveto(218.01912525,50.00439354)(218.01412526,49.93439361)(218.0241272,49.86439941)
\curveto(218.04412523,49.79439375)(218.06412521,49.73439381)(218.0841272,49.68439941)
\curveto(218.14412513,49.53439401)(218.21412506,49.42439412)(218.2941272,49.35439941)
\curveto(218.38412489,49.29439425)(218.49412478,49.22439432)(218.6241272,49.14439941)
\curveto(218.78412449,49.0443945)(218.96412431,48.96939458)(219.1641272,48.91939941)
\curveto(219.36412391,48.87939467)(219.56412371,48.82939472)(219.7641272,48.76939941)
\curveto(219.89412338,48.72939482)(220.02412325,48.69939485)(220.1541272,48.67939941)
\curveto(220.28412299,48.65939489)(220.41412286,48.62939492)(220.5441272,48.58939941)
\curveto(220.75412252,48.52939502)(220.95912231,48.46939508)(221.1591272,48.40939941)
\curveto(221.35912191,48.35939519)(221.55912171,48.29439525)(221.7591272,48.21439941)
\lineto(221.9091272,48.15439941)
\curveto(221.95912131,48.13439541)(222.00912126,48.10939544)(222.0591272,48.07939941)
\curveto(222.25912101,47.95939559)(222.43412084,47.82439572)(222.5841272,47.67439941)
\curveto(222.73412054,47.52439602)(222.85912041,47.33439621)(222.9591272,47.10439941)
\curveto(222.97912029,47.03439651)(222.99912027,46.93939661)(223.0191272,46.81939941)
\curveto(223.03912023,46.7493968)(223.04912022,46.67439687)(223.0491272,46.59439941)
\curveto(223.05912021,46.52439702)(223.06412021,46.4443971)(223.0641272,46.35439941)
\lineto(223.0641272,46.20439941)
\curveto(223.04412023,46.13439741)(223.03412024,46.06439748)(223.0341272,45.99439941)
\curveto(223.03412024,45.92439762)(223.02412025,45.85439769)(223.0041272,45.78439941)
\curveto(222.9741203,45.67439787)(222.93912033,45.56939798)(222.8991272,45.46939941)
\curveto(222.85912041,45.36939818)(222.81412046,45.27939827)(222.7641272,45.19939941)
\curveto(222.60412067,44.93939861)(222.39912087,44.72939882)(222.1491272,44.56939941)
\curveto(221.89912137,44.41939913)(221.61912165,44.28939926)(221.3091272,44.17939941)
\curveto(221.21912205,44.1493994)(221.12412215,44.12939942)(221.0241272,44.11939941)
\curveto(220.93412234,44.09939945)(220.84412243,44.07439947)(220.7541272,44.04439941)
\curveto(220.65412262,44.02439952)(220.55412272,44.01439953)(220.4541272,44.01439941)
\curveto(220.35412292,44.01439953)(220.25412302,44.00439954)(220.1541272,43.98439941)
\lineto(220.0041272,43.98439941)
\curveto(219.95412332,43.97439957)(219.88412339,43.96939958)(219.7941272,43.96939941)
\curveto(219.70412357,43.96939958)(219.63412364,43.97439957)(219.5841272,43.98439941)
\lineto(219.4191272,43.98439941)
\curveto(219.35912391,44.00439954)(219.29412398,44.01439953)(219.2241272,44.01439941)
\curveto(219.15412412,44.00439954)(219.09412418,44.00939954)(219.0441272,44.02939941)
\curveto(218.99412428,44.03939951)(218.92912434,44.0443995)(218.8491272,44.04439941)
\lineto(218.6091272,44.10439941)
\curveto(218.53912473,44.11439943)(218.46412481,44.13439941)(218.3841272,44.16439941)
\curveto(218.0741252,44.26439928)(217.80412547,44.38939916)(217.5741272,44.53939941)
\curveto(217.34412593,44.68939886)(217.14412613,44.88439866)(216.9741272,45.12439941)
\curveto(216.88412639,45.25439829)(216.80912646,45.38939816)(216.7491272,45.52939941)
\curveto(216.68912658,45.66939788)(216.63412664,45.82439772)(216.5841272,45.99439941)
\curveto(216.56412671,46.05439749)(216.55412672,46.12439742)(216.5541272,46.20439941)
\curveto(216.56412671,46.29439725)(216.57912669,46.36439718)(216.5991272,46.41439941)
\curveto(216.62912664,46.45439709)(216.67912659,46.49439705)(216.7491272,46.53439941)
\curveto(216.79912647,46.55439699)(216.8691264,46.56439698)(216.9591272,46.56439941)
\curveto(217.04912622,46.57439697)(217.13912613,46.57439697)(217.2291272,46.56439941)
\curveto(217.31912595,46.55439699)(217.40412587,46.53939701)(217.4841272,46.51939941)
\curveto(217.5741257,46.50939704)(217.63412564,46.49439705)(217.6641272,46.47439941)
\curveto(217.73412554,46.42439712)(217.77912549,46.3493972)(217.7991272,46.24939941)
\curveto(217.82912544,46.15939739)(217.86412541,46.07439747)(217.9041272,45.99439941)
\curveto(218.00412527,45.77439777)(218.13912513,45.60439794)(218.3091272,45.48439941)
\curveto(218.42912484,45.39439815)(218.56412471,45.32439822)(218.7141272,45.27439941)
\curveto(218.86412441,45.22439832)(219.02412425,45.17439837)(219.1941272,45.12439941)
\lineto(219.5091272,45.07939941)
\lineto(219.5991272,45.07939941)
\curveto(219.6691236,45.05939849)(219.75912351,45.0493985)(219.8691272,45.04939941)
\curveto(219.98912328,45.0493985)(220.08912318,45.05939849)(220.1691272,45.07939941)
\curveto(220.23912303,45.07939847)(220.29412298,45.08439846)(220.3341272,45.09439941)
\curveto(220.39412288,45.10439844)(220.45412282,45.10939844)(220.5141272,45.10939941)
\curveto(220.5741227,45.11939843)(220.62912264,45.12939842)(220.6791272,45.13939941)
\curveto(220.9691223,45.21939833)(221.19912207,45.32439822)(221.3691272,45.45439941)
\curveto(221.53912173,45.58439796)(221.65912161,45.80439774)(221.7291272,46.11439941)
\curveto(221.74912152,46.16439738)(221.75412152,46.21939733)(221.7441272,46.27939941)
\curveto(221.73412154,46.33939721)(221.72412155,46.38439716)(221.7141272,46.41439941)
\curveto(221.66412161,46.60439694)(221.59412168,46.7443968)(221.5041272,46.83439941)
\curveto(221.41412186,46.93439661)(221.29912197,47.02439652)(221.1591272,47.10439941)
\curveto(221.0691222,47.16439638)(220.9691223,47.21439633)(220.8591272,47.25439941)
\lineto(220.5291272,47.37439941)
\curveto(220.49912277,47.38439616)(220.4691228,47.38939616)(220.4391272,47.38939941)
\curveto(220.41912285,47.38939616)(220.39412288,47.39939615)(220.3641272,47.41939941)
\curveto(220.02412325,47.52939602)(219.6691236,47.60939594)(219.2991272,47.65939941)
\curveto(218.93912433,47.71939583)(218.59912467,47.81439573)(218.2791272,47.94439941)
\curveto(218.17912509,47.98439556)(218.08412519,48.01939553)(217.9941272,48.04939941)
\curveto(217.90412537,48.07939547)(217.81912545,48.11939543)(217.7391272,48.16939941)
\curveto(217.54912572,48.27939527)(217.3741259,48.40439514)(217.2141272,48.54439941)
\curveto(217.05412622,48.68439486)(216.92912634,48.85939469)(216.8391272,49.06939941)
\curveto(216.80912646,49.13939441)(216.78412649,49.20939434)(216.7641272,49.27939941)
\curveto(216.75412652,49.3493942)(216.73912653,49.42439412)(216.7191272,49.50439941)
\curveto(216.68912658,49.62439392)(216.67912659,49.75939379)(216.6891272,49.90939941)
\curveto(216.69912657,50.06939348)(216.71412656,50.20439334)(216.7341272,50.31439941)
\curveto(216.75412652,50.36439318)(216.76412651,50.40439314)(216.7641272,50.43439941)
\curveto(216.7741265,50.47439307)(216.78912648,50.51439303)(216.8091272,50.55439941)
\curveto(216.89912637,50.78439276)(217.01912625,50.98439256)(217.1691272,51.15439941)
\curveto(217.32912594,51.32439222)(217.50912576,51.47439207)(217.7091272,51.60439941)
\curveto(217.85912541,51.69439185)(218.02412525,51.76439178)(218.2041272,51.81439941)
\curveto(218.38412489,51.87439167)(218.5741247,51.92939162)(218.7741272,51.97939941)
\curveto(218.84412443,51.98939156)(218.90912436,51.99939155)(218.9691272,52.00939941)
\curveto(219.03912423,52.01939153)(219.11412416,52.02939152)(219.1941272,52.03939941)
\curveto(219.22412405,52.0493915)(219.26412401,52.0493915)(219.3141272,52.03939941)
\curveto(219.36412391,52.02939152)(219.39912387,52.03439151)(219.4191272,52.05439941)
}
}
{
\newrgbcolor{curcolor}{0 0 0}
\pscustom[linestyle=none,fillstyle=solid,fillcolor=curcolor]
{
\newpath
\moveto(225.4341272,54.21439941)
\curveto(225.58412519,54.21438933)(225.73412504,54.20938934)(225.8841272,54.19939941)
\curveto(226.03412474,54.19938935)(226.13912463,54.15938939)(226.1991272,54.07939941)
\curveto(226.24912452,54.01938953)(226.2741245,53.93438961)(226.2741272,53.82439941)
\curveto(226.28412449,53.72438982)(226.28912448,53.61938993)(226.2891272,53.50939941)
\lineto(226.2891272,52.63939941)
\curveto(226.28912448,52.55939099)(226.28412449,52.47439107)(226.2741272,52.38439941)
\curveto(226.2741245,52.30439124)(226.28412449,52.23439131)(226.3041272,52.17439941)
\curveto(226.34412443,52.03439151)(226.43412434,51.9443916)(226.5741272,51.90439941)
\curveto(226.62412415,51.89439165)(226.6691241,51.88939166)(226.7091272,51.88939941)
\lineto(226.8591272,51.88939941)
\lineto(227.2641272,51.88939941)
\curveto(227.42412335,51.89939165)(227.53912323,51.88939166)(227.6091272,51.85939941)
\curveto(227.69912307,51.79939175)(227.75912301,51.73939181)(227.7891272,51.67939941)
\curveto(227.80912296,51.63939191)(227.81912295,51.59439195)(227.8191272,51.54439941)
\lineto(227.8191272,51.39439941)
\curveto(227.81912295,51.28439226)(227.81412296,51.17939237)(227.8041272,51.07939941)
\curveto(227.79412298,50.98939256)(227.75912301,50.91939263)(227.6991272,50.86939941)
\curveto(227.63912313,50.81939273)(227.55412322,50.78939276)(227.4441272,50.77939941)
\lineto(227.1141272,50.77939941)
\curveto(227.00412377,50.78939276)(226.89412388,50.79439275)(226.7841272,50.79439941)
\curveto(226.6741241,50.79439275)(226.57912419,50.77939277)(226.4991272,50.74939941)
\curveto(226.42912434,50.71939283)(226.37912439,50.66939288)(226.3491272,50.59939941)
\curveto(226.31912445,50.52939302)(226.29912447,50.4443931)(226.2891272,50.34439941)
\curveto(226.27912449,50.25439329)(226.2741245,50.15439339)(226.2741272,50.04439941)
\curveto(226.28412449,49.9443936)(226.28912448,49.8443937)(226.2891272,49.74439941)
\lineto(226.2891272,46.77439941)
\curveto(226.28912448,46.55439699)(226.28412449,46.31939723)(226.2741272,46.06939941)
\curveto(226.2741245,45.82939772)(226.31912445,45.6443979)(226.4091272,45.51439941)
\curveto(226.45912431,45.43439811)(226.52412425,45.37939817)(226.6041272,45.34939941)
\curveto(226.68412409,45.31939823)(226.77912399,45.29439825)(226.8891272,45.27439941)
\curveto(226.91912385,45.26439828)(226.94912382,45.25939829)(226.9791272,45.25939941)
\curveto(227.01912375,45.26939828)(227.05412372,45.26939828)(227.0841272,45.25939941)
\lineto(227.2791272,45.25939941)
\curveto(227.37912339,45.25939829)(227.4691233,45.2493983)(227.5491272,45.22939941)
\curveto(227.63912313,45.21939833)(227.70412307,45.18439836)(227.7441272,45.12439941)
\curveto(227.76412301,45.09439845)(227.77912299,45.03939851)(227.7891272,44.95939941)
\curveto(227.80912296,44.88939866)(227.81912295,44.81439873)(227.8191272,44.73439941)
\curveto(227.82912294,44.65439889)(227.82912294,44.57439897)(227.8191272,44.49439941)
\curveto(227.80912296,44.42439912)(227.78912298,44.36939918)(227.7591272,44.32939941)
\curveto(227.71912305,44.25939929)(227.64412313,44.20939934)(227.5341272,44.17939941)
\curveto(227.45412332,44.15939939)(227.36412341,44.1493994)(227.2641272,44.14939941)
\curveto(227.16412361,44.15939939)(227.0741237,44.16439938)(226.9941272,44.16439941)
\curveto(226.93412384,44.16439938)(226.8741239,44.15939939)(226.8141272,44.14939941)
\curveto(226.75412402,44.1493994)(226.69912407,44.15439939)(226.6491272,44.16439941)
\lineto(226.4691272,44.16439941)
\curveto(226.41912435,44.17439937)(226.3691244,44.17939937)(226.3191272,44.17939941)
\curveto(226.27912449,44.18939936)(226.23412454,44.19439935)(226.1841272,44.19439941)
\curveto(225.98412479,44.2443993)(225.80912496,44.29939925)(225.6591272,44.35939941)
\curveto(225.51912525,44.41939913)(225.39912537,44.52439902)(225.2991272,44.67439941)
\curveto(225.15912561,44.87439867)(225.07912569,45.12439842)(225.0591272,45.42439941)
\curveto(225.03912573,45.73439781)(225.02912574,46.06439748)(225.0291272,46.41439941)
\lineto(225.0291272,50.34439941)
\curveto(224.99912577,50.47439307)(224.9691258,50.56939298)(224.9391272,50.62939941)
\curveto(224.91912585,50.68939286)(224.84912592,50.73939281)(224.7291272,50.77939941)
\curveto(224.68912608,50.78939276)(224.64912612,50.78939276)(224.6091272,50.77939941)
\curveto(224.5691262,50.76939278)(224.52912624,50.77439277)(224.4891272,50.79439941)
\lineto(224.2491272,50.79439941)
\curveto(224.11912665,50.79439275)(224.00912676,50.80439274)(223.9191272,50.82439941)
\curveto(223.83912693,50.85439269)(223.78412699,50.91439263)(223.7541272,51.00439941)
\curveto(223.73412704,51.0443925)(223.71912705,51.08939246)(223.7091272,51.13939941)
\lineto(223.7091272,51.28939941)
\curveto(223.70912706,51.42939212)(223.71912705,51.544392)(223.7391272,51.63439941)
\curveto(223.75912701,51.73439181)(223.81912695,51.80939174)(223.9191272,51.85939941)
\curveto(224.02912674,51.89939165)(224.1691266,51.90939164)(224.3391272,51.88939941)
\curveto(224.51912625,51.86939168)(224.6691261,51.87939167)(224.7891272,51.91939941)
\curveto(224.87912589,51.96939158)(224.94912582,52.03939151)(224.9991272,52.12939941)
\curveto(225.01912575,52.18939136)(225.02912574,52.26439128)(225.0291272,52.35439941)
\lineto(225.0291272,52.60939941)
\lineto(225.0291272,53.53939941)
\lineto(225.0291272,53.77939941)
\curveto(225.02912574,53.86938968)(225.03912573,53.9443896)(225.0591272,54.00439941)
\curveto(225.09912567,54.08438946)(225.1741256,54.1493894)(225.2841272,54.19939941)
\curveto(225.31412546,54.19938935)(225.33912543,54.19938935)(225.3591272,54.19939941)
\curveto(225.38912538,54.20938934)(225.41412536,54.21438933)(225.4341272,54.21439941)
}
}
{
\newrgbcolor{curcolor}{0 0 0}
\pscustom[linestyle=none,fillstyle=solid,fillcolor=curcolor]
{
\newpath
\moveto(229.67092407,51.87439941)
\lineto(230.10592407,51.87439941)
\curveto(230.25592211,51.87439167)(230.360922,51.83439171)(230.42092407,51.75439941)
\curveto(230.47092189,51.67439187)(230.49592187,51.57439197)(230.49592407,51.45439941)
\curveto(230.50592186,51.33439221)(230.51092185,51.21439233)(230.51092407,51.09439941)
\lineto(230.51092407,49.66939941)
\lineto(230.51092407,47.40439941)
\lineto(230.51092407,46.71439941)
\curveto(230.51092185,46.48439706)(230.53592183,46.28439726)(230.58592407,46.11439941)
\curveto(230.74592162,45.66439788)(231.04592132,45.3493982)(231.48592407,45.16939941)
\curveto(231.70592066,45.07939847)(231.97092039,45.0443985)(232.28092407,45.06439941)
\curveto(232.59091977,45.09439845)(232.84091952,45.1493984)(233.03092407,45.22939941)
\curveto(233.360919,45.36939818)(233.62091874,45.544398)(233.81092407,45.75439941)
\curveto(234.01091835,45.97439757)(234.1659182,46.25939729)(234.27592407,46.60939941)
\curveto(234.30591806,46.68939686)(234.32591804,46.76939678)(234.33592407,46.84939941)
\curveto(234.34591802,46.92939662)(234.360918,47.01439653)(234.38092407,47.10439941)
\curveto(234.39091797,47.15439639)(234.39091797,47.19939635)(234.38092407,47.23939941)
\curveto(234.38091798,47.27939627)(234.39091797,47.32439622)(234.41092407,47.37439941)
\lineto(234.41092407,47.68939941)
\curveto(234.43091793,47.76939578)(234.43591793,47.85939569)(234.42592407,47.95939941)
\curveto(234.41591795,48.06939548)(234.41091795,48.16939538)(234.41092407,48.25939941)
\lineto(234.41092407,49.42939941)
\lineto(234.41092407,51.01939941)
\curveto(234.41091795,51.13939241)(234.40591796,51.26439228)(234.39592407,51.39439941)
\curveto(234.39591797,51.53439201)(234.42091794,51.6443919)(234.47092407,51.72439941)
\curveto(234.51091785,51.77439177)(234.55591781,51.80439174)(234.60592407,51.81439941)
\curveto(234.6659177,51.83439171)(234.73591763,51.85439169)(234.81592407,51.87439941)
\lineto(235.04092407,51.87439941)
\curveto(235.1609172,51.87439167)(235.2659171,51.86939168)(235.35592407,51.85939941)
\curveto(235.45591691,51.8493917)(235.53091683,51.80439174)(235.58092407,51.72439941)
\curveto(235.63091673,51.67439187)(235.65591671,51.59939195)(235.65592407,51.49939941)
\lineto(235.65592407,51.21439941)
\lineto(235.65592407,50.19439941)
\lineto(235.65592407,46.15939941)
\lineto(235.65592407,44.80939941)
\curveto(235.65591671,44.68939886)(235.65091671,44.57439897)(235.64092407,44.46439941)
\curveto(235.64091672,44.36439918)(235.60591676,44.28939926)(235.53592407,44.23939941)
\curveto(235.49591687,44.20939934)(235.43591693,44.18439936)(235.35592407,44.16439941)
\curveto(235.27591709,44.15439939)(235.18591718,44.1443994)(235.08592407,44.13439941)
\curveto(234.99591737,44.13439941)(234.90591746,44.13939941)(234.81592407,44.14939941)
\curveto(234.73591763,44.15939939)(234.67591769,44.17939937)(234.63592407,44.20939941)
\curveto(234.58591778,44.2493993)(234.54091782,44.31439923)(234.50092407,44.40439941)
\curveto(234.49091787,44.4443991)(234.48091788,44.49939905)(234.47092407,44.56939941)
\curveto(234.47091789,44.63939891)(234.4659179,44.70439884)(234.45592407,44.76439941)
\curveto(234.44591792,44.83439871)(234.42591794,44.88939866)(234.39592407,44.92939941)
\curveto(234.365918,44.96939858)(234.32091804,44.98439856)(234.26092407,44.97439941)
\curveto(234.18091818,44.95439859)(234.10091826,44.89439865)(234.02092407,44.79439941)
\curveto(233.94091842,44.70439884)(233.8659185,44.63439891)(233.79592407,44.58439941)
\curveto(233.57591879,44.42439912)(233.32591904,44.28439926)(233.04592407,44.16439941)
\curveto(232.93591943,44.11439943)(232.82091954,44.08439946)(232.70092407,44.07439941)
\curveto(232.59091977,44.05439949)(232.47591989,44.02939952)(232.35592407,43.99939941)
\curveto(232.30592006,43.98939956)(232.25092011,43.98939956)(232.19092407,43.99939941)
\curveto(232.14092022,44.00939954)(232.09092027,44.00439954)(232.04092407,43.98439941)
\curveto(231.94092042,43.96439958)(231.85092051,43.96439958)(231.77092407,43.98439941)
\lineto(231.62092407,43.98439941)
\curveto(231.57092079,44.00439954)(231.51092085,44.01439953)(231.44092407,44.01439941)
\curveto(231.38092098,44.01439953)(231.32592104,44.01939953)(231.27592407,44.02939941)
\curveto(231.23592113,44.0493995)(231.19592117,44.05939949)(231.15592407,44.05939941)
\curveto(231.12592124,44.0493995)(231.08592128,44.05439949)(231.03592407,44.07439941)
\lineto(230.79592407,44.13439941)
\curveto(230.72592164,44.15439939)(230.65092171,44.18439936)(230.57092407,44.22439941)
\curveto(230.31092205,44.33439921)(230.09092227,44.47939907)(229.91092407,44.65939941)
\curveto(229.74092262,44.8493987)(229.60092276,45.07439847)(229.49092407,45.33439941)
\curveto(229.45092291,45.42439812)(229.42092294,45.51439803)(229.40092407,45.60439941)
\lineto(229.34092407,45.90439941)
\curveto(229.32092304,45.96439758)(229.31092305,46.01939753)(229.31092407,46.06939941)
\curveto(229.32092304,46.12939742)(229.31592305,46.19439735)(229.29592407,46.26439941)
\curveto(229.28592308,46.28439726)(229.28092308,46.30939724)(229.28092407,46.33939941)
\curveto(229.28092308,46.37939717)(229.27592309,46.41439713)(229.26592407,46.44439941)
\lineto(229.26592407,46.59439941)
\curveto(229.25592311,46.63439691)(229.25092311,46.67939687)(229.25092407,46.72939941)
\curveto(229.2609231,46.78939676)(229.2659231,46.8443967)(229.26592407,46.89439941)
\lineto(229.26592407,47.49439941)
\lineto(229.26592407,50.25439941)
\lineto(229.26592407,51.21439941)
\lineto(229.26592407,51.48439941)
\curveto(229.2659231,51.57439197)(229.28592308,51.6493919)(229.32592407,51.70939941)
\curveto(229.365923,51.77939177)(229.44092292,51.82939172)(229.55092407,51.85939941)
\curveto(229.57092279,51.86939168)(229.59092277,51.86939168)(229.61092407,51.85939941)
\curveto(229.63092273,51.85939169)(229.65092271,51.86439168)(229.67092407,51.87439941)
}
}
{
\newrgbcolor{curcolor}{0 0 0}
\pscustom[linestyle=none,fillstyle=solid,fillcolor=curcolor]
{
\newpath
\moveto(244.51553345,44.95939941)
\lineto(244.51553345,44.56939941)
\curveto(244.51552557,44.4493991)(244.4905256,44.3493992)(244.44053345,44.26939941)
\curveto(244.3905257,44.19939935)(244.30552578,44.15939939)(244.18553345,44.14939941)
\lineto(243.84053345,44.14939941)
\curveto(243.78052631,44.1493994)(243.72052637,44.1443994)(243.66053345,44.13439941)
\curveto(243.61052648,44.13439941)(243.56552652,44.1443994)(243.52553345,44.16439941)
\curveto(243.43552665,44.18439936)(243.37552671,44.22439932)(243.34553345,44.28439941)
\curveto(243.30552678,44.33439921)(243.28052681,44.39439915)(243.27053345,44.46439941)
\curveto(243.27052682,44.53439901)(243.25552683,44.60439894)(243.22553345,44.67439941)
\curveto(243.21552687,44.69439885)(243.20052689,44.70939884)(243.18053345,44.71939941)
\curveto(243.17052692,44.73939881)(243.15552693,44.75939879)(243.13553345,44.77939941)
\curveto(243.03552705,44.78939876)(242.95552713,44.76939878)(242.89553345,44.71939941)
\curveto(242.84552724,44.66939888)(242.7905273,44.61939893)(242.73053345,44.56939941)
\curveto(242.53052756,44.41939913)(242.33052776,44.30439924)(242.13053345,44.22439941)
\curveto(241.95052814,44.1443994)(241.74052835,44.08439946)(241.50053345,44.04439941)
\curveto(241.27052882,44.00439954)(241.03052906,43.98439956)(240.78053345,43.98439941)
\curveto(240.54052955,43.97439957)(240.30052979,43.98939956)(240.06053345,44.02939941)
\curveto(239.82053027,44.05939949)(239.61053048,44.11439943)(239.43053345,44.19439941)
\curveto(238.91053118,44.41439913)(238.4905316,44.70939884)(238.17053345,45.07939941)
\curveto(237.85053224,45.45939809)(237.60053249,45.92939762)(237.42053345,46.48939941)
\curveto(237.38053271,46.57939697)(237.35053274,46.66939688)(237.33053345,46.75939941)
\curveto(237.32053277,46.85939669)(237.30053279,46.95939659)(237.27053345,47.05939941)
\curveto(237.26053283,47.10939644)(237.25553283,47.15939639)(237.25553345,47.20939941)
\curveto(237.25553283,47.25939629)(237.25053284,47.30939624)(237.24053345,47.35939941)
\curveto(237.22053287,47.40939614)(237.21053288,47.45939609)(237.21053345,47.50939941)
\curveto(237.22053287,47.56939598)(237.22053287,47.62439592)(237.21053345,47.67439941)
\lineto(237.21053345,47.82439941)
\curveto(237.1905329,47.87439567)(237.18053291,47.93939561)(237.18053345,48.01939941)
\curveto(237.18053291,48.09939545)(237.1905329,48.16439538)(237.21053345,48.21439941)
\lineto(237.21053345,48.37939941)
\curveto(237.23053286,48.4493951)(237.23553285,48.51939503)(237.22553345,48.58939941)
\curveto(237.22553286,48.66939488)(237.23553285,48.7443948)(237.25553345,48.81439941)
\curveto(237.26553282,48.86439468)(237.27053282,48.90939464)(237.27053345,48.94939941)
\curveto(237.27053282,48.98939456)(237.27553281,49.03439451)(237.28553345,49.08439941)
\curveto(237.31553277,49.18439436)(237.34053275,49.27939427)(237.36053345,49.36939941)
\curveto(237.38053271,49.46939408)(237.40553268,49.56439398)(237.43553345,49.65439941)
\curveto(237.56553252,50.03439351)(237.73053236,50.37439317)(237.93053345,50.67439941)
\curveto(238.14053195,50.98439256)(238.3905317,51.23939231)(238.68053345,51.43939941)
\curveto(238.85053124,51.55939199)(239.02553106,51.65939189)(239.20553345,51.73939941)
\curveto(239.39553069,51.81939173)(239.60053049,51.88939166)(239.82053345,51.94939941)
\curveto(239.8905302,51.95939159)(239.95553013,51.96939158)(240.01553345,51.97939941)
\curveto(240.08553,51.98939156)(240.15552993,52.00439154)(240.22553345,52.02439941)
\lineto(240.37553345,52.02439941)
\curveto(240.45552963,52.0443915)(240.57052952,52.05439149)(240.72053345,52.05439941)
\curveto(240.88052921,52.05439149)(241.00052909,52.0443915)(241.08053345,52.02439941)
\curveto(241.12052897,52.01439153)(241.17552891,52.00939154)(241.24553345,52.00939941)
\curveto(241.35552873,51.97939157)(241.46552862,51.95439159)(241.57553345,51.93439941)
\curveto(241.6855284,51.92439162)(241.7905283,51.89439165)(241.89053345,51.84439941)
\curveto(242.04052805,51.78439176)(242.18052791,51.71939183)(242.31053345,51.64939941)
\curveto(242.45052764,51.57939197)(242.58052751,51.49939205)(242.70053345,51.40939941)
\curveto(242.76052733,51.35939219)(242.82052727,51.30439224)(242.88053345,51.24439941)
\curveto(242.95052714,51.19439235)(243.04052705,51.17939237)(243.15053345,51.19939941)
\curveto(243.17052692,51.22939232)(243.1855269,51.25439229)(243.19553345,51.27439941)
\curveto(243.21552687,51.29439225)(243.23052686,51.32439222)(243.24053345,51.36439941)
\curveto(243.27052682,51.45439209)(243.28052681,51.56939198)(243.27053345,51.70939941)
\lineto(243.27053345,52.08439941)
\lineto(243.27053345,53.80939941)
\lineto(243.27053345,54.27439941)
\curveto(243.27052682,54.45438909)(243.29552679,54.58438896)(243.34553345,54.66439941)
\curveto(243.3855267,54.73438881)(243.44552664,54.77938877)(243.52553345,54.79939941)
\curveto(243.54552654,54.79938875)(243.57052652,54.79938875)(243.60053345,54.79939941)
\curveto(243.63052646,54.80938874)(243.65552643,54.81438873)(243.67553345,54.81439941)
\curveto(243.81552627,54.82438872)(243.96052613,54.82438872)(244.11053345,54.81439941)
\curveto(244.27052582,54.81438873)(244.38052571,54.77438877)(244.44053345,54.69439941)
\curveto(244.4905256,54.61438893)(244.51552557,54.51438903)(244.51553345,54.39439941)
\lineto(244.51553345,54.01939941)
\lineto(244.51553345,44.95939941)
\moveto(243.30053345,47.79439941)
\curveto(243.32052677,47.8443957)(243.33052676,47.90939564)(243.33053345,47.98939941)
\curveto(243.33052676,48.07939547)(243.32052677,48.1493954)(243.30053345,48.19939941)
\lineto(243.30053345,48.42439941)
\curveto(243.28052681,48.51439503)(243.26552682,48.60439494)(243.25553345,48.69439941)
\curveto(243.24552684,48.79439475)(243.22552686,48.88439466)(243.19553345,48.96439941)
\curveto(243.17552691,49.0443945)(243.15552693,49.11939443)(243.13553345,49.18939941)
\curveto(243.12552696,49.25939429)(243.10552698,49.32939422)(243.07553345,49.39939941)
\curveto(242.95552713,49.69939385)(242.80052729,49.96439358)(242.61053345,50.19439941)
\curveto(242.42052767,50.42439312)(242.18052791,50.60439294)(241.89053345,50.73439941)
\curveto(241.7905283,50.78439276)(241.6855284,50.81939273)(241.57553345,50.83939941)
\curveto(241.47552861,50.86939268)(241.36552872,50.89439265)(241.24553345,50.91439941)
\curveto(241.16552892,50.93439261)(241.07552901,50.9443926)(240.97553345,50.94439941)
\lineto(240.70553345,50.94439941)
\curveto(240.65552943,50.93439261)(240.61052948,50.92439262)(240.57053345,50.91439941)
\lineto(240.43553345,50.91439941)
\curveto(240.35552973,50.89439265)(240.27052982,50.87439267)(240.18053345,50.85439941)
\curveto(240.10052999,50.83439271)(240.02053007,50.80939274)(239.94053345,50.77939941)
\curveto(239.62053047,50.63939291)(239.36053073,50.43439311)(239.16053345,50.16439941)
\curveto(238.97053112,49.90439364)(238.81553127,49.59939395)(238.69553345,49.24939941)
\curveto(238.65553143,49.13939441)(238.62553146,49.02439452)(238.60553345,48.90439941)
\curveto(238.59553149,48.79439475)(238.58053151,48.68439486)(238.56053345,48.57439941)
\curveto(238.56053153,48.53439501)(238.55553153,48.49439505)(238.54553345,48.45439941)
\lineto(238.54553345,48.34939941)
\curveto(238.52553156,48.29939525)(238.51553157,48.2443953)(238.51553345,48.18439941)
\curveto(238.52553156,48.12439542)(238.53053156,48.06939548)(238.53053345,48.01939941)
\lineto(238.53053345,47.68939941)
\curveto(238.53053156,47.58939596)(238.54053155,47.49439605)(238.56053345,47.40439941)
\curveto(238.57053152,47.37439617)(238.57553151,47.32439622)(238.57553345,47.25439941)
\curveto(238.59553149,47.18439636)(238.61053148,47.11439643)(238.62053345,47.04439941)
\lineto(238.68053345,46.83439941)
\curveto(238.7905313,46.48439706)(238.94053115,46.18439736)(239.13053345,45.93439941)
\curveto(239.32053077,45.68439786)(239.56053053,45.47939807)(239.85053345,45.31939941)
\curveto(239.94053015,45.26939828)(240.03053006,45.22939832)(240.12053345,45.19939941)
\curveto(240.21052988,45.16939838)(240.31052978,45.13939841)(240.42053345,45.10939941)
\curveto(240.47052962,45.08939846)(240.52052957,45.08439846)(240.57053345,45.09439941)
\curveto(240.63052946,45.10439844)(240.6855294,45.09939845)(240.73553345,45.07939941)
\curveto(240.77552931,45.06939848)(240.81552927,45.06439848)(240.85553345,45.06439941)
\lineto(240.99053345,45.06439941)
\lineto(241.12553345,45.06439941)
\curveto(241.15552893,45.07439847)(241.20552888,45.07939847)(241.27553345,45.07939941)
\curveto(241.35552873,45.09939845)(241.43552865,45.11439843)(241.51553345,45.12439941)
\curveto(241.59552849,45.1443984)(241.67052842,45.16939838)(241.74053345,45.19939941)
\curveto(242.07052802,45.33939821)(242.33552775,45.51439803)(242.53553345,45.72439941)
\curveto(242.74552734,45.9443976)(242.92052717,46.21939733)(243.06053345,46.54939941)
\curveto(243.11052698,46.65939689)(243.14552694,46.76939678)(243.16553345,46.87939941)
\curveto(243.1855269,46.98939656)(243.21052688,47.09939645)(243.24053345,47.20939941)
\curveto(243.26052683,47.2493963)(243.27052682,47.28439626)(243.27053345,47.31439941)
\curveto(243.27052682,47.35439619)(243.27552681,47.39439615)(243.28553345,47.43439941)
\curveto(243.29552679,47.49439605)(243.29552679,47.55439599)(243.28553345,47.61439941)
\curveto(243.2855268,47.67439587)(243.2905268,47.73439581)(243.30053345,47.79439941)
}
}
{
\newrgbcolor{curcolor}{0 0 0}
\pscustom[linestyle=none,fillstyle=solid,fillcolor=curcolor]
{
\newpath
\moveto(246.74678345,53.37439941)
\curveto(246.66678233,53.43439011)(246.62178237,53.53939001)(246.61178345,53.68939941)
\lineto(246.61178345,54.15439941)
\lineto(246.61178345,54.40939941)
\curveto(246.61178238,54.49938905)(246.62678237,54.57438897)(246.65678345,54.63439941)
\curveto(246.6967823,54.71438883)(246.77678222,54.77438877)(246.89678345,54.81439941)
\curveto(246.91678208,54.82438872)(246.93678206,54.82438872)(246.95678345,54.81439941)
\curveto(246.98678201,54.81438873)(247.01178198,54.81938873)(247.03178345,54.82939941)
\curveto(247.20178179,54.82938872)(247.36178163,54.82438872)(247.51178345,54.81439941)
\curveto(247.66178133,54.80438874)(247.76178123,54.7443888)(247.81178345,54.63439941)
\curveto(247.84178115,54.57438897)(247.85678114,54.49938905)(247.85678345,54.40939941)
\lineto(247.85678345,54.15439941)
\curveto(247.85678114,53.97438957)(247.85178114,53.80438974)(247.84178345,53.64439941)
\curveto(247.84178115,53.48439006)(247.77678122,53.37939017)(247.64678345,53.32939941)
\curveto(247.5967814,53.30939024)(247.54178145,53.29939025)(247.48178345,53.29939941)
\lineto(247.31678345,53.29939941)
\lineto(247.00178345,53.29939941)
\curveto(246.90178209,53.29939025)(246.81678218,53.32439022)(246.74678345,53.37439941)
\moveto(247.85678345,44.86939941)
\lineto(247.85678345,44.55439941)
\curveto(247.86678113,44.45439909)(247.84678115,44.37439917)(247.79678345,44.31439941)
\curveto(247.76678123,44.25439929)(247.72178127,44.21439933)(247.66178345,44.19439941)
\curveto(247.60178139,44.18439936)(247.53178146,44.16939938)(247.45178345,44.14939941)
\lineto(247.22678345,44.14939941)
\curveto(247.0967819,44.1493994)(246.98178201,44.15439939)(246.88178345,44.16439941)
\curveto(246.7917822,44.18439936)(246.72178227,44.23439931)(246.67178345,44.31439941)
\curveto(246.63178236,44.37439917)(246.61178238,44.4493991)(246.61178345,44.53939941)
\lineto(246.61178345,44.82439941)
\lineto(246.61178345,51.16939941)
\lineto(246.61178345,51.48439941)
\curveto(246.61178238,51.59439195)(246.63678236,51.67939187)(246.68678345,51.73939941)
\curveto(246.71678228,51.78939176)(246.75678224,51.81939173)(246.80678345,51.82939941)
\curveto(246.85678214,51.83939171)(246.91178208,51.85439169)(246.97178345,51.87439941)
\curveto(246.991782,51.87439167)(247.01178198,51.86939168)(247.03178345,51.85939941)
\curveto(247.06178193,51.85939169)(247.08678191,51.86439168)(247.10678345,51.87439941)
\curveto(247.23678176,51.87439167)(247.36678163,51.86939168)(247.49678345,51.85939941)
\curveto(247.63678136,51.85939169)(247.73178126,51.81939173)(247.78178345,51.73939941)
\curveto(247.83178116,51.67939187)(247.85678114,51.59939195)(247.85678345,51.49939941)
\lineto(247.85678345,51.21439941)
\lineto(247.85678345,44.86939941)
}
}
{
\newrgbcolor{curcolor}{0 0 0}
\pscustom[linestyle=none,fillstyle=solid,fillcolor=curcolor]
{
\newpath
\moveto(256.6866272,44.70439941)
\curveto(256.71661937,44.544399)(256.70161938,44.40939914)(256.6416272,44.29939941)
\curveto(256.5816195,44.19939935)(256.50161958,44.12439942)(256.4016272,44.07439941)
\curveto(256.35161973,44.05439949)(256.29661979,44.0443995)(256.2366272,44.04439941)
\curveto(256.1866199,44.0443995)(256.13161995,44.03439951)(256.0716272,44.01439941)
\curveto(255.85162023,43.96439958)(255.63162045,43.97939957)(255.4116272,44.05939941)
\curveto(255.20162088,44.12939942)(255.05662103,44.21939933)(254.9766272,44.32939941)
\curveto(254.92662116,44.39939915)(254.8816212,44.47939907)(254.8416272,44.56939941)
\curveto(254.80162128,44.66939888)(254.75162133,44.7493988)(254.6916272,44.80939941)
\curveto(254.67162141,44.82939872)(254.64662144,44.8493987)(254.6166272,44.86939941)
\curveto(254.59662149,44.88939866)(254.56662152,44.89439865)(254.5266272,44.88439941)
\curveto(254.41662167,44.85439869)(254.31162177,44.79939875)(254.2116272,44.71939941)
\curveto(254.12162196,44.63939891)(254.03162205,44.56939898)(253.9416272,44.50939941)
\curveto(253.81162227,44.42939912)(253.67162241,44.35439919)(253.5216272,44.28439941)
\curveto(253.37162271,44.22439932)(253.21162287,44.16939938)(253.0416272,44.11939941)
\curveto(252.94162314,44.08939946)(252.83162325,44.06939948)(252.7116272,44.05939941)
\curveto(252.60162348,44.0493995)(252.49162359,44.03439951)(252.3816272,44.01439941)
\curveto(252.33162375,44.00439954)(252.2866238,43.99939955)(252.2466272,43.99939941)
\lineto(252.1416272,43.99939941)
\curveto(252.03162405,43.97939957)(251.92662416,43.97939957)(251.8266272,43.99939941)
\lineto(251.6916272,43.99939941)
\curveto(251.64162444,44.00939954)(251.59162449,44.01439953)(251.5416272,44.01439941)
\curveto(251.49162459,44.01439953)(251.44662464,44.02439952)(251.4066272,44.04439941)
\curveto(251.36662472,44.05439949)(251.33162475,44.05939949)(251.3016272,44.05939941)
\curveto(251.2816248,44.0493995)(251.25662483,44.0493995)(251.2266272,44.05939941)
\lineto(250.9866272,44.11939941)
\curveto(250.90662518,44.12939942)(250.83162525,44.1493994)(250.7616272,44.17939941)
\curveto(250.46162562,44.30939924)(250.21662587,44.45439909)(250.0266272,44.61439941)
\curveto(249.84662624,44.78439876)(249.69662639,45.01939853)(249.5766272,45.31939941)
\curveto(249.4866266,45.53939801)(249.44162664,45.80439774)(249.4416272,46.11439941)
\lineto(249.4416272,46.42939941)
\curveto(249.45162663,46.47939707)(249.45662663,46.52939702)(249.4566272,46.57939941)
\lineto(249.4866272,46.75939941)
\lineto(249.6066272,47.08939941)
\curveto(249.64662644,47.19939635)(249.69662639,47.29939625)(249.7566272,47.38939941)
\curveto(249.93662615,47.67939587)(250.1816259,47.89439565)(250.4916272,48.03439941)
\curveto(250.80162528,48.17439537)(251.14162494,48.29939525)(251.5116272,48.40939941)
\curveto(251.65162443,48.4493951)(251.79662429,48.47939507)(251.9466272,48.49939941)
\curveto(252.09662399,48.51939503)(252.24662384,48.544395)(252.3966272,48.57439941)
\curveto(252.46662362,48.59439495)(252.53162355,48.60439494)(252.5916272,48.60439941)
\curveto(252.66162342,48.60439494)(252.73662335,48.61439493)(252.8166272,48.63439941)
\curveto(252.8866232,48.65439489)(252.95662313,48.66439488)(253.0266272,48.66439941)
\curveto(253.09662299,48.67439487)(253.17162291,48.68939486)(253.2516272,48.70939941)
\curveto(253.50162258,48.76939478)(253.73662235,48.81939473)(253.9566272,48.85939941)
\curveto(254.17662191,48.90939464)(254.35162173,49.02439452)(254.4816272,49.20439941)
\curveto(254.54162154,49.28439426)(254.59162149,49.38439416)(254.6316272,49.50439941)
\curveto(254.67162141,49.63439391)(254.67162141,49.77439377)(254.6316272,49.92439941)
\curveto(254.57162151,50.16439338)(254.4816216,50.35439319)(254.3616272,50.49439941)
\curveto(254.25162183,50.63439291)(254.09162199,50.7443928)(253.8816272,50.82439941)
\curveto(253.76162232,50.87439267)(253.61662247,50.90939264)(253.4466272,50.92939941)
\curveto(253.2866228,50.9493926)(253.11662297,50.95939259)(252.9366272,50.95939941)
\curveto(252.75662333,50.95939259)(252.5816235,50.9493926)(252.4116272,50.92939941)
\curveto(252.24162384,50.90939264)(252.09662399,50.87939267)(251.9766272,50.83939941)
\curveto(251.80662428,50.77939277)(251.64162444,50.69439285)(251.4816272,50.58439941)
\curveto(251.40162468,50.52439302)(251.32662476,50.4443931)(251.2566272,50.34439941)
\curveto(251.19662489,50.25439329)(251.14162494,50.15439339)(251.0916272,50.04439941)
\curveto(251.06162502,49.96439358)(251.03162505,49.87939367)(251.0016272,49.78939941)
\curveto(250.9816251,49.69939385)(250.93662515,49.62939392)(250.8666272,49.57939941)
\curveto(250.82662526,49.549394)(250.75662533,49.52439402)(250.6566272,49.50439941)
\curveto(250.56662552,49.49439405)(250.47162561,49.48939406)(250.3716272,49.48939941)
\curveto(250.27162581,49.48939406)(250.17162591,49.49439405)(250.0716272,49.50439941)
\curveto(249.9816261,49.52439402)(249.91662617,49.549394)(249.8766272,49.57939941)
\curveto(249.83662625,49.60939394)(249.80662628,49.65939389)(249.7866272,49.72939941)
\curveto(249.76662632,49.79939375)(249.76662632,49.87439367)(249.7866272,49.95439941)
\curveto(249.81662627,50.08439346)(249.84662624,50.20439334)(249.8766272,50.31439941)
\curveto(249.91662617,50.43439311)(249.96162612,50.549393)(250.0116272,50.65939941)
\curveto(250.20162588,51.00939254)(250.44162564,51.27939227)(250.7316272,51.46939941)
\curveto(251.02162506,51.66939188)(251.3816247,51.82939172)(251.8116272,51.94939941)
\curveto(251.91162417,51.96939158)(252.01162407,51.98439156)(252.1116272,51.99439941)
\curveto(252.22162386,52.00439154)(252.33162375,52.01939153)(252.4416272,52.03939941)
\curveto(252.4816236,52.0493915)(252.54662354,52.0493915)(252.6366272,52.03939941)
\curveto(252.72662336,52.03939151)(252.7816233,52.0493915)(252.8016272,52.06939941)
\curveto(253.50162258,52.07939147)(254.11162197,51.99939155)(254.6316272,51.82939941)
\curveto(255.15162093,51.65939189)(255.51662057,51.33439221)(255.7266272,50.85439941)
\curveto(255.81662027,50.65439289)(255.86662022,50.41939313)(255.8766272,50.14939941)
\curveto(255.89662019,49.88939366)(255.90662018,49.61439393)(255.9066272,49.32439941)
\lineto(255.9066272,46.00939941)
\curveto(255.90662018,45.86939768)(255.91162017,45.73439781)(255.9216272,45.60439941)
\curveto(255.93162015,45.47439807)(255.96162012,45.36939818)(256.0116272,45.28939941)
\curveto(256.06162002,45.21939833)(256.12661996,45.16939838)(256.2066272,45.13939941)
\curveto(256.29661979,45.09939845)(256.3816197,45.06939848)(256.4616272,45.04939941)
\curveto(256.54161954,45.03939851)(256.60161948,44.99439855)(256.6416272,44.91439941)
\curveto(256.66161942,44.88439866)(256.67161941,44.85439869)(256.6716272,44.82439941)
\curveto(256.67161941,44.79439875)(256.67661941,44.75439879)(256.6866272,44.70439941)
\moveto(254.5416272,46.36939941)
\curveto(254.60162148,46.50939704)(254.63162145,46.66939688)(254.6316272,46.84939941)
\curveto(254.64162144,47.03939651)(254.64662144,47.23439631)(254.6466272,47.43439941)
\curveto(254.64662144,47.544396)(254.64162144,47.6443959)(254.6316272,47.73439941)
\curveto(254.62162146,47.82439572)(254.5816215,47.89439565)(254.5116272,47.94439941)
\curveto(254.4816216,47.96439558)(254.41162167,47.97439557)(254.3016272,47.97439941)
\curveto(254.2816218,47.95439559)(254.24662184,47.9443956)(254.1966272,47.94439941)
\curveto(254.14662194,47.9443956)(254.10162198,47.93439561)(254.0616272,47.91439941)
\curveto(253.9816221,47.89439565)(253.89162219,47.87439567)(253.7916272,47.85439941)
\lineto(253.4916272,47.79439941)
\curveto(253.46162262,47.79439575)(253.42662266,47.78939576)(253.3866272,47.77939941)
\lineto(253.2816272,47.77939941)
\curveto(253.13162295,47.73939581)(252.96662312,47.71439583)(252.7866272,47.70439941)
\curveto(252.61662347,47.70439584)(252.45662363,47.68439586)(252.3066272,47.64439941)
\curveto(252.22662386,47.62439592)(252.15162393,47.60439594)(252.0816272,47.58439941)
\curveto(252.02162406,47.57439597)(251.95162413,47.55939599)(251.8716272,47.53939941)
\curveto(251.71162437,47.48939606)(251.56162452,47.42439612)(251.4216272,47.34439941)
\curveto(251.2816248,47.27439627)(251.16162492,47.18439636)(251.0616272,47.07439941)
\curveto(250.96162512,46.96439658)(250.8866252,46.82939672)(250.8366272,46.66939941)
\curveto(250.7866253,46.51939703)(250.76662532,46.33439721)(250.7766272,46.11439941)
\curveto(250.77662531,46.01439753)(250.79162529,45.91939763)(250.8216272,45.82939941)
\curveto(250.86162522,45.7493978)(250.90662518,45.67439787)(250.9566272,45.60439941)
\curveto(251.03662505,45.49439805)(251.14162494,45.39939815)(251.2716272,45.31939941)
\curveto(251.40162468,45.2493983)(251.54162454,45.18939836)(251.6916272,45.13939941)
\curveto(251.74162434,45.12939842)(251.79162429,45.12439842)(251.8416272,45.12439941)
\curveto(251.89162419,45.12439842)(251.94162414,45.11939843)(251.9916272,45.10939941)
\curveto(252.06162402,45.08939846)(252.14662394,45.07439847)(252.2466272,45.06439941)
\curveto(252.35662373,45.06439848)(252.44662364,45.07439847)(252.5166272,45.09439941)
\curveto(252.57662351,45.11439843)(252.63662345,45.11939843)(252.6966272,45.10939941)
\curveto(252.75662333,45.10939844)(252.81662327,45.11939843)(252.8766272,45.13939941)
\curveto(252.95662313,45.15939839)(253.03162305,45.17439837)(253.1016272,45.18439941)
\curveto(253.1816229,45.19439835)(253.25662283,45.21439833)(253.3266272,45.24439941)
\curveto(253.61662247,45.36439818)(253.86162222,45.50939804)(254.0616272,45.67939941)
\curveto(254.27162181,45.8493977)(254.43162165,46.07939747)(254.5416272,46.36939941)
}
}
{
\newrgbcolor{curcolor}{0 0 0}
\pscustom[linestyle=none,fillstyle=solid,fillcolor=curcolor]
{
\newpath
\moveto(261.54826782,52.02439941)
\curveto(262.17826259,52.0443915)(262.68326208,51.95939159)(263.06326782,51.76939941)
\curveto(263.44326132,51.57939197)(263.74826102,51.29439225)(263.97826782,50.91439941)
\curveto(264.03826073,50.81439273)(264.08326068,50.70439284)(264.11326782,50.58439941)
\curveto(264.15326061,50.47439307)(264.18826058,50.35939319)(264.21826782,50.23939941)
\curveto(264.2682605,50.0493935)(264.29826047,49.8443937)(264.30826782,49.62439941)
\curveto(264.31826045,49.40439414)(264.32326044,49.17939437)(264.32326782,48.94939941)
\lineto(264.32326782,47.34439941)
\lineto(264.32326782,45.00439941)
\curveto(264.32326044,44.83439871)(264.31826045,44.66439888)(264.30826782,44.49439941)
\curveto(264.30826046,44.32439922)(264.24326052,44.21439933)(264.11326782,44.16439941)
\curveto(264.0632607,44.1443994)(264.00826076,44.13439941)(263.94826782,44.13439941)
\curveto(263.89826087,44.12439942)(263.84326092,44.11939943)(263.78326782,44.11939941)
\curveto(263.65326111,44.11939943)(263.52826124,44.12439942)(263.40826782,44.13439941)
\curveto(263.28826148,44.13439941)(263.20326156,44.17439937)(263.15326782,44.25439941)
\curveto(263.10326166,44.32439922)(263.07826169,44.41439913)(263.07826782,44.52439941)
\lineto(263.07826782,44.85439941)
\lineto(263.07826782,46.14439941)
\lineto(263.07826782,48.58939941)
\curveto(263.07826169,48.85939469)(263.07326169,49.12439442)(263.06326782,49.38439941)
\curveto(263.05326171,49.65439389)(263.00826176,49.88439366)(262.92826782,50.07439941)
\curveto(262.84826192,50.27439327)(262.72826204,50.43439311)(262.56826782,50.55439941)
\curveto(262.40826236,50.68439286)(262.22326254,50.78439276)(262.01326782,50.85439941)
\curveto(261.95326281,50.87439267)(261.88826288,50.88439266)(261.81826782,50.88439941)
\curveto(261.75826301,50.89439265)(261.69826307,50.90939264)(261.63826782,50.92939941)
\curveto(261.58826318,50.93939261)(261.50826326,50.93939261)(261.39826782,50.92939941)
\curveto(261.29826347,50.92939262)(261.22826354,50.92439262)(261.18826782,50.91439941)
\curveto(261.14826362,50.89439265)(261.11326365,50.88439266)(261.08326782,50.88439941)
\curveto(261.05326371,50.89439265)(261.01826375,50.89439265)(260.97826782,50.88439941)
\curveto(260.84826392,50.85439269)(260.72326404,50.81939273)(260.60326782,50.77939941)
\curveto(260.49326427,50.7493928)(260.38826438,50.70439284)(260.28826782,50.64439941)
\curveto(260.24826452,50.62439292)(260.21326455,50.60439294)(260.18326782,50.58439941)
\curveto(260.15326461,50.56439298)(260.11826465,50.544393)(260.07826782,50.52439941)
\curveto(259.72826504,50.27439327)(259.47326529,49.89939365)(259.31326782,49.39939941)
\curveto(259.28326548,49.31939423)(259.2632655,49.23439431)(259.25326782,49.14439941)
\curveto(259.24326552,49.06439448)(259.22826554,48.98439456)(259.20826782,48.90439941)
\curveto(259.18826558,48.85439469)(259.18326558,48.80439474)(259.19326782,48.75439941)
\curveto(259.20326556,48.71439483)(259.19826557,48.67439487)(259.17826782,48.63439941)
\lineto(259.17826782,48.31939941)
\curveto(259.1682656,48.28939526)(259.1632656,48.25439529)(259.16326782,48.21439941)
\curveto(259.17326559,48.17439537)(259.17826559,48.12939542)(259.17826782,48.07939941)
\lineto(259.17826782,47.62939941)
\lineto(259.17826782,46.18939941)
\lineto(259.17826782,44.86939941)
\lineto(259.17826782,44.52439941)
\curveto(259.17826559,44.41439913)(259.15326561,44.32439922)(259.10326782,44.25439941)
\curveto(259.05326571,44.17439937)(258.9632658,44.13439941)(258.83326782,44.13439941)
\curveto(258.71326605,44.12439942)(258.58826618,44.11939943)(258.45826782,44.11939941)
\curveto(258.37826639,44.11939943)(258.30326646,44.12439942)(258.23326782,44.13439941)
\curveto(258.1632666,44.1443994)(258.10326666,44.16939938)(258.05326782,44.20939941)
\curveto(257.97326679,44.25939929)(257.93326683,44.35439919)(257.93326782,44.49439941)
\lineto(257.93326782,44.89939941)
\lineto(257.93326782,46.66939941)
\lineto(257.93326782,50.29939941)
\lineto(257.93326782,51.21439941)
\lineto(257.93326782,51.48439941)
\curveto(257.93326683,51.57439197)(257.95326681,51.6443919)(257.99326782,51.69439941)
\curveto(258.02326674,51.75439179)(258.07326669,51.79439175)(258.14326782,51.81439941)
\curveto(258.18326658,51.82439172)(258.23826653,51.83439171)(258.30826782,51.84439941)
\curveto(258.38826638,51.85439169)(258.4682663,51.85939169)(258.54826782,51.85939941)
\curveto(258.62826614,51.85939169)(258.70326606,51.85439169)(258.77326782,51.84439941)
\curveto(258.85326591,51.83439171)(258.90826586,51.81939173)(258.93826782,51.79939941)
\curveto(259.04826572,51.72939182)(259.09826567,51.63939191)(259.08826782,51.52939941)
\curveto(259.07826569,51.42939212)(259.09326567,51.31439223)(259.13326782,51.18439941)
\curveto(259.15326561,51.12439242)(259.19326557,51.07439247)(259.25326782,51.03439941)
\curveto(259.37326539,51.02439252)(259.4682653,51.06939248)(259.53826782,51.16939941)
\curveto(259.61826515,51.26939228)(259.69826507,51.3493922)(259.77826782,51.40939941)
\curveto(259.91826485,51.50939204)(260.05826471,51.59939195)(260.19826782,51.67939941)
\curveto(260.34826442,51.76939178)(260.51826425,51.8443917)(260.70826782,51.90439941)
\curveto(260.78826398,51.93439161)(260.87326389,51.95439159)(260.96326782,51.96439941)
\curveto(261.0632637,51.97439157)(261.15826361,51.98939156)(261.24826782,52.00939941)
\curveto(261.29826347,52.01939153)(261.34826342,52.02439152)(261.39826782,52.02439941)
\lineto(261.54826782,52.02439941)
}
}
{
\newrgbcolor{curcolor}{0 0 0}
\pscustom[linestyle=none,fillstyle=solid,fillcolor=curcolor]
{
\newpath
\moveto(267.1528772,54.21439941)
\curveto(267.30287519,54.21438933)(267.45287504,54.20938934)(267.6028772,54.19939941)
\curveto(267.75287474,54.19938935)(267.85787463,54.15938939)(267.9178772,54.07939941)
\curveto(267.96787452,54.01938953)(267.9928745,53.93438961)(267.9928772,53.82439941)
\curveto(268.00287449,53.72438982)(268.00787448,53.61938993)(268.0078772,53.50939941)
\lineto(268.0078772,52.63939941)
\curveto(268.00787448,52.55939099)(268.00287449,52.47439107)(267.9928772,52.38439941)
\curveto(267.9928745,52.30439124)(268.00287449,52.23439131)(268.0228772,52.17439941)
\curveto(268.06287443,52.03439151)(268.15287434,51.9443916)(268.2928772,51.90439941)
\curveto(268.34287415,51.89439165)(268.3878741,51.88939166)(268.4278772,51.88939941)
\lineto(268.5778772,51.88939941)
\lineto(268.9828772,51.88939941)
\curveto(269.14287335,51.89939165)(269.25787323,51.88939166)(269.3278772,51.85939941)
\curveto(269.41787307,51.79939175)(269.47787301,51.73939181)(269.5078772,51.67939941)
\curveto(269.52787296,51.63939191)(269.53787295,51.59439195)(269.5378772,51.54439941)
\lineto(269.5378772,51.39439941)
\curveto(269.53787295,51.28439226)(269.53287296,51.17939237)(269.5228772,51.07939941)
\curveto(269.51287298,50.98939256)(269.47787301,50.91939263)(269.4178772,50.86939941)
\curveto(269.35787313,50.81939273)(269.27287322,50.78939276)(269.1628772,50.77939941)
\lineto(268.8328772,50.77939941)
\curveto(268.72287377,50.78939276)(268.61287388,50.79439275)(268.5028772,50.79439941)
\curveto(268.3928741,50.79439275)(268.29787419,50.77939277)(268.2178772,50.74939941)
\curveto(268.14787434,50.71939283)(268.09787439,50.66939288)(268.0678772,50.59939941)
\curveto(268.03787445,50.52939302)(268.01787447,50.4443931)(268.0078772,50.34439941)
\curveto(267.99787449,50.25439329)(267.9928745,50.15439339)(267.9928772,50.04439941)
\curveto(268.00287449,49.9443936)(268.00787448,49.8443937)(268.0078772,49.74439941)
\lineto(268.0078772,46.77439941)
\curveto(268.00787448,46.55439699)(268.00287449,46.31939723)(267.9928772,46.06939941)
\curveto(267.9928745,45.82939772)(268.03787445,45.6443979)(268.1278772,45.51439941)
\curveto(268.17787431,45.43439811)(268.24287425,45.37939817)(268.3228772,45.34939941)
\curveto(268.40287409,45.31939823)(268.49787399,45.29439825)(268.6078772,45.27439941)
\curveto(268.63787385,45.26439828)(268.66787382,45.25939829)(268.6978772,45.25939941)
\curveto(268.73787375,45.26939828)(268.77287372,45.26939828)(268.8028772,45.25939941)
\lineto(268.9978772,45.25939941)
\curveto(269.09787339,45.25939829)(269.1878733,45.2493983)(269.2678772,45.22939941)
\curveto(269.35787313,45.21939833)(269.42287307,45.18439836)(269.4628772,45.12439941)
\curveto(269.48287301,45.09439845)(269.49787299,45.03939851)(269.5078772,44.95939941)
\curveto(269.52787296,44.88939866)(269.53787295,44.81439873)(269.5378772,44.73439941)
\curveto(269.54787294,44.65439889)(269.54787294,44.57439897)(269.5378772,44.49439941)
\curveto(269.52787296,44.42439912)(269.50787298,44.36939918)(269.4778772,44.32939941)
\curveto(269.43787305,44.25939929)(269.36287313,44.20939934)(269.2528772,44.17939941)
\curveto(269.17287332,44.15939939)(269.08287341,44.1493994)(268.9828772,44.14939941)
\curveto(268.88287361,44.15939939)(268.7928737,44.16439938)(268.7128772,44.16439941)
\curveto(268.65287384,44.16439938)(268.5928739,44.15939939)(268.5328772,44.14939941)
\curveto(268.47287402,44.1493994)(268.41787407,44.15439939)(268.3678772,44.16439941)
\lineto(268.1878772,44.16439941)
\curveto(268.13787435,44.17439937)(268.0878744,44.17939937)(268.0378772,44.17939941)
\curveto(267.99787449,44.18939936)(267.95287454,44.19439935)(267.9028772,44.19439941)
\curveto(267.70287479,44.2443993)(267.52787496,44.29939925)(267.3778772,44.35939941)
\curveto(267.23787525,44.41939913)(267.11787537,44.52439902)(267.0178772,44.67439941)
\curveto(266.87787561,44.87439867)(266.79787569,45.12439842)(266.7778772,45.42439941)
\curveto(266.75787573,45.73439781)(266.74787574,46.06439748)(266.7478772,46.41439941)
\lineto(266.7478772,50.34439941)
\curveto(266.71787577,50.47439307)(266.6878758,50.56939298)(266.6578772,50.62939941)
\curveto(266.63787585,50.68939286)(266.56787592,50.73939281)(266.4478772,50.77939941)
\curveto(266.40787608,50.78939276)(266.36787612,50.78939276)(266.3278772,50.77939941)
\curveto(266.2878762,50.76939278)(266.24787624,50.77439277)(266.2078772,50.79439941)
\lineto(265.9678772,50.79439941)
\curveto(265.83787665,50.79439275)(265.72787676,50.80439274)(265.6378772,50.82439941)
\curveto(265.55787693,50.85439269)(265.50287699,50.91439263)(265.4728772,51.00439941)
\curveto(265.45287704,51.0443925)(265.43787705,51.08939246)(265.4278772,51.13939941)
\lineto(265.4278772,51.28939941)
\curveto(265.42787706,51.42939212)(265.43787705,51.544392)(265.4578772,51.63439941)
\curveto(265.47787701,51.73439181)(265.53787695,51.80939174)(265.6378772,51.85939941)
\curveto(265.74787674,51.89939165)(265.8878766,51.90939164)(266.0578772,51.88939941)
\curveto(266.23787625,51.86939168)(266.3878761,51.87939167)(266.5078772,51.91939941)
\curveto(266.59787589,51.96939158)(266.66787582,52.03939151)(266.7178772,52.12939941)
\curveto(266.73787575,52.18939136)(266.74787574,52.26439128)(266.7478772,52.35439941)
\lineto(266.7478772,52.60939941)
\lineto(266.7478772,53.53939941)
\lineto(266.7478772,53.77939941)
\curveto(266.74787574,53.86938968)(266.75787573,53.9443896)(266.7778772,54.00439941)
\curveto(266.81787567,54.08438946)(266.8928756,54.1493894)(267.0028772,54.19939941)
\curveto(267.03287546,54.19938935)(267.05787543,54.19938935)(267.0778772,54.19939941)
\curveto(267.10787538,54.20938934)(267.13287536,54.21438933)(267.1528772,54.21439941)
}
}
{
\newrgbcolor{curcolor}{0 0 0}
\pscustom[linestyle=none,fillstyle=solid,fillcolor=curcolor]
{
\newpath
\moveto(277.67467407,48.31939941)
\curveto(277.69466639,48.21939533)(277.69466639,48.10439544)(277.67467407,47.97439941)
\curveto(277.66466642,47.85439569)(277.63466645,47.76939578)(277.58467407,47.71939941)
\curveto(277.53466655,47.67939587)(277.45966662,47.6493959)(277.35967407,47.62939941)
\curveto(277.26966681,47.61939593)(277.16466692,47.61439593)(277.04467407,47.61439941)
\lineto(276.68467407,47.61439941)
\curveto(276.56466752,47.62439592)(276.45966762,47.62939592)(276.36967407,47.62939941)
\lineto(272.52967407,47.62939941)
\curveto(272.44967163,47.62939592)(272.36967171,47.62439592)(272.28967407,47.61439941)
\curveto(272.20967187,47.61439593)(272.14467194,47.59939595)(272.09467407,47.56939941)
\curveto(272.05467203,47.549396)(272.01467207,47.50939604)(271.97467407,47.44939941)
\curveto(271.95467213,47.41939613)(271.93467215,47.37439617)(271.91467407,47.31439941)
\curveto(271.89467219,47.26439628)(271.89467219,47.21439633)(271.91467407,47.16439941)
\curveto(271.92467216,47.11439643)(271.92967215,47.06939648)(271.92967407,47.02939941)
\curveto(271.92967215,46.98939656)(271.93467215,46.9493966)(271.94467407,46.90939941)
\curveto(271.96467212,46.82939672)(271.9846721,46.7443968)(272.00467407,46.65439941)
\curveto(272.02467206,46.57439697)(272.05467203,46.49439705)(272.09467407,46.41439941)
\curveto(272.32467176,45.87439767)(272.70467138,45.48939806)(273.23467407,45.25939941)
\curveto(273.29467079,45.22939832)(273.35967072,45.20439834)(273.42967407,45.18439941)
\lineto(273.63967407,45.12439941)
\curveto(273.66967041,45.11439843)(273.71967036,45.10939844)(273.78967407,45.10939941)
\curveto(273.92967015,45.06939848)(274.11466997,45.0493985)(274.34467407,45.04939941)
\curveto(274.57466951,45.0493985)(274.75966932,45.06939848)(274.89967407,45.10939941)
\curveto(275.03966904,45.1493984)(275.16466892,45.18939836)(275.27467407,45.22939941)
\curveto(275.39466869,45.27939827)(275.50466858,45.33939821)(275.60467407,45.40939941)
\curveto(275.71466837,45.47939807)(275.80966827,45.55939799)(275.88967407,45.64939941)
\curveto(275.96966811,45.7493978)(276.03966804,45.85439769)(276.09967407,45.96439941)
\curveto(276.15966792,46.06439748)(276.20966787,46.16939738)(276.24967407,46.27939941)
\curveto(276.29966778,46.38939716)(276.3796677,46.46939708)(276.48967407,46.51939941)
\curveto(276.52966755,46.53939701)(276.59466749,46.55439699)(276.68467407,46.56439941)
\curveto(276.77466731,46.57439697)(276.86466722,46.57439697)(276.95467407,46.56439941)
\curveto(277.04466704,46.56439698)(277.12966695,46.55939699)(277.20967407,46.54939941)
\curveto(277.28966679,46.53939701)(277.34466674,46.51939703)(277.37467407,46.48939941)
\curveto(277.47466661,46.41939713)(277.49966658,46.30439724)(277.44967407,46.14439941)
\curveto(277.36966671,45.87439767)(277.26466682,45.63439791)(277.13467407,45.42439941)
\curveto(276.93466715,45.10439844)(276.70466738,44.83939871)(276.44467407,44.62939941)
\curveto(276.19466789,44.42939912)(275.87466821,44.26439928)(275.48467407,44.13439941)
\curveto(275.3846687,44.09439945)(275.2846688,44.06939948)(275.18467407,44.05939941)
\curveto(275.084669,44.03939951)(274.9796691,44.01939953)(274.86967407,43.99939941)
\curveto(274.81966926,43.98939956)(274.76966931,43.98439956)(274.71967407,43.98439941)
\curveto(274.6796694,43.98439956)(274.63466945,43.97939957)(274.58467407,43.96939941)
\lineto(274.43467407,43.96939941)
\curveto(274.3846697,43.95939959)(274.32466976,43.95439959)(274.25467407,43.95439941)
\curveto(274.19466989,43.95439959)(274.14466994,43.95939959)(274.10467407,43.96939941)
\lineto(273.96967407,43.96939941)
\curveto(273.91967016,43.97939957)(273.87467021,43.98439956)(273.83467407,43.98439941)
\curveto(273.79467029,43.98439956)(273.75467033,43.98939956)(273.71467407,43.99939941)
\curveto(273.66467042,44.00939954)(273.60967047,44.01939953)(273.54967407,44.02939941)
\curveto(273.48967059,44.02939952)(273.43467065,44.03439951)(273.38467407,44.04439941)
\curveto(273.29467079,44.06439948)(273.20467088,44.08939946)(273.11467407,44.11939941)
\curveto(273.02467106,44.13939941)(272.93967114,44.16439938)(272.85967407,44.19439941)
\curveto(272.81967126,44.21439933)(272.7846713,44.22439932)(272.75467407,44.22439941)
\curveto(272.72467136,44.23439931)(272.68967139,44.2493993)(272.64967407,44.26939941)
\curveto(272.49967158,44.33939921)(272.33967174,44.42439912)(272.16967407,44.52439941)
\curveto(271.8796722,44.71439883)(271.62967245,44.9443986)(271.41967407,45.21439941)
\curveto(271.21967286,45.49439805)(271.04967303,45.80439774)(270.90967407,46.14439941)
\curveto(270.85967322,46.25439729)(270.81967326,46.36939718)(270.78967407,46.48939941)
\curveto(270.76967331,46.60939694)(270.73967334,46.72939682)(270.69967407,46.84939941)
\curveto(270.68967339,46.88939666)(270.6846734,46.92439662)(270.68467407,46.95439941)
\curveto(270.6846734,46.98439656)(270.6796734,47.02439652)(270.66967407,47.07439941)
\curveto(270.64967343,47.15439639)(270.63467345,47.23939631)(270.62467407,47.32939941)
\curveto(270.61467347,47.41939613)(270.59967348,47.50939604)(270.57967407,47.59939941)
\lineto(270.57967407,47.80939941)
\curveto(270.56967351,47.8493957)(270.55967352,47.90439564)(270.54967407,47.97439941)
\curveto(270.54967353,48.05439549)(270.55467353,48.11939543)(270.56467407,48.16939941)
\lineto(270.56467407,48.33439941)
\curveto(270.5846735,48.38439516)(270.58967349,48.43439511)(270.57967407,48.48439941)
\curveto(270.5796735,48.544395)(270.5846735,48.59939495)(270.59467407,48.64939941)
\curveto(270.63467345,48.80939474)(270.66467342,48.96939458)(270.68467407,49.12939941)
\curveto(270.71467337,49.28939426)(270.75967332,49.43939411)(270.81967407,49.57939941)
\curveto(270.86967321,49.68939386)(270.91467317,49.79939375)(270.95467407,49.90939941)
\curveto(271.00467308,50.02939352)(271.05967302,50.1443934)(271.11967407,50.25439941)
\curveto(271.33967274,50.60439294)(271.58967249,50.90439264)(271.86967407,51.15439941)
\curveto(272.14967193,51.41439213)(272.49467159,51.62939192)(272.90467407,51.79939941)
\curveto(273.02467106,51.8493917)(273.14467094,51.88439166)(273.26467407,51.90439941)
\curveto(273.39467069,51.93439161)(273.52967055,51.96439158)(273.66967407,51.99439941)
\curveto(273.71967036,52.00439154)(273.76467032,52.00939154)(273.80467407,52.00939941)
\curveto(273.84467024,52.01939153)(273.88967019,52.02439152)(273.93967407,52.02439941)
\curveto(273.95967012,52.03439151)(273.9846701,52.03439151)(274.01467407,52.02439941)
\curveto(274.04467004,52.01439153)(274.06967001,52.01939153)(274.08967407,52.03939941)
\curveto(274.50966957,52.0493915)(274.87466921,52.00439154)(275.18467407,51.90439941)
\curveto(275.49466859,51.81439173)(275.77466831,51.68939186)(276.02467407,51.52939941)
\curveto(276.07466801,51.50939204)(276.11466797,51.47939207)(276.14467407,51.43939941)
\curveto(276.17466791,51.40939214)(276.20966787,51.38439216)(276.24967407,51.36439941)
\curveto(276.32966775,51.30439224)(276.40966767,51.23439231)(276.48967407,51.15439941)
\curveto(276.5796675,51.07439247)(276.65466743,50.99439255)(276.71467407,50.91439941)
\curveto(276.87466721,50.70439284)(277.00966707,50.50439304)(277.11967407,50.31439941)
\curveto(277.18966689,50.20439334)(277.24466684,50.08439346)(277.28467407,49.95439941)
\curveto(277.32466676,49.82439372)(277.36966671,49.69439385)(277.41967407,49.56439941)
\curveto(277.46966661,49.43439411)(277.50466658,49.29939425)(277.52467407,49.15939941)
\curveto(277.55466653,49.01939453)(277.58966649,48.87939467)(277.62967407,48.73939941)
\curveto(277.63966644,48.66939488)(277.64466644,48.59939495)(277.64467407,48.52939941)
\lineto(277.67467407,48.31939941)
\moveto(276.21967407,48.82939941)
\curveto(276.24966783,48.86939468)(276.27466781,48.91939463)(276.29467407,48.97939941)
\curveto(276.31466777,49.0493945)(276.31466777,49.11939443)(276.29467407,49.18939941)
\curveto(276.23466785,49.40939414)(276.14966793,49.61439393)(276.03967407,49.80439941)
\curveto(275.89966818,50.03439351)(275.74466834,50.22939332)(275.57467407,50.38939941)
\curveto(275.40466868,50.549393)(275.1846689,50.68439286)(274.91467407,50.79439941)
\curveto(274.84466924,50.81439273)(274.77466931,50.82939272)(274.70467407,50.83939941)
\curveto(274.63466945,50.85939269)(274.55966952,50.87939267)(274.47967407,50.89939941)
\curveto(274.39966968,50.91939263)(274.31466977,50.92939262)(274.22467407,50.92939941)
\lineto(273.96967407,50.92939941)
\curveto(273.93967014,50.90939264)(273.90467018,50.89939265)(273.86467407,50.89939941)
\curveto(273.82467026,50.90939264)(273.78967029,50.90939264)(273.75967407,50.89939941)
\lineto(273.51967407,50.83939941)
\curveto(273.44967063,50.82939272)(273.3796707,50.81439273)(273.30967407,50.79439941)
\curveto(273.01967106,50.67439287)(272.7846713,50.52439302)(272.60467407,50.34439941)
\curveto(272.43467165,50.16439338)(272.2796718,49.93939361)(272.13967407,49.66939941)
\curveto(272.10967197,49.61939393)(272.079672,49.55439399)(272.04967407,49.47439941)
\curveto(272.01967206,49.40439414)(271.99467209,49.32439422)(271.97467407,49.23439941)
\curveto(271.95467213,49.1443944)(271.94967213,49.05939449)(271.95967407,48.97939941)
\curveto(271.96967211,48.89939465)(272.00467208,48.83939471)(272.06467407,48.79939941)
\curveto(272.14467194,48.73939481)(272.2796718,48.70939484)(272.46967407,48.70939941)
\curveto(272.66967141,48.71939483)(272.83967124,48.72439482)(272.97967407,48.72439941)
\lineto(275.25967407,48.72439941)
\curveto(275.40966867,48.72439482)(275.58966849,48.71939483)(275.79967407,48.70939941)
\curveto(276.00966807,48.70939484)(276.14966793,48.7493948)(276.21967407,48.82939941)
}
}
{
\newrgbcolor{curcolor}{0 0 0}
\pscustom[linestyle=none,fillstyle=solid,fillcolor=curcolor]
{
\newpath
\moveto(405.67456909,49.72938477)
\lineto(405.67456909,49.45938477)
\curveto(405.68455912,49.36937952)(405.67955913,49.2893796)(405.65956909,49.21938477)
\lineto(405.65956909,49.06938477)
\curveto(405.64955916,49.03937985)(405.64455916,49.00437988)(405.64456909,48.96438477)
\curveto(405.65455915,48.92437996)(405.65455915,48.89437999)(405.64456909,48.87438477)
\curveto(405.63455917,48.82438006)(405.62955918,48.76938012)(405.62956909,48.70938477)
\curveto(405.62955918,48.65938023)(405.62455918,48.60938028)(405.61456909,48.55938477)
\curveto(405.58455922,48.41938047)(405.56455924,48.26938062)(405.55456909,48.10938477)
\curveto(405.54455926,47.95938093)(405.51455929,47.81438107)(405.46456909,47.67438477)
\curveto(405.43455937,47.55438133)(405.39955941,47.42938146)(405.35956909,47.29938477)
\curveto(405.32955948,47.17938171)(405.28955952,47.05938183)(405.23956909,46.93938477)
\curveto(405.06955974,46.50938238)(404.85455995,46.11938277)(404.59456909,45.76938477)
\curveto(404.34456046,45.42938346)(404.02956078,45.13938375)(403.64956909,44.89938477)
\curveto(403.45956135,44.77938411)(403.25456155,44.67438421)(403.03456909,44.58438477)
\curveto(402.82456198,44.50438438)(402.59456221,44.42438446)(402.34456909,44.34438477)
\curveto(402.23456257,44.30438458)(402.11456269,44.27438461)(401.98456909,44.25438477)
\curveto(401.86456294,44.24438464)(401.74456306,44.22438466)(401.62456909,44.19438477)
\curveto(401.51456329,44.17438471)(401.4045634,44.16438472)(401.29456909,44.16438477)
\curveto(401.19456361,44.16438472)(401.09456371,44.15438473)(400.99456909,44.13438477)
\lineto(400.78456909,44.13438477)
\curveto(400.75456405,44.12438476)(400.71956409,44.11938477)(400.67956909,44.11938477)
\curveto(400.63956417,44.12938476)(400.59956421,44.13438475)(400.55956909,44.13438477)
\lineto(397.55956909,44.13438477)
\curveto(397.4095674,44.13438475)(397.27456753,44.13938475)(397.15456909,44.14938477)
\curveto(397.04456776,44.16938472)(396.96956784,44.23438465)(396.92956909,44.34438477)
\curveto(396.88956792,44.42438446)(396.86956794,44.53938435)(396.86956909,44.68938477)
\curveto(396.87956793,44.83938405)(396.88456792,44.97438391)(396.88456909,45.09438477)
\lineto(396.88456909,53.95938477)
\curveto(396.88456792,54.07937481)(396.87956793,54.20437468)(396.86956909,54.33438477)
\curveto(396.86956794,54.47437441)(396.89456791,54.5843743)(396.94456909,54.66438477)
\curveto(396.98456782,54.73437415)(397.05956775,54.77937411)(397.16956909,54.79938477)
\curveto(397.18956762,54.80937408)(397.2095676,54.80937408)(397.22956909,54.79938477)
\curveto(397.24956756,54.79937409)(397.26956754,54.80437408)(397.28956909,54.81438477)
\lineto(400.54456909,54.81438477)
\curveto(400.59456421,54.81437407)(400.63956417,54.81437407)(400.67956909,54.81438477)
\curveto(400.72956408,54.82437406)(400.77456403,54.82437406)(400.81456909,54.81438477)
\curveto(400.86456394,54.79437409)(400.91456389,54.7893741)(400.96456909,54.79938477)
\curveto(401.02456378,54.80937408)(401.07956373,54.80937408)(401.12956909,54.79938477)
\curveto(401.17956363,54.7893741)(401.23456357,54.7843741)(401.29456909,54.78438477)
\curveto(401.35456345,54.7843741)(401.4095634,54.77937411)(401.45956909,54.76938477)
\curveto(401.5095633,54.75937413)(401.55456325,54.75437413)(401.59456909,54.75438477)
\curveto(401.64456316,54.75437413)(401.69456311,54.74937414)(401.74456909,54.73938477)
\curveto(401.85456295,54.71937417)(401.95956285,54.69937419)(402.05956909,54.67938477)
\curveto(402.15956265,54.66937422)(402.25956255,54.64937424)(402.35956909,54.61938477)
\curveto(402.57956223,54.54937434)(402.78956202,54.47937441)(402.98956909,54.40938477)
\curveto(403.18956162,54.34937454)(403.37456143,54.26437462)(403.54456909,54.15438477)
\curveto(403.68456112,54.07437481)(403.809561,53.99437489)(403.91956909,53.91438477)
\curveto(403.94956086,53.89437499)(403.97956083,53.86937502)(404.00956909,53.83938477)
\curveto(404.03956077,53.81937507)(404.06956074,53.79937509)(404.09956909,53.77938477)
\curveto(404.15956065,53.72937516)(404.21456059,53.67937521)(404.26456909,53.62938477)
\curveto(404.31456049,53.57937531)(404.36456044,53.52937536)(404.41456909,53.47938477)
\curveto(404.46456034,53.42937546)(404.5045603,53.39437549)(404.53456909,53.37438477)
\curveto(404.57456023,53.31437557)(404.61456019,53.25937563)(404.65456909,53.20938477)
\curveto(404.7045601,53.15937573)(404.74956006,53.10437578)(404.78956909,53.04438477)
\curveto(404.83955997,52.9843759)(404.87955993,52.91937597)(404.90956909,52.84938477)
\curveto(404.94955986,52.7893761)(404.99455981,52.72437616)(405.04456909,52.65438477)
\curveto(405.06455974,52.61437627)(405.07955973,52.57937631)(405.08956909,52.54938477)
\curveto(405.09955971,52.51937637)(405.11455969,52.4843764)(405.13456909,52.44438477)
\curveto(405.17455963,52.36437652)(405.2095596,52.2843766)(405.23956909,52.20438477)
\curveto(405.26955954,52.13437675)(405.3045595,52.05937683)(405.34456909,51.97938477)
\curveto(405.38455942,51.86937702)(405.41455939,51.75437713)(405.43456909,51.63438477)
\curveto(405.46455934,51.52437736)(405.49455931,51.41437747)(405.52456909,51.30438477)
\curveto(405.54455926,51.24437764)(405.55455925,51.1843777)(405.55456909,51.12438477)
\curveto(405.55455925,51.07437781)(405.56455924,51.01937787)(405.58456909,50.95938477)
\curveto(405.63455917,50.77937811)(405.65955915,50.57937831)(405.65956909,50.35938477)
\curveto(405.66955914,50.14937874)(405.67455913,49.93937895)(405.67456909,49.72938477)
\moveto(404.24956909,48.94938477)
\curveto(404.26956054,49.04937984)(404.27956053,49.15437973)(404.27956909,49.26438477)
\lineto(404.27956909,49.60938477)
\lineto(404.27956909,49.83438477)
\curveto(404.28956052,49.91437897)(404.28456052,49.9893789)(404.26456909,50.05938477)
\curveto(404.26456054,50.0893788)(404.25956055,50.11937877)(404.24956909,50.14938477)
\lineto(404.24956909,50.25438477)
\curveto(404.22956058,50.36437852)(404.21456059,50.47437841)(404.20456909,50.58438477)
\curveto(404.2045606,50.69437819)(404.18956062,50.80437808)(404.15956909,50.91438477)
\curveto(404.13956067,50.99437789)(404.11956069,51.06937782)(404.09956909,51.13938477)
\curveto(404.08956072,51.21937767)(404.07456073,51.29937759)(404.05456909,51.37938477)
\curveto(403.94456086,51.73937715)(403.804561,52.05437683)(403.63456909,52.32438477)
\curveto(403.35456145,52.77437611)(402.93956187,53.11437577)(402.38956909,53.34438477)
\curveto(402.29956251,53.39437549)(402.2045626,53.42937546)(402.10456909,53.44938477)
\curveto(402.0045628,53.47937541)(401.89956291,53.50937538)(401.78956909,53.53938477)
\curveto(401.67956313,53.56937532)(401.56456324,53.5843753)(401.44456909,53.58438477)
\curveto(401.33456347,53.59437529)(401.22456358,53.60937528)(401.11456909,53.62938477)
\lineto(400.79956909,53.62938477)
\curveto(400.76956404,53.63937525)(400.73456407,53.64437524)(400.69456909,53.64438477)
\lineto(400.57456909,53.64438477)
\lineto(398.74456909,53.64438477)
\curveto(398.72456608,53.63437525)(398.69956611,53.62937526)(398.66956909,53.62938477)
\curveto(398.63956617,53.63937525)(398.61456619,53.63937525)(398.59456909,53.62938477)
\lineto(398.44456909,53.56938477)
\curveto(398.4045664,53.54937534)(398.37456643,53.51937537)(398.35456909,53.47938477)
\curveto(398.33456647,53.43937545)(398.31456649,53.36937552)(398.29456909,53.26938477)
\lineto(398.29456909,53.14938477)
\curveto(398.28456652,53.10937578)(398.27956653,53.06437582)(398.27956909,53.01438477)
\lineto(398.27956909,52.87938477)
\lineto(398.27956909,46.06938477)
\lineto(398.27956909,45.91938477)
\curveto(398.27956653,45.87938301)(398.28456652,45.83938305)(398.29456909,45.79938477)
\lineto(398.29456909,45.67938477)
\curveto(398.31456649,45.57938331)(398.33456647,45.50938338)(398.35456909,45.46938477)
\curveto(398.43456637,45.34938354)(398.58456622,45.2893836)(398.80456909,45.28938477)
\curveto(399.02456578,45.29938359)(399.23456557,45.30438358)(399.43456909,45.30438477)
\lineto(400.30456909,45.30438477)
\curveto(400.37456443,45.30438358)(400.44956436,45.29938359)(400.52956909,45.28938477)
\curveto(400.6095642,45.2893836)(400.67956413,45.29938359)(400.73956909,45.31938477)
\lineto(400.90456909,45.31938477)
\curveto(400.95456385,45.32938356)(401.0095638,45.32938356)(401.06956909,45.31938477)
\curveto(401.12956368,45.31938357)(401.18956362,45.32438356)(401.24956909,45.33438477)
\curveto(401.3095635,45.35438353)(401.36956344,45.36438352)(401.42956909,45.36438477)
\curveto(401.48956332,45.37438351)(401.55456325,45.3893835)(401.62456909,45.40938477)
\curveto(401.73456307,45.43938345)(401.83956297,45.46938342)(401.93956909,45.49938477)
\curveto(402.04956276,45.52938336)(402.15956265,45.56938332)(402.26956909,45.61938477)
\curveto(402.63956217,45.77938311)(402.95456185,45.99438289)(403.21456909,46.26438477)
\curveto(403.48456132,46.54438234)(403.7045611,46.87438201)(403.87456909,47.25438477)
\curveto(403.92456088,47.36438152)(403.96456084,47.47938141)(403.99456909,47.59938477)
\lineto(404.11456909,47.98938477)
\curveto(404.14456066,48.09938079)(404.16456064,48.21438067)(404.17456909,48.33438477)
\curveto(404.19456061,48.46438042)(404.21456059,48.5893803)(404.23456909,48.70938477)
\curveto(404.24456056,48.75938013)(404.24956056,48.79938009)(404.24956909,48.82938477)
\lineto(404.24956909,48.94938477)
}
}
{
\newrgbcolor{curcolor}{0 0 0}
\pscustom[linestyle=none,fillstyle=solid,fillcolor=curcolor]
{
\newpath
\moveto(414.30144409,48.33438477)
\curveto(414.32143603,48.27438061)(414.33143602,48.17938071)(414.33144409,48.04938477)
\curveto(414.33143602,47.92938096)(414.32643603,47.84438104)(414.31644409,47.79438477)
\lineto(414.31644409,47.64438477)
\curveto(414.30643605,47.56438132)(414.29643606,47.4893814)(414.28644409,47.41938477)
\curveto(414.28643607,47.35938153)(414.28143607,47.2893816)(414.27144409,47.20938477)
\curveto(414.2514361,47.14938174)(414.23643612,47.0893818)(414.22644409,47.02938477)
\curveto(414.22643613,46.96938192)(414.21643614,46.90938198)(414.19644409,46.84938477)
\curveto(414.1564362,46.71938217)(414.12143623,46.5893823)(414.09144409,46.45938477)
\curveto(414.06143629,46.32938256)(414.02143633,46.20938268)(413.97144409,46.09938477)
\curveto(413.76143659,45.61938327)(413.48143687,45.21438367)(413.13144409,44.88438477)
\curveto(412.78143757,44.56438432)(412.351438,44.31938457)(411.84144409,44.14938477)
\curveto(411.73143862,44.10938478)(411.61143874,44.07938481)(411.48144409,44.05938477)
\curveto(411.36143899,44.03938485)(411.23643912,44.01938487)(411.10644409,43.99938477)
\curveto(411.04643931,43.9893849)(410.98143937,43.9843849)(410.91144409,43.98438477)
\curveto(410.8514395,43.97438491)(410.79143956,43.96938492)(410.73144409,43.96938477)
\curveto(410.69143966,43.95938493)(410.63143972,43.95438493)(410.55144409,43.95438477)
\curveto(410.48143987,43.95438493)(410.43143992,43.95938493)(410.40144409,43.96938477)
\curveto(410.36143999,43.97938491)(410.32144003,43.9843849)(410.28144409,43.98438477)
\curveto(410.24144011,43.97438491)(410.20644015,43.97438491)(410.17644409,43.98438477)
\lineto(410.08644409,43.98438477)
\lineto(409.72644409,44.02938477)
\curveto(409.58644077,44.06938482)(409.4514409,44.10938478)(409.32144409,44.14938477)
\curveto(409.19144116,44.1893847)(409.06644129,44.23438465)(408.94644409,44.28438477)
\curveto(408.49644186,44.4843844)(408.12644223,44.74438414)(407.83644409,45.06438477)
\curveto(407.54644281,45.3843835)(407.30644305,45.77438311)(407.11644409,46.23438477)
\curveto(407.06644329,46.33438255)(407.02644333,46.43438245)(406.99644409,46.53438477)
\curveto(406.97644338,46.63438225)(406.9564434,46.73938215)(406.93644409,46.84938477)
\curveto(406.91644344,46.889382)(406.90644345,46.91938197)(406.90644409,46.93938477)
\curveto(406.91644344,46.96938192)(406.91644344,47.00438188)(406.90644409,47.04438477)
\curveto(406.88644347,47.12438176)(406.87144348,47.20438168)(406.86144409,47.28438477)
\curveto(406.86144349,47.37438151)(406.8514435,47.45938143)(406.83144409,47.53938477)
\lineto(406.83144409,47.65938477)
\curveto(406.83144352,47.69938119)(406.82644353,47.74438114)(406.81644409,47.79438477)
\curveto(406.80644355,47.84438104)(406.80144355,47.92938096)(406.80144409,48.04938477)
\curveto(406.80144355,48.17938071)(406.81144354,48.27438061)(406.83144409,48.33438477)
\curveto(406.8514435,48.40438048)(406.8564435,48.47438041)(406.84644409,48.54438477)
\curveto(406.83644352,48.61438027)(406.84144351,48.6843802)(406.86144409,48.75438477)
\curveto(406.87144348,48.80438008)(406.87644348,48.84438004)(406.87644409,48.87438477)
\curveto(406.88644347,48.91437997)(406.89644346,48.95937993)(406.90644409,49.00938477)
\curveto(406.93644342,49.12937976)(406.96144339,49.24937964)(406.98144409,49.36938477)
\curveto(407.01144334,49.4893794)(407.0514433,49.60437928)(407.10144409,49.71438477)
\curveto(407.2514431,50.0843788)(407.43144292,50.41437847)(407.64144409,50.70438477)
\curveto(407.86144249,51.00437788)(408.12644223,51.25437763)(408.43644409,51.45438477)
\curveto(408.5564418,51.53437735)(408.68144167,51.59937729)(408.81144409,51.64938477)
\curveto(408.94144141,51.70937718)(409.07644128,51.76937712)(409.21644409,51.82938477)
\curveto(409.33644102,51.87937701)(409.46644089,51.90937698)(409.60644409,51.91938477)
\curveto(409.74644061,51.93937695)(409.88644047,51.96937692)(410.02644409,52.00938477)
\lineto(410.22144409,52.00938477)
\curveto(410.29144006,52.01937687)(410.35644,52.02937686)(410.41644409,52.03938477)
\curveto(411.30643905,52.04937684)(412.04643831,51.86437702)(412.63644409,51.48438477)
\curveto(413.22643713,51.10437778)(413.6514367,50.60937828)(413.91144409,49.99938477)
\curveto(413.96143639,49.89937899)(414.00143635,49.79937909)(414.03144409,49.69938477)
\curveto(414.06143629,49.59937929)(414.09643626,49.49437939)(414.13644409,49.38438477)
\curveto(414.16643619,49.27437961)(414.19143616,49.15437973)(414.21144409,49.02438477)
\curveto(414.23143612,48.90437998)(414.2564361,48.77938011)(414.28644409,48.64938477)
\curveto(414.29643606,48.59938029)(414.29643606,48.54438034)(414.28644409,48.48438477)
\curveto(414.28643607,48.43438045)(414.29143606,48.3843805)(414.30144409,48.33438477)
\moveto(412.96644409,47.47938477)
\curveto(412.98643737,47.54938134)(412.99143736,47.62938126)(412.98144409,47.71938477)
\lineto(412.98144409,47.97438477)
\curveto(412.98143737,48.36438052)(412.94643741,48.69438019)(412.87644409,48.96438477)
\curveto(412.84643751,49.04437984)(412.82143753,49.12437976)(412.80144409,49.20438477)
\curveto(412.78143757,49.2843796)(412.7564376,49.35937953)(412.72644409,49.42938477)
\curveto(412.44643791,50.07937881)(412.00143835,50.52937836)(411.39144409,50.77938477)
\curveto(411.32143903,50.80937808)(411.24643911,50.82937806)(411.16644409,50.83938477)
\lineto(410.92644409,50.89938477)
\curveto(410.84643951,50.91937797)(410.76143959,50.92937796)(410.67144409,50.92938477)
\lineto(410.40144409,50.92938477)
\lineto(410.13144409,50.88438477)
\curveto(410.03144032,50.86437802)(409.93644042,50.83937805)(409.84644409,50.80938477)
\curveto(409.76644059,50.7893781)(409.68644067,50.75937813)(409.60644409,50.71938477)
\curveto(409.53644082,50.69937819)(409.47144088,50.66937822)(409.41144409,50.62938477)
\curveto(409.351441,50.5893783)(409.29644106,50.54937834)(409.24644409,50.50938477)
\curveto(409.00644135,50.33937855)(408.81144154,50.13437875)(408.66144409,49.89438477)
\curveto(408.51144184,49.65437923)(408.38144197,49.37437951)(408.27144409,49.05438477)
\curveto(408.24144211,48.95437993)(408.22144213,48.84938004)(408.21144409,48.73938477)
\curveto(408.20144215,48.63938025)(408.18644217,48.53438035)(408.16644409,48.42438477)
\curveto(408.1564422,48.3843805)(408.1514422,48.31938057)(408.15144409,48.22938477)
\curveto(408.14144221,48.19938069)(408.13644222,48.16438072)(408.13644409,48.12438477)
\curveto(408.14644221,48.0843808)(408.1514422,48.03938085)(408.15144409,47.98938477)
\lineto(408.15144409,47.68938477)
\curveto(408.1514422,47.5893813)(408.16144219,47.49938139)(408.18144409,47.41938477)
\lineto(408.21144409,47.23938477)
\curveto(408.23144212,47.13938175)(408.24644211,47.03938185)(408.25644409,46.93938477)
\curveto(408.27644208,46.84938204)(408.30644205,46.76438212)(408.34644409,46.68438477)
\curveto(408.44644191,46.44438244)(408.56144179,46.21938267)(408.69144409,46.00938477)
\curveto(408.83144152,45.79938309)(409.00144135,45.62438326)(409.20144409,45.48438477)
\curveto(409.2514411,45.45438343)(409.29644106,45.42938346)(409.33644409,45.40938477)
\curveto(409.37644098,45.3893835)(409.42144093,45.36438352)(409.47144409,45.33438477)
\curveto(409.5514408,45.2843836)(409.63644072,45.23938365)(409.72644409,45.19938477)
\curveto(409.82644053,45.16938372)(409.93144042,45.13938375)(410.04144409,45.10938477)
\curveto(410.09144026,45.0893838)(410.13644022,45.07938381)(410.17644409,45.07938477)
\curveto(410.22644013,45.0893838)(410.27644008,45.0893838)(410.32644409,45.07938477)
\curveto(410.35644,45.06938382)(410.41643994,45.05938383)(410.50644409,45.04938477)
\curveto(410.60643975,45.03938385)(410.68143967,45.04438384)(410.73144409,45.06438477)
\curveto(410.77143958,45.07438381)(410.81143954,45.07438381)(410.85144409,45.06438477)
\curveto(410.89143946,45.06438382)(410.93143942,45.07438381)(410.97144409,45.09438477)
\curveto(411.0514393,45.11438377)(411.13143922,45.12938376)(411.21144409,45.13938477)
\curveto(411.29143906,45.15938373)(411.36643899,45.1843837)(411.43644409,45.21438477)
\curveto(411.77643858,45.35438353)(412.0514383,45.54938334)(412.26144409,45.79938477)
\curveto(412.47143788,46.04938284)(412.64643771,46.34438254)(412.78644409,46.68438477)
\curveto(412.83643752,46.80438208)(412.86643749,46.92938196)(412.87644409,47.05938477)
\curveto(412.89643746,47.19938169)(412.92643743,47.33938155)(412.96644409,47.47938477)
}
}
{
\newrgbcolor{curcolor}{0 0 0}
\pscustom[linestyle=none,fillstyle=solid,fillcolor=curcolor]
{
\newpath
\moveto(418.92472534,52.03938477)
\curveto(419.66472055,52.04937684)(420.27971994,51.93937695)(420.76972534,51.70938477)
\curveto(421.26971895,51.4893774)(421.66471855,51.15437773)(421.95472534,50.70438477)
\curveto(422.08471813,50.50437838)(422.19471802,50.25937863)(422.28472534,49.96938477)
\curveto(422.30471791,49.91937897)(422.3197179,49.85437903)(422.32972534,49.77438477)
\curveto(422.33971788,49.69437919)(422.33471788,49.62437926)(422.31472534,49.56438477)
\curveto(422.28471793,49.51437937)(422.23471798,49.46937942)(422.16472534,49.42938477)
\curveto(422.13471808,49.40937948)(422.10471811,49.39937949)(422.07472534,49.39938477)
\curveto(422.04471817,49.40937948)(422.00971821,49.40937948)(421.96972534,49.39938477)
\curveto(421.92971829,49.3893795)(421.88971833,49.3843795)(421.84972534,49.38438477)
\curveto(421.80971841,49.39437949)(421.76971845,49.39937949)(421.72972534,49.39938477)
\lineto(421.41472534,49.39938477)
\curveto(421.3147189,49.40937948)(421.22971899,49.43937945)(421.15972534,49.48938477)
\curveto(421.07971914,49.54937934)(421.02471919,49.63437925)(420.99472534,49.74438477)
\curveto(420.96471925,49.85437903)(420.92471929,49.94937894)(420.87472534,50.02938477)
\curveto(420.72471949,50.2893786)(420.52971969,50.49437839)(420.28972534,50.64438477)
\curveto(420.20972001,50.69437819)(420.12472009,50.73437815)(420.03472534,50.76438477)
\curveto(419.94472027,50.80437808)(419.84972037,50.83937805)(419.74972534,50.86938477)
\curveto(419.60972061,50.90937798)(419.42472079,50.92937796)(419.19472534,50.92938477)
\curveto(418.96472125,50.93937795)(418.77472144,50.91937797)(418.62472534,50.86938477)
\curveto(418.55472166,50.84937804)(418.48972173,50.83437805)(418.42972534,50.82438477)
\curveto(418.36972185,50.81437807)(418.30472191,50.79937809)(418.23472534,50.77938477)
\curveto(417.97472224,50.66937822)(417.74472247,50.51937837)(417.54472534,50.32938477)
\curveto(417.34472287,50.13937875)(417.18972303,49.91437897)(417.07972534,49.65438477)
\curveto(417.03972318,49.56437932)(417.00472321,49.46937942)(416.97472534,49.36938477)
\curveto(416.94472327,49.27937961)(416.9147233,49.17937971)(416.88472534,49.06938477)
\lineto(416.79472534,48.66438477)
\curveto(416.78472343,48.61438027)(416.77972344,48.55938033)(416.77972534,48.49938477)
\curveto(416.78972343,48.43938045)(416.78472343,48.3843805)(416.76472534,48.33438477)
\lineto(416.76472534,48.21438477)
\curveto(416.75472346,48.17438071)(416.74472347,48.10938078)(416.73472534,48.01938477)
\curveto(416.73472348,47.92938096)(416.74472347,47.86438102)(416.76472534,47.82438477)
\curveto(416.77472344,47.77438111)(416.77472344,47.72438116)(416.76472534,47.67438477)
\curveto(416.75472346,47.62438126)(416.75472346,47.57438131)(416.76472534,47.52438477)
\curveto(416.77472344,47.4843814)(416.77972344,47.41438147)(416.77972534,47.31438477)
\curveto(416.79972342,47.23438165)(416.8147234,47.14938174)(416.82472534,47.05938477)
\curveto(416.84472337,46.96938192)(416.86472335,46.884382)(416.88472534,46.80438477)
\curveto(416.99472322,46.4843824)(417.1197231,46.20438268)(417.25972534,45.96438477)
\curveto(417.40972281,45.73438315)(417.6147226,45.53438335)(417.87472534,45.36438477)
\curveto(417.96472225,45.31438357)(418.05472216,45.26938362)(418.14472534,45.22938477)
\curveto(418.24472197,45.1893837)(418.34972187,45.14938374)(418.45972534,45.10938477)
\curveto(418.50972171,45.09938379)(418.54972167,45.09438379)(418.57972534,45.09438477)
\curveto(418.60972161,45.09438379)(418.64972157,45.0893838)(418.69972534,45.07938477)
\curveto(418.72972149,45.06938382)(418.77972144,45.06438382)(418.84972534,45.06438477)
\lineto(419.01472534,45.06438477)
\curveto(419.0147212,45.05438383)(419.03472118,45.04938384)(419.07472534,45.04938477)
\curveto(419.09472112,45.05938383)(419.1197211,45.05938383)(419.14972534,45.04938477)
\curveto(419.17972104,45.04938384)(419.20972101,45.05438383)(419.23972534,45.06438477)
\curveto(419.30972091,45.0843838)(419.37472084,45.0893838)(419.43472534,45.07938477)
\curveto(419.50472071,45.07938381)(419.57472064,45.0893838)(419.64472534,45.10938477)
\curveto(419.90472031,45.1893837)(420.12972009,45.2893836)(420.31972534,45.40938477)
\curveto(420.50971971,45.53938335)(420.66971955,45.70438318)(420.79972534,45.90438477)
\curveto(420.84971937,45.9843829)(420.89471932,46.06938282)(420.93472534,46.15938477)
\lineto(421.05472534,46.42938477)
\curveto(421.07471914,46.50938238)(421.09471912,46.5843823)(421.11472534,46.65438477)
\curveto(421.14471907,46.73438215)(421.19471902,46.79938209)(421.26472534,46.84938477)
\curveto(421.29471892,46.87938201)(421.35471886,46.89938199)(421.44472534,46.90938477)
\curveto(421.53471868,46.92938196)(421.62971859,46.93938195)(421.72972534,46.93938477)
\curveto(421.83971838,46.94938194)(421.93971828,46.94938194)(422.02972534,46.93938477)
\curveto(422.12971809,46.92938196)(422.19971802,46.90938198)(422.23972534,46.87938477)
\curveto(422.29971792,46.83938205)(422.33471788,46.77938211)(422.34472534,46.69938477)
\curveto(422.36471785,46.61938227)(422.36471785,46.53438235)(422.34472534,46.44438477)
\curveto(422.29471792,46.29438259)(422.24471797,46.14938274)(422.19472534,46.00938477)
\curveto(422.15471806,45.87938301)(422.09971812,45.74938314)(422.02972534,45.61938477)
\curveto(421.87971834,45.31938357)(421.68971853,45.05438383)(421.45972534,44.82438477)
\curveto(421.23971898,44.59438429)(420.96971925,44.40938448)(420.64972534,44.26938477)
\curveto(420.56971965,44.22938466)(420.48471973,44.19438469)(420.39472534,44.16438477)
\curveto(420.30471991,44.14438474)(420.20972001,44.11938477)(420.10972534,44.08938477)
\curveto(419.99972022,44.04938484)(419.88972033,44.02938486)(419.77972534,44.02938477)
\curveto(419.66972055,44.01938487)(419.55972066,44.00438488)(419.44972534,43.98438477)
\curveto(419.40972081,43.96438492)(419.36972085,43.95938493)(419.32972534,43.96938477)
\curveto(419.28972093,43.97938491)(419.24972097,43.97938491)(419.20972534,43.96938477)
\lineto(419.07472534,43.96938477)
\lineto(418.83472534,43.96938477)
\curveto(418.76472145,43.95938493)(418.69972152,43.96438492)(418.63972534,43.98438477)
\lineto(418.56472534,43.98438477)
\lineto(418.20472534,44.02938477)
\curveto(418.07472214,44.06938482)(417.94972227,44.10438478)(417.82972534,44.13438477)
\curveto(417.70972251,44.16438472)(417.59472262,44.20438468)(417.48472534,44.25438477)
\curveto(417.12472309,44.41438447)(416.82472339,44.60438428)(416.58472534,44.82438477)
\curveto(416.35472386,45.04438384)(416.13972408,45.31438357)(415.93972534,45.63438477)
\curveto(415.88972433,45.71438317)(415.84472437,45.80438308)(415.80472534,45.90438477)
\lineto(415.68472534,46.20438477)
\curveto(415.63472458,46.31438257)(415.59972462,46.42938246)(415.57972534,46.54938477)
\curveto(415.55972466,46.66938222)(415.53472468,46.7893821)(415.50472534,46.90938477)
\curveto(415.49472472,46.94938194)(415.48972473,46.9893819)(415.48972534,47.02938477)
\curveto(415.48972473,47.06938182)(415.48472473,47.10938178)(415.47472534,47.14938477)
\curveto(415.45472476,47.20938168)(415.44472477,47.27438161)(415.44472534,47.34438477)
\curveto(415.45472476,47.41438147)(415.44972477,47.47938141)(415.42972534,47.53938477)
\lineto(415.42972534,47.68938477)
\curveto(415.4197248,47.73938115)(415.4147248,47.80938108)(415.41472534,47.89938477)
\curveto(415.4147248,47.9893809)(415.4197248,48.05938083)(415.42972534,48.10938477)
\curveto(415.43972478,48.15938073)(415.43972478,48.20438068)(415.42972534,48.24438477)
\curveto(415.42972479,48.2843806)(415.43472478,48.32438056)(415.44472534,48.36438477)
\curveto(415.46472475,48.43438045)(415.46972475,48.50438038)(415.45972534,48.57438477)
\curveto(415.45972476,48.64438024)(415.46972475,48.70938018)(415.48972534,48.76938477)
\curveto(415.52972469,48.93937995)(415.56472465,49.10937978)(415.59472534,49.27938477)
\curveto(415.62472459,49.44937944)(415.66972455,49.60937928)(415.72972534,49.75938477)
\curveto(415.93972428,50.27937861)(416.19472402,50.69937819)(416.49472534,51.01938477)
\curveto(416.79472342,51.33937755)(417.20472301,51.60437728)(417.72472534,51.81438477)
\curveto(417.83472238,51.86437702)(417.95472226,51.89937699)(418.08472534,51.91938477)
\curveto(418.214722,51.93937695)(418.34972187,51.96437692)(418.48972534,51.99438477)
\curveto(418.55972166,52.00437688)(418.62972159,52.00937688)(418.69972534,52.00938477)
\curveto(418.76972145,52.01937687)(418.84472137,52.02937686)(418.92472534,52.03938477)
}
}
{
\newrgbcolor{curcolor}{0 0 0}
\pscustom[linestyle=none,fillstyle=solid,fillcolor=curcolor]
{
\newpath
\moveto(430.59636597,48.30438477)
\curveto(430.61635828,48.20438068)(430.61635828,48.0893808)(430.59636597,47.95938477)
\curveto(430.58635831,47.83938105)(430.55635834,47.75438113)(430.50636597,47.70438477)
\curveto(430.45635844,47.66438122)(430.38135852,47.63438125)(430.28136597,47.61438477)
\curveto(430.19135871,47.60438128)(430.08635881,47.59938129)(429.96636597,47.59938477)
\lineto(429.60636597,47.59938477)
\curveto(429.48635941,47.60938128)(429.38135952,47.61438127)(429.29136597,47.61438477)
\lineto(425.45136597,47.61438477)
\curveto(425.37136353,47.61438127)(425.29136361,47.60938128)(425.21136597,47.59938477)
\curveto(425.13136377,47.59938129)(425.06636383,47.5843813)(425.01636597,47.55438477)
\curveto(424.97636392,47.53438135)(424.93636396,47.49438139)(424.89636597,47.43438477)
\curveto(424.87636402,47.40438148)(424.85636404,47.35938153)(424.83636597,47.29938477)
\curveto(424.81636408,47.24938164)(424.81636408,47.19938169)(424.83636597,47.14938477)
\curveto(424.84636405,47.09938179)(424.85136405,47.05438183)(424.85136597,47.01438477)
\curveto(424.85136405,46.97438191)(424.85636404,46.93438195)(424.86636597,46.89438477)
\curveto(424.88636401,46.81438207)(424.90636399,46.72938216)(424.92636597,46.63938477)
\curveto(424.94636395,46.55938233)(424.97636392,46.47938241)(425.01636597,46.39938477)
\curveto(425.24636365,45.85938303)(425.62636327,45.47438341)(426.15636597,45.24438477)
\curveto(426.21636268,45.21438367)(426.28136262,45.1893837)(426.35136597,45.16938477)
\lineto(426.56136597,45.10938477)
\curveto(426.59136231,45.09938379)(426.64136226,45.09438379)(426.71136597,45.09438477)
\curveto(426.85136205,45.05438383)(427.03636186,45.03438385)(427.26636597,45.03438477)
\curveto(427.4963614,45.03438385)(427.68136122,45.05438383)(427.82136597,45.09438477)
\curveto(427.96136094,45.13438375)(428.08636081,45.17438371)(428.19636597,45.21438477)
\curveto(428.31636058,45.26438362)(428.42636047,45.32438356)(428.52636597,45.39438477)
\curveto(428.63636026,45.46438342)(428.73136017,45.54438334)(428.81136597,45.63438477)
\curveto(428.89136001,45.73438315)(428.96135994,45.83938305)(429.02136597,45.94938477)
\curveto(429.08135982,46.04938284)(429.13135977,46.15438273)(429.17136597,46.26438477)
\curveto(429.22135968,46.37438251)(429.3013596,46.45438243)(429.41136597,46.50438477)
\curveto(429.45135945,46.52438236)(429.51635938,46.53938235)(429.60636597,46.54938477)
\curveto(429.6963592,46.55938233)(429.78635911,46.55938233)(429.87636597,46.54938477)
\curveto(429.96635893,46.54938234)(430.05135885,46.54438234)(430.13136597,46.53438477)
\curveto(430.21135869,46.52438236)(430.26635863,46.50438238)(430.29636597,46.47438477)
\curveto(430.3963585,46.40438248)(430.42135848,46.2893826)(430.37136597,46.12938477)
\curveto(430.29135861,45.85938303)(430.18635871,45.61938327)(430.05636597,45.40938477)
\curveto(429.85635904,45.0893838)(429.62635927,44.82438406)(429.36636597,44.61438477)
\curveto(429.11635978,44.41438447)(428.7963601,44.24938464)(428.40636597,44.11938477)
\curveto(428.30636059,44.07938481)(428.20636069,44.05438483)(428.10636597,44.04438477)
\curveto(428.00636089,44.02438486)(427.901361,44.00438488)(427.79136597,43.98438477)
\curveto(427.74136116,43.97438491)(427.69136121,43.96938492)(427.64136597,43.96938477)
\curveto(427.6013613,43.96938492)(427.55636134,43.96438492)(427.50636597,43.95438477)
\lineto(427.35636597,43.95438477)
\curveto(427.30636159,43.94438494)(427.24636165,43.93938495)(427.17636597,43.93938477)
\curveto(427.11636178,43.93938495)(427.06636183,43.94438494)(427.02636597,43.95438477)
\lineto(426.89136597,43.95438477)
\curveto(426.84136206,43.96438492)(426.7963621,43.96938492)(426.75636597,43.96938477)
\curveto(426.71636218,43.96938492)(426.67636222,43.97438491)(426.63636597,43.98438477)
\curveto(426.58636231,43.99438489)(426.53136237,44.00438488)(426.47136597,44.01438477)
\curveto(426.41136249,44.01438487)(426.35636254,44.01938487)(426.30636597,44.02938477)
\curveto(426.21636268,44.04938484)(426.12636277,44.07438481)(426.03636597,44.10438477)
\curveto(425.94636295,44.12438476)(425.86136304,44.14938474)(425.78136597,44.17938477)
\curveto(425.74136316,44.19938469)(425.70636319,44.20938468)(425.67636597,44.20938477)
\curveto(425.64636325,44.21938467)(425.61136329,44.23438465)(425.57136597,44.25438477)
\curveto(425.42136348,44.32438456)(425.26136364,44.40938448)(425.09136597,44.50938477)
\curveto(424.8013641,44.69938419)(424.55136435,44.92938396)(424.34136597,45.19938477)
\curveto(424.14136476,45.47938341)(423.97136493,45.7893831)(423.83136597,46.12938477)
\curveto(423.78136512,46.23938265)(423.74136516,46.35438253)(423.71136597,46.47438477)
\curveto(423.69136521,46.59438229)(423.66136524,46.71438217)(423.62136597,46.83438477)
\curveto(423.61136529,46.87438201)(423.60636529,46.90938198)(423.60636597,46.93938477)
\curveto(423.60636529,46.96938192)(423.6013653,47.00938188)(423.59136597,47.05938477)
\curveto(423.57136533,47.13938175)(423.55636534,47.22438166)(423.54636597,47.31438477)
\curveto(423.53636536,47.40438148)(423.52136538,47.49438139)(423.50136597,47.58438477)
\lineto(423.50136597,47.79438477)
\curveto(423.49136541,47.83438105)(423.48136542,47.889381)(423.47136597,47.95938477)
\curveto(423.47136543,48.03938085)(423.47636542,48.10438078)(423.48636597,48.15438477)
\lineto(423.48636597,48.31938477)
\curveto(423.50636539,48.36938052)(423.51136539,48.41938047)(423.50136597,48.46938477)
\curveto(423.5013654,48.52938036)(423.50636539,48.5843803)(423.51636597,48.63438477)
\curveto(423.55636534,48.79438009)(423.58636531,48.95437993)(423.60636597,49.11438477)
\curveto(423.63636526,49.27437961)(423.68136522,49.42437946)(423.74136597,49.56438477)
\curveto(423.79136511,49.67437921)(423.83636506,49.7843791)(423.87636597,49.89438477)
\curveto(423.92636497,50.01437887)(423.98136492,50.12937876)(424.04136597,50.23938477)
\curveto(424.26136464,50.5893783)(424.51136439,50.889378)(424.79136597,51.13938477)
\curveto(425.07136383,51.39937749)(425.41636348,51.61437727)(425.82636597,51.78438477)
\curveto(425.94636295,51.83437705)(426.06636283,51.86937702)(426.18636597,51.88938477)
\curveto(426.31636258,51.91937697)(426.45136245,51.94937694)(426.59136597,51.97938477)
\curveto(426.64136226,51.9893769)(426.68636221,51.99437689)(426.72636597,51.99438477)
\curveto(426.76636213,52.00437688)(426.81136209,52.00937688)(426.86136597,52.00938477)
\curveto(426.88136202,52.01937687)(426.90636199,52.01937687)(426.93636597,52.00938477)
\curveto(426.96636193,51.99937689)(426.99136191,52.00437688)(427.01136597,52.02438477)
\curveto(427.43136147,52.03437685)(427.7963611,51.9893769)(428.10636597,51.88938477)
\curveto(428.41636048,51.79937709)(428.6963602,51.67437721)(428.94636597,51.51438477)
\curveto(428.9963599,51.49437739)(429.03635986,51.46437742)(429.06636597,51.42438477)
\curveto(429.0963598,51.39437749)(429.13135977,51.36937752)(429.17136597,51.34938477)
\curveto(429.25135965,51.2893776)(429.33135957,51.21937767)(429.41136597,51.13938477)
\curveto(429.5013594,51.05937783)(429.57635932,50.97937791)(429.63636597,50.89938477)
\curveto(429.7963591,50.6893782)(429.93135897,50.4893784)(430.04136597,50.29938477)
\curveto(430.11135879,50.1893787)(430.16635873,50.06937882)(430.20636597,49.93938477)
\curveto(430.24635865,49.80937908)(430.29135861,49.67937921)(430.34136597,49.54938477)
\curveto(430.39135851,49.41937947)(430.42635847,49.2843796)(430.44636597,49.14438477)
\curveto(430.47635842,49.00437988)(430.51135839,48.86438002)(430.55136597,48.72438477)
\curveto(430.56135834,48.65438023)(430.56635833,48.5843803)(430.56636597,48.51438477)
\lineto(430.59636597,48.30438477)
\moveto(429.14136597,48.81438477)
\curveto(429.17135973,48.85438003)(429.1963597,48.90437998)(429.21636597,48.96438477)
\curveto(429.23635966,49.03437985)(429.23635966,49.10437978)(429.21636597,49.17438477)
\curveto(429.15635974,49.39437949)(429.07135983,49.59937929)(428.96136597,49.78938477)
\curveto(428.82136008,50.01937887)(428.66636023,50.21437867)(428.49636597,50.37438477)
\curveto(428.32636057,50.53437835)(428.10636079,50.66937822)(427.83636597,50.77938477)
\curveto(427.76636113,50.79937809)(427.6963612,50.81437807)(427.62636597,50.82438477)
\curveto(427.55636134,50.84437804)(427.48136142,50.86437802)(427.40136597,50.88438477)
\curveto(427.32136158,50.90437798)(427.23636166,50.91437797)(427.14636597,50.91438477)
\lineto(426.89136597,50.91438477)
\curveto(426.86136204,50.89437799)(426.82636207,50.884378)(426.78636597,50.88438477)
\curveto(426.74636215,50.89437799)(426.71136219,50.89437799)(426.68136597,50.88438477)
\lineto(426.44136597,50.82438477)
\curveto(426.37136253,50.81437807)(426.3013626,50.79937809)(426.23136597,50.77938477)
\curveto(425.94136296,50.65937823)(425.70636319,50.50937838)(425.52636597,50.32938477)
\curveto(425.35636354,50.14937874)(425.2013637,49.92437896)(425.06136597,49.65438477)
\curveto(425.03136387,49.60437928)(425.0013639,49.53937935)(424.97136597,49.45938477)
\curveto(424.94136396,49.3893795)(424.91636398,49.30937958)(424.89636597,49.21938477)
\curveto(424.87636402,49.12937976)(424.87136403,49.04437984)(424.88136597,48.96438477)
\curveto(424.89136401,48.88438)(424.92636397,48.82438006)(424.98636597,48.78438477)
\curveto(425.06636383,48.72438016)(425.2013637,48.69438019)(425.39136597,48.69438477)
\curveto(425.59136331,48.70438018)(425.76136314,48.70938018)(425.90136597,48.70938477)
\lineto(428.18136597,48.70938477)
\curveto(428.33136057,48.70938018)(428.51136039,48.70438018)(428.72136597,48.69438477)
\curveto(428.93135997,48.69438019)(429.07135983,48.73438015)(429.14136597,48.81438477)
}
}
{
\newrgbcolor{curcolor}{0 0 0}
\pscustom[linestyle=none,fillstyle=solid,fillcolor=curcolor]
{
\newpath
\moveto(435.59300659,52.00938477)
\curveto(436.22300136,52.02937686)(436.72800085,51.94437694)(437.10800659,51.75438477)
\curveto(437.48800009,51.56437732)(437.79299979,51.27937761)(438.02300659,50.89938477)
\curveto(438.0829995,50.79937809)(438.12799945,50.6893782)(438.15800659,50.56938477)
\curveto(438.19799938,50.45937843)(438.23299935,50.34437854)(438.26300659,50.22438477)
\curveto(438.31299927,50.03437885)(438.34299924,49.82937906)(438.35300659,49.60938477)
\curveto(438.36299922,49.3893795)(438.36799921,49.16437972)(438.36800659,48.93438477)
\lineto(438.36800659,47.32938477)
\lineto(438.36800659,44.98938477)
\curveto(438.36799921,44.81938407)(438.36299922,44.64938424)(438.35300659,44.47938477)
\curveto(438.35299923,44.30938458)(438.28799929,44.19938469)(438.15800659,44.14938477)
\curveto(438.10799947,44.12938476)(438.05299953,44.11938477)(437.99300659,44.11938477)
\curveto(437.94299964,44.10938478)(437.88799969,44.10438478)(437.82800659,44.10438477)
\curveto(437.69799988,44.10438478)(437.57300001,44.10938478)(437.45300659,44.11938477)
\curveto(437.33300025,44.11938477)(437.24800033,44.15938473)(437.19800659,44.23938477)
\curveto(437.14800043,44.30938458)(437.12300046,44.39938449)(437.12300659,44.50938477)
\lineto(437.12300659,44.83938477)
\lineto(437.12300659,46.12938477)
\lineto(437.12300659,48.57438477)
\curveto(437.12300046,48.84438004)(437.11800046,49.10937978)(437.10800659,49.36938477)
\curveto(437.09800048,49.63937925)(437.05300053,49.86937902)(436.97300659,50.05938477)
\curveto(436.89300069,50.25937863)(436.77300081,50.41937847)(436.61300659,50.53938477)
\curveto(436.45300113,50.66937822)(436.26800131,50.76937812)(436.05800659,50.83938477)
\curveto(435.99800158,50.85937803)(435.93300165,50.86937802)(435.86300659,50.86938477)
\curveto(435.80300178,50.87937801)(435.74300184,50.89437799)(435.68300659,50.91438477)
\curveto(435.63300195,50.92437796)(435.55300203,50.92437796)(435.44300659,50.91438477)
\curveto(435.34300224,50.91437797)(435.27300231,50.90937798)(435.23300659,50.89938477)
\curveto(435.19300239,50.87937801)(435.15800242,50.86937802)(435.12800659,50.86938477)
\curveto(435.09800248,50.87937801)(435.06300252,50.87937801)(435.02300659,50.86938477)
\curveto(434.89300269,50.83937805)(434.76800281,50.80437808)(434.64800659,50.76438477)
\curveto(434.53800304,50.73437815)(434.43300315,50.6893782)(434.33300659,50.62938477)
\curveto(434.29300329,50.60937828)(434.25800332,50.5893783)(434.22800659,50.56938477)
\curveto(434.19800338,50.54937834)(434.16300342,50.52937836)(434.12300659,50.50938477)
\curveto(433.77300381,50.25937863)(433.51800406,49.884379)(433.35800659,49.38438477)
\curveto(433.32800425,49.30437958)(433.30800427,49.21937967)(433.29800659,49.12938477)
\curveto(433.28800429,49.04937984)(433.27300431,48.96937992)(433.25300659,48.88938477)
\curveto(433.23300435,48.83938005)(433.22800435,48.7893801)(433.23800659,48.73938477)
\curveto(433.24800433,48.69938019)(433.24300434,48.65938023)(433.22300659,48.61938477)
\lineto(433.22300659,48.30438477)
\curveto(433.21300437,48.27438061)(433.20800437,48.23938065)(433.20800659,48.19938477)
\curveto(433.21800436,48.15938073)(433.22300436,48.11438077)(433.22300659,48.06438477)
\lineto(433.22300659,47.61438477)
\lineto(433.22300659,46.17438477)
\lineto(433.22300659,44.85438477)
\lineto(433.22300659,44.50938477)
\curveto(433.22300436,44.39938449)(433.19800438,44.30938458)(433.14800659,44.23938477)
\curveto(433.09800448,44.15938473)(433.00800457,44.11938477)(432.87800659,44.11938477)
\curveto(432.75800482,44.10938478)(432.63300495,44.10438478)(432.50300659,44.10438477)
\curveto(432.42300516,44.10438478)(432.34800523,44.10938478)(432.27800659,44.11938477)
\curveto(432.20800537,44.12938476)(432.14800543,44.15438473)(432.09800659,44.19438477)
\curveto(432.01800556,44.24438464)(431.9780056,44.33938455)(431.97800659,44.47938477)
\lineto(431.97800659,44.88438477)
\lineto(431.97800659,46.65438477)
\lineto(431.97800659,50.28438477)
\lineto(431.97800659,51.19938477)
\lineto(431.97800659,51.46938477)
\curveto(431.9780056,51.55937733)(431.99800558,51.62937726)(432.03800659,51.67938477)
\curveto(432.06800551,51.73937715)(432.11800546,51.77937711)(432.18800659,51.79938477)
\curveto(432.22800535,51.80937708)(432.2830053,51.81937707)(432.35300659,51.82938477)
\curveto(432.43300515,51.83937705)(432.51300507,51.84437704)(432.59300659,51.84438477)
\curveto(432.67300491,51.84437704)(432.74800483,51.83937705)(432.81800659,51.82938477)
\curveto(432.89800468,51.81937707)(432.95300463,51.80437708)(432.98300659,51.78438477)
\curveto(433.09300449,51.71437717)(433.14300444,51.62437726)(433.13300659,51.51438477)
\curveto(433.12300446,51.41437747)(433.13800444,51.29937759)(433.17800659,51.16938477)
\curveto(433.19800438,51.10937778)(433.23800434,51.05937783)(433.29800659,51.01938477)
\curveto(433.41800416,51.00937788)(433.51300407,51.05437783)(433.58300659,51.15438477)
\curveto(433.66300392,51.25437763)(433.74300384,51.33437755)(433.82300659,51.39438477)
\curveto(433.96300362,51.49437739)(434.10300348,51.5843773)(434.24300659,51.66438477)
\curveto(434.39300319,51.75437713)(434.56300302,51.82937706)(434.75300659,51.88938477)
\curveto(434.83300275,51.91937697)(434.91800266,51.93937695)(435.00800659,51.94938477)
\curveto(435.10800247,51.95937693)(435.20300238,51.97437691)(435.29300659,51.99438477)
\curveto(435.34300224,52.00437688)(435.39300219,52.00937688)(435.44300659,52.00938477)
\lineto(435.59300659,52.00938477)
}
}
{
\newrgbcolor{curcolor}{0 0 0}
\pscustom[linestyle=none,fillstyle=solid,fillcolor=curcolor]
{
\newpath
\moveto(441.19761597,54.19938477)
\curveto(441.34761396,54.19937469)(441.49761381,54.19437469)(441.64761597,54.18438477)
\curveto(441.79761351,54.1843747)(441.9026134,54.14437474)(441.96261597,54.06438477)
\curveto(442.01261329,54.00437488)(442.03761327,53.91937497)(442.03761597,53.80938477)
\curveto(442.04761326,53.70937518)(442.05261325,53.60437528)(442.05261597,53.49438477)
\lineto(442.05261597,52.62438477)
\curveto(442.05261325,52.54437634)(442.04761326,52.45937643)(442.03761597,52.36938477)
\curveto(442.03761327,52.2893766)(442.04761326,52.21937667)(442.06761597,52.15938477)
\curveto(442.1076132,52.01937687)(442.19761311,51.92937696)(442.33761597,51.88938477)
\curveto(442.38761292,51.87937701)(442.43261287,51.87437701)(442.47261597,51.87438477)
\lineto(442.62261597,51.87438477)
\lineto(443.02761597,51.87438477)
\curveto(443.18761212,51.884377)(443.302612,51.87437701)(443.37261597,51.84438477)
\curveto(443.46261184,51.7843771)(443.52261178,51.72437716)(443.55261597,51.66438477)
\curveto(443.57261173,51.62437726)(443.58261172,51.57937731)(443.58261597,51.52938477)
\lineto(443.58261597,51.37938477)
\curveto(443.58261172,51.26937762)(443.57761173,51.16437772)(443.56761597,51.06438477)
\curveto(443.55761175,50.97437791)(443.52261178,50.90437798)(443.46261597,50.85438477)
\curveto(443.4026119,50.80437808)(443.31761199,50.77437811)(443.20761597,50.76438477)
\lineto(442.87761597,50.76438477)
\curveto(442.76761254,50.77437811)(442.65761265,50.77937811)(442.54761597,50.77938477)
\curveto(442.43761287,50.77937811)(442.34261296,50.76437812)(442.26261597,50.73438477)
\curveto(442.19261311,50.70437818)(442.14261316,50.65437823)(442.11261597,50.58438477)
\curveto(442.08261322,50.51437837)(442.06261324,50.42937846)(442.05261597,50.32938477)
\curveto(442.04261326,50.23937865)(442.03761327,50.13937875)(442.03761597,50.02938477)
\curveto(442.04761326,49.92937896)(442.05261325,49.82937906)(442.05261597,49.72938477)
\lineto(442.05261597,46.75938477)
\curveto(442.05261325,46.53938235)(442.04761326,46.30438258)(442.03761597,46.05438477)
\curveto(442.03761327,45.81438307)(442.08261322,45.62938326)(442.17261597,45.49938477)
\curveto(442.22261308,45.41938347)(442.28761302,45.36438352)(442.36761597,45.33438477)
\curveto(442.44761286,45.30438358)(442.54261276,45.27938361)(442.65261597,45.25938477)
\curveto(442.68261262,45.24938364)(442.71261259,45.24438364)(442.74261597,45.24438477)
\curveto(442.78261252,45.25438363)(442.81761249,45.25438363)(442.84761597,45.24438477)
\lineto(443.04261597,45.24438477)
\curveto(443.14261216,45.24438364)(443.23261207,45.23438365)(443.31261597,45.21438477)
\curveto(443.4026119,45.20438368)(443.46761184,45.16938372)(443.50761597,45.10938477)
\curveto(443.52761178,45.07938381)(443.54261176,45.02438386)(443.55261597,44.94438477)
\curveto(443.57261173,44.87438401)(443.58261172,44.79938409)(443.58261597,44.71938477)
\curveto(443.59261171,44.63938425)(443.59261171,44.55938433)(443.58261597,44.47938477)
\curveto(443.57261173,44.40938448)(443.55261175,44.35438453)(443.52261597,44.31438477)
\curveto(443.48261182,44.24438464)(443.4076119,44.19438469)(443.29761597,44.16438477)
\curveto(443.21761209,44.14438474)(443.12761218,44.13438475)(443.02761597,44.13438477)
\curveto(442.92761238,44.14438474)(442.83761247,44.14938474)(442.75761597,44.14938477)
\curveto(442.69761261,44.14938474)(442.63761267,44.14438474)(442.57761597,44.13438477)
\curveto(442.51761279,44.13438475)(442.46261284,44.13938475)(442.41261597,44.14938477)
\lineto(442.23261597,44.14938477)
\curveto(442.18261312,44.15938473)(442.13261317,44.16438472)(442.08261597,44.16438477)
\curveto(442.04261326,44.17438471)(441.99761331,44.17938471)(441.94761597,44.17938477)
\curveto(441.74761356,44.22938466)(441.57261373,44.2843846)(441.42261597,44.34438477)
\curveto(441.28261402,44.40438448)(441.16261414,44.50938438)(441.06261597,44.65938477)
\curveto(440.92261438,44.85938403)(440.84261446,45.10938378)(440.82261597,45.40938477)
\curveto(440.8026145,45.71938317)(440.79261451,46.04938284)(440.79261597,46.39938477)
\lineto(440.79261597,50.32938477)
\curveto(440.76261454,50.45937843)(440.73261457,50.55437833)(440.70261597,50.61438477)
\curveto(440.68261462,50.67437821)(440.61261469,50.72437816)(440.49261597,50.76438477)
\curveto(440.45261485,50.77437811)(440.41261489,50.77437811)(440.37261597,50.76438477)
\curveto(440.33261497,50.75437813)(440.29261501,50.75937813)(440.25261597,50.77938477)
\lineto(440.01261597,50.77938477)
\curveto(439.88261542,50.77937811)(439.77261553,50.7893781)(439.68261597,50.80938477)
\curveto(439.6026157,50.83937805)(439.54761576,50.89937799)(439.51761597,50.98938477)
\curveto(439.49761581,51.02937786)(439.48261582,51.07437781)(439.47261597,51.12438477)
\lineto(439.47261597,51.27438477)
\curveto(439.47261583,51.41437747)(439.48261582,51.52937736)(439.50261597,51.61938477)
\curveto(439.52261578,51.71937717)(439.58261572,51.79437709)(439.68261597,51.84438477)
\curveto(439.79261551,51.884377)(439.93261537,51.89437699)(440.10261597,51.87438477)
\curveto(440.28261502,51.85437703)(440.43261487,51.86437702)(440.55261597,51.90438477)
\curveto(440.64261466,51.95437693)(440.71261459,52.02437686)(440.76261597,52.11438477)
\curveto(440.78261452,52.17437671)(440.79261451,52.24937664)(440.79261597,52.33938477)
\lineto(440.79261597,52.59438477)
\lineto(440.79261597,53.52438477)
\lineto(440.79261597,53.76438477)
\curveto(440.79261451,53.85437503)(440.8026145,53.92937496)(440.82261597,53.98938477)
\curveto(440.86261444,54.06937482)(440.93761437,54.13437475)(441.04761597,54.18438477)
\curveto(441.07761423,54.1843747)(441.1026142,54.1843747)(441.12261597,54.18438477)
\curveto(441.15261415,54.19437469)(441.17761413,54.19937469)(441.19761597,54.19938477)
}
}
{
\newrgbcolor{curcolor}{0 0 0}
\pscustom[linestyle=none,fillstyle=solid,fillcolor=curcolor]
{
\newpath
\moveto(451.71941284,48.30438477)
\curveto(451.73940516,48.20438068)(451.73940516,48.0893808)(451.71941284,47.95938477)
\curveto(451.70940519,47.83938105)(451.67940522,47.75438113)(451.62941284,47.70438477)
\curveto(451.57940532,47.66438122)(451.50440539,47.63438125)(451.40441284,47.61438477)
\curveto(451.31440558,47.60438128)(451.20940569,47.59938129)(451.08941284,47.59938477)
\lineto(450.72941284,47.59938477)
\curveto(450.60940629,47.60938128)(450.50440639,47.61438127)(450.41441284,47.61438477)
\lineto(446.57441284,47.61438477)
\curveto(446.4944104,47.61438127)(446.41441048,47.60938128)(446.33441284,47.59938477)
\curveto(446.25441064,47.59938129)(446.18941071,47.5843813)(446.13941284,47.55438477)
\curveto(446.0994108,47.53438135)(446.05941084,47.49438139)(446.01941284,47.43438477)
\curveto(445.9994109,47.40438148)(445.97941092,47.35938153)(445.95941284,47.29938477)
\curveto(445.93941096,47.24938164)(445.93941096,47.19938169)(445.95941284,47.14938477)
\curveto(445.96941093,47.09938179)(445.97441092,47.05438183)(445.97441284,47.01438477)
\curveto(445.97441092,46.97438191)(445.97941092,46.93438195)(445.98941284,46.89438477)
\curveto(446.00941089,46.81438207)(446.02941087,46.72938216)(446.04941284,46.63938477)
\curveto(446.06941083,46.55938233)(446.0994108,46.47938241)(446.13941284,46.39938477)
\curveto(446.36941053,45.85938303)(446.74941015,45.47438341)(447.27941284,45.24438477)
\curveto(447.33940956,45.21438367)(447.40440949,45.1893837)(447.47441284,45.16938477)
\lineto(447.68441284,45.10938477)
\curveto(447.71440918,45.09938379)(447.76440913,45.09438379)(447.83441284,45.09438477)
\curveto(447.97440892,45.05438383)(448.15940874,45.03438385)(448.38941284,45.03438477)
\curveto(448.61940828,45.03438385)(448.80440809,45.05438383)(448.94441284,45.09438477)
\curveto(449.08440781,45.13438375)(449.20940769,45.17438371)(449.31941284,45.21438477)
\curveto(449.43940746,45.26438362)(449.54940735,45.32438356)(449.64941284,45.39438477)
\curveto(449.75940714,45.46438342)(449.85440704,45.54438334)(449.93441284,45.63438477)
\curveto(450.01440688,45.73438315)(450.08440681,45.83938305)(450.14441284,45.94938477)
\curveto(450.20440669,46.04938284)(450.25440664,46.15438273)(450.29441284,46.26438477)
\curveto(450.34440655,46.37438251)(450.42440647,46.45438243)(450.53441284,46.50438477)
\curveto(450.57440632,46.52438236)(450.63940626,46.53938235)(450.72941284,46.54938477)
\curveto(450.81940608,46.55938233)(450.90940599,46.55938233)(450.99941284,46.54938477)
\curveto(451.08940581,46.54938234)(451.17440572,46.54438234)(451.25441284,46.53438477)
\curveto(451.33440556,46.52438236)(451.38940551,46.50438238)(451.41941284,46.47438477)
\curveto(451.51940538,46.40438248)(451.54440535,46.2893826)(451.49441284,46.12938477)
\curveto(451.41440548,45.85938303)(451.30940559,45.61938327)(451.17941284,45.40938477)
\curveto(450.97940592,45.0893838)(450.74940615,44.82438406)(450.48941284,44.61438477)
\curveto(450.23940666,44.41438447)(449.91940698,44.24938464)(449.52941284,44.11938477)
\curveto(449.42940747,44.07938481)(449.32940757,44.05438483)(449.22941284,44.04438477)
\curveto(449.12940777,44.02438486)(449.02440787,44.00438488)(448.91441284,43.98438477)
\curveto(448.86440803,43.97438491)(448.81440808,43.96938492)(448.76441284,43.96938477)
\curveto(448.72440817,43.96938492)(448.67940822,43.96438492)(448.62941284,43.95438477)
\lineto(448.47941284,43.95438477)
\curveto(448.42940847,43.94438494)(448.36940853,43.93938495)(448.29941284,43.93938477)
\curveto(448.23940866,43.93938495)(448.18940871,43.94438494)(448.14941284,43.95438477)
\lineto(448.01441284,43.95438477)
\curveto(447.96440893,43.96438492)(447.91940898,43.96938492)(447.87941284,43.96938477)
\curveto(447.83940906,43.96938492)(447.7994091,43.97438491)(447.75941284,43.98438477)
\curveto(447.70940919,43.99438489)(447.65440924,44.00438488)(447.59441284,44.01438477)
\curveto(447.53440936,44.01438487)(447.47940942,44.01938487)(447.42941284,44.02938477)
\curveto(447.33940956,44.04938484)(447.24940965,44.07438481)(447.15941284,44.10438477)
\curveto(447.06940983,44.12438476)(446.98440991,44.14938474)(446.90441284,44.17938477)
\curveto(446.86441003,44.19938469)(446.82941007,44.20938468)(446.79941284,44.20938477)
\curveto(446.76941013,44.21938467)(446.73441016,44.23438465)(446.69441284,44.25438477)
\curveto(446.54441035,44.32438456)(446.38441051,44.40938448)(446.21441284,44.50938477)
\curveto(445.92441097,44.69938419)(445.67441122,44.92938396)(445.46441284,45.19938477)
\curveto(445.26441163,45.47938341)(445.0944118,45.7893831)(444.95441284,46.12938477)
\curveto(444.90441199,46.23938265)(444.86441203,46.35438253)(444.83441284,46.47438477)
\curveto(444.81441208,46.59438229)(444.78441211,46.71438217)(444.74441284,46.83438477)
\curveto(444.73441216,46.87438201)(444.72941217,46.90938198)(444.72941284,46.93938477)
\curveto(444.72941217,46.96938192)(444.72441217,47.00938188)(444.71441284,47.05938477)
\curveto(444.6944122,47.13938175)(444.67941222,47.22438166)(444.66941284,47.31438477)
\curveto(444.65941224,47.40438148)(444.64441225,47.49438139)(444.62441284,47.58438477)
\lineto(444.62441284,47.79438477)
\curveto(444.61441228,47.83438105)(444.60441229,47.889381)(444.59441284,47.95938477)
\curveto(444.5944123,48.03938085)(444.5994123,48.10438078)(444.60941284,48.15438477)
\lineto(444.60941284,48.31938477)
\curveto(444.62941227,48.36938052)(444.63441226,48.41938047)(444.62441284,48.46938477)
\curveto(444.62441227,48.52938036)(444.62941227,48.5843803)(444.63941284,48.63438477)
\curveto(444.67941222,48.79438009)(444.70941219,48.95437993)(444.72941284,49.11438477)
\curveto(444.75941214,49.27437961)(444.80441209,49.42437946)(444.86441284,49.56438477)
\curveto(444.91441198,49.67437921)(444.95941194,49.7843791)(444.99941284,49.89438477)
\curveto(445.04941185,50.01437887)(445.10441179,50.12937876)(445.16441284,50.23938477)
\curveto(445.38441151,50.5893783)(445.63441126,50.889378)(445.91441284,51.13938477)
\curveto(446.1944107,51.39937749)(446.53941036,51.61437727)(446.94941284,51.78438477)
\curveto(447.06940983,51.83437705)(447.18940971,51.86937702)(447.30941284,51.88938477)
\curveto(447.43940946,51.91937697)(447.57440932,51.94937694)(447.71441284,51.97938477)
\curveto(447.76440913,51.9893769)(447.80940909,51.99437689)(447.84941284,51.99438477)
\curveto(447.88940901,52.00437688)(447.93440896,52.00937688)(447.98441284,52.00938477)
\curveto(448.00440889,52.01937687)(448.02940887,52.01937687)(448.05941284,52.00938477)
\curveto(448.08940881,51.99937689)(448.11440878,52.00437688)(448.13441284,52.02438477)
\curveto(448.55440834,52.03437685)(448.91940798,51.9893769)(449.22941284,51.88938477)
\curveto(449.53940736,51.79937709)(449.81940708,51.67437721)(450.06941284,51.51438477)
\curveto(450.11940678,51.49437739)(450.15940674,51.46437742)(450.18941284,51.42438477)
\curveto(450.21940668,51.39437749)(450.25440664,51.36937752)(450.29441284,51.34938477)
\curveto(450.37440652,51.2893776)(450.45440644,51.21937767)(450.53441284,51.13938477)
\curveto(450.62440627,51.05937783)(450.6994062,50.97937791)(450.75941284,50.89938477)
\curveto(450.91940598,50.6893782)(451.05440584,50.4893784)(451.16441284,50.29938477)
\curveto(451.23440566,50.1893787)(451.28940561,50.06937882)(451.32941284,49.93938477)
\curveto(451.36940553,49.80937908)(451.41440548,49.67937921)(451.46441284,49.54938477)
\curveto(451.51440538,49.41937947)(451.54940535,49.2843796)(451.56941284,49.14438477)
\curveto(451.5994053,49.00437988)(451.63440526,48.86438002)(451.67441284,48.72438477)
\curveto(451.68440521,48.65438023)(451.68940521,48.5843803)(451.68941284,48.51438477)
\lineto(451.71941284,48.30438477)
\moveto(450.26441284,48.81438477)
\curveto(450.2944066,48.85438003)(450.31940658,48.90437998)(450.33941284,48.96438477)
\curveto(450.35940654,49.03437985)(450.35940654,49.10437978)(450.33941284,49.17438477)
\curveto(450.27940662,49.39437949)(450.1944067,49.59937929)(450.08441284,49.78938477)
\curveto(449.94440695,50.01937887)(449.78940711,50.21437867)(449.61941284,50.37438477)
\curveto(449.44940745,50.53437835)(449.22940767,50.66937822)(448.95941284,50.77938477)
\curveto(448.88940801,50.79937809)(448.81940808,50.81437807)(448.74941284,50.82438477)
\curveto(448.67940822,50.84437804)(448.60440829,50.86437802)(448.52441284,50.88438477)
\curveto(448.44440845,50.90437798)(448.35940854,50.91437797)(448.26941284,50.91438477)
\lineto(448.01441284,50.91438477)
\curveto(447.98440891,50.89437799)(447.94940895,50.884378)(447.90941284,50.88438477)
\curveto(447.86940903,50.89437799)(447.83440906,50.89437799)(447.80441284,50.88438477)
\lineto(447.56441284,50.82438477)
\curveto(447.4944094,50.81437807)(447.42440947,50.79937809)(447.35441284,50.77938477)
\curveto(447.06440983,50.65937823)(446.82941007,50.50937838)(446.64941284,50.32938477)
\curveto(446.47941042,50.14937874)(446.32441057,49.92437896)(446.18441284,49.65438477)
\curveto(446.15441074,49.60437928)(446.12441077,49.53937935)(446.09441284,49.45938477)
\curveto(446.06441083,49.3893795)(446.03941086,49.30937958)(446.01941284,49.21938477)
\curveto(445.9994109,49.12937976)(445.9944109,49.04437984)(446.00441284,48.96438477)
\curveto(446.01441088,48.88438)(446.04941085,48.82438006)(446.10941284,48.78438477)
\curveto(446.18941071,48.72438016)(446.32441057,48.69438019)(446.51441284,48.69438477)
\curveto(446.71441018,48.70438018)(446.88441001,48.70938018)(447.02441284,48.70938477)
\lineto(449.30441284,48.70938477)
\curveto(449.45440744,48.70938018)(449.63440726,48.70438018)(449.84441284,48.69438477)
\curveto(450.05440684,48.69438019)(450.1944067,48.73438015)(450.26441284,48.81438477)
}
}
{
\newrgbcolor{curcolor}{0 0 0}
\pscustom[linestyle=none,fillstyle=solid,fillcolor=curcolor]
{
\newpath
\moveto(572.93253418,49.74439941)
\lineto(572.93253418,49.47439941)
\curveto(572.94252421,49.38439416)(572.93752421,49.30439424)(572.91753418,49.23439941)
\lineto(572.91753418,49.08439941)
\curveto(572.90752424,49.05439449)(572.90252425,49.01939453)(572.90253418,48.97939941)
\curveto(572.91252424,48.93939461)(572.91252424,48.90939464)(572.90253418,48.88939941)
\curveto(572.89252426,48.83939471)(572.88752426,48.78439476)(572.88753418,48.72439941)
\curveto(572.88752426,48.67439487)(572.88252427,48.62439492)(572.87253418,48.57439941)
\curveto(572.84252431,48.43439511)(572.82252433,48.28439526)(572.81253418,48.12439941)
\curveto(572.80252435,47.97439557)(572.77252438,47.82939572)(572.72253418,47.68939941)
\curveto(572.69252446,47.56939598)(572.65752449,47.4443961)(572.61753418,47.31439941)
\curveto(572.58752456,47.19439635)(572.5475246,47.07439647)(572.49753418,46.95439941)
\curveto(572.32752482,46.52439702)(572.11252504,46.13439741)(571.85253418,45.78439941)
\curveto(571.60252555,45.4443981)(571.28752586,45.15439839)(570.90753418,44.91439941)
\curveto(570.71752643,44.79439875)(570.51252664,44.68939886)(570.29253418,44.59939941)
\curveto(570.08252707,44.51939903)(569.8525273,44.43939911)(569.60253418,44.35939941)
\curveto(569.49252766,44.31939923)(569.37252778,44.28939926)(569.24253418,44.26939941)
\curveto(569.12252803,44.25939929)(569.00252815,44.23939931)(568.88253418,44.20939941)
\curveto(568.77252838,44.18939936)(568.66252849,44.17939937)(568.55253418,44.17939941)
\curveto(568.4525287,44.17939937)(568.3525288,44.16939938)(568.25253418,44.14939941)
\lineto(568.04253418,44.14939941)
\curveto(568.01252914,44.13939941)(567.97752917,44.13439941)(567.93753418,44.13439941)
\curveto(567.89752925,44.1443994)(567.85752929,44.1493994)(567.81753418,44.14939941)
\lineto(564.81753418,44.14939941)
\curveto(564.66753248,44.1493994)(564.53253262,44.15439939)(564.41253418,44.16439941)
\curveto(564.30253285,44.18439936)(564.22753292,44.2493993)(564.18753418,44.35939941)
\curveto(564.147533,44.43939911)(564.12753302,44.55439899)(564.12753418,44.70439941)
\curveto(564.13753301,44.85439869)(564.14253301,44.98939856)(564.14253418,45.10939941)
\lineto(564.14253418,53.97439941)
\curveto(564.14253301,54.09438945)(564.13753301,54.21938933)(564.12753418,54.34939941)
\curveto(564.12753302,54.48938906)(564.152533,54.59938895)(564.20253418,54.67939941)
\curveto(564.24253291,54.7493888)(564.31753283,54.79438875)(564.42753418,54.81439941)
\curveto(564.4475327,54.82438872)(564.46753268,54.82438872)(564.48753418,54.81439941)
\curveto(564.50753264,54.81438873)(564.52753262,54.81938873)(564.54753418,54.82939941)
\lineto(567.80253418,54.82939941)
\curveto(567.8525293,54.82938872)(567.89752925,54.82938872)(567.93753418,54.82939941)
\curveto(567.98752916,54.83938871)(568.03252912,54.83938871)(568.07253418,54.82939941)
\curveto(568.12252903,54.80938874)(568.17252898,54.80438874)(568.22253418,54.81439941)
\curveto(568.28252887,54.82438872)(568.33752881,54.82438872)(568.38753418,54.81439941)
\curveto(568.43752871,54.80438874)(568.49252866,54.79938875)(568.55253418,54.79939941)
\curveto(568.61252854,54.79938875)(568.66752848,54.79438875)(568.71753418,54.78439941)
\curveto(568.76752838,54.77438877)(568.81252834,54.76938878)(568.85253418,54.76939941)
\curveto(568.90252825,54.76938878)(568.9525282,54.76438878)(569.00253418,54.75439941)
\curveto(569.11252804,54.73438881)(569.21752793,54.71438883)(569.31753418,54.69439941)
\curveto(569.41752773,54.68438886)(569.51752763,54.66438888)(569.61753418,54.63439941)
\curveto(569.83752731,54.56438898)(570.0475271,54.49438905)(570.24753418,54.42439941)
\curveto(570.4475267,54.36438918)(570.63252652,54.27938927)(570.80253418,54.16939941)
\curveto(570.94252621,54.08938946)(571.06752608,54.00938954)(571.17753418,53.92939941)
\curveto(571.20752594,53.90938964)(571.23752591,53.88438966)(571.26753418,53.85439941)
\curveto(571.29752585,53.83438971)(571.32752582,53.81438973)(571.35753418,53.79439941)
\curveto(571.41752573,53.7443898)(571.47252568,53.69438985)(571.52253418,53.64439941)
\curveto(571.57252558,53.59438995)(571.62252553,53.54439)(571.67253418,53.49439941)
\curveto(571.72252543,53.4443901)(571.76252539,53.40939014)(571.79253418,53.38939941)
\curveto(571.83252532,53.32939022)(571.87252528,53.27439027)(571.91253418,53.22439941)
\curveto(571.96252519,53.17439037)(572.00752514,53.11939043)(572.04753418,53.05939941)
\curveto(572.09752505,52.99939055)(572.13752501,52.93439061)(572.16753418,52.86439941)
\curveto(572.20752494,52.80439074)(572.2525249,52.73939081)(572.30253418,52.66939941)
\curveto(572.32252483,52.62939092)(572.33752481,52.59439095)(572.34753418,52.56439941)
\curveto(572.35752479,52.53439101)(572.37252478,52.49939105)(572.39253418,52.45939941)
\curveto(572.43252472,52.37939117)(572.46752468,52.29939125)(572.49753418,52.21939941)
\curveto(572.52752462,52.1493914)(572.56252459,52.07439147)(572.60253418,51.99439941)
\curveto(572.64252451,51.88439166)(572.67252448,51.76939178)(572.69253418,51.64939941)
\curveto(572.72252443,51.53939201)(572.7525244,51.42939212)(572.78253418,51.31939941)
\curveto(572.80252435,51.25939229)(572.81252434,51.19939235)(572.81253418,51.13939941)
\curveto(572.81252434,51.08939246)(572.82252433,51.03439251)(572.84253418,50.97439941)
\curveto(572.89252426,50.79439275)(572.91752423,50.59439295)(572.91753418,50.37439941)
\curveto(572.92752422,50.16439338)(572.93252422,49.95439359)(572.93253418,49.74439941)
\moveto(571.50753418,48.96439941)
\curveto(571.52752562,49.06439448)(571.53752561,49.16939438)(571.53753418,49.27939941)
\lineto(571.53753418,49.62439941)
\lineto(571.53753418,49.84939941)
\curveto(571.5475256,49.92939362)(571.54252561,50.00439354)(571.52253418,50.07439941)
\curveto(571.52252563,50.10439344)(571.51752563,50.13439341)(571.50753418,50.16439941)
\lineto(571.50753418,50.26939941)
\curveto(571.48752566,50.37939317)(571.47252568,50.48939306)(571.46253418,50.59939941)
\curveto(571.46252569,50.70939284)(571.4475257,50.81939273)(571.41753418,50.92939941)
\curveto(571.39752575,51.00939254)(571.37752577,51.08439246)(571.35753418,51.15439941)
\curveto(571.3475258,51.23439231)(571.33252582,51.31439223)(571.31253418,51.39439941)
\curveto(571.20252595,51.75439179)(571.06252609,52.06939148)(570.89253418,52.33939941)
\curveto(570.61252654,52.78939076)(570.19752695,53.12939042)(569.64753418,53.35939941)
\curveto(569.55752759,53.40939014)(569.46252769,53.4443901)(569.36253418,53.46439941)
\curveto(569.26252789,53.49439005)(569.15752799,53.52439002)(569.04753418,53.55439941)
\curveto(568.93752821,53.58438996)(568.82252833,53.59938995)(568.70253418,53.59939941)
\curveto(568.59252856,53.60938994)(568.48252867,53.62438992)(568.37253418,53.64439941)
\lineto(568.05753418,53.64439941)
\curveto(568.02752912,53.65438989)(567.99252916,53.65938989)(567.95253418,53.65939941)
\lineto(567.83253418,53.65939941)
\lineto(566.00253418,53.65939941)
\curveto(565.98253117,53.6493899)(565.95753119,53.6443899)(565.92753418,53.64439941)
\curveto(565.89753125,53.65438989)(565.87253128,53.65438989)(565.85253418,53.64439941)
\lineto(565.70253418,53.58439941)
\curveto(565.66253149,53.56438998)(565.63253152,53.53439001)(565.61253418,53.49439941)
\curveto(565.59253156,53.45439009)(565.57253158,53.38439016)(565.55253418,53.28439941)
\lineto(565.55253418,53.16439941)
\curveto(565.54253161,53.12439042)(565.53753161,53.07939047)(565.53753418,53.02939941)
\lineto(565.53753418,52.89439941)
\lineto(565.53753418,46.08439941)
\lineto(565.53753418,45.93439941)
\curveto(565.53753161,45.89439765)(565.54253161,45.85439769)(565.55253418,45.81439941)
\lineto(565.55253418,45.69439941)
\curveto(565.57253158,45.59439795)(565.59253156,45.52439802)(565.61253418,45.48439941)
\curveto(565.69253146,45.36439818)(565.84253131,45.30439824)(566.06253418,45.30439941)
\curveto(566.28253087,45.31439823)(566.49253066,45.31939823)(566.69253418,45.31939941)
\lineto(567.56253418,45.31939941)
\curveto(567.63252952,45.31939823)(567.70752944,45.31439823)(567.78753418,45.30439941)
\curveto(567.86752928,45.30439824)(567.93752921,45.31439823)(567.99753418,45.33439941)
\lineto(568.16253418,45.33439941)
\curveto(568.21252894,45.3443982)(568.26752888,45.3443982)(568.32753418,45.33439941)
\curveto(568.38752876,45.33439821)(568.4475287,45.33939821)(568.50753418,45.34939941)
\curveto(568.56752858,45.36939818)(568.62752852,45.37939817)(568.68753418,45.37939941)
\curveto(568.7475284,45.38939816)(568.81252834,45.40439814)(568.88253418,45.42439941)
\curveto(568.99252816,45.45439809)(569.09752805,45.48439806)(569.19753418,45.51439941)
\curveto(569.30752784,45.544398)(569.41752773,45.58439796)(569.52753418,45.63439941)
\curveto(569.89752725,45.79439775)(570.21252694,46.00939754)(570.47253418,46.27939941)
\curveto(570.74252641,46.55939699)(570.96252619,46.88939666)(571.13253418,47.26939941)
\curveto(571.18252597,47.37939617)(571.22252593,47.49439605)(571.25253418,47.61439941)
\lineto(571.37253418,48.00439941)
\curveto(571.40252575,48.11439543)(571.42252573,48.22939532)(571.43253418,48.34939941)
\curveto(571.4525257,48.47939507)(571.47252568,48.60439494)(571.49253418,48.72439941)
\curveto(571.50252565,48.77439477)(571.50752564,48.81439473)(571.50753418,48.84439941)
\lineto(571.50753418,48.96439941)
}
}
{
\newrgbcolor{curcolor}{0 0 0}
\pscustom[linestyle=none,fillstyle=solid,fillcolor=curcolor]
{
\newpath
\moveto(581.18440918,48.31939941)
\curveto(581.20440149,48.21939533)(581.20440149,48.10439544)(581.18440918,47.97439941)
\curveto(581.17440152,47.85439569)(581.14440155,47.76939578)(581.09440918,47.71939941)
\curveto(581.04440165,47.67939587)(580.96940173,47.6493959)(580.86940918,47.62939941)
\curveto(580.77940192,47.61939593)(580.67440202,47.61439593)(580.55440918,47.61439941)
\lineto(580.19440918,47.61439941)
\curveto(580.07440262,47.62439592)(579.96940273,47.62939592)(579.87940918,47.62939941)
\lineto(576.03940918,47.62939941)
\curveto(575.95940674,47.62939592)(575.87940682,47.62439592)(575.79940918,47.61439941)
\curveto(575.71940698,47.61439593)(575.65440704,47.59939595)(575.60440918,47.56939941)
\curveto(575.56440713,47.549396)(575.52440717,47.50939604)(575.48440918,47.44939941)
\curveto(575.46440723,47.41939613)(575.44440725,47.37439617)(575.42440918,47.31439941)
\curveto(575.40440729,47.26439628)(575.40440729,47.21439633)(575.42440918,47.16439941)
\curveto(575.43440726,47.11439643)(575.43940726,47.06939648)(575.43940918,47.02939941)
\curveto(575.43940726,46.98939656)(575.44440725,46.9493966)(575.45440918,46.90939941)
\curveto(575.47440722,46.82939672)(575.4944072,46.7443968)(575.51440918,46.65439941)
\curveto(575.53440716,46.57439697)(575.56440713,46.49439705)(575.60440918,46.41439941)
\curveto(575.83440686,45.87439767)(576.21440648,45.48939806)(576.74440918,45.25939941)
\curveto(576.80440589,45.22939832)(576.86940583,45.20439834)(576.93940918,45.18439941)
\lineto(577.14940918,45.12439941)
\curveto(577.17940552,45.11439843)(577.22940547,45.10939844)(577.29940918,45.10939941)
\curveto(577.43940526,45.06939848)(577.62440507,45.0493985)(577.85440918,45.04939941)
\curveto(578.08440461,45.0493985)(578.26940443,45.06939848)(578.40940918,45.10939941)
\curveto(578.54940415,45.1493984)(578.67440402,45.18939836)(578.78440918,45.22939941)
\curveto(578.90440379,45.27939827)(579.01440368,45.33939821)(579.11440918,45.40939941)
\curveto(579.22440347,45.47939807)(579.31940338,45.55939799)(579.39940918,45.64939941)
\curveto(579.47940322,45.7493978)(579.54940315,45.85439769)(579.60940918,45.96439941)
\curveto(579.66940303,46.06439748)(579.71940298,46.16939738)(579.75940918,46.27939941)
\curveto(579.80940289,46.38939716)(579.88940281,46.46939708)(579.99940918,46.51939941)
\curveto(580.03940266,46.53939701)(580.10440259,46.55439699)(580.19440918,46.56439941)
\curveto(580.28440241,46.57439697)(580.37440232,46.57439697)(580.46440918,46.56439941)
\curveto(580.55440214,46.56439698)(580.63940206,46.55939699)(580.71940918,46.54939941)
\curveto(580.7994019,46.53939701)(580.85440184,46.51939703)(580.88440918,46.48939941)
\curveto(580.98440171,46.41939713)(581.00940169,46.30439724)(580.95940918,46.14439941)
\curveto(580.87940182,45.87439767)(580.77440192,45.63439791)(580.64440918,45.42439941)
\curveto(580.44440225,45.10439844)(580.21440248,44.83939871)(579.95440918,44.62939941)
\curveto(579.70440299,44.42939912)(579.38440331,44.26439928)(578.99440918,44.13439941)
\curveto(578.8944038,44.09439945)(578.7944039,44.06939948)(578.69440918,44.05939941)
\curveto(578.5944041,44.03939951)(578.48940421,44.01939953)(578.37940918,43.99939941)
\curveto(578.32940437,43.98939956)(578.27940442,43.98439956)(578.22940918,43.98439941)
\curveto(578.18940451,43.98439956)(578.14440455,43.97939957)(578.09440918,43.96939941)
\lineto(577.94440918,43.96939941)
\curveto(577.8944048,43.95939959)(577.83440486,43.95439959)(577.76440918,43.95439941)
\curveto(577.70440499,43.95439959)(577.65440504,43.95939959)(577.61440918,43.96939941)
\lineto(577.47940918,43.96939941)
\curveto(577.42940527,43.97939957)(577.38440531,43.98439956)(577.34440918,43.98439941)
\curveto(577.30440539,43.98439956)(577.26440543,43.98939956)(577.22440918,43.99939941)
\curveto(577.17440552,44.00939954)(577.11940558,44.01939953)(577.05940918,44.02939941)
\curveto(576.9994057,44.02939952)(576.94440575,44.03439951)(576.89440918,44.04439941)
\curveto(576.80440589,44.06439948)(576.71440598,44.08939946)(576.62440918,44.11939941)
\curveto(576.53440616,44.13939941)(576.44940625,44.16439938)(576.36940918,44.19439941)
\curveto(576.32940637,44.21439933)(576.2944064,44.22439932)(576.26440918,44.22439941)
\curveto(576.23440646,44.23439931)(576.1994065,44.2493993)(576.15940918,44.26939941)
\curveto(576.00940669,44.33939921)(575.84940685,44.42439912)(575.67940918,44.52439941)
\curveto(575.38940731,44.71439883)(575.13940756,44.9443986)(574.92940918,45.21439941)
\curveto(574.72940797,45.49439805)(574.55940814,45.80439774)(574.41940918,46.14439941)
\curveto(574.36940833,46.25439729)(574.32940837,46.36939718)(574.29940918,46.48939941)
\curveto(574.27940842,46.60939694)(574.24940845,46.72939682)(574.20940918,46.84939941)
\curveto(574.1994085,46.88939666)(574.1944085,46.92439662)(574.19440918,46.95439941)
\curveto(574.1944085,46.98439656)(574.18940851,47.02439652)(574.17940918,47.07439941)
\curveto(574.15940854,47.15439639)(574.14440855,47.23939631)(574.13440918,47.32939941)
\curveto(574.12440857,47.41939613)(574.10940859,47.50939604)(574.08940918,47.59939941)
\lineto(574.08940918,47.80939941)
\curveto(574.07940862,47.8493957)(574.06940863,47.90439564)(574.05940918,47.97439941)
\curveto(574.05940864,48.05439549)(574.06440863,48.11939543)(574.07440918,48.16939941)
\lineto(574.07440918,48.33439941)
\curveto(574.0944086,48.38439516)(574.0994086,48.43439511)(574.08940918,48.48439941)
\curveto(574.08940861,48.544395)(574.0944086,48.59939495)(574.10440918,48.64939941)
\curveto(574.14440855,48.80939474)(574.17440852,48.96939458)(574.19440918,49.12939941)
\curveto(574.22440847,49.28939426)(574.26940843,49.43939411)(574.32940918,49.57939941)
\curveto(574.37940832,49.68939386)(574.42440827,49.79939375)(574.46440918,49.90939941)
\curveto(574.51440818,50.02939352)(574.56940813,50.1443934)(574.62940918,50.25439941)
\curveto(574.84940785,50.60439294)(575.0994076,50.90439264)(575.37940918,51.15439941)
\curveto(575.65940704,51.41439213)(576.00440669,51.62939192)(576.41440918,51.79939941)
\curveto(576.53440616,51.8493917)(576.65440604,51.88439166)(576.77440918,51.90439941)
\curveto(576.90440579,51.93439161)(577.03940566,51.96439158)(577.17940918,51.99439941)
\curveto(577.22940547,52.00439154)(577.27440542,52.00939154)(577.31440918,52.00939941)
\curveto(577.35440534,52.01939153)(577.3994053,52.02439152)(577.44940918,52.02439941)
\curveto(577.46940523,52.03439151)(577.4944052,52.03439151)(577.52440918,52.02439941)
\curveto(577.55440514,52.01439153)(577.57940512,52.01939153)(577.59940918,52.03939941)
\curveto(578.01940468,52.0493915)(578.38440431,52.00439154)(578.69440918,51.90439941)
\curveto(579.00440369,51.81439173)(579.28440341,51.68939186)(579.53440918,51.52939941)
\curveto(579.58440311,51.50939204)(579.62440307,51.47939207)(579.65440918,51.43939941)
\curveto(579.68440301,51.40939214)(579.71940298,51.38439216)(579.75940918,51.36439941)
\curveto(579.83940286,51.30439224)(579.91940278,51.23439231)(579.99940918,51.15439941)
\curveto(580.08940261,51.07439247)(580.16440253,50.99439255)(580.22440918,50.91439941)
\curveto(580.38440231,50.70439284)(580.51940218,50.50439304)(580.62940918,50.31439941)
\curveto(580.699402,50.20439334)(580.75440194,50.08439346)(580.79440918,49.95439941)
\curveto(580.83440186,49.82439372)(580.87940182,49.69439385)(580.92940918,49.56439941)
\curveto(580.97940172,49.43439411)(581.01440168,49.29939425)(581.03440918,49.15939941)
\curveto(581.06440163,49.01939453)(581.0994016,48.87939467)(581.13940918,48.73939941)
\curveto(581.14940155,48.66939488)(581.15440154,48.59939495)(581.15440918,48.52939941)
\lineto(581.18440918,48.31939941)
\moveto(579.72940918,48.82939941)
\curveto(579.75940294,48.86939468)(579.78440291,48.91939463)(579.80440918,48.97939941)
\curveto(579.82440287,49.0493945)(579.82440287,49.11939443)(579.80440918,49.18939941)
\curveto(579.74440295,49.40939414)(579.65940304,49.61439393)(579.54940918,49.80439941)
\curveto(579.40940329,50.03439351)(579.25440344,50.22939332)(579.08440918,50.38939941)
\curveto(578.91440378,50.549393)(578.694404,50.68439286)(578.42440918,50.79439941)
\curveto(578.35440434,50.81439273)(578.28440441,50.82939272)(578.21440918,50.83939941)
\curveto(578.14440455,50.85939269)(578.06940463,50.87939267)(577.98940918,50.89939941)
\curveto(577.90940479,50.91939263)(577.82440487,50.92939262)(577.73440918,50.92939941)
\lineto(577.47940918,50.92939941)
\curveto(577.44940525,50.90939264)(577.41440528,50.89939265)(577.37440918,50.89939941)
\curveto(577.33440536,50.90939264)(577.2994054,50.90939264)(577.26940918,50.89939941)
\lineto(577.02940918,50.83939941)
\curveto(576.95940574,50.82939272)(576.88940581,50.81439273)(576.81940918,50.79439941)
\curveto(576.52940617,50.67439287)(576.2944064,50.52439302)(576.11440918,50.34439941)
\curveto(575.94440675,50.16439338)(575.78940691,49.93939361)(575.64940918,49.66939941)
\curveto(575.61940708,49.61939393)(575.58940711,49.55439399)(575.55940918,49.47439941)
\curveto(575.52940717,49.40439414)(575.50440719,49.32439422)(575.48440918,49.23439941)
\curveto(575.46440723,49.1443944)(575.45940724,49.05939449)(575.46940918,48.97939941)
\curveto(575.47940722,48.89939465)(575.51440718,48.83939471)(575.57440918,48.79939941)
\curveto(575.65440704,48.73939481)(575.78940691,48.70939484)(575.97940918,48.70939941)
\curveto(576.17940652,48.71939483)(576.34940635,48.72439482)(576.48940918,48.72439941)
\lineto(578.76940918,48.72439941)
\curveto(578.91940378,48.72439482)(579.0994036,48.71939483)(579.30940918,48.70939941)
\curveto(579.51940318,48.70939484)(579.65940304,48.7493948)(579.72940918,48.82939941)
}
}
{
\newrgbcolor{curcolor}{0 0 0}
\pscustom[linestyle=none,fillstyle=solid,fillcolor=curcolor]
{
\newpath
\moveto(584.9210498,52.05439941)
\curveto(585.64104574,52.06439148)(586.24604513,51.97939157)(586.7360498,51.79939941)
\curveto(587.22604415,51.62939192)(587.60604377,51.32439222)(587.8760498,50.88439941)
\curveto(587.94604343,50.77439277)(588.00104338,50.65939289)(588.0410498,50.53939941)
\curveto(588.0810433,50.42939312)(588.12104326,50.30439324)(588.1610498,50.16439941)
\curveto(588.1810432,50.09439345)(588.18604319,50.01939353)(588.1760498,49.93939941)
\curveto(588.16604321,49.86939368)(588.15104323,49.81439373)(588.1310498,49.77439941)
\curveto(588.11104327,49.75439379)(588.08604329,49.73439381)(588.0560498,49.71439941)
\curveto(588.02604335,49.70439384)(588.00104338,49.68939386)(587.9810498,49.66939941)
\curveto(587.93104345,49.6493939)(587.8810435,49.6443939)(587.8310498,49.65439941)
\curveto(587.7810436,49.66439388)(587.73104365,49.66439388)(587.6810498,49.65439941)
\curveto(587.60104378,49.63439391)(587.49604388,49.62939392)(587.3660498,49.63939941)
\curveto(587.23604414,49.65939389)(587.14604423,49.68439386)(587.0960498,49.71439941)
\curveto(587.01604436,49.76439378)(586.96104442,49.82939372)(586.9310498,49.90939941)
\curveto(586.91104447,49.99939355)(586.8760445,50.08439346)(586.8260498,50.16439941)
\curveto(586.73604464,50.32439322)(586.61104477,50.46939308)(586.4510498,50.59939941)
\curveto(586.34104504,50.67939287)(586.22104516,50.73939281)(586.0910498,50.77939941)
\curveto(585.96104542,50.81939273)(585.82104556,50.85939269)(585.6710498,50.89939941)
\curveto(585.62104576,50.91939263)(585.57104581,50.92439262)(585.5210498,50.91439941)
\curveto(585.47104591,50.91439263)(585.42104596,50.91939263)(585.3710498,50.92939941)
\curveto(585.31104607,50.9493926)(585.23604614,50.95939259)(585.1460498,50.95939941)
\curveto(585.05604632,50.95939259)(584.9810464,50.9493926)(584.9210498,50.92939941)
\lineto(584.8310498,50.92939941)
\lineto(584.6810498,50.89939941)
\curveto(584.63104675,50.89939265)(584.5810468,50.89439265)(584.5310498,50.88439941)
\curveto(584.27104711,50.82439272)(584.05604732,50.73939281)(583.8860498,50.62939941)
\curveto(583.71604766,50.51939303)(583.60104778,50.33439321)(583.5410498,50.07439941)
\curveto(583.52104786,50.00439354)(583.51604786,49.93439361)(583.5260498,49.86439941)
\curveto(583.54604783,49.79439375)(583.56604781,49.73439381)(583.5860498,49.68439941)
\curveto(583.64604773,49.53439401)(583.71604766,49.42439412)(583.7960498,49.35439941)
\curveto(583.88604749,49.29439425)(583.99604738,49.22439432)(584.1260498,49.14439941)
\curveto(584.28604709,49.0443945)(584.46604691,48.96939458)(584.6660498,48.91939941)
\curveto(584.86604651,48.87939467)(585.06604631,48.82939472)(585.2660498,48.76939941)
\curveto(585.39604598,48.72939482)(585.52604585,48.69939485)(585.6560498,48.67939941)
\curveto(585.78604559,48.65939489)(585.91604546,48.62939492)(586.0460498,48.58939941)
\curveto(586.25604512,48.52939502)(586.46104492,48.46939508)(586.6610498,48.40939941)
\curveto(586.86104452,48.35939519)(587.06104432,48.29439525)(587.2610498,48.21439941)
\lineto(587.4110498,48.15439941)
\curveto(587.46104392,48.13439541)(587.51104387,48.10939544)(587.5610498,48.07939941)
\curveto(587.76104362,47.95939559)(587.93604344,47.82439572)(588.0860498,47.67439941)
\curveto(588.23604314,47.52439602)(588.36104302,47.33439621)(588.4610498,47.10439941)
\curveto(588.4810429,47.03439651)(588.50104288,46.93939661)(588.5210498,46.81939941)
\curveto(588.54104284,46.7493968)(588.55104283,46.67439687)(588.5510498,46.59439941)
\curveto(588.56104282,46.52439702)(588.56604281,46.4443971)(588.5660498,46.35439941)
\lineto(588.5660498,46.20439941)
\curveto(588.54604283,46.13439741)(588.53604284,46.06439748)(588.5360498,45.99439941)
\curveto(588.53604284,45.92439762)(588.52604285,45.85439769)(588.5060498,45.78439941)
\curveto(588.4760429,45.67439787)(588.44104294,45.56939798)(588.4010498,45.46939941)
\curveto(588.36104302,45.36939818)(588.31604306,45.27939827)(588.2660498,45.19939941)
\curveto(588.10604327,44.93939861)(587.90104348,44.72939882)(587.6510498,44.56939941)
\curveto(587.40104398,44.41939913)(587.12104426,44.28939926)(586.8110498,44.17939941)
\curveto(586.72104466,44.1493994)(586.62604475,44.12939942)(586.5260498,44.11939941)
\curveto(586.43604494,44.09939945)(586.34604503,44.07439947)(586.2560498,44.04439941)
\curveto(586.15604522,44.02439952)(586.05604532,44.01439953)(585.9560498,44.01439941)
\curveto(585.85604552,44.01439953)(585.75604562,44.00439954)(585.6560498,43.98439941)
\lineto(585.5060498,43.98439941)
\curveto(585.45604592,43.97439957)(585.38604599,43.96939958)(585.2960498,43.96939941)
\curveto(585.20604617,43.96939958)(585.13604624,43.97439957)(585.0860498,43.98439941)
\lineto(584.9210498,43.98439941)
\curveto(584.86104652,44.00439954)(584.79604658,44.01439953)(584.7260498,44.01439941)
\curveto(584.65604672,44.00439954)(584.59604678,44.00939954)(584.5460498,44.02939941)
\curveto(584.49604688,44.03939951)(584.43104695,44.0443995)(584.3510498,44.04439941)
\lineto(584.1110498,44.10439941)
\curveto(584.04104734,44.11439943)(583.96604741,44.13439941)(583.8860498,44.16439941)
\curveto(583.5760478,44.26439928)(583.30604807,44.38939916)(583.0760498,44.53939941)
\curveto(582.84604853,44.68939886)(582.64604873,44.88439866)(582.4760498,45.12439941)
\curveto(582.38604899,45.25439829)(582.31104907,45.38939816)(582.2510498,45.52939941)
\curveto(582.19104919,45.66939788)(582.13604924,45.82439772)(582.0860498,45.99439941)
\curveto(582.06604931,46.05439749)(582.05604932,46.12439742)(582.0560498,46.20439941)
\curveto(582.06604931,46.29439725)(582.0810493,46.36439718)(582.1010498,46.41439941)
\curveto(582.13104925,46.45439709)(582.1810492,46.49439705)(582.2510498,46.53439941)
\curveto(582.30104908,46.55439699)(582.37104901,46.56439698)(582.4610498,46.56439941)
\curveto(582.55104883,46.57439697)(582.64104874,46.57439697)(582.7310498,46.56439941)
\curveto(582.82104856,46.55439699)(582.90604847,46.53939701)(582.9860498,46.51939941)
\curveto(583.0760483,46.50939704)(583.13604824,46.49439705)(583.1660498,46.47439941)
\curveto(583.23604814,46.42439712)(583.2810481,46.3493972)(583.3010498,46.24939941)
\curveto(583.33104805,46.15939739)(583.36604801,46.07439747)(583.4060498,45.99439941)
\curveto(583.50604787,45.77439777)(583.64104774,45.60439794)(583.8110498,45.48439941)
\curveto(583.93104745,45.39439815)(584.06604731,45.32439822)(584.2160498,45.27439941)
\curveto(584.36604701,45.22439832)(584.52604685,45.17439837)(584.6960498,45.12439941)
\lineto(585.0110498,45.07939941)
\lineto(585.1010498,45.07939941)
\curveto(585.17104621,45.05939849)(585.26104612,45.0493985)(585.3710498,45.04939941)
\curveto(585.49104589,45.0493985)(585.59104579,45.05939849)(585.6710498,45.07939941)
\curveto(585.74104564,45.07939847)(585.79604558,45.08439846)(585.8360498,45.09439941)
\curveto(585.89604548,45.10439844)(585.95604542,45.10939844)(586.0160498,45.10939941)
\curveto(586.0760453,45.11939843)(586.13104525,45.12939842)(586.1810498,45.13939941)
\curveto(586.47104491,45.21939833)(586.70104468,45.32439822)(586.8710498,45.45439941)
\curveto(587.04104434,45.58439796)(587.16104422,45.80439774)(587.2310498,46.11439941)
\curveto(587.25104413,46.16439738)(587.25604412,46.21939733)(587.2460498,46.27939941)
\curveto(587.23604414,46.33939721)(587.22604415,46.38439716)(587.2160498,46.41439941)
\curveto(587.16604421,46.60439694)(587.09604428,46.7443968)(587.0060498,46.83439941)
\curveto(586.91604446,46.93439661)(586.80104458,47.02439652)(586.6610498,47.10439941)
\curveto(586.57104481,47.16439638)(586.47104491,47.21439633)(586.3610498,47.25439941)
\lineto(586.0310498,47.37439941)
\curveto(586.00104538,47.38439616)(585.97104541,47.38939616)(585.9410498,47.38939941)
\curveto(585.92104546,47.38939616)(585.89604548,47.39939615)(585.8660498,47.41939941)
\curveto(585.52604585,47.52939602)(585.17104621,47.60939594)(584.8010498,47.65939941)
\curveto(584.44104694,47.71939583)(584.10104728,47.81439573)(583.7810498,47.94439941)
\curveto(583.6810477,47.98439556)(583.58604779,48.01939553)(583.4960498,48.04939941)
\curveto(583.40604797,48.07939547)(583.32104806,48.11939543)(583.2410498,48.16939941)
\curveto(583.05104833,48.27939527)(582.8760485,48.40439514)(582.7160498,48.54439941)
\curveto(582.55604882,48.68439486)(582.43104895,48.85939469)(582.3410498,49.06939941)
\curveto(582.31104907,49.13939441)(582.28604909,49.20939434)(582.2660498,49.27939941)
\curveto(582.25604912,49.3493942)(582.24104914,49.42439412)(582.2210498,49.50439941)
\curveto(582.19104919,49.62439392)(582.1810492,49.75939379)(582.1910498,49.90939941)
\curveto(582.20104918,50.06939348)(582.21604916,50.20439334)(582.2360498,50.31439941)
\curveto(582.25604912,50.36439318)(582.26604911,50.40439314)(582.2660498,50.43439941)
\curveto(582.2760491,50.47439307)(582.29104909,50.51439303)(582.3110498,50.55439941)
\curveto(582.40104898,50.78439276)(582.52104886,50.98439256)(582.6710498,51.15439941)
\curveto(582.83104855,51.32439222)(583.01104837,51.47439207)(583.2110498,51.60439941)
\curveto(583.36104802,51.69439185)(583.52604785,51.76439178)(583.7060498,51.81439941)
\curveto(583.88604749,51.87439167)(584.0760473,51.92939162)(584.2760498,51.97939941)
\curveto(584.34604703,51.98939156)(584.41104697,51.99939155)(584.4710498,52.00939941)
\curveto(584.54104684,52.01939153)(584.61604676,52.02939152)(584.6960498,52.03939941)
\curveto(584.72604665,52.0493915)(584.76604661,52.0493915)(584.8160498,52.03939941)
\curveto(584.86604651,52.02939152)(584.90104648,52.03439151)(584.9210498,52.05439941)
}
}
{
\newrgbcolor{curcolor}{0 0 0}
\pscustom[linestyle=none,fillstyle=solid,fillcolor=curcolor]
{
\newpath
\moveto(596.8760498,44.70439941)
\curveto(596.90604197,44.544399)(596.89104199,44.40939914)(596.8310498,44.29939941)
\curveto(596.77104211,44.19939935)(596.69104219,44.12439942)(596.5910498,44.07439941)
\curveto(596.54104234,44.05439949)(596.48604239,44.0443995)(596.4260498,44.04439941)
\curveto(596.3760425,44.0443995)(596.32104256,44.03439951)(596.2610498,44.01439941)
\curveto(596.04104284,43.96439958)(595.82104306,43.97939957)(595.6010498,44.05939941)
\curveto(595.39104349,44.12939942)(595.24604363,44.21939933)(595.1660498,44.32939941)
\curveto(595.11604376,44.39939915)(595.07104381,44.47939907)(595.0310498,44.56939941)
\curveto(594.99104389,44.66939888)(594.94104394,44.7493988)(594.8810498,44.80939941)
\curveto(594.86104402,44.82939872)(594.83604404,44.8493987)(594.8060498,44.86939941)
\curveto(594.78604409,44.88939866)(594.75604412,44.89439865)(594.7160498,44.88439941)
\curveto(594.60604427,44.85439869)(594.50104438,44.79939875)(594.4010498,44.71939941)
\curveto(594.31104457,44.63939891)(594.22104466,44.56939898)(594.1310498,44.50939941)
\curveto(594.00104488,44.42939912)(593.86104502,44.35439919)(593.7110498,44.28439941)
\curveto(593.56104532,44.22439932)(593.40104548,44.16939938)(593.2310498,44.11939941)
\curveto(593.13104575,44.08939946)(593.02104586,44.06939948)(592.9010498,44.05939941)
\curveto(592.79104609,44.0493995)(592.6810462,44.03439951)(592.5710498,44.01439941)
\curveto(592.52104636,44.00439954)(592.4760464,43.99939955)(592.4360498,43.99939941)
\lineto(592.3310498,43.99939941)
\curveto(592.22104666,43.97939957)(592.11604676,43.97939957)(592.0160498,43.99939941)
\lineto(591.8810498,43.99939941)
\curveto(591.83104705,44.00939954)(591.7810471,44.01439953)(591.7310498,44.01439941)
\curveto(591.6810472,44.01439953)(591.63604724,44.02439952)(591.5960498,44.04439941)
\curveto(591.55604732,44.05439949)(591.52104736,44.05939949)(591.4910498,44.05939941)
\curveto(591.47104741,44.0493995)(591.44604743,44.0493995)(591.4160498,44.05939941)
\lineto(591.1760498,44.11939941)
\curveto(591.09604778,44.12939942)(591.02104786,44.1493994)(590.9510498,44.17939941)
\curveto(590.65104823,44.30939924)(590.40604847,44.45439909)(590.2160498,44.61439941)
\curveto(590.03604884,44.78439876)(589.88604899,45.01939853)(589.7660498,45.31939941)
\curveto(589.6760492,45.53939801)(589.63104925,45.80439774)(589.6310498,46.11439941)
\lineto(589.6310498,46.42939941)
\curveto(589.64104924,46.47939707)(589.64604923,46.52939702)(589.6460498,46.57939941)
\lineto(589.6760498,46.75939941)
\lineto(589.7960498,47.08939941)
\curveto(589.83604904,47.19939635)(589.88604899,47.29939625)(589.9460498,47.38939941)
\curveto(590.12604875,47.67939587)(590.37104851,47.89439565)(590.6810498,48.03439941)
\curveto(590.99104789,48.17439537)(591.33104755,48.29939525)(591.7010498,48.40939941)
\curveto(591.84104704,48.4493951)(591.98604689,48.47939507)(592.1360498,48.49939941)
\curveto(592.28604659,48.51939503)(592.43604644,48.544395)(592.5860498,48.57439941)
\curveto(592.65604622,48.59439495)(592.72104616,48.60439494)(592.7810498,48.60439941)
\curveto(592.85104603,48.60439494)(592.92604595,48.61439493)(593.0060498,48.63439941)
\curveto(593.0760458,48.65439489)(593.14604573,48.66439488)(593.2160498,48.66439941)
\curveto(593.28604559,48.67439487)(593.36104552,48.68939486)(593.4410498,48.70939941)
\curveto(593.69104519,48.76939478)(593.92604495,48.81939473)(594.1460498,48.85939941)
\curveto(594.36604451,48.90939464)(594.54104434,49.02439452)(594.6710498,49.20439941)
\curveto(594.73104415,49.28439426)(594.7810441,49.38439416)(594.8210498,49.50439941)
\curveto(594.86104402,49.63439391)(594.86104402,49.77439377)(594.8210498,49.92439941)
\curveto(594.76104412,50.16439338)(594.67104421,50.35439319)(594.5510498,50.49439941)
\curveto(594.44104444,50.63439291)(594.2810446,50.7443928)(594.0710498,50.82439941)
\curveto(593.95104493,50.87439267)(593.80604507,50.90939264)(593.6360498,50.92939941)
\curveto(593.4760454,50.9493926)(593.30604557,50.95939259)(593.1260498,50.95939941)
\curveto(592.94604593,50.95939259)(592.77104611,50.9493926)(592.6010498,50.92939941)
\curveto(592.43104645,50.90939264)(592.28604659,50.87939267)(592.1660498,50.83939941)
\curveto(591.99604688,50.77939277)(591.83104705,50.69439285)(591.6710498,50.58439941)
\curveto(591.59104729,50.52439302)(591.51604736,50.4443931)(591.4460498,50.34439941)
\curveto(591.38604749,50.25439329)(591.33104755,50.15439339)(591.2810498,50.04439941)
\curveto(591.25104763,49.96439358)(591.22104766,49.87939367)(591.1910498,49.78939941)
\curveto(591.17104771,49.69939385)(591.12604775,49.62939392)(591.0560498,49.57939941)
\curveto(591.01604786,49.549394)(590.94604793,49.52439402)(590.8460498,49.50439941)
\curveto(590.75604812,49.49439405)(590.66104822,49.48939406)(590.5610498,49.48939941)
\curveto(590.46104842,49.48939406)(590.36104852,49.49439405)(590.2610498,49.50439941)
\curveto(590.17104871,49.52439402)(590.10604877,49.549394)(590.0660498,49.57939941)
\curveto(590.02604885,49.60939394)(589.99604888,49.65939389)(589.9760498,49.72939941)
\curveto(589.95604892,49.79939375)(589.95604892,49.87439367)(589.9760498,49.95439941)
\curveto(590.00604887,50.08439346)(590.03604884,50.20439334)(590.0660498,50.31439941)
\curveto(590.10604877,50.43439311)(590.15104873,50.549393)(590.2010498,50.65939941)
\curveto(590.39104849,51.00939254)(590.63104825,51.27939227)(590.9210498,51.46939941)
\curveto(591.21104767,51.66939188)(591.57104731,51.82939172)(592.0010498,51.94939941)
\curveto(592.10104678,51.96939158)(592.20104668,51.98439156)(592.3010498,51.99439941)
\curveto(592.41104647,52.00439154)(592.52104636,52.01939153)(592.6310498,52.03939941)
\curveto(592.67104621,52.0493915)(592.73604614,52.0493915)(592.8260498,52.03939941)
\curveto(592.91604596,52.03939151)(592.97104591,52.0493915)(592.9910498,52.06939941)
\curveto(593.69104519,52.07939147)(594.30104458,51.99939155)(594.8210498,51.82939941)
\curveto(595.34104354,51.65939189)(595.70604317,51.33439221)(595.9160498,50.85439941)
\curveto(596.00604287,50.65439289)(596.05604282,50.41939313)(596.0660498,50.14939941)
\curveto(596.08604279,49.88939366)(596.09604278,49.61439393)(596.0960498,49.32439941)
\lineto(596.0960498,46.00939941)
\curveto(596.09604278,45.86939768)(596.10104278,45.73439781)(596.1110498,45.60439941)
\curveto(596.12104276,45.47439807)(596.15104273,45.36939818)(596.2010498,45.28939941)
\curveto(596.25104263,45.21939833)(596.31604256,45.16939838)(596.3960498,45.13939941)
\curveto(596.48604239,45.09939845)(596.57104231,45.06939848)(596.6510498,45.04939941)
\curveto(596.73104215,45.03939851)(596.79104209,44.99439855)(596.8310498,44.91439941)
\curveto(596.85104203,44.88439866)(596.86104202,44.85439869)(596.8610498,44.82439941)
\curveto(596.86104202,44.79439875)(596.86604201,44.75439879)(596.8760498,44.70439941)
\moveto(594.7310498,46.36939941)
\curveto(594.79104409,46.50939704)(594.82104406,46.66939688)(594.8210498,46.84939941)
\curveto(594.83104405,47.03939651)(594.83604404,47.23439631)(594.8360498,47.43439941)
\curveto(594.83604404,47.544396)(594.83104405,47.6443959)(594.8210498,47.73439941)
\curveto(594.81104407,47.82439572)(594.77104411,47.89439565)(594.7010498,47.94439941)
\curveto(594.67104421,47.96439558)(594.60104428,47.97439557)(594.4910498,47.97439941)
\curveto(594.47104441,47.95439559)(594.43604444,47.9443956)(594.3860498,47.94439941)
\curveto(594.33604454,47.9443956)(594.29104459,47.93439561)(594.2510498,47.91439941)
\curveto(594.17104471,47.89439565)(594.0810448,47.87439567)(593.9810498,47.85439941)
\lineto(593.6810498,47.79439941)
\curveto(593.65104523,47.79439575)(593.61604526,47.78939576)(593.5760498,47.77939941)
\lineto(593.4710498,47.77939941)
\curveto(593.32104556,47.73939581)(593.15604572,47.71439583)(592.9760498,47.70439941)
\curveto(592.80604607,47.70439584)(592.64604623,47.68439586)(592.4960498,47.64439941)
\curveto(592.41604646,47.62439592)(592.34104654,47.60439594)(592.2710498,47.58439941)
\curveto(592.21104667,47.57439597)(592.14104674,47.55939599)(592.0610498,47.53939941)
\curveto(591.90104698,47.48939606)(591.75104713,47.42439612)(591.6110498,47.34439941)
\curveto(591.47104741,47.27439627)(591.35104753,47.18439636)(591.2510498,47.07439941)
\curveto(591.15104773,46.96439658)(591.0760478,46.82939672)(591.0260498,46.66939941)
\curveto(590.9760479,46.51939703)(590.95604792,46.33439721)(590.9660498,46.11439941)
\curveto(590.96604791,46.01439753)(590.9810479,45.91939763)(591.0110498,45.82939941)
\curveto(591.05104783,45.7493978)(591.09604778,45.67439787)(591.1460498,45.60439941)
\curveto(591.22604765,45.49439805)(591.33104755,45.39939815)(591.4610498,45.31939941)
\curveto(591.59104729,45.2493983)(591.73104715,45.18939836)(591.8810498,45.13939941)
\curveto(591.93104695,45.12939842)(591.9810469,45.12439842)(592.0310498,45.12439941)
\curveto(592.0810468,45.12439842)(592.13104675,45.11939843)(592.1810498,45.10939941)
\curveto(592.25104663,45.08939846)(592.33604654,45.07439847)(592.4360498,45.06439941)
\curveto(592.54604633,45.06439848)(592.63604624,45.07439847)(592.7060498,45.09439941)
\curveto(592.76604611,45.11439843)(592.82604605,45.11939843)(592.8860498,45.10939941)
\curveto(592.94604593,45.10939844)(593.00604587,45.11939843)(593.0660498,45.13939941)
\curveto(593.14604573,45.15939839)(593.22104566,45.17439837)(593.2910498,45.18439941)
\curveto(593.37104551,45.19439835)(593.44604543,45.21439833)(593.5160498,45.24439941)
\curveto(593.80604507,45.36439818)(594.05104483,45.50939804)(594.2510498,45.67939941)
\curveto(594.46104442,45.8493977)(594.62104426,46.07939747)(594.7310498,46.36939941)
}
}
{
\newrgbcolor{curcolor}{0 0 0}
\pscustom[linestyle=none,fillstyle=solid,fillcolor=curcolor]
{
\newpath
\moveto(601.69269043,52.05439941)
\curveto(601.92268564,52.05439149)(602.05268551,51.99439155)(602.08269043,51.87439941)
\curveto(602.11268545,51.76439178)(602.12768543,51.59939195)(602.12769043,51.37939941)
\lineto(602.12769043,51.09439941)
\curveto(602.12768543,51.00439254)(602.10268546,50.92939262)(602.05269043,50.86939941)
\curveto(601.99268557,50.78939276)(601.90768565,50.7443928)(601.79769043,50.73439941)
\curveto(601.68768587,50.73439281)(601.57768598,50.71939283)(601.46769043,50.68939941)
\curveto(601.32768623,50.65939289)(601.19268637,50.62939292)(601.06269043,50.59939941)
\curveto(600.94268662,50.56939298)(600.82768673,50.52939302)(600.71769043,50.47939941)
\curveto(600.42768713,50.3493932)(600.19268737,50.16939338)(600.01269043,49.93939941)
\curveto(599.83268773,49.71939383)(599.67768788,49.46439408)(599.54769043,49.17439941)
\curveto(599.50768805,49.06439448)(599.47768808,48.9493946)(599.45769043,48.82939941)
\curveto(599.43768812,48.71939483)(599.41268815,48.60439494)(599.38269043,48.48439941)
\curveto(599.37268819,48.43439511)(599.36768819,48.38439516)(599.36769043,48.33439941)
\curveto(599.37768818,48.28439526)(599.37768818,48.23439531)(599.36769043,48.18439941)
\curveto(599.33768822,48.06439548)(599.32268824,47.92439562)(599.32269043,47.76439941)
\curveto(599.33268823,47.61439593)(599.33768822,47.46939608)(599.33769043,47.32939941)
\lineto(599.33769043,45.48439941)
\lineto(599.33769043,45.13939941)
\curveto(599.33768822,45.01939853)(599.33268823,44.90439864)(599.32269043,44.79439941)
\curveto(599.31268825,44.68439886)(599.30768825,44.58939896)(599.30769043,44.50939941)
\curveto(599.31768824,44.42939912)(599.29768826,44.35939919)(599.24769043,44.29939941)
\curveto(599.19768836,44.22939932)(599.11768844,44.18939936)(599.00769043,44.17939941)
\curveto(598.90768865,44.16939938)(598.79768876,44.16439938)(598.67769043,44.16439941)
\lineto(598.40769043,44.16439941)
\curveto(598.3576892,44.18439936)(598.30768925,44.19939935)(598.25769043,44.20939941)
\curveto(598.21768934,44.22939932)(598.18768937,44.25439929)(598.16769043,44.28439941)
\curveto(598.11768944,44.35439919)(598.08768947,44.43939911)(598.07769043,44.53939941)
\lineto(598.07769043,44.86939941)
\lineto(598.07769043,46.02439941)
\lineto(598.07769043,50.17939941)
\lineto(598.07769043,51.21439941)
\lineto(598.07769043,51.51439941)
\curveto(598.08768947,51.61439193)(598.11768944,51.69939185)(598.16769043,51.76939941)
\curveto(598.19768936,51.80939174)(598.24768931,51.83939171)(598.31769043,51.85939941)
\curveto(598.39768916,51.87939167)(598.48268908,51.88939166)(598.57269043,51.88939941)
\curveto(598.6626889,51.89939165)(598.75268881,51.89939165)(598.84269043,51.88939941)
\curveto(598.93268863,51.87939167)(599.00268856,51.86439168)(599.05269043,51.84439941)
\curveto(599.13268843,51.81439173)(599.18268838,51.75439179)(599.20269043,51.66439941)
\curveto(599.23268833,51.58439196)(599.24768831,51.49439205)(599.24769043,51.39439941)
\lineto(599.24769043,51.09439941)
\curveto(599.24768831,50.99439255)(599.26768829,50.90439264)(599.30769043,50.82439941)
\curveto(599.31768824,50.80439274)(599.32768823,50.78939276)(599.33769043,50.77939941)
\lineto(599.38269043,50.73439941)
\curveto(599.49268807,50.73439281)(599.58268798,50.77939277)(599.65269043,50.86939941)
\curveto(599.72268784,50.96939258)(599.78268778,51.0493925)(599.83269043,51.10939941)
\lineto(599.92269043,51.19939941)
\curveto(600.01268755,51.30939224)(600.13768742,51.42439212)(600.29769043,51.54439941)
\curveto(600.4576871,51.66439188)(600.60768695,51.75439179)(600.74769043,51.81439941)
\curveto(600.83768672,51.86439168)(600.93268663,51.89939165)(601.03269043,51.91939941)
\curveto(601.13268643,51.9493916)(601.23768632,51.97939157)(601.34769043,52.00939941)
\curveto(601.40768615,52.01939153)(601.46768609,52.02439152)(601.52769043,52.02439941)
\curveto(601.58768597,52.03439151)(601.64268592,52.0443915)(601.69269043,52.05439941)
}
}
{
\newrgbcolor{curcolor}{0 0 0}
\pscustom[linestyle=none,fillstyle=solid,fillcolor=curcolor]
{
\newpath
\moveto(606.70245605,52.05439941)
\curveto(606.93245126,52.05439149)(607.06245113,51.99439155)(607.09245605,51.87439941)
\curveto(607.12245107,51.76439178)(607.13745106,51.59939195)(607.13745605,51.37939941)
\lineto(607.13745605,51.09439941)
\curveto(607.13745106,51.00439254)(607.11245108,50.92939262)(607.06245605,50.86939941)
\curveto(607.00245119,50.78939276)(606.91745128,50.7443928)(606.80745605,50.73439941)
\curveto(606.6974515,50.73439281)(606.58745161,50.71939283)(606.47745605,50.68939941)
\curveto(606.33745186,50.65939289)(606.20245199,50.62939292)(606.07245605,50.59939941)
\curveto(605.95245224,50.56939298)(605.83745236,50.52939302)(605.72745605,50.47939941)
\curveto(605.43745276,50.3493932)(605.20245299,50.16939338)(605.02245605,49.93939941)
\curveto(604.84245335,49.71939383)(604.68745351,49.46439408)(604.55745605,49.17439941)
\curveto(604.51745368,49.06439448)(604.48745371,48.9493946)(604.46745605,48.82939941)
\curveto(604.44745375,48.71939483)(604.42245377,48.60439494)(604.39245605,48.48439941)
\curveto(604.38245381,48.43439511)(604.37745382,48.38439516)(604.37745605,48.33439941)
\curveto(604.38745381,48.28439526)(604.38745381,48.23439531)(604.37745605,48.18439941)
\curveto(604.34745385,48.06439548)(604.33245386,47.92439562)(604.33245605,47.76439941)
\curveto(604.34245385,47.61439593)(604.34745385,47.46939608)(604.34745605,47.32939941)
\lineto(604.34745605,45.48439941)
\lineto(604.34745605,45.13939941)
\curveto(604.34745385,45.01939853)(604.34245385,44.90439864)(604.33245605,44.79439941)
\curveto(604.32245387,44.68439886)(604.31745388,44.58939896)(604.31745605,44.50939941)
\curveto(604.32745387,44.42939912)(604.30745389,44.35939919)(604.25745605,44.29939941)
\curveto(604.20745399,44.22939932)(604.12745407,44.18939936)(604.01745605,44.17939941)
\curveto(603.91745428,44.16939938)(603.80745439,44.16439938)(603.68745605,44.16439941)
\lineto(603.41745605,44.16439941)
\curveto(603.36745483,44.18439936)(603.31745488,44.19939935)(603.26745605,44.20939941)
\curveto(603.22745497,44.22939932)(603.197455,44.25439929)(603.17745605,44.28439941)
\curveto(603.12745507,44.35439919)(603.0974551,44.43939911)(603.08745605,44.53939941)
\lineto(603.08745605,44.86939941)
\lineto(603.08745605,46.02439941)
\lineto(603.08745605,50.17939941)
\lineto(603.08745605,51.21439941)
\lineto(603.08745605,51.51439941)
\curveto(603.0974551,51.61439193)(603.12745507,51.69939185)(603.17745605,51.76939941)
\curveto(603.20745499,51.80939174)(603.25745494,51.83939171)(603.32745605,51.85939941)
\curveto(603.40745479,51.87939167)(603.4924547,51.88939166)(603.58245605,51.88939941)
\curveto(603.67245452,51.89939165)(603.76245443,51.89939165)(603.85245605,51.88939941)
\curveto(603.94245425,51.87939167)(604.01245418,51.86439168)(604.06245605,51.84439941)
\curveto(604.14245405,51.81439173)(604.192454,51.75439179)(604.21245605,51.66439941)
\curveto(604.24245395,51.58439196)(604.25745394,51.49439205)(604.25745605,51.39439941)
\lineto(604.25745605,51.09439941)
\curveto(604.25745394,50.99439255)(604.27745392,50.90439264)(604.31745605,50.82439941)
\curveto(604.32745387,50.80439274)(604.33745386,50.78939276)(604.34745605,50.77939941)
\lineto(604.39245605,50.73439941)
\curveto(604.50245369,50.73439281)(604.5924536,50.77939277)(604.66245605,50.86939941)
\curveto(604.73245346,50.96939258)(604.7924534,51.0493925)(604.84245605,51.10939941)
\lineto(604.93245605,51.19939941)
\curveto(605.02245317,51.30939224)(605.14745305,51.42439212)(605.30745605,51.54439941)
\curveto(605.46745273,51.66439188)(605.61745258,51.75439179)(605.75745605,51.81439941)
\curveto(605.84745235,51.86439168)(605.94245225,51.89939165)(606.04245605,51.91939941)
\curveto(606.14245205,51.9493916)(606.24745195,51.97939157)(606.35745605,52.00939941)
\curveto(606.41745178,52.01939153)(606.47745172,52.02439152)(606.53745605,52.02439941)
\curveto(606.5974516,52.03439151)(606.65245154,52.0443915)(606.70245605,52.05439941)
}
}
{
\newrgbcolor{curcolor}{0 0 0}
\pscustom[linestyle=none,fillstyle=solid,fillcolor=curcolor]
{
\newpath
\moveto(615.19222168,48.34939941)
\curveto(615.21221362,48.28939526)(615.22221361,48.19439535)(615.22222168,48.06439941)
\curveto(615.22221361,47.9443956)(615.21721361,47.85939569)(615.20722168,47.80939941)
\lineto(615.20722168,47.65939941)
\curveto(615.19721363,47.57939597)(615.18721364,47.50439604)(615.17722168,47.43439941)
\curveto(615.17721365,47.37439617)(615.17221366,47.30439624)(615.16222168,47.22439941)
\curveto(615.14221369,47.16439638)(615.1272137,47.10439644)(615.11722168,47.04439941)
\curveto(615.11721371,46.98439656)(615.10721372,46.92439662)(615.08722168,46.86439941)
\curveto(615.04721378,46.73439681)(615.01221382,46.60439694)(614.98222168,46.47439941)
\curveto(614.95221388,46.3443972)(614.91221392,46.22439732)(614.86222168,46.11439941)
\curveto(614.65221418,45.63439791)(614.37221446,45.22939832)(614.02222168,44.89939941)
\curveto(613.67221516,44.57939897)(613.24221559,44.33439921)(612.73222168,44.16439941)
\curveto(612.62221621,44.12439942)(612.50221633,44.09439945)(612.37222168,44.07439941)
\curveto(612.25221658,44.05439949)(612.1272167,44.03439951)(611.99722168,44.01439941)
\curveto(611.93721689,44.00439954)(611.87221696,43.99939955)(611.80222168,43.99939941)
\curveto(611.74221709,43.98939956)(611.68221715,43.98439956)(611.62222168,43.98439941)
\curveto(611.58221725,43.97439957)(611.52221731,43.96939958)(611.44222168,43.96939941)
\curveto(611.37221746,43.96939958)(611.32221751,43.97439957)(611.29222168,43.98439941)
\curveto(611.25221758,43.99439955)(611.21221762,43.99939955)(611.17222168,43.99939941)
\curveto(611.1322177,43.98939956)(611.09721773,43.98939956)(611.06722168,43.99939941)
\lineto(610.97722168,43.99939941)
\lineto(610.61722168,44.04439941)
\curveto(610.47721835,44.08439946)(610.34221849,44.12439942)(610.21222168,44.16439941)
\curveto(610.08221875,44.20439934)(609.95721887,44.2493993)(609.83722168,44.29939941)
\curveto(609.38721944,44.49939905)(609.01721981,44.75939879)(608.72722168,45.07939941)
\curveto(608.43722039,45.39939815)(608.19722063,45.78939776)(608.00722168,46.24939941)
\curveto(607.95722087,46.3493972)(607.91722091,46.4493971)(607.88722168,46.54939941)
\curveto(607.86722096,46.6493969)(607.84722098,46.75439679)(607.82722168,46.86439941)
\curveto(607.80722102,46.90439664)(607.79722103,46.93439661)(607.79722168,46.95439941)
\curveto(607.80722102,46.98439656)(607.80722102,47.01939653)(607.79722168,47.05939941)
\curveto(607.77722105,47.13939641)(607.76222107,47.21939633)(607.75222168,47.29939941)
\curveto(607.75222108,47.38939616)(607.74222109,47.47439607)(607.72222168,47.55439941)
\lineto(607.72222168,47.67439941)
\curveto(607.72222111,47.71439583)(607.71722111,47.75939579)(607.70722168,47.80939941)
\curveto(607.69722113,47.85939569)(607.69222114,47.9443956)(607.69222168,48.06439941)
\curveto(607.69222114,48.19439535)(607.70222113,48.28939526)(607.72222168,48.34939941)
\curveto(607.74222109,48.41939513)(607.74722108,48.48939506)(607.73722168,48.55939941)
\curveto(607.7272211,48.62939492)(607.7322211,48.69939485)(607.75222168,48.76939941)
\curveto(607.76222107,48.81939473)(607.76722106,48.85939469)(607.76722168,48.88939941)
\curveto(607.77722105,48.92939462)(607.78722104,48.97439457)(607.79722168,49.02439941)
\curveto(607.827221,49.1443944)(607.85222098,49.26439428)(607.87222168,49.38439941)
\curveto(607.90222093,49.50439404)(607.94222089,49.61939393)(607.99222168,49.72939941)
\curveto(608.14222069,50.09939345)(608.32222051,50.42939312)(608.53222168,50.71939941)
\curveto(608.75222008,51.01939253)(609.01721981,51.26939228)(609.32722168,51.46939941)
\curveto(609.44721938,51.549392)(609.57221926,51.61439193)(609.70222168,51.66439941)
\curveto(609.832219,51.72439182)(609.96721886,51.78439176)(610.10722168,51.84439941)
\curveto(610.2272186,51.89439165)(610.35721847,51.92439162)(610.49722168,51.93439941)
\curveto(610.63721819,51.95439159)(610.77721805,51.98439156)(610.91722168,52.02439941)
\lineto(611.11222168,52.02439941)
\curveto(611.18221765,52.03439151)(611.24721758,52.0443915)(611.30722168,52.05439941)
\curveto(612.19721663,52.06439148)(612.93721589,51.87939167)(613.52722168,51.49939941)
\curveto(614.11721471,51.11939243)(614.54221429,50.62439292)(614.80222168,50.01439941)
\curveto(614.85221398,49.91439363)(614.89221394,49.81439373)(614.92222168,49.71439941)
\curveto(614.95221388,49.61439393)(614.98721384,49.50939404)(615.02722168,49.39939941)
\curveto(615.05721377,49.28939426)(615.08221375,49.16939438)(615.10222168,49.03939941)
\curveto(615.12221371,48.91939463)(615.14721368,48.79439475)(615.17722168,48.66439941)
\curveto(615.18721364,48.61439493)(615.18721364,48.55939499)(615.17722168,48.49939941)
\curveto(615.17721365,48.4493951)(615.18221365,48.39939515)(615.19222168,48.34939941)
\moveto(613.85722168,47.49439941)
\curveto(613.87721495,47.56439598)(613.88221495,47.6443959)(613.87222168,47.73439941)
\lineto(613.87222168,47.98939941)
\curveto(613.87221496,48.37939517)(613.83721499,48.70939484)(613.76722168,48.97939941)
\curveto(613.73721509,49.05939449)(613.71221512,49.13939441)(613.69222168,49.21939941)
\curveto(613.67221516,49.29939425)(613.64721518,49.37439417)(613.61722168,49.44439941)
\curveto(613.33721549,50.09439345)(612.89221594,50.544393)(612.28222168,50.79439941)
\curveto(612.21221662,50.82439272)(612.13721669,50.8443927)(612.05722168,50.85439941)
\lineto(611.81722168,50.91439941)
\curveto(611.73721709,50.93439261)(611.65221718,50.9443926)(611.56222168,50.94439941)
\lineto(611.29222168,50.94439941)
\lineto(611.02222168,50.89939941)
\curveto(610.92221791,50.87939267)(610.827218,50.85439269)(610.73722168,50.82439941)
\curveto(610.65721817,50.80439274)(610.57721825,50.77439277)(610.49722168,50.73439941)
\curveto(610.4272184,50.71439283)(610.36221847,50.68439286)(610.30222168,50.64439941)
\curveto(610.24221859,50.60439294)(610.18721864,50.56439298)(610.13722168,50.52439941)
\curveto(609.89721893,50.35439319)(609.70221913,50.1493934)(609.55222168,49.90939941)
\curveto(609.40221943,49.66939388)(609.27221956,49.38939416)(609.16222168,49.06939941)
\curveto(609.1322197,48.96939458)(609.11221972,48.86439468)(609.10222168,48.75439941)
\curveto(609.09221974,48.65439489)(609.07721975,48.549395)(609.05722168,48.43939941)
\curveto(609.04721978,48.39939515)(609.04221979,48.33439521)(609.04222168,48.24439941)
\curveto(609.0322198,48.21439533)(609.0272198,48.17939537)(609.02722168,48.13939941)
\curveto(609.03721979,48.09939545)(609.04221979,48.05439549)(609.04222168,48.00439941)
\lineto(609.04222168,47.70439941)
\curveto(609.04221979,47.60439594)(609.05221978,47.51439603)(609.07222168,47.43439941)
\lineto(609.10222168,47.25439941)
\curveto(609.12221971,47.15439639)(609.13721969,47.05439649)(609.14722168,46.95439941)
\curveto(609.16721966,46.86439668)(609.19721963,46.77939677)(609.23722168,46.69939941)
\curveto(609.33721949,46.45939709)(609.45221938,46.23439731)(609.58222168,46.02439941)
\curveto(609.72221911,45.81439773)(609.89221894,45.63939791)(610.09222168,45.49939941)
\curveto(610.14221869,45.46939808)(610.18721864,45.4443981)(610.22722168,45.42439941)
\curveto(610.26721856,45.40439814)(610.31221852,45.37939817)(610.36222168,45.34939941)
\curveto(610.44221839,45.29939825)(610.5272183,45.25439829)(610.61722168,45.21439941)
\curveto(610.71721811,45.18439836)(610.82221801,45.15439839)(610.93222168,45.12439941)
\curveto(610.98221785,45.10439844)(611.0272178,45.09439845)(611.06722168,45.09439941)
\curveto(611.11721771,45.10439844)(611.16721766,45.10439844)(611.21722168,45.09439941)
\curveto(611.24721758,45.08439846)(611.30721752,45.07439847)(611.39722168,45.06439941)
\curveto(611.49721733,45.05439849)(611.57221726,45.05939849)(611.62222168,45.07939941)
\curveto(611.66221717,45.08939846)(611.70221713,45.08939846)(611.74222168,45.07939941)
\curveto(611.78221705,45.07939847)(611.82221701,45.08939846)(611.86222168,45.10939941)
\curveto(611.94221689,45.12939842)(612.02221681,45.1443984)(612.10222168,45.15439941)
\curveto(612.18221665,45.17439837)(612.25721657,45.19939835)(612.32722168,45.22939941)
\curveto(612.66721616,45.36939818)(612.94221589,45.56439798)(613.15222168,45.81439941)
\curveto(613.36221547,46.06439748)(613.53721529,46.35939719)(613.67722168,46.69939941)
\curveto(613.7272151,46.81939673)(613.75721507,46.9443966)(613.76722168,47.07439941)
\curveto(613.78721504,47.21439633)(613.81721501,47.35439619)(613.85722168,47.49439941)
}
}
{
\newrgbcolor{curcolor}{0 0 0}
\pscustom[linestyle=none,fillstyle=solid,fillcolor=curcolor]
{
\newpath
\moveto(617.25050293,54.82939941)
\curveto(617.38050131,54.82938872)(617.51550118,54.82938872)(617.65550293,54.82939941)
\curveto(617.80550089,54.82938872)(617.91550078,54.79438875)(617.98550293,54.72439941)
\curveto(618.03550066,54.65438889)(618.06050063,54.55938899)(618.06050293,54.43939941)
\curveto(618.07050062,54.32938922)(618.07550062,54.21438933)(618.07550293,54.09439941)
\lineto(618.07550293,52.75939941)
\lineto(618.07550293,46.68439941)
\lineto(618.07550293,45.00439941)
\lineto(618.07550293,44.61439941)
\curveto(618.07550062,44.47439907)(618.05050064,44.36439918)(618.00050293,44.28439941)
\curveto(617.97050072,44.23439931)(617.92550077,44.20439934)(617.86550293,44.19439941)
\curveto(617.81550088,44.18439936)(617.75050094,44.16939938)(617.67050293,44.14939941)
\lineto(617.46050293,44.14939941)
\lineto(617.14550293,44.14939941)
\curveto(617.04550165,44.15939939)(616.97050172,44.19439935)(616.92050293,44.25439941)
\curveto(616.87050182,44.33439921)(616.84050185,44.43439911)(616.83050293,44.55439941)
\lineto(616.83050293,44.92939941)
\lineto(616.83050293,46.30939941)
\lineto(616.83050293,52.54939941)
\lineto(616.83050293,54.01939941)
\curveto(616.83050186,54.12938942)(616.82550187,54.2443893)(616.81550293,54.36439941)
\curveto(616.81550188,54.49438905)(616.84050185,54.59438895)(616.89050293,54.66439941)
\curveto(616.93050176,54.72438882)(617.00550169,54.77438877)(617.11550293,54.81439941)
\curveto(617.13550156,54.82438872)(617.15550154,54.82438872)(617.17550293,54.81439941)
\curveto(617.20550149,54.81438873)(617.23050146,54.81938873)(617.25050293,54.82939941)
}
}
{
\newrgbcolor{curcolor}{0 0 0}
\pscustom[linestyle=none,fillstyle=solid,fillcolor=curcolor]
{
\newpath
\moveto(620.59034668,54.82939941)
\curveto(620.72034506,54.82938872)(620.85534493,54.82938872)(620.99534668,54.82939941)
\curveto(621.14534464,54.82938872)(621.25534453,54.79438875)(621.32534668,54.72439941)
\curveto(621.37534441,54.65438889)(621.40034438,54.55938899)(621.40034668,54.43939941)
\curveto(621.41034437,54.32938922)(621.41534437,54.21438933)(621.41534668,54.09439941)
\lineto(621.41534668,52.75939941)
\lineto(621.41534668,46.68439941)
\lineto(621.41534668,45.00439941)
\lineto(621.41534668,44.61439941)
\curveto(621.41534437,44.47439907)(621.39034439,44.36439918)(621.34034668,44.28439941)
\curveto(621.31034447,44.23439931)(621.26534452,44.20439934)(621.20534668,44.19439941)
\curveto(621.15534463,44.18439936)(621.09034469,44.16939938)(621.01034668,44.14939941)
\lineto(620.80034668,44.14939941)
\lineto(620.48534668,44.14939941)
\curveto(620.3853454,44.15939939)(620.31034547,44.19439935)(620.26034668,44.25439941)
\curveto(620.21034557,44.33439921)(620.1803456,44.43439911)(620.17034668,44.55439941)
\lineto(620.17034668,44.92939941)
\lineto(620.17034668,46.30939941)
\lineto(620.17034668,52.54939941)
\lineto(620.17034668,54.01939941)
\curveto(620.17034561,54.12938942)(620.16534562,54.2443893)(620.15534668,54.36439941)
\curveto(620.15534563,54.49438905)(620.1803456,54.59438895)(620.23034668,54.66439941)
\curveto(620.27034551,54.72438882)(620.34534544,54.77438877)(620.45534668,54.81439941)
\curveto(620.47534531,54.82438872)(620.49534529,54.82438872)(620.51534668,54.81439941)
\curveto(620.54534524,54.81438873)(620.57034521,54.81938873)(620.59034668,54.82939941)
}
}
{
\newrgbcolor{curcolor}{0 0 0}
\pscustom[linestyle=none,fillstyle=solid,fillcolor=curcolor]
{
\newpath
\moveto(630.24519043,44.70439941)
\curveto(630.2751826,44.544399)(630.26018261,44.40939914)(630.20019043,44.29939941)
\curveto(630.14018273,44.19939935)(630.06018281,44.12439942)(629.96019043,44.07439941)
\curveto(629.91018296,44.05439949)(629.85518302,44.0443995)(629.79519043,44.04439941)
\curveto(629.74518313,44.0443995)(629.69018318,44.03439951)(629.63019043,44.01439941)
\curveto(629.41018346,43.96439958)(629.19018368,43.97939957)(628.97019043,44.05939941)
\curveto(628.76018411,44.12939942)(628.61518426,44.21939933)(628.53519043,44.32939941)
\curveto(628.48518439,44.39939915)(628.44018443,44.47939907)(628.40019043,44.56939941)
\curveto(628.36018451,44.66939888)(628.31018456,44.7493988)(628.25019043,44.80939941)
\curveto(628.23018464,44.82939872)(628.20518467,44.8493987)(628.17519043,44.86939941)
\curveto(628.15518472,44.88939866)(628.12518475,44.89439865)(628.08519043,44.88439941)
\curveto(627.9751849,44.85439869)(627.870185,44.79939875)(627.77019043,44.71939941)
\curveto(627.68018519,44.63939891)(627.59018528,44.56939898)(627.50019043,44.50939941)
\curveto(627.3701855,44.42939912)(627.23018564,44.35439919)(627.08019043,44.28439941)
\curveto(626.93018594,44.22439932)(626.7701861,44.16939938)(626.60019043,44.11939941)
\curveto(626.50018637,44.08939946)(626.39018648,44.06939948)(626.27019043,44.05939941)
\curveto(626.16018671,44.0493995)(626.05018682,44.03439951)(625.94019043,44.01439941)
\curveto(625.89018698,44.00439954)(625.84518703,43.99939955)(625.80519043,43.99939941)
\lineto(625.70019043,43.99939941)
\curveto(625.59018728,43.97939957)(625.48518739,43.97939957)(625.38519043,43.99939941)
\lineto(625.25019043,43.99939941)
\curveto(625.20018767,44.00939954)(625.15018772,44.01439953)(625.10019043,44.01439941)
\curveto(625.05018782,44.01439953)(625.00518787,44.02439952)(624.96519043,44.04439941)
\curveto(624.92518795,44.05439949)(624.89018798,44.05939949)(624.86019043,44.05939941)
\curveto(624.84018803,44.0493995)(624.81518806,44.0493995)(624.78519043,44.05939941)
\lineto(624.54519043,44.11939941)
\curveto(624.46518841,44.12939942)(624.39018848,44.1493994)(624.32019043,44.17939941)
\curveto(624.02018885,44.30939924)(623.7751891,44.45439909)(623.58519043,44.61439941)
\curveto(623.40518947,44.78439876)(623.25518962,45.01939853)(623.13519043,45.31939941)
\curveto(623.04518983,45.53939801)(623.00018987,45.80439774)(623.00019043,46.11439941)
\lineto(623.00019043,46.42939941)
\curveto(623.01018986,46.47939707)(623.01518986,46.52939702)(623.01519043,46.57939941)
\lineto(623.04519043,46.75939941)
\lineto(623.16519043,47.08939941)
\curveto(623.20518967,47.19939635)(623.25518962,47.29939625)(623.31519043,47.38939941)
\curveto(623.49518938,47.67939587)(623.74018913,47.89439565)(624.05019043,48.03439941)
\curveto(624.36018851,48.17439537)(624.70018817,48.29939525)(625.07019043,48.40939941)
\curveto(625.21018766,48.4493951)(625.35518752,48.47939507)(625.50519043,48.49939941)
\curveto(625.65518722,48.51939503)(625.80518707,48.544395)(625.95519043,48.57439941)
\curveto(626.02518685,48.59439495)(626.09018678,48.60439494)(626.15019043,48.60439941)
\curveto(626.22018665,48.60439494)(626.29518658,48.61439493)(626.37519043,48.63439941)
\curveto(626.44518643,48.65439489)(626.51518636,48.66439488)(626.58519043,48.66439941)
\curveto(626.65518622,48.67439487)(626.73018614,48.68939486)(626.81019043,48.70939941)
\curveto(627.06018581,48.76939478)(627.29518558,48.81939473)(627.51519043,48.85939941)
\curveto(627.73518514,48.90939464)(627.91018496,49.02439452)(628.04019043,49.20439941)
\curveto(628.10018477,49.28439426)(628.15018472,49.38439416)(628.19019043,49.50439941)
\curveto(628.23018464,49.63439391)(628.23018464,49.77439377)(628.19019043,49.92439941)
\curveto(628.13018474,50.16439338)(628.04018483,50.35439319)(627.92019043,50.49439941)
\curveto(627.81018506,50.63439291)(627.65018522,50.7443928)(627.44019043,50.82439941)
\curveto(627.32018555,50.87439267)(627.1751857,50.90939264)(627.00519043,50.92939941)
\curveto(626.84518603,50.9493926)(626.6751862,50.95939259)(626.49519043,50.95939941)
\curveto(626.31518656,50.95939259)(626.14018673,50.9493926)(625.97019043,50.92939941)
\curveto(625.80018707,50.90939264)(625.65518722,50.87939267)(625.53519043,50.83939941)
\curveto(625.36518751,50.77939277)(625.20018767,50.69439285)(625.04019043,50.58439941)
\curveto(624.96018791,50.52439302)(624.88518799,50.4443931)(624.81519043,50.34439941)
\curveto(624.75518812,50.25439329)(624.70018817,50.15439339)(624.65019043,50.04439941)
\curveto(624.62018825,49.96439358)(624.59018828,49.87939367)(624.56019043,49.78939941)
\curveto(624.54018833,49.69939385)(624.49518838,49.62939392)(624.42519043,49.57939941)
\curveto(624.38518849,49.549394)(624.31518856,49.52439402)(624.21519043,49.50439941)
\curveto(624.12518875,49.49439405)(624.03018884,49.48939406)(623.93019043,49.48939941)
\curveto(623.83018904,49.48939406)(623.73018914,49.49439405)(623.63019043,49.50439941)
\curveto(623.54018933,49.52439402)(623.4751894,49.549394)(623.43519043,49.57939941)
\curveto(623.39518948,49.60939394)(623.36518951,49.65939389)(623.34519043,49.72939941)
\curveto(623.32518955,49.79939375)(623.32518955,49.87439367)(623.34519043,49.95439941)
\curveto(623.3751895,50.08439346)(623.40518947,50.20439334)(623.43519043,50.31439941)
\curveto(623.4751894,50.43439311)(623.52018935,50.549393)(623.57019043,50.65939941)
\curveto(623.76018911,51.00939254)(624.00018887,51.27939227)(624.29019043,51.46939941)
\curveto(624.58018829,51.66939188)(624.94018793,51.82939172)(625.37019043,51.94939941)
\curveto(625.4701874,51.96939158)(625.5701873,51.98439156)(625.67019043,51.99439941)
\curveto(625.78018709,52.00439154)(625.89018698,52.01939153)(626.00019043,52.03939941)
\curveto(626.04018683,52.0493915)(626.10518677,52.0493915)(626.19519043,52.03939941)
\curveto(626.28518659,52.03939151)(626.34018653,52.0493915)(626.36019043,52.06939941)
\curveto(627.06018581,52.07939147)(627.6701852,51.99939155)(628.19019043,51.82939941)
\curveto(628.71018416,51.65939189)(629.0751838,51.33439221)(629.28519043,50.85439941)
\curveto(629.3751835,50.65439289)(629.42518345,50.41939313)(629.43519043,50.14939941)
\curveto(629.45518342,49.88939366)(629.46518341,49.61439393)(629.46519043,49.32439941)
\lineto(629.46519043,46.00939941)
\curveto(629.46518341,45.86939768)(629.4701834,45.73439781)(629.48019043,45.60439941)
\curveto(629.49018338,45.47439807)(629.52018335,45.36939818)(629.57019043,45.28939941)
\curveto(629.62018325,45.21939833)(629.68518319,45.16939838)(629.76519043,45.13939941)
\curveto(629.85518302,45.09939845)(629.94018293,45.06939848)(630.02019043,45.04939941)
\curveto(630.10018277,45.03939851)(630.16018271,44.99439855)(630.20019043,44.91439941)
\curveto(630.22018265,44.88439866)(630.23018264,44.85439869)(630.23019043,44.82439941)
\curveto(630.23018264,44.79439875)(630.23518264,44.75439879)(630.24519043,44.70439941)
\moveto(628.10019043,46.36939941)
\curveto(628.16018471,46.50939704)(628.19018468,46.66939688)(628.19019043,46.84939941)
\curveto(628.20018467,47.03939651)(628.20518467,47.23439631)(628.20519043,47.43439941)
\curveto(628.20518467,47.544396)(628.20018467,47.6443959)(628.19019043,47.73439941)
\curveto(628.18018469,47.82439572)(628.14018473,47.89439565)(628.07019043,47.94439941)
\curveto(628.04018483,47.96439558)(627.9701849,47.97439557)(627.86019043,47.97439941)
\curveto(627.84018503,47.95439559)(627.80518507,47.9443956)(627.75519043,47.94439941)
\curveto(627.70518517,47.9443956)(627.66018521,47.93439561)(627.62019043,47.91439941)
\curveto(627.54018533,47.89439565)(627.45018542,47.87439567)(627.35019043,47.85439941)
\lineto(627.05019043,47.79439941)
\curveto(627.02018585,47.79439575)(626.98518589,47.78939576)(626.94519043,47.77939941)
\lineto(626.84019043,47.77939941)
\curveto(626.69018618,47.73939581)(626.52518635,47.71439583)(626.34519043,47.70439941)
\curveto(626.1751867,47.70439584)(626.01518686,47.68439586)(625.86519043,47.64439941)
\curveto(625.78518709,47.62439592)(625.71018716,47.60439594)(625.64019043,47.58439941)
\curveto(625.58018729,47.57439597)(625.51018736,47.55939599)(625.43019043,47.53939941)
\curveto(625.2701876,47.48939606)(625.12018775,47.42439612)(624.98019043,47.34439941)
\curveto(624.84018803,47.27439627)(624.72018815,47.18439636)(624.62019043,47.07439941)
\curveto(624.52018835,46.96439658)(624.44518843,46.82939672)(624.39519043,46.66939941)
\curveto(624.34518853,46.51939703)(624.32518855,46.33439721)(624.33519043,46.11439941)
\curveto(624.33518854,46.01439753)(624.35018852,45.91939763)(624.38019043,45.82939941)
\curveto(624.42018845,45.7493978)(624.46518841,45.67439787)(624.51519043,45.60439941)
\curveto(624.59518828,45.49439805)(624.70018817,45.39939815)(624.83019043,45.31939941)
\curveto(624.96018791,45.2493983)(625.10018777,45.18939836)(625.25019043,45.13939941)
\curveto(625.30018757,45.12939842)(625.35018752,45.12439842)(625.40019043,45.12439941)
\curveto(625.45018742,45.12439842)(625.50018737,45.11939843)(625.55019043,45.10939941)
\curveto(625.62018725,45.08939846)(625.70518717,45.07439847)(625.80519043,45.06439941)
\curveto(625.91518696,45.06439848)(626.00518687,45.07439847)(626.07519043,45.09439941)
\curveto(626.13518674,45.11439843)(626.19518668,45.11939843)(626.25519043,45.10939941)
\curveto(626.31518656,45.10939844)(626.3751865,45.11939843)(626.43519043,45.13939941)
\curveto(626.51518636,45.15939839)(626.59018628,45.17439837)(626.66019043,45.18439941)
\curveto(626.74018613,45.19439835)(626.81518606,45.21439833)(626.88519043,45.24439941)
\curveto(627.1751857,45.36439818)(627.42018545,45.50939804)(627.62019043,45.67939941)
\curveto(627.83018504,45.8493977)(627.99018488,46.07939747)(628.10019043,46.36939941)
}
}
{
\newrgbcolor{curcolor}{0 0 0}
\pscustom[linestyle=none,fillstyle=solid,fillcolor=curcolor]
{
\newpath
\moveto(638.37683105,44.95939941)
\lineto(638.37683105,44.56939941)
\curveto(638.37682318,44.4493991)(638.3518232,44.3493992)(638.30183105,44.26939941)
\curveto(638.2518233,44.19939935)(638.16682339,44.15939939)(638.04683105,44.14939941)
\lineto(637.70183105,44.14939941)
\curveto(637.64182391,44.1493994)(637.58182397,44.1443994)(637.52183105,44.13439941)
\curveto(637.47182408,44.13439941)(637.42682413,44.1443994)(637.38683105,44.16439941)
\curveto(637.29682426,44.18439936)(637.23682432,44.22439932)(637.20683105,44.28439941)
\curveto(637.16682439,44.33439921)(637.14182441,44.39439915)(637.13183105,44.46439941)
\curveto(637.13182442,44.53439901)(637.11682444,44.60439894)(637.08683105,44.67439941)
\curveto(637.07682448,44.69439885)(637.06182449,44.70939884)(637.04183105,44.71939941)
\curveto(637.03182452,44.73939881)(637.01682454,44.75939879)(636.99683105,44.77939941)
\curveto(636.89682466,44.78939876)(636.81682474,44.76939878)(636.75683105,44.71939941)
\curveto(636.70682485,44.66939888)(636.6518249,44.61939893)(636.59183105,44.56939941)
\curveto(636.39182516,44.41939913)(636.19182536,44.30439924)(635.99183105,44.22439941)
\curveto(635.81182574,44.1443994)(635.60182595,44.08439946)(635.36183105,44.04439941)
\curveto(635.13182642,44.00439954)(634.89182666,43.98439956)(634.64183105,43.98439941)
\curveto(634.40182715,43.97439957)(634.16182739,43.98939956)(633.92183105,44.02939941)
\curveto(633.68182787,44.05939949)(633.47182808,44.11439943)(633.29183105,44.19439941)
\curveto(632.77182878,44.41439913)(632.3518292,44.70939884)(632.03183105,45.07939941)
\curveto(631.71182984,45.45939809)(631.46183009,45.92939762)(631.28183105,46.48939941)
\curveto(631.24183031,46.57939697)(631.21183034,46.66939688)(631.19183105,46.75939941)
\curveto(631.18183037,46.85939669)(631.16183039,46.95939659)(631.13183105,47.05939941)
\curveto(631.12183043,47.10939644)(631.11683044,47.15939639)(631.11683105,47.20939941)
\curveto(631.11683044,47.25939629)(631.11183044,47.30939624)(631.10183105,47.35939941)
\curveto(631.08183047,47.40939614)(631.07183048,47.45939609)(631.07183105,47.50939941)
\curveto(631.08183047,47.56939598)(631.08183047,47.62439592)(631.07183105,47.67439941)
\lineto(631.07183105,47.82439941)
\curveto(631.0518305,47.87439567)(631.04183051,47.93939561)(631.04183105,48.01939941)
\curveto(631.04183051,48.09939545)(631.0518305,48.16439538)(631.07183105,48.21439941)
\lineto(631.07183105,48.37939941)
\curveto(631.09183046,48.4493951)(631.09683046,48.51939503)(631.08683105,48.58939941)
\curveto(631.08683047,48.66939488)(631.09683046,48.7443948)(631.11683105,48.81439941)
\curveto(631.12683043,48.86439468)(631.13183042,48.90939464)(631.13183105,48.94939941)
\curveto(631.13183042,48.98939456)(631.13683042,49.03439451)(631.14683105,49.08439941)
\curveto(631.17683038,49.18439436)(631.20183035,49.27939427)(631.22183105,49.36939941)
\curveto(631.24183031,49.46939408)(631.26683029,49.56439398)(631.29683105,49.65439941)
\curveto(631.42683013,50.03439351)(631.59182996,50.37439317)(631.79183105,50.67439941)
\curveto(632.00182955,50.98439256)(632.2518293,51.23939231)(632.54183105,51.43939941)
\curveto(632.71182884,51.55939199)(632.88682867,51.65939189)(633.06683105,51.73939941)
\curveto(633.2568283,51.81939173)(633.46182809,51.88939166)(633.68183105,51.94939941)
\curveto(633.7518278,51.95939159)(633.81682774,51.96939158)(633.87683105,51.97939941)
\curveto(633.94682761,51.98939156)(634.01682754,52.00439154)(634.08683105,52.02439941)
\lineto(634.23683105,52.02439941)
\curveto(634.31682724,52.0443915)(634.43182712,52.05439149)(634.58183105,52.05439941)
\curveto(634.74182681,52.05439149)(634.86182669,52.0443915)(634.94183105,52.02439941)
\curveto(634.98182657,52.01439153)(635.03682652,52.00939154)(635.10683105,52.00939941)
\curveto(635.21682634,51.97939157)(635.32682623,51.95439159)(635.43683105,51.93439941)
\curveto(635.54682601,51.92439162)(635.6518259,51.89439165)(635.75183105,51.84439941)
\curveto(635.90182565,51.78439176)(636.04182551,51.71939183)(636.17183105,51.64939941)
\curveto(636.31182524,51.57939197)(636.44182511,51.49939205)(636.56183105,51.40939941)
\curveto(636.62182493,51.35939219)(636.68182487,51.30439224)(636.74183105,51.24439941)
\curveto(636.81182474,51.19439235)(636.90182465,51.17939237)(637.01183105,51.19939941)
\curveto(637.03182452,51.22939232)(637.04682451,51.25439229)(637.05683105,51.27439941)
\curveto(637.07682448,51.29439225)(637.09182446,51.32439222)(637.10183105,51.36439941)
\curveto(637.13182442,51.45439209)(637.14182441,51.56939198)(637.13183105,51.70939941)
\lineto(637.13183105,52.08439941)
\lineto(637.13183105,53.80939941)
\lineto(637.13183105,54.27439941)
\curveto(637.13182442,54.45438909)(637.1568244,54.58438896)(637.20683105,54.66439941)
\curveto(637.24682431,54.73438881)(637.30682425,54.77938877)(637.38683105,54.79939941)
\curveto(637.40682415,54.79938875)(637.43182412,54.79938875)(637.46183105,54.79939941)
\curveto(637.49182406,54.80938874)(637.51682404,54.81438873)(637.53683105,54.81439941)
\curveto(637.67682388,54.82438872)(637.82182373,54.82438872)(637.97183105,54.81439941)
\curveto(638.13182342,54.81438873)(638.24182331,54.77438877)(638.30183105,54.69439941)
\curveto(638.3518232,54.61438893)(638.37682318,54.51438903)(638.37683105,54.39439941)
\lineto(638.37683105,54.01939941)
\lineto(638.37683105,44.95939941)
\moveto(637.16183105,47.79439941)
\curveto(637.18182437,47.8443957)(637.19182436,47.90939564)(637.19183105,47.98939941)
\curveto(637.19182436,48.07939547)(637.18182437,48.1493954)(637.16183105,48.19939941)
\lineto(637.16183105,48.42439941)
\curveto(637.14182441,48.51439503)(637.12682443,48.60439494)(637.11683105,48.69439941)
\curveto(637.10682445,48.79439475)(637.08682447,48.88439466)(637.05683105,48.96439941)
\curveto(637.03682452,49.0443945)(637.01682454,49.11939443)(636.99683105,49.18939941)
\curveto(636.98682457,49.25939429)(636.96682459,49.32939422)(636.93683105,49.39939941)
\curveto(636.81682474,49.69939385)(636.66182489,49.96439358)(636.47183105,50.19439941)
\curveto(636.28182527,50.42439312)(636.04182551,50.60439294)(635.75183105,50.73439941)
\curveto(635.6518259,50.78439276)(635.54682601,50.81939273)(635.43683105,50.83939941)
\curveto(635.33682622,50.86939268)(635.22682633,50.89439265)(635.10683105,50.91439941)
\curveto(635.02682653,50.93439261)(634.93682662,50.9443926)(634.83683105,50.94439941)
\lineto(634.56683105,50.94439941)
\curveto(634.51682704,50.93439261)(634.47182708,50.92439262)(634.43183105,50.91439941)
\lineto(634.29683105,50.91439941)
\curveto(634.21682734,50.89439265)(634.13182742,50.87439267)(634.04183105,50.85439941)
\curveto(633.96182759,50.83439271)(633.88182767,50.80939274)(633.80183105,50.77939941)
\curveto(633.48182807,50.63939291)(633.22182833,50.43439311)(633.02183105,50.16439941)
\curveto(632.83182872,49.90439364)(632.67682888,49.59939395)(632.55683105,49.24939941)
\curveto(632.51682904,49.13939441)(632.48682907,49.02439452)(632.46683105,48.90439941)
\curveto(632.4568291,48.79439475)(632.44182911,48.68439486)(632.42183105,48.57439941)
\curveto(632.42182913,48.53439501)(632.41682914,48.49439505)(632.40683105,48.45439941)
\lineto(632.40683105,48.34939941)
\curveto(632.38682917,48.29939525)(632.37682918,48.2443953)(632.37683105,48.18439941)
\curveto(632.38682917,48.12439542)(632.39182916,48.06939548)(632.39183105,48.01939941)
\lineto(632.39183105,47.68939941)
\curveto(632.39182916,47.58939596)(632.40182915,47.49439605)(632.42183105,47.40439941)
\curveto(632.43182912,47.37439617)(632.43682912,47.32439622)(632.43683105,47.25439941)
\curveto(632.4568291,47.18439636)(632.47182908,47.11439643)(632.48183105,47.04439941)
\lineto(632.54183105,46.83439941)
\curveto(632.6518289,46.48439706)(632.80182875,46.18439736)(632.99183105,45.93439941)
\curveto(633.18182837,45.68439786)(633.42182813,45.47939807)(633.71183105,45.31939941)
\curveto(633.80182775,45.26939828)(633.89182766,45.22939832)(633.98183105,45.19939941)
\curveto(634.07182748,45.16939838)(634.17182738,45.13939841)(634.28183105,45.10939941)
\curveto(634.33182722,45.08939846)(634.38182717,45.08439846)(634.43183105,45.09439941)
\curveto(634.49182706,45.10439844)(634.54682701,45.09939845)(634.59683105,45.07939941)
\curveto(634.63682692,45.06939848)(634.67682688,45.06439848)(634.71683105,45.06439941)
\lineto(634.85183105,45.06439941)
\lineto(634.98683105,45.06439941)
\curveto(635.01682654,45.07439847)(635.06682649,45.07939847)(635.13683105,45.07939941)
\curveto(635.21682634,45.09939845)(635.29682626,45.11439843)(635.37683105,45.12439941)
\curveto(635.4568261,45.1443984)(635.53182602,45.16939838)(635.60183105,45.19939941)
\curveto(635.93182562,45.33939821)(636.19682536,45.51439803)(636.39683105,45.72439941)
\curveto(636.60682495,45.9443976)(636.78182477,46.21939733)(636.92183105,46.54939941)
\curveto(636.97182458,46.65939689)(637.00682455,46.76939678)(637.02683105,46.87939941)
\curveto(637.04682451,46.98939656)(637.07182448,47.09939645)(637.10183105,47.20939941)
\curveto(637.12182443,47.2493963)(637.13182442,47.28439626)(637.13183105,47.31439941)
\curveto(637.13182442,47.35439619)(637.13682442,47.39439615)(637.14683105,47.43439941)
\curveto(637.1568244,47.49439605)(637.1568244,47.55439599)(637.14683105,47.61439941)
\curveto(637.14682441,47.67439587)(637.1518244,47.73439581)(637.16183105,47.79439941)
}
}
{
\newrgbcolor{curcolor}{0 0 0}
\pscustom[linestyle=none,fillstyle=solid,fillcolor=curcolor]
{
\newpath
\moveto(647.44808105,48.34939941)
\curveto(647.46807299,48.28939526)(647.47807298,48.19439535)(647.47808105,48.06439941)
\curveto(647.47807298,47.9443956)(647.47307299,47.85939569)(647.46308105,47.80939941)
\lineto(647.46308105,47.65939941)
\curveto(647.45307301,47.57939597)(647.44307302,47.50439604)(647.43308105,47.43439941)
\curveto(647.43307303,47.37439617)(647.42807303,47.30439624)(647.41808105,47.22439941)
\curveto(647.39807306,47.16439638)(647.38307308,47.10439644)(647.37308105,47.04439941)
\curveto(647.37307309,46.98439656)(647.3630731,46.92439662)(647.34308105,46.86439941)
\curveto(647.30307316,46.73439681)(647.26807319,46.60439694)(647.23808105,46.47439941)
\curveto(647.20807325,46.3443972)(647.16807329,46.22439732)(647.11808105,46.11439941)
\curveto(646.90807355,45.63439791)(646.62807383,45.22939832)(646.27808105,44.89939941)
\curveto(645.92807453,44.57939897)(645.49807496,44.33439921)(644.98808105,44.16439941)
\curveto(644.87807558,44.12439942)(644.7580757,44.09439945)(644.62808105,44.07439941)
\curveto(644.50807595,44.05439949)(644.38307608,44.03439951)(644.25308105,44.01439941)
\curveto(644.19307627,44.00439954)(644.12807633,43.99939955)(644.05808105,43.99939941)
\curveto(643.99807646,43.98939956)(643.93807652,43.98439956)(643.87808105,43.98439941)
\curveto(643.83807662,43.97439957)(643.77807668,43.96939958)(643.69808105,43.96939941)
\curveto(643.62807683,43.96939958)(643.57807688,43.97439957)(643.54808105,43.98439941)
\curveto(643.50807695,43.99439955)(643.46807699,43.99939955)(643.42808105,43.99939941)
\curveto(643.38807707,43.98939956)(643.35307711,43.98939956)(643.32308105,43.99939941)
\lineto(643.23308105,43.99939941)
\lineto(642.87308105,44.04439941)
\curveto(642.73307773,44.08439946)(642.59807786,44.12439942)(642.46808105,44.16439941)
\curveto(642.33807812,44.20439934)(642.21307825,44.2493993)(642.09308105,44.29939941)
\curveto(641.64307882,44.49939905)(641.27307919,44.75939879)(640.98308105,45.07939941)
\curveto(640.69307977,45.39939815)(640.45308001,45.78939776)(640.26308105,46.24939941)
\curveto(640.21308025,46.3493972)(640.17308029,46.4493971)(640.14308105,46.54939941)
\curveto(640.12308034,46.6493969)(640.10308036,46.75439679)(640.08308105,46.86439941)
\curveto(640.0630804,46.90439664)(640.05308041,46.93439661)(640.05308105,46.95439941)
\curveto(640.0630804,46.98439656)(640.0630804,47.01939653)(640.05308105,47.05939941)
\curveto(640.03308043,47.13939641)(640.01808044,47.21939633)(640.00808105,47.29939941)
\curveto(640.00808045,47.38939616)(639.99808046,47.47439607)(639.97808105,47.55439941)
\lineto(639.97808105,47.67439941)
\curveto(639.97808048,47.71439583)(639.97308049,47.75939579)(639.96308105,47.80939941)
\curveto(639.95308051,47.85939569)(639.94808051,47.9443956)(639.94808105,48.06439941)
\curveto(639.94808051,48.19439535)(639.9580805,48.28939526)(639.97808105,48.34939941)
\curveto(639.99808046,48.41939513)(640.00308046,48.48939506)(639.99308105,48.55939941)
\curveto(639.98308048,48.62939492)(639.98808047,48.69939485)(640.00808105,48.76939941)
\curveto(640.01808044,48.81939473)(640.02308044,48.85939469)(640.02308105,48.88939941)
\curveto(640.03308043,48.92939462)(640.04308042,48.97439457)(640.05308105,49.02439941)
\curveto(640.08308038,49.1443944)(640.10808035,49.26439428)(640.12808105,49.38439941)
\curveto(640.1580803,49.50439404)(640.19808026,49.61939393)(640.24808105,49.72939941)
\curveto(640.39808006,50.09939345)(640.57807988,50.42939312)(640.78808105,50.71939941)
\curveto(641.00807945,51.01939253)(641.27307919,51.26939228)(641.58308105,51.46939941)
\curveto(641.70307876,51.549392)(641.82807863,51.61439193)(641.95808105,51.66439941)
\curveto(642.08807837,51.72439182)(642.22307824,51.78439176)(642.36308105,51.84439941)
\curveto(642.48307798,51.89439165)(642.61307785,51.92439162)(642.75308105,51.93439941)
\curveto(642.89307757,51.95439159)(643.03307743,51.98439156)(643.17308105,52.02439941)
\lineto(643.36808105,52.02439941)
\curveto(643.43807702,52.03439151)(643.50307696,52.0443915)(643.56308105,52.05439941)
\curveto(644.45307601,52.06439148)(645.19307527,51.87939167)(645.78308105,51.49939941)
\curveto(646.37307409,51.11939243)(646.79807366,50.62439292)(647.05808105,50.01439941)
\curveto(647.10807335,49.91439363)(647.14807331,49.81439373)(647.17808105,49.71439941)
\curveto(647.20807325,49.61439393)(647.24307322,49.50939404)(647.28308105,49.39939941)
\curveto(647.31307315,49.28939426)(647.33807312,49.16939438)(647.35808105,49.03939941)
\curveto(647.37807308,48.91939463)(647.40307306,48.79439475)(647.43308105,48.66439941)
\curveto(647.44307302,48.61439493)(647.44307302,48.55939499)(647.43308105,48.49939941)
\curveto(647.43307303,48.4493951)(647.43807302,48.39939515)(647.44808105,48.34939941)
\moveto(646.11308105,47.49439941)
\curveto(646.13307433,47.56439598)(646.13807432,47.6443959)(646.12808105,47.73439941)
\lineto(646.12808105,47.98939941)
\curveto(646.12807433,48.37939517)(646.09307437,48.70939484)(646.02308105,48.97939941)
\curveto(645.99307447,49.05939449)(645.96807449,49.13939441)(645.94808105,49.21939941)
\curveto(645.92807453,49.29939425)(645.90307456,49.37439417)(645.87308105,49.44439941)
\curveto(645.59307487,50.09439345)(645.14807531,50.544393)(644.53808105,50.79439941)
\curveto(644.46807599,50.82439272)(644.39307607,50.8443927)(644.31308105,50.85439941)
\lineto(644.07308105,50.91439941)
\curveto(643.99307647,50.93439261)(643.90807655,50.9443926)(643.81808105,50.94439941)
\lineto(643.54808105,50.94439941)
\lineto(643.27808105,50.89939941)
\curveto(643.17807728,50.87939267)(643.08307738,50.85439269)(642.99308105,50.82439941)
\curveto(642.91307755,50.80439274)(642.83307763,50.77439277)(642.75308105,50.73439941)
\curveto(642.68307778,50.71439283)(642.61807784,50.68439286)(642.55808105,50.64439941)
\curveto(642.49807796,50.60439294)(642.44307802,50.56439298)(642.39308105,50.52439941)
\curveto(642.15307831,50.35439319)(641.9580785,50.1493934)(641.80808105,49.90939941)
\curveto(641.6580788,49.66939388)(641.52807893,49.38939416)(641.41808105,49.06939941)
\curveto(641.38807907,48.96939458)(641.36807909,48.86439468)(641.35808105,48.75439941)
\curveto(641.34807911,48.65439489)(641.33307913,48.549395)(641.31308105,48.43939941)
\curveto(641.30307916,48.39939515)(641.29807916,48.33439521)(641.29808105,48.24439941)
\curveto(641.28807917,48.21439533)(641.28307918,48.17939537)(641.28308105,48.13939941)
\curveto(641.29307917,48.09939545)(641.29807916,48.05439549)(641.29808105,48.00439941)
\lineto(641.29808105,47.70439941)
\curveto(641.29807916,47.60439594)(641.30807915,47.51439603)(641.32808105,47.43439941)
\lineto(641.35808105,47.25439941)
\curveto(641.37807908,47.15439639)(641.39307907,47.05439649)(641.40308105,46.95439941)
\curveto(641.42307904,46.86439668)(641.45307901,46.77939677)(641.49308105,46.69939941)
\curveto(641.59307887,46.45939709)(641.70807875,46.23439731)(641.83808105,46.02439941)
\curveto(641.97807848,45.81439773)(642.14807831,45.63939791)(642.34808105,45.49939941)
\curveto(642.39807806,45.46939808)(642.44307802,45.4443981)(642.48308105,45.42439941)
\curveto(642.52307794,45.40439814)(642.56807789,45.37939817)(642.61808105,45.34939941)
\curveto(642.69807776,45.29939825)(642.78307768,45.25439829)(642.87308105,45.21439941)
\curveto(642.97307749,45.18439836)(643.07807738,45.15439839)(643.18808105,45.12439941)
\curveto(643.23807722,45.10439844)(643.28307718,45.09439845)(643.32308105,45.09439941)
\curveto(643.37307709,45.10439844)(643.42307704,45.10439844)(643.47308105,45.09439941)
\curveto(643.50307696,45.08439846)(643.5630769,45.07439847)(643.65308105,45.06439941)
\curveto(643.75307671,45.05439849)(643.82807663,45.05939849)(643.87808105,45.07939941)
\curveto(643.91807654,45.08939846)(643.9580765,45.08939846)(643.99808105,45.07939941)
\curveto(644.03807642,45.07939847)(644.07807638,45.08939846)(644.11808105,45.10939941)
\curveto(644.19807626,45.12939842)(644.27807618,45.1443984)(644.35808105,45.15439941)
\curveto(644.43807602,45.17439837)(644.51307595,45.19939835)(644.58308105,45.22939941)
\curveto(644.92307554,45.36939818)(645.19807526,45.56439798)(645.40808105,45.81439941)
\curveto(645.61807484,46.06439748)(645.79307467,46.35939719)(645.93308105,46.69939941)
\curveto(645.98307448,46.81939673)(646.01307445,46.9443966)(646.02308105,47.07439941)
\curveto(646.04307442,47.21439633)(646.07307439,47.35439619)(646.11308105,47.49439941)
}
}
{
\newrgbcolor{curcolor}{0 0 0}
\pscustom[linestyle=none,fillstyle=solid,fillcolor=curcolor]
{
\newpath
\moveto(652.5813623,52.05439941)
\curveto(652.81135751,52.05439149)(652.94135738,51.99439155)(652.9713623,51.87439941)
\curveto(653.00135732,51.76439178)(653.01635731,51.59939195)(653.0163623,51.37939941)
\lineto(653.0163623,51.09439941)
\curveto(653.01635731,51.00439254)(652.99135733,50.92939262)(652.9413623,50.86939941)
\curveto(652.88135744,50.78939276)(652.79635753,50.7443928)(652.6863623,50.73439941)
\curveto(652.57635775,50.73439281)(652.46635786,50.71939283)(652.3563623,50.68939941)
\curveto(652.21635811,50.65939289)(652.08135824,50.62939292)(651.9513623,50.59939941)
\curveto(651.83135849,50.56939298)(651.71635861,50.52939302)(651.6063623,50.47939941)
\curveto(651.31635901,50.3493932)(651.08135924,50.16939338)(650.9013623,49.93939941)
\curveto(650.7213596,49.71939383)(650.56635976,49.46439408)(650.4363623,49.17439941)
\curveto(650.39635993,49.06439448)(650.36635996,48.9493946)(650.3463623,48.82939941)
\curveto(650.32636,48.71939483)(650.30136002,48.60439494)(650.2713623,48.48439941)
\curveto(650.26136006,48.43439511)(650.25636007,48.38439516)(650.2563623,48.33439941)
\curveto(650.26636006,48.28439526)(650.26636006,48.23439531)(650.2563623,48.18439941)
\curveto(650.2263601,48.06439548)(650.21136011,47.92439562)(650.2113623,47.76439941)
\curveto(650.2213601,47.61439593)(650.2263601,47.46939608)(650.2263623,47.32939941)
\lineto(650.2263623,45.48439941)
\lineto(650.2263623,45.13939941)
\curveto(650.2263601,45.01939853)(650.2213601,44.90439864)(650.2113623,44.79439941)
\curveto(650.20136012,44.68439886)(650.19636013,44.58939896)(650.1963623,44.50939941)
\curveto(650.20636012,44.42939912)(650.18636014,44.35939919)(650.1363623,44.29939941)
\curveto(650.08636024,44.22939932)(650.00636032,44.18939936)(649.8963623,44.17939941)
\curveto(649.79636053,44.16939938)(649.68636064,44.16439938)(649.5663623,44.16439941)
\lineto(649.2963623,44.16439941)
\curveto(649.24636108,44.18439936)(649.19636113,44.19939935)(649.1463623,44.20939941)
\curveto(649.10636122,44.22939932)(649.07636125,44.25439929)(649.0563623,44.28439941)
\curveto(649.00636132,44.35439919)(648.97636135,44.43939911)(648.9663623,44.53939941)
\lineto(648.9663623,44.86939941)
\lineto(648.9663623,46.02439941)
\lineto(648.9663623,50.17939941)
\lineto(648.9663623,51.21439941)
\lineto(648.9663623,51.51439941)
\curveto(648.97636135,51.61439193)(649.00636132,51.69939185)(649.0563623,51.76939941)
\curveto(649.08636124,51.80939174)(649.13636119,51.83939171)(649.2063623,51.85939941)
\curveto(649.28636104,51.87939167)(649.37136095,51.88939166)(649.4613623,51.88939941)
\curveto(649.55136077,51.89939165)(649.64136068,51.89939165)(649.7313623,51.88939941)
\curveto(649.8213605,51.87939167)(649.89136043,51.86439168)(649.9413623,51.84439941)
\curveto(650.0213603,51.81439173)(650.07136025,51.75439179)(650.0913623,51.66439941)
\curveto(650.1213602,51.58439196)(650.13636019,51.49439205)(650.1363623,51.39439941)
\lineto(650.1363623,51.09439941)
\curveto(650.13636019,50.99439255)(650.15636017,50.90439264)(650.1963623,50.82439941)
\curveto(650.20636012,50.80439274)(650.21636011,50.78939276)(650.2263623,50.77939941)
\lineto(650.2713623,50.73439941)
\curveto(650.38135994,50.73439281)(650.47135985,50.77939277)(650.5413623,50.86939941)
\curveto(650.61135971,50.96939258)(650.67135965,51.0493925)(650.7213623,51.10939941)
\lineto(650.8113623,51.19939941)
\curveto(650.90135942,51.30939224)(651.0263593,51.42439212)(651.1863623,51.54439941)
\curveto(651.34635898,51.66439188)(651.49635883,51.75439179)(651.6363623,51.81439941)
\curveto(651.7263586,51.86439168)(651.8213585,51.89939165)(651.9213623,51.91939941)
\curveto(652.0213583,51.9493916)(652.1263582,51.97939157)(652.2363623,52.00939941)
\curveto(652.29635803,52.01939153)(652.35635797,52.02439152)(652.4163623,52.02439941)
\curveto(652.47635785,52.03439151)(652.53135779,52.0443915)(652.5813623,52.05439941)
}
}
{
\newrgbcolor{curcolor}{0 0 0}
\pscustom[linestyle=none,fillstyle=solid,fillcolor=curcolor]
{
\newpath
\moveto(87.59249146,83.16295776)
\lineto(87.59249146,82.90795776)
\curveto(87.60248375,82.827953)(87.59748376,82.75295307)(87.57749146,82.68295776)
\lineto(87.57749146,82.44295776)
\lineto(87.57749146,82.27795776)
\curveto(87.5574838,82.17795365)(87.54748381,82.07295375)(87.54749146,81.96295776)
\curveto(87.54748381,81.86295396)(87.53748382,81.76295406)(87.51749146,81.66295776)
\lineto(87.51749146,81.51295776)
\curveto(87.48748387,81.37295445)(87.46748389,81.23295459)(87.45749146,81.09295776)
\curveto(87.44748391,80.96295486)(87.42248393,80.83295499)(87.38249146,80.70295776)
\curveto(87.36248399,80.6229552)(87.34248401,80.53795529)(87.32249146,80.44795776)
\lineto(87.26249146,80.20795776)
\lineto(87.14249146,79.90795776)
\curveto(87.11248424,79.81795601)(87.07748428,79.7279561)(87.03749146,79.63795776)
\curveto(86.93748442,79.41795641)(86.80248455,79.20295662)(86.63249146,78.99295776)
\curveto(86.47248488,78.78295704)(86.29748506,78.61295721)(86.10749146,78.48295776)
\curveto(86.0574853,78.44295738)(85.99748536,78.40295742)(85.92749146,78.36295776)
\curveto(85.86748549,78.33295749)(85.80748555,78.29795753)(85.74749146,78.25795776)
\curveto(85.66748569,78.20795762)(85.57248578,78.16795766)(85.46249146,78.13795776)
\curveto(85.352486,78.10795772)(85.24748611,78.07795775)(85.14749146,78.04795776)
\curveto(85.03748632,78.00795782)(84.92748643,77.98295784)(84.81749146,77.97295776)
\curveto(84.70748665,77.96295786)(84.59248676,77.94795788)(84.47249146,77.92795776)
\curveto(84.43248692,77.91795791)(84.38748697,77.91795791)(84.33749146,77.92795776)
\curveto(84.29748706,77.9279579)(84.2574871,77.9229579)(84.21749146,77.91295776)
\curveto(84.17748718,77.90295792)(84.12248723,77.89795793)(84.05249146,77.89795776)
\curveto(83.98248737,77.89795793)(83.93248742,77.90295792)(83.90249146,77.91295776)
\curveto(83.8524875,77.93295789)(83.80748755,77.93795789)(83.76749146,77.92795776)
\curveto(83.72748763,77.91795791)(83.69248766,77.91795791)(83.66249146,77.92795776)
\lineto(83.57249146,77.92795776)
\curveto(83.51248784,77.94795788)(83.44748791,77.96295786)(83.37749146,77.97295776)
\curveto(83.31748804,77.97295785)(83.2524881,77.97795785)(83.18249146,77.98795776)
\curveto(83.01248834,78.03795779)(82.8524885,78.08795774)(82.70249146,78.13795776)
\curveto(82.5524888,78.18795764)(82.40748895,78.25295757)(82.26749146,78.33295776)
\curveto(82.21748914,78.37295745)(82.16248919,78.40295742)(82.10249146,78.42295776)
\curveto(82.0524893,78.45295737)(82.00248935,78.48795734)(81.95249146,78.52795776)
\curveto(81.71248964,78.70795712)(81.51248984,78.9279569)(81.35249146,79.18795776)
\curveto(81.19249016,79.44795638)(81.0524903,79.73295609)(80.93249146,80.04295776)
\curveto(80.87249048,80.18295564)(80.82749053,80.3229555)(80.79749146,80.46295776)
\curveto(80.76749059,80.61295521)(80.73249062,80.76795506)(80.69249146,80.92795776)
\curveto(80.67249068,81.03795479)(80.6574907,81.14795468)(80.64749146,81.25795776)
\curveto(80.63749072,81.36795446)(80.62249073,81.47795435)(80.60249146,81.58795776)
\curveto(80.59249076,81.6279542)(80.58749077,81.66795416)(80.58749146,81.70795776)
\curveto(80.59749076,81.74795408)(80.59749076,81.78795404)(80.58749146,81.82795776)
\curveto(80.57749078,81.87795395)(80.57249078,81.9279539)(80.57249146,81.97795776)
\lineto(80.57249146,82.14295776)
\curveto(80.5524908,82.19295363)(80.54749081,82.24295358)(80.55749146,82.29295776)
\curveto(80.56749079,82.35295347)(80.56749079,82.40795342)(80.55749146,82.45795776)
\curveto(80.54749081,82.49795333)(80.54749081,82.54295328)(80.55749146,82.59295776)
\curveto(80.56749079,82.64295318)(80.56249079,82.69295313)(80.54249146,82.74295776)
\curveto(80.52249083,82.81295301)(80.51749084,82.88795294)(80.52749146,82.96795776)
\curveto(80.53749082,83.05795277)(80.54249081,83.14295268)(80.54249146,83.22295776)
\curveto(80.54249081,83.31295251)(80.53749082,83.41295241)(80.52749146,83.52295776)
\curveto(80.51749084,83.64295218)(80.52249083,83.74295208)(80.54249146,83.82295776)
\lineto(80.54249146,84.10795776)
\lineto(80.58749146,84.73795776)
\curveto(80.59749076,84.83795099)(80.60749075,84.93295089)(80.61749146,85.02295776)
\lineto(80.64749146,85.32295776)
\curveto(80.66749069,85.37295045)(80.67249068,85.4229504)(80.66249146,85.47295776)
\curveto(80.66249069,85.53295029)(80.67249068,85.58795024)(80.69249146,85.63795776)
\curveto(80.74249061,85.80795002)(80.78249057,85.97294985)(80.81249146,86.13295776)
\curveto(80.84249051,86.30294952)(80.89249046,86.46294936)(80.96249146,86.61295776)
\curveto(81.1524902,87.07294875)(81.37248998,87.44794838)(81.62249146,87.73795776)
\curveto(81.88248947,88.0279478)(82.24248911,88.27294755)(82.70249146,88.47295776)
\curveto(82.83248852,88.5229473)(82.96248839,88.55794727)(83.09249146,88.57795776)
\curveto(83.23248812,88.59794723)(83.37248798,88.6229472)(83.51249146,88.65295776)
\curveto(83.58248777,88.66294716)(83.64748771,88.66794716)(83.70749146,88.66795776)
\curveto(83.76748759,88.66794716)(83.83248752,88.67294715)(83.90249146,88.68295776)
\curveto(84.73248662,88.70294712)(85.40248595,88.55294727)(85.91249146,88.23295776)
\curveto(86.42248493,87.9229479)(86.80248455,87.48294834)(87.05249146,86.91295776)
\curveto(87.10248425,86.79294903)(87.14748421,86.66794916)(87.18749146,86.53795776)
\curveto(87.22748413,86.40794942)(87.27248408,86.27294955)(87.32249146,86.13295776)
\curveto(87.34248401,86.05294977)(87.357484,85.96794986)(87.36749146,85.87795776)
\lineto(87.42749146,85.63795776)
\curveto(87.4574839,85.5279503)(87.47248388,85.41795041)(87.47249146,85.30795776)
\curveto(87.48248387,85.19795063)(87.49748386,85.08795074)(87.51749146,84.97795776)
\curveto(87.53748382,84.9279509)(87.54248381,84.88295094)(87.53249146,84.84295776)
\curveto(87.53248382,84.80295102)(87.53748382,84.76295106)(87.54749146,84.72295776)
\curveto(87.5574838,84.67295115)(87.5574838,84.61795121)(87.54749146,84.55795776)
\curveto(87.54748381,84.50795132)(87.5524838,84.45795137)(87.56249146,84.40795776)
\lineto(87.56249146,84.27295776)
\curveto(87.58248377,84.21295161)(87.58248377,84.14295168)(87.56249146,84.06295776)
\curveto(87.5524838,83.99295183)(87.5574838,83.9279519)(87.57749146,83.86795776)
\curveto(87.58748377,83.83795199)(87.59248376,83.79795203)(87.59249146,83.74795776)
\lineto(87.59249146,83.62795776)
\lineto(87.59249146,83.16295776)
\moveto(86.04749146,80.83795776)
\curveto(86.14748521,81.15795467)(86.20748515,81.5229543)(86.22749146,81.93295776)
\curveto(86.24748511,82.34295348)(86.2574851,82.75295307)(86.25749146,83.16295776)
\curveto(86.2574851,83.59295223)(86.24748511,84.01295181)(86.22749146,84.42295776)
\curveto(86.20748515,84.83295099)(86.16248519,85.21795061)(86.09249146,85.57795776)
\curveto(86.02248533,85.93794989)(85.91248544,86.25794957)(85.76249146,86.53795776)
\curveto(85.62248573,86.827949)(85.42748593,87.06294876)(85.17749146,87.24295776)
\curveto(85.01748634,87.35294847)(84.83748652,87.43294839)(84.63749146,87.48295776)
\curveto(84.43748692,87.54294828)(84.19248716,87.57294825)(83.90249146,87.57295776)
\curveto(83.88248747,87.55294827)(83.84748751,87.54294828)(83.79749146,87.54295776)
\curveto(83.74748761,87.55294827)(83.70748765,87.55294827)(83.67749146,87.54295776)
\curveto(83.59748776,87.5229483)(83.52248783,87.50294832)(83.45249146,87.48295776)
\curveto(83.39248796,87.47294835)(83.32748803,87.45294837)(83.25749146,87.42295776)
\curveto(82.98748837,87.30294852)(82.76748859,87.13294869)(82.59749146,86.91295776)
\curveto(82.43748892,86.70294912)(82.30248905,86.45794937)(82.19249146,86.17795776)
\curveto(82.14248921,86.06794976)(82.10248925,85.94794988)(82.07249146,85.81795776)
\curveto(82.0524893,85.69795013)(82.02748933,85.57295025)(81.99749146,85.44295776)
\curveto(81.97748938,85.39295043)(81.96748939,85.33795049)(81.96749146,85.27795776)
\curveto(81.96748939,85.2279506)(81.96248939,85.17795065)(81.95249146,85.12795776)
\curveto(81.94248941,85.03795079)(81.93248942,84.94295088)(81.92249146,84.84295776)
\curveto(81.91248944,84.75295107)(81.90248945,84.65795117)(81.89249146,84.55795776)
\curveto(81.89248946,84.47795135)(81.88748947,84.39295143)(81.87749146,84.30295776)
\lineto(81.87749146,84.06295776)
\lineto(81.87749146,83.88295776)
\curveto(81.86748949,83.85295197)(81.86248949,83.81795201)(81.86249146,83.77795776)
\lineto(81.86249146,83.64295776)
\lineto(81.86249146,83.19295776)
\curveto(81.86248949,83.11295271)(81.8574895,83.0279528)(81.84749146,82.93795776)
\curveto(81.84748951,82.85795297)(81.8574895,82.78295304)(81.87749146,82.71295776)
\lineto(81.87749146,82.44295776)
\curveto(81.87748948,82.4229534)(81.87248948,82.39295343)(81.86249146,82.35295776)
\curveto(81.86248949,82.3229535)(81.86748949,82.29795353)(81.87749146,82.27795776)
\curveto(81.88748947,82.17795365)(81.89248946,82.07795375)(81.89249146,81.97795776)
\curveto(81.90248945,81.88795394)(81.91248944,81.78795404)(81.92249146,81.67795776)
\curveto(81.9524894,81.55795427)(81.96748939,81.43295439)(81.96749146,81.30295776)
\curveto(81.97748938,81.18295464)(82.00248935,81.06795476)(82.04249146,80.95795776)
\curveto(82.12248923,80.65795517)(82.20748915,80.39295543)(82.29749146,80.16295776)
\curveto(82.39748896,79.93295589)(82.54248881,79.71795611)(82.73249146,79.51795776)
\curveto(82.94248841,79.31795651)(83.20748815,79.16795666)(83.52749146,79.06795776)
\curveto(83.56748779,79.04795678)(83.60248775,79.03795679)(83.63249146,79.03795776)
\curveto(83.67248768,79.04795678)(83.71748764,79.04295678)(83.76749146,79.02295776)
\curveto(83.80748755,79.01295681)(83.87748748,79.00295682)(83.97749146,78.99295776)
\curveto(84.08748727,78.98295684)(84.17248718,78.98795684)(84.23249146,79.00795776)
\curveto(84.30248705,79.0279568)(84.37248698,79.03795679)(84.44249146,79.03795776)
\curveto(84.51248684,79.04795678)(84.57748678,79.06295676)(84.63749146,79.08295776)
\curveto(84.83748652,79.14295668)(85.01748634,79.2279566)(85.17749146,79.33795776)
\curveto(85.20748615,79.35795647)(85.23248612,79.37795645)(85.25249146,79.39795776)
\lineto(85.31249146,79.45795776)
\curveto(85.352486,79.47795635)(85.40248595,79.51795631)(85.46249146,79.57795776)
\curveto(85.56248579,79.71795611)(85.64748571,79.84795598)(85.71749146,79.96795776)
\curveto(85.78748557,80.08795574)(85.8574855,80.23295559)(85.92749146,80.40295776)
\curveto(85.9574854,80.47295535)(85.97748538,80.54295528)(85.98749146,80.61295776)
\curveto(86.00748535,80.68295514)(86.02748533,80.75795507)(86.04749146,80.83795776)
}
}
{
\newrgbcolor{curcolor}{0 0 0}
\pscustom[linestyle=none,fillstyle=solid,fillcolor=curcolor]
{
\newpath
\moveto(67.95343292,162.96866333)
\curveto(68.05342807,162.96865271)(68.14842797,162.95865272)(68.23843292,162.93866333)
\curveto(68.32842779,162.92865275)(68.39342773,162.89865278)(68.43343292,162.84866333)
\curveto(68.49342763,162.76865291)(68.5234276,162.66365302)(68.52343292,162.53366333)
\lineto(68.52343292,162.14366333)
\lineto(68.52343292,160.64366333)
\lineto(68.52343292,154.25366333)
\lineto(68.52343292,153.08366333)
\lineto(68.52343292,152.76866333)
\curveto(68.53342759,152.66866301)(68.5184276,152.58866309)(68.47843292,152.52866333)
\curveto(68.42842769,152.44866323)(68.35342777,152.39866328)(68.25343292,152.37866333)
\curveto(68.16342796,152.36866331)(68.05342807,152.36366332)(67.92343292,152.36366333)
\lineto(67.69843292,152.36366333)
\curveto(67.6184285,152.3836633)(67.54842857,152.39866328)(67.48843292,152.40866333)
\curveto(67.42842869,152.42866325)(67.37842874,152.46866321)(67.33843292,152.52866333)
\curveto(67.29842882,152.58866309)(67.27842884,152.66366302)(67.27843292,152.75366333)
\lineto(67.27843292,153.05366333)
\lineto(67.27843292,154.14866333)
\lineto(67.27843292,159.48866333)
\curveto(67.25842886,159.5786561)(67.24342888,159.65365603)(67.23343292,159.71366333)
\curveto(67.23342889,159.7836559)(67.20342892,159.84365584)(67.14343292,159.89366333)
\curveto(67.07342905,159.94365574)(66.98342914,159.96865571)(66.87343292,159.96866333)
\curveto(66.77342935,159.9786557)(66.66342946,159.9836557)(66.54343292,159.98366333)
\lineto(65.40343292,159.98366333)
\lineto(64.90843292,159.98366333)
\curveto(64.74843137,159.99365569)(64.63843148,160.05365563)(64.57843292,160.16366333)
\curveto(64.55843156,160.19365549)(64.54843157,160.22365546)(64.54843292,160.25366333)
\curveto(64.54843157,160.29365539)(64.54343158,160.33865534)(64.53343292,160.38866333)
\curveto(64.51343161,160.50865517)(64.5184316,160.61865506)(64.54843292,160.71866333)
\curveto(64.58843153,160.81865486)(64.64343148,160.88865479)(64.71343292,160.92866333)
\curveto(64.79343133,160.9786547)(64.91343121,161.00365468)(65.07343292,161.00366333)
\curveto(65.23343089,161.00365468)(65.36843075,161.01865466)(65.47843292,161.04866333)
\curveto(65.52843059,161.05865462)(65.58343054,161.06365462)(65.64343292,161.06366333)
\curveto(65.70343042,161.07365461)(65.76343036,161.08865459)(65.82343292,161.10866333)
\curveto(65.97343015,161.15865452)(66.11843,161.20865447)(66.25843292,161.25866333)
\curveto(66.39842972,161.31865436)(66.53342959,161.38865429)(66.66343292,161.46866333)
\curveto(66.80342932,161.55865412)(66.9234292,161.66365402)(67.02343292,161.78366333)
\curveto(67.123429,161.90365378)(67.2184289,162.03365365)(67.30843292,162.17366333)
\curveto(67.36842875,162.27365341)(67.41342871,162.3836533)(67.44343292,162.50366333)
\curveto(67.48342864,162.62365306)(67.53342859,162.72865295)(67.59343292,162.81866333)
\curveto(67.64342848,162.8786528)(67.71342841,162.91865276)(67.80343292,162.93866333)
\curveto(67.8234283,162.94865273)(67.84842827,162.95365273)(67.87843292,162.95366333)
\curveto(67.90842821,162.95365273)(67.93342819,162.95865272)(67.95343292,162.96866333)
}
}
{
\newrgbcolor{curcolor}{0 0 0}
\pscustom[linestyle=none,fillstyle=solid,fillcolor=curcolor]
{
\newpath
\moveto(73.7530423,162.77366333)
\lineto(77.3530423,162.77366333)
\lineto(77.9980423,162.77366333)
\curveto(78.07803577,162.77365291)(78.15303569,162.76865291)(78.2230423,162.75866333)
\curveto(78.29303555,162.75865292)(78.35303549,162.74865293)(78.4030423,162.72866333)
\curveto(78.47303537,162.69865298)(78.52803532,162.63865304)(78.5680423,162.54866333)
\curveto(78.58803526,162.51865316)(78.59803525,162.4786532)(78.5980423,162.42866333)
\lineto(78.5980423,162.29366333)
\curveto(78.60803524,162.1836535)(78.60303524,162.0786536)(78.5830423,161.97866333)
\curveto(78.57303527,161.8786538)(78.53803531,161.80865387)(78.4780423,161.76866333)
\curveto(78.38803546,161.69865398)(78.25303559,161.66365402)(78.0730423,161.66366333)
\curveto(77.89303595,161.67365401)(77.72803612,161.678654)(77.5780423,161.67866333)
\lineto(75.5830423,161.67866333)
\lineto(75.0880423,161.67866333)
\lineto(74.9530423,161.67866333)
\curveto(74.91303893,161.678654)(74.87303897,161.67365401)(74.8330423,161.66366333)
\lineto(74.6230423,161.66366333)
\curveto(74.51303933,161.63365405)(74.43303941,161.59365409)(74.3830423,161.54366333)
\curveto(74.33303951,161.50365418)(74.29803955,161.44865423)(74.2780423,161.37866333)
\curveto(74.25803959,161.31865436)(74.2430396,161.24865443)(74.2330423,161.16866333)
\curveto(74.22303962,161.08865459)(74.20303964,160.99865468)(74.1730423,160.89866333)
\curveto(74.12303972,160.69865498)(74.08303976,160.49365519)(74.0530423,160.28366333)
\curveto(74.02303982,160.07365561)(73.98303986,159.86865581)(73.9330423,159.66866333)
\curveto(73.91303993,159.59865608)(73.90303994,159.52865615)(73.9030423,159.45866333)
\curveto(73.90303994,159.39865628)(73.89303995,159.33365635)(73.8730423,159.26366333)
\curveto(73.86303998,159.23365645)(73.85303999,159.19365649)(73.8430423,159.14366333)
\curveto(73.84304,159.10365658)(73.84804,159.06365662)(73.8580423,159.02366333)
\curveto(73.87803997,158.97365671)(73.90303994,158.92865675)(73.9330423,158.88866333)
\curveto(73.97303987,158.85865682)(74.03303981,158.85365683)(74.1130423,158.87366333)
\curveto(74.17303967,158.89365679)(74.23303961,158.91865676)(74.2930423,158.94866333)
\curveto(74.35303949,158.98865669)(74.41303943,159.02365666)(74.4730423,159.05366333)
\curveto(74.53303931,159.07365661)(74.58303926,159.08865659)(74.6230423,159.09866333)
\curveto(74.81303903,159.1786565)(75.01803883,159.23365645)(75.2380423,159.26366333)
\curveto(75.46803838,159.29365639)(75.69803815,159.30365638)(75.9280423,159.29366333)
\curveto(76.16803768,159.29365639)(76.39803745,159.26865641)(76.6180423,159.21866333)
\curveto(76.83803701,159.1786565)(77.03803681,159.11865656)(77.2180423,159.03866333)
\curveto(77.26803658,159.01865666)(77.31303653,158.99865668)(77.3530423,158.97866333)
\curveto(77.40303644,158.95865672)(77.45303639,158.93365675)(77.5030423,158.90366333)
\curveto(77.85303599,158.69365699)(78.13303571,158.46365722)(78.3430423,158.21366333)
\curveto(78.56303528,157.96365772)(78.75803509,157.63865804)(78.9280423,157.23866333)
\curveto(78.97803487,157.12865855)(79.01303483,157.01865866)(79.0330423,156.90866333)
\curveto(79.05303479,156.79865888)(79.07803477,156.683659)(79.1080423,156.56366333)
\curveto(79.11803473,156.53365915)(79.12303472,156.48865919)(79.1230423,156.42866333)
\curveto(79.1430347,156.36865931)(79.15303469,156.29865938)(79.1530423,156.21866333)
\curveto(79.15303469,156.14865953)(79.16303468,156.0836596)(79.1830423,156.02366333)
\lineto(79.1830423,155.85866333)
\curveto(79.19303465,155.80865987)(79.19803465,155.73865994)(79.1980423,155.64866333)
\curveto(79.19803465,155.55866012)(79.18803466,155.48866019)(79.1680423,155.43866333)
\curveto(79.1480347,155.3786603)(79.1430347,155.31866036)(79.1530423,155.25866333)
\curveto(79.16303468,155.20866047)(79.15803469,155.15866052)(79.1380423,155.10866333)
\curveto(79.09803475,154.94866073)(79.06303478,154.79866088)(79.0330423,154.65866333)
\curveto(79.00303484,154.51866116)(78.95803489,154.3836613)(78.8980423,154.25366333)
\curveto(78.73803511,153.8836618)(78.51803533,153.54866213)(78.2380423,153.24866333)
\curveto(77.95803589,152.94866273)(77.63803621,152.71866296)(77.2780423,152.55866333)
\curveto(77.10803674,152.4786632)(76.90803694,152.40366328)(76.6780423,152.33366333)
\curveto(76.56803728,152.29366339)(76.45303739,152.26866341)(76.3330423,152.25866333)
\curveto(76.21303763,152.24866343)(76.09303775,152.22866345)(75.9730423,152.19866333)
\curveto(75.92303792,152.1786635)(75.86803798,152.1786635)(75.8080423,152.19866333)
\curveto(75.7480381,152.20866347)(75.68803816,152.20366348)(75.6280423,152.18366333)
\curveto(75.52803832,152.16366352)(75.42803842,152.16366352)(75.3280423,152.18366333)
\lineto(75.1930423,152.18366333)
\curveto(75.1430387,152.20366348)(75.08303876,152.21366347)(75.0130423,152.21366333)
\curveto(74.95303889,152.20366348)(74.89803895,152.20866347)(74.8480423,152.22866333)
\curveto(74.80803904,152.23866344)(74.77303907,152.24366344)(74.7430423,152.24366333)
\curveto(74.71303913,152.24366344)(74.67803917,152.24866343)(74.6380423,152.25866333)
\lineto(74.3680423,152.31866333)
\curveto(74.27803957,152.33866334)(74.19303965,152.36866331)(74.1130423,152.40866333)
\curveto(73.77304007,152.54866313)(73.48304036,152.70366298)(73.2430423,152.87366333)
\curveto(73.00304084,153.05366263)(72.78304106,153.2836624)(72.5830423,153.56366333)
\curveto(72.43304141,153.79366189)(72.31804153,154.03366165)(72.2380423,154.28366333)
\curveto(72.21804163,154.33366135)(72.20804164,154.3786613)(72.2080423,154.41866333)
\curveto(72.20804164,154.46866121)(72.19804165,154.51866116)(72.1780423,154.56866333)
\curveto(72.15804169,154.62866105)(72.1430417,154.70866097)(72.1330423,154.80866333)
\curveto(72.13304171,154.90866077)(72.15304169,154.9836607)(72.1930423,155.03366333)
\curveto(72.2430416,155.11366057)(72.32304152,155.15866052)(72.4330423,155.16866333)
\curveto(72.5430413,155.1786605)(72.65804119,155.1836605)(72.7780423,155.18366333)
\lineto(72.9430423,155.18366333)
\curveto(73.00304084,155.1836605)(73.05804079,155.17366051)(73.1080423,155.15366333)
\curveto(73.19804065,155.13366055)(73.26804058,155.09366059)(73.3180423,155.03366333)
\curveto(73.38804046,154.94366074)(73.43304041,154.83366085)(73.4530423,154.70366333)
\curveto(73.48304036,154.5836611)(73.52804032,154.4786612)(73.5880423,154.38866333)
\curveto(73.77804007,154.04866163)(74.03803981,153.7786619)(74.3680423,153.57866333)
\curveto(74.46803938,153.51866216)(74.57303927,153.46866221)(74.6830423,153.42866333)
\curveto(74.80303904,153.39866228)(74.92303892,153.36366232)(75.0430423,153.32366333)
\curveto(75.21303863,153.27366241)(75.41803843,153.25366243)(75.6580423,153.26366333)
\curveto(75.90803794,153.2836624)(76.10803774,153.31866236)(76.2580423,153.36866333)
\curveto(76.62803722,153.48866219)(76.91803693,153.64866203)(77.1280423,153.84866333)
\curveto(77.3480365,154.05866162)(77.52803632,154.33866134)(77.6680423,154.68866333)
\curveto(77.71803613,154.78866089)(77.7480361,154.89366079)(77.7580423,155.00366333)
\curveto(77.77803607,155.11366057)(77.80303604,155.22866045)(77.8330423,155.34866333)
\lineto(77.8330423,155.45366333)
\curveto(77.843036,155.49366019)(77.848036,155.53366015)(77.8480423,155.57366333)
\curveto(77.85803599,155.60366008)(77.85803599,155.63866004)(77.8480423,155.67866333)
\lineto(77.8480423,155.79866333)
\curveto(77.848036,156.05865962)(77.81803603,156.30365938)(77.7580423,156.53366333)
\curveto(77.6480362,156.8836588)(77.49303635,157.1786585)(77.2930423,157.41866333)
\curveto(77.09303675,157.66865801)(76.83303701,157.86365782)(76.5130423,158.00366333)
\lineto(76.3330423,158.06366333)
\curveto(76.28303756,158.0836576)(76.22303762,158.10365758)(76.1530423,158.12366333)
\curveto(76.10303774,158.14365754)(76.0430378,158.15365753)(75.9730423,158.15366333)
\curveto(75.91303793,158.16365752)(75.848038,158.1786575)(75.7780423,158.19866333)
\lineto(75.6280423,158.19866333)
\curveto(75.58803826,158.21865746)(75.53303831,158.22865745)(75.4630423,158.22866333)
\curveto(75.40303844,158.22865745)(75.3480385,158.21865746)(75.2980423,158.19866333)
\lineto(75.1930423,158.19866333)
\curveto(75.16303868,158.19865748)(75.12803872,158.19365749)(75.0880423,158.18366333)
\lineto(74.8480423,158.12366333)
\curveto(74.76803908,158.11365757)(74.68803916,158.09365759)(74.6080423,158.06366333)
\curveto(74.36803948,157.96365772)(74.13803971,157.82865785)(73.9180423,157.65866333)
\curveto(73.82804002,157.58865809)(73.7430401,157.51365817)(73.6630423,157.43366333)
\curveto(73.58304026,157.36365832)(73.48304036,157.30865837)(73.3630423,157.26866333)
\curveto(73.27304057,157.23865844)(73.13304071,157.22865845)(72.9430423,157.23866333)
\curveto(72.76304108,157.24865843)(72.6430412,157.27365841)(72.5830423,157.31366333)
\curveto(72.53304131,157.35365833)(72.49304135,157.41365827)(72.4630423,157.49366333)
\curveto(72.4430414,157.57365811)(72.4430414,157.65865802)(72.4630423,157.74866333)
\curveto(72.49304135,157.86865781)(72.51304133,157.98865769)(72.5230423,158.10866333)
\curveto(72.5430413,158.23865744)(72.56804128,158.36365732)(72.5980423,158.48366333)
\curveto(72.61804123,158.52365716)(72.62304122,158.55865712)(72.6130423,158.58866333)
\curveto(72.61304123,158.62865705)(72.62304122,158.67365701)(72.6430423,158.72366333)
\curveto(72.66304118,158.81365687)(72.67804117,158.90365678)(72.6880423,158.99366333)
\curveto(72.69804115,159.09365659)(72.71804113,159.18865649)(72.7480423,159.27866333)
\curveto(72.75804109,159.33865634)(72.76304108,159.39865628)(72.7630423,159.45866333)
\curveto(72.77304107,159.51865616)(72.78804106,159.5786561)(72.8080423,159.63866333)
\curveto(72.85804099,159.83865584)(72.89304095,160.04365564)(72.9130423,160.25366333)
\curveto(72.9430409,160.47365521)(72.98304086,160.683655)(73.0330423,160.88366333)
\curveto(73.06304078,160.9836547)(73.08304076,161.0836546)(73.0930423,161.18366333)
\curveto(73.10304074,161.2836544)(73.11804073,161.3836543)(73.1380423,161.48366333)
\curveto(73.1480407,161.51365417)(73.15304069,161.55365413)(73.1530423,161.60366333)
\curveto(73.18304066,161.71365397)(73.20304064,161.81865386)(73.2130423,161.91866333)
\curveto(73.23304061,162.02865365)(73.25804059,162.13865354)(73.2880423,162.24866333)
\curveto(73.30804054,162.32865335)(73.32304052,162.39865328)(73.3330423,162.45866333)
\curveto(73.3430405,162.52865315)(73.36804048,162.58865309)(73.4080423,162.63866333)
\curveto(73.42804042,162.66865301)(73.45804039,162.68865299)(73.4980423,162.69866333)
\curveto(73.53804031,162.71865296)(73.58304026,162.73865294)(73.6330423,162.75866333)
\curveto(73.69304015,162.75865292)(73.73304011,162.76365292)(73.7530423,162.77366333)
}
}
{
\newrgbcolor{curcolor}{0 0 0}
\pscustom[linestyle=none,fillstyle=solid,fillcolor=curcolor]
{
\newpath
\moveto(87.59265167,157.44866333)
\lineto(87.59265167,157.19366333)
\curveto(87.60264397,157.11365857)(87.59764397,157.03865864)(87.57765167,156.96866333)
\lineto(87.57765167,156.72866333)
\lineto(87.57765167,156.56366333)
\curveto(87.55764401,156.46365922)(87.54764402,156.35865932)(87.54765167,156.24866333)
\curveto(87.54764402,156.14865953)(87.53764403,156.04865963)(87.51765167,155.94866333)
\lineto(87.51765167,155.79866333)
\curveto(87.48764408,155.65866002)(87.4676441,155.51866016)(87.45765167,155.37866333)
\curveto(87.44764412,155.24866043)(87.42264415,155.11866056)(87.38265167,154.98866333)
\curveto(87.36264421,154.90866077)(87.34264423,154.82366086)(87.32265167,154.73366333)
\lineto(87.26265167,154.49366333)
\lineto(87.14265167,154.19366333)
\curveto(87.11264446,154.10366158)(87.07764449,154.01366167)(87.03765167,153.92366333)
\curveto(86.93764463,153.70366198)(86.80264477,153.48866219)(86.63265167,153.27866333)
\curveto(86.4726451,153.06866261)(86.29764527,152.89866278)(86.10765167,152.76866333)
\curveto(86.05764551,152.72866295)(85.99764557,152.68866299)(85.92765167,152.64866333)
\curveto(85.8676457,152.61866306)(85.80764576,152.5836631)(85.74765167,152.54366333)
\curveto(85.6676459,152.49366319)(85.572646,152.45366323)(85.46265167,152.42366333)
\curveto(85.35264622,152.39366329)(85.24764632,152.36366332)(85.14765167,152.33366333)
\curveto(85.03764653,152.29366339)(84.92764664,152.26866341)(84.81765167,152.25866333)
\curveto(84.70764686,152.24866343)(84.59264698,152.23366345)(84.47265167,152.21366333)
\curveto(84.43264714,152.20366348)(84.38764718,152.20366348)(84.33765167,152.21366333)
\curveto(84.29764727,152.21366347)(84.25764731,152.20866347)(84.21765167,152.19866333)
\curveto(84.17764739,152.18866349)(84.12264745,152.1836635)(84.05265167,152.18366333)
\curveto(83.98264759,152.1836635)(83.93264764,152.18866349)(83.90265167,152.19866333)
\curveto(83.85264772,152.21866346)(83.80764776,152.22366346)(83.76765167,152.21366333)
\curveto(83.72764784,152.20366348)(83.69264788,152.20366348)(83.66265167,152.21366333)
\lineto(83.57265167,152.21366333)
\curveto(83.51264806,152.23366345)(83.44764812,152.24866343)(83.37765167,152.25866333)
\curveto(83.31764825,152.25866342)(83.25264832,152.26366342)(83.18265167,152.27366333)
\curveto(83.01264856,152.32366336)(82.85264872,152.37366331)(82.70265167,152.42366333)
\curveto(82.55264902,152.47366321)(82.40764916,152.53866314)(82.26765167,152.61866333)
\curveto(82.21764935,152.65866302)(82.16264941,152.68866299)(82.10265167,152.70866333)
\curveto(82.05264952,152.73866294)(82.00264957,152.77366291)(81.95265167,152.81366333)
\curveto(81.71264986,152.99366269)(81.51265006,153.21366247)(81.35265167,153.47366333)
\curveto(81.19265038,153.73366195)(81.05265052,154.01866166)(80.93265167,154.32866333)
\curveto(80.8726507,154.46866121)(80.82765074,154.60866107)(80.79765167,154.74866333)
\curveto(80.7676508,154.89866078)(80.73265084,155.05366063)(80.69265167,155.21366333)
\curveto(80.6726509,155.32366036)(80.65765091,155.43366025)(80.64765167,155.54366333)
\curveto(80.63765093,155.65366003)(80.62265095,155.76365992)(80.60265167,155.87366333)
\curveto(80.59265098,155.91365977)(80.58765098,155.95365973)(80.58765167,155.99366333)
\curveto(80.59765097,156.03365965)(80.59765097,156.07365961)(80.58765167,156.11366333)
\curveto(80.57765099,156.16365952)(80.572651,156.21365947)(80.57265167,156.26366333)
\lineto(80.57265167,156.42866333)
\curveto(80.55265102,156.4786592)(80.54765102,156.52865915)(80.55765167,156.57866333)
\curveto(80.567651,156.63865904)(80.567651,156.69365899)(80.55765167,156.74366333)
\curveto(80.54765102,156.7836589)(80.54765102,156.82865885)(80.55765167,156.87866333)
\curveto(80.567651,156.92865875)(80.56265101,156.9786587)(80.54265167,157.02866333)
\curveto(80.52265105,157.09865858)(80.51765105,157.17365851)(80.52765167,157.25366333)
\curveto(80.53765103,157.34365834)(80.54265103,157.42865825)(80.54265167,157.50866333)
\curveto(80.54265103,157.59865808)(80.53765103,157.69865798)(80.52765167,157.80866333)
\curveto(80.51765105,157.92865775)(80.52265105,158.02865765)(80.54265167,158.10866333)
\lineto(80.54265167,158.39366333)
\lineto(80.58765167,159.02366333)
\curveto(80.59765097,159.12365656)(80.60765096,159.21865646)(80.61765167,159.30866333)
\lineto(80.64765167,159.60866333)
\curveto(80.6676509,159.65865602)(80.6726509,159.70865597)(80.66265167,159.75866333)
\curveto(80.66265091,159.81865586)(80.6726509,159.87365581)(80.69265167,159.92366333)
\curveto(80.74265083,160.09365559)(80.78265079,160.25865542)(80.81265167,160.41866333)
\curveto(80.84265073,160.58865509)(80.89265068,160.74865493)(80.96265167,160.89866333)
\curveto(81.15265042,161.35865432)(81.3726502,161.73365395)(81.62265167,162.02366333)
\curveto(81.88264969,162.31365337)(82.24264933,162.55865312)(82.70265167,162.75866333)
\curveto(82.83264874,162.80865287)(82.96264861,162.84365284)(83.09265167,162.86366333)
\curveto(83.23264834,162.8836528)(83.3726482,162.90865277)(83.51265167,162.93866333)
\curveto(83.58264799,162.94865273)(83.64764792,162.95365273)(83.70765167,162.95366333)
\curveto(83.7676478,162.95365273)(83.83264774,162.95865272)(83.90265167,162.96866333)
\curveto(84.73264684,162.98865269)(85.40264617,162.83865284)(85.91265167,162.51866333)
\curveto(86.42264515,162.20865347)(86.80264477,161.76865391)(87.05265167,161.19866333)
\curveto(87.10264447,161.0786546)(87.14764442,160.95365473)(87.18765167,160.82366333)
\curveto(87.22764434,160.69365499)(87.2726443,160.55865512)(87.32265167,160.41866333)
\curveto(87.34264423,160.33865534)(87.35764421,160.25365543)(87.36765167,160.16366333)
\lineto(87.42765167,159.92366333)
\curveto(87.45764411,159.81365587)(87.4726441,159.70365598)(87.47265167,159.59366333)
\curveto(87.48264409,159.4836562)(87.49764407,159.37365631)(87.51765167,159.26366333)
\curveto(87.53764403,159.21365647)(87.54264403,159.16865651)(87.53265167,159.12866333)
\curveto(87.53264404,159.08865659)(87.53764403,159.04865663)(87.54765167,159.00866333)
\curveto(87.55764401,158.95865672)(87.55764401,158.90365678)(87.54765167,158.84366333)
\curveto(87.54764402,158.79365689)(87.55264402,158.74365694)(87.56265167,158.69366333)
\lineto(87.56265167,158.55866333)
\curveto(87.58264399,158.49865718)(87.58264399,158.42865725)(87.56265167,158.34866333)
\curveto(87.55264402,158.2786574)(87.55764401,158.21365747)(87.57765167,158.15366333)
\curveto(87.58764398,158.12365756)(87.59264398,158.0836576)(87.59265167,158.03366333)
\lineto(87.59265167,157.91366333)
\lineto(87.59265167,157.44866333)
\moveto(86.04765167,155.12366333)
\curveto(86.14764542,155.44366024)(86.20764536,155.80865987)(86.22765167,156.21866333)
\curveto(86.24764532,156.62865905)(86.25764531,157.03865864)(86.25765167,157.44866333)
\curveto(86.25764531,157.8786578)(86.24764532,158.29865738)(86.22765167,158.70866333)
\curveto(86.20764536,159.11865656)(86.16264541,159.50365618)(86.09265167,159.86366333)
\curveto(86.02264555,160.22365546)(85.91264566,160.54365514)(85.76265167,160.82366333)
\curveto(85.62264595,161.11365457)(85.42764614,161.34865433)(85.17765167,161.52866333)
\curveto(85.01764655,161.63865404)(84.83764673,161.71865396)(84.63765167,161.76866333)
\curveto(84.43764713,161.82865385)(84.19264738,161.85865382)(83.90265167,161.85866333)
\curveto(83.88264769,161.83865384)(83.84764772,161.82865385)(83.79765167,161.82866333)
\curveto(83.74764782,161.83865384)(83.70764786,161.83865384)(83.67765167,161.82866333)
\curveto(83.59764797,161.80865387)(83.52264805,161.78865389)(83.45265167,161.76866333)
\curveto(83.39264818,161.75865392)(83.32764824,161.73865394)(83.25765167,161.70866333)
\curveto(82.98764858,161.58865409)(82.7676488,161.41865426)(82.59765167,161.19866333)
\curveto(82.43764913,160.98865469)(82.30264927,160.74365494)(82.19265167,160.46366333)
\curveto(82.14264943,160.35365533)(82.10264947,160.23365545)(82.07265167,160.10366333)
\curveto(82.05264952,159.9836557)(82.02764954,159.85865582)(81.99765167,159.72866333)
\curveto(81.97764959,159.678656)(81.9676496,159.62365606)(81.96765167,159.56366333)
\curveto(81.9676496,159.51365617)(81.96264961,159.46365622)(81.95265167,159.41366333)
\curveto(81.94264963,159.32365636)(81.93264964,159.22865645)(81.92265167,159.12866333)
\curveto(81.91264966,159.03865664)(81.90264967,158.94365674)(81.89265167,158.84366333)
\curveto(81.89264968,158.76365692)(81.88764968,158.678657)(81.87765167,158.58866333)
\lineto(81.87765167,158.34866333)
\lineto(81.87765167,158.16866333)
\curveto(81.8676497,158.13865754)(81.86264971,158.10365758)(81.86265167,158.06366333)
\lineto(81.86265167,157.92866333)
\lineto(81.86265167,157.47866333)
\curveto(81.86264971,157.39865828)(81.85764971,157.31365837)(81.84765167,157.22366333)
\curveto(81.84764972,157.14365854)(81.85764971,157.06865861)(81.87765167,156.99866333)
\lineto(81.87765167,156.72866333)
\curveto(81.87764969,156.70865897)(81.8726497,156.678659)(81.86265167,156.63866333)
\curveto(81.86264971,156.60865907)(81.8676497,156.5836591)(81.87765167,156.56366333)
\curveto(81.88764968,156.46365922)(81.89264968,156.36365932)(81.89265167,156.26366333)
\curveto(81.90264967,156.17365951)(81.91264966,156.07365961)(81.92265167,155.96366333)
\curveto(81.95264962,155.84365984)(81.9676496,155.71865996)(81.96765167,155.58866333)
\curveto(81.97764959,155.46866021)(82.00264957,155.35366033)(82.04265167,155.24366333)
\curveto(82.12264945,154.94366074)(82.20764936,154.678661)(82.29765167,154.44866333)
\curveto(82.39764917,154.21866146)(82.54264903,154.00366168)(82.73265167,153.80366333)
\curveto(82.94264863,153.60366208)(83.20764836,153.45366223)(83.52765167,153.35366333)
\curveto(83.567648,153.33366235)(83.60264797,153.32366236)(83.63265167,153.32366333)
\curveto(83.6726479,153.33366235)(83.71764785,153.32866235)(83.76765167,153.30866333)
\curveto(83.80764776,153.29866238)(83.87764769,153.28866239)(83.97765167,153.27866333)
\curveto(84.08764748,153.26866241)(84.1726474,153.27366241)(84.23265167,153.29366333)
\curveto(84.30264727,153.31366237)(84.3726472,153.32366236)(84.44265167,153.32366333)
\curveto(84.51264706,153.33366235)(84.57764699,153.34866233)(84.63765167,153.36866333)
\curveto(84.83764673,153.42866225)(85.01764655,153.51366217)(85.17765167,153.62366333)
\curveto(85.20764636,153.64366204)(85.23264634,153.66366202)(85.25265167,153.68366333)
\lineto(85.31265167,153.74366333)
\curveto(85.35264622,153.76366192)(85.40264617,153.80366188)(85.46265167,153.86366333)
\curveto(85.56264601,154.00366168)(85.64764592,154.13366155)(85.71765167,154.25366333)
\curveto(85.78764578,154.37366131)(85.85764571,154.51866116)(85.92765167,154.68866333)
\curveto(85.95764561,154.75866092)(85.97764559,154.82866085)(85.98765167,154.89866333)
\curveto(86.00764556,154.96866071)(86.02764554,155.04366064)(86.04765167,155.12366333)
}
}
{
\newrgbcolor{curcolor}{0 0 0}
\pscustom[linestyle=none,fillstyle=solid,fillcolor=curcolor]
{
\newpath
\moveto(67.00843292,236.89723694)
\curveto(68.63842748,236.92722629)(69.68842643,236.37222684)(70.15843292,235.23223694)
\curveto(70.25842586,235.00222821)(70.3234258,234.7122285)(70.35343292,234.36223694)
\curveto(70.39342573,234.02222919)(70.36842575,233.7122295)(70.27843292,233.43223694)
\curveto(70.18842593,233.17223004)(70.06842605,232.94723027)(69.91843292,232.75723694)
\curveto(69.89842622,232.7172305)(69.87342625,232.68223053)(69.84343292,232.65223694)
\curveto(69.81342631,232.63223058)(69.78842633,232.60723061)(69.76843292,232.57723694)
\lineto(69.67843292,232.45723694)
\curveto(69.64842647,232.42723079)(69.61342651,232.40223081)(69.57343292,232.38223694)
\curveto(69.5234266,232.33223088)(69.46842665,232.28723093)(69.40843292,232.24723694)
\curveto(69.35842676,232.20723101)(69.31342681,232.15723106)(69.27343292,232.09723694)
\curveto(69.23342689,232.05723116)(69.2184269,232.00723121)(69.22843292,231.94723694)
\curveto(69.23842688,231.89723132)(69.26842685,231.85223136)(69.31843292,231.81223694)
\curveto(69.36842675,231.77223144)(69.4234267,231.73223148)(69.48343292,231.69223694)
\curveto(69.55342657,231.66223155)(69.6184265,231.63223158)(69.67843292,231.60223694)
\curveto(69.73842638,231.57223164)(69.78842633,231.54223167)(69.82843292,231.51223694)
\curveto(70.14842597,231.29223192)(70.40342572,230.98223223)(70.59343292,230.58223694)
\curveto(70.63342549,230.49223272)(70.66342546,230.39723282)(70.68343292,230.29723694)
\curveto(70.71342541,230.20723301)(70.73842538,230.1172331)(70.75843292,230.02723694)
\curveto(70.76842535,229.97723324)(70.77342535,229.92723329)(70.77343292,229.87723694)
\curveto(70.78342534,229.83723338)(70.79342533,229.79223342)(70.80343292,229.74223694)
\curveto(70.81342531,229.69223352)(70.81342531,229.64223357)(70.80343292,229.59223694)
\curveto(70.79342533,229.54223367)(70.79842532,229.49223372)(70.81843292,229.44223694)
\curveto(70.82842529,229.39223382)(70.83342529,229.33223388)(70.83343292,229.26223694)
\curveto(70.83342529,229.19223402)(70.8234253,229.13223408)(70.80343292,229.08223694)
\lineto(70.80343292,228.85723694)
\lineto(70.74343292,228.61723694)
\curveto(70.73342539,228.54723467)(70.7184254,228.47723474)(70.69843292,228.40723694)
\curveto(70.66842545,228.3172349)(70.63842548,228.23223498)(70.60843292,228.15223694)
\curveto(70.58842553,228.07223514)(70.55842556,227.99223522)(70.51843292,227.91223694)
\curveto(70.49842562,227.85223536)(70.46842565,227.79223542)(70.42843292,227.73223694)
\curveto(70.39842572,227.68223553)(70.36342576,227.63223558)(70.32343292,227.58223694)
\curveto(70.123426,227.27223594)(69.87342625,227.0122362)(69.57343292,226.80223694)
\curveto(69.27342685,226.60223661)(68.92842719,226.43723678)(68.53843292,226.30723694)
\curveto(68.4184277,226.26723695)(68.28842783,226.24223697)(68.14843292,226.23223694)
\curveto(68.0184281,226.212237)(67.88342824,226.18723703)(67.74343292,226.15723694)
\curveto(67.67342845,226.14723707)(67.60342852,226.14223707)(67.53343292,226.14223694)
\curveto(67.47342865,226.14223707)(67.40842871,226.13723708)(67.33843292,226.12723694)
\curveto(67.29842882,226.1172371)(67.23842888,226.1122371)(67.15843292,226.11223694)
\curveto(67.08842903,226.1122371)(67.03842908,226.1172371)(67.00843292,226.12723694)
\curveto(66.95842916,226.13723708)(66.91342921,226.14223707)(66.87343292,226.14223694)
\lineto(66.75343292,226.14223694)
\curveto(66.65342947,226.16223705)(66.55342957,226.17723704)(66.45343292,226.18723694)
\curveto(66.35342977,226.19723702)(66.25842986,226.212237)(66.16843292,226.23223694)
\curveto(66.05843006,226.26223695)(65.94843017,226.28723693)(65.83843292,226.30723694)
\curveto(65.73843038,226.33723688)(65.63343049,226.37723684)(65.52343292,226.42723694)
\curveto(65.15343097,226.58723663)(64.83843128,226.78723643)(64.57843292,227.02723694)
\curveto(64.3184318,227.27723594)(64.10843201,227.58723563)(63.94843292,227.95723694)
\curveto(63.90843221,228.04723517)(63.87343225,228.14223507)(63.84343292,228.24223694)
\curveto(63.81343231,228.34223487)(63.78343234,228.44723477)(63.75343292,228.55723694)
\curveto(63.73343239,228.60723461)(63.7234324,228.65723456)(63.72343292,228.70723694)
\curveto(63.7234324,228.76723445)(63.71343241,228.82723439)(63.69343292,228.88723694)
\curveto(63.67343245,228.94723427)(63.66343246,229.02723419)(63.66343292,229.12723694)
\curveto(63.66343246,229.22723399)(63.67843244,229.30223391)(63.70843292,229.35223694)
\curveto(63.7184324,229.38223383)(63.73343239,229.40723381)(63.75343292,229.42723694)
\lineto(63.81343292,229.48723694)
\curveto(63.85343227,229.50723371)(63.91343221,229.52223369)(63.99343292,229.53223694)
\curveto(64.08343204,229.54223367)(64.17343195,229.54723367)(64.26343292,229.54723694)
\curveto(64.35343177,229.54723367)(64.43843168,229.54223367)(64.51843292,229.53223694)
\curveto(64.60843151,229.52223369)(64.67343145,229.5122337)(64.71343292,229.50223694)
\curveto(64.73343139,229.48223373)(64.75343137,229.46723375)(64.77343292,229.45723694)
\curveto(64.79343133,229.45723376)(64.81343131,229.44723377)(64.83343292,229.42723694)
\curveto(64.90343122,229.33723388)(64.94343118,229.22223399)(64.95343292,229.08223694)
\curveto(64.97343115,228.94223427)(65.00343112,228.8172344)(65.04343292,228.70723694)
\lineto(65.19343292,228.34723694)
\curveto(65.24343088,228.23723498)(65.30843081,228.13223508)(65.38843292,228.03223694)
\curveto(65.40843071,228.00223521)(65.42843069,227.97723524)(65.44843292,227.95723694)
\curveto(65.47843064,227.93723528)(65.50343062,227.9122353)(65.52343292,227.88223694)
\curveto(65.56343056,227.82223539)(65.59843052,227.77723544)(65.62843292,227.74723694)
\curveto(65.66843045,227.7172355)(65.70343042,227.68723553)(65.73343292,227.65723694)
\curveto(65.77343035,227.62723559)(65.8184303,227.59723562)(65.86843292,227.56723694)
\curveto(65.95843016,227.50723571)(66.05343007,227.45723576)(66.15343292,227.41723694)
\lineto(66.48343292,227.29723694)
\curveto(66.63342949,227.24723597)(66.83342929,227.217236)(67.08343292,227.20723694)
\curveto(67.33342879,227.19723602)(67.54342858,227.217236)(67.71343292,227.26723694)
\curveto(67.79342833,227.28723593)(67.86342826,227.30223591)(67.92343292,227.31223694)
\lineto(68.13343292,227.37223694)
\curveto(68.41342771,227.49223572)(68.65342747,227.64223557)(68.85343292,227.82223694)
\curveto(69.06342706,228.00223521)(69.22842689,228.23223498)(69.34843292,228.51223694)
\curveto(69.37842674,228.58223463)(69.39842672,228.65223456)(69.40843292,228.72223694)
\lineto(69.46843292,228.96223694)
\curveto(69.50842661,229.10223411)(69.5184266,229.26223395)(69.49843292,229.44223694)
\curveto(69.47842664,229.63223358)(69.44842667,229.78223343)(69.40843292,229.89223694)
\curveto(69.27842684,230.27223294)(69.09342703,230.56223265)(68.85343292,230.76223694)
\curveto(68.6234275,230.96223225)(68.31342781,231.12223209)(67.92343292,231.24223694)
\curveto(67.81342831,231.27223194)(67.69342843,231.29223192)(67.56343292,231.30223694)
\curveto(67.44342868,231.3122319)(67.3184288,231.3172319)(67.18843292,231.31723694)
\curveto(67.02842909,231.3172319)(66.88842923,231.32223189)(66.76843292,231.33223694)
\curveto(66.64842947,231.34223187)(66.56342956,231.40223181)(66.51343292,231.51223694)
\curveto(66.49342963,231.54223167)(66.48342964,231.57723164)(66.48343292,231.61723694)
\lineto(66.48343292,231.75223694)
\curveto(66.47342965,231.85223136)(66.47342965,231.94723127)(66.48343292,232.03723694)
\curveto(66.50342962,232.12723109)(66.54342958,232.19223102)(66.60343292,232.23223694)
\curveto(66.64342948,232.26223095)(66.68342944,232.28223093)(66.72343292,232.29223694)
\curveto(66.77342935,232.30223091)(66.82842929,232.3122309)(66.88843292,232.32223694)
\curveto(66.90842921,232.33223088)(66.93342919,232.33223088)(66.96343292,232.32223694)
\curveto(66.99342913,232.32223089)(67.0184291,232.32723089)(67.03843292,232.33723694)
\lineto(67.17343292,232.33723694)
\curveto(67.28342884,232.35723086)(67.38342874,232.36723085)(67.47343292,232.36723694)
\curveto(67.57342855,232.37723084)(67.66842845,232.39723082)(67.75843292,232.42723694)
\curveto(68.07842804,232.53723068)(68.33342779,232.68223053)(68.52343292,232.86223694)
\curveto(68.71342741,233.04223017)(68.86342726,233.29222992)(68.97343292,233.61223694)
\curveto(69.00342712,233.7122295)(69.0234271,233.83722938)(69.03343292,233.98723694)
\curveto(69.05342707,234.14722907)(69.04842707,234.29222892)(69.01843292,234.42223694)
\curveto(68.99842712,234.49222872)(68.97842714,234.55722866)(68.95843292,234.61723694)
\curveto(68.94842717,234.68722853)(68.92842719,234.75222846)(68.89843292,234.81223694)
\curveto(68.79842732,235.05222816)(68.65342747,235.24222797)(68.46343292,235.38223694)
\curveto(68.27342785,235.52222769)(68.04842807,235.63222758)(67.78843292,235.71223694)
\curveto(67.72842839,235.73222748)(67.66842845,235.74222747)(67.60843292,235.74223694)
\curveto(67.54842857,235.74222747)(67.48342864,235.75222746)(67.41343292,235.77223694)
\curveto(67.33342879,235.79222742)(67.23842888,235.80222741)(67.12843292,235.80223694)
\curveto(67.0184291,235.80222741)(66.9234292,235.79222742)(66.84343292,235.77223694)
\curveto(66.79342933,235.75222746)(66.74342938,235.74222747)(66.69343292,235.74223694)
\curveto(66.65342947,235.74222747)(66.60842951,235.73222748)(66.55843292,235.71223694)
\curveto(66.37842974,235.66222755)(66.20842991,235.58722763)(66.04843292,235.48723694)
\curveto(65.89843022,235.39722782)(65.76843035,235.28222793)(65.65843292,235.14223694)
\curveto(65.56843055,235.02222819)(65.48843063,234.89222832)(65.41843292,234.75223694)
\curveto(65.34843077,234.6122286)(65.28343084,234.45722876)(65.22343292,234.28723694)
\curveto(65.19343093,234.17722904)(65.17343095,234.05722916)(65.16343292,233.92723694)
\curveto(65.15343097,233.80722941)(65.118431,233.70722951)(65.05843292,233.62723694)
\curveto(65.03843108,233.58722963)(64.97843114,233.54722967)(64.87843292,233.50723694)
\curveto(64.83843128,233.49722972)(64.77843134,233.48722973)(64.69843292,233.47723694)
\lineto(64.44343292,233.47723694)
\curveto(64.35343177,233.48722973)(64.26843185,233.49722972)(64.18843292,233.50723694)
\curveto(64.118432,233.5172297)(64.06843205,233.53222968)(64.03843292,233.55223694)
\curveto(63.99843212,233.58222963)(63.96343216,233.63722958)(63.93343292,233.71723694)
\curveto(63.90343222,233.79722942)(63.89843222,233.88222933)(63.91843292,233.97223694)
\curveto(63.92843219,234.02222919)(63.93343219,234.07222914)(63.93343292,234.12223694)
\lineto(63.96343292,234.30223694)
\curveto(63.99343213,234.40222881)(64.0184321,234.50222871)(64.03843292,234.60223694)
\curveto(64.06843205,234.70222851)(64.10343202,234.79222842)(64.14343292,234.87223694)
\curveto(64.19343193,234.98222823)(64.23843188,235.08722813)(64.27843292,235.18723694)
\curveto(64.3184318,235.29722792)(64.36843175,235.40222781)(64.42843292,235.50223694)
\curveto(64.75843136,236.04222717)(65.22843089,236.43722678)(65.83843292,236.68723694)
\curveto(65.95843016,236.73722648)(66.08343004,236.77222644)(66.21343292,236.79223694)
\curveto(66.35342977,236.8122264)(66.49342963,236.83722638)(66.63343292,236.86723694)
\curveto(66.69342943,236.87722634)(66.75342937,236.88222633)(66.81343292,236.88223694)
\curveto(66.88342924,236.88222633)(66.94842917,236.88722633)(67.00843292,236.89723694)
}
}
{
\newrgbcolor{curcolor}{0 0 0}
\pscustom[linestyle=none,fillstyle=solid,fillcolor=curcolor]
{
\newpath
\moveto(79.2430423,231.37723694)
\lineto(79.2430423,231.12223694)
\curveto(79.25303459,231.04223217)(79.2480346,230.96723225)(79.2280423,230.89723694)
\lineto(79.2280423,230.65723694)
\lineto(79.2280423,230.49223694)
\curveto(79.20803464,230.39223282)(79.19803465,230.28723293)(79.1980423,230.17723694)
\curveto(79.19803465,230.07723314)(79.18803466,229.97723324)(79.1680423,229.87723694)
\lineto(79.1680423,229.72723694)
\curveto(79.13803471,229.58723363)(79.11803473,229.44723377)(79.1080423,229.30723694)
\curveto(79.09803475,229.17723404)(79.07303477,229.04723417)(79.0330423,228.91723694)
\curveto(79.01303483,228.83723438)(78.99303485,228.75223446)(78.9730423,228.66223694)
\lineto(78.9130423,228.42223694)
\lineto(78.7930423,228.12223694)
\curveto(78.76303508,228.03223518)(78.72803512,227.94223527)(78.6880423,227.85223694)
\curveto(78.58803526,227.63223558)(78.45303539,227.4172358)(78.2830423,227.20723694)
\curveto(78.12303572,226.99723622)(77.9480359,226.82723639)(77.7580423,226.69723694)
\curveto(77.70803614,226.65723656)(77.6480362,226.6172366)(77.5780423,226.57723694)
\curveto(77.51803633,226.54723667)(77.45803639,226.5122367)(77.3980423,226.47223694)
\curveto(77.31803653,226.42223679)(77.22303662,226.38223683)(77.1130423,226.35223694)
\curveto(77.00303684,226.32223689)(76.89803695,226.29223692)(76.7980423,226.26223694)
\curveto(76.68803716,226.22223699)(76.57803727,226.19723702)(76.4680423,226.18723694)
\curveto(76.35803749,226.17723704)(76.2430376,226.16223705)(76.1230423,226.14223694)
\curveto(76.08303776,226.13223708)(76.03803781,226.13223708)(75.9880423,226.14223694)
\curveto(75.9480379,226.14223707)(75.90803794,226.13723708)(75.8680423,226.12723694)
\curveto(75.82803802,226.1172371)(75.77303807,226.1122371)(75.7030423,226.11223694)
\curveto(75.63303821,226.1122371)(75.58303826,226.1172371)(75.5530423,226.12723694)
\curveto(75.50303834,226.14723707)(75.45803839,226.15223706)(75.4180423,226.14223694)
\curveto(75.37803847,226.13223708)(75.3430385,226.13223708)(75.3130423,226.14223694)
\lineto(75.2230423,226.14223694)
\curveto(75.16303868,226.16223705)(75.09803875,226.17723704)(75.0280423,226.18723694)
\curveto(74.96803888,226.18723703)(74.90303894,226.19223702)(74.8330423,226.20223694)
\curveto(74.66303918,226.25223696)(74.50303934,226.30223691)(74.3530423,226.35223694)
\curveto(74.20303964,226.40223681)(74.05803979,226.46723675)(73.9180423,226.54723694)
\curveto(73.86803998,226.58723663)(73.81304003,226.6172366)(73.7530423,226.63723694)
\curveto(73.70304014,226.66723655)(73.65304019,226.70223651)(73.6030423,226.74223694)
\curveto(73.36304048,226.92223629)(73.16304068,227.14223607)(73.0030423,227.40223694)
\curveto(72.843041,227.66223555)(72.70304114,227.94723527)(72.5830423,228.25723694)
\curveto(72.52304132,228.39723482)(72.47804137,228.53723468)(72.4480423,228.67723694)
\curveto(72.41804143,228.82723439)(72.38304146,228.98223423)(72.3430423,229.14223694)
\curveto(72.32304152,229.25223396)(72.30804154,229.36223385)(72.2980423,229.47223694)
\curveto(72.28804156,229.58223363)(72.27304157,229.69223352)(72.2530423,229.80223694)
\curveto(72.2430416,229.84223337)(72.23804161,229.88223333)(72.2380423,229.92223694)
\curveto(72.2480416,229.96223325)(72.2480416,230.00223321)(72.2380423,230.04223694)
\curveto(72.22804162,230.09223312)(72.22304162,230.14223307)(72.2230423,230.19223694)
\lineto(72.2230423,230.35723694)
\curveto(72.20304164,230.40723281)(72.19804165,230.45723276)(72.2080423,230.50723694)
\curveto(72.21804163,230.56723265)(72.21804163,230.62223259)(72.2080423,230.67223694)
\curveto(72.19804165,230.7122325)(72.19804165,230.75723246)(72.2080423,230.80723694)
\curveto(72.21804163,230.85723236)(72.21304163,230.90723231)(72.1930423,230.95723694)
\curveto(72.17304167,231.02723219)(72.16804168,231.10223211)(72.1780423,231.18223694)
\curveto(72.18804166,231.27223194)(72.19304165,231.35723186)(72.1930423,231.43723694)
\curveto(72.19304165,231.52723169)(72.18804166,231.62723159)(72.1780423,231.73723694)
\curveto(72.16804168,231.85723136)(72.17304167,231.95723126)(72.1930423,232.03723694)
\lineto(72.1930423,232.32223694)
\lineto(72.2380423,232.95223694)
\curveto(72.2480416,233.05223016)(72.25804159,233.14723007)(72.2680423,233.23723694)
\lineto(72.2980423,233.53723694)
\curveto(72.31804153,233.58722963)(72.32304152,233.63722958)(72.3130423,233.68723694)
\curveto(72.31304153,233.74722947)(72.32304152,233.80222941)(72.3430423,233.85223694)
\curveto(72.39304145,234.02222919)(72.43304141,234.18722903)(72.4630423,234.34723694)
\curveto(72.49304135,234.5172287)(72.5430413,234.67722854)(72.6130423,234.82723694)
\curveto(72.80304104,235.28722793)(73.02304082,235.66222755)(73.2730423,235.95223694)
\curveto(73.53304031,236.24222697)(73.89303995,236.48722673)(74.3530423,236.68723694)
\curveto(74.48303936,236.73722648)(74.61303923,236.77222644)(74.7430423,236.79223694)
\curveto(74.88303896,236.8122264)(75.02303882,236.83722638)(75.1630423,236.86723694)
\curveto(75.23303861,236.87722634)(75.29803855,236.88222633)(75.3580423,236.88223694)
\curveto(75.41803843,236.88222633)(75.48303836,236.88722633)(75.5530423,236.89723694)
\curveto(76.38303746,236.9172263)(77.05303679,236.76722645)(77.5630423,236.44723694)
\curveto(78.07303577,236.13722708)(78.45303539,235.69722752)(78.7030423,235.12723694)
\curveto(78.75303509,235.00722821)(78.79803505,234.88222833)(78.8380423,234.75223694)
\curveto(78.87803497,234.62222859)(78.92303492,234.48722873)(78.9730423,234.34723694)
\curveto(78.99303485,234.26722895)(79.00803484,234.18222903)(79.0180423,234.09223694)
\lineto(79.0780423,233.85223694)
\curveto(79.10803474,233.74222947)(79.12303472,233.63222958)(79.1230423,233.52223694)
\curveto(79.13303471,233.4122298)(79.1480347,233.30222991)(79.1680423,233.19223694)
\curveto(79.18803466,233.14223007)(79.19303465,233.09723012)(79.1830423,233.05723694)
\curveto(79.18303466,233.0172302)(79.18803466,232.97723024)(79.1980423,232.93723694)
\curveto(79.20803464,232.88723033)(79.20803464,232.83223038)(79.1980423,232.77223694)
\curveto(79.19803465,232.72223049)(79.20303464,232.67223054)(79.2130423,232.62223694)
\lineto(79.2130423,232.48723694)
\curveto(79.23303461,232.42723079)(79.23303461,232.35723086)(79.2130423,232.27723694)
\curveto(79.20303464,232.20723101)(79.20803464,232.14223107)(79.2280423,232.08223694)
\curveto(79.23803461,232.05223116)(79.2430346,232.0122312)(79.2430423,231.96223694)
\lineto(79.2430423,231.84223694)
\lineto(79.2430423,231.37723694)
\moveto(77.6980423,229.05223694)
\curveto(77.79803605,229.37223384)(77.85803599,229.73723348)(77.8780423,230.14723694)
\curveto(77.89803595,230.55723266)(77.90803594,230.96723225)(77.9080423,231.37723694)
\curveto(77.90803594,231.80723141)(77.89803595,232.22723099)(77.8780423,232.63723694)
\curveto(77.85803599,233.04723017)(77.81303603,233.43222978)(77.7430423,233.79223694)
\curveto(77.67303617,234.15222906)(77.56303628,234.47222874)(77.4130423,234.75223694)
\curveto(77.27303657,235.04222817)(77.07803677,235.27722794)(76.8280423,235.45723694)
\curveto(76.66803718,235.56722765)(76.48803736,235.64722757)(76.2880423,235.69723694)
\curveto(76.08803776,235.75722746)(75.843038,235.78722743)(75.5530423,235.78723694)
\curveto(75.53303831,235.76722745)(75.49803835,235.75722746)(75.4480423,235.75723694)
\curveto(75.39803845,235.76722745)(75.35803849,235.76722745)(75.3280423,235.75723694)
\curveto(75.2480386,235.73722748)(75.17303867,235.7172275)(75.1030423,235.69723694)
\curveto(75.0430388,235.68722753)(74.97803887,235.66722755)(74.9080423,235.63723694)
\curveto(74.63803921,235.5172277)(74.41803943,235.34722787)(74.2480423,235.12723694)
\curveto(74.08803976,234.9172283)(73.95303989,234.67222854)(73.8430423,234.39223694)
\curveto(73.79304005,234.28222893)(73.75304009,234.16222905)(73.7230423,234.03223694)
\curveto(73.70304014,233.9122293)(73.67804017,233.78722943)(73.6480423,233.65723694)
\curveto(73.62804022,233.60722961)(73.61804023,233.55222966)(73.6180423,233.49223694)
\curveto(73.61804023,233.44222977)(73.61304023,233.39222982)(73.6030423,233.34223694)
\curveto(73.59304025,233.25222996)(73.58304026,233.15723006)(73.5730423,233.05723694)
\curveto(73.56304028,232.96723025)(73.55304029,232.87223034)(73.5430423,232.77223694)
\curveto(73.5430403,232.69223052)(73.53804031,232.60723061)(73.5280423,232.51723694)
\lineto(73.5280423,232.27723694)
\lineto(73.5280423,232.09723694)
\curveto(73.51804033,232.06723115)(73.51304033,232.03223118)(73.5130423,231.99223694)
\lineto(73.5130423,231.85723694)
\lineto(73.5130423,231.40723694)
\curveto(73.51304033,231.32723189)(73.50804034,231.24223197)(73.4980423,231.15223694)
\curveto(73.49804035,231.07223214)(73.50804034,230.99723222)(73.5280423,230.92723694)
\lineto(73.5280423,230.65723694)
\curveto(73.52804032,230.63723258)(73.52304032,230.60723261)(73.5130423,230.56723694)
\curveto(73.51304033,230.53723268)(73.51804033,230.5122327)(73.5280423,230.49223694)
\curveto(73.53804031,230.39223282)(73.5430403,230.29223292)(73.5430423,230.19223694)
\curveto(73.55304029,230.10223311)(73.56304028,230.00223321)(73.5730423,229.89223694)
\curveto(73.60304024,229.77223344)(73.61804023,229.64723357)(73.6180423,229.51723694)
\curveto(73.62804022,229.39723382)(73.65304019,229.28223393)(73.6930423,229.17223694)
\curveto(73.77304007,228.87223434)(73.85803999,228.60723461)(73.9480423,228.37723694)
\curveto(74.0480398,228.14723507)(74.19303965,227.93223528)(74.3830423,227.73223694)
\curveto(74.59303925,227.53223568)(74.85803899,227.38223583)(75.1780423,227.28223694)
\curveto(75.21803863,227.26223595)(75.25303859,227.25223596)(75.2830423,227.25223694)
\curveto(75.32303852,227.26223595)(75.36803848,227.25723596)(75.4180423,227.23723694)
\curveto(75.45803839,227.22723599)(75.52803832,227.217236)(75.6280423,227.20723694)
\curveto(75.73803811,227.19723602)(75.82303802,227.20223601)(75.8830423,227.22223694)
\curveto(75.95303789,227.24223597)(76.02303782,227.25223596)(76.0930423,227.25223694)
\curveto(76.16303768,227.26223595)(76.22803762,227.27723594)(76.2880423,227.29723694)
\curveto(76.48803736,227.35723586)(76.66803718,227.44223577)(76.8280423,227.55223694)
\curveto(76.85803699,227.57223564)(76.88303696,227.59223562)(76.9030423,227.61223694)
\lineto(76.9630423,227.67223694)
\curveto(77.00303684,227.69223552)(77.05303679,227.73223548)(77.1130423,227.79223694)
\curveto(77.21303663,227.93223528)(77.29803655,228.06223515)(77.3680423,228.18223694)
\curveto(77.43803641,228.30223491)(77.50803634,228.44723477)(77.5780423,228.61723694)
\curveto(77.60803624,228.68723453)(77.62803622,228.75723446)(77.6380423,228.82723694)
\curveto(77.65803619,228.89723432)(77.67803617,228.97223424)(77.6980423,229.05223694)
}
}
{
\newrgbcolor{curcolor}{0 0 0}
\pscustom[linestyle=none,fillstyle=solid,fillcolor=curcolor]
{
\newpath
\moveto(87.59265167,231.37723694)
\lineto(87.59265167,231.12223694)
\curveto(87.60264397,231.04223217)(87.59764397,230.96723225)(87.57765167,230.89723694)
\lineto(87.57765167,230.65723694)
\lineto(87.57765167,230.49223694)
\curveto(87.55764401,230.39223282)(87.54764402,230.28723293)(87.54765167,230.17723694)
\curveto(87.54764402,230.07723314)(87.53764403,229.97723324)(87.51765167,229.87723694)
\lineto(87.51765167,229.72723694)
\curveto(87.48764408,229.58723363)(87.4676441,229.44723377)(87.45765167,229.30723694)
\curveto(87.44764412,229.17723404)(87.42264415,229.04723417)(87.38265167,228.91723694)
\curveto(87.36264421,228.83723438)(87.34264423,228.75223446)(87.32265167,228.66223694)
\lineto(87.26265167,228.42223694)
\lineto(87.14265167,228.12223694)
\curveto(87.11264446,228.03223518)(87.07764449,227.94223527)(87.03765167,227.85223694)
\curveto(86.93764463,227.63223558)(86.80264477,227.4172358)(86.63265167,227.20723694)
\curveto(86.4726451,226.99723622)(86.29764527,226.82723639)(86.10765167,226.69723694)
\curveto(86.05764551,226.65723656)(85.99764557,226.6172366)(85.92765167,226.57723694)
\curveto(85.8676457,226.54723667)(85.80764576,226.5122367)(85.74765167,226.47223694)
\curveto(85.6676459,226.42223679)(85.572646,226.38223683)(85.46265167,226.35223694)
\curveto(85.35264622,226.32223689)(85.24764632,226.29223692)(85.14765167,226.26223694)
\curveto(85.03764653,226.22223699)(84.92764664,226.19723702)(84.81765167,226.18723694)
\curveto(84.70764686,226.17723704)(84.59264698,226.16223705)(84.47265167,226.14223694)
\curveto(84.43264714,226.13223708)(84.38764718,226.13223708)(84.33765167,226.14223694)
\curveto(84.29764727,226.14223707)(84.25764731,226.13723708)(84.21765167,226.12723694)
\curveto(84.17764739,226.1172371)(84.12264745,226.1122371)(84.05265167,226.11223694)
\curveto(83.98264759,226.1122371)(83.93264764,226.1172371)(83.90265167,226.12723694)
\curveto(83.85264772,226.14723707)(83.80764776,226.15223706)(83.76765167,226.14223694)
\curveto(83.72764784,226.13223708)(83.69264788,226.13223708)(83.66265167,226.14223694)
\lineto(83.57265167,226.14223694)
\curveto(83.51264806,226.16223705)(83.44764812,226.17723704)(83.37765167,226.18723694)
\curveto(83.31764825,226.18723703)(83.25264832,226.19223702)(83.18265167,226.20223694)
\curveto(83.01264856,226.25223696)(82.85264872,226.30223691)(82.70265167,226.35223694)
\curveto(82.55264902,226.40223681)(82.40764916,226.46723675)(82.26765167,226.54723694)
\curveto(82.21764935,226.58723663)(82.16264941,226.6172366)(82.10265167,226.63723694)
\curveto(82.05264952,226.66723655)(82.00264957,226.70223651)(81.95265167,226.74223694)
\curveto(81.71264986,226.92223629)(81.51265006,227.14223607)(81.35265167,227.40223694)
\curveto(81.19265038,227.66223555)(81.05265052,227.94723527)(80.93265167,228.25723694)
\curveto(80.8726507,228.39723482)(80.82765074,228.53723468)(80.79765167,228.67723694)
\curveto(80.7676508,228.82723439)(80.73265084,228.98223423)(80.69265167,229.14223694)
\curveto(80.6726509,229.25223396)(80.65765091,229.36223385)(80.64765167,229.47223694)
\curveto(80.63765093,229.58223363)(80.62265095,229.69223352)(80.60265167,229.80223694)
\curveto(80.59265098,229.84223337)(80.58765098,229.88223333)(80.58765167,229.92223694)
\curveto(80.59765097,229.96223325)(80.59765097,230.00223321)(80.58765167,230.04223694)
\curveto(80.57765099,230.09223312)(80.572651,230.14223307)(80.57265167,230.19223694)
\lineto(80.57265167,230.35723694)
\curveto(80.55265102,230.40723281)(80.54765102,230.45723276)(80.55765167,230.50723694)
\curveto(80.567651,230.56723265)(80.567651,230.62223259)(80.55765167,230.67223694)
\curveto(80.54765102,230.7122325)(80.54765102,230.75723246)(80.55765167,230.80723694)
\curveto(80.567651,230.85723236)(80.56265101,230.90723231)(80.54265167,230.95723694)
\curveto(80.52265105,231.02723219)(80.51765105,231.10223211)(80.52765167,231.18223694)
\curveto(80.53765103,231.27223194)(80.54265103,231.35723186)(80.54265167,231.43723694)
\curveto(80.54265103,231.52723169)(80.53765103,231.62723159)(80.52765167,231.73723694)
\curveto(80.51765105,231.85723136)(80.52265105,231.95723126)(80.54265167,232.03723694)
\lineto(80.54265167,232.32223694)
\lineto(80.58765167,232.95223694)
\curveto(80.59765097,233.05223016)(80.60765096,233.14723007)(80.61765167,233.23723694)
\lineto(80.64765167,233.53723694)
\curveto(80.6676509,233.58722963)(80.6726509,233.63722958)(80.66265167,233.68723694)
\curveto(80.66265091,233.74722947)(80.6726509,233.80222941)(80.69265167,233.85223694)
\curveto(80.74265083,234.02222919)(80.78265079,234.18722903)(80.81265167,234.34723694)
\curveto(80.84265073,234.5172287)(80.89265068,234.67722854)(80.96265167,234.82723694)
\curveto(81.15265042,235.28722793)(81.3726502,235.66222755)(81.62265167,235.95223694)
\curveto(81.88264969,236.24222697)(82.24264933,236.48722673)(82.70265167,236.68723694)
\curveto(82.83264874,236.73722648)(82.96264861,236.77222644)(83.09265167,236.79223694)
\curveto(83.23264834,236.8122264)(83.3726482,236.83722638)(83.51265167,236.86723694)
\curveto(83.58264799,236.87722634)(83.64764792,236.88222633)(83.70765167,236.88223694)
\curveto(83.7676478,236.88222633)(83.83264774,236.88722633)(83.90265167,236.89723694)
\curveto(84.73264684,236.9172263)(85.40264617,236.76722645)(85.91265167,236.44723694)
\curveto(86.42264515,236.13722708)(86.80264477,235.69722752)(87.05265167,235.12723694)
\curveto(87.10264447,235.00722821)(87.14764442,234.88222833)(87.18765167,234.75223694)
\curveto(87.22764434,234.62222859)(87.2726443,234.48722873)(87.32265167,234.34723694)
\curveto(87.34264423,234.26722895)(87.35764421,234.18222903)(87.36765167,234.09223694)
\lineto(87.42765167,233.85223694)
\curveto(87.45764411,233.74222947)(87.4726441,233.63222958)(87.47265167,233.52223694)
\curveto(87.48264409,233.4122298)(87.49764407,233.30222991)(87.51765167,233.19223694)
\curveto(87.53764403,233.14223007)(87.54264403,233.09723012)(87.53265167,233.05723694)
\curveto(87.53264404,233.0172302)(87.53764403,232.97723024)(87.54765167,232.93723694)
\curveto(87.55764401,232.88723033)(87.55764401,232.83223038)(87.54765167,232.77223694)
\curveto(87.54764402,232.72223049)(87.55264402,232.67223054)(87.56265167,232.62223694)
\lineto(87.56265167,232.48723694)
\curveto(87.58264399,232.42723079)(87.58264399,232.35723086)(87.56265167,232.27723694)
\curveto(87.55264402,232.20723101)(87.55764401,232.14223107)(87.57765167,232.08223694)
\curveto(87.58764398,232.05223116)(87.59264398,232.0122312)(87.59265167,231.96223694)
\lineto(87.59265167,231.84223694)
\lineto(87.59265167,231.37723694)
\moveto(86.04765167,229.05223694)
\curveto(86.14764542,229.37223384)(86.20764536,229.73723348)(86.22765167,230.14723694)
\curveto(86.24764532,230.55723266)(86.25764531,230.96723225)(86.25765167,231.37723694)
\curveto(86.25764531,231.80723141)(86.24764532,232.22723099)(86.22765167,232.63723694)
\curveto(86.20764536,233.04723017)(86.16264541,233.43222978)(86.09265167,233.79223694)
\curveto(86.02264555,234.15222906)(85.91264566,234.47222874)(85.76265167,234.75223694)
\curveto(85.62264595,235.04222817)(85.42764614,235.27722794)(85.17765167,235.45723694)
\curveto(85.01764655,235.56722765)(84.83764673,235.64722757)(84.63765167,235.69723694)
\curveto(84.43764713,235.75722746)(84.19264738,235.78722743)(83.90265167,235.78723694)
\curveto(83.88264769,235.76722745)(83.84764772,235.75722746)(83.79765167,235.75723694)
\curveto(83.74764782,235.76722745)(83.70764786,235.76722745)(83.67765167,235.75723694)
\curveto(83.59764797,235.73722748)(83.52264805,235.7172275)(83.45265167,235.69723694)
\curveto(83.39264818,235.68722753)(83.32764824,235.66722755)(83.25765167,235.63723694)
\curveto(82.98764858,235.5172277)(82.7676488,235.34722787)(82.59765167,235.12723694)
\curveto(82.43764913,234.9172283)(82.30264927,234.67222854)(82.19265167,234.39223694)
\curveto(82.14264943,234.28222893)(82.10264947,234.16222905)(82.07265167,234.03223694)
\curveto(82.05264952,233.9122293)(82.02764954,233.78722943)(81.99765167,233.65723694)
\curveto(81.97764959,233.60722961)(81.9676496,233.55222966)(81.96765167,233.49223694)
\curveto(81.9676496,233.44222977)(81.96264961,233.39222982)(81.95265167,233.34223694)
\curveto(81.94264963,233.25222996)(81.93264964,233.15723006)(81.92265167,233.05723694)
\curveto(81.91264966,232.96723025)(81.90264967,232.87223034)(81.89265167,232.77223694)
\curveto(81.89264968,232.69223052)(81.88764968,232.60723061)(81.87765167,232.51723694)
\lineto(81.87765167,232.27723694)
\lineto(81.87765167,232.09723694)
\curveto(81.8676497,232.06723115)(81.86264971,232.03223118)(81.86265167,231.99223694)
\lineto(81.86265167,231.85723694)
\lineto(81.86265167,231.40723694)
\curveto(81.86264971,231.32723189)(81.85764971,231.24223197)(81.84765167,231.15223694)
\curveto(81.84764972,231.07223214)(81.85764971,230.99723222)(81.87765167,230.92723694)
\lineto(81.87765167,230.65723694)
\curveto(81.87764969,230.63723258)(81.8726497,230.60723261)(81.86265167,230.56723694)
\curveto(81.86264971,230.53723268)(81.8676497,230.5122327)(81.87765167,230.49223694)
\curveto(81.88764968,230.39223282)(81.89264968,230.29223292)(81.89265167,230.19223694)
\curveto(81.90264967,230.10223311)(81.91264966,230.00223321)(81.92265167,229.89223694)
\curveto(81.95264962,229.77223344)(81.9676496,229.64723357)(81.96765167,229.51723694)
\curveto(81.97764959,229.39723382)(82.00264957,229.28223393)(82.04265167,229.17223694)
\curveto(82.12264945,228.87223434)(82.20764936,228.60723461)(82.29765167,228.37723694)
\curveto(82.39764917,228.14723507)(82.54264903,227.93223528)(82.73265167,227.73223694)
\curveto(82.94264863,227.53223568)(83.20764836,227.38223583)(83.52765167,227.28223694)
\curveto(83.567648,227.26223595)(83.60264797,227.25223596)(83.63265167,227.25223694)
\curveto(83.6726479,227.26223595)(83.71764785,227.25723596)(83.76765167,227.23723694)
\curveto(83.80764776,227.22723599)(83.87764769,227.217236)(83.97765167,227.20723694)
\curveto(84.08764748,227.19723602)(84.1726474,227.20223601)(84.23265167,227.22223694)
\curveto(84.30264727,227.24223597)(84.3726472,227.25223596)(84.44265167,227.25223694)
\curveto(84.51264706,227.26223595)(84.57764699,227.27723594)(84.63765167,227.29723694)
\curveto(84.83764673,227.35723586)(85.01764655,227.44223577)(85.17765167,227.55223694)
\curveto(85.20764636,227.57223564)(85.23264634,227.59223562)(85.25265167,227.61223694)
\lineto(85.31265167,227.67223694)
\curveto(85.35264622,227.69223552)(85.40264617,227.73223548)(85.46265167,227.79223694)
\curveto(85.56264601,227.93223528)(85.64764592,228.06223515)(85.71765167,228.18223694)
\curveto(85.78764578,228.30223491)(85.85764571,228.44723477)(85.92765167,228.61723694)
\curveto(85.95764561,228.68723453)(85.97764559,228.75723446)(85.98765167,228.82723694)
\curveto(86.00764556,228.89723432)(86.02764554,228.97223424)(86.04765167,229.05223694)
}
}
{
\newrgbcolor{curcolor}{0 0 0}
\pscustom[linestyle=none,fillstyle=solid,fillcolor=curcolor]
{
\newpath
\moveto(70.77343292,304.78723694)
\curveto(70.84342528,304.73723348)(70.88342524,304.66723355)(70.89343292,304.57723694)
\curveto(70.91342521,304.48723373)(70.9234252,304.38223383)(70.92343292,304.26223694)
\curveto(70.9234252,304.212234)(70.9184252,304.16223405)(70.90843292,304.11223694)
\curveto(70.90842521,304.06223415)(70.89842522,304.0172342)(70.87843292,303.97723694)
\curveto(70.84842527,303.88723433)(70.78842533,303.82723439)(70.69843292,303.79723694)
\curveto(70.6184255,303.77723444)(70.5234256,303.76723445)(70.41343292,303.76723694)
\lineto(70.09843292,303.76723694)
\curveto(69.98842613,303.77723444)(69.88342624,303.76723445)(69.78343292,303.73723694)
\curveto(69.64342648,303.70723451)(69.55342657,303.62723459)(69.51343292,303.49723694)
\curveto(69.49342663,303.42723479)(69.48342664,303.34223487)(69.48343292,303.24223694)
\lineto(69.48343292,302.97223694)
\lineto(69.48343292,302.02723694)
\lineto(69.48343292,301.69723694)
\curveto(69.48342664,301.58723663)(69.46342666,301.50223671)(69.42343292,301.44223694)
\curveto(69.38342674,301.38223683)(69.33342679,301.34223687)(69.27343292,301.32223694)
\curveto(69.2234269,301.3122369)(69.15842696,301.29723692)(69.07843292,301.27723694)
\lineto(68.88343292,301.27723694)
\curveto(68.76342736,301.27723694)(68.65842746,301.28223693)(68.56843292,301.29223694)
\curveto(68.47842764,301.3122369)(68.40842771,301.36223685)(68.35843292,301.44223694)
\curveto(68.32842779,301.49223672)(68.31342781,301.56223665)(68.31343292,301.65223694)
\lineto(68.31343292,301.95223694)
\lineto(68.31343292,302.98723694)
\curveto(68.31342781,303.14723507)(68.30342782,303.29223492)(68.28343292,303.42223694)
\curveto(68.27342785,303.56223465)(68.2184279,303.65723456)(68.11843292,303.70723694)
\curveto(68.06842805,303.72723449)(67.99842812,303.74223447)(67.90843292,303.75223694)
\curveto(67.82842829,303.76223445)(67.73842838,303.76723445)(67.63843292,303.76723694)
\lineto(67.35343292,303.76723694)
\lineto(67.11343292,303.76723694)
\lineto(64.84843292,303.76723694)
\curveto(64.75843136,303.76723445)(64.65343147,303.76223445)(64.53343292,303.75223694)
\lineto(64.20343292,303.75223694)
\curveto(64.09343203,303.75223446)(63.99343213,303.76223445)(63.90343292,303.78223694)
\curveto(63.81343231,303.80223441)(63.75343237,303.83723438)(63.72343292,303.88723694)
\curveto(63.67343245,303.95723426)(63.64843247,304.05223416)(63.64843292,304.17223694)
\lineto(63.64843292,304.51723694)
\lineto(63.64843292,304.78723694)
\curveto(63.68843243,304.95723326)(63.74343238,305.09723312)(63.81343292,305.20723694)
\curveto(63.88343224,305.3172329)(63.96343216,305.43223278)(64.05343292,305.55223694)
\lineto(64.41343292,306.09223694)
\curveto(64.85343127,306.72223149)(65.28843083,307.34223087)(65.71843292,307.95223694)
\lineto(67.03843292,309.81223694)
\curveto(67.19842892,310.04222817)(67.35342877,310.26222795)(67.50343292,310.47223694)
\curveto(67.65342847,310.69222752)(67.80842831,310.9172273)(67.96843292,311.14723694)
\curveto(68.0184281,311.217227)(68.06842805,311.28222693)(68.11843292,311.34223694)
\curveto(68.16842795,311.4122268)(68.2184279,311.48722673)(68.26843292,311.56723694)
\lineto(68.32843292,311.65723694)
\curveto(68.35842776,311.69722652)(68.38842773,311.72722649)(68.41843292,311.74723694)
\curveto(68.45842766,311.77722644)(68.49842762,311.79722642)(68.53843292,311.80723694)
\curveto(68.57842754,311.82722639)(68.6234275,311.84722637)(68.67343292,311.86723694)
\curveto(68.69342743,311.86722635)(68.71342741,311.86222635)(68.73343292,311.85223694)
\curveto(68.76342736,311.85222636)(68.78842733,311.86222635)(68.80843292,311.88223694)
\curveto(68.93842718,311.88222633)(69.05842706,311.87722634)(69.16843292,311.86723694)
\curveto(69.27842684,311.85722636)(69.35842676,311.8122264)(69.40843292,311.73223694)
\curveto(69.44842667,311.68222653)(69.46842665,311.6122266)(69.46843292,311.52223694)
\curveto(69.47842664,311.43222678)(69.48342664,311.33722688)(69.48343292,311.23723694)
\lineto(69.48343292,305.77723694)
\curveto(69.48342664,305.70723251)(69.47842664,305.63223258)(69.46843292,305.55223694)
\curveto(69.46842665,305.48223273)(69.47342665,305.4122328)(69.48343292,305.34223694)
\lineto(69.48343292,305.23723694)
\curveto(69.50342662,305.18723303)(69.5184266,305.13223308)(69.52843292,305.07223694)
\curveto(69.53842658,305.02223319)(69.56342656,304.98223323)(69.60343292,304.95223694)
\curveto(69.67342645,304.90223331)(69.75842636,304.87223334)(69.85843292,304.86223694)
\lineto(70.18843292,304.86223694)
\curveto(70.29842582,304.86223335)(70.40342572,304.85723336)(70.50343292,304.84723694)
\curveto(70.61342551,304.84723337)(70.70342542,304.82723339)(70.77343292,304.78723694)
\moveto(68.20843292,304.98223694)
\curveto(68.28842783,305.09223312)(68.3234278,305.26223295)(68.31343292,305.49223694)
\lineto(68.31343292,306.10723694)
\lineto(68.31343292,308.58223694)
\lineto(68.31343292,308.89723694)
\curveto(68.3234278,309.0172292)(68.3184278,309.1172291)(68.29843292,309.19723694)
\lineto(68.29843292,309.34723694)
\curveto(68.29842782,309.43722878)(68.28342784,309.52222869)(68.25343292,309.60223694)
\curveto(68.24342788,309.62222859)(68.23342789,309.63222858)(68.22343292,309.63223694)
\lineto(68.17843292,309.67723694)
\curveto(68.15842796,309.68722853)(68.12842799,309.69222852)(68.08843292,309.69223694)
\curveto(68.06842805,309.67222854)(68.04842807,309.65722856)(68.02843292,309.64723694)
\curveto(68.0184281,309.64722857)(68.00342812,309.64222857)(67.98343292,309.63223694)
\curveto(67.9234282,309.58222863)(67.86342826,309.5122287)(67.80343292,309.42223694)
\curveto(67.74342838,309.33222888)(67.68842843,309.25222896)(67.63843292,309.18223694)
\curveto(67.53842858,309.04222917)(67.44342868,308.89722932)(67.35343292,308.74723694)
\curveto(67.26342886,308.60722961)(67.16842895,308.46722975)(67.06843292,308.32723694)
\lineto(66.52843292,307.54723694)
\curveto(66.35842976,307.28723093)(66.18342994,307.02723119)(66.00343292,306.76723694)
\curveto(65.9234302,306.65723156)(65.84843027,306.55223166)(65.77843292,306.45223694)
\lineto(65.56843292,306.15223694)
\curveto(65.5184306,306.07223214)(65.46843065,305.99723222)(65.41843292,305.92723694)
\curveto(65.37843074,305.85723236)(65.33343079,305.78223243)(65.28343292,305.70223694)
\curveto(65.23343089,305.64223257)(65.18343094,305.57723264)(65.13343292,305.50723694)
\curveto(65.09343103,305.44723277)(65.05343107,305.37723284)(65.01343292,305.29723694)
\curveto(64.97343115,305.23723298)(64.94843117,305.16723305)(64.93843292,305.08723694)
\curveto(64.92843119,305.0172332)(64.96343116,304.96223325)(65.04343292,304.92223694)
\curveto(65.11343101,304.87223334)(65.2234309,304.84723337)(65.37343292,304.84723694)
\curveto(65.53343059,304.85723336)(65.66843045,304.86223335)(65.77843292,304.86223694)
\lineto(67.45843292,304.86223694)
\lineto(67.89343292,304.86223694)
\curveto(68.04342808,304.86223335)(68.14842797,304.90223331)(68.20843292,304.98223694)
}
}
{
\newrgbcolor{curcolor}{0 0 0}
\pscustom[linestyle=none,fillstyle=solid,fillcolor=curcolor]
{
\newpath
\moveto(73.7530423,311.70223694)
\lineto(77.3530423,311.70223694)
\lineto(77.9980423,311.70223694)
\curveto(78.07803577,311.70222651)(78.15303569,311.69722652)(78.2230423,311.68723694)
\curveto(78.29303555,311.68722653)(78.35303549,311.67722654)(78.4030423,311.65723694)
\curveto(78.47303537,311.62722659)(78.52803532,311.56722665)(78.5680423,311.47723694)
\curveto(78.58803526,311.44722677)(78.59803525,311.40722681)(78.5980423,311.35723694)
\lineto(78.5980423,311.22223694)
\curveto(78.60803524,311.1122271)(78.60303524,311.00722721)(78.5830423,310.90723694)
\curveto(78.57303527,310.80722741)(78.53803531,310.73722748)(78.4780423,310.69723694)
\curveto(78.38803546,310.62722759)(78.25303559,310.59222762)(78.0730423,310.59223694)
\curveto(77.89303595,310.60222761)(77.72803612,310.60722761)(77.5780423,310.60723694)
\lineto(75.5830423,310.60723694)
\lineto(75.0880423,310.60723694)
\lineto(74.9530423,310.60723694)
\curveto(74.91303893,310.60722761)(74.87303897,310.60222761)(74.8330423,310.59223694)
\lineto(74.6230423,310.59223694)
\curveto(74.51303933,310.56222765)(74.43303941,310.52222769)(74.3830423,310.47223694)
\curveto(74.33303951,310.43222778)(74.29803955,310.37722784)(74.2780423,310.30723694)
\curveto(74.25803959,310.24722797)(74.2430396,310.17722804)(74.2330423,310.09723694)
\curveto(74.22303962,310.0172282)(74.20303964,309.92722829)(74.1730423,309.82723694)
\curveto(74.12303972,309.62722859)(74.08303976,309.42222879)(74.0530423,309.21223694)
\curveto(74.02303982,309.00222921)(73.98303986,308.79722942)(73.9330423,308.59723694)
\curveto(73.91303993,308.52722969)(73.90303994,308.45722976)(73.9030423,308.38723694)
\curveto(73.90303994,308.32722989)(73.89303995,308.26222995)(73.8730423,308.19223694)
\curveto(73.86303998,308.16223005)(73.85303999,308.12223009)(73.8430423,308.07223694)
\curveto(73.84304,308.03223018)(73.84804,307.99223022)(73.8580423,307.95223694)
\curveto(73.87803997,307.90223031)(73.90303994,307.85723036)(73.9330423,307.81723694)
\curveto(73.97303987,307.78723043)(74.03303981,307.78223043)(74.1130423,307.80223694)
\curveto(74.17303967,307.82223039)(74.23303961,307.84723037)(74.2930423,307.87723694)
\curveto(74.35303949,307.9172303)(74.41303943,307.95223026)(74.4730423,307.98223694)
\curveto(74.53303931,308.00223021)(74.58303926,308.0172302)(74.6230423,308.02723694)
\curveto(74.81303903,308.10723011)(75.01803883,308.16223005)(75.2380423,308.19223694)
\curveto(75.46803838,308.22222999)(75.69803815,308.23222998)(75.9280423,308.22223694)
\curveto(76.16803768,308.22222999)(76.39803745,308.19723002)(76.6180423,308.14723694)
\curveto(76.83803701,308.10723011)(77.03803681,308.04723017)(77.2180423,307.96723694)
\curveto(77.26803658,307.94723027)(77.31303653,307.92723029)(77.3530423,307.90723694)
\curveto(77.40303644,307.88723033)(77.45303639,307.86223035)(77.5030423,307.83223694)
\curveto(77.85303599,307.62223059)(78.13303571,307.39223082)(78.3430423,307.14223694)
\curveto(78.56303528,306.89223132)(78.75803509,306.56723165)(78.9280423,306.16723694)
\curveto(78.97803487,306.05723216)(79.01303483,305.94723227)(79.0330423,305.83723694)
\curveto(79.05303479,305.72723249)(79.07803477,305.6122326)(79.1080423,305.49223694)
\curveto(79.11803473,305.46223275)(79.12303472,305.4172328)(79.1230423,305.35723694)
\curveto(79.1430347,305.29723292)(79.15303469,305.22723299)(79.1530423,305.14723694)
\curveto(79.15303469,305.07723314)(79.16303468,305.0122332)(79.1830423,304.95223694)
\lineto(79.1830423,304.78723694)
\curveto(79.19303465,304.73723348)(79.19803465,304.66723355)(79.1980423,304.57723694)
\curveto(79.19803465,304.48723373)(79.18803466,304.4172338)(79.1680423,304.36723694)
\curveto(79.1480347,304.30723391)(79.1430347,304.24723397)(79.1530423,304.18723694)
\curveto(79.16303468,304.13723408)(79.15803469,304.08723413)(79.1380423,304.03723694)
\curveto(79.09803475,303.87723434)(79.06303478,303.72723449)(79.0330423,303.58723694)
\curveto(79.00303484,303.44723477)(78.95803489,303.3122349)(78.8980423,303.18223694)
\curveto(78.73803511,302.8122354)(78.51803533,302.47723574)(78.2380423,302.17723694)
\curveto(77.95803589,301.87723634)(77.63803621,301.64723657)(77.2780423,301.48723694)
\curveto(77.10803674,301.40723681)(76.90803694,301.33223688)(76.6780423,301.26223694)
\curveto(76.56803728,301.22223699)(76.45303739,301.19723702)(76.3330423,301.18723694)
\curveto(76.21303763,301.17723704)(76.09303775,301.15723706)(75.9730423,301.12723694)
\curveto(75.92303792,301.10723711)(75.86803798,301.10723711)(75.8080423,301.12723694)
\curveto(75.7480381,301.13723708)(75.68803816,301.13223708)(75.6280423,301.11223694)
\curveto(75.52803832,301.09223712)(75.42803842,301.09223712)(75.3280423,301.11223694)
\lineto(75.1930423,301.11223694)
\curveto(75.1430387,301.13223708)(75.08303876,301.14223707)(75.0130423,301.14223694)
\curveto(74.95303889,301.13223708)(74.89803895,301.13723708)(74.8480423,301.15723694)
\curveto(74.80803904,301.16723705)(74.77303907,301.17223704)(74.7430423,301.17223694)
\curveto(74.71303913,301.17223704)(74.67803917,301.17723704)(74.6380423,301.18723694)
\lineto(74.3680423,301.24723694)
\curveto(74.27803957,301.26723695)(74.19303965,301.29723692)(74.1130423,301.33723694)
\curveto(73.77304007,301.47723674)(73.48304036,301.63223658)(73.2430423,301.80223694)
\curveto(73.00304084,301.98223623)(72.78304106,302.212236)(72.5830423,302.49223694)
\curveto(72.43304141,302.72223549)(72.31804153,302.96223525)(72.2380423,303.21223694)
\curveto(72.21804163,303.26223495)(72.20804164,303.30723491)(72.2080423,303.34723694)
\curveto(72.20804164,303.39723482)(72.19804165,303.44723477)(72.1780423,303.49723694)
\curveto(72.15804169,303.55723466)(72.1430417,303.63723458)(72.1330423,303.73723694)
\curveto(72.13304171,303.83723438)(72.15304169,303.9122343)(72.1930423,303.96223694)
\curveto(72.2430416,304.04223417)(72.32304152,304.08723413)(72.4330423,304.09723694)
\curveto(72.5430413,304.10723411)(72.65804119,304.1122341)(72.7780423,304.11223694)
\lineto(72.9430423,304.11223694)
\curveto(73.00304084,304.1122341)(73.05804079,304.10223411)(73.1080423,304.08223694)
\curveto(73.19804065,304.06223415)(73.26804058,304.02223419)(73.3180423,303.96223694)
\curveto(73.38804046,303.87223434)(73.43304041,303.76223445)(73.4530423,303.63223694)
\curveto(73.48304036,303.5122347)(73.52804032,303.40723481)(73.5880423,303.31723694)
\curveto(73.77804007,302.97723524)(74.03803981,302.70723551)(74.3680423,302.50723694)
\curveto(74.46803938,302.44723577)(74.57303927,302.39723582)(74.6830423,302.35723694)
\curveto(74.80303904,302.32723589)(74.92303892,302.29223592)(75.0430423,302.25223694)
\curveto(75.21303863,302.20223601)(75.41803843,302.18223603)(75.6580423,302.19223694)
\curveto(75.90803794,302.212236)(76.10803774,302.24723597)(76.2580423,302.29723694)
\curveto(76.62803722,302.4172358)(76.91803693,302.57723564)(77.1280423,302.77723694)
\curveto(77.3480365,302.98723523)(77.52803632,303.26723495)(77.6680423,303.61723694)
\curveto(77.71803613,303.7172345)(77.7480361,303.82223439)(77.7580423,303.93223694)
\curveto(77.77803607,304.04223417)(77.80303604,304.15723406)(77.8330423,304.27723694)
\lineto(77.8330423,304.38223694)
\curveto(77.843036,304.42223379)(77.848036,304.46223375)(77.8480423,304.50223694)
\curveto(77.85803599,304.53223368)(77.85803599,304.56723365)(77.8480423,304.60723694)
\lineto(77.8480423,304.72723694)
\curveto(77.848036,304.98723323)(77.81803603,305.23223298)(77.7580423,305.46223694)
\curveto(77.6480362,305.8122324)(77.49303635,306.10723211)(77.2930423,306.34723694)
\curveto(77.09303675,306.59723162)(76.83303701,306.79223142)(76.5130423,306.93223694)
\lineto(76.3330423,306.99223694)
\curveto(76.28303756,307.0122312)(76.22303762,307.03223118)(76.1530423,307.05223694)
\curveto(76.10303774,307.07223114)(76.0430378,307.08223113)(75.9730423,307.08223694)
\curveto(75.91303793,307.09223112)(75.848038,307.10723111)(75.7780423,307.12723694)
\lineto(75.6280423,307.12723694)
\curveto(75.58803826,307.14723107)(75.53303831,307.15723106)(75.4630423,307.15723694)
\curveto(75.40303844,307.15723106)(75.3480385,307.14723107)(75.2980423,307.12723694)
\lineto(75.1930423,307.12723694)
\curveto(75.16303868,307.12723109)(75.12803872,307.12223109)(75.0880423,307.11223694)
\lineto(74.8480423,307.05223694)
\curveto(74.76803908,307.04223117)(74.68803916,307.02223119)(74.6080423,306.99223694)
\curveto(74.36803948,306.89223132)(74.13803971,306.75723146)(73.9180423,306.58723694)
\curveto(73.82804002,306.5172317)(73.7430401,306.44223177)(73.6630423,306.36223694)
\curveto(73.58304026,306.29223192)(73.48304036,306.23723198)(73.3630423,306.19723694)
\curveto(73.27304057,306.16723205)(73.13304071,306.15723206)(72.9430423,306.16723694)
\curveto(72.76304108,306.17723204)(72.6430412,306.20223201)(72.5830423,306.24223694)
\curveto(72.53304131,306.28223193)(72.49304135,306.34223187)(72.4630423,306.42223694)
\curveto(72.4430414,306.50223171)(72.4430414,306.58723163)(72.4630423,306.67723694)
\curveto(72.49304135,306.79723142)(72.51304133,306.9172313)(72.5230423,307.03723694)
\curveto(72.5430413,307.16723105)(72.56804128,307.29223092)(72.5980423,307.41223694)
\curveto(72.61804123,307.45223076)(72.62304122,307.48723073)(72.6130423,307.51723694)
\curveto(72.61304123,307.55723066)(72.62304122,307.60223061)(72.6430423,307.65223694)
\curveto(72.66304118,307.74223047)(72.67804117,307.83223038)(72.6880423,307.92223694)
\curveto(72.69804115,308.02223019)(72.71804113,308.1172301)(72.7480423,308.20723694)
\curveto(72.75804109,308.26722995)(72.76304108,308.32722989)(72.7630423,308.38723694)
\curveto(72.77304107,308.44722977)(72.78804106,308.50722971)(72.8080423,308.56723694)
\curveto(72.85804099,308.76722945)(72.89304095,308.97222924)(72.9130423,309.18223694)
\curveto(72.9430409,309.40222881)(72.98304086,309.6122286)(73.0330423,309.81223694)
\curveto(73.06304078,309.9122283)(73.08304076,310.0122282)(73.0930423,310.11223694)
\curveto(73.10304074,310.212228)(73.11804073,310.3122279)(73.1380423,310.41223694)
\curveto(73.1480407,310.44222777)(73.15304069,310.48222773)(73.1530423,310.53223694)
\curveto(73.18304066,310.64222757)(73.20304064,310.74722747)(73.2130423,310.84723694)
\curveto(73.23304061,310.95722726)(73.25804059,311.06722715)(73.2880423,311.17723694)
\curveto(73.30804054,311.25722696)(73.32304052,311.32722689)(73.3330423,311.38723694)
\curveto(73.3430405,311.45722676)(73.36804048,311.5172267)(73.4080423,311.56723694)
\curveto(73.42804042,311.59722662)(73.45804039,311.6172266)(73.4980423,311.62723694)
\curveto(73.53804031,311.64722657)(73.58304026,311.66722655)(73.6330423,311.68723694)
\curveto(73.69304015,311.68722653)(73.73304011,311.69222652)(73.7530423,311.70223694)
}
}
{
\newrgbcolor{curcolor}{0 0 0}
\pscustom[linestyle=none,fillstyle=solid,fillcolor=curcolor]
{
\newpath
\moveto(87.59265167,306.37723694)
\lineto(87.59265167,306.12223694)
\curveto(87.60264397,306.04223217)(87.59764397,305.96723225)(87.57765167,305.89723694)
\lineto(87.57765167,305.65723694)
\lineto(87.57765167,305.49223694)
\curveto(87.55764401,305.39223282)(87.54764402,305.28723293)(87.54765167,305.17723694)
\curveto(87.54764402,305.07723314)(87.53764403,304.97723324)(87.51765167,304.87723694)
\lineto(87.51765167,304.72723694)
\curveto(87.48764408,304.58723363)(87.4676441,304.44723377)(87.45765167,304.30723694)
\curveto(87.44764412,304.17723404)(87.42264415,304.04723417)(87.38265167,303.91723694)
\curveto(87.36264421,303.83723438)(87.34264423,303.75223446)(87.32265167,303.66223694)
\lineto(87.26265167,303.42223694)
\lineto(87.14265167,303.12223694)
\curveto(87.11264446,303.03223518)(87.07764449,302.94223527)(87.03765167,302.85223694)
\curveto(86.93764463,302.63223558)(86.80264477,302.4172358)(86.63265167,302.20723694)
\curveto(86.4726451,301.99723622)(86.29764527,301.82723639)(86.10765167,301.69723694)
\curveto(86.05764551,301.65723656)(85.99764557,301.6172366)(85.92765167,301.57723694)
\curveto(85.8676457,301.54723667)(85.80764576,301.5122367)(85.74765167,301.47223694)
\curveto(85.6676459,301.42223679)(85.572646,301.38223683)(85.46265167,301.35223694)
\curveto(85.35264622,301.32223689)(85.24764632,301.29223692)(85.14765167,301.26223694)
\curveto(85.03764653,301.22223699)(84.92764664,301.19723702)(84.81765167,301.18723694)
\curveto(84.70764686,301.17723704)(84.59264698,301.16223705)(84.47265167,301.14223694)
\curveto(84.43264714,301.13223708)(84.38764718,301.13223708)(84.33765167,301.14223694)
\curveto(84.29764727,301.14223707)(84.25764731,301.13723708)(84.21765167,301.12723694)
\curveto(84.17764739,301.1172371)(84.12264745,301.1122371)(84.05265167,301.11223694)
\curveto(83.98264759,301.1122371)(83.93264764,301.1172371)(83.90265167,301.12723694)
\curveto(83.85264772,301.14723707)(83.80764776,301.15223706)(83.76765167,301.14223694)
\curveto(83.72764784,301.13223708)(83.69264788,301.13223708)(83.66265167,301.14223694)
\lineto(83.57265167,301.14223694)
\curveto(83.51264806,301.16223705)(83.44764812,301.17723704)(83.37765167,301.18723694)
\curveto(83.31764825,301.18723703)(83.25264832,301.19223702)(83.18265167,301.20223694)
\curveto(83.01264856,301.25223696)(82.85264872,301.30223691)(82.70265167,301.35223694)
\curveto(82.55264902,301.40223681)(82.40764916,301.46723675)(82.26765167,301.54723694)
\curveto(82.21764935,301.58723663)(82.16264941,301.6172366)(82.10265167,301.63723694)
\curveto(82.05264952,301.66723655)(82.00264957,301.70223651)(81.95265167,301.74223694)
\curveto(81.71264986,301.92223629)(81.51265006,302.14223607)(81.35265167,302.40223694)
\curveto(81.19265038,302.66223555)(81.05265052,302.94723527)(80.93265167,303.25723694)
\curveto(80.8726507,303.39723482)(80.82765074,303.53723468)(80.79765167,303.67723694)
\curveto(80.7676508,303.82723439)(80.73265084,303.98223423)(80.69265167,304.14223694)
\curveto(80.6726509,304.25223396)(80.65765091,304.36223385)(80.64765167,304.47223694)
\curveto(80.63765093,304.58223363)(80.62265095,304.69223352)(80.60265167,304.80223694)
\curveto(80.59265098,304.84223337)(80.58765098,304.88223333)(80.58765167,304.92223694)
\curveto(80.59765097,304.96223325)(80.59765097,305.00223321)(80.58765167,305.04223694)
\curveto(80.57765099,305.09223312)(80.572651,305.14223307)(80.57265167,305.19223694)
\lineto(80.57265167,305.35723694)
\curveto(80.55265102,305.40723281)(80.54765102,305.45723276)(80.55765167,305.50723694)
\curveto(80.567651,305.56723265)(80.567651,305.62223259)(80.55765167,305.67223694)
\curveto(80.54765102,305.7122325)(80.54765102,305.75723246)(80.55765167,305.80723694)
\curveto(80.567651,305.85723236)(80.56265101,305.90723231)(80.54265167,305.95723694)
\curveto(80.52265105,306.02723219)(80.51765105,306.10223211)(80.52765167,306.18223694)
\curveto(80.53765103,306.27223194)(80.54265103,306.35723186)(80.54265167,306.43723694)
\curveto(80.54265103,306.52723169)(80.53765103,306.62723159)(80.52765167,306.73723694)
\curveto(80.51765105,306.85723136)(80.52265105,306.95723126)(80.54265167,307.03723694)
\lineto(80.54265167,307.32223694)
\lineto(80.58765167,307.95223694)
\curveto(80.59765097,308.05223016)(80.60765096,308.14723007)(80.61765167,308.23723694)
\lineto(80.64765167,308.53723694)
\curveto(80.6676509,308.58722963)(80.6726509,308.63722958)(80.66265167,308.68723694)
\curveto(80.66265091,308.74722947)(80.6726509,308.80222941)(80.69265167,308.85223694)
\curveto(80.74265083,309.02222919)(80.78265079,309.18722903)(80.81265167,309.34723694)
\curveto(80.84265073,309.5172287)(80.89265068,309.67722854)(80.96265167,309.82723694)
\curveto(81.15265042,310.28722793)(81.3726502,310.66222755)(81.62265167,310.95223694)
\curveto(81.88264969,311.24222697)(82.24264933,311.48722673)(82.70265167,311.68723694)
\curveto(82.83264874,311.73722648)(82.96264861,311.77222644)(83.09265167,311.79223694)
\curveto(83.23264834,311.8122264)(83.3726482,311.83722638)(83.51265167,311.86723694)
\curveto(83.58264799,311.87722634)(83.64764792,311.88222633)(83.70765167,311.88223694)
\curveto(83.7676478,311.88222633)(83.83264774,311.88722633)(83.90265167,311.89723694)
\curveto(84.73264684,311.9172263)(85.40264617,311.76722645)(85.91265167,311.44723694)
\curveto(86.42264515,311.13722708)(86.80264477,310.69722752)(87.05265167,310.12723694)
\curveto(87.10264447,310.00722821)(87.14764442,309.88222833)(87.18765167,309.75223694)
\curveto(87.22764434,309.62222859)(87.2726443,309.48722873)(87.32265167,309.34723694)
\curveto(87.34264423,309.26722895)(87.35764421,309.18222903)(87.36765167,309.09223694)
\lineto(87.42765167,308.85223694)
\curveto(87.45764411,308.74222947)(87.4726441,308.63222958)(87.47265167,308.52223694)
\curveto(87.48264409,308.4122298)(87.49764407,308.30222991)(87.51765167,308.19223694)
\curveto(87.53764403,308.14223007)(87.54264403,308.09723012)(87.53265167,308.05723694)
\curveto(87.53264404,308.0172302)(87.53764403,307.97723024)(87.54765167,307.93723694)
\curveto(87.55764401,307.88723033)(87.55764401,307.83223038)(87.54765167,307.77223694)
\curveto(87.54764402,307.72223049)(87.55264402,307.67223054)(87.56265167,307.62223694)
\lineto(87.56265167,307.48723694)
\curveto(87.58264399,307.42723079)(87.58264399,307.35723086)(87.56265167,307.27723694)
\curveto(87.55264402,307.20723101)(87.55764401,307.14223107)(87.57765167,307.08223694)
\curveto(87.58764398,307.05223116)(87.59264398,307.0122312)(87.59265167,306.96223694)
\lineto(87.59265167,306.84223694)
\lineto(87.59265167,306.37723694)
\moveto(86.04765167,304.05223694)
\curveto(86.14764542,304.37223384)(86.20764536,304.73723348)(86.22765167,305.14723694)
\curveto(86.24764532,305.55723266)(86.25764531,305.96723225)(86.25765167,306.37723694)
\curveto(86.25764531,306.80723141)(86.24764532,307.22723099)(86.22765167,307.63723694)
\curveto(86.20764536,308.04723017)(86.16264541,308.43222978)(86.09265167,308.79223694)
\curveto(86.02264555,309.15222906)(85.91264566,309.47222874)(85.76265167,309.75223694)
\curveto(85.62264595,310.04222817)(85.42764614,310.27722794)(85.17765167,310.45723694)
\curveto(85.01764655,310.56722765)(84.83764673,310.64722757)(84.63765167,310.69723694)
\curveto(84.43764713,310.75722746)(84.19264738,310.78722743)(83.90265167,310.78723694)
\curveto(83.88264769,310.76722745)(83.84764772,310.75722746)(83.79765167,310.75723694)
\curveto(83.74764782,310.76722745)(83.70764786,310.76722745)(83.67765167,310.75723694)
\curveto(83.59764797,310.73722748)(83.52264805,310.7172275)(83.45265167,310.69723694)
\curveto(83.39264818,310.68722753)(83.32764824,310.66722755)(83.25765167,310.63723694)
\curveto(82.98764858,310.5172277)(82.7676488,310.34722787)(82.59765167,310.12723694)
\curveto(82.43764913,309.9172283)(82.30264927,309.67222854)(82.19265167,309.39223694)
\curveto(82.14264943,309.28222893)(82.10264947,309.16222905)(82.07265167,309.03223694)
\curveto(82.05264952,308.9122293)(82.02764954,308.78722943)(81.99765167,308.65723694)
\curveto(81.97764959,308.60722961)(81.9676496,308.55222966)(81.96765167,308.49223694)
\curveto(81.9676496,308.44222977)(81.96264961,308.39222982)(81.95265167,308.34223694)
\curveto(81.94264963,308.25222996)(81.93264964,308.15723006)(81.92265167,308.05723694)
\curveto(81.91264966,307.96723025)(81.90264967,307.87223034)(81.89265167,307.77223694)
\curveto(81.89264968,307.69223052)(81.88764968,307.60723061)(81.87765167,307.51723694)
\lineto(81.87765167,307.27723694)
\lineto(81.87765167,307.09723694)
\curveto(81.8676497,307.06723115)(81.86264971,307.03223118)(81.86265167,306.99223694)
\lineto(81.86265167,306.85723694)
\lineto(81.86265167,306.40723694)
\curveto(81.86264971,306.32723189)(81.85764971,306.24223197)(81.84765167,306.15223694)
\curveto(81.84764972,306.07223214)(81.85764971,305.99723222)(81.87765167,305.92723694)
\lineto(81.87765167,305.65723694)
\curveto(81.87764969,305.63723258)(81.8726497,305.60723261)(81.86265167,305.56723694)
\curveto(81.86264971,305.53723268)(81.8676497,305.5122327)(81.87765167,305.49223694)
\curveto(81.88764968,305.39223282)(81.89264968,305.29223292)(81.89265167,305.19223694)
\curveto(81.90264967,305.10223311)(81.91264966,305.00223321)(81.92265167,304.89223694)
\curveto(81.95264962,304.77223344)(81.9676496,304.64723357)(81.96765167,304.51723694)
\curveto(81.97764959,304.39723382)(82.00264957,304.28223393)(82.04265167,304.17223694)
\curveto(82.12264945,303.87223434)(82.20764936,303.60723461)(82.29765167,303.37723694)
\curveto(82.39764917,303.14723507)(82.54264903,302.93223528)(82.73265167,302.73223694)
\curveto(82.94264863,302.53223568)(83.20764836,302.38223583)(83.52765167,302.28223694)
\curveto(83.567648,302.26223595)(83.60264797,302.25223596)(83.63265167,302.25223694)
\curveto(83.6726479,302.26223595)(83.71764785,302.25723596)(83.76765167,302.23723694)
\curveto(83.80764776,302.22723599)(83.87764769,302.217236)(83.97765167,302.20723694)
\curveto(84.08764748,302.19723602)(84.1726474,302.20223601)(84.23265167,302.22223694)
\curveto(84.30264727,302.24223597)(84.3726472,302.25223596)(84.44265167,302.25223694)
\curveto(84.51264706,302.26223595)(84.57764699,302.27723594)(84.63765167,302.29723694)
\curveto(84.83764673,302.35723586)(85.01764655,302.44223577)(85.17765167,302.55223694)
\curveto(85.20764636,302.57223564)(85.23264634,302.59223562)(85.25265167,302.61223694)
\lineto(85.31265167,302.67223694)
\curveto(85.35264622,302.69223552)(85.40264617,302.73223548)(85.46265167,302.79223694)
\curveto(85.56264601,302.93223528)(85.64764592,303.06223515)(85.71765167,303.18223694)
\curveto(85.78764578,303.30223491)(85.85764571,303.44723477)(85.92765167,303.61723694)
\curveto(85.95764561,303.68723453)(85.97764559,303.75723446)(85.98765167,303.82723694)
\curveto(86.00764556,303.89723432)(86.02764554,303.97223424)(86.04765167,304.05223694)
}
}
{
\newrgbcolor{curcolor}{0 0 0}
\pscustom[linestyle=none,fillstyle=solid,fillcolor=curcolor]
{
\newpath
\moveto(70.89343292,379.89795013)
\curveto(70.9234252,379.77794592)(70.94842517,379.63794606)(70.96843292,379.47795013)
\curveto(70.98842513,379.31794638)(70.99842512,379.15294654)(70.99843292,378.98295013)
\curveto(70.99842512,378.81294688)(70.98842513,378.64794705)(70.96843292,378.48795013)
\curveto(70.94842517,378.32794737)(70.9234252,378.18794751)(70.89343292,378.06795013)
\curveto(70.85342527,377.92794777)(70.8184253,377.80294789)(70.78843292,377.69295013)
\curveto(70.75842536,377.58294811)(70.7184254,377.47294822)(70.66843292,377.36295013)
\curveto(70.39842572,376.72294897)(69.98342614,376.23794946)(69.42343292,375.90795013)
\curveto(69.34342678,375.84794985)(69.25842686,375.7979499)(69.16843292,375.75795013)
\curveto(69.07842704,375.72794997)(68.97842714,375.69295)(68.86843292,375.65295013)
\curveto(68.75842736,375.60295009)(68.63842748,375.56795013)(68.50843292,375.54795013)
\curveto(68.38842773,375.51795018)(68.25842786,375.48795021)(68.11843292,375.45795013)
\curveto(68.05842806,375.43795026)(67.99842812,375.43295026)(67.93843292,375.44295013)
\curveto(67.88842823,375.45295024)(67.82842829,375.44795025)(67.75843292,375.42795013)
\curveto(67.73842838,375.41795028)(67.71342841,375.41795028)(67.68343292,375.42795013)
\curveto(67.65342847,375.42795027)(67.62842849,375.42295027)(67.60843292,375.41295013)
\lineto(67.45843292,375.41295013)
\curveto(67.38842873,375.40295029)(67.33842878,375.40295029)(67.30843292,375.41295013)
\curveto(67.26842885,375.42295027)(67.2234289,375.42795027)(67.17343292,375.42795013)
\curveto(67.13342899,375.41795028)(67.09342903,375.41795028)(67.05343292,375.42795013)
\curveto(66.96342916,375.44795025)(66.87342925,375.46295023)(66.78343292,375.47295013)
\curveto(66.69342943,375.47295022)(66.60342952,375.48295021)(66.51343292,375.50295013)
\curveto(66.4234297,375.53295016)(66.33342979,375.55795014)(66.24343292,375.57795013)
\curveto(66.15342997,375.5979501)(66.06843005,375.62795007)(65.98843292,375.66795013)
\curveto(65.74843037,375.77794992)(65.5234306,375.90794979)(65.31343292,376.05795013)
\curveto(65.10343102,376.21794948)(64.9234312,376.3979493)(64.77343292,376.59795013)
\curveto(64.65343147,376.76794893)(64.54843157,376.94294875)(64.45843292,377.12295013)
\curveto(64.36843175,377.30294839)(64.27843184,377.4929482)(64.18843292,377.69295013)
\curveto(64.14843197,377.7929479)(64.11343201,377.8929478)(64.08343292,377.99295013)
\curveto(64.06343206,378.10294759)(64.03843208,378.21294748)(64.00843292,378.32295013)
\curveto(63.96843215,378.46294723)(63.94343218,378.60294709)(63.93343292,378.74295013)
\curveto(63.9234322,378.88294681)(63.90343222,379.02294667)(63.87343292,379.16295013)
\curveto(63.86343226,379.27294642)(63.85343227,379.37294632)(63.84343292,379.46295013)
\curveto(63.84343228,379.56294613)(63.83343229,379.66294603)(63.81343292,379.76295013)
\lineto(63.81343292,379.85295013)
\curveto(63.8234323,379.88294581)(63.8234323,379.90794579)(63.81343292,379.92795013)
\lineto(63.81343292,380.13795013)
\curveto(63.79343233,380.1979455)(63.78343234,380.26294543)(63.78343292,380.33295013)
\curveto(63.79343233,380.41294528)(63.79843232,380.48794521)(63.79843292,380.55795013)
\lineto(63.79843292,380.70795013)
\curveto(63.79843232,380.75794494)(63.80343232,380.80794489)(63.81343292,380.85795013)
\lineto(63.81343292,381.23295013)
\curveto(63.8234323,381.26294443)(63.8234323,381.2979444)(63.81343292,381.33795013)
\curveto(63.81343231,381.37794432)(63.8184323,381.41794428)(63.82843292,381.45795013)
\curveto(63.84843227,381.56794413)(63.86343226,381.67794402)(63.87343292,381.78795013)
\curveto(63.88343224,381.90794379)(63.89343223,382.02294367)(63.90343292,382.13295013)
\curveto(63.94343218,382.28294341)(63.96843215,382.42794327)(63.97843292,382.56795013)
\curveto(63.99843212,382.71794298)(64.02843209,382.86294283)(64.06843292,383.00295013)
\curveto(64.15843196,383.30294239)(64.25343187,383.58794211)(64.35343292,383.85795013)
\curveto(64.45343167,384.12794157)(64.57843154,384.37794132)(64.72843292,384.60795013)
\curveto(64.92843119,384.92794077)(65.17343095,385.20794049)(65.46343292,385.44795013)
\curveto(65.75343037,385.68794001)(66.09343003,385.87293982)(66.48343292,386.00295013)
\curveto(66.59342953,386.04293965)(66.70342942,386.06793963)(66.81343292,386.07795013)
\curveto(66.93342919,386.0979396)(67.05342907,386.12293957)(67.17343292,386.15295013)
\curveto(67.24342888,386.16293953)(67.30842881,386.16793953)(67.36843292,386.16795013)
\curveto(67.42842869,386.16793953)(67.49342863,386.17293952)(67.56343292,386.18295013)
\curveto(68.26342786,386.20293949)(68.83842728,386.08793961)(69.28843292,385.83795013)
\curveto(69.73842638,385.58794011)(70.08342604,385.23794046)(70.32343292,384.78795013)
\curveto(70.43342569,384.55794114)(70.53342559,384.28294141)(70.62343292,383.96295013)
\curveto(70.64342548,383.8929418)(70.64342548,383.81794188)(70.62343292,383.73795013)
\curveto(70.61342551,383.66794203)(70.58842553,383.61794208)(70.54843292,383.58795013)
\curveto(70.5184256,383.55794214)(70.45842566,383.53294216)(70.36843292,383.51295013)
\curveto(70.27842584,383.50294219)(70.17842594,383.4929422)(70.06843292,383.48295013)
\curveto(69.96842615,383.48294221)(69.86842625,383.48794221)(69.76843292,383.49795013)
\curveto(69.67842644,383.50794219)(69.61342651,383.52794217)(69.57343292,383.55795013)
\curveto(69.46342666,383.62794207)(69.38342674,383.73794196)(69.33343292,383.88795013)
\curveto(69.29342683,384.03794166)(69.23842688,384.16794153)(69.16843292,384.27795013)
\curveto(68.97842714,384.58794111)(68.69842742,384.81794088)(68.32843292,384.96795013)
\curveto(68.25842786,384.9979407)(68.18342794,385.01794068)(68.10343292,385.02795013)
\curveto(68.03342809,385.03794066)(67.95842816,385.05294064)(67.87843292,385.07295013)
\curveto(67.82842829,385.08294061)(67.75842836,385.08794061)(67.66843292,385.08795013)
\curveto(67.58842853,385.08794061)(67.5234286,385.08294061)(67.47343292,385.07295013)
\curveto(67.43342869,385.05294064)(67.39842872,385.04794065)(67.36843292,385.05795013)
\curveto(67.33842878,385.06794063)(67.30342882,385.06794063)(67.26343292,385.05795013)
\lineto(67.02343292,384.99795013)
\curveto(66.95342917,384.97794072)(66.88342924,384.95294074)(66.81343292,384.92295013)
\curveto(66.43342969,384.76294093)(66.14342998,384.55294114)(65.94343292,384.29295013)
\curveto(65.75343037,384.03294166)(65.57843054,383.71794198)(65.41843292,383.34795013)
\curveto(65.38843073,383.26794243)(65.36343076,383.18794251)(65.34343292,383.10795013)
\curveto(65.33343079,383.02794267)(65.31343081,382.94794275)(65.28343292,382.86795013)
\curveto(65.25343087,382.75794294)(65.22843089,382.64294305)(65.20843292,382.52295013)
\curveto(65.19843092,382.40294329)(65.17843094,382.28294341)(65.14843292,382.16295013)
\curveto(65.12843099,382.11294358)(65.118431,382.06294363)(65.11843292,382.01295013)
\curveto(65.12843099,381.96294373)(65.123431,381.91294378)(65.10343292,381.86295013)
\curveto(65.09343103,381.80294389)(65.09343103,381.72294397)(65.10343292,381.62295013)
\curveto(65.11343101,381.53294416)(65.12843099,381.47794422)(65.14843292,381.45795013)
\curveto(65.16843095,381.41794428)(65.19843092,381.3979443)(65.23843292,381.39795013)
\curveto(65.28843083,381.3979443)(65.33343079,381.40794429)(65.37343292,381.42795013)
\curveto(65.44343068,381.46794423)(65.50343062,381.51294418)(65.55343292,381.56295013)
\curveto(65.60343052,381.61294408)(65.66343046,381.66294403)(65.73343292,381.71295013)
\lineto(65.79343292,381.77295013)
\curveto(65.8234303,381.80294389)(65.85343027,381.82794387)(65.88343292,381.84795013)
\curveto(66.11343001,382.00794369)(66.38842973,382.14294355)(66.70843292,382.25295013)
\curveto(66.77842934,382.27294342)(66.84842927,382.28794341)(66.91843292,382.29795013)
\curveto(66.98842913,382.30794339)(67.06342906,382.32294337)(67.14343292,382.34295013)
\curveto(67.18342894,382.34294335)(67.2184289,382.34794335)(67.24843292,382.35795013)
\curveto(67.27842884,382.36794333)(67.31342881,382.36794333)(67.35343292,382.35795013)
\curveto(67.40342872,382.35794334)(67.44342868,382.36794333)(67.47343292,382.38795013)
\lineto(67.63843292,382.38795013)
\lineto(67.72843292,382.38795013)
\curveto(67.77842834,382.3979433)(67.8184283,382.3979433)(67.84843292,382.38795013)
\curveto(67.89842822,382.37794332)(67.94842817,382.37294332)(67.99843292,382.37295013)
\curveto(68.05842806,382.38294331)(68.11342801,382.38294331)(68.16343292,382.37295013)
\curveto(68.27342785,382.34294335)(68.37842774,382.32294337)(68.47843292,382.31295013)
\curveto(68.58842753,382.30294339)(68.69342743,382.27794342)(68.79343292,382.23795013)
\curveto(69.21342691,382.0979436)(69.55842656,381.91294378)(69.82843292,381.68295013)
\curveto(70.09842602,381.46294423)(70.33842578,381.17794452)(70.54843292,380.82795013)
\curveto(70.62842549,380.68794501)(70.69342543,380.53794516)(70.74343292,380.37795013)
\curveto(70.79342533,380.22794547)(70.84342528,380.06794563)(70.89343292,379.89795013)
\moveto(69.64843292,378.59295013)
\curveto(69.65842646,378.64294705)(69.66342646,378.68794701)(69.66343292,378.72795013)
\lineto(69.66343292,378.87795013)
\curveto(69.66342646,379.18794651)(69.6234265,379.47294622)(69.54343292,379.73295013)
\curveto(69.5234266,379.7929459)(69.50342662,379.84794585)(69.48343292,379.89795013)
\curveto(69.47342665,379.95794574)(69.45842666,380.01294568)(69.43843292,380.06295013)
\curveto(69.2184269,380.55294514)(68.87342725,380.90294479)(68.40343292,381.11295013)
\curveto(68.3234278,381.14294455)(68.24342788,381.16794453)(68.16343292,381.18795013)
\lineto(67.92343292,381.24795013)
\curveto(67.84342828,381.26794443)(67.75342837,381.27794442)(67.65343292,381.27795013)
\lineto(67.33843292,381.27795013)
\curveto(67.3184288,381.25794444)(67.27842884,381.24794445)(67.21843292,381.24795013)
\curveto(67.16842895,381.25794444)(67.123429,381.25794444)(67.08343292,381.24795013)
\lineto(66.84343292,381.18795013)
\curveto(66.77342935,381.17794452)(66.70342942,381.15794454)(66.63343292,381.12795013)
\curveto(66.03343009,380.86794483)(65.62843049,380.40294529)(65.41843292,379.73295013)
\curveto(65.38843073,379.65294604)(65.36843075,379.57294612)(65.35843292,379.49295013)
\curveto(65.34843077,379.41294628)(65.33343079,379.32794637)(65.31343292,379.23795013)
\lineto(65.31343292,379.08795013)
\curveto(65.30343082,379.04794665)(65.29843082,378.97794672)(65.29843292,378.87795013)
\curveto(65.29843082,378.64794705)(65.3184308,378.45294724)(65.35843292,378.29295013)
\curveto(65.37843074,378.22294747)(65.39343073,378.15794754)(65.40343292,378.09795013)
\curveto(65.41343071,378.03794766)(65.43343069,377.97294772)(65.46343292,377.90295013)
\curveto(65.57343055,377.62294807)(65.7184304,377.37794832)(65.89843292,377.16795013)
\curveto(66.07843004,376.96794873)(66.31342981,376.80794889)(66.60343292,376.68795013)
\lineto(66.84343292,376.59795013)
\lineto(67.08343292,376.53795013)
\curveto(67.13342899,376.51794918)(67.17342895,376.51294918)(67.20343292,376.52295013)
\curveto(67.24342888,376.53294916)(67.28842883,376.52794917)(67.33843292,376.50795013)
\curveto(67.36842875,376.4979492)(67.4234287,376.4929492)(67.50343292,376.49295013)
\curveto(67.58342854,376.4929492)(67.64342848,376.4979492)(67.68343292,376.50795013)
\curveto(67.79342833,376.52794917)(67.89842822,376.54294915)(67.99843292,376.55295013)
\curveto(68.09842802,376.56294913)(68.19342793,376.5929491)(68.28343292,376.64295013)
\curveto(68.81342731,376.84294885)(69.20342692,377.21794848)(69.45343292,377.76795013)
\curveto(69.49342663,377.86794783)(69.5234266,377.97294772)(69.54343292,378.08295013)
\lineto(69.63343292,378.41295013)
\curveto(69.63342649,378.4929472)(69.63842648,378.55294714)(69.64843292,378.59295013)
}
}
{
\newrgbcolor{curcolor}{0 0 0}
\pscustom[linestyle=none,fillstyle=solid,fillcolor=curcolor]
{
\newpath
\moveto(79.2430423,380.66295013)
\lineto(79.2430423,380.40795013)
\curveto(79.25303459,380.32794537)(79.2480346,380.25294544)(79.2280423,380.18295013)
\lineto(79.2280423,379.94295013)
\lineto(79.2280423,379.77795013)
\curveto(79.20803464,379.67794602)(79.19803465,379.57294612)(79.1980423,379.46295013)
\curveto(79.19803465,379.36294633)(79.18803466,379.26294643)(79.1680423,379.16295013)
\lineto(79.1680423,379.01295013)
\curveto(79.13803471,378.87294682)(79.11803473,378.73294696)(79.1080423,378.59295013)
\curveto(79.09803475,378.46294723)(79.07303477,378.33294736)(79.0330423,378.20295013)
\curveto(79.01303483,378.12294757)(78.99303485,378.03794766)(78.9730423,377.94795013)
\lineto(78.9130423,377.70795013)
\lineto(78.7930423,377.40795013)
\curveto(78.76303508,377.31794838)(78.72803512,377.22794847)(78.6880423,377.13795013)
\curveto(78.58803526,376.91794878)(78.45303539,376.70294899)(78.2830423,376.49295013)
\curveto(78.12303572,376.28294941)(77.9480359,376.11294958)(77.7580423,375.98295013)
\curveto(77.70803614,375.94294975)(77.6480362,375.90294979)(77.5780423,375.86295013)
\curveto(77.51803633,375.83294986)(77.45803639,375.7979499)(77.3980423,375.75795013)
\curveto(77.31803653,375.70794999)(77.22303662,375.66795003)(77.1130423,375.63795013)
\curveto(77.00303684,375.60795009)(76.89803695,375.57795012)(76.7980423,375.54795013)
\curveto(76.68803716,375.50795019)(76.57803727,375.48295021)(76.4680423,375.47295013)
\curveto(76.35803749,375.46295023)(76.2430376,375.44795025)(76.1230423,375.42795013)
\curveto(76.08303776,375.41795028)(76.03803781,375.41795028)(75.9880423,375.42795013)
\curveto(75.9480379,375.42795027)(75.90803794,375.42295027)(75.8680423,375.41295013)
\curveto(75.82803802,375.40295029)(75.77303807,375.3979503)(75.7030423,375.39795013)
\curveto(75.63303821,375.3979503)(75.58303826,375.40295029)(75.5530423,375.41295013)
\curveto(75.50303834,375.43295026)(75.45803839,375.43795026)(75.4180423,375.42795013)
\curveto(75.37803847,375.41795028)(75.3430385,375.41795028)(75.3130423,375.42795013)
\lineto(75.2230423,375.42795013)
\curveto(75.16303868,375.44795025)(75.09803875,375.46295023)(75.0280423,375.47295013)
\curveto(74.96803888,375.47295022)(74.90303894,375.47795022)(74.8330423,375.48795013)
\curveto(74.66303918,375.53795016)(74.50303934,375.58795011)(74.3530423,375.63795013)
\curveto(74.20303964,375.68795001)(74.05803979,375.75294994)(73.9180423,375.83295013)
\curveto(73.86803998,375.87294982)(73.81304003,375.90294979)(73.7530423,375.92295013)
\curveto(73.70304014,375.95294974)(73.65304019,375.98794971)(73.6030423,376.02795013)
\curveto(73.36304048,376.20794949)(73.16304068,376.42794927)(73.0030423,376.68795013)
\curveto(72.843041,376.94794875)(72.70304114,377.23294846)(72.5830423,377.54295013)
\curveto(72.52304132,377.68294801)(72.47804137,377.82294787)(72.4480423,377.96295013)
\curveto(72.41804143,378.11294758)(72.38304146,378.26794743)(72.3430423,378.42795013)
\curveto(72.32304152,378.53794716)(72.30804154,378.64794705)(72.2980423,378.75795013)
\curveto(72.28804156,378.86794683)(72.27304157,378.97794672)(72.2530423,379.08795013)
\curveto(72.2430416,379.12794657)(72.23804161,379.16794653)(72.2380423,379.20795013)
\curveto(72.2480416,379.24794645)(72.2480416,379.28794641)(72.2380423,379.32795013)
\curveto(72.22804162,379.37794632)(72.22304162,379.42794627)(72.2230423,379.47795013)
\lineto(72.2230423,379.64295013)
\curveto(72.20304164,379.692946)(72.19804165,379.74294595)(72.2080423,379.79295013)
\curveto(72.21804163,379.85294584)(72.21804163,379.90794579)(72.2080423,379.95795013)
\curveto(72.19804165,379.9979457)(72.19804165,380.04294565)(72.2080423,380.09295013)
\curveto(72.21804163,380.14294555)(72.21304163,380.1929455)(72.1930423,380.24295013)
\curveto(72.17304167,380.31294538)(72.16804168,380.38794531)(72.1780423,380.46795013)
\curveto(72.18804166,380.55794514)(72.19304165,380.64294505)(72.1930423,380.72295013)
\curveto(72.19304165,380.81294488)(72.18804166,380.91294478)(72.1780423,381.02295013)
\curveto(72.16804168,381.14294455)(72.17304167,381.24294445)(72.1930423,381.32295013)
\lineto(72.1930423,381.60795013)
\lineto(72.2380423,382.23795013)
\curveto(72.2480416,382.33794336)(72.25804159,382.43294326)(72.2680423,382.52295013)
\lineto(72.2980423,382.82295013)
\curveto(72.31804153,382.87294282)(72.32304152,382.92294277)(72.3130423,382.97295013)
\curveto(72.31304153,383.03294266)(72.32304152,383.08794261)(72.3430423,383.13795013)
\curveto(72.39304145,383.30794239)(72.43304141,383.47294222)(72.4630423,383.63295013)
\curveto(72.49304135,383.80294189)(72.5430413,383.96294173)(72.6130423,384.11295013)
\curveto(72.80304104,384.57294112)(73.02304082,384.94794075)(73.2730423,385.23795013)
\curveto(73.53304031,385.52794017)(73.89303995,385.77293992)(74.3530423,385.97295013)
\curveto(74.48303936,386.02293967)(74.61303923,386.05793964)(74.7430423,386.07795013)
\curveto(74.88303896,386.0979396)(75.02303882,386.12293957)(75.1630423,386.15295013)
\curveto(75.23303861,386.16293953)(75.29803855,386.16793953)(75.3580423,386.16795013)
\curveto(75.41803843,386.16793953)(75.48303836,386.17293952)(75.5530423,386.18295013)
\curveto(76.38303746,386.20293949)(77.05303679,386.05293964)(77.5630423,385.73295013)
\curveto(78.07303577,385.42294027)(78.45303539,384.98294071)(78.7030423,384.41295013)
\curveto(78.75303509,384.2929414)(78.79803505,384.16794153)(78.8380423,384.03795013)
\curveto(78.87803497,383.90794179)(78.92303492,383.77294192)(78.9730423,383.63295013)
\curveto(78.99303485,383.55294214)(79.00803484,383.46794223)(79.0180423,383.37795013)
\lineto(79.0780423,383.13795013)
\curveto(79.10803474,383.02794267)(79.12303472,382.91794278)(79.1230423,382.80795013)
\curveto(79.13303471,382.697943)(79.1480347,382.58794311)(79.1680423,382.47795013)
\curveto(79.18803466,382.42794327)(79.19303465,382.38294331)(79.1830423,382.34295013)
\curveto(79.18303466,382.30294339)(79.18803466,382.26294343)(79.1980423,382.22295013)
\curveto(79.20803464,382.17294352)(79.20803464,382.11794358)(79.1980423,382.05795013)
\curveto(79.19803465,382.00794369)(79.20303464,381.95794374)(79.2130423,381.90795013)
\lineto(79.2130423,381.77295013)
\curveto(79.23303461,381.71294398)(79.23303461,381.64294405)(79.2130423,381.56295013)
\curveto(79.20303464,381.4929442)(79.20803464,381.42794427)(79.2280423,381.36795013)
\curveto(79.23803461,381.33794436)(79.2430346,381.2979444)(79.2430423,381.24795013)
\lineto(79.2430423,381.12795013)
\lineto(79.2430423,380.66295013)
\moveto(77.6980423,378.33795013)
\curveto(77.79803605,378.65794704)(77.85803599,379.02294667)(77.8780423,379.43295013)
\curveto(77.89803595,379.84294585)(77.90803594,380.25294544)(77.9080423,380.66295013)
\curveto(77.90803594,381.0929446)(77.89803595,381.51294418)(77.8780423,381.92295013)
\curveto(77.85803599,382.33294336)(77.81303603,382.71794298)(77.7430423,383.07795013)
\curveto(77.67303617,383.43794226)(77.56303628,383.75794194)(77.4130423,384.03795013)
\curveto(77.27303657,384.32794137)(77.07803677,384.56294113)(76.8280423,384.74295013)
\curveto(76.66803718,384.85294084)(76.48803736,384.93294076)(76.2880423,384.98295013)
\curveto(76.08803776,385.04294065)(75.843038,385.07294062)(75.5530423,385.07295013)
\curveto(75.53303831,385.05294064)(75.49803835,385.04294065)(75.4480423,385.04295013)
\curveto(75.39803845,385.05294064)(75.35803849,385.05294064)(75.3280423,385.04295013)
\curveto(75.2480386,385.02294067)(75.17303867,385.00294069)(75.1030423,384.98295013)
\curveto(75.0430388,384.97294072)(74.97803887,384.95294074)(74.9080423,384.92295013)
\curveto(74.63803921,384.80294089)(74.41803943,384.63294106)(74.2480423,384.41295013)
\curveto(74.08803976,384.20294149)(73.95303989,383.95794174)(73.8430423,383.67795013)
\curveto(73.79304005,383.56794213)(73.75304009,383.44794225)(73.7230423,383.31795013)
\curveto(73.70304014,383.1979425)(73.67804017,383.07294262)(73.6480423,382.94295013)
\curveto(73.62804022,382.8929428)(73.61804023,382.83794286)(73.6180423,382.77795013)
\curveto(73.61804023,382.72794297)(73.61304023,382.67794302)(73.6030423,382.62795013)
\curveto(73.59304025,382.53794316)(73.58304026,382.44294325)(73.5730423,382.34295013)
\curveto(73.56304028,382.25294344)(73.55304029,382.15794354)(73.5430423,382.05795013)
\curveto(73.5430403,381.97794372)(73.53804031,381.8929438)(73.5280423,381.80295013)
\lineto(73.5280423,381.56295013)
\lineto(73.5280423,381.38295013)
\curveto(73.51804033,381.35294434)(73.51304033,381.31794438)(73.5130423,381.27795013)
\lineto(73.5130423,381.14295013)
\lineto(73.5130423,380.69295013)
\curveto(73.51304033,380.61294508)(73.50804034,380.52794517)(73.4980423,380.43795013)
\curveto(73.49804035,380.35794534)(73.50804034,380.28294541)(73.5280423,380.21295013)
\lineto(73.5280423,379.94295013)
\curveto(73.52804032,379.92294577)(73.52304032,379.8929458)(73.5130423,379.85295013)
\curveto(73.51304033,379.82294587)(73.51804033,379.7979459)(73.5280423,379.77795013)
\curveto(73.53804031,379.67794602)(73.5430403,379.57794612)(73.5430423,379.47795013)
\curveto(73.55304029,379.38794631)(73.56304028,379.28794641)(73.5730423,379.17795013)
\curveto(73.60304024,379.05794664)(73.61804023,378.93294676)(73.6180423,378.80295013)
\curveto(73.62804022,378.68294701)(73.65304019,378.56794713)(73.6930423,378.45795013)
\curveto(73.77304007,378.15794754)(73.85803999,377.8929478)(73.9480423,377.66295013)
\curveto(74.0480398,377.43294826)(74.19303965,377.21794848)(74.3830423,377.01795013)
\curveto(74.59303925,376.81794888)(74.85803899,376.66794903)(75.1780423,376.56795013)
\curveto(75.21803863,376.54794915)(75.25303859,376.53794916)(75.2830423,376.53795013)
\curveto(75.32303852,376.54794915)(75.36803848,376.54294915)(75.4180423,376.52295013)
\curveto(75.45803839,376.51294918)(75.52803832,376.50294919)(75.6280423,376.49295013)
\curveto(75.73803811,376.48294921)(75.82303802,376.48794921)(75.8830423,376.50795013)
\curveto(75.95303789,376.52794917)(76.02303782,376.53794916)(76.0930423,376.53795013)
\curveto(76.16303768,376.54794915)(76.22803762,376.56294913)(76.2880423,376.58295013)
\curveto(76.48803736,376.64294905)(76.66803718,376.72794897)(76.8280423,376.83795013)
\curveto(76.85803699,376.85794884)(76.88303696,376.87794882)(76.9030423,376.89795013)
\lineto(76.9630423,376.95795013)
\curveto(77.00303684,376.97794872)(77.05303679,377.01794868)(77.1130423,377.07795013)
\curveto(77.21303663,377.21794848)(77.29803655,377.34794835)(77.3680423,377.46795013)
\curveto(77.43803641,377.58794811)(77.50803634,377.73294796)(77.5780423,377.90295013)
\curveto(77.60803624,377.97294772)(77.62803622,378.04294765)(77.6380423,378.11295013)
\curveto(77.65803619,378.18294751)(77.67803617,378.25794744)(77.6980423,378.33795013)
}
}
{
\newrgbcolor{curcolor}{0 0 0}
\pscustom[linestyle=none,fillstyle=solid,fillcolor=curcolor]
{
\newpath
\moveto(87.59265167,380.66295013)
\lineto(87.59265167,380.40795013)
\curveto(87.60264397,380.32794537)(87.59764397,380.25294544)(87.57765167,380.18295013)
\lineto(87.57765167,379.94295013)
\lineto(87.57765167,379.77795013)
\curveto(87.55764401,379.67794602)(87.54764402,379.57294612)(87.54765167,379.46295013)
\curveto(87.54764402,379.36294633)(87.53764403,379.26294643)(87.51765167,379.16295013)
\lineto(87.51765167,379.01295013)
\curveto(87.48764408,378.87294682)(87.4676441,378.73294696)(87.45765167,378.59295013)
\curveto(87.44764412,378.46294723)(87.42264415,378.33294736)(87.38265167,378.20295013)
\curveto(87.36264421,378.12294757)(87.34264423,378.03794766)(87.32265167,377.94795013)
\lineto(87.26265167,377.70795013)
\lineto(87.14265167,377.40795013)
\curveto(87.11264446,377.31794838)(87.07764449,377.22794847)(87.03765167,377.13795013)
\curveto(86.93764463,376.91794878)(86.80264477,376.70294899)(86.63265167,376.49295013)
\curveto(86.4726451,376.28294941)(86.29764527,376.11294958)(86.10765167,375.98295013)
\curveto(86.05764551,375.94294975)(85.99764557,375.90294979)(85.92765167,375.86295013)
\curveto(85.8676457,375.83294986)(85.80764576,375.7979499)(85.74765167,375.75795013)
\curveto(85.6676459,375.70794999)(85.572646,375.66795003)(85.46265167,375.63795013)
\curveto(85.35264622,375.60795009)(85.24764632,375.57795012)(85.14765167,375.54795013)
\curveto(85.03764653,375.50795019)(84.92764664,375.48295021)(84.81765167,375.47295013)
\curveto(84.70764686,375.46295023)(84.59264698,375.44795025)(84.47265167,375.42795013)
\curveto(84.43264714,375.41795028)(84.38764718,375.41795028)(84.33765167,375.42795013)
\curveto(84.29764727,375.42795027)(84.25764731,375.42295027)(84.21765167,375.41295013)
\curveto(84.17764739,375.40295029)(84.12264745,375.3979503)(84.05265167,375.39795013)
\curveto(83.98264759,375.3979503)(83.93264764,375.40295029)(83.90265167,375.41295013)
\curveto(83.85264772,375.43295026)(83.80764776,375.43795026)(83.76765167,375.42795013)
\curveto(83.72764784,375.41795028)(83.69264788,375.41795028)(83.66265167,375.42795013)
\lineto(83.57265167,375.42795013)
\curveto(83.51264806,375.44795025)(83.44764812,375.46295023)(83.37765167,375.47295013)
\curveto(83.31764825,375.47295022)(83.25264832,375.47795022)(83.18265167,375.48795013)
\curveto(83.01264856,375.53795016)(82.85264872,375.58795011)(82.70265167,375.63795013)
\curveto(82.55264902,375.68795001)(82.40764916,375.75294994)(82.26765167,375.83295013)
\curveto(82.21764935,375.87294982)(82.16264941,375.90294979)(82.10265167,375.92295013)
\curveto(82.05264952,375.95294974)(82.00264957,375.98794971)(81.95265167,376.02795013)
\curveto(81.71264986,376.20794949)(81.51265006,376.42794927)(81.35265167,376.68795013)
\curveto(81.19265038,376.94794875)(81.05265052,377.23294846)(80.93265167,377.54295013)
\curveto(80.8726507,377.68294801)(80.82765074,377.82294787)(80.79765167,377.96295013)
\curveto(80.7676508,378.11294758)(80.73265084,378.26794743)(80.69265167,378.42795013)
\curveto(80.6726509,378.53794716)(80.65765091,378.64794705)(80.64765167,378.75795013)
\curveto(80.63765093,378.86794683)(80.62265095,378.97794672)(80.60265167,379.08795013)
\curveto(80.59265098,379.12794657)(80.58765098,379.16794653)(80.58765167,379.20795013)
\curveto(80.59765097,379.24794645)(80.59765097,379.28794641)(80.58765167,379.32795013)
\curveto(80.57765099,379.37794632)(80.572651,379.42794627)(80.57265167,379.47795013)
\lineto(80.57265167,379.64295013)
\curveto(80.55265102,379.692946)(80.54765102,379.74294595)(80.55765167,379.79295013)
\curveto(80.567651,379.85294584)(80.567651,379.90794579)(80.55765167,379.95795013)
\curveto(80.54765102,379.9979457)(80.54765102,380.04294565)(80.55765167,380.09295013)
\curveto(80.567651,380.14294555)(80.56265101,380.1929455)(80.54265167,380.24295013)
\curveto(80.52265105,380.31294538)(80.51765105,380.38794531)(80.52765167,380.46795013)
\curveto(80.53765103,380.55794514)(80.54265103,380.64294505)(80.54265167,380.72295013)
\curveto(80.54265103,380.81294488)(80.53765103,380.91294478)(80.52765167,381.02295013)
\curveto(80.51765105,381.14294455)(80.52265105,381.24294445)(80.54265167,381.32295013)
\lineto(80.54265167,381.60795013)
\lineto(80.58765167,382.23795013)
\curveto(80.59765097,382.33794336)(80.60765096,382.43294326)(80.61765167,382.52295013)
\lineto(80.64765167,382.82295013)
\curveto(80.6676509,382.87294282)(80.6726509,382.92294277)(80.66265167,382.97295013)
\curveto(80.66265091,383.03294266)(80.6726509,383.08794261)(80.69265167,383.13795013)
\curveto(80.74265083,383.30794239)(80.78265079,383.47294222)(80.81265167,383.63295013)
\curveto(80.84265073,383.80294189)(80.89265068,383.96294173)(80.96265167,384.11295013)
\curveto(81.15265042,384.57294112)(81.3726502,384.94794075)(81.62265167,385.23795013)
\curveto(81.88264969,385.52794017)(82.24264933,385.77293992)(82.70265167,385.97295013)
\curveto(82.83264874,386.02293967)(82.96264861,386.05793964)(83.09265167,386.07795013)
\curveto(83.23264834,386.0979396)(83.3726482,386.12293957)(83.51265167,386.15295013)
\curveto(83.58264799,386.16293953)(83.64764792,386.16793953)(83.70765167,386.16795013)
\curveto(83.7676478,386.16793953)(83.83264774,386.17293952)(83.90265167,386.18295013)
\curveto(84.73264684,386.20293949)(85.40264617,386.05293964)(85.91265167,385.73295013)
\curveto(86.42264515,385.42294027)(86.80264477,384.98294071)(87.05265167,384.41295013)
\curveto(87.10264447,384.2929414)(87.14764442,384.16794153)(87.18765167,384.03795013)
\curveto(87.22764434,383.90794179)(87.2726443,383.77294192)(87.32265167,383.63295013)
\curveto(87.34264423,383.55294214)(87.35764421,383.46794223)(87.36765167,383.37795013)
\lineto(87.42765167,383.13795013)
\curveto(87.45764411,383.02794267)(87.4726441,382.91794278)(87.47265167,382.80795013)
\curveto(87.48264409,382.697943)(87.49764407,382.58794311)(87.51765167,382.47795013)
\curveto(87.53764403,382.42794327)(87.54264403,382.38294331)(87.53265167,382.34295013)
\curveto(87.53264404,382.30294339)(87.53764403,382.26294343)(87.54765167,382.22295013)
\curveto(87.55764401,382.17294352)(87.55764401,382.11794358)(87.54765167,382.05795013)
\curveto(87.54764402,382.00794369)(87.55264402,381.95794374)(87.56265167,381.90795013)
\lineto(87.56265167,381.77295013)
\curveto(87.58264399,381.71294398)(87.58264399,381.64294405)(87.56265167,381.56295013)
\curveto(87.55264402,381.4929442)(87.55764401,381.42794427)(87.57765167,381.36795013)
\curveto(87.58764398,381.33794436)(87.59264398,381.2979444)(87.59265167,381.24795013)
\lineto(87.59265167,381.12795013)
\lineto(87.59265167,380.66295013)
\moveto(86.04765167,378.33795013)
\curveto(86.14764542,378.65794704)(86.20764536,379.02294667)(86.22765167,379.43295013)
\curveto(86.24764532,379.84294585)(86.25764531,380.25294544)(86.25765167,380.66295013)
\curveto(86.25764531,381.0929446)(86.24764532,381.51294418)(86.22765167,381.92295013)
\curveto(86.20764536,382.33294336)(86.16264541,382.71794298)(86.09265167,383.07795013)
\curveto(86.02264555,383.43794226)(85.91264566,383.75794194)(85.76265167,384.03795013)
\curveto(85.62264595,384.32794137)(85.42764614,384.56294113)(85.17765167,384.74295013)
\curveto(85.01764655,384.85294084)(84.83764673,384.93294076)(84.63765167,384.98295013)
\curveto(84.43764713,385.04294065)(84.19264738,385.07294062)(83.90265167,385.07295013)
\curveto(83.88264769,385.05294064)(83.84764772,385.04294065)(83.79765167,385.04295013)
\curveto(83.74764782,385.05294064)(83.70764786,385.05294064)(83.67765167,385.04295013)
\curveto(83.59764797,385.02294067)(83.52264805,385.00294069)(83.45265167,384.98295013)
\curveto(83.39264818,384.97294072)(83.32764824,384.95294074)(83.25765167,384.92295013)
\curveto(82.98764858,384.80294089)(82.7676488,384.63294106)(82.59765167,384.41295013)
\curveto(82.43764913,384.20294149)(82.30264927,383.95794174)(82.19265167,383.67795013)
\curveto(82.14264943,383.56794213)(82.10264947,383.44794225)(82.07265167,383.31795013)
\curveto(82.05264952,383.1979425)(82.02764954,383.07294262)(81.99765167,382.94295013)
\curveto(81.97764959,382.8929428)(81.9676496,382.83794286)(81.96765167,382.77795013)
\curveto(81.9676496,382.72794297)(81.96264961,382.67794302)(81.95265167,382.62795013)
\curveto(81.94264963,382.53794316)(81.93264964,382.44294325)(81.92265167,382.34295013)
\curveto(81.91264966,382.25294344)(81.90264967,382.15794354)(81.89265167,382.05795013)
\curveto(81.89264968,381.97794372)(81.88764968,381.8929438)(81.87765167,381.80295013)
\lineto(81.87765167,381.56295013)
\lineto(81.87765167,381.38295013)
\curveto(81.8676497,381.35294434)(81.86264971,381.31794438)(81.86265167,381.27795013)
\lineto(81.86265167,381.14295013)
\lineto(81.86265167,380.69295013)
\curveto(81.86264971,380.61294508)(81.85764971,380.52794517)(81.84765167,380.43795013)
\curveto(81.84764972,380.35794534)(81.85764971,380.28294541)(81.87765167,380.21295013)
\lineto(81.87765167,379.94295013)
\curveto(81.87764969,379.92294577)(81.8726497,379.8929458)(81.86265167,379.85295013)
\curveto(81.86264971,379.82294587)(81.8676497,379.7979459)(81.87765167,379.77795013)
\curveto(81.88764968,379.67794602)(81.89264968,379.57794612)(81.89265167,379.47795013)
\curveto(81.90264967,379.38794631)(81.91264966,379.28794641)(81.92265167,379.17795013)
\curveto(81.95264962,379.05794664)(81.9676496,378.93294676)(81.96765167,378.80295013)
\curveto(81.97764959,378.68294701)(82.00264957,378.56794713)(82.04265167,378.45795013)
\curveto(82.12264945,378.15794754)(82.20764936,377.8929478)(82.29765167,377.66295013)
\curveto(82.39764917,377.43294826)(82.54264903,377.21794848)(82.73265167,377.01795013)
\curveto(82.94264863,376.81794888)(83.20764836,376.66794903)(83.52765167,376.56795013)
\curveto(83.567648,376.54794915)(83.60264797,376.53794916)(83.63265167,376.53795013)
\curveto(83.6726479,376.54794915)(83.71764785,376.54294915)(83.76765167,376.52295013)
\curveto(83.80764776,376.51294918)(83.87764769,376.50294919)(83.97765167,376.49295013)
\curveto(84.08764748,376.48294921)(84.1726474,376.48794921)(84.23265167,376.50795013)
\curveto(84.30264727,376.52794917)(84.3726472,376.53794916)(84.44265167,376.53795013)
\curveto(84.51264706,376.54794915)(84.57764699,376.56294913)(84.63765167,376.58295013)
\curveto(84.83764673,376.64294905)(85.01764655,376.72794897)(85.17765167,376.83795013)
\curveto(85.20764636,376.85794884)(85.23264634,376.87794882)(85.25265167,376.89795013)
\lineto(85.31265167,376.95795013)
\curveto(85.35264622,376.97794872)(85.40264617,377.01794868)(85.46265167,377.07795013)
\curveto(85.56264601,377.21794848)(85.64764592,377.34794835)(85.71765167,377.46795013)
\curveto(85.78764578,377.58794811)(85.85764571,377.73294796)(85.92765167,377.90295013)
\curveto(85.95764561,377.97294772)(85.97764559,378.04294765)(85.98765167,378.11295013)
\curveto(86.00764556,378.18294751)(86.02764554,378.25794744)(86.04765167,378.33795013)
}
}
{
\newrgbcolor{curcolor}{0 0 0}
\pscustom[linestyle=none,fillstyle=solid,fillcolor=curcolor]
{
\newpath
\moveto(783.95647461,368.68897308)
\curveto(784.00646499,368.55897282)(783.98646501,368.45897292)(783.89647461,368.38897308)
\curveto(783.84646515,368.35897302)(783.78146521,368.33897304)(783.70147461,368.32897308)
\lineto(783.47647461,368.32897308)
\lineto(782.99647461,368.32897308)
\curveto(782.83646616,368.32897305)(782.71146628,368.36397302)(782.62147461,368.43397308)
\curveto(782.54146645,368.4839729)(782.48646651,368.55897282)(782.45647461,368.65897308)
\lineto(782.39647461,368.98897308)
\curveto(782.38646661,369.02897235)(782.38146661,369.06397232)(782.38147461,369.09397308)
\lineto(782.38147461,369.19897308)
\curveto(782.36146663,369.24897213)(782.35646664,369.29397209)(782.36647461,369.33397308)
\curveto(782.37646662,369.37397201)(782.37646662,369.41397197)(782.36647461,369.45397308)
\curveto(782.35646664,369.51397187)(782.35146664,369.57397181)(782.35147461,369.63397308)
\lineto(782.35147461,369.81397308)
\lineto(782.30647461,370.48897308)
\curveto(782.28646671,370.55897082)(782.27646672,370.62897075)(782.27647461,370.69897308)
\curveto(782.27646672,370.76897061)(782.26646673,370.84397054)(782.24647461,370.92397308)
\curveto(782.1964668,371.10397028)(782.15646684,371.2839701)(782.12647461,371.46397308)
\curveto(782.10646689,371.64396974)(782.06146693,371.81396957)(781.99147461,371.97397308)
\curveto(781.80146719,372.39396899)(781.48646751,372.67396871)(781.04647461,372.81397308)
\curveto(780.91646808,372.86396852)(780.77146822,372.88896849)(780.61147461,372.88897308)
\curveto(780.46146853,372.89896848)(780.30146869,372.90396848)(780.13147461,372.90397308)
\lineto(777.37147461,372.90397308)
\curveto(777.30147169,372.8839685)(777.23647176,372.86396852)(777.17647461,372.84397308)
\curveto(777.12647187,372.83396855)(777.08147191,372.80396858)(777.04147461,372.75397308)
\curveto(776.97147202,372.65396873)(776.93647206,372.48896889)(776.93647461,372.25897308)
\curveto(776.94647205,372.03896934)(776.95147204,371.84396954)(776.95147461,371.67397308)
\lineto(776.95147461,369.49897308)
\curveto(776.95147204,369.35897202)(776.95647204,369.1839722)(776.96647461,368.97397308)
\curveto(776.97647202,368.77397261)(776.95647204,368.62397276)(776.90647461,368.52397308)
\curveto(776.88647211,368.45397293)(776.84647215,368.40897297)(776.78647461,368.38897308)
\curveto(776.74647225,368.36897301)(776.70647229,368.35897302)(776.66647461,368.35897308)
\curveto(776.63647236,368.35897302)(776.5964724,368.34897303)(776.54647461,368.32897308)
\curveto(776.50647249,368.31897306)(776.46147253,368.31397307)(776.41147461,368.31397308)
\curveto(776.36147263,368.32397306)(776.31147268,368.32897305)(776.26147461,368.32897308)
\lineto(775.93147461,368.32897308)
\curveto(775.83147316,368.33897304)(775.74647325,368.36897301)(775.67647461,368.41897308)
\curveto(775.5964734,368.46897291)(775.55647344,368.55897282)(775.55647461,368.68897308)
\lineto(775.55647461,369.09397308)
\lineto(775.55647461,378.21397308)
\curveto(775.55647344,378.32396306)(775.55147344,378.43896294)(775.54147461,378.55897308)
\curveto(775.54147345,378.6789627)(775.56647343,378.77396261)(775.61647461,378.84397308)
\curveto(775.65647334,378.90396248)(775.73147326,378.95396243)(775.84147461,378.99397308)
\curveto(775.86147313,379.00396238)(775.88147311,379.00396238)(775.90147461,378.99397308)
\curveto(775.92147307,378.99396239)(775.94147305,378.99896238)(775.96147461,379.00897308)
\lineto(780.31147461,379.00897308)
\curveto(780.38146861,379.00896237)(780.45646854,379.00896237)(780.53647461,379.00897308)
\curveto(780.61646838,379.01896236)(780.68646831,379.01896236)(780.74647461,379.00897308)
\lineto(780.91147461,379.00897308)
\curveto(780.97146802,378.99896238)(781.03146796,378.98896239)(781.09147461,378.97897308)
\curveto(781.15146784,378.9789624)(781.21646778,378.97396241)(781.28647461,378.96397308)
\curveto(781.36646763,378.94396244)(781.44646755,378.92896245)(781.52647461,378.91897308)
\curveto(781.61646738,378.90896247)(781.70146729,378.89396249)(781.78147461,378.87397308)
\curveto(781.97146702,378.81396257)(782.14646685,378.74896263)(782.30647461,378.67897308)
\curveto(782.46646653,378.60896277)(782.61646638,378.52396286)(782.75647461,378.42397308)
\curveto(783.00646599,378.25396313)(783.20646579,378.04396334)(783.35647461,377.79397308)
\curveto(783.51646548,377.55396383)(783.64646535,377.26896411)(783.74647461,376.93897308)
\curveto(783.76646523,376.85896452)(783.77646522,376.77396461)(783.77647461,376.68397308)
\curveto(783.78646521,376.60396478)(783.80146519,376.52396486)(783.82147461,376.44397308)
\lineto(783.82147461,376.29397308)
\curveto(783.83146516,376.24396514)(783.83146516,376.1839652)(783.82147461,376.11397308)
\curveto(783.82146517,376.05396533)(783.81646518,375.99896538)(783.80647461,375.94897308)
\lineto(783.80647461,375.78397308)
\curveto(783.78646521,375.70396568)(783.77146522,375.62896575)(783.76147461,375.55897308)
\curveto(783.76146523,375.48896589)(783.75146524,375.41896596)(783.73147461,375.34897308)
\curveto(783.68146531,375.19896618)(783.63146536,375.05396633)(783.58147461,374.91397308)
\curveto(783.54146545,374.7839666)(783.48146551,374.65896672)(783.40147461,374.53897308)
\curveto(783.37146562,374.48896689)(783.33646566,374.44396694)(783.29647461,374.40397308)
\curveto(783.26646573,374.36396702)(783.23646576,374.31896706)(783.20647461,374.26897308)
\lineto(783.17647461,374.23897308)
\curveto(783.16646583,374.23896714)(783.15646584,374.23396715)(783.14647461,374.22397308)
\lineto(783.07147461,374.14897308)
\curveto(783.05146594,374.11896726)(783.03146596,374.09396729)(783.01147461,374.07397308)
\curveto(782.93146606,374.01396737)(782.85646614,373.95396743)(782.78647461,373.89397308)
\curveto(782.71646628,373.84396754)(782.64146635,373.79396759)(782.56147461,373.74397308)
\curveto(782.51146648,373.71396767)(782.46646653,373.6789677)(782.42647461,373.63897308)
\curveto(782.38646661,373.60896777)(782.36146663,373.56396782)(782.35147461,373.50397308)
\curveto(782.34146665,373.44396794)(782.36146663,373.39396799)(782.41147461,373.35397308)
\curveto(782.47146652,373.31396807)(782.52146647,373.2839681)(782.56147461,373.26397308)
\curveto(782.67146632,373.19396819)(782.77146622,373.11896826)(782.86147461,373.03897308)
\curveto(782.96146603,372.95896842)(783.04646595,372.86396852)(783.11647461,372.75397308)
\curveto(783.22646577,372.61396877)(783.30646569,372.45396893)(783.35647461,372.27397308)
\curveto(783.40646559,372.10396928)(783.45646554,371.91896946)(783.50647461,371.71897308)
\lineto(783.53647461,371.47897308)
\curveto(783.54646545,371.40896997)(783.55646544,371.33397005)(783.56647461,371.25397308)
\curveto(783.58646541,371.1839702)(783.5914654,371.11397027)(783.58147461,371.04397308)
\curveto(783.57146542,370.97397041)(783.57646542,370.90397048)(783.59647461,370.83397308)
\lineto(783.59647461,370.69897308)
\curveto(783.61646538,370.62897075)(783.62146537,370.55397083)(783.61147461,370.47397308)
\curveto(783.60146539,370.39397099)(783.60646539,370.31397107)(783.62647461,370.23397308)
\curveto(783.63646536,370.19397119)(783.63646536,370.15397123)(783.62647461,370.11397308)
\curveto(783.62646537,370.0839713)(783.63146536,370.04397134)(783.64147461,369.99397308)
\curveto(783.66146533,369.89397149)(783.67646532,369.78897159)(783.68647461,369.67897308)
\curveto(783.6964653,369.5789718)(783.71646528,369.4839719)(783.74647461,369.39397308)
\curveto(783.76646523,369.33397205)(783.77646522,369.27397211)(783.77647461,369.21397308)
\curveto(783.78646521,369.16397222)(783.80146519,369.10897227)(783.82147461,369.04897308)
\lineto(783.95647461,368.68897308)
\moveto(782.14147461,374.91397308)
\curveto(782.21146678,375.02396636)(782.26146673,375.13896624)(782.29147461,375.25897308)
\curveto(782.33146666,375.378966)(782.36646663,375.50896587)(782.39647461,375.64897308)
\lineto(782.39647461,375.78397308)
\curveto(782.42646657,375.92396546)(782.43146656,376.07396531)(782.41147461,376.23397308)
\curveto(782.3914666,376.40396498)(782.36146663,376.54396484)(782.32147461,376.65397308)
\curveto(782.16146683,377.15396423)(781.84646715,377.49896388)(781.37647461,377.68897308)
\curveto(781.17646782,377.76896361)(780.94146805,377.81396357)(780.67147461,377.82397308)
\curveto(780.41146858,377.83396355)(780.14146885,377.83896354)(779.86147461,377.83897308)
\lineto(777.38647461,377.83897308)
\curveto(777.36647163,377.82896355)(777.34147165,377.82396356)(777.31147461,377.82397308)
\curveto(777.2914717,377.82396356)(777.26647173,377.81896356)(777.23647461,377.80897308)
\curveto(777.11647188,377.7789636)(777.03647196,377.71396367)(776.99647461,377.61397308)
\curveto(776.95647204,377.52396386)(776.93647206,377.39896398)(776.93647461,377.23897308)
\curveto(776.94647205,377.0789643)(776.95147204,376.93396445)(776.95147461,376.80397308)
\lineto(776.95147461,375.07897308)
\curveto(776.95147204,374.92896645)(776.94647205,374.76896661)(776.93647461,374.59897308)
\curveto(776.93647206,374.43896694)(776.97147202,374.31396707)(777.04147461,374.22397308)
\curveto(777.0914719,374.15396723)(777.16647183,374.10896727)(777.26647461,374.08897308)
\curveto(777.36647163,374.0789673)(777.47647152,374.07396731)(777.59647461,374.07397308)
\lineto(778.52647461,374.07397308)
\curveto(778.91647008,374.07396731)(779.2964697,374.06896731)(779.66647461,374.05897308)
\curveto(780.03646896,374.05896732)(780.37646862,374.0789673)(780.68647461,374.11897308)
\curveto(781.00646799,374.16896721)(781.2914677,374.25396713)(781.54147461,374.37397308)
\curveto(781.7914672,374.49396689)(781.991467,374.67396671)(782.14147461,374.91397308)
}
}
{
\newrgbcolor{curcolor}{0 0 0}
\pscustom[linestyle=none,fillstyle=solid,fillcolor=curcolor]
{
\newpath
\moveto(792.33467773,372.48397308)
\curveto(792.35467005,372.383969)(792.35467005,372.26896911)(792.33467773,372.13897308)
\curveto(792.32467008,372.01896936)(792.29467011,371.93396945)(792.24467773,371.88397308)
\curveto(792.19467021,371.84396954)(792.11967028,371.81396957)(792.01967773,371.79397308)
\curveto(791.92967047,371.7839696)(791.82467058,371.7789696)(791.70467773,371.77897308)
\lineto(791.34467773,371.77897308)
\curveto(791.22467118,371.78896959)(791.11967128,371.79396959)(791.02967773,371.79397308)
\lineto(787.18967773,371.79397308)
\curveto(787.10967529,371.79396959)(787.02967537,371.78896959)(786.94967773,371.77897308)
\curveto(786.86967553,371.7789696)(786.8046756,371.76396962)(786.75467773,371.73397308)
\curveto(786.71467569,371.71396967)(786.67467573,371.67396971)(786.63467773,371.61397308)
\curveto(786.61467579,371.5839698)(786.59467581,371.53896984)(786.57467773,371.47897308)
\curveto(786.55467585,371.42896995)(786.55467585,371.37897)(786.57467773,371.32897308)
\curveto(786.58467582,371.2789701)(786.58967581,371.23397015)(786.58967773,371.19397308)
\curveto(786.58967581,371.15397023)(786.59467581,371.11397027)(786.60467773,371.07397308)
\curveto(786.62467578,370.99397039)(786.64467576,370.90897047)(786.66467773,370.81897308)
\curveto(786.68467572,370.73897064)(786.71467569,370.65897072)(786.75467773,370.57897308)
\curveto(786.98467542,370.03897134)(787.36467504,369.65397173)(787.89467773,369.42397308)
\curveto(787.95467445,369.39397199)(788.01967438,369.36897201)(788.08967773,369.34897308)
\lineto(788.29967773,369.28897308)
\curveto(788.32967407,369.2789721)(788.37967402,369.27397211)(788.44967773,369.27397308)
\curveto(788.58967381,369.23397215)(788.77467363,369.21397217)(789.00467773,369.21397308)
\curveto(789.23467317,369.21397217)(789.41967298,369.23397215)(789.55967773,369.27397308)
\curveto(789.6996727,369.31397207)(789.82467258,369.35397203)(789.93467773,369.39397308)
\curveto(790.05467235,369.44397194)(790.16467224,369.50397188)(790.26467773,369.57397308)
\curveto(790.37467203,369.64397174)(790.46967193,369.72397166)(790.54967773,369.81397308)
\curveto(790.62967177,369.91397147)(790.6996717,370.01897136)(790.75967773,370.12897308)
\curveto(790.81967158,370.22897115)(790.86967153,370.33397105)(790.90967773,370.44397308)
\curveto(790.95967144,370.55397083)(791.03967136,370.63397075)(791.14967773,370.68397308)
\curveto(791.18967121,370.70397068)(791.25467115,370.71897066)(791.34467773,370.72897308)
\curveto(791.43467097,370.73897064)(791.52467088,370.73897064)(791.61467773,370.72897308)
\curveto(791.7046707,370.72897065)(791.78967061,370.72397066)(791.86967773,370.71397308)
\curveto(791.94967045,370.70397068)(792.0046704,370.6839707)(792.03467773,370.65397308)
\curveto(792.13467027,370.5839708)(792.15967024,370.46897091)(792.10967773,370.30897308)
\curveto(792.02967037,370.03897134)(791.92467048,369.79897158)(791.79467773,369.58897308)
\curveto(791.59467081,369.26897211)(791.36467104,369.00397238)(791.10467773,368.79397308)
\curveto(790.85467155,368.59397279)(790.53467187,368.42897295)(790.14467773,368.29897308)
\curveto(790.04467236,368.25897312)(789.94467246,368.23397315)(789.84467773,368.22397308)
\curveto(789.74467266,368.20397318)(789.63967276,368.1839732)(789.52967773,368.16397308)
\curveto(789.47967292,368.15397323)(789.42967297,368.14897323)(789.37967773,368.14897308)
\curveto(789.33967306,368.14897323)(789.29467311,368.14397324)(789.24467773,368.13397308)
\lineto(789.09467773,368.13397308)
\curveto(789.04467336,368.12397326)(788.98467342,368.11897326)(788.91467773,368.11897308)
\curveto(788.85467355,368.11897326)(788.8046736,368.12397326)(788.76467773,368.13397308)
\lineto(788.62967773,368.13397308)
\curveto(788.57967382,368.14397324)(788.53467387,368.14897323)(788.49467773,368.14897308)
\curveto(788.45467395,368.14897323)(788.41467399,368.15397323)(788.37467773,368.16397308)
\curveto(788.32467408,368.17397321)(788.26967413,368.1839732)(788.20967773,368.19397308)
\curveto(788.14967425,368.19397319)(788.09467431,368.19897318)(788.04467773,368.20897308)
\curveto(787.95467445,368.22897315)(787.86467454,368.25397313)(787.77467773,368.28397308)
\curveto(787.68467472,368.30397308)(787.5996748,368.32897305)(787.51967773,368.35897308)
\curveto(787.47967492,368.378973)(787.44467496,368.38897299)(787.41467773,368.38897308)
\curveto(787.38467502,368.39897298)(787.34967505,368.41397297)(787.30967773,368.43397308)
\curveto(787.15967524,368.50397288)(786.9996754,368.58897279)(786.82967773,368.68897308)
\curveto(786.53967586,368.8789725)(786.28967611,369.10897227)(786.07967773,369.37897308)
\curveto(785.87967652,369.65897172)(785.70967669,369.96897141)(785.56967773,370.30897308)
\curveto(785.51967688,370.41897096)(785.47967692,370.53397085)(785.44967773,370.65397308)
\curveto(785.42967697,370.77397061)(785.399677,370.89397049)(785.35967773,371.01397308)
\curveto(785.34967705,371.05397033)(785.34467706,371.08897029)(785.34467773,371.11897308)
\curveto(785.34467706,371.14897023)(785.33967706,371.18897019)(785.32967773,371.23897308)
\curveto(785.30967709,371.31897006)(785.29467711,371.40396998)(785.28467773,371.49397308)
\curveto(785.27467713,371.5839698)(785.25967714,371.67396971)(785.23967773,371.76397308)
\lineto(785.23967773,371.97397308)
\curveto(785.22967717,372.01396937)(785.21967718,372.06896931)(785.20967773,372.13897308)
\curveto(785.20967719,372.21896916)(785.21467719,372.2839691)(785.22467773,372.33397308)
\lineto(785.22467773,372.49897308)
\curveto(785.24467716,372.54896883)(785.24967715,372.59896878)(785.23967773,372.64897308)
\curveto(785.23967716,372.70896867)(785.24467716,372.76396862)(785.25467773,372.81397308)
\curveto(785.29467711,372.97396841)(785.32467708,373.13396825)(785.34467773,373.29397308)
\curveto(785.37467703,373.45396793)(785.41967698,373.60396778)(785.47967773,373.74397308)
\curveto(785.52967687,373.85396753)(785.57467683,373.96396742)(785.61467773,374.07397308)
\curveto(785.66467674,374.19396719)(785.71967668,374.30896707)(785.77967773,374.41897308)
\curveto(785.9996764,374.76896661)(786.24967615,375.06896631)(786.52967773,375.31897308)
\curveto(786.80967559,375.5789658)(787.15467525,375.79396559)(787.56467773,375.96397308)
\curveto(787.68467472,376.01396537)(787.8046746,376.04896533)(787.92467773,376.06897308)
\curveto(788.05467435,376.09896528)(788.18967421,376.12896525)(788.32967773,376.15897308)
\curveto(788.37967402,376.16896521)(788.42467398,376.17396521)(788.46467773,376.17397308)
\curveto(788.5046739,376.1839652)(788.54967385,376.18896519)(788.59967773,376.18897308)
\curveto(788.61967378,376.19896518)(788.64467376,376.19896518)(788.67467773,376.18897308)
\curveto(788.7046737,376.1789652)(788.72967367,376.1839652)(788.74967773,376.20397308)
\curveto(789.16967323,376.21396517)(789.53467287,376.16896521)(789.84467773,376.06897308)
\curveto(790.15467225,375.9789654)(790.43467197,375.85396553)(790.68467773,375.69397308)
\curveto(790.73467167,375.67396571)(790.77467163,375.64396574)(790.80467773,375.60397308)
\curveto(790.83467157,375.57396581)(790.86967153,375.54896583)(790.90967773,375.52897308)
\curveto(790.98967141,375.46896591)(791.06967133,375.39896598)(791.14967773,375.31897308)
\curveto(791.23967116,375.23896614)(791.31467109,375.15896622)(791.37467773,375.07897308)
\curveto(791.53467087,374.86896651)(791.66967073,374.66896671)(791.77967773,374.47897308)
\curveto(791.84967055,374.36896701)(791.9046705,374.24896713)(791.94467773,374.11897308)
\curveto(791.98467042,373.98896739)(792.02967037,373.85896752)(792.07967773,373.72897308)
\curveto(792.12967027,373.59896778)(792.16467024,373.46396792)(792.18467773,373.32397308)
\curveto(792.21467019,373.1839682)(792.24967015,373.04396834)(792.28967773,372.90397308)
\curveto(792.2996701,372.83396855)(792.3046701,372.76396862)(792.30467773,372.69397308)
\lineto(792.33467773,372.48397308)
\moveto(790.87967773,372.99397308)
\curveto(790.90967149,373.03396835)(790.93467147,373.0839683)(790.95467773,373.14397308)
\curveto(790.97467143,373.21396817)(790.97467143,373.2839681)(790.95467773,373.35397308)
\curveto(790.89467151,373.57396781)(790.80967159,373.7789676)(790.69967773,373.96897308)
\curveto(790.55967184,374.19896718)(790.404672,374.39396699)(790.23467773,374.55397308)
\curveto(790.06467234,374.71396667)(789.84467256,374.84896653)(789.57467773,374.95897308)
\curveto(789.5046729,374.9789664)(789.43467297,374.99396639)(789.36467773,375.00397308)
\curveto(789.29467311,375.02396636)(789.21967318,375.04396634)(789.13967773,375.06397308)
\curveto(789.05967334,375.0839663)(788.97467343,375.09396629)(788.88467773,375.09397308)
\lineto(788.62967773,375.09397308)
\curveto(788.5996738,375.07396631)(788.56467384,375.06396632)(788.52467773,375.06397308)
\curveto(788.48467392,375.07396631)(788.44967395,375.07396631)(788.41967773,375.06397308)
\lineto(788.17967773,375.00397308)
\curveto(788.10967429,374.99396639)(788.03967436,374.9789664)(787.96967773,374.95897308)
\curveto(787.67967472,374.83896654)(787.44467496,374.68896669)(787.26467773,374.50897308)
\curveto(787.09467531,374.32896705)(786.93967546,374.10396728)(786.79967773,373.83397308)
\curveto(786.76967563,373.7839676)(786.73967566,373.71896766)(786.70967773,373.63897308)
\curveto(786.67967572,373.56896781)(786.65467575,373.48896789)(786.63467773,373.39897308)
\curveto(786.61467579,373.30896807)(786.60967579,373.22396816)(786.61967773,373.14397308)
\curveto(786.62967577,373.06396832)(786.66467574,373.00396838)(786.72467773,372.96397308)
\curveto(786.8046756,372.90396848)(786.93967546,372.87396851)(787.12967773,372.87397308)
\curveto(787.32967507,372.8839685)(787.4996749,372.88896849)(787.63967773,372.88897308)
\lineto(789.91967773,372.88897308)
\curveto(790.06967233,372.88896849)(790.24967215,372.8839685)(790.45967773,372.87397308)
\curveto(790.66967173,372.87396851)(790.80967159,372.91396847)(790.87967773,372.99397308)
}
}
{
\newrgbcolor{curcolor}{0 0 0}
\pscustom[linestyle=none,fillstyle=solid,fillcolor=curcolor]
{
\newpath
\moveto(800.25631836,375.93397308)
\curveto(800.32631076,375.8839655)(800.36131072,375.80896557)(800.36131836,375.70897308)
\curveto(800.37131071,375.60896577)(800.37631071,375.50396588)(800.37631836,375.39397308)
\lineto(800.37631836,369.12397308)
\lineto(800.37631836,368.52397308)
\curveto(800.35631073,368.47397291)(800.35131073,368.42397296)(800.36131836,368.37397308)
\curveto(800.37131071,368.33397305)(800.36631072,368.28897309)(800.34631836,368.23897308)
\curveto(800.32631076,368.13897324)(800.31131077,368.03897334)(800.30131836,367.93897308)
\curveto(800.30131078,367.82897355)(800.2863108,367.72397366)(800.25631836,367.62397308)
\curveto(800.22631086,367.51397387)(800.19631089,367.40897397)(800.16631836,367.30897308)
\curveto(800.14631094,367.20897417)(800.11131097,367.10897427)(800.06131836,367.00897308)
\curveto(799.96131112,366.74897463)(799.83131125,366.51397487)(799.67131836,366.30397308)
\curveto(799.52131156,366.09397529)(799.34131174,365.91897546)(799.13131836,365.77897308)
\curveto(798.96131212,365.65897572)(798.7813123,365.56397582)(798.59131836,365.49397308)
\curveto(798.40131268,365.41397597)(798.19631289,365.33897604)(797.97631836,365.26897308)
\curveto(797.8863132,365.24897613)(797.79631329,365.23897614)(797.70631836,365.23897308)
\curveto(797.61631347,365.22897615)(797.52631356,365.21397617)(797.43631836,365.19397308)
\lineto(797.34631836,365.19397308)
\curveto(797.32631376,365.1839762)(797.30631378,365.1789762)(797.28631836,365.17897308)
\curveto(797.23631385,365.16897621)(797.1863139,365.16897621)(797.13631836,365.17897308)
\curveto(797.09631399,365.18897619)(797.05131403,365.1839762)(797.00131836,365.16397308)
\curveto(796.93131415,365.14397624)(796.82131426,365.13897624)(796.67131836,365.14897308)
\curveto(796.53131455,365.14897623)(796.43131465,365.15897622)(796.37131836,365.17897308)
\curveto(796.34131474,365.1789762)(796.31131477,365.1839762)(796.28131836,365.19397308)
\lineto(796.22131836,365.19397308)
\curveto(796.13131495,365.21397617)(796.04131504,365.22897615)(795.95131836,365.23897308)
\curveto(795.86131522,365.23897614)(795.77631531,365.24897613)(795.69631836,365.26897308)
\curveto(795.61631547,365.28897609)(795.53631555,365.31397607)(795.45631836,365.34397308)
\curveto(795.37631571,365.36397602)(795.29631579,365.38897599)(795.21631836,365.41897308)
\curveto(794.89631619,365.54897583)(794.62631646,365.69397569)(794.40631836,365.85397308)
\curveto(794.19631689,366.01397537)(794.00631708,366.23897514)(793.83631836,366.52897308)
\curveto(793.81631727,366.54897483)(793.80131728,366.57397481)(793.79131836,366.60397308)
\curveto(793.79131729,366.62397476)(793.7813173,366.64897473)(793.76131836,366.67897308)
\curveto(793.73131735,366.75897462)(793.69631739,366.87397451)(793.65631836,367.02397308)
\curveto(793.62631746,367.16397422)(793.65631743,367.26897411)(793.74631836,367.33897308)
\curveto(793.80631728,367.38897399)(793.8863172,367.41397397)(793.98631836,367.41397308)
\lineto(794.31631836,367.41397308)
\lineto(794.48131836,367.41397308)
\curveto(794.54131654,367.41397397)(794.59631649,367.40397398)(794.64631836,367.38397308)
\curveto(794.73631635,367.35397403)(794.80131628,367.30397408)(794.84131836,367.23397308)
\curveto(794.8813162,367.16397422)(794.92631616,367.08897429)(794.97631836,367.00897308)
\lineto(795.09631836,366.82897308)
\curveto(795.14631594,366.75897462)(795.19631589,366.70397468)(795.24631836,366.66397308)
\curveto(795.49631559,366.47397491)(795.79631529,366.33397505)(796.14631836,366.24397308)
\curveto(796.20631488,366.22397516)(796.26631482,366.21397517)(796.32631836,366.21397308)
\curveto(796.39631469,366.20397518)(796.46131462,366.18897519)(796.52131836,366.16897308)
\lineto(796.61131836,366.16897308)
\curveto(796.6813144,366.14897523)(796.76631432,366.13897524)(796.86631836,366.13897308)
\curveto(796.96631412,366.13897524)(797.05631403,366.14897523)(797.13631836,366.16897308)
\curveto(797.16631392,366.1789752)(797.20631388,366.1839752)(797.25631836,366.18397308)
\curveto(797.35631373,366.20397518)(797.45131363,366.22397516)(797.54131836,366.24397308)
\curveto(797.63131345,366.25397513)(797.71631337,366.2789751)(797.79631836,366.31897308)
\curveto(798.086313,366.43897494)(798.32131276,366.60397478)(798.50131836,366.81397308)
\curveto(798.69131239,367.01397437)(798.84631224,367.25897412)(798.96631836,367.54897308)
\curveto(799.00631208,367.63897374)(799.03131205,367.73397365)(799.04131836,367.83397308)
\curveto(799.06131202,367.93397345)(799.086312,368.03897334)(799.11631836,368.14897308)
\curveto(799.13631195,368.19897318)(799.14631194,368.24897313)(799.14631836,368.29897308)
\curveto(799.14631194,368.34897303)(799.15131193,368.39897298)(799.16131836,368.44897308)
\curveto(799.17131191,368.4789729)(799.17631191,368.52897285)(799.17631836,368.59897308)
\curveto(799.19631189,368.6789727)(799.19631189,368.76397262)(799.17631836,368.85397308)
\curveto(799.16631192,368.90397248)(799.16131192,368.94897243)(799.16131836,368.98897308)
\curveto(799.17131191,369.02897235)(799.16631192,369.06397232)(799.14631836,369.09397308)
\curveto(799.12631196,369.11397227)(799.11131197,369.12397226)(799.10131836,369.12397308)
\lineto(799.05631836,369.16897308)
\curveto(798.95631213,369.16897221)(798.8813122,369.13897224)(798.83131836,369.07897308)
\curveto(798.79131229,369.02897235)(798.74131234,368.9839724)(798.68131836,368.94397308)
\lineto(798.44131836,368.73397308)
\curveto(798.36131272,368.67397271)(798.27131281,368.61897276)(798.17131836,368.56897308)
\curveto(798.03131305,368.4789729)(797.85631323,368.40397298)(797.64631836,368.34397308)
\curveto(797.43631365,368.29397309)(797.21631387,368.25897312)(796.98631836,368.23897308)
\curveto(796.75631433,368.21897316)(796.52631456,368.22397316)(796.29631836,368.25397308)
\curveto(796.06631502,368.27397311)(795.85631523,368.31397307)(795.66631836,368.37397308)
\curveto(794.72631636,368.6839727)(794.06631702,369.2789721)(793.68631836,370.15897308)
\curveto(793.63631745,370.25897112)(793.59631749,370.35397103)(793.56631836,370.44397308)
\curveto(793.53631755,370.54397084)(793.50131758,370.64897073)(793.46131836,370.75897308)
\curveto(793.44131764,370.80897057)(793.43131765,370.85397053)(793.43131836,370.89397308)
\curveto(793.43131765,370.93397045)(793.42131766,370.9789704)(793.40131836,371.02897308)
\curveto(793.3813177,371.09897028)(793.36631772,371.16897021)(793.35631836,371.23897308)
\curveto(793.35631773,371.31897006)(793.34631774,371.39396999)(793.32631836,371.46397308)
\curveto(793.31631777,371.50396988)(793.31131777,371.53896984)(793.31131836,371.56897308)
\curveto(793.32131776,371.60896977)(793.32131776,371.64896973)(793.31131836,371.68897308)
\curveto(793.31131777,371.72896965)(793.30631778,371.76896961)(793.29631836,371.80897308)
\lineto(793.29631836,371.92897308)
\curveto(793.27631781,372.04896933)(793.27631781,372.17396921)(793.29631836,372.30397308)
\curveto(793.30631778,372.36396902)(793.31131777,372.42396896)(793.31131836,372.48397308)
\lineto(793.31131836,372.64897308)
\curveto(793.32131776,372.69896868)(793.32631776,372.73896864)(793.32631836,372.76897308)
\curveto(793.32631776,372.80896857)(793.33131775,372.85396853)(793.34131836,372.90397308)
\curveto(793.37131771,373.01396837)(793.39131769,373.11896826)(793.40131836,373.21897308)
\curveto(793.41131767,373.32896805)(793.43631765,373.43896794)(793.47631836,373.54897308)
\curveto(793.51631757,373.66896771)(793.55131753,373.7839676)(793.58131836,373.89397308)
\curveto(793.62131746,374.01396737)(793.66631742,374.12896725)(793.71631836,374.23897308)
\curveto(793.7863173,374.39896698)(793.86631722,374.54396684)(793.95631836,374.67397308)
\curveto(794.04631704,374.81396657)(794.14131694,374.94896643)(794.24131836,375.07897308)
\curveto(794.31131677,375.18896619)(794.40131668,375.2789661)(794.51131836,375.34897308)
\lineto(794.57131836,375.40897308)
\lineto(794.63131836,375.46897308)
\lineto(794.78131836,375.58897308)
\lineto(794.96131836,375.70897308)
\curveto(795.09131599,375.78896559)(795.22631586,375.85896552)(795.36631836,375.91897308)
\curveto(795.51631557,375.9789654)(795.67631541,376.03396535)(795.84631836,376.08397308)
\curveto(795.94631514,376.11396527)(796.04631504,376.13396525)(796.14631836,376.14397308)
\curveto(796.25631483,376.15396523)(796.36631472,376.16896521)(796.47631836,376.18897308)
\curveto(796.51631457,376.19896518)(796.56631452,376.19896518)(796.62631836,376.18897308)
\curveto(796.69631439,376.1789652)(796.74631434,376.1839652)(796.77631836,376.20397308)
\curveto(797.09631399,376.21396517)(797.3813137,376.1839652)(797.63131836,376.11397308)
\curveto(797.89131319,376.04396534)(798.12131296,375.94396544)(798.32131836,375.81397308)
\curveto(798.39131269,375.77396561)(798.45631263,375.72896565)(798.51631836,375.67897308)
\lineto(798.69631836,375.52897308)
\curveto(798.74631234,375.48896589)(798.79131229,375.44396594)(798.83131836,375.39397308)
\curveto(798.8813122,375.35396603)(798.95631213,375.33396605)(799.05631836,375.33397308)
\lineto(799.10131836,375.37897308)
\curveto(799.12131196,375.39896598)(799.14131194,375.42396596)(799.16131836,375.45397308)
\curveto(799.19131189,375.53396585)(799.20631188,375.61396577)(799.20631836,375.69397308)
\curveto(799.21631187,375.77396561)(799.24631184,375.84396554)(799.29631836,375.90397308)
\curveto(799.32631176,375.94396544)(799.3863117,375.97396541)(799.47631836,375.99397308)
\curveto(799.56631152,376.02396536)(799.66131142,376.03896534)(799.76131836,376.03897308)
\curveto(799.86131122,376.03896534)(799.95631113,376.02896535)(800.04631836,376.00897308)
\curveto(800.14631094,375.98896539)(800.21631087,375.96396542)(800.25631836,375.93397308)
\moveto(799.13131836,372.15397308)
\curveto(799.14131194,372.19396919)(799.14631194,372.24396914)(799.14631836,372.30397308)
\curveto(799.14631194,372.37396901)(799.14131194,372.42896895)(799.13131836,372.46897308)
\lineto(799.13131836,372.70897308)
\curveto(799.11131197,372.79896858)(799.09631199,372.8839685)(799.08631836,372.96397308)
\curveto(799.07631201,373.05396833)(799.06131202,373.13896824)(799.04131836,373.21897308)
\curveto(799.02131206,373.29896808)(799.00131208,373.37396801)(798.98131836,373.44397308)
\curveto(798.97131211,373.52396786)(798.95131213,373.59896778)(798.92131836,373.66897308)
\curveto(798.81131227,373.94896743)(798.66631242,374.19896718)(798.48631836,374.41897308)
\curveto(798.31631277,374.63896674)(798.09631299,374.80396658)(797.82631836,374.91397308)
\curveto(797.74631334,374.95396643)(797.66131342,374.9839664)(797.57131836,375.00397308)
\curveto(797.4813136,375.03396635)(797.3863137,375.05896632)(797.28631836,375.07897308)
\curveto(797.20631388,375.09896628)(797.11631397,375.10396628)(797.01631836,375.09397308)
\lineto(796.74631836,375.09397308)
\curveto(796.69631439,375.0839663)(796.64631444,375.0789663)(796.59631836,375.07897308)
\curveto(796.55631453,375.0789663)(796.51131457,375.07396631)(796.46131836,375.06397308)
\curveto(796.27131481,375.01396637)(796.11131497,374.96396642)(795.98131836,374.91397308)
\curveto(795.64131544,374.77396661)(795.37631571,374.56396682)(795.18631836,374.28397308)
\curveto(794.99631609,374.00396738)(794.84631624,373.6789677)(794.73631836,373.30897308)
\curveto(794.71631637,373.22896815)(794.70131638,373.14896823)(794.69131836,373.06897308)
\curveto(794.69131639,372.99896838)(794.6813164,372.92396846)(794.66131836,372.84397308)
\curveto(794.64131644,372.81396857)(794.63131645,372.7789686)(794.63131836,372.73897308)
\curveto(794.64131644,372.69896868)(794.64131644,372.66396872)(794.63131836,372.63397308)
\lineto(794.63131836,372.30397308)
\lineto(794.63131836,371.95897308)
\curveto(794.63131645,371.84896953)(794.64131644,371.74396964)(794.66131836,371.64397308)
\lineto(794.66131836,371.56897308)
\curveto(794.67131641,371.53896984)(794.67631641,371.51396987)(794.67631836,371.49397308)
\curveto(794.69631639,371.40396998)(794.71131637,371.31397007)(794.72131836,371.22397308)
\curveto(794.74131634,371.13397025)(794.76631632,371.04897033)(794.79631836,370.96897308)
\curveto(794.87631621,370.70897067)(794.97631611,370.46897091)(795.09631836,370.24897308)
\curveto(795.21631587,370.02897135)(795.37631571,369.84897153)(795.57631836,369.70897308)
\lineto(795.69631836,369.61897308)
\curveto(795.73631535,369.59897178)(795.7813153,369.5789718)(795.83131836,369.55897308)
\curveto(795.91131517,369.50897187)(795.99631509,369.46897191)(796.08631836,369.43897308)
\curveto(796.17631491,369.40897197)(796.27631481,369.378972)(796.38631836,369.34897308)
\curveto(796.43631465,369.33897204)(796.4813146,369.33397205)(796.52131836,369.33397308)
\curveto(796.57131451,369.34397204)(796.62131446,369.33897204)(796.67131836,369.31897308)
\curveto(796.70131438,369.30897207)(796.75131433,369.30397208)(796.82131836,369.30397308)
\curveto(796.89131419,369.30397208)(796.94131414,369.30897207)(796.97131836,369.31897308)
\curveto(797.00131408,369.32897205)(797.03131405,369.32897205)(797.06131836,369.31897308)
\curveto(797.10131398,369.31897206)(797.14131394,369.32397206)(797.18131836,369.33397308)
\curveto(797.27131381,369.35397203)(797.35631373,369.37397201)(797.43631836,369.39397308)
\curveto(797.51631357,369.41397197)(797.59631349,369.43897194)(797.67631836,369.46897308)
\curveto(798.01631307,369.61897176)(798.2863128,369.82897155)(798.48631836,370.09897308)
\curveto(798.6863124,370.36897101)(798.84631224,370.6839707)(798.96631836,371.04397308)
\curveto(798.99631209,371.13397025)(799.01631207,371.22397016)(799.02631836,371.31397308)
\curveto(799.04631204,371.41396997)(799.06631202,371.50896987)(799.08631836,371.59897308)
\curveto(799.09631199,371.63896974)(799.10131198,371.67396971)(799.10131836,371.70397308)
\curveto(799.10131198,371.74396964)(799.10631198,371.7839696)(799.11631836,371.82397308)
\curveto(799.13631195,371.87396951)(799.13631195,371.92396946)(799.11631836,371.97397308)
\curveto(799.10631198,372.03396935)(799.11131197,372.09396929)(799.13131836,372.15397308)
}
}
{
\newrgbcolor{curcolor}{0 0 0}
\pscustom[linestyle=none,fillstyle=solid,fillcolor=curcolor]
{
\newpath
\moveto(802.53959961,377.53897308)
\curveto(802.45959849,377.59896378)(802.41459853,377.70396368)(802.40459961,377.85397308)
\lineto(802.40459961,378.31897308)
\lineto(802.40459961,378.57397308)
\curveto(802.40459854,378.66396272)(802.41959853,378.73896264)(802.44959961,378.79897308)
\curveto(802.48959846,378.8789625)(802.56959838,378.93896244)(802.68959961,378.97897308)
\curveto(802.70959824,378.98896239)(802.72959822,378.98896239)(802.74959961,378.97897308)
\curveto(802.77959817,378.9789624)(802.80459814,378.9839624)(802.82459961,378.99397308)
\curveto(802.99459795,378.99396239)(803.15459779,378.98896239)(803.30459961,378.97897308)
\curveto(803.45459749,378.96896241)(803.55459739,378.90896247)(803.60459961,378.79897308)
\curveto(803.63459731,378.73896264)(803.6495973,378.66396272)(803.64959961,378.57397308)
\lineto(803.64959961,378.31897308)
\curveto(803.6495973,378.13896324)(803.6445973,377.96896341)(803.63459961,377.80897308)
\curveto(803.63459731,377.64896373)(803.56959738,377.54396384)(803.43959961,377.49397308)
\curveto(803.38959756,377.47396391)(803.33459761,377.46396392)(803.27459961,377.46397308)
\lineto(803.10959961,377.46397308)
\lineto(802.79459961,377.46397308)
\curveto(802.69459825,377.46396392)(802.60959834,377.48896389)(802.53959961,377.53897308)
\moveto(803.64959961,369.03397308)
\lineto(803.64959961,368.71897308)
\curveto(803.65959729,368.61897276)(803.63959731,368.53897284)(803.58959961,368.47897308)
\curveto(803.55959739,368.41897296)(803.51459743,368.378973)(803.45459961,368.35897308)
\curveto(803.39459755,368.34897303)(803.32459762,368.33397305)(803.24459961,368.31397308)
\lineto(803.01959961,368.31397308)
\curveto(802.88959806,368.31397307)(802.77459817,368.31897306)(802.67459961,368.32897308)
\curveto(802.58459836,368.34897303)(802.51459843,368.39897298)(802.46459961,368.47897308)
\curveto(802.42459852,368.53897284)(802.40459854,368.61397277)(802.40459961,368.70397308)
\lineto(802.40459961,368.98897308)
\lineto(802.40459961,375.33397308)
\lineto(802.40459961,375.64897308)
\curveto(802.40459854,375.75896562)(802.42959852,375.84396554)(802.47959961,375.90397308)
\curveto(802.50959844,375.95396543)(802.5495984,375.9839654)(802.59959961,375.99397308)
\curveto(802.6495983,376.00396538)(802.70459824,376.01896536)(802.76459961,376.03897308)
\curveto(802.78459816,376.03896534)(802.80459814,376.03396535)(802.82459961,376.02397308)
\curveto(802.85459809,376.02396536)(802.87959807,376.02896535)(802.89959961,376.03897308)
\curveto(803.02959792,376.03896534)(803.15959779,376.03396535)(803.28959961,376.02397308)
\curveto(803.42959752,376.02396536)(803.52459742,375.9839654)(803.57459961,375.90397308)
\curveto(803.62459732,375.84396554)(803.6495973,375.76396562)(803.64959961,375.66397308)
\lineto(803.64959961,375.37897308)
\lineto(803.64959961,369.03397308)
}
}
{
\newrgbcolor{curcolor}{0 0 0}
\pscustom[linestyle=none,fillstyle=solid,fillcolor=curcolor]
{
\newpath
\moveto(808.02444336,376.21897308)
\curveto(808.74443929,376.22896515)(809.34943869,376.14396524)(809.83944336,375.96397308)
\curveto(810.32943771,375.79396559)(810.70943733,375.48896589)(810.97944336,375.04897308)
\curveto(811.04943699,374.93896644)(811.10443693,374.82396656)(811.14444336,374.70397308)
\curveto(811.18443685,374.59396679)(811.22443681,374.46896691)(811.26444336,374.32897308)
\curveto(811.28443675,374.25896712)(811.28943675,374.1839672)(811.27944336,374.10397308)
\curveto(811.26943677,374.03396735)(811.25443678,373.9789674)(811.23444336,373.93897308)
\curveto(811.21443682,373.91896746)(811.18943685,373.89896748)(811.15944336,373.87897308)
\curveto(811.12943691,373.86896751)(811.10443693,373.85396753)(811.08444336,373.83397308)
\curveto(811.034437,373.81396757)(810.98443705,373.80896757)(810.93444336,373.81897308)
\curveto(810.88443715,373.82896755)(810.8344372,373.82896755)(810.78444336,373.81897308)
\curveto(810.70443733,373.79896758)(810.59943744,373.79396759)(810.46944336,373.80397308)
\curveto(810.3394377,373.82396756)(810.24943779,373.84896753)(810.19944336,373.87897308)
\curveto(810.11943792,373.92896745)(810.06443797,373.99396739)(810.03444336,374.07397308)
\curveto(810.01443802,374.16396722)(809.97943806,374.24896713)(809.92944336,374.32897308)
\curveto(809.8394382,374.48896689)(809.71443832,374.63396675)(809.55444336,374.76397308)
\curveto(809.44443859,374.84396654)(809.32443871,374.90396648)(809.19444336,374.94397308)
\curveto(809.06443897,374.9839664)(808.92443911,375.02396636)(808.77444336,375.06397308)
\curveto(808.72443931,375.0839663)(808.67443936,375.08896629)(808.62444336,375.07897308)
\curveto(808.57443946,375.0789663)(808.52443951,375.0839663)(808.47444336,375.09397308)
\curveto(808.41443962,375.11396627)(808.3394397,375.12396626)(808.24944336,375.12397308)
\curveto(808.15943988,375.12396626)(808.08443995,375.11396627)(808.02444336,375.09397308)
\lineto(807.93444336,375.09397308)
\lineto(807.78444336,375.06397308)
\curveto(807.7344403,375.06396632)(807.68444035,375.05896632)(807.63444336,375.04897308)
\curveto(807.37444066,374.98896639)(807.15944088,374.90396648)(806.98944336,374.79397308)
\curveto(806.81944122,374.6839667)(806.70444133,374.49896688)(806.64444336,374.23897308)
\curveto(806.62444141,374.16896721)(806.61944142,374.09896728)(806.62944336,374.02897308)
\curveto(806.64944139,373.95896742)(806.66944137,373.89896748)(806.68944336,373.84897308)
\curveto(806.74944129,373.69896768)(806.81944122,373.58896779)(806.89944336,373.51897308)
\curveto(806.98944105,373.45896792)(807.09944094,373.38896799)(807.22944336,373.30897308)
\curveto(807.38944065,373.20896817)(807.56944047,373.13396825)(807.76944336,373.08397308)
\curveto(807.96944007,373.04396834)(808.16943987,372.99396839)(808.36944336,372.93397308)
\curveto(808.49943954,372.89396849)(808.62943941,372.86396852)(808.75944336,372.84397308)
\curveto(808.88943915,372.82396856)(809.01943902,372.79396859)(809.14944336,372.75397308)
\curveto(809.35943868,372.69396869)(809.56443847,372.63396875)(809.76444336,372.57397308)
\curveto(809.96443807,372.52396886)(810.16443787,372.45896892)(810.36444336,372.37897308)
\lineto(810.51444336,372.31897308)
\curveto(810.56443747,372.29896908)(810.61443742,372.27396911)(810.66444336,372.24397308)
\curveto(810.86443717,372.12396926)(811.039437,371.98896939)(811.18944336,371.83897308)
\curveto(811.3394367,371.68896969)(811.46443657,371.49896988)(811.56444336,371.26897308)
\curveto(811.58443645,371.19897018)(811.60443643,371.10397028)(811.62444336,370.98397308)
\curveto(811.64443639,370.91397047)(811.65443638,370.83897054)(811.65444336,370.75897308)
\curveto(811.66443637,370.68897069)(811.66943637,370.60897077)(811.66944336,370.51897308)
\lineto(811.66944336,370.36897308)
\curveto(811.64943639,370.29897108)(811.6394364,370.22897115)(811.63944336,370.15897308)
\curveto(811.6394364,370.08897129)(811.62943641,370.01897136)(811.60944336,369.94897308)
\curveto(811.57943646,369.83897154)(811.54443649,369.73397165)(811.50444336,369.63397308)
\curveto(811.46443657,369.53397185)(811.41943662,369.44397194)(811.36944336,369.36397308)
\curveto(811.20943683,369.10397228)(811.00443703,368.89397249)(810.75444336,368.73397308)
\curveto(810.50443753,368.5839728)(810.22443781,368.45397293)(809.91444336,368.34397308)
\curveto(809.82443821,368.31397307)(809.72943831,368.29397309)(809.62944336,368.28397308)
\curveto(809.5394385,368.26397312)(809.44943859,368.23897314)(809.35944336,368.20897308)
\curveto(809.25943878,368.18897319)(809.15943888,368.1789732)(809.05944336,368.17897308)
\curveto(808.95943908,368.1789732)(808.85943918,368.16897321)(808.75944336,368.14897308)
\lineto(808.60944336,368.14897308)
\curveto(808.55943948,368.13897324)(808.48943955,368.13397325)(808.39944336,368.13397308)
\curveto(808.30943973,368.13397325)(808.2394398,368.13897324)(808.18944336,368.14897308)
\lineto(808.02444336,368.14897308)
\curveto(807.96444007,368.16897321)(807.89944014,368.1789732)(807.82944336,368.17897308)
\curveto(807.75944028,368.16897321)(807.69944034,368.17397321)(807.64944336,368.19397308)
\curveto(807.59944044,368.20397318)(807.5344405,368.20897317)(807.45444336,368.20897308)
\lineto(807.21444336,368.26897308)
\curveto(807.14444089,368.2789731)(807.06944097,368.29897308)(806.98944336,368.32897308)
\curveto(806.67944136,368.42897295)(806.40944163,368.55397283)(806.17944336,368.70397308)
\curveto(805.94944209,368.85397253)(805.74944229,369.04897233)(805.57944336,369.28897308)
\curveto(805.48944255,369.41897196)(805.41444262,369.55397183)(805.35444336,369.69397308)
\curveto(805.29444274,369.83397155)(805.2394428,369.98897139)(805.18944336,370.15897308)
\curveto(805.16944287,370.21897116)(805.15944288,370.28897109)(805.15944336,370.36897308)
\curveto(805.16944287,370.45897092)(805.18444285,370.52897085)(805.20444336,370.57897308)
\curveto(805.2344428,370.61897076)(805.28444275,370.65897072)(805.35444336,370.69897308)
\curveto(805.40444263,370.71897066)(805.47444256,370.72897065)(805.56444336,370.72897308)
\curveto(805.65444238,370.73897064)(805.74444229,370.73897064)(805.83444336,370.72897308)
\curveto(805.92444211,370.71897066)(806.00944203,370.70397068)(806.08944336,370.68397308)
\curveto(806.17944186,370.67397071)(806.2394418,370.65897072)(806.26944336,370.63897308)
\curveto(806.3394417,370.58897079)(806.38444165,370.51397087)(806.40444336,370.41397308)
\curveto(806.4344416,370.32397106)(806.46944157,370.23897114)(806.50944336,370.15897308)
\curveto(806.60944143,369.93897144)(806.74444129,369.76897161)(806.91444336,369.64897308)
\curveto(807.034441,369.55897182)(807.16944087,369.48897189)(807.31944336,369.43897308)
\curveto(807.46944057,369.38897199)(807.62944041,369.33897204)(807.79944336,369.28897308)
\lineto(808.11444336,369.24397308)
\lineto(808.20444336,369.24397308)
\curveto(808.27443976,369.22397216)(808.36443967,369.21397217)(808.47444336,369.21397308)
\curveto(808.59443944,369.21397217)(808.69443934,369.22397216)(808.77444336,369.24397308)
\curveto(808.84443919,369.24397214)(808.89943914,369.24897213)(808.93944336,369.25897308)
\curveto(808.99943904,369.26897211)(809.05943898,369.27397211)(809.11944336,369.27397308)
\curveto(809.17943886,369.2839721)(809.2344388,369.29397209)(809.28444336,369.30397308)
\curveto(809.57443846,369.383972)(809.80443823,369.48897189)(809.97444336,369.61897308)
\curveto(810.14443789,369.74897163)(810.26443777,369.96897141)(810.33444336,370.27897308)
\curveto(810.35443768,370.32897105)(810.35943768,370.383971)(810.34944336,370.44397308)
\curveto(810.3394377,370.50397088)(810.32943771,370.54897083)(810.31944336,370.57897308)
\curveto(810.26943777,370.76897061)(810.19943784,370.90897047)(810.10944336,370.99897308)
\curveto(810.01943802,371.09897028)(809.90443813,371.18897019)(809.76444336,371.26897308)
\curveto(809.67443836,371.32897005)(809.57443846,371.37897)(809.46444336,371.41897308)
\lineto(809.13444336,371.53897308)
\curveto(809.10443893,371.54896983)(809.07443896,371.55396983)(809.04444336,371.55397308)
\curveto(809.02443901,371.55396983)(808.99943904,371.56396982)(808.96944336,371.58397308)
\curveto(808.62943941,371.69396969)(808.27443976,371.77396961)(807.90444336,371.82397308)
\curveto(807.54444049,371.8839695)(807.20444083,371.9789694)(806.88444336,372.10897308)
\curveto(806.78444125,372.14896923)(806.68944135,372.1839692)(806.59944336,372.21397308)
\curveto(806.50944153,372.24396914)(806.42444161,372.2839691)(806.34444336,372.33397308)
\curveto(806.15444188,372.44396894)(805.97944206,372.56896881)(805.81944336,372.70897308)
\curveto(805.65944238,372.84896853)(805.5344425,373.02396836)(805.44444336,373.23397308)
\curveto(805.41444262,373.30396808)(805.38944265,373.37396801)(805.36944336,373.44397308)
\curveto(805.35944268,373.51396787)(805.34444269,373.58896779)(805.32444336,373.66897308)
\curveto(805.29444274,373.78896759)(805.28444275,373.92396746)(805.29444336,374.07397308)
\curveto(805.30444273,374.23396715)(805.31944272,374.36896701)(805.33944336,374.47897308)
\curveto(805.35944268,374.52896685)(805.36944267,374.56896681)(805.36944336,374.59897308)
\curveto(805.37944266,374.63896674)(805.39444264,374.6789667)(805.41444336,374.71897308)
\curveto(805.50444253,374.94896643)(805.62444241,375.14896623)(805.77444336,375.31897308)
\curveto(805.9344421,375.48896589)(806.11444192,375.63896574)(806.31444336,375.76897308)
\curveto(806.46444157,375.85896552)(806.62944141,375.92896545)(806.80944336,375.97897308)
\curveto(806.98944105,376.03896534)(807.17944086,376.09396529)(807.37944336,376.14397308)
\curveto(807.44944059,376.15396523)(807.51444052,376.16396522)(807.57444336,376.17397308)
\curveto(807.64444039,376.1839652)(807.71944032,376.19396519)(807.79944336,376.20397308)
\curveto(807.82944021,376.21396517)(807.86944017,376.21396517)(807.91944336,376.20397308)
\curveto(807.96944007,376.19396519)(808.00444003,376.19896518)(808.02444336,376.21897308)
}
}
{
\newrgbcolor{curcolor}{0 0 0}
\pscustom[linestyle=none,fillstyle=solid,fillcolor=curcolor]
{
\newpath
\moveto(814.03944336,378.37897308)
\curveto(814.18944135,378.378963)(814.3394412,378.37396301)(814.48944336,378.36397308)
\curveto(814.6394409,378.36396302)(814.74444079,378.32396306)(814.80444336,378.24397308)
\curveto(814.85444068,378.1839632)(814.87944066,378.09896328)(814.87944336,377.98897308)
\curveto(814.88944065,377.88896349)(814.89444064,377.7839636)(814.89444336,377.67397308)
\lineto(814.89444336,376.80397308)
\curveto(814.89444064,376.72396466)(814.88944065,376.63896474)(814.87944336,376.54897308)
\curveto(814.87944066,376.46896491)(814.88944065,376.39896498)(814.90944336,376.33897308)
\curveto(814.94944059,376.19896518)(815.0394405,376.10896527)(815.17944336,376.06897308)
\curveto(815.22944031,376.05896532)(815.27444026,376.05396533)(815.31444336,376.05397308)
\lineto(815.46444336,376.05397308)
\lineto(815.86944336,376.05397308)
\curveto(816.02943951,376.06396532)(816.14443939,376.05396533)(816.21444336,376.02397308)
\curveto(816.30443923,375.96396542)(816.36443917,375.90396548)(816.39444336,375.84397308)
\curveto(816.41443912,375.80396558)(816.42443911,375.75896562)(816.42444336,375.70897308)
\lineto(816.42444336,375.55897308)
\curveto(816.42443911,375.44896593)(816.41943912,375.34396604)(816.40944336,375.24397308)
\curveto(816.39943914,375.15396623)(816.36443917,375.0839663)(816.30444336,375.03397308)
\curveto(816.24443929,374.9839664)(816.15943938,374.95396643)(816.04944336,374.94397308)
\lineto(815.71944336,374.94397308)
\curveto(815.60943993,374.95396643)(815.49944004,374.95896642)(815.38944336,374.95897308)
\curveto(815.27944026,374.95896642)(815.18444035,374.94396644)(815.10444336,374.91397308)
\curveto(815.0344405,374.8839665)(814.98444055,374.83396655)(814.95444336,374.76397308)
\curveto(814.92444061,374.69396669)(814.90444063,374.60896677)(814.89444336,374.50897308)
\curveto(814.88444065,374.41896696)(814.87944066,374.31896706)(814.87944336,374.20897308)
\curveto(814.88944065,374.10896727)(814.89444064,374.00896737)(814.89444336,373.90897308)
\lineto(814.89444336,370.93897308)
\curveto(814.89444064,370.71897066)(814.88944065,370.4839709)(814.87944336,370.23397308)
\curveto(814.87944066,369.99397139)(814.92444061,369.80897157)(815.01444336,369.67897308)
\curveto(815.06444047,369.59897178)(815.12944041,369.54397184)(815.20944336,369.51397308)
\curveto(815.28944025,369.4839719)(815.38444015,369.45897192)(815.49444336,369.43897308)
\curveto(815.52444001,369.42897195)(815.55443998,369.42397196)(815.58444336,369.42397308)
\curveto(815.62443991,369.43397195)(815.65943988,369.43397195)(815.68944336,369.42397308)
\lineto(815.88444336,369.42397308)
\curveto(815.98443955,369.42397196)(816.07443946,369.41397197)(816.15444336,369.39397308)
\curveto(816.24443929,369.383972)(816.30943923,369.34897203)(816.34944336,369.28897308)
\curveto(816.36943917,369.25897212)(816.38443915,369.20397218)(816.39444336,369.12397308)
\curveto(816.41443912,369.05397233)(816.42443911,368.9789724)(816.42444336,368.89897308)
\curveto(816.4344391,368.81897256)(816.4344391,368.73897264)(816.42444336,368.65897308)
\curveto(816.41443912,368.58897279)(816.39443914,368.53397285)(816.36444336,368.49397308)
\curveto(816.32443921,368.42397296)(816.24943929,368.37397301)(816.13944336,368.34397308)
\curveto(816.05943948,368.32397306)(815.96943957,368.31397307)(815.86944336,368.31397308)
\curveto(815.76943977,368.32397306)(815.67943986,368.32897305)(815.59944336,368.32897308)
\curveto(815.53944,368.32897305)(815.47944006,368.32397306)(815.41944336,368.31397308)
\curveto(815.35944018,368.31397307)(815.30444023,368.31897306)(815.25444336,368.32897308)
\lineto(815.07444336,368.32897308)
\curveto(815.02444051,368.33897304)(814.97444056,368.34397304)(814.92444336,368.34397308)
\curveto(814.88444065,368.35397303)(814.8394407,368.35897302)(814.78944336,368.35897308)
\curveto(814.58944095,368.40897297)(814.41444112,368.46397292)(814.26444336,368.52397308)
\curveto(814.12444141,368.5839728)(814.00444153,368.68897269)(813.90444336,368.83897308)
\curveto(813.76444177,369.03897234)(813.68444185,369.28897209)(813.66444336,369.58897308)
\curveto(813.64444189,369.89897148)(813.6344419,370.22897115)(813.63444336,370.57897308)
\lineto(813.63444336,374.50897308)
\curveto(813.60444193,374.63896674)(813.57444196,374.73396665)(813.54444336,374.79397308)
\curveto(813.52444201,374.85396653)(813.45444208,374.90396648)(813.33444336,374.94397308)
\curveto(813.29444224,374.95396643)(813.25444228,374.95396643)(813.21444336,374.94397308)
\curveto(813.17444236,374.93396645)(813.1344424,374.93896644)(813.09444336,374.95897308)
\lineto(812.85444336,374.95897308)
\curveto(812.72444281,374.95896642)(812.61444292,374.96896641)(812.52444336,374.98897308)
\curveto(812.44444309,375.01896636)(812.38944315,375.0789663)(812.35944336,375.16897308)
\curveto(812.3394432,375.20896617)(812.32444321,375.25396613)(812.31444336,375.30397308)
\lineto(812.31444336,375.45397308)
\curveto(812.31444322,375.59396579)(812.32444321,375.70896567)(812.34444336,375.79897308)
\curveto(812.36444317,375.89896548)(812.42444311,375.97396541)(812.52444336,376.02397308)
\curveto(812.6344429,376.06396532)(812.77444276,376.07396531)(812.94444336,376.05397308)
\curveto(813.12444241,376.03396535)(813.27444226,376.04396534)(813.39444336,376.08397308)
\curveto(813.48444205,376.13396525)(813.55444198,376.20396518)(813.60444336,376.29397308)
\curveto(813.62444191,376.35396503)(813.6344419,376.42896495)(813.63444336,376.51897308)
\lineto(813.63444336,376.77397308)
\lineto(813.63444336,377.70397308)
\lineto(813.63444336,377.94397308)
\curveto(813.6344419,378.03396335)(813.64444189,378.10896327)(813.66444336,378.16897308)
\curveto(813.70444183,378.24896313)(813.77944176,378.31396307)(813.88944336,378.36397308)
\curveto(813.91944162,378.36396302)(813.94444159,378.36396302)(813.96444336,378.36397308)
\curveto(813.99444154,378.37396301)(814.01944152,378.378963)(814.03944336,378.37897308)
}
}
{
\newrgbcolor{curcolor}{0 0 0}
\pscustom[linestyle=none,fillstyle=solid,fillcolor=curcolor]
{
\newpath
\moveto(821.45624023,376.21897308)
\curveto(821.68623544,376.21896516)(821.81623531,376.15896522)(821.84624023,376.03897308)
\curveto(821.87623525,375.92896545)(821.89123524,375.76396562)(821.89124023,375.54397308)
\lineto(821.89124023,375.25897308)
\curveto(821.89123524,375.16896621)(821.86623526,375.09396629)(821.81624023,375.03397308)
\curveto(821.75623537,374.95396643)(821.67123546,374.90896647)(821.56124023,374.89897308)
\curveto(821.45123568,374.89896648)(821.34123579,374.8839665)(821.23124023,374.85397308)
\curveto(821.09123604,374.82396656)(820.95623617,374.79396659)(820.82624023,374.76397308)
\curveto(820.70623642,374.73396665)(820.59123654,374.69396669)(820.48124023,374.64397308)
\curveto(820.19123694,374.51396687)(819.95623717,374.33396705)(819.77624023,374.10397308)
\curveto(819.59623753,373.8839675)(819.44123769,373.62896775)(819.31124023,373.33897308)
\curveto(819.27123786,373.22896815)(819.24123789,373.11396827)(819.22124023,372.99397308)
\curveto(819.20123793,372.8839685)(819.17623795,372.76896861)(819.14624023,372.64897308)
\curveto(819.13623799,372.59896878)(819.131238,372.54896883)(819.13124023,372.49897308)
\curveto(819.14123799,372.44896893)(819.14123799,372.39896898)(819.13124023,372.34897308)
\curveto(819.10123803,372.22896915)(819.08623804,372.08896929)(819.08624023,371.92897308)
\curveto(819.09623803,371.7789696)(819.10123803,371.63396975)(819.10124023,371.49397308)
\lineto(819.10124023,369.64897308)
\lineto(819.10124023,369.30397308)
\curveto(819.10123803,369.1839722)(819.09623803,369.06897231)(819.08624023,368.95897308)
\curveto(819.07623805,368.84897253)(819.07123806,368.75397263)(819.07124023,368.67397308)
\curveto(819.08123805,368.59397279)(819.06123807,368.52397286)(819.01124023,368.46397308)
\curveto(818.96123817,368.39397299)(818.88123825,368.35397303)(818.77124023,368.34397308)
\curveto(818.67123846,368.33397305)(818.56123857,368.32897305)(818.44124023,368.32897308)
\lineto(818.17124023,368.32897308)
\curveto(818.12123901,368.34897303)(818.07123906,368.36397302)(818.02124023,368.37397308)
\curveto(817.98123915,368.39397299)(817.95123918,368.41897296)(817.93124023,368.44897308)
\curveto(817.88123925,368.51897286)(817.85123928,368.60397278)(817.84124023,368.70397308)
\lineto(817.84124023,369.03397308)
\lineto(817.84124023,370.18897308)
\lineto(817.84124023,374.34397308)
\lineto(817.84124023,375.37897308)
\lineto(817.84124023,375.67897308)
\curveto(817.85123928,375.7789656)(817.88123925,375.86396552)(817.93124023,375.93397308)
\curveto(817.96123917,375.97396541)(818.01123912,376.00396538)(818.08124023,376.02397308)
\curveto(818.16123897,376.04396534)(818.24623888,376.05396533)(818.33624023,376.05397308)
\curveto(818.4262387,376.06396532)(818.51623861,376.06396532)(818.60624023,376.05397308)
\curveto(818.69623843,376.04396534)(818.76623836,376.02896535)(818.81624023,376.00897308)
\curveto(818.89623823,375.9789654)(818.94623818,375.91896546)(818.96624023,375.82897308)
\curveto(818.99623813,375.74896563)(819.01123812,375.65896572)(819.01124023,375.55897308)
\lineto(819.01124023,375.25897308)
\curveto(819.01123812,375.15896622)(819.0312381,375.06896631)(819.07124023,374.98897308)
\curveto(819.08123805,374.96896641)(819.09123804,374.95396643)(819.10124023,374.94397308)
\lineto(819.14624023,374.89897308)
\curveto(819.25623787,374.89896648)(819.34623778,374.94396644)(819.41624023,375.03397308)
\curveto(819.48623764,375.13396625)(819.54623758,375.21396617)(819.59624023,375.27397308)
\lineto(819.68624023,375.36397308)
\curveto(819.77623735,375.47396591)(819.90123723,375.58896579)(820.06124023,375.70897308)
\curveto(820.22123691,375.82896555)(820.37123676,375.91896546)(820.51124023,375.97897308)
\curveto(820.60123653,376.02896535)(820.69623643,376.06396532)(820.79624023,376.08397308)
\curveto(820.89623623,376.11396527)(821.00123613,376.14396524)(821.11124023,376.17397308)
\curveto(821.17123596,376.1839652)(821.2312359,376.18896519)(821.29124023,376.18897308)
\curveto(821.35123578,376.19896518)(821.40623572,376.20896517)(821.45624023,376.21897308)
}
}
{
\newrgbcolor{curcolor}{0 0 0}
\pscustom[linestyle=none,fillstyle=solid,fillcolor=curcolor]
{
\newpath
\moveto(829.70600586,368.86897308)
\curveto(829.73599803,368.70897267)(829.72099804,368.57397281)(829.66100586,368.46397308)
\curveto(829.60099816,368.36397302)(829.52099824,368.28897309)(829.42100586,368.23897308)
\curveto(829.37099839,368.21897316)(829.31599845,368.20897317)(829.25600586,368.20897308)
\curveto(829.20599856,368.20897317)(829.15099861,368.19897318)(829.09100586,368.17897308)
\curveto(828.87099889,368.12897325)(828.65099911,368.14397324)(828.43100586,368.22397308)
\curveto(828.22099954,368.29397309)(828.07599969,368.383973)(827.99600586,368.49397308)
\curveto(827.94599982,368.56397282)(827.90099986,368.64397274)(827.86100586,368.73397308)
\curveto(827.82099994,368.83397255)(827.77099999,368.91397247)(827.71100586,368.97397308)
\curveto(827.69100007,368.99397239)(827.6660001,369.01397237)(827.63600586,369.03397308)
\curveto(827.61600015,369.05397233)(827.58600018,369.05897232)(827.54600586,369.04897308)
\curveto(827.43600033,369.01897236)(827.33100043,368.96397242)(827.23100586,368.88397308)
\curveto(827.14100062,368.80397258)(827.05100071,368.73397265)(826.96100586,368.67397308)
\curveto(826.83100093,368.59397279)(826.69100107,368.51897286)(826.54100586,368.44897308)
\curveto(826.39100137,368.38897299)(826.23100153,368.33397305)(826.06100586,368.28397308)
\curveto(825.9610018,368.25397313)(825.85100191,368.23397315)(825.73100586,368.22397308)
\curveto(825.62100214,368.21397317)(825.51100225,368.19897318)(825.40100586,368.17897308)
\curveto(825.35100241,368.16897321)(825.30600246,368.16397322)(825.26600586,368.16397308)
\lineto(825.16100586,368.16397308)
\curveto(825.05100271,368.14397324)(824.94600282,368.14397324)(824.84600586,368.16397308)
\lineto(824.71100586,368.16397308)
\curveto(824.6610031,368.17397321)(824.61100315,368.1789732)(824.56100586,368.17897308)
\curveto(824.51100325,368.1789732)(824.4660033,368.18897319)(824.42600586,368.20897308)
\curveto(824.38600338,368.21897316)(824.35100341,368.22397316)(824.32100586,368.22397308)
\curveto(824.30100346,368.21397317)(824.27600349,368.21397317)(824.24600586,368.22397308)
\lineto(824.00600586,368.28397308)
\curveto(823.92600384,368.29397309)(823.85100391,368.31397307)(823.78100586,368.34397308)
\curveto(823.48100428,368.47397291)(823.23600453,368.61897276)(823.04600586,368.77897308)
\curveto(822.8660049,368.94897243)(822.71600505,369.1839722)(822.59600586,369.48397308)
\curveto(822.50600526,369.70397168)(822.4610053,369.96897141)(822.46100586,370.27897308)
\lineto(822.46100586,370.59397308)
\curveto(822.47100529,370.64397074)(822.47600529,370.69397069)(822.47600586,370.74397308)
\lineto(822.50600586,370.92397308)
\lineto(822.62600586,371.25397308)
\curveto(822.6660051,371.36397002)(822.71600505,371.46396992)(822.77600586,371.55397308)
\curveto(822.95600481,371.84396954)(823.20100456,372.05896932)(823.51100586,372.19897308)
\curveto(823.82100394,372.33896904)(824.1610036,372.46396892)(824.53100586,372.57397308)
\curveto(824.67100309,372.61396877)(824.81600295,372.64396874)(824.96600586,372.66397308)
\curveto(825.11600265,372.6839687)(825.2660025,372.70896867)(825.41600586,372.73897308)
\curveto(825.48600228,372.75896862)(825.55100221,372.76896861)(825.61100586,372.76897308)
\curveto(825.68100208,372.76896861)(825.75600201,372.7789686)(825.83600586,372.79897308)
\curveto(825.90600186,372.81896856)(825.97600179,372.82896855)(826.04600586,372.82897308)
\curveto(826.11600165,372.83896854)(826.19100157,372.85396853)(826.27100586,372.87397308)
\curveto(826.52100124,372.93396845)(826.75600101,372.9839684)(826.97600586,373.02397308)
\curveto(827.19600057,373.07396831)(827.37100039,373.18896819)(827.50100586,373.36897308)
\curveto(827.5610002,373.44896793)(827.61100015,373.54896783)(827.65100586,373.66897308)
\curveto(827.69100007,373.79896758)(827.69100007,373.93896744)(827.65100586,374.08897308)
\curveto(827.59100017,374.32896705)(827.50100026,374.51896686)(827.38100586,374.65897308)
\curveto(827.27100049,374.79896658)(827.11100065,374.90896647)(826.90100586,374.98897308)
\curveto(826.78100098,375.03896634)(826.63600113,375.07396631)(826.46600586,375.09397308)
\curveto(826.30600146,375.11396627)(826.13600163,375.12396626)(825.95600586,375.12397308)
\curveto(825.77600199,375.12396626)(825.60100216,375.11396627)(825.43100586,375.09397308)
\curveto(825.2610025,375.07396631)(825.11600265,375.04396634)(824.99600586,375.00397308)
\curveto(824.82600294,374.94396644)(824.6610031,374.85896652)(824.50100586,374.74897308)
\curveto(824.42100334,374.68896669)(824.34600342,374.60896677)(824.27600586,374.50897308)
\curveto(824.21600355,374.41896696)(824.1610036,374.31896706)(824.11100586,374.20897308)
\curveto(824.08100368,374.12896725)(824.05100371,374.04396734)(824.02100586,373.95397308)
\curveto(824.00100376,373.86396752)(823.95600381,373.79396759)(823.88600586,373.74397308)
\curveto(823.84600392,373.71396767)(823.77600399,373.68896769)(823.67600586,373.66897308)
\curveto(823.58600418,373.65896772)(823.49100427,373.65396773)(823.39100586,373.65397308)
\curveto(823.29100447,373.65396773)(823.19100457,373.65896772)(823.09100586,373.66897308)
\curveto(823.00100476,373.68896769)(822.93600483,373.71396767)(822.89600586,373.74397308)
\curveto(822.85600491,373.77396761)(822.82600494,373.82396756)(822.80600586,373.89397308)
\curveto(822.78600498,373.96396742)(822.78600498,374.03896734)(822.80600586,374.11897308)
\curveto(822.83600493,374.24896713)(822.8660049,374.36896701)(822.89600586,374.47897308)
\curveto(822.93600483,374.59896678)(822.98100478,374.71396667)(823.03100586,374.82397308)
\curveto(823.22100454,375.17396621)(823.4610043,375.44396594)(823.75100586,375.63397308)
\curveto(824.04100372,375.83396555)(824.40100336,375.99396539)(824.83100586,376.11397308)
\curveto(824.93100283,376.13396525)(825.03100273,376.14896523)(825.13100586,376.15897308)
\curveto(825.24100252,376.16896521)(825.35100241,376.1839652)(825.46100586,376.20397308)
\curveto(825.50100226,376.21396517)(825.5660022,376.21396517)(825.65600586,376.20397308)
\curveto(825.74600202,376.20396518)(825.80100196,376.21396517)(825.82100586,376.23397308)
\curveto(826.52100124,376.24396514)(827.13100063,376.16396522)(827.65100586,375.99397308)
\curveto(828.17099959,375.82396556)(828.53599923,375.49896588)(828.74600586,375.01897308)
\curveto(828.83599893,374.81896656)(828.88599888,374.5839668)(828.89600586,374.31397308)
\curveto(828.91599885,374.05396733)(828.92599884,373.7789676)(828.92600586,373.48897308)
\lineto(828.92600586,370.17397308)
\curveto(828.92599884,370.03397135)(828.93099883,369.89897148)(828.94100586,369.76897308)
\curveto(828.95099881,369.63897174)(828.98099878,369.53397185)(829.03100586,369.45397308)
\curveto(829.08099868,369.383972)(829.14599862,369.33397205)(829.22600586,369.30397308)
\curveto(829.31599845,369.26397212)(829.40099836,369.23397215)(829.48100586,369.21397308)
\curveto(829.5609982,369.20397218)(829.62099814,369.15897222)(829.66100586,369.07897308)
\curveto(829.68099808,369.04897233)(829.69099807,369.01897236)(829.69100586,368.98897308)
\curveto(829.69099807,368.95897242)(829.69599807,368.91897246)(829.70600586,368.86897308)
\moveto(827.56100586,370.53397308)
\curveto(827.62100014,370.67397071)(827.65100011,370.83397055)(827.65100586,371.01397308)
\curveto(827.6610001,371.20397018)(827.6660001,371.39896998)(827.66600586,371.59897308)
\curveto(827.6660001,371.70896967)(827.6610001,371.80896957)(827.65100586,371.89897308)
\curveto(827.64100012,371.98896939)(827.60100016,372.05896932)(827.53100586,372.10897308)
\curveto(827.50100026,372.12896925)(827.43100033,372.13896924)(827.32100586,372.13897308)
\curveto(827.30100046,372.11896926)(827.2660005,372.10896927)(827.21600586,372.10897308)
\curveto(827.1660006,372.10896927)(827.12100064,372.09896928)(827.08100586,372.07897308)
\curveto(827.00100076,372.05896932)(826.91100085,372.03896934)(826.81100586,372.01897308)
\lineto(826.51100586,371.95897308)
\curveto(826.48100128,371.95896942)(826.44600132,371.95396943)(826.40600586,371.94397308)
\lineto(826.30100586,371.94397308)
\curveto(826.15100161,371.90396948)(825.98600178,371.8789695)(825.80600586,371.86897308)
\curveto(825.63600213,371.86896951)(825.47600229,371.84896953)(825.32600586,371.80897308)
\curveto(825.24600252,371.78896959)(825.17100259,371.76896961)(825.10100586,371.74897308)
\curveto(825.04100272,371.73896964)(824.97100279,371.72396966)(824.89100586,371.70397308)
\curveto(824.73100303,371.65396973)(824.58100318,371.58896979)(824.44100586,371.50897308)
\curveto(824.30100346,371.43896994)(824.18100358,371.34897003)(824.08100586,371.23897308)
\curveto(823.98100378,371.12897025)(823.90600386,370.99397039)(823.85600586,370.83397308)
\curveto(823.80600396,370.6839707)(823.78600398,370.49897088)(823.79600586,370.27897308)
\curveto(823.79600397,370.1789712)(823.81100395,370.0839713)(823.84100586,369.99397308)
\curveto(823.88100388,369.91397147)(823.92600384,369.83897154)(823.97600586,369.76897308)
\curveto(824.05600371,369.65897172)(824.1610036,369.56397182)(824.29100586,369.48397308)
\curveto(824.42100334,369.41397197)(824.5610032,369.35397203)(824.71100586,369.30397308)
\curveto(824.761003,369.29397209)(824.81100295,369.28897209)(824.86100586,369.28897308)
\curveto(824.91100285,369.28897209)(824.9610028,369.2839721)(825.01100586,369.27397308)
\curveto(825.08100268,369.25397213)(825.1660026,369.23897214)(825.26600586,369.22897308)
\curveto(825.37600239,369.22897215)(825.4660023,369.23897214)(825.53600586,369.25897308)
\curveto(825.59600217,369.2789721)(825.65600211,369.2839721)(825.71600586,369.27397308)
\curveto(825.77600199,369.27397211)(825.83600193,369.2839721)(825.89600586,369.30397308)
\curveto(825.97600179,369.32397206)(826.05100171,369.33897204)(826.12100586,369.34897308)
\curveto(826.20100156,369.35897202)(826.27600149,369.378972)(826.34600586,369.40897308)
\curveto(826.63600113,369.52897185)(826.88100088,369.67397171)(827.08100586,369.84397308)
\curveto(827.29100047,370.01397137)(827.45100031,370.24397114)(827.56100586,370.53397308)
}
}
{
\newrgbcolor{curcolor}{0 0 0}
\pscustom[linestyle=none,fillstyle=solid,fillcolor=curcolor]
{
\newpath
\moveto(837.83764648,369.12397308)
\lineto(837.83764648,368.73397308)
\curveto(837.83763861,368.61397277)(837.81263863,368.51397287)(837.76264648,368.43397308)
\curveto(837.71263873,368.36397302)(837.62763882,368.32397306)(837.50764648,368.31397308)
\lineto(837.16264648,368.31397308)
\curveto(837.10263934,368.31397307)(837.0426394,368.30897307)(836.98264648,368.29897308)
\curveto(836.93263951,368.29897308)(836.88763956,368.30897307)(836.84764648,368.32897308)
\curveto(836.75763969,368.34897303)(836.69763975,368.38897299)(836.66764648,368.44897308)
\curveto(836.62763982,368.49897288)(836.60263984,368.55897282)(836.59264648,368.62897308)
\curveto(836.59263985,368.69897268)(836.57763987,368.76897261)(836.54764648,368.83897308)
\curveto(836.53763991,368.85897252)(836.52263992,368.87397251)(836.50264648,368.88397308)
\curveto(836.49263995,368.90397248)(836.47763997,368.92397246)(836.45764648,368.94397308)
\curveto(836.35764009,368.95397243)(836.27764017,368.93397245)(836.21764648,368.88397308)
\curveto(836.16764028,368.83397255)(836.11264033,368.7839726)(836.05264648,368.73397308)
\curveto(835.85264059,368.5839728)(835.65264079,368.46897291)(835.45264648,368.38897308)
\curveto(835.27264117,368.30897307)(835.06264138,368.24897313)(834.82264648,368.20897308)
\curveto(834.59264185,368.16897321)(834.35264209,368.14897323)(834.10264648,368.14897308)
\curveto(833.86264258,368.13897324)(833.62264282,368.15397323)(833.38264648,368.19397308)
\curveto(833.1426433,368.22397316)(832.93264351,368.2789731)(832.75264648,368.35897308)
\curveto(832.23264421,368.5789728)(831.81264463,368.87397251)(831.49264648,369.24397308)
\curveto(831.17264527,369.62397176)(830.92264552,370.09397129)(830.74264648,370.65397308)
\curveto(830.70264574,370.74397064)(830.67264577,370.83397055)(830.65264648,370.92397308)
\curveto(830.6426458,371.02397036)(830.62264582,371.12397026)(830.59264648,371.22397308)
\curveto(830.58264586,371.27397011)(830.57764587,371.32397006)(830.57764648,371.37397308)
\curveto(830.57764587,371.42396996)(830.57264587,371.47396991)(830.56264648,371.52397308)
\curveto(830.5426459,371.57396981)(830.53264591,371.62396976)(830.53264648,371.67397308)
\curveto(830.5426459,371.73396965)(830.5426459,371.78896959)(830.53264648,371.83897308)
\lineto(830.53264648,371.98897308)
\curveto(830.51264593,372.03896934)(830.50264594,372.10396928)(830.50264648,372.18397308)
\curveto(830.50264594,372.26396912)(830.51264593,372.32896905)(830.53264648,372.37897308)
\lineto(830.53264648,372.54397308)
\curveto(830.55264589,372.61396877)(830.55764589,372.6839687)(830.54764648,372.75397308)
\curveto(830.5476459,372.83396855)(830.55764589,372.90896847)(830.57764648,372.97897308)
\curveto(830.58764586,373.02896835)(830.59264585,373.07396831)(830.59264648,373.11397308)
\curveto(830.59264585,373.15396823)(830.59764585,373.19896818)(830.60764648,373.24897308)
\curveto(830.63764581,373.34896803)(830.66264578,373.44396794)(830.68264648,373.53397308)
\curveto(830.70264574,373.63396775)(830.72764572,373.72896765)(830.75764648,373.81897308)
\curveto(830.88764556,374.19896718)(831.05264539,374.53896684)(831.25264648,374.83897308)
\curveto(831.46264498,375.14896623)(831.71264473,375.40396598)(832.00264648,375.60397308)
\curveto(832.17264427,375.72396566)(832.3476441,375.82396556)(832.52764648,375.90397308)
\curveto(832.71764373,375.9839654)(832.92264352,376.05396533)(833.14264648,376.11397308)
\curveto(833.21264323,376.12396526)(833.27764317,376.13396525)(833.33764648,376.14397308)
\curveto(833.40764304,376.15396523)(833.47764297,376.16896521)(833.54764648,376.18897308)
\lineto(833.69764648,376.18897308)
\curveto(833.77764267,376.20896517)(833.89264255,376.21896516)(834.04264648,376.21897308)
\curveto(834.20264224,376.21896516)(834.32264212,376.20896517)(834.40264648,376.18897308)
\curveto(834.442642,376.1789652)(834.49764195,376.17396521)(834.56764648,376.17397308)
\curveto(834.67764177,376.14396524)(834.78764166,376.11896526)(834.89764648,376.09897308)
\curveto(835.00764144,376.08896529)(835.11264133,376.05896532)(835.21264648,376.00897308)
\curveto(835.36264108,375.94896543)(835.50264094,375.8839655)(835.63264648,375.81397308)
\curveto(835.77264067,375.74396564)(835.90264054,375.66396572)(836.02264648,375.57397308)
\curveto(836.08264036,375.52396586)(836.1426403,375.46896591)(836.20264648,375.40897308)
\curveto(836.27264017,375.35896602)(836.36264008,375.34396604)(836.47264648,375.36397308)
\curveto(836.49263995,375.39396599)(836.50763994,375.41896596)(836.51764648,375.43897308)
\curveto(836.53763991,375.45896592)(836.55263989,375.48896589)(836.56264648,375.52897308)
\curveto(836.59263985,375.61896576)(836.60263984,375.73396565)(836.59264648,375.87397308)
\lineto(836.59264648,376.24897308)
\lineto(836.59264648,377.97397308)
\lineto(836.59264648,378.43897308)
\curveto(836.59263985,378.61896276)(836.61763983,378.74896263)(836.66764648,378.82897308)
\curveto(836.70763974,378.89896248)(836.76763968,378.94396244)(836.84764648,378.96397308)
\curveto(836.86763958,378.96396242)(836.89263955,378.96396242)(836.92264648,378.96397308)
\curveto(836.95263949,378.97396241)(836.97763947,378.9789624)(836.99764648,378.97897308)
\curveto(837.13763931,378.98896239)(837.28263916,378.98896239)(837.43264648,378.97897308)
\curveto(837.59263885,378.9789624)(837.70263874,378.93896244)(837.76264648,378.85897308)
\curveto(837.81263863,378.7789626)(837.83763861,378.6789627)(837.83764648,378.55897308)
\lineto(837.83764648,378.18397308)
\lineto(837.83764648,369.12397308)
\moveto(836.62264648,371.95897308)
\curveto(836.6426398,372.00896937)(836.65263979,372.07396931)(836.65264648,372.15397308)
\curveto(836.65263979,372.24396914)(836.6426398,372.31396907)(836.62264648,372.36397308)
\lineto(836.62264648,372.58897308)
\curveto(836.60263984,372.6789687)(836.58763986,372.76896861)(836.57764648,372.85897308)
\curveto(836.56763988,372.95896842)(836.5476399,373.04896833)(836.51764648,373.12897308)
\curveto(836.49763995,373.20896817)(836.47763997,373.2839681)(836.45764648,373.35397308)
\curveto(836.44764,373.42396796)(836.42764002,373.49396789)(836.39764648,373.56397308)
\curveto(836.27764017,373.86396752)(836.12264032,374.12896725)(835.93264648,374.35897308)
\curveto(835.7426407,374.58896679)(835.50264094,374.76896661)(835.21264648,374.89897308)
\curveto(835.11264133,374.94896643)(835.00764144,374.9839664)(834.89764648,375.00397308)
\curveto(834.79764165,375.03396635)(834.68764176,375.05896632)(834.56764648,375.07897308)
\curveto(834.48764196,375.09896628)(834.39764205,375.10896627)(834.29764648,375.10897308)
\lineto(834.02764648,375.10897308)
\curveto(833.97764247,375.09896628)(833.93264251,375.08896629)(833.89264648,375.07897308)
\lineto(833.75764648,375.07897308)
\curveto(833.67764277,375.05896632)(833.59264285,375.03896634)(833.50264648,375.01897308)
\curveto(833.42264302,374.99896638)(833.3426431,374.97396641)(833.26264648,374.94397308)
\curveto(832.9426435,374.80396658)(832.68264376,374.59896678)(832.48264648,374.32897308)
\curveto(832.29264415,374.06896731)(832.13764431,373.76396762)(832.01764648,373.41397308)
\curveto(831.97764447,373.30396808)(831.9476445,373.18896819)(831.92764648,373.06897308)
\curveto(831.91764453,372.95896842)(831.90264454,372.84896853)(831.88264648,372.73897308)
\curveto(831.88264456,372.69896868)(831.87764457,372.65896872)(831.86764648,372.61897308)
\lineto(831.86764648,372.51397308)
\curveto(831.8476446,372.46396892)(831.83764461,372.40896897)(831.83764648,372.34897308)
\curveto(831.8476446,372.28896909)(831.85264459,372.23396915)(831.85264648,372.18397308)
\lineto(831.85264648,371.85397308)
\curveto(831.85264459,371.75396963)(831.86264458,371.65896972)(831.88264648,371.56897308)
\curveto(831.89264455,371.53896984)(831.89764455,371.48896989)(831.89764648,371.41897308)
\curveto(831.91764453,371.34897003)(831.93264451,371.2789701)(831.94264648,371.20897308)
\lineto(832.00264648,370.99897308)
\curveto(832.11264433,370.64897073)(832.26264418,370.34897103)(832.45264648,370.09897308)
\curveto(832.6426438,369.84897153)(832.88264356,369.64397174)(833.17264648,369.48397308)
\curveto(833.26264318,369.43397195)(833.35264309,369.39397199)(833.44264648,369.36397308)
\curveto(833.53264291,369.33397205)(833.63264281,369.30397208)(833.74264648,369.27397308)
\curveto(833.79264265,369.25397213)(833.8426426,369.24897213)(833.89264648,369.25897308)
\curveto(833.95264249,369.26897211)(834.00764244,369.26397212)(834.05764648,369.24397308)
\curveto(834.09764235,369.23397215)(834.13764231,369.22897215)(834.17764648,369.22897308)
\lineto(834.31264648,369.22897308)
\lineto(834.44764648,369.22897308)
\curveto(834.47764197,369.23897214)(834.52764192,369.24397214)(834.59764648,369.24397308)
\curveto(834.67764177,369.26397212)(834.75764169,369.2789721)(834.83764648,369.28897308)
\curveto(834.91764153,369.30897207)(834.99264145,369.33397205)(835.06264648,369.36397308)
\curveto(835.39264105,369.50397188)(835.65764079,369.6789717)(835.85764648,369.88897308)
\curveto(836.06764038,370.10897127)(836.2426402,370.383971)(836.38264648,370.71397308)
\curveto(836.43264001,370.82397056)(836.46763998,370.93397045)(836.48764648,371.04397308)
\curveto(836.50763994,371.15397023)(836.53263991,371.26397012)(836.56264648,371.37397308)
\curveto(836.58263986,371.41396997)(836.59263985,371.44896993)(836.59264648,371.47897308)
\curveto(836.59263985,371.51896986)(836.59763985,371.55896982)(836.60764648,371.59897308)
\curveto(836.61763983,371.65896972)(836.61763983,371.71896966)(836.60764648,371.77897308)
\curveto(836.60763984,371.83896954)(836.61263983,371.89896948)(836.62264648,371.95897308)
}
}
{
\newrgbcolor{curcolor}{0 0 0}
\pscustom[linestyle=none,fillstyle=solid,fillcolor=curcolor]
{
\newpath
\moveto(846.90889648,372.51397308)
\curveto(846.92888842,372.45396893)(846.93888841,372.35896902)(846.93889648,372.22897308)
\curveto(846.93888841,372.10896927)(846.93388842,372.02396936)(846.92389648,371.97397308)
\lineto(846.92389648,371.82397308)
\curveto(846.91388844,371.74396964)(846.90388845,371.66896971)(846.89389648,371.59897308)
\curveto(846.89388846,371.53896984)(846.88888846,371.46896991)(846.87889648,371.38897308)
\curveto(846.85888849,371.32897005)(846.84388851,371.26897011)(846.83389648,371.20897308)
\curveto(846.83388852,371.14897023)(846.82388853,371.08897029)(846.80389648,371.02897308)
\curveto(846.76388859,370.89897048)(846.72888862,370.76897061)(846.69889648,370.63897308)
\curveto(846.66888868,370.50897087)(846.62888872,370.38897099)(846.57889648,370.27897308)
\curveto(846.36888898,369.79897158)(846.08888926,369.39397199)(845.73889648,369.06397308)
\curveto(845.38888996,368.74397264)(844.95889039,368.49897288)(844.44889648,368.32897308)
\curveto(844.33889101,368.28897309)(844.21889113,368.25897312)(844.08889648,368.23897308)
\curveto(843.96889138,368.21897316)(843.84389151,368.19897318)(843.71389648,368.17897308)
\curveto(843.6538917,368.16897321)(843.58889176,368.16397322)(843.51889648,368.16397308)
\curveto(843.45889189,368.15397323)(843.39889195,368.14897323)(843.33889648,368.14897308)
\curveto(843.29889205,368.13897324)(843.23889211,368.13397325)(843.15889648,368.13397308)
\curveto(843.08889226,368.13397325)(843.03889231,368.13897324)(843.00889648,368.14897308)
\curveto(842.96889238,368.15897322)(842.92889242,368.16397322)(842.88889648,368.16397308)
\curveto(842.8488925,368.15397323)(842.81389254,368.15397323)(842.78389648,368.16397308)
\lineto(842.69389648,368.16397308)
\lineto(842.33389648,368.20897308)
\curveto(842.19389316,368.24897313)(842.05889329,368.28897309)(841.92889648,368.32897308)
\curveto(841.79889355,368.36897301)(841.67389368,368.41397297)(841.55389648,368.46397308)
\curveto(841.10389425,368.66397272)(840.73389462,368.92397246)(840.44389648,369.24397308)
\curveto(840.1538952,369.56397182)(839.91389544,369.95397143)(839.72389648,370.41397308)
\curveto(839.67389568,370.51397087)(839.63389572,370.61397077)(839.60389648,370.71397308)
\curveto(839.58389577,370.81397057)(839.56389579,370.91897046)(839.54389648,371.02897308)
\curveto(839.52389583,371.06897031)(839.51389584,371.09897028)(839.51389648,371.11897308)
\curveto(839.52389583,371.14897023)(839.52389583,371.1839702)(839.51389648,371.22397308)
\curveto(839.49389586,371.30397008)(839.47889587,371.38397)(839.46889648,371.46397308)
\curveto(839.46889588,371.55396983)(839.45889589,371.63896974)(839.43889648,371.71897308)
\lineto(839.43889648,371.83897308)
\curveto(839.43889591,371.8789695)(839.43389592,371.92396946)(839.42389648,371.97397308)
\curveto(839.41389594,372.02396936)(839.40889594,372.10896927)(839.40889648,372.22897308)
\curveto(839.40889594,372.35896902)(839.41889593,372.45396893)(839.43889648,372.51397308)
\curveto(839.45889589,372.5839688)(839.46389589,372.65396873)(839.45389648,372.72397308)
\curveto(839.44389591,372.79396859)(839.4488959,372.86396852)(839.46889648,372.93397308)
\curveto(839.47889587,372.9839684)(839.48389587,373.02396836)(839.48389648,373.05397308)
\curveto(839.49389586,373.09396829)(839.50389585,373.13896824)(839.51389648,373.18897308)
\curveto(839.54389581,373.30896807)(839.56889578,373.42896795)(839.58889648,373.54897308)
\curveto(839.61889573,373.66896771)(839.65889569,373.7839676)(839.70889648,373.89397308)
\curveto(839.85889549,374.26396712)(840.03889531,374.59396679)(840.24889648,374.88397308)
\curveto(840.46889488,375.1839662)(840.73389462,375.43396595)(841.04389648,375.63397308)
\curveto(841.16389419,375.71396567)(841.28889406,375.7789656)(841.41889648,375.82897308)
\curveto(841.5488938,375.88896549)(841.68389367,375.94896543)(841.82389648,376.00897308)
\curveto(841.94389341,376.05896532)(842.07389328,376.08896529)(842.21389648,376.09897308)
\curveto(842.353893,376.11896526)(842.49389286,376.14896523)(842.63389648,376.18897308)
\lineto(842.82889648,376.18897308)
\curveto(842.89889245,376.19896518)(842.96389239,376.20896517)(843.02389648,376.21897308)
\curveto(843.91389144,376.22896515)(844.6538907,376.04396534)(845.24389648,375.66397308)
\curveto(845.83388952,375.2839661)(846.25888909,374.78896659)(846.51889648,374.17897308)
\curveto(846.56888878,374.0789673)(846.60888874,373.9789674)(846.63889648,373.87897308)
\curveto(846.66888868,373.7789676)(846.70388865,373.67396771)(846.74389648,373.56397308)
\curveto(846.77388858,373.45396793)(846.79888855,373.33396805)(846.81889648,373.20397308)
\curveto(846.83888851,373.0839683)(846.86388849,372.95896842)(846.89389648,372.82897308)
\curveto(846.90388845,372.7789686)(846.90388845,372.72396866)(846.89389648,372.66397308)
\curveto(846.89388846,372.61396877)(846.89888845,372.56396882)(846.90889648,372.51397308)
\moveto(845.57389648,371.65897308)
\curveto(845.59388976,371.72896965)(845.59888975,371.80896957)(845.58889648,371.89897308)
\lineto(845.58889648,372.15397308)
\curveto(845.58888976,372.54396884)(845.5538898,372.87396851)(845.48389648,373.14397308)
\curveto(845.4538899,373.22396816)(845.42888992,373.30396808)(845.40889648,373.38397308)
\curveto(845.38888996,373.46396792)(845.36388999,373.53896784)(845.33389648,373.60897308)
\curveto(845.0538903,374.25896712)(844.60889074,374.70896667)(843.99889648,374.95897308)
\curveto(843.92889142,374.98896639)(843.8538915,375.00896637)(843.77389648,375.01897308)
\lineto(843.53389648,375.07897308)
\curveto(843.4538919,375.09896628)(843.36889198,375.10896627)(843.27889648,375.10897308)
\lineto(843.00889648,375.10897308)
\lineto(842.73889648,375.06397308)
\curveto(842.63889271,375.04396634)(842.54389281,375.01896636)(842.45389648,374.98897308)
\curveto(842.37389298,374.96896641)(842.29389306,374.93896644)(842.21389648,374.89897308)
\curveto(842.14389321,374.8789665)(842.07889327,374.84896653)(842.01889648,374.80897308)
\curveto(841.95889339,374.76896661)(841.90389345,374.72896665)(841.85389648,374.68897308)
\curveto(841.61389374,374.51896686)(841.41889393,374.31396707)(841.26889648,374.07397308)
\curveto(841.11889423,373.83396755)(840.98889436,373.55396783)(840.87889648,373.23397308)
\curveto(840.8488945,373.13396825)(840.82889452,373.02896835)(840.81889648,372.91897308)
\curveto(840.80889454,372.81896856)(840.79389456,372.71396867)(840.77389648,372.60397308)
\curveto(840.76389459,372.56396882)(840.75889459,372.49896888)(840.75889648,372.40897308)
\curveto(840.7488946,372.378969)(840.74389461,372.34396904)(840.74389648,372.30397308)
\curveto(840.7538946,372.26396912)(840.75889459,372.21896916)(840.75889648,372.16897308)
\lineto(840.75889648,371.86897308)
\curveto(840.75889459,371.76896961)(840.76889458,371.6789697)(840.78889648,371.59897308)
\lineto(840.81889648,371.41897308)
\curveto(840.83889451,371.31897006)(840.8538945,371.21897016)(840.86389648,371.11897308)
\curveto(840.88389447,371.02897035)(840.91389444,370.94397044)(840.95389648,370.86397308)
\curveto(841.0538943,370.62397076)(841.16889418,370.39897098)(841.29889648,370.18897308)
\curveto(841.43889391,369.9789714)(841.60889374,369.80397158)(841.80889648,369.66397308)
\curveto(841.85889349,369.63397175)(841.90389345,369.60897177)(841.94389648,369.58897308)
\curveto(841.98389337,369.56897181)(842.02889332,369.54397184)(842.07889648,369.51397308)
\curveto(842.15889319,369.46397192)(842.24389311,369.41897196)(842.33389648,369.37897308)
\curveto(842.43389292,369.34897203)(842.53889281,369.31897206)(842.64889648,369.28897308)
\curveto(842.69889265,369.26897211)(842.74389261,369.25897212)(842.78389648,369.25897308)
\curveto(842.83389252,369.26897211)(842.88389247,369.26897211)(842.93389648,369.25897308)
\curveto(842.96389239,369.24897213)(843.02389233,369.23897214)(843.11389648,369.22897308)
\curveto(843.21389214,369.21897216)(843.28889206,369.22397216)(843.33889648,369.24397308)
\curveto(843.37889197,369.25397213)(843.41889193,369.25397213)(843.45889648,369.24397308)
\curveto(843.49889185,369.24397214)(843.53889181,369.25397213)(843.57889648,369.27397308)
\curveto(843.65889169,369.29397209)(843.73889161,369.30897207)(843.81889648,369.31897308)
\curveto(843.89889145,369.33897204)(843.97389138,369.36397202)(844.04389648,369.39397308)
\curveto(844.38389097,369.53397185)(844.65889069,369.72897165)(844.86889648,369.97897308)
\curveto(845.07889027,370.22897115)(845.2538901,370.52397086)(845.39389648,370.86397308)
\curveto(845.44388991,370.9839704)(845.47388988,371.10897027)(845.48389648,371.23897308)
\curveto(845.50388985,371.37897)(845.53388982,371.51896986)(845.57389648,371.65897308)
}
}
{
\newrgbcolor{curcolor}{0 0 0}
\pscustom[linestyle=none,fillstyle=solid,fillcolor=curcolor]
{
\newpath
\moveto(850.82717773,376.21897308)
\curveto(851.54717367,376.22896515)(852.15217306,376.14396524)(852.64217773,375.96397308)
\curveto(853.13217208,375.79396559)(853.5121717,375.48896589)(853.78217773,375.04897308)
\curveto(853.85217136,374.93896644)(853.90717131,374.82396656)(853.94717773,374.70397308)
\curveto(853.98717123,374.59396679)(854.02717119,374.46896691)(854.06717773,374.32897308)
\curveto(854.08717113,374.25896712)(854.09217112,374.1839672)(854.08217773,374.10397308)
\curveto(854.07217114,374.03396735)(854.05717116,373.9789674)(854.03717773,373.93897308)
\curveto(854.0171712,373.91896746)(853.99217122,373.89896748)(853.96217773,373.87897308)
\curveto(853.93217128,373.86896751)(853.90717131,373.85396753)(853.88717773,373.83397308)
\curveto(853.83717138,373.81396757)(853.78717143,373.80896757)(853.73717773,373.81897308)
\curveto(853.68717153,373.82896755)(853.63717158,373.82896755)(853.58717773,373.81897308)
\curveto(853.50717171,373.79896758)(853.40217181,373.79396759)(853.27217773,373.80397308)
\curveto(853.14217207,373.82396756)(853.05217216,373.84896753)(853.00217773,373.87897308)
\curveto(852.92217229,373.92896745)(852.86717235,373.99396739)(852.83717773,374.07397308)
\curveto(852.8171724,374.16396722)(852.78217243,374.24896713)(852.73217773,374.32897308)
\curveto(852.64217257,374.48896689)(852.5171727,374.63396675)(852.35717773,374.76397308)
\curveto(852.24717297,374.84396654)(852.12717309,374.90396648)(851.99717773,374.94397308)
\curveto(851.86717335,374.9839664)(851.72717349,375.02396636)(851.57717773,375.06397308)
\curveto(851.52717369,375.0839663)(851.47717374,375.08896629)(851.42717773,375.07897308)
\curveto(851.37717384,375.0789663)(851.32717389,375.0839663)(851.27717773,375.09397308)
\curveto(851.217174,375.11396627)(851.14217407,375.12396626)(851.05217773,375.12397308)
\curveto(850.96217425,375.12396626)(850.88717433,375.11396627)(850.82717773,375.09397308)
\lineto(850.73717773,375.09397308)
\lineto(850.58717773,375.06397308)
\curveto(850.53717468,375.06396632)(850.48717473,375.05896632)(850.43717773,375.04897308)
\curveto(850.17717504,374.98896639)(849.96217525,374.90396648)(849.79217773,374.79397308)
\curveto(849.62217559,374.6839667)(849.50717571,374.49896688)(849.44717773,374.23897308)
\curveto(849.42717579,374.16896721)(849.42217579,374.09896728)(849.43217773,374.02897308)
\curveto(849.45217576,373.95896742)(849.47217574,373.89896748)(849.49217773,373.84897308)
\curveto(849.55217566,373.69896768)(849.62217559,373.58896779)(849.70217773,373.51897308)
\curveto(849.79217542,373.45896792)(849.90217531,373.38896799)(850.03217773,373.30897308)
\curveto(850.19217502,373.20896817)(850.37217484,373.13396825)(850.57217773,373.08397308)
\curveto(850.77217444,373.04396834)(850.97217424,372.99396839)(851.17217773,372.93397308)
\curveto(851.30217391,372.89396849)(851.43217378,372.86396852)(851.56217773,372.84397308)
\curveto(851.69217352,372.82396856)(851.82217339,372.79396859)(851.95217773,372.75397308)
\curveto(852.16217305,372.69396869)(852.36717285,372.63396875)(852.56717773,372.57397308)
\curveto(852.76717245,372.52396886)(852.96717225,372.45896892)(853.16717773,372.37897308)
\lineto(853.31717773,372.31897308)
\curveto(853.36717185,372.29896908)(853.4171718,372.27396911)(853.46717773,372.24397308)
\curveto(853.66717155,372.12396926)(853.84217137,371.98896939)(853.99217773,371.83897308)
\curveto(854.14217107,371.68896969)(854.26717095,371.49896988)(854.36717773,371.26897308)
\curveto(854.38717083,371.19897018)(854.40717081,371.10397028)(854.42717773,370.98397308)
\curveto(854.44717077,370.91397047)(854.45717076,370.83897054)(854.45717773,370.75897308)
\curveto(854.46717075,370.68897069)(854.47217074,370.60897077)(854.47217773,370.51897308)
\lineto(854.47217773,370.36897308)
\curveto(854.45217076,370.29897108)(854.44217077,370.22897115)(854.44217773,370.15897308)
\curveto(854.44217077,370.08897129)(854.43217078,370.01897136)(854.41217773,369.94897308)
\curveto(854.38217083,369.83897154)(854.34717087,369.73397165)(854.30717773,369.63397308)
\curveto(854.26717095,369.53397185)(854.22217099,369.44397194)(854.17217773,369.36397308)
\curveto(854.0121712,369.10397228)(853.80717141,368.89397249)(853.55717773,368.73397308)
\curveto(853.30717191,368.5839728)(853.02717219,368.45397293)(852.71717773,368.34397308)
\curveto(852.62717259,368.31397307)(852.53217268,368.29397309)(852.43217773,368.28397308)
\curveto(852.34217287,368.26397312)(852.25217296,368.23897314)(852.16217773,368.20897308)
\curveto(852.06217315,368.18897319)(851.96217325,368.1789732)(851.86217773,368.17897308)
\curveto(851.76217345,368.1789732)(851.66217355,368.16897321)(851.56217773,368.14897308)
\lineto(851.41217773,368.14897308)
\curveto(851.36217385,368.13897324)(851.29217392,368.13397325)(851.20217773,368.13397308)
\curveto(851.1121741,368.13397325)(851.04217417,368.13897324)(850.99217773,368.14897308)
\lineto(850.82717773,368.14897308)
\curveto(850.76717445,368.16897321)(850.70217451,368.1789732)(850.63217773,368.17897308)
\curveto(850.56217465,368.16897321)(850.50217471,368.17397321)(850.45217773,368.19397308)
\curveto(850.40217481,368.20397318)(850.33717488,368.20897317)(850.25717773,368.20897308)
\lineto(850.01717773,368.26897308)
\curveto(849.94717527,368.2789731)(849.87217534,368.29897308)(849.79217773,368.32897308)
\curveto(849.48217573,368.42897295)(849.212176,368.55397283)(848.98217773,368.70397308)
\curveto(848.75217646,368.85397253)(848.55217666,369.04897233)(848.38217773,369.28897308)
\curveto(848.29217692,369.41897196)(848.217177,369.55397183)(848.15717773,369.69397308)
\curveto(848.09717712,369.83397155)(848.04217717,369.98897139)(847.99217773,370.15897308)
\curveto(847.97217724,370.21897116)(847.96217725,370.28897109)(847.96217773,370.36897308)
\curveto(847.97217724,370.45897092)(847.98717723,370.52897085)(848.00717773,370.57897308)
\curveto(848.03717718,370.61897076)(848.08717713,370.65897072)(848.15717773,370.69897308)
\curveto(848.20717701,370.71897066)(848.27717694,370.72897065)(848.36717773,370.72897308)
\curveto(848.45717676,370.73897064)(848.54717667,370.73897064)(848.63717773,370.72897308)
\curveto(848.72717649,370.71897066)(848.8121764,370.70397068)(848.89217773,370.68397308)
\curveto(848.98217623,370.67397071)(849.04217617,370.65897072)(849.07217773,370.63897308)
\curveto(849.14217607,370.58897079)(849.18717603,370.51397087)(849.20717773,370.41397308)
\curveto(849.23717598,370.32397106)(849.27217594,370.23897114)(849.31217773,370.15897308)
\curveto(849.4121758,369.93897144)(849.54717567,369.76897161)(849.71717773,369.64897308)
\curveto(849.83717538,369.55897182)(849.97217524,369.48897189)(850.12217773,369.43897308)
\curveto(850.27217494,369.38897199)(850.43217478,369.33897204)(850.60217773,369.28897308)
\lineto(850.91717773,369.24397308)
\lineto(851.00717773,369.24397308)
\curveto(851.07717414,369.22397216)(851.16717405,369.21397217)(851.27717773,369.21397308)
\curveto(851.39717382,369.21397217)(851.49717372,369.22397216)(851.57717773,369.24397308)
\curveto(851.64717357,369.24397214)(851.70217351,369.24897213)(851.74217773,369.25897308)
\curveto(851.80217341,369.26897211)(851.86217335,369.27397211)(851.92217773,369.27397308)
\curveto(851.98217323,369.2839721)(852.03717318,369.29397209)(852.08717773,369.30397308)
\curveto(852.37717284,369.383972)(852.60717261,369.48897189)(852.77717773,369.61897308)
\curveto(852.94717227,369.74897163)(853.06717215,369.96897141)(853.13717773,370.27897308)
\curveto(853.15717206,370.32897105)(853.16217205,370.383971)(853.15217773,370.44397308)
\curveto(853.14217207,370.50397088)(853.13217208,370.54897083)(853.12217773,370.57897308)
\curveto(853.07217214,370.76897061)(853.00217221,370.90897047)(852.91217773,370.99897308)
\curveto(852.82217239,371.09897028)(852.70717251,371.18897019)(852.56717773,371.26897308)
\curveto(852.47717274,371.32897005)(852.37717284,371.37897)(852.26717773,371.41897308)
\lineto(851.93717773,371.53897308)
\curveto(851.90717331,371.54896983)(851.87717334,371.55396983)(851.84717773,371.55397308)
\curveto(851.82717339,371.55396983)(851.80217341,371.56396982)(851.77217773,371.58397308)
\curveto(851.43217378,371.69396969)(851.07717414,371.77396961)(850.70717773,371.82397308)
\curveto(850.34717487,371.8839695)(850.00717521,371.9789694)(849.68717773,372.10897308)
\curveto(849.58717563,372.14896923)(849.49217572,372.1839692)(849.40217773,372.21397308)
\curveto(849.3121759,372.24396914)(849.22717599,372.2839691)(849.14717773,372.33397308)
\curveto(848.95717626,372.44396894)(848.78217643,372.56896881)(848.62217773,372.70897308)
\curveto(848.46217675,372.84896853)(848.33717688,373.02396836)(848.24717773,373.23397308)
\curveto(848.217177,373.30396808)(848.19217702,373.37396801)(848.17217773,373.44397308)
\curveto(848.16217705,373.51396787)(848.14717707,373.58896779)(848.12717773,373.66897308)
\curveto(848.09717712,373.78896759)(848.08717713,373.92396746)(848.09717773,374.07397308)
\curveto(848.10717711,374.23396715)(848.12217709,374.36896701)(848.14217773,374.47897308)
\curveto(848.16217705,374.52896685)(848.17217704,374.56896681)(848.17217773,374.59897308)
\curveto(848.18217703,374.63896674)(848.19717702,374.6789667)(848.21717773,374.71897308)
\curveto(848.30717691,374.94896643)(848.42717679,375.14896623)(848.57717773,375.31897308)
\curveto(848.73717648,375.48896589)(848.9171763,375.63896574)(849.11717773,375.76897308)
\curveto(849.26717595,375.85896552)(849.43217578,375.92896545)(849.61217773,375.97897308)
\curveto(849.79217542,376.03896534)(849.98217523,376.09396529)(850.18217773,376.14397308)
\curveto(850.25217496,376.15396523)(850.3171749,376.16396522)(850.37717773,376.17397308)
\curveto(850.44717477,376.1839652)(850.52217469,376.19396519)(850.60217773,376.20397308)
\curveto(850.63217458,376.21396517)(850.67217454,376.21396517)(850.72217773,376.20397308)
\curveto(850.77217444,376.19396519)(850.80717441,376.19896518)(850.82717773,376.21897308)
}
}
{
\newrgbcolor{curcolor}{0 0 0}
\pscustom[linestyle=none,fillstyle=solid,fillcolor=curcolor]
{
\newpath
\moveto(775.96144531,355.67765076)
\curveto(776.03144364,355.67764006)(776.11144356,355.67764006)(776.20144531,355.67765076)
\curveto(776.29144338,355.68764005)(776.37644329,355.68764005)(776.45644531,355.67765076)
\curveto(776.54644312,355.67764006)(776.62644304,355.66764007)(776.69644531,355.64765076)
\curveto(776.7664429,355.62764011)(776.81644285,355.59764014)(776.84644531,355.55765076)
\curveto(776.90644276,355.47764026)(776.93644273,355.37264037)(776.93644531,355.24265076)
\lineto(776.93644531,354.83765076)
\lineto(776.93644531,353.26265076)
\lineto(776.93644531,348.44765076)
\lineto(776.93644531,347.06765076)
\lineto(776.93644531,346.70765076)
\curveto(776.93644273,346.57764916)(776.95144272,346.47264927)(776.98144531,346.39265076)
\curveto(777.01144266,346.32264942)(777.0664426,346.26264948)(777.14644531,346.21265076)
\curveto(777.19644247,346.19264955)(777.26144241,346.17764956)(777.34144531,346.16765076)
\lineto(777.58144531,346.16765076)
\lineto(778.34644531,346.16765076)
\lineto(781.04644531,346.16765076)
\lineto(781.88644531,346.16765076)
\lineto(782.09644531,346.16765076)
\curveto(782.17643749,346.17764956)(782.24643742,346.17264957)(782.30644531,346.15265076)
\curveto(782.43643723,346.11264963)(782.52143715,346.05764968)(782.56144531,345.98765076)
\curveto(782.5714371,345.95764978)(782.58143709,345.90764983)(782.59144531,345.83765076)
\curveto(782.60143707,345.76764997)(782.60643706,345.69265005)(782.60644531,345.61265076)
\curveto(782.61643705,345.5426502)(782.61643705,345.46765027)(782.60644531,345.38765076)
\curveto(782.59643707,345.31765042)(782.58643708,345.26265048)(782.57644531,345.22265076)
\curveto(782.55643711,345.15265059)(782.51143716,345.09765064)(782.44144531,345.05765076)
\curveto(782.39143728,345.01765072)(782.31643735,344.99765074)(782.21644531,344.99765076)
\lineto(781.94644531,344.99765076)
\lineto(780.95644531,344.99765076)
\lineto(777.16144531,344.99765076)
\lineto(776.18644531,344.99765076)
\curveto(776.04644362,344.99765074)(775.92644374,345.00265074)(775.82644531,345.01265076)
\curveto(775.72644394,345.03265071)(775.65144402,345.08265066)(775.60144531,345.16265076)
\curveto(775.56144411,345.22265052)(775.54144413,345.29765044)(775.54144531,345.38765076)
\lineto(775.54144531,345.67265076)
\lineto(775.54144531,346.75265076)
\lineto(775.54144531,350.84765076)
\lineto(775.54144531,354.10265076)
\lineto(775.54144531,355.03265076)
\lineto(775.54144531,355.30265076)
\curveto(775.55144412,355.39264035)(775.5714441,355.46264028)(775.60144531,355.51265076)
\curveto(775.64144403,355.57264017)(775.71644395,355.62264012)(775.82644531,355.66265076)
\curveto(775.84644382,355.67264007)(775.8664438,355.67264007)(775.88644531,355.66265076)
\curveto(775.91644375,355.66264008)(775.94144373,355.66764007)(775.96144531,355.67765076)
}
}
{
\newrgbcolor{curcolor}{0 0 0}
\pscustom[linestyle=none,fillstyle=solid,fillcolor=curcolor]
{
\newpath
\moveto(790.74605469,349.19765076)
\curveto(790.76604663,349.1376466)(790.77604662,349.0426467)(790.77605469,348.91265076)
\curveto(790.77604662,348.79264695)(790.77104662,348.70764703)(790.76105469,348.65765076)
\lineto(790.76105469,348.50765076)
\curveto(790.75104664,348.42764731)(790.74104665,348.35264739)(790.73105469,348.28265076)
\curveto(790.73104666,348.22264752)(790.72604667,348.15264759)(790.71605469,348.07265076)
\curveto(790.6960467,348.01264773)(790.68104671,347.95264779)(790.67105469,347.89265076)
\curveto(790.67104672,347.83264791)(790.66104673,347.77264797)(790.64105469,347.71265076)
\curveto(790.60104679,347.58264816)(790.56604683,347.45264829)(790.53605469,347.32265076)
\curveto(790.50604689,347.19264855)(790.46604693,347.07264867)(790.41605469,346.96265076)
\curveto(790.20604719,346.48264926)(789.92604747,346.07764966)(789.57605469,345.74765076)
\curveto(789.22604817,345.42765031)(788.7960486,345.18265056)(788.28605469,345.01265076)
\curveto(788.17604922,344.97265077)(788.05604934,344.9426508)(787.92605469,344.92265076)
\curveto(787.80604959,344.90265084)(787.68104971,344.88265086)(787.55105469,344.86265076)
\curveto(787.4910499,344.85265089)(787.42604997,344.84765089)(787.35605469,344.84765076)
\curveto(787.2960501,344.8376509)(787.23605016,344.83265091)(787.17605469,344.83265076)
\curveto(787.13605026,344.82265092)(787.07605032,344.81765092)(786.99605469,344.81765076)
\curveto(786.92605047,344.81765092)(786.87605052,344.82265092)(786.84605469,344.83265076)
\curveto(786.80605059,344.8426509)(786.76605063,344.84765089)(786.72605469,344.84765076)
\curveto(786.68605071,344.8376509)(786.65105074,344.8376509)(786.62105469,344.84765076)
\lineto(786.53105469,344.84765076)
\lineto(786.17105469,344.89265076)
\curveto(786.03105136,344.93265081)(785.8960515,344.97265077)(785.76605469,345.01265076)
\curveto(785.63605176,345.05265069)(785.51105188,345.09765064)(785.39105469,345.14765076)
\curveto(784.94105245,345.34765039)(784.57105282,345.60765013)(784.28105469,345.92765076)
\curveto(783.9910534,346.24764949)(783.75105364,346.6376491)(783.56105469,347.09765076)
\curveto(783.51105388,347.19764854)(783.47105392,347.29764844)(783.44105469,347.39765076)
\curveto(783.42105397,347.49764824)(783.40105399,347.60264814)(783.38105469,347.71265076)
\curveto(783.36105403,347.75264799)(783.35105404,347.78264796)(783.35105469,347.80265076)
\curveto(783.36105403,347.83264791)(783.36105403,347.86764787)(783.35105469,347.90765076)
\curveto(783.33105406,347.98764775)(783.31605408,348.06764767)(783.30605469,348.14765076)
\curveto(783.30605409,348.2376475)(783.2960541,348.32264742)(783.27605469,348.40265076)
\lineto(783.27605469,348.52265076)
\curveto(783.27605412,348.56264718)(783.27105412,348.60764713)(783.26105469,348.65765076)
\curveto(783.25105414,348.70764703)(783.24605415,348.79264695)(783.24605469,348.91265076)
\curveto(783.24605415,349.0426467)(783.25605414,349.1376466)(783.27605469,349.19765076)
\curveto(783.2960541,349.26764647)(783.30105409,349.3376464)(783.29105469,349.40765076)
\curveto(783.28105411,349.47764626)(783.28605411,349.54764619)(783.30605469,349.61765076)
\curveto(783.31605408,349.66764607)(783.32105407,349.70764603)(783.32105469,349.73765076)
\curveto(783.33105406,349.77764596)(783.34105405,349.82264592)(783.35105469,349.87265076)
\curveto(783.38105401,349.99264575)(783.40605399,350.11264563)(783.42605469,350.23265076)
\curveto(783.45605394,350.35264539)(783.4960539,350.46764527)(783.54605469,350.57765076)
\curveto(783.6960537,350.94764479)(783.87605352,351.27764446)(784.08605469,351.56765076)
\curveto(784.30605309,351.86764387)(784.57105282,352.11764362)(784.88105469,352.31765076)
\curveto(785.00105239,352.39764334)(785.12605227,352.46264328)(785.25605469,352.51265076)
\curveto(785.38605201,352.57264317)(785.52105187,352.63264311)(785.66105469,352.69265076)
\curveto(785.78105161,352.742643)(785.91105148,352.77264297)(786.05105469,352.78265076)
\curveto(786.1910512,352.80264294)(786.33105106,352.83264291)(786.47105469,352.87265076)
\lineto(786.66605469,352.87265076)
\curveto(786.73605066,352.88264286)(786.80105059,352.89264285)(786.86105469,352.90265076)
\curveto(787.75104964,352.91264283)(788.4910489,352.72764301)(789.08105469,352.34765076)
\curveto(789.67104772,351.96764377)(790.0960473,351.47264427)(790.35605469,350.86265076)
\curveto(790.40604699,350.76264498)(790.44604695,350.66264508)(790.47605469,350.56265076)
\curveto(790.50604689,350.46264528)(790.54104685,350.35764538)(790.58105469,350.24765076)
\curveto(790.61104678,350.1376456)(790.63604676,350.01764572)(790.65605469,349.88765076)
\curveto(790.67604672,349.76764597)(790.70104669,349.6426461)(790.73105469,349.51265076)
\curveto(790.74104665,349.46264628)(790.74104665,349.40764633)(790.73105469,349.34765076)
\curveto(790.73104666,349.29764644)(790.73604666,349.24764649)(790.74605469,349.19765076)
\moveto(789.41105469,348.34265076)
\curveto(789.43104796,348.41264733)(789.43604796,348.49264725)(789.42605469,348.58265076)
\lineto(789.42605469,348.83765076)
\curveto(789.42604797,349.22764651)(789.391048,349.55764618)(789.32105469,349.82765076)
\curveto(789.2910481,349.90764583)(789.26604813,349.98764575)(789.24605469,350.06765076)
\curveto(789.22604817,350.14764559)(789.20104819,350.22264552)(789.17105469,350.29265076)
\curveto(788.8910485,350.9426448)(788.44604895,351.39264435)(787.83605469,351.64265076)
\curveto(787.76604963,351.67264407)(787.6910497,351.69264405)(787.61105469,351.70265076)
\lineto(787.37105469,351.76265076)
\curveto(787.2910501,351.78264396)(787.20605019,351.79264395)(787.11605469,351.79265076)
\lineto(786.84605469,351.79265076)
\lineto(786.57605469,351.74765076)
\curveto(786.47605092,351.72764401)(786.38105101,351.70264404)(786.29105469,351.67265076)
\curveto(786.21105118,351.65264409)(786.13105126,351.62264412)(786.05105469,351.58265076)
\curveto(785.98105141,351.56264418)(785.91605148,351.53264421)(785.85605469,351.49265076)
\curveto(785.7960516,351.45264429)(785.74105165,351.41264433)(785.69105469,351.37265076)
\curveto(785.45105194,351.20264454)(785.25605214,350.99764474)(785.10605469,350.75765076)
\curveto(784.95605244,350.51764522)(784.82605257,350.2376455)(784.71605469,349.91765076)
\curveto(784.68605271,349.81764592)(784.66605273,349.71264603)(784.65605469,349.60265076)
\curveto(784.64605275,349.50264624)(784.63105276,349.39764634)(784.61105469,349.28765076)
\curveto(784.60105279,349.24764649)(784.5960528,349.18264656)(784.59605469,349.09265076)
\curveto(784.58605281,349.06264668)(784.58105281,349.02764671)(784.58105469,348.98765076)
\curveto(784.5910528,348.94764679)(784.5960528,348.90264684)(784.59605469,348.85265076)
\lineto(784.59605469,348.55265076)
\curveto(784.5960528,348.45264729)(784.60605279,348.36264738)(784.62605469,348.28265076)
\lineto(784.65605469,348.10265076)
\curveto(784.67605272,348.00264774)(784.6910527,347.90264784)(784.70105469,347.80265076)
\curveto(784.72105267,347.71264803)(784.75105264,347.62764811)(784.79105469,347.54765076)
\curveto(784.8910525,347.30764843)(785.00605239,347.08264866)(785.13605469,346.87265076)
\curveto(785.27605212,346.66264908)(785.44605195,346.48764925)(785.64605469,346.34765076)
\curveto(785.6960517,346.31764942)(785.74105165,346.29264945)(785.78105469,346.27265076)
\curveto(785.82105157,346.25264949)(785.86605153,346.22764951)(785.91605469,346.19765076)
\curveto(785.9960514,346.14764959)(786.08105131,346.10264964)(786.17105469,346.06265076)
\curveto(786.27105112,346.03264971)(786.37605102,346.00264974)(786.48605469,345.97265076)
\curveto(786.53605086,345.95264979)(786.58105081,345.9426498)(786.62105469,345.94265076)
\curveto(786.67105072,345.95264979)(786.72105067,345.95264979)(786.77105469,345.94265076)
\curveto(786.80105059,345.93264981)(786.86105053,345.92264982)(786.95105469,345.91265076)
\curveto(787.05105034,345.90264984)(787.12605027,345.90764983)(787.17605469,345.92765076)
\curveto(787.21605018,345.9376498)(787.25605014,345.9376498)(787.29605469,345.92765076)
\curveto(787.33605006,345.92764981)(787.37605002,345.9376498)(787.41605469,345.95765076)
\curveto(787.4960499,345.97764976)(787.57604982,345.99264975)(787.65605469,346.00265076)
\curveto(787.73604966,346.02264972)(787.81104958,346.04764969)(787.88105469,346.07765076)
\curveto(788.22104917,346.21764952)(788.4960489,346.41264933)(788.70605469,346.66265076)
\curveto(788.91604848,346.91264883)(789.0910483,347.20764853)(789.23105469,347.54765076)
\curveto(789.28104811,347.66764807)(789.31104808,347.79264795)(789.32105469,347.92265076)
\curveto(789.34104805,348.06264768)(789.37104802,348.20264754)(789.41105469,348.34265076)
}
}
{
\newrgbcolor{curcolor}{0 0 0}
\pscustom[linestyle=none,fillstyle=solid,fillcolor=curcolor]
{
\newpath
\moveto(798.84933594,352.61765076)
\curveto(798.91932834,352.56764317)(798.9543283,352.49264325)(798.95433594,352.39265076)
\curveto(798.96432829,352.29264345)(798.96932829,352.18764355)(798.96933594,352.07765076)
\lineto(798.96933594,345.80765076)
\lineto(798.96933594,345.20765076)
\curveto(798.94932831,345.15765058)(798.94432831,345.10765063)(798.95433594,345.05765076)
\curveto(798.96432829,345.01765072)(798.9593283,344.97265077)(798.93933594,344.92265076)
\curveto(798.91932834,344.82265092)(798.90432835,344.72265102)(798.89433594,344.62265076)
\curveto(798.89432836,344.51265123)(798.87932838,344.40765133)(798.84933594,344.30765076)
\curveto(798.81932844,344.19765154)(798.78932847,344.09265165)(798.75933594,343.99265076)
\curveto(798.73932852,343.89265185)(798.70432855,343.79265195)(798.65433594,343.69265076)
\curveto(798.5543287,343.43265231)(798.42432883,343.19765254)(798.26433594,342.98765076)
\curveto(798.11432914,342.77765296)(797.93432932,342.60265314)(797.72433594,342.46265076)
\curveto(797.5543297,342.3426534)(797.37432988,342.24765349)(797.18433594,342.17765076)
\curveto(796.99433026,342.09765364)(796.78933047,342.02265372)(796.56933594,341.95265076)
\curveto(796.47933078,341.93265381)(796.38933087,341.92265382)(796.29933594,341.92265076)
\curveto(796.20933105,341.91265383)(796.11933114,341.89765384)(796.02933594,341.87765076)
\lineto(795.93933594,341.87765076)
\curveto(795.91933134,341.86765387)(795.89933136,341.86265388)(795.87933594,341.86265076)
\curveto(795.82933143,341.85265389)(795.77933148,341.85265389)(795.72933594,341.86265076)
\curveto(795.68933157,341.87265387)(795.64433161,341.86765387)(795.59433594,341.84765076)
\curveto(795.52433173,341.82765391)(795.41433184,341.82265392)(795.26433594,341.83265076)
\curveto(795.12433213,341.83265391)(795.02433223,341.8426539)(794.96433594,341.86265076)
\curveto(794.93433232,341.86265388)(794.90433235,341.86765387)(794.87433594,341.87765076)
\lineto(794.81433594,341.87765076)
\curveto(794.72433253,341.89765384)(794.63433262,341.91265383)(794.54433594,341.92265076)
\curveto(794.4543328,341.92265382)(794.36933289,341.93265381)(794.28933594,341.95265076)
\curveto(794.20933305,341.97265377)(794.12933313,341.99765374)(794.04933594,342.02765076)
\curveto(793.96933329,342.04765369)(793.88933337,342.07265367)(793.80933594,342.10265076)
\curveto(793.48933377,342.23265351)(793.21933404,342.37765336)(792.99933594,342.53765076)
\curveto(792.78933447,342.69765304)(792.59933466,342.92265282)(792.42933594,343.21265076)
\curveto(792.40933485,343.23265251)(792.39433486,343.25765248)(792.38433594,343.28765076)
\curveto(792.38433487,343.30765243)(792.37433488,343.33265241)(792.35433594,343.36265076)
\curveto(792.32433493,343.4426523)(792.28933497,343.55765218)(792.24933594,343.70765076)
\curveto(792.21933504,343.84765189)(792.24933501,343.95265179)(792.33933594,344.02265076)
\curveto(792.39933486,344.07265167)(792.47933478,344.09765164)(792.57933594,344.09765076)
\lineto(792.90933594,344.09765076)
\lineto(793.07433594,344.09765076)
\curveto(793.13433412,344.09765164)(793.18933407,344.08765165)(793.23933594,344.06765076)
\curveto(793.32933393,344.0376517)(793.39433386,343.98765175)(793.43433594,343.91765076)
\curveto(793.47433378,343.84765189)(793.51933374,343.77265197)(793.56933594,343.69265076)
\lineto(793.68933594,343.51265076)
\curveto(793.73933352,343.4426523)(793.78933347,343.38765235)(793.83933594,343.34765076)
\curveto(794.08933317,343.15765258)(794.38933287,343.01765272)(794.73933594,342.92765076)
\curveto(794.79933246,342.90765283)(794.8593324,342.89765284)(794.91933594,342.89765076)
\curveto(794.98933227,342.88765285)(795.0543322,342.87265287)(795.11433594,342.85265076)
\lineto(795.20433594,342.85265076)
\curveto(795.27433198,342.83265291)(795.3593319,342.82265292)(795.45933594,342.82265076)
\curveto(795.5593317,342.82265292)(795.64933161,342.83265291)(795.72933594,342.85265076)
\curveto(795.7593315,342.86265288)(795.79933146,342.86765287)(795.84933594,342.86765076)
\curveto(795.94933131,342.88765285)(796.04433121,342.90765283)(796.13433594,342.92765076)
\curveto(796.22433103,342.9376528)(796.30933095,342.96265278)(796.38933594,343.00265076)
\curveto(796.67933058,343.12265262)(796.91433034,343.28765245)(797.09433594,343.49765076)
\curveto(797.28432997,343.69765204)(797.43932982,343.9426518)(797.55933594,344.23265076)
\curveto(797.59932966,344.32265142)(797.62432963,344.41765132)(797.63433594,344.51765076)
\curveto(797.6543296,344.61765112)(797.67932958,344.72265102)(797.70933594,344.83265076)
\curveto(797.72932953,344.88265086)(797.73932952,344.93265081)(797.73933594,344.98265076)
\curveto(797.73932952,345.03265071)(797.74432951,345.08265066)(797.75433594,345.13265076)
\curveto(797.76432949,345.16265058)(797.76932949,345.21265053)(797.76933594,345.28265076)
\curveto(797.78932947,345.36265038)(797.78932947,345.44765029)(797.76933594,345.53765076)
\curveto(797.7593295,345.58765015)(797.7543295,345.63265011)(797.75433594,345.67265076)
\curveto(797.76432949,345.71265003)(797.7593295,345.74764999)(797.73933594,345.77765076)
\curveto(797.71932954,345.79764994)(797.70432955,345.80764993)(797.69433594,345.80765076)
\lineto(797.64933594,345.85265076)
\curveto(797.54932971,345.85264989)(797.47432978,345.82264992)(797.42433594,345.76265076)
\curveto(797.38432987,345.71265003)(797.33432992,345.66765007)(797.27433594,345.62765076)
\lineto(797.03433594,345.41765076)
\curveto(796.9543303,345.35765038)(796.86433039,345.30265044)(796.76433594,345.25265076)
\curveto(796.62433063,345.16265058)(796.44933081,345.08765065)(796.23933594,345.02765076)
\curveto(796.02933123,344.97765076)(795.80933145,344.9426508)(795.57933594,344.92265076)
\curveto(795.34933191,344.90265084)(795.11933214,344.90765083)(794.88933594,344.93765076)
\curveto(794.6593326,344.95765078)(794.44933281,344.99765074)(794.25933594,345.05765076)
\curveto(793.31933394,345.36765037)(792.6593346,345.96264978)(792.27933594,346.84265076)
\curveto(792.22933503,346.9426488)(792.18933507,347.0376487)(792.15933594,347.12765076)
\curveto(792.12933513,347.22764851)(792.09433516,347.33264841)(792.05433594,347.44265076)
\curveto(792.03433522,347.49264825)(792.02433523,347.5376482)(792.02433594,347.57765076)
\curveto(792.02433523,347.61764812)(792.01433524,347.66264808)(791.99433594,347.71265076)
\curveto(791.97433528,347.78264796)(791.9593353,347.85264789)(791.94933594,347.92265076)
\curveto(791.94933531,348.00264774)(791.93933532,348.07764766)(791.91933594,348.14765076)
\curveto(791.90933535,348.18764755)(791.90433535,348.22264752)(791.90433594,348.25265076)
\curveto(791.91433534,348.29264745)(791.91433534,348.33264741)(791.90433594,348.37265076)
\curveto(791.90433535,348.41264733)(791.89933536,348.45264729)(791.88933594,348.49265076)
\lineto(791.88933594,348.61265076)
\curveto(791.86933539,348.73264701)(791.86933539,348.85764688)(791.88933594,348.98765076)
\curveto(791.89933536,349.04764669)(791.90433535,349.10764663)(791.90433594,349.16765076)
\lineto(791.90433594,349.33265076)
\curveto(791.91433534,349.38264636)(791.91933534,349.42264632)(791.91933594,349.45265076)
\curveto(791.91933534,349.49264625)(791.92433533,349.5376462)(791.93433594,349.58765076)
\curveto(791.96433529,349.69764604)(791.98433527,349.80264594)(791.99433594,349.90265076)
\curveto(792.00433525,350.01264573)(792.02933523,350.12264562)(792.06933594,350.23265076)
\curveto(792.10933515,350.35264539)(792.14433511,350.46764527)(792.17433594,350.57765076)
\curveto(792.21433504,350.69764504)(792.259335,350.81264493)(792.30933594,350.92265076)
\curveto(792.37933488,351.08264466)(792.4593348,351.22764451)(792.54933594,351.35765076)
\curveto(792.63933462,351.49764424)(792.73433452,351.63264411)(792.83433594,351.76265076)
\curveto(792.90433435,351.87264387)(792.99433426,351.96264378)(793.10433594,352.03265076)
\lineto(793.16433594,352.09265076)
\lineto(793.22433594,352.15265076)
\lineto(793.37433594,352.27265076)
\lineto(793.55433594,352.39265076)
\curveto(793.68433357,352.47264327)(793.81933344,352.5426432)(793.95933594,352.60265076)
\curveto(794.10933315,352.66264308)(794.26933299,352.71764302)(794.43933594,352.76765076)
\curveto(794.53933272,352.79764294)(794.63933262,352.81764292)(794.73933594,352.82765076)
\curveto(794.84933241,352.8376429)(794.9593323,352.85264289)(795.06933594,352.87265076)
\curveto(795.10933215,352.88264286)(795.1593321,352.88264286)(795.21933594,352.87265076)
\curveto(795.28933197,352.86264288)(795.33933192,352.86764287)(795.36933594,352.88765076)
\curveto(795.68933157,352.89764284)(795.97433128,352.86764287)(796.22433594,352.79765076)
\curveto(796.48433077,352.72764301)(796.71433054,352.62764311)(796.91433594,352.49765076)
\curveto(796.98433027,352.45764328)(797.04933021,352.41264333)(797.10933594,352.36265076)
\lineto(797.28933594,352.21265076)
\curveto(797.33932992,352.17264357)(797.38432987,352.12764361)(797.42433594,352.07765076)
\curveto(797.47432978,352.0376437)(797.54932971,352.01764372)(797.64933594,352.01765076)
\lineto(797.69433594,352.06265076)
\curveto(797.71432954,352.08264366)(797.73432952,352.10764363)(797.75433594,352.13765076)
\curveto(797.78432947,352.21764352)(797.79932946,352.29764344)(797.79933594,352.37765076)
\curveto(797.80932945,352.45764328)(797.83932942,352.52764321)(797.88933594,352.58765076)
\curveto(797.91932934,352.62764311)(797.97932928,352.65764308)(798.06933594,352.67765076)
\curveto(798.1593291,352.70764303)(798.254329,352.72264302)(798.35433594,352.72265076)
\curveto(798.4543288,352.72264302)(798.54932871,352.71264303)(798.63933594,352.69265076)
\curveto(798.73932852,352.67264307)(798.80932845,352.64764309)(798.84933594,352.61765076)
\moveto(797.72433594,348.83765076)
\curveto(797.73432952,348.87764686)(797.73932952,348.92764681)(797.73933594,348.98765076)
\curveto(797.73932952,349.05764668)(797.73432952,349.11264663)(797.72433594,349.15265076)
\lineto(797.72433594,349.39265076)
\curveto(797.70432955,349.48264626)(797.68932957,349.56764617)(797.67933594,349.64765076)
\curveto(797.66932959,349.737646)(797.6543296,349.82264592)(797.63433594,349.90265076)
\curveto(797.61432964,349.98264576)(797.59432966,350.05764568)(797.57433594,350.12765076)
\curveto(797.56432969,350.20764553)(797.54432971,350.28264546)(797.51433594,350.35265076)
\curveto(797.40432985,350.63264511)(797.25933,350.88264486)(797.07933594,351.10265076)
\curveto(796.90933035,351.32264442)(796.68933057,351.48764425)(796.41933594,351.59765076)
\curveto(796.33933092,351.6376441)(796.254331,351.66764407)(796.16433594,351.68765076)
\curveto(796.07433118,351.71764402)(795.97933128,351.742644)(795.87933594,351.76265076)
\curveto(795.79933146,351.78264396)(795.70933155,351.78764395)(795.60933594,351.77765076)
\lineto(795.33933594,351.77765076)
\curveto(795.28933197,351.76764397)(795.23933202,351.76264398)(795.18933594,351.76265076)
\curveto(795.14933211,351.76264398)(795.10433215,351.75764398)(795.05433594,351.74765076)
\curveto(794.86433239,351.69764404)(794.70433255,351.64764409)(794.57433594,351.59765076)
\curveto(794.23433302,351.45764428)(793.96933329,351.24764449)(793.77933594,350.96765076)
\curveto(793.58933367,350.68764505)(793.43933382,350.36264538)(793.32933594,349.99265076)
\curveto(793.30933395,349.91264583)(793.29433396,349.83264591)(793.28433594,349.75265076)
\curveto(793.28433397,349.68264606)(793.27433398,349.60764613)(793.25433594,349.52765076)
\curveto(793.23433402,349.49764624)(793.22433403,349.46264628)(793.22433594,349.42265076)
\curveto(793.23433402,349.38264636)(793.23433402,349.34764639)(793.22433594,349.31765076)
\lineto(793.22433594,348.98765076)
\lineto(793.22433594,348.64265076)
\curveto(793.22433403,348.53264721)(793.23433402,348.42764731)(793.25433594,348.32765076)
\lineto(793.25433594,348.25265076)
\curveto(793.26433399,348.22264752)(793.26933399,348.19764754)(793.26933594,348.17765076)
\curveto(793.28933397,348.08764765)(793.30433395,347.99764774)(793.31433594,347.90765076)
\curveto(793.33433392,347.81764792)(793.3593339,347.73264801)(793.38933594,347.65265076)
\curveto(793.46933379,347.39264835)(793.56933369,347.15264859)(793.68933594,346.93265076)
\curveto(793.80933345,346.71264903)(793.96933329,346.53264921)(794.16933594,346.39265076)
\lineto(794.28933594,346.30265076)
\curveto(794.32933293,346.28264946)(794.37433288,346.26264948)(794.42433594,346.24265076)
\curveto(794.50433275,346.19264955)(794.58933267,346.15264959)(794.67933594,346.12265076)
\curveto(794.76933249,346.09264965)(794.86933239,346.06264968)(794.97933594,346.03265076)
\curveto(795.02933223,346.02264972)(795.07433218,346.01764972)(795.11433594,346.01765076)
\curveto(795.16433209,346.02764971)(795.21433204,346.02264972)(795.26433594,346.00265076)
\curveto(795.29433196,345.99264975)(795.34433191,345.98764975)(795.41433594,345.98765076)
\curveto(795.48433177,345.98764975)(795.53433172,345.99264975)(795.56433594,346.00265076)
\curveto(795.59433166,346.01264973)(795.62433163,346.01264973)(795.65433594,346.00265076)
\curveto(795.69433156,346.00264974)(795.73433152,346.00764973)(795.77433594,346.01765076)
\curveto(795.86433139,346.0376497)(795.94933131,346.05764968)(796.02933594,346.07765076)
\curveto(796.10933115,346.09764964)(796.18933107,346.12264962)(796.26933594,346.15265076)
\curveto(796.60933065,346.30264944)(796.87933038,346.51264923)(797.07933594,346.78265076)
\curveto(797.27932998,347.05264869)(797.43932982,347.36764837)(797.55933594,347.72765076)
\curveto(797.58932967,347.81764792)(797.60932965,347.90764783)(797.61933594,347.99765076)
\curveto(797.63932962,348.09764764)(797.6593296,348.19264755)(797.67933594,348.28265076)
\curveto(797.68932957,348.32264742)(797.69432956,348.35764738)(797.69433594,348.38765076)
\curveto(797.69432956,348.42764731)(797.69932956,348.46764727)(797.70933594,348.50765076)
\curveto(797.72932953,348.55764718)(797.72932953,348.60764713)(797.70933594,348.65765076)
\curveto(797.69932956,348.71764702)(797.70432955,348.77764696)(797.72433594,348.83765076)
}
}
{
\newrgbcolor{curcolor}{0 0 0}
\pscustom[linestyle=none,fillstyle=solid,fillcolor=curcolor]
{
\newpath
\moveto(807.59761719,349.16765076)
\curveto(807.6176095,349.06764667)(807.6176095,348.95264679)(807.59761719,348.82265076)
\curveto(807.58760953,348.70264704)(807.55760956,348.61764712)(807.50761719,348.56765076)
\curveto(807.45760966,348.52764721)(807.38260974,348.49764724)(807.28261719,348.47765076)
\curveto(807.19260993,348.46764727)(807.08761003,348.46264728)(806.96761719,348.46265076)
\lineto(806.60761719,348.46265076)
\curveto(806.48761063,348.47264727)(806.38261074,348.47764726)(806.29261719,348.47765076)
\lineto(802.45261719,348.47765076)
\curveto(802.37261475,348.47764726)(802.29261483,348.47264727)(802.21261719,348.46265076)
\curveto(802.13261499,348.46264728)(802.06761505,348.44764729)(802.01761719,348.41765076)
\curveto(801.97761514,348.39764734)(801.93761518,348.35764738)(801.89761719,348.29765076)
\curveto(801.87761524,348.26764747)(801.85761526,348.22264752)(801.83761719,348.16265076)
\curveto(801.8176153,348.11264763)(801.8176153,348.06264768)(801.83761719,348.01265076)
\curveto(801.84761527,347.96264778)(801.85261527,347.91764782)(801.85261719,347.87765076)
\curveto(801.85261527,347.8376479)(801.85761526,347.79764794)(801.86761719,347.75765076)
\curveto(801.88761523,347.67764806)(801.90761521,347.59264815)(801.92761719,347.50265076)
\curveto(801.94761517,347.42264832)(801.97761514,347.3426484)(802.01761719,347.26265076)
\curveto(802.24761487,346.72264902)(802.62761449,346.3376494)(803.15761719,346.10765076)
\curveto(803.2176139,346.07764966)(803.28261384,346.05264969)(803.35261719,346.03265076)
\lineto(803.56261719,345.97265076)
\curveto(803.59261353,345.96264978)(803.64261348,345.95764978)(803.71261719,345.95765076)
\curveto(803.85261327,345.91764982)(804.03761308,345.89764984)(804.26761719,345.89765076)
\curveto(804.49761262,345.89764984)(804.68261244,345.91764982)(804.82261719,345.95765076)
\curveto(804.96261216,345.99764974)(805.08761203,346.0376497)(805.19761719,346.07765076)
\curveto(805.3176118,346.12764961)(805.42761169,346.18764955)(805.52761719,346.25765076)
\curveto(805.63761148,346.32764941)(805.73261139,346.40764933)(805.81261719,346.49765076)
\curveto(805.89261123,346.59764914)(805.96261116,346.70264904)(806.02261719,346.81265076)
\curveto(806.08261104,346.91264883)(806.13261099,347.01764872)(806.17261719,347.12765076)
\curveto(806.2226109,347.2376485)(806.30261082,347.31764842)(806.41261719,347.36765076)
\curveto(806.45261067,347.38764835)(806.5176106,347.40264834)(806.60761719,347.41265076)
\curveto(806.69761042,347.42264832)(806.78761033,347.42264832)(806.87761719,347.41265076)
\curveto(806.96761015,347.41264833)(807.05261007,347.40764833)(807.13261719,347.39765076)
\curveto(807.21260991,347.38764835)(807.26760985,347.36764837)(807.29761719,347.33765076)
\curveto(807.39760972,347.26764847)(807.4226097,347.15264859)(807.37261719,346.99265076)
\curveto(807.29260983,346.72264902)(807.18760993,346.48264926)(807.05761719,346.27265076)
\curveto(806.85761026,345.95264979)(806.62761049,345.68765005)(806.36761719,345.47765076)
\curveto(806.117611,345.27765046)(805.79761132,345.11265063)(805.40761719,344.98265076)
\curveto(805.30761181,344.9426508)(805.20761191,344.91765082)(805.10761719,344.90765076)
\curveto(805.00761211,344.88765085)(804.90261222,344.86765087)(804.79261719,344.84765076)
\curveto(804.74261238,344.8376509)(804.69261243,344.83265091)(804.64261719,344.83265076)
\curveto(804.60261252,344.83265091)(804.55761256,344.82765091)(804.50761719,344.81765076)
\lineto(804.35761719,344.81765076)
\curveto(804.30761281,344.80765093)(804.24761287,344.80265094)(804.17761719,344.80265076)
\curveto(804.117613,344.80265094)(804.06761305,344.80765093)(804.02761719,344.81765076)
\lineto(803.89261719,344.81765076)
\curveto(803.84261328,344.82765091)(803.79761332,344.83265091)(803.75761719,344.83265076)
\curveto(803.7176134,344.83265091)(803.67761344,344.8376509)(803.63761719,344.84765076)
\curveto(803.58761353,344.85765088)(803.53261359,344.86765087)(803.47261719,344.87765076)
\curveto(803.41261371,344.87765086)(803.35761376,344.88265086)(803.30761719,344.89265076)
\curveto(803.2176139,344.91265083)(803.12761399,344.9376508)(803.03761719,344.96765076)
\curveto(802.94761417,344.98765075)(802.86261426,345.01265073)(802.78261719,345.04265076)
\curveto(802.74261438,345.06265068)(802.70761441,345.07265067)(802.67761719,345.07265076)
\curveto(802.64761447,345.08265066)(802.61261451,345.09765064)(802.57261719,345.11765076)
\curveto(802.4226147,345.18765055)(802.26261486,345.27265047)(802.09261719,345.37265076)
\curveto(801.80261532,345.56265018)(801.55261557,345.79264995)(801.34261719,346.06265076)
\curveto(801.14261598,346.3426494)(800.97261615,346.65264909)(800.83261719,346.99265076)
\curveto(800.78261634,347.10264864)(800.74261638,347.21764852)(800.71261719,347.33765076)
\curveto(800.69261643,347.45764828)(800.66261646,347.57764816)(800.62261719,347.69765076)
\curveto(800.61261651,347.737648)(800.60761651,347.77264797)(800.60761719,347.80265076)
\curveto(800.60761651,347.83264791)(800.60261652,347.87264787)(800.59261719,347.92265076)
\curveto(800.57261655,348.00264774)(800.55761656,348.08764765)(800.54761719,348.17765076)
\curveto(800.53761658,348.26764747)(800.5226166,348.35764738)(800.50261719,348.44765076)
\lineto(800.50261719,348.65765076)
\curveto(800.49261663,348.69764704)(800.48261664,348.75264699)(800.47261719,348.82265076)
\curveto(800.47261665,348.90264684)(800.47761664,348.96764677)(800.48761719,349.01765076)
\lineto(800.48761719,349.18265076)
\curveto(800.50761661,349.23264651)(800.51261661,349.28264646)(800.50261719,349.33265076)
\curveto(800.50261662,349.39264635)(800.50761661,349.44764629)(800.51761719,349.49765076)
\curveto(800.55761656,349.65764608)(800.58761653,349.81764592)(800.60761719,349.97765076)
\curveto(800.63761648,350.1376456)(800.68261644,350.28764545)(800.74261719,350.42765076)
\curveto(800.79261633,350.5376452)(800.83761628,350.64764509)(800.87761719,350.75765076)
\curveto(800.92761619,350.87764486)(800.98261614,350.99264475)(801.04261719,351.10265076)
\curveto(801.26261586,351.45264429)(801.51261561,351.75264399)(801.79261719,352.00265076)
\curveto(802.07261505,352.26264348)(802.4176147,352.47764326)(802.82761719,352.64765076)
\curveto(802.94761417,352.69764304)(803.06761405,352.73264301)(803.18761719,352.75265076)
\curveto(803.3176138,352.78264296)(803.45261367,352.81264293)(803.59261719,352.84265076)
\curveto(803.64261348,352.85264289)(803.68761343,352.85764288)(803.72761719,352.85765076)
\curveto(803.76761335,352.86764287)(803.81261331,352.87264287)(803.86261719,352.87265076)
\curveto(803.88261324,352.88264286)(803.90761321,352.88264286)(803.93761719,352.87265076)
\curveto(803.96761315,352.86264288)(803.99261313,352.86764287)(804.01261719,352.88765076)
\curveto(804.43261269,352.89764284)(804.79761232,352.85264289)(805.10761719,352.75265076)
\curveto(805.4176117,352.66264308)(805.69761142,352.5376432)(805.94761719,352.37765076)
\curveto(805.99761112,352.35764338)(806.03761108,352.32764341)(806.06761719,352.28765076)
\curveto(806.09761102,352.25764348)(806.13261099,352.23264351)(806.17261719,352.21265076)
\curveto(806.25261087,352.15264359)(806.33261079,352.08264366)(806.41261719,352.00265076)
\curveto(806.50261062,351.92264382)(806.57761054,351.8426439)(806.63761719,351.76265076)
\curveto(806.79761032,351.55264419)(806.93261019,351.35264439)(807.04261719,351.16265076)
\curveto(807.11261001,351.05264469)(807.16760995,350.93264481)(807.20761719,350.80265076)
\curveto(807.24760987,350.67264507)(807.29260983,350.5426452)(807.34261719,350.41265076)
\curveto(807.39260973,350.28264546)(807.42760969,350.14764559)(807.44761719,350.00765076)
\curveto(807.47760964,349.86764587)(807.51260961,349.72764601)(807.55261719,349.58765076)
\curveto(807.56260956,349.51764622)(807.56760955,349.44764629)(807.56761719,349.37765076)
\lineto(807.59761719,349.16765076)
\moveto(806.14261719,349.67765076)
\curveto(806.17261095,349.71764602)(806.19761092,349.76764597)(806.21761719,349.82765076)
\curveto(806.23761088,349.89764584)(806.23761088,349.96764577)(806.21761719,350.03765076)
\curveto(806.15761096,350.25764548)(806.07261105,350.46264528)(805.96261719,350.65265076)
\curveto(805.8226113,350.88264486)(805.66761145,351.07764466)(805.49761719,351.23765076)
\curveto(805.32761179,351.39764434)(805.10761201,351.53264421)(804.83761719,351.64265076)
\curveto(804.76761235,351.66264408)(804.69761242,351.67764406)(804.62761719,351.68765076)
\curveto(804.55761256,351.70764403)(804.48261264,351.72764401)(804.40261719,351.74765076)
\curveto(804.3226128,351.76764397)(804.23761288,351.77764396)(804.14761719,351.77765076)
\lineto(803.89261719,351.77765076)
\curveto(803.86261326,351.75764398)(803.82761329,351.74764399)(803.78761719,351.74765076)
\curveto(803.74761337,351.75764398)(803.71261341,351.75764398)(803.68261719,351.74765076)
\lineto(803.44261719,351.68765076)
\curveto(803.37261375,351.67764406)(803.30261382,351.66264408)(803.23261719,351.64265076)
\curveto(802.94261418,351.52264422)(802.70761441,351.37264437)(802.52761719,351.19265076)
\curveto(802.35761476,351.01264473)(802.20261492,350.78764495)(802.06261719,350.51765076)
\curveto(802.03261509,350.46764527)(802.00261512,350.40264534)(801.97261719,350.32265076)
\curveto(801.94261518,350.25264549)(801.9176152,350.17264557)(801.89761719,350.08265076)
\curveto(801.87761524,349.99264575)(801.87261525,349.90764583)(801.88261719,349.82765076)
\curveto(801.89261523,349.74764599)(801.92761519,349.68764605)(801.98761719,349.64765076)
\curveto(802.06761505,349.58764615)(802.20261492,349.55764618)(802.39261719,349.55765076)
\curveto(802.59261453,349.56764617)(802.76261436,349.57264617)(802.90261719,349.57265076)
\lineto(805.18261719,349.57265076)
\curveto(805.33261179,349.57264617)(805.51261161,349.56764617)(805.72261719,349.55765076)
\curveto(805.93261119,349.55764618)(806.07261105,349.59764614)(806.14261719,349.67765076)
}
}
{
\newrgbcolor{curcolor}{0 0 0}
\pscustom[linestyle=none,fillstyle=solid,fillcolor=curcolor]
{
\newpath
\moveto(815.78925781,345.55265076)
\curveto(815.81924998,345.39265035)(815.80425,345.25765048)(815.74425781,345.14765076)
\curveto(815.68425012,345.04765069)(815.6042502,344.97265077)(815.50425781,344.92265076)
\curveto(815.45425035,344.90265084)(815.3992504,344.89265085)(815.33925781,344.89265076)
\curveto(815.28925051,344.89265085)(815.23425057,344.88265086)(815.17425781,344.86265076)
\curveto(814.95425085,344.81265093)(814.73425107,344.82765091)(814.51425781,344.90765076)
\curveto(814.3042515,344.97765076)(814.15925164,345.06765067)(814.07925781,345.17765076)
\curveto(814.02925177,345.24765049)(813.98425182,345.32765041)(813.94425781,345.41765076)
\curveto(813.9042519,345.51765022)(813.85425195,345.59765014)(813.79425781,345.65765076)
\curveto(813.77425203,345.67765006)(813.74925205,345.69765004)(813.71925781,345.71765076)
\curveto(813.6992521,345.73765)(813.66925213,345.74265)(813.62925781,345.73265076)
\curveto(813.51925228,345.70265004)(813.41425239,345.64765009)(813.31425781,345.56765076)
\curveto(813.22425258,345.48765025)(813.13425267,345.41765032)(813.04425781,345.35765076)
\curveto(812.91425289,345.27765046)(812.77425303,345.20265054)(812.62425781,345.13265076)
\curveto(812.47425333,345.07265067)(812.31425349,345.01765072)(812.14425781,344.96765076)
\curveto(812.04425376,344.9376508)(811.93425387,344.91765082)(811.81425781,344.90765076)
\curveto(811.7042541,344.89765084)(811.59425421,344.88265086)(811.48425781,344.86265076)
\curveto(811.43425437,344.85265089)(811.38925441,344.84765089)(811.34925781,344.84765076)
\lineto(811.24425781,344.84765076)
\curveto(811.13425467,344.82765091)(811.02925477,344.82765091)(810.92925781,344.84765076)
\lineto(810.79425781,344.84765076)
\curveto(810.74425506,344.85765088)(810.69425511,344.86265088)(810.64425781,344.86265076)
\curveto(810.59425521,344.86265088)(810.54925525,344.87265087)(810.50925781,344.89265076)
\curveto(810.46925533,344.90265084)(810.43425537,344.90765083)(810.40425781,344.90765076)
\curveto(810.38425542,344.89765084)(810.35925544,344.89765084)(810.32925781,344.90765076)
\lineto(810.08925781,344.96765076)
\curveto(810.00925579,344.97765076)(809.93425587,344.99765074)(809.86425781,345.02765076)
\curveto(809.56425624,345.15765058)(809.31925648,345.30265044)(809.12925781,345.46265076)
\curveto(808.94925685,345.63265011)(808.799257,345.86764987)(808.67925781,346.16765076)
\curveto(808.58925721,346.38764935)(808.54425726,346.65264909)(808.54425781,346.96265076)
\lineto(808.54425781,347.27765076)
\curveto(808.55425725,347.32764841)(808.55925724,347.37764836)(808.55925781,347.42765076)
\lineto(808.58925781,347.60765076)
\lineto(808.70925781,347.93765076)
\curveto(808.74925705,348.04764769)(808.799257,348.14764759)(808.85925781,348.23765076)
\curveto(809.03925676,348.52764721)(809.28425652,348.742647)(809.59425781,348.88265076)
\curveto(809.9042559,349.02264672)(810.24425556,349.14764659)(810.61425781,349.25765076)
\curveto(810.75425505,349.29764644)(810.8992549,349.32764641)(811.04925781,349.34765076)
\curveto(811.1992546,349.36764637)(811.34925445,349.39264635)(811.49925781,349.42265076)
\curveto(811.56925423,349.4426463)(811.63425417,349.45264629)(811.69425781,349.45265076)
\curveto(811.76425404,349.45264629)(811.83925396,349.46264628)(811.91925781,349.48265076)
\curveto(811.98925381,349.50264624)(812.05925374,349.51264623)(812.12925781,349.51265076)
\curveto(812.1992536,349.52264622)(812.27425353,349.5376462)(812.35425781,349.55765076)
\curveto(812.6042532,349.61764612)(812.83925296,349.66764607)(813.05925781,349.70765076)
\curveto(813.27925252,349.75764598)(813.45425235,349.87264587)(813.58425781,350.05265076)
\curveto(813.64425216,350.13264561)(813.69425211,350.23264551)(813.73425781,350.35265076)
\curveto(813.77425203,350.48264526)(813.77425203,350.62264512)(813.73425781,350.77265076)
\curveto(813.67425213,351.01264473)(813.58425222,351.20264454)(813.46425781,351.34265076)
\curveto(813.35425245,351.48264426)(813.19425261,351.59264415)(812.98425781,351.67265076)
\curveto(812.86425294,351.72264402)(812.71925308,351.75764398)(812.54925781,351.77765076)
\curveto(812.38925341,351.79764394)(812.21925358,351.80764393)(812.03925781,351.80765076)
\curveto(811.85925394,351.80764393)(811.68425412,351.79764394)(811.51425781,351.77765076)
\curveto(811.34425446,351.75764398)(811.1992546,351.72764401)(811.07925781,351.68765076)
\curveto(810.90925489,351.62764411)(810.74425506,351.5426442)(810.58425781,351.43265076)
\curveto(810.5042553,351.37264437)(810.42925537,351.29264445)(810.35925781,351.19265076)
\curveto(810.2992555,351.10264464)(810.24425556,351.00264474)(810.19425781,350.89265076)
\curveto(810.16425564,350.81264493)(810.13425567,350.72764501)(810.10425781,350.63765076)
\curveto(810.08425572,350.54764519)(810.03925576,350.47764526)(809.96925781,350.42765076)
\curveto(809.92925587,350.39764534)(809.85925594,350.37264537)(809.75925781,350.35265076)
\curveto(809.66925613,350.3426454)(809.57425623,350.3376454)(809.47425781,350.33765076)
\curveto(809.37425643,350.3376454)(809.27425653,350.3426454)(809.17425781,350.35265076)
\curveto(809.08425672,350.37264537)(809.01925678,350.39764534)(808.97925781,350.42765076)
\curveto(808.93925686,350.45764528)(808.90925689,350.50764523)(808.88925781,350.57765076)
\curveto(808.86925693,350.64764509)(808.86925693,350.72264502)(808.88925781,350.80265076)
\curveto(808.91925688,350.93264481)(808.94925685,351.05264469)(808.97925781,351.16265076)
\curveto(809.01925678,351.28264446)(809.06425674,351.39764434)(809.11425781,351.50765076)
\curveto(809.3042565,351.85764388)(809.54425626,352.12764361)(809.83425781,352.31765076)
\curveto(810.12425568,352.51764322)(810.48425532,352.67764306)(810.91425781,352.79765076)
\curveto(811.01425479,352.81764292)(811.11425469,352.83264291)(811.21425781,352.84265076)
\curveto(811.32425448,352.85264289)(811.43425437,352.86764287)(811.54425781,352.88765076)
\curveto(811.58425422,352.89764284)(811.64925415,352.89764284)(811.73925781,352.88765076)
\curveto(811.82925397,352.88764285)(811.88425392,352.89764284)(811.90425781,352.91765076)
\curveto(812.6042532,352.92764281)(813.21425259,352.84764289)(813.73425781,352.67765076)
\curveto(814.25425155,352.50764323)(814.61925118,352.18264356)(814.82925781,351.70265076)
\curveto(814.91925088,351.50264424)(814.96925083,351.26764447)(814.97925781,350.99765076)
\curveto(814.9992508,350.737645)(815.00925079,350.46264528)(815.00925781,350.17265076)
\lineto(815.00925781,346.85765076)
\curveto(815.00925079,346.71764902)(815.01425079,346.58264916)(815.02425781,346.45265076)
\curveto(815.03425077,346.32264942)(815.06425074,346.21764952)(815.11425781,346.13765076)
\curveto(815.16425064,346.06764967)(815.22925057,346.01764972)(815.30925781,345.98765076)
\curveto(815.3992504,345.94764979)(815.48425032,345.91764982)(815.56425781,345.89765076)
\curveto(815.64425016,345.88764985)(815.7042501,345.8426499)(815.74425781,345.76265076)
\curveto(815.76425004,345.73265001)(815.77425003,345.70265004)(815.77425781,345.67265076)
\curveto(815.77425003,345.6426501)(815.77925002,345.60265014)(815.78925781,345.55265076)
\moveto(813.64425781,347.21765076)
\curveto(813.7042521,347.35764838)(813.73425207,347.51764822)(813.73425781,347.69765076)
\curveto(813.74425206,347.88764785)(813.74925205,348.08264766)(813.74925781,348.28265076)
\curveto(813.74925205,348.39264735)(813.74425206,348.49264725)(813.73425781,348.58265076)
\curveto(813.72425208,348.67264707)(813.68425212,348.742647)(813.61425781,348.79265076)
\curveto(813.58425222,348.81264693)(813.51425229,348.82264692)(813.40425781,348.82265076)
\curveto(813.38425242,348.80264694)(813.34925245,348.79264695)(813.29925781,348.79265076)
\curveto(813.24925255,348.79264695)(813.2042526,348.78264696)(813.16425781,348.76265076)
\curveto(813.08425272,348.742647)(812.99425281,348.72264702)(812.89425781,348.70265076)
\lineto(812.59425781,348.64265076)
\curveto(812.56425324,348.6426471)(812.52925327,348.6376471)(812.48925781,348.62765076)
\lineto(812.38425781,348.62765076)
\curveto(812.23425357,348.58764715)(812.06925373,348.56264718)(811.88925781,348.55265076)
\curveto(811.71925408,348.55264719)(811.55925424,348.53264721)(811.40925781,348.49265076)
\curveto(811.32925447,348.47264727)(811.25425455,348.45264729)(811.18425781,348.43265076)
\curveto(811.12425468,348.42264732)(811.05425475,348.40764733)(810.97425781,348.38765076)
\curveto(810.81425499,348.3376474)(810.66425514,348.27264747)(810.52425781,348.19265076)
\curveto(810.38425542,348.12264762)(810.26425554,348.03264771)(810.16425781,347.92265076)
\curveto(810.06425574,347.81264793)(809.98925581,347.67764806)(809.93925781,347.51765076)
\curveto(809.88925591,347.36764837)(809.86925593,347.18264856)(809.87925781,346.96265076)
\curveto(809.87925592,346.86264888)(809.89425591,346.76764897)(809.92425781,346.67765076)
\curveto(809.96425584,346.59764914)(810.00925579,346.52264922)(810.05925781,346.45265076)
\curveto(810.13925566,346.3426494)(810.24425556,346.24764949)(810.37425781,346.16765076)
\curveto(810.5042553,346.09764964)(810.64425516,346.0376497)(810.79425781,345.98765076)
\curveto(810.84425496,345.97764976)(810.89425491,345.97264977)(810.94425781,345.97265076)
\curveto(810.99425481,345.97264977)(811.04425476,345.96764977)(811.09425781,345.95765076)
\curveto(811.16425464,345.9376498)(811.24925455,345.92264982)(811.34925781,345.91265076)
\curveto(811.45925434,345.91264983)(811.54925425,345.92264982)(811.61925781,345.94265076)
\curveto(811.67925412,345.96264978)(811.73925406,345.96764977)(811.79925781,345.95765076)
\curveto(811.85925394,345.95764978)(811.91925388,345.96764977)(811.97925781,345.98765076)
\curveto(812.05925374,346.00764973)(812.13425367,346.02264972)(812.20425781,346.03265076)
\curveto(812.28425352,346.0426497)(812.35925344,346.06264968)(812.42925781,346.09265076)
\curveto(812.71925308,346.21264953)(812.96425284,346.35764938)(813.16425781,346.52765076)
\curveto(813.37425243,346.69764904)(813.53425227,346.92764881)(813.64425781,347.21765076)
}
}
{
\newrgbcolor{curcolor}{0 0 0}
\pscustom[linestyle=none,fillstyle=solid,fillcolor=curcolor]
{
\newpath
\moveto(823.92089844,345.80765076)
\lineto(823.92089844,345.41765076)
\curveto(823.92089056,345.29765044)(823.89589059,345.19765054)(823.84589844,345.11765076)
\curveto(823.79589069,345.04765069)(823.71089077,345.00765073)(823.59089844,344.99765076)
\lineto(823.24589844,344.99765076)
\curveto(823.1858913,344.99765074)(823.12589136,344.99265075)(823.06589844,344.98265076)
\curveto(823.01589147,344.98265076)(822.97089151,344.99265075)(822.93089844,345.01265076)
\curveto(822.84089164,345.03265071)(822.7808917,345.07265067)(822.75089844,345.13265076)
\curveto(822.71089177,345.18265056)(822.6858918,345.2426505)(822.67589844,345.31265076)
\curveto(822.67589181,345.38265036)(822.66089182,345.45265029)(822.63089844,345.52265076)
\curveto(822.62089186,345.5426502)(822.60589188,345.55765018)(822.58589844,345.56765076)
\curveto(822.57589191,345.58765015)(822.56089192,345.60765013)(822.54089844,345.62765076)
\curveto(822.44089204,345.6376501)(822.36089212,345.61765012)(822.30089844,345.56765076)
\curveto(822.25089223,345.51765022)(822.19589229,345.46765027)(822.13589844,345.41765076)
\curveto(821.93589255,345.26765047)(821.73589275,345.15265059)(821.53589844,345.07265076)
\curveto(821.35589313,344.99265075)(821.14589334,344.93265081)(820.90589844,344.89265076)
\curveto(820.67589381,344.85265089)(820.43589405,344.83265091)(820.18589844,344.83265076)
\curveto(819.94589454,344.82265092)(819.70589478,344.8376509)(819.46589844,344.87765076)
\curveto(819.22589526,344.90765083)(819.01589547,344.96265078)(818.83589844,345.04265076)
\curveto(818.31589617,345.26265048)(817.89589659,345.55765018)(817.57589844,345.92765076)
\curveto(817.25589723,346.30764943)(817.00589748,346.77764896)(816.82589844,347.33765076)
\curveto(816.7858977,347.42764831)(816.75589773,347.51764822)(816.73589844,347.60765076)
\curveto(816.72589776,347.70764803)(816.70589778,347.80764793)(816.67589844,347.90765076)
\curveto(816.66589782,347.95764778)(816.66089782,348.00764773)(816.66089844,348.05765076)
\curveto(816.66089782,348.10764763)(816.65589783,348.15764758)(816.64589844,348.20765076)
\curveto(816.62589786,348.25764748)(816.61589787,348.30764743)(816.61589844,348.35765076)
\curveto(816.62589786,348.41764732)(816.62589786,348.47264727)(816.61589844,348.52265076)
\lineto(816.61589844,348.67265076)
\curveto(816.59589789,348.72264702)(816.5858979,348.78764695)(816.58589844,348.86765076)
\curveto(816.5858979,348.94764679)(816.59589789,349.01264673)(816.61589844,349.06265076)
\lineto(816.61589844,349.22765076)
\curveto(816.63589785,349.29764644)(816.64089784,349.36764637)(816.63089844,349.43765076)
\curveto(816.63089785,349.51764622)(816.64089784,349.59264615)(816.66089844,349.66265076)
\curveto(816.67089781,349.71264603)(816.67589781,349.75764598)(816.67589844,349.79765076)
\curveto(816.67589781,349.8376459)(816.6808978,349.88264586)(816.69089844,349.93265076)
\curveto(816.72089776,350.03264571)(816.74589774,350.12764561)(816.76589844,350.21765076)
\curveto(816.7858977,350.31764542)(816.81089767,350.41264533)(816.84089844,350.50265076)
\curveto(816.97089751,350.88264486)(817.13589735,351.22264452)(817.33589844,351.52265076)
\curveto(817.54589694,351.83264391)(817.79589669,352.08764365)(818.08589844,352.28765076)
\curveto(818.25589623,352.40764333)(818.43089605,352.50764323)(818.61089844,352.58765076)
\curveto(818.80089568,352.66764307)(819.00589548,352.737643)(819.22589844,352.79765076)
\curveto(819.29589519,352.80764293)(819.36089512,352.81764292)(819.42089844,352.82765076)
\curveto(819.49089499,352.8376429)(819.56089492,352.85264289)(819.63089844,352.87265076)
\lineto(819.78089844,352.87265076)
\curveto(819.86089462,352.89264285)(819.97589451,352.90264284)(820.12589844,352.90265076)
\curveto(820.2858942,352.90264284)(820.40589408,352.89264285)(820.48589844,352.87265076)
\curveto(820.52589396,352.86264288)(820.5808939,352.85764288)(820.65089844,352.85765076)
\curveto(820.76089372,352.82764291)(820.87089361,352.80264294)(820.98089844,352.78265076)
\curveto(821.09089339,352.77264297)(821.19589329,352.742643)(821.29589844,352.69265076)
\curveto(821.44589304,352.63264311)(821.5858929,352.56764317)(821.71589844,352.49765076)
\curveto(821.85589263,352.42764331)(821.9858925,352.34764339)(822.10589844,352.25765076)
\curveto(822.16589232,352.20764353)(822.22589226,352.15264359)(822.28589844,352.09265076)
\curveto(822.35589213,352.0426437)(822.44589204,352.02764371)(822.55589844,352.04765076)
\curveto(822.57589191,352.07764366)(822.59089189,352.10264364)(822.60089844,352.12265076)
\curveto(822.62089186,352.1426436)(822.63589185,352.17264357)(822.64589844,352.21265076)
\curveto(822.67589181,352.30264344)(822.6858918,352.41764332)(822.67589844,352.55765076)
\lineto(822.67589844,352.93265076)
\lineto(822.67589844,354.65765076)
\lineto(822.67589844,355.12265076)
\curveto(822.67589181,355.30264044)(822.70089178,355.43264031)(822.75089844,355.51265076)
\curveto(822.79089169,355.58264016)(822.85089163,355.62764011)(822.93089844,355.64765076)
\curveto(822.95089153,355.64764009)(822.97589151,355.64764009)(823.00589844,355.64765076)
\curveto(823.03589145,355.65764008)(823.06089142,355.66264008)(823.08089844,355.66265076)
\curveto(823.22089126,355.67264007)(823.36589112,355.67264007)(823.51589844,355.66265076)
\curveto(823.67589081,355.66264008)(823.7858907,355.62264012)(823.84589844,355.54265076)
\curveto(823.89589059,355.46264028)(823.92089056,355.36264038)(823.92089844,355.24265076)
\lineto(823.92089844,354.86765076)
\lineto(823.92089844,345.80765076)
\moveto(822.70589844,348.64265076)
\curveto(822.72589176,348.69264705)(822.73589175,348.75764698)(822.73589844,348.83765076)
\curveto(822.73589175,348.92764681)(822.72589176,348.99764674)(822.70589844,349.04765076)
\lineto(822.70589844,349.27265076)
\curveto(822.6858918,349.36264638)(822.67089181,349.45264629)(822.66089844,349.54265076)
\curveto(822.65089183,349.6426461)(822.63089185,349.73264601)(822.60089844,349.81265076)
\curveto(822.5808919,349.89264585)(822.56089192,349.96764577)(822.54089844,350.03765076)
\curveto(822.53089195,350.10764563)(822.51089197,350.17764556)(822.48089844,350.24765076)
\curveto(822.36089212,350.54764519)(822.20589228,350.81264493)(822.01589844,351.04265076)
\curveto(821.82589266,351.27264447)(821.5858929,351.45264429)(821.29589844,351.58265076)
\curveto(821.19589329,351.63264411)(821.09089339,351.66764407)(820.98089844,351.68765076)
\curveto(820.8808936,351.71764402)(820.77089371,351.742644)(820.65089844,351.76265076)
\curveto(820.57089391,351.78264396)(820.480894,351.79264395)(820.38089844,351.79265076)
\lineto(820.11089844,351.79265076)
\curveto(820.06089442,351.78264396)(820.01589447,351.77264397)(819.97589844,351.76265076)
\lineto(819.84089844,351.76265076)
\curveto(819.76089472,351.742644)(819.67589481,351.72264402)(819.58589844,351.70265076)
\curveto(819.50589498,351.68264406)(819.42589506,351.65764408)(819.34589844,351.62765076)
\curveto(819.02589546,351.48764425)(818.76589572,351.28264446)(818.56589844,351.01265076)
\curveto(818.37589611,350.75264499)(818.22089626,350.44764529)(818.10089844,350.09765076)
\curveto(818.06089642,349.98764575)(818.03089645,349.87264587)(818.01089844,349.75265076)
\curveto(818.00089648,349.6426461)(817.9858965,349.53264621)(817.96589844,349.42265076)
\curveto(817.96589652,349.38264636)(817.96089652,349.3426464)(817.95089844,349.30265076)
\lineto(817.95089844,349.19765076)
\curveto(817.93089655,349.14764659)(817.92089656,349.09264665)(817.92089844,349.03265076)
\curveto(817.93089655,348.97264677)(817.93589655,348.91764682)(817.93589844,348.86765076)
\lineto(817.93589844,348.53765076)
\curveto(817.93589655,348.4376473)(817.94589654,348.3426474)(817.96589844,348.25265076)
\curveto(817.97589651,348.22264752)(817.9808965,348.17264757)(817.98089844,348.10265076)
\curveto(818.00089648,348.03264771)(818.01589647,347.96264778)(818.02589844,347.89265076)
\lineto(818.08589844,347.68265076)
\curveto(818.19589629,347.33264841)(818.34589614,347.03264871)(818.53589844,346.78265076)
\curveto(818.72589576,346.53264921)(818.96589552,346.32764941)(819.25589844,346.16765076)
\curveto(819.34589514,346.11764962)(819.43589505,346.07764966)(819.52589844,346.04765076)
\curveto(819.61589487,346.01764972)(819.71589477,345.98764975)(819.82589844,345.95765076)
\curveto(819.87589461,345.9376498)(819.92589456,345.93264981)(819.97589844,345.94265076)
\curveto(820.03589445,345.95264979)(820.09089439,345.94764979)(820.14089844,345.92765076)
\curveto(820.1808943,345.91764982)(820.22089426,345.91264983)(820.26089844,345.91265076)
\lineto(820.39589844,345.91265076)
\lineto(820.53089844,345.91265076)
\curveto(820.56089392,345.92264982)(820.61089387,345.92764981)(820.68089844,345.92765076)
\curveto(820.76089372,345.94764979)(820.84089364,345.96264978)(820.92089844,345.97265076)
\curveto(821.00089348,345.99264975)(821.07589341,346.01764972)(821.14589844,346.04765076)
\curveto(821.47589301,346.18764955)(821.74089274,346.36264938)(821.94089844,346.57265076)
\curveto(822.15089233,346.79264895)(822.32589216,347.06764867)(822.46589844,347.39765076)
\curveto(822.51589197,347.50764823)(822.55089193,347.61764812)(822.57089844,347.72765076)
\curveto(822.59089189,347.8376479)(822.61589187,347.94764779)(822.64589844,348.05765076)
\curveto(822.66589182,348.09764764)(822.67589181,348.13264761)(822.67589844,348.16265076)
\curveto(822.67589181,348.20264754)(822.6808918,348.2426475)(822.69089844,348.28265076)
\curveto(822.70089178,348.3426474)(822.70089178,348.40264734)(822.69089844,348.46265076)
\curveto(822.69089179,348.52264722)(822.69589179,348.58264716)(822.70589844,348.64265076)
}
}
{
\newrgbcolor{curcolor}{0 0 0}
\pscustom[linestyle=none,fillstyle=solid,fillcolor=curcolor]
{
\newpath
\moveto(832.99214844,349.19765076)
\curveto(833.01214038,349.1376466)(833.02214037,349.0426467)(833.02214844,348.91265076)
\curveto(833.02214037,348.79264695)(833.01714037,348.70764703)(833.00714844,348.65765076)
\lineto(833.00714844,348.50765076)
\curveto(832.99714039,348.42764731)(832.9871404,348.35264739)(832.97714844,348.28265076)
\curveto(832.97714041,348.22264752)(832.97214042,348.15264759)(832.96214844,348.07265076)
\curveto(832.94214045,348.01264773)(832.92714046,347.95264779)(832.91714844,347.89265076)
\curveto(832.91714047,347.83264791)(832.90714048,347.77264797)(832.88714844,347.71265076)
\curveto(832.84714054,347.58264816)(832.81214058,347.45264829)(832.78214844,347.32265076)
\curveto(832.75214064,347.19264855)(832.71214068,347.07264867)(832.66214844,346.96265076)
\curveto(832.45214094,346.48264926)(832.17214122,346.07764966)(831.82214844,345.74765076)
\curveto(831.47214192,345.42765031)(831.04214235,345.18265056)(830.53214844,345.01265076)
\curveto(830.42214297,344.97265077)(830.30214309,344.9426508)(830.17214844,344.92265076)
\curveto(830.05214334,344.90265084)(829.92714346,344.88265086)(829.79714844,344.86265076)
\curveto(829.73714365,344.85265089)(829.67214372,344.84765089)(829.60214844,344.84765076)
\curveto(829.54214385,344.8376509)(829.48214391,344.83265091)(829.42214844,344.83265076)
\curveto(829.38214401,344.82265092)(829.32214407,344.81765092)(829.24214844,344.81765076)
\curveto(829.17214422,344.81765092)(829.12214427,344.82265092)(829.09214844,344.83265076)
\curveto(829.05214434,344.8426509)(829.01214438,344.84765089)(828.97214844,344.84765076)
\curveto(828.93214446,344.8376509)(828.89714449,344.8376509)(828.86714844,344.84765076)
\lineto(828.77714844,344.84765076)
\lineto(828.41714844,344.89265076)
\curveto(828.27714511,344.93265081)(828.14214525,344.97265077)(828.01214844,345.01265076)
\curveto(827.88214551,345.05265069)(827.75714563,345.09765064)(827.63714844,345.14765076)
\curveto(827.1871462,345.34765039)(826.81714657,345.60765013)(826.52714844,345.92765076)
\curveto(826.23714715,346.24764949)(825.99714739,346.6376491)(825.80714844,347.09765076)
\curveto(825.75714763,347.19764854)(825.71714767,347.29764844)(825.68714844,347.39765076)
\curveto(825.66714772,347.49764824)(825.64714774,347.60264814)(825.62714844,347.71265076)
\curveto(825.60714778,347.75264799)(825.59714779,347.78264796)(825.59714844,347.80265076)
\curveto(825.60714778,347.83264791)(825.60714778,347.86764787)(825.59714844,347.90765076)
\curveto(825.57714781,347.98764775)(825.56214783,348.06764767)(825.55214844,348.14765076)
\curveto(825.55214784,348.2376475)(825.54214785,348.32264742)(825.52214844,348.40265076)
\lineto(825.52214844,348.52265076)
\curveto(825.52214787,348.56264718)(825.51714787,348.60764713)(825.50714844,348.65765076)
\curveto(825.49714789,348.70764703)(825.4921479,348.79264695)(825.49214844,348.91265076)
\curveto(825.4921479,349.0426467)(825.50214789,349.1376466)(825.52214844,349.19765076)
\curveto(825.54214785,349.26764647)(825.54714784,349.3376464)(825.53714844,349.40765076)
\curveto(825.52714786,349.47764626)(825.53214786,349.54764619)(825.55214844,349.61765076)
\curveto(825.56214783,349.66764607)(825.56714782,349.70764603)(825.56714844,349.73765076)
\curveto(825.57714781,349.77764596)(825.5871478,349.82264592)(825.59714844,349.87265076)
\curveto(825.62714776,349.99264575)(825.65214774,350.11264563)(825.67214844,350.23265076)
\curveto(825.70214769,350.35264539)(825.74214765,350.46764527)(825.79214844,350.57765076)
\curveto(825.94214745,350.94764479)(826.12214727,351.27764446)(826.33214844,351.56765076)
\curveto(826.55214684,351.86764387)(826.81714657,352.11764362)(827.12714844,352.31765076)
\curveto(827.24714614,352.39764334)(827.37214602,352.46264328)(827.50214844,352.51265076)
\curveto(827.63214576,352.57264317)(827.76714562,352.63264311)(827.90714844,352.69265076)
\curveto(828.02714536,352.742643)(828.15714523,352.77264297)(828.29714844,352.78265076)
\curveto(828.43714495,352.80264294)(828.57714481,352.83264291)(828.71714844,352.87265076)
\lineto(828.91214844,352.87265076)
\curveto(828.98214441,352.88264286)(829.04714434,352.89264285)(829.10714844,352.90265076)
\curveto(829.99714339,352.91264283)(830.73714265,352.72764301)(831.32714844,352.34765076)
\curveto(831.91714147,351.96764377)(832.34214105,351.47264427)(832.60214844,350.86265076)
\curveto(832.65214074,350.76264498)(832.6921407,350.66264508)(832.72214844,350.56265076)
\curveto(832.75214064,350.46264528)(832.7871406,350.35764538)(832.82714844,350.24765076)
\curveto(832.85714053,350.1376456)(832.88214051,350.01764572)(832.90214844,349.88765076)
\curveto(832.92214047,349.76764597)(832.94714044,349.6426461)(832.97714844,349.51265076)
\curveto(832.9871404,349.46264628)(832.9871404,349.40764633)(832.97714844,349.34765076)
\curveto(832.97714041,349.29764644)(832.98214041,349.24764649)(832.99214844,349.19765076)
\moveto(831.65714844,348.34265076)
\curveto(831.67714171,348.41264733)(831.68214171,348.49264725)(831.67214844,348.58265076)
\lineto(831.67214844,348.83765076)
\curveto(831.67214172,349.22764651)(831.63714175,349.55764618)(831.56714844,349.82765076)
\curveto(831.53714185,349.90764583)(831.51214188,349.98764575)(831.49214844,350.06765076)
\curveto(831.47214192,350.14764559)(831.44714194,350.22264552)(831.41714844,350.29265076)
\curveto(831.13714225,350.9426448)(830.6921427,351.39264435)(830.08214844,351.64265076)
\curveto(830.01214338,351.67264407)(829.93714345,351.69264405)(829.85714844,351.70265076)
\lineto(829.61714844,351.76265076)
\curveto(829.53714385,351.78264396)(829.45214394,351.79264395)(829.36214844,351.79265076)
\lineto(829.09214844,351.79265076)
\lineto(828.82214844,351.74765076)
\curveto(828.72214467,351.72764401)(828.62714476,351.70264404)(828.53714844,351.67265076)
\curveto(828.45714493,351.65264409)(828.37714501,351.62264412)(828.29714844,351.58265076)
\curveto(828.22714516,351.56264418)(828.16214523,351.53264421)(828.10214844,351.49265076)
\curveto(828.04214535,351.45264429)(827.9871454,351.41264433)(827.93714844,351.37265076)
\curveto(827.69714569,351.20264454)(827.50214589,350.99764474)(827.35214844,350.75765076)
\curveto(827.20214619,350.51764522)(827.07214632,350.2376455)(826.96214844,349.91765076)
\curveto(826.93214646,349.81764592)(826.91214648,349.71264603)(826.90214844,349.60265076)
\curveto(826.8921465,349.50264624)(826.87714651,349.39764634)(826.85714844,349.28765076)
\curveto(826.84714654,349.24764649)(826.84214655,349.18264656)(826.84214844,349.09265076)
\curveto(826.83214656,349.06264668)(826.82714656,349.02764671)(826.82714844,348.98765076)
\curveto(826.83714655,348.94764679)(826.84214655,348.90264684)(826.84214844,348.85265076)
\lineto(826.84214844,348.55265076)
\curveto(826.84214655,348.45264729)(826.85214654,348.36264738)(826.87214844,348.28265076)
\lineto(826.90214844,348.10265076)
\curveto(826.92214647,348.00264774)(826.93714645,347.90264784)(826.94714844,347.80265076)
\curveto(826.96714642,347.71264803)(826.99714639,347.62764811)(827.03714844,347.54765076)
\curveto(827.13714625,347.30764843)(827.25214614,347.08264866)(827.38214844,346.87265076)
\curveto(827.52214587,346.66264908)(827.6921457,346.48764925)(827.89214844,346.34765076)
\curveto(827.94214545,346.31764942)(827.9871454,346.29264945)(828.02714844,346.27265076)
\curveto(828.06714532,346.25264949)(828.11214528,346.22764951)(828.16214844,346.19765076)
\curveto(828.24214515,346.14764959)(828.32714506,346.10264964)(828.41714844,346.06265076)
\curveto(828.51714487,346.03264971)(828.62214477,346.00264974)(828.73214844,345.97265076)
\curveto(828.78214461,345.95264979)(828.82714456,345.9426498)(828.86714844,345.94265076)
\curveto(828.91714447,345.95264979)(828.96714442,345.95264979)(829.01714844,345.94265076)
\curveto(829.04714434,345.93264981)(829.10714428,345.92264982)(829.19714844,345.91265076)
\curveto(829.29714409,345.90264984)(829.37214402,345.90764983)(829.42214844,345.92765076)
\curveto(829.46214393,345.9376498)(829.50214389,345.9376498)(829.54214844,345.92765076)
\curveto(829.58214381,345.92764981)(829.62214377,345.9376498)(829.66214844,345.95765076)
\curveto(829.74214365,345.97764976)(829.82214357,345.99264975)(829.90214844,346.00265076)
\curveto(829.98214341,346.02264972)(830.05714333,346.04764969)(830.12714844,346.07765076)
\curveto(830.46714292,346.21764952)(830.74214265,346.41264933)(830.95214844,346.66265076)
\curveto(831.16214223,346.91264883)(831.33714205,347.20764853)(831.47714844,347.54765076)
\curveto(831.52714186,347.66764807)(831.55714183,347.79264795)(831.56714844,347.92265076)
\curveto(831.5871418,348.06264768)(831.61714177,348.20264754)(831.65714844,348.34265076)
}
}
{
\newrgbcolor{curcolor}{0 0 0}
\pscustom[linestyle=none,fillstyle=solid,fillcolor=curcolor]
{
\newpath
\moveto(836.91042969,352.90265076)
\curveto(837.63042562,352.91264283)(838.23542502,352.82764291)(838.72542969,352.64765076)
\curveto(839.21542404,352.47764326)(839.59542366,352.17264357)(839.86542969,351.73265076)
\curveto(839.93542332,351.62264412)(839.99042326,351.50764423)(840.03042969,351.38765076)
\curveto(840.07042318,351.27764446)(840.11042314,351.15264459)(840.15042969,351.01265076)
\curveto(840.17042308,350.9426448)(840.17542308,350.86764487)(840.16542969,350.78765076)
\curveto(840.1554231,350.71764502)(840.14042311,350.66264508)(840.12042969,350.62265076)
\curveto(840.10042315,350.60264514)(840.07542318,350.58264516)(840.04542969,350.56265076)
\curveto(840.01542324,350.55264519)(839.99042326,350.5376452)(839.97042969,350.51765076)
\curveto(839.92042333,350.49764524)(839.87042338,350.49264525)(839.82042969,350.50265076)
\curveto(839.77042348,350.51264523)(839.72042353,350.51264523)(839.67042969,350.50265076)
\curveto(839.59042366,350.48264526)(839.48542377,350.47764526)(839.35542969,350.48765076)
\curveto(839.22542403,350.50764523)(839.13542412,350.53264521)(839.08542969,350.56265076)
\curveto(839.00542425,350.61264513)(838.9504243,350.67764506)(838.92042969,350.75765076)
\curveto(838.90042435,350.84764489)(838.86542439,350.93264481)(838.81542969,351.01265076)
\curveto(838.72542453,351.17264457)(838.60042465,351.31764442)(838.44042969,351.44765076)
\curveto(838.33042492,351.52764421)(838.21042504,351.58764415)(838.08042969,351.62765076)
\curveto(837.9504253,351.66764407)(837.81042544,351.70764403)(837.66042969,351.74765076)
\curveto(837.61042564,351.76764397)(837.56042569,351.77264397)(837.51042969,351.76265076)
\curveto(837.46042579,351.76264398)(837.41042584,351.76764397)(837.36042969,351.77765076)
\curveto(837.30042595,351.79764394)(837.22542603,351.80764393)(837.13542969,351.80765076)
\curveto(837.04542621,351.80764393)(836.97042628,351.79764394)(836.91042969,351.77765076)
\lineto(836.82042969,351.77765076)
\lineto(836.67042969,351.74765076)
\curveto(836.62042663,351.74764399)(836.57042668,351.742644)(836.52042969,351.73265076)
\curveto(836.26042699,351.67264407)(836.04542721,351.58764415)(835.87542969,351.47765076)
\curveto(835.70542755,351.36764437)(835.59042766,351.18264456)(835.53042969,350.92265076)
\curveto(835.51042774,350.85264489)(835.50542775,350.78264496)(835.51542969,350.71265076)
\curveto(835.53542772,350.6426451)(835.5554277,350.58264516)(835.57542969,350.53265076)
\curveto(835.63542762,350.38264536)(835.70542755,350.27264547)(835.78542969,350.20265076)
\curveto(835.87542738,350.1426456)(835.98542727,350.07264567)(836.11542969,349.99265076)
\curveto(836.27542698,349.89264585)(836.4554268,349.81764592)(836.65542969,349.76765076)
\curveto(836.8554264,349.72764601)(837.0554262,349.67764606)(837.25542969,349.61765076)
\curveto(837.38542587,349.57764616)(837.51542574,349.54764619)(837.64542969,349.52765076)
\curveto(837.77542548,349.50764623)(837.90542535,349.47764626)(838.03542969,349.43765076)
\curveto(838.24542501,349.37764636)(838.4504248,349.31764642)(838.65042969,349.25765076)
\curveto(838.8504244,349.20764653)(839.0504242,349.1426466)(839.25042969,349.06265076)
\lineto(839.40042969,349.00265076)
\curveto(839.4504238,348.98264676)(839.50042375,348.95764678)(839.55042969,348.92765076)
\curveto(839.7504235,348.80764693)(839.92542333,348.67264707)(840.07542969,348.52265076)
\curveto(840.22542303,348.37264737)(840.3504229,348.18264756)(840.45042969,347.95265076)
\curveto(840.47042278,347.88264786)(840.49042276,347.78764795)(840.51042969,347.66765076)
\curveto(840.53042272,347.59764814)(840.54042271,347.52264822)(840.54042969,347.44265076)
\curveto(840.5504227,347.37264837)(840.5554227,347.29264845)(840.55542969,347.20265076)
\lineto(840.55542969,347.05265076)
\curveto(840.53542272,346.98264876)(840.52542273,346.91264883)(840.52542969,346.84265076)
\curveto(840.52542273,346.77264897)(840.51542274,346.70264904)(840.49542969,346.63265076)
\curveto(840.46542279,346.52264922)(840.43042282,346.41764932)(840.39042969,346.31765076)
\curveto(840.3504229,346.21764952)(840.30542295,346.12764961)(840.25542969,346.04765076)
\curveto(840.09542316,345.78764995)(839.89042336,345.57765016)(839.64042969,345.41765076)
\curveto(839.39042386,345.26765047)(839.11042414,345.1376506)(838.80042969,345.02765076)
\curveto(838.71042454,344.99765074)(838.61542464,344.97765076)(838.51542969,344.96765076)
\curveto(838.42542483,344.94765079)(838.33542492,344.92265082)(838.24542969,344.89265076)
\curveto(838.14542511,344.87265087)(838.04542521,344.86265088)(837.94542969,344.86265076)
\curveto(837.84542541,344.86265088)(837.74542551,344.85265089)(837.64542969,344.83265076)
\lineto(837.49542969,344.83265076)
\curveto(837.44542581,344.82265092)(837.37542588,344.81765092)(837.28542969,344.81765076)
\curveto(837.19542606,344.81765092)(837.12542613,344.82265092)(837.07542969,344.83265076)
\lineto(836.91042969,344.83265076)
\curveto(836.8504264,344.85265089)(836.78542647,344.86265088)(836.71542969,344.86265076)
\curveto(836.64542661,344.85265089)(836.58542667,344.85765088)(836.53542969,344.87765076)
\curveto(836.48542677,344.88765085)(836.42042683,344.89265085)(836.34042969,344.89265076)
\lineto(836.10042969,344.95265076)
\curveto(836.03042722,344.96265078)(835.9554273,344.98265076)(835.87542969,345.01265076)
\curveto(835.56542769,345.11265063)(835.29542796,345.2376505)(835.06542969,345.38765076)
\curveto(834.83542842,345.5376502)(834.63542862,345.73265001)(834.46542969,345.97265076)
\curveto(834.37542888,346.10264964)(834.30042895,346.2376495)(834.24042969,346.37765076)
\curveto(834.18042907,346.51764922)(834.12542913,346.67264907)(834.07542969,346.84265076)
\curveto(834.0554292,346.90264884)(834.04542921,346.97264877)(834.04542969,347.05265076)
\curveto(834.0554292,347.1426486)(834.07042918,347.21264853)(834.09042969,347.26265076)
\curveto(834.12042913,347.30264844)(834.17042908,347.3426484)(834.24042969,347.38265076)
\curveto(834.29042896,347.40264834)(834.36042889,347.41264833)(834.45042969,347.41265076)
\curveto(834.54042871,347.42264832)(834.63042862,347.42264832)(834.72042969,347.41265076)
\curveto(834.81042844,347.40264834)(834.89542836,347.38764835)(834.97542969,347.36765076)
\curveto(835.06542819,347.35764838)(835.12542813,347.3426484)(835.15542969,347.32265076)
\curveto(835.22542803,347.27264847)(835.27042798,347.19764854)(835.29042969,347.09765076)
\curveto(835.32042793,347.00764873)(835.3554279,346.92264882)(835.39542969,346.84265076)
\curveto(835.49542776,346.62264912)(835.63042762,346.45264929)(835.80042969,346.33265076)
\curveto(835.92042733,346.2426495)(836.0554272,346.17264957)(836.20542969,346.12265076)
\curveto(836.3554269,346.07264967)(836.51542674,346.02264972)(836.68542969,345.97265076)
\lineto(837.00042969,345.92765076)
\lineto(837.09042969,345.92765076)
\curveto(837.16042609,345.90764983)(837.250426,345.89764984)(837.36042969,345.89765076)
\curveto(837.48042577,345.89764984)(837.58042567,345.90764983)(837.66042969,345.92765076)
\curveto(837.73042552,345.92764981)(837.78542547,345.93264981)(837.82542969,345.94265076)
\curveto(837.88542537,345.95264979)(837.94542531,345.95764978)(838.00542969,345.95765076)
\curveto(838.06542519,345.96764977)(838.12042513,345.97764976)(838.17042969,345.98765076)
\curveto(838.46042479,346.06764967)(838.69042456,346.17264957)(838.86042969,346.30265076)
\curveto(839.03042422,346.43264931)(839.1504241,346.65264909)(839.22042969,346.96265076)
\curveto(839.24042401,347.01264873)(839.24542401,347.06764867)(839.23542969,347.12765076)
\curveto(839.22542403,347.18764855)(839.21542404,347.23264851)(839.20542969,347.26265076)
\curveto(839.1554241,347.45264829)(839.08542417,347.59264815)(838.99542969,347.68265076)
\curveto(838.90542435,347.78264796)(838.79042446,347.87264787)(838.65042969,347.95265076)
\curveto(838.56042469,348.01264773)(838.46042479,348.06264768)(838.35042969,348.10265076)
\lineto(838.02042969,348.22265076)
\curveto(837.99042526,348.23264751)(837.96042529,348.2376475)(837.93042969,348.23765076)
\curveto(837.91042534,348.2376475)(837.88542537,348.24764749)(837.85542969,348.26765076)
\curveto(837.51542574,348.37764736)(837.16042609,348.45764728)(836.79042969,348.50765076)
\curveto(836.43042682,348.56764717)(836.09042716,348.66264708)(835.77042969,348.79265076)
\curveto(835.67042758,348.83264691)(835.57542768,348.86764687)(835.48542969,348.89765076)
\curveto(835.39542786,348.92764681)(835.31042794,348.96764677)(835.23042969,349.01765076)
\curveto(835.04042821,349.12764661)(834.86542839,349.25264649)(834.70542969,349.39265076)
\curveto(834.54542871,349.53264621)(834.42042883,349.70764603)(834.33042969,349.91765076)
\curveto(834.30042895,349.98764575)(834.27542898,350.05764568)(834.25542969,350.12765076)
\curveto(834.24542901,350.19764554)(834.23042902,350.27264547)(834.21042969,350.35265076)
\curveto(834.18042907,350.47264527)(834.17042908,350.60764513)(834.18042969,350.75765076)
\curveto(834.19042906,350.91764482)(834.20542905,351.05264469)(834.22542969,351.16265076)
\curveto(834.24542901,351.21264453)(834.255429,351.25264449)(834.25542969,351.28265076)
\curveto(834.26542899,351.32264442)(834.28042897,351.36264438)(834.30042969,351.40265076)
\curveto(834.39042886,351.63264411)(834.51042874,351.83264391)(834.66042969,352.00265076)
\curveto(834.82042843,352.17264357)(835.00042825,352.32264342)(835.20042969,352.45265076)
\curveto(835.3504279,352.5426432)(835.51542774,352.61264313)(835.69542969,352.66265076)
\curveto(835.87542738,352.72264302)(836.06542719,352.77764296)(836.26542969,352.82765076)
\curveto(836.33542692,352.8376429)(836.40042685,352.84764289)(836.46042969,352.85765076)
\curveto(836.53042672,352.86764287)(836.60542665,352.87764286)(836.68542969,352.88765076)
\curveto(836.71542654,352.89764284)(836.7554265,352.89764284)(836.80542969,352.88765076)
\curveto(836.8554264,352.87764286)(836.89042636,352.88264286)(836.91042969,352.90265076)
}
}
{
\newrgbcolor{curcolor}{0 0 0}
\pscustom[linestyle=none,fillstyle=solid,fillcolor=curcolor]
{
\newpath
\moveto(780.41642334,332.34634644)
\curveto(781.39641684,332.36633548)(782.21641602,332.20633564)(782.87642334,331.86634644)
\curveto(783.54641469,331.53633631)(784.06641417,331.07633677)(784.43642334,330.48634644)
\curveto(784.5364137,330.32633752)(784.61641362,330.17133768)(784.67642334,330.02134644)
\curveto(784.74641349,329.88133797)(784.81141342,329.71133814)(784.87142334,329.51134644)
\curveto(784.89141334,329.46133839)(784.91141332,329.39133846)(784.93142334,329.30134644)
\curveto(784.95141328,329.22133863)(784.94641329,329.1463387)(784.91642334,329.07634644)
\curveto(784.89641334,329.01633883)(784.85641338,328.97633887)(784.79642334,328.95634644)
\curveto(784.74641349,328.9463389)(784.69141354,328.93133892)(784.63142334,328.91134644)
\lineto(784.48142334,328.91134644)
\curveto(784.45141378,328.90133895)(784.41141382,328.89633895)(784.36142334,328.89634644)
\lineto(784.24142334,328.89634644)
\curveto(784.10141413,328.89633895)(783.97141426,328.90133895)(783.85142334,328.91134644)
\curveto(783.74141449,328.93133892)(783.66141457,328.98133887)(783.61142334,329.06134644)
\curveto(783.54141469,329.16133869)(783.48641475,329.27633857)(783.44642334,329.40634644)
\curveto(783.40641483,329.53633831)(783.35141488,329.65633819)(783.28142334,329.76634644)
\curveto(783.15141508,329.98633786)(783.00141523,330.17633767)(782.83142334,330.33634644)
\curveto(782.67141556,330.49633735)(782.48141575,330.6463372)(782.26142334,330.78634644)
\curveto(782.14141609,330.86633698)(782.00641623,330.92633692)(781.85642334,330.96634644)
\curveto(781.71641652,331.00633684)(781.57141666,331.0463368)(781.42142334,331.08634644)
\curveto(781.31141692,331.11633673)(781.18641705,331.13633671)(781.04642334,331.14634644)
\curveto(780.90641733,331.16633668)(780.75641748,331.17633667)(780.59642334,331.17634644)
\curveto(780.44641779,331.17633667)(780.29641794,331.16633668)(780.14642334,331.14634644)
\curveto(780.00641823,331.13633671)(779.88641835,331.11633673)(779.78642334,331.08634644)
\curveto(779.68641855,331.06633678)(779.59141864,331.0463368)(779.50142334,331.02634644)
\curveto(779.41141882,331.00633684)(779.32141891,330.97633687)(779.23142334,330.93634644)
\curveto(778.39141984,330.58633726)(777.78642045,329.98633786)(777.41642334,329.13634644)
\curveto(777.34642089,328.99633885)(777.28642095,328.846339)(777.23642334,328.68634644)
\curveto(777.19642104,328.53633931)(777.15142108,328.38133947)(777.10142334,328.22134644)
\curveto(777.08142115,328.16133969)(777.07142116,328.09633975)(777.07142334,328.02634644)
\curveto(777.07142116,327.96633988)(777.06142117,327.90633994)(777.04142334,327.84634644)
\curveto(777.0314212,327.80634004)(777.02642121,327.77134008)(777.02642334,327.74134644)
\curveto(777.02642121,327.71134014)(777.02142121,327.67634017)(777.01142334,327.63634644)
\curveto(776.99142124,327.52634032)(776.97642126,327.41134044)(776.96642334,327.29134644)
\lineto(776.96642334,326.94634644)
\curveto(776.96642127,326.87634097)(776.96142127,326.80134105)(776.95142334,326.72134644)
\curveto(776.95142128,326.6513412)(776.95642128,326.58634126)(776.96642334,326.52634644)
\lineto(776.96642334,326.37634644)
\curveto(776.98642125,326.30634154)(776.99142124,326.23634161)(776.98142334,326.16634644)
\curveto(776.98142125,326.09634175)(776.99142124,326.02634182)(777.01142334,325.95634644)
\curveto(777.0314212,325.89634195)(777.0364212,325.83634201)(777.02642334,325.77634644)
\curveto(777.02642121,325.71634213)(777.0364212,325.66134219)(777.05642334,325.61134644)
\curveto(777.08642115,325.48134237)(777.11142112,325.3513425)(777.13142334,325.22134644)
\curveto(777.16142107,325.10134275)(777.19642104,324.98134287)(777.23642334,324.86134644)
\curveto(777.40642083,324.36134349)(777.62642061,323.93134392)(777.89642334,323.57134644)
\curveto(778.16642007,323.22134463)(778.52141971,322.93134492)(778.96142334,322.70134644)
\curveto(779.10141913,322.63134522)(779.24141899,322.57634527)(779.38142334,322.53634644)
\curveto(779.5314187,322.49634535)(779.69141854,322.4513454)(779.86142334,322.40134644)
\curveto(779.9314183,322.38134547)(779.99641824,322.37134548)(780.05642334,322.37134644)
\curveto(780.11641812,322.38134547)(780.18641805,322.37634547)(780.26642334,322.35634644)
\curveto(780.31641792,322.3463455)(780.40641783,322.33634551)(780.53642334,322.32634644)
\curveto(780.66641757,322.32634552)(780.76141747,322.33634551)(780.82142334,322.35634644)
\lineto(780.92642334,322.35634644)
\curveto(780.96641727,322.36634548)(781.00641723,322.36634548)(781.04642334,322.35634644)
\curveto(781.08641715,322.35634549)(781.12641711,322.36634548)(781.16642334,322.38634644)
\curveto(781.26641697,322.40634544)(781.36141687,322.42134543)(781.45142334,322.43134644)
\curveto(781.55141668,322.4513454)(781.64641659,322.48134537)(781.73642334,322.52134644)
\curveto(782.51641572,322.84134501)(783.06641517,323.36634448)(783.38642334,324.09634644)
\curveto(783.46641477,324.27634357)(783.54141469,324.49134336)(783.61142334,324.74134644)
\curveto(783.6314146,324.83134302)(783.64641459,324.92134293)(783.65642334,325.01134644)
\curveto(783.67641456,325.11134274)(783.71141452,325.20134265)(783.76142334,325.28134644)
\curveto(783.81141442,325.36134249)(783.89141434,325.40634244)(784.00142334,325.41634644)
\curveto(784.11141412,325.42634242)(784.231414,325.43134242)(784.36142334,325.43134644)
\lineto(784.51142334,325.43134644)
\curveto(784.56141367,325.43134242)(784.60641363,325.42634242)(784.64642334,325.41634644)
\lineto(784.75142334,325.41634644)
\lineto(784.84142334,325.38634644)
\curveto(784.88141335,325.38634246)(784.91141332,325.37634247)(784.93142334,325.35634644)
\curveto(785.00141323,325.31634253)(785.04141319,325.24134261)(785.05142334,325.13134644)
\curveto(785.06141317,325.03134282)(785.05141318,324.93134292)(785.02142334,324.83134644)
\curveto(784.96141327,324.60134325)(784.90641333,324.38134347)(784.85642334,324.17134644)
\curveto(784.80641343,323.96134389)(784.7314135,323.76134409)(784.63142334,323.57134644)
\curveto(784.55141368,323.44134441)(784.47641376,323.31634453)(784.40642334,323.19634644)
\curveto(784.34641389,323.07634477)(784.27641396,322.95634489)(784.19642334,322.83634644)
\curveto(784.01641422,322.57634527)(783.79141444,322.33634551)(783.52142334,322.11634644)
\curveto(783.26141497,321.90634594)(782.97641526,321.73134612)(782.66642334,321.59134644)
\curveto(782.55641568,321.54134631)(782.44641579,321.50134635)(782.33642334,321.47134644)
\curveto(782.236416,321.44134641)(782.1314161,321.40634644)(782.02142334,321.36634644)
\curveto(781.91141632,321.32634652)(781.79641644,321.30134655)(781.67642334,321.29134644)
\curveto(781.56641667,321.27134658)(781.45141678,321.2513466)(781.33142334,321.23134644)
\curveto(781.28141695,321.21134664)(781.236417,321.20634664)(781.19642334,321.21634644)
\curveto(781.15641708,321.21634663)(781.11641712,321.21134664)(781.07642334,321.20134644)
\curveto(781.01641722,321.19134666)(780.95641728,321.18634666)(780.89642334,321.18634644)
\curveto(780.8364174,321.18634666)(780.77141746,321.18134667)(780.70142334,321.17134644)
\curveto(780.67141756,321.16134669)(780.60141763,321.16134669)(780.49142334,321.17134644)
\curveto(780.39141784,321.17134668)(780.32641791,321.17634667)(780.29642334,321.18634644)
\curveto(780.24641799,321.19634665)(780.19641804,321.20134665)(780.14642334,321.20134644)
\curveto(780.10641813,321.19134666)(780.06141817,321.19134666)(780.01142334,321.20134644)
\lineto(779.86142334,321.20134644)
\curveto(779.78141845,321.22134663)(779.70641853,321.23634661)(779.63642334,321.24634644)
\curveto(779.56641867,321.2463466)(779.49141874,321.25634659)(779.41142334,321.27634644)
\lineto(779.14142334,321.33634644)
\curveto(779.05141918,321.3463465)(778.96641927,321.36634648)(778.88642334,321.39634644)
\curveto(778.67641956,321.45634639)(778.48641975,321.53134632)(778.31642334,321.62134644)
\curveto(777.68642055,321.89134596)(777.17642106,322.27634557)(776.78642334,322.77634644)
\curveto(776.39642184,323.27634457)(776.08642215,323.86634398)(775.85642334,324.54634644)
\curveto(775.81642242,324.66634318)(775.78142245,324.79134306)(775.75142334,324.92134644)
\curveto(775.7314225,325.0513428)(775.70642253,325.18634266)(775.67642334,325.32634644)
\curveto(775.65642258,325.37634247)(775.64642259,325.42134243)(775.64642334,325.46134644)
\curveto(775.65642258,325.50134235)(775.65642258,325.5463423)(775.64642334,325.59634644)
\curveto(775.62642261,325.68634216)(775.61142262,325.78134207)(775.60142334,325.88134644)
\curveto(775.60142263,325.98134187)(775.59142264,326.07634177)(775.57142334,326.16634644)
\lineto(775.57142334,326.45134644)
\curveto(775.55142268,326.50134135)(775.54142269,326.58634126)(775.54142334,326.70634644)
\curveto(775.54142269,326.82634102)(775.55142268,326.91134094)(775.57142334,326.96134644)
\curveto(775.58142265,326.99134086)(775.58142265,327.02134083)(775.57142334,327.05134644)
\curveto(775.56142267,327.09134076)(775.56142267,327.12134073)(775.57142334,327.14134644)
\lineto(775.57142334,327.27634644)
\curveto(775.58142265,327.35634049)(775.58642265,327.43634041)(775.58642334,327.51634644)
\curveto(775.59642264,327.60634024)(775.61142262,327.69134016)(775.63142334,327.77134644)
\curveto(775.65142258,327.83134002)(775.66142257,327.89133996)(775.66142334,327.95134644)
\curveto(775.66142257,328.02133983)(775.67142256,328.09133976)(775.69142334,328.16134644)
\curveto(775.74142249,328.33133952)(775.78142245,328.49633935)(775.81142334,328.65634644)
\curveto(775.84142239,328.81633903)(775.88642235,328.96633888)(775.94642334,329.10634644)
\lineto(776.09642334,329.49634644)
\curveto(776.15642208,329.63633821)(776.22142201,329.76133809)(776.29142334,329.87134644)
\curveto(776.44142179,330.13133772)(776.59142164,330.36633748)(776.74142334,330.57634644)
\curveto(776.77142146,330.62633722)(776.80642143,330.66633718)(776.84642334,330.69634644)
\curveto(776.89642134,330.73633711)(776.9364213,330.78133707)(776.96642334,330.83134644)
\curveto(777.02642121,330.91133694)(777.08642115,330.98133687)(777.14642334,331.04134644)
\lineto(777.35642334,331.22134644)
\curveto(777.41642082,331.27133658)(777.47142076,331.31633653)(777.52142334,331.35634644)
\curveto(777.58142065,331.40633644)(777.64642059,331.45633639)(777.71642334,331.50634644)
\curveto(777.86642037,331.61633623)(778.02142021,331.71133614)(778.18142334,331.79134644)
\curveto(778.35141988,331.87133598)(778.52641971,331.9513359)(778.70642334,332.03134644)
\curveto(778.81641942,332.08133577)(778.9314193,332.11633573)(779.05142334,332.13634644)
\curveto(779.18141905,332.16633568)(779.30641893,332.20133565)(779.42642334,332.24134644)
\curveto(779.49641874,332.2513356)(779.56141867,332.26133559)(779.62142334,332.27134644)
\lineto(779.80142334,332.30134644)
\curveto(779.88141835,332.31133554)(779.95641828,332.31633553)(780.02642334,332.31634644)
\curveto(780.10641813,332.32633552)(780.18641805,332.33633551)(780.26642334,332.34634644)
\curveto(780.28641795,332.35633549)(780.31141792,332.35633549)(780.34142334,332.34634644)
\curveto(780.37141786,332.33633551)(780.39641784,332.33633551)(780.41642334,332.34634644)
}
}
{
\newrgbcolor{curcolor}{0 0 0}
\pscustom[linestyle=none,fillstyle=solid,fillcolor=curcolor]
{
\newpath
\moveto(793.77626709,325.62634644)
\curveto(793.79625903,325.56634228)(793.80625902,325.47134238)(793.80626709,325.34134644)
\curveto(793.80625902,325.22134263)(793.80125902,325.13634271)(793.79126709,325.08634644)
\lineto(793.79126709,324.93634644)
\curveto(793.78125904,324.85634299)(793.77125905,324.78134307)(793.76126709,324.71134644)
\curveto(793.76125906,324.6513432)(793.75625907,324.58134327)(793.74626709,324.50134644)
\curveto(793.7262591,324.44134341)(793.71125911,324.38134347)(793.70126709,324.32134644)
\curveto(793.70125912,324.26134359)(793.69125913,324.20134365)(793.67126709,324.14134644)
\curveto(793.63125919,324.01134384)(793.59625923,323.88134397)(793.56626709,323.75134644)
\curveto(793.53625929,323.62134423)(793.49625933,323.50134435)(793.44626709,323.39134644)
\curveto(793.23625959,322.91134494)(792.95625987,322.50634534)(792.60626709,322.17634644)
\curveto(792.25626057,321.85634599)(791.826261,321.61134624)(791.31626709,321.44134644)
\curveto(791.20626162,321.40134645)(791.08626174,321.37134648)(790.95626709,321.35134644)
\curveto(790.83626199,321.33134652)(790.71126211,321.31134654)(790.58126709,321.29134644)
\curveto(790.5212623,321.28134657)(790.45626237,321.27634657)(790.38626709,321.27634644)
\curveto(790.3262625,321.26634658)(790.26626256,321.26134659)(790.20626709,321.26134644)
\curveto(790.16626266,321.2513466)(790.10626272,321.2463466)(790.02626709,321.24634644)
\curveto(789.95626287,321.2463466)(789.90626292,321.2513466)(789.87626709,321.26134644)
\curveto(789.83626299,321.27134658)(789.79626303,321.27634657)(789.75626709,321.27634644)
\curveto(789.71626311,321.26634658)(789.68126314,321.26634658)(789.65126709,321.27634644)
\lineto(789.56126709,321.27634644)
\lineto(789.20126709,321.32134644)
\curveto(789.06126376,321.36134649)(788.9262639,321.40134645)(788.79626709,321.44134644)
\curveto(788.66626416,321.48134637)(788.54126428,321.52634632)(788.42126709,321.57634644)
\curveto(787.97126485,321.77634607)(787.60126522,322.03634581)(787.31126709,322.35634644)
\curveto(787.0212658,322.67634517)(786.78126604,323.06634478)(786.59126709,323.52634644)
\curveto(786.54126628,323.62634422)(786.50126632,323.72634412)(786.47126709,323.82634644)
\curveto(786.45126637,323.92634392)(786.43126639,324.03134382)(786.41126709,324.14134644)
\curveto(786.39126643,324.18134367)(786.38126644,324.21134364)(786.38126709,324.23134644)
\curveto(786.39126643,324.26134359)(786.39126643,324.29634355)(786.38126709,324.33634644)
\curveto(786.36126646,324.41634343)(786.34626648,324.49634335)(786.33626709,324.57634644)
\curveto(786.33626649,324.66634318)(786.3262665,324.7513431)(786.30626709,324.83134644)
\lineto(786.30626709,324.95134644)
\curveto(786.30626652,324.99134286)(786.30126652,325.03634281)(786.29126709,325.08634644)
\curveto(786.28126654,325.13634271)(786.27626655,325.22134263)(786.27626709,325.34134644)
\curveto(786.27626655,325.47134238)(786.28626654,325.56634228)(786.30626709,325.62634644)
\curveto(786.3262665,325.69634215)(786.33126649,325.76634208)(786.32126709,325.83634644)
\curveto(786.31126651,325.90634194)(786.31626651,325.97634187)(786.33626709,326.04634644)
\curveto(786.34626648,326.09634175)(786.35126647,326.13634171)(786.35126709,326.16634644)
\curveto(786.36126646,326.20634164)(786.37126645,326.2513416)(786.38126709,326.30134644)
\curveto(786.41126641,326.42134143)(786.43626639,326.54134131)(786.45626709,326.66134644)
\curveto(786.48626634,326.78134107)(786.5262663,326.89634095)(786.57626709,327.00634644)
\curveto(786.7262661,327.37634047)(786.90626592,327.70634014)(787.11626709,327.99634644)
\curveto(787.33626549,328.29633955)(787.60126522,328.5463393)(787.91126709,328.74634644)
\curveto(788.03126479,328.82633902)(788.15626467,328.89133896)(788.28626709,328.94134644)
\curveto(788.41626441,329.00133885)(788.55126427,329.06133879)(788.69126709,329.12134644)
\curveto(788.81126401,329.17133868)(788.94126388,329.20133865)(789.08126709,329.21134644)
\curveto(789.2212636,329.23133862)(789.36126346,329.26133859)(789.50126709,329.30134644)
\lineto(789.69626709,329.30134644)
\curveto(789.76626306,329.31133854)(789.83126299,329.32133853)(789.89126709,329.33134644)
\curveto(790.78126204,329.34133851)(791.5212613,329.15633869)(792.11126709,328.77634644)
\curveto(792.70126012,328.39633945)(793.1262597,327.90133995)(793.38626709,327.29134644)
\curveto(793.43625939,327.19134066)(793.47625935,327.09134076)(793.50626709,326.99134644)
\curveto(793.53625929,326.89134096)(793.57125925,326.78634106)(793.61126709,326.67634644)
\curveto(793.64125918,326.56634128)(793.66625916,326.4463414)(793.68626709,326.31634644)
\curveto(793.70625912,326.19634165)(793.73125909,326.07134178)(793.76126709,325.94134644)
\curveto(793.77125905,325.89134196)(793.77125905,325.83634201)(793.76126709,325.77634644)
\curveto(793.76125906,325.72634212)(793.76625906,325.67634217)(793.77626709,325.62634644)
\moveto(792.44126709,324.77134644)
\curveto(792.46126036,324.84134301)(792.46626036,324.92134293)(792.45626709,325.01134644)
\lineto(792.45626709,325.26634644)
\curveto(792.45626037,325.65634219)(792.4212604,325.98634186)(792.35126709,326.25634644)
\curveto(792.3212605,326.33634151)(792.29626053,326.41634143)(792.27626709,326.49634644)
\curveto(792.25626057,326.57634127)(792.23126059,326.6513412)(792.20126709,326.72134644)
\curveto(791.9212609,327.37134048)(791.47626135,327.82134003)(790.86626709,328.07134644)
\curveto(790.79626203,328.10133975)(790.7212621,328.12133973)(790.64126709,328.13134644)
\lineto(790.40126709,328.19134644)
\curveto(790.3212625,328.21133964)(790.23626259,328.22133963)(790.14626709,328.22134644)
\lineto(789.87626709,328.22134644)
\lineto(789.60626709,328.17634644)
\curveto(789.50626332,328.15633969)(789.41126341,328.13133972)(789.32126709,328.10134644)
\curveto(789.24126358,328.08133977)(789.16126366,328.0513398)(789.08126709,328.01134644)
\curveto(789.01126381,327.99133986)(788.94626388,327.96133989)(788.88626709,327.92134644)
\curveto(788.826264,327.88133997)(788.77126405,327.84134001)(788.72126709,327.80134644)
\curveto(788.48126434,327.63134022)(788.28626454,327.42634042)(788.13626709,327.18634644)
\curveto(787.98626484,326.9463409)(787.85626497,326.66634118)(787.74626709,326.34634644)
\curveto(787.71626511,326.2463416)(787.69626513,326.14134171)(787.68626709,326.03134644)
\curveto(787.67626515,325.93134192)(787.66126516,325.82634202)(787.64126709,325.71634644)
\curveto(787.63126519,325.67634217)(787.6262652,325.61134224)(787.62626709,325.52134644)
\curveto(787.61626521,325.49134236)(787.61126521,325.45634239)(787.61126709,325.41634644)
\curveto(787.6212652,325.37634247)(787.6262652,325.33134252)(787.62626709,325.28134644)
\lineto(787.62626709,324.98134644)
\curveto(787.6262652,324.88134297)(787.63626519,324.79134306)(787.65626709,324.71134644)
\lineto(787.68626709,324.53134644)
\curveto(787.70626512,324.43134342)(787.7212651,324.33134352)(787.73126709,324.23134644)
\curveto(787.75126507,324.14134371)(787.78126504,324.05634379)(787.82126709,323.97634644)
\curveto(787.9212649,323.73634411)(788.03626479,323.51134434)(788.16626709,323.30134644)
\curveto(788.30626452,323.09134476)(788.47626435,322.91634493)(788.67626709,322.77634644)
\curveto(788.7262641,322.7463451)(788.77126405,322.72134513)(788.81126709,322.70134644)
\curveto(788.85126397,322.68134517)(788.89626393,322.65634519)(788.94626709,322.62634644)
\curveto(789.0262638,322.57634527)(789.11126371,322.53134532)(789.20126709,322.49134644)
\curveto(789.30126352,322.46134539)(789.40626342,322.43134542)(789.51626709,322.40134644)
\curveto(789.56626326,322.38134547)(789.61126321,322.37134548)(789.65126709,322.37134644)
\curveto(789.70126312,322.38134547)(789.75126307,322.38134547)(789.80126709,322.37134644)
\curveto(789.83126299,322.36134549)(789.89126293,322.3513455)(789.98126709,322.34134644)
\curveto(790.08126274,322.33134552)(790.15626267,322.33634551)(790.20626709,322.35634644)
\curveto(790.24626258,322.36634548)(790.28626254,322.36634548)(790.32626709,322.35634644)
\curveto(790.36626246,322.35634549)(790.40626242,322.36634548)(790.44626709,322.38634644)
\curveto(790.5262623,322.40634544)(790.60626222,322.42134543)(790.68626709,322.43134644)
\curveto(790.76626206,322.4513454)(790.84126198,322.47634537)(790.91126709,322.50634644)
\curveto(791.25126157,322.6463452)(791.5262613,322.84134501)(791.73626709,323.09134644)
\curveto(791.94626088,323.34134451)(792.1212607,323.63634421)(792.26126709,323.97634644)
\curveto(792.31126051,324.09634375)(792.34126048,324.22134363)(792.35126709,324.35134644)
\curveto(792.37126045,324.49134336)(792.40126042,324.63134322)(792.44126709,324.77134644)
}
}
{
\newrgbcolor{curcolor}{0 0 0}
\pscustom[linestyle=none,fillstyle=solid,fillcolor=curcolor]
{
\newpath
\moveto(798.90954834,329.33134644)
\curveto(799.13954355,329.33133852)(799.26954342,329.27133858)(799.29954834,329.15134644)
\curveto(799.32954336,329.04133881)(799.34454334,328.87633897)(799.34454834,328.65634644)
\lineto(799.34454834,328.37134644)
\curveto(799.34454334,328.28133957)(799.31954337,328.20633964)(799.26954834,328.14634644)
\curveto(799.20954348,328.06633978)(799.12454356,328.02133983)(799.01454834,328.01134644)
\curveto(798.90454378,328.01133984)(798.79454389,327.99633985)(798.68454834,327.96634644)
\curveto(798.54454414,327.93633991)(798.40954428,327.90633994)(798.27954834,327.87634644)
\curveto(798.15954453,327.84634)(798.04454464,327.80634004)(797.93454834,327.75634644)
\curveto(797.64454504,327.62634022)(797.40954528,327.4463404)(797.22954834,327.21634644)
\curveto(797.04954564,326.99634085)(796.89454579,326.74134111)(796.76454834,326.45134644)
\curveto(796.72454596,326.34134151)(796.69454599,326.22634162)(796.67454834,326.10634644)
\curveto(796.65454603,325.99634185)(796.62954606,325.88134197)(796.59954834,325.76134644)
\curveto(796.5895461,325.71134214)(796.5845461,325.66134219)(796.58454834,325.61134644)
\curveto(796.59454609,325.56134229)(796.59454609,325.51134234)(796.58454834,325.46134644)
\curveto(796.55454613,325.34134251)(796.53954615,325.20134265)(796.53954834,325.04134644)
\curveto(796.54954614,324.89134296)(796.55454613,324.7463431)(796.55454834,324.60634644)
\lineto(796.55454834,322.76134644)
\lineto(796.55454834,322.41634644)
\curveto(796.55454613,322.29634555)(796.54954614,322.18134567)(796.53954834,322.07134644)
\curveto(796.52954616,321.96134589)(796.52454616,321.86634598)(796.52454834,321.78634644)
\curveto(796.53454615,321.70634614)(796.51454617,321.63634621)(796.46454834,321.57634644)
\curveto(796.41454627,321.50634634)(796.33454635,321.46634638)(796.22454834,321.45634644)
\curveto(796.12454656,321.4463464)(796.01454667,321.44134641)(795.89454834,321.44134644)
\lineto(795.62454834,321.44134644)
\curveto(795.57454711,321.46134639)(795.52454716,321.47634637)(795.47454834,321.48634644)
\curveto(795.43454725,321.50634634)(795.40454728,321.53134632)(795.38454834,321.56134644)
\curveto(795.33454735,321.63134622)(795.30454738,321.71634613)(795.29454834,321.81634644)
\lineto(795.29454834,322.14634644)
\lineto(795.29454834,323.30134644)
\lineto(795.29454834,327.45634644)
\lineto(795.29454834,328.49134644)
\lineto(795.29454834,328.79134644)
\curveto(795.30454738,328.89133896)(795.33454735,328.97633887)(795.38454834,329.04634644)
\curveto(795.41454727,329.08633876)(795.46454722,329.11633873)(795.53454834,329.13634644)
\curveto(795.61454707,329.15633869)(795.69954699,329.16633868)(795.78954834,329.16634644)
\curveto(795.87954681,329.17633867)(795.96954672,329.17633867)(796.05954834,329.16634644)
\curveto(796.14954654,329.15633869)(796.21954647,329.14133871)(796.26954834,329.12134644)
\curveto(796.34954634,329.09133876)(796.39954629,329.03133882)(796.41954834,328.94134644)
\curveto(796.44954624,328.86133899)(796.46454622,328.77133908)(796.46454834,328.67134644)
\lineto(796.46454834,328.37134644)
\curveto(796.46454622,328.27133958)(796.4845462,328.18133967)(796.52454834,328.10134644)
\curveto(796.53454615,328.08133977)(796.54454614,328.06633978)(796.55454834,328.05634644)
\lineto(796.59954834,328.01134644)
\curveto(796.70954598,328.01133984)(796.79954589,328.05633979)(796.86954834,328.14634644)
\curveto(796.93954575,328.2463396)(796.99954569,328.32633952)(797.04954834,328.38634644)
\lineto(797.13954834,328.47634644)
\curveto(797.22954546,328.58633926)(797.35454533,328.70133915)(797.51454834,328.82134644)
\curveto(797.67454501,328.94133891)(797.82454486,329.03133882)(797.96454834,329.09134644)
\curveto(798.05454463,329.14133871)(798.14954454,329.17633867)(798.24954834,329.19634644)
\curveto(798.34954434,329.22633862)(798.45454423,329.25633859)(798.56454834,329.28634644)
\curveto(798.62454406,329.29633855)(798.684544,329.30133855)(798.74454834,329.30134644)
\curveto(798.80454388,329.31133854)(798.85954383,329.32133853)(798.90954834,329.33134644)
}
}
{
\newrgbcolor{curcolor}{0 0 0}
\pscustom[linestyle=none,fillstyle=solid,fillcolor=curcolor]
{
\newpath
\moveto(803.91931396,329.33134644)
\curveto(804.14930917,329.33133852)(804.27930904,329.27133858)(804.30931396,329.15134644)
\curveto(804.33930898,329.04133881)(804.35430897,328.87633897)(804.35431396,328.65634644)
\lineto(804.35431396,328.37134644)
\curveto(804.35430897,328.28133957)(804.32930899,328.20633964)(804.27931396,328.14634644)
\curveto(804.2193091,328.06633978)(804.13430919,328.02133983)(804.02431396,328.01134644)
\curveto(803.91430941,328.01133984)(803.80430952,327.99633985)(803.69431396,327.96634644)
\curveto(803.55430977,327.93633991)(803.4193099,327.90633994)(803.28931396,327.87634644)
\curveto(803.16931015,327.84634)(803.05431027,327.80634004)(802.94431396,327.75634644)
\curveto(802.65431067,327.62634022)(802.4193109,327.4463404)(802.23931396,327.21634644)
\curveto(802.05931126,326.99634085)(801.90431142,326.74134111)(801.77431396,326.45134644)
\curveto(801.73431159,326.34134151)(801.70431162,326.22634162)(801.68431396,326.10634644)
\curveto(801.66431166,325.99634185)(801.63931168,325.88134197)(801.60931396,325.76134644)
\curveto(801.59931172,325.71134214)(801.59431173,325.66134219)(801.59431396,325.61134644)
\curveto(801.60431172,325.56134229)(801.60431172,325.51134234)(801.59431396,325.46134644)
\curveto(801.56431176,325.34134251)(801.54931177,325.20134265)(801.54931396,325.04134644)
\curveto(801.55931176,324.89134296)(801.56431176,324.7463431)(801.56431396,324.60634644)
\lineto(801.56431396,322.76134644)
\lineto(801.56431396,322.41634644)
\curveto(801.56431176,322.29634555)(801.55931176,322.18134567)(801.54931396,322.07134644)
\curveto(801.53931178,321.96134589)(801.53431179,321.86634598)(801.53431396,321.78634644)
\curveto(801.54431178,321.70634614)(801.5243118,321.63634621)(801.47431396,321.57634644)
\curveto(801.4243119,321.50634634)(801.34431198,321.46634638)(801.23431396,321.45634644)
\curveto(801.13431219,321.4463464)(801.0243123,321.44134641)(800.90431396,321.44134644)
\lineto(800.63431396,321.44134644)
\curveto(800.58431274,321.46134639)(800.53431279,321.47634637)(800.48431396,321.48634644)
\curveto(800.44431288,321.50634634)(800.41431291,321.53134632)(800.39431396,321.56134644)
\curveto(800.34431298,321.63134622)(800.31431301,321.71634613)(800.30431396,321.81634644)
\lineto(800.30431396,322.14634644)
\lineto(800.30431396,323.30134644)
\lineto(800.30431396,327.45634644)
\lineto(800.30431396,328.49134644)
\lineto(800.30431396,328.79134644)
\curveto(800.31431301,328.89133896)(800.34431298,328.97633887)(800.39431396,329.04634644)
\curveto(800.4243129,329.08633876)(800.47431285,329.11633873)(800.54431396,329.13634644)
\curveto(800.6243127,329.15633869)(800.70931261,329.16633868)(800.79931396,329.16634644)
\curveto(800.88931243,329.17633867)(800.97931234,329.17633867)(801.06931396,329.16634644)
\curveto(801.15931216,329.15633869)(801.22931209,329.14133871)(801.27931396,329.12134644)
\curveto(801.35931196,329.09133876)(801.40931191,329.03133882)(801.42931396,328.94134644)
\curveto(801.45931186,328.86133899)(801.47431185,328.77133908)(801.47431396,328.67134644)
\lineto(801.47431396,328.37134644)
\curveto(801.47431185,328.27133958)(801.49431183,328.18133967)(801.53431396,328.10134644)
\curveto(801.54431178,328.08133977)(801.55431177,328.06633978)(801.56431396,328.05634644)
\lineto(801.60931396,328.01134644)
\curveto(801.7193116,328.01133984)(801.80931151,328.05633979)(801.87931396,328.14634644)
\curveto(801.94931137,328.2463396)(802.00931131,328.32633952)(802.05931396,328.38634644)
\lineto(802.14931396,328.47634644)
\curveto(802.23931108,328.58633926)(802.36431096,328.70133915)(802.52431396,328.82134644)
\curveto(802.68431064,328.94133891)(802.83431049,329.03133882)(802.97431396,329.09134644)
\curveto(803.06431026,329.14133871)(803.15931016,329.17633867)(803.25931396,329.19634644)
\curveto(803.35930996,329.22633862)(803.46430986,329.25633859)(803.57431396,329.28634644)
\curveto(803.63430969,329.29633855)(803.69430963,329.30133855)(803.75431396,329.30134644)
\curveto(803.81430951,329.31133854)(803.86930945,329.32133853)(803.91931396,329.33134644)
}
}
{
\newrgbcolor{curcolor}{0 0 0}
\pscustom[linestyle=none,fillstyle=solid,fillcolor=curcolor]
{
\newpath
\moveto(812.03407959,325.59634644)
\curveto(812.0540719,325.49634235)(812.0540719,325.38134247)(812.03407959,325.25134644)
\curveto(812.02407193,325.13134272)(811.99407196,325.0463428)(811.94407959,324.99634644)
\curveto(811.89407206,324.95634289)(811.81907214,324.92634292)(811.71907959,324.90634644)
\curveto(811.62907233,324.89634295)(811.52407243,324.89134296)(811.40407959,324.89134644)
\lineto(811.04407959,324.89134644)
\curveto(810.92407303,324.90134295)(810.81907314,324.90634294)(810.72907959,324.90634644)
\lineto(806.88907959,324.90634644)
\curveto(806.80907715,324.90634294)(806.72907723,324.90134295)(806.64907959,324.89134644)
\curveto(806.56907739,324.89134296)(806.50407745,324.87634297)(806.45407959,324.84634644)
\curveto(806.41407754,324.82634302)(806.37407758,324.78634306)(806.33407959,324.72634644)
\curveto(806.31407764,324.69634315)(806.29407766,324.6513432)(806.27407959,324.59134644)
\curveto(806.2540777,324.54134331)(806.2540777,324.49134336)(806.27407959,324.44134644)
\curveto(806.28407767,324.39134346)(806.28907767,324.3463435)(806.28907959,324.30634644)
\curveto(806.28907767,324.26634358)(806.29407766,324.22634362)(806.30407959,324.18634644)
\curveto(806.32407763,324.10634374)(806.34407761,324.02134383)(806.36407959,323.93134644)
\curveto(806.38407757,323.851344)(806.41407754,323.77134408)(806.45407959,323.69134644)
\curveto(806.68407727,323.1513447)(807.06407689,322.76634508)(807.59407959,322.53634644)
\curveto(807.6540763,322.50634534)(807.71907624,322.48134537)(807.78907959,322.46134644)
\lineto(807.99907959,322.40134644)
\curveto(808.02907593,322.39134546)(808.07907588,322.38634546)(808.14907959,322.38634644)
\curveto(808.28907567,322.3463455)(808.47407548,322.32634552)(808.70407959,322.32634644)
\curveto(808.93407502,322.32634552)(809.11907484,322.3463455)(809.25907959,322.38634644)
\curveto(809.39907456,322.42634542)(809.52407443,322.46634538)(809.63407959,322.50634644)
\curveto(809.7540742,322.55634529)(809.86407409,322.61634523)(809.96407959,322.68634644)
\curveto(810.07407388,322.75634509)(810.16907379,322.83634501)(810.24907959,322.92634644)
\curveto(810.32907363,323.02634482)(810.39907356,323.13134472)(810.45907959,323.24134644)
\curveto(810.51907344,323.34134451)(810.56907339,323.4463444)(810.60907959,323.55634644)
\curveto(810.6590733,323.66634418)(810.73907322,323.7463441)(810.84907959,323.79634644)
\curveto(810.88907307,323.81634403)(810.954073,323.83134402)(811.04407959,323.84134644)
\curveto(811.13407282,323.851344)(811.22407273,323.851344)(811.31407959,323.84134644)
\curveto(811.40407255,323.84134401)(811.48907247,323.83634401)(811.56907959,323.82634644)
\curveto(811.64907231,323.81634403)(811.70407225,323.79634405)(811.73407959,323.76634644)
\curveto(811.83407212,323.69634415)(811.8590721,323.58134427)(811.80907959,323.42134644)
\curveto(811.72907223,323.1513447)(811.62407233,322.91134494)(811.49407959,322.70134644)
\curveto(811.29407266,322.38134547)(811.06407289,322.11634573)(810.80407959,321.90634644)
\curveto(810.5540734,321.70634614)(810.23407372,321.54134631)(809.84407959,321.41134644)
\curveto(809.74407421,321.37134648)(809.64407431,321.3463465)(809.54407959,321.33634644)
\curveto(809.44407451,321.31634653)(809.33907462,321.29634655)(809.22907959,321.27634644)
\curveto(809.17907478,321.26634658)(809.12907483,321.26134659)(809.07907959,321.26134644)
\curveto(809.03907492,321.26134659)(808.99407496,321.25634659)(808.94407959,321.24634644)
\lineto(808.79407959,321.24634644)
\curveto(808.74407521,321.23634661)(808.68407527,321.23134662)(808.61407959,321.23134644)
\curveto(808.5540754,321.23134662)(808.50407545,321.23634661)(808.46407959,321.24634644)
\lineto(808.32907959,321.24634644)
\curveto(808.27907568,321.25634659)(808.23407572,321.26134659)(808.19407959,321.26134644)
\curveto(808.1540758,321.26134659)(808.11407584,321.26634658)(808.07407959,321.27634644)
\curveto(808.02407593,321.28634656)(807.96907599,321.29634655)(807.90907959,321.30634644)
\curveto(807.84907611,321.30634654)(807.79407616,321.31134654)(807.74407959,321.32134644)
\curveto(807.6540763,321.34134651)(807.56407639,321.36634648)(807.47407959,321.39634644)
\curveto(807.38407657,321.41634643)(807.29907666,321.44134641)(807.21907959,321.47134644)
\curveto(807.17907678,321.49134636)(807.14407681,321.50134635)(807.11407959,321.50134644)
\curveto(807.08407687,321.51134634)(807.04907691,321.52634632)(807.00907959,321.54634644)
\curveto(806.8590771,321.61634623)(806.69907726,321.70134615)(806.52907959,321.80134644)
\curveto(806.23907772,321.99134586)(805.98907797,322.22134563)(805.77907959,322.49134644)
\curveto(805.57907838,322.77134508)(805.40907855,323.08134477)(805.26907959,323.42134644)
\curveto(805.21907874,323.53134432)(805.17907878,323.6463442)(805.14907959,323.76634644)
\curveto(805.12907883,323.88634396)(805.09907886,324.00634384)(805.05907959,324.12634644)
\curveto(805.04907891,324.16634368)(805.04407891,324.20134365)(805.04407959,324.23134644)
\curveto(805.04407891,324.26134359)(805.03907892,324.30134355)(805.02907959,324.35134644)
\curveto(805.00907895,324.43134342)(804.99407896,324.51634333)(804.98407959,324.60634644)
\curveto(804.97407898,324.69634315)(804.959079,324.78634306)(804.93907959,324.87634644)
\lineto(804.93907959,325.08634644)
\curveto(804.92907903,325.12634272)(804.91907904,325.18134267)(804.90907959,325.25134644)
\curveto(804.90907905,325.33134252)(804.91407904,325.39634245)(804.92407959,325.44634644)
\lineto(804.92407959,325.61134644)
\curveto(804.94407901,325.66134219)(804.94907901,325.71134214)(804.93907959,325.76134644)
\curveto(804.93907902,325.82134203)(804.94407901,325.87634197)(804.95407959,325.92634644)
\curveto(804.99407896,326.08634176)(805.02407893,326.2463416)(805.04407959,326.40634644)
\curveto(805.07407888,326.56634128)(805.11907884,326.71634113)(805.17907959,326.85634644)
\curveto(805.22907873,326.96634088)(805.27407868,327.07634077)(805.31407959,327.18634644)
\curveto(805.36407859,327.30634054)(805.41907854,327.42134043)(805.47907959,327.53134644)
\curveto(805.69907826,327.88133997)(805.94907801,328.18133967)(806.22907959,328.43134644)
\curveto(806.50907745,328.69133916)(806.8540771,328.90633894)(807.26407959,329.07634644)
\curveto(807.38407657,329.12633872)(807.50407645,329.16133869)(807.62407959,329.18134644)
\curveto(807.7540762,329.21133864)(807.88907607,329.24133861)(808.02907959,329.27134644)
\curveto(808.07907588,329.28133857)(808.12407583,329.28633856)(808.16407959,329.28634644)
\curveto(808.20407575,329.29633855)(808.24907571,329.30133855)(808.29907959,329.30134644)
\curveto(808.31907564,329.31133854)(808.34407561,329.31133854)(808.37407959,329.30134644)
\curveto(808.40407555,329.29133856)(808.42907553,329.29633855)(808.44907959,329.31634644)
\curveto(808.86907509,329.32633852)(809.23407472,329.28133857)(809.54407959,329.18134644)
\curveto(809.8540741,329.09133876)(810.13407382,328.96633888)(810.38407959,328.80634644)
\curveto(810.43407352,328.78633906)(810.47407348,328.75633909)(810.50407959,328.71634644)
\curveto(810.53407342,328.68633916)(810.56907339,328.66133919)(810.60907959,328.64134644)
\curveto(810.68907327,328.58133927)(810.76907319,328.51133934)(810.84907959,328.43134644)
\curveto(810.93907302,328.3513395)(811.01407294,328.27133958)(811.07407959,328.19134644)
\curveto(811.23407272,327.98133987)(811.36907259,327.78134007)(811.47907959,327.59134644)
\curveto(811.54907241,327.48134037)(811.60407235,327.36134049)(811.64407959,327.23134644)
\curveto(811.68407227,327.10134075)(811.72907223,326.97134088)(811.77907959,326.84134644)
\curveto(811.82907213,326.71134114)(811.86407209,326.57634127)(811.88407959,326.43634644)
\curveto(811.91407204,326.29634155)(811.94907201,326.15634169)(811.98907959,326.01634644)
\curveto(811.99907196,325.9463419)(812.00407195,325.87634197)(812.00407959,325.80634644)
\lineto(812.03407959,325.59634644)
\moveto(810.57907959,326.10634644)
\curveto(810.60907335,326.1463417)(810.63407332,326.19634165)(810.65407959,326.25634644)
\curveto(810.67407328,326.32634152)(810.67407328,326.39634145)(810.65407959,326.46634644)
\curveto(810.59407336,326.68634116)(810.50907345,326.89134096)(810.39907959,327.08134644)
\curveto(810.2590737,327.31134054)(810.10407385,327.50634034)(809.93407959,327.66634644)
\curveto(809.76407419,327.82634002)(809.54407441,327.96133989)(809.27407959,328.07134644)
\curveto(809.20407475,328.09133976)(809.13407482,328.10633974)(809.06407959,328.11634644)
\curveto(808.99407496,328.13633971)(808.91907504,328.15633969)(808.83907959,328.17634644)
\curveto(808.7590752,328.19633965)(808.67407528,328.20633964)(808.58407959,328.20634644)
\lineto(808.32907959,328.20634644)
\curveto(808.29907566,328.18633966)(808.26407569,328.17633967)(808.22407959,328.17634644)
\curveto(808.18407577,328.18633966)(808.14907581,328.18633966)(808.11907959,328.17634644)
\lineto(807.87907959,328.11634644)
\curveto(807.80907615,328.10633974)(807.73907622,328.09133976)(807.66907959,328.07134644)
\curveto(807.37907658,327.9513399)(807.14407681,327.80134005)(806.96407959,327.62134644)
\curveto(806.79407716,327.44134041)(806.63907732,327.21634063)(806.49907959,326.94634644)
\curveto(806.46907749,326.89634095)(806.43907752,326.83134102)(806.40907959,326.75134644)
\curveto(806.37907758,326.68134117)(806.3540776,326.60134125)(806.33407959,326.51134644)
\curveto(806.31407764,326.42134143)(806.30907765,326.33634151)(806.31907959,326.25634644)
\curveto(806.32907763,326.17634167)(806.36407759,326.11634173)(806.42407959,326.07634644)
\curveto(806.50407745,326.01634183)(806.63907732,325.98634186)(806.82907959,325.98634644)
\curveto(807.02907693,325.99634185)(807.19907676,326.00134185)(807.33907959,326.00134644)
\lineto(809.61907959,326.00134644)
\curveto(809.76907419,326.00134185)(809.94907401,325.99634185)(810.15907959,325.98634644)
\curveto(810.36907359,325.98634186)(810.50907345,326.02634182)(810.57907959,326.10634644)
}
}
{
\newrgbcolor{curcolor}{0 0 0}
\pscustom[linestyle=none,fillstyle=solid,fillcolor=curcolor]
{
\newpath
\moveto(820.46572021,325.62634644)
\curveto(820.48571215,325.56634228)(820.49571214,325.47134238)(820.49572021,325.34134644)
\curveto(820.49571214,325.22134263)(820.49071215,325.13634271)(820.48072021,325.08634644)
\lineto(820.48072021,324.93634644)
\curveto(820.47071217,324.85634299)(820.46071218,324.78134307)(820.45072021,324.71134644)
\curveto(820.45071219,324.6513432)(820.44571219,324.58134327)(820.43572021,324.50134644)
\curveto(820.41571222,324.44134341)(820.40071224,324.38134347)(820.39072021,324.32134644)
\curveto(820.39071225,324.26134359)(820.38071226,324.20134365)(820.36072021,324.14134644)
\curveto(820.32071232,324.01134384)(820.28571235,323.88134397)(820.25572021,323.75134644)
\curveto(820.22571241,323.62134423)(820.18571245,323.50134435)(820.13572021,323.39134644)
\curveto(819.92571271,322.91134494)(819.64571299,322.50634534)(819.29572021,322.17634644)
\curveto(818.94571369,321.85634599)(818.51571412,321.61134624)(818.00572021,321.44134644)
\curveto(817.89571474,321.40134645)(817.77571486,321.37134648)(817.64572021,321.35134644)
\curveto(817.52571511,321.33134652)(817.40071524,321.31134654)(817.27072021,321.29134644)
\curveto(817.21071543,321.28134657)(817.14571549,321.27634657)(817.07572021,321.27634644)
\curveto(817.01571562,321.26634658)(816.95571568,321.26134659)(816.89572021,321.26134644)
\curveto(816.85571578,321.2513466)(816.79571584,321.2463466)(816.71572021,321.24634644)
\curveto(816.64571599,321.2463466)(816.59571604,321.2513466)(816.56572021,321.26134644)
\curveto(816.52571611,321.27134658)(816.48571615,321.27634657)(816.44572021,321.27634644)
\curveto(816.40571623,321.26634658)(816.37071627,321.26634658)(816.34072021,321.27634644)
\lineto(816.25072021,321.27634644)
\lineto(815.89072021,321.32134644)
\curveto(815.75071689,321.36134649)(815.61571702,321.40134645)(815.48572021,321.44134644)
\curveto(815.35571728,321.48134637)(815.23071741,321.52634632)(815.11072021,321.57634644)
\curveto(814.66071798,321.77634607)(814.29071835,322.03634581)(814.00072021,322.35634644)
\curveto(813.71071893,322.67634517)(813.47071917,323.06634478)(813.28072021,323.52634644)
\curveto(813.23071941,323.62634422)(813.19071945,323.72634412)(813.16072021,323.82634644)
\curveto(813.1407195,323.92634392)(813.12071952,324.03134382)(813.10072021,324.14134644)
\curveto(813.08071956,324.18134367)(813.07071957,324.21134364)(813.07072021,324.23134644)
\curveto(813.08071956,324.26134359)(813.08071956,324.29634355)(813.07072021,324.33634644)
\curveto(813.05071959,324.41634343)(813.0357196,324.49634335)(813.02572021,324.57634644)
\curveto(813.02571961,324.66634318)(813.01571962,324.7513431)(812.99572021,324.83134644)
\lineto(812.99572021,324.95134644)
\curveto(812.99571964,324.99134286)(812.99071965,325.03634281)(812.98072021,325.08634644)
\curveto(812.97071967,325.13634271)(812.96571967,325.22134263)(812.96572021,325.34134644)
\curveto(812.96571967,325.47134238)(812.97571966,325.56634228)(812.99572021,325.62634644)
\curveto(813.01571962,325.69634215)(813.02071962,325.76634208)(813.01072021,325.83634644)
\curveto(813.00071964,325.90634194)(813.00571963,325.97634187)(813.02572021,326.04634644)
\curveto(813.0357196,326.09634175)(813.0407196,326.13634171)(813.04072021,326.16634644)
\curveto(813.05071959,326.20634164)(813.06071958,326.2513416)(813.07072021,326.30134644)
\curveto(813.10071954,326.42134143)(813.12571951,326.54134131)(813.14572021,326.66134644)
\curveto(813.17571946,326.78134107)(813.21571942,326.89634095)(813.26572021,327.00634644)
\curveto(813.41571922,327.37634047)(813.59571904,327.70634014)(813.80572021,327.99634644)
\curveto(814.02571861,328.29633955)(814.29071835,328.5463393)(814.60072021,328.74634644)
\curveto(814.72071792,328.82633902)(814.84571779,328.89133896)(814.97572021,328.94134644)
\curveto(815.10571753,329.00133885)(815.2407174,329.06133879)(815.38072021,329.12134644)
\curveto(815.50071714,329.17133868)(815.63071701,329.20133865)(815.77072021,329.21134644)
\curveto(815.91071673,329.23133862)(816.05071659,329.26133859)(816.19072021,329.30134644)
\lineto(816.38572021,329.30134644)
\curveto(816.45571618,329.31133854)(816.52071612,329.32133853)(816.58072021,329.33134644)
\curveto(817.47071517,329.34133851)(818.21071443,329.15633869)(818.80072021,328.77634644)
\curveto(819.39071325,328.39633945)(819.81571282,327.90133995)(820.07572021,327.29134644)
\curveto(820.12571251,327.19134066)(820.16571247,327.09134076)(820.19572021,326.99134644)
\curveto(820.22571241,326.89134096)(820.26071238,326.78634106)(820.30072021,326.67634644)
\curveto(820.33071231,326.56634128)(820.35571228,326.4463414)(820.37572021,326.31634644)
\curveto(820.39571224,326.19634165)(820.42071222,326.07134178)(820.45072021,325.94134644)
\curveto(820.46071218,325.89134196)(820.46071218,325.83634201)(820.45072021,325.77634644)
\curveto(820.45071219,325.72634212)(820.45571218,325.67634217)(820.46572021,325.62634644)
\moveto(819.13072021,324.77134644)
\curveto(819.15071349,324.84134301)(819.15571348,324.92134293)(819.14572021,325.01134644)
\lineto(819.14572021,325.26634644)
\curveto(819.14571349,325.65634219)(819.11071353,325.98634186)(819.04072021,326.25634644)
\curveto(819.01071363,326.33634151)(818.98571365,326.41634143)(818.96572021,326.49634644)
\curveto(818.94571369,326.57634127)(818.92071372,326.6513412)(818.89072021,326.72134644)
\curveto(818.61071403,327.37134048)(818.16571447,327.82134003)(817.55572021,328.07134644)
\curveto(817.48571515,328.10133975)(817.41071523,328.12133973)(817.33072021,328.13134644)
\lineto(817.09072021,328.19134644)
\curveto(817.01071563,328.21133964)(816.92571571,328.22133963)(816.83572021,328.22134644)
\lineto(816.56572021,328.22134644)
\lineto(816.29572021,328.17634644)
\curveto(816.19571644,328.15633969)(816.10071654,328.13133972)(816.01072021,328.10134644)
\curveto(815.93071671,328.08133977)(815.85071679,328.0513398)(815.77072021,328.01134644)
\curveto(815.70071694,327.99133986)(815.635717,327.96133989)(815.57572021,327.92134644)
\curveto(815.51571712,327.88133997)(815.46071718,327.84134001)(815.41072021,327.80134644)
\curveto(815.17071747,327.63134022)(814.97571766,327.42634042)(814.82572021,327.18634644)
\curveto(814.67571796,326.9463409)(814.54571809,326.66634118)(814.43572021,326.34634644)
\curveto(814.40571823,326.2463416)(814.38571825,326.14134171)(814.37572021,326.03134644)
\curveto(814.36571827,325.93134192)(814.35071829,325.82634202)(814.33072021,325.71634644)
\curveto(814.32071832,325.67634217)(814.31571832,325.61134224)(814.31572021,325.52134644)
\curveto(814.30571833,325.49134236)(814.30071834,325.45634239)(814.30072021,325.41634644)
\curveto(814.31071833,325.37634247)(814.31571832,325.33134252)(814.31572021,325.28134644)
\lineto(814.31572021,324.98134644)
\curveto(814.31571832,324.88134297)(814.32571831,324.79134306)(814.34572021,324.71134644)
\lineto(814.37572021,324.53134644)
\curveto(814.39571824,324.43134342)(814.41071823,324.33134352)(814.42072021,324.23134644)
\curveto(814.4407182,324.14134371)(814.47071817,324.05634379)(814.51072021,323.97634644)
\curveto(814.61071803,323.73634411)(814.72571791,323.51134434)(814.85572021,323.30134644)
\curveto(814.99571764,323.09134476)(815.16571747,322.91634493)(815.36572021,322.77634644)
\curveto(815.41571722,322.7463451)(815.46071718,322.72134513)(815.50072021,322.70134644)
\curveto(815.5407171,322.68134517)(815.58571705,322.65634519)(815.63572021,322.62634644)
\curveto(815.71571692,322.57634527)(815.80071684,322.53134532)(815.89072021,322.49134644)
\curveto(815.99071665,322.46134539)(816.09571654,322.43134542)(816.20572021,322.40134644)
\curveto(816.25571638,322.38134547)(816.30071634,322.37134548)(816.34072021,322.37134644)
\curveto(816.39071625,322.38134547)(816.4407162,322.38134547)(816.49072021,322.37134644)
\curveto(816.52071612,322.36134549)(816.58071606,322.3513455)(816.67072021,322.34134644)
\curveto(816.77071587,322.33134552)(816.84571579,322.33634551)(816.89572021,322.35634644)
\curveto(816.9357157,322.36634548)(816.97571566,322.36634548)(817.01572021,322.35634644)
\curveto(817.05571558,322.35634549)(817.09571554,322.36634548)(817.13572021,322.38634644)
\curveto(817.21571542,322.40634544)(817.29571534,322.42134543)(817.37572021,322.43134644)
\curveto(817.45571518,322.4513454)(817.53071511,322.47634537)(817.60072021,322.50634644)
\curveto(817.9407147,322.6463452)(818.21571442,322.84134501)(818.42572021,323.09134644)
\curveto(818.635714,323.34134451)(818.81071383,323.63634421)(818.95072021,323.97634644)
\curveto(819.00071364,324.09634375)(819.03071361,324.22134363)(819.04072021,324.35134644)
\curveto(819.06071358,324.49134336)(819.09071355,324.63134322)(819.13072021,324.77134644)
}
}
{
\newrgbcolor{curcolor}{0 0 0}
\pscustom[linestyle=none,fillstyle=solid,fillcolor=curcolor]
{
\newpath
\moveto(775.96145996,291.38369019)
\curveto(776.14145819,291.38367949)(776.34145799,291.38367949)(776.56145996,291.38369019)
\curveto(776.78145755,291.39367948)(776.94645739,291.35867952)(777.05645996,291.27869019)
\curveto(777.1364572,291.21867966)(777.21145712,291.12867975)(777.28145996,291.00869019)
\curveto(777.35145698,290.89867998)(777.41645692,290.79868008)(777.47645996,290.70869019)
\curveto(777.60645673,290.50868037)(777.7364566,290.30368057)(777.86645996,290.09369019)
\curveto(778.00645633,289.89368098)(778.14145619,289.68868119)(778.27145996,289.47869019)
\lineto(778.48145996,289.14869019)
\curveto(778.56145577,289.04868183)(778.6364557,288.94368193)(778.70645996,288.83369019)
\curveto(779.00645533,288.35368252)(779.31145502,287.873683)(779.62145996,287.39369019)
\curveto(779.9314544,286.92368395)(780.24145409,286.44868443)(780.55145996,285.96869019)
\curveto(780.6314537,285.82868505)(780.71645362,285.69368518)(780.80645996,285.56369019)
\curveto(780.90645343,285.44368543)(780.99645334,285.31368556)(781.07645996,285.17369019)
\lineto(781.58645996,284.36369019)
\curveto(781.76645257,284.10368677)(781.94145239,283.84368703)(782.11145996,283.58369019)
\curveto(782.16145217,283.50368737)(782.22145211,283.40368747)(782.29145996,283.28369019)
\curveto(782.37145196,283.1736877)(782.46645187,283.11868776)(782.57645996,283.11869019)
\curveto(782.62645171,283.13868774)(782.65145168,283.15368772)(782.65145996,283.16369019)
\curveto(782.70145163,283.22368765)(782.72645161,283.30868757)(782.72645996,283.41869019)
\lineto(782.72645996,283.73369019)
\lineto(782.72645996,284.91869019)
\lineto(782.72645996,289.50869019)
\lineto(782.72645996,290.40869019)
\curveto(782.72645161,290.4786804)(782.72145161,290.55368032)(782.71145996,290.63369019)
\curveto(782.70145163,290.71368016)(782.70645163,290.78868009)(782.72645996,290.85869019)
\lineto(782.72645996,291.02369019)
\curveto(782.74645159,291.06367981)(782.75645158,291.10367977)(782.75645996,291.14369019)
\curveto(782.76645157,291.18367969)(782.78145155,291.21867966)(782.80145996,291.24869019)
\curveto(782.86145147,291.32867955)(782.95145138,291.36867951)(783.07145996,291.36869019)
\curveto(783.19145114,291.3786795)(783.32145101,291.38367949)(783.46145996,291.38369019)
\curveto(783.52145081,291.38367949)(783.58145075,291.38367949)(783.64145996,291.38369019)
\curveto(783.71145062,291.38367949)(783.77145056,291.3736795)(783.82145996,291.35369019)
\curveto(783.94145039,291.30367957)(784.00645033,291.21367966)(784.01645996,291.08369019)
\curveto(784.0364503,290.96367991)(784.04645029,290.81868006)(784.04645996,290.64869019)
\lineto(784.04645996,288.99869019)
\lineto(784.04645996,282.71369019)
\lineto(784.04645996,281.45369019)
\lineto(784.04645996,281.12369019)
\curveto(784.05645028,281.01368986)(784.0364503,280.92868995)(783.98645996,280.86869019)
\curveto(783.94645039,280.80869007)(783.89645044,280.76869011)(783.83645996,280.74869019)
\curveto(783.78645055,280.73869014)(783.72145061,280.72369015)(783.64145996,280.70369019)
\lineto(783.25145996,280.70369019)
\lineto(782.87645996,280.70369019)
\curveto(782.75645158,280.70369017)(782.65645168,280.72369015)(782.57645996,280.76369019)
\curveto(782.49645184,280.79369008)(782.4314519,280.84369003)(782.38145996,280.91369019)
\curveto(782.34145199,280.98368989)(782.29645204,281.05368982)(782.24645996,281.12369019)
\curveto(782.16645217,281.24368963)(782.08145225,281.36868951)(781.99145996,281.49869019)
\lineto(781.75145996,281.88869019)
\curveto(781.39145294,282.42868845)(781.0364533,282.96368791)(780.68645996,283.49369019)
\curveto(780.336454,284.02368685)(779.98645435,284.56368631)(779.63645996,285.11369019)
\curveto(779.44645489,285.41368546)(779.25145508,285.70868517)(779.05145996,285.99869019)
\curveto(778.86145547,286.28868459)(778.67145566,286.58368429)(778.48145996,286.88369019)
\curveto(778.15145618,287.40368347)(777.80645653,287.92868295)(777.44645996,288.45869019)
\curveto(777.40645693,288.52868235)(777.36645697,288.59368228)(777.32645996,288.65369019)
\curveto(777.28645705,288.72368215)(777.2314571,288.78368209)(777.16145996,288.83369019)
\curveto(777.14145719,288.84368203)(777.12145721,288.85868202)(777.10145996,288.87869019)
\curveto(777.08145725,288.89868198)(777.05645728,288.90368197)(777.02645996,288.89369019)
\curveto(776.96645737,288.873682)(776.9314574,288.83368204)(776.92145996,288.77369019)
\curveto(776.91145742,288.71368216)(776.89645744,288.65368222)(776.87645996,288.59369019)
\lineto(776.87645996,288.48869019)
\curveto(776.85645748,288.41868246)(776.85145748,288.33868254)(776.86145996,288.24869019)
\curveto(776.87145746,288.16868271)(776.87645746,288.08868279)(776.87645996,288.00869019)
\lineto(776.87645996,287.01869019)
\lineto(776.87645996,282.24869019)
\lineto(776.87645996,281.54369019)
\lineto(776.87645996,281.36369019)
\curveto(776.88645745,281.29368958)(776.88145745,281.23368964)(776.86145996,281.18369019)
\lineto(776.86145996,281.06369019)
\curveto(776.84145749,280.96368991)(776.82145751,280.89368998)(776.80145996,280.85369019)
\curveto(776.78145755,280.80369007)(776.74645759,280.76869011)(776.69645996,280.74869019)
\curveto(776.64645769,280.73869014)(776.59145774,280.72369015)(776.53145996,280.70369019)
\lineto(776.23145996,280.70369019)
\curveto(776.09145824,280.70369017)(775.96645837,280.70869017)(775.85645996,280.71869019)
\curveto(775.74645859,280.72869015)(775.66645867,280.7736901)(775.61645996,280.85369019)
\curveto(775.56645877,280.91368996)(775.54145879,280.99368988)(775.54145996,281.09369019)
\lineto(775.54145996,281.42369019)
\lineto(775.54145996,282.63869019)
\lineto(775.54145996,288.90869019)
\lineto(775.54145996,290.52869019)
\lineto(775.54145996,290.90369019)
\curveto(775.54145879,291.04367983)(775.56645877,291.15367972)(775.61645996,291.23369019)
\curveto(775.64645869,291.28367959)(775.70645863,291.32867955)(775.79645996,291.36869019)
\curveto(775.81645852,291.3786795)(775.84145849,291.3786795)(775.87145996,291.36869019)
\curveto(775.91145842,291.36867951)(775.94145839,291.3736795)(775.96145996,291.38369019)
}
}
{
\newrgbcolor{curcolor}{0 0 0}
\pscustom[linestyle=none,fillstyle=solid,fillcolor=curcolor]
{
\newpath
\moveto(793.22200684,284.90369019)
\curveto(793.24199878,284.84368603)(793.25199877,284.74868613)(793.25200684,284.61869019)
\curveto(793.25199877,284.49868638)(793.24699877,284.41368646)(793.23700684,284.36369019)
\lineto(793.23700684,284.21369019)
\curveto(793.22699879,284.13368674)(793.2169988,284.05868682)(793.20700684,283.98869019)
\curveto(793.20699881,283.92868695)(793.20199882,283.85868702)(793.19200684,283.77869019)
\curveto(793.17199885,283.71868716)(793.15699886,283.65868722)(793.14700684,283.59869019)
\curveto(793.14699887,283.53868734)(793.13699888,283.4786874)(793.11700684,283.41869019)
\curveto(793.07699894,283.28868759)(793.04199898,283.15868772)(793.01200684,283.02869019)
\curveto(792.98199904,282.89868798)(792.94199908,282.7786881)(792.89200684,282.66869019)
\curveto(792.68199934,282.18868869)(792.40199962,281.78368909)(792.05200684,281.45369019)
\curveto(791.70200032,281.13368974)(791.27200075,280.88868999)(790.76200684,280.71869019)
\curveto(790.65200137,280.6786902)(790.53200149,280.64869023)(790.40200684,280.62869019)
\curveto(790.28200174,280.60869027)(790.15700186,280.58869029)(790.02700684,280.56869019)
\curveto(789.96700205,280.55869032)(789.90200212,280.55369032)(789.83200684,280.55369019)
\curveto(789.77200225,280.54369033)(789.71200231,280.53869034)(789.65200684,280.53869019)
\curveto(789.61200241,280.52869035)(789.55200247,280.52369035)(789.47200684,280.52369019)
\curveto(789.40200262,280.52369035)(789.35200267,280.52869035)(789.32200684,280.53869019)
\curveto(789.28200274,280.54869033)(789.24200278,280.55369032)(789.20200684,280.55369019)
\curveto(789.16200286,280.54369033)(789.12700289,280.54369033)(789.09700684,280.55369019)
\lineto(789.00700684,280.55369019)
\lineto(788.64700684,280.59869019)
\curveto(788.50700351,280.63869024)(788.37200365,280.6786902)(788.24200684,280.71869019)
\curveto(788.11200391,280.75869012)(787.98700403,280.80369007)(787.86700684,280.85369019)
\curveto(787.4170046,281.05368982)(787.04700497,281.31368956)(786.75700684,281.63369019)
\curveto(786.46700555,281.95368892)(786.22700579,282.34368853)(786.03700684,282.80369019)
\curveto(785.98700603,282.90368797)(785.94700607,283.00368787)(785.91700684,283.10369019)
\curveto(785.89700612,283.20368767)(785.87700614,283.30868757)(785.85700684,283.41869019)
\curveto(785.83700618,283.45868742)(785.82700619,283.48868739)(785.82700684,283.50869019)
\curveto(785.83700618,283.53868734)(785.83700618,283.5736873)(785.82700684,283.61369019)
\curveto(785.80700621,283.69368718)(785.79200623,283.7736871)(785.78200684,283.85369019)
\curveto(785.78200624,283.94368693)(785.77200625,284.02868685)(785.75200684,284.10869019)
\lineto(785.75200684,284.22869019)
\curveto(785.75200627,284.26868661)(785.74700627,284.31368656)(785.73700684,284.36369019)
\curveto(785.72700629,284.41368646)(785.7220063,284.49868638)(785.72200684,284.61869019)
\curveto(785.7220063,284.74868613)(785.73200629,284.84368603)(785.75200684,284.90369019)
\curveto(785.77200625,284.9736859)(785.77700624,285.04368583)(785.76700684,285.11369019)
\curveto(785.75700626,285.18368569)(785.76200626,285.25368562)(785.78200684,285.32369019)
\curveto(785.79200623,285.3736855)(785.79700622,285.41368546)(785.79700684,285.44369019)
\curveto(785.80700621,285.48368539)(785.8170062,285.52868535)(785.82700684,285.57869019)
\curveto(785.85700616,285.69868518)(785.88200614,285.81868506)(785.90200684,285.93869019)
\curveto(785.93200609,286.05868482)(785.97200605,286.1736847)(786.02200684,286.28369019)
\curveto(786.17200585,286.65368422)(786.35200567,286.98368389)(786.56200684,287.27369019)
\curveto(786.78200524,287.5736833)(787.04700497,287.82368305)(787.35700684,288.02369019)
\curveto(787.47700454,288.10368277)(787.60200442,288.16868271)(787.73200684,288.21869019)
\curveto(787.86200416,288.2786826)(787.99700402,288.33868254)(788.13700684,288.39869019)
\curveto(788.25700376,288.44868243)(788.38700363,288.4786824)(788.52700684,288.48869019)
\curveto(788.66700335,288.50868237)(788.80700321,288.53868234)(788.94700684,288.57869019)
\lineto(789.14200684,288.57869019)
\curveto(789.21200281,288.58868229)(789.27700274,288.59868228)(789.33700684,288.60869019)
\curveto(790.22700179,288.61868226)(790.96700105,288.43368244)(791.55700684,288.05369019)
\curveto(792.14699987,287.6736832)(792.57199945,287.1786837)(792.83200684,286.56869019)
\curveto(792.88199914,286.46868441)(792.9219991,286.36868451)(792.95200684,286.26869019)
\curveto(792.98199904,286.16868471)(793.016999,286.06368481)(793.05700684,285.95369019)
\curveto(793.08699893,285.84368503)(793.11199891,285.72368515)(793.13200684,285.59369019)
\curveto(793.15199887,285.4736854)(793.17699884,285.34868553)(793.20700684,285.21869019)
\curveto(793.2169988,285.16868571)(793.2169988,285.11368576)(793.20700684,285.05369019)
\curveto(793.20699881,285.00368587)(793.21199881,284.95368592)(793.22200684,284.90369019)
\moveto(791.88700684,284.04869019)
\curveto(791.90700011,284.11868676)(791.91200011,284.19868668)(791.90200684,284.28869019)
\lineto(791.90200684,284.54369019)
\curveto(791.90200012,284.93368594)(791.86700015,285.26368561)(791.79700684,285.53369019)
\curveto(791.76700025,285.61368526)(791.74200028,285.69368518)(791.72200684,285.77369019)
\curveto(791.70200032,285.85368502)(791.67700034,285.92868495)(791.64700684,285.99869019)
\curveto(791.36700065,286.64868423)(790.9220011,287.09868378)(790.31200684,287.34869019)
\curveto(790.24200178,287.3786835)(790.16700185,287.39868348)(790.08700684,287.40869019)
\lineto(789.84700684,287.46869019)
\curveto(789.76700225,287.48868339)(789.68200234,287.49868338)(789.59200684,287.49869019)
\lineto(789.32200684,287.49869019)
\lineto(789.05200684,287.45369019)
\curveto(788.95200307,287.43368344)(788.85700316,287.40868347)(788.76700684,287.37869019)
\curveto(788.68700333,287.35868352)(788.60700341,287.32868355)(788.52700684,287.28869019)
\curveto(788.45700356,287.26868361)(788.39200363,287.23868364)(788.33200684,287.19869019)
\curveto(788.27200375,287.15868372)(788.2170038,287.11868376)(788.16700684,287.07869019)
\curveto(787.92700409,286.90868397)(787.73200429,286.70368417)(787.58200684,286.46369019)
\curveto(787.43200459,286.22368465)(787.30200472,285.94368493)(787.19200684,285.62369019)
\curveto(787.16200486,285.52368535)(787.14200488,285.41868546)(787.13200684,285.30869019)
\curveto(787.1220049,285.20868567)(787.10700491,285.10368577)(787.08700684,284.99369019)
\curveto(787.07700494,284.95368592)(787.07200495,284.88868599)(787.07200684,284.79869019)
\curveto(787.06200496,284.76868611)(787.05700496,284.73368614)(787.05700684,284.69369019)
\curveto(787.06700495,284.65368622)(787.07200495,284.60868627)(787.07200684,284.55869019)
\lineto(787.07200684,284.25869019)
\curveto(787.07200495,284.15868672)(787.08200494,284.06868681)(787.10200684,283.98869019)
\lineto(787.13200684,283.80869019)
\curveto(787.15200487,283.70868717)(787.16700485,283.60868727)(787.17700684,283.50869019)
\curveto(787.19700482,283.41868746)(787.22700479,283.33368754)(787.26700684,283.25369019)
\curveto(787.36700465,283.01368786)(787.48200454,282.78868809)(787.61200684,282.57869019)
\curveto(787.75200427,282.36868851)(787.9220041,282.19368868)(788.12200684,282.05369019)
\curveto(788.17200385,282.02368885)(788.2170038,281.99868888)(788.25700684,281.97869019)
\curveto(788.29700372,281.95868892)(788.34200368,281.93368894)(788.39200684,281.90369019)
\curveto(788.47200355,281.85368902)(788.55700346,281.80868907)(788.64700684,281.76869019)
\curveto(788.74700327,281.73868914)(788.85200317,281.70868917)(788.96200684,281.67869019)
\curveto(789.01200301,281.65868922)(789.05700296,281.64868923)(789.09700684,281.64869019)
\curveto(789.14700287,281.65868922)(789.19700282,281.65868922)(789.24700684,281.64869019)
\curveto(789.27700274,281.63868924)(789.33700268,281.62868925)(789.42700684,281.61869019)
\curveto(789.52700249,281.60868927)(789.60200242,281.61368926)(789.65200684,281.63369019)
\curveto(789.69200233,281.64368923)(789.73200229,281.64368923)(789.77200684,281.63369019)
\curveto(789.81200221,281.63368924)(789.85200217,281.64368923)(789.89200684,281.66369019)
\curveto(789.97200205,281.68368919)(790.05200197,281.69868918)(790.13200684,281.70869019)
\curveto(790.21200181,281.72868915)(790.28700173,281.75368912)(790.35700684,281.78369019)
\curveto(790.69700132,281.92368895)(790.97200105,282.11868876)(791.18200684,282.36869019)
\curveto(791.39200063,282.61868826)(791.56700045,282.91368796)(791.70700684,283.25369019)
\curveto(791.75700026,283.3736875)(791.78700023,283.49868738)(791.79700684,283.62869019)
\curveto(791.8170002,283.76868711)(791.84700017,283.90868697)(791.88700684,284.04869019)
}
}
{
\newrgbcolor{curcolor}{0 0 0}
\pscustom[linestyle=none,fillstyle=solid,fillcolor=curcolor]
{
\newpath
\moveto(798.40028809,288.60869019)
\curveto(798.7802831,288.61868226)(799.10028278,288.5786823)(799.36028809,288.48869019)
\curveto(799.63028225,288.39868248)(799.87528201,288.26868261)(800.09528809,288.09869019)
\curveto(800.17528171,288.04868283)(800.24028164,287.9786829)(800.29028809,287.88869019)
\curveto(800.35028153,287.80868307)(800.41528147,287.73368314)(800.48528809,287.66369019)
\curveto(800.50528138,287.64368323)(800.53528135,287.61868326)(800.57528809,287.58869019)
\curveto(800.61528127,287.55868332)(800.66528122,287.54868333)(800.72528809,287.55869019)
\curveto(800.82528106,287.58868329)(800.91028097,287.64868323)(800.98028809,287.73869019)
\curveto(801.06028082,287.83868304)(801.14028074,287.91368296)(801.22028809,287.96369019)
\curveto(801.36028052,288.0736828)(801.50528038,288.16868271)(801.65528809,288.24869019)
\curveto(801.80528008,288.33868254)(801.97027991,288.41368246)(802.15028809,288.47369019)
\curveto(802.23027965,288.50368237)(802.31527957,288.52368235)(802.40528809,288.53369019)
\curveto(802.50527938,288.55368232)(802.60027928,288.5736823)(802.69028809,288.59369019)
\curveto(802.74027914,288.60368227)(802.7852791,288.60868227)(802.82528809,288.60869019)
\lineto(802.97528809,288.60869019)
\curveto(803.02527886,288.62868225)(803.09527879,288.63368224)(803.18528809,288.62369019)
\curveto(803.27527861,288.62368225)(803.34027854,288.61868226)(803.38028809,288.60869019)
\curveto(803.43027845,288.59868228)(803.50527838,288.59368228)(803.60528809,288.59369019)
\curveto(803.69527819,288.5736823)(803.7802781,288.55368232)(803.86028809,288.53369019)
\curveto(803.95027793,288.52368235)(804.03527785,288.50368237)(804.11528809,288.47369019)
\curveto(804.16527772,288.45368242)(804.21027767,288.43868244)(804.25028809,288.42869019)
\curveto(804.30027758,288.42868245)(804.35027753,288.41868246)(804.40028809,288.39869019)
\curveto(804.90027698,288.1786827)(805.24527664,287.83868304)(805.43528809,287.37869019)
\curveto(805.47527641,287.29868358)(805.50527638,287.20868367)(805.52528809,287.10869019)
\curveto(805.54527634,287.01868386)(805.56527632,286.91868396)(805.58528809,286.80869019)
\curveto(805.60527628,286.7786841)(805.61027627,286.74368413)(805.60028809,286.70369019)
\curveto(805.60027628,286.6736842)(805.60527628,286.64368423)(805.61528809,286.61369019)
\lineto(805.61528809,286.47869019)
\curveto(805.62527626,286.43868444)(805.62527626,286.39368448)(805.61528809,286.34369019)
\curveto(805.61527627,286.29368458)(805.61527627,286.24368463)(805.61528809,286.19369019)
\lineto(805.61528809,285.60869019)
\lineto(805.61528809,284.64869019)
\lineto(805.61528809,281.79869019)
\curveto(805.61527627,281.63868924)(805.61527627,281.44868943)(805.61528809,281.22869019)
\curveto(805.62527626,281.00868987)(805.5852763,280.86369001)(805.49528809,280.79369019)
\curveto(805.45527643,280.76369011)(805.39027649,280.73869014)(805.30028809,280.71869019)
\curveto(805.21027667,280.70869017)(805.11527677,280.70369017)(805.01528809,280.70369019)
\curveto(804.91527697,280.70369017)(804.81527707,280.70869017)(804.71528809,280.71869019)
\curveto(804.62527726,280.72869015)(804.56027732,280.74869013)(804.52028809,280.77869019)
\curveto(804.46027742,280.80869007)(804.42027746,280.86869001)(804.40028809,280.95869019)
\curveto(804.3802775,281.01868986)(804.37527751,281.0786898)(804.38528809,281.13869019)
\curveto(804.39527749,281.20868967)(804.39027749,281.2736896)(804.37028809,281.33369019)
\curveto(804.36027752,281.38368949)(804.35527753,281.43868944)(804.35528809,281.49869019)
\curveto(804.36527752,281.56868931)(804.37027751,281.63368924)(804.37028809,281.69369019)
\lineto(804.37028809,282.36869019)
\lineto(804.37028809,285.23369019)
\curveto(804.37027751,285.56368531)(804.36027752,285.873685)(804.34028809,286.16369019)
\curveto(804.33027755,286.46368441)(804.26027762,286.71368416)(804.13028809,286.91369019)
\curveto(803.9802779,287.15368372)(803.75027813,287.32868355)(803.44028809,287.43869019)
\curveto(803.3802785,287.45868342)(803.31527857,287.46868341)(803.24528809,287.46869019)
\curveto(803.1852787,287.4786834)(803.12027876,287.49368338)(803.05028809,287.51369019)
\curveto(803.01027887,287.52368335)(802.94527894,287.52368335)(802.85528809,287.51369019)
\curveto(802.76527912,287.51368336)(802.70527918,287.50868337)(802.67528809,287.49869019)
\curveto(802.62527926,287.48868339)(802.57527931,287.48368339)(802.52528809,287.48369019)
\curveto(802.47527941,287.49368338)(802.42527946,287.48868339)(802.37528809,287.46869019)
\curveto(802.23527965,287.43868344)(802.10027978,287.39868348)(801.97028809,287.34869019)
\curveto(801.45028043,287.12868375)(801.10028078,286.74368413)(800.92028809,286.19369019)
\curveto(800.87028101,286.02368485)(800.84028104,285.82868505)(800.83028809,285.60869019)
\lineto(800.83028809,284.93369019)
\lineto(800.83028809,282.96869019)
\lineto(800.83028809,281.51369019)
\lineto(800.83028809,281.13869019)
\curveto(800.83028105,281.01868986)(800.80528108,280.92368995)(800.75528809,280.85369019)
\curveto(800.70528118,280.7736901)(800.62028126,280.72869015)(800.50028809,280.71869019)
\curveto(800.3802815,280.70869017)(800.25528163,280.70369017)(800.12528809,280.70369019)
\curveto(799.95528193,280.70369017)(799.83028205,280.72369015)(799.75028809,280.76369019)
\curveto(799.66028222,280.81369006)(799.60528228,280.89368998)(799.58528809,281.00369019)
\curveto(799.57528231,281.12368975)(799.57028231,281.25368962)(799.57028809,281.39369019)
\lineto(799.57028809,282.81869019)
\lineto(799.57028809,285.29369019)
\curveto(799.57028231,285.61368526)(799.56028232,285.90868497)(799.54028809,286.17869019)
\curveto(799.52028236,286.45868442)(799.45028243,286.69868418)(799.33028809,286.89869019)
\curveto(799.22028266,287.0786838)(799.09528279,287.20868367)(798.95528809,287.28869019)
\curveto(798.81528307,287.3786835)(798.62528326,287.44868343)(798.38528809,287.49869019)
\curveto(798.34528354,287.50868337)(798.30028358,287.51368336)(798.25028809,287.51369019)
\lineto(798.11528809,287.51369019)
\curveto(797.89528399,287.51368336)(797.70028418,287.48868339)(797.53028809,287.43869019)
\curveto(797.37028451,287.38868349)(797.22528466,287.32368355)(797.09528809,287.24369019)
\curveto(796.5852853,286.93368394)(796.24528564,286.46868441)(796.07528809,285.84869019)
\curveto(796.03528585,285.71868516)(796.01528587,285.56868531)(796.01528809,285.39869019)
\curveto(796.02528586,285.23868564)(796.03028585,285.0786858)(796.03028809,284.91869019)
\lineto(796.03028809,283.22369019)
\lineto(796.03028809,281.57369019)
\lineto(796.03028809,281.16869019)
\curveto(796.03028585,281.02868985)(796.00028588,280.91868996)(795.94028809,280.83869019)
\curveto(795.89028599,280.76869011)(795.81528607,280.72869015)(795.71528809,280.71869019)
\curveto(795.61528627,280.70869017)(795.51028637,280.70369017)(795.40028809,280.70369019)
\lineto(795.17528809,280.70369019)
\curveto(795.11528677,280.72369015)(795.05528683,280.73869014)(794.99528809,280.74869019)
\curveto(794.94528694,280.75869012)(794.90028698,280.78869009)(794.86028809,280.83869019)
\curveto(794.81028707,280.89868998)(794.7852871,280.9736899)(794.78528809,281.06369019)
\lineto(794.78528809,281.37869019)
\lineto(794.78528809,282.35369019)
\lineto(794.78528809,286.64369019)
\lineto(794.78528809,287.75369019)
\lineto(794.78528809,288.03869019)
\curveto(794.7852871,288.13868274)(794.80528708,288.21868266)(794.84528809,288.27869019)
\curveto(794.87528701,288.33868254)(794.92028696,288.3786825)(794.98028809,288.39869019)
\curveto(795.06028682,288.42868245)(795.1852867,288.44368243)(795.35528809,288.44369019)
\curveto(795.53528635,288.44368243)(795.66528622,288.42868245)(795.74528809,288.39869019)
\curveto(795.82528606,288.35868252)(795.880286,288.30868257)(795.91028809,288.24869019)
\curveto(795.93028595,288.19868268)(795.94028594,288.13868274)(795.94028809,288.06869019)
\curveto(795.95028593,287.99868288)(795.96028592,287.93368294)(795.97028809,287.87369019)
\curveto(795.9802859,287.81368306)(796.00028588,287.76368311)(796.03028809,287.72369019)
\curveto(796.06028582,287.68368319)(796.11028577,287.66368321)(796.18028809,287.66369019)
\curveto(796.20028568,287.68368319)(796.22028566,287.69368318)(796.24028809,287.69369019)
\curveto(796.27028561,287.69368318)(796.29528559,287.70368317)(796.31528809,287.72369019)
\curveto(796.37528551,287.7736831)(796.43028545,287.82368305)(796.48028809,287.87369019)
\lineto(796.66028809,288.02369019)
\curveto(796.880285,288.18368269)(797.13028475,288.32368255)(797.41028809,288.44369019)
\curveto(797.51028437,288.48368239)(797.61028427,288.50868237)(797.71028809,288.51869019)
\curveto(797.81028407,288.53868234)(797.91528397,288.56368231)(798.02528809,288.59369019)
\lineto(798.20528809,288.59369019)
\curveto(798.27528361,288.60368227)(798.34028354,288.60868227)(798.40028809,288.60869019)
}
}
{
\newrgbcolor{curcolor}{0 0 0}
\pscustom[linestyle=none,fillstyle=solid,fillcolor=curcolor]
{
\newpath
\moveto(814.93802246,284.73869019)
\curveto(814.94801411,284.68868619)(814.95301411,284.61868626)(814.95302246,284.52869019)
\curveto(814.95301411,284.44868643)(814.94801411,284.38368649)(814.93802246,284.33369019)
\curveto(814.93801412,284.29368658)(814.93301413,284.25368662)(814.92302246,284.21369019)
\lineto(814.92302246,284.09369019)
\curveto(814.90301416,284.01368686)(814.89301417,283.93368694)(814.89302246,283.85369019)
\curveto(814.89301417,283.7736871)(814.88301418,283.69368718)(814.86302246,283.61369019)
\curveto(814.85301421,283.5736873)(814.84801421,283.53368734)(814.84802246,283.49369019)
\curveto(814.84801421,283.46368741)(814.84301422,283.42868745)(814.83302246,283.38869019)
\curveto(814.80301426,283.2786876)(814.77301429,283.1736877)(814.74302246,283.07369019)
\curveto(814.72301434,282.9736879)(814.69301437,282.873688)(814.65302246,282.77369019)
\curveto(814.51301455,282.42368845)(814.34301472,282.10868877)(814.14302246,281.82869019)
\curveto(813.94301512,281.54868933)(813.69301537,281.30868957)(813.39302246,281.10869019)
\curveto(813.24301582,281.00868987)(813.09801596,280.92368995)(812.95802246,280.85369019)
\curveto(812.84801621,280.80369007)(812.73801632,280.76369011)(812.62802246,280.73369019)
\curveto(812.52801653,280.70369017)(812.42301664,280.6736902)(812.31302246,280.64369019)
\curveto(812.24301682,280.62369025)(812.17801688,280.61369026)(812.11802246,280.61369019)
\curveto(812.058017,280.60369027)(811.99801706,280.58869029)(811.93802246,280.56869019)
\lineto(811.78802246,280.56869019)
\curveto(811.73801732,280.54869033)(811.6630174,280.53869034)(811.56302246,280.53869019)
\curveto(811.4630176,280.52869035)(811.38301768,280.53369034)(811.32302246,280.55369019)
\lineto(811.17302246,280.55369019)
\curveto(811.13301793,280.56369031)(811.08801797,280.56869031)(811.03802246,280.56869019)
\curveto(810.99801806,280.56869031)(810.95301811,280.5736903)(810.90302246,280.58369019)
\curveto(810.75301831,280.62369025)(810.60301846,280.65869022)(810.45302246,280.68869019)
\curveto(810.31301875,280.71869016)(810.17301889,280.76369011)(810.03302246,280.82369019)
\curveto(809.83301923,280.90368997)(809.65301941,281.00368987)(809.49302246,281.12369019)
\lineto(809.31302246,281.27369019)
\curveto(809.25301981,281.33368954)(809.18301988,281.3736895)(809.10302246,281.39369019)
\curveto(809.04302002,281.40368947)(808.99302007,281.38868949)(808.95302246,281.34869019)
\curveto(808.92302014,281.31868956)(808.89802016,281.2736896)(808.87802246,281.21369019)
\curveto(808.86802019,281.15368972)(808.8580202,281.08868979)(808.84802246,281.01869019)
\curveto(808.84802021,280.95868992)(808.83802022,280.91368996)(808.81802246,280.88369019)
\curveto(808.77802028,280.83369004)(808.73302033,280.78869009)(808.68302246,280.74869019)
\curveto(808.63302043,280.72869015)(808.5630205,280.71369016)(808.47302246,280.70369019)
\lineto(808.20302246,280.70369019)
\curveto(808.11302095,280.70369017)(808.02802103,280.70869017)(807.94802246,280.71869019)
\curveto(807.86802119,280.73869014)(807.80802125,280.75869012)(807.76802246,280.77869019)
\curveto(807.74802131,280.79869008)(807.72802133,280.82369005)(807.70802246,280.85369019)
\lineto(807.64802246,280.94369019)
\curveto(807.61802144,281.02368985)(807.60302146,281.14368973)(807.60302246,281.30369019)
\curveto(807.61302145,281.46368941)(807.61802144,281.59868928)(807.61802246,281.70869019)
\lineto(807.61802246,290.51369019)
\curveto(807.61802144,290.63368024)(807.61302145,290.75868012)(807.60302246,290.88869019)
\curveto(807.60302146,291.02867985)(807.62802143,291.13867974)(807.67802246,291.21869019)
\curveto(807.71802134,291.2786796)(807.78302128,291.32867955)(807.87302246,291.36869019)
\curveto(807.89302117,291.36867951)(807.91802114,291.36867951)(807.94802246,291.36869019)
\curveto(807.97802108,291.3786795)(808.00302106,291.38367949)(808.02302246,291.38369019)
\curveto(808.1630209,291.39367948)(808.30802075,291.39367948)(808.45802246,291.38369019)
\curveto(808.61802044,291.38367949)(808.72802033,291.34367953)(808.78802246,291.26369019)
\curveto(808.83802022,291.18367969)(808.8630202,291.06867981)(808.86302246,290.91869019)
\lineto(808.86302246,290.51369019)
\lineto(808.86302246,288.75869019)
\lineto(808.86302246,288.50369019)
\lineto(808.86302246,288.21869019)
\curveto(808.87302019,288.12868275)(808.88302018,288.04368283)(808.89302246,287.96369019)
\curveto(808.91302015,287.89368298)(808.94302012,287.84368303)(808.98302246,287.81369019)
\curveto(809.02302004,287.78368309)(809.06801999,287.7786831)(809.11802246,287.79869019)
\curveto(809.16801989,287.81868306)(809.20801985,287.83868304)(809.23802246,287.85869019)
\curveto(809.28801977,287.89868298)(809.33301973,287.93868294)(809.37302246,287.97869019)
\lineto(809.52302246,288.09869019)
\curveto(809.59301947,288.14868273)(809.6630194,288.19368268)(809.73302246,288.23369019)
\lineto(809.97302246,288.35369019)
\curveto(810.15301891,288.44368243)(810.36801869,288.50868237)(810.61802246,288.54869019)
\curveto(810.86801819,288.58868229)(811.12301794,288.60868227)(811.38302246,288.60869019)
\curveto(811.64301742,288.60868227)(811.89801716,288.58368229)(812.14802246,288.53369019)
\curveto(812.39801666,288.49368238)(812.61801644,288.43368244)(812.80802246,288.35369019)
\curveto(813.20801585,288.18368269)(813.55301551,287.94868293)(813.84302246,287.64869019)
\curveto(814.13301493,287.34868353)(814.3630147,286.99868388)(814.53302246,286.59869019)
\curveto(814.58301448,286.48868439)(814.62301444,286.3786845)(814.65302246,286.26869019)
\curveto(814.69301437,286.16868471)(814.73301433,286.06368481)(814.77302246,285.95369019)
\curveto(814.80301426,285.84368503)(814.82301424,285.72868515)(814.83302246,285.60869019)
\lineto(814.89302246,285.27869019)
\curveto(814.90301416,285.24868563)(814.90801415,285.21368566)(814.90802246,285.17369019)
\curveto(814.90801415,285.14368573)(814.91301415,285.11368576)(814.92302246,285.08369019)
\curveto(814.94301412,285.02368585)(814.94301412,284.96368591)(814.92302246,284.90369019)
\curveto(814.91301415,284.85368602)(814.91801414,284.79868608)(814.93802246,284.73869019)
\moveto(813.60302246,284.34869019)
\curveto(813.62301544,284.39868648)(813.62801543,284.45868642)(813.61802246,284.52869019)
\curveto(813.60801545,284.59868628)(813.60301546,284.66368621)(813.60302246,284.72369019)
\curveto(813.60301546,284.89368598)(813.59301547,285.05368582)(813.57302246,285.20369019)
\curveto(813.5630155,285.35368552)(813.53301553,285.48868539)(813.48302246,285.60869019)
\curveto(813.45301561,285.70868517)(813.42801563,285.79868508)(813.40802246,285.87869019)
\curveto(813.38801567,285.95868492)(813.3580157,286.03868484)(813.31802246,286.11869019)
\curveto(813.20801585,286.36868451)(813.058016,286.59868428)(812.86802246,286.80869019)
\curveto(812.67801638,287.02868385)(812.4580166,287.19368368)(812.20802246,287.30369019)
\curveto(812.12801693,287.33368354)(812.04801701,287.35868352)(811.96802246,287.37869019)
\curveto(811.89801716,287.40868347)(811.82301724,287.43368344)(811.74302246,287.45369019)
\curveto(811.63301743,287.48368339)(811.52301754,287.49868338)(811.41302246,287.49869019)
\curveto(811.30301776,287.50868337)(811.18301788,287.51368336)(811.05302246,287.51369019)
\curveto(811.00301806,287.50368337)(810.9580181,287.49368338)(810.91802246,287.48369019)
\lineto(810.78302246,287.48369019)
\lineto(810.51302246,287.42369019)
\curveto(810.43301863,287.40368347)(810.35301871,287.3736835)(810.27302246,287.33369019)
\curveto(809.93301913,287.19368368)(809.6630194,286.98368389)(809.46302246,286.70369019)
\curveto(809.2630198,286.43368444)(809.10301996,286.11368476)(808.98302246,285.74369019)
\curveto(808.94302012,285.63368524)(808.91802014,285.52368535)(808.90802246,285.41369019)
\curveto(808.89802016,285.30368557)(808.87802018,285.18868569)(808.84802246,285.06869019)
\curveto(808.83802022,285.01868586)(808.83802022,284.9736859)(808.84802246,284.93369019)
\curveto(808.8580202,284.89368598)(808.85302021,284.84868603)(808.83302246,284.79869019)
\curveto(808.82302024,284.74868613)(808.81802024,284.6736862)(808.81802246,284.57369019)
\curveto(808.81802024,284.48368639)(808.82302024,284.41368646)(808.83302246,284.36369019)
\lineto(808.83302246,284.24369019)
\curveto(808.84302022,284.20368667)(808.84802021,284.16368671)(808.84802246,284.12369019)
\curveto(808.84802021,284.08368679)(808.85302021,284.04868683)(808.86302246,284.01869019)
\curveto(808.87302019,283.98868689)(808.87802018,283.95368692)(808.87802246,283.91369019)
\curveto(808.87802018,283.88368699)(808.88302018,283.85368702)(808.89302246,283.82369019)
\curveto(808.91302015,283.74368713)(808.92802013,283.66368721)(808.93802246,283.58369019)
\lineto(808.99802246,283.34369019)
\curveto(809.10801995,283.00368787)(809.2580198,282.71368816)(809.44802246,282.47369019)
\curveto(809.64801941,282.23368864)(809.89301917,282.03368884)(810.18302246,281.87369019)
\curveto(810.27301879,281.82368905)(810.36801869,281.78368909)(810.46802246,281.75369019)
\curveto(810.56801849,281.73368914)(810.67301839,281.70868917)(810.78302246,281.67869019)
\curveto(810.83301823,281.65868922)(810.87801818,281.64868923)(810.91802246,281.64869019)
\curveto(810.96801809,281.65868922)(811.01801804,281.65868922)(811.06802246,281.64869019)
\curveto(811.10801795,281.63868924)(811.15301791,281.63368924)(811.20302246,281.63369019)
\lineto(811.33802246,281.63369019)
\lineto(811.47302246,281.63369019)
\curveto(811.51301755,281.64368923)(811.54801751,281.64868923)(811.57802246,281.64869019)
\curveto(811.60801745,281.64868923)(811.64301742,281.65368922)(811.68302246,281.66369019)
\curveto(811.7630173,281.68368919)(811.83801722,281.69868918)(811.90802246,281.70869019)
\curveto(811.97801708,281.72868915)(812.05301701,281.75368912)(812.13302246,281.78369019)
\curveto(812.44301662,281.91368896)(812.69301637,282.08368879)(812.88302246,282.29369019)
\curveto(813.07301599,282.51368836)(813.23301583,282.7786881)(813.36302246,283.08869019)
\curveto(813.41301565,283.22868765)(813.44801561,283.36868751)(813.46802246,283.50869019)
\curveto(813.49801556,283.65868722)(813.53301553,283.80868707)(813.57302246,283.95869019)
\curveto(813.59301547,284.00868687)(813.59801546,284.05368682)(813.58802246,284.09369019)
\curveto(813.58801547,284.14368673)(813.59301547,284.19368668)(813.60302246,284.24369019)
\lineto(813.60302246,284.34869019)
}
}
{
\newrgbcolor{curcolor}{0 0 0}
\pscustom[linestyle=none,fillstyle=solid,fillcolor=curcolor]
{
\newpath
\moveto(820.06427246,288.60869019)
\curveto(820.29426767,288.60868227)(820.42426754,288.54868233)(820.45427246,288.42869019)
\curveto(820.48426748,288.31868256)(820.49926747,288.15368272)(820.49927246,287.93369019)
\lineto(820.49927246,287.64869019)
\curveto(820.49926747,287.55868332)(820.47426749,287.48368339)(820.42427246,287.42369019)
\curveto(820.3642676,287.34368353)(820.27926769,287.29868358)(820.16927246,287.28869019)
\curveto(820.05926791,287.28868359)(819.94926802,287.2736836)(819.83927246,287.24369019)
\curveto(819.69926827,287.21368366)(819.5642684,287.18368369)(819.43427246,287.15369019)
\curveto(819.31426865,287.12368375)(819.19926877,287.08368379)(819.08927246,287.03369019)
\curveto(818.79926917,286.90368397)(818.5642694,286.72368415)(818.38427246,286.49369019)
\curveto(818.20426976,286.2736846)(818.04926992,286.01868486)(817.91927246,285.72869019)
\curveto(817.87927009,285.61868526)(817.84927012,285.50368537)(817.82927246,285.38369019)
\curveto(817.80927016,285.2736856)(817.78427018,285.15868572)(817.75427246,285.03869019)
\curveto(817.74427022,284.98868589)(817.73927023,284.93868594)(817.73927246,284.88869019)
\curveto(817.74927022,284.83868604)(817.74927022,284.78868609)(817.73927246,284.73869019)
\curveto(817.70927026,284.61868626)(817.69427027,284.4786864)(817.69427246,284.31869019)
\curveto(817.70427026,284.16868671)(817.70927026,284.02368685)(817.70927246,283.88369019)
\lineto(817.70927246,282.03869019)
\lineto(817.70927246,281.69369019)
\curveto(817.70927026,281.5736893)(817.70427026,281.45868942)(817.69427246,281.34869019)
\curveto(817.68427028,281.23868964)(817.67927029,281.14368973)(817.67927246,281.06369019)
\curveto(817.68927028,280.98368989)(817.6692703,280.91368996)(817.61927246,280.85369019)
\curveto(817.5692704,280.78369009)(817.48927048,280.74369013)(817.37927246,280.73369019)
\curveto(817.27927069,280.72369015)(817.1692708,280.71869016)(817.04927246,280.71869019)
\lineto(816.77927246,280.71869019)
\curveto(816.72927124,280.73869014)(816.67927129,280.75369012)(816.62927246,280.76369019)
\curveto(816.58927138,280.78369009)(816.55927141,280.80869007)(816.53927246,280.83869019)
\curveto(816.48927148,280.90868997)(816.45927151,280.99368988)(816.44927246,281.09369019)
\lineto(816.44927246,281.42369019)
\lineto(816.44927246,282.57869019)
\lineto(816.44927246,286.73369019)
\lineto(816.44927246,287.76869019)
\lineto(816.44927246,288.06869019)
\curveto(816.45927151,288.16868271)(816.48927148,288.25368262)(816.53927246,288.32369019)
\curveto(816.5692714,288.36368251)(816.61927135,288.39368248)(816.68927246,288.41369019)
\curveto(816.7692712,288.43368244)(816.85427111,288.44368243)(816.94427246,288.44369019)
\curveto(817.03427093,288.45368242)(817.12427084,288.45368242)(817.21427246,288.44369019)
\curveto(817.30427066,288.43368244)(817.37427059,288.41868246)(817.42427246,288.39869019)
\curveto(817.50427046,288.36868251)(817.55427041,288.30868257)(817.57427246,288.21869019)
\curveto(817.60427036,288.13868274)(817.61927035,288.04868283)(817.61927246,287.94869019)
\lineto(817.61927246,287.64869019)
\curveto(817.61927035,287.54868333)(817.63927033,287.45868342)(817.67927246,287.37869019)
\curveto(817.68927028,287.35868352)(817.69927027,287.34368353)(817.70927246,287.33369019)
\lineto(817.75427246,287.28869019)
\curveto(817.8642701,287.28868359)(817.95427001,287.33368354)(818.02427246,287.42369019)
\curveto(818.09426987,287.52368335)(818.15426981,287.60368327)(818.20427246,287.66369019)
\lineto(818.29427246,287.75369019)
\curveto(818.38426958,287.86368301)(818.50926946,287.9786829)(818.66927246,288.09869019)
\curveto(818.82926914,288.21868266)(818.97926899,288.30868257)(819.11927246,288.36869019)
\curveto(819.20926876,288.41868246)(819.30426866,288.45368242)(819.40427246,288.47369019)
\curveto(819.50426846,288.50368237)(819.60926836,288.53368234)(819.71927246,288.56369019)
\curveto(819.77926819,288.5736823)(819.83926813,288.5786823)(819.89927246,288.57869019)
\curveto(819.95926801,288.58868229)(820.01426795,288.59868228)(820.06427246,288.60869019)
}
}
{
\newrgbcolor{curcolor}{0 0 0}
\pscustom[linestyle=none,fillstyle=solid,fillcolor=curcolor]
{
\newpath
\moveto(828.17903809,284.87369019)
\curveto(828.1990304,284.7736861)(828.1990304,284.65868622)(828.17903809,284.52869019)
\curveto(828.16903043,284.40868647)(828.13903046,284.32368655)(828.08903809,284.27369019)
\curveto(828.03903056,284.23368664)(827.96403064,284.20368667)(827.86403809,284.18369019)
\curveto(827.77403083,284.1736867)(827.66903093,284.16868671)(827.54903809,284.16869019)
\lineto(827.18903809,284.16869019)
\curveto(827.06903153,284.1786867)(826.96403164,284.18368669)(826.87403809,284.18369019)
\lineto(823.03403809,284.18369019)
\curveto(822.95403565,284.18368669)(822.87403573,284.1786867)(822.79403809,284.16869019)
\curveto(822.71403589,284.16868671)(822.64903595,284.15368672)(822.59903809,284.12369019)
\curveto(822.55903604,284.10368677)(822.51903608,284.06368681)(822.47903809,284.00369019)
\curveto(822.45903614,283.9736869)(822.43903616,283.92868695)(822.41903809,283.86869019)
\curveto(822.3990362,283.81868706)(822.3990362,283.76868711)(822.41903809,283.71869019)
\curveto(822.42903617,283.66868721)(822.43403617,283.62368725)(822.43403809,283.58369019)
\curveto(822.43403617,283.54368733)(822.43903616,283.50368737)(822.44903809,283.46369019)
\curveto(822.46903613,283.38368749)(822.48903611,283.29868758)(822.50903809,283.20869019)
\curveto(822.52903607,283.12868775)(822.55903604,283.04868783)(822.59903809,282.96869019)
\curveto(822.82903577,282.42868845)(823.20903539,282.04368883)(823.73903809,281.81369019)
\curveto(823.7990348,281.78368909)(823.86403474,281.75868912)(823.93403809,281.73869019)
\lineto(824.14403809,281.67869019)
\curveto(824.17403443,281.66868921)(824.22403438,281.66368921)(824.29403809,281.66369019)
\curveto(824.43403417,281.62368925)(824.61903398,281.60368927)(824.84903809,281.60369019)
\curveto(825.07903352,281.60368927)(825.26403334,281.62368925)(825.40403809,281.66369019)
\curveto(825.54403306,281.70368917)(825.66903293,281.74368913)(825.77903809,281.78369019)
\curveto(825.8990327,281.83368904)(826.00903259,281.89368898)(826.10903809,281.96369019)
\curveto(826.21903238,282.03368884)(826.31403229,282.11368876)(826.39403809,282.20369019)
\curveto(826.47403213,282.30368857)(826.54403206,282.40868847)(826.60403809,282.51869019)
\curveto(826.66403194,282.61868826)(826.71403189,282.72368815)(826.75403809,282.83369019)
\curveto(826.8040318,282.94368793)(826.88403172,283.02368785)(826.99403809,283.07369019)
\curveto(827.03403157,283.09368778)(827.0990315,283.10868777)(827.18903809,283.11869019)
\curveto(827.27903132,283.12868775)(827.36903123,283.12868775)(827.45903809,283.11869019)
\curveto(827.54903105,283.11868776)(827.63403097,283.11368776)(827.71403809,283.10369019)
\curveto(827.79403081,283.09368778)(827.84903075,283.0736878)(827.87903809,283.04369019)
\curveto(827.97903062,282.9736879)(828.0040306,282.85868802)(827.95403809,282.69869019)
\curveto(827.87403073,282.42868845)(827.76903083,282.18868869)(827.63903809,281.97869019)
\curveto(827.43903116,281.65868922)(827.20903139,281.39368948)(826.94903809,281.18369019)
\curveto(826.6990319,280.98368989)(826.37903222,280.81869006)(825.98903809,280.68869019)
\curveto(825.88903271,280.64869023)(825.78903281,280.62369025)(825.68903809,280.61369019)
\curveto(825.58903301,280.59369028)(825.48403312,280.5736903)(825.37403809,280.55369019)
\curveto(825.32403328,280.54369033)(825.27403333,280.53869034)(825.22403809,280.53869019)
\curveto(825.18403342,280.53869034)(825.13903346,280.53369034)(825.08903809,280.52369019)
\lineto(824.93903809,280.52369019)
\curveto(824.88903371,280.51369036)(824.82903377,280.50869037)(824.75903809,280.50869019)
\curveto(824.6990339,280.50869037)(824.64903395,280.51369036)(824.60903809,280.52369019)
\lineto(824.47403809,280.52369019)
\curveto(824.42403418,280.53369034)(824.37903422,280.53869034)(824.33903809,280.53869019)
\curveto(824.2990343,280.53869034)(824.25903434,280.54369033)(824.21903809,280.55369019)
\curveto(824.16903443,280.56369031)(824.11403449,280.5736903)(824.05403809,280.58369019)
\curveto(823.99403461,280.58369029)(823.93903466,280.58869029)(823.88903809,280.59869019)
\curveto(823.7990348,280.61869026)(823.70903489,280.64369023)(823.61903809,280.67369019)
\curveto(823.52903507,280.69369018)(823.44403516,280.71869016)(823.36403809,280.74869019)
\curveto(823.32403528,280.76869011)(823.28903531,280.7786901)(823.25903809,280.77869019)
\curveto(823.22903537,280.78869009)(823.19403541,280.80369007)(823.15403809,280.82369019)
\curveto(823.0040356,280.89368998)(822.84403576,280.9786899)(822.67403809,281.07869019)
\curveto(822.38403622,281.26868961)(822.13403647,281.49868938)(821.92403809,281.76869019)
\curveto(821.72403688,282.04868883)(821.55403705,282.35868852)(821.41403809,282.69869019)
\curveto(821.36403724,282.80868807)(821.32403728,282.92368795)(821.29403809,283.04369019)
\curveto(821.27403733,283.16368771)(821.24403736,283.28368759)(821.20403809,283.40369019)
\curveto(821.19403741,283.44368743)(821.18903741,283.4786874)(821.18903809,283.50869019)
\curveto(821.18903741,283.53868734)(821.18403742,283.5786873)(821.17403809,283.62869019)
\curveto(821.15403745,283.70868717)(821.13903746,283.79368708)(821.12903809,283.88369019)
\curveto(821.11903748,283.9736869)(821.1040375,284.06368681)(821.08403809,284.15369019)
\lineto(821.08403809,284.36369019)
\curveto(821.07403753,284.40368647)(821.06403754,284.45868642)(821.05403809,284.52869019)
\curveto(821.05403755,284.60868627)(821.05903754,284.6736862)(821.06903809,284.72369019)
\lineto(821.06903809,284.88869019)
\curveto(821.08903751,284.93868594)(821.09403751,284.98868589)(821.08403809,285.03869019)
\curveto(821.08403752,285.09868578)(821.08903751,285.15368572)(821.09903809,285.20369019)
\curveto(821.13903746,285.36368551)(821.16903743,285.52368535)(821.18903809,285.68369019)
\curveto(821.21903738,285.84368503)(821.26403734,285.99368488)(821.32403809,286.13369019)
\curveto(821.37403723,286.24368463)(821.41903718,286.35368452)(821.45903809,286.46369019)
\curveto(821.50903709,286.58368429)(821.56403704,286.69868418)(821.62403809,286.80869019)
\curveto(821.84403676,287.15868372)(822.09403651,287.45868342)(822.37403809,287.70869019)
\curveto(822.65403595,287.96868291)(822.9990356,288.18368269)(823.40903809,288.35369019)
\curveto(823.52903507,288.40368247)(823.64903495,288.43868244)(823.76903809,288.45869019)
\curveto(823.8990347,288.48868239)(824.03403457,288.51868236)(824.17403809,288.54869019)
\curveto(824.22403438,288.55868232)(824.26903433,288.56368231)(824.30903809,288.56369019)
\curveto(824.34903425,288.5736823)(824.39403421,288.5786823)(824.44403809,288.57869019)
\curveto(824.46403414,288.58868229)(824.48903411,288.58868229)(824.51903809,288.57869019)
\curveto(824.54903405,288.56868231)(824.57403403,288.5736823)(824.59403809,288.59369019)
\curveto(825.01403359,288.60368227)(825.37903322,288.55868232)(825.68903809,288.45869019)
\curveto(825.9990326,288.36868251)(826.27903232,288.24368263)(826.52903809,288.08369019)
\curveto(826.57903202,288.06368281)(826.61903198,288.03368284)(826.64903809,287.99369019)
\curveto(826.67903192,287.96368291)(826.71403189,287.93868294)(826.75403809,287.91869019)
\curveto(826.83403177,287.85868302)(826.91403169,287.78868309)(826.99403809,287.70869019)
\curveto(827.08403152,287.62868325)(827.15903144,287.54868333)(827.21903809,287.46869019)
\curveto(827.37903122,287.25868362)(827.51403109,287.05868382)(827.62403809,286.86869019)
\curveto(827.69403091,286.75868412)(827.74903085,286.63868424)(827.78903809,286.50869019)
\curveto(827.82903077,286.3786845)(827.87403073,286.24868463)(827.92403809,286.11869019)
\curveto(827.97403063,285.98868489)(828.00903059,285.85368502)(828.02903809,285.71369019)
\curveto(828.05903054,285.5736853)(828.09403051,285.43368544)(828.13403809,285.29369019)
\curveto(828.14403046,285.22368565)(828.14903045,285.15368572)(828.14903809,285.08369019)
\lineto(828.17903809,284.87369019)
\moveto(826.72403809,285.38369019)
\curveto(826.75403185,285.42368545)(826.77903182,285.4736854)(826.79903809,285.53369019)
\curveto(826.81903178,285.60368527)(826.81903178,285.6736852)(826.79903809,285.74369019)
\curveto(826.73903186,285.96368491)(826.65403195,286.16868471)(826.54403809,286.35869019)
\curveto(826.4040322,286.58868429)(826.24903235,286.78368409)(826.07903809,286.94369019)
\curveto(825.90903269,287.10368377)(825.68903291,287.23868364)(825.41903809,287.34869019)
\curveto(825.34903325,287.36868351)(825.27903332,287.38368349)(825.20903809,287.39369019)
\curveto(825.13903346,287.41368346)(825.06403354,287.43368344)(824.98403809,287.45369019)
\curveto(824.9040337,287.4736834)(824.81903378,287.48368339)(824.72903809,287.48369019)
\lineto(824.47403809,287.48369019)
\curveto(824.44403416,287.46368341)(824.40903419,287.45368342)(824.36903809,287.45369019)
\curveto(824.32903427,287.46368341)(824.29403431,287.46368341)(824.26403809,287.45369019)
\lineto(824.02403809,287.39369019)
\curveto(823.95403465,287.38368349)(823.88403472,287.36868351)(823.81403809,287.34869019)
\curveto(823.52403508,287.22868365)(823.28903531,287.0786838)(823.10903809,286.89869019)
\curveto(822.93903566,286.71868416)(822.78403582,286.49368438)(822.64403809,286.22369019)
\curveto(822.61403599,286.1736847)(822.58403602,286.10868477)(822.55403809,286.02869019)
\curveto(822.52403608,285.95868492)(822.4990361,285.878685)(822.47903809,285.78869019)
\curveto(822.45903614,285.69868518)(822.45403615,285.61368526)(822.46403809,285.53369019)
\curveto(822.47403613,285.45368542)(822.50903609,285.39368548)(822.56903809,285.35369019)
\curveto(822.64903595,285.29368558)(822.78403582,285.26368561)(822.97403809,285.26369019)
\curveto(823.17403543,285.2736856)(823.34403526,285.2786856)(823.48403809,285.27869019)
\lineto(825.76403809,285.27869019)
\curveto(825.91403269,285.2786856)(826.09403251,285.2736856)(826.30403809,285.26369019)
\curveto(826.51403209,285.26368561)(826.65403195,285.30368557)(826.72403809,285.38369019)
}
}
{
\newrgbcolor{curcolor}{0 0 0}
\pscustom[linestyle=none,fillstyle=solid,fillcolor=curcolor]
{
}
}
{
\newrgbcolor{curcolor}{0 0 0}
\pscustom[linestyle=none,fillstyle=solid,fillcolor=curcolor]
{
\newpath
\moveto(840.60583496,281.51369019)
\lineto(840.60583496,281.12369019)
\curveto(840.60582709,281.00368987)(840.58082711,280.90368997)(840.53083496,280.82369019)
\curveto(840.48082721,280.75369012)(840.3958273,280.71369016)(840.27583496,280.70369019)
\lineto(839.93083496,280.70369019)
\curveto(839.87082782,280.70369017)(839.81082788,280.69869018)(839.75083496,280.68869019)
\curveto(839.70082799,280.68869019)(839.65582804,280.69869018)(839.61583496,280.71869019)
\curveto(839.52582817,280.73869014)(839.46582823,280.7786901)(839.43583496,280.83869019)
\curveto(839.3958283,280.88868999)(839.37082832,280.94868993)(839.36083496,281.01869019)
\curveto(839.36082833,281.08868979)(839.34582835,281.15868972)(839.31583496,281.22869019)
\curveto(839.30582839,281.24868963)(839.2908284,281.26368961)(839.27083496,281.27369019)
\curveto(839.26082843,281.29368958)(839.24582845,281.31368956)(839.22583496,281.33369019)
\curveto(839.12582857,281.34368953)(839.04582865,281.32368955)(838.98583496,281.27369019)
\curveto(838.93582876,281.22368965)(838.88082881,281.1736897)(838.82083496,281.12369019)
\curveto(838.62082907,280.9736899)(838.42082927,280.85869002)(838.22083496,280.77869019)
\curveto(838.04082965,280.69869018)(837.83082986,280.63869024)(837.59083496,280.59869019)
\curveto(837.36083033,280.55869032)(837.12083057,280.53869034)(836.87083496,280.53869019)
\curveto(836.63083106,280.52869035)(836.3908313,280.54369033)(836.15083496,280.58369019)
\curveto(835.91083178,280.61369026)(835.70083199,280.66869021)(835.52083496,280.74869019)
\curveto(835.00083269,280.96868991)(834.58083311,281.26368961)(834.26083496,281.63369019)
\curveto(833.94083375,282.01368886)(833.690834,282.48368839)(833.51083496,283.04369019)
\curveto(833.47083422,283.13368774)(833.44083425,283.22368765)(833.42083496,283.31369019)
\curveto(833.41083428,283.41368746)(833.3908343,283.51368736)(833.36083496,283.61369019)
\curveto(833.35083434,283.66368721)(833.34583435,283.71368716)(833.34583496,283.76369019)
\curveto(833.34583435,283.81368706)(833.34083435,283.86368701)(833.33083496,283.91369019)
\curveto(833.31083438,283.96368691)(833.30083439,284.01368686)(833.30083496,284.06369019)
\curveto(833.31083438,284.12368675)(833.31083438,284.1786867)(833.30083496,284.22869019)
\lineto(833.30083496,284.37869019)
\curveto(833.28083441,284.42868645)(833.27083442,284.49368638)(833.27083496,284.57369019)
\curveto(833.27083442,284.65368622)(833.28083441,284.71868616)(833.30083496,284.76869019)
\lineto(833.30083496,284.93369019)
\curveto(833.32083437,285.00368587)(833.32583437,285.0736858)(833.31583496,285.14369019)
\curveto(833.31583438,285.22368565)(833.32583437,285.29868558)(833.34583496,285.36869019)
\curveto(833.35583434,285.41868546)(833.36083433,285.46368541)(833.36083496,285.50369019)
\curveto(833.36083433,285.54368533)(833.36583433,285.58868529)(833.37583496,285.63869019)
\curveto(833.40583429,285.73868514)(833.43083426,285.83368504)(833.45083496,285.92369019)
\curveto(833.47083422,286.02368485)(833.4958342,286.11868476)(833.52583496,286.20869019)
\curveto(833.65583404,286.58868429)(833.82083387,286.92868395)(834.02083496,287.22869019)
\curveto(834.23083346,287.53868334)(834.48083321,287.79368308)(834.77083496,287.99369019)
\curveto(834.94083275,288.11368276)(835.11583258,288.21368266)(835.29583496,288.29369019)
\curveto(835.48583221,288.3736825)(835.690832,288.44368243)(835.91083496,288.50369019)
\curveto(835.98083171,288.51368236)(836.04583165,288.52368235)(836.10583496,288.53369019)
\curveto(836.17583152,288.54368233)(836.24583145,288.55868232)(836.31583496,288.57869019)
\lineto(836.46583496,288.57869019)
\curveto(836.54583115,288.59868228)(836.66083103,288.60868227)(836.81083496,288.60869019)
\curveto(836.97083072,288.60868227)(837.0908306,288.59868228)(837.17083496,288.57869019)
\curveto(837.21083048,288.56868231)(837.26583043,288.56368231)(837.33583496,288.56369019)
\curveto(837.44583025,288.53368234)(837.55583014,288.50868237)(837.66583496,288.48869019)
\curveto(837.77582992,288.4786824)(837.88082981,288.44868243)(837.98083496,288.39869019)
\curveto(838.13082956,288.33868254)(838.27082942,288.2736826)(838.40083496,288.20369019)
\curveto(838.54082915,288.13368274)(838.67082902,288.05368282)(838.79083496,287.96369019)
\curveto(838.85082884,287.91368296)(838.91082878,287.85868302)(838.97083496,287.79869019)
\curveto(839.04082865,287.74868313)(839.13082856,287.73368314)(839.24083496,287.75369019)
\curveto(839.26082843,287.78368309)(839.27582842,287.80868307)(839.28583496,287.82869019)
\curveto(839.30582839,287.84868303)(839.32082837,287.878683)(839.33083496,287.91869019)
\curveto(839.36082833,288.00868287)(839.37082832,288.12368275)(839.36083496,288.26369019)
\lineto(839.36083496,288.63869019)
\lineto(839.36083496,290.36369019)
\lineto(839.36083496,290.82869019)
\curveto(839.36082833,291.00867987)(839.38582831,291.13867974)(839.43583496,291.21869019)
\curveto(839.47582822,291.28867959)(839.53582816,291.33367954)(839.61583496,291.35369019)
\curveto(839.63582806,291.35367952)(839.66082803,291.35367952)(839.69083496,291.35369019)
\curveto(839.72082797,291.36367951)(839.74582795,291.36867951)(839.76583496,291.36869019)
\curveto(839.90582779,291.3786795)(840.05082764,291.3786795)(840.20083496,291.36869019)
\curveto(840.36082733,291.36867951)(840.47082722,291.32867955)(840.53083496,291.24869019)
\curveto(840.58082711,291.16867971)(840.60582709,291.06867981)(840.60583496,290.94869019)
\lineto(840.60583496,290.57369019)
\lineto(840.60583496,281.51369019)
\moveto(839.39083496,284.34869019)
\curveto(839.41082828,284.39868648)(839.42082827,284.46368641)(839.42083496,284.54369019)
\curveto(839.42082827,284.63368624)(839.41082828,284.70368617)(839.39083496,284.75369019)
\lineto(839.39083496,284.97869019)
\curveto(839.37082832,285.06868581)(839.35582834,285.15868572)(839.34583496,285.24869019)
\curveto(839.33582836,285.34868553)(839.31582838,285.43868544)(839.28583496,285.51869019)
\curveto(839.26582843,285.59868528)(839.24582845,285.6736852)(839.22583496,285.74369019)
\curveto(839.21582848,285.81368506)(839.1958285,285.88368499)(839.16583496,285.95369019)
\curveto(839.04582865,286.25368462)(838.8908288,286.51868436)(838.70083496,286.74869019)
\curveto(838.51082918,286.9786839)(838.27082942,287.15868372)(837.98083496,287.28869019)
\curveto(837.88082981,287.33868354)(837.77582992,287.3736835)(837.66583496,287.39369019)
\curveto(837.56583013,287.42368345)(837.45583024,287.44868343)(837.33583496,287.46869019)
\curveto(837.25583044,287.48868339)(837.16583053,287.49868338)(837.06583496,287.49869019)
\lineto(836.79583496,287.49869019)
\curveto(836.74583095,287.48868339)(836.70083099,287.4786834)(836.66083496,287.46869019)
\lineto(836.52583496,287.46869019)
\curveto(836.44583125,287.44868343)(836.36083133,287.42868345)(836.27083496,287.40869019)
\curveto(836.1908315,287.38868349)(836.11083158,287.36368351)(836.03083496,287.33369019)
\curveto(835.71083198,287.19368368)(835.45083224,286.98868389)(835.25083496,286.71869019)
\curveto(835.06083263,286.45868442)(834.90583279,286.15368472)(834.78583496,285.80369019)
\curveto(834.74583295,285.69368518)(834.71583298,285.5786853)(834.69583496,285.45869019)
\curveto(834.68583301,285.34868553)(834.67083302,285.23868564)(834.65083496,285.12869019)
\curveto(834.65083304,285.08868579)(834.64583305,285.04868583)(834.63583496,285.00869019)
\lineto(834.63583496,284.90369019)
\curveto(834.61583308,284.85368602)(834.60583309,284.79868608)(834.60583496,284.73869019)
\curveto(834.61583308,284.6786862)(834.62083307,284.62368625)(834.62083496,284.57369019)
\lineto(834.62083496,284.24369019)
\curveto(834.62083307,284.14368673)(834.63083306,284.04868683)(834.65083496,283.95869019)
\curveto(834.66083303,283.92868695)(834.66583303,283.878687)(834.66583496,283.80869019)
\curveto(834.68583301,283.73868714)(834.70083299,283.66868721)(834.71083496,283.59869019)
\lineto(834.77083496,283.38869019)
\curveto(834.88083281,283.03868784)(835.03083266,282.73868814)(835.22083496,282.48869019)
\curveto(835.41083228,282.23868864)(835.65083204,282.03368884)(835.94083496,281.87369019)
\curveto(836.03083166,281.82368905)(836.12083157,281.78368909)(836.21083496,281.75369019)
\curveto(836.30083139,281.72368915)(836.40083129,281.69368918)(836.51083496,281.66369019)
\curveto(836.56083113,281.64368923)(836.61083108,281.63868924)(836.66083496,281.64869019)
\curveto(836.72083097,281.65868922)(836.77583092,281.65368922)(836.82583496,281.63369019)
\curveto(836.86583083,281.62368925)(836.90583079,281.61868926)(836.94583496,281.61869019)
\lineto(837.08083496,281.61869019)
\lineto(837.21583496,281.61869019)
\curveto(837.24583045,281.62868925)(837.2958304,281.63368924)(837.36583496,281.63369019)
\curveto(837.44583025,281.65368922)(837.52583017,281.66868921)(837.60583496,281.67869019)
\curveto(837.68583001,281.69868918)(837.76082993,281.72368915)(837.83083496,281.75369019)
\curveto(838.16082953,281.89368898)(838.42582927,282.06868881)(838.62583496,282.27869019)
\curveto(838.83582886,282.49868838)(839.01082868,282.7736881)(839.15083496,283.10369019)
\curveto(839.20082849,283.21368766)(839.23582846,283.32368755)(839.25583496,283.43369019)
\curveto(839.27582842,283.54368733)(839.30082839,283.65368722)(839.33083496,283.76369019)
\curveto(839.35082834,283.80368707)(839.36082833,283.83868704)(839.36083496,283.86869019)
\curveto(839.36082833,283.90868697)(839.36582833,283.94868693)(839.37583496,283.98869019)
\curveto(839.38582831,284.04868683)(839.38582831,284.10868677)(839.37583496,284.16869019)
\curveto(839.37582832,284.22868665)(839.38082831,284.28868659)(839.39083496,284.34869019)
}
}
{
\newrgbcolor{curcolor}{0 0 0}
\pscustom[linestyle=none,fillstyle=solid,fillcolor=curcolor]
{
\newpath
\moveto(849.30208496,284.87369019)
\curveto(849.32207728,284.7736861)(849.32207728,284.65868622)(849.30208496,284.52869019)
\curveto(849.29207731,284.40868647)(849.26207734,284.32368655)(849.21208496,284.27369019)
\curveto(849.16207744,284.23368664)(849.08707751,284.20368667)(848.98708496,284.18369019)
\curveto(848.8970777,284.1736867)(848.79207781,284.16868671)(848.67208496,284.16869019)
\lineto(848.31208496,284.16869019)
\curveto(848.19207841,284.1786867)(848.08707851,284.18368669)(847.99708496,284.18369019)
\lineto(844.15708496,284.18369019)
\curveto(844.07708252,284.18368669)(843.9970826,284.1786867)(843.91708496,284.16869019)
\curveto(843.83708276,284.16868671)(843.77208283,284.15368672)(843.72208496,284.12369019)
\curveto(843.68208292,284.10368677)(843.64208296,284.06368681)(843.60208496,284.00369019)
\curveto(843.58208302,283.9736869)(843.56208304,283.92868695)(843.54208496,283.86869019)
\curveto(843.52208308,283.81868706)(843.52208308,283.76868711)(843.54208496,283.71869019)
\curveto(843.55208305,283.66868721)(843.55708304,283.62368725)(843.55708496,283.58369019)
\curveto(843.55708304,283.54368733)(843.56208304,283.50368737)(843.57208496,283.46369019)
\curveto(843.59208301,283.38368749)(843.61208299,283.29868758)(843.63208496,283.20869019)
\curveto(843.65208295,283.12868775)(843.68208292,283.04868783)(843.72208496,282.96869019)
\curveto(843.95208265,282.42868845)(844.33208227,282.04368883)(844.86208496,281.81369019)
\curveto(844.92208168,281.78368909)(844.98708161,281.75868912)(845.05708496,281.73869019)
\lineto(845.26708496,281.67869019)
\curveto(845.2970813,281.66868921)(845.34708125,281.66368921)(845.41708496,281.66369019)
\curveto(845.55708104,281.62368925)(845.74208086,281.60368927)(845.97208496,281.60369019)
\curveto(846.2020804,281.60368927)(846.38708021,281.62368925)(846.52708496,281.66369019)
\curveto(846.66707993,281.70368917)(846.79207981,281.74368913)(846.90208496,281.78369019)
\curveto(847.02207958,281.83368904)(847.13207947,281.89368898)(847.23208496,281.96369019)
\curveto(847.34207926,282.03368884)(847.43707916,282.11368876)(847.51708496,282.20369019)
\curveto(847.597079,282.30368857)(847.66707893,282.40868847)(847.72708496,282.51869019)
\curveto(847.78707881,282.61868826)(847.83707876,282.72368815)(847.87708496,282.83369019)
\curveto(847.92707867,282.94368793)(848.00707859,283.02368785)(848.11708496,283.07369019)
\curveto(848.15707844,283.09368778)(848.22207838,283.10868777)(848.31208496,283.11869019)
\curveto(848.4020782,283.12868775)(848.49207811,283.12868775)(848.58208496,283.11869019)
\curveto(848.67207793,283.11868776)(848.75707784,283.11368776)(848.83708496,283.10369019)
\curveto(848.91707768,283.09368778)(848.97207763,283.0736878)(849.00208496,283.04369019)
\curveto(849.1020775,282.9736879)(849.12707747,282.85868802)(849.07708496,282.69869019)
\curveto(848.9970776,282.42868845)(848.89207771,282.18868869)(848.76208496,281.97869019)
\curveto(848.56207804,281.65868922)(848.33207827,281.39368948)(848.07208496,281.18369019)
\curveto(847.82207878,280.98368989)(847.5020791,280.81869006)(847.11208496,280.68869019)
\curveto(847.01207959,280.64869023)(846.91207969,280.62369025)(846.81208496,280.61369019)
\curveto(846.71207989,280.59369028)(846.60707999,280.5736903)(846.49708496,280.55369019)
\curveto(846.44708015,280.54369033)(846.3970802,280.53869034)(846.34708496,280.53869019)
\curveto(846.30708029,280.53869034)(846.26208034,280.53369034)(846.21208496,280.52369019)
\lineto(846.06208496,280.52369019)
\curveto(846.01208059,280.51369036)(845.95208065,280.50869037)(845.88208496,280.50869019)
\curveto(845.82208078,280.50869037)(845.77208083,280.51369036)(845.73208496,280.52369019)
\lineto(845.59708496,280.52369019)
\curveto(845.54708105,280.53369034)(845.5020811,280.53869034)(845.46208496,280.53869019)
\curveto(845.42208118,280.53869034)(845.38208122,280.54369033)(845.34208496,280.55369019)
\curveto(845.29208131,280.56369031)(845.23708136,280.5736903)(845.17708496,280.58369019)
\curveto(845.11708148,280.58369029)(845.06208154,280.58869029)(845.01208496,280.59869019)
\curveto(844.92208168,280.61869026)(844.83208177,280.64369023)(844.74208496,280.67369019)
\curveto(844.65208195,280.69369018)(844.56708203,280.71869016)(844.48708496,280.74869019)
\curveto(844.44708215,280.76869011)(844.41208219,280.7786901)(844.38208496,280.77869019)
\curveto(844.35208225,280.78869009)(844.31708228,280.80369007)(844.27708496,280.82369019)
\curveto(844.12708247,280.89368998)(843.96708263,280.9786899)(843.79708496,281.07869019)
\curveto(843.50708309,281.26868961)(843.25708334,281.49868938)(843.04708496,281.76869019)
\curveto(842.84708375,282.04868883)(842.67708392,282.35868852)(842.53708496,282.69869019)
\curveto(842.48708411,282.80868807)(842.44708415,282.92368795)(842.41708496,283.04369019)
\curveto(842.3970842,283.16368771)(842.36708423,283.28368759)(842.32708496,283.40369019)
\curveto(842.31708428,283.44368743)(842.31208429,283.4786874)(842.31208496,283.50869019)
\curveto(842.31208429,283.53868734)(842.30708429,283.5786873)(842.29708496,283.62869019)
\curveto(842.27708432,283.70868717)(842.26208434,283.79368708)(842.25208496,283.88369019)
\curveto(842.24208436,283.9736869)(842.22708437,284.06368681)(842.20708496,284.15369019)
\lineto(842.20708496,284.36369019)
\curveto(842.1970844,284.40368647)(842.18708441,284.45868642)(842.17708496,284.52869019)
\curveto(842.17708442,284.60868627)(842.18208442,284.6736862)(842.19208496,284.72369019)
\lineto(842.19208496,284.88869019)
\curveto(842.21208439,284.93868594)(842.21708438,284.98868589)(842.20708496,285.03869019)
\curveto(842.20708439,285.09868578)(842.21208439,285.15368572)(842.22208496,285.20369019)
\curveto(842.26208434,285.36368551)(842.29208431,285.52368535)(842.31208496,285.68369019)
\curveto(842.34208426,285.84368503)(842.38708421,285.99368488)(842.44708496,286.13369019)
\curveto(842.4970841,286.24368463)(842.54208406,286.35368452)(842.58208496,286.46369019)
\curveto(842.63208397,286.58368429)(842.68708391,286.69868418)(842.74708496,286.80869019)
\curveto(842.96708363,287.15868372)(843.21708338,287.45868342)(843.49708496,287.70869019)
\curveto(843.77708282,287.96868291)(844.12208248,288.18368269)(844.53208496,288.35369019)
\curveto(844.65208195,288.40368247)(844.77208183,288.43868244)(844.89208496,288.45869019)
\curveto(845.02208158,288.48868239)(845.15708144,288.51868236)(845.29708496,288.54869019)
\curveto(845.34708125,288.55868232)(845.39208121,288.56368231)(845.43208496,288.56369019)
\curveto(845.47208113,288.5736823)(845.51708108,288.5786823)(845.56708496,288.57869019)
\curveto(845.58708101,288.58868229)(845.61208099,288.58868229)(845.64208496,288.57869019)
\curveto(845.67208093,288.56868231)(845.6970809,288.5736823)(845.71708496,288.59369019)
\curveto(846.13708046,288.60368227)(846.5020801,288.55868232)(846.81208496,288.45869019)
\curveto(847.12207948,288.36868251)(847.4020792,288.24368263)(847.65208496,288.08369019)
\curveto(847.7020789,288.06368281)(847.74207886,288.03368284)(847.77208496,287.99369019)
\curveto(847.8020788,287.96368291)(847.83707876,287.93868294)(847.87708496,287.91869019)
\curveto(847.95707864,287.85868302)(848.03707856,287.78868309)(848.11708496,287.70869019)
\curveto(848.20707839,287.62868325)(848.28207832,287.54868333)(848.34208496,287.46869019)
\curveto(848.5020781,287.25868362)(848.63707796,287.05868382)(848.74708496,286.86869019)
\curveto(848.81707778,286.75868412)(848.87207773,286.63868424)(848.91208496,286.50869019)
\curveto(848.95207765,286.3786845)(848.9970776,286.24868463)(849.04708496,286.11869019)
\curveto(849.0970775,285.98868489)(849.13207747,285.85368502)(849.15208496,285.71369019)
\curveto(849.18207742,285.5736853)(849.21707738,285.43368544)(849.25708496,285.29369019)
\curveto(849.26707733,285.22368565)(849.27207733,285.15368572)(849.27208496,285.08369019)
\lineto(849.30208496,284.87369019)
\moveto(847.84708496,285.38369019)
\curveto(847.87707872,285.42368545)(847.9020787,285.4736854)(847.92208496,285.53369019)
\curveto(847.94207866,285.60368527)(847.94207866,285.6736852)(847.92208496,285.74369019)
\curveto(847.86207874,285.96368491)(847.77707882,286.16868471)(847.66708496,286.35869019)
\curveto(847.52707907,286.58868429)(847.37207923,286.78368409)(847.20208496,286.94369019)
\curveto(847.03207957,287.10368377)(846.81207979,287.23868364)(846.54208496,287.34869019)
\curveto(846.47208013,287.36868351)(846.4020802,287.38368349)(846.33208496,287.39369019)
\curveto(846.26208034,287.41368346)(846.18708041,287.43368344)(846.10708496,287.45369019)
\curveto(846.02708057,287.4736834)(845.94208066,287.48368339)(845.85208496,287.48369019)
\lineto(845.59708496,287.48369019)
\curveto(845.56708103,287.46368341)(845.53208107,287.45368342)(845.49208496,287.45369019)
\curveto(845.45208115,287.46368341)(845.41708118,287.46368341)(845.38708496,287.45369019)
\lineto(845.14708496,287.39369019)
\curveto(845.07708152,287.38368349)(845.00708159,287.36868351)(844.93708496,287.34869019)
\curveto(844.64708195,287.22868365)(844.41208219,287.0786838)(844.23208496,286.89869019)
\curveto(844.06208254,286.71868416)(843.90708269,286.49368438)(843.76708496,286.22369019)
\curveto(843.73708286,286.1736847)(843.70708289,286.10868477)(843.67708496,286.02869019)
\curveto(843.64708295,285.95868492)(843.62208298,285.878685)(843.60208496,285.78869019)
\curveto(843.58208302,285.69868518)(843.57708302,285.61368526)(843.58708496,285.53369019)
\curveto(843.597083,285.45368542)(843.63208297,285.39368548)(843.69208496,285.35369019)
\curveto(843.77208283,285.29368558)(843.90708269,285.26368561)(844.09708496,285.26369019)
\curveto(844.2970823,285.2736856)(844.46708213,285.2786856)(844.60708496,285.27869019)
\lineto(846.88708496,285.27869019)
\curveto(847.03707956,285.2786856)(847.21707938,285.2736856)(847.42708496,285.26369019)
\curveto(847.63707896,285.26368561)(847.77707882,285.30368557)(847.84708496,285.38369019)
}
}
{
\newrgbcolor{curcolor}{0 0 0}
\pscustom[linestyle=none,fillstyle=solid,fillcolor=curcolor]
{
\newpath
\moveto(782.66646484,308.23541504)
\curveto(782.68645716,308.13541095)(782.68645716,308.02041107)(782.66646484,307.89041504)
\curveto(782.65645719,307.77041132)(782.62645722,307.6854114)(782.57646484,307.63541504)
\curveto(782.52645732,307.59541149)(782.45145739,307.56541152)(782.35146484,307.54541504)
\curveto(782.26145758,307.53541155)(782.15645769,307.53041156)(782.03646484,307.53041504)
\lineto(781.67646484,307.53041504)
\curveto(781.55645829,307.54041155)(781.45145839,307.54541154)(781.36146484,307.54541504)
\lineto(777.52146484,307.54541504)
\curveto(777.4414624,307.54541154)(777.36146248,307.54041155)(777.28146484,307.53041504)
\curveto(777.20146264,307.53041156)(777.13646271,307.51541157)(777.08646484,307.48541504)
\curveto(777.0464628,307.46541162)(777.00646284,307.42541166)(776.96646484,307.36541504)
\curveto(776.9464629,307.33541175)(776.92646292,307.2904118)(776.90646484,307.23041504)
\curveto(776.88646296,307.18041191)(776.88646296,307.13041196)(776.90646484,307.08041504)
\curveto(776.91646293,307.03041206)(776.92146292,306.9854121)(776.92146484,306.94541504)
\curveto(776.92146292,306.90541218)(776.92646292,306.86541222)(776.93646484,306.82541504)
\curveto(776.95646289,306.74541234)(776.97646287,306.66041243)(776.99646484,306.57041504)
\curveto(777.01646283,306.4904126)(777.0464628,306.41041268)(777.08646484,306.33041504)
\curveto(777.31646253,305.7904133)(777.69646215,305.40541368)(778.22646484,305.17541504)
\curveto(778.28646156,305.14541394)(778.35146149,305.12041397)(778.42146484,305.10041504)
\lineto(778.63146484,305.04041504)
\curveto(778.66146118,305.03041406)(778.71146113,305.02541406)(778.78146484,305.02541504)
\curveto(778.92146092,304.9854141)(779.10646074,304.96541412)(779.33646484,304.96541504)
\curveto(779.56646028,304.96541412)(779.75146009,304.9854141)(779.89146484,305.02541504)
\curveto(780.03145981,305.06541402)(780.15645969,305.10541398)(780.26646484,305.14541504)
\curveto(780.38645946,305.19541389)(780.49645935,305.25541383)(780.59646484,305.32541504)
\curveto(780.70645914,305.39541369)(780.80145904,305.47541361)(780.88146484,305.56541504)
\curveto(780.96145888,305.66541342)(781.03145881,305.77041332)(781.09146484,305.88041504)
\curveto(781.15145869,305.98041311)(781.20145864,306.085413)(781.24146484,306.19541504)
\curveto(781.29145855,306.30541278)(781.37145847,306.3854127)(781.48146484,306.43541504)
\curveto(781.52145832,306.45541263)(781.58645826,306.47041262)(781.67646484,306.48041504)
\curveto(781.76645808,306.4904126)(781.85645799,306.4904126)(781.94646484,306.48041504)
\curveto(782.03645781,306.48041261)(782.12145772,306.47541261)(782.20146484,306.46541504)
\curveto(782.28145756,306.45541263)(782.33645751,306.43541265)(782.36646484,306.40541504)
\curveto(782.46645738,306.33541275)(782.49145735,306.22041287)(782.44146484,306.06041504)
\curveto(782.36145748,305.7904133)(782.25645759,305.55041354)(782.12646484,305.34041504)
\curveto(781.92645792,305.02041407)(781.69645815,304.75541433)(781.43646484,304.54541504)
\curveto(781.18645866,304.34541474)(780.86645898,304.18041491)(780.47646484,304.05041504)
\curveto(780.37645947,304.01041508)(780.27645957,303.9854151)(780.17646484,303.97541504)
\curveto(780.07645977,303.95541513)(779.97145987,303.93541515)(779.86146484,303.91541504)
\curveto(779.81146003,303.90541518)(779.76146008,303.90041519)(779.71146484,303.90041504)
\curveto(779.67146017,303.90041519)(779.62646022,303.89541519)(779.57646484,303.88541504)
\lineto(779.42646484,303.88541504)
\curveto(779.37646047,303.87541521)(779.31646053,303.87041522)(779.24646484,303.87041504)
\curveto(779.18646066,303.87041522)(779.13646071,303.87541521)(779.09646484,303.88541504)
\lineto(778.96146484,303.88541504)
\curveto(778.91146093,303.89541519)(778.86646098,303.90041519)(778.82646484,303.90041504)
\curveto(778.78646106,303.90041519)(778.7464611,303.90541518)(778.70646484,303.91541504)
\curveto(778.65646119,303.92541516)(778.60146124,303.93541515)(778.54146484,303.94541504)
\curveto(778.48146136,303.94541514)(778.42646142,303.95041514)(778.37646484,303.96041504)
\curveto(778.28646156,303.98041511)(778.19646165,304.00541508)(778.10646484,304.03541504)
\curveto(778.01646183,304.05541503)(777.93146191,304.08041501)(777.85146484,304.11041504)
\curveto(777.81146203,304.13041496)(777.77646207,304.14041495)(777.74646484,304.14041504)
\curveto(777.71646213,304.15041494)(777.68146216,304.16541492)(777.64146484,304.18541504)
\curveto(777.49146235,304.25541483)(777.33146251,304.34041475)(777.16146484,304.44041504)
\curveto(776.87146297,304.63041446)(776.62146322,304.86041423)(776.41146484,305.13041504)
\curveto(776.21146363,305.41041368)(776.0414638,305.72041337)(775.90146484,306.06041504)
\curveto(775.85146399,306.17041292)(775.81146403,306.2854128)(775.78146484,306.40541504)
\curveto(775.76146408,306.52541256)(775.73146411,306.64541244)(775.69146484,306.76541504)
\curveto(775.68146416,306.80541228)(775.67646417,306.84041225)(775.67646484,306.87041504)
\curveto(775.67646417,306.90041219)(775.67146417,306.94041215)(775.66146484,306.99041504)
\curveto(775.6414642,307.07041202)(775.62646422,307.15541193)(775.61646484,307.24541504)
\curveto(775.60646424,307.33541175)(775.59146425,307.42541166)(775.57146484,307.51541504)
\lineto(775.57146484,307.72541504)
\curveto(775.56146428,307.76541132)(775.55146429,307.82041127)(775.54146484,307.89041504)
\curveto(775.5414643,307.97041112)(775.5464643,308.03541105)(775.55646484,308.08541504)
\lineto(775.55646484,308.25041504)
\curveto(775.57646427,308.30041079)(775.58146426,308.35041074)(775.57146484,308.40041504)
\curveto(775.57146427,308.46041063)(775.57646427,308.51541057)(775.58646484,308.56541504)
\curveto(775.62646422,308.72541036)(775.65646419,308.8854102)(775.67646484,309.04541504)
\curveto(775.70646414,309.20540988)(775.75146409,309.35540973)(775.81146484,309.49541504)
\curveto(775.86146398,309.60540948)(775.90646394,309.71540937)(775.94646484,309.82541504)
\curveto(775.99646385,309.94540914)(776.05146379,310.06040903)(776.11146484,310.17041504)
\curveto(776.33146351,310.52040857)(776.58146326,310.82040827)(776.86146484,311.07041504)
\curveto(777.1414627,311.33040776)(777.48646236,311.54540754)(777.89646484,311.71541504)
\curveto(778.01646183,311.76540732)(778.13646171,311.80040729)(778.25646484,311.82041504)
\curveto(778.38646146,311.85040724)(778.52146132,311.88040721)(778.66146484,311.91041504)
\curveto(778.71146113,311.92040717)(778.75646109,311.92540716)(778.79646484,311.92541504)
\curveto(778.83646101,311.93540715)(778.88146096,311.94040715)(778.93146484,311.94041504)
\curveto(778.95146089,311.95040714)(778.97646087,311.95040714)(779.00646484,311.94041504)
\curveto(779.03646081,311.93040716)(779.06146078,311.93540715)(779.08146484,311.95541504)
\curveto(779.50146034,311.96540712)(779.86645998,311.92040717)(780.17646484,311.82041504)
\curveto(780.48645936,311.73040736)(780.76645908,311.60540748)(781.01646484,311.44541504)
\curveto(781.06645878,311.42540766)(781.10645874,311.39540769)(781.13646484,311.35541504)
\curveto(781.16645868,311.32540776)(781.20145864,311.30040779)(781.24146484,311.28041504)
\curveto(781.32145852,311.22040787)(781.40145844,311.15040794)(781.48146484,311.07041504)
\curveto(781.57145827,310.9904081)(781.6464582,310.91040818)(781.70646484,310.83041504)
\curveto(781.86645798,310.62040847)(782.00145784,310.42040867)(782.11146484,310.23041504)
\curveto(782.18145766,310.12040897)(782.23645761,310.00040909)(782.27646484,309.87041504)
\curveto(782.31645753,309.74040935)(782.36145748,309.61040948)(782.41146484,309.48041504)
\curveto(782.46145738,309.35040974)(782.49645735,309.21540987)(782.51646484,309.07541504)
\curveto(782.5464573,308.93541015)(782.58145726,308.79541029)(782.62146484,308.65541504)
\curveto(782.63145721,308.5854105)(782.63645721,308.51541057)(782.63646484,308.44541504)
\lineto(782.66646484,308.23541504)
\moveto(781.21146484,308.74541504)
\curveto(781.2414586,308.7854103)(781.26645858,308.83541025)(781.28646484,308.89541504)
\curveto(781.30645854,308.96541012)(781.30645854,309.03541005)(781.28646484,309.10541504)
\curveto(781.22645862,309.32540976)(781.1414587,309.53040956)(781.03146484,309.72041504)
\curveto(780.89145895,309.95040914)(780.73645911,310.14540894)(780.56646484,310.30541504)
\curveto(780.39645945,310.46540862)(780.17645967,310.60040849)(779.90646484,310.71041504)
\curveto(779.83646001,310.73040836)(779.76646008,310.74540834)(779.69646484,310.75541504)
\curveto(779.62646022,310.77540831)(779.55146029,310.79540829)(779.47146484,310.81541504)
\curveto(779.39146045,310.83540825)(779.30646054,310.84540824)(779.21646484,310.84541504)
\lineto(778.96146484,310.84541504)
\curveto(778.93146091,310.82540826)(778.89646095,310.81540827)(778.85646484,310.81541504)
\curveto(778.81646103,310.82540826)(778.78146106,310.82540826)(778.75146484,310.81541504)
\lineto(778.51146484,310.75541504)
\curveto(778.4414614,310.74540834)(778.37146147,310.73040836)(778.30146484,310.71041504)
\curveto(778.01146183,310.5904085)(777.77646207,310.44040865)(777.59646484,310.26041504)
\curveto(777.42646242,310.08040901)(777.27146257,309.85540923)(777.13146484,309.58541504)
\curveto(777.10146274,309.53540955)(777.07146277,309.47040962)(777.04146484,309.39041504)
\curveto(777.01146283,309.32040977)(776.98646286,309.24040985)(776.96646484,309.15041504)
\curveto(776.9464629,309.06041003)(776.9414629,308.97541011)(776.95146484,308.89541504)
\curveto(776.96146288,308.81541027)(776.99646285,308.75541033)(777.05646484,308.71541504)
\curveto(777.13646271,308.65541043)(777.27146257,308.62541046)(777.46146484,308.62541504)
\curveto(777.66146218,308.63541045)(777.83146201,308.64041045)(777.97146484,308.64041504)
\lineto(780.25146484,308.64041504)
\curveto(780.40145944,308.64041045)(780.58145926,308.63541045)(780.79146484,308.62541504)
\curveto(781.00145884,308.62541046)(781.1414587,308.66541042)(781.21146484,308.74541504)
}
}
{
\newrgbcolor{curcolor}{0 0 0}
\pscustom[linestyle=none,fillstyle=solid,fillcolor=curcolor]
{
\newpath
\moveto(784.54310547,314.74541504)
\curveto(784.67310385,314.74540434)(784.80810372,314.74540434)(784.94810547,314.74541504)
\curveto(785.09810343,314.74540434)(785.20810332,314.71040438)(785.27810547,314.64041504)
\curveto(785.3281032,314.57040452)(785.35310317,314.47540461)(785.35310547,314.35541504)
\curveto(785.36310316,314.24540484)(785.36810316,314.13040496)(785.36810547,314.01041504)
\lineto(785.36810547,312.67541504)
\lineto(785.36810547,306.60041504)
\lineto(785.36810547,304.92041504)
\lineto(785.36810547,304.53041504)
\curveto(785.36810316,304.3904147)(785.34310318,304.28041481)(785.29310547,304.20041504)
\curveto(785.26310326,304.15041494)(785.21810331,304.12041497)(785.15810547,304.11041504)
\curveto(785.10810342,304.10041499)(785.04310348,304.085415)(784.96310547,304.06541504)
\lineto(784.75310547,304.06541504)
\lineto(784.43810547,304.06541504)
\curveto(784.33810419,304.07541501)(784.26310426,304.11041498)(784.21310547,304.17041504)
\curveto(784.16310436,304.25041484)(784.13310439,304.35041474)(784.12310547,304.47041504)
\lineto(784.12310547,304.84541504)
\lineto(784.12310547,306.22541504)
\lineto(784.12310547,312.46541504)
\lineto(784.12310547,313.93541504)
\curveto(784.1231044,314.04540504)(784.11810441,314.16040493)(784.10810547,314.28041504)
\curveto(784.10810442,314.41040468)(784.13310439,314.51040458)(784.18310547,314.58041504)
\curveto(784.2231043,314.64040445)(784.29810423,314.6904044)(784.40810547,314.73041504)
\curveto(784.4281041,314.74040435)(784.44810408,314.74040435)(784.46810547,314.73041504)
\curveto(784.49810403,314.73040436)(784.523104,314.73540435)(784.54310547,314.74541504)
}
}
{
\newrgbcolor{curcolor}{0 0 0}
\pscustom[linestyle=none,fillstyle=solid,fillcolor=curcolor]
{
\newpath
\moveto(794.06294922,308.23541504)
\curveto(794.08294153,308.13541095)(794.08294153,308.02041107)(794.06294922,307.89041504)
\curveto(794.05294156,307.77041132)(794.02294159,307.6854114)(793.97294922,307.63541504)
\curveto(793.92294169,307.59541149)(793.84794177,307.56541152)(793.74794922,307.54541504)
\curveto(793.65794196,307.53541155)(793.55294206,307.53041156)(793.43294922,307.53041504)
\lineto(793.07294922,307.53041504)
\curveto(792.95294266,307.54041155)(792.84794277,307.54541154)(792.75794922,307.54541504)
\lineto(788.91794922,307.54541504)
\curveto(788.83794678,307.54541154)(788.75794686,307.54041155)(788.67794922,307.53041504)
\curveto(788.59794702,307.53041156)(788.53294708,307.51541157)(788.48294922,307.48541504)
\curveto(788.44294717,307.46541162)(788.40294721,307.42541166)(788.36294922,307.36541504)
\curveto(788.34294727,307.33541175)(788.32294729,307.2904118)(788.30294922,307.23041504)
\curveto(788.28294733,307.18041191)(788.28294733,307.13041196)(788.30294922,307.08041504)
\curveto(788.3129473,307.03041206)(788.3179473,306.9854121)(788.31794922,306.94541504)
\curveto(788.3179473,306.90541218)(788.32294729,306.86541222)(788.33294922,306.82541504)
\curveto(788.35294726,306.74541234)(788.37294724,306.66041243)(788.39294922,306.57041504)
\curveto(788.4129472,306.4904126)(788.44294717,306.41041268)(788.48294922,306.33041504)
\curveto(788.7129469,305.7904133)(789.09294652,305.40541368)(789.62294922,305.17541504)
\curveto(789.68294593,305.14541394)(789.74794587,305.12041397)(789.81794922,305.10041504)
\lineto(790.02794922,305.04041504)
\curveto(790.05794556,305.03041406)(790.10794551,305.02541406)(790.17794922,305.02541504)
\curveto(790.3179453,304.9854141)(790.50294511,304.96541412)(790.73294922,304.96541504)
\curveto(790.96294465,304.96541412)(791.14794447,304.9854141)(791.28794922,305.02541504)
\curveto(791.42794419,305.06541402)(791.55294406,305.10541398)(791.66294922,305.14541504)
\curveto(791.78294383,305.19541389)(791.89294372,305.25541383)(791.99294922,305.32541504)
\curveto(792.10294351,305.39541369)(792.19794342,305.47541361)(792.27794922,305.56541504)
\curveto(792.35794326,305.66541342)(792.42794319,305.77041332)(792.48794922,305.88041504)
\curveto(792.54794307,305.98041311)(792.59794302,306.085413)(792.63794922,306.19541504)
\curveto(792.68794293,306.30541278)(792.76794285,306.3854127)(792.87794922,306.43541504)
\curveto(792.9179427,306.45541263)(792.98294263,306.47041262)(793.07294922,306.48041504)
\curveto(793.16294245,306.4904126)(793.25294236,306.4904126)(793.34294922,306.48041504)
\curveto(793.43294218,306.48041261)(793.5179421,306.47541261)(793.59794922,306.46541504)
\curveto(793.67794194,306.45541263)(793.73294188,306.43541265)(793.76294922,306.40541504)
\curveto(793.86294175,306.33541275)(793.88794173,306.22041287)(793.83794922,306.06041504)
\curveto(793.75794186,305.7904133)(793.65294196,305.55041354)(793.52294922,305.34041504)
\curveto(793.32294229,305.02041407)(793.09294252,304.75541433)(792.83294922,304.54541504)
\curveto(792.58294303,304.34541474)(792.26294335,304.18041491)(791.87294922,304.05041504)
\curveto(791.77294384,304.01041508)(791.67294394,303.9854151)(791.57294922,303.97541504)
\curveto(791.47294414,303.95541513)(791.36794425,303.93541515)(791.25794922,303.91541504)
\curveto(791.20794441,303.90541518)(791.15794446,303.90041519)(791.10794922,303.90041504)
\curveto(791.06794455,303.90041519)(791.02294459,303.89541519)(790.97294922,303.88541504)
\lineto(790.82294922,303.88541504)
\curveto(790.77294484,303.87541521)(790.7129449,303.87041522)(790.64294922,303.87041504)
\curveto(790.58294503,303.87041522)(790.53294508,303.87541521)(790.49294922,303.88541504)
\lineto(790.35794922,303.88541504)
\curveto(790.30794531,303.89541519)(790.26294535,303.90041519)(790.22294922,303.90041504)
\curveto(790.18294543,303.90041519)(790.14294547,303.90541518)(790.10294922,303.91541504)
\curveto(790.05294556,303.92541516)(789.99794562,303.93541515)(789.93794922,303.94541504)
\curveto(789.87794574,303.94541514)(789.82294579,303.95041514)(789.77294922,303.96041504)
\curveto(789.68294593,303.98041511)(789.59294602,304.00541508)(789.50294922,304.03541504)
\curveto(789.4129462,304.05541503)(789.32794629,304.08041501)(789.24794922,304.11041504)
\curveto(789.20794641,304.13041496)(789.17294644,304.14041495)(789.14294922,304.14041504)
\curveto(789.1129465,304.15041494)(789.07794654,304.16541492)(789.03794922,304.18541504)
\curveto(788.88794673,304.25541483)(788.72794689,304.34041475)(788.55794922,304.44041504)
\curveto(788.26794735,304.63041446)(788.0179476,304.86041423)(787.80794922,305.13041504)
\curveto(787.60794801,305.41041368)(787.43794818,305.72041337)(787.29794922,306.06041504)
\curveto(787.24794837,306.17041292)(787.20794841,306.2854128)(787.17794922,306.40541504)
\curveto(787.15794846,306.52541256)(787.12794849,306.64541244)(787.08794922,306.76541504)
\curveto(787.07794854,306.80541228)(787.07294854,306.84041225)(787.07294922,306.87041504)
\curveto(787.07294854,306.90041219)(787.06794855,306.94041215)(787.05794922,306.99041504)
\curveto(787.03794858,307.07041202)(787.02294859,307.15541193)(787.01294922,307.24541504)
\curveto(787.00294861,307.33541175)(786.98794863,307.42541166)(786.96794922,307.51541504)
\lineto(786.96794922,307.72541504)
\curveto(786.95794866,307.76541132)(786.94794867,307.82041127)(786.93794922,307.89041504)
\curveto(786.93794868,307.97041112)(786.94294867,308.03541105)(786.95294922,308.08541504)
\lineto(786.95294922,308.25041504)
\curveto(786.97294864,308.30041079)(786.97794864,308.35041074)(786.96794922,308.40041504)
\curveto(786.96794865,308.46041063)(786.97294864,308.51541057)(786.98294922,308.56541504)
\curveto(787.02294859,308.72541036)(787.05294856,308.8854102)(787.07294922,309.04541504)
\curveto(787.10294851,309.20540988)(787.14794847,309.35540973)(787.20794922,309.49541504)
\curveto(787.25794836,309.60540948)(787.30294831,309.71540937)(787.34294922,309.82541504)
\curveto(787.39294822,309.94540914)(787.44794817,310.06040903)(787.50794922,310.17041504)
\curveto(787.72794789,310.52040857)(787.97794764,310.82040827)(788.25794922,311.07041504)
\curveto(788.53794708,311.33040776)(788.88294673,311.54540754)(789.29294922,311.71541504)
\curveto(789.4129462,311.76540732)(789.53294608,311.80040729)(789.65294922,311.82041504)
\curveto(789.78294583,311.85040724)(789.9179457,311.88040721)(790.05794922,311.91041504)
\curveto(790.10794551,311.92040717)(790.15294546,311.92540716)(790.19294922,311.92541504)
\curveto(790.23294538,311.93540715)(790.27794534,311.94040715)(790.32794922,311.94041504)
\curveto(790.34794527,311.95040714)(790.37294524,311.95040714)(790.40294922,311.94041504)
\curveto(790.43294518,311.93040716)(790.45794516,311.93540715)(790.47794922,311.95541504)
\curveto(790.89794472,311.96540712)(791.26294435,311.92040717)(791.57294922,311.82041504)
\curveto(791.88294373,311.73040736)(792.16294345,311.60540748)(792.41294922,311.44541504)
\curveto(792.46294315,311.42540766)(792.50294311,311.39540769)(792.53294922,311.35541504)
\curveto(792.56294305,311.32540776)(792.59794302,311.30040779)(792.63794922,311.28041504)
\curveto(792.7179429,311.22040787)(792.79794282,311.15040794)(792.87794922,311.07041504)
\curveto(792.96794265,310.9904081)(793.04294257,310.91040818)(793.10294922,310.83041504)
\curveto(793.26294235,310.62040847)(793.39794222,310.42040867)(793.50794922,310.23041504)
\curveto(793.57794204,310.12040897)(793.63294198,310.00040909)(793.67294922,309.87041504)
\curveto(793.7129419,309.74040935)(793.75794186,309.61040948)(793.80794922,309.48041504)
\curveto(793.85794176,309.35040974)(793.89294172,309.21540987)(793.91294922,309.07541504)
\curveto(793.94294167,308.93541015)(793.97794164,308.79541029)(794.01794922,308.65541504)
\curveto(794.02794159,308.5854105)(794.03294158,308.51541057)(794.03294922,308.44541504)
\lineto(794.06294922,308.23541504)
\moveto(792.60794922,308.74541504)
\curveto(792.63794298,308.7854103)(792.66294295,308.83541025)(792.68294922,308.89541504)
\curveto(792.70294291,308.96541012)(792.70294291,309.03541005)(792.68294922,309.10541504)
\curveto(792.62294299,309.32540976)(792.53794308,309.53040956)(792.42794922,309.72041504)
\curveto(792.28794333,309.95040914)(792.13294348,310.14540894)(791.96294922,310.30541504)
\curveto(791.79294382,310.46540862)(791.57294404,310.60040849)(791.30294922,310.71041504)
\curveto(791.23294438,310.73040836)(791.16294445,310.74540834)(791.09294922,310.75541504)
\curveto(791.02294459,310.77540831)(790.94794467,310.79540829)(790.86794922,310.81541504)
\curveto(790.78794483,310.83540825)(790.70294491,310.84540824)(790.61294922,310.84541504)
\lineto(790.35794922,310.84541504)
\curveto(790.32794529,310.82540826)(790.29294532,310.81540827)(790.25294922,310.81541504)
\curveto(790.2129454,310.82540826)(790.17794544,310.82540826)(790.14794922,310.81541504)
\lineto(789.90794922,310.75541504)
\curveto(789.83794578,310.74540834)(789.76794585,310.73040836)(789.69794922,310.71041504)
\curveto(789.40794621,310.5904085)(789.17294644,310.44040865)(788.99294922,310.26041504)
\curveto(788.82294679,310.08040901)(788.66794695,309.85540923)(788.52794922,309.58541504)
\curveto(788.49794712,309.53540955)(788.46794715,309.47040962)(788.43794922,309.39041504)
\curveto(788.40794721,309.32040977)(788.38294723,309.24040985)(788.36294922,309.15041504)
\curveto(788.34294727,309.06041003)(788.33794728,308.97541011)(788.34794922,308.89541504)
\curveto(788.35794726,308.81541027)(788.39294722,308.75541033)(788.45294922,308.71541504)
\curveto(788.53294708,308.65541043)(788.66794695,308.62541046)(788.85794922,308.62541504)
\curveto(789.05794656,308.63541045)(789.22794639,308.64041045)(789.36794922,308.64041504)
\lineto(791.64794922,308.64041504)
\curveto(791.79794382,308.64041045)(791.97794364,308.63541045)(792.18794922,308.62541504)
\curveto(792.39794322,308.62541046)(792.53794308,308.66541042)(792.60794922,308.74541504)
}
}
{
\newrgbcolor{curcolor}{0 0 0}
\pscustom[linestyle=none,fillstyle=solid,fillcolor=curcolor]
{
\newpath
\moveto(798.50458984,311.97041504)
\curveto(799.24458505,311.98040711)(799.85958444,311.87040722)(800.34958984,311.64041504)
\curveto(800.84958345,311.42040767)(801.24458305,311.085408)(801.53458984,310.63541504)
\curveto(801.66458263,310.43540865)(801.77458252,310.1904089)(801.86458984,309.90041504)
\curveto(801.88458241,309.85040924)(801.8995824,309.7854093)(801.90958984,309.70541504)
\curveto(801.91958238,309.62540946)(801.91458238,309.55540953)(801.89458984,309.49541504)
\curveto(801.86458243,309.44540964)(801.81458248,309.40040969)(801.74458984,309.36041504)
\curveto(801.71458258,309.34040975)(801.68458261,309.33040976)(801.65458984,309.33041504)
\curveto(801.62458267,309.34040975)(801.58958271,309.34040975)(801.54958984,309.33041504)
\curveto(801.50958279,309.32040977)(801.46958283,309.31540977)(801.42958984,309.31541504)
\curveto(801.38958291,309.32540976)(801.34958295,309.33040976)(801.30958984,309.33041504)
\lineto(800.99458984,309.33041504)
\curveto(800.8945834,309.34040975)(800.80958349,309.37040972)(800.73958984,309.42041504)
\curveto(800.65958364,309.48040961)(800.60458369,309.56540952)(800.57458984,309.67541504)
\curveto(800.54458375,309.7854093)(800.50458379,309.88040921)(800.45458984,309.96041504)
\curveto(800.30458399,310.22040887)(800.10958419,310.42540866)(799.86958984,310.57541504)
\curveto(799.78958451,310.62540846)(799.70458459,310.66540842)(799.61458984,310.69541504)
\curveto(799.52458477,310.73540835)(799.42958487,310.77040832)(799.32958984,310.80041504)
\curveto(799.18958511,310.84040825)(799.00458529,310.86040823)(798.77458984,310.86041504)
\curveto(798.54458575,310.87040822)(798.35458594,310.85040824)(798.20458984,310.80041504)
\curveto(798.13458616,310.78040831)(798.06958623,310.76540832)(798.00958984,310.75541504)
\curveto(797.94958635,310.74540834)(797.88458641,310.73040836)(797.81458984,310.71041504)
\curveto(797.55458674,310.60040849)(797.32458697,310.45040864)(797.12458984,310.26041504)
\curveto(796.92458737,310.07040902)(796.76958753,309.84540924)(796.65958984,309.58541504)
\curveto(796.61958768,309.49540959)(796.58458771,309.40040969)(796.55458984,309.30041504)
\curveto(796.52458777,309.21040988)(796.4945878,309.11040998)(796.46458984,309.00041504)
\lineto(796.37458984,308.59541504)
\curveto(796.36458793,308.54541054)(796.35958794,308.4904106)(796.35958984,308.43041504)
\curveto(796.36958793,308.37041072)(796.36458793,308.31541077)(796.34458984,308.26541504)
\lineto(796.34458984,308.14541504)
\curveto(796.33458796,308.10541098)(796.32458797,308.04041105)(796.31458984,307.95041504)
\curveto(796.31458798,307.86041123)(796.32458797,307.79541129)(796.34458984,307.75541504)
\curveto(796.35458794,307.70541138)(796.35458794,307.65541143)(796.34458984,307.60541504)
\curveto(796.33458796,307.55541153)(796.33458796,307.50541158)(796.34458984,307.45541504)
\curveto(796.35458794,307.41541167)(796.35958794,307.34541174)(796.35958984,307.24541504)
\curveto(796.37958792,307.16541192)(796.3945879,307.08041201)(796.40458984,306.99041504)
\curveto(796.42458787,306.90041219)(796.44458785,306.81541227)(796.46458984,306.73541504)
\curveto(796.57458772,306.41541267)(796.6995876,306.13541295)(796.83958984,305.89541504)
\curveto(796.98958731,305.66541342)(797.1945871,305.46541362)(797.45458984,305.29541504)
\curveto(797.54458675,305.24541384)(797.63458666,305.20041389)(797.72458984,305.16041504)
\curveto(797.82458647,305.12041397)(797.92958637,305.08041401)(798.03958984,305.04041504)
\curveto(798.08958621,305.03041406)(798.12958617,305.02541406)(798.15958984,305.02541504)
\curveto(798.18958611,305.02541406)(798.22958607,305.02041407)(798.27958984,305.01041504)
\curveto(798.30958599,305.00041409)(798.35958594,304.99541409)(798.42958984,304.99541504)
\lineto(798.59458984,304.99541504)
\curveto(798.5945857,304.9854141)(798.61458568,304.98041411)(798.65458984,304.98041504)
\curveto(798.67458562,304.9904141)(798.6995856,304.9904141)(798.72958984,304.98041504)
\curveto(798.75958554,304.98041411)(798.78958551,304.9854141)(798.81958984,304.99541504)
\curveto(798.88958541,305.01541407)(798.95458534,305.02041407)(799.01458984,305.01041504)
\curveto(799.08458521,305.01041408)(799.15458514,305.02041407)(799.22458984,305.04041504)
\curveto(799.48458481,305.12041397)(799.70958459,305.22041387)(799.89958984,305.34041504)
\curveto(800.08958421,305.47041362)(800.24958405,305.63541345)(800.37958984,305.83541504)
\curveto(800.42958387,305.91541317)(800.47458382,306.00041309)(800.51458984,306.09041504)
\lineto(800.63458984,306.36041504)
\curveto(800.65458364,306.44041265)(800.67458362,306.51541257)(800.69458984,306.58541504)
\curveto(800.72458357,306.66541242)(800.77458352,306.73041236)(800.84458984,306.78041504)
\curveto(800.87458342,306.81041228)(800.93458336,306.83041226)(801.02458984,306.84041504)
\curveto(801.11458318,306.86041223)(801.20958309,306.87041222)(801.30958984,306.87041504)
\curveto(801.41958288,306.88041221)(801.51958278,306.88041221)(801.60958984,306.87041504)
\curveto(801.70958259,306.86041223)(801.77958252,306.84041225)(801.81958984,306.81041504)
\curveto(801.87958242,306.77041232)(801.91458238,306.71041238)(801.92458984,306.63041504)
\curveto(801.94458235,306.55041254)(801.94458235,306.46541262)(801.92458984,306.37541504)
\curveto(801.87458242,306.22541286)(801.82458247,306.08041301)(801.77458984,305.94041504)
\curveto(801.73458256,305.81041328)(801.67958262,305.68041341)(801.60958984,305.55041504)
\curveto(801.45958284,305.25041384)(801.26958303,304.9854141)(801.03958984,304.75541504)
\curveto(800.81958348,304.52541456)(800.54958375,304.34041475)(800.22958984,304.20041504)
\curveto(800.14958415,304.16041493)(800.06458423,304.12541496)(799.97458984,304.09541504)
\curveto(799.88458441,304.07541501)(799.78958451,304.05041504)(799.68958984,304.02041504)
\curveto(799.57958472,303.98041511)(799.46958483,303.96041513)(799.35958984,303.96041504)
\curveto(799.24958505,303.95041514)(799.13958516,303.93541515)(799.02958984,303.91541504)
\curveto(798.98958531,303.89541519)(798.94958535,303.8904152)(798.90958984,303.90041504)
\curveto(798.86958543,303.91041518)(798.82958547,303.91041518)(798.78958984,303.90041504)
\lineto(798.65458984,303.90041504)
\lineto(798.41458984,303.90041504)
\curveto(798.34458595,303.8904152)(798.27958602,303.89541519)(798.21958984,303.91541504)
\lineto(798.14458984,303.91541504)
\lineto(797.78458984,303.96041504)
\curveto(797.65458664,304.00041509)(797.52958677,304.03541505)(797.40958984,304.06541504)
\curveto(797.28958701,304.09541499)(797.17458712,304.13541495)(797.06458984,304.18541504)
\curveto(796.70458759,304.34541474)(796.40458789,304.53541455)(796.16458984,304.75541504)
\curveto(795.93458836,304.97541411)(795.71958858,305.24541384)(795.51958984,305.56541504)
\curveto(795.46958883,305.64541344)(795.42458887,305.73541335)(795.38458984,305.83541504)
\lineto(795.26458984,306.13541504)
\curveto(795.21458908,306.24541284)(795.17958912,306.36041273)(795.15958984,306.48041504)
\curveto(795.13958916,306.60041249)(795.11458918,306.72041237)(795.08458984,306.84041504)
\curveto(795.07458922,306.88041221)(795.06958923,306.92041217)(795.06958984,306.96041504)
\curveto(795.06958923,307.00041209)(795.06458923,307.04041205)(795.05458984,307.08041504)
\curveto(795.03458926,307.14041195)(795.02458927,307.20541188)(795.02458984,307.27541504)
\curveto(795.03458926,307.34541174)(795.02958927,307.41041168)(795.00958984,307.47041504)
\lineto(795.00958984,307.62041504)
\curveto(794.9995893,307.67041142)(794.9945893,307.74041135)(794.99458984,307.83041504)
\curveto(794.9945893,307.92041117)(794.9995893,307.9904111)(795.00958984,308.04041504)
\curveto(795.01958928,308.090411)(795.01958928,308.13541095)(795.00958984,308.17541504)
\curveto(795.00958929,308.21541087)(795.01458928,308.25541083)(795.02458984,308.29541504)
\curveto(795.04458925,308.36541072)(795.04958925,308.43541065)(795.03958984,308.50541504)
\curveto(795.03958926,308.57541051)(795.04958925,308.64041045)(795.06958984,308.70041504)
\curveto(795.10958919,308.87041022)(795.14458915,309.04041005)(795.17458984,309.21041504)
\curveto(795.20458909,309.38040971)(795.24958905,309.54040955)(795.30958984,309.69041504)
\curveto(795.51958878,310.21040888)(795.77458852,310.63040846)(796.07458984,310.95041504)
\curveto(796.37458792,311.27040782)(796.78458751,311.53540755)(797.30458984,311.74541504)
\curveto(797.41458688,311.79540729)(797.53458676,311.83040726)(797.66458984,311.85041504)
\curveto(797.7945865,311.87040722)(797.92958637,311.89540719)(798.06958984,311.92541504)
\curveto(798.13958616,311.93540715)(798.20958609,311.94040715)(798.27958984,311.94041504)
\curveto(798.34958595,311.95040714)(798.42458587,311.96040713)(798.50458984,311.97041504)
}
}
{
\newrgbcolor{curcolor}{0 0 0}
\pscustom[linestyle=none,fillstyle=solid,fillcolor=curcolor]
{
\newpath
\moveto(804.37123047,314.13041504)
\curveto(804.52122846,314.13040496)(804.67122831,314.12540496)(804.82123047,314.11541504)
\curveto(804.97122801,314.11540497)(805.0762279,314.07540501)(805.13623047,313.99541504)
\curveto(805.18622779,313.93540515)(805.21122777,313.85040524)(805.21123047,313.74041504)
\curveto(805.22122776,313.64040545)(805.22622775,313.53540555)(805.22623047,313.42541504)
\lineto(805.22623047,312.55541504)
\curveto(805.22622775,312.47540661)(805.22122776,312.3904067)(805.21123047,312.30041504)
\curveto(805.21122777,312.22040687)(805.22122776,312.15040694)(805.24123047,312.09041504)
\curveto(805.2812277,311.95040714)(805.37122761,311.86040723)(805.51123047,311.82041504)
\curveto(805.56122742,311.81040728)(805.60622737,311.80540728)(805.64623047,311.80541504)
\lineto(805.79623047,311.80541504)
\lineto(806.20123047,311.80541504)
\curveto(806.36122662,311.81540727)(806.4762265,311.80540728)(806.54623047,311.77541504)
\curveto(806.63622634,311.71540737)(806.69622628,311.65540743)(806.72623047,311.59541504)
\curveto(806.74622623,311.55540753)(806.75622622,311.51040758)(806.75623047,311.46041504)
\lineto(806.75623047,311.31041504)
\curveto(806.75622622,311.20040789)(806.75122623,311.09540799)(806.74123047,310.99541504)
\curveto(806.73122625,310.90540818)(806.69622628,310.83540825)(806.63623047,310.78541504)
\curveto(806.5762264,310.73540835)(806.49122649,310.70540838)(806.38123047,310.69541504)
\lineto(806.05123047,310.69541504)
\curveto(805.94122704,310.70540838)(805.83122715,310.71040838)(805.72123047,310.71041504)
\curveto(805.61122737,310.71040838)(805.51622746,310.69540839)(805.43623047,310.66541504)
\curveto(805.36622761,310.63540845)(805.31622766,310.5854085)(805.28623047,310.51541504)
\curveto(805.25622772,310.44540864)(805.23622774,310.36040873)(805.22623047,310.26041504)
\curveto(805.21622776,310.17040892)(805.21122777,310.07040902)(805.21123047,309.96041504)
\curveto(805.22122776,309.86040923)(805.22622775,309.76040933)(805.22623047,309.66041504)
\lineto(805.22623047,306.69041504)
\curveto(805.22622775,306.47041262)(805.22122776,306.23541285)(805.21123047,305.98541504)
\curveto(805.21122777,305.74541334)(805.25622772,305.56041353)(805.34623047,305.43041504)
\curveto(805.39622758,305.35041374)(805.46122752,305.29541379)(805.54123047,305.26541504)
\curveto(805.62122736,305.23541385)(805.71622726,305.21041388)(805.82623047,305.19041504)
\curveto(805.85622712,305.18041391)(805.88622709,305.17541391)(805.91623047,305.17541504)
\curveto(805.95622702,305.1854139)(805.99122699,305.1854139)(806.02123047,305.17541504)
\lineto(806.21623047,305.17541504)
\curveto(806.31622666,305.17541391)(806.40622657,305.16541392)(806.48623047,305.14541504)
\curveto(806.5762264,305.13541395)(806.64122634,305.10041399)(806.68123047,305.04041504)
\curveto(806.70122628,305.01041408)(806.71622626,304.95541413)(806.72623047,304.87541504)
\curveto(806.74622623,304.80541428)(806.75622622,304.73041436)(806.75623047,304.65041504)
\curveto(806.76622621,304.57041452)(806.76622621,304.4904146)(806.75623047,304.41041504)
\curveto(806.74622623,304.34041475)(806.72622625,304.2854148)(806.69623047,304.24541504)
\curveto(806.65622632,304.17541491)(806.5812264,304.12541496)(806.47123047,304.09541504)
\curveto(806.39122659,304.07541501)(806.30122668,304.06541502)(806.20123047,304.06541504)
\curveto(806.10122688,304.07541501)(806.01122697,304.08041501)(805.93123047,304.08041504)
\curveto(805.87122711,304.08041501)(805.81122717,304.07541501)(805.75123047,304.06541504)
\curveto(805.69122729,304.06541502)(805.63622734,304.07041502)(805.58623047,304.08041504)
\lineto(805.40623047,304.08041504)
\curveto(805.35622762,304.090415)(805.30622767,304.09541499)(805.25623047,304.09541504)
\curveto(805.21622776,304.10541498)(805.17122781,304.11041498)(805.12123047,304.11041504)
\curveto(804.92122806,304.16041493)(804.74622823,304.21541487)(804.59623047,304.27541504)
\curveto(804.45622852,304.33541475)(804.33622864,304.44041465)(804.23623047,304.59041504)
\curveto(804.09622888,304.7904143)(804.01622896,305.04041405)(803.99623047,305.34041504)
\curveto(803.976229,305.65041344)(803.96622901,305.98041311)(803.96623047,306.33041504)
\lineto(803.96623047,310.26041504)
\curveto(803.93622904,310.3904087)(803.90622907,310.4854086)(803.87623047,310.54541504)
\curveto(803.85622912,310.60540848)(803.78622919,310.65540843)(803.66623047,310.69541504)
\curveto(803.62622935,310.70540838)(803.58622939,310.70540838)(803.54623047,310.69541504)
\curveto(803.50622947,310.6854084)(803.46622951,310.6904084)(803.42623047,310.71041504)
\lineto(803.18623047,310.71041504)
\curveto(803.05622992,310.71040838)(802.94623003,310.72040837)(802.85623047,310.74041504)
\curveto(802.7762302,310.77040832)(802.72123026,310.83040826)(802.69123047,310.92041504)
\curveto(802.67123031,310.96040813)(802.65623032,311.00540808)(802.64623047,311.05541504)
\lineto(802.64623047,311.20541504)
\curveto(802.64623033,311.34540774)(802.65623032,311.46040763)(802.67623047,311.55041504)
\curveto(802.69623028,311.65040744)(802.75623022,311.72540736)(802.85623047,311.77541504)
\curveto(802.96623001,311.81540727)(803.10622987,311.82540726)(803.27623047,311.80541504)
\curveto(803.45622952,311.7854073)(803.60622937,311.79540729)(803.72623047,311.83541504)
\curveto(803.81622916,311.8854072)(803.88622909,311.95540713)(803.93623047,312.04541504)
\curveto(803.95622902,312.10540698)(803.96622901,312.18040691)(803.96623047,312.27041504)
\lineto(803.96623047,312.52541504)
\lineto(803.96623047,313.45541504)
\lineto(803.96623047,313.69541504)
\curveto(803.96622901,313.7854053)(803.976229,313.86040523)(803.99623047,313.92041504)
\curveto(804.03622894,314.00040509)(804.11122887,314.06540502)(804.22123047,314.11541504)
\curveto(804.25122873,314.11540497)(804.2762287,314.11540497)(804.29623047,314.11541504)
\curveto(804.32622865,314.12540496)(804.35122863,314.13040496)(804.37123047,314.13041504)
}
}
{
\newrgbcolor{curcolor}{0 0 0}
\pscustom[linestyle=none,fillstyle=solid,fillcolor=curcolor]
{
\newpath
\moveto(811.78802734,311.97041504)
\curveto(812.01802255,311.97040712)(812.14802242,311.91040718)(812.17802734,311.79041504)
\curveto(812.20802236,311.68040741)(812.22302235,311.51540757)(812.22302734,311.29541504)
\lineto(812.22302734,311.01041504)
\curveto(812.22302235,310.92040817)(812.19802237,310.84540824)(812.14802734,310.78541504)
\curveto(812.08802248,310.70540838)(812.00302257,310.66040843)(811.89302734,310.65041504)
\curveto(811.78302279,310.65040844)(811.6730229,310.63540845)(811.56302734,310.60541504)
\curveto(811.42302315,310.57540851)(811.28802328,310.54540854)(811.15802734,310.51541504)
\curveto(811.03802353,310.4854086)(810.92302365,310.44540864)(810.81302734,310.39541504)
\curveto(810.52302405,310.26540882)(810.28802428,310.085409)(810.10802734,309.85541504)
\curveto(809.92802464,309.63540945)(809.7730248,309.38040971)(809.64302734,309.09041504)
\curveto(809.60302497,308.98041011)(809.573025,308.86541022)(809.55302734,308.74541504)
\curveto(809.53302504,308.63541045)(809.50802506,308.52041057)(809.47802734,308.40041504)
\curveto(809.4680251,308.35041074)(809.46302511,308.30041079)(809.46302734,308.25041504)
\curveto(809.4730251,308.20041089)(809.4730251,308.15041094)(809.46302734,308.10041504)
\curveto(809.43302514,307.98041111)(809.41802515,307.84041125)(809.41802734,307.68041504)
\curveto(809.42802514,307.53041156)(809.43302514,307.3854117)(809.43302734,307.24541504)
\lineto(809.43302734,305.40041504)
\lineto(809.43302734,305.05541504)
\curveto(809.43302514,304.93541415)(809.42802514,304.82041427)(809.41802734,304.71041504)
\curveto(809.40802516,304.60041449)(809.40302517,304.50541458)(809.40302734,304.42541504)
\curveto(809.41302516,304.34541474)(809.39302518,304.27541481)(809.34302734,304.21541504)
\curveto(809.29302528,304.14541494)(809.21302536,304.10541498)(809.10302734,304.09541504)
\curveto(809.00302557,304.085415)(808.89302568,304.08041501)(808.77302734,304.08041504)
\lineto(808.50302734,304.08041504)
\curveto(808.45302612,304.10041499)(808.40302617,304.11541497)(808.35302734,304.12541504)
\curveto(808.31302626,304.14541494)(808.28302629,304.17041492)(808.26302734,304.20041504)
\curveto(808.21302636,304.27041482)(808.18302639,304.35541473)(808.17302734,304.45541504)
\lineto(808.17302734,304.78541504)
\lineto(808.17302734,305.94041504)
\lineto(808.17302734,310.09541504)
\lineto(808.17302734,311.13041504)
\lineto(808.17302734,311.43041504)
\curveto(808.18302639,311.53040756)(808.21302636,311.61540747)(808.26302734,311.68541504)
\curveto(808.29302628,311.72540736)(808.34302623,311.75540733)(808.41302734,311.77541504)
\curveto(808.49302608,311.79540729)(808.57802599,311.80540728)(808.66802734,311.80541504)
\curveto(808.75802581,311.81540727)(808.84802572,311.81540727)(808.93802734,311.80541504)
\curveto(809.02802554,311.79540729)(809.09802547,311.78040731)(809.14802734,311.76041504)
\curveto(809.22802534,311.73040736)(809.27802529,311.67040742)(809.29802734,311.58041504)
\curveto(809.32802524,311.50040759)(809.34302523,311.41040768)(809.34302734,311.31041504)
\lineto(809.34302734,311.01041504)
\curveto(809.34302523,310.91040818)(809.36302521,310.82040827)(809.40302734,310.74041504)
\curveto(809.41302516,310.72040837)(809.42302515,310.70540838)(809.43302734,310.69541504)
\lineto(809.47802734,310.65041504)
\curveto(809.58802498,310.65040844)(809.67802489,310.69540839)(809.74802734,310.78541504)
\curveto(809.81802475,310.8854082)(809.87802469,310.96540812)(809.92802734,311.02541504)
\lineto(810.01802734,311.11541504)
\curveto(810.10802446,311.22540786)(810.23302434,311.34040775)(810.39302734,311.46041504)
\curveto(810.55302402,311.58040751)(810.70302387,311.67040742)(810.84302734,311.73041504)
\curveto(810.93302364,311.78040731)(811.02802354,311.81540727)(811.12802734,311.83541504)
\curveto(811.22802334,311.86540722)(811.33302324,311.89540719)(811.44302734,311.92541504)
\curveto(811.50302307,311.93540715)(811.56302301,311.94040715)(811.62302734,311.94041504)
\curveto(811.68302289,311.95040714)(811.73802283,311.96040713)(811.78802734,311.97041504)
}
}
{
\newrgbcolor{curcolor}{0 0 0}
\pscustom[linestyle=none,fillstyle=solid,fillcolor=curcolor]
{
\newpath
\moveto(817.23279297,314.98541504)
\curveto(817.3027879,314.9854041)(817.38778782,314.9854041)(817.48779297,314.98541504)
\curveto(817.59778761,314.99540409)(817.69778751,314.99540409)(817.78779297,314.98541504)
\curveto(817.88778732,314.9854041)(817.97778723,314.97540411)(818.05779297,314.95541504)
\curveto(818.13778707,314.93540415)(818.19278701,314.90540418)(818.22279297,314.86541504)
\curveto(818.23278697,314.82540426)(818.22778698,314.77040432)(818.20779297,314.70041504)
\curveto(818.18778702,314.64040445)(818.14778706,314.58040451)(818.08779297,314.52041504)
\lineto(817.92279297,314.35541504)
\curveto(817.87278733,314.30540478)(817.82278738,314.25040484)(817.77279297,314.19041504)
\curveto(817.73278747,314.14040495)(817.68778752,314.085405)(817.63779297,314.02541504)
\curveto(817.6077876,313.97540511)(817.56778764,313.93040516)(817.51779297,313.89041504)
\curveto(817.47778773,313.86040523)(817.43778777,313.82040527)(817.39779297,313.77041504)
\lineto(817.35279297,313.72541504)
\curveto(817.35278785,313.71540537)(817.34278786,313.70540538)(817.32279297,313.69541504)
\curveto(817.28278792,313.64540544)(817.24278796,313.60040549)(817.20279297,313.56041504)
\curveto(817.16278804,313.53040556)(817.12278808,313.4904056)(817.08279297,313.44041504)
\curveto(817.06278814,313.40040569)(817.03278817,313.36540572)(816.99279297,313.33541504)
\lineto(816.90279297,313.24541504)
\curveto(816.86278834,313.19540589)(816.81778839,313.14540594)(816.76779297,313.09541504)
\curveto(816.72778848,313.04540604)(816.68278852,313.00540608)(816.63279297,312.97541504)
\curveto(816.56278864,312.93540615)(816.44778876,312.90040619)(816.28779297,312.87041504)
\curveto(816.13778907,312.85040624)(816.01778919,312.86540622)(815.92779297,312.91541504)
\curveto(815.89778931,312.93540615)(815.86778934,312.96540612)(815.83779297,313.00541504)
\curveto(815.81778939,313.05540603)(815.81778939,313.11040598)(815.83779297,313.17041504)
\curveto(815.85778935,313.25040584)(815.88778932,313.32040577)(815.92779297,313.38041504)
\curveto(815.96778924,313.45040564)(816.01278919,313.51540557)(816.06279297,313.57541504)
\curveto(816.14278906,313.71540537)(816.22778898,313.86040523)(816.31779297,314.01041504)
\curveto(816.4077888,314.16040493)(816.49778871,314.30540478)(816.58779297,314.44541504)
\lineto(816.70779297,314.65541504)
\curveto(816.74778846,314.73540435)(816.8027884,314.80040429)(816.87279297,314.85041504)
\curveto(816.94278826,314.90040419)(817.01278819,314.94040415)(817.08279297,314.97041504)
\curveto(817.11278809,314.97040412)(817.13778807,314.97040412)(817.15779297,314.97041504)
\curveto(817.18778802,314.98040411)(817.21278799,314.9854041)(817.23279297,314.98541504)
\moveto(820.27779297,308.26541504)
\curveto(820.26778494,308.31541077)(820.26278494,308.36541072)(820.26279297,308.41541504)
\curveto(820.27278493,308.47541061)(820.27278493,308.53041056)(820.26279297,308.58041504)
\curveto(820.23278497,308.71041038)(820.207785,308.83541025)(820.18779297,308.95541504)
\curveto(820.16778504,309.08541)(820.14278506,309.20540988)(820.11279297,309.31541504)
\curveto(820.07278513,309.42540966)(820.03778517,309.53040956)(820.00779297,309.63041504)
\curveto(819.97778523,309.73040936)(819.93778527,309.83040926)(819.88779297,309.93041504)
\curveto(819.62778558,310.54040855)(819.202786,311.03540805)(818.61279297,311.41541504)
\curveto(818.02278718,311.79540729)(817.28278792,311.98040711)(816.39279297,311.97041504)
\curveto(816.33278887,311.96040713)(816.26778894,311.95040714)(816.19779297,311.94041504)
\lineto(816.00279297,311.94041504)
\curveto(815.86278934,311.90040719)(815.72278948,311.87040722)(815.58279297,311.85041504)
\curveto(815.44278976,311.84040725)(815.31278989,311.81040728)(815.19279297,311.76041504)
\curveto(815.05279015,311.70040739)(814.91779029,311.64040745)(814.78779297,311.58041504)
\curveto(814.65779055,311.53040756)(814.53279067,311.46540762)(814.41279297,311.38541504)
\curveto(814.1027911,311.1854079)(813.83779137,310.93540815)(813.61779297,310.63541504)
\curveto(813.4077918,310.34540874)(813.22779198,310.01540907)(813.07779297,309.64541504)
\curveto(813.02779218,309.53540955)(812.98779222,309.42040967)(812.95779297,309.30041504)
\curveto(812.93779227,309.18040991)(812.91279229,309.06041003)(812.88279297,308.94041504)
\curveto(812.87279233,308.8904102)(812.86279234,308.84541024)(812.85279297,308.80541504)
\curveto(812.85279235,308.77541031)(812.84779236,308.73541035)(812.83779297,308.68541504)
\curveto(812.81779239,308.61541047)(812.81279239,308.54541054)(812.82279297,308.47541504)
\curveto(812.83279237,308.40541068)(812.82779238,308.33541075)(812.80779297,308.26541504)
\curveto(812.78779242,308.20541088)(812.77779243,308.11041098)(812.77779297,307.98041504)
\curveto(812.77779243,307.86041123)(812.78279242,307.77541131)(812.79279297,307.72541504)
\curveto(812.8027924,307.67541141)(812.8077924,307.63041146)(812.80779297,307.59041504)
\lineto(812.80779297,307.47041504)
\curveto(812.82779238,307.3904117)(812.83779237,307.30541178)(812.83779297,307.21541504)
\curveto(812.84779236,307.13541195)(812.86279234,307.05541203)(812.88279297,306.97541504)
\curveto(812.89279231,306.93541215)(812.89279231,306.90041219)(812.88279297,306.87041504)
\curveto(812.88279232,306.85041224)(812.89279231,306.82041227)(812.91279297,306.78041504)
\curveto(812.93279227,306.67041242)(812.95279225,306.56541252)(812.97279297,306.46541504)
\curveto(813.0027922,306.36541272)(813.04279216,306.26541282)(813.09279297,306.16541504)
\curveto(813.28279192,305.70541338)(813.52279168,305.31541377)(813.81279297,304.99541504)
\curveto(814.1027911,304.67541441)(814.47279073,304.41541467)(814.92279297,304.21541504)
\curveto(815.04279016,304.16541492)(815.16779004,304.12041497)(815.29779297,304.08041504)
\curveto(815.42778978,304.04041505)(815.56278964,304.00041509)(815.70279297,303.96041504)
\lineto(816.06279297,303.91541504)
\lineto(816.15279297,303.91541504)
\curveto(816.18278902,303.90541518)(816.21778899,303.90541518)(816.25779297,303.91541504)
\curveto(816.29778891,303.91541517)(816.33778887,303.91041518)(816.37779297,303.90041504)
\curveto(816.4077888,303.8904152)(816.45778875,303.8854152)(816.52779297,303.88541504)
\curveto(816.6077886,303.8854152)(816.66778854,303.8904152)(816.70779297,303.90041504)
\curveto(816.76778844,303.90041519)(816.82778838,303.90541518)(816.88779297,303.91541504)
\curveto(816.95778825,303.91541517)(817.02278818,303.92041517)(817.08279297,303.93041504)
\curveto(817.21278799,303.95041514)(817.33778787,303.97041512)(817.45779297,303.99041504)
\curveto(817.58778762,304.01041508)(817.7077875,304.04041505)(817.81779297,304.08041504)
\curveto(818.32778688,304.25041484)(818.75778645,304.49541459)(819.10779297,304.81541504)
\curveto(819.45778575,305.14541394)(819.73778547,305.55041354)(819.94779297,306.03041504)
\curveto(819.99778521,306.14041295)(820.03778517,306.26041283)(820.06779297,306.39041504)
\curveto(820.09778511,306.52041257)(820.13278507,306.65041244)(820.17279297,306.78041504)
\curveto(820.19278501,306.84041225)(820.202785,306.90041219)(820.20279297,306.96041504)
\curveto(820.21278499,307.02041207)(820.22778498,307.08041201)(820.24779297,307.14041504)
\curveto(820.25778495,307.22041187)(820.26278494,307.2904118)(820.26279297,307.35041504)
\curveto(820.27278493,307.42041167)(820.28278492,307.49541159)(820.29279297,307.57541504)
\lineto(820.29279297,307.72541504)
\curveto(820.3027849,307.77541131)(820.3077849,307.86041123)(820.30779297,307.98041504)
\curveto(820.3077849,308.11041098)(820.29778491,308.20541088)(820.27779297,308.26541504)
\moveto(818.94279297,307.41041504)
\curveto(818.9027863,307.27041182)(818.87278633,307.13041196)(818.85279297,306.99041504)
\curveto(818.84278636,306.86041223)(818.81278639,306.73541235)(818.76279297,306.61541504)
\curveto(818.62278658,306.27541281)(818.44778676,305.98041311)(818.23779297,305.73041504)
\curveto(818.02778718,305.48041361)(817.75278745,305.2854138)(817.41279297,305.14541504)
\curveto(817.34278786,305.11541397)(817.26778794,305.090414)(817.18779297,305.07041504)
\curveto(817.1077881,305.06041403)(817.02778818,305.04541404)(816.94779297,305.02541504)
\curveto(816.9077883,305.00541408)(816.86778834,304.99541409)(816.82779297,304.99541504)
\curveto(816.78778842,305.00541408)(816.74778846,305.00541408)(816.70779297,304.99541504)
\curveto(816.65778855,304.97541411)(816.58278862,304.97041412)(816.48279297,304.98041504)
\curveto(816.39278881,304.9904141)(816.33278887,305.00041409)(816.30279297,305.01041504)
\curveto(816.25278895,305.02041407)(816.202789,305.02041407)(816.15279297,305.01041504)
\curveto(816.11278909,305.01041408)(816.06778914,305.02041407)(816.01779297,305.04041504)
\curveto(815.9077893,305.07041402)(815.8027894,305.10041399)(815.70279297,305.13041504)
\curveto(815.61278959,305.17041392)(815.52778968,305.21541387)(815.44779297,305.26541504)
\curveto(815.39778981,305.29541379)(815.35278985,305.32041377)(815.31279297,305.34041504)
\curveto(815.27278993,305.36041373)(815.22778998,305.3854137)(815.17779297,305.41541504)
\curveto(814.97779023,305.55541353)(814.8077904,305.73041336)(814.66779297,305.94041504)
\curveto(814.53779067,306.15041294)(814.42279078,306.37541271)(814.32279297,306.61541504)
\curveto(814.28279092,306.69541239)(814.25279095,306.78041231)(814.23279297,306.87041504)
\curveto(814.22279098,306.97041212)(814.207791,307.07041202)(814.18779297,307.17041504)
\lineto(814.15779297,307.35041504)
\curveto(814.13779107,307.43041166)(814.12779108,307.52041157)(814.12779297,307.62041504)
\lineto(814.12779297,307.92041504)
\curveto(814.12779108,307.97041112)(814.12279108,308.01541107)(814.11279297,308.05541504)
\curveto(814.11279109,308.09541099)(814.11779109,308.13041096)(814.12779297,308.16041504)
\curveto(814.12779108,308.25041084)(814.13279107,308.31541077)(814.14279297,308.35541504)
\curveto(814.16279104,308.46541062)(814.17779103,308.57041052)(814.18779297,308.67041504)
\curveto(814.19779101,308.78041031)(814.21779099,308.8854102)(814.24779297,308.98541504)
\curveto(814.35779085,309.30540978)(814.48779072,309.5854095)(814.63779297,309.82541504)
\curveto(814.78779042,310.06540902)(814.98279022,310.27040882)(815.22279297,310.44041504)
\curveto(815.27278993,310.48040861)(815.32778988,310.52040857)(815.38779297,310.56041504)
\curveto(815.44778976,310.60040849)(815.51278969,310.63040846)(815.58279297,310.65041504)
\curveto(815.66278954,310.6904084)(815.74278946,310.72040837)(815.82279297,310.74041504)
\curveto(815.91278929,310.77040832)(816.0077892,310.79540829)(816.10779297,310.81541504)
\lineto(816.37779297,310.86041504)
\lineto(816.64779297,310.86041504)
\curveto(816.73778847,310.86040823)(816.82278838,310.85040824)(816.90279297,310.83041504)
\lineto(817.14279297,310.77041504)
\curveto(817.22278798,310.76040833)(817.29778791,310.74040835)(817.36779297,310.71041504)
\curveto(817.97778723,310.46040863)(818.42278678,310.01040908)(818.70279297,309.36041504)
\curveto(818.73278647,309.2904098)(818.75778645,309.21540987)(818.77779297,309.13541504)
\curveto(818.79778641,309.05541003)(818.82278638,308.97541011)(818.85279297,308.89541504)
\curveto(818.92278628,308.62541046)(818.95778625,308.29541079)(818.95779297,307.90541504)
\lineto(818.95779297,307.65041504)
\curveto(818.96778624,307.56041153)(818.96278624,307.48041161)(818.94279297,307.41041504)
}
}
{
\newrgbcolor{curcolor}{0 0 0}
\pscustom[linestyle=none,fillstyle=solid,fillcolor=curcolor]
{
\newpath
\moveto(825.45607422,311.94041504)
\curveto(826.08606898,311.96040713)(826.59106848,311.87540721)(826.97107422,311.68541504)
\curveto(827.35106772,311.49540759)(827.65606741,311.21040788)(827.88607422,310.83041504)
\curveto(827.94606712,310.73040836)(827.99106708,310.62040847)(828.02107422,310.50041504)
\curveto(828.06106701,310.3904087)(828.09606697,310.27540881)(828.12607422,310.15541504)
\curveto(828.17606689,309.96540912)(828.20606686,309.76040933)(828.21607422,309.54041504)
\curveto(828.22606684,309.32040977)(828.23106684,309.09540999)(828.23107422,308.86541504)
\lineto(828.23107422,307.26041504)
\lineto(828.23107422,304.92041504)
\curveto(828.23106684,304.75041434)(828.22606684,304.58041451)(828.21607422,304.41041504)
\curveto(828.21606685,304.24041485)(828.15106692,304.13041496)(828.02107422,304.08041504)
\curveto(827.9710671,304.06041503)(827.91606715,304.05041504)(827.85607422,304.05041504)
\curveto(827.80606726,304.04041505)(827.75106732,304.03541505)(827.69107422,304.03541504)
\curveto(827.56106751,304.03541505)(827.43606763,304.04041505)(827.31607422,304.05041504)
\curveto(827.19606787,304.05041504)(827.11106796,304.090415)(827.06107422,304.17041504)
\curveto(827.01106806,304.24041485)(826.98606808,304.33041476)(826.98607422,304.44041504)
\lineto(826.98607422,304.77041504)
\lineto(826.98607422,306.06041504)
\lineto(826.98607422,308.50541504)
\curveto(826.98606808,308.77541031)(826.98106809,309.04041005)(826.97107422,309.30041504)
\curveto(826.96106811,309.57040952)(826.91606815,309.80040929)(826.83607422,309.99041504)
\curveto(826.75606831,310.1904089)(826.63606843,310.35040874)(826.47607422,310.47041504)
\curveto(826.31606875,310.60040849)(826.13106894,310.70040839)(825.92107422,310.77041504)
\curveto(825.86106921,310.7904083)(825.79606927,310.80040829)(825.72607422,310.80041504)
\curveto(825.6660694,310.81040828)(825.60606946,310.82540826)(825.54607422,310.84541504)
\curveto(825.49606957,310.85540823)(825.41606965,310.85540823)(825.30607422,310.84541504)
\curveto(825.20606986,310.84540824)(825.13606993,310.84040825)(825.09607422,310.83041504)
\curveto(825.05607001,310.81040828)(825.02107005,310.80040829)(824.99107422,310.80041504)
\curveto(824.96107011,310.81040828)(824.92607014,310.81040828)(824.88607422,310.80041504)
\curveto(824.75607031,310.77040832)(824.63107044,310.73540835)(824.51107422,310.69541504)
\curveto(824.40107067,310.66540842)(824.29607077,310.62040847)(824.19607422,310.56041504)
\curveto(824.15607091,310.54040855)(824.12107095,310.52040857)(824.09107422,310.50041504)
\curveto(824.06107101,310.48040861)(824.02607104,310.46040863)(823.98607422,310.44041504)
\curveto(823.63607143,310.1904089)(823.38107169,309.81540927)(823.22107422,309.31541504)
\curveto(823.19107188,309.23540985)(823.1710719,309.15040994)(823.16107422,309.06041504)
\curveto(823.15107192,308.98041011)(823.13607193,308.90041019)(823.11607422,308.82041504)
\curveto(823.09607197,308.77041032)(823.09107198,308.72041037)(823.10107422,308.67041504)
\curveto(823.11107196,308.63041046)(823.10607196,308.5904105)(823.08607422,308.55041504)
\lineto(823.08607422,308.23541504)
\curveto(823.07607199,308.20541088)(823.071072,308.17041092)(823.07107422,308.13041504)
\curveto(823.08107199,308.090411)(823.08607198,308.04541104)(823.08607422,307.99541504)
\lineto(823.08607422,307.54541504)
\lineto(823.08607422,306.10541504)
\lineto(823.08607422,304.78541504)
\lineto(823.08607422,304.44041504)
\curveto(823.08607198,304.33041476)(823.06107201,304.24041485)(823.01107422,304.17041504)
\curveto(822.96107211,304.090415)(822.8710722,304.05041504)(822.74107422,304.05041504)
\curveto(822.62107245,304.04041505)(822.49607257,304.03541505)(822.36607422,304.03541504)
\curveto(822.28607278,304.03541505)(822.21107286,304.04041505)(822.14107422,304.05041504)
\curveto(822.071073,304.06041503)(822.01107306,304.085415)(821.96107422,304.12541504)
\curveto(821.88107319,304.17541491)(821.84107323,304.27041482)(821.84107422,304.41041504)
\lineto(821.84107422,304.81541504)
\lineto(821.84107422,306.58541504)
\lineto(821.84107422,310.21541504)
\lineto(821.84107422,311.13041504)
\lineto(821.84107422,311.40041504)
\curveto(821.84107323,311.4904076)(821.86107321,311.56040753)(821.90107422,311.61041504)
\curveto(821.93107314,311.67040742)(821.98107309,311.71040738)(822.05107422,311.73041504)
\curveto(822.09107298,311.74040735)(822.14607292,311.75040734)(822.21607422,311.76041504)
\curveto(822.29607277,311.77040732)(822.37607269,311.77540731)(822.45607422,311.77541504)
\curveto(822.53607253,311.77540731)(822.61107246,311.77040732)(822.68107422,311.76041504)
\curveto(822.76107231,311.75040734)(822.81607225,311.73540735)(822.84607422,311.71541504)
\curveto(822.95607211,311.64540744)(823.00607206,311.55540753)(822.99607422,311.44541504)
\curveto(822.98607208,311.34540774)(823.00107207,311.23040786)(823.04107422,311.10041504)
\curveto(823.06107201,311.04040805)(823.10107197,310.9904081)(823.16107422,310.95041504)
\curveto(823.28107179,310.94040815)(823.37607169,310.9854081)(823.44607422,311.08541504)
\curveto(823.52607154,311.1854079)(823.60607146,311.26540782)(823.68607422,311.32541504)
\curveto(823.82607124,311.42540766)(823.9660711,311.51540757)(824.10607422,311.59541504)
\curveto(824.25607081,311.6854074)(824.42607064,311.76040733)(824.61607422,311.82041504)
\curveto(824.69607037,311.85040724)(824.78107029,311.87040722)(824.87107422,311.88041504)
\curveto(824.9710701,311.8904072)(825.06607,311.90540718)(825.15607422,311.92541504)
\curveto(825.20606986,311.93540715)(825.25606981,311.94040715)(825.30607422,311.94041504)
\lineto(825.45607422,311.94041504)
}
}
{
\newrgbcolor{curcolor}{0 0 0}
\pscustom[linestyle=none,fillstyle=solid,fillcolor=curcolor]
{
\newpath
\moveto(830.40068359,313.29041504)
\curveto(830.32068247,313.35040574)(830.27568252,313.45540563)(830.26568359,313.60541504)
\lineto(830.26568359,314.07041504)
\lineto(830.26568359,314.32541504)
\curveto(830.26568253,314.41540467)(830.28068251,314.4904046)(830.31068359,314.55041504)
\curveto(830.35068244,314.63040446)(830.43068236,314.6904044)(830.55068359,314.73041504)
\curveto(830.57068222,314.74040435)(830.5906822,314.74040435)(830.61068359,314.73041504)
\curveto(830.64068215,314.73040436)(830.66568213,314.73540435)(830.68568359,314.74541504)
\curveto(830.85568194,314.74540434)(831.01568178,314.74040435)(831.16568359,314.73041504)
\curveto(831.31568148,314.72040437)(831.41568138,314.66040443)(831.46568359,314.55041504)
\curveto(831.4956813,314.4904046)(831.51068128,314.41540467)(831.51068359,314.32541504)
\lineto(831.51068359,314.07041504)
\curveto(831.51068128,313.8904052)(831.50568129,313.72040537)(831.49568359,313.56041504)
\curveto(831.4956813,313.40040569)(831.43068136,313.29540579)(831.30068359,313.24541504)
\curveto(831.25068154,313.22540586)(831.1956816,313.21540587)(831.13568359,313.21541504)
\lineto(830.97068359,313.21541504)
\lineto(830.65568359,313.21541504)
\curveto(830.55568224,313.21540587)(830.47068232,313.24040585)(830.40068359,313.29041504)
\moveto(831.51068359,304.78541504)
\lineto(831.51068359,304.47041504)
\curveto(831.52068127,304.37041472)(831.50068129,304.2904148)(831.45068359,304.23041504)
\curveto(831.42068137,304.17041492)(831.37568142,304.13041496)(831.31568359,304.11041504)
\curveto(831.25568154,304.10041499)(831.18568161,304.085415)(831.10568359,304.06541504)
\lineto(830.88068359,304.06541504)
\curveto(830.75068204,304.06541502)(830.63568216,304.07041502)(830.53568359,304.08041504)
\curveto(830.44568235,304.10041499)(830.37568242,304.15041494)(830.32568359,304.23041504)
\curveto(830.28568251,304.2904148)(830.26568253,304.36541472)(830.26568359,304.45541504)
\lineto(830.26568359,304.74041504)
\lineto(830.26568359,311.08541504)
\lineto(830.26568359,311.40041504)
\curveto(830.26568253,311.51040758)(830.2906825,311.59540749)(830.34068359,311.65541504)
\curveto(830.37068242,311.70540738)(830.41068238,311.73540735)(830.46068359,311.74541504)
\curveto(830.51068228,311.75540733)(830.56568223,311.77040732)(830.62568359,311.79041504)
\curveto(830.64568215,311.7904073)(830.66568213,311.7854073)(830.68568359,311.77541504)
\curveto(830.71568208,311.77540731)(830.74068205,311.78040731)(830.76068359,311.79041504)
\curveto(830.8906819,311.7904073)(831.02068177,311.7854073)(831.15068359,311.77541504)
\curveto(831.2906815,311.77540731)(831.38568141,311.73540735)(831.43568359,311.65541504)
\curveto(831.48568131,311.59540749)(831.51068128,311.51540757)(831.51068359,311.41541504)
\lineto(831.51068359,311.13041504)
\lineto(831.51068359,304.78541504)
}
}
{
\newrgbcolor{curcolor}{0 0 0}
\pscustom[linestyle=none,fillstyle=solid,fillcolor=curcolor]
{
\newpath
\moveto(836.59052734,311.97041504)
\curveto(837.33052255,311.98040711)(837.94552194,311.87040722)(838.43552734,311.64041504)
\curveto(838.93552095,311.42040767)(839.33052055,311.085408)(839.62052734,310.63541504)
\curveto(839.75052013,310.43540865)(839.86052002,310.1904089)(839.95052734,309.90041504)
\curveto(839.97051991,309.85040924)(839.9855199,309.7854093)(839.99552734,309.70541504)
\curveto(840.00551988,309.62540946)(840.00051988,309.55540953)(839.98052734,309.49541504)
\curveto(839.95051993,309.44540964)(839.90051998,309.40040969)(839.83052734,309.36041504)
\curveto(839.80052008,309.34040975)(839.77052011,309.33040976)(839.74052734,309.33041504)
\curveto(839.71052017,309.34040975)(839.67552021,309.34040975)(839.63552734,309.33041504)
\curveto(839.59552029,309.32040977)(839.55552033,309.31540977)(839.51552734,309.31541504)
\curveto(839.47552041,309.32540976)(839.43552045,309.33040976)(839.39552734,309.33041504)
\lineto(839.08052734,309.33041504)
\curveto(838.9805209,309.34040975)(838.89552099,309.37040972)(838.82552734,309.42041504)
\curveto(838.74552114,309.48040961)(838.69052119,309.56540952)(838.66052734,309.67541504)
\curveto(838.63052125,309.7854093)(838.59052129,309.88040921)(838.54052734,309.96041504)
\curveto(838.39052149,310.22040887)(838.19552169,310.42540866)(837.95552734,310.57541504)
\curveto(837.87552201,310.62540846)(837.79052209,310.66540842)(837.70052734,310.69541504)
\curveto(837.61052227,310.73540835)(837.51552237,310.77040832)(837.41552734,310.80041504)
\curveto(837.27552261,310.84040825)(837.09052279,310.86040823)(836.86052734,310.86041504)
\curveto(836.63052325,310.87040822)(836.44052344,310.85040824)(836.29052734,310.80041504)
\curveto(836.22052366,310.78040831)(836.15552373,310.76540832)(836.09552734,310.75541504)
\curveto(836.03552385,310.74540834)(835.97052391,310.73040836)(835.90052734,310.71041504)
\curveto(835.64052424,310.60040849)(835.41052447,310.45040864)(835.21052734,310.26041504)
\curveto(835.01052487,310.07040902)(834.85552503,309.84540924)(834.74552734,309.58541504)
\curveto(834.70552518,309.49540959)(834.67052521,309.40040969)(834.64052734,309.30041504)
\curveto(834.61052527,309.21040988)(834.5805253,309.11040998)(834.55052734,309.00041504)
\lineto(834.46052734,308.59541504)
\curveto(834.45052543,308.54541054)(834.44552544,308.4904106)(834.44552734,308.43041504)
\curveto(834.45552543,308.37041072)(834.45052543,308.31541077)(834.43052734,308.26541504)
\lineto(834.43052734,308.14541504)
\curveto(834.42052546,308.10541098)(834.41052547,308.04041105)(834.40052734,307.95041504)
\curveto(834.40052548,307.86041123)(834.41052547,307.79541129)(834.43052734,307.75541504)
\curveto(834.44052544,307.70541138)(834.44052544,307.65541143)(834.43052734,307.60541504)
\curveto(834.42052546,307.55541153)(834.42052546,307.50541158)(834.43052734,307.45541504)
\curveto(834.44052544,307.41541167)(834.44552544,307.34541174)(834.44552734,307.24541504)
\curveto(834.46552542,307.16541192)(834.4805254,307.08041201)(834.49052734,306.99041504)
\curveto(834.51052537,306.90041219)(834.53052535,306.81541227)(834.55052734,306.73541504)
\curveto(834.66052522,306.41541267)(834.7855251,306.13541295)(834.92552734,305.89541504)
\curveto(835.07552481,305.66541342)(835.2805246,305.46541362)(835.54052734,305.29541504)
\curveto(835.63052425,305.24541384)(835.72052416,305.20041389)(835.81052734,305.16041504)
\curveto(835.91052397,305.12041397)(836.01552387,305.08041401)(836.12552734,305.04041504)
\curveto(836.17552371,305.03041406)(836.21552367,305.02541406)(836.24552734,305.02541504)
\curveto(836.27552361,305.02541406)(836.31552357,305.02041407)(836.36552734,305.01041504)
\curveto(836.39552349,305.00041409)(836.44552344,304.99541409)(836.51552734,304.99541504)
\lineto(836.68052734,304.99541504)
\curveto(836.6805232,304.9854141)(836.70052318,304.98041411)(836.74052734,304.98041504)
\curveto(836.76052312,304.9904141)(836.7855231,304.9904141)(836.81552734,304.98041504)
\curveto(836.84552304,304.98041411)(836.87552301,304.9854141)(836.90552734,304.99541504)
\curveto(836.97552291,305.01541407)(837.04052284,305.02041407)(837.10052734,305.01041504)
\curveto(837.17052271,305.01041408)(837.24052264,305.02041407)(837.31052734,305.04041504)
\curveto(837.57052231,305.12041397)(837.79552209,305.22041387)(837.98552734,305.34041504)
\curveto(838.17552171,305.47041362)(838.33552155,305.63541345)(838.46552734,305.83541504)
\curveto(838.51552137,305.91541317)(838.56052132,306.00041309)(838.60052734,306.09041504)
\lineto(838.72052734,306.36041504)
\curveto(838.74052114,306.44041265)(838.76052112,306.51541257)(838.78052734,306.58541504)
\curveto(838.81052107,306.66541242)(838.86052102,306.73041236)(838.93052734,306.78041504)
\curveto(838.96052092,306.81041228)(839.02052086,306.83041226)(839.11052734,306.84041504)
\curveto(839.20052068,306.86041223)(839.29552059,306.87041222)(839.39552734,306.87041504)
\curveto(839.50552038,306.88041221)(839.60552028,306.88041221)(839.69552734,306.87041504)
\curveto(839.79552009,306.86041223)(839.86552002,306.84041225)(839.90552734,306.81041504)
\curveto(839.96551992,306.77041232)(840.00051988,306.71041238)(840.01052734,306.63041504)
\curveto(840.03051985,306.55041254)(840.03051985,306.46541262)(840.01052734,306.37541504)
\curveto(839.96051992,306.22541286)(839.91051997,306.08041301)(839.86052734,305.94041504)
\curveto(839.82052006,305.81041328)(839.76552012,305.68041341)(839.69552734,305.55041504)
\curveto(839.54552034,305.25041384)(839.35552053,304.9854141)(839.12552734,304.75541504)
\curveto(838.90552098,304.52541456)(838.63552125,304.34041475)(838.31552734,304.20041504)
\curveto(838.23552165,304.16041493)(838.15052173,304.12541496)(838.06052734,304.09541504)
\curveto(837.97052191,304.07541501)(837.87552201,304.05041504)(837.77552734,304.02041504)
\curveto(837.66552222,303.98041511)(837.55552233,303.96041513)(837.44552734,303.96041504)
\curveto(837.33552255,303.95041514)(837.22552266,303.93541515)(837.11552734,303.91541504)
\curveto(837.07552281,303.89541519)(837.03552285,303.8904152)(836.99552734,303.90041504)
\curveto(836.95552293,303.91041518)(836.91552297,303.91041518)(836.87552734,303.90041504)
\lineto(836.74052734,303.90041504)
\lineto(836.50052734,303.90041504)
\curveto(836.43052345,303.8904152)(836.36552352,303.89541519)(836.30552734,303.91541504)
\lineto(836.23052734,303.91541504)
\lineto(835.87052734,303.96041504)
\curveto(835.74052414,304.00041509)(835.61552427,304.03541505)(835.49552734,304.06541504)
\curveto(835.37552451,304.09541499)(835.26052462,304.13541495)(835.15052734,304.18541504)
\curveto(834.79052509,304.34541474)(834.49052539,304.53541455)(834.25052734,304.75541504)
\curveto(834.02052586,304.97541411)(833.80552608,305.24541384)(833.60552734,305.56541504)
\curveto(833.55552633,305.64541344)(833.51052637,305.73541335)(833.47052734,305.83541504)
\lineto(833.35052734,306.13541504)
\curveto(833.30052658,306.24541284)(833.26552662,306.36041273)(833.24552734,306.48041504)
\curveto(833.22552666,306.60041249)(833.20052668,306.72041237)(833.17052734,306.84041504)
\curveto(833.16052672,306.88041221)(833.15552673,306.92041217)(833.15552734,306.96041504)
\curveto(833.15552673,307.00041209)(833.15052673,307.04041205)(833.14052734,307.08041504)
\curveto(833.12052676,307.14041195)(833.11052677,307.20541188)(833.11052734,307.27541504)
\curveto(833.12052676,307.34541174)(833.11552677,307.41041168)(833.09552734,307.47041504)
\lineto(833.09552734,307.62041504)
\curveto(833.0855268,307.67041142)(833.0805268,307.74041135)(833.08052734,307.83041504)
\curveto(833.0805268,307.92041117)(833.0855268,307.9904111)(833.09552734,308.04041504)
\curveto(833.10552678,308.090411)(833.10552678,308.13541095)(833.09552734,308.17541504)
\curveto(833.09552679,308.21541087)(833.10052678,308.25541083)(833.11052734,308.29541504)
\curveto(833.13052675,308.36541072)(833.13552675,308.43541065)(833.12552734,308.50541504)
\curveto(833.12552676,308.57541051)(833.13552675,308.64041045)(833.15552734,308.70041504)
\curveto(833.19552669,308.87041022)(833.23052665,309.04041005)(833.26052734,309.21041504)
\curveto(833.29052659,309.38040971)(833.33552655,309.54040955)(833.39552734,309.69041504)
\curveto(833.60552628,310.21040888)(833.86052602,310.63040846)(834.16052734,310.95041504)
\curveto(834.46052542,311.27040782)(834.87052501,311.53540755)(835.39052734,311.74541504)
\curveto(835.50052438,311.79540729)(835.62052426,311.83040726)(835.75052734,311.85041504)
\curveto(835.880524,311.87040722)(836.01552387,311.89540719)(836.15552734,311.92541504)
\curveto(836.22552366,311.93540715)(836.29552359,311.94040715)(836.36552734,311.94041504)
\curveto(836.43552345,311.95040714)(836.51052337,311.96040713)(836.59052734,311.97041504)
}
}
{
\newrgbcolor{curcolor}{0 0 0}
\pscustom[linestyle=none,fillstyle=solid,fillcolor=curcolor]
{
\newpath
\moveto(848.63716797,308.26541504)
\curveto(848.65715991,308.20541088)(848.6671599,308.11041098)(848.66716797,307.98041504)
\curveto(848.6671599,307.86041123)(848.6621599,307.77541131)(848.65216797,307.72541504)
\lineto(848.65216797,307.57541504)
\curveto(848.64215992,307.49541159)(848.63215993,307.42041167)(848.62216797,307.35041504)
\curveto(848.62215994,307.2904118)(848.61715995,307.22041187)(848.60716797,307.14041504)
\curveto(848.58715998,307.08041201)(848.57215999,307.02041207)(848.56216797,306.96041504)
\curveto(848.56216,306.90041219)(848.55216001,306.84041225)(848.53216797,306.78041504)
\curveto(848.49216007,306.65041244)(848.45716011,306.52041257)(848.42716797,306.39041504)
\curveto(848.39716017,306.26041283)(848.35716021,306.14041295)(848.30716797,306.03041504)
\curveto(848.09716047,305.55041354)(847.81716075,305.14541394)(847.46716797,304.81541504)
\curveto(847.11716145,304.49541459)(846.68716188,304.25041484)(846.17716797,304.08041504)
\curveto(846.0671625,304.04041505)(845.94716262,304.01041508)(845.81716797,303.99041504)
\curveto(845.69716287,303.97041512)(845.57216299,303.95041514)(845.44216797,303.93041504)
\curveto(845.38216318,303.92041517)(845.31716325,303.91541517)(845.24716797,303.91541504)
\curveto(845.18716338,303.90541518)(845.12716344,303.90041519)(845.06716797,303.90041504)
\curveto(845.02716354,303.8904152)(844.9671636,303.8854152)(844.88716797,303.88541504)
\curveto(844.81716375,303.8854152)(844.7671638,303.8904152)(844.73716797,303.90041504)
\curveto(844.69716387,303.91041518)(844.65716391,303.91541517)(844.61716797,303.91541504)
\curveto(844.57716399,303.90541518)(844.54216402,303.90541518)(844.51216797,303.91541504)
\lineto(844.42216797,303.91541504)
\lineto(844.06216797,303.96041504)
\curveto(843.92216464,304.00041509)(843.78716478,304.04041505)(843.65716797,304.08041504)
\curveto(843.52716504,304.12041497)(843.40216516,304.16541492)(843.28216797,304.21541504)
\curveto(842.83216573,304.41541467)(842.4621661,304.67541441)(842.17216797,304.99541504)
\curveto(841.88216668,305.31541377)(841.64216692,305.70541338)(841.45216797,306.16541504)
\curveto(841.40216716,306.26541282)(841.3621672,306.36541272)(841.33216797,306.46541504)
\curveto(841.31216725,306.56541252)(841.29216727,306.67041242)(841.27216797,306.78041504)
\curveto(841.25216731,306.82041227)(841.24216732,306.85041224)(841.24216797,306.87041504)
\curveto(841.25216731,306.90041219)(841.25216731,306.93541215)(841.24216797,306.97541504)
\curveto(841.22216734,307.05541203)(841.20716736,307.13541195)(841.19716797,307.21541504)
\curveto(841.19716737,307.30541178)(841.18716738,307.3904117)(841.16716797,307.47041504)
\lineto(841.16716797,307.59041504)
\curveto(841.1671674,307.63041146)(841.1621674,307.67541141)(841.15216797,307.72541504)
\curveto(841.14216742,307.77541131)(841.13716743,307.86041123)(841.13716797,307.98041504)
\curveto(841.13716743,308.11041098)(841.14716742,308.20541088)(841.16716797,308.26541504)
\curveto(841.18716738,308.33541075)(841.19216737,308.40541068)(841.18216797,308.47541504)
\curveto(841.17216739,308.54541054)(841.17716739,308.61541047)(841.19716797,308.68541504)
\curveto(841.20716736,308.73541035)(841.21216735,308.77541031)(841.21216797,308.80541504)
\curveto(841.22216734,308.84541024)(841.23216733,308.8904102)(841.24216797,308.94041504)
\curveto(841.27216729,309.06041003)(841.29716727,309.18040991)(841.31716797,309.30041504)
\curveto(841.34716722,309.42040967)(841.38716718,309.53540955)(841.43716797,309.64541504)
\curveto(841.58716698,310.01540907)(841.7671668,310.34540874)(841.97716797,310.63541504)
\curveto(842.19716637,310.93540815)(842.4621661,311.1854079)(842.77216797,311.38541504)
\curveto(842.89216567,311.46540762)(843.01716555,311.53040756)(843.14716797,311.58041504)
\curveto(843.27716529,311.64040745)(843.41216515,311.70040739)(843.55216797,311.76041504)
\curveto(843.67216489,311.81040728)(843.80216476,311.84040725)(843.94216797,311.85041504)
\curveto(844.08216448,311.87040722)(844.22216434,311.90040719)(844.36216797,311.94041504)
\lineto(844.55716797,311.94041504)
\curveto(844.62716394,311.95040714)(844.69216387,311.96040713)(844.75216797,311.97041504)
\curveto(845.64216292,311.98040711)(846.38216218,311.79540729)(846.97216797,311.41541504)
\curveto(847.562161,311.03540805)(847.98716058,310.54040855)(848.24716797,309.93041504)
\curveto(848.29716027,309.83040926)(848.33716023,309.73040936)(848.36716797,309.63041504)
\curveto(848.39716017,309.53040956)(848.43216013,309.42540966)(848.47216797,309.31541504)
\curveto(848.50216006,309.20540988)(848.52716004,309.08541)(848.54716797,308.95541504)
\curveto(848.56716,308.83541025)(848.59215997,308.71041038)(848.62216797,308.58041504)
\curveto(848.63215993,308.53041056)(848.63215993,308.47541061)(848.62216797,308.41541504)
\curveto(848.62215994,308.36541072)(848.62715994,308.31541077)(848.63716797,308.26541504)
\moveto(847.30216797,307.41041504)
\curveto(847.32216124,307.48041161)(847.32716124,307.56041153)(847.31716797,307.65041504)
\lineto(847.31716797,307.90541504)
\curveto(847.31716125,308.29541079)(847.28216128,308.62541046)(847.21216797,308.89541504)
\curveto(847.18216138,308.97541011)(847.15716141,309.05541003)(847.13716797,309.13541504)
\curveto(847.11716145,309.21540987)(847.09216147,309.2904098)(847.06216797,309.36041504)
\curveto(846.78216178,310.01040908)(846.33716223,310.46040863)(845.72716797,310.71041504)
\curveto(845.65716291,310.74040835)(845.58216298,310.76040833)(845.50216797,310.77041504)
\lineto(845.26216797,310.83041504)
\curveto(845.18216338,310.85040824)(845.09716347,310.86040823)(845.00716797,310.86041504)
\lineto(844.73716797,310.86041504)
\lineto(844.46716797,310.81541504)
\curveto(844.3671642,310.79540829)(844.27216429,310.77040832)(844.18216797,310.74041504)
\curveto(844.10216446,310.72040837)(844.02216454,310.6904084)(843.94216797,310.65041504)
\curveto(843.87216469,310.63040846)(843.80716476,310.60040849)(843.74716797,310.56041504)
\curveto(843.68716488,310.52040857)(843.63216493,310.48040861)(843.58216797,310.44041504)
\curveto(843.34216522,310.27040882)(843.14716542,310.06540902)(842.99716797,309.82541504)
\curveto(842.84716572,309.5854095)(842.71716585,309.30540978)(842.60716797,308.98541504)
\curveto(842.57716599,308.8854102)(842.55716601,308.78041031)(842.54716797,308.67041504)
\curveto(842.53716603,308.57041052)(842.52216604,308.46541062)(842.50216797,308.35541504)
\curveto(842.49216607,308.31541077)(842.48716608,308.25041084)(842.48716797,308.16041504)
\curveto(842.47716609,308.13041096)(842.47216609,308.09541099)(842.47216797,308.05541504)
\curveto(842.48216608,308.01541107)(842.48716608,307.97041112)(842.48716797,307.92041504)
\lineto(842.48716797,307.62041504)
\curveto(842.48716608,307.52041157)(842.49716607,307.43041166)(842.51716797,307.35041504)
\lineto(842.54716797,307.17041504)
\curveto(842.567166,307.07041202)(842.58216598,306.97041212)(842.59216797,306.87041504)
\curveto(842.61216595,306.78041231)(842.64216592,306.69541239)(842.68216797,306.61541504)
\curveto(842.78216578,306.37541271)(842.89716567,306.15041294)(843.02716797,305.94041504)
\curveto(843.1671654,305.73041336)(843.33716523,305.55541353)(843.53716797,305.41541504)
\curveto(843.58716498,305.3854137)(843.63216493,305.36041373)(843.67216797,305.34041504)
\curveto(843.71216485,305.32041377)(843.75716481,305.29541379)(843.80716797,305.26541504)
\curveto(843.88716468,305.21541387)(843.97216459,305.17041392)(844.06216797,305.13041504)
\curveto(844.1621644,305.10041399)(844.2671643,305.07041402)(844.37716797,305.04041504)
\curveto(844.42716414,305.02041407)(844.47216409,305.01041408)(844.51216797,305.01041504)
\curveto(844.562164,305.02041407)(844.61216395,305.02041407)(844.66216797,305.01041504)
\curveto(844.69216387,305.00041409)(844.75216381,304.9904141)(844.84216797,304.98041504)
\curveto(844.94216362,304.97041412)(845.01716355,304.97541411)(845.06716797,304.99541504)
\curveto(845.10716346,305.00541408)(845.14716342,305.00541408)(845.18716797,304.99541504)
\curveto(845.22716334,304.99541409)(845.2671633,305.00541408)(845.30716797,305.02541504)
\curveto(845.38716318,305.04541404)(845.4671631,305.06041403)(845.54716797,305.07041504)
\curveto(845.62716294,305.090414)(845.70216286,305.11541397)(845.77216797,305.14541504)
\curveto(846.11216245,305.2854138)(846.38716218,305.48041361)(846.59716797,305.73041504)
\curveto(846.80716176,305.98041311)(846.98216158,306.27541281)(847.12216797,306.61541504)
\curveto(847.17216139,306.73541235)(847.20216136,306.86041223)(847.21216797,306.99041504)
\curveto(847.23216133,307.13041196)(847.2621613,307.27041182)(847.30216797,307.41041504)
}
}
{
\newrgbcolor{curcolor}{0 0 0}
\pscustom[linestyle=none,fillstyle=solid,fillcolor=curcolor]
{
\newpath
\moveto(775.9614209,270.63770508)
\lineto(776.3964209,270.63770508)
\curveto(776.54641893,270.63769734)(776.65141883,270.59769738)(776.7114209,270.51770508)
\curveto(776.76141872,270.43769754)(776.78641869,270.33769764)(776.7864209,270.21770508)
\curveto(776.79641868,270.09769788)(776.80141868,269.977698)(776.8014209,269.85770508)
\lineto(776.8014209,268.43270508)
\lineto(776.8014209,266.16770508)
\lineto(776.8014209,265.47770508)
\curveto(776.80141868,265.24770273)(776.82641865,265.04770293)(776.8764209,264.87770508)
\curveto(777.03641844,264.42770355)(777.33641814,264.11270386)(777.7764209,263.93270508)
\curveto(777.99641748,263.84270413)(778.26141722,263.80770417)(778.5714209,263.82770508)
\curveto(778.8814166,263.85770412)(779.13141635,263.91270406)(779.3214209,263.99270508)
\curveto(779.65141583,264.13270384)(779.91141557,264.30770367)(780.1014209,264.51770508)
\curveto(780.30141518,264.73770324)(780.45641502,265.02270295)(780.5664209,265.37270508)
\curveto(780.59641488,265.45270252)(780.61641486,265.53270244)(780.6264209,265.61270508)
\curveto(780.63641484,265.69270228)(780.65141483,265.7777022)(780.6714209,265.86770508)
\curveto(780.6814148,265.91770206)(780.6814148,265.96270201)(780.6714209,266.00270508)
\curveto(780.67141481,266.04270193)(780.6814148,266.08770189)(780.7014209,266.13770508)
\lineto(780.7014209,266.45270508)
\curveto(780.72141476,266.53270144)(780.72641475,266.62270135)(780.7164209,266.72270508)
\curveto(780.70641477,266.83270114)(780.70141478,266.93270104)(780.7014209,267.02270508)
\lineto(780.7014209,268.19270508)
\lineto(780.7014209,269.78270508)
\curveto(780.70141478,269.90269807)(780.69641478,270.02769795)(780.6864209,270.15770508)
\curveto(780.68641479,270.29769768)(780.71141477,270.40769757)(780.7614209,270.48770508)
\curveto(780.80141468,270.53769744)(780.84641463,270.56769741)(780.8964209,270.57770508)
\curveto(780.95641452,270.59769738)(781.02641445,270.61769736)(781.1064209,270.63770508)
\lineto(781.3314209,270.63770508)
\curveto(781.45141403,270.63769734)(781.55641392,270.63269734)(781.6464209,270.62270508)
\curveto(781.74641373,270.61269736)(781.82141366,270.56769741)(781.8714209,270.48770508)
\curveto(781.92141356,270.43769754)(781.94641353,270.36269761)(781.9464209,270.26270508)
\lineto(781.9464209,269.97770508)
\lineto(781.9464209,268.95770508)
\lineto(781.9464209,264.92270508)
\lineto(781.9464209,263.57270508)
\curveto(781.94641353,263.45270452)(781.94141354,263.33770464)(781.9314209,263.22770508)
\curveto(781.93141355,263.12770485)(781.89641358,263.05270492)(781.8264209,263.00270508)
\curveto(781.78641369,262.972705)(781.72641375,262.94770503)(781.6464209,262.92770508)
\curveto(781.56641391,262.91770506)(781.476414,262.90770507)(781.3764209,262.89770508)
\curveto(781.28641419,262.89770508)(781.19641428,262.90270507)(781.1064209,262.91270508)
\curveto(781.02641445,262.92270505)(780.96641451,262.94270503)(780.9264209,262.97270508)
\curveto(780.8764146,263.01270496)(780.83141465,263.0777049)(780.7914209,263.16770508)
\curveto(780.7814147,263.20770477)(780.77141471,263.26270471)(780.7614209,263.33270508)
\curveto(780.76141472,263.40270457)(780.75641472,263.46770451)(780.7464209,263.52770508)
\curveto(780.73641474,263.59770438)(780.71641476,263.65270432)(780.6864209,263.69270508)
\curveto(780.65641482,263.73270424)(780.61141487,263.74770423)(780.5514209,263.73770508)
\curveto(780.47141501,263.71770426)(780.39141509,263.65770432)(780.3114209,263.55770508)
\curveto(780.23141525,263.46770451)(780.15641532,263.39770458)(780.0864209,263.34770508)
\curveto(779.86641561,263.18770479)(779.61641586,263.04770493)(779.3364209,262.92770508)
\curveto(779.22641625,262.8777051)(779.11141637,262.84770513)(778.9914209,262.83770508)
\curveto(778.8814166,262.81770516)(778.76641671,262.79270518)(778.6464209,262.76270508)
\curveto(778.59641688,262.75270522)(778.54141694,262.75270522)(778.4814209,262.76270508)
\curveto(778.43141705,262.7727052)(778.3814171,262.76770521)(778.3314209,262.74770508)
\curveto(778.23141725,262.72770525)(778.14141734,262.72770525)(778.0614209,262.74770508)
\lineto(777.9114209,262.74770508)
\curveto(777.86141762,262.76770521)(777.80141768,262.7777052)(777.7314209,262.77770508)
\curveto(777.67141781,262.7777052)(777.61641786,262.78270519)(777.5664209,262.79270508)
\curveto(777.52641795,262.81270516)(777.48641799,262.82270515)(777.4464209,262.82270508)
\curveto(777.41641806,262.81270516)(777.3764181,262.81770516)(777.3264209,262.83770508)
\lineto(777.0864209,262.89770508)
\curveto(777.01641846,262.91770506)(776.94141854,262.94770503)(776.8614209,262.98770508)
\curveto(776.60141888,263.09770488)(776.3814191,263.24270473)(776.2014209,263.42270508)
\curveto(776.03141945,263.61270436)(775.89141959,263.83770414)(775.7814209,264.09770508)
\curveto(775.74141974,264.18770379)(775.71141977,264.2777037)(775.6914209,264.36770508)
\lineto(775.6314209,264.66770508)
\curveto(775.61141987,264.72770325)(775.60141988,264.78270319)(775.6014209,264.83270508)
\curveto(775.61141987,264.89270308)(775.60641987,264.95770302)(775.5864209,265.02770508)
\curveto(775.5764199,265.04770293)(775.57141991,265.0727029)(775.5714209,265.10270508)
\curveto(775.57141991,265.14270283)(775.56641991,265.1777028)(775.5564209,265.20770508)
\lineto(775.5564209,265.35770508)
\curveto(775.54641993,265.39770258)(775.54141994,265.44270253)(775.5414209,265.49270508)
\curveto(775.55141993,265.55270242)(775.55641992,265.60770237)(775.5564209,265.65770508)
\lineto(775.5564209,266.25770508)
\lineto(775.5564209,269.01770508)
\lineto(775.5564209,269.97770508)
\lineto(775.5564209,270.24770508)
\curveto(775.55641992,270.33769764)(775.5764199,270.41269756)(775.6164209,270.47270508)
\curveto(775.65641982,270.54269743)(775.73141975,270.59269738)(775.8414209,270.62270508)
\curveto(775.86141962,270.63269734)(775.8814196,270.63269734)(775.9014209,270.62270508)
\curveto(775.92141956,270.62269735)(775.94141954,270.62769735)(775.9614209,270.63770508)
}
}
{
\newrgbcolor{curcolor}{0 0 0}
\pscustom[linestyle=none,fillstyle=solid,fillcolor=curcolor]
{
\newpath
\moveto(786.27603027,270.81770508)
\curveto(786.99602621,270.82769715)(787.6010256,270.74269723)(788.09103027,270.56270508)
\curveto(788.58102462,270.39269758)(788.96102424,270.08769789)(789.23103027,269.64770508)
\curveto(789.3010239,269.53769844)(789.35602385,269.42269855)(789.39603027,269.30270508)
\curveto(789.43602377,269.19269878)(789.47602373,269.06769891)(789.51603027,268.92770508)
\curveto(789.53602367,268.85769912)(789.54102366,268.78269919)(789.53103027,268.70270508)
\curveto(789.52102368,268.63269934)(789.5060237,268.5776994)(789.48603027,268.53770508)
\curveto(789.46602374,268.51769946)(789.44102376,268.49769948)(789.41103027,268.47770508)
\curveto(789.38102382,268.46769951)(789.35602385,268.45269952)(789.33603027,268.43270508)
\curveto(789.28602392,268.41269956)(789.23602397,268.40769957)(789.18603027,268.41770508)
\curveto(789.13602407,268.42769955)(789.08602412,268.42769955)(789.03603027,268.41770508)
\curveto(788.95602425,268.39769958)(788.85102435,268.39269958)(788.72103027,268.40270508)
\curveto(788.59102461,268.42269955)(788.5010247,268.44769953)(788.45103027,268.47770508)
\curveto(788.37102483,268.52769945)(788.31602489,268.59269938)(788.28603027,268.67270508)
\curveto(788.26602494,268.76269921)(788.23102497,268.84769913)(788.18103027,268.92770508)
\curveto(788.09102511,269.08769889)(787.96602524,269.23269874)(787.80603027,269.36270508)
\curveto(787.69602551,269.44269853)(787.57602563,269.50269847)(787.44603027,269.54270508)
\curveto(787.31602589,269.58269839)(787.17602603,269.62269835)(787.02603027,269.66270508)
\curveto(786.97602623,269.68269829)(786.92602628,269.68769829)(786.87603027,269.67770508)
\curveto(786.82602638,269.6776983)(786.77602643,269.68269829)(786.72603027,269.69270508)
\curveto(786.66602654,269.71269826)(786.59102661,269.72269825)(786.50103027,269.72270508)
\curveto(786.41102679,269.72269825)(786.33602687,269.71269826)(786.27603027,269.69270508)
\lineto(786.18603027,269.69270508)
\lineto(786.03603027,269.66270508)
\curveto(785.98602722,269.66269831)(785.93602727,269.65769832)(785.88603027,269.64770508)
\curveto(785.62602758,269.58769839)(785.41102779,269.50269847)(785.24103027,269.39270508)
\curveto(785.07102813,269.28269869)(784.95602825,269.09769888)(784.89603027,268.83770508)
\curveto(784.87602833,268.76769921)(784.87102833,268.69769928)(784.88103027,268.62770508)
\curveto(784.9010283,268.55769942)(784.92102828,268.49769948)(784.94103027,268.44770508)
\curveto(785.0010282,268.29769968)(785.07102813,268.18769979)(785.15103027,268.11770508)
\curveto(785.24102796,268.05769992)(785.35102785,267.98769999)(785.48103027,267.90770508)
\curveto(785.64102756,267.80770017)(785.82102738,267.73270024)(786.02103027,267.68270508)
\curveto(786.22102698,267.64270033)(786.42102678,267.59270038)(786.62103027,267.53270508)
\curveto(786.75102645,267.49270048)(786.88102632,267.46270051)(787.01103027,267.44270508)
\curveto(787.14102606,267.42270055)(787.27102593,267.39270058)(787.40103027,267.35270508)
\curveto(787.61102559,267.29270068)(787.81602539,267.23270074)(788.01603027,267.17270508)
\curveto(788.21602499,267.12270085)(788.41602479,267.05770092)(788.61603027,266.97770508)
\lineto(788.76603027,266.91770508)
\curveto(788.81602439,266.89770108)(788.86602434,266.8727011)(788.91603027,266.84270508)
\curveto(789.11602409,266.72270125)(789.29102391,266.58770139)(789.44103027,266.43770508)
\curveto(789.59102361,266.28770169)(789.71602349,266.09770188)(789.81603027,265.86770508)
\curveto(789.83602337,265.79770218)(789.85602335,265.70270227)(789.87603027,265.58270508)
\curveto(789.89602331,265.51270246)(789.9060233,265.43770254)(789.90603027,265.35770508)
\curveto(789.91602329,265.28770269)(789.92102328,265.20770277)(789.92103027,265.11770508)
\lineto(789.92103027,264.96770508)
\curveto(789.9010233,264.89770308)(789.89102331,264.82770315)(789.89103027,264.75770508)
\curveto(789.89102331,264.68770329)(789.88102332,264.61770336)(789.86103027,264.54770508)
\curveto(789.83102337,264.43770354)(789.79602341,264.33270364)(789.75603027,264.23270508)
\curveto(789.71602349,264.13270384)(789.67102353,264.04270393)(789.62103027,263.96270508)
\curveto(789.46102374,263.70270427)(789.25602395,263.49270448)(789.00603027,263.33270508)
\curveto(788.75602445,263.18270479)(788.47602473,263.05270492)(788.16603027,262.94270508)
\curveto(788.07602513,262.91270506)(787.98102522,262.89270508)(787.88103027,262.88270508)
\curveto(787.79102541,262.86270511)(787.7010255,262.83770514)(787.61103027,262.80770508)
\curveto(787.51102569,262.78770519)(787.41102579,262.7777052)(787.31103027,262.77770508)
\curveto(787.21102599,262.7777052)(787.11102609,262.76770521)(787.01103027,262.74770508)
\lineto(786.86103027,262.74770508)
\curveto(786.81102639,262.73770524)(786.74102646,262.73270524)(786.65103027,262.73270508)
\curveto(786.56102664,262.73270524)(786.49102671,262.73770524)(786.44103027,262.74770508)
\lineto(786.27603027,262.74770508)
\curveto(786.21602699,262.76770521)(786.15102705,262.7777052)(786.08103027,262.77770508)
\curveto(786.01102719,262.76770521)(785.95102725,262.7727052)(785.90103027,262.79270508)
\curveto(785.85102735,262.80270517)(785.78602742,262.80770517)(785.70603027,262.80770508)
\lineto(785.46603027,262.86770508)
\curveto(785.39602781,262.8777051)(785.32102788,262.89770508)(785.24103027,262.92770508)
\curveto(784.93102827,263.02770495)(784.66102854,263.15270482)(784.43103027,263.30270508)
\curveto(784.201029,263.45270452)(784.0010292,263.64770433)(783.83103027,263.88770508)
\curveto(783.74102946,264.01770396)(783.66602954,264.15270382)(783.60603027,264.29270508)
\curveto(783.54602966,264.43270354)(783.49102971,264.58770339)(783.44103027,264.75770508)
\curveto(783.42102978,264.81770316)(783.41102979,264.88770309)(783.41103027,264.96770508)
\curveto(783.42102978,265.05770292)(783.43602977,265.12770285)(783.45603027,265.17770508)
\curveto(783.48602972,265.21770276)(783.53602967,265.25770272)(783.60603027,265.29770508)
\curveto(783.65602955,265.31770266)(783.72602948,265.32770265)(783.81603027,265.32770508)
\curveto(783.9060293,265.33770264)(783.99602921,265.33770264)(784.08603027,265.32770508)
\curveto(784.17602903,265.31770266)(784.26102894,265.30270267)(784.34103027,265.28270508)
\curveto(784.43102877,265.2727027)(784.49102871,265.25770272)(784.52103027,265.23770508)
\curveto(784.59102861,265.18770279)(784.63602857,265.11270286)(784.65603027,265.01270508)
\curveto(784.68602852,264.92270305)(784.72102848,264.83770314)(784.76103027,264.75770508)
\curveto(784.86102834,264.53770344)(784.99602821,264.36770361)(785.16603027,264.24770508)
\curveto(785.28602792,264.15770382)(785.42102778,264.08770389)(785.57103027,264.03770508)
\curveto(785.72102748,263.98770399)(785.88102732,263.93770404)(786.05103027,263.88770508)
\lineto(786.36603027,263.84270508)
\lineto(786.45603027,263.84270508)
\curveto(786.52602668,263.82270415)(786.61602659,263.81270416)(786.72603027,263.81270508)
\curveto(786.84602636,263.81270416)(786.94602626,263.82270415)(787.02603027,263.84270508)
\curveto(787.09602611,263.84270413)(787.15102605,263.84770413)(787.19103027,263.85770508)
\curveto(787.25102595,263.86770411)(787.31102589,263.8727041)(787.37103027,263.87270508)
\curveto(787.43102577,263.88270409)(787.48602572,263.89270408)(787.53603027,263.90270508)
\curveto(787.82602538,263.98270399)(788.05602515,264.08770389)(788.22603027,264.21770508)
\curveto(788.39602481,264.34770363)(788.51602469,264.56770341)(788.58603027,264.87770508)
\curveto(788.6060246,264.92770305)(788.61102459,264.98270299)(788.60103027,265.04270508)
\curveto(788.59102461,265.10270287)(788.58102462,265.14770283)(788.57103027,265.17770508)
\curveto(788.52102468,265.36770261)(788.45102475,265.50770247)(788.36103027,265.59770508)
\curveto(788.27102493,265.69770228)(788.15602505,265.78770219)(788.01603027,265.86770508)
\curveto(787.92602528,265.92770205)(787.82602538,265.977702)(787.71603027,266.01770508)
\lineto(787.38603027,266.13770508)
\curveto(787.35602585,266.14770183)(787.32602588,266.15270182)(787.29603027,266.15270508)
\curveto(787.27602593,266.15270182)(787.25102595,266.16270181)(787.22103027,266.18270508)
\curveto(786.88102632,266.29270168)(786.52602668,266.3727016)(786.15603027,266.42270508)
\curveto(785.79602741,266.48270149)(785.45602775,266.5777014)(785.13603027,266.70770508)
\curveto(785.03602817,266.74770123)(784.94102826,266.78270119)(784.85103027,266.81270508)
\curveto(784.76102844,266.84270113)(784.67602853,266.88270109)(784.59603027,266.93270508)
\curveto(784.4060288,267.04270093)(784.23102897,267.16770081)(784.07103027,267.30770508)
\curveto(783.91102929,267.44770053)(783.78602942,267.62270035)(783.69603027,267.83270508)
\curveto(783.66602954,267.90270007)(783.64102956,267.9727)(783.62103027,268.04270508)
\curveto(783.61102959,268.11269986)(783.59602961,268.18769979)(783.57603027,268.26770508)
\curveto(783.54602966,268.38769959)(783.53602967,268.52269945)(783.54603027,268.67270508)
\curveto(783.55602965,268.83269914)(783.57102963,268.96769901)(783.59103027,269.07770508)
\curveto(783.61102959,269.12769885)(783.62102958,269.16769881)(783.62103027,269.19770508)
\curveto(783.63102957,269.23769874)(783.64602956,269.2776987)(783.66603027,269.31770508)
\curveto(783.75602945,269.54769843)(783.87602933,269.74769823)(784.02603027,269.91770508)
\curveto(784.18602902,270.08769789)(784.36602884,270.23769774)(784.56603027,270.36770508)
\curveto(784.71602849,270.45769752)(784.88102832,270.52769745)(785.06103027,270.57770508)
\curveto(785.24102796,270.63769734)(785.43102777,270.69269728)(785.63103027,270.74270508)
\curveto(785.7010275,270.75269722)(785.76602744,270.76269721)(785.82603027,270.77270508)
\curveto(785.89602731,270.78269719)(785.97102723,270.79269718)(786.05103027,270.80270508)
\curveto(786.08102712,270.81269716)(786.12102708,270.81269716)(786.17103027,270.80270508)
\curveto(786.22102698,270.79269718)(786.25602695,270.79769718)(786.27603027,270.81770508)
}
}
{
\newrgbcolor{curcolor}{0 0 0}
\pscustom[linestyle=none,fillstyle=solid,fillcolor=curcolor]
{
\newpath
\moveto(791.81103027,270.63770508)
\lineto(792.24603027,270.63770508)
\curveto(792.39602831,270.63769734)(792.5010282,270.59769738)(792.56103027,270.51770508)
\curveto(792.61102809,270.43769754)(792.63602807,270.33769764)(792.63603027,270.21770508)
\curveto(792.64602806,270.09769788)(792.65102805,269.977698)(792.65103027,269.85770508)
\lineto(792.65103027,268.43270508)
\lineto(792.65103027,266.16770508)
\lineto(792.65103027,265.47770508)
\curveto(792.65102805,265.24770273)(792.67602803,265.04770293)(792.72603027,264.87770508)
\curveto(792.88602782,264.42770355)(793.18602752,264.11270386)(793.62603027,263.93270508)
\curveto(793.84602686,263.84270413)(794.11102659,263.80770417)(794.42103027,263.82770508)
\curveto(794.73102597,263.85770412)(794.98102572,263.91270406)(795.17103027,263.99270508)
\curveto(795.5010252,264.13270384)(795.76102494,264.30770367)(795.95103027,264.51770508)
\curveto(796.15102455,264.73770324)(796.3060244,265.02270295)(796.41603027,265.37270508)
\curveto(796.44602426,265.45270252)(796.46602424,265.53270244)(796.47603027,265.61270508)
\curveto(796.48602422,265.69270228)(796.5010242,265.7777022)(796.52103027,265.86770508)
\curveto(796.53102417,265.91770206)(796.53102417,265.96270201)(796.52103027,266.00270508)
\curveto(796.52102418,266.04270193)(796.53102417,266.08770189)(796.55103027,266.13770508)
\lineto(796.55103027,266.45270508)
\curveto(796.57102413,266.53270144)(796.57602413,266.62270135)(796.56603027,266.72270508)
\curveto(796.55602415,266.83270114)(796.55102415,266.93270104)(796.55103027,267.02270508)
\lineto(796.55103027,268.19270508)
\lineto(796.55103027,269.78270508)
\curveto(796.55102415,269.90269807)(796.54602416,270.02769795)(796.53603027,270.15770508)
\curveto(796.53602417,270.29769768)(796.56102414,270.40769757)(796.61103027,270.48770508)
\curveto(796.65102405,270.53769744)(796.69602401,270.56769741)(796.74603027,270.57770508)
\curveto(796.8060239,270.59769738)(796.87602383,270.61769736)(796.95603027,270.63770508)
\lineto(797.18103027,270.63770508)
\curveto(797.3010234,270.63769734)(797.4060233,270.63269734)(797.49603027,270.62270508)
\curveto(797.59602311,270.61269736)(797.67102303,270.56769741)(797.72103027,270.48770508)
\curveto(797.77102293,270.43769754)(797.79602291,270.36269761)(797.79603027,270.26270508)
\lineto(797.79603027,269.97770508)
\lineto(797.79603027,268.95770508)
\lineto(797.79603027,264.92270508)
\lineto(797.79603027,263.57270508)
\curveto(797.79602291,263.45270452)(797.79102291,263.33770464)(797.78103027,263.22770508)
\curveto(797.78102292,263.12770485)(797.74602296,263.05270492)(797.67603027,263.00270508)
\curveto(797.63602307,262.972705)(797.57602313,262.94770503)(797.49603027,262.92770508)
\curveto(797.41602329,262.91770506)(797.32602338,262.90770507)(797.22603027,262.89770508)
\curveto(797.13602357,262.89770508)(797.04602366,262.90270507)(796.95603027,262.91270508)
\curveto(796.87602383,262.92270505)(796.81602389,262.94270503)(796.77603027,262.97270508)
\curveto(796.72602398,263.01270496)(796.68102402,263.0777049)(796.64103027,263.16770508)
\curveto(796.63102407,263.20770477)(796.62102408,263.26270471)(796.61103027,263.33270508)
\curveto(796.61102409,263.40270457)(796.6060241,263.46770451)(796.59603027,263.52770508)
\curveto(796.58602412,263.59770438)(796.56602414,263.65270432)(796.53603027,263.69270508)
\curveto(796.5060242,263.73270424)(796.46102424,263.74770423)(796.40103027,263.73770508)
\curveto(796.32102438,263.71770426)(796.24102446,263.65770432)(796.16103027,263.55770508)
\curveto(796.08102462,263.46770451)(796.0060247,263.39770458)(795.93603027,263.34770508)
\curveto(795.71602499,263.18770479)(795.46602524,263.04770493)(795.18603027,262.92770508)
\curveto(795.07602563,262.8777051)(794.96102574,262.84770513)(794.84103027,262.83770508)
\curveto(794.73102597,262.81770516)(794.61602609,262.79270518)(794.49603027,262.76270508)
\curveto(794.44602626,262.75270522)(794.39102631,262.75270522)(794.33103027,262.76270508)
\curveto(794.28102642,262.7727052)(794.23102647,262.76770521)(794.18103027,262.74770508)
\curveto(794.08102662,262.72770525)(793.99102671,262.72770525)(793.91103027,262.74770508)
\lineto(793.76103027,262.74770508)
\curveto(793.71102699,262.76770521)(793.65102705,262.7777052)(793.58103027,262.77770508)
\curveto(793.52102718,262.7777052)(793.46602724,262.78270519)(793.41603027,262.79270508)
\curveto(793.37602733,262.81270516)(793.33602737,262.82270515)(793.29603027,262.82270508)
\curveto(793.26602744,262.81270516)(793.22602748,262.81770516)(793.17603027,262.83770508)
\lineto(792.93603027,262.89770508)
\curveto(792.86602784,262.91770506)(792.79102791,262.94770503)(792.71103027,262.98770508)
\curveto(792.45102825,263.09770488)(792.23102847,263.24270473)(792.05103027,263.42270508)
\curveto(791.88102882,263.61270436)(791.74102896,263.83770414)(791.63103027,264.09770508)
\curveto(791.59102911,264.18770379)(791.56102914,264.2777037)(791.54103027,264.36770508)
\lineto(791.48103027,264.66770508)
\curveto(791.46102924,264.72770325)(791.45102925,264.78270319)(791.45103027,264.83270508)
\curveto(791.46102924,264.89270308)(791.45602925,264.95770302)(791.43603027,265.02770508)
\curveto(791.42602928,265.04770293)(791.42102928,265.0727029)(791.42103027,265.10270508)
\curveto(791.42102928,265.14270283)(791.41602929,265.1777028)(791.40603027,265.20770508)
\lineto(791.40603027,265.35770508)
\curveto(791.39602931,265.39770258)(791.39102931,265.44270253)(791.39103027,265.49270508)
\curveto(791.4010293,265.55270242)(791.4060293,265.60770237)(791.40603027,265.65770508)
\lineto(791.40603027,266.25770508)
\lineto(791.40603027,269.01770508)
\lineto(791.40603027,269.97770508)
\lineto(791.40603027,270.24770508)
\curveto(791.4060293,270.33769764)(791.42602928,270.41269756)(791.46603027,270.47270508)
\curveto(791.5060292,270.54269743)(791.58102912,270.59269738)(791.69103027,270.62270508)
\curveto(791.71102899,270.63269734)(791.73102897,270.63269734)(791.75103027,270.62270508)
\curveto(791.77102893,270.62269735)(791.79102891,270.62769735)(791.81103027,270.63770508)
}
}
{
\newrgbcolor{curcolor}{0 0 0}
\pscustom[linestyle=none,fillstyle=solid,fillcolor=curcolor]
{
\newpath
\moveto(806.58063965,263.46770508)
\curveto(806.61063182,263.30770467)(806.59563183,263.1727048)(806.53563965,263.06270508)
\curveto(806.47563195,262.96270501)(806.39563203,262.88770509)(806.29563965,262.83770508)
\curveto(806.24563218,262.81770516)(806.19063224,262.80770517)(806.13063965,262.80770508)
\curveto(806.08063235,262.80770517)(806.0256324,262.79770518)(805.96563965,262.77770508)
\curveto(805.74563268,262.72770525)(805.5256329,262.74270523)(805.30563965,262.82270508)
\curveto(805.09563333,262.89270508)(804.95063348,262.98270499)(804.87063965,263.09270508)
\curveto(804.82063361,263.16270481)(804.77563365,263.24270473)(804.73563965,263.33270508)
\curveto(804.69563373,263.43270454)(804.64563378,263.51270446)(804.58563965,263.57270508)
\curveto(804.56563386,263.59270438)(804.54063389,263.61270436)(804.51063965,263.63270508)
\curveto(804.49063394,263.65270432)(804.46063397,263.65770432)(804.42063965,263.64770508)
\curveto(804.31063412,263.61770436)(804.20563422,263.56270441)(804.10563965,263.48270508)
\curveto(804.01563441,263.40270457)(803.9256345,263.33270464)(803.83563965,263.27270508)
\curveto(803.70563472,263.19270478)(803.56563486,263.11770486)(803.41563965,263.04770508)
\curveto(803.26563516,262.98770499)(803.10563532,262.93270504)(802.93563965,262.88270508)
\curveto(802.83563559,262.85270512)(802.7256357,262.83270514)(802.60563965,262.82270508)
\curveto(802.49563593,262.81270516)(802.38563604,262.79770518)(802.27563965,262.77770508)
\curveto(802.2256362,262.76770521)(802.18063625,262.76270521)(802.14063965,262.76270508)
\lineto(802.03563965,262.76270508)
\curveto(801.9256365,262.74270523)(801.82063661,262.74270523)(801.72063965,262.76270508)
\lineto(801.58563965,262.76270508)
\curveto(801.53563689,262.7727052)(801.48563694,262.7777052)(801.43563965,262.77770508)
\curveto(801.38563704,262.7777052)(801.34063709,262.78770519)(801.30063965,262.80770508)
\curveto(801.26063717,262.81770516)(801.2256372,262.82270515)(801.19563965,262.82270508)
\curveto(801.17563725,262.81270516)(801.15063728,262.81270516)(801.12063965,262.82270508)
\lineto(800.88063965,262.88270508)
\curveto(800.80063763,262.89270508)(800.7256377,262.91270506)(800.65563965,262.94270508)
\curveto(800.35563807,263.0727049)(800.11063832,263.21770476)(799.92063965,263.37770508)
\curveto(799.74063869,263.54770443)(799.59063884,263.78270419)(799.47063965,264.08270508)
\curveto(799.38063905,264.30270367)(799.33563909,264.56770341)(799.33563965,264.87770508)
\lineto(799.33563965,265.19270508)
\curveto(799.34563908,265.24270273)(799.35063908,265.29270268)(799.35063965,265.34270508)
\lineto(799.38063965,265.52270508)
\lineto(799.50063965,265.85270508)
\curveto(799.54063889,265.96270201)(799.59063884,266.06270191)(799.65063965,266.15270508)
\curveto(799.8306386,266.44270153)(800.07563835,266.65770132)(800.38563965,266.79770508)
\curveto(800.69563773,266.93770104)(801.03563739,267.06270091)(801.40563965,267.17270508)
\curveto(801.54563688,267.21270076)(801.69063674,267.24270073)(801.84063965,267.26270508)
\curveto(801.99063644,267.28270069)(802.14063629,267.30770067)(802.29063965,267.33770508)
\curveto(802.36063607,267.35770062)(802.425636,267.36770061)(802.48563965,267.36770508)
\curveto(802.55563587,267.36770061)(802.6306358,267.3777006)(802.71063965,267.39770508)
\curveto(802.78063565,267.41770056)(802.85063558,267.42770055)(802.92063965,267.42770508)
\curveto(802.99063544,267.43770054)(803.06563536,267.45270052)(803.14563965,267.47270508)
\curveto(803.39563503,267.53270044)(803.6306348,267.58270039)(803.85063965,267.62270508)
\curveto(804.07063436,267.6727003)(804.24563418,267.78770019)(804.37563965,267.96770508)
\curveto(804.43563399,268.04769993)(804.48563394,268.14769983)(804.52563965,268.26770508)
\curveto(804.56563386,268.39769958)(804.56563386,268.53769944)(804.52563965,268.68770508)
\curveto(804.46563396,268.92769905)(804.37563405,269.11769886)(804.25563965,269.25770508)
\curveto(804.14563428,269.39769858)(803.98563444,269.50769847)(803.77563965,269.58770508)
\curveto(803.65563477,269.63769834)(803.51063492,269.6726983)(803.34063965,269.69270508)
\curveto(803.18063525,269.71269826)(803.01063542,269.72269825)(802.83063965,269.72270508)
\curveto(802.65063578,269.72269825)(802.47563595,269.71269826)(802.30563965,269.69270508)
\curveto(802.13563629,269.6726983)(801.99063644,269.64269833)(801.87063965,269.60270508)
\curveto(801.70063673,269.54269843)(801.53563689,269.45769852)(801.37563965,269.34770508)
\curveto(801.29563713,269.28769869)(801.22063721,269.20769877)(801.15063965,269.10770508)
\curveto(801.09063734,269.01769896)(801.03563739,268.91769906)(800.98563965,268.80770508)
\curveto(800.95563747,268.72769925)(800.9256375,268.64269933)(800.89563965,268.55270508)
\curveto(800.87563755,268.46269951)(800.8306376,268.39269958)(800.76063965,268.34270508)
\curveto(800.72063771,268.31269966)(800.65063778,268.28769969)(800.55063965,268.26770508)
\curveto(800.46063797,268.25769972)(800.36563806,268.25269972)(800.26563965,268.25270508)
\curveto(800.16563826,268.25269972)(800.06563836,268.25769972)(799.96563965,268.26770508)
\curveto(799.87563855,268.28769969)(799.81063862,268.31269966)(799.77063965,268.34270508)
\curveto(799.7306387,268.3726996)(799.70063873,268.42269955)(799.68063965,268.49270508)
\curveto(799.66063877,268.56269941)(799.66063877,268.63769934)(799.68063965,268.71770508)
\curveto(799.71063872,268.84769913)(799.74063869,268.96769901)(799.77063965,269.07770508)
\curveto(799.81063862,269.19769878)(799.85563857,269.31269866)(799.90563965,269.42270508)
\curveto(800.09563833,269.7726982)(800.33563809,270.04269793)(800.62563965,270.23270508)
\curveto(800.91563751,270.43269754)(801.27563715,270.59269738)(801.70563965,270.71270508)
\curveto(801.80563662,270.73269724)(801.90563652,270.74769723)(802.00563965,270.75770508)
\curveto(802.11563631,270.76769721)(802.2256362,270.78269719)(802.33563965,270.80270508)
\curveto(802.37563605,270.81269716)(802.44063599,270.81269716)(802.53063965,270.80270508)
\curveto(802.62063581,270.80269717)(802.67563575,270.81269716)(802.69563965,270.83270508)
\curveto(803.39563503,270.84269713)(804.00563442,270.76269721)(804.52563965,270.59270508)
\curveto(805.04563338,270.42269755)(805.41063302,270.09769788)(805.62063965,269.61770508)
\curveto(805.71063272,269.41769856)(805.76063267,269.18269879)(805.77063965,268.91270508)
\curveto(805.79063264,268.65269932)(805.80063263,268.3776996)(805.80063965,268.08770508)
\lineto(805.80063965,264.77270508)
\curveto(805.80063263,264.63270334)(805.80563262,264.49770348)(805.81563965,264.36770508)
\curveto(805.8256326,264.23770374)(805.85563257,264.13270384)(805.90563965,264.05270508)
\curveto(805.95563247,263.98270399)(806.02063241,263.93270404)(806.10063965,263.90270508)
\curveto(806.19063224,263.86270411)(806.27563215,263.83270414)(806.35563965,263.81270508)
\curveto(806.43563199,263.80270417)(806.49563193,263.75770422)(806.53563965,263.67770508)
\curveto(806.55563187,263.64770433)(806.56563186,263.61770436)(806.56563965,263.58770508)
\curveto(806.56563186,263.55770442)(806.57063186,263.51770446)(806.58063965,263.46770508)
\moveto(804.43563965,265.13270508)
\curveto(804.49563393,265.2727027)(804.5256339,265.43270254)(804.52563965,265.61270508)
\curveto(804.53563389,265.80270217)(804.54063389,265.99770198)(804.54063965,266.19770508)
\curveto(804.54063389,266.30770167)(804.53563389,266.40770157)(804.52563965,266.49770508)
\curveto(804.51563391,266.58770139)(804.47563395,266.65770132)(804.40563965,266.70770508)
\curveto(804.37563405,266.72770125)(804.30563412,266.73770124)(804.19563965,266.73770508)
\curveto(804.17563425,266.71770126)(804.14063429,266.70770127)(804.09063965,266.70770508)
\curveto(804.04063439,266.70770127)(803.99563443,266.69770128)(803.95563965,266.67770508)
\curveto(803.87563455,266.65770132)(803.78563464,266.63770134)(803.68563965,266.61770508)
\lineto(803.38563965,266.55770508)
\curveto(803.35563507,266.55770142)(803.32063511,266.55270142)(803.28063965,266.54270508)
\lineto(803.17563965,266.54270508)
\curveto(803.0256354,266.50270147)(802.86063557,266.4777015)(802.68063965,266.46770508)
\curveto(802.51063592,266.46770151)(802.35063608,266.44770153)(802.20063965,266.40770508)
\curveto(802.12063631,266.38770159)(802.04563638,266.36770161)(801.97563965,266.34770508)
\curveto(801.91563651,266.33770164)(801.84563658,266.32270165)(801.76563965,266.30270508)
\curveto(801.60563682,266.25270172)(801.45563697,266.18770179)(801.31563965,266.10770508)
\curveto(801.17563725,266.03770194)(801.05563737,265.94770203)(800.95563965,265.83770508)
\curveto(800.85563757,265.72770225)(800.78063765,265.59270238)(800.73063965,265.43270508)
\curveto(800.68063775,265.28270269)(800.66063777,265.09770288)(800.67063965,264.87770508)
\curveto(800.67063776,264.7777032)(800.68563774,264.68270329)(800.71563965,264.59270508)
\curveto(800.75563767,264.51270346)(800.80063763,264.43770354)(800.85063965,264.36770508)
\curveto(800.9306375,264.25770372)(801.03563739,264.16270381)(801.16563965,264.08270508)
\curveto(801.29563713,264.01270396)(801.43563699,263.95270402)(801.58563965,263.90270508)
\curveto(801.63563679,263.89270408)(801.68563674,263.88770409)(801.73563965,263.88770508)
\curveto(801.78563664,263.88770409)(801.83563659,263.88270409)(801.88563965,263.87270508)
\curveto(801.95563647,263.85270412)(802.04063639,263.83770414)(802.14063965,263.82770508)
\curveto(802.25063618,263.82770415)(802.34063609,263.83770414)(802.41063965,263.85770508)
\curveto(802.47063596,263.8777041)(802.5306359,263.88270409)(802.59063965,263.87270508)
\curveto(802.65063578,263.8727041)(802.71063572,263.88270409)(802.77063965,263.90270508)
\curveto(802.85063558,263.92270405)(802.9256355,263.93770404)(802.99563965,263.94770508)
\curveto(803.07563535,263.95770402)(803.15063528,263.977704)(803.22063965,264.00770508)
\curveto(803.51063492,264.12770385)(803.75563467,264.2727037)(803.95563965,264.44270508)
\curveto(804.16563426,264.61270336)(804.3256341,264.84270313)(804.43563965,265.13270508)
}
}
{
\newrgbcolor{curcolor}{0 0 0}
\pscustom[linestyle=none,fillstyle=solid,fillcolor=curcolor]
{
\newpath
\moveto(811.39728027,270.81770508)
\curveto(811.62727548,270.81769716)(811.75727535,270.75769722)(811.78728027,270.63770508)
\curveto(811.81727529,270.52769745)(811.83227528,270.36269761)(811.83228027,270.14270508)
\lineto(811.83228027,269.85770508)
\curveto(811.83227528,269.76769821)(811.8072753,269.69269828)(811.75728027,269.63270508)
\curveto(811.69727541,269.55269842)(811.6122755,269.50769847)(811.50228027,269.49770508)
\curveto(811.39227572,269.49769848)(811.28227583,269.48269849)(811.17228027,269.45270508)
\curveto(811.03227608,269.42269855)(810.89727621,269.39269858)(810.76728027,269.36270508)
\curveto(810.64727646,269.33269864)(810.53227658,269.29269868)(810.42228027,269.24270508)
\curveto(810.13227698,269.11269886)(809.89727721,268.93269904)(809.71728027,268.70270508)
\curveto(809.53727757,268.48269949)(809.38227773,268.22769975)(809.25228027,267.93770508)
\curveto(809.2122779,267.82770015)(809.18227793,267.71270026)(809.16228027,267.59270508)
\curveto(809.14227797,267.48270049)(809.11727799,267.36770061)(809.08728027,267.24770508)
\curveto(809.07727803,267.19770078)(809.07227804,267.14770083)(809.07228027,267.09770508)
\curveto(809.08227803,267.04770093)(809.08227803,266.99770098)(809.07228027,266.94770508)
\curveto(809.04227807,266.82770115)(809.02727808,266.68770129)(809.02728027,266.52770508)
\curveto(809.03727807,266.3777016)(809.04227807,266.23270174)(809.04228027,266.09270508)
\lineto(809.04228027,264.24770508)
\lineto(809.04228027,263.90270508)
\curveto(809.04227807,263.78270419)(809.03727807,263.66770431)(809.02728027,263.55770508)
\curveto(809.01727809,263.44770453)(809.0122781,263.35270462)(809.01228027,263.27270508)
\curveto(809.02227809,263.19270478)(809.00227811,263.12270485)(808.95228027,263.06270508)
\curveto(808.90227821,262.99270498)(808.82227829,262.95270502)(808.71228027,262.94270508)
\curveto(808.6122785,262.93270504)(808.50227861,262.92770505)(808.38228027,262.92770508)
\lineto(808.11228027,262.92770508)
\curveto(808.06227905,262.94770503)(808.0122791,262.96270501)(807.96228027,262.97270508)
\curveto(807.92227919,262.99270498)(807.89227922,263.01770496)(807.87228027,263.04770508)
\curveto(807.82227929,263.11770486)(807.79227932,263.20270477)(807.78228027,263.30270508)
\lineto(807.78228027,263.63270508)
\lineto(807.78228027,264.78770508)
\lineto(807.78228027,268.94270508)
\lineto(807.78228027,269.97770508)
\lineto(807.78228027,270.27770508)
\curveto(807.79227932,270.3776976)(807.82227929,270.46269751)(807.87228027,270.53270508)
\curveto(807.90227921,270.5726974)(807.95227916,270.60269737)(808.02228027,270.62270508)
\curveto(808.10227901,270.64269733)(808.18727892,270.65269732)(808.27728027,270.65270508)
\curveto(808.36727874,270.66269731)(808.45727865,270.66269731)(808.54728027,270.65270508)
\curveto(808.63727847,270.64269733)(808.7072784,270.62769735)(808.75728027,270.60770508)
\curveto(808.83727827,270.5776974)(808.88727822,270.51769746)(808.90728027,270.42770508)
\curveto(808.93727817,270.34769763)(808.95227816,270.25769772)(808.95228027,270.15770508)
\lineto(808.95228027,269.85770508)
\curveto(808.95227816,269.75769822)(808.97227814,269.66769831)(809.01228027,269.58770508)
\curveto(809.02227809,269.56769841)(809.03227808,269.55269842)(809.04228027,269.54270508)
\lineto(809.08728027,269.49770508)
\curveto(809.19727791,269.49769848)(809.28727782,269.54269843)(809.35728027,269.63270508)
\curveto(809.42727768,269.73269824)(809.48727762,269.81269816)(809.53728027,269.87270508)
\lineto(809.62728027,269.96270508)
\curveto(809.71727739,270.0726979)(809.84227727,270.18769779)(810.00228027,270.30770508)
\curveto(810.16227695,270.42769755)(810.3122768,270.51769746)(810.45228027,270.57770508)
\curveto(810.54227657,270.62769735)(810.63727647,270.66269731)(810.73728027,270.68270508)
\curveto(810.83727627,270.71269726)(810.94227617,270.74269723)(811.05228027,270.77270508)
\curveto(811.112276,270.78269719)(811.17227594,270.78769719)(811.23228027,270.78770508)
\curveto(811.29227582,270.79769718)(811.34727576,270.80769717)(811.39728027,270.81770508)
}
}
{
\newrgbcolor{curcolor}{0 0 0}
\pscustom[linestyle=none,fillstyle=solid,fillcolor=curcolor]
{
\newpath
\moveto(813.0470459,272.13770508)
\curveto(812.96704478,272.19769578)(812.92204482,272.30269567)(812.9120459,272.45270508)
\lineto(812.9120459,272.91770508)
\lineto(812.9120459,273.17270508)
\curveto(812.91204483,273.26269471)(812.92704482,273.33769464)(812.9570459,273.39770508)
\curveto(812.99704475,273.4776945)(813.07704467,273.53769444)(813.1970459,273.57770508)
\curveto(813.21704453,273.58769439)(813.23704451,273.58769439)(813.2570459,273.57770508)
\curveto(813.28704446,273.5776944)(813.31204443,273.58269439)(813.3320459,273.59270508)
\curveto(813.50204424,273.59269438)(813.66204408,273.58769439)(813.8120459,273.57770508)
\curveto(813.96204378,273.56769441)(814.06204368,273.50769447)(814.1120459,273.39770508)
\curveto(814.1420436,273.33769464)(814.15704359,273.26269471)(814.1570459,273.17270508)
\lineto(814.1570459,272.91770508)
\curveto(814.15704359,272.73769524)(814.15204359,272.56769541)(814.1420459,272.40770508)
\curveto(814.1420436,272.24769573)(814.07704367,272.14269583)(813.9470459,272.09270508)
\curveto(813.89704385,272.0726959)(813.8420439,272.06269591)(813.7820459,272.06270508)
\lineto(813.6170459,272.06270508)
\lineto(813.3020459,272.06270508)
\curveto(813.20204454,272.06269591)(813.11704463,272.08769589)(813.0470459,272.13770508)
\moveto(814.1570459,263.63270508)
\lineto(814.1570459,263.31770508)
\curveto(814.16704358,263.21770476)(814.1470436,263.13770484)(814.0970459,263.07770508)
\curveto(814.06704368,263.01770496)(814.02204372,262.977705)(813.9620459,262.95770508)
\curveto(813.90204384,262.94770503)(813.83204391,262.93270504)(813.7520459,262.91270508)
\lineto(813.5270459,262.91270508)
\curveto(813.39704435,262.91270506)(813.28204446,262.91770506)(813.1820459,262.92770508)
\curveto(813.09204465,262.94770503)(813.02204472,262.99770498)(812.9720459,263.07770508)
\curveto(812.93204481,263.13770484)(812.91204483,263.21270476)(812.9120459,263.30270508)
\lineto(812.9120459,263.58770508)
\lineto(812.9120459,269.93270508)
\lineto(812.9120459,270.24770508)
\curveto(812.91204483,270.35769762)(812.93704481,270.44269753)(812.9870459,270.50270508)
\curveto(813.01704473,270.55269742)(813.05704469,270.58269739)(813.1070459,270.59270508)
\curveto(813.15704459,270.60269737)(813.21204453,270.61769736)(813.2720459,270.63770508)
\curveto(813.29204445,270.63769734)(813.31204443,270.63269734)(813.3320459,270.62270508)
\curveto(813.36204438,270.62269735)(813.38704436,270.62769735)(813.4070459,270.63770508)
\curveto(813.53704421,270.63769734)(813.66704408,270.63269734)(813.7970459,270.62270508)
\curveto(813.93704381,270.62269735)(814.03204371,270.58269739)(814.0820459,270.50270508)
\curveto(814.13204361,270.44269753)(814.15704359,270.36269761)(814.1570459,270.26270508)
\lineto(814.1570459,269.97770508)
\lineto(814.1570459,263.63270508)
}
}
{
\newrgbcolor{curcolor}{0 0 0}
\pscustom[linestyle=none,fillstyle=solid,fillcolor=curcolor]
{
\newpath
\moveto(823.22688965,267.11270508)
\curveto(823.24688159,267.05270092)(823.25688158,266.95770102)(823.25688965,266.82770508)
\curveto(823.25688158,266.70770127)(823.25188158,266.62270135)(823.24188965,266.57270508)
\lineto(823.24188965,266.42270508)
\curveto(823.2318816,266.34270163)(823.22188161,266.26770171)(823.21188965,266.19770508)
\curveto(823.21188162,266.13770184)(823.20688163,266.06770191)(823.19688965,265.98770508)
\curveto(823.17688166,265.92770205)(823.16188167,265.86770211)(823.15188965,265.80770508)
\curveto(823.15188168,265.74770223)(823.14188169,265.68770229)(823.12188965,265.62770508)
\curveto(823.08188175,265.49770248)(823.04688179,265.36770261)(823.01688965,265.23770508)
\curveto(822.98688185,265.10770287)(822.94688189,264.98770299)(822.89688965,264.87770508)
\curveto(822.68688215,264.39770358)(822.40688243,263.99270398)(822.05688965,263.66270508)
\curveto(821.70688313,263.34270463)(821.27688356,263.09770488)(820.76688965,262.92770508)
\curveto(820.65688418,262.88770509)(820.5368843,262.85770512)(820.40688965,262.83770508)
\curveto(820.28688455,262.81770516)(820.16188467,262.79770518)(820.03188965,262.77770508)
\curveto(819.97188486,262.76770521)(819.90688493,262.76270521)(819.83688965,262.76270508)
\curveto(819.77688506,262.75270522)(819.71688512,262.74770523)(819.65688965,262.74770508)
\curveto(819.61688522,262.73770524)(819.55688528,262.73270524)(819.47688965,262.73270508)
\curveto(819.40688543,262.73270524)(819.35688548,262.73770524)(819.32688965,262.74770508)
\curveto(819.28688555,262.75770522)(819.24688559,262.76270521)(819.20688965,262.76270508)
\curveto(819.16688567,262.75270522)(819.1318857,262.75270522)(819.10188965,262.76270508)
\lineto(819.01188965,262.76270508)
\lineto(818.65188965,262.80770508)
\curveto(818.51188632,262.84770513)(818.37688646,262.88770509)(818.24688965,262.92770508)
\curveto(818.11688672,262.96770501)(817.99188684,263.01270496)(817.87188965,263.06270508)
\curveto(817.42188741,263.26270471)(817.05188778,263.52270445)(816.76188965,263.84270508)
\curveto(816.47188836,264.16270381)(816.2318886,264.55270342)(816.04188965,265.01270508)
\curveto(815.99188884,265.11270286)(815.95188888,265.21270276)(815.92188965,265.31270508)
\curveto(815.90188893,265.41270256)(815.88188895,265.51770246)(815.86188965,265.62770508)
\curveto(815.84188899,265.66770231)(815.831889,265.69770228)(815.83188965,265.71770508)
\curveto(815.84188899,265.74770223)(815.84188899,265.78270219)(815.83188965,265.82270508)
\curveto(815.81188902,265.90270207)(815.79688904,265.98270199)(815.78688965,266.06270508)
\curveto(815.78688905,266.15270182)(815.77688906,266.23770174)(815.75688965,266.31770508)
\lineto(815.75688965,266.43770508)
\curveto(815.75688908,266.4777015)(815.75188908,266.52270145)(815.74188965,266.57270508)
\curveto(815.7318891,266.62270135)(815.72688911,266.70770127)(815.72688965,266.82770508)
\curveto(815.72688911,266.95770102)(815.7368891,267.05270092)(815.75688965,267.11270508)
\curveto(815.77688906,267.18270079)(815.78188905,267.25270072)(815.77188965,267.32270508)
\curveto(815.76188907,267.39270058)(815.76688907,267.46270051)(815.78688965,267.53270508)
\curveto(815.79688904,267.58270039)(815.80188903,267.62270035)(815.80188965,267.65270508)
\curveto(815.81188902,267.69270028)(815.82188901,267.73770024)(815.83188965,267.78770508)
\curveto(815.86188897,267.90770007)(815.88688895,268.02769995)(815.90688965,268.14770508)
\curveto(815.9368889,268.26769971)(815.97688886,268.38269959)(816.02688965,268.49270508)
\curveto(816.17688866,268.86269911)(816.35688848,269.19269878)(816.56688965,269.48270508)
\curveto(816.78688805,269.78269819)(817.05188778,270.03269794)(817.36188965,270.23270508)
\curveto(817.48188735,270.31269766)(817.60688723,270.3776976)(817.73688965,270.42770508)
\curveto(817.86688697,270.48769749)(818.00188683,270.54769743)(818.14188965,270.60770508)
\curveto(818.26188657,270.65769732)(818.39188644,270.68769729)(818.53188965,270.69770508)
\curveto(818.67188616,270.71769726)(818.81188602,270.74769723)(818.95188965,270.78770508)
\lineto(819.14688965,270.78770508)
\curveto(819.21688562,270.79769718)(819.28188555,270.80769717)(819.34188965,270.81770508)
\curveto(820.2318846,270.82769715)(820.97188386,270.64269733)(821.56188965,270.26270508)
\curveto(822.15188268,269.88269809)(822.57688226,269.38769859)(822.83688965,268.77770508)
\curveto(822.88688195,268.6776993)(822.92688191,268.5776994)(822.95688965,268.47770508)
\curveto(822.98688185,268.3776996)(823.02188181,268.2726997)(823.06188965,268.16270508)
\curveto(823.09188174,268.05269992)(823.11688172,267.93270004)(823.13688965,267.80270508)
\curveto(823.15688168,267.68270029)(823.18188165,267.55770042)(823.21188965,267.42770508)
\curveto(823.22188161,267.3777006)(823.22188161,267.32270065)(823.21188965,267.26270508)
\curveto(823.21188162,267.21270076)(823.21688162,267.16270081)(823.22688965,267.11270508)
\moveto(821.89188965,266.25770508)
\curveto(821.91188292,266.32770165)(821.91688292,266.40770157)(821.90688965,266.49770508)
\lineto(821.90688965,266.75270508)
\curveto(821.90688293,267.14270083)(821.87188296,267.4727005)(821.80188965,267.74270508)
\curveto(821.77188306,267.82270015)(821.74688309,267.90270007)(821.72688965,267.98270508)
\curveto(821.70688313,268.06269991)(821.68188315,268.13769984)(821.65188965,268.20770508)
\curveto(821.37188346,268.85769912)(820.92688391,269.30769867)(820.31688965,269.55770508)
\curveto(820.24688459,269.58769839)(820.17188466,269.60769837)(820.09188965,269.61770508)
\lineto(819.85188965,269.67770508)
\curveto(819.77188506,269.69769828)(819.68688515,269.70769827)(819.59688965,269.70770508)
\lineto(819.32688965,269.70770508)
\lineto(819.05688965,269.66270508)
\curveto(818.95688588,269.64269833)(818.86188597,269.61769836)(818.77188965,269.58770508)
\curveto(818.69188614,269.56769841)(818.61188622,269.53769844)(818.53188965,269.49770508)
\curveto(818.46188637,269.4776985)(818.39688644,269.44769853)(818.33688965,269.40770508)
\curveto(818.27688656,269.36769861)(818.22188661,269.32769865)(818.17188965,269.28770508)
\curveto(817.9318869,269.11769886)(817.7368871,268.91269906)(817.58688965,268.67270508)
\curveto(817.4368874,268.43269954)(817.30688753,268.15269982)(817.19688965,267.83270508)
\curveto(817.16688767,267.73270024)(817.14688769,267.62770035)(817.13688965,267.51770508)
\curveto(817.12688771,267.41770056)(817.11188772,267.31270066)(817.09188965,267.20270508)
\curveto(817.08188775,267.16270081)(817.07688776,267.09770088)(817.07688965,267.00770508)
\curveto(817.06688777,266.977701)(817.06188777,266.94270103)(817.06188965,266.90270508)
\curveto(817.07188776,266.86270111)(817.07688776,266.81770116)(817.07688965,266.76770508)
\lineto(817.07688965,266.46770508)
\curveto(817.07688776,266.36770161)(817.08688775,266.2777017)(817.10688965,266.19770508)
\lineto(817.13688965,266.01770508)
\curveto(817.15688768,265.91770206)(817.17188766,265.81770216)(817.18188965,265.71770508)
\curveto(817.20188763,265.62770235)(817.2318876,265.54270243)(817.27188965,265.46270508)
\curveto(817.37188746,265.22270275)(817.48688735,264.99770298)(817.61688965,264.78770508)
\curveto(817.75688708,264.5777034)(817.92688691,264.40270357)(818.12688965,264.26270508)
\curveto(818.17688666,264.23270374)(818.22188661,264.20770377)(818.26188965,264.18770508)
\curveto(818.30188653,264.16770381)(818.34688649,264.14270383)(818.39688965,264.11270508)
\curveto(818.47688636,264.06270391)(818.56188627,264.01770396)(818.65188965,263.97770508)
\curveto(818.75188608,263.94770403)(818.85688598,263.91770406)(818.96688965,263.88770508)
\curveto(819.01688582,263.86770411)(819.06188577,263.85770412)(819.10188965,263.85770508)
\curveto(819.15188568,263.86770411)(819.20188563,263.86770411)(819.25188965,263.85770508)
\curveto(819.28188555,263.84770413)(819.34188549,263.83770414)(819.43188965,263.82770508)
\curveto(819.5318853,263.81770416)(819.60688523,263.82270415)(819.65688965,263.84270508)
\curveto(819.69688514,263.85270412)(819.7368851,263.85270412)(819.77688965,263.84270508)
\curveto(819.81688502,263.84270413)(819.85688498,263.85270412)(819.89688965,263.87270508)
\curveto(819.97688486,263.89270408)(820.05688478,263.90770407)(820.13688965,263.91770508)
\curveto(820.21688462,263.93770404)(820.29188454,263.96270401)(820.36188965,263.99270508)
\curveto(820.70188413,264.13270384)(820.97688386,264.32770365)(821.18688965,264.57770508)
\curveto(821.39688344,264.82770315)(821.57188326,265.12270285)(821.71188965,265.46270508)
\curveto(821.76188307,265.58270239)(821.79188304,265.70770227)(821.80188965,265.83770508)
\curveto(821.82188301,265.977702)(821.85188298,266.11770186)(821.89188965,266.25770508)
}
}
{
\newrgbcolor{curcolor}{0 0 0}
\pscustom[linestyle=none,fillstyle=solid,fillcolor=curcolor]
{
\newpath
\moveto(775.97647461,250.60103638)
\lineto(780.61147461,250.60103638)
\lineto(781.82647461,250.60103638)
\curveto(781.93646706,250.60102568)(782.04146695,250.60102568)(782.14147461,250.60103638)
\curveto(782.25146674,250.60102568)(782.33646666,250.5810257)(782.39647461,250.54103638)
\curveto(782.47646652,250.49102579)(782.52146647,250.41602587)(782.53147461,250.31603638)
\curveto(782.55146644,250.22602606)(782.56146643,250.11602617)(782.56147461,249.98603638)
\lineto(782.56147461,249.83603638)
\curveto(782.57146642,249.79602649)(782.56646643,249.75602653)(782.54647461,249.71603638)
\curveto(782.50646649,249.55602673)(782.41646658,249.46602682)(782.27647461,249.44603638)
\curveto(782.14646685,249.43602685)(781.98146701,249.43102685)(781.78147461,249.43103638)
\lineto(780.22147461,249.43103638)
\lineto(778.01647461,249.43103638)
\lineto(777.50647461,249.43103638)
\curveto(777.32647167,249.44102684)(777.1914718,249.41102687)(777.10147461,249.34103638)
\curveto(777.01147198,249.281027)(776.96147203,249.17602711)(776.95147461,249.02603638)
\lineto(776.95147461,248.57603638)
\lineto(776.95147461,247.09103638)
\curveto(776.95147204,247.01102927)(776.94647205,246.91102937)(776.93647461,246.79103638)
\curveto(776.93647206,246.67102961)(776.94647205,246.57102971)(776.96647461,246.49103638)
\lineto(776.96647461,246.37103638)
\curveto(776.98647201,246.31102997)(777.00147199,246.25103003)(777.01147461,246.19103638)
\curveto(777.03147196,246.14103014)(777.06647193,246.10103018)(777.11647461,246.07103638)
\curveto(777.20647179,246.01103027)(777.34647165,245.9810303)(777.53647461,245.98103638)
\curveto(777.72647127,245.99103029)(777.8914711,245.99603029)(778.03147461,245.99603638)
\lineto(780.73147461,245.99603638)
\lineto(781.01647461,245.99603638)
\curveto(781.12646787,246.00603028)(781.23146776,246.00603028)(781.33147461,245.99603638)
\curveto(781.44146755,245.99603029)(781.53646746,245.9860303)(781.61647461,245.96603638)
\curveto(781.70646729,245.94603034)(781.76646723,245.91103037)(781.79647461,245.86103638)
\curveto(781.84646715,245.80103048)(781.87146712,245.72603056)(781.87147461,245.63603638)
\lineto(781.87147461,245.33603638)
\lineto(781.87147461,245.17103638)
\curveto(781.87146712,245.12103116)(781.86146713,245.07603121)(781.84147461,245.03603638)
\curveto(781.80146719,244.93603135)(781.74646725,244.8810314)(781.67647461,244.87103638)
\curveto(781.63646736,244.85103143)(781.5964674,244.84103144)(781.55647461,244.84103638)
\curveto(781.52646747,244.84103144)(781.48646751,244.83603145)(781.43647461,244.82603638)
\curveto(781.3964676,244.81603147)(781.35146764,244.81103147)(781.30147461,244.81103638)
\curveto(781.26146773,244.82103146)(781.22146777,244.82603146)(781.18147461,244.82603638)
\lineto(780.65647461,244.82603638)
\lineto(778.12147461,244.82603638)
\lineto(777.55147461,244.82603638)
\curveto(777.34147165,244.83603145)(777.1914718,244.80603148)(777.10147461,244.73603638)
\curveto(777.05147194,244.69603159)(777.01147198,244.63103165)(776.98147461,244.54103638)
\curveto(776.96147203,244.46103182)(776.94647205,244.36603192)(776.93647461,244.25603638)
\lineto(776.93647461,243.91103638)
\curveto(776.94647205,243.80103248)(776.95147204,243.70103258)(776.95147461,243.61103638)
\lineto(776.95147461,241.03103638)
\curveto(776.95147204,240.86103542)(776.95647204,240.67603561)(776.96647461,240.47603638)
\curveto(776.97647202,240.27603601)(776.94147205,240.12603616)(776.86147461,240.02603638)
\curveto(776.83147216,239.9860363)(776.78647221,239.96103632)(776.72647461,239.95103638)
\curveto(776.66647233,239.95103633)(776.60647239,239.94103634)(776.54647461,239.92103638)
\lineto(776.26147461,239.92103638)
\curveto(776.12147287,239.92103636)(775.991473,239.92603636)(775.87147461,239.93603638)
\curveto(775.75147324,239.94603634)(775.66647333,239.99603629)(775.61647461,240.08603638)
\curveto(775.57647342,240.14603614)(775.55647344,240.22603606)(775.55647461,240.32603638)
\lineto(775.55647461,240.65603638)
\lineto(775.55647461,241.85603638)
\lineto(775.55647461,248.12603638)
\lineto(775.55647461,249.74603638)
\curveto(775.55647344,249.85602643)(775.55147344,249.97602631)(775.54147461,250.10603638)
\curveto(775.54147345,250.24602604)(775.56647343,250.35602593)(775.61647461,250.43603638)
\curveto(775.65647334,250.49602579)(775.73147326,250.54602574)(775.84147461,250.58603638)
\curveto(775.86147313,250.59602569)(775.88147311,250.59602569)(775.90147461,250.58603638)
\curveto(775.93147306,250.5860257)(775.95647304,250.59102569)(775.97647461,250.60103638)
}
}
{
\newrgbcolor{curcolor}{0 0 0}
\pscustom[linestyle=none,fillstyle=solid,fillcolor=curcolor]
{
\newpath
\moveto(791.03975586,244.12103638)
\curveto(791.0597478,244.06103222)(791.06974779,243.96603232)(791.06975586,243.83603638)
\curveto(791.06974779,243.71603257)(791.06474779,243.63103265)(791.05475586,243.58103638)
\lineto(791.05475586,243.43103638)
\curveto(791.04474781,243.35103293)(791.03474782,243.27603301)(791.02475586,243.20603638)
\curveto(791.02474783,243.14603314)(791.01974784,243.07603321)(791.00975586,242.99603638)
\curveto(790.98974787,242.93603335)(790.97474788,242.87603341)(790.96475586,242.81603638)
\curveto(790.96474789,242.75603353)(790.9547479,242.69603359)(790.93475586,242.63603638)
\curveto(790.89474796,242.50603378)(790.859748,242.37603391)(790.82975586,242.24603638)
\curveto(790.79974806,242.11603417)(790.7597481,241.99603429)(790.70975586,241.88603638)
\curveto(790.49974836,241.40603488)(790.21974864,241.00103528)(789.86975586,240.67103638)
\curveto(789.51974934,240.35103593)(789.08974977,240.10603618)(788.57975586,239.93603638)
\curveto(788.46975039,239.89603639)(788.34975051,239.86603642)(788.21975586,239.84603638)
\curveto(788.09975076,239.82603646)(787.97475088,239.80603648)(787.84475586,239.78603638)
\curveto(787.78475107,239.77603651)(787.71975114,239.77103651)(787.64975586,239.77103638)
\curveto(787.58975127,239.76103652)(787.52975133,239.75603653)(787.46975586,239.75603638)
\curveto(787.42975143,239.74603654)(787.36975149,239.74103654)(787.28975586,239.74103638)
\curveto(787.21975164,239.74103654)(787.16975169,239.74603654)(787.13975586,239.75603638)
\curveto(787.09975176,239.76603652)(787.0597518,239.77103651)(787.01975586,239.77103638)
\curveto(786.97975188,239.76103652)(786.94475191,239.76103652)(786.91475586,239.77103638)
\lineto(786.82475586,239.77103638)
\lineto(786.46475586,239.81603638)
\curveto(786.32475253,239.85603643)(786.18975267,239.89603639)(786.05975586,239.93603638)
\curveto(785.92975293,239.97603631)(785.80475305,240.02103626)(785.68475586,240.07103638)
\curveto(785.23475362,240.27103601)(784.86475399,240.53103575)(784.57475586,240.85103638)
\curveto(784.28475457,241.17103511)(784.04475481,241.56103472)(783.85475586,242.02103638)
\curveto(783.80475505,242.12103416)(783.76475509,242.22103406)(783.73475586,242.32103638)
\curveto(783.71475514,242.42103386)(783.69475516,242.52603376)(783.67475586,242.63603638)
\curveto(783.6547552,242.67603361)(783.64475521,242.70603358)(783.64475586,242.72603638)
\curveto(783.6547552,242.75603353)(783.6547552,242.79103349)(783.64475586,242.83103638)
\curveto(783.62475523,242.91103337)(783.60975525,242.99103329)(783.59975586,243.07103638)
\curveto(783.59975526,243.16103312)(783.58975527,243.24603304)(783.56975586,243.32603638)
\lineto(783.56975586,243.44603638)
\curveto(783.56975529,243.4860328)(783.56475529,243.53103275)(783.55475586,243.58103638)
\curveto(783.54475531,243.63103265)(783.53975532,243.71603257)(783.53975586,243.83603638)
\curveto(783.53975532,243.96603232)(783.54975531,244.06103222)(783.56975586,244.12103638)
\curveto(783.58975527,244.19103209)(783.59475526,244.26103202)(783.58475586,244.33103638)
\curveto(783.57475528,244.40103188)(783.57975528,244.47103181)(783.59975586,244.54103638)
\curveto(783.60975525,244.59103169)(783.61475524,244.63103165)(783.61475586,244.66103638)
\curveto(783.62475523,244.70103158)(783.63475522,244.74603154)(783.64475586,244.79603638)
\curveto(783.67475518,244.91603137)(783.69975516,245.03603125)(783.71975586,245.15603638)
\curveto(783.74975511,245.27603101)(783.78975507,245.39103089)(783.83975586,245.50103638)
\curveto(783.98975487,245.87103041)(784.16975469,246.20103008)(784.37975586,246.49103638)
\curveto(784.59975426,246.79102949)(784.86475399,247.04102924)(785.17475586,247.24103638)
\curveto(785.29475356,247.32102896)(785.41975344,247.3860289)(785.54975586,247.43603638)
\curveto(785.67975318,247.49602879)(785.81475304,247.55602873)(785.95475586,247.61603638)
\curveto(786.07475278,247.66602862)(786.20475265,247.69602859)(786.34475586,247.70603638)
\curveto(786.48475237,247.72602856)(786.62475223,247.75602853)(786.76475586,247.79603638)
\lineto(786.95975586,247.79603638)
\curveto(787.02975183,247.80602848)(787.09475176,247.81602847)(787.15475586,247.82603638)
\curveto(788.04475081,247.83602845)(788.78475007,247.65102863)(789.37475586,247.27103638)
\curveto(789.96474889,246.89102939)(790.38974847,246.39602989)(790.64975586,245.78603638)
\curveto(790.69974816,245.6860306)(790.73974812,245.5860307)(790.76975586,245.48603638)
\curveto(790.79974806,245.3860309)(790.83474802,245.281031)(790.87475586,245.17103638)
\curveto(790.90474795,245.06103122)(790.92974793,244.94103134)(790.94975586,244.81103638)
\curveto(790.96974789,244.69103159)(790.99474786,244.56603172)(791.02475586,244.43603638)
\curveto(791.03474782,244.3860319)(791.03474782,244.33103195)(791.02475586,244.27103638)
\curveto(791.02474783,244.22103206)(791.02974783,244.17103211)(791.03975586,244.12103638)
\moveto(789.70475586,243.26603638)
\curveto(789.72474913,243.33603295)(789.72974913,243.41603287)(789.71975586,243.50603638)
\lineto(789.71975586,243.76103638)
\curveto(789.71974914,244.15103213)(789.68474917,244.4810318)(789.61475586,244.75103638)
\curveto(789.58474927,244.83103145)(789.5597493,244.91103137)(789.53975586,244.99103638)
\curveto(789.51974934,245.07103121)(789.49474936,245.14603114)(789.46475586,245.21603638)
\curveto(789.18474967,245.86603042)(788.73975012,246.31602997)(788.12975586,246.56603638)
\curveto(788.0597508,246.59602969)(787.98475087,246.61602967)(787.90475586,246.62603638)
\lineto(787.66475586,246.68603638)
\curveto(787.58475127,246.70602958)(787.49975136,246.71602957)(787.40975586,246.71603638)
\lineto(787.13975586,246.71603638)
\lineto(786.86975586,246.67103638)
\curveto(786.76975209,246.65102963)(786.67475218,246.62602966)(786.58475586,246.59603638)
\curveto(786.50475235,246.57602971)(786.42475243,246.54602974)(786.34475586,246.50603638)
\curveto(786.27475258,246.4860298)(786.20975265,246.45602983)(786.14975586,246.41603638)
\curveto(786.08975277,246.37602991)(786.03475282,246.33602995)(785.98475586,246.29603638)
\curveto(785.74475311,246.12603016)(785.54975331,245.92103036)(785.39975586,245.68103638)
\curveto(785.24975361,245.44103084)(785.11975374,245.16103112)(785.00975586,244.84103638)
\curveto(784.97975388,244.74103154)(784.9597539,244.63603165)(784.94975586,244.52603638)
\curveto(784.93975392,244.42603186)(784.92475393,244.32103196)(784.90475586,244.21103638)
\curveto(784.89475396,244.17103211)(784.88975397,244.10603218)(784.88975586,244.01603638)
\curveto(784.87975398,243.9860323)(784.87475398,243.95103233)(784.87475586,243.91103638)
\curveto(784.88475397,243.87103241)(784.88975397,243.82603246)(784.88975586,243.77603638)
\lineto(784.88975586,243.47603638)
\curveto(784.88975397,243.37603291)(784.89975396,243.286033)(784.91975586,243.20603638)
\lineto(784.94975586,243.02603638)
\curveto(784.96975389,242.92603336)(784.98475387,242.82603346)(784.99475586,242.72603638)
\curveto(785.01475384,242.63603365)(785.04475381,242.55103373)(785.08475586,242.47103638)
\curveto(785.18475367,242.23103405)(785.29975356,242.00603428)(785.42975586,241.79603638)
\curveto(785.56975329,241.5860347)(785.73975312,241.41103487)(785.93975586,241.27103638)
\curveto(785.98975287,241.24103504)(786.03475282,241.21603507)(786.07475586,241.19603638)
\curveto(786.11475274,241.17603511)(786.1597527,241.15103513)(786.20975586,241.12103638)
\curveto(786.28975257,241.07103521)(786.37475248,241.02603526)(786.46475586,240.98603638)
\curveto(786.56475229,240.95603533)(786.66975219,240.92603536)(786.77975586,240.89603638)
\curveto(786.82975203,240.87603541)(786.87475198,240.86603542)(786.91475586,240.86603638)
\curveto(786.96475189,240.87603541)(787.01475184,240.87603541)(787.06475586,240.86603638)
\curveto(787.09475176,240.85603543)(787.1547517,240.84603544)(787.24475586,240.83603638)
\curveto(787.34475151,240.82603546)(787.41975144,240.83103545)(787.46975586,240.85103638)
\curveto(787.50975135,240.86103542)(787.54975131,240.86103542)(787.58975586,240.85103638)
\curveto(787.62975123,240.85103543)(787.66975119,240.86103542)(787.70975586,240.88103638)
\curveto(787.78975107,240.90103538)(787.86975099,240.91603537)(787.94975586,240.92603638)
\curveto(788.02975083,240.94603534)(788.10475075,240.97103531)(788.17475586,241.00103638)
\curveto(788.51475034,241.14103514)(788.78975007,241.33603495)(788.99975586,241.58603638)
\curveto(789.20974965,241.83603445)(789.38474947,242.13103415)(789.52475586,242.47103638)
\curveto(789.57474928,242.59103369)(789.60474925,242.71603357)(789.61475586,242.84603638)
\curveto(789.63474922,242.9860333)(789.66474919,243.12603316)(789.70475586,243.26603638)
}
}
{
\newrgbcolor{curcolor}{0 0 0}
\pscustom[linestyle=none,fillstyle=solid,fillcolor=curcolor]
{
\newpath
\moveto(793.47303711,249.98603638)
\curveto(793.6230351,249.9860263)(793.77303495,249.9810263)(793.92303711,249.97103638)
\curveto(794.07303465,249.97102631)(794.17803454,249.93102635)(794.23803711,249.85103638)
\curveto(794.28803443,249.79102649)(794.31303441,249.70602658)(794.31303711,249.59603638)
\curveto(794.3230344,249.49602679)(794.32803439,249.39102689)(794.32803711,249.28103638)
\lineto(794.32803711,248.41103638)
\curveto(794.32803439,248.33102795)(794.3230344,248.24602804)(794.31303711,248.15603638)
\curveto(794.31303441,248.07602821)(794.3230344,248.00602828)(794.34303711,247.94603638)
\curveto(794.38303434,247.80602848)(794.47303425,247.71602857)(794.61303711,247.67603638)
\curveto(794.66303406,247.66602862)(794.70803401,247.66102862)(794.74803711,247.66103638)
\lineto(794.89803711,247.66103638)
\lineto(795.30303711,247.66103638)
\curveto(795.46303326,247.67102861)(795.57803314,247.66102862)(795.64803711,247.63103638)
\curveto(795.73803298,247.57102871)(795.79803292,247.51102877)(795.82803711,247.45103638)
\curveto(795.84803287,247.41102887)(795.85803286,247.36602892)(795.85803711,247.31603638)
\lineto(795.85803711,247.16603638)
\curveto(795.85803286,247.05602923)(795.85303287,246.95102933)(795.84303711,246.85103638)
\curveto(795.83303289,246.76102952)(795.79803292,246.69102959)(795.73803711,246.64103638)
\curveto(795.67803304,246.59102969)(795.59303313,246.56102972)(795.48303711,246.55103638)
\lineto(795.15303711,246.55103638)
\curveto(795.04303368,246.56102972)(794.93303379,246.56602972)(794.82303711,246.56603638)
\curveto(794.71303401,246.56602972)(794.6180341,246.55102973)(794.53803711,246.52103638)
\curveto(794.46803425,246.49102979)(794.4180343,246.44102984)(794.38803711,246.37103638)
\curveto(794.35803436,246.30102998)(794.33803438,246.21603007)(794.32803711,246.11603638)
\curveto(794.3180344,246.02603026)(794.31303441,245.92603036)(794.31303711,245.81603638)
\curveto(794.3230344,245.71603057)(794.32803439,245.61603067)(794.32803711,245.51603638)
\lineto(794.32803711,242.54603638)
\curveto(794.32803439,242.32603396)(794.3230344,242.09103419)(794.31303711,241.84103638)
\curveto(794.31303441,241.60103468)(794.35803436,241.41603487)(794.44803711,241.28603638)
\curveto(794.49803422,241.20603508)(794.56303416,241.15103513)(794.64303711,241.12103638)
\curveto(794.723034,241.09103519)(794.8180339,241.06603522)(794.92803711,241.04603638)
\curveto(794.95803376,241.03603525)(794.98803373,241.03103525)(795.01803711,241.03103638)
\curveto(795.05803366,241.04103524)(795.09303363,241.04103524)(795.12303711,241.03103638)
\lineto(795.31803711,241.03103638)
\curveto(795.4180333,241.03103525)(795.50803321,241.02103526)(795.58803711,241.00103638)
\curveto(795.67803304,240.99103529)(795.74303298,240.95603533)(795.78303711,240.89603638)
\curveto(795.80303292,240.86603542)(795.8180329,240.81103547)(795.82803711,240.73103638)
\curveto(795.84803287,240.66103562)(795.85803286,240.5860357)(795.85803711,240.50603638)
\curveto(795.86803285,240.42603586)(795.86803285,240.34603594)(795.85803711,240.26603638)
\curveto(795.84803287,240.19603609)(795.82803289,240.14103614)(795.79803711,240.10103638)
\curveto(795.75803296,240.03103625)(795.68303304,239.9810363)(795.57303711,239.95103638)
\curveto(795.49303323,239.93103635)(795.40303332,239.92103636)(795.30303711,239.92103638)
\curveto(795.20303352,239.93103635)(795.11303361,239.93603635)(795.03303711,239.93603638)
\curveto(794.97303375,239.93603635)(794.91303381,239.93103635)(794.85303711,239.92103638)
\curveto(794.79303393,239.92103636)(794.73803398,239.92603636)(794.68803711,239.93603638)
\lineto(794.50803711,239.93603638)
\curveto(794.45803426,239.94603634)(794.40803431,239.95103633)(794.35803711,239.95103638)
\curveto(794.3180344,239.96103632)(794.27303445,239.96603632)(794.22303711,239.96603638)
\curveto(794.0230347,240.01603627)(793.84803487,240.07103621)(793.69803711,240.13103638)
\curveto(793.55803516,240.19103609)(793.43803528,240.29603599)(793.33803711,240.44603638)
\curveto(793.19803552,240.64603564)(793.1180356,240.89603539)(793.09803711,241.19603638)
\curveto(793.07803564,241.50603478)(793.06803565,241.83603445)(793.06803711,242.18603638)
\lineto(793.06803711,246.11603638)
\curveto(793.03803568,246.24603004)(793.00803571,246.34102994)(792.97803711,246.40103638)
\curveto(792.95803576,246.46102982)(792.88803583,246.51102977)(792.76803711,246.55103638)
\curveto(792.72803599,246.56102972)(792.68803603,246.56102972)(792.64803711,246.55103638)
\curveto(792.60803611,246.54102974)(792.56803615,246.54602974)(792.52803711,246.56603638)
\lineto(792.28803711,246.56603638)
\curveto(792.15803656,246.56602972)(792.04803667,246.57602971)(791.95803711,246.59603638)
\curveto(791.87803684,246.62602966)(791.8230369,246.6860296)(791.79303711,246.77603638)
\curveto(791.77303695,246.81602947)(791.75803696,246.86102942)(791.74803711,246.91103638)
\lineto(791.74803711,247.06103638)
\curveto(791.74803697,247.20102908)(791.75803696,247.31602897)(791.77803711,247.40603638)
\curveto(791.79803692,247.50602878)(791.85803686,247.5810287)(791.95803711,247.63103638)
\curveto(792.06803665,247.67102861)(792.20803651,247.6810286)(792.37803711,247.66103638)
\curveto(792.55803616,247.64102864)(792.70803601,247.65102863)(792.82803711,247.69103638)
\curveto(792.9180358,247.74102854)(792.98803573,247.81102847)(793.03803711,247.90103638)
\curveto(793.05803566,247.96102832)(793.06803565,248.03602825)(793.06803711,248.12603638)
\lineto(793.06803711,248.38103638)
\lineto(793.06803711,249.31103638)
\lineto(793.06803711,249.55103638)
\curveto(793.06803565,249.64102664)(793.07803564,249.71602657)(793.09803711,249.77603638)
\curveto(793.13803558,249.85602643)(793.21303551,249.92102636)(793.32303711,249.97103638)
\curveto(793.35303537,249.97102631)(793.37803534,249.97102631)(793.39803711,249.97103638)
\curveto(793.42803529,249.9810263)(793.45303527,249.9860263)(793.47303711,249.98603638)
}
}
{
\newrgbcolor{curcolor}{0 0 0}
\pscustom[linestyle=none,fillstyle=solid,fillcolor=curcolor]
{
\newpath
\moveto(804.36983398,244.12103638)
\curveto(804.38982592,244.06103222)(804.39982591,243.96603232)(804.39983398,243.83603638)
\curveto(804.39982591,243.71603257)(804.39482592,243.63103265)(804.38483398,243.58103638)
\lineto(804.38483398,243.43103638)
\curveto(804.37482594,243.35103293)(804.36482595,243.27603301)(804.35483398,243.20603638)
\curveto(804.35482596,243.14603314)(804.34982596,243.07603321)(804.33983398,242.99603638)
\curveto(804.31982599,242.93603335)(804.30482601,242.87603341)(804.29483398,242.81603638)
\curveto(804.29482602,242.75603353)(804.28482603,242.69603359)(804.26483398,242.63603638)
\curveto(804.22482609,242.50603378)(804.18982612,242.37603391)(804.15983398,242.24603638)
\curveto(804.12982618,242.11603417)(804.08982622,241.99603429)(804.03983398,241.88603638)
\curveto(803.82982648,241.40603488)(803.54982676,241.00103528)(803.19983398,240.67103638)
\curveto(802.84982746,240.35103593)(802.41982789,240.10603618)(801.90983398,239.93603638)
\curveto(801.79982851,239.89603639)(801.67982863,239.86603642)(801.54983398,239.84603638)
\curveto(801.42982888,239.82603646)(801.30482901,239.80603648)(801.17483398,239.78603638)
\curveto(801.1148292,239.77603651)(801.04982926,239.77103651)(800.97983398,239.77103638)
\curveto(800.91982939,239.76103652)(800.85982945,239.75603653)(800.79983398,239.75603638)
\curveto(800.75982955,239.74603654)(800.69982961,239.74103654)(800.61983398,239.74103638)
\curveto(800.54982976,239.74103654)(800.49982981,239.74603654)(800.46983398,239.75603638)
\curveto(800.42982988,239.76603652)(800.38982992,239.77103651)(800.34983398,239.77103638)
\curveto(800.30983,239.76103652)(800.27483004,239.76103652)(800.24483398,239.77103638)
\lineto(800.15483398,239.77103638)
\lineto(799.79483398,239.81603638)
\curveto(799.65483066,239.85603643)(799.51983079,239.89603639)(799.38983398,239.93603638)
\curveto(799.25983105,239.97603631)(799.13483118,240.02103626)(799.01483398,240.07103638)
\curveto(798.56483175,240.27103601)(798.19483212,240.53103575)(797.90483398,240.85103638)
\curveto(797.6148327,241.17103511)(797.37483294,241.56103472)(797.18483398,242.02103638)
\curveto(797.13483318,242.12103416)(797.09483322,242.22103406)(797.06483398,242.32103638)
\curveto(797.04483327,242.42103386)(797.02483329,242.52603376)(797.00483398,242.63603638)
\curveto(796.98483333,242.67603361)(796.97483334,242.70603358)(796.97483398,242.72603638)
\curveto(796.98483333,242.75603353)(796.98483333,242.79103349)(796.97483398,242.83103638)
\curveto(796.95483336,242.91103337)(796.93983337,242.99103329)(796.92983398,243.07103638)
\curveto(796.92983338,243.16103312)(796.91983339,243.24603304)(796.89983398,243.32603638)
\lineto(796.89983398,243.44603638)
\curveto(796.89983341,243.4860328)(796.89483342,243.53103275)(796.88483398,243.58103638)
\curveto(796.87483344,243.63103265)(796.86983344,243.71603257)(796.86983398,243.83603638)
\curveto(796.86983344,243.96603232)(796.87983343,244.06103222)(796.89983398,244.12103638)
\curveto(796.91983339,244.19103209)(796.92483339,244.26103202)(796.91483398,244.33103638)
\curveto(796.90483341,244.40103188)(796.9098334,244.47103181)(796.92983398,244.54103638)
\curveto(796.93983337,244.59103169)(796.94483337,244.63103165)(796.94483398,244.66103638)
\curveto(796.95483336,244.70103158)(796.96483335,244.74603154)(796.97483398,244.79603638)
\curveto(797.00483331,244.91603137)(797.02983328,245.03603125)(797.04983398,245.15603638)
\curveto(797.07983323,245.27603101)(797.11983319,245.39103089)(797.16983398,245.50103638)
\curveto(797.31983299,245.87103041)(797.49983281,246.20103008)(797.70983398,246.49103638)
\curveto(797.92983238,246.79102949)(798.19483212,247.04102924)(798.50483398,247.24103638)
\curveto(798.62483169,247.32102896)(798.74983156,247.3860289)(798.87983398,247.43603638)
\curveto(799.0098313,247.49602879)(799.14483117,247.55602873)(799.28483398,247.61603638)
\curveto(799.40483091,247.66602862)(799.53483078,247.69602859)(799.67483398,247.70603638)
\curveto(799.8148305,247.72602856)(799.95483036,247.75602853)(800.09483398,247.79603638)
\lineto(800.28983398,247.79603638)
\curveto(800.35982995,247.80602848)(800.42482989,247.81602847)(800.48483398,247.82603638)
\curveto(801.37482894,247.83602845)(802.1148282,247.65102863)(802.70483398,247.27103638)
\curveto(803.29482702,246.89102939)(803.71982659,246.39602989)(803.97983398,245.78603638)
\curveto(804.02982628,245.6860306)(804.06982624,245.5860307)(804.09983398,245.48603638)
\curveto(804.12982618,245.3860309)(804.16482615,245.281031)(804.20483398,245.17103638)
\curveto(804.23482608,245.06103122)(804.25982605,244.94103134)(804.27983398,244.81103638)
\curveto(804.29982601,244.69103159)(804.32482599,244.56603172)(804.35483398,244.43603638)
\curveto(804.36482595,244.3860319)(804.36482595,244.33103195)(804.35483398,244.27103638)
\curveto(804.35482596,244.22103206)(804.35982595,244.17103211)(804.36983398,244.12103638)
\moveto(803.03483398,243.26603638)
\curveto(803.05482726,243.33603295)(803.05982725,243.41603287)(803.04983398,243.50603638)
\lineto(803.04983398,243.76103638)
\curveto(803.04982726,244.15103213)(803.0148273,244.4810318)(802.94483398,244.75103638)
\curveto(802.9148274,244.83103145)(802.88982742,244.91103137)(802.86983398,244.99103638)
\curveto(802.84982746,245.07103121)(802.82482749,245.14603114)(802.79483398,245.21603638)
\curveto(802.5148278,245.86603042)(802.06982824,246.31602997)(801.45983398,246.56603638)
\curveto(801.38982892,246.59602969)(801.314829,246.61602967)(801.23483398,246.62603638)
\lineto(800.99483398,246.68603638)
\curveto(800.9148294,246.70602958)(800.82982948,246.71602957)(800.73983398,246.71603638)
\lineto(800.46983398,246.71603638)
\lineto(800.19983398,246.67103638)
\curveto(800.09983021,246.65102963)(800.00483031,246.62602966)(799.91483398,246.59603638)
\curveto(799.83483048,246.57602971)(799.75483056,246.54602974)(799.67483398,246.50603638)
\curveto(799.60483071,246.4860298)(799.53983077,246.45602983)(799.47983398,246.41603638)
\curveto(799.41983089,246.37602991)(799.36483095,246.33602995)(799.31483398,246.29603638)
\curveto(799.07483124,246.12603016)(798.87983143,245.92103036)(798.72983398,245.68103638)
\curveto(798.57983173,245.44103084)(798.44983186,245.16103112)(798.33983398,244.84103638)
\curveto(798.309832,244.74103154)(798.28983202,244.63603165)(798.27983398,244.52603638)
\curveto(798.26983204,244.42603186)(798.25483206,244.32103196)(798.23483398,244.21103638)
\curveto(798.22483209,244.17103211)(798.21983209,244.10603218)(798.21983398,244.01603638)
\curveto(798.2098321,243.9860323)(798.20483211,243.95103233)(798.20483398,243.91103638)
\curveto(798.2148321,243.87103241)(798.21983209,243.82603246)(798.21983398,243.77603638)
\lineto(798.21983398,243.47603638)
\curveto(798.21983209,243.37603291)(798.22983208,243.286033)(798.24983398,243.20603638)
\lineto(798.27983398,243.02603638)
\curveto(798.29983201,242.92603336)(798.314832,242.82603346)(798.32483398,242.72603638)
\curveto(798.34483197,242.63603365)(798.37483194,242.55103373)(798.41483398,242.47103638)
\curveto(798.5148318,242.23103405)(798.62983168,242.00603428)(798.75983398,241.79603638)
\curveto(798.89983141,241.5860347)(799.06983124,241.41103487)(799.26983398,241.27103638)
\curveto(799.31983099,241.24103504)(799.36483095,241.21603507)(799.40483398,241.19603638)
\curveto(799.44483087,241.17603511)(799.48983082,241.15103513)(799.53983398,241.12103638)
\curveto(799.61983069,241.07103521)(799.70483061,241.02603526)(799.79483398,240.98603638)
\curveto(799.89483042,240.95603533)(799.99983031,240.92603536)(800.10983398,240.89603638)
\curveto(800.15983015,240.87603541)(800.20483011,240.86603542)(800.24483398,240.86603638)
\curveto(800.29483002,240.87603541)(800.34482997,240.87603541)(800.39483398,240.86603638)
\curveto(800.42482989,240.85603543)(800.48482983,240.84603544)(800.57483398,240.83603638)
\curveto(800.67482964,240.82603546)(800.74982956,240.83103545)(800.79983398,240.85103638)
\curveto(800.83982947,240.86103542)(800.87982943,240.86103542)(800.91983398,240.85103638)
\curveto(800.95982935,240.85103543)(800.99982931,240.86103542)(801.03983398,240.88103638)
\curveto(801.11982919,240.90103538)(801.19982911,240.91603537)(801.27983398,240.92603638)
\curveto(801.35982895,240.94603534)(801.43482888,240.97103531)(801.50483398,241.00103638)
\curveto(801.84482847,241.14103514)(802.11982819,241.33603495)(802.32983398,241.58603638)
\curveto(802.53982777,241.83603445)(802.7148276,242.13103415)(802.85483398,242.47103638)
\curveto(802.90482741,242.59103369)(802.93482738,242.71603357)(802.94483398,242.84603638)
\curveto(802.96482735,242.9860333)(802.99482732,243.12603316)(803.03483398,243.26603638)
}
}
{
\newrgbcolor{curcolor}{0 0 0}
\pscustom[linestyle=none,fillstyle=solid,fillcolor=curcolor]
{
\newpath
\moveto(812.47311523,247.54103638)
\curveto(812.54310763,247.49102879)(812.5781076,247.41602887)(812.57811523,247.31603638)
\curveto(812.58810759,247.21602907)(812.59310758,247.11102917)(812.59311523,247.00103638)
\lineto(812.59311523,240.73103638)
\lineto(812.59311523,240.13103638)
\curveto(812.5731076,240.0810362)(812.56810761,240.03103625)(812.57811523,239.98103638)
\curveto(812.58810759,239.94103634)(812.58310759,239.89603639)(812.56311523,239.84603638)
\curveto(812.54310763,239.74603654)(812.52810765,239.64603664)(812.51811523,239.54603638)
\curveto(812.51810766,239.43603685)(812.50310767,239.33103695)(812.47311523,239.23103638)
\curveto(812.44310773,239.12103716)(812.41310776,239.01603727)(812.38311523,238.91603638)
\curveto(812.36310781,238.81603747)(812.32810785,238.71603757)(812.27811523,238.61603638)
\curveto(812.178108,238.35603793)(812.04810813,238.12103816)(811.88811523,237.91103638)
\curveto(811.73810844,237.70103858)(811.55810862,237.52603876)(811.34811523,237.38603638)
\curveto(811.178109,237.26603902)(810.99810918,237.17103911)(810.80811523,237.10103638)
\curveto(810.61810956,237.02103926)(810.41310976,236.94603934)(810.19311523,236.87603638)
\curveto(810.10311007,236.85603943)(810.01311016,236.84603944)(809.92311523,236.84603638)
\curveto(809.83311034,236.83603945)(809.74311043,236.82103946)(809.65311523,236.80103638)
\lineto(809.56311523,236.80103638)
\curveto(809.54311063,236.79103949)(809.52311065,236.7860395)(809.50311523,236.78603638)
\curveto(809.45311072,236.77603951)(809.40311077,236.77603951)(809.35311523,236.78603638)
\curveto(809.31311086,236.79603949)(809.26811091,236.79103949)(809.21811523,236.77103638)
\curveto(809.14811103,236.75103953)(809.03811114,236.74603954)(808.88811523,236.75603638)
\curveto(808.74811143,236.75603953)(808.64811153,236.76603952)(808.58811523,236.78603638)
\curveto(808.55811162,236.7860395)(808.52811165,236.79103949)(808.49811523,236.80103638)
\lineto(808.43811523,236.80103638)
\curveto(808.34811183,236.82103946)(808.25811192,236.83603945)(808.16811523,236.84603638)
\curveto(808.0781121,236.84603944)(807.99311218,236.85603943)(807.91311523,236.87603638)
\curveto(807.83311234,236.89603939)(807.75311242,236.92103936)(807.67311523,236.95103638)
\curveto(807.59311258,236.97103931)(807.51311266,236.99603929)(807.43311523,237.02603638)
\curveto(807.11311306,237.15603913)(806.84311333,237.30103898)(806.62311523,237.46103638)
\curveto(806.41311376,237.62103866)(806.22311395,237.84603844)(806.05311523,238.13603638)
\curveto(806.03311414,238.15603813)(806.01811416,238.1810381)(806.00811523,238.21103638)
\curveto(806.00811417,238.23103805)(805.99811418,238.25603803)(805.97811523,238.28603638)
\curveto(805.94811423,238.36603792)(805.91311426,238.4810378)(805.87311523,238.63103638)
\curveto(805.84311433,238.77103751)(805.8731143,238.87603741)(805.96311523,238.94603638)
\curveto(806.02311415,238.99603729)(806.10311407,239.02103726)(806.20311523,239.02103638)
\lineto(806.53311523,239.02103638)
\lineto(806.69811523,239.02103638)
\curveto(806.75811342,239.02103726)(806.81311336,239.01103727)(806.86311523,238.99103638)
\curveto(806.95311322,238.96103732)(807.01811316,238.91103737)(807.05811523,238.84103638)
\curveto(807.09811308,238.77103751)(807.14311303,238.69603759)(807.19311523,238.61603638)
\lineto(807.31311523,238.43603638)
\curveto(807.36311281,238.36603792)(807.41311276,238.31103797)(807.46311523,238.27103638)
\curveto(807.71311246,238.0810382)(808.01311216,237.94103834)(808.36311523,237.85103638)
\curveto(808.42311175,237.83103845)(808.48311169,237.82103846)(808.54311523,237.82103638)
\curveto(808.61311156,237.81103847)(808.6781115,237.79603849)(808.73811523,237.77603638)
\lineto(808.82811523,237.77603638)
\curveto(808.89811128,237.75603853)(808.98311119,237.74603854)(809.08311523,237.74603638)
\curveto(809.18311099,237.74603854)(809.2731109,237.75603853)(809.35311523,237.77603638)
\curveto(809.38311079,237.7860385)(809.42311075,237.79103849)(809.47311523,237.79103638)
\curveto(809.5731106,237.81103847)(809.66811051,237.83103845)(809.75811523,237.85103638)
\curveto(809.84811033,237.86103842)(809.93311024,237.8860384)(810.01311523,237.92603638)
\curveto(810.30310987,238.04603824)(810.53810964,238.21103807)(810.71811523,238.42103638)
\curveto(810.90810927,238.62103766)(811.06310911,238.86603742)(811.18311523,239.15603638)
\curveto(811.22310895,239.24603704)(811.24810893,239.34103694)(811.25811523,239.44103638)
\curveto(811.2781089,239.54103674)(811.30310887,239.64603664)(811.33311523,239.75603638)
\curveto(811.35310882,239.80603648)(811.36310881,239.85603643)(811.36311523,239.90603638)
\curveto(811.36310881,239.95603633)(811.36810881,240.00603628)(811.37811523,240.05603638)
\curveto(811.38810879,240.0860362)(811.39310878,240.13603615)(811.39311523,240.20603638)
\curveto(811.41310876,240.286036)(811.41310876,240.37103591)(811.39311523,240.46103638)
\curveto(811.38310879,240.51103577)(811.3781088,240.55603573)(811.37811523,240.59603638)
\curveto(811.38810879,240.63603565)(811.38310879,240.67103561)(811.36311523,240.70103638)
\curveto(811.34310883,240.72103556)(811.32810885,240.73103555)(811.31811523,240.73103638)
\lineto(811.27311523,240.77603638)
\curveto(811.173109,240.77603551)(811.09810908,240.74603554)(811.04811523,240.68603638)
\curveto(811.00810917,240.63603565)(810.95810922,240.59103569)(810.89811523,240.55103638)
\lineto(810.65811523,240.34103638)
\curveto(810.5781096,240.281036)(810.48810969,240.22603606)(810.38811523,240.17603638)
\curveto(810.24810993,240.0860362)(810.0731101,240.01103627)(809.86311523,239.95103638)
\curveto(809.65311052,239.90103638)(809.43311074,239.86603642)(809.20311523,239.84603638)
\curveto(808.9731112,239.82603646)(808.74311143,239.83103645)(808.51311523,239.86103638)
\curveto(808.28311189,239.8810364)(808.0731121,239.92103636)(807.88311523,239.98103638)
\curveto(806.94311323,240.29103599)(806.28311389,240.8860354)(805.90311523,241.76603638)
\curveto(805.85311432,241.86603442)(805.81311436,241.96103432)(805.78311523,242.05103638)
\curveto(805.75311442,242.15103413)(805.71811446,242.25603403)(805.67811523,242.36603638)
\curveto(805.65811452,242.41603387)(805.64811453,242.46103382)(805.64811523,242.50103638)
\curveto(805.64811453,242.54103374)(805.63811454,242.5860337)(805.61811523,242.63603638)
\curveto(805.59811458,242.70603358)(805.58311459,242.77603351)(805.57311523,242.84603638)
\curveto(805.5731146,242.92603336)(805.56311461,243.00103328)(805.54311523,243.07103638)
\curveto(805.53311464,243.11103317)(805.52811465,243.14603314)(805.52811523,243.17603638)
\curveto(805.53811464,243.21603307)(805.53811464,243.25603303)(805.52811523,243.29603638)
\curveto(805.52811465,243.33603295)(805.52311465,243.37603291)(805.51311523,243.41603638)
\lineto(805.51311523,243.53603638)
\curveto(805.49311468,243.65603263)(805.49311468,243.7810325)(805.51311523,243.91103638)
\curveto(805.52311465,243.97103231)(805.52811465,244.03103225)(805.52811523,244.09103638)
\lineto(805.52811523,244.25603638)
\curveto(805.53811464,244.30603198)(805.54311463,244.34603194)(805.54311523,244.37603638)
\curveto(805.54311463,244.41603187)(805.54811463,244.46103182)(805.55811523,244.51103638)
\curveto(805.58811459,244.62103166)(805.60811457,244.72603156)(805.61811523,244.82603638)
\curveto(805.62811455,244.93603135)(805.65311452,245.04603124)(805.69311523,245.15603638)
\curveto(805.73311444,245.27603101)(805.76811441,245.39103089)(805.79811523,245.50103638)
\curveto(805.83811434,245.62103066)(805.88311429,245.73603055)(805.93311523,245.84603638)
\curveto(806.00311417,246.00603028)(806.08311409,246.15103013)(806.17311523,246.28103638)
\curveto(806.26311391,246.42102986)(806.35811382,246.55602973)(806.45811523,246.68603638)
\curveto(806.52811365,246.79602949)(806.61811356,246.8860294)(806.72811523,246.95603638)
\lineto(806.78811523,247.01603638)
\lineto(806.84811523,247.07603638)
\lineto(806.99811523,247.19603638)
\lineto(807.17811523,247.31603638)
\curveto(807.30811287,247.39602889)(807.44311273,247.46602882)(807.58311523,247.52603638)
\curveto(807.73311244,247.5860287)(807.89311228,247.64102864)(808.06311523,247.69103638)
\curveto(808.16311201,247.72102856)(808.26311191,247.74102854)(808.36311523,247.75103638)
\curveto(808.4731117,247.76102852)(808.58311159,247.77602851)(808.69311523,247.79603638)
\curveto(808.73311144,247.80602848)(808.78311139,247.80602848)(808.84311523,247.79603638)
\curveto(808.91311126,247.7860285)(808.96311121,247.79102849)(808.99311523,247.81103638)
\curveto(809.31311086,247.82102846)(809.59811058,247.79102849)(809.84811523,247.72103638)
\curveto(810.10811007,247.65102863)(810.33810984,247.55102873)(810.53811523,247.42103638)
\curveto(810.60810957,247.3810289)(810.6731095,247.33602895)(810.73311523,247.28603638)
\lineto(810.91311523,247.13603638)
\curveto(810.96310921,247.09602919)(811.00810917,247.05102923)(811.04811523,247.00103638)
\curveto(811.09810908,246.96102932)(811.173109,246.94102934)(811.27311523,246.94103638)
\lineto(811.31811523,246.98603638)
\curveto(811.33810884,247.00602928)(811.35810882,247.03102925)(811.37811523,247.06103638)
\curveto(811.40810877,247.14102914)(811.42310875,247.22102906)(811.42311523,247.30103638)
\curveto(811.43310874,247.3810289)(811.46310871,247.45102883)(811.51311523,247.51103638)
\curveto(811.54310863,247.55102873)(811.60310857,247.5810287)(811.69311523,247.60103638)
\curveto(811.78310839,247.63102865)(811.8781083,247.64602864)(811.97811523,247.64603638)
\curveto(812.0781081,247.64602864)(812.173108,247.63602865)(812.26311523,247.61603638)
\curveto(812.36310781,247.59602869)(812.43310774,247.57102871)(812.47311523,247.54103638)
\moveto(811.34811523,243.76103638)
\curveto(811.35810882,243.80103248)(811.36310881,243.85103243)(811.36311523,243.91103638)
\curveto(811.36310881,243.9810323)(811.35810882,244.03603225)(811.34811523,244.07603638)
\lineto(811.34811523,244.31603638)
\curveto(811.32810885,244.40603188)(811.31310886,244.49103179)(811.30311523,244.57103638)
\curveto(811.29310888,244.66103162)(811.2781089,244.74603154)(811.25811523,244.82603638)
\curveto(811.23810894,244.90603138)(811.21810896,244.9810313)(811.19811523,245.05103638)
\curveto(811.18810899,245.13103115)(811.16810901,245.20603108)(811.13811523,245.27603638)
\curveto(811.02810915,245.55603073)(810.88310929,245.80603048)(810.70311523,246.02603638)
\curveto(810.53310964,246.24603004)(810.31310986,246.41102987)(810.04311523,246.52103638)
\curveto(809.96311021,246.56102972)(809.8781103,246.59102969)(809.78811523,246.61103638)
\curveto(809.69811048,246.64102964)(809.60311057,246.66602962)(809.50311523,246.68603638)
\curveto(809.42311075,246.70602958)(809.33311084,246.71102957)(809.23311523,246.70103638)
\lineto(808.96311523,246.70103638)
\curveto(808.91311126,246.69102959)(808.86311131,246.6860296)(808.81311523,246.68603638)
\curveto(808.7731114,246.6860296)(808.72811145,246.6810296)(808.67811523,246.67103638)
\curveto(808.48811169,246.62102966)(808.32811185,246.57102971)(808.19811523,246.52103638)
\curveto(807.85811232,246.3810299)(807.59311258,246.17103011)(807.40311523,245.89103638)
\curveto(807.21311296,245.61103067)(807.06311311,245.286031)(806.95311523,244.91603638)
\curveto(806.93311324,244.83603145)(806.91811326,244.75603153)(806.90811523,244.67603638)
\curveto(806.90811327,244.60603168)(806.89811328,244.53103175)(806.87811523,244.45103638)
\curveto(806.85811332,244.42103186)(806.84811333,244.3860319)(806.84811523,244.34603638)
\curveto(806.85811332,244.30603198)(806.85811332,244.27103201)(806.84811523,244.24103638)
\lineto(806.84811523,243.91103638)
\lineto(806.84811523,243.56603638)
\curveto(806.84811333,243.45603283)(806.85811332,243.35103293)(806.87811523,243.25103638)
\lineto(806.87811523,243.17603638)
\curveto(806.88811329,243.14603314)(806.89311328,243.12103316)(806.89311523,243.10103638)
\curveto(806.91311326,243.01103327)(806.92811325,242.92103336)(806.93811523,242.83103638)
\curveto(806.95811322,242.74103354)(806.98311319,242.65603363)(807.01311523,242.57603638)
\curveto(807.09311308,242.31603397)(807.19311298,242.07603421)(807.31311523,241.85603638)
\curveto(807.43311274,241.63603465)(807.59311258,241.45603483)(807.79311523,241.31603638)
\lineto(807.91311523,241.22603638)
\curveto(807.95311222,241.20603508)(807.99811218,241.1860351)(808.04811523,241.16603638)
\curveto(808.12811205,241.11603517)(808.21311196,241.07603521)(808.30311523,241.04603638)
\curveto(808.39311178,241.01603527)(808.49311168,240.9860353)(808.60311523,240.95603638)
\curveto(808.65311152,240.94603534)(808.69811148,240.94103534)(808.73811523,240.94103638)
\curveto(808.78811139,240.95103533)(808.83811134,240.94603534)(808.88811523,240.92603638)
\curveto(808.91811126,240.91603537)(808.96811121,240.91103537)(809.03811523,240.91103638)
\curveto(809.10811107,240.91103537)(809.15811102,240.91603537)(809.18811523,240.92603638)
\curveto(809.21811096,240.93603535)(809.24811093,240.93603535)(809.27811523,240.92603638)
\curveto(809.31811086,240.92603536)(809.35811082,240.93103535)(809.39811523,240.94103638)
\curveto(809.48811069,240.96103532)(809.5731106,240.9810353)(809.65311523,241.00103638)
\curveto(809.73311044,241.02103526)(809.81311036,241.04603524)(809.89311523,241.07603638)
\curveto(810.23310994,241.22603506)(810.50310967,241.43603485)(810.70311523,241.70603638)
\curveto(810.90310927,241.97603431)(811.06310911,242.29103399)(811.18311523,242.65103638)
\curveto(811.21310896,242.74103354)(811.23310894,242.83103345)(811.24311523,242.92103638)
\curveto(811.26310891,243.02103326)(811.28310889,243.11603317)(811.30311523,243.20603638)
\curveto(811.31310886,243.24603304)(811.31810886,243.281033)(811.31811523,243.31103638)
\curveto(811.31810886,243.35103293)(811.32310885,243.39103289)(811.33311523,243.43103638)
\curveto(811.35310882,243.4810328)(811.35310882,243.53103275)(811.33311523,243.58103638)
\curveto(811.32310885,243.64103264)(811.32810885,243.70103258)(811.34811523,243.76103638)
}
}
{
\newrgbcolor{curcolor}{0 0 0}
\pscustom[linestyle=none,fillstyle=solid,fillcolor=curcolor]
{
\newpath
\moveto(818.11639648,247.82603638)
\curveto(818.34639169,247.82602846)(818.47639156,247.76602852)(818.50639648,247.64603638)
\curveto(818.5363915,247.53602875)(818.55139149,247.37102891)(818.55139648,247.15103638)
\lineto(818.55139648,246.86603638)
\curveto(818.55139149,246.77602951)(818.52639151,246.70102958)(818.47639648,246.64103638)
\curveto(818.41639162,246.56102972)(818.33139171,246.51602977)(818.22139648,246.50603638)
\curveto(818.11139193,246.50602978)(818.00139204,246.49102979)(817.89139648,246.46103638)
\curveto(817.75139229,246.43102985)(817.61639242,246.40102988)(817.48639648,246.37103638)
\curveto(817.36639267,246.34102994)(817.25139279,246.30102998)(817.14139648,246.25103638)
\curveto(816.85139319,246.12103016)(816.61639342,245.94103034)(816.43639648,245.71103638)
\curveto(816.25639378,245.49103079)(816.10139394,245.23603105)(815.97139648,244.94603638)
\curveto(815.93139411,244.83603145)(815.90139414,244.72103156)(815.88139648,244.60103638)
\curveto(815.86139418,244.49103179)(815.8363942,244.37603191)(815.80639648,244.25603638)
\curveto(815.79639424,244.20603208)(815.79139425,244.15603213)(815.79139648,244.10603638)
\curveto(815.80139424,244.05603223)(815.80139424,244.00603228)(815.79139648,243.95603638)
\curveto(815.76139428,243.83603245)(815.74639429,243.69603259)(815.74639648,243.53603638)
\curveto(815.75639428,243.3860329)(815.76139428,243.24103304)(815.76139648,243.10103638)
\lineto(815.76139648,241.25603638)
\lineto(815.76139648,240.91103638)
\curveto(815.76139428,240.79103549)(815.75639428,240.67603561)(815.74639648,240.56603638)
\curveto(815.7363943,240.45603583)(815.73139431,240.36103592)(815.73139648,240.28103638)
\curveto(815.7413943,240.20103608)(815.72139432,240.13103615)(815.67139648,240.07103638)
\curveto(815.62139442,240.00103628)(815.5413945,239.96103632)(815.43139648,239.95103638)
\curveto(815.33139471,239.94103634)(815.22139482,239.93603635)(815.10139648,239.93603638)
\lineto(814.83139648,239.93603638)
\curveto(814.78139526,239.95603633)(814.73139531,239.97103631)(814.68139648,239.98103638)
\curveto(814.6413954,240.00103628)(814.61139543,240.02603626)(814.59139648,240.05603638)
\curveto(814.5413955,240.12603616)(814.51139553,240.21103607)(814.50139648,240.31103638)
\lineto(814.50139648,240.64103638)
\lineto(814.50139648,241.79603638)
\lineto(814.50139648,245.95103638)
\lineto(814.50139648,246.98603638)
\lineto(814.50139648,247.28603638)
\curveto(814.51139553,247.3860289)(814.5413955,247.47102881)(814.59139648,247.54103638)
\curveto(814.62139542,247.5810287)(814.67139537,247.61102867)(814.74139648,247.63103638)
\curveto(814.82139522,247.65102863)(814.90639513,247.66102862)(814.99639648,247.66103638)
\curveto(815.08639495,247.67102861)(815.17639486,247.67102861)(815.26639648,247.66103638)
\curveto(815.35639468,247.65102863)(815.42639461,247.63602865)(815.47639648,247.61603638)
\curveto(815.55639448,247.5860287)(815.60639443,247.52602876)(815.62639648,247.43603638)
\curveto(815.65639438,247.35602893)(815.67139437,247.26602902)(815.67139648,247.16603638)
\lineto(815.67139648,246.86603638)
\curveto(815.67139437,246.76602952)(815.69139435,246.67602961)(815.73139648,246.59603638)
\curveto(815.7413943,246.57602971)(815.75139429,246.56102972)(815.76139648,246.55103638)
\lineto(815.80639648,246.50603638)
\curveto(815.91639412,246.50602978)(816.00639403,246.55102973)(816.07639648,246.64103638)
\curveto(816.14639389,246.74102954)(816.20639383,246.82102946)(816.25639648,246.88103638)
\lineto(816.34639648,246.97103638)
\curveto(816.4363936,247.0810292)(816.56139348,247.19602909)(816.72139648,247.31603638)
\curveto(816.88139316,247.43602885)(817.03139301,247.52602876)(817.17139648,247.58603638)
\curveto(817.26139278,247.63602865)(817.35639268,247.67102861)(817.45639648,247.69103638)
\curveto(817.55639248,247.72102856)(817.66139238,247.75102853)(817.77139648,247.78103638)
\curveto(817.83139221,247.79102849)(817.89139215,247.79602849)(817.95139648,247.79603638)
\curveto(818.01139203,247.80602848)(818.06639197,247.81602847)(818.11639648,247.82603638)
}
}
{
\newrgbcolor{curcolor}{0 0 0}
\pscustom[linestyle=none,fillstyle=solid,fillcolor=curcolor]
{
\newpath
\moveto(826.36616211,240.47603638)
\curveto(826.39615428,240.31603597)(826.38115429,240.1810361)(826.32116211,240.07103638)
\curveto(826.26115441,239.97103631)(826.18115449,239.89603639)(826.08116211,239.84603638)
\curveto(826.03115464,239.82603646)(825.9761547,239.81603647)(825.91616211,239.81603638)
\curveto(825.86615481,239.81603647)(825.81115486,239.80603648)(825.75116211,239.78603638)
\curveto(825.53115514,239.73603655)(825.31115536,239.75103653)(825.09116211,239.83103638)
\curveto(824.88115579,239.90103638)(824.73615594,239.99103629)(824.65616211,240.10103638)
\curveto(824.60615607,240.17103611)(824.56115611,240.25103603)(824.52116211,240.34103638)
\curveto(824.48115619,240.44103584)(824.43115624,240.52103576)(824.37116211,240.58103638)
\curveto(824.35115632,240.60103568)(824.32615635,240.62103566)(824.29616211,240.64103638)
\curveto(824.2761564,240.66103562)(824.24615643,240.66603562)(824.20616211,240.65603638)
\curveto(824.09615658,240.62603566)(823.99115668,240.57103571)(823.89116211,240.49103638)
\curveto(823.80115687,240.41103587)(823.71115696,240.34103594)(823.62116211,240.28103638)
\curveto(823.49115718,240.20103608)(823.35115732,240.12603616)(823.20116211,240.05603638)
\curveto(823.05115762,239.99603629)(822.89115778,239.94103634)(822.72116211,239.89103638)
\curveto(822.62115805,239.86103642)(822.51115816,239.84103644)(822.39116211,239.83103638)
\curveto(822.28115839,239.82103646)(822.1711585,239.80603648)(822.06116211,239.78603638)
\curveto(822.01115866,239.77603651)(821.96615871,239.77103651)(821.92616211,239.77103638)
\lineto(821.82116211,239.77103638)
\curveto(821.71115896,239.75103653)(821.60615907,239.75103653)(821.50616211,239.77103638)
\lineto(821.37116211,239.77103638)
\curveto(821.32115935,239.7810365)(821.2711594,239.7860365)(821.22116211,239.78603638)
\curveto(821.1711595,239.7860365)(821.12615955,239.79603649)(821.08616211,239.81603638)
\curveto(821.04615963,239.82603646)(821.01115966,239.83103645)(820.98116211,239.83103638)
\curveto(820.96115971,239.82103646)(820.93615974,239.82103646)(820.90616211,239.83103638)
\lineto(820.66616211,239.89103638)
\curveto(820.58616009,239.90103638)(820.51116016,239.92103636)(820.44116211,239.95103638)
\curveto(820.14116053,240.0810362)(819.89616078,240.22603606)(819.70616211,240.38603638)
\curveto(819.52616115,240.55603573)(819.3761613,240.79103549)(819.25616211,241.09103638)
\curveto(819.16616151,241.31103497)(819.12116155,241.57603471)(819.12116211,241.88603638)
\lineto(819.12116211,242.20103638)
\curveto(819.13116154,242.25103403)(819.13616154,242.30103398)(819.13616211,242.35103638)
\lineto(819.16616211,242.53103638)
\lineto(819.28616211,242.86103638)
\curveto(819.32616135,242.97103331)(819.3761613,243.07103321)(819.43616211,243.16103638)
\curveto(819.61616106,243.45103283)(819.86116081,243.66603262)(820.17116211,243.80603638)
\curveto(820.48116019,243.94603234)(820.82115985,244.07103221)(821.19116211,244.18103638)
\curveto(821.33115934,244.22103206)(821.4761592,244.25103203)(821.62616211,244.27103638)
\curveto(821.7761589,244.29103199)(821.92615875,244.31603197)(822.07616211,244.34603638)
\curveto(822.14615853,244.36603192)(822.21115846,244.37603191)(822.27116211,244.37603638)
\curveto(822.34115833,244.37603191)(822.41615826,244.3860319)(822.49616211,244.40603638)
\curveto(822.56615811,244.42603186)(822.63615804,244.43603185)(822.70616211,244.43603638)
\curveto(822.7761579,244.44603184)(822.85115782,244.46103182)(822.93116211,244.48103638)
\curveto(823.18115749,244.54103174)(823.41615726,244.59103169)(823.63616211,244.63103638)
\curveto(823.85615682,244.6810316)(824.03115664,244.79603149)(824.16116211,244.97603638)
\curveto(824.22115645,245.05603123)(824.2711564,245.15603113)(824.31116211,245.27603638)
\curveto(824.35115632,245.40603088)(824.35115632,245.54603074)(824.31116211,245.69603638)
\curveto(824.25115642,245.93603035)(824.16115651,246.12603016)(824.04116211,246.26603638)
\curveto(823.93115674,246.40602988)(823.7711569,246.51602977)(823.56116211,246.59603638)
\curveto(823.44115723,246.64602964)(823.29615738,246.6810296)(823.12616211,246.70103638)
\curveto(822.96615771,246.72102956)(822.79615788,246.73102955)(822.61616211,246.73103638)
\curveto(822.43615824,246.73102955)(822.26115841,246.72102956)(822.09116211,246.70103638)
\curveto(821.92115875,246.6810296)(821.7761589,246.65102963)(821.65616211,246.61103638)
\curveto(821.48615919,246.55102973)(821.32115935,246.46602982)(821.16116211,246.35603638)
\curveto(821.08115959,246.29602999)(821.00615967,246.21603007)(820.93616211,246.11603638)
\curveto(820.8761598,246.02603026)(820.82115985,245.92603036)(820.77116211,245.81603638)
\curveto(820.74115993,245.73603055)(820.71115996,245.65103063)(820.68116211,245.56103638)
\curveto(820.66116001,245.47103081)(820.61616006,245.40103088)(820.54616211,245.35103638)
\curveto(820.50616017,245.32103096)(820.43616024,245.29603099)(820.33616211,245.27603638)
\curveto(820.24616043,245.26603102)(820.15116052,245.26103102)(820.05116211,245.26103638)
\curveto(819.95116072,245.26103102)(819.85116082,245.26603102)(819.75116211,245.27603638)
\curveto(819.66116101,245.29603099)(819.59616108,245.32103096)(819.55616211,245.35103638)
\curveto(819.51616116,245.3810309)(819.48616119,245.43103085)(819.46616211,245.50103638)
\curveto(819.44616123,245.57103071)(819.44616123,245.64603064)(819.46616211,245.72603638)
\curveto(819.49616118,245.85603043)(819.52616115,245.97603031)(819.55616211,246.08603638)
\curveto(819.59616108,246.20603008)(819.64116103,246.32102996)(819.69116211,246.43103638)
\curveto(819.88116079,246.7810295)(820.12116055,247.05102923)(820.41116211,247.24103638)
\curveto(820.70115997,247.44102884)(821.06115961,247.60102868)(821.49116211,247.72103638)
\curveto(821.59115908,247.74102854)(821.69115898,247.75602853)(821.79116211,247.76603638)
\curveto(821.90115877,247.77602851)(822.01115866,247.79102849)(822.12116211,247.81103638)
\curveto(822.16115851,247.82102846)(822.22615845,247.82102846)(822.31616211,247.81103638)
\curveto(822.40615827,247.81102847)(822.46115821,247.82102846)(822.48116211,247.84103638)
\curveto(823.18115749,247.85102843)(823.79115688,247.77102851)(824.31116211,247.60103638)
\curveto(824.83115584,247.43102885)(825.19615548,247.10602918)(825.40616211,246.62603638)
\curveto(825.49615518,246.42602986)(825.54615513,246.19103009)(825.55616211,245.92103638)
\curveto(825.5761551,245.66103062)(825.58615509,245.3860309)(825.58616211,245.09603638)
\lineto(825.58616211,241.78103638)
\curveto(825.58615509,241.64103464)(825.59115508,241.50603478)(825.60116211,241.37603638)
\curveto(825.61115506,241.24603504)(825.64115503,241.14103514)(825.69116211,241.06103638)
\curveto(825.74115493,240.99103529)(825.80615487,240.94103534)(825.88616211,240.91103638)
\curveto(825.9761547,240.87103541)(826.06115461,240.84103544)(826.14116211,240.82103638)
\curveto(826.22115445,240.81103547)(826.28115439,240.76603552)(826.32116211,240.68603638)
\curveto(826.34115433,240.65603563)(826.35115432,240.62603566)(826.35116211,240.59603638)
\curveto(826.35115432,240.56603572)(826.35615432,240.52603576)(826.36616211,240.47603638)
\moveto(824.22116211,242.14103638)
\curveto(824.28115639,242.281034)(824.31115636,242.44103384)(824.31116211,242.62103638)
\curveto(824.32115635,242.81103347)(824.32615635,243.00603328)(824.32616211,243.20603638)
\curveto(824.32615635,243.31603297)(824.32115635,243.41603287)(824.31116211,243.50603638)
\curveto(824.30115637,243.59603269)(824.26115641,243.66603262)(824.19116211,243.71603638)
\curveto(824.16115651,243.73603255)(824.09115658,243.74603254)(823.98116211,243.74603638)
\curveto(823.96115671,243.72603256)(823.92615675,243.71603257)(823.87616211,243.71603638)
\curveto(823.82615685,243.71603257)(823.78115689,243.70603258)(823.74116211,243.68603638)
\curveto(823.66115701,243.66603262)(823.5711571,243.64603264)(823.47116211,243.62603638)
\lineto(823.17116211,243.56603638)
\curveto(823.14115753,243.56603272)(823.10615757,243.56103272)(823.06616211,243.55103638)
\lineto(822.96116211,243.55103638)
\curveto(822.81115786,243.51103277)(822.64615803,243.4860328)(822.46616211,243.47603638)
\curveto(822.29615838,243.47603281)(822.13615854,243.45603283)(821.98616211,243.41603638)
\curveto(821.90615877,243.39603289)(821.83115884,243.37603291)(821.76116211,243.35603638)
\curveto(821.70115897,243.34603294)(821.63115904,243.33103295)(821.55116211,243.31103638)
\curveto(821.39115928,243.26103302)(821.24115943,243.19603309)(821.10116211,243.11603638)
\curveto(820.96115971,243.04603324)(820.84115983,242.95603333)(820.74116211,242.84603638)
\curveto(820.64116003,242.73603355)(820.56616011,242.60103368)(820.51616211,242.44103638)
\curveto(820.46616021,242.29103399)(820.44616023,242.10603418)(820.45616211,241.88603638)
\curveto(820.45616022,241.7860345)(820.4711602,241.69103459)(820.50116211,241.60103638)
\curveto(820.54116013,241.52103476)(820.58616009,241.44603484)(820.63616211,241.37603638)
\curveto(820.71615996,241.26603502)(820.82115985,241.17103511)(820.95116211,241.09103638)
\curveto(821.08115959,241.02103526)(821.22115945,240.96103532)(821.37116211,240.91103638)
\curveto(821.42115925,240.90103538)(821.4711592,240.89603539)(821.52116211,240.89603638)
\curveto(821.5711591,240.89603539)(821.62115905,240.89103539)(821.67116211,240.88103638)
\curveto(821.74115893,240.86103542)(821.82615885,240.84603544)(821.92616211,240.83603638)
\curveto(822.03615864,240.83603545)(822.12615855,240.84603544)(822.19616211,240.86603638)
\curveto(822.25615842,240.8860354)(822.31615836,240.89103539)(822.37616211,240.88103638)
\curveto(822.43615824,240.8810354)(822.49615818,240.89103539)(822.55616211,240.91103638)
\curveto(822.63615804,240.93103535)(822.71115796,240.94603534)(822.78116211,240.95603638)
\curveto(822.86115781,240.96603532)(822.93615774,240.9860353)(823.00616211,241.01603638)
\curveto(823.29615738,241.13603515)(823.54115713,241.281035)(823.74116211,241.45103638)
\curveto(823.95115672,241.62103466)(824.11115656,241.85103443)(824.22116211,242.14103638)
}
}
{
\newrgbcolor{curcolor}{0 0 0}
\pscustom[linestyle=none,fillstyle=solid,fillcolor=curcolor]
{
\newpath
\moveto(829.98280273,250.72103638)
\curveto(830.16279919,250.73102555)(830.352799,250.73102555)(830.55280273,250.72103638)
\curveto(830.7527986,250.71102557)(830.89279846,250.65102563)(830.97280273,250.54103638)
\curveto(831.01279834,250.4810258)(831.03779832,250.40602588)(831.04780273,250.31603638)
\curveto(831.0577983,250.23602605)(831.06279829,250.14602614)(831.06280273,250.04603638)
\curveto(831.06279829,249.91602637)(831.03779832,249.81102647)(830.98780273,249.73103638)
\curveto(830.94779841,249.6810266)(830.88779847,249.64602664)(830.80780273,249.62603638)
\curveto(830.73779862,249.61602667)(830.6577987,249.61102667)(830.56780273,249.61103638)
\lineto(830.28280273,249.61103638)
\curveto(830.19279916,249.62102666)(830.11279924,249.62102666)(830.04280273,249.61103638)
\curveto(829.76279959,249.53102675)(829.57779978,249.40102688)(829.48780273,249.22103638)
\curveto(829.40779995,249.05102723)(829.36779999,248.79102749)(829.36780273,248.44103638)
\curveto(829.36779999,248.37102791)(829.36279999,248.29602799)(829.35280273,248.21603638)
\curveto(829.34280001,248.14602814)(829.34780001,248.0810282)(829.36780273,248.02103638)
\curveto(829.39779996,247.87102841)(829.46279989,247.76602852)(829.56280273,247.70603638)
\curveto(829.64279971,247.67602861)(829.74279961,247.66102862)(829.86280273,247.66103638)
\lineto(830.22280273,247.66103638)
\lineto(830.44780273,247.66103638)
\curveto(830.47779888,247.64102864)(830.50779885,247.63602865)(830.53780273,247.64603638)
\curveto(830.56779879,247.65602863)(830.59779876,247.65102863)(830.62780273,247.63103638)
\curveto(830.72779863,247.60102868)(830.79279856,247.54102874)(830.82280273,247.45103638)
\curveto(830.8527985,247.37102891)(830.86779849,247.26602902)(830.86780273,247.13603638)
\curveto(830.8577985,247.09602919)(830.8527985,247.05602923)(830.85280273,247.01603638)
\lineto(830.85280273,246.89603638)
\curveto(830.82279853,246.74602954)(830.7577986,246.64602964)(830.65780273,246.59603638)
\curveto(830.52779883,246.54602974)(830.357799,246.53102975)(830.14780273,246.55103638)
\curveto(829.94779941,246.5810297)(829.77779958,246.57602971)(829.63780273,246.53603638)
\curveto(829.5577998,246.51602977)(829.49779986,246.47602981)(829.45780273,246.41603638)
\curveto(829.41779994,246.36602992)(829.38779997,246.29602999)(829.36780273,246.20603638)
\curveto(829.34780001,246.13603015)(829.34280001,246.05603023)(829.35280273,245.96603638)
\curveto(829.36279999,245.87603041)(829.36779999,245.79103049)(829.36780273,245.71103638)
\lineto(829.36780273,244.72103638)
\lineto(829.36780273,241.54103638)
\lineto(829.36780273,240.79103638)
\lineto(829.36780273,240.59603638)
\curveto(829.37779998,240.52603576)(829.37279998,240.46603582)(829.35280273,240.41603638)
\lineto(829.35280273,240.29603638)
\lineto(829.32280273,240.17603638)
\curveto(829.31280004,240.13603615)(829.29780006,240.10103618)(829.27780273,240.07103638)
\curveto(829.22780013,240.00103628)(829.1528002,239.96103632)(829.05280273,239.95103638)
\curveto(828.9528004,239.94103634)(828.84280051,239.93603635)(828.72280273,239.93603638)
\lineto(828.43780273,239.93603638)
\curveto(828.38780097,239.95603633)(828.33780102,239.97103631)(828.28780273,239.98103638)
\curveto(828.24780111,240.00103628)(828.21280114,240.03603625)(828.18280273,240.08603638)
\curveto(828.16280119,240.11603617)(828.14280121,240.1810361)(828.12280273,240.28103638)
\lineto(828.12280273,240.38603638)
\curveto(828.10280125,240.43603585)(828.09280126,240.4860358)(828.09280273,240.53603638)
\curveto(828.10280125,240.59603569)(828.10780125,240.65103563)(828.10780273,240.70103638)
\lineto(828.10780273,241.30103638)
\lineto(828.10780273,245.39603638)
\lineto(828.10780273,245.74103638)
\curveto(828.11780124,245.86103042)(828.11780124,245.97103031)(828.10780273,246.07103638)
\curveto(828.10780125,246.1810301)(828.08780127,246.27603001)(828.04780273,246.35603638)
\curveto(828.01780134,246.43602985)(827.96280139,246.49102979)(827.88280273,246.52103638)
\curveto(827.82280153,246.55102973)(827.7528016,246.56602972)(827.67280273,246.56603638)
\lineto(827.44780273,246.56603638)
\lineto(827.20780273,246.56603638)
\curveto(827.13780222,246.56602972)(827.07280228,246.57602971)(827.01280273,246.59603638)
\curveto(826.92280243,246.63602965)(826.8578025,246.72102956)(826.81780273,246.85103638)
\curveto(826.80780255,246.90102938)(826.80280255,246.94602934)(826.80280273,246.98603638)
\lineto(826.80280273,247.12103638)
\curveto(826.80280255,247.26102902)(826.81780254,247.37102891)(826.84780273,247.45103638)
\curveto(826.87780248,247.54102874)(826.94280241,247.60102868)(827.04280273,247.63103638)
\curveto(827.11280224,247.66102862)(827.19280216,247.67102861)(827.28280273,247.66103638)
\lineto(827.56780273,247.66103638)
\curveto(827.66780169,247.66102862)(827.7528016,247.67102861)(827.82280273,247.69103638)
\curveto(827.90280145,247.71102857)(827.96780139,247.75102853)(828.01780273,247.81103638)
\curveto(828.08780127,247.89102839)(828.11780124,248.01602827)(828.10780273,248.18603638)
\lineto(828.10780273,248.66603638)
\curveto(828.10780125,248.86602742)(828.11780124,249.05102723)(828.13780273,249.22103638)
\curveto(828.16780119,249.40102688)(828.21280114,249.56102672)(828.27280273,249.70103638)
\curveto(828.38280097,249.94102634)(828.52780083,250.13602615)(828.70780273,250.28603638)
\curveto(828.89780046,250.43602585)(829.12280023,250.55102573)(829.38280273,250.63103638)
\curveto(829.44279991,250.65102563)(829.50279985,250.66102562)(829.56280273,250.66103638)
\curveto(829.63279972,250.67102561)(829.70279965,250.6860256)(829.77280273,250.70603638)
\curveto(829.79279956,250.71602557)(829.82779953,250.71602557)(829.87780273,250.70603638)
\curveto(829.92779943,250.70602558)(829.96279939,250.71102557)(829.98280273,250.72103638)
}
}
{
\newrgbcolor{curcolor}{0 0 0}
\pscustom[linestyle=none,fillstyle=solid,fillcolor=curcolor]
{
\newpath
\moveto(833.43092773,250.84103638)
\curveto(833.50092531,250.84102544)(833.58592522,250.84102544)(833.68592773,250.84103638)
\curveto(833.79592501,250.85102543)(833.89592491,250.85102543)(833.98592773,250.84103638)
\curveto(834.08592472,250.84102544)(834.17592463,250.83102545)(834.25592773,250.81103638)
\curveto(834.33592447,250.79102549)(834.39092442,250.76102552)(834.42092773,250.72103638)
\curveto(834.43092438,250.6810256)(834.42592438,250.62602566)(834.40592773,250.55603638)
\curveto(834.38592442,250.49602579)(834.34592446,250.43602585)(834.28592773,250.37603638)
\lineto(834.12092773,250.21103638)
\curveto(834.07092474,250.16102612)(834.02092479,250.10602618)(833.97092773,250.04603638)
\curveto(833.93092488,249.99602629)(833.88592492,249.94102634)(833.83592773,249.88103638)
\curveto(833.805925,249.83102645)(833.76592504,249.7860265)(833.71592773,249.74603638)
\curveto(833.67592513,249.71602657)(833.63592517,249.67602661)(833.59592773,249.62603638)
\lineto(833.55092773,249.58103638)
\curveto(833.55092526,249.57102671)(833.54092527,249.56102672)(833.52092773,249.55103638)
\curveto(833.48092533,249.50102678)(833.44092537,249.45602683)(833.40092773,249.41603638)
\curveto(833.36092545,249.3860269)(833.32092549,249.34602694)(833.28092773,249.29603638)
\curveto(833.26092555,249.25602703)(833.23092558,249.22102706)(833.19092773,249.19103638)
\lineto(833.10092773,249.10103638)
\curveto(833.06092575,249.05102723)(833.01592579,249.00102728)(832.96592773,248.95103638)
\curveto(832.92592588,248.90102738)(832.88092593,248.86102742)(832.83092773,248.83103638)
\curveto(832.76092605,248.79102749)(832.64592616,248.75602753)(832.48592773,248.72603638)
\curveto(832.33592647,248.70602758)(832.21592659,248.72102756)(832.12592773,248.77103638)
\curveto(832.09592671,248.79102749)(832.06592674,248.82102746)(832.03592773,248.86103638)
\curveto(832.01592679,248.91102737)(832.01592679,248.96602732)(832.03592773,249.02603638)
\curveto(832.05592675,249.10602718)(832.08592672,249.17602711)(832.12592773,249.23603638)
\curveto(832.16592664,249.30602698)(832.2109266,249.37102691)(832.26092773,249.43103638)
\curveto(832.34092647,249.57102671)(832.42592638,249.71602657)(832.51592773,249.86603638)
\curveto(832.6059262,250.01602627)(832.69592611,250.16102612)(832.78592773,250.30103638)
\lineto(832.90592773,250.51103638)
\curveto(832.94592586,250.59102569)(833.00092581,250.65602563)(833.07092773,250.70603638)
\curveto(833.14092567,250.75602553)(833.2109256,250.79602549)(833.28092773,250.82603638)
\curveto(833.3109255,250.82602546)(833.33592547,250.82602546)(833.35592773,250.82603638)
\curveto(833.38592542,250.83602545)(833.4109254,250.84102544)(833.43092773,250.84103638)
\moveto(833.38592773,246.98603638)
\lineto(833.38592773,247.27103638)
\curveto(833.38592542,247.37102891)(833.36092545,247.45102883)(833.31092773,247.51103638)
\curveto(833.26092555,247.59102869)(833.16592564,247.63102865)(833.02592773,247.63103638)
\curveto(832.89592591,247.64102864)(832.76592604,247.64602864)(832.63592773,247.64603638)
\curveto(832.61592619,247.63602865)(832.59092622,247.63102865)(832.56092773,247.63103638)
\curveto(832.54092627,247.64102864)(832.52092629,247.64602864)(832.50092773,247.64603638)
\curveto(832.44092637,247.62602866)(832.38592642,247.61102867)(832.33592773,247.60103638)
\curveto(832.28592652,247.59102869)(832.24592656,247.56102872)(832.21592773,247.51103638)
\curveto(832.16592664,247.45102883)(832.14092667,247.36602892)(832.14092773,247.25603638)
\lineto(832.14092773,246.94103638)
\lineto(832.14092773,240.59603638)
\lineto(832.14092773,240.31103638)
\curveto(832.14092667,240.22103606)(832.16092665,240.14603614)(832.20092773,240.08603638)
\curveto(832.25092656,240.00603628)(832.32092649,239.95603633)(832.41092773,239.93603638)
\curveto(832.5109263,239.92603636)(832.62592618,239.92103636)(832.75592773,239.92103638)
\lineto(832.98092773,239.92103638)
\curveto(833.06092575,239.94103634)(833.13092568,239.95603633)(833.19092773,239.96603638)
\curveto(833.25092556,239.9860363)(833.29592551,240.02603626)(833.32592773,240.08603638)
\curveto(833.37592543,240.14603614)(833.39592541,240.22603606)(833.38592773,240.32603638)
\lineto(833.38592773,240.64103638)
\lineto(833.38592773,246.98603638)
}
}
{
\newrgbcolor{curcolor}{0 0 0}
\pscustom[linestyle=none,fillstyle=solid,fillcolor=curcolor]
{
\newpath
\moveto(842.21577148,240.47603638)
\curveto(842.24576365,240.31603597)(842.23076367,240.1810361)(842.17077148,240.07103638)
\curveto(842.11076379,239.97103631)(842.03076387,239.89603639)(841.93077148,239.84603638)
\curveto(841.88076402,239.82603646)(841.82576407,239.81603647)(841.76577148,239.81603638)
\curveto(841.71576418,239.81603647)(841.66076424,239.80603648)(841.60077148,239.78603638)
\curveto(841.38076452,239.73603655)(841.16076474,239.75103653)(840.94077148,239.83103638)
\curveto(840.73076517,239.90103638)(840.58576531,239.99103629)(840.50577148,240.10103638)
\curveto(840.45576544,240.17103611)(840.41076549,240.25103603)(840.37077148,240.34103638)
\curveto(840.33076557,240.44103584)(840.28076562,240.52103576)(840.22077148,240.58103638)
\curveto(840.2007657,240.60103568)(840.17576572,240.62103566)(840.14577148,240.64103638)
\curveto(840.12576577,240.66103562)(840.0957658,240.66603562)(840.05577148,240.65603638)
\curveto(839.94576595,240.62603566)(839.84076606,240.57103571)(839.74077148,240.49103638)
\curveto(839.65076625,240.41103587)(839.56076634,240.34103594)(839.47077148,240.28103638)
\curveto(839.34076656,240.20103608)(839.2007667,240.12603616)(839.05077148,240.05603638)
\curveto(838.900767,239.99603629)(838.74076716,239.94103634)(838.57077148,239.89103638)
\curveto(838.47076743,239.86103642)(838.36076754,239.84103644)(838.24077148,239.83103638)
\curveto(838.13076777,239.82103646)(838.02076788,239.80603648)(837.91077148,239.78603638)
\curveto(837.86076804,239.77603651)(837.81576808,239.77103651)(837.77577148,239.77103638)
\lineto(837.67077148,239.77103638)
\curveto(837.56076834,239.75103653)(837.45576844,239.75103653)(837.35577148,239.77103638)
\lineto(837.22077148,239.77103638)
\curveto(837.17076873,239.7810365)(837.12076878,239.7860365)(837.07077148,239.78603638)
\curveto(837.02076888,239.7860365)(836.97576892,239.79603649)(836.93577148,239.81603638)
\curveto(836.895769,239.82603646)(836.86076904,239.83103645)(836.83077148,239.83103638)
\curveto(836.81076909,239.82103646)(836.78576911,239.82103646)(836.75577148,239.83103638)
\lineto(836.51577148,239.89103638)
\curveto(836.43576946,239.90103638)(836.36076954,239.92103636)(836.29077148,239.95103638)
\curveto(835.99076991,240.0810362)(835.74577015,240.22603606)(835.55577148,240.38603638)
\curveto(835.37577052,240.55603573)(835.22577067,240.79103549)(835.10577148,241.09103638)
\curveto(835.01577088,241.31103497)(834.97077093,241.57603471)(834.97077148,241.88603638)
\lineto(834.97077148,242.20103638)
\curveto(834.98077092,242.25103403)(834.98577091,242.30103398)(834.98577148,242.35103638)
\lineto(835.01577148,242.53103638)
\lineto(835.13577148,242.86103638)
\curveto(835.17577072,242.97103331)(835.22577067,243.07103321)(835.28577148,243.16103638)
\curveto(835.46577043,243.45103283)(835.71077019,243.66603262)(836.02077148,243.80603638)
\curveto(836.33076957,243.94603234)(836.67076923,244.07103221)(837.04077148,244.18103638)
\curveto(837.18076872,244.22103206)(837.32576857,244.25103203)(837.47577148,244.27103638)
\curveto(837.62576827,244.29103199)(837.77576812,244.31603197)(837.92577148,244.34603638)
\curveto(837.9957679,244.36603192)(838.06076784,244.37603191)(838.12077148,244.37603638)
\curveto(838.19076771,244.37603191)(838.26576763,244.3860319)(838.34577148,244.40603638)
\curveto(838.41576748,244.42603186)(838.48576741,244.43603185)(838.55577148,244.43603638)
\curveto(838.62576727,244.44603184)(838.7007672,244.46103182)(838.78077148,244.48103638)
\curveto(839.03076687,244.54103174)(839.26576663,244.59103169)(839.48577148,244.63103638)
\curveto(839.70576619,244.6810316)(839.88076602,244.79603149)(840.01077148,244.97603638)
\curveto(840.07076583,245.05603123)(840.12076578,245.15603113)(840.16077148,245.27603638)
\curveto(840.2007657,245.40603088)(840.2007657,245.54603074)(840.16077148,245.69603638)
\curveto(840.1007658,245.93603035)(840.01076589,246.12603016)(839.89077148,246.26603638)
\curveto(839.78076612,246.40602988)(839.62076628,246.51602977)(839.41077148,246.59603638)
\curveto(839.29076661,246.64602964)(839.14576675,246.6810296)(838.97577148,246.70103638)
\curveto(838.81576708,246.72102956)(838.64576725,246.73102955)(838.46577148,246.73103638)
\curveto(838.28576761,246.73102955)(838.11076779,246.72102956)(837.94077148,246.70103638)
\curveto(837.77076813,246.6810296)(837.62576827,246.65102963)(837.50577148,246.61103638)
\curveto(837.33576856,246.55102973)(837.17076873,246.46602982)(837.01077148,246.35603638)
\curveto(836.93076897,246.29602999)(836.85576904,246.21603007)(836.78577148,246.11603638)
\curveto(836.72576917,246.02603026)(836.67076923,245.92603036)(836.62077148,245.81603638)
\curveto(836.59076931,245.73603055)(836.56076934,245.65103063)(836.53077148,245.56103638)
\curveto(836.51076939,245.47103081)(836.46576943,245.40103088)(836.39577148,245.35103638)
\curveto(836.35576954,245.32103096)(836.28576961,245.29603099)(836.18577148,245.27603638)
\curveto(836.0957698,245.26603102)(836.0007699,245.26103102)(835.90077148,245.26103638)
\curveto(835.8007701,245.26103102)(835.7007702,245.26603102)(835.60077148,245.27603638)
\curveto(835.51077039,245.29603099)(835.44577045,245.32103096)(835.40577148,245.35103638)
\curveto(835.36577053,245.3810309)(835.33577056,245.43103085)(835.31577148,245.50103638)
\curveto(835.2957706,245.57103071)(835.2957706,245.64603064)(835.31577148,245.72603638)
\curveto(835.34577055,245.85603043)(835.37577052,245.97603031)(835.40577148,246.08603638)
\curveto(835.44577045,246.20603008)(835.49077041,246.32102996)(835.54077148,246.43103638)
\curveto(835.73077017,246.7810295)(835.97076993,247.05102923)(836.26077148,247.24103638)
\curveto(836.55076935,247.44102884)(836.91076899,247.60102868)(837.34077148,247.72103638)
\curveto(837.44076846,247.74102854)(837.54076836,247.75602853)(837.64077148,247.76603638)
\curveto(837.75076815,247.77602851)(837.86076804,247.79102849)(837.97077148,247.81103638)
\curveto(838.01076789,247.82102846)(838.07576782,247.82102846)(838.16577148,247.81103638)
\curveto(838.25576764,247.81102847)(838.31076759,247.82102846)(838.33077148,247.84103638)
\curveto(839.03076687,247.85102843)(839.64076626,247.77102851)(840.16077148,247.60103638)
\curveto(840.68076522,247.43102885)(841.04576485,247.10602918)(841.25577148,246.62603638)
\curveto(841.34576455,246.42602986)(841.3957645,246.19103009)(841.40577148,245.92103638)
\curveto(841.42576447,245.66103062)(841.43576446,245.3860309)(841.43577148,245.09603638)
\lineto(841.43577148,241.78103638)
\curveto(841.43576446,241.64103464)(841.44076446,241.50603478)(841.45077148,241.37603638)
\curveto(841.46076444,241.24603504)(841.49076441,241.14103514)(841.54077148,241.06103638)
\curveto(841.59076431,240.99103529)(841.65576424,240.94103534)(841.73577148,240.91103638)
\curveto(841.82576407,240.87103541)(841.91076399,240.84103544)(841.99077148,240.82103638)
\curveto(842.07076383,240.81103547)(842.13076377,240.76603552)(842.17077148,240.68603638)
\curveto(842.19076371,240.65603563)(842.2007637,240.62603566)(842.20077148,240.59603638)
\curveto(842.2007637,240.56603572)(842.20576369,240.52603576)(842.21577148,240.47603638)
\moveto(840.07077148,242.14103638)
\curveto(840.13076577,242.281034)(840.16076574,242.44103384)(840.16077148,242.62103638)
\curveto(840.17076573,242.81103347)(840.17576572,243.00603328)(840.17577148,243.20603638)
\curveto(840.17576572,243.31603297)(840.17076573,243.41603287)(840.16077148,243.50603638)
\curveto(840.15076575,243.59603269)(840.11076579,243.66603262)(840.04077148,243.71603638)
\curveto(840.01076589,243.73603255)(839.94076596,243.74603254)(839.83077148,243.74603638)
\curveto(839.81076609,243.72603256)(839.77576612,243.71603257)(839.72577148,243.71603638)
\curveto(839.67576622,243.71603257)(839.63076627,243.70603258)(839.59077148,243.68603638)
\curveto(839.51076639,243.66603262)(839.42076648,243.64603264)(839.32077148,243.62603638)
\lineto(839.02077148,243.56603638)
\curveto(838.99076691,243.56603272)(838.95576694,243.56103272)(838.91577148,243.55103638)
\lineto(838.81077148,243.55103638)
\curveto(838.66076724,243.51103277)(838.4957674,243.4860328)(838.31577148,243.47603638)
\curveto(838.14576775,243.47603281)(837.98576791,243.45603283)(837.83577148,243.41603638)
\curveto(837.75576814,243.39603289)(837.68076822,243.37603291)(837.61077148,243.35603638)
\curveto(837.55076835,243.34603294)(837.48076842,243.33103295)(837.40077148,243.31103638)
\curveto(837.24076866,243.26103302)(837.09076881,243.19603309)(836.95077148,243.11603638)
\curveto(836.81076909,243.04603324)(836.69076921,242.95603333)(836.59077148,242.84603638)
\curveto(836.49076941,242.73603355)(836.41576948,242.60103368)(836.36577148,242.44103638)
\curveto(836.31576958,242.29103399)(836.2957696,242.10603418)(836.30577148,241.88603638)
\curveto(836.30576959,241.7860345)(836.32076958,241.69103459)(836.35077148,241.60103638)
\curveto(836.39076951,241.52103476)(836.43576946,241.44603484)(836.48577148,241.37603638)
\curveto(836.56576933,241.26603502)(836.67076923,241.17103511)(836.80077148,241.09103638)
\curveto(836.93076897,241.02103526)(837.07076883,240.96103532)(837.22077148,240.91103638)
\curveto(837.27076863,240.90103538)(837.32076858,240.89603539)(837.37077148,240.89603638)
\curveto(837.42076848,240.89603539)(837.47076843,240.89103539)(837.52077148,240.88103638)
\curveto(837.59076831,240.86103542)(837.67576822,240.84603544)(837.77577148,240.83603638)
\curveto(837.88576801,240.83603545)(837.97576792,240.84603544)(838.04577148,240.86603638)
\curveto(838.10576779,240.8860354)(838.16576773,240.89103539)(838.22577148,240.88103638)
\curveto(838.28576761,240.8810354)(838.34576755,240.89103539)(838.40577148,240.91103638)
\curveto(838.48576741,240.93103535)(838.56076734,240.94603534)(838.63077148,240.95603638)
\curveto(838.71076719,240.96603532)(838.78576711,240.9860353)(838.85577148,241.01603638)
\curveto(839.14576675,241.13603515)(839.39076651,241.281035)(839.59077148,241.45103638)
\curveto(839.8007661,241.62103466)(839.96076594,241.85103443)(840.07077148,242.14103638)
}
}
{
\newrgbcolor{curcolor}{0.80000001 0.80000001 0.80000001}
\pscustom[linestyle=none,fillstyle=solid,fillcolor=curcolor]
{
\newpath
\moveto(755.9732666,379.02397156)
\lineto(770.9732666,379.02397156)
\lineto(770.9732666,364.02397156)
\lineto(755.9732666,364.02397156)
\closepath
}
}
{
\newrgbcolor{curcolor}{0.7019608 0.7019608 0.7019608}
\pscustom[linestyle=none,fillstyle=solid,fillcolor=curcolor]
{
\newpath
\moveto(755.9732666,355.69264984)
\lineto(770.9732666,355.69264984)
\lineto(770.9732666,340.69264984)
\lineto(755.9732666,340.69264984)
\closepath
}
}
{
\newrgbcolor{curcolor}{0.60000002 0.60000002 0.60000002}
\pscustom[linestyle=none,fillstyle=solid,fillcolor=curcolor]
{
\newpath
\moveto(755.9732666,332.37634277)
\lineto(770.9732666,332.37634277)
\lineto(770.9732666,317.37634277)
\lineto(755.9732666,317.37634277)
\closepath
}
}
{
\newrgbcolor{curcolor}{0.50196081 0.50196081 0.50196081}
\pscustom[linestyle=none,fillstyle=solid,fillcolor=curcolor]
{
\newpath
\moveto(755.9732666,291.3986969)
\lineto(770.9732666,291.3986969)
\lineto(770.9732666,276.3986969)
\lineto(755.9732666,276.3986969)
\closepath
}
}
{
\newrgbcolor{curcolor}{0.40000001 0.40000001 0.40000001}
\pscustom[linestyle=none,fillstyle=solid,fillcolor=curcolor]
{
\newpath
\moveto(755.9732666,250.73603821)
\lineto(770.9732666,250.73603821)
\lineto(770.9732666,235.73603821)
\lineto(755.9732666,235.73603821)
\closepath
}
}
{
\newrgbcolor{curcolor}{0.80000001 0.80000001 0.80000001}
\pscustom[linestyle=none,fillstyle=solid,fillcolor=curcolor]
{
\newpath
\moveto(212.06921387,337.9569397)
\lineto(222.99412441,337.9569397)
\lineto(222.99412441,83.08282471)
\lineto(212.06921387,83.08282471)
\closepath
}
}
{
\newrgbcolor{curcolor}{0.7019608 0.7019608 0.7019608}
\pscustom[linestyle=none,fillstyle=solid,fillcolor=curcolor]
{
\newpath
\moveto(223.05358887,133.0038147)
\lineto(233.97849941,133.0038147)
\lineto(233.97849941,83.08282471)
\lineto(223.05358887,83.08282471)
\closepath
}
}
{
\newrgbcolor{curcolor}{0.60000002 0.60000002 0.60000002}
\pscustom[linestyle=none,fillstyle=solid,fillcolor=curcolor]
{
\newpath
\moveto(234.03796387,105.0194397)
\lineto(244.96287441,105.0194397)
\lineto(244.96287441,83.08282471)
\lineto(234.03796387,83.08282471)
\closepath
}
}
{
\newrgbcolor{curcolor}{0.50196081 0.50196081 0.50196081}
\pscustom[linestyle=none,fillstyle=solid,fillcolor=curcolor]
{
\newpath
\moveto(245.02233887,88.98904419)
\lineto(255.94724941,88.98904419)
\lineto(255.94724941,83.08280468)
\lineto(245.02233887,83.08280468)
\closepath
}
}
{
\newrgbcolor{curcolor}{0.40000001 0.40000001 0.40000001}
\pscustom[linestyle=none,fillstyle=solid,fillcolor=curcolor]
{
\newpath
\moveto(256.00671387,86.98480225)
\lineto(266.93162441,86.98480225)
\lineto(266.93162441,83.08281684)
\lineto(256.00671387,83.08281684)
\closepath
}
}
{
\newrgbcolor{curcolor}{0.80000001 0.80000001 0.80000001}
\pscustom[linestyle=none,fillstyle=solid,fillcolor=curcolor]
{
\newpath
\moveto(121.00671387,86.99966431)
\lineto(131.93162441,86.99966431)
\lineto(131.93162441,83.0828166)
\lineto(121.00671387,83.0828166)
\closepath
}
}
{
\newrgbcolor{curcolor}{0.7019608 0.7019608 0.7019608}
\pscustom[linestyle=none,fillstyle=solid,fillcolor=curcolor]
{
\newpath
\moveto(131.99108887,86.66955566)
\lineto(142.91599941,86.66955566)
\lineto(142.91599941,83.08283734)
\lineto(131.99108887,83.08283734)
\closepath
}
}
{
\newrgbcolor{curcolor}{0.60000002 0.60000002 0.60000002}
\pscustom[linestyle=none,fillstyle=solid,fillcolor=curcolor]
{
\newpath
\moveto(142.97546387,86.9881897)
\lineto(153.90037441,86.9881897)
\lineto(153.90037441,83.08282471)
\lineto(142.97546387,83.08282471)
\closepath
}
}
{
\newrgbcolor{curcolor}{0.50196081 0.50196081 0.50196081}
\pscustom[linestyle=none,fillstyle=solid,fillcolor=curcolor]
{
\newpath
\moveto(153.95983887,86.98904419)
\lineto(164.88474941,86.98904419)
\lineto(164.88474941,83.08280468)
\lineto(153.95983887,83.08280468)
\closepath
}
}
{
\newrgbcolor{curcolor}{0.40000001 0.40000001 0.40000001}
\pscustom[linestyle=none,fillstyle=solid,fillcolor=curcolor]
{
\newpath
\moveto(164.94421387,83.70355225)
\lineto(175.86912441,83.70355225)
\lineto(175.86912441,83.08281684)
\lineto(164.94421387,83.08281684)
\closepath
}
}
{
\newrgbcolor{curcolor}{0.80000001 0.80000001 0.80000001}
\pscustom[linestyle=none,fillstyle=solid,fillcolor=curcolor]
{
\newpath
\moveto(304.01507568,85.98318481)
\lineto(314.93998623,85.98318481)
\lineto(314.93998623,83.08280301)
\lineto(304.01507568,83.08280301)
\closepath
}
}
{
\newrgbcolor{curcolor}{0.7019608 0.7019608 0.7019608}
\pscustom[linestyle=none,fillstyle=solid,fillcolor=curcolor]
{
\newpath
\moveto(314.99945068,85.07855225)
\lineto(325.92436123,85.07855225)
\lineto(325.92436123,83.08282423)
\lineto(314.99945068,83.08282423)
\closepath
}
}
{
\newrgbcolor{curcolor}{0.60000002 0.60000002 0.60000002}
\pscustom[linestyle=none,fillstyle=solid,fillcolor=curcolor]
{
\newpath
\moveto(325.98382568,86.01593018)
\lineto(336.90873623,86.01593018)
\lineto(336.90873623,83.0828371)
\lineto(325.98382568,83.0828371)
\closepath
}
}
{
\newrgbcolor{curcolor}{0.50196081 0.50196081 0.50196081}
\pscustom[linestyle=none,fillstyle=solid,fillcolor=curcolor]
{
\newpath
\moveto(336.96820068,84.51416016)
\lineto(347.89311123,84.51416016)
\lineto(347.89311123,83.08279443)
\lineto(336.96820068,83.08279443)
\closepath
}
}
{
\newrgbcolor{curcolor}{0.40000001 0.40000001 0.40000001}
\pscustom[linestyle=none,fillstyle=solid,fillcolor=curcolor]
{
\newpath
\moveto(347.95257568,83.96871948)
\lineto(358.87748623,83.96871948)
\lineto(358.87748623,83.08281904)
\lineto(347.95257568,83.08281904)
\closepath
}
}
{
\newrgbcolor{curcolor}{0.80000001 0.80000001 0.80000001}
\pscustom[linestyle=none,fillstyle=solid,fillcolor=curcolor]
{
\newpath
\moveto(394.96664429,84.03863525)
\lineto(405.89155483,84.03863525)
\lineto(405.89155483,83.08279711)
\lineto(394.96664429,83.08279711)
\closepath
}
}
{
\newrgbcolor{curcolor}{0.7019608 0.7019608 0.7019608}
\pscustom[linestyle=none,fillstyle=solid,fillcolor=curcolor]
{
\newpath
\moveto(405.95101929,84.03997803)
\lineto(416.87592983,84.03997803)
\lineto(416.87592983,83.08281308)
\lineto(405.95101929,83.08281308)
\closepath
}
}
{
\newrgbcolor{curcolor}{0.60000002 0.60000002 0.60000002}
\pscustom[linestyle=none,fillstyle=solid,fillcolor=curcolor]
{
\newpath
\moveto(416.93539429,84.03634644)
\lineto(427.86030483,84.03634644)
\lineto(427.86030483,83.0828383)
\lineto(416.93539429,83.0828383)
\closepath
}
}
{
\newrgbcolor{curcolor}{0.50196081 0.50196081 0.50196081}
\pscustom[linestyle=none,fillstyle=solid,fillcolor=curcolor]
{
\newpath
\moveto(427.91976929,84.02801514)
\lineto(438.84467983,84.02801514)
\lineto(438.84467983,83.08278531)
\lineto(427.91976929,83.08278531)
\closepath
}
}
{
\newrgbcolor{curcolor}{0.40000001 0.40000001 0.40000001}
\pscustom[linestyle=none,fillstyle=solid,fillcolor=curcolor]
{
\newpath
\moveto(438.90414429,83.47439575)
\lineto(449.82905483,83.47439575)
\lineto(449.82905483,83.08280843)
\lineto(438.90414429,83.08280843)
\closepath
}
}
{
\newrgbcolor{curcolor}{0.80000001 0.80000001 0.80000001}
\pscustom[linestyle=none,fillstyle=solid,fillcolor=curcolor]
{
\newpath
\moveto(486.02270508,83.51934814)
\lineto(496.94761562,83.51934814)
\lineto(496.94761562,83.08279154)
\lineto(486.02270508,83.08279154)
\closepath
}
}
{
\newrgbcolor{curcolor}{0.7019608 0.7019608 0.7019608}
\pscustom[linestyle=none,fillstyle=solid,fillcolor=curcolor]
{
\newpath
\moveto(497.00708008,83.49859619)
\lineto(507.93199062,83.49859619)
\lineto(507.93199062,83.08280987)
\lineto(497.00708008,83.08280987)
\closepath
}
}
{
\newrgbcolor{curcolor}{0.60000002 0.60000002 0.60000002}
\pscustom[linestyle=none,fillstyle=solid,fillcolor=curcolor]
{
\newpath
\moveto(507.99145508,83.48391724)
\lineto(518.91636562,83.48391724)
\lineto(518.91636562,83.08283627)
\lineto(507.99145508,83.08283627)
\closepath
}
}
{
\newrgbcolor{curcolor}{0.50196081 0.50196081 0.50196081}
\pscustom[linestyle=none,fillstyle=solid,fillcolor=curcolor]
{
\newpath
\moveto(518.97583008,83.47802734)
\lineto(529.90074062,83.47802734)
\lineto(529.90074062,83.08275223)
\lineto(518.97583008,83.08275223)
\closepath
}
}
{
\newrgbcolor{curcolor}{0.80000001 0.80000001 0.80000001}
\pscustom[linestyle=none,fillstyle=solid,fillcolor=curcolor]
{
\newpath
\moveto(578.01263428,84.03863525)
\lineto(588.93754482,84.03863525)
\lineto(588.93754482,83.08279711)
\lineto(578.01263428,83.08279711)
\closepath
}
}
{
\newrgbcolor{curcolor}{0.7019608 0.7019608 0.7019608}
\pscustom[linestyle=none,fillstyle=solid,fillcolor=curcolor]
{
\newpath
\moveto(588.99700928,84.03997803)
\lineto(599.92191982,84.03997803)
\lineto(599.92191982,83.08281308)
\lineto(588.99700928,83.08281308)
\closepath
}
}
{
\newrgbcolor{curcolor}{0.60000002 0.60000002 0.60000002}
\pscustom[linestyle=none,fillstyle=solid,fillcolor=curcolor]
{
\newpath
\moveto(599.98138428,84.03634644)
\lineto(610.90629482,84.03634644)
\lineto(610.90629482,83.0828383)
\lineto(599.98138428,83.0828383)
\closepath
}
}
{
\newrgbcolor{curcolor}{0.50196081 0.50196081 0.50196081}
\pscustom[linestyle=none,fillstyle=solid,fillcolor=curcolor]
{
\newpath
\moveto(610.96575928,84.02801514)
\lineto(621.89066982,84.02801514)
\lineto(621.89066982,83.08278531)
\lineto(610.96575928,83.08278531)
\closepath
}
}
{
\newrgbcolor{curcolor}{0.40000001 0.40000001 0.40000001}
\pscustom[linestyle=none,fillstyle=solid,fillcolor=curcolor]
{
\newpath
\moveto(621.95013428,83.7064209)
\lineto(632.87504482,83.7064209)
\lineto(632.87504482,83.08281416)
\lineto(621.95013428,83.08281416)
\closepath
}
}
{
\newrgbcolor{curcolor}{0.80000001 0.80000001 0.80000001}
\pscustom[linestyle=none,fillstyle=solid,fillcolor=curcolor]
{
\newpath
\moveto(668.96020508,83.51934814)
\lineto(679.88511562,83.51934814)
\lineto(679.88511562,83.08279154)
\lineto(668.96020508,83.08279154)
\closepath
}
}
{
\newrgbcolor{curcolor}{0.7019608 0.7019608 0.7019608}
\pscustom[linestyle=none,fillstyle=solid,fillcolor=curcolor]
{
\newpath
\moveto(679.94458008,83.49859619)
\lineto(690.86949062,83.49859619)
\lineto(690.86949062,83.08280987)
\lineto(679.94458008,83.08280987)
\closepath
}
}
{
\newrgbcolor{curcolor}{0.60000002 0.60000002 0.60000002}
\pscustom[linestyle=none,fillstyle=solid,fillcolor=curcolor]
{
\newpath
\moveto(690.92895508,83.48391724)
\lineto(701.85386562,83.48391724)
\lineto(701.85386562,83.08283627)
\lineto(690.92895508,83.08283627)
\closepath
}
}
{
\newrgbcolor{curcolor}{0.50196081 0.50196081 0.50196081}
\pscustom[linestyle=none,fillstyle=solid,fillcolor=curcolor]
{
\newpath
\moveto(701.91333008,83.47802734)
\lineto(712.83824062,83.47802734)
\lineto(712.83824062,83.08275223)
\lineto(701.91333008,83.08275223)
\closepath
}
}
{
\newrgbcolor{curcolor}{0.40000001 0.40000001 0.40000001}
\pscustom[linestyle=none,fillstyle=solid,fillcolor=curcolor]
{
\newpath
\moveto(712.9041748,83.47802734)
\lineto(723.82908535,83.47802734)
\lineto(723.82908535,83.08275223)
\lineto(712.9041748,83.08275223)
\closepath
}
}
\end{pspicture}

\caption{Diagrama de barras de los niveles de intención de los usuarios
clasificados por rol}
\label{usuarios_bars_1}
\end{figure}

En la figura \ref{usuarios_pie_1}, se pueden apreciar las gráficas circulares,
de cada una de las variables de actividad presentadas en el cuadro
\ref{usuarios_tabla_1}; puede notarse en esta, como el predominio en cantidad de
los estudiantes va decayendo progresivamente en intención frente a los otros
roles.

\begin{figure}
\centering
%LaTeX with PSTricks extensions
%%Creator: inkscape 0.48.5
%%Please note this file requires PSTricks extensions
\psset{xunit=.5pt,yunit=.5pt,runit=.5pt}
\begin{pspicture}(763,1078)
{
\newrgbcolor{curcolor}{0 0 0}
\pscustom[linestyle=none,fillstyle=solid,fillcolor=curcolor]
{
\newpath
\moveto(26.7416285,1051.08118698)
\lineto(28.0166285,1051.08118698)
\curveto(28.12662572,1051.08117627)(28.23162561,1051.07617628)(28.3316285,1051.06618698)
\curveto(28.4416254,1051.0561763)(28.52162532,1051.02117633)(28.5716285,1050.96118698)
\curveto(28.62162522,1050.88117647)(28.6466252,1050.77617658)(28.6466285,1050.64618698)
\curveto(28.65662519,1050.52617683)(28.66162518,1050.40117695)(28.6616285,1050.27118698)
\lineto(28.6616285,1048.75618698)
\lineto(28.6616285,1045.66618698)
\lineto(28.6616285,1045.14118698)
\curveto(28.66162518,1045.10118225)(28.65662519,1045.0561823)(28.6466285,1045.00618698)
\curveto(28.6466252,1044.96618239)(28.65162519,1044.92618243)(28.6616285,1044.88618698)
\lineto(28.6616285,1044.64618698)
\curveto(28.66162518,1044.5561828)(28.65662519,1044.46118289)(28.6466285,1044.36118698)
\curveto(28.6466252,1044.26118309)(28.65662519,1044.17118318)(28.6766285,1044.09118698)
\curveto(28.67662517,1044.02118333)(28.68162516,1043.96618339)(28.6916285,1043.92618698)
\curveto(28.71162513,1043.81618354)(28.72662512,1043.70618365)(28.7366285,1043.59618698)
\curveto(28.75662509,1043.48618387)(28.78662506,1043.37618398)(28.8266285,1043.26618698)
\curveto(28.93662491,1043.00618435)(29.07662477,1042.79118456)(29.2466285,1042.62118698)
\curveto(29.42662442,1042.4511849)(29.66162418,1042.31618504)(29.9516285,1042.21618698)
\curveto(30.03162381,1042.19618516)(30.11162373,1042.18118517)(30.1916285,1042.17118698)
\curveto(30.27162357,1042.16118519)(30.35162349,1042.14618521)(30.4316285,1042.12618698)
\curveto(30.48162336,1042.10618525)(30.52662332,1042.09618526)(30.5666285,1042.09618698)
\curveto(30.60662324,1042.10618525)(30.65162319,1042.10618525)(30.7016285,1042.09618698)
\curveto(30.7416231,1042.08618527)(30.80662304,1042.08118527)(30.8966285,1042.08118698)
\curveto(30.98662286,1042.09118526)(31.0466228,1042.10118525)(31.0766285,1042.11118698)
\lineto(31.3016285,1042.11118698)
\curveto(31.38162246,1042.13118522)(31.46162238,1042.14618521)(31.5416285,1042.15618698)
\curveto(31.62162222,1042.16618519)(31.69662215,1042.18118517)(31.7666285,1042.20118698)
\curveto(31.90662194,1042.23118512)(32.01662183,1042.26618509)(32.0966285,1042.30618698)
\curveto(32.27662157,1042.38618497)(32.43162141,1042.49118486)(32.5616285,1042.62118698)
\curveto(32.70162114,1042.76118459)(32.81162103,1042.91618444)(32.8916285,1043.08618698)
\curveto(33.00162084,1043.34618401)(33.06662078,1043.6511837)(33.0866285,1044.00118698)
\curveto(33.10662074,1044.36118299)(33.11662073,1044.73118262)(33.1166285,1045.11118698)
\lineto(33.1166285,1048.09618698)
\lineto(33.1166285,1050.10618698)
\curveto(33.11662073,1050.24617711)(33.11162073,1050.40117695)(33.1016285,1050.57118698)
\curveto(33.10162074,1050.74117661)(33.13162071,1050.86617649)(33.1916285,1050.94618698)
\curveto(33.2416206,1051.00617635)(33.31162053,1051.04117631)(33.4016285,1051.05118698)
\curveto(33.49162035,1051.07117628)(33.59162025,1051.08117627)(33.7016285,1051.08118698)
\lineto(34.6616285,1051.08118698)
\curveto(34.7416191,1051.08117627)(34.81661903,1051.08117627)(34.8866285,1051.08118698)
\curveto(34.96661888,1051.09117626)(35.0416188,1051.08617627)(35.1116285,1051.06618698)
\curveto(35.25161859,1051.03617632)(35.3416185,1050.98617637)(35.3816285,1050.91618698)
\curveto(35.43161841,1050.83617652)(35.45161839,1050.72117663)(35.4416285,1050.57118698)
\curveto(35.4416184,1050.43117692)(35.4416184,1050.30117705)(35.4416285,1050.18118698)
\lineto(35.4416285,1048.17118698)
\lineto(35.4416285,1045.14118698)
\curveto(35.4416184,1044.76118259)(35.43661841,1044.39118296)(35.4266285,1044.03118698)
\curveto(35.41661843,1043.67118368)(35.37161847,1043.34618401)(35.2916285,1043.05618698)
\curveto(35.15161869,1042.58618477)(34.97161887,1042.17618518)(34.7516285,1041.82618698)
\curveto(34.5416193,1041.48618587)(34.26161958,1041.19618616)(33.9116285,1040.95618698)
\curveto(33.60162024,1040.73618662)(33.23662061,1040.5561868)(32.8166285,1040.41618698)
\curveto(32.72662112,1040.38618697)(32.63162121,1040.36118699)(32.5316285,1040.34118698)
\lineto(32.2616285,1040.28118698)
\curveto(32.20162164,1040.26118709)(32.1416217,1040.2511871)(32.0816285,1040.25118698)
\curveto(32.03162181,1040.2511871)(31.97662187,1040.24118711)(31.9166285,1040.22118698)
\curveto(31.79662205,1040.20118715)(31.66162218,1040.18618717)(31.5116285,1040.17618698)
\curveto(31.36162248,1040.16618719)(31.21662263,1040.16118719)(31.0766285,1040.16118698)
\curveto(30.12662372,1040.1511872)(29.31662453,1040.26618709)(28.6466285,1040.50618698)
\curveto(27.97662587,1040.7561866)(27.45162639,1041.1561862)(27.0716285,1041.70618698)
\curveto(26.9416269,1041.88618547)(26.83162701,1042.07118528)(26.7416285,1042.26118698)
\curveto(26.66162718,1042.46118489)(26.58662726,1042.67618468)(26.5166285,1042.90618698)
\curveto(26.49662735,1042.9561844)(26.48662736,1042.99618436)(26.4866285,1043.02618698)
\curveto(26.48662736,1043.06618429)(26.47662737,1043.11118424)(26.4566285,1043.16118698)
\curveto(26.37662747,1043.44118391)(26.33662751,1043.7561836)(26.3366285,1044.10618698)
\lineto(26.3366285,1045.15618698)
\lineto(26.3366285,1049.34118698)
\lineto(26.3366285,1050.39118698)
\lineto(26.3366285,1050.67618698)
\curveto(26.33662751,1050.77617658)(26.35162749,1050.8561765)(26.3816285,1050.91618698)
\curveto(26.4416274,1050.98617637)(26.52162732,1051.03617632)(26.6216285,1051.06618698)
\curveto(26.6416272,1051.06617629)(26.66162718,1051.06617629)(26.6816285,1051.06618698)
\curveto(26.70162714,1051.06617629)(26.72162712,1051.07117628)(26.7416285,1051.08118698)
}
}
{
\newrgbcolor{curcolor}{0 0 0}
\pscustom[linestyle=none,fillstyle=solid,fillcolor=curcolor]
{
\newpath
\moveto(40.20014412,1048.32118698)
\curveto(40.95013962,1048.34117901)(41.60013897,1048.2561791)(42.15014412,1048.06618698)
\curveto(42.71013786,1047.88617947)(43.13513744,1047.57117978)(43.42514412,1047.12118698)
\curveto(43.49513708,1047.01118034)(43.55513702,1046.89618046)(43.60514412,1046.77618698)
\curveto(43.66513691,1046.66618069)(43.71513686,1046.54118081)(43.75514412,1046.40118698)
\curveto(43.7751368,1046.34118101)(43.78513679,1046.27618108)(43.78514412,1046.20618698)
\curveto(43.78513679,1046.13618122)(43.7751368,1046.07618128)(43.75514412,1046.02618698)
\curveto(43.71513686,1045.96618139)(43.66013691,1045.92618143)(43.59014412,1045.90618698)
\curveto(43.54013703,1045.88618147)(43.48013709,1045.87618148)(43.41014412,1045.87618698)
\lineto(43.20014412,1045.87618698)
\lineto(42.54014412,1045.87618698)
\curveto(42.4701381,1045.87618148)(42.40013817,1045.87118148)(42.33014412,1045.86118698)
\curveto(42.26013831,1045.86118149)(42.19513838,1045.87118148)(42.13514412,1045.89118698)
\curveto(42.03513854,1045.91118144)(41.96013861,1045.9511814)(41.91014412,1046.01118698)
\curveto(41.86013871,1046.07118128)(41.81513876,1046.13118122)(41.77514412,1046.19118698)
\lineto(41.65514412,1046.40118698)
\curveto(41.62513895,1046.48118087)(41.575139,1046.54618081)(41.50514412,1046.59618698)
\curveto(41.40513917,1046.67618068)(41.30513927,1046.73618062)(41.20514412,1046.77618698)
\curveto(41.11513946,1046.81618054)(41.00013957,1046.8511805)(40.86014412,1046.88118698)
\curveto(40.79013978,1046.90118045)(40.68513989,1046.91618044)(40.54514412,1046.92618698)
\curveto(40.41514016,1046.93618042)(40.31514026,1046.93118042)(40.24514412,1046.91118698)
\lineto(40.14014412,1046.91118698)
\lineto(39.99014412,1046.88118698)
\curveto(39.95014062,1046.88118047)(39.90514067,1046.87618048)(39.85514412,1046.86618698)
\curveto(39.68514089,1046.81618054)(39.54514103,1046.74618061)(39.43514412,1046.65618698)
\curveto(39.33514124,1046.57618078)(39.26514131,1046.4511809)(39.22514412,1046.28118698)
\curveto(39.20514137,1046.21118114)(39.20514137,1046.14618121)(39.22514412,1046.08618698)
\curveto(39.24514133,1046.02618133)(39.26514131,1045.97618138)(39.28514412,1045.93618698)
\curveto(39.35514122,1045.81618154)(39.43514114,1045.72118163)(39.52514412,1045.65118698)
\curveto(39.62514095,1045.58118177)(39.74014083,1045.52118183)(39.87014412,1045.47118698)
\curveto(40.06014051,1045.39118196)(40.26514031,1045.32118203)(40.48514412,1045.26118698)
\lineto(41.17514412,1045.11118698)
\curveto(41.41513916,1045.07118228)(41.64513893,1045.02118233)(41.86514412,1044.96118698)
\curveto(42.09513848,1044.91118244)(42.31013826,1044.84618251)(42.51014412,1044.76618698)
\curveto(42.60013797,1044.72618263)(42.68513789,1044.69118266)(42.76514412,1044.66118698)
\curveto(42.85513772,1044.64118271)(42.94013763,1044.60618275)(43.02014412,1044.55618698)
\curveto(43.21013736,1044.43618292)(43.38013719,1044.30618305)(43.53014412,1044.16618698)
\curveto(43.69013688,1044.02618333)(43.81513676,1043.8511835)(43.90514412,1043.64118698)
\curveto(43.93513664,1043.57118378)(43.96013661,1043.50118385)(43.98014412,1043.43118698)
\curveto(44.00013657,1043.36118399)(44.02013655,1043.28618407)(44.04014412,1043.20618698)
\curveto(44.05013652,1043.14618421)(44.05513652,1043.0511843)(44.05514412,1042.92118698)
\curveto(44.06513651,1042.80118455)(44.06513651,1042.70618465)(44.05514412,1042.63618698)
\lineto(44.05514412,1042.56118698)
\curveto(44.03513654,1042.50118485)(44.02013655,1042.44118491)(44.01014412,1042.38118698)
\curveto(44.01013656,1042.33118502)(44.00513657,1042.28118507)(43.99514412,1042.23118698)
\curveto(43.92513665,1041.93118542)(43.81513676,1041.66618569)(43.66514412,1041.43618698)
\curveto(43.50513707,1041.19618616)(43.31013726,1041.00118635)(43.08014412,1040.85118698)
\curveto(42.85013772,1040.70118665)(42.59013798,1040.57118678)(42.30014412,1040.46118698)
\curveto(42.19013838,1040.41118694)(42.0701385,1040.37618698)(41.94014412,1040.35618698)
\curveto(41.82013875,1040.33618702)(41.70013887,1040.31118704)(41.58014412,1040.28118698)
\curveto(41.49013908,1040.26118709)(41.39513918,1040.2511871)(41.29514412,1040.25118698)
\curveto(41.20513937,1040.24118711)(41.11513946,1040.22618713)(41.02514412,1040.20618698)
\lineto(40.75514412,1040.20618698)
\curveto(40.69513988,1040.18618717)(40.59013998,1040.17618718)(40.44014412,1040.17618698)
\curveto(40.30014027,1040.17618718)(40.20014037,1040.18618717)(40.14014412,1040.20618698)
\curveto(40.11014046,1040.20618715)(40.0751405,1040.21118714)(40.03514412,1040.22118698)
\lineto(39.93014412,1040.22118698)
\curveto(39.81014076,1040.24118711)(39.69014088,1040.2561871)(39.57014412,1040.26618698)
\curveto(39.45014112,1040.27618708)(39.33514124,1040.29618706)(39.22514412,1040.32618698)
\curveto(38.83514174,1040.43618692)(38.49014208,1040.56118679)(38.19014412,1040.70118698)
\curveto(37.89014268,1040.8511865)(37.63514294,1041.07118628)(37.42514412,1041.36118698)
\curveto(37.28514329,1041.5511858)(37.16514341,1041.77118558)(37.06514412,1042.02118698)
\curveto(37.04514353,1042.08118527)(37.02514355,1042.16118519)(37.00514412,1042.26118698)
\curveto(36.98514359,1042.31118504)(36.9701436,1042.38118497)(36.96014412,1042.47118698)
\curveto(36.95014362,1042.56118479)(36.95514362,1042.63618472)(36.97514412,1042.69618698)
\curveto(37.00514357,1042.76618459)(37.05514352,1042.81618454)(37.12514412,1042.84618698)
\curveto(37.1751434,1042.86618449)(37.23514334,1042.87618448)(37.30514412,1042.87618698)
\lineto(37.53014412,1042.87618698)
\lineto(38.23514412,1042.87618698)
\lineto(38.47514412,1042.87618698)
\curveto(38.55514202,1042.87618448)(38.62514195,1042.86618449)(38.68514412,1042.84618698)
\curveto(38.79514178,1042.80618455)(38.86514171,1042.74118461)(38.89514412,1042.65118698)
\curveto(38.93514164,1042.56118479)(38.98014159,1042.46618489)(39.03014412,1042.36618698)
\curveto(39.05014152,1042.31618504)(39.08514149,1042.2511851)(39.13514412,1042.17118698)
\curveto(39.19514138,1042.09118526)(39.24514133,1042.04118531)(39.28514412,1042.02118698)
\curveto(39.40514117,1041.92118543)(39.52014105,1041.84118551)(39.63014412,1041.78118698)
\curveto(39.74014083,1041.73118562)(39.88014069,1041.68118567)(40.05014412,1041.63118698)
\curveto(40.10014047,1041.61118574)(40.15014042,1041.60118575)(40.20014412,1041.60118698)
\curveto(40.25014032,1041.61118574)(40.30014027,1041.61118574)(40.35014412,1041.60118698)
\curveto(40.43014014,1041.58118577)(40.51514006,1041.57118578)(40.60514412,1041.57118698)
\curveto(40.70513987,1041.58118577)(40.79013978,1041.59618576)(40.86014412,1041.61618698)
\curveto(40.91013966,1041.62618573)(40.95513962,1041.63118572)(40.99514412,1041.63118698)
\curveto(41.04513953,1041.63118572)(41.09513948,1041.64118571)(41.14514412,1041.66118698)
\curveto(41.28513929,1041.71118564)(41.41013916,1041.77118558)(41.52014412,1041.84118698)
\curveto(41.64013893,1041.91118544)(41.73513884,1042.00118535)(41.80514412,1042.11118698)
\curveto(41.85513872,1042.19118516)(41.89513868,1042.31618504)(41.92514412,1042.48618698)
\curveto(41.94513863,1042.5561848)(41.94513863,1042.62118473)(41.92514412,1042.68118698)
\curveto(41.90513867,1042.74118461)(41.88513869,1042.79118456)(41.86514412,1042.83118698)
\curveto(41.79513878,1042.97118438)(41.70513887,1043.07618428)(41.59514412,1043.14618698)
\curveto(41.49513908,1043.21618414)(41.3751392,1043.28118407)(41.23514412,1043.34118698)
\curveto(41.04513953,1043.42118393)(40.84513973,1043.48618387)(40.63514412,1043.53618698)
\curveto(40.42514015,1043.58618377)(40.21514036,1043.64118371)(40.00514412,1043.70118698)
\curveto(39.92514065,1043.72118363)(39.84014073,1043.73618362)(39.75014412,1043.74618698)
\curveto(39.6701409,1043.7561836)(39.59014098,1043.77118358)(39.51014412,1043.79118698)
\curveto(39.19014138,1043.88118347)(38.88514169,1043.96618339)(38.59514412,1044.04618698)
\curveto(38.30514227,1044.13618322)(38.04014253,1044.26618309)(37.80014412,1044.43618698)
\curveto(37.52014305,1044.63618272)(37.31514326,1044.90618245)(37.18514412,1045.24618698)
\curveto(37.16514341,1045.31618204)(37.14514343,1045.41118194)(37.12514412,1045.53118698)
\curveto(37.10514347,1045.60118175)(37.09014348,1045.68618167)(37.08014412,1045.78618698)
\curveto(37.0701435,1045.88618147)(37.0751435,1045.97618138)(37.09514412,1046.05618698)
\curveto(37.11514346,1046.10618125)(37.12014345,1046.14618121)(37.11014412,1046.17618698)
\curveto(37.10014347,1046.21618114)(37.10514347,1046.26118109)(37.12514412,1046.31118698)
\curveto(37.14514343,1046.42118093)(37.16514341,1046.52118083)(37.18514412,1046.61118698)
\curveto(37.21514336,1046.71118064)(37.25014332,1046.80618055)(37.29014412,1046.89618698)
\curveto(37.42014315,1047.18618017)(37.60014297,1047.42117993)(37.83014412,1047.60118698)
\curveto(38.06014251,1047.78117957)(38.32014225,1047.92617943)(38.61014412,1048.03618698)
\curveto(38.72014185,1048.08617927)(38.83514174,1048.12117923)(38.95514412,1048.14118698)
\curveto(39.0751415,1048.17117918)(39.20014137,1048.20117915)(39.33014412,1048.23118698)
\curveto(39.39014118,1048.2511791)(39.45014112,1048.26117909)(39.51014412,1048.26118698)
\lineto(39.69014412,1048.29118698)
\curveto(39.7701408,1048.30117905)(39.85514072,1048.30617905)(39.94514412,1048.30618698)
\curveto(40.03514054,1048.30617905)(40.12014045,1048.31117904)(40.20014412,1048.32118698)
}
}
{
\newrgbcolor{curcolor}{0 0 0}
\pscustom[linestyle=none,fillstyle=solid,fillcolor=curcolor]
{
\newpath
\moveto(45.70678475,1048.09618698)
\lineto(46.83178475,1048.09618698)
\curveto(46.94178231,1048.09617926)(47.04178221,1048.09117926)(47.13178475,1048.08118698)
\curveto(47.22178203,1048.07117928)(47.28678197,1048.03617932)(47.32678475,1047.97618698)
\curveto(47.37678188,1047.91617944)(47.40678185,1047.83117952)(47.41678475,1047.72118698)
\curveto(47.42678183,1047.62117973)(47.43178182,1047.51617984)(47.43178475,1047.40618698)
\lineto(47.43178475,1046.35618698)
\lineto(47.43178475,1044.12118698)
\curveto(47.43178182,1043.76118359)(47.44678181,1043.42118393)(47.47678475,1043.10118698)
\curveto(47.50678175,1042.78118457)(47.59678166,1042.51618484)(47.74678475,1042.30618698)
\curveto(47.88678137,1042.09618526)(48.11178114,1041.94618541)(48.42178475,1041.85618698)
\curveto(48.47178078,1041.84618551)(48.51178074,1041.84118551)(48.54178475,1041.84118698)
\curveto(48.58178067,1041.84118551)(48.62678063,1041.83618552)(48.67678475,1041.82618698)
\curveto(48.72678053,1041.81618554)(48.78178047,1041.81118554)(48.84178475,1041.81118698)
\curveto(48.90178035,1041.81118554)(48.94678031,1041.81618554)(48.97678475,1041.82618698)
\curveto(49.02678023,1041.84618551)(49.06678019,1041.8511855)(49.09678475,1041.84118698)
\curveto(49.13678012,1041.83118552)(49.17678008,1041.83618552)(49.21678475,1041.85618698)
\curveto(49.42677983,1041.90618545)(49.59177966,1041.97118538)(49.71178475,1042.05118698)
\curveto(49.89177936,1042.16118519)(50.03177922,1042.30118505)(50.13178475,1042.47118698)
\curveto(50.24177901,1042.6511847)(50.31677894,1042.84618451)(50.35678475,1043.05618698)
\curveto(50.40677885,1043.27618408)(50.43677882,1043.51618384)(50.44678475,1043.77618698)
\curveto(50.4567788,1044.04618331)(50.46177879,1044.32618303)(50.46178475,1044.61618698)
\lineto(50.46178475,1046.43118698)
\lineto(50.46178475,1047.40618698)
\lineto(50.46178475,1047.67618698)
\curveto(50.46177879,1047.77617958)(50.48177877,1047.8561795)(50.52178475,1047.91618698)
\curveto(50.57177868,1048.00617935)(50.64677861,1048.0561793)(50.74678475,1048.06618698)
\curveto(50.84677841,1048.08617927)(50.96677829,1048.09617926)(51.10678475,1048.09618698)
\lineto(51.90178475,1048.09618698)
\lineto(52.18678475,1048.09618698)
\curveto(52.27677698,1048.09617926)(52.3517769,1048.07617928)(52.41178475,1048.03618698)
\curveto(52.49177676,1047.98617937)(52.53677672,1047.91117944)(52.54678475,1047.81118698)
\curveto(52.5567767,1047.71117964)(52.56177669,1047.59617976)(52.56178475,1047.46618698)
\lineto(52.56178475,1046.32618698)
\lineto(52.56178475,1042.11118698)
\lineto(52.56178475,1041.04618698)
\lineto(52.56178475,1040.74618698)
\curveto(52.56177669,1040.64618671)(52.54177671,1040.57118678)(52.50178475,1040.52118698)
\curveto(52.4517768,1040.44118691)(52.37677688,1040.39618696)(52.27678475,1040.38618698)
\curveto(52.17677708,1040.37618698)(52.07177718,1040.37118698)(51.96178475,1040.37118698)
\lineto(51.15178475,1040.37118698)
\curveto(51.04177821,1040.37118698)(50.94177831,1040.37618698)(50.85178475,1040.38618698)
\curveto(50.77177848,1040.39618696)(50.70677855,1040.43618692)(50.65678475,1040.50618698)
\curveto(50.63677862,1040.53618682)(50.61677864,1040.58118677)(50.59678475,1040.64118698)
\curveto(50.58677867,1040.70118665)(50.57177868,1040.76118659)(50.55178475,1040.82118698)
\curveto(50.54177871,1040.88118647)(50.52677873,1040.93618642)(50.50678475,1040.98618698)
\curveto(50.48677877,1041.03618632)(50.4567788,1041.06618629)(50.41678475,1041.07618698)
\curveto(50.39677886,1041.09618626)(50.37177888,1041.10118625)(50.34178475,1041.09118698)
\curveto(50.31177894,1041.08118627)(50.28677897,1041.07118628)(50.26678475,1041.06118698)
\curveto(50.19677906,1041.02118633)(50.13677912,1040.97618638)(50.08678475,1040.92618698)
\curveto(50.03677922,1040.87618648)(49.98177927,1040.83118652)(49.92178475,1040.79118698)
\curveto(49.88177937,1040.76118659)(49.84177941,1040.72618663)(49.80178475,1040.68618698)
\curveto(49.77177948,1040.6561867)(49.73177952,1040.62618673)(49.68178475,1040.59618698)
\curveto(49.4517798,1040.4561869)(49.18178007,1040.34618701)(48.87178475,1040.26618698)
\curveto(48.80178045,1040.24618711)(48.73178052,1040.23618712)(48.66178475,1040.23618698)
\curveto(48.59178066,1040.22618713)(48.51678074,1040.21118714)(48.43678475,1040.19118698)
\curveto(48.39678086,1040.18118717)(48.3517809,1040.18118717)(48.30178475,1040.19118698)
\curveto(48.26178099,1040.19118716)(48.22178103,1040.18618717)(48.18178475,1040.17618698)
\curveto(48.1517811,1040.16618719)(48.08678117,1040.16618719)(47.98678475,1040.17618698)
\curveto(47.89678136,1040.17618718)(47.83678142,1040.18118717)(47.80678475,1040.19118698)
\curveto(47.7567815,1040.19118716)(47.70678155,1040.19618716)(47.65678475,1040.20618698)
\lineto(47.50678475,1040.20618698)
\curveto(47.38678187,1040.23618712)(47.27178198,1040.26118709)(47.16178475,1040.28118698)
\curveto(47.0517822,1040.30118705)(46.94178231,1040.33118702)(46.83178475,1040.37118698)
\curveto(46.78178247,1040.39118696)(46.73678252,1040.40618695)(46.69678475,1040.41618698)
\curveto(46.66678259,1040.43618692)(46.62678263,1040.4561869)(46.57678475,1040.47618698)
\curveto(46.22678303,1040.66618669)(45.94678331,1040.93118642)(45.73678475,1041.27118698)
\curveto(45.60678365,1041.48118587)(45.51178374,1041.73118562)(45.45178475,1042.02118698)
\curveto(45.39178386,1042.32118503)(45.3517839,1042.63618472)(45.33178475,1042.96618698)
\curveto(45.32178393,1043.30618405)(45.31678394,1043.6511837)(45.31678475,1044.00118698)
\curveto(45.32678393,1044.36118299)(45.33178392,1044.71618264)(45.33178475,1045.06618698)
\lineto(45.33178475,1047.10618698)
\curveto(45.33178392,1047.23618012)(45.32678393,1047.38617997)(45.31678475,1047.55618698)
\curveto(45.31678394,1047.73617962)(45.34178391,1047.86617949)(45.39178475,1047.94618698)
\curveto(45.42178383,1047.99617936)(45.48178377,1048.04117931)(45.57178475,1048.08118698)
\curveto(45.63178362,1048.08117927)(45.67678358,1048.08617927)(45.70678475,1048.09618698)
}
}
{
\newrgbcolor{curcolor}{0 0 0}
\pscustom[linestyle=none,fillstyle=solid,fillcolor=curcolor]
{
\newpath
\moveto(61.24303475,1040.97118698)
\curveto(61.2630269,1040.86118649)(61.27302689,1040.7511866)(61.27303475,1040.64118698)
\curveto(61.28302688,1040.53118682)(61.23302693,1040.4561869)(61.12303475,1040.41618698)
\curveto(61.0630271,1040.38618697)(60.99302717,1040.37118698)(60.91303475,1040.37118698)
\lineto(60.67303475,1040.37118698)
\lineto(59.86303475,1040.37118698)
\lineto(59.59303475,1040.37118698)
\curveto(59.51302865,1040.38118697)(59.44802871,1040.40618695)(59.39803475,1040.44618698)
\curveto(59.32802883,1040.48618687)(59.27302889,1040.54118681)(59.23303475,1040.61118698)
\curveto(59.20302896,1040.69118666)(59.158029,1040.7561866)(59.09803475,1040.80618698)
\curveto(59.07802908,1040.82618653)(59.05302911,1040.84118651)(59.02303475,1040.85118698)
\curveto(58.99302917,1040.87118648)(58.95302921,1040.87618648)(58.90303475,1040.86618698)
\curveto(58.85302931,1040.84618651)(58.80302936,1040.82118653)(58.75303475,1040.79118698)
\curveto(58.71302945,1040.76118659)(58.66802949,1040.73618662)(58.61803475,1040.71618698)
\curveto(58.56802959,1040.67618668)(58.51302965,1040.64118671)(58.45303475,1040.61118698)
\lineto(58.27303475,1040.52118698)
\curveto(58.14303002,1040.46118689)(58.00803015,1040.41118694)(57.86803475,1040.37118698)
\curveto(57.72803043,1040.34118701)(57.58303058,1040.30618705)(57.43303475,1040.26618698)
\curveto(57.3630308,1040.24618711)(57.29303087,1040.23618712)(57.22303475,1040.23618698)
\curveto(57.163031,1040.22618713)(57.09803106,1040.21618714)(57.02803475,1040.20618698)
\lineto(56.93803475,1040.20618698)
\curveto(56.90803125,1040.19618716)(56.87803128,1040.19118716)(56.84803475,1040.19118698)
\lineto(56.68303475,1040.19118698)
\curveto(56.58303158,1040.17118718)(56.48303168,1040.17118718)(56.38303475,1040.19118698)
\lineto(56.24803475,1040.19118698)
\curveto(56.17803198,1040.21118714)(56.10803205,1040.22118713)(56.03803475,1040.22118698)
\curveto(55.97803218,1040.21118714)(55.91803224,1040.21618714)(55.85803475,1040.23618698)
\curveto(55.7580324,1040.2561871)(55.6630325,1040.27618708)(55.57303475,1040.29618698)
\curveto(55.48303268,1040.30618705)(55.39803276,1040.33118702)(55.31803475,1040.37118698)
\curveto(55.02803313,1040.48118687)(54.77803338,1040.62118673)(54.56803475,1040.79118698)
\curveto(54.36803379,1040.97118638)(54.20803395,1041.20618615)(54.08803475,1041.49618698)
\curveto(54.0580341,1041.56618579)(54.02803413,1041.64118571)(53.99803475,1041.72118698)
\curveto(53.97803418,1041.80118555)(53.9580342,1041.88618547)(53.93803475,1041.97618698)
\curveto(53.91803424,1042.02618533)(53.90803425,1042.07618528)(53.90803475,1042.12618698)
\curveto(53.91803424,1042.17618518)(53.91803424,1042.22618513)(53.90803475,1042.27618698)
\curveto(53.89803426,1042.30618505)(53.88803427,1042.36618499)(53.87803475,1042.45618698)
\curveto(53.87803428,1042.5561848)(53.88303428,1042.62618473)(53.89303475,1042.66618698)
\curveto(53.91303425,1042.76618459)(53.92303424,1042.8511845)(53.92303475,1042.92118698)
\lineto(54.01303475,1043.25118698)
\curveto(54.04303412,1043.37118398)(54.08303408,1043.47618388)(54.13303475,1043.56618698)
\curveto(54.30303386,1043.8561835)(54.49803366,1044.07618328)(54.71803475,1044.22618698)
\curveto(54.93803322,1044.37618298)(55.21803294,1044.50618285)(55.55803475,1044.61618698)
\curveto(55.68803247,1044.66618269)(55.82303234,1044.70118265)(55.96303475,1044.72118698)
\curveto(56.10303206,1044.74118261)(56.24303192,1044.76618259)(56.38303475,1044.79618698)
\curveto(56.4630317,1044.81618254)(56.54803161,1044.82618253)(56.63803475,1044.82618698)
\curveto(56.72803143,1044.83618252)(56.81803134,1044.8511825)(56.90803475,1044.87118698)
\curveto(56.97803118,1044.89118246)(57.04803111,1044.89618246)(57.11803475,1044.88618698)
\curveto(57.18803097,1044.88618247)(57.2630309,1044.89618246)(57.34303475,1044.91618698)
\curveto(57.41303075,1044.93618242)(57.48303068,1044.94618241)(57.55303475,1044.94618698)
\curveto(57.62303054,1044.94618241)(57.69803046,1044.9561824)(57.77803475,1044.97618698)
\curveto(57.98803017,1045.02618233)(58.17802998,1045.06618229)(58.34803475,1045.09618698)
\curveto(58.52802963,1045.13618222)(58.68802947,1045.22618213)(58.82803475,1045.36618698)
\curveto(58.91802924,1045.4561819)(58.97802918,1045.5561818)(59.00803475,1045.66618698)
\curveto(59.01802914,1045.69618166)(59.01802914,1045.72118163)(59.00803475,1045.74118698)
\curveto(59.00802915,1045.76118159)(59.01302915,1045.78118157)(59.02303475,1045.80118698)
\curveto(59.03302913,1045.82118153)(59.03802912,1045.8511815)(59.03803475,1045.89118698)
\lineto(59.03803475,1045.98118698)
\lineto(59.00803475,1046.10118698)
\curveto(59.00802915,1046.14118121)(59.00302916,1046.17618118)(58.99303475,1046.20618698)
\curveto(58.89302927,1046.50618085)(58.68302948,1046.71118064)(58.36303475,1046.82118698)
\curveto(58.27302989,1046.8511805)(58.16303,1046.87118048)(58.03303475,1046.88118698)
\curveto(57.91303025,1046.90118045)(57.78803037,1046.90618045)(57.65803475,1046.89618698)
\curveto(57.52803063,1046.89618046)(57.40303076,1046.88618047)(57.28303475,1046.86618698)
\curveto(57.163031,1046.84618051)(57.0580311,1046.82118053)(56.96803475,1046.79118698)
\curveto(56.90803125,1046.77118058)(56.84803131,1046.74118061)(56.78803475,1046.70118698)
\curveto(56.73803142,1046.67118068)(56.68803147,1046.63618072)(56.63803475,1046.59618698)
\curveto(56.58803157,1046.5561808)(56.53303163,1046.50118085)(56.47303475,1046.43118698)
\curveto(56.42303174,1046.36118099)(56.38803177,1046.29618106)(56.36803475,1046.23618698)
\curveto(56.31803184,1046.13618122)(56.27303189,1046.04118131)(56.23303475,1045.95118698)
\curveto(56.20303196,1045.86118149)(56.13303203,1045.80118155)(56.02303475,1045.77118698)
\curveto(55.94303222,1045.7511816)(55.8580323,1045.74118161)(55.76803475,1045.74118698)
\lineto(55.49803475,1045.74118698)
\lineto(54.92803475,1045.74118698)
\curveto(54.87803328,1045.74118161)(54.82803333,1045.73618162)(54.77803475,1045.72618698)
\curveto(54.72803343,1045.72618163)(54.68303348,1045.73118162)(54.64303475,1045.74118698)
\lineto(54.50803475,1045.74118698)
\curveto(54.48803367,1045.7511816)(54.4630337,1045.7561816)(54.43303475,1045.75618698)
\curveto(54.40303376,1045.7561816)(54.37803378,1045.76618159)(54.35803475,1045.78618698)
\curveto(54.27803388,1045.80618155)(54.22303394,1045.87118148)(54.19303475,1045.98118698)
\curveto(54.18303398,1046.03118132)(54.18303398,1046.08118127)(54.19303475,1046.13118698)
\curveto(54.20303396,1046.18118117)(54.21303395,1046.22618113)(54.22303475,1046.26618698)
\curveto(54.25303391,1046.37618098)(54.28303388,1046.47618088)(54.31303475,1046.56618698)
\curveto(54.35303381,1046.66618069)(54.39803376,1046.7561806)(54.44803475,1046.83618698)
\lineto(54.53803475,1046.98618698)
\lineto(54.62803475,1047.13618698)
\curveto(54.70803345,1047.24618011)(54.80803335,1047.35118)(54.92803475,1047.45118698)
\curveto(54.94803321,1047.46117989)(54.97803318,1047.48617987)(55.01803475,1047.52618698)
\curveto(55.06803309,1047.56617979)(55.11303305,1047.60117975)(55.15303475,1047.63118698)
\curveto(55.19303297,1047.66117969)(55.23803292,1047.69117966)(55.28803475,1047.72118698)
\curveto(55.4580327,1047.83117952)(55.63803252,1047.91617944)(55.82803475,1047.97618698)
\curveto(56.01803214,1048.04617931)(56.21303195,1048.11117924)(56.41303475,1048.17118698)
\curveto(56.53303163,1048.20117915)(56.6580315,1048.22117913)(56.78803475,1048.23118698)
\curveto(56.91803124,1048.24117911)(57.04803111,1048.26117909)(57.17803475,1048.29118698)
\curveto(57.21803094,1048.30117905)(57.27803088,1048.30117905)(57.35803475,1048.29118698)
\curveto(57.44803071,1048.28117907)(57.50303066,1048.28617907)(57.52303475,1048.30618698)
\curveto(57.93303023,1048.31617904)(58.32302984,1048.30117905)(58.69303475,1048.26118698)
\curveto(59.07302909,1048.22117913)(59.41302875,1048.14617921)(59.71303475,1048.03618698)
\curveto(60.02302814,1047.92617943)(60.28802787,1047.77617958)(60.50803475,1047.58618698)
\curveto(60.72802743,1047.40617995)(60.89802726,1047.17118018)(61.01803475,1046.88118698)
\curveto(61.08802707,1046.71118064)(61.12802703,1046.51618084)(61.13803475,1046.29618698)
\curveto(61.14802701,1046.07618128)(61.15302701,1045.8511815)(61.15303475,1045.62118698)
\lineto(61.15303475,1042.27618698)
\lineto(61.15303475,1041.69118698)
\curveto(61.15302701,1041.50118585)(61.17302699,1041.32618603)(61.21303475,1041.16618698)
\curveto(61.22302694,1041.13618622)(61.22802693,1041.10118625)(61.22803475,1041.06118698)
\curveto(61.22802693,1041.03118632)(61.23302693,1041.00118635)(61.24303475,1040.97118698)
\moveto(59.03803475,1043.28118698)
\curveto(59.04802911,1043.33118402)(59.05302911,1043.38618397)(59.05303475,1043.44618698)
\curveto(59.05302911,1043.51618384)(59.04802911,1043.57618378)(59.03803475,1043.62618698)
\curveto(59.01802914,1043.68618367)(59.00802915,1043.74118361)(59.00803475,1043.79118698)
\curveto(59.00802915,1043.84118351)(58.98802917,1043.88118347)(58.94803475,1043.91118698)
\curveto(58.89802926,1043.9511834)(58.82302934,1043.97118338)(58.72303475,1043.97118698)
\curveto(58.68302948,1043.96118339)(58.64802951,1043.9511834)(58.61803475,1043.94118698)
\curveto(58.58802957,1043.94118341)(58.55302961,1043.93618342)(58.51303475,1043.92618698)
\curveto(58.44302972,1043.90618345)(58.36802979,1043.89118346)(58.28803475,1043.88118698)
\curveto(58.20802995,1043.87118348)(58.12803003,1043.8561835)(58.04803475,1043.83618698)
\curveto(58.01803014,1043.82618353)(57.97303019,1043.82118353)(57.91303475,1043.82118698)
\curveto(57.78303038,1043.79118356)(57.65303051,1043.77118358)(57.52303475,1043.76118698)
\curveto(57.39303077,1043.7511836)(57.26803089,1043.72618363)(57.14803475,1043.68618698)
\curveto(57.06803109,1043.66618369)(56.99303117,1043.64618371)(56.92303475,1043.62618698)
\curveto(56.85303131,1043.61618374)(56.78303138,1043.59618376)(56.71303475,1043.56618698)
\curveto(56.50303166,1043.47618388)(56.32303184,1043.34118401)(56.17303475,1043.16118698)
\curveto(56.03303213,1042.98118437)(55.98303218,1042.73118462)(56.02303475,1042.41118698)
\curveto(56.04303212,1042.24118511)(56.09803206,1042.10118525)(56.18803475,1041.99118698)
\curveto(56.2580319,1041.88118547)(56.3630318,1041.79118556)(56.50303475,1041.72118698)
\curveto(56.64303152,1041.66118569)(56.79303137,1041.61618574)(56.95303475,1041.58618698)
\curveto(57.12303104,1041.5561858)(57.29803086,1041.54618581)(57.47803475,1041.55618698)
\curveto(57.66803049,1041.57618578)(57.84303032,1041.61118574)(58.00303475,1041.66118698)
\curveto(58.2630299,1041.74118561)(58.46802969,1041.86618549)(58.61803475,1042.03618698)
\curveto(58.76802939,1042.21618514)(58.88302928,1042.43618492)(58.96303475,1042.69618698)
\curveto(58.98302918,1042.76618459)(58.99302917,1042.83618452)(58.99303475,1042.90618698)
\curveto(59.00302916,1042.98618437)(59.01802914,1043.06618429)(59.03803475,1043.14618698)
\lineto(59.03803475,1043.28118698)
}
}
{
\newrgbcolor{curcolor}{0 0 0}
\pscustom[linestyle=none,fillstyle=solid,fillcolor=curcolor]
{
\newpath
\moveto(67.231316,1048.30618698)
\curveto(67.34131068,1048.30617905)(67.43631059,1048.29617906)(67.516316,1048.27618698)
\curveto(67.60631042,1048.2561791)(67.67631035,1048.21117914)(67.726316,1048.14118698)
\curveto(67.78631024,1048.06117929)(67.81631021,1047.92117943)(67.816316,1047.72118698)
\lineto(67.816316,1047.21118698)
\lineto(67.816316,1046.83618698)
\curveto(67.8263102,1046.69618066)(67.81131021,1046.58618077)(67.771316,1046.50618698)
\curveto(67.73131029,1046.43618092)(67.67131035,1046.39118096)(67.591316,1046.37118698)
\curveto(67.5213105,1046.351181)(67.43631059,1046.34118101)(67.336316,1046.34118698)
\curveto(67.24631078,1046.34118101)(67.14631088,1046.34618101)(67.036316,1046.35618698)
\curveto(66.93631109,1046.36618099)(66.84131118,1046.36118099)(66.751316,1046.34118698)
\curveto(66.68131134,1046.32118103)(66.61131141,1046.30618105)(66.541316,1046.29618698)
\curveto(66.47131155,1046.29618106)(66.40631162,1046.28618107)(66.346316,1046.26618698)
\curveto(66.18631184,1046.21618114)(66.026312,1046.14118121)(65.866316,1046.04118698)
\curveto(65.70631232,1045.9511814)(65.58131244,1045.84618151)(65.491316,1045.72618698)
\curveto(65.44131258,1045.64618171)(65.38631264,1045.56118179)(65.326316,1045.47118698)
\curveto(65.27631275,1045.39118196)(65.2263128,1045.30618205)(65.176316,1045.21618698)
\curveto(65.14631288,1045.13618222)(65.11631291,1045.0511823)(65.086316,1044.96118698)
\lineto(65.026316,1044.72118698)
\curveto(65.00631302,1044.6511827)(64.99631303,1044.57618278)(64.996316,1044.49618698)
\curveto(64.99631303,1044.42618293)(64.98631304,1044.356183)(64.966316,1044.28618698)
\curveto(64.95631307,1044.24618311)(64.95131307,1044.20618315)(64.951316,1044.16618698)
\curveto(64.96131306,1044.13618322)(64.96131306,1044.10618325)(64.951316,1044.07618698)
\lineto(64.951316,1043.83618698)
\curveto(64.93131309,1043.76618359)(64.9263131,1043.68618367)(64.936316,1043.59618698)
\curveto(64.94631308,1043.51618384)(64.95131307,1043.43618392)(64.951316,1043.35618698)
\lineto(64.951316,1042.39618698)
\lineto(64.951316,1041.12118698)
\curveto(64.95131307,1040.99118636)(64.94631308,1040.87118648)(64.936316,1040.76118698)
\curveto(64.9263131,1040.6511867)(64.89631313,1040.56118679)(64.846316,1040.49118698)
\curveto(64.8263132,1040.46118689)(64.79131323,1040.43618692)(64.741316,1040.41618698)
\curveto(64.70131332,1040.40618695)(64.65631337,1040.39618696)(64.606316,1040.38618698)
\lineto(64.531316,1040.38618698)
\curveto(64.48131354,1040.37618698)(64.4263136,1040.37118698)(64.366316,1040.37118698)
\lineto(64.201316,1040.37118698)
\lineto(63.556316,1040.37118698)
\curveto(63.49631453,1040.38118697)(63.43131459,1040.38618697)(63.361316,1040.38618698)
\lineto(63.166316,1040.38618698)
\curveto(63.11631491,1040.40618695)(63.06631496,1040.42118693)(63.016316,1040.43118698)
\curveto(62.96631506,1040.4511869)(62.93131509,1040.48618687)(62.911316,1040.53618698)
\curveto(62.87131515,1040.58618677)(62.84631518,1040.6561867)(62.836316,1040.74618698)
\lineto(62.836316,1041.04618698)
\lineto(62.836316,1042.06618698)
\lineto(62.836316,1046.29618698)
\lineto(62.836316,1047.40618698)
\lineto(62.836316,1047.69118698)
\curveto(62.83631519,1047.79117956)(62.85631517,1047.87117948)(62.896316,1047.93118698)
\curveto(62.94631508,1048.01117934)(63.021315,1048.06117929)(63.121316,1048.08118698)
\curveto(63.2213148,1048.10117925)(63.34131468,1048.11117924)(63.481316,1048.11118698)
\lineto(64.246316,1048.11118698)
\curveto(64.36631366,1048.11117924)(64.47131355,1048.10117925)(64.561316,1048.08118698)
\curveto(64.65131337,1048.07117928)(64.7213133,1048.02617933)(64.771316,1047.94618698)
\curveto(64.80131322,1047.89617946)(64.81631321,1047.82617953)(64.816316,1047.73618698)
\lineto(64.846316,1047.46618698)
\curveto(64.85631317,1047.38617997)(64.87131315,1047.31118004)(64.891316,1047.24118698)
\curveto(64.9213131,1047.17118018)(64.97131305,1047.13618022)(65.041316,1047.13618698)
\curveto(65.06131296,1047.1561802)(65.08131294,1047.16618019)(65.101316,1047.16618698)
\curveto(65.1213129,1047.16618019)(65.14131288,1047.17618018)(65.161316,1047.19618698)
\curveto(65.2213128,1047.24618011)(65.27131275,1047.30118005)(65.311316,1047.36118698)
\curveto(65.36131266,1047.43117992)(65.4213126,1047.49117986)(65.491316,1047.54118698)
\curveto(65.53131249,1047.57117978)(65.56631246,1047.60117975)(65.596316,1047.63118698)
\curveto(65.6263124,1047.67117968)(65.66131236,1047.70617965)(65.701316,1047.73618698)
\lineto(65.971316,1047.91618698)
\curveto(66.07131195,1047.97617938)(66.17131185,1048.03117932)(66.271316,1048.08118698)
\curveto(66.37131165,1048.12117923)(66.47131155,1048.1561792)(66.571316,1048.18618698)
\lineto(66.901316,1048.27618698)
\curveto(66.93131109,1048.28617907)(66.98631104,1048.28617907)(67.066316,1048.27618698)
\curveto(67.15631087,1048.27617908)(67.21131081,1048.28617907)(67.231316,1048.30618698)
}
}
{
\newrgbcolor{curcolor}{0 0 0}
\pscustom[linestyle=none,fillstyle=solid,fillcolor=curcolor]
{
\newpath
\moveto(70.73639412,1050.96118698)
\curveto(70.80639117,1050.88117647)(70.84139114,1050.76117659)(70.84139412,1050.60118698)
\lineto(70.84139412,1050.13618698)
\lineto(70.84139412,1049.73118698)
\curveto(70.84139114,1049.59117776)(70.80639117,1049.49617786)(70.73639412,1049.44618698)
\curveto(70.6763913,1049.39617796)(70.59639138,1049.36617799)(70.49639412,1049.35618698)
\curveto(70.40639157,1049.34617801)(70.30639167,1049.34117801)(70.19639412,1049.34118698)
\lineto(69.35639412,1049.34118698)
\curveto(69.24639273,1049.34117801)(69.14639283,1049.34617801)(69.05639412,1049.35618698)
\curveto(68.976393,1049.36617799)(68.90639307,1049.39617796)(68.84639412,1049.44618698)
\curveto(68.80639317,1049.47617788)(68.7763932,1049.53117782)(68.75639412,1049.61118698)
\curveto(68.74639323,1049.70117765)(68.73639324,1049.79617756)(68.72639412,1049.89618698)
\lineto(68.72639412,1050.22618698)
\curveto(68.73639324,1050.33617702)(68.74139324,1050.43117692)(68.74139412,1050.51118698)
\lineto(68.74139412,1050.72118698)
\curveto(68.75139323,1050.79117656)(68.77139321,1050.8511765)(68.80139412,1050.90118698)
\curveto(68.82139316,1050.94117641)(68.84639313,1050.97117638)(68.87639412,1050.99118698)
\lineto(68.99639412,1051.05118698)
\curveto(69.01639296,1051.0511763)(69.04139294,1051.0511763)(69.07139412,1051.05118698)
\curveto(69.10139288,1051.06117629)(69.12639285,1051.06617629)(69.14639412,1051.06618698)
\lineto(70.24139412,1051.06618698)
\curveto(70.34139164,1051.06617629)(70.43639154,1051.06117629)(70.52639412,1051.05118698)
\curveto(70.61639136,1051.04117631)(70.68639129,1051.01117634)(70.73639412,1050.96118698)
\moveto(70.84139412,1041.19618698)
\curveto(70.84139114,1040.99618636)(70.83639114,1040.82618653)(70.82639412,1040.68618698)
\curveto(70.81639116,1040.54618681)(70.72639125,1040.4511869)(70.55639412,1040.40118698)
\curveto(70.49639148,1040.38118697)(70.43139155,1040.37118698)(70.36139412,1040.37118698)
\curveto(70.29139169,1040.38118697)(70.21639176,1040.38618697)(70.13639412,1040.38618698)
\lineto(69.29639412,1040.38618698)
\curveto(69.20639277,1040.38618697)(69.11639286,1040.39118696)(69.02639412,1040.40118698)
\curveto(68.94639303,1040.41118694)(68.88639309,1040.44118691)(68.84639412,1040.49118698)
\curveto(68.78639319,1040.56118679)(68.75139323,1040.64618671)(68.74139412,1040.74618698)
\lineto(68.74139412,1041.09118698)
\lineto(68.74139412,1047.42118698)
\lineto(68.74139412,1047.72118698)
\curveto(68.74139324,1047.82117953)(68.76139322,1047.90117945)(68.80139412,1047.96118698)
\curveto(68.86139312,1048.03117932)(68.94639303,1048.07617928)(69.05639412,1048.09618698)
\curveto(69.0763929,1048.10617925)(69.10139288,1048.10617925)(69.13139412,1048.09618698)
\curveto(69.17139281,1048.09617926)(69.20139278,1048.10117925)(69.22139412,1048.11118698)
\lineto(69.97139412,1048.11118698)
\lineto(70.16639412,1048.11118698)
\curveto(70.24639173,1048.12117923)(70.31139167,1048.12117923)(70.36139412,1048.11118698)
\lineto(70.48139412,1048.11118698)
\curveto(70.54139144,1048.09117926)(70.59639138,1048.07617928)(70.64639412,1048.06618698)
\curveto(70.69639128,1048.0561793)(70.73639124,1048.02617933)(70.76639412,1047.97618698)
\curveto(70.80639117,1047.92617943)(70.82639115,1047.8561795)(70.82639412,1047.76618698)
\curveto(70.83639114,1047.67617968)(70.84139114,1047.58117977)(70.84139412,1047.48118698)
\lineto(70.84139412,1041.19618698)
}
}
{
\newrgbcolor{curcolor}{0 0 0}
\pscustom[linestyle=none,fillstyle=solid,fillcolor=curcolor]
{
\newpath
\moveto(80.27358162,1044.55618698)
\curveto(80.29357305,1044.49618286)(80.30357304,1044.41118294)(80.30358162,1044.30118698)
\curveto(80.30357304,1044.19118316)(80.29357305,1044.10618325)(80.27358162,1044.04618698)
\lineto(80.27358162,1043.89618698)
\curveto(80.25357309,1043.81618354)(80.2435731,1043.73618362)(80.24358162,1043.65618698)
\curveto(80.25357309,1043.57618378)(80.2485731,1043.49618386)(80.22858162,1043.41618698)
\curveto(80.20857314,1043.34618401)(80.19357315,1043.28118407)(80.18358162,1043.22118698)
\curveto(80.17357317,1043.16118419)(80.16357318,1043.09618426)(80.15358162,1043.02618698)
\curveto(80.11357323,1042.91618444)(80.07857327,1042.80118455)(80.04858162,1042.68118698)
\curveto(80.01857333,1042.57118478)(79.97857337,1042.46618489)(79.92858162,1042.36618698)
\curveto(79.71857363,1041.88618547)(79.4435739,1041.49618586)(79.10358162,1041.19618698)
\curveto(78.76357458,1040.89618646)(78.35357499,1040.64618671)(77.87358162,1040.44618698)
\curveto(77.75357559,1040.39618696)(77.62857572,1040.36118699)(77.49858162,1040.34118698)
\curveto(77.37857597,1040.31118704)(77.25357609,1040.28118707)(77.12358162,1040.25118698)
\curveto(77.07357627,1040.23118712)(77.01857633,1040.22118713)(76.95858162,1040.22118698)
\curveto(76.89857645,1040.22118713)(76.8435765,1040.21618714)(76.79358162,1040.20618698)
\lineto(76.68858162,1040.20618698)
\curveto(76.65857669,1040.19618716)(76.62857672,1040.19118716)(76.59858162,1040.19118698)
\curveto(76.5485768,1040.18118717)(76.46857688,1040.17618718)(76.35858162,1040.17618698)
\curveto(76.2485771,1040.16618719)(76.16357718,1040.17118718)(76.10358162,1040.19118698)
\lineto(75.95358162,1040.19118698)
\curveto(75.90357744,1040.20118715)(75.8485775,1040.20618715)(75.78858162,1040.20618698)
\curveto(75.73857761,1040.19618716)(75.68857766,1040.20118715)(75.63858162,1040.22118698)
\curveto(75.59857775,1040.23118712)(75.55857779,1040.23618712)(75.51858162,1040.23618698)
\curveto(75.48857786,1040.23618712)(75.4485779,1040.24118711)(75.39858162,1040.25118698)
\curveto(75.29857805,1040.28118707)(75.19857815,1040.30618705)(75.09858162,1040.32618698)
\curveto(74.99857835,1040.34618701)(74.90357844,1040.37618698)(74.81358162,1040.41618698)
\curveto(74.69357865,1040.4561869)(74.57857877,1040.49618686)(74.46858162,1040.53618698)
\curveto(74.36857898,1040.57618678)(74.26357908,1040.62618673)(74.15358162,1040.68618698)
\curveto(73.80357954,1040.89618646)(73.50357984,1041.14118621)(73.25358162,1041.42118698)
\curveto(73.00358034,1041.70118565)(72.79358055,1042.03618532)(72.62358162,1042.42618698)
\curveto(72.57358077,1042.51618484)(72.53358081,1042.61118474)(72.50358162,1042.71118698)
\curveto(72.48358086,1042.81118454)(72.45858089,1042.91618444)(72.42858162,1043.02618698)
\curveto(72.40858094,1043.07618428)(72.39858095,1043.12118423)(72.39858162,1043.16118698)
\curveto(72.39858095,1043.20118415)(72.38858096,1043.24618411)(72.36858162,1043.29618698)
\curveto(72.348581,1043.37618398)(72.33858101,1043.4561839)(72.33858162,1043.53618698)
\curveto(72.33858101,1043.62618373)(72.32858102,1043.71118364)(72.30858162,1043.79118698)
\curveto(72.29858105,1043.84118351)(72.29358105,1043.88618347)(72.29358162,1043.92618698)
\lineto(72.29358162,1044.06118698)
\curveto(72.27358107,1044.12118323)(72.26358108,1044.20618315)(72.26358162,1044.31618698)
\curveto(72.27358107,1044.42618293)(72.28858106,1044.51118284)(72.30858162,1044.57118698)
\lineto(72.30858162,1044.67618698)
\curveto(72.31858103,1044.72618263)(72.31858103,1044.77618258)(72.30858162,1044.82618698)
\curveto(72.30858104,1044.88618247)(72.31858103,1044.94118241)(72.33858162,1044.99118698)
\curveto(72.348581,1045.04118231)(72.35358099,1045.08618227)(72.35358162,1045.12618698)
\curveto(72.35358099,1045.17618218)(72.36358098,1045.22618213)(72.38358162,1045.27618698)
\curveto(72.42358092,1045.40618195)(72.45858089,1045.53118182)(72.48858162,1045.65118698)
\curveto(72.51858083,1045.78118157)(72.55858079,1045.90618145)(72.60858162,1046.02618698)
\curveto(72.78858056,1046.43618092)(73.00358034,1046.77618058)(73.25358162,1047.04618698)
\curveto(73.50357984,1047.32618003)(73.80857954,1047.58117977)(74.16858162,1047.81118698)
\curveto(74.26857908,1047.86117949)(74.37357897,1047.90617945)(74.48358162,1047.94618698)
\curveto(74.59357875,1047.98617937)(74.70357864,1048.03117932)(74.81358162,1048.08118698)
\curveto(74.9435784,1048.13117922)(75.07857827,1048.16617919)(75.21858162,1048.18618698)
\curveto(75.35857799,1048.20617915)(75.50357784,1048.23617912)(75.65358162,1048.27618698)
\curveto(75.73357761,1048.28617907)(75.80857754,1048.29117906)(75.87858162,1048.29118698)
\curveto(75.9485774,1048.29117906)(76.01857733,1048.29617906)(76.08858162,1048.30618698)
\curveto(76.66857668,1048.31617904)(77.16857618,1048.2561791)(77.58858162,1048.12618698)
\curveto(78.01857533,1047.99617936)(78.39857495,1047.81617954)(78.72858162,1047.58618698)
\curveto(78.83857451,1047.50617985)(78.9485744,1047.41617994)(79.05858162,1047.31618698)
\curveto(79.17857417,1047.22618013)(79.27857407,1047.12618023)(79.35858162,1047.01618698)
\curveto(79.43857391,1046.91618044)(79.50857384,1046.81618054)(79.56858162,1046.71618698)
\curveto(79.63857371,1046.61618074)(79.70857364,1046.51118084)(79.77858162,1046.40118698)
\curveto(79.8485735,1046.29118106)(79.90357344,1046.17118118)(79.94358162,1046.04118698)
\curveto(79.98357336,1045.92118143)(80.02857332,1045.79118156)(80.07858162,1045.65118698)
\curveto(80.10857324,1045.57118178)(80.13357321,1045.48618187)(80.15358162,1045.39618698)
\lineto(80.21358162,1045.12618698)
\curveto(80.22357312,1045.08618227)(80.22857312,1045.04618231)(80.22858162,1045.00618698)
\curveto(80.22857312,1044.96618239)(80.23357311,1044.92618243)(80.24358162,1044.88618698)
\curveto(80.26357308,1044.83618252)(80.26857308,1044.78118257)(80.25858162,1044.72118698)
\curveto(80.2485731,1044.66118269)(80.25357309,1044.60618275)(80.27358162,1044.55618698)
\moveto(78.17358162,1044.01618698)
\curveto(78.18357516,1044.06618329)(78.18857516,1044.13618322)(78.18858162,1044.22618698)
\curveto(78.18857516,1044.32618303)(78.18357516,1044.40118295)(78.17358162,1044.45118698)
\lineto(78.17358162,1044.57118698)
\curveto(78.15357519,1044.62118273)(78.1435752,1044.67618268)(78.14358162,1044.73618698)
\curveto(78.1435752,1044.79618256)(78.13857521,1044.8511825)(78.12858162,1044.90118698)
\curveto(78.12857522,1044.94118241)(78.12357522,1044.97118238)(78.11358162,1044.99118698)
\lineto(78.05358162,1045.23118698)
\curveto(78.0435753,1045.32118203)(78.02357532,1045.40618195)(77.99358162,1045.48618698)
\curveto(77.88357546,1045.74618161)(77.75357559,1045.96618139)(77.60358162,1046.14618698)
\curveto(77.45357589,1046.33618102)(77.25357609,1046.48618087)(77.00358162,1046.59618698)
\curveto(76.9435764,1046.61618074)(76.88357646,1046.63118072)(76.82358162,1046.64118698)
\curveto(76.76357658,1046.66118069)(76.69857665,1046.68118067)(76.62858162,1046.70118698)
\curveto(76.5485768,1046.72118063)(76.46357688,1046.72618063)(76.37358162,1046.71618698)
\lineto(76.10358162,1046.71618698)
\curveto(76.07357727,1046.69618066)(76.03857731,1046.68618067)(75.99858162,1046.68618698)
\curveto(75.95857739,1046.69618066)(75.92357742,1046.69618066)(75.89358162,1046.68618698)
\lineto(75.68358162,1046.62618698)
\curveto(75.62357772,1046.61618074)(75.56857778,1046.59618076)(75.51858162,1046.56618698)
\curveto(75.26857808,1046.4561809)(75.06357828,1046.29618106)(74.90358162,1046.08618698)
\curveto(74.75357859,1045.88618147)(74.63357871,1045.6511817)(74.54358162,1045.38118698)
\curveto(74.51357883,1045.28118207)(74.48857886,1045.17618218)(74.46858162,1045.06618698)
\curveto(74.45857889,1044.9561824)(74.4435789,1044.84618251)(74.42358162,1044.73618698)
\curveto(74.41357893,1044.68618267)(74.40857894,1044.63618272)(74.40858162,1044.58618698)
\lineto(74.40858162,1044.43618698)
\curveto(74.38857896,1044.36618299)(74.37857897,1044.26118309)(74.37858162,1044.12118698)
\curveto(74.38857896,1043.98118337)(74.40357894,1043.87618348)(74.42358162,1043.80618698)
\lineto(74.42358162,1043.67118698)
\curveto(74.4435789,1043.59118376)(74.45857889,1043.51118384)(74.46858162,1043.43118698)
\curveto(74.47857887,1043.36118399)(74.49357885,1043.28618407)(74.51358162,1043.20618698)
\curveto(74.61357873,1042.90618445)(74.71857863,1042.66118469)(74.82858162,1042.47118698)
\curveto(74.9485784,1042.29118506)(75.13357821,1042.12618523)(75.38358162,1041.97618698)
\curveto(75.45357789,1041.92618543)(75.52857782,1041.88618547)(75.60858162,1041.85618698)
\curveto(75.69857765,1041.82618553)(75.78857756,1041.80118555)(75.87858162,1041.78118698)
\curveto(75.91857743,1041.77118558)(75.95357739,1041.76618559)(75.98358162,1041.76618698)
\curveto(76.01357733,1041.77618558)(76.0485773,1041.77618558)(76.08858162,1041.76618698)
\lineto(76.20858162,1041.73618698)
\curveto(76.25857709,1041.73618562)(76.30357704,1041.74118561)(76.34358162,1041.75118698)
\lineto(76.46358162,1041.75118698)
\curveto(76.5435768,1041.77118558)(76.62357672,1041.78618557)(76.70358162,1041.79618698)
\curveto(76.78357656,1041.80618555)(76.85857649,1041.82618553)(76.92858162,1041.85618698)
\curveto(77.18857616,1041.9561854)(77.39857595,1042.09118526)(77.55858162,1042.26118698)
\curveto(77.71857563,1042.43118492)(77.85357549,1042.64118471)(77.96358162,1042.89118698)
\curveto(78.00357534,1042.99118436)(78.03357531,1043.09118426)(78.05358162,1043.19118698)
\curveto(78.07357527,1043.29118406)(78.09857525,1043.39618396)(78.12858162,1043.50618698)
\curveto(78.13857521,1043.54618381)(78.1435752,1043.58118377)(78.14358162,1043.61118698)
\curveto(78.1435752,1043.6511837)(78.1485752,1043.69118366)(78.15858162,1043.73118698)
\lineto(78.15858162,1043.86618698)
\curveto(78.15857519,1043.91618344)(78.16357518,1043.96618339)(78.17358162,1044.01618698)
}
}
{
\newrgbcolor{curcolor}{0 0 0}
\pscustom[linestyle=none,fillstyle=solid,fillcolor=curcolor]
{
\newpath
\moveto(84.6435035,1048.32118698)
\curveto(85.393499,1048.34117901)(86.04349835,1048.2561791)(86.5935035,1048.06618698)
\curveto(87.15349724,1047.88617947)(87.57849681,1047.57117978)(87.8685035,1047.12118698)
\curveto(87.93849645,1047.01118034)(87.99849639,1046.89618046)(88.0485035,1046.77618698)
\curveto(88.10849628,1046.66618069)(88.15849623,1046.54118081)(88.1985035,1046.40118698)
\curveto(88.21849617,1046.34118101)(88.22849616,1046.27618108)(88.2285035,1046.20618698)
\curveto(88.22849616,1046.13618122)(88.21849617,1046.07618128)(88.1985035,1046.02618698)
\curveto(88.15849623,1045.96618139)(88.10349629,1045.92618143)(88.0335035,1045.90618698)
\curveto(87.98349641,1045.88618147)(87.92349647,1045.87618148)(87.8535035,1045.87618698)
\lineto(87.6435035,1045.87618698)
\lineto(86.9835035,1045.87618698)
\curveto(86.91349748,1045.87618148)(86.84349755,1045.87118148)(86.7735035,1045.86118698)
\curveto(86.70349769,1045.86118149)(86.63849775,1045.87118148)(86.5785035,1045.89118698)
\curveto(86.47849791,1045.91118144)(86.40349799,1045.9511814)(86.3535035,1046.01118698)
\curveto(86.30349809,1046.07118128)(86.25849813,1046.13118122)(86.2185035,1046.19118698)
\lineto(86.0985035,1046.40118698)
\curveto(86.06849832,1046.48118087)(86.01849837,1046.54618081)(85.9485035,1046.59618698)
\curveto(85.84849854,1046.67618068)(85.74849864,1046.73618062)(85.6485035,1046.77618698)
\curveto(85.55849883,1046.81618054)(85.44349895,1046.8511805)(85.3035035,1046.88118698)
\curveto(85.23349916,1046.90118045)(85.12849926,1046.91618044)(84.9885035,1046.92618698)
\curveto(84.85849953,1046.93618042)(84.75849963,1046.93118042)(84.6885035,1046.91118698)
\lineto(84.5835035,1046.91118698)
\lineto(84.4335035,1046.88118698)
\curveto(84.3935,1046.88118047)(84.34850004,1046.87618048)(84.2985035,1046.86618698)
\curveto(84.12850026,1046.81618054)(83.9885004,1046.74618061)(83.8785035,1046.65618698)
\curveto(83.77850061,1046.57618078)(83.70850068,1046.4511809)(83.6685035,1046.28118698)
\curveto(83.64850074,1046.21118114)(83.64850074,1046.14618121)(83.6685035,1046.08618698)
\curveto(83.6885007,1046.02618133)(83.70850068,1045.97618138)(83.7285035,1045.93618698)
\curveto(83.79850059,1045.81618154)(83.87850051,1045.72118163)(83.9685035,1045.65118698)
\curveto(84.06850032,1045.58118177)(84.18350021,1045.52118183)(84.3135035,1045.47118698)
\curveto(84.50349989,1045.39118196)(84.70849968,1045.32118203)(84.9285035,1045.26118698)
\lineto(85.6185035,1045.11118698)
\curveto(85.85849853,1045.07118228)(86.0884983,1045.02118233)(86.3085035,1044.96118698)
\curveto(86.53849785,1044.91118244)(86.75349764,1044.84618251)(86.9535035,1044.76618698)
\curveto(87.04349735,1044.72618263)(87.12849726,1044.69118266)(87.2085035,1044.66118698)
\curveto(87.29849709,1044.64118271)(87.38349701,1044.60618275)(87.4635035,1044.55618698)
\curveto(87.65349674,1044.43618292)(87.82349657,1044.30618305)(87.9735035,1044.16618698)
\curveto(88.13349626,1044.02618333)(88.25849613,1043.8511835)(88.3485035,1043.64118698)
\curveto(88.37849601,1043.57118378)(88.40349599,1043.50118385)(88.4235035,1043.43118698)
\curveto(88.44349595,1043.36118399)(88.46349593,1043.28618407)(88.4835035,1043.20618698)
\curveto(88.4934959,1043.14618421)(88.49849589,1043.0511843)(88.4985035,1042.92118698)
\curveto(88.50849588,1042.80118455)(88.50849588,1042.70618465)(88.4985035,1042.63618698)
\lineto(88.4985035,1042.56118698)
\curveto(88.47849591,1042.50118485)(88.46349593,1042.44118491)(88.4535035,1042.38118698)
\curveto(88.45349594,1042.33118502)(88.44849594,1042.28118507)(88.4385035,1042.23118698)
\curveto(88.36849602,1041.93118542)(88.25849613,1041.66618569)(88.1085035,1041.43618698)
\curveto(87.94849644,1041.19618616)(87.75349664,1041.00118635)(87.5235035,1040.85118698)
\curveto(87.2934971,1040.70118665)(87.03349736,1040.57118678)(86.7435035,1040.46118698)
\curveto(86.63349776,1040.41118694)(86.51349788,1040.37618698)(86.3835035,1040.35618698)
\curveto(86.26349813,1040.33618702)(86.14349825,1040.31118704)(86.0235035,1040.28118698)
\curveto(85.93349846,1040.26118709)(85.83849855,1040.2511871)(85.7385035,1040.25118698)
\curveto(85.64849874,1040.24118711)(85.55849883,1040.22618713)(85.4685035,1040.20618698)
\lineto(85.1985035,1040.20618698)
\curveto(85.13849925,1040.18618717)(85.03349936,1040.17618718)(84.8835035,1040.17618698)
\curveto(84.74349965,1040.17618718)(84.64349975,1040.18618717)(84.5835035,1040.20618698)
\curveto(84.55349984,1040.20618715)(84.51849987,1040.21118714)(84.4785035,1040.22118698)
\lineto(84.3735035,1040.22118698)
\curveto(84.25350014,1040.24118711)(84.13350026,1040.2561871)(84.0135035,1040.26618698)
\curveto(83.8935005,1040.27618708)(83.77850061,1040.29618706)(83.6685035,1040.32618698)
\curveto(83.27850111,1040.43618692)(82.93350146,1040.56118679)(82.6335035,1040.70118698)
\curveto(82.33350206,1040.8511865)(82.07850231,1041.07118628)(81.8685035,1041.36118698)
\curveto(81.72850266,1041.5511858)(81.60850278,1041.77118558)(81.5085035,1042.02118698)
\curveto(81.4885029,1042.08118527)(81.46850292,1042.16118519)(81.4485035,1042.26118698)
\curveto(81.42850296,1042.31118504)(81.41350298,1042.38118497)(81.4035035,1042.47118698)
\curveto(81.393503,1042.56118479)(81.39850299,1042.63618472)(81.4185035,1042.69618698)
\curveto(81.44850294,1042.76618459)(81.49850289,1042.81618454)(81.5685035,1042.84618698)
\curveto(81.61850277,1042.86618449)(81.67850271,1042.87618448)(81.7485035,1042.87618698)
\lineto(81.9735035,1042.87618698)
\lineto(82.6785035,1042.87618698)
\lineto(82.9185035,1042.87618698)
\curveto(82.99850139,1042.87618448)(83.06850132,1042.86618449)(83.1285035,1042.84618698)
\curveto(83.23850115,1042.80618455)(83.30850108,1042.74118461)(83.3385035,1042.65118698)
\curveto(83.37850101,1042.56118479)(83.42350097,1042.46618489)(83.4735035,1042.36618698)
\curveto(83.4935009,1042.31618504)(83.52850086,1042.2511851)(83.5785035,1042.17118698)
\curveto(83.63850075,1042.09118526)(83.6885007,1042.04118531)(83.7285035,1042.02118698)
\curveto(83.84850054,1041.92118543)(83.96350043,1041.84118551)(84.0735035,1041.78118698)
\curveto(84.18350021,1041.73118562)(84.32350007,1041.68118567)(84.4935035,1041.63118698)
\curveto(84.54349985,1041.61118574)(84.5934998,1041.60118575)(84.6435035,1041.60118698)
\curveto(84.6934997,1041.61118574)(84.74349965,1041.61118574)(84.7935035,1041.60118698)
\curveto(84.87349952,1041.58118577)(84.95849943,1041.57118578)(85.0485035,1041.57118698)
\curveto(85.14849924,1041.58118577)(85.23349916,1041.59618576)(85.3035035,1041.61618698)
\curveto(85.35349904,1041.62618573)(85.39849899,1041.63118572)(85.4385035,1041.63118698)
\curveto(85.4884989,1041.63118572)(85.53849885,1041.64118571)(85.5885035,1041.66118698)
\curveto(85.72849866,1041.71118564)(85.85349854,1041.77118558)(85.9635035,1041.84118698)
\curveto(86.08349831,1041.91118544)(86.17849821,1042.00118535)(86.2485035,1042.11118698)
\curveto(86.29849809,1042.19118516)(86.33849805,1042.31618504)(86.3685035,1042.48618698)
\curveto(86.388498,1042.5561848)(86.388498,1042.62118473)(86.3685035,1042.68118698)
\curveto(86.34849804,1042.74118461)(86.32849806,1042.79118456)(86.3085035,1042.83118698)
\curveto(86.23849815,1042.97118438)(86.14849824,1043.07618428)(86.0385035,1043.14618698)
\curveto(85.93849845,1043.21618414)(85.81849857,1043.28118407)(85.6785035,1043.34118698)
\curveto(85.4884989,1043.42118393)(85.2884991,1043.48618387)(85.0785035,1043.53618698)
\curveto(84.86849952,1043.58618377)(84.65849973,1043.64118371)(84.4485035,1043.70118698)
\curveto(84.36850002,1043.72118363)(84.28350011,1043.73618362)(84.1935035,1043.74618698)
\curveto(84.11350028,1043.7561836)(84.03350036,1043.77118358)(83.9535035,1043.79118698)
\curveto(83.63350076,1043.88118347)(83.32850106,1043.96618339)(83.0385035,1044.04618698)
\curveto(82.74850164,1044.13618322)(82.48350191,1044.26618309)(82.2435035,1044.43618698)
\curveto(81.96350243,1044.63618272)(81.75850263,1044.90618245)(81.6285035,1045.24618698)
\curveto(81.60850278,1045.31618204)(81.5885028,1045.41118194)(81.5685035,1045.53118698)
\curveto(81.54850284,1045.60118175)(81.53350286,1045.68618167)(81.5235035,1045.78618698)
\curveto(81.51350288,1045.88618147)(81.51850287,1045.97618138)(81.5385035,1046.05618698)
\curveto(81.55850283,1046.10618125)(81.56350283,1046.14618121)(81.5535035,1046.17618698)
\curveto(81.54350285,1046.21618114)(81.54850284,1046.26118109)(81.5685035,1046.31118698)
\curveto(81.5885028,1046.42118093)(81.60850278,1046.52118083)(81.6285035,1046.61118698)
\curveto(81.65850273,1046.71118064)(81.6935027,1046.80618055)(81.7335035,1046.89618698)
\curveto(81.86350253,1047.18618017)(82.04350235,1047.42117993)(82.2735035,1047.60118698)
\curveto(82.50350189,1047.78117957)(82.76350163,1047.92617943)(83.0535035,1048.03618698)
\curveto(83.16350123,1048.08617927)(83.27850111,1048.12117923)(83.3985035,1048.14118698)
\curveto(83.51850087,1048.17117918)(83.64350075,1048.20117915)(83.7735035,1048.23118698)
\curveto(83.83350056,1048.2511791)(83.8935005,1048.26117909)(83.9535035,1048.26118698)
\lineto(84.1335035,1048.29118698)
\curveto(84.21350018,1048.30117905)(84.29850009,1048.30617905)(84.3885035,1048.30618698)
\curveto(84.47849991,1048.30617905)(84.56349983,1048.31117904)(84.6435035,1048.32118698)
}
}
{
\newrgbcolor{curcolor}{0 0 0}
\pscustom[linestyle=none,fillstyle=solid,fillcolor=curcolor]
{
}
}
{
\newrgbcolor{curcolor}{0 0 0}
\pscustom[linestyle=none,fillstyle=solid,fillcolor=curcolor]
{
\newpath
\moveto(98.31530037,1048.30618698)
\curveto(98.42529506,1048.30617905)(98.52029496,1048.29617906)(98.60030037,1048.27618698)
\curveto(98.69029479,1048.2561791)(98.76029472,1048.21117914)(98.81030037,1048.14118698)
\curveto(98.87029461,1048.06117929)(98.90029458,1047.92117943)(98.90030037,1047.72118698)
\lineto(98.90030037,1047.21118698)
\lineto(98.90030037,1046.83618698)
\curveto(98.91029457,1046.69618066)(98.89529459,1046.58618077)(98.85530037,1046.50618698)
\curveto(98.81529467,1046.43618092)(98.75529473,1046.39118096)(98.67530037,1046.37118698)
\curveto(98.60529488,1046.351181)(98.52029496,1046.34118101)(98.42030037,1046.34118698)
\curveto(98.33029515,1046.34118101)(98.23029525,1046.34618101)(98.12030037,1046.35618698)
\curveto(98.02029546,1046.36618099)(97.92529556,1046.36118099)(97.83530037,1046.34118698)
\curveto(97.76529572,1046.32118103)(97.69529579,1046.30618105)(97.62530037,1046.29618698)
\curveto(97.55529593,1046.29618106)(97.49029599,1046.28618107)(97.43030037,1046.26618698)
\curveto(97.27029621,1046.21618114)(97.11029637,1046.14118121)(96.95030037,1046.04118698)
\curveto(96.79029669,1045.9511814)(96.66529682,1045.84618151)(96.57530037,1045.72618698)
\curveto(96.52529696,1045.64618171)(96.47029701,1045.56118179)(96.41030037,1045.47118698)
\curveto(96.36029712,1045.39118196)(96.31029717,1045.30618205)(96.26030037,1045.21618698)
\curveto(96.23029725,1045.13618222)(96.20029728,1045.0511823)(96.17030037,1044.96118698)
\lineto(96.11030037,1044.72118698)
\curveto(96.09029739,1044.6511827)(96.0802974,1044.57618278)(96.08030037,1044.49618698)
\curveto(96.0802974,1044.42618293)(96.07029741,1044.356183)(96.05030037,1044.28618698)
\curveto(96.04029744,1044.24618311)(96.03529745,1044.20618315)(96.03530037,1044.16618698)
\curveto(96.04529744,1044.13618322)(96.04529744,1044.10618325)(96.03530037,1044.07618698)
\lineto(96.03530037,1043.83618698)
\curveto(96.01529747,1043.76618359)(96.01029747,1043.68618367)(96.02030037,1043.59618698)
\curveto(96.03029745,1043.51618384)(96.03529745,1043.43618392)(96.03530037,1043.35618698)
\lineto(96.03530037,1042.39618698)
\lineto(96.03530037,1041.12118698)
\curveto(96.03529745,1040.99118636)(96.03029745,1040.87118648)(96.02030037,1040.76118698)
\curveto(96.01029747,1040.6511867)(95.9802975,1040.56118679)(95.93030037,1040.49118698)
\curveto(95.91029757,1040.46118689)(95.87529761,1040.43618692)(95.82530037,1040.41618698)
\curveto(95.7852977,1040.40618695)(95.74029774,1040.39618696)(95.69030037,1040.38618698)
\lineto(95.61530037,1040.38618698)
\curveto(95.56529792,1040.37618698)(95.51029797,1040.37118698)(95.45030037,1040.37118698)
\lineto(95.28530037,1040.37118698)
\lineto(94.64030037,1040.37118698)
\curveto(94.5802989,1040.38118697)(94.51529897,1040.38618697)(94.44530037,1040.38618698)
\lineto(94.25030037,1040.38618698)
\curveto(94.20029928,1040.40618695)(94.15029933,1040.42118693)(94.10030037,1040.43118698)
\curveto(94.05029943,1040.4511869)(94.01529947,1040.48618687)(93.99530037,1040.53618698)
\curveto(93.95529953,1040.58618677)(93.93029955,1040.6561867)(93.92030037,1040.74618698)
\lineto(93.92030037,1041.04618698)
\lineto(93.92030037,1042.06618698)
\lineto(93.92030037,1046.29618698)
\lineto(93.92030037,1047.40618698)
\lineto(93.92030037,1047.69118698)
\curveto(93.92029956,1047.79117956)(93.94029954,1047.87117948)(93.98030037,1047.93118698)
\curveto(94.03029945,1048.01117934)(94.10529938,1048.06117929)(94.20530037,1048.08118698)
\curveto(94.30529918,1048.10117925)(94.42529906,1048.11117924)(94.56530037,1048.11118698)
\lineto(95.33030037,1048.11118698)
\curveto(95.45029803,1048.11117924)(95.55529793,1048.10117925)(95.64530037,1048.08118698)
\curveto(95.73529775,1048.07117928)(95.80529768,1048.02617933)(95.85530037,1047.94618698)
\curveto(95.8852976,1047.89617946)(95.90029758,1047.82617953)(95.90030037,1047.73618698)
\lineto(95.93030037,1047.46618698)
\curveto(95.94029754,1047.38617997)(95.95529753,1047.31118004)(95.97530037,1047.24118698)
\curveto(96.00529748,1047.17118018)(96.05529743,1047.13618022)(96.12530037,1047.13618698)
\curveto(96.14529734,1047.1561802)(96.16529732,1047.16618019)(96.18530037,1047.16618698)
\curveto(96.20529728,1047.16618019)(96.22529726,1047.17618018)(96.24530037,1047.19618698)
\curveto(96.30529718,1047.24618011)(96.35529713,1047.30118005)(96.39530037,1047.36118698)
\curveto(96.44529704,1047.43117992)(96.50529698,1047.49117986)(96.57530037,1047.54118698)
\curveto(96.61529687,1047.57117978)(96.65029683,1047.60117975)(96.68030037,1047.63118698)
\curveto(96.71029677,1047.67117968)(96.74529674,1047.70617965)(96.78530037,1047.73618698)
\lineto(97.05530037,1047.91618698)
\curveto(97.15529633,1047.97617938)(97.25529623,1048.03117932)(97.35530037,1048.08118698)
\curveto(97.45529603,1048.12117923)(97.55529593,1048.1561792)(97.65530037,1048.18618698)
\lineto(97.98530037,1048.27618698)
\curveto(98.01529547,1048.28617907)(98.07029541,1048.28617907)(98.15030037,1048.27618698)
\curveto(98.24029524,1048.27617908)(98.29529519,1048.28617907)(98.31530037,1048.30618698)
}
}
{
\newrgbcolor{curcolor}{0 0 0}
\pscustom[linestyle=none,fillstyle=solid,fillcolor=curcolor]
{
\newpath
\moveto(106.82170662,1044.31618698)
\curveto(106.84169846,1044.23618312)(106.84169846,1044.14618321)(106.82170662,1044.04618698)
\curveto(106.8016985,1043.94618341)(106.76669853,1043.88118347)(106.71670662,1043.85118698)
\curveto(106.66669863,1043.81118354)(106.59169871,1043.78118357)(106.49170662,1043.76118698)
\curveto(106.4016989,1043.7511836)(106.296699,1043.74118361)(106.17670662,1043.73118698)
\lineto(105.83170662,1043.73118698)
\curveto(105.72169958,1043.74118361)(105.62169968,1043.74618361)(105.53170662,1043.74618698)
\lineto(101.87170662,1043.74618698)
\lineto(101.66170662,1043.74618698)
\curveto(101.6017037,1043.74618361)(101.54670375,1043.73618362)(101.49670662,1043.71618698)
\curveto(101.41670388,1043.67618368)(101.36670393,1043.63618372)(101.34670662,1043.59618698)
\curveto(101.32670397,1043.57618378)(101.30670399,1043.53618382)(101.28670662,1043.47618698)
\curveto(101.26670403,1043.42618393)(101.26170404,1043.37618398)(101.27170662,1043.32618698)
\curveto(101.29170401,1043.26618409)(101.301704,1043.20618415)(101.30170662,1043.14618698)
\curveto(101.31170399,1043.09618426)(101.32670397,1043.04118431)(101.34670662,1042.98118698)
\curveto(101.42670387,1042.74118461)(101.52170378,1042.54118481)(101.63170662,1042.38118698)
\curveto(101.75170355,1042.23118512)(101.91170339,1042.09618526)(102.11170662,1041.97618698)
\curveto(102.19170311,1041.92618543)(102.27170303,1041.89118546)(102.35170662,1041.87118698)
\curveto(102.44170286,1041.86118549)(102.53170277,1041.84118551)(102.62170662,1041.81118698)
\curveto(102.7017026,1041.79118556)(102.81170249,1041.77618558)(102.95170662,1041.76618698)
\curveto(103.09170221,1041.7561856)(103.21170209,1041.76118559)(103.31170662,1041.78118698)
\lineto(103.44670662,1041.78118698)
\curveto(103.54670175,1041.80118555)(103.63670166,1041.82118553)(103.71670662,1041.84118698)
\curveto(103.80670149,1041.87118548)(103.89170141,1041.90118545)(103.97170662,1041.93118698)
\curveto(104.07170123,1041.98118537)(104.18170112,1042.04618531)(104.30170662,1042.12618698)
\curveto(104.43170087,1042.20618515)(104.52670077,1042.28618507)(104.58670662,1042.36618698)
\curveto(104.63670066,1042.43618492)(104.68670061,1042.50118485)(104.73670662,1042.56118698)
\curveto(104.7967005,1042.63118472)(104.86670043,1042.68118467)(104.94670662,1042.71118698)
\curveto(105.04670025,1042.76118459)(105.17170013,1042.78118457)(105.32170662,1042.77118698)
\lineto(105.75670662,1042.77118698)
\lineto(105.93670662,1042.77118698)
\curveto(106.00669929,1042.78118457)(106.06669923,1042.77618458)(106.11670662,1042.75618698)
\lineto(106.26670662,1042.75618698)
\curveto(106.36669893,1042.73618462)(106.43669886,1042.71118464)(106.47670662,1042.68118698)
\curveto(106.51669878,1042.66118469)(106.53669876,1042.61618474)(106.53670662,1042.54618698)
\curveto(106.54669875,1042.47618488)(106.54169876,1042.41618494)(106.52170662,1042.36618698)
\curveto(106.47169883,1042.22618513)(106.41669888,1042.10118525)(106.35670662,1041.99118698)
\curveto(106.296699,1041.88118547)(106.22669907,1041.77118558)(106.14670662,1041.66118698)
\curveto(105.92669937,1041.33118602)(105.67669962,1041.06618629)(105.39670662,1040.86618698)
\curveto(105.11670018,1040.66618669)(104.76670053,1040.49618686)(104.34670662,1040.35618698)
\curveto(104.23670106,1040.31618704)(104.12670117,1040.29118706)(104.01670662,1040.28118698)
\curveto(103.90670139,1040.27118708)(103.79170151,1040.2511871)(103.67170662,1040.22118698)
\curveto(103.63170167,1040.21118714)(103.58670171,1040.21118714)(103.53670662,1040.22118698)
\curveto(103.4967018,1040.22118713)(103.45670184,1040.21618714)(103.41670662,1040.20618698)
\lineto(103.25170662,1040.20618698)
\curveto(103.2017021,1040.18618717)(103.14170216,1040.18118717)(103.07170662,1040.19118698)
\curveto(103.01170229,1040.19118716)(102.95670234,1040.19618716)(102.90670662,1040.20618698)
\curveto(102.82670247,1040.21618714)(102.75670254,1040.21618714)(102.69670662,1040.20618698)
\curveto(102.63670266,1040.19618716)(102.57170273,1040.20118715)(102.50170662,1040.22118698)
\curveto(102.45170285,1040.24118711)(102.3967029,1040.2511871)(102.33670662,1040.25118698)
\curveto(102.27670302,1040.2511871)(102.22170308,1040.26118709)(102.17170662,1040.28118698)
\curveto(102.06170324,1040.30118705)(101.95170335,1040.32618703)(101.84170662,1040.35618698)
\curveto(101.73170357,1040.37618698)(101.63170367,1040.41118694)(101.54170662,1040.46118698)
\curveto(101.43170387,1040.50118685)(101.32670397,1040.53618682)(101.22670662,1040.56618698)
\curveto(101.13670416,1040.60618675)(101.05170425,1040.6511867)(100.97170662,1040.70118698)
\curveto(100.65170465,1040.90118645)(100.36670493,1041.13118622)(100.11670662,1041.39118698)
\curveto(99.86670543,1041.66118569)(99.66170564,1041.97118538)(99.50170662,1042.32118698)
\curveto(99.45170585,1042.43118492)(99.41170589,1042.54118481)(99.38170662,1042.65118698)
\curveto(99.35170595,1042.77118458)(99.31170599,1042.89118446)(99.26170662,1043.01118698)
\curveto(99.25170605,1043.0511843)(99.24670605,1043.08618427)(99.24670662,1043.11618698)
\curveto(99.24670605,1043.1561842)(99.24170606,1043.19618416)(99.23170662,1043.23618698)
\curveto(99.19170611,1043.356184)(99.16670613,1043.48618387)(99.15670662,1043.62618698)
\lineto(99.12670662,1044.04618698)
\curveto(99.12670617,1044.09618326)(99.12170618,1044.1511832)(99.11170662,1044.21118698)
\curveto(99.11170619,1044.27118308)(99.11670618,1044.32618303)(99.12670662,1044.37618698)
\lineto(99.12670662,1044.55618698)
\lineto(99.17170662,1044.91618698)
\curveto(99.21170609,1045.08618227)(99.24670605,1045.2511821)(99.27670662,1045.41118698)
\curveto(99.30670599,1045.57118178)(99.35170595,1045.72118163)(99.41170662,1045.86118698)
\curveto(99.84170546,1046.90118045)(100.57170473,1047.63617972)(101.60170662,1048.06618698)
\curveto(101.74170356,1048.12617923)(101.88170342,1048.16617919)(102.02170662,1048.18618698)
\curveto(102.17170313,1048.21617914)(102.32670297,1048.2511791)(102.48670662,1048.29118698)
\curveto(102.56670273,1048.30117905)(102.64170266,1048.30617905)(102.71170662,1048.30618698)
\curveto(102.78170252,1048.30617905)(102.85670244,1048.31117904)(102.93670662,1048.32118698)
\curveto(103.44670185,1048.33117902)(103.88170142,1048.27117908)(104.24170662,1048.14118698)
\curveto(104.61170069,1048.02117933)(104.94170036,1047.86117949)(105.23170662,1047.66118698)
\curveto(105.32169998,1047.60117975)(105.41169989,1047.53117982)(105.50170662,1047.45118698)
\curveto(105.59169971,1047.38117997)(105.67169963,1047.30618005)(105.74170662,1047.22618698)
\curveto(105.77169953,1047.17618018)(105.81169949,1047.13618022)(105.86170662,1047.10618698)
\curveto(105.94169936,1046.99618036)(106.01669928,1046.88118047)(106.08670662,1046.76118698)
\curveto(106.15669914,1046.6511807)(106.23169907,1046.53618082)(106.31170662,1046.41618698)
\curveto(106.36169894,1046.32618103)(106.4016989,1046.23118112)(106.43170662,1046.13118698)
\curveto(106.47169883,1046.04118131)(106.51169879,1045.94118141)(106.55170662,1045.83118698)
\curveto(106.6016987,1045.70118165)(106.64169866,1045.56618179)(106.67170662,1045.42618698)
\curveto(106.7016986,1045.28618207)(106.73669856,1045.14618221)(106.77670662,1045.00618698)
\curveto(106.7966985,1044.92618243)(106.8016985,1044.83618252)(106.79170662,1044.73618698)
\curveto(106.79169851,1044.64618271)(106.8016985,1044.56118279)(106.82170662,1044.48118698)
\lineto(106.82170662,1044.31618698)
\moveto(104.57170662,1045.20118698)
\curveto(104.64170066,1045.30118205)(104.64670065,1045.42118193)(104.58670662,1045.56118698)
\curveto(104.53670076,1045.71118164)(104.4967008,1045.82118153)(104.46670662,1045.89118698)
\curveto(104.32670097,1046.16118119)(104.14170116,1046.36618099)(103.91170662,1046.50618698)
\curveto(103.68170162,1046.6561807)(103.36170194,1046.73618062)(102.95170662,1046.74618698)
\curveto(102.92170238,1046.72618063)(102.88670241,1046.72118063)(102.84670662,1046.73118698)
\curveto(102.80670249,1046.74118061)(102.77170253,1046.74118061)(102.74170662,1046.73118698)
\curveto(102.69170261,1046.71118064)(102.63670266,1046.69618066)(102.57670662,1046.68618698)
\curveto(102.51670278,1046.68618067)(102.46170284,1046.67618068)(102.41170662,1046.65618698)
\curveto(101.97170333,1046.51618084)(101.64670365,1046.24118111)(101.43670662,1045.83118698)
\curveto(101.41670388,1045.79118156)(101.39170391,1045.73618162)(101.36170662,1045.66618698)
\curveto(101.34170396,1045.60618175)(101.32670397,1045.54118181)(101.31670662,1045.47118698)
\curveto(101.30670399,1045.41118194)(101.30670399,1045.351182)(101.31670662,1045.29118698)
\curveto(101.33670396,1045.23118212)(101.37170393,1045.18118217)(101.42170662,1045.14118698)
\curveto(101.5017038,1045.09118226)(101.61170369,1045.06618229)(101.75170662,1045.06618698)
\lineto(102.15670662,1045.06618698)
\lineto(103.82170662,1045.06618698)
\lineto(104.25670662,1045.06618698)
\curveto(104.41670088,1045.07618228)(104.52170078,1045.12118223)(104.57170662,1045.20118698)
}
}
{
\newrgbcolor{curcolor}{0 0 0}
\pscustom[linestyle=none,fillstyle=solid,fillcolor=curcolor]
{
\newpath
\moveto(115.41998787,1048.02118698)
\curveto(115.48997967,1047.97117938)(115.52497964,1047.88617947)(115.52498787,1047.76618698)
\curveto(115.53497963,1047.6561797)(115.53997962,1047.54117981)(115.53998787,1047.42118698)
\lineto(115.53998787,1041.01618698)
\curveto(115.53997962,1040.93618642)(115.53497963,1040.8561865)(115.52498787,1040.77618698)
\lineto(115.52498787,1040.55118698)
\curveto(115.51497965,1040.47118688)(115.50497966,1040.40118695)(115.49498787,1040.34118698)
\curveto(115.49497967,1040.27118708)(115.48997967,1040.19618716)(115.47998787,1040.11618698)
\curveto(115.43997972,1039.97618738)(115.40497976,1039.84618751)(115.37498787,1039.72618698)
\curveto(115.35497981,1039.59618776)(115.31997984,1039.47618788)(115.26998787,1039.36618698)
\curveto(115.09998006,1038.98618837)(114.87998028,1038.67118868)(114.60998787,1038.42118698)
\curveto(114.34998081,1038.17118918)(114.02998113,1037.96618939)(113.64998787,1037.80618698)
\curveto(113.53998162,1037.7561896)(113.42998173,1037.71618964)(113.31998787,1037.68618698)
\curveto(113.20998195,1037.6561897)(113.09498207,1037.62618973)(112.97498787,1037.59618698)
\curveto(112.8649823,1037.56618979)(112.75498241,1037.54618981)(112.64498787,1037.53618698)
\curveto(112.53498263,1037.52618983)(112.42498274,1037.51118984)(112.31498787,1037.49118698)
\lineto(112.19498787,1037.49118698)
\curveto(112.15498301,1037.48118987)(112.10998305,1037.47618988)(112.05998787,1037.47618698)
\curveto(112.01998314,1037.46618989)(111.97498319,1037.46618989)(111.92498787,1037.47618698)
\curveto(111.87498329,1037.47618988)(111.82498334,1037.47118988)(111.77498787,1037.46118698)
\curveto(111.72498344,1037.4511899)(111.6599835,1037.44618991)(111.57998787,1037.44618698)
\curveto(111.49998366,1037.44618991)(111.43498373,1037.4511899)(111.38498787,1037.46118698)
\lineto(111.24998787,1037.46118698)
\curveto(111.20998395,1037.46118989)(111.16998399,1037.46618989)(111.12998787,1037.47618698)
\curveto(111.04998411,1037.49618986)(110.9649842,1037.50618985)(110.87498787,1037.50618698)
\curveto(110.79498437,1037.50618985)(110.71998444,1037.51618984)(110.64998787,1037.53618698)
\curveto(110.62998453,1037.54618981)(110.60498456,1037.5511898)(110.57498787,1037.55118698)
\curveto(110.54498462,1037.5511898)(110.51998464,1037.5561898)(110.49998787,1037.56618698)
\curveto(110.39998476,1037.58618977)(110.29998486,1037.61118974)(110.19998787,1037.64118698)
\curveto(110.10998505,1037.66118969)(110.01998514,1037.69118966)(109.92998787,1037.73118698)
\curveto(109.54998561,1037.89118946)(109.20998595,1038.09618926)(108.90998787,1038.34618698)
\curveto(108.60998655,1038.58618877)(108.38998677,1038.91118844)(108.24998787,1039.32118698)
\curveto(108.22998693,1039.351188)(108.21998694,1039.38118797)(108.21998787,1039.41118698)
\curveto(108.21998694,1039.44118791)(108.21498695,1039.46618789)(108.20498787,1039.48618698)
\curveto(108.17498699,1039.61618774)(108.18498698,1039.71618764)(108.23498787,1039.78618698)
\curveto(108.29498687,1039.84618751)(108.37498679,1039.88618747)(108.47498787,1039.90618698)
\curveto(108.57498659,1039.92618743)(108.68498648,1039.93618742)(108.80498787,1039.93618698)
\curveto(108.93498623,1039.92618743)(109.05498611,1039.92118743)(109.16498787,1039.92118698)
\lineto(109.67498787,1039.92118698)
\lineto(109.79498787,1039.92118698)
\curveto(109.83498533,1039.91118744)(109.87998528,1039.90618745)(109.92998787,1039.90618698)
\curveto(110.08998507,1039.86618749)(110.18998497,1039.81618754)(110.22998787,1039.75618698)
\curveto(110.26998489,1039.68618767)(110.32998483,1039.59618776)(110.40998787,1039.48618698)
\curveto(110.43998472,1039.44618791)(110.48498468,1039.39618796)(110.54498787,1039.33618698)
\curveto(110.55498461,1039.31618804)(110.5649846,1039.30118805)(110.57498787,1039.29118698)
\curveto(110.58498458,1039.28118807)(110.59498457,1039.26618809)(110.60498787,1039.24618698)
\curveto(110.68498448,1039.18618817)(110.76998439,1039.13118822)(110.85998787,1039.08118698)
\curveto(110.94998421,1039.03118832)(111.04998411,1038.98618837)(111.15998787,1038.94618698)
\curveto(111.22998393,1038.92618843)(111.29998386,1038.91618844)(111.36998787,1038.91618698)
\curveto(111.43998372,1038.90618845)(111.51498365,1038.89118846)(111.59498787,1038.87118698)
\lineto(111.75998787,1038.87118698)
\curveto(111.82998333,1038.8511885)(111.91998324,1038.8511885)(112.02998787,1038.87118698)
\curveto(112.13998302,1038.88118847)(112.22498294,1038.89618846)(112.28498787,1038.91618698)
\curveto(112.33498283,1038.93618842)(112.37498279,1038.94618841)(112.40498787,1038.94618698)
\curveto(112.44498272,1038.94618841)(112.48498268,1038.9561884)(112.52498787,1038.97618698)
\curveto(112.73498243,1039.06618829)(112.90998225,1039.18618817)(113.04998787,1039.33618698)
\curveto(113.18998197,1039.48618787)(113.30498186,1039.66118769)(113.39498787,1039.86118698)
\curveto(113.41498175,1039.92118743)(113.42998173,1039.98118737)(113.43998787,1040.04118698)
\curveto(113.44998171,1040.10118725)(113.4649817,1040.16618719)(113.48498787,1040.23618698)
\curveto(113.50498166,1040.32618703)(113.51498165,1040.42118693)(113.51498787,1040.52118698)
\curveto(113.52498164,1040.63118672)(113.52998163,1040.74118661)(113.52998787,1040.85118698)
\lineto(113.52998787,1040.97118698)
\curveto(113.53998162,1041.01118634)(113.53998162,1041.04618631)(113.52998787,1041.07618698)
\curveto(113.50998165,1041.12618623)(113.49998166,1041.17118618)(113.49998787,1041.21118698)
\curveto(113.50998165,1041.2511861)(113.50498166,1041.29118606)(113.48498787,1041.33118698)
\curveto(113.47498169,1041.351186)(113.4599817,1041.36618599)(113.43998787,1041.37618698)
\lineto(113.39498787,1041.42118698)
\curveto(113.30498186,1041.43118592)(113.22998193,1041.41118594)(113.16998787,1041.36118698)
\curveto(113.11998204,1041.31118604)(113.06998209,1041.26618609)(113.01998787,1041.22618698)
\curveto(112.92998223,1041.1561862)(112.83998232,1041.09118626)(112.74998787,1041.03118698)
\curveto(112.6599825,1040.97118638)(112.5599826,1040.91618644)(112.44998787,1040.86618698)
\curveto(112.33998282,1040.81618654)(112.22998293,1040.77618658)(112.11998787,1040.74618698)
\curveto(112.00998315,1040.71618664)(111.89498327,1040.68618667)(111.77498787,1040.65618698)
\lineto(111.59498787,1040.62618698)
\curveto(111.54498362,1040.62618673)(111.49498367,1040.62118673)(111.44498787,1040.61118698)
\curveto(111.39498377,1040.60118675)(111.31498385,1040.59618676)(111.20498787,1040.59618698)
\curveto(111.09498407,1040.59618676)(111.01498415,1040.60118675)(110.96498787,1040.61118698)
\lineto(110.84498787,1040.61118698)
\curveto(110.81498435,1040.62118673)(110.77998438,1040.62618673)(110.73998787,1040.62618698)
\curveto(110.70998445,1040.62618673)(110.67498449,1040.63118672)(110.63498787,1040.64118698)
\curveto(110.49498467,1040.67118668)(110.3599848,1040.69618666)(110.22998787,1040.71618698)
\curveto(110.09998506,1040.74618661)(109.97998518,1040.78618657)(109.86998787,1040.83618698)
\curveto(109.43998572,1041.00618635)(109.08998607,1041.24118611)(108.81998787,1041.54118698)
\curveto(108.5599866,1041.8511855)(108.33998682,1042.22118513)(108.15998787,1042.65118698)
\curveto(108.10998705,1042.76118459)(108.07498709,1042.87618448)(108.05498787,1042.99618698)
\curveto(108.03498713,1043.11618424)(108.00498716,1043.23618412)(107.96498787,1043.35618698)
\curveto(107.9649872,1043.40618395)(107.9599872,1043.44618391)(107.94998787,1043.47618698)
\curveto(107.92998723,1043.5561838)(107.91998724,1043.64118371)(107.91998787,1043.73118698)
\curveto(107.91998724,1043.83118352)(107.90998725,1043.92118343)(107.88998787,1044.00118698)
\curveto(107.87998728,1044.0511833)(107.87498729,1044.09618326)(107.87498787,1044.13618698)
\lineto(107.87498787,1044.28618698)
\curveto(107.8649873,1044.33618302)(107.8599873,1044.39618296)(107.85998787,1044.46618698)
\curveto(107.8599873,1044.54618281)(107.8649873,1044.61118274)(107.87498787,1044.66118698)
\lineto(107.87498787,1044.81118698)
\curveto(107.88498728,1044.8511825)(107.88498728,1044.89118246)(107.87498787,1044.93118698)
\curveto(107.87498729,1044.97118238)(107.88498728,1045.01118234)(107.90498787,1045.05118698)
\curveto(107.92498724,1045.1511822)(107.93998722,1045.24618211)(107.94998787,1045.33618698)
\curveto(107.9599872,1045.43618192)(107.97498719,1045.53618182)(107.99498787,1045.63618698)
\curveto(108.05498711,1045.83618152)(108.11498705,1046.02618133)(108.17498787,1046.20618698)
\curveto(108.24498692,1046.38618097)(108.32998683,1046.5561808)(108.42998787,1046.71618698)
\curveto(108.47998668,1046.81618054)(108.53498663,1046.90618045)(108.59498787,1046.98618698)
\lineto(108.80498787,1047.25618698)
\curveto(108.83498633,1047.30618005)(108.87498629,1047.35618)(108.92498787,1047.40618698)
\curveto(108.98498618,1047.4561799)(109.03998612,1047.50117985)(109.08998787,1047.54118698)
\lineto(109.17998787,1047.63118698)
\curveto(109.22998593,1047.67117968)(109.27998588,1047.70617965)(109.32998787,1047.73618698)
\curveto(109.37998578,1047.77617958)(109.42998573,1047.81117954)(109.47998787,1047.84118698)
\curveto(109.60998555,1047.92117943)(109.74498542,1047.99117936)(109.88498787,1048.05118698)
\curveto(110.02498514,1048.11117924)(110.17998498,1048.16617919)(110.34998787,1048.21618698)
\curveto(110.42998473,1048.24617911)(110.50998465,1048.26117909)(110.58998787,1048.26118698)
\curveto(110.67998448,1048.27117908)(110.7649844,1048.28617907)(110.84498787,1048.30618698)
\curveto(110.88498428,1048.31617904)(110.93998422,1048.31617904)(111.00998787,1048.30618698)
\curveto(111.07998408,1048.29617906)(111.12498404,1048.30117905)(111.14498787,1048.32118698)
\curveto(111.4649837,1048.33117902)(111.74998341,1048.30117905)(111.99998787,1048.23118698)
\curveto(112.2599829,1048.16117919)(112.48998267,1048.06117929)(112.68998787,1047.93118698)
\curveto(112.71998244,1047.91117944)(112.74998241,1047.88617947)(112.77998787,1047.85618698)
\curveto(112.80998235,1047.83617952)(112.84498232,1047.81117954)(112.88498787,1047.78118698)
\curveto(112.94498222,1047.73117962)(112.99998216,1047.68117967)(113.04998787,1047.63118698)
\curveto(113.09998206,1047.58117977)(113.159982,1047.53617982)(113.22998787,1047.49618698)
\curveto(113.24998191,1047.48617987)(113.27498189,1047.47617988)(113.30498787,1047.46618698)
\curveto(113.34498182,1047.4561799)(113.37498179,1047.46117989)(113.39498787,1047.48118698)
\curveto(113.44498172,1047.50117985)(113.47498169,1047.53617982)(113.48498787,1047.58618698)
\curveto(113.49498167,1047.63617972)(113.50998165,1047.68617967)(113.52998787,1047.73618698)
\curveto(113.54998161,1047.78617957)(113.5649816,1047.83617952)(113.57498787,1047.88618698)
\curveto(113.59498157,1047.94617941)(113.62498154,1047.99617936)(113.66498787,1048.03618698)
\curveto(113.72498144,1048.07617928)(113.79498137,1048.09617926)(113.87498787,1048.09618698)
\curveto(113.9649812,1048.10617925)(114.05498111,1048.11117924)(114.14498787,1048.11118698)
\lineto(114.90998787,1048.11118698)
\curveto(115.01998014,1048.11117924)(115.11498005,1048.10617925)(115.19498787,1048.09618698)
\curveto(115.28497988,1048.09617926)(115.3599798,1048.07117928)(115.41998787,1048.02118698)
\moveto(113.36498787,1043.38618698)
\curveto(113.40498176,1043.47618388)(113.43998172,1043.59118376)(113.46998787,1043.73118698)
\curveto(113.49998166,1043.87118348)(113.51998164,1044.01618334)(113.52998787,1044.16618698)
\curveto(113.53998162,1044.32618303)(113.53998162,1044.48118287)(113.52998787,1044.63118698)
\curveto(113.52998163,1044.78118257)(113.51498165,1044.91618244)(113.48498787,1045.03618698)
\curveto(113.4649817,1045.07618228)(113.45498171,1045.10618225)(113.45498787,1045.12618698)
\curveto(113.4649817,1045.1561822)(113.4649817,1045.19118216)(113.45498787,1045.23118698)
\lineto(113.39498787,1045.44118698)
\curveto(113.37498179,1045.51118184)(113.34998181,1045.57618178)(113.31998787,1045.63618698)
\curveto(113.17998198,1045.98618137)(112.97998218,1046.2561811)(112.71998787,1046.44618698)
\curveto(112.4599827,1046.63618072)(112.07998308,1046.73118062)(111.57998787,1046.73118698)
\curveto(111.5599836,1046.71118064)(111.52998363,1046.70118065)(111.48998787,1046.70118698)
\curveto(111.4599837,1046.71118064)(111.42998373,1046.71118064)(111.39998787,1046.70118698)
\curveto(111.32998383,1046.68118067)(111.2649839,1046.66118069)(111.20498787,1046.64118698)
\curveto(111.14498402,1046.63118072)(111.08498408,1046.61618074)(111.02498787,1046.59618698)
\curveto(110.7649844,1046.48618087)(110.5649846,1046.32118103)(110.42498787,1046.10118698)
\curveto(110.28498488,1045.88118147)(110.16998499,1045.63618172)(110.07998787,1045.36618698)
\curveto(110.0599851,1045.31618204)(110.04998511,1045.27618208)(110.04998787,1045.24618698)
\curveto(110.04998511,1045.21618214)(110.04498512,1045.17618218)(110.03498787,1045.12618698)
\curveto(110.00498516,1045.01618234)(109.98498518,1044.8561825)(109.97498787,1044.64618698)
\curveto(109.9649852,1044.43618292)(109.97498519,1044.26618309)(110.00498787,1044.13618698)
\lineto(110.00498787,1043.98618698)
\curveto(110.02498514,1043.90618345)(110.03998512,1043.82618353)(110.04998787,1043.74618698)
\curveto(110.0599851,1043.67618368)(110.07498509,1043.60118375)(110.09498787,1043.52118698)
\curveto(110.18498498,1043.26118409)(110.29498487,1043.03118432)(110.42498787,1042.83118698)
\curveto(110.55498461,1042.64118471)(110.73498443,1042.48618487)(110.96498787,1042.36618698)
\curveto(111.0649841,1042.31618504)(111.20498396,1042.26618509)(111.38498787,1042.21618698)
\curveto(111.45498371,1042.21618514)(111.50998365,1042.21118514)(111.54998787,1042.20118698)
\curveto(111.56998359,1042.20118515)(111.59998356,1042.19618516)(111.63998787,1042.18618698)
\curveto(111.67998348,1042.18618517)(111.70998345,1042.19118516)(111.72998787,1042.20118698)
\lineto(111.87998787,1042.20118698)
\curveto(111.96998319,1042.22118513)(112.05498311,1042.23618512)(112.13498787,1042.24618698)
\curveto(112.21498295,1042.2561851)(112.29498287,1042.28118507)(112.37498787,1042.32118698)
\curveto(112.62498254,1042.42118493)(112.82498234,1042.56118479)(112.97498787,1042.74118698)
\curveto(113.13498203,1042.92118443)(113.2649819,1043.13618422)(113.36498787,1043.38618698)
}
}
{
\newrgbcolor{curcolor}{0 0 0}
\pscustom[linestyle=none,fillstyle=solid,fillcolor=curcolor]
{
\newpath
\moveto(119.33990975,1050.96118698)
\curveto(119.4099068,1050.88117647)(119.44490676,1050.76117659)(119.44490975,1050.60118698)
\lineto(119.44490975,1050.13618698)
\lineto(119.44490975,1049.73118698)
\curveto(119.44490676,1049.59117776)(119.4099068,1049.49617786)(119.33990975,1049.44618698)
\curveto(119.27990693,1049.39617796)(119.19990701,1049.36617799)(119.09990975,1049.35618698)
\curveto(119.0099072,1049.34617801)(118.9099073,1049.34117801)(118.79990975,1049.34118698)
\lineto(117.95990975,1049.34118698)
\curveto(117.84990836,1049.34117801)(117.74990846,1049.34617801)(117.65990975,1049.35618698)
\curveto(117.57990863,1049.36617799)(117.5099087,1049.39617796)(117.44990975,1049.44618698)
\curveto(117.4099088,1049.47617788)(117.37990883,1049.53117782)(117.35990975,1049.61118698)
\curveto(117.34990886,1049.70117765)(117.33990887,1049.79617756)(117.32990975,1049.89618698)
\lineto(117.32990975,1050.22618698)
\curveto(117.33990887,1050.33617702)(117.34490886,1050.43117692)(117.34490975,1050.51118698)
\lineto(117.34490975,1050.72118698)
\curveto(117.35490885,1050.79117656)(117.37490883,1050.8511765)(117.40490975,1050.90118698)
\curveto(117.42490878,1050.94117641)(117.44990876,1050.97117638)(117.47990975,1050.99118698)
\lineto(117.59990975,1051.05118698)
\curveto(117.61990859,1051.0511763)(117.64490856,1051.0511763)(117.67490975,1051.05118698)
\curveto(117.7049085,1051.06117629)(117.72990848,1051.06617629)(117.74990975,1051.06618698)
\lineto(118.84490975,1051.06618698)
\curveto(118.94490726,1051.06617629)(119.03990717,1051.06117629)(119.12990975,1051.05118698)
\curveto(119.21990699,1051.04117631)(119.28990692,1051.01117634)(119.33990975,1050.96118698)
\moveto(119.44490975,1041.19618698)
\curveto(119.44490676,1040.99618636)(119.43990677,1040.82618653)(119.42990975,1040.68618698)
\curveto(119.41990679,1040.54618681)(119.32990688,1040.4511869)(119.15990975,1040.40118698)
\curveto(119.09990711,1040.38118697)(119.03490717,1040.37118698)(118.96490975,1040.37118698)
\curveto(118.89490731,1040.38118697)(118.81990739,1040.38618697)(118.73990975,1040.38618698)
\lineto(117.89990975,1040.38618698)
\curveto(117.8099084,1040.38618697)(117.71990849,1040.39118696)(117.62990975,1040.40118698)
\curveto(117.54990866,1040.41118694)(117.48990872,1040.44118691)(117.44990975,1040.49118698)
\curveto(117.38990882,1040.56118679)(117.35490885,1040.64618671)(117.34490975,1040.74618698)
\lineto(117.34490975,1041.09118698)
\lineto(117.34490975,1047.42118698)
\lineto(117.34490975,1047.72118698)
\curveto(117.34490886,1047.82117953)(117.36490884,1047.90117945)(117.40490975,1047.96118698)
\curveto(117.46490874,1048.03117932)(117.54990866,1048.07617928)(117.65990975,1048.09618698)
\curveto(117.67990853,1048.10617925)(117.7049085,1048.10617925)(117.73490975,1048.09618698)
\curveto(117.77490843,1048.09617926)(117.8049084,1048.10117925)(117.82490975,1048.11118698)
\lineto(118.57490975,1048.11118698)
\lineto(118.76990975,1048.11118698)
\curveto(118.84990736,1048.12117923)(118.91490729,1048.12117923)(118.96490975,1048.11118698)
\lineto(119.08490975,1048.11118698)
\curveto(119.14490706,1048.09117926)(119.19990701,1048.07617928)(119.24990975,1048.06618698)
\curveto(119.29990691,1048.0561793)(119.33990687,1048.02617933)(119.36990975,1047.97618698)
\curveto(119.4099068,1047.92617943)(119.42990678,1047.8561795)(119.42990975,1047.76618698)
\curveto(119.43990677,1047.67617968)(119.44490676,1047.58117977)(119.44490975,1047.48118698)
\lineto(119.44490975,1041.19618698)
}
}
{
\newrgbcolor{curcolor}{0 0 0}
\pscustom[linestyle=none,fillstyle=solid,fillcolor=curcolor]
{
\newpath
\moveto(124.07709725,1048.32118698)
\curveto(124.82709275,1048.34117901)(125.4770921,1048.2561791)(126.02709725,1048.06618698)
\curveto(126.58709099,1047.88617947)(127.01209056,1047.57117978)(127.30209725,1047.12118698)
\curveto(127.3720902,1047.01118034)(127.43209014,1046.89618046)(127.48209725,1046.77618698)
\curveto(127.54209003,1046.66618069)(127.59208998,1046.54118081)(127.63209725,1046.40118698)
\curveto(127.65208992,1046.34118101)(127.66208991,1046.27618108)(127.66209725,1046.20618698)
\curveto(127.66208991,1046.13618122)(127.65208992,1046.07618128)(127.63209725,1046.02618698)
\curveto(127.59208998,1045.96618139)(127.53709004,1045.92618143)(127.46709725,1045.90618698)
\curveto(127.41709016,1045.88618147)(127.35709022,1045.87618148)(127.28709725,1045.87618698)
\lineto(127.07709725,1045.87618698)
\lineto(126.41709725,1045.87618698)
\curveto(126.34709123,1045.87618148)(126.2770913,1045.87118148)(126.20709725,1045.86118698)
\curveto(126.13709144,1045.86118149)(126.0720915,1045.87118148)(126.01209725,1045.89118698)
\curveto(125.91209166,1045.91118144)(125.83709174,1045.9511814)(125.78709725,1046.01118698)
\curveto(125.73709184,1046.07118128)(125.69209188,1046.13118122)(125.65209725,1046.19118698)
\lineto(125.53209725,1046.40118698)
\curveto(125.50209207,1046.48118087)(125.45209212,1046.54618081)(125.38209725,1046.59618698)
\curveto(125.28209229,1046.67618068)(125.18209239,1046.73618062)(125.08209725,1046.77618698)
\curveto(124.99209258,1046.81618054)(124.8770927,1046.8511805)(124.73709725,1046.88118698)
\curveto(124.66709291,1046.90118045)(124.56209301,1046.91618044)(124.42209725,1046.92618698)
\curveto(124.29209328,1046.93618042)(124.19209338,1046.93118042)(124.12209725,1046.91118698)
\lineto(124.01709725,1046.91118698)
\lineto(123.86709725,1046.88118698)
\curveto(123.82709375,1046.88118047)(123.78209379,1046.87618048)(123.73209725,1046.86618698)
\curveto(123.56209401,1046.81618054)(123.42209415,1046.74618061)(123.31209725,1046.65618698)
\curveto(123.21209436,1046.57618078)(123.14209443,1046.4511809)(123.10209725,1046.28118698)
\curveto(123.08209449,1046.21118114)(123.08209449,1046.14618121)(123.10209725,1046.08618698)
\curveto(123.12209445,1046.02618133)(123.14209443,1045.97618138)(123.16209725,1045.93618698)
\curveto(123.23209434,1045.81618154)(123.31209426,1045.72118163)(123.40209725,1045.65118698)
\curveto(123.50209407,1045.58118177)(123.61709396,1045.52118183)(123.74709725,1045.47118698)
\curveto(123.93709364,1045.39118196)(124.14209343,1045.32118203)(124.36209725,1045.26118698)
\lineto(125.05209725,1045.11118698)
\curveto(125.29209228,1045.07118228)(125.52209205,1045.02118233)(125.74209725,1044.96118698)
\curveto(125.9720916,1044.91118244)(126.18709139,1044.84618251)(126.38709725,1044.76618698)
\curveto(126.4770911,1044.72618263)(126.56209101,1044.69118266)(126.64209725,1044.66118698)
\curveto(126.73209084,1044.64118271)(126.81709076,1044.60618275)(126.89709725,1044.55618698)
\curveto(127.08709049,1044.43618292)(127.25709032,1044.30618305)(127.40709725,1044.16618698)
\curveto(127.56709001,1044.02618333)(127.69208988,1043.8511835)(127.78209725,1043.64118698)
\curveto(127.81208976,1043.57118378)(127.83708974,1043.50118385)(127.85709725,1043.43118698)
\curveto(127.8770897,1043.36118399)(127.89708968,1043.28618407)(127.91709725,1043.20618698)
\curveto(127.92708965,1043.14618421)(127.93208964,1043.0511843)(127.93209725,1042.92118698)
\curveto(127.94208963,1042.80118455)(127.94208963,1042.70618465)(127.93209725,1042.63618698)
\lineto(127.93209725,1042.56118698)
\curveto(127.91208966,1042.50118485)(127.89708968,1042.44118491)(127.88709725,1042.38118698)
\curveto(127.88708969,1042.33118502)(127.88208969,1042.28118507)(127.87209725,1042.23118698)
\curveto(127.80208977,1041.93118542)(127.69208988,1041.66618569)(127.54209725,1041.43618698)
\curveto(127.38209019,1041.19618616)(127.18709039,1041.00118635)(126.95709725,1040.85118698)
\curveto(126.72709085,1040.70118665)(126.46709111,1040.57118678)(126.17709725,1040.46118698)
\curveto(126.06709151,1040.41118694)(125.94709163,1040.37618698)(125.81709725,1040.35618698)
\curveto(125.69709188,1040.33618702)(125.577092,1040.31118704)(125.45709725,1040.28118698)
\curveto(125.36709221,1040.26118709)(125.2720923,1040.2511871)(125.17209725,1040.25118698)
\curveto(125.08209249,1040.24118711)(124.99209258,1040.22618713)(124.90209725,1040.20618698)
\lineto(124.63209725,1040.20618698)
\curveto(124.572093,1040.18618717)(124.46709311,1040.17618718)(124.31709725,1040.17618698)
\curveto(124.1770934,1040.17618718)(124.0770935,1040.18618717)(124.01709725,1040.20618698)
\curveto(123.98709359,1040.20618715)(123.95209362,1040.21118714)(123.91209725,1040.22118698)
\lineto(123.80709725,1040.22118698)
\curveto(123.68709389,1040.24118711)(123.56709401,1040.2561871)(123.44709725,1040.26618698)
\curveto(123.32709425,1040.27618708)(123.21209436,1040.29618706)(123.10209725,1040.32618698)
\curveto(122.71209486,1040.43618692)(122.36709521,1040.56118679)(122.06709725,1040.70118698)
\curveto(121.76709581,1040.8511865)(121.51209606,1041.07118628)(121.30209725,1041.36118698)
\curveto(121.16209641,1041.5511858)(121.04209653,1041.77118558)(120.94209725,1042.02118698)
\curveto(120.92209665,1042.08118527)(120.90209667,1042.16118519)(120.88209725,1042.26118698)
\curveto(120.86209671,1042.31118504)(120.84709673,1042.38118497)(120.83709725,1042.47118698)
\curveto(120.82709675,1042.56118479)(120.83209674,1042.63618472)(120.85209725,1042.69618698)
\curveto(120.88209669,1042.76618459)(120.93209664,1042.81618454)(121.00209725,1042.84618698)
\curveto(121.05209652,1042.86618449)(121.11209646,1042.87618448)(121.18209725,1042.87618698)
\lineto(121.40709725,1042.87618698)
\lineto(122.11209725,1042.87618698)
\lineto(122.35209725,1042.87618698)
\curveto(122.43209514,1042.87618448)(122.50209507,1042.86618449)(122.56209725,1042.84618698)
\curveto(122.6720949,1042.80618455)(122.74209483,1042.74118461)(122.77209725,1042.65118698)
\curveto(122.81209476,1042.56118479)(122.85709472,1042.46618489)(122.90709725,1042.36618698)
\curveto(122.92709465,1042.31618504)(122.96209461,1042.2511851)(123.01209725,1042.17118698)
\curveto(123.0720945,1042.09118526)(123.12209445,1042.04118531)(123.16209725,1042.02118698)
\curveto(123.28209429,1041.92118543)(123.39709418,1041.84118551)(123.50709725,1041.78118698)
\curveto(123.61709396,1041.73118562)(123.75709382,1041.68118567)(123.92709725,1041.63118698)
\curveto(123.9770936,1041.61118574)(124.02709355,1041.60118575)(124.07709725,1041.60118698)
\curveto(124.12709345,1041.61118574)(124.1770934,1041.61118574)(124.22709725,1041.60118698)
\curveto(124.30709327,1041.58118577)(124.39209318,1041.57118578)(124.48209725,1041.57118698)
\curveto(124.58209299,1041.58118577)(124.66709291,1041.59618576)(124.73709725,1041.61618698)
\curveto(124.78709279,1041.62618573)(124.83209274,1041.63118572)(124.87209725,1041.63118698)
\curveto(124.92209265,1041.63118572)(124.9720926,1041.64118571)(125.02209725,1041.66118698)
\curveto(125.16209241,1041.71118564)(125.28709229,1041.77118558)(125.39709725,1041.84118698)
\curveto(125.51709206,1041.91118544)(125.61209196,1042.00118535)(125.68209725,1042.11118698)
\curveto(125.73209184,1042.19118516)(125.7720918,1042.31618504)(125.80209725,1042.48618698)
\curveto(125.82209175,1042.5561848)(125.82209175,1042.62118473)(125.80209725,1042.68118698)
\curveto(125.78209179,1042.74118461)(125.76209181,1042.79118456)(125.74209725,1042.83118698)
\curveto(125.6720919,1042.97118438)(125.58209199,1043.07618428)(125.47209725,1043.14618698)
\curveto(125.3720922,1043.21618414)(125.25209232,1043.28118407)(125.11209725,1043.34118698)
\curveto(124.92209265,1043.42118393)(124.72209285,1043.48618387)(124.51209725,1043.53618698)
\curveto(124.30209327,1043.58618377)(124.09209348,1043.64118371)(123.88209725,1043.70118698)
\curveto(123.80209377,1043.72118363)(123.71709386,1043.73618362)(123.62709725,1043.74618698)
\curveto(123.54709403,1043.7561836)(123.46709411,1043.77118358)(123.38709725,1043.79118698)
\curveto(123.06709451,1043.88118347)(122.76209481,1043.96618339)(122.47209725,1044.04618698)
\curveto(122.18209539,1044.13618322)(121.91709566,1044.26618309)(121.67709725,1044.43618698)
\curveto(121.39709618,1044.63618272)(121.19209638,1044.90618245)(121.06209725,1045.24618698)
\curveto(121.04209653,1045.31618204)(121.02209655,1045.41118194)(121.00209725,1045.53118698)
\curveto(120.98209659,1045.60118175)(120.96709661,1045.68618167)(120.95709725,1045.78618698)
\curveto(120.94709663,1045.88618147)(120.95209662,1045.97618138)(120.97209725,1046.05618698)
\curveto(120.99209658,1046.10618125)(120.99709658,1046.14618121)(120.98709725,1046.17618698)
\curveto(120.9770966,1046.21618114)(120.98209659,1046.26118109)(121.00209725,1046.31118698)
\curveto(121.02209655,1046.42118093)(121.04209653,1046.52118083)(121.06209725,1046.61118698)
\curveto(121.09209648,1046.71118064)(121.12709645,1046.80618055)(121.16709725,1046.89618698)
\curveto(121.29709628,1047.18618017)(121.4770961,1047.42117993)(121.70709725,1047.60118698)
\curveto(121.93709564,1047.78117957)(122.19709538,1047.92617943)(122.48709725,1048.03618698)
\curveto(122.59709498,1048.08617927)(122.71209486,1048.12117923)(122.83209725,1048.14118698)
\curveto(122.95209462,1048.17117918)(123.0770945,1048.20117915)(123.20709725,1048.23118698)
\curveto(123.26709431,1048.2511791)(123.32709425,1048.26117909)(123.38709725,1048.26118698)
\lineto(123.56709725,1048.29118698)
\curveto(123.64709393,1048.30117905)(123.73209384,1048.30617905)(123.82209725,1048.30618698)
\curveto(123.91209366,1048.30617905)(123.99709358,1048.31117904)(124.07709725,1048.32118698)
}
}
{
\newrgbcolor{curcolor}{0 0 0}
\pscustom[linestyle=none,fillstyle=solid,fillcolor=curcolor]
{
\newpath
\moveto(130.21373787,1050.42118698)
\lineto(131.21873787,1050.42118698)
\curveto(131.36873489,1050.42117693)(131.49873476,1050.41117694)(131.60873787,1050.39118698)
\curveto(131.72873453,1050.38117697)(131.81373444,1050.32117703)(131.86373787,1050.21118698)
\curveto(131.88373437,1050.16117719)(131.89373436,1050.10117725)(131.89373787,1050.03118698)
\lineto(131.89373787,1049.82118698)
\lineto(131.89373787,1049.14618698)
\curveto(131.89373436,1049.09617826)(131.88873437,1049.03617832)(131.87873787,1048.96618698)
\curveto(131.87873438,1048.90617845)(131.88373437,1048.8511785)(131.89373787,1048.80118698)
\lineto(131.89373787,1048.63618698)
\curveto(131.89373436,1048.5561788)(131.89873436,1048.48117887)(131.90873787,1048.41118698)
\curveto(131.91873434,1048.351179)(131.94373431,1048.29617906)(131.98373787,1048.24618698)
\curveto(132.0537342,1048.1561792)(132.17873408,1048.10617925)(132.35873787,1048.09618698)
\lineto(132.89873787,1048.09618698)
\lineto(133.07873787,1048.09618698)
\curveto(133.13873312,1048.09617926)(133.19373306,1048.08617927)(133.24373787,1048.06618698)
\curveto(133.3537329,1048.01617934)(133.41373284,1047.92617943)(133.42373787,1047.79618698)
\curveto(133.44373281,1047.66617969)(133.4537328,1047.52117983)(133.45373787,1047.36118698)
\lineto(133.45373787,1047.15118698)
\curveto(133.46373279,1047.08118027)(133.4587328,1047.02118033)(133.43873787,1046.97118698)
\curveto(133.38873287,1046.81118054)(133.28373297,1046.72618063)(133.12373787,1046.71618698)
\curveto(132.96373329,1046.70618065)(132.78373347,1046.70118065)(132.58373787,1046.70118698)
\lineto(132.44873787,1046.70118698)
\curveto(132.40873385,1046.71118064)(132.37373388,1046.71118064)(132.34373787,1046.70118698)
\curveto(132.30373395,1046.69118066)(132.26873399,1046.68618067)(132.23873787,1046.68618698)
\curveto(132.20873405,1046.69618066)(132.17873408,1046.69118066)(132.14873787,1046.67118698)
\curveto(132.06873419,1046.6511807)(132.00873425,1046.60618075)(131.96873787,1046.53618698)
\curveto(131.93873432,1046.47618088)(131.91373434,1046.40118095)(131.89373787,1046.31118698)
\curveto(131.88373437,1046.26118109)(131.88373437,1046.20618115)(131.89373787,1046.14618698)
\curveto(131.90373435,1046.08618127)(131.90373435,1046.03118132)(131.89373787,1045.98118698)
\lineto(131.89373787,1045.05118698)
\lineto(131.89373787,1043.29618698)
\curveto(131.89373436,1043.04618431)(131.89873436,1042.82618453)(131.90873787,1042.63618698)
\curveto(131.92873433,1042.4561849)(131.99373426,1042.29618506)(132.10373787,1042.15618698)
\curveto(132.1537341,1042.09618526)(132.21873404,1042.0511853)(132.29873787,1042.02118698)
\lineto(132.56873787,1041.96118698)
\curveto(132.59873366,1041.9511854)(132.62873363,1041.94618541)(132.65873787,1041.94618698)
\curveto(132.69873356,1041.9561854)(132.72873353,1041.9561854)(132.74873787,1041.94618698)
\lineto(132.91373787,1041.94618698)
\curveto(133.02373323,1041.94618541)(133.11873314,1041.94118541)(133.19873787,1041.93118698)
\curveto(133.27873298,1041.92118543)(133.34373291,1041.88118547)(133.39373787,1041.81118698)
\curveto(133.43373282,1041.7511856)(133.4537328,1041.67118568)(133.45373787,1041.57118698)
\lineto(133.45373787,1041.28618698)
\curveto(133.4537328,1041.07618628)(133.44873281,1040.88118647)(133.43873787,1040.70118698)
\curveto(133.43873282,1040.53118682)(133.3587329,1040.41618694)(133.19873787,1040.35618698)
\curveto(133.14873311,1040.33618702)(133.10373315,1040.33118702)(133.06373787,1040.34118698)
\curveto(133.02373323,1040.34118701)(132.97873328,1040.33118702)(132.92873787,1040.31118698)
\lineto(132.77873787,1040.31118698)
\curveto(132.7587335,1040.31118704)(132.72873353,1040.31618704)(132.68873787,1040.32618698)
\curveto(132.64873361,1040.32618703)(132.61373364,1040.32118703)(132.58373787,1040.31118698)
\curveto(132.53373372,1040.30118705)(132.47873378,1040.30118705)(132.41873787,1040.31118698)
\lineto(132.26873787,1040.31118698)
\lineto(132.11873787,1040.31118698)
\curveto(132.06873419,1040.30118705)(132.02373423,1040.30118705)(131.98373787,1040.31118698)
\lineto(131.81873787,1040.31118698)
\curveto(131.76873449,1040.32118703)(131.71373454,1040.32618703)(131.65373787,1040.32618698)
\curveto(131.59373466,1040.32618703)(131.53873472,1040.33118702)(131.48873787,1040.34118698)
\curveto(131.41873484,1040.351187)(131.3537349,1040.36118699)(131.29373787,1040.37118698)
\lineto(131.11373787,1040.40118698)
\curveto(131.00373525,1040.43118692)(130.89873536,1040.46618689)(130.79873787,1040.50618698)
\curveto(130.69873556,1040.54618681)(130.60373565,1040.59118676)(130.51373787,1040.64118698)
\lineto(130.42373787,1040.70118698)
\curveto(130.39373586,1040.73118662)(130.3587359,1040.76118659)(130.31873787,1040.79118698)
\curveto(130.29873596,1040.81118654)(130.27373598,1040.83118652)(130.24373787,1040.85118698)
\lineto(130.16873787,1040.92618698)
\curveto(130.02873623,1041.11618624)(129.92373633,1041.32618603)(129.85373787,1041.55618698)
\curveto(129.83373642,1041.59618576)(129.82373643,1041.63118572)(129.82373787,1041.66118698)
\curveto(129.83373642,1041.70118565)(129.83373642,1041.74618561)(129.82373787,1041.79618698)
\curveto(129.81373644,1041.81618554)(129.80873645,1041.84118551)(129.80873787,1041.87118698)
\curveto(129.80873645,1041.90118545)(129.80373645,1041.92618543)(129.79373787,1041.94618698)
\lineto(129.79373787,1042.09618698)
\curveto(129.78373647,1042.13618522)(129.77873648,1042.18118517)(129.77873787,1042.23118698)
\curveto(129.78873647,1042.28118507)(129.79373646,1042.33118502)(129.79373787,1042.38118698)
\lineto(129.79373787,1042.95118698)
\lineto(129.79373787,1045.18618698)
\lineto(129.79373787,1045.98118698)
\lineto(129.79373787,1046.19118698)
\curveto(129.80373645,1046.26118109)(129.79873646,1046.32618103)(129.77873787,1046.38618698)
\curveto(129.73873652,1046.52618083)(129.66873659,1046.61618074)(129.56873787,1046.65618698)
\curveto(129.4587368,1046.70618065)(129.31873694,1046.72118063)(129.14873787,1046.70118698)
\curveto(128.97873728,1046.68118067)(128.83373742,1046.69618066)(128.71373787,1046.74618698)
\curveto(128.63373762,1046.77618058)(128.58373767,1046.82118053)(128.56373787,1046.88118698)
\curveto(128.54373771,1046.94118041)(128.52373773,1047.01618034)(128.50373787,1047.10618698)
\lineto(128.50373787,1047.42118698)
\curveto(128.50373775,1047.60117975)(128.51373774,1047.74617961)(128.53373787,1047.85618698)
\curveto(128.5537377,1047.96617939)(128.63873762,1048.04117931)(128.78873787,1048.08118698)
\curveto(128.82873743,1048.10117925)(128.86873739,1048.10617925)(128.90873787,1048.09618698)
\lineto(129.04373787,1048.09618698)
\curveto(129.19373706,1048.09617926)(129.33373692,1048.10117925)(129.46373787,1048.11118698)
\curveto(129.59373666,1048.13117922)(129.68373657,1048.19117916)(129.73373787,1048.29118698)
\curveto(129.76373649,1048.36117899)(129.77873648,1048.44117891)(129.77873787,1048.53118698)
\curveto(129.78873647,1048.62117873)(129.79373646,1048.71117864)(129.79373787,1048.80118698)
\lineto(129.79373787,1049.73118698)
\lineto(129.79373787,1049.98618698)
\curveto(129.79373646,1050.07617728)(129.80373645,1050.1511772)(129.82373787,1050.21118698)
\curveto(129.87373638,1050.31117704)(129.94873631,1050.37617698)(130.04873787,1050.40618698)
\curveto(130.06873619,1050.41617694)(130.09373616,1050.41617694)(130.12373787,1050.40618698)
\curveto(130.16373609,1050.40617695)(130.19373606,1050.41117694)(130.21373787,1050.42118698)
}
}
{
\newrgbcolor{curcolor}{0 0 0}
\pscustom[linestyle=none,fillstyle=solid,fillcolor=curcolor]
{
\newpath
\moveto(138.86217537,1048.30618698)
\curveto(138.97217006,1048.30617905)(139.06716996,1048.29617906)(139.14717537,1048.27618698)
\curveto(139.23716979,1048.2561791)(139.30716972,1048.21117914)(139.35717537,1048.14118698)
\curveto(139.41716961,1048.06117929)(139.44716958,1047.92117943)(139.44717537,1047.72118698)
\lineto(139.44717537,1047.21118698)
\lineto(139.44717537,1046.83618698)
\curveto(139.45716957,1046.69618066)(139.44216959,1046.58618077)(139.40217537,1046.50618698)
\curveto(139.36216967,1046.43618092)(139.30216973,1046.39118096)(139.22217537,1046.37118698)
\curveto(139.15216988,1046.351181)(139.06716996,1046.34118101)(138.96717537,1046.34118698)
\curveto(138.87717015,1046.34118101)(138.77717025,1046.34618101)(138.66717537,1046.35618698)
\curveto(138.56717046,1046.36618099)(138.47217056,1046.36118099)(138.38217537,1046.34118698)
\curveto(138.31217072,1046.32118103)(138.24217079,1046.30618105)(138.17217537,1046.29618698)
\curveto(138.10217093,1046.29618106)(138.03717099,1046.28618107)(137.97717537,1046.26618698)
\curveto(137.81717121,1046.21618114)(137.65717137,1046.14118121)(137.49717537,1046.04118698)
\curveto(137.33717169,1045.9511814)(137.21217182,1045.84618151)(137.12217537,1045.72618698)
\curveto(137.07217196,1045.64618171)(137.01717201,1045.56118179)(136.95717537,1045.47118698)
\curveto(136.90717212,1045.39118196)(136.85717217,1045.30618205)(136.80717537,1045.21618698)
\curveto(136.77717225,1045.13618222)(136.74717228,1045.0511823)(136.71717537,1044.96118698)
\lineto(136.65717537,1044.72118698)
\curveto(136.63717239,1044.6511827)(136.6271724,1044.57618278)(136.62717537,1044.49618698)
\curveto(136.6271724,1044.42618293)(136.61717241,1044.356183)(136.59717537,1044.28618698)
\curveto(136.58717244,1044.24618311)(136.58217245,1044.20618315)(136.58217537,1044.16618698)
\curveto(136.59217244,1044.13618322)(136.59217244,1044.10618325)(136.58217537,1044.07618698)
\lineto(136.58217537,1043.83618698)
\curveto(136.56217247,1043.76618359)(136.55717247,1043.68618367)(136.56717537,1043.59618698)
\curveto(136.57717245,1043.51618384)(136.58217245,1043.43618392)(136.58217537,1043.35618698)
\lineto(136.58217537,1042.39618698)
\lineto(136.58217537,1041.12118698)
\curveto(136.58217245,1040.99118636)(136.57717245,1040.87118648)(136.56717537,1040.76118698)
\curveto(136.55717247,1040.6511867)(136.5271725,1040.56118679)(136.47717537,1040.49118698)
\curveto(136.45717257,1040.46118689)(136.42217261,1040.43618692)(136.37217537,1040.41618698)
\curveto(136.3321727,1040.40618695)(136.28717274,1040.39618696)(136.23717537,1040.38618698)
\lineto(136.16217537,1040.38618698)
\curveto(136.11217292,1040.37618698)(136.05717297,1040.37118698)(135.99717537,1040.37118698)
\lineto(135.83217537,1040.37118698)
\lineto(135.18717537,1040.37118698)
\curveto(135.1271739,1040.38118697)(135.06217397,1040.38618697)(134.99217537,1040.38618698)
\lineto(134.79717537,1040.38618698)
\curveto(134.74717428,1040.40618695)(134.69717433,1040.42118693)(134.64717537,1040.43118698)
\curveto(134.59717443,1040.4511869)(134.56217447,1040.48618687)(134.54217537,1040.53618698)
\curveto(134.50217453,1040.58618677)(134.47717455,1040.6561867)(134.46717537,1040.74618698)
\lineto(134.46717537,1041.04618698)
\lineto(134.46717537,1042.06618698)
\lineto(134.46717537,1046.29618698)
\lineto(134.46717537,1047.40618698)
\lineto(134.46717537,1047.69118698)
\curveto(134.46717456,1047.79117956)(134.48717454,1047.87117948)(134.52717537,1047.93118698)
\curveto(134.57717445,1048.01117934)(134.65217438,1048.06117929)(134.75217537,1048.08118698)
\curveto(134.85217418,1048.10117925)(134.97217406,1048.11117924)(135.11217537,1048.11118698)
\lineto(135.87717537,1048.11118698)
\curveto(135.99717303,1048.11117924)(136.10217293,1048.10117925)(136.19217537,1048.08118698)
\curveto(136.28217275,1048.07117928)(136.35217268,1048.02617933)(136.40217537,1047.94618698)
\curveto(136.4321726,1047.89617946)(136.44717258,1047.82617953)(136.44717537,1047.73618698)
\lineto(136.47717537,1047.46618698)
\curveto(136.48717254,1047.38617997)(136.50217253,1047.31118004)(136.52217537,1047.24118698)
\curveto(136.55217248,1047.17118018)(136.60217243,1047.13618022)(136.67217537,1047.13618698)
\curveto(136.69217234,1047.1561802)(136.71217232,1047.16618019)(136.73217537,1047.16618698)
\curveto(136.75217228,1047.16618019)(136.77217226,1047.17618018)(136.79217537,1047.19618698)
\curveto(136.85217218,1047.24618011)(136.90217213,1047.30118005)(136.94217537,1047.36118698)
\curveto(136.99217204,1047.43117992)(137.05217198,1047.49117986)(137.12217537,1047.54118698)
\curveto(137.16217187,1047.57117978)(137.19717183,1047.60117975)(137.22717537,1047.63118698)
\curveto(137.25717177,1047.67117968)(137.29217174,1047.70617965)(137.33217537,1047.73618698)
\lineto(137.60217537,1047.91618698)
\curveto(137.70217133,1047.97617938)(137.80217123,1048.03117932)(137.90217537,1048.08118698)
\curveto(138.00217103,1048.12117923)(138.10217093,1048.1561792)(138.20217537,1048.18618698)
\lineto(138.53217537,1048.27618698)
\curveto(138.56217047,1048.28617907)(138.61717041,1048.28617907)(138.69717537,1048.27618698)
\curveto(138.78717024,1048.27617908)(138.84217019,1048.28617907)(138.86217537,1048.30618698)
}
}
{
\newrgbcolor{curcolor}{0 0 0}
\pscustom[linestyle=none,fillstyle=solid,fillcolor=curcolor]
{
\newpath
\moveto(147.3172535,1040.97118698)
\curveto(147.33724565,1040.86118649)(147.34724564,1040.7511866)(147.3472535,1040.64118698)
\curveto(147.35724563,1040.53118682)(147.30724568,1040.4561869)(147.1972535,1040.41618698)
\curveto(147.13724585,1040.38618697)(147.06724592,1040.37118698)(146.9872535,1040.37118698)
\lineto(146.7472535,1040.37118698)
\lineto(145.9372535,1040.37118698)
\lineto(145.6672535,1040.37118698)
\curveto(145.5872474,1040.38118697)(145.52224746,1040.40618695)(145.4722535,1040.44618698)
\curveto(145.40224758,1040.48618687)(145.34724764,1040.54118681)(145.3072535,1040.61118698)
\curveto(145.27724771,1040.69118666)(145.23224775,1040.7561866)(145.1722535,1040.80618698)
\curveto(145.15224783,1040.82618653)(145.12724786,1040.84118651)(145.0972535,1040.85118698)
\curveto(145.06724792,1040.87118648)(145.02724796,1040.87618648)(144.9772535,1040.86618698)
\curveto(144.92724806,1040.84618651)(144.87724811,1040.82118653)(144.8272535,1040.79118698)
\curveto(144.7872482,1040.76118659)(144.74224824,1040.73618662)(144.6922535,1040.71618698)
\curveto(144.64224834,1040.67618668)(144.5872484,1040.64118671)(144.5272535,1040.61118698)
\lineto(144.3472535,1040.52118698)
\curveto(144.21724877,1040.46118689)(144.0822489,1040.41118694)(143.9422535,1040.37118698)
\curveto(143.80224918,1040.34118701)(143.65724933,1040.30618705)(143.5072535,1040.26618698)
\curveto(143.43724955,1040.24618711)(143.36724962,1040.23618712)(143.2972535,1040.23618698)
\curveto(143.23724975,1040.22618713)(143.17224981,1040.21618714)(143.1022535,1040.20618698)
\lineto(143.0122535,1040.20618698)
\curveto(142.98225,1040.19618716)(142.95225003,1040.19118716)(142.9222535,1040.19118698)
\lineto(142.7572535,1040.19118698)
\curveto(142.65725033,1040.17118718)(142.55725043,1040.17118718)(142.4572535,1040.19118698)
\lineto(142.3222535,1040.19118698)
\curveto(142.25225073,1040.21118714)(142.1822508,1040.22118713)(142.1122535,1040.22118698)
\curveto(142.05225093,1040.21118714)(141.99225099,1040.21618714)(141.9322535,1040.23618698)
\curveto(141.83225115,1040.2561871)(141.73725125,1040.27618708)(141.6472535,1040.29618698)
\curveto(141.55725143,1040.30618705)(141.47225151,1040.33118702)(141.3922535,1040.37118698)
\curveto(141.10225188,1040.48118687)(140.85225213,1040.62118673)(140.6422535,1040.79118698)
\curveto(140.44225254,1040.97118638)(140.2822527,1041.20618615)(140.1622535,1041.49618698)
\curveto(140.13225285,1041.56618579)(140.10225288,1041.64118571)(140.0722535,1041.72118698)
\curveto(140.05225293,1041.80118555)(140.03225295,1041.88618547)(140.0122535,1041.97618698)
\curveto(139.99225299,1042.02618533)(139.982253,1042.07618528)(139.9822535,1042.12618698)
\curveto(139.99225299,1042.17618518)(139.99225299,1042.22618513)(139.9822535,1042.27618698)
\curveto(139.97225301,1042.30618505)(139.96225302,1042.36618499)(139.9522535,1042.45618698)
\curveto(139.95225303,1042.5561848)(139.95725303,1042.62618473)(139.9672535,1042.66618698)
\curveto(139.987253,1042.76618459)(139.99725299,1042.8511845)(139.9972535,1042.92118698)
\lineto(140.0872535,1043.25118698)
\curveto(140.11725287,1043.37118398)(140.15725283,1043.47618388)(140.2072535,1043.56618698)
\curveto(140.37725261,1043.8561835)(140.57225241,1044.07618328)(140.7922535,1044.22618698)
\curveto(141.01225197,1044.37618298)(141.29225169,1044.50618285)(141.6322535,1044.61618698)
\curveto(141.76225122,1044.66618269)(141.89725109,1044.70118265)(142.0372535,1044.72118698)
\curveto(142.17725081,1044.74118261)(142.31725067,1044.76618259)(142.4572535,1044.79618698)
\curveto(142.53725045,1044.81618254)(142.62225036,1044.82618253)(142.7122535,1044.82618698)
\curveto(142.80225018,1044.83618252)(142.89225009,1044.8511825)(142.9822535,1044.87118698)
\curveto(143.05224993,1044.89118246)(143.12224986,1044.89618246)(143.1922535,1044.88618698)
\curveto(143.26224972,1044.88618247)(143.33724965,1044.89618246)(143.4172535,1044.91618698)
\curveto(143.4872495,1044.93618242)(143.55724943,1044.94618241)(143.6272535,1044.94618698)
\curveto(143.69724929,1044.94618241)(143.77224921,1044.9561824)(143.8522535,1044.97618698)
\curveto(144.06224892,1045.02618233)(144.25224873,1045.06618229)(144.4222535,1045.09618698)
\curveto(144.60224838,1045.13618222)(144.76224822,1045.22618213)(144.9022535,1045.36618698)
\curveto(144.99224799,1045.4561819)(145.05224793,1045.5561818)(145.0822535,1045.66618698)
\curveto(145.09224789,1045.69618166)(145.09224789,1045.72118163)(145.0822535,1045.74118698)
\curveto(145.0822479,1045.76118159)(145.0872479,1045.78118157)(145.0972535,1045.80118698)
\curveto(145.10724788,1045.82118153)(145.11224787,1045.8511815)(145.1122535,1045.89118698)
\lineto(145.1122535,1045.98118698)
\lineto(145.0822535,1046.10118698)
\curveto(145.0822479,1046.14118121)(145.07724791,1046.17618118)(145.0672535,1046.20618698)
\curveto(144.96724802,1046.50618085)(144.75724823,1046.71118064)(144.4372535,1046.82118698)
\curveto(144.34724864,1046.8511805)(144.23724875,1046.87118048)(144.1072535,1046.88118698)
\curveto(143.987249,1046.90118045)(143.86224912,1046.90618045)(143.7322535,1046.89618698)
\curveto(143.60224938,1046.89618046)(143.47724951,1046.88618047)(143.3572535,1046.86618698)
\curveto(143.23724975,1046.84618051)(143.13224985,1046.82118053)(143.0422535,1046.79118698)
\curveto(142.98225,1046.77118058)(142.92225006,1046.74118061)(142.8622535,1046.70118698)
\curveto(142.81225017,1046.67118068)(142.76225022,1046.63618072)(142.7122535,1046.59618698)
\curveto(142.66225032,1046.5561808)(142.60725038,1046.50118085)(142.5472535,1046.43118698)
\curveto(142.49725049,1046.36118099)(142.46225052,1046.29618106)(142.4422535,1046.23618698)
\curveto(142.39225059,1046.13618122)(142.34725064,1046.04118131)(142.3072535,1045.95118698)
\curveto(142.27725071,1045.86118149)(142.20725078,1045.80118155)(142.0972535,1045.77118698)
\curveto(142.01725097,1045.7511816)(141.93225105,1045.74118161)(141.8422535,1045.74118698)
\lineto(141.5722535,1045.74118698)
\lineto(141.0022535,1045.74118698)
\curveto(140.95225203,1045.74118161)(140.90225208,1045.73618162)(140.8522535,1045.72618698)
\curveto(140.80225218,1045.72618163)(140.75725223,1045.73118162)(140.7172535,1045.74118698)
\lineto(140.5822535,1045.74118698)
\curveto(140.56225242,1045.7511816)(140.53725245,1045.7561816)(140.5072535,1045.75618698)
\curveto(140.47725251,1045.7561816)(140.45225253,1045.76618159)(140.4322535,1045.78618698)
\curveto(140.35225263,1045.80618155)(140.29725269,1045.87118148)(140.2672535,1045.98118698)
\curveto(140.25725273,1046.03118132)(140.25725273,1046.08118127)(140.2672535,1046.13118698)
\curveto(140.27725271,1046.18118117)(140.2872527,1046.22618113)(140.2972535,1046.26618698)
\curveto(140.32725266,1046.37618098)(140.35725263,1046.47618088)(140.3872535,1046.56618698)
\curveto(140.42725256,1046.66618069)(140.47225251,1046.7561806)(140.5222535,1046.83618698)
\lineto(140.6122535,1046.98618698)
\lineto(140.7022535,1047.13618698)
\curveto(140.7822522,1047.24618011)(140.8822521,1047.35118)(141.0022535,1047.45118698)
\curveto(141.02225196,1047.46117989)(141.05225193,1047.48617987)(141.0922535,1047.52618698)
\curveto(141.14225184,1047.56617979)(141.1872518,1047.60117975)(141.2272535,1047.63118698)
\curveto(141.26725172,1047.66117969)(141.31225167,1047.69117966)(141.3622535,1047.72118698)
\curveto(141.53225145,1047.83117952)(141.71225127,1047.91617944)(141.9022535,1047.97618698)
\curveto(142.09225089,1048.04617931)(142.2872507,1048.11117924)(142.4872535,1048.17118698)
\curveto(142.60725038,1048.20117915)(142.73225025,1048.22117913)(142.8622535,1048.23118698)
\curveto(142.99224999,1048.24117911)(143.12224986,1048.26117909)(143.2522535,1048.29118698)
\curveto(143.29224969,1048.30117905)(143.35224963,1048.30117905)(143.4322535,1048.29118698)
\curveto(143.52224946,1048.28117907)(143.57724941,1048.28617907)(143.5972535,1048.30618698)
\curveto(144.00724898,1048.31617904)(144.39724859,1048.30117905)(144.7672535,1048.26118698)
\curveto(145.14724784,1048.22117913)(145.4872475,1048.14617921)(145.7872535,1048.03618698)
\curveto(146.09724689,1047.92617943)(146.36224662,1047.77617958)(146.5822535,1047.58618698)
\curveto(146.80224618,1047.40617995)(146.97224601,1047.17118018)(147.0922535,1046.88118698)
\curveto(147.16224582,1046.71118064)(147.20224578,1046.51618084)(147.2122535,1046.29618698)
\curveto(147.22224576,1046.07618128)(147.22724576,1045.8511815)(147.2272535,1045.62118698)
\lineto(147.2272535,1042.27618698)
\lineto(147.2272535,1041.69118698)
\curveto(147.22724576,1041.50118585)(147.24724574,1041.32618603)(147.2872535,1041.16618698)
\curveto(147.29724569,1041.13618622)(147.30224568,1041.10118625)(147.3022535,1041.06118698)
\curveto(147.30224568,1041.03118632)(147.30724568,1041.00118635)(147.3172535,1040.97118698)
\moveto(145.1122535,1043.28118698)
\curveto(145.12224786,1043.33118402)(145.12724786,1043.38618397)(145.1272535,1043.44618698)
\curveto(145.12724786,1043.51618384)(145.12224786,1043.57618378)(145.1122535,1043.62618698)
\curveto(145.09224789,1043.68618367)(145.0822479,1043.74118361)(145.0822535,1043.79118698)
\curveto(145.0822479,1043.84118351)(145.06224792,1043.88118347)(145.0222535,1043.91118698)
\curveto(144.97224801,1043.9511834)(144.89724809,1043.97118338)(144.7972535,1043.97118698)
\curveto(144.75724823,1043.96118339)(144.72224826,1043.9511834)(144.6922535,1043.94118698)
\curveto(144.66224832,1043.94118341)(144.62724836,1043.93618342)(144.5872535,1043.92618698)
\curveto(144.51724847,1043.90618345)(144.44224854,1043.89118346)(144.3622535,1043.88118698)
\curveto(144.2822487,1043.87118348)(144.20224878,1043.8561835)(144.1222535,1043.83618698)
\curveto(144.09224889,1043.82618353)(144.04724894,1043.82118353)(143.9872535,1043.82118698)
\curveto(143.85724913,1043.79118356)(143.72724926,1043.77118358)(143.5972535,1043.76118698)
\curveto(143.46724952,1043.7511836)(143.34224964,1043.72618363)(143.2222535,1043.68618698)
\curveto(143.14224984,1043.66618369)(143.06724992,1043.64618371)(142.9972535,1043.62618698)
\curveto(142.92725006,1043.61618374)(142.85725013,1043.59618376)(142.7872535,1043.56618698)
\curveto(142.57725041,1043.47618388)(142.39725059,1043.34118401)(142.2472535,1043.16118698)
\curveto(142.10725088,1042.98118437)(142.05725093,1042.73118462)(142.0972535,1042.41118698)
\curveto(142.11725087,1042.24118511)(142.17225081,1042.10118525)(142.2622535,1041.99118698)
\curveto(142.33225065,1041.88118547)(142.43725055,1041.79118556)(142.5772535,1041.72118698)
\curveto(142.71725027,1041.66118569)(142.86725012,1041.61618574)(143.0272535,1041.58618698)
\curveto(143.19724979,1041.5561858)(143.37224961,1041.54618581)(143.5522535,1041.55618698)
\curveto(143.74224924,1041.57618578)(143.91724907,1041.61118574)(144.0772535,1041.66118698)
\curveto(144.33724865,1041.74118561)(144.54224844,1041.86618549)(144.6922535,1042.03618698)
\curveto(144.84224814,1042.21618514)(144.95724803,1042.43618492)(145.0372535,1042.69618698)
\curveto(145.05724793,1042.76618459)(145.06724792,1042.83618452)(145.0672535,1042.90618698)
\curveto(145.07724791,1042.98618437)(145.09224789,1043.06618429)(145.1122535,1043.14618698)
\lineto(145.1122535,1043.28118698)
}
}
{
\newrgbcolor{curcolor}{0 0 0}
\pscustom[linestyle=none,fillstyle=solid,fillcolor=curcolor]
{
\newpath
\moveto(156.47053475,1041.22618698)
\lineto(156.47053475,1040.80618698)
\curveto(156.47052638,1040.67618668)(156.44052641,1040.57118678)(156.38053475,1040.49118698)
\curveto(156.33052652,1040.44118691)(156.26552658,1040.40618695)(156.18553475,1040.38618698)
\curveto(156.10552674,1040.37618698)(156.01552683,1040.37118698)(155.91553475,1040.37118698)
\lineto(155.09053475,1040.37118698)
\lineto(154.80553475,1040.37118698)
\curveto(154.72552812,1040.38118697)(154.66052819,1040.40618695)(154.61053475,1040.44618698)
\curveto(154.54052831,1040.49618686)(154.50052835,1040.56118679)(154.49053475,1040.64118698)
\curveto(154.48052837,1040.72118663)(154.46052839,1040.80118655)(154.43053475,1040.88118698)
\curveto(154.41052844,1040.90118645)(154.39052846,1040.91618644)(154.37053475,1040.92618698)
\curveto(154.36052849,1040.94618641)(154.3455285,1040.96618639)(154.32553475,1040.98618698)
\curveto(154.21552863,1040.98618637)(154.13552871,1040.96118639)(154.08553475,1040.91118698)
\lineto(153.93553475,1040.76118698)
\curveto(153.86552898,1040.71118664)(153.80052905,1040.66618669)(153.74053475,1040.62618698)
\curveto(153.68052917,1040.59618676)(153.61552923,1040.5561868)(153.54553475,1040.50618698)
\curveto(153.50552934,1040.48618687)(153.46052939,1040.46618689)(153.41053475,1040.44618698)
\curveto(153.37052948,1040.42618693)(153.32552952,1040.40618695)(153.27553475,1040.38618698)
\curveto(153.13552971,1040.33618702)(152.98552986,1040.29118706)(152.82553475,1040.25118698)
\curveto(152.77553007,1040.23118712)(152.73053012,1040.22118713)(152.69053475,1040.22118698)
\curveto(152.6505302,1040.22118713)(152.61053024,1040.21618714)(152.57053475,1040.20618698)
\lineto(152.43553475,1040.20618698)
\curveto(152.40553044,1040.19618716)(152.36553048,1040.19118716)(152.31553475,1040.19118698)
\lineto(152.18053475,1040.19118698)
\curveto(152.12053073,1040.17118718)(152.03053082,1040.16618719)(151.91053475,1040.17618698)
\curveto(151.79053106,1040.17618718)(151.70553114,1040.18618717)(151.65553475,1040.20618698)
\curveto(151.58553126,1040.22618713)(151.52053133,1040.23618712)(151.46053475,1040.23618698)
\curveto(151.41053144,1040.22618713)(151.35553149,1040.23118712)(151.29553475,1040.25118698)
\lineto(150.93553475,1040.37118698)
\curveto(150.82553202,1040.40118695)(150.71553213,1040.44118691)(150.60553475,1040.49118698)
\curveto(150.25553259,1040.64118671)(149.94053291,1040.87118648)(149.66053475,1041.18118698)
\curveto(149.39053346,1041.50118585)(149.17553367,1041.83618552)(149.01553475,1042.18618698)
\curveto(148.96553388,1042.29618506)(148.92553392,1042.40118495)(148.89553475,1042.50118698)
\curveto(148.86553398,1042.61118474)(148.83053402,1042.72118463)(148.79053475,1042.83118698)
\curveto(148.78053407,1042.87118448)(148.77553407,1042.90618445)(148.77553475,1042.93618698)
\curveto(148.77553407,1042.97618438)(148.76553408,1043.02118433)(148.74553475,1043.07118698)
\curveto(148.72553412,1043.1511842)(148.70553414,1043.23618412)(148.68553475,1043.32618698)
\curveto(148.67553417,1043.42618393)(148.66053419,1043.52618383)(148.64053475,1043.62618698)
\curveto(148.63053422,1043.6561837)(148.62553422,1043.69118366)(148.62553475,1043.73118698)
\curveto(148.63553421,1043.77118358)(148.63553421,1043.80618355)(148.62553475,1043.83618698)
\lineto(148.62553475,1043.97118698)
\curveto(148.62553422,1044.02118333)(148.62053423,1044.07118328)(148.61053475,1044.12118698)
\curveto(148.60053425,1044.17118318)(148.59553425,1044.22618313)(148.59553475,1044.28618698)
\curveto(148.59553425,1044.356183)(148.60053425,1044.41118294)(148.61053475,1044.45118698)
\curveto(148.62053423,1044.50118285)(148.62553422,1044.54618281)(148.62553475,1044.58618698)
\lineto(148.62553475,1044.73618698)
\curveto(148.63553421,1044.78618257)(148.63553421,1044.83118252)(148.62553475,1044.87118698)
\curveto(148.62553422,1044.92118243)(148.63553421,1044.97118238)(148.65553475,1045.02118698)
\curveto(148.67553417,1045.13118222)(148.69053416,1045.23618212)(148.70053475,1045.33618698)
\curveto(148.72053413,1045.43618192)(148.7455341,1045.53618182)(148.77553475,1045.63618698)
\curveto(148.81553403,1045.7561816)(148.850534,1045.87118148)(148.88053475,1045.98118698)
\curveto(148.91053394,1046.09118126)(148.9505339,1046.20118115)(149.00053475,1046.31118698)
\curveto(149.14053371,1046.61118074)(149.31553353,1046.89618046)(149.52553475,1047.16618698)
\curveto(149.5455333,1047.19618016)(149.57053328,1047.22118013)(149.60053475,1047.24118698)
\curveto(149.64053321,1047.27118008)(149.67053318,1047.30118005)(149.69053475,1047.33118698)
\curveto(149.73053312,1047.38117997)(149.77053308,1047.42617993)(149.81053475,1047.46618698)
\curveto(149.850533,1047.50617985)(149.89553295,1047.54617981)(149.94553475,1047.58618698)
\curveto(149.98553286,1047.60617975)(150.02053283,1047.63117972)(150.05053475,1047.66118698)
\curveto(150.08053277,1047.70117965)(150.11553273,1047.73117962)(150.15553475,1047.75118698)
\curveto(150.40553244,1047.92117943)(150.69553215,1048.06117929)(151.02553475,1048.17118698)
\curveto(151.09553175,1048.19117916)(151.16553168,1048.20617915)(151.23553475,1048.21618698)
\curveto(151.31553153,1048.22617913)(151.39553145,1048.24117911)(151.47553475,1048.26118698)
\curveto(151.5455313,1048.28117907)(151.63553121,1048.29117906)(151.74553475,1048.29118698)
\curveto(151.85553099,1048.30117905)(151.96553088,1048.30617905)(152.07553475,1048.30618698)
\curveto(152.18553066,1048.30617905)(152.29053056,1048.30117905)(152.39053475,1048.29118698)
\curveto(152.50053035,1048.28117907)(152.59053026,1048.26617909)(152.66053475,1048.24618698)
\curveto(152.81053004,1048.19617916)(152.95552989,1048.1511792)(153.09553475,1048.11118698)
\curveto(153.23552961,1048.07117928)(153.36552948,1048.01617934)(153.48553475,1047.94618698)
\curveto(153.55552929,1047.89617946)(153.62052923,1047.84617951)(153.68053475,1047.79618698)
\curveto(153.74052911,1047.7561796)(153.80552904,1047.71117964)(153.87553475,1047.66118698)
\curveto(153.91552893,1047.63117972)(153.97052888,1047.59117976)(154.04053475,1047.54118698)
\curveto(154.12052873,1047.49117986)(154.19552865,1047.49117986)(154.26553475,1047.54118698)
\curveto(154.30552854,1047.56117979)(154.32552852,1047.59617976)(154.32553475,1047.64618698)
\curveto(154.32552852,1047.69617966)(154.33552851,1047.74617961)(154.35553475,1047.79618698)
\lineto(154.35553475,1047.94618698)
\curveto(154.36552848,1047.97617938)(154.37052848,1048.01117934)(154.37053475,1048.05118698)
\lineto(154.37053475,1048.17118698)
\lineto(154.37053475,1050.21118698)
\curveto(154.37052848,1050.32117703)(154.36552848,1050.44117691)(154.35553475,1050.57118698)
\curveto(154.35552849,1050.71117664)(154.38052847,1050.81617654)(154.43053475,1050.88618698)
\curveto(154.47052838,1050.96617639)(154.5455283,1051.01617634)(154.65553475,1051.03618698)
\curveto(154.67552817,1051.04617631)(154.69552815,1051.04617631)(154.71553475,1051.03618698)
\curveto(154.73552811,1051.03617632)(154.75552809,1051.04117631)(154.77553475,1051.05118698)
\lineto(155.84053475,1051.05118698)
\curveto(155.96052689,1051.0511763)(156.07052678,1051.04617631)(156.17053475,1051.03618698)
\curveto(156.27052658,1051.02617633)(156.3455265,1050.98617637)(156.39553475,1050.91618698)
\curveto(156.4455264,1050.83617652)(156.47052638,1050.73117662)(156.47053475,1050.60118698)
\lineto(156.47053475,1050.24118698)
\lineto(156.47053475,1041.22618698)
\moveto(154.43053475,1044.16618698)
\curveto(154.44052841,1044.20618315)(154.44052841,1044.24618311)(154.43053475,1044.28618698)
\lineto(154.43053475,1044.42118698)
\curveto(154.43052842,1044.52118283)(154.42552842,1044.62118273)(154.41553475,1044.72118698)
\curveto(154.40552844,1044.82118253)(154.39052846,1044.91118244)(154.37053475,1044.99118698)
\curveto(154.3505285,1045.10118225)(154.33052852,1045.20118215)(154.31053475,1045.29118698)
\curveto(154.30052855,1045.38118197)(154.27552857,1045.46618189)(154.23553475,1045.54618698)
\curveto(154.09552875,1045.90618145)(153.89052896,1046.19118116)(153.62053475,1046.40118698)
\curveto(153.36052949,1046.61118074)(152.98052987,1046.71618064)(152.48053475,1046.71618698)
\curveto(152.42053043,1046.71618064)(152.34053051,1046.70618065)(152.24053475,1046.68618698)
\curveto(152.16053069,1046.66618069)(152.08553076,1046.64618071)(152.01553475,1046.62618698)
\curveto(151.95553089,1046.61618074)(151.89553095,1046.59618076)(151.83553475,1046.56618698)
\curveto(151.56553128,1046.4561809)(151.35553149,1046.28618107)(151.20553475,1046.05618698)
\curveto(151.05553179,1045.82618153)(150.93553191,1045.56618179)(150.84553475,1045.27618698)
\curveto(150.81553203,1045.17618218)(150.79553205,1045.07618228)(150.78553475,1044.97618698)
\curveto(150.77553207,1044.87618248)(150.75553209,1044.77118258)(150.72553475,1044.66118698)
\lineto(150.72553475,1044.45118698)
\curveto(150.70553214,1044.36118299)(150.70053215,1044.23618312)(150.71053475,1044.07618698)
\curveto(150.72053213,1043.92618343)(150.73553211,1043.81618354)(150.75553475,1043.74618698)
\lineto(150.75553475,1043.65618698)
\curveto(150.76553208,1043.63618372)(150.77053208,1043.61618374)(150.77053475,1043.59618698)
\curveto(150.79053206,1043.51618384)(150.80553204,1043.44118391)(150.81553475,1043.37118698)
\curveto(150.83553201,1043.30118405)(150.85553199,1043.22618413)(150.87553475,1043.14618698)
\curveto(151.0455318,1042.62618473)(151.33553151,1042.24118511)(151.74553475,1041.99118698)
\curveto(151.87553097,1041.90118545)(152.05553079,1041.83118552)(152.28553475,1041.78118698)
\curveto(152.32553052,1041.77118558)(152.38553046,1041.76618559)(152.46553475,1041.76618698)
\curveto(152.49553035,1041.7561856)(152.54053031,1041.74618561)(152.60053475,1041.73618698)
\curveto(152.67053018,1041.73618562)(152.72553012,1041.74118561)(152.76553475,1041.75118698)
\curveto(152.84553,1041.77118558)(152.92552992,1041.78618557)(153.00553475,1041.79618698)
\curveto(153.08552976,1041.80618555)(153.16552968,1041.82618553)(153.24553475,1041.85618698)
\curveto(153.49552935,1041.96618539)(153.69552915,1042.10618525)(153.84553475,1042.27618698)
\curveto(153.99552885,1042.44618491)(154.12552872,1042.66118469)(154.23553475,1042.92118698)
\curveto(154.27552857,1043.01118434)(154.30552854,1043.10118425)(154.32553475,1043.19118698)
\curveto(154.3455285,1043.29118406)(154.36552848,1043.39618396)(154.38553475,1043.50618698)
\curveto(154.39552845,1043.5561838)(154.39552845,1043.60118375)(154.38553475,1043.64118698)
\curveto(154.38552846,1043.69118366)(154.39552845,1043.74118361)(154.41553475,1043.79118698)
\curveto(154.42552842,1043.82118353)(154.43052842,1043.8561835)(154.43053475,1043.89618698)
\lineto(154.43053475,1044.03118698)
\lineto(154.43053475,1044.16618698)
}
}
{
\newrgbcolor{curcolor}{0 0 0}
\pscustom[linestyle=none,fillstyle=solid,fillcolor=curcolor]
{
\newpath
\moveto(165.82045662,1044.55618698)
\curveto(165.84044805,1044.49618286)(165.85044804,1044.41118294)(165.85045662,1044.30118698)
\curveto(165.85044804,1044.19118316)(165.84044805,1044.10618325)(165.82045662,1044.04618698)
\lineto(165.82045662,1043.89618698)
\curveto(165.80044809,1043.81618354)(165.7904481,1043.73618362)(165.79045662,1043.65618698)
\curveto(165.80044809,1043.57618378)(165.7954481,1043.49618386)(165.77545662,1043.41618698)
\curveto(165.75544814,1043.34618401)(165.74044815,1043.28118407)(165.73045662,1043.22118698)
\curveto(165.72044817,1043.16118419)(165.71044818,1043.09618426)(165.70045662,1043.02618698)
\curveto(165.66044823,1042.91618444)(165.62544827,1042.80118455)(165.59545662,1042.68118698)
\curveto(165.56544833,1042.57118478)(165.52544837,1042.46618489)(165.47545662,1042.36618698)
\curveto(165.26544863,1041.88618547)(164.9904489,1041.49618586)(164.65045662,1041.19618698)
\curveto(164.31044958,1040.89618646)(163.90044999,1040.64618671)(163.42045662,1040.44618698)
\curveto(163.30045059,1040.39618696)(163.17545072,1040.36118699)(163.04545662,1040.34118698)
\curveto(162.92545097,1040.31118704)(162.80045109,1040.28118707)(162.67045662,1040.25118698)
\curveto(162.62045127,1040.23118712)(162.56545133,1040.22118713)(162.50545662,1040.22118698)
\curveto(162.44545145,1040.22118713)(162.3904515,1040.21618714)(162.34045662,1040.20618698)
\lineto(162.23545662,1040.20618698)
\curveto(162.20545169,1040.19618716)(162.17545172,1040.19118716)(162.14545662,1040.19118698)
\curveto(162.0954518,1040.18118717)(162.01545188,1040.17618718)(161.90545662,1040.17618698)
\curveto(161.7954521,1040.16618719)(161.71045218,1040.17118718)(161.65045662,1040.19118698)
\lineto(161.50045662,1040.19118698)
\curveto(161.45045244,1040.20118715)(161.3954525,1040.20618715)(161.33545662,1040.20618698)
\curveto(161.28545261,1040.19618716)(161.23545266,1040.20118715)(161.18545662,1040.22118698)
\curveto(161.14545275,1040.23118712)(161.10545279,1040.23618712)(161.06545662,1040.23618698)
\curveto(161.03545286,1040.23618712)(160.9954529,1040.24118711)(160.94545662,1040.25118698)
\curveto(160.84545305,1040.28118707)(160.74545315,1040.30618705)(160.64545662,1040.32618698)
\curveto(160.54545335,1040.34618701)(160.45045344,1040.37618698)(160.36045662,1040.41618698)
\curveto(160.24045365,1040.4561869)(160.12545377,1040.49618686)(160.01545662,1040.53618698)
\curveto(159.91545398,1040.57618678)(159.81045408,1040.62618673)(159.70045662,1040.68618698)
\curveto(159.35045454,1040.89618646)(159.05045484,1041.14118621)(158.80045662,1041.42118698)
\curveto(158.55045534,1041.70118565)(158.34045555,1042.03618532)(158.17045662,1042.42618698)
\curveto(158.12045577,1042.51618484)(158.08045581,1042.61118474)(158.05045662,1042.71118698)
\curveto(158.03045586,1042.81118454)(158.00545589,1042.91618444)(157.97545662,1043.02618698)
\curveto(157.95545594,1043.07618428)(157.94545595,1043.12118423)(157.94545662,1043.16118698)
\curveto(157.94545595,1043.20118415)(157.93545596,1043.24618411)(157.91545662,1043.29618698)
\curveto(157.895456,1043.37618398)(157.88545601,1043.4561839)(157.88545662,1043.53618698)
\curveto(157.88545601,1043.62618373)(157.87545602,1043.71118364)(157.85545662,1043.79118698)
\curveto(157.84545605,1043.84118351)(157.84045605,1043.88618347)(157.84045662,1043.92618698)
\lineto(157.84045662,1044.06118698)
\curveto(157.82045607,1044.12118323)(157.81045608,1044.20618315)(157.81045662,1044.31618698)
\curveto(157.82045607,1044.42618293)(157.83545606,1044.51118284)(157.85545662,1044.57118698)
\lineto(157.85545662,1044.67618698)
\curveto(157.86545603,1044.72618263)(157.86545603,1044.77618258)(157.85545662,1044.82618698)
\curveto(157.85545604,1044.88618247)(157.86545603,1044.94118241)(157.88545662,1044.99118698)
\curveto(157.895456,1045.04118231)(157.90045599,1045.08618227)(157.90045662,1045.12618698)
\curveto(157.90045599,1045.17618218)(157.91045598,1045.22618213)(157.93045662,1045.27618698)
\curveto(157.97045592,1045.40618195)(158.00545589,1045.53118182)(158.03545662,1045.65118698)
\curveto(158.06545583,1045.78118157)(158.10545579,1045.90618145)(158.15545662,1046.02618698)
\curveto(158.33545556,1046.43618092)(158.55045534,1046.77618058)(158.80045662,1047.04618698)
\curveto(159.05045484,1047.32618003)(159.35545454,1047.58117977)(159.71545662,1047.81118698)
\curveto(159.81545408,1047.86117949)(159.92045397,1047.90617945)(160.03045662,1047.94618698)
\curveto(160.14045375,1047.98617937)(160.25045364,1048.03117932)(160.36045662,1048.08118698)
\curveto(160.4904534,1048.13117922)(160.62545327,1048.16617919)(160.76545662,1048.18618698)
\curveto(160.90545299,1048.20617915)(161.05045284,1048.23617912)(161.20045662,1048.27618698)
\curveto(161.28045261,1048.28617907)(161.35545254,1048.29117906)(161.42545662,1048.29118698)
\curveto(161.4954524,1048.29117906)(161.56545233,1048.29617906)(161.63545662,1048.30618698)
\curveto(162.21545168,1048.31617904)(162.71545118,1048.2561791)(163.13545662,1048.12618698)
\curveto(163.56545033,1047.99617936)(163.94544995,1047.81617954)(164.27545662,1047.58618698)
\curveto(164.38544951,1047.50617985)(164.4954494,1047.41617994)(164.60545662,1047.31618698)
\curveto(164.72544917,1047.22618013)(164.82544907,1047.12618023)(164.90545662,1047.01618698)
\curveto(164.98544891,1046.91618044)(165.05544884,1046.81618054)(165.11545662,1046.71618698)
\curveto(165.18544871,1046.61618074)(165.25544864,1046.51118084)(165.32545662,1046.40118698)
\curveto(165.3954485,1046.29118106)(165.45044844,1046.17118118)(165.49045662,1046.04118698)
\curveto(165.53044836,1045.92118143)(165.57544832,1045.79118156)(165.62545662,1045.65118698)
\curveto(165.65544824,1045.57118178)(165.68044821,1045.48618187)(165.70045662,1045.39618698)
\lineto(165.76045662,1045.12618698)
\curveto(165.77044812,1045.08618227)(165.77544812,1045.04618231)(165.77545662,1045.00618698)
\curveto(165.77544812,1044.96618239)(165.78044811,1044.92618243)(165.79045662,1044.88618698)
\curveto(165.81044808,1044.83618252)(165.81544808,1044.78118257)(165.80545662,1044.72118698)
\curveto(165.7954481,1044.66118269)(165.80044809,1044.60618275)(165.82045662,1044.55618698)
\moveto(163.72045662,1044.01618698)
\curveto(163.73045016,1044.06618329)(163.73545016,1044.13618322)(163.73545662,1044.22618698)
\curveto(163.73545016,1044.32618303)(163.73045016,1044.40118295)(163.72045662,1044.45118698)
\lineto(163.72045662,1044.57118698)
\curveto(163.70045019,1044.62118273)(163.6904502,1044.67618268)(163.69045662,1044.73618698)
\curveto(163.6904502,1044.79618256)(163.68545021,1044.8511825)(163.67545662,1044.90118698)
\curveto(163.67545022,1044.94118241)(163.67045022,1044.97118238)(163.66045662,1044.99118698)
\lineto(163.60045662,1045.23118698)
\curveto(163.5904503,1045.32118203)(163.57045032,1045.40618195)(163.54045662,1045.48618698)
\curveto(163.43045046,1045.74618161)(163.30045059,1045.96618139)(163.15045662,1046.14618698)
\curveto(163.00045089,1046.33618102)(162.80045109,1046.48618087)(162.55045662,1046.59618698)
\curveto(162.4904514,1046.61618074)(162.43045146,1046.63118072)(162.37045662,1046.64118698)
\curveto(162.31045158,1046.66118069)(162.24545165,1046.68118067)(162.17545662,1046.70118698)
\curveto(162.0954518,1046.72118063)(162.01045188,1046.72618063)(161.92045662,1046.71618698)
\lineto(161.65045662,1046.71618698)
\curveto(161.62045227,1046.69618066)(161.58545231,1046.68618067)(161.54545662,1046.68618698)
\curveto(161.50545239,1046.69618066)(161.47045242,1046.69618066)(161.44045662,1046.68618698)
\lineto(161.23045662,1046.62618698)
\curveto(161.17045272,1046.61618074)(161.11545278,1046.59618076)(161.06545662,1046.56618698)
\curveto(160.81545308,1046.4561809)(160.61045328,1046.29618106)(160.45045662,1046.08618698)
\curveto(160.30045359,1045.88618147)(160.18045371,1045.6511817)(160.09045662,1045.38118698)
\curveto(160.06045383,1045.28118207)(160.03545386,1045.17618218)(160.01545662,1045.06618698)
\curveto(160.00545389,1044.9561824)(159.9904539,1044.84618251)(159.97045662,1044.73618698)
\curveto(159.96045393,1044.68618267)(159.95545394,1044.63618272)(159.95545662,1044.58618698)
\lineto(159.95545662,1044.43618698)
\curveto(159.93545396,1044.36618299)(159.92545397,1044.26118309)(159.92545662,1044.12118698)
\curveto(159.93545396,1043.98118337)(159.95045394,1043.87618348)(159.97045662,1043.80618698)
\lineto(159.97045662,1043.67118698)
\curveto(159.9904539,1043.59118376)(160.00545389,1043.51118384)(160.01545662,1043.43118698)
\curveto(160.02545387,1043.36118399)(160.04045385,1043.28618407)(160.06045662,1043.20618698)
\curveto(160.16045373,1042.90618445)(160.26545363,1042.66118469)(160.37545662,1042.47118698)
\curveto(160.4954534,1042.29118506)(160.68045321,1042.12618523)(160.93045662,1041.97618698)
\curveto(161.00045289,1041.92618543)(161.07545282,1041.88618547)(161.15545662,1041.85618698)
\curveto(161.24545265,1041.82618553)(161.33545256,1041.80118555)(161.42545662,1041.78118698)
\curveto(161.46545243,1041.77118558)(161.50045239,1041.76618559)(161.53045662,1041.76618698)
\curveto(161.56045233,1041.77618558)(161.5954523,1041.77618558)(161.63545662,1041.76618698)
\lineto(161.75545662,1041.73618698)
\curveto(161.80545209,1041.73618562)(161.85045204,1041.74118561)(161.89045662,1041.75118698)
\lineto(162.01045662,1041.75118698)
\curveto(162.0904518,1041.77118558)(162.17045172,1041.78618557)(162.25045662,1041.79618698)
\curveto(162.33045156,1041.80618555)(162.40545149,1041.82618553)(162.47545662,1041.85618698)
\curveto(162.73545116,1041.9561854)(162.94545095,1042.09118526)(163.10545662,1042.26118698)
\curveto(163.26545063,1042.43118492)(163.40045049,1042.64118471)(163.51045662,1042.89118698)
\curveto(163.55045034,1042.99118436)(163.58045031,1043.09118426)(163.60045662,1043.19118698)
\curveto(163.62045027,1043.29118406)(163.64545025,1043.39618396)(163.67545662,1043.50618698)
\curveto(163.68545021,1043.54618381)(163.6904502,1043.58118377)(163.69045662,1043.61118698)
\curveto(163.6904502,1043.6511837)(163.6954502,1043.69118366)(163.70545662,1043.73118698)
\lineto(163.70545662,1043.86618698)
\curveto(163.70545019,1043.91618344)(163.71045018,1043.96618339)(163.72045662,1044.01618698)
}
}
{
\newrgbcolor{curcolor}{0 0 0}
\pscustom[linestyle=none,fillstyle=solid,fillcolor=curcolor]
{
\newpath
\moveto(170.1903785,1048.32118698)
\curveto(170.940374,1048.34117901)(171.59037335,1048.2561791)(172.1403785,1048.06618698)
\curveto(172.70037224,1047.88617947)(173.12537181,1047.57117978)(173.4153785,1047.12118698)
\curveto(173.48537145,1047.01118034)(173.54537139,1046.89618046)(173.5953785,1046.77618698)
\curveto(173.65537128,1046.66618069)(173.70537123,1046.54118081)(173.7453785,1046.40118698)
\curveto(173.76537117,1046.34118101)(173.77537116,1046.27618108)(173.7753785,1046.20618698)
\curveto(173.77537116,1046.13618122)(173.76537117,1046.07618128)(173.7453785,1046.02618698)
\curveto(173.70537123,1045.96618139)(173.65037129,1045.92618143)(173.5803785,1045.90618698)
\curveto(173.53037141,1045.88618147)(173.47037147,1045.87618148)(173.4003785,1045.87618698)
\lineto(173.1903785,1045.87618698)
\lineto(172.5303785,1045.87618698)
\curveto(172.46037248,1045.87618148)(172.39037255,1045.87118148)(172.3203785,1045.86118698)
\curveto(172.25037269,1045.86118149)(172.18537275,1045.87118148)(172.1253785,1045.89118698)
\curveto(172.02537291,1045.91118144)(171.95037299,1045.9511814)(171.9003785,1046.01118698)
\curveto(171.85037309,1046.07118128)(171.80537313,1046.13118122)(171.7653785,1046.19118698)
\lineto(171.6453785,1046.40118698)
\curveto(171.61537332,1046.48118087)(171.56537337,1046.54618081)(171.4953785,1046.59618698)
\curveto(171.39537354,1046.67618068)(171.29537364,1046.73618062)(171.1953785,1046.77618698)
\curveto(171.10537383,1046.81618054)(170.99037395,1046.8511805)(170.8503785,1046.88118698)
\curveto(170.78037416,1046.90118045)(170.67537426,1046.91618044)(170.5353785,1046.92618698)
\curveto(170.40537453,1046.93618042)(170.30537463,1046.93118042)(170.2353785,1046.91118698)
\lineto(170.1303785,1046.91118698)
\lineto(169.9803785,1046.88118698)
\curveto(169.940375,1046.88118047)(169.89537504,1046.87618048)(169.8453785,1046.86618698)
\curveto(169.67537526,1046.81618054)(169.5353754,1046.74618061)(169.4253785,1046.65618698)
\curveto(169.32537561,1046.57618078)(169.25537568,1046.4511809)(169.2153785,1046.28118698)
\curveto(169.19537574,1046.21118114)(169.19537574,1046.14618121)(169.2153785,1046.08618698)
\curveto(169.2353757,1046.02618133)(169.25537568,1045.97618138)(169.2753785,1045.93618698)
\curveto(169.34537559,1045.81618154)(169.42537551,1045.72118163)(169.5153785,1045.65118698)
\curveto(169.61537532,1045.58118177)(169.73037521,1045.52118183)(169.8603785,1045.47118698)
\curveto(170.05037489,1045.39118196)(170.25537468,1045.32118203)(170.4753785,1045.26118698)
\lineto(171.1653785,1045.11118698)
\curveto(171.40537353,1045.07118228)(171.6353733,1045.02118233)(171.8553785,1044.96118698)
\curveto(172.08537285,1044.91118244)(172.30037264,1044.84618251)(172.5003785,1044.76618698)
\curveto(172.59037235,1044.72618263)(172.67537226,1044.69118266)(172.7553785,1044.66118698)
\curveto(172.84537209,1044.64118271)(172.93037201,1044.60618275)(173.0103785,1044.55618698)
\curveto(173.20037174,1044.43618292)(173.37037157,1044.30618305)(173.5203785,1044.16618698)
\curveto(173.68037126,1044.02618333)(173.80537113,1043.8511835)(173.8953785,1043.64118698)
\curveto(173.92537101,1043.57118378)(173.95037099,1043.50118385)(173.9703785,1043.43118698)
\curveto(173.99037095,1043.36118399)(174.01037093,1043.28618407)(174.0303785,1043.20618698)
\curveto(174.0403709,1043.14618421)(174.04537089,1043.0511843)(174.0453785,1042.92118698)
\curveto(174.05537088,1042.80118455)(174.05537088,1042.70618465)(174.0453785,1042.63618698)
\lineto(174.0453785,1042.56118698)
\curveto(174.02537091,1042.50118485)(174.01037093,1042.44118491)(174.0003785,1042.38118698)
\curveto(174.00037094,1042.33118502)(173.99537094,1042.28118507)(173.9853785,1042.23118698)
\curveto(173.91537102,1041.93118542)(173.80537113,1041.66618569)(173.6553785,1041.43618698)
\curveto(173.49537144,1041.19618616)(173.30037164,1041.00118635)(173.0703785,1040.85118698)
\curveto(172.8403721,1040.70118665)(172.58037236,1040.57118678)(172.2903785,1040.46118698)
\curveto(172.18037276,1040.41118694)(172.06037288,1040.37618698)(171.9303785,1040.35618698)
\curveto(171.81037313,1040.33618702)(171.69037325,1040.31118704)(171.5703785,1040.28118698)
\curveto(171.48037346,1040.26118709)(171.38537355,1040.2511871)(171.2853785,1040.25118698)
\curveto(171.19537374,1040.24118711)(171.10537383,1040.22618713)(171.0153785,1040.20618698)
\lineto(170.7453785,1040.20618698)
\curveto(170.68537425,1040.18618717)(170.58037436,1040.17618718)(170.4303785,1040.17618698)
\curveto(170.29037465,1040.17618718)(170.19037475,1040.18618717)(170.1303785,1040.20618698)
\curveto(170.10037484,1040.20618715)(170.06537487,1040.21118714)(170.0253785,1040.22118698)
\lineto(169.9203785,1040.22118698)
\curveto(169.80037514,1040.24118711)(169.68037526,1040.2561871)(169.5603785,1040.26618698)
\curveto(169.4403755,1040.27618708)(169.32537561,1040.29618706)(169.2153785,1040.32618698)
\curveto(168.82537611,1040.43618692)(168.48037646,1040.56118679)(168.1803785,1040.70118698)
\curveto(167.88037706,1040.8511865)(167.62537731,1041.07118628)(167.4153785,1041.36118698)
\curveto(167.27537766,1041.5511858)(167.15537778,1041.77118558)(167.0553785,1042.02118698)
\curveto(167.0353779,1042.08118527)(167.01537792,1042.16118519)(166.9953785,1042.26118698)
\curveto(166.97537796,1042.31118504)(166.96037798,1042.38118497)(166.9503785,1042.47118698)
\curveto(166.940378,1042.56118479)(166.94537799,1042.63618472)(166.9653785,1042.69618698)
\curveto(166.99537794,1042.76618459)(167.04537789,1042.81618454)(167.1153785,1042.84618698)
\curveto(167.16537777,1042.86618449)(167.22537771,1042.87618448)(167.2953785,1042.87618698)
\lineto(167.5203785,1042.87618698)
\lineto(168.2253785,1042.87618698)
\lineto(168.4653785,1042.87618698)
\curveto(168.54537639,1042.87618448)(168.61537632,1042.86618449)(168.6753785,1042.84618698)
\curveto(168.78537615,1042.80618455)(168.85537608,1042.74118461)(168.8853785,1042.65118698)
\curveto(168.92537601,1042.56118479)(168.97037597,1042.46618489)(169.0203785,1042.36618698)
\curveto(169.0403759,1042.31618504)(169.07537586,1042.2511851)(169.1253785,1042.17118698)
\curveto(169.18537575,1042.09118526)(169.2353757,1042.04118531)(169.2753785,1042.02118698)
\curveto(169.39537554,1041.92118543)(169.51037543,1041.84118551)(169.6203785,1041.78118698)
\curveto(169.73037521,1041.73118562)(169.87037507,1041.68118567)(170.0403785,1041.63118698)
\curveto(170.09037485,1041.61118574)(170.1403748,1041.60118575)(170.1903785,1041.60118698)
\curveto(170.2403747,1041.61118574)(170.29037465,1041.61118574)(170.3403785,1041.60118698)
\curveto(170.42037452,1041.58118577)(170.50537443,1041.57118578)(170.5953785,1041.57118698)
\curveto(170.69537424,1041.58118577)(170.78037416,1041.59618576)(170.8503785,1041.61618698)
\curveto(170.90037404,1041.62618573)(170.94537399,1041.63118572)(170.9853785,1041.63118698)
\curveto(171.0353739,1041.63118572)(171.08537385,1041.64118571)(171.1353785,1041.66118698)
\curveto(171.27537366,1041.71118564)(171.40037354,1041.77118558)(171.5103785,1041.84118698)
\curveto(171.63037331,1041.91118544)(171.72537321,1042.00118535)(171.7953785,1042.11118698)
\curveto(171.84537309,1042.19118516)(171.88537305,1042.31618504)(171.9153785,1042.48618698)
\curveto(171.935373,1042.5561848)(171.935373,1042.62118473)(171.9153785,1042.68118698)
\curveto(171.89537304,1042.74118461)(171.87537306,1042.79118456)(171.8553785,1042.83118698)
\curveto(171.78537315,1042.97118438)(171.69537324,1043.07618428)(171.5853785,1043.14618698)
\curveto(171.48537345,1043.21618414)(171.36537357,1043.28118407)(171.2253785,1043.34118698)
\curveto(171.0353739,1043.42118393)(170.8353741,1043.48618387)(170.6253785,1043.53618698)
\curveto(170.41537452,1043.58618377)(170.20537473,1043.64118371)(169.9953785,1043.70118698)
\curveto(169.91537502,1043.72118363)(169.83037511,1043.73618362)(169.7403785,1043.74618698)
\curveto(169.66037528,1043.7561836)(169.58037536,1043.77118358)(169.5003785,1043.79118698)
\curveto(169.18037576,1043.88118347)(168.87537606,1043.96618339)(168.5853785,1044.04618698)
\curveto(168.29537664,1044.13618322)(168.03037691,1044.26618309)(167.7903785,1044.43618698)
\curveto(167.51037743,1044.63618272)(167.30537763,1044.90618245)(167.1753785,1045.24618698)
\curveto(167.15537778,1045.31618204)(167.1353778,1045.41118194)(167.1153785,1045.53118698)
\curveto(167.09537784,1045.60118175)(167.08037786,1045.68618167)(167.0703785,1045.78618698)
\curveto(167.06037788,1045.88618147)(167.06537787,1045.97618138)(167.0853785,1046.05618698)
\curveto(167.10537783,1046.10618125)(167.11037783,1046.14618121)(167.1003785,1046.17618698)
\curveto(167.09037785,1046.21618114)(167.09537784,1046.26118109)(167.1153785,1046.31118698)
\curveto(167.1353778,1046.42118093)(167.15537778,1046.52118083)(167.1753785,1046.61118698)
\curveto(167.20537773,1046.71118064)(167.2403777,1046.80618055)(167.2803785,1046.89618698)
\curveto(167.41037753,1047.18618017)(167.59037735,1047.42117993)(167.8203785,1047.60118698)
\curveto(168.05037689,1047.78117957)(168.31037663,1047.92617943)(168.6003785,1048.03618698)
\curveto(168.71037623,1048.08617927)(168.82537611,1048.12117923)(168.9453785,1048.14118698)
\curveto(169.06537587,1048.17117918)(169.19037575,1048.20117915)(169.3203785,1048.23118698)
\curveto(169.38037556,1048.2511791)(169.4403755,1048.26117909)(169.5003785,1048.26118698)
\lineto(169.6803785,1048.29118698)
\curveto(169.76037518,1048.30117905)(169.84537509,1048.30617905)(169.9353785,1048.30618698)
\curveto(170.02537491,1048.30617905)(170.11037483,1048.31117904)(170.1903785,1048.32118698)
}
}
{
\newrgbcolor{curcolor}{0 0 0}
\pscustom[linestyle=none,fillstyle=solid,fillcolor=curcolor]
{
\newpath
\moveto(393.83927712,1051.79547188)
\lineto(395.11427712,1051.79547188)
\curveto(395.22427434,1051.79546117)(395.32927423,1051.79046118)(395.42927712,1051.78047188)
\curveto(395.53927402,1051.7704612)(395.61927394,1051.73546123)(395.66927712,1051.67547188)
\curveto(395.71927384,1051.59546137)(395.74427382,1051.49046148)(395.74427712,1051.36047188)
\curveto(395.75427381,1051.24046173)(395.7592738,1051.11546185)(395.75927712,1050.98547188)
\lineto(395.75927712,1049.47047188)
\lineto(395.75927712,1046.38047188)
\lineto(395.75927712,1045.85547188)
\curveto(395.7592738,1045.81546715)(395.75427381,1045.7704672)(395.74427712,1045.72047188)
\curveto(395.74427382,1045.68046729)(395.74927381,1045.64046733)(395.75927712,1045.60047188)
\lineto(395.75927712,1045.36047188)
\curveto(395.7592738,1045.2704677)(395.75427381,1045.17546779)(395.74427712,1045.07547188)
\curveto(395.74427382,1044.97546799)(395.75427381,1044.88546808)(395.77427712,1044.80547188)
\curveto(395.77427379,1044.73546823)(395.77927378,1044.68046829)(395.78927712,1044.64047188)
\curveto(395.80927375,1044.53046844)(395.82427374,1044.42046855)(395.83427712,1044.31047188)
\curveto(395.85427371,1044.20046877)(395.88427368,1044.09046888)(395.92427712,1043.98047188)
\curveto(396.03427353,1043.72046925)(396.17427339,1043.50546946)(396.34427712,1043.33547188)
\curveto(396.52427304,1043.1654698)(396.7592728,1043.03046994)(397.04927712,1042.93047188)
\curveto(397.12927243,1042.91047006)(397.20927235,1042.89547007)(397.28927712,1042.88547188)
\curveto(397.36927219,1042.87547009)(397.44927211,1042.86047011)(397.52927712,1042.84047188)
\curveto(397.57927198,1042.82047015)(397.62427194,1042.81047016)(397.66427712,1042.81047188)
\curveto(397.70427186,1042.82047015)(397.74927181,1042.82047015)(397.79927712,1042.81047188)
\curveto(397.83927172,1042.80047017)(397.90427166,1042.79547017)(397.99427712,1042.79547188)
\curveto(398.08427148,1042.80547016)(398.14427142,1042.81547015)(398.17427712,1042.82547188)
\lineto(398.39927712,1042.82547188)
\curveto(398.47927108,1042.84547012)(398.559271,1042.86047011)(398.63927712,1042.87047188)
\curveto(398.71927084,1042.88047009)(398.79427077,1042.89547007)(398.86427712,1042.91547188)
\curveto(399.00427056,1042.94547002)(399.11427045,1042.98046999)(399.19427712,1043.02047188)
\curveto(399.37427019,1043.10046987)(399.52927003,1043.20546976)(399.65927712,1043.33547188)
\curveto(399.79926976,1043.47546949)(399.90926965,1043.63046934)(399.98927712,1043.80047188)
\curveto(400.09926946,1044.06046891)(400.1642694,1044.3654686)(400.18427712,1044.71547188)
\curveto(400.20426936,1045.07546789)(400.21426935,1045.44546752)(400.21427712,1045.82547188)
\lineto(400.21427712,1048.81047188)
\lineto(400.21427712,1050.82047188)
\curveto(400.21426935,1050.96046201)(400.20926935,1051.11546185)(400.19927712,1051.28547188)
\curveto(400.19926936,1051.45546151)(400.22926933,1051.58046139)(400.28927712,1051.66047188)
\curveto(400.33926922,1051.72046125)(400.40926915,1051.75546121)(400.49927712,1051.76547188)
\curveto(400.58926897,1051.78546118)(400.68926887,1051.79546117)(400.79927712,1051.79547188)
\lineto(401.75927712,1051.79547188)
\curveto(401.83926772,1051.79546117)(401.91426765,1051.79546117)(401.98427712,1051.79547188)
\curveto(402.0642675,1051.80546116)(402.13926742,1051.80046117)(402.20927712,1051.78047188)
\curveto(402.34926721,1051.75046122)(402.43926712,1051.70046127)(402.47927712,1051.63047188)
\curveto(402.52926703,1051.55046142)(402.54926701,1051.43546153)(402.53927712,1051.28547188)
\curveto(402.53926702,1051.14546182)(402.53926702,1051.01546195)(402.53927712,1050.89547188)
\lineto(402.53927712,1048.88547188)
\lineto(402.53927712,1045.85547188)
\curveto(402.53926702,1045.47546749)(402.53426703,1045.10546786)(402.52427712,1044.74547188)
\curveto(402.51426705,1044.38546858)(402.46926709,1044.06046891)(402.38927712,1043.77047188)
\curveto(402.24926731,1043.30046967)(402.06926749,1042.89047008)(401.84927712,1042.54047188)
\curveto(401.63926792,1042.20047077)(401.3592682,1041.91047106)(401.00927712,1041.67047188)
\curveto(400.69926886,1041.45047152)(400.33426923,1041.2704717)(399.91427712,1041.13047188)
\curveto(399.82426974,1041.10047187)(399.72926983,1041.07547189)(399.62927712,1041.05547188)
\lineto(399.35927712,1040.99547188)
\curveto(399.29927026,1040.97547199)(399.23927032,1040.965472)(399.17927712,1040.96547188)
\curveto(399.12927043,1040.965472)(399.07427049,1040.95547201)(399.01427712,1040.93547188)
\curveto(398.89427067,1040.91547205)(398.7592708,1040.90047207)(398.60927712,1040.89047188)
\curveto(398.4592711,1040.88047209)(398.31427125,1040.87547209)(398.17427712,1040.87547188)
\curveto(397.22427234,1040.8654721)(396.41427315,1040.98047199)(395.74427712,1041.22047188)
\curveto(395.07427449,1041.4704715)(394.54927501,1041.8704711)(394.16927712,1042.42047188)
\curveto(394.03927552,1042.60047037)(393.92927563,1042.78547018)(393.83927712,1042.97547188)
\curveto(393.7592758,1043.17546979)(393.68427588,1043.39046958)(393.61427712,1043.62047188)
\curveto(393.59427597,1043.6704693)(393.58427598,1043.71046926)(393.58427712,1043.74047188)
\curveto(393.58427598,1043.78046919)(393.57427599,1043.82546914)(393.55427712,1043.87547188)
\curveto(393.47427609,1044.15546881)(393.43427613,1044.4704685)(393.43427712,1044.82047188)
\lineto(393.43427712,1045.87047188)
\lineto(393.43427712,1050.05547188)
\lineto(393.43427712,1051.10547188)
\lineto(393.43427712,1051.39047188)
\curveto(393.43427613,1051.49046148)(393.44927611,1051.5704614)(393.47927712,1051.63047188)
\curveto(393.53927602,1051.70046127)(393.61927594,1051.75046122)(393.71927712,1051.78047188)
\curveto(393.73927582,1051.78046119)(393.7592758,1051.78046119)(393.77927712,1051.78047188)
\curveto(393.79927576,1051.78046119)(393.81927574,1051.78546118)(393.83927712,1051.79547188)
}
}
{
\newrgbcolor{curcolor}{0 0 0}
\pscustom[linestyle=none,fillstyle=solid,fillcolor=curcolor]
{
\newpath
\moveto(407.29779274,1049.03547188)
\curveto(408.04778824,1049.05546391)(408.69778759,1048.970464)(409.24779274,1048.78047188)
\curveto(409.80778648,1048.60046437)(410.23278606,1048.28546468)(410.52279274,1047.83547188)
\curveto(410.5927857,1047.72546524)(410.65278564,1047.61046536)(410.70279274,1047.49047188)
\curveto(410.76278553,1047.38046559)(410.81278548,1047.25546571)(410.85279274,1047.11547188)
\curveto(410.87278542,1047.05546591)(410.88278541,1046.99046598)(410.88279274,1046.92047188)
\curveto(410.88278541,1046.85046612)(410.87278542,1046.79046618)(410.85279274,1046.74047188)
\curveto(410.81278548,1046.68046629)(410.75778553,1046.64046633)(410.68779274,1046.62047188)
\curveto(410.63778565,1046.60046637)(410.57778571,1046.59046638)(410.50779274,1046.59047188)
\lineto(410.29779274,1046.59047188)
\lineto(409.63779274,1046.59047188)
\curveto(409.56778672,1046.59046638)(409.49778679,1046.58546638)(409.42779274,1046.57547188)
\curveto(409.35778693,1046.57546639)(409.292787,1046.58546638)(409.23279274,1046.60547188)
\curveto(409.13278716,1046.62546634)(409.05778723,1046.6654663)(409.00779274,1046.72547188)
\curveto(408.95778733,1046.78546618)(408.91278738,1046.84546612)(408.87279274,1046.90547188)
\lineto(408.75279274,1047.11547188)
\curveto(408.72278757,1047.19546577)(408.67278762,1047.26046571)(408.60279274,1047.31047188)
\curveto(408.50278779,1047.39046558)(408.40278789,1047.45046552)(408.30279274,1047.49047188)
\curveto(408.21278808,1047.53046544)(408.09778819,1047.5654654)(407.95779274,1047.59547188)
\curveto(407.8877884,1047.61546535)(407.78278851,1047.63046534)(407.64279274,1047.64047188)
\curveto(407.51278878,1047.65046532)(407.41278888,1047.64546532)(407.34279274,1047.62547188)
\lineto(407.23779274,1047.62547188)
\lineto(407.08779274,1047.59547188)
\curveto(407.04778924,1047.59546537)(407.00278929,1047.59046538)(406.95279274,1047.58047188)
\curveto(406.78278951,1047.53046544)(406.64278965,1047.46046551)(406.53279274,1047.37047188)
\curveto(406.43278986,1047.29046568)(406.36278993,1047.1654658)(406.32279274,1046.99547188)
\curveto(406.30278999,1046.92546604)(406.30278999,1046.86046611)(406.32279274,1046.80047188)
\curveto(406.34278995,1046.74046623)(406.36278993,1046.69046628)(406.38279274,1046.65047188)
\curveto(406.45278984,1046.53046644)(406.53278976,1046.43546653)(406.62279274,1046.36547188)
\curveto(406.72278957,1046.29546667)(406.83778945,1046.23546673)(406.96779274,1046.18547188)
\curveto(407.15778913,1046.10546686)(407.36278893,1046.03546693)(407.58279274,1045.97547188)
\lineto(408.27279274,1045.82547188)
\curveto(408.51278778,1045.78546718)(408.74278755,1045.73546723)(408.96279274,1045.67547188)
\curveto(409.1927871,1045.62546734)(409.40778688,1045.56046741)(409.60779274,1045.48047188)
\curveto(409.69778659,1045.44046753)(409.78278651,1045.40546756)(409.86279274,1045.37547188)
\curveto(409.95278634,1045.35546761)(410.03778625,1045.32046765)(410.11779274,1045.27047188)
\curveto(410.30778598,1045.15046782)(410.47778581,1045.02046795)(410.62779274,1044.88047188)
\curveto(410.7877855,1044.74046823)(410.91278538,1044.5654684)(411.00279274,1044.35547188)
\curveto(411.03278526,1044.28546868)(411.05778523,1044.21546875)(411.07779274,1044.14547188)
\curveto(411.09778519,1044.07546889)(411.11778517,1044.00046897)(411.13779274,1043.92047188)
\curveto(411.14778514,1043.86046911)(411.15278514,1043.7654692)(411.15279274,1043.63547188)
\curveto(411.16278513,1043.51546945)(411.16278513,1043.42046955)(411.15279274,1043.35047188)
\lineto(411.15279274,1043.27547188)
\curveto(411.13278516,1043.21546975)(411.11778517,1043.15546981)(411.10779274,1043.09547188)
\curveto(411.10778518,1043.04546992)(411.10278519,1042.99546997)(411.09279274,1042.94547188)
\curveto(411.02278527,1042.64547032)(410.91278538,1042.38047059)(410.76279274,1042.15047188)
\curveto(410.60278569,1041.91047106)(410.40778588,1041.71547125)(410.17779274,1041.56547188)
\curveto(409.94778634,1041.41547155)(409.6877866,1041.28547168)(409.39779274,1041.17547188)
\curveto(409.287787,1041.12547184)(409.16778712,1041.09047188)(409.03779274,1041.07047188)
\curveto(408.91778737,1041.05047192)(408.79778749,1041.02547194)(408.67779274,1040.99547188)
\curveto(408.5877877,1040.97547199)(408.4927878,1040.965472)(408.39279274,1040.96547188)
\curveto(408.30278799,1040.95547201)(408.21278808,1040.94047203)(408.12279274,1040.92047188)
\lineto(407.85279274,1040.92047188)
\curveto(407.7927885,1040.90047207)(407.6877886,1040.89047208)(407.53779274,1040.89047188)
\curveto(407.39778889,1040.89047208)(407.29778899,1040.90047207)(407.23779274,1040.92047188)
\curveto(407.20778908,1040.92047205)(407.17278912,1040.92547204)(407.13279274,1040.93547188)
\lineto(407.02779274,1040.93547188)
\curveto(406.90778938,1040.95547201)(406.7877895,1040.970472)(406.66779274,1040.98047188)
\curveto(406.54778974,1040.99047198)(406.43278986,1041.01047196)(406.32279274,1041.04047188)
\curveto(405.93279036,1041.15047182)(405.5877907,1041.27547169)(405.28779274,1041.41547188)
\curveto(404.9877913,1041.5654714)(404.73279156,1041.78547118)(404.52279274,1042.07547188)
\curveto(404.38279191,1042.2654707)(404.26279203,1042.48547048)(404.16279274,1042.73547188)
\curveto(404.14279215,1042.79547017)(404.12279217,1042.87547009)(404.10279274,1042.97547188)
\curveto(404.08279221,1043.02546994)(404.06779222,1043.09546987)(404.05779274,1043.18547188)
\curveto(404.04779224,1043.27546969)(404.05279224,1043.35046962)(404.07279274,1043.41047188)
\curveto(404.10279219,1043.48046949)(404.15279214,1043.53046944)(404.22279274,1043.56047188)
\curveto(404.27279202,1043.58046939)(404.33279196,1043.59046938)(404.40279274,1043.59047188)
\lineto(404.62779274,1043.59047188)
\lineto(405.33279274,1043.59047188)
\lineto(405.57279274,1043.59047188)
\curveto(405.65279064,1043.59046938)(405.72279057,1043.58046939)(405.78279274,1043.56047188)
\curveto(405.8927904,1043.52046945)(405.96279033,1043.45546951)(405.99279274,1043.36547188)
\curveto(406.03279026,1043.27546969)(406.07779021,1043.18046979)(406.12779274,1043.08047188)
\curveto(406.14779014,1043.03046994)(406.18279011,1042.96547)(406.23279274,1042.88547188)
\curveto(406.29279,1042.80547016)(406.34278995,1042.75547021)(406.38279274,1042.73547188)
\curveto(406.50278979,1042.63547033)(406.61778967,1042.55547041)(406.72779274,1042.49547188)
\curveto(406.83778945,1042.44547052)(406.97778931,1042.39547057)(407.14779274,1042.34547188)
\curveto(407.19778909,1042.32547064)(407.24778904,1042.31547065)(407.29779274,1042.31547188)
\curveto(407.34778894,1042.32547064)(407.39778889,1042.32547064)(407.44779274,1042.31547188)
\curveto(407.52778876,1042.29547067)(407.61278868,1042.28547068)(407.70279274,1042.28547188)
\curveto(407.80278849,1042.29547067)(407.8877884,1042.31047066)(407.95779274,1042.33047188)
\curveto(408.00778828,1042.34047063)(408.05278824,1042.34547062)(408.09279274,1042.34547188)
\curveto(408.14278815,1042.34547062)(408.1927881,1042.35547061)(408.24279274,1042.37547188)
\curveto(408.38278791,1042.42547054)(408.50778778,1042.48547048)(408.61779274,1042.55547188)
\curveto(408.73778755,1042.62547034)(408.83278746,1042.71547025)(408.90279274,1042.82547188)
\curveto(408.95278734,1042.90547006)(408.9927873,1043.03046994)(409.02279274,1043.20047188)
\curveto(409.04278725,1043.2704697)(409.04278725,1043.33546963)(409.02279274,1043.39547188)
\curveto(409.00278729,1043.45546951)(408.98278731,1043.50546946)(408.96279274,1043.54547188)
\curveto(408.8927874,1043.68546928)(408.80278749,1043.79046918)(408.69279274,1043.86047188)
\curveto(408.5927877,1043.93046904)(408.47278782,1043.99546897)(408.33279274,1044.05547188)
\curveto(408.14278815,1044.13546883)(407.94278835,1044.20046877)(407.73279274,1044.25047188)
\curveto(407.52278877,1044.30046867)(407.31278898,1044.35546861)(407.10279274,1044.41547188)
\curveto(407.02278927,1044.43546853)(406.93778935,1044.45046852)(406.84779274,1044.46047188)
\curveto(406.76778952,1044.4704685)(406.6877896,1044.48546848)(406.60779274,1044.50547188)
\curveto(406.28779,1044.59546837)(405.98279031,1044.68046829)(405.69279274,1044.76047188)
\curveto(405.40279089,1044.85046812)(405.13779115,1044.98046799)(404.89779274,1045.15047188)
\curveto(404.61779167,1045.35046762)(404.41279188,1045.62046735)(404.28279274,1045.96047188)
\curveto(404.26279203,1046.03046694)(404.24279205,1046.12546684)(404.22279274,1046.24547188)
\curveto(404.20279209,1046.31546665)(404.1877921,1046.40046657)(404.17779274,1046.50047188)
\curveto(404.16779212,1046.60046637)(404.17279212,1046.69046628)(404.19279274,1046.77047188)
\curveto(404.21279208,1046.82046615)(404.21779207,1046.86046611)(404.20779274,1046.89047188)
\curveto(404.19779209,1046.93046604)(404.20279209,1046.97546599)(404.22279274,1047.02547188)
\curveto(404.24279205,1047.13546583)(404.26279203,1047.23546573)(404.28279274,1047.32547188)
\curveto(404.31279198,1047.42546554)(404.34779194,1047.52046545)(404.38779274,1047.61047188)
\curveto(404.51779177,1047.90046507)(404.69779159,1048.13546483)(404.92779274,1048.31547188)
\curveto(405.15779113,1048.49546447)(405.41779087,1048.64046433)(405.70779274,1048.75047188)
\curveto(405.81779047,1048.80046417)(405.93279036,1048.83546413)(406.05279274,1048.85547188)
\curveto(406.17279012,1048.88546408)(406.29778999,1048.91546405)(406.42779274,1048.94547188)
\curveto(406.4877898,1048.965464)(406.54778974,1048.97546399)(406.60779274,1048.97547188)
\lineto(406.78779274,1049.00547188)
\curveto(406.86778942,1049.01546395)(406.95278934,1049.02046395)(407.04279274,1049.02047188)
\curveto(407.13278916,1049.02046395)(407.21778907,1049.02546394)(407.29779274,1049.03547188)
}
}
{
\newrgbcolor{curcolor}{0 0 0}
\pscustom[linestyle=none,fillstyle=solid,fillcolor=curcolor]
{
\newpath
\moveto(412.80443337,1048.81047188)
\lineto(413.92943337,1048.81047188)
\curveto(414.03943093,1048.81046416)(414.13943083,1048.80546416)(414.22943337,1048.79547188)
\curveto(414.31943065,1048.78546418)(414.38443059,1048.75046422)(414.42443337,1048.69047188)
\curveto(414.4744305,1048.63046434)(414.50443047,1048.54546442)(414.51443337,1048.43547188)
\curveto(414.52443045,1048.33546463)(414.52943044,1048.23046474)(414.52943337,1048.12047188)
\lineto(414.52943337,1047.07047188)
\lineto(414.52943337,1044.83547188)
\curveto(414.52943044,1044.47546849)(414.54443043,1044.13546883)(414.57443337,1043.81547188)
\curveto(414.60443037,1043.49546947)(414.69443028,1043.23046974)(414.84443337,1043.02047188)
\curveto(414.98442999,1042.81047016)(415.20942976,1042.66047031)(415.51943337,1042.57047188)
\curveto(415.5694294,1042.56047041)(415.60942936,1042.55547041)(415.63943337,1042.55547188)
\curveto(415.67942929,1042.55547041)(415.72442925,1042.55047042)(415.77443337,1042.54047188)
\curveto(415.82442915,1042.53047044)(415.87942909,1042.52547044)(415.93943337,1042.52547188)
\curveto(415.99942897,1042.52547044)(416.04442893,1042.53047044)(416.07443337,1042.54047188)
\curveto(416.12442885,1042.56047041)(416.16442881,1042.5654704)(416.19443337,1042.55547188)
\curveto(416.23442874,1042.54547042)(416.2744287,1042.55047042)(416.31443337,1042.57047188)
\curveto(416.52442845,1042.62047035)(416.68942828,1042.68547028)(416.80943337,1042.76547188)
\curveto(416.98942798,1042.87547009)(417.12942784,1043.01546995)(417.22943337,1043.18547188)
\curveto(417.33942763,1043.3654696)(417.41442756,1043.56046941)(417.45443337,1043.77047188)
\curveto(417.50442747,1043.99046898)(417.53442744,1044.23046874)(417.54443337,1044.49047188)
\curveto(417.55442742,1044.76046821)(417.55942741,1045.04046793)(417.55943337,1045.33047188)
\lineto(417.55943337,1047.14547188)
\lineto(417.55943337,1048.12047188)
\lineto(417.55943337,1048.39047188)
\curveto(417.55942741,1048.49046448)(417.57942739,1048.5704644)(417.61943337,1048.63047188)
\curveto(417.6694273,1048.72046425)(417.74442723,1048.7704642)(417.84443337,1048.78047188)
\curveto(417.94442703,1048.80046417)(418.06442691,1048.81046416)(418.20443337,1048.81047188)
\lineto(418.99943337,1048.81047188)
\lineto(419.28443337,1048.81047188)
\curveto(419.3744256,1048.81046416)(419.44942552,1048.79046418)(419.50943337,1048.75047188)
\curveto(419.58942538,1048.70046427)(419.63442534,1048.62546434)(419.64443337,1048.52547188)
\curveto(419.65442532,1048.42546454)(419.65942531,1048.31046466)(419.65943337,1048.18047188)
\lineto(419.65943337,1047.04047188)
\lineto(419.65943337,1042.82547188)
\lineto(419.65943337,1041.76047188)
\lineto(419.65943337,1041.46047188)
\curveto(419.65942531,1041.36047161)(419.63942533,1041.28547168)(419.59943337,1041.23547188)
\curveto(419.54942542,1041.15547181)(419.4744255,1041.11047186)(419.37443337,1041.10047188)
\curveto(419.2744257,1041.09047188)(419.1694258,1041.08547188)(419.05943337,1041.08547188)
\lineto(418.24943337,1041.08547188)
\curveto(418.13942683,1041.08547188)(418.03942693,1041.09047188)(417.94943337,1041.10047188)
\curveto(417.8694271,1041.11047186)(417.80442717,1041.15047182)(417.75443337,1041.22047188)
\curveto(417.73442724,1041.25047172)(417.71442726,1041.29547167)(417.69443337,1041.35547188)
\curveto(417.68442729,1041.41547155)(417.6694273,1041.47547149)(417.64943337,1041.53547188)
\curveto(417.63942733,1041.59547137)(417.62442735,1041.65047132)(417.60443337,1041.70047188)
\curveto(417.58442739,1041.75047122)(417.55442742,1041.78047119)(417.51443337,1041.79047188)
\curveto(417.49442748,1041.81047116)(417.4694275,1041.81547115)(417.43943337,1041.80547188)
\curveto(417.40942756,1041.79547117)(417.38442759,1041.78547118)(417.36443337,1041.77547188)
\curveto(417.29442768,1041.73547123)(417.23442774,1041.69047128)(417.18443337,1041.64047188)
\curveto(417.13442784,1041.59047138)(417.07942789,1041.54547142)(417.01943337,1041.50547188)
\curveto(416.97942799,1041.47547149)(416.93942803,1041.44047153)(416.89943337,1041.40047188)
\curveto(416.8694281,1041.3704716)(416.82942814,1041.34047163)(416.77943337,1041.31047188)
\curveto(416.54942842,1041.1704718)(416.27942869,1041.06047191)(415.96943337,1040.98047188)
\curveto(415.89942907,1040.96047201)(415.82942914,1040.95047202)(415.75943337,1040.95047188)
\curveto(415.68942928,1040.94047203)(415.61442936,1040.92547204)(415.53443337,1040.90547188)
\curveto(415.49442948,1040.89547207)(415.44942952,1040.89547207)(415.39943337,1040.90547188)
\curveto(415.35942961,1040.90547206)(415.31942965,1040.90047207)(415.27943337,1040.89047188)
\curveto(415.24942972,1040.88047209)(415.18442979,1040.88047209)(415.08443337,1040.89047188)
\curveto(414.99442998,1040.89047208)(414.93443004,1040.89547207)(414.90443337,1040.90547188)
\curveto(414.85443012,1040.90547206)(414.80443017,1040.91047206)(414.75443337,1040.92047188)
\lineto(414.60443337,1040.92047188)
\curveto(414.48443049,1040.95047202)(414.3694306,1040.97547199)(414.25943337,1040.99547188)
\curveto(414.14943082,1041.01547195)(414.03943093,1041.04547192)(413.92943337,1041.08547188)
\curveto(413.87943109,1041.10547186)(413.83443114,1041.12047185)(413.79443337,1041.13047188)
\curveto(413.76443121,1041.15047182)(413.72443125,1041.1704718)(413.67443337,1041.19047188)
\curveto(413.32443165,1041.38047159)(413.04443193,1041.64547132)(412.83443337,1041.98547188)
\curveto(412.70443227,1042.19547077)(412.60943236,1042.44547052)(412.54943337,1042.73547188)
\curveto(412.48943248,1043.03546993)(412.44943252,1043.35046962)(412.42943337,1043.68047188)
\curveto(412.41943255,1044.02046895)(412.41443256,1044.3654686)(412.41443337,1044.71547188)
\curveto(412.42443255,1045.07546789)(412.42943254,1045.43046754)(412.42943337,1045.78047188)
\lineto(412.42943337,1047.82047188)
\curveto(412.42943254,1047.95046502)(412.42443255,1048.10046487)(412.41443337,1048.27047188)
\curveto(412.41443256,1048.45046452)(412.43943253,1048.58046439)(412.48943337,1048.66047188)
\curveto(412.51943245,1048.71046426)(412.57943239,1048.75546421)(412.66943337,1048.79547188)
\curveto(412.72943224,1048.79546417)(412.7744322,1048.80046417)(412.80443337,1048.81047188)
}
}
{
\newrgbcolor{curcolor}{0 0 0}
\pscustom[linestyle=none,fillstyle=solid,fillcolor=curcolor]
{
\newpath
\moveto(428.34068337,1041.68547188)
\curveto(428.36067552,1041.57547139)(428.37067551,1041.4654715)(428.37068337,1041.35547188)
\curveto(428.3806755,1041.24547172)(428.33067555,1041.1704718)(428.22068337,1041.13047188)
\curveto(428.16067572,1041.10047187)(428.09067579,1041.08547188)(428.01068337,1041.08547188)
\lineto(427.77068337,1041.08547188)
\lineto(426.96068337,1041.08547188)
\lineto(426.69068337,1041.08547188)
\curveto(426.61067727,1041.09547187)(426.54567733,1041.12047185)(426.49568337,1041.16047188)
\curveto(426.42567745,1041.20047177)(426.37067751,1041.25547171)(426.33068337,1041.32547188)
\curveto(426.30067758,1041.40547156)(426.25567762,1041.4704715)(426.19568337,1041.52047188)
\curveto(426.1756777,1041.54047143)(426.15067773,1041.55547141)(426.12068337,1041.56547188)
\curveto(426.09067779,1041.58547138)(426.05067783,1041.59047138)(426.00068337,1041.58047188)
\curveto(425.95067793,1041.56047141)(425.90067798,1041.53547143)(425.85068337,1041.50547188)
\curveto(425.81067807,1041.47547149)(425.76567811,1041.45047152)(425.71568337,1041.43047188)
\curveto(425.66567821,1041.39047158)(425.61067827,1041.35547161)(425.55068337,1041.32547188)
\lineto(425.37068337,1041.23547188)
\curveto(425.24067864,1041.17547179)(425.10567877,1041.12547184)(424.96568337,1041.08547188)
\curveto(424.82567905,1041.05547191)(424.6806792,1041.02047195)(424.53068337,1040.98047188)
\curveto(424.46067942,1040.96047201)(424.39067949,1040.95047202)(424.32068337,1040.95047188)
\curveto(424.26067962,1040.94047203)(424.19567968,1040.93047204)(424.12568337,1040.92047188)
\lineto(424.03568337,1040.92047188)
\curveto(424.00567987,1040.91047206)(423.9756799,1040.90547206)(423.94568337,1040.90547188)
\lineto(423.78068337,1040.90547188)
\curveto(423.6806802,1040.88547208)(423.5806803,1040.88547208)(423.48068337,1040.90547188)
\lineto(423.34568337,1040.90547188)
\curveto(423.2756806,1040.92547204)(423.20568067,1040.93547203)(423.13568337,1040.93547188)
\curveto(423.0756808,1040.92547204)(423.01568086,1040.93047204)(422.95568337,1040.95047188)
\curveto(422.85568102,1040.970472)(422.76068112,1040.99047198)(422.67068337,1041.01047188)
\curveto(422.5806813,1041.02047195)(422.49568138,1041.04547192)(422.41568337,1041.08547188)
\curveto(422.12568175,1041.19547177)(421.875682,1041.33547163)(421.66568337,1041.50547188)
\curveto(421.46568241,1041.68547128)(421.30568257,1041.92047105)(421.18568337,1042.21047188)
\curveto(421.15568272,1042.28047069)(421.12568275,1042.35547061)(421.09568337,1042.43547188)
\curveto(421.0756828,1042.51547045)(421.05568282,1042.60047037)(421.03568337,1042.69047188)
\curveto(421.01568286,1042.74047023)(421.00568287,1042.79047018)(421.00568337,1042.84047188)
\curveto(421.01568286,1042.89047008)(421.01568286,1042.94047003)(421.00568337,1042.99047188)
\curveto(420.99568288,1043.02046995)(420.98568289,1043.08046989)(420.97568337,1043.17047188)
\curveto(420.9756829,1043.2704697)(420.9806829,1043.34046963)(420.99068337,1043.38047188)
\curveto(421.01068287,1043.48046949)(421.02068286,1043.5654694)(421.02068337,1043.63547188)
\lineto(421.11068337,1043.96547188)
\curveto(421.14068274,1044.08546888)(421.1806827,1044.19046878)(421.23068337,1044.28047188)
\curveto(421.40068248,1044.5704684)(421.59568228,1044.79046818)(421.81568337,1044.94047188)
\curveto(422.03568184,1045.09046788)(422.31568156,1045.22046775)(422.65568337,1045.33047188)
\curveto(422.78568109,1045.38046759)(422.92068096,1045.41546755)(423.06068337,1045.43547188)
\curveto(423.20068068,1045.45546751)(423.34068054,1045.48046749)(423.48068337,1045.51047188)
\curveto(423.56068032,1045.53046744)(423.64568023,1045.54046743)(423.73568337,1045.54047188)
\curveto(423.82568005,1045.55046742)(423.91567996,1045.5654674)(424.00568337,1045.58547188)
\curveto(424.0756798,1045.60546736)(424.14567973,1045.61046736)(424.21568337,1045.60047188)
\curveto(424.28567959,1045.60046737)(424.36067952,1045.61046736)(424.44068337,1045.63047188)
\curveto(424.51067937,1045.65046732)(424.5806793,1045.66046731)(424.65068337,1045.66047188)
\curveto(424.72067916,1045.66046731)(424.79567908,1045.6704673)(424.87568337,1045.69047188)
\curveto(425.08567879,1045.74046723)(425.2756786,1045.78046719)(425.44568337,1045.81047188)
\curveto(425.62567825,1045.85046712)(425.78567809,1045.94046703)(425.92568337,1046.08047188)
\curveto(426.01567786,1046.1704668)(426.0756778,1046.2704667)(426.10568337,1046.38047188)
\curveto(426.11567776,1046.41046656)(426.11567776,1046.43546653)(426.10568337,1046.45547188)
\curveto(426.10567777,1046.47546649)(426.11067777,1046.49546647)(426.12068337,1046.51547188)
\curveto(426.13067775,1046.53546643)(426.13567774,1046.5654664)(426.13568337,1046.60547188)
\lineto(426.13568337,1046.69547188)
\lineto(426.10568337,1046.81547188)
\curveto(426.10567777,1046.85546611)(426.10067778,1046.89046608)(426.09068337,1046.92047188)
\curveto(425.99067789,1047.22046575)(425.7806781,1047.42546554)(425.46068337,1047.53547188)
\curveto(425.37067851,1047.5654654)(425.26067862,1047.58546538)(425.13068337,1047.59547188)
\curveto(425.01067887,1047.61546535)(424.88567899,1047.62046535)(424.75568337,1047.61047188)
\curveto(424.62567925,1047.61046536)(424.50067938,1047.60046537)(424.38068337,1047.58047188)
\curveto(424.26067962,1047.56046541)(424.15567972,1047.53546543)(424.06568337,1047.50547188)
\curveto(424.00567987,1047.48546548)(423.94567993,1047.45546551)(423.88568337,1047.41547188)
\curveto(423.83568004,1047.38546558)(423.78568009,1047.35046562)(423.73568337,1047.31047188)
\curveto(423.68568019,1047.2704657)(423.63068025,1047.21546575)(423.57068337,1047.14547188)
\curveto(423.52068036,1047.07546589)(423.48568039,1047.01046596)(423.46568337,1046.95047188)
\curveto(423.41568046,1046.85046612)(423.37068051,1046.75546621)(423.33068337,1046.66547188)
\curveto(423.30068058,1046.57546639)(423.23068065,1046.51546645)(423.12068337,1046.48547188)
\curveto(423.04068084,1046.4654665)(422.95568092,1046.45546651)(422.86568337,1046.45547188)
\lineto(422.59568337,1046.45547188)
\lineto(422.02568337,1046.45547188)
\curveto(421.9756819,1046.45546651)(421.92568195,1046.45046652)(421.87568337,1046.44047188)
\curveto(421.82568205,1046.44046653)(421.7806821,1046.44546652)(421.74068337,1046.45547188)
\lineto(421.60568337,1046.45547188)
\curveto(421.58568229,1046.4654665)(421.56068232,1046.4704665)(421.53068337,1046.47047188)
\curveto(421.50068238,1046.4704665)(421.4756824,1046.48046649)(421.45568337,1046.50047188)
\curveto(421.3756825,1046.52046645)(421.32068256,1046.58546638)(421.29068337,1046.69547188)
\curveto(421.2806826,1046.74546622)(421.2806826,1046.79546617)(421.29068337,1046.84547188)
\curveto(421.30068258,1046.89546607)(421.31068257,1046.94046603)(421.32068337,1046.98047188)
\curveto(421.35068253,1047.09046588)(421.3806825,1047.19046578)(421.41068337,1047.28047188)
\curveto(421.45068243,1047.38046559)(421.49568238,1047.4704655)(421.54568337,1047.55047188)
\lineto(421.63568337,1047.70047188)
\lineto(421.72568337,1047.85047188)
\curveto(421.80568207,1047.96046501)(421.90568197,1048.0654649)(422.02568337,1048.16547188)
\curveto(422.04568183,1048.17546479)(422.0756818,1048.20046477)(422.11568337,1048.24047188)
\curveto(422.16568171,1048.28046469)(422.21068167,1048.31546465)(422.25068337,1048.34547188)
\curveto(422.29068159,1048.37546459)(422.33568154,1048.40546456)(422.38568337,1048.43547188)
\curveto(422.55568132,1048.54546442)(422.73568114,1048.63046434)(422.92568337,1048.69047188)
\curveto(423.11568076,1048.76046421)(423.31068057,1048.82546414)(423.51068337,1048.88547188)
\curveto(423.63068025,1048.91546405)(423.75568012,1048.93546403)(423.88568337,1048.94547188)
\curveto(424.01567986,1048.95546401)(424.14567973,1048.97546399)(424.27568337,1049.00547188)
\curveto(424.31567956,1049.01546395)(424.3756795,1049.01546395)(424.45568337,1049.00547188)
\curveto(424.54567933,1048.99546397)(424.60067928,1049.00046397)(424.62068337,1049.02047188)
\curveto(425.03067885,1049.03046394)(425.42067846,1049.01546395)(425.79068337,1048.97547188)
\curveto(426.17067771,1048.93546403)(426.51067737,1048.86046411)(426.81068337,1048.75047188)
\curveto(427.12067676,1048.64046433)(427.38567649,1048.49046448)(427.60568337,1048.30047188)
\curveto(427.82567605,1048.12046485)(427.99567588,1047.88546508)(428.11568337,1047.59547188)
\curveto(428.18567569,1047.42546554)(428.22567565,1047.23046574)(428.23568337,1047.01047188)
\curveto(428.24567563,1046.79046618)(428.25067563,1046.5654664)(428.25068337,1046.33547188)
\lineto(428.25068337,1042.99047188)
\lineto(428.25068337,1042.40547188)
\curveto(428.25067563,1042.21547075)(428.27067561,1042.04047093)(428.31068337,1041.88047188)
\curveto(428.32067556,1041.85047112)(428.32567555,1041.81547115)(428.32568337,1041.77547188)
\curveto(428.32567555,1041.74547122)(428.33067555,1041.71547125)(428.34068337,1041.68547188)
\moveto(426.13568337,1043.99547188)
\curveto(426.14567773,1044.04546892)(426.15067773,1044.10046887)(426.15068337,1044.16047188)
\curveto(426.15067773,1044.23046874)(426.14567773,1044.29046868)(426.13568337,1044.34047188)
\curveto(426.11567776,1044.40046857)(426.10567777,1044.45546851)(426.10568337,1044.50547188)
\curveto(426.10567777,1044.55546841)(426.08567779,1044.59546837)(426.04568337,1044.62547188)
\curveto(425.99567788,1044.6654683)(425.92067796,1044.68546828)(425.82068337,1044.68547188)
\curveto(425.7806781,1044.67546829)(425.74567813,1044.6654683)(425.71568337,1044.65547188)
\curveto(425.68567819,1044.65546831)(425.65067823,1044.65046832)(425.61068337,1044.64047188)
\curveto(425.54067834,1044.62046835)(425.46567841,1044.60546836)(425.38568337,1044.59547188)
\curveto(425.30567857,1044.58546838)(425.22567865,1044.5704684)(425.14568337,1044.55047188)
\curveto(425.11567876,1044.54046843)(425.07067881,1044.53546843)(425.01068337,1044.53547188)
\curveto(424.880679,1044.50546846)(424.75067913,1044.48546848)(424.62068337,1044.47547188)
\curveto(424.49067939,1044.4654685)(424.36567951,1044.44046853)(424.24568337,1044.40047188)
\curveto(424.16567971,1044.38046859)(424.09067979,1044.36046861)(424.02068337,1044.34047188)
\curveto(423.95067993,1044.33046864)(423.88068,1044.31046866)(423.81068337,1044.28047188)
\curveto(423.60068028,1044.19046878)(423.42068046,1044.05546891)(423.27068337,1043.87547188)
\curveto(423.13068075,1043.69546927)(423.0806808,1043.44546952)(423.12068337,1043.12547188)
\curveto(423.14068074,1042.95547001)(423.19568068,1042.81547015)(423.28568337,1042.70547188)
\curveto(423.35568052,1042.59547037)(423.46068042,1042.50547046)(423.60068337,1042.43547188)
\curveto(423.74068014,1042.37547059)(423.89067999,1042.33047064)(424.05068337,1042.30047188)
\curveto(424.22067966,1042.2704707)(424.39567948,1042.26047071)(424.57568337,1042.27047188)
\curveto(424.76567911,1042.29047068)(424.94067894,1042.32547064)(425.10068337,1042.37547188)
\curveto(425.36067852,1042.45547051)(425.56567831,1042.58047039)(425.71568337,1042.75047188)
\curveto(425.86567801,1042.93047004)(425.9806779,1043.15046982)(426.06068337,1043.41047188)
\curveto(426.0806778,1043.48046949)(426.09067779,1043.55046942)(426.09068337,1043.62047188)
\curveto(426.10067778,1043.70046927)(426.11567776,1043.78046919)(426.13568337,1043.86047188)
\lineto(426.13568337,1043.99547188)
}
}
{
\newrgbcolor{curcolor}{0 0 0}
\pscustom[linestyle=none,fillstyle=solid,fillcolor=curcolor]
{
\newpath
\moveto(434.32896462,1049.02047188)
\curveto(434.4389593,1049.02046395)(434.53395921,1049.01046396)(434.61396462,1048.99047188)
\curveto(434.70395904,1048.970464)(434.77395897,1048.92546404)(434.82396462,1048.85547188)
\curveto(434.88395886,1048.77546419)(434.91395883,1048.63546433)(434.91396462,1048.43547188)
\lineto(434.91396462,1047.92547188)
\lineto(434.91396462,1047.55047188)
\curveto(434.92395882,1047.41046556)(434.90895883,1047.30046567)(434.86896462,1047.22047188)
\curveto(434.82895891,1047.15046582)(434.76895897,1047.10546586)(434.68896462,1047.08547188)
\curveto(434.61895912,1047.0654659)(434.53395921,1047.05546591)(434.43396462,1047.05547188)
\curveto(434.3439594,1047.05546591)(434.2439595,1047.06046591)(434.13396462,1047.07047188)
\curveto(434.03395971,1047.08046589)(433.9389598,1047.07546589)(433.84896462,1047.05547188)
\curveto(433.77895996,1047.03546593)(433.70896003,1047.02046595)(433.63896462,1047.01047188)
\curveto(433.56896017,1047.01046596)(433.50396024,1047.00046597)(433.44396462,1046.98047188)
\curveto(433.28396046,1046.93046604)(433.12396062,1046.85546611)(432.96396462,1046.75547188)
\curveto(432.80396094,1046.6654663)(432.67896106,1046.56046641)(432.58896462,1046.44047188)
\curveto(432.5389612,1046.36046661)(432.48396126,1046.27546669)(432.42396462,1046.18547188)
\curveto(432.37396137,1046.10546686)(432.32396142,1046.02046695)(432.27396462,1045.93047188)
\curveto(432.2439615,1045.85046712)(432.21396153,1045.7654672)(432.18396462,1045.67547188)
\lineto(432.12396462,1045.43547188)
\curveto(432.10396164,1045.3654676)(432.09396165,1045.29046768)(432.09396462,1045.21047188)
\curveto(432.09396165,1045.14046783)(432.08396166,1045.0704679)(432.06396462,1045.00047188)
\curveto(432.05396169,1044.96046801)(432.04896169,1044.92046805)(432.04896462,1044.88047188)
\curveto(432.05896168,1044.85046812)(432.05896168,1044.82046815)(432.04896462,1044.79047188)
\lineto(432.04896462,1044.55047188)
\curveto(432.02896171,1044.48046849)(432.02396172,1044.40046857)(432.03396462,1044.31047188)
\curveto(432.0439617,1044.23046874)(432.04896169,1044.15046882)(432.04896462,1044.07047188)
\lineto(432.04896462,1043.11047188)
\lineto(432.04896462,1041.83547188)
\curveto(432.04896169,1041.70547126)(432.0439617,1041.58547138)(432.03396462,1041.47547188)
\curveto(432.02396172,1041.3654716)(431.99396175,1041.27547169)(431.94396462,1041.20547188)
\curveto(431.92396182,1041.17547179)(431.88896185,1041.15047182)(431.83896462,1041.13047188)
\curveto(431.79896194,1041.12047185)(431.75396199,1041.11047186)(431.70396462,1041.10047188)
\lineto(431.62896462,1041.10047188)
\curveto(431.57896216,1041.09047188)(431.52396222,1041.08547188)(431.46396462,1041.08547188)
\lineto(431.29896462,1041.08547188)
\lineto(430.65396462,1041.08547188)
\curveto(430.59396315,1041.09547187)(430.52896321,1041.10047187)(430.45896462,1041.10047188)
\lineto(430.26396462,1041.10047188)
\curveto(430.21396353,1041.12047185)(430.16396358,1041.13547183)(430.11396462,1041.14547188)
\curveto(430.06396368,1041.1654718)(430.02896371,1041.20047177)(430.00896462,1041.25047188)
\curveto(429.96896377,1041.30047167)(429.9439638,1041.3704716)(429.93396462,1041.46047188)
\lineto(429.93396462,1041.76047188)
\lineto(429.93396462,1042.78047188)
\lineto(429.93396462,1047.01047188)
\lineto(429.93396462,1048.12047188)
\lineto(429.93396462,1048.40547188)
\curveto(429.93396381,1048.50546446)(429.95396379,1048.58546438)(429.99396462,1048.64547188)
\curveto(430.0439637,1048.72546424)(430.11896362,1048.77546419)(430.21896462,1048.79547188)
\curveto(430.31896342,1048.81546415)(430.4389633,1048.82546414)(430.57896462,1048.82547188)
\lineto(431.34396462,1048.82547188)
\curveto(431.46396228,1048.82546414)(431.56896217,1048.81546415)(431.65896462,1048.79547188)
\curveto(431.74896199,1048.78546418)(431.81896192,1048.74046423)(431.86896462,1048.66047188)
\curveto(431.89896184,1048.61046436)(431.91396183,1048.54046443)(431.91396462,1048.45047188)
\lineto(431.94396462,1048.18047188)
\curveto(431.95396179,1048.10046487)(431.96896177,1048.02546494)(431.98896462,1047.95547188)
\curveto(432.01896172,1047.88546508)(432.06896167,1047.85046512)(432.13896462,1047.85047188)
\curveto(432.15896158,1047.8704651)(432.17896156,1047.88046509)(432.19896462,1047.88047188)
\curveto(432.21896152,1047.88046509)(432.2389615,1047.89046508)(432.25896462,1047.91047188)
\curveto(432.31896142,1047.96046501)(432.36896137,1048.01546495)(432.40896462,1048.07547188)
\curveto(432.45896128,1048.14546482)(432.51896122,1048.20546476)(432.58896462,1048.25547188)
\curveto(432.62896111,1048.28546468)(432.66396108,1048.31546465)(432.69396462,1048.34547188)
\curveto(432.72396102,1048.38546458)(432.75896098,1048.42046455)(432.79896462,1048.45047188)
\lineto(433.06896462,1048.63047188)
\curveto(433.16896057,1048.69046428)(433.26896047,1048.74546422)(433.36896462,1048.79547188)
\curveto(433.46896027,1048.83546413)(433.56896017,1048.8704641)(433.66896462,1048.90047188)
\lineto(433.99896462,1048.99047188)
\curveto(434.02895971,1049.00046397)(434.08395966,1049.00046397)(434.16396462,1048.99047188)
\curveto(434.25395949,1048.99046398)(434.30895943,1049.00046397)(434.32896462,1049.02047188)
}
}
{
\newrgbcolor{curcolor}{0 0 0}
\pscustom[linestyle=none,fillstyle=solid,fillcolor=curcolor]
{
\newpath
\moveto(437.83404274,1051.67547188)
\curveto(437.90403979,1051.59546137)(437.93903976,1051.47546149)(437.93904274,1051.31547188)
\lineto(437.93904274,1050.85047188)
\lineto(437.93904274,1050.44547188)
\curveto(437.93903976,1050.30546266)(437.90403979,1050.21046276)(437.83404274,1050.16047188)
\curveto(437.77403992,1050.11046286)(437.69404,1050.08046289)(437.59404274,1050.07047188)
\curveto(437.50404019,1050.06046291)(437.40404029,1050.05546291)(437.29404274,1050.05547188)
\lineto(436.45404274,1050.05547188)
\curveto(436.34404135,1050.05546291)(436.24404145,1050.06046291)(436.15404274,1050.07047188)
\curveto(436.07404162,1050.08046289)(436.00404169,1050.11046286)(435.94404274,1050.16047188)
\curveto(435.90404179,1050.19046278)(435.87404182,1050.24546272)(435.85404274,1050.32547188)
\curveto(435.84404185,1050.41546255)(435.83404186,1050.51046246)(435.82404274,1050.61047188)
\lineto(435.82404274,1050.94047188)
\curveto(435.83404186,1051.05046192)(435.83904186,1051.14546182)(435.83904274,1051.22547188)
\lineto(435.83904274,1051.43547188)
\curveto(435.84904185,1051.50546146)(435.86904183,1051.5654614)(435.89904274,1051.61547188)
\curveto(435.91904178,1051.65546131)(435.94404175,1051.68546128)(435.97404274,1051.70547188)
\lineto(436.09404274,1051.76547188)
\curveto(436.11404158,1051.7654612)(436.13904156,1051.7654612)(436.16904274,1051.76547188)
\curveto(436.1990415,1051.77546119)(436.22404147,1051.78046119)(436.24404274,1051.78047188)
\lineto(437.33904274,1051.78047188)
\curveto(437.43904026,1051.78046119)(437.53404016,1051.77546119)(437.62404274,1051.76547188)
\curveto(437.71403998,1051.75546121)(437.78403991,1051.72546124)(437.83404274,1051.67547188)
\moveto(437.93904274,1041.91047188)
\curveto(437.93903976,1041.71047126)(437.93403976,1041.54047143)(437.92404274,1041.40047188)
\curveto(437.91403978,1041.26047171)(437.82403987,1041.1654718)(437.65404274,1041.11547188)
\curveto(437.5940401,1041.09547187)(437.52904017,1041.08547188)(437.45904274,1041.08547188)
\curveto(437.38904031,1041.09547187)(437.31404038,1041.10047187)(437.23404274,1041.10047188)
\lineto(436.39404274,1041.10047188)
\curveto(436.30404139,1041.10047187)(436.21404148,1041.10547186)(436.12404274,1041.11547188)
\curveto(436.04404165,1041.12547184)(435.98404171,1041.15547181)(435.94404274,1041.20547188)
\curveto(435.88404181,1041.27547169)(435.84904185,1041.36047161)(435.83904274,1041.46047188)
\lineto(435.83904274,1041.80547188)
\lineto(435.83904274,1048.13547188)
\lineto(435.83904274,1048.43547188)
\curveto(435.83904186,1048.53546443)(435.85904184,1048.61546435)(435.89904274,1048.67547188)
\curveto(435.95904174,1048.74546422)(436.04404165,1048.79046418)(436.15404274,1048.81047188)
\curveto(436.17404152,1048.82046415)(436.1990415,1048.82046415)(436.22904274,1048.81047188)
\curveto(436.26904143,1048.81046416)(436.2990414,1048.81546415)(436.31904274,1048.82547188)
\lineto(437.06904274,1048.82547188)
\lineto(437.26404274,1048.82547188)
\curveto(437.34404035,1048.83546413)(437.40904029,1048.83546413)(437.45904274,1048.82547188)
\lineto(437.57904274,1048.82547188)
\curveto(437.63904006,1048.80546416)(437.69404,1048.79046418)(437.74404274,1048.78047188)
\curveto(437.7940399,1048.7704642)(437.83403986,1048.74046423)(437.86404274,1048.69047188)
\curveto(437.90403979,1048.64046433)(437.92403977,1048.5704644)(437.92404274,1048.48047188)
\curveto(437.93403976,1048.39046458)(437.93903976,1048.29546467)(437.93904274,1048.19547188)
\lineto(437.93904274,1041.91047188)
}
}
{
\newrgbcolor{curcolor}{0 0 0}
\pscustom[linestyle=none,fillstyle=solid,fillcolor=curcolor]
{
\newpath
\moveto(447.37123024,1045.27047188)
\curveto(447.39122167,1045.21046776)(447.40122166,1045.12546784)(447.40123024,1045.01547188)
\curveto(447.40122166,1044.90546806)(447.39122167,1044.82046815)(447.37123024,1044.76047188)
\lineto(447.37123024,1044.61047188)
\curveto(447.35122171,1044.53046844)(447.34122172,1044.45046852)(447.34123024,1044.37047188)
\curveto(447.35122171,1044.29046868)(447.34622172,1044.21046876)(447.32623024,1044.13047188)
\curveto(447.30622176,1044.06046891)(447.29122177,1043.99546897)(447.28123024,1043.93547188)
\curveto(447.27122179,1043.87546909)(447.2612218,1043.81046916)(447.25123024,1043.74047188)
\curveto(447.21122185,1043.63046934)(447.17622189,1043.51546945)(447.14623024,1043.39547188)
\curveto(447.11622195,1043.28546968)(447.07622199,1043.18046979)(447.02623024,1043.08047188)
\curveto(446.81622225,1042.60047037)(446.54122252,1042.21047076)(446.20123024,1041.91047188)
\curveto(445.8612232,1041.61047136)(445.45122361,1041.36047161)(444.97123024,1041.16047188)
\curveto(444.85122421,1041.11047186)(444.72622434,1041.07547189)(444.59623024,1041.05547188)
\curveto(444.47622459,1041.02547194)(444.35122471,1040.99547197)(444.22123024,1040.96547188)
\curveto(444.17122489,1040.94547202)(444.11622495,1040.93547203)(444.05623024,1040.93547188)
\curveto(443.99622507,1040.93547203)(443.94122512,1040.93047204)(443.89123024,1040.92047188)
\lineto(443.78623024,1040.92047188)
\curveto(443.75622531,1040.91047206)(443.72622534,1040.90547206)(443.69623024,1040.90547188)
\curveto(443.64622542,1040.89547207)(443.5662255,1040.89047208)(443.45623024,1040.89047188)
\curveto(443.34622572,1040.88047209)(443.2612258,1040.88547208)(443.20123024,1040.90547188)
\lineto(443.05123024,1040.90547188)
\curveto(443.00122606,1040.91547205)(442.94622612,1040.92047205)(442.88623024,1040.92047188)
\curveto(442.83622623,1040.91047206)(442.78622628,1040.91547205)(442.73623024,1040.93547188)
\curveto(442.69622637,1040.94547202)(442.65622641,1040.95047202)(442.61623024,1040.95047188)
\curveto(442.58622648,1040.95047202)(442.54622652,1040.95547201)(442.49623024,1040.96547188)
\curveto(442.39622667,1040.99547197)(442.29622677,1041.02047195)(442.19623024,1041.04047188)
\curveto(442.09622697,1041.06047191)(442.00122706,1041.09047188)(441.91123024,1041.13047188)
\curveto(441.79122727,1041.1704718)(441.67622739,1041.21047176)(441.56623024,1041.25047188)
\curveto(441.4662276,1041.29047168)(441.3612277,1041.34047163)(441.25123024,1041.40047188)
\curveto(440.90122816,1041.61047136)(440.60122846,1041.85547111)(440.35123024,1042.13547188)
\curveto(440.10122896,1042.41547055)(439.89122917,1042.75047022)(439.72123024,1043.14047188)
\curveto(439.67122939,1043.23046974)(439.63122943,1043.32546964)(439.60123024,1043.42547188)
\curveto(439.58122948,1043.52546944)(439.55622951,1043.63046934)(439.52623024,1043.74047188)
\curveto(439.50622956,1043.79046918)(439.49622957,1043.83546913)(439.49623024,1043.87547188)
\curveto(439.49622957,1043.91546905)(439.48622958,1043.96046901)(439.46623024,1044.01047188)
\curveto(439.44622962,1044.09046888)(439.43622963,1044.1704688)(439.43623024,1044.25047188)
\curveto(439.43622963,1044.34046863)(439.42622964,1044.42546854)(439.40623024,1044.50547188)
\curveto(439.39622967,1044.55546841)(439.39122967,1044.60046837)(439.39123024,1044.64047188)
\lineto(439.39123024,1044.77547188)
\curveto(439.37122969,1044.83546813)(439.3612297,1044.92046805)(439.36123024,1045.03047188)
\curveto(439.37122969,1045.14046783)(439.38622968,1045.22546774)(439.40623024,1045.28547188)
\lineto(439.40623024,1045.39047188)
\curveto(439.41622965,1045.44046753)(439.41622965,1045.49046748)(439.40623024,1045.54047188)
\curveto(439.40622966,1045.60046737)(439.41622965,1045.65546731)(439.43623024,1045.70547188)
\curveto(439.44622962,1045.75546721)(439.45122961,1045.80046717)(439.45123024,1045.84047188)
\curveto(439.45122961,1045.89046708)(439.4612296,1045.94046703)(439.48123024,1045.99047188)
\curveto(439.52122954,1046.12046685)(439.55622951,1046.24546672)(439.58623024,1046.36547188)
\curveto(439.61622945,1046.49546647)(439.65622941,1046.62046635)(439.70623024,1046.74047188)
\curveto(439.88622918,1047.15046582)(440.10122896,1047.49046548)(440.35123024,1047.76047188)
\curveto(440.60122846,1048.04046493)(440.90622816,1048.29546467)(441.26623024,1048.52547188)
\curveto(441.3662277,1048.57546439)(441.47122759,1048.62046435)(441.58123024,1048.66047188)
\curveto(441.69122737,1048.70046427)(441.80122726,1048.74546422)(441.91123024,1048.79547188)
\curveto(442.04122702,1048.84546412)(442.17622689,1048.88046409)(442.31623024,1048.90047188)
\curveto(442.45622661,1048.92046405)(442.60122646,1048.95046402)(442.75123024,1048.99047188)
\curveto(442.83122623,1049.00046397)(442.90622616,1049.00546396)(442.97623024,1049.00547188)
\curveto(443.04622602,1049.00546396)(443.11622595,1049.01046396)(443.18623024,1049.02047188)
\curveto(443.7662253,1049.03046394)(444.2662248,1048.970464)(444.68623024,1048.84047188)
\curveto(445.11622395,1048.71046426)(445.49622357,1048.53046444)(445.82623024,1048.30047188)
\curveto(445.93622313,1048.22046475)(446.04622302,1048.13046484)(446.15623024,1048.03047188)
\curveto(446.27622279,1047.94046503)(446.37622269,1047.84046513)(446.45623024,1047.73047188)
\curveto(446.53622253,1047.63046534)(446.60622246,1047.53046544)(446.66623024,1047.43047188)
\curveto(446.73622233,1047.33046564)(446.80622226,1047.22546574)(446.87623024,1047.11547188)
\curveto(446.94622212,1047.00546596)(447.00122206,1046.88546608)(447.04123024,1046.75547188)
\curveto(447.08122198,1046.63546633)(447.12622194,1046.50546646)(447.17623024,1046.36547188)
\curveto(447.20622186,1046.28546668)(447.23122183,1046.20046677)(447.25123024,1046.11047188)
\lineto(447.31123024,1045.84047188)
\curveto(447.32122174,1045.80046717)(447.32622174,1045.76046721)(447.32623024,1045.72047188)
\curveto(447.32622174,1045.68046729)(447.33122173,1045.64046733)(447.34123024,1045.60047188)
\curveto(447.3612217,1045.55046742)(447.3662217,1045.49546747)(447.35623024,1045.43547188)
\curveto(447.34622172,1045.37546759)(447.35122171,1045.32046765)(447.37123024,1045.27047188)
\moveto(445.27123024,1044.73047188)
\curveto(445.28122378,1044.78046819)(445.28622378,1044.85046812)(445.28623024,1044.94047188)
\curveto(445.28622378,1045.04046793)(445.28122378,1045.11546785)(445.27123024,1045.16547188)
\lineto(445.27123024,1045.28547188)
\curveto(445.25122381,1045.33546763)(445.24122382,1045.39046758)(445.24123024,1045.45047188)
\curveto(445.24122382,1045.51046746)(445.23622383,1045.5654674)(445.22623024,1045.61547188)
\curveto(445.22622384,1045.65546731)(445.22122384,1045.68546728)(445.21123024,1045.70547188)
\lineto(445.15123024,1045.94547188)
\curveto(445.14122392,1046.03546693)(445.12122394,1046.12046685)(445.09123024,1046.20047188)
\curveto(444.98122408,1046.46046651)(444.85122421,1046.68046629)(444.70123024,1046.86047188)
\curveto(444.55122451,1047.05046592)(444.35122471,1047.20046577)(444.10123024,1047.31047188)
\curveto(444.04122502,1047.33046564)(443.98122508,1047.34546562)(443.92123024,1047.35547188)
\curveto(443.8612252,1047.37546559)(443.79622527,1047.39546557)(443.72623024,1047.41547188)
\curveto(443.64622542,1047.43546553)(443.5612255,1047.44046553)(443.47123024,1047.43047188)
\lineto(443.20123024,1047.43047188)
\curveto(443.17122589,1047.41046556)(443.13622593,1047.40046557)(443.09623024,1047.40047188)
\curveto(443.05622601,1047.41046556)(443.02122604,1047.41046556)(442.99123024,1047.40047188)
\lineto(442.78123024,1047.34047188)
\curveto(442.72122634,1047.33046564)(442.6662264,1047.31046566)(442.61623024,1047.28047188)
\curveto(442.3662267,1047.1704658)(442.1612269,1047.01046596)(442.00123024,1046.80047188)
\curveto(441.85122721,1046.60046637)(441.73122733,1046.3654666)(441.64123024,1046.09547188)
\curveto(441.61122745,1045.99546697)(441.58622748,1045.89046708)(441.56623024,1045.78047188)
\curveto(441.55622751,1045.6704673)(441.54122752,1045.56046741)(441.52123024,1045.45047188)
\curveto(441.51122755,1045.40046757)(441.50622756,1045.35046762)(441.50623024,1045.30047188)
\lineto(441.50623024,1045.15047188)
\curveto(441.48622758,1045.08046789)(441.47622759,1044.97546799)(441.47623024,1044.83547188)
\curveto(441.48622758,1044.69546827)(441.50122756,1044.59046838)(441.52123024,1044.52047188)
\lineto(441.52123024,1044.38547188)
\curveto(441.54122752,1044.30546866)(441.55622751,1044.22546874)(441.56623024,1044.14547188)
\curveto(441.57622749,1044.07546889)(441.59122747,1044.00046897)(441.61123024,1043.92047188)
\curveto(441.71122735,1043.62046935)(441.81622725,1043.37546959)(441.92623024,1043.18547188)
\curveto(442.04622702,1043.00546996)(442.23122683,1042.84047013)(442.48123024,1042.69047188)
\curveto(442.55122651,1042.64047033)(442.62622644,1042.60047037)(442.70623024,1042.57047188)
\curveto(442.79622627,1042.54047043)(442.88622618,1042.51547045)(442.97623024,1042.49547188)
\curveto(443.01622605,1042.48547048)(443.05122601,1042.48047049)(443.08123024,1042.48047188)
\curveto(443.11122595,1042.49047048)(443.14622592,1042.49047048)(443.18623024,1042.48047188)
\lineto(443.30623024,1042.45047188)
\curveto(443.35622571,1042.45047052)(443.40122566,1042.45547051)(443.44123024,1042.46547188)
\lineto(443.56123024,1042.46547188)
\curveto(443.64122542,1042.48547048)(443.72122534,1042.50047047)(443.80123024,1042.51047188)
\curveto(443.88122518,1042.52047045)(443.95622511,1042.54047043)(444.02623024,1042.57047188)
\curveto(444.28622478,1042.6704703)(444.49622457,1042.80547016)(444.65623024,1042.97547188)
\curveto(444.81622425,1043.14546982)(444.95122411,1043.35546961)(445.06123024,1043.60547188)
\curveto(445.10122396,1043.70546926)(445.13122393,1043.80546916)(445.15123024,1043.90547188)
\curveto(445.17122389,1044.00546896)(445.19622387,1044.11046886)(445.22623024,1044.22047188)
\curveto(445.23622383,1044.26046871)(445.24122382,1044.29546867)(445.24123024,1044.32547188)
\curveto(445.24122382,1044.3654686)(445.24622382,1044.40546856)(445.25623024,1044.44547188)
\lineto(445.25623024,1044.58047188)
\curveto(445.25622381,1044.63046834)(445.2612238,1044.68046829)(445.27123024,1044.73047188)
}
}
{
\newrgbcolor{curcolor}{0 0 0}
\pscustom[linestyle=none,fillstyle=solid,fillcolor=curcolor]
{
\newpath
\moveto(451.74115212,1049.03547188)
\curveto(452.49114762,1049.05546391)(453.14114697,1048.970464)(453.69115212,1048.78047188)
\curveto(454.25114586,1048.60046437)(454.67614543,1048.28546468)(454.96615212,1047.83547188)
\curveto(455.03614507,1047.72546524)(455.09614501,1047.61046536)(455.14615212,1047.49047188)
\curveto(455.2061449,1047.38046559)(455.25614485,1047.25546571)(455.29615212,1047.11547188)
\curveto(455.31614479,1047.05546591)(455.32614478,1046.99046598)(455.32615212,1046.92047188)
\curveto(455.32614478,1046.85046612)(455.31614479,1046.79046618)(455.29615212,1046.74047188)
\curveto(455.25614485,1046.68046629)(455.20114491,1046.64046633)(455.13115212,1046.62047188)
\curveto(455.08114503,1046.60046637)(455.02114509,1046.59046638)(454.95115212,1046.59047188)
\lineto(454.74115212,1046.59047188)
\lineto(454.08115212,1046.59047188)
\curveto(454.0111461,1046.59046638)(453.94114617,1046.58546638)(453.87115212,1046.57547188)
\curveto(453.80114631,1046.57546639)(453.73614637,1046.58546638)(453.67615212,1046.60547188)
\curveto(453.57614653,1046.62546634)(453.50114661,1046.6654663)(453.45115212,1046.72547188)
\curveto(453.40114671,1046.78546618)(453.35614675,1046.84546612)(453.31615212,1046.90547188)
\lineto(453.19615212,1047.11547188)
\curveto(453.16614694,1047.19546577)(453.11614699,1047.26046571)(453.04615212,1047.31047188)
\curveto(452.94614716,1047.39046558)(452.84614726,1047.45046552)(452.74615212,1047.49047188)
\curveto(452.65614745,1047.53046544)(452.54114757,1047.5654654)(452.40115212,1047.59547188)
\curveto(452.33114778,1047.61546535)(452.22614788,1047.63046534)(452.08615212,1047.64047188)
\curveto(451.95614815,1047.65046532)(451.85614825,1047.64546532)(451.78615212,1047.62547188)
\lineto(451.68115212,1047.62547188)
\lineto(451.53115212,1047.59547188)
\curveto(451.49114862,1047.59546537)(451.44614866,1047.59046538)(451.39615212,1047.58047188)
\curveto(451.22614888,1047.53046544)(451.08614902,1047.46046551)(450.97615212,1047.37047188)
\curveto(450.87614923,1047.29046568)(450.8061493,1047.1654658)(450.76615212,1046.99547188)
\curveto(450.74614936,1046.92546604)(450.74614936,1046.86046611)(450.76615212,1046.80047188)
\curveto(450.78614932,1046.74046623)(450.8061493,1046.69046628)(450.82615212,1046.65047188)
\curveto(450.89614921,1046.53046644)(450.97614913,1046.43546653)(451.06615212,1046.36547188)
\curveto(451.16614894,1046.29546667)(451.28114883,1046.23546673)(451.41115212,1046.18547188)
\curveto(451.60114851,1046.10546686)(451.8061483,1046.03546693)(452.02615212,1045.97547188)
\lineto(452.71615212,1045.82547188)
\curveto(452.95614715,1045.78546718)(453.18614692,1045.73546723)(453.40615212,1045.67547188)
\curveto(453.63614647,1045.62546734)(453.85114626,1045.56046741)(454.05115212,1045.48047188)
\curveto(454.14114597,1045.44046753)(454.22614588,1045.40546756)(454.30615212,1045.37547188)
\curveto(454.39614571,1045.35546761)(454.48114563,1045.32046765)(454.56115212,1045.27047188)
\curveto(454.75114536,1045.15046782)(454.92114519,1045.02046795)(455.07115212,1044.88047188)
\curveto(455.23114488,1044.74046823)(455.35614475,1044.5654684)(455.44615212,1044.35547188)
\curveto(455.47614463,1044.28546868)(455.50114461,1044.21546875)(455.52115212,1044.14547188)
\curveto(455.54114457,1044.07546889)(455.56114455,1044.00046897)(455.58115212,1043.92047188)
\curveto(455.59114452,1043.86046911)(455.59614451,1043.7654692)(455.59615212,1043.63547188)
\curveto(455.6061445,1043.51546945)(455.6061445,1043.42046955)(455.59615212,1043.35047188)
\lineto(455.59615212,1043.27547188)
\curveto(455.57614453,1043.21546975)(455.56114455,1043.15546981)(455.55115212,1043.09547188)
\curveto(455.55114456,1043.04546992)(455.54614456,1042.99546997)(455.53615212,1042.94547188)
\curveto(455.46614464,1042.64547032)(455.35614475,1042.38047059)(455.20615212,1042.15047188)
\curveto(455.04614506,1041.91047106)(454.85114526,1041.71547125)(454.62115212,1041.56547188)
\curveto(454.39114572,1041.41547155)(454.13114598,1041.28547168)(453.84115212,1041.17547188)
\curveto(453.73114638,1041.12547184)(453.6111465,1041.09047188)(453.48115212,1041.07047188)
\curveto(453.36114675,1041.05047192)(453.24114687,1041.02547194)(453.12115212,1040.99547188)
\curveto(453.03114708,1040.97547199)(452.93614717,1040.965472)(452.83615212,1040.96547188)
\curveto(452.74614736,1040.95547201)(452.65614745,1040.94047203)(452.56615212,1040.92047188)
\lineto(452.29615212,1040.92047188)
\curveto(452.23614787,1040.90047207)(452.13114798,1040.89047208)(451.98115212,1040.89047188)
\curveto(451.84114827,1040.89047208)(451.74114837,1040.90047207)(451.68115212,1040.92047188)
\curveto(451.65114846,1040.92047205)(451.61614849,1040.92547204)(451.57615212,1040.93547188)
\lineto(451.47115212,1040.93547188)
\curveto(451.35114876,1040.95547201)(451.23114888,1040.970472)(451.11115212,1040.98047188)
\curveto(450.99114912,1040.99047198)(450.87614923,1041.01047196)(450.76615212,1041.04047188)
\curveto(450.37614973,1041.15047182)(450.03115008,1041.27547169)(449.73115212,1041.41547188)
\curveto(449.43115068,1041.5654714)(449.17615093,1041.78547118)(448.96615212,1042.07547188)
\curveto(448.82615128,1042.2654707)(448.7061514,1042.48547048)(448.60615212,1042.73547188)
\curveto(448.58615152,1042.79547017)(448.56615154,1042.87547009)(448.54615212,1042.97547188)
\curveto(448.52615158,1043.02546994)(448.5111516,1043.09546987)(448.50115212,1043.18547188)
\curveto(448.49115162,1043.27546969)(448.49615161,1043.35046962)(448.51615212,1043.41047188)
\curveto(448.54615156,1043.48046949)(448.59615151,1043.53046944)(448.66615212,1043.56047188)
\curveto(448.71615139,1043.58046939)(448.77615133,1043.59046938)(448.84615212,1043.59047188)
\lineto(449.07115212,1043.59047188)
\lineto(449.77615212,1043.59047188)
\lineto(450.01615212,1043.59047188)
\curveto(450.09615001,1043.59046938)(450.16614994,1043.58046939)(450.22615212,1043.56047188)
\curveto(450.33614977,1043.52046945)(450.4061497,1043.45546951)(450.43615212,1043.36547188)
\curveto(450.47614963,1043.27546969)(450.52114959,1043.18046979)(450.57115212,1043.08047188)
\curveto(450.59114952,1043.03046994)(450.62614948,1042.96547)(450.67615212,1042.88547188)
\curveto(450.73614937,1042.80547016)(450.78614932,1042.75547021)(450.82615212,1042.73547188)
\curveto(450.94614916,1042.63547033)(451.06114905,1042.55547041)(451.17115212,1042.49547188)
\curveto(451.28114883,1042.44547052)(451.42114869,1042.39547057)(451.59115212,1042.34547188)
\curveto(451.64114847,1042.32547064)(451.69114842,1042.31547065)(451.74115212,1042.31547188)
\curveto(451.79114832,1042.32547064)(451.84114827,1042.32547064)(451.89115212,1042.31547188)
\curveto(451.97114814,1042.29547067)(452.05614805,1042.28547068)(452.14615212,1042.28547188)
\curveto(452.24614786,1042.29547067)(452.33114778,1042.31047066)(452.40115212,1042.33047188)
\curveto(452.45114766,1042.34047063)(452.49614761,1042.34547062)(452.53615212,1042.34547188)
\curveto(452.58614752,1042.34547062)(452.63614747,1042.35547061)(452.68615212,1042.37547188)
\curveto(452.82614728,1042.42547054)(452.95114716,1042.48547048)(453.06115212,1042.55547188)
\curveto(453.18114693,1042.62547034)(453.27614683,1042.71547025)(453.34615212,1042.82547188)
\curveto(453.39614671,1042.90547006)(453.43614667,1043.03046994)(453.46615212,1043.20047188)
\curveto(453.48614662,1043.2704697)(453.48614662,1043.33546963)(453.46615212,1043.39547188)
\curveto(453.44614666,1043.45546951)(453.42614668,1043.50546946)(453.40615212,1043.54547188)
\curveto(453.33614677,1043.68546928)(453.24614686,1043.79046918)(453.13615212,1043.86047188)
\curveto(453.03614707,1043.93046904)(452.91614719,1043.99546897)(452.77615212,1044.05547188)
\curveto(452.58614752,1044.13546883)(452.38614772,1044.20046877)(452.17615212,1044.25047188)
\curveto(451.96614814,1044.30046867)(451.75614835,1044.35546861)(451.54615212,1044.41547188)
\curveto(451.46614864,1044.43546853)(451.38114873,1044.45046852)(451.29115212,1044.46047188)
\curveto(451.2111489,1044.4704685)(451.13114898,1044.48546848)(451.05115212,1044.50547188)
\curveto(450.73114938,1044.59546837)(450.42614968,1044.68046829)(450.13615212,1044.76047188)
\curveto(449.84615026,1044.85046812)(449.58115053,1044.98046799)(449.34115212,1045.15047188)
\curveto(449.06115105,1045.35046762)(448.85615125,1045.62046735)(448.72615212,1045.96047188)
\curveto(448.7061514,1046.03046694)(448.68615142,1046.12546684)(448.66615212,1046.24547188)
\curveto(448.64615146,1046.31546665)(448.63115148,1046.40046657)(448.62115212,1046.50047188)
\curveto(448.6111515,1046.60046637)(448.61615149,1046.69046628)(448.63615212,1046.77047188)
\curveto(448.65615145,1046.82046615)(448.66115145,1046.86046611)(448.65115212,1046.89047188)
\curveto(448.64115147,1046.93046604)(448.64615146,1046.97546599)(448.66615212,1047.02547188)
\curveto(448.68615142,1047.13546583)(448.7061514,1047.23546573)(448.72615212,1047.32547188)
\curveto(448.75615135,1047.42546554)(448.79115132,1047.52046545)(448.83115212,1047.61047188)
\curveto(448.96115115,1047.90046507)(449.14115097,1048.13546483)(449.37115212,1048.31547188)
\curveto(449.60115051,1048.49546447)(449.86115025,1048.64046433)(450.15115212,1048.75047188)
\curveto(450.26114985,1048.80046417)(450.37614973,1048.83546413)(450.49615212,1048.85547188)
\curveto(450.61614949,1048.88546408)(450.74114937,1048.91546405)(450.87115212,1048.94547188)
\curveto(450.93114918,1048.965464)(450.99114912,1048.97546399)(451.05115212,1048.97547188)
\lineto(451.23115212,1049.00547188)
\curveto(451.3111488,1049.01546395)(451.39614871,1049.02046395)(451.48615212,1049.02047188)
\curveto(451.57614853,1049.02046395)(451.66114845,1049.02546394)(451.74115212,1049.03547188)
}
}
{
\newrgbcolor{curcolor}{0 0 0}
\pscustom[linestyle=none,fillstyle=solid,fillcolor=curcolor]
{
}
}
{
\newrgbcolor{curcolor}{0 0 0}
\pscustom[linestyle=none,fillstyle=solid,fillcolor=curcolor]
{
\newpath
\moveto(468.54794899,1039.24047188)
\curveto(468.54794065,1039.08047389)(468.54294066,1038.92547404)(468.53294899,1038.77547188)
\curveto(468.53294067,1038.61547435)(468.48294072,1038.50547446)(468.38294899,1038.44547188)
\curveto(468.3029409,1038.39547457)(468.18794101,1038.37547459)(468.03794899,1038.38547188)
\lineto(467.61794899,1038.38547188)
\lineto(467.30294899,1038.38547188)
\curveto(467.19294201,1038.37547459)(467.08294212,1038.37547459)(466.97294899,1038.38547188)
\curveto(466.87294233,1038.38547458)(466.77794242,1038.40047457)(466.68794899,1038.43047188)
\curveto(466.60794259,1038.45047452)(466.54794265,1038.49047448)(466.50794899,1038.55047188)
\curveto(466.45794274,1038.63047434)(466.43294277,1038.74547422)(466.43294899,1038.89547188)
\curveto(466.44294276,1039.03547393)(466.44794275,1039.1654738)(466.44794899,1039.28547188)
\lineto(466.44794899,1040.92047188)
\lineto(466.44794899,1041.29547188)
\curveto(466.44794275,1041.43547153)(466.43294277,1041.54047143)(466.40294899,1041.61047188)
\curveto(466.38294282,1041.63047134)(466.36294284,1041.64547132)(466.34294899,1041.65547188)
\curveto(466.33294287,1041.67547129)(466.31794288,1041.69547127)(466.29794899,1041.71547188)
\curveto(466.20794299,1041.72547124)(466.13794306,1041.70547126)(466.08794899,1041.65547188)
\curveto(466.03794316,1041.61547135)(465.98294322,1041.57547139)(465.92294899,1041.53547188)
\curveto(465.83294337,1041.4654715)(465.73794346,1041.40047157)(465.63794899,1041.34047188)
\curveto(465.54794365,1041.28047169)(465.44794375,1041.22547174)(465.33794899,1041.17547188)
\curveto(465.15794404,1041.09547187)(464.95794424,1041.03547193)(464.73794899,1040.99547188)
\curveto(464.51794468,1040.94547202)(464.29294491,1040.92047205)(464.06294899,1040.92047188)
\curveto(463.83294537,1040.91047206)(463.6029456,1040.92547204)(463.37294899,1040.96547188)
\curveto(463.15294605,1041.00547196)(462.95294625,1041.0654719)(462.77294899,1041.14547188)
\curveto(462.32294688,1041.34547162)(461.95794724,1041.60047137)(461.67794899,1041.91047188)
\curveto(461.3979478,1042.23047074)(461.16294804,1042.62047035)(460.97294899,1043.08047188)
\curveto(460.92294828,1043.19046978)(460.88794831,1043.30046967)(460.86794899,1043.41047188)
\curveto(460.84794835,1043.53046944)(460.82294838,1043.64546932)(460.79294899,1043.75547188)
\curveto(460.77294843,1043.79546917)(460.76294844,1043.83046914)(460.76294899,1043.86047188)
\curveto(460.77294843,1043.90046907)(460.77294843,1043.94046903)(460.76294899,1043.98047188)
\curveto(460.74294846,1044.06046891)(460.72794847,1044.14546882)(460.71794899,1044.23547188)
\curveto(460.71794848,1044.33546863)(460.70794849,1044.43046854)(460.68794899,1044.52047188)
\lineto(460.68794899,1044.71547188)
\curveto(460.67794852,1044.7654682)(460.67294853,1044.82546814)(460.67294899,1044.89547188)
\curveto(460.67294853,1044.97546799)(460.67794852,1045.04046793)(460.68794899,1045.09047188)
\curveto(460.6979485,1045.14046783)(460.7029485,1045.18546778)(460.70294899,1045.22547188)
\lineto(460.70294899,1045.36047188)
\curveto(460.71294849,1045.41046756)(460.71294849,1045.46046751)(460.70294899,1045.51047188)
\curveto(460.7029485,1045.56046741)(460.71294849,1045.61046736)(460.73294899,1045.66047188)
\curveto(460.75294845,1045.75046722)(460.76794843,1045.84046713)(460.77794899,1045.93047188)
\curveto(460.78794841,1046.03046694)(460.8029484,1046.12546684)(460.82294899,1046.21547188)
\curveto(460.87294833,1046.38546658)(460.92294828,1046.54546642)(460.97294899,1046.69547188)
\curveto(461.03294817,1046.84546612)(461.09294811,1046.99046598)(461.15294899,1047.13047188)
\curveto(461.21294799,1047.2704657)(461.28794791,1047.40546556)(461.37794899,1047.53547188)
\curveto(461.46794773,1047.6654653)(461.55794764,1047.79046518)(461.64794899,1047.91047188)
\curveto(461.73794746,1048.02046495)(461.83794736,1048.12046485)(461.94794899,1048.21047188)
\curveto(461.97794722,1048.24046473)(461.9979472,1048.2654647)(462.00794899,1048.28547188)
\curveto(462.05794714,1048.31546465)(462.1029471,1048.34546462)(462.14294899,1048.37547188)
\curveto(462.18294702,1048.41546455)(462.22294698,1048.45046452)(462.26294899,1048.48047188)
\curveto(462.4029468,1048.58046439)(462.54794665,1048.66046431)(462.69794899,1048.72047188)
\curveto(462.85794634,1048.79046418)(463.02294618,1048.85546411)(463.19294899,1048.91547188)
\curveto(463.28294592,1048.94546402)(463.37294583,1048.965464)(463.46294899,1048.97547188)
\curveto(463.55294565,1048.98546398)(463.64294556,1049.00046397)(463.73294899,1049.02047188)
\curveto(463.76294544,1049.03046394)(463.81794538,1049.03046394)(463.89794899,1049.02047188)
\curveto(463.97794522,1049.01046396)(464.02794517,1049.01546395)(464.04794899,1049.03547188)
\curveto(464.36794483,1049.04546392)(464.66794453,1049.01546395)(464.94794899,1048.94547188)
\curveto(465.22794397,1048.88546408)(465.46794373,1048.79546417)(465.66794899,1048.67547188)
\lineto(465.84794899,1048.55547188)
\curveto(465.90794329,1048.51546445)(465.96294324,1048.47546449)(466.01294899,1048.43547188)
\curveto(466.07294313,1048.38546458)(466.12294308,1048.33546463)(466.16294899,1048.28547188)
\curveto(466.21294299,1048.24546472)(466.29294291,1048.22546474)(466.40294899,1048.22547188)
\lineto(466.44794899,1048.27047188)
\lineto(466.50794899,1048.33047188)
\curveto(466.53794266,1048.41046456)(466.55794264,1048.48546448)(466.56794899,1048.55547188)
\curveto(466.57794262,1048.63546433)(466.61794258,1048.70046427)(466.68794899,1048.75047188)
\curveto(466.73794246,1048.79046418)(466.80794239,1048.81046416)(466.89794899,1048.81047188)
\curveto(466.9979422,1048.82046415)(467.0979421,1048.82546414)(467.19794899,1048.82547188)
\lineto(467.91794899,1048.82547188)
\lineto(468.12794899,1048.82547188)
\curveto(468.197941,1048.82546414)(468.26294094,1048.81546415)(468.32294899,1048.79547188)
\curveto(468.39294081,1048.77546419)(468.44794075,1048.73046424)(468.48794899,1048.66047188)
\curveto(468.53794066,1048.59046438)(468.55794064,1048.49546447)(468.54794899,1048.37547188)
\lineto(468.54794899,1048.03047188)
\lineto(468.54794899,1039.24047188)
\moveto(466.50794899,1044.85047188)
\curveto(466.51794268,1044.8704681)(466.51794268,1044.89546807)(466.50794899,1044.92547188)
\lineto(466.50794899,1045.00047188)
\curveto(466.4979427,1045.10046787)(466.49294271,1045.19546777)(466.49294899,1045.28547188)
\curveto(466.49294271,1045.37546759)(466.48294272,1045.46046751)(466.46294899,1045.54047188)
\curveto(466.45294275,1045.5704674)(466.44794275,1045.59546737)(466.44794899,1045.61547188)
\curveto(466.45794274,1045.64546732)(466.45794274,1045.67546729)(466.44794899,1045.70547188)
\curveto(466.42794277,1045.78546718)(466.40794279,1045.85546711)(466.38794899,1045.91547188)
\curveto(466.37794282,1045.98546698)(466.36294284,1046.05546691)(466.34294899,1046.12547188)
\curveto(466.24294296,1046.41546655)(466.10794309,1046.6654663)(465.93794899,1046.87547188)
\curveto(465.76794343,1047.08546588)(465.54794365,1047.24546572)(465.27794899,1047.35547188)
\curveto(465.16794403,1047.40546556)(465.04794415,1047.43046554)(464.91794899,1047.43047188)
\curveto(464.7979444,1047.44046553)(464.66794453,1047.44546552)(464.52794899,1047.44547188)
\curveto(464.4979447,1047.42546554)(464.46294474,1047.41546555)(464.42294899,1047.41547188)
\curveto(464.38294482,1047.42546554)(464.34294486,1047.42546554)(464.30294899,1047.41547188)
\lineto(464.12294899,1047.35547188)
\curveto(464.06294514,1047.34546562)(464.00794519,1047.33046564)(463.95794899,1047.31047188)
\curveto(463.66794553,1047.18046579)(463.43794576,1046.99046598)(463.26794899,1046.74047188)
\curveto(463.10794609,1046.49046648)(462.98294622,1046.20046677)(462.89294899,1045.87047188)
\curveto(462.87294633,1045.79046718)(462.85794634,1045.71546725)(462.84794899,1045.64547188)
\curveto(462.84794635,1045.58546738)(462.83794636,1045.51546745)(462.81794899,1045.43547188)
\curveto(462.81794638,1045.3654676)(462.81294639,1045.31546765)(462.80294899,1045.28547188)
\curveto(462.79294641,1045.23546773)(462.78294642,1045.14546782)(462.77294899,1045.01547188)
\curveto(462.77294643,1044.89546807)(462.78294642,1044.81046816)(462.80294899,1044.76047188)
\lineto(462.80294899,1044.62547188)
\curveto(462.81294639,1044.58546838)(462.81794638,1044.54546842)(462.81794899,1044.50547188)
\curveto(462.81794638,1044.4654685)(462.82294638,1044.43046854)(462.83294899,1044.40047188)
\lineto(462.83294899,1044.32547188)
\curveto(462.84294636,1044.29546867)(462.84794635,1044.2704687)(462.84794899,1044.25047188)
\curveto(462.86794633,1044.1704688)(462.88294632,1044.09546887)(462.89294899,1044.02547188)
\curveto(462.9029463,1043.95546901)(462.92294628,1043.88546908)(462.95294899,1043.81547188)
\curveto(463.03294617,1043.5654694)(463.13794606,1043.35046962)(463.26794899,1043.17047188)
\curveto(463.3979458,1042.99046998)(463.56294564,1042.83547013)(463.76294899,1042.70547188)
\curveto(463.9029453,1042.62547034)(464.05794514,1042.5654704)(464.22794899,1042.52547188)
\curveto(464.25794494,1042.51547045)(464.28294492,1042.51047046)(464.30294899,1042.51047188)
\curveto(464.33294487,1042.51047046)(464.36794483,1042.50547046)(464.40794899,1042.49547188)
\curveto(464.43794476,1042.48547048)(464.48294472,1042.47547049)(464.54294899,1042.46547188)
\curveto(464.61294459,1042.4654705)(464.67294453,1042.4704705)(464.72294899,1042.48047188)
\curveto(464.74294446,1042.49047048)(464.76794443,1042.49047048)(464.79794899,1042.48047188)
\curveto(464.83794436,1042.48047049)(464.87294433,1042.48547048)(464.90294899,1042.49547188)
\curveto(464.97294423,1042.51547045)(465.03794416,1042.53047044)(465.09794899,1042.54047188)
\curveto(465.16794403,1042.55047042)(465.23794396,1042.5654704)(465.30794899,1042.58547188)
\curveto(465.56794363,1042.69547027)(465.77294343,1042.84047013)(465.92294899,1043.02047188)
\curveto(466.08294312,1043.20046977)(466.21794298,1043.42046955)(466.32794899,1043.68047188)
\curveto(466.35794284,1043.76046921)(466.38294282,1043.84546912)(466.40294899,1043.93547188)
\lineto(466.46294899,1044.20547188)
\lineto(466.46294899,1044.31047188)
\curveto(466.47294273,1044.34046863)(466.47794272,1044.37546859)(466.47794899,1044.41547188)
\curveto(466.4979427,1044.51546845)(466.50794269,1044.60046837)(466.50794899,1044.67047188)
\lineto(466.50794899,1044.85047188)
}
}
{
\newrgbcolor{curcolor}{0 0 0}
\pscustom[linestyle=none,fillstyle=solid,fillcolor=curcolor]
{
\newpath
\moveto(470.57787087,1048.81047188)
\lineto(471.70287087,1048.81047188)
\curveto(471.81286843,1048.81046416)(471.91286833,1048.80546416)(472.00287087,1048.79547188)
\curveto(472.09286815,1048.78546418)(472.15786809,1048.75046422)(472.19787087,1048.69047188)
\curveto(472.247868,1048.63046434)(472.27786797,1048.54546442)(472.28787087,1048.43547188)
\curveto(472.29786795,1048.33546463)(472.30286794,1048.23046474)(472.30287087,1048.12047188)
\lineto(472.30287087,1047.07047188)
\lineto(472.30287087,1044.83547188)
\curveto(472.30286794,1044.47546849)(472.31786793,1044.13546883)(472.34787087,1043.81547188)
\curveto(472.37786787,1043.49546947)(472.46786778,1043.23046974)(472.61787087,1043.02047188)
\curveto(472.75786749,1042.81047016)(472.98286726,1042.66047031)(473.29287087,1042.57047188)
\curveto(473.3428669,1042.56047041)(473.38286686,1042.55547041)(473.41287087,1042.55547188)
\curveto(473.45286679,1042.55547041)(473.49786675,1042.55047042)(473.54787087,1042.54047188)
\curveto(473.59786665,1042.53047044)(473.65286659,1042.52547044)(473.71287087,1042.52547188)
\curveto(473.77286647,1042.52547044)(473.81786643,1042.53047044)(473.84787087,1042.54047188)
\curveto(473.89786635,1042.56047041)(473.93786631,1042.5654704)(473.96787087,1042.55547188)
\curveto(474.00786624,1042.54547042)(474.0478662,1042.55047042)(474.08787087,1042.57047188)
\curveto(474.29786595,1042.62047035)(474.46286578,1042.68547028)(474.58287087,1042.76547188)
\curveto(474.76286548,1042.87547009)(474.90286534,1043.01546995)(475.00287087,1043.18547188)
\curveto(475.11286513,1043.3654696)(475.18786506,1043.56046941)(475.22787087,1043.77047188)
\curveto(475.27786497,1043.99046898)(475.30786494,1044.23046874)(475.31787087,1044.49047188)
\curveto(475.32786492,1044.76046821)(475.33286491,1045.04046793)(475.33287087,1045.33047188)
\lineto(475.33287087,1047.14547188)
\lineto(475.33287087,1048.12047188)
\lineto(475.33287087,1048.39047188)
\curveto(475.33286491,1048.49046448)(475.35286489,1048.5704644)(475.39287087,1048.63047188)
\curveto(475.4428648,1048.72046425)(475.51786473,1048.7704642)(475.61787087,1048.78047188)
\curveto(475.71786453,1048.80046417)(475.83786441,1048.81046416)(475.97787087,1048.81047188)
\lineto(476.77287087,1048.81047188)
\lineto(477.05787087,1048.81047188)
\curveto(477.1478631,1048.81046416)(477.22286302,1048.79046418)(477.28287087,1048.75047188)
\curveto(477.36286288,1048.70046427)(477.40786284,1048.62546434)(477.41787087,1048.52547188)
\curveto(477.42786282,1048.42546454)(477.43286281,1048.31046466)(477.43287087,1048.18047188)
\lineto(477.43287087,1047.04047188)
\lineto(477.43287087,1042.82547188)
\lineto(477.43287087,1041.76047188)
\lineto(477.43287087,1041.46047188)
\curveto(477.43286281,1041.36047161)(477.41286283,1041.28547168)(477.37287087,1041.23547188)
\curveto(477.32286292,1041.15547181)(477.247863,1041.11047186)(477.14787087,1041.10047188)
\curveto(477.0478632,1041.09047188)(476.9428633,1041.08547188)(476.83287087,1041.08547188)
\lineto(476.02287087,1041.08547188)
\curveto(475.91286433,1041.08547188)(475.81286443,1041.09047188)(475.72287087,1041.10047188)
\curveto(475.6428646,1041.11047186)(475.57786467,1041.15047182)(475.52787087,1041.22047188)
\curveto(475.50786474,1041.25047172)(475.48786476,1041.29547167)(475.46787087,1041.35547188)
\curveto(475.45786479,1041.41547155)(475.4428648,1041.47547149)(475.42287087,1041.53547188)
\curveto(475.41286483,1041.59547137)(475.39786485,1041.65047132)(475.37787087,1041.70047188)
\curveto(475.35786489,1041.75047122)(475.32786492,1041.78047119)(475.28787087,1041.79047188)
\curveto(475.26786498,1041.81047116)(475.242865,1041.81547115)(475.21287087,1041.80547188)
\curveto(475.18286506,1041.79547117)(475.15786509,1041.78547118)(475.13787087,1041.77547188)
\curveto(475.06786518,1041.73547123)(475.00786524,1041.69047128)(474.95787087,1041.64047188)
\curveto(474.90786534,1041.59047138)(474.85286539,1041.54547142)(474.79287087,1041.50547188)
\curveto(474.75286549,1041.47547149)(474.71286553,1041.44047153)(474.67287087,1041.40047188)
\curveto(474.6428656,1041.3704716)(474.60286564,1041.34047163)(474.55287087,1041.31047188)
\curveto(474.32286592,1041.1704718)(474.05286619,1041.06047191)(473.74287087,1040.98047188)
\curveto(473.67286657,1040.96047201)(473.60286664,1040.95047202)(473.53287087,1040.95047188)
\curveto(473.46286678,1040.94047203)(473.38786686,1040.92547204)(473.30787087,1040.90547188)
\curveto(473.26786698,1040.89547207)(473.22286702,1040.89547207)(473.17287087,1040.90547188)
\curveto(473.13286711,1040.90547206)(473.09286715,1040.90047207)(473.05287087,1040.89047188)
\curveto(473.02286722,1040.88047209)(472.95786729,1040.88047209)(472.85787087,1040.89047188)
\curveto(472.76786748,1040.89047208)(472.70786754,1040.89547207)(472.67787087,1040.90547188)
\curveto(472.62786762,1040.90547206)(472.57786767,1040.91047206)(472.52787087,1040.92047188)
\lineto(472.37787087,1040.92047188)
\curveto(472.25786799,1040.95047202)(472.1428681,1040.97547199)(472.03287087,1040.99547188)
\curveto(471.92286832,1041.01547195)(471.81286843,1041.04547192)(471.70287087,1041.08547188)
\curveto(471.65286859,1041.10547186)(471.60786864,1041.12047185)(471.56787087,1041.13047188)
\curveto(471.53786871,1041.15047182)(471.49786875,1041.1704718)(471.44787087,1041.19047188)
\curveto(471.09786915,1041.38047159)(470.81786943,1041.64547132)(470.60787087,1041.98547188)
\curveto(470.47786977,1042.19547077)(470.38286986,1042.44547052)(470.32287087,1042.73547188)
\curveto(470.26286998,1043.03546993)(470.22287002,1043.35046962)(470.20287087,1043.68047188)
\curveto(470.19287005,1044.02046895)(470.18787006,1044.3654686)(470.18787087,1044.71547188)
\curveto(470.19787005,1045.07546789)(470.20287004,1045.43046754)(470.20287087,1045.78047188)
\lineto(470.20287087,1047.82047188)
\curveto(470.20287004,1047.95046502)(470.19787005,1048.10046487)(470.18787087,1048.27047188)
\curveto(470.18787006,1048.45046452)(470.21287003,1048.58046439)(470.26287087,1048.66047188)
\curveto(470.29286995,1048.71046426)(470.35286989,1048.75546421)(470.44287087,1048.79547188)
\curveto(470.50286974,1048.79546417)(470.5478697,1048.80046417)(470.57787087,1048.81047188)
}
}
{
\newrgbcolor{curcolor}{0 0 0}
\pscustom[linestyle=none,fillstyle=solid,fillcolor=curcolor]
{
\newpath
\moveto(486.42912087,1045.03047188)
\curveto(486.4491127,1044.95046802)(486.4491127,1044.86046811)(486.42912087,1044.76047188)
\curveto(486.40911274,1044.66046831)(486.37411278,1044.59546837)(486.32412087,1044.56547188)
\curveto(486.27411288,1044.52546844)(486.19911295,1044.49546847)(486.09912087,1044.47547188)
\curveto(486.00911314,1044.4654685)(485.90411325,1044.45546851)(485.78412087,1044.44547188)
\lineto(485.43912087,1044.44547188)
\curveto(485.32911382,1044.45546851)(485.22911392,1044.46046851)(485.13912087,1044.46047188)
\lineto(481.47912087,1044.46047188)
\lineto(481.26912087,1044.46047188)
\curveto(481.20911794,1044.46046851)(481.154118,1044.45046852)(481.10412087,1044.43047188)
\curveto(481.02411813,1044.39046858)(480.97411818,1044.35046862)(480.95412087,1044.31047188)
\curveto(480.93411822,1044.29046868)(480.91411824,1044.25046872)(480.89412087,1044.19047188)
\curveto(480.87411828,1044.14046883)(480.86911828,1044.09046888)(480.87912087,1044.04047188)
\curveto(480.89911825,1043.98046899)(480.90911824,1043.92046905)(480.90912087,1043.86047188)
\curveto(480.91911823,1043.81046916)(480.93411822,1043.75546921)(480.95412087,1043.69547188)
\curveto(481.03411812,1043.45546951)(481.12911802,1043.25546971)(481.23912087,1043.09547188)
\curveto(481.35911779,1042.94547002)(481.51911763,1042.81047016)(481.71912087,1042.69047188)
\curveto(481.79911735,1042.64047033)(481.87911727,1042.60547036)(481.95912087,1042.58547188)
\curveto(482.0491171,1042.57547039)(482.13911701,1042.55547041)(482.22912087,1042.52547188)
\curveto(482.30911684,1042.50547046)(482.41911673,1042.49047048)(482.55912087,1042.48047188)
\curveto(482.69911645,1042.4704705)(482.81911633,1042.47547049)(482.91912087,1042.49547188)
\lineto(483.05412087,1042.49547188)
\curveto(483.154116,1042.51547045)(483.24411591,1042.53547043)(483.32412087,1042.55547188)
\curveto(483.41411574,1042.58547038)(483.49911565,1042.61547035)(483.57912087,1042.64547188)
\curveto(483.67911547,1042.69547027)(483.78911536,1042.76047021)(483.90912087,1042.84047188)
\curveto(484.03911511,1042.92047005)(484.13411502,1043.00046997)(484.19412087,1043.08047188)
\curveto(484.24411491,1043.15046982)(484.29411486,1043.21546975)(484.34412087,1043.27547188)
\curveto(484.40411475,1043.34546962)(484.47411468,1043.39546957)(484.55412087,1043.42547188)
\curveto(484.6541145,1043.47546949)(484.77911437,1043.49546947)(484.92912087,1043.48547188)
\lineto(485.36412087,1043.48547188)
\lineto(485.54412087,1043.48547188)
\curveto(485.61411354,1043.49546947)(485.67411348,1043.49046948)(485.72412087,1043.47047188)
\lineto(485.87412087,1043.47047188)
\curveto(485.97411318,1043.45046952)(486.04411311,1043.42546954)(486.08412087,1043.39547188)
\curveto(486.12411303,1043.37546959)(486.14411301,1043.33046964)(486.14412087,1043.26047188)
\curveto(486.154113,1043.19046978)(486.149113,1043.13046984)(486.12912087,1043.08047188)
\curveto(486.07911307,1042.94047003)(486.02411313,1042.81547015)(485.96412087,1042.70547188)
\curveto(485.90411325,1042.59547037)(485.83411332,1042.48547048)(485.75412087,1042.37547188)
\curveto(485.53411362,1042.04547092)(485.28411387,1041.78047119)(485.00412087,1041.58047188)
\curveto(484.72411443,1041.38047159)(484.37411478,1041.21047176)(483.95412087,1041.07047188)
\curveto(483.84411531,1041.03047194)(483.73411542,1041.00547196)(483.62412087,1040.99547188)
\curveto(483.51411564,1040.98547198)(483.39911575,1040.965472)(483.27912087,1040.93547188)
\curveto(483.23911591,1040.92547204)(483.19411596,1040.92547204)(483.14412087,1040.93547188)
\curveto(483.10411605,1040.93547203)(483.06411609,1040.93047204)(483.02412087,1040.92047188)
\lineto(482.85912087,1040.92047188)
\curveto(482.80911634,1040.90047207)(482.7491164,1040.89547207)(482.67912087,1040.90547188)
\curveto(482.61911653,1040.90547206)(482.56411659,1040.91047206)(482.51412087,1040.92047188)
\curveto(482.43411672,1040.93047204)(482.36411679,1040.93047204)(482.30412087,1040.92047188)
\curveto(482.24411691,1040.91047206)(482.17911697,1040.91547205)(482.10912087,1040.93547188)
\curveto(482.05911709,1040.95547201)(482.00411715,1040.965472)(481.94412087,1040.96547188)
\curveto(481.88411727,1040.965472)(481.82911732,1040.97547199)(481.77912087,1040.99547188)
\curveto(481.66911748,1041.01547195)(481.55911759,1041.04047193)(481.44912087,1041.07047188)
\curveto(481.33911781,1041.09047188)(481.23911791,1041.12547184)(481.14912087,1041.17547188)
\curveto(481.03911811,1041.21547175)(480.93411822,1041.25047172)(480.83412087,1041.28047188)
\curveto(480.74411841,1041.32047165)(480.65911849,1041.3654716)(480.57912087,1041.41547188)
\curveto(480.25911889,1041.61547135)(479.97411918,1041.84547112)(479.72412087,1042.10547188)
\curveto(479.47411968,1042.37547059)(479.26911988,1042.68547028)(479.10912087,1043.03547188)
\curveto(479.05912009,1043.14546982)(479.01912013,1043.25546971)(478.98912087,1043.36547188)
\curveto(478.95912019,1043.48546948)(478.91912023,1043.60546936)(478.86912087,1043.72547188)
\curveto(478.85912029,1043.7654692)(478.8541203,1043.80046917)(478.85412087,1043.83047188)
\curveto(478.8541203,1043.8704691)(478.8491203,1043.91046906)(478.83912087,1043.95047188)
\curveto(478.79912035,1044.0704689)(478.77412038,1044.20046877)(478.76412087,1044.34047188)
\lineto(478.73412087,1044.76047188)
\curveto(478.73412042,1044.81046816)(478.72912042,1044.8654681)(478.71912087,1044.92547188)
\curveto(478.71912043,1044.98546798)(478.72412043,1045.04046793)(478.73412087,1045.09047188)
\lineto(478.73412087,1045.27047188)
\lineto(478.77912087,1045.63047188)
\curveto(478.81912033,1045.80046717)(478.8541203,1045.965467)(478.88412087,1046.12547188)
\curveto(478.91412024,1046.28546668)(478.95912019,1046.43546653)(479.01912087,1046.57547188)
\curveto(479.4491197,1047.61546535)(480.17911897,1048.35046462)(481.20912087,1048.78047188)
\curveto(481.3491178,1048.84046413)(481.48911766,1048.88046409)(481.62912087,1048.90047188)
\curveto(481.77911737,1048.93046404)(481.93411722,1048.965464)(482.09412087,1049.00547188)
\curveto(482.17411698,1049.01546395)(482.2491169,1049.02046395)(482.31912087,1049.02047188)
\curveto(482.38911676,1049.02046395)(482.46411669,1049.02546394)(482.54412087,1049.03547188)
\curveto(483.0541161,1049.04546392)(483.48911566,1048.98546398)(483.84912087,1048.85547188)
\curveto(484.21911493,1048.73546423)(484.5491146,1048.57546439)(484.83912087,1048.37547188)
\curveto(484.92911422,1048.31546465)(485.01911413,1048.24546472)(485.10912087,1048.16547188)
\curveto(485.19911395,1048.09546487)(485.27911387,1048.02046495)(485.34912087,1047.94047188)
\curveto(485.37911377,1047.89046508)(485.41911373,1047.85046512)(485.46912087,1047.82047188)
\curveto(485.5491136,1047.71046526)(485.62411353,1047.59546537)(485.69412087,1047.47547188)
\curveto(485.76411339,1047.3654656)(485.83911331,1047.25046572)(485.91912087,1047.13047188)
\curveto(485.96911318,1047.04046593)(486.00911314,1046.94546602)(486.03912087,1046.84547188)
\curveto(486.07911307,1046.75546621)(486.11911303,1046.65546631)(486.15912087,1046.54547188)
\curveto(486.20911294,1046.41546655)(486.2491129,1046.28046669)(486.27912087,1046.14047188)
\curveto(486.30911284,1046.00046697)(486.34411281,1045.86046711)(486.38412087,1045.72047188)
\curveto(486.40411275,1045.64046733)(486.40911274,1045.55046742)(486.39912087,1045.45047188)
\curveto(486.39911275,1045.36046761)(486.40911274,1045.27546769)(486.42912087,1045.19547188)
\lineto(486.42912087,1045.03047188)
\moveto(484.17912087,1045.91547188)
\curveto(484.2491149,1046.01546695)(484.2541149,1046.13546683)(484.19412087,1046.27547188)
\curveto(484.14411501,1046.42546654)(484.10411505,1046.53546643)(484.07412087,1046.60547188)
\curveto(483.93411522,1046.87546609)(483.7491154,1047.08046589)(483.51912087,1047.22047188)
\curveto(483.28911586,1047.3704656)(482.96911618,1047.45046552)(482.55912087,1047.46047188)
\curveto(482.52911662,1047.44046553)(482.49411666,1047.43546553)(482.45412087,1047.44547188)
\curveto(482.41411674,1047.45546551)(482.37911677,1047.45546551)(482.34912087,1047.44547188)
\curveto(482.29911685,1047.42546554)(482.24411691,1047.41046556)(482.18412087,1047.40047188)
\curveto(482.12411703,1047.40046557)(482.06911708,1047.39046558)(482.01912087,1047.37047188)
\curveto(481.57911757,1047.23046574)(481.2541179,1046.95546601)(481.04412087,1046.54547188)
\curveto(481.02411813,1046.50546646)(480.99911815,1046.45046652)(480.96912087,1046.38047188)
\curveto(480.9491182,1046.32046665)(480.93411822,1046.25546671)(480.92412087,1046.18547188)
\curveto(480.91411824,1046.12546684)(480.91411824,1046.0654669)(480.92412087,1046.00547188)
\curveto(480.94411821,1045.94546702)(480.97911817,1045.89546707)(481.02912087,1045.85547188)
\curveto(481.10911804,1045.80546716)(481.21911793,1045.78046719)(481.35912087,1045.78047188)
\lineto(481.76412087,1045.78047188)
\lineto(483.42912087,1045.78047188)
\lineto(483.86412087,1045.78047188)
\curveto(484.02411513,1045.79046718)(484.12911502,1045.83546713)(484.17912087,1045.91547188)
}
}
{
\newrgbcolor{curcolor}{0 0 0}
\pscustom[linestyle=none,fillstyle=solid,fillcolor=curcolor]
{
}
}
{
\newrgbcolor{curcolor}{0 0 0}
\pscustom[linestyle=none,fillstyle=solid,fillcolor=curcolor]
{
\newpath
\moveto(493.93755837,1051.67547188)
\curveto(494.00755542,1051.59546137)(494.04255538,1051.47546149)(494.04255837,1051.31547188)
\lineto(494.04255837,1050.85047188)
\lineto(494.04255837,1050.44547188)
\curveto(494.04255538,1050.30546266)(494.00755542,1050.21046276)(493.93755837,1050.16047188)
\curveto(493.87755555,1050.11046286)(493.79755563,1050.08046289)(493.69755837,1050.07047188)
\curveto(493.60755582,1050.06046291)(493.50755592,1050.05546291)(493.39755837,1050.05547188)
\lineto(492.55755837,1050.05547188)
\curveto(492.44755698,1050.05546291)(492.34755708,1050.06046291)(492.25755837,1050.07047188)
\curveto(492.17755725,1050.08046289)(492.10755732,1050.11046286)(492.04755837,1050.16047188)
\curveto(492.00755742,1050.19046278)(491.97755745,1050.24546272)(491.95755837,1050.32547188)
\curveto(491.94755748,1050.41546255)(491.93755749,1050.51046246)(491.92755837,1050.61047188)
\lineto(491.92755837,1050.94047188)
\curveto(491.93755749,1051.05046192)(491.94255748,1051.14546182)(491.94255837,1051.22547188)
\lineto(491.94255837,1051.43547188)
\curveto(491.95255747,1051.50546146)(491.97255745,1051.5654614)(492.00255837,1051.61547188)
\curveto(492.0225574,1051.65546131)(492.04755738,1051.68546128)(492.07755837,1051.70547188)
\lineto(492.19755837,1051.76547188)
\curveto(492.21755721,1051.7654612)(492.24255718,1051.7654612)(492.27255837,1051.76547188)
\curveto(492.30255712,1051.77546119)(492.3275571,1051.78046119)(492.34755837,1051.78047188)
\lineto(493.44255837,1051.78047188)
\curveto(493.54255588,1051.78046119)(493.63755579,1051.77546119)(493.72755837,1051.76547188)
\curveto(493.81755561,1051.75546121)(493.88755554,1051.72546124)(493.93755837,1051.67547188)
\moveto(494.04255837,1041.91047188)
\curveto(494.04255538,1041.71047126)(494.03755539,1041.54047143)(494.02755837,1041.40047188)
\curveto(494.01755541,1041.26047171)(493.9275555,1041.1654718)(493.75755837,1041.11547188)
\curveto(493.69755573,1041.09547187)(493.63255579,1041.08547188)(493.56255837,1041.08547188)
\curveto(493.49255593,1041.09547187)(493.41755601,1041.10047187)(493.33755837,1041.10047188)
\lineto(492.49755837,1041.10047188)
\curveto(492.40755702,1041.10047187)(492.31755711,1041.10547186)(492.22755837,1041.11547188)
\curveto(492.14755728,1041.12547184)(492.08755734,1041.15547181)(492.04755837,1041.20547188)
\curveto(491.98755744,1041.27547169)(491.95255747,1041.36047161)(491.94255837,1041.46047188)
\lineto(491.94255837,1041.80547188)
\lineto(491.94255837,1048.13547188)
\lineto(491.94255837,1048.43547188)
\curveto(491.94255748,1048.53546443)(491.96255746,1048.61546435)(492.00255837,1048.67547188)
\curveto(492.06255736,1048.74546422)(492.14755728,1048.79046418)(492.25755837,1048.81047188)
\curveto(492.27755715,1048.82046415)(492.30255712,1048.82046415)(492.33255837,1048.81047188)
\curveto(492.37255705,1048.81046416)(492.40255702,1048.81546415)(492.42255837,1048.82547188)
\lineto(493.17255837,1048.82547188)
\lineto(493.36755837,1048.82547188)
\curveto(493.44755598,1048.83546413)(493.51255591,1048.83546413)(493.56255837,1048.82547188)
\lineto(493.68255837,1048.82547188)
\curveto(493.74255568,1048.80546416)(493.79755563,1048.79046418)(493.84755837,1048.78047188)
\curveto(493.89755553,1048.7704642)(493.93755549,1048.74046423)(493.96755837,1048.69047188)
\curveto(494.00755542,1048.64046433)(494.0275554,1048.5704644)(494.02755837,1048.48047188)
\curveto(494.03755539,1048.39046458)(494.04255538,1048.29546467)(494.04255837,1048.19547188)
\lineto(494.04255837,1041.91047188)
}
}
{
\newrgbcolor{curcolor}{0 0 0}
\pscustom[linestyle=none,fillstyle=solid,fillcolor=curcolor]
{
\newpath
\moveto(500.12974587,1049.02047188)
\curveto(500.72974006,1049.04046393)(501.22973956,1048.95546401)(501.62974587,1048.76547188)
\curveto(502.02973876,1048.57546439)(502.34473845,1048.29546467)(502.57474587,1047.92547188)
\curveto(502.64473815,1047.81546515)(502.69973809,1047.69546527)(502.73974587,1047.56547188)
\curveto(502.77973801,1047.44546552)(502.81973797,1047.32046565)(502.85974587,1047.19047188)
\curveto(502.87973791,1047.11046586)(502.8897379,1047.03546593)(502.88974587,1046.96547188)
\curveto(502.89973789,1046.89546607)(502.91473788,1046.82546614)(502.93474587,1046.75547188)
\curveto(502.93473786,1046.69546627)(502.93973785,1046.65546631)(502.94974587,1046.63547188)
\curveto(502.96973782,1046.49546647)(502.97973781,1046.35046662)(502.97974587,1046.20047188)
\lineto(502.97974587,1045.76547188)
\lineto(502.97974587,1044.43047188)
\lineto(502.97974587,1042.00047188)
\curveto(502.97973781,1041.81047116)(502.97473782,1041.62547134)(502.96474587,1041.44547188)
\curveto(502.96473783,1041.27547169)(502.8947379,1041.1654718)(502.75474587,1041.11547188)
\curveto(502.6947381,1041.09547187)(502.62473817,1041.08547188)(502.54474587,1041.08547188)
\lineto(502.30474587,1041.08547188)
\lineto(501.49474587,1041.08547188)
\curveto(501.37473942,1041.08547188)(501.26473953,1041.09047188)(501.16474587,1041.10047188)
\curveto(501.07473972,1041.12047185)(501.00473979,1041.1654718)(500.95474587,1041.23547188)
\curveto(500.91473988,1041.29547167)(500.8897399,1041.3704716)(500.87974587,1041.46047188)
\lineto(500.87974587,1041.77547188)
\lineto(500.87974587,1042.82547188)
\lineto(500.87974587,1045.06047188)
\curveto(500.87973991,1045.43046754)(500.86473993,1045.7704672)(500.83474587,1046.08047188)
\curveto(500.80473999,1046.40046657)(500.71474008,1046.6704663)(500.56474587,1046.89047188)
\curveto(500.42474037,1047.09046588)(500.21974057,1047.23046574)(499.94974587,1047.31047188)
\curveto(499.89974089,1047.33046564)(499.84474095,1047.34046563)(499.78474587,1047.34047188)
\curveto(499.73474106,1047.34046563)(499.67974111,1047.35046562)(499.61974587,1047.37047188)
\curveto(499.56974122,1047.38046559)(499.50474129,1047.38046559)(499.42474587,1047.37047188)
\curveto(499.35474144,1047.3704656)(499.29974149,1047.3654656)(499.25974587,1047.35547188)
\curveto(499.21974157,1047.34546562)(499.18474161,1047.34046563)(499.15474587,1047.34047188)
\curveto(499.12474167,1047.34046563)(499.0947417,1047.33546563)(499.06474587,1047.32547188)
\curveto(498.83474196,1047.2654657)(498.64974214,1047.18546578)(498.50974587,1047.08547188)
\curveto(498.1897426,1046.85546611)(497.99974279,1046.52046645)(497.93974587,1046.08047188)
\curveto(497.87974291,1045.64046733)(497.84974294,1045.14546782)(497.84974587,1044.59547188)
\lineto(497.84974587,1042.72047188)
\lineto(497.84974587,1041.80547188)
\lineto(497.84974587,1041.53547188)
\curveto(497.84974294,1041.44547152)(497.83474296,1041.3704716)(497.80474587,1041.31047188)
\curveto(497.75474304,1041.20047177)(497.67474312,1041.13547183)(497.56474587,1041.11547188)
\curveto(497.45474334,1041.09547187)(497.31974347,1041.08547188)(497.15974587,1041.08547188)
\lineto(496.40974587,1041.08547188)
\curveto(496.29974449,1041.08547188)(496.1897446,1041.09047188)(496.07974587,1041.10047188)
\curveto(495.96974482,1041.11047186)(495.8897449,1041.14547182)(495.83974587,1041.20547188)
\curveto(495.76974502,1041.29547167)(495.73474506,1041.42547154)(495.73474587,1041.59547188)
\curveto(495.74474505,1041.7654712)(495.74974504,1041.92547104)(495.74974587,1042.07547188)
\lineto(495.74974587,1044.11547188)
\lineto(495.74974587,1047.41547188)
\lineto(495.74974587,1048.18047188)
\lineto(495.74974587,1048.48047188)
\curveto(495.75974503,1048.5704644)(495.789745,1048.64546432)(495.83974587,1048.70547188)
\curveto(495.85974493,1048.73546423)(495.8897449,1048.75546421)(495.92974587,1048.76547188)
\curveto(495.97974481,1048.78546418)(496.02974476,1048.80046417)(496.07974587,1048.81047188)
\lineto(496.15474587,1048.81047188)
\curveto(496.20474459,1048.82046415)(496.25474454,1048.82546414)(496.30474587,1048.82547188)
\lineto(496.46974587,1048.82547188)
\lineto(497.09974587,1048.82547188)
\curveto(497.17974361,1048.82546414)(497.25474354,1048.82046415)(497.32474587,1048.81047188)
\curveto(497.40474339,1048.81046416)(497.47474332,1048.80046417)(497.53474587,1048.78047188)
\curveto(497.60474319,1048.75046422)(497.64974314,1048.70546426)(497.66974587,1048.64547188)
\curveto(497.69974309,1048.58546438)(497.72474307,1048.51546445)(497.74474587,1048.43547188)
\curveto(497.75474304,1048.39546457)(497.75474304,1048.36046461)(497.74474587,1048.33047188)
\curveto(497.74474305,1048.30046467)(497.75474304,1048.2704647)(497.77474587,1048.24047188)
\curveto(497.794743,1048.19046478)(497.80974298,1048.16046481)(497.81974587,1048.15047188)
\curveto(497.83974295,1048.14046483)(497.86474293,1048.12546484)(497.89474587,1048.10547188)
\curveto(498.00474279,1048.09546487)(498.0947427,1048.13046484)(498.16474587,1048.21047188)
\curveto(498.23474256,1048.30046467)(498.30974248,1048.3704646)(498.38974587,1048.42047188)
\curveto(498.65974213,1048.62046435)(498.95974183,1048.78046419)(499.28974587,1048.90047188)
\curveto(499.37974141,1048.93046404)(499.46974132,1048.95046402)(499.55974587,1048.96047188)
\curveto(499.65974113,1048.970464)(499.76474103,1048.98546398)(499.87474587,1049.00547188)
\curveto(499.90474089,1049.01546395)(499.94974084,1049.01546395)(500.00974587,1049.00547188)
\curveto(500.06974072,1049.00546396)(500.10974068,1049.01046396)(500.12974587,1049.02047188)
}
}
{
\newrgbcolor{curcolor}{0 0 0}
\pscustom[linestyle=none,fillstyle=solid,fillcolor=curcolor]
{
\newpath
\moveto(511.96099587,1048.73547188)
\curveto(512.03098767,1048.68546428)(512.06598763,1048.60046437)(512.06599587,1048.48047188)
\curveto(512.07598762,1048.3704646)(512.08098762,1048.25546471)(512.08099587,1048.13547188)
\lineto(512.08099587,1041.73047188)
\curveto(512.08098762,1041.65047132)(512.07598762,1041.5704714)(512.06599587,1041.49047188)
\lineto(512.06599587,1041.26547188)
\curveto(512.05598764,1041.18547178)(512.04598765,1041.11547185)(512.03599587,1041.05547188)
\curveto(512.03598766,1040.98547198)(512.03098767,1040.91047206)(512.02099587,1040.83047188)
\curveto(511.98098772,1040.69047228)(511.94598775,1040.56047241)(511.91599587,1040.44047188)
\curveto(511.8959878,1040.31047266)(511.86098784,1040.19047278)(511.81099587,1040.08047188)
\curveto(511.64098806,1039.70047327)(511.42098828,1039.38547358)(511.15099587,1039.13547188)
\curveto(510.89098881,1038.88547408)(510.57098913,1038.68047429)(510.19099587,1038.52047188)
\curveto(510.08098962,1038.4704745)(509.97098973,1038.43047454)(509.86099587,1038.40047188)
\curveto(509.75098995,1038.3704746)(509.63599006,1038.34047463)(509.51599587,1038.31047188)
\curveto(509.40599029,1038.28047469)(509.2959904,1038.26047471)(509.18599587,1038.25047188)
\curveto(509.07599062,1038.24047473)(508.96599073,1038.22547474)(508.85599587,1038.20547188)
\lineto(508.73599587,1038.20547188)
\curveto(508.695991,1038.19547477)(508.65099105,1038.19047478)(508.60099587,1038.19047188)
\curveto(508.56099114,1038.18047479)(508.51599118,1038.18047479)(508.46599587,1038.19047188)
\curveto(508.41599128,1038.19047478)(508.36599133,1038.18547478)(508.31599587,1038.17547188)
\curveto(508.26599143,1038.1654748)(508.2009915,1038.16047481)(508.12099587,1038.16047188)
\curveto(508.04099166,1038.16047481)(507.97599172,1038.1654748)(507.92599587,1038.17547188)
\lineto(507.79099587,1038.17547188)
\curveto(507.75099195,1038.17547479)(507.71099199,1038.18047479)(507.67099587,1038.19047188)
\curveto(507.59099211,1038.21047476)(507.50599219,1038.22047475)(507.41599587,1038.22047188)
\curveto(507.33599236,1038.22047475)(507.26099244,1038.23047474)(507.19099587,1038.25047188)
\curveto(507.17099253,1038.26047471)(507.14599255,1038.2654747)(507.11599587,1038.26547188)
\curveto(507.08599261,1038.2654747)(507.06099264,1038.2704747)(507.04099587,1038.28047188)
\curveto(506.94099276,1038.30047467)(506.84099286,1038.32547464)(506.74099587,1038.35547188)
\curveto(506.65099305,1038.37547459)(506.56099314,1038.40547456)(506.47099587,1038.44547188)
\curveto(506.09099361,1038.60547436)(505.75099395,1038.81047416)(505.45099587,1039.06047188)
\curveto(505.15099455,1039.30047367)(504.93099477,1039.62547334)(504.79099587,1040.03547188)
\curveto(504.77099493,1040.0654729)(504.76099494,1040.09547287)(504.76099587,1040.12547188)
\curveto(504.76099494,1040.15547281)(504.75599494,1040.18047279)(504.74599587,1040.20047188)
\curveto(504.71599498,1040.33047264)(504.72599497,1040.43047254)(504.77599587,1040.50047188)
\curveto(504.83599486,1040.56047241)(504.91599478,1040.60047237)(505.01599587,1040.62047188)
\curveto(505.11599458,1040.64047233)(505.22599447,1040.65047232)(505.34599587,1040.65047188)
\curveto(505.47599422,1040.64047233)(505.5959941,1040.63547233)(505.70599587,1040.63547188)
\lineto(506.21599587,1040.63547188)
\lineto(506.33599587,1040.63547188)
\curveto(506.37599332,1040.62547234)(506.42099328,1040.62047235)(506.47099587,1040.62047188)
\curveto(506.63099307,1040.58047239)(506.73099297,1040.53047244)(506.77099587,1040.47047188)
\curveto(506.81099289,1040.40047257)(506.87099283,1040.31047266)(506.95099587,1040.20047188)
\curveto(506.98099272,1040.16047281)(507.02599267,1040.11047286)(507.08599587,1040.05047188)
\curveto(507.0959926,1040.03047294)(507.10599259,1040.01547295)(507.11599587,1040.00547188)
\curveto(507.12599257,1039.99547297)(507.13599256,1039.98047299)(507.14599587,1039.96047188)
\curveto(507.22599247,1039.90047307)(507.31099239,1039.84547312)(507.40099587,1039.79547188)
\curveto(507.49099221,1039.74547322)(507.59099211,1039.70047327)(507.70099587,1039.66047188)
\curveto(507.77099193,1039.64047333)(507.84099186,1039.63047334)(507.91099587,1039.63047188)
\curveto(507.98099172,1039.62047335)(508.05599164,1039.60547336)(508.13599587,1039.58547188)
\lineto(508.30099587,1039.58547188)
\curveto(508.37099133,1039.5654734)(508.46099124,1039.5654734)(508.57099587,1039.58547188)
\curveto(508.68099102,1039.59547337)(508.76599093,1039.61047336)(508.82599587,1039.63047188)
\curveto(508.87599082,1039.65047332)(508.91599078,1039.66047331)(508.94599587,1039.66047188)
\curveto(508.98599071,1039.66047331)(509.02599067,1039.6704733)(509.06599587,1039.69047188)
\curveto(509.27599042,1039.78047319)(509.45099025,1039.90047307)(509.59099587,1040.05047188)
\curveto(509.73098997,1040.20047277)(509.84598985,1040.37547259)(509.93599587,1040.57547188)
\curveto(509.95598974,1040.63547233)(509.97098973,1040.69547227)(509.98099587,1040.75547188)
\curveto(509.99098971,1040.81547215)(510.00598969,1040.88047209)(510.02599587,1040.95047188)
\curveto(510.04598965,1041.04047193)(510.05598964,1041.13547183)(510.05599587,1041.23547188)
\curveto(510.06598963,1041.34547162)(510.07098963,1041.45547151)(510.07099587,1041.56547188)
\lineto(510.07099587,1041.68547188)
\curveto(510.08098962,1041.72547124)(510.08098962,1041.76047121)(510.07099587,1041.79047188)
\curveto(510.05098965,1041.84047113)(510.04098966,1041.88547108)(510.04099587,1041.92547188)
\curveto(510.05098965,1041.965471)(510.04598965,1042.00547096)(510.02599587,1042.04547188)
\curveto(510.01598968,1042.0654709)(510.0009897,1042.08047089)(509.98099587,1042.09047188)
\lineto(509.93599587,1042.13547188)
\curveto(509.84598985,1042.14547082)(509.77098993,1042.12547084)(509.71099587,1042.07547188)
\curveto(509.66099004,1042.02547094)(509.61099009,1041.98047099)(509.56099587,1041.94047188)
\curveto(509.47099023,1041.8704711)(509.38099032,1041.80547116)(509.29099587,1041.74547188)
\curveto(509.2009905,1041.68547128)(509.1009906,1041.63047134)(508.99099587,1041.58047188)
\curveto(508.88099082,1041.53047144)(508.77099093,1041.49047148)(508.66099587,1041.46047188)
\curveto(508.55099115,1041.43047154)(508.43599126,1041.40047157)(508.31599587,1041.37047188)
\lineto(508.13599587,1041.34047188)
\curveto(508.08599161,1041.34047163)(508.03599166,1041.33547163)(507.98599587,1041.32547188)
\curveto(507.93599176,1041.31547165)(507.85599184,1041.31047166)(507.74599587,1041.31047188)
\curveto(507.63599206,1041.31047166)(507.55599214,1041.31547165)(507.50599587,1041.32547188)
\lineto(507.38599587,1041.32547188)
\curveto(507.35599234,1041.33547163)(507.32099238,1041.34047163)(507.28099587,1041.34047188)
\curveto(507.25099245,1041.34047163)(507.21599248,1041.34547162)(507.17599587,1041.35547188)
\curveto(507.03599266,1041.38547158)(506.9009928,1041.41047156)(506.77099587,1041.43047188)
\curveto(506.64099306,1041.46047151)(506.52099318,1041.50047147)(506.41099587,1041.55047188)
\curveto(505.98099372,1041.72047125)(505.63099407,1041.95547101)(505.36099587,1042.25547188)
\curveto(505.1009946,1042.5654704)(504.88099482,1042.93547003)(504.70099587,1043.36547188)
\curveto(504.65099505,1043.47546949)(504.61599508,1043.59046938)(504.59599587,1043.71047188)
\curveto(504.57599512,1043.83046914)(504.54599515,1043.95046902)(504.50599587,1044.07047188)
\curveto(504.50599519,1044.12046885)(504.5009952,1044.16046881)(504.49099587,1044.19047188)
\curveto(504.47099523,1044.2704687)(504.46099524,1044.35546861)(504.46099587,1044.44547188)
\curveto(504.46099524,1044.54546842)(504.45099525,1044.63546833)(504.43099587,1044.71547188)
\curveto(504.42099528,1044.7654682)(504.41599528,1044.81046816)(504.41599587,1044.85047188)
\lineto(504.41599587,1045.00047188)
\curveto(504.40599529,1045.05046792)(504.4009953,1045.11046786)(504.40099587,1045.18047188)
\curveto(504.4009953,1045.26046771)(504.40599529,1045.32546764)(504.41599587,1045.37547188)
\lineto(504.41599587,1045.52547188)
\curveto(504.42599527,1045.5654674)(504.42599527,1045.60546736)(504.41599587,1045.64547188)
\curveto(504.41599528,1045.68546728)(504.42599527,1045.72546724)(504.44599587,1045.76547188)
\curveto(504.46599523,1045.8654671)(504.48099522,1045.96046701)(504.49099587,1046.05047188)
\curveto(504.5009952,1046.15046682)(504.51599518,1046.25046672)(504.53599587,1046.35047188)
\curveto(504.5959951,1046.55046642)(504.65599504,1046.74046623)(504.71599587,1046.92047188)
\curveto(504.78599491,1047.10046587)(504.87099483,1047.2704657)(504.97099587,1047.43047188)
\curveto(505.02099468,1047.53046544)(505.07599462,1047.62046535)(505.13599587,1047.70047188)
\lineto(505.34599587,1047.97047188)
\curveto(505.37599432,1048.02046495)(505.41599428,1048.0704649)(505.46599587,1048.12047188)
\curveto(505.52599417,1048.1704648)(505.58099412,1048.21546475)(505.63099587,1048.25547188)
\lineto(505.72099587,1048.34547188)
\curveto(505.77099393,1048.38546458)(505.82099388,1048.42046455)(505.87099587,1048.45047188)
\curveto(505.92099378,1048.49046448)(505.97099373,1048.52546444)(506.02099587,1048.55547188)
\curveto(506.15099355,1048.63546433)(506.28599341,1048.70546426)(506.42599587,1048.76547188)
\curveto(506.56599313,1048.82546414)(506.72099298,1048.88046409)(506.89099587,1048.93047188)
\curveto(506.97099273,1048.96046401)(507.05099265,1048.97546399)(507.13099587,1048.97547188)
\curveto(507.22099248,1048.98546398)(507.30599239,1049.00046397)(507.38599587,1049.02047188)
\curveto(507.42599227,1049.03046394)(507.48099222,1049.03046394)(507.55099587,1049.02047188)
\curveto(507.62099208,1049.01046396)(507.66599203,1049.01546395)(507.68599587,1049.03547188)
\curveto(508.00599169,1049.04546392)(508.29099141,1049.01546395)(508.54099587,1048.94547188)
\curveto(508.8009909,1048.87546409)(509.03099067,1048.77546419)(509.23099587,1048.64547188)
\curveto(509.26099044,1048.62546434)(509.29099041,1048.60046437)(509.32099587,1048.57047188)
\curveto(509.35099035,1048.55046442)(509.38599031,1048.52546444)(509.42599587,1048.49547188)
\curveto(509.48599021,1048.44546452)(509.54099016,1048.39546457)(509.59099587,1048.34547188)
\curveto(509.64099006,1048.29546467)(509.70099,1048.25046472)(509.77099587,1048.21047188)
\curveto(509.79098991,1048.20046477)(509.81598988,1048.19046478)(509.84599587,1048.18047188)
\curveto(509.88598981,1048.1704648)(509.91598978,1048.17546479)(509.93599587,1048.19547188)
\curveto(509.98598971,1048.21546475)(510.01598968,1048.25046472)(510.02599587,1048.30047188)
\curveto(510.03598966,1048.35046462)(510.05098965,1048.40046457)(510.07099587,1048.45047188)
\curveto(510.09098961,1048.50046447)(510.10598959,1048.55046442)(510.11599587,1048.60047188)
\curveto(510.13598956,1048.66046431)(510.16598953,1048.71046426)(510.20599587,1048.75047188)
\curveto(510.26598943,1048.79046418)(510.33598936,1048.81046416)(510.41599587,1048.81047188)
\curveto(510.50598919,1048.82046415)(510.5959891,1048.82546414)(510.68599587,1048.82547188)
\lineto(511.45099587,1048.82547188)
\curveto(511.56098814,1048.82546414)(511.65598804,1048.82046415)(511.73599587,1048.81047188)
\curveto(511.82598787,1048.81046416)(511.9009878,1048.78546418)(511.96099587,1048.73547188)
\moveto(509.90599587,1044.10047188)
\curveto(509.94598975,1044.19046878)(509.98098972,1044.30546866)(510.01099587,1044.44547188)
\curveto(510.04098966,1044.58546838)(510.06098964,1044.73046824)(510.07099587,1044.88047188)
\curveto(510.08098962,1045.04046793)(510.08098962,1045.19546777)(510.07099587,1045.34547188)
\curveto(510.07098963,1045.49546747)(510.05598964,1045.63046734)(510.02599587,1045.75047188)
\curveto(510.00598969,1045.79046718)(509.9959897,1045.82046715)(509.99599587,1045.84047188)
\curveto(510.00598969,1045.8704671)(510.00598969,1045.90546706)(509.99599587,1045.94547188)
\lineto(509.93599587,1046.15547188)
\curveto(509.91598978,1046.22546674)(509.89098981,1046.29046668)(509.86099587,1046.35047188)
\curveto(509.72098998,1046.70046627)(509.52099018,1046.970466)(509.26099587,1047.16047188)
\curveto(509.0009907,1047.35046562)(508.62099108,1047.44546552)(508.12099587,1047.44547188)
\curveto(508.1009916,1047.42546554)(508.07099163,1047.41546555)(508.03099587,1047.41547188)
\curveto(508.0009917,1047.42546554)(507.97099173,1047.42546554)(507.94099587,1047.41547188)
\curveto(507.87099183,1047.39546557)(507.80599189,1047.37546559)(507.74599587,1047.35547188)
\curveto(507.68599201,1047.34546562)(507.62599207,1047.33046564)(507.56599587,1047.31047188)
\curveto(507.30599239,1047.20046577)(507.10599259,1047.03546593)(506.96599587,1046.81547188)
\curveto(506.82599287,1046.59546637)(506.71099299,1046.35046662)(506.62099587,1046.08047188)
\curveto(506.6009931,1046.03046694)(506.59099311,1045.99046698)(506.59099587,1045.96047188)
\curveto(506.59099311,1045.93046704)(506.58599311,1045.89046708)(506.57599587,1045.84047188)
\curveto(506.54599315,1045.73046724)(506.52599317,1045.5704674)(506.51599587,1045.36047188)
\curveto(506.50599319,1045.15046782)(506.51599318,1044.98046799)(506.54599587,1044.85047188)
\lineto(506.54599587,1044.70047188)
\curveto(506.56599313,1044.62046835)(506.58099312,1044.54046843)(506.59099587,1044.46047188)
\curveto(506.6009931,1044.39046858)(506.61599308,1044.31546865)(506.63599587,1044.23547188)
\curveto(506.72599297,1043.97546899)(506.83599286,1043.74546922)(506.96599587,1043.54547188)
\curveto(507.0959926,1043.35546961)(507.27599242,1043.20046977)(507.50599587,1043.08047188)
\curveto(507.60599209,1043.03046994)(507.74599195,1042.98046999)(507.92599587,1042.93047188)
\curveto(507.9959917,1042.93047004)(508.05099165,1042.92547004)(508.09099587,1042.91547188)
\curveto(508.11099159,1042.91547005)(508.14099156,1042.91047006)(508.18099587,1042.90047188)
\curveto(508.22099148,1042.90047007)(508.25099145,1042.90547006)(508.27099587,1042.91547188)
\lineto(508.42099587,1042.91547188)
\curveto(508.51099119,1042.93547003)(508.5959911,1042.95047002)(508.67599587,1042.96047188)
\curveto(508.75599094,1042.97047)(508.83599086,1042.99546997)(508.91599587,1043.03547188)
\curveto(509.16599053,1043.13546983)(509.36599033,1043.27546969)(509.51599587,1043.45547188)
\curveto(509.67599002,1043.63546933)(509.80598989,1043.85046912)(509.90599587,1044.10047188)
}
}
{
\newrgbcolor{curcolor}{0 0 0}
\pscustom[linestyle=none,fillstyle=solid,fillcolor=curcolor]
{
\newpath
\moveto(518.20591774,1049.02047188)
\curveto(518.31591243,1049.02046395)(518.41091233,1049.01046396)(518.49091774,1048.99047188)
\curveto(518.58091216,1048.970464)(518.65091209,1048.92546404)(518.70091774,1048.85547188)
\curveto(518.76091198,1048.77546419)(518.79091195,1048.63546433)(518.79091774,1048.43547188)
\lineto(518.79091774,1047.92547188)
\lineto(518.79091774,1047.55047188)
\curveto(518.80091194,1047.41046556)(518.78591196,1047.30046567)(518.74591774,1047.22047188)
\curveto(518.70591204,1047.15046582)(518.6459121,1047.10546586)(518.56591774,1047.08547188)
\curveto(518.49591225,1047.0654659)(518.41091233,1047.05546591)(518.31091774,1047.05547188)
\curveto(518.22091252,1047.05546591)(518.12091262,1047.06046591)(518.01091774,1047.07047188)
\curveto(517.91091283,1047.08046589)(517.81591293,1047.07546589)(517.72591774,1047.05547188)
\curveto(517.65591309,1047.03546593)(517.58591316,1047.02046595)(517.51591774,1047.01047188)
\curveto(517.4459133,1047.01046596)(517.38091336,1047.00046597)(517.32091774,1046.98047188)
\curveto(517.16091358,1046.93046604)(517.00091374,1046.85546611)(516.84091774,1046.75547188)
\curveto(516.68091406,1046.6654663)(516.55591419,1046.56046641)(516.46591774,1046.44047188)
\curveto(516.41591433,1046.36046661)(516.36091438,1046.27546669)(516.30091774,1046.18547188)
\curveto(516.25091449,1046.10546686)(516.20091454,1046.02046695)(516.15091774,1045.93047188)
\curveto(516.12091462,1045.85046712)(516.09091465,1045.7654672)(516.06091774,1045.67547188)
\lineto(516.00091774,1045.43547188)
\curveto(515.98091476,1045.3654676)(515.97091477,1045.29046768)(515.97091774,1045.21047188)
\curveto(515.97091477,1045.14046783)(515.96091478,1045.0704679)(515.94091774,1045.00047188)
\curveto(515.93091481,1044.96046801)(515.92591482,1044.92046805)(515.92591774,1044.88047188)
\curveto(515.93591481,1044.85046812)(515.93591481,1044.82046815)(515.92591774,1044.79047188)
\lineto(515.92591774,1044.55047188)
\curveto(515.90591484,1044.48046849)(515.90091484,1044.40046857)(515.91091774,1044.31047188)
\curveto(515.92091482,1044.23046874)(515.92591482,1044.15046882)(515.92591774,1044.07047188)
\lineto(515.92591774,1043.11047188)
\lineto(515.92591774,1041.83547188)
\curveto(515.92591482,1041.70547126)(515.92091482,1041.58547138)(515.91091774,1041.47547188)
\curveto(515.90091484,1041.3654716)(515.87091487,1041.27547169)(515.82091774,1041.20547188)
\curveto(515.80091494,1041.17547179)(515.76591498,1041.15047182)(515.71591774,1041.13047188)
\curveto(515.67591507,1041.12047185)(515.63091511,1041.11047186)(515.58091774,1041.10047188)
\lineto(515.50591774,1041.10047188)
\curveto(515.45591529,1041.09047188)(515.40091534,1041.08547188)(515.34091774,1041.08547188)
\lineto(515.17591774,1041.08547188)
\lineto(514.53091774,1041.08547188)
\curveto(514.47091627,1041.09547187)(514.40591634,1041.10047187)(514.33591774,1041.10047188)
\lineto(514.14091774,1041.10047188)
\curveto(514.09091665,1041.12047185)(514.0409167,1041.13547183)(513.99091774,1041.14547188)
\curveto(513.9409168,1041.1654718)(513.90591684,1041.20047177)(513.88591774,1041.25047188)
\curveto(513.8459169,1041.30047167)(513.82091692,1041.3704716)(513.81091774,1041.46047188)
\lineto(513.81091774,1041.76047188)
\lineto(513.81091774,1042.78047188)
\lineto(513.81091774,1047.01047188)
\lineto(513.81091774,1048.12047188)
\lineto(513.81091774,1048.40547188)
\curveto(513.81091693,1048.50546446)(513.83091691,1048.58546438)(513.87091774,1048.64547188)
\curveto(513.92091682,1048.72546424)(513.99591675,1048.77546419)(514.09591774,1048.79547188)
\curveto(514.19591655,1048.81546415)(514.31591643,1048.82546414)(514.45591774,1048.82547188)
\lineto(515.22091774,1048.82547188)
\curveto(515.3409154,1048.82546414)(515.4459153,1048.81546415)(515.53591774,1048.79547188)
\curveto(515.62591512,1048.78546418)(515.69591505,1048.74046423)(515.74591774,1048.66047188)
\curveto(515.77591497,1048.61046436)(515.79091495,1048.54046443)(515.79091774,1048.45047188)
\lineto(515.82091774,1048.18047188)
\curveto(515.83091491,1048.10046487)(515.8459149,1048.02546494)(515.86591774,1047.95547188)
\curveto(515.89591485,1047.88546508)(515.9459148,1047.85046512)(516.01591774,1047.85047188)
\curveto(516.03591471,1047.8704651)(516.05591469,1047.88046509)(516.07591774,1047.88047188)
\curveto(516.09591465,1047.88046509)(516.11591463,1047.89046508)(516.13591774,1047.91047188)
\curveto(516.19591455,1047.96046501)(516.2459145,1048.01546495)(516.28591774,1048.07547188)
\curveto(516.33591441,1048.14546482)(516.39591435,1048.20546476)(516.46591774,1048.25547188)
\curveto(516.50591424,1048.28546468)(516.5409142,1048.31546465)(516.57091774,1048.34547188)
\curveto(516.60091414,1048.38546458)(516.63591411,1048.42046455)(516.67591774,1048.45047188)
\lineto(516.94591774,1048.63047188)
\curveto(517.0459137,1048.69046428)(517.1459136,1048.74546422)(517.24591774,1048.79547188)
\curveto(517.3459134,1048.83546413)(517.4459133,1048.8704641)(517.54591774,1048.90047188)
\lineto(517.87591774,1048.99047188)
\curveto(517.90591284,1049.00046397)(517.96091278,1049.00046397)(518.04091774,1048.99047188)
\curveto(518.13091261,1048.99046398)(518.18591256,1049.00046397)(518.20591774,1049.02047188)
}
}
{
\newrgbcolor{curcolor}{0 0 0}
\pscustom[linestyle=none,fillstyle=solid,fillcolor=curcolor]
{
\newpath
\moveto(526.71232399,1045.03047188)
\curveto(526.73231583,1044.95046802)(526.73231583,1044.86046811)(526.71232399,1044.76047188)
\curveto(526.69231587,1044.66046831)(526.6573159,1044.59546837)(526.60732399,1044.56547188)
\curveto(526.557316,1044.52546844)(526.48231608,1044.49546847)(526.38232399,1044.47547188)
\curveto(526.29231627,1044.4654685)(526.18731637,1044.45546851)(526.06732399,1044.44547188)
\lineto(525.72232399,1044.44547188)
\curveto(525.61231695,1044.45546851)(525.51231705,1044.46046851)(525.42232399,1044.46047188)
\lineto(521.76232399,1044.46047188)
\lineto(521.55232399,1044.46047188)
\curveto(521.49232107,1044.46046851)(521.43732112,1044.45046852)(521.38732399,1044.43047188)
\curveto(521.30732125,1044.39046858)(521.2573213,1044.35046862)(521.23732399,1044.31047188)
\curveto(521.21732134,1044.29046868)(521.19732136,1044.25046872)(521.17732399,1044.19047188)
\curveto(521.1573214,1044.14046883)(521.15232141,1044.09046888)(521.16232399,1044.04047188)
\curveto(521.18232138,1043.98046899)(521.19232137,1043.92046905)(521.19232399,1043.86047188)
\curveto(521.20232136,1043.81046916)(521.21732134,1043.75546921)(521.23732399,1043.69547188)
\curveto(521.31732124,1043.45546951)(521.41232115,1043.25546971)(521.52232399,1043.09547188)
\curveto(521.64232092,1042.94547002)(521.80232076,1042.81047016)(522.00232399,1042.69047188)
\curveto(522.08232048,1042.64047033)(522.1623204,1042.60547036)(522.24232399,1042.58547188)
\curveto(522.33232023,1042.57547039)(522.42232014,1042.55547041)(522.51232399,1042.52547188)
\curveto(522.59231997,1042.50547046)(522.70231986,1042.49047048)(522.84232399,1042.48047188)
\curveto(522.98231958,1042.4704705)(523.10231946,1042.47547049)(523.20232399,1042.49547188)
\lineto(523.33732399,1042.49547188)
\curveto(523.43731912,1042.51547045)(523.52731903,1042.53547043)(523.60732399,1042.55547188)
\curveto(523.69731886,1042.58547038)(523.78231878,1042.61547035)(523.86232399,1042.64547188)
\curveto(523.9623186,1042.69547027)(524.07231849,1042.76047021)(524.19232399,1042.84047188)
\curveto(524.32231824,1042.92047005)(524.41731814,1043.00046997)(524.47732399,1043.08047188)
\curveto(524.52731803,1043.15046982)(524.57731798,1043.21546975)(524.62732399,1043.27547188)
\curveto(524.68731787,1043.34546962)(524.7573178,1043.39546957)(524.83732399,1043.42547188)
\curveto(524.93731762,1043.47546949)(525.0623175,1043.49546947)(525.21232399,1043.48547188)
\lineto(525.64732399,1043.48547188)
\lineto(525.82732399,1043.48547188)
\curveto(525.89731666,1043.49546947)(525.9573166,1043.49046948)(526.00732399,1043.47047188)
\lineto(526.15732399,1043.47047188)
\curveto(526.2573163,1043.45046952)(526.32731623,1043.42546954)(526.36732399,1043.39547188)
\curveto(526.40731615,1043.37546959)(526.42731613,1043.33046964)(526.42732399,1043.26047188)
\curveto(526.43731612,1043.19046978)(526.43231613,1043.13046984)(526.41232399,1043.08047188)
\curveto(526.3623162,1042.94047003)(526.30731625,1042.81547015)(526.24732399,1042.70547188)
\curveto(526.18731637,1042.59547037)(526.11731644,1042.48547048)(526.03732399,1042.37547188)
\curveto(525.81731674,1042.04547092)(525.56731699,1041.78047119)(525.28732399,1041.58047188)
\curveto(525.00731755,1041.38047159)(524.6573179,1041.21047176)(524.23732399,1041.07047188)
\curveto(524.12731843,1041.03047194)(524.01731854,1041.00547196)(523.90732399,1040.99547188)
\curveto(523.79731876,1040.98547198)(523.68231888,1040.965472)(523.56232399,1040.93547188)
\curveto(523.52231904,1040.92547204)(523.47731908,1040.92547204)(523.42732399,1040.93547188)
\curveto(523.38731917,1040.93547203)(523.34731921,1040.93047204)(523.30732399,1040.92047188)
\lineto(523.14232399,1040.92047188)
\curveto(523.09231947,1040.90047207)(523.03231953,1040.89547207)(522.96232399,1040.90547188)
\curveto(522.90231966,1040.90547206)(522.84731971,1040.91047206)(522.79732399,1040.92047188)
\curveto(522.71731984,1040.93047204)(522.64731991,1040.93047204)(522.58732399,1040.92047188)
\curveto(522.52732003,1040.91047206)(522.4623201,1040.91547205)(522.39232399,1040.93547188)
\curveto(522.34232022,1040.95547201)(522.28732027,1040.965472)(522.22732399,1040.96547188)
\curveto(522.16732039,1040.965472)(522.11232045,1040.97547199)(522.06232399,1040.99547188)
\curveto(521.95232061,1041.01547195)(521.84232072,1041.04047193)(521.73232399,1041.07047188)
\curveto(521.62232094,1041.09047188)(521.52232104,1041.12547184)(521.43232399,1041.17547188)
\curveto(521.32232124,1041.21547175)(521.21732134,1041.25047172)(521.11732399,1041.28047188)
\curveto(521.02732153,1041.32047165)(520.94232162,1041.3654716)(520.86232399,1041.41547188)
\curveto(520.54232202,1041.61547135)(520.2573223,1041.84547112)(520.00732399,1042.10547188)
\curveto(519.7573228,1042.37547059)(519.55232301,1042.68547028)(519.39232399,1043.03547188)
\curveto(519.34232322,1043.14546982)(519.30232326,1043.25546971)(519.27232399,1043.36547188)
\curveto(519.24232332,1043.48546948)(519.20232336,1043.60546936)(519.15232399,1043.72547188)
\curveto(519.14232342,1043.7654692)(519.13732342,1043.80046917)(519.13732399,1043.83047188)
\curveto(519.13732342,1043.8704691)(519.13232343,1043.91046906)(519.12232399,1043.95047188)
\curveto(519.08232348,1044.0704689)(519.0573235,1044.20046877)(519.04732399,1044.34047188)
\lineto(519.01732399,1044.76047188)
\curveto(519.01732354,1044.81046816)(519.01232355,1044.8654681)(519.00232399,1044.92547188)
\curveto(519.00232356,1044.98546798)(519.00732355,1045.04046793)(519.01732399,1045.09047188)
\lineto(519.01732399,1045.27047188)
\lineto(519.06232399,1045.63047188)
\curveto(519.10232346,1045.80046717)(519.13732342,1045.965467)(519.16732399,1046.12547188)
\curveto(519.19732336,1046.28546668)(519.24232332,1046.43546653)(519.30232399,1046.57547188)
\curveto(519.73232283,1047.61546535)(520.4623221,1048.35046462)(521.49232399,1048.78047188)
\curveto(521.63232093,1048.84046413)(521.77232079,1048.88046409)(521.91232399,1048.90047188)
\curveto(522.0623205,1048.93046404)(522.21732034,1048.965464)(522.37732399,1049.00547188)
\curveto(522.4573201,1049.01546395)(522.53232003,1049.02046395)(522.60232399,1049.02047188)
\curveto(522.67231989,1049.02046395)(522.74731981,1049.02546394)(522.82732399,1049.03547188)
\curveto(523.33731922,1049.04546392)(523.77231879,1048.98546398)(524.13232399,1048.85547188)
\curveto(524.50231806,1048.73546423)(524.83231773,1048.57546439)(525.12232399,1048.37547188)
\curveto(525.21231735,1048.31546465)(525.30231726,1048.24546472)(525.39232399,1048.16547188)
\curveto(525.48231708,1048.09546487)(525.562317,1048.02046495)(525.63232399,1047.94047188)
\curveto(525.6623169,1047.89046508)(525.70231686,1047.85046512)(525.75232399,1047.82047188)
\curveto(525.83231673,1047.71046526)(525.90731665,1047.59546537)(525.97732399,1047.47547188)
\curveto(526.04731651,1047.3654656)(526.12231644,1047.25046572)(526.20232399,1047.13047188)
\curveto(526.25231631,1047.04046593)(526.29231627,1046.94546602)(526.32232399,1046.84547188)
\curveto(526.3623162,1046.75546621)(526.40231616,1046.65546631)(526.44232399,1046.54547188)
\curveto(526.49231607,1046.41546655)(526.53231603,1046.28046669)(526.56232399,1046.14047188)
\curveto(526.59231597,1046.00046697)(526.62731593,1045.86046711)(526.66732399,1045.72047188)
\curveto(526.68731587,1045.64046733)(526.69231587,1045.55046742)(526.68232399,1045.45047188)
\curveto(526.68231588,1045.36046761)(526.69231587,1045.27546769)(526.71232399,1045.19547188)
\lineto(526.71232399,1045.03047188)
\moveto(524.46232399,1045.91547188)
\curveto(524.53231803,1046.01546695)(524.53731802,1046.13546683)(524.47732399,1046.27547188)
\curveto(524.42731813,1046.42546654)(524.38731817,1046.53546643)(524.35732399,1046.60547188)
\curveto(524.21731834,1046.87546609)(524.03231853,1047.08046589)(523.80232399,1047.22047188)
\curveto(523.57231899,1047.3704656)(523.25231931,1047.45046552)(522.84232399,1047.46047188)
\curveto(522.81231975,1047.44046553)(522.77731978,1047.43546553)(522.73732399,1047.44547188)
\curveto(522.69731986,1047.45546551)(522.6623199,1047.45546551)(522.63232399,1047.44547188)
\curveto(522.58231998,1047.42546554)(522.52732003,1047.41046556)(522.46732399,1047.40047188)
\curveto(522.40732015,1047.40046557)(522.35232021,1047.39046558)(522.30232399,1047.37047188)
\curveto(521.8623207,1047.23046574)(521.53732102,1046.95546601)(521.32732399,1046.54547188)
\curveto(521.30732125,1046.50546646)(521.28232128,1046.45046652)(521.25232399,1046.38047188)
\curveto(521.23232133,1046.32046665)(521.21732134,1046.25546671)(521.20732399,1046.18547188)
\curveto(521.19732136,1046.12546684)(521.19732136,1046.0654669)(521.20732399,1046.00547188)
\curveto(521.22732133,1045.94546702)(521.2623213,1045.89546707)(521.31232399,1045.85547188)
\curveto(521.39232117,1045.80546716)(521.50232106,1045.78046719)(521.64232399,1045.78047188)
\lineto(522.04732399,1045.78047188)
\lineto(523.71232399,1045.78047188)
\lineto(524.14732399,1045.78047188)
\curveto(524.30731825,1045.79046718)(524.41231815,1045.83546713)(524.46232399,1045.91547188)
}
}
{
\newrgbcolor{curcolor}{0 0 0}
\pscustom[linestyle=none,fillstyle=solid,fillcolor=curcolor]
{
\newpath
\moveto(530.93060524,1049.03547188)
\curveto(531.68060074,1049.05546391)(532.33060009,1048.970464)(532.88060524,1048.78047188)
\curveto(533.44059898,1048.60046437)(533.86559856,1048.28546468)(534.15560524,1047.83547188)
\curveto(534.2255982,1047.72546524)(534.28559814,1047.61046536)(534.33560524,1047.49047188)
\curveto(534.39559803,1047.38046559)(534.44559798,1047.25546571)(534.48560524,1047.11547188)
\curveto(534.50559792,1047.05546591)(534.51559791,1046.99046598)(534.51560524,1046.92047188)
\curveto(534.51559791,1046.85046612)(534.50559792,1046.79046618)(534.48560524,1046.74047188)
\curveto(534.44559798,1046.68046629)(534.39059803,1046.64046633)(534.32060524,1046.62047188)
\curveto(534.27059815,1046.60046637)(534.21059821,1046.59046638)(534.14060524,1046.59047188)
\lineto(533.93060524,1046.59047188)
\lineto(533.27060524,1046.59047188)
\curveto(533.20059922,1046.59046638)(533.13059929,1046.58546638)(533.06060524,1046.57547188)
\curveto(532.99059943,1046.57546639)(532.9255995,1046.58546638)(532.86560524,1046.60547188)
\curveto(532.76559966,1046.62546634)(532.69059973,1046.6654663)(532.64060524,1046.72547188)
\curveto(532.59059983,1046.78546618)(532.54559988,1046.84546612)(532.50560524,1046.90547188)
\lineto(532.38560524,1047.11547188)
\curveto(532.35560007,1047.19546577)(532.30560012,1047.26046571)(532.23560524,1047.31047188)
\curveto(532.13560029,1047.39046558)(532.03560039,1047.45046552)(531.93560524,1047.49047188)
\curveto(531.84560058,1047.53046544)(531.73060069,1047.5654654)(531.59060524,1047.59547188)
\curveto(531.5206009,1047.61546535)(531.41560101,1047.63046534)(531.27560524,1047.64047188)
\curveto(531.14560128,1047.65046532)(531.04560138,1047.64546532)(530.97560524,1047.62547188)
\lineto(530.87060524,1047.62547188)
\lineto(530.72060524,1047.59547188)
\curveto(530.68060174,1047.59546537)(530.63560179,1047.59046538)(530.58560524,1047.58047188)
\curveto(530.41560201,1047.53046544)(530.27560215,1047.46046551)(530.16560524,1047.37047188)
\curveto(530.06560236,1047.29046568)(529.99560243,1047.1654658)(529.95560524,1046.99547188)
\curveto(529.93560249,1046.92546604)(529.93560249,1046.86046611)(529.95560524,1046.80047188)
\curveto(529.97560245,1046.74046623)(529.99560243,1046.69046628)(530.01560524,1046.65047188)
\curveto(530.08560234,1046.53046644)(530.16560226,1046.43546653)(530.25560524,1046.36547188)
\curveto(530.35560207,1046.29546667)(530.47060195,1046.23546673)(530.60060524,1046.18547188)
\curveto(530.79060163,1046.10546686)(530.99560143,1046.03546693)(531.21560524,1045.97547188)
\lineto(531.90560524,1045.82547188)
\curveto(532.14560028,1045.78546718)(532.37560005,1045.73546723)(532.59560524,1045.67547188)
\curveto(532.8255996,1045.62546734)(533.04059938,1045.56046741)(533.24060524,1045.48047188)
\curveto(533.33059909,1045.44046753)(533.41559901,1045.40546756)(533.49560524,1045.37547188)
\curveto(533.58559884,1045.35546761)(533.67059875,1045.32046765)(533.75060524,1045.27047188)
\curveto(533.94059848,1045.15046782)(534.11059831,1045.02046795)(534.26060524,1044.88047188)
\curveto(534.420598,1044.74046823)(534.54559788,1044.5654684)(534.63560524,1044.35547188)
\curveto(534.66559776,1044.28546868)(534.69059773,1044.21546875)(534.71060524,1044.14547188)
\curveto(534.73059769,1044.07546889)(534.75059767,1044.00046897)(534.77060524,1043.92047188)
\curveto(534.78059764,1043.86046911)(534.78559764,1043.7654692)(534.78560524,1043.63547188)
\curveto(534.79559763,1043.51546945)(534.79559763,1043.42046955)(534.78560524,1043.35047188)
\lineto(534.78560524,1043.27547188)
\curveto(534.76559766,1043.21546975)(534.75059767,1043.15546981)(534.74060524,1043.09547188)
\curveto(534.74059768,1043.04546992)(534.73559769,1042.99546997)(534.72560524,1042.94547188)
\curveto(534.65559777,1042.64547032)(534.54559788,1042.38047059)(534.39560524,1042.15047188)
\curveto(534.23559819,1041.91047106)(534.04059838,1041.71547125)(533.81060524,1041.56547188)
\curveto(533.58059884,1041.41547155)(533.3205991,1041.28547168)(533.03060524,1041.17547188)
\curveto(532.9205995,1041.12547184)(532.80059962,1041.09047188)(532.67060524,1041.07047188)
\curveto(532.55059987,1041.05047192)(532.43059999,1041.02547194)(532.31060524,1040.99547188)
\curveto(532.2206002,1040.97547199)(532.1256003,1040.965472)(532.02560524,1040.96547188)
\curveto(531.93560049,1040.95547201)(531.84560058,1040.94047203)(531.75560524,1040.92047188)
\lineto(531.48560524,1040.92047188)
\curveto(531.425601,1040.90047207)(531.3206011,1040.89047208)(531.17060524,1040.89047188)
\curveto(531.03060139,1040.89047208)(530.93060149,1040.90047207)(530.87060524,1040.92047188)
\curveto(530.84060158,1040.92047205)(530.80560162,1040.92547204)(530.76560524,1040.93547188)
\lineto(530.66060524,1040.93547188)
\curveto(530.54060188,1040.95547201)(530.420602,1040.970472)(530.30060524,1040.98047188)
\curveto(530.18060224,1040.99047198)(530.06560236,1041.01047196)(529.95560524,1041.04047188)
\curveto(529.56560286,1041.15047182)(529.2206032,1041.27547169)(528.92060524,1041.41547188)
\curveto(528.6206038,1041.5654714)(528.36560406,1041.78547118)(528.15560524,1042.07547188)
\curveto(528.01560441,1042.2654707)(527.89560453,1042.48547048)(527.79560524,1042.73547188)
\curveto(527.77560465,1042.79547017)(527.75560467,1042.87547009)(527.73560524,1042.97547188)
\curveto(527.71560471,1043.02546994)(527.70060472,1043.09546987)(527.69060524,1043.18547188)
\curveto(527.68060474,1043.27546969)(527.68560474,1043.35046962)(527.70560524,1043.41047188)
\curveto(527.73560469,1043.48046949)(527.78560464,1043.53046944)(527.85560524,1043.56047188)
\curveto(527.90560452,1043.58046939)(527.96560446,1043.59046938)(528.03560524,1043.59047188)
\lineto(528.26060524,1043.59047188)
\lineto(528.96560524,1043.59047188)
\lineto(529.20560524,1043.59047188)
\curveto(529.28560314,1043.59046938)(529.35560307,1043.58046939)(529.41560524,1043.56047188)
\curveto(529.5256029,1043.52046945)(529.59560283,1043.45546951)(529.62560524,1043.36547188)
\curveto(529.66560276,1043.27546969)(529.71060271,1043.18046979)(529.76060524,1043.08047188)
\curveto(529.78060264,1043.03046994)(529.81560261,1042.96547)(529.86560524,1042.88547188)
\curveto(529.9256025,1042.80547016)(529.97560245,1042.75547021)(530.01560524,1042.73547188)
\curveto(530.13560229,1042.63547033)(530.25060217,1042.55547041)(530.36060524,1042.49547188)
\curveto(530.47060195,1042.44547052)(530.61060181,1042.39547057)(530.78060524,1042.34547188)
\curveto(530.83060159,1042.32547064)(530.88060154,1042.31547065)(530.93060524,1042.31547188)
\curveto(530.98060144,1042.32547064)(531.03060139,1042.32547064)(531.08060524,1042.31547188)
\curveto(531.16060126,1042.29547067)(531.24560118,1042.28547068)(531.33560524,1042.28547188)
\curveto(531.43560099,1042.29547067)(531.5206009,1042.31047066)(531.59060524,1042.33047188)
\curveto(531.64060078,1042.34047063)(531.68560074,1042.34547062)(531.72560524,1042.34547188)
\curveto(531.77560065,1042.34547062)(531.8256006,1042.35547061)(531.87560524,1042.37547188)
\curveto(532.01560041,1042.42547054)(532.14060028,1042.48547048)(532.25060524,1042.55547188)
\curveto(532.37060005,1042.62547034)(532.46559996,1042.71547025)(532.53560524,1042.82547188)
\curveto(532.58559984,1042.90547006)(532.6255998,1043.03046994)(532.65560524,1043.20047188)
\curveto(532.67559975,1043.2704697)(532.67559975,1043.33546963)(532.65560524,1043.39547188)
\curveto(532.63559979,1043.45546951)(532.61559981,1043.50546946)(532.59560524,1043.54547188)
\curveto(532.5255999,1043.68546928)(532.43559999,1043.79046918)(532.32560524,1043.86047188)
\curveto(532.2256002,1043.93046904)(532.10560032,1043.99546897)(531.96560524,1044.05547188)
\curveto(531.77560065,1044.13546883)(531.57560085,1044.20046877)(531.36560524,1044.25047188)
\curveto(531.15560127,1044.30046867)(530.94560148,1044.35546861)(530.73560524,1044.41547188)
\curveto(530.65560177,1044.43546853)(530.57060185,1044.45046852)(530.48060524,1044.46047188)
\curveto(530.40060202,1044.4704685)(530.3206021,1044.48546848)(530.24060524,1044.50547188)
\curveto(529.9206025,1044.59546837)(529.61560281,1044.68046829)(529.32560524,1044.76047188)
\curveto(529.03560339,1044.85046812)(528.77060365,1044.98046799)(528.53060524,1045.15047188)
\curveto(528.25060417,1045.35046762)(528.04560438,1045.62046735)(527.91560524,1045.96047188)
\curveto(527.89560453,1046.03046694)(527.87560455,1046.12546684)(527.85560524,1046.24547188)
\curveto(527.83560459,1046.31546665)(527.8206046,1046.40046657)(527.81060524,1046.50047188)
\curveto(527.80060462,1046.60046637)(527.80560462,1046.69046628)(527.82560524,1046.77047188)
\curveto(527.84560458,1046.82046615)(527.85060457,1046.86046611)(527.84060524,1046.89047188)
\curveto(527.83060459,1046.93046604)(527.83560459,1046.97546599)(527.85560524,1047.02547188)
\curveto(527.87560455,1047.13546583)(527.89560453,1047.23546573)(527.91560524,1047.32547188)
\curveto(527.94560448,1047.42546554)(527.98060444,1047.52046545)(528.02060524,1047.61047188)
\curveto(528.15060427,1047.90046507)(528.33060409,1048.13546483)(528.56060524,1048.31547188)
\curveto(528.79060363,1048.49546447)(529.05060337,1048.64046433)(529.34060524,1048.75047188)
\curveto(529.45060297,1048.80046417)(529.56560286,1048.83546413)(529.68560524,1048.85547188)
\curveto(529.80560262,1048.88546408)(529.93060249,1048.91546405)(530.06060524,1048.94547188)
\curveto(530.1206023,1048.965464)(530.18060224,1048.97546399)(530.24060524,1048.97547188)
\lineto(530.42060524,1049.00547188)
\curveto(530.50060192,1049.01546395)(530.58560184,1049.02046395)(530.67560524,1049.02047188)
\curveto(530.76560166,1049.02046395)(530.85060157,1049.02546394)(530.93060524,1049.03547188)
}
}
{
\newrgbcolor{curcolor}{0 0 0}
\pscustom[linestyle=none,fillstyle=solid,fillcolor=curcolor]
{
\newpath
\moveto(543.06724587,1041.68547188)
\curveto(543.08723802,1041.57547139)(543.09723801,1041.4654715)(543.09724587,1041.35547188)
\curveto(543.107238,1041.24547172)(543.05723805,1041.1704718)(542.94724587,1041.13047188)
\curveto(542.88723822,1041.10047187)(542.81723829,1041.08547188)(542.73724587,1041.08547188)
\lineto(542.49724587,1041.08547188)
\lineto(541.68724587,1041.08547188)
\lineto(541.41724587,1041.08547188)
\curveto(541.33723977,1041.09547187)(541.27223983,1041.12047185)(541.22224587,1041.16047188)
\curveto(541.15223995,1041.20047177)(541.09724001,1041.25547171)(541.05724587,1041.32547188)
\curveto(541.02724008,1041.40547156)(540.98224012,1041.4704715)(540.92224587,1041.52047188)
\curveto(540.9022402,1041.54047143)(540.87724023,1041.55547141)(540.84724587,1041.56547188)
\curveto(540.81724029,1041.58547138)(540.77724033,1041.59047138)(540.72724587,1041.58047188)
\curveto(540.67724043,1041.56047141)(540.62724048,1041.53547143)(540.57724587,1041.50547188)
\curveto(540.53724057,1041.47547149)(540.49224061,1041.45047152)(540.44224587,1041.43047188)
\curveto(540.39224071,1041.39047158)(540.33724077,1041.35547161)(540.27724587,1041.32547188)
\lineto(540.09724587,1041.23547188)
\curveto(539.96724114,1041.17547179)(539.83224127,1041.12547184)(539.69224587,1041.08547188)
\curveto(539.55224155,1041.05547191)(539.4072417,1041.02047195)(539.25724587,1040.98047188)
\curveto(539.18724192,1040.96047201)(539.11724199,1040.95047202)(539.04724587,1040.95047188)
\curveto(538.98724212,1040.94047203)(538.92224218,1040.93047204)(538.85224587,1040.92047188)
\lineto(538.76224587,1040.92047188)
\curveto(538.73224237,1040.91047206)(538.7022424,1040.90547206)(538.67224587,1040.90547188)
\lineto(538.50724587,1040.90547188)
\curveto(538.4072427,1040.88547208)(538.3072428,1040.88547208)(538.20724587,1040.90547188)
\lineto(538.07224587,1040.90547188)
\curveto(538.0022431,1040.92547204)(537.93224317,1040.93547203)(537.86224587,1040.93547188)
\curveto(537.8022433,1040.92547204)(537.74224336,1040.93047204)(537.68224587,1040.95047188)
\curveto(537.58224352,1040.970472)(537.48724362,1040.99047198)(537.39724587,1041.01047188)
\curveto(537.3072438,1041.02047195)(537.22224388,1041.04547192)(537.14224587,1041.08547188)
\curveto(536.85224425,1041.19547177)(536.6022445,1041.33547163)(536.39224587,1041.50547188)
\curveto(536.19224491,1041.68547128)(536.03224507,1041.92047105)(535.91224587,1042.21047188)
\curveto(535.88224522,1042.28047069)(535.85224525,1042.35547061)(535.82224587,1042.43547188)
\curveto(535.8022453,1042.51547045)(535.78224532,1042.60047037)(535.76224587,1042.69047188)
\curveto(535.74224536,1042.74047023)(535.73224537,1042.79047018)(535.73224587,1042.84047188)
\curveto(535.74224536,1042.89047008)(535.74224536,1042.94047003)(535.73224587,1042.99047188)
\curveto(535.72224538,1043.02046995)(535.71224539,1043.08046989)(535.70224587,1043.17047188)
\curveto(535.7022454,1043.2704697)(535.7072454,1043.34046963)(535.71724587,1043.38047188)
\curveto(535.73724537,1043.48046949)(535.74724536,1043.5654694)(535.74724587,1043.63547188)
\lineto(535.83724587,1043.96547188)
\curveto(535.86724524,1044.08546888)(535.9072452,1044.19046878)(535.95724587,1044.28047188)
\curveto(536.12724498,1044.5704684)(536.32224478,1044.79046818)(536.54224587,1044.94047188)
\curveto(536.76224434,1045.09046788)(537.04224406,1045.22046775)(537.38224587,1045.33047188)
\curveto(537.51224359,1045.38046759)(537.64724346,1045.41546755)(537.78724587,1045.43547188)
\curveto(537.92724318,1045.45546751)(538.06724304,1045.48046749)(538.20724587,1045.51047188)
\curveto(538.28724282,1045.53046744)(538.37224273,1045.54046743)(538.46224587,1045.54047188)
\curveto(538.55224255,1045.55046742)(538.64224246,1045.5654674)(538.73224587,1045.58547188)
\curveto(538.8022423,1045.60546736)(538.87224223,1045.61046736)(538.94224587,1045.60047188)
\curveto(539.01224209,1045.60046737)(539.08724202,1045.61046736)(539.16724587,1045.63047188)
\curveto(539.23724187,1045.65046732)(539.3072418,1045.66046731)(539.37724587,1045.66047188)
\curveto(539.44724166,1045.66046731)(539.52224158,1045.6704673)(539.60224587,1045.69047188)
\curveto(539.81224129,1045.74046723)(540.0022411,1045.78046719)(540.17224587,1045.81047188)
\curveto(540.35224075,1045.85046712)(540.51224059,1045.94046703)(540.65224587,1046.08047188)
\curveto(540.74224036,1046.1704668)(540.8022403,1046.2704667)(540.83224587,1046.38047188)
\curveto(540.84224026,1046.41046656)(540.84224026,1046.43546653)(540.83224587,1046.45547188)
\curveto(540.83224027,1046.47546649)(540.83724027,1046.49546647)(540.84724587,1046.51547188)
\curveto(540.85724025,1046.53546643)(540.86224024,1046.5654664)(540.86224587,1046.60547188)
\lineto(540.86224587,1046.69547188)
\lineto(540.83224587,1046.81547188)
\curveto(540.83224027,1046.85546611)(540.82724028,1046.89046608)(540.81724587,1046.92047188)
\curveto(540.71724039,1047.22046575)(540.5072406,1047.42546554)(540.18724587,1047.53547188)
\curveto(540.09724101,1047.5654654)(539.98724112,1047.58546538)(539.85724587,1047.59547188)
\curveto(539.73724137,1047.61546535)(539.61224149,1047.62046535)(539.48224587,1047.61047188)
\curveto(539.35224175,1047.61046536)(539.22724188,1047.60046537)(539.10724587,1047.58047188)
\curveto(538.98724212,1047.56046541)(538.88224222,1047.53546543)(538.79224587,1047.50547188)
\curveto(538.73224237,1047.48546548)(538.67224243,1047.45546551)(538.61224587,1047.41547188)
\curveto(538.56224254,1047.38546558)(538.51224259,1047.35046562)(538.46224587,1047.31047188)
\curveto(538.41224269,1047.2704657)(538.35724275,1047.21546575)(538.29724587,1047.14547188)
\curveto(538.24724286,1047.07546589)(538.21224289,1047.01046596)(538.19224587,1046.95047188)
\curveto(538.14224296,1046.85046612)(538.09724301,1046.75546621)(538.05724587,1046.66547188)
\curveto(538.02724308,1046.57546639)(537.95724315,1046.51546645)(537.84724587,1046.48547188)
\curveto(537.76724334,1046.4654665)(537.68224342,1046.45546651)(537.59224587,1046.45547188)
\lineto(537.32224587,1046.45547188)
\lineto(536.75224587,1046.45547188)
\curveto(536.7022444,1046.45546651)(536.65224445,1046.45046652)(536.60224587,1046.44047188)
\curveto(536.55224455,1046.44046653)(536.5072446,1046.44546652)(536.46724587,1046.45547188)
\lineto(536.33224587,1046.45547188)
\curveto(536.31224479,1046.4654665)(536.28724482,1046.4704665)(536.25724587,1046.47047188)
\curveto(536.22724488,1046.4704665)(536.2022449,1046.48046649)(536.18224587,1046.50047188)
\curveto(536.102245,1046.52046645)(536.04724506,1046.58546638)(536.01724587,1046.69547188)
\curveto(536.0072451,1046.74546622)(536.0072451,1046.79546617)(536.01724587,1046.84547188)
\curveto(536.02724508,1046.89546607)(536.03724507,1046.94046603)(536.04724587,1046.98047188)
\curveto(536.07724503,1047.09046588)(536.107245,1047.19046578)(536.13724587,1047.28047188)
\curveto(536.17724493,1047.38046559)(536.22224488,1047.4704655)(536.27224587,1047.55047188)
\lineto(536.36224587,1047.70047188)
\lineto(536.45224587,1047.85047188)
\curveto(536.53224457,1047.96046501)(536.63224447,1048.0654649)(536.75224587,1048.16547188)
\curveto(536.77224433,1048.17546479)(536.8022443,1048.20046477)(536.84224587,1048.24047188)
\curveto(536.89224421,1048.28046469)(536.93724417,1048.31546465)(536.97724587,1048.34547188)
\curveto(537.01724409,1048.37546459)(537.06224404,1048.40546456)(537.11224587,1048.43547188)
\curveto(537.28224382,1048.54546442)(537.46224364,1048.63046434)(537.65224587,1048.69047188)
\curveto(537.84224326,1048.76046421)(538.03724307,1048.82546414)(538.23724587,1048.88547188)
\curveto(538.35724275,1048.91546405)(538.48224262,1048.93546403)(538.61224587,1048.94547188)
\curveto(538.74224236,1048.95546401)(538.87224223,1048.97546399)(539.00224587,1049.00547188)
\curveto(539.04224206,1049.01546395)(539.102242,1049.01546395)(539.18224587,1049.00547188)
\curveto(539.27224183,1048.99546397)(539.32724178,1049.00046397)(539.34724587,1049.02047188)
\curveto(539.75724135,1049.03046394)(540.14724096,1049.01546395)(540.51724587,1048.97547188)
\curveto(540.89724021,1048.93546403)(541.23723987,1048.86046411)(541.53724587,1048.75047188)
\curveto(541.84723926,1048.64046433)(542.11223899,1048.49046448)(542.33224587,1048.30047188)
\curveto(542.55223855,1048.12046485)(542.72223838,1047.88546508)(542.84224587,1047.59547188)
\curveto(542.91223819,1047.42546554)(542.95223815,1047.23046574)(542.96224587,1047.01047188)
\curveto(542.97223813,1046.79046618)(542.97723813,1046.5654664)(542.97724587,1046.33547188)
\lineto(542.97724587,1042.99047188)
\lineto(542.97724587,1042.40547188)
\curveto(542.97723813,1042.21547075)(542.99723811,1042.04047093)(543.03724587,1041.88047188)
\curveto(543.04723806,1041.85047112)(543.05223805,1041.81547115)(543.05224587,1041.77547188)
\curveto(543.05223805,1041.74547122)(543.05723805,1041.71547125)(543.06724587,1041.68547188)
\moveto(540.86224587,1043.99547188)
\curveto(540.87224023,1044.04546892)(540.87724023,1044.10046887)(540.87724587,1044.16047188)
\curveto(540.87724023,1044.23046874)(540.87224023,1044.29046868)(540.86224587,1044.34047188)
\curveto(540.84224026,1044.40046857)(540.83224027,1044.45546851)(540.83224587,1044.50547188)
\curveto(540.83224027,1044.55546841)(540.81224029,1044.59546837)(540.77224587,1044.62547188)
\curveto(540.72224038,1044.6654683)(540.64724046,1044.68546828)(540.54724587,1044.68547188)
\curveto(540.5072406,1044.67546829)(540.47224063,1044.6654683)(540.44224587,1044.65547188)
\curveto(540.41224069,1044.65546831)(540.37724073,1044.65046832)(540.33724587,1044.64047188)
\curveto(540.26724084,1044.62046835)(540.19224091,1044.60546836)(540.11224587,1044.59547188)
\curveto(540.03224107,1044.58546838)(539.95224115,1044.5704684)(539.87224587,1044.55047188)
\curveto(539.84224126,1044.54046843)(539.79724131,1044.53546843)(539.73724587,1044.53547188)
\curveto(539.6072415,1044.50546846)(539.47724163,1044.48546848)(539.34724587,1044.47547188)
\curveto(539.21724189,1044.4654685)(539.09224201,1044.44046853)(538.97224587,1044.40047188)
\curveto(538.89224221,1044.38046859)(538.81724229,1044.36046861)(538.74724587,1044.34047188)
\curveto(538.67724243,1044.33046864)(538.6072425,1044.31046866)(538.53724587,1044.28047188)
\curveto(538.32724278,1044.19046878)(538.14724296,1044.05546891)(537.99724587,1043.87547188)
\curveto(537.85724325,1043.69546927)(537.8072433,1043.44546952)(537.84724587,1043.12547188)
\curveto(537.86724324,1042.95547001)(537.92224318,1042.81547015)(538.01224587,1042.70547188)
\curveto(538.08224302,1042.59547037)(538.18724292,1042.50547046)(538.32724587,1042.43547188)
\curveto(538.46724264,1042.37547059)(538.61724249,1042.33047064)(538.77724587,1042.30047188)
\curveto(538.94724216,1042.2704707)(539.12224198,1042.26047071)(539.30224587,1042.27047188)
\curveto(539.49224161,1042.29047068)(539.66724144,1042.32547064)(539.82724587,1042.37547188)
\curveto(540.08724102,1042.45547051)(540.29224081,1042.58047039)(540.44224587,1042.75047188)
\curveto(540.59224051,1042.93047004)(540.7072404,1043.15046982)(540.78724587,1043.41047188)
\curveto(540.8072403,1043.48046949)(540.81724029,1043.55046942)(540.81724587,1043.62047188)
\curveto(540.82724028,1043.70046927)(540.84224026,1043.78046919)(540.86224587,1043.86047188)
\lineto(540.86224587,1043.99547188)
}
}
{
\newrgbcolor{curcolor}{0 0 0}
\pscustom[linestyle=none,fillstyle=solid,fillcolor=curcolor]
{
\newpath
\moveto(549.05552712,1049.02047188)
\curveto(549.1655218,1049.02046395)(549.26052171,1049.01046396)(549.34052712,1048.99047188)
\curveto(549.43052154,1048.970464)(549.50052147,1048.92546404)(549.55052712,1048.85547188)
\curveto(549.61052136,1048.77546419)(549.64052133,1048.63546433)(549.64052712,1048.43547188)
\lineto(549.64052712,1047.92547188)
\lineto(549.64052712,1047.55047188)
\curveto(549.65052132,1047.41046556)(549.63552133,1047.30046567)(549.59552712,1047.22047188)
\curveto(549.55552141,1047.15046582)(549.49552147,1047.10546586)(549.41552712,1047.08547188)
\curveto(549.34552162,1047.0654659)(549.26052171,1047.05546591)(549.16052712,1047.05547188)
\curveto(549.0705219,1047.05546591)(548.970522,1047.06046591)(548.86052712,1047.07047188)
\curveto(548.76052221,1047.08046589)(548.6655223,1047.07546589)(548.57552712,1047.05547188)
\curveto(548.50552246,1047.03546593)(548.43552253,1047.02046595)(548.36552712,1047.01047188)
\curveto(548.29552267,1047.01046596)(548.23052274,1047.00046597)(548.17052712,1046.98047188)
\curveto(548.01052296,1046.93046604)(547.85052312,1046.85546611)(547.69052712,1046.75547188)
\curveto(547.53052344,1046.6654663)(547.40552356,1046.56046641)(547.31552712,1046.44047188)
\curveto(547.2655237,1046.36046661)(547.21052376,1046.27546669)(547.15052712,1046.18547188)
\curveto(547.10052387,1046.10546686)(547.05052392,1046.02046695)(547.00052712,1045.93047188)
\curveto(546.970524,1045.85046712)(546.94052403,1045.7654672)(546.91052712,1045.67547188)
\lineto(546.85052712,1045.43547188)
\curveto(546.83052414,1045.3654676)(546.82052415,1045.29046768)(546.82052712,1045.21047188)
\curveto(546.82052415,1045.14046783)(546.81052416,1045.0704679)(546.79052712,1045.00047188)
\curveto(546.78052419,1044.96046801)(546.77552419,1044.92046805)(546.77552712,1044.88047188)
\curveto(546.78552418,1044.85046812)(546.78552418,1044.82046815)(546.77552712,1044.79047188)
\lineto(546.77552712,1044.55047188)
\curveto(546.75552421,1044.48046849)(546.75052422,1044.40046857)(546.76052712,1044.31047188)
\curveto(546.7705242,1044.23046874)(546.77552419,1044.15046882)(546.77552712,1044.07047188)
\lineto(546.77552712,1043.11047188)
\lineto(546.77552712,1041.83547188)
\curveto(546.77552419,1041.70547126)(546.7705242,1041.58547138)(546.76052712,1041.47547188)
\curveto(546.75052422,1041.3654716)(546.72052425,1041.27547169)(546.67052712,1041.20547188)
\curveto(546.65052432,1041.17547179)(546.61552435,1041.15047182)(546.56552712,1041.13047188)
\curveto(546.52552444,1041.12047185)(546.48052449,1041.11047186)(546.43052712,1041.10047188)
\lineto(546.35552712,1041.10047188)
\curveto(546.30552466,1041.09047188)(546.25052472,1041.08547188)(546.19052712,1041.08547188)
\lineto(546.02552712,1041.08547188)
\lineto(545.38052712,1041.08547188)
\curveto(545.32052565,1041.09547187)(545.25552571,1041.10047187)(545.18552712,1041.10047188)
\lineto(544.99052712,1041.10047188)
\curveto(544.94052603,1041.12047185)(544.89052608,1041.13547183)(544.84052712,1041.14547188)
\curveto(544.79052618,1041.1654718)(544.75552621,1041.20047177)(544.73552712,1041.25047188)
\curveto(544.69552627,1041.30047167)(544.6705263,1041.3704716)(544.66052712,1041.46047188)
\lineto(544.66052712,1041.76047188)
\lineto(544.66052712,1042.78047188)
\lineto(544.66052712,1047.01047188)
\lineto(544.66052712,1048.12047188)
\lineto(544.66052712,1048.40547188)
\curveto(544.66052631,1048.50546446)(544.68052629,1048.58546438)(544.72052712,1048.64547188)
\curveto(544.7705262,1048.72546424)(544.84552612,1048.77546419)(544.94552712,1048.79547188)
\curveto(545.04552592,1048.81546415)(545.1655258,1048.82546414)(545.30552712,1048.82547188)
\lineto(546.07052712,1048.82547188)
\curveto(546.19052478,1048.82546414)(546.29552467,1048.81546415)(546.38552712,1048.79547188)
\curveto(546.47552449,1048.78546418)(546.54552442,1048.74046423)(546.59552712,1048.66047188)
\curveto(546.62552434,1048.61046436)(546.64052433,1048.54046443)(546.64052712,1048.45047188)
\lineto(546.67052712,1048.18047188)
\curveto(546.68052429,1048.10046487)(546.69552427,1048.02546494)(546.71552712,1047.95547188)
\curveto(546.74552422,1047.88546508)(546.79552417,1047.85046512)(546.86552712,1047.85047188)
\curveto(546.88552408,1047.8704651)(546.90552406,1047.88046509)(546.92552712,1047.88047188)
\curveto(546.94552402,1047.88046509)(546.965524,1047.89046508)(546.98552712,1047.91047188)
\curveto(547.04552392,1047.96046501)(547.09552387,1048.01546495)(547.13552712,1048.07547188)
\curveto(547.18552378,1048.14546482)(547.24552372,1048.20546476)(547.31552712,1048.25547188)
\curveto(547.35552361,1048.28546468)(547.39052358,1048.31546465)(547.42052712,1048.34547188)
\curveto(547.45052352,1048.38546458)(547.48552348,1048.42046455)(547.52552712,1048.45047188)
\lineto(547.79552712,1048.63047188)
\curveto(547.89552307,1048.69046428)(547.99552297,1048.74546422)(548.09552712,1048.79547188)
\curveto(548.19552277,1048.83546413)(548.29552267,1048.8704641)(548.39552712,1048.90047188)
\lineto(548.72552712,1048.99047188)
\curveto(548.75552221,1049.00046397)(548.81052216,1049.00046397)(548.89052712,1048.99047188)
\curveto(548.98052199,1048.99046398)(549.03552193,1049.00046397)(549.05552712,1049.02047188)
}
}
{
\newrgbcolor{curcolor}{0 0 0}
\pscustom[linestyle=none,fillstyle=solid,fillcolor=curcolor]
{
\newpath
\moveto(557.96693337,1045.27047188)
\curveto(557.9869248,1045.21046776)(557.99692479,1045.12546784)(557.99693337,1045.01547188)
\curveto(557.99692479,1044.90546806)(557.9869248,1044.82046815)(557.96693337,1044.76047188)
\lineto(557.96693337,1044.61047188)
\curveto(557.94692484,1044.53046844)(557.93692485,1044.45046852)(557.93693337,1044.37047188)
\curveto(557.94692484,1044.29046868)(557.94192484,1044.21046876)(557.92193337,1044.13047188)
\curveto(557.90192488,1044.06046891)(557.8869249,1043.99546897)(557.87693337,1043.93547188)
\curveto(557.86692492,1043.87546909)(557.85692493,1043.81046916)(557.84693337,1043.74047188)
\curveto(557.80692498,1043.63046934)(557.77192501,1043.51546945)(557.74193337,1043.39547188)
\curveto(557.71192507,1043.28546968)(557.67192511,1043.18046979)(557.62193337,1043.08047188)
\curveto(557.41192537,1042.60047037)(557.13692565,1042.21047076)(556.79693337,1041.91047188)
\curveto(556.45692633,1041.61047136)(556.04692674,1041.36047161)(555.56693337,1041.16047188)
\curveto(555.44692734,1041.11047186)(555.32192746,1041.07547189)(555.19193337,1041.05547188)
\curveto(555.07192771,1041.02547194)(554.94692784,1040.99547197)(554.81693337,1040.96547188)
\curveto(554.76692802,1040.94547202)(554.71192807,1040.93547203)(554.65193337,1040.93547188)
\curveto(554.59192819,1040.93547203)(554.53692825,1040.93047204)(554.48693337,1040.92047188)
\lineto(554.38193337,1040.92047188)
\curveto(554.35192843,1040.91047206)(554.32192846,1040.90547206)(554.29193337,1040.90547188)
\curveto(554.24192854,1040.89547207)(554.16192862,1040.89047208)(554.05193337,1040.89047188)
\curveto(553.94192884,1040.88047209)(553.85692893,1040.88547208)(553.79693337,1040.90547188)
\lineto(553.64693337,1040.90547188)
\curveto(553.59692919,1040.91547205)(553.54192924,1040.92047205)(553.48193337,1040.92047188)
\curveto(553.43192935,1040.91047206)(553.3819294,1040.91547205)(553.33193337,1040.93547188)
\curveto(553.29192949,1040.94547202)(553.25192953,1040.95047202)(553.21193337,1040.95047188)
\curveto(553.1819296,1040.95047202)(553.14192964,1040.95547201)(553.09193337,1040.96547188)
\curveto(552.99192979,1040.99547197)(552.89192989,1041.02047195)(552.79193337,1041.04047188)
\curveto(552.69193009,1041.06047191)(552.59693019,1041.09047188)(552.50693337,1041.13047188)
\curveto(552.3869304,1041.1704718)(552.27193051,1041.21047176)(552.16193337,1041.25047188)
\curveto(552.06193072,1041.29047168)(551.95693083,1041.34047163)(551.84693337,1041.40047188)
\curveto(551.49693129,1041.61047136)(551.19693159,1041.85547111)(550.94693337,1042.13547188)
\curveto(550.69693209,1042.41547055)(550.4869323,1042.75047022)(550.31693337,1043.14047188)
\curveto(550.26693252,1043.23046974)(550.22693256,1043.32546964)(550.19693337,1043.42547188)
\curveto(550.17693261,1043.52546944)(550.15193263,1043.63046934)(550.12193337,1043.74047188)
\curveto(550.10193268,1043.79046918)(550.09193269,1043.83546913)(550.09193337,1043.87547188)
\curveto(550.09193269,1043.91546905)(550.0819327,1043.96046901)(550.06193337,1044.01047188)
\curveto(550.04193274,1044.09046888)(550.03193275,1044.1704688)(550.03193337,1044.25047188)
\curveto(550.03193275,1044.34046863)(550.02193276,1044.42546854)(550.00193337,1044.50547188)
\curveto(549.99193279,1044.55546841)(549.9869328,1044.60046837)(549.98693337,1044.64047188)
\lineto(549.98693337,1044.77547188)
\curveto(549.96693282,1044.83546813)(549.95693283,1044.92046805)(549.95693337,1045.03047188)
\curveto(549.96693282,1045.14046783)(549.9819328,1045.22546774)(550.00193337,1045.28547188)
\lineto(550.00193337,1045.39047188)
\curveto(550.01193277,1045.44046753)(550.01193277,1045.49046748)(550.00193337,1045.54047188)
\curveto(550.00193278,1045.60046737)(550.01193277,1045.65546731)(550.03193337,1045.70547188)
\curveto(550.04193274,1045.75546721)(550.04693274,1045.80046717)(550.04693337,1045.84047188)
\curveto(550.04693274,1045.89046708)(550.05693273,1045.94046703)(550.07693337,1045.99047188)
\curveto(550.11693267,1046.12046685)(550.15193263,1046.24546672)(550.18193337,1046.36547188)
\curveto(550.21193257,1046.49546647)(550.25193253,1046.62046635)(550.30193337,1046.74047188)
\curveto(550.4819323,1047.15046582)(550.69693209,1047.49046548)(550.94693337,1047.76047188)
\curveto(551.19693159,1048.04046493)(551.50193128,1048.29546467)(551.86193337,1048.52547188)
\curveto(551.96193082,1048.57546439)(552.06693072,1048.62046435)(552.17693337,1048.66047188)
\curveto(552.2869305,1048.70046427)(552.39693039,1048.74546422)(552.50693337,1048.79547188)
\curveto(552.63693015,1048.84546412)(552.77193001,1048.88046409)(552.91193337,1048.90047188)
\curveto(553.05192973,1048.92046405)(553.19692959,1048.95046402)(553.34693337,1048.99047188)
\curveto(553.42692936,1049.00046397)(553.50192928,1049.00546396)(553.57193337,1049.00547188)
\curveto(553.64192914,1049.00546396)(553.71192907,1049.01046396)(553.78193337,1049.02047188)
\curveto(554.36192842,1049.03046394)(554.86192792,1048.970464)(555.28193337,1048.84047188)
\curveto(555.71192707,1048.71046426)(556.09192669,1048.53046444)(556.42193337,1048.30047188)
\curveto(556.53192625,1048.22046475)(556.64192614,1048.13046484)(556.75193337,1048.03047188)
\curveto(556.87192591,1047.94046503)(556.97192581,1047.84046513)(557.05193337,1047.73047188)
\curveto(557.13192565,1047.63046534)(557.20192558,1047.53046544)(557.26193337,1047.43047188)
\curveto(557.33192545,1047.33046564)(557.40192538,1047.22546574)(557.47193337,1047.11547188)
\curveto(557.54192524,1047.00546596)(557.59692519,1046.88546608)(557.63693337,1046.75547188)
\curveto(557.67692511,1046.63546633)(557.72192506,1046.50546646)(557.77193337,1046.36547188)
\curveto(557.80192498,1046.28546668)(557.82692496,1046.20046677)(557.84693337,1046.11047188)
\lineto(557.90693337,1045.84047188)
\curveto(557.91692487,1045.80046717)(557.92192486,1045.76046721)(557.92193337,1045.72047188)
\curveto(557.92192486,1045.68046729)(557.92692486,1045.64046733)(557.93693337,1045.60047188)
\curveto(557.95692483,1045.55046742)(557.96192482,1045.49546747)(557.95193337,1045.43547188)
\curveto(557.94192484,1045.37546759)(557.94692484,1045.32046765)(557.96693337,1045.27047188)
\moveto(555.86693337,1044.73047188)
\curveto(555.87692691,1044.78046819)(555.8819269,1044.85046812)(555.88193337,1044.94047188)
\curveto(555.8819269,1045.04046793)(555.87692691,1045.11546785)(555.86693337,1045.16547188)
\lineto(555.86693337,1045.28547188)
\curveto(555.84692694,1045.33546763)(555.83692695,1045.39046758)(555.83693337,1045.45047188)
\curveto(555.83692695,1045.51046746)(555.83192695,1045.5654674)(555.82193337,1045.61547188)
\curveto(555.82192696,1045.65546731)(555.81692697,1045.68546728)(555.80693337,1045.70547188)
\lineto(555.74693337,1045.94547188)
\curveto(555.73692705,1046.03546693)(555.71692707,1046.12046685)(555.68693337,1046.20047188)
\curveto(555.57692721,1046.46046651)(555.44692734,1046.68046629)(555.29693337,1046.86047188)
\curveto(555.14692764,1047.05046592)(554.94692784,1047.20046577)(554.69693337,1047.31047188)
\curveto(554.63692815,1047.33046564)(554.57692821,1047.34546562)(554.51693337,1047.35547188)
\curveto(554.45692833,1047.37546559)(554.39192839,1047.39546557)(554.32193337,1047.41547188)
\curveto(554.24192854,1047.43546553)(554.15692863,1047.44046553)(554.06693337,1047.43047188)
\lineto(553.79693337,1047.43047188)
\curveto(553.76692902,1047.41046556)(553.73192905,1047.40046557)(553.69193337,1047.40047188)
\curveto(553.65192913,1047.41046556)(553.61692917,1047.41046556)(553.58693337,1047.40047188)
\lineto(553.37693337,1047.34047188)
\curveto(553.31692947,1047.33046564)(553.26192952,1047.31046566)(553.21193337,1047.28047188)
\curveto(552.96192982,1047.1704658)(552.75693003,1047.01046596)(552.59693337,1046.80047188)
\curveto(552.44693034,1046.60046637)(552.32693046,1046.3654666)(552.23693337,1046.09547188)
\curveto(552.20693058,1045.99546697)(552.1819306,1045.89046708)(552.16193337,1045.78047188)
\curveto(552.15193063,1045.6704673)(552.13693065,1045.56046741)(552.11693337,1045.45047188)
\curveto(552.10693068,1045.40046757)(552.10193068,1045.35046762)(552.10193337,1045.30047188)
\lineto(552.10193337,1045.15047188)
\curveto(552.0819307,1045.08046789)(552.07193071,1044.97546799)(552.07193337,1044.83547188)
\curveto(552.0819307,1044.69546827)(552.09693069,1044.59046838)(552.11693337,1044.52047188)
\lineto(552.11693337,1044.38547188)
\curveto(552.13693065,1044.30546866)(552.15193063,1044.22546874)(552.16193337,1044.14547188)
\curveto(552.17193061,1044.07546889)(552.1869306,1044.00046897)(552.20693337,1043.92047188)
\curveto(552.30693048,1043.62046935)(552.41193037,1043.37546959)(552.52193337,1043.18547188)
\curveto(552.64193014,1043.00546996)(552.82692996,1042.84047013)(553.07693337,1042.69047188)
\curveto(553.14692964,1042.64047033)(553.22192956,1042.60047037)(553.30193337,1042.57047188)
\curveto(553.39192939,1042.54047043)(553.4819293,1042.51547045)(553.57193337,1042.49547188)
\curveto(553.61192917,1042.48547048)(553.64692914,1042.48047049)(553.67693337,1042.48047188)
\curveto(553.70692908,1042.49047048)(553.74192904,1042.49047048)(553.78193337,1042.48047188)
\lineto(553.90193337,1042.45047188)
\curveto(553.95192883,1042.45047052)(553.99692879,1042.45547051)(554.03693337,1042.46547188)
\lineto(554.15693337,1042.46547188)
\curveto(554.23692855,1042.48547048)(554.31692847,1042.50047047)(554.39693337,1042.51047188)
\curveto(554.47692831,1042.52047045)(554.55192823,1042.54047043)(554.62193337,1042.57047188)
\curveto(554.8819279,1042.6704703)(555.09192769,1042.80547016)(555.25193337,1042.97547188)
\curveto(555.41192737,1043.14546982)(555.54692724,1043.35546961)(555.65693337,1043.60547188)
\curveto(555.69692709,1043.70546926)(555.72692706,1043.80546916)(555.74693337,1043.90547188)
\curveto(555.76692702,1044.00546896)(555.79192699,1044.11046886)(555.82193337,1044.22047188)
\curveto(555.83192695,1044.26046871)(555.83692695,1044.29546867)(555.83693337,1044.32547188)
\curveto(555.83692695,1044.3654686)(555.84192694,1044.40546856)(555.85193337,1044.44547188)
\lineto(555.85193337,1044.58047188)
\curveto(555.85192693,1044.63046834)(555.85692693,1044.68046829)(555.86693337,1044.73047188)
}
}
{
\newrgbcolor{curcolor}{0 0 0}
\pscustom[linestyle=none,fillstyle=solid,fillcolor=curcolor]
{
\newpath
\moveto(563.79185524,1049.02047188)
\curveto(564.39184944,1049.04046393)(564.89184894,1048.95546401)(565.29185524,1048.76547188)
\curveto(565.69184814,1048.57546439)(566.00684782,1048.29546467)(566.23685524,1047.92547188)
\curveto(566.30684752,1047.81546515)(566.36184747,1047.69546527)(566.40185524,1047.56547188)
\curveto(566.44184739,1047.44546552)(566.48184735,1047.32046565)(566.52185524,1047.19047188)
\curveto(566.54184729,1047.11046586)(566.55184728,1047.03546593)(566.55185524,1046.96547188)
\curveto(566.56184727,1046.89546607)(566.57684725,1046.82546614)(566.59685524,1046.75547188)
\curveto(566.59684723,1046.69546627)(566.60184723,1046.65546631)(566.61185524,1046.63547188)
\curveto(566.6318472,1046.49546647)(566.64184719,1046.35046662)(566.64185524,1046.20047188)
\lineto(566.64185524,1045.76547188)
\lineto(566.64185524,1044.43047188)
\lineto(566.64185524,1042.00047188)
\curveto(566.64184719,1041.81047116)(566.63684719,1041.62547134)(566.62685524,1041.44547188)
\curveto(566.6268472,1041.27547169)(566.55684727,1041.1654718)(566.41685524,1041.11547188)
\curveto(566.35684747,1041.09547187)(566.28684754,1041.08547188)(566.20685524,1041.08547188)
\lineto(565.96685524,1041.08547188)
\lineto(565.15685524,1041.08547188)
\curveto(565.03684879,1041.08547188)(564.9268489,1041.09047188)(564.82685524,1041.10047188)
\curveto(564.73684909,1041.12047185)(564.66684916,1041.1654718)(564.61685524,1041.23547188)
\curveto(564.57684925,1041.29547167)(564.55184928,1041.3704716)(564.54185524,1041.46047188)
\lineto(564.54185524,1041.77547188)
\lineto(564.54185524,1042.82547188)
\lineto(564.54185524,1045.06047188)
\curveto(564.54184929,1045.43046754)(564.5268493,1045.7704672)(564.49685524,1046.08047188)
\curveto(564.46684936,1046.40046657)(564.37684945,1046.6704663)(564.22685524,1046.89047188)
\curveto(564.08684974,1047.09046588)(563.88184995,1047.23046574)(563.61185524,1047.31047188)
\curveto(563.56185027,1047.33046564)(563.50685032,1047.34046563)(563.44685524,1047.34047188)
\curveto(563.39685043,1047.34046563)(563.34185049,1047.35046562)(563.28185524,1047.37047188)
\curveto(563.2318506,1047.38046559)(563.16685066,1047.38046559)(563.08685524,1047.37047188)
\curveto(563.01685081,1047.3704656)(562.96185087,1047.3654656)(562.92185524,1047.35547188)
\curveto(562.88185095,1047.34546562)(562.84685098,1047.34046563)(562.81685524,1047.34047188)
\curveto(562.78685104,1047.34046563)(562.75685107,1047.33546563)(562.72685524,1047.32547188)
\curveto(562.49685133,1047.2654657)(562.31185152,1047.18546578)(562.17185524,1047.08547188)
\curveto(561.85185198,1046.85546611)(561.66185217,1046.52046645)(561.60185524,1046.08047188)
\curveto(561.54185229,1045.64046733)(561.51185232,1045.14546782)(561.51185524,1044.59547188)
\lineto(561.51185524,1042.72047188)
\lineto(561.51185524,1041.80547188)
\lineto(561.51185524,1041.53547188)
\curveto(561.51185232,1041.44547152)(561.49685233,1041.3704716)(561.46685524,1041.31047188)
\curveto(561.41685241,1041.20047177)(561.33685249,1041.13547183)(561.22685524,1041.11547188)
\curveto(561.11685271,1041.09547187)(560.98185285,1041.08547188)(560.82185524,1041.08547188)
\lineto(560.07185524,1041.08547188)
\curveto(559.96185387,1041.08547188)(559.85185398,1041.09047188)(559.74185524,1041.10047188)
\curveto(559.6318542,1041.11047186)(559.55185428,1041.14547182)(559.50185524,1041.20547188)
\curveto(559.4318544,1041.29547167)(559.39685443,1041.42547154)(559.39685524,1041.59547188)
\curveto(559.40685442,1041.7654712)(559.41185442,1041.92547104)(559.41185524,1042.07547188)
\lineto(559.41185524,1044.11547188)
\lineto(559.41185524,1047.41547188)
\lineto(559.41185524,1048.18047188)
\lineto(559.41185524,1048.48047188)
\curveto(559.42185441,1048.5704644)(559.45185438,1048.64546432)(559.50185524,1048.70547188)
\curveto(559.52185431,1048.73546423)(559.55185428,1048.75546421)(559.59185524,1048.76547188)
\curveto(559.64185419,1048.78546418)(559.69185414,1048.80046417)(559.74185524,1048.81047188)
\lineto(559.81685524,1048.81047188)
\curveto(559.86685396,1048.82046415)(559.91685391,1048.82546414)(559.96685524,1048.82547188)
\lineto(560.13185524,1048.82547188)
\lineto(560.76185524,1048.82547188)
\curveto(560.84185299,1048.82546414)(560.91685291,1048.82046415)(560.98685524,1048.81047188)
\curveto(561.06685276,1048.81046416)(561.13685269,1048.80046417)(561.19685524,1048.78047188)
\curveto(561.26685256,1048.75046422)(561.31185252,1048.70546426)(561.33185524,1048.64547188)
\curveto(561.36185247,1048.58546438)(561.38685244,1048.51546445)(561.40685524,1048.43547188)
\curveto(561.41685241,1048.39546457)(561.41685241,1048.36046461)(561.40685524,1048.33047188)
\curveto(561.40685242,1048.30046467)(561.41685241,1048.2704647)(561.43685524,1048.24047188)
\curveto(561.45685237,1048.19046478)(561.47185236,1048.16046481)(561.48185524,1048.15047188)
\curveto(561.50185233,1048.14046483)(561.5268523,1048.12546484)(561.55685524,1048.10547188)
\curveto(561.66685216,1048.09546487)(561.75685207,1048.13046484)(561.82685524,1048.21047188)
\curveto(561.89685193,1048.30046467)(561.97185186,1048.3704646)(562.05185524,1048.42047188)
\curveto(562.32185151,1048.62046435)(562.62185121,1048.78046419)(562.95185524,1048.90047188)
\curveto(563.04185079,1048.93046404)(563.1318507,1048.95046402)(563.22185524,1048.96047188)
\curveto(563.32185051,1048.970464)(563.4268504,1048.98546398)(563.53685524,1049.00547188)
\curveto(563.56685026,1049.01546395)(563.61185022,1049.01546395)(563.67185524,1049.00547188)
\curveto(563.7318501,1049.00546396)(563.77185006,1049.01046396)(563.79185524,1049.02047188)
}
}
{
\newrgbcolor{curcolor}{0 0 0}
\pscustom[linestyle=none,fillstyle=solid,fillcolor=curcolor]
{
}
}
{
\newrgbcolor{curcolor}{0 0 0}
\pscustom[linestyle=none,fillstyle=solid,fillcolor=curcolor]
{
\newpath
\moveto(579.48326149,1041.68547188)
\curveto(579.50325364,1041.57547139)(579.51325363,1041.4654715)(579.51326149,1041.35547188)
\curveto(579.52325362,1041.24547172)(579.47325367,1041.1704718)(579.36326149,1041.13047188)
\curveto(579.30325384,1041.10047187)(579.23325391,1041.08547188)(579.15326149,1041.08547188)
\lineto(578.91326149,1041.08547188)
\lineto(578.10326149,1041.08547188)
\lineto(577.83326149,1041.08547188)
\curveto(577.75325539,1041.09547187)(577.68825546,1041.12047185)(577.63826149,1041.16047188)
\curveto(577.56825558,1041.20047177)(577.51325563,1041.25547171)(577.47326149,1041.32547188)
\curveto(577.4432557,1041.40547156)(577.39825575,1041.4704715)(577.33826149,1041.52047188)
\curveto(577.31825583,1041.54047143)(577.29325585,1041.55547141)(577.26326149,1041.56547188)
\curveto(577.23325591,1041.58547138)(577.19325595,1041.59047138)(577.14326149,1041.58047188)
\curveto(577.09325605,1041.56047141)(577.0432561,1041.53547143)(576.99326149,1041.50547188)
\curveto(576.95325619,1041.47547149)(576.90825624,1041.45047152)(576.85826149,1041.43047188)
\curveto(576.80825634,1041.39047158)(576.75325639,1041.35547161)(576.69326149,1041.32547188)
\lineto(576.51326149,1041.23547188)
\curveto(576.38325676,1041.17547179)(576.2482569,1041.12547184)(576.10826149,1041.08547188)
\curveto(575.96825718,1041.05547191)(575.82325732,1041.02047195)(575.67326149,1040.98047188)
\curveto(575.60325754,1040.96047201)(575.53325761,1040.95047202)(575.46326149,1040.95047188)
\curveto(575.40325774,1040.94047203)(575.33825781,1040.93047204)(575.26826149,1040.92047188)
\lineto(575.17826149,1040.92047188)
\curveto(575.148258,1040.91047206)(575.11825803,1040.90547206)(575.08826149,1040.90547188)
\lineto(574.92326149,1040.90547188)
\curveto(574.82325832,1040.88547208)(574.72325842,1040.88547208)(574.62326149,1040.90547188)
\lineto(574.48826149,1040.90547188)
\curveto(574.41825873,1040.92547204)(574.3482588,1040.93547203)(574.27826149,1040.93547188)
\curveto(574.21825893,1040.92547204)(574.15825899,1040.93047204)(574.09826149,1040.95047188)
\curveto(573.99825915,1040.970472)(573.90325924,1040.99047198)(573.81326149,1041.01047188)
\curveto(573.72325942,1041.02047195)(573.63825951,1041.04547192)(573.55826149,1041.08547188)
\curveto(573.26825988,1041.19547177)(573.01826013,1041.33547163)(572.80826149,1041.50547188)
\curveto(572.60826054,1041.68547128)(572.4482607,1041.92047105)(572.32826149,1042.21047188)
\curveto(572.29826085,1042.28047069)(572.26826088,1042.35547061)(572.23826149,1042.43547188)
\curveto(572.21826093,1042.51547045)(572.19826095,1042.60047037)(572.17826149,1042.69047188)
\curveto(572.15826099,1042.74047023)(572.148261,1042.79047018)(572.14826149,1042.84047188)
\curveto(572.15826099,1042.89047008)(572.15826099,1042.94047003)(572.14826149,1042.99047188)
\curveto(572.13826101,1043.02046995)(572.12826102,1043.08046989)(572.11826149,1043.17047188)
\curveto(572.11826103,1043.2704697)(572.12326102,1043.34046963)(572.13326149,1043.38047188)
\curveto(572.15326099,1043.48046949)(572.16326098,1043.5654694)(572.16326149,1043.63547188)
\lineto(572.25326149,1043.96547188)
\curveto(572.28326086,1044.08546888)(572.32326082,1044.19046878)(572.37326149,1044.28047188)
\curveto(572.5432606,1044.5704684)(572.73826041,1044.79046818)(572.95826149,1044.94047188)
\curveto(573.17825997,1045.09046788)(573.45825969,1045.22046775)(573.79826149,1045.33047188)
\curveto(573.92825922,1045.38046759)(574.06325908,1045.41546755)(574.20326149,1045.43547188)
\curveto(574.3432588,1045.45546751)(574.48325866,1045.48046749)(574.62326149,1045.51047188)
\curveto(574.70325844,1045.53046744)(574.78825836,1045.54046743)(574.87826149,1045.54047188)
\curveto(574.96825818,1045.55046742)(575.05825809,1045.5654674)(575.14826149,1045.58547188)
\curveto(575.21825793,1045.60546736)(575.28825786,1045.61046736)(575.35826149,1045.60047188)
\curveto(575.42825772,1045.60046737)(575.50325764,1045.61046736)(575.58326149,1045.63047188)
\curveto(575.65325749,1045.65046732)(575.72325742,1045.66046731)(575.79326149,1045.66047188)
\curveto(575.86325728,1045.66046731)(575.93825721,1045.6704673)(576.01826149,1045.69047188)
\curveto(576.22825692,1045.74046723)(576.41825673,1045.78046719)(576.58826149,1045.81047188)
\curveto(576.76825638,1045.85046712)(576.92825622,1045.94046703)(577.06826149,1046.08047188)
\curveto(577.15825599,1046.1704668)(577.21825593,1046.2704667)(577.24826149,1046.38047188)
\curveto(577.25825589,1046.41046656)(577.25825589,1046.43546653)(577.24826149,1046.45547188)
\curveto(577.2482559,1046.47546649)(577.25325589,1046.49546647)(577.26326149,1046.51547188)
\curveto(577.27325587,1046.53546643)(577.27825587,1046.5654664)(577.27826149,1046.60547188)
\lineto(577.27826149,1046.69547188)
\lineto(577.24826149,1046.81547188)
\curveto(577.2482559,1046.85546611)(577.2432559,1046.89046608)(577.23326149,1046.92047188)
\curveto(577.13325601,1047.22046575)(576.92325622,1047.42546554)(576.60326149,1047.53547188)
\curveto(576.51325663,1047.5654654)(576.40325674,1047.58546538)(576.27326149,1047.59547188)
\curveto(576.15325699,1047.61546535)(576.02825712,1047.62046535)(575.89826149,1047.61047188)
\curveto(575.76825738,1047.61046536)(575.6432575,1047.60046537)(575.52326149,1047.58047188)
\curveto(575.40325774,1047.56046541)(575.29825785,1047.53546543)(575.20826149,1047.50547188)
\curveto(575.148258,1047.48546548)(575.08825806,1047.45546551)(575.02826149,1047.41547188)
\curveto(574.97825817,1047.38546558)(574.92825822,1047.35046562)(574.87826149,1047.31047188)
\curveto(574.82825832,1047.2704657)(574.77325837,1047.21546575)(574.71326149,1047.14547188)
\curveto(574.66325848,1047.07546589)(574.62825852,1047.01046596)(574.60826149,1046.95047188)
\curveto(574.55825859,1046.85046612)(574.51325863,1046.75546621)(574.47326149,1046.66547188)
\curveto(574.4432587,1046.57546639)(574.37325877,1046.51546645)(574.26326149,1046.48547188)
\curveto(574.18325896,1046.4654665)(574.09825905,1046.45546651)(574.00826149,1046.45547188)
\lineto(573.73826149,1046.45547188)
\lineto(573.16826149,1046.45547188)
\curveto(573.11826003,1046.45546651)(573.06826008,1046.45046652)(573.01826149,1046.44047188)
\curveto(572.96826018,1046.44046653)(572.92326022,1046.44546652)(572.88326149,1046.45547188)
\lineto(572.74826149,1046.45547188)
\curveto(572.72826042,1046.4654665)(572.70326044,1046.4704665)(572.67326149,1046.47047188)
\curveto(572.6432605,1046.4704665)(572.61826053,1046.48046649)(572.59826149,1046.50047188)
\curveto(572.51826063,1046.52046645)(572.46326068,1046.58546638)(572.43326149,1046.69547188)
\curveto(572.42326072,1046.74546622)(572.42326072,1046.79546617)(572.43326149,1046.84547188)
\curveto(572.4432607,1046.89546607)(572.45326069,1046.94046603)(572.46326149,1046.98047188)
\curveto(572.49326065,1047.09046588)(572.52326062,1047.19046578)(572.55326149,1047.28047188)
\curveto(572.59326055,1047.38046559)(572.63826051,1047.4704655)(572.68826149,1047.55047188)
\lineto(572.77826149,1047.70047188)
\lineto(572.86826149,1047.85047188)
\curveto(572.9482602,1047.96046501)(573.0482601,1048.0654649)(573.16826149,1048.16547188)
\curveto(573.18825996,1048.17546479)(573.21825993,1048.20046477)(573.25826149,1048.24047188)
\curveto(573.30825984,1048.28046469)(573.35325979,1048.31546465)(573.39326149,1048.34547188)
\curveto(573.43325971,1048.37546459)(573.47825967,1048.40546456)(573.52826149,1048.43547188)
\curveto(573.69825945,1048.54546442)(573.87825927,1048.63046434)(574.06826149,1048.69047188)
\curveto(574.25825889,1048.76046421)(574.45325869,1048.82546414)(574.65326149,1048.88547188)
\curveto(574.77325837,1048.91546405)(574.89825825,1048.93546403)(575.02826149,1048.94547188)
\curveto(575.15825799,1048.95546401)(575.28825786,1048.97546399)(575.41826149,1049.00547188)
\curveto(575.45825769,1049.01546395)(575.51825763,1049.01546395)(575.59826149,1049.00547188)
\curveto(575.68825746,1048.99546397)(575.7432574,1049.00046397)(575.76326149,1049.02047188)
\curveto(576.17325697,1049.03046394)(576.56325658,1049.01546395)(576.93326149,1048.97547188)
\curveto(577.31325583,1048.93546403)(577.65325549,1048.86046411)(577.95326149,1048.75047188)
\curveto(578.26325488,1048.64046433)(578.52825462,1048.49046448)(578.74826149,1048.30047188)
\curveto(578.96825418,1048.12046485)(579.13825401,1047.88546508)(579.25826149,1047.59547188)
\curveto(579.32825382,1047.42546554)(579.36825378,1047.23046574)(579.37826149,1047.01047188)
\curveto(579.38825376,1046.79046618)(579.39325375,1046.5654664)(579.39326149,1046.33547188)
\lineto(579.39326149,1042.99047188)
\lineto(579.39326149,1042.40547188)
\curveto(579.39325375,1042.21547075)(579.41325373,1042.04047093)(579.45326149,1041.88047188)
\curveto(579.46325368,1041.85047112)(579.46825368,1041.81547115)(579.46826149,1041.77547188)
\curveto(579.46825368,1041.74547122)(579.47325367,1041.71547125)(579.48326149,1041.68547188)
\moveto(577.27826149,1043.99547188)
\curveto(577.28825586,1044.04546892)(577.29325585,1044.10046887)(577.29326149,1044.16047188)
\curveto(577.29325585,1044.23046874)(577.28825586,1044.29046868)(577.27826149,1044.34047188)
\curveto(577.25825589,1044.40046857)(577.2482559,1044.45546851)(577.24826149,1044.50547188)
\curveto(577.2482559,1044.55546841)(577.22825592,1044.59546837)(577.18826149,1044.62547188)
\curveto(577.13825601,1044.6654683)(577.06325608,1044.68546828)(576.96326149,1044.68547188)
\curveto(576.92325622,1044.67546829)(576.88825626,1044.6654683)(576.85826149,1044.65547188)
\curveto(576.82825632,1044.65546831)(576.79325635,1044.65046832)(576.75326149,1044.64047188)
\curveto(576.68325646,1044.62046835)(576.60825654,1044.60546836)(576.52826149,1044.59547188)
\curveto(576.4482567,1044.58546838)(576.36825678,1044.5704684)(576.28826149,1044.55047188)
\curveto(576.25825689,1044.54046843)(576.21325693,1044.53546843)(576.15326149,1044.53547188)
\curveto(576.02325712,1044.50546846)(575.89325725,1044.48546848)(575.76326149,1044.47547188)
\curveto(575.63325751,1044.4654685)(575.50825764,1044.44046853)(575.38826149,1044.40047188)
\curveto(575.30825784,1044.38046859)(575.23325791,1044.36046861)(575.16326149,1044.34047188)
\curveto(575.09325805,1044.33046864)(575.02325812,1044.31046866)(574.95326149,1044.28047188)
\curveto(574.7432584,1044.19046878)(574.56325858,1044.05546891)(574.41326149,1043.87547188)
\curveto(574.27325887,1043.69546927)(574.22325892,1043.44546952)(574.26326149,1043.12547188)
\curveto(574.28325886,1042.95547001)(574.33825881,1042.81547015)(574.42826149,1042.70547188)
\curveto(574.49825865,1042.59547037)(574.60325854,1042.50547046)(574.74326149,1042.43547188)
\curveto(574.88325826,1042.37547059)(575.03325811,1042.33047064)(575.19326149,1042.30047188)
\curveto(575.36325778,1042.2704707)(575.53825761,1042.26047071)(575.71826149,1042.27047188)
\curveto(575.90825724,1042.29047068)(576.08325706,1042.32547064)(576.24326149,1042.37547188)
\curveto(576.50325664,1042.45547051)(576.70825644,1042.58047039)(576.85826149,1042.75047188)
\curveto(577.00825614,1042.93047004)(577.12325602,1043.15046982)(577.20326149,1043.41047188)
\curveto(577.22325592,1043.48046949)(577.23325591,1043.55046942)(577.23326149,1043.62047188)
\curveto(577.2432559,1043.70046927)(577.25825589,1043.78046919)(577.27826149,1043.86047188)
\lineto(577.27826149,1043.99547188)
}
}
{
\newrgbcolor{curcolor}{0 0 0}
\pscustom[linestyle=none,fillstyle=solid,fillcolor=curcolor]
{
\newpath
\moveto(581.55654274,1051.78047188)
\lineto(582.65154274,1051.78047188)
\curveto(582.75154026,1051.78046119)(582.84654016,1051.77546119)(582.93654274,1051.76547188)
\curveto(583.02653998,1051.75546121)(583.09653991,1051.72546124)(583.14654274,1051.67547188)
\curveto(583.2065398,1051.60546136)(583.23653977,1051.51046146)(583.23654274,1051.39047188)
\curveto(583.24653976,1051.28046169)(583.25153976,1051.1654618)(583.25154274,1051.04547188)
\lineto(583.25154274,1049.71047188)
\lineto(583.25154274,1044.32547188)
\lineto(583.25154274,1042.03047188)
\lineto(583.25154274,1041.61047188)
\curveto(583.26153975,1041.46047151)(583.24153977,1041.34547162)(583.19154274,1041.26547188)
\curveto(583.14153987,1041.18547178)(583.05153996,1041.13047184)(582.92154274,1041.10047188)
\curveto(582.86154015,1041.08047189)(582.79154022,1041.07547189)(582.71154274,1041.08547188)
\curveto(582.64154037,1041.09547187)(582.57154044,1041.10047187)(582.50154274,1041.10047188)
\lineto(581.78154274,1041.10047188)
\curveto(581.67154134,1041.10047187)(581.57154144,1041.10547186)(581.48154274,1041.11547188)
\curveto(581.39154162,1041.12547184)(581.31654169,1041.15547181)(581.25654274,1041.20547188)
\curveto(581.19654181,1041.25547171)(581.16154185,1041.33047164)(581.15154274,1041.43047188)
\lineto(581.15154274,1041.76047188)
\lineto(581.15154274,1043.09547188)
\lineto(581.15154274,1048.72047188)
\lineto(581.15154274,1050.76047188)
\curveto(581.15154186,1050.89046208)(581.14654186,1051.04546192)(581.13654274,1051.22547188)
\curveto(581.13654187,1051.40546156)(581.16154185,1051.53546143)(581.21154274,1051.61547188)
\curveto(581.23154178,1051.65546131)(581.25654175,1051.68546128)(581.28654274,1051.70547188)
\lineto(581.40654274,1051.76547188)
\curveto(581.42654158,1051.7654612)(581.45154156,1051.7654612)(581.48154274,1051.76547188)
\curveto(581.5115415,1051.77546119)(581.53654147,1051.78046119)(581.55654274,1051.78047188)
}
}
{
\newrgbcolor{curcolor}{0 0 0}
\pscustom[linestyle=none,fillstyle=solid,fillcolor=curcolor]
{
}
}
{
\newrgbcolor{curcolor}{0 0 0}
\pscustom[linestyle=none,fillstyle=solid,fillcolor=curcolor]
{
\newpath
\moveto(592.04388649,1049.03547188)
\curveto(592.79388199,1049.05546391)(593.44388134,1048.970464)(593.99388649,1048.78047188)
\curveto(594.55388023,1048.60046437)(594.97887981,1048.28546468)(595.26888649,1047.83547188)
\curveto(595.33887945,1047.72546524)(595.39887939,1047.61046536)(595.44888649,1047.49047188)
\curveto(595.50887928,1047.38046559)(595.55887923,1047.25546571)(595.59888649,1047.11547188)
\curveto(595.61887917,1047.05546591)(595.62887916,1046.99046598)(595.62888649,1046.92047188)
\curveto(595.62887916,1046.85046612)(595.61887917,1046.79046618)(595.59888649,1046.74047188)
\curveto(595.55887923,1046.68046629)(595.50387928,1046.64046633)(595.43388649,1046.62047188)
\curveto(595.3838794,1046.60046637)(595.32387946,1046.59046638)(595.25388649,1046.59047188)
\lineto(595.04388649,1046.59047188)
\lineto(594.38388649,1046.59047188)
\curveto(594.31388047,1046.59046638)(594.24388054,1046.58546638)(594.17388649,1046.57547188)
\curveto(594.10388068,1046.57546639)(594.03888075,1046.58546638)(593.97888649,1046.60547188)
\curveto(593.87888091,1046.62546634)(593.80388098,1046.6654663)(593.75388649,1046.72547188)
\curveto(593.70388108,1046.78546618)(593.65888113,1046.84546612)(593.61888649,1046.90547188)
\lineto(593.49888649,1047.11547188)
\curveto(593.46888132,1047.19546577)(593.41888137,1047.26046571)(593.34888649,1047.31047188)
\curveto(593.24888154,1047.39046558)(593.14888164,1047.45046552)(593.04888649,1047.49047188)
\curveto(592.95888183,1047.53046544)(592.84388194,1047.5654654)(592.70388649,1047.59547188)
\curveto(592.63388215,1047.61546535)(592.52888226,1047.63046534)(592.38888649,1047.64047188)
\curveto(592.25888253,1047.65046532)(592.15888263,1047.64546532)(592.08888649,1047.62547188)
\lineto(591.98388649,1047.62547188)
\lineto(591.83388649,1047.59547188)
\curveto(591.79388299,1047.59546537)(591.74888304,1047.59046538)(591.69888649,1047.58047188)
\curveto(591.52888326,1047.53046544)(591.3888834,1047.46046551)(591.27888649,1047.37047188)
\curveto(591.17888361,1047.29046568)(591.10888368,1047.1654658)(591.06888649,1046.99547188)
\curveto(591.04888374,1046.92546604)(591.04888374,1046.86046611)(591.06888649,1046.80047188)
\curveto(591.0888837,1046.74046623)(591.10888368,1046.69046628)(591.12888649,1046.65047188)
\curveto(591.19888359,1046.53046644)(591.27888351,1046.43546653)(591.36888649,1046.36547188)
\curveto(591.46888332,1046.29546667)(591.5838832,1046.23546673)(591.71388649,1046.18547188)
\curveto(591.90388288,1046.10546686)(592.10888268,1046.03546693)(592.32888649,1045.97547188)
\lineto(593.01888649,1045.82547188)
\curveto(593.25888153,1045.78546718)(593.4888813,1045.73546723)(593.70888649,1045.67547188)
\curveto(593.93888085,1045.62546734)(594.15388063,1045.56046741)(594.35388649,1045.48047188)
\curveto(594.44388034,1045.44046753)(594.52888026,1045.40546756)(594.60888649,1045.37547188)
\curveto(594.69888009,1045.35546761)(594.78388,1045.32046765)(594.86388649,1045.27047188)
\curveto(595.05387973,1045.15046782)(595.22387956,1045.02046795)(595.37388649,1044.88047188)
\curveto(595.53387925,1044.74046823)(595.65887913,1044.5654684)(595.74888649,1044.35547188)
\curveto(595.77887901,1044.28546868)(595.80387898,1044.21546875)(595.82388649,1044.14547188)
\curveto(595.84387894,1044.07546889)(595.86387892,1044.00046897)(595.88388649,1043.92047188)
\curveto(595.89387889,1043.86046911)(595.89887889,1043.7654692)(595.89888649,1043.63547188)
\curveto(595.90887888,1043.51546945)(595.90887888,1043.42046955)(595.89888649,1043.35047188)
\lineto(595.89888649,1043.27547188)
\curveto(595.87887891,1043.21546975)(595.86387892,1043.15546981)(595.85388649,1043.09547188)
\curveto(595.85387893,1043.04546992)(595.84887894,1042.99546997)(595.83888649,1042.94547188)
\curveto(595.76887902,1042.64547032)(595.65887913,1042.38047059)(595.50888649,1042.15047188)
\curveto(595.34887944,1041.91047106)(595.15387963,1041.71547125)(594.92388649,1041.56547188)
\curveto(594.69388009,1041.41547155)(594.43388035,1041.28547168)(594.14388649,1041.17547188)
\curveto(594.03388075,1041.12547184)(593.91388087,1041.09047188)(593.78388649,1041.07047188)
\curveto(593.66388112,1041.05047192)(593.54388124,1041.02547194)(593.42388649,1040.99547188)
\curveto(593.33388145,1040.97547199)(593.23888155,1040.965472)(593.13888649,1040.96547188)
\curveto(593.04888174,1040.95547201)(592.95888183,1040.94047203)(592.86888649,1040.92047188)
\lineto(592.59888649,1040.92047188)
\curveto(592.53888225,1040.90047207)(592.43388235,1040.89047208)(592.28388649,1040.89047188)
\curveto(592.14388264,1040.89047208)(592.04388274,1040.90047207)(591.98388649,1040.92047188)
\curveto(591.95388283,1040.92047205)(591.91888287,1040.92547204)(591.87888649,1040.93547188)
\lineto(591.77388649,1040.93547188)
\curveto(591.65388313,1040.95547201)(591.53388325,1040.970472)(591.41388649,1040.98047188)
\curveto(591.29388349,1040.99047198)(591.17888361,1041.01047196)(591.06888649,1041.04047188)
\curveto(590.67888411,1041.15047182)(590.33388445,1041.27547169)(590.03388649,1041.41547188)
\curveto(589.73388505,1041.5654714)(589.47888531,1041.78547118)(589.26888649,1042.07547188)
\curveto(589.12888566,1042.2654707)(589.00888578,1042.48547048)(588.90888649,1042.73547188)
\curveto(588.8888859,1042.79547017)(588.86888592,1042.87547009)(588.84888649,1042.97547188)
\curveto(588.82888596,1043.02546994)(588.81388597,1043.09546987)(588.80388649,1043.18547188)
\curveto(588.79388599,1043.27546969)(588.79888599,1043.35046962)(588.81888649,1043.41047188)
\curveto(588.84888594,1043.48046949)(588.89888589,1043.53046944)(588.96888649,1043.56047188)
\curveto(589.01888577,1043.58046939)(589.07888571,1043.59046938)(589.14888649,1043.59047188)
\lineto(589.37388649,1043.59047188)
\lineto(590.07888649,1043.59047188)
\lineto(590.31888649,1043.59047188)
\curveto(590.39888439,1043.59046938)(590.46888432,1043.58046939)(590.52888649,1043.56047188)
\curveto(590.63888415,1043.52046945)(590.70888408,1043.45546951)(590.73888649,1043.36547188)
\curveto(590.77888401,1043.27546969)(590.82388396,1043.18046979)(590.87388649,1043.08047188)
\curveto(590.89388389,1043.03046994)(590.92888386,1042.96547)(590.97888649,1042.88547188)
\curveto(591.03888375,1042.80547016)(591.0888837,1042.75547021)(591.12888649,1042.73547188)
\curveto(591.24888354,1042.63547033)(591.36388342,1042.55547041)(591.47388649,1042.49547188)
\curveto(591.5838832,1042.44547052)(591.72388306,1042.39547057)(591.89388649,1042.34547188)
\curveto(591.94388284,1042.32547064)(591.99388279,1042.31547065)(592.04388649,1042.31547188)
\curveto(592.09388269,1042.32547064)(592.14388264,1042.32547064)(592.19388649,1042.31547188)
\curveto(592.27388251,1042.29547067)(592.35888243,1042.28547068)(592.44888649,1042.28547188)
\curveto(592.54888224,1042.29547067)(592.63388215,1042.31047066)(592.70388649,1042.33047188)
\curveto(592.75388203,1042.34047063)(592.79888199,1042.34547062)(592.83888649,1042.34547188)
\curveto(592.8888819,1042.34547062)(592.93888185,1042.35547061)(592.98888649,1042.37547188)
\curveto(593.12888166,1042.42547054)(593.25388153,1042.48547048)(593.36388649,1042.55547188)
\curveto(593.4838813,1042.62547034)(593.57888121,1042.71547025)(593.64888649,1042.82547188)
\curveto(593.69888109,1042.90547006)(593.73888105,1043.03046994)(593.76888649,1043.20047188)
\curveto(593.788881,1043.2704697)(593.788881,1043.33546963)(593.76888649,1043.39547188)
\curveto(593.74888104,1043.45546951)(593.72888106,1043.50546946)(593.70888649,1043.54547188)
\curveto(593.63888115,1043.68546928)(593.54888124,1043.79046918)(593.43888649,1043.86047188)
\curveto(593.33888145,1043.93046904)(593.21888157,1043.99546897)(593.07888649,1044.05547188)
\curveto(592.8888819,1044.13546883)(592.6888821,1044.20046877)(592.47888649,1044.25047188)
\curveto(592.26888252,1044.30046867)(592.05888273,1044.35546861)(591.84888649,1044.41547188)
\curveto(591.76888302,1044.43546853)(591.6838831,1044.45046852)(591.59388649,1044.46047188)
\curveto(591.51388327,1044.4704685)(591.43388335,1044.48546848)(591.35388649,1044.50547188)
\curveto(591.03388375,1044.59546837)(590.72888406,1044.68046829)(590.43888649,1044.76047188)
\curveto(590.14888464,1044.85046812)(589.8838849,1044.98046799)(589.64388649,1045.15047188)
\curveto(589.36388542,1045.35046762)(589.15888563,1045.62046735)(589.02888649,1045.96047188)
\curveto(589.00888578,1046.03046694)(588.9888858,1046.12546684)(588.96888649,1046.24547188)
\curveto(588.94888584,1046.31546665)(588.93388585,1046.40046657)(588.92388649,1046.50047188)
\curveto(588.91388587,1046.60046637)(588.91888587,1046.69046628)(588.93888649,1046.77047188)
\curveto(588.95888583,1046.82046615)(588.96388582,1046.86046611)(588.95388649,1046.89047188)
\curveto(588.94388584,1046.93046604)(588.94888584,1046.97546599)(588.96888649,1047.02547188)
\curveto(588.9888858,1047.13546583)(589.00888578,1047.23546573)(589.02888649,1047.32547188)
\curveto(589.05888573,1047.42546554)(589.09388569,1047.52046545)(589.13388649,1047.61047188)
\curveto(589.26388552,1047.90046507)(589.44388534,1048.13546483)(589.67388649,1048.31547188)
\curveto(589.90388488,1048.49546447)(590.16388462,1048.64046433)(590.45388649,1048.75047188)
\curveto(590.56388422,1048.80046417)(590.67888411,1048.83546413)(590.79888649,1048.85547188)
\curveto(590.91888387,1048.88546408)(591.04388374,1048.91546405)(591.17388649,1048.94547188)
\curveto(591.23388355,1048.965464)(591.29388349,1048.97546399)(591.35388649,1048.97547188)
\lineto(591.53388649,1049.00547188)
\curveto(591.61388317,1049.01546395)(591.69888309,1049.02046395)(591.78888649,1049.02047188)
\curveto(591.87888291,1049.02046395)(591.96388282,1049.02546394)(592.04388649,1049.03547188)
}
}
{
\newrgbcolor{curcolor}{0 0 0}
\pscustom[linestyle=none,fillstyle=solid,fillcolor=curcolor]
{
\newpath
\moveto(599.23052712,1051.67547188)
\curveto(599.30052417,1051.59546137)(599.33552413,1051.47546149)(599.33552712,1051.31547188)
\lineto(599.33552712,1050.85047188)
\lineto(599.33552712,1050.44547188)
\curveto(599.33552413,1050.30546266)(599.30052417,1050.21046276)(599.23052712,1050.16047188)
\curveto(599.1705243,1050.11046286)(599.09052438,1050.08046289)(598.99052712,1050.07047188)
\curveto(598.90052457,1050.06046291)(598.80052467,1050.05546291)(598.69052712,1050.05547188)
\lineto(597.85052712,1050.05547188)
\curveto(597.74052573,1050.05546291)(597.64052583,1050.06046291)(597.55052712,1050.07047188)
\curveto(597.470526,1050.08046289)(597.40052607,1050.11046286)(597.34052712,1050.16047188)
\curveto(597.30052617,1050.19046278)(597.2705262,1050.24546272)(597.25052712,1050.32547188)
\curveto(597.24052623,1050.41546255)(597.23052624,1050.51046246)(597.22052712,1050.61047188)
\lineto(597.22052712,1050.94047188)
\curveto(597.23052624,1051.05046192)(597.23552623,1051.14546182)(597.23552712,1051.22547188)
\lineto(597.23552712,1051.43547188)
\curveto(597.24552622,1051.50546146)(597.2655262,1051.5654614)(597.29552712,1051.61547188)
\curveto(597.31552615,1051.65546131)(597.34052613,1051.68546128)(597.37052712,1051.70547188)
\lineto(597.49052712,1051.76547188)
\curveto(597.51052596,1051.7654612)(597.53552593,1051.7654612)(597.56552712,1051.76547188)
\curveto(597.59552587,1051.77546119)(597.62052585,1051.78046119)(597.64052712,1051.78047188)
\lineto(598.73552712,1051.78047188)
\curveto(598.83552463,1051.78046119)(598.93052454,1051.77546119)(599.02052712,1051.76547188)
\curveto(599.11052436,1051.75546121)(599.18052429,1051.72546124)(599.23052712,1051.67547188)
\moveto(599.33552712,1041.91047188)
\curveto(599.33552413,1041.71047126)(599.33052414,1041.54047143)(599.32052712,1041.40047188)
\curveto(599.31052416,1041.26047171)(599.22052425,1041.1654718)(599.05052712,1041.11547188)
\curveto(598.99052448,1041.09547187)(598.92552454,1041.08547188)(598.85552712,1041.08547188)
\curveto(598.78552468,1041.09547187)(598.71052476,1041.10047187)(598.63052712,1041.10047188)
\lineto(597.79052712,1041.10047188)
\curveto(597.70052577,1041.10047187)(597.61052586,1041.10547186)(597.52052712,1041.11547188)
\curveto(597.44052603,1041.12547184)(597.38052609,1041.15547181)(597.34052712,1041.20547188)
\curveto(597.28052619,1041.27547169)(597.24552622,1041.36047161)(597.23552712,1041.46047188)
\lineto(597.23552712,1041.80547188)
\lineto(597.23552712,1048.13547188)
\lineto(597.23552712,1048.43547188)
\curveto(597.23552623,1048.53546443)(597.25552621,1048.61546435)(597.29552712,1048.67547188)
\curveto(597.35552611,1048.74546422)(597.44052603,1048.79046418)(597.55052712,1048.81047188)
\curveto(597.5705259,1048.82046415)(597.59552587,1048.82046415)(597.62552712,1048.81047188)
\curveto(597.6655258,1048.81046416)(597.69552577,1048.81546415)(597.71552712,1048.82547188)
\lineto(598.46552712,1048.82547188)
\lineto(598.66052712,1048.82547188)
\curveto(598.74052473,1048.83546413)(598.80552466,1048.83546413)(598.85552712,1048.82547188)
\lineto(598.97552712,1048.82547188)
\curveto(599.03552443,1048.80546416)(599.09052438,1048.79046418)(599.14052712,1048.78047188)
\curveto(599.19052428,1048.7704642)(599.23052424,1048.74046423)(599.26052712,1048.69047188)
\curveto(599.30052417,1048.64046433)(599.32052415,1048.5704644)(599.32052712,1048.48047188)
\curveto(599.33052414,1048.39046458)(599.33552413,1048.29546467)(599.33552712,1048.19547188)
\lineto(599.33552712,1041.91047188)
}
}
{
\newrgbcolor{curcolor}{0 0 0}
\pscustom[linestyle=none,fillstyle=solid,fillcolor=curcolor]
{
\newpath
\moveto(603.96771462,1049.03547188)
\curveto(604.71771012,1049.05546391)(605.36770947,1048.970464)(605.91771462,1048.78047188)
\curveto(606.47770836,1048.60046437)(606.90270793,1048.28546468)(607.19271462,1047.83547188)
\curveto(607.26270757,1047.72546524)(607.32270751,1047.61046536)(607.37271462,1047.49047188)
\curveto(607.4327074,1047.38046559)(607.48270735,1047.25546571)(607.52271462,1047.11547188)
\curveto(607.54270729,1047.05546591)(607.55270728,1046.99046598)(607.55271462,1046.92047188)
\curveto(607.55270728,1046.85046612)(607.54270729,1046.79046618)(607.52271462,1046.74047188)
\curveto(607.48270735,1046.68046629)(607.42770741,1046.64046633)(607.35771462,1046.62047188)
\curveto(607.30770753,1046.60046637)(607.24770759,1046.59046638)(607.17771462,1046.59047188)
\lineto(606.96771462,1046.59047188)
\lineto(606.30771462,1046.59047188)
\curveto(606.2377086,1046.59046638)(606.16770867,1046.58546638)(606.09771462,1046.57547188)
\curveto(606.02770881,1046.57546639)(605.96270887,1046.58546638)(605.90271462,1046.60547188)
\curveto(605.80270903,1046.62546634)(605.72770911,1046.6654663)(605.67771462,1046.72547188)
\curveto(605.62770921,1046.78546618)(605.58270925,1046.84546612)(605.54271462,1046.90547188)
\lineto(605.42271462,1047.11547188)
\curveto(605.39270944,1047.19546577)(605.34270949,1047.26046571)(605.27271462,1047.31047188)
\curveto(605.17270966,1047.39046558)(605.07270976,1047.45046552)(604.97271462,1047.49047188)
\curveto(604.88270995,1047.53046544)(604.76771007,1047.5654654)(604.62771462,1047.59547188)
\curveto(604.55771028,1047.61546535)(604.45271038,1047.63046534)(604.31271462,1047.64047188)
\curveto(604.18271065,1047.65046532)(604.08271075,1047.64546532)(604.01271462,1047.62547188)
\lineto(603.90771462,1047.62547188)
\lineto(603.75771462,1047.59547188)
\curveto(603.71771112,1047.59546537)(603.67271116,1047.59046538)(603.62271462,1047.58047188)
\curveto(603.45271138,1047.53046544)(603.31271152,1047.46046551)(603.20271462,1047.37047188)
\curveto(603.10271173,1047.29046568)(603.0327118,1047.1654658)(602.99271462,1046.99547188)
\curveto(602.97271186,1046.92546604)(602.97271186,1046.86046611)(602.99271462,1046.80047188)
\curveto(603.01271182,1046.74046623)(603.0327118,1046.69046628)(603.05271462,1046.65047188)
\curveto(603.12271171,1046.53046644)(603.20271163,1046.43546653)(603.29271462,1046.36547188)
\curveto(603.39271144,1046.29546667)(603.50771133,1046.23546673)(603.63771462,1046.18547188)
\curveto(603.82771101,1046.10546686)(604.0327108,1046.03546693)(604.25271462,1045.97547188)
\lineto(604.94271462,1045.82547188)
\curveto(605.18270965,1045.78546718)(605.41270942,1045.73546723)(605.63271462,1045.67547188)
\curveto(605.86270897,1045.62546734)(606.07770876,1045.56046741)(606.27771462,1045.48047188)
\curveto(606.36770847,1045.44046753)(606.45270838,1045.40546756)(606.53271462,1045.37547188)
\curveto(606.62270821,1045.35546761)(606.70770813,1045.32046765)(606.78771462,1045.27047188)
\curveto(606.97770786,1045.15046782)(607.14770769,1045.02046795)(607.29771462,1044.88047188)
\curveto(607.45770738,1044.74046823)(607.58270725,1044.5654684)(607.67271462,1044.35547188)
\curveto(607.70270713,1044.28546868)(607.72770711,1044.21546875)(607.74771462,1044.14547188)
\curveto(607.76770707,1044.07546889)(607.78770705,1044.00046897)(607.80771462,1043.92047188)
\curveto(607.81770702,1043.86046911)(607.82270701,1043.7654692)(607.82271462,1043.63547188)
\curveto(607.832707,1043.51546945)(607.832707,1043.42046955)(607.82271462,1043.35047188)
\lineto(607.82271462,1043.27547188)
\curveto(607.80270703,1043.21546975)(607.78770705,1043.15546981)(607.77771462,1043.09547188)
\curveto(607.77770706,1043.04546992)(607.77270706,1042.99546997)(607.76271462,1042.94547188)
\curveto(607.69270714,1042.64547032)(607.58270725,1042.38047059)(607.43271462,1042.15047188)
\curveto(607.27270756,1041.91047106)(607.07770776,1041.71547125)(606.84771462,1041.56547188)
\curveto(606.61770822,1041.41547155)(606.35770848,1041.28547168)(606.06771462,1041.17547188)
\curveto(605.95770888,1041.12547184)(605.837709,1041.09047188)(605.70771462,1041.07047188)
\curveto(605.58770925,1041.05047192)(605.46770937,1041.02547194)(605.34771462,1040.99547188)
\curveto(605.25770958,1040.97547199)(605.16270967,1040.965472)(605.06271462,1040.96547188)
\curveto(604.97270986,1040.95547201)(604.88270995,1040.94047203)(604.79271462,1040.92047188)
\lineto(604.52271462,1040.92047188)
\curveto(604.46271037,1040.90047207)(604.35771048,1040.89047208)(604.20771462,1040.89047188)
\curveto(604.06771077,1040.89047208)(603.96771087,1040.90047207)(603.90771462,1040.92047188)
\curveto(603.87771096,1040.92047205)(603.84271099,1040.92547204)(603.80271462,1040.93547188)
\lineto(603.69771462,1040.93547188)
\curveto(603.57771126,1040.95547201)(603.45771138,1040.970472)(603.33771462,1040.98047188)
\curveto(603.21771162,1040.99047198)(603.10271173,1041.01047196)(602.99271462,1041.04047188)
\curveto(602.60271223,1041.15047182)(602.25771258,1041.27547169)(601.95771462,1041.41547188)
\curveto(601.65771318,1041.5654714)(601.40271343,1041.78547118)(601.19271462,1042.07547188)
\curveto(601.05271378,1042.2654707)(600.9327139,1042.48547048)(600.83271462,1042.73547188)
\curveto(600.81271402,1042.79547017)(600.79271404,1042.87547009)(600.77271462,1042.97547188)
\curveto(600.75271408,1043.02546994)(600.7377141,1043.09546987)(600.72771462,1043.18547188)
\curveto(600.71771412,1043.27546969)(600.72271411,1043.35046962)(600.74271462,1043.41047188)
\curveto(600.77271406,1043.48046949)(600.82271401,1043.53046944)(600.89271462,1043.56047188)
\curveto(600.94271389,1043.58046939)(601.00271383,1043.59046938)(601.07271462,1043.59047188)
\lineto(601.29771462,1043.59047188)
\lineto(602.00271462,1043.59047188)
\lineto(602.24271462,1043.59047188)
\curveto(602.32271251,1043.59046938)(602.39271244,1043.58046939)(602.45271462,1043.56047188)
\curveto(602.56271227,1043.52046945)(602.6327122,1043.45546951)(602.66271462,1043.36547188)
\curveto(602.70271213,1043.27546969)(602.74771209,1043.18046979)(602.79771462,1043.08047188)
\curveto(602.81771202,1043.03046994)(602.85271198,1042.96547)(602.90271462,1042.88547188)
\curveto(602.96271187,1042.80547016)(603.01271182,1042.75547021)(603.05271462,1042.73547188)
\curveto(603.17271166,1042.63547033)(603.28771155,1042.55547041)(603.39771462,1042.49547188)
\curveto(603.50771133,1042.44547052)(603.64771119,1042.39547057)(603.81771462,1042.34547188)
\curveto(603.86771097,1042.32547064)(603.91771092,1042.31547065)(603.96771462,1042.31547188)
\curveto(604.01771082,1042.32547064)(604.06771077,1042.32547064)(604.11771462,1042.31547188)
\curveto(604.19771064,1042.29547067)(604.28271055,1042.28547068)(604.37271462,1042.28547188)
\curveto(604.47271036,1042.29547067)(604.55771028,1042.31047066)(604.62771462,1042.33047188)
\curveto(604.67771016,1042.34047063)(604.72271011,1042.34547062)(604.76271462,1042.34547188)
\curveto(604.81271002,1042.34547062)(604.86270997,1042.35547061)(604.91271462,1042.37547188)
\curveto(605.05270978,1042.42547054)(605.17770966,1042.48547048)(605.28771462,1042.55547188)
\curveto(605.40770943,1042.62547034)(605.50270933,1042.71547025)(605.57271462,1042.82547188)
\curveto(605.62270921,1042.90547006)(605.66270917,1043.03046994)(605.69271462,1043.20047188)
\curveto(605.71270912,1043.2704697)(605.71270912,1043.33546963)(605.69271462,1043.39547188)
\curveto(605.67270916,1043.45546951)(605.65270918,1043.50546946)(605.63271462,1043.54547188)
\curveto(605.56270927,1043.68546928)(605.47270936,1043.79046918)(605.36271462,1043.86047188)
\curveto(605.26270957,1043.93046904)(605.14270969,1043.99546897)(605.00271462,1044.05547188)
\curveto(604.81271002,1044.13546883)(604.61271022,1044.20046877)(604.40271462,1044.25047188)
\curveto(604.19271064,1044.30046867)(603.98271085,1044.35546861)(603.77271462,1044.41547188)
\curveto(603.69271114,1044.43546853)(603.60771123,1044.45046852)(603.51771462,1044.46047188)
\curveto(603.4377114,1044.4704685)(603.35771148,1044.48546848)(603.27771462,1044.50547188)
\curveto(602.95771188,1044.59546837)(602.65271218,1044.68046829)(602.36271462,1044.76047188)
\curveto(602.07271276,1044.85046812)(601.80771303,1044.98046799)(601.56771462,1045.15047188)
\curveto(601.28771355,1045.35046762)(601.08271375,1045.62046735)(600.95271462,1045.96047188)
\curveto(600.9327139,1046.03046694)(600.91271392,1046.12546684)(600.89271462,1046.24547188)
\curveto(600.87271396,1046.31546665)(600.85771398,1046.40046657)(600.84771462,1046.50047188)
\curveto(600.837714,1046.60046637)(600.84271399,1046.69046628)(600.86271462,1046.77047188)
\curveto(600.88271395,1046.82046615)(600.88771395,1046.86046611)(600.87771462,1046.89047188)
\curveto(600.86771397,1046.93046604)(600.87271396,1046.97546599)(600.89271462,1047.02547188)
\curveto(600.91271392,1047.13546583)(600.9327139,1047.23546573)(600.95271462,1047.32547188)
\curveto(600.98271385,1047.42546554)(601.01771382,1047.52046545)(601.05771462,1047.61047188)
\curveto(601.18771365,1047.90046507)(601.36771347,1048.13546483)(601.59771462,1048.31547188)
\curveto(601.82771301,1048.49546447)(602.08771275,1048.64046433)(602.37771462,1048.75047188)
\curveto(602.48771235,1048.80046417)(602.60271223,1048.83546413)(602.72271462,1048.85547188)
\curveto(602.84271199,1048.88546408)(602.96771187,1048.91546405)(603.09771462,1048.94547188)
\curveto(603.15771168,1048.965464)(603.21771162,1048.97546399)(603.27771462,1048.97547188)
\lineto(603.45771462,1049.00547188)
\curveto(603.5377113,1049.01546395)(603.62271121,1049.02046395)(603.71271462,1049.02047188)
\curveto(603.80271103,1049.02046395)(603.88771095,1049.02546394)(603.96771462,1049.03547188)
}
}
{
\newrgbcolor{curcolor}{0 0 0}
\pscustom[linestyle=none,fillstyle=solid,fillcolor=curcolor]
{
\newpath
\moveto(610.10435524,1051.13547188)
\lineto(611.10935524,1051.13547188)
\curveto(611.25935226,1051.13546183)(611.38935213,1051.12546184)(611.49935524,1051.10547188)
\curveto(611.6193519,1051.09546187)(611.70435181,1051.03546193)(611.75435524,1050.92547188)
\curveto(611.77435174,1050.87546209)(611.78435173,1050.81546215)(611.78435524,1050.74547188)
\lineto(611.78435524,1050.53547188)
\lineto(611.78435524,1049.86047188)
\curveto(611.78435173,1049.81046316)(611.77935174,1049.75046322)(611.76935524,1049.68047188)
\curveto(611.76935175,1049.62046335)(611.77435174,1049.5654634)(611.78435524,1049.51547188)
\lineto(611.78435524,1049.35047188)
\curveto(611.78435173,1049.2704637)(611.78935173,1049.19546377)(611.79935524,1049.12547188)
\curveto(611.80935171,1049.0654639)(611.83435168,1049.01046396)(611.87435524,1048.96047188)
\curveto(611.94435157,1048.8704641)(612.06935145,1048.82046415)(612.24935524,1048.81047188)
\lineto(612.78935524,1048.81047188)
\lineto(612.96935524,1048.81047188)
\curveto(613.02935049,1048.81046416)(613.08435043,1048.80046417)(613.13435524,1048.78047188)
\curveto(613.24435027,1048.73046424)(613.30435021,1048.64046433)(613.31435524,1048.51047188)
\curveto(613.33435018,1048.38046459)(613.34435017,1048.23546473)(613.34435524,1048.07547188)
\lineto(613.34435524,1047.86547188)
\curveto(613.35435016,1047.79546517)(613.34935017,1047.73546523)(613.32935524,1047.68547188)
\curveto(613.27935024,1047.52546544)(613.17435034,1047.44046553)(613.01435524,1047.43047188)
\curveto(612.85435066,1047.42046555)(612.67435084,1047.41546555)(612.47435524,1047.41547188)
\lineto(612.33935524,1047.41547188)
\curveto(612.29935122,1047.42546554)(612.26435125,1047.42546554)(612.23435524,1047.41547188)
\curveto(612.19435132,1047.40546556)(612.15935136,1047.40046557)(612.12935524,1047.40047188)
\curveto(612.09935142,1047.41046556)(612.06935145,1047.40546556)(612.03935524,1047.38547188)
\curveto(611.95935156,1047.3654656)(611.89935162,1047.32046565)(611.85935524,1047.25047188)
\curveto(611.82935169,1047.19046578)(611.80435171,1047.11546585)(611.78435524,1047.02547188)
\curveto(611.77435174,1046.97546599)(611.77435174,1046.92046605)(611.78435524,1046.86047188)
\curveto(611.79435172,1046.80046617)(611.79435172,1046.74546622)(611.78435524,1046.69547188)
\lineto(611.78435524,1045.76547188)
\lineto(611.78435524,1044.01047188)
\curveto(611.78435173,1043.76046921)(611.78935173,1043.54046943)(611.79935524,1043.35047188)
\curveto(611.8193517,1043.1704698)(611.88435163,1043.01046996)(611.99435524,1042.87047188)
\curveto(612.04435147,1042.81047016)(612.10935141,1042.7654702)(612.18935524,1042.73547188)
\lineto(612.45935524,1042.67547188)
\curveto(612.48935103,1042.6654703)(612.519351,1042.66047031)(612.54935524,1042.66047188)
\curveto(612.58935093,1042.6704703)(612.6193509,1042.6704703)(612.63935524,1042.66047188)
\lineto(612.80435524,1042.66047188)
\curveto(612.9143506,1042.66047031)(613.00935051,1042.65547031)(613.08935524,1042.64547188)
\curveto(613.16935035,1042.63547033)(613.23435028,1042.59547037)(613.28435524,1042.52547188)
\curveto(613.32435019,1042.4654705)(613.34435017,1042.38547058)(613.34435524,1042.28547188)
\lineto(613.34435524,1042.00047188)
\curveto(613.34435017,1041.79047118)(613.33935018,1041.59547137)(613.32935524,1041.41547188)
\curveto(613.32935019,1041.24547172)(613.24935027,1041.13047184)(613.08935524,1041.07047188)
\curveto(613.03935048,1041.05047192)(612.99435052,1041.04547192)(612.95435524,1041.05547188)
\curveto(612.9143506,1041.05547191)(612.86935065,1041.04547192)(612.81935524,1041.02547188)
\lineto(612.66935524,1041.02547188)
\curveto(612.64935087,1041.02547194)(612.6193509,1041.03047194)(612.57935524,1041.04047188)
\curveto(612.53935098,1041.04047193)(612.50435101,1041.03547193)(612.47435524,1041.02547188)
\curveto(612.42435109,1041.01547195)(612.36935115,1041.01547195)(612.30935524,1041.02547188)
\lineto(612.15935524,1041.02547188)
\lineto(612.00935524,1041.02547188)
\curveto(611.95935156,1041.01547195)(611.9143516,1041.01547195)(611.87435524,1041.02547188)
\lineto(611.70935524,1041.02547188)
\curveto(611.65935186,1041.03547193)(611.60435191,1041.04047193)(611.54435524,1041.04047188)
\curveto(611.48435203,1041.04047193)(611.42935209,1041.04547192)(611.37935524,1041.05547188)
\curveto(611.30935221,1041.0654719)(611.24435227,1041.07547189)(611.18435524,1041.08547188)
\lineto(611.00435524,1041.11547188)
\curveto(610.89435262,1041.14547182)(610.78935273,1041.18047179)(610.68935524,1041.22047188)
\curveto(610.58935293,1041.26047171)(610.49435302,1041.30547166)(610.40435524,1041.35547188)
\lineto(610.31435524,1041.41547188)
\curveto(610.28435323,1041.44547152)(610.24935327,1041.47547149)(610.20935524,1041.50547188)
\curveto(610.18935333,1041.52547144)(610.16435335,1041.54547142)(610.13435524,1041.56547188)
\lineto(610.05935524,1041.64047188)
\curveto(609.9193536,1041.83047114)(609.8143537,1042.04047093)(609.74435524,1042.27047188)
\curveto(609.72435379,1042.31047066)(609.7143538,1042.34547062)(609.71435524,1042.37547188)
\curveto(609.72435379,1042.41547055)(609.72435379,1042.46047051)(609.71435524,1042.51047188)
\curveto(609.70435381,1042.53047044)(609.69935382,1042.55547041)(609.69935524,1042.58547188)
\curveto(609.69935382,1042.61547035)(609.69435382,1042.64047033)(609.68435524,1042.66047188)
\lineto(609.68435524,1042.81047188)
\curveto(609.67435384,1042.85047012)(609.66935385,1042.89547007)(609.66935524,1042.94547188)
\curveto(609.67935384,1042.99546997)(609.68435383,1043.04546992)(609.68435524,1043.09547188)
\lineto(609.68435524,1043.66547188)
\lineto(609.68435524,1045.90047188)
\lineto(609.68435524,1046.69547188)
\lineto(609.68435524,1046.90547188)
\curveto(609.69435382,1046.97546599)(609.68935383,1047.04046593)(609.66935524,1047.10047188)
\curveto(609.62935389,1047.24046573)(609.55935396,1047.33046564)(609.45935524,1047.37047188)
\curveto(609.34935417,1047.42046555)(609.20935431,1047.43546553)(609.03935524,1047.41547188)
\curveto(608.86935465,1047.39546557)(608.72435479,1047.41046556)(608.60435524,1047.46047188)
\curveto(608.52435499,1047.49046548)(608.47435504,1047.53546543)(608.45435524,1047.59547188)
\curveto(608.43435508,1047.65546531)(608.4143551,1047.73046524)(608.39435524,1047.82047188)
\lineto(608.39435524,1048.13547188)
\curveto(608.39435512,1048.31546465)(608.40435511,1048.46046451)(608.42435524,1048.57047188)
\curveto(608.44435507,1048.68046429)(608.52935499,1048.75546421)(608.67935524,1048.79547188)
\curveto(608.7193548,1048.81546415)(608.75935476,1048.82046415)(608.79935524,1048.81047188)
\lineto(608.93435524,1048.81047188)
\curveto(609.08435443,1048.81046416)(609.22435429,1048.81546415)(609.35435524,1048.82547188)
\curveto(609.48435403,1048.84546412)(609.57435394,1048.90546406)(609.62435524,1049.00547188)
\curveto(609.65435386,1049.07546389)(609.66935385,1049.15546381)(609.66935524,1049.24547188)
\curveto(609.67935384,1049.33546363)(609.68435383,1049.42546354)(609.68435524,1049.51547188)
\lineto(609.68435524,1050.44547188)
\lineto(609.68435524,1050.70047188)
\curveto(609.68435383,1050.79046218)(609.69435382,1050.8654621)(609.71435524,1050.92547188)
\curveto(609.76435375,1051.02546194)(609.83935368,1051.09046188)(609.93935524,1051.12047188)
\curveto(609.95935356,1051.13046184)(609.98435353,1051.13046184)(610.01435524,1051.12047188)
\curveto(610.05435346,1051.12046185)(610.08435343,1051.12546184)(610.10435524,1051.13547188)
}
}
{
\newrgbcolor{curcolor}{0 0 0}
\pscustom[linestyle=none,fillstyle=solid,fillcolor=curcolor]
{
\newpath
\moveto(621.69279274,1045.03047188)
\curveto(621.71278458,1044.95046802)(621.71278458,1044.86046811)(621.69279274,1044.76047188)
\curveto(621.67278462,1044.66046831)(621.63778465,1044.59546837)(621.58779274,1044.56547188)
\curveto(621.53778475,1044.52546844)(621.46278483,1044.49546847)(621.36279274,1044.47547188)
\curveto(621.27278502,1044.4654685)(621.16778512,1044.45546851)(621.04779274,1044.44547188)
\lineto(620.70279274,1044.44547188)
\curveto(620.5927857,1044.45546851)(620.4927858,1044.46046851)(620.40279274,1044.46047188)
\lineto(616.74279274,1044.46047188)
\lineto(616.53279274,1044.46047188)
\curveto(616.47278982,1044.46046851)(616.41778987,1044.45046852)(616.36779274,1044.43047188)
\curveto(616.28779,1044.39046858)(616.23779005,1044.35046862)(616.21779274,1044.31047188)
\curveto(616.19779009,1044.29046868)(616.17779011,1044.25046872)(616.15779274,1044.19047188)
\curveto(616.13779015,1044.14046883)(616.13279016,1044.09046888)(616.14279274,1044.04047188)
\curveto(616.16279013,1043.98046899)(616.17279012,1043.92046905)(616.17279274,1043.86047188)
\curveto(616.18279011,1043.81046916)(616.19779009,1043.75546921)(616.21779274,1043.69547188)
\curveto(616.29778999,1043.45546951)(616.3927899,1043.25546971)(616.50279274,1043.09547188)
\curveto(616.62278967,1042.94547002)(616.78278951,1042.81047016)(616.98279274,1042.69047188)
\curveto(617.06278923,1042.64047033)(617.14278915,1042.60547036)(617.22279274,1042.58547188)
\curveto(617.31278898,1042.57547039)(617.40278889,1042.55547041)(617.49279274,1042.52547188)
\curveto(617.57278872,1042.50547046)(617.68278861,1042.49047048)(617.82279274,1042.48047188)
\curveto(617.96278833,1042.4704705)(618.08278821,1042.47547049)(618.18279274,1042.49547188)
\lineto(618.31779274,1042.49547188)
\curveto(618.41778787,1042.51547045)(618.50778778,1042.53547043)(618.58779274,1042.55547188)
\curveto(618.67778761,1042.58547038)(618.76278753,1042.61547035)(618.84279274,1042.64547188)
\curveto(618.94278735,1042.69547027)(619.05278724,1042.76047021)(619.17279274,1042.84047188)
\curveto(619.30278699,1042.92047005)(619.39778689,1043.00046997)(619.45779274,1043.08047188)
\curveto(619.50778678,1043.15046982)(619.55778673,1043.21546975)(619.60779274,1043.27547188)
\curveto(619.66778662,1043.34546962)(619.73778655,1043.39546957)(619.81779274,1043.42547188)
\curveto(619.91778637,1043.47546949)(620.04278625,1043.49546947)(620.19279274,1043.48547188)
\lineto(620.62779274,1043.48547188)
\lineto(620.80779274,1043.48547188)
\curveto(620.87778541,1043.49546947)(620.93778535,1043.49046948)(620.98779274,1043.47047188)
\lineto(621.13779274,1043.47047188)
\curveto(621.23778505,1043.45046952)(621.30778498,1043.42546954)(621.34779274,1043.39547188)
\curveto(621.3877849,1043.37546959)(621.40778488,1043.33046964)(621.40779274,1043.26047188)
\curveto(621.41778487,1043.19046978)(621.41278488,1043.13046984)(621.39279274,1043.08047188)
\curveto(621.34278495,1042.94047003)(621.287785,1042.81547015)(621.22779274,1042.70547188)
\curveto(621.16778512,1042.59547037)(621.09778519,1042.48547048)(621.01779274,1042.37547188)
\curveto(620.79778549,1042.04547092)(620.54778574,1041.78047119)(620.26779274,1041.58047188)
\curveto(619.9877863,1041.38047159)(619.63778665,1041.21047176)(619.21779274,1041.07047188)
\curveto(619.10778718,1041.03047194)(618.99778729,1041.00547196)(618.88779274,1040.99547188)
\curveto(618.77778751,1040.98547198)(618.66278763,1040.965472)(618.54279274,1040.93547188)
\curveto(618.50278779,1040.92547204)(618.45778783,1040.92547204)(618.40779274,1040.93547188)
\curveto(618.36778792,1040.93547203)(618.32778796,1040.93047204)(618.28779274,1040.92047188)
\lineto(618.12279274,1040.92047188)
\curveto(618.07278822,1040.90047207)(618.01278828,1040.89547207)(617.94279274,1040.90547188)
\curveto(617.88278841,1040.90547206)(617.82778846,1040.91047206)(617.77779274,1040.92047188)
\curveto(617.69778859,1040.93047204)(617.62778866,1040.93047204)(617.56779274,1040.92047188)
\curveto(617.50778878,1040.91047206)(617.44278885,1040.91547205)(617.37279274,1040.93547188)
\curveto(617.32278897,1040.95547201)(617.26778902,1040.965472)(617.20779274,1040.96547188)
\curveto(617.14778914,1040.965472)(617.0927892,1040.97547199)(617.04279274,1040.99547188)
\curveto(616.93278936,1041.01547195)(616.82278947,1041.04047193)(616.71279274,1041.07047188)
\curveto(616.60278969,1041.09047188)(616.50278979,1041.12547184)(616.41279274,1041.17547188)
\curveto(616.30278999,1041.21547175)(616.19779009,1041.25047172)(616.09779274,1041.28047188)
\curveto(616.00779028,1041.32047165)(615.92279037,1041.3654716)(615.84279274,1041.41547188)
\curveto(615.52279077,1041.61547135)(615.23779105,1041.84547112)(614.98779274,1042.10547188)
\curveto(614.73779155,1042.37547059)(614.53279176,1042.68547028)(614.37279274,1043.03547188)
\curveto(614.32279197,1043.14546982)(614.28279201,1043.25546971)(614.25279274,1043.36547188)
\curveto(614.22279207,1043.48546948)(614.18279211,1043.60546936)(614.13279274,1043.72547188)
\curveto(614.12279217,1043.7654692)(614.11779217,1043.80046917)(614.11779274,1043.83047188)
\curveto(614.11779217,1043.8704691)(614.11279218,1043.91046906)(614.10279274,1043.95047188)
\curveto(614.06279223,1044.0704689)(614.03779225,1044.20046877)(614.02779274,1044.34047188)
\lineto(613.99779274,1044.76047188)
\curveto(613.99779229,1044.81046816)(613.9927923,1044.8654681)(613.98279274,1044.92547188)
\curveto(613.98279231,1044.98546798)(613.9877923,1045.04046793)(613.99779274,1045.09047188)
\lineto(613.99779274,1045.27047188)
\lineto(614.04279274,1045.63047188)
\curveto(614.08279221,1045.80046717)(614.11779217,1045.965467)(614.14779274,1046.12547188)
\curveto(614.17779211,1046.28546668)(614.22279207,1046.43546653)(614.28279274,1046.57547188)
\curveto(614.71279158,1047.61546535)(615.44279085,1048.35046462)(616.47279274,1048.78047188)
\curveto(616.61278968,1048.84046413)(616.75278954,1048.88046409)(616.89279274,1048.90047188)
\curveto(617.04278925,1048.93046404)(617.19778909,1048.965464)(617.35779274,1049.00547188)
\curveto(617.43778885,1049.01546395)(617.51278878,1049.02046395)(617.58279274,1049.02047188)
\curveto(617.65278864,1049.02046395)(617.72778856,1049.02546394)(617.80779274,1049.03547188)
\curveto(618.31778797,1049.04546392)(618.75278754,1048.98546398)(619.11279274,1048.85547188)
\curveto(619.48278681,1048.73546423)(619.81278648,1048.57546439)(620.10279274,1048.37547188)
\curveto(620.1927861,1048.31546465)(620.28278601,1048.24546472)(620.37279274,1048.16547188)
\curveto(620.46278583,1048.09546487)(620.54278575,1048.02046495)(620.61279274,1047.94047188)
\curveto(620.64278565,1047.89046508)(620.68278561,1047.85046512)(620.73279274,1047.82047188)
\curveto(620.81278548,1047.71046526)(620.8877854,1047.59546537)(620.95779274,1047.47547188)
\curveto(621.02778526,1047.3654656)(621.10278519,1047.25046572)(621.18279274,1047.13047188)
\curveto(621.23278506,1047.04046593)(621.27278502,1046.94546602)(621.30279274,1046.84547188)
\curveto(621.34278495,1046.75546621)(621.38278491,1046.65546631)(621.42279274,1046.54547188)
\curveto(621.47278482,1046.41546655)(621.51278478,1046.28046669)(621.54279274,1046.14047188)
\curveto(621.57278472,1046.00046697)(621.60778468,1045.86046711)(621.64779274,1045.72047188)
\curveto(621.66778462,1045.64046733)(621.67278462,1045.55046742)(621.66279274,1045.45047188)
\curveto(621.66278463,1045.36046761)(621.67278462,1045.27546769)(621.69279274,1045.19547188)
\lineto(621.69279274,1045.03047188)
\moveto(619.44279274,1045.91547188)
\curveto(619.51278678,1046.01546695)(619.51778677,1046.13546683)(619.45779274,1046.27547188)
\curveto(619.40778688,1046.42546654)(619.36778692,1046.53546643)(619.33779274,1046.60547188)
\curveto(619.19778709,1046.87546609)(619.01278728,1047.08046589)(618.78279274,1047.22047188)
\curveto(618.55278774,1047.3704656)(618.23278806,1047.45046552)(617.82279274,1047.46047188)
\curveto(617.7927885,1047.44046553)(617.75778853,1047.43546553)(617.71779274,1047.44547188)
\curveto(617.67778861,1047.45546551)(617.64278865,1047.45546551)(617.61279274,1047.44547188)
\curveto(617.56278873,1047.42546554)(617.50778878,1047.41046556)(617.44779274,1047.40047188)
\curveto(617.3877889,1047.40046557)(617.33278896,1047.39046558)(617.28279274,1047.37047188)
\curveto(616.84278945,1047.23046574)(616.51778977,1046.95546601)(616.30779274,1046.54547188)
\curveto(616.28779,1046.50546646)(616.26279003,1046.45046652)(616.23279274,1046.38047188)
\curveto(616.21279008,1046.32046665)(616.19779009,1046.25546671)(616.18779274,1046.18547188)
\curveto(616.17779011,1046.12546684)(616.17779011,1046.0654669)(616.18779274,1046.00547188)
\curveto(616.20779008,1045.94546702)(616.24279005,1045.89546707)(616.29279274,1045.85547188)
\curveto(616.37278992,1045.80546716)(616.48278981,1045.78046719)(616.62279274,1045.78047188)
\lineto(617.02779274,1045.78047188)
\lineto(618.69279274,1045.78047188)
\lineto(619.12779274,1045.78047188)
\curveto(619.287787,1045.79046718)(619.3927869,1045.83546713)(619.44279274,1045.91547188)
}
}
{
\newrgbcolor{curcolor}{0 0 0}
\pscustom[linestyle=none,fillstyle=solid,fillcolor=curcolor]
{
\newpath
\moveto(627.39607399,1049.02047188)
\curveto(627.76606839,1049.03046394)(628.09106806,1048.99046398)(628.37107399,1048.90047188)
\curveto(628.6510675,1048.81046416)(628.89606726,1048.68546428)(629.10607399,1048.52547188)
\curveto(629.18606697,1048.4654645)(629.2560669,1048.39546457)(629.31607399,1048.31547188)
\curveto(629.38606677,1048.23546473)(629.46106669,1048.15546481)(629.54107399,1048.07547188)
\curveto(629.56106659,1048.05546491)(629.59106656,1048.02546494)(629.63107399,1047.98547188)
\curveto(629.68106647,1047.95546501)(629.73106642,1047.95046502)(629.78107399,1047.97047188)
\curveto(629.89106626,1048.00046497)(629.99606616,1048.0704649)(630.09607399,1048.18047188)
\curveto(630.19606596,1048.30046467)(630.29106586,1048.39046458)(630.38107399,1048.45047188)
\curveto(630.52106563,1048.56046441)(630.67106548,1048.65046432)(630.83107399,1048.72047188)
\curveto(630.99106516,1048.80046417)(631.17106498,1048.87546409)(631.37107399,1048.94547188)
\curveto(631.4510647,1048.965464)(631.54606461,1048.98046399)(631.65607399,1048.99047188)
\curveto(631.77606438,1049.01046396)(631.89606426,1049.02046395)(632.01607399,1049.02047188)
\curveto(632.14606401,1049.03046394)(632.26606389,1049.03046394)(632.37607399,1049.02047188)
\curveto(632.49606366,1049.01046396)(632.60106355,1048.99546397)(632.69107399,1048.97547188)
\curveto(632.74106341,1048.965464)(632.78606337,1048.96046401)(632.82607399,1048.96047188)
\curveto(632.86606329,1048.96046401)(632.91106324,1048.95046402)(632.96107399,1048.93047188)
\curveto(633.10106305,1048.89046408)(633.23606292,1048.85046412)(633.36607399,1048.81047188)
\curveto(633.49606266,1048.7704642)(633.61606254,1048.71546425)(633.72607399,1048.64547188)
\curveto(634.14606201,1048.38546458)(634.46106169,1048.00546496)(634.67107399,1047.50547188)
\curveto(634.71106144,1047.41546555)(634.74106141,1047.32046565)(634.76107399,1047.22047188)
\curveto(634.78106137,1047.13046584)(634.80106135,1047.04046593)(634.82107399,1046.95047188)
\curveto(634.83106132,1046.88046609)(634.83606132,1046.81546615)(634.83607399,1046.75547188)
\curveto(634.84606131,1046.69546627)(634.8560613,1046.63546633)(634.86607399,1046.57547188)
\lineto(634.86607399,1046.42547188)
\curveto(634.87606128,1046.3654666)(634.87606128,1046.29546667)(634.86607399,1046.21547188)
\curveto(634.86606129,1046.13546683)(634.86606129,1046.06046691)(634.86607399,1045.99047188)
\lineto(634.86607399,1045.12047188)
\lineto(634.86607399,1042.19547188)
\curveto(634.86606129,1042.11547085)(634.86606129,1042.02047095)(634.86607399,1041.91047188)
\curveto(634.87606128,1041.81047116)(634.87606128,1041.71047126)(634.86607399,1041.61047188)
\curveto(634.86606129,1041.52047145)(634.8560613,1041.43047154)(634.83607399,1041.34047188)
\curveto(634.81606134,1041.26047171)(634.78606137,1041.20547176)(634.74607399,1041.17547188)
\curveto(634.68606147,1041.12547184)(634.60606155,1041.09547187)(634.50607399,1041.08547188)
\lineto(634.20607399,1041.08547188)
\lineto(633.41107399,1041.08547188)
\curveto(633.27106288,1041.08547188)(633.14606301,1041.09547187)(633.03607399,1041.11547188)
\curveto(632.92606323,1041.13547183)(632.8510633,1041.19047178)(632.81107399,1041.28047188)
\curveto(632.78106337,1041.35047162)(632.76606339,1041.42547154)(632.76607399,1041.50547188)
\curveto(632.76606339,1041.59547137)(632.76606339,1041.68047129)(632.76607399,1041.76047188)
\lineto(632.76607399,1042.60047188)
\lineto(632.76607399,1044.62547188)
\lineto(632.76607399,1045.25547188)
\curveto(632.76606339,1045.30546766)(632.76606339,1045.36046761)(632.76607399,1045.42047188)
\curveto(632.77606338,1045.48046749)(632.77106338,1045.53546743)(632.75107399,1045.58547188)
\lineto(632.75107399,1045.70547188)
\curveto(632.7510634,1045.7654672)(632.7510634,1045.82546714)(632.75107399,1045.88547188)
\curveto(632.7510634,1045.94546702)(632.74606341,1046.00546696)(632.73607399,1046.06547188)
\curveto(632.72606343,1046.10546686)(632.72106343,1046.14546682)(632.72107399,1046.18547188)
\curveto(632.72106343,1046.23546673)(632.71606344,1046.28046669)(632.70607399,1046.32047188)
\curveto(632.66606349,1046.4704665)(632.62106353,1046.60046637)(632.57107399,1046.71047188)
\curveto(632.53106362,1046.83046614)(632.46606369,1046.93546603)(632.37607399,1047.02547188)
\curveto(632.23606392,1047.1654658)(632.06606409,1047.2654657)(631.86607399,1047.32547188)
\curveto(631.82606433,1047.33546563)(631.79106436,1047.33546563)(631.76107399,1047.32547188)
\curveto(631.73106442,1047.32546564)(631.69606446,1047.33546563)(631.65607399,1047.35547188)
\curveto(631.61606454,1047.3654656)(631.56606459,1047.3704656)(631.50607399,1047.37047188)
\curveto(631.4560647,1047.38046559)(631.40606475,1047.38046559)(631.35607399,1047.37047188)
\curveto(631.29606486,1047.35046562)(631.23606492,1047.34046563)(631.17607399,1047.34047188)
\curveto(631.11606504,1047.34046563)(631.0560651,1047.33046564)(630.99607399,1047.31047188)
\curveto(630.70606545,1047.21046576)(630.49606566,1047.06046591)(630.36607399,1046.86047188)
\curveto(630.19606596,1046.63046634)(630.09106606,1046.34046663)(630.05107399,1045.99047188)
\curveto(630.02106613,1045.65046732)(630.00606615,1045.27546769)(630.00607399,1044.86547188)
\lineto(630.00607399,1042.88547188)
\lineto(630.00607399,1041.77547188)
\lineto(630.00607399,1041.47547188)
\curveto(630.00606615,1041.37547159)(629.98106617,1041.29547167)(629.93107399,1041.23547188)
\curveto(629.88106627,1041.1654718)(629.80606635,1041.12047185)(629.70607399,1041.10047188)
\curveto(629.61606654,1041.09047188)(629.51106664,1041.08547188)(629.39107399,1041.08547188)
\lineto(628.58107399,1041.08547188)
\lineto(628.31107399,1041.08547188)
\curveto(628.23106792,1041.09547187)(628.16106799,1041.11047186)(628.10107399,1041.13047188)
\curveto(628.00106815,1041.18047179)(627.94106821,1041.26047171)(627.92107399,1041.37047188)
\curveto(627.91106824,1041.48047149)(627.90606825,1041.60547136)(627.90607399,1041.74547188)
\lineto(627.90607399,1043.02047188)
\lineto(627.90607399,1045.37547188)
\curveto(627.90606825,1045.6654673)(627.89606826,1045.94046703)(627.87607399,1046.20047188)
\curveto(627.8560683,1046.46046651)(627.79106836,1046.67546629)(627.68107399,1046.84547188)
\curveto(627.60106855,1046.98546598)(627.49606866,1047.09046588)(627.36607399,1047.16047188)
\curveto(627.24606891,1047.23046574)(627.09606906,1047.29046568)(626.91607399,1047.34047188)
\curveto(626.87606928,1047.35046562)(626.83606932,1047.35046562)(626.79607399,1047.34047188)
\curveto(626.7560694,1047.34046563)(626.71106944,1047.34546562)(626.66107399,1047.35547188)
\curveto(626.5510696,1047.37546559)(626.44606971,1047.3654656)(626.34607399,1047.32547188)
\curveto(626.32606983,1047.32546564)(626.30606985,1047.32046565)(626.28607399,1047.31047188)
\lineto(626.22607399,1047.31047188)
\curveto(626.06607009,1047.26046571)(625.91107024,1047.17546579)(625.76107399,1047.05547188)
\curveto(625.60107055,1046.93546603)(625.47607068,1046.79546617)(625.38607399,1046.63547188)
\curveto(625.30607085,1046.48546648)(625.24607091,1046.31046666)(625.20607399,1046.11047188)
\curveto(625.17607098,1045.92046705)(625.156071,1045.71046726)(625.14607399,1045.48047188)
\lineto(625.14607399,1044.73047188)
\lineto(625.14607399,1042.70547188)
\lineto(625.14607399,1041.79047188)
\lineto(625.14607399,1041.52047188)
\curveto(625.14607101,1041.43047154)(625.13107102,1041.35047162)(625.10107399,1041.28047188)
\curveto(625.06107109,1041.19047178)(624.98607117,1041.13547183)(624.87607399,1041.11547188)
\curveto(624.76607139,1041.09547187)(624.64107151,1041.08547188)(624.50107399,1041.08547188)
\lineto(623.72107399,1041.08547188)
\lineto(623.42107399,1041.08547188)
\curveto(623.33107282,1041.09547187)(623.2560729,1041.12047185)(623.19607399,1041.16047188)
\curveto(623.10607305,1041.21047176)(623.0560731,1041.30047167)(623.04607399,1041.43047188)
\lineto(623.04607399,1041.86547188)
\lineto(623.04607399,1043.62047188)
\lineto(623.04607399,1047.28047188)
\lineto(623.04607399,1048.18047188)
\lineto(623.04607399,1048.46547188)
\curveto(623.0560731,1048.55546441)(623.08107307,1048.63046434)(623.12107399,1048.69047188)
\curveto(623.17107298,1048.75046422)(623.2510729,1048.79046418)(623.36107399,1048.81047188)
\lineto(623.45107399,1048.81047188)
\curveto(623.50107265,1048.82046415)(623.5510726,1048.82546414)(623.60107399,1048.82547188)
\lineto(623.76607399,1048.82547188)
\lineto(624.38107399,1048.82547188)
\curveto(624.46107169,1048.82546414)(624.53607162,1048.82046415)(624.60607399,1048.81047188)
\curveto(624.68607147,1048.81046416)(624.7560714,1048.80046417)(624.81607399,1048.78047188)
\curveto(624.89607126,1048.75046422)(624.94607121,1048.70046427)(624.96607399,1048.63047188)
\curveto(624.99607116,1048.56046441)(625.02107113,1048.48046449)(625.04107399,1048.39047188)
\curveto(625.0510711,1048.36046461)(625.0510711,1048.33046464)(625.04107399,1048.30047188)
\curveto(625.04107111,1048.28046469)(625.0510711,1048.26046471)(625.07107399,1048.24047188)
\curveto(625.08107107,1048.21046476)(625.09107106,1048.18546478)(625.10107399,1048.16547188)
\curveto(625.12107103,1048.15546481)(625.14107101,1048.14046483)(625.16107399,1048.12047188)
\curveto(625.28107087,1048.11046486)(625.38107077,1048.14546482)(625.46107399,1048.22547188)
\curveto(625.54107061,1048.31546465)(625.61607054,1048.38546458)(625.68607399,1048.43547188)
\curveto(625.82607033,1048.53546443)(625.96607019,1048.62546434)(626.10607399,1048.70547188)
\curveto(626.2560699,1048.78546418)(626.41606974,1048.85046412)(626.58607399,1048.90047188)
\curveto(626.67606948,1048.93046404)(626.76606939,1048.95046402)(626.85607399,1048.96047188)
\curveto(626.94606921,1048.970464)(627.04106911,1048.98546398)(627.14107399,1049.00547188)
\curveto(627.17106898,1049.01546395)(627.21606894,1049.01546395)(627.27607399,1049.00547188)
\curveto(627.33606882,1049.00546396)(627.37606878,1049.01046396)(627.39607399,1049.02047188)
}
}
{
\newrgbcolor{curcolor}{0 0 0}
\pscustom[linestyle=none,fillstyle=solid,fillcolor=curcolor]
{
\newpath
\moveto(643.58482399,1041.68547188)
\curveto(643.60481614,1041.57547139)(643.61481613,1041.4654715)(643.61482399,1041.35547188)
\curveto(643.62481612,1041.24547172)(643.57481617,1041.1704718)(643.46482399,1041.13047188)
\curveto(643.40481634,1041.10047187)(643.33481641,1041.08547188)(643.25482399,1041.08547188)
\lineto(643.01482399,1041.08547188)
\lineto(642.20482399,1041.08547188)
\lineto(641.93482399,1041.08547188)
\curveto(641.85481789,1041.09547187)(641.78981796,1041.12047185)(641.73982399,1041.16047188)
\curveto(641.66981808,1041.20047177)(641.61481813,1041.25547171)(641.57482399,1041.32547188)
\curveto(641.5448182,1041.40547156)(641.49981825,1041.4704715)(641.43982399,1041.52047188)
\curveto(641.41981833,1041.54047143)(641.39481835,1041.55547141)(641.36482399,1041.56547188)
\curveto(641.33481841,1041.58547138)(641.29481845,1041.59047138)(641.24482399,1041.58047188)
\curveto(641.19481855,1041.56047141)(641.1448186,1041.53547143)(641.09482399,1041.50547188)
\curveto(641.05481869,1041.47547149)(641.00981874,1041.45047152)(640.95982399,1041.43047188)
\curveto(640.90981884,1041.39047158)(640.85481889,1041.35547161)(640.79482399,1041.32547188)
\lineto(640.61482399,1041.23547188)
\curveto(640.48481926,1041.17547179)(640.3498194,1041.12547184)(640.20982399,1041.08547188)
\curveto(640.06981968,1041.05547191)(639.92481982,1041.02047195)(639.77482399,1040.98047188)
\curveto(639.70482004,1040.96047201)(639.63482011,1040.95047202)(639.56482399,1040.95047188)
\curveto(639.50482024,1040.94047203)(639.43982031,1040.93047204)(639.36982399,1040.92047188)
\lineto(639.27982399,1040.92047188)
\curveto(639.2498205,1040.91047206)(639.21982053,1040.90547206)(639.18982399,1040.90547188)
\lineto(639.02482399,1040.90547188)
\curveto(638.92482082,1040.88547208)(638.82482092,1040.88547208)(638.72482399,1040.90547188)
\lineto(638.58982399,1040.90547188)
\curveto(638.51982123,1040.92547204)(638.4498213,1040.93547203)(638.37982399,1040.93547188)
\curveto(638.31982143,1040.92547204)(638.25982149,1040.93047204)(638.19982399,1040.95047188)
\curveto(638.09982165,1040.970472)(638.00482174,1040.99047198)(637.91482399,1041.01047188)
\curveto(637.82482192,1041.02047195)(637.73982201,1041.04547192)(637.65982399,1041.08547188)
\curveto(637.36982238,1041.19547177)(637.11982263,1041.33547163)(636.90982399,1041.50547188)
\curveto(636.70982304,1041.68547128)(636.5498232,1041.92047105)(636.42982399,1042.21047188)
\curveto(636.39982335,1042.28047069)(636.36982338,1042.35547061)(636.33982399,1042.43547188)
\curveto(636.31982343,1042.51547045)(636.29982345,1042.60047037)(636.27982399,1042.69047188)
\curveto(636.25982349,1042.74047023)(636.2498235,1042.79047018)(636.24982399,1042.84047188)
\curveto(636.25982349,1042.89047008)(636.25982349,1042.94047003)(636.24982399,1042.99047188)
\curveto(636.23982351,1043.02046995)(636.22982352,1043.08046989)(636.21982399,1043.17047188)
\curveto(636.21982353,1043.2704697)(636.22482352,1043.34046963)(636.23482399,1043.38047188)
\curveto(636.25482349,1043.48046949)(636.26482348,1043.5654694)(636.26482399,1043.63547188)
\lineto(636.35482399,1043.96547188)
\curveto(636.38482336,1044.08546888)(636.42482332,1044.19046878)(636.47482399,1044.28047188)
\curveto(636.6448231,1044.5704684)(636.83982291,1044.79046818)(637.05982399,1044.94047188)
\curveto(637.27982247,1045.09046788)(637.55982219,1045.22046775)(637.89982399,1045.33047188)
\curveto(638.02982172,1045.38046759)(638.16482158,1045.41546755)(638.30482399,1045.43547188)
\curveto(638.4448213,1045.45546751)(638.58482116,1045.48046749)(638.72482399,1045.51047188)
\curveto(638.80482094,1045.53046744)(638.88982086,1045.54046743)(638.97982399,1045.54047188)
\curveto(639.06982068,1045.55046742)(639.15982059,1045.5654674)(639.24982399,1045.58547188)
\curveto(639.31982043,1045.60546736)(639.38982036,1045.61046736)(639.45982399,1045.60047188)
\curveto(639.52982022,1045.60046737)(639.60482014,1045.61046736)(639.68482399,1045.63047188)
\curveto(639.75481999,1045.65046732)(639.82481992,1045.66046731)(639.89482399,1045.66047188)
\curveto(639.96481978,1045.66046731)(640.03981971,1045.6704673)(640.11982399,1045.69047188)
\curveto(640.32981942,1045.74046723)(640.51981923,1045.78046719)(640.68982399,1045.81047188)
\curveto(640.86981888,1045.85046712)(641.02981872,1045.94046703)(641.16982399,1046.08047188)
\curveto(641.25981849,1046.1704668)(641.31981843,1046.2704667)(641.34982399,1046.38047188)
\curveto(641.35981839,1046.41046656)(641.35981839,1046.43546653)(641.34982399,1046.45547188)
\curveto(641.3498184,1046.47546649)(641.35481839,1046.49546647)(641.36482399,1046.51547188)
\curveto(641.37481837,1046.53546643)(641.37981837,1046.5654664)(641.37982399,1046.60547188)
\lineto(641.37982399,1046.69547188)
\lineto(641.34982399,1046.81547188)
\curveto(641.3498184,1046.85546611)(641.3448184,1046.89046608)(641.33482399,1046.92047188)
\curveto(641.23481851,1047.22046575)(641.02481872,1047.42546554)(640.70482399,1047.53547188)
\curveto(640.61481913,1047.5654654)(640.50481924,1047.58546538)(640.37482399,1047.59547188)
\curveto(640.25481949,1047.61546535)(640.12981962,1047.62046535)(639.99982399,1047.61047188)
\curveto(639.86981988,1047.61046536)(639.74482,1047.60046537)(639.62482399,1047.58047188)
\curveto(639.50482024,1047.56046541)(639.39982035,1047.53546543)(639.30982399,1047.50547188)
\curveto(639.2498205,1047.48546548)(639.18982056,1047.45546551)(639.12982399,1047.41547188)
\curveto(639.07982067,1047.38546558)(639.02982072,1047.35046562)(638.97982399,1047.31047188)
\curveto(638.92982082,1047.2704657)(638.87482087,1047.21546575)(638.81482399,1047.14547188)
\curveto(638.76482098,1047.07546589)(638.72982102,1047.01046596)(638.70982399,1046.95047188)
\curveto(638.65982109,1046.85046612)(638.61482113,1046.75546621)(638.57482399,1046.66547188)
\curveto(638.5448212,1046.57546639)(638.47482127,1046.51546645)(638.36482399,1046.48547188)
\curveto(638.28482146,1046.4654665)(638.19982155,1046.45546651)(638.10982399,1046.45547188)
\lineto(637.83982399,1046.45547188)
\lineto(637.26982399,1046.45547188)
\curveto(637.21982253,1046.45546651)(637.16982258,1046.45046652)(637.11982399,1046.44047188)
\curveto(637.06982268,1046.44046653)(637.02482272,1046.44546652)(636.98482399,1046.45547188)
\lineto(636.84982399,1046.45547188)
\curveto(636.82982292,1046.4654665)(636.80482294,1046.4704665)(636.77482399,1046.47047188)
\curveto(636.744823,1046.4704665)(636.71982303,1046.48046649)(636.69982399,1046.50047188)
\curveto(636.61982313,1046.52046645)(636.56482318,1046.58546638)(636.53482399,1046.69547188)
\curveto(636.52482322,1046.74546622)(636.52482322,1046.79546617)(636.53482399,1046.84547188)
\curveto(636.5448232,1046.89546607)(636.55482319,1046.94046603)(636.56482399,1046.98047188)
\curveto(636.59482315,1047.09046588)(636.62482312,1047.19046578)(636.65482399,1047.28047188)
\curveto(636.69482305,1047.38046559)(636.73982301,1047.4704655)(636.78982399,1047.55047188)
\lineto(636.87982399,1047.70047188)
\lineto(636.96982399,1047.85047188)
\curveto(637.0498227,1047.96046501)(637.1498226,1048.0654649)(637.26982399,1048.16547188)
\curveto(637.28982246,1048.17546479)(637.31982243,1048.20046477)(637.35982399,1048.24047188)
\curveto(637.40982234,1048.28046469)(637.45482229,1048.31546465)(637.49482399,1048.34547188)
\curveto(637.53482221,1048.37546459)(637.57982217,1048.40546456)(637.62982399,1048.43547188)
\curveto(637.79982195,1048.54546442)(637.97982177,1048.63046434)(638.16982399,1048.69047188)
\curveto(638.35982139,1048.76046421)(638.55482119,1048.82546414)(638.75482399,1048.88547188)
\curveto(638.87482087,1048.91546405)(638.99982075,1048.93546403)(639.12982399,1048.94547188)
\curveto(639.25982049,1048.95546401)(639.38982036,1048.97546399)(639.51982399,1049.00547188)
\curveto(639.55982019,1049.01546395)(639.61982013,1049.01546395)(639.69982399,1049.00547188)
\curveto(639.78981996,1048.99546397)(639.8448199,1049.00046397)(639.86482399,1049.02047188)
\curveto(640.27481947,1049.03046394)(640.66481908,1049.01546395)(641.03482399,1048.97547188)
\curveto(641.41481833,1048.93546403)(641.75481799,1048.86046411)(642.05482399,1048.75047188)
\curveto(642.36481738,1048.64046433)(642.62981712,1048.49046448)(642.84982399,1048.30047188)
\curveto(643.06981668,1048.12046485)(643.23981651,1047.88546508)(643.35982399,1047.59547188)
\curveto(643.42981632,1047.42546554)(643.46981628,1047.23046574)(643.47982399,1047.01047188)
\curveto(643.48981626,1046.79046618)(643.49481625,1046.5654664)(643.49482399,1046.33547188)
\lineto(643.49482399,1042.99047188)
\lineto(643.49482399,1042.40547188)
\curveto(643.49481625,1042.21547075)(643.51481623,1042.04047093)(643.55482399,1041.88047188)
\curveto(643.56481618,1041.85047112)(643.56981618,1041.81547115)(643.56982399,1041.77547188)
\curveto(643.56981618,1041.74547122)(643.57481617,1041.71547125)(643.58482399,1041.68547188)
\moveto(641.37982399,1043.99547188)
\curveto(641.38981836,1044.04546892)(641.39481835,1044.10046887)(641.39482399,1044.16047188)
\curveto(641.39481835,1044.23046874)(641.38981836,1044.29046868)(641.37982399,1044.34047188)
\curveto(641.35981839,1044.40046857)(641.3498184,1044.45546851)(641.34982399,1044.50547188)
\curveto(641.3498184,1044.55546841)(641.32981842,1044.59546837)(641.28982399,1044.62547188)
\curveto(641.23981851,1044.6654683)(641.16481858,1044.68546828)(641.06482399,1044.68547188)
\curveto(641.02481872,1044.67546829)(640.98981876,1044.6654683)(640.95982399,1044.65547188)
\curveto(640.92981882,1044.65546831)(640.89481885,1044.65046832)(640.85482399,1044.64047188)
\curveto(640.78481896,1044.62046835)(640.70981904,1044.60546836)(640.62982399,1044.59547188)
\curveto(640.5498192,1044.58546838)(640.46981928,1044.5704684)(640.38982399,1044.55047188)
\curveto(640.35981939,1044.54046843)(640.31481943,1044.53546843)(640.25482399,1044.53547188)
\curveto(640.12481962,1044.50546846)(639.99481975,1044.48546848)(639.86482399,1044.47547188)
\curveto(639.73482001,1044.4654685)(639.60982014,1044.44046853)(639.48982399,1044.40047188)
\curveto(639.40982034,1044.38046859)(639.33482041,1044.36046861)(639.26482399,1044.34047188)
\curveto(639.19482055,1044.33046864)(639.12482062,1044.31046866)(639.05482399,1044.28047188)
\curveto(638.8448209,1044.19046878)(638.66482108,1044.05546891)(638.51482399,1043.87547188)
\curveto(638.37482137,1043.69546927)(638.32482142,1043.44546952)(638.36482399,1043.12547188)
\curveto(638.38482136,1042.95547001)(638.43982131,1042.81547015)(638.52982399,1042.70547188)
\curveto(638.59982115,1042.59547037)(638.70482104,1042.50547046)(638.84482399,1042.43547188)
\curveto(638.98482076,1042.37547059)(639.13482061,1042.33047064)(639.29482399,1042.30047188)
\curveto(639.46482028,1042.2704707)(639.63982011,1042.26047071)(639.81982399,1042.27047188)
\curveto(640.00981974,1042.29047068)(640.18481956,1042.32547064)(640.34482399,1042.37547188)
\curveto(640.60481914,1042.45547051)(640.80981894,1042.58047039)(640.95982399,1042.75047188)
\curveto(641.10981864,1042.93047004)(641.22481852,1043.15046982)(641.30482399,1043.41047188)
\curveto(641.32481842,1043.48046949)(641.33481841,1043.55046942)(641.33482399,1043.62047188)
\curveto(641.3448184,1043.70046927)(641.35981839,1043.78046919)(641.37982399,1043.86047188)
\lineto(641.37982399,1043.99547188)
}
}
{
\newrgbcolor{curcolor}{0 0 0}
\pscustom[linestyle=none,fillstyle=solid,fillcolor=curcolor]
{
\newpath
\moveto(401.11427712,1026.29547188)
\curveto(401.12426844,1026.23546798)(401.12926843,1026.14546807)(401.12927712,1026.02547188)
\curveto(401.12926843,1025.90546831)(401.11926844,1025.8204684)(401.09927712,1025.77047188)
\lineto(401.09927712,1025.57547188)
\curveto(401.06926849,1025.46546875)(401.04926851,1025.36046886)(401.03927712,1025.26047188)
\curveto(401.03926852,1025.16046906)(401.02426854,1025.06046916)(400.99427712,1024.96047188)
\curveto(400.97426859,1024.87046935)(400.95426861,1024.77546944)(400.93427712,1024.67547188)
\curveto(400.91426865,1024.58546963)(400.88426868,1024.49546972)(400.84427712,1024.40547188)
\curveto(400.77426879,1024.23546998)(400.70426886,1024.07547014)(400.63427712,1023.92547188)
\curveto(400.564269,1023.78547043)(400.48426908,1023.64547057)(400.39427712,1023.50547188)
\curveto(400.33426923,1023.4154708)(400.26926929,1023.33047089)(400.19927712,1023.25047188)
\curveto(400.13926942,1023.18047104)(400.06926949,1023.10547111)(399.98927712,1023.02547188)
\lineto(399.88427712,1022.92047188)
\curveto(399.83426973,1022.87047135)(399.77926978,1022.82547139)(399.71927712,1022.78547188)
\lineto(399.56927712,1022.66547188)
\curveto(399.48927007,1022.60547161)(399.39927016,1022.55047167)(399.29927712,1022.50047188)
\curveto(399.20927035,1022.46047176)(399.11427045,1022.4154718)(399.01427712,1022.36547188)
\curveto(398.91427065,1022.3154719)(398.80927075,1022.28047194)(398.69927712,1022.26047188)
\curveto(398.59927096,1022.24047198)(398.49427107,1022.220472)(398.38427712,1022.20047188)
\curveto(398.32427124,1022.18047204)(398.2592713,1022.17047205)(398.18927712,1022.17047188)
\curveto(398.12927143,1022.17047205)(398.0642715,1022.16047206)(397.99427712,1022.14047188)
\lineto(397.85927712,1022.14047188)
\curveto(397.77927178,1022.1204721)(397.70427186,1022.1204721)(397.63427712,1022.14047188)
\lineto(397.48427712,1022.14047188)
\curveto(397.42427214,1022.16047206)(397.3592722,1022.17047205)(397.28927712,1022.17047188)
\curveto(397.22927233,1022.16047206)(397.16927239,1022.16547205)(397.10927712,1022.18547188)
\curveto(396.94927261,1022.23547198)(396.79427277,1022.28047194)(396.64427712,1022.32047188)
\curveto(396.50427306,1022.36047186)(396.37427319,1022.4204718)(396.25427712,1022.50047188)
\curveto(396.18427338,1022.54047168)(396.11927344,1022.58047164)(396.05927712,1022.62047188)
\curveto(395.99927356,1022.67047155)(395.93427363,1022.7204715)(395.86427712,1022.77047188)
\lineto(395.68427712,1022.90547188)
\curveto(395.60427396,1022.96547125)(395.53427403,1022.97047125)(395.47427712,1022.92047188)
\curveto(395.42427414,1022.89047133)(395.39927416,1022.85047137)(395.39927712,1022.80047188)
\curveto(395.39927416,1022.76047146)(395.38927417,1022.71047151)(395.36927712,1022.65047188)
\curveto(395.34927421,1022.55047167)(395.33927422,1022.43547178)(395.33927712,1022.30547188)
\curveto(395.34927421,1022.17547204)(395.35427421,1022.05547216)(395.35427712,1021.94547188)
\lineto(395.35427712,1020.41547188)
\curveto(395.35427421,1020.28547393)(395.34927421,1020.16047406)(395.33927712,1020.04047188)
\curveto(395.33927422,1019.91047431)(395.31427425,1019.80547441)(395.26427712,1019.72547188)
\curveto(395.23427433,1019.68547453)(395.17927438,1019.65547456)(395.09927712,1019.63547188)
\curveto(395.01927454,1019.6154746)(394.92927463,1019.60547461)(394.82927712,1019.60547188)
\curveto(394.72927483,1019.59547462)(394.62927493,1019.59547462)(394.52927712,1019.60547188)
\lineto(394.27427712,1019.60547188)
\lineto(393.86927712,1019.60547188)
\lineto(393.76427712,1019.60547188)
\curveto(393.72427584,1019.60547461)(393.68927587,1019.61047461)(393.65927712,1019.62047188)
\lineto(393.53927712,1019.62047188)
\curveto(393.36927619,1019.67047455)(393.27927628,1019.77047445)(393.26927712,1019.92047188)
\curveto(393.2592763,1020.06047416)(393.25427631,1020.23047399)(393.25427712,1020.43047188)
\lineto(393.25427712,1029.23547188)
\curveto(393.25427631,1029.34546487)(393.24927631,1029.46046476)(393.23927712,1029.58047188)
\curveto(393.23927632,1029.71046451)(393.2642763,1029.81046441)(393.31427712,1029.88047188)
\curveto(393.35427621,1029.95046427)(393.40927615,1029.99546422)(393.47927712,1030.01547188)
\curveto(393.52927603,1030.03546418)(393.58927597,1030.04546417)(393.65927712,1030.04547188)
\lineto(393.88427712,1030.04547188)
\lineto(394.60427712,1030.04547188)
\lineto(394.88927712,1030.04547188)
\curveto(394.97927458,1030.04546417)(395.05427451,1030.0204642)(395.11427712,1029.97047188)
\curveto(395.18427438,1029.9204643)(395.21927434,1029.85546436)(395.21927712,1029.77547188)
\curveto(395.22927433,1029.70546451)(395.25427431,1029.63046459)(395.29427712,1029.55047188)
\curveto(395.30427426,1029.5204647)(395.31427425,1029.49546472)(395.32427712,1029.47547188)
\curveto(395.34427422,1029.46546475)(395.3642742,1029.45046477)(395.38427712,1029.43047188)
\curveto(395.49427407,1029.4204648)(395.58427398,1029.45046477)(395.65427712,1029.52047188)
\curveto(395.72427384,1029.59046463)(395.79427377,1029.65046457)(395.86427712,1029.70047188)
\curveto(395.99427357,1029.79046443)(396.12927343,1029.87046435)(396.26927712,1029.94047188)
\curveto(396.40927315,1030.0204642)(396.564273,1030.08546413)(396.73427712,1030.13547188)
\curveto(396.81427275,1030.16546405)(396.89927266,1030.18546403)(396.98927712,1030.19547188)
\curveto(397.08927247,1030.20546401)(397.18427238,1030.220464)(397.27427712,1030.24047188)
\curveto(397.31427225,1030.25046397)(397.35427221,1030.25046397)(397.39427712,1030.24047188)
\curveto(397.44427212,1030.23046399)(397.48427208,1030.23546398)(397.51427712,1030.25547188)
\curveto(398.08427148,1030.27546394)(398.564271,1030.19546402)(398.95427712,1030.01547188)
\curveto(399.35427021,1029.84546437)(399.69426987,1029.6204646)(399.97427712,1029.34047188)
\curveto(400.02426954,1029.29046493)(400.06926949,1029.24046498)(400.10927712,1029.19047188)
\curveto(400.14926941,1029.15046507)(400.18926937,1029.10546511)(400.22927712,1029.05547188)
\curveto(400.29926926,1028.96546525)(400.3592692,1028.87546534)(400.40927712,1028.78547188)
\curveto(400.46926909,1028.69546552)(400.52426904,1028.60546561)(400.57427712,1028.51547188)
\curveto(400.59426897,1028.49546572)(400.60426896,1028.47046575)(400.60427712,1028.44047188)
\curveto(400.61426895,1028.41046581)(400.62926893,1028.37546584)(400.64927712,1028.33547188)
\curveto(400.70926885,1028.23546598)(400.7642688,1028.1154661)(400.81427712,1027.97547188)
\curveto(400.83426873,1027.9154663)(400.85426871,1027.85046637)(400.87427712,1027.78047188)
\curveto(400.89426867,1027.7204665)(400.91426865,1027.65546656)(400.93427712,1027.58547188)
\curveto(400.97426859,1027.46546675)(400.99926856,1027.34046688)(401.00927712,1027.21047188)
\curveto(401.02926853,1027.08046714)(401.05426851,1026.94546727)(401.08427712,1026.80547188)
\lineto(401.08427712,1026.64047188)
\lineto(401.11427712,1026.46047188)
\lineto(401.11427712,1026.29547188)
\moveto(398.99927712,1025.95047188)
\curveto(399.00927055,1026.00046822)(399.01427055,1026.06546815)(399.01427712,1026.14547188)
\curveto(399.01427055,1026.23546798)(399.00927055,1026.30546791)(398.99927712,1026.35547188)
\lineto(398.99927712,1026.49047188)
\curveto(398.97927058,1026.55046767)(398.96927059,1026.6154676)(398.96927712,1026.68547188)
\curveto(398.96927059,1026.75546746)(398.9592706,1026.82546739)(398.93927712,1026.89547188)
\curveto(398.91927064,1026.99546722)(398.89927066,1027.09046713)(398.87927712,1027.18047188)
\curveto(398.8592707,1027.28046694)(398.82927073,1027.37046685)(398.78927712,1027.45047188)
\curveto(398.66927089,1027.77046645)(398.51427105,1028.02546619)(398.32427712,1028.21547188)
\curveto(398.13427143,1028.40546581)(397.8642717,1028.54546567)(397.51427712,1028.63547188)
\curveto(397.43427213,1028.65546556)(397.34427222,1028.66546555)(397.24427712,1028.66547188)
\lineto(396.97427712,1028.66547188)
\curveto(396.93427263,1028.65546556)(396.89927266,1028.65046557)(396.86927712,1028.65047188)
\curveto(396.83927272,1028.65046557)(396.80427276,1028.64546557)(396.76427712,1028.63547188)
\lineto(396.55427712,1028.57547188)
\curveto(396.49427307,1028.56546565)(396.43427313,1028.54546567)(396.37427712,1028.51547188)
\curveto(396.11427345,1028.40546581)(395.90927365,1028.23546598)(395.75927712,1028.00547188)
\curveto(395.61927394,1027.77546644)(395.50427406,1027.5204667)(395.41427712,1027.24047188)
\curveto(395.39427417,1027.16046706)(395.37927418,1027.07546714)(395.36927712,1026.98547188)
\curveto(395.3592742,1026.90546731)(395.34427422,1026.82546739)(395.32427712,1026.74547188)
\curveto(395.31427425,1026.70546751)(395.30927425,1026.64046758)(395.30927712,1026.55047188)
\curveto(395.28927427,1026.51046771)(395.28427428,1026.46046776)(395.29427712,1026.40047188)
\curveto(395.30427426,1026.35046787)(395.30427426,1026.30046792)(395.29427712,1026.25047188)
\curveto(395.27427429,1026.19046803)(395.27427429,1026.13546808)(395.29427712,1026.08547188)
\lineto(395.29427712,1025.90547188)
\lineto(395.29427712,1025.77047188)
\curveto(395.29427427,1025.73046849)(395.30427426,1025.69046853)(395.32427712,1025.65047188)
\curveto(395.32427424,1025.58046864)(395.32927423,1025.52546869)(395.33927712,1025.48547188)
\lineto(395.36927712,1025.30547188)
\curveto(395.37927418,1025.24546897)(395.39427417,1025.18546903)(395.41427712,1025.12547188)
\curveto(395.50427406,1024.83546938)(395.60927395,1024.59546962)(395.72927712,1024.40547188)
\curveto(395.8592737,1024.22546999)(396.03927352,1024.06547015)(396.26927712,1023.92547188)
\curveto(396.40927315,1023.84547037)(396.57427299,1023.78047044)(396.76427712,1023.73047188)
\curveto(396.80427276,1023.7204705)(396.83927272,1023.7154705)(396.86927712,1023.71547188)
\curveto(396.89927266,1023.72547049)(396.93427263,1023.72547049)(396.97427712,1023.71547188)
\curveto(397.01427255,1023.70547051)(397.07427249,1023.69547052)(397.15427712,1023.68547188)
\curveto(397.23427233,1023.68547053)(397.29927226,1023.69047053)(397.34927712,1023.70047188)
\curveto(397.42927213,1023.7204705)(397.50927205,1023.73547048)(397.58927712,1023.74547188)
\curveto(397.67927188,1023.76547045)(397.7642718,1023.79047043)(397.84427712,1023.82047188)
\curveto(398.08427148,1023.9204703)(398.27927128,1024.06047016)(398.42927712,1024.24047188)
\curveto(398.57927098,1024.4204698)(398.70427086,1024.63046959)(398.80427712,1024.87047188)
\curveto(398.85427071,1024.99046923)(398.88927067,1025.1154691)(398.90927712,1025.24547188)
\curveto(398.92927063,1025.37546884)(398.95427061,1025.51046871)(398.98427712,1025.65047188)
\lineto(398.98427712,1025.80047188)
\curveto(398.99427057,1025.85046837)(398.99927056,1025.90046832)(398.99927712,1025.95047188)
}
}
{
\newrgbcolor{curcolor}{0 0 0}
\pscustom[linestyle=none,fillstyle=solid,fillcolor=curcolor]
{
\newpath
\moveto(410.16419899,1026.52047188)
\curveto(410.18419042,1026.46046776)(410.19419041,1026.37546784)(410.19419899,1026.26547188)
\curveto(410.19419041,1026.15546806)(410.18419042,1026.07046815)(410.16419899,1026.01047188)
\lineto(410.16419899,1025.86047188)
\curveto(410.14419046,1025.78046844)(410.13419047,1025.70046852)(410.13419899,1025.62047188)
\curveto(410.14419046,1025.54046868)(410.13919047,1025.46046876)(410.11919899,1025.38047188)
\curveto(410.09919051,1025.31046891)(410.08419052,1025.24546897)(410.07419899,1025.18547188)
\curveto(410.06419054,1025.12546909)(410.05419055,1025.06046916)(410.04419899,1024.99047188)
\curveto(410.0041906,1024.88046934)(409.96919064,1024.76546945)(409.93919899,1024.64547188)
\curveto(409.9091907,1024.53546968)(409.86919074,1024.43046979)(409.81919899,1024.33047188)
\curveto(409.609191,1023.85047037)(409.33419127,1023.46047076)(408.99419899,1023.16047188)
\curveto(408.65419195,1022.86047136)(408.24419236,1022.61047161)(407.76419899,1022.41047188)
\curveto(407.64419296,1022.36047186)(407.51919309,1022.32547189)(407.38919899,1022.30547188)
\curveto(407.26919334,1022.27547194)(407.14419346,1022.24547197)(407.01419899,1022.21547188)
\curveto(406.96419364,1022.19547202)(406.9091937,1022.18547203)(406.84919899,1022.18547188)
\curveto(406.78919382,1022.18547203)(406.73419387,1022.18047204)(406.68419899,1022.17047188)
\lineto(406.57919899,1022.17047188)
\curveto(406.54919406,1022.16047206)(406.51919409,1022.15547206)(406.48919899,1022.15547188)
\curveto(406.43919417,1022.14547207)(406.35919425,1022.14047208)(406.24919899,1022.14047188)
\curveto(406.13919447,1022.13047209)(406.05419455,1022.13547208)(405.99419899,1022.15547188)
\lineto(405.84419899,1022.15547188)
\curveto(405.79419481,1022.16547205)(405.73919487,1022.17047205)(405.67919899,1022.17047188)
\curveto(405.62919498,1022.16047206)(405.57919503,1022.16547205)(405.52919899,1022.18547188)
\curveto(405.48919512,1022.19547202)(405.44919516,1022.20047202)(405.40919899,1022.20047188)
\curveto(405.37919523,1022.20047202)(405.33919527,1022.20547201)(405.28919899,1022.21547188)
\curveto(405.18919542,1022.24547197)(405.08919552,1022.27047195)(404.98919899,1022.29047188)
\curveto(404.88919572,1022.31047191)(404.79419581,1022.34047188)(404.70419899,1022.38047188)
\curveto(404.58419602,1022.4204718)(404.46919614,1022.46047176)(404.35919899,1022.50047188)
\curveto(404.25919635,1022.54047168)(404.15419645,1022.59047163)(404.04419899,1022.65047188)
\curveto(403.69419691,1022.86047136)(403.39419721,1023.10547111)(403.14419899,1023.38547188)
\curveto(402.89419771,1023.66547055)(402.68419792,1024.00047022)(402.51419899,1024.39047188)
\curveto(402.46419814,1024.48046974)(402.42419818,1024.57546964)(402.39419899,1024.67547188)
\curveto(402.37419823,1024.77546944)(402.34919826,1024.88046934)(402.31919899,1024.99047188)
\curveto(402.29919831,1025.04046918)(402.28919832,1025.08546913)(402.28919899,1025.12547188)
\curveto(402.28919832,1025.16546905)(402.27919833,1025.21046901)(402.25919899,1025.26047188)
\curveto(402.23919837,1025.34046888)(402.22919838,1025.4204688)(402.22919899,1025.50047188)
\curveto(402.22919838,1025.59046863)(402.21919839,1025.67546854)(402.19919899,1025.75547188)
\curveto(402.18919842,1025.80546841)(402.18419842,1025.85046837)(402.18419899,1025.89047188)
\lineto(402.18419899,1026.02547188)
\curveto(402.16419844,1026.08546813)(402.15419845,1026.17046805)(402.15419899,1026.28047188)
\curveto(402.16419844,1026.39046783)(402.17919843,1026.47546774)(402.19919899,1026.53547188)
\lineto(402.19919899,1026.64047188)
\curveto(402.2091984,1026.69046753)(402.2091984,1026.74046748)(402.19919899,1026.79047188)
\curveto(402.19919841,1026.85046737)(402.2091984,1026.90546731)(402.22919899,1026.95547188)
\curveto(402.23919837,1027.00546721)(402.24419836,1027.05046717)(402.24419899,1027.09047188)
\curveto(402.24419836,1027.14046708)(402.25419835,1027.19046703)(402.27419899,1027.24047188)
\curveto(402.31419829,1027.37046685)(402.34919826,1027.49546672)(402.37919899,1027.61547188)
\curveto(402.4091982,1027.74546647)(402.44919816,1027.87046635)(402.49919899,1027.99047188)
\curveto(402.67919793,1028.40046582)(402.89419771,1028.74046548)(403.14419899,1029.01047188)
\curveto(403.39419721,1029.29046493)(403.69919691,1029.54546467)(404.05919899,1029.77547188)
\curveto(404.15919645,1029.82546439)(404.26419634,1029.87046435)(404.37419899,1029.91047188)
\curveto(404.48419612,1029.95046427)(404.59419601,1029.99546422)(404.70419899,1030.04547188)
\curveto(404.83419577,1030.09546412)(404.96919564,1030.13046409)(405.10919899,1030.15047188)
\curveto(405.24919536,1030.17046405)(405.39419521,1030.20046402)(405.54419899,1030.24047188)
\curveto(405.62419498,1030.25046397)(405.69919491,1030.25546396)(405.76919899,1030.25547188)
\curveto(405.83919477,1030.25546396)(405.9091947,1030.26046396)(405.97919899,1030.27047188)
\curveto(406.55919405,1030.28046394)(407.05919355,1030.220464)(407.47919899,1030.09047188)
\curveto(407.9091927,1029.96046426)(408.28919232,1029.78046444)(408.61919899,1029.55047188)
\curveto(408.72919188,1029.47046475)(408.83919177,1029.38046484)(408.94919899,1029.28047188)
\curveto(409.06919154,1029.19046503)(409.16919144,1029.09046513)(409.24919899,1028.98047188)
\curveto(409.32919128,1028.88046534)(409.39919121,1028.78046544)(409.45919899,1028.68047188)
\curveto(409.52919108,1028.58046564)(409.59919101,1028.47546574)(409.66919899,1028.36547188)
\curveto(409.73919087,1028.25546596)(409.79419081,1028.13546608)(409.83419899,1028.00547188)
\curveto(409.87419073,1027.88546633)(409.91919069,1027.75546646)(409.96919899,1027.61547188)
\curveto(409.99919061,1027.53546668)(410.02419058,1027.45046677)(410.04419899,1027.36047188)
\lineto(410.10419899,1027.09047188)
\curveto(410.11419049,1027.05046717)(410.11919049,1027.01046721)(410.11919899,1026.97047188)
\curveto(410.11919049,1026.93046729)(410.12419048,1026.89046733)(410.13419899,1026.85047188)
\curveto(410.15419045,1026.80046742)(410.15919045,1026.74546747)(410.14919899,1026.68547188)
\curveto(410.13919047,1026.62546759)(410.14419046,1026.57046765)(410.16419899,1026.52047188)
\moveto(408.06419899,1025.98047188)
\curveto(408.07419253,1026.03046819)(408.07919253,1026.10046812)(408.07919899,1026.19047188)
\curveto(408.07919253,1026.29046793)(408.07419253,1026.36546785)(408.06419899,1026.41547188)
\lineto(408.06419899,1026.53547188)
\curveto(408.04419256,1026.58546763)(408.03419257,1026.64046758)(408.03419899,1026.70047188)
\curveto(408.03419257,1026.76046746)(408.02919258,1026.8154674)(408.01919899,1026.86547188)
\curveto(408.01919259,1026.90546731)(408.01419259,1026.93546728)(408.00419899,1026.95547188)
\lineto(407.94419899,1027.19547188)
\curveto(407.93419267,1027.28546693)(407.91419269,1027.37046685)(407.88419899,1027.45047188)
\curveto(407.77419283,1027.71046651)(407.64419296,1027.93046629)(407.49419899,1028.11047188)
\curveto(407.34419326,1028.30046592)(407.14419346,1028.45046577)(406.89419899,1028.56047188)
\curveto(406.83419377,1028.58046564)(406.77419383,1028.59546562)(406.71419899,1028.60547188)
\curveto(406.65419395,1028.62546559)(406.58919402,1028.64546557)(406.51919899,1028.66547188)
\curveto(406.43919417,1028.68546553)(406.35419425,1028.69046553)(406.26419899,1028.68047188)
\lineto(405.99419899,1028.68047188)
\curveto(405.96419464,1028.66046556)(405.92919468,1028.65046557)(405.88919899,1028.65047188)
\curveto(405.84919476,1028.66046556)(405.81419479,1028.66046556)(405.78419899,1028.65047188)
\lineto(405.57419899,1028.59047188)
\curveto(405.51419509,1028.58046564)(405.45919515,1028.56046566)(405.40919899,1028.53047188)
\curveto(405.15919545,1028.4204658)(404.95419565,1028.26046596)(404.79419899,1028.05047188)
\curveto(404.64419596,1027.85046637)(404.52419608,1027.6154666)(404.43419899,1027.34547188)
\curveto(404.4041962,1027.24546697)(404.37919623,1027.14046708)(404.35919899,1027.03047188)
\curveto(404.34919626,1026.9204673)(404.33419627,1026.81046741)(404.31419899,1026.70047188)
\curveto(404.3041963,1026.65046757)(404.29919631,1026.60046762)(404.29919899,1026.55047188)
\lineto(404.29919899,1026.40047188)
\curveto(404.27919633,1026.33046789)(404.26919634,1026.22546799)(404.26919899,1026.08547188)
\curveto(404.27919633,1025.94546827)(404.29419631,1025.84046838)(404.31419899,1025.77047188)
\lineto(404.31419899,1025.63547188)
\curveto(404.33419627,1025.55546866)(404.34919626,1025.47546874)(404.35919899,1025.39547188)
\curveto(404.36919624,1025.32546889)(404.38419622,1025.25046897)(404.40419899,1025.17047188)
\curveto(404.5041961,1024.87046935)(404.609196,1024.62546959)(404.71919899,1024.43547188)
\curveto(404.83919577,1024.25546996)(405.02419558,1024.09047013)(405.27419899,1023.94047188)
\curveto(405.34419526,1023.89047033)(405.41919519,1023.85047037)(405.49919899,1023.82047188)
\curveto(405.58919502,1023.79047043)(405.67919493,1023.76547045)(405.76919899,1023.74547188)
\curveto(405.8091948,1023.73547048)(405.84419476,1023.73047049)(405.87419899,1023.73047188)
\curveto(405.9041947,1023.74047048)(405.93919467,1023.74047048)(405.97919899,1023.73047188)
\lineto(406.09919899,1023.70047188)
\curveto(406.14919446,1023.70047052)(406.19419441,1023.70547051)(406.23419899,1023.71547188)
\lineto(406.35419899,1023.71547188)
\curveto(406.43419417,1023.73547048)(406.51419409,1023.75047047)(406.59419899,1023.76047188)
\curveto(406.67419393,1023.77047045)(406.74919386,1023.79047043)(406.81919899,1023.82047188)
\curveto(407.07919353,1023.9204703)(407.28919332,1024.05547016)(407.44919899,1024.22547188)
\curveto(407.609193,1024.39546982)(407.74419286,1024.60546961)(407.85419899,1024.85547188)
\curveto(407.89419271,1024.95546926)(407.92419268,1025.05546916)(407.94419899,1025.15547188)
\curveto(407.96419264,1025.25546896)(407.98919262,1025.36046886)(408.01919899,1025.47047188)
\curveto(408.02919258,1025.51046871)(408.03419257,1025.54546867)(408.03419899,1025.57547188)
\curveto(408.03419257,1025.6154686)(408.03919257,1025.65546856)(408.04919899,1025.69547188)
\lineto(408.04919899,1025.83047188)
\curveto(408.04919256,1025.88046834)(408.05419255,1025.93046829)(408.06419899,1025.98047188)
}
}
{
\newrgbcolor{curcolor}{0 0 0}
\pscustom[linestyle=none,fillstyle=solid,fillcolor=curcolor]
{
\newpath
\moveto(415.98912087,1030.27047188)
\curveto(416.09911555,1030.27046395)(416.19411546,1030.26046396)(416.27412087,1030.24047188)
\curveto(416.36411529,1030.220464)(416.43411522,1030.17546404)(416.48412087,1030.10547188)
\curveto(416.54411511,1030.02546419)(416.57411508,1029.88546433)(416.57412087,1029.68547188)
\lineto(416.57412087,1029.17547188)
\lineto(416.57412087,1028.80047188)
\curveto(416.58411507,1028.66046556)(416.56911508,1028.55046567)(416.52912087,1028.47047188)
\curveto(416.48911516,1028.40046582)(416.42911522,1028.35546586)(416.34912087,1028.33547188)
\curveto(416.27911537,1028.3154659)(416.19411546,1028.30546591)(416.09412087,1028.30547188)
\curveto(416.00411565,1028.30546591)(415.90411575,1028.31046591)(415.79412087,1028.32047188)
\curveto(415.69411596,1028.33046589)(415.59911605,1028.32546589)(415.50912087,1028.30547188)
\curveto(415.43911621,1028.28546593)(415.36911628,1028.27046595)(415.29912087,1028.26047188)
\curveto(415.22911642,1028.26046596)(415.16411649,1028.25046597)(415.10412087,1028.23047188)
\curveto(414.94411671,1028.18046604)(414.78411687,1028.10546611)(414.62412087,1028.00547188)
\curveto(414.46411719,1027.9154663)(414.33911731,1027.81046641)(414.24912087,1027.69047188)
\curveto(414.19911745,1027.61046661)(414.14411751,1027.52546669)(414.08412087,1027.43547188)
\curveto(414.03411762,1027.35546686)(413.98411767,1027.27046695)(413.93412087,1027.18047188)
\curveto(413.90411775,1027.10046712)(413.87411778,1027.0154672)(413.84412087,1026.92547188)
\lineto(413.78412087,1026.68547188)
\curveto(413.76411789,1026.6154676)(413.7541179,1026.54046768)(413.75412087,1026.46047188)
\curveto(413.7541179,1026.39046783)(413.74411791,1026.3204679)(413.72412087,1026.25047188)
\curveto(413.71411794,1026.21046801)(413.70911794,1026.17046805)(413.70912087,1026.13047188)
\curveto(413.71911793,1026.10046812)(413.71911793,1026.07046815)(413.70912087,1026.04047188)
\lineto(413.70912087,1025.80047188)
\curveto(413.68911796,1025.73046849)(413.68411797,1025.65046857)(413.69412087,1025.56047188)
\curveto(413.70411795,1025.48046874)(413.70911794,1025.40046882)(413.70912087,1025.32047188)
\lineto(413.70912087,1024.36047188)
\lineto(413.70912087,1023.08547188)
\curveto(413.70911794,1022.95547126)(413.70411795,1022.83547138)(413.69412087,1022.72547188)
\curveto(413.68411797,1022.6154716)(413.654118,1022.52547169)(413.60412087,1022.45547188)
\curveto(413.58411807,1022.42547179)(413.5491181,1022.40047182)(413.49912087,1022.38047188)
\curveto(413.45911819,1022.37047185)(413.41411824,1022.36047186)(413.36412087,1022.35047188)
\lineto(413.28912087,1022.35047188)
\curveto(413.23911841,1022.34047188)(413.18411847,1022.33547188)(413.12412087,1022.33547188)
\lineto(412.95912087,1022.33547188)
\lineto(412.31412087,1022.33547188)
\curveto(412.2541194,1022.34547187)(412.18911946,1022.35047187)(412.11912087,1022.35047188)
\lineto(411.92412087,1022.35047188)
\curveto(411.87411978,1022.37047185)(411.82411983,1022.38547183)(411.77412087,1022.39547188)
\curveto(411.72411993,1022.4154718)(411.68911996,1022.45047177)(411.66912087,1022.50047188)
\curveto(411.62912002,1022.55047167)(411.60412005,1022.6204716)(411.59412087,1022.71047188)
\lineto(411.59412087,1023.01047188)
\lineto(411.59412087,1024.03047188)
\lineto(411.59412087,1028.26047188)
\lineto(411.59412087,1029.37047188)
\lineto(411.59412087,1029.65547188)
\curveto(411.59412006,1029.75546446)(411.61412004,1029.83546438)(411.65412087,1029.89547188)
\curveto(411.70411995,1029.97546424)(411.77911987,1030.02546419)(411.87912087,1030.04547188)
\curveto(411.97911967,1030.06546415)(412.09911955,1030.07546414)(412.23912087,1030.07547188)
\lineto(413.00412087,1030.07547188)
\curveto(413.12411853,1030.07546414)(413.22911842,1030.06546415)(413.31912087,1030.04547188)
\curveto(413.40911824,1030.03546418)(413.47911817,1029.99046423)(413.52912087,1029.91047188)
\curveto(413.55911809,1029.86046436)(413.57411808,1029.79046443)(413.57412087,1029.70047188)
\lineto(413.60412087,1029.43047188)
\curveto(413.61411804,1029.35046487)(413.62911802,1029.27546494)(413.64912087,1029.20547188)
\curveto(413.67911797,1029.13546508)(413.72911792,1029.10046512)(413.79912087,1029.10047188)
\curveto(413.81911783,1029.1204651)(413.83911781,1029.13046509)(413.85912087,1029.13047188)
\curveto(413.87911777,1029.13046509)(413.89911775,1029.14046508)(413.91912087,1029.16047188)
\curveto(413.97911767,1029.21046501)(414.02911762,1029.26546495)(414.06912087,1029.32547188)
\curveto(414.11911753,1029.39546482)(414.17911747,1029.45546476)(414.24912087,1029.50547188)
\curveto(414.28911736,1029.53546468)(414.32411733,1029.56546465)(414.35412087,1029.59547188)
\curveto(414.38411727,1029.63546458)(414.41911723,1029.67046455)(414.45912087,1029.70047188)
\lineto(414.72912087,1029.88047188)
\curveto(414.82911682,1029.94046428)(414.92911672,1029.99546422)(415.02912087,1030.04547188)
\curveto(415.12911652,1030.08546413)(415.22911642,1030.1204641)(415.32912087,1030.15047188)
\lineto(415.65912087,1030.24047188)
\curveto(415.68911596,1030.25046397)(415.74411591,1030.25046397)(415.82412087,1030.24047188)
\curveto(415.91411574,1030.24046398)(415.96911568,1030.25046397)(415.98912087,1030.27047188)
}
}
{
\newrgbcolor{curcolor}{0 0 0}
\pscustom[linestyle=none,fillstyle=solid,fillcolor=curcolor]
{
}
}
{
\newrgbcolor{curcolor}{0 0 0}
\pscustom[linestyle=none,fillstyle=solid,fillcolor=curcolor]
{
\newpath
\moveto(422.06435524,1033.03047188)
\lineto(423.15935524,1033.03047188)
\curveto(423.25935276,1033.03046119)(423.35435266,1033.02546119)(423.44435524,1033.01547188)
\curveto(423.53435248,1033.00546121)(423.60435241,1032.97546124)(423.65435524,1032.92547188)
\curveto(423.7143523,1032.85546136)(423.74435227,1032.76046146)(423.74435524,1032.64047188)
\curveto(423.75435226,1032.53046169)(423.75935226,1032.4154618)(423.75935524,1032.29547188)
\lineto(423.75935524,1030.96047188)
\lineto(423.75935524,1025.57547188)
\lineto(423.75935524,1023.28047188)
\lineto(423.75935524,1022.86047188)
\curveto(423.76935225,1022.71047151)(423.74935227,1022.59547162)(423.69935524,1022.51547188)
\curveto(423.64935237,1022.43547178)(423.55935246,1022.38047184)(423.42935524,1022.35047188)
\curveto(423.36935265,1022.33047189)(423.29935272,1022.32547189)(423.21935524,1022.33547188)
\curveto(423.14935287,1022.34547187)(423.07935294,1022.35047187)(423.00935524,1022.35047188)
\lineto(422.28935524,1022.35047188)
\curveto(422.17935384,1022.35047187)(422.07935394,1022.35547186)(421.98935524,1022.36547188)
\curveto(421.89935412,1022.37547184)(421.82435419,1022.40547181)(421.76435524,1022.45547188)
\curveto(421.70435431,1022.50547171)(421.66935435,1022.58047164)(421.65935524,1022.68047188)
\lineto(421.65935524,1023.01047188)
\lineto(421.65935524,1024.34547188)
\lineto(421.65935524,1029.97047188)
\lineto(421.65935524,1032.01047188)
\curveto(421.65935436,1032.14046208)(421.65435436,1032.29546192)(421.64435524,1032.47547188)
\curveto(421.64435437,1032.65546156)(421.66935435,1032.78546143)(421.71935524,1032.86547188)
\curveto(421.73935428,1032.90546131)(421.76435425,1032.93546128)(421.79435524,1032.95547188)
\lineto(421.91435524,1033.01547188)
\curveto(421.93435408,1033.0154612)(421.95935406,1033.0154612)(421.98935524,1033.01547188)
\curveto(422.019354,1033.02546119)(422.04435397,1033.03046119)(422.06435524,1033.03047188)
}
}
{
\newrgbcolor{curcolor}{0 0 0}
\pscustom[linestyle=none,fillstyle=solid,fillcolor=curcolor]
{
\newpath
\moveto(433.19154274,1026.52047188)
\curveto(433.21153417,1026.46046776)(433.22153416,1026.37546784)(433.22154274,1026.26547188)
\curveto(433.22153416,1026.15546806)(433.21153417,1026.07046815)(433.19154274,1026.01047188)
\lineto(433.19154274,1025.86047188)
\curveto(433.17153421,1025.78046844)(433.16153422,1025.70046852)(433.16154274,1025.62047188)
\curveto(433.17153421,1025.54046868)(433.16653422,1025.46046876)(433.14654274,1025.38047188)
\curveto(433.12653426,1025.31046891)(433.11153427,1025.24546897)(433.10154274,1025.18547188)
\curveto(433.09153429,1025.12546909)(433.0815343,1025.06046916)(433.07154274,1024.99047188)
\curveto(433.03153435,1024.88046934)(432.99653439,1024.76546945)(432.96654274,1024.64547188)
\curveto(432.93653445,1024.53546968)(432.89653449,1024.43046979)(432.84654274,1024.33047188)
\curveto(432.63653475,1023.85047037)(432.36153502,1023.46047076)(432.02154274,1023.16047188)
\curveto(431.6815357,1022.86047136)(431.27153611,1022.61047161)(430.79154274,1022.41047188)
\curveto(430.67153671,1022.36047186)(430.54653684,1022.32547189)(430.41654274,1022.30547188)
\curveto(430.29653709,1022.27547194)(430.17153721,1022.24547197)(430.04154274,1022.21547188)
\curveto(429.99153739,1022.19547202)(429.93653745,1022.18547203)(429.87654274,1022.18547188)
\curveto(429.81653757,1022.18547203)(429.76153762,1022.18047204)(429.71154274,1022.17047188)
\lineto(429.60654274,1022.17047188)
\curveto(429.57653781,1022.16047206)(429.54653784,1022.15547206)(429.51654274,1022.15547188)
\curveto(429.46653792,1022.14547207)(429.386538,1022.14047208)(429.27654274,1022.14047188)
\curveto(429.16653822,1022.13047209)(429.0815383,1022.13547208)(429.02154274,1022.15547188)
\lineto(428.87154274,1022.15547188)
\curveto(428.82153856,1022.16547205)(428.76653862,1022.17047205)(428.70654274,1022.17047188)
\curveto(428.65653873,1022.16047206)(428.60653878,1022.16547205)(428.55654274,1022.18547188)
\curveto(428.51653887,1022.19547202)(428.47653891,1022.20047202)(428.43654274,1022.20047188)
\curveto(428.40653898,1022.20047202)(428.36653902,1022.20547201)(428.31654274,1022.21547188)
\curveto(428.21653917,1022.24547197)(428.11653927,1022.27047195)(428.01654274,1022.29047188)
\curveto(427.91653947,1022.31047191)(427.82153956,1022.34047188)(427.73154274,1022.38047188)
\curveto(427.61153977,1022.4204718)(427.49653989,1022.46047176)(427.38654274,1022.50047188)
\curveto(427.2865401,1022.54047168)(427.1815402,1022.59047163)(427.07154274,1022.65047188)
\curveto(426.72154066,1022.86047136)(426.42154096,1023.10547111)(426.17154274,1023.38547188)
\curveto(425.92154146,1023.66547055)(425.71154167,1024.00047022)(425.54154274,1024.39047188)
\curveto(425.49154189,1024.48046974)(425.45154193,1024.57546964)(425.42154274,1024.67547188)
\curveto(425.40154198,1024.77546944)(425.37654201,1024.88046934)(425.34654274,1024.99047188)
\curveto(425.32654206,1025.04046918)(425.31654207,1025.08546913)(425.31654274,1025.12547188)
\curveto(425.31654207,1025.16546905)(425.30654208,1025.21046901)(425.28654274,1025.26047188)
\curveto(425.26654212,1025.34046888)(425.25654213,1025.4204688)(425.25654274,1025.50047188)
\curveto(425.25654213,1025.59046863)(425.24654214,1025.67546854)(425.22654274,1025.75547188)
\curveto(425.21654217,1025.80546841)(425.21154217,1025.85046837)(425.21154274,1025.89047188)
\lineto(425.21154274,1026.02547188)
\curveto(425.19154219,1026.08546813)(425.1815422,1026.17046805)(425.18154274,1026.28047188)
\curveto(425.19154219,1026.39046783)(425.20654218,1026.47546774)(425.22654274,1026.53547188)
\lineto(425.22654274,1026.64047188)
\curveto(425.23654215,1026.69046753)(425.23654215,1026.74046748)(425.22654274,1026.79047188)
\curveto(425.22654216,1026.85046737)(425.23654215,1026.90546731)(425.25654274,1026.95547188)
\curveto(425.26654212,1027.00546721)(425.27154211,1027.05046717)(425.27154274,1027.09047188)
\curveto(425.27154211,1027.14046708)(425.2815421,1027.19046703)(425.30154274,1027.24047188)
\curveto(425.34154204,1027.37046685)(425.37654201,1027.49546672)(425.40654274,1027.61547188)
\curveto(425.43654195,1027.74546647)(425.47654191,1027.87046635)(425.52654274,1027.99047188)
\curveto(425.70654168,1028.40046582)(425.92154146,1028.74046548)(426.17154274,1029.01047188)
\curveto(426.42154096,1029.29046493)(426.72654066,1029.54546467)(427.08654274,1029.77547188)
\curveto(427.1865402,1029.82546439)(427.29154009,1029.87046435)(427.40154274,1029.91047188)
\curveto(427.51153987,1029.95046427)(427.62153976,1029.99546422)(427.73154274,1030.04547188)
\curveto(427.86153952,1030.09546412)(427.99653939,1030.13046409)(428.13654274,1030.15047188)
\curveto(428.27653911,1030.17046405)(428.42153896,1030.20046402)(428.57154274,1030.24047188)
\curveto(428.65153873,1030.25046397)(428.72653866,1030.25546396)(428.79654274,1030.25547188)
\curveto(428.86653852,1030.25546396)(428.93653845,1030.26046396)(429.00654274,1030.27047188)
\curveto(429.5865378,1030.28046394)(430.0865373,1030.220464)(430.50654274,1030.09047188)
\curveto(430.93653645,1029.96046426)(431.31653607,1029.78046444)(431.64654274,1029.55047188)
\curveto(431.75653563,1029.47046475)(431.86653552,1029.38046484)(431.97654274,1029.28047188)
\curveto(432.09653529,1029.19046503)(432.19653519,1029.09046513)(432.27654274,1028.98047188)
\curveto(432.35653503,1028.88046534)(432.42653496,1028.78046544)(432.48654274,1028.68047188)
\curveto(432.55653483,1028.58046564)(432.62653476,1028.47546574)(432.69654274,1028.36547188)
\curveto(432.76653462,1028.25546596)(432.82153456,1028.13546608)(432.86154274,1028.00547188)
\curveto(432.90153448,1027.88546633)(432.94653444,1027.75546646)(432.99654274,1027.61547188)
\curveto(433.02653436,1027.53546668)(433.05153433,1027.45046677)(433.07154274,1027.36047188)
\lineto(433.13154274,1027.09047188)
\curveto(433.14153424,1027.05046717)(433.14653424,1027.01046721)(433.14654274,1026.97047188)
\curveto(433.14653424,1026.93046729)(433.15153423,1026.89046733)(433.16154274,1026.85047188)
\curveto(433.1815342,1026.80046742)(433.1865342,1026.74546747)(433.17654274,1026.68547188)
\curveto(433.16653422,1026.62546759)(433.17153421,1026.57046765)(433.19154274,1026.52047188)
\moveto(431.09154274,1025.98047188)
\curveto(431.10153628,1026.03046819)(431.10653628,1026.10046812)(431.10654274,1026.19047188)
\curveto(431.10653628,1026.29046793)(431.10153628,1026.36546785)(431.09154274,1026.41547188)
\lineto(431.09154274,1026.53547188)
\curveto(431.07153631,1026.58546763)(431.06153632,1026.64046758)(431.06154274,1026.70047188)
\curveto(431.06153632,1026.76046746)(431.05653633,1026.8154674)(431.04654274,1026.86547188)
\curveto(431.04653634,1026.90546731)(431.04153634,1026.93546728)(431.03154274,1026.95547188)
\lineto(430.97154274,1027.19547188)
\curveto(430.96153642,1027.28546693)(430.94153644,1027.37046685)(430.91154274,1027.45047188)
\curveto(430.80153658,1027.71046651)(430.67153671,1027.93046629)(430.52154274,1028.11047188)
\curveto(430.37153701,1028.30046592)(430.17153721,1028.45046577)(429.92154274,1028.56047188)
\curveto(429.86153752,1028.58046564)(429.80153758,1028.59546562)(429.74154274,1028.60547188)
\curveto(429.6815377,1028.62546559)(429.61653777,1028.64546557)(429.54654274,1028.66547188)
\curveto(429.46653792,1028.68546553)(429.381538,1028.69046553)(429.29154274,1028.68047188)
\lineto(429.02154274,1028.68047188)
\curveto(428.99153839,1028.66046556)(428.95653843,1028.65046557)(428.91654274,1028.65047188)
\curveto(428.87653851,1028.66046556)(428.84153854,1028.66046556)(428.81154274,1028.65047188)
\lineto(428.60154274,1028.59047188)
\curveto(428.54153884,1028.58046564)(428.4865389,1028.56046566)(428.43654274,1028.53047188)
\curveto(428.1865392,1028.4204658)(427.9815394,1028.26046596)(427.82154274,1028.05047188)
\curveto(427.67153971,1027.85046637)(427.55153983,1027.6154666)(427.46154274,1027.34547188)
\curveto(427.43153995,1027.24546697)(427.40653998,1027.14046708)(427.38654274,1027.03047188)
\curveto(427.37654001,1026.9204673)(427.36154002,1026.81046741)(427.34154274,1026.70047188)
\curveto(427.33154005,1026.65046757)(427.32654006,1026.60046762)(427.32654274,1026.55047188)
\lineto(427.32654274,1026.40047188)
\curveto(427.30654008,1026.33046789)(427.29654009,1026.22546799)(427.29654274,1026.08547188)
\curveto(427.30654008,1025.94546827)(427.32154006,1025.84046838)(427.34154274,1025.77047188)
\lineto(427.34154274,1025.63547188)
\curveto(427.36154002,1025.55546866)(427.37654001,1025.47546874)(427.38654274,1025.39547188)
\curveto(427.39653999,1025.32546889)(427.41153997,1025.25046897)(427.43154274,1025.17047188)
\curveto(427.53153985,1024.87046935)(427.63653975,1024.62546959)(427.74654274,1024.43547188)
\curveto(427.86653952,1024.25546996)(428.05153933,1024.09047013)(428.30154274,1023.94047188)
\curveto(428.37153901,1023.89047033)(428.44653894,1023.85047037)(428.52654274,1023.82047188)
\curveto(428.61653877,1023.79047043)(428.70653868,1023.76547045)(428.79654274,1023.74547188)
\curveto(428.83653855,1023.73547048)(428.87153851,1023.73047049)(428.90154274,1023.73047188)
\curveto(428.93153845,1023.74047048)(428.96653842,1023.74047048)(429.00654274,1023.73047188)
\lineto(429.12654274,1023.70047188)
\curveto(429.17653821,1023.70047052)(429.22153816,1023.70547051)(429.26154274,1023.71547188)
\lineto(429.38154274,1023.71547188)
\curveto(429.46153792,1023.73547048)(429.54153784,1023.75047047)(429.62154274,1023.76047188)
\curveto(429.70153768,1023.77047045)(429.77653761,1023.79047043)(429.84654274,1023.82047188)
\curveto(430.10653728,1023.9204703)(430.31653707,1024.05547016)(430.47654274,1024.22547188)
\curveto(430.63653675,1024.39546982)(430.77153661,1024.60546961)(430.88154274,1024.85547188)
\curveto(430.92153646,1024.95546926)(430.95153643,1025.05546916)(430.97154274,1025.15547188)
\curveto(430.99153639,1025.25546896)(431.01653637,1025.36046886)(431.04654274,1025.47047188)
\curveto(431.05653633,1025.51046871)(431.06153632,1025.54546867)(431.06154274,1025.57547188)
\curveto(431.06153632,1025.6154686)(431.06653632,1025.65546856)(431.07654274,1025.69547188)
\lineto(431.07654274,1025.83047188)
\curveto(431.07653631,1025.88046834)(431.0815363,1025.93046829)(431.09154274,1025.98047188)
}
}
{
\newrgbcolor{curcolor}{0 0 0}
\pscustom[linestyle=none,fillstyle=solid,fillcolor=curcolor]
{
}
}
{
\newrgbcolor{curcolor}{0 0 0}
\pscustom[linestyle=none,fillstyle=solid,fillcolor=curcolor]
{
\newpath
\moveto(443.20662087,1030.27047188)
\curveto(443.57661526,1030.28046394)(443.90161494,1030.24046398)(444.18162087,1030.15047188)
\curveto(444.46161438,1030.06046416)(444.70661413,1029.93546428)(444.91662087,1029.77547188)
\curveto(444.99661384,1029.7154645)(445.06661377,1029.64546457)(445.12662087,1029.56547188)
\curveto(445.19661364,1029.48546473)(445.27161357,1029.40546481)(445.35162087,1029.32547188)
\curveto(445.37161347,1029.30546491)(445.40161344,1029.27546494)(445.44162087,1029.23547188)
\curveto(445.49161335,1029.20546501)(445.5416133,1029.20046502)(445.59162087,1029.22047188)
\curveto(445.70161314,1029.25046497)(445.80661303,1029.3204649)(445.90662087,1029.43047188)
\curveto(446.00661283,1029.55046467)(446.10161274,1029.64046458)(446.19162087,1029.70047188)
\curveto(446.33161251,1029.81046441)(446.48161236,1029.90046432)(446.64162087,1029.97047188)
\curveto(446.80161204,1030.05046417)(446.98161186,1030.12546409)(447.18162087,1030.19547188)
\curveto(447.26161158,1030.215464)(447.35661148,1030.23046399)(447.46662087,1030.24047188)
\curveto(447.58661125,1030.26046396)(447.70661113,1030.27046395)(447.82662087,1030.27047188)
\curveto(447.95661088,1030.28046394)(448.07661076,1030.28046394)(448.18662087,1030.27047188)
\curveto(448.30661053,1030.26046396)(448.41161043,1030.24546397)(448.50162087,1030.22547188)
\curveto(448.55161029,1030.215464)(448.59661024,1030.21046401)(448.63662087,1030.21047188)
\curveto(448.67661016,1030.21046401)(448.72161012,1030.20046402)(448.77162087,1030.18047188)
\curveto(448.91160993,1030.14046408)(449.04660979,1030.10046412)(449.17662087,1030.06047188)
\curveto(449.30660953,1030.0204642)(449.42660941,1029.96546425)(449.53662087,1029.89547188)
\curveto(449.95660888,1029.63546458)(450.27160857,1029.25546496)(450.48162087,1028.75547188)
\curveto(450.52160832,1028.66546555)(450.55160829,1028.57046565)(450.57162087,1028.47047188)
\curveto(450.59160825,1028.38046584)(450.61160823,1028.29046593)(450.63162087,1028.20047188)
\curveto(450.6416082,1028.13046609)(450.64660819,1028.06546615)(450.64662087,1028.00547188)
\curveto(450.65660818,1027.94546627)(450.66660817,1027.88546633)(450.67662087,1027.82547188)
\lineto(450.67662087,1027.67547188)
\curveto(450.68660815,1027.6154666)(450.68660815,1027.54546667)(450.67662087,1027.46547188)
\curveto(450.67660816,1027.38546683)(450.67660816,1027.31046691)(450.67662087,1027.24047188)
\lineto(450.67662087,1026.37047188)
\lineto(450.67662087,1023.44547188)
\curveto(450.67660816,1023.36547085)(450.67660816,1023.27047095)(450.67662087,1023.16047188)
\curveto(450.68660815,1023.06047116)(450.68660815,1022.96047126)(450.67662087,1022.86047188)
\curveto(450.67660816,1022.77047145)(450.66660817,1022.68047154)(450.64662087,1022.59047188)
\curveto(450.62660821,1022.51047171)(450.59660824,1022.45547176)(450.55662087,1022.42547188)
\curveto(450.49660834,1022.37547184)(450.41660842,1022.34547187)(450.31662087,1022.33547188)
\lineto(450.01662087,1022.33547188)
\lineto(449.22162087,1022.33547188)
\curveto(449.08160976,1022.33547188)(448.95660988,1022.34547187)(448.84662087,1022.36547188)
\curveto(448.7366101,1022.38547183)(448.66161018,1022.44047178)(448.62162087,1022.53047188)
\curveto(448.59161025,1022.60047162)(448.57661026,1022.67547154)(448.57662087,1022.75547188)
\curveto(448.57661026,1022.84547137)(448.57661026,1022.93047129)(448.57662087,1023.01047188)
\lineto(448.57662087,1023.85047188)
\lineto(448.57662087,1025.87547188)
\lineto(448.57662087,1026.50547188)
\curveto(448.57661026,1026.55546766)(448.57661026,1026.61046761)(448.57662087,1026.67047188)
\curveto(448.58661025,1026.73046749)(448.58161026,1026.78546743)(448.56162087,1026.83547188)
\lineto(448.56162087,1026.95547188)
\curveto(448.56161028,1027.0154672)(448.56161028,1027.07546714)(448.56162087,1027.13547188)
\curveto(448.56161028,1027.19546702)(448.55661028,1027.25546696)(448.54662087,1027.31547188)
\curveto(448.5366103,1027.35546686)(448.53161031,1027.39546682)(448.53162087,1027.43547188)
\curveto(448.53161031,1027.48546673)(448.52661031,1027.53046669)(448.51662087,1027.57047188)
\curveto(448.47661036,1027.7204665)(448.43161041,1027.85046637)(448.38162087,1027.96047188)
\curveto(448.3416105,1028.08046614)(448.27661056,1028.18546603)(448.18662087,1028.27547188)
\curveto(448.04661079,1028.4154658)(447.87661096,1028.5154657)(447.67662087,1028.57547188)
\curveto(447.6366112,1028.58546563)(447.60161124,1028.58546563)(447.57162087,1028.57547188)
\curveto(447.5416113,1028.57546564)(447.50661133,1028.58546563)(447.46662087,1028.60547188)
\curveto(447.42661141,1028.6154656)(447.37661146,1028.6204656)(447.31662087,1028.62047188)
\curveto(447.26661157,1028.63046559)(447.21661162,1028.63046559)(447.16662087,1028.62047188)
\curveto(447.10661173,1028.60046562)(447.04661179,1028.59046563)(446.98662087,1028.59047188)
\curveto(446.92661191,1028.59046563)(446.86661197,1028.58046564)(446.80662087,1028.56047188)
\curveto(446.51661232,1028.46046576)(446.30661253,1028.31046591)(446.17662087,1028.11047188)
\curveto(446.00661283,1027.88046634)(445.90161294,1027.59046663)(445.86162087,1027.24047188)
\curveto(445.83161301,1026.90046732)(445.81661302,1026.52546769)(445.81662087,1026.11547188)
\lineto(445.81662087,1024.13547188)
\lineto(445.81662087,1023.02547188)
\lineto(445.81662087,1022.72547188)
\curveto(445.81661302,1022.62547159)(445.79161305,1022.54547167)(445.74162087,1022.48547188)
\curveto(445.69161315,1022.4154718)(445.61661322,1022.37047185)(445.51662087,1022.35047188)
\curveto(445.42661341,1022.34047188)(445.32161352,1022.33547188)(445.20162087,1022.33547188)
\lineto(444.39162087,1022.33547188)
\lineto(444.12162087,1022.33547188)
\curveto(444.0416148,1022.34547187)(443.97161487,1022.36047186)(443.91162087,1022.38047188)
\curveto(443.81161503,1022.43047179)(443.75161509,1022.51047171)(443.73162087,1022.62047188)
\curveto(443.72161512,1022.73047149)(443.71661512,1022.85547136)(443.71662087,1022.99547188)
\lineto(443.71662087,1024.27047188)
\lineto(443.71662087,1026.62547188)
\curveto(443.71661512,1026.9154673)(443.70661513,1027.19046703)(443.68662087,1027.45047188)
\curveto(443.66661517,1027.71046651)(443.60161524,1027.92546629)(443.49162087,1028.09547188)
\curveto(443.41161543,1028.23546598)(443.30661553,1028.34046588)(443.17662087,1028.41047188)
\curveto(443.05661578,1028.48046574)(442.90661593,1028.54046568)(442.72662087,1028.59047188)
\curveto(442.68661615,1028.60046562)(442.64661619,1028.60046562)(442.60662087,1028.59047188)
\curveto(442.56661627,1028.59046563)(442.52161632,1028.59546562)(442.47162087,1028.60547188)
\curveto(442.36161648,1028.62546559)(442.25661658,1028.6154656)(442.15662087,1028.57547188)
\curveto(442.1366167,1028.57546564)(442.11661672,1028.57046565)(442.09662087,1028.56047188)
\lineto(442.03662087,1028.56047188)
\curveto(441.87661696,1028.51046571)(441.72161712,1028.42546579)(441.57162087,1028.30547188)
\curveto(441.41161743,1028.18546603)(441.28661755,1028.04546617)(441.19662087,1027.88547188)
\curveto(441.11661772,1027.73546648)(441.05661778,1027.56046666)(441.01662087,1027.36047188)
\curveto(440.98661785,1027.17046705)(440.96661787,1026.96046726)(440.95662087,1026.73047188)
\lineto(440.95662087,1025.98047188)
\lineto(440.95662087,1023.95547188)
\lineto(440.95662087,1023.04047188)
\lineto(440.95662087,1022.77047188)
\curveto(440.95661788,1022.68047154)(440.9416179,1022.60047162)(440.91162087,1022.53047188)
\curveto(440.87161797,1022.44047178)(440.79661804,1022.38547183)(440.68662087,1022.36547188)
\curveto(440.57661826,1022.34547187)(440.45161839,1022.33547188)(440.31162087,1022.33547188)
\lineto(439.53162087,1022.33547188)
\lineto(439.23162087,1022.33547188)
\curveto(439.1416197,1022.34547187)(439.06661977,1022.37047185)(439.00662087,1022.41047188)
\curveto(438.91661992,1022.46047176)(438.86661997,1022.55047167)(438.85662087,1022.68047188)
\lineto(438.85662087,1023.11547188)
\lineto(438.85662087,1024.87047188)
\lineto(438.85662087,1028.53047188)
\lineto(438.85662087,1029.43047188)
\lineto(438.85662087,1029.71547188)
\curveto(438.86661997,1029.80546441)(438.89161995,1029.88046434)(438.93162087,1029.94047188)
\curveto(438.98161986,1030.00046422)(439.06161978,1030.04046418)(439.17162087,1030.06047188)
\lineto(439.26162087,1030.06047188)
\curveto(439.31161953,1030.07046415)(439.36161948,1030.07546414)(439.41162087,1030.07547188)
\lineto(439.57662087,1030.07547188)
\lineto(440.19162087,1030.07547188)
\curveto(440.27161857,1030.07546414)(440.34661849,1030.07046415)(440.41662087,1030.06047188)
\curveto(440.49661834,1030.06046416)(440.56661827,1030.05046417)(440.62662087,1030.03047188)
\curveto(440.70661813,1030.00046422)(440.75661808,1029.95046427)(440.77662087,1029.88047188)
\curveto(440.80661803,1029.81046441)(440.83161801,1029.73046449)(440.85162087,1029.64047188)
\curveto(440.86161798,1029.61046461)(440.86161798,1029.58046464)(440.85162087,1029.55047188)
\curveto(440.85161799,1029.53046469)(440.86161798,1029.51046471)(440.88162087,1029.49047188)
\curveto(440.89161795,1029.46046476)(440.90161794,1029.43546478)(440.91162087,1029.41547188)
\curveto(440.93161791,1029.40546481)(440.95161789,1029.39046483)(440.97162087,1029.37047188)
\curveto(441.09161775,1029.36046486)(441.19161765,1029.39546482)(441.27162087,1029.47547188)
\curveto(441.35161749,1029.56546465)(441.42661741,1029.63546458)(441.49662087,1029.68547188)
\curveto(441.6366172,1029.78546443)(441.77661706,1029.87546434)(441.91662087,1029.95547188)
\curveto(442.06661677,1030.03546418)(442.22661661,1030.10046412)(442.39662087,1030.15047188)
\curveto(442.48661635,1030.18046404)(442.57661626,1030.20046402)(442.66662087,1030.21047188)
\curveto(442.75661608,1030.220464)(442.85161599,1030.23546398)(442.95162087,1030.25547188)
\curveto(442.98161586,1030.26546395)(443.02661581,1030.26546395)(443.08662087,1030.25547188)
\curveto(443.14661569,1030.25546396)(443.18661565,1030.26046396)(443.20662087,1030.27047188)
}
}
{
\newrgbcolor{curcolor}{0 0 0}
\pscustom[linestyle=none,fillstyle=solid,fillcolor=curcolor]
{
\newpath
\moveto(459.71037087,1026.28047188)
\curveto(459.7303627,1026.20046802)(459.7303627,1026.11046811)(459.71037087,1026.01047188)
\curveto(459.69036274,1025.91046831)(459.65536278,1025.84546837)(459.60537087,1025.81547188)
\curveto(459.55536288,1025.77546844)(459.48036295,1025.74546847)(459.38037087,1025.72547188)
\curveto(459.29036314,1025.7154685)(459.18536325,1025.70546851)(459.06537087,1025.69547188)
\lineto(458.72037087,1025.69547188)
\curveto(458.61036382,1025.70546851)(458.51036392,1025.71046851)(458.42037087,1025.71047188)
\lineto(454.76037087,1025.71047188)
\lineto(454.55037087,1025.71047188)
\curveto(454.49036794,1025.71046851)(454.435368,1025.70046852)(454.38537087,1025.68047188)
\curveto(454.30536813,1025.64046858)(454.25536818,1025.60046862)(454.23537087,1025.56047188)
\curveto(454.21536822,1025.54046868)(454.19536824,1025.50046872)(454.17537087,1025.44047188)
\curveto(454.15536828,1025.39046883)(454.15036828,1025.34046888)(454.16037087,1025.29047188)
\curveto(454.18036825,1025.23046899)(454.19036824,1025.17046905)(454.19037087,1025.11047188)
\curveto(454.20036823,1025.06046916)(454.21536822,1025.00546921)(454.23537087,1024.94547188)
\curveto(454.31536812,1024.70546951)(454.41036802,1024.50546971)(454.52037087,1024.34547188)
\curveto(454.64036779,1024.19547002)(454.80036763,1024.06047016)(455.00037087,1023.94047188)
\curveto(455.08036735,1023.89047033)(455.16036727,1023.85547036)(455.24037087,1023.83547188)
\curveto(455.3303671,1023.82547039)(455.42036701,1023.80547041)(455.51037087,1023.77547188)
\curveto(455.59036684,1023.75547046)(455.70036673,1023.74047048)(455.84037087,1023.73047188)
\curveto(455.98036645,1023.7204705)(456.10036633,1023.72547049)(456.20037087,1023.74547188)
\lineto(456.33537087,1023.74547188)
\curveto(456.435366,1023.76547045)(456.52536591,1023.78547043)(456.60537087,1023.80547188)
\curveto(456.69536574,1023.83547038)(456.78036565,1023.86547035)(456.86037087,1023.89547188)
\curveto(456.96036547,1023.94547027)(457.07036536,1024.01047021)(457.19037087,1024.09047188)
\curveto(457.32036511,1024.17047005)(457.41536502,1024.25046997)(457.47537087,1024.33047188)
\curveto(457.52536491,1024.40046982)(457.57536486,1024.46546975)(457.62537087,1024.52547188)
\curveto(457.68536475,1024.59546962)(457.75536468,1024.64546957)(457.83537087,1024.67547188)
\curveto(457.9353645,1024.72546949)(458.06036437,1024.74546947)(458.21037087,1024.73547188)
\lineto(458.64537087,1024.73547188)
\lineto(458.82537087,1024.73547188)
\curveto(458.89536354,1024.74546947)(458.95536348,1024.74046948)(459.00537087,1024.72047188)
\lineto(459.15537087,1024.72047188)
\curveto(459.25536318,1024.70046952)(459.32536311,1024.67546954)(459.36537087,1024.64547188)
\curveto(459.40536303,1024.62546959)(459.42536301,1024.58046964)(459.42537087,1024.51047188)
\curveto(459.435363,1024.44046978)(459.430363,1024.38046984)(459.41037087,1024.33047188)
\curveto(459.36036307,1024.19047003)(459.30536313,1024.06547015)(459.24537087,1023.95547188)
\curveto(459.18536325,1023.84547037)(459.11536332,1023.73547048)(459.03537087,1023.62547188)
\curveto(458.81536362,1023.29547092)(458.56536387,1023.03047119)(458.28537087,1022.83047188)
\curveto(458.00536443,1022.63047159)(457.65536478,1022.46047176)(457.23537087,1022.32047188)
\curveto(457.12536531,1022.28047194)(457.01536542,1022.25547196)(456.90537087,1022.24547188)
\curveto(456.79536564,1022.23547198)(456.68036575,1022.215472)(456.56037087,1022.18547188)
\curveto(456.52036591,1022.17547204)(456.47536596,1022.17547204)(456.42537087,1022.18547188)
\curveto(456.38536605,1022.18547203)(456.34536609,1022.18047204)(456.30537087,1022.17047188)
\lineto(456.14037087,1022.17047188)
\curveto(456.09036634,1022.15047207)(456.0303664,1022.14547207)(455.96037087,1022.15547188)
\curveto(455.90036653,1022.15547206)(455.84536659,1022.16047206)(455.79537087,1022.17047188)
\curveto(455.71536672,1022.18047204)(455.64536679,1022.18047204)(455.58537087,1022.17047188)
\curveto(455.52536691,1022.16047206)(455.46036697,1022.16547205)(455.39037087,1022.18547188)
\curveto(455.34036709,1022.20547201)(455.28536715,1022.215472)(455.22537087,1022.21547188)
\curveto(455.16536727,1022.215472)(455.11036732,1022.22547199)(455.06037087,1022.24547188)
\curveto(454.95036748,1022.26547195)(454.84036759,1022.29047193)(454.73037087,1022.32047188)
\curveto(454.62036781,1022.34047188)(454.52036791,1022.37547184)(454.43037087,1022.42547188)
\curveto(454.32036811,1022.46547175)(454.21536822,1022.50047172)(454.11537087,1022.53047188)
\curveto(454.02536841,1022.57047165)(453.94036849,1022.6154716)(453.86037087,1022.66547188)
\curveto(453.54036889,1022.86547135)(453.25536918,1023.09547112)(453.00537087,1023.35547188)
\curveto(452.75536968,1023.62547059)(452.55036988,1023.93547028)(452.39037087,1024.28547188)
\curveto(452.34037009,1024.39546982)(452.30037013,1024.50546971)(452.27037087,1024.61547188)
\curveto(452.24037019,1024.73546948)(452.20037023,1024.85546936)(452.15037087,1024.97547188)
\curveto(452.14037029,1025.0154692)(452.1353703,1025.05046917)(452.13537087,1025.08047188)
\curveto(452.1353703,1025.1204691)(452.1303703,1025.16046906)(452.12037087,1025.20047188)
\curveto(452.08037035,1025.3204689)(452.05537038,1025.45046877)(452.04537087,1025.59047188)
\lineto(452.01537087,1026.01047188)
\curveto(452.01537042,1026.06046816)(452.01037042,1026.1154681)(452.00037087,1026.17547188)
\curveto(452.00037043,1026.23546798)(452.00537043,1026.29046793)(452.01537087,1026.34047188)
\lineto(452.01537087,1026.52047188)
\lineto(452.06037087,1026.88047188)
\curveto(452.10037033,1027.05046717)(452.1353703,1027.215467)(452.16537087,1027.37547188)
\curveto(452.19537024,1027.53546668)(452.24037019,1027.68546653)(452.30037087,1027.82547188)
\curveto(452.7303697,1028.86546535)(453.46036897,1029.60046462)(454.49037087,1030.03047188)
\curveto(454.6303678,1030.09046413)(454.77036766,1030.13046409)(454.91037087,1030.15047188)
\curveto(455.06036737,1030.18046404)(455.21536722,1030.215464)(455.37537087,1030.25547188)
\curveto(455.45536698,1030.26546395)(455.5303669,1030.27046395)(455.60037087,1030.27047188)
\curveto(455.67036676,1030.27046395)(455.74536669,1030.27546394)(455.82537087,1030.28547188)
\curveto(456.3353661,1030.29546392)(456.77036566,1030.23546398)(457.13037087,1030.10547188)
\curveto(457.50036493,1029.98546423)(457.8303646,1029.82546439)(458.12037087,1029.62547188)
\curveto(458.21036422,1029.56546465)(458.30036413,1029.49546472)(458.39037087,1029.41547188)
\curveto(458.48036395,1029.34546487)(458.56036387,1029.27046495)(458.63037087,1029.19047188)
\curveto(458.66036377,1029.14046508)(458.70036373,1029.10046512)(458.75037087,1029.07047188)
\curveto(458.8303636,1028.96046526)(458.90536353,1028.84546537)(458.97537087,1028.72547188)
\curveto(459.04536339,1028.6154656)(459.12036331,1028.50046572)(459.20037087,1028.38047188)
\curveto(459.25036318,1028.29046593)(459.29036314,1028.19546602)(459.32037087,1028.09547188)
\curveto(459.36036307,1028.00546621)(459.40036303,1027.90546631)(459.44037087,1027.79547188)
\curveto(459.49036294,1027.66546655)(459.5303629,1027.53046669)(459.56037087,1027.39047188)
\curveto(459.59036284,1027.25046697)(459.62536281,1027.11046711)(459.66537087,1026.97047188)
\curveto(459.68536275,1026.89046733)(459.69036274,1026.80046742)(459.68037087,1026.70047188)
\curveto(459.68036275,1026.61046761)(459.69036274,1026.52546769)(459.71037087,1026.44547188)
\lineto(459.71037087,1026.28047188)
\moveto(457.46037087,1027.16547188)
\curveto(457.5303649,1027.26546695)(457.5353649,1027.38546683)(457.47537087,1027.52547188)
\curveto(457.42536501,1027.67546654)(457.38536505,1027.78546643)(457.35537087,1027.85547188)
\curveto(457.21536522,1028.12546609)(457.0303654,1028.33046589)(456.80037087,1028.47047188)
\curveto(456.57036586,1028.6204656)(456.25036618,1028.70046552)(455.84037087,1028.71047188)
\curveto(455.81036662,1028.69046553)(455.77536666,1028.68546553)(455.73537087,1028.69547188)
\curveto(455.69536674,1028.70546551)(455.66036677,1028.70546551)(455.63037087,1028.69547188)
\curveto(455.58036685,1028.67546554)(455.52536691,1028.66046556)(455.46537087,1028.65047188)
\curveto(455.40536703,1028.65046557)(455.35036708,1028.64046558)(455.30037087,1028.62047188)
\curveto(454.86036757,1028.48046574)(454.5353679,1028.20546601)(454.32537087,1027.79547188)
\curveto(454.30536813,1027.75546646)(454.28036815,1027.70046652)(454.25037087,1027.63047188)
\curveto(454.2303682,1027.57046665)(454.21536822,1027.50546671)(454.20537087,1027.43547188)
\curveto(454.19536824,1027.37546684)(454.19536824,1027.3154669)(454.20537087,1027.25547188)
\curveto(454.22536821,1027.19546702)(454.26036817,1027.14546707)(454.31037087,1027.10547188)
\curveto(454.39036804,1027.05546716)(454.50036793,1027.03046719)(454.64037087,1027.03047188)
\lineto(455.04537087,1027.03047188)
\lineto(456.71037087,1027.03047188)
\lineto(457.14537087,1027.03047188)
\curveto(457.30536513,1027.04046718)(457.41036502,1027.08546713)(457.46037087,1027.16547188)
}
}
{
\newrgbcolor{curcolor}{0 0 0}
\pscustom[linestyle=none,fillstyle=solid,fillcolor=curcolor]
{
\newpath
\moveto(465.38365212,1030.27047188)
\curveto(465.98364631,1030.29046393)(466.48364581,1030.20546401)(466.88365212,1030.01547188)
\curveto(467.28364501,1029.82546439)(467.5986447,1029.54546467)(467.82865212,1029.17547188)
\curveto(467.8986444,1029.06546515)(467.95364434,1028.94546527)(467.99365212,1028.81547188)
\curveto(468.03364426,1028.69546552)(468.07364422,1028.57046565)(468.11365212,1028.44047188)
\curveto(468.13364416,1028.36046586)(468.14364415,1028.28546593)(468.14365212,1028.21547188)
\curveto(468.15364414,1028.14546607)(468.16864413,1028.07546614)(468.18865212,1028.00547188)
\curveto(468.18864411,1027.94546627)(468.1936441,1027.90546631)(468.20365212,1027.88547188)
\curveto(468.22364407,1027.74546647)(468.23364406,1027.60046662)(468.23365212,1027.45047188)
\lineto(468.23365212,1027.01547188)
\lineto(468.23365212,1025.68047188)
\lineto(468.23365212,1023.25047188)
\curveto(468.23364406,1023.06047116)(468.22864407,1022.87547134)(468.21865212,1022.69547188)
\curveto(468.21864408,1022.52547169)(468.14864415,1022.4154718)(468.00865212,1022.36547188)
\curveto(467.94864435,1022.34547187)(467.87864442,1022.33547188)(467.79865212,1022.33547188)
\lineto(467.55865212,1022.33547188)
\lineto(466.74865212,1022.33547188)
\curveto(466.62864567,1022.33547188)(466.51864578,1022.34047188)(466.41865212,1022.35047188)
\curveto(466.32864597,1022.37047185)(466.25864604,1022.4154718)(466.20865212,1022.48547188)
\curveto(466.16864613,1022.54547167)(466.14364615,1022.6204716)(466.13365212,1022.71047188)
\lineto(466.13365212,1023.02547188)
\lineto(466.13365212,1024.07547188)
\lineto(466.13365212,1026.31047188)
\curveto(466.13364616,1026.68046754)(466.11864618,1027.0204672)(466.08865212,1027.33047188)
\curveto(466.05864624,1027.65046657)(465.96864633,1027.9204663)(465.81865212,1028.14047188)
\curveto(465.67864662,1028.34046588)(465.47364682,1028.48046574)(465.20365212,1028.56047188)
\curveto(465.15364714,1028.58046564)(465.0986472,1028.59046563)(465.03865212,1028.59047188)
\curveto(464.98864731,1028.59046563)(464.93364736,1028.60046562)(464.87365212,1028.62047188)
\curveto(464.82364747,1028.63046559)(464.75864754,1028.63046559)(464.67865212,1028.62047188)
\curveto(464.60864769,1028.6204656)(464.55364774,1028.6154656)(464.51365212,1028.60547188)
\curveto(464.47364782,1028.59546562)(464.43864786,1028.59046563)(464.40865212,1028.59047188)
\curveto(464.37864792,1028.59046563)(464.34864795,1028.58546563)(464.31865212,1028.57547188)
\curveto(464.08864821,1028.5154657)(463.90364839,1028.43546578)(463.76365212,1028.33547188)
\curveto(463.44364885,1028.10546611)(463.25364904,1027.77046645)(463.19365212,1027.33047188)
\curveto(463.13364916,1026.89046733)(463.10364919,1026.39546782)(463.10365212,1025.84547188)
\lineto(463.10365212,1023.97047188)
\lineto(463.10365212,1023.05547188)
\lineto(463.10365212,1022.78547188)
\curveto(463.10364919,1022.69547152)(463.08864921,1022.6204716)(463.05865212,1022.56047188)
\curveto(463.00864929,1022.45047177)(462.92864937,1022.38547183)(462.81865212,1022.36547188)
\curveto(462.70864959,1022.34547187)(462.57364972,1022.33547188)(462.41365212,1022.33547188)
\lineto(461.66365212,1022.33547188)
\curveto(461.55365074,1022.33547188)(461.44365085,1022.34047188)(461.33365212,1022.35047188)
\curveto(461.22365107,1022.36047186)(461.14365115,1022.39547182)(461.09365212,1022.45547188)
\curveto(461.02365127,1022.54547167)(460.98865131,1022.67547154)(460.98865212,1022.84547188)
\curveto(460.9986513,1023.0154712)(461.00365129,1023.17547104)(461.00365212,1023.32547188)
\lineto(461.00365212,1025.36547188)
\lineto(461.00365212,1028.66547188)
\lineto(461.00365212,1029.43047188)
\lineto(461.00365212,1029.73047188)
\curveto(461.01365128,1029.8204644)(461.04365125,1029.89546432)(461.09365212,1029.95547188)
\curveto(461.11365118,1029.98546423)(461.14365115,1030.00546421)(461.18365212,1030.01547188)
\curveto(461.23365106,1030.03546418)(461.28365101,1030.05046417)(461.33365212,1030.06047188)
\lineto(461.40865212,1030.06047188)
\curveto(461.45865084,1030.07046415)(461.50865079,1030.07546414)(461.55865212,1030.07547188)
\lineto(461.72365212,1030.07547188)
\lineto(462.35365212,1030.07547188)
\curveto(462.43364986,1030.07546414)(462.50864979,1030.07046415)(462.57865212,1030.06047188)
\curveto(462.65864964,1030.06046416)(462.72864957,1030.05046417)(462.78865212,1030.03047188)
\curveto(462.85864944,1030.00046422)(462.90364939,1029.95546426)(462.92365212,1029.89547188)
\curveto(462.95364934,1029.83546438)(462.97864932,1029.76546445)(462.99865212,1029.68547188)
\curveto(463.00864929,1029.64546457)(463.00864929,1029.61046461)(462.99865212,1029.58047188)
\curveto(462.9986493,1029.55046467)(463.00864929,1029.5204647)(463.02865212,1029.49047188)
\curveto(463.04864925,1029.44046478)(463.06364923,1029.41046481)(463.07365212,1029.40047188)
\curveto(463.0936492,1029.39046483)(463.11864918,1029.37546484)(463.14865212,1029.35547188)
\curveto(463.25864904,1029.34546487)(463.34864895,1029.38046484)(463.41865212,1029.46047188)
\curveto(463.48864881,1029.55046467)(463.56364873,1029.6204646)(463.64365212,1029.67047188)
\curveto(463.91364838,1029.87046435)(464.21364808,1030.03046419)(464.54365212,1030.15047188)
\curveto(464.63364766,1030.18046404)(464.72364757,1030.20046402)(464.81365212,1030.21047188)
\curveto(464.91364738,1030.220464)(465.01864728,1030.23546398)(465.12865212,1030.25547188)
\curveto(465.15864714,1030.26546395)(465.20364709,1030.26546395)(465.26365212,1030.25547188)
\curveto(465.32364697,1030.25546396)(465.36364693,1030.26046396)(465.38365212,1030.27047188)
}
}
{
\newrgbcolor{curcolor}{0 0 0}
\pscustom[linestyle=none,fillstyle=solid,fillcolor=curcolor]
{
\newpath
\moveto(477.63490212,1026.52047188)
\curveto(477.65489355,1026.46046776)(477.66489354,1026.37546784)(477.66490212,1026.26547188)
\curveto(477.66489354,1026.15546806)(477.65489355,1026.07046815)(477.63490212,1026.01047188)
\lineto(477.63490212,1025.86047188)
\curveto(477.61489359,1025.78046844)(477.6048936,1025.70046852)(477.60490212,1025.62047188)
\curveto(477.61489359,1025.54046868)(477.60989359,1025.46046876)(477.58990212,1025.38047188)
\curveto(477.56989363,1025.31046891)(477.55489365,1025.24546897)(477.54490212,1025.18547188)
\curveto(477.53489367,1025.12546909)(477.52489368,1025.06046916)(477.51490212,1024.99047188)
\curveto(477.47489373,1024.88046934)(477.43989376,1024.76546945)(477.40990212,1024.64547188)
\curveto(477.37989382,1024.53546968)(477.33989386,1024.43046979)(477.28990212,1024.33047188)
\curveto(477.07989412,1023.85047037)(476.8048944,1023.46047076)(476.46490212,1023.16047188)
\curveto(476.12489508,1022.86047136)(475.71489549,1022.61047161)(475.23490212,1022.41047188)
\curveto(475.11489609,1022.36047186)(474.98989621,1022.32547189)(474.85990212,1022.30547188)
\curveto(474.73989646,1022.27547194)(474.61489659,1022.24547197)(474.48490212,1022.21547188)
\curveto(474.43489677,1022.19547202)(474.37989682,1022.18547203)(474.31990212,1022.18547188)
\curveto(474.25989694,1022.18547203)(474.204897,1022.18047204)(474.15490212,1022.17047188)
\lineto(474.04990212,1022.17047188)
\curveto(474.01989718,1022.16047206)(473.98989721,1022.15547206)(473.95990212,1022.15547188)
\curveto(473.90989729,1022.14547207)(473.82989737,1022.14047208)(473.71990212,1022.14047188)
\curveto(473.60989759,1022.13047209)(473.52489768,1022.13547208)(473.46490212,1022.15547188)
\lineto(473.31490212,1022.15547188)
\curveto(473.26489794,1022.16547205)(473.20989799,1022.17047205)(473.14990212,1022.17047188)
\curveto(473.0998981,1022.16047206)(473.04989815,1022.16547205)(472.99990212,1022.18547188)
\curveto(472.95989824,1022.19547202)(472.91989828,1022.20047202)(472.87990212,1022.20047188)
\curveto(472.84989835,1022.20047202)(472.80989839,1022.20547201)(472.75990212,1022.21547188)
\curveto(472.65989854,1022.24547197)(472.55989864,1022.27047195)(472.45990212,1022.29047188)
\curveto(472.35989884,1022.31047191)(472.26489894,1022.34047188)(472.17490212,1022.38047188)
\curveto(472.05489915,1022.4204718)(471.93989926,1022.46047176)(471.82990212,1022.50047188)
\curveto(471.72989947,1022.54047168)(471.62489958,1022.59047163)(471.51490212,1022.65047188)
\curveto(471.16490004,1022.86047136)(470.86490034,1023.10547111)(470.61490212,1023.38547188)
\curveto(470.36490084,1023.66547055)(470.15490105,1024.00047022)(469.98490212,1024.39047188)
\curveto(469.93490127,1024.48046974)(469.89490131,1024.57546964)(469.86490212,1024.67547188)
\curveto(469.84490136,1024.77546944)(469.81990138,1024.88046934)(469.78990212,1024.99047188)
\curveto(469.76990143,1025.04046918)(469.75990144,1025.08546913)(469.75990212,1025.12547188)
\curveto(469.75990144,1025.16546905)(469.74990145,1025.21046901)(469.72990212,1025.26047188)
\curveto(469.70990149,1025.34046888)(469.6999015,1025.4204688)(469.69990212,1025.50047188)
\curveto(469.6999015,1025.59046863)(469.68990151,1025.67546854)(469.66990212,1025.75547188)
\curveto(469.65990154,1025.80546841)(469.65490155,1025.85046837)(469.65490212,1025.89047188)
\lineto(469.65490212,1026.02547188)
\curveto(469.63490157,1026.08546813)(469.62490158,1026.17046805)(469.62490212,1026.28047188)
\curveto(469.63490157,1026.39046783)(469.64990155,1026.47546774)(469.66990212,1026.53547188)
\lineto(469.66990212,1026.64047188)
\curveto(469.67990152,1026.69046753)(469.67990152,1026.74046748)(469.66990212,1026.79047188)
\curveto(469.66990153,1026.85046737)(469.67990152,1026.90546731)(469.69990212,1026.95547188)
\curveto(469.70990149,1027.00546721)(469.71490149,1027.05046717)(469.71490212,1027.09047188)
\curveto(469.71490149,1027.14046708)(469.72490148,1027.19046703)(469.74490212,1027.24047188)
\curveto(469.78490142,1027.37046685)(469.81990138,1027.49546672)(469.84990212,1027.61547188)
\curveto(469.87990132,1027.74546647)(469.91990128,1027.87046635)(469.96990212,1027.99047188)
\curveto(470.14990105,1028.40046582)(470.36490084,1028.74046548)(470.61490212,1029.01047188)
\curveto(470.86490034,1029.29046493)(471.16990003,1029.54546467)(471.52990212,1029.77547188)
\curveto(471.62989957,1029.82546439)(471.73489947,1029.87046435)(471.84490212,1029.91047188)
\curveto(471.95489925,1029.95046427)(472.06489914,1029.99546422)(472.17490212,1030.04547188)
\curveto(472.3048989,1030.09546412)(472.43989876,1030.13046409)(472.57990212,1030.15047188)
\curveto(472.71989848,1030.17046405)(472.86489834,1030.20046402)(473.01490212,1030.24047188)
\curveto(473.09489811,1030.25046397)(473.16989803,1030.25546396)(473.23990212,1030.25547188)
\curveto(473.30989789,1030.25546396)(473.37989782,1030.26046396)(473.44990212,1030.27047188)
\curveto(474.02989717,1030.28046394)(474.52989667,1030.220464)(474.94990212,1030.09047188)
\curveto(475.37989582,1029.96046426)(475.75989544,1029.78046444)(476.08990212,1029.55047188)
\curveto(476.199895,1029.47046475)(476.30989489,1029.38046484)(476.41990212,1029.28047188)
\curveto(476.53989466,1029.19046503)(476.63989456,1029.09046513)(476.71990212,1028.98047188)
\curveto(476.7998944,1028.88046534)(476.86989433,1028.78046544)(476.92990212,1028.68047188)
\curveto(476.9998942,1028.58046564)(477.06989413,1028.47546574)(477.13990212,1028.36547188)
\curveto(477.20989399,1028.25546596)(477.26489394,1028.13546608)(477.30490212,1028.00547188)
\curveto(477.34489386,1027.88546633)(477.38989381,1027.75546646)(477.43990212,1027.61547188)
\curveto(477.46989373,1027.53546668)(477.49489371,1027.45046677)(477.51490212,1027.36047188)
\lineto(477.57490212,1027.09047188)
\curveto(477.58489362,1027.05046717)(477.58989361,1027.01046721)(477.58990212,1026.97047188)
\curveto(477.58989361,1026.93046729)(477.59489361,1026.89046733)(477.60490212,1026.85047188)
\curveto(477.62489358,1026.80046742)(477.62989357,1026.74546747)(477.61990212,1026.68547188)
\curveto(477.60989359,1026.62546759)(477.61489359,1026.57046765)(477.63490212,1026.52047188)
\moveto(475.53490212,1025.98047188)
\curveto(475.54489566,1026.03046819)(475.54989565,1026.10046812)(475.54990212,1026.19047188)
\curveto(475.54989565,1026.29046793)(475.54489566,1026.36546785)(475.53490212,1026.41547188)
\lineto(475.53490212,1026.53547188)
\curveto(475.51489569,1026.58546763)(475.5048957,1026.64046758)(475.50490212,1026.70047188)
\curveto(475.5048957,1026.76046746)(475.4998957,1026.8154674)(475.48990212,1026.86547188)
\curveto(475.48989571,1026.90546731)(475.48489572,1026.93546728)(475.47490212,1026.95547188)
\lineto(475.41490212,1027.19547188)
\curveto(475.4048958,1027.28546693)(475.38489582,1027.37046685)(475.35490212,1027.45047188)
\curveto(475.24489596,1027.71046651)(475.11489609,1027.93046629)(474.96490212,1028.11047188)
\curveto(474.81489639,1028.30046592)(474.61489659,1028.45046577)(474.36490212,1028.56047188)
\curveto(474.3048969,1028.58046564)(474.24489696,1028.59546562)(474.18490212,1028.60547188)
\curveto(474.12489708,1028.62546559)(474.05989714,1028.64546557)(473.98990212,1028.66547188)
\curveto(473.90989729,1028.68546553)(473.82489738,1028.69046553)(473.73490212,1028.68047188)
\lineto(473.46490212,1028.68047188)
\curveto(473.43489777,1028.66046556)(473.3998978,1028.65046557)(473.35990212,1028.65047188)
\curveto(473.31989788,1028.66046556)(473.28489792,1028.66046556)(473.25490212,1028.65047188)
\lineto(473.04490212,1028.59047188)
\curveto(472.98489822,1028.58046564)(472.92989827,1028.56046566)(472.87990212,1028.53047188)
\curveto(472.62989857,1028.4204658)(472.42489878,1028.26046596)(472.26490212,1028.05047188)
\curveto(472.11489909,1027.85046637)(471.99489921,1027.6154666)(471.90490212,1027.34547188)
\curveto(471.87489933,1027.24546697)(471.84989935,1027.14046708)(471.82990212,1027.03047188)
\curveto(471.81989938,1026.9204673)(471.8048994,1026.81046741)(471.78490212,1026.70047188)
\curveto(471.77489943,1026.65046757)(471.76989943,1026.60046762)(471.76990212,1026.55047188)
\lineto(471.76990212,1026.40047188)
\curveto(471.74989945,1026.33046789)(471.73989946,1026.22546799)(471.73990212,1026.08547188)
\curveto(471.74989945,1025.94546827)(471.76489944,1025.84046838)(471.78490212,1025.77047188)
\lineto(471.78490212,1025.63547188)
\curveto(471.8048994,1025.55546866)(471.81989938,1025.47546874)(471.82990212,1025.39547188)
\curveto(471.83989936,1025.32546889)(471.85489935,1025.25046897)(471.87490212,1025.17047188)
\curveto(471.97489923,1024.87046935)(472.07989912,1024.62546959)(472.18990212,1024.43547188)
\curveto(472.30989889,1024.25546996)(472.49489871,1024.09047013)(472.74490212,1023.94047188)
\curveto(472.81489839,1023.89047033)(472.88989831,1023.85047037)(472.96990212,1023.82047188)
\curveto(473.05989814,1023.79047043)(473.14989805,1023.76547045)(473.23990212,1023.74547188)
\curveto(473.27989792,1023.73547048)(473.31489789,1023.73047049)(473.34490212,1023.73047188)
\curveto(473.37489783,1023.74047048)(473.40989779,1023.74047048)(473.44990212,1023.73047188)
\lineto(473.56990212,1023.70047188)
\curveto(473.61989758,1023.70047052)(473.66489754,1023.70547051)(473.70490212,1023.71547188)
\lineto(473.82490212,1023.71547188)
\curveto(473.9048973,1023.73547048)(473.98489722,1023.75047047)(474.06490212,1023.76047188)
\curveto(474.14489706,1023.77047045)(474.21989698,1023.79047043)(474.28990212,1023.82047188)
\curveto(474.54989665,1023.9204703)(474.75989644,1024.05547016)(474.91990212,1024.22547188)
\curveto(475.07989612,1024.39546982)(475.21489599,1024.60546961)(475.32490212,1024.85547188)
\curveto(475.36489584,1024.95546926)(475.39489581,1025.05546916)(475.41490212,1025.15547188)
\curveto(475.43489577,1025.25546896)(475.45989574,1025.36046886)(475.48990212,1025.47047188)
\curveto(475.4998957,1025.51046871)(475.5048957,1025.54546867)(475.50490212,1025.57547188)
\curveto(475.5048957,1025.6154686)(475.50989569,1025.65546856)(475.51990212,1025.69547188)
\lineto(475.51990212,1025.83047188)
\curveto(475.51989568,1025.88046834)(475.52489568,1025.93046829)(475.53490212,1025.98047188)
}
}
{
\newrgbcolor{curcolor}{0 0 0}
\pscustom[linestyle=none,fillstyle=solid,fillcolor=curcolor]
{
\newpath
\moveto(482.00482399,1030.28547188)
\curveto(482.75481949,1030.30546391)(483.40481884,1030.220464)(483.95482399,1030.03047188)
\curveto(484.51481773,1029.85046437)(484.93981731,1029.53546468)(485.22982399,1029.08547188)
\curveto(485.29981695,1028.97546524)(485.35981689,1028.86046536)(485.40982399,1028.74047188)
\curveto(485.46981678,1028.63046559)(485.51981673,1028.50546571)(485.55982399,1028.36547188)
\curveto(485.57981667,1028.30546591)(485.58981666,1028.24046598)(485.58982399,1028.17047188)
\curveto(485.58981666,1028.10046612)(485.57981667,1028.04046618)(485.55982399,1027.99047188)
\curveto(485.51981673,1027.93046629)(485.46481678,1027.89046633)(485.39482399,1027.87047188)
\curveto(485.3448169,1027.85046637)(485.28481696,1027.84046638)(485.21482399,1027.84047188)
\lineto(485.00482399,1027.84047188)
\lineto(484.34482399,1027.84047188)
\curveto(484.27481797,1027.84046638)(484.20481804,1027.83546638)(484.13482399,1027.82547188)
\curveto(484.06481818,1027.82546639)(483.99981825,1027.83546638)(483.93982399,1027.85547188)
\curveto(483.83981841,1027.87546634)(483.76481848,1027.9154663)(483.71482399,1027.97547188)
\curveto(483.66481858,1028.03546618)(483.61981863,1028.09546612)(483.57982399,1028.15547188)
\lineto(483.45982399,1028.36547188)
\curveto(483.42981882,1028.44546577)(483.37981887,1028.51046571)(483.30982399,1028.56047188)
\curveto(483.20981904,1028.64046558)(483.10981914,1028.70046552)(483.00982399,1028.74047188)
\curveto(482.91981933,1028.78046544)(482.80481944,1028.8154654)(482.66482399,1028.84547188)
\curveto(482.59481965,1028.86546535)(482.48981976,1028.88046534)(482.34982399,1028.89047188)
\curveto(482.21982003,1028.90046532)(482.11982013,1028.89546532)(482.04982399,1028.87547188)
\lineto(481.94482399,1028.87547188)
\lineto(481.79482399,1028.84547188)
\curveto(481.75482049,1028.84546537)(481.70982054,1028.84046538)(481.65982399,1028.83047188)
\curveto(481.48982076,1028.78046544)(481.3498209,1028.71046551)(481.23982399,1028.62047188)
\curveto(481.13982111,1028.54046568)(481.06982118,1028.4154658)(481.02982399,1028.24547188)
\curveto(481.00982124,1028.17546604)(481.00982124,1028.11046611)(481.02982399,1028.05047188)
\curveto(481.0498212,1027.99046623)(481.06982118,1027.94046628)(481.08982399,1027.90047188)
\curveto(481.15982109,1027.78046644)(481.23982101,1027.68546653)(481.32982399,1027.61547188)
\curveto(481.42982082,1027.54546667)(481.5448207,1027.48546673)(481.67482399,1027.43547188)
\curveto(481.86482038,1027.35546686)(482.06982018,1027.28546693)(482.28982399,1027.22547188)
\lineto(482.97982399,1027.07547188)
\curveto(483.21981903,1027.03546718)(483.4498188,1026.98546723)(483.66982399,1026.92547188)
\curveto(483.89981835,1026.87546734)(484.11481813,1026.81046741)(484.31482399,1026.73047188)
\curveto(484.40481784,1026.69046753)(484.48981776,1026.65546756)(484.56982399,1026.62547188)
\curveto(484.65981759,1026.60546761)(484.7448175,1026.57046765)(484.82482399,1026.52047188)
\curveto(485.01481723,1026.40046782)(485.18481706,1026.27046795)(485.33482399,1026.13047188)
\curveto(485.49481675,1025.99046823)(485.61981663,1025.8154684)(485.70982399,1025.60547188)
\curveto(485.73981651,1025.53546868)(485.76481648,1025.46546875)(485.78482399,1025.39547188)
\curveto(485.80481644,1025.32546889)(485.82481642,1025.25046897)(485.84482399,1025.17047188)
\curveto(485.85481639,1025.11046911)(485.85981639,1025.0154692)(485.85982399,1024.88547188)
\curveto(485.86981638,1024.76546945)(485.86981638,1024.67046955)(485.85982399,1024.60047188)
\lineto(485.85982399,1024.52547188)
\curveto(485.83981641,1024.46546975)(485.82481642,1024.40546981)(485.81482399,1024.34547188)
\curveto(485.81481643,1024.29546992)(485.80981644,1024.24546997)(485.79982399,1024.19547188)
\curveto(485.72981652,1023.89547032)(485.61981663,1023.63047059)(485.46982399,1023.40047188)
\curveto(485.30981694,1023.16047106)(485.11481713,1022.96547125)(484.88482399,1022.81547188)
\curveto(484.65481759,1022.66547155)(484.39481785,1022.53547168)(484.10482399,1022.42547188)
\curveto(483.99481825,1022.37547184)(483.87481837,1022.34047188)(483.74482399,1022.32047188)
\curveto(483.62481862,1022.30047192)(483.50481874,1022.27547194)(483.38482399,1022.24547188)
\curveto(483.29481895,1022.22547199)(483.19981905,1022.215472)(483.09982399,1022.21547188)
\curveto(483.00981924,1022.20547201)(482.91981933,1022.19047203)(482.82982399,1022.17047188)
\lineto(482.55982399,1022.17047188)
\curveto(482.49981975,1022.15047207)(482.39481985,1022.14047208)(482.24482399,1022.14047188)
\curveto(482.10482014,1022.14047208)(482.00482024,1022.15047207)(481.94482399,1022.17047188)
\curveto(481.91482033,1022.17047205)(481.87982037,1022.17547204)(481.83982399,1022.18547188)
\lineto(481.73482399,1022.18547188)
\curveto(481.61482063,1022.20547201)(481.49482075,1022.220472)(481.37482399,1022.23047188)
\curveto(481.25482099,1022.24047198)(481.13982111,1022.26047196)(481.02982399,1022.29047188)
\curveto(480.63982161,1022.40047182)(480.29482195,1022.52547169)(479.99482399,1022.66547188)
\curveto(479.69482255,1022.8154714)(479.43982281,1023.03547118)(479.22982399,1023.32547188)
\curveto(479.08982316,1023.5154707)(478.96982328,1023.73547048)(478.86982399,1023.98547188)
\curveto(478.8498234,1024.04547017)(478.82982342,1024.12547009)(478.80982399,1024.22547188)
\curveto(478.78982346,1024.27546994)(478.77482347,1024.34546987)(478.76482399,1024.43547188)
\curveto(478.75482349,1024.52546969)(478.75982349,1024.60046962)(478.77982399,1024.66047188)
\curveto(478.80982344,1024.73046949)(478.85982339,1024.78046944)(478.92982399,1024.81047188)
\curveto(478.97982327,1024.83046939)(479.03982321,1024.84046938)(479.10982399,1024.84047188)
\lineto(479.33482399,1024.84047188)
\lineto(480.03982399,1024.84047188)
\lineto(480.27982399,1024.84047188)
\curveto(480.35982189,1024.84046938)(480.42982182,1024.83046939)(480.48982399,1024.81047188)
\curveto(480.59982165,1024.77046945)(480.66982158,1024.70546951)(480.69982399,1024.61547188)
\curveto(480.73982151,1024.52546969)(480.78482146,1024.43046979)(480.83482399,1024.33047188)
\curveto(480.85482139,1024.28046994)(480.88982136,1024.21547)(480.93982399,1024.13547188)
\curveto(480.99982125,1024.05547016)(481.0498212,1024.00547021)(481.08982399,1023.98547188)
\curveto(481.20982104,1023.88547033)(481.32482092,1023.80547041)(481.43482399,1023.74547188)
\curveto(481.5448207,1023.69547052)(481.68482056,1023.64547057)(481.85482399,1023.59547188)
\curveto(481.90482034,1023.57547064)(481.95482029,1023.56547065)(482.00482399,1023.56547188)
\curveto(482.05482019,1023.57547064)(482.10482014,1023.57547064)(482.15482399,1023.56547188)
\curveto(482.23482001,1023.54547067)(482.31981993,1023.53547068)(482.40982399,1023.53547188)
\curveto(482.50981974,1023.54547067)(482.59481965,1023.56047066)(482.66482399,1023.58047188)
\curveto(482.71481953,1023.59047063)(482.75981949,1023.59547062)(482.79982399,1023.59547188)
\curveto(482.8498194,1023.59547062)(482.89981935,1023.60547061)(482.94982399,1023.62547188)
\curveto(483.08981916,1023.67547054)(483.21481903,1023.73547048)(483.32482399,1023.80547188)
\curveto(483.4448188,1023.87547034)(483.53981871,1023.96547025)(483.60982399,1024.07547188)
\curveto(483.65981859,1024.15547006)(483.69981855,1024.28046994)(483.72982399,1024.45047188)
\curveto(483.7498185,1024.5204697)(483.7498185,1024.58546963)(483.72982399,1024.64547188)
\curveto(483.70981854,1024.70546951)(483.68981856,1024.75546946)(483.66982399,1024.79547188)
\curveto(483.59981865,1024.93546928)(483.50981874,1025.04046918)(483.39982399,1025.11047188)
\curveto(483.29981895,1025.18046904)(483.17981907,1025.24546897)(483.03982399,1025.30547188)
\curveto(482.8498194,1025.38546883)(482.6498196,1025.45046877)(482.43982399,1025.50047188)
\curveto(482.22982002,1025.55046867)(482.01982023,1025.60546861)(481.80982399,1025.66547188)
\curveto(481.72982052,1025.68546853)(481.6448206,1025.70046852)(481.55482399,1025.71047188)
\curveto(481.47482077,1025.7204685)(481.39482085,1025.73546848)(481.31482399,1025.75547188)
\curveto(480.99482125,1025.84546837)(480.68982156,1025.93046829)(480.39982399,1026.01047188)
\curveto(480.10982214,1026.10046812)(479.8448224,1026.23046799)(479.60482399,1026.40047188)
\curveto(479.32482292,1026.60046762)(479.11982313,1026.87046735)(478.98982399,1027.21047188)
\curveto(478.96982328,1027.28046694)(478.9498233,1027.37546684)(478.92982399,1027.49547188)
\curveto(478.90982334,1027.56546665)(478.89482335,1027.65046657)(478.88482399,1027.75047188)
\curveto(478.87482337,1027.85046637)(478.87982337,1027.94046628)(478.89982399,1028.02047188)
\curveto(478.91982333,1028.07046615)(478.92482332,1028.11046611)(478.91482399,1028.14047188)
\curveto(478.90482334,1028.18046604)(478.90982334,1028.22546599)(478.92982399,1028.27547188)
\curveto(478.9498233,1028.38546583)(478.96982328,1028.48546573)(478.98982399,1028.57547188)
\curveto(479.01982323,1028.67546554)(479.05482319,1028.77046545)(479.09482399,1028.86047188)
\curveto(479.22482302,1029.15046507)(479.40482284,1029.38546483)(479.63482399,1029.56547188)
\curveto(479.86482238,1029.74546447)(480.12482212,1029.89046433)(480.41482399,1030.00047188)
\curveto(480.52482172,1030.05046417)(480.63982161,1030.08546413)(480.75982399,1030.10547188)
\curveto(480.87982137,1030.13546408)(481.00482124,1030.16546405)(481.13482399,1030.19547188)
\curveto(481.19482105,1030.215464)(481.25482099,1030.22546399)(481.31482399,1030.22547188)
\lineto(481.49482399,1030.25547188)
\curveto(481.57482067,1030.26546395)(481.65982059,1030.27046395)(481.74982399,1030.27047188)
\curveto(481.83982041,1030.27046395)(481.92482032,1030.27546394)(482.00482399,1030.28547188)
}
}
{
\newrgbcolor{curcolor}{0 0 0}
\pscustom[linestyle=none,fillstyle=solid,fillcolor=curcolor]
{
}
}
{
\newrgbcolor{curcolor}{0 0 0}
\pscustom[linestyle=none,fillstyle=solid,fillcolor=curcolor]
{
\newpath
\moveto(491.67162087,1030.06047188)
\lineto(492.79662087,1030.06047188)
\curveto(492.90661843,1030.06046416)(493.00661833,1030.05546416)(493.09662087,1030.04547188)
\curveto(493.18661815,1030.03546418)(493.25161809,1030.00046422)(493.29162087,1029.94047188)
\curveto(493.341618,1029.88046434)(493.37161797,1029.79546442)(493.38162087,1029.68547188)
\curveto(493.39161795,1029.58546463)(493.39661794,1029.48046474)(493.39662087,1029.37047188)
\lineto(493.39662087,1028.32047188)
\lineto(493.39662087,1026.08547188)
\curveto(493.39661794,1025.72546849)(493.41161793,1025.38546883)(493.44162087,1025.06547188)
\curveto(493.47161787,1024.74546947)(493.56161778,1024.48046974)(493.71162087,1024.27047188)
\curveto(493.85161749,1024.06047016)(494.07661726,1023.91047031)(494.38662087,1023.82047188)
\curveto(494.4366169,1023.81047041)(494.47661686,1023.80547041)(494.50662087,1023.80547188)
\curveto(494.54661679,1023.80547041)(494.59161675,1023.80047042)(494.64162087,1023.79047188)
\curveto(494.69161665,1023.78047044)(494.74661659,1023.77547044)(494.80662087,1023.77547188)
\curveto(494.86661647,1023.77547044)(494.91161643,1023.78047044)(494.94162087,1023.79047188)
\curveto(494.99161635,1023.81047041)(495.03161631,1023.8154704)(495.06162087,1023.80547188)
\curveto(495.10161624,1023.79547042)(495.1416162,1023.80047042)(495.18162087,1023.82047188)
\curveto(495.39161595,1023.87047035)(495.55661578,1023.93547028)(495.67662087,1024.01547188)
\curveto(495.85661548,1024.12547009)(495.99661534,1024.26546995)(496.09662087,1024.43547188)
\curveto(496.20661513,1024.6154696)(496.28161506,1024.81046941)(496.32162087,1025.02047188)
\curveto(496.37161497,1025.24046898)(496.40161494,1025.48046874)(496.41162087,1025.74047188)
\curveto(496.42161492,1026.01046821)(496.42661491,1026.29046793)(496.42662087,1026.58047188)
\lineto(496.42662087,1028.39547188)
\lineto(496.42662087,1029.37047188)
\lineto(496.42662087,1029.64047188)
\curveto(496.42661491,1029.74046448)(496.44661489,1029.8204644)(496.48662087,1029.88047188)
\curveto(496.5366148,1029.97046425)(496.61161473,1030.0204642)(496.71162087,1030.03047188)
\curveto(496.81161453,1030.05046417)(496.93161441,1030.06046416)(497.07162087,1030.06047188)
\lineto(497.86662087,1030.06047188)
\lineto(498.15162087,1030.06047188)
\curveto(498.2416131,1030.06046416)(498.31661302,1030.04046418)(498.37662087,1030.00047188)
\curveto(498.45661288,1029.95046427)(498.50161284,1029.87546434)(498.51162087,1029.77547188)
\curveto(498.52161282,1029.67546454)(498.52661281,1029.56046466)(498.52662087,1029.43047188)
\lineto(498.52662087,1028.29047188)
\lineto(498.52662087,1024.07547188)
\lineto(498.52662087,1023.01047188)
\lineto(498.52662087,1022.71047188)
\curveto(498.52661281,1022.61047161)(498.50661283,1022.53547168)(498.46662087,1022.48547188)
\curveto(498.41661292,1022.40547181)(498.341613,1022.36047186)(498.24162087,1022.35047188)
\curveto(498.1416132,1022.34047188)(498.0366133,1022.33547188)(497.92662087,1022.33547188)
\lineto(497.11662087,1022.33547188)
\curveto(497.00661433,1022.33547188)(496.90661443,1022.34047188)(496.81662087,1022.35047188)
\curveto(496.7366146,1022.36047186)(496.67161467,1022.40047182)(496.62162087,1022.47047188)
\curveto(496.60161474,1022.50047172)(496.58161476,1022.54547167)(496.56162087,1022.60547188)
\curveto(496.55161479,1022.66547155)(496.5366148,1022.72547149)(496.51662087,1022.78547188)
\curveto(496.50661483,1022.84547137)(496.49161485,1022.90047132)(496.47162087,1022.95047188)
\curveto(496.45161489,1023.00047122)(496.42161492,1023.03047119)(496.38162087,1023.04047188)
\curveto(496.36161498,1023.06047116)(496.336615,1023.06547115)(496.30662087,1023.05547188)
\curveto(496.27661506,1023.04547117)(496.25161509,1023.03547118)(496.23162087,1023.02547188)
\curveto(496.16161518,1022.98547123)(496.10161524,1022.94047128)(496.05162087,1022.89047188)
\curveto(496.00161534,1022.84047138)(495.94661539,1022.79547142)(495.88662087,1022.75547188)
\curveto(495.84661549,1022.72547149)(495.80661553,1022.69047153)(495.76662087,1022.65047188)
\curveto(495.7366156,1022.6204716)(495.69661564,1022.59047163)(495.64662087,1022.56047188)
\curveto(495.41661592,1022.4204718)(495.14661619,1022.31047191)(494.83662087,1022.23047188)
\curveto(494.76661657,1022.21047201)(494.69661664,1022.20047202)(494.62662087,1022.20047188)
\curveto(494.55661678,1022.19047203)(494.48161686,1022.17547204)(494.40162087,1022.15547188)
\curveto(494.36161698,1022.14547207)(494.31661702,1022.14547207)(494.26662087,1022.15547188)
\curveto(494.22661711,1022.15547206)(494.18661715,1022.15047207)(494.14662087,1022.14047188)
\curveto(494.11661722,1022.13047209)(494.05161729,1022.13047209)(493.95162087,1022.14047188)
\curveto(493.86161748,1022.14047208)(493.80161754,1022.14547207)(493.77162087,1022.15547188)
\curveto(493.72161762,1022.15547206)(493.67161767,1022.16047206)(493.62162087,1022.17047188)
\lineto(493.47162087,1022.17047188)
\curveto(493.35161799,1022.20047202)(493.2366181,1022.22547199)(493.12662087,1022.24547188)
\curveto(493.01661832,1022.26547195)(492.90661843,1022.29547192)(492.79662087,1022.33547188)
\curveto(492.74661859,1022.35547186)(492.70161864,1022.37047185)(492.66162087,1022.38047188)
\curveto(492.63161871,1022.40047182)(492.59161875,1022.4204718)(492.54162087,1022.44047188)
\curveto(492.19161915,1022.63047159)(491.91161943,1022.89547132)(491.70162087,1023.23547188)
\curveto(491.57161977,1023.44547077)(491.47661986,1023.69547052)(491.41662087,1023.98547188)
\curveto(491.35661998,1024.28546993)(491.31662002,1024.60046962)(491.29662087,1024.93047188)
\curveto(491.28662005,1025.27046895)(491.28162006,1025.6154686)(491.28162087,1025.96547188)
\curveto(491.29162005,1026.32546789)(491.29662004,1026.68046754)(491.29662087,1027.03047188)
\lineto(491.29662087,1029.07047188)
\curveto(491.29662004,1029.20046502)(491.29162005,1029.35046487)(491.28162087,1029.52047188)
\curveto(491.28162006,1029.70046452)(491.30662003,1029.83046439)(491.35662087,1029.91047188)
\curveto(491.38661995,1029.96046426)(491.44661989,1030.00546421)(491.53662087,1030.04547188)
\curveto(491.59661974,1030.04546417)(491.6416197,1030.05046417)(491.67162087,1030.06047188)
}
}
{
\newrgbcolor{curcolor}{0 0 0}
\pscustom[linestyle=none,fillstyle=solid,fillcolor=curcolor]
{
\newpath
\moveto(504.58287087,1030.27047188)
\curveto(505.18286506,1030.29046393)(505.68286456,1030.20546401)(506.08287087,1030.01547188)
\curveto(506.48286376,1029.82546439)(506.79786345,1029.54546467)(507.02787087,1029.17547188)
\curveto(507.09786315,1029.06546515)(507.15286309,1028.94546527)(507.19287087,1028.81547188)
\curveto(507.23286301,1028.69546552)(507.27286297,1028.57046565)(507.31287087,1028.44047188)
\curveto(507.33286291,1028.36046586)(507.3428629,1028.28546593)(507.34287087,1028.21547188)
\curveto(507.35286289,1028.14546607)(507.36786288,1028.07546614)(507.38787087,1028.00547188)
\curveto(507.38786286,1027.94546627)(507.39286285,1027.90546631)(507.40287087,1027.88547188)
\curveto(507.42286282,1027.74546647)(507.43286281,1027.60046662)(507.43287087,1027.45047188)
\lineto(507.43287087,1027.01547188)
\lineto(507.43287087,1025.68047188)
\lineto(507.43287087,1023.25047188)
\curveto(507.43286281,1023.06047116)(507.42786282,1022.87547134)(507.41787087,1022.69547188)
\curveto(507.41786283,1022.52547169)(507.3478629,1022.4154718)(507.20787087,1022.36547188)
\curveto(507.1478631,1022.34547187)(507.07786317,1022.33547188)(506.99787087,1022.33547188)
\lineto(506.75787087,1022.33547188)
\lineto(505.94787087,1022.33547188)
\curveto(505.82786442,1022.33547188)(505.71786453,1022.34047188)(505.61787087,1022.35047188)
\curveto(505.52786472,1022.37047185)(505.45786479,1022.4154718)(505.40787087,1022.48547188)
\curveto(505.36786488,1022.54547167)(505.3428649,1022.6204716)(505.33287087,1022.71047188)
\lineto(505.33287087,1023.02547188)
\lineto(505.33287087,1024.07547188)
\lineto(505.33287087,1026.31047188)
\curveto(505.33286491,1026.68046754)(505.31786493,1027.0204672)(505.28787087,1027.33047188)
\curveto(505.25786499,1027.65046657)(505.16786508,1027.9204663)(505.01787087,1028.14047188)
\curveto(504.87786537,1028.34046588)(504.67286557,1028.48046574)(504.40287087,1028.56047188)
\curveto(504.35286589,1028.58046564)(504.29786595,1028.59046563)(504.23787087,1028.59047188)
\curveto(504.18786606,1028.59046563)(504.13286611,1028.60046562)(504.07287087,1028.62047188)
\curveto(504.02286622,1028.63046559)(503.95786629,1028.63046559)(503.87787087,1028.62047188)
\curveto(503.80786644,1028.6204656)(503.75286649,1028.6154656)(503.71287087,1028.60547188)
\curveto(503.67286657,1028.59546562)(503.63786661,1028.59046563)(503.60787087,1028.59047188)
\curveto(503.57786667,1028.59046563)(503.5478667,1028.58546563)(503.51787087,1028.57547188)
\curveto(503.28786696,1028.5154657)(503.10286714,1028.43546578)(502.96287087,1028.33547188)
\curveto(502.6428676,1028.10546611)(502.45286779,1027.77046645)(502.39287087,1027.33047188)
\curveto(502.33286791,1026.89046733)(502.30286794,1026.39546782)(502.30287087,1025.84547188)
\lineto(502.30287087,1023.97047188)
\lineto(502.30287087,1023.05547188)
\lineto(502.30287087,1022.78547188)
\curveto(502.30286794,1022.69547152)(502.28786796,1022.6204716)(502.25787087,1022.56047188)
\curveto(502.20786804,1022.45047177)(502.12786812,1022.38547183)(502.01787087,1022.36547188)
\curveto(501.90786834,1022.34547187)(501.77286847,1022.33547188)(501.61287087,1022.33547188)
\lineto(500.86287087,1022.33547188)
\curveto(500.75286949,1022.33547188)(500.6428696,1022.34047188)(500.53287087,1022.35047188)
\curveto(500.42286982,1022.36047186)(500.3428699,1022.39547182)(500.29287087,1022.45547188)
\curveto(500.22287002,1022.54547167)(500.18787006,1022.67547154)(500.18787087,1022.84547188)
\curveto(500.19787005,1023.0154712)(500.20287004,1023.17547104)(500.20287087,1023.32547188)
\lineto(500.20287087,1025.36547188)
\lineto(500.20287087,1028.66547188)
\lineto(500.20287087,1029.43047188)
\lineto(500.20287087,1029.73047188)
\curveto(500.21287003,1029.8204644)(500.24287,1029.89546432)(500.29287087,1029.95547188)
\curveto(500.31286993,1029.98546423)(500.3428699,1030.00546421)(500.38287087,1030.01547188)
\curveto(500.43286981,1030.03546418)(500.48286976,1030.05046417)(500.53287087,1030.06047188)
\lineto(500.60787087,1030.06047188)
\curveto(500.65786959,1030.07046415)(500.70786954,1030.07546414)(500.75787087,1030.07547188)
\lineto(500.92287087,1030.07547188)
\lineto(501.55287087,1030.07547188)
\curveto(501.63286861,1030.07546414)(501.70786854,1030.07046415)(501.77787087,1030.06047188)
\curveto(501.85786839,1030.06046416)(501.92786832,1030.05046417)(501.98787087,1030.03047188)
\curveto(502.05786819,1030.00046422)(502.10286814,1029.95546426)(502.12287087,1029.89547188)
\curveto(502.15286809,1029.83546438)(502.17786807,1029.76546445)(502.19787087,1029.68547188)
\curveto(502.20786804,1029.64546457)(502.20786804,1029.61046461)(502.19787087,1029.58047188)
\curveto(502.19786805,1029.55046467)(502.20786804,1029.5204647)(502.22787087,1029.49047188)
\curveto(502.247868,1029.44046478)(502.26286798,1029.41046481)(502.27287087,1029.40047188)
\curveto(502.29286795,1029.39046483)(502.31786793,1029.37546484)(502.34787087,1029.35547188)
\curveto(502.45786779,1029.34546487)(502.5478677,1029.38046484)(502.61787087,1029.46047188)
\curveto(502.68786756,1029.55046467)(502.76286748,1029.6204646)(502.84287087,1029.67047188)
\curveto(503.11286713,1029.87046435)(503.41286683,1030.03046419)(503.74287087,1030.15047188)
\curveto(503.83286641,1030.18046404)(503.92286632,1030.20046402)(504.01287087,1030.21047188)
\curveto(504.11286613,1030.220464)(504.21786603,1030.23546398)(504.32787087,1030.25547188)
\curveto(504.35786589,1030.26546395)(504.40286584,1030.26546395)(504.46287087,1030.25547188)
\curveto(504.52286572,1030.25546396)(504.56286568,1030.26046396)(504.58287087,1030.27047188)
}
}
{
\newrgbcolor{curcolor}{0 0 0}
\pscustom[linestyle=none,fillstyle=solid,fillcolor=curcolor]
{
\newpath
\moveto(516.11412087,1022.93547188)
\curveto(516.13411302,1022.82547139)(516.14411301,1022.7154715)(516.14412087,1022.60547188)
\curveto(516.154113,1022.49547172)(516.10411305,1022.4204718)(515.99412087,1022.38047188)
\curveto(515.93411322,1022.35047187)(515.86411329,1022.33547188)(515.78412087,1022.33547188)
\lineto(515.54412087,1022.33547188)
\lineto(514.73412087,1022.33547188)
\lineto(514.46412087,1022.33547188)
\curveto(514.38411477,1022.34547187)(514.31911483,1022.37047185)(514.26912087,1022.41047188)
\curveto(514.19911495,1022.45047177)(514.14411501,1022.50547171)(514.10412087,1022.57547188)
\curveto(514.07411508,1022.65547156)(514.02911512,1022.7204715)(513.96912087,1022.77047188)
\curveto(513.9491152,1022.79047143)(513.92411523,1022.80547141)(513.89412087,1022.81547188)
\curveto(513.86411529,1022.83547138)(513.82411533,1022.84047138)(513.77412087,1022.83047188)
\curveto(513.72411543,1022.81047141)(513.67411548,1022.78547143)(513.62412087,1022.75547188)
\curveto(513.58411557,1022.72547149)(513.53911561,1022.70047152)(513.48912087,1022.68047188)
\curveto(513.43911571,1022.64047158)(513.38411577,1022.60547161)(513.32412087,1022.57547188)
\lineto(513.14412087,1022.48547188)
\curveto(513.01411614,1022.42547179)(512.87911627,1022.37547184)(512.73912087,1022.33547188)
\curveto(512.59911655,1022.30547191)(512.4541167,1022.27047195)(512.30412087,1022.23047188)
\curveto(512.23411692,1022.21047201)(512.16411699,1022.20047202)(512.09412087,1022.20047188)
\curveto(512.03411712,1022.19047203)(511.96911718,1022.18047204)(511.89912087,1022.17047188)
\lineto(511.80912087,1022.17047188)
\curveto(511.77911737,1022.16047206)(511.7491174,1022.15547206)(511.71912087,1022.15547188)
\lineto(511.55412087,1022.15547188)
\curveto(511.4541177,1022.13547208)(511.3541178,1022.13547208)(511.25412087,1022.15547188)
\lineto(511.11912087,1022.15547188)
\curveto(511.0491181,1022.17547204)(510.97911817,1022.18547203)(510.90912087,1022.18547188)
\curveto(510.8491183,1022.17547204)(510.78911836,1022.18047204)(510.72912087,1022.20047188)
\curveto(510.62911852,1022.220472)(510.53411862,1022.24047198)(510.44412087,1022.26047188)
\curveto(510.3541188,1022.27047195)(510.26911888,1022.29547192)(510.18912087,1022.33547188)
\curveto(509.89911925,1022.44547177)(509.6491195,1022.58547163)(509.43912087,1022.75547188)
\curveto(509.23911991,1022.93547128)(509.07912007,1023.17047105)(508.95912087,1023.46047188)
\curveto(508.92912022,1023.53047069)(508.89912025,1023.60547061)(508.86912087,1023.68547188)
\curveto(508.8491203,1023.76547045)(508.82912032,1023.85047037)(508.80912087,1023.94047188)
\curveto(508.78912036,1023.99047023)(508.77912037,1024.04047018)(508.77912087,1024.09047188)
\curveto(508.78912036,1024.14047008)(508.78912036,1024.19047003)(508.77912087,1024.24047188)
\curveto(508.76912038,1024.27046995)(508.75912039,1024.33046989)(508.74912087,1024.42047188)
\curveto(508.7491204,1024.5204697)(508.7541204,1024.59046963)(508.76412087,1024.63047188)
\curveto(508.78412037,1024.73046949)(508.79412036,1024.8154694)(508.79412087,1024.88547188)
\lineto(508.88412087,1025.21547188)
\curveto(508.91412024,1025.33546888)(508.9541202,1025.44046878)(509.00412087,1025.53047188)
\curveto(509.17411998,1025.8204684)(509.36911978,1026.04046818)(509.58912087,1026.19047188)
\curveto(509.80911934,1026.34046788)(510.08911906,1026.47046775)(510.42912087,1026.58047188)
\curveto(510.55911859,1026.63046759)(510.69411846,1026.66546755)(510.83412087,1026.68547188)
\curveto(510.97411818,1026.70546751)(511.11411804,1026.73046749)(511.25412087,1026.76047188)
\curveto(511.33411782,1026.78046744)(511.41911773,1026.79046743)(511.50912087,1026.79047188)
\curveto(511.59911755,1026.80046742)(511.68911746,1026.8154674)(511.77912087,1026.83547188)
\curveto(511.8491173,1026.85546736)(511.91911723,1026.86046736)(511.98912087,1026.85047188)
\curveto(512.05911709,1026.85046737)(512.13411702,1026.86046736)(512.21412087,1026.88047188)
\curveto(512.28411687,1026.90046732)(512.3541168,1026.91046731)(512.42412087,1026.91047188)
\curveto(512.49411666,1026.91046731)(512.56911658,1026.9204673)(512.64912087,1026.94047188)
\curveto(512.85911629,1026.99046723)(513.0491161,1027.03046719)(513.21912087,1027.06047188)
\curveto(513.39911575,1027.10046712)(513.55911559,1027.19046703)(513.69912087,1027.33047188)
\curveto(513.78911536,1027.4204668)(513.8491153,1027.5204667)(513.87912087,1027.63047188)
\curveto(513.88911526,1027.66046656)(513.88911526,1027.68546653)(513.87912087,1027.70547188)
\curveto(513.87911527,1027.72546649)(513.88411527,1027.74546647)(513.89412087,1027.76547188)
\curveto(513.90411525,1027.78546643)(513.90911524,1027.8154664)(513.90912087,1027.85547188)
\lineto(513.90912087,1027.94547188)
\lineto(513.87912087,1028.06547188)
\curveto(513.87911527,1028.10546611)(513.87411528,1028.14046608)(513.86412087,1028.17047188)
\curveto(513.76411539,1028.47046575)(513.5541156,1028.67546554)(513.23412087,1028.78547188)
\curveto(513.14411601,1028.8154654)(513.03411612,1028.83546538)(512.90412087,1028.84547188)
\curveto(512.78411637,1028.86546535)(512.65911649,1028.87046535)(512.52912087,1028.86047188)
\curveto(512.39911675,1028.86046536)(512.27411688,1028.85046537)(512.15412087,1028.83047188)
\curveto(512.03411712,1028.81046541)(511.92911722,1028.78546543)(511.83912087,1028.75547188)
\curveto(511.77911737,1028.73546548)(511.71911743,1028.70546551)(511.65912087,1028.66547188)
\curveto(511.60911754,1028.63546558)(511.55911759,1028.60046562)(511.50912087,1028.56047188)
\curveto(511.45911769,1028.5204657)(511.40411775,1028.46546575)(511.34412087,1028.39547188)
\curveto(511.29411786,1028.32546589)(511.25911789,1028.26046596)(511.23912087,1028.20047188)
\curveto(511.18911796,1028.10046612)(511.14411801,1028.00546621)(511.10412087,1027.91547188)
\curveto(511.07411808,1027.82546639)(511.00411815,1027.76546645)(510.89412087,1027.73547188)
\curveto(510.81411834,1027.7154665)(510.72911842,1027.70546651)(510.63912087,1027.70547188)
\lineto(510.36912087,1027.70547188)
\lineto(509.79912087,1027.70547188)
\curveto(509.7491194,1027.70546651)(509.69911945,1027.70046652)(509.64912087,1027.69047188)
\curveto(509.59911955,1027.69046653)(509.5541196,1027.69546652)(509.51412087,1027.70547188)
\lineto(509.37912087,1027.70547188)
\curveto(509.35911979,1027.7154665)(509.33411982,1027.7204665)(509.30412087,1027.72047188)
\curveto(509.27411988,1027.7204665)(509.2491199,1027.73046649)(509.22912087,1027.75047188)
\curveto(509.14912,1027.77046645)(509.09412006,1027.83546638)(509.06412087,1027.94547188)
\curveto(509.0541201,1027.99546622)(509.0541201,1028.04546617)(509.06412087,1028.09547188)
\curveto(509.07412008,1028.14546607)(509.08412007,1028.19046603)(509.09412087,1028.23047188)
\curveto(509.12412003,1028.34046588)(509.15412,1028.44046578)(509.18412087,1028.53047188)
\curveto(509.22411993,1028.63046559)(509.26911988,1028.7204655)(509.31912087,1028.80047188)
\lineto(509.40912087,1028.95047188)
\lineto(509.49912087,1029.10047188)
\curveto(509.57911957,1029.21046501)(509.67911947,1029.3154649)(509.79912087,1029.41547188)
\curveto(509.81911933,1029.42546479)(509.8491193,1029.45046477)(509.88912087,1029.49047188)
\curveto(509.93911921,1029.53046469)(509.98411917,1029.56546465)(510.02412087,1029.59547188)
\curveto(510.06411909,1029.62546459)(510.10911904,1029.65546456)(510.15912087,1029.68547188)
\curveto(510.32911882,1029.79546442)(510.50911864,1029.88046434)(510.69912087,1029.94047188)
\curveto(510.88911826,1030.01046421)(511.08411807,1030.07546414)(511.28412087,1030.13547188)
\curveto(511.40411775,1030.16546405)(511.52911762,1030.18546403)(511.65912087,1030.19547188)
\curveto(511.78911736,1030.20546401)(511.91911723,1030.22546399)(512.04912087,1030.25547188)
\curveto(512.08911706,1030.26546395)(512.149117,1030.26546395)(512.22912087,1030.25547188)
\curveto(512.31911683,1030.24546397)(512.37411678,1030.25046397)(512.39412087,1030.27047188)
\curveto(512.80411635,1030.28046394)(513.19411596,1030.26546395)(513.56412087,1030.22547188)
\curveto(513.94411521,1030.18546403)(514.28411487,1030.11046411)(514.58412087,1030.00047188)
\curveto(514.89411426,1029.89046433)(515.15911399,1029.74046448)(515.37912087,1029.55047188)
\curveto(515.59911355,1029.37046485)(515.76911338,1029.13546508)(515.88912087,1028.84547188)
\curveto(515.95911319,1028.67546554)(515.99911315,1028.48046574)(516.00912087,1028.26047188)
\curveto(516.01911313,1028.04046618)(516.02411313,1027.8154664)(516.02412087,1027.58547188)
\lineto(516.02412087,1024.24047188)
\lineto(516.02412087,1023.65547188)
\curveto(516.02411313,1023.46547075)(516.04411311,1023.29047093)(516.08412087,1023.13047188)
\curveto(516.09411306,1023.10047112)(516.09911305,1023.06547115)(516.09912087,1023.02547188)
\curveto(516.09911305,1022.99547122)(516.10411305,1022.96547125)(516.11412087,1022.93547188)
\moveto(513.90912087,1025.24547188)
\curveto(513.91911523,1025.29546892)(513.92411523,1025.35046887)(513.92412087,1025.41047188)
\curveto(513.92411523,1025.48046874)(513.91911523,1025.54046868)(513.90912087,1025.59047188)
\curveto(513.88911526,1025.65046857)(513.87911527,1025.70546851)(513.87912087,1025.75547188)
\curveto(513.87911527,1025.80546841)(513.85911529,1025.84546837)(513.81912087,1025.87547188)
\curveto(513.76911538,1025.9154683)(513.69411546,1025.93546828)(513.59412087,1025.93547188)
\curveto(513.5541156,1025.92546829)(513.51911563,1025.9154683)(513.48912087,1025.90547188)
\curveto(513.45911569,1025.90546831)(513.42411573,1025.90046832)(513.38412087,1025.89047188)
\curveto(513.31411584,1025.87046835)(513.23911591,1025.85546836)(513.15912087,1025.84547188)
\curveto(513.07911607,1025.83546838)(512.99911615,1025.8204684)(512.91912087,1025.80047188)
\curveto(512.88911626,1025.79046843)(512.84411631,1025.78546843)(512.78412087,1025.78547188)
\curveto(512.6541165,1025.75546846)(512.52411663,1025.73546848)(512.39412087,1025.72547188)
\curveto(512.26411689,1025.7154685)(512.13911701,1025.69046853)(512.01912087,1025.65047188)
\curveto(511.93911721,1025.63046859)(511.86411729,1025.61046861)(511.79412087,1025.59047188)
\curveto(511.72411743,1025.58046864)(511.6541175,1025.56046866)(511.58412087,1025.53047188)
\curveto(511.37411778,1025.44046878)(511.19411796,1025.30546891)(511.04412087,1025.12547188)
\curveto(510.90411825,1024.94546927)(510.8541183,1024.69546952)(510.89412087,1024.37547188)
\curveto(510.91411824,1024.20547001)(510.96911818,1024.06547015)(511.05912087,1023.95547188)
\curveto(511.12911802,1023.84547037)(511.23411792,1023.75547046)(511.37412087,1023.68547188)
\curveto(511.51411764,1023.62547059)(511.66411749,1023.58047064)(511.82412087,1023.55047188)
\curveto(511.99411716,1023.5204707)(512.16911698,1023.51047071)(512.34912087,1023.52047188)
\curveto(512.53911661,1023.54047068)(512.71411644,1023.57547064)(512.87412087,1023.62547188)
\curveto(513.13411602,1023.70547051)(513.33911581,1023.83047039)(513.48912087,1024.00047188)
\curveto(513.63911551,1024.18047004)(513.7541154,1024.40046982)(513.83412087,1024.66047188)
\curveto(513.8541153,1024.73046949)(513.86411529,1024.80046942)(513.86412087,1024.87047188)
\curveto(513.87411528,1024.95046927)(513.88911526,1025.03046919)(513.90912087,1025.11047188)
\lineto(513.90912087,1025.24547188)
}
}
{
\newrgbcolor{curcolor}{0 0 0}
\pscustom[linestyle=none,fillstyle=solid,fillcolor=curcolor]
{
}
}
{
\newrgbcolor{curcolor}{0 0 0}
\pscustom[linestyle=none,fillstyle=solid,fillcolor=curcolor]
{
\newpath
\moveto(521.58255837,1030.07547188)
\lineto(522.67755837,1030.07547188)
\curveto(522.78755664,1030.07546414)(522.89255653,1030.07046415)(522.99255837,1030.06047188)
\curveto(523.09255633,1030.06046416)(523.17755625,1030.04046418)(523.24755837,1030.00047188)
\curveto(523.35755607,1029.93046429)(523.43255599,1029.80046442)(523.47255837,1029.61047188)
\curveto(523.5225559,1029.43046479)(523.57255585,1029.27046495)(523.62255837,1029.13047188)
\curveto(523.73255569,1028.80046542)(523.83755559,1028.46046576)(523.93755837,1028.11047188)
\curveto(524.03755539,1027.77046645)(524.14255528,1027.43046679)(524.25255837,1027.09047188)
\curveto(524.33255509,1026.85046737)(524.40755502,1026.60546761)(524.47755837,1026.35547188)
\curveto(524.54755488,1026.10546811)(524.6275548,1025.86546835)(524.71755837,1025.63547188)
\curveto(524.74755468,1025.54546867)(524.77755465,1025.45046877)(524.80755837,1025.35047188)
\curveto(524.84755458,1025.25046897)(524.9225545,1025.20046902)(525.03255837,1025.20047188)
\curveto(525.05255437,1025.220469)(525.06755436,1025.23046899)(525.07755837,1025.23047188)
\lineto(525.12255837,1025.27547188)
\curveto(525.15255427,1025.32546889)(525.17755425,1025.37546884)(525.19755837,1025.42547188)
\curveto(525.21755421,1025.47546874)(525.23755419,1025.53046869)(525.25755837,1025.59047188)
\curveto(525.30755412,1025.70046852)(525.34255408,1025.8154684)(525.36255837,1025.93547188)
\curveto(525.39255403,1026.05546816)(525.427554,1026.17046805)(525.46755837,1026.28047188)
\curveto(525.60755382,1026.70046752)(525.73755369,1027.1204671)(525.85755837,1027.54047188)
\curveto(525.98755344,1027.97046625)(526.1225533,1028.39546582)(526.26255837,1028.81547188)
\curveto(526.31255311,1028.93546528)(526.35255307,1029.05546516)(526.38255837,1029.17547188)
\curveto(526.41255301,1029.30546491)(526.44755298,1029.42546479)(526.48755837,1029.53547188)
\curveto(526.50755292,1029.6154646)(526.53255289,1029.69546452)(526.56255837,1029.77547188)
\curveto(526.59255283,1029.85546436)(526.63755279,1029.9204643)(526.69755837,1029.97047188)
\curveto(526.7275527,1029.99046423)(526.79255263,1030.0204642)(526.89255837,1030.06047188)
\curveto(526.95255247,1030.08046414)(527.01755241,1030.08546413)(527.08755837,1030.07547188)
\lineto(527.28255837,1030.07547188)
\lineto(528.01755837,1030.07547188)
\curveto(528.1275513,1030.07546414)(528.2275512,1030.07046415)(528.31755837,1030.06047188)
\curveto(528.41755101,1030.06046416)(528.49255093,1030.03546418)(528.54255837,1029.98547188)
\curveto(528.60255082,1029.93546428)(528.6225508,1029.86046436)(528.60255837,1029.76047188)
\curveto(528.59255083,1029.67046455)(528.57755085,1029.59546462)(528.55755837,1029.53547188)
\curveto(528.50755092,1029.39546482)(528.45755097,1029.24546497)(528.40755837,1029.08547188)
\curveto(528.35755107,1028.93546528)(528.30755112,1028.79046543)(528.25755837,1028.65047188)
\curveto(528.2275512,1028.58046564)(528.20255122,1028.51046571)(528.18255837,1028.44047188)
\curveto(528.17255125,1028.38046584)(528.15755127,1028.3204659)(528.13755837,1028.26047188)
\curveto(528.08755134,1028.15046607)(528.04255138,1028.04046618)(528.00255837,1027.93047188)
\curveto(527.97255145,1027.8204664)(527.93755149,1027.71046651)(527.89755837,1027.60047188)
\curveto(527.88755154,1027.57046665)(527.87755155,1027.53546668)(527.86755837,1027.49547188)
\curveto(527.86755156,1027.46546675)(527.85755157,1027.43546678)(527.83755837,1027.40547188)
\curveto(527.77755165,1027.25546696)(527.7225517,1027.10046712)(527.67255837,1026.94047188)
\curveto(527.63255179,1026.79046743)(527.58755184,1026.64046758)(527.53755837,1026.49047188)
\curveto(527.427552,1026.19046803)(527.3225521,1025.88046834)(527.22255837,1025.56047188)
\curveto(527.1225523,1025.25046897)(527.01255241,1024.94546927)(526.89255837,1024.64547188)
\curveto(526.84255258,1024.50546971)(526.79755263,1024.36546985)(526.75755837,1024.22547188)
\curveto(526.71755271,1024.09547012)(526.67255275,1023.96047026)(526.62255837,1023.82047188)
\curveto(526.60255282,1023.77047045)(526.58755284,1023.72547049)(526.57755837,1023.68547188)
\curveto(526.56755286,1023.65547056)(526.55255287,1023.6154706)(526.53255837,1023.56547188)
\curveto(526.49255293,1023.45547076)(526.45255297,1023.34047088)(526.41255837,1023.22047188)
\curveto(526.38255304,1023.11047111)(526.34755308,1023.00047122)(526.30755837,1022.89047188)
\curveto(526.25755317,1022.78047144)(526.21255321,1022.67547154)(526.17255837,1022.57547188)
\curveto(526.13255329,1022.48547173)(526.05755337,1022.4204718)(525.94755837,1022.38047188)
\curveto(525.86755356,1022.35047187)(525.78255364,1022.33547188)(525.69255837,1022.33547188)
\curveto(525.60255382,1022.34547187)(525.51255391,1022.35047187)(525.42255837,1022.35047188)
\lineto(524.44755837,1022.35047188)
\lineto(524.11755837,1022.35047188)
\curveto(524.01755541,1022.35047187)(523.9275555,1022.37047185)(523.84755837,1022.41047188)
\curveto(523.77755565,1022.46047176)(523.7275557,1022.5204717)(523.69755837,1022.59047188)
\curveto(523.66755576,1022.67047155)(523.63755579,1022.75547146)(523.60755837,1022.84547188)
\curveto(523.55755587,1022.96547125)(523.51255591,1023.09047113)(523.47255837,1023.22047188)
\curveto(523.43255599,1023.35047087)(523.38755604,1023.47547074)(523.33755837,1023.59547188)
\curveto(523.31755611,1023.63547058)(523.30255612,1023.67047055)(523.29255837,1023.70047188)
\curveto(523.29255613,1023.74047048)(523.28255614,1023.78547043)(523.26255837,1023.83547188)
\curveto(523.21255621,1023.95547026)(523.16755626,1024.08047014)(523.12755837,1024.21047188)
\curveto(523.08755634,1024.34046988)(523.04255638,1024.47046975)(522.99255837,1024.60047188)
\curveto(522.97255645,1024.65046957)(522.95755647,1024.69546952)(522.94755837,1024.73547188)
\curveto(522.93755649,1024.78546943)(522.9225565,1024.83546938)(522.90255837,1024.88547188)
\curveto(522.86255656,1024.97546924)(522.8275566,1025.06546915)(522.79755837,1025.15547188)
\curveto(522.76755666,1025.25546896)(522.73755669,1025.35046887)(522.70755837,1025.44047188)
\curveto(522.68755674,1025.47046875)(522.67755675,1025.50046872)(522.67755837,1025.53047188)
\curveto(522.67755675,1025.57046865)(522.66755676,1025.60546861)(522.64755837,1025.63547188)
\curveto(522.60755682,1025.72546849)(522.57255685,1025.8154684)(522.54255837,1025.90547188)
\curveto(522.5225569,1025.99546822)(522.49255693,1026.09046813)(522.45255837,1026.19047188)
\curveto(522.36255706,1026.4204678)(522.27755715,1026.65546756)(522.19755837,1026.89547188)
\curveto(522.11755731,1027.14546707)(522.03755739,1027.39046683)(521.95755837,1027.63047188)
\curveto(521.84755758,1027.90046632)(521.75255767,1028.17046605)(521.67255837,1028.44047188)
\curveto(521.59255783,1028.7204655)(521.50255792,1028.99546522)(521.40255837,1029.26547188)
\lineto(521.26755837,1029.67047188)
\curveto(521.25755817,1029.70046452)(521.24755818,1029.73546448)(521.23755837,1029.77547188)
\curveto(521.23755819,1029.82546439)(521.24755818,1029.87046435)(521.26755837,1029.91047188)
\curveto(521.29755813,1029.98046424)(521.36755806,1030.03046419)(521.47755837,1030.06047188)
\curveto(521.5275579,1030.06046416)(521.56255786,1030.06546415)(521.58255837,1030.07547188)
}
}
{
\newrgbcolor{curcolor}{0 0 0}
\pscustom[linestyle=none,fillstyle=solid,fillcolor=curcolor]
{
\newpath
\moveto(536.99552712,1026.28047188)
\curveto(537.01551895,1026.20046802)(537.01551895,1026.11046811)(536.99552712,1026.01047188)
\curveto(536.97551899,1025.91046831)(536.94051903,1025.84546837)(536.89052712,1025.81547188)
\curveto(536.84051913,1025.77546844)(536.7655192,1025.74546847)(536.66552712,1025.72547188)
\curveto(536.57551939,1025.7154685)(536.4705195,1025.70546851)(536.35052712,1025.69547188)
\lineto(536.00552712,1025.69547188)
\curveto(535.89552007,1025.70546851)(535.79552017,1025.71046851)(535.70552712,1025.71047188)
\lineto(532.04552712,1025.71047188)
\lineto(531.83552712,1025.71047188)
\curveto(531.77552419,1025.71046851)(531.72052425,1025.70046852)(531.67052712,1025.68047188)
\curveto(531.59052438,1025.64046858)(531.54052443,1025.60046862)(531.52052712,1025.56047188)
\curveto(531.50052447,1025.54046868)(531.48052449,1025.50046872)(531.46052712,1025.44047188)
\curveto(531.44052453,1025.39046883)(531.43552453,1025.34046888)(531.44552712,1025.29047188)
\curveto(531.4655245,1025.23046899)(531.47552449,1025.17046905)(531.47552712,1025.11047188)
\curveto(531.48552448,1025.06046916)(531.50052447,1025.00546921)(531.52052712,1024.94547188)
\curveto(531.60052437,1024.70546951)(531.69552427,1024.50546971)(531.80552712,1024.34547188)
\curveto(531.92552404,1024.19547002)(532.08552388,1024.06047016)(532.28552712,1023.94047188)
\curveto(532.3655236,1023.89047033)(532.44552352,1023.85547036)(532.52552712,1023.83547188)
\curveto(532.61552335,1023.82547039)(532.70552326,1023.80547041)(532.79552712,1023.77547188)
\curveto(532.87552309,1023.75547046)(532.98552298,1023.74047048)(533.12552712,1023.73047188)
\curveto(533.2655227,1023.7204705)(533.38552258,1023.72547049)(533.48552712,1023.74547188)
\lineto(533.62052712,1023.74547188)
\curveto(533.72052225,1023.76547045)(533.81052216,1023.78547043)(533.89052712,1023.80547188)
\curveto(533.98052199,1023.83547038)(534.0655219,1023.86547035)(534.14552712,1023.89547188)
\curveto(534.24552172,1023.94547027)(534.35552161,1024.01047021)(534.47552712,1024.09047188)
\curveto(534.60552136,1024.17047005)(534.70052127,1024.25046997)(534.76052712,1024.33047188)
\curveto(534.81052116,1024.40046982)(534.86052111,1024.46546975)(534.91052712,1024.52547188)
\curveto(534.970521,1024.59546962)(535.04052093,1024.64546957)(535.12052712,1024.67547188)
\curveto(535.22052075,1024.72546949)(535.34552062,1024.74546947)(535.49552712,1024.73547188)
\lineto(535.93052712,1024.73547188)
\lineto(536.11052712,1024.73547188)
\curveto(536.18051979,1024.74546947)(536.24051973,1024.74046948)(536.29052712,1024.72047188)
\lineto(536.44052712,1024.72047188)
\curveto(536.54051943,1024.70046952)(536.61051936,1024.67546954)(536.65052712,1024.64547188)
\curveto(536.69051928,1024.62546959)(536.71051926,1024.58046964)(536.71052712,1024.51047188)
\curveto(536.72051925,1024.44046978)(536.71551925,1024.38046984)(536.69552712,1024.33047188)
\curveto(536.64551932,1024.19047003)(536.59051938,1024.06547015)(536.53052712,1023.95547188)
\curveto(536.4705195,1023.84547037)(536.40051957,1023.73547048)(536.32052712,1023.62547188)
\curveto(536.10051987,1023.29547092)(535.85052012,1023.03047119)(535.57052712,1022.83047188)
\curveto(535.29052068,1022.63047159)(534.94052103,1022.46047176)(534.52052712,1022.32047188)
\curveto(534.41052156,1022.28047194)(534.30052167,1022.25547196)(534.19052712,1022.24547188)
\curveto(534.08052189,1022.23547198)(533.965522,1022.215472)(533.84552712,1022.18547188)
\curveto(533.80552216,1022.17547204)(533.76052221,1022.17547204)(533.71052712,1022.18547188)
\curveto(533.6705223,1022.18547203)(533.63052234,1022.18047204)(533.59052712,1022.17047188)
\lineto(533.42552712,1022.17047188)
\curveto(533.37552259,1022.15047207)(533.31552265,1022.14547207)(533.24552712,1022.15547188)
\curveto(533.18552278,1022.15547206)(533.13052284,1022.16047206)(533.08052712,1022.17047188)
\curveto(533.00052297,1022.18047204)(532.93052304,1022.18047204)(532.87052712,1022.17047188)
\curveto(532.81052316,1022.16047206)(532.74552322,1022.16547205)(532.67552712,1022.18547188)
\curveto(532.62552334,1022.20547201)(532.5705234,1022.215472)(532.51052712,1022.21547188)
\curveto(532.45052352,1022.215472)(532.39552357,1022.22547199)(532.34552712,1022.24547188)
\curveto(532.23552373,1022.26547195)(532.12552384,1022.29047193)(532.01552712,1022.32047188)
\curveto(531.90552406,1022.34047188)(531.80552416,1022.37547184)(531.71552712,1022.42547188)
\curveto(531.60552436,1022.46547175)(531.50052447,1022.50047172)(531.40052712,1022.53047188)
\curveto(531.31052466,1022.57047165)(531.22552474,1022.6154716)(531.14552712,1022.66547188)
\curveto(530.82552514,1022.86547135)(530.54052543,1023.09547112)(530.29052712,1023.35547188)
\curveto(530.04052593,1023.62547059)(529.83552613,1023.93547028)(529.67552712,1024.28547188)
\curveto(529.62552634,1024.39546982)(529.58552638,1024.50546971)(529.55552712,1024.61547188)
\curveto(529.52552644,1024.73546948)(529.48552648,1024.85546936)(529.43552712,1024.97547188)
\curveto(529.42552654,1025.0154692)(529.42052655,1025.05046917)(529.42052712,1025.08047188)
\curveto(529.42052655,1025.1204691)(529.41552655,1025.16046906)(529.40552712,1025.20047188)
\curveto(529.3655266,1025.3204689)(529.34052663,1025.45046877)(529.33052712,1025.59047188)
\lineto(529.30052712,1026.01047188)
\curveto(529.30052667,1026.06046816)(529.29552667,1026.1154681)(529.28552712,1026.17547188)
\curveto(529.28552668,1026.23546798)(529.29052668,1026.29046793)(529.30052712,1026.34047188)
\lineto(529.30052712,1026.52047188)
\lineto(529.34552712,1026.88047188)
\curveto(529.38552658,1027.05046717)(529.42052655,1027.215467)(529.45052712,1027.37547188)
\curveto(529.48052649,1027.53546668)(529.52552644,1027.68546653)(529.58552712,1027.82547188)
\curveto(530.01552595,1028.86546535)(530.74552522,1029.60046462)(531.77552712,1030.03047188)
\curveto(531.91552405,1030.09046413)(532.05552391,1030.13046409)(532.19552712,1030.15047188)
\curveto(532.34552362,1030.18046404)(532.50052347,1030.215464)(532.66052712,1030.25547188)
\curveto(532.74052323,1030.26546395)(532.81552315,1030.27046395)(532.88552712,1030.27047188)
\curveto(532.95552301,1030.27046395)(533.03052294,1030.27546394)(533.11052712,1030.28547188)
\curveto(533.62052235,1030.29546392)(534.05552191,1030.23546398)(534.41552712,1030.10547188)
\curveto(534.78552118,1029.98546423)(535.11552085,1029.82546439)(535.40552712,1029.62547188)
\curveto(535.49552047,1029.56546465)(535.58552038,1029.49546472)(535.67552712,1029.41547188)
\curveto(535.7655202,1029.34546487)(535.84552012,1029.27046495)(535.91552712,1029.19047188)
\curveto(535.94552002,1029.14046508)(535.98551998,1029.10046512)(536.03552712,1029.07047188)
\curveto(536.11551985,1028.96046526)(536.19051978,1028.84546537)(536.26052712,1028.72547188)
\curveto(536.33051964,1028.6154656)(536.40551956,1028.50046572)(536.48552712,1028.38047188)
\curveto(536.53551943,1028.29046593)(536.57551939,1028.19546602)(536.60552712,1028.09547188)
\curveto(536.64551932,1028.00546621)(536.68551928,1027.90546631)(536.72552712,1027.79547188)
\curveto(536.77551919,1027.66546655)(536.81551915,1027.53046669)(536.84552712,1027.39047188)
\curveto(536.87551909,1027.25046697)(536.91051906,1027.11046711)(536.95052712,1026.97047188)
\curveto(536.970519,1026.89046733)(536.97551899,1026.80046742)(536.96552712,1026.70047188)
\curveto(536.965519,1026.61046761)(536.97551899,1026.52546769)(536.99552712,1026.44547188)
\lineto(536.99552712,1026.28047188)
\moveto(534.74552712,1027.16547188)
\curveto(534.81552115,1027.26546695)(534.82052115,1027.38546683)(534.76052712,1027.52547188)
\curveto(534.71052126,1027.67546654)(534.6705213,1027.78546643)(534.64052712,1027.85547188)
\curveto(534.50052147,1028.12546609)(534.31552165,1028.33046589)(534.08552712,1028.47047188)
\curveto(533.85552211,1028.6204656)(533.53552243,1028.70046552)(533.12552712,1028.71047188)
\curveto(533.09552287,1028.69046553)(533.06052291,1028.68546553)(533.02052712,1028.69547188)
\curveto(532.98052299,1028.70546551)(532.94552302,1028.70546551)(532.91552712,1028.69547188)
\curveto(532.8655231,1028.67546554)(532.81052316,1028.66046556)(532.75052712,1028.65047188)
\curveto(532.69052328,1028.65046557)(532.63552333,1028.64046558)(532.58552712,1028.62047188)
\curveto(532.14552382,1028.48046574)(531.82052415,1028.20546601)(531.61052712,1027.79547188)
\curveto(531.59052438,1027.75546646)(531.5655244,1027.70046652)(531.53552712,1027.63047188)
\curveto(531.51552445,1027.57046665)(531.50052447,1027.50546671)(531.49052712,1027.43547188)
\curveto(531.48052449,1027.37546684)(531.48052449,1027.3154669)(531.49052712,1027.25547188)
\curveto(531.51052446,1027.19546702)(531.54552442,1027.14546707)(531.59552712,1027.10547188)
\curveto(531.67552429,1027.05546716)(531.78552418,1027.03046719)(531.92552712,1027.03047188)
\lineto(532.33052712,1027.03047188)
\lineto(533.99552712,1027.03047188)
\lineto(534.43052712,1027.03047188)
\curveto(534.59052138,1027.04046718)(534.69552127,1027.08546713)(534.74552712,1027.16547188)
}
}
{
\newrgbcolor{curcolor}{0 0 0}
\pscustom[linestyle=none,fillstyle=solid,fillcolor=curcolor]
{
\newpath
\moveto(538.49880837,1030.07547188)
\lineto(543.10380837,1030.07547188)
\lineto(543.97380837,1030.07547188)
\curveto(544.06380177,1030.07546414)(544.15380168,1030.07046415)(544.24380837,1030.06047188)
\curveto(544.3338015,1030.06046416)(544.40380143,1030.04046418)(544.45380837,1030.00047188)
\curveto(544.51380132,1029.96046426)(544.56380127,1029.88546433)(544.60380837,1029.77547188)
\lineto(544.60380837,1029.55047188)
\curveto(544.62380121,1029.50046472)(544.6288012,1029.44046478)(544.61880837,1029.37047188)
\lineto(544.61880837,1029.20547188)
\lineto(544.61880837,1028.81547188)
\curveto(544.61880121,1028.68546553)(544.59880123,1028.57546564)(544.55880837,1028.48547188)
\curveto(544.51880131,1028.40546581)(544.47380136,1028.33546588)(544.42380837,1028.27547188)
\curveto(544.37380146,1028.22546599)(544.32380151,1028.17046605)(544.27380837,1028.11047188)
\curveto(544.21380162,1028.03046619)(544.14880168,1027.95046627)(544.07880837,1027.87047188)
\curveto(544.00880182,1027.80046642)(543.94380189,1027.72546649)(543.88380837,1027.64547188)
\lineto(543.82380837,1027.58547188)
\curveto(543.80380203,1027.57546664)(543.78380205,1027.56046666)(543.76380837,1027.54047188)
\curveto(543.71380212,1027.47046675)(543.65380218,1027.40546681)(543.58380837,1027.34547188)
\curveto(543.52380231,1027.28546693)(543.46880236,1027.220467)(543.41880837,1027.15047188)
\curveto(543.25880257,1026.94046728)(543.08880274,1026.74546747)(542.90880837,1026.56547188)
\curveto(542.7288031,1026.38546783)(542.55880327,1026.19046803)(542.39880837,1025.98047188)
\lineto(542.27880837,1025.86047188)
\curveto(542.2288036,1025.79046843)(542.16880366,1025.72546849)(542.09880837,1025.66547188)
\curveto(542.03880379,1025.60546861)(541.98380385,1025.54046868)(541.93380837,1025.47047188)
\curveto(541.88380395,1025.40046882)(541.82380401,1025.33546888)(541.75380837,1025.27547188)
\curveto(541.69380414,1025.215469)(541.63880419,1025.15546906)(541.58880837,1025.09547188)
\curveto(541.53880429,1025.03546918)(541.48880434,1024.98046924)(541.43880837,1024.93047188)
\curveto(541.39880443,1024.88046934)(541.35380448,1024.82546939)(541.30380837,1024.76547188)
\lineto(541.18380837,1024.61547188)
\curveto(541.14380469,1024.57546964)(541.10380473,1024.53046969)(541.06380837,1024.48047188)
\curveto(541.02380481,1024.44046978)(540.98880484,1024.40046982)(540.95880837,1024.36047188)
\curveto(540.9288049,1024.3204699)(540.90380493,1024.27546994)(540.88380837,1024.22547188)
\curveto(540.87380496,1024.20547001)(540.86380497,1024.18047004)(540.85380837,1024.15047188)
\curveto(540.85380498,1024.1204701)(540.86380497,1024.09547012)(540.88380837,1024.07547188)
\curveto(540.90380493,1024.0154702)(540.93880489,1023.98047024)(540.98880837,1023.97047188)
\curveto(541.03880479,1023.97047025)(541.09380474,1023.96047026)(541.15380837,1023.94047188)
\curveto(541.25380458,1023.9204703)(541.36380447,1023.91047031)(541.48380837,1023.91047188)
\curveto(541.61380422,1023.9204703)(541.7338041,1023.92547029)(541.84380837,1023.92547188)
\lineto(544.04880837,1023.92547188)
\curveto(544.2288016,1023.92547029)(544.39880143,1023.9204703)(544.55880837,1023.91047188)
\curveto(544.71880111,1023.90047032)(544.82380101,1023.82547039)(544.87380837,1023.68547188)
\curveto(544.89380094,1023.63547058)(544.90380093,1023.58047064)(544.90380837,1023.52047188)
\lineto(544.90380837,1023.32547188)
\lineto(544.90380837,1022.81547188)
\curveto(544.90380093,1022.62547159)(544.85880097,1022.49547172)(544.76880837,1022.42547188)
\curveto(544.70880112,1022.37547184)(544.6338012,1022.35047187)(544.54380837,1022.35047188)
\lineto(544.24380837,1022.35047188)
\lineto(543.31380837,1022.35047188)
\lineto(539.48880837,1022.35047188)
\lineto(538.46880837,1022.35047188)
\lineto(538.15380837,1022.35047188)
\curveto(538.06380777,1022.35047187)(537.98380785,1022.37547184)(537.91380837,1022.42547188)
\curveto(537.85380798,1022.47547174)(537.81880801,1022.55547166)(537.80880837,1022.66547188)
\curveto(537.79880803,1022.77547144)(537.79380804,1022.88547133)(537.79380837,1022.99547188)
\curveto(537.79380804,1023.13547108)(537.78880804,1023.28547093)(537.77880837,1023.44547188)
\curveto(537.77880805,1023.6154706)(537.79880803,1023.75547046)(537.83880837,1023.86547188)
\curveto(537.85880797,1023.92547029)(537.88380795,1023.98047024)(537.91380837,1024.03047188)
\curveto(537.94380789,1024.09047013)(537.97880785,1024.14547007)(538.01880837,1024.19547188)
\curveto(538.09880773,1024.28546993)(538.17880765,1024.37046985)(538.25880837,1024.45047188)
\curveto(538.33880749,1024.54046968)(538.41880741,1024.63546958)(538.49880837,1024.73547188)
\curveto(538.54880728,1024.78546943)(538.58880724,1024.8204694)(538.61880837,1024.84047188)
\curveto(538.75880707,1025.03046919)(538.90880692,1025.20546901)(539.06880837,1025.36547188)
\curveto(539.2288066,1025.52546869)(539.37880645,1025.69546852)(539.51880837,1025.87547188)
\curveto(539.55880627,1025.92546829)(539.59880623,1025.97046825)(539.63880837,1026.01047188)
\curveto(539.67880615,1026.05046817)(539.71380612,1026.09546812)(539.74380837,1026.14547188)
\curveto(539.79380604,1026.215468)(539.84880598,1026.27546794)(539.90880837,1026.32547188)
\curveto(539.97880585,1026.38546783)(540.03880579,1026.45546776)(540.08880837,1026.53547188)
\curveto(540.10880572,1026.55546766)(540.1288057,1026.57046765)(540.14880837,1026.58047188)
\lineto(540.20880837,1026.64047188)
\curveto(540.25880557,1026.7204675)(540.31380552,1026.79046743)(540.37380837,1026.85047188)
\curveto(540.44380539,1026.91046731)(540.50380533,1026.97546724)(540.55380837,1027.04547188)
\curveto(540.57380526,1027.07546714)(540.59880523,1027.10546711)(540.62880837,1027.13547188)
\curveto(540.66880516,1027.16546705)(540.70380513,1027.19546702)(540.73380837,1027.22547188)
\curveto(540.81380502,1027.33546688)(540.89880493,1027.43546678)(540.98880837,1027.52547188)
\curveto(541.07880475,1027.6154666)(541.16380467,1027.71046651)(541.24380837,1027.81047188)
\curveto(541.29380454,1027.88046634)(541.34880448,1027.94546627)(541.40880837,1028.00547188)
\curveto(541.47880435,1028.06546615)(541.52380431,1028.14046608)(541.54380837,1028.23047188)
\curveto(541.56380427,1028.29046593)(541.55380428,1028.33546588)(541.51380837,1028.36547188)
\curveto(541.48380435,1028.39546582)(541.45380438,1028.4204658)(541.42380837,1028.44047188)
\curveto(541.31380452,1028.49046573)(541.17880465,1028.51046571)(541.01880837,1028.50047188)
\lineto(540.58380837,1028.50047188)
\lineto(538.96380837,1028.50047188)
\curveto(538.84380699,1028.50046572)(538.71380712,1028.49546572)(538.57380837,1028.48547188)
\curveto(538.44380739,1028.48546573)(538.33880749,1028.51046571)(538.25880837,1028.56047188)
\curveto(538.16880766,1028.61046561)(538.11880771,1028.70546551)(538.10880837,1028.84547188)
\curveto(538.09880773,1028.98546523)(538.09380774,1029.12546509)(538.09380837,1029.26547188)
\curveto(538.09380774,1029.3154649)(538.08880774,1029.37046485)(538.07880837,1029.43047188)
\curveto(538.07880775,1029.49046473)(538.08380775,1029.54546467)(538.09380837,1029.59547188)
\lineto(538.09380837,1029.71547188)
\curveto(538.10380773,1029.74546447)(538.10880772,1029.78046444)(538.10880837,1029.82047188)
\curveto(538.11880771,1029.86046436)(538.1338077,1029.89046433)(538.15380837,1029.91047188)
\curveto(538.21380762,1029.99046423)(538.27880755,1030.03546418)(538.34880837,1030.04547188)
\curveto(538.36880746,1030.05546416)(538.39380744,1030.05546416)(538.42380837,1030.04547188)
\curveto(538.45380738,1030.04546417)(538.47880735,1030.05546416)(538.49880837,1030.07547188)
}
}
{
\newrgbcolor{curcolor}{0 0 0}
\pscustom[linestyle=none,fillstyle=solid,fillcolor=curcolor]
{
\newpath
\moveto(26.7416285,703.05834076)
\lineto(28.0166285,703.05834076)
\curveto(28.12662572,703.05833005)(28.23162561,703.05333006)(28.3316285,703.04334076)
\curveto(28.4416254,703.03333008)(28.52162532,702.99833011)(28.5716285,702.93834076)
\curveto(28.62162522,702.85833025)(28.6466252,702.75333036)(28.6466285,702.62334076)
\curveto(28.65662519,702.50333061)(28.66162518,702.37833073)(28.6616285,702.24834076)
\lineto(28.6616285,700.73334076)
\lineto(28.6616285,697.64334076)
\lineto(28.6616285,697.11834076)
\curveto(28.66162518,697.07833603)(28.65662519,697.03333608)(28.6466285,696.98334076)
\curveto(28.6466252,696.94333617)(28.65162519,696.90333621)(28.6616285,696.86334076)
\lineto(28.6616285,696.62334076)
\curveto(28.66162518,696.53333658)(28.65662519,696.43833667)(28.6466285,696.33834076)
\curveto(28.6466252,696.23833687)(28.65662519,696.14833696)(28.6766285,696.06834076)
\curveto(28.67662517,695.99833711)(28.68162516,695.94333717)(28.6916285,695.90334076)
\curveto(28.71162513,695.79333732)(28.72662512,695.68333743)(28.7366285,695.57334076)
\curveto(28.75662509,695.46333765)(28.78662506,695.35333776)(28.8266285,695.24334076)
\curveto(28.93662491,694.98333813)(29.07662477,694.76833834)(29.2466285,694.59834076)
\curveto(29.42662442,694.42833868)(29.66162418,694.29333882)(29.9516285,694.19334076)
\curveto(30.03162381,694.17333894)(30.11162373,694.15833895)(30.1916285,694.14834076)
\curveto(30.27162357,694.13833897)(30.35162349,694.12333899)(30.4316285,694.10334076)
\curveto(30.48162336,694.08333903)(30.52662332,694.07333904)(30.5666285,694.07334076)
\curveto(30.60662324,694.08333903)(30.65162319,694.08333903)(30.7016285,694.07334076)
\curveto(30.7416231,694.06333905)(30.80662304,694.05833905)(30.8966285,694.05834076)
\curveto(30.98662286,694.06833904)(31.0466228,694.07833903)(31.0766285,694.08834076)
\lineto(31.3016285,694.08834076)
\curveto(31.38162246,694.108339)(31.46162238,694.12333899)(31.5416285,694.13334076)
\curveto(31.62162222,694.14333897)(31.69662215,694.15833895)(31.7666285,694.17834076)
\curveto(31.90662194,694.2083389)(32.01662183,694.24333887)(32.0966285,694.28334076)
\curveto(32.27662157,694.36333875)(32.43162141,694.46833864)(32.5616285,694.59834076)
\curveto(32.70162114,694.73833837)(32.81162103,694.89333822)(32.8916285,695.06334076)
\curveto(33.00162084,695.32333779)(33.06662078,695.62833748)(33.0866285,695.97834076)
\curveto(33.10662074,696.33833677)(33.11662073,696.7083364)(33.1166285,697.08834076)
\lineto(33.1166285,700.07334076)
\lineto(33.1166285,702.08334076)
\curveto(33.11662073,702.22333089)(33.11162073,702.37833073)(33.1016285,702.54834076)
\curveto(33.10162074,702.71833039)(33.13162071,702.84333027)(33.1916285,702.92334076)
\curveto(33.2416206,702.98333013)(33.31162053,703.01833009)(33.4016285,703.02834076)
\curveto(33.49162035,703.04833006)(33.59162025,703.05833005)(33.7016285,703.05834076)
\lineto(34.6616285,703.05834076)
\curveto(34.7416191,703.05833005)(34.81661903,703.05833005)(34.8866285,703.05834076)
\curveto(34.96661888,703.06833004)(35.0416188,703.06333005)(35.1116285,703.04334076)
\curveto(35.25161859,703.0133301)(35.3416185,702.96333015)(35.3816285,702.89334076)
\curveto(35.43161841,702.8133303)(35.45161839,702.69833041)(35.4416285,702.54834076)
\curveto(35.4416184,702.4083307)(35.4416184,702.27833083)(35.4416285,702.15834076)
\lineto(35.4416285,700.14834076)
\lineto(35.4416285,697.11834076)
\curveto(35.4416184,696.73833637)(35.43661841,696.36833674)(35.4266285,696.00834076)
\curveto(35.41661843,695.64833746)(35.37161847,695.32333779)(35.2916285,695.03334076)
\curveto(35.15161869,694.56333855)(34.97161887,694.15333896)(34.7516285,693.80334076)
\curveto(34.5416193,693.46333965)(34.26161958,693.17333994)(33.9116285,692.93334076)
\curveto(33.60162024,692.7133404)(33.23662061,692.53334058)(32.8166285,692.39334076)
\curveto(32.72662112,692.36334075)(32.63162121,692.33834077)(32.5316285,692.31834076)
\lineto(32.2616285,692.25834076)
\curveto(32.20162164,692.23834087)(32.1416217,692.22834088)(32.0816285,692.22834076)
\curveto(32.03162181,692.22834088)(31.97662187,692.21834089)(31.9166285,692.19834076)
\curveto(31.79662205,692.17834093)(31.66162218,692.16334095)(31.5116285,692.15334076)
\curveto(31.36162248,692.14334097)(31.21662263,692.13834097)(31.0766285,692.13834076)
\curveto(30.12662372,692.12834098)(29.31662453,692.24334087)(28.6466285,692.48334076)
\curveto(27.97662587,692.73334038)(27.45162639,693.13333998)(27.0716285,693.68334076)
\curveto(26.9416269,693.86333925)(26.83162701,694.04833906)(26.7416285,694.23834076)
\curveto(26.66162718,694.43833867)(26.58662726,694.65333846)(26.5166285,694.88334076)
\curveto(26.49662735,694.93333818)(26.48662736,694.97333814)(26.4866285,695.00334076)
\curveto(26.48662736,695.04333807)(26.47662737,695.08833802)(26.4566285,695.13834076)
\curveto(26.37662747,695.41833769)(26.33662751,695.73333738)(26.3366285,696.08334076)
\lineto(26.3366285,697.13334076)
\lineto(26.3366285,701.31834076)
\lineto(26.3366285,702.36834076)
\lineto(26.3366285,702.65334076)
\curveto(26.33662751,702.75333036)(26.35162749,702.83333028)(26.3816285,702.89334076)
\curveto(26.4416274,702.96333015)(26.52162732,703.0133301)(26.6216285,703.04334076)
\curveto(26.6416272,703.04333007)(26.66162718,703.04333007)(26.6816285,703.04334076)
\curveto(26.70162714,703.04333007)(26.72162712,703.04833006)(26.7416285,703.05834076)
}
}
{
\newrgbcolor{curcolor}{0 0 0}
\pscustom[linestyle=none,fillstyle=solid,fillcolor=curcolor]
{
\newpath
\moveto(40.20014412,700.29834076)
\curveto(40.95013962,700.31833279)(41.60013897,700.23333288)(42.15014412,700.04334076)
\curveto(42.71013786,699.86333325)(43.13513744,699.54833356)(43.42514412,699.09834076)
\curveto(43.49513708,698.98833412)(43.55513702,698.87333424)(43.60514412,698.75334076)
\curveto(43.66513691,698.64333447)(43.71513686,698.51833459)(43.75514412,698.37834076)
\curveto(43.7751368,698.31833479)(43.78513679,698.25333486)(43.78514412,698.18334076)
\curveto(43.78513679,698.113335)(43.7751368,698.05333506)(43.75514412,698.00334076)
\curveto(43.71513686,697.94333517)(43.66013691,697.90333521)(43.59014412,697.88334076)
\curveto(43.54013703,697.86333525)(43.48013709,697.85333526)(43.41014412,697.85334076)
\lineto(43.20014412,697.85334076)
\lineto(42.54014412,697.85334076)
\curveto(42.4701381,697.85333526)(42.40013817,697.84833526)(42.33014412,697.83834076)
\curveto(42.26013831,697.83833527)(42.19513838,697.84833526)(42.13514412,697.86834076)
\curveto(42.03513854,697.88833522)(41.96013861,697.92833518)(41.91014412,697.98834076)
\curveto(41.86013871,698.04833506)(41.81513876,698.108335)(41.77514412,698.16834076)
\lineto(41.65514412,698.37834076)
\curveto(41.62513895,698.45833465)(41.575139,698.52333459)(41.50514412,698.57334076)
\curveto(41.40513917,698.65333446)(41.30513927,698.7133344)(41.20514412,698.75334076)
\curveto(41.11513946,698.79333432)(41.00013957,698.82833428)(40.86014412,698.85834076)
\curveto(40.79013978,698.87833423)(40.68513989,698.89333422)(40.54514412,698.90334076)
\curveto(40.41514016,698.9133342)(40.31514026,698.9083342)(40.24514412,698.88834076)
\lineto(40.14014412,698.88834076)
\lineto(39.99014412,698.85834076)
\curveto(39.95014062,698.85833425)(39.90514067,698.85333426)(39.85514412,698.84334076)
\curveto(39.68514089,698.79333432)(39.54514103,698.72333439)(39.43514412,698.63334076)
\curveto(39.33514124,698.55333456)(39.26514131,698.42833468)(39.22514412,698.25834076)
\curveto(39.20514137,698.18833492)(39.20514137,698.12333499)(39.22514412,698.06334076)
\curveto(39.24514133,698.00333511)(39.26514131,697.95333516)(39.28514412,697.91334076)
\curveto(39.35514122,697.79333532)(39.43514114,697.69833541)(39.52514412,697.62834076)
\curveto(39.62514095,697.55833555)(39.74014083,697.49833561)(39.87014412,697.44834076)
\curveto(40.06014051,697.36833574)(40.26514031,697.29833581)(40.48514412,697.23834076)
\lineto(41.17514412,697.08834076)
\curveto(41.41513916,697.04833606)(41.64513893,696.99833611)(41.86514412,696.93834076)
\curveto(42.09513848,696.88833622)(42.31013826,696.82333629)(42.51014412,696.74334076)
\curveto(42.60013797,696.70333641)(42.68513789,696.66833644)(42.76514412,696.63834076)
\curveto(42.85513772,696.61833649)(42.94013763,696.58333653)(43.02014412,696.53334076)
\curveto(43.21013736,696.4133367)(43.38013719,696.28333683)(43.53014412,696.14334076)
\curveto(43.69013688,696.00333711)(43.81513676,695.82833728)(43.90514412,695.61834076)
\curveto(43.93513664,695.54833756)(43.96013661,695.47833763)(43.98014412,695.40834076)
\curveto(44.00013657,695.33833777)(44.02013655,695.26333785)(44.04014412,695.18334076)
\curveto(44.05013652,695.12333799)(44.05513652,695.02833808)(44.05514412,694.89834076)
\curveto(44.06513651,694.77833833)(44.06513651,694.68333843)(44.05514412,694.61334076)
\lineto(44.05514412,694.53834076)
\curveto(44.03513654,694.47833863)(44.02013655,694.41833869)(44.01014412,694.35834076)
\curveto(44.01013656,694.3083388)(44.00513657,694.25833885)(43.99514412,694.20834076)
\curveto(43.92513665,693.9083392)(43.81513676,693.64333947)(43.66514412,693.41334076)
\curveto(43.50513707,693.17333994)(43.31013726,692.97834013)(43.08014412,692.82834076)
\curveto(42.85013772,692.67834043)(42.59013798,692.54834056)(42.30014412,692.43834076)
\curveto(42.19013838,692.38834072)(42.0701385,692.35334076)(41.94014412,692.33334076)
\curveto(41.82013875,692.3133408)(41.70013887,692.28834082)(41.58014412,692.25834076)
\curveto(41.49013908,692.23834087)(41.39513918,692.22834088)(41.29514412,692.22834076)
\curveto(41.20513937,692.21834089)(41.11513946,692.20334091)(41.02514412,692.18334076)
\lineto(40.75514412,692.18334076)
\curveto(40.69513988,692.16334095)(40.59013998,692.15334096)(40.44014412,692.15334076)
\curveto(40.30014027,692.15334096)(40.20014037,692.16334095)(40.14014412,692.18334076)
\curveto(40.11014046,692.18334093)(40.0751405,692.18834092)(40.03514412,692.19834076)
\lineto(39.93014412,692.19834076)
\curveto(39.81014076,692.21834089)(39.69014088,692.23334088)(39.57014412,692.24334076)
\curveto(39.45014112,692.25334086)(39.33514124,692.27334084)(39.22514412,692.30334076)
\curveto(38.83514174,692.4133407)(38.49014208,692.53834057)(38.19014412,692.67834076)
\curveto(37.89014268,692.82834028)(37.63514294,693.04834006)(37.42514412,693.33834076)
\curveto(37.28514329,693.52833958)(37.16514341,693.74833936)(37.06514412,693.99834076)
\curveto(37.04514353,694.05833905)(37.02514355,694.13833897)(37.00514412,694.23834076)
\curveto(36.98514359,694.28833882)(36.9701436,694.35833875)(36.96014412,694.44834076)
\curveto(36.95014362,694.53833857)(36.95514362,694.6133385)(36.97514412,694.67334076)
\curveto(37.00514357,694.74333837)(37.05514352,694.79333832)(37.12514412,694.82334076)
\curveto(37.1751434,694.84333827)(37.23514334,694.85333826)(37.30514412,694.85334076)
\lineto(37.53014412,694.85334076)
\lineto(38.23514412,694.85334076)
\lineto(38.47514412,694.85334076)
\curveto(38.55514202,694.85333826)(38.62514195,694.84333827)(38.68514412,694.82334076)
\curveto(38.79514178,694.78333833)(38.86514171,694.71833839)(38.89514412,694.62834076)
\curveto(38.93514164,694.53833857)(38.98014159,694.44333867)(39.03014412,694.34334076)
\curveto(39.05014152,694.29333882)(39.08514149,694.22833888)(39.13514412,694.14834076)
\curveto(39.19514138,694.06833904)(39.24514133,694.01833909)(39.28514412,693.99834076)
\curveto(39.40514117,693.89833921)(39.52014105,693.81833929)(39.63014412,693.75834076)
\curveto(39.74014083,693.7083394)(39.88014069,693.65833945)(40.05014412,693.60834076)
\curveto(40.10014047,693.58833952)(40.15014042,693.57833953)(40.20014412,693.57834076)
\curveto(40.25014032,693.58833952)(40.30014027,693.58833952)(40.35014412,693.57834076)
\curveto(40.43014014,693.55833955)(40.51514006,693.54833956)(40.60514412,693.54834076)
\curveto(40.70513987,693.55833955)(40.79013978,693.57333954)(40.86014412,693.59334076)
\curveto(40.91013966,693.60333951)(40.95513962,693.6083395)(40.99514412,693.60834076)
\curveto(41.04513953,693.6083395)(41.09513948,693.61833949)(41.14514412,693.63834076)
\curveto(41.28513929,693.68833942)(41.41013916,693.74833936)(41.52014412,693.81834076)
\curveto(41.64013893,693.88833922)(41.73513884,693.97833913)(41.80514412,694.08834076)
\curveto(41.85513872,694.16833894)(41.89513868,694.29333882)(41.92514412,694.46334076)
\curveto(41.94513863,694.53333858)(41.94513863,694.59833851)(41.92514412,694.65834076)
\curveto(41.90513867,694.71833839)(41.88513869,694.76833834)(41.86514412,694.80834076)
\curveto(41.79513878,694.94833816)(41.70513887,695.05333806)(41.59514412,695.12334076)
\curveto(41.49513908,695.19333792)(41.3751392,695.25833785)(41.23514412,695.31834076)
\curveto(41.04513953,695.39833771)(40.84513973,695.46333765)(40.63514412,695.51334076)
\curveto(40.42514015,695.56333755)(40.21514036,695.61833749)(40.00514412,695.67834076)
\curveto(39.92514065,695.69833741)(39.84014073,695.7133374)(39.75014412,695.72334076)
\curveto(39.6701409,695.73333738)(39.59014098,695.74833736)(39.51014412,695.76834076)
\curveto(39.19014138,695.85833725)(38.88514169,695.94333717)(38.59514412,696.02334076)
\curveto(38.30514227,696.113337)(38.04014253,696.24333687)(37.80014412,696.41334076)
\curveto(37.52014305,696.6133365)(37.31514326,696.88333623)(37.18514412,697.22334076)
\curveto(37.16514341,697.29333582)(37.14514343,697.38833572)(37.12514412,697.50834076)
\curveto(37.10514347,697.57833553)(37.09014348,697.66333545)(37.08014412,697.76334076)
\curveto(37.0701435,697.86333525)(37.0751435,697.95333516)(37.09514412,698.03334076)
\curveto(37.11514346,698.08333503)(37.12014345,698.12333499)(37.11014412,698.15334076)
\curveto(37.10014347,698.19333492)(37.10514347,698.23833487)(37.12514412,698.28834076)
\curveto(37.14514343,698.39833471)(37.16514341,698.49833461)(37.18514412,698.58834076)
\curveto(37.21514336,698.68833442)(37.25014332,698.78333433)(37.29014412,698.87334076)
\curveto(37.42014315,699.16333395)(37.60014297,699.39833371)(37.83014412,699.57834076)
\curveto(38.06014251,699.75833335)(38.32014225,699.90333321)(38.61014412,700.01334076)
\curveto(38.72014185,700.06333305)(38.83514174,700.09833301)(38.95514412,700.11834076)
\curveto(39.0751415,700.14833296)(39.20014137,700.17833293)(39.33014412,700.20834076)
\curveto(39.39014118,700.22833288)(39.45014112,700.23833287)(39.51014412,700.23834076)
\lineto(39.69014412,700.26834076)
\curveto(39.7701408,700.27833283)(39.85514072,700.28333283)(39.94514412,700.28334076)
\curveto(40.03514054,700.28333283)(40.12014045,700.28833282)(40.20014412,700.29834076)
}
}
{
\newrgbcolor{curcolor}{0 0 0}
\pscustom[linestyle=none,fillstyle=solid,fillcolor=curcolor]
{
\newpath
\moveto(45.70678475,700.07334076)
\lineto(46.83178475,700.07334076)
\curveto(46.94178231,700.07333304)(47.04178221,700.06833304)(47.13178475,700.05834076)
\curveto(47.22178203,700.04833306)(47.28678197,700.0133331)(47.32678475,699.95334076)
\curveto(47.37678188,699.89333322)(47.40678185,699.8083333)(47.41678475,699.69834076)
\curveto(47.42678183,699.59833351)(47.43178182,699.49333362)(47.43178475,699.38334076)
\lineto(47.43178475,698.33334076)
\lineto(47.43178475,696.09834076)
\curveto(47.43178182,695.73833737)(47.44678181,695.39833771)(47.47678475,695.07834076)
\curveto(47.50678175,694.75833835)(47.59678166,694.49333862)(47.74678475,694.28334076)
\curveto(47.88678137,694.07333904)(48.11178114,693.92333919)(48.42178475,693.83334076)
\curveto(48.47178078,693.82333929)(48.51178074,693.81833929)(48.54178475,693.81834076)
\curveto(48.58178067,693.81833929)(48.62678063,693.8133393)(48.67678475,693.80334076)
\curveto(48.72678053,693.79333932)(48.78178047,693.78833932)(48.84178475,693.78834076)
\curveto(48.90178035,693.78833932)(48.94678031,693.79333932)(48.97678475,693.80334076)
\curveto(49.02678023,693.82333929)(49.06678019,693.82833928)(49.09678475,693.81834076)
\curveto(49.13678012,693.8083393)(49.17678008,693.8133393)(49.21678475,693.83334076)
\curveto(49.42677983,693.88333923)(49.59177966,693.94833916)(49.71178475,694.02834076)
\curveto(49.89177936,694.13833897)(50.03177922,694.27833883)(50.13178475,694.44834076)
\curveto(50.24177901,694.62833848)(50.31677894,694.82333829)(50.35678475,695.03334076)
\curveto(50.40677885,695.25333786)(50.43677882,695.49333762)(50.44678475,695.75334076)
\curveto(50.4567788,696.02333709)(50.46177879,696.30333681)(50.46178475,696.59334076)
\lineto(50.46178475,698.40834076)
\lineto(50.46178475,699.38334076)
\lineto(50.46178475,699.65334076)
\curveto(50.46177879,699.75333336)(50.48177877,699.83333328)(50.52178475,699.89334076)
\curveto(50.57177868,699.98333313)(50.64677861,700.03333308)(50.74678475,700.04334076)
\curveto(50.84677841,700.06333305)(50.96677829,700.07333304)(51.10678475,700.07334076)
\lineto(51.90178475,700.07334076)
\lineto(52.18678475,700.07334076)
\curveto(52.27677698,700.07333304)(52.3517769,700.05333306)(52.41178475,700.01334076)
\curveto(52.49177676,699.96333315)(52.53677672,699.88833322)(52.54678475,699.78834076)
\curveto(52.5567767,699.68833342)(52.56177669,699.57333354)(52.56178475,699.44334076)
\lineto(52.56178475,698.30334076)
\lineto(52.56178475,694.08834076)
\lineto(52.56178475,693.02334076)
\lineto(52.56178475,692.72334076)
\curveto(52.56177669,692.62334049)(52.54177671,692.54834056)(52.50178475,692.49834076)
\curveto(52.4517768,692.41834069)(52.37677688,692.37334074)(52.27678475,692.36334076)
\curveto(52.17677708,692.35334076)(52.07177718,692.34834076)(51.96178475,692.34834076)
\lineto(51.15178475,692.34834076)
\curveto(51.04177821,692.34834076)(50.94177831,692.35334076)(50.85178475,692.36334076)
\curveto(50.77177848,692.37334074)(50.70677855,692.4133407)(50.65678475,692.48334076)
\curveto(50.63677862,692.5133406)(50.61677864,692.55834055)(50.59678475,692.61834076)
\curveto(50.58677867,692.67834043)(50.57177868,692.73834037)(50.55178475,692.79834076)
\curveto(50.54177871,692.85834025)(50.52677873,692.9133402)(50.50678475,692.96334076)
\curveto(50.48677877,693.0133401)(50.4567788,693.04334007)(50.41678475,693.05334076)
\curveto(50.39677886,693.07334004)(50.37177888,693.07834003)(50.34178475,693.06834076)
\curveto(50.31177894,693.05834005)(50.28677897,693.04834006)(50.26678475,693.03834076)
\curveto(50.19677906,692.99834011)(50.13677912,692.95334016)(50.08678475,692.90334076)
\curveto(50.03677922,692.85334026)(49.98177927,692.8083403)(49.92178475,692.76834076)
\curveto(49.88177937,692.73834037)(49.84177941,692.70334041)(49.80178475,692.66334076)
\curveto(49.77177948,692.63334048)(49.73177952,692.60334051)(49.68178475,692.57334076)
\curveto(49.4517798,692.43334068)(49.18178007,692.32334079)(48.87178475,692.24334076)
\curveto(48.80178045,692.22334089)(48.73178052,692.2133409)(48.66178475,692.21334076)
\curveto(48.59178066,692.20334091)(48.51678074,692.18834092)(48.43678475,692.16834076)
\curveto(48.39678086,692.15834095)(48.3517809,692.15834095)(48.30178475,692.16834076)
\curveto(48.26178099,692.16834094)(48.22178103,692.16334095)(48.18178475,692.15334076)
\curveto(48.1517811,692.14334097)(48.08678117,692.14334097)(47.98678475,692.15334076)
\curveto(47.89678136,692.15334096)(47.83678142,692.15834095)(47.80678475,692.16834076)
\curveto(47.7567815,692.16834094)(47.70678155,692.17334094)(47.65678475,692.18334076)
\lineto(47.50678475,692.18334076)
\curveto(47.38678187,692.2133409)(47.27178198,692.23834087)(47.16178475,692.25834076)
\curveto(47.0517822,692.27834083)(46.94178231,692.3083408)(46.83178475,692.34834076)
\curveto(46.78178247,692.36834074)(46.73678252,692.38334073)(46.69678475,692.39334076)
\curveto(46.66678259,692.4133407)(46.62678263,692.43334068)(46.57678475,692.45334076)
\curveto(46.22678303,692.64334047)(45.94678331,692.9083402)(45.73678475,693.24834076)
\curveto(45.60678365,693.45833965)(45.51178374,693.7083394)(45.45178475,693.99834076)
\curveto(45.39178386,694.29833881)(45.3517839,694.6133385)(45.33178475,694.94334076)
\curveto(45.32178393,695.28333783)(45.31678394,695.62833748)(45.31678475,695.97834076)
\curveto(45.32678393,696.33833677)(45.33178392,696.69333642)(45.33178475,697.04334076)
\lineto(45.33178475,699.08334076)
\curveto(45.33178392,699.2133339)(45.32678393,699.36333375)(45.31678475,699.53334076)
\curveto(45.31678394,699.7133334)(45.34178391,699.84333327)(45.39178475,699.92334076)
\curveto(45.42178383,699.97333314)(45.48178377,700.01833309)(45.57178475,700.05834076)
\curveto(45.63178362,700.05833305)(45.67678358,700.06333305)(45.70678475,700.07334076)
}
}
{
\newrgbcolor{curcolor}{0 0 0}
\pscustom[linestyle=none,fillstyle=solid,fillcolor=curcolor]
{
\newpath
\moveto(61.24303475,692.94834076)
\curveto(61.2630269,692.83834027)(61.27302689,692.72834038)(61.27303475,692.61834076)
\curveto(61.28302688,692.5083406)(61.23302693,692.43334068)(61.12303475,692.39334076)
\curveto(61.0630271,692.36334075)(60.99302717,692.34834076)(60.91303475,692.34834076)
\lineto(60.67303475,692.34834076)
\lineto(59.86303475,692.34834076)
\lineto(59.59303475,692.34834076)
\curveto(59.51302865,692.35834075)(59.44802871,692.38334073)(59.39803475,692.42334076)
\curveto(59.32802883,692.46334065)(59.27302889,692.51834059)(59.23303475,692.58834076)
\curveto(59.20302896,692.66834044)(59.158029,692.73334038)(59.09803475,692.78334076)
\curveto(59.07802908,692.80334031)(59.05302911,692.81834029)(59.02303475,692.82834076)
\curveto(58.99302917,692.84834026)(58.95302921,692.85334026)(58.90303475,692.84334076)
\curveto(58.85302931,692.82334029)(58.80302936,692.79834031)(58.75303475,692.76834076)
\curveto(58.71302945,692.73834037)(58.66802949,692.7133404)(58.61803475,692.69334076)
\curveto(58.56802959,692.65334046)(58.51302965,692.61834049)(58.45303475,692.58834076)
\lineto(58.27303475,692.49834076)
\curveto(58.14303002,692.43834067)(58.00803015,692.38834072)(57.86803475,692.34834076)
\curveto(57.72803043,692.31834079)(57.58303058,692.28334083)(57.43303475,692.24334076)
\curveto(57.3630308,692.22334089)(57.29303087,692.2133409)(57.22303475,692.21334076)
\curveto(57.163031,692.20334091)(57.09803106,692.19334092)(57.02803475,692.18334076)
\lineto(56.93803475,692.18334076)
\curveto(56.90803125,692.17334094)(56.87803128,692.16834094)(56.84803475,692.16834076)
\lineto(56.68303475,692.16834076)
\curveto(56.58303158,692.14834096)(56.48303168,692.14834096)(56.38303475,692.16834076)
\lineto(56.24803475,692.16834076)
\curveto(56.17803198,692.18834092)(56.10803205,692.19834091)(56.03803475,692.19834076)
\curveto(55.97803218,692.18834092)(55.91803224,692.19334092)(55.85803475,692.21334076)
\curveto(55.7580324,692.23334088)(55.6630325,692.25334086)(55.57303475,692.27334076)
\curveto(55.48303268,692.28334083)(55.39803276,692.3083408)(55.31803475,692.34834076)
\curveto(55.02803313,692.45834065)(54.77803338,692.59834051)(54.56803475,692.76834076)
\curveto(54.36803379,692.94834016)(54.20803395,693.18333993)(54.08803475,693.47334076)
\curveto(54.0580341,693.54333957)(54.02803413,693.61833949)(53.99803475,693.69834076)
\curveto(53.97803418,693.77833933)(53.9580342,693.86333925)(53.93803475,693.95334076)
\curveto(53.91803424,694.00333911)(53.90803425,694.05333906)(53.90803475,694.10334076)
\curveto(53.91803424,694.15333896)(53.91803424,694.20333891)(53.90803475,694.25334076)
\curveto(53.89803426,694.28333883)(53.88803427,694.34333877)(53.87803475,694.43334076)
\curveto(53.87803428,694.53333858)(53.88303428,694.60333851)(53.89303475,694.64334076)
\curveto(53.91303425,694.74333837)(53.92303424,694.82833828)(53.92303475,694.89834076)
\lineto(54.01303475,695.22834076)
\curveto(54.04303412,695.34833776)(54.08303408,695.45333766)(54.13303475,695.54334076)
\curveto(54.30303386,695.83333728)(54.49803366,696.05333706)(54.71803475,696.20334076)
\curveto(54.93803322,696.35333676)(55.21803294,696.48333663)(55.55803475,696.59334076)
\curveto(55.68803247,696.64333647)(55.82303234,696.67833643)(55.96303475,696.69834076)
\curveto(56.10303206,696.71833639)(56.24303192,696.74333637)(56.38303475,696.77334076)
\curveto(56.4630317,696.79333632)(56.54803161,696.80333631)(56.63803475,696.80334076)
\curveto(56.72803143,696.8133363)(56.81803134,696.82833628)(56.90803475,696.84834076)
\curveto(56.97803118,696.86833624)(57.04803111,696.87333624)(57.11803475,696.86334076)
\curveto(57.18803097,696.86333625)(57.2630309,696.87333624)(57.34303475,696.89334076)
\curveto(57.41303075,696.9133362)(57.48303068,696.92333619)(57.55303475,696.92334076)
\curveto(57.62303054,696.92333619)(57.69803046,696.93333618)(57.77803475,696.95334076)
\curveto(57.98803017,697.00333611)(58.17802998,697.04333607)(58.34803475,697.07334076)
\curveto(58.52802963,697.113336)(58.68802947,697.20333591)(58.82803475,697.34334076)
\curveto(58.91802924,697.43333568)(58.97802918,697.53333558)(59.00803475,697.64334076)
\curveto(59.01802914,697.67333544)(59.01802914,697.69833541)(59.00803475,697.71834076)
\curveto(59.00802915,697.73833537)(59.01302915,697.75833535)(59.02303475,697.77834076)
\curveto(59.03302913,697.79833531)(59.03802912,697.82833528)(59.03803475,697.86834076)
\lineto(59.03803475,697.95834076)
\lineto(59.00803475,698.07834076)
\curveto(59.00802915,698.11833499)(59.00302916,698.15333496)(58.99303475,698.18334076)
\curveto(58.89302927,698.48333463)(58.68302948,698.68833442)(58.36303475,698.79834076)
\curveto(58.27302989,698.82833428)(58.16303,698.84833426)(58.03303475,698.85834076)
\curveto(57.91303025,698.87833423)(57.78803037,698.88333423)(57.65803475,698.87334076)
\curveto(57.52803063,698.87333424)(57.40303076,698.86333425)(57.28303475,698.84334076)
\curveto(57.163031,698.82333429)(57.0580311,698.79833431)(56.96803475,698.76834076)
\curveto(56.90803125,698.74833436)(56.84803131,698.71833439)(56.78803475,698.67834076)
\curveto(56.73803142,698.64833446)(56.68803147,698.6133345)(56.63803475,698.57334076)
\curveto(56.58803157,698.53333458)(56.53303163,698.47833463)(56.47303475,698.40834076)
\curveto(56.42303174,698.33833477)(56.38803177,698.27333484)(56.36803475,698.21334076)
\curveto(56.31803184,698.113335)(56.27303189,698.01833509)(56.23303475,697.92834076)
\curveto(56.20303196,697.83833527)(56.13303203,697.77833533)(56.02303475,697.74834076)
\curveto(55.94303222,697.72833538)(55.8580323,697.71833539)(55.76803475,697.71834076)
\lineto(55.49803475,697.71834076)
\lineto(54.92803475,697.71834076)
\curveto(54.87803328,697.71833539)(54.82803333,697.7133354)(54.77803475,697.70334076)
\curveto(54.72803343,697.70333541)(54.68303348,697.7083354)(54.64303475,697.71834076)
\lineto(54.50803475,697.71834076)
\curveto(54.48803367,697.72833538)(54.4630337,697.73333538)(54.43303475,697.73334076)
\curveto(54.40303376,697.73333538)(54.37803378,697.74333537)(54.35803475,697.76334076)
\curveto(54.27803388,697.78333533)(54.22303394,697.84833526)(54.19303475,697.95834076)
\curveto(54.18303398,698.0083351)(54.18303398,698.05833505)(54.19303475,698.10834076)
\curveto(54.20303396,698.15833495)(54.21303395,698.20333491)(54.22303475,698.24334076)
\curveto(54.25303391,698.35333476)(54.28303388,698.45333466)(54.31303475,698.54334076)
\curveto(54.35303381,698.64333447)(54.39803376,698.73333438)(54.44803475,698.81334076)
\lineto(54.53803475,698.96334076)
\lineto(54.62803475,699.11334076)
\curveto(54.70803345,699.22333389)(54.80803335,699.32833378)(54.92803475,699.42834076)
\curveto(54.94803321,699.43833367)(54.97803318,699.46333365)(55.01803475,699.50334076)
\curveto(55.06803309,699.54333357)(55.11303305,699.57833353)(55.15303475,699.60834076)
\curveto(55.19303297,699.63833347)(55.23803292,699.66833344)(55.28803475,699.69834076)
\curveto(55.4580327,699.8083333)(55.63803252,699.89333322)(55.82803475,699.95334076)
\curveto(56.01803214,700.02333309)(56.21303195,700.08833302)(56.41303475,700.14834076)
\curveto(56.53303163,700.17833293)(56.6580315,700.19833291)(56.78803475,700.20834076)
\curveto(56.91803124,700.21833289)(57.04803111,700.23833287)(57.17803475,700.26834076)
\curveto(57.21803094,700.27833283)(57.27803088,700.27833283)(57.35803475,700.26834076)
\curveto(57.44803071,700.25833285)(57.50303066,700.26333285)(57.52303475,700.28334076)
\curveto(57.93303023,700.29333282)(58.32302984,700.27833283)(58.69303475,700.23834076)
\curveto(59.07302909,700.19833291)(59.41302875,700.12333299)(59.71303475,700.01334076)
\curveto(60.02302814,699.90333321)(60.28802787,699.75333336)(60.50803475,699.56334076)
\curveto(60.72802743,699.38333373)(60.89802726,699.14833396)(61.01803475,698.85834076)
\curveto(61.08802707,698.68833442)(61.12802703,698.49333462)(61.13803475,698.27334076)
\curveto(61.14802701,698.05333506)(61.15302701,697.82833528)(61.15303475,697.59834076)
\lineto(61.15303475,694.25334076)
\lineto(61.15303475,693.66834076)
\curveto(61.15302701,693.47833963)(61.17302699,693.30333981)(61.21303475,693.14334076)
\curveto(61.22302694,693.11334)(61.22802693,693.07834003)(61.22803475,693.03834076)
\curveto(61.22802693,693.0083401)(61.23302693,692.97834013)(61.24303475,692.94834076)
\moveto(59.03803475,695.25834076)
\curveto(59.04802911,695.3083378)(59.05302911,695.36333775)(59.05303475,695.42334076)
\curveto(59.05302911,695.49333762)(59.04802911,695.55333756)(59.03803475,695.60334076)
\curveto(59.01802914,695.66333745)(59.00802915,695.71833739)(59.00803475,695.76834076)
\curveto(59.00802915,695.81833729)(58.98802917,695.85833725)(58.94803475,695.88834076)
\curveto(58.89802926,695.92833718)(58.82302934,695.94833716)(58.72303475,695.94834076)
\curveto(58.68302948,695.93833717)(58.64802951,695.92833718)(58.61803475,695.91834076)
\curveto(58.58802957,695.91833719)(58.55302961,695.9133372)(58.51303475,695.90334076)
\curveto(58.44302972,695.88333723)(58.36802979,695.86833724)(58.28803475,695.85834076)
\curveto(58.20802995,695.84833726)(58.12803003,695.83333728)(58.04803475,695.81334076)
\curveto(58.01803014,695.80333731)(57.97303019,695.79833731)(57.91303475,695.79834076)
\curveto(57.78303038,695.76833734)(57.65303051,695.74833736)(57.52303475,695.73834076)
\curveto(57.39303077,695.72833738)(57.26803089,695.70333741)(57.14803475,695.66334076)
\curveto(57.06803109,695.64333747)(56.99303117,695.62333749)(56.92303475,695.60334076)
\curveto(56.85303131,695.59333752)(56.78303138,695.57333754)(56.71303475,695.54334076)
\curveto(56.50303166,695.45333766)(56.32303184,695.31833779)(56.17303475,695.13834076)
\curveto(56.03303213,694.95833815)(55.98303218,694.7083384)(56.02303475,694.38834076)
\curveto(56.04303212,694.21833889)(56.09803206,694.07833903)(56.18803475,693.96834076)
\curveto(56.2580319,693.85833925)(56.3630318,693.76833934)(56.50303475,693.69834076)
\curveto(56.64303152,693.63833947)(56.79303137,693.59333952)(56.95303475,693.56334076)
\curveto(57.12303104,693.53333958)(57.29803086,693.52333959)(57.47803475,693.53334076)
\curveto(57.66803049,693.55333956)(57.84303032,693.58833952)(58.00303475,693.63834076)
\curveto(58.2630299,693.71833939)(58.46802969,693.84333927)(58.61803475,694.01334076)
\curveto(58.76802939,694.19333892)(58.88302928,694.4133387)(58.96303475,694.67334076)
\curveto(58.98302918,694.74333837)(58.99302917,694.8133383)(58.99303475,694.88334076)
\curveto(59.00302916,694.96333815)(59.01802914,695.04333807)(59.03803475,695.12334076)
\lineto(59.03803475,695.25834076)
}
}
{
\newrgbcolor{curcolor}{0 0 0}
\pscustom[linestyle=none,fillstyle=solid,fillcolor=curcolor]
{
\newpath
\moveto(67.231316,700.28334076)
\curveto(67.34131068,700.28333283)(67.43631059,700.27333284)(67.516316,700.25334076)
\curveto(67.60631042,700.23333288)(67.67631035,700.18833292)(67.726316,700.11834076)
\curveto(67.78631024,700.03833307)(67.81631021,699.89833321)(67.816316,699.69834076)
\lineto(67.816316,699.18834076)
\lineto(67.816316,698.81334076)
\curveto(67.8263102,698.67333444)(67.81131021,698.56333455)(67.771316,698.48334076)
\curveto(67.73131029,698.4133347)(67.67131035,698.36833474)(67.591316,698.34834076)
\curveto(67.5213105,698.32833478)(67.43631059,698.31833479)(67.336316,698.31834076)
\curveto(67.24631078,698.31833479)(67.14631088,698.32333479)(67.036316,698.33334076)
\curveto(66.93631109,698.34333477)(66.84131118,698.33833477)(66.751316,698.31834076)
\curveto(66.68131134,698.29833481)(66.61131141,698.28333483)(66.541316,698.27334076)
\curveto(66.47131155,698.27333484)(66.40631162,698.26333485)(66.346316,698.24334076)
\curveto(66.18631184,698.19333492)(66.026312,698.11833499)(65.866316,698.01834076)
\curveto(65.70631232,697.92833518)(65.58131244,697.82333529)(65.491316,697.70334076)
\curveto(65.44131258,697.62333549)(65.38631264,697.53833557)(65.326316,697.44834076)
\curveto(65.27631275,697.36833574)(65.2263128,697.28333583)(65.176316,697.19334076)
\curveto(65.14631288,697.113336)(65.11631291,697.02833608)(65.086316,696.93834076)
\lineto(65.026316,696.69834076)
\curveto(65.00631302,696.62833648)(64.99631303,696.55333656)(64.996316,696.47334076)
\curveto(64.99631303,696.40333671)(64.98631304,696.33333678)(64.966316,696.26334076)
\curveto(64.95631307,696.22333689)(64.95131307,696.18333693)(64.951316,696.14334076)
\curveto(64.96131306,696.113337)(64.96131306,696.08333703)(64.951316,696.05334076)
\lineto(64.951316,695.81334076)
\curveto(64.93131309,695.74333737)(64.9263131,695.66333745)(64.936316,695.57334076)
\curveto(64.94631308,695.49333762)(64.95131307,695.4133377)(64.951316,695.33334076)
\lineto(64.951316,694.37334076)
\lineto(64.951316,693.09834076)
\curveto(64.95131307,692.96834014)(64.94631308,692.84834026)(64.936316,692.73834076)
\curveto(64.9263131,692.62834048)(64.89631313,692.53834057)(64.846316,692.46834076)
\curveto(64.8263132,692.43834067)(64.79131323,692.4133407)(64.741316,692.39334076)
\curveto(64.70131332,692.38334073)(64.65631337,692.37334074)(64.606316,692.36334076)
\lineto(64.531316,692.36334076)
\curveto(64.48131354,692.35334076)(64.4263136,692.34834076)(64.366316,692.34834076)
\lineto(64.201316,692.34834076)
\lineto(63.556316,692.34834076)
\curveto(63.49631453,692.35834075)(63.43131459,692.36334075)(63.361316,692.36334076)
\lineto(63.166316,692.36334076)
\curveto(63.11631491,692.38334073)(63.06631496,692.39834071)(63.016316,692.40834076)
\curveto(62.96631506,692.42834068)(62.93131509,692.46334065)(62.911316,692.51334076)
\curveto(62.87131515,692.56334055)(62.84631518,692.63334048)(62.836316,692.72334076)
\lineto(62.836316,693.02334076)
\lineto(62.836316,694.04334076)
\lineto(62.836316,698.27334076)
\lineto(62.836316,699.38334076)
\lineto(62.836316,699.66834076)
\curveto(62.83631519,699.76833334)(62.85631517,699.84833326)(62.896316,699.90834076)
\curveto(62.94631508,699.98833312)(63.021315,700.03833307)(63.121316,700.05834076)
\curveto(63.2213148,700.07833303)(63.34131468,700.08833302)(63.481316,700.08834076)
\lineto(64.246316,700.08834076)
\curveto(64.36631366,700.08833302)(64.47131355,700.07833303)(64.561316,700.05834076)
\curveto(64.65131337,700.04833306)(64.7213133,700.00333311)(64.771316,699.92334076)
\curveto(64.80131322,699.87333324)(64.81631321,699.80333331)(64.816316,699.71334076)
\lineto(64.846316,699.44334076)
\curveto(64.85631317,699.36333375)(64.87131315,699.28833382)(64.891316,699.21834076)
\curveto(64.9213131,699.14833396)(64.97131305,699.113334)(65.041316,699.11334076)
\curveto(65.06131296,699.13333398)(65.08131294,699.14333397)(65.101316,699.14334076)
\curveto(65.1213129,699.14333397)(65.14131288,699.15333396)(65.161316,699.17334076)
\curveto(65.2213128,699.22333389)(65.27131275,699.27833383)(65.311316,699.33834076)
\curveto(65.36131266,699.4083337)(65.4213126,699.46833364)(65.491316,699.51834076)
\curveto(65.53131249,699.54833356)(65.56631246,699.57833353)(65.596316,699.60834076)
\curveto(65.6263124,699.64833346)(65.66131236,699.68333343)(65.701316,699.71334076)
\lineto(65.971316,699.89334076)
\curveto(66.07131195,699.95333316)(66.17131185,700.0083331)(66.271316,700.05834076)
\curveto(66.37131165,700.09833301)(66.47131155,700.13333298)(66.571316,700.16334076)
\lineto(66.901316,700.25334076)
\curveto(66.93131109,700.26333285)(66.98631104,700.26333285)(67.066316,700.25334076)
\curveto(67.15631087,700.25333286)(67.21131081,700.26333285)(67.231316,700.28334076)
}
}
{
\newrgbcolor{curcolor}{0 0 0}
\pscustom[linestyle=none,fillstyle=solid,fillcolor=curcolor]
{
\newpath
\moveto(70.73639412,702.93834076)
\curveto(70.80639117,702.85833025)(70.84139114,702.73833037)(70.84139412,702.57834076)
\lineto(70.84139412,702.11334076)
\lineto(70.84139412,701.70834076)
\curveto(70.84139114,701.56833154)(70.80639117,701.47333164)(70.73639412,701.42334076)
\curveto(70.6763913,701.37333174)(70.59639138,701.34333177)(70.49639412,701.33334076)
\curveto(70.40639157,701.32333179)(70.30639167,701.31833179)(70.19639412,701.31834076)
\lineto(69.35639412,701.31834076)
\curveto(69.24639273,701.31833179)(69.14639283,701.32333179)(69.05639412,701.33334076)
\curveto(68.976393,701.34333177)(68.90639307,701.37333174)(68.84639412,701.42334076)
\curveto(68.80639317,701.45333166)(68.7763932,701.5083316)(68.75639412,701.58834076)
\curveto(68.74639323,701.67833143)(68.73639324,701.77333134)(68.72639412,701.87334076)
\lineto(68.72639412,702.20334076)
\curveto(68.73639324,702.3133308)(68.74139324,702.4083307)(68.74139412,702.48834076)
\lineto(68.74139412,702.69834076)
\curveto(68.75139323,702.76833034)(68.77139321,702.82833028)(68.80139412,702.87834076)
\curveto(68.82139316,702.91833019)(68.84639313,702.94833016)(68.87639412,702.96834076)
\lineto(68.99639412,703.02834076)
\curveto(69.01639296,703.02833008)(69.04139294,703.02833008)(69.07139412,703.02834076)
\curveto(69.10139288,703.03833007)(69.12639285,703.04333007)(69.14639412,703.04334076)
\lineto(70.24139412,703.04334076)
\curveto(70.34139164,703.04333007)(70.43639154,703.03833007)(70.52639412,703.02834076)
\curveto(70.61639136,703.01833009)(70.68639129,702.98833012)(70.73639412,702.93834076)
\moveto(70.84139412,693.17334076)
\curveto(70.84139114,692.97334014)(70.83639114,692.80334031)(70.82639412,692.66334076)
\curveto(70.81639116,692.52334059)(70.72639125,692.42834068)(70.55639412,692.37834076)
\curveto(70.49639148,692.35834075)(70.43139155,692.34834076)(70.36139412,692.34834076)
\curveto(70.29139169,692.35834075)(70.21639176,692.36334075)(70.13639412,692.36334076)
\lineto(69.29639412,692.36334076)
\curveto(69.20639277,692.36334075)(69.11639286,692.36834074)(69.02639412,692.37834076)
\curveto(68.94639303,692.38834072)(68.88639309,692.41834069)(68.84639412,692.46834076)
\curveto(68.78639319,692.53834057)(68.75139323,692.62334049)(68.74139412,692.72334076)
\lineto(68.74139412,693.06834076)
\lineto(68.74139412,699.39834076)
\lineto(68.74139412,699.69834076)
\curveto(68.74139324,699.79833331)(68.76139322,699.87833323)(68.80139412,699.93834076)
\curveto(68.86139312,700.0083331)(68.94639303,700.05333306)(69.05639412,700.07334076)
\curveto(69.0763929,700.08333303)(69.10139288,700.08333303)(69.13139412,700.07334076)
\curveto(69.17139281,700.07333304)(69.20139278,700.07833303)(69.22139412,700.08834076)
\lineto(69.97139412,700.08834076)
\lineto(70.16639412,700.08834076)
\curveto(70.24639173,700.09833301)(70.31139167,700.09833301)(70.36139412,700.08834076)
\lineto(70.48139412,700.08834076)
\curveto(70.54139144,700.06833304)(70.59639138,700.05333306)(70.64639412,700.04334076)
\curveto(70.69639128,700.03333308)(70.73639124,700.00333311)(70.76639412,699.95334076)
\curveto(70.80639117,699.90333321)(70.82639115,699.83333328)(70.82639412,699.74334076)
\curveto(70.83639114,699.65333346)(70.84139114,699.55833355)(70.84139412,699.45834076)
\lineto(70.84139412,693.17334076)
}
}
{
\newrgbcolor{curcolor}{0 0 0}
\pscustom[linestyle=none,fillstyle=solid,fillcolor=curcolor]
{
\newpath
\moveto(80.27358162,696.53334076)
\curveto(80.29357305,696.47333664)(80.30357304,696.38833672)(80.30358162,696.27834076)
\curveto(80.30357304,696.16833694)(80.29357305,696.08333703)(80.27358162,696.02334076)
\lineto(80.27358162,695.87334076)
\curveto(80.25357309,695.79333732)(80.2435731,695.7133374)(80.24358162,695.63334076)
\curveto(80.25357309,695.55333756)(80.2485731,695.47333764)(80.22858162,695.39334076)
\curveto(80.20857314,695.32333779)(80.19357315,695.25833785)(80.18358162,695.19834076)
\curveto(80.17357317,695.13833797)(80.16357318,695.07333804)(80.15358162,695.00334076)
\curveto(80.11357323,694.89333822)(80.07857327,694.77833833)(80.04858162,694.65834076)
\curveto(80.01857333,694.54833856)(79.97857337,694.44333867)(79.92858162,694.34334076)
\curveto(79.71857363,693.86333925)(79.4435739,693.47333964)(79.10358162,693.17334076)
\curveto(78.76357458,692.87334024)(78.35357499,692.62334049)(77.87358162,692.42334076)
\curveto(77.75357559,692.37334074)(77.62857572,692.33834077)(77.49858162,692.31834076)
\curveto(77.37857597,692.28834082)(77.25357609,692.25834085)(77.12358162,692.22834076)
\curveto(77.07357627,692.2083409)(77.01857633,692.19834091)(76.95858162,692.19834076)
\curveto(76.89857645,692.19834091)(76.8435765,692.19334092)(76.79358162,692.18334076)
\lineto(76.68858162,692.18334076)
\curveto(76.65857669,692.17334094)(76.62857672,692.16834094)(76.59858162,692.16834076)
\curveto(76.5485768,692.15834095)(76.46857688,692.15334096)(76.35858162,692.15334076)
\curveto(76.2485771,692.14334097)(76.16357718,692.14834096)(76.10358162,692.16834076)
\lineto(75.95358162,692.16834076)
\curveto(75.90357744,692.17834093)(75.8485775,692.18334093)(75.78858162,692.18334076)
\curveto(75.73857761,692.17334094)(75.68857766,692.17834093)(75.63858162,692.19834076)
\curveto(75.59857775,692.2083409)(75.55857779,692.2133409)(75.51858162,692.21334076)
\curveto(75.48857786,692.2133409)(75.4485779,692.21834089)(75.39858162,692.22834076)
\curveto(75.29857805,692.25834085)(75.19857815,692.28334083)(75.09858162,692.30334076)
\curveto(74.99857835,692.32334079)(74.90357844,692.35334076)(74.81358162,692.39334076)
\curveto(74.69357865,692.43334068)(74.57857877,692.47334064)(74.46858162,692.51334076)
\curveto(74.36857898,692.55334056)(74.26357908,692.60334051)(74.15358162,692.66334076)
\curveto(73.80357954,692.87334024)(73.50357984,693.11833999)(73.25358162,693.39834076)
\curveto(73.00358034,693.67833943)(72.79358055,694.0133391)(72.62358162,694.40334076)
\curveto(72.57358077,694.49333862)(72.53358081,694.58833852)(72.50358162,694.68834076)
\curveto(72.48358086,694.78833832)(72.45858089,694.89333822)(72.42858162,695.00334076)
\curveto(72.40858094,695.05333806)(72.39858095,695.09833801)(72.39858162,695.13834076)
\curveto(72.39858095,695.17833793)(72.38858096,695.22333789)(72.36858162,695.27334076)
\curveto(72.348581,695.35333776)(72.33858101,695.43333768)(72.33858162,695.51334076)
\curveto(72.33858101,695.60333751)(72.32858102,695.68833742)(72.30858162,695.76834076)
\curveto(72.29858105,695.81833729)(72.29358105,695.86333725)(72.29358162,695.90334076)
\lineto(72.29358162,696.03834076)
\curveto(72.27358107,696.09833701)(72.26358108,696.18333693)(72.26358162,696.29334076)
\curveto(72.27358107,696.40333671)(72.28858106,696.48833662)(72.30858162,696.54834076)
\lineto(72.30858162,696.65334076)
\curveto(72.31858103,696.70333641)(72.31858103,696.75333636)(72.30858162,696.80334076)
\curveto(72.30858104,696.86333625)(72.31858103,696.91833619)(72.33858162,696.96834076)
\curveto(72.348581,697.01833609)(72.35358099,697.06333605)(72.35358162,697.10334076)
\curveto(72.35358099,697.15333596)(72.36358098,697.20333591)(72.38358162,697.25334076)
\curveto(72.42358092,697.38333573)(72.45858089,697.5083356)(72.48858162,697.62834076)
\curveto(72.51858083,697.75833535)(72.55858079,697.88333523)(72.60858162,698.00334076)
\curveto(72.78858056,698.4133347)(73.00358034,698.75333436)(73.25358162,699.02334076)
\curveto(73.50357984,699.30333381)(73.80857954,699.55833355)(74.16858162,699.78834076)
\curveto(74.26857908,699.83833327)(74.37357897,699.88333323)(74.48358162,699.92334076)
\curveto(74.59357875,699.96333315)(74.70357864,700.0083331)(74.81358162,700.05834076)
\curveto(74.9435784,700.108333)(75.07857827,700.14333297)(75.21858162,700.16334076)
\curveto(75.35857799,700.18333293)(75.50357784,700.2133329)(75.65358162,700.25334076)
\curveto(75.73357761,700.26333285)(75.80857754,700.26833284)(75.87858162,700.26834076)
\curveto(75.9485774,700.26833284)(76.01857733,700.27333284)(76.08858162,700.28334076)
\curveto(76.66857668,700.29333282)(77.16857618,700.23333288)(77.58858162,700.10334076)
\curveto(78.01857533,699.97333314)(78.39857495,699.79333332)(78.72858162,699.56334076)
\curveto(78.83857451,699.48333363)(78.9485744,699.39333372)(79.05858162,699.29334076)
\curveto(79.17857417,699.20333391)(79.27857407,699.10333401)(79.35858162,698.99334076)
\curveto(79.43857391,698.89333422)(79.50857384,698.79333432)(79.56858162,698.69334076)
\curveto(79.63857371,698.59333452)(79.70857364,698.48833462)(79.77858162,698.37834076)
\curveto(79.8485735,698.26833484)(79.90357344,698.14833496)(79.94358162,698.01834076)
\curveto(79.98357336,697.89833521)(80.02857332,697.76833534)(80.07858162,697.62834076)
\curveto(80.10857324,697.54833556)(80.13357321,697.46333565)(80.15358162,697.37334076)
\lineto(80.21358162,697.10334076)
\curveto(80.22357312,697.06333605)(80.22857312,697.02333609)(80.22858162,696.98334076)
\curveto(80.22857312,696.94333617)(80.23357311,696.90333621)(80.24358162,696.86334076)
\curveto(80.26357308,696.8133363)(80.26857308,696.75833635)(80.25858162,696.69834076)
\curveto(80.2485731,696.63833647)(80.25357309,696.58333653)(80.27358162,696.53334076)
\moveto(78.17358162,695.99334076)
\curveto(78.18357516,696.04333707)(78.18857516,696.113337)(78.18858162,696.20334076)
\curveto(78.18857516,696.30333681)(78.18357516,696.37833673)(78.17358162,696.42834076)
\lineto(78.17358162,696.54834076)
\curveto(78.15357519,696.59833651)(78.1435752,696.65333646)(78.14358162,696.71334076)
\curveto(78.1435752,696.77333634)(78.13857521,696.82833628)(78.12858162,696.87834076)
\curveto(78.12857522,696.91833619)(78.12357522,696.94833616)(78.11358162,696.96834076)
\lineto(78.05358162,697.20834076)
\curveto(78.0435753,697.29833581)(78.02357532,697.38333573)(77.99358162,697.46334076)
\curveto(77.88357546,697.72333539)(77.75357559,697.94333517)(77.60358162,698.12334076)
\curveto(77.45357589,698.3133348)(77.25357609,698.46333465)(77.00358162,698.57334076)
\curveto(76.9435764,698.59333452)(76.88357646,698.6083345)(76.82358162,698.61834076)
\curveto(76.76357658,698.63833447)(76.69857665,698.65833445)(76.62858162,698.67834076)
\curveto(76.5485768,698.69833441)(76.46357688,698.70333441)(76.37358162,698.69334076)
\lineto(76.10358162,698.69334076)
\curveto(76.07357727,698.67333444)(76.03857731,698.66333445)(75.99858162,698.66334076)
\curveto(75.95857739,698.67333444)(75.92357742,698.67333444)(75.89358162,698.66334076)
\lineto(75.68358162,698.60334076)
\curveto(75.62357772,698.59333452)(75.56857778,698.57333454)(75.51858162,698.54334076)
\curveto(75.26857808,698.43333468)(75.06357828,698.27333484)(74.90358162,698.06334076)
\curveto(74.75357859,697.86333525)(74.63357871,697.62833548)(74.54358162,697.35834076)
\curveto(74.51357883,697.25833585)(74.48857886,697.15333596)(74.46858162,697.04334076)
\curveto(74.45857889,696.93333618)(74.4435789,696.82333629)(74.42358162,696.71334076)
\curveto(74.41357893,696.66333645)(74.40857894,696.6133365)(74.40858162,696.56334076)
\lineto(74.40858162,696.41334076)
\curveto(74.38857896,696.34333677)(74.37857897,696.23833687)(74.37858162,696.09834076)
\curveto(74.38857896,695.95833715)(74.40357894,695.85333726)(74.42358162,695.78334076)
\lineto(74.42358162,695.64834076)
\curveto(74.4435789,695.56833754)(74.45857889,695.48833762)(74.46858162,695.40834076)
\curveto(74.47857887,695.33833777)(74.49357885,695.26333785)(74.51358162,695.18334076)
\curveto(74.61357873,694.88333823)(74.71857863,694.63833847)(74.82858162,694.44834076)
\curveto(74.9485784,694.26833884)(75.13357821,694.10333901)(75.38358162,693.95334076)
\curveto(75.45357789,693.90333921)(75.52857782,693.86333925)(75.60858162,693.83334076)
\curveto(75.69857765,693.80333931)(75.78857756,693.77833933)(75.87858162,693.75834076)
\curveto(75.91857743,693.74833936)(75.95357739,693.74333937)(75.98358162,693.74334076)
\curveto(76.01357733,693.75333936)(76.0485773,693.75333936)(76.08858162,693.74334076)
\lineto(76.20858162,693.71334076)
\curveto(76.25857709,693.7133394)(76.30357704,693.71833939)(76.34358162,693.72834076)
\lineto(76.46358162,693.72834076)
\curveto(76.5435768,693.74833936)(76.62357672,693.76333935)(76.70358162,693.77334076)
\curveto(76.78357656,693.78333933)(76.85857649,693.80333931)(76.92858162,693.83334076)
\curveto(77.18857616,693.93333918)(77.39857595,694.06833904)(77.55858162,694.23834076)
\curveto(77.71857563,694.4083387)(77.85357549,694.61833849)(77.96358162,694.86834076)
\curveto(78.00357534,694.96833814)(78.03357531,695.06833804)(78.05358162,695.16834076)
\curveto(78.07357527,695.26833784)(78.09857525,695.37333774)(78.12858162,695.48334076)
\curveto(78.13857521,695.52333759)(78.1435752,695.55833755)(78.14358162,695.58834076)
\curveto(78.1435752,695.62833748)(78.1485752,695.66833744)(78.15858162,695.70834076)
\lineto(78.15858162,695.84334076)
\curveto(78.15857519,695.89333722)(78.16357518,695.94333717)(78.17358162,695.99334076)
}
}
{
\newrgbcolor{curcolor}{0 0 0}
\pscustom[linestyle=none,fillstyle=solid,fillcolor=curcolor]
{
\newpath
\moveto(84.6435035,700.29834076)
\curveto(85.393499,700.31833279)(86.04349835,700.23333288)(86.5935035,700.04334076)
\curveto(87.15349724,699.86333325)(87.57849681,699.54833356)(87.8685035,699.09834076)
\curveto(87.93849645,698.98833412)(87.99849639,698.87333424)(88.0485035,698.75334076)
\curveto(88.10849628,698.64333447)(88.15849623,698.51833459)(88.1985035,698.37834076)
\curveto(88.21849617,698.31833479)(88.22849616,698.25333486)(88.2285035,698.18334076)
\curveto(88.22849616,698.113335)(88.21849617,698.05333506)(88.1985035,698.00334076)
\curveto(88.15849623,697.94333517)(88.10349629,697.90333521)(88.0335035,697.88334076)
\curveto(87.98349641,697.86333525)(87.92349647,697.85333526)(87.8535035,697.85334076)
\lineto(87.6435035,697.85334076)
\lineto(86.9835035,697.85334076)
\curveto(86.91349748,697.85333526)(86.84349755,697.84833526)(86.7735035,697.83834076)
\curveto(86.70349769,697.83833527)(86.63849775,697.84833526)(86.5785035,697.86834076)
\curveto(86.47849791,697.88833522)(86.40349799,697.92833518)(86.3535035,697.98834076)
\curveto(86.30349809,698.04833506)(86.25849813,698.108335)(86.2185035,698.16834076)
\lineto(86.0985035,698.37834076)
\curveto(86.06849832,698.45833465)(86.01849837,698.52333459)(85.9485035,698.57334076)
\curveto(85.84849854,698.65333446)(85.74849864,698.7133344)(85.6485035,698.75334076)
\curveto(85.55849883,698.79333432)(85.44349895,698.82833428)(85.3035035,698.85834076)
\curveto(85.23349916,698.87833423)(85.12849926,698.89333422)(84.9885035,698.90334076)
\curveto(84.85849953,698.9133342)(84.75849963,698.9083342)(84.6885035,698.88834076)
\lineto(84.5835035,698.88834076)
\lineto(84.4335035,698.85834076)
\curveto(84.3935,698.85833425)(84.34850004,698.85333426)(84.2985035,698.84334076)
\curveto(84.12850026,698.79333432)(83.9885004,698.72333439)(83.8785035,698.63334076)
\curveto(83.77850061,698.55333456)(83.70850068,698.42833468)(83.6685035,698.25834076)
\curveto(83.64850074,698.18833492)(83.64850074,698.12333499)(83.6685035,698.06334076)
\curveto(83.6885007,698.00333511)(83.70850068,697.95333516)(83.7285035,697.91334076)
\curveto(83.79850059,697.79333532)(83.87850051,697.69833541)(83.9685035,697.62834076)
\curveto(84.06850032,697.55833555)(84.18350021,697.49833561)(84.3135035,697.44834076)
\curveto(84.50349989,697.36833574)(84.70849968,697.29833581)(84.9285035,697.23834076)
\lineto(85.6185035,697.08834076)
\curveto(85.85849853,697.04833606)(86.0884983,696.99833611)(86.3085035,696.93834076)
\curveto(86.53849785,696.88833622)(86.75349764,696.82333629)(86.9535035,696.74334076)
\curveto(87.04349735,696.70333641)(87.12849726,696.66833644)(87.2085035,696.63834076)
\curveto(87.29849709,696.61833649)(87.38349701,696.58333653)(87.4635035,696.53334076)
\curveto(87.65349674,696.4133367)(87.82349657,696.28333683)(87.9735035,696.14334076)
\curveto(88.13349626,696.00333711)(88.25849613,695.82833728)(88.3485035,695.61834076)
\curveto(88.37849601,695.54833756)(88.40349599,695.47833763)(88.4235035,695.40834076)
\curveto(88.44349595,695.33833777)(88.46349593,695.26333785)(88.4835035,695.18334076)
\curveto(88.4934959,695.12333799)(88.49849589,695.02833808)(88.4985035,694.89834076)
\curveto(88.50849588,694.77833833)(88.50849588,694.68333843)(88.4985035,694.61334076)
\lineto(88.4985035,694.53834076)
\curveto(88.47849591,694.47833863)(88.46349593,694.41833869)(88.4535035,694.35834076)
\curveto(88.45349594,694.3083388)(88.44849594,694.25833885)(88.4385035,694.20834076)
\curveto(88.36849602,693.9083392)(88.25849613,693.64333947)(88.1085035,693.41334076)
\curveto(87.94849644,693.17333994)(87.75349664,692.97834013)(87.5235035,692.82834076)
\curveto(87.2934971,692.67834043)(87.03349736,692.54834056)(86.7435035,692.43834076)
\curveto(86.63349776,692.38834072)(86.51349788,692.35334076)(86.3835035,692.33334076)
\curveto(86.26349813,692.3133408)(86.14349825,692.28834082)(86.0235035,692.25834076)
\curveto(85.93349846,692.23834087)(85.83849855,692.22834088)(85.7385035,692.22834076)
\curveto(85.64849874,692.21834089)(85.55849883,692.20334091)(85.4685035,692.18334076)
\lineto(85.1985035,692.18334076)
\curveto(85.13849925,692.16334095)(85.03349936,692.15334096)(84.8835035,692.15334076)
\curveto(84.74349965,692.15334096)(84.64349975,692.16334095)(84.5835035,692.18334076)
\curveto(84.55349984,692.18334093)(84.51849987,692.18834092)(84.4785035,692.19834076)
\lineto(84.3735035,692.19834076)
\curveto(84.25350014,692.21834089)(84.13350026,692.23334088)(84.0135035,692.24334076)
\curveto(83.8935005,692.25334086)(83.77850061,692.27334084)(83.6685035,692.30334076)
\curveto(83.27850111,692.4133407)(82.93350146,692.53834057)(82.6335035,692.67834076)
\curveto(82.33350206,692.82834028)(82.07850231,693.04834006)(81.8685035,693.33834076)
\curveto(81.72850266,693.52833958)(81.60850278,693.74833936)(81.5085035,693.99834076)
\curveto(81.4885029,694.05833905)(81.46850292,694.13833897)(81.4485035,694.23834076)
\curveto(81.42850296,694.28833882)(81.41350298,694.35833875)(81.4035035,694.44834076)
\curveto(81.393503,694.53833857)(81.39850299,694.6133385)(81.4185035,694.67334076)
\curveto(81.44850294,694.74333837)(81.49850289,694.79333832)(81.5685035,694.82334076)
\curveto(81.61850277,694.84333827)(81.67850271,694.85333826)(81.7485035,694.85334076)
\lineto(81.9735035,694.85334076)
\lineto(82.6785035,694.85334076)
\lineto(82.9185035,694.85334076)
\curveto(82.99850139,694.85333826)(83.06850132,694.84333827)(83.1285035,694.82334076)
\curveto(83.23850115,694.78333833)(83.30850108,694.71833839)(83.3385035,694.62834076)
\curveto(83.37850101,694.53833857)(83.42350097,694.44333867)(83.4735035,694.34334076)
\curveto(83.4935009,694.29333882)(83.52850086,694.22833888)(83.5785035,694.14834076)
\curveto(83.63850075,694.06833904)(83.6885007,694.01833909)(83.7285035,693.99834076)
\curveto(83.84850054,693.89833921)(83.96350043,693.81833929)(84.0735035,693.75834076)
\curveto(84.18350021,693.7083394)(84.32350007,693.65833945)(84.4935035,693.60834076)
\curveto(84.54349985,693.58833952)(84.5934998,693.57833953)(84.6435035,693.57834076)
\curveto(84.6934997,693.58833952)(84.74349965,693.58833952)(84.7935035,693.57834076)
\curveto(84.87349952,693.55833955)(84.95849943,693.54833956)(85.0485035,693.54834076)
\curveto(85.14849924,693.55833955)(85.23349916,693.57333954)(85.3035035,693.59334076)
\curveto(85.35349904,693.60333951)(85.39849899,693.6083395)(85.4385035,693.60834076)
\curveto(85.4884989,693.6083395)(85.53849885,693.61833949)(85.5885035,693.63834076)
\curveto(85.72849866,693.68833942)(85.85349854,693.74833936)(85.9635035,693.81834076)
\curveto(86.08349831,693.88833922)(86.17849821,693.97833913)(86.2485035,694.08834076)
\curveto(86.29849809,694.16833894)(86.33849805,694.29333882)(86.3685035,694.46334076)
\curveto(86.388498,694.53333858)(86.388498,694.59833851)(86.3685035,694.65834076)
\curveto(86.34849804,694.71833839)(86.32849806,694.76833834)(86.3085035,694.80834076)
\curveto(86.23849815,694.94833816)(86.14849824,695.05333806)(86.0385035,695.12334076)
\curveto(85.93849845,695.19333792)(85.81849857,695.25833785)(85.6785035,695.31834076)
\curveto(85.4884989,695.39833771)(85.2884991,695.46333765)(85.0785035,695.51334076)
\curveto(84.86849952,695.56333755)(84.65849973,695.61833749)(84.4485035,695.67834076)
\curveto(84.36850002,695.69833741)(84.28350011,695.7133374)(84.1935035,695.72334076)
\curveto(84.11350028,695.73333738)(84.03350036,695.74833736)(83.9535035,695.76834076)
\curveto(83.63350076,695.85833725)(83.32850106,695.94333717)(83.0385035,696.02334076)
\curveto(82.74850164,696.113337)(82.48350191,696.24333687)(82.2435035,696.41334076)
\curveto(81.96350243,696.6133365)(81.75850263,696.88333623)(81.6285035,697.22334076)
\curveto(81.60850278,697.29333582)(81.5885028,697.38833572)(81.5685035,697.50834076)
\curveto(81.54850284,697.57833553)(81.53350286,697.66333545)(81.5235035,697.76334076)
\curveto(81.51350288,697.86333525)(81.51850287,697.95333516)(81.5385035,698.03334076)
\curveto(81.55850283,698.08333503)(81.56350283,698.12333499)(81.5535035,698.15334076)
\curveto(81.54350285,698.19333492)(81.54850284,698.23833487)(81.5685035,698.28834076)
\curveto(81.5885028,698.39833471)(81.60850278,698.49833461)(81.6285035,698.58834076)
\curveto(81.65850273,698.68833442)(81.6935027,698.78333433)(81.7335035,698.87334076)
\curveto(81.86350253,699.16333395)(82.04350235,699.39833371)(82.2735035,699.57834076)
\curveto(82.50350189,699.75833335)(82.76350163,699.90333321)(83.0535035,700.01334076)
\curveto(83.16350123,700.06333305)(83.27850111,700.09833301)(83.3985035,700.11834076)
\curveto(83.51850087,700.14833296)(83.64350075,700.17833293)(83.7735035,700.20834076)
\curveto(83.83350056,700.22833288)(83.8935005,700.23833287)(83.9535035,700.23834076)
\lineto(84.1335035,700.26834076)
\curveto(84.21350018,700.27833283)(84.29850009,700.28333283)(84.3885035,700.28334076)
\curveto(84.47849991,700.28333283)(84.56349983,700.28833282)(84.6435035,700.29834076)
}
}
{
\newrgbcolor{curcolor}{0 0 0}
\pscustom[linestyle=none,fillstyle=solid,fillcolor=curcolor]
{
}
}
{
\newrgbcolor{curcolor}{0 0 0}
\pscustom[linestyle=none,fillstyle=solid,fillcolor=curcolor]
{
\newpath
\moveto(101.45030037,690.50334076)
\curveto(101.45029203,690.34334277)(101.44529204,690.18834292)(101.43530037,690.03834076)
\curveto(101.43529205,689.87834323)(101.3852921,689.76834334)(101.28530037,689.70834076)
\curveto(101.20529228,689.65834345)(101.09029239,689.63834347)(100.94030037,689.64834076)
\lineto(100.52030037,689.64834076)
\lineto(100.20530037,689.64834076)
\curveto(100.09529339,689.63834347)(99.9852935,689.63834347)(99.87530037,689.64834076)
\curveto(99.77529371,689.64834346)(99.6802938,689.66334345)(99.59030037,689.69334076)
\curveto(99.51029397,689.7133434)(99.45029403,689.75334336)(99.41030037,689.81334076)
\curveto(99.36029412,689.89334322)(99.33529415,690.0083431)(99.33530037,690.15834076)
\curveto(99.34529414,690.29834281)(99.35029413,690.42834268)(99.35030037,690.54834076)
\lineto(99.35030037,692.18334076)
\lineto(99.35030037,692.55834076)
\curveto(99.35029413,692.69834041)(99.33529415,692.80334031)(99.30530037,692.87334076)
\curveto(99.2852942,692.89334022)(99.26529422,692.9083402)(99.24530037,692.91834076)
\curveto(99.23529425,692.93834017)(99.22029426,692.95834015)(99.20030037,692.97834076)
\curveto(99.11029437,692.98834012)(99.04029444,692.96834014)(98.99030037,692.91834076)
\curveto(98.94029454,692.87834023)(98.8852946,692.83834027)(98.82530037,692.79834076)
\curveto(98.73529475,692.72834038)(98.64029484,692.66334045)(98.54030037,692.60334076)
\curveto(98.45029503,692.54334057)(98.35029513,692.48834062)(98.24030037,692.43834076)
\curveto(98.06029542,692.35834075)(97.86029562,692.29834081)(97.64030037,692.25834076)
\curveto(97.42029606,692.2083409)(97.19529629,692.18334093)(96.96530037,692.18334076)
\curveto(96.73529675,692.17334094)(96.50529698,692.18834092)(96.27530037,692.22834076)
\curveto(96.05529743,692.26834084)(95.85529763,692.32834078)(95.67530037,692.40834076)
\curveto(95.22529826,692.6083405)(94.86029862,692.86334025)(94.58030037,693.17334076)
\curveto(94.30029918,693.49333962)(94.06529942,693.88333923)(93.87530037,694.34334076)
\curveto(93.82529966,694.45333866)(93.79029969,694.56333855)(93.77030037,694.67334076)
\curveto(93.75029973,694.79333832)(93.72529976,694.9083382)(93.69530037,695.01834076)
\curveto(93.67529981,695.05833805)(93.66529982,695.09333802)(93.66530037,695.12334076)
\curveto(93.67529981,695.16333795)(93.67529981,695.20333791)(93.66530037,695.24334076)
\curveto(93.64529984,695.32333779)(93.63029985,695.4083377)(93.62030037,695.49834076)
\curveto(93.62029986,695.59833751)(93.61029987,695.69333742)(93.59030037,695.78334076)
\lineto(93.59030037,695.97834076)
\curveto(93.5802999,696.02833708)(93.57529991,696.08833702)(93.57530037,696.15834076)
\curveto(93.57529991,696.23833687)(93.5802999,696.30333681)(93.59030037,696.35334076)
\curveto(93.60029988,696.40333671)(93.60529988,696.44833666)(93.60530037,696.48834076)
\lineto(93.60530037,696.62334076)
\curveto(93.61529987,696.67333644)(93.61529987,696.72333639)(93.60530037,696.77334076)
\curveto(93.60529988,696.82333629)(93.61529987,696.87333624)(93.63530037,696.92334076)
\curveto(93.65529983,697.0133361)(93.67029981,697.10333601)(93.68030037,697.19334076)
\curveto(93.69029979,697.29333582)(93.70529978,697.38833572)(93.72530037,697.47834076)
\curveto(93.77529971,697.64833546)(93.82529966,697.8083353)(93.87530037,697.95834076)
\curveto(93.93529955,698.108335)(93.99529949,698.25333486)(94.05530037,698.39334076)
\curveto(94.11529937,698.53333458)(94.19029929,698.66833444)(94.28030037,698.79834076)
\curveto(94.37029911,698.92833418)(94.46029902,699.05333406)(94.55030037,699.17334076)
\curveto(94.64029884,699.28333383)(94.74029874,699.38333373)(94.85030037,699.47334076)
\curveto(94.8802986,699.50333361)(94.90029858,699.52833358)(94.91030037,699.54834076)
\curveto(94.96029852,699.57833353)(95.00529848,699.6083335)(95.04530037,699.63834076)
\curveto(95.0852984,699.67833343)(95.12529836,699.7133334)(95.16530037,699.74334076)
\curveto(95.30529818,699.84333327)(95.45029803,699.92333319)(95.60030037,699.98334076)
\curveto(95.76029772,700.05333306)(95.92529756,700.11833299)(96.09530037,700.17834076)
\curveto(96.1852973,700.2083329)(96.27529721,700.22833288)(96.36530037,700.23834076)
\curveto(96.45529703,700.24833286)(96.54529694,700.26333285)(96.63530037,700.28334076)
\curveto(96.66529682,700.29333282)(96.72029676,700.29333282)(96.80030037,700.28334076)
\curveto(96.8802966,700.27333284)(96.93029655,700.27833283)(96.95030037,700.29834076)
\curveto(97.27029621,700.3083328)(97.57029591,700.27833283)(97.85030037,700.20834076)
\curveto(98.13029535,700.14833296)(98.37029511,700.05833305)(98.57030037,699.93834076)
\lineto(98.75030037,699.81834076)
\curveto(98.81029467,699.77833333)(98.86529462,699.73833337)(98.91530037,699.69834076)
\curveto(98.97529451,699.64833346)(99.02529446,699.59833351)(99.06530037,699.54834076)
\curveto(99.11529437,699.5083336)(99.19529429,699.48833362)(99.30530037,699.48834076)
\lineto(99.35030037,699.53334076)
\lineto(99.41030037,699.59334076)
\curveto(99.44029404,699.67333344)(99.46029402,699.74833336)(99.47030037,699.81834076)
\curveto(99.480294,699.89833321)(99.52029396,699.96333315)(99.59030037,700.01334076)
\curveto(99.64029384,700.05333306)(99.71029377,700.07333304)(99.80030037,700.07334076)
\curveto(99.90029358,700.08333303)(100.00029348,700.08833302)(100.10030037,700.08834076)
\lineto(100.82030037,700.08834076)
\lineto(101.03030037,700.08834076)
\curveto(101.10029238,700.08833302)(101.16529232,700.07833303)(101.22530037,700.05834076)
\curveto(101.29529219,700.03833307)(101.35029213,699.99333312)(101.39030037,699.92334076)
\curveto(101.44029204,699.85333326)(101.46029202,699.75833335)(101.45030037,699.63834076)
\lineto(101.45030037,699.29334076)
\lineto(101.45030037,690.50334076)
\moveto(99.41030037,696.11334076)
\curveto(99.42029406,696.13333698)(99.42029406,696.15833695)(99.41030037,696.18834076)
\lineto(99.41030037,696.26334076)
\curveto(99.40029408,696.36333675)(99.39529409,696.45833665)(99.39530037,696.54834076)
\curveto(99.39529409,696.63833647)(99.3852941,696.72333639)(99.36530037,696.80334076)
\curveto(99.35529413,696.83333628)(99.35029413,696.85833625)(99.35030037,696.87834076)
\curveto(99.36029412,696.9083362)(99.36029412,696.93833617)(99.35030037,696.96834076)
\curveto(99.33029415,697.04833606)(99.31029417,697.11833599)(99.29030037,697.17834076)
\curveto(99.2802942,697.24833586)(99.26529422,697.31833579)(99.24530037,697.38834076)
\curveto(99.14529434,697.67833543)(99.01029447,697.92833518)(98.84030037,698.13834076)
\curveto(98.67029481,698.34833476)(98.45029503,698.5083346)(98.18030037,698.61834076)
\curveto(98.07029541,698.66833444)(97.95029553,698.69333442)(97.82030037,698.69334076)
\curveto(97.70029578,698.70333441)(97.57029591,698.7083344)(97.43030037,698.70834076)
\curveto(97.40029608,698.68833442)(97.36529612,698.67833443)(97.32530037,698.67834076)
\curveto(97.2852962,698.68833442)(97.24529624,698.68833442)(97.20530037,698.67834076)
\lineto(97.02530037,698.61834076)
\curveto(96.96529652,698.6083345)(96.91029657,698.59333452)(96.86030037,698.57334076)
\curveto(96.57029691,698.44333467)(96.34029714,698.25333486)(96.17030037,698.00334076)
\curveto(96.01029747,697.75333536)(95.8852976,697.46333565)(95.79530037,697.13334076)
\curveto(95.77529771,697.05333606)(95.76029772,696.97833613)(95.75030037,696.90834076)
\curveto(95.75029773,696.84833626)(95.74029774,696.77833633)(95.72030037,696.69834076)
\curveto(95.72029776,696.62833648)(95.71529777,696.57833653)(95.70530037,696.54834076)
\curveto(95.69529779,696.49833661)(95.6852978,696.4083367)(95.67530037,696.27834076)
\curveto(95.67529781,696.15833695)(95.6852978,696.07333704)(95.70530037,696.02334076)
\lineto(95.70530037,695.88834076)
\curveto(95.71529777,695.84833726)(95.72029776,695.8083373)(95.72030037,695.76834076)
\curveto(95.72029776,695.72833738)(95.72529776,695.69333742)(95.73530037,695.66334076)
\lineto(95.73530037,695.58834076)
\curveto(95.74529774,695.55833755)(95.75029773,695.53333758)(95.75030037,695.51334076)
\curveto(95.77029771,695.43333768)(95.7852977,695.35833775)(95.79530037,695.28834076)
\curveto(95.80529768,695.21833789)(95.82529766,695.14833796)(95.85530037,695.07834076)
\curveto(95.93529755,694.82833828)(96.04029744,694.6133385)(96.17030037,694.43334076)
\curveto(96.30029718,694.25333886)(96.46529702,694.09833901)(96.66530037,693.96834076)
\curveto(96.80529668,693.88833922)(96.96029652,693.82833928)(97.13030037,693.78834076)
\curveto(97.16029632,693.77833933)(97.1852963,693.77333934)(97.20530037,693.77334076)
\curveto(97.23529625,693.77333934)(97.27029621,693.76833934)(97.31030037,693.75834076)
\curveto(97.34029614,693.74833936)(97.3852961,693.73833937)(97.44530037,693.72834076)
\curveto(97.51529597,693.72833938)(97.57529591,693.73333938)(97.62530037,693.74334076)
\curveto(97.64529584,693.75333936)(97.67029581,693.75333936)(97.70030037,693.74334076)
\curveto(97.74029574,693.74333937)(97.77529571,693.74833936)(97.80530037,693.75834076)
\curveto(97.87529561,693.77833933)(97.94029554,693.79333932)(98.00030037,693.80334076)
\curveto(98.07029541,693.8133393)(98.14029534,693.82833928)(98.21030037,693.84834076)
\curveto(98.47029501,693.95833915)(98.67529481,694.10333901)(98.82530037,694.28334076)
\curveto(98.9852945,694.46333865)(99.12029436,694.68333843)(99.23030037,694.94334076)
\curveto(99.26029422,695.02333809)(99.2852942,695.108338)(99.30530037,695.19834076)
\lineto(99.36530037,695.46834076)
\lineto(99.36530037,695.57334076)
\curveto(99.37529411,695.60333751)(99.3802941,695.63833747)(99.38030037,695.67834076)
\curveto(99.40029408,695.77833733)(99.41029407,695.86333725)(99.41030037,695.93334076)
\lineto(99.41030037,696.11334076)
}
}
{
\newrgbcolor{curcolor}{0 0 0}
\pscustom[linestyle=none,fillstyle=solid,fillcolor=curcolor]
{
\newpath
\moveto(103.48022225,700.07334076)
\lineto(104.60522225,700.07334076)
\curveto(104.71521981,700.07333304)(104.81521971,700.06833304)(104.90522225,700.05834076)
\curveto(104.99521953,700.04833306)(105.06021947,700.0133331)(105.10022225,699.95334076)
\curveto(105.15021938,699.89333322)(105.18021935,699.8083333)(105.19022225,699.69834076)
\curveto(105.20021933,699.59833351)(105.20521932,699.49333362)(105.20522225,699.38334076)
\lineto(105.20522225,698.33334076)
\lineto(105.20522225,696.09834076)
\curveto(105.20521932,695.73833737)(105.22021931,695.39833771)(105.25022225,695.07834076)
\curveto(105.28021925,694.75833835)(105.37021916,694.49333862)(105.52022225,694.28334076)
\curveto(105.66021887,694.07333904)(105.88521864,693.92333919)(106.19522225,693.83334076)
\curveto(106.24521828,693.82333929)(106.28521824,693.81833929)(106.31522225,693.81834076)
\curveto(106.35521817,693.81833929)(106.40021813,693.8133393)(106.45022225,693.80334076)
\curveto(106.50021803,693.79333932)(106.55521797,693.78833932)(106.61522225,693.78834076)
\curveto(106.67521785,693.78833932)(106.72021781,693.79333932)(106.75022225,693.80334076)
\curveto(106.80021773,693.82333929)(106.84021769,693.82833928)(106.87022225,693.81834076)
\curveto(106.91021762,693.8083393)(106.95021758,693.8133393)(106.99022225,693.83334076)
\curveto(107.20021733,693.88333923)(107.36521716,693.94833916)(107.48522225,694.02834076)
\curveto(107.66521686,694.13833897)(107.80521672,694.27833883)(107.90522225,694.44834076)
\curveto(108.01521651,694.62833848)(108.09021644,694.82333829)(108.13022225,695.03334076)
\curveto(108.18021635,695.25333786)(108.21021632,695.49333762)(108.22022225,695.75334076)
\curveto(108.2302163,696.02333709)(108.23521629,696.30333681)(108.23522225,696.59334076)
\lineto(108.23522225,698.40834076)
\lineto(108.23522225,699.38334076)
\lineto(108.23522225,699.65334076)
\curveto(108.23521629,699.75333336)(108.25521627,699.83333328)(108.29522225,699.89334076)
\curveto(108.34521618,699.98333313)(108.42021611,700.03333308)(108.52022225,700.04334076)
\curveto(108.62021591,700.06333305)(108.74021579,700.07333304)(108.88022225,700.07334076)
\lineto(109.67522225,700.07334076)
\lineto(109.96022225,700.07334076)
\curveto(110.05021448,700.07333304)(110.1252144,700.05333306)(110.18522225,700.01334076)
\curveto(110.26521426,699.96333315)(110.31021422,699.88833322)(110.32022225,699.78834076)
\curveto(110.3302142,699.68833342)(110.33521419,699.57333354)(110.33522225,699.44334076)
\lineto(110.33522225,698.30334076)
\lineto(110.33522225,694.08834076)
\lineto(110.33522225,693.02334076)
\lineto(110.33522225,692.72334076)
\curveto(110.33521419,692.62334049)(110.31521421,692.54834056)(110.27522225,692.49834076)
\curveto(110.2252143,692.41834069)(110.15021438,692.37334074)(110.05022225,692.36334076)
\curveto(109.95021458,692.35334076)(109.84521468,692.34834076)(109.73522225,692.34834076)
\lineto(108.92522225,692.34834076)
\curveto(108.81521571,692.34834076)(108.71521581,692.35334076)(108.62522225,692.36334076)
\curveto(108.54521598,692.37334074)(108.48021605,692.4133407)(108.43022225,692.48334076)
\curveto(108.41021612,692.5133406)(108.39021614,692.55834055)(108.37022225,692.61834076)
\curveto(108.36021617,692.67834043)(108.34521618,692.73834037)(108.32522225,692.79834076)
\curveto(108.31521621,692.85834025)(108.30021623,692.9133402)(108.28022225,692.96334076)
\curveto(108.26021627,693.0133401)(108.2302163,693.04334007)(108.19022225,693.05334076)
\curveto(108.17021636,693.07334004)(108.14521638,693.07834003)(108.11522225,693.06834076)
\curveto(108.08521644,693.05834005)(108.06021647,693.04834006)(108.04022225,693.03834076)
\curveto(107.97021656,692.99834011)(107.91021662,692.95334016)(107.86022225,692.90334076)
\curveto(107.81021672,692.85334026)(107.75521677,692.8083403)(107.69522225,692.76834076)
\curveto(107.65521687,692.73834037)(107.61521691,692.70334041)(107.57522225,692.66334076)
\curveto(107.54521698,692.63334048)(107.50521702,692.60334051)(107.45522225,692.57334076)
\curveto(107.2252173,692.43334068)(106.95521757,692.32334079)(106.64522225,692.24334076)
\curveto(106.57521795,692.22334089)(106.50521802,692.2133409)(106.43522225,692.21334076)
\curveto(106.36521816,692.20334091)(106.29021824,692.18834092)(106.21022225,692.16834076)
\curveto(106.17021836,692.15834095)(106.1252184,692.15834095)(106.07522225,692.16834076)
\curveto(106.03521849,692.16834094)(105.99521853,692.16334095)(105.95522225,692.15334076)
\curveto(105.9252186,692.14334097)(105.86021867,692.14334097)(105.76022225,692.15334076)
\curveto(105.67021886,692.15334096)(105.61021892,692.15834095)(105.58022225,692.16834076)
\curveto(105.530219,692.16834094)(105.48021905,692.17334094)(105.43022225,692.18334076)
\lineto(105.28022225,692.18334076)
\curveto(105.16021937,692.2133409)(105.04521948,692.23834087)(104.93522225,692.25834076)
\curveto(104.8252197,692.27834083)(104.71521981,692.3083408)(104.60522225,692.34834076)
\curveto(104.55521997,692.36834074)(104.51022002,692.38334073)(104.47022225,692.39334076)
\curveto(104.44022009,692.4133407)(104.40022013,692.43334068)(104.35022225,692.45334076)
\curveto(104.00022053,692.64334047)(103.72022081,692.9083402)(103.51022225,693.24834076)
\curveto(103.38022115,693.45833965)(103.28522124,693.7083394)(103.22522225,693.99834076)
\curveto(103.16522136,694.29833881)(103.1252214,694.6133385)(103.10522225,694.94334076)
\curveto(103.09522143,695.28333783)(103.09022144,695.62833748)(103.09022225,695.97834076)
\curveto(103.10022143,696.33833677)(103.10522142,696.69333642)(103.10522225,697.04334076)
\lineto(103.10522225,699.08334076)
\curveto(103.10522142,699.2133339)(103.10022143,699.36333375)(103.09022225,699.53334076)
\curveto(103.09022144,699.7133334)(103.11522141,699.84333327)(103.16522225,699.92334076)
\curveto(103.19522133,699.97333314)(103.25522127,700.01833309)(103.34522225,700.05834076)
\curveto(103.40522112,700.05833305)(103.45022108,700.06333305)(103.48022225,700.07334076)
}
}
{
\newrgbcolor{curcolor}{0 0 0}
\pscustom[linestyle=none,fillstyle=solid,fillcolor=curcolor]
{
\newpath
\moveto(119.33147225,696.29334076)
\curveto(119.35146408,696.2133369)(119.35146408,696.12333699)(119.33147225,696.02334076)
\curveto(119.31146412,695.92333719)(119.27646416,695.85833725)(119.22647225,695.82834076)
\curveto(119.17646426,695.78833732)(119.10146433,695.75833735)(119.00147225,695.73834076)
\curveto(118.91146452,695.72833738)(118.80646463,695.71833739)(118.68647225,695.70834076)
\lineto(118.34147225,695.70834076)
\curveto(118.2314652,695.71833739)(118.1314653,695.72333739)(118.04147225,695.72334076)
\lineto(114.38147225,695.72334076)
\lineto(114.17147225,695.72334076)
\curveto(114.11146932,695.72333739)(114.05646938,695.7133374)(114.00647225,695.69334076)
\curveto(113.92646951,695.65333746)(113.87646956,695.6133375)(113.85647225,695.57334076)
\curveto(113.8364696,695.55333756)(113.81646962,695.5133376)(113.79647225,695.45334076)
\curveto(113.77646966,695.40333771)(113.77146966,695.35333776)(113.78147225,695.30334076)
\curveto(113.80146963,695.24333787)(113.81146962,695.18333793)(113.81147225,695.12334076)
\curveto(113.82146961,695.07333804)(113.8364696,695.01833809)(113.85647225,694.95834076)
\curveto(113.9364695,694.71833839)(114.0314694,694.51833859)(114.14147225,694.35834076)
\curveto(114.26146917,694.2083389)(114.42146901,694.07333904)(114.62147225,693.95334076)
\curveto(114.70146873,693.90333921)(114.78146865,693.86833924)(114.86147225,693.84834076)
\curveto(114.95146848,693.83833927)(115.04146839,693.81833929)(115.13147225,693.78834076)
\curveto(115.21146822,693.76833934)(115.32146811,693.75333936)(115.46147225,693.74334076)
\curveto(115.60146783,693.73333938)(115.72146771,693.73833937)(115.82147225,693.75834076)
\lineto(115.95647225,693.75834076)
\curveto(116.05646738,693.77833933)(116.14646729,693.79833931)(116.22647225,693.81834076)
\curveto(116.31646712,693.84833926)(116.40146703,693.87833923)(116.48147225,693.90834076)
\curveto(116.58146685,693.95833915)(116.69146674,694.02333909)(116.81147225,694.10334076)
\curveto(116.94146649,694.18333893)(117.0364664,694.26333885)(117.09647225,694.34334076)
\curveto(117.14646629,694.4133387)(117.19646624,694.47833863)(117.24647225,694.53834076)
\curveto(117.30646613,694.6083385)(117.37646606,694.65833845)(117.45647225,694.68834076)
\curveto(117.55646588,694.73833837)(117.68146575,694.75833835)(117.83147225,694.74834076)
\lineto(118.26647225,694.74834076)
\lineto(118.44647225,694.74834076)
\curveto(118.51646492,694.75833835)(118.57646486,694.75333836)(118.62647225,694.73334076)
\lineto(118.77647225,694.73334076)
\curveto(118.87646456,694.7133384)(118.94646449,694.68833842)(118.98647225,694.65834076)
\curveto(119.02646441,694.63833847)(119.04646439,694.59333852)(119.04647225,694.52334076)
\curveto(119.05646438,694.45333866)(119.05146438,694.39333872)(119.03147225,694.34334076)
\curveto(118.98146445,694.20333891)(118.92646451,694.07833903)(118.86647225,693.96834076)
\curveto(118.80646463,693.85833925)(118.7364647,693.74833936)(118.65647225,693.63834076)
\curveto(118.436465,693.3083398)(118.18646525,693.04334007)(117.90647225,692.84334076)
\curveto(117.62646581,692.64334047)(117.27646616,692.47334064)(116.85647225,692.33334076)
\curveto(116.74646669,692.29334082)(116.6364668,692.26834084)(116.52647225,692.25834076)
\curveto(116.41646702,692.24834086)(116.30146713,692.22834088)(116.18147225,692.19834076)
\curveto(116.14146729,692.18834092)(116.09646734,692.18834092)(116.04647225,692.19834076)
\curveto(116.00646743,692.19834091)(115.96646747,692.19334092)(115.92647225,692.18334076)
\lineto(115.76147225,692.18334076)
\curveto(115.71146772,692.16334095)(115.65146778,692.15834095)(115.58147225,692.16834076)
\curveto(115.52146791,692.16834094)(115.46646797,692.17334094)(115.41647225,692.18334076)
\curveto(115.3364681,692.19334092)(115.26646817,692.19334092)(115.20647225,692.18334076)
\curveto(115.14646829,692.17334094)(115.08146835,692.17834093)(115.01147225,692.19834076)
\curveto(114.96146847,692.21834089)(114.90646853,692.22834088)(114.84647225,692.22834076)
\curveto(114.78646865,692.22834088)(114.7314687,692.23834087)(114.68147225,692.25834076)
\curveto(114.57146886,692.27834083)(114.46146897,692.30334081)(114.35147225,692.33334076)
\curveto(114.24146919,692.35334076)(114.14146929,692.38834072)(114.05147225,692.43834076)
\curveto(113.94146949,692.47834063)(113.8364696,692.5133406)(113.73647225,692.54334076)
\curveto(113.64646979,692.58334053)(113.56146987,692.62834048)(113.48147225,692.67834076)
\curveto(113.16147027,692.87834023)(112.87647056,693.10834)(112.62647225,693.36834076)
\curveto(112.37647106,693.63833947)(112.17147126,693.94833916)(112.01147225,694.29834076)
\curveto(111.96147147,694.4083387)(111.92147151,694.51833859)(111.89147225,694.62834076)
\curveto(111.86147157,694.74833836)(111.82147161,694.86833824)(111.77147225,694.98834076)
\curveto(111.76147167,695.02833808)(111.75647168,695.06333805)(111.75647225,695.09334076)
\curveto(111.75647168,695.13333798)(111.75147168,695.17333794)(111.74147225,695.21334076)
\curveto(111.70147173,695.33333778)(111.67647176,695.46333765)(111.66647225,695.60334076)
\lineto(111.63647225,696.02334076)
\curveto(111.6364718,696.07333704)(111.6314718,696.12833698)(111.62147225,696.18834076)
\curveto(111.62147181,696.24833686)(111.62647181,696.30333681)(111.63647225,696.35334076)
\lineto(111.63647225,696.53334076)
\lineto(111.68147225,696.89334076)
\curveto(111.72147171,697.06333605)(111.75647168,697.22833588)(111.78647225,697.38834076)
\curveto(111.81647162,697.54833556)(111.86147157,697.69833541)(111.92147225,697.83834076)
\curveto(112.35147108,698.87833423)(113.08147035,699.6133335)(114.11147225,700.04334076)
\curveto(114.25146918,700.10333301)(114.39146904,700.14333297)(114.53147225,700.16334076)
\curveto(114.68146875,700.19333292)(114.8364686,700.22833288)(114.99647225,700.26834076)
\curveto(115.07646836,700.27833283)(115.15146828,700.28333283)(115.22147225,700.28334076)
\curveto(115.29146814,700.28333283)(115.36646807,700.28833282)(115.44647225,700.29834076)
\curveto(115.95646748,700.3083328)(116.39146704,700.24833286)(116.75147225,700.11834076)
\curveto(117.12146631,699.99833311)(117.45146598,699.83833327)(117.74147225,699.63834076)
\curveto(117.8314656,699.57833353)(117.92146551,699.5083336)(118.01147225,699.42834076)
\curveto(118.10146533,699.35833375)(118.18146525,699.28333383)(118.25147225,699.20334076)
\curveto(118.28146515,699.15333396)(118.32146511,699.113334)(118.37147225,699.08334076)
\curveto(118.45146498,698.97333414)(118.52646491,698.85833425)(118.59647225,698.73834076)
\curveto(118.66646477,698.62833448)(118.74146469,698.5133346)(118.82147225,698.39334076)
\curveto(118.87146456,698.30333481)(118.91146452,698.2083349)(118.94147225,698.10834076)
\curveto(118.98146445,698.01833509)(119.02146441,697.91833519)(119.06147225,697.80834076)
\curveto(119.11146432,697.67833543)(119.15146428,697.54333557)(119.18147225,697.40334076)
\curveto(119.21146422,697.26333585)(119.24646419,697.12333599)(119.28647225,696.98334076)
\curveto(119.30646413,696.90333621)(119.31146412,696.8133363)(119.30147225,696.71334076)
\curveto(119.30146413,696.62333649)(119.31146412,696.53833657)(119.33147225,696.45834076)
\lineto(119.33147225,696.29334076)
\moveto(117.08147225,697.17834076)
\curveto(117.15146628,697.27833583)(117.15646628,697.39833571)(117.09647225,697.53834076)
\curveto(117.04646639,697.68833542)(117.00646643,697.79833531)(116.97647225,697.86834076)
\curveto(116.8364666,698.13833497)(116.65146678,698.34333477)(116.42147225,698.48334076)
\curveto(116.19146724,698.63333448)(115.87146756,698.7133344)(115.46147225,698.72334076)
\curveto(115.431468,698.70333441)(115.39646804,698.69833441)(115.35647225,698.70834076)
\curveto(115.31646812,698.71833439)(115.28146815,698.71833439)(115.25147225,698.70834076)
\curveto(115.20146823,698.68833442)(115.14646829,698.67333444)(115.08647225,698.66334076)
\curveto(115.02646841,698.66333445)(114.97146846,698.65333446)(114.92147225,698.63334076)
\curveto(114.48146895,698.49333462)(114.15646928,698.21833489)(113.94647225,697.80834076)
\curveto(113.92646951,697.76833534)(113.90146953,697.7133354)(113.87147225,697.64334076)
\curveto(113.85146958,697.58333553)(113.8364696,697.51833559)(113.82647225,697.44834076)
\curveto(113.81646962,697.38833572)(113.81646962,697.32833578)(113.82647225,697.26834076)
\curveto(113.84646959,697.2083359)(113.88146955,697.15833595)(113.93147225,697.11834076)
\curveto(114.01146942,697.06833604)(114.12146931,697.04333607)(114.26147225,697.04334076)
\lineto(114.66647225,697.04334076)
\lineto(116.33147225,697.04334076)
\lineto(116.76647225,697.04334076)
\curveto(116.92646651,697.05333606)(117.0314664,697.09833601)(117.08147225,697.17834076)
}
}
{
\newrgbcolor{curcolor}{0 0 0}
\pscustom[linestyle=none,fillstyle=solid,fillcolor=curcolor]
{
}
}
{
\newrgbcolor{curcolor}{0 0 0}
\pscustom[linestyle=none,fillstyle=solid,fillcolor=curcolor]
{
\newpath
\moveto(132.10490975,696.29334076)
\curveto(132.12490158,696.2133369)(132.12490158,696.12333699)(132.10490975,696.02334076)
\curveto(132.08490162,695.92333719)(132.04990166,695.85833725)(131.99990975,695.82834076)
\curveto(131.94990176,695.78833732)(131.87490183,695.75833735)(131.77490975,695.73834076)
\curveto(131.68490202,695.72833738)(131.57990213,695.71833739)(131.45990975,695.70834076)
\lineto(131.11490975,695.70834076)
\curveto(131.0049027,695.71833739)(130.9049028,695.72333739)(130.81490975,695.72334076)
\lineto(127.15490975,695.72334076)
\lineto(126.94490975,695.72334076)
\curveto(126.88490682,695.72333739)(126.82990688,695.7133374)(126.77990975,695.69334076)
\curveto(126.69990701,695.65333746)(126.64990706,695.6133375)(126.62990975,695.57334076)
\curveto(126.6099071,695.55333756)(126.58990712,695.5133376)(126.56990975,695.45334076)
\curveto(126.54990716,695.40333771)(126.54490716,695.35333776)(126.55490975,695.30334076)
\curveto(126.57490713,695.24333787)(126.58490712,695.18333793)(126.58490975,695.12334076)
\curveto(126.59490711,695.07333804)(126.6099071,695.01833809)(126.62990975,694.95834076)
\curveto(126.709907,694.71833839)(126.8049069,694.51833859)(126.91490975,694.35834076)
\curveto(127.03490667,694.2083389)(127.19490651,694.07333904)(127.39490975,693.95334076)
\curveto(127.47490623,693.90333921)(127.55490615,693.86833924)(127.63490975,693.84834076)
\curveto(127.72490598,693.83833927)(127.81490589,693.81833929)(127.90490975,693.78834076)
\curveto(127.98490572,693.76833934)(128.09490561,693.75333936)(128.23490975,693.74334076)
\curveto(128.37490533,693.73333938)(128.49490521,693.73833937)(128.59490975,693.75834076)
\lineto(128.72990975,693.75834076)
\curveto(128.82990488,693.77833933)(128.91990479,693.79833931)(128.99990975,693.81834076)
\curveto(129.08990462,693.84833926)(129.17490453,693.87833923)(129.25490975,693.90834076)
\curveto(129.35490435,693.95833915)(129.46490424,694.02333909)(129.58490975,694.10334076)
\curveto(129.71490399,694.18333893)(129.8099039,694.26333885)(129.86990975,694.34334076)
\curveto(129.91990379,694.4133387)(129.96990374,694.47833863)(130.01990975,694.53834076)
\curveto(130.07990363,694.6083385)(130.14990356,694.65833845)(130.22990975,694.68834076)
\curveto(130.32990338,694.73833837)(130.45490325,694.75833835)(130.60490975,694.74834076)
\lineto(131.03990975,694.74834076)
\lineto(131.21990975,694.74834076)
\curveto(131.28990242,694.75833835)(131.34990236,694.75333836)(131.39990975,694.73334076)
\lineto(131.54990975,694.73334076)
\curveto(131.64990206,694.7133384)(131.71990199,694.68833842)(131.75990975,694.65834076)
\curveto(131.79990191,694.63833847)(131.81990189,694.59333852)(131.81990975,694.52334076)
\curveto(131.82990188,694.45333866)(131.82490188,694.39333872)(131.80490975,694.34334076)
\curveto(131.75490195,694.20333891)(131.69990201,694.07833903)(131.63990975,693.96834076)
\curveto(131.57990213,693.85833925)(131.5099022,693.74833936)(131.42990975,693.63834076)
\curveto(131.2099025,693.3083398)(130.95990275,693.04334007)(130.67990975,692.84334076)
\curveto(130.39990331,692.64334047)(130.04990366,692.47334064)(129.62990975,692.33334076)
\curveto(129.51990419,692.29334082)(129.4099043,692.26834084)(129.29990975,692.25834076)
\curveto(129.18990452,692.24834086)(129.07490463,692.22834088)(128.95490975,692.19834076)
\curveto(128.91490479,692.18834092)(128.86990484,692.18834092)(128.81990975,692.19834076)
\curveto(128.77990493,692.19834091)(128.73990497,692.19334092)(128.69990975,692.18334076)
\lineto(128.53490975,692.18334076)
\curveto(128.48490522,692.16334095)(128.42490528,692.15834095)(128.35490975,692.16834076)
\curveto(128.29490541,692.16834094)(128.23990547,692.17334094)(128.18990975,692.18334076)
\curveto(128.1099056,692.19334092)(128.03990567,692.19334092)(127.97990975,692.18334076)
\curveto(127.91990579,692.17334094)(127.85490585,692.17834093)(127.78490975,692.19834076)
\curveto(127.73490597,692.21834089)(127.67990603,692.22834088)(127.61990975,692.22834076)
\curveto(127.55990615,692.22834088)(127.5049062,692.23834087)(127.45490975,692.25834076)
\curveto(127.34490636,692.27834083)(127.23490647,692.30334081)(127.12490975,692.33334076)
\curveto(127.01490669,692.35334076)(126.91490679,692.38834072)(126.82490975,692.43834076)
\curveto(126.71490699,692.47834063)(126.6099071,692.5133406)(126.50990975,692.54334076)
\curveto(126.41990729,692.58334053)(126.33490737,692.62834048)(126.25490975,692.67834076)
\curveto(125.93490777,692.87834023)(125.64990806,693.10834)(125.39990975,693.36834076)
\curveto(125.14990856,693.63833947)(124.94490876,693.94833916)(124.78490975,694.29834076)
\curveto(124.73490897,694.4083387)(124.69490901,694.51833859)(124.66490975,694.62834076)
\curveto(124.63490907,694.74833836)(124.59490911,694.86833824)(124.54490975,694.98834076)
\curveto(124.53490917,695.02833808)(124.52990918,695.06333805)(124.52990975,695.09334076)
\curveto(124.52990918,695.13333798)(124.52490918,695.17333794)(124.51490975,695.21334076)
\curveto(124.47490923,695.33333778)(124.44990926,695.46333765)(124.43990975,695.60334076)
\lineto(124.40990975,696.02334076)
\curveto(124.4099093,696.07333704)(124.4049093,696.12833698)(124.39490975,696.18834076)
\curveto(124.39490931,696.24833686)(124.39990931,696.30333681)(124.40990975,696.35334076)
\lineto(124.40990975,696.53334076)
\lineto(124.45490975,696.89334076)
\curveto(124.49490921,697.06333605)(124.52990918,697.22833588)(124.55990975,697.38834076)
\curveto(124.58990912,697.54833556)(124.63490907,697.69833541)(124.69490975,697.83834076)
\curveto(125.12490858,698.87833423)(125.85490785,699.6133335)(126.88490975,700.04334076)
\curveto(127.02490668,700.10333301)(127.16490654,700.14333297)(127.30490975,700.16334076)
\curveto(127.45490625,700.19333292)(127.6099061,700.22833288)(127.76990975,700.26834076)
\curveto(127.84990586,700.27833283)(127.92490578,700.28333283)(127.99490975,700.28334076)
\curveto(128.06490564,700.28333283)(128.13990557,700.28833282)(128.21990975,700.29834076)
\curveto(128.72990498,700.3083328)(129.16490454,700.24833286)(129.52490975,700.11834076)
\curveto(129.89490381,699.99833311)(130.22490348,699.83833327)(130.51490975,699.63834076)
\curveto(130.6049031,699.57833353)(130.69490301,699.5083336)(130.78490975,699.42834076)
\curveto(130.87490283,699.35833375)(130.95490275,699.28333383)(131.02490975,699.20334076)
\curveto(131.05490265,699.15333396)(131.09490261,699.113334)(131.14490975,699.08334076)
\curveto(131.22490248,698.97333414)(131.29990241,698.85833425)(131.36990975,698.73834076)
\curveto(131.43990227,698.62833448)(131.51490219,698.5133346)(131.59490975,698.39334076)
\curveto(131.64490206,698.30333481)(131.68490202,698.2083349)(131.71490975,698.10834076)
\curveto(131.75490195,698.01833509)(131.79490191,697.91833519)(131.83490975,697.80834076)
\curveto(131.88490182,697.67833543)(131.92490178,697.54333557)(131.95490975,697.40334076)
\curveto(131.98490172,697.26333585)(132.01990169,697.12333599)(132.05990975,696.98334076)
\curveto(132.07990163,696.90333621)(132.08490162,696.8133363)(132.07490975,696.71334076)
\curveto(132.07490163,696.62333649)(132.08490162,696.53833657)(132.10490975,696.45834076)
\lineto(132.10490975,696.29334076)
\moveto(129.85490975,697.17834076)
\curveto(129.92490378,697.27833583)(129.92990378,697.39833571)(129.86990975,697.53834076)
\curveto(129.81990389,697.68833542)(129.77990393,697.79833531)(129.74990975,697.86834076)
\curveto(129.6099041,698.13833497)(129.42490428,698.34333477)(129.19490975,698.48334076)
\curveto(128.96490474,698.63333448)(128.64490506,698.7133344)(128.23490975,698.72334076)
\curveto(128.2049055,698.70333441)(128.16990554,698.69833441)(128.12990975,698.70834076)
\curveto(128.08990562,698.71833439)(128.05490565,698.71833439)(128.02490975,698.70834076)
\curveto(127.97490573,698.68833442)(127.91990579,698.67333444)(127.85990975,698.66334076)
\curveto(127.79990591,698.66333445)(127.74490596,698.65333446)(127.69490975,698.63334076)
\curveto(127.25490645,698.49333462)(126.92990678,698.21833489)(126.71990975,697.80834076)
\curveto(126.69990701,697.76833534)(126.67490703,697.7133354)(126.64490975,697.64334076)
\curveto(126.62490708,697.58333553)(126.6099071,697.51833559)(126.59990975,697.44834076)
\curveto(126.58990712,697.38833572)(126.58990712,697.32833578)(126.59990975,697.26834076)
\curveto(126.61990709,697.2083359)(126.65490705,697.15833595)(126.70490975,697.11834076)
\curveto(126.78490692,697.06833604)(126.89490681,697.04333607)(127.03490975,697.04334076)
\lineto(127.43990975,697.04334076)
\lineto(129.10490975,697.04334076)
\lineto(129.53990975,697.04334076)
\curveto(129.69990401,697.05333606)(129.8049039,697.09833601)(129.85490975,697.17834076)
}
}
{
\newrgbcolor{curcolor}{0 0 0}
\pscustom[linestyle=none,fillstyle=solid,fillcolor=curcolor]
{
\newpath
\moveto(136.323191,700.29834076)
\curveto(137.0731865,700.31833279)(137.72318585,700.23333288)(138.273191,700.04334076)
\curveto(138.83318474,699.86333325)(139.25818431,699.54833356)(139.548191,699.09834076)
\curveto(139.61818395,698.98833412)(139.67818389,698.87333424)(139.728191,698.75334076)
\curveto(139.78818378,698.64333447)(139.83818373,698.51833459)(139.878191,698.37834076)
\curveto(139.89818367,698.31833479)(139.90818366,698.25333486)(139.908191,698.18334076)
\curveto(139.90818366,698.113335)(139.89818367,698.05333506)(139.878191,698.00334076)
\curveto(139.83818373,697.94333517)(139.78318379,697.90333521)(139.713191,697.88334076)
\curveto(139.66318391,697.86333525)(139.60318397,697.85333526)(139.533191,697.85334076)
\lineto(139.323191,697.85334076)
\lineto(138.663191,697.85334076)
\curveto(138.59318498,697.85333526)(138.52318505,697.84833526)(138.453191,697.83834076)
\curveto(138.38318519,697.83833527)(138.31818525,697.84833526)(138.258191,697.86834076)
\curveto(138.15818541,697.88833522)(138.08318549,697.92833518)(138.033191,697.98834076)
\curveto(137.98318559,698.04833506)(137.93818563,698.108335)(137.898191,698.16834076)
\lineto(137.778191,698.37834076)
\curveto(137.74818582,698.45833465)(137.69818587,698.52333459)(137.628191,698.57334076)
\curveto(137.52818604,698.65333446)(137.42818614,698.7133344)(137.328191,698.75334076)
\curveto(137.23818633,698.79333432)(137.12318645,698.82833428)(136.983191,698.85834076)
\curveto(136.91318666,698.87833423)(136.80818676,698.89333422)(136.668191,698.90334076)
\curveto(136.53818703,698.9133342)(136.43818713,698.9083342)(136.368191,698.88834076)
\lineto(136.263191,698.88834076)
\lineto(136.113191,698.85834076)
\curveto(136.0731875,698.85833425)(136.02818754,698.85333426)(135.978191,698.84334076)
\curveto(135.80818776,698.79333432)(135.6681879,698.72333439)(135.558191,698.63334076)
\curveto(135.45818811,698.55333456)(135.38818818,698.42833468)(135.348191,698.25834076)
\curveto(135.32818824,698.18833492)(135.32818824,698.12333499)(135.348191,698.06334076)
\curveto(135.3681882,698.00333511)(135.38818818,697.95333516)(135.408191,697.91334076)
\curveto(135.47818809,697.79333532)(135.55818801,697.69833541)(135.648191,697.62834076)
\curveto(135.74818782,697.55833555)(135.86318771,697.49833561)(135.993191,697.44834076)
\curveto(136.18318739,697.36833574)(136.38818718,697.29833581)(136.608191,697.23834076)
\lineto(137.298191,697.08834076)
\curveto(137.53818603,697.04833606)(137.7681858,696.99833611)(137.988191,696.93834076)
\curveto(138.21818535,696.88833622)(138.43318514,696.82333629)(138.633191,696.74334076)
\curveto(138.72318485,696.70333641)(138.80818476,696.66833644)(138.888191,696.63834076)
\curveto(138.97818459,696.61833649)(139.06318451,696.58333653)(139.143191,696.53334076)
\curveto(139.33318424,696.4133367)(139.50318407,696.28333683)(139.653191,696.14334076)
\curveto(139.81318376,696.00333711)(139.93818363,695.82833728)(140.028191,695.61834076)
\curveto(140.05818351,695.54833756)(140.08318349,695.47833763)(140.103191,695.40834076)
\curveto(140.12318345,695.33833777)(140.14318343,695.26333785)(140.163191,695.18334076)
\curveto(140.1731834,695.12333799)(140.17818339,695.02833808)(140.178191,694.89834076)
\curveto(140.18818338,694.77833833)(140.18818338,694.68333843)(140.178191,694.61334076)
\lineto(140.178191,694.53834076)
\curveto(140.15818341,694.47833863)(140.14318343,694.41833869)(140.133191,694.35834076)
\curveto(140.13318344,694.3083388)(140.12818344,694.25833885)(140.118191,694.20834076)
\curveto(140.04818352,693.9083392)(139.93818363,693.64333947)(139.788191,693.41334076)
\curveto(139.62818394,693.17333994)(139.43318414,692.97834013)(139.203191,692.82834076)
\curveto(138.9731846,692.67834043)(138.71318486,692.54834056)(138.423191,692.43834076)
\curveto(138.31318526,692.38834072)(138.19318538,692.35334076)(138.063191,692.33334076)
\curveto(137.94318563,692.3133408)(137.82318575,692.28834082)(137.703191,692.25834076)
\curveto(137.61318596,692.23834087)(137.51818605,692.22834088)(137.418191,692.22834076)
\curveto(137.32818624,692.21834089)(137.23818633,692.20334091)(137.148191,692.18334076)
\lineto(136.878191,692.18334076)
\curveto(136.81818675,692.16334095)(136.71318686,692.15334096)(136.563191,692.15334076)
\curveto(136.42318715,692.15334096)(136.32318725,692.16334095)(136.263191,692.18334076)
\curveto(136.23318734,692.18334093)(136.19818737,692.18834092)(136.158191,692.19834076)
\lineto(136.053191,692.19834076)
\curveto(135.93318764,692.21834089)(135.81318776,692.23334088)(135.693191,692.24334076)
\curveto(135.573188,692.25334086)(135.45818811,692.27334084)(135.348191,692.30334076)
\curveto(134.95818861,692.4133407)(134.61318896,692.53834057)(134.313191,692.67834076)
\curveto(134.01318956,692.82834028)(133.75818981,693.04834006)(133.548191,693.33834076)
\curveto(133.40819016,693.52833958)(133.28819028,693.74833936)(133.188191,693.99834076)
\curveto(133.1681904,694.05833905)(133.14819042,694.13833897)(133.128191,694.23834076)
\curveto(133.10819046,694.28833882)(133.09319048,694.35833875)(133.083191,694.44834076)
\curveto(133.0731905,694.53833857)(133.07819049,694.6133385)(133.098191,694.67334076)
\curveto(133.12819044,694.74333837)(133.17819039,694.79333832)(133.248191,694.82334076)
\curveto(133.29819027,694.84333827)(133.35819021,694.85333826)(133.428191,694.85334076)
\lineto(133.653191,694.85334076)
\lineto(134.358191,694.85334076)
\lineto(134.598191,694.85334076)
\curveto(134.67818889,694.85333826)(134.74818882,694.84333827)(134.808191,694.82334076)
\curveto(134.91818865,694.78333833)(134.98818858,694.71833839)(135.018191,694.62834076)
\curveto(135.05818851,694.53833857)(135.10318847,694.44333867)(135.153191,694.34334076)
\curveto(135.1731884,694.29333882)(135.20818836,694.22833888)(135.258191,694.14834076)
\curveto(135.31818825,694.06833904)(135.3681882,694.01833909)(135.408191,693.99834076)
\curveto(135.52818804,693.89833921)(135.64318793,693.81833929)(135.753191,693.75834076)
\curveto(135.86318771,693.7083394)(136.00318757,693.65833945)(136.173191,693.60834076)
\curveto(136.22318735,693.58833952)(136.2731873,693.57833953)(136.323191,693.57834076)
\curveto(136.3731872,693.58833952)(136.42318715,693.58833952)(136.473191,693.57834076)
\curveto(136.55318702,693.55833955)(136.63818693,693.54833956)(136.728191,693.54834076)
\curveto(136.82818674,693.55833955)(136.91318666,693.57333954)(136.983191,693.59334076)
\curveto(137.03318654,693.60333951)(137.07818649,693.6083395)(137.118191,693.60834076)
\curveto(137.1681864,693.6083395)(137.21818635,693.61833949)(137.268191,693.63834076)
\curveto(137.40818616,693.68833942)(137.53318604,693.74833936)(137.643191,693.81834076)
\curveto(137.76318581,693.88833922)(137.85818571,693.97833913)(137.928191,694.08834076)
\curveto(137.97818559,694.16833894)(138.01818555,694.29333882)(138.048191,694.46334076)
\curveto(138.0681855,694.53333858)(138.0681855,694.59833851)(138.048191,694.65834076)
\curveto(138.02818554,694.71833839)(138.00818556,694.76833834)(137.988191,694.80834076)
\curveto(137.91818565,694.94833816)(137.82818574,695.05333806)(137.718191,695.12334076)
\curveto(137.61818595,695.19333792)(137.49818607,695.25833785)(137.358191,695.31834076)
\curveto(137.1681864,695.39833771)(136.9681866,695.46333765)(136.758191,695.51334076)
\curveto(136.54818702,695.56333755)(136.33818723,695.61833749)(136.128191,695.67834076)
\curveto(136.04818752,695.69833741)(135.96318761,695.7133374)(135.873191,695.72334076)
\curveto(135.79318778,695.73333738)(135.71318786,695.74833736)(135.633191,695.76834076)
\curveto(135.31318826,695.85833725)(135.00818856,695.94333717)(134.718191,696.02334076)
\curveto(134.42818914,696.113337)(134.16318941,696.24333687)(133.923191,696.41334076)
\curveto(133.64318993,696.6133365)(133.43819013,696.88333623)(133.308191,697.22334076)
\curveto(133.28819028,697.29333582)(133.2681903,697.38833572)(133.248191,697.50834076)
\curveto(133.22819034,697.57833553)(133.21319036,697.66333545)(133.203191,697.76334076)
\curveto(133.19319038,697.86333525)(133.19819037,697.95333516)(133.218191,698.03334076)
\curveto(133.23819033,698.08333503)(133.24319033,698.12333499)(133.233191,698.15334076)
\curveto(133.22319035,698.19333492)(133.22819034,698.23833487)(133.248191,698.28834076)
\curveto(133.2681903,698.39833471)(133.28819028,698.49833461)(133.308191,698.58834076)
\curveto(133.33819023,698.68833442)(133.3731902,698.78333433)(133.413191,698.87334076)
\curveto(133.54319003,699.16333395)(133.72318985,699.39833371)(133.953191,699.57834076)
\curveto(134.18318939,699.75833335)(134.44318913,699.90333321)(134.733191,700.01334076)
\curveto(134.84318873,700.06333305)(134.95818861,700.09833301)(135.078191,700.11834076)
\curveto(135.19818837,700.14833296)(135.32318825,700.17833293)(135.453191,700.20834076)
\curveto(135.51318806,700.22833288)(135.573188,700.23833287)(135.633191,700.23834076)
\lineto(135.813191,700.26834076)
\curveto(135.89318768,700.27833283)(135.97818759,700.28333283)(136.068191,700.28334076)
\curveto(136.15818741,700.28333283)(136.24318733,700.28833282)(136.323191,700.29834076)
}
}
{
\newrgbcolor{curcolor}{0 0 0}
\pscustom[linestyle=none,fillstyle=solid,fillcolor=curcolor]
{
\newpath
\moveto(142.45983162,702.39834076)
\lineto(143.46483162,702.39834076)
\curveto(143.61482864,702.39833071)(143.74482851,702.38833072)(143.85483162,702.36834076)
\curveto(143.97482828,702.35833075)(144.05982819,702.29833081)(144.10983162,702.18834076)
\curveto(144.12982812,702.13833097)(144.13982811,702.07833103)(144.13983162,702.00834076)
\lineto(144.13983162,701.79834076)
\lineto(144.13983162,701.12334076)
\curveto(144.13982811,701.07333204)(144.13482812,701.0133321)(144.12483162,700.94334076)
\curveto(144.12482813,700.88333223)(144.12982812,700.82833228)(144.13983162,700.77834076)
\lineto(144.13983162,700.61334076)
\curveto(144.13982811,700.53333258)(144.14482811,700.45833265)(144.15483162,700.38834076)
\curveto(144.16482809,700.32833278)(144.18982806,700.27333284)(144.22983162,700.22334076)
\curveto(144.29982795,700.13333298)(144.42482783,700.08333303)(144.60483162,700.07334076)
\lineto(145.14483162,700.07334076)
\lineto(145.32483162,700.07334076)
\curveto(145.38482687,700.07333304)(145.43982681,700.06333305)(145.48983162,700.04334076)
\curveto(145.59982665,699.99333312)(145.65982659,699.90333321)(145.66983162,699.77334076)
\curveto(145.68982656,699.64333347)(145.69982655,699.49833361)(145.69983162,699.33834076)
\lineto(145.69983162,699.12834076)
\curveto(145.70982654,699.05833405)(145.70482655,698.99833411)(145.68483162,698.94834076)
\curveto(145.63482662,698.78833432)(145.52982672,698.70333441)(145.36983162,698.69334076)
\curveto(145.20982704,698.68333443)(145.02982722,698.67833443)(144.82983162,698.67834076)
\lineto(144.69483162,698.67834076)
\curveto(144.6548276,698.68833442)(144.61982763,698.68833442)(144.58983162,698.67834076)
\curveto(144.5498277,698.66833444)(144.51482774,698.66333445)(144.48483162,698.66334076)
\curveto(144.4548278,698.67333444)(144.42482783,698.66833444)(144.39483162,698.64834076)
\curveto(144.31482794,698.62833448)(144.254828,698.58333453)(144.21483162,698.51334076)
\curveto(144.18482807,698.45333466)(144.15982809,698.37833473)(144.13983162,698.28834076)
\curveto(144.12982812,698.23833487)(144.12982812,698.18333493)(144.13983162,698.12334076)
\curveto(144.1498281,698.06333505)(144.1498281,698.0083351)(144.13983162,697.95834076)
\lineto(144.13983162,697.02834076)
\lineto(144.13983162,695.27334076)
\curveto(144.13982811,695.02333809)(144.14482811,694.80333831)(144.15483162,694.61334076)
\curveto(144.17482808,694.43333868)(144.23982801,694.27333884)(144.34983162,694.13334076)
\curveto(144.39982785,694.07333904)(144.46482779,694.02833908)(144.54483162,693.99834076)
\lineto(144.81483162,693.93834076)
\curveto(144.84482741,693.92833918)(144.87482738,693.92333919)(144.90483162,693.92334076)
\curveto(144.94482731,693.93333918)(144.97482728,693.93333918)(144.99483162,693.92334076)
\lineto(145.15983162,693.92334076)
\curveto(145.26982698,693.92333919)(145.36482689,693.91833919)(145.44483162,693.90834076)
\curveto(145.52482673,693.89833921)(145.58982666,693.85833925)(145.63983162,693.78834076)
\curveto(145.67982657,693.72833938)(145.69982655,693.64833946)(145.69983162,693.54834076)
\lineto(145.69983162,693.26334076)
\curveto(145.69982655,693.05334006)(145.69482656,692.85834025)(145.68483162,692.67834076)
\curveto(145.68482657,692.5083406)(145.60482665,692.39334072)(145.44483162,692.33334076)
\curveto(145.39482686,692.3133408)(145.3498269,692.3083408)(145.30983162,692.31834076)
\curveto(145.26982698,692.31834079)(145.22482703,692.3083408)(145.17483162,692.28834076)
\lineto(145.02483162,692.28834076)
\curveto(145.00482725,692.28834082)(144.97482728,692.29334082)(144.93483162,692.30334076)
\curveto(144.89482736,692.30334081)(144.85982739,692.29834081)(144.82983162,692.28834076)
\curveto(144.77982747,692.27834083)(144.72482753,692.27834083)(144.66483162,692.28834076)
\lineto(144.51483162,692.28834076)
\lineto(144.36483162,692.28834076)
\curveto(144.31482794,692.27834083)(144.26982798,692.27834083)(144.22983162,692.28834076)
\lineto(144.06483162,692.28834076)
\curveto(144.01482824,692.29834081)(143.95982829,692.30334081)(143.89983162,692.30334076)
\curveto(143.83982841,692.30334081)(143.78482847,692.3083408)(143.73483162,692.31834076)
\curveto(143.66482859,692.32834078)(143.59982865,692.33834077)(143.53983162,692.34834076)
\lineto(143.35983162,692.37834076)
\curveto(143.249829,692.4083407)(143.14482911,692.44334067)(143.04483162,692.48334076)
\curveto(142.94482931,692.52334059)(142.8498294,692.56834054)(142.75983162,692.61834076)
\lineto(142.66983162,692.67834076)
\curveto(142.63982961,692.7083404)(142.60482965,692.73834037)(142.56483162,692.76834076)
\curveto(142.54482971,692.78834032)(142.51982973,692.8083403)(142.48983162,692.82834076)
\lineto(142.41483162,692.90334076)
\curveto(142.27482998,693.09334002)(142.16983008,693.30333981)(142.09983162,693.53334076)
\curveto(142.07983017,693.57333954)(142.06983018,693.6083395)(142.06983162,693.63834076)
\curveto(142.07983017,693.67833943)(142.07983017,693.72333939)(142.06983162,693.77334076)
\curveto(142.05983019,693.79333932)(142.0548302,693.81833929)(142.05483162,693.84834076)
\curveto(142.0548302,693.87833923)(142.0498302,693.90333921)(142.03983162,693.92334076)
\lineto(142.03983162,694.07334076)
\curveto(142.02983022,694.113339)(142.02483023,694.15833895)(142.02483162,694.20834076)
\curveto(142.03483022,694.25833885)(142.03983021,694.3083388)(142.03983162,694.35834076)
\lineto(142.03983162,694.92834076)
\lineto(142.03983162,697.16334076)
\lineto(142.03983162,697.95834076)
\lineto(142.03983162,698.16834076)
\curveto(142.0498302,698.23833487)(142.04483021,698.30333481)(142.02483162,698.36334076)
\curveto(141.98483027,698.50333461)(141.91483034,698.59333452)(141.81483162,698.63334076)
\curveto(141.70483055,698.68333443)(141.56483069,698.69833441)(141.39483162,698.67834076)
\curveto(141.22483103,698.65833445)(141.07983117,698.67333444)(140.95983162,698.72334076)
\curveto(140.87983137,698.75333436)(140.82983142,698.79833431)(140.80983162,698.85834076)
\curveto(140.78983146,698.91833419)(140.76983148,698.99333412)(140.74983162,699.08334076)
\lineto(140.74983162,699.39834076)
\curveto(140.7498315,699.57833353)(140.75983149,699.72333339)(140.77983162,699.83334076)
\curveto(140.79983145,699.94333317)(140.88483137,700.01833309)(141.03483162,700.05834076)
\curveto(141.07483118,700.07833303)(141.11483114,700.08333303)(141.15483162,700.07334076)
\lineto(141.28983162,700.07334076)
\curveto(141.43983081,700.07333304)(141.57983067,700.07833303)(141.70983162,700.08834076)
\curveto(141.83983041,700.108333)(141.92983032,700.16833294)(141.97983162,700.26834076)
\curveto(142.00983024,700.33833277)(142.02483023,700.41833269)(142.02483162,700.50834076)
\curveto(142.03483022,700.59833251)(142.03983021,700.68833242)(142.03983162,700.77834076)
\lineto(142.03983162,701.70834076)
\lineto(142.03983162,701.96334076)
\curveto(142.03983021,702.05333106)(142.0498302,702.12833098)(142.06983162,702.18834076)
\curveto(142.11983013,702.28833082)(142.19483006,702.35333076)(142.29483162,702.38334076)
\curveto(142.31482994,702.39333072)(142.33982991,702.39333072)(142.36983162,702.38334076)
\curveto(142.40982984,702.38333073)(142.43982981,702.38833072)(142.45983162,702.39834076)
}
}
{
\newrgbcolor{curcolor}{0 0 0}
\pscustom[linestyle=none,fillstyle=solid,fillcolor=curcolor]
{
\newpath
\moveto(153.73326912,692.94834076)
\curveto(153.75326127,692.83834027)(153.76326126,692.72834038)(153.76326912,692.61834076)
\curveto(153.77326125,692.5083406)(153.7232613,692.43334068)(153.61326912,692.39334076)
\curveto(153.55326147,692.36334075)(153.48326154,692.34834076)(153.40326912,692.34834076)
\lineto(153.16326912,692.34834076)
\lineto(152.35326912,692.34834076)
\lineto(152.08326912,692.34834076)
\curveto(152.00326302,692.35834075)(151.93826309,692.38334073)(151.88826912,692.42334076)
\curveto(151.81826321,692.46334065)(151.76326326,692.51834059)(151.72326912,692.58834076)
\curveto(151.69326333,692.66834044)(151.64826338,692.73334038)(151.58826912,692.78334076)
\curveto(151.56826346,692.80334031)(151.54326348,692.81834029)(151.51326912,692.82834076)
\curveto(151.48326354,692.84834026)(151.44326358,692.85334026)(151.39326912,692.84334076)
\curveto(151.34326368,692.82334029)(151.29326373,692.79834031)(151.24326912,692.76834076)
\curveto(151.20326382,692.73834037)(151.15826387,692.7133404)(151.10826912,692.69334076)
\curveto(151.05826397,692.65334046)(151.00326402,692.61834049)(150.94326912,692.58834076)
\lineto(150.76326912,692.49834076)
\curveto(150.63326439,692.43834067)(150.49826453,692.38834072)(150.35826912,692.34834076)
\curveto(150.21826481,692.31834079)(150.07326495,692.28334083)(149.92326912,692.24334076)
\curveto(149.85326517,692.22334089)(149.78326524,692.2133409)(149.71326912,692.21334076)
\curveto(149.65326537,692.20334091)(149.58826544,692.19334092)(149.51826912,692.18334076)
\lineto(149.42826912,692.18334076)
\curveto(149.39826563,692.17334094)(149.36826566,692.16834094)(149.33826912,692.16834076)
\lineto(149.17326912,692.16834076)
\curveto(149.07326595,692.14834096)(148.97326605,692.14834096)(148.87326912,692.16834076)
\lineto(148.73826912,692.16834076)
\curveto(148.66826636,692.18834092)(148.59826643,692.19834091)(148.52826912,692.19834076)
\curveto(148.46826656,692.18834092)(148.40826662,692.19334092)(148.34826912,692.21334076)
\curveto(148.24826678,692.23334088)(148.15326687,692.25334086)(148.06326912,692.27334076)
\curveto(147.97326705,692.28334083)(147.88826714,692.3083408)(147.80826912,692.34834076)
\curveto(147.51826751,692.45834065)(147.26826776,692.59834051)(147.05826912,692.76834076)
\curveto(146.85826817,692.94834016)(146.69826833,693.18333993)(146.57826912,693.47334076)
\curveto(146.54826848,693.54333957)(146.51826851,693.61833949)(146.48826912,693.69834076)
\curveto(146.46826856,693.77833933)(146.44826858,693.86333925)(146.42826912,693.95334076)
\curveto(146.40826862,694.00333911)(146.39826863,694.05333906)(146.39826912,694.10334076)
\curveto(146.40826862,694.15333896)(146.40826862,694.20333891)(146.39826912,694.25334076)
\curveto(146.38826864,694.28333883)(146.37826865,694.34333877)(146.36826912,694.43334076)
\curveto(146.36826866,694.53333858)(146.37326865,694.60333851)(146.38326912,694.64334076)
\curveto(146.40326862,694.74333837)(146.41326861,694.82833828)(146.41326912,694.89834076)
\lineto(146.50326912,695.22834076)
\curveto(146.53326849,695.34833776)(146.57326845,695.45333766)(146.62326912,695.54334076)
\curveto(146.79326823,695.83333728)(146.98826804,696.05333706)(147.20826912,696.20334076)
\curveto(147.4282676,696.35333676)(147.70826732,696.48333663)(148.04826912,696.59334076)
\curveto(148.17826685,696.64333647)(148.31326671,696.67833643)(148.45326912,696.69834076)
\curveto(148.59326643,696.71833639)(148.73326629,696.74333637)(148.87326912,696.77334076)
\curveto(148.95326607,696.79333632)(149.03826599,696.80333631)(149.12826912,696.80334076)
\curveto(149.21826581,696.8133363)(149.30826572,696.82833628)(149.39826912,696.84834076)
\curveto(149.46826556,696.86833624)(149.53826549,696.87333624)(149.60826912,696.86334076)
\curveto(149.67826535,696.86333625)(149.75326527,696.87333624)(149.83326912,696.89334076)
\curveto(149.90326512,696.9133362)(149.97326505,696.92333619)(150.04326912,696.92334076)
\curveto(150.11326491,696.92333619)(150.18826484,696.93333618)(150.26826912,696.95334076)
\curveto(150.47826455,697.00333611)(150.66826436,697.04333607)(150.83826912,697.07334076)
\curveto(151.01826401,697.113336)(151.17826385,697.20333591)(151.31826912,697.34334076)
\curveto(151.40826362,697.43333568)(151.46826356,697.53333558)(151.49826912,697.64334076)
\curveto(151.50826352,697.67333544)(151.50826352,697.69833541)(151.49826912,697.71834076)
\curveto(151.49826353,697.73833537)(151.50326352,697.75833535)(151.51326912,697.77834076)
\curveto(151.5232635,697.79833531)(151.5282635,697.82833528)(151.52826912,697.86834076)
\lineto(151.52826912,697.95834076)
\lineto(151.49826912,698.07834076)
\curveto(151.49826353,698.11833499)(151.49326353,698.15333496)(151.48326912,698.18334076)
\curveto(151.38326364,698.48333463)(151.17326385,698.68833442)(150.85326912,698.79834076)
\curveto(150.76326426,698.82833428)(150.65326437,698.84833426)(150.52326912,698.85834076)
\curveto(150.40326462,698.87833423)(150.27826475,698.88333423)(150.14826912,698.87334076)
\curveto(150.01826501,698.87333424)(149.89326513,698.86333425)(149.77326912,698.84334076)
\curveto(149.65326537,698.82333429)(149.54826548,698.79833431)(149.45826912,698.76834076)
\curveto(149.39826563,698.74833436)(149.33826569,698.71833439)(149.27826912,698.67834076)
\curveto(149.2282658,698.64833446)(149.17826585,698.6133345)(149.12826912,698.57334076)
\curveto(149.07826595,698.53333458)(149.023266,698.47833463)(148.96326912,698.40834076)
\curveto(148.91326611,698.33833477)(148.87826615,698.27333484)(148.85826912,698.21334076)
\curveto(148.80826622,698.113335)(148.76326626,698.01833509)(148.72326912,697.92834076)
\curveto(148.69326633,697.83833527)(148.6232664,697.77833533)(148.51326912,697.74834076)
\curveto(148.43326659,697.72833538)(148.34826668,697.71833539)(148.25826912,697.71834076)
\lineto(147.98826912,697.71834076)
\lineto(147.41826912,697.71834076)
\curveto(147.36826766,697.71833539)(147.31826771,697.7133354)(147.26826912,697.70334076)
\curveto(147.21826781,697.70333541)(147.17326785,697.7083354)(147.13326912,697.71834076)
\lineto(146.99826912,697.71834076)
\curveto(146.97826805,697.72833538)(146.95326807,697.73333538)(146.92326912,697.73334076)
\curveto(146.89326813,697.73333538)(146.86826816,697.74333537)(146.84826912,697.76334076)
\curveto(146.76826826,697.78333533)(146.71326831,697.84833526)(146.68326912,697.95834076)
\curveto(146.67326835,698.0083351)(146.67326835,698.05833505)(146.68326912,698.10834076)
\curveto(146.69326833,698.15833495)(146.70326832,698.20333491)(146.71326912,698.24334076)
\curveto(146.74326828,698.35333476)(146.77326825,698.45333466)(146.80326912,698.54334076)
\curveto(146.84326818,698.64333447)(146.88826814,698.73333438)(146.93826912,698.81334076)
\lineto(147.02826912,698.96334076)
\lineto(147.11826912,699.11334076)
\curveto(147.19826783,699.22333389)(147.29826773,699.32833378)(147.41826912,699.42834076)
\curveto(147.43826759,699.43833367)(147.46826756,699.46333365)(147.50826912,699.50334076)
\curveto(147.55826747,699.54333357)(147.60326742,699.57833353)(147.64326912,699.60834076)
\curveto(147.68326734,699.63833347)(147.7282673,699.66833344)(147.77826912,699.69834076)
\curveto(147.94826708,699.8083333)(148.1282669,699.89333322)(148.31826912,699.95334076)
\curveto(148.50826652,700.02333309)(148.70326632,700.08833302)(148.90326912,700.14834076)
\curveto(149.023266,700.17833293)(149.14826588,700.19833291)(149.27826912,700.20834076)
\curveto(149.40826562,700.21833289)(149.53826549,700.23833287)(149.66826912,700.26834076)
\curveto(149.70826532,700.27833283)(149.76826526,700.27833283)(149.84826912,700.26834076)
\curveto(149.93826509,700.25833285)(149.99326503,700.26333285)(150.01326912,700.28334076)
\curveto(150.4232646,700.29333282)(150.81326421,700.27833283)(151.18326912,700.23834076)
\curveto(151.56326346,700.19833291)(151.90326312,700.12333299)(152.20326912,700.01334076)
\curveto(152.51326251,699.90333321)(152.77826225,699.75333336)(152.99826912,699.56334076)
\curveto(153.21826181,699.38333373)(153.38826164,699.14833396)(153.50826912,698.85834076)
\curveto(153.57826145,698.68833442)(153.61826141,698.49333462)(153.62826912,698.27334076)
\curveto(153.63826139,698.05333506)(153.64326138,697.82833528)(153.64326912,697.59834076)
\lineto(153.64326912,694.25334076)
\lineto(153.64326912,693.66834076)
\curveto(153.64326138,693.47833963)(153.66326136,693.30333981)(153.70326912,693.14334076)
\curveto(153.71326131,693.11334)(153.71826131,693.07834003)(153.71826912,693.03834076)
\curveto(153.71826131,693.0083401)(153.7232613,692.97834013)(153.73326912,692.94834076)
\moveto(151.52826912,695.25834076)
\curveto(151.53826349,695.3083378)(151.54326348,695.36333775)(151.54326912,695.42334076)
\curveto(151.54326348,695.49333762)(151.53826349,695.55333756)(151.52826912,695.60334076)
\curveto(151.50826352,695.66333745)(151.49826353,695.71833739)(151.49826912,695.76834076)
\curveto(151.49826353,695.81833729)(151.47826355,695.85833725)(151.43826912,695.88834076)
\curveto(151.38826364,695.92833718)(151.31326371,695.94833716)(151.21326912,695.94834076)
\curveto(151.17326385,695.93833717)(151.13826389,695.92833718)(151.10826912,695.91834076)
\curveto(151.07826395,695.91833719)(151.04326398,695.9133372)(151.00326912,695.90334076)
\curveto(150.93326409,695.88333723)(150.85826417,695.86833724)(150.77826912,695.85834076)
\curveto(150.69826433,695.84833726)(150.61826441,695.83333728)(150.53826912,695.81334076)
\curveto(150.50826452,695.80333731)(150.46326456,695.79833731)(150.40326912,695.79834076)
\curveto(150.27326475,695.76833734)(150.14326488,695.74833736)(150.01326912,695.73834076)
\curveto(149.88326514,695.72833738)(149.75826527,695.70333741)(149.63826912,695.66334076)
\curveto(149.55826547,695.64333747)(149.48326554,695.62333749)(149.41326912,695.60334076)
\curveto(149.34326568,695.59333752)(149.27326575,695.57333754)(149.20326912,695.54334076)
\curveto(148.99326603,695.45333766)(148.81326621,695.31833779)(148.66326912,695.13834076)
\curveto(148.5232665,694.95833815)(148.47326655,694.7083384)(148.51326912,694.38834076)
\curveto(148.53326649,694.21833889)(148.58826644,694.07833903)(148.67826912,693.96834076)
\curveto(148.74826628,693.85833925)(148.85326617,693.76833934)(148.99326912,693.69834076)
\curveto(149.13326589,693.63833947)(149.28326574,693.59333952)(149.44326912,693.56334076)
\curveto(149.61326541,693.53333958)(149.78826524,693.52333959)(149.96826912,693.53334076)
\curveto(150.15826487,693.55333956)(150.33326469,693.58833952)(150.49326912,693.63834076)
\curveto(150.75326427,693.71833939)(150.95826407,693.84333927)(151.10826912,694.01334076)
\curveto(151.25826377,694.19333892)(151.37326365,694.4133387)(151.45326912,694.67334076)
\curveto(151.47326355,694.74333837)(151.48326354,694.8133383)(151.48326912,694.88334076)
\curveto(151.49326353,694.96333815)(151.50826352,695.04333807)(151.52826912,695.12334076)
\lineto(151.52826912,695.25834076)
}
}
{
\newrgbcolor{curcolor}{0 0 0}
\pscustom[linestyle=none,fillstyle=solid,fillcolor=curcolor]
{
\newpath
\moveto(163.09655037,696.60834076)
\curveto(163.11654177,696.54833656)(163.12654176,696.44333667)(163.12655037,696.29334076)
\curveto(163.12654176,696.15333696)(163.12154177,696.05333706)(163.11155037,695.99334076)
\curveto(163.11154178,695.94333717)(163.10654178,695.89833721)(163.09655037,695.85834076)
\lineto(163.09655037,695.73834076)
\curveto(163.07654181,695.65833745)(163.06654182,695.57833753)(163.06655037,695.49834076)
\curveto(163.06654182,695.42833768)(163.05654183,695.35333776)(163.03655037,695.27334076)
\curveto(163.03654185,695.23333788)(163.02654186,695.16333795)(163.00655037,695.06334076)
\curveto(162.97654191,694.94333817)(162.94654194,694.81833829)(162.91655037,694.68834076)
\curveto(162.89654199,694.56833854)(162.86154203,694.45333866)(162.81155037,694.34334076)
\curveto(162.63154226,693.89333922)(162.40654248,693.50333961)(162.13655037,693.17334076)
\curveto(161.86654302,692.84334027)(161.51154338,692.58334053)(161.07155037,692.39334076)
\curveto(160.98154391,692.35334076)(160.886544,692.32334079)(160.78655037,692.30334076)
\curveto(160.69654419,692.27334084)(160.59654429,692.24334087)(160.48655037,692.21334076)
\curveto(160.42654446,692.19334092)(160.36154453,692.18334093)(160.29155037,692.18334076)
\curveto(160.23154466,692.18334093)(160.17154472,692.17834093)(160.11155037,692.16834076)
\lineto(159.97655037,692.16834076)
\curveto(159.91654497,692.14834096)(159.83654505,692.14334097)(159.73655037,692.15334076)
\curveto(159.63654525,692.15334096)(159.55654533,692.16334095)(159.49655037,692.18334076)
\lineto(159.40655037,692.18334076)
\curveto(159.35654553,692.19334092)(159.30154559,692.20334091)(159.24155037,692.21334076)
\curveto(159.18154571,692.2133409)(159.12154577,692.21834089)(159.06155037,692.22834076)
\curveto(158.87154602,692.27834083)(158.69654619,692.32834078)(158.53655037,692.37834076)
\curveto(158.37654651,692.42834068)(158.22654666,692.49834061)(158.08655037,692.58834076)
\lineto(157.90655037,692.70834076)
\curveto(157.85654703,692.74834036)(157.80654708,692.79334032)(157.75655037,692.84334076)
\lineto(157.66655037,692.90334076)
\curveto(157.63654725,692.92334019)(157.60654728,692.93834017)(157.57655037,692.94834076)
\curveto(157.4865474,692.97834013)(157.43154746,692.95834015)(157.41155037,692.88834076)
\curveto(157.36154753,692.81834029)(157.32654756,692.73334038)(157.30655037,692.63334076)
\curveto(157.29654759,692.54334057)(157.26154763,692.47334064)(157.20155037,692.42334076)
\curveto(157.14154775,692.38334073)(157.07154782,692.35834075)(156.99155037,692.34834076)
\lineto(156.72155037,692.34834076)
\lineto(156.00155037,692.34834076)
\lineto(155.77655037,692.34834076)
\curveto(155.70654918,692.33834077)(155.64154925,692.34334077)(155.58155037,692.36334076)
\curveto(155.44154945,692.4133407)(155.36154953,692.50334061)(155.34155037,692.63334076)
\curveto(155.33154956,692.77334034)(155.32654956,692.92834018)(155.32655037,693.09834076)
\lineto(155.32655037,702.24834076)
\lineto(155.32655037,702.59334076)
\curveto(155.32654956,702.7133304)(155.35154954,702.8083303)(155.40155037,702.87834076)
\curveto(155.44154945,702.94833016)(155.51154938,702.99333012)(155.61155037,703.01334076)
\curveto(155.63154926,703.02333009)(155.65154924,703.02333009)(155.67155037,703.01334076)
\curveto(155.70154919,703.0133301)(155.72654916,703.01833009)(155.74655037,703.02834076)
\lineto(156.69155037,703.02834076)
\curveto(156.87154802,703.02833008)(157.02654786,703.01833009)(157.15655037,702.99834076)
\curveto(157.2865476,702.98833012)(157.37154752,702.9133302)(157.41155037,702.77334076)
\curveto(157.44154745,702.67333044)(157.45154744,702.53833057)(157.44155037,702.36834076)
\curveto(157.43154746,702.2083309)(157.42654746,702.06833104)(157.42655037,701.94834076)
\lineto(157.42655037,700.31334076)
\lineto(157.42655037,699.98334076)
\curveto(157.42654746,699.87333324)(157.43654745,699.77833333)(157.45655037,699.69834076)
\curveto(157.46654742,699.64833346)(157.47654741,699.60333351)(157.48655037,699.56334076)
\curveto(157.49654739,699.53333358)(157.52154737,699.5133336)(157.56155037,699.50334076)
\curveto(157.58154731,699.48333363)(157.60654728,699.47333364)(157.63655037,699.47334076)
\curveto(157.67654721,699.47333364)(157.70654718,699.47833363)(157.72655037,699.48834076)
\curveto(157.79654709,699.52833358)(157.86154703,699.56833354)(157.92155037,699.60834076)
\curveto(157.98154691,699.65833345)(158.04654684,699.7083334)(158.11655037,699.75834076)
\curveto(158.24654664,699.84833326)(158.38154651,699.92333319)(158.52155037,699.98334076)
\curveto(158.66154623,700.05333306)(158.81654607,700.113333)(158.98655037,700.16334076)
\curveto(159.06654582,700.19333292)(159.14654574,700.2083329)(159.22655037,700.20834076)
\curveto(159.30654558,700.21833289)(159.3865455,700.23333288)(159.46655037,700.25334076)
\curveto(159.53654535,700.27333284)(159.61154528,700.28333283)(159.69155037,700.28334076)
\lineto(159.93155037,700.28334076)
\lineto(160.08155037,700.28334076)
\curveto(160.11154478,700.27333284)(160.14654474,700.26833284)(160.18655037,700.26834076)
\curveto(160.22654466,700.27833283)(160.26654462,700.27833283)(160.30655037,700.26834076)
\curveto(160.41654447,700.23833287)(160.51654437,700.2133329)(160.60655037,700.19334076)
\curveto(160.70654418,700.18333293)(160.80154409,700.15833295)(160.89155037,700.11834076)
\curveto(161.35154354,699.92833318)(161.72654316,699.68333343)(162.01655037,699.38334076)
\curveto(162.30654258,699.08333403)(162.55154234,698.7083344)(162.75155037,698.25834076)
\curveto(162.80154209,698.13833497)(162.84154205,698.0133351)(162.87155037,697.88334076)
\curveto(162.91154198,697.75333536)(162.95154194,697.61833549)(162.99155037,697.47834076)
\curveto(163.01154188,697.4083357)(163.02154187,697.33833577)(163.02155037,697.26834076)
\curveto(163.03154186,697.2083359)(163.04654184,697.13833597)(163.06655037,697.05834076)
\curveto(163.0865418,697.0083361)(163.0915418,696.95333616)(163.08155037,696.89334076)
\curveto(163.08154181,696.83333628)(163.0865418,696.77333634)(163.09655037,696.71334076)
\lineto(163.09655037,696.60834076)
\moveto(160.87655037,695.19834076)
\curveto(160.90654398,695.29833781)(160.93154396,695.42333769)(160.95155037,695.57334076)
\curveto(160.98154391,695.72333739)(160.99654389,695.87333724)(160.99655037,696.02334076)
\curveto(161.00654388,696.18333693)(161.00654388,696.33833677)(160.99655037,696.48834076)
\curveto(160.99654389,696.64833646)(160.98154391,696.78333633)(160.95155037,696.89334076)
\curveto(160.92154397,696.99333612)(160.90154399,697.08833602)(160.89155037,697.17834076)
\curveto(160.88154401,697.26833584)(160.85654403,697.35333576)(160.81655037,697.43334076)
\curveto(160.67654421,697.78333533)(160.47654441,698.07833503)(160.21655037,698.31834076)
\curveto(159.96654492,698.56833454)(159.59654529,698.69333442)(159.10655037,698.69334076)
\curveto(159.06654582,698.69333442)(159.03154586,698.68833442)(159.00155037,698.67834076)
\lineto(158.89655037,698.67834076)
\curveto(158.82654606,698.65833445)(158.76154613,698.63833447)(158.70155037,698.61834076)
\curveto(158.64154625,698.6083345)(158.58154631,698.59333452)(158.52155037,698.57334076)
\curveto(158.23154666,698.44333467)(158.01154688,698.25833485)(157.86155037,698.01834076)
\curveto(157.71154718,697.78833532)(157.5865473,697.52333559)(157.48655037,697.22334076)
\curveto(157.45654743,697.14333597)(157.43654745,697.05833605)(157.42655037,696.96834076)
\curveto(157.42654746,696.88833622)(157.41654747,696.8083363)(157.39655037,696.72834076)
\curveto(157.3865475,696.69833641)(157.38154751,696.64833646)(157.38155037,696.57834076)
\curveto(157.37154752,696.53833657)(157.36654752,696.49833661)(157.36655037,696.45834076)
\curveto(157.37654751,696.41833669)(157.37654751,696.37833673)(157.36655037,696.33834076)
\curveto(157.34654754,696.25833685)(157.34154755,696.14833696)(157.35155037,696.00834076)
\curveto(157.36154753,695.86833724)(157.37654751,695.76833734)(157.39655037,695.70834076)
\curveto(157.41654747,695.61833749)(157.42654746,695.53333758)(157.42655037,695.45334076)
\curveto(157.43654745,695.37333774)(157.45654743,695.29333782)(157.48655037,695.21334076)
\curveto(157.57654731,694.93333818)(157.68154721,694.68833842)(157.80155037,694.47834076)
\curveto(157.93154696,694.27833883)(158.11154678,694.108339)(158.34155037,693.96834076)
\curveto(158.50154639,693.86833924)(158.66654622,693.79833931)(158.83655037,693.75834076)
\curveto(158.85654603,693.75833935)(158.87654601,693.75333936)(158.89655037,693.74334076)
\lineto(158.98655037,693.74334076)
\curveto(159.01654587,693.73333938)(159.06654582,693.72333939)(159.13655037,693.71334076)
\curveto(159.20654568,693.7133394)(159.26654562,693.71833939)(159.31655037,693.72834076)
\curveto(159.41654547,693.74833936)(159.50654538,693.76333935)(159.58655037,693.77334076)
\curveto(159.67654521,693.79333932)(159.76154513,693.81833929)(159.84155037,693.84834076)
\curveto(160.12154477,693.97833913)(160.33654455,694.15833895)(160.48655037,694.38834076)
\curveto(160.64654424,694.61833849)(160.77654411,694.88833822)(160.87655037,695.19834076)
}
}
{
\newrgbcolor{curcolor}{0 0 0}
\pscustom[linestyle=none,fillstyle=solid,fillcolor=curcolor]
{
\newpath
\moveto(164.97647225,703.04334076)
\lineto(166.07147225,703.04334076)
\curveto(166.17146976,703.04333007)(166.26646967,703.03833007)(166.35647225,703.02834076)
\curveto(166.44646949,703.01833009)(166.51646942,702.98833012)(166.56647225,702.93834076)
\curveto(166.62646931,702.86833024)(166.65646928,702.77333034)(166.65647225,702.65334076)
\curveto(166.66646927,702.54333057)(166.67146926,702.42833068)(166.67147225,702.30834076)
\lineto(166.67147225,700.97334076)
\lineto(166.67147225,695.58834076)
\lineto(166.67147225,693.29334076)
\lineto(166.67147225,692.87334076)
\curveto(166.68146925,692.72334039)(166.66146927,692.6083405)(166.61147225,692.52834076)
\curveto(166.56146937,692.44834066)(166.47146946,692.39334072)(166.34147225,692.36334076)
\curveto(166.28146965,692.34334077)(166.21146972,692.33834077)(166.13147225,692.34834076)
\curveto(166.06146987,692.35834075)(165.99146994,692.36334075)(165.92147225,692.36334076)
\lineto(165.20147225,692.36334076)
\curveto(165.09147084,692.36334075)(164.99147094,692.36834074)(164.90147225,692.37834076)
\curveto(164.81147112,692.38834072)(164.7364712,692.41834069)(164.67647225,692.46834076)
\curveto(164.61647132,692.51834059)(164.58147135,692.59334052)(164.57147225,692.69334076)
\lineto(164.57147225,693.02334076)
\lineto(164.57147225,694.35834076)
\lineto(164.57147225,699.98334076)
\lineto(164.57147225,702.02334076)
\curveto(164.57147136,702.15333096)(164.56647137,702.3083308)(164.55647225,702.48834076)
\curveto(164.55647138,702.66833044)(164.58147135,702.79833031)(164.63147225,702.87834076)
\curveto(164.65147128,702.91833019)(164.67647126,702.94833016)(164.70647225,702.96834076)
\lineto(164.82647225,703.02834076)
\curveto(164.84647109,703.02833008)(164.87147106,703.02833008)(164.90147225,703.02834076)
\curveto(164.931471,703.03833007)(164.95647098,703.04333007)(164.97647225,703.04334076)
}
}
{
\newrgbcolor{curcolor}{0 0 0}
\pscustom[linestyle=none,fillstyle=solid,fillcolor=curcolor]
{
\newpath
\moveto(175.69865975,696.29334076)
\curveto(175.71865158,696.2133369)(175.71865158,696.12333699)(175.69865975,696.02334076)
\curveto(175.67865162,695.92333719)(175.64365166,695.85833725)(175.59365975,695.82834076)
\curveto(175.54365176,695.78833732)(175.46865183,695.75833735)(175.36865975,695.73834076)
\curveto(175.27865202,695.72833738)(175.17365213,695.71833739)(175.05365975,695.70834076)
\lineto(174.70865975,695.70834076)
\curveto(174.5986527,695.71833739)(174.4986528,695.72333739)(174.40865975,695.72334076)
\lineto(170.74865975,695.72334076)
\lineto(170.53865975,695.72334076)
\curveto(170.47865682,695.72333739)(170.42365688,695.7133374)(170.37365975,695.69334076)
\curveto(170.29365701,695.65333746)(170.24365706,695.6133375)(170.22365975,695.57334076)
\curveto(170.2036571,695.55333756)(170.18365712,695.5133376)(170.16365975,695.45334076)
\curveto(170.14365716,695.40333771)(170.13865716,695.35333776)(170.14865975,695.30334076)
\curveto(170.16865713,695.24333787)(170.17865712,695.18333793)(170.17865975,695.12334076)
\curveto(170.18865711,695.07333804)(170.2036571,695.01833809)(170.22365975,694.95834076)
\curveto(170.303657,694.71833839)(170.3986569,694.51833859)(170.50865975,694.35834076)
\curveto(170.62865667,694.2083389)(170.78865651,694.07333904)(170.98865975,693.95334076)
\curveto(171.06865623,693.90333921)(171.14865615,693.86833924)(171.22865975,693.84834076)
\curveto(171.31865598,693.83833927)(171.40865589,693.81833929)(171.49865975,693.78834076)
\curveto(171.57865572,693.76833934)(171.68865561,693.75333936)(171.82865975,693.74334076)
\curveto(171.96865533,693.73333938)(172.08865521,693.73833937)(172.18865975,693.75834076)
\lineto(172.32365975,693.75834076)
\curveto(172.42365488,693.77833933)(172.51365479,693.79833931)(172.59365975,693.81834076)
\curveto(172.68365462,693.84833926)(172.76865453,693.87833923)(172.84865975,693.90834076)
\curveto(172.94865435,693.95833915)(173.05865424,694.02333909)(173.17865975,694.10334076)
\curveto(173.30865399,694.18333893)(173.4036539,694.26333885)(173.46365975,694.34334076)
\curveto(173.51365379,694.4133387)(173.56365374,694.47833863)(173.61365975,694.53834076)
\curveto(173.67365363,694.6083385)(173.74365356,694.65833845)(173.82365975,694.68834076)
\curveto(173.92365338,694.73833837)(174.04865325,694.75833835)(174.19865975,694.74834076)
\lineto(174.63365975,694.74834076)
\lineto(174.81365975,694.74834076)
\curveto(174.88365242,694.75833835)(174.94365236,694.75333836)(174.99365975,694.73334076)
\lineto(175.14365975,694.73334076)
\curveto(175.24365206,694.7133384)(175.31365199,694.68833842)(175.35365975,694.65834076)
\curveto(175.39365191,694.63833847)(175.41365189,694.59333852)(175.41365975,694.52334076)
\curveto(175.42365188,694.45333866)(175.41865188,694.39333872)(175.39865975,694.34334076)
\curveto(175.34865195,694.20333891)(175.29365201,694.07833903)(175.23365975,693.96834076)
\curveto(175.17365213,693.85833925)(175.1036522,693.74833936)(175.02365975,693.63834076)
\curveto(174.8036525,693.3083398)(174.55365275,693.04334007)(174.27365975,692.84334076)
\curveto(173.99365331,692.64334047)(173.64365366,692.47334064)(173.22365975,692.33334076)
\curveto(173.11365419,692.29334082)(173.0036543,692.26834084)(172.89365975,692.25834076)
\curveto(172.78365452,692.24834086)(172.66865463,692.22834088)(172.54865975,692.19834076)
\curveto(172.50865479,692.18834092)(172.46365484,692.18834092)(172.41365975,692.19834076)
\curveto(172.37365493,692.19834091)(172.33365497,692.19334092)(172.29365975,692.18334076)
\lineto(172.12865975,692.18334076)
\curveto(172.07865522,692.16334095)(172.01865528,692.15834095)(171.94865975,692.16834076)
\curveto(171.88865541,692.16834094)(171.83365547,692.17334094)(171.78365975,692.18334076)
\curveto(171.7036556,692.19334092)(171.63365567,692.19334092)(171.57365975,692.18334076)
\curveto(171.51365579,692.17334094)(171.44865585,692.17834093)(171.37865975,692.19834076)
\curveto(171.32865597,692.21834089)(171.27365603,692.22834088)(171.21365975,692.22834076)
\curveto(171.15365615,692.22834088)(171.0986562,692.23834087)(171.04865975,692.25834076)
\curveto(170.93865636,692.27834083)(170.82865647,692.30334081)(170.71865975,692.33334076)
\curveto(170.60865669,692.35334076)(170.50865679,692.38834072)(170.41865975,692.43834076)
\curveto(170.30865699,692.47834063)(170.2036571,692.5133406)(170.10365975,692.54334076)
\curveto(170.01365729,692.58334053)(169.92865737,692.62834048)(169.84865975,692.67834076)
\curveto(169.52865777,692.87834023)(169.24365806,693.10834)(168.99365975,693.36834076)
\curveto(168.74365856,693.63833947)(168.53865876,693.94833916)(168.37865975,694.29834076)
\curveto(168.32865897,694.4083387)(168.28865901,694.51833859)(168.25865975,694.62834076)
\curveto(168.22865907,694.74833836)(168.18865911,694.86833824)(168.13865975,694.98834076)
\curveto(168.12865917,695.02833808)(168.12365918,695.06333805)(168.12365975,695.09334076)
\curveto(168.12365918,695.13333798)(168.11865918,695.17333794)(168.10865975,695.21334076)
\curveto(168.06865923,695.33333778)(168.04365926,695.46333765)(168.03365975,695.60334076)
\lineto(168.00365975,696.02334076)
\curveto(168.0036593,696.07333704)(167.9986593,696.12833698)(167.98865975,696.18834076)
\curveto(167.98865931,696.24833686)(167.99365931,696.30333681)(168.00365975,696.35334076)
\lineto(168.00365975,696.53334076)
\lineto(168.04865975,696.89334076)
\curveto(168.08865921,697.06333605)(168.12365918,697.22833588)(168.15365975,697.38834076)
\curveto(168.18365912,697.54833556)(168.22865907,697.69833541)(168.28865975,697.83834076)
\curveto(168.71865858,698.87833423)(169.44865785,699.6133335)(170.47865975,700.04334076)
\curveto(170.61865668,700.10333301)(170.75865654,700.14333297)(170.89865975,700.16334076)
\curveto(171.04865625,700.19333292)(171.2036561,700.22833288)(171.36365975,700.26834076)
\curveto(171.44365586,700.27833283)(171.51865578,700.28333283)(171.58865975,700.28334076)
\curveto(171.65865564,700.28333283)(171.73365557,700.28833282)(171.81365975,700.29834076)
\curveto(172.32365498,700.3083328)(172.75865454,700.24833286)(173.11865975,700.11834076)
\curveto(173.48865381,699.99833311)(173.81865348,699.83833327)(174.10865975,699.63834076)
\curveto(174.1986531,699.57833353)(174.28865301,699.5083336)(174.37865975,699.42834076)
\curveto(174.46865283,699.35833375)(174.54865275,699.28333383)(174.61865975,699.20334076)
\curveto(174.64865265,699.15333396)(174.68865261,699.113334)(174.73865975,699.08334076)
\curveto(174.81865248,698.97333414)(174.89365241,698.85833425)(174.96365975,698.73834076)
\curveto(175.03365227,698.62833448)(175.10865219,698.5133346)(175.18865975,698.39334076)
\curveto(175.23865206,698.30333481)(175.27865202,698.2083349)(175.30865975,698.10834076)
\curveto(175.34865195,698.01833509)(175.38865191,697.91833519)(175.42865975,697.80834076)
\curveto(175.47865182,697.67833543)(175.51865178,697.54333557)(175.54865975,697.40334076)
\curveto(175.57865172,697.26333585)(175.61365169,697.12333599)(175.65365975,696.98334076)
\curveto(175.67365163,696.90333621)(175.67865162,696.8133363)(175.66865975,696.71334076)
\curveto(175.66865163,696.62333649)(175.67865162,696.53833657)(175.69865975,696.45834076)
\lineto(175.69865975,696.29334076)
\moveto(173.44865975,697.17834076)
\curveto(173.51865378,697.27833583)(173.52365378,697.39833571)(173.46365975,697.53834076)
\curveto(173.41365389,697.68833542)(173.37365393,697.79833531)(173.34365975,697.86834076)
\curveto(173.2036541,698.13833497)(173.01865428,698.34333477)(172.78865975,698.48334076)
\curveto(172.55865474,698.63333448)(172.23865506,698.7133344)(171.82865975,698.72334076)
\curveto(171.7986555,698.70333441)(171.76365554,698.69833441)(171.72365975,698.70834076)
\curveto(171.68365562,698.71833439)(171.64865565,698.71833439)(171.61865975,698.70834076)
\curveto(171.56865573,698.68833442)(171.51365579,698.67333444)(171.45365975,698.66334076)
\curveto(171.39365591,698.66333445)(171.33865596,698.65333446)(171.28865975,698.63334076)
\curveto(170.84865645,698.49333462)(170.52365678,698.21833489)(170.31365975,697.80834076)
\curveto(170.29365701,697.76833534)(170.26865703,697.7133354)(170.23865975,697.64334076)
\curveto(170.21865708,697.58333553)(170.2036571,697.51833559)(170.19365975,697.44834076)
\curveto(170.18365712,697.38833572)(170.18365712,697.32833578)(170.19365975,697.26834076)
\curveto(170.21365709,697.2083359)(170.24865705,697.15833595)(170.29865975,697.11834076)
\curveto(170.37865692,697.06833604)(170.48865681,697.04333607)(170.62865975,697.04334076)
\lineto(171.03365975,697.04334076)
\lineto(172.69865975,697.04334076)
\lineto(173.13365975,697.04334076)
\curveto(173.29365401,697.05333606)(173.3986539,697.09833601)(173.44865975,697.17834076)
}
}
{
\newrgbcolor{curcolor}{0 0 0}
\pscustom[linestyle=none,fillstyle=solid,fillcolor=curcolor]
{
\newpath
\moveto(180.516941,700.29834076)
\curveto(181.32693584,700.31833279)(182.00193516,700.19833291)(182.541941,699.93834076)
\curveto(183.09193407,699.67833343)(183.52693364,699.3083338)(183.846941,698.82834076)
\curveto(184.00693316,698.58833452)(184.12693304,698.3133348)(184.206941,698.00334076)
\curveto(184.22693294,697.95333516)(184.24193292,697.88833522)(184.251941,697.80834076)
\curveto(184.27193289,697.72833538)(184.27193289,697.65833545)(184.251941,697.59834076)
\curveto(184.21193295,697.48833562)(184.14193302,697.42333569)(184.041941,697.40334076)
\curveto(183.94193322,697.39333572)(183.82193334,697.38833572)(183.681941,697.38834076)
\lineto(182.901941,697.38834076)
\lineto(182.616941,697.38834076)
\curveto(182.52693464,697.38833572)(182.45193471,697.4083357)(182.391941,697.44834076)
\curveto(182.31193485,697.48833562)(182.25693491,697.54833556)(182.226941,697.62834076)
\curveto(182.19693497,697.71833539)(182.15693501,697.8083353)(182.106941,697.89834076)
\curveto(182.04693512,698.0083351)(181.98193518,698.108335)(181.911941,698.19834076)
\curveto(181.84193532,698.28833482)(181.7619354,698.36833474)(181.671941,698.43834076)
\curveto(181.53193563,698.52833458)(181.37693579,698.59833451)(181.206941,698.64834076)
\curveto(181.14693602,698.66833444)(181.08693608,698.67833443)(181.026941,698.67834076)
\curveto(180.9669362,698.67833443)(180.91193625,698.68833442)(180.861941,698.70834076)
\lineto(180.711941,698.70834076)
\curveto(180.51193665,698.7083344)(180.35193681,698.68833442)(180.231941,698.64834076)
\curveto(179.94193722,698.55833455)(179.70693746,698.41833469)(179.526941,698.22834076)
\curveto(179.34693782,698.04833506)(179.20193796,697.82833528)(179.091941,697.56834076)
\curveto(179.04193812,697.45833565)(179.00193816,697.33833577)(178.971941,697.20834076)
\curveto(178.95193821,697.08833602)(178.92693824,696.95833615)(178.896941,696.81834076)
\curveto(178.88693828,696.77833633)(178.88193828,696.73833637)(178.881941,696.69834076)
\curveto(178.88193828,696.65833645)(178.87693829,696.61833649)(178.866941,696.57834076)
\curveto(178.84693832,696.47833663)(178.83693833,696.33833677)(178.836941,696.15834076)
\curveto(178.84693832,695.97833713)(178.8619383,695.83833727)(178.881941,695.73834076)
\curveto(178.88193828,695.65833745)(178.88693828,695.60333751)(178.896941,695.57334076)
\curveto(178.91693825,695.50333761)(178.92693824,695.43333768)(178.926941,695.36334076)
\curveto(178.93693823,695.29333782)(178.95193821,695.22333789)(178.971941,695.15334076)
\curveto(179.05193811,694.92333819)(179.14693802,694.7133384)(179.256941,694.52334076)
\curveto(179.3669378,694.33333878)(179.50693766,694.17333894)(179.676941,694.04334076)
\curveto(179.71693745,694.0133391)(179.77693739,693.97833913)(179.856941,693.93834076)
\curveto(179.9669372,693.86833924)(180.07693709,693.82333929)(180.186941,693.80334076)
\curveto(180.30693686,693.78333933)(180.45193671,693.76333935)(180.621941,693.74334076)
\lineto(180.711941,693.74334076)
\curveto(180.75193641,693.74333937)(180.78193638,693.74833936)(180.801941,693.75834076)
\lineto(180.936941,693.75834076)
\curveto(181.00693616,693.77833933)(181.07193609,693.79333932)(181.131941,693.80334076)
\curveto(181.20193596,693.82333929)(181.2669359,693.84333927)(181.326941,693.86334076)
\curveto(181.62693554,693.99333912)(181.85693531,694.18333893)(182.016941,694.43334076)
\curveto(182.05693511,694.48333863)(182.09193507,694.53833857)(182.121941,694.59834076)
\curveto(182.15193501,694.66833844)(182.17693499,694.72833838)(182.196941,694.77834076)
\curveto(182.23693493,694.88833822)(182.27193489,694.98333813)(182.301941,695.06334076)
\curveto(182.33193483,695.15333796)(182.40193476,695.22333789)(182.511941,695.27334076)
\curveto(182.60193456,695.3133378)(182.74693442,695.32833778)(182.946941,695.31834076)
\lineto(183.441941,695.31834076)
\lineto(183.651941,695.31834076)
\curveto(183.73193343,695.32833778)(183.79693337,695.32333779)(183.846941,695.30334076)
\lineto(183.966941,695.30334076)
\lineto(184.086941,695.27334076)
\curveto(184.12693304,695.27333784)(184.15693301,695.26333785)(184.176941,695.24334076)
\curveto(184.22693294,695.20333791)(184.25693291,695.14333797)(184.266941,695.06334076)
\curveto(184.28693288,694.99333812)(184.28693288,694.91833819)(184.266941,694.83834076)
\curveto(184.17693299,694.5083386)(184.0669331,694.2133389)(183.936941,693.95334076)
\curveto(183.52693364,693.18333993)(182.87193429,692.64834046)(181.971941,692.34834076)
\curveto(181.87193529,692.31834079)(181.7669354,692.29834081)(181.656941,692.28834076)
\curveto(181.54693562,692.26834084)(181.43693573,692.24334087)(181.326941,692.21334076)
\curveto(181.2669359,692.20334091)(181.20693596,692.19834091)(181.146941,692.19834076)
\curveto(181.08693608,692.19834091)(181.02693614,692.19334092)(180.966941,692.18334076)
\lineto(180.801941,692.18334076)
\curveto(180.75193641,692.16334095)(180.67693649,692.15834095)(180.576941,692.16834076)
\curveto(180.47693669,692.16834094)(180.40193676,692.17334094)(180.351941,692.18334076)
\curveto(180.27193689,692.20334091)(180.19693697,692.2133409)(180.126941,692.21334076)
\curveto(180.0669371,692.20334091)(180.00193716,692.2083409)(179.931941,692.22834076)
\lineto(179.781941,692.25834076)
\curveto(179.73193743,692.25834085)(179.68193748,692.26334085)(179.631941,692.27334076)
\curveto(179.52193764,692.30334081)(179.41693775,692.33334078)(179.316941,692.36334076)
\curveto(179.21693795,692.39334072)(179.12193804,692.42834068)(179.031941,692.46834076)
\curveto(178.5619386,692.66834044)(178.166939,692.92334019)(177.846941,693.23334076)
\curveto(177.52693964,693.55333956)(177.2669399,693.94833916)(177.066941,694.41834076)
\curveto(177.01694015,694.5083386)(176.97694019,694.60333851)(176.946941,694.70334076)
\lineto(176.856941,695.03334076)
\curveto(176.84694032,695.07333804)(176.84194032,695.108338)(176.841941,695.13834076)
\curveto(176.84194032,695.17833793)(176.83194033,695.22333789)(176.811941,695.27334076)
\curveto(176.79194037,695.34333777)(176.78194038,695.4133377)(176.781941,695.48334076)
\curveto(176.78194038,695.56333755)(176.77194039,695.63833747)(176.751941,695.70834076)
\lineto(176.751941,695.96334076)
\curveto(176.73194043,696.0133371)(176.72194044,696.06833704)(176.721941,696.12834076)
\curveto(176.72194044,696.19833691)(176.73194043,696.25833685)(176.751941,696.30834076)
\curveto(176.7619404,696.35833675)(176.7619404,696.40333671)(176.751941,696.44334076)
\curveto(176.74194042,696.48333663)(176.74194042,696.52333659)(176.751941,696.56334076)
\curveto(176.77194039,696.63333648)(176.77694039,696.69833641)(176.766941,696.75834076)
\curveto(176.7669404,696.81833629)(176.77694039,696.87833623)(176.796941,696.93834076)
\curveto(176.84694032,697.11833599)(176.88694028,697.28833582)(176.916941,697.44834076)
\curveto(176.94694022,697.61833549)(176.99194017,697.78333533)(177.051941,697.94334076)
\curveto(177.27193989,698.45333466)(177.54693962,698.87833423)(177.876941,699.21834076)
\curveto(178.21693895,699.55833355)(178.64693852,699.83333328)(179.166941,700.04334076)
\curveto(179.30693786,700.10333301)(179.45193771,700.14333297)(179.601941,700.16334076)
\curveto(179.75193741,700.19333292)(179.90693726,700.22833288)(180.066941,700.26834076)
\curveto(180.14693702,700.27833283)(180.22193694,700.28333283)(180.291941,700.28334076)
\curveto(180.3619368,700.28333283)(180.43693673,700.28833282)(180.516941,700.29834076)
}
}
{
\newrgbcolor{curcolor}{0 0 0}
\pscustom[linestyle=none,fillstyle=solid,fillcolor=curcolor]
{
\newpath
\moveto(187.66022225,702.93834076)
\curveto(187.7302193,702.85833025)(187.76521926,702.73833037)(187.76522225,702.57834076)
\lineto(187.76522225,702.11334076)
\lineto(187.76522225,701.70834076)
\curveto(187.76521926,701.56833154)(187.7302193,701.47333164)(187.66022225,701.42334076)
\curveto(187.60021943,701.37333174)(187.52021951,701.34333177)(187.42022225,701.33334076)
\curveto(187.3302197,701.32333179)(187.2302198,701.31833179)(187.12022225,701.31834076)
\lineto(186.28022225,701.31834076)
\curveto(186.17022086,701.31833179)(186.07022096,701.32333179)(185.98022225,701.33334076)
\curveto(185.90022113,701.34333177)(185.8302212,701.37333174)(185.77022225,701.42334076)
\curveto(185.7302213,701.45333166)(185.70022133,701.5083316)(185.68022225,701.58834076)
\curveto(185.67022136,701.67833143)(185.66022137,701.77333134)(185.65022225,701.87334076)
\lineto(185.65022225,702.20334076)
\curveto(185.66022137,702.3133308)(185.66522136,702.4083307)(185.66522225,702.48834076)
\lineto(185.66522225,702.69834076)
\curveto(185.67522135,702.76833034)(185.69522133,702.82833028)(185.72522225,702.87834076)
\curveto(185.74522128,702.91833019)(185.77022126,702.94833016)(185.80022225,702.96834076)
\lineto(185.92022225,703.02834076)
\curveto(185.94022109,703.02833008)(185.96522106,703.02833008)(185.99522225,703.02834076)
\curveto(186.025221,703.03833007)(186.05022098,703.04333007)(186.07022225,703.04334076)
\lineto(187.16522225,703.04334076)
\curveto(187.26521976,703.04333007)(187.36021967,703.03833007)(187.45022225,703.02834076)
\curveto(187.54021949,703.01833009)(187.61021942,702.98833012)(187.66022225,702.93834076)
\moveto(187.76522225,693.17334076)
\curveto(187.76521926,692.97334014)(187.76021927,692.80334031)(187.75022225,692.66334076)
\curveto(187.74021929,692.52334059)(187.65021938,692.42834068)(187.48022225,692.37834076)
\curveto(187.42021961,692.35834075)(187.35521967,692.34834076)(187.28522225,692.34834076)
\curveto(187.21521981,692.35834075)(187.14021989,692.36334075)(187.06022225,692.36334076)
\lineto(186.22022225,692.36334076)
\curveto(186.1302209,692.36334075)(186.04022099,692.36834074)(185.95022225,692.37834076)
\curveto(185.87022116,692.38834072)(185.81022122,692.41834069)(185.77022225,692.46834076)
\curveto(185.71022132,692.53834057)(185.67522135,692.62334049)(185.66522225,692.72334076)
\lineto(185.66522225,693.06834076)
\lineto(185.66522225,699.39834076)
\lineto(185.66522225,699.69834076)
\curveto(185.66522136,699.79833331)(185.68522134,699.87833323)(185.72522225,699.93834076)
\curveto(185.78522124,700.0083331)(185.87022116,700.05333306)(185.98022225,700.07334076)
\curveto(186.00022103,700.08333303)(186.025221,700.08333303)(186.05522225,700.07334076)
\curveto(186.09522093,700.07333304)(186.1252209,700.07833303)(186.14522225,700.08834076)
\lineto(186.89522225,700.08834076)
\lineto(187.09022225,700.08834076)
\curveto(187.17021986,700.09833301)(187.23521979,700.09833301)(187.28522225,700.08834076)
\lineto(187.40522225,700.08834076)
\curveto(187.46521956,700.06833304)(187.52021951,700.05333306)(187.57022225,700.04334076)
\curveto(187.62021941,700.03333308)(187.66021937,700.00333311)(187.69022225,699.95334076)
\curveto(187.7302193,699.90333321)(187.75021928,699.83333328)(187.75022225,699.74334076)
\curveto(187.76021927,699.65333346)(187.76521926,699.55833355)(187.76522225,699.45834076)
\lineto(187.76522225,693.17334076)
}
}
{
\newrgbcolor{curcolor}{0 0 0}
\pscustom[linestyle=none,fillstyle=solid,fillcolor=curcolor]
{
\newpath
\moveto(196.79240975,696.29334076)
\curveto(196.81240158,696.2133369)(196.81240158,696.12333699)(196.79240975,696.02334076)
\curveto(196.77240162,695.92333719)(196.73740166,695.85833725)(196.68740975,695.82834076)
\curveto(196.63740176,695.78833732)(196.56240183,695.75833735)(196.46240975,695.73834076)
\curveto(196.37240202,695.72833738)(196.26740213,695.71833739)(196.14740975,695.70834076)
\lineto(195.80240975,695.70834076)
\curveto(195.6924027,695.71833739)(195.5924028,695.72333739)(195.50240975,695.72334076)
\lineto(191.84240975,695.72334076)
\lineto(191.63240975,695.72334076)
\curveto(191.57240682,695.72333739)(191.51740688,695.7133374)(191.46740975,695.69334076)
\curveto(191.38740701,695.65333746)(191.33740706,695.6133375)(191.31740975,695.57334076)
\curveto(191.2974071,695.55333756)(191.27740712,695.5133376)(191.25740975,695.45334076)
\curveto(191.23740716,695.40333771)(191.23240716,695.35333776)(191.24240975,695.30334076)
\curveto(191.26240713,695.24333787)(191.27240712,695.18333793)(191.27240975,695.12334076)
\curveto(191.28240711,695.07333804)(191.2974071,695.01833809)(191.31740975,694.95834076)
\curveto(191.397407,694.71833839)(191.4924069,694.51833859)(191.60240975,694.35834076)
\curveto(191.72240667,694.2083389)(191.88240651,694.07333904)(192.08240975,693.95334076)
\curveto(192.16240623,693.90333921)(192.24240615,693.86833924)(192.32240975,693.84834076)
\curveto(192.41240598,693.83833927)(192.50240589,693.81833929)(192.59240975,693.78834076)
\curveto(192.67240572,693.76833934)(192.78240561,693.75333936)(192.92240975,693.74334076)
\curveto(193.06240533,693.73333938)(193.18240521,693.73833937)(193.28240975,693.75834076)
\lineto(193.41740975,693.75834076)
\curveto(193.51740488,693.77833933)(193.60740479,693.79833931)(193.68740975,693.81834076)
\curveto(193.77740462,693.84833926)(193.86240453,693.87833923)(193.94240975,693.90834076)
\curveto(194.04240435,693.95833915)(194.15240424,694.02333909)(194.27240975,694.10334076)
\curveto(194.40240399,694.18333893)(194.4974039,694.26333885)(194.55740975,694.34334076)
\curveto(194.60740379,694.4133387)(194.65740374,694.47833863)(194.70740975,694.53834076)
\curveto(194.76740363,694.6083385)(194.83740356,694.65833845)(194.91740975,694.68834076)
\curveto(195.01740338,694.73833837)(195.14240325,694.75833835)(195.29240975,694.74834076)
\lineto(195.72740975,694.74834076)
\lineto(195.90740975,694.74834076)
\curveto(195.97740242,694.75833835)(196.03740236,694.75333836)(196.08740975,694.73334076)
\lineto(196.23740975,694.73334076)
\curveto(196.33740206,694.7133384)(196.40740199,694.68833842)(196.44740975,694.65834076)
\curveto(196.48740191,694.63833847)(196.50740189,694.59333852)(196.50740975,694.52334076)
\curveto(196.51740188,694.45333866)(196.51240188,694.39333872)(196.49240975,694.34334076)
\curveto(196.44240195,694.20333891)(196.38740201,694.07833903)(196.32740975,693.96834076)
\curveto(196.26740213,693.85833925)(196.1974022,693.74833936)(196.11740975,693.63834076)
\curveto(195.8974025,693.3083398)(195.64740275,693.04334007)(195.36740975,692.84334076)
\curveto(195.08740331,692.64334047)(194.73740366,692.47334064)(194.31740975,692.33334076)
\curveto(194.20740419,692.29334082)(194.0974043,692.26834084)(193.98740975,692.25834076)
\curveto(193.87740452,692.24834086)(193.76240463,692.22834088)(193.64240975,692.19834076)
\curveto(193.60240479,692.18834092)(193.55740484,692.18834092)(193.50740975,692.19834076)
\curveto(193.46740493,692.19834091)(193.42740497,692.19334092)(193.38740975,692.18334076)
\lineto(193.22240975,692.18334076)
\curveto(193.17240522,692.16334095)(193.11240528,692.15834095)(193.04240975,692.16834076)
\curveto(192.98240541,692.16834094)(192.92740547,692.17334094)(192.87740975,692.18334076)
\curveto(192.7974056,692.19334092)(192.72740567,692.19334092)(192.66740975,692.18334076)
\curveto(192.60740579,692.17334094)(192.54240585,692.17834093)(192.47240975,692.19834076)
\curveto(192.42240597,692.21834089)(192.36740603,692.22834088)(192.30740975,692.22834076)
\curveto(192.24740615,692.22834088)(192.1924062,692.23834087)(192.14240975,692.25834076)
\curveto(192.03240636,692.27834083)(191.92240647,692.30334081)(191.81240975,692.33334076)
\curveto(191.70240669,692.35334076)(191.60240679,692.38834072)(191.51240975,692.43834076)
\curveto(191.40240699,692.47834063)(191.2974071,692.5133406)(191.19740975,692.54334076)
\curveto(191.10740729,692.58334053)(191.02240737,692.62834048)(190.94240975,692.67834076)
\curveto(190.62240777,692.87834023)(190.33740806,693.10834)(190.08740975,693.36834076)
\curveto(189.83740856,693.63833947)(189.63240876,693.94833916)(189.47240975,694.29834076)
\curveto(189.42240897,694.4083387)(189.38240901,694.51833859)(189.35240975,694.62834076)
\curveto(189.32240907,694.74833836)(189.28240911,694.86833824)(189.23240975,694.98834076)
\curveto(189.22240917,695.02833808)(189.21740918,695.06333805)(189.21740975,695.09334076)
\curveto(189.21740918,695.13333798)(189.21240918,695.17333794)(189.20240975,695.21334076)
\curveto(189.16240923,695.33333778)(189.13740926,695.46333765)(189.12740975,695.60334076)
\lineto(189.09740975,696.02334076)
\curveto(189.0974093,696.07333704)(189.0924093,696.12833698)(189.08240975,696.18834076)
\curveto(189.08240931,696.24833686)(189.08740931,696.30333681)(189.09740975,696.35334076)
\lineto(189.09740975,696.53334076)
\lineto(189.14240975,696.89334076)
\curveto(189.18240921,697.06333605)(189.21740918,697.22833588)(189.24740975,697.38834076)
\curveto(189.27740912,697.54833556)(189.32240907,697.69833541)(189.38240975,697.83834076)
\curveto(189.81240858,698.87833423)(190.54240785,699.6133335)(191.57240975,700.04334076)
\curveto(191.71240668,700.10333301)(191.85240654,700.14333297)(191.99240975,700.16334076)
\curveto(192.14240625,700.19333292)(192.2974061,700.22833288)(192.45740975,700.26834076)
\curveto(192.53740586,700.27833283)(192.61240578,700.28333283)(192.68240975,700.28334076)
\curveto(192.75240564,700.28333283)(192.82740557,700.28833282)(192.90740975,700.29834076)
\curveto(193.41740498,700.3083328)(193.85240454,700.24833286)(194.21240975,700.11834076)
\curveto(194.58240381,699.99833311)(194.91240348,699.83833327)(195.20240975,699.63834076)
\curveto(195.2924031,699.57833353)(195.38240301,699.5083336)(195.47240975,699.42834076)
\curveto(195.56240283,699.35833375)(195.64240275,699.28333383)(195.71240975,699.20334076)
\curveto(195.74240265,699.15333396)(195.78240261,699.113334)(195.83240975,699.08334076)
\curveto(195.91240248,698.97333414)(195.98740241,698.85833425)(196.05740975,698.73834076)
\curveto(196.12740227,698.62833448)(196.20240219,698.5133346)(196.28240975,698.39334076)
\curveto(196.33240206,698.30333481)(196.37240202,698.2083349)(196.40240975,698.10834076)
\curveto(196.44240195,698.01833509)(196.48240191,697.91833519)(196.52240975,697.80834076)
\curveto(196.57240182,697.67833543)(196.61240178,697.54333557)(196.64240975,697.40334076)
\curveto(196.67240172,697.26333585)(196.70740169,697.12333599)(196.74740975,696.98334076)
\curveto(196.76740163,696.90333621)(196.77240162,696.8133363)(196.76240975,696.71334076)
\curveto(196.76240163,696.62333649)(196.77240162,696.53833657)(196.79240975,696.45834076)
\lineto(196.79240975,696.29334076)
\moveto(194.54240975,697.17834076)
\curveto(194.61240378,697.27833583)(194.61740378,697.39833571)(194.55740975,697.53834076)
\curveto(194.50740389,697.68833542)(194.46740393,697.79833531)(194.43740975,697.86834076)
\curveto(194.2974041,698.13833497)(194.11240428,698.34333477)(193.88240975,698.48334076)
\curveto(193.65240474,698.63333448)(193.33240506,698.7133344)(192.92240975,698.72334076)
\curveto(192.8924055,698.70333441)(192.85740554,698.69833441)(192.81740975,698.70834076)
\curveto(192.77740562,698.71833439)(192.74240565,698.71833439)(192.71240975,698.70834076)
\curveto(192.66240573,698.68833442)(192.60740579,698.67333444)(192.54740975,698.66334076)
\curveto(192.48740591,698.66333445)(192.43240596,698.65333446)(192.38240975,698.63334076)
\curveto(191.94240645,698.49333462)(191.61740678,698.21833489)(191.40740975,697.80834076)
\curveto(191.38740701,697.76833534)(191.36240703,697.7133354)(191.33240975,697.64334076)
\curveto(191.31240708,697.58333553)(191.2974071,697.51833559)(191.28740975,697.44834076)
\curveto(191.27740712,697.38833572)(191.27740712,697.32833578)(191.28740975,697.26834076)
\curveto(191.30740709,697.2083359)(191.34240705,697.15833595)(191.39240975,697.11834076)
\curveto(191.47240692,697.06833604)(191.58240681,697.04333607)(191.72240975,697.04334076)
\lineto(192.12740975,697.04334076)
\lineto(193.79240975,697.04334076)
\lineto(194.22740975,697.04334076)
\curveto(194.38740401,697.05333606)(194.4924039,697.09833601)(194.54240975,697.17834076)
}
}
{
\newrgbcolor{curcolor}{0 0 0}
\pscustom[linestyle=none,fillstyle=solid,fillcolor=curcolor]
{
\newpath
\moveto(202.465691,700.28334076)
\curveto(202.57568568,700.28333283)(202.67068559,700.27333284)(202.750691,700.25334076)
\curveto(202.84068542,700.23333288)(202.91068535,700.18833292)(202.960691,700.11834076)
\curveto(203.02068524,700.03833307)(203.05068521,699.89833321)(203.050691,699.69834076)
\lineto(203.050691,699.18834076)
\lineto(203.050691,698.81334076)
\curveto(203.0606852,698.67333444)(203.04568521,698.56333455)(203.005691,698.48334076)
\curveto(202.96568529,698.4133347)(202.90568535,698.36833474)(202.825691,698.34834076)
\curveto(202.7556855,698.32833478)(202.67068559,698.31833479)(202.570691,698.31834076)
\curveto(202.48068578,698.31833479)(202.38068588,698.32333479)(202.270691,698.33334076)
\curveto(202.17068609,698.34333477)(202.07568618,698.33833477)(201.985691,698.31834076)
\curveto(201.91568634,698.29833481)(201.84568641,698.28333483)(201.775691,698.27334076)
\curveto(201.70568655,698.27333484)(201.64068662,698.26333485)(201.580691,698.24334076)
\curveto(201.42068684,698.19333492)(201.260687,698.11833499)(201.100691,698.01834076)
\curveto(200.94068732,697.92833518)(200.81568744,697.82333529)(200.725691,697.70334076)
\curveto(200.67568758,697.62333549)(200.62068764,697.53833557)(200.560691,697.44834076)
\curveto(200.51068775,697.36833574)(200.4606878,697.28333583)(200.410691,697.19334076)
\curveto(200.38068788,697.113336)(200.35068791,697.02833608)(200.320691,696.93834076)
\lineto(200.260691,696.69834076)
\curveto(200.24068802,696.62833648)(200.23068803,696.55333656)(200.230691,696.47334076)
\curveto(200.23068803,696.40333671)(200.22068804,696.33333678)(200.200691,696.26334076)
\curveto(200.19068807,696.22333689)(200.18568807,696.18333693)(200.185691,696.14334076)
\curveto(200.19568806,696.113337)(200.19568806,696.08333703)(200.185691,696.05334076)
\lineto(200.185691,695.81334076)
\curveto(200.16568809,695.74333737)(200.1606881,695.66333745)(200.170691,695.57334076)
\curveto(200.18068808,695.49333762)(200.18568807,695.4133377)(200.185691,695.33334076)
\lineto(200.185691,694.37334076)
\lineto(200.185691,693.09834076)
\curveto(200.18568807,692.96834014)(200.18068808,692.84834026)(200.170691,692.73834076)
\curveto(200.1606881,692.62834048)(200.13068813,692.53834057)(200.080691,692.46834076)
\curveto(200.0606882,692.43834067)(200.02568823,692.4133407)(199.975691,692.39334076)
\curveto(199.93568832,692.38334073)(199.89068837,692.37334074)(199.840691,692.36334076)
\lineto(199.765691,692.36334076)
\curveto(199.71568854,692.35334076)(199.6606886,692.34834076)(199.600691,692.34834076)
\lineto(199.435691,692.34834076)
\lineto(198.790691,692.34834076)
\curveto(198.73068953,692.35834075)(198.66568959,692.36334075)(198.595691,692.36334076)
\lineto(198.400691,692.36334076)
\curveto(198.35068991,692.38334073)(198.30068996,692.39834071)(198.250691,692.40834076)
\curveto(198.20069006,692.42834068)(198.16569009,692.46334065)(198.145691,692.51334076)
\curveto(198.10569015,692.56334055)(198.08069018,692.63334048)(198.070691,692.72334076)
\lineto(198.070691,693.02334076)
\lineto(198.070691,694.04334076)
\lineto(198.070691,698.27334076)
\lineto(198.070691,699.38334076)
\lineto(198.070691,699.66834076)
\curveto(198.07069019,699.76833334)(198.09069017,699.84833326)(198.130691,699.90834076)
\curveto(198.18069008,699.98833312)(198.25569,700.03833307)(198.355691,700.05834076)
\curveto(198.4556898,700.07833303)(198.57568968,700.08833302)(198.715691,700.08834076)
\lineto(199.480691,700.08834076)
\curveto(199.60068866,700.08833302)(199.70568855,700.07833303)(199.795691,700.05834076)
\curveto(199.88568837,700.04833306)(199.9556883,700.00333311)(200.005691,699.92334076)
\curveto(200.03568822,699.87333324)(200.05068821,699.80333331)(200.050691,699.71334076)
\lineto(200.080691,699.44334076)
\curveto(200.09068817,699.36333375)(200.10568815,699.28833382)(200.125691,699.21834076)
\curveto(200.1556881,699.14833396)(200.20568805,699.113334)(200.275691,699.11334076)
\curveto(200.29568796,699.13333398)(200.31568794,699.14333397)(200.335691,699.14334076)
\curveto(200.3556879,699.14333397)(200.37568788,699.15333396)(200.395691,699.17334076)
\curveto(200.4556878,699.22333389)(200.50568775,699.27833383)(200.545691,699.33834076)
\curveto(200.59568766,699.4083337)(200.6556876,699.46833364)(200.725691,699.51834076)
\curveto(200.76568749,699.54833356)(200.80068746,699.57833353)(200.830691,699.60834076)
\curveto(200.8606874,699.64833346)(200.89568736,699.68333343)(200.935691,699.71334076)
\lineto(201.205691,699.89334076)
\curveto(201.30568695,699.95333316)(201.40568685,700.0083331)(201.505691,700.05834076)
\curveto(201.60568665,700.09833301)(201.70568655,700.13333298)(201.805691,700.16334076)
\lineto(202.135691,700.25334076)
\curveto(202.16568609,700.26333285)(202.22068604,700.26333285)(202.300691,700.25334076)
\curveto(202.39068587,700.25333286)(202.44568581,700.26333285)(202.465691,700.28334076)
}
}
{
\newrgbcolor{curcolor}{0 0 0}
\pscustom[linestyle=none,fillstyle=solid,fillcolor=curcolor]
{
\newpath
\moveto(211.37709725,696.53334076)
\curveto(211.39708868,696.47333664)(211.40708867,696.38833672)(211.40709725,696.27834076)
\curveto(211.40708867,696.16833694)(211.39708868,696.08333703)(211.37709725,696.02334076)
\lineto(211.37709725,695.87334076)
\curveto(211.35708872,695.79333732)(211.34708873,695.7133374)(211.34709725,695.63334076)
\curveto(211.35708872,695.55333756)(211.35208872,695.47333764)(211.33209725,695.39334076)
\curveto(211.31208876,695.32333779)(211.29708878,695.25833785)(211.28709725,695.19834076)
\curveto(211.2770888,695.13833797)(211.26708881,695.07333804)(211.25709725,695.00334076)
\curveto(211.21708886,694.89333822)(211.18208889,694.77833833)(211.15209725,694.65834076)
\curveto(211.12208895,694.54833856)(211.08208899,694.44333867)(211.03209725,694.34334076)
\curveto(210.82208925,693.86333925)(210.54708953,693.47333964)(210.20709725,693.17334076)
\curveto(209.86709021,692.87334024)(209.45709062,692.62334049)(208.97709725,692.42334076)
\curveto(208.85709122,692.37334074)(208.73209134,692.33834077)(208.60209725,692.31834076)
\curveto(208.48209159,692.28834082)(208.35709172,692.25834085)(208.22709725,692.22834076)
\curveto(208.1770919,692.2083409)(208.12209195,692.19834091)(208.06209725,692.19834076)
\curveto(208.00209207,692.19834091)(207.94709213,692.19334092)(207.89709725,692.18334076)
\lineto(207.79209725,692.18334076)
\curveto(207.76209231,692.17334094)(207.73209234,692.16834094)(207.70209725,692.16834076)
\curveto(207.65209242,692.15834095)(207.5720925,692.15334096)(207.46209725,692.15334076)
\curveto(207.35209272,692.14334097)(207.26709281,692.14834096)(207.20709725,692.16834076)
\lineto(207.05709725,692.16834076)
\curveto(207.00709307,692.17834093)(206.95209312,692.18334093)(206.89209725,692.18334076)
\curveto(206.84209323,692.17334094)(206.79209328,692.17834093)(206.74209725,692.19834076)
\curveto(206.70209337,692.2083409)(206.66209341,692.2133409)(206.62209725,692.21334076)
\curveto(206.59209348,692.2133409)(206.55209352,692.21834089)(206.50209725,692.22834076)
\curveto(206.40209367,692.25834085)(206.30209377,692.28334083)(206.20209725,692.30334076)
\curveto(206.10209397,692.32334079)(206.00709407,692.35334076)(205.91709725,692.39334076)
\curveto(205.79709428,692.43334068)(205.68209439,692.47334064)(205.57209725,692.51334076)
\curveto(205.4720946,692.55334056)(205.36709471,692.60334051)(205.25709725,692.66334076)
\curveto(204.90709517,692.87334024)(204.60709547,693.11833999)(204.35709725,693.39834076)
\curveto(204.10709597,693.67833943)(203.89709618,694.0133391)(203.72709725,694.40334076)
\curveto(203.6770964,694.49333862)(203.63709644,694.58833852)(203.60709725,694.68834076)
\curveto(203.58709649,694.78833832)(203.56209651,694.89333822)(203.53209725,695.00334076)
\curveto(203.51209656,695.05333806)(203.50209657,695.09833801)(203.50209725,695.13834076)
\curveto(203.50209657,695.17833793)(203.49209658,695.22333789)(203.47209725,695.27334076)
\curveto(203.45209662,695.35333776)(203.44209663,695.43333768)(203.44209725,695.51334076)
\curveto(203.44209663,695.60333751)(203.43209664,695.68833742)(203.41209725,695.76834076)
\curveto(203.40209667,695.81833729)(203.39709668,695.86333725)(203.39709725,695.90334076)
\lineto(203.39709725,696.03834076)
\curveto(203.3770967,696.09833701)(203.36709671,696.18333693)(203.36709725,696.29334076)
\curveto(203.3770967,696.40333671)(203.39209668,696.48833662)(203.41209725,696.54834076)
\lineto(203.41209725,696.65334076)
\curveto(203.42209665,696.70333641)(203.42209665,696.75333636)(203.41209725,696.80334076)
\curveto(203.41209666,696.86333625)(203.42209665,696.91833619)(203.44209725,696.96834076)
\curveto(203.45209662,697.01833609)(203.45709662,697.06333605)(203.45709725,697.10334076)
\curveto(203.45709662,697.15333596)(203.46709661,697.20333591)(203.48709725,697.25334076)
\curveto(203.52709655,697.38333573)(203.56209651,697.5083356)(203.59209725,697.62834076)
\curveto(203.62209645,697.75833535)(203.66209641,697.88333523)(203.71209725,698.00334076)
\curveto(203.89209618,698.4133347)(204.10709597,698.75333436)(204.35709725,699.02334076)
\curveto(204.60709547,699.30333381)(204.91209516,699.55833355)(205.27209725,699.78834076)
\curveto(205.3720947,699.83833327)(205.4770946,699.88333323)(205.58709725,699.92334076)
\curveto(205.69709438,699.96333315)(205.80709427,700.0083331)(205.91709725,700.05834076)
\curveto(206.04709403,700.108333)(206.18209389,700.14333297)(206.32209725,700.16334076)
\curveto(206.46209361,700.18333293)(206.60709347,700.2133329)(206.75709725,700.25334076)
\curveto(206.83709324,700.26333285)(206.91209316,700.26833284)(206.98209725,700.26834076)
\curveto(207.05209302,700.26833284)(207.12209295,700.27333284)(207.19209725,700.28334076)
\curveto(207.7720923,700.29333282)(208.2720918,700.23333288)(208.69209725,700.10334076)
\curveto(209.12209095,699.97333314)(209.50209057,699.79333332)(209.83209725,699.56334076)
\curveto(209.94209013,699.48333363)(210.05209002,699.39333372)(210.16209725,699.29334076)
\curveto(210.28208979,699.20333391)(210.38208969,699.10333401)(210.46209725,698.99334076)
\curveto(210.54208953,698.89333422)(210.61208946,698.79333432)(210.67209725,698.69334076)
\curveto(210.74208933,698.59333452)(210.81208926,698.48833462)(210.88209725,698.37834076)
\curveto(210.95208912,698.26833484)(211.00708907,698.14833496)(211.04709725,698.01834076)
\curveto(211.08708899,697.89833521)(211.13208894,697.76833534)(211.18209725,697.62834076)
\curveto(211.21208886,697.54833556)(211.23708884,697.46333565)(211.25709725,697.37334076)
\lineto(211.31709725,697.10334076)
\curveto(211.32708875,697.06333605)(211.33208874,697.02333609)(211.33209725,696.98334076)
\curveto(211.33208874,696.94333617)(211.33708874,696.90333621)(211.34709725,696.86334076)
\curveto(211.36708871,696.8133363)(211.3720887,696.75833635)(211.36209725,696.69834076)
\curveto(211.35208872,696.63833647)(211.35708872,696.58333653)(211.37709725,696.53334076)
\moveto(209.27709725,695.99334076)
\curveto(209.28709079,696.04333707)(209.29209078,696.113337)(209.29209725,696.20334076)
\curveto(209.29209078,696.30333681)(209.28709079,696.37833673)(209.27709725,696.42834076)
\lineto(209.27709725,696.54834076)
\curveto(209.25709082,696.59833651)(209.24709083,696.65333646)(209.24709725,696.71334076)
\curveto(209.24709083,696.77333634)(209.24209083,696.82833628)(209.23209725,696.87834076)
\curveto(209.23209084,696.91833619)(209.22709085,696.94833616)(209.21709725,696.96834076)
\lineto(209.15709725,697.20834076)
\curveto(209.14709093,697.29833581)(209.12709095,697.38333573)(209.09709725,697.46334076)
\curveto(208.98709109,697.72333539)(208.85709122,697.94333517)(208.70709725,698.12334076)
\curveto(208.55709152,698.3133348)(208.35709172,698.46333465)(208.10709725,698.57334076)
\curveto(208.04709203,698.59333452)(207.98709209,698.6083345)(207.92709725,698.61834076)
\curveto(207.86709221,698.63833447)(207.80209227,698.65833445)(207.73209725,698.67834076)
\curveto(207.65209242,698.69833441)(207.56709251,698.70333441)(207.47709725,698.69334076)
\lineto(207.20709725,698.69334076)
\curveto(207.1770929,698.67333444)(207.14209293,698.66333445)(207.10209725,698.66334076)
\curveto(207.06209301,698.67333444)(207.02709305,698.67333444)(206.99709725,698.66334076)
\lineto(206.78709725,698.60334076)
\curveto(206.72709335,698.59333452)(206.6720934,698.57333454)(206.62209725,698.54334076)
\curveto(206.3720937,698.43333468)(206.16709391,698.27333484)(206.00709725,698.06334076)
\curveto(205.85709422,697.86333525)(205.73709434,697.62833548)(205.64709725,697.35834076)
\curveto(205.61709446,697.25833585)(205.59209448,697.15333596)(205.57209725,697.04334076)
\curveto(205.56209451,696.93333618)(205.54709453,696.82333629)(205.52709725,696.71334076)
\curveto(205.51709456,696.66333645)(205.51209456,696.6133365)(205.51209725,696.56334076)
\lineto(205.51209725,696.41334076)
\curveto(205.49209458,696.34333677)(205.48209459,696.23833687)(205.48209725,696.09834076)
\curveto(205.49209458,695.95833715)(205.50709457,695.85333726)(205.52709725,695.78334076)
\lineto(205.52709725,695.64834076)
\curveto(205.54709453,695.56833754)(205.56209451,695.48833762)(205.57209725,695.40834076)
\curveto(205.58209449,695.33833777)(205.59709448,695.26333785)(205.61709725,695.18334076)
\curveto(205.71709436,694.88333823)(205.82209425,694.63833847)(205.93209725,694.44834076)
\curveto(206.05209402,694.26833884)(206.23709384,694.10333901)(206.48709725,693.95334076)
\curveto(206.55709352,693.90333921)(206.63209344,693.86333925)(206.71209725,693.83334076)
\curveto(206.80209327,693.80333931)(206.89209318,693.77833933)(206.98209725,693.75834076)
\curveto(207.02209305,693.74833936)(207.05709302,693.74333937)(207.08709725,693.74334076)
\curveto(207.11709296,693.75333936)(207.15209292,693.75333936)(207.19209725,693.74334076)
\lineto(207.31209725,693.71334076)
\curveto(207.36209271,693.7133394)(207.40709267,693.71833939)(207.44709725,693.72834076)
\lineto(207.56709725,693.72834076)
\curveto(207.64709243,693.74833936)(207.72709235,693.76333935)(207.80709725,693.77334076)
\curveto(207.88709219,693.78333933)(207.96209211,693.80333931)(208.03209725,693.83334076)
\curveto(208.29209178,693.93333918)(208.50209157,694.06833904)(208.66209725,694.23834076)
\curveto(208.82209125,694.4083387)(208.95709112,694.61833849)(209.06709725,694.86834076)
\curveto(209.10709097,694.96833814)(209.13709094,695.06833804)(209.15709725,695.16834076)
\curveto(209.1770909,695.26833784)(209.20209087,695.37333774)(209.23209725,695.48334076)
\curveto(209.24209083,695.52333759)(209.24709083,695.55833755)(209.24709725,695.58834076)
\curveto(209.24709083,695.62833748)(209.25209082,695.66833744)(209.26209725,695.70834076)
\lineto(209.26209725,695.84334076)
\curveto(209.26209081,695.89333722)(209.26709081,695.94333717)(209.27709725,695.99334076)
}
}
{
\newrgbcolor{curcolor}{0 0 0}
\pscustom[linestyle=none,fillstyle=solid,fillcolor=curcolor]
{
\newpath
\moveto(217.20201912,700.28334076)
\curveto(217.80201332,700.30333281)(218.30201282,700.21833289)(218.70201912,700.02834076)
\curveto(219.10201202,699.83833327)(219.4170117,699.55833355)(219.64701912,699.18834076)
\curveto(219.7170114,699.07833403)(219.77201135,698.95833415)(219.81201912,698.82834076)
\curveto(219.85201127,698.7083344)(219.89201123,698.58333453)(219.93201912,698.45334076)
\curveto(219.95201117,698.37333474)(219.96201116,698.29833481)(219.96201912,698.22834076)
\curveto(219.97201115,698.15833495)(219.98701113,698.08833502)(220.00701912,698.01834076)
\curveto(220.00701111,697.95833515)(220.01201111,697.91833519)(220.02201912,697.89834076)
\curveto(220.04201108,697.75833535)(220.05201107,697.6133355)(220.05201912,697.46334076)
\lineto(220.05201912,697.02834076)
\lineto(220.05201912,695.69334076)
\lineto(220.05201912,693.26334076)
\curveto(220.05201107,693.07334004)(220.04701107,692.88834022)(220.03701912,692.70834076)
\curveto(220.03701108,692.53834057)(219.96701115,692.42834068)(219.82701912,692.37834076)
\curveto(219.76701135,692.35834075)(219.69701142,692.34834076)(219.61701912,692.34834076)
\lineto(219.37701912,692.34834076)
\lineto(218.56701912,692.34834076)
\curveto(218.44701267,692.34834076)(218.33701278,692.35334076)(218.23701912,692.36334076)
\curveto(218.14701297,692.38334073)(218.07701304,692.42834068)(218.02701912,692.49834076)
\curveto(217.98701313,692.55834055)(217.96201316,692.63334048)(217.95201912,692.72334076)
\lineto(217.95201912,693.03834076)
\lineto(217.95201912,694.08834076)
\lineto(217.95201912,696.32334076)
\curveto(217.95201317,696.69333642)(217.93701318,697.03333608)(217.90701912,697.34334076)
\curveto(217.87701324,697.66333545)(217.78701333,697.93333518)(217.63701912,698.15334076)
\curveto(217.49701362,698.35333476)(217.29201383,698.49333462)(217.02201912,698.57334076)
\curveto(216.97201415,698.59333452)(216.9170142,698.60333451)(216.85701912,698.60334076)
\curveto(216.80701431,698.60333451)(216.75201437,698.6133345)(216.69201912,698.63334076)
\curveto(216.64201448,698.64333447)(216.57701454,698.64333447)(216.49701912,698.63334076)
\curveto(216.42701469,698.63333448)(216.37201475,698.62833448)(216.33201912,698.61834076)
\curveto(216.29201483,698.6083345)(216.25701486,698.60333451)(216.22701912,698.60334076)
\curveto(216.19701492,698.60333451)(216.16701495,698.59833451)(216.13701912,698.58834076)
\curveto(215.90701521,698.52833458)(215.7220154,698.44833466)(215.58201912,698.34834076)
\curveto(215.26201586,698.11833499)(215.07201605,697.78333533)(215.01201912,697.34334076)
\curveto(214.95201617,696.90333621)(214.9220162,696.4083367)(214.92201912,695.85834076)
\lineto(214.92201912,693.98334076)
\lineto(214.92201912,693.06834076)
\lineto(214.92201912,692.79834076)
\curveto(214.9220162,692.7083404)(214.90701621,692.63334048)(214.87701912,692.57334076)
\curveto(214.82701629,692.46334065)(214.74701637,692.39834071)(214.63701912,692.37834076)
\curveto(214.52701659,692.35834075)(214.39201673,692.34834076)(214.23201912,692.34834076)
\lineto(213.48201912,692.34834076)
\curveto(213.37201775,692.34834076)(213.26201786,692.35334076)(213.15201912,692.36334076)
\curveto(213.04201808,692.37334074)(212.96201816,692.4083407)(212.91201912,692.46834076)
\curveto(212.84201828,692.55834055)(212.80701831,692.68834042)(212.80701912,692.85834076)
\curveto(212.8170183,693.02834008)(212.8220183,693.18833992)(212.82201912,693.33834076)
\lineto(212.82201912,695.37834076)
\lineto(212.82201912,698.67834076)
\lineto(212.82201912,699.44334076)
\lineto(212.82201912,699.74334076)
\curveto(212.83201829,699.83333328)(212.86201826,699.9083332)(212.91201912,699.96834076)
\curveto(212.93201819,699.99833311)(212.96201816,700.01833309)(213.00201912,700.02834076)
\curveto(213.05201807,700.04833306)(213.10201802,700.06333305)(213.15201912,700.07334076)
\lineto(213.22701912,700.07334076)
\curveto(213.27701784,700.08333303)(213.32701779,700.08833302)(213.37701912,700.08834076)
\lineto(213.54201912,700.08834076)
\lineto(214.17201912,700.08834076)
\curveto(214.25201687,700.08833302)(214.32701679,700.08333303)(214.39701912,700.07334076)
\curveto(214.47701664,700.07333304)(214.54701657,700.06333305)(214.60701912,700.04334076)
\curveto(214.67701644,700.0133331)(214.7220164,699.96833314)(214.74201912,699.90834076)
\curveto(214.77201635,699.84833326)(214.79701632,699.77833333)(214.81701912,699.69834076)
\curveto(214.82701629,699.65833345)(214.82701629,699.62333349)(214.81701912,699.59334076)
\curveto(214.8170163,699.56333355)(214.82701629,699.53333358)(214.84701912,699.50334076)
\curveto(214.86701625,699.45333366)(214.88201624,699.42333369)(214.89201912,699.41334076)
\curveto(214.91201621,699.40333371)(214.93701618,699.38833372)(214.96701912,699.36834076)
\curveto(215.07701604,699.35833375)(215.16701595,699.39333372)(215.23701912,699.47334076)
\curveto(215.30701581,699.56333355)(215.38201574,699.63333348)(215.46201912,699.68334076)
\curveto(215.73201539,699.88333323)(216.03201509,700.04333307)(216.36201912,700.16334076)
\curveto(216.45201467,700.19333292)(216.54201458,700.2133329)(216.63201912,700.22334076)
\curveto(216.73201439,700.23333288)(216.83701428,700.24833286)(216.94701912,700.26834076)
\curveto(216.97701414,700.27833283)(217.0220141,700.27833283)(217.08201912,700.26834076)
\curveto(217.14201398,700.26833284)(217.18201394,700.27333284)(217.20201912,700.28334076)
}
}
{
\newrgbcolor{curcolor}{0 0 0}
\pscustom[linestyle=none,fillstyle=solid,fillcolor=curcolor]
{
\newpath
\moveto(29.0966285,681.54834076)
\curveto(29.846624,681.56833279)(30.49662335,681.48333288)(31.0466285,681.29334076)
\curveto(31.60662224,681.11333325)(32.03162181,680.79833356)(32.3216285,680.34834076)
\curveto(32.39162145,680.23833412)(32.45162139,680.12333424)(32.5016285,680.00334076)
\curveto(32.56162128,679.89333447)(32.61162123,679.76833459)(32.6516285,679.62834076)
\curveto(32.67162117,679.56833479)(32.68162116,679.50333486)(32.6816285,679.43334076)
\curveto(32.68162116,679.363335)(32.67162117,679.30333506)(32.6516285,679.25334076)
\curveto(32.61162123,679.19333517)(32.55662129,679.15333521)(32.4866285,679.13334076)
\curveto(32.43662141,679.11333525)(32.37662147,679.10333526)(32.3066285,679.10334076)
\lineto(32.0966285,679.10334076)
\lineto(31.4366285,679.10334076)
\curveto(31.36662248,679.10333526)(31.29662255,679.09833526)(31.2266285,679.08834076)
\curveto(31.15662269,679.08833527)(31.09162275,679.09833526)(31.0316285,679.11834076)
\curveto(30.93162291,679.13833522)(30.85662299,679.17833518)(30.8066285,679.23834076)
\curveto(30.75662309,679.29833506)(30.71162313,679.358335)(30.6716285,679.41834076)
\lineto(30.5516285,679.62834076)
\curveto(30.52162332,679.70833465)(30.47162337,679.77333459)(30.4016285,679.82334076)
\curveto(30.30162354,679.90333446)(30.20162364,679.9633344)(30.1016285,680.00334076)
\curveto(30.01162383,680.04333432)(29.89662395,680.07833428)(29.7566285,680.10834076)
\curveto(29.68662416,680.12833423)(29.58162426,680.14333422)(29.4416285,680.15334076)
\curveto(29.31162453,680.1633342)(29.21162463,680.1583342)(29.1416285,680.13834076)
\lineto(29.0366285,680.13834076)
\lineto(28.8866285,680.10834076)
\curveto(28.846625,680.10833425)(28.80162504,680.10333426)(28.7516285,680.09334076)
\curveto(28.58162526,680.04333432)(28.4416254,679.97333439)(28.3316285,679.88334076)
\curveto(28.23162561,679.80333456)(28.16162568,679.67833468)(28.1216285,679.50834076)
\curveto(28.10162574,679.43833492)(28.10162574,679.37333499)(28.1216285,679.31334076)
\curveto(28.1416257,679.25333511)(28.16162568,679.20333516)(28.1816285,679.16334076)
\curveto(28.25162559,679.04333532)(28.33162551,678.94833541)(28.4216285,678.87834076)
\curveto(28.52162532,678.80833555)(28.63662521,678.74833561)(28.7666285,678.69834076)
\curveto(28.95662489,678.61833574)(29.16162468,678.54833581)(29.3816285,678.48834076)
\lineto(30.0716285,678.33834076)
\curveto(30.31162353,678.29833606)(30.5416233,678.24833611)(30.7616285,678.18834076)
\curveto(30.99162285,678.13833622)(31.20662264,678.07333629)(31.4066285,677.99334076)
\curveto(31.49662235,677.95333641)(31.58162226,677.91833644)(31.6616285,677.88834076)
\curveto(31.75162209,677.86833649)(31.83662201,677.83333653)(31.9166285,677.78334076)
\curveto(32.10662174,677.6633367)(32.27662157,677.53333683)(32.4266285,677.39334076)
\curveto(32.58662126,677.25333711)(32.71162113,677.07833728)(32.8016285,676.86834076)
\curveto(32.83162101,676.79833756)(32.85662099,676.72833763)(32.8766285,676.65834076)
\curveto(32.89662095,676.58833777)(32.91662093,676.51333785)(32.9366285,676.43334076)
\curveto(32.9466209,676.37333799)(32.95162089,676.27833808)(32.9516285,676.14834076)
\curveto(32.96162088,676.02833833)(32.96162088,675.93333843)(32.9516285,675.86334076)
\lineto(32.9516285,675.78834076)
\curveto(32.93162091,675.72833863)(32.91662093,675.66833869)(32.9066285,675.60834076)
\curveto(32.90662094,675.5583388)(32.90162094,675.50833885)(32.8916285,675.45834076)
\curveto(32.82162102,675.1583392)(32.71162113,674.89333947)(32.5616285,674.66334076)
\curveto(32.40162144,674.42333994)(32.20662164,674.22834013)(31.9766285,674.07834076)
\curveto(31.7466221,673.92834043)(31.48662236,673.79834056)(31.1966285,673.68834076)
\curveto(31.08662276,673.63834072)(30.96662288,673.60334076)(30.8366285,673.58334076)
\curveto(30.71662313,673.5633408)(30.59662325,673.53834082)(30.4766285,673.50834076)
\curveto(30.38662346,673.48834087)(30.29162355,673.47834088)(30.1916285,673.47834076)
\curveto(30.10162374,673.46834089)(30.01162383,673.45334091)(29.9216285,673.43334076)
\lineto(29.6516285,673.43334076)
\curveto(29.59162425,673.41334095)(29.48662436,673.40334096)(29.3366285,673.40334076)
\curveto(29.19662465,673.40334096)(29.09662475,673.41334095)(29.0366285,673.43334076)
\curveto(29.00662484,673.43334093)(28.97162487,673.43834092)(28.9316285,673.44834076)
\lineto(28.8266285,673.44834076)
\curveto(28.70662514,673.46834089)(28.58662526,673.48334088)(28.4666285,673.49334076)
\curveto(28.3466255,673.50334086)(28.23162561,673.52334084)(28.1216285,673.55334076)
\curveto(27.73162611,673.6633407)(27.38662646,673.78834057)(27.0866285,673.92834076)
\curveto(26.78662706,674.07834028)(26.53162731,674.29834006)(26.3216285,674.58834076)
\curveto(26.18162766,674.77833958)(26.06162778,674.99833936)(25.9616285,675.24834076)
\curveto(25.9416279,675.30833905)(25.92162792,675.38833897)(25.9016285,675.48834076)
\curveto(25.88162796,675.53833882)(25.86662798,675.60833875)(25.8566285,675.69834076)
\curveto(25.846628,675.78833857)(25.85162799,675.8633385)(25.8716285,675.92334076)
\curveto(25.90162794,675.99333837)(25.95162789,676.04333832)(26.0216285,676.07334076)
\curveto(26.07162777,676.09333827)(26.13162771,676.10333826)(26.2016285,676.10334076)
\lineto(26.4266285,676.10334076)
\lineto(27.1316285,676.10334076)
\lineto(27.3716285,676.10334076)
\curveto(27.45162639,676.10333826)(27.52162632,676.09333827)(27.5816285,676.07334076)
\curveto(27.69162615,676.03333833)(27.76162608,675.96833839)(27.7916285,675.87834076)
\curveto(27.83162601,675.78833857)(27.87662597,675.69333867)(27.9266285,675.59334076)
\curveto(27.9466259,675.54333882)(27.98162586,675.47833888)(28.0316285,675.39834076)
\curveto(28.09162575,675.31833904)(28.1416257,675.26833909)(28.1816285,675.24834076)
\curveto(28.30162554,675.14833921)(28.41662543,675.06833929)(28.5266285,675.00834076)
\curveto(28.63662521,674.9583394)(28.77662507,674.90833945)(28.9466285,674.85834076)
\curveto(28.99662485,674.83833952)(29.0466248,674.82833953)(29.0966285,674.82834076)
\curveto(29.1466247,674.83833952)(29.19662465,674.83833952)(29.2466285,674.82834076)
\curveto(29.32662452,674.80833955)(29.41162443,674.79833956)(29.5016285,674.79834076)
\curveto(29.60162424,674.80833955)(29.68662416,674.82333954)(29.7566285,674.84334076)
\curveto(29.80662404,674.85333951)(29.85162399,674.8583395)(29.8916285,674.85834076)
\curveto(29.9416239,674.8583395)(29.99162385,674.86833949)(30.0416285,674.88834076)
\curveto(30.18162366,674.93833942)(30.30662354,674.99833936)(30.4166285,675.06834076)
\curveto(30.53662331,675.13833922)(30.63162321,675.22833913)(30.7016285,675.33834076)
\curveto(30.75162309,675.41833894)(30.79162305,675.54333882)(30.8216285,675.71334076)
\curveto(30.841623,675.78333858)(30.841623,675.84833851)(30.8216285,675.90834076)
\curveto(30.80162304,675.96833839)(30.78162306,676.01833834)(30.7616285,676.05834076)
\curveto(30.69162315,676.19833816)(30.60162324,676.30333806)(30.4916285,676.37334076)
\curveto(30.39162345,676.44333792)(30.27162357,676.50833785)(30.1316285,676.56834076)
\curveto(29.9416239,676.64833771)(29.7416241,676.71333765)(29.5316285,676.76334076)
\curveto(29.32162452,676.81333755)(29.11162473,676.86833749)(28.9016285,676.92834076)
\curveto(28.82162502,676.94833741)(28.73662511,676.9633374)(28.6466285,676.97334076)
\curveto(28.56662528,676.98333738)(28.48662536,676.99833736)(28.4066285,677.01834076)
\curveto(28.08662576,677.10833725)(27.78162606,677.19333717)(27.4916285,677.27334076)
\curveto(27.20162664,677.363337)(26.93662691,677.49333687)(26.6966285,677.66334076)
\curveto(26.41662743,677.8633365)(26.21162763,678.13333623)(26.0816285,678.47334076)
\curveto(26.06162778,678.54333582)(26.0416278,678.63833572)(26.0216285,678.75834076)
\curveto(26.00162784,678.82833553)(25.98662786,678.91333545)(25.9766285,679.01334076)
\curveto(25.96662788,679.11333525)(25.97162787,679.20333516)(25.9916285,679.28334076)
\curveto(26.01162783,679.33333503)(26.01662783,679.37333499)(26.0066285,679.40334076)
\curveto(25.99662785,679.44333492)(26.00162784,679.48833487)(26.0216285,679.53834076)
\curveto(26.0416278,679.64833471)(26.06162778,679.74833461)(26.0816285,679.83834076)
\curveto(26.11162773,679.93833442)(26.1466277,680.03333433)(26.1866285,680.12334076)
\curveto(26.31662753,680.41333395)(26.49662735,680.64833371)(26.7266285,680.82834076)
\curveto(26.95662689,681.00833335)(27.21662663,681.15333321)(27.5066285,681.26334076)
\curveto(27.61662623,681.31333305)(27.73162611,681.34833301)(27.8516285,681.36834076)
\curveto(27.97162587,681.39833296)(28.09662575,681.42833293)(28.2266285,681.45834076)
\curveto(28.28662556,681.47833288)(28.3466255,681.48833287)(28.4066285,681.48834076)
\lineto(28.5866285,681.51834076)
\curveto(28.66662518,681.52833283)(28.75162509,681.53333283)(28.8416285,681.53334076)
\curveto(28.93162491,681.53333283)(29.01662483,681.53833282)(29.0966285,681.54834076)
}
}
{
\newrgbcolor{curcolor}{0 0 0}
\pscustom[linestyle=none,fillstyle=solid,fillcolor=curcolor]
{
\newpath
\moveto(34.60326912,681.32334076)
\lineto(35.72826912,681.32334076)
\curveto(35.83826669,681.32333304)(35.93826659,681.31833304)(36.02826912,681.30834076)
\curveto(36.11826641,681.29833306)(36.18326634,681.2633331)(36.22326912,681.20334076)
\curveto(36.27326625,681.14333322)(36.30326622,681.0583333)(36.31326912,680.94834076)
\curveto(36.3232662,680.84833351)(36.3282662,680.74333362)(36.32826912,680.63334076)
\lineto(36.32826912,679.58334076)
\lineto(36.32826912,677.34834076)
\curveto(36.3282662,676.98833737)(36.34326618,676.64833771)(36.37326912,676.32834076)
\curveto(36.40326612,676.00833835)(36.49326603,675.74333862)(36.64326912,675.53334076)
\curveto(36.78326574,675.32333904)(37.00826552,675.17333919)(37.31826912,675.08334076)
\curveto(37.36826516,675.07333929)(37.40826512,675.06833929)(37.43826912,675.06834076)
\curveto(37.47826505,675.06833929)(37.523265,675.0633393)(37.57326912,675.05334076)
\curveto(37.6232649,675.04333932)(37.67826485,675.03833932)(37.73826912,675.03834076)
\curveto(37.79826473,675.03833932)(37.84326468,675.04333932)(37.87326912,675.05334076)
\curveto(37.9232646,675.07333929)(37.96326456,675.07833928)(37.99326912,675.06834076)
\curveto(38.03326449,675.0583393)(38.07326445,675.0633393)(38.11326912,675.08334076)
\curveto(38.3232642,675.13333923)(38.48826404,675.19833916)(38.60826912,675.27834076)
\curveto(38.78826374,675.38833897)(38.9282636,675.52833883)(39.02826912,675.69834076)
\curveto(39.13826339,675.87833848)(39.21326331,676.07333829)(39.25326912,676.28334076)
\curveto(39.30326322,676.50333786)(39.33326319,676.74333762)(39.34326912,677.00334076)
\curveto(39.35326317,677.27333709)(39.35826317,677.55333681)(39.35826912,677.84334076)
\lineto(39.35826912,679.65834076)
\lineto(39.35826912,680.63334076)
\lineto(39.35826912,680.90334076)
\curveto(39.35826317,681.00333336)(39.37826315,681.08333328)(39.41826912,681.14334076)
\curveto(39.46826306,681.23333313)(39.54326298,681.28333308)(39.64326912,681.29334076)
\curveto(39.74326278,681.31333305)(39.86326266,681.32333304)(40.00326912,681.32334076)
\lineto(40.79826912,681.32334076)
\lineto(41.08326912,681.32334076)
\curveto(41.17326135,681.32333304)(41.24826128,681.30333306)(41.30826912,681.26334076)
\curveto(41.38826114,681.21333315)(41.43326109,681.13833322)(41.44326912,681.03834076)
\curveto(41.45326107,680.93833342)(41.45826107,680.82333354)(41.45826912,680.69334076)
\lineto(41.45826912,679.55334076)
\lineto(41.45826912,675.33834076)
\lineto(41.45826912,674.27334076)
\lineto(41.45826912,673.97334076)
\curveto(41.45826107,673.87334049)(41.43826109,673.79834056)(41.39826912,673.74834076)
\curveto(41.34826118,673.66834069)(41.27326125,673.62334074)(41.17326912,673.61334076)
\curveto(41.07326145,673.60334076)(40.96826156,673.59834076)(40.85826912,673.59834076)
\lineto(40.04826912,673.59834076)
\curveto(39.93826259,673.59834076)(39.83826269,673.60334076)(39.74826912,673.61334076)
\curveto(39.66826286,673.62334074)(39.60326292,673.6633407)(39.55326912,673.73334076)
\curveto(39.53326299,673.7633406)(39.51326301,673.80834055)(39.49326912,673.86834076)
\curveto(39.48326304,673.92834043)(39.46826306,673.98834037)(39.44826912,674.04834076)
\curveto(39.43826309,674.10834025)(39.4232631,674.1633402)(39.40326912,674.21334076)
\curveto(39.38326314,674.2633401)(39.35326317,674.29334007)(39.31326912,674.30334076)
\curveto(39.29326323,674.32334004)(39.26826326,674.32834003)(39.23826912,674.31834076)
\curveto(39.20826332,674.30834005)(39.18326334,674.29834006)(39.16326912,674.28834076)
\curveto(39.09326343,674.24834011)(39.03326349,674.20334016)(38.98326912,674.15334076)
\curveto(38.93326359,674.10334026)(38.87826365,674.0583403)(38.81826912,674.01834076)
\curveto(38.77826375,673.98834037)(38.73826379,673.95334041)(38.69826912,673.91334076)
\curveto(38.66826386,673.88334048)(38.6282639,673.85334051)(38.57826912,673.82334076)
\curveto(38.34826418,673.68334068)(38.07826445,673.57334079)(37.76826912,673.49334076)
\curveto(37.69826483,673.47334089)(37.6282649,673.4633409)(37.55826912,673.46334076)
\curveto(37.48826504,673.45334091)(37.41326511,673.43834092)(37.33326912,673.41834076)
\curveto(37.29326523,673.40834095)(37.24826528,673.40834095)(37.19826912,673.41834076)
\curveto(37.15826537,673.41834094)(37.11826541,673.41334095)(37.07826912,673.40334076)
\curveto(37.04826548,673.39334097)(36.98326554,673.39334097)(36.88326912,673.40334076)
\curveto(36.79326573,673.40334096)(36.73326579,673.40834095)(36.70326912,673.41834076)
\curveto(36.65326587,673.41834094)(36.60326592,673.42334094)(36.55326912,673.43334076)
\lineto(36.40326912,673.43334076)
\curveto(36.28326624,673.4633409)(36.16826636,673.48834087)(36.05826912,673.50834076)
\curveto(35.94826658,673.52834083)(35.83826669,673.5583408)(35.72826912,673.59834076)
\curveto(35.67826685,673.61834074)(35.63326689,673.63334073)(35.59326912,673.64334076)
\curveto(35.56326696,673.6633407)(35.523267,673.68334068)(35.47326912,673.70334076)
\curveto(35.1232674,673.89334047)(34.84326768,674.1583402)(34.63326912,674.49834076)
\curveto(34.50326802,674.70833965)(34.40826812,674.9583394)(34.34826912,675.24834076)
\curveto(34.28826824,675.54833881)(34.24826828,675.8633385)(34.22826912,676.19334076)
\curveto(34.21826831,676.53333783)(34.21326831,676.87833748)(34.21326912,677.22834076)
\curveto(34.2232683,677.58833677)(34.2282683,677.94333642)(34.22826912,678.29334076)
\lineto(34.22826912,680.33334076)
\curveto(34.2282683,680.4633339)(34.2232683,680.61333375)(34.21326912,680.78334076)
\curveto(34.21326831,680.9633334)(34.23826829,681.09333327)(34.28826912,681.17334076)
\curveto(34.31826821,681.22333314)(34.37826815,681.26833309)(34.46826912,681.30834076)
\curveto(34.528268,681.30833305)(34.57326795,681.31333305)(34.60326912,681.32334076)
}
}
{
\newrgbcolor{curcolor}{0 0 0}
\pscustom[linestyle=none,fillstyle=solid,fillcolor=curcolor]
{
}
}
{
\newrgbcolor{curcolor}{0 0 0}
\pscustom[linestyle=none,fillstyle=solid,fillcolor=curcolor]
{
\newpath
\moveto(50.81967537,681.54834076)
\curveto(51.62967021,681.56833279)(52.30466954,681.44833291)(52.84467537,681.18834076)
\curveto(53.39466845,680.92833343)(53.82966801,680.5583338)(54.14967537,680.07834076)
\curveto(54.30966753,679.83833452)(54.42966741,679.5633348)(54.50967537,679.25334076)
\curveto(54.52966731,679.20333516)(54.5446673,679.13833522)(54.55467537,679.05834076)
\curveto(54.57466727,678.97833538)(54.57466727,678.90833545)(54.55467537,678.84834076)
\curveto(54.51466733,678.73833562)(54.4446674,678.67333569)(54.34467537,678.65334076)
\curveto(54.2446676,678.64333572)(54.12466772,678.63833572)(53.98467537,678.63834076)
\lineto(53.20467537,678.63834076)
\lineto(52.91967537,678.63834076)
\curveto(52.82966901,678.63833572)(52.75466909,678.6583357)(52.69467537,678.69834076)
\curveto(52.61466923,678.73833562)(52.55966928,678.79833556)(52.52967537,678.87834076)
\curveto(52.49966934,678.96833539)(52.45966938,679.0583353)(52.40967537,679.14834076)
\curveto(52.34966949,679.2583351)(52.28466956,679.358335)(52.21467537,679.44834076)
\curveto(52.1446697,679.53833482)(52.06466978,679.61833474)(51.97467537,679.68834076)
\curveto(51.83467001,679.77833458)(51.67967016,679.84833451)(51.50967537,679.89834076)
\curveto(51.44967039,679.91833444)(51.38967045,679.92833443)(51.32967537,679.92834076)
\curveto(51.26967057,679.92833443)(51.21467063,679.93833442)(51.16467537,679.95834076)
\lineto(51.01467537,679.95834076)
\curveto(50.81467103,679.9583344)(50.65467119,679.93833442)(50.53467537,679.89834076)
\curveto(50.2446716,679.80833455)(50.00967183,679.66833469)(49.82967537,679.47834076)
\curveto(49.64967219,679.29833506)(49.50467234,679.07833528)(49.39467537,678.81834076)
\curveto(49.3446725,678.70833565)(49.30467254,678.58833577)(49.27467537,678.45834076)
\curveto(49.25467259,678.33833602)(49.22967261,678.20833615)(49.19967537,678.06834076)
\curveto(49.18967265,678.02833633)(49.18467266,677.98833637)(49.18467537,677.94834076)
\curveto(49.18467266,677.90833645)(49.17967266,677.86833649)(49.16967537,677.82834076)
\curveto(49.14967269,677.72833663)(49.1396727,677.58833677)(49.13967537,677.40834076)
\curveto(49.14967269,677.22833713)(49.16467268,677.08833727)(49.18467537,676.98834076)
\curveto(49.18467266,676.90833745)(49.18967265,676.85333751)(49.19967537,676.82334076)
\curveto(49.21967262,676.75333761)(49.22967261,676.68333768)(49.22967537,676.61334076)
\curveto(49.2396726,676.54333782)(49.25467259,676.47333789)(49.27467537,676.40334076)
\curveto(49.35467249,676.17333819)(49.44967239,675.9633384)(49.55967537,675.77334076)
\curveto(49.66967217,675.58333878)(49.80967203,675.42333894)(49.97967537,675.29334076)
\curveto(50.01967182,675.2633391)(50.07967176,675.22833913)(50.15967537,675.18834076)
\curveto(50.26967157,675.11833924)(50.37967146,675.07333929)(50.48967537,675.05334076)
\curveto(50.60967123,675.03333933)(50.75467109,675.01333935)(50.92467537,674.99334076)
\lineto(51.01467537,674.99334076)
\curveto(51.05467079,674.99333937)(51.08467076,674.99833936)(51.10467537,675.00834076)
\lineto(51.23967537,675.00834076)
\curveto(51.30967053,675.02833933)(51.37467047,675.04333932)(51.43467537,675.05334076)
\curveto(51.50467034,675.07333929)(51.56967027,675.09333927)(51.62967537,675.11334076)
\curveto(51.92966991,675.24333912)(52.15966968,675.43333893)(52.31967537,675.68334076)
\curveto(52.35966948,675.73333863)(52.39466945,675.78833857)(52.42467537,675.84834076)
\curveto(52.45466939,675.91833844)(52.47966936,675.97833838)(52.49967537,676.02834076)
\curveto(52.5396693,676.13833822)(52.57466927,676.23333813)(52.60467537,676.31334076)
\curveto(52.63466921,676.40333796)(52.70466914,676.47333789)(52.81467537,676.52334076)
\curveto(52.90466894,676.5633378)(53.04966879,676.57833778)(53.24967537,676.56834076)
\lineto(53.74467537,676.56834076)
\lineto(53.95467537,676.56834076)
\curveto(54.03466781,676.57833778)(54.09966774,676.57333779)(54.14967537,676.55334076)
\lineto(54.26967537,676.55334076)
\lineto(54.38967537,676.52334076)
\curveto(54.42966741,676.52333784)(54.45966738,676.51333785)(54.47967537,676.49334076)
\curveto(54.52966731,676.45333791)(54.55966728,676.39333797)(54.56967537,676.31334076)
\curveto(54.58966725,676.24333812)(54.58966725,676.16833819)(54.56967537,676.08834076)
\curveto(54.47966736,675.7583386)(54.36966747,675.4633389)(54.23967537,675.20334076)
\curveto(53.82966801,674.43333993)(53.17466867,673.89834046)(52.27467537,673.59834076)
\curveto(52.17466967,673.56834079)(52.06966977,673.54834081)(51.95967537,673.53834076)
\curveto(51.84966999,673.51834084)(51.7396701,673.49334087)(51.62967537,673.46334076)
\curveto(51.56967027,673.45334091)(51.50967033,673.44834091)(51.44967537,673.44834076)
\curveto(51.38967045,673.44834091)(51.32967051,673.44334092)(51.26967537,673.43334076)
\lineto(51.10467537,673.43334076)
\curveto(51.05467079,673.41334095)(50.97967086,673.40834095)(50.87967537,673.41834076)
\curveto(50.77967106,673.41834094)(50.70467114,673.42334094)(50.65467537,673.43334076)
\curveto(50.57467127,673.45334091)(50.49967134,673.4633409)(50.42967537,673.46334076)
\curveto(50.36967147,673.45334091)(50.30467154,673.4583409)(50.23467537,673.47834076)
\lineto(50.08467537,673.50834076)
\curveto(50.03467181,673.50834085)(49.98467186,673.51334085)(49.93467537,673.52334076)
\curveto(49.82467202,673.55334081)(49.71967212,673.58334078)(49.61967537,673.61334076)
\curveto(49.51967232,673.64334072)(49.42467242,673.67834068)(49.33467537,673.71834076)
\curveto(48.86467298,673.91834044)(48.46967337,674.17334019)(48.14967537,674.48334076)
\curveto(47.82967401,674.80333956)(47.56967427,675.19833916)(47.36967537,675.66834076)
\curveto(47.31967452,675.7583386)(47.27967456,675.85333851)(47.24967537,675.95334076)
\lineto(47.15967537,676.28334076)
\curveto(47.14967469,676.32333804)(47.1446747,676.358338)(47.14467537,676.38834076)
\curveto(47.1446747,676.42833793)(47.13467471,676.47333789)(47.11467537,676.52334076)
\curveto(47.09467475,676.59333777)(47.08467476,676.6633377)(47.08467537,676.73334076)
\curveto(47.08467476,676.81333755)(47.07467477,676.88833747)(47.05467537,676.95834076)
\lineto(47.05467537,677.21334076)
\curveto(47.03467481,677.2633371)(47.02467482,677.31833704)(47.02467537,677.37834076)
\curveto(47.02467482,677.44833691)(47.03467481,677.50833685)(47.05467537,677.55834076)
\curveto(47.06467478,677.60833675)(47.06467478,677.65333671)(47.05467537,677.69334076)
\curveto(47.0446748,677.73333663)(47.0446748,677.77333659)(47.05467537,677.81334076)
\curveto(47.07467477,677.88333648)(47.07967476,677.94833641)(47.06967537,678.00834076)
\curveto(47.06967477,678.06833629)(47.07967476,678.12833623)(47.09967537,678.18834076)
\curveto(47.14967469,678.36833599)(47.18967465,678.53833582)(47.21967537,678.69834076)
\curveto(47.24967459,678.86833549)(47.29467455,679.03333533)(47.35467537,679.19334076)
\curveto(47.57467427,679.70333466)(47.84967399,680.12833423)(48.17967537,680.46834076)
\curveto(48.51967332,680.80833355)(48.94967289,681.08333328)(49.46967537,681.29334076)
\curveto(49.60967223,681.35333301)(49.75467209,681.39333297)(49.90467537,681.41334076)
\curveto(50.05467179,681.44333292)(50.20967163,681.47833288)(50.36967537,681.51834076)
\curveto(50.44967139,681.52833283)(50.52467132,681.53333283)(50.59467537,681.53334076)
\curveto(50.66467118,681.53333283)(50.7396711,681.53833282)(50.81967537,681.54834076)
}
}
{
\newrgbcolor{curcolor}{0 0 0}
\pscustom[linestyle=none,fillstyle=solid,fillcolor=curcolor]
{
\newpath
\moveto(63.63295662,677.78334076)
\curveto(63.65294805,677.72333664)(63.66294804,677.63833672)(63.66295662,677.52834076)
\curveto(63.66294804,677.41833694)(63.65294805,677.33333703)(63.63295662,677.27334076)
\lineto(63.63295662,677.12334076)
\curveto(63.61294809,677.04333732)(63.6029481,676.9633374)(63.60295662,676.88334076)
\curveto(63.61294809,676.80333756)(63.6079481,676.72333764)(63.58795662,676.64334076)
\curveto(63.56794814,676.57333779)(63.55294815,676.50833785)(63.54295662,676.44834076)
\curveto(63.53294817,676.38833797)(63.52294818,676.32333804)(63.51295662,676.25334076)
\curveto(63.47294823,676.14333822)(63.43794827,676.02833833)(63.40795662,675.90834076)
\curveto(63.37794833,675.79833856)(63.33794837,675.69333867)(63.28795662,675.59334076)
\curveto(63.07794863,675.11333925)(62.8029489,674.72333964)(62.46295662,674.42334076)
\curveto(62.12294958,674.12334024)(61.71294999,673.87334049)(61.23295662,673.67334076)
\curveto(61.11295059,673.62334074)(60.98795072,673.58834077)(60.85795662,673.56834076)
\curveto(60.73795097,673.53834082)(60.61295109,673.50834085)(60.48295662,673.47834076)
\curveto(60.43295127,673.4583409)(60.37795133,673.44834091)(60.31795662,673.44834076)
\curveto(60.25795145,673.44834091)(60.2029515,673.44334092)(60.15295662,673.43334076)
\lineto(60.04795662,673.43334076)
\curveto(60.01795169,673.42334094)(59.98795172,673.41834094)(59.95795662,673.41834076)
\curveto(59.9079518,673.40834095)(59.82795188,673.40334096)(59.71795662,673.40334076)
\curveto(59.6079521,673.39334097)(59.52295218,673.39834096)(59.46295662,673.41834076)
\lineto(59.31295662,673.41834076)
\curveto(59.26295244,673.42834093)(59.2079525,673.43334093)(59.14795662,673.43334076)
\curveto(59.09795261,673.42334094)(59.04795266,673.42834093)(58.99795662,673.44834076)
\curveto(58.95795275,673.4583409)(58.91795279,673.4633409)(58.87795662,673.46334076)
\curveto(58.84795286,673.4633409)(58.8079529,673.46834089)(58.75795662,673.47834076)
\curveto(58.65795305,673.50834085)(58.55795315,673.53334083)(58.45795662,673.55334076)
\curveto(58.35795335,673.57334079)(58.26295344,673.60334076)(58.17295662,673.64334076)
\curveto(58.05295365,673.68334068)(57.93795377,673.72334064)(57.82795662,673.76334076)
\curveto(57.72795398,673.80334056)(57.62295408,673.85334051)(57.51295662,673.91334076)
\curveto(57.16295454,674.12334024)(56.86295484,674.36833999)(56.61295662,674.64834076)
\curveto(56.36295534,674.92833943)(56.15295555,675.2633391)(55.98295662,675.65334076)
\curveto(55.93295577,675.74333862)(55.89295581,675.83833852)(55.86295662,675.93834076)
\curveto(55.84295586,676.03833832)(55.81795589,676.14333822)(55.78795662,676.25334076)
\curveto(55.76795594,676.30333806)(55.75795595,676.34833801)(55.75795662,676.38834076)
\curveto(55.75795595,676.42833793)(55.74795596,676.47333789)(55.72795662,676.52334076)
\curveto(55.707956,676.60333776)(55.69795601,676.68333768)(55.69795662,676.76334076)
\curveto(55.69795601,676.85333751)(55.68795602,676.93833742)(55.66795662,677.01834076)
\curveto(55.65795605,677.06833729)(55.65295605,677.11333725)(55.65295662,677.15334076)
\lineto(55.65295662,677.28834076)
\curveto(55.63295607,677.34833701)(55.62295608,677.43333693)(55.62295662,677.54334076)
\curveto(55.63295607,677.65333671)(55.64795606,677.73833662)(55.66795662,677.79834076)
\lineto(55.66795662,677.90334076)
\curveto(55.67795603,677.95333641)(55.67795603,678.00333636)(55.66795662,678.05334076)
\curveto(55.66795604,678.11333625)(55.67795603,678.16833619)(55.69795662,678.21834076)
\curveto(55.707956,678.26833609)(55.71295599,678.31333605)(55.71295662,678.35334076)
\curveto(55.71295599,678.40333596)(55.72295598,678.45333591)(55.74295662,678.50334076)
\curveto(55.78295592,678.63333573)(55.81795589,678.7583356)(55.84795662,678.87834076)
\curveto(55.87795583,679.00833535)(55.91795579,679.13333523)(55.96795662,679.25334076)
\curveto(56.14795556,679.6633347)(56.36295534,680.00333436)(56.61295662,680.27334076)
\curveto(56.86295484,680.55333381)(57.16795454,680.80833355)(57.52795662,681.03834076)
\curveto(57.62795408,681.08833327)(57.73295397,681.13333323)(57.84295662,681.17334076)
\curveto(57.95295375,681.21333315)(58.06295364,681.2583331)(58.17295662,681.30834076)
\curveto(58.3029534,681.358333)(58.43795327,681.39333297)(58.57795662,681.41334076)
\curveto(58.71795299,681.43333293)(58.86295284,681.4633329)(59.01295662,681.50334076)
\curveto(59.09295261,681.51333285)(59.16795254,681.51833284)(59.23795662,681.51834076)
\curveto(59.3079524,681.51833284)(59.37795233,681.52333284)(59.44795662,681.53334076)
\curveto(60.02795168,681.54333282)(60.52795118,681.48333288)(60.94795662,681.35334076)
\curveto(61.37795033,681.22333314)(61.75794995,681.04333332)(62.08795662,680.81334076)
\curveto(62.19794951,680.73333363)(62.3079494,680.64333372)(62.41795662,680.54334076)
\curveto(62.53794917,680.45333391)(62.63794907,680.35333401)(62.71795662,680.24334076)
\curveto(62.79794891,680.14333422)(62.86794884,680.04333432)(62.92795662,679.94334076)
\curveto(62.99794871,679.84333452)(63.06794864,679.73833462)(63.13795662,679.62834076)
\curveto(63.2079485,679.51833484)(63.26294844,679.39833496)(63.30295662,679.26834076)
\curveto(63.34294836,679.14833521)(63.38794832,679.01833534)(63.43795662,678.87834076)
\curveto(63.46794824,678.79833556)(63.49294821,678.71333565)(63.51295662,678.62334076)
\lineto(63.57295662,678.35334076)
\curveto(63.58294812,678.31333605)(63.58794812,678.27333609)(63.58795662,678.23334076)
\curveto(63.58794812,678.19333617)(63.59294811,678.15333621)(63.60295662,678.11334076)
\curveto(63.62294808,678.0633363)(63.62794808,678.00833635)(63.61795662,677.94834076)
\curveto(63.6079481,677.88833647)(63.61294809,677.83333653)(63.63295662,677.78334076)
\moveto(61.53295662,677.24334076)
\curveto(61.54295016,677.29333707)(61.54795016,677.363337)(61.54795662,677.45334076)
\curveto(61.54795016,677.55333681)(61.54295016,677.62833673)(61.53295662,677.67834076)
\lineto(61.53295662,677.79834076)
\curveto(61.51295019,677.84833651)(61.5029502,677.90333646)(61.50295662,677.96334076)
\curveto(61.5029502,678.02333634)(61.49795021,678.07833628)(61.48795662,678.12834076)
\curveto(61.48795022,678.16833619)(61.48295022,678.19833616)(61.47295662,678.21834076)
\lineto(61.41295662,678.45834076)
\curveto(61.4029503,678.54833581)(61.38295032,678.63333573)(61.35295662,678.71334076)
\curveto(61.24295046,678.97333539)(61.11295059,679.19333517)(60.96295662,679.37334076)
\curveto(60.81295089,679.5633348)(60.61295109,679.71333465)(60.36295662,679.82334076)
\curveto(60.3029514,679.84333452)(60.24295146,679.8583345)(60.18295662,679.86834076)
\curveto(60.12295158,679.88833447)(60.05795165,679.90833445)(59.98795662,679.92834076)
\curveto(59.9079518,679.94833441)(59.82295188,679.95333441)(59.73295662,679.94334076)
\lineto(59.46295662,679.94334076)
\curveto(59.43295227,679.92333444)(59.39795231,679.91333445)(59.35795662,679.91334076)
\curveto(59.31795239,679.92333444)(59.28295242,679.92333444)(59.25295662,679.91334076)
\lineto(59.04295662,679.85334076)
\curveto(58.98295272,679.84333452)(58.92795278,679.82333454)(58.87795662,679.79334076)
\curveto(58.62795308,679.68333468)(58.42295328,679.52333484)(58.26295662,679.31334076)
\curveto(58.11295359,679.11333525)(57.99295371,678.87833548)(57.90295662,678.60834076)
\curveto(57.87295383,678.50833585)(57.84795386,678.40333596)(57.82795662,678.29334076)
\curveto(57.81795389,678.18333618)(57.8029539,678.07333629)(57.78295662,677.96334076)
\curveto(57.77295393,677.91333645)(57.76795394,677.8633365)(57.76795662,677.81334076)
\lineto(57.76795662,677.66334076)
\curveto(57.74795396,677.59333677)(57.73795397,677.48833687)(57.73795662,677.34834076)
\curveto(57.74795396,677.20833715)(57.76295394,677.10333726)(57.78295662,677.03334076)
\lineto(57.78295662,676.89834076)
\curveto(57.8029539,676.81833754)(57.81795389,676.73833762)(57.82795662,676.65834076)
\curveto(57.83795387,676.58833777)(57.85295385,676.51333785)(57.87295662,676.43334076)
\curveto(57.97295373,676.13333823)(58.07795363,675.88833847)(58.18795662,675.69834076)
\curveto(58.3079534,675.51833884)(58.49295321,675.35333901)(58.74295662,675.20334076)
\curveto(58.81295289,675.15333921)(58.88795282,675.11333925)(58.96795662,675.08334076)
\curveto(59.05795265,675.05333931)(59.14795256,675.02833933)(59.23795662,675.00834076)
\curveto(59.27795243,674.99833936)(59.31295239,674.99333937)(59.34295662,674.99334076)
\curveto(59.37295233,675.00333936)(59.4079523,675.00333936)(59.44795662,674.99334076)
\lineto(59.56795662,674.96334076)
\curveto(59.61795209,674.9633394)(59.66295204,674.96833939)(59.70295662,674.97834076)
\lineto(59.82295662,674.97834076)
\curveto(59.9029518,674.99833936)(59.98295172,675.01333935)(60.06295662,675.02334076)
\curveto(60.14295156,675.03333933)(60.21795149,675.05333931)(60.28795662,675.08334076)
\curveto(60.54795116,675.18333918)(60.75795095,675.31833904)(60.91795662,675.48834076)
\curveto(61.07795063,675.6583387)(61.21295049,675.86833849)(61.32295662,676.11834076)
\curveto(61.36295034,676.21833814)(61.39295031,676.31833804)(61.41295662,676.41834076)
\curveto(61.43295027,676.51833784)(61.45795025,676.62333774)(61.48795662,676.73334076)
\curveto(61.49795021,676.77333759)(61.5029502,676.80833755)(61.50295662,676.83834076)
\curveto(61.5029502,676.87833748)(61.5079502,676.91833744)(61.51795662,676.95834076)
\lineto(61.51795662,677.09334076)
\curveto(61.51795019,677.14333722)(61.52295018,677.19333717)(61.53295662,677.24334076)
}
}
{
\newrgbcolor{curcolor}{0 0 0}
\pscustom[linestyle=none,fillstyle=solid,fillcolor=curcolor]
{
\newpath
\moveto(69.4578785,681.53334076)
\curveto(69.56787318,681.53333283)(69.66287309,681.52333284)(69.7428785,681.50334076)
\curveto(69.83287292,681.48333288)(69.90287285,681.43833292)(69.9528785,681.36834076)
\curveto(70.01287274,681.28833307)(70.04287271,681.14833321)(70.0428785,680.94834076)
\lineto(70.0428785,680.43834076)
\lineto(70.0428785,680.06334076)
\curveto(70.0528727,679.92333444)(70.03787271,679.81333455)(69.9978785,679.73334076)
\curveto(69.95787279,679.6633347)(69.89787285,679.61833474)(69.8178785,679.59834076)
\curveto(69.747873,679.57833478)(69.66287309,679.56833479)(69.5628785,679.56834076)
\curveto(69.47287328,679.56833479)(69.37287338,679.57333479)(69.2628785,679.58334076)
\curveto(69.16287359,679.59333477)(69.06787368,679.58833477)(68.9778785,679.56834076)
\curveto(68.90787384,679.54833481)(68.83787391,679.53333483)(68.7678785,679.52334076)
\curveto(68.69787405,679.52333484)(68.63287412,679.51333485)(68.5728785,679.49334076)
\curveto(68.41287434,679.44333492)(68.2528745,679.36833499)(68.0928785,679.26834076)
\curveto(67.93287482,679.17833518)(67.80787494,679.07333529)(67.7178785,678.95334076)
\curveto(67.66787508,678.87333549)(67.61287514,678.78833557)(67.5528785,678.69834076)
\curveto(67.50287525,678.61833574)(67.4528753,678.53333583)(67.4028785,678.44334076)
\curveto(67.37287538,678.363336)(67.34287541,678.27833608)(67.3128785,678.18834076)
\lineto(67.2528785,677.94834076)
\curveto(67.23287552,677.87833648)(67.22287553,677.80333656)(67.2228785,677.72334076)
\curveto(67.22287553,677.65333671)(67.21287554,677.58333678)(67.1928785,677.51334076)
\curveto(67.18287557,677.47333689)(67.17787557,677.43333693)(67.1778785,677.39334076)
\curveto(67.18787556,677.363337)(67.18787556,677.33333703)(67.1778785,677.30334076)
\lineto(67.1778785,677.06334076)
\curveto(67.15787559,676.99333737)(67.1528756,676.91333745)(67.1628785,676.82334076)
\curveto(67.17287558,676.74333762)(67.17787557,676.6633377)(67.1778785,676.58334076)
\lineto(67.1778785,675.62334076)
\lineto(67.1778785,674.34834076)
\curveto(67.17787557,674.21834014)(67.17287558,674.09834026)(67.1628785,673.98834076)
\curveto(67.1528756,673.87834048)(67.12287563,673.78834057)(67.0728785,673.71834076)
\curveto(67.0528757,673.68834067)(67.01787573,673.6633407)(66.9678785,673.64334076)
\curveto(66.92787582,673.63334073)(66.88287587,673.62334074)(66.8328785,673.61334076)
\lineto(66.7578785,673.61334076)
\curveto(66.70787604,673.60334076)(66.6528761,673.59834076)(66.5928785,673.59834076)
\lineto(66.4278785,673.59834076)
\lineto(65.7828785,673.59834076)
\curveto(65.72287703,673.60834075)(65.65787709,673.61334075)(65.5878785,673.61334076)
\lineto(65.3928785,673.61334076)
\curveto(65.34287741,673.63334073)(65.29287746,673.64834071)(65.2428785,673.65834076)
\curveto(65.19287756,673.67834068)(65.15787759,673.71334065)(65.1378785,673.76334076)
\curveto(65.09787765,673.81334055)(65.07287768,673.88334048)(65.0628785,673.97334076)
\lineto(65.0628785,674.27334076)
\lineto(65.0628785,675.29334076)
\lineto(65.0628785,679.52334076)
\lineto(65.0628785,680.63334076)
\lineto(65.0628785,680.91834076)
\curveto(65.06287769,681.01833334)(65.08287767,681.09833326)(65.1228785,681.15834076)
\curveto(65.17287758,681.23833312)(65.2478775,681.28833307)(65.3478785,681.30834076)
\curveto(65.4478773,681.32833303)(65.56787718,681.33833302)(65.7078785,681.33834076)
\lineto(66.4728785,681.33834076)
\curveto(66.59287616,681.33833302)(66.69787605,681.32833303)(66.7878785,681.30834076)
\curveto(66.87787587,681.29833306)(66.9478758,681.25333311)(66.9978785,681.17334076)
\curveto(67.02787572,681.12333324)(67.04287571,681.05333331)(67.0428785,680.96334076)
\lineto(67.0728785,680.69334076)
\curveto(67.08287567,680.61333375)(67.09787565,680.53833382)(67.1178785,680.46834076)
\curveto(67.1478756,680.39833396)(67.19787555,680.363334)(67.2678785,680.36334076)
\curveto(67.28787546,680.38333398)(67.30787544,680.39333397)(67.3278785,680.39334076)
\curveto(67.3478754,680.39333397)(67.36787538,680.40333396)(67.3878785,680.42334076)
\curveto(67.4478753,680.47333389)(67.49787525,680.52833383)(67.5378785,680.58834076)
\curveto(67.58787516,680.6583337)(67.6478751,680.71833364)(67.7178785,680.76834076)
\curveto(67.75787499,680.79833356)(67.79287496,680.82833353)(67.8228785,680.85834076)
\curveto(67.8528749,680.89833346)(67.88787486,680.93333343)(67.9278785,680.96334076)
\lineto(68.1978785,681.14334076)
\curveto(68.29787445,681.20333316)(68.39787435,681.2583331)(68.4978785,681.30834076)
\curveto(68.59787415,681.34833301)(68.69787405,681.38333298)(68.7978785,681.41334076)
\lineto(69.1278785,681.50334076)
\curveto(69.15787359,681.51333285)(69.21287354,681.51333285)(69.2928785,681.50334076)
\curveto(69.38287337,681.50333286)(69.43787331,681.51333285)(69.4578785,681.53334076)
}
}
{
\newrgbcolor{curcolor}{0 0 0}
\pscustom[linestyle=none,fillstyle=solid,fillcolor=curcolor]
{
\newpath
\moveto(75.28795662,681.53334076)
\curveto(75.39795131,681.53333283)(75.49295121,681.52333284)(75.57295662,681.50334076)
\curveto(75.66295104,681.48333288)(75.73295097,681.43833292)(75.78295662,681.36834076)
\curveto(75.84295086,681.28833307)(75.87295083,681.14833321)(75.87295662,680.94834076)
\lineto(75.87295662,680.43834076)
\lineto(75.87295662,680.06334076)
\curveto(75.88295082,679.92333444)(75.86795084,679.81333455)(75.82795662,679.73334076)
\curveto(75.78795092,679.6633347)(75.72795098,679.61833474)(75.64795662,679.59834076)
\curveto(75.57795113,679.57833478)(75.49295121,679.56833479)(75.39295662,679.56834076)
\curveto(75.3029514,679.56833479)(75.2029515,679.57333479)(75.09295662,679.58334076)
\curveto(74.99295171,679.59333477)(74.89795181,679.58833477)(74.80795662,679.56834076)
\curveto(74.73795197,679.54833481)(74.66795204,679.53333483)(74.59795662,679.52334076)
\curveto(74.52795218,679.52333484)(74.46295224,679.51333485)(74.40295662,679.49334076)
\curveto(74.24295246,679.44333492)(74.08295262,679.36833499)(73.92295662,679.26834076)
\curveto(73.76295294,679.17833518)(73.63795307,679.07333529)(73.54795662,678.95334076)
\curveto(73.49795321,678.87333549)(73.44295326,678.78833557)(73.38295662,678.69834076)
\curveto(73.33295337,678.61833574)(73.28295342,678.53333583)(73.23295662,678.44334076)
\curveto(73.2029535,678.363336)(73.17295353,678.27833608)(73.14295662,678.18834076)
\lineto(73.08295662,677.94834076)
\curveto(73.06295364,677.87833648)(73.05295365,677.80333656)(73.05295662,677.72334076)
\curveto(73.05295365,677.65333671)(73.04295366,677.58333678)(73.02295662,677.51334076)
\curveto(73.01295369,677.47333689)(73.0079537,677.43333693)(73.00795662,677.39334076)
\curveto(73.01795369,677.363337)(73.01795369,677.33333703)(73.00795662,677.30334076)
\lineto(73.00795662,677.06334076)
\curveto(72.98795372,676.99333737)(72.98295372,676.91333745)(72.99295662,676.82334076)
\curveto(73.0029537,676.74333762)(73.0079537,676.6633377)(73.00795662,676.58334076)
\lineto(73.00795662,675.62334076)
\lineto(73.00795662,674.34834076)
\curveto(73.0079537,674.21834014)(73.0029537,674.09834026)(72.99295662,673.98834076)
\curveto(72.98295372,673.87834048)(72.95295375,673.78834057)(72.90295662,673.71834076)
\curveto(72.88295382,673.68834067)(72.84795386,673.6633407)(72.79795662,673.64334076)
\curveto(72.75795395,673.63334073)(72.71295399,673.62334074)(72.66295662,673.61334076)
\lineto(72.58795662,673.61334076)
\curveto(72.53795417,673.60334076)(72.48295422,673.59834076)(72.42295662,673.59834076)
\lineto(72.25795662,673.59834076)
\lineto(71.61295662,673.59834076)
\curveto(71.55295515,673.60834075)(71.48795522,673.61334075)(71.41795662,673.61334076)
\lineto(71.22295662,673.61334076)
\curveto(71.17295553,673.63334073)(71.12295558,673.64834071)(71.07295662,673.65834076)
\curveto(71.02295568,673.67834068)(70.98795572,673.71334065)(70.96795662,673.76334076)
\curveto(70.92795578,673.81334055)(70.9029558,673.88334048)(70.89295662,673.97334076)
\lineto(70.89295662,674.27334076)
\lineto(70.89295662,675.29334076)
\lineto(70.89295662,679.52334076)
\lineto(70.89295662,680.63334076)
\lineto(70.89295662,680.91834076)
\curveto(70.89295581,681.01833334)(70.91295579,681.09833326)(70.95295662,681.15834076)
\curveto(71.0029557,681.23833312)(71.07795563,681.28833307)(71.17795662,681.30834076)
\curveto(71.27795543,681.32833303)(71.39795531,681.33833302)(71.53795662,681.33834076)
\lineto(72.30295662,681.33834076)
\curveto(72.42295428,681.33833302)(72.52795418,681.32833303)(72.61795662,681.30834076)
\curveto(72.707954,681.29833306)(72.77795393,681.25333311)(72.82795662,681.17334076)
\curveto(72.85795385,681.12333324)(72.87295383,681.05333331)(72.87295662,680.96334076)
\lineto(72.90295662,680.69334076)
\curveto(72.91295379,680.61333375)(72.92795378,680.53833382)(72.94795662,680.46834076)
\curveto(72.97795373,680.39833396)(73.02795368,680.363334)(73.09795662,680.36334076)
\curveto(73.11795359,680.38333398)(73.13795357,680.39333397)(73.15795662,680.39334076)
\curveto(73.17795353,680.39333397)(73.19795351,680.40333396)(73.21795662,680.42334076)
\curveto(73.27795343,680.47333389)(73.32795338,680.52833383)(73.36795662,680.58834076)
\curveto(73.41795329,680.6583337)(73.47795323,680.71833364)(73.54795662,680.76834076)
\curveto(73.58795312,680.79833356)(73.62295308,680.82833353)(73.65295662,680.85834076)
\curveto(73.68295302,680.89833346)(73.71795299,680.93333343)(73.75795662,680.96334076)
\lineto(74.02795662,681.14334076)
\curveto(74.12795258,681.20333316)(74.22795248,681.2583331)(74.32795662,681.30834076)
\curveto(74.42795228,681.34833301)(74.52795218,681.38333298)(74.62795662,681.41334076)
\lineto(74.95795662,681.50334076)
\curveto(74.98795172,681.51333285)(75.04295166,681.51333285)(75.12295662,681.50334076)
\curveto(75.21295149,681.50333286)(75.26795144,681.51333285)(75.28795662,681.53334076)
}
}
{
\newrgbcolor{curcolor}{0 0 0}
\pscustom[linestyle=none,fillstyle=solid,fillcolor=curcolor]
{
\newpath
\moveto(83.79436287,677.54334076)
\curveto(83.81435471,677.4633369)(83.81435471,677.37333699)(83.79436287,677.27334076)
\curveto(83.77435475,677.17333719)(83.73935478,677.10833725)(83.68936287,677.07834076)
\curveto(83.63935488,677.03833732)(83.56435496,677.00833735)(83.46436287,676.98834076)
\curveto(83.37435515,676.97833738)(83.26935525,676.96833739)(83.14936287,676.95834076)
\lineto(82.80436287,676.95834076)
\curveto(82.69435583,676.96833739)(82.59435593,676.97333739)(82.50436287,676.97334076)
\lineto(78.84436287,676.97334076)
\lineto(78.63436287,676.97334076)
\curveto(78.57435995,676.97333739)(78.51936,676.9633374)(78.46936287,676.94334076)
\curveto(78.38936013,676.90333746)(78.33936018,676.8633375)(78.31936287,676.82334076)
\curveto(78.29936022,676.80333756)(78.27936024,676.7633376)(78.25936287,676.70334076)
\curveto(78.23936028,676.65333771)(78.23436029,676.60333776)(78.24436287,676.55334076)
\curveto(78.26436026,676.49333787)(78.27436025,676.43333793)(78.27436287,676.37334076)
\curveto(78.28436024,676.32333804)(78.29936022,676.26833809)(78.31936287,676.20834076)
\curveto(78.39936012,675.96833839)(78.49436003,675.76833859)(78.60436287,675.60834076)
\curveto(78.7243598,675.4583389)(78.88435964,675.32333904)(79.08436287,675.20334076)
\curveto(79.16435936,675.15333921)(79.24435928,675.11833924)(79.32436287,675.09834076)
\curveto(79.41435911,675.08833927)(79.50435902,675.06833929)(79.59436287,675.03834076)
\curveto(79.67435885,675.01833934)(79.78435874,675.00333936)(79.92436287,674.99334076)
\curveto(80.06435846,674.98333938)(80.18435834,674.98833937)(80.28436287,675.00834076)
\lineto(80.41936287,675.00834076)
\curveto(80.519358,675.02833933)(80.60935791,675.04833931)(80.68936287,675.06834076)
\curveto(80.77935774,675.09833926)(80.86435766,675.12833923)(80.94436287,675.15834076)
\curveto(81.04435748,675.20833915)(81.15435737,675.27333909)(81.27436287,675.35334076)
\curveto(81.40435712,675.43333893)(81.49935702,675.51333885)(81.55936287,675.59334076)
\curveto(81.60935691,675.6633387)(81.65935686,675.72833863)(81.70936287,675.78834076)
\curveto(81.76935675,675.8583385)(81.83935668,675.90833845)(81.91936287,675.93834076)
\curveto(82.0193565,675.98833837)(82.14435638,676.00833835)(82.29436287,675.99834076)
\lineto(82.72936287,675.99834076)
\lineto(82.90936287,675.99834076)
\curveto(82.97935554,676.00833835)(83.03935548,676.00333836)(83.08936287,675.98334076)
\lineto(83.23936287,675.98334076)
\curveto(83.33935518,675.9633384)(83.40935511,675.93833842)(83.44936287,675.90834076)
\curveto(83.48935503,675.88833847)(83.50935501,675.84333852)(83.50936287,675.77334076)
\curveto(83.519355,675.70333866)(83.51435501,675.64333872)(83.49436287,675.59334076)
\curveto(83.44435508,675.45333891)(83.38935513,675.32833903)(83.32936287,675.21834076)
\curveto(83.26935525,675.10833925)(83.19935532,674.99833936)(83.11936287,674.88834076)
\curveto(82.89935562,674.5583398)(82.64935587,674.29334007)(82.36936287,674.09334076)
\curveto(82.08935643,673.89334047)(81.73935678,673.72334064)(81.31936287,673.58334076)
\curveto(81.20935731,673.54334082)(81.09935742,673.51834084)(80.98936287,673.50834076)
\curveto(80.87935764,673.49834086)(80.76435776,673.47834088)(80.64436287,673.44834076)
\curveto(80.60435792,673.43834092)(80.55935796,673.43834092)(80.50936287,673.44834076)
\curveto(80.46935805,673.44834091)(80.42935809,673.44334092)(80.38936287,673.43334076)
\lineto(80.22436287,673.43334076)
\curveto(80.17435835,673.41334095)(80.11435841,673.40834095)(80.04436287,673.41834076)
\curveto(79.98435854,673.41834094)(79.92935859,673.42334094)(79.87936287,673.43334076)
\curveto(79.79935872,673.44334092)(79.72935879,673.44334092)(79.66936287,673.43334076)
\curveto(79.60935891,673.42334094)(79.54435898,673.42834093)(79.47436287,673.44834076)
\curveto(79.4243591,673.46834089)(79.36935915,673.47834088)(79.30936287,673.47834076)
\curveto(79.24935927,673.47834088)(79.19435933,673.48834087)(79.14436287,673.50834076)
\curveto(79.03435949,673.52834083)(78.9243596,673.55334081)(78.81436287,673.58334076)
\curveto(78.70435982,673.60334076)(78.60435992,673.63834072)(78.51436287,673.68834076)
\curveto(78.40436012,673.72834063)(78.29936022,673.7633406)(78.19936287,673.79334076)
\curveto(78.10936041,673.83334053)(78.0243605,673.87834048)(77.94436287,673.92834076)
\curveto(77.6243609,674.12834023)(77.33936118,674.35834)(77.08936287,674.61834076)
\curveto(76.83936168,674.88833947)(76.63436189,675.19833916)(76.47436287,675.54834076)
\curveto(76.4243621,675.6583387)(76.38436214,675.76833859)(76.35436287,675.87834076)
\curveto(76.3243622,675.99833836)(76.28436224,676.11833824)(76.23436287,676.23834076)
\curveto(76.2243623,676.27833808)(76.2193623,676.31333805)(76.21936287,676.34334076)
\curveto(76.2193623,676.38333798)(76.21436231,676.42333794)(76.20436287,676.46334076)
\curveto(76.16436236,676.58333778)(76.13936238,676.71333765)(76.12936287,676.85334076)
\lineto(76.09936287,677.27334076)
\curveto(76.09936242,677.32333704)(76.09436243,677.37833698)(76.08436287,677.43834076)
\curveto(76.08436244,677.49833686)(76.08936243,677.55333681)(76.09936287,677.60334076)
\lineto(76.09936287,677.78334076)
\lineto(76.14436287,678.14334076)
\curveto(76.18436234,678.31333605)(76.2193623,678.47833588)(76.24936287,678.63834076)
\curveto(76.27936224,678.79833556)(76.3243622,678.94833541)(76.38436287,679.08834076)
\curveto(76.81436171,680.12833423)(77.54436098,680.8633335)(78.57436287,681.29334076)
\curveto(78.71435981,681.35333301)(78.85435967,681.39333297)(78.99436287,681.41334076)
\curveto(79.14435938,681.44333292)(79.29935922,681.47833288)(79.45936287,681.51834076)
\curveto(79.53935898,681.52833283)(79.61435891,681.53333283)(79.68436287,681.53334076)
\curveto(79.75435877,681.53333283)(79.82935869,681.53833282)(79.90936287,681.54834076)
\curveto(80.4193581,681.5583328)(80.85435767,681.49833286)(81.21436287,681.36834076)
\curveto(81.58435694,681.24833311)(81.91435661,681.08833327)(82.20436287,680.88834076)
\curveto(82.29435623,680.82833353)(82.38435614,680.7583336)(82.47436287,680.67834076)
\curveto(82.56435596,680.60833375)(82.64435588,680.53333383)(82.71436287,680.45334076)
\curveto(82.74435578,680.40333396)(82.78435574,680.363334)(82.83436287,680.33334076)
\curveto(82.91435561,680.22333414)(82.98935553,680.10833425)(83.05936287,679.98834076)
\curveto(83.12935539,679.87833448)(83.20435532,679.7633346)(83.28436287,679.64334076)
\curveto(83.33435519,679.55333481)(83.37435515,679.4583349)(83.40436287,679.35834076)
\curveto(83.44435508,679.26833509)(83.48435504,679.16833519)(83.52436287,679.05834076)
\curveto(83.57435495,678.92833543)(83.61435491,678.79333557)(83.64436287,678.65334076)
\curveto(83.67435485,678.51333585)(83.70935481,678.37333599)(83.74936287,678.23334076)
\curveto(83.76935475,678.15333621)(83.77435475,678.0633363)(83.76436287,677.96334076)
\curveto(83.76435476,677.87333649)(83.77435475,677.78833657)(83.79436287,677.70834076)
\lineto(83.79436287,677.54334076)
\moveto(81.54436287,678.42834076)
\curveto(81.61435691,678.52833583)(81.6193569,678.64833571)(81.55936287,678.78834076)
\curveto(81.50935701,678.93833542)(81.46935705,679.04833531)(81.43936287,679.11834076)
\curveto(81.29935722,679.38833497)(81.11435741,679.59333477)(80.88436287,679.73334076)
\curveto(80.65435787,679.88333448)(80.33435819,679.9633344)(79.92436287,679.97334076)
\curveto(79.89435863,679.95333441)(79.85935866,679.94833441)(79.81936287,679.95834076)
\curveto(79.77935874,679.96833439)(79.74435878,679.96833439)(79.71436287,679.95834076)
\curveto(79.66435886,679.93833442)(79.60935891,679.92333444)(79.54936287,679.91334076)
\curveto(79.48935903,679.91333445)(79.43435909,679.90333446)(79.38436287,679.88334076)
\curveto(78.94435958,679.74333462)(78.6193599,679.46833489)(78.40936287,679.05834076)
\curveto(78.38936013,679.01833534)(78.36436016,678.9633354)(78.33436287,678.89334076)
\curveto(78.31436021,678.83333553)(78.29936022,678.76833559)(78.28936287,678.69834076)
\curveto(78.27936024,678.63833572)(78.27936024,678.57833578)(78.28936287,678.51834076)
\curveto(78.30936021,678.4583359)(78.34436018,678.40833595)(78.39436287,678.36834076)
\curveto(78.47436005,678.31833604)(78.58435994,678.29333607)(78.72436287,678.29334076)
\lineto(79.12936287,678.29334076)
\lineto(80.79436287,678.29334076)
\lineto(81.22936287,678.29334076)
\curveto(81.38935713,678.30333606)(81.49435703,678.34833601)(81.54436287,678.42834076)
}
}
{
\newrgbcolor{curcolor}{0 0 0}
\pscustom[linestyle=none,fillstyle=solid,fillcolor=curcolor]
{
\newpath
\moveto(92.81264412,677.78334076)
\curveto(92.83263555,677.72333664)(92.84263554,677.63833672)(92.84264412,677.52834076)
\curveto(92.84263554,677.41833694)(92.83263555,677.33333703)(92.81264412,677.27334076)
\lineto(92.81264412,677.12334076)
\curveto(92.79263559,677.04333732)(92.7826356,676.9633374)(92.78264412,676.88334076)
\curveto(92.79263559,676.80333756)(92.7876356,676.72333764)(92.76764412,676.64334076)
\curveto(92.74763564,676.57333779)(92.73263565,676.50833785)(92.72264412,676.44834076)
\curveto(92.71263567,676.38833797)(92.70263568,676.32333804)(92.69264412,676.25334076)
\curveto(92.65263573,676.14333822)(92.61763577,676.02833833)(92.58764412,675.90834076)
\curveto(92.55763583,675.79833856)(92.51763587,675.69333867)(92.46764412,675.59334076)
\curveto(92.25763613,675.11333925)(91.9826364,674.72333964)(91.64264412,674.42334076)
\curveto(91.30263708,674.12334024)(90.89263749,673.87334049)(90.41264412,673.67334076)
\curveto(90.29263809,673.62334074)(90.16763822,673.58834077)(90.03764412,673.56834076)
\curveto(89.91763847,673.53834082)(89.79263859,673.50834085)(89.66264412,673.47834076)
\curveto(89.61263877,673.4583409)(89.55763883,673.44834091)(89.49764412,673.44834076)
\curveto(89.43763895,673.44834091)(89.382639,673.44334092)(89.33264412,673.43334076)
\lineto(89.22764412,673.43334076)
\curveto(89.19763919,673.42334094)(89.16763922,673.41834094)(89.13764412,673.41834076)
\curveto(89.0876393,673.40834095)(89.00763938,673.40334096)(88.89764412,673.40334076)
\curveto(88.7876396,673.39334097)(88.70263968,673.39834096)(88.64264412,673.41834076)
\lineto(88.49264412,673.41834076)
\curveto(88.44263994,673.42834093)(88.38764,673.43334093)(88.32764412,673.43334076)
\curveto(88.27764011,673.42334094)(88.22764016,673.42834093)(88.17764412,673.44834076)
\curveto(88.13764025,673.4583409)(88.09764029,673.4633409)(88.05764412,673.46334076)
\curveto(88.02764036,673.4633409)(87.9876404,673.46834089)(87.93764412,673.47834076)
\curveto(87.83764055,673.50834085)(87.73764065,673.53334083)(87.63764412,673.55334076)
\curveto(87.53764085,673.57334079)(87.44264094,673.60334076)(87.35264412,673.64334076)
\curveto(87.23264115,673.68334068)(87.11764127,673.72334064)(87.00764412,673.76334076)
\curveto(86.90764148,673.80334056)(86.80264158,673.85334051)(86.69264412,673.91334076)
\curveto(86.34264204,674.12334024)(86.04264234,674.36833999)(85.79264412,674.64834076)
\curveto(85.54264284,674.92833943)(85.33264305,675.2633391)(85.16264412,675.65334076)
\curveto(85.11264327,675.74333862)(85.07264331,675.83833852)(85.04264412,675.93834076)
\curveto(85.02264336,676.03833832)(84.99764339,676.14333822)(84.96764412,676.25334076)
\curveto(84.94764344,676.30333806)(84.93764345,676.34833801)(84.93764412,676.38834076)
\curveto(84.93764345,676.42833793)(84.92764346,676.47333789)(84.90764412,676.52334076)
\curveto(84.8876435,676.60333776)(84.87764351,676.68333768)(84.87764412,676.76334076)
\curveto(84.87764351,676.85333751)(84.86764352,676.93833742)(84.84764412,677.01834076)
\curveto(84.83764355,677.06833729)(84.83264355,677.11333725)(84.83264412,677.15334076)
\lineto(84.83264412,677.28834076)
\curveto(84.81264357,677.34833701)(84.80264358,677.43333693)(84.80264412,677.54334076)
\curveto(84.81264357,677.65333671)(84.82764356,677.73833662)(84.84764412,677.79834076)
\lineto(84.84764412,677.90334076)
\curveto(84.85764353,677.95333641)(84.85764353,678.00333636)(84.84764412,678.05334076)
\curveto(84.84764354,678.11333625)(84.85764353,678.16833619)(84.87764412,678.21834076)
\curveto(84.8876435,678.26833609)(84.89264349,678.31333605)(84.89264412,678.35334076)
\curveto(84.89264349,678.40333596)(84.90264348,678.45333591)(84.92264412,678.50334076)
\curveto(84.96264342,678.63333573)(84.99764339,678.7583356)(85.02764412,678.87834076)
\curveto(85.05764333,679.00833535)(85.09764329,679.13333523)(85.14764412,679.25334076)
\curveto(85.32764306,679.6633347)(85.54264284,680.00333436)(85.79264412,680.27334076)
\curveto(86.04264234,680.55333381)(86.34764204,680.80833355)(86.70764412,681.03834076)
\curveto(86.80764158,681.08833327)(86.91264147,681.13333323)(87.02264412,681.17334076)
\curveto(87.13264125,681.21333315)(87.24264114,681.2583331)(87.35264412,681.30834076)
\curveto(87.4826409,681.358333)(87.61764077,681.39333297)(87.75764412,681.41334076)
\curveto(87.89764049,681.43333293)(88.04264034,681.4633329)(88.19264412,681.50334076)
\curveto(88.27264011,681.51333285)(88.34764004,681.51833284)(88.41764412,681.51834076)
\curveto(88.4876399,681.51833284)(88.55763983,681.52333284)(88.62764412,681.53334076)
\curveto(89.20763918,681.54333282)(89.70763868,681.48333288)(90.12764412,681.35334076)
\curveto(90.55763783,681.22333314)(90.93763745,681.04333332)(91.26764412,680.81334076)
\curveto(91.37763701,680.73333363)(91.4876369,680.64333372)(91.59764412,680.54334076)
\curveto(91.71763667,680.45333391)(91.81763657,680.35333401)(91.89764412,680.24334076)
\curveto(91.97763641,680.14333422)(92.04763634,680.04333432)(92.10764412,679.94334076)
\curveto(92.17763621,679.84333452)(92.24763614,679.73833462)(92.31764412,679.62834076)
\curveto(92.387636,679.51833484)(92.44263594,679.39833496)(92.48264412,679.26834076)
\curveto(92.52263586,679.14833521)(92.56763582,679.01833534)(92.61764412,678.87834076)
\curveto(92.64763574,678.79833556)(92.67263571,678.71333565)(92.69264412,678.62334076)
\lineto(92.75264412,678.35334076)
\curveto(92.76263562,678.31333605)(92.76763562,678.27333609)(92.76764412,678.23334076)
\curveto(92.76763562,678.19333617)(92.77263561,678.15333621)(92.78264412,678.11334076)
\curveto(92.80263558,678.0633363)(92.80763558,678.00833635)(92.79764412,677.94834076)
\curveto(92.7876356,677.88833647)(92.79263559,677.83333653)(92.81264412,677.78334076)
\moveto(90.71264412,677.24334076)
\curveto(90.72263766,677.29333707)(90.72763766,677.363337)(90.72764412,677.45334076)
\curveto(90.72763766,677.55333681)(90.72263766,677.62833673)(90.71264412,677.67834076)
\lineto(90.71264412,677.79834076)
\curveto(90.69263769,677.84833651)(90.6826377,677.90333646)(90.68264412,677.96334076)
\curveto(90.6826377,678.02333634)(90.67763771,678.07833628)(90.66764412,678.12834076)
\curveto(90.66763772,678.16833619)(90.66263772,678.19833616)(90.65264412,678.21834076)
\lineto(90.59264412,678.45834076)
\curveto(90.5826378,678.54833581)(90.56263782,678.63333573)(90.53264412,678.71334076)
\curveto(90.42263796,678.97333539)(90.29263809,679.19333517)(90.14264412,679.37334076)
\curveto(89.99263839,679.5633348)(89.79263859,679.71333465)(89.54264412,679.82334076)
\curveto(89.4826389,679.84333452)(89.42263896,679.8583345)(89.36264412,679.86834076)
\curveto(89.30263908,679.88833447)(89.23763915,679.90833445)(89.16764412,679.92834076)
\curveto(89.0876393,679.94833441)(89.00263938,679.95333441)(88.91264412,679.94334076)
\lineto(88.64264412,679.94334076)
\curveto(88.61263977,679.92333444)(88.57763981,679.91333445)(88.53764412,679.91334076)
\curveto(88.49763989,679.92333444)(88.46263992,679.92333444)(88.43264412,679.91334076)
\lineto(88.22264412,679.85334076)
\curveto(88.16264022,679.84333452)(88.10764028,679.82333454)(88.05764412,679.79334076)
\curveto(87.80764058,679.68333468)(87.60264078,679.52333484)(87.44264412,679.31334076)
\curveto(87.29264109,679.11333525)(87.17264121,678.87833548)(87.08264412,678.60834076)
\curveto(87.05264133,678.50833585)(87.02764136,678.40333596)(87.00764412,678.29334076)
\curveto(86.99764139,678.18333618)(86.9826414,678.07333629)(86.96264412,677.96334076)
\curveto(86.95264143,677.91333645)(86.94764144,677.8633365)(86.94764412,677.81334076)
\lineto(86.94764412,677.66334076)
\curveto(86.92764146,677.59333677)(86.91764147,677.48833687)(86.91764412,677.34834076)
\curveto(86.92764146,677.20833715)(86.94264144,677.10333726)(86.96264412,677.03334076)
\lineto(86.96264412,676.89834076)
\curveto(86.9826414,676.81833754)(86.99764139,676.73833762)(87.00764412,676.65834076)
\curveto(87.01764137,676.58833777)(87.03264135,676.51333785)(87.05264412,676.43334076)
\curveto(87.15264123,676.13333823)(87.25764113,675.88833847)(87.36764412,675.69834076)
\curveto(87.4876409,675.51833884)(87.67264071,675.35333901)(87.92264412,675.20334076)
\curveto(87.99264039,675.15333921)(88.06764032,675.11333925)(88.14764412,675.08334076)
\curveto(88.23764015,675.05333931)(88.32764006,675.02833933)(88.41764412,675.00834076)
\curveto(88.45763993,674.99833936)(88.49263989,674.99333937)(88.52264412,674.99334076)
\curveto(88.55263983,675.00333936)(88.5876398,675.00333936)(88.62764412,674.99334076)
\lineto(88.74764412,674.96334076)
\curveto(88.79763959,674.9633394)(88.84263954,674.96833939)(88.88264412,674.97834076)
\lineto(89.00264412,674.97834076)
\curveto(89.0826393,674.99833936)(89.16263922,675.01333935)(89.24264412,675.02334076)
\curveto(89.32263906,675.03333933)(89.39763899,675.05333931)(89.46764412,675.08334076)
\curveto(89.72763866,675.18333918)(89.93763845,675.31833904)(90.09764412,675.48834076)
\curveto(90.25763813,675.6583387)(90.39263799,675.86833849)(90.50264412,676.11834076)
\curveto(90.54263784,676.21833814)(90.57263781,676.31833804)(90.59264412,676.41834076)
\curveto(90.61263777,676.51833784)(90.63763775,676.62333774)(90.66764412,676.73334076)
\curveto(90.67763771,676.77333759)(90.6826377,676.80833755)(90.68264412,676.83834076)
\curveto(90.6826377,676.87833748)(90.6876377,676.91833744)(90.69764412,676.95834076)
\lineto(90.69764412,677.09334076)
\curveto(90.69763769,677.14333722)(90.70263768,677.19333717)(90.71264412,677.24334076)
}
}
{
\newrgbcolor{curcolor}{0 0 0}
\pscustom[linestyle=none,fillstyle=solid,fillcolor=curcolor]
{
}
}
{
\newrgbcolor{curcolor}{0 0 0}
\pscustom[linestyle=none,fillstyle=solid,fillcolor=curcolor]
{
\newpath
\moveto(105.73772225,677.54334076)
\curveto(105.75771408,677.4633369)(105.75771408,677.37333699)(105.73772225,677.27334076)
\curveto(105.71771412,677.17333719)(105.68271416,677.10833725)(105.63272225,677.07834076)
\curveto(105.58271426,677.03833732)(105.50771433,677.00833735)(105.40772225,676.98834076)
\curveto(105.31771452,676.97833738)(105.21271463,676.96833739)(105.09272225,676.95834076)
\lineto(104.74772225,676.95834076)
\curveto(104.6377152,676.96833739)(104.5377153,676.97333739)(104.44772225,676.97334076)
\lineto(100.78772225,676.97334076)
\lineto(100.57772225,676.97334076)
\curveto(100.51771932,676.97333739)(100.46271938,676.9633374)(100.41272225,676.94334076)
\curveto(100.33271951,676.90333746)(100.28271956,676.8633375)(100.26272225,676.82334076)
\curveto(100.2427196,676.80333756)(100.22271962,676.7633376)(100.20272225,676.70334076)
\curveto(100.18271966,676.65333771)(100.17771966,676.60333776)(100.18772225,676.55334076)
\curveto(100.20771963,676.49333787)(100.21771962,676.43333793)(100.21772225,676.37334076)
\curveto(100.22771961,676.32333804)(100.2427196,676.26833809)(100.26272225,676.20834076)
\curveto(100.3427195,675.96833839)(100.4377194,675.76833859)(100.54772225,675.60834076)
\curveto(100.66771917,675.4583389)(100.82771901,675.32333904)(101.02772225,675.20334076)
\curveto(101.10771873,675.15333921)(101.18771865,675.11833924)(101.26772225,675.09834076)
\curveto(101.35771848,675.08833927)(101.44771839,675.06833929)(101.53772225,675.03834076)
\curveto(101.61771822,675.01833934)(101.72771811,675.00333936)(101.86772225,674.99334076)
\curveto(102.00771783,674.98333938)(102.12771771,674.98833937)(102.22772225,675.00834076)
\lineto(102.36272225,675.00834076)
\curveto(102.46271738,675.02833933)(102.55271729,675.04833931)(102.63272225,675.06834076)
\curveto(102.72271712,675.09833926)(102.80771703,675.12833923)(102.88772225,675.15834076)
\curveto(102.98771685,675.20833915)(103.09771674,675.27333909)(103.21772225,675.35334076)
\curveto(103.34771649,675.43333893)(103.4427164,675.51333885)(103.50272225,675.59334076)
\curveto(103.55271629,675.6633387)(103.60271624,675.72833863)(103.65272225,675.78834076)
\curveto(103.71271613,675.8583385)(103.78271606,675.90833845)(103.86272225,675.93834076)
\curveto(103.96271588,675.98833837)(104.08771575,676.00833835)(104.23772225,675.99834076)
\lineto(104.67272225,675.99834076)
\lineto(104.85272225,675.99834076)
\curveto(104.92271492,676.00833835)(104.98271486,676.00333836)(105.03272225,675.98334076)
\lineto(105.18272225,675.98334076)
\curveto(105.28271456,675.9633384)(105.35271449,675.93833842)(105.39272225,675.90834076)
\curveto(105.43271441,675.88833847)(105.45271439,675.84333852)(105.45272225,675.77334076)
\curveto(105.46271438,675.70333866)(105.45771438,675.64333872)(105.43772225,675.59334076)
\curveto(105.38771445,675.45333891)(105.33271451,675.32833903)(105.27272225,675.21834076)
\curveto(105.21271463,675.10833925)(105.1427147,674.99833936)(105.06272225,674.88834076)
\curveto(104.842715,674.5583398)(104.59271525,674.29334007)(104.31272225,674.09334076)
\curveto(104.03271581,673.89334047)(103.68271616,673.72334064)(103.26272225,673.58334076)
\curveto(103.15271669,673.54334082)(103.0427168,673.51834084)(102.93272225,673.50834076)
\curveto(102.82271702,673.49834086)(102.70771713,673.47834088)(102.58772225,673.44834076)
\curveto(102.54771729,673.43834092)(102.50271734,673.43834092)(102.45272225,673.44834076)
\curveto(102.41271743,673.44834091)(102.37271747,673.44334092)(102.33272225,673.43334076)
\lineto(102.16772225,673.43334076)
\curveto(102.11771772,673.41334095)(102.05771778,673.40834095)(101.98772225,673.41834076)
\curveto(101.92771791,673.41834094)(101.87271797,673.42334094)(101.82272225,673.43334076)
\curveto(101.7427181,673.44334092)(101.67271817,673.44334092)(101.61272225,673.43334076)
\curveto(101.55271829,673.42334094)(101.48771835,673.42834093)(101.41772225,673.44834076)
\curveto(101.36771847,673.46834089)(101.31271853,673.47834088)(101.25272225,673.47834076)
\curveto(101.19271865,673.47834088)(101.1377187,673.48834087)(101.08772225,673.50834076)
\curveto(100.97771886,673.52834083)(100.86771897,673.55334081)(100.75772225,673.58334076)
\curveto(100.64771919,673.60334076)(100.54771929,673.63834072)(100.45772225,673.68834076)
\curveto(100.34771949,673.72834063)(100.2427196,673.7633406)(100.14272225,673.79334076)
\curveto(100.05271979,673.83334053)(99.96771987,673.87834048)(99.88772225,673.92834076)
\curveto(99.56772027,674.12834023)(99.28272056,674.35834)(99.03272225,674.61834076)
\curveto(98.78272106,674.88833947)(98.57772126,675.19833916)(98.41772225,675.54834076)
\curveto(98.36772147,675.6583387)(98.32772151,675.76833859)(98.29772225,675.87834076)
\curveto(98.26772157,675.99833836)(98.22772161,676.11833824)(98.17772225,676.23834076)
\curveto(98.16772167,676.27833808)(98.16272168,676.31333805)(98.16272225,676.34334076)
\curveto(98.16272168,676.38333798)(98.15772168,676.42333794)(98.14772225,676.46334076)
\curveto(98.10772173,676.58333778)(98.08272176,676.71333765)(98.07272225,676.85334076)
\lineto(98.04272225,677.27334076)
\curveto(98.0427218,677.32333704)(98.0377218,677.37833698)(98.02772225,677.43834076)
\curveto(98.02772181,677.49833686)(98.03272181,677.55333681)(98.04272225,677.60334076)
\lineto(98.04272225,677.78334076)
\lineto(98.08772225,678.14334076)
\curveto(98.12772171,678.31333605)(98.16272168,678.47833588)(98.19272225,678.63834076)
\curveto(98.22272162,678.79833556)(98.26772157,678.94833541)(98.32772225,679.08834076)
\curveto(98.75772108,680.12833423)(99.48772035,680.8633335)(100.51772225,681.29334076)
\curveto(100.65771918,681.35333301)(100.79771904,681.39333297)(100.93772225,681.41334076)
\curveto(101.08771875,681.44333292)(101.2427186,681.47833288)(101.40272225,681.51834076)
\curveto(101.48271836,681.52833283)(101.55771828,681.53333283)(101.62772225,681.53334076)
\curveto(101.69771814,681.53333283)(101.77271807,681.53833282)(101.85272225,681.54834076)
\curveto(102.36271748,681.5583328)(102.79771704,681.49833286)(103.15772225,681.36834076)
\curveto(103.52771631,681.24833311)(103.85771598,681.08833327)(104.14772225,680.88834076)
\curveto(104.2377156,680.82833353)(104.32771551,680.7583336)(104.41772225,680.67834076)
\curveto(104.50771533,680.60833375)(104.58771525,680.53333383)(104.65772225,680.45334076)
\curveto(104.68771515,680.40333396)(104.72771511,680.363334)(104.77772225,680.33334076)
\curveto(104.85771498,680.22333414)(104.93271491,680.10833425)(105.00272225,679.98834076)
\curveto(105.07271477,679.87833448)(105.14771469,679.7633346)(105.22772225,679.64334076)
\curveto(105.27771456,679.55333481)(105.31771452,679.4583349)(105.34772225,679.35834076)
\curveto(105.38771445,679.26833509)(105.42771441,679.16833519)(105.46772225,679.05834076)
\curveto(105.51771432,678.92833543)(105.55771428,678.79333557)(105.58772225,678.65334076)
\curveto(105.61771422,678.51333585)(105.65271419,678.37333599)(105.69272225,678.23334076)
\curveto(105.71271413,678.15333621)(105.71771412,678.0633363)(105.70772225,677.96334076)
\curveto(105.70771413,677.87333649)(105.71771412,677.78833657)(105.73772225,677.70834076)
\lineto(105.73772225,677.54334076)
\moveto(103.48772225,678.42834076)
\curveto(103.55771628,678.52833583)(103.56271628,678.64833571)(103.50272225,678.78834076)
\curveto(103.45271639,678.93833542)(103.41271643,679.04833531)(103.38272225,679.11834076)
\curveto(103.2427166,679.38833497)(103.05771678,679.59333477)(102.82772225,679.73334076)
\curveto(102.59771724,679.88333448)(102.27771756,679.9633344)(101.86772225,679.97334076)
\curveto(101.837718,679.95333441)(101.80271804,679.94833441)(101.76272225,679.95834076)
\curveto(101.72271812,679.96833439)(101.68771815,679.96833439)(101.65772225,679.95834076)
\curveto(101.60771823,679.93833442)(101.55271829,679.92333444)(101.49272225,679.91334076)
\curveto(101.43271841,679.91333445)(101.37771846,679.90333446)(101.32772225,679.88334076)
\curveto(100.88771895,679.74333462)(100.56271928,679.46833489)(100.35272225,679.05834076)
\curveto(100.33271951,679.01833534)(100.30771953,678.9633354)(100.27772225,678.89334076)
\curveto(100.25771958,678.83333553)(100.2427196,678.76833559)(100.23272225,678.69834076)
\curveto(100.22271962,678.63833572)(100.22271962,678.57833578)(100.23272225,678.51834076)
\curveto(100.25271959,678.4583359)(100.28771955,678.40833595)(100.33772225,678.36834076)
\curveto(100.41771942,678.31833604)(100.52771931,678.29333607)(100.66772225,678.29334076)
\lineto(101.07272225,678.29334076)
\lineto(102.73772225,678.29334076)
\lineto(103.17272225,678.29334076)
\curveto(103.33271651,678.30333606)(103.4377164,678.34833601)(103.48772225,678.42834076)
}
}
{
\newrgbcolor{curcolor}{0 0 0}
\pscustom[linestyle=none,fillstyle=solid,fillcolor=curcolor]
{
\newpath
\moveto(107.4960035,684.29334076)
\lineto(108.5910035,684.29334076)
\curveto(108.69100101,684.29333007)(108.78600092,684.28833007)(108.8760035,684.27834076)
\curveto(108.96600074,684.26833009)(109.03600067,684.23833012)(109.0860035,684.18834076)
\curveto(109.14600056,684.11833024)(109.17600053,684.02333034)(109.1760035,683.90334076)
\curveto(109.18600052,683.79333057)(109.19100051,683.67833068)(109.1910035,683.55834076)
\lineto(109.1910035,682.22334076)
\lineto(109.1910035,676.83834076)
\lineto(109.1910035,674.54334076)
\lineto(109.1910035,674.12334076)
\curveto(109.2010005,673.97334039)(109.18100052,673.8583405)(109.1310035,673.77834076)
\curveto(109.08100062,673.69834066)(108.99100071,673.64334072)(108.8610035,673.61334076)
\curveto(108.8010009,673.59334077)(108.73100097,673.58834077)(108.6510035,673.59834076)
\curveto(108.58100112,673.60834075)(108.51100119,673.61334075)(108.4410035,673.61334076)
\lineto(107.7210035,673.61334076)
\curveto(107.61100209,673.61334075)(107.51100219,673.61834074)(107.4210035,673.62834076)
\curveto(107.33100237,673.63834072)(107.25600245,673.66834069)(107.1960035,673.71834076)
\curveto(107.13600257,673.76834059)(107.1010026,673.84334052)(107.0910035,673.94334076)
\lineto(107.0910035,674.27334076)
\lineto(107.0910035,675.60834076)
\lineto(107.0910035,681.23334076)
\lineto(107.0910035,683.27334076)
\curveto(107.09100261,683.40333096)(107.08600262,683.5583308)(107.0760035,683.73834076)
\curveto(107.07600263,683.91833044)(107.1010026,684.04833031)(107.1510035,684.12834076)
\curveto(107.17100253,684.16833019)(107.19600251,684.19833016)(107.2260035,684.21834076)
\lineto(107.3460035,684.27834076)
\curveto(107.36600234,684.27833008)(107.39100231,684.27833008)(107.4210035,684.27834076)
\curveto(107.45100225,684.28833007)(107.47600223,684.29333007)(107.4960035,684.29334076)
}
}
{
\newrgbcolor{curcolor}{0 0 0}
\pscustom[linestyle=none,fillstyle=solid,fillcolor=curcolor]
{
\newpath
\moveto(118.218191,677.54334076)
\curveto(118.23818283,677.4633369)(118.23818283,677.37333699)(118.218191,677.27334076)
\curveto(118.19818287,677.17333719)(118.16318291,677.10833725)(118.113191,677.07834076)
\curveto(118.06318301,677.03833732)(117.98818308,677.00833735)(117.888191,676.98834076)
\curveto(117.79818327,676.97833738)(117.69318338,676.96833739)(117.573191,676.95834076)
\lineto(117.228191,676.95834076)
\curveto(117.11818395,676.96833739)(117.01818405,676.97333739)(116.928191,676.97334076)
\lineto(113.268191,676.97334076)
\lineto(113.058191,676.97334076)
\curveto(112.99818807,676.97333739)(112.94318813,676.9633374)(112.893191,676.94334076)
\curveto(112.81318826,676.90333746)(112.76318831,676.8633375)(112.743191,676.82334076)
\curveto(112.72318835,676.80333756)(112.70318837,676.7633376)(112.683191,676.70334076)
\curveto(112.66318841,676.65333771)(112.65818841,676.60333776)(112.668191,676.55334076)
\curveto(112.68818838,676.49333787)(112.69818837,676.43333793)(112.698191,676.37334076)
\curveto(112.70818836,676.32333804)(112.72318835,676.26833809)(112.743191,676.20834076)
\curveto(112.82318825,675.96833839)(112.91818815,675.76833859)(113.028191,675.60834076)
\curveto(113.14818792,675.4583389)(113.30818776,675.32333904)(113.508191,675.20334076)
\curveto(113.58818748,675.15333921)(113.6681874,675.11833924)(113.748191,675.09834076)
\curveto(113.83818723,675.08833927)(113.92818714,675.06833929)(114.018191,675.03834076)
\curveto(114.09818697,675.01833934)(114.20818686,675.00333936)(114.348191,674.99334076)
\curveto(114.48818658,674.98333938)(114.60818646,674.98833937)(114.708191,675.00834076)
\lineto(114.843191,675.00834076)
\curveto(114.94318613,675.02833933)(115.03318604,675.04833931)(115.113191,675.06834076)
\curveto(115.20318587,675.09833926)(115.28818578,675.12833923)(115.368191,675.15834076)
\curveto(115.4681856,675.20833915)(115.57818549,675.27333909)(115.698191,675.35334076)
\curveto(115.82818524,675.43333893)(115.92318515,675.51333885)(115.983191,675.59334076)
\curveto(116.03318504,675.6633387)(116.08318499,675.72833863)(116.133191,675.78834076)
\curveto(116.19318488,675.8583385)(116.26318481,675.90833845)(116.343191,675.93834076)
\curveto(116.44318463,675.98833837)(116.5681845,676.00833835)(116.718191,675.99834076)
\lineto(117.153191,675.99834076)
\lineto(117.333191,675.99834076)
\curveto(117.40318367,676.00833835)(117.46318361,676.00333836)(117.513191,675.98334076)
\lineto(117.663191,675.98334076)
\curveto(117.76318331,675.9633384)(117.83318324,675.93833842)(117.873191,675.90834076)
\curveto(117.91318316,675.88833847)(117.93318314,675.84333852)(117.933191,675.77334076)
\curveto(117.94318313,675.70333866)(117.93818313,675.64333872)(117.918191,675.59334076)
\curveto(117.8681832,675.45333891)(117.81318326,675.32833903)(117.753191,675.21834076)
\curveto(117.69318338,675.10833925)(117.62318345,674.99833936)(117.543191,674.88834076)
\curveto(117.32318375,674.5583398)(117.073184,674.29334007)(116.793191,674.09334076)
\curveto(116.51318456,673.89334047)(116.16318491,673.72334064)(115.743191,673.58334076)
\curveto(115.63318544,673.54334082)(115.52318555,673.51834084)(115.413191,673.50834076)
\curveto(115.30318577,673.49834086)(115.18818588,673.47834088)(115.068191,673.44834076)
\curveto(115.02818604,673.43834092)(114.98318609,673.43834092)(114.933191,673.44834076)
\curveto(114.89318618,673.44834091)(114.85318622,673.44334092)(114.813191,673.43334076)
\lineto(114.648191,673.43334076)
\curveto(114.59818647,673.41334095)(114.53818653,673.40834095)(114.468191,673.41834076)
\curveto(114.40818666,673.41834094)(114.35318672,673.42334094)(114.303191,673.43334076)
\curveto(114.22318685,673.44334092)(114.15318692,673.44334092)(114.093191,673.43334076)
\curveto(114.03318704,673.42334094)(113.9681871,673.42834093)(113.898191,673.44834076)
\curveto(113.84818722,673.46834089)(113.79318728,673.47834088)(113.733191,673.47834076)
\curveto(113.6731874,673.47834088)(113.61818745,673.48834087)(113.568191,673.50834076)
\curveto(113.45818761,673.52834083)(113.34818772,673.55334081)(113.238191,673.58334076)
\curveto(113.12818794,673.60334076)(113.02818804,673.63834072)(112.938191,673.68834076)
\curveto(112.82818824,673.72834063)(112.72318835,673.7633406)(112.623191,673.79334076)
\curveto(112.53318854,673.83334053)(112.44818862,673.87834048)(112.368191,673.92834076)
\curveto(112.04818902,674.12834023)(111.76318931,674.35834)(111.513191,674.61834076)
\curveto(111.26318981,674.88833947)(111.05819001,675.19833916)(110.898191,675.54834076)
\curveto(110.84819022,675.6583387)(110.80819026,675.76833859)(110.778191,675.87834076)
\curveto(110.74819032,675.99833836)(110.70819036,676.11833824)(110.658191,676.23834076)
\curveto(110.64819042,676.27833808)(110.64319043,676.31333805)(110.643191,676.34334076)
\curveto(110.64319043,676.38333798)(110.63819043,676.42333794)(110.628191,676.46334076)
\curveto(110.58819048,676.58333778)(110.56319051,676.71333765)(110.553191,676.85334076)
\lineto(110.523191,677.27334076)
\curveto(110.52319055,677.32333704)(110.51819055,677.37833698)(110.508191,677.43834076)
\curveto(110.50819056,677.49833686)(110.51319056,677.55333681)(110.523191,677.60334076)
\lineto(110.523191,677.78334076)
\lineto(110.568191,678.14334076)
\curveto(110.60819046,678.31333605)(110.64319043,678.47833588)(110.673191,678.63834076)
\curveto(110.70319037,678.79833556)(110.74819032,678.94833541)(110.808191,679.08834076)
\curveto(111.23818983,680.12833423)(111.9681891,680.8633335)(112.998191,681.29334076)
\curveto(113.13818793,681.35333301)(113.27818779,681.39333297)(113.418191,681.41334076)
\curveto(113.5681875,681.44333292)(113.72318735,681.47833288)(113.883191,681.51834076)
\curveto(113.96318711,681.52833283)(114.03818703,681.53333283)(114.108191,681.53334076)
\curveto(114.17818689,681.53333283)(114.25318682,681.53833282)(114.333191,681.54834076)
\curveto(114.84318623,681.5583328)(115.27818579,681.49833286)(115.638191,681.36834076)
\curveto(116.00818506,681.24833311)(116.33818473,681.08833327)(116.628191,680.88834076)
\curveto(116.71818435,680.82833353)(116.80818426,680.7583336)(116.898191,680.67834076)
\curveto(116.98818408,680.60833375)(117.068184,680.53333383)(117.138191,680.45334076)
\curveto(117.1681839,680.40333396)(117.20818386,680.363334)(117.258191,680.33334076)
\curveto(117.33818373,680.22333414)(117.41318366,680.10833425)(117.483191,679.98834076)
\curveto(117.55318352,679.87833448)(117.62818344,679.7633346)(117.708191,679.64334076)
\curveto(117.75818331,679.55333481)(117.79818327,679.4583349)(117.828191,679.35834076)
\curveto(117.8681832,679.26833509)(117.90818316,679.16833519)(117.948191,679.05834076)
\curveto(117.99818307,678.92833543)(118.03818303,678.79333557)(118.068191,678.65334076)
\curveto(118.09818297,678.51333585)(118.13318294,678.37333599)(118.173191,678.23334076)
\curveto(118.19318288,678.15333621)(118.19818287,678.0633363)(118.188191,677.96334076)
\curveto(118.18818288,677.87333649)(118.19818287,677.78833657)(118.218191,677.70834076)
\lineto(118.218191,677.54334076)
\moveto(115.968191,678.42834076)
\curveto(116.03818503,678.52833583)(116.04318503,678.64833571)(115.983191,678.78834076)
\curveto(115.93318514,678.93833542)(115.89318518,679.04833531)(115.863191,679.11834076)
\curveto(115.72318535,679.38833497)(115.53818553,679.59333477)(115.308191,679.73334076)
\curveto(115.07818599,679.88333448)(114.75818631,679.9633344)(114.348191,679.97334076)
\curveto(114.31818675,679.95333441)(114.28318679,679.94833441)(114.243191,679.95834076)
\curveto(114.20318687,679.96833439)(114.1681869,679.96833439)(114.138191,679.95834076)
\curveto(114.08818698,679.93833442)(114.03318704,679.92333444)(113.973191,679.91334076)
\curveto(113.91318716,679.91333445)(113.85818721,679.90333446)(113.808191,679.88334076)
\curveto(113.3681877,679.74333462)(113.04318803,679.46833489)(112.833191,679.05834076)
\curveto(112.81318826,679.01833534)(112.78818828,678.9633354)(112.758191,678.89334076)
\curveto(112.73818833,678.83333553)(112.72318835,678.76833559)(112.713191,678.69834076)
\curveto(112.70318837,678.63833572)(112.70318837,678.57833578)(112.713191,678.51834076)
\curveto(112.73318834,678.4583359)(112.7681883,678.40833595)(112.818191,678.36834076)
\curveto(112.89818817,678.31833604)(113.00818806,678.29333607)(113.148191,678.29334076)
\lineto(113.553191,678.29334076)
\lineto(115.218191,678.29334076)
\lineto(115.653191,678.29334076)
\curveto(115.81318526,678.30333606)(115.91818515,678.34833601)(115.968191,678.42834076)
}
}
{
\newrgbcolor{curcolor}{0 0 0}
\pscustom[linestyle=none,fillstyle=solid,fillcolor=curcolor]
{
\newpath
\moveto(123.03647225,681.54834076)
\curveto(123.84646709,681.56833279)(124.52146641,681.44833291)(125.06147225,681.18834076)
\curveto(125.61146532,680.92833343)(126.04646489,680.5583338)(126.36647225,680.07834076)
\curveto(126.52646441,679.83833452)(126.64646429,679.5633348)(126.72647225,679.25334076)
\curveto(126.74646419,679.20333516)(126.76146417,679.13833522)(126.77147225,679.05834076)
\curveto(126.79146414,678.97833538)(126.79146414,678.90833545)(126.77147225,678.84834076)
\curveto(126.7314642,678.73833562)(126.66146427,678.67333569)(126.56147225,678.65334076)
\curveto(126.46146447,678.64333572)(126.34146459,678.63833572)(126.20147225,678.63834076)
\lineto(125.42147225,678.63834076)
\lineto(125.13647225,678.63834076)
\curveto(125.04646589,678.63833572)(124.97146596,678.6583357)(124.91147225,678.69834076)
\curveto(124.8314661,678.73833562)(124.77646616,678.79833556)(124.74647225,678.87834076)
\curveto(124.71646622,678.96833539)(124.67646626,679.0583353)(124.62647225,679.14834076)
\curveto(124.56646637,679.2583351)(124.50146643,679.358335)(124.43147225,679.44834076)
\curveto(124.36146657,679.53833482)(124.28146665,679.61833474)(124.19147225,679.68834076)
\curveto(124.05146688,679.77833458)(123.89646704,679.84833451)(123.72647225,679.89834076)
\curveto(123.66646727,679.91833444)(123.60646733,679.92833443)(123.54647225,679.92834076)
\curveto(123.48646745,679.92833443)(123.4314675,679.93833442)(123.38147225,679.95834076)
\lineto(123.23147225,679.95834076)
\curveto(123.0314679,679.9583344)(122.87146806,679.93833442)(122.75147225,679.89834076)
\curveto(122.46146847,679.80833455)(122.22646871,679.66833469)(122.04647225,679.47834076)
\curveto(121.86646907,679.29833506)(121.72146921,679.07833528)(121.61147225,678.81834076)
\curveto(121.56146937,678.70833565)(121.52146941,678.58833577)(121.49147225,678.45834076)
\curveto(121.47146946,678.33833602)(121.44646949,678.20833615)(121.41647225,678.06834076)
\curveto(121.40646953,678.02833633)(121.40146953,677.98833637)(121.40147225,677.94834076)
\curveto(121.40146953,677.90833645)(121.39646954,677.86833649)(121.38647225,677.82834076)
\curveto(121.36646957,677.72833663)(121.35646958,677.58833677)(121.35647225,677.40834076)
\curveto(121.36646957,677.22833713)(121.38146955,677.08833727)(121.40147225,676.98834076)
\curveto(121.40146953,676.90833745)(121.40646953,676.85333751)(121.41647225,676.82334076)
\curveto(121.4364695,676.75333761)(121.44646949,676.68333768)(121.44647225,676.61334076)
\curveto(121.45646948,676.54333782)(121.47146946,676.47333789)(121.49147225,676.40334076)
\curveto(121.57146936,676.17333819)(121.66646927,675.9633384)(121.77647225,675.77334076)
\curveto(121.88646905,675.58333878)(122.02646891,675.42333894)(122.19647225,675.29334076)
\curveto(122.2364687,675.2633391)(122.29646864,675.22833913)(122.37647225,675.18834076)
\curveto(122.48646845,675.11833924)(122.59646834,675.07333929)(122.70647225,675.05334076)
\curveto(122.82646811,675.03333933)(122.97146796,675.01333935)(123.14147225,674.99334076)
\lineto(123.23147225,674.99334076)
\curveto(123.27146766,674.99333937)(123.30146763,674.99833936)(123.32147225,675.00834076)
\lineto(123.45647225,675.00834076)
\curveto(123.52646741,675.02833933)(123.59146734,675.04333932)(123.65147225,675.05334076)
\curveto(123.72146721,675.07333929)(123.78646715,675.09333927)(123.84647225,675.11334076)
\curveto(124.14646679,675.24333912)(124.37646656,675.43333893)(124.53647225,675.68334076)
\curveto(124.57646636,675.73333863)(124.61146632,675.78833857)(124.64147225,675.84834076)
\curveto(124.67146626,675.91833844)(124.69646624,675.97833838)(124.71647225,676.02834076)
\curveto(124.75646618,676.13833822)(124.79146614,676.23333813)(124.82147225,676.31334076)
\curveto(124.85146608,676.40333796)(124.92146601,676.47333789)(125.03147225,676.52334076)
\curveto(125.12146581,676.5633378)(125.26646567,676.57833778)(125.46647225,676.56834076)
\lineto(125.96147225,676.56834076)
\lineto(126.17147225,676.56834076)
\curveto(126.25146468,676.57833778)(126.31646462,676.57333779)(126.36647225,676.55334076)
\lineto(126.48647225,676.55334076)
\lineto(126.60647225,676.52334076)
\curveto(126.64646429,676.52333784)(126.67646426,676.51333785)(126.69647225,676.49334076)
\curveto(126.74646419,676.45333791)(126.77646416,676.39333797)(126.78647225,676.31334076)
\curveto(126.80646413,676.24333812)(126.80646413,676.16833819)(126.78647225,676.08834076)
\curveto(126.69646424,675.7583386)(126.58646435,675.4633389)(126.45647225,675.20334076)
\curveto(126.04646489,674.43333993)(125.39146554,673.89834046)(124.49147225,673.59834076)
\curveto(124.39146654,673.56834079)(124.28646665,673.54834081)(124.17647225,673.53834076)
\curveto(124.06646687,673.51834084)(123.95646698,673.49334087)(123.84647225,673.46334076)
\curveto(123.78646715,673.45334091)(123.72646721,673.44834091)(123.66647225,673.44834076)
\curveto(123.60646733,673.44834091)(123.54646739,673.44334092)(123.48647225,673.43334076)
\lineto(123.32147225,673.43334076)
\curveto(123.27146766,673.41334095)(123.19646774,673.40834095)(123.09647225,673.41834076)
\curveto(122.99646794,673.41834094)(122.92146801,673.42334094)(122.87147225,673.43334076)
\curveto(122.79146814,673.45334091)(122.71646822,673.4633409)(122.64647225,673.46334076)
\curveto(122.58646835,673.45334091)(122.52146841,673.4583409)(122.45147225,673.47834076)
\lineto(122.30147225,673.50834076)
\curveto(122.25146868,673.50834085)(122.20146873,673.51334085)(122.15147225,673.52334076)
\curveto(122.04146889,673.55334081)(121.936469,673.58334078)(121.83647225,673.61334076)
\curveto(121.7364692,673.64334072)(121.64146929,673.67834068)(121.55147225,673.71834076)
\curveto(121.08146985,673.91834044)(120.68647025,674.17334019)(120.36647225,674.48334076)
\curveto(120.04647089,674.80333956)(119.78647115,675.19833916)(119.58647225,675.66834076)
\curveto(119.5364714,675.7583386)(119.49647144,675.85333851)(119.46647225,675.95334076)
\lineto(119.37647225,676.28334076)
\curveto(119.36647157,676.32333804)(119.36147157,676.358338)(119.36147225,676.38834076)
\curveto(119.36147157,676.42833793)(119.35147158,676.47333789)(119.33147225,676.52334076)
\curveto(119.31147162,676.59333777)(119.30147163,676.6633377)(119.30147225,676.73334076)
\curveto(119.30147163,676.81333755)(119.29147164,676.88833747)(119.27147225,676.95834076)
\lineto(119.27147225,677.21334076)
\curveto(119.25147168,677.2633371)(119.24147169,677.31833704)(119.24147225,677.37834076)
\curveto(119.24147169,677.44833691)(119.25147168,677.50833685)(119.27147225,677.55834076)
\curveto(119.28147165,677.60833675)(119.28147165,677.65333671)(119.27147225,677.69334076)
\curveto(119.26147167,677.73333663)(119.26147167,677.77333659)(119.27147225,677.81334076)
\curveto(119.29147164,677.88333648)(119.29647164,677.94833641)(119.28647225,678.00834076)
\curveto(119.28647165,678.06833629)(119.29647164,678.12833623)(119.31647225,678.18834076)
\curveto(119.36647157,678.36833599)(119.40647153,678.53833582)(119.43647225,678.69834076)
\curveto(119.46647147,678.86833549)(119.51147142,679.03333533)(119.57147225,679.19334076)
\curveto(119.79147114,679.70333466)(120.06647087,680.12833423)(120.39647225,680.46834076)
\curveto(120.7364702,680.80833355)(121.16646977,681.08333328)(121.68647225,681.29334076)
\curveto(121.82646911,681.35333301)(121.97146896,681.39333297)(122.12147225,681.41334076)
\curveto(122.27146866,681.44333292)(122.42646851,681.47833288)(122.58647225,681.51834076)
\curveto(122.66646827,681.52833283)(122.74146819,681.53333283)(122.81147225,681.53334076)
\curveto(122.88146805,681.53333283)(122.95646798,681.53833282)(123.03647225,681.54834076)
}
}
{
\newrgbcolor{curcolor}{0 0 0}
\pscustom[linestyle=none,fillstyle=solid,fillcolor=curcolor]
{
\newpath
\moveto(129.1297535,683.64834076)
\lineto(130.1347535,683.64834076)
\curveto(130.28475051,683.64833071)(130.41475038,683.63833072)(130.5247535,683.61834076)
\curveto(130.64475015,683.60833075)(130.72975007,683.54833081)(130.7797535,683.43834076)
\curveto(130.79975,683.38833097)(130.80974999,683.32833103)(130.8097535,683.25834076)
\lineto(130.8097535,683.04834076)
\lineto(130.8097535,682.37334076)
\curveto(130.80974999,682.32333204)(130.80474999,682.2633321)(130.7947535,682.19334076)
\curveto(130.79475,682.13333223)(130.79975,682.07833228)(130.8097535,682.02834076)
\lineto(130.8097535,681.86334076)
\curveto(130.80974999,681.78333258)(130.81474998,681.70833265)(130.8247535,681.63834076)
\curveto(130.83474996,681.57833278)(130.85974994,681.52333284)(130.8997535,681.47334076)
\curveto(130.96974983,681.38333298)(131.0947497,681.33333303)(131.2747535,681.32334076)
\lineto(131.8147535,681.32334076)
\lineto(131.9947535,681.32334076)
\curveto(132.05474874,681.32333304)(132.10974869,681.31333305)(132.1597535,681.29334076)
\curveto(132.26974853,681.24333312)(132.32974847,681.15333321)(132.3397535,681.02334076)
\curveto(132.35974844,680.89333347)(132.36974843,680.74833361)(132.3697535,680.58834076)
\lineto(132.3697535,680.37834076)
\curveto(132.37974842,680.30833405)(132.37474842,680.24833411)(132.3547535,680.19834076)
\curveto(132.30474849,680.03833432)(132.1997486,679.95333441)(132.0397535,679.94334076)
\curveto(131.87974892,679.93333443)(131.6997491,679.92833443)(131.4997535,679.92834076)
\lineto(131.3647535,679.92834076)
\curveto(131.32474947,679.93833442)(131.28974951,679.93833442)(131.2597535,679.92834076)
\curveto(131.21974958,679.91833444)(131.18474961,679.91333445)(131.1547535,679.91334076)
\curveto(131.12474967,679.92333444)(131.0947497,679.91833444)(131.0647535,679.89834076)
\curveto(130.98474981,679.87833448)(130.92474987,679.83333453)(130.8847535,679.76334076)
\curveto(130.85474994,679.70333466)(130.82974997,679.62833473)(130.8097535,679.53834076)
\curveto(130.79975,679.48833487)(130.79975,679.43333493)(130.8097535,679.37334076)
\curveto(130.81974998,679.31333505)(130.81974998,679.2583351)(130.8097535,679.20834076)
\lineto(130.8097535,678.27834076)
\lineto(130.8097535,676.52334076)
\curveto(130.80974999,676.27333809)(130.81474998,676.05333831)(130.8247535,675.86334076)
\curveto(130.84474995,675.68333868)(130.90974989,675.52333884)(131.0197535,675.38334076)
\curveto(131.06974973,675.32333904)(131.13474966,675.27833908)(131.2147535,675.24834076)
\lineto(131.4847535,675.18834076)
\curveto(131.51474928,675.17833918)(131.54474925,675.17333919)(131.5747535,675.17334076)
\curveto(131.61474918,675.18333918)(131.64474915,675.18333918)(131.6647535,675.17334076)
\lineto(131.8297535,675.17334076)
\curveto(131.93974886,675.17333919)(132.03474876,675.16833919)(132.1147535,675.15834076)
\curveto(132.1947486,675.14833921)(132.25974854,675.10833925)(132.3097535,675.03834076)
\curveto(132.34974845,674.97833938)(132.36974843,674.89833946)(132.3697535,674.79834076)
\lineto(132.3697535,674.51334076)
\curveto(132.36974843,674.30334006)(132.36474843,674.10834025)(132.3547535,673.92834076)
\curveto(132.35474844,673.7583406)(132.27474852,673.64334072)(132.1147535,673.58334076)
\curveto(132.06474873,673.5633408)(132.01974878,673.5583408)(131.9797535,673.56834076)
\curveto(131.93974886,673.56834079)(131.8947489,673.5583408)(131.8447535,673.53834076)
\lineto(131.6947535,673.53834076)
\curveto(131.67474912,673.53834082)(131.64474915,673.54334082)(131.6047535,673.55334076)
\curveto(131.56474923,673.55334081)(131.52974927,673.54834081)(131.4997535,673.53834076)
\curveto(131.44974935,673.52834083)(131.3947494,673.52834083)(131.3347535,673.53834076)
\lineto(131.1847535,673.53834076)
\lineto(131.0347535,673.53834076)
\curveto(130.98474981,673.52834083)(130.93974986,673.52834083)(130.8997535,673.53834076)
\lineto(130.7347535,673.53834076)
\curveto(130.68475011,673.54834081)(130.62975017,673.55334081)(130.5697535,673.55334076)
\curveto(130.50975029,673.55334081)(130.45475034,673.5583408)(130.4047535,673.56834076)
\curveto(130.33475046,673.57834078)(130.26975053,673.58834077)(130.2097535,673.59834076)
\lineto(130.0297535,673.62834076)
\curveto(129.91975088,673.6583407)(129.81475098,673.69334067)(129.7147535,673.73334076)
\curveto(129.61475118,673.77334059)(129.51975128,673.81834054)(129.4297535,673.86834076)
\lineto(129.3397535,673.92834076)
\curveto(129.30975149,673.9583404)(129.27475152,673.98834037)(129.2347535,674.01834076)
\curveto(129.21475158,674.03834032)(129.18975161,674.0583403)(129.1597535,674.07834076)
\lineto(129.0847535,674.15334076)
\curveto(128.94475185,674.34334002)(128.83975196,674.55333981)(128.7697535,674.78334076)
\curveto(128.74975205,674.82333954)(128.73975206,674.8583395)(128.7397535,674.88834076)
\curveto(128.74975205,674.92833943)(128.74975205,674.97333939)(128.7397535,675.02334076)
\curveto(128.72975207,675.04333932)(128.72475207,675.06833929)(128.7247535,675.09834076)
\curveto(128.72475207,675.12833923)(128.71975208,675.15333921)(128.7097535,675.17334076)
\lineto(128.7097535,675.32334076)
\curveto(128.6997521,675.363339)(128.6947521,675.40833895)(128.6947535,675.45834076)
\curveto(128.70475209,675.50833885)(128.70975209,675.5583388)(128.7097535,675.60834076)
\lineto(128.7097535,676.17834076)
\lineto(128.7097535,678.41334076)
\lineto(128.7097535,679.20834076)
\lineto(128.7097535,679.41834076)
\curveto(128.71975208,679.48833487)(128.71475208,679.55333481)(128.6947535,679.61334076)
\curveto(128.65475214,679.75333461)(128.58475221,679.84333452)(128.4847535,679.88334076)
\curveto(128.37475242,679.93333443)(128.23475256,679.94833441)(128.0647535,679.92834076)
\curveto(127.8947529,679.90833445)(127.74975305,679.92333444)(127.6297535,679.97334076)
\curveto(127.54975325,680.00333436)(127.4997533,680.04833431)(127.4797535,680.10834076)
\curveto(127.45975334,680.16833419)(127.43975336,680.24333412)(127.4197535,680.33334076)
\lineto(127.4197535,680.64834076)
\curveto(127.41975338,680.82833353)(127.42975337,680.97333339)(127.4497535,681.08334076)
\curveto(127.46975333,681.19333317)(127.55475324,681.26833309)(127.7047535,681.30834076)
\curveto(127.74475305,681.32833303)(127.78475301,681.33333303)(127.8247535,681.32334076)
\lineto(127.9597535,681.32334076)
\curveto(128.10975269,681.32333304)(128.24975255,681.32833303)(128.3797535,681.33834076)
\curveto(128.50975229,681.358333)(128.5997522,681.41833294)(128.6497535,681.51834076)
\curveto(128.67975212,681.58833277)(128.6947521,681.66833269)(128.6947535,681.75834076)
\curveto(128.70475209,681.84833251)(128.70975209,681.93833242)(128.7097535,682.02834076)
\lineto(128.7097535,682.95834076)
\lineto(128.7097535,683.21334076)
\curveto(128.70975209,683.30333106)(128.71975208,683.37833098)(128.7397535,683.43834076)
\curveto(128.78975201,683.53833082)(128.86475193,683.60333076)(128.9647535,683.63334076)
\curveto(128.98475181,683.64333072)(129.00975179,683.64333072)(129.0397535,683.63334076)
\curveto(129.07975172,683.63333073)(129.10975169,683.63833072)(129.1297535,683.64834076)
}
}
{
\newrgbcolor{curcolor}{0 0 0}
\pscustom[linestyle=none,fillstyle=solid,fillcolor=curcolor]
{
\newpath
\moveto(137.778191,681.53334076)
\curveto(137.88818568,681.53333283)(137.98318559,681.52333284)(138.063191,681.50334076)
\curveto(138.15318542,681.48333288)(138.22318535,681.43833292)(138.273191,681.36834076)
\curveto(138.33318524,681.28833307)(138.36318521,681.14833321)(138.363191,680.94834076)
\lineto(138.363191,680.43834076)
\lineto(138.363191,680.06334076)
\curveto(138.3731852,679.92333444)(138.35818521,679.81333455)(138.318191,679.73334076)
\curveto(138.27818529,679.6633347)(138.21818535,679.61833474)(138.138191,679.59834076)
\curveto(138.0681855,679.57833478)(137.98318559,679.56833479)(137.883191,679.56834076)
\curveto(137.79318578,679.56833479)(137.69318588,679.57333479)(137.583191,679.58334076)
\curveto(137.48318609,679.59333477)(137.38818618,679.58833477)(137.298191,679.56834076)
\curveto(137.22818634,679.54833481)(137.15818641,679.53333483)(137.088191,679.52334076)
\curveto(137.01818655,679.52333484)(136.95318662,679.51333485)(136.893191,679.49334076)
\curveto(136.73318684,679.44333492)(136.573187,679.36833499)(136.413191,679.26834076)
\curveto(136.25318732,679.17833518)(136.12818744,679.07333529)(136.038191,678.95334076)
\curveto(135.98818758,678.87333549)(135.93318764,678.78833557)(135.873191,678.69834076)
\curveto(135.82318775,678.61833574)(135.7731878,678.53333583)(135.723191,678.44334076)
\curveto(135.69318788,678.363336)(135.66318791,678.27833608)(135.633191,678.18834076)
\lineto(135.573191,677.94834076)
\curveto(135.55318802,677.87833648)(135.54318803,677.80333656)(135.543191,677.72334076)
\curveto(135.54318803,677.65333671)(135.53318804,677.58333678)(135.513191,677.51334076)
\curveto(135.50318807,677.47333689)(135.49818807,677.43333693)(135.498191,677.39334076)
\curveto(135.50818806,677.363337)(135.50818806,677.33333703)(135.498191,677.30334076)
\lineto(135.498191,677.06334076)
\curveto(135.47818809,676.99333737)(135.4731881,676.91333745)(135.483191,676.82334076)
\curveto(135.49318808,676.74333762)(135.49818807,676.6633377)(135.498191,676.58334076)
\lineto(135.498191,675.62334076)
\lineto(135.498191,674.34834076)
\curveto(135.49818807,674.21834014)(135.49318808,674.09834026)(135.483191,673.98834076)
\curveto(135.4731881,673.87834048)(135.44318813,673.78834057)(135.393191,673.71834076)
\curveto(135.3731882,673.68834067)(135.33818823,673.6633407)(135.288191,673.64334076)
\curveto(135.24818832,673.63334073)(135.20318837,673.62334074)(135.153191,673.61334076)
\lineto(135.078191,673.61334076)
\curveto(135.02818854,673.60334076)(134.9731886,673.59834076)(134.913191,673.59834076)
\lineto(134.748191,673.59834076)
\lineto(134.103191,673.59834076)
\curveto(134.04318953,673.60834075)(133.97818959,673.61334075)(133.908191,673.61334076)
\lineto(133.713191,673.61334076)
\curveto(133.66318991,673.63334073)(133.61318996,673.64834071)(133.563191,673.65834076)
\curveto(133.51319006,673.67834068)(133.47819009,673.71334065)(133.458191,673.76334076)
\curveto(133.41819015,673.81334055)(133.39319018,673.88334048)(133.383191,673.97334076)
\lineto(133.383191,674.27334076)
\lineto(133.383191,675.29334076)
\lineto(133.383191,679.52334076)
\lineto(133.383191,680.63334076)
\lineto(133.383191,680.91834076)
\curveto(133.38319019,681.01833334)(133.40319017,681.09833326)(133.443191,681.15834076)
\curveto(133.49319008,681.23833312)(133.56819,681.28833307)(133.668191,681.30834076)
\curveto(133.7681898,681.32833303)(133.88818968,681.33833302)(134.028191,681.33834076)
\lineto(134.793191,681.33834076)
\curveto(134.91318866,681.33833302)(135.01818855,681.32833303)(135.108191,681.30834076)
\curveto(135.19818837,681.29833306)(135.2681883,681.25333311)(135.318191,681.17334076)
\curveto(135.34818822,681.12333324)(135.36318821,681.05333331)(135.363191,680.96334076)
\lineto(135.393191,680.69334076)
\curveto(135.40318817,680.61333375)(135.41818815,680.53833382)(135.438191,680.46834076)
\curveto(135.4681881,680.39833396)(135.51818805,680.363334)(135.588191,680.36334076)
\curveto(135.60818796,680.38333398)(135.62818794,680.39333397)(135.648191,680.39334076)
\curveto(135.6681879,680.39333397)(135.68818788,680.40333396)(135.708191,680.42334076)
\curveto(135.7681878,680.47333389)(135.81818775,680.52833383)(135.858191,680.58834076)
\curveto(135.90818766,680.6583337)(135.9681876,680.71833364)(136.038191,680.76834076)
\curveto(136.07818749,680.79833356)(136.11318746,680.82833353)(136.143191,680.85834076)
\curveto(136.1731874,680.89833346)(136.20818736,680.93333343)(136.248191,680.96334076)
\lineto(136.518191,681.14334076)
\curveto(136.61818695,681.20333316)(136.71818685,681.2583331)(136.818191,681.30834076)
\curveto(136.91818665,681.34833301)(137.01818655,681.38333298)(137.118191,681.41334076)
\lineto(137.448191,681.50334076)
\curveto(137.47818609,681.51333285)(137.53318604,681.51333285)(137.613191,681.50334076)
\curveto(137.70318587,681.50333286)(137.75818581,681.51333285)(137.778191,681.53334076)
}
}
{
\newrgbcolor{curcolor}{0 0 0}
\pscustom[linestyle=none,fillstyle=solid,fillcolor=curcolor]
{
\newpath
\moveto(146.95326912,677.78334076)
\curveto(146.93326059,677.83333653)(146.9282606,677.88833647)(146.93826912,677.94834076)
\curveto(146.94826058,678.00833635)(146.94326058,678.0633363)(146.92326912,678.11334076)
\curveto(146.91326061,678.15333621)(146.90826062,678.19333617)(146.90826912,678.23334076)
\curveto(146.90826062,678.27333609)(146.90326062,678.31333605)(146.89326912,678.35334076)
\lineto(146.83326912,678.62334076)
\curveto(146.81326071,678.71333565)(146.78826074,678.79833556)(146.75826912,678.87834076)
\curveto(146.70826082,679.01833534)(146.66326086,679.14833521)(146.62326912,679.26834076)
\curveto(146.58326094,679.39833496)(146.528261,679.51833484)(146.45826912,679.62834076)
\curveto(146.38826114,679.73833462)(146.31826121,679.84333452)(146.24826912,679.94334076)
\curveto(146.18826134,680.04333432)(146.11826141,680.14333422)(146.03826912,680.24334076)
\curveto(145.95826157,680.35333401)(145.85826167,680.45333391)(145.73826912,680.54334076)
\curveto(145.6282619,680.64333372)(145.51826201,680.73333363)(145.40826912,680.81334076)
\curveto(145.07826245,681.04333332)(144.69826283,681.22333314)(144.26826912,681.35334076)
\curveto(143.84826368,681.48333288)(143.34826418,681.54333282)(142.76826912,681.53334076)
\curveto(142.69826483,681.52333284)(142.6282649,681.51833284)(142.55826912,681.51834076)
\curveto(142.48826504,681.51833284)(142.41326511,681.51333285)(142.33326912,681.50334076)
\curveto(142.18326534,681.4633329)(142.03826549,681.43333293)(141.89826912,681.41334076)
\curveto(141.75826577,681.39333297)(141.6232659,681.358333)(141.49326912,681.30834076)
\curveto(141.38326614,681.2583331)(141.27326625,681.21333315)(141.16326912,681.17334076)
\curveto(141.05326647,681.13333323)(140.94826658,681.08833327)(140.84826912,681.03834076)
\curveto(140.48826704,680.80833355)(140.18326734,680.55333381)(139.93326912,680.27334076)
\curveto(139.68326784,680.00333436)(139.46826806,679.6633347)(139.28826912,679.25334076)
\curveto(139.23826829,679.13333523)(139.19826833,679.00833535)(139.16826912,678.87834076)
\curveto(139.13826839,678.7583356)(139.10326842,678.63333573)(139.06326912,678.50334076)
\curveto(139.04326848,678.45333591)(139.03326849,678.40333596)(139.03326912,678.35334076)
\curveto(139.03326849,678.31333605)(139.0282685,678.26833609)(139.01826912,678.21834076)
\curveto(138.99826853,678.16833619)(138.98826854,678.11333625)(138.98826912,678.05334076)
\curveto(138.99826853,678.00333636)(138.99826853,677.95333641)(138.98826912,677.90334076)
\lineto(138.98826912,677.79834076)
\curveto(138.96826856,677.73833662)(138.95326857,677.65333671)(138.94326912,677.54334076)
\curveto(138.94326858,677.43333693)(138.95326857,677.34833701)(138.97326912,677.28834076)
\lineto(138.97326912,677.15334076)
\curveto(138.97326855,677.11333725)(138.97826855,677.06833729)(138.98826912,677.01834076)
\curveto(139.00826852,676.93833742)(139.01826851,676.85333751)(139.01826912,676.76334076)
\curveto(139.01826851,676.68333768)(139.0282685,676.60333776)(139.04826912,676.52334076)
\curveto(139.06826846,676.47333789)(139.07826845,676.42833793)(139.07826912,676.38834076)
\curveto(139.07826845,676.34833801)(139.08826844,676.30333806)(139.10826912,676.25334076)
\curveto(139.13826839,676.14333822)(139.16326836,676.03833832)(139.18326912,675.93834076)
\curveto(139.21326831,675.83833852)(139.25326827,675.74333862)(139.30326912,675.65334076)
\curveto(139.47326805,675.2633391)(139.68326784,674.92833943)(139.93326912,674.64834076)
\curveto(140.18326734,674.36833999)(140.48326704,674.12334024)(140.83326912,673.91334076)
\curveto(140.94326658,673.85334051)(141.04826648,673.80334056)(141.14826912,673.76334076)
\curveto(141.25826627,673.72334064)(141.37326615,673.68334068)(141.49326912,673.64334076)
\curveto(141.58326594,673.60334076)(141.67826585,673.57334079)(141.77826912,673.55334076)
\curveto(141.87826565,673.53334083)(141.97826555,673.50834085)(142.07826912,673.47834076)
\curveto(142.1282654,673.46834089)(142.16826536,673.4633409)(142.19826912,673.46334076)
\curveto(142.23826529,673.4633409)(142.27826525,673.4583409)(142.31826912,673.44834076)
\curveto(142.36826516,673.42834093)(142.41826511,673.42334094)(142.46826912,673.43334076)
\curveto(142.528265,673.43334093)(142.58326494,673.42834093)(142.63326912,673.41834076)
\lineto(142.78326912,673.41834076)
\curveto(142.84326468,673.39834096)(142.9282646,673.39334097)(143.03826912,673.40334076)
\curveto(143.14826438,673.40334096)(143.2282643,673.40834095)(143.27826912,673.41834076)
\curveto(143.30826422,673.41834094)(143.33826419,673.42334094)(143.36826912,673.43334076)
\lineto(143.47326912,673.43334076)
\curveto(143.523264,673.44334092)(143.57826395,673.44834091)(143.63826912,673.44834076)
\curveto(143.69826383,673.44834091)(143.75326377,673.4583409)(143.80326912,673.47834076)
\curveto(143.93326359,673.50834085)(144.05826347,673.53834082)(144.17826912,673.56834076)
\curveto(144.30826322,673.58834077)(144.43326309,673.62334074)(144.55326912,673.67334076)
\curveto(145.03326249,673.87334049)(145.44326208,674.12334024)(145.78326912,674.42334076)
\curveto(146.1232614,674.72333964)(146.39826113,675.11333925)(146.60826912,675.59334076)
\curveto(146.65826087,675.69333867)(146.69826083,675.79833856)(146.72826912,675.90834076)
\curveto(146.75826077,676.02833833)(146.79326073,676.14333822)(146.83326912,676.25334076)
\curveto(146.84326068,676.32333804)(146.85326067,676.38833797)(146.86326912,676.44834076)
\curveto(146.87326065,676.50833785)(146.88826064,676.57333779)(146.90826912,676.64334076)
\curveto(146.9282606,676.72333764)(146.93326059,676.80333756)(146.92326912,676.88334076)
\curveto(146.9232606,676.9633374)(146.93326059,677.04333732)(146.95326912,677.12334076)
\lineto(146.95326912,677.27334076)
\curveto(146.97326055,677.33333703)(146.98326054,677.41833694)(146.98326912,677.52834076)
\curveto(146.98326054,677.63833672)(146.97326055,677.72333664)(146.95326912,677.78334076)
\moveto(144.85326912,677.24334076)
\curveto(144.84326268,677.19333717)(144.83826269,677.14333722)(144.83826912,677.09334076)
\lineto(144.83826912,676.95834076)
\curveto(144.8282627,676.91833744)(144.8232627,676.87833748)(144.82326912,676.83834076)
\curveto(144.8232627,676.80833755)(144.81826271,676.77333759)(144.80826912,676.73334076)
\curveto(144.77826275,676.62333774)(144.75326277,676.51833784)(144.73326912,676.41834076)
\curveto(144.71326281,676.31833804)(144.68326284,676.21833814)(144.64326912,676.11834076)
\curveto(144.53326299,675.86833849)(144.39826313,675.6583387)(144.23826912,675.48834076)
\curveto(144.07826345,675.31833904)(143.86826366,675.18333918)(143.60826912,675.08334076)
\curveto(143.53826399,675.05333931)(143.46326406,675.03333933)(143.38326912,675.02334076)
\curveto(143.30326422,675.01333935)(143.2232643,674.99833936)(143.14326912,674.97834076)
\lineto(143.02326912,674.97834076)
\curveto(142.98326454,674.96833939)(142.93826459,674.9633394)(142.88826912,674.96334076)
\lineto(142.76826912,674.99334076)
\curveto(142.7282648,675.00333936)(142.69326483,675.00333936)(142.66326912,674.99334076)
\curveto(142.63326489,674.99333937)(142.59826493,674.99833936)(142.55826912,675.00834076)
\curveto(142.46826506,675.02833933)(142.37826515,675.05333931)(142.28826912,675.08334076)
\curveto(142.20826532,675.11333925)(142.13326539,675.15333921)(142.06326912,675.20334076)
\curveto(141.81326571,675.35333901)(141.6282659,675.51833884)(141.50826912,675.69834076)
\curveto(141.39826613,675.88833847)(141.29326623,676.13333823)(141.19326912,676.43334076)
\curveto(141.17326635,676.51333785)(141.15826637,676.58833777)(141.14826912,676.65834076)
\curveto(141.13826639,676.73833762)(141.1232664,676.81833754)(141.10326912,676.89834076)
\lineto(141.10326912,677.03334076)
\curveto(141.08326644,677.10333726)(141.06826646,677.20833715)(141.05826912,677.34834076)
\curveto(141.05826647,677.48833687)(141.06826646,677.59333677)(141.08826912,677.66334076)
\lineto(141.08826912,677.81334076)
\curveto(141.08826644,677.8633365)(141.09326643,677.91333645)(141.10326912,677.96334076)
\curveto(141.1232664,678.07333629)(141.13826639,678.18333618)(141.14826912,678.29334076)
\curveto(141.16826636,678.40333596)(141.19326633,678.50833585)(141.22326912,678.60834076)
\curveto(141.31326621,678.87833548)(141.43326609,679.11333525)(141.58326912,679.31334076)
\curveto(141.74326578,679.52333484)(141.94826558,679.68333468)(142.19826912,679.79334076)
\curveto(142.24826528,679.82333454)(142.30326522,679.84333452)(142.36326912,679.85334076)
\lineto(142.57326912,679.91334076)
\curveto(142.60326492,679.92333444)(142.63826489,679.92333444)(142.67826912,679.91334076)
\curveto(142.71826481,679.91333445)(142.75326477,679.92333444)(142.78326912,679.94334076)
\lineto(143.05326912,679.94334076)
\curveto(143.14326438,679.95333441)(143.2282643,679.94833441)(143.30826912,679.92834076)
\curveto(143.37826415,679.90833445)(143.44326408,679.88833447)(143.50326912,679.86834076)
\curveto(143.56326396,679.8583345)(143.6232639,679.84333452)(143.68326912,679.82334076)
\curveto(143.93326359,679.71333465)(144.13326339,679.5633348)(144.28326912,679.37334076)
\curveto(144.43326309,679.19333517)(144.56326296,678.97333539)(144.67326912,678.71334076)
\curveto(144.70326282,678.63333573)(144.7232628,678.54833581)(144.73326912,678.45834076)
\lineto(144.79326912,678.21834076)
\curveto(144.80326272,678.19833616)(144.80826272,678.16833619)(144.80826912,678.12834076)
\curveto(144.81826271,678.07833628)(144.8232627,678.02333634)(144.82326912,677.96334076)
\curveto(144.8232627,677.90333646)(144.83326269,677.84833651)(144.85326912,677.79834076)
\lineto(144.85326912,677.67834076)
\curveto(144.86326266,677.62833673)(144.86826266,677.55333681)(144.86826912,677.45334076)
\curveto(144.86826266,677.363337)(144.86326266,677.29333707)(144.85326912,677.24334076)
\moveto(143.62326912,684.41334076)
\lineto(144.68826912,684.41334076)
\curveto(144.76826276,684.41332995)(144.86326266,684.41332995)(144.97326912,684.41334076)
\curveto(145.08326244,684.41332995)(145.16326236,684.39832996)(145.21326912,684.36834076)
\curveto(145.23326229,684.35833)(145.24326228,684.34333002)(145.24326912,684.32334076)
\curveto(145.25326227,684.31333005)(145.26826226,684.30333006)(145.28826912,684.29334076)
\curveto(145.29826223,684.17333019)(145.24826228,684.06833029)(145.13826912,683.97834076)
\curveto(145.03826249,683.88833047)(144.95326257,683.80833055)(144.88326912,683.73834076)
\curveto(144.80326272,683.66833069)(144.7232628,683.59333077)(144.64326912,683.51334076)
\curveto(144.57326295,683.44333092)(144.49826303,683.37833098)(144.41826912,683.31834076)
\curveto(144.37826315,683.28833107)(144.34326318,683.25333111)(144.31326912,683.21334076)
\curveto(144.29326323,683.18333118)(144.26326326,683.1583312)(144.22326912,683.13834076)
\curveto(144.20326332,683.10833125)(144.17826335,683.08333128)(144.14826912,683.06334076)
\lineto(143.99826912,682.91334076)
\lineto(143.84826912,682.79334076)
\lineto(143.80326912,682.74834076)
\curveto(143.80326372,682.73833162)(143.79326373,682.72333164)(143.77326912,682.70334076)
\curveto(143.69326383,682.64333172)(143.61326391,682.57833178)(143.53326912,682.50834076)
\curveto(143.46326406,682.43833192)(143.37326415,682.38333198)(143.26326912,682.34334076)
\curveto(143.2232643,682.33333203)(143.18326434,682.32833203)(143.14326912,682.32834076)
\curveto(143.11326441,682.32833203)(143.07326445,682.32333204)(143.02326912,682.31334076)
\curveto(142.99326453,682.30333206)(142.95326457,682.29833206)(142.90326912,682.29834076)
\curveto(142.85326467,682.30833205)(142.80826472,682.31333205)(142.76826912,682.31334076)
\lineto(142.42326912,682.31334076)
\curveto(142.30326522,682.31333205)(142.21326531,682.33833202)(142.15326912,682.38834076)
\curveto(142.09326543,682.42833193)(142.07826545,682.49833186)(142.10826912,682.59834076)
\curveto(142.1282654,682.67833168)(142.16326536,682.74833161)(142.21326912,682.80834076)
\curveto(142.26326526,682.87833148)(142.30826522,682.94833141)(142.34826912,683.01834076)
\curveto(142.44826508,683.1583312)(142.54326498,683.29333107)(142.63326912,683.42334076)
\curveto(142.7232648,683.55333081)(142.81326471,683.68833067)(142.90326912,683.82834076)
\curveto(142.95326457,683.90833045)(143.00326452,683.99333037)(143.05326912,684.08334076)
\curveto(143.11326441,684.17333019)(143.17826435,684.24333012)(143.24826912,684.29334076)
\curveto(143.28826424,684.32333004)(143.35826417,684.35833)(143.45826912,684.39834076)
\curveto(143.47826405,684.40832995)(143.50326402,684.40832995)(143.53326912,684.39834076)
\curveto(143.57326395,684.39832996)(143.60326392,684.40332996)(143.62326912,684.41334076)
}
}
{
\newrgbcolor{curcolor}{0 0 0}
\pscustom[linestyle=none,fillstyle=solid,fillcolor=curcolor]
{
\newpath
\moveto(152.778191,681.53334076)
\curveto(153.37818519,681.55333281)(153.87818469,681.46833289)(154.278191,681.27834076)
\curveto(154.67818389,681.08833327)(154.99318358,680.80833355)(155.223191,680.43834076)
\curveto(155.29318328,680.32833403)(155.34818322,680.20833415)(155.388191,680.07834076)
\curveto(155.42818314,679.9583344)(155.4681831,679.83333453)(155.508191,679.70334076)
\curveto(155.52818304,679.62333474)(155.53818303,679.54833481)(155.538191,679.47834076)
\curveto(155.54818302,679.40833495)(155.56318301,679.33833502)(155.583191,679.26834076)
\curveto(155.58318299,679.20833515)(155.58818298,679.16833519)(155.598191,679.14834076)
\curveto(155.61818295,679.00833535)(155.62818294,678.8633355)(155.628191,678.71334076)
\lineto(155.628191,678.27834076)
\lineto(155.628191,676.94334076)
\lineto(155.628191,674.51334076)
\curveto(155.62818294,674.32334004)(155.62318295,674.13834022)(155.613191,673.95834076)
\curveto(155.61318296,673.78834057)(155.54318303,673.67834068)(155.403191,673.62834076)
\curveto(155.34318323,673.60834075)(155.2731833,673.59834076)(155.193191,673.59834076)
\lineto(154.953191,673.59834076)
\lineto(154.143191,673.59834076)
\curveto(154.02318455,673.59834076)(153.91318466,673.60334076)(153.813191,673.61334076)
\curveto(153.72318485,673.63334073)(153.65318492,673.67834068)(153.603191,673.74834076)
\curveto(153.56318501,673.80834055)(153.53818503,673.88334048)(153.528191,673.97334076)
\lineto(153.528191,674.28834076)
\lineto(153.528191,675.33834076)
\lineto(153.528191,677.57334076)
\curveto(153.52818504,677.94333642)(153.51318506,678.28333608)(153.483191,678.59334076)
\curveto(153.45318512,678.91333545)(153.36318521,679.18333518)(153.213191,679.40334076)
\curveto(153.0731855,679.60333476)(152.8681857,679.74333462)(152.598191,679.82334076)
\curveto(152.54818602,679.84333452)(152.49318608,679.85333451)(152.433191,679.85334076)
\curveto(152.38318619,679.85333451)(152.32818624,679.8633345)(152.268191,679.88334076)
\curveto(152.21818635,679.89333447)(152.15318642,679.89333447)(152.073191,679.88334076)
\curveto(152.00318657,679.88333448)(151.94818662,679.87833448)(151.908191,679.86834076)
\curveto(151.8681867,679.8583345)(151.83318674,679.85333451)(151.803191,679.85334076)
\curveto(151.7731868,679.85333451)(151.74318683,679.84833451)(151.713191,679.83834076)
\curveto(151.48318709,679.77833458)(151.29818727,679.69833466)(151.158191,679.59834076)
\curveto(150.83818773,679.36833499)(150.64818792,679.03333533)(150.588191,678.59334076)
\curveto(150.52818804,678.15333621)(150.49818807,677.6583367)(150.498191,677.10834076)
\lineto(150.498191,675.23334076)
\lineto(150.498191,674.31834076)
\lineto(150.498191,674.04834076)
\curveto(150.49818807,673.9583404)(150.48318809,673.88334048)(150.453191,673.82334076)
\curveto(150.40318817,673.71334065)(150.32318825,673.64834071)(150.213191,673.62834076)
\curveto(150.10318847,673.60834075)(149.9681886,673.59834076)(149.808191,673.59834076)
\lineto(149.058191,673.59834076)
\curveto(148.94818962,673.59834076)(148.83818973,673.60334076)(148.728191,673.61334076)
\curveto(148.61818995,673.62334074)(148.53819003,673.6583407)(148.488191,673.71834076)
\curveto(148.41819015,673.80834055)(148.38319019,673.93834042)(148.383191,674.10834076)
\curveto(148.39319018,674.27834008)(148.39819017,674.43833992)(148.398191,674.58834076)
\lineto(148.398191,676.62834076)
\lineto(148.398191,679.92834076)
\lineto(148.398191,680.69334076)
\lineto(148.398191,680.99334076)
\curveto(148.40819016,681.08333328)(148.43819013,681.1583332)(148.488191,681.21834076)
\curveto(148.50819006,681.24833311)(148.53819003,681.26833309)(148.578191,681.27834076)
\curveto(148.62818994,681.29833306)(148.67818989,681.31333305)(148.728191,681.32334076)
\lineto(148.803191,681.32334076)
\curveto(148.85318972,681.33333303)(148.90318967,681.33833302)(148.953191,681.33834076)
\lineto(149.118191,681.33834076)
\lineto(149.748191,681.33834076)
\curveto(149.82818874,681.33833302)(149.90318867,681.33333303)(149.973191,681.32334076)
\curveto(150.05318852,681.32333304)(150.12318845,681.31333305)(150.183191,681.29334076)
\curveto(150.25318832,681.2633331)(150.29818827,681.21833314)(150.318191,681.15834076)
\curveto(150.34818822,681.09833326)(150.3731882,681.02833333)(150.393191,680.94834076)
\curveto(150.40318817,680.90833345)(150.40318817,680.87333349)(150.393191,680.84334076)
\curveto(150.39318818,680.81333355)(150.40318817,680.78333358)(150.423191,680.75334076)
\curveto(150.44318813,680.70333366)(150.45818811,680.67333369)(150.468191,680.66334076)
\curveto(150.48818808,680.65333371)(150.51318806,680.63833372)(150.543191,680.61834076)
\curveto(150.65318792,680.60833375)(150.74318783,680.64333372)(150.813191,680.72334076)
\curveto(150.88318769,680.81333355)(150.95818761,680.88333348)(151.038191,680.93334076)
\curveto(151.30818726,681.13333323)(151.60818696,681.29333307)(151.938191,681.41334076)
\curveto(152.02818654,681.44333292)(152.11818645,681.4633329)(152.208191,681.47334076)
\curveto(152.30818626,681.48333288)(152.41318616,681.49833286)(152.523191,681.51834076)
\curveto(152.55318602,681.52833283)(152.59818597,681.52833283)(152.658191,681.51834076)
\curveto(152.71818585,681.51833284)(152.75818581,681.52333284)(152.778191,681.53334076)
}
}
{
\newrgbcolor{curcolor}{0 0 0}
\pscustom[linestyle=none,fillstyle=solid,fillcolor=curcolor]
{
\newpath
\moveto(159.359441,684.18834076)
\curveto(159.42943805,684.10833025)(159.46443801,683.98833037)(159.464441,683.82834076)
\lineto(159.464441,683.36334076)
\lineto(159.464441,682.95834076)
\curveto(159.46443801,682.81833154)(159.42943805,682.72333164)(159.359441,682.67334076)
\curveto(159.29943818,682.62333174)(159.21943826,682.59333177)(159.119441,682.58334076)
\curveto(159.02943845,682.57333179)(158.92943855,682.56833179)(158.819441,682.56834076)
\lineto(157.979441,682.56834076)
\curveto(157.86943961,682.56833179)(157.76943971,682.57333179)(157.679441,682.58334076)
\curveto(157.59943988,682.59333177)(157.52943995,682.62333174)(157.469441,682.67334076)
\curveto(157.42944005,682.70333166)(157.39944008,682.7583316)(157.379441,682.83834076)
\curveto(157.36944011,682.92833143)(157.35944012,683.02333134)(157.349441,683.12334076)
\lineto(157.349441,683.45334076)
\curveto(157.35944012,683.5633308)(157.36444011,683.6583307)(157.364441,683.73834076)
\lineto(157.364441,683.94834076)
\curveto(157.3744401,684.01833034)(157.39444008,684.07833028)(157.424441,684.12834076)
\curveto(157.44444003,684.16833019)(157.46944001,684.19833016)(157.499441,684.21834076)
\lineto(157.619441,684.27834076)
\curveto(157.63943984,684.27833008)(157.66443981,684.27833008)(157.694441,684.27834076)
\curveto(157.72443975,684.28833007)(157.74943973,684.29333007)(157.769441,684.29334076)
\lineto(158.864441,684.29334076)
\curveto(158.96443851,684.29333007)(159.05943842,684.28833007)(159.149441,684.27834076)
\curveto(159.23943824,684.26833009)(159.30943817,684.23833012)(159.359441,684.18834076)
\moveto(159.464441,674.42334076)
\curveto(159.46443801,674.22334014)(159.45943802,674.05334031)(159.449441,673.91334076)
\curveto(159.43943804,673.77334059)(159.34943813,673.67834068)(159.179441,673.62834076)
\curveto(159.11943836,673.60834075)(159.05443842,673.59834076)(158.984441,673.59834076)
\curveto(158.91443856,673.60834075)(158.83943864,673.61334075)(158.759441,673.61334076)
\lineto(157.919441,673.61334076)
\curveto(157.82943965,673.61334075)(157.73943974,673.61834074)(157.649441,673.62834076)
\curveto(157.56943991,673.63834072)(157.50943997,673.66834069)(157.469441,673.71834076)
\curveto(157.40944007,673.78834057)(157.3744401,673.87334049)(157.364441,673.97334076)
\lineto(157.364441,674.31834076)
\lineto(157.364441,680.64834076)
\lineto(157.364441,680.94834076)
\curveto(157.36444011,681.04833331)(157.38444009,681.12833323)(157.424441,681.18834076)
\curveto(157.48443999,681.2583331)(157.56943991,681.30333306)(157.679441,681.32334076)
\curveto(157.69943978,681.33333303)(157.72443975,681.33333303)(157.754441,681.32334076)
\curveto(157.79443968,681.32333304)(157.82443965,681.32833303)(157.844441,681.33834076)
\lineto(158.594441,681.33834076)
\lineto(158.789441,681.33834076)
\curveto(158.86943861,681.34833301)(158.93443854,681.34833301)(158.984441,681.33834076)
\lineto(159.104441,681.33834076)
\curveto(159.16443831,681.31833304)(159.21943826,681.30333306)(159.269441,681.29334076)
\curveto(159.31943816,681.28333308)(159.35943812,681.25333311)(159.389441,681.20334076)
\curveto(159.42943805,681.15333321)(159.44943803,681.08333328)(159.449441,680.99334076)
\curveto(159.45943802,680.90333346)(159.46443801,680.80833355)(159.464441,680.70834076)
\lineto(159.464441,674.42334076)
}
}
{
\newrgbcolor{curcolor}{0 0 0}
\pscustom[linestyle=none,fillstyle=solid,fillcolor=curcolor]
{
\newpath
\moveto(164.6966285,681.54834076)
\curveto(165.50662334,681.56833279)(166.18162266,681.44833291)(166.7216285,681.18834076)
\curveto(167.27162157,680.92833343)(167.70662114,680.5583338)(168.0266285,680.07834076)
\curveto(168.18662066,679.83833452)(168.30662054,679.5633348)(168.3866285,679.25334076)
\curveto(168.40662044,679.20333516)(168.42162042,679.13833522)(168.4316285,679.05834076)
\curveto(168.45162039,678.97833538)(168.45162039,678.90833545)(168.4316285,678.84834076)
\curveto(168.39162045,678.73833562)(168.32162052,678.67333569)(168.2216285,678.65334076)
\curveto(168.12162072,678.64333572)(168.00162084,678.63833572)(167.8616285,678.63834076)
\lineto(167.0816285,678.63834076)
\lineto(166.7966285,678.63834076)
\curveto(166.70662214,678.63833572)(166.63162221,678.6583357)(166.5716285,678.69834076)
\curveto(166.49162235,678.73833562)(166.43662241,678.79833556)(166.4066285,678.87834076)
\curveto(166.37662247,678.96833539)(166.33662251,679.0583353)(166.2866285,679.14834076)
\curveto(166.22662262,679.2583351)(166.16162268,679.358335)(166.0916285,679.44834076)
\curveto(166.02162282,679.53833482)(165.9416229,679.61833474)(165.8516285,679.68834076)
\curveto(165.71162313,679.77833458)(165.55662329,679.84833451)(165.3866285,679.89834076)
\curveto(165.32662352,679.91833444)(165.26662358,679.92833443)(165.2066285,679.92834076)
\curveto(165.1466237,679.92833443)(165.09162375,679.93833442)(165.0416285,679.95834076)
\lineto(164.8916285,679.95834076)
\curveto(164.69162415,679.9583344)(164.53162431,679.93833442)(164.4116285,679.89834076)
\curveto(164.12162472,679.80833455)(163.88662496,679.66833469)(163.7066285,679.47834076)
\curveto(163.52662532,679.29833506)(163.38162546,679.07833528)(163.2716285,678.81834076)
\curveto(163.22162562,678.70833565)(163.18162566,678.58833577)(163.1516285,678.45834076)
\curveto(163.13162571,678.33833602)(163.10662574,678.20833615)(163.0766285,678.06834076)
\curveto(163.06662578,678.02833633)(163.06162578,677.98833637)(163.0616285,677.94834076)
\curveto(163.06162578,677.90833645)(163.05662579,677.86833649)(163.0466285,677.82834076)
\curveto(163.02662582,677.72833663)(163.01662583,677.58833677)(163.0166285,677.40834076)
\curveto(163.02662582,677.22833713)(163.0416258,677.08833727)(163.0616285,676.98834076)
\curveto(163.06162578,676.90833745)(163.06662578,676.85333751)(163.0766285,676.82334076)
\curveto(163.09662575,676.75333761)(163.10662574,676.68333768)(163.1066285,676.61334076)
\curveto(163.11662573,676.54333782)(163.13162571,676.47333789)(163.1516285,676.40334076)
\curveto(163.23162561,676.17333819)(163.32662552,675.9633384)(163.4366285,675.77334076)
\curveto(163.5466253,675.58333878)(163.68662516,675.42333894)(163.8566285,675.29334076)
\curveto(163.89662495,675.2633391)(163.95662489,675.22833913)(164.0366285,675.18834076)
\curveto(164.1466247,675.11833924)(164.25662459,675.07333929)(164.3666285,675.05334076)
\curveto(164.48662436,675.03333933)(164.63162421,675.01333935)(164.8016285,674.99334076)
\lineto(164.8916285,674.99334076)
\curveto(164.93162391,674.99333937)(164.96162388,674.99833936)(164.9816285,675.00834076)
\lineto(165.1166285,675.00834076)
\curveto(165.18662366,675.02833933)(165.25162359,675.04333932)(165.3116285,675.05334076)
\curveto(165.38162346,675.07333929)(165.4466234,675.09333927)(165.5066285,675.11334076)
\curveto(165.80662304,675.24333912)(166.03662281,675.43333893)(166.1966285,675.68334076)
\curveto(166.23662261,675.73333863)(166.27162257,675.78833857)(166.3016285,675.84834076)
\curveto(166.33162251,675.91833844)(166.35662249,675.97833838)(166.3766285,676.02834076)
\curveto(166.41662243,676.13833822)(166.45162239,676.23333813)(166.4816285,676.31334076)
\curveto(166.51162233,676.40333796)(166.58162226,676.47333789)(166.6916285,676.52334076)
\curveto(166.78162206,676.5633378)(166.92662192,676.57833778)(167.1266285,676.56834076)
\lineto(167.6216285,676.56834076)
\lineto(167.8316285,676.56834076)
\curveto(167.91162093,676.57833778)(167.97662087,676.57333779)(168.0266285,676.55334076)
\lineto(168.1466285,676.55334076)
\lineto(168.2666285,676.52334076)
\curveto(168.30662054,676.52333784)(168.33662051,676.51333785)(168.3566285,676.49334076)
\curveto(168.40662044,676.45333791)(168.43662041,676.39333797)(168.4466285,676.31334076)
\curveto(168.46662038,676.24333812)(168.46662038,676.16833819)(168.4466285,676.08834076)
\curveto(168.35662049,675.7583386)(168.2466206,675.4633389)(168.1166285,675.20334076)
\curveto(167.70662114,674.43333993)(167.05162179,673.89834046)(166.1516285,673.59834076)
\curveto(166.05162279,673.56834079)(165.9466229,673.54834081)(165.8366285,673.53834076)
\curveto(165.72662312,673.51834084)(165.61662323,673.49334087)(165.5066285,673.46334076)
\curveto(165.4466234,673.45334091)(165.38662346,673.44834091)(165.3266285,673.44834076)
\curveto(165.26662358,673.44834091)(165.20662364,673.44334092)(165.1466285,673.43334076)
\lineto(164.9816285,673.43334076)
\curveto(164.93162391,673.41334095)(164.85662399,673.40834095)(164.7566285,673.41834076)
\curveto(164.65662419,673.41834094)(164.58162426,673.42334094)(164.5316285,673.43334076)
\curveto(164.45162439,673.45334091)(164.37662447,673.4633409)(164.3066285,673.46334076)
\curveto(164.2466246,673.45334091)(164.18162466,673.4583409)(164.1116285,673.47834076)
\lineto(163.9616285,673.50834076)
\curveto(163.91162493,673.50834085)(163.86162498,673.51334085)(163.8116285,673.52334076)
\curveto(163.70162514,673.55334081)(163.59662525,673.58334078)(163.4966285,673.61334076)
\curveto(163.39662545,673.64334072)(163.30162554,673.67834068)(163.2116285,673.71834076)
\curveto(162.7416261,673.91834044)(162.3466265,674.17334019)(162.0266285,674.48334076)
\curveto(161.70662714,674.80333956)(161.4466274,675.19833916)(161.2466285,675.66834076)
\curveto(161.19662765,675.7583386)(161.15662769,675.85333851)(161.1266285,675.95334076)
\lineto(161.0366285,676.28334076)
\curveto(161.02662782,676.32333804)(161.02162782,676.358338)(161.0216285,676.38834076)
\curveto(161.02162782,676.42833793)(161.01162783,676.47333789)(160.9916285,676.52334076)
\curveto(160.97162787,676.59333777)(160.96162788,676.6633377)(160.9616285,676.73334076)
\curveto(160.96162788,676.81333755)(160.95162789,676.88833747)(160.9316285,676.95834076)
\lineto(160.9316285,677.21334076)
\curveto(160.91162793,677.2633371)(160.90162794,677.31833704)(160.9016285,677.37834076)
\curveto(160.90162794,677.44833691)(160.91162793,677.50833685)(160.9316285,677.55834076)
\curveto(160.9416279,677.60833675)(160.9416279,677.65333671)(160.9316285,677.69334076)
\curveto(160.92162792,677.73333663)(160.92162792,677.77333659)(160.9316285,677.81334076)
\curveto(160.95162789,677.88333648)(160.95662789,677.94833641)(160.9466285,678.00834076)
\curveto(160.9466279,678.06833629)(160.95662789,678.12833623)(160.9766285,678.18834076)
\curveto(161.02662782,678.36833599)(161.06662778,678.53833582)(161.0966285,678.69834076)
\curveto(161.12662772,678.86833549)(161.17162767,679.03333533)(161.2316285,679.19334076)
\curveto(161.45162739,679.70333466)(161.72662712,680.12833423)(162.0566285,680.46834076)
\curveto(162.39662645,680.80833355)(162.82662602,681.08333328)(163.3466285,681.29334076)
\curveto(163.48662536,681.35333301)(163.63162521,681.39333297)(163.7816285,681.41334076)
\curveto(163.93162491,681.44333292)(164.08662476,681.47833288)(164.2466285,681.51834076)
\curveto(164.32662452,681.52833283)(164.40162444,681.53333283)(164.4716285,681.53334076)
\curveto(164.5416243,681.53333283)(164.61662423,681.53833282)(164.6966285,681.54834076)
}
}
{
\newrgbcolor{curcolor}{0 0 0}
\pscustom[linestyle=none,fillstyle=solid,fillcolor=curcolor]
{
\newpath
\moveto(177.50990975,677.78334076)
\curveto(177.52990118,677.72333664)(177.53990117,677.63833672)(177.53990975,677.52834076)
\curveto(177.53990117,677.41833694)(177.52990118,677.33333703)(177.50990975,677.27334076)
\lineto(177.50990975,677.12334076)
\curveto(177.48990122,677.04333732)(177.47990123,676.9633374)(177.47990975,676.88334076)
\curveto(177.48990122,676.80333756)(177.48490122,676.72333764)(177.46490975,676.64334076)
\curveto(177.44490126,676.57333779)(177.42990128,676.50833785)(177.41990975,676.44834076)
\curveto(177.4099013,676.38833797)(177.39990131,676.32333804)(177.38990975,676.25334076)
\curveto(177.34990136,676.14333822)(177.31490139,676.02833833)(177.28490975,675.90834076)
\curveto(177.25490145,675.79833856)(177.21490149,675.69333867)(177.16490975,675.59334076)
\curveto(176.95490175,675.11333925)(176.67990203,674.72333964)(176.33990975,674.42334076)
\curveto(175.99990271,674.12334024)(175.58990312,673.87334049)(175.10990975,673.67334076)
\curveto(174.98990372,673.62334074)(174.86490384,673.58834077)(174.73490975,673.56834076)
\curveto(174.61490409,673.53834082)(174.48990422,673.50834085)(174.35990975,673.47834076)
\curveto(174.3099044,673.4583409)(174.25490445,673.44834091)(174.19490975,673.44834076)
\curveto(174.13490457,673.44834091)(174.07990463,673.44334092)(174.02990975,673.43334076)
\lineto(173.92490975,673.43334076)
\curveto(173.89490481,673.42334094)(173.86490484,673.41834094)(173.83490975,673.41834076)
\curveto(173.78490492,673.40834095)(173.704905,673.40334096)(173.59490975,673.40334076)
\curveto(173.48490522,673.39334097)(173.39990531,673.39834096)(173.33990975,673.41834076)
\lineto(173.18990975,673.41834076)
\curveto(173.13990557,673.42834093)(173.08490562,673.43334093)(173.02490975,673.43334076)
\curveto(172.97490573,673.42334094)(172.92490578,673.42834093)(172.87490975,673.44834076)
\curveto(172.83490587,673.4583409)(172.79490591,673.4633409)(172.75490975,673.46334076)
\curveto(172.72490598,673.4633409)(172.68490602,673.46834089)(172.63490975,673.47834076)
\curveto(172.53490617,673.50834085)(172.43490627,673.53334083)(172.33490975,673.55334076)
\curveto(172.23490647,673.57334079)(172.13990657,673.60334076)(172.04990975,673.64334076)
\curveto(171.92990678,673.68334068)(171.81490689,673.72334064)(171.70490975,673.76334076)
\curveto(171.6049071,673.80334056)(171.49990721,673.85334051)(171.38990975,673.91334076)
\curveto(171.03990767,674.12334024)(170.73990797,674.36833999)(170.48990975,674.64834076)
\curveto(170.23990847,674.92833943)(170.02990868,675.2633391)(169.85990975,675.65334076)
\curveto(169.8099089,675.74333862)(169.76990894,675.83833852)(169.73990975,675.93834076)
\curveto(169.71990899,676.03833832)(169.69490901,676.14333822)(169.66490975,676.25334076)
\curveto(169.64490906,676.30333806)(169.63490907,676.34833801)(169.63490975,676.38834076)
\curveto(169.63490907,676.42833793)(169.62490908,676.47333789)(169.60490975,676.52334076)
\curveto(169.58490912,676.60333776)(169.57490913,676.68333768)(169.57490975,676.76334076)
\curveto(169.57490913,676.85333751)(169.56490914,676.93833742)(169.54490975,677.01834076)
\curveto(169.53490917,677.06833729)(169.52990918,677.11333725)(169.52990975,677.15334076)
\lineto(169.52990975,677.28834076)
\curveto(169.5099092,677.34833701)(169.49990921,677.43333693)(169.49990975,677.54334076)
\curveto(169.5099092,677.65333671)(169.52490918,677.73833662)(169.54490975,677.79834076)
\lineto(169.54490975,677.90334076)
\curveto(169.55490915,677.95333641)(169.55490915,678.00333636)(169.54490975,678.05334076)
\curveto(169.54490916,678.11333625)(169.55490915,678.16833619)(169.57490975,678.21834076)
\curveto(169.58490912,678.26833609)(169.58990912,678.31333605)(169.58990975,678.35334076)
\curveto(169.58990912,678.40333596)(169.59990911,678.45333591)(169.61990975,678.50334076)
\curveto(169.65990905,678.63333573)(169.69490901,678.7583356)(169.72490975,678.87834076)
\curveto(169.75490895,679.00833535)(169.79490891,679.13333523)(169.84490975,679.25334076)
\curveto(170.02490868,679.6633347)(170.23990847,680.00333436)(170.48990975,680.27334076)
\curveto(170.73990797,680.55333381)(171.04490766,680.80833355)(171.40490975,681.03834076)
\curveto(171.5049072,681.08833327)(171.6099071,681.13333323)(171.71990975,681.17334076)
\curveto(171.82990688,681.21333315)(171.93990677,681.2583331)(172.04990975,681.30834076)
\curveto(172.17990653,681.358333)(172.31490639,681.39333297)(172.45490975,681.41334076)
\curveto(172.59490611,681.43333293)(172.73990597,681.4633329)(172.88990975,681.50334076)
\curveto(172.96990574,681.51333285)(173.04490566,681.51833284)(173.11490975,681.51834076)
\curveto(173.18490552,681.51833284)(173.25490545,681.52333284)(173.32490975,681.53334076)
\curveto(173.9049048,681.54333282)(174.4049043,681.48333288)(174.82490975,681.35334076)
\curveto(175.25490345,681.22333314)(175.63490307,681.04333332)(175.96490975,680.81334076)
\curveto(176.07490263,680.73333363)(176.18490252,680.64333372)(176.29490975,680.54334076)
\curveto(176.41490229,680.45333391)(176.51490219,680.35333401)(176.59490975,680.24334076)
\curveto(176.67490203,680.14333422)(176.74490196,680.04333432)(176.80490975,679.94334076)
\curveto(176.87490183,679.84333452)(176.94490176,679.73833462)(177.01490975,679.62834076)
\curveto(177.08490162,679.51833484)(177.13990157,679.39833496)(177.17990975,679.26834076)
\curveto(177.21990149,679.14833521)(177.26490144,679.01833534)(177.31490975,678.87834076)
\curveto(177.34490136,678.79833556)(177.36990134,678.71333565)(177.38990975,678.62334076)
\lineto(177.44990975,678.35334076)
\curveto(177.45990125,678.31333605)(177.46490124,678.27333609)(177.46490975,678.23334076)
\curveto(177.46490124,678.19333617)(177.46990124,678.15333621)(177.47990975,678.11334076)
\curveto(177.49990121,678.0633363)(177.5049012,678.00833635)(177.49490975,677.94834076)
\curveto(177.48490122,677.88833647)(177.48990122,677.83333653)(177.50990975,677.78334076)
\moveto(175.40990975,677.24334076)
\curveto(175.41990329,677.29333707)(175.42490328,677.363337)(175.42490975,677.45334076)
\curveto(175.42490328,677.55333681)(175.41990329,677.62833673)(175.40990975,677.67834076)
\lineto(175.40990975,677.79834076)
\curveto(175.38990332,677.84833651)(175.37990333,677.90333646)(175.37990975,677.96334076)
\curveto(175.37990333,678.02333634)(175.37490333,678.07833628)(175.36490975,678.12834076)
\curveto(175.36490334,678.16833619)(175.35990335,678.19833616)(175.34990975,678.21834076)
\lineto(175.28990975,678.45834076)
\curveto(175.27990343,678.54833581)(175.25990345,678.63333573)(175.22990975,678.71334076)
\curveto(175.11990359,678.97333539)(174.98990372,679.19333517)(174.83990975,679.37334076)
\curveto(174.68990402,679.5633348)(174.48990422,679.71333465)(174.23990975,679.82334076)
\curveto(174.17990453,679.84333452)(174.11990459,679.8583345)(174.05990975,679.86834076)
\curveto(173.99990471,679.88833447)(173.93490477,679.90833445)(173.86490975,679.92834076)
\curveto(173.78490492,679.94833441)(173.69990501,679.95333441)(173.60990975,679.94334076)
\lineto(173.33990975,679.94334076)
\curveto(173.3099054,679.92333444)(173.27490543,679.91333445)(173.23490975,679.91334076)
\curveto(173.19490551,679.92333444)(173.15990555,679.92333444)(173.12990975,679.91334076)
\lineto(172.91990975,679.85334076)
\curveto(172.85990585,679.84333452)(172.8049059,679.82333454)(172.75490975,679.79334076)
\curveto(172.5049062,679.68333468)(172.29990641,679.52333484)(172.13990975,679.31334076)
\curveto(171.98990672,679.11333525)(171.86990684,678.87833548)(171.77990975,678.60834076)
\curveto(171.74990696,678.50833585)(171.72490698,678.40333596)(171.70490975,678.29334076)
\curveto(171.69490701,678.18333618)(171.67990703,678.07333629)(171.65990975,677.96334076)
\curveto(171.64990706,677.91333645)(171.64490706,677.8633365)(171.64490975,677.81334076)
\lineto(171.64490975,677.66334076)
\curveto(171.62490708,677.59333677)(171.61490709,677.48833687)(171.61490975,677.34834076)
\curveto(171.62490708,677.20833715)(171.63990707,677.10333726)(171.65990975,677.03334076)
\lineto(171.65990975,676.89834076)
\curveto(171.67990703,676.81833754)(171.69490701,676.73833762)(171.70490975,676.65834076)
\curveto(171.71490699,676.58833777)(171.72990698,676.51333785)(171.74990975,676.43334076)
\curveto(171.84990686,676.13333823)(171.95490675,675.88833847)(172.06490975,675.69834076)
\curveto(172.18490652,675.51833884)(172.36990634,675.35333901)(172.61990975,675.20334076)
\curveto(172.68990602,675.15333921)(172.76490594,675.11333925)(172.84490975,675.08334076)
\curveto(172.93490577,675.05333931)(173.02490568,675.02833933)(173.11490975,675.00834076)
\curveto(173.15490555,674.99833936)(173.18990552,674.99333937)(173.21990975,674.99334076)
\curveto(173.24990546,675.00333936)(173.28490542,675.00333936)(173.32490975,674.99334076)
\lineto(173.44490975,674.96334076)
\curveto(173.49490521,674.9633394)(173.53990517,674.96833939)(173.57990975,674.97834076)
\lineto(173.69990975,674.97834076)
\curveto(173.77990493,674.99833936)(173.85990485,675.01333935)(173.93990975,675.02334076)
\curveto(174.01990469,675.03333933)(174.09490461,675.05333931)(174.16490975,675.08334076)
\curveto(174.42490428,675.18333918)(174.63490407,675.31833904)(174.79490975,675.48834076)
\curveto(174.95490375,675.6583387)(175.08990362,675.86833849)(175.19990975,676.11834076)
\curveto(175.23990347,676.21833814)(175.26990344,676.31833804)(175.28990975,676.41834076)
\curveto(175.3099034,676.51833784)(175.33490337,676.62333774)(175.36490975,676.73334076)
\curveto(175.37490333,676.77333759)(175.37990333,676.80833755)(175.37990975,676.83834076)
\curveto(175.37990333,676.87833748)(175.38490332,676.91833744)(175.39490975,676.95834076)
\lineto(175.39490975,677.09334076)
\curveto(175.39490331,677.14333722)(175.39990331,677.19333717)(175.40990975,677.24334076)
}
}
{
\newrgbcolor{curcolor}{0 0 0}
\pscustom[linestyle=none,fillstyle=solid,fillcolor=curcolor]
{
\newpath
\moveto(393.83927712,702.34389374)
\lineto(395.11427712,702.34389374)
\curveto(395.22427434,702.34388303)(395.32927423,702.33888303)(395.42927712,702.32889374)
\curveto(395.53927402,702.31888305)(395.61927394,702.28388309)(395.66927712,702.22389374)
\curveto(395.71927384,702.14388323)(395.74427382,702.03888333)(395.74427712,701.90889374)
\curveto(395.75427381,701.78888358)(395.7592738,701.66388371)(395.75927712,701.53389374)
\lineto(395.75927712,700.01889374)
\lineto(395.75927712,696.92889374)
\lineto(395.75927712,696.40389374)
\curveto(395.7592738,696.36388901)(395.75427381,696.31888905)(395.74427712,696.26889374)
\curveto(395.74427382,696.22888914)(395.74927381,696.18888918)(395.75927712,696.14889374)
\lineto(395.75927712,695.90889374)
\curveto(395.7592738,695.81888955)(395.75427381,695.72388965)(395.74427712,695.62389374)
\curveto(395.74427382,695.52388985)(395.75427381,695.43388994)(395.77427712,695.35389374)
\curveto(395.77427379,695.28389009)(395.77927378,695.22889014)(395.78927712,695.18889374)
\curveto(395.80927375,695.07889029)(395.82427374,694.9688904)(395.83427712,694.85889374)
\curveto(395.85427371,694.74889062)(395.88427368,694.63889073)(395.92427712,694.52889374)
\curveto(396.03427353,694.2688911)(396.17427339,694.05389132)(396.34427712,693.88389374)
\curveto(396.52427304,693.71389166)(396.7592728,693.57889179)(397.04927712,693.47889374)
\curveto(397.12927243,693.45889191)(397.20927235,693.44389193)(397.28927712,693.43389374)
\curveto(397.36927219,693.42389195)(397.44927211,693.40889196)(397.52927712,693.38889374)
\curveto(397.57927198,693.368892)(397.62427194,693.35889201)(397.66427712,693.35889374)
\curveto(397.70427186,693.368892)(397.74927181,693.368892)(397.79927712,693.35889374)
\curveto(397.83927172,693.34889202)(397.90427166,693.34389203)(397.99427712,693.34389374)
\curveto(398.08427148,693.35389202)(398.14427142,693.36389201)(398.17427712,693.37389374)
\lineto(398.39927712,693.37389374)
\curveto(398.47927108,693.39389198)(398.559271,693.40889196)(398.63927712,693.41889374)
\curveto(398.71927084,693.42889194)(398.79427077,693.44389193)(398.86427712,693.46389374)
\curveto(399.00427056,693.49389188)(399.11427045,693.52889184)(399.19427712,693.56889374)
\curveto(399.37427019,693.64889172)(399.52927003,693.75389162)(399.65927712,693.88389374)
\curveto(399.79926976,694.02389135)(399.90926965,694.17889119)(399.98927712,694.34889374)
\curveto(400.09926946,694.60889076)(400.1642694,694.91389046)(400.18427712,695.26389374)
\curveto(400.20426936,695.62388975)(400.21426935,695.99388938)(400.21427712,696.37389374)
\lineto(400.21427712,699.35889374)
\lineto(400.21427712,701.36889374)
\curveto(400.21426935,701.50888386)(400.20926935,701.66388371)(400.19927712,701.83389374)
\curveto(400.19926936,702.00388337)(400.22926933,702.12888324)(400.28927712,702.20889374)
\curveto(400.33926922,702.2688831)(400.40926915,702.30388307)(400.49927712,702.31389374)
\curveto(400.58926897,702.33388304)(400.68926887,702.34388303)(400.79927712,702.34389374)
\lineto(401.75927712,702.34389374)
\curveto(401.83926772,702.34388303)(401.91426765,702.34388303)(401.98427712,702.34389374)
\curveto(402.0642675,702.35388302)(402.13926742,702.34888302)(402.20927712,702.32889374)
\curveto(402.34926721,702.29888307)(402.43926712,702.24888312)(402.47927712,702.17889374)
\curveto(402.52926703,702.09888327)(402.54926701,701.98388339)(402.53927712,701.83389374)
\curveto(402.53926702,701.69388368)(402.53926702,701.56388381)(402.53927712,701.44389374)
\lineto(402.53927712,699.43389374)
\lineto(402.53927712,696.40389374)
\curveto(402.53926702,696.02388935)(402.53426703,695.65388972)(402.52427712,695.29389374)
\curveto(402.51426705,694.93389044)(402.46926709,694.60889076)(402.38927712,694.31889374)
\curveto(402.24926731,693.84889152)(402.06926749,693.43889193)(401.84927712,693.08889374)
\curveto(401.63926792,692.74889262)(401.3592682,692.45889291)(401.00927712,692.21889374)
\curveto(400.69926886,691.99889337)(400.33426923,691.81889355)(399.91427712,691.67889374)
\curveto(399.82426974,691.64889372)(399.72926983,691.62389375)(399.62927712,691.60389374)
\lineto(399.35927712,691.54389374)
\curveto(399.29927026,691.52389385)(399.23927032,691.51389386)(399.17927712,691.51389374)
\curveto(399.12927043,691.51389386)(399.07427049,691.50389387)(399.01427712,691.48389374)
\curveto(398.89427067,691.46389391)(398.7592708,691.44889392)(398.60927712,691.43889374)
\curveto(398.4592711,691.42889394)(398.31427125,691.42389395)(398.17427712,691.42389374)
\curveto(397.22427234,691.41389396)(396.41427315,691.52889384)(395.74427712,691.76889374)
\curveto(395.07427449,692.01889335)(394.54927501,692.41889295)(394.16927712,692.96889374)
\curveto(394.03927552,693.14889222)(393.92927563,693.33389204)(393.83927712,693.52389374)
\curveto(393.7592758,693.72389165)(393.68427588,693.93889143)(393.61427712,694.16889374)
\curveto(393.59427597,694.21889115)(393.58427598,694.25889111)(393.58427712,694.28889374)
\curveto(393.58427598,694.32889104)(393.57427599,694.373891)(393.55427712,694.42389374)
\curveto(393.47427609,694.70389067)(393.43427613,695.01889035)(393.43427712,695.36889374)
\lineto(393.43427712,696.41889374)
\lineto(393.43427712,700.60389374)
\lineto(393.43427712,701.65389374)
\lineto(393.43427712,701.93889374)
\curveto(393.43427613,702.03888333)(393.44927611,702.11888325)(393.47927712,702.17889374)
\curveto(393.53927602,702.24888312)(393.61927594,702.29888307)(393.71927712,702.32889374)
\curveto(393.73927582,702.32888304)(393.7592758,702.32888304)(393.77927712,702.32889374)
\curveto(393.79927576,702.32888304)(393.81927574,702.33388304)(393.83927712,702.34389374)
}
}
{
\newrgbcolor{curcolor}{0 0 0}
\pscustom[linestyle=none,fillstyle=solid,fillcolor=curcolor]
{
\newpath
\moveto(407.29779274,699.58389374)
\curveto(408.04778824,699.60388577)(408.69778759,699.51888585)(409.24779274,699.32889374)
\curveto(409.80778648,699.14888622)(410.23278606,698.83388654)(410.52279274,698.38389374)
\curveto(410.5927857,698.2738871)(410.65278564,698.15888721)(410.70279274,698.03889374)
\curveto(410.76278553,697.92888744)(410.81278548,697.80388757)(410.85279274,697.66389374)
\curveto(410.87278542,697.60388777)(410.88278541,697.53888783)(410.88279274,697.46889374)
\curveto(410.88278541,697.39888797)(410.87278542,697.33888803)(410.85279274,697.28889374)
\curveto(410.81278548,697.22888814)(410.75778553,697.18888818)(410.68779274,697.16889374)
\curveto(410.63778565,697.14888822)(410.57778571,697.13888823)(410.50779274,697.13889374)
\lineto(410.29779274,697.13889374)
\lineto(409.63779274,697.13889374)
\curveto(409.56778672,697.13888823)(409.49778679,697.13388824)(409.42779274,697.12389374)
\curveto(409.35778693,697.12388825)(409.292787,697.13388824)(409.23279274,697.15389374)
\curveto(409.13278716,697.1738882)(409.05778723,697.21388816)(409.00779274,697.27389374)
\curveto(408.95778733,697.33388804)(408.91278738,697.39388798)(408.87279274,697.45389374)
\lineto(408.75279274,697.66389374)
\curveto(408.72278757,697.74388763)(408.67278762,697.80888756)(408.60279274,697.85889374)
\curveto(408.50278779,697.93888743)(408.40278789,697.99888737)(408.30279274,698.03889374)
\curveto(408.21278808,698.07888729)(408.09778819,698.11388726)(407.95779274,698.14389374)
\curveto(407.8877884,698.16388721)(407.78278851,698.17888719)(407.64279274,698.18889374)
\curveto(407.51278878,698.19888717)(407.41278888,698.19388718)(407.34279274,698.17389374)
\lineto(407.23779274,698.17389374)
\lineto(407.08779274,698.14389374)
\curveto(407.04778924,698.14388723)(407.00278929,698.13888723)(406.95279274,698.12889374)
\curveto(406.78278951,698.07888729)(406.64278965,698.00888736)(406.53279274,697.91889374)
\curveto(406.43278986,697.83888753)(406.36278993,697.71388766)(406.32279274,697.54389374)
\curveto(406.30278999,697.4738879)(406.30278999,697.40888796)(406.32279274,697.34889374)
\curveto(406.34278995,697.28888808)(406.36278993,697.23888813)(406.38279274,697.19889374)
\curveto(406.45278984,697.07888829)(406.53278976,696.98388839)(406.62279274,696.91389374)
\curveto(406.72278957,696.84388853)(406.83778945,696.78388859)(406.96779274,696.73389374)
\curveto(407.15778913,696.65388872)(407.36278893,696.58388879)(407.58279274,696.52389374)
\lineto(408.27279274,696.37389374)
\curveto(408.51278778,696.33388904)(408.74278755,696.28388909)(408.96279274,696.22389374)
\curveto(409.1927871,696.1738892)(409.40778688,696.10888926)(409.60779274,696.02889374)
\curveto(409.69778659,695.98888938)(409.78278651,695.95388942)(409.86279274,695.92389374)
\curveto(409.95278634,695.90388947)(410.03778625,695.8688895)(410.11779274,695.81889374)
\curveto(410.30778598,695.69888967)(410.47778581,695.5688898)(410.62779274,695.42889374)
\curveto(410.7877855,695.28889008)(410.91278538,695.11389026)(411.00279274,694.90389374)
\curveto(411.03278526,694.83389054)(411.05778523,694.76389061)(411.07779274,694.69389374)
\curveto(411.09778519,694.62389075)(411.11778517,694.54889082)(411.13779274,694.46889374)
\curveto(411.14778514,694.40889096)(411.15278514,694.31389106)(411.15279274,694.18389374)
\curveto(411.16278513,694.06389131)(411.16278513,693.9688914)(411.15279274,693.89889374)
\lineto(411.15279274,693.82389374)
\curveto(411.13278516,693.76389161)(411.11778517,693.70389167)(411.10779274,693.64389374)
\curveto(411.10778518,693.59389178)(411.10278519,693.54389183)(411.09279274,693.49389374)
\curveto(411.02278527,693.19389218)(410.91278538,692.92889244)(410.76279274,692.69889374)
\curveto(410.60278569,692.45889291)(410.40778588,692.26389311)(410.17779274,692.11389374)
\curveto(409.94778634,691.96389341)(409.6877866,691.83389354)(409.39779274,691.72389374)
\curveto(409.287787,691.6738937)(409.16778712,691.63889373)(409.03779274,691.61889374)
\curveto(408.91778737,691.59889377)(408.79778749,691.5738938)(408.67779274,691.54389374)
\curveto(408.5877877,691.52389385)(408.4927878,691.51389386)(408.39279274,691.51389374)
\curveto(408.30278799,691.50389387)(408.21278808,691.48889388)(408.12279274,691.46889374)
\lineto(407.85279274,691.46889374)
\curveto(407.7927885,691.44889392)(407.6877886,691.43889393)(407.53779274,691.43889374)
\curveto(407.39778889,691.43889393)(407.29778899,691.44889392)(407.23779274,691.46889374)
\curveto(407.20778908,691.4688939)(407.17278912,691.4738939)(407.13279274,691.48389374)
\lineto(407.02779274,691.48389374)
\curveto(406.90778938,691.50389387)(406.7877895,691.51889385)(406.66779274,691.52889374)
\curveto(406.54778974,691.53889383)(406.43278986,691.55889381)(406.32279274,691.58889374)
\curveto(405.93279036,691.69889367)(405.5877907,691.82389355)(405.28779274,691.96389374)
\curveto(404.9877913,692.11389326)(404.73279156,692.33389304)(404.52279274,692.62389374)
\curveto(404.38279191,692.81389256)(404.26279203,693.03389234)(404.16279274,693.28389374)
\curveto(404.14279215,693.34389203)(404.12279217,693.42389195)(404.10279274,693.52389374)
\curveto(404.08279221,693.5738918)(404.06779222,693.64389173)(404.05779274,693.73389374)
\curveto(404.04779224,693.82389155)(404.05279224,693.89889147)(404.07279274,693.95889374)
\curveto(404.10279219,694.02889134)(404.15279214,694.07889129)(404.22279274,694.10889374)
\curveto(404.27279202,694.12889124)(404.33279196,694.13889123)(404.40279274,694.13889374)
\lineto(404.62779274,694.13889374)
\lineto(405.33279274,694.13889374)
\lineto(405.57279274,694.13889374)
\curveto(405.65279064,694.13889123)(405.72279057,694.12889124)(405.78279274,694.10889374)
\curveto(405.8927904,694.0688913)(405.96279033,694.00389137)(405.99279274,693.91389374)
\curveto(406.03279026,693.82389155)(406.07779021,693.72889164)(406.12779274,693.62889374)
\curveto(406.14779014,693.57889179)(406.18279011,693.51389186)(406.23279274,693.43389374)
\curveto(406.29279,693.35389202)(406.34278995,693.30389207)(406.38279274,693.28389374)
\curveto(406.50278979,693.18389219)(406.61778967,693.10389227)(406.72779274,693.04389374)
\curveto(406.83778945,692.99389238)(406.97778931,692.94389243)(407.14779274,692.89389374)
\curveto(407.19778909,692.8738925)(407.24778904,692.86389251)(407.29779274,692.86389374)
\curveto(407.34778894,692.8738925)(407.39778889,692.8738925)(407.44779274,692.86389374)
\curveto(407.52778876,692.84389253)(407.61278868,692.83389254)(407.70279274,692.83389374)
\curveto(407.80278849,692.84389253)(407.8877884,692.85889251)(407.95779274,692.87889374)
\curveto(408.00778828,692.88889248)(408.05278824,692.89389248)(408.09279274,692.89389374)
\curveto(408.14278815,692.89389248)(408.1927881,692.90389247)(408.24279274,692.92389374)
\curveto(408.38278791,692.9738924)(408.50778778,693.03389234)(408.61779274,693.10389374)
\curveto(408.73778755,693.1738922)(408.83278746,693.26389211)(408.90279274,693.37389374)
\curveto(408.95278734,693.45389192)(408.9927873,693.57889179)(409.02279274,693.74889374)
\curveto(409.04278725,693.81889155)(409.04278725,693.88389149)(409.02279274,693.94389374)
\curveto(409.00278729,694.00389137)(408.98278731,694.05389132)(408.96279274,694.09389374)
\curveto(408.8927874,694.23389114)(408.80278749,694.33889103)(408.69279274,694.40889374)
\curveto(408.5927877,694.47889089)(408.47278782,694.54389083)(408.33279274,694.60389374)
\curveto(408.14278815,694.68389069)(407.94278835,694.74889062)(407.73279274,694.79889374)
\curveto(407.52278877,694.84889052)(407.31278898,694.90389047)(407.10279274,694.96389374)
\curveto(407.02278927,694.98389039)(406.93778935,694.99889037)(406.84779274,695.00889374)
\curveto(406.76778952,695.01889035)(406.6877896,695.03389034)(406.60779274,695.05389374)
\curveto(406.28779,695.14389023)(405.98279031,695.22889014)(405.69279274,695.30889374)
\curveto(405.40279089,695.39888997)(405.13779115,695.52888984)(404.89779274,695.69889374)
\curveto(404.61779167,695.89888947)(404.41279188,696.1688892)(404.28279274,696.50889374)
\curveto(404.26279203,696.57888879)(404.24279205,696.6738887)(404.22279274,696.79389374)
\curveto(404.20279209,696.86388851)(404.1877921,696.94888842)(404.17779274,697.04889374)
\curveto(404.16779212,697.14888822)(404.17279212,697.23888813)(404.19279274,697.31889374)
\curveto(404.21279208,697.368888)(404.21779207,697.40888796)(404.20779274,697.43889374)
\curveto(404.19779209,697.47888789)(404.20279209,697.52388785)(404.22279274,697.57389374)
\curveto(404.24279205,697.68388769)(404.26279203,697.78388759)(404.28279274,697.87389374)
\curveto(404.31279198,697.9738874)(404.34779194,698.0688873)(404.38779274,698.15889374)
\curveto(404.51779177,698.44888692)(404.69779159,698.68388669)(404.92779274,698.86389374)
\curveto(405.15779113,699.04388633)(405.41779087,699.18888618)(405.70779274,699.29889374)
\curveto(405.81779047,699.34888602)(405.93279036,699.38388599)(406.05279274,699.40389374)
\curveto(406.17279012,699.43388594)(406.29778999,699.46388591)(406.42779274,699.49389374)
\curveto(406.4877898,699.51388586)(406.54778974,699.52388585)(406.60779274,699.52389374)
\lineto(406.78779274,699.55389374)
\curveto(406.86778942,699.56388581)(406.95278934,699.5688858)(407.04279274,699.56889374)
\curveto(407.13278916,699.5688858)(407.21778907,699.5738858)(407.29779274,699.58389374)
}
}
{
\newrgbcolor{curcolor}{0 0 0}
\pscustom[linestyle=none,fillstyle=solid,fillcolor=curcolor]
{
\newpath
\moveto(412.80443337,699.35889374)
\lineto(413.92943337,699.35889374)
\curveto(414.03943093,699.35888601)(414.13943083,699.35388602)(414.22943337,699.34389374)
\curveto(414.31943065,699.33388604)(414.38443059,699.29888607)(414.42443337,699.23889374)
\curveto(414.4744305,699.17888619)(414.50443047,699.09388628)(414.51443337,698.98389374)
\curveto(414.52443045,698.88388649)(414.52943044,698.77888659)(414.52943337,698.66889374)
\lineto(414.52943337,697.61889374)
\lineto(414.52943337,695.38389374)
\curveto(414.52943044,695.02389035)(414.54443043,694.68389069)(414.57443337,694.36389374)
\curveto(414.60443037,694.04389133)(414.69443028,693.77889159)(414.84443337,693.56889374)
\curveto(414.98442999,693.35889201)(415.20942976,693.20889216)(415.51943337,693.11889374)
\curveto(415.5694294,693.10889226)(415.60942936,693.10389227)(415.63943337,693.10389374)
\curveto(415.67942929,693.10389227)(415.72442925,693.09889227)(415.77443337,693.08889374)
\curveto(415.82442915,693.07889229)(415.87942909,693.0738923)(415.93943337,693.07389374)
\curveto(415.99942897,693.0738923)(416.04442893,693.07889229)(416.07443337,693.08889374)
\curveto(416.12442885,693.10889226)(416.16442881,693.11389226)(416.19443337,693.10389374)
\curveto(416.23442874,693.09389228)(416.2744287,693.09889227)(416.31443337,693.11889374)
\curveto(416.52442845,693.1688922)(416.68942828,693.23389214)(416.80943337,693.31389374)
\curveto(416.98942798,693.42389195)(417.12942784,693.56389181)(417.22943337,693.73389374)
\curveto(417.33942763,693.91389146)(417.41442756,694.10889126)(417.45443337,694.31889374)
\curveto(417.50442747,694.53889083)(417.53442744,694.77889059)(417.54443337,695.03889374)
\curveto(417.55442742,695.30889006)(417.55942741,695.58888978)(417.55943337,695.87889374)
\lineto(417.55943337,697.69389374)
\lineto(417.55943337,698.66889374)
\lineto(417.55943337,698.93889374)
\curveto(417.55942741,699.03888633)(417.57942739,699.11888625)(417.61943337,699.17889374)
\curveto(417.6694273,699.2688861)(417.74442723,699.31888605)(417.84443337,699.32889374)
\curveto(417.94442703,699.34888602)(418.06442691,699.35888601)(418.20443337,699.35889374)
\lineto(418.99943337,699.35889374)
\lineto(419.28443337,699.35889374)
\curveto(419.3744256,699.35888601)(419.44942552,699.33888603)(419.50943337,699.29889374)
\curveto(419.58942538,699.24888612)(419.63442534,699.1738862)(419.64443337,699.07389374)
\curveto(419.65442532,698.9738864)(419.65942531,698.85888651)(419.65943337,698.72889374)
\lineto(419.65943337,697.58889374)
\lineto(419.65943337,693.37389374)
\lineto(419.65943337,692.30889374)
\lineto(419.65943337,692.00889374)
\curveto(419.65942531,691.90889346)(419.63942533,691.83389354)(419.59943337,691.78389374)
\curveto(419.54942542,691.70389367)(419.4744255,691.65889371)(419.37443337,691.64889374)
\curveto(419.2744257,691.63889373)(419.1694258,691.63389374)(419.05943337,691.63389374)
\lineto(418.24943337,691.63389374)
\curveto(418.13942683,691.63389374)(418.03942693,691.63889373)(417.94943337,691.64889374)
\curveto(417.8694271,691.65889371)(417.80442717,691.69889367)(417.75443337,691.76889374)
\curveto(417.73442724,691.79889357)(417.71442726,691.84389353)(417.69443337,691.90389374)
\curveto(417.68442729,691.96389341)(417.6694273,692.02389335)(417.64943337,692.08389374)
\curveto(417.63942733,692.14389323)(417.62442735,692.19889317)(417.60443337,692.24889374)
\curveto(417.58442739,692.29889307)(417.55442742,692.32889304)(417.51443337,692.33889374)
\curveto(417.49442748,692.35889301)(417.4694275,692.36389301)(417.43943337,692.35389374)
\curveto(417.40942756,692.34389303)(417.38442759,692.33389304)(417.36443337,692.32389374)
\curveto(417.29442768,692.28389309)(417.23442774,692.23889313)(417.18443337,692.18889374)
\curveto(417.13442784,692.13889323)(417.07942789,692.09389328)(417.01943337,692.05389374)
\curveto(416.97942799,692.02389335)(416.93942803,691.98889338)(416.89943337,691.94889374)
\curveto(416.8694281,691.91889345)(416.82942814,691.88889348)(416.77943337,691.85889374)
\curveto(416.54942842,691.71889365)(416.27942869,691.60889376)(415.96943337,691.52889374)
\curveto(415.89942907,691.50889386)(415.82942914,691.49889387)(415.75943337,691.49889374)
\curveto(415.68942928,691.48889388)(415.61442936,691.4738939)(415.53443337,691.45389374)
\curveto(415.49442948,691.44389393)(415.44942952,691.44389393)(415.39943337,691.45389374)
\curveto(415.35942961,691.45389392)(415.31942965,691.44889392)(415.27943337,691.43889374)
\curveto(415.24942972,691.42889394)(415.18442979,691.42889394)(415.08443337,691.43889374)
\curveto(414.99442998,691.43889393)(414.93443004,691.44389393)(414.90443337,691.45389374)
\curveto(414.85443012,691.45389392)(414.80443017,691.45889391)(414.75443337,691.46889374)
\lineto(414.60443337,691.46889374)
\curveto(414.48443049,691.49889387)(414.3694306,691.52389385)(414.25943337,691.54389374)
\curveto(414.14943082,691.56389381)(414.03943093,691.59389378)(413.92943337,691.63389374)
\curveto(413.87943109,691.65389372)(413.83443114,691.6688937)(413.79443337,691.67889374)
\curveto(413.76443121,691.69889367)(413.72443125,691.71889365)(413.67443337,691.73889374)
\curveto(413.32443165,691.92889344)(413.04443193,692.19389318)(412.83443337,692.53389374)
\curveto(412.70443227,692.74389263)(412.60943236,692.99389238)(412.54943337,693.28389374)
\curveto(412.48943248,693.58389179)(412.44943252,693.89889147)(412.42943337,694.22889374)
\curveto(412.41943255,694.5688908)(412.41443256,694.91389046)(412.41443337,695.26389374)
\curveto(412.42443255,695.62388975)(412.42943254,695.97888939)(412.42943337,696.32889374)
\lineto(412.42943337,698.36889374)
\curveto(412.42943254,698.49888687)(412.42443255,698.64888672)(412.41443337,698.81889374)
\curveto(412.41443256,698.99888637)(412.43943253,699.12888624)(412.48943337,699.20889374)
\curveto(412.51943245,699.25888611)(412.57943239,699.30388607)(412.66943337,699.34389374)
\curveto(412.72943224,699.34388603)(412.7744322,699.34888602)(412.80443337,699.35889374)
}
}
{
\newrgbcolor{curcolor}{0 0 0}
\pscustom[linestyle=none,fillstyle=solid,fillcolor=curcolor]
{
\newpath
\moveto(428.34068337,692.23389374)
\curveto(428.36067552,692.12389325)(428.37067551,692.01389336)(428.37068337,691.90389374)
\curveto(428.3806755,691.79389358)(428.33067555,691.71889365)(428.22068337,691.67889374)
\curveto(428.16067572,691.64889372)(428.09067579,691.63389374)(428.01068337,691.63389374)
\lineto(427.77068337,691.63389374)
\lineto(426.96068337,691.63389374)
\lineto(426.69068337,691.63389374)
\curveto(426.61067727,691.64389373)(426.54567733,691.6688937)(426.49568337,691.70889374)
\curveto(426.42567745,691.74889362)(426.37067751,691.80389357)(426.33068337,691.87389374)
\curveto(426.30067758,691.95389342)(426.25567762,692.01889335)(426.19568337,692.06889374)
\curveto(426.1756777,692.08889328)(426.15067773,692.10389327)(426.12068337,692.11389374)
\curveto(426.09067779,692.13389324)(426.05067783,692.13889323)(426.00068337,692.12889374)
\curveto(425.95067793,692.10889326)(425.90067798,692.08389329)(425.85068337,692.05389374)
\curveto(425.81067807,692.02389335)(425.76567811,691.99889337)(425.71568337,691.97889374)
\curveto(425.66567821,691.93889343)(425.61067827,691.90389347)(425.55068337,691.87389374)
\lineto(425.37068337,691.78389374)
\curveto(425.24067864,691.72389365)(425.10567877,691.6738937)(424.96568337,691.63389374)
\curveto(424.82567905,691.60389377)(424.6806792,691.5688938)(424.53068337,691.52889374)
\curveto(424.46067942,691.50889386)(424.39067949,691.49889387)(424.32068337,691.49889374)
\curveto(424.26067962,691.48889388)(424.19567968,691.47889389)(424.12568337,691.46889374)
\lineto(424.03568337,691.46889374)
\curveto(424.00567987,691.45889391)(423.9756799,691.45389392)(423.94568337,691.45389374)
\lineto(423.78068337,691.45389374)
\curveto(423.6806802,691.43389394)(423.5806803,691.43389394)(423.48068337,691.45389374)
\lineto(423.34568337,691.45389374)
\curveto(423.2756806,691.4738939)(423.20568067,691.48389389)(423.13568337,691.48389374)
\curveto(423.0756808,691.4738939)(423.01568086,691.47889389)(422.95568337,691.49889374)
\curveto(422.85568102,691.51889385)(422.76068112,691.53889383)(422.67068337,691.55889374)
\curveto(422.5806813,691.5688938)(422.49568138,691.59389378)(422.41568337,691.63389374)
\curveto(422.12568175,691.74389363)(421.875682,691.88389349)(421.66568337,692.05389374)
\curveto(421.46568241,692.23389314)(421.30568257,692.4688929)(421.18568337,692.75889374)
\curveto(421.15568272,692.82889254)(421.12568275,692.90389247)(421.09568337,692.98389374)
\curveto(421.0756828,693.06389231)(421.05568282,693.14889222)(421.03568337,693.23889374)
\curveto(421.01568286,693.28889208)(421.00568287,693.33889203)(421.00568337,693.38889374)
\curveto(421.01568286,693.43889193)(421.01568286,693.48889188)(421.00568337,693.53889374)
\curveto(420.99568288,693.5688918)(420.98568289,693.62889174)(420.97568337,693.71889374)
\curveto(420.9756829,693.81889155)(420.9806829,693.88889148)(420.99068337,693.92889374)
\curveto(421.01068287,694.02889134)(421.02068286,694.11389126)(421.02068337,694.18389374)
\lineto(421.11068337,694.51389374)
\curveto(421.14068274,694.63389074)(421.1806827,694.73889063)(421.23068337,694.82889374)
\curveto(421.40068248,695.11889025)(421.59568228,695.33889003)(421.81568337,695.48889374)
\curveto(422.03568184,695.63888973)(422.31568156,695.7688896)(422.65568337,695.87889374)
\curveto(422.78568109,695.92888944)(422.92068096,695.96388941)(423.06068337,695.98389374)
\curveto(423.20068068,696.00388937)(423.34068054,696.02888934)(423.48068337,696.05889374)
\curveto(423.56068032,696.07888929)(423.64568023,696.08888928)(423.73568337,696.08889374)
\curveto(423.82568005,696.09888927)(423.91567996,696.11388926)(424.00568337,696.13389374)
\curveto(424.0756798,696.15388922)(424.14567973,696.15888921)(424.21568337,696.14889374)
\curveto(424.28567959,696.14888922)(424.36067952,696.15888921)(424.44068337,696.17889374)
\curveto(424.51067937,696.19888917)(424.5806793,696.20888916)(424.65068337,696.20889374)
\curveto(424.72067916,696.20888916)(424.79567908,696.21888915)(424.87568337,696.23889374)
\curveto(425.08567879,696.28888908)(425.2756786,696.32888904)(425.44568337,696.35889374)
\curveto(425.62567825,696.39888897)(425.78567809,696.48888888)(425.92568337,696.62889374)
\curveto(426.01567786,696.71888865)(426.0756778,696.81888855)(426.10568337,696.92889374)
\curveto(426.11567776,696.95888841)(426.11567776,696.98388839)(426.10568337,697.00389374)
\curveto(426.10567777,697.02388835)(426.11067777,697.04388833)(426.12068337,697.06389374)
\curveto(426.13067775,697.08388829)(426.13567774,697.11388826)(426.13568337,697.15389374)
\lineto(426.13568337,697.24389374)
\lineto(426.10568337,697.36389374)
\curveto(426.10567777,697.40388797)(426.10067778,697.43888793)(426.09068337,697.46889374)
\curveto(425.99067789,697.7688876)(425.7806781,697.9738874)(425.46068337,698.08389374)
\curveto(425.37067851,698.11388726)(425.26067862,698.13388724)(425.13068337,698.14389374)
\curveto(425.01067887,698.16388721)(424.88567899,698.1688872)(424.75568337,698.15889374)
\curveto(424.62567925,698.15888721)(424.50067938,698.14888722)(424.38068337,698.12889374)
\curveto(424.26067962,698.10888726)(424.15567972,698.08388729)(424.06568337,698.05389374)
\curveto(424.00567987,698.03388734)(423.94567993,698.00388737)(423.88568337,697.96389374)
\curveto(423.83568004,697.93388744)(423.78568009,697.89888747)(423.73568337,697.85889374)
\curveto(423.68568019,697.81888755)(423.63068025,697.76388761)(423.57068337,697.69389374)
\curveto(423.52068036,697.62388775)(423.48568039,697.55888781)(423.46568337,697.49889374)
\curveto(423.41568046,697.39888797)(423.37068051,697.30388807)(423.33068337,697.21389374)
\curveto(423.30068058,697.12388825)(423.23068065,697.06388831)(423.12068337,697.03389374)
\curveto(423.04068084,697.01388836)(422.95568092,697.00388837)(422.86568337,697.00389374)
\lineto(422.59568337,697.00389374)
\lineto(422.02568337,697.00389374)
\curveto(421.9756819,697.00388837)(421.92568195,696.99888837)(421.87568337,696.98889374)
\curveto(421.82568205,696.98888838)(421.7806821,696.99388838)(421.74068337,697.00389374)
\lineto(421.60568337,697.00389374)
\curveto(421.58568229,697.01388836)(421.56068232,697.01888835)(421.53068337,697.01889374)
\curveto(421.50068238,697.01888835)(421.4756824,697.02888834)(421.45568337,697.04889374)
\curveto(421.3756825,697.0688883)(421.32068256,697.13388824)(421.29068337,697.24389374)
\curveto(421.2806826,697.29388808)(421.2806826,697.34388803)(421.29068337,697.39389374)
\curveto(421.30068258,697.44388793)(421.31068257,697.48888788)(421.32068337,697.52889374)
\curveto(421.35068253,697.63888773)(421.3806825,697.73888763)(421.41068337,697.82889374)
\curveto(421.45068243,697.92888744)(421.49568238,698.01888735)(421.54568337,698.09889374)
\lineto(421.63568337,698.24889374)
\lineto(421.72568337,698.39889374)
\curveto(421.80568207,698.50888686)(421.90568197,698.61388676)(422.02568337,698.71389374)
\curveto(422.04568183,698.72388665)(422.0756818,698.74888662)(422.11568337,698.78889374)
\curveto(422.16568171,698.82888654)(422.21068167,698.86388651)(422.25068337,698.89389374)
\curveto(422.29068159,698.92388645)(422.33568154,698.95388642)(422.38568337,698.98389374)
\curveto(422.55568132,699.09388628)(422.73568114,699.17888619)(422.92568337,699.23889374)
\curveto(423.11568076,699.30888606)(423.31068057,699.373886)(423.51068337,699.43389374)
\curveto(423.63068025,699.46388591)(423.75568012,699.48388589)(423.88568337,699.49389374)
\curveto(424.01567986,699.50388587)(424.14567973,699.52388585)(424.27568337,699.55389374)
\curveto(424.31567956,699.56388581)(424.3756795,699.56388581)(424.45568337,699.55389374)
\curveto(424.54567933,699.54388583)(424.60067928,699.54888582)(424.62068337,699.56889374)
\curveto(425.03067885,699.57888579)(425.42067846,699.56388581)(425.79068337,699.52389374)
\curveto(426.17067771,699.48388589)(426.51067737,699.40888596)(426.81068337,699.29889374)
\curveto(427.12067676,699.18888618)(427.38567649,699.03888633)(427.60568337,698.84889374)
\curveto(427.82567605,698.6688867)(427.99567588,698.43388694)(428.11568337,698.14389374)
\curveto(428.18567569,697.9738874)(428.22567565,697.77888759)(428.23568337,697.55889374)
\curveto(428.24567563,697.33888803)(428.25067563,697.11388826)(428.25068337,696.88389374)
\lineto(428.25068337,693.53889374)
\lineto(428.25068337,692.95389374)
\curveto(428.25067563,692.76389261)(428.27067561,692.58889278)(428.31068337,692.42889374)
\curveto(428.32067556,692.39889297)(428.32567555,692.36389301)(428.32568337,692.32389374)
\curveto(428.32567555,692.29389308)(428.33067555,692.26389311)(428.34068337,692.23389374)
\moveto(426.13568337,694.54389374)
\curveto(426.14567773,694.59389078)(426.15067773,694.64889072)(426.15068337,694.70889374)
\curveto(426.15067773,694.77889059)(426.14567773,694.83889053)(426.13568337,694.88889374)
\curveto(426.11567776,694.94889042)(426.10567777,695.00389037)(426.10568337,695.05389374)
\curveto(426.10567777,695.10389027)(426.08567779,695.14389023)(426.04568337,695.17389374)
\curveto(425.99567788,695.21389016)(425.92067796,695.23389014)(425.82068337,695.23389374)
\curveto(425.7806781,695.22389015)(425.74567813,695.21389016)(425.71568337,695.20389374)
\curveto(425.68567819,695.20389017)(425.65067823,695.19889017)(425.61068337,695.18889374)
\curveto(425.54067834,695.1688902)(425.46567841,695.15389022)(425.38568337,695.14389374)
\curveto(425.30567857,695.13389024)(425.22567865,695.11889025)(425.14568337,695.09889374)
\curveto(425.11567876,695.08889028)(425.07067881,695.08389029)(425.01068337,695.08389374)
\curveto(424.880679,695.05389032)(424.75067913,695.03389034)(424.62068337,695.02389374)
\curveto(424.49067939,695.01389036)(424.36567951,694.98889038)(424.24568337,694.94889374)
\curveto(424.16567971,694.92889044)(424.09067979,694.90889046)(424.02068337,694.88889374)
\curveto(423.95067993,694.87889049)(423.88068,694.85889051)(423.81068337,694.82889374)
\curveto(423.60068028,694.73889063)(423.42068046,694.60389077)(423.27068337,694.42389374)
\curveto(423.13068075,694.24389113)(423.0806808,693.99389138)(423.12068337,693.67389374)
\curveto(423.14068074,693.50389187)(423.19568068,693.36389201)(423.28568337,693.25389374)
\curveto(423.35568052,693.14389223)(423.46068042,693.05389232)(423.60068337,692.98389374)
\curveto(423.74068014,692.92389245)(423.89067999,692.87889249)(424.05068337,692.84889374)
\curveto(424.22067966,692.81889255)(424.39567948,692.80889256)(424.57568337,692.81889374)
\curveto(424.76567911,692.83889253)(424.94067894,692.8738925)(425.10068337,692.92389374)
\curveto(425.36067852,693.00389237)(425.56567831,693.12889224)(425.71568337,693.29889374)
\curveto(425.86567801,693.47889189)(425.9806779,693.69889167)(426.06068337,693.95889374)
\curveto(426.0806778,694.02889134)(426.09067779,694.09889127)(426.09068337,694.16889374)
\curveto(426.10067778,694.24889112)(426.11567776,694.32889104)(426.13568337,694.40889374)
\lineto(426.13568337,694.54389374)
}
}
{
\newrgbcolor{curcolor}{0 0 0}
\pscustom[linestyle=none,fillstyle=solid,fillcolor=curcolor]
{
\newpath
\moveto(434.32896462,699.56889374)
\curveto(434.4389593,699.5688858)(434.53395921,699.55888581)(434.61396462,699.53889374)
\curveto(434.70395904,699.51888585)(434.77395897,699.4738859)(434.82396462,699.40389374)
\curveto(434.88395886,699.32388605)(434.91395883,699.18388619)(434.91396462,698.98389374)
\lineto(434.91396462,698.47389374)
\lineto(434.91396462,698.09889374)
\curveto(434.92395882,697.95888741)(434.90895883,697.84888752)(434.86896462,697.76889374)
\curveto(434.82895891,697.69888767)(434.76895897,697.65388772)(434.68896462,697.63389374)
\curveto(434.61895912,697.61388776)(434.53395921,697.60388777)(434.43396462,697.60389374)
\curveto(434.3439594,697.60388777)(434.2439595,697.60888776)(434.13396462,697.61889374)
\curveto(434.03395971,697.62888774)(433.9389598,697.62388775)(433.84896462,697.60389374)
\curveto(433.77895996,697.58388779)(433.70896003,697.5688878)(433.63896462,697.55889374)
\curveto(433.56896017,697.55888781)(433.50396024,697.54888782)(433.44396462,697.52889374)
\curveto(433.28396046,697.47888789)(433.12396062,697.40388797)(432.96396462,697.30389374)
\curveto(432.80396094,697.21388816)(432.67896106,697.10888826)(432.58896462,696.98889374)
\curveto(432.5389612,696.90888846)(432.48396126,696.82388855)(432.42396462,696.73389374)
\curveto(432.37396137,696.65388872)(432.32396142,696.5688888)(432.27396462,696.47889374)
\curveto(432.2439615,696.39888897)(432.21396153,696.31388906)(432.18396462,696.22389374)
\lineto(432.12396462,695.98389374)
\curveto(432.10396164,695.91388946)(432.09396165,695.83888953)(432.09396462,695.75889374)
\curveto(432.09396165,695.68888968)(432.08396166,695.61888975)(432.06396462,695.54889374)
\curveto(432.05396169,695.50888986)(432.04896169,695.4688899)(432.04896462,695.42889374)
\curveto(432.05896168,695.39888997)(432.05896168,695.36889)(432.04896462,695.33889374)
\lineto(432.04896462,695.09889374)
\curveto(432.02896171,695.02889034)(432.02396172,694.94889042)(432.03396462,694.85889374)
\curveto(432.0439617,694.77889059)(432.04896169,694.69889067)(432.04896462,694.61889374)
\lineto(432.04896462,693.65889374)
\lineto(432.04896462,692.38389374)
\curveto(432.04896169,692.25389312)(432.0439617,692.13389324)(432.03396462,692.02389374)
\curveto(432.02396172,691.91389346)(431.99396175,691.82389355)(431.94396462,691.75389374)
\curveto(431.92396182,691.72389365)(431.88896185,691.69889367)(431.83896462,691.67889374)
\curveto(431.79896194,691.6688937)(431.75396199,691.65889371)(431.70396462,691.64889374)
\lineto(431.62896462,691.64889374)
\curveto(431.57896216,691.63889373)(431.52396222,691.63389374)(431.46396462,691.63389374)
\lineto(431.29896462,691.63389374)
\lineto(430.65396462,691.63389374)
\curveto(430.59396315,691.64389373)(430.52896321,691.64889372)(430.45896462,691.64889374)
\lineto(430.26396462,691.64889374)
\curveto(430.21396353,691.6688937)(430.16396358,691.68389369)(430.11396462,691.69389374)
\curveto(430.06396368,691.71389366)(430.02896371,691.74889362)(430.00896462,691.79889374)
\curveto(429.96896377,691.84889352)(429.9439638,691.91889345)(429.93396462,692.00889374)
\lineto(429.93396462,692.30889374)
\lineto(429.93396462,693.32889374)
\lineto(429.93396462,697.55889374)
\lineto(429.93396462,698.66889374)
\lineto(429.93396462,698.95389374)
\curveto(429.93396381,699.05388632)(429.95396379,699.13388624)(429.99396462,699.19389374)
\curveto(430.0439637,699.2738861)(430.11896362,699.32388605)(430.21896462,699.34389374)
\curveto(430.31896342,699.36388601)(430.4389633,699.373886)(430.57896462,699.37389374)
\lineto(431.34396462,699.37389374)
\curveto(431.46396228,699.373886)(431.56896217,699.36388601)(431.65896462,699.34389374)
\curveto(431.74896199,699.33388604)(431.81896192,699.28888608)(431.86896462,699.20889374)
\curveto(431.89896184,699.15888621)(431.91396183,699.08888628)(431.91396462,698.99889374)
\lineto(431.94396462,698.72889374)
\curveto(431.95396179,698.64888672)(431.96896177,698.5738868)(431.98896462,698.50389374)
\curveto(432.01896172,698.43388694)(432.06896167,698.39888697)(432.13896462,698.39889374)
\curveto(432.15896158,698.41888695)(432.17896156,698.42888694)(432.19896462,698.42889374)
\curveto(432.21896152,698.42888694)(432.2389615,698.43888693)(432.25896462,698.45889374)
\curveto(432.31896142,698.50888686)(432.36896137,698.56388681)(432.40896462,698.62389374)
\curveto(432.45896128,698.69388668)(432.51896122,698.75388662)(432.58896462,698.80389374)
\curveto(432.62896111,698.83388654)(432.66396108,698.86388651)(432.69396462,698.89389374)
\curveto(432.72396102,698.93388644)(432.75896098,698.9688864)(432.79896462,698.99889374)
\lineto(433.06896462,699.17889374)
\curveto(433.16896057,699.23888613)(433.26896047,699.29388608)(433.36896462,699.34389374)
\curveto(433.46896027,699.38388599)(433.56896017,699.41888595)(433.66896462,699.44889374)
\lineto(433.99896462,699.53889374)
\curveto(434.02895971,699.54888582)(434.08395966,699.54888582)(434.16396462,699.53889374)
\curveto(434.25395949,699.53888583)(434.30895943,699.54888582)(434.32896462,699.56889374)
}
}
{
\newrgbcolor{curcolor}{0 0 0}
\pscustom[linestyle=none,fillstyle=solid,fillcolor=curcolor]
{
\newpath
\moveto(437.83404274,702.22389374)
\curveto(437.90403979,702.14388323)(437.93903976,702.02388335)(437.93904274,701.86389374)
\lineto(437.93904274,701.39889374)
\lineto(437.93904274,700.99389374)
\curveto(437.93903976,700.85388452)(437.90403979,700.75888461)(437.83404274,700.70889374)
\curveto(437.77403992,700.65888471)(437.69404,700.62888474)(437.59404274,700.61889374)
\curveto(437.50404019,700.60888476)(437.40404029,700.60388477)(437.29404274,700.60389374)
\lineto(436.45404274,700.60389374)
\curveto(436.34404135,700.60388477)(436.24404145,700.60888476)(436.15404274,700.61889374)
\curveto(436.07404162,700.62888474)(436.00404169,700.65888471)(435.94404274,700.70889374)
\curveto(435.90404179,700.73888463)(435.87404182,700.79388458)(435.85404274,700.87389374)
\curveto(435.84404185,700.96388441)(435.83404186,701.05888431)(435.82404274,701.15889374)
\lineto(435.82404274,701.48889374)
\curveto(435.83404186,701.59888377)(435.83904186,701.69388368)(435.83904274,701.77389374)
\lineto(435.83904274,701.98389374)
\curveto(435.84904185,702.05388332)(435.86904183,702.11388326)(435.89904274,702.16389374)
\curveto(435.91904178,702.20388317)(435.94404175,702.23388314)(435.97404274,702.25389374)
\lineto(436.09404274,702.31389374)
\curveto(436.11404158,702.31388306)(436.13904156,702.31388306)(436.16904274,702.31389374)
\curveto(436.1990415,702.32388305)(436.22404147,702.32888304)(436.24404274,702.32889374)
\lineto(437.33904274,702.32889374)
\curveto(437.43904026,702.32888304)(437.53404016,702.32388305)(437.62404274,702.31389374)
\curveto(437.71403998,702.30388307)(437.78403991,702.2738831)(437.83404274,702.22389374)
\moveto(437.93904274,692.45889374)
\curveto(437.93903976,692.25889311)(437.93403976,692.08889328)(437.92404274,691.94889374)
\curveto(437.91403978,691.80889356)(437.82403987,691.71389366)(437.65404274,691.66389374)
\curveto(437.5940401,691.64389373)(437.52904017,691.63389374)(437.45904274,691.63389374)
\curveto(437.38904031,691.64389373)(437.31404038,691.64889372)(437.23404274,691.64889374)
\lineto(436.39404274,691.64889374)
\curveto(436.30404139,691.64889372)(436.21404148,691.65389372)(436.12404274,691.66389374)
\curveto(436.04404165,691.6738937)(435.98404171,691.70389367)(435.94404274,691.75389374)
\curveto(435.88404181,691.82389355)(435.84904185,691.90889346)(435.83904274,692.00889374)
\lineto(435.83904274,692.35389374)
\lineto(435.83904274,698.68389374)
\lineto(435.83904274,698.98389374)
\curveto(435.83904186,699.08388629)(435.85904184,699.16388621)(435.89904274,699.22389374)
\curveto(435.95904174,699.29388608)(436.04404165,699.33888603)(436.15404274,699.35889374)
\curveto(436.17404152,699.368886)(436.1990415,699.368886)(436.22904274,699.35889374)
\curveto(436.26904143,699.35888601)(436.2990414,699.36388601)(436.31904274,699.37389374)
\lineto(437.06904274,699.37389374)
\lineto(437.26404274,699.37389374)
\curveto(437.34404035,699.38388599)(437.40904029,699.38388599)(437.45904274,699.37389374)
\lineto(437.57904274,699.37389374)
\curveto(437.63904006,699.35388602)(437.69404,699.33888603)(437.74404274,699.32889374)
\curveto(437.7940399,699.31888605)(437.83403986,699.28888608)(437.86404274,699.23889374)
\curveto(437.90403979,699.18888618)(437.92403977,699.11888625)(437.92404274,699.02889374)
\curveto(437.93403976,698.93888643)(437.93903976,698.84388653)(437.93904274,698.74389374)
\lineto(437.93904274,692.45889374)
}
}
{
\newrgbcolor{curcolor}{0 0 0}
\pscustom[linestyle=none,fillstyle=solid,fillcolor=curcolor]
{
\newpath
\moveto(447.37123024,695.81889374)
\curveto(447.39122167,695.75888961)(447.40122166,695.6738897)(447.40123024,695.56389374)
\curveto(447.40122166,695.45388992)(447.39122167,695.36889)(447.37123024,695.30889374)
\lineto(447.37123024,695.15889374)
\curveto(447.35122171,695.07889029)(447.34122172,694.99889037)(447.34123024,694.91889374)
\curveto(447.35122171,694.83889053)(447.34622172,694.75889061)(447.32623024,694.67889374)
\curveto(447.30622176,694.60889076)(447.29122177,694.54389083)(447.28123024,694.48389374)
\curveto(447.27122179,694.42389095)(447.2612218,694.35889101)(447.25123024,694.28889374)
\curveto(447.21122185,694.17889119)(447.17622189,694.06389131)(447.14623024,693.94389374)
\curveto(447.11622195,693.83389154)(447.07622199,693.72889164)(447.02623024,693.62889374)
\curveto(446.81622225,693.14889222)(446.54122252,692.75889261)(446.20123024,692.45889374)
\curveto(445.8612232,692.15889321)(445.45122361,691.90889346)(444.97123024,691.70889374)
\curveto(444.85122421,691.65889371)(444.72622434,691.62389375)(444.59623024,691.60389374)
\curveto(444.47622459,691.5738938)(444.35122471,691.54389383)(444.22123024,691.51389374)
\curveto(444.17122489,691.49389388)(444.11622495,691.48389389)(444.05623024,691.48389374)
\curveto(443.99622507,691.48389389)(443.94122512,691.47889389)(443.89123024,691.46889374)
\lineto(443.78623024,691.46889374)
\curveto(443.75622531,691.45889391)(443.72622534,691.45389392)(443.69623024,691.45389374)
\curveto(443.64622542,691.44389393)(443.5662255,691.43889393)(443.45623024,691.43889374)
\curveto(443.34622572,691.42889394)(443.2612258,691.43389394)(443.20123024,691.45389374)
\lineto(443.05123024,691.45389374)
\curveto(443.00122606,691.46389391)(442.94622612,691.4688939)(442.88623024,691.46889374)
\curveto(442.83622623,691.45889391)(442.78622628,691.46389391)(442.73623024,691.48389374)
\curveto(442.69622637,691.49389388)(442.65622641,691.49889387)(442.61623024,691.49889374)
\curveto(442.58622648,691.49889387)(442.54622652,691.50389387)(442.49623024,691.51389374)
\curveto(442.39622667,691.54389383)(442.29622677,691.5688938)(442.19623024,691.58889374)
\curveto(442.09622697,691.60889376)(442.00122706,691.63889373)(441.91123024,691.67889374)
\curveto(441.79122727,691.71889365)(441.67622739,691.75889361)(441.56623024,691.79889374)
\curveto(441.4662276,691.83889353)(441.3612277,691.88889348)(441.25123024,691.94889374)
\curveto(440.90122816,692.15889321)(440.60122846,692.40389297)(440.35123024,692.68389374)
\curveto(440.10122896,692.96389241)(439.89122917,693.29889207)(439.72123024,693.68889374)
\curveto(439.67122939,693.77889159)(439.63122943,693.8738915)(439.60123024,693.97389374)
\curveto(439.58122948,694.0738913)(439.55622951,694.17889119)(439.52623024,694.28889374)
\curveto(439.50622956,694.33889103)(439.49622957,694.38389099)(439.49623024,694.42389374)
\curveto(439.49622957,694.46389091)(439.48622958,694.50889086)(439.46623024,694.55889374)
\curveto(439.44622962,694.63889073)(439.43622963,694.71889065)(439.43623024,694.79889374)
\curveto(439.43622963,694.88889048)(439.42622964,694.9738904)(439.40623024,695.05389374)
\curveto(439.39622967,695.10389027)(439.39122967,695.14889022)(439.39123024,695.18889374)
\lineto(439.39123024,695.32389374)
\curveto(439.37122969,695.38388999)(439.3612297,695.4688899)(439.36123024,695.57889374)
\curveto(439.37122969,695.68888968)(439.38622968,695.7738896)(439.40623024,695.83389374)
\lineto(439.40623024,695.93889374)
\curveto(439.41622965,695.98888938)(439.41622965,696.03888933)(439.40623024,696.08889374)
\curveto(439.40622966,696.14888922)(439.41622965,696.20388917)(439.43623024,696.25389374)
\curveto(439.44622962,696.30388907)(439.45122961,696.34888902)(439.45123024,696.38889374)
\curveto(439.45122961,696.43888893)(439.4612296,696.48888888)(439.48123024,696.53889374)
\curveto(439.52122954,696.6688887)(439.55622951,696.79388858)(439.58623024,696.91389374)
\curveto(439.61622945,697.04388833)(439.65622941,697.1688882)(439.70623024,697.28889374)
\curveto(439.88622918,697.69888767)(440.10122896,698.03888733)(440.35123024,698.30889374)
\curveto(440.60122846,698.58888678)(440.90622816,698.84388653)(441.26623024,699.07389374)
\curveto(441.3662277,699.12388625)(441.47122759,699.1688862)(441.58123024,699.20889374)
\curveto(441.69122737,699.24888612)(441.80122726,699.29388608)(441.91123024,699.34389374)
\curveto(442.04122702,699.39388598)(442.17622689,699.42888594)(442.31623024,699.44889374)
\curveto(442.45622661,699.4688859)(442.60122646,699.49888587)(442.75123024,699.53889374)
\curveto(442.83122623,699.54888582)(442.90622616,699.55388582)(442.97623024,699.55389374)
\curveto(443.04622602,699.55388582)(443.11622595,699.55888581)(443.18623024,699.56889374)
\curveto(443.7662253,699.57888579)(444.2662248,699.51888585)(444.68623024,699.38889374)
\curveto(445.11622395,699.25888611)(445.49622357,699.07888629)(445.82623024,698.84889374)
\curveto(445.93622313,698.7688866)(446.04622302,698.67888669)(446.15623024,698.57889374)
\curveto(446.27622279,698.48888688)(446.37622269,698.38888698)(446.45623024,698.27889374)
\curveto(446.53622253,698.17888719)(446.60622246,698.07888729)(446.66623024,697.97889374)
\curveto(446.73622233,697.87888749)(446.80622226,697.7738876)(446.87623024,697.66389374)
\curveto(446.94622212,697.55388782)(447.00122206,697.43388794)(447.04123024,697.30389374)
\curveto(447.08122198,697.18388819)(447.12622194,697.05388832)(447.17623024,696.91389374)
\curveto(447.20622186,696.83388854)(447.23122183,696.74888862)(447.25123024,696.65889374)
\lineto(447.31123024,696.38889374)
\curveto(447.32122174,696.34888902)(447.32622174,696.30888906)(447.32623024,696.26889374)
\curveto(447.32622174,696.22888914)(447.33122173,696.18888918)(447.34123024,696.14889374)
\curveto(447.3612217,696.09888927)(447.3662217,696.04388933)(447.35623024,695.98389374)
\curveto(447.34622172,695.92388945)(447.35122171,695.8688895)(447.37123024,695.81889374)
\moveto(445.27123024,695.27889374)
\curveto(445.28122378,695.32889004)(445.28622378,695.39888997)(445.28623024,695.48889374)
\curveto(445.28622378,695.58888978)(445.28122378,695.66388971)(445.27123024,695.71389374)
\lineto(445.27123024,695.83389374)
\curveto(445.25122381,695.88388949)(445.24122382,695.93888943)(445.24123024,695.99889374)
\curveto(445.24122382,696.05888931)(445.23622383,696.11388926)(445.22623024,696.16389374)
\curveto(445.22622384,696.20388917)(445.22122384,696.23388914)(445.21123024,696.25389374)
\lineto(445.15123024,696.49389374)
\curveto(445.14122392,696.58388879)(445.12122394,696.6688887)(445.09123024,696.74889374)
\curveto(444.98122408,697.00888836)(444.85122421,697.22888814)(444.70123024,697.40889374)
\curveto(444.55122451,697.59888777)(444.35122471,697.74888762)(444.10123024,697.85889374)
\curveto(444.04122502,697.87888749)(443.98122508,697.89388748)(443.92123024,697.90389374)
\curveto(443.8612252,697.92388745)(443.79622527,697.94388743)(443.72623024,697.96389374)
\curveto(443.64622542,697.98388739)(443.5612255,697.98888738)(443.47123024,697.97889374)
\lineto(443.20123024,697.97889374)
\curveto(443.17122589,697.95888741)(443.13622593,697.94888742)(443.09623024,697.94889374)
\curveto(443.05622601,697.95888741)(443.02122604,697.95888741)(442.99123024,697.94889374)
\lineto(442.78123024,697.88889374)
\curveto(442.72122634,697.87888749)(442.6662264,697.85888751)(442.61623024,697.82889374)
\curveto(442.3662267,697.71888765)(442.1612269,697.55888781)(442.00123024,697.34889374)
\curveto(441.85122721,697.14888822)(441.73122733,696.91388846)(441.64123024,696.64389374)
\curveto(441.61122745,696.54388883)(441.58622748,696.43888893)(441.56623024,696.32889374)
\curveto(441.55622751,696.21888915)(441.54122752,696.10888926)(441.52123024,695.99889374)
\curveto(441.51122755,695.94888942)(441.50622756,695.89888947)(441.50623024,695.84889374)
\lineto(441.50623024,695.69889374)
\curveto(441.48622758,695.62888974)(441.47622759,695.52388985)(441.47623024,695.38389374)
\curveto(441.48622758,695.24389013)(441.50122756,695.13889023)(441.52123024,695.06889374)
\lineto(441.52123024,694.93389374)
\curveto(441.54122752,694.85389052)(441.55622751,694.7738906)(441.56623024,694.69389374)
\curveto(441.57622749,694.62389075)(441.59122747,694.54889082)(441.61123024,694.46889374)
\curveto(441.71122735,694.1688912)(441.81622725,693.92389145)(441.92623024,693.73389374)
\curveto(442.04622702,693.55389182)(442.23122683,693.38889198)(442.48123024,693.23889374)
\curveto(442.55122651,693.18889218)(442.62622644,693.14889222)(442.70623024,693.11889374)
\curveto(442.79622627,693.08889228)(442.88622618,693.06389231)(442.97623024,693.04389374)
\curveto(443.01622605,693.03389234)(443.05122601,693.02889234)(443.08123024,693.02889374)
\curveto(443.11122595,693.03889233)(443.14622592,693.03889233)(443.18623024,693.02889374)
\lineto(443.30623024,692.99889374)
\curveto(443.35622571,692.99889237)(443.40122566,693.00389237)(443.44123024,693.01389374)
\lineto(443.56123024,693.01389374)
\curveto(443.64122542,693.03389234)(443.72122534,693.04889232)(443.80123024,693.05889374)
\curveto(443.88122518,693.0688923)(443.95622511,693.08889228)(444.02623024,693.11889374)
\curveto(444.28622478,693.21889215)(444.49622457,693.35389202)(444.65623024,693.52389374)
\curveto(444.81622425,693.69389168)(444.95122411,693.90389147)(445.06123024,694.15389374)
\curveto(445.10122396,694.25389112)(445.13122393,694.35389102)(445.15123024,694.45389374)
\curveto(445.17122389,694.55389082)(445.19622387,694.65889071)(445.22623024,694.76889374)
\curveto(445.23622383,694.80889056)(445.24122382,694.84389053)(445.24123024,694.87389374)
\curveto(445.24122382,694.91389046)(445.24622382,694.95389042)(445.25623024,694.99389374)
\lineto(445.25623024,695.12889374)
\curveto(445.25622381,695.17889019)(445.2612238,695.22889014)(445.27123024,695.27889374)
}
}
{
\newrgbcolor{curcolor}{0 0 0}
\pscustom[linestyle=none,fillstyle=solid,fillcolor=curcolor]
{
\newpath
\moveto(451.74115212,699.58389374)
\curveto(452.49114762,699.60388577)(453.14114697,699.51888585)(453.69115212,699.32889374)
\curveto(454.25114586,699.14888622)(454.67614543,698.83388654)(454.96615212,698.38389374)
\curveto(455.03614507,698.2738871)(455.09614501,698.15888721)(455.14615212,698.03889374)
\curveto(455.2061449,697.92888744)(455.25614485,697.80388757)(455.29615212,697.66389374)
\curveto(455.31614479,697.60388777)(455.32614478,697.53888783)(455.32615212,697.46889374)
\curveto(455.32614478,697.39888797)(455.31614479,697.33888803)(455.29615212,697.28889374)
\curveto(455.25614485,697.22888814)(455.20114491,697.18888818)(455.13115212,697.16889374)
\curveto(455.08114503,697.14888822)(455.02114509,697.13888823)(454.95115212,697.13889374)
\lineto(454.74115212,697.13889374)
\lineto(454.08115212,697.13889374)
\curveto(454.0111461,697.13888823)(453.94114617,697.13388824)(453.87115212,697.12389374)
\curveto(453.80114631,697.12388825)(453.73614637,697.13388824)(453.67615212,697.15389374)
\curveto(453.57614653,697.1738882)(453.50114661,697.21388816)(453.45115212,697.27389374)
\curveto(453.40114671,697.33388804)(453.35614675,697.39388798)(453.31615212,697.45389374)
\lineto(453.19615212,697.66389374)
\curveto(453.16614694,697.74388763)(453.11614699,697.80888756)(453.04615212,697.85889374)
\curveto(452.94614716,697.93888743)(452.84614726,697.99888737)(452.74615212,698.03889374)
\curveto(452.65614745,698.07888729)(452.54114757,698.11388726)(452.40115212,698.14389374)
\curveto(452.33114778,698.16388721)(452.22614788,698.17888719)(452.08615212,698.18889374)
\curveto(451.95614815,698.19888717)(451.85614825,698.19388718)(451.78615212,698.17389374)
\lineto(451.68115212,698.17389374)
\lineto(451.53115212,698.14389374)
\curveto(451.49114862,698.14388723)(451.44614866,698.13888723)(451.39615212,698.12889374)
\curveto(451.22614888,698.07888729)(451.08614902,698.00888736)(450.97615212,697.91889374)
\curveto(450.87614923,697.83888753)(450.8061493,697.71388766)(450.76615212,697.54389374)
\curveto(450.74614936,697.4738879)(450.74614936,697.40888796)(450.76615212,697.34889374)
\curveto(450.78614932,697.28888808)(450.8061493,697.23888813)(450.82615212,697.19889374)
\curveto(450.89614921,697.07888829)(450.97614913,696.98388839)(451.06615212,696.91389374)
\curveto(451.16614894,696.84388853)(451.28114883,696.78388859)(451.41115212,696.73389374)
\curveto(451.60114851,696.65388872)(451.8061483,696.58388879)(452.02615212,696.52389374)
\lineto(452.71615212,696.37389374)
\curveto(452.95614715,696.33388904)(453.18614692,696.28388909)(453.40615212,696.22389374)
\curveto(453.63614647,696.1738892)(453.85114626,696.10888926)(454.05115212,696.02889374)
\curveto(454.14114597,695.98888938)(454.22614588,695.95388942)(454.30615212,695.92389374)
\curveto(454.39614571,695.90388947)(454.48114563,695.8688895)(454.56115212,695.81889374)
\curveto(454.75114536,695.69888967)(454.92114519,695.5688898)(455.07115212,695.42889374)
\curveto(455.23114488,695.28889008)(455.35614475,695.11389026)(455.44615212,694.90389374)
\curveto(455.47614463,694.83389054)(455.50114461,694.76389061)(455.52115212,694.69389374)
\curveto(455.54114457,694.62389075)(455.56114455,694.54889082)(455.58115212,694.46889374)
\curveto(455.59114452,694.40889096)(455.59614451,694.31389106)(455.59615212,694.18389374)
\curveto(455.6061445,694.06389131)(455.6061445,693.9688914)(455.59615212,693.89889374)
\lineto(455.59615212,693.82389374)
\curveto(455.57614453,693.76389161)(455.56114455,693.70389167)(455.55115212,693.64389374)
\curveto(455.55114456,693.59389178)(455.54614456,693.54389183)(455.53615212,693.49389374)
\curveto(455.46614464,693.19389218)(455.35614475,692.92889244)(455.20615212,692.69889374)
\curveto(455.04614506,692.45889291)(454.85114526,692.26389311)(454.62115212,692.11389374)
\curveto(454.39114572,691.96389341)(454.13114598,691.83389354)(453.84115212,691.72389374)
\curveto(453.73114638,691.6738937)(453.6111465,691.63889373)(453.48115212,691.61889374)
\curveto(453.36114675,691.59889377)(453.24114687,691.5738938)(453.12115212,691.54389374)
\curveto(453.03114708,691.52389385)(452.93614717,691.51389386)(452.83615212,691.51389374)
\curveto(452.74614736,691.50389387)(452.65614745,691.48889388)(452.56615212,691.46889374)
\lineto(452.29615212,691.46889374)
\curveto(452.23614787,691.44889392)(452.13114798,691.43889393)(451.98115212,691.43889374)
\curveto(451.84114827,691.43889393)(451.74114837,691.44889392)(451.68115212,691.46889374)
\curveto(451.65114846,691.4688939)(451.61614849,691.4738939)(451.57615212,691.48389374)
\lineto(451.47115212,691.48389374)
\curveto(451.35114876,691.50389387)(451.23114888,691.51889385)(451.11115212,691.52889374)
\curveto(450.99114912,691.53889383)(450.87614923,691.55889381)(450.76615212,691.58889374)
\curveto(450.37614973,691.69889367)(450.03115008,691.82389355)(449.73115212,691.96389374)
\curveto(449.43115068,692.11389326)(449.17615093,692.33389304)(448.96615212,692.62389374)
\curveto(448.82615128,692.81389256)(448.7061514,693.03389234)(448.60615212,693.28389374)
\curveto(448.58615152,693.34389203)(448.56615154,693.42389195)(448.54615212,693.52389374)
\curveto(448.52615158,693.5738918)(448.5111516,693.64389173)(448.50115212,693.73389374)
\curveto(448.49115162,693.82389155)(448.49615161,693.89889147)(448.51615212,693.95889374)
\curveto(448.54615156,694.02889134)(448.59615151,694.07889129)(448.66615212,694.10889374)
\curveto(448.71615139,694.12889124)(448.77615133,694.13889123)(448.84615212,694.13889374)
\lineto(449.07115212,694.13889374)
\lineto(449.77615212,694.13889374)
\lineto(450.01615212,694.13889374)
\curveto(450.09615001,694.13889123)(450.16614994,694.12889124)(450.22615212,694.10889374)
\curveto(450.33614977,694.0688913)(450.4061497,694.00389137)(450.43615212,693.91389374)
\curveto(450.47614963,693.82389155)(450.52114959,693.72889164)(450.57115212,693.62889374)
\curveto(450.59114952,693.57889179)(450.62614948,693.51389186)(450.67615212,693.43389374)
\curveto(450.73614937,693.35389202)(450.78614932,693.30389207)(450.82615212,693.28389374)
\curveto(450.94614916,693.18389219)(451.06114905,693.10389227)(451.17115212,693.04389374)
\curveto(451.28114883,692.99389238)(451.42114869,692.94389243)(451.59115212,692.89389374)
\curveto(451.64114847,692.8738925)(451.69114842,692.86389251)(451.74115212,692.86389374)
\curveto(451.79114832,692.8738925)(451.84114827,692.8738925)(451.89115212,692.86389374)
\curveto(451.97114814,692.84389253)(452.05614805,692.83389254)(452.14615212,692.83389374)
\curveto(452.24614786,692.84389253)(452.33114778,692.85889251)(452.40115212,692.87889374)
\curveto(452.45114766,692.88889248)(452.49614761,692.89389248)(452.53615212,692.89389374)
\curveto(452.58614752,692.89389248)(452.63614747,692.90389247)(452.68615212,692.92389374)
\curveto(452.82614728,692.9738924)(452.95114716,693.03389234)(453.06115212,693.10389374)
\curveto(453.18114693,693.1738922)(453.27614683,693.26389211)(453.34615212,693.37389374)
\curveto(453.39614671,693.45389192)(453.43614667,693.57889179)(453.46615212,693.74889374)
\curveto(453.48614662,693.81889155)(453.48614662,693.88389149)(453.46615212,693.94389374)
\curveto(453.44614666,694.00389137)(453.42614668,694.05389132)(453.40615212,694.09389374)
\curveto(453.33614677,694.23389114)(453.24614686,694.33889103)(453.13615212,694.40889374)
\curveto(453.03614707,694.47889089)(452.91614719,694.54389083)(452.77615212,694.60389374)
\curveto(452.58614752,694.68389069)(452.38614772,694.74889062)(452.17615212,694.79889374)
\curveto(451.96614814,694.84889052)(451.75614835,694.90389047)(451.54615212,694.96389374)
\curveto(451.46614864,694.98389039)(451.38114873,694.99889037)(451.29115212,695.00889374)
\curveto(451.2111489,695.01889035)(451.13114898,695.03389034)(451.05115212,695.05389374)
\curveto(450.73114938,695.14389023)(450.42614968,695.22889014)(450.13615212,695.30889374)
\curveto(449.84615026,695.39888997)(449.58115053,695.52888984)(449.34115212,695.69889374)
\curveto(449.06115105,695.89888947)(448.85615125,696.1688892)(448.72615212,696.50889374)
\curveto(448.7061514,696.57888879)(448.68615142,696.6738887)(448.66615212,696.79389374)
\curveto(448.64615146,696.86388851)(448.63115148,696.94888842)(448.62115212,697.04889374)
\curveto(448.6111515,697.14888822)(448.61615149,697.23888813)(448.63615212,697.31889374)
\curveto(448.65615145,697.368888)(448.66115145,697.40888796)(448.65115212,697.43889374)
\curveto(448.64115147,697.47888789)(448.64615146,697.52388785)(448.66615212,697.57389374)
\curveto(448.68615142,697.68388769)(448.7061514,697.78388759)(448.72615212,697.87389374)
\curveto(448.75615135,697.9738874)(448.79115132,698.0688873)(448.83115212,698.15889374)
\curveto(448.96115115,698.44888692)(449.14115097,698.68388669)(449.37115212,698.86389374)
\curveto(449.60115051,699.04388633)(449.86115025,699.18888618)(450.15115212,699.29889374)
\curveto(450.26114985,699.34888602)(450.37614973,699.38388599)(450.49615212,699.40389374)
\curveto(450.61614949,699.43388594)(450.74114937,699.46388591)(450.87115212,699.49389374)
\curveto(450.93114918,699.51388586)(450.99114912,699.52388585)(451.05115212,699.52389374)
\lineto(451.23115212,699.55389374)
\curveto(451.3111488,699.56388581)(451.39614871,699.5688858)(451.48615212,699.56889374)
\curveto(451.57614853,699.5688858)(451.66114845,699.5738858)(451.74115212,699.58389374)
}
}
{
\newrgbcolor{curcolor}{0 0 0}
\pscustom[linestyle=none,fillstyle=solid,fillcolor=curcolor]
{
}
}
{
\newrgbcolor{curcolor}{0 0 0}
\pscustom[linestyle=none,fillstyle=solid,fillcolor=curcolor]
{
\newpath
\moveto(468.54794899,689.78889374)
\curveto(468.54794065,689.62889574)(468.54294066,689.4738959)(468.53294899,689.32389374)
\curveto(468.53294067,689.16389621)(468.48294072,689.05389632)(468.38294899,688.99389374)
\curveto(468.3029409,688.94389643)(468.18794101,688.92389645)(468.03794899,688.93389374)
\lineto(467.61794899,688.93389374)
\lineto(467.30294899,688.93389374)
\curveto(467.19294201,688.92389645)(467.08294212,688.92389645)(466.97294899,688.93389374)
\curveto(466.87294233,688.93389644)(466.77794242,688.94889642)(466.68794899,688.97889374)
\curveto(466.60794259,688.99889637)(466.54794265,689.03889633)(466.50794899,689.09889374)
\curveto(466.45794274,689.17889619)(466.43294277,689.29389608)(466.43294899,689.44389374)
\curveto(466.44294276,689.58389579)(466.44794275,689.71389566)(466.44794899,689.83389374)
\lineto(466.44794899,691.46889374)
\lineto(466.44794899,691.84389374)
\curveto(466.44794275,691.98389339)(466.43294277,692.08889328)(466.40294899,692.15889374)
\curveto(466.38294282,692.17889319)(466.36294284,692.19389318)(466.34294899,692.20389374)
\curveto(466.33294287,692.22389315)(466.31794288,692.24389313)(466.29794899,692.26389374)
\curveto(466.20794299,692.2738931)(466.13794306,692.25389312)(466.08794899,692.20389374)
\curveto(466.03794316,692.16389321)(465.98294322,692.12389325)(465.92294899,692.08389374)
\curveto(465.83294337,692.01389336)(465.73794346,691.94889342)(465.63794899,691.88889374)
\curveto(465.54794365,691.82889354)(465.44794375,691.7738936)(465.33794899,691.72389374)
\curveto(465.15794404,691.64389373)(464.95794424,691.58389379)(464.73794899,691.54389374)
\curveto(464.51794468,691.49389388)(464.29294491,691.4688939)(464.06294899,691.46889374)
\curveto(463.83294537,691.45889391)(463.6029456,691.4738939)(463.37294899,691.51389374)
\curveto(463.15294605,691.55389382)(462.95294625,691.61389376)(462.77294899,691.69389374)
\curveto(462.32294688,691.89389348)(461.95794724,692.14889322)(461.67794899,692.45889374)
\curveto(461.3979478,692.77889259)(461.16294804,693.1688922)(460.97294899,693.62889374)
\curveto(460.92294828,693.73889163)(460.88794831,693.84889152)(460.86794899,693.95889374)
\curveto(460.84794835,694.07889129)(460.82294838,694.19389118)(460.79294899,694.30389374)
\curveto(460.77294843,694.34389103)(460.76294844,694.37889099)(460.76294899,694.40889374)
\curveto(460.77294843,694.44889092)(460.77294843,694.48889088)(460.76294899,694.52889374)
\curveto(460.74294846,694.60889076)(460.72794847,694.69389068)(460.71794899,694.78389374)
\curveto(460.71794848,694.88389049)(460.70794849,694.97889039)(460.68794899,695.06889374)
\lineto(460.68794899,695.26389374)
\curveto(460.67794852,695.31389006)(460.67294853,695.37389)(460.67294899,695.44389374)
\curveto(460.67294853,695.52388985)(460.67794852,695.58888978)(460.68794899,695.63889374)
\curveto(460.6979485,695.68888968)(460.7029485,695.73388964)(460.70294899,695.77389374)
\lineto(460.70294899,695.90889374)
\curveto(460.71294849,695.95888941)(460.71294849,696.00888936)(460.70294899,696.05889374)
\curveto(460.7029485,696.10888926)(460.71294849,696.15888921)(460.73294899,696.20889374)
\curveto(460.75294845,696.29888907)(460.76794843,696.38888898)(460.77794899,696.47889374)
\curveto(460.78794841,696.57888879)(460.8029484,696.6738887)(460.82294899,696.76389374)
\curveto(460.87294833,696.93388844)(460.92294828,697.09388828)(460.97294899,697.24389374)
\curveto(461.03294817,697.39388798)(461.09294811,697.53888783)(461.15294899,697.67889374)
\curveto(461.21294799,697.81888755)(461.28794791,697.95388742)(461.37794899,698.08389374)
\curveto(461.46794773,698.21388716)(461.55794764,698.33888703)(461.64794899,698.45889374)
\curveto(461.73794746,698.5688868)(461.83794736,698.6688867)(461.94794899,698.75889374)
\curveto(461.97794722,698.78888658)(461.9979472,698.81388656)(462.00794899,698.83389374)
\curveto(462.05794714,698.86388651)(462.1029471,698.89388648)(462.14294899,698.92389374)
\curveto(462.18294702,698.96388641)(462.22294698,698.99888637)(462.26294899,699.02889374)
\curveto(462.4029468,699.12888624)(462.54794665,699.20888616)(462.69794899,699.26889374)
\curveto(462.85794634,699.33888603)(463.02294618,699.40388597)(463.19294899,699.46389374)
\curveto(463.28294592,699.49388588)(463.37294583,699.51388586)(463.46294899,699.52389374)
\curveto(463.55294565,699.53388584)(463.64294556,699.54888582)(463.73294899,699.56889374)
\curveto(463.76294544,699.57888579)(463.81794538,699.57888579)(463.89794899,699.56889374)
\curveto(463.97794522,699.55888581)(464.02794517,699.56388581)(464.04794899,699.58389374)
\curveto(464.36794483,699.59388578)(464.66794453,699.56388581)(464.94794899,699.49389374)
\curveto(465.22794397,699.43388594)(465.46794373,699.34388603)(465.66794899,699.22389374)
\lineto(465.84794899,699.10389374)
\curveto(465.90794329,699.06388631)(465.96294324,699.02388635)(466.01294899,698.98389374)
\curveto(466.07294313,698.93388644)(466.12294308,698.88388649)(466.16294899,698.83389374)
\curveto(466.21294299,698.79388658)(466.29294291,698.7738866)(466.40294899,698.77389374)
\lineto(466.44794899,698.81889374)
\lineto(466.50794899,698.87889374)
\curveto(466.53794266,698.95888641)(466.55794264,699.03388634)(466.56794899,699.10389374)
\curveto(466.57794262,699.18388619)(466.61794258,699.24888612)(466.68794899,699.29889374)
\curveto(466.73794246,699.33888603)(466.80794239,699.35888601)(466.89794899,699.35889374)
\curveto(466.9979422,699.368886)(467.0979421,699.373886)(467.19794899,699.37389374)
\lineto(467.91794899,699.37389374)
\lineto(468.12794899,699.37389374)
\curveto(468.197941,699.373886)(468.26294094,699.36388601)(468.32294899,699.34389374)
\curveto(468.39294081,699.32388605)(468.44794075,699.27888609)(468.48794899,699.20889374)
\curveto(468.53794066,699.13888623)(468.55794064,699.04388633)(468.54794899,698.92389374)
\lineto(468.54794899,698.57889374)
\lineto(468.54794899,689.78889374)
\moveto(466.50794899,695.39889374)
\curveto(466.51794268,695.41888995)(466.51794268,695.44388993)(466.50794899,695.47389374)
\lineto(466.50794899,695.54889374)
\curveto(466.4979427,695.64888972)(466.49294271,695.74388963)(466.49294899,695.83389374)
\curveto(466.49294271,695.92388945)(466.48294272,696.00888936)(466.46294899,696.08889374)
\curveto(466.45294275,696.11888925)(466.44794275,696.14388923)(466.44794899,696.16389374)
\curveto(466.45794274,696.19388918)(466.45794274,696.22388915)(466.44794899,696.25389374)
\curveto(466.42794277,696.33388904)(466.40794279,696.40388897)(466.38794899,696.46389374)
\curveto(466.37794282,696.53388884)(466.36294284,696.60388877)(466.34294899,696.67389374)
\curveto(466.24294296,696.96388841)(466.10794309,697.21388816)(465.93794899,697.42389374)
\curveto(465.76794343,697.63388774)(465.54794365,697.79388758)(465.27794899,697.90389374)
\curveto(465.16794403,697.95388742)(465.04794415,697.97888739)(464.91794899,697.97889374)
\curveto(464.7979444,697.98888738)(464.66794453,697.99388738)(464.52794899,697.99389374)
\curveto(464.4979447,697.9738874)(464.46294474,697.96388741)(464.42294899,697.96389374)
\curveto(464.38294482,697.9738874)(464.34294486,697.9738874)(464.30294899,697.96389374)
\lineto(464.12294899,697.90389374)
\curveto(464.06294514,697.89388748)(464.00794519,697.87888749)(463.95794899,697.85889374)
\curveto(463.66794553,697.72888764)(463.43794576,697.53888783)(463.26794899,697.28889374)
\curveto(463.10794609,697.03888833)(462.98294622,696.74888862)(462.89294899,696.41889374)
\curveto(462.87294633,696.33888903)(462.85794634,696.26388911)(462.84794899,696.19389374)
\curveto(462.84794635,696.13388924)(462.83794636,696.06388931)(462.81794899,695.98389374)
\curveto(462.81794638,695.91388946)(462.81294639,695.86388951)(462.80294899,695.83389374)
\curveto(462.79294641,695.78388959)(462.78294642,695.69388968)(462.77294899,695.56389374)
\curveto(462.77294643,695.44388993)(462.78294642,695.35889001)(462.80294899,695.30889374)
\lineto(462.80294899,695.17389374)
\curveto(462.81294639,695.13389024)(462.81794638,695.09389028)(462.81794899,695.05389374)
\curveto(462.81794638,695.01389036)(462.82294638,694.97889039)(462.83294899,694.94889374)
\lineto(462.83294899,694.87389374)
\curveto(462.84294636,694.84389053)(462.84794635,694.81889055)(462.84794899,694.79889374)
\curveto(462.86794633,694.71889065)(462.88294632,694.64389073)(462.89294899,694.57389374)
\curveto(462.9029463,694.50389087)(462.92294628,694.43389094)(462.95294899,694.36389374)
\curveto(463.03294617,694.11389126)(463.13794606,693.89889147)(463.26794899,693.71889374)
\curveto(463.3979458,693.53889183)(463.56294564,693.38389199)(463.76294899,693.25389374)
\curveto(463.9029453,693.1738922)(464.05794514,693.11389226)(464.22794899,693.07389374)
\curveto(464.25794494,693.06389231)(464.28294492,693.05889231)(464.30294899,693.05889374)
\curveto(464.33294487,693.05889231)(464.36794483,693.05389232)(464.40794899,693.04389374)
\curveto(464.43794476,693.03389234)(464.48294472,693.02389235)(464.54294899,693.01389374)
\curveto(464.61294459,693.01389236)(464.67294453,693.01889235)(464.72294899,693.02889374)
\curveto(464.74294446,693.03889233)(464.76794443,693.03889233)(464.79794899,693.02889374)
\curveto(464.83794436,693.02889234)(464.87294433,693.03389234)(464.90294899,693.04389374)
\curveto(464.97294423,693.06389231)(465.03794416,693.07889229)(465.09794899,693.08889374)
\curveto(465.16794403,693.09889227)(465.23794396,693.11389226)(465.30794899,693.13389374)
\curveto(465.56794363,693.24389213)(465.77294343,693.38889198)(465.92294899,693.56889374)
\curveto(466.08294312,693.74889162)(466.21794298,693.9688914)(466.32794899,694.22889374)
\curveto(466.35794284,694.30889106)(466.38294282,694.39389098)(466.40294899,694.48389374)
\lineto(466.46294899,694.75389374)
\lineto(466.46294899,694.85889374)
\curveto(466.47294273,694.88889048)(466.47794272,694.92389045)(466.47794899,694.96389374)
\curveto(466.4979427,695.06389031)(466.50794269,695.14889022)(466.50794899,695.21889374)
\lineto(466.50794899,695.39889374)
}
}
{
\newrgbcolor{curcolor}{0 0 0}
\pscustom[linestyle=none,fillstyle=solid,fillcolor=curcolor]
{
\newpath
\moveto(470.57787087,699.35889374)
\lineto(471.70287087,699.35889374)
\curveto(471.81286843,699.35888601)(471.91286833,699.35388602)(472.00287087,699.34389374)
\curveto(472.09286815,699.33388604)(472.15786809,699.29888607)(472.19787087,699.23889374)
\curveto(472.247868,699.17888619)(472.27786797,699.09388628)(472.28787087,698.98389374)
\curveto(472.29786795,698.88388649)(472.30286794,698.77888659)(472.30287087,698.66889374)
\lineto(472.30287087,697.61889374)
\lineto(472.30287087,695.38389374)
\curveto(472.30286794,695.02389035)(472.31786793,694.68389069)(472.34787087,694.36389374)
\curveto(472.37786787,694.04389133)(472.46786778,693.77889159)(472.61787087,693.56889374)
\curveto(472.75786749,693.35889201)(472.98286726,693.20889216)(473.29287087,693.11889374)
\curveto(473.3428669,693.10889226)(473.38286686,693.10389227)(473.41287087,693.10389374)
\curveto(473.45286679,693.10389227)(473.49786675,693.09889227)(473.54787087,693.08889374)
\curveto(473.59786665,693.07889229)(473.65286659,693.0738923)(473.71287087,693.07389374)
\curveto(473.77286647,693.0738923)(473.81786643,693.07889229)(473.84787087,693.08889374)
\curveto(473.89786635,693.10889226)(473.93786631,693.11389226)(473.96787087,693.10389374)
\curveto(474.00786624,693.09389228)(474.0478662,693.09889227)(474.08787087,693.11889374)
\curveto(474.29786595,693.1688922)(474.46286578,693.23389214)(474.58287087,693.31389374)
\curveto(474.76286548,693.42389195)(474.90286534,693.56389181)(475.00287087,693.73389374)
\curveto(475.11286513,693.91389146)(475.18786506,694.10889126)(475.22787087,694.31889374)
\curveto(475.27786497,694.53889083)(475.30786494,694.77889059)(475.31787087,695.03889374)
\curveto(475.32786492,695.30889006)(475.33286491,695.58888978)(475.33287087,695.87889374)
\lineto(475.33287087,697.69389374)
\lineto(475.33287087,698.66889374)
\lineto(475.33287087,698.93889374)
\curveto(475.33286491,699.03888633)(475.35286489,699.11888625)(475.39287087,699.17889374)
\curveto(475.4428648,699.2688861)(475.51786473,699.31888605)(475.61787087,699.32889374)
\curveto(475.71786453,699.34888602)(475.83786441,699.35888601)(475.97787087,699.35889374)
\lineto(476.77287087,699.35889374)
\lineto(477.05787087,699.35889374)
\curveto(477.1478631,699.35888601)(477.22286302,699.33888603)(477.28287087,699.29889374)
\curveto(477.36286288,699.24888612)(477.40786284,699.1738862)(477.41787087,699.07389374)
\curveto(477.42786282,698.9738864)(477.43286281,698.85888651)(477.43287087,698.72889374)
\lineto(477.43287087,697.58889374)
\lineto(477.43287087,693.37389374)
\lineto(477.43287087,692.30889374)
\lineto(477.43287087,692.00889374)
\curveto(477.43286281,691.90889346)(477.41286283,691.83389354)(477.37287087,691.78389374)
\curveto(477.32286292,691.70389367)(477.247863,691.65889371)(477.14787087,691.64889374)
\curveto(477.0478632,691.63889373)(476.9428633,691.63389374)(476.83287087,691.63389374)
\lineto(476.02287087,691.63389374)
\curveto(475.91286433,691.63389374)(475.81286443,691.63889373)(475.72287087,691.64889374)
\curveto(475.6428646,691.65889371)(475.57786467,691.69889367)(475.52787087,691.76889374)
\curveto(475.50786474,691.79889357)(475.48786476,691.84389353)(475.46787087,691.90389374)
\curveto(475.45786479,691.96389341)(475.4428648,692.02389335)(475.42287087,692.08389374)
\curveto(475.41286483,692.14389323)(475.39786485,692.19889317)(475.37787087,692.24889374)
\curveto(475.35786489,692.29889307)(475.32786492,692.32889304)(475.28787087,692.33889374)
\curveto(475.26786498,692.35889301)(475.242865,692.36389301)(475.21287087,692.35389374)
\curveto(475.18286506,692.34389303)(475.15786509,692.33389304)(475.13787087,692.32389374)
\curveto(475.06786518,692.28389309)(475.00786524,692.23889313)(474.95787087,692.18889374)
\curveto(474.90786534,692.13889323)(474.85286539,692.09389328)(474.79287087,692.05389374)
\curveto(474.75286549,692.02389335)(474.71286553,691.98889338)(474.67287087,691.94889374)
\curveto(474.6428656,691.91889345)(474.60286564,691.88889348)(474.55287087,691.85889374)
\curveto(474.32286592,691.71889365)(474.05286619,691.60889376)(473.74287087,691.52889374)
\curveto(473.67286657,691.50889386)(473.60286664,691.49889387)(473.53287087,691.49889374)
\curveto(473.46286678,691.48889388)(473.38786686,691.4738939)(473.30787087,691.45389374)
\curveto(473.26786698,691.44389393)(473.22286702,691.44389393)(473.17287087,691.45389374)
\curveto(473.13286711,691.45389392)(473.09286715,691.44889392)(473.05287087,691.43889374)
\curveto(473.02286722,691.42889394)(472.95786729,691.42889394)(472.85787087,691.43889374)
\curveto(472.76786748,691.43889393)(472.70786754,691.44389393)(472.67787087,691.45389374)
\curveto(472.62786762,691.45389392)(472.57786767,691.45889391)(472.52787087,691.46889374)
\lineto(472.37787087,691.46889374)
\curveto(472.25786799,691.49889387)(472.1428681,691.52389385)(472.03287087,691.54389374)
\curveto(471.92286832,691.56389381)(471.81286843,691.59389378)(471.70287087,691.63389374)
\curveto(471.65286859,691.65389372)(471.60786864,691.6688937)(471.56787087,691.67889374)
\curveto(471.53786871,691.69889367)(471.49786875,691.71889365)(471.44787087,691.73889374)
\curveto(471.09786915,691.92889344)(470.81786943,692.19389318)(470.60787087,692.53389374)
\curveto(470.47786977,692.74389263)(470.38286986,692.99389238)(470.32287087,693.28389374)
\curveto(470.26286998,693.58389179)(470.22287002,693.89889147)(470.20287087,694.22889374)
\curveto(470.19287005,694.5688908)(470.18787006,694.91389046)(470.18787087,695.26389374)
\curveto(470.19787005,695.62388975)(470.20287004,695.97888939)(470.20287087,696.32889374)
\lineto(470.20287087,698.36889374)
\curveto(470.20287004,698.49888687)(470.19787005,698.64888672)(470.18787087,698.81889374)
\curveto(470.18787006,698.99888637)(470.21287003,699.12888624)(470.26287087,699.20889374)
\curveto(470.29286995,699.25888611)(470.35286989,699.30388607)(470.44287087,699.34389374)
\curveto(470.50286974,699.34388603)(470.5478697,699.34888602)(470.57787087,699.35889374)
}
}
{
\newrgbcolor{curcolor}{0 0 0}
\pscustom[linestyle=none,fillstyle=solid,fillcolor=curcolor]
{
\newpath
\moveto(486.42912087,695.57889374)
\curveto(486.4491127,695.49888987)(486.4491127,695.40888996)(486.42912087,695.30889374)
\curveto(486.40911274,695.20889016)(486.37411278,695.14389023)(486.32412087,695.11389374)
\curveto(486.27411288,695.0738903)(486.19911295,695.04389033)(486.09912087,695.02389374)
\curveto(486.00911314,695.01389036)(485.90411325,695.00389037)(485.78412087,694.99389374)
\lineto(485.43912087,694.99389374)
\curveto(485.32911382,695.00389037)(485.22911392,695.00889036)(485.13912087,695.00889374)
\lineto(481.47912087,695.00889374)
\lineto(481.26912087,695.00889374)
\curveto(481.20911794,695.00889036)(481.154118,694.99889037)(481.10412087,694.97889374)
\curveto(481.02411813,694.93889043)(480.97411818,694.89889047)(480.95412087,694.85889374)
\curveto(480.93411822,694.83889053)(480.91411824,694.79889057)(480.89412087,694.73889374)
\curveto(480.87411828,694.68889068)(480.86911828,694.63889073)(480.87912087,694.58889374)
\curveto(480.89911825,694.52889084)(480.90911824,694.4688909)(480.90912087,694.40889374)
\curveto(480.91911823,694.35889101)(480.93411822,694.30389107)(480.95412087,694.24389374)
\curveto(481.03411812,694.00389137)(481.12911802,693.80389157)(481.23912087,693.64389374)
\curveto(481.35911779,693.49389188)(481.51911763,693.35889201)(481.71912087,693.23889374)
\curveto(481.79911735,693.18889218)(481.87911727,693.15389222)(481.95912087,693.13389374)
\curveto(482.0491171,693.12389225)(482.13911701,693.10389227)(482.22912087,693.07389374)
\curveto(482.30911684,693.05389232)(482.41911673,693.03889233)(482.55912087,693.02889374)
\curveto(482.69911645,693.01889235)(482.81911633,693.02389235)(482.91912087,693.04389374)
\lineto(483.05412087,693.04389374)
\curveto(483.154116,693.06389231)(483.24411591,693.08389229)(483.32412087,693.10389374)
\curveto(483.41411574,693.13389224)(483.49911565,693.16389221)(483.57912087,693.19389374)
\curveto(483.67911547,693.24389213)(483.78911536,693.30889206)(483.90912087,693.38889374)
\curveto(484.03911511,693.4688919)(484.13411502,693.54889182)(484.19412087,693.62889374)
\curveto(484.24411491,693.69889167)(484.29411486,693.76389161)(484.34412087,693.82389374)
\curveto(484.40411475,693.89389148)(484.47411468,693.94389143)(484.55412087,693.97389374)
\curveto(484.6541145,694.02389135)(484.77911437,694.04389133)(484.92912087,694.03389374)
\lineto(485.36412087,694.03389374)
\lineto(485.54412087,694.03389374)
\curveto(485.61411354,694.04389133)(485.67411348,694.03889133)(485.72412087,694.01889374)
\lineto(485.87412087,694.01889374)
\curveto(485.97411318,693.99889137)(486.04411311,693.9738914)(486.08412087,693.94389374)
\curveto(486.12411303,693.92389145)(486.14411301,693.87889149)(486.14412087,693.80889374)
\curveto(486.154113,693.73889163)(486.149113,693.67889169)(486.12912087,693.62889374)
\curveto(486.07911307,693.48889188)(486.02411313,693.36389201)(485.96412087,693.25389374)
\curveto(485.90411325,693.14389223)(485.83411332,693.03389234)(485.75412087,692.92389374)
\curveto(485.53411362,692.59389278)(485.28411387,692.32889304)(485.00412087,692.12889374)
\curveto(484.72411443,691.92889344)(484.37411478,691.75889361)(483.95412087,691.61889374)
\curveto(483.84411531,691.57889379)(483.73411542,691.55389382)(483.62412087,691.54389374)
\curveto(483.51411564,691.53389384)(483.39911575,691.51389386)(483.27912087,691.48389374)
\curveto(483.23911591,691.4738939)(483.19411596,691.4738939)(483.14412087,691.48389374)
\curveto(483.10411605,691.48389389)(483.06411609,691.47889389)(483.02412087,691.46889374)
\lineto(482.85912087,691.46889374)
\curveto(482.80911634,691.44889392)(482.7491164,691.44389393)(482.67912087,691.45389374)
\curveto(482.61911653,691.45389392)(482.56411659,691.45889391)(482.51412087,691.46889374)
\curveto(482.43411672,691.47889389)(482.36411679,691.47889389)(482.30412087,691.46889374)
\curveto(482.24411691,691.45889391)(482.17911697,691.46389391)(482.10912087,691.48389374)
\curveto(482.05911709,691.50389387)(482.00411715,691.51389386)(481.94412087,691.51389374)
\curveto(481.88411727,691.51389386)(481.82911732,691.52389385)(481.77912087,691.54389374)
\curveto(481.66911748,691.56389381)(481.55911759,691.58889378)(481.44912087,691.61889374)
\curveto(481.33911781,691.63889373)(481.23911791,691.6738937)(481.14912087,691.72389374)
\curveto(481.03911811,691.76389361)(480.93411822,691.79889357)(480.83412087,691.82889374)
\curveto(480.74411841,691.8688935)(480.65911849,691.91389346)(480.57912087,691.96389374)
\curveto(480.25911889,692.16389321)(479.97411918,692.39389298)(479.72412087,692.65389374)
\curveto(479.47411968,692.92389245)(479.26911988,693.23389214)(479.10912087,693.58389374)
\curveto(479.05912009,693.69389168)(479.01912013,693.80389157)(478.98912087,693.91389374)
\curveto(478.95912019,694.03389134)(478.91912023,694.15389122)(478.86912087,694.27389374)
\curveto(478.85912029,694.31389106)(478.8541203,694.34889102)(478.85412087,694.37889374)
\curveto(478.8541203,694.41889095)(478.8491203,694.45889091)(478.83912087,694.49889374)
\curveto(478.79912035,694.61889075)(478.77412038,694.74889062)(478.76412087,694.88889374)
\lineto(478.73412087,695.30889374)
\curveto(478.73412042,695.35889001)(478.72912042,695.41388996)(478.71912087,695.47389374)
\curveto(478.71912043,695.53388984)(478.72412043,695.58888978)(478.73412087,695.63889374)
\lineto(478.73412087,695.81889374)
\lineto(478.77912087,696.17889374)
\curveto(478.81912033,696.34888902)(478.8541203,696.51388886)(478.88412087,696.67389374)
\curveto(478.91412024,696.83388854)(478.95912019,696.98388839)(479.01912087,697.12389374)
\curveto(479.4491197,698.16388721)(480.17911897,698.89888647)(481.20912087,699.32889374)
\curveto(481.3491178,699.38888598)(481.48911766,699.42888594)(481.62912087,699.44889374)
\curveto(481.77911737,699.47888589)(481.93411722,699.51388586)(482.09412087,699.55389374)
\curveto(482.17411698,699.56388581)(482.2491169,699.5688858)(482.31912087,699.56889374)
\curveto(482.38911676,699.5688858)(482.46411669,699.5738858)(482.54412087,699.58389374)
\curveto(483.0541161,699.59388578)(483.48911566,699.53388584)(483.84912087,699.40389374)
\curveto(484.21911493,699.28388609)(484.5491146,699.12388625)(484.83912087,698.92389374)
\curveto(484.92911422,698.86388651)(485.01911413,698.79388658)(485.10912087,698.71389374)
\curveto(485.19911395,698.64388673)(485.27911387,698.5688868)(485.34912087,698.48889374)
\curveto(485.37911377,698.43888693)(485.41911373,698.39888697)(485.46912087,698.36889374)
\curveto(485.5491136,698.25888711)(485.62411353,698.14388723)(485.69412087,698.02389374)
\curveto(485.76411339,697.91388746)(485.83911331,697.79888757)(485.91912087,697.67889374)
\curveto(485.96911318,697.58888778)(486.00911314,697.49388788)(486.03912087,697.39389374)
\curveto(486.07911307,697.30388807)(486.11911303,697.20388817)(486.15912087,697.09389374)
\curveto(486.20911294,696.96388841)(486.2491129,696.82888854)(486.27912087,696.68889374)
\curveto(486.30911284,696.54888882)(486.34411281,696.40888896)(486.38412087,696.26889374)
\curveto(486.40411275,696.18888918)(486.40911274,696.09888927)(486.39912087,695.99889374)
\curveto(486.39911275,695.90888946)(486.40911274,695.82388955)(486.42912087,695.74389374)
\lineto(486.42912087,695.57889374)
\moveto(484.17912087,696.46389374)
\curveto(484.2491149,696.56388881)(484.2541149,696.68388869)(484.19412087,696.82389374)
\curveto(484.14411501,696.9738884)(484.10411505,697.08388829)(484.07412087,697.15389374)
\curveto(483.93411522,697.42388795)(483.7491154,697.62888774)(483.51912087,697.76889374)
\curveto(483.28911586,697.91888745)(482.96911618,697.99888737)(482.55912087,698.00889374)
\curveto(482.52911662,697.98888738)(482.49411666,697.98388739)(482.45412087,697.99389374)
\curveto(482.41411674,698.00388737)(482.37911677,698.00388737)(482.34912087,697.99389374)
\curveto(482.29911685,697.9738874)(482.24411691,697.95888741)(482.18412087,697.94889374)
\curveto(482.12411703,697.94888742)(482.06911708,697.93888743)(482.01912087,697.91889374)
\curveto(481.57911757,697.77888759)(481.2541179,697.50388787)(481.04412087,697.09389374)
\curveto(481.02411813,697.05388832)(480.99911815,696.99888837)(480.96912087,696.92889374)
\curveto(480.9491182,696.8688885)(480.93411822,696.80388857)(480.92412087,696.73389374)
\curveto(480.91411824,696.6738887)(480.91411824,696.61388876)(480.92412087,696.55389374)
\curveto(480.94411821,696.49388888)(480.97911817,696.44388893)(481.02912087,696.40389374)
\curveto(481.10911804,696.35388902)(481.21911793,696.32888904)(481.35912087,696.32889374)
\lineto(481.76412087,696.32889374)
\lineto(483.42912087,696.32889374)
\lineto(483.86412087,696.32889374)
\curveto(484.02411513,696.33888903)(484.12911502,696.38388899)(484.17912087,696.46389374)
}
}
{
\newrgbcolor{curcolor}{0 0 0}
\pscustom[linestyle=none,fillstyle=solid,fillcolor=curcolor]
{
}
}
{
\newrgbcolor{curcolor}{0 0 0}
\pscustom[linestyle=none,fillstyle=solid,fillcolor=curcolor]
{
\newpath
\moveto(499.20255837,695.57889374)
\curveto(499.2225502,695.49888987)(499.2225502,695.40888996)(499.20255837,695.30889374)
\curveto(499.18255024,695.20889016)(499.14755028,695.14389023)(499.09755837,695.11389374)
\curveto(499.04755038,695.0738903)(498.97255045,695.04389033)(498.87255837,695.02389374)
\curveto(498.78255064,695.01389036)(498.67755075,695.00389037)(498.55755837,694.99389374)
\lineto(498.21255837,694.99389374)
\curveto(498.10255132,695.00389037)(498.00255142,695.00889036)(497.91255837,695.00889374)
\lineto(494.25255837,695.00889374)
\lineto(494.04255837,695.00889374)
\curveto(493.98255544,695.00889036)(493.9275555,694.99889037)(493.87755837,694.97889374)
\curveto(493.79755563,694.93889043)(493.74755568,694.89889047)(493.72755837,694.85889374)
\curveto(493.70755572,694.83889053)(493.68755574,694.79889057)(493.66755837,694.73889374)
\curveto(493.64755578,694.68889068)(493.64255578,694.63889073)(493.65255837,694.58889374)
\curveto(493.67255575,694.52889084)(493.68255574,694.4688909)(493.68255837,694.40889374)
\curveto(493.69255573,694.35889101)(493.70755572,694.30389107)(493.72755837,694.24389374)
\curveto(493.80755562,694.00389137)(493.90255552,693.80389157)(494.01255837,693.64389374)
\curveto(494.13255529,693.49389188)(494.29255513,693.35889201)(494.49255837,693.23889374)
\curveto(494.57255485,693.18889218)(494.65255477,693.15389222)(494.73255837,693.13389374)
\curveto(494.8225546,693.12389225)(494.91255451,693.10389227)(495.00255837,693.07389374)
\curveto(495.08255434,693.05389232)(495.19255423,693.03889233)(495.33255837,693.02889374)
\curveto(495.47255395,693.01889235)(495.59255383,693.02389235)(495.69255837,693.04389374)
\lineto(495.82755837,693.04389374)
\curveto(495.9275535,693.06389231)(496.01755341,693.08389229)(496.09755837,693.10389374)
\curveto(496.18755324,693.13389224)(496.27255315,693.16389221)(496.35255837,693.19389374)
\curveto(496.45255297,693.24389213)(496.56255286,693.30889206)(496.68255837,693.38889374)
\curveto(496.81255261,693.4688919)(496.90755252,693.54889182)(496.96755837,693.62889374)
\curveto(497.01755241,693.69889167)(497.06755236,693.76389161)(497.11755837,693.82389374)
\curveto(497.17755225,693.89389148)(497.24755218,693.94389143)(497.32755837,693.97389374)
\curveto(497.427552,694.02389135)(497.55255187,694.04389133)(497.70255837,694.03389374)
\lineto(498.13755837,694.03389374)
\lineto(498.31755837,694.03389374)
\curveto(498.38755104,694.04389133)(498.44755098,694.03889133)(498.49755837,694.01889374)
\lineto(498.64755837,694.01889374)
\curveto(498.74755068,693.99889137)(498.81755061,693.9738914)(498.85755837,693.94389374)
\curveto(498.89755053,693.92389145)(498.91755051,693.87889149)(498.91755837,693.80889374)
\curveto(498.9275505,693.73889163)(498.9225505,693.67889169)(498.90255837,693.62889374)
\curveto(498.85255057,693.48889188)(498.79755063,693.36389201)(498.73755837,693.25389374)
\curveto(498.67755075,693.14389223)(498.60755082,693.03389234)(498.52755837,692.92389374)
\curveto(498.30755112,692.59389278)(498.05755137,692.32889304)(497.77755837,692.12889374)
\curveto(497.49755193,691.92889344)(497.14755228,691.75889361)(496.72755837,691.61889374)
\curveto(496.61755281,691.57889379)(496.50755292,691.55389382)(496.39755837,691.54389374)
\curveto(496.28755314,691.53389384)(496.17255325,691.51389386)(496.05255837,691.48389374)
\curveto(496.01255341,691.4738939)(495.96755346,691.4738939)(495.91755837,691.48389374)
\curveto(495.87755355,691.48389389)(495.83755359,691.47889389)(495.79755837,691.46889374)
\lineto(495.63255837,691.46889374)
\curveto(495.58255384,691.44889392)(495.5225539,691.44389393)(495.45255837,691.45389374)
\curveto(495.39255403,691.45389392)(495.33755409,691.45889391)(495.28755837,691.46889374)
\curveto(495.20755422,691.47889389)(495.13755429,691.47889389)(495.07755837,691.46889374)
\curveto(495.01755441,691.45889391)(494.95255447,691.46389391)(494.88255837,691.48389374)
\curveto(494.83255459,691.50389387)(494.77755465,691.51389386)(494.71755837,691.51389374)
\curveto(494.65755477,691.51389386)(494.60255482,691.52389385)(494.55255837,691.54389374)
\curveto(494.44255498,691.56389381)(494.33255509,691.58889378)(494.22255837,691.61889374)
\curveto(494.11255531,691.63889373)(494.01255541,691.6738937)(493.92255837,691.72389374)
\curveto(493.81255561,691.76389361)(493.70755572,691.79889357)(493.60755837,691.82889374)
\curveto(493.51755591,691.8688935)(493.43255599,691.91389346)(493.35255837,691.96389374)
\curveto(493.03255639,692.16389321)(492.74755668,692.39389298)(492.49755837,692.65389374)
\curveto(492.24755718,692.92389245)(492.04255738,693.23389214)(491.88255837,693.58389374)
\curveto(491.83255759,693.69389168)(491.79255763,693.80389157)(491.76255837,693.91389374)
\curveto(491.73255769,694.03389134)(491.69255773,694.15389122)(491.64255837,694.27389374)
\curveto(491.63255779,694.31389106)(491.6275578,694.34889102)(491.62755837,694.37889374)
\curveto(491.6275578,694.41889095)(491.6225578,694.45889091)(491.61255837,694.49889374)
\curveto(491.57255785,694.61889075)(491.54755788,694.74889062)(491.53755837,694.88889374)
\lineto(491.50755837,695.30889374)
\curveto(491.50755792,695.35889001)(491.50255792,695.41388996)(491.49255837,695.47389374)
\curveto(491.49255793,695.53388984)(491.49755793,695.58888978)(491.50755837,695.63889374)
\lineto(491.50755837,695.81889374)
\lineto(491.55255837,696.17889374)
\curveto(491.59255783,696.34888902)(491.6275578,696.51388886)(491.65755837,696.67389374)
\curveto(491.68755774,696.83388854)(491.73255769,696.98388839)(491.79255837,697.12389374)
\curveto(492.2225572,698.16388721)(492.95255647,698.89888647)(493.98255837,699.32889374)
\curveto(494.1225553,699.38888598)(494.26255516,699.42888594)(494.40255837,699.44889374)
\curveto(494.55255487,699.47888589)(494.70755472,699.51388586)(494.86755837,699.55389374)
\curveto(494.94755448,699.56388581)(495.0225544,699.5688858)(495.09255837,699.56889374)
\curveto(495.16255426,699.5688858)(495.23755419,699.5738858)(495.31755837,699.58389374)
\curveto(495.8275536,699.59388578)(496.26255316,699.53388584)(496.62255837,699.40389374)
\curveto(496.99255243,699.28388609)(497.3225521,699.12388625)(497.61255837,698.92389374)
\curveto(497.70255172,698.86388651)(497.79255163,698.79388658)(497.88255837,698.71389374)
\curveto(497.97255145,698.64388673)(498.05255137,698.5688868)(498.12255837,698.48889374)
\curveto(498.15255127,698.43888693)(498.19255123,698.39888697)(498.24255837,698.36889374)
\curveto(498.3225511,698.25888711)(498.39755103,698.14388723)(498.46755837,698.02389374)
\curveto(498.53755089,697.91388746)(498.61255081,697.79888757)(498.69255837,697.67889374)
\curveto(498.74255068,697.58888778)(498.78255064,697.49388788)(498.81255837,697.39389374)
\curveto(498.85255057,697.30388807)(498.89255053,697.20388817)(498.93255837,697.09389374)
\curveto(498.98255044,696.96388841)(499.0225504,696.82888854)(499.05255837,696.68889374)
\curveto(499.08255034,696.54888882)(499.11755031,696.40888896)(499.15755837,696.26889374)
\curveto(499.17755025,696.18888918)(499.18255024,696.09888927)(499.17255837,695.99889374)
\curveto(499.17255025,695.90888946)(499.18255024,695.82388955)(499.20255837,695.74389374)
\lineto(499.20255837,695.57889374)
\moveto(496.95255837,696.46389374)
\curveto(497.0225524,696.56388881)(497.0275524,696.68388869)(496.96755837,696.82389374)
\curveto(496.91755251,696.9738884)(496.87755255,697.08388829)(496.84755837,697.15389374)
\curveto(496.70755272,697.42388795)(496.5225529,697.62888774)(496.29255837,697.76889374)
\curveto(496.06255336,697.91888745)(495.74255368,697.99888737)(495.33255837,698.00889374)
\curveto(495.30255412,697.98888738)(495.26755416,697.98388739)(495.22755837,697.99389374)
\curveto(495.18755424,698.00388737)(495.15255427,698.00388737)(495.12255837,697.99389374)
\curveto(495.07255435,697.9738874)(495.01755441,697.95888741)(494.95755837,697.94889374)
\curveto(494.89755453,697.94888742)(494.84255458,697.93888743)(494.79255837,697.91889374)
\curveto(494.35255507,697.77888759)(494.0275554,697.50388787)(493.81755837,697.09389374)
\curveto(493.79755563,697.05388832)(493.77255565,696.99888837)(493.74255837,696.92889374)
\curveto(493.7225557,696.8688885)(493.70755572,696.80388857)(493.69755837,696.73389374)
\curveto(493.68755574,696.6738887)(493.68755574,696.61388876)(493.69755837,696.55389374)
\curveto(493.71755571,696.49388888)(493.75255567,696.44388893)(493.80255837,696.40389374)
\curveto(493.88255554,696.35388902)(493.99255543,696.32888904)(494.13255837,696.32889374)
\lineto(494.53755837,696.32889374)
\lineto(496.20255837,696.32889374)
\lineto(496.63755837,696.32889374)
\curveto(496.79755263,696.33888903)(496.90255252,696.38388899)(496.95255837,696.46389374)
}
}
{
\newrgbcolor{curcolor}{0 0 0}
\pscustom[linestyle=none,fillstyle=solid,fillcolor=curcolor]
{
\newpath
\moveto(503.42083962,699.58389374)
\curveto(504.17083512,699.60388577)(504.82083447,699.51888585)(505.37083962,699.32889374)
\curveto(505.93083336,699.14888622)(506.35583293,698.83388654)(506.64583962,698.38389374)
\curveto(506.71583257,698.2738871)(506.77583251,698.15888721)(506.82583962,698.03889374)
\curveto(506.8858324,697.92888744)(506.93583235,697.80388757)(506.97583962,697.66389374)
\curveto(506.99583229,697.60388777)(507.00583228,697.53888783)(507.00583962,697.46889374)
\curveto(507.00583228,697.39888797)(506.99583229,697.33888803)(506.97583962,697.28889374)
\curveto(506.93583235,697.22888814)(506.88083241,697.18888818)(506.81083962,697.16889374)
\curveto(506.76083253,697.14888822)(506.70083259,697.13888823)(506.63083962,697.13889374)
\lineto(506.42083962,697.13889374)
\lineto(505.76083962,697.13889374)
\curveto(505.6908336,697.13888823)(505.62083367,697.13388824)(505.55083962,697.12389374)
\curveto(505.48083381,697.12388825)(505.41583387,697.13388824)(505.35583962,697.15389374)
\curveto(505.25583403,697.1738882)(505.18083411,697.21388816)(505.13083962,697.27389374)
\curveto(505.08083421,697.33388804)(505.03583425,697.39388798)(504.99583962,697.45389374)
\lineto(504.87583962,697.66389374)
\curveto(504.84583444,697.74388763)(504.79583449,697.80888756)(504.72583962,697.85889374)
\curveto(504.62583466,697.93888743)(504.52583476,697.99888737)(504.42583962,698.03889374)
\curveto(504.33583495,698.07888729)(504.22083507,698.11388726)(504.08083962,698.14389374)
\curveto(504.01083528,698.16388721)(503.90583538,698.17888719)(503.76583962,698.18889374)
\curveto(503.63583565,698.19888717)(503.53583575,698.19388718)(503.46583962,698.17389374)
\lineto(503.36083962,698.17389374)
\lineto(503.21083962,698.14389374)
\curveto(503.17083612,698.14388723)(503.12583616,698.13888723)(503.07583962,698.12889374)
\curveto(502.90583638,698.07888729)(502.76583652,698.00888736)(502.65583962,697.91889374)
\curveto(502.55583673,697.83888753)(502.4858368,697.71388766)(502.44583962,697.54389374)
\curveto(502.42583686,697.4738879)(502.42583686,697.40888796)(502.44583962,697.34889374)
\curveto(502.46583682,697.28888808)(502.4858368,697.23888813)(502.50583962,697.19889374)
\curveto(502.57583671,697.07888829)(502.65583663,696.98388839)(502.74583962,696.91389374)
\curveto(502.84583644,696.84388853)(502.96083633,696.78388859)(503.09083962,696.73389374)
\curveto(503.28083601,696.65388872)(503.4858358,696.58388879)(503.70583962,696.52389374)
\lineto(504.39583962,696.37389374)
\curveto(504.63583465,696.33388904)(504.86583442,696.28388909)(505.08583962,696.22389374)
\curveto(505.31583397,696.1738892)(505.53083376,696.10888926)(505.73083962,696.02889374)
\curveto(505.82083347,695.98888938)(505.90583338,695.95388942)(505.98583962,695.92389374)
\curveto(506.07583321,695.90388947)(506.16083313,695.8688895)(506.24083962,695.81889374)
\curveto(506.43083286,695.69888967)(506.60083269,695.5688898)(506.75083962,695.42889374)
\curveto(506.91083238,695.28889008)(507.03583225,695.11389026)(507.12583962,694.90389374)
\curveto(507.15583213,694.83389054)(507.18083211,694.76389061)(507.20083962,694.69389374)
\curveto(507.22083207,694.62389075)(507.24083205,694.54889082)(507.26083962,694.46889374)
\curveto(507.27083202,694.40889096)(507.27583201,694.31389106)(507.27583962,694.18389374)
\curveto(507.285832,694.06389131)(507.285832,693.9688914)(507.27583962,693.89889374)
\lineto(507.27583962,693.82389374)
\curveto(507.25583203,693.76389161)(507.24083205,693.70389167)(507.23083962,693.64389374)
\curveto(507.23083206,693.59389178)(507.22583206,693.54389183)(507.21583962,693.49389374)
\curveto(507.14583214,693.19389218)(507.03583225,692.92889244)(506.88583962,692.69889374)
\curveto(506.72583256,692.45889291)(506.53083276,692.26389311)(506.30083962,692.11389374)
\curveto(506.07083322,691.96389341)(505.81083348,691.83389354)(505.52083962,691.72389374)
\curveto(505.41083388,691.6738937)(505.290834,691.63889373)(505.16083962,691.61889374)
\curveto(505.04083425,691.59889377)(504.92083437,691.5738938)(504.80083962,691.54389374)
\curveto(504.71083458,691.52389385)(504.61583467,691.51389386)(504.51583962,691.51389374)
\curveto(504.42583486,691.50389387)(504.33583495,691.48889388)(504.24583962,691.46889374)
\lineto(503.97583962,691.46889374)
\curveto(503.91583537,691.44889392)(503.81083548,691.43889393)(503.66083962,691.43889374)
\curveto(503.52083577,691.43889393)(503.42083587,691.44889392)(503.36083962,691.46889374)
\curveto(503.33083596,691.4688939)(503.29583599,691.4738939)(503.25583962,691.48389374)
\lineto(503.15083962,691.48389374)
\curveto(503.03083626,691.50389387)(502.91083638,691.51889385)(502.79083962,691.52889374)
\curveto(502.67083662,691.53889383)(502.55583673,691.55889381)(502.44583962,691.58889374)
\curveto(502.05583723,691.69889367)(501.71083758,691.82389355)(501.41083962,691.96389374)
\curveto(501.11083818,692.11389326)(500.85583843,692.33389304)(500.64583962,692.62389374)
\curveto(500.50583878,692.81389256)(500.3858389,693.03389234)(500.28583962,693.28389374)
\curveto(500.26583902,693.34389203)(500.24583904,693.42389195)(500.22583962,693.52389374)
\curveto(500.20583908,693.5738918)(500.1908391,693.64389173)(500.18083962,693.73389374)
\curveto(500.17083912,693.82389155)(500.17583911,693.89889147)(500.19583962,693.95889374)
\curveto(500.22583906,694.02889134)(500.27583901,694.07889129)(500.34583962,694.10889374)
\curveto(500.39583889,694.12889124)(500.45583883,694.13889123)(500.52583962,694.13889374)
\lineto(500.75083962,694.13889374)
\lineto(501.45583962,694.13889374)
\lineto(501.69583962,694.13889374)
\curveto(501.77583751,694.13889123)(501.84583744,694.12889124)(501.90583962,694.10889374)
\curveto(502.01583727,694.0688913)(502.0858372,694.00389137)(502.11583962,693.91389374)
\curveto(502.15583713,693.82389155)(502.20083709,693.72889164)(502.25083962,693.62889374)
\curveto(502.27083702,693.57889179)(502.30583698,693.51389186)(502.35583962,693.43389374)
\curveto(502.41583687,693.35389202)(502.46583682,693.30389207)(502.50583962,693.28389374)
\curveto(502.62583666,693.18389219)(502.74083655,693.10389227)(502.85083962,693.04389374)
\curveto(502.96083633,692.99389238)(503.10083619,692.94389243)(503.27083962,692.89389374)
\curveto(503.32083597,692.8738925)(503.37083592,692.86389251)(503.42083962,692.86389374)
\curveto(503.47083582,692.8738925)(503.52083577,692.8738925)(503.57083962,692.86389374)
\curveto(503.65083564,692.84389253)(503.73583555,692.83389254)(503.82583962,692.83389374)
\curveto(503.92583536,692.84389253)(504.01083528,692.85889251)(504.08083962,692.87889374)
\curveto(504.13083516,692.88889248)(504.17583511,692.89389248)(504.21583962,692.89389374)
\curveto(504.26583502,692.89389248)(504.31583497,692.90389247)(504.36583962,692.92389374)
\curveto(504.50583478,692.9738924)(504.63083466,693.03389234)(504.74083962,693.10389374)
\curveto(504.86083443,693.1738922)(504.95583433,693.26389211)(505.02583962,693.37389374)
\curveto(505.07583421,693.45389192)(505.11583417,693.57889179)(505.14583962,693.74889374)
\curveto(505.16583412,693.81889155)(505.16583412,693.88389149)(505.14583962,693.94389374)
\curveto(505.12583416,694.00389137)(505.10583418,694.05389132)(505.08583962,694.09389374)
\curveto(505.01583427,694.23389114)(504.92583436,694.33889103)(504.81583962,694.40889374)
\curveto(504.71583457,694.47889089)(504.59583469,694.54389083)(504.45583962,694.60389374)
\curveto(504.26583502,694.68389069)(504.06583522,694.74889062)(503.85583962,694.79889374)
\curveto(503.64583564,694.84889052)(503.43583585,694.90389047)(503.22583962,694.96389374)
\curveto(503.14583614,694.98389039)(503.06083623,694.99889037)(502.97083962,695.00889374)
\curveto(502.8908364,695.01889035)(502.81083648,695.03389034)(502.73083962,695.05389374)
\curveto(502.41083688,695.14389023)(502.10583718,695.22889014)(501.81583962,695.30889374)
\curveto(501.52583776,695.39888997)(501.26083803,695.52888984)(501.02083962,695.69889374)
\curveto(500.74083855,695.89888947)(500.53583875,696.1688892)(500.40583962,696.50889374)
\curveto(500.3858389,696.57888879)(500.36583892,696.6738887)(500.34583962,696.79389374)
\curveto(500.32583896,696.86388851)(500.31083898,696.94888842)(500.30083962,697.04889374)
\curveto(500.290839,697.14888822)(500.29583899,697.23888813)(500.31583962,697.31889374)
\curveto(500.33583895,697.368888)(500.34083895,697.40888796)(500.33083962,697.43889374)
\curveto(500.32083897,697.47888789)(500.32583896,697.52388785)(500.34583962,697.57389374)
\curveto(500.36583892,697.68388769)(500.3858389,697.78388759)(500.40583962,697.87389374)
\curveto(500.43583885,697.9738874)(500.47083882,698.0688873)(500.51083962,698.15889374)
\curveto(500.64083865,698.44888692)(500.82083847,698.68388669)(501.05083962,698.86389374)
\curveto(501.28083801,699.04388633)(501.54083775,699.18888618)(501.83083962,699.29889374)
\curveto(501.94083735,699.34888602)(502.05583723,699.38388599)(502.17583962,699.40389374)
\curveto(502.29583699,699.43388594)(502.42083687,699.46388591)(502.55083962,699.49389374)
\curveto(502.61083668,699.51388586)(502.67083662,699.52388585)(502.73083962,699.52389374)
\lineto(502.91083962,699.55389374)
\curveto(502.9908363,699.56388581)(503.07583621,699.5688858)(503.16583962,699.56889374)
\curveto(503.25583603,699.5688858)(503.34083595,699.5738858)(503.42083962,699.58389374)
}
}
{
\newrgbcolor{curcolor}{0 0 0}
\pscustom[linestyle=none,fillstyle=solid,fillcolor=curcolor]
{
\newpath
\moveto(509.55748024,701.68389374)
\lineto(510.56248024,701.68389374)
\curveto(510.71247726,701.68388369)(510.84247713,701.6738837)(510.95248024,701.65389374)
\curveto(511.0724769,701.64388373)(511.15747681,701.58388379)(511.20748024,701.47389374)
\curveto(511.22747674,701.42388395)(511.23747673,701.36388401)(511.23748024,701.29389374)
\lineto(511.23748024,701.08389374)
\lineto(511.23748024,700.40889374)
\curveto(511.23747673,700.35888501)(511.23247674,700.29888507)(511.22248024,700.22889374)
\curveto(511.22247675,700.1688852)(511.22747674,700.11388526)(511.23748024,700.06389374)
\lineto(511.23748024,699.89889374)
\curveto(511.23747673,699.81888555)(511.24247673,699.74388563)(511.25248024,699.67389374)
\curveto(511.26247671,699.61388576)(511.28747668,699.55888581)(511.32748024,699.50889374)
\curveto(511.39747657,699.41888595)(511.52247645,699.368886)(511.70248024,699.35889374)
\lineto(512.24248024,699.35889374)
\lineto(512.42248024,699.35889374)
\curveto(512.48247549,699.35888601)(512.53747543,699.34888602)(512.58748024,699.32889374)
\curveto(512.69747527,699.27888609)(512.75747521,699.18888618)(512.76748024,699.05889374)
\curveto(512.78747518,698.92888644)(512.79747517,698.78388659)(512.79748024,698.62389374)
\lineto(512.79748024,698.41389374)
\curveto(512.80747516,698.34388703)(512.80247517,698.28388709)(512.78248024,698.23389374)
\curveto(512.73247524,698.0738873)(512.62747534,697.98888738)(512.46748024,697.97889374)
\curveto(512.30747566,697.9688874)(512.12747584,697.96388741)(511.92748024,697.96389374)
\lineto(511.79248024,697.96389374)
\curveto(511.75247622,697.9738874)(511.71747625,697.9738874)(511.68748024,697.96389374)
\curveto(511.64747632,697.95388742)(511.61247636,697.94888742)(511.58248024,697.94889374)
\curveto(511.55247642,697.95888741)(511.52247645,697.95388742)(511.49248024,697.93389374)
\curveto(511.41247656,697.91388746)(511.35247662,697.8688875)(511.31248024,697.79889374)
\curveto(511.28247669,697.73888763)(511.25747671,697.66388771)(511.23748024,697.57389374)
\curveto(511.22747674,697.52388785)(511.22747674,697.4688879)(511.23748024,697.40889374)
\curveto(511.24747672,697.34888802)(511.24747672,697.29388808)(511.23748024,697.24389374)
\lineto(511.23748024,696.31389374)
\lineto(511.23748024,694.55889374)
\curveto(511.23747673,694.30889106)(511.24247673,694.08889128)(511.25248024,693.89889374)
\curveto(511.2724767,693.71889165)(511.33747663,693.55889181)(511.44748024,693.41889374)
\curveto(511.49747647,693.35889201)(511.56247641,693.31389206)(511.64248024,693.28389374)
\lineto(511.91248024,693.22389374)
\curveto(511.94247603,693.21389216)(511.972476,693.20889216)(512.00248024,693.20889374)
\curveto(512.04247593,693.21889215)(512.0724759,693.21889215)(512.09248024,693.20889374)
\lineto(512.25748024,693.20889374)
\curveto(512.3674756,693.20889216)(512.46247551,693.20389217)(512.54248024,693.19389374)
\curveto(512.62247535,693.18389219)(512.68747528,693.14389223)(512.73748024,693.07389374)
\curveto(512.77747519,693.01389236)(512.79747517,692.93389244)(512.79748024,692.83389374)
\lineto(512.79748024,692.54889374)
\curveto(512.79747517,692.33889303)(512.79247518,692.14389323)(512.78248024,691.96389374)
\curveto(512.78247519,691.79389358)(512.70247527,691.67889369)(512.54248024,691.61889374)
\curveto(512.49247548,691.59889377)(512.44747552,691.59389378)(512.40748024,691.60389374)
\curveto(512.3674756,691.60389377)(512.32247565,691.59389378)(512.27248024,691.57389374)
\lineto(512.12248024,691.57389374)
\curveto(512.10247587,691.5738938)(512.0724759,691.57889379)(512.03248024,691.58889374)
\curveto(511.99247598,691.58889378)(511.95747601,691.58389379)(511.92748024,691.57389374)
\curveto(511.87747609,691.56389381)(511.82247615,691.56389381)(511.76248024,691.57389374)
\lineto(511.61248024,691.57389374)
\lineto(511.46248024,691.57389374)
\curveto(511.41247656,691.56389381)(511.3674766,691.56389381)(511.32748024,691.57389374)
\lineto(511.16248024,691.57389374)
\curveto(511.11247686,691.58389379)(511.05747691,691.58889378)(510.99748024,691.58889374)
\curveto(510.93747703,691.58889378)(510.88247709,691.59389378)(510.83248024,691.60389374)
\curveto(510.76247721,691.61389376)(510.69747727,691.62389375)(510.63748024,691.63389374)
\lineto(510.45748024,691.66389374)
\curveto(510.34747762,691.69389368)(510.24247773,691.72889364)(510.14248024,691.76889374)
\curveto(510.04247793,691.80889356)(509.94747802,691.85389352)(509.85748024,691.90389374)
\lineto(509.76748024,691.96389374)
\curveto(509.73747823,691.99389338)(509.70247827,692.02389335)(509.66248024,692.05389374)
\curveto(509.64247833,692.0738933)(509.61747835,692.09389328)(509.58748024,692.11389374)
\lineto(509.51248024,692.18889374)
\curveto(509.3724786,692.37889299)(509.2674787,692.58889278)(509.19748024,692.81889374)
\curveto(509.17747879,692.85889251)(509.1674788,692.89389248)(509.16748024,692.92389374)
\curveto(509.17747879,692.96389241)(509.17747879,693.00889236)(509.16748024,693.05889374)
\curveto(509.15747881,693.07889229)(509.15247882,693.10389227)(509.15248024,693.13389374)
\curveto(509.15247882,693.16389221)(509.14747882,693.18889218)(509.13748024,693.20889374)
\lineto(509.13748024,693.35889374)
\curveto(509.12747884,693.39889197)(509.12247885,693.44389193)(509.12248024,693.49389374)
\curveto(509.13247884,693.54389183)(509.13747883,693.59389178)(509.13748024,693.64389374)
\lineto(509.13748024,694.21389374)
\lineto(509.13748024,696.44889374)
\lineto(509.13748024,697.24389374)
\lineto(509.13748024,697.45389374)
\curveto(509.14747882,697.52388785)(509.14247883,697.58888778)(509.12248024,697.64889374)
\curveto(509.08247889,697.78888758)(509.01247896,697.87888749)(508.91248024,697.91889374)
\curveto(508.80247917,697.9688874)(508.66247931,697.98388739)(508.49248024,697.96389374)
\curveto(508.32247965,697.94388743)(508.17747979,697.95888741)(508.05748024,698.00889374)
\curveto(507.97747999,698.03888733)(507.92748004,698.08388729)(507.90748024,698.14389374)
\curveto(507.88748008,698.20388717)(507.8674801,698.27888709)(507.84748024,698.36889374)
\lineto(507.84748024,698.68389374)
\curveto(507.84748012,698.86388651)(507.85748011,699.00888636)(507.87748024,699.11889374)
\curveto(507.89748007,699.22888614)(507.98247999,699.30388607)(508.13248024,699.34389374)
\curveto(508.1724798,699.36388601)(508.21247976,699.368886)(508.25248024,699.35889374)
\lineto(508.38748024,699.35889374)
\curveto(508.53747943,699.35888601)(508.67747929,699.36388601)(508.80748024,699.37389374)
\curveto(508.93747903,699.39388598)(509.02747894,699.45388592)(509.07748024,699.55389374)
\curveto(509.10747886,699.62388575)(509.12247885,699.70388567)(509.12248024,699.79389374)
\curveto(509.13247884,699.88388549)(509.13747883,699.9738854)(509.13748024,700.06389374)
\lineto(509.13748024,700.99389374)
\lineto(509.13748024,701.24889374)
\curveto(509.13747883,701.33888403)(509.14747882,701.41388396)(509.16748024,701.47389374)
\curveto(509.21747875,701.5738838)(509.29247868,701.63888373)(509.39248024,701.66889374)
\curveto(509.41247856,701.67888369)(509.43747853,701.67888369)(509.46748024,701.66889374)
\curveto(509.50747846,701.6688837)(509.53747843,701.6738837)(509.55748024,701.68389374)
}
}
{
\newrgbcolor{curcolor}{0 0 0}
\pscustom[linestyle=none,fillstyle=solid,fillcolor=curcolor]
{
\newpath
\moveto(520.83091774,692.23389374)
\curveto(520.85090989,692.12389325)(520.86090988,692.01389336)(520.86091774,691.90389374)
\curveto(520.87090987,691.79389358)(520.82090992,691.71889365)(520.71091774,691.67889374)
\curveto(520.65091009,691.64889372)(520.58091016,691.63389374)(520.50091774,691.63389374)
\lineto(520.26091774,691.63389374)
\lineto(519.45091774,691.63389374)
\lineto(519.18091774,691.63389374)
\curveto(519.10091164,691.64389373)(519.03591171,691.6688937)(518.98591774,691.70889374)
\curveto(518.91591183,691.74889362)(518.86091188,691.80389357)(518.82091774,691.87389374)
\curveto(518.79091195,691.95389342)(518.745912,692.01889335)(518.68591774,692.06889374)
\curveto(518.66591208,692.08889328)(518.6409121,692.10389327)(518.61091774,692.11389374)
\curveto(518.58091216,692.13389324)(518.5409122,692.13889323)(518.49091774,692.12889374)
\curveto(518.4409123,692.10889326)(518.39091235,692.08389329)(518.34091774,692.05389374)
\curveto(518.30091244,692.02389335)(518.25591249,691.99889337)(518.20591774,691.97889374)
\curveto(518.15591259,691.93889343)(518.10091264,691.90389347)(518.04091774,691.87389374)
\lineto(517.86091774,691.78389374)
\curveto(517.73091301,691.72389365)(517.59591315,691.6738937)(517.45591774,691.63389374)
\curveto(517.31591343,691.60389377)(517.17091357,691.5688938)(517.02091774,691.52889374)
\curveto(516.95091379,691.50889386)(516.88091386,691.49889387)(516.81091774,691.49889374)
\curveto(516.75091399,691.48889388)(516.68591406,691.47889389)(516.61591774,691.46889374)
\lineto(516.52591774,691.46889374)
\curveto(516.49591425,691.45889391)(516.46591428,691.45389392)(516.43591774,691.45389374)
\lineto(516.27091774,691.45389374)
\curveto(516.17091457,691.43389394)(516.07091467,691.43389394)(515.97091774,691.45389374)
\lineto(515.83591774,691.45389374)
\curveto(515.76591498,691.4738939)(515.69591505,691.48389389)(515.62591774,691.48389374)
\curveto(515.56591518,691.4738939)(515.50591524,691.47889389)(515.44591774,691.49889374)
\curveto(515.3459154,691.51889385)(515.25091549,691.53889383)(515.16091774,691.55889374)
\curveto(515.07091567,691.5688938)(514.98591576,691.59389378)(514.90591774,691.63389374)
\curveto(514.61591613,691.74389363)(514.36591638,691.88389349)(514.15591774,692.05389374)
\curveto(513.95591679,692.23389314)(513.79591695,692.4688929)(513.67591774,692.75889374)
\curveto(513.6459171,692.82889254)(513.61591713,692.90389247)(513.58591774,692.98389374)
\curveto(513.56591718,693.06389231)(513.5459172,693.14889222)(513.52591774,693.23889374)
\curveto(513.50591724,693.28889208)(513.49591725,693.33889203)(513.49591774,693.38889374)
\curveto(513.50591724,693.43889193)(513.50591724,693.48889188)(513.49591774,693.53889374)
\curveto(513.48591726,693.5688918)(513.47591727,693.62889174)(513.46591774,693.71889374)
\curveto(513.46591728,693.81889155)(513.47091727,693.88889148)(513.48091774,693.92889374)
\curveto(513.50091724,694.02889134)(513.51091723,694.11389126)(513.51091774,694.18389374)
\lineto(513.60091774,694.51389374)
\curveto(513.63091711,694.63389074)(513.67091707,694.73889063)(513.72091774,694.82889374)
\curveto(513.89091685,695.11889025)(514.08591666,695.33889003)(514.30591774,695.48889374)
\curveto(514.52591622,695.63888973)(514.80591594,695.7688896)(515.14591774,695.87889374)
\curveto(515.27591547,695.92888944)(515.41091533,695.96388941)(515.55091774,695.98389374)
\curveto(515.69091505,696.00388937)(515.83091491,696.02888934)(515.97091774,696.05889374)
\curveto(516.05091469,696.07888929)(516.13591461,696.08888928)(516.22591774,696.08889374)
\curveto(516.31591443,696.09888927)(516.40591434,696.11388926)(516.49591774,696.13389374)
\curveto(516.56591418,696.15388922)(516.63591411,696.15888921)(516.70591774,696.14889374)
\curveto(516.77591397,696.14888922)(516.85091389,696.15888921)(516.93091774,696.17889374)
\curveto(517.00091374,696.19888917)(517.07091367,696.20888916)(517.14091774,696.20889374)
\curveto(517.21091353,696.20888916)(517.28591346,696.21888915)(517.36591774,696.23889374)
\curveto(517.57591317,696.28888908)(517.76591298,696.32888904)(517.93591774,696.35889374)
\curveto(518.11591263,696.39888897)(518.27591247,696.48888888)(518.41591774,696.62889374)
\curveto(518.50591224,696.71888865)(518.56591218,696.81888855)(518.59591774,696.92889374)
\curveto(518.60591214,696.95888841)(518.60591214,696.98388839)(518.59591774,697.00389374)
\curveto(518.59591215,697.02388835)(518.60091214,697.04388833)(518.61091774,697.06389374)
\curveto(518.62091212,697.08388829)(518.62591212,697.11388826)(518.62591774,697.15389374)
\lineto(518.62591774,697.24389374)
\lineto(518.59591774,697.36389374)
\curveto(518.59591215,697.40388797)(518.59091215,697.43888793)(518.58091774,697.46889374)
\curveto(518.48091226,697.7688876)(518.27091247,697.9738874)(517.95091774,698.08389374)
\curveto(517.86091288,698.11388726)(517.75091299,698.13388724)(517.62091774,698.14389374)
\curveto(517.50091324,698.16388721)(517.37591337,698.1688872)(517.24591774,698.15889374)
\curveto(517.11591363,698.15888721)(516.99091375,698.14888722)(516.87091774,698.12889374)
\curveto(516.75091399,698.10888726)(516.6459141,698.08388729)(516.55591774,698.05389374)
\curveto(516.49591425,698.03388734)(516.43591431,698.00388737)(516.37591774,697.96389374)
\curveto(516.32591442,697.93388744)(516.27591447,697.89888747)(516.22591774,697.85889374)
\curveto(516.17591457,697.81888755)(516.12091462,697.76388761)(516.06091774,697.69389374)
\curveto(516.01091473,697.62388775)(515.97591477,697.55888781)(515.95591774,697.49889374)
\curveto(515.90591484,697.39888797)(515.86091488,697.30388807)(515.82091774,697.21389374)
\curveto(515.79091495,697.12388825)(515.72091502,697.06388831)(515.61091774,697.03389374)
\curveto(515.53091521,697.01388836)(515.4459153,697.00388837)(515.35591774,697.00389374)
\lineto(515.08591774,697.00389374)
\lineto(514.51591774,697.00389374)
\curveto(514.46591628,697.00388837)(514.41591633,696.99888837)(514.36591774,696.98889374)
\curveto(514.31591643,696.98888838)(514.27091647,696.99388838)(514.23091774,697.00389374)
\lineto(514.09591774,697.00389374)
\curveto(514.07591667,697.01388836)(514.05091669,697.01888835)(514.02091774,697.01889374)
\curveto(513.99091675,697.01888835)(513.96591678,697.02888834)(513.94591774,697.04889374)
\curveto(513.86591688,697.0688883)(513.81091693,697.13388824)(513.78091774,697.24389374)
\curveto(513.77091697,697.29388808)(513.77091697,697.34388803)(513.78091774,697.39389374)
\curveto(513.79091695,697.44388793)(513.80091694,697.48888788)(513.81091774,697.52889374)
\curveto(513.8409169,697.63888773)(513.87091687,697.73888763)(513.90091774,697.82889374)
\curveto(513.9409168,697.92888744)(513.98591676,698.01888735)(514.03591774,698.09889374)
\lineto(514.12591774,698.24889374)
\lineto(514.21591774,698.39889374)
\curveto(514.29591645,698.50888686)(514.39591635,698.61388676)(514.51591774,698.71389374)
\curveto(514.53591621,698.72388665)(514.56591618,698.74888662)(514.60591774,698.78889374)
\curveto(514.65591609,698.82888654)(514.70091604,698.86388651)(514.74091774,698.89389374)
\curveto(514.78091596,698.92388645)(514.82591592,698.95388642)(514.87591774,698.98389374)
\curveto(515.0459157,699.09388628)(515.22591552,699.17888619)(515.41591774,699.23889374)
\curveto(515.60591514,699.30888606)(515.80091494,699.373886)(516.00091774,699.43389374)
\curveto(516.12091462,699.46388591)(516.2459145,699.48388589)(516.37591774,699.49389374)
\curveto(516.50591424,699.50388587)(516.63591411,699.52388585)(516.76591774,699.55389374)
\curveto(516.80591394,699.56388581)(516.86591388,699.56388581)(516.94591774,699.55389374)
\curveto(517.03591371,699.54388583)(517.09091365,699.54888582)(517.11091774,699.56889374)
\curveto(517.52091322,699.57888579)(517.91091283,699.56388581)(518.28091774,699.52389374)
\curveto(518.66091208,699.48388589)(519.00091174,699.40888596)(519.30091774,699.29889374)
\curveto(519.61091113,699.18888618)(519.87591087,699.03888633)(520.09591774,698.84889374)
\curveto(520.31591043,698.6688867)(520.48591026,698.43388694)(520.60591774,698.14389374)
\curveto(520.67591007,697.9738874)(520.71591003,697.77888759)(520.72591774,697.55889374)
\curveto(520.73591001,697.33888803)(520.74091,697.11388826)(520.74091774,696.88389374)
\lineto(520.74091774,693.53889374)
\lineto(520.74091774,692.95389374)
\curveto(520.74091,692.76389261)(520.76090998,692.58889278)(520.80091774,692.42889374)
\curveto(520.81090993,692.39889297)(520.81590993,692.36389301)(520.81591774,692.32389374)
\curveto(520.81590993,692.29389308)(520.82090992,692.26389311)(520.83091774,692.23389374)
\moveto(518.62591774,694.54389374)
\curveto(518.63591211,694.59389078)(518.6409121,694.64889072)(518.64091774,694.70889374)
\curveto(518.6409121,694.77889059)(518.63591211,694.83889053)(518.62591774,694.88889374)
\curveto(518.60591214,694.94889042)(518.59591215,695.00389037)(518.59591774,695.05389374)
\curveto(518.59591215,695.10389027)(518.57591217,695.14389023)(518.53591774,695.17389374)
\curveto(518.48591226,695.21389016)(518.41091233,695.23389014)(518.31091774,695.23389374)
\curveto(518.27091247,695.22389015)(518.23591251,695.21389016)(518.20591774,695.20389374)
\curveto(518.17591257,695.20389017)(518.1409126,695.19889017)(518.10091774,695.18889374)
\curveto(518.03091271,695.1688902)(517.95591279,695.15389022)(517.87591774,695.14389374)
\curveto(517.79591295,695.13389024)(517.71591303,695.11889025)(517.63591774,695.09889374)
\curveto(517.60591314,695.08889028)(517.56091318,695.08389029)(517.50091774,695.08389374)
\curveto(517.37091337,695.05389032)(517.2409135,695.03389034)(517.11091774,695.02389374)
\curveto(516.98091376,695.01389036)(516.85591389,694.98889038)(516.73591774,694.94889374)
\curveto(516.65591409,694.92889044)(516.58091416,694.90889046)(516.51091774,694.88889374)
\curveto(516.4409143,694.87889049)(516.37091437,694.85889051)(516.30091774,694.82889374)
\curveto(516.09091465,694.73889063)(515.91091483,694.60389077)(515.76091774,694.42389374)
\curveto(515.62091512,694.24389113)(515.57091517,693.99389138)(515.61091774,693.67389374)
\curveto(515.63091511,693.50389187)(515.68591506,693.36389201)(515.77591774,693.25389374)
\curveto(515.8459149,693.14389223)(515.95091479,693.05389232)(516.09091774,692.98389374)
\curveto(516.23091451,692.92389245)(516.38091436,692.87889249)(516.54091774,692.84889374)
\curveto(516.71091403,692.81889255)(516.88591386,692.80889256)(517.06591774,692.81889374)
\curveto(517.25591349,692.83889253)(517.43091331,692.8738925)(517.59091774,692.92389374)
\curveto(517.85091289,693.00389237)(518.05591269,693.12889224)(518.20591774,693.29889374)
\curveto(518.35591239,693.47889189)(518.47091227,693.69889167)(518.55091774,693.95889374)
\curveto(518.57091217,694.02889134)(518.58091216,694.09889127)(518.58091774,694.16889374)
\curveto(518.59091215,694.24889112)(518.60591214,694.32889104)(518.62591774,694.40889374)
\lineto(518.62591774,694.54389374)
}
}
{
\newrgbcolor{curcolor}{0 0 0}
\pscustom[linestyle=none,fillstyle=solid,fillcolor=curcolor]
{
\newpath
\moveto(530.19419899,695.89389374)
\curveto(530.21419039,695.83388954)(530.22419038,695.72888964)(530.22419899,695.57889374)
\curveto(530.22419038,695.43888993)(530.21919039,695.33889003)(530.20919899,695.27889374)
\curveto(530.2091904,695.22889014)(530.2041904,695.18389019)(530.19419899,695.14389374)
\lineto(530.19419899,695.02389374)
\curveto(530.17419043,694.94389043)(530.16419044,694.86389051)(530.16419899,694.78389374)
\curveto(530.16419044,694.71389066)(530.15419045,694.63889073)(530.13419899,694.55889374)
\curveto(530.13419047,694.51889085)(530.12419048,694.44889092)(530.10419899,694.34889374)
\curveto(530.07419053,694.22889114)(530.04419056,694.10389127)(530.01419899,693.97389374)
\curveto(529.99419061,693.85389152)(529.95919065,693.73889163)(529.90919899,693.62889374)
\curveto(529.72919088,693.17889219)(529.5041911,692.78889258)(529.23419899,692.45889374)
\curveto(528.96419164,692.12889324)(528.609192,691.8688935)(528.16919899,691.67889374)
\curveto(528.07919253,691.63889373)(527.98419262,691.60889376)(527.88419899,691.58889374)
\curveto(527.79419281,691.55889381)(527.69419291,691.52889384)(527.58419899,691.49889374)
\curveto(527.52419308,691.47889389)(527.45919315,691.4688939)(527.38919899,691.46889374)
\curveto(527.32919328,691.4688939)(527.26919334,691.46389391)(527.20919899,691.45389374)
\lineto(527.07419899,691.45389374)
\curveto(527.01419359,691.43389394)(526.93419367,691.42889394)(526.83419899,691.43889374)
\curveto(526.73419387,691.43889393)(526.65419395,691.44889392)(526.59419899,691.46889374)
\lineto(526.50419899,691.46889374)
\curveto(526.45419415,691.47889389)(526.39919421,691.48889388)(526.33919899,691.49889374)
\curveto(526.27919433,691.49889387)(526.21919439,691.50389387)(526.15919899,691.51389374)
\curveto(525.96919464,691.56389381)(525.79419481,691.61389376)(525.63419899,691.66389374)
\curveto(525.47419513,691.71389366)(525.32419528,691.78389359)(525.18419899,691.87389374)
\lineto(525.00419899,691.99389374)
\curveto(524.95419565,692.03389334)(524.9041957,692.07889329)(524.85419899,692.12889374)
\lineto(524.76419899,692.18889374)
\curveto(524.73419587,692.20889316)(524.7041959,692.22389315)(524.67419899,692.23389374)
\curveto(524.58419602,692.26389311)(524.52919608,692.24389313)(524.50919899,692.17389374)
\curveto(524.45919615,692.10389327)(524.42419618,692.01889335)(524.40419899,691.91889374)
\curveto(524.39419621,691.82889354)(524.35919625,691.75889361)(524.29919899,691.70889374)
\curveto(524.23919637,691.6688937)(524.16919644,691.64389373)(524.08919899,691.63389374)
\lineto(523.81919899,691.63389374)
\lineto(523.09919899,691.63389374)
\lineto(522.87419899,691.63389374)
\curveto(522.8041978,691.62389375)(522.73919787,691.62889374)(522.67919899,691.64889374)
\curveto(522.53919807,691.69889367)(522.45919815,691.78889358)(522.43919899,691.91889374)
\curveto(522.42919818,692.05889331)(522.42419818,692.21389316)(522.42419899,692.38389374)
\lineto(522.42419899,701.53389374)
\lineto(522.42419899,701.87889374)
\curveto(522.42419818,701.99888337)(522.44919816,702.09388328)(522.49919899,702.16389374)
\curveto(522.53919807,702.23388314)(522.609198,702.27888309)(522.70919899,702.29889374)
\curveto(522.72919788,702.30888306)(522.74919786,702.30888306)(522.76919899,702.29889374)
\curveto(522.79919781,702.29888307)(522.82419778,702.30388307)(522.84419899,702.31389374)
\lineto(523.78919899,702.31389374)
\curveto(523.96919664,702.31388306)(524.12419648,702.30388307)(524.25419899,702.28389374)
\curveto(524.38419622,702.2738831)(524.46919614,702.19888317)(524.50919899,702.05889374)
\curveto(524.53919607,701.95888341)(524.54919606,701.82388355)(524.53919899,701.65389374)
\curveto(524.52919608,701.49388388)(524.52419608,701.35388402)(524.52419899,701.23389374)
\lineto(524.52419899,699.59889374)
\lineto(524.52419899,699.26889374)
\curveto(524.52419608,699.15888621)(524.53419607,699.06388631)(524.55419899,698.98389374)
\curveto(524.56419604,698.93388644)(524.57419603,698.88888648)(524.58419899,698.84889374)
\curveto(524.59419601,698.81888655)(524.61919599,698.79888657)(524.65919899,698.78889374)
\curveto(524.67919593,698.7688866)(524.7041959,698.75888661)(524.73419899,698.75889374)
\curveto(524.77419583,698.75888661)(524.8041958,698.76388661)(524.82419899,698.77389374)
\curveto(524.89419571,698.81388656)(524.95919565,698.85388652)(525.01919899,698.89389374)
\curveto(525.07919553,698.94388643)(525.14419546,698.99388638)(525.21419899,699.04389374)
\curveto(525.34419526,699.13388624)(525.47919513,699.20888616)(525.61919899,699.26889374)
\curveto(525.75919485,699.33888603)(525.91419469,699.39888597)(526.08419899,699.44889374)
\curveto(526.16419444,699.47888589)(526.24419436,699.49388588)(526.32419899,699.49389374)
\curveto(526.4041942,699.50388587)(526.48419412,699.51888585)(526.56419899,699.53889374)
\curveto(526.63419397,699.55888581)(526.7091939,699.5688858)(526.78919899,699.56889374)
\lineto(527.02919899,699.56889374)
\lineto(527.17919899,699.56889374)
\curveto(527.2091934,699.55888581)(527.24419336,699.55388582)(527.28419899,699.55389374)
\curveto(527.32419328,699.56388581)(527.36419324,699.56388581)(527.40419899,699.55389374)
\curveto(527.51419309,699.52388585)(527.61419299,699.49888587)(527.70419899,699.47889374)
\curveto(527.8041928,699.4688859)(527.89919271,699.44388593)(527.98919899,699.40389374)
\curveto(528.44919216,699.21388616)(528.82419178,698.9688864)(529.11419899,698.66889374)
\curveto(529.4041912,698.368887)(529.64919096,697.99388738)(529.84919899,697.54389374)
\curveto(529.89919071,697.42388795)(529.93919067,697.29888807)(529.96919899,697.16889374)
\curveto(530.0091906,697.03888833)(530.04919056,696.90388847)(530.08919899,696.76389374)
\curveto(530.1091905,696.69388868)(530.11919049,696.62388875)(530.11919899,696.55389374)
\curveto(530.12919048,696.49388888)(530.14419046,696.42388895)(530.16419899,696.34389374)
\curveto(530.18419042,696.29388908)(530.18919042,696.23888913)(530.17919899,696.17889374)
\curveto(530.17919043,696.11888925)(530.18419042,696.05888931)(530.19419899,695.99889374)
\lineto(530.19419899,695.89389374)
\moveto(527.97419899,694.48389374)
\curveto(528.0041926,694.58389079)(528.02919258,694.70889066)(528.04919899,694.85889374)
\curveto(528.07919253,695.00889036)(528.09419251,695.15889021)(528.09419899,695.30889374)
\curveto(528.1041925,695.4688899)(528.1041925,695.62388975)(528.09419899,695.77389374)
\curveto(528.09419251,695.93388944)(528.07919253,696.0688893)(528.04919899,696.17889374)
\curveto(528.01919259,696.27888909)(527.99919261,696.373889)(527.98919899,696.46389374)
\curveto(527.97919263,696.55388882)(527.95419265,696.63888873)(527.91419899,696.71889374)
\curveto(527.77419283,697.0688883)(527.57419303,697.36388801)(527.31419899,697.60389374)
\curveto(527.06419354,697.85388752)(526.69419391,697.97888739)(526.20419899,697.97889374)
\curveto(526.16419444,697.97888739)(526.12919448,697.9738874)(526.09919899,697.96389374)
\lineto(525.99419899,697.96389374)
\curveto(525.92419468,697.94388743)(525.85919475,697.92388745)(525.79919899,697.90389374)
\curveto(525.73919487,697.89388748)(525.67919493,697.87888749)(525.61919899,697.85889374)
\curveto(525.32919528,697.72888764)(525.1091955,697.54388783)(524.95919899,697.30389374)
\curveto(524.8091958,697.0738883)(524.68419592,696.80888856)(524.58419899,696.50889374)
\curveto(524.55419605,696.42888894)(524.53419607,696.34388903)(524.52419899,696.25389374)
\curveto(524.52419608,696.1738892)(524.51419609,696.09388928)(524.49419899,696.01389374)
\curveto(524.48419612,695.98388939)(524.47919613,695.93388944)(524.47919899,695.86389374)
\curveto(524.46919614,695.82388955)(524.46419614,695.78388959)(524.46419899,695.74389374)
\curveto(524.47419613,695.70388967)(524.47419613,695.66388971)(524.46419899,695.62389374)
\curveto(524.44419616,695.54388983)(524.43919617,695.43388994)(524.44919899,695.29389374)
\curveto(524.45919615,695.15389022)(524.47419613,695.05389032)(524.49419899,694.99389374)
\curveto(524.51419609,694.90389047)(524.52419608,694.81889055)(524.52419899,694.73889374)
\curveto(524.53419607,694.65889071)(524.55419605,694.57889079)(524.58419899,694.49889374)
\curveto(524.67419593,694.21889115)(524.77919583,693.9738914)(524.89919899,693.76389374)
\curveto(525.02919558,693.56389181)(525.2091954,693.39389198)(525.43919899,693.25389374)
\curveto(525.59919501,693.15389222)(525.76419484,693.08389229)(525.93419899,693.04389374)
\curveto(525.95419465,693.04389233)(525.97419463,693.03889233)(525.99419899,693.02889374)
\lineto(526.08419899,693.02889374)
\curveto(526.11419449,693.01889235)(526.16419444,693.00889236)(526.23419899,692.99889374)
\curveto(526.3041943,692.99889237)(526.36419424,693.00389237)(526.41419899,693.01389374)
\curveto(526.51419409,693.03389234)(526.604194,693.04889232)(526.68419899,693.05889374)
\curveto(526.77419383,693.07889229)(526.85919375,693.10389227)(526.93919899,693.13389374)
\curveto(527.21919339,693.26389211)(527.43419317,693.44389193)(527.58419899,693.67389374)
\curveto(527.74419286,693.90389147)(527.87419273,694.1738912)(527.97419899,694.48389374)
}
}
{
\newrgbcolor{curcolor}{0 0 0}
\pscustom[linestyle=none,fillstyle=solid,fillcolor=curcolor]
{
\newpath
\moveto(532.07412087,702.32889374)
\lineto(533.16912087,702.32889374)
\curveto(533.26911838,702.32888304)(533.36411829,702.32388305)(533.45412087,702.31389374)
\curveto(533.54411811,702.30388307)(533.61411804,702.2738831)(533.66412087,702.22389374)
\curveto(533.72411793,702.15388322)(533.7541179,702.05888331)(533.75412087,701.93889374)
\curveto(533.76411789,701.82888354)(533.76911788,701.71388366)(533.76912087,701.59389374)
\lineto(533.76912087,700.25889374)
\lineto(533.76912087,694.87389374)
\lineto(533.76912087,692.57889374)
\lineto(533.76912087,692.15889374)
\curveto(533.77911787,692.00889336)(533.75911789,691.89389348)(533.70912087,691.81389374)
\curveto(533.65911799,691.73389364)(533.56911808,691.67889369)(533.43912087,691.64889374)
\curveto(533.37911827,691.62889374)(533.30911834,691.62389375)(533.22912087,691.63389374)
\curveto(533.15911849,691.64389373)(533.08911856,691.64889372)(533.01912087,691.64889374)
\lineto(532.29912087,691.64889374)
\curveto(532.18911946,691.64889372)(532.08911956,691.65389372)(531.99912087,691.66389374)
\curveto(531.90911974,691.6738937)(531.83411982,691.70389367)(531.77412087,691.75389374)
\curveto(531.71411994,691.80389357)(531.67911997,691.87889349)(531.66912087,691.97889374)
\lineto(531.66912087,692.30889374)
\lineto(531.66912087,693.64389374)
\lineto(531.66912087,699.26889374)
\lineto(531.66912087,701.30889374)
\curveto(531.66911998,701.43888393)(531.66411999,701.59388378)(531.65412087,701.77389374)
\curveto(531.65412,701.95388342)(531.67911997,702.08388329)(531.72912087,702.16389374)
\curveto(531.7491199,702.20388317)(531.77411988,702.23388314)(531.80412087,702.25389374)
\lineto(531.92412087,702.31389374)
\curveto(531.94411971,702.31388306)(531.96911968,702.31388306)(531.99912087,702.31389374)
\curveto(532.02911962,702.32388305)(532.0541196,702.32888304)(532.07412087,702.32889374)
}
}
{
\newrgbcolor{curcolor}{0 0 0}
\pscustom[linestyle=none,fillstyle=solid,fillcolor=curcolor]
{
\newpath
\moveto(542.79630837,695.57889374)
\curveto(542.8163002,695.49888987)(542.8163002,695.40888996)(542.79630837,695.30889374)
\curveto(542.77630024,695.20889016)(542.74130028,695.14389023)(542.69130837,695.11389374)
\curveto(542.64130038,695.0738903)(542.56630045,695.04389033)(542.46630837,695.02389374)
\curveto(542.37630064,695.01389036)(542.27130075,695.00389037)(542.15130837,694.99389374)
\lineto(541.80630837,694.99389374)
\curveto(541.69630132,695.00389037)(541.59630142,695.00889036)(541.50630837,695.00889374)
\lineto(537.84630837,695.00889374)
\lineto(537.63630837,695.00889374)
\curveto(537.57630544,695.00889036)(537.5213055,694.99889037)(537.47130837,694.97889374)
\curveto(537.39130563,694.93889043)(537.34130568,694.89889047)(537.32130837,694.85889374)
\curveto(537.30130572,694.83889053)(537.28130574,694.79889057)(537.26130837,694.73889374)
\curveto(537.24130578,694.68889068)(537.23630578,694.63889073)(537.24630837,694.58889374)
\curveto(537.26630575,694.52889084)(537.27630574,694.4688909)(537.27630837,694.40889374)
\curveto(537.28630573,694.35889101)(537.30130572,694.30389107)(537.32130837,694.24389374)
\curveto(537.40130562,694.00389137)(537.49630552,693.80389157)(537.60630837,693.64389374)
\curveto(537.72630529,693.49389188)(537.88630513,693.35889201)(538.08630837,693.23889374)
\curveto(538.16630485,693.18889218)(538.24630477,693.15389222)(538.32630837,693.13389374)
\curveto(538.4163046,693.12389225)(538.50630451,693.10389227)(538.59630837,693.07389374)
\curveto(538.67630434,693.05389232)(538.78630423,693.03889233)(538.92630837,693.02889374)
\curveto(539.06630395,693.01889235)(539.18630383,693.02389235)(539.28630837,693.04389374)
\lineto(539.42130837,693.04389374)
\curveto(539.5213035,693.06389231)(539.61130341,693.08389229)(539.69130837,693.10389374)
\curveto(539.78130324,693.13389224)(539.86630315,693.16389221)(539.94630837,693.19389374)
\curveto(540.04630297,693.24389213)(540.15630286,693.30889206)(540.27630837,693.38889374)
\curveto(540.40630261,693.4688919)(540.50130252,693.54889182)(540.56130837,693.62889374)
\curveto(540.61130241,693.69889167)(540.66130236,693.76389161)(540.71130837,693.82389374)
\curveto(540.77130225,693.89389148)(540.84130218,693.94389143)(540.92130837,693.97389374)
\curveto(541.021302,694.02389135)(541.14630187,694.04389133)(541.29630837,694.03389374)
\lineto(541.73130837,694.03389374)
\lineto(541.91130837,694.03389374)
\curveto(541.98130104,694.04389133)(542.04130098,694.03889133)(542.09130837,694.01889374)
\lineto(542.24130837,694.01889374)
\curveto(542.34130068,693.99889137)(542.41130061,693.9738914)(542.45130837,693.94389374)
\curveto(542.49130053,693.92389145)(542.51130051,693.87889149)(542.51130837,693.80889374)
\curveto(542.5213005,693.73889163)(542.5163005,693.67889169)(542.49630837,693.62889374)
\curveto(542.44630057,693.48889188)(542.39130063,693.36389201)(542.33130837,693.25389374)
\curveto(542.27130075,693.14389223)(542.20130082,693.03389234)(542.12130837,692.92389374)
\curveto(541.90130112,692.59389278)(541.65130137,692.32889304)(541.37130837,692.12889374)
\curveto(541.09130193,691.92889344)(540.74130228,691.75889361)(540.32130837,691.61889374)
\curveto(540.21130281,691.57889379)(540.10130292,691.55389382)(539.99130837,691.54389374)
\curveto(539.88130314,691.53389384)(539.76630325,691.51389386)(539.64630837,691.48389374)
\curveto(539.60630341,691.4738939)(539.56130346,691.4738939)(539.51130837,691.48389374)
\curveto(539.47130355,691.48389389)(539.43130359,691.47889389)(539.39130837,691.46889374)
\lineto(539.22630837,691.46889374)
\curveto(539.17630384,691.44889392)(539.1163039,691.44389393)(539.04630837,691.45389374)
\curveto(538.98630403,691.45389392)(538.93130409,691.45889391)(538.88130837,691.46889374)
\curveto(538.80130422,691.47889389)(538.73130429,691.47889389)(538.67130837,691.46889374)
\curveto(538.61130441,691.45889391)(538.54630447,691.46389391)(538.47630837,691.48389374)
\curveto(538.42630459,691.50389387)(538.37130465,691.51389386)(538.31130837,691.51389374)
\curveto(538.25130477,691.51389386)(538.19630482,691.52389385)(538.14630837,691.54389374)
\curveto(538.03630498,691.56389381)(537.92630509,691.58889378)(537.81630837,691.61889374)
\curveto(537.70630531,691.63889373)(537.60630541,691.6738937)(537.51630837,691.72389374)
\curveto(537.40630561,691.76389361)(537.30130572,691.79889357)(537.20130837,691.82889374)
\curveto(537.11130591,691.8688935)(537.02630599,691.91389346)(536.94630837,691.96389374)
\curveto(536.62630639,692.16389321)(536.34130668,692.39389298)(536.09130837,692.65389374)
\curveto(535.84130718,692.92389245)(535.63630738,693.23389214)(535.47630837,693.58389374)
\curveto(535.42630759,693.69389168)(535.38630763,693.80389157)(535.35630837,693.91389374)
\curveto(535.32630769,694.03389134)(535.28630773,694.15389122)(535.23630837,694.27389374)
\curveto(535.22630779,694.31389106)(535.2213078,694.34889102)(535.22130837,694.37889374)
\curveto(535.2213078,694.41889095)(535.2163078,694.45889091)(535.20630837,694.49889374)
\curveto(535.16630785,694.61889075)(535.14130788,694.74889062)(535.13130837,694.88889374)
\lineto(535.10130837,695.30889374)
\curveto(535.10130792,695.35889001)(535.09630792,695.41388996)(535.08630837,695.47389374)
\curveto(535.08630793,695.53388984)(535.09130793,695.58888978)(535.10130837,695.63889374)
\lineto(535.10130837,695.81889374)
\lineto(535.14630837,696.17889374)
\curveto(535.18630783,696.34888902)(535.2213078,696.51388886)(535.25130837,696.67389374)
\curveto(535.28130774,696.83388854)(535.32630769,696.98388839)(535.38630837,697.12389374)
\curveto(535.8163072,698.16388721)(536.54630647,698.89888647)(537.57630837,699.32889374)
\curveto(537.7163053,699.38888598)(537.85630516,699.42888594)(537.99630837,699.44889374)
\curveto(538.14630487,699.47888589)(538.30130472,699.51388586)(538.46130837,699.55389374)
\curveto(538.54130448,699.56388581)(538.6163044,699.5688858)(538.68630837,699.56889374)
\curveto(538.75630426,699.5688858)(538.83130419,699.5738858)(538.91130837,699.58389374)
\curveto(539.4213036,699.59388578)(539.85630316,699.53388584)(540.21630837,699.40389374)
\curveto(540.58630243,699.28388609)(540.9163021,699.12388625)(541.20630837,698.92389374)
\curveto(541.29630172,698.86388651)(541.38630163,698.79388658)(541.47630837,698.71389374)
\curveto(541.56630145,698.64388673)(541.64630137,698.5688868)(541.71630837,698.48889374)
\curveto(541.74630127,698.43888693)(541.78630123,698.39888697)(541.83630837,698.36889374)
\curveto(541.9163011,698.25888711)(541.99130103,698.14388723)(542.06130837,698.02389374)
\curveto(542.13130089,697.91388746)(542.20630081,697.79888757)(542.28630837,697.67889374)
\curveto(542.33630068,697.58888778)(542.37630064,697.49388788)(542.40630837,697.39389374)
\curveto(542.44630057,697.30388807)(542.48630053,697.20388817)(542.52630837,697.09389374)
\curveto(542.57630044,696.96388841)(542.6163004,696.82888854)(542.64630837,696.68889374)
\curveto(542.67630034,696.54888882)(542.71130031,696.40888896)(542.75130837,696.26889374)
\curveto(542.77130025,696.18888918)(542.77630024,696.09888927)(542.76630837,695.99889374)
\curveto(542.76630025,695.90888946)(542.77630024,695.82388955)(542.79630837,695.74389374)
\lineto(542.79630837,695.57889374)
\moveto(540.54630837,696.46389374)
\curveto(540.6163024,696.56388881)(540.6213024,696.68388869)(540.56130837,696.82389374)
\curveto(540.51130251,696.9738884)(540.47130255,697.08388829)(540.44130837,697.15389374)
\curveto(540.30130272,697.42388795)(540.1163029,697.62888774)(539.88630837,697.76889374)
\curveto(539.65630336,697.91888745)(539.33630368,697.99888737)(538.92630837,698.00889374)
\curveto(538.89630412,697.98888738)(538.86130416,697.98388739)(538.82130837,697.99389374)
\curveto(538.78130424,698.00388737)(538.74630427,698.00388737)(538.71630837,697.99389374)
\curveto(538.66630435,697.9738874)(538.61130441,697.95888741)(538.55130837,697.94889374)
\curveto(538.49130453,697.94888742)(538.43630458,697.93888743)(538.38630837,697.91889374)
\curveto(537.94630507,697.77888759)(537.6213054,697.50388787)(537.41130837,697.09389374)
\curveto(537.39130563,697.05388832)(537.36630565,696.99888837)(537.33630837,696.92889374)
\curveto(537.3163057,696.8688885)(537.30130572,696.80388857)(537.29130837,696.73389374)
\curveto(537.28130574,696.6738887)(537.28130574,696.61388876)(537.29130837,696.55389374)
\curveto(537.31130571,696.49388888)(537.34630567,696.44388893)(537.39630837,696.40389374)
\curveto(537.47630554,696.35388902)(537.58630543,696.32888904)(537.72630837,696.32889374)
\lineto(538.13130837,696.32889374)
\lineto(539.79630837,696.32889374)
\lineto(540.23130837,696.32889374)
\curveto(540.39130263,696.33888903)(540.49630252,696.38388899)(540.54630837,696.46389374)
}
}
{
\newrgbcolor{curcolor}{0 0 0}
\pscustom[linestyle=none,fillstyle=solid,fillcolor=curcolor]
{
\newpath
\moveto(547.61458962,699.58389374)
\curveto(548.42458446,699.60388577)(549.09958378,699.48388589)(549.63958962,699.22389374)
\curveto(550.18958269,698.96388641)(550.62458226,698.59388678)(550.94458962,698.11389374)
\curveto(551.10458178,697.8738875)(551.22458166,697.59888777)(551.30458962,697.28889374)
\curveto(551.32458156,697.23888813)(551.33958154,697.1738882)(551.34958962,697.09389374)
\curveto(551.36958151,697.01388836)(551.36958151,696.94388843)(551.34958962,696.88389374)
\curveto(551.30958157,696.7738886)(551.23958164,696.70888866)(551.13958962,696.68889374)
\curveto(551.03958184,696.67888869)(550.91958196,696.6738887)(550.77958962,696.67389374)
\lineto(549.99958962,696.67389374)
\lineto(549.71458962,696.67389374)
\curveto(549.62458326,696.6738887)(549.54958333,696.69388868)(549.48958962,696.73389374)
\curveto(549.40958347,696.7738886)(549.35458353,696.83388854)(549.32458962,696.91389374)
\curveto(549.29458359,697.00388837)(549.25458363,697.09388828)(549.20458962,697.18389374)
\curveto(549.14458374,697.29388808)(549.0795838,697.39388798)(549.00958962,697.48389374)
\curveto(548.93958394,697.5738878)(548.85958402,697.65388772)(548.76958962,697.72389374)
\curveto(548.62958425,697.81388756)(548.47458441,697.88388749)(548.30458962,697.93389374)
\curveto(548.24458464,697.95388742)(548.1845847,697.96388741)(548.12458962,697.96389374)
\curveto(548.06458482,697.96388741)(548.00958487,697.9738874)(547.95958962,697.99389374)
\lineto(547.80958962,697.99389374)
\curveto(547.60958527,697.99388738)(547.44958543,697.9738874)(547.32958962,697.93389374)
\curveto(547.03958584,697.84388753)(546.80458608,697.70388767)(546.62458962,697.51389374)
\curveto(546.44458644,697.33388804)(546.29958658,697.11388826)(546.18958962,696.85389374)
\curveto(546.13958674,696.74388863)(546.09958678,696.62388875)(546.06958962,696.49389374)
\curveto(546.04958683,696.373889)(546.02458686,696.24388913)(545.99458962,696.10389374)
\curveto(545.9845869,696.06388931)(545.9795869,696.02388935)(545.97958962,695.98389374)
\curveto(545.9795869,695.94388943)(545.97458691,695.90388947)(545.96458962,695.86389374)
\curveto(545.94458694,695.76388961)(545.93458695,695.62388975)(545.93458962,695.44389374)
\curveto(545.94458694,695.26389011)(545.95958692,695.12389025)(545.97958962,695.02389374)
\curveto(545.9795869,694.94389043)(545.9845869,694.88889048)(545.99458962,694.85889374)
\curveto(546.01458687,694.78889058)(546.02458686,694.71889065)(546.02458962,694.64889374)
\curveto(546.03458685,694.57889079)(546.04958683,694.50889086)(546.06958962,694.43889374)
\curveto(546.14958673,694.20889116)(546.24458664,693.99889137)(546.35458962,693.80889374)
\curveto(546.46458642,693.61889175)(546.60458628,693.45889191)(546.77458962,693.32889374)
\curveto(546.81458607,693.29889207)(546.87458601,693.26389211)(546.95458962,693.22389374)
\curveto(547.06458582,693.15389222)(547.17458571,693.10889226)(547.28458962,693.08889374)
\curveto(547.40458548,693.0688923)(547.54958533,693.04889232)(547.71958962,693.02889374)
\lineto(547.80958962,693.02889374)
\curveto(547.84958503,693.02889234)(547.879585,693.03389234)(547.89958962,693.04389374)
\lineto(548.03458962,693.04389374)
\curveto(548.10458478,693.06389231)(548.16958471,693.07889229)(548.22958962,693.08889374)
\curveto(548.29958458,693.10889226)(548.36458452,693.12889224)(548.42458962,693.14889374)
\curveto(548.72458416,693.27889209)(548.95458393,693.4688919)(549.11458962,693.71889374)
\curveto(549.15458373,693.7688916)(549.18958369,693.82389155)(549.21958962,693.88389374)
\curveto(549.24958363,693.95389142)(549.27458361,694.01389136)(549.29458962,694.06389374)
\curveto(549.33458355,694.1738912)(549.36958351,694.2688911)(549.39958962,694.34889374)
\curveto(549.42958345,694.43889093)(549.49958338,694.50889086)(549.60958962,694.55889374)
\curveto(549.69958318,694.59889077)(549.84458304,694.61389076)(550.04458962,694.60389374)
\lineto(550.53958962,694.60389374)
\lineto(550.74958962,694.60389374)
\curveto(550.82958205,694.61389076)(550.89458199,694.60889076)(550.94458962,694.58889374)
\lineto(551.06458962,694.58889374)
\lineto(551.18458962,694.55889374)
\curveto(551.22458166,694.55889081)(551.25458163,694.54889082)(551.27458962,694.52889374)
\curveto(551.32458156,694.48889088)(551.35458153,694.42889094)(551.36458962,694.34889374)
\curveto(551.3845815,694.27889109)(551.3845815,694.20389117)(551.36458962,694.12389374)
\curveto(551.27458161,693.79389158)(551.16458172,693.49889187)(551.03458962,693.23889374)
\curveto(550.62458226,692.4688929)(549.96958291,691.93389344)(549.06958962,691.63389374)
\curveto(548.96958391,691.60389377)(548.86458402,691.58389379)(548.75458962,691.57389374)
\curveto(548.64458424,691.55389382)(548.53458435,691.52889384)(548.42458962,691.49889374)
\curveto(548.36458452,691.48889388)(548.30458458,691.48389389)(548.24458962,691.48389374)
\curveto(548.1845847,691.48389389)(548.12458476,691.47889389)(548.06458962,691.46889374)
\lineto(547.89958962,691.46889374)
\curveto(547.84958503,691.44889392)(547.77458511,691.44389393)(547.67458962,691.45389374)
\curveto(547.57458531,691.45389392)(547.49958538,691.45889391)(547.44958962,691.46889374)
\curveto(547.36958551,691.48889388)(547.29458559,691.49889387)(547.22458962,691.49889374)
\curveto(547.16458572,691.48889388)(547.09958578,691.49389388)(547.02958962,691.51389374)
\lineto(546.87958962,691.54389374)
\curveto(546.82958605,691.54389383)(546.7795861,691.54889382)(546.72958962,691.55889374)
\curveto(546.61958626,691.58889378)(546.51458637,691.61889375)(546.41458962,691.64889374)
\curveto(546.31458657,691.67889369)(546.21958666,691.71389366)(546.12958962,691.75389374)
\curveto(545.65958722,691.95389342)(545.26458762,692.20889316)(544.94458962,692.51889374)
\curveto(544.62458826,692.83889253)(544.36458852,693.23389214)(544.16458962,693.70389374)
\curveto(544.11458877,693.79389158)(544.07458881,693.88889148)(544.04458962,693.98889374)
\lineto(543.95458962,694.31889374)
\curveto(543.94458894,694.35889101)(543.93958894,694.39389098)(543.93958962,694.42389374)
\curveto(543.93958894,694.46389091)(543.92958895,694.50889086)(543.90958962,694.55889374)
\curveto(543.88958899,694.62889074)(543.879589,694.69889067)(543.87958962,694.76889374)
\curveto(543.879589,694.84889052)(543.86958901,694.92389045)(543.84958962,694.99389374)
\lineto(543.84958962,695.24889374)
\curveto(543.82958905,695.29889007)(543.81958906,695.35389002)(543.81958962,695.41389374)
\curveto(543.81958906,695.48388989)(543.82958905,695.54388983)(543.84958962,695.59389374)
\curveto(543.85958902,695.64388973)(543.85958902,695.68888968)(543.84958962,695.72889374)
\curveto(543.83958904,695.7688896)(543.83958904,695.80888956)(543.84958962,695.84889374)
\curveto(543.86958901,695.91888945)(543.87458901,695.98388939)(543.86458962,696.04389374)
\curveto(543.86458902,696.10388927)(543.87458901,696.16388921)(543.89458962,696.22389374)
\curveto(543.94458894,696.40388897)(543.9845889,696.5738888)(544.01458962,696.73389374)
\curveto(544.04458884,696.90388847)(544.08958879,697.0688883)(544.14958962,697.22889374)
\curveto(544.36958851,697.73888763)(544.64458824,698.16388721)(544.97458962,698.50389374)
\curveto(545.31458757,698.84388653)(545.74458714,699.11888625)(546.26458962,699.32889374)
\curveto(546.40458648,699.38888598)(546.54958633,699.42888594)(546.69958962,699.44889374)
\curveto(546.84958603,699.47888589)(547.00458588,699.51388586)(547.16458962,699.55389374)
\curveto(547.24458564,699.56388581)(547.31958556,699.5688858)(547.38958962,699.56889374)
\curveto(547.45958542,699.5688858)(547.53458535,699.5738858)(547.61458962,699.58389374)
}
}
{
\newrgbcolor{curcolor}{0 0 0}
\pscustom[linestyle=none,fillstyle=solid,fillcolor=curcolor]
{
\newpath
\moveto(554.75787087,702.22389374)
\curveto(554.82786792,702.14388323)(554.86286788,702.02388335)(554.86287087,701.86389374)
\lineto(554.86287087,701.39889374)
\lineto(554.86287087,700.99389374)
\curveto(554.86286788,700.85388452)(554.82786792,700.75888461)(554.75787087,700.70889374)
\curveto(554.69786805,700.65888471)(554.61786813,700.62888474)(554.51787087,700.61889374)
\curveto(554.42786832,700.60888476)(554.32786842,700.60388477)(554.21787087,700.60389374)
\lineto(553.37787087,700.60389374)
\curveto(553.26786948,700.60388477)(553.16786958,700.60888476)(553.07787087,700.61889374)
\curveto(552.99786975,700.62888474)(552.92786982,700.65888471)(552.86787087,700.70889374)
\curveto(552.82786992,700.73888463)(552.79786995,700.79388458)(552.77787087,700.87389374)
\curveto(552.76786998,700.96388441)(552.75786999,701.05888431)(552.74787087,701.15889374)
\lineto(552.74787087,701.48889374)
\curveto(552.75786999,701.59888377)(552.76286998,701.69388368)(552.76287087,701.77389374)
\lineto(552.76287087,701.98389374)
\curveto(552.77286997,702.05388332)(552.79286995,702.11388326)(552.82287087,702.16389374)
\curveto(552.8428699,702.20388317)(552.86786988,702.23388314)(552.89787087,702.25389374)
\lineto(553.01787087,702.31389374)
\curveto(553.03786971,702.31388306)(553.06286968,702.31388306)(553.09287087,702.31389374)
\curveto(553.12286962,702.32388305)(553.1478696,702.32888304)(553.16787087,702.32889374)
\lineto(554.26287087,702.32889374)
\curveto(554.36286838,702.32888304)(554.45786829,702.32388305)(554.54787087,702.31389374)
\curveto(554.63786811,702.30388307)(554.70786804,702.2738831)(554.75787087,702.22389374)
\moveto(554.86287087,692.45889374)
\curveto(554.86286788,692.25889311)(554.85786789,692.08889328)(554.84787087,691.94889374)
\curveto(554.83786791,691.80889356)(554.747868,691.71389366)(554.57787087,691.66389374)
\curveto(554.51786823,691.64389373)(554.45286829,691.63389374)(554.38287087,691.63389374)
\curveto(554.31286843,691.64389373)(554.23786851,691.64889372)(554.15787087,691.64889374)
\lineto(553.31787087,691.64889374)
\curveto(553.22786952,691.64889372)(553.13786961,691.65389372)(553.04787087,691.66389374)
\curveto(552.96786978,691.6738937)(552.90786984,691.70389367)(552.86787087,691.75389374)
\curveto(552.80786994,691.82389355)(552.77286997,691.90889346)(552.76287087,692.00889374)
\lineto(552.76287087,692.35389374)
\lineto(552.76287087,698.68389374)
\lineto(552.76287087,698.98389374)
\curveto(552.76286998,699.08388629)(552.78286996,699.16388621)(552.82287087,699.22389374)
\curveto(552.88286986,699.29388608)(552.96786978,699.33888603)(553.07787087,699.35889374)
\curveto(553.09786965,699.368886)(553.12286962,699.368886)(553.15287087,699.35889374)
\curveto(553.19286955,699.35888601)(553.22286952,699.36388601)(553.24287087,699.37389374)
\lineto(553.99287087,699.37389374)
\lineto(554.18787087,699.37389374)
\curveto(554.26786848,699.38388599)(554.33286841,699.38388599)(554.38287087,699.37389374)
\lineto(554.50287087,699.37389374)
\curveto(554.56286818,699.35388602)(554.61786813,699.33888603)(554.66787087,699.32889374)
\curveto(554.71786803,699.31888605)(554.75786799,699.28888608)(554.78787087,699.23889374)
\curveto(554.82786792,699.18888618)(554.8478679,699.11888625)(554.84787087,699.02889374)
\curveto(554.85786789,698.93888643)(554.86286788,698.84388653)(554.86287087,698.74389374)
\lineto(554.86287087,692.45889374)
}
}
{
\newrgbcolor{curcolor}{0 0 0}
\pscustom[linestyle=none,fillstyle=solid,fillcolor=curcolor]
{
\newpath
\moveto(563.89005837,695.57889374)
\curveto(563.9100502,695.49888987)(563.9100502,695.40888996)(563.89005837,695.30889374)
\curveto(563.87005024,695.20889016)(563.83505028,695.14389023)(563.78505837,695.11389374)
\curveto(563.73505038,695.0738903)(563.66005045,695.04389033)(563.56005837,695.02389374)
\curveto(563.47005064,695.01389036)(563.36505075,695.00389037)(563.24505837,694.99389374)
\lineto(562.90005837,694.99389374)
\curveto(562.79005132,695.00389037)(562.69005142,695.00889036)(562.60005837,695.00889374)
\lineto(558.94005837,695.00889374)
\lineto(558.73005837,695.00889374)
\curveto(558.67005544,695.00889036)(558.6150555,694.99889037)(558.56505837,694.97889374)
\curveto(558.48505563,694.93889043)(558.43505568,694.89889047)(558.41505837,694.85889374)
\curveto(558.39505572,694.83889053)(558.37505574,694.79889057)(558.35505837,694.73889374)
\curveto(558.33505578,694.68889068)(558.33005578,694.63889073)(558.34005837,694.58889374)
\curveto(558.36005575,694.52889084)(558.37005574,694.4688909)(558.37005837,694.40889374)
\curveto(558.38005573,694.35889101)(558.39505572,694.30389107)(558.41505837,694.24389374)
\curveto(558.49505562,694.00389137)(558.59005552,693.80389157)(558.70005837,693.64389374)
\curveto(558.82005529,693.49389188)(558.98005513,693.35889201)(559.18005837,693.23889374)
\curveto(559.26005485,693.18889218)(559.34005477,693.15389222)(559.42005837,693.13389374)
\curveto(559.5100546,693.12389225)(559.60005451,693.10389227)(559.69005837,693.07389374)
\curveto(559.77005434,693.05389232)(559.88005423,693.03889233)(560.02005837,693.02889374)
\curveto(560.16005395,693.01889235)(560.28005383,693.02389235)(560.38005837,693.04389374)
\lineto(560.51505837,693.04389374)
\curveto(560.6150535,693.06389231)(560.70505341,693.08389229)(560.78505837,693.10389374)
\curveto(560.87505324,693.13389224)(560.96005315,693.16389221)(561.04005837,693.19389374)
\curveto(561.14005297,693.24389213)(561.25005286,693.30889206)(561.37005837,693.38889374)
\curveto(561.50005261,693.4688919)(561.59505252,693.54889182)(561.65505837,693.62889374)
\curveto(561.70505241,693.69889167)(561.75505236,693.76389161)(561.80505837,693.82389374)
\curveto(561.86505225,693.89389148)(561.93505218,693.94389143)(562.01505837,693.97389374)
\curveto(562.115052,694.02389135)(562.24005187,694.04389133)(562.39005837,694.03389374)
\lineto(562.82505837,694.03389374)
\lineto(563.00505837,694.03389374)
\curveto(563.07505104,694.04389133)(563.13505098,694.03889133)(563.18505837,694.01889374)
\lineto(563.33505837,694.01889374)
\curveto(563.43505068,693.99889137)(563.50505061,693.9738914)(563.54505837,693.94389374)
\curveto(563.58505053,693.92389145)(563.60505051,693.87889149)(563.60505837,693.80889374)
\curveto(563.6150505,693.73889163)(563.6100505,693.67889169)(563.59005837,693.62889374)
\curveto(563.54005057,693.48889188)(563.48505063,693.36389201)(563.42505837,693.25389374)
\curveto(563.36505075,693.14389223)(563.29505082,693.03389234)(563.21505837,692.92389374)
\curveto(562.99505112,692.59389278)(562.74505137,692.32889304)(562.46505837,692.12889374)
\curveto(562.18505193,691.92889344)(561.83505228,691.75889361)(561.41505837,691.61889374)
\curveto(561.30505281,691.57889379)(561.19505292,691.55389382)(561.08505837,691.54389374)
\curveto(560.97505314,691.53389384)(560.86005325,691.51389386)(560.74005837,691.48389374)
\curveto(560.70005341,691.4738939)(560.65505346,691.4738939)(560.60505837,691.48389374)
\curveto(560.56505355,691.48389389)(560.52505359,691.47889389)(560.48505837,691.46889374)
\lineto(560.32005837,691.46889374)
\curveto(560.27005384,691.44889392)(560.2100539,691.44389393)(560.14005837,691.45389374)
\curveto(560.08005403,691.45389392)(560.02505409,691.45889391)(559.97505837,691.46889374)
\curveto(559.89505422,691.47889389)(559.82505429,691.47889389)(559.76505837,691.46889374)
\curveto(559.70505441,691.45889391)(559.64005447,691.46389391)(559.57005837,691.48389374)
\curveto(559.52005459,691.50389387)(559.46505465,691.51389386)(559.40505837,691.51389374)
\curveto(559.34505477,691.51389386)(559.29005482,691.52389385)(559.24005837,691.54389374)
\curveto(559.13005498,691.56389381)(559.02005509,691.58889378)(558.91005837,691.61889374)
\curveto(558.80005531,691.63889373)(558.70005541,691.6738937)(558.61005837,691.72389374)
\curveto(558.50005561,691.76389361)(558.39505572,691.79889357)(558.29505837,691.82889374)
\curveto(558.20505591,691.8688935)(558.12005599,691.91389346)(558.04005837,691.96389374)
\curveto(557.72005639,692.16389321)(557.43505668,692.39389298)(557.18505837,692.65389374)
\curveto(556.93505718,692.92389245)(556.73005738,693.23389214)(556.57005837,693.58389374)
\curveto(556.52005759,693.69389168)(556.48005763,693.80389157)(556.45005837,693.91389374)
\curveto(556.42005769,694.03389134)(556.38005773,694.15389122)(556.33005837,694.27389374)
\curveto(556.32005779,694.31389106)(556.3150578,694.34889102)(556.31505837,694.37889374)
\curveto(556.3150578,694.41889095)(556.3100578,694.45889091)(556.30005837,694.49889374)
\curveto(556.26005785,694.61889075)(556.23505788,694.74889062)(556.22505837,694.88889374)
\lineto(556.19505837,695.30889374)
\curveto(556.19505792,695.35889001)(556.19005792,695.41388996)(556.18005837,695.47389374)
\curveto(556.18005793,695.53388984)(556.18505793,695.58888978)(556.19505837,695.63889374)
\lineto(556.19505837,695.81889374)
\lineto(556.24005837,696.17889374)
\curveto(556.28005783,696.34888902)(556.3150578,696.51388886)(556.34505837,696.67389374)
\curveto(556.37505774,696.83388854)(556.42005769,696.98388839)(556.48005837,697.12389374)
\curveto(556.9100572,698.16388721)(557.64005647,698.89888647)(558.67005837,699.32889374)
\curveto(558.8100553,699.38888598)(558.95005516,699.42888594)(559.09005837,699.44889374)
\curveto(559.24005487,699.47888589)(559.39505472,699.51388586)(559.55505837,699.55389374)
\curveto(559.63505448,699.56388581)(559.7100544,699.5688858)(559.78005837,699.56889374)
\curveto(559.85005426,699.5688858)(559.92505419,699.5738858)(560.00505837,699.58389374)
\curveto(560.5150536,699.59388578)(560.95005316,699.53388584)(561.31005837,699.40389374)
\curveto(561.68005243,699.28388609)(562.0100521,699.12388625)(562.30005837,698.92389374)
\curveto(562.39005172,698.86388651)(562.48005163,698.79388658)(562.57005837,698.71389374)
\curveto(562.66005145,698.64388673)(562.74005137,698.5688868)(562.81005837,698.48889374)
\curveto(562.84005127,698.43888693)(562.88005123,698.39888697)(562.93005837,698.36889374)
\curveto(563.0100511,698.25888711)(563.08505103,698.14388723)(563.15505837,698.02389374)
\curveto(563.22505089,697.91388746)(563.30005081,697.79888757)(563.38005837,697.67889374)
\curveto(563.43005068,697.58888778)(563.47005064,697.49388788)(563.50005837,697.39389374)
\curveto(563.54005057,697.30388807)(563.58005053,697.20388817)(563.62005837,697.09389374)
\curveto(563.67005044,696.96388841)(563.7100504,696.82888854)(563.74005837,696.68889374)
\curveto(563.77005034,696.54888882)(563.80505031,696.40888896)(563.84505837,696.26889374)
\curveto(563.86505025,696.18888918)(563.87005024,696.09888927)(563.86005837,695.99889374)
\curveto(563.86005025,695.90888946)(563.87005024,695.82388955)(563.89005837,695.74389374)
\lineto(563.89005837,695.57889374)
\moveto(561.64005837,696.46389374)
\curveto(561.7100524,696.56388881)(561.7150524,696.68388869)(561.65505837,696.82389374)
\curveto(561.60505251,696.9738884)(561.56505255,697.08388829)(561.53505837,697.15389374)
\curveto(561.39505272,697.42388795)(561.2100529,697.62888774)(560.98005837,697.76889374)
\curveto(560.75005336,697.91888745)(560.43005368,697.99888737)(560.02005837,698.00889374)
\curveto(559.99005412,697.98888738)(559.95505416,697.98388739)(559.91505837,697.99389374)
\curveto(559.87505424,698.00388737)(559.84005427,698.00388737)(559.81005837,697.99389374)
\curveto(559.76005435,697.9738874)(559.70505441,697.95888741)(559.64505837,697.94889374)
\curveto(559.58505453,697.94888742)(559.53005458,697.93888743)(559.48005837,697.91889374)
\curveto(559.04005507,697.77888759)(558.7150554,697.50388787)(558.50505837,697.09389374)
\curveto(558.48505563,697.05388832)(558.46005565,696.99888837)(558.43005837,696.92889374)
\curveto(558.4100557,696.8688885)(558.39505572,696.80388857)(558.38505837,696.73389374)
\curveto(558.37505574,696.6738887)(558.37505574,696.61388876)(558.38505837,696.55389374)
\curveto(558.40505571,696.49388888)(558.44005567,696.44388893)(558.49005837,696.40389374)
\curveto(558.57005554,696.35388902)(558.68005543,696.32888904)(558.82005837,696.32889374)
\lineto(559.22505837,696.32889374)
\lineto(560.89005837,696.32889374)
\lineto(561.32505837,696.32889374)
\curveto(561.48505263,696.33888903)(561.59005252,696.38388899)(561.64005837,696.46389374)
}
}
{
\newrgbcolor{curcolor}{0 0 0}
\pscustom[linestyle=none,fillstyle=solid,fillcolor=curcolor]
{
\newpath
\moveto(569.56333962,699.56889374)
\curveto(569.6733343,699.5688858)(569.76833421,699.55888581)(569.84833962,699.53889374)
\curveto(569.93833404,699.51888585)(570.00833397,699.4738859)(570.05833962,699.40389374)
\curveto(570.11833386,699.32388605)(570.14833383,699.18388619)(570.14833962,698.98389374)
\lineto(570.14833962,698.47389374)
\lineto(570.14833962,698.09889374)
\curveto(570.15833382,697.95888741)(570.14333383,697.84888752)(570.10333962,697.76889374)
\curveto(570.06333391,697.69888767)(570.00333397,697.65388772)(569.92333962,697.63389374)
\curveto(569.85333412,697.61388776)(569.76833421,697.60388777)(569.66833962,697.60389374)
\curveto(569.5783344,697.60388777)(569.4783345,697.60888776)(569.36833962,697.61889374)
\curveto(569.26833471,697.62888774)(569.1733348,697.62388775)(569.08333962,697.60389374)
\curveto(569.01333496,697.58388779)(568.94333503,697.5688878)(568.87333962,697.55889374)
\curveto(568.80333517,697.55888781)(568.73833524,697.54888782)(568.67833962,697.52889374)
\curveto(568.51833546,697.47888789)(568.35833562,697.40388797)(568.19833962,697.30389374)
\curveto(568.03833594,697.21388816)(567.91333606,697.10888826)(567.82333962,696.98889374)
\curveto(567.7733362,696.90888846)(567.71833626,696.82388855)(567.65833962,696.73389374)
\curveto(567.60833637,696.65388872)(567.55833642,696.5688888)(567.50833962,696.47889374)
\curveto(567.4783365,696.39888897)(567.44833653,696.31388906)(567.41833962,696.22389374)
\lineto(567.35833962,695.98389374)
\curveto(567.33833664,695.91388946)(567.32833665,695.83888953)(567.32833962,695.75889374)
\curveto(567.32833665,695.68888968)(567.31833666,695.61888975)(567.29833962,695.54889374)
\curveto(567.28833669,695.50888986)(567.28333669,695.4688899)(567.28333962,695.42889374)
\curveto(567.29333668,695.39888997)(567.29333668,695.36889)(567.28333962,695.33889374)
\lineto(567.28333962,695.09889374)
\curveto(567.26333671,695.02889034)(567.25833672,694.94889042)(567.26833962,694.85889374)
\curveto(567.2783367,694.77889059)(567.28333669,694.69889067)(567.28333962,694.61889374)
\lineto(567.28333962,693.65889374)
\lineto(567.28333962,692.38389374)
\curveto(567.28333669,692.25389312)(567.2783367,692.13389324)(567.26833962,692.02389374)
\curveto(567.25833672,691.91389346)(567.22833675,691.82389355)(567.17833962,691.75389374)
\curveto(567.15833682,691.72389365)(567.12333685,691.69889367)(567.07333962,691.67889374)
\curveto(567.03333694,691.6688937)(566.98833699,691.65889371)(566.93833962,691.64889374)
\lineto(566.86333962,691.64889374)
\curveto(566.81333716,691.63889373)(566.75833722,691.63389374)(566.69833962,691.63389374)
\lineto(566.53333962,691.63389374)
\lineto(565.88833962,691.63389374)
\curveto(565.82833815,691.64389373)(565.76333821,691.64889372)(565.69333962,691.64889374)
\lineto(565.49833962,691.64889374)
\curveto(565.44833853,691.6688937)(565.39833858,691.68389369)(565.34833962,691.69389374)
\curveto(565.29833868,691.71389366)(565.26333871,691.74889362)(565.24333962,691.79889374)
\curveto(565.20333877,691.84889352)(565.1783388,691.91889345)(565.16833962,692.00889374)
\lineto(565.16833962,692.30889374)
\lineto(565.16833962,693.32889374)
\lineto(565.16833962,697.55889374)
\lineto(565.16833962,698.66889374)
\lineto(565.16833962,698.95389374)
\curveto(565.16833881,699.05388632)(565.18833879,699.13388624)(565.22833962,699.19389374)
\curveto(565.2783387,699.2738861)(565.35333862,699.32388605)(565.45333962,699.34389374)
\curveto(565.55333842,699.36388601)(565.6733383,699.373886)(565.81333962,699.37389374)
\lineto(566.57833962,699.37389374)
\curveto(566.69833728,699.373886)(566.80333717,699.36388601)(566.89333962,699.34389374)
\curveto(566.98333699,699.33388604)(567.05333692,699.28888608)(567.10333962,699.20889374)
\curveto(567.13333684,699.15888621)(567.14833683,699.08888628)(567.14833962,698.99889374)
\lineto(567.17833962,698.72889374)
\curveto(567.18833679,698.64888672)(567.20333677,698.5738868)(567.22333962,698.50389374)
\curveto(567.25333672,698.43388694)(567.30333667,698.39888697)(567.37333962,698.39889374)
\curveto(567.39333658,698.41888695)(567.41333656,698.42888694)(567.43333962,698.42889374)
\curveto(567.45333652,698.42888694)(567.4733365,698.43888693)(567.49333962,698.45889374)
\curveto(567.55333642,698.50888686)(567.60333637,698.56388681)(567.64333962,698.62389374)
\curveto(567.69333628,698.69388668)(567.75333622,698.75388662)(567.82333962,698.80389374)
\curveto(567.86333611,698.83388654)(567.89833608,698.86388651)(567.92833962,698.89389374)
\curveto(567.95833602,698.93388644)(567.99333598,698.9688864)(568.03333962,698.99889374)
\lineto(568.30333962,699.17889374)
\curveto(568.40333557,699.23888613)(568.50333547,699.29388608)(568.60333962,699.34389374)
\curveto(568.70333527,699.38388599)(568.80333517,699.41888595)(568.90333962,699.44889374)
\lineto(569.23333962,699.53889374)
\curveto(569.26333471,699.54888582)(569.31833466,699.54888582)(569.39833962,699.53889374)
\curveto(569.48833449,699.53888583)(569.54333443,699.54888582)(569.56333962,699.56889374)
}
}
{
\newrgbcolor{curcolor}{0 0 0}
\pscustom[linestyle=none,fillstyle=solid,fillcolor=curcolor]
{
\newpath
\moveto(578.47474587,695.81889374)
\curveto(578.4947373,695.75888961)(578.50473729,695.6738897)(578.50474587,695.56389374)
\curveto(578.50473729,695.45388992)(578.4947373,695.36889)(578.47474587,695.30889374)
\lineto(578.47474587,695.15889374)
\curveto(578.45473734,695.07889029)(578.44473735,694.99889037)(578.44474587,694.91889374)
\curveto(578.45473734,694.83889053)(578.44973734,694.75889061)(578.42974587,694.67889374)
\curveto(578.40973738,694.60889076)(578.3947374,694.54389083)(578.38474587,694.48389374)
\curveto(578.37473742,694.42389095)(578.36473743,694.35889101)(578.35474587,694.28889374)
\curveto(578.31473748,694.17889119)(578.27973751,694.06389131)(578.24974587,693.94389374)
\curveto(578.21973757,693.83389154)(578.17973761,693.72889164)(578.12974587,693.62889374)
\curveto(577.91973787,693.14889222)(577.64473815,692.75889261)(577.30474587,692.45889374)
\curveto(576.96473883,692.15889321)(576.55473924,691.90889346)(576.07474587,691.70889374)
\curveto(575.95473984,691.65889371)(575.82973996,691.62389375)(575.69974587,691.60389374)
\curveto(575.57974021,691.5738938)(575.45474034,691.54389383)(575.32474587,691.51389374)
\curveto(575.27474052,691.49389388)(575.21974057,691.48389389)(575.15974587,691.48389374)
\curveto(575.09974069,691.48389389)(575.04474075,691.47889389)(574.99474587,691.46889374)
\lineto(574.88974587,691.46889374)
\curveto(574.85974093,691.45889391)(574.82974096,691.45389392)(574.79974587,691.45389374)
\curveto(574.74974104,691.44389393)(574.66974112,691.43889393)(574.55974587,691.43889374)
\curveto(574.44974134,691.42889394)(574.36474143,691.43389394)(574.30474587,691.45389374)
\lineto(574.15474587,691.45389374)
\curveto(574.10474169,691.46389391)(574.04974174,691.4688939)(573.98974587,691.46889374)
\curveto(573.93974185,691.45889391)(573.8897419,691.46389391)(573.83974587,691.48389374)
\curveto(573.79974199,691.49389388)(573.75974203,691.49889387)(573.71974587,691.49889374)
\curveto(573.6897421,691.49889387)(573.64974214,691.50389387)(573.59974587,691.51389374)
\curveto(573.49974229,691.54389383)(573.39974239,691.5688938)(573.29974587,691.58889374)
\curveto(573.19974259,691.60889376)(573.10474269,691.63889373)(573.01474587,691.67889374)
\curveto(572.8947429,691.71889365)(572.77974301,691.75889361)(572.66974587,691.79889374)
\curveto(572.56974322,691.83889353)(572.46474333,691.88889348)(572.35474587,691.94889374)
\curveto(572.00474379,692.15889321)(571.70474409,692.40389297)(571.45474587,692.68389374)
\curveto(571.20474459,692.96389241)(570.9947448,693.29889207)(570.82474587,693.68889374)
\curveto(570.77474502,693.77889159)(570.73474506,693.8738915)(570.70474587,693.97389374)
\curveto(570.68474511,694.0738913)(570.65974513,694.17889119)(570.62974587,694.28889374)
\curveto(570.60974518,694.33889103)(570.59974519,694.38389099)(570.59974587,694.42389374)
\curveto(570.59974519,694.46389091)(570.5897452,694.50889086)(570.56974587,694.55889374)
\curveto(570.54974524,694.63889073)(570.53974525,694.71889065)(570.53974587,694.79889374)
\curveto(570.53974525,694.88889048)(570.52974526,694.9738904)(570.50974587,695.05389374)
\curveto(570.49974529,695.10389027)(570.4947453,695.14889022)(570.49474587,695.18889374)
\lineto(570.49474587,695.32389374)
\curveto(570.47474532,695.38388999)(570.46474533,695.4688899)(570.46474587,695.57889374)
\curveto(570.47474532,695.68888968)(570.4897453,695.7738896)(570.50974587,695.83389374)
\lineto(570.50974587,695.93889374)
\curveto(570.51974527,695.98888938)(570.51974527,696.03888933)(570.50974587,696.08889374)
\curveto(570.50974528,696.14888922)(570.51974527,696.20388917)(570.53974587,696.25389374)
\curveto(570.54974524,696.30388907)(570.55474524,696.34888902)(570.55474587,696.38889374)
\curveto(570.55474524,696.43888893)(570.56474523,696.48888888)(570.58474587,696.53889374)
\curveto(570.62474517,696.6688887)(570.65974513,696.79388858)(570.68974587,696.91389374)
\curveto(570.71974507,697.04388833)(570.75974503,697.1688882)(570.80974587,697.28889374)
\curveto(570.9897448,697.69888767)(571.20474459,698.03888733)(571.45474587,698.30889374)
\curveto(571.70474409,698.58888678)(572.00974378,698.84388653)(572.36974587,699.07389374)
\curveto(572.46974332,699.12388625)(572.57474322,699.1688862)(572.68474587,699.20889374)
\curveto(572.794743,699.24888612)(572.90474289,699.29388608)(573.01474587,699.34389374)
\curveto(573.14474265,699.39388598)(573.27974251,699.42888594)(573.41974587,699.44889374)
\curveto(573.55974223,699.4688859)(573.70474209,699.49888587)(573.85474587,699.53889374)
\curveto(573.93474186,699.54888582)(574.00974178,699.55388582)(574.07974587,699.55389374)
\curveto(574.14974164,699.55388582)(574.21974157,699.55888581)(574.28974587,699.56889374)
\curveto(574.86974092,699.57888579)(575.36974042,699.51888585)(575.78974587,699.38889374)
\curveto(576.21973957,699.25888611)(576.59973919,699.07888629)(576.92974587,698.84889374)
\curveto(577.03973875,698.7688866)(577.14973864,698.67888669)(577.25974587,698.57889374)
\curveto(577.37973841,698.48888688)(577.47973831,698.38888698)(577.55974587,698.27889374)
\curveto(577.63973815,698.17888719)(577.70973808,698.07888729)(577.76974587,697.97889374)
\curveto(577.83973795,697.87888749)(577.90973788,697.7738876)(577.97974587,697.66389374)
\curveto(578.04973774,697.55388782)(578.10473769,697.43388794)(578.14474587,697.30389374)
\curveto(578.18473761,697.18388819)(578.22973756,697.05388832)(578.27974587,696.91389374)
\curveto(578.30973748,696.83388854)(578.33473746,696.74888862)(578.35474587,696.65889374)
\lineto(578.41474587,696.38889374)
\curveto(578.42473737,696.34888902)(578.42973736,696.30888906)(578.42974587,696.26889374)
\curveto(578.42973736,696.22888914)(578.43473736,696.18888918)(578.44474587,696.14889374)
\curveto(578.46473733,696.09888927)(578.46973732,696.04388933)(578.45974587,695.98389374)
\curveto(578.44973734,695.92388945)(578.45473734,695.8688895)(578.47474587,695.81889374)
\moveto(576.37474587,695.27889374)
\curveto(576.38473941,695.32889004)(576.3897394,695.39888997)(576.38974587,695.48889374)
\curveto(576.3897394,695.58888978)(576.38473941,695.66388971)(576.37474587,695.71389374)
\lineto(576.37474587,695.83389374)
\curveto(576.35473944,695.88388949)(576.34473945,695.93888943)(576.34474587,695.99889374)
\curveto(576.34473945,696.05888931)(576.33973945,696.11388926)(576.32974587,696.16389374)
\curveto(576.32973946,696.20388917)(576.32473947,696.23388914)(576.31474587,696.25389374)
\lineto(576.25474587,696.49389374)
\curveto(576.24473955,696.58388879)(576.22473957,696.6688887)(576.19474587,696.74889374)
\curveto(576.08473971,697.00888836)(575.95473984,697.22888814)(575.80474587,697.40889374)
\curveto(575.65474014,697.59888777)(575.45474034,697.74888762)(575.20474587,697.85889374)
\curveto(575.14474065,697.87888749)(575.08474071,697.89388748)(575.02474587,697.90389374)
\curveto(574.96474083,697.92388745)(574.89974089,697.94388743)(574.82974587,697.96389374)
\curveto(574.74974104,697.98388739)(574.66474113,697.98888738)(574.57474587,697.97889374)
\lineto(574.30474587,697.97889374)
\curveto(574.27474152,697.95888741)(574.23974155,697.94888742)(574.19974587,697.94889374)
\curveto(574.15974163,697.95888741)(574.12474167,697.95888741)(574.09474587,697.94889374)
\lineto(573.88474587,697.88889374)
\curveto(573.82474197,697.87888749)(573.76974202,697.85888751)(573.71974587,697.82889374)
\curveto(573.46974232,697.71888765)(573.26474253,697.55888781)(573.10474587,697.34889374)
\curveto(572.95474284,697.14888822)(572.83474296,696.91388846)(572.74474587,696.64389374)
\curveto(572.71474308,696.54388883)(572.6897431,696.43888893)(572.66974587,696.32889374)
\curveto(572.65974313,696.21888915)(572.64474315,696.10888926)(572.62474587,695.99889374)
\curveto(572.61474318,695.94888942)(572.60974318,695.89888947)(572.60974587,695.84889374)
\lineto(572.60974587,695.69889374)
\curveto(572.5897432,695.62888974)(572.57974321,695.52388985)(572.57974587,695.38389374)
\curveto(572.5897432,695.24389013)(572.60474319,695.13889023)(572.62474587,695.06889374)
\lineto(572.62474587,694.93389374)
\curveto(572.64474315,694.85389052)(572.65974313,694.7738906)(572.66974587,694.69389374)
\curveto(572.67974311,694.62389075)(572.6947431,694.54889082)(572.71474587,694.46889374)
\curveto(572.81474298,694.1688912)(572.91974287,693.92389145)(573.02974587,693.73389374)
\curveto(573.14974264,693.55389182)(573.33474246,693.38889198)(573.58474587,693.23889374)
\curveto(573.65474214,693.18889218)(573.72974206,693.14889222)(573.80974587,693.11889374)
\curveto(573.89974189,693.08889228)(573.9897418,693.06389231)(574.07974587,693.04389374)
\curveto(574.11974167,693.03389234)(574.15474164,693.02889234)(574.18474587,693.02889374)
\curveto(574.21474158,693.03889233)(574.24974154,693.03889233)(574.28974587,693.02889374)
\lineto(574.40974587,692.99889374)
\curveto(574.45974133,692.99889237)(574.50474129,693.00389237)(574.54474587,693.01389374)
\lineto(574.66474587,693.01389374)
\curveto(574.74474105,693.03389234)(574.82474097,693.04889232)(574.90474587,693.05889374)
\curveto(574.98474081,693.0688923)(575.05974073,693.08889228)(575.12974587,693.11889374)
\curveto(575.3897404,693.21889215)(575.59974019,693.35389202)(575.75974587,693.52389374)
\curveto(575.91973987,693.69389168)(576.05473974,693.90389147)(576.16474587,694.15389374)
\curveto(576.20473959,694.25389112)(576.23473956,694.35389102)(576.25474587,694.45389374)
\curveto(576.27473952,694.55389082)(576.29973949,694.65889071)(576.32974587,694.76889374)
\curveto(576.33973945,694.80889056)(576.34473945,694.84389053)(576.34474587,694.87389374)
\curveto(576.34473945,694.91389046)(576.34973944,694.95389042)(576.35974587,694.99389374)
\lineto(576.35974587,695.12889374)
\curveto(576.35973943,695.17889019)(576.36473943,695.22889014)(576.37474587,695.27889374)
}
}
{
\newrgbcolor{curcolor}{0 0 0}
\pscustom[linestyle=none,fillstyle=solid,fillcolor=curcolor]
{
\newpath
\moveto(584.29966774,699.56889374)
\curveto(584.89966194,699.58888578)(585.39966144,699.50388587)(585.79966774,699.31389374)
\curveto(586.19966064,699.12388625)(586.51466032,698.84388653)(586.74466774,698.47389374)
\curveto(586.81466002,698.36388701)(586.86965997,698.24388713)(586.90966774,698.11389374)
\curveto(586.94965989,697.99388738)(586.98965985,697.8688875)(587.02966774,697.73889374)
\curveto(587.04965979,697.65888771)(587.05965978,697.58388779)(587.05966774,697.51389374)
\curveto(587.06965977,697.44388793)(587.08465975,697.373888)(587.10466774,697.30389374)
\curveto(587.10465973,697.24388813)(587.10965973,697.20388817)(587.11966774,697.18389374)
\curveto(587.1396597,697.04388833)(587.14965969,696.89888847)(587.14966774,696.74889374)
\lineto(587.14966774,696.31389374)
\lineto(587.14966774,694.97889374)
\lineto(587.14966774,692.54889374)
\curveto(587.14965969,692.35889301)(587.14465969,692.1738932)(587.13466774,691.99389374)
\curveto(587.1346597,691.82389355)(587.06465977,691.71389366)(586.92466774,691.66389374)
\curveto(586.86465997,691.64389373)(586.79466004,691.63389374)(586.71466774,691.63389374)
\lineto(586.47466774,691.63389374)
\lineto(585.66466774,691.63389374)
\curveto(585.54466129,691.63389374)(585.4346614,691.63889373)(585.33466774,691.64889374)
\curveto(585.24466159,691.6688937)(585.17466166,691.71389366)(585.12466774,691.78389374)
\curveto(585.08466175,691.84389353)(585.05966178,691.91889345)(585.04966774,692.00889374)
\lineto(585.04966774,692.32389374)
\lineto(585.04966774,693.37389374)
\lineto(585.04966774,695.60889374)
\curveto(585.04966179,695.97888939)(585.0346618,696.31888905)(585.00466774,696.62889374)
\curveto(584.97466186,696.94888842)(584.88466195,697.21888815)(584.73466774,697.43889374)
\curveto(584.59466224,697.63888773)(584.38966245,697.77888759)(584.11966774,697.85889374)
\curveto(584.06966277,697.87888749)(584.01466282,697.88888748)(583.95466774,697.88889374)
\curveto(583.90466293,697.88888748)(583.84966299,697.89888747)(583.78966774,697.91889374)
\curveto(583.7396631,697.92888744)(583.67466316,697.92888744)(583.59466774,697.91889374)
\curveto(583.52466331,697.91888745)(583.46966337,697.91388746)(583.42966774,697.90389374)
\curveto(583.38966345,697.89388748)(583.35466348,697.88888748)(583.32466774,697.88889374)
\curveto(583.29466354,697.88888748)(583.26466357,697.88388749)(583.23466774,697.87389374)
\curveto(583.00466383,697.81388756)(582.81966402,697.73388764)(582.67966774,697.63389374)
\curveto(582.35966448,697.40388797)(582.16966467,697.0688883)(582.10966774,696.62889374)
\curveto(582.04966479,696.18888918)(582.01966482,695.69388968)(582.01966774,695.14389374)
\lineto(582.01966774,693.26889374)
\lineto(582.01966774,692.35389374)
\lineto(582.01966774,692.08389374)
\curveto(582.01966482,691.99389338)(582.00466483,691.91889345)(581.97466774,691.85889374)
\curveto(581.92466491,691.74889362)(581.84466499,691.68389369)(581.73466774,691.66389374)
\curveto(581.62466521,691.64389373)(581.48966535,691.63389374)(581.32966774,691.63389374)
\lineto(580.57966774,691.63389374)
\curveto(580.46966637,691.63389374)(580.35966648,691.63889373)(580.24966774,691.64889374)
\curveto(580.1396667,691.65889371)(580.05966678,691.69389368)(580.00966774,691.75389374)
\curveto(579.9396669,691.84389353)(579.90466693,691.9738934)(579.90466774,692.14389374)
\curveto(579.91466692,692.31389306)(579.91966692,692.4738929)(579.91966774,692.62389374)
\lineto(579.91966774,694.66389374)
\lineto(579.91966774,697.96389374)
\lineto(579.91966774,698.72889374)
\lineto(579.91966774,699.02889374)
\curveto(579.92966691,699.11888625)(579.95966688,699.19388618)(580.00966774,699.25389374)
\curveto(580.02966681,699.28388609)(580.05966678,699.30388607)(580.09966774,699.31389374)
\curveto(580.14966669,699.33388604)(580.19966664,699.34888602)(580.24966774,699.35889374)
\lineto(580.32466774,699.35889374)
\curveto(580.37466646,699.368886)(580.42466641,699.373886)(580.47466774,699.37389374)
\lineto(580.63966774,699.37389374)
\lineto(581.26966774,699.37389374)
\curveto(581.34966549,699.373886)(581.42466541,699.368886)(581.49466774,699.35889374)
\curveto(581.57466526,699.35888601)(581.64466519,699.34888602)(581.70466774,699.32889374)
\curveto(581.77466506,699.29888607)(581.81966502,699.25388612)(581.83966774,699.19389374)
\curveto(581.86966497,699.13388624)(581.89466494,699.06388631)(581.91466774,698.98389374)
\curveto(581.92466491,698.94388643)(581.92466491,698.90888646)(581.91466774,698.87889374)
\curveto(581.91466492,698.84888652)(581.92466491,698.81888655)(581.94466774,698.78889374)
\curveto(581.96466487,698.73888663)(581.97966486,698.70888666)(581.98966774,698.69889374)
\curveto(582.00966483,698.68888668)(582.0346648,698.6738867)(582.06466774,698.65389374)
\curveto(582.17466466,698.64388673)(582.26466457,698.67888669)(582.33466774,698.75889374)
\curveto(582.40466443,698.84888652)(582.47966436,698.91888645)(582.55966774,698.96889374)
\curveto(582.82966401,699.1688862)(583.12966371,699.32888604)(583.45966774,699.44889374)
\curveto(583.54966329,699.47888589)(583.6396632,699.49888587)(583.72966774,699.50889374)
\curveto(583.82966301,699.51888585)(583.9346629,699.53388584)(584.04466774,699.55389374)
\curveto(584.07466276,699.56388581)(584.11966272,699.56388581)(584.17966774,699.55389374)
\curveto(584.2396626,699.55388582)(584.27966256,699.55888581)(584.29966774,699.56889374)
}
}
{
\newrgbcolor{curcolor}{0 0 0}
\pscustom[linestyle=none,fillstyle=solid,fillcolor=curcolor]
{
}
}
{
\newrgbcolor{curcolor}{0 0 0}
\pscustom[linestyle=none,fillstyle=solid,fillcolor=curcolor]
{
\newpath
\moveto(593.36107399,699.35889374)
\lineto(594.48607399,699.35889374)
\curveto(594.59607156,699.35888601)(594.69607146,699.35388602)(594.78607399,699.34389374)
\curveto(594.87607128,699.33388604)(594.94107121,699.29888607)(594.98107399,699.23889374)
\curveto(595.03107112,699.17888619)(595.06107109,699.09388628)(595.07107399,698.98389374)
\curveto(595.08107107,698.88388649)(595.08607107,698.77888659)(595.08607399,698.66889374)
\lineto(595.08607399,697.61889374)
\lineto(595.08607399,695.38389374)
\curveto(595.08607107,695.02389035)(595.10107105,694.68389069)(595.13107399,694.36389374)
\curveto(595.16107099,694.04389133)(595.2510709,693.77889159)(595.40107399,693.56889374)
\curveto(595.54107061,693.35889201)(595.76607039,693.20889216)(596.07607399,693.11889374)
\curveto(596.12607003,693.10889226)(596.16606999,693.10389227)(596.19607399,693.10389374)
\curveto(596.23606992,693.10389227)(596.28106987,693.09889227)(596.33107399,693.08889374)
\curveto(596.38106977,693.07889229)(596.43606972,693.0738923)(596.49607399,693.07389374)
\curveto(596.5560696,693.0738923)(596.60106955,693.07889229)(596.63107399,693.08889374)
\curveto(596.68106947,693.10889226)(596.72106943,693.11389226)(596.75107399,693.10389374)
\curveto(596.79106936,693.09389228)(596.83106932,693.09889227)(596.87107399,693.11889374)
\curveto(597.08106907,693.1688922)(597.24606891,693.23389214)(597.36607399,693.31389374)
\curveto(597.54606861,693.42389195)(597.68606847,693.56389181)(597.78607399,693.73389374)
\curveto(597.89606826,693.91389146)(597.97106818,694.10889126)(598.01107399,694.31889374)
\curveto(598.06106809,694.53889083)(598.09106806,694.77889059)(598.10107399,695.03889374)
\curveto(598.11106804,695.30889006)(598.11606804,695.58888978)(598.11607399,695.87889374)
\lineto(598.11607399,697.69389374)
\lineto(598.11607399,698.66889374)
\lineto(598.11607399,698.93889374)
\curveto(598.11606804,699.03888633)(598.13606802,699.11888625)(598.17607399,699.17889374)
\curveto(598.22606793,699.2688861)(598.30106785,699.31888605)(598.40107399,699.32889374)
\curveto(598.50106765,699.34888602)(598.62106753,699.35888601)(598.76107399,699.35889374)
\lineto(599.55607399,699.35889374)
\lineto(599.84107399,699.35889374)
\curveto(599.93106622,699.35888601)(600.00606615,699.33888603)(600.06607399,699.29889374)
\curveto(600.14606601,699.24888612)(600.19106596,699.1738862)(600.20107399,699.07389374)
\curveto(600.21106594,698.9738864)(600.21606594,698.85888651)(600.21607399,698.72889374)
\lineto(600.21607399,697.58889374)
\lineto(600.21607399,693.37389374)
\lineto(600.21607399,692.30889374)
\lineto(600.21607399,692.00889374)
\curveto(600.21606594,691.90889346)(600.19606596,691.83389354)(600.15607399,691.78389374)
\curveto(600.10606605,691.70389367)(600.03106612,691.65889371)(599.93107399,691.64889374)
\curveto(599.83106632,691.63889373)(599.72606643,691.63389374)(599.61607399,691.63389374)
\lineto(598.80607399,691.63389374)
\curveto(598.69606746,691.63389374)(598.59606756,691.63889373)(598.50607399,691.64889374)
\curveto(598.42606773,691.65889371)(598.36106779,691.69889367)(598.31107399,691.76889374)
\curveto(598.29106786,691.79889357)(598.27106788,691.84389353)(598.25107399,691.90389374)
\curveto(598.24106791,691.96389341)(598.22606793,692.02389335)(598.20607399,692.08389374)
\curveto(598.19606796,692.14389323)(598.18106797,692.19889317)(598.16107399,692.24889374)
\curveto(598.14106801,692.29889307)(598.11106804,692.32889304)(598.07107399,692.33889374)
\curveto(598.0510681,692.35889301)(598.02606813,692.36389301)(597.99607399,692.35389374)
\curveto(597.96606819,692.34389303)(597.94106821,692.33389304)(597.92107399,692.32389374)
\curveto(597.8510683,692.28389309)(597.79106836,692.23889313)(597.74107399,692.18889374)
\curveto(597.69106846,692.13889323)(597.63606852,692.09389328)(597.57607399,692.05389374)
\curveto(597.53606862,692.02389335)(597.49606866,691.98889338)(597.45607399,691.94889374)
\curveto(597.42606873,691.91889345)(597.38606877,691.88889348)(597.33607399,691.85889374)
\curveto(597.10606905,691.71889365)(596.83606932,691.60889376)(596.52607399,691.52889374)
\curveto(596.4560697,691.50889386)(596.38606977,691.49889387)(596.31607399,691.49889374)
\curveto(596.24606991,691.48889388)(596.17106998,691.4738939)(596.09107399,691.45389374)
\curveto(596.0510701,691.44389393)(596.00607015,691.44389393)(595.95607399,691.45389374)
\curveto(595.91607024,691.45389392)(595.87607028,691.44889392)(595.83607399,691.43889374)
\curveto(595.80607035,691.42889394)(595.74107041,691.42889394)(595.64107399,691.43889374)
\curveto(595.5510706,691.43889393)(595.49107066,691.44389393)(595.46107399,691.45389374)
\curveto(595.41107074,691.45389392)(595.36107079,691.45889391)(595.31107399,691.46889374)
\lineto(595.16107399,691.46889374)
\curveto(595.04107111,691.49889387)(594.92607123,691.52389385)(594.81607399,691.54389374)
\curveto(594.70607145,691.56389381)(594.59607156,691.59389378)(594.48607399,691.63389374)
\curveto(594.43607172,691.65389372)(594.39107176,691.6688937)(594.35107399,691.67889374)
\curveto(594.32107183,691.69889367)(594.28107187,691.71889365)(594.23107399,691.73889374)
\curveto(593.88107227,691.92889344)(593.60107255,692.19389318)(593.39107399,692.53389374)
\curveto(593.26107289,692.74389263)(593.16607299,692.99389238)(593.10607399,693.28389374)
\curveto(593.04607311,693.58389179)(593.00607315,693.89889147)(592.98607399,694.22889374)
\curveto(592.97607318,694.5688908)(592.97107318,694.91389046)(592.97107399,695.26389374)
\curveto(592.98107317,695.62388975)(592.98607317,695.97888939)(592.98607399,696.32889374)
\lineto(592.98607399,698.36889374)
\curveto(592.98607317,698.49888687)(592.98107317,698.64888672)(592.97107399,698.81889374)
\curveto(592.97107318,698.99888637)(592.99607316,699.12888624)(593.04607399,699.20889374)
\curveto(593.07607308,699.25888611)(593.13607302,699.30388607)(593.22607399,699.34389374)
\curveto(593.28607287,699.34388603)(593.33107282,699.34888602)(593.36107399,699.35889374)
}
}
{
\newrgbcolor{curcolor}{0 0 0}
\pscustom[linestyle=none,fillstyle=solid,fillcolor=curcolor]
{
\newpath
\moveto(606.27232399,699.56889374)
\curveto(606.87231819,699.58888578)(607.37231769,699.50388587)(607.77232399,699.31389374)
\curveto(608.17231689,699.12388625)(608.48731657,698.84388653)(608.71732399,698.47389374)
\curveto(608.78731627,698.36388701)(608.84231622,698.24388713)(608.88232399,698.11389374)
\curveto(608.92231614,697.99388738)(608.9623161,697.8688875)(609.00232399,697.73889374)
\curveto(609.02231604,697.65888771)(609.03231603,697.58388779)(609.03232399,697.51389374)
\curveto(609.04231602,697.44388793)(609.057316,697.373888)(609.07732399,697.30389374)
\curveto(609.07731598,697.24388813)(609.08231598,697.20388817)(609.09232399,697.18389374)
\curveto(609.11231595,697.04388833)(609.12231594,696.89888847)(609.12232399,696.74889374)
\lineto(609.12232399,696.31389374)
\lineto(609.12232399,694.97889374)
\lineto(609.12232399,692.54889374)
\curveto(609.12231594,692.35889301)(609.11731594,692.1738932)(609.10732399,691.99389374)
\curveto(609.10731595,691.82389355)(609.03731602,691.71389366)(608.89732399,691.66389374)
\curveto(608.83731622,691.64389373)(608.76731629,691.63389374)(608.68732399,691.63389374)
\lineto(608.44732399,691.63389374)
\lineto(607.63732399,691.63389374)
\curveto(607.51731754,691.63389374)(607.40731765,691.63889373)(607.30732399,691.64889374)
\curveto(607.21731784,691.6688937)(607.14731791,691.71389366)(607.09732399,691.78389374)
\curveto(607.057318,691.84389353)(607.03231803,691.91889345)(607.02232399,692.00889374)
\lineto(607.02232399,692.32389374)
\lineto(607.02232399,693.37389374)
\lineto(607.02232399,695.60889374)
\curveto(607.02231804,695.97888939)(607.00731805,696.31888905)(606.97732399,696.62889374)
\curveto(606.94731811,696.94888842)(606.8573182,697.21888815)(606.70732399,697.43889374)
\curveto(606.56731849,697.63888773)(606.3623187,697.77888759)(606.09232399,697.85889374)
\curveto(606.04231902,697.87888749)(605.98731907,697.88888748)(605.92732399,697.88889374)
\curveto(605.87731918,697.88888748)(605.82231924,697.89888747)(605.76232399,697.91889374)
\curveto(605.71231935,697.92888744)(605.64731941,697.92888744)(605.56732399,697.91889374)
\curveto(605.49731956,697.91888745)(605.44231962,697.91388746)(605.40232399,697.90389374)
\curveto(605.3623197,697.89388748)(605.32731973,697.88888748)(605.29732399,697.88889374)
\curveto(605.26731979,697.88888748)(605.23731982,697.88388749)(605.20732399,697.87389374)
\curveto(604.97732008,697.81388756)(604.79232027,697.73388764)(604.65232399,697.63389374)
\curveto(604.33232073,697.40388797)(604.14232092,697.0688883)(604.08232399,696.62889374)
\curveto(604.02232104,696.18888918)(603.99232107,695.69388968)(603.99232399,695.14389374)
\lineto(603.99232399,693.26889374)
\lineto(603.99232399,692.35389374)
\lineto(603.99232399,692.08389374)
\curveto(603.99232107,691.99389338)(603.97732108,691.91889345)(603.94732399,691.85889374)
\curveto(603.89732116,691.74889362)(603.81732124,691.68389369)(603.70732399,691.66389374)
\curveto(603.59732146,691.64389373)(603.4623216,691.63389374)(603.30232399,691.63389374)
\lineto(602.55232399,691.63389374)
\curveto(602.44232262,691.63389374)(602.33232273,691.63889373)(602.22232399,691.64889374)
\curveto(602.11232295,691.65889371)(602.03232303,691.69389368)(601.98232399,691.75389374)
\curveto(601.91232315,691.84389353)(601.87732318,691.9738934)(601.87732399,692.14389374)
\curveto(601.88732317,692.31389306)(601.89232317,692.4738929)(601.89232399,692.62389374)
\lineto(601.89232399,694.66389374)
\lineto(601.89232399,697.96389374)
\lineto(601.89232399,698.72889374)
\lineto(601.89232399,699.02889374)
\curveto(601.90232316,699.11888625)(601.93232313,699.19388618)(601.98232399,699.25389374)
\curveto(602.00232306,699.28388609)(602.03232303,699.30388607)(602.07232399,699.31389374)
\curveto(602.12232294,699.33388604)(602.17232289,699.34888602)(602.22232399,699.35889374)
\lineto(602.29732399,699.35889374)
\curveto(602.34732271,699.368886)(602.39732266,699.373886)(602.44732399,699.37389374)
\lineto(602.61232399,699.37389374)
\lineto(603.24232399,699.37389374)
\curveto(603.32232174,699.373886)(603.39732166,699.368886)(603.46732399,699.35889374)
\curveto(603.54732151,699.35888601)(603.61732144,699.34888602)(603.67732399,699.32889374)
\curveto(603.74732131,699.29888607)(603.79232127,699.25388612)(603.81232399,699.19389374)
\curveto(603.84232122,699.13388624)(603.86732119,699.06388631)(603.88732399,698.98389374)
\curveto(603.89732116,698.94388643)(603.89732116,698.90888646)(603.88732399,698.87889374)
\curveto(603.88732117,698.84888652)(603.89732116,698.81888655)(603.91732399,698.78889374)
\curveto(603.93732112,698.73888663)(603.95232111,698.70888666)(603.96232399,698.69889374)
\curveto(603.98232108,698.68888668)(604.00732105,698.6738867)(604.03732399,698.65389374)
\curveto(604.14732091,698.64388673)(604.23732082,698.67888669)(604.30732399,698.75889374)
\curveto(604.37732068,698.84888652)(604.45232061,698.91888645)(604.53232399,698.96889374)
\curveto(604.80232026,699.1688862)(605.10231996,699.32888604)(605.43232399,699.44889374)
\curveto(605.52231954,699.47888589)(605.61231945,699.49888587)(605.70232399,699.50889374)
\curveto(605.80231926,699.51888585)(605.90731915,699.53388584)(606.01732399,699.55389374)
\curveto(606.04731901,699.56388581)(606.09231897,699.56388581)(606.15232399,699.55389374)
\curveto(606.21231885,699.55388582)(606.25231881,699.55888581)(606.27232399,699.56889374)
}
}
{
\newrgbcolor{curcolor}{0 0 0}
\pscustom[linestyle=none,fillstyle=solid,fillcolor=curcolor]
{
\newpath
\moveto(397.64927712,680.81889374)
\curveto(398.24927131,680.83888578)(398.74927081,680.75388587)(399.14927712,680.56389374)
\curveto(399.54927001,680.37388625)(399.8642697,680.09388653)(400.09427712,679.72389374)
\curveto(400.1642694,679.61388701)(400.21926934,679.49388713)(400.25927712,679.36389374)
\curveto(400.29926926,679.24388738)(400.33926922,679.1188875)(400.37927712,678.98889374)
\curveto(400.39926916,678.90888771)(400.40926915,678.83388779)(400.40927712,678.76389374)
\curveto(400.41926914,678.69388793)(400.43426913,678.623888)(400.45427712,678.55389374)
\curveto(400.45426911,678.49388813)(400.4592691,678.45388817)(400.46927712,678.43389374)
\curveto(400.48926907,678.29388833)(400.49926906,678.14888847)(400.49927712,677.99889374)
\lineto(400.49927712,677.56389374)
\lineto(400.49927712,676.22889374)
\lineto(400.49927712,673.79889374)
\curveto(400.49926906,673.60889301)(400.49426907,673.4238932)(400.48427712,673.24389374)
\curveto(400.48426908,673.07389355)(400.41426915,672.96389366)(400.27427712,672.91389374)
\curveto(400.21426935,672.89389373)(400.14426942,672.88389374)(400.06427712,672.88389374)
\lineto(399.82427712,672.88389374)
\lineto(399.01427712,672.88389374)
\curveto(398.89427067,672.88389374)(398.78427078,672.88889373)(398.68427712,672.89889374)
\curveto(398.59427097,672.9188937)(398.52427104,672.96389366)(398.47427712,673.03389374)
\curveto(398.43427113,673.09389353)(398.40927115,673.16889345)(398.39927712,673.25889374)
\lineto(398.39927712,673.57389374)
\lineto(398.39927712,674.62389374)
\lineto(398.39927712,676.85889374)
\curveto(398.39927116,677.22888939)(398.38427118,677.56888905)(398.35427712,677.87889374)
\curveto(398.32427124,678.19888842)(398.23427133,678.46888815)(398.08427712,678.68889374)
\curveto(397.94427162,678.88888773)(397.73927182,679.02888759)(397.46927712,679.10889374)
\curveto(397.41927214,679.12888749)(397.3642722,679.13888748)(397.30427712,679.13889374)
\curveto(397.25427231,679.13888748)(397.19927236,679.14888747)(397.13927712,679.16889374)
\curveto(397.08927247,679.17888744)(397.02427254,679.17888744)(396.94427712,679.16889374)
\curveto(396.87427269,679.16888745)(396.81927274,679.16388746)(396.77927712,679.15389374)
\curveto(396.73927282,679.14388748)(396.70427286,679.13888748)(396.67427712,679.13889374)
\curveto(396.64427292,679.13888748)(396.61427295,679.13388749)(396.58427712,679.12389374)
\curveto(396.35427321,679.06388756)(396.16927339,678.98388764)(396.02927712,678.88389374)
\curveto(395.70927385,678.65388797)(395.51927404,678.3188883)(395.45927712,677.87889374)
\curveto(395.39927416,677.43888918)(395.36927419,676.94388968)(395.36927712,676.39389374)
\lineto(395.36927712,674.51889374)
\lineto(395.36927712,673.60389374)
\lineto(395.36927712,673.33389374)
\curveto(395.36927419,673.24389338)(395.35427421,673.16889345)(395.32427712,673.10889374)
\curveto(395.27427429,672.99889362)(395.19427437,672.93389369)(395.08427712,672.91389374)
\curveto(394.97427459,672.89389373)(394.83927472,672.88389374)(394.67927712,672.88389374)
\lineto(393.92927712,672.88389374)
\curveto(393.81927574,672.88389374)(393.70927585,672.88889373)(393.59927712,672.89889374)
\curveto(393.48927607,672.90889371)(393.40927615,672.94389368)(393.35927712,673.00389374)
\curveto(393.28927627,673.09389353)(393.25427631,673.2238934)(393.25427712,673.39389374)
\curveto(393.2642763,673.56389306)(393.26927629,673.7238929)(393.26927712,673.87389374)
\lineto(393.26927712,675.91389374)
\lineto(393.26927712,679.21389374)
\lineto(393.26927712,679.97889374)
\lineto(393.26927712,680.27889374)
\curveto(393.27927628,680.36888625)(393.30927625,680.44388618)(393.35927712,680.50389374)
\curveto(393.37927618,680.53388609)(393.40927615,680.55388607)(393.44927712,680.56389374)
\curveto(393.49927606,680.58388604)(393.54927601,680.59888602)(393.59927712,680.60889374)
\lineto(393.67427712,680.60889374)
\curveto(393.72427584,680.618886)(393.77427579,680.623886)(393.82427712,680.62389374)
\lineto(393.98927712,680.62389374)
\lineto(394.61927712,680.62389374)
\curveto(394.69927486,680.623886)(394.77427479,680.618886)(394.84427712,680.60889374)
\curveto(394.92427464,680.60888601)(394.99427457,680.59888602)(395.05427712,680.57889374)
\curveto(395.12427444,680.54888607)(395.16927439,680.50388612)(395.18927712,680.44389374)
\curveto(395.21927434,680.38388624)(395.24427432,680.31388631)(395.26427712,680.23389374)
\curveto(395.27427429,680.19388643)(395.27427429,680.15888646)(395.26427712,680.12889374)
\curveto(395.2642743,680.09888652)(395.27427429,680.06888655)(395.29427712,680.03889374)
\curveto(395.31427425,679.98888663)(395.32927423,679.95888666)(395.33927712,679.94889374)
\curveto(395.3592742,679.93888668)(395.38427418,679.9238867)(395.41427712,679.90389374)
\curveto(395.52427404,679.89388673)(395.61427395,679.92888669)(395.68427712,680.00889374)
\curveto(395.75427381,680.09888652)(395.82927373,680.16888645)(395.90927712,680.21889374)
\curveto(396.17927338,680.4188862)(396.47927308,680.57888604)(396.80927712,680.69889374)
\curveto(396.89927266,680.72888589)(396.98927257,680.74888587)(397.07927712,680.75889374)
\curveto(397.17927238,680.76888585)(397.28427228,680.78388584)(397.39427712,680.80389374)
\curveto(397.42427214,680.81388581)(397.46927209,680.81388581)(397.52927712,680.80389374)
\curveto(397.58927197,680.80388582)(397.62927193,680.80888581)(397.64927712,680.81889374)
}
}
{
\newrgbcolor{curcolor}{0 0 0}
\pscustom[linestyle=none,fillstyle=solid,fillcolor=curcolor]
{
\newpath
\moveto(409.90052712,677.06889374)
\curveto(409.92051855,677.00888961)(409.93051854,676.9238897)(409.93052712,676.81389374)
\curveto(409.93051854,676.70388992)(409.92051855,676.61889)(409.90052712,676.55889374)
\lineto(409.90052712,676.40889374)
\curveto(409.88051859,676.32889029)(409.8705186,676.24889037)(409.87052712,676.16889374)
\curveto(409.88051859,676.08889053)(409.87551859,676.00889061)(409.85552712,675.92889374)
\curveto(409.83551863,675.85889076)(409.82051865,675.79389083)(409.81052712,675.73389374)
\curveto(409.80051867,675.67389095)(409.79051868,675.60889101)(409.78052712,675.53889374)
\curveto(409.74051873,675.42889119)(409.70551876,675.31389131)(409.67552712,675.19389374)
\curveto(409.64551882,675.08389154)(409.60551886,674.97889164)(409.55552712,674.87889374)
\curveto(409.34551912,674.39889222)(409.0705194,674.00889261)(408.73052712,673.70889374)
\curveto(408.39052008,673.40889321)(407.98052049,673.15889346)(407.50052712,672.95889374)
\curveto(407.38052109,672.90889371)(407.25552121,672.87389375)(407.12552712,672.85389374)
\curveto(407.00552146,672.8238938)(406.88052159,672.79389383)(406.75052712,672.76389374)
\curveto(406.70052177,672.74389388)(406.64552182,672.73389389)(406.58552712,672.73389374)
\curveto(406.52552194,672.73389389)(406.470522,672.72889389)(406.42052712,672.71889374)
\lineto(406.31552712,672.71889374)
\curveto(406.28552218,672.70889391)(406.25552221,672.70389392)(406.22552712,672.70389374)
\curveto(406.17552229,672.69389393)(406.09552237,672.68889393)(405.98552712,672.68889374)
\curveto(405.87552259,672.67889394)(405.79052268,672.68389394)(405.73052712,672.70389374)
\lineto(405.58052712,672.70389374)
\curveto(405.53052294,672.71389391)(405.47552299,672.7188939)(405.41552712,672.71889374)
\curveto(405.3655231,672.70889391)(405.31552315,672.71389391)(405.26552712,672.73389374)
\curveto(405.22552324,672.74389388)(405.18552328,672.74889387)(405.14552712,672.74889374)
\curveto(405.11552335,672.74889387)(405.07552339,672.75389387)(405.02552712,672.76389374)
\curveto(404.92552354,672.79389383)(404.82552364,672.8188938)(404.72552712,672.83889374)
\curveto(404.62552384,672.85889376)(404.53052394,672.88889373)(404.44052712,672.92889374)
\curveto(404.32052415,672.96889365)(404.20552426,673.00889361)(404.09552712,673.04889374)
\curveto(403.99552447,673.08889353)(403.89052458,673.13889348)(403.78052712,673.19889374)
\curveto(403.43052504,673.40889321)(403.13052534,673.65389297)(402.88052712,673.93389374)
\curveto(402.63052584,674.21389241)(402.42052605,674.54889207)(402.25052712,674.93889374)
\curveto(402.20052627,675.02889159)(402.16052631,675.1238915)(402.13052712,675.22389374)
\curveto(402.11052636,675.3238913)(402.08552638,675.42889119)(402.05552712,675.53889374)
\curveto(402.03552643,675.58889103)(402.02552644,675.63389099)(402.02552712,675.67389374)
\curveto(402.02552644,675.71389091)(402.01552645,675.75889086)(401.99552712,675.80889374)
\curveto(401.97552649,675.88889073)(401.9655265,675.96889065)(401.96552712,676.04889374)
\curveto(401.9655265,676.13889048)(401.95552651,676.2238904)(401.93552712,676.30389374)
\curveto(401.92552654,676.35389027)(401.92052655,676.39889022)(401.92052712,676.43889374)
\lineto(401.92052712,676.57389374)
\curveto(401.90052657,676.63388999)(401.89052658,676.7188899)(401.89052712,676.82889374)
\curveto(401.90052657,676.93888968)(401.91552655,677.0238896)(401.93552712,677.08389374)
\lineto(401.93552712,677.18889374)
\curveto(401.94552652,677.23888938)(401.94552652,677.28888933)(401.93552712,677.33889374)
\curveto(401.93552653,677.39888922)(401.94552652,677.45388917)(401.96552712,677.50389374)
\curveto(401.97552649,677.55388907)(401.98052649,677.59888902)(401.98052712,677.63889374)
\curveto(401.98052649,677.68888893)(401.99052648,677.73888888)(402.01052712,677.78889374)
\curveto(402.05052642,677.9188887)(402.08552638,678.04388858)(402.11552712,678.16389374)
\curveto(402.14552632,678.29388833)(402.18552628,678.4188882)(402.23552712,678.53889374)
\curveto(402.41552605,678.94888767)(402.63052584,679.28888733)(402.88052712,679.55889374)
\curveto(403.13052534,679.83888678)(403.43552503,680.09388653)(403.79552712,680.32389374)
\curveto(403.89552457,680.37388625)(404.00052447,680.4188862)(404.11052712,680.45889374)
\curveto(404.22052425,680.49888612)(404.33052414,680.54388608)(404.44052712,680.59389374)
\curveto(404.5705239,680.64388598)(404.70552376,680.67888594)(404.84552712,680.69889374)
\curveto(404.98552348,680.7188859)(405.13052334,680.74888587)(405.28052712,680.78889374)
\curveto(405.36052311,680.79888582)(405.43552303,680.80388582)(405.50552712,680.80389374)
\curveto(405.57552289,680.80388582)(405.64552282,680.80888581)(405.71552712,680.81889374)
\curveto(406.29552217,680.82888579)(406.79552167,680.76888585)(407.21552712,680.63889374)
\curveto(407.64552082,680.50888611)(408.02552044,680.32888629)(408.35552712,680.09889374)
\curveto(408.46552,680.0188866)(408.57551989,679.92888669)(408.68552712,679.82889374)
\curveto(408.80551966,679.73888688)(408.90551956,679.63888698)(408.98552712,679.52889374)
\curveto(409.0655194,679.42888719)(409.13551933,679.32888729)(409.19552712,679.22889374)
\curveto(409.2655192,679.12888749)(409.33551913,679.0238876)(409.40552712,678.91389374)
\curveto(409.47551899,678.80388782)(409.53051894,678.68388794)(409.57052712,678.55389374)
\curveto(409.61051886,678.43388819)(409.65551881,678.30388832)(409.70552712,678.16389374)
\curveto(409.73551873,678.08388854)(409.76051871,677.99888862)(409.78052712,677.90889374)
\lineto(409.84052712,677.63889374)
\curveto(409.85051862,677.59888902)(409.85551861,677.55888906)(409.85552712,677.51889374)
\curveto(409.85551861,677.47888914)(409.86051861,677.43888918)(409.87052712,677.39889374)
\curveto(409.89051858,677.34888927)(409.89551857,677.29388933)(409.88552712,677.23389374)
\curveto(409.87551859,677.17388945)(409.88051859,677.1188895)(409.90052712,677.06889374)
\moveto(407.80052712,676.52889374)
\curveto(407.81052066,676.57889004)(407.81552065,676.64888997)(407.81552712,676.73889374)
\curveto(407.81552065,676.83888978)(407.81052066,676.91388971)(407.80052712,676.96389374)
\lineto(407.80052712,677.08389374)
\curveto(407.78052069,677.13388949)(407.7705207,677.18888943)(407.77052712,677.24889374)
\curveto(407.7705207,677.30888931)(407.7655207,677.36388926)(407.75552712,677.41389374)
\curveto(407.75552071,677.45388917)(407.75052072,677.48388914)(407.74052712,677.50389374)
\lineto(407.68052712,677.74389374)
\curveto(407.6705208,677.83388879)(407.65052082,677.9188887)(407.62052712,677.99889374)
\curveto(407.51052096,678.25888836)(407.38052109,678.47888814)(407.23052712,678.65889374)
\curveto(407.08052139,678.84888777)(406.88052159,678.99888762)(406.63052712,679.10889374)
\curveto(406.5705219,679.12888749)(406.51052196,679.14388748)(406.45052712,679.15389374)
\curveto(406.39052208,679.17388745)(406.32552214,679.19388743)(406.25552712,679.21389374)
\curveto(406.17552229,679.23388739)(406.09052238,679.23888738)(406.00052712,679.22889374)
\lineto(405.73052712,679.22889374)
\curveto(405.70052277,679.20888741)(405.6655228,679.19888742)(405.62552712,679.19889374)
\curveto(405.58552288,679.20888741)(405.55052292,679.20888741)(405.52052712,679.19889374)
\lineto(405.31052712,679.13889374)
\curveto(405.25052322,679.12888749)(405.19552327,679.10888751)(405.14552712,679.07889374)
\curveto(404.89552357,678.96888765)(404.69052378,678.80888781)(404.53052712,678.59889374)
\curveto(404.38052409,678.39888822)(404.26052421,678.16388846)(404.17052712,677.89389374)
\curveto(404.14052433,677.79388883)(404.11552435,677.68888893)(404.09552712,677.57889374)
\curveto(404.08552438,677.46888915)(404.0705244,677.35888926)(404.05052712,677.24889374)
\curveto(404.04052443,677.19888942)(404.03552443,677.14888947)(404.03552712,677.09889374)
\lineto(404.03552712,676.94889374)
\curveto(404.01552445,676.87888974)(404.00552446,676.77388985)(404.00552712,676.63389374)
\curveto(404.01552445,676.49389013)(404.03052444,676.38889023)(404.05052712,676.31889374)
\lineto(404.05052712,676.18389374)
\curveto(404.0705244,676.10389052)(404.08552438,676.0238906)(404.09552712,675.94389374)
\curveto(404.10552436,675.87389075)(404.12052435,675.79889082)(404.14052712,675.71889374)
\curveto(404.24052423,675.4188912)(404.34552412,675.17389145)(404.45552712,674.98389374)
\curveto(404.57552389,674.80389182)(404.76052371,674.63889198)(405.01052712,674.48889374)
\curveto(405.08052339,674.43889218)(405.15552331,674.39889222)(405.23552712,674.36889374)
\curveto(405.32552314,674.33889228)(405.41552305,674.31389231)(405.50552712,674.29389374)
\curveto(405.54552292,674.28389234)(405.58052289,674.27889234)(405.61052712,674.27889374)
\curveto(405.64052283,674.28889233)(405.67552279,674.28889233)(405.71552712,674.27889374)
\lineto(405.83552712,674.24889374)
\curveto(405.88552258,674.24889237)(405.93052254,674.25389237)(405.97052712,674.26389374)
\lineto(406.09052712,674.26389374)
\curveto(406.1705223,674.28389234)(406.25052222,674.29889232)(406.33052712,674.30889374)
\curveto(406.41052206,674.3188923)(406.48552198,674.33889228)(406.55552712,674.36889374)
\curveto(406.81552165,674.46889215)(407.02552144,674.60389202)(407.18552712,674.77389374)
\curveto(407.34552112,674.94389168)(407.48052099,675.15389147)(407.59052712,675.40389374)
\curveto(407.63052084,675.50389112)(407.66052081,675.60389102)(407.68052712,675.70389374)
\curveto(407.70052077,675.80389082)(407.72552074,675.90889071)(407.75552712,676.01889374)
\curveto(407.7655207,676.05889056)(407.7705207,676.09389053)(407.77052712,676.12389374)
\curveto(407.7705207,676.16389046)(407.77552069,676.20389042)(407.78552712,676.24389374)
\lineto(407.78552712,676.37889374)
\curveto(407.78552068,676.42889019)(407.79052068,676.47889014)(407.80052712,676.52889374)
}
}
{
\newrgbcolor{curcolor}{0 0 0}
\pscustom[linestyle=none,fillstyle=solid,fillcolor=curcolor]
{
\newpath
\moveto(415.75544899,680.81889374)
\curveto(416.12544339,680.82888579)(416.45044306,680.78888583)(416.73044899,680.69889374)
\curveto(417.0104425,680.60888601)(417.25544226,680.48388614)(417.46544899,680.32389374)
\curveto(417.54544197,680.26388636)(417.6154419,680.19388643)(417.67544899,680.11389374)
\curveto(417.74544177,680.03388659)(417.82044169,679.95388667)(417.90044899,679.87389374)
\curveto(417.92044159,679.85388677)(417.95044156,679.8238868)(417.99044899,679.78389374)
\curveto(418.04044147,679.75388687)(418.09044142,679.74888687)(418.14044899,679.76889374)
\curveto(418.25044126,679.79888682)(418.35544116,679.86888675)(418.45544899,679.97889374)
\curveto(418.55544096,680.09888652)(418.65044086,680.18888643)(418.74044899,680.24889374)
\curveto(418.88044063,680.35888626)(419.03044048,680.44888617)(419.19044899,680.51889374)
\curveto(419.35044016,680.59888602)(419.53043998,680.67388595)(419.73044899,680.74389374)
\curveto(419.8104397,680.76388586)(419.90543961,680.77888584)(420.01544899,680.78889374)
\curveto(420.13543938,680.80888581)(420.25543926,680.8188858)(420.37544899,680.81889374)
\curveto(420.50543901,680.82888579)(420.62543889,680.82888579)(420.73544899,680.81889374)
\curveto(420.85543866,680.80888581)(420.96043855,680.79388583)(421.05044899,680.77389374)
\curveto(421.10043841,680.76388586)(421.14543837,680.75888586)(421.18544899,680.75889374)
\curveto(421.22543829,680.75888586)(421.27043824,680.74888587)(421.32044899,680.72889374)
\curveto(421.46043805,680.68888593)(421.59543792,680.64888597)(421.72544899,680.60889374)
\curveto(421.85543766,680.56888605)(421.97543754,680.51388611)(422.08544899,680.44389374)
\curveto(422.50543701,680.18388644)(422.82043669,679.80388682)(423.03044899,679.30389374)
\curveto(423.07043644,679.21388741)(423.10043641,679.1188875)(423.12044899,679.01889374)
\curveto(423.14043637,678.92888769)(423.16043635,678.83888778)(423.18044899,678.74889374)
\curveto(423.19043632,678.67888794)(423.19543632,678.61388801)(423.19544899,678.55389374)
\curveto(423.20543631,678.49388813)(423.2154363,678.43388819)(423.22544899,678.37389374)
\lineto(423.22544899,678.22389374)
\curveto(423.23543628,678.16388846)(423.23543628,678.09388853)(423.22544899,678.01389374)
\curveto(423.22543629,677.93388869)(423.22543629,677.85888876)(423.22544899,677.78889374)
\lineto(423.22544899,676.91889374)
\lineto(423.22544899,673.99389374)
\curveto(423.22543629,673.91389271)(423.22543629,673.8188928)(423.22544899,673.70889374)
\curveto(423.23543628,673.60889301)(423.23543628,673.50889311)(423.22544899,673.40889374)
\curveto(423.22543629,673.3188933)(423.2154363,673.22889339)(423.19544899,673.13889374)
\curveto(423.17543634,673.05889356)(423.14543637,673.00389362)(423.10544899,672.97389374)
\curveto(423.04543647,672.9238937)(422.96543655,672.89389373)(422.86544899,672.88389374)
\lineto(422.56544899,672.88389374)
\lineto(421.77044899,672.88389374)
\curveto(421.63043788,672.88389374)(421.50543801,672.89389373)(421.39544899,672.91389374)
\curveto(421.28543823,672.93389369)(421.2104383,672.98889363)(421.17044899,673.07889374)
\curveto(421.14043837,673.14889347)(421.12543839,673.2238934)(421.12544899,673.30389374)
\curveto(421.12543839,673.39389323)(421.12543839,673.47889314)(421.12544899,673.55889374)
\lineto(421.12544899,674.39889374)
\lineto(421.12544899,676.42389374)
\lineto(421.12544899,677.05389374)
\curveto(421.12543839,677.10388952)(421.12543839,677.15888946)(421.12544899,677.21889374)
\curveto(421.13543838,677.27888934)(421.13043838,677.33388929)(421.11044899,677.38389374)
\lineto(421.11044899,677.50389374)
\curveto(421.1104384,677.56388906)(421.1104384,677.623889)(421.11044899,677.68389374)
\curveto(421.1104384,677.74388888)(421.10543841,677.80388882)(421.09544899,677.86389374)
\curveto(421.08543843,677.90388872)(421.08043843,677.94388868)(421.08044899,677.98389374)
\curveto(421.08043843,678.03388859)(421.07543844,678.07888854)(421.06544899,678.11889374)
\curveto(421.02543849,678.26888835)(420.98043853,678.39888822)(420.93044899,678.50889374)
\curveto(420.89043862,678.62888799)(420.82543869,678.73388789)(420.73544899,678.82389374)
\curveto(420.59543892,678.96388766)(420.42543909,679.06388756)(420.22544899,679.12389374)
\curveto(420.18543933,679.13388749)(420.15043936,679.13388749)(420.12044899,679.12389374)
\curveto(420.09043942,679.1238875)(420.05543946,679.13388749)(420.01544899,679.15389374)
\curveto(419.97543954,679.16388746)(419.92543959,679.16888745)(419.86544899,679.16889374)
\curveto(419.8154397,679.17888744)(419.76543975,679.17888744)(419.71544899,679.16889374)
\curveto(419.65543986,679.14888747)(419.59543992,679.13888748)(419.53544899,679.13889374)
\curveto(419.47544004,679.13888748)(419.4154401,679.12888749)(419.35544899,679.10889374)
\curveto(419.06544045,679.00888761)(418.85544066,678.85888776)(418.72544899,678.65889374)
\curveto(418.55544096,678.42888819)(418.45044106,678.13888848)(418.41044899,677.78889374)
\curveto(418.38044113,677.44888917)(418.36544115,677.07388955)(418.36544899,676.66389374)
\lineto(418.36544899,674.68389374)
\lineto(418.36544899,673.57389374)
\lineto(418.36544899,673.27389374)
\curveto(418.36544115,673.17389345)(418.34044117,673.09389353)(418.29044899,673.03389374)
\curveto(418.24044127,672.96389366)(418.16544135,672.9188937)(418.06544899,672.89889374)
\curveto(417.97544154,672.88889373)(417.87044164,672.88389374)(417.75044899,672.88389374)
\lineto(416.94044899,672.88389374)
\lineto(416.67044899,672.88389374)
\curveto(416.59044292,672.89389373)(416.52044299,672.90889371)(416.46044899,672.92889374)
\curveto(416.36044315,672.97889364)(416.30044321,673.05889356)(416.28044899,673.16889374)
\curveto(416.27044324,673.27889334)(416.26544325,673.40389322)(416.26544899,673.54389374)
\lineto(416.26544899,674.81889374)
\lineto(416.26544899,677.17389374)
\curveto(416.26544325,677.46388916)(416.25544326,677.73888888)(416.23544899,677.99889374)
\curveto(416.2154433,678.25888836)(416.15044336,678.47388815)(416.04044899,678.64389374)
\curveto(415.96044355,678.78388784)(415.85544366,678.88888773)(415.72544899,678.95889374)
\curveto(415.60544391,679.02888759)(415.45544406,679.08888753)(415.27544899,679.13889374)
\curveto(415.23544428,679.14888747)(415.19544432,679.14888747)(415.15544899,679.13889374)
\curveto(415.1154444,679.13888748)(415.07044444,679.14388748)(415.02044899,679.15389374)
\curveto(414.9104446,679.17388745)(414.80544471,679.16388746)(414.70544899,679.12389374)
\curveto(414.68544483,679.1238875)(414.66544485,679.1188875)(414.64544899,679.10889374)
\lineto(414.58544899,679.10889374)
\curveto(414.42544509,679.05888756)(414.27044524,678.97388765)(414.12044899,678.85389374)
\curveto(413.96044555,678.73388789)(413.83544568,678.59388803)(413.74544899,678.43389374)
\curveto(413.66544585,678.28388834)(413.60544591,678.10888851)(413.56544899,677.90889374)
\curveto(413.53544598,677.7188889)(413.515446,677.50888911)(413.50544899,677.27889374)
\lineto(413.50544899,676.52889374)
\lineto(413.50544899,674.50389374)
\lineto(413.50544899,673.58889374)
\lineto(413.50544899,673.31889374)
\curveto(413.50544601,673.22889339)(413.49044602,673.14889347)(413.46044899,673.07889374)
\curveto(413.42044609,672.98889363)(413.34544617,672.93389369)(413.23544899,672.91389374)
\curveto(413.12544639,672.89389373)(413.00044651,672.88389374)(412.86044899,672.88389374)
\lineto(412.08044899,672.88389374)
\lineto(411.78044899,672.88389374)
\curveto(411.69044782,672.89389373)(411.6154479,672.9188937)(411.55544899,672.95889374)
\curveto(411.46544805,673.00889361)(411.4154481,673.09889352)(411.40544899,673.22889374)
\lineto(411.40544899,673.66389374)
\lineto(411.40544899,675.41889374)
\lineto(411.40544899,679.07889374)
\lineto(411.40544899,679.97889374)
\lineto(411.40544899,680.26389374)
\curveto(411.4154481,680.35388627)(411.44044807,680.42888619)(411.48044899,680.48889374)
\curveto(411.53044798,680.54888607)(411.6104479,680.58888603)(411.72044899,680.60889374)
\lineto(411.81044899,680.60889374)
\curveto(411.86044765,680.618886)(411.9104476,680.623886)(411.96044899,680.62389374)
\lineto(412.12544899,680.62389374)
\lineto(412.74044899,680.62389374)
\curveto(412.82044669,680.623886)(412.89544662,680.618886)(412.96544899,680.60889374)
\curveto(413.04544647,680.60888601)(413.1154464,680.59888602)(413.17544899,680.57889374)
\curveto(413.25544626,680.54888607)(413.30544621,680.49888612)(413.32544899,680.42889374)
\curveto(413.35544616,680.35888626)(413.38044613,680.27888634)(413.40044899,680.18889374)
\curveto(413.4104461,680.15888646)(413.4104461,680.12888649)(413.40044899,680.09889374)
\curveto(413.40044611,680.07888654)(413.4104461,680.05888656)(413.43044899,680.03889374)
\curveto(413.44044607,680.00888661)(413.45044606,679.98388664)(413.46044899,679.96389374)
\curveto(413.48044603,679.95388667)(413.50044601,679.93888668)(413.52044899,679.91889374)
\curveto(413.64044587,679.90888671)(413.74044577,679.94388668)(413.82044899,680.02389374)
\curveto(413.90044561,680.11388651)(413.97544554,680.18388644)(414.04544899,680.23389374)
\curveto(414.18544533,680.33388629)(414.32544519,680.4238862)(414.46544899,680.50389374)
\curveto(414.6154449,680.58388604)(414.77544474,680.64888597)(414.94544899,680.69889374)
\curveto(415.03544448,680.72888589)(415.12544439,680.74888587)(415.21544899,680.75889374)
\curveto(415.30544421,680.76888585)(415.40044411,680.78388584)(415.50044899,680.80389374)
\curveto(415.53044398,680.81388581)(415.57544394,680.81388581)(415.63544899,680.80389374)
\curveto(415.69544382,680.80388582)(415.73544378,680.80888581)(415.75544899,680.81889374)
}
}
{
\newrgbcolor{curcolor}{0 0 0}
\pscustom[linestyle=none,fillstyle=solid,fillcolor=curcolor]
{
\newpath
\moveto(432.69419899,677.14389374)
\curveto(432.71419039,677.08388954)(432.72419038,676.97888964)(432.72419899,676.82889374)
\curveto(432.72419038,676.68888993)(432.71919039,676.58889003)(432.70919899,676.52889374)
\curveto(432.7091904,676.47889014)(432.7041904,676.43389019)(432.69419899,676.39389374)
\lineto(432.69419899,676.27389374)
\curveto(432.67419043,676.19389043)(432.66419044,676.11389051)(432.66419899,676.03389374)
\curveto(432.66419044,675.96389066)(432.65419045,675.88889073)(432.63419899,675.80889374)
\curveto(432.63419047,675.76889085)(432.62419048,675.69889092)(432.60419899,675.59889374)
\curveto(432.57419053,675.47889114)(432.54419056,675.35389127)(432.51419899,675.22389374)
\curveto(432.49419061,675.10389152)(432.45919065,674.98889163)(432.40919899,674.87889374)
\curveto(432.22919088,674.42889219)(432.0041911,674.03889258)(431.73419899,673.70889374)
\curveto(431.46419164,673.37889324)(431.109192,673.1188935)(430.66919899,672.92889374)
\curveto(430.57919253,672.88889373)(430.48419262,672.85889376)(430.38419899,672.83889374)
\curveto(430.29419281,672.80889381)(430.19419291,672.77889384)(430.08419899,672.74889374)
\curveto(430.02419308,672.72889389)(429.95919315,672.7188939)(429.88919899,672.71889374)
\curveto(429.82919328,672.7188939)(429.76919334,672.71389391)(429.70919899,672.70389374)
\lineto(429.57419899,672.70389374)
\curveto(429.51419359,672.68389394)(429.43419367,672.67889394)(429.33419899,672.68889374)
\curveto(429.23419387,672.68889393)(429.15419395,672.69889392)(429.09419899,672.71889374)
\lineto(429.00419899,672.71889374)
\curveto(428.95419415,672.72889389)(428.89919421,672.73889388)(428.83919899,672.74889374)
\curveto(428.77919433,672.74889387)(428.71919439,672.75389387)(428.65919899,672.76389374)
\curveto(428.46919464,672.81389381)(428.29419481,672.86389376)(428.13419899,672.91389374)
\curveto(427.97419513,672.96389366)(427.82419528,673.03389359)(427.68419899,673.12389374)
\lineto(427.50419899,673.24389374)
\curveto(427.45419565,673.28389334)(427.4041957,673.32889329)(427.35419899,673.37889374)
\lineto(427.26419899,673.43889374)
\curveto(427.23419587,673.45889316)(427.2041959,673.47389315)(427.17419899,673.48389374)
\curveto(427.08419602,673.51389311)(427.02919608,673.49389313)(427.00919899,673.42389374)
\curveto(426.95919615,673.35389327)(426.92419618,673.26889335)(426.90419899,673.16889374)
\curveto(426.89419621,673.07889354)(426.85919625,673.00889361)(426.79919899,672.95889374)
\curveto(426.73919637,672.9188937)(426.66919644,672.89389373)(426.58919899,672.88389374)
\lineto(426.31919899,672.88389374)
\lineto(425.59919899,672.88389374)
\lineto(425.37419899,672.88389374)
\curveto(425.3041978,672.87389375)(425.23919787,672.87889374)(425.17919899,672.89889374)
\curveto(425.03919807,672.94889367)(424.95919815,673.03889358)(424.93919899,673.16889374)
\curveto(424.92919818,673.30889331)(424.92419818,673.46389316)(424.92419899,673.63389374)
\lineto(424.92419899,682.78389374)
\lineto(424.92419899,683.12889374)
\curveto(424.92419818,683.24888337)(424.94919816,683.34388328)(424.99919899,683.41389374)
\curveto(425.03919807,683.48388314)(425.109198,683.52888309)(425.20919899,683.54889374)
\curveto(425.22919788,683.55888306)(425.24919786,683.55888306)(425.26919899,683.54889374)
\curveto(425.29919781,683.54888307)(425.32419778,683.55388307)(425.34419899,683.56389374)
\lineto(426.28919899,683.56389374)
\curveto(426.46919664,683.56388306)(426.62419648,683.55388307)(426.75419899,683.53389374)
\curveto(426.88419622,683.5238831)(426.96919614,683.44888317)(427.00919899,683.30889374)
\curveto(427.03919607,683.20888341)(427.04919606,683.07388355)(427.03919899,682.90389374)
\curveto(427.02919608,682.74388388)(427.02419608,682.60388402)(427.02419899,682.48389374)
\lineto(427.02419899,680.84889374)
\lineto(427.02419899,680.51889374)
\curveto(427.02419608,680.40888621)(427.03419607,680.31388631)(427.05419899,680.23389374)
\curveto(427.06419604,680.18388644)(427.07419603,680.13888648)(427.08419899,680.09889374)
\curveto(427.09419601,680.06888655)(427.11919599,680.04888657)(427.15919899,680.03889374)
\curveto(427.17919593,680.0188866)(427.2041959,680.00888661)(427.23419899,680.00889374)
\curveto(427.27419583,680.00888661)(427.3041958,680.01388661)(427.32419899,680.02389374)
\curveto(427.39419571,680.06388656)(427.45919565,680.10388652)(427.51919899,680.14389374)
\curveto(427.57919553,680.19388643)(427.64419546,680.24388638)(427.71419899,680.29389374)
\curveto(427.84419526,680.38388624)(427.97919513,680.45888616)(428.11919899,680.51889374)
\curveto(428.25919485,680.58888603)(428.41419469,680.64888597)(428.58419899,680.69889374)
\curveto(428.66419444,680.72888589)(428.74419436,680.74388588)(428.82419899,680.74389374)
\curveto(428.9041942,680.75388587)(428.98419412,680.76888585)(429.06419899,680.78889374)
\curveto(429.13419397,680.80888581)(429.2091939,680.8188858)(429.28919899,680.81889374)
\lineto(429.52919899,680.81889374)
\lineto(429.67919899,680.81889374)
\curveto(429.7091934,680.80888581)(429.74419336,680.80388582)(429.78419899,680.80389374)
\curveto(429.82419328,680.81388581)(429.86419324,680.81388581)(429.90419899,680.80389374)
\curveto(430.01419309,680.77388585)(430.11419299,680.74888587)(430.20419899,680.72889374)
\curveto(430.3041928,680.7188859)(430.39919271,680.69388593)(430.48919899,680.65389374)
\curveto(430.94919216,680.46388616)(431.32419178,680.2188864)(431.61419899,679.91889374)
\curveto(431.9041912,679.618887)(432.14919096,679.24388738)(432.34919899,678.79389374)
\curveto(432.39919071,678.67388795)(432.43919067,678.54888807)(432.46919899,678.41889374)
\curveto(432.5091906,678.28888833)(432.54919056,678.15388847)(432.58919899,678.01389374)
\curveto(432.6091905,677.94388868)(432.61919049,677.87388875)(432.61919899,677.80389374)
\curveto(432.62919048,677.74388888)(432.64419046,677.67388895)(432.66419899,677.59389374)
\curveto(432.68419042,677.54388908)(432.68919042,677.48888913)(432.67919899,677.42889374)
\curveto(432.67919043,677.36888925)(432.68419042,677.30888931)(432.69419899,677.24889374)
\lineto(432.69419899,677.14389374)
\moveto(430.47419899,675.73389374)
\curveto(430.5041926,675.83389079)(430.52919258,675.95889066)(430.54919899,676.10889374)
\curveto(430.57919253,676.25889036)(430.59419251,676.40889021)(430.59419899,676.55889374)
\curveto(430.6041925,676.7188899)(430.6041925,676.87388975)(430.59419899,677.02389374)
\curveto(430.59419251,677.18388944)(430.57919253,677.3188893)(430.54919899,677.42889374)
\curveto(430.51919259,677.52888909)(430.49919261,677.623889)(430.48919899,677.71389374)
\curveto(430.47919263,677.80388882)(430.45419265,677.88888873)(430.41419899,677.96889374)
\curveto(430.27419283,678.3188883)(430.07419303,678.61388801)(429.81419899,678.85389374)
\curveto(429.56419354,679.10388752)(429.19419391,679.22888739)(428.70419899,679.22889374)
\curveto(428.66419444,679.22888739)(428.62919448,679.2238874)(428.59919899,679.21389374)
\lineto(428.49419899,679.21389374)
\curveto(428.42419468,679.19388743)(428.35919475,679.17388745)(428.29919899,679.15389374)
\curveto(428.23919487,679.14388748)(428.17919493,679.12888749)(428.11919899,679.10889374)
\curveto(427.82919528,678.97888764)(427.6091955,678.79388783)(427.45919899,678.55389374)
\curveto(427.3091958,678.3238883)(427.18419592,678.05888856)(427.08419899,677.75889374)
\curveto(427.05419605,677.67888894)(427.03419607,677.59388903)(427.02419899,677.50389374)
\curveto(427.02419608,677.4238892)(427.01419609,677.34388928)(426.99419899,677.26389374)
\curveto(426.98419612,677.23388939)(426.97919613,677.18388944)(426.97919899,677.11389374)
\curveto(426.96919614,677.07388955)(426.96419614,677.03388959)(426.96419899,676.99389374)
\curveto(426.97419613,676.95388967)(426.97419613,676.91388971)(426.96419899,676.87389374)
\curveto(426.94419616,676.79388983)(426.93919617,676.68388994)(426.94919899,676.54389374)
\curveto(426.95919615,676.40389022)(426.97419613,676.30389032)(426.99419899,676.24389374)
\curveto(427.01419609,676.15389047)(427.02419608,676.06889055)(427.02419899,675.98889374)
\curveto(427.03419607,675.90889071)(427.05419605,675.82889079)(427.08419899,675.74889374)
\curveto(427.17419593,675.46889115)(427.27919583,675.2238914)(427.39919899,675.01389374)
\curveto(427.52919558,674.81389181)(427.7091954,674.64389198)(427.93919899,674.50389374)
\curveto(428.09919501,674.40389222)(428.26419484,674.33389229)(428.43419899,674.29389374)
\curveto(428.45419465,674.29389233)(428.47419463,674.28889233)(428.49419899,674.27889374)
\lineto(428.58419899,674.27889374)
\curveto(428.61419449,674.26889235)(428.66419444,674.25889236)(428.73419899,674.24889374)
\curveto(428.8041943,674.24889237)(428.86419424,674.25389237)(428.91419899,674.26389374)
\curveto(429.01419409,674.28389234)(429.104194,674.29889232)(429.18419899,674.30889374)
\curveto(429.27419383,674.32889229)(429.35919375,674.35389227)(429.43919899,674.38389374)
\curveto(429.71919339,674.51389211)(429.93419317,674.69389193)(430.08419899,674.92389374)
\curveto(430.24419286,675.15389147)(430.37419273,675.4238912)(430.47419899,675.73389374)
}
}
{
\newrgbcolor{curcolor}{0 0 0}
\pscustom[linestyle=none,fillstyle=solid,fillcolor=curcolor]
{
\newpath
\moveto(438.48912087,680.81889374)
\curveto(438.59911555,680.8188858)(438.69411546,680.80888581)(438.77412087,680.78889374)
\curveto(438.86411529,680.76888585)(438.93411522,680.7238859)(438.98412087,680.65389374)
\curveto(439.04411511,680.57388605)(439.07411508,680.43388619)(439.07412087,680.23389374)
\lineto(439.07412087,679.72389374)
\lineto(439.07412087,679.34889374)
\curveto(439.08411507,679.20888741)(439.06911508,679.09888752)(439.02912087,679.01889374)
\curveto(438.98911516,678.94888767)(438.92911522,678.90388772)(438.84912087,678.88389374)
\curveto(438.77911537,678.86388776)(438.69411546,678.85388777)(438.59412087,678.85389374)
\curveto(438.50411565,678.85388777)(438.40411575,678.85888776)(438.29412087,678.86889374)
\curveto(438.19411596,678.87888774)(438.09911605,678.87388775)(438.00912087,678.85389374)
\curveto(437.93911621,678.83388779)(437.86911628,678.8188878)(437.79912087,678.80889374)
\curveto(437.72911642,678.80888781)(437.66411649,678.79888782)(437.60412087,678.77889374)
\curveto(437.44411671,678.72888789)(437.28411687,678.65388797)(437.12412087,678.55389374)
\curveto(436.96411719,678.46388816)(436.83911731,678.35888826)(436.74912087,678.23889374)
\curveto(436.69911745,678.15888846)(436.64411751,678.07388855)(436.58412087,677.98389374)
\curveto(436.53411762,677.90388872)(436.48411767,677.8188888)(436.43412087,677.72889374)
\curveto(436.40411775,677.64888897)(436.37411778,677.56388906)(436.34412087,677.47389374)
\lineto(436.28412087,677.23389374)
\curveto(436.26411789,677.16388946)(436.2541179,677.08888953)(436.25412087,677.00889374)
\curveto(436.2541179,676.93888968)(436.24411791,676.86888975)(436.22412087,676.79889374)
\curveto(436.21411794,676.75888986)(436.20911794,676.7188899)(436.20912087,676.67889374)
\curveto(436.21911793,676.64888997)(436.21911793,676.61889)(436.20912087,676.58889374)
\lineto(436.20912087,676.34889374)
\curveto(436.18911796,676.27889034)(436.18411797,676.19889042)(436.19412087,676.10889374)
\curveto(436.20411795,676.02889059)(436.20911794,675.94889067)(436.20912087,675.86889374)
\lineto(436.20912087,674.90889374)
\lineto(436.20912087,673.63389374)
\curveto(436.20911794,673.50389312)(436.20411795,673.38389324)(436.19412087,673.27389374)
\curveto(436.18411797,673.16389346)(436.154118,673.07389355)(436.10412087,673.00389374)
\curveto(436.08411807,672.97389365)(436.0491181,672.94889367)(435.99912087,672.92889374)
\curveto(435.95911819,672.9188937)(435.91411824,672.90889371)(435.86412087,672.89889374)
\lineto(435.78912087,672.89889374)
\curveto(435.73911841,672.88889373)(435.68411847,672.88389374)(435.62412087,672.88389374)
\lineto(435.45912087,672.88389374)
\lineto(434.81412087,672.88389374)
\curveto(434.7541194,672.89389373)(434.68911946,672.89889372)(434.61912087,672.89889374)
\lineto(434.42412087,672.89889374)
\curveto(434.37411978,672.9188937)(434.32411983,672.93389369)(434.27412087,672.94389374)
\curveto(434.22411993,672.96389366)(434.18911996,672.99889362)(434.16912087,673.04889374)
\curveto(434.12912002,673.09889352)(434.10412005,673.16889345)(434.09412087,673.25889374)
\lineto(434.09412087,673.55889374)
\lineto(434.09412087,674.57889374)
\lineto(434.09412087,678.80889374)
\lineto(434.09412087,679.91889374)
\lineto(434.09412087,680.20389374)
\curveto(434.09412006,680.30388632)(434.11412004,680.38388624)(434.15412087,680.44389374)
\curveto(434.20411995,680.5238861)(434.27911987,680.57388605)(434.37912087,680.59389374)
\curveto(434.47911967,680.61388601)(434.59911955,680.623886)(434.73912087,680.62389374)
\lineto(435.50412087,680.62389374)
\curveto(435.62411853,680.623886)(435.72911842,680.61388601)(435.81912087,680.59389374)
\curveto(435.90911824,680.58388604)(435.97911817,680.53888608)(436.02912087,680.45889374)
\curveto(436.05911809,680.40888621)(436.07411808,680.33888628)(436.07412087,680.24889374)
\lineto(436.10412087,679.97889374)
\curveto(436.11411804,679.89888672)(436.12911802,679.8238868)(436.14912087,679.75389374)
\curveto(436.17911797,679.68388694)(436.22911792,679.64888697)(436.29912087,679.64889374)
\curveto(436.31911783,679.66888695)(436.33911781,679.67888694)(436.35912087,679.67889374)
\curveto(436.37911777,679.67888694)(436.39911775,679.68888693)(436.41912087,679.70889374)
\curveto(436.47911767,679.75888686)(436.52911762,679.81388681)(436.56912087,679.87389374)
\curveto(436.61911753,679.94388668)(436.67911747,680.00388662)(436.74912087,680.05389374)
\curveto(436.78911736,680.08388654)(436.82411733,680.11388651)(436.85412087,680.14389374)
\curveto(436.88411727,680.18388644)(436.91911723,680.2188864)(436.95912087,680.24889374)
\lineto(437.22912087,680.42889374)
\curveto(437.32911682,680.48888613)(437.42911672,680.54388608)(437.52912087,680.59389374)
\curveto(437.62911652,680.63388599)(437.72911642,680.66888595)(437.82912087,680.69889374)
\lineto(438.15912087,680.78889374)
\curveto(438.18911596,680.79888582)(438.24411591,680.79888582)(438.32412087,680.78889374)
\curveto(438.41411574,680.78888583)(438.46911568,680.79888582)(438.48912087,680.81889374)
}
}
{
\newrgbcolor{curcolor}{0 0 0}
\pscustom[linestyle=none,fillstyle=solid,fillcolor=curcolor]
{
\newpath
\moveto(446.99552712,676.82889374)
\curveto(447.01551895,676.74888987)(447.01551895,676.65888996)(446.99552712,676.55889374)
\curveto(446.97551899,676.45889016)(446.94051903,676.39389023)(446.89052712,676.36389374)
\curveto(446.84051913,676.3238903)(446.7655192,676.29389033)(446.66552712,676.27389374)
\curveto(446.57551939,676.26389036)(446.4705195,676.25389037)(446.35052712,676.24389374)
\lineto(446.00552712,676.24389374)
\curveto(445.89552007,676.25389037)(445.79552017,676.25889036)(445.70552712,676.25889374)
\lineto(442.04552712,676.25889374)
\lineto(441.83552712,676.25889374)
\curveto(441.77552419,676.25889036)(441.72052425,676.24889037)(441.67052712,676.22889374)
\curveto(441.59052438,676.18889043)(441.54052443,676.14889047)(441.52052712,676.10889374)
\curveto(441.50052447,676.08889053)(441.48052449,676.04889057)(441.46052712,675.98889374)
\curveto(441.44052453,675.93889068)(441.43552453,675.88889073)(441.44552712,675.83889374)
\curveto(441.4655245,675.77889084)(441.47552449,675.7188909)(441.47552712,675.65889374)
\curveto(441.48552448,675.60889101)(441.50052447,675.55389107)(441.52052712,675.49389374)
\curveto(441.60052437,675.25389137)(441.69552427,675.05389157)(441.80552712,674.89389374)
\curveto(441.92552404,674.74389188)(442.08552388,674.60889201)(442.28552712,674.48889374)
\curveto(442.3655236,674.43889218)(442.44552352,674.40389222)(442.52552712,674.38389374)
\curveto(442.61552335,674.37389225)(442.70552326,674.35389227)(442.79552712,674.32389374)
\curveto(442.87552309,674.30389232)(442.98552298,674.28889233)(443.12552712,674.27889374)
\curveto(443.2655227,674.26889235)(443.38552258,674.27389235)(443.48552712,674.29389374)
\lineto(443.62052712,674.29389374)
\curveto(443.72052225,674.31389231)(443.81052216,674.33389229)(443.89052712,674.35389374)
\curveto(443.98052199,674.38389224)(444.0655219,674.41389221)(444.14552712,674.44389374)
\curveto(444.24552172,674.49389213)(444.35552161,674.55889206)(444.47552712,674.63889374)
\curveto(444.60552136,674.7188919)(444.70052127,674.79889182)(444.76052712,674.87889374)
\curveto(444.81052116,674.94889167)(444.86052111,675.01389161)(444.91052712,675.07389374)
\curveto(444.970521,675.14389148)(445.04052093,675.19389143)(445.12052712,675.22389374)
\curveto(445.22052075,675.27389135)(445.34552062,675.29389133)(445.49552712,675.28389374)
\lineto(445.93052712,675.28389374)
\lineto(446.11052712,675.28389374)
\curveto(446.18051979,675.29389133)(446.24051973,675.28889133)(446.29052712,675.26889374)
\lineto(446.44052712,675.26889374)
\curveto(446.54051943,675.24889137)(446.61051936,675.2238914)(446.65052712,675.19389374)
\curveto(446.69051928,675.17389145)(446.71051926,675.12889149)(446.71052712,675.05889374)
\curveto(446.72051925,674.98889163)(446.71551925,674.92889169)(446.69552712,674.87889374)
\curveto(446.64551932,674.73889188)(446.59051938,674.61389201)(446.53052712,674.50389374)
\curveto(446.4705195,674.39389223)(446.40051957,674.28389234)(446.32052712,674.17389374)
\curveto(446.10051987,673.84389278)(445.85052012,673.57889304)(445.57052712,673.37889374)
\curveto(445.29052068,673.17889344)(444.94052103,673.00889361)(444.52052712,672.86889374)
\curveto(444.41052156,672.82889379)(444.30052167,672.80389382)(444.19052712,672.79389374)
\curveto(444.08052189,672.78389384)(443.965522,672.76389386)(443.84552712,672.73389374)
\curveto(443.80552216,672.7238939)(443.76052221,672.7238939)(443.71052712,672.73389374)
\curveto(443.6705223,672.73389389)(443.63052234,672.72889389)(443.59052712,672.71889374)
\lineto(443.42552712,672.71889374)
\curveto(443.37552259,672.69889392)(443.31552265,672.69389393)(443.24552712,672.70389374)
\curveto(443.18552278,672.70389392)(443.13052284,672.70889391)(443.08052712,672.71889374)
\curveto(443.00052297,672.72889389)(442.93052304,672.72889389)(442.87052712,672.71889374)
\curveto(442.81052316,672.70889391)(442.74552322,672.71389391)(442.67552712,672.73389374)
\curveto(442.62552334,672.75389387)(442.5705234,672.76389386)(442.51052712,672.76389374)
\curveto(442.45052352,672.76389386)(442.39552357,672.77389385)(442.34552712,672.79389374)
\curveto(442.23552373,672.81389381)(442.12552384,672.83889378)(442.01552712,672.86889374)
\curveto(441.90552406,672.88889373)(441.80552416,672.9238937)(441.71552712,672.97389374)
\curveto(441.60552436,673.01389361)(441.50052447,673.04889357)(441.40052712,673.07889374)
\curveto(441.31052466,673.1188935)(441.22552474,673.16389346)(441.14552712,673.21389374)
\curveto(440.82552514,673.41389321)(440.54052543,673.64389298)(440.29052712,673.90389374)
\curveto(440.04052593,674.17389245)(439.83552613,674.48389214)(439.67552712,674.83389374)
\curveto(439.62552634,674.94389168)(439.58552638,675.05389157)(439.55552712,675.16389374)
\curveto(439.52552644,675.28389134)(439.48552648,675.40389122)(439.43552712,675.52389374)
\curveto(439.42552654,675.56389106)(439.42052655,675.59889102)(439.42052712,675.62889374)
\curveto(439.42052655,675.66889095)(439.41552655,675.70889091)(439.40552712,675.74889374)
\curveto(439.3655266,675.86889075)(439.34052663,675.99889062)(439.33052712,676.13889374)
\lineto(439.30052712,676.55889374)
\curveto(439.30052667,676.60889001)(439.29552667,676.66388996)(439.28552712,676.72389374)
\curveto(439.28552668,676.78388984)(439.29052668,676.83888978)(439.30052712,676.88889374)
\lineto(439.30052712,677.06889374)
\lineto(439.34552712,677.42889374)
\curveto(439.38552658,677.59888902)(439.42052655,677.76388886)(439.45052712,677.92389374)
\curveto(439.48052649,678.08388854)(439.52552644,678.23388839)(439.58552712,678.37389374)
\curveto(440.01552595,679.41388721)(440.74552522,680.14888647)(441.77552712,680.57889374)
\curveto(441.91552405,680.63888598)(442.05552391,680.67888594)(442.19552712,680.69889374)
\curveto(442.34552362,680.72888589)(442.50052347,680.76388586)(442.66052712,680.80389374)
\curveto(442.74052323,680.81388581)(442.81552315,680.8188858)(442.88552712,680.81889374)
\curveto(442.95552301,680.8188858)(443.03052294,680.8238858)(443.11052712,680.83389374)
\curveto(443.62052235,680.84388578)(444.05552191,680.78388584)(444.41552712,680.65389374)
\curveto(444.78552118,680.53388609)(445.11552085,680.37388625)(445.40552712,680.17389374)
\curveto(445.49552047,680.11388651)(445.58552038,680.04388658)(445.67552712,679.96389374)
\curveto(445.7655202,679.89388673)(445.84552012,679.8188868)(445.91552712,679.73889374)
\curveto(445.94552002,679.68888693)(445.98551998,679.64888697)(446.03552712,679.61889374)
\curveto(446.11551985,679.50888711)(446.19051978,679.39388723)(446.26052712,679.27389374)
\curveto(446.33051964,679.16388746)(446.40551956,679.04888757)(446.48552712,678.92889374)
\curveto(446.53551943,678.83888778)(446.57551939,678.74388788)(446.60552712,678.64389374)
\curveto(446.64551932,678.55388807)(446.68551928,678.45388817)(446.72552712,678.34389374)
\curveto(446.77551919,678.21388841)(446.81551915,678.07888854)(446.84552712,677.93889374)
\curveto(446.87551909,677.79888882)(446.91051906,677.65888896)(446.95052712,677.51889374)
\curveto(446.970519,677.43888918)(446.97551899,677.34888927)(446.96552712,677.24889374)
\curveto(446.965519,677.15888946)(446.97551899,677.07388955)(446.99552712,676.99389374)
\lineto(446.99552712,676.82889374)
\moveto(444.74552712,677.71389374)
\curveto(444.81552115,677.81388881)(444.82052115,677.93388869)(444.76052712,678.07389374)
\curveto(444.71052126,678.2238884)(444.6705213,678.33388829)(444.64052712,678.40389374)
\curveto(444.50052147,678.67388795)(444.31552165,678.87888774)(444.08552712,679.01889374)
\curveto(443.85552211,679.16888745)(443.53552243,679.24888737)(443.12552712,679.25889374)
\curveto(443.09552287,679.23888738)(443.06052291,679.23388739)(443.02052712,679.24389374)
\curveto(442.98052299,679.25388737)(442.94552302,679.25388737)(442.91552712,679.24389374)
\curveto(442.8655231,679.2238874)(442.81052316,679.20888741)(442.75052712,679.19889374)
\curveto(442.69052328,679.19888742)(442.63552333,679.18888743)(442.58552712,679.16889374)
\curveto(442.14552382,679.02888759)(441.82052415,678.75388787)(441.61052712,678.34389374)
\curveto(441.59052438,678.30388832)(441.5655244,678.24888837)(441.53552712,678.17889374)
\curveto(441.51552445,678.1188885)(441.50052447,678.05388857)(441.49052712,677.98389374)
\curveto(441.48052449,677.9238887)(441.48052449,677.86388876)(441.49052712,677.80389374)
\curveto(441.51052446,677.74388888)(441.54552442,677.69388893)(441.59552712,677.65389374)
\curveto(441.67552429,677.60388902)(441.78552418,677.57888904)(441.92552712,677.57889374)
\lineto(442.33052712,677.57889374)
\lineto(443.99552712,677.57889374)
\lineto(444.43052712,677.57889374)
\curveto(444.59052138,677.58888903)(444.69552127,677.63388899)(444.74552712,677.71389374)
}
}
{
\newrgbcolor{curcolor}{0 0 0}
\pscustom[linestyle=none,fillstyle=solid,fillcolor=curcolor]
{
}
}
{
\newrgbcolor{curcolor}{0 0 0}
\pscustom[linestyle=none,fillstyle=solid,fillcolor=curcolor]
{
\newpath
\moveto(459.99396462,673.73889374)
\lineto(459.99396462,673.31889374)
\curveto(459.99395625,673.18889343)(459.96395628,673.08389354)(459.90396462,673.00389374)
\curveto(459.85395639,672.95389367)(459.78895645,672.9188937)(459.70896462,672.89889374)
\curveto(459.62895661,672.88889373)(459.5389567,672.88389374)(459.43896462,672.88389374)
\lineto(458.61396462,672.88389374)
\lineto(458.32896462,672.88389374)
\curveto(458.24895799,672.89389373)(458.18395806,672.9188937)(458.13396462,672.95889374)
\curveto(458.06395818,673.00889361)(458.02395822,673.07389355)(458.01396462,673.15389374)
\curveto(458.00395824,673.23389339)(457.98395826,673.31389331)(457.95396462,673.39389374)
\curveto(457.93395831,673.41389321)(457.91395833,673.42889319)(457.89396462,673.43889374)
\curveto(457.88395836,673.45889316)(457.86895837,673.47889314)(457.84896462,673.49889374)
\curveto(457.7389585,673.49889312)(457.65895858,673.47389315)(457.60896462,673.42389374)
\lineto(457.45896462,673.27389374)
\curveto(457.38895885,673.2238934)(457.32395892,673.17889344)(457.26396462,673.13889374)
\curveto(457.20395904,673.10889351)(457.1389591,673.06889355)(457.06896462,673.01889374)
\curveto(457.02895921,672.99889362)(456.98395926,672.97889364)(456.93396462,672.95889374)
\curveto(456.89395935,672.93889368)(456.84895939,672.9188937)(456.79896462,672.89889374)
\curveto(456.65895958,672.84889377)(456.50895973,672.80389382)(456.34896462,672.76389374)
\curveto(456.29895994,672.74389388)(456.25395999,672.73389389)(456.21396462,672.73389374)
\curveto(456.17396007,672.73389389)(456.13396011,672.72889389)(456.09396462,672.71889374)
\lineto(455.95896462,672.71889374)
\curveto(455.92896031,672.70889391)(455.88896035,672.70389392)(455.83896462,672.70389374)
\lineto(455.70396462,672.70389374)
\curveto(455.6439606,672.68389394)(455.55396069,672.67889394)(455.43396462,672.68889374)
\curveto(455.31396093,672.68889393)(455.22896101,672.69889392)(455.17896462,672.71889374)
\curveto(455.10896113,672.73889388)(455.0439612,672.74889387)(454.98396462,672.74889374)
\curveto(454.93396131,672.73889388)(454.87896136,672.74389388)(454.81896462,672.76389374)
\lineto(454.45896462,672.88389374)
\curveto(454.34896189,672.91389371)(454.238962,672.95389367)(454.12896462,673.00389374)
\curveto(453.77896246,673.15389347)(453.46396278,673.38389324)(453.18396462,673.69389374)
\curveto(452.91396333,674.01389261)(452.69896354,674.34889227)(452.53896462,674.69889374)
\curveto(452.48896375,674.80889181)(452.44896379,674.91389171)(452.41896462,675.01389374)
\curveto(452.38896385,675.1238915)(452.35396389,675.23389139)(452.31396462,675.34389374)
\curveto(452.30396394,675.38389124)(452.29896394,675.4188912)(452.29896462,675.44889374)
\curveto(452.29896394,675.48889113)(452.28896395,675.53389109)(452.26896462,675.58389374)
\curveto(452.24896399,675.66389096)(452.22896401,675.74889087)(452.20896462,675.83889374)
\curveto(452.19896404,675.93889068)(452.18396406,676.03889058)(452.16396462,676.13889374)
\curveto(452.15396409,676.16889045)(452.14896409,676.20389042)(452.14896462,676.24389374)
\curveto(452.15896408,676.28389034)(452.15896408,676.3188903)(452.14896462,676.34889374)
\lineto(452.14896462,676.48389374)
\curveto(452.14896409,676.53389009)(452.1439641,676.58389004)(452.13396462,676.63389374)
\curveto(452.12396412,676.68388994)(452.11896412,676.73888988)(452.11896462,676.79889374)
\curveto(452.11896412,676.86888975)(452.12396412,676.9238897)(452.13396462,676.96389374)
\curveto(452.1439641,677.01388961)(452.14896409,677.05888956)(452.14896462,677.09889374)
\lineto(452.14896462,677.24889374)
\curveto(452.15896408,677.29888932)(452.15896408,677.34388928)(452.14896462,677.38389374)
\curveto(452.14896409,677.43388919)(452.15896408,677.48388914)(452.17896462,677.53389374)
\curveto(452.19896404,677.64388898)(452.21396403,677.74888887)(452.22396462,677.84889374)
\curveto(452.243964,677.94888867)(452.26896397,678.04888857)(452.29896462,678.14889374)
\curveto(452.3389639,678.26888835)(452.37396387,678.38388824)(452.40396462,678.49389374)
\curveto(452.43396381,678.60388802)(452.47396377,678.71388791)(452.52396462,678.82389374)
\curveto(452.66396358,679.1238875)(452.8389634,679.40888721)(453.04896462,679.67889374)
\curveto(453.06896317,679.70888691)(453.09396315,679.73388689)(453.12396462,679.75389374)
\curveto(453.16396308,679.78388684)(453.19396305,679.81388681)(453.21396462,679.84389374)
\curveto(453.25396299,679.89388673)(453.29396295,679.93888668)(453.33396462,679.97889374)
\curveto(453.37396287,680.0188866)(453.41896282,680.05888656)(453.46896462,680.09889374)
\curveto(453.50896273,680.1188865)(453.5439627,680.14388648)(453.57396462,680.17389374)
\curveto(453.60396264,680.21388641)(453.6389626,680.24388638)(453.67896462,680.26389374)
\curveto(453.92896231,680.43388619)(454.21896202,680.57388605)(454.54896462,680.68389374)
\curveto(454.61896162,680.70388592)(454.68896155,680.7188859)(454.75896462,680.72889374)
\curveto(454.8389614,680.73888588)(454.91896132,680.75388587)(454.99896462,680.77389374)
\curveto(455.06896117,680.79388583)(455.15896108,680.80388582)(455.26896462,680.80389374)
\curveto(455.37896086,680.81388581)(455.48896075,680.8188858)(455.59896462,680.81889374)
\curveto(455.70896053,680.8188858)(455.81396043,680.81388581)(455.91396462,680.80389374)
\curveto(456.02396022,680.79388583)(456.11396013,680.77888584)(456.18396462,680.75889374)
\curveto(456.33395991,680.70888591)(456.47895976,680.66388596)(456.61896462,680.62389374)
\curveto(456.75895948,680.58388604)(456.88895935,680.52888609)(457.00896462,680.45889374)
\curveto(457.07895916,680.40888621)(457.1439591,680.35888626)(457.20396462,680.30889374)
\curveto(457.26395898,680.26888635)(457.32895891,680.2238864)(457.39896462,680.17389374)
\curveto(457.4389588,680.14388648)(457.49395875,680.10388652)(457.56396462,680.05389374)
\curveto(457.6439586,680.00388662)(457.71895852,680.00388662)(457.78896462,680.05389374)
\curveto(457.82895841,680.07388655)(457.84895839,680.10888651)(457.84896462,680.15889374)
\curveto(457.84895839,680.20888641)(457.85895838,680.25888636)(457.87896462,680.30889374)
\lineto(457.87896462,680.45889374)
\curveto(457.88895835,680.48888613)(457.89395835,680.5238861)(457.89396462,680.56389374)
\lineto(457.89396462,680.68389374)
\lineto(457.89396462,682.72389374)
\curveto(457.89395835,682.83388379)(457.88895835,682.95388367)(457.87896462,683.08389374)
\curveto(457.87895836,683.2238834)(457.90395834,683.32888329)(457.95396462,683.39889374)
\curveto(457.99395825,683.47888314)(458.06895817,683.52888309)(458.17896462,683.54889374)
\curveto(458.19895804,683.55888306)(458.21895802,683.55888306)(458.23896462,683.54889374)
\curveto(458.25895798,683.54888307)(458.27895796,683.55388307)(458.29896462,683.56389374)
\lineto(459.36396462,683.56389374)
\curveto(459.48395676,683.56388306)(459.59395665,683.55888306)(459.69396462,683.54889374)
\curveto(459.79395645,683.53888308)(459.86895637,683.49888312)(459.91896462,683.42889374)
\curveto(459.96895627,683.34888327)(459.99395625,683.24388338)(459.99396462,683.11389374)
\lineto(459.99396462,682.75389374)
\lineto(459.99396462,673.73889374)
\moveto(457.95396462,676.67889374)
\curveto(457.96395828,676.7188899)(457.96395828,676.75888986)(457.95396462,676.79889374)
\lineto(457.95396462,676.93389374)
\curveto(457.95395829,677.03388959)(457.94895829,677.13388949)(457.93896462,677.23389374)
\curveto(457.92895831,677.33388929)(457.91395833,677.4238892)(457.89396462,677.50389374)
\curveto(457.87395837,677.61388901)(457.85395839,677.71388891)(457.83396462,677.80389374)
\curveto(457.82395842,677.89388873)(457.79895844,677.97888864)(457.75896462,678.05889374)
\curveto(457.61895862,678.4188882)(457.41395883,678.70388792)(457.14396462,678.91389374)
\curveto(456.88395936,679.1238875)(456.50395974,679.22888739)(456.00396462,679.22889374)
\curveto(455.9439603,679.22888739)(455.86396038,679.2188874)(455.76396462,679.19889374)
\curveto(455.68396056,679.17888744)(455.60896063,679.15888746)(455.53896462,679.13889374)
\curveto(455.47896076,679.12888749)(455.41896082,679.10888751)(455.35896462,679.07889374)
\curveto(455.08896115,678.96888765)(454.87896136,678.79888782)(454.72896462,678.56889374)
\curveto(454.57896166,678.33888828)(454.45896178,678.07888854)(454.36896462,677.78889374)
\curveto(454.3389619,677.68888893)(454.31896192,677.58888903)(454.30896462,677.48889374)
\curveto(454.29896194,677.38888923)(454.27896196,677.28388934)(454.24896462,677.17389374)
\lineto(454.24896462,676.96389374)
\curveto(454.22896201,676.87388975)(454.22396202,676.74888987)(454.23396462,676.58889374)
\curveto(454.243962,676.43889018)(454.25896198,676.32889029)(454.27896462,676.25889374)
\lineto(454.27896462,676.16889374)
\curveto(454.28896195,676.14889047)(454.29396195,676.12889049)(454.29396462,676.10889374)
\curveto(454.31396193,676.02889059)(454.32896191,675.95389067)(454.33896462,675.88389374)
\curveto(454.35896188,675.81389081)(454.37896186,675.73889088)(454.39896462,675.65889374)
\curveto(454.56896167,675.13889148)(454.85896138,674.75389187)(455.26896462,674.50389374)
\curveto(455.39896084,674.41389221)(455.57896066,674.34389228)(455.80896462,674.29389374)
\curveto(455.84896039,674.28389234)(455.90896033,674.27889234)(455.98896462,674.27889374)
\curveto(456.01896022,674.26889235)(456.06396018,674.25889236)(456.12396462,674.24889374)
\curveto(456.19396005,674.24889237)(456.24895999,674.25389237)(456.28896462,674.26389374)
\curveto(456.36895987,674.28389234)(456.44895979,674.29889232)(456.52896462,674.30889374)
\curveto(456.60895963,674.3188923)(456.68895955,674.33889228)(456.76896462,674.36889374)
\curveto(457.01895922,674.47889214)(457.21895902,674.618892)(457.36896462,674.78889374)
\curveto(457.51895872,674.95889166)(457.64895859,675.17389145)(457.75896462,675.43389374)
\curveto(457.79895844,675.5238911)(457.82895841,675.61389101)(457.84896462,675.70389374)
\curveto(457.86895837,675.80389082)(457.88895835,675.90889071)(457.90896462,676.01889374)
\curveto(457.91895832,676.06889055)(457.91895832,676.11389051)(457.90896462,676.15389374)
\curveto(457.90895833,676.20389042)(457.91895832,676.25389037)(457.93896462,676.30389374)
\curveto(457.94895829,676.33389029)(457.95395829,676.36889025)(457.95396462,676.40889374)
\lineto(457.95396462,676.54389374)
\lineto(457.95396462,676.67889374)
}
}
{
\newrgbcolor{curcolor}{0 0 0}
\pscustom[linestyle=none,fillstyle=solid,fillcolor=curcolor]
{
\newpath
\moveto(468.93888649,676.82889374)
\curveto(468.95887833,676.74888987)(468.95887833,676.65888996)(468.93888649,676.55889374)
\curveto(468.91887837,676.45889016)(468.8838784,676.39389023)(468.83388649,676.36389374)
\curveto(468.7838785,676.3238903)(468.70887858,676.29389033)(468.60888649,676.27389374)
\curveto(468.51887877,676.26389036)(468.41387887,676.25389037)(468.29388649,676.24389374)
\lineto(467.94888649,676.24389374)
\curveto(467.83887945,676.25389037)(467.73887955,676.25889036)(467.64888649,676.25889374)
\lineto(463.98888649,676.25889374)
\lineto(463.77888649,676.25889374)
\curveto(463.71888357,676.25889036)(463.66388362,676.24889037)(463.61388649,676.22889374)
\curveto(463.53388375,676.18889043)(463.4838838,676.14889047)(463.46388649,676.10889374)
\curveto(463.44388384,676.08889053)(463.42388386,676.04889057)(463.40388649,675.98889374)
\curveto(463.3838839,675.93889068)(463.37888391,675.88889073)(463.38888649,675.83889374)
\curveto(463.40888388,675.77889084)(463.41888387,675.7188909)(463.41888649,675.65889374)
\curveto(463.42888386,675.60889101)(463.44388384,675.55389107)(463.46388649,675.49389374)
\curveto(463.54388374,675.25389137)(463.63888365,675.05389157)(463.74888649,674.89389374)
\curveto(463.86888342,674.74389188)(464.02888326,674.60889201)(464.22888649,674.48889374)
\curveto(464.30888298,674.43889218)(464.3888829,674.40389222)(464.46888649,674.38389374)
\curveto(464.55888273,674.37389225)(464.64888264,674.35389227)(464.73888649,674.32389374)
\curveto(464.81888247,674.30389232)(464.92888236,674.28889233)(465.06888649,674.27889374)
\curveto(465.20888208,674.26889235)(465.32888196,674.27389235)(465.42888649,674.29389374)
\lineto(465.56388649,674.29389374)
\curveto(465.66388162,674.31389231)(465.75388153,674.33389229)(465.83388649,674.35389374)
\curveto(465.92388136,674.38389224)(466.00888128,674.41389221)(466.08888649,674.44389374)
\curveto(466.1888811,674.49389213)(466.29888099,674.55889206)(466.41888649,674.63889374)
\curveto(466.54888074,674.7188919)(466.64388064,674.79889182)(466.70388649,674.87889374)
\curveto(466.75388053,674.94889167)(466.80388048,675.01389161)(466.85388649,675.07389374)
\curveto(466.91388037,675.14389148)(466.9838803,675.19389143)(467.06388649,675.22389374)
\curveto(467.16388012,675.27389135)(467.28888,675.29389133)(467.43888649,675.28389374)
\lineto(467.87388649,675.28389374)
\lineto(468.05388649,675.28389374)
\curveto(468.12387916,675.29389133)(468.1838791,675.28889133)(468.23388649,675.26889374)
\lineto(468.38388649,675.26889374)
\curveto(468.4838788,675.24889137)(468.55387873,675.2238914)(468.59388649,675.19389374)
\curveto(468.63387865,675.17389145)(468.65387863,675.12889149)(468.65388649,675.05889374)
\curveto(468.66387862,674.98889163)(468.65887863,674.92889169)(468.63888649,674.87889374)
\curveto(468.5888787,674.73889188)(468.53387875,674.61389201)(468.47388649,674.50389374)
\curveto(468.41387887,674.39389223)(468.34387894,674.28389234)(468.26388649,674.17389374)
\curveto(468.04387924,673.84389278)(467.79387949,673.57889304)(467.51388649,673.37889374)
\curveto(467.23388005,673.17889344)(466.8838804,673.00889361)(466.46388649,672.86889374)
\curveto(466.35388093,672.82889379)(466.24388104,672.80389382)(466.13388649,672.79389374)
\curveto(466.02388126,672.78389384)(465.90888138,672.76389386)(465.78888649,672.73389374)
\curveto(465.74888154,672.7238939)(465.70388158,672.7238939)(465.65388649,672.73389374)
\curveto(465.61388167,672.73389389)(465.57388171,672.72889389)(465.53388649,672.71889374)
\lineto(465.36888649,672.71889374)
\curveto(465.31888197,672.69889392)(465.25888203,672.69389393)(465.18888649,672.70389374)
\curveto(465.12888216,672.70389392)(465.07388221,672.70889391)(465.02388649,672.71889374)
\curveto(464.94388234,672.72889389)(464.87388241,672.72889389)(464.81388649,672.71889374)
\curveto(464.75388253,672.70889391)(464.6888826,672.71389391)(464.61888649,672.73389374)
\curveto(464.56888272,672.75389387)(464.51388277,672.76389386)(464.45388649,672.76389374)
\curveto(464.39388289,672.76389386)(464.33888295,672.77389385)(464.28888649,672.79389374)
\curveto(464.17888311,672.81389381)(464.06888322,672.83889378)(463.95888649,672.86889374)
\curveto(463.84888344,672.88889373)(463.74888354,672.9238937)(463.65888649,672.97389374)
\curveto(463.54888374,673.01389361)(463.44388384,673.04889357)(463.34388649,673.07889374)
\curveto(463.25388403,673.1188935)(463.16888412,673.16389346)(463.08888649,673.21389374)
\curveto(462.76888452,673.41389321)(462.4838848,673.64389298)(462.23388649,673.90389374)
\curveto(461.9838853,674.17389245)(461.77888551,674.48389214)(461.61888649,674.83389374)
\curveto(461.56888572,674.94389168)(461.52888576,675.05389157)(461.49888649,675.16389374)
\curveto(461.46888582,675.28389134)(461.42888586,675.40389122)(461.37888649,675.52389374)
\curveto(461.36888592,675.56389106)(461.36388592,675.59889102)(461.36388649,675.62889374)
\curveto(461.36388592,675.66889095)(461.35888593,675.70889091)(461.34888649,675.74889374)
\curveto(461.30888598,675.86889075)(461.283886,675.99889062)(461.27388649,676.13889374)
\lineto(461.24388649,676.55889374)
\curveto(461.24388604,676.60889001)(461.23888605,676.66388996)(461.22888649,676.72389374)
\curveto(461.22888606,676.78388984)(461.23388605,676.83888978)(461.24388649,676.88889374)
\lineto(461.24388649,677.06889374)
\lineto(461.28888649,677.42889374)
\curveto(461.32888596,677.59888902)(461.36388592,677.76388886)(461.39388649,677.92389374)
\curveto(461.42388586,678.08388854)(461.46888582,678.23388839)(461.52888649,678.37389374)
\curveto(461.95888533,679.41388721)(462.6888846,680.14888647)(463.71888649,680.57889374)
\curveto(463.85888343,680.63888598)(463.99888329,680.67888594)(464.13888649,680.69889374)
\curveto(464.288883,680.72888589)(464.44388284,680.76388586)(464.60388649,680.80389374)
\curveto(464.6838826,680.81388581)(464.75888253,680.8188858)(464.82888649,680.81889374)
\curveto(464.89888239,680.8188858)(464.97388231,680.8238858)(465.05388649,680.83389374)
\curveto(465.56388172,680.84388578)(465.99888129,680.78388584)(466.35888649,680.65389374)
\curveto(466.72888056,680.53388609)(467.05888023,680.37388625)(467.34888649,680.17389374)
\curveto(467.43887985,680.11388651)(467.52887976,680.04388658)(467.61888649,679.96389374)
\curveto(467.70887958,679.89388673)(467.7888795,679.8188868)(467.85888649,679.73889374)
\curveto(467.8888794,679.68888693)(467.92887936,679.64888697)(467.97888649,679.61889374)
\curveto(468.05887923,679.50888711)(468.13387915,679.39388723)(468.20388649,679.27389374)
\curveto(468.27387901,679.16388746)(468.34887894,679.04888757)(468.42888649,678.92889374)
\curveto(468.47887881,678.83888778)(468.51887877,678.74388788)(468.54888649,678.64389374)
\curveto(468.5888787,678.55388807)(468.62887866,678.45388817)(468.66888649,678.34389374)
\curveto(468.71887857,678.21388841)(468.75887853,678.07888854)(468.78888649,677.93889374)
\curveto(468.81887847,677.79888882)(468.85387843,677.65888896)(468.89388649,677.51889374)
\curveto(468.91387837,677.43888918)(468.91887837,677.34888927)(468.90888649,677.24889374)
\curveto(468.90887838,677.15888946)(468.91887837,677.07388955)(468.93888649,676.99389374)
\lineto(468.93888649,676.82889374)
\moveto(466.68888649,677.71389374)
\curveto(466.75888053,677.81388881)(466.76388052,677.93388869)(466.70388649,678.07389374)
\curveto(466.65388063,678.2238884)(466.61388067,678.33388829)(466.58388649,678.40389374)
\curveto(466.44388084,678.67388795)(466.25888103,678.87888774)(466.02888649,679.01889374)
\curveto(465.79888149,679.16888745)(465.47888181,679.24888737)(465.06888649,679.25889374)
\curveto(465.03888225,679.23888738)(465.00388228,679.23388739)(464.96388649,679.24389374)
\curveto(464.92388236,679.25388737)(464.8888824,679.25388737)(464.85888649,679.24389374)
\curveto(464.80888248,679.2238874)(464.75388253,679.20888741)(464.69388649,679.19889374)
\curveto(464.63388265,679.19888742)(464.57888271,679.18888743)(464.52888649,679.16889374)
\curveto(464.0888832,679.02888759)(463.76388352,678.75388787)(463.55388649,678.34389374)
\curveto(463.53388375,678.30388832)(463.50888378,678.24888837)(463.47888649,678.17889374)
\curveto(463.45888383,678.1188885)(463.44388384,678.05388857)(463.43388649,677.98389374)
\curveto(463.42388386,677.9238887)(463.42388386,677.86388876)(463.43388649,677.80389374)
\curveto(463.45388383,677.74388888)(463.4888838,677.69388893)(463.53888649,677.65389374)
\curveto(463.61888367,677.60388902)(463.72888356,677.57888904)(463.86888649,677.57889374)
\lineto(464.27388649,677.57889374)
\lineto(465.93888649,677.57889374)
\lineto(466.37388649,677.57889374)
\curveto(466.53388075,677.58888903)(466.63888065,677.63388899)(466.68888649,677.71389374)
}
}
{
\newrgbcolor{curcolor}{0 0 0}
\pscustom[linestyle=none,fillstyle=solid,fillcolor=curcolor]
{
}
}
{
\newrgbcolor{curcolor}{0 0 0}
\pscustom[linestyle=none,fillstyle=solid,fillcolor=curcolor]
{
\newpath
\moveto(474.76732399,680.60889374)
\lineto(475.89232399,680.60889374)
\curveto(476.00232156,680.60888601)(476.10232146,680.60388602)(476.19232399,680.59389374)
\curveto(476.28232128,680.58388604)(476.34732121,680.54888607)(476.38732399,680.48889374)
\curveto(476.43732112,680.42888619)(476.46732109,680.34388628)(476.47732399,680.23389374)
\curveto(476.48732107,680.13388649)(476.49232107,680.02888659)(476.49232399,679.91889374)
\lineto(476.49232399,678.86889374)
\lineto(476.49232399,676.63389374)
\curveto(476.49232107,676.27389035)(476.50732105,675.93389069)(476.53732399,675.61389374)
\curveto(476.56732099,675.29389133)(476.6573209,675.02889159)(476.80732399,674.81889374)
\curveto(476.94732061,674.60889201)(477.17232039,674.45889216)(477.48232399,674.36889374)
\curveto(477.53232003,674.35889226)(477.57231999,674.35389227)(477.60232399,674.35389374)
\curveto(477.64231992,674.35389227)(477.68731987,674.34889227)(477.73732399,674.33889374)
\curveto(477.78731977,674.32889229)(477.84231972,674.3238923)(477.90232399,674.32389374)
\curveto(477.9623196,674.3238923)(478.00731955,674.32889229)(478.03732399,674.33889374)
\curveto(478.08731947,674.35889226)(478.12731943,674.36389226)(478.15732399,674.35389374)
\curveto(478.19731936,674.34389228)(478.23731932,674.34889227)(478.27732399,674.36889374)
\curveto(478.48731907,674.4188922)(478.65231891,674.48389214)(478.77232399,674.56389374)
\curveto(478.95231861,674.67389195)(479.09231847,674.81389181)(479.19232399,674.98389374)
\curveto(479.30231826,675.16389146)(479.37731818,675.35889126)(479.41732399,675.56889374)
\curveto(479.46731809,675.78889083)(479.49731806,676.02889059)(479.50732399,676.28889374)
\curveto(479.51731804,676.55889006)(479.52231804,676.83888978)(479.52232399,677.12889374)
\lineto(479.52232399,678.94389374)
\lineto(479.52232399,679.91889374)
\lineto(479.52232399,680.18889374)
\curveto(479.52231804,680.28888633)(479.54231802,680.36888625)(479.58232399,680.42889374)
\curveto(479.63231793,680.5188861)(479.70731785,680.56888605)(479.80732399,680.57889374)
\curveto(479.90731765,680.59888602)(480.02731753,680.60888601)(480.16732399,680.60889374)
\lineto(480.96232399,680.60889374)
\lineto(481.24732399,680.60889374)
\curveto(481.33731622,680.60888601)(481.41231615,680.58888603)(481.47232399,680.54889374)
\curveto(481.55231601,680.49888612)(481.59731596,680.4238862)(481.60732399,680.32389374)
\curveto(481.61731594,680.2238864)(481.62231594,680.10888651)(481.62232399,679.97889374)
\lineto(481.62232399,678.83889374)
\lineto(481.62232399,674.62389374)
\lineto(481.62232399,673.55889374)
\lineto(481.62232399,673.25889374)
\curveto(481.62231594,673.15889346)(481.60231596,673.08389354)(481.56232399,673.03389374)
\curveto(481.51231605,672.95389367)(481.43731612,672.90889371)(481.33732399,672.89889374)
\curveto(481.23731632,672.88889373)(481.13231643,672.88389374)(481.02232399,672.88389374)
\lineto(480.21232399,672.88389374)
\curveto(480.10231746,672.88389374)(480.00231756,672.88889373)(479.91232399,672.89889374)
\curveto(479.83231773,672.90889371)(479.76731779,672.94889367)(479.71732399,673.01889374)
\curveto(479.69731786,673.04889357)(479.67731788,673.09389353)(479.65732399,673.15389374)
\curveto(479.64731791,673.21389341)(479.63231793,673.27389335)(479.61232399,673.33389374)
\curveto(479.60231796,673.39389323)(479.58731797,673.44889317)(479.56732399,673.49889374)
\curveto(479.54731801,673.54889307)(479.51731804,673.57889304)(479.47732399,673.58889374)
\curveto(479.4573181,673.60889301)(479.43231813,673.61389301)(479.40232399,673.60389374)
\curveto(479.37231819,673.59389303)(479.34731821,673.58389304)(479.32732399,673.57389374)
\curveto(479.2573183,673.53389309)(479.19731836,673.48889313)(479.14732399,673.43889374)
\curveto(479.09731846,673.38889323)(479.04231852,673.34389328)(478.98232399,673.30389374)
\curveto(478.94231862,673.27389335)(478.90231866,673.23889338)(478.86232399,673.19889374)
\curveto(478.83231873,673.16889345)(478.79231877,673.13889348)(478.74232399,673.10889374)
\curveto(478.51231905,672.96889365)(478.24231932,672.85889376)(477.93232399,672.77889374)
\curveto(477.8623197,672.75889386)(477.79231977,672.74889387)(477.72232399,672.74889374)
\curveto(477.65231991,672.73889388)(477.57731998,672.7238939)(477.49732399,672.70389374)
\curveto(477.4573201,672.69389393)(477.41232015,672.69389393)(477.36232399,672.70389374)
\curveto(477.32232024,672.70389392)(477.28232028,672.69889392)(477.24232399,672.68889374)
\curveto(477.21232035,672.67889394)(477.14732041,672.67889394)(477.04732399,672.68889374)
\curveto(476.9573206,672.68889393)(476.89732066,672.69389393)(476.86732399,672.70389374)
\curveto(476.81732074,672.70389392)(476.76732079,672.70889391)(476.71732399,672.71889374)
\lineto(476.56732399,672.71889374)
\curveto(476.44732111,672.74889387)(476.33232123,672.77389385)(476.22232399,672.79389374)
\curveto(476.11232145,672.81389381)(476.00232156,672.84389378)(475.89232399,672.88389374)
\curveto(475.84232172,672.90389372)(475.79732176,672.9188937)(475.75732399,672.92889374)
\curveto(475.72732183,672.94889367)(475.68732187,672.96889365)(475.63732399,672.98889374)
\curveto(475.28732227,673.17889344)(475.00732255,673.44389318)(474.79732399,673.78389374)
\curveto(474.66732289,673.99389263)(474.57232299,674.24389238)(474.51232399,674.53389374)
\curveto(474.45232311,674.83389179)(474.41232315,675.14889147)(474.39232399,675.47889374)
\curveto(474.38232318,675.8188908)(474.37732318,676.16389046)(474.37732399,676.51389374)
\curveto(474.38732317,676.87388975)(474.39232317,677.22888939)(474.39232399,677.57889374)
\lineto(474.39232399,679.61889374)
\curveto(474.39232317,679.74888687)(474.38732317,679.89888672)(474.37732399,680.06889374)
\curveto(474.37732318,680.24888637)(474.40232316,680.37888624)(474.45232399,680.45889374)
\curveto(474.48232308,680.50888611)(474.54232302,680.55388607)(474.63232399,680.59389374)
\curveto(474.69232287,680.59388603)(474.73732282,680.59888602)(474.76732399,680.60889374)
}
}
{
\newrgbcolor{curcolor}{0 0 0}
\pscustom[linestyle=none,fillstyle=solid,fillcolor=curcolor]
{
\newpath
\moveto(486.22357399,680.83389374)
\curveto(486.97356949,680.85388577)(487.62356884,680.76888585)(488.17357399,680.57889374)
\curveto(488.73356773,680.39888622)(489.15856731,680.08388654)(489.44857399,679.63389374)
\curveto(489.51856695,679.5238871)(489.57856689,679.40888721)(489.62857399,679.28889374)
\curveto(489.68856678,679.17888744)(489.73856673,679.05388757)(489.77857399,678.91389374)
\curveto(489.79856667,678.85388777)(489.80856666,678.78888783)(489.80857399,678.71889374)
\curveto(489.80856666,678.64888797)(489.79856667,678.58888803)(489.77857399,678.53889374)
\curveto(489.73856673,678.47888814)(489.68356678,678.43888818)(489.61357399,678.41889374)
\curveto(489.5635669,678.39888822)(489.50356696,678.38888823)(489.43357399,678.38889374)
\lineto(489.22357399,678.38889374)
\lineto(488.56357399,678.38889374)
\curveto(488.49356797,678.38888823)(488.42356804,678.38388824)(488.35357399,678.37389374)
\curveto(488.28356818,678.37388825)(488.21856825,678.38388824)(488.15857399,678.40389374)
\curveto(488.05856841,678.4238882)(487.98356848,678.46388816)(487.93357399,678.52389374)
\curveto(487.88356858,678.58388804)(487.83856863,678.64388798)(487.79857399,678.70389374)
\lineto(487.67857399,678.91389374)
\curveto(487.64856882,678.99388763)(487.59856887,679.05888756)(487.52857399,679.10889374)
\curveto(487.42856904,679.18888743)(487.32856914,679.24888737)(487.22857399,679.28889374)
\curveto(487.13856933,679.32888729)(487.02356944,679.36388726)(486.88357399,679.39389374)
\curveto(486.81356965,679.41388721)(486.70856976,679.42888719)(486.56857399,679.43889374)
\curveto(486.43857003,679.44888717)(486.33857013,679.44388718)(486.26857399,679.42389374)
\lineto(486.16357399,679.42389374)
\lineto(486.01357399,679.39389374)
\curveto(485.97357049,679.39388723)(485.92857054,679.38888723)(485.87857399,679.37889374)
\curveto(485.70857076,679.32888729)(485.5685709,679.25888736)(485.45857399,679.16889374)
\curveto(485.35857111,679.08888753)(485.28857118,678.96388766)(485.24857399,678.79389374)
\curveto(485.22857124,678.7238879)(485.22857124,678.65888796)(485.24857399,678.59889374)
\curveto(485.2685712,678.53888808)(485.28857118,678.48888813)(485.30857399,678.44889374)
\curveto(485.37857109,678.32888829)(485.45857101,678.23388839)(485.54857399,678.16389374)
\curveto(485.64857082,678.09388853)(485.7635707,678.03388859)(485.89357399,677.98389374)
\curveto(486.08357038,677.90388872)(486.28857018,677.83388879)(486.50857399,677.77389374)
\lineto(487.19857399,677.62389374)
\curveto(487.43856903,677.58388904)(487.6685688,677.53388909)(487.88857399,677.47389374)
\curveto(488.11856835,677.4238892)(488.33356813,677.35888926)(488.53357399,677.27889374)
\curveto(488.62356784,677.23888938)(488.70856776,677.20388942)(488.78857399,677.17389374)
\curveto(488.87856759,677.15388947)(488.9635675,677.1188895)(489.04357399,677.06889374)
\curveto(489.23356723,676.94888967)(489.40356706,676.8188898)(489.55357399,676.67889374)
\curveto(489.71356675,676.53889008)(489.83856663,676.36389026)(489.92857399,676.15389374)
\curveto(489.95856651,676.08389054)(489.98356648,676.01389061)(490.00357399,675.94389374)
\curveto(490.02356644,675.87389075)(490.04356642,675.79889082)(490.06357399,675.71889374)
\curveto(490.07356639,675.65889096)(490.07856639,675.56389106)(490.07857399,675.43389374)
\curveto(490.08856638,675.31389131)(490.08856638,675.2188914)(490.07857399,675.14889374)
\lineto(490.07857399,675.07389374)
\curveto(490.05856641,675.01389161)(490.04356642,674.95389167)(490.03357399,674.89389374)
\curveto(490.03356643,674.84389178)(490.02856644,674.79389183)(490.01857399,674.74389374)
\curveto(489.94856652,674.44389218)(489.83856663,674.17889244)(489.68857399,673.94889374)
\curveto(489.52856694,673.70889291)(489.33356713,673.51389311)(489.10357399,673.36389374)
\curveto(488.87356759,673.21389341)(488.61356785,673.08389354)(488.32357399,672.97389374)
\curveto(488.21356825,672.9238937)(488.09356837,672.88889373)(487.96357399,672.86889374)
\curveto(487.84356862,672.84889377)(487.72356874,672.8238938)(487.60357399,672.79389374)
\curveto(487.51356895,672.77389385)(487.41856905,672.76389386)(487.31857399,672.76389374)
\curveto(487.22856924,672.75389387)(487.13856933,672.73889388)(487.04857399,672.71889374)
\lineto(486.77857399,672.71889374)
\curveto(486.71856975,672.69889392)(486.61356985,672.68889393)(486.46357399,672.68889374)
\curveto(486.32357014,672.68889393)(486.22357024,672.69889392)(486.16357399,672.71889374)
\curveto(486.13357033,672.7188939)(486.09857037,672.7238939)(486.05857399,672.73389374)
\lineto(485.95357399,672.73389374)
\curveto(485.83357063,672.75389387)(485.71357075,672.76889385)(485.59357399,672.77889374)
\curveto(485.47357099,672.78889383)(485.35857111,672.80889381)(485.24857399,672.83889374)
\curveto(484.85857161,672.94889367)(484.51357195,673.07389355)(484.21357399,673.21389374)
\curveto(483.91357255,673.36389326)(483.65857281,673.58389304)(483.44857399,673.87389374)
\curveto(483.30857316,674.06389256)(483.18857328,674.28389234)(483.08857399,674.53389374)
\curveto(483.0685734,674.59389203)(483.04857342,674.67389195)(483.02857399,674.77389374)
\curveto(483.00857346,674.8238918)(482.99357347,674.89389173)(482.98357399,674.98389374)
\curveto(482.97357349,675.07389155)(482.97857349,675.14889147)(482.99857399,675.20889374)
\curveto(483.02857344,675.27889134)(483.07857339,675.32889129)(483.14857399,675.35889374)
\curveto(483.19857327,675.37889124)(483.25857321,675.38889123)(483.32857399,675.38889374)
\lineto(483.55357399,675.38889374)
\lineto(484.25857399,675.38889374)
\lineto(484.49857399,675.38889374)
\curveto(484.57857189,675.38889123)(484.64857182,675.37889124)(484.70857399,675.35889374)
\curveto(484.81857165,675.3188913)(484.88857158,675.25389137)(484.91857399,675.16389374)
\curveto(484.95857151,675.07389155)(485.00357146,674.97889164)(485.05357399,674.87889374)
\curveto(485.07357139,674.82889179)(485.10857136,674.76389186)(485.15857399,674.68389374)
\curveto(485.21857125,674.60389202)(485.2685712,674.55389207)(485.30857399,674.53389374)
\curveto(485.42857104,674.43389219)(485.54357092,674.35389227)(485.65357399,674.29389374)
\curveto(485.7635707,674.24389238)(485.90357056,674.19389243)(486.07357399,674.14389374)
\curveto(486.12357034,674.1238925)(486.17357029,674.11389251)(486.22357399,674.11389374)
\curveto(486.27357019,674.1238925)(486.32357014,674.1238925)(486.37357399,674.11389374)
\curveto(486.45357001,674.09389253)(486.53856993,674.08389254)(486.62857399,674.08389374)
\curveto(486.72856974,674.09389253)(486.81356965,674.10889251)(486.88357399,674.12889374)
\curveto(486.93356953,674.13889248)(486.97856949,674.14389248)(487.01857399,674.14389374)
\curveto(487.0685694,674.14389248)(487.11856935,674.15389247)(487.16857399,674.17389374)
\curveto(487.30856916,674.2238924)(487.43356903,674.28389234)(487.54357399,674.35389374)
\curveto(487.6635688,674.4238922)(487.75856871,674.51389211)(487.82857399,674.62389374)
\curveto(487.87856859,674.70389192)(487.91856855,674.82889179)(487.94857399,674.99889374)
\curveto(487.9685685,675.06889155)(487.9685685,675.13389149)(487.94857399,675.19389374)
\curveto(487.92856854,675.25389137)(487.90856856,675.30389132)(487.88857399,675.34389374)
\curveto(487.81856865,675.48389114)(487.72856874,675.58889103)(487.61857399,675.65889374)
\curveto(487.51856895,675.72889089)(487.39856907,675.79389083)(487.25857399,675.85389374)
\curveto(487.0685694,675.93389069)(486.8685696,675.99889062)(486.65857399,676.04889374)
\curveto(486.44857002,676.09889052)(486.23857023,676.15389047)(486.02857399,676.21389374)
\curveto(485.94857052,676.23389039)(485.8635706,676.24889037)(485.77357399,676.25889374)
\curveto(485.69357077,676.26889035)(485.61357085,676.28389034)(485.53357399,676.30389374)
\curveto(485.21357125,676.39389023)(484.90857156,676.47889014)(484.61857399,676.55889374)
\curveto(484.32857214,676.64888997)(484.0635724,676.77888984)(483.82357399,676.94889374)
\curveto(483.54357292,677.14888947)(483.33857313,677.4188892)(483.20857399,677.75889374)
\curveto(483.18857328,677.82888879)(483.1685733,677.9238887)(483.14857399,678.04389374)
\curveto(483.12857334,678.11388851)(483.11357335,678.19888842)(483.10357399,678.29889374)
\curveto(483.09357337,678.39888822)(483.09857337,678.48888813)(483.11857399,678.56889374)
\curveto(483.13857333,678.618888)(483.14357332,678.65888796)(483.13357399,678.68889374)
\curveto(483.12357334,678.72888789)(483.12857334,678.77388785)(483.14857399,678.82389374)
\curveto(483.1685733,678.93388769)(483.18857328,679.03388759)(483.20857399,679.12389374)
\curveto(483.23857323,679.2238874)(483.27357319,679.3188873)(483.31357399,679.40889374)
\curveto(483.44357302,679.69888692)(483.62357284,679.93388669)(483.85357399,680.11389374)
\curveto(484.08357238,680.29388633)(484.34357212,680.43888618)(484.63357399,680.54889374)
\curveto(484.74357172,680.59888602)(484.85857161,680.63388599)(484.97857399,680.65389374)
\curveto(485.09857137,680.68388594)(485.22357124,680.71388591)(485.35357399,680.74389374)
\curveto(485.41357105,680.76388586)(485.47357099,680.77388585)(485.53357399,680.77389374)
\lineto(485.71357399,680.80389374)
\curveto(485.79357067,680.81388581)(485.87857059,680.8188858)(485.96857399,680.81889374)
\curveto(486.05857041,680.8188858)(486.14357032,680.8238858)(486.22357399,680.83389374)
}
}
{
\newrgbcolor{curcolor}{0 0 0}
\pscustom[linestyle=none,fillstyle=solid,fillcolor=curcolor]
{
\newpath
\moveto(491.73021462,680.60889374)
\lineto(492.85521462,680.60889374)
\curveto(492.96521218,680.60888601)(493.06521208,680.60388602)(493.15521462,680.59389374)
\curveto(493.2452119,680.58388604)(493.31021184,680.54888607)(493.35021462,680.48889374)
\curveto(493.40021175,680.42888619)(493.43021172,680.34388628)(493.44021462,680.23389374)
\curveto(493.4502117,680.13388649)(493.45521169,680.02888659)(493.45521462,679.91889374)
\lineto(493.45521462,678.86889374)
\lineto(493.45521462,676.63389374)
\curveto(493.45521169,676.27389035)(493.47021168,675.93389069)(493.50021462,675.61389374)
\curveto(493.53021162,675.29389133)(493.62021153,675.02889159)(493.77021462,674.81889374)
\curveto(493.91021124,674.60889201)(494.13521101,674.45889216)(494.44521462,674.36889374)
\curveto(494.49521065,674.35889226)(494.53521061,674.35389227)(494.56521462,674.35389374)
\curveto(494.60521054,674.35389227)(494.6502105,674.34889227)(494.70021462,674.33889374)
\curveto(494.7502104,674.32889229)(494.80521034,674.3238923)(494.86521462,674.32389374)
\curveto(494.92521022,674.3238923)(494.97021018,674.32889229)(495.00021462,674.33889374)
\curveto(495.0502101,674.35889226)(495.09021006,674.36389226)(495.12021462,674.35389374)
\curveto(495.16020999,674.34389228)(495.20020995,674.34889227)(495.24021462,674.36889374)
\curveto(495.4502097,674.4188922)(495.61520953,674.48389214)(495.73521462,674.56389374)
\curveto(495.91520923,674.67389195)(496.05520909,674.81389181)(496.15521462,674.98389374)
\curveto(496.26520888,675.16389146)(496.34020881,675.35889126)(496.38021462,675.56889374)
\curveto(496.43020872,675.78889083)(496.46020869,676.02889059)(496.47021462,676.28889374)
\curveto(496.48020867,676.55889006)(496.48520866,676.83888978)(496.48521462,677.12889374)
\lineto(496.48521462,678.94389374)
\lineto(496.48521462,679.91889374)
\lineto(496.48521462,680.18889374)
\curveto(496.48520866,680.28888633)(496.50520864,680.36888625)(496.54521462,680.42889374)
\curveto(496.59520855,680.5188861)(496.67020848,680.56888605)(496.77021462,680.57889374)
\curveto(496.87020828,680.59888602)(496.99020816,680.60888601)(497.13021462,680.60889374)
\lineto(497.92521462,680.60889374)
\lineto(498.21021462,680.60889374)
\curveto(498.30020685,680.60888601)(498.37520677,680.58888603)(498.43521462,680.54889374)
\curveto(498.51520663,680.49888612)(498.56020659,680.4238862)(498.57021462,680.32389374)
\curveto(498.58020657,680.2238864)(498.58520656,680.10888651)(498.58521462,679.97889374)
\lineto(498.58521462,678.83889374)
\lineto(498.58521462,674.62389374)
\lineto(498.58521462,673.55889374)
\lineto(498.58521462,673.25889374)
\curveto(498.58520656,673.15889346)(498.56520658,673.08389354)(498.52521462,673.03389374)
\curveto(498.47520667,672.95389367)(498.40020675,672.90889371)(498.30021462,672.89889374)
\curveto(498.20020695,672.88889373)(498.09520705,672.88389374)(497.98521462,672.88389374)
\lineto(497.17521462,672.88389374)
\curveto(497.06520808,672.88389374)(496.96520818,672.88889373)(496.87521462,672.89889374)
\curveto(496.79520835,672.90889371)(496.73020842,672.94889367)(496.68021462,673.01889374)
\curveto(496.66020849,673.04889357)(496.64020851,673.09389353)(496.62021462,673.15389374)
\curveto(496.61020854,673.21389341)(496.59520855,673.27389335)(496.57521462,673.33389374)
\curveto(496.56520858,673.39389323)(496.5502086,673.44889317)(496.53021462,673.49889374)
\curveto(496.51020864,673.54889307)(496.48020867,673.57889304)(496.44021462,673.58889374)
\curveto(496.42020873,673.60889301)(496.39520875,673.61389301)(496.36521462,673.60389374)
\curveto(496.33520881,673.59389303)(496.31020884,673.58389304)(496.29021462,673.57389374)
\curveto(496.22020893,673.53389309)(496.16020899,673.48889313)(496.11021462,673.43889374)
\curveto(496.06020909,673.38889323)(496.00520914,673.34389328)(495.94521462,673.30389374)
\curveto(495.90520924,673.27389335)(495.86520928,673.23889338)(495.82521462,673.19889374)
\curveto(495.79520935,673.16889345)(495.75520939,673.13889348)(495.70521462,673.10889374)
\curveto(495.47520967,672.96889365)(495.20520994,672.85889376)(494.89521462,672.77889374)
\curveto(494.82521032,672.75889386)(494.75521039,672.74889387)(494.68521462,672.74889374)
\curveto(494.61521053,672.73889388)(494.54021061,672.7238939)(494.46021462,672.70389374)
\curveto(494.42021073,672.69389393)(494.37521077,672.69389393)(494.32521462,672.70389374)
\curveto(494.28521086,672.70389392)(494.2452109,672.69889392)(494.20521462,672.68889374)
\curveto(494.17521097,672.67889394)(494.11021104,672.67889394)(494.01021462,672.68889374)
\curveto(493.92021123,672.68889393)(493.86021129,672.69389393)(493.83021462,672.70389374)
\curveto(493.78021137,672.70389392)(493.73021142,672.70889391)(493.68021462,672.71889374)
\lineto(493.53021462,672.71889374)
\curveto(493.41021174,672.74889387)(493.29521185,672.77389385)(493.18521462,672.79389374)
\curveto(493.07521207,672.81389381)(492.96521218,672.84389378)(492.85521462,672.88389374)
\curveto(492.80521234,672.90389372)(492.76021239,672.9188937)(492.72021462,672.92889374)
\curveto(492.69021246,672.94889367)(492.6502125,672.96889365)(492.60021462,672.98889374)
\curveto(492.2502129,673.17889344)(491.97021318,673.44389318)(491.76021462,673.78389374)
\curveto(491.63021352,673.99389263)(491.53521361,674.24389238)(491.47521462,674.53389374)
\curveto(491.41521373,674.83389179)(491.37521377,675.14889147)(491.35521462,675.47889374)
\curveto(491.3452138,675.8188908)(491.34021381,676.16389046)(491.34021462,676.51389374)
\curveto(491.3502138,676.87388975)(491.35521379,677.22888939)(491.35521462,677.57889374)
\lineto(491.35521462,679.61889374)
\curveto(491.35521379,679.74888687)(491.3502138,679.89888672)(491.34021462,680.06889374)
\curveto(491.34021381,680.24888637)(491.36521378,680.37888624)(491.41521462,680.45889374)
\curveto(491.4452137,680.50888611)(491.50521364,680.55388607)(491.59521462,680.59389374)
\curveto(491.65521349,680.59388603)(491.70021345,680.59888602)(491.73021462,680.60889374)
}
}
{
\newrgbcolor{curcolor}{0 0 0}
\pscustom[linestyle=none,fillstyle=solid,fillcolor=curcolor]
{
\newpath
\moveto(507.26646462,673.48389374)
\curveto(507.28645677,673.37389325)(507.29645676,673.26389336)(507.29646462,673.15389374)
\curveto(507.30645675,673.04389358)(507.2564568,672.96889365)(507.14646462,672.92889374)
\curveto(507.08645697,672.89889372)(507.01645704,672.88389374)(506.93646462,672.88389374)
\lineto(506.69646462,672.88389374)
\lineto(505.88646462,672.88389374)
\lineto(505.61646462,672.88389374)
\curveto(505.53645852,672.89389373)(505.47145858,672.9188937)(505.42146462,672.95889374)
\curveto(505.3514587,672.99889362)(505.29645876,673.05389357)(505.25646462,673.12389374)
\curveto(505.22645883,673.20389342)(505.18145887,673.26889335)(505.12146462,673.31889374)
\curveto(505.10145895,673.33889328)(505.07645898,673.35389327)(505.04646462,673.36389374)
\curveto(505.01645904,673.38389324)(504.97645908,673.38889323)(504.92646462,673.37889374)
\curveto(504.87645918,673.35889326)(504.82645923,673.33389329)(504.77646462,673.30389374)
\curveto(504.73645932,673.27389335)(504.69145936,673.24889337)(504.64146462,673.22889374)
\curveto(504.59145946,673.18889343)(504.53645952,673.15389347)(504.47646462,673.12389374)
\lineto(504.29646462,673.03389374)
\curveto(504.16645989,672.97389365)(504.03146002,672.9238937)(503.89146462,672.88389374)
\curveto(503.7514603,672.85389377)(503.60646045,672.8188938)(503.45646462,672.77889374)
\curveto(503.38646067,672.75889386)(503.31646074,672.74889387)(503.24646462,672.74889374)
\curveto(503.18646087,672.73889388)(503.12146093,672.72889389)(503.05146462,672.71889374)
\lineto(502.96146462,672.71889374)
\curveto(502.93146112,672.70889391)(502.90146115,672.70389392)(502.87146462,672.70389374)
\lineto(502.70646462,672.70389374)
\curveto(502.60646145,672.68389394)(502.50646155,672.68389394)(502.40646462,672.70389374)
\lineto(502.27146462,672.70389374)
\curveto(502.20146185,672.7238939)(502.13146192,672.73389389)(502.06146462,672.73389374)
\curveto(502.00146205,672.7238939)(501.94146211,672.72889389)(501.88146462,672.74889374)
\curveto(501.78146227,672.76889385)(501.68646237,672.78889383)(501.59646462,672.80889374)
\curveto(501.50646255,672.8188938)(501.42146263,672.84389378)(501.34146462,672.88389374)
\curveto(501.051463,672.99389363)(500.80146325,673.13389349)(500.59146462,673.30389374)
\curveto(500.39146366,673.48389314)(500.23146382,673.7188929)(500.11146462,674.00889374)
\curveto(500.08146397,674.07889254)(500.051464,674.15389247)(500.02146462,674.23389374)
\curveto(500.00146405,674.31389231)(499.98146407,674.39889222)(499.96146462,674.48889374)
\curveto(499.94146411,674.53889208)(499.93146412,674.58889203)(499.93146462,674.63889374)
\curveto(499.94146411,674.68889193)(499.94146411,674.73889188)(499.93146462,674.78889374)
\curveto(499.92146413,674.8188918)(499.91146414,674.87889174)(499.90146462,674.96889374)
\curveto(499.90146415,675.06889155)(499.90646415,675.13889148)(499.91646462,675.17889374)
\curveto(499.93646412,675.27889134)(499.94646411,675.36389126)(499.94646462,675.43389374)
\lineto(500.03646462,675.76389374)
\curveto(500.06646399,675.88389074)(500.10646395,675.98889063)(500.15646462,676.07889374)
\curveto(500.32646373,676.36889025)(500.52146353,676.58889003)(500.74146462,676.73889374)
\curveto(500.96146309,676.88888973)(501.24146281,677.0188896)(501.58146462,677.12889374)
\curveto(501.71146234,677.17888944)(501.84646221,677.21388941)(501.98646462,677.23389374)
\curveto(502.12646193,677.25388937)(502.26646179,677.27888934)(502.40646462,677.30889374)
\curveto(502.48646157,677.32888929)(502.57146148,677.33888928)(502.66146462,677.33889374)
\curveto(502.7514613,677.34888927)(502.84146121,677.36388926)(502.93146462,677.38389374)
\curveto(503.00146105,677.40388922)(503.07146098,677.40888921)(503.14146462,677.39889374)
\curveto(503.21146084,677.39888922)(503.28646077,677.40888921)(503.36646462,677.42889374)
\curveto(503.43646062,677.44888917)(503.50646055,677.45888916)(503.57646462,677.45889374)
\curveto(503.64646041,677.45888916)(503.72146033,677.46888915)(503.80146462,677.48889374)
\curveto(504.01146004,677.53888908)(504.20145985,677.57888904)(504.37146462,677.60889374)
\curveto(504.5514595,677.64888897)(504.71145934,677.73888888)(504.85146462,677.87889374)
\curveto(504.94145911,677.96888865)(505.00145905,678.06888855)(505.03146462,678.17889374)
\curveto(505.04145901,678.20888841)(505.04145901,678.23388839)(505.03146462,678.25389374)
\curveto(505.03145902,678.27388835)(505.03645902,678.29388833)(505.04646462,678.31389374)
\curveto(505.056459,678.33388829)(505.06145899,678.36388826)(505.06146462,678.40389374)
\lineto(505.06146462,678.49389374)
\lineto(505.03146462,678.61389374)
\curveto(505.03145902,678.65388797)(505.02645903,678.68888793)(505.01646462,678.71889374)
\curveto(504.91645914,679.0188876)(504.70645935,679.2238874)(504.38646462,679.33389374)
\curveto(504.29645976,679.36388726)(504.18645987,679.38388724)(504.05646462,679.39389374)
\curveto(503.93646012,679.41388721)(503.81146024,679.4188872)(503.68146462,679.40889374)
\curveto(503.5514605,679.40888721)(503.42646063,679.39888722)(503.30646462,679.37889374)
\curveto(503.18646087,679.35888726)(503.08146097,679.33388729)(502.99146462,679.30389374)
\curveto(502.93146112,679.28388734)(502.87146118,679.25388737)(502.81146462,679.21389374)
\curveto(502.76146129,679.18388744)(502.71146134,679.14888747)(502.66146462,679.10889374)
\curveto(502.61146144,679.06888755)(502.5564615,679.01388761)(502.49646462,678.94389374)
\curveto(502.44646161,678.87388775)(502.41146164,678.80888781)(502.39146462,678.74889374)
\curveto(502.34146171,678.64888797)(502.29646176,678.55388807)(502.25646462,678.46389374)
\curveto(502.22646183,678.37388825)(502.1564619,678.31388831)(502.04646462,678.28389374)
\curveto(501.96646209,678.26388836)(501.88146217,678.25388837)(501.79146462,678.25389374)
\lineto(501.52146462,678.25389374)
\lineto(500.95146462,678.25389374)
\curveto(500.90146315,678.25388837)(500.8514632,678.24888837)(500.80146462,678.23889374)
\curveto(500.7514633,678.23888838)(500.70646335,678.24388838)(500.66646462,678.25389374)
\lineto(500.53146462,678.25389374)
\curveto(500.51146354,678.26388836)(500.48646357,678.26888835)(500.45646462,678.26889374)
\curveto(500.42646363,678.26888835)(500.40146365,678.27888834)(500.38146462,678.29889374)
\curveto(500.30146375,678.3188883)(500.24646381,678.38388824)(500.21646462,678.49389374)
\curveto(500.20646385,678.54388808)(500.20646385,678.59388803)(500.21646462,678.64389374)
\curveto(500.22646383,678.69388793)(500.23646382,678.73888788)(500.24646462,678.77889374)
\curveto(500.27646378,678.88888773)(500.30646375,678.98888763)(500.33646462,679.07889374)
\curveto(500.37646368,679.17888744)(500.42146363,679.26888735)(500.47146462,679.34889374)
\lineto(500.56146462,679.49889374)
\lineto(500.65146462,679.64889374)
\curveto(500.73146332,679.75888686)(500.83146322,679.86388676)(500.95146462,679.96389374)
\curveto(500.97146308,679.97388665)(501.00146305,679.99888662)(501.04146462,680.03889374)
\curveto(501.09146296,680.07888654)(501.13646292,680.11388651)(501.17646462,680.14389374)
\curveto(501.21646284,680.17388645)(501.26146279,680.20388642)(501.31146462,680.23389374)
\curveto(501.48146257,680.34388628)(501.66146239,680.42888619)(501.85146462,680.48889374)
\curveto(502.04146201,680.55888606)(502.23646182,680.623886)(502.43646462,680.68389374)
\curveto(502.5564615,680.71388591)(502.68146137,680.73388589)(502.81146462,680.74389374)
\curveto(502.94146111,680.75388587)(503.07146098,680.77388585)(503.20146462,680.80389374)
\curveto(503.24146081,680.81388581)(503.30146075,680.81388581)(503.38146462,680.80389374)
\curveto(503.47146058,680.79388583)(503.52646053,680.79888582)(503.54646462,680.81889374)
\curveto(503.9564601,680.82888579)(504.34645971,680.81388581)(504.71646462,680.77389374)
\curveto(505.09645896,680.73388589)(505.43645862,680.65888596)(505.73646462,680.54889374)
\curveto(506.04645801,680.43888618)(506.31145774,680.28888633)(506.53146462,680.09889374)
\curveto(506.7514573,679.9188867)(506.92145713,679.68388694)(507.04146462,679.39389374)
\curveto(507.11145694,679.2238874)(507.1514569,679.02888759)(507.16146462,678.80889374)
\curveto(507.17145688,678.58888803)(507.17645688,678.36388826)(507.17646462,678.13389374)
\lineto(507.17646462,674.78889374)
\lineto(507.17646462,674.20389374)
\curveto(507.17645688,674.01389261)(507.19645686,673.83889278)(507.23646462,673.67889374)
\curveto(507.24645681,673.64889297)(507.2514568,673.61389301)(507.25146462,673.57389374)
\curveto(507.2514568,673.54389308)(507.2564568,673.51389311)(507.26646462,673.48389374)
\moveto(505.06146462,675.79389374)
\curveto(505.07145898,675.84389078)(505.07645898,675.89889072)(505.07646462,675.95889374)
\curveto(505.07645898,676.02889059)(505.07145898,676.08889053)(505.06146462,676.13889374)
\curveto(505.04145901,676.19889042)(505.03145902,676.25389037)(505.03146462,676.30389374)
\curveto(505.03145902,676.35389027)(505.01145904,676.39389023)(504.97146462,676.42389374)
\curveto(504.92145913,676.46389016)(504.84645921,676.48389014)(504.74646462,676.48389374)
\curveto(504.70645935,676.47389015)(504.67145938,676.46389016)(504.64146462,676.45389374)
\curveto(504.61145944,676.45389017)(504.57645948,676.44889017)(504.53646462,676.43889374)
\curveto(504.46645959,676.4188902)(504.39145966,676.40389022)(504.31146462,676.39389374)
\curveto(504.23145982,676.38389024)(504.1514599,676.36889025)(504.07146462,676.34889374)
\curveto(504.04146001,676.33889028)(503.99646006,676.33389029)(503.93646462,676.33389374)
\curveto(503.80646025,676.30389032)(503.67646038,676.28389034)(503.54646462,676.27389374)
\curveto(503.41646064,676.26389036)(503.29146076,676.23889038)(503.17146462,676.19889374)
\curveto(503.09146096,676.17889044)(503.01646104,676.15889046)(502.94646462,676.13889374)
\curveto(502.87646118,676.12889049)(502.80646125,676.10889051)(502.73646462,676.07889374)
\curveto(502.52646153,675.98889063)(502.34646171,675.85389077)(502.19646462,675.67389374)
\curveto(502.056462,675.49389113)(502.00646205,675.24389138)(502.04646462,674.92389374)
\curveto(502.06646199,674.75389187)(502.12146193,674.61389201)(502.21146462,674.50389374)
\curveto(502.28146177,674.39389223)(502.38646167,674.30389232)(502.52646462,674.23389374)
\curveto(502.66646139,674.17389245)(502.81646124,674.12889249)(502.97646462,674.09889374)
\curveto(503.14646091,674.06889255)(503.32146073,674.05889256)(503.50146462,674.06889374)
\curveto(503.69146036,674.08889253)(503.86646019,674.1238925)(504.02646462,674.17389374)
\curveto(504.28645977,674.25389237)(504.49145956,674.37889224)(504.64146462,674.54889374)
\curveto(504.79145926,674.72889189)(504.90645915,674.94889167)(504.98646462,675.20889374)
\curveto(505.00645905,675.27889134)(505.01645904,675.34889127)(505.01646462,675.41889374)
\curveto(505.02645903,675.49889112)(505.04145901,675.57889104)(505.06146462,675.65889374)
\lineto(505.06146462,675.79389374)
}
}
{
\newrgbcolor{curcolor}{0 0 0}
\pscustom[linestyle=none,fillstyle=solid,fillcolor=curcolor]
{
\newpath
\moveto(513.25474587,680.81889374)
\curveto(513.36474055,680.8188858)(513.45974046,680.80888581)(513.53974587,680.78889374)
\curveto(513.62974029,680.76888585)(513.69974022,680.7238859)(513.74974587,680.65389374)
\curveto(513.80974011,680.57388605)(513.83974008,680.43388619)(513.83974587,680.23389374)
\lineto(513.83974587,679.72389374)
\lineto(513.83974587,679.34889374)
\curveto(513.84974007,679.20888741)(513.83474008,679.09888752)(513.79474587,679.01889374)
\curveto(513.75474016,678.94888767)(513.69474022,678.90388772)(513.61474587,678.88389374)
\curveto(513.54474037,678.86388776)(513.45974046,678.85388777)(513.35974587,678.85389374)
\curveto(513.26974065,678.85388777)(513.16974075,678.85888776)(513.05974587,678.86889374)
\curveto(512.95974096,678.87888774)(512.86474105,678.87388775)(512.77474587,678.85389374)
\curveto(512.70474121,678.83388779)(512.63474128,678.8188878)(512.56474587,678.80889374)
\curveto(512.49474142,678.80888781)(512.42974149,678.79888782)(512.36974587,678.77889374)
\curveto(512.20974171,678.72888789)(512.04974187,678.65388797)(511.88974587,678.55389374)
\curveto(511.72974219,678.46388816)(511.60474231,678.35888826)(511.51474587,678.23889374)
\curveto(511.46474245,678.15888846)(511.40974251,678.07388855)(511.34974587,677.98389374)
\curveto(511.29974262,677.90388872)(511.24974267,677.8188888)(511.19974587,677.72889374)
\curveto(511.16974275,677.64888897)(511.13974278,677.56388906)(511.10974587,677.47389374)
\lineto(511.04974587,677.23389374)
\curveto(511.02974289,677.16388946)(511.0197429,677.08888953)(511.01974587,677.00889374)
\curveto(511.0197429,676.93888968)(511.00974291,676.86888975)(510.98974587,676.79889374)
\curveto(510.97974294,676.75888986)(510.97474294,676.7188899)(510.97474587,676.67889374)
\curveto(510.98474293,676.64888997)(510.98474293,676.61889)(510.97474587,676.58889374)
\lineto(510.97474587,676.34889374)
\curveto(510.95474296,676.27889034)(510.94974297,676.19889042)(510.95974587,676.10889374)
\curveto(510.96974295,676.02889059)(510.97474294,675.94889067)(510.97474587,675.86889374)
\lineto(510.97474587,674.90889374)
\lineto(510.97474587,673.63389374)
\curveto(510.97474294,673.50389312)(510.96974295,673.38389324)(510.95974587,673.27389374)
\curveto(510.94974297,673.16389346)(510.919743,673.07389355)(510.86974587,673.00389374)
\curveto(510.84974307,672.97389365)(510.8147431,672.94889367)(510.76474587,672.92889374)
\curveto(510.72474319,672.9188937)(510.67974324,672.90889371)(510.62974587,672.89889374)
\lineto(510.55474587,672.89889374)
\curveto(510.50474341,672.88889373)(510.44974347,672.88389374)(510.38974587,672.88389374)
\lineto(510.22474587,672.88389374)
\lineto(509.57974587,672.88389374)
\curveto(509.5197444,672.89389373)(509.45474446,672.89889372)(509.38474587,672.89889374)
\lineto(509.18974587,672.89889374)
\curveto(509.13974478,672.9188937)(509.08974483,672.93389369)(509.03974587,672.94389374)
\curveto(508.98974493,672.96389366)(508.95474496,672.99889362)(508.93474587,673.04889374)
\curveto(508.89474502,673.09889352)(508.86974505,673.16889345)(508.85974587,673.25889374)
\lineto(508.85974587,673.55889374)
\lineto(508.85974587,674.57889374)
\lineto(508.85974587,678.80889374)
\lineto(508.85974587,679.91889374)
\lineto(508.85974587,680.20389374)
\curveto(508.85974506,680.30388632)(508.87974504,680.38388624)(508.91974587,680.44389374)
\curveto(508.96974495,680.5238861)(509.04474487,680.57388605)(509.14474587,680.59389374)
\curveto(509.24474467,680.61388601)(509.36474455,680.623886)(509.50474587,680.62389374)
\lineto(510.26974587,680.62389374)
\curveto(510.38974353,680.623886)(510.49474342,680.61388601)(510.58474587,680.59389374)
\curveto(510.67474324,680.58388604)(510.74474317,680.53888608)(510.79474587,680.45889374)
\curveto(510.82474309,680.40888621)(510.83974308,680.33888628)(510.83974587,680.24889374)
\lineto(510.86974587,679.97889374)
\curveto(510.87974304,679.89888672)(510.89474302,679.8238868)(510.91474587,679.75389374)
\curveto(510.94474297,679.68388694)(510.99474292,679.64888697)(511.06474587,679.64889374)
\curveto(511.08474283,679.66888695)(511.10474281,679.67888694)(511.12474587,679.67889374)
\curveto(511.14474277,679.67888694)(511.16474275,679.68888693)(511.18474587,679.70889374)
\curveto(511.24474267,679.75888686)(511.29474262,679.81388681)(511.33474587,679.87389374)
\curveto(511.38474253,679.94388668)(511.44474247,680.00388662)(511.51474587,680.05389374)
\curveto(511.55474236,680.08388654)(511.58974233,680.11388651)(511.61974587,680.14389374)
\curveto(511.64974227,680.18388644)(511.68474223,680.2188864)(511.72474587,680.24889374)
\lineto(511.99474587,680.42889374)
\curveto(512.09474182,680.48888613)(512.19474172,680.54388608)(512.29474587,680.59389374)
\curveto(512.39474152,680.63388599)(512.49474142,680.66888595)(512.59474587,680.69889374)
\lineto(512.92474587,680.78889374)
\curveto(512.95474096,680.79888582)(513.00974091,680.79888582)(513.08974587,680.78889374)
\curveto(513.17974074,680.78888583)(513.23474068,680.79888582)(513.25474587,680.81889374)
}
}
{
\newrgbcolor{curcolor}{0 0 0}
\pscustom[linestyle=none,fillstyle=solid,fillcolor=curcolor]
{
\newpath
\moveto(516.75982399,683.47389374)
\curveto(516.82982104,683.39388323)(516.86482101,683.27388335)(516.86482399,683.11389374)
\lineto(516.86482399,682.64889374)
\lineto(516.86482399,682.24389374)
\curveto(516.86482101,682.10388452)(516.82982104,682.00888461)(516.75982399,681.95889374)
\curveto(516.69982117,681.90888471)(516.61982125,681.87888474)(516.51982399,681.86889374)
\curveto(516.42982144,681.85888476)(516.32982154,681.85388477)(516.21982399,681.85389374)
\lineto(515.37982399,681.85389374)
\curveto(515.2698226,681.85388477)(515.1698227,681.85888476)(515.07982399,681.86889374)
\curveto(514.99982287,681.87888474)(514.92982294,681.90888471)(514.86982399,681.95889374)
\curveto(514.82982304,681.98888463)(514.79982307,682.04388458)(514.77982399,682.12389374)
\curveto(514.7698231,682.21388441)(514.75982311,682.30888431)(514.74982399,682.40889374)
\lineto(514.74982399,682.73889374)
\curveto(514.75982311,682.84888377)(514.76482311,682.94388368)(514.76482399,683.02389374)
\lineto(514.76482399,683.23389374)
\curveto(514.7748231,683.30388332)(514.79482308,683.36388326)(514.82482399,683.41389374)
\curveto(514.84482303,683.45388317)(514.869823,683.48388314)(514.89982399,683.50389374)
\lineto(515.01982399,683.56389374)
\curveto(515.03982283,683.56388306)(515.06482281,683.56388306)(515.09482399,683.56389374)
\curveto(515.12482275,683.57388305)(515.14982272,683.57888304)(515.16982399,683.57889374)
\lineto(516.26482399,683.57889374)
\curveto(516.36482151,683.57888304)(516.45982141,683.57388305)(516.54982399,683.56389374)
\curveto(516.63982123,683.55388307)(516.70982116,683.5238831)(516.75982399,683.47389374)
\moveto(516.86482399,673.70889374)
\curveto(516.86482101,673.50889311)(516.85982101,673.33889328)(516.84982399,673.19889374)
\curveto(516.83982103,673.05889356)(516.74982112,672.96389366)(516.57982399,672.91389374)
\curveto(516.51982135,672.89389373)(516.45482142,672.88389374)(516.38482399,672.88389374)
\curveto(516.31482156,672.89389373)(516.23982163,672.89889372)(516.15982399,672.89889374)
\lineto(515.31982399,672.89889374)
\curveto(515.22982264,672.89889372)(515.13982273,672.90389372)(515.04982399,672.91389374)
\curveto(514.9698229,672.9238937)(514.90982296,672.95389367)(514.86982399,673.00389374)
\curveto(514.80982306,673.07389355)(514.7748231,673.15889346)(514.76482399,673.25889374)
\lineto(514.76482399,673.60389374)
\lineto(514.76482399,679.93389374)
\lineto(514.76482399,680.23389374)
\curveto(514.76482311,680.33388629)(514.78482309,680.41388621)(514.82482399,680.47389374)
\curveto(514.88482299,680.54388608)(514.9698229,680.58888603)(515.07982399,680.60889374)
\curveto(515.09982277,680.618886)(515.12482275,680.618886)(515.15482399,680.60889374)
\curveto(515.19482268,680.60888601)(515.22482265,680.61388601)(515.24482399,680.62389374)
\lineto(515.99482399,680.62389374)
\lineto(516.18982399,680.62389374)
\curveto(516.2698216,680.63388599)(516.33482154,680.63388599)(516.38482399,680.62389374)
\lineto(516.50482399,680.62389374)
\curveto(516.56482131,680.60388602)(516.61982125,680.58888603)(516.66982399,680.57889374)
\curveto(516.71982115,680.56888605)(516.75982111,680.53888608)(516.78982399,680.48889374)
\curveto(516.82982104,680.43888618)(516.84982102,680.36888625)(516.84982399,680.27889374)
\curveto(516.85982101,680.18888643)(516.86482101,680.09388653)(516.86482399,679.99389374)
\lineto(516.86482399,673.70889374)
}
}
{
\newrgbcolor{curcolor}{0 0 0}
\pscustom[linestyle=none,fillstyle=solid,fillcolor=curcolor]
{
\newpath
\moveto(526.29701149,677.06889374)
\curveto(526.31700292,677.00888961)(526.32700291,676.9238897)(526.32701149,676.81389374)
\curveto(526.32700291,676.70388992)(526.31700292,676.61889)(526.29701149,676.55889374)
\lineto(526.29701149,676.40889374)
\curveto(526.27700296,676.32889029)(526.26700297,676.24889037)(526.26701149,676.16889374)
\curveto(526.27700296,676.08889053)(526.27200297,676.00889061)(526.25201149,675.92889374)
\curveto(526.23200301,675.85889076)(526.21700302,675.79389083)(526.20701149,675.73389374)
\curveto(526.19700304,675.67389095)(526.18700305,675.60889101)(526.17701149,675.53889374)
\curveto(526.1370031,675.42889119)(526.10200314,675.31389131)(526.07201149,675.19389374)
\curveto(526.0420032,675.08389154)(526.00200324,674.97889164)(525.95201149,674.87889374)
\curveto(525.7420035,674.39889222)(525.46700377,674.00889261)(525.12701149,673.70889374)
\curveto(524.78700445,673.40889321)(524.37700486,673.15889346)(523.89701149,672.95889374)
\curveto(523.77700546,672.90889371)(523.65200559,672.87389375)(523.52201149,672.85389374)
\curveto(523.40200584,672.8238938)(523.27700596,672.79389383)(523.14701149,672.76389374)
\curveto(523.09700614,672.74389388)(523.0420062,672.73389389)(522.98201149,672.73389374)
\curveto(522.92200632,672.73389389)(522.86700637,672.72889389)(522.81701149,672.71889374)
\lineto(522.71201149,672.71889374)
\curveto(522.68200656,672.70889391)(522.65200659,672.70389392)(522.62201149,672.70389374)
\curveto(522.57200667,672.69389393)(522.49200675,672.68889393)(522.38201149,672.68889374)
\curveto(522.27200697,672.67889394)(522.18700705,672.68389394)(522.12701149,672.70389374)
\lineto(521.97701149,672.70389374)
\curveto(521.92700731,672.71389391)(521.87200737,672.7188939)(521.81201149,672.71889374)
\curveto(521.76200748,672.70889391)(521.71200753,672.71389391)(521.66201149,672.73389374)
\curveto(521.62200762,672.74389388)(521.58200766,672.74889387)(521.54201149,672.74889374)
\curveto(521.51200773,672.74889387)(521.47200777,672.75389387)(521.42201149,672.76389374)
\curveto(521.32200792,672.79389383)(521.22200802,672.8188938)(521.12201149,672.83889374)
\curveto(521.02200822,672.85889376)(520.92700831,672.88889373)(520.83701149,672.92889374)
\curveto(520.71700852,672.96889365)(520.60200864,673.00889361)(520.49201149,673.04889374)
\curveto(520.39200885,673.08889353)(520.28700895,673.13889348)(520.17701149,673.19889374)
\curveto(519.82700941,673.40889321)(519.52700971,673.65389297)(519.27701149,673.93389374)
\curveto(519.02701021,674.21389241)(518.81701042,674.54889207)(518.64701149,674.93889374)
\curveto(518.59701064,675.02889159)(518.55701068,675.1238915)(518.52701149,675.22389374)
\curveto(518.50701073,675.3238913)(518.48201076,675.42889119)(518.45201149,675.53889374)
\curveto(518.43201081,675.58889103)(518.42201082,675.63389099)(518.42201149,675.67389374)
\curveto(518.42201082,675.71389091)(518.41201083,675.75889086)(518.39201149,675.80889374)
\curveto(518.37201087,675.88889073)(518.36201088,675.96889065)(518.36201149,676.04889374)
\curveto(518.36201088,676.13889048)(518.35201089,676.2238904)(518.33201149,676.30389374)
\curveto(518.32201092,676.35389027)(518.31701092,676.39889022)(518.31701149,676.43889374)
\lineto(518.31701149,676.57389374)
\curveto(518.29701094,676.63388999)(518.28701095,676.7188899)(518.28701149,676.82889374)
\curveto(518.29701094,676.93888968)(518.31201093,677.0238896)(518.33201149,677.08389374)
\lineto(518.33201149,677.18889374)
\curveto(518.3420109,677.23888938)(518.3420109,677.28888933)(518.33201149,677.33889374)
\curveto(518.33201091,677.39888922)(518.3420109,677.45388917)(518.36201149,677.50389374)
\curveto(518.37201087,677.55388907)(518.37701086,677.59888902)(518.37701149,677.63889374)
\curveto(518.37701086,677.68888893)(518.38701085,677.73888888)(518.40701149,677.78889374)
\curveto(518.44701079,677.9188887)(518.48201076,678.04388858)(518.51201149,678.16389374)
\curveto(518.5420107,678.29388833)(518.58201066,678.4188882)(518.63201149,678.53889374)
\curveto(518.81201043,678.94888767)(519.02701021,679.28888733)(519.27701149,679.55889374)
\curveto(519.52700971,679.83888678)(519.83200941,680.09388653)(520.19201149,680.32389374)
\curveto(520.29200895,680.37388625)(520.39700884,680.4188862)(520.50701149,680.45889374)
\curveto(520.61700862,680.49888612)(520.72700851,680.54388608)(520.83701149,680.59389374)
\curveto(520.96700827,680.64388598)(521.10200814,680.67888594)(521.24201149,680.69889374)
\curveto(521.38200786,680.7188859)(521.52700771,680.74888587)(521.67701149,680.78889374)
\curveto(521.75700748,680.79888582)(521.83200741,680.80388582)(521.90201149,680.80389374)
\curveto(521.97200727,680.80388582)(522.0420072,680.80888581)(522.11201149,680.81889374)
\curveto(522.69200655,680.82888579)(523.19200605,680.76888585)(523.61201149,680.63889374)
\curveto(524.0420052,680.50888611)(524.42200482,680.32888629)(524.75201149,680.09889374)
\curveto(524.86200438,680.0188866)(524.97200427,679.92888669)(525.08201149,679.82889374)
\curveto(525.20200404,679.73888688)(525.30200394,679.63888698)(525.38201149,679.52889374)
\curveto(525.46200378,679.42888719)(525.53200371,679.32888729)(525.59201149,679.22889374)
\curveto(525.66200358,679.12888749)(525.73200351,679.0238876)(525.80201149,678.91389374)
\curveto(525.87200337,678.80388782)(525.92700331,678.68388794)(525.96701149,678.55389374)
\curveto(526.00700323,678.43388819)(526.05200319,678.30388832)(526.10201149,678.16389374)
\curveto(526.13200311,678.08388854)(526.15700308,677.99888862)(526.17701149,677.90889374)
\lineto(526.23701149,677.63889374)
\curveto(526.24700299,677.59888902)(526.25200299,677.55888906)(526.25201149,677.51889374)
\curveto(526.25200299,677.47888914)(526.25700298,677.43888918)(526.26701149,677.39889374)
\curveto(526.28700295,677.34888927)(526.29200295,677.29388933)(526.28201149,677.23389374)
\curveto(526.27200297,677.17388945)(526.27700296,677.1188895)(526.29701149,677.06889374)
\moveto(524.19701149,676.52889374)
\curveto(524.20700503,676.57889004)(524.21200503,676.64888997)(524.21201149,676.73889374)
\curveto(524.21200503,676.83888978)(524.20700503,676.91388971)(524.19701149,676.96389374)
\lineto(524.19701149,677.08389374)
\curveto(524.17700506,677.13388949)(524.16700507,677.18888943)(524.16701149,677.24889374)
\curveto(524.16700507,677.30888931)(524.16200508,677.36388926)(524.15201149,677.41389374)
\curveto(524.15200509,677.45388917)(524.14700509,677.48388914)(524.13701149,677.50389374)
\lineto(524.07701149,677.74389374)
\curveto(524.06700517,677.83388879)(524.04700519,677.9188887)(524.01701149,677.99889374)
\curveto(523.90700533,678.25888836)(523.77700546,678.47888814)(523.62701149,678.65889374)
\curveto(523.47700576,678.84888777)(523.27700596,678.99888762)(523.02701149,679.10889374)
\curveto(522.96700627,679.12888749)(522.90700633,679.14388748)(522.84701149,679.15389374)
\curveto(522.78700645,679.17388745)(522.72200652,679.19388743)(522.65201149,679.21389374)
\curveto(522.57200667,679.23388739)(522.48700675,679.23888738)(522.39701149,679.22889374)
\lineto(522.12701149,679.22889374)
\curveto(522.09700714,679.20888741)(522.06200718,679.19888742)(522.02201149,679.19889374)
\curveto(521.98200726,679.20888741)(521.94700729,679.20888741)(521.91701149,679.19889374)
\lineto(521.70701149,679.13889374)
\curveto(521.64700759,679.12888749)(521.59200765,679.10888751)(521.54201149,679.07889374)
\curveto(521.29200795,678.96888765)(521.08700815,678.80888781)(520.92701149,678.59889374)
\curveto(520.77700846,678.39888822)(520.65700858,678.16388846)(520.56701149,677.89389374)
\curveto(520.5370087,677.79388883)(520.51200873,677.68888893)(520.49201149,677.57889374)
\curveto(520.48200876,677.46888915)(520.46700877,677.35888926)(520.44701149,677.24889374)
\curveto(520.4370088,677.19888942)(520.43200881,677.14888947)(520.43201149,677.09889374)
\lineto(520.43201149,676.94889374)
\curveto(520.41200883,676.87888974)(520.40200884,676.77388985)(520.40201149,676.63389374)
\curveto(520.41200883,676.49389013)(520.42700881,676.38889023)(520.44701149,676.31889374)
\lineto(520.44701149,676.18389374)
\curveto(520.46700877,676.10389052)(520.48200876,676.0238906)(520.49201149,675.94389374)
\curveto(520.50200874,675.87389075)(520.51700872,675.79889082)(520.53701149,675.71889374)
\curveto(520.6370086,675.4188912)(520.7420085,675.17389145)(520.85201149,674.98389374)
\curveto(520.97200827,674.80389182)(521.15700808,674.63889198)(521.40701149,674.48889374)
\curveto(521.47700776,674.43889218)(521.55200769,674.39889222)(521.63201149,674.36889374)
\curveto(521.72200752,674.33889228)(521.81200743,674.31389231)(521.90201149,674.29389374)
\curveto(521.9420073,674.28389234)(521.97700726,674.27889234)(522.00701149,674.27889374)
\curveto(522.0370072,674.28889233)(522.07200717,674.28889233)(522.11201149,674.27889374)
\lineto(522.23201149,674.24889374)
\curveto(522.28200696,674.24889237)(522.32700691,674.25389237)(522.36701149,674.26389374)
\lineto(522.48701149,674.26389374)
\curveto(522.56700667,674.28389234)(522.64700659,674.29889232)(522.72701149,674.30889374)
\curveto(522.80700643,674.3188923)(522.88200636,674.33889228)(522.95201149,674.36889374)
\curveto(523.21200603,674.46889215)(523.42200582,674.60389202)(523.58201149,674.77389374)
\curveto(523.7420055,674.94389168)(523.87700536,675.15389147)(523.98701149,675.40389374)
\curveto(524.02700521,675.50389112)(524.05700518,675.60389102)(524.07701149,675.70389374)
\curveto(524.09700514,675.80389082)(524.12200512,675.90889071)(524.15201149,676.01889374)
\curveto(524.16200508,676.05889056)(524.16700507,676.09389053)(524.16701149,676.12389374)
\curveto(524.16700507,676.16389046)(524.17200507,676.20389042)(524.18201149,676.24389374)
\lineto(524.18201149,676.37889374)
\curveto(524.18200506,676.42889019)(524.18700505,676.47889014)(524.19701149,676.52889374)
}
}
{
\newrgbcolor{curcolor}{0 0 0}
\pscustom[linestyle=none,fillstyle=solid,fillcolor=curcolor]
{
\newpath
\moveto(26.7416285,358.05818817)
\lineto(28.0166285,358.05818817)
\curveto(28.12662572,358.05817746)(28.23162561,358.05317747)(28.3316285,358.04318817)
\curveto(28.4416254,358.03317749)(28.52162532,357.99817752)(28.5716285,357.93818817)
\curveto(28.62162522,357.85817766)(28.6466252,357.75317777)(28.6466285,357.62318817)
\curveto(28.65662519,357.50317802)(28.66162518,357.37817814)(28.6616285,357.24818817)
\lineto(28.6616285,355.73318817)
\lineto(28.6616285,352.64318817)
\lineto(28.6616285,352.11818817)
\curveto(28.66162518,352.07818344)(28.65662519,352.03318349)(28.6466285,351.98318817)
\curveto(28.6466252,351.94318358)(28.65162519,351.90318362)(28.6616285,351.86318817)
\lineto(28.6616285,351.62318817)
\curveto(28.66162518,351.53318399)(28.65662519,351.43818408)(28.6466285,351.33818817)
\curveto(28.6466252,351.23818428)(28.65662519,351.14818437)(28.6766285,351.06818817)
\curveto(28.67662517,350.99818452)(28.68162516,350.94318458)(28.6916285,350.90318817)
\curveto(28.71162513,350.79318473)(28.72662512,350.68318484)(28.7366285,350.57318817)
\curveto(28.75662509,350.46318506)(28.78662506,350.35318517)(28.8266285,350.24318817)
\curveto(28.93662491,349.98318554)(29.07662477,349.76818575)(29.2466285,349.59818817)
\curveto(29.42662442,349.42818609)(29.66162418,349.29318623)(29.9516285,349.19318817)
\curveto(30.03162381,349.17318635)(30.11162373,349.15818636)(30.1916285,349.14818817)
\curveto(30.27162357,349.13818638)(30.35162349,349.1231864)(30.4316285,349.10318817)
\curveto(30.48162336,349.08318644)(30.52662332,349.07318645)(30.5666285,349.07318817)
\curveto(30.60662324,349.08318644)(30.65162319,349.08318644)(30.7016285,349.07318817)
\curveto(30.7416231,349.06318646)(30.80662304,349.05818646)(30.8966285,349.05818817)
\curveto(30.98662286,349.06818645)(31.0466228,349.07818644)(31.0766285,349.08818817)
\lineto(31.3016285,349.08818817)
\curveto(31.38162246,349.10818641)(31.46162238,349.1231864)(31.5416285,349.13318817)
\curveto(31.62162222,349.14318638)(31.69662215,349.15818636)(31.7666285,349.17818817)
\curveto(31.90662194,349.20818631)(32.01662183,349.24318628)(32.0966285,349.28318817)
\curveto(32.27662157,349.36318616)(32.43162141,349.46818605)(32.5616285,349.59818817)
\curveto(32.70162114,349.73818578)(32.81162103,349.89318563)(32.8916285,350.06318817)
\curveto(33.00162084,350.3231852)(33.06662078,350.62818489)(33.0866285,350.97818817)
\curveto(33.10662074,351.33818418)(33.11662073,351.70818381)(33.1166285,352.08818817)
\lineto(33.1166285,355.07318817)
\lineto(33.1166285,357.08318817)
\curveto(33.11662073,357.2231783)(33.11162073,357.37817814)(33.1016285,357.54818817)
\curveto(33.10162074,357.7181778)(33.13162071,357.84317768)(33.1916285,357.92318817)
\curveto(33.2416206,357.98317754)(33.31162053,358.0181775)(33.4016285,358.02818817)
\curveto(33.49162035,358.04817747)(33.59162025,358.05817746)(33.7016285,358.05818817)
\lineto(34.6616285,358.05818817)
\curveto(34.7416191,358.05817746)(34.81661903,358.05817746)(34.8866285,358.05818817)
\curveto(34.96661888,358.06817745)(35.0416188,358.06317746)(35.1116285,358.04318817)
\curveto(35.25161859,358.01317751)(35.3416185,357.96317756)(35.3816285,357.89318817)
\curveto(35.43161841,357.81317771)(35.45161839,357.69817782)(35.4416285,357.54818817)
\curveto(35.4416184,357.40817811)(35.4416184,357.27817824)(35.4416285,357.15818817)
\lineto(35.4416285,355.14818817)
\lineto(35.4416285,352.11818817)
\curveto(35.4416184,351.73818378)(35.43661841,351.36818415)(35.4266285,351.00818817)
\curveto(35.41661843,350.64818487)(35.37161847,350.3231852)(35.2916285,350.03318817)
\curveto(35.15161869,349.56318596)(34.97161887,349.15318637)(34.7516285,348.80318817)
\curveto(34.5416193,348.46318706)(34.26161958,348.17318735)(33.9116285,347.93318817)
\curveto(33.60162024,347.71318781)(33.23662061,347.53318799)(32.8166285,347.39318817)
\curveto(32.72662112,347.36318816)(32.63162121,347.33818818)(32.5316285,347.31818817)
\lineto(32.2616285,347.25818817)
\curveto(32.20162164,347.23818828)(32.1416217,347.22818829)(32.0816285,347.22818817)
\curveto(32.03162181,347.22818829)(31.97662187,347.2181883)(31.9166285,347.19818817)
\curveto(31.79662205,347.17818834)(31.66162218,347.16318836)(31.5116285,347.15318817)
\curveto(31.36162248,347.14318838)(31.21662263,347.13818838)(31.0766285,347.13818817)
\curveto(30.12662372,347.12818839)(29.31662453,347.24318828)(28.6466285,347.48318817)
\curveto(27.97662587,347.73318779)(27.45162639,348.13318739)(27.0716285,348.68318817)
\curveto(26.9416269,348.86318666)(26.83162701,349.04818647)(26.7416285,349.23818817)
\curveto(26.66162718,349.43818608)(26.58662726,349.65318587)(26.5166285,349.88318817)
\curveto(26.49662735,349.93318559)(26.48662736,349.97318555)(26.4866285,350.00318817)
\curveto(26.48662736,350.04318548)(26.47662737,350.08818543)(26.4566285,350.13818817)
\curveto(26.37662747,350.4181851)(26.33662751,350.73318479)(26.3366285,351.08318817)
\lineto(26.3366285,352.13318817)
\lineto(26.3366285,356.31818817)
\lineto(26.3366285,357.36818817)
\lineto(26.3366285,357.65318817)
\curveto(26.33662751,357.75317777)(26.35162749,357.83317769)(26.3816285,357.89318817)
\curveto(26.4416274,357.96317756)(26.52162732,358.01317751)(26.6216285,358.04318817)
\curveto(26.6416272,358.04317748)(26.66162718,358.04317748)(26.6816285,358.04318817)
\curveto(26.70162714,358.04317748)(26.72162712,358.04817747)(26.7416285,358.05818817)
}
}
{
\newrgbcolor{curcolor}{0 0 0}
\pscustom[linestyle=none,fillstyle=solid,fillcolor=curcolor]
{
\newpath
\moveto(40.20014412,355.29818817)
\curveto(40.95013962,355.3181802)(41.60013897,355.23318029)(42.15014412,355.04318817)
\curveto(42.71013786,354.86318066)(43.13513744,354.54818097)(43.42514412,354.09818817)
\curveto(43.49513708,353.98818153)(43.55513702,353.87318165)(43.60514412,353.75318817)
\curveto(43.66513691,353.64318188)(43.71513686,353.518182)(43.75514412,353.37818817)
\curveto(43.7751368,353.3181822)(43.78513679,353.25318227)(43.78514412,353.18318817)
\curveto(43.78513679,353.11318241)(43.7751368,353.05318247)(43.75514412,353.00318817)
\curveto(43.71513686,352.94318258)(43.66013691,352.90318262)(43.59014412,352.88318817)
\curveto(43.54013703,352.86318266)(43.48013709,352.85318267)(43.41014412,352.85318817)
\lineto(43.20014412,352.85318817)
\lineto(42.54014412,352.85318817)
\curveto(42.4701381,352.85318267)(42.40013817,352.84818267)(42.33014412,352.83818817)
\curveto(42.26013831,352.83818268)(42.19513838,352.84818267)(42.13514412,352.86818817)
\curveto(42.03513854,352.88818263)(41.96013861,352.92818259)(41.91014412,352.98818817)
\curveto(41.86013871,353.04818247)(41.81513876,353.10818241)(41.77514412,353.16818817)
\lineto(41.65514412,353.37818817)
\curveto(41.62513895,353.45818206)(41.575139,353.523182)(41.50514412,353.57318817)
\curveto(41.40513917,353.65318187)(41.30513927,353.71318181)(41.20514412,353.75318817)
\curveto(41.11513946,353.79318173)(41.00013957,353.82818169)(40.86014412,353.85818817)
\curveto(40.79013978,353.87818164)(40.68513989,353.89318163)(40.54514412,353.90318817)
\curveto(40.41514016,353.91318161)(40.31514026,353.90818161)(40.24514412,353.88818817)
\lineto(40.14014412,353.88818817)
\lineto(39.99014412,353.85818817)
\curveto(39.95014062,353.85818166)(39.90514067,353.85318167)(39.85514412,353.84318817)
\curveto(39.68514089,353.79318173)(39.54514103,353.7231818)(39.43514412,353.63318817)
\curveto(39.33514124,353.55318197)(39.26514131,353.42818209)(39.22514412,353.25818817)
\curveto(39.20514137,353.18818233)(39.20514137,353.1231824)(39.22514412,353.06318817)
\curveto(39.24514133,353.00318252)(39.26514131,352.95318257)(39.28514412,352.91318817)
\curveto(39.35514122,352.79318273)(39.43514114,352.69818282)(39.52514412,352.62818817)
\curveto(39.62514095,352.55818296)(39.74014083,352.49818302)(39.87014412,352.44818817)
\curveto(40.06014051,352.36818315)(40.26514031,352.29818322)(40.48514412,352.23818817)
\lineto(41.17514412,352.08818817)
\curveto(41.41513916,352.04818347)(41.64513893,351.99818352)(41.86514412,351.93818817)
\curveto(42.09513848,351.88818363)(42.31013826,351.8231837)(42.51014412,351.74318817)
\curveto(42.60013797,351.70318382)(42.68513789,351.66818385)(42.76514412,351.63818817)
\curveto(42.85513772,351.6181839)(42.94013763,351.58318394)(43.02014412,351.53318817)
\curveto(43.21013736,351.41318411)(43.38013719,351.28318424)(43.53014412,351.14318817)
\curveto(43.69013688,351.00318452)(43.81513676,350.82818469)(43.90514412,350.61818817)
\curveto(43.93513664,350.54818497)(43.96013661,350.47818504)(43.98014412,350.40818817)
\curveto(44.00013657,350.33818518)(44.02013655,350.26318526)(44.04014412,350.18318817)
\curveto(44.05013652,350.1231854)(44.05513652,350.02818549)(44.05514412,349.89818817)
\curveto(44.06513651,349.77818574)(44.06513651,349.68318584)(44.05514412,349.61318817)
\lineto(44.05514412,349.53818817)
\curveto(44.03513654,349.47818604)(44.02013655,349.4181861)(44.01014412,349.35818817)
\curveto(44.01013656,349.30818621)(44.00513657,349.25818626)(43.99514412,349.20818817)
\curveto(43.92513665,348.90818661)(43.81513676,348.64318688)(43.66514412,348.41318817)
\curveto(43.50513707,348.17318735)(43.31013726,347.97818754)(43.08014412,347.82818817)
\curveto(42.85013772,347.67818784)(42.59013798,347.54818797)(42.30014412,347.43818817)
\curveto(42.19013838,347.38818813)(42.0701385,347.35318817)(41.94014412,347.33318817)
\curveto(41.82013875,347.31318821)(41.70013887,347.28818823)(41.58014412,347.25818817)
\curveto(41.49013908,347.23818828)(41.39513918,347.22818829)(41.29514412,347.22818817)
\curveto(41.20513937,347.2181883)(41.11513946,347.20318832)(41.02514412,347.18318817)
\lineto(40.75514412,347.18318817)
\curveto(40.69513988,347.16318836)(40.59013998,347.15318837)(40.44014412,347.15318817)
\curveto(40.30014027,347.15318837)(40.20014037,347.16318836)(40.14014412,347.18318817)
\curveto(40.11014046,347.18318834)(40.0751405,347.18818833)(40.03514412,347.19818817)
\lineto(39.93014412,347.19818817)
\curveto(39.81014076,347.2181883)(39.69014088,347.23318829)(39.57014412,347.24318817)
\curveto(39.45014112,347.25318827)(39.33514124,347.27318825)(39.22514412,347.30318817)
\curveto(38.83514174,347.41318811)(38.49014208,347.53818798)(38.19014412,347.67818817)
\curveto(37.89014268,347.82818769)(37.63514294,348.04818747)(37.42514412,348.33818817)
\curveto(37.28514329,348.52818699)(37.16514341,348.74818677)(37.06514412,348.99818817)
\curveto(37.04514353,349.05818646)(37.02514355,349.13818638)(37.00514412,349.23818817)
\curveto(36.98514359,349.28818623)(36.9701436,349.35818616)(36.96014412,349.44818817)
\curveto(36.95014362,349.53818598)(36.95514362,349.61318591)(36.97514412,349.67318817)
\curveto(37.00514357,349.74318578)(37.05514352,349.79318573)(37.12514412,349.82318817)
\curveto(37.1751434,349.84318568)(37.23514334,349.85318567)(37.30514412,349.85318817)
\lineto(37.53014412,349.85318817)
\lineto(38.23514412,349.85318817)
\lineto(38.47514412,349.85318817)
\curveto(38.55514202,349.85318567)(38.62514195,349.84318568)(38.68514412,349.82318817)
\curveto(38.79514178,349.78318574)(38.86514171,349.7181858)(38.89514412,349.62818817)
\curveto(38.93514164,349.53818598)(38.98014159,349.44318608)(39.03014412,349.34318817)
\curveto(39.05014152,349.29318623)(39.08514149,349.22818629)(39.13514412,349.14818817)
\curveto(39.19514138,349.06818645)(39.24514133,349.0181865)(39.28514412,348.99818817)
\curveto(39.40514117,348.89818662)(39.52014105,348.8181867)(39.63014412,348.75818817)
\curveto(39.74014083,348.70818681)(39.88014069,348.65818686)(40.05014412,348.60818817)
\curveto(40.10014047,348.58818693)(40.15014042,348.57818694)(40.20014412,348.57818817)
\curveto(40.25014032,348.58818693)(40.30014027,348.58818693)(40.35014412,348.57818817)
\curveto(40.43014014,348.55818696)(40.51514006,348.54818697)(40.60514412,348.54818817)
\curveto(40.70513987,348.55818696)(40.79013978,348.57318695)(40.86014412,348.59318817)
\curveto(40.91013966,348.60318692)(40.95513962,348.60818691)(40.99514412,348.60818817)
\curveto(41.04513953,348.60818691)(41.09513948,348.6181869)(41.14514412,348.63818817)
\curveto(41.28513929,348.68818683)(41.41013916,348.74818677)(41.52014412,348.81818817)
\curveto(41.64013893,348.88818663)(41.73513884,348.97818654)(41.80514412,349.08818817)
\curveto(41.85513872,349.16818635)(41.89513868,349.29318623)(41.92514412,349.46318817)
\curveto(41.94513863,349.53318599)(41.94513863,349.59818592)(41.92514412,349.65818817)
\curveto(41.90513867,349.7181858)(41.88513869,349.76818575)(41.86514412,349.80818817)
\curveto(41.79513878,349.94818557)(41.70513887,350.05318547)(41.59514412,350.12318817)
\curveto(41.49513908,350.19318533)(41.3751392,350.25818526)(41.23514412,350.31818817)
\curveto(41.04513953,350.39818512)(40.84513973,350.46318506)(40.63514412,350.51318817)
\curveto(40.42514015,350.56318496)(40.21514036,350.6181849)(40.00514412,350.67818817)
\curveto(39.92514065,350.69818482)(39.84014073,350.71318481)(39.75014412,350.72318817)
\curveto(39.6701409,350.73318479)(39.59014098,350.74818477)(39.51014412,350.76818817)
\curveto(39.19014138,350.85818466)(38.88514169,350.94318458)(38.59514412,351.02318817)
\curveto(38.30514227,351.11318441)(38.04014253,351.24318428)(37.80014412,351.41318817)
\curveto(37.52014305,351.61318391)(37.31514326,351.88318364)(37.18514412,352.22318817)
\curveto(37.16514341,352.29318323)(37.14514343,352.38818313)(37.12514412,352.50818817)
\curveto(37.10514347,352.57818294)(37.09014348,352.66318286)(37.08014412,352.76318817)
\curveto(37.0701435,352.86318266)(37.0751435,352.95318257)(37.09514412,353.03318817)
\curveto(37.11514346,353.08318244)(37.12014345,353.1231824)(37.11014412,353.15318817)
\curveto(37.10014347,353.19318233)(37.10514347,353.23818228)(37.12514412,353.28818817)
\curveto(37.14514343,353.39818212)(37.16514341,353.49818202)(37.18514412,353.58818817)
\curveto(37.21514336,353.68818183)(37.25014332,353.78318174)(37.29014412,353.87318817)
\curveto(37.42014315,354.16318136)(37.60014297,354.39818112)(37.83014412,354.57818817)
\curveto(38.06014251,354.75818076)(38.32014225,354.90318062)(38.61014412,355.01318817)
\curveto(38.72014185,355.06318046)(38.83514174,355.09818042)(38.95514412,355.11818817)
\curveto(39.0751415,355.14818037)(39.20014137,355.17818034)(39.33014412,355.20818817)
\curveto(39.39014118,355.22818029)(39.45014112,355.23818028)(39.51014412,355.23818817)
\lineto(39.69014412,355.26818817)
\curveto(39.7701408,355.27818024)(39.85514072,355.28318024)(39.94514412,355.28318817)
\curveto(40.03514054,355.28318024)(40.12014045,355.28818023)(40.20014412,355.29818817)
}
}
{
\newrgbcolor{curcolor}{0 0 0}
\pscustom[linestyle=none,fillstyle=solid,fillcolor=curcolor]
{
\newpath
\moveto(45.70678475,355.07318817)
\lineto(46.83178475,355.07318817)
\curveto(46.94178231,355.07318045)(47.04178221,355.06818045)(47.13178475,355.05818817)
\curveto(47.22178203,355.04818047)(47.28678197,355.01318051)(47.32678475,354.95318817)
\curveto(47.37678188,354.89318063)(47.40678185,354.80818071)(47.41678475,354.69818817)
\curveto(47.42678183,354.59818092)(47.43178182,354.49318103)(47.43178475,354.38318817)
\lineto(47.43178475,353.33318817)
\lineto(47.43178475,351.09818817)
\curveto(47.43178182,350.73818478)(47.44678181,350.39818512)(47.47678475,350.07818817)
\curveto(47.50678175,349.75818576)(47.59678166,349.49318603)(47.74678475,349.28318817)
\curveto(47.88678137,349.07318645)(48.11178114,348.9231866)(48.42178475,348.83318817)
\curveto(48.47178078,348.8231867)(48.51178074,348.8181867)(48.54178475,348.81818817)
\curveto(48.58178067,348.8181867)(48.62678063,348.81318671)(48.67678475,348.80318817)
\curveto(48.72678053,348.79318673)(48.78178047,348.78818673)(48.84178475,348.78818817)
\curveto(48.90178035,348.78818673)(48.94678031,348.79318673)(48.97678475,348.80318817)
\curveto(49.02678023,348.8231867)(49.06678019,348.82818669)(49.09678475,348.81818817)
\curveto(49.13678012,348.80818671)(49.17678008,348.81318671)(49.21678475,348.83318817)
\curveto(49.42677983,348.88318664)(49.59177966,348.94818657)(49.71178475,349.02818817)
\curveto(49.89177936,349.13818638)(50.03177922,349.27818624)(50.13178475,349.44818817)
\curveto(50.24177901,349.62818589)(50.31677894,349.8231857)(50.35678475,350.03318817)
\curveto(50.40677885,350.25318527)(50.43677882,350.49318503)(50.44678475,350.75318817)
\curveto(50.4567788,351.0231845)(50.46177879,351.30318422)(50.46178475,351.59318817)
\lineto(50.46178475,353.40818817)
\lineto(50.46178475,354.38318817)
\lineto(50.46178475,354.65318817)
\curveto(50.46177879,354.75318077)(50.48177877,354.83318069)(50.52178475,354.89318817)
\curveto(50.57177868,354.98318054)(50.64677861,355.03318049)(50.74678475,355.04318817)
\curveto(50.84677841,355.06318046)(50.96677829,355.07318045)(51.10678475,355.07318817)
\lineto(51.90178475,355.07318817)
\lineto(52.18678475,355.07318817)
\curveto(52.27677698,355.07318045)(52.3517769,355.05318047)(52.41178475,355.01318817)
\curveto(52.49177676,354.96318056)(52.53677672,354.88818063)(52.54678475,354.78818817)
\curveto(52.5567767,354.68818083)(52.56177669,354.57318095)(52.56178475,354.44318817)
\lineto(52.56178475,353.30318817)
\lineto(52.56178475,349.08818817)
\lineto(52.56178475,348.02318817)
\lineto(52.56178475,347.72318817)
\curveto(52.56177669,347.6231879)(52.54177671,347.54818797)(52.50178475,347.49818817)
\curveto(52.4517768,347.4181881)(52.37677688,347.37318815)(52.27678475,347.36318817)
\curveto(52.17677708,347.35318817)(52.07177718,347.34818817)(51.96178475,347.34818817)
\lineto(51.15178475,347.34818817)
\curveto(51.04177821,347.34818817)(50.94177831,347.35318817)(50.85178475,347.36318817)
\curveto(50.77177848,347.37318815)(50.70677855,347.41318811)(50.65678475,347.48318817)
\curveto(50.63677862,347.51318801)(50.61677864,347.55818796)(50.59678475,347.61818817)
\curveto(50.58677867,347.67818784)(50.57177868,347.73818778)(50.55178475,347.79818817)
\curveto(50.54177871,347.85818766)(50.52677873,347.91318761)(50.50678475,347.96318817)
\curveto(50.48677877,348.01318751)(50.4567788,348.04318748)(50.41678475,348.05318817)
\curveto(50.39677886,348.07318745)(50.37177888,348.07818744)(50.34178475,348.06818817)
\curveto(50.31177894,348.05818746)(50.28677897,348.04818747)(50.26678475,348.03818817)
\curveto(50.19677906,347.99818752)(50.13677912,347.95318757)(50.08678475,347.90318817)
\curveto(50.03677922,347.85318767)(49.98177927,347.80818771)(49.92178475,347.76818817)
\curveto(49.88177937,347.73818778)(49.84177941,347.70318782)(49.80178475,347.66318817)
\curveto(49.77177948,347.63318789)(49.73177952,347.60318792)(49.68178475,347.57318817)
\curveto(49.4517798,347.43318809)(49.18178007,347.3231882)(48.87178475,347.24318817)
\curveto(48.80178045,347.2231883)(48.73178052,347.21318831)(48.66178475,347.21318817)
\curveto(48.59178066,347.20318832)(48.51678074,347.18818833)(48.43678475,347.16818817)
\curveto(48.39678086,347.15818836)(48.3517809,347.15818836)(48.30178475,347.16818817)
\curveto(48.26178099,347.16818835)(48.22178103,347.16318836)(48.18178475,347.15318817)
\curveto(48.1517811,347.14318838)(48.08678117,347.14318838)(47.98678475,347.15318817)
\curveto(47.89678136,347.15318837)(47.83678142,347.15818836)(47.80678475,347.16818817)
\curveto(47.7567815,347.16818835)(47.70678155,347.17318835)(47.65678475,347.18318817)
\lineto(47.50678475,347.18318817)
\curveto(47.38678187,347.21318831)(47.27178198,347.23818828)(47.16178475,347.25818817)
\curveto(47.0517822,347.27818824)(46.94178231,347.30818821)(46.83178475,347.34818817)
\curveto(46.78178247,347.36818815)(46.73678252,347.38318814)(46.69678475,347.39318817)
\curveto(46.66678259,347.41318811)(46.62678263,347.43318809)(46.57678475,347.45318817)
\curveto(46.22678303,347.64318788)(45.94678331,347.90818761)(45.73678475,348.24818817)
\curveto(45.60678365,348.45818706)(45.51178374,348.70818681)(45.45178475,348.99818817)
\curveto(45.39178386,349.29818622)(45.3517839,349.61318591)(45.33178475,349.94318817)
\curveto(45.32178393,350.28318524)(45.31678394,350.62818489)(45.31678475,350.97818817)
\curveto(45.32678393,351.33818418)(45.33178392,351.69318383)(45.33178475,352.04318817)
\lineto(45.33178475,354.08318817)
\curveto(45.33178392,354.21318131)(45.32678393,354.36318116)(45.31678475,354.53318817)
\curveto(45.31678394,354.71318081)(45.34178391,354.84318068)(45.39178475,354.92318817)
\curveto(45.42178383,354.97318055)(45.48178377,355.0181805)(45.57178475,355.05818817)
\curveto(45.63178362,355.05818046)(45.67678358,355.06318046)(45.70678475,355.07318817)
}
}
{
\newrgbcolor{curcolor}{0 0 0}
\pscustom[linestyle=none,fillstyle=solid,fillcolor=curcolor]
{
\newpath
\moveto(61.24303475,347.94818817)
\curveto(61.2630269,347.83818768)(61.27302689,347.72818779)(61.27303475,347.61818817)
\curveto(61.28302688,347.50818801)(61.23302693,347.43318809)(61.12303475,347.39318817)
\curveto(61.0630271,347.36318816)(60.99302717,347.34818817)(60.91303475,347.34818817)
\lineto(60.67303475,347.34818817)
\lineto(59.86303475,347.34818817)
\lineto(59.59303475,347.34818817)
\curveto(59.51302865,347.35818816)(59.44802871,347.38318814)(59.39803475,347.42318817)
\curveto(59.32802883,347.46318806)(59.27302889,347.518188)(59.23303475,347.58818817)
\curveto(59.20302896,347.66818785)(59.158029,347.73318779)(59.09803475,347.78318817)
\curveto(59.07802908,347.80318772)(59.05302911,347.8181877)(59.02303475,347.82818817)
\curveto(58.99302917,347.84818767)(58.95302921,347.85318767)(58.90303475,347.84318817)
\curveto(58.85302931,347.8231877)(58.80302936,347.79818772)(58.75303475,347.76818817)
\curveto(58.71302945,347.73818778)(58.66802949,347.71318781)(58.61803475,347.69318817)
\curveto(58.56802959,347.65318787)(58.51302965,347.6181879)(58.45303475,347.58818817)
\lineto(58.27303475,347.49818817)
\curveto(58.14303002,347.43818808)(58.00803015,347.38818813)(57.86803475,347.34818817)
\curveto(57.72803043,347.3181882)(57.58303058,347.28318824)(57.43303475,347.24318817)
\curveto(57.3630308,347.2231883)(57.29303087,347.21318831)(57.22303475,347.21318817)
\curveto(57.163031,347.20318832)(57.09803106,347.19318833)(57.02803475,347.18318817)
\lineto(56.93803475,347.18318817)
\curveto(56.90803125,347.17318835)(56.87803128,347.16818835)(56.84803475,347.16818817)
\lineto(56.68303475,347.16818817)
\curveto(56.58303158,347.14818837)(56.48303168,347.14818837)(56.38303475,347.16818817)
\lineto(56.24803475,347.16818817)
\curveto(56.17803198,347.18818833)(56.10803205,347.19818832)(56.03803475,347.19818817)
\curveto(55.97803218,347.18818833)(55.91803224,347.19318833)(55.85803475,347.21318817)
\curveto(55.7580324,347.23318829)(55.6630325,347.25318827)(55.57303475,347.27318817)
\curveto(55.48303268,347.28318824)(55.39803276,347.30818821)(55.31803475,347.34818817)
\curveto(55.02803313,347.45818806)(54.77803338,347.59818792)(54.56803475,347.76818817)
\curveto(54.36803379,347.94818757)(54.20803395,348.18318734)(54.08803475,348.47318817)
\curveto(54.0580341,348.54318698)(54.02803413,348.6181869)(53.99803475,348.69818817)
\curveto(53.97803418,348.77818674)(53.9580342,348.86318666)(53.93803475,348.95318817)
\curveto(53.91803424,349.00318652)(53.90803425,349.05318647)(53.90803475,349.10318817)
\curveto(53.91803424,349.15318637)(53.91803424,349.20318632)(53.90803475,349.25318817)
\curveto(53.89803426,349.28318624)(53.88803427,349.34318618)(53.87803475,349.43318817)
\curveto(53.87803428,349.53318599)(53.88303428,349.60318592)(53.89303475,349.64318817)
\curveto(53.91303425,349.74318578)(53.92303424,349.82818569)(53.92303475,349.89818817)
\lineto(54.01303475,350.22818817)
\curveto(54.04303412,350.34818517)(54.08303408,350.45318507)(54.13303475,350.54318817)
\curveto(54.30303386,350.83318469)(54.49803366,351.05318447)(54.71803475,351.20318817)
\curveto(54.93803322,351.35318417)(55.21803294,351.48318404)(55.55803475,351.59318817)
\curveto(55.68803247,351.64318388)(55.82303234,351.67818384)(55.96303475,351.69818817)
\curveto(56.10303206,351.7181838)(56.24303192,351.74318378)(56.38303475,351.77318817)
\curveto(56.4630317,351.79318373)(56.54803161,351.80318372)(56.63803475,351.80318817)
\curveto(56.72803143,351.81318371)(56.81803134,351.82818369)(56.90803475,351.84818817)
\curveto(56.97803118,351.86818365)(57.04803111,351.87318365)(57.11803475,351.86318817)
\curveto(57.18803097,351.86318366)(57.2630309,351.87318365)(57.34303475,351.89318817)
\curveto(57.41303075,351.91318361)(57.48303068,351.9231836)(57.55303475,351.92318817)
\curveto(57.62303054,351.9231836)(57.69803046,351.93318359)(57.77803475,351.95318817)
\curveto(57.98803017,352.00318352)(58.17802998,352.04318348)(58.34803475,352.07318817)
\curveto(58.52802963,352.11318341)(58.68802947,352.20318332)(58.82803475,352.34318817)
\curveto(58.91802924,352.43318309)(58.97802918,352.53318299)(59.00803475,352.64318817)
\curveto(59.01802914,352.67318285)(59.01802914,352.69818282)(59.00803475,352.71818817)
\curveto(59.00802915,352.73818278)(59.01302915,352.75818276)(59.02303475,352.77818817)
\curveto(59.03302913,352.79818272)(59.03802912,352.82818269)(59.03803475,352.86818817)
\lineto(59.03803475,352.95818817)
\lineto(59.00803475,353.07818817)
\curveto(59.00802915,353.1181824)(59.00302916,353.15318237)(58.99303475,353.18318817)
\curveto(58.89302927,353.48318204)(58.68302948,353.68818183)(58.36303475,353.79818817)
\curveto(58.27302989,353.82818169)(58.16303,353.84818167)(58.03303475,353.85818817)
\curveto(57.91303025,353.87818164)(57.78803037,353.88318164)(57.65803475,353.87318817)
\curveto(57.52803063,353.87318165)(57.40303076,353.86318166)(57.28303475,353.84318817)
\curveto(57.163031,353.8231817)(57.0580311,353.79818172)(56.96803475,353.76818817)
\curveto(56.90803125,353.74818177)(56.84803131,353.7181818)(56.78803475,353.67818817)
\curveto(56.73803142,353.64818187)(56.68803147,353.61318191)(56.63803475,353.57318817)
\curveto(56.58803157,353.53318199)(56.53303163,353.47818204)(56.47303475,353.40818817)
\curveto(56.42303174,353.33818218)(56.38803177,353.27318225)(56.36803475,353.21318817)
\curveto(56.31803184,353.11318241)(56.27303189,353.0181825)(56.23303475,352.92818817)
\curveto(56.20303196,352.83818268)(56.13303203,352.77818274)(56.02303475,352.74818817)
\curveto(55.94303222,352.72818279)(55.8580323,352.7181828)(55.76803475,352.71818817)
\lineto(55.49803475,352.71818817)
\lineto(54.92803475,352.71818817)
\curveto(54.87803328,352.7181828)(54.82803333,352.71318281)(54.77803475,352.70318817)
\curveto(54.72803343,352.70318282)(54.68303348,352.70818281)(54.64303475,352.71818817)
\lineto(54.50803475,352.71818817)
\curveto(54.48803367,352.72818279)(54.4630337,352.73318279)(54.43303475,352.73318817)
\curveto(54.40303376,352.73318279)(54.37803378,352.74318278)(54.35803475,352.76318817)
\curveto(54.27803388,352.78318274)(54.22303394,352.84818267)(54.19303475,352.95818817)
\curveto(54.18303398,353.00818251)(54.18303398,353.05818246)(54.19303475,353.10818817)
\curveto(54.20303396,353.15818236)(54.21303395,353.20318232)(54.22303475,353.24318817)
\curveto(54.25303391,353.35318217)(54.28303388,353.45318207)(54.31303475,353.54318817)
\curveto(54.35303381,353.64318188)(54.39803376,353.73318179)(54.44803475,353.81318817)
\lineto(54.53803475,353.96318817)
\lineto(54.62803475,354.11318817)
\curveto(54.70803345,354.2231813)(54.80803335,354.32818119)(54.92803475,354.42818817)
\curveto(54.94803321,354.43818108)(54.97803318,354.46318106)(55.01803475,354.50318817)
\curveto(55.06803309,354.54318098)(55.11303305,354.57818094)(55.15303475,354.60818817)
\curveto(55.19303297,354.63818088)(55.23803292,354.66818085)(55.28803475,354.69818817)
\curveto(55.4580327,354.80818071)(55.63803252,354.89318063)(55.82803475,354.95318817)
\curveto(56.01803214,355.0231805)(56.21303195,355.08818043)(56.41303475,355.14818817)
\curveto(56.53303163,355.17818034)(56.6580315,355.19818032)(56.78803475,355.20818817)
\curveto(56.91803124,355.2181803)(57.04803111,355.23818028)(57.17803475,355.26818817)
\curveto(57.21803094,355.27818024)(57.27803088,355.27818024)(57.35803475,355.26818817)
\curveto(57.44803071,355.25818026)(57.50303066,355.26318026)(57.52303475,355.28318817)
\curveto(57.93303023,355.29318023)(58.32302984,355.27818024)(58.69303475,355.23818817)
\curveto(59.07302909,355.19818032)(59.41302875,355.1231804)(59.71303475,355.01318817)
\curveto(60.02302814,354.90318062)(60.28802787,354.75318077)(60.50803475,354.56318817)
\curveto(60.72802743,354.38318114)(60.89802726,354.14818137)(61.01803475,353.85818817)
\curveto(61.08802707,353.68818183)(61.12802703,353.49318203)(61.13803475,353.27318817)
\curveto(61.14802701,353.05318247)(61.15302701,352.82818269)(61.15303475,352.59818817)
\lineto(61.15303475,349.25318817)
\lineto(61.15303475,348.66818817)
\curveto(61.15302701,348.47818704)(61.17302699,348.30318722)(61.21303475,348.14318817)
\curveto(61.22302694,348.11318741)(61.22802693,348.07818744)(61.22803475,348.03818817)
\curveto(61.22802693,348.00818751)(61.23302693,347.97818754)(61.24303475,347.94818817)
\moveto(59.03803475,350.25818817)
\curveto(59.04802911,350.30818521)(59.05302911,350.36318516)(59.05303475,350.42318817)
\curveto(59.05302911,350.49318503)(59.04802911,350.55318497)(59.03803475,350.60318817)
\curveto(59.01802914,350.66318486)(59.00802915,350.7181848)(59.00803475,350.76818817)
\curveto(59.00802915,350.8181847)(58.98802917,350.85818466)(58.94803475,350.88818817)
\curveto(58.89802926,350.92818459)(58.82302934,350.94818457)(58.72303475,350.94818817)
\curveto(58.68302948,350.93818458)(58.64802951,350.92818459)(58.61803475,350.91818817)
\curveto(58.58802957,350.9181846)(58.55302961,350.91318461)(58.51303475,350.90318817)
\curveto(58.44302972,350.88318464)(58.36802979,350.86818465)(58.28803475,350.85818817)
\curveto(58.20802995,350.84818467)(58.12803003,350.83318469)(58.04803475,350.81318817)
\curveto(58.01803014,350.80318472)(57.97303019,350.79818472)(57.91303475,350.79818817)
\curveto(57.78303038,350.76818475)(57.65303051,350.74818477)(57.52303475,350.73818817)
\curveto(57.39303077,350.72818479)(57.26803089,350.70318482)(57.14803475,350.66318817)
\curveto(57.06803109,350.64318488)(56.99303117,350.6231849)(56.92303475,350.60318817)
\curveto(56.85303131,350.59318493)(56.78303138,350.57318495)(56.71303475,350.54318817)
\curveto(56.50303166,350.45318507)(56.32303184,350.3181852)(56.17303475,350.13818817)
\curveto(56.03303213,349.95818556)(55.98303218,349.70818581)(56.02303475,349.38818817)
\curveto(56.04303212,349.2181863)(56.09803206,349.07818644)(56.18803475,348.96818817)
\curveto(56.2580319,348.85818666)(56.3630318,348.76818675)(56.50303475,348.69818817)
\curveto(56.64303152,348.63818688)(56.79303137,348.59318693)(56.95303475,348.56318817)
\curveto(57.12303104,348.53318699)(57.29803086,348.523187)(57.47803475,348.53318817)
\curveto(57.66803049,348.55318697)(57.84303032,348.58818693)(58.00303475,348.63818817)
\curveto(58.2630299,348.7181868)(58.46802969,348.84318668)(58.61803475,349.01318817)
\curveto(58.76802939,349.19318633)(58.88302928,349.41318611)(58.96303475,349.67318817)
\curveto(58.98302918,349.74318578)(58.99302917,349.81318571)(58.99303475,349.88318817)
\curveto(59.00302916,349.96318556)(59.01802914,350.04318548)(59.03803475,350.12318817)
\lineto(59.03803475,350.25818817)
}
}
{
\newrgbcolor{curcolor}{0 0 0}
\pscustom[linestyle=none,fillstyle=solid,fillcolor=curcolor]
{
\newpath
\moveto(67.231316,355.28318817)
\curveto(67.34131068,355.28318024)(67.43631059,355.27318025)(67.516316,355.25318817)
\curveto(67.60631042,355.23318029)(67.67631035,355.18818033)(67.726316,355.11818817)
\curveto(67.78631024,355.03818048)(67.81631021,354.89818062)(67.816316,354.69818817)
\lineto(67.816316,354.18818817)
\lineto(67.816316,353.81318817)
\curveto(67.8263102,353.67318185)(67.81131021,353.56318196)(67.771316,353.48318817)
\curveto(67.73131029,353.41318211)(67.67131035,353.36818215)(67.591316,353.34818817)
\curveto(67.5213105,353.32818219)(67.43631059,353.3181822)(67.336316,353.31818817)
\curveto(67.24631078,353.3181822)(67.14631088,353.3231822)(67.036316,353.33318817)
\curveto(66.93631109,353.34318218)(66.84131118,353.33818218)(66.751316,353.31818817)
\curveto(66.68131134,353.29818222)(66.61131141,353.28318224)(66.541316,353.27318817)
\curveto(66.47131155,353.27318225)(66.40631162,353.26318226)(66.346316,353.24318817)
\curveto(66.18631184,353.19318233)(66.026312,353.1181824)(65.866316,353.01818817)
\curveto(65.70631232,352.92818259)(65.58131244,352.8231827)(65.491316,352.70318817)
\curveto(65.44131258,352.6231829)(65.38631264,352.53818298)(65.326316,352.44818817)
\curveto(65.27631275,352.36818315)(65.2263128,352.28318324)(65.176316,352.19318817)
\curveto(65.14631288,352.11318341)(65.11631291,352.02818349)(65.086316,351.93818817)
\lineto(65.026316,351.69818817)
\curveto(65.00631302,351.62818389)(64.99631303,351.55318397)(64.996316,351.47318817)
\curveto(64.99631303,351.40318412)(64.98631304,351.33318419)(64.966316,351.26318817)
\curveto(64.95631307,351.2231843)(64.95131307,351.18318434)(64.951316,351.14318817)
\curveto(64.96131306,351.11318441)(64.96131306,351.08318444)(64.951316,351.05318817)
\lineto(64.951316,350.81318817)
\curveto(64.93131309,350.74318478)(64.9263131,350.66318486)(64.936316,350.57318817)
\curveto(64.94631308,350.49318503)(64.95131307,350.41318511)(64.951316,350.33318817)
\lineto(64.951316,349.37318817)
\lineto(64.951316,348.09818817)
\curveto(64.95131307,347.96818755)(64.94631308,347.84818767)(64.936316,347.73818817)
\curveto(64.9263131,347.62818789)(64.89631313,347.53818798)(64.846316,347.46818817)
\curveto(64.8263132,347.43818808)(64.79131323,347.41318811)(64.741316,347.39318817)
\curveto(64.70131332,347.38318814)(64.65631337,347.37318815)(64.606316,347.36318817)
\lineto(64.531316,347.36318817)
\curveto(64.48131354,347.35318817)(64.4263136,347.34818817)(64.366316,347.34818817)
\lineto(64.201316,347.34818817)
\lineto(63.556316,347.34818817)
\curveto(63.49631453,347.35818816)(63.43131459,347.36318816)(63.361316,347.36318817)
\lineto(63.166316,347.36318817)
\curveto(63.11631491,347.38318814)(63.06631496,347.39818812)(63.016316,347.40818817)
\curveto(62.96631506,347.42818809)(62.93131509,347.46318806)(62.911316,347.51318817)
\curveto(62.87131515,347.56318796)(62.84631518,347.63318789)(62.836316,347.72318817)
\lineto(62.836316,348.02318817)
\lineto(62.836316,349.04318817)
\lineto(62.836316,353.27318817)
\lineto(62.836316,354.38318817)
\lineto(62.836316,354.66818817)
\curveto(62.83631519,354.76818075)(62.85631517,354.84818067)(62.896316,354.90818817)
\curveto(62.94631508,354.98818053)(63.021315,355.03818048)(63.121316,355.05818817)
\curveto(63.2213148,355.07818044)(63.34131468,355.08818043)(63.481316,355.08818817)
\lineto(64.246316,355.08818817)
\curveto(64.36631366,355.08818043)(64.47131355,355.07818044)(64.561316,355.05818817)
\curveto(64.65131337,355.04818047)(64.7213133,355.00318052)(64.771316,354.92318817)
\curveto(64.80131322,354.87318065)(64.81631321,354.80318072)(64.816316,354.71318817)
\lineto(64.846316,354.44318817)
\curveto(64.85631317,354.36318116)(64.87131315,354.28818123)(64.891316,354.21818817)
\curveto(64.9213131,354.14818137)(64.97131305,354.11318141)(65.041316,354.11318817)
\curveto(65.06131296,354.13318139)(65.08131294,354.14318138)(65.101316,354.14318817)
\curveto(65.1213129,354.14318138)(65.14131288,354.15318137)(65.161316,354.17318817)
\curveto(65.2213128,354.2231813)(65.27131275,354.27818124)(65.311316,354.33818817)
\curveto(65.36131266,354.40818111)(65.4213126,354.46818105)(65.491316,354.51818817)
\curveto(65.53131249,354.54818097)(65.56631246,354.57818094)(65.596316,354.60818817)
\curveto(65.6263124,354.64818087)(65.66131236,354.68318084)(65.701316,354.71318817)
\lineto(65.971316,354.89318817)
\curveto(66.07131195,354.95318057)(66.17131185,355.00818051)(66.271316,355.05818817)
\curveto(66.37131165,355.09818042)(66.47131155,355.13318039)(66.571316,355.16318817)
\lineto(66.901316,355.25318817)
\curveto(66.93131109,355.26318026)(66.98631104,355.26318026)(67.066316,355.25318817)
\curveto(67.15631087,355.25318027)(67.21131081,355.26318026)(67.231316,355.28318817)
}
}
{
\newrgbcolor{curcolor}{0 0 0}
\pscustom[linestyle=none,fillstyle=solid,fillcolor=curcolor]
{
\newpath
\moveto(70.73639412,357.93818817)
\curveto(70.80639117,357.85817766)(70.84139114,357.73817778)(70.84139412,357.57818817)
\lineto(70.84139412,357.11318817)
\lineto(70.84139412,356.70818817)
\curveto(70.84139114,356.56817895)(70.80639117,356.47317905)(70.73639412,356.42318817)
\curveto(70.6763913,356.37317915)(70.59639138,356.34317918)(70.49639412,356.33318817)
\curveto(70.40639157,356.3231792)(70.30639167,356.3181792)(70.19639412,356.31818817)
\lineto(69.35639412,356.31818817)
\curveto(69.24639273,356.3181792)(69.14639283,356.3231792)(69.05639412,356.33318817)
\curveto(68.976393,356.34317918)(68.90639307,356.37317915)(68.84639412,356.42318817)
\curveto(68.80639317,356.45317907)(68.7763932,356.50817901)(68.75639412,356.58818817)
\curveto(68.74639323,356.67817884)(68.73639324,356.77317875)(68.72639412,356.87318817)
\lineto(68.72639412,357.20318817)
\curveto(68.73639324,357.31317821)(68.74139324,357.40817811)(68.74139412,357.48818817)
\lineto(68.74139412,357.69818817)
\curveto(68.75139323,357.76817775)(68.77139321,357.82817769)(68.80139412,357.87818817)
\curveto(68.82139316,357.9181776)(68.84639313,357.94817757)(68.87639412,357.96818817)
\lineto(68.99639412,358.02818817)
\curveto(69.01639296,358.02817749)(69.04139294,358.02817749)(69.07139412,358.02818817)
\curveto(69.10139288,358.03817748)(69.12639285,358.04317748)(69.14639412,358.04318817)
\lineto(70.24139412,358.04318817)
\curveto(70.34139164,358.04317748)(70.43639154,358.03817748)(70.52639412,358.02818817)
\curveto(70.61639136,358.0181775)(70.68639129,357.98817753)(70.73639412,357.93818817)
\moveto(70.84139412,348.17318817)
\curveto(70.84139114,347.97318755)(70.83639114,347.80318772)(70.82639412,347.66318817)
\curveto(70.81639116,347.523188)(70.72639125,347.42818809)(70.55639412,347.37818817)
\curveto(70.49639148,347.35818816)(70.43139155,347.34818817)(70.36139412,347.34818817)
\curveto(70.29139169,347.35818816)(70.21639176,347.36318816)(70.13639412,347.36318817)
\lineto(69.29639412,347.36318817)
\curveto(69.20639277,347.36318816)(69.11639286,347.36818815)(69.02639412,347.37818817)
\curveto(68.94639303,347.38818813)(68.88639309,347.4181881)(68.84639412,347.46818817)
\curveto(68.78639319,347.53818798)(68.75139323,347.6231879)(68.74139412,347.72318817)
\lineto(68.74139412,348.06818817)
\lineto(68.74139412,354.39818817)
\lineto(68.74139412,354.69818817)
\curveto(68.74139324,354.79818072)(68.76139322,354.87818064)(68.80139412,354.93818817)
\curveto(68.86139312,355.00818051)(68.94639303,355.05318047)(69.05639412,355.07318817)
\curveto(69.0763929,355.08318044)(69.10139288,355.08318044)(69.13139412,355.07318817)
\curveto(69.17139281,355.07318045)(69.20139278,355.07818044)(69.22139412,355.08818817)
\lineto(69.97139412,355.08818817)
\lineto(70.16639412,355.08818817)
\curveto(70.24639173,355.09818042)(70.31139167,355.09818042)(70.36139412,355.08818817)
\lineto(70.48139412,355.08818817)
\curveto(70.54139144,355.06818045)(70.59639138,355.05318047)(70.64639412,355.04318817)
\curveto(70.69639128,355.03318049)(70.73639124,355.00318052)(70.76639412,354.95318817)
\curveto(70.80639117,354.90318062)(70.82639115,354.83318069)(70.82639412,354.74318817)
\curveto(70.83639114,354.65318087)(70.84139114,354.55818096)(70.84139412,354.45818817)
\lineto(70.84139412,348.17318817)
}
}
{
\newrgbcolor{curcolor}{0 0 0}
\pscustom[linestyle=none,fillstyle=solid,fillcolor=curcolor]
{
\newpath
\moveto(80.27358162,351.53318817)
\curveto(80.29357305,351.47318405)(80.30357304,351.38818413)(80.30358162,351.27818817)
\curveto(80.30357304,351.16818435)(80.29357305,351.08318444)(80.27358162,351.02318817)
\lineto(80.27358162,350.87318817)
\curveto(80.25357309,350.79318473)(80.2435731,350.71318481)(80.24358162,350.63318817)
\curveto(80.25357309,350.55318497)(80.2485731,350.47318505)(80.22858162,350.39318817)
\curveto(80.20857314,350.3231852)(80.19357315,350.25818526)(80.18358162,350.19818817)
\curveto(80.17357317,350.13818538)(80.16357318,350.07318545)(80.15358162,350.00318817)
\curveto(80.11357323,349.89318563)(80.07857327,349.77818574)(80.04858162,349.65818817)
\curveto(80.01857333,349.54818597)(79.97857337,349.44318608)(79.92858162,349.34318817)
\curveto(79.71857363,348.86318666)(79.4435739,348.47318705)(79.10358162,348.17318817)
\curveto(78.76357458,347.87318765)(78.35357499,347.6231879)(77.87358162,347.42318817)
\curveto(77.75357559,347.37318815)(77.62857572,347.33818818)(77.49858162,347.31818817)
\curveto(77.37857597,347.28818823)(77.25357609,347.25818826)(77.12358162,347.22818817)
\curveto(77.07357627,347.20818831)(77.01857633,347.19818832)(76.95858162,347.19818817)
\curveto(76.89857645,347.19818832)(76.8435765,347.19318833)(76.79358162,347.18318817)
\lineto(76.68858162,347.18318817)
\curveto(76.65857669,347.17318835)(76.62857672,347.16818835)(76.59858162,347.16818817)
\curveto(76.5485768,347.15818836)(76.46857688,347.15318837)(76.35858162,347.15318817)
\curveto(76.2485771,347.14318838)(76.16357718,347.14818837)(76.10358162,347.16818817)
\lineto(75.95358162,347.16818817)
\curveto(75.90357744,347.17818834)(75.8485775,347.18318834)(75.78858162,347.18318817)
\curveto(75.73857761,347.17318835)(75.68857766,347.17818834)(75.63858162,347.19818817)
\curveto(75.59857775,347.20818831)(75.55857779,347.21318831)(75.51858162,347.21318817)
\curveto(75.48857786,347.21318831)(75.4485779,347.2181883)(75.39858162,347.22818817)
\curveto(75.29857805,347.25818826)(75.19857815,347.28318824)(75.09858162,347.30318817)
\curveto(74.99857835,347.3231882)(74.90357844,347.35318817)(74.81358162,347.39318817)
\curveto(74.69357865,347.43318809)(74.57857877,347.47318805)(74.46858162,347.51318817)
\curveto(74.36857898,347.55318797)(74.26357908,347.60318792)(74.15358162,347.66318817)
\curveto(73.80357954,347.87318765)(73.50357984,348.1181874)(73.25358162,348.39818817)
\curveto(73.00358034,348.67818684)(72.79358055,349.01318651)(72.62358162,349.40318817)
\curveto(72.57358077,349.49318603)(72.53358081,349.58818593)(72.50358162,349.68818817)
\curveto(72.48358086,349.78818573)(72.45858089,349.89318563)(72.42858162,350.00318817)
\curveto(72.40858094,350.05318547)(72.39858095,350.09818542)(72.39858162,350.13818817)
\curveto(72.39858095,350.17818534)(72.38858096,350.2231853)(72.36858162,350.27318817)
\curveto(72.348581,350.35318517)(72.33858101,350.43318509)(72.33858162,350.51318817)
\curveto(72.33858101,350.60318492)(72.32858102,350.68818483)(72.30858162,350.76818817)
\curveto(72.29858105,350.8181847)(72.29358105,350.86318466)(72.29358162,350.90318817)
\lineto(72.29358162,351.03818817)
\curveto(72.27358107,351.09818442)(72.26358108,351.18318434)(72.26358162,351.29318817)
\curveto(72.27358107,351.40318412)(72.28858106,351.48818403)(72.30858162,351.54818817)
\lineto(72.30858162,351.65318817)
\curveto(72.31858103,351.70318382)(72.31858103,351.75318377)(72.30858162,351.80318817)
\curveto(72.30858104,351.86318366)(72.31858103,351.9181836)(72.33858162,351.96818817)
\curveto(72.348581,352.0181835)(72.35358099,352.06318346)(72.35358162,352.10318817)
\curveto(72.35358099,352.15318337)(72.36358098,352.20318332)(72.38358162,352.25318817)
\curveto(72.42358092,352.38318314)(72.45858089,352.50818301)(72.48858162,352.62818817)
\curveto(72.51858083,352.75818276)(72.55858079,352.88318264)(72.60858162,353.00318817)
\curveto(72.78858056,353.41318211)(73.00358034,353.75318177)(73.25358162,354.02318817)
\curveto(73.50357984,354.30318122)(73.80857954,354.55818096)(74.16858162,354.78818817)
\curveto(74.26857908,354.83818068)(74.37357897,354.88318064)(74.48358162,354.92318817)
\curveto(74.59357875,354.96318056)(74.70357864,355.00818051)(74.81358162,355.05818817)
\curveto(74.9435784,355.10818041)(75.07857827,355.14318038)(75.21858162,355.16318817)
\curveto(75.35857799,355.18318034)(75.50357784,355.21318031)(75.65358162,355.25318817)
\curveto(75.73357761,355.26318026)(75.80857754,355.26818025)(75.87858162,355.26818817)
\curveto(75.9485774,355.26818025)(76.01857733,355.27318025)(76.08858162,355.28318817)
\curveto(76.66857668,355.29318023)(77.16857618,355.23318029)(77.58858162,355.10318817)
\curveto(78.01857533,354.97318055)(78.39857495,354.79318073)(78.72858162,354.56318817)
\curveto(78.83857451,354.48318104)(78.9485744,354.39318113)(79.05858162,354.29318817)
\curveto(79.17857417,354.20318132)(79.27857407,354.10318142)(79.35858162,353.99318817)
\curveto(79.43857391,353.89318163)(79.50857384,353.79318173)(79.56858162,353.69318817)
\curveto(79.63857371,353.59318193)(79.70857364,353.48818203)(79.77858162,353.37818817)
\curveto(79.8485735,353.26818225)(79.90357344,353.14818237)(79.94358162,353.01818817)
\curveto(79.98357336,352.89818262)(80.02857332,352.76818275)(80.07858162,352.62818817)
\curveto(80.10857324,352.54818297)(80.13357321,352.46318306)(80.15358162,352.37318817)
\lineto(80.21358162,352.10318817)
\curveto(80.22357312,352.06318346)(80.22857312,352.0231835)(80.22858162,351.98318817)
\curveto(80.22857312,351.94318358)(80.23357311,351.90318362)(80.24358162,351.86318817)
\curveto(80.26357308,351.81318371)(80.26857308,351.75818376)(80.25858162,351.69818817)
\curveto(80.2485731,351.63818388)(80.25357309,351.58318394)(80.27358162,351.53318817)
\moveto(78.17358162,350.99318817)
\curveto(78.18357516,351.04318448)(78.18857516,351.11318441)(78.18858162,351.20318817)
\curveto(78.18857516,351.30318422)(78.18357516,351.37818414)(78.17358162,351.42818817)
\lineto(78.17358162,351.54818817)
\curveto(78.15357519,351.59818392)(78.1435752,351.65318387)(78.14358162,351.71318817)
\curveto(78.1435752,351.77318375)(78.13857521,351.82818369)(78.12858162,351.87818817)
\curveto(78.12857522,351.9181836)(78.12357522,351.94818357)(78.11358162,351.96818817)
\lineto(78.05358162,352.20818817)
\curveto(78.0435753,352.29818322)(78.02357532,352.38318314)(77.99358162,352.46318817)
\curveto(77.88357546,352.7231828)(77.75357559,352.94318258)(77.60358162,353.12318817)
\curveto(77.45357589,353.31318221)(77.25357609,353.46318206)(77.00358162,353.57318817)
\curveto(76.9435764,353.59318193)(76.88357646,353.60818191)(76.82358162,353.61818817)
\curveto(76.76357658,353.63818188)(76.69857665,353.65818186)(76.62858162,353.67818817)
\curveto(76.5485768,353.69818182)(76.46357688,353.70318182)(76.37358162,353.69318817)
\lineto(76.10358162,353.69318817)
\curveto(76.07357727,353.67318185)(76.03857731,353.66318186)(75.99858162,353.66318817)
\curveto(75.95857739,353.67318185)(75.92357742,353.67318185)(75.89358162,353.66318817)
\lineto(75.68358162,353.60318817)
\curveto(75.62357772,353.59318193)(75.56857778,353.57318195)(75.51858162,353.54318817)
\curveto(75.26857808,353.43318209)(75.06357828,353.27318225)(74.90358162,353.06318817)
\curveto(74.75357859,352.86318266)(74.63357871,352.62818289)(74.54358162,352.35818817)
\curveto(74.51357883,352.25818326)(74.48857886,352.15318337)(74.46858162,352.04318817)
\curveto(74.45857889,351.93318359)(74.4435789,351.8231837)(74.42358162,351.71318817)
\curveto(74.41357893,351.66318386)(74.40857894,351.61318391)(74.40858162,351.56318817)
\lineto(74.40858162,351.41318817)
\curveto(74.38857896,351.34318418)(74.37857897,351.23818428)(74.37858162,351.09818817)
\curveto(74.38857896,350.95818456)(74.40357894,350.85318467)(74.42358162,350.78318817)
\lineto(74.42358162,350.64818817)
\curveto(74.4435789,350.56818495)(74.45857889,350.48818503)(74.46858162,350.40818817)
\curveto(74.47857887,350.33818518)(74.49357885,350.26318526)(74.51358162,350.18318817)
\curveto(74.61357873,349.88318564)(74.71857863,349.63818588)(74.82858162,349.44818817)
\curveto(74.9485784,349.26818625)(75.13357821,349.10318642)(75.38358162,348.95318817)
\curveto(75.45357789,348.90318662)(75.52857782,348.86318666)(75.60858162,348.83318817)
\curveto(75.69857765,348.80318672)(75.78857756,348.77818674)(75.87858162,348.75818817)
\curveto(75.91857743,348.74818677)(75.95357739,348.74318678)(75.98358162,348.74318817)
\curveto(76.01357733,348.75318677)(76.0485773,348.75318677)(76.08858162,348.74318817)
\lineto(76.20858162,348.71318817)
\curveto(76.25857709,348.71318681)(76.30357704,348.7181868)(76.34358162,348.72818817)
\lineto(76.46358162,348.72818817)
\curveto(76.5435768,348.74818677)(76.62357672,348.76318676)(76.70358162,348.77318817)
\curveto(76.78357656,348.78318674)(76.85857649,348.80318672)(76.92858162,348.83318817)
\curveto(77.18857616,348.93318659)(77.39857595,349.06818645)(77.55858162,349.23818817)
\curveto(77.71857563,349.40818611)(77.85357549,349.6181859)(77.96358162,349.86818817)
\curveto(78.00357534,349.96818555)(78.03357531,350.06818545)(78.05358162,350.16818817)
\curveto(78.07357527,350.26818525)(78.09857525,350.37318515)(78.12858162,350.48318817)
\curveto(78.13857521,350.523185)(78.1435752,350.55818496)(78.14358162,350.58818817)
\curveto(78.1435752,350.62818489)(78.1485752,350.66818485)(78.15858162,350.70818817)
\lineto(78.15858162,350.84318817)
\curveto(78.15857519,350.89318463)(78.16357518,350.94318458)(78.17358162,350.99318817)
}
}
{
\newrgbcolor{curcolor}{0 0 0}
\pscustom[linestyle=none,fillstyle=solid,fillcolor=curcolor]
{
\newpath
\moveto(84.6435035,355.29818817)
\curveto(85.393499,355.3181802)(86.04349835,355.23318029)(86.5935035,355.04318817)
\curveto(87.15349724,354.86318066)(87.57849681,354.54818097)(87.8685035,354.09818817)
\curveto(87.93849645,353.98818153)(87.99849639,353.87318165)(88.0485035,353.75318817)
\curveto(88.10849628,353.64318188)(88.15849623,353.518182)(88.1985035,353.37818817)
\curveto(88.21849617,353.3181822)(88.22849616,353.25318227)(88.2285035,353.18318817)
\curveto(88.22849616,353.11318241)(88.21849617,353.05318247)(88.1985035,353.00318817)
\curveto(88.15849623,352.94318258)(88.10349629,352.90318262)(88.0335035,352.88318817)
\curveto(87.98349641,352.86318266)(87.92349647,352.85318267)(87.8535035,352.85318817)
\lineto(87.6435035,352.85318817)
\lineto(86.9835035,352.85318817)
\curveto(86.91349748,352.85318267)(86.84349755,352.84818267)(86.7735035,352.83818817)
\curveto(86.70349769,352.83818268)(86.63849775,352.84818267)(86.5785035,352.86818817)
\curveto(86.47849791,352.88818263)(86.40349799,352.92818259)(86.3535035,352.98818817)
\curveto(86.30349809,353.04818247)(86.25849813,353.10818241)(86.2185035,353.16818817)
\lineto(86.0985035,353.37818817)
\curveto(86.06849832,353.45818206)(86.01849837,353.523182)(85.9485035,353.57318817)
\curveto(85.84849854,353.65318187)(85.74849864,353.71318181)(85.6485035,353.75318817)
\curveto(85.55849883,353.79318173)(85.44349895,353.82818169)(85.3035035,353.85818817)
\curveto(85.23349916,353.87818164)(85.12849926,353.89318163)(84.9885035,353.90318817)
\curveto(84.85849953,353.91318161)(84.75849963,353.90818161)(84.6885035,353.88818817)
\lineto(84.5835035,353.88818817)
\lineto(84.4335035,353.85818817)
\curveto(84.3935,353.85818166)(84.34850004,353.85318167)(84.2985035,353.84318817)
\curveto(84.12850026,353.79318173)(83.9885004,353.7231818)(83.8785035,353.63318817)
\curveto(83.77850061,353.55318197)(83.70850068,353.42818209)(83.6685035,353.25818817)
\curveto(83.64850074,353.18818233)(83.64850074,353.1231824)(83.6685035,353.06318817)
\curveto(83.6885007,353.00318252)(83.70850068,352.95318257)(83.7285035,352.91318817)
\curveto(83.79850059,352.79318273)(83.87850051,352.69818282)(83.9685035,352.62818817)
\curveto(84.06850032,352.55818296)(84.18350021,352.49818302)(84.3135035,352.44818817)
\curveto(84.50349989,352.36818315)(84.70849968,352.29818322)(84.9285035,352.23818817)
\lineto(85.6185035,352.08818817)
\curveto(85.85849853,352.04818347)(86.0884983,351.99818352)(86.3085035,351.93818817)
\curveto(86.53849785,351.88818363)(86.75349764,351.8231837)(86.9535035,351.74318817)
\curveto(87.04349735,351.70318382)(87.12849726,351.66818385)(87.2085035,351.63818817)
\curveto(87.29849709,351.6181839)(87.38349701,351.58318394)(87.4635035,351.53318817)
\curveto(87.65349674,351.41318411)(87.82349657,351.28318424)(87.9735035,351.14318817)
\curveto(88.13349626,351.00318452)(88.25849613,350.82818469)(88.3485035,350.61818817)
\curveto(88.37849601,350.54818497)(88.40349599,350.47818504)(88.4235035,350.40818817)
\curveto(88.44349595,350.33818518)(88.46349593,350.26318526)(88.4835035,350.18318817)
\curveto(88.4934959,350.1231854)(88.49849589,350.02818549)(88.4985035,349.89818817)
\curveto(88.50849588,349.77818574)(88.50849588,349.68318584)(88.4985035,349.61318817)
\lineto(88.4985035,349.53818817)
\curveto(88.47849591,349.47818604)(88.46349593,349.4181861)(88.4535035,349.35818817)
\curveto(88.45349594,349.30818621)(88.44849594,349.25818626)(88.4385035,349.20818817)
\curveto(88.36849602,348.90818661)(88.25849613,348.64318688)(88.1085035,348.41318817)
\curveto(87.94849644,348.17318735)(87.75349664,347.97818754)(87.5235035,347.82818817)
\curveto(87.2934971,347.67818784)(87.03349736,347.54818797)(86.7435035,347.43818817)
\curveto(86.63349776,347.38818813)(86.51349788,347.35318817)(86.3835035,347.33318817)
\curveto(86.26349813,347.31318821)(86.14349825,347.28818823)(86.0235035,347.25818817)
\curveto(85.93349846,347.23818828)(85.83849855,347.22818829)(85.7385035,347.22818817)
\curveto(85.64849874,347.2181883)(85.55849883,347.20318832)(85.4685035,347.18318817)
\lineto(85.1985035,347.18318817)
\curveto(85.13849925,347.16318836)(85.03349936,347.15318837)(84.8835035,347.15318817)
\curveto(84.74349965,347.15318837)(84.64349975,347.16318836)(84.5835035,347.18318817)
\curveto(84.55349984,347.18318834)(84.51849987,347.18818833)(84.4785035,347.19818817)
\lineto(84.3735035,347.19818817)
\curveto(84.25350014,347.2181883)(84.13350026,347.23318829)(84.0135035,347.24318817)
\curveto(83.8935005,347.25318827)(83.77850061,347.27318825)(83.6685035,347.30318817)
\curveto(83.27850111,347.41318811)(82.93350146,347.53818798)(82.6335035,347.67818817)
\curveto(82.33350206,347.82818769)(82.07850231,348.04818747)(81.8685035,348.33818817)
\curveto(81.72850266,348.52818699)(81.60850278,348.74818677)(81.5085035,348.99818817)
\curveto(81.4885029,349.05818646)(81.46850292,349.13818638)(81.4485035,349.23818817)
\curveto(81.42850296,349.28818623)(81.41350298,349.35818616)(81.4035035,349.44818817)
\curveto(81.393503,349.53818598)(81.39850299,349.61318591)(81.4185035,349.67318817)
\curveto(81.44850294,349.74318578)(81.49850289,349.79318573)(81.5685035,349.82318817)
\curveto(81.61850277,349.84318568)(81.67850271,349.85318567)(81.7485035,349.85318817)
\lineto(81.9735035,349.85318817)
\lineto(82.6785035,349.85318817)
\lineto(82.9185035,349.85318817)
\curveto(82.99850139,349.85318567)(83.06850132,349.84318568)(83.1285035,349.82318817)
\curveto(83.23850115,349.78318574)(83.30850108,349.7181858)(83.3385035,349.62818817)
\curveto(83.37850101,349.53818598)(83.42350097,349.44318608)(83.4735035,349.34318817)
\curveto(83.4935009,349.29318623)(83.52850086,349.22818629)(83.5785035,349.14818817)
\curveto(83.63850075,349.06818645)(83.6885007,349.0181865)(83.7285035,348.99818817)
\curveto(83.84850054,348.89818662)(83.96350043,348.8181867)(84.0735035,348.75818817)
\curveto(84.18350021,348.70818681)(84.32350007,348.65818686)(84.4935035,348.60818817)
\curveto(84.54349985,348.58818693)(84.5934998,348.57818694)(84.6435035,348.57818817)
\curveto(84.6934997,348.58818693)(84.74349965,348.58818693)(84.7935035,348.57818817)
\curveto(84.87349952,348.55818696)(84.95849943,348.54818697)(85.0485035,348.54818817)
\curveto(85.14849924,348.55818696)(85.23349916,348.57318695)(85.3035035,348.59318817)
\curveto(85.35349904,348.60318692)(85.39849899,348.60818691)(85.4385035,348.60818817)
\curveto(85.4884989,348.60818691)(85.53849885,348.6181869)(85.5885035,348.63818817)
\curveto(85.72849866,348.68818683)(85.85349854,348.74818677)(85.9635035,348.81818817)
\curveto(86.08349831,348.88818663)(86.17849821,348.97818654)(86.2485035,349.08818817)
\curveto(86.29849809,349.16818635)(86.33849805,349.29318623)(86.3685035,349.46318817)
\curveto(86.388498,349.53318599)(86.388498,349.59818592)(86.3685035,349.65818817)
\curveto(86.34849804,349.7181858)(86.32849806,349.76818575)(86.3085035,349.80818817)
\curveto(86.23849815,349.94818557)(86.14849824,350.05318547)(86.0385035,350.12318817)
\curveto(85.93849845,350.19318533)(85.81849857,350.25818526)(85.6785035,350.31818817)
\curveto(85.4884989,350.39818512)(85.2884991,350.46318506)(85.0785035,350.51318817)
\curveto(84.86849952,350.56318496)(84.65849973,350.6181849)(84.4485035,350.67818817)
\curveto(84.36850002,350.69818482)(84.28350011,350.71318481)(84.1935035,350.72318817)
\curveto(84.11350028,350.73318479)(84.03350036,350.74818477)(83.9535035,350.76818817)
\curveto(83.63350076,350.85818466)(83.32850106,350.94318458)(83.0385035,351.02318817)
\curveto(82.74850164,351.11318441)(82.48350191,351.24318428)(82.2435035,351.41318817)
\curveto(81.96350243,351.61318391)(81.75850263,351.88318364)(81.6285035,352.22318817)
\curveto(81.60850278,352.29318323)(81.5885028,352.38818313)(81.5685035,352.50818817)
\curveto(81.54850284,352.57818294)(81.53350286,352.66318286)(81.5235035,352.76318817)
\curveto(81.51350288,352.86318266)(81.51850287,352.95318257)(81.5385035,353.03318817)
\curveto(81.55850283,353.08318244)(81.56350283,353.1231824)(81.5535035,353.15318817)
\curveto(81.54350285,353.19318233)(81.54850284,353.23818228)(81.5685035,353.28818817)
\curveto(81.5885028,353.39818212)(81.60850278,353.49818202)(81.6285035,353.58818817)
\curveto(81.65850273,353.68818183)(81.6935027,353.78318174)(81.7335035,353.87318817)
\curveto(81.86350253,354.16318136)(82.04350235,354.39818112)(82.2735035,354.57818817)
\curveto(82.50350189,354.75818076)(82.76350163,354.90318062)(83.0535035,355.01318817)
\curveto(83.16350123,355.06318046)(83.27850111,355.09818042)(83.3985035,355.11818817)
\curveto(83.51850087,355.14818037)(83.64350075,355.17818034)(83.7735035,355.20818817)
\curveto(83.83350056,355.22818029)(83.8935005,355.23818028)(83.9535035,355.23818817)
\lineto(84.1335035,355.26818817)
\curveto(84.21350018,355.27818024)(84.29850009,355.28318024)(84.3885035,355.28318817)
\curveto(84.47849991,355.28318024)(84.56349983,355.28818023)(84.6435035,355.29818817)
}
}
{
\newrgbcolor{curcolor}{0 0 0}
\pscustom[linestyle=none,fillstyle=solid,fillcolor=curcolor]
{
}
}
{
\newrgbcolor{curcolor}{0 0 0}
\pscustom[linestyle=none,fillstyle=solid,fillcolor=curcolor]
{
\newpath
\moveto(101.45030037,345.50318817)
\curveto(101.45029203,345.34319018)(101.44529204,345.18819033)(101.43530037,345.03818817)
\curveto(101.43529205,344.87819064)(101.3852921,344.76819075)(101.28530037,344.70818817)
\curveto(101.20529228,344.65819086)(101.09029239,344.63819088)(100.94030037,344.64818817)
\lineto(100.52030037,344.64818817)
\lineto(100.20530037,344.64818817)
\curveto(100.09529339,344.63819088)(99.9852935,344.63819088)(99.87530037,344.64818817)
\curveto(99.77529371,344.64819087)(99.6802938,344.66319086)(99.59030037,344.69318817)
\curveto(99.51029397,344.71319081)(99.45029403,344.75319077)(99.41030037,344.81318817)
\curveto(99.36029412,344.89319063)(99.33529415,345.00819051)(99.33530037,345.15818817)
\curveto(99.34529414,345.29819022)(99.35029413,345.42819009)(99.35030037,345.54818817)
\lineto(99.35030037,347.18318817)
\lineto(99.35030037,347.55818817)
\curveto(99.35029413,347.69818782)(99.33529415,347.80318772)(99.30530037,347.87318817)
\curveto(99.2852942,347.89318763)(99.26529422,347.90818761)(99.24530037,347.91818817)
\curveto(99.23529425,347.93818758)(99.22029426,347.95818756)(99.20030037,347.97818817)
\curveto(99.11029437,347.98818753)(99.04029444,347.96818755)(98.99030037,347.91818817)
\curveto(98.94029454,347.87818764)(98.8852946,347.83818768)(98.82530037,347.79818817)
\curveto(98.73529475,347.72818779)(98.64029484,347.66318786)(98.54030037,347.60318817)
\curveto(98.45029503,347.54318798)(98.35029513,347.48818803)(98.24030037,347.43818817)
\curveto(98.06029542,347.35818816)(97.86029562,347.29818822)(97.64030037,347.25818817)
\curveto(97.42029606,347.20818831)(97.19529629,347.18318834)(96.96530037,347.18318817)
\curveto(96.73529675,347.17318835)(96.50529698,347.18818833)(96.27530037,347.22818817)
\curveto(96.05529743,347.26818825)(95.85529763,347.32818819)(95.67530037,347.40818817)
\curveto(95.22529826,347.60818791)(94.86029862,347.86318766)(94.58030037,348.17318817)
\curveto(94.30029918,348.49318703)(94.06529942,348.88318664)(93.87530037,349.34318817)
\curveto(93.82529966,349.45318607)(93.79029969,349.56318596)(93.77030037,349.67318817)
\curveto(93.75029973,349.79318573)(93.72529976,349.90818561)(93.69530037,350.01818817)
\curveto(93.67529981,350.05818546)(93.66529982,350.09318543)(93.66530037,350.12318817)
\curveto(93.67529981,350.16318536)(93.67529981,350.20318532)(93.66530037,350.24318817)
\curveto(93.64529984,350.3231852)(93.63029985,350.40818511)(93.62030037,350.49818817)
\curveto(93.62029986,350.59818492)(93.61029987,350.69318483)(93.59030037,350.78318817)
\lineto(93.59030037,350.97818817)
\curveto(93.5802999,351.02818449)(93.57529991,351.08818443)(93.57530037,351.15818817)
\curveto(93.57529991,351.23818428)(93.5802999,351.30318422)(93.59030037,351.35318817)
\curveto(93.60029988,351.40318412)(93.60529988,351.44818407)(93.60530037,351.48818817)
\lineto(93.60530037,351.62318817)
\curveto(93.61529987,351.67318385)(93.61529987,351.7231838)(93.60530037,351.77318817)
\curveto(93.60529988,351.8231837)(93.61529987,351.87318365)(93.63530037,351.92318817)
\curveto(93.65529983,352.01318351)(93.67029981,352.10318342)(93.68030037,352.19318817)
\curveto(93.69029979,352.29318323)(93.70529978,352.38818313)(93.72530037,352.47818817)
\curveto(93.77529971,352.64818287)(93.82529966,352.80818271)(93.87530037,352.95818817)
\curveto(93.93529955,353.10818241)(93.99529949,353.25318227)(94.05530037,353.39318817)
\curveto(94.11529937,353.53318199)(94.19029929,353.66818185)(94.28030037,353.79818817)
\curveto(94.37029911,353.92818159)(94.46029902,354.05318147)(94.55030037,354.17318817)
\curveto(94.64029884,354.28318124)(94.74029874,354.38318114)(94.85030037,354.47318817)
\curveto(94.8802986,354.50318102)(94.90029858,354.52818099)(94.91030037,354.54818817)
\curveto(94.96029852,354.57818094)(95.00529848,354.60818091)(95.04530037,354.63818817)
\curveto(95.0852984,354.67818084)(95.12529836,354.71318081)(95.16530037,354.74318817)
\curveto(95.30529818,354.84318068)(95.45029803,354.9231806)(95.60030037,354.98318817)
\curveto(95.76029772,355.05318047)(95.92529756,355.1181804)(96.09530037,355.17818817)
\curveto(96.1852973,355.20818031)(96.27529721,355.22818029)(96.36530037,355.23818817)
\curveto(96.45529703,355.24818027)(96.54529694,355.26318026)(96.63530037,355.28318817)
\curveto(96.66529682,355.29318023)(96.72029676,355.29318023)(96.80030037,355.28318817)
\curveto(96.8802966,355.27318025)(96.93029655,355.27818024)(96.95030037,355.29818817)
\curveto(97.27029621,355.30818021)(97.57029591,355.27818024)(97.85030037,355.20818817)
\curveto(98.13029535,355.14818037)(98.37029511,355.05818046)(98.57030037,354.93818817)
\lineto(98.75030037,354.81818817)
\curveto(98.81029467,354.77818074)(98.86529462,354.73818078)(98.91530037,354.69818817)
\curveto(98.97529451,354.64818087)(99.02529446,354.59818092)(99.06530037,354.54818817)
\curveto(99.11529437,354.50818101)(99.19529429,354.48818103)(99.30530037,354.48818817)
\lineto(99.35030037,354.53318817)
\lineto(99.41030037,354.59318817)
\curveto(99.44029404,354.67318085)(99.46029402,354.74818077)(99.47030037,354.81818817)
\curveto(99.480294,354.89818062)(99.52029396,354.96318056)(99.59030037,355.01318817)
\curveto(99.64029384,355.05318047)(99.71029377,355.07318045)(99.80030037,355.07318817)
\curveto(99.90029358,355.08318044)(100.00029348,355.08818043)(100.10030037,355.08818817)
\lineto(100.82030037,355.08818817)
\lineto(101.03030037,355.08818817)
\curveto(101.10029238,355.08818043)(101.16529232,355.07818044)(101.22530037,355.05818817)
\curveto(101.29529219,355.03818048)(101.35029213,354.99318053)(101.39030037,354.92318817)
\curveto(101.44029204,354.85318067)(101.46029202,354.75818076)(101.45030037,354.63818817)
\lineto(101.45030037,354.29318817)
\lineto(101.45030037,345.50318817)
\moveto(99.41030037,351.11318817)
\curveto(99.42029406,351.13318439)(99.42029406,351.15818436)(99.41030037,351.18818817)
\lineto(99.41030037,351.26318817)
\curveto(99.40029408,351.36318416)(99.39529409,351.45818406)(99.39530037,351.54818817)
\curveto(99.39529409,351.63818388)(99.3852941,351.7231838)(99.36530037,351.80318817)
\curveto(99.35529413,351.83318369)(99.35029413,351.85818366)(99.35030037,351.87818817)
\curveto(99.36029412,351.90818361)(99.36029412,351.93818358)(99.35030037,351.96818817)
\curveto(99.33029415,352.04818347)(99.31029417,352.1181834)(99.29030037,352.17818817)
\curveto(99.2802942,352.24818327)(99.26529422,352.3181832)(99.24530037,352.38818817)
\curveto(99.14529434,352.67818284)(99.01029447,352.92818259)(98.84030037,353.13818817)
\curveto(98.67029481,353.34818217)(98.45029503,353.50818201)(98.18030037,353.61818817)
\curveto(98.07029541,353.66818185)(97.95029553,353.69318183)(97.82030037,353.69318817)
\curveto(97.70029578,353.70318182)(97.57029591,353.70818181)(97.43030037,353.70818817)
\curveto(97.40029608,353.68818183)(97.36529612,353.67818184)(97.32530037,353.67818817)
\curveto(97.2852962,353.68818183)(97.24529624,353.68818183)(97.20530037,353.67818817)
\lineto(97.02530037,353.61818817)
\curveto(96.96529652,353.60818191)(96.91029657,353.59318193)(96.86030037,353.57318817)
\curveto(96.57029691,353.44318208)(96.34029714,353.25318227)(96.17030037,353.00318817)
\curveto(96.01029747,352.75318277)(95.8852976,352.46318306)(95.79530037,352.13318817)
\curveto(95.77529771,352.05318347)(95.76029772,351.97818354)(95.75030037,351.90818817)
\curveto(95.75029773,351.84818367)(95.74029774,351.77818374)(95.72030037,351.69818817)
\curveto(95.72029776,351.62818389)(95.71529777,351.57818394)(95.70530037,351.54818817)
\curveto(95.69529779,351.49818402)(95.6852978,351.40818411)(95.67530037,351.27818817)
\curveto(95.67529781,351.15818436)(95.6852978,351.07318445)(95.70530037,351.02318817)
\lineto(95.70530037,350.88818817)
\curveto(95.71529777,350.84818467)(95.72029776,350.80818471)(95.72030037,350.76818817)
\curveto(95.72029776,350.72818479)(95.72529776,350.69318483)(95.73530037,350.66318817)
\lineto(95.73530037,350.58818817)
\curveto(95.74529774,350.55818496)(95.75029773,350.53318499)(95.75030037,350.51318817)
\curveto(95.77029771,350.43318509)(95.7852977,350.35818516)(95.79530037,350.28818817)
\curveto(95.80529768,350.2181853)(95.82529766,350.14818537)(95.85530037,350.07818817)
\curveto(95.93529755,349.82818569)(96.04029744,349.61318591)(96.17030037,349.43318817)
\curveto(96.30029718,349.25318627)(96.46529702,349.09818642)(96.66530037,348.96818817)
\curveto(96.80529668,348.88818663)(96.96029652,348.82818669)(97.13030037,348.78818817)
\curveto(97.16029632,348.77818674)(97.1852963,348.77318675)(97.20530037,348.77318817)
\curveto(97.23529625,348.77318675)(97.27029621,348.76818675)(97.31030037,348.75818817)
\curveto(97.34029614,348.74818677)(97.3852961,348.73818678)(97.44530037,348.72818817)
\curveto(97.51529597,348.72818679)(97.57529591,348.73318679)(97.62530037,348.74318817)
\curveto(97.64529584,348.75318677)(97.67029581,348.75318677)(97.70030037,348.74318817)
\curveto(97.74029574,348.74318678)(97.77529571,348.74818677)(97.80530037,348.75818817)
\curveto(97.87529561,348.77818674)(97.94029554,348.79318673)(98.00030037,348.80318817)
\curveto(98.07029541,348.81318671)(98.14029534,348.82818669)(98.21030037,348.84818817)
\curveto(98.47029501,348.95818656)(98.67529481,349.10318642)(98.82530037,349.28318817)
\curveto(98.9852945,349.46318606)(99.12029436,349.68318584)(99.23030037,349.94318817)
\curveto(99.26029422,350.0231855)(99.2852942,350.10818541)(99.30530037,350.19818817)
\lineto(99.36530037,350.46818817)
\lineto(99.36530037,350.57318817)
\curveto(99.37529411,350.60318492)(99.3802941,350.63818488)(99.38030037,350.67818817)
\curveto(99.40029408,350.77818474)(99.41029407,350.86318466)(99.41030037,350.93318817)
\lineto(99.41030037,351.11318817)
}
}
{
\newrgbcolor{curcolor}{0 0 0}
\pscustom[linestyle=none,fillstyle=solid,fillcolor=curcolor]
{
\newpath
\moveto(103.48022225,355.07318817)
\lineto(104.60522225,355.07318817)
\curveto(104.71521981,355.07318045)(104.81521971,355.06818045)(104.90522225,355.05818817)
\curveto(104.99521953,355.04818047)(105.06021947,355.01318051)(105.10022225,354.95318817)
\curveto(105.15021938,354.89318063)(105.18021935,354.80818071)(105.19022225,354.69818817)
\curveto(105.20021933,354.59818092)(105.20521932,354.49318103)(105.20522225,354.38318817)
\lineto(105.20522225,353.33318817)
\lineto(105.20522225,351.09818817)
\curveto(105.20521932,350.73818478)(105.22021931,350.39818512)(105.25022225,350.07818817)
\curveto(105.28021925,349.75818576)(105.37021916,349.49318603)(105.52022225,349.28318817)
\curveto(105.66021887,349.07318645)(105.88521864,348.9231866)(106.19522225,348.83318817)
\curveto(106.24521828,348.8231867)(106.28521824,348.8181867)(106.31522225,348.81818817)
\curveto(106.35521817,348.8181867)(106.40021813,348.81318671)(106.45022225,348.80318817)
\curveto(106.50021803,348.79318673)(106.55521797,348.78818673)(106.61522225,348.78818817)
\curveto(106.67521785,348.78818673)(106.72021781,348.79318673)(106.75022225,348.80318817)
\curveto(106.80021773,348.8231867)(106.84021769,348.82818669)(106.87022225,348.81818817)
\curveto(106.91021762,348.80818671)(106.95021758,348.81318671)(106.99022225,348.83318817)
\curveto(107.20021733,348.88318664)(107.36521716,348.94818657)(107.48522225,349.02818817)
\curveto(107.66521686,349.13818638)(107.80521672,349.27818624)(107.90522225,349.44818817)
\curveto(108.01521651,349.62818589)(108.09021644,349.8231857)(108.13022225,350.03318817)
\curveto(108.18021635,350.25318527)(108.21021632,350.49318503)(108.22022225,350.75318817)
\curveto(108.2302163,351.0231845)(108.23521629,351.30318422)(108.23522225,351.59318817)
\lineto(108.23522225,353.40818817)
\lineto(108.23522225,354.38318817)
\lineto(108.23522225,354.65318817)
\curveto(108.23521629,354.75318077)(108.25521627,354.83318069)(108.29522225,354.89318817)
\curveto(108.34521618,354.98318054)(108.42021611,355.03318049)(108.52022225,355.04318817)
\curveto(108.62021591,355.06318046)(108.74021579,355.07318045)(108.88022225,355.07318817)
\lineto(109.67522225,355.07318817)
\lineto(109.96022225,355.07318817)
\curveto(110.05021448,355.07318045)(110.1252144,355.05318047)(110.18522225,355.01318817)
\curveto(110.26521426,354.96318056)(110.31021422,354.88818063)(110.32022225,354.78818817)
\curveto(110.3302142,354.68818083)(110.33521419,354.57318095)(110.33522225,354.44318817)
\lineto(110.33522225,353.30318817)
\lineto(110.33522225,349.08818817)
\lineto(110.33522225,348.02318817)
\lineto(110.33522225,347.72318817)
\curveto(110.33521419,347.6231879)(110.31521421,347.54818797)(110.27522225,347.49818817)
\curveto(110.2252143,347.4181881)(110.15021438,347.37318815)(110.05022225,347.36318817)
\curveto(109.95021458,347.35318817)(109.84521468,347.34818817)(109.73522225,347.34818817)
\lineto(108.92522225,347.34818817)
\curveto(108.81521571,347.34818817)(108.71521581,347.35318817)(108.62522225,347.36318817)
\curveto(108.54521598,347.37318815)(108.48021605,347.41318811)(108.43022225,347.48318817)
\curveto(108.41021612,347.51318801)(108.39021614,347.55818796)(108.37022225,347.61818817)
\curveto(108.36021617,347.67818784)(108.34521618,347.73818778)(108.32522225,347.79818817)
\curveto(108.31521621,347.85818766)(108.30021623,347.91318761)(108.28022225,347.96318817)
\curveto(108.26021627,348.01318751)(108.2302163,348.04318748)(108.19022225,348.05318817)
\curveto(108.17021636,348.07318745)(108.14521638,348.07818744)(108.11522225,348.06818817)
\curveto(108.08521644,348.05818746)(108.06021647,348.04818747)(108.04022225,348.03818817)
\curveto(107.97021656,347.99818752)(107.91021662,347.95318757)(107.86022225,347.90318817)
\curveto(107.81021672,347.85318767)(107.75521677,347.80818771)(107.69522225,347.76818817)
\curveto(107.65521687,347.73818778)(107.61521691,347.70318782)(107.57522225,347.66318817)
\curveto(107.54521698,347.63318789)(107.50521702,347.60318792)(107.45522225,347.57318817)
\curveto(107.2252173,347.43318809)(106.95521757,347.3231882)(106.64522225,347.24318817)
\curveto(106.57521795,347.2231883)(106.50521802,347.21318831)(106.43522225,347.21318817)
\curveto(106.36521816,347.20318832)(106.29021824,347.18818833)(106.21022225,347.16818817)
\curveto(106.17021836,347.15818836)(106.1252184,347.15818836)(106.07522225,347.16818817)
\curveto(106.03521849,347.16818835)(105.99521853,347.16318836)(105.95522225,347.15318817)
\curveto(105.9252186,347.14318838)(105.86021867,347.14318838)(105.76022225,347.15318817)
\curveto(105.67021886,347.15318837)(105.61021892,347.15818836)(105.58022225,347.16818817)
\curveto(105.530219,347.16818835)(105.48021905,347.17318835)(105.43022225,347.18318817)
\lineto(105.28022225,347.18318817)
\curveto(105.16021937,347.21318831)(105.04521948,347.23818828)(104.93522225,347.25818817)
\curveto(104.8252197,347.27818824)(104.71521981,347.30818821)(104.60522225,347.34818817)
\curveto(104.55521997,347.36818815)(104.51022002,347.38318814)(104.47022225,347.39318817)
\curveto(104.44022009,347.41318811)(104.40022013,347.43318809)(104.35022225,347.45318817)
\curveto(104.00022053,347.64318788)(103.72022081,347.90818761)(103.51022225,348.24818817)
\curveto(103.38022115,348.45818706)(103.28522124,348.70818681)(103.22522225,348.99818817)
\curveto(103.16522136,349.29818622)(103.1252214,349.61318591)(103.10522225,349.94318817)
\curveto(103.09522143,350.28318524)(103.09022144,350.62818489)(103.09022225,350.97818817)
\curveto(103.10022143,351.33818418)(103.10522142,351.69318383)(103.10522225,352.04318817)
\lineto(103.10522225,354.08318817)
\curveto(103.10522142,354.21318131)(103.10022143,354.36318116)(103.09022225,354.53318817)
\curveto(103.09022144,354.71318081)(103.11522141,354.84318068)(103.16522225,354.92318817)
\curveto(103.19522133,354.97318055)(103.25522127,355.0181805)(103.34522225,355.05818817)
\curveto(103.40522112,355.05818046)(103.45022108,355.06318046)(103.48022225,355.07318817)
}
}
{
\newrgbcolor{curcolor}{0 0 0}
\pscustom[linestyle=none,fillstyle=solid,fillcolor=curcolor]
{
\newpath
\moveto(119.33147225,351.29318817)
\curveto(119.35146408,351.21318431)(119.35146408,351.1231844)(119.33147225,351.02318817)
\curveto(119.31146412,350.9231846)(119.27646416,350.85818466)(119.22647225,350.82818817)
\curveto(119.17646426,350.78818473)(119.10146433,350.75818476)(119.00147225,350.73818817)
\curveto(118.91146452,350.72818479)(118.80646463,350.7181848)(118.68647225,350.70818817)
\lineto(118.34147225,350.70818817)
\curveto(118.2314652,350.7181848)(118.1314653,350.7231848)(118.04147225,350.72318817)
\lineto(114.38147225,350.72318817)
\lineto(114.17147225,350.72318817)
\curveto(114.11146932,350.7231848)(114.05646938,350.71318481)(114.00647225,350.69318817)
\curveto(113.92646951,350.65318487)(113.87646956,350.61318491)(113.85647225,350.57318817)
\curveto(113.8364696,350.55318497)(113.81646962,350.51318501)(113.79647225,350.45318817)
\curveto(113.77646966,350.40318512)(113.77146966,350.35318517)(113.78147225,350.30318817)
\curveto(113.80146963,350.24318528)(113.81146962,350.18318534)(113.81147225,350.12318817)
\curveto(113.82146961,350.07318545)(113.8364696,350.0181855)(113.85647225,349.95818817)
\curveto(113.9364695,349.7181858)(114.0314694,349.518186)(114.14147225,349.35818817)
\curveto(114.26146917,349.20818631)(114.42146901,349.07318645)(114.62147225,348.95318817)
\curveto(114.70146873,348.90318662)(114.78146865,348.86818665)(114.86147225,348.84818817)
\curveto(114.95146848,348.83818668)(115.04146839,348.8181867)(115.13147225,348.78818817)
\curveto(115.21146822,348.76818675)(115.32146811,348.75318677)(115.46147225,348.74318817)
\curveto(115.60146783,348.73318679)(115.72146771,348.73818678)(115.82147225,348.75818817)
\lineto(115.95647225,348.75818817)
\curveto(116.05646738,348.77818674)(116.14646729,348.79818672)(116.22647225,348.81818817)
\curveto(116.31646712,348.84818667)(116.40146703,348.87818664)(116.48147225,348.90818817)
\curveto(116.58146685,348.95818656)(116.69146674,349.0231865)(116.81147225,349.10318817)
\curveto(116.94146649,349.18318634)(117.0364664,349.26318626)(117.09647225,349.34318817)
\curveto(117.14646629,349.41318611)(117.19646624,349.47818604)(117.24647225,349.53818817)
\curveto(117.30646613,349.60818591)(117.37646606,349.65818586)(117.45647225,349.68818817)
\curveto(117.55646588,349.73818578)(117.68146575,349.75818576)(117.83147225,349.74818817)
\lineto(118.26647225,349.74818817)
\lineto(118.44647225,349.74818817)
\curveto(118.51646492,349.75818576)(118.57646486,349.75318577)(118.62647225,349.73318817)
\lineto(118.77647225,349.73318817)
\curveto(118.87646456,349.71318581)(118.94646449,349.68818583)(118.98647225,349.65818817)
\curveto(119.02646441,349.63818588)(119.04646439,349.59318593)(119.04647225,349.52318817)
\curveto(119.05646438,349.45318607)(119.05146438,349.39318613)(119.03147225,349.34318817)
\curveto(118.98146445,349.20318632)(118.92646451,349.07818644)(118.86647225,348.96818817)
\curveto(118.80646463,348.85818666)(118.7364647,348.74818677)(118.65647225,348.63818817)
\curveto(118.436465,348.30818721)(118.18646525,348.04318748)(117.90647225,347.84318817)
\curveto(117.62646581,347.64318788)(117.27646616,347.47318805)(116.85647225,347.33318817)
\curveto(116.74646669,347.29318823)(116.6364668,347.26818825)(116.52647225,347.25818817)
\curveto(116.41646702,347.24818827)(116.30146713,347.22818829)(116.18147225,347.19818817)
\curveto(116.14146729,347.18818833)(116.09646734,347.18818833)(116.04647225,347.19818817)
\curveto(116.00646743,347.19818832)(115.96646747,347.19318833)(115.92647225,347.18318817)
\lineto(115.76147225,347.18318817)
\curveto(115.71146772,347.16318836)(115.65146778,347.15818836)(115.58147225,347.16818817)
\curveto(115.52146791,347.16818835)(115.46646797,347.17318835)(115.41647225,347.18318817)
\curveto(115.3364681,347.19318833)(115.26646817,347.19318833)(115.20647225,347.18318817)
\curveto(115.14646829,347.17318835)(115.08146835,347.17818834)(115.01147225,347.19818817)
\curveto(114.96146847,347.2181883)(114.90646853,347.22818829)(114.84647225,347.22818817)
\curveto(114.78646865,347.22818829)(114.7314687,347.23818828)(114.68147225,347.25818817)
\curveto(114.57146886,347.27818824)(114.46146897,347.30318822)(114.35147225,347.33318817)
\curveto(114.24146919,347.35318817)(114.14146929,347.38818813)(114.05147225,347.43818817)
\curveto(113.94146949,347.47818804)(113.8364696,347.51318801)(113.73647225,347.54318817)
\curveto(113.64646979,347.58318794)(113.56146987,347.62818789)(113.48147225,347.67818817)
\curveto(113.16147027,347.87818764)(112.87647056,348.10818741)(112.62647225,348.36818817)
\curveto(112.37647106,348.63818688)(112.17147126,348.94818657)(112.01147225,349.29818817)
\curveto(111.96147147,349.40818611)(111.92147151,349.518186)(111.89147225,349.62818817)
\curveto(111.86147157,349.74818577)(111.82147161,349.86818565)(111.77147225,349.98818817)
\curveto(111.76147167,350.02818549)(111.75647168,350.06318546)(111.75647225,350.09318817)
\curveto(111.75647168,350.13318539)(111.75147168,350.17318535)(111.74147225,350.21318817)
\curveto(111.70147173,350.33318519)(111.67647176,350.46318506)(111.66647225,350.60318817)
\lineto(111.63647225,351.02318817)
\curveto(111.6364718,351.07318445)(111.6314718,351.12818439)(111.62147225,351.18818817)
\curveto(111.62147181,351.24818427)(111.62647181,351.30318422)(111.63647225,351.35318817)
\lineto(111.63647225,351.53318817)
\lineto(111.68147225,351.89318817)
\curveto(111.72147171,352.06318346)(111.75647168,352.22818329)(111.78647225,352.38818817)
\curveto(111.81647162,352.54818297)(111.86147157,352.69818282)(111.92147225,352.83818817)
\curveto(112.35147108,353.87818164)(113.08147035,354.61318091)(114.11147225,355.04318817)
\curveto(114.25146918,355.10318042)(114.39146904,355.14318038)(114.53147225,355.16318817)
\curveto(114.68146875,355.19318033)(114.8364686,355.22818029)(114.99647225,355.26818817)
\curveto(115.07646836,355.27818024)(115.15146828,355.28318024)(115.22147225,355.28318817)
\curveto(115.29146814,355.28318024)(115.36646807,355.28818023)(115.44647225,355.29818817)
\curveto(115.95646748,355.30818021)(116.39146704,355.24818027)(116.75147225,355.11818817)
\curveto(117.12146631,354.99818052)(117.45146598,354.83818068)(117.74147225,354.63818817)
\curveto(117.8314656,354.57818094)(117.92146551,354.50818101)(118.01147225,354.42818817)
\curveto(118.10146533,354.35818116)(118.18146525,354.28318124)(118.25147225,354.20318817)
\curveto(118.28146515,354.15318137)(118.32146511,354.11318141)(118.37147225,354.08318817)
\curveto(118.45146498,353.97318155)(118.52646491,353.85818166)(118.59647225,353.73818817)
\curveto(118.66646477,353.62818189)(118.74146469,353.51318201)(118.82147225,353.39318817)
\curveto(118.87146456,353.30318222)(118.91146452,353.20818231)(118.94147225,353.10818817)
\curveto(118.98146445,353.0181825)(119.02146441,352.9181826)(119.06147225,352.80818817)
\curveto(119.11146432,352.67818284)(119.15146428,352.54318298)(119.18147225,352.40318817)
\curveto(119.21146422,352.26318326)(119.24646419,352.1231834)(119.28647225,351.98318817)
\curveto(119.30646413,351.90318362)(119.31146412,351.81318371)(119.30147225,351.71318817)
\curveto(119.30146413,351.6231839)(119.31146412,351.53818398)(119.33147225,351.45818817)
\lineto(119.33147225,351.29318817)
\moveto(117.08147225,352.17818817)
\curveto(117.15146628,352.27818324)(117.15646628,352.39818312)(117.09647225,352.53818817)
\curveto(117.04646639,352.68818283)(117.00646643,352.79818272)(116.97647225,352.86818817)
\curveto(116.8364666,353.13818238)(116.65146678,353.34318218)(116.42147225,353.48318817)
\curveto(116.19146724,353.63318189)(115.87146756,353.71318181)(115.46147225,353.72318817)
\curveto(115.431468,353.70318182)(115.39646804,353.69818182)(115.35647225,353.70818817)
\curveto(115.31646812,353.7181818)(115.28146815,353.7181818)(115.25147225,353.70818817)
\curveto(115.20146823,353.68818183)(115.14646829,353.67318185)(115.08647225,353.66318817)
\curveto(115.02646841,353.66318186)(114.97146846,353.65318187)(114.92147225,353.63318817)
\curveto(114.48146895,353.49318203)(114.15646928,353.2181823)(113.94647225,352.80818817)
\curveto(113.92646951,352.76818275)(113.90146953,352.71318281)(113.87147225,352.64318817)
\curveto(113.85146958,352.58318294)(113.8364696,352.518183)(113.82647225,352.44818817)
\curveto(113.81646962,352.38818313)(113.81646962,352.32818319)(113.82647225,352.26818817)
\curveto(113.84646959,352.20818331)(113.88146955,352.15818336)(113.93147225,352.11818817)
\curveto(114.01146942,352.06818345)(114.12146931,352.04318348)(114.26147225,352.04318817)
\lineto(114.66647225,352.04318817)
\lineto(116.33147225,352.04318817)
\lineto(116.76647225,352.04318817)
\curveto(116.92646651,352.05318347)(117.0314664,352.09818342)(117.08147225,352.17818817)
}
}
{
\newrgbcolor{curcolor}{0 0 0}
\pscustom[linestyle=none,fillstyle=solid,fillcolor=curcolor]
{
}
}
{
\newrgbcolor{curcolor}{0 0 0}
\pscustom[linestyle=none,fillstyle=solid,fillcolor=curcolor]
{
\newpath
\moveto(132.10490975,351.29318817)
\curveto(132.12490158,351.21318431)(132.12490158,351.1231844)(132.10490975,351.02318817)
\curveto(132.08490162,350.9231846)(132.04990166,350.85818466)(131.99990975,350.82818817)
\curveto(131.94990176,350.78818473)(131.87490183,350.75818476)(131.77490975,350.73818817)
\curveto(131.68490202,350.72818479)(131.57990213,350.7181848)(131.45990975,350.70818817)
\lineto(131.11490975,350.70818817)
\curveto(131.0049027,350.7181848)(130.9049028,350.7231848)(130.81490975,350.72318817)
\lineto(127.15490975,350.72318817)
\lineto(126.94490975,350.72318817)
\curveto(126.88490682,350.7231848)(126.82990688,350.71318481)(126.77990975,350.69318817)
\curveto(126.69990701,350.65318487)(126.64990706,350.61318491)(126.62990975,350.57318817)
\curveto(126.6099071,350.55318497)(126.58990712,350.51318501)(126.56990975,350.45318817)
\curveto(126.54990716,350.40318512)(126.54490716,350.35318517)(126.55490975,350.30318817)
\curveto(126.57490713,350.24318528)(126.58490712,350.18318534)(126.58490975,350.12318817)
\curveto(126.59490711,350.07318545)(126.6099071,350.0181855)(126.62990975,349.95818817)
\curveto(126.709907,349.7181858)(126.8049069,349.518186)(126.91490975,349.35818817)
\curveto(127.03490667,349.20818631)(127.19490651,349.07318645)(127.39490975,348.95318817)
\curveto(127.47490623,348.90318662)(127.55490615,348.86818665)(127.63490975,348.84818817)
\curveto(127.72490598,348.83818668)(127.81490589,348.8181867)(127.90490975,348.78818817)
\curveto(127.98490572,348.76818675)(128.09490561,348.75318677)(128.23490975,348.74318817)
\curveto(128.37490533,348.73318679)(128.49490521,348.73818678)(128.59490975,348.75818817)
\lineto(128.72990975,348.75818817)
\curveto(128.82990488,348.77818674)(128.91990479,348.79818672)(128.99990975,348.81818817)
\curveto(129.08990462,348.84818667)(129.17490453,348.87818664)(129.25490975,348.90818817)
\curveto(129.35490435,348.95818656)(129.46490424,349.0231865)(129.58490975,349.10318817)
\curveto(129.71490399,349.18318634)(129.8099039,349.26318626)(129.86990975,349.34318817)
\curveto(129.91990379,349.41318611)(129.96990374,349.47818604)(130.01990975,349.53818817)
\curveto(130.07990363,349.60818591)(130.14990356,349.65818586)(130.22990975,349.68818817)
\curveto(130.32990338,349.73818578)(130.45490325,349.75818576)(130.60490975,349.74818817)
\lineto(131.03990975,349.74818817)
\lineto(131.21990975,349.74818817)
\curveto(131.28990242,349.75818576)(131.34990236,349.75318577)(131.39990975,349.73318817)
\lineto(131.54990975,349.73318817)
\curveto(131.64990206,349.71318581)(131.71990199,349.68818583)(131.75990975,349.65818817)
\curveto(131.79990191,349.63818588)(131.81990189,349.59318593)(131.81990975,349.52318817)
\curveto(131.82990188,349.45318607)(131.82490188,349.39318613)(131.80490975,349.34318817)
\curveto(131.75490195,349.20318632)(131.69990201,349.07818644)(131.63990975,348.96818817)
\curveto(131.57990213,348.85818666)(131.5099022,348.74818677)(131.42990975,348.63818817)
\curveto(131.2099025,348.30818721)(130.95990275,348.04318748)(130.67990975,347.84318817)
\curveto(130.39990331,347.64318788)(130.04990366,347.47318805)(129.62990975,347.33318817)
\curveto(129.51990419,347.29318823)(129.4099043,347.26818825)(129.29990975,347.25818817)
\curveto(129.18990452,347.24818827)(129.07490463,347.22818829)(128.95490975,347.19818817)
\curveto(128.91490479,347.18818833)(128.86990484,347.18818833)(128.81990975,347.19818817)
\curveto(128.77990493,347.19818832)(128.73990497,347.19318833)(128.69990975,347.18318817)
\lineto(128.53490975,347.18318817)
\curveto(128.48490522,347.16318836)(128.42490528,347.15818836)(128.35490975,347.16818817)
\curveto(128.29490541,347.16818835)(128.23990547,347.17318835)(128.18990975,347.18318817)
\curveto(128.1099056,347.19318833)(128.03990567,347.19318833)(127.97990975,347.18318817)
\curveto(127.91990579,347.17318835)(127.85490585,347.17818834)(127.78490975,347.19818817)
\curveto(127.73490597,347.2181883)(127.67990603,347.22818829)(127.61990975,347.22818817)
\curveto(127.55990615,347.22818829)(127.5049062,347.23818828)(127.45490975,347.25818817)
\curveto(127.34490636,347.27818824)(127.23490647,347.30318822)(127.12490975,347.33318817)
\curveto(127.01490669,347.35318817)(126.91490679,347.38818813)(126.82490975,347.43818817)
\curveto(126.71490699,347.47818804)(126.6099071,347.51318801)(126.50990975,347.54318817)
\curveto(126.41990729,347.58318794)(126.33490737,347.62818789)(126.25490975,347.67818817)
\curveto(125.93490777,347.87818764)(125.64990806,348.10818741)(125.39990975,348.36818817)
\curveto(125.14990856,348.63818688)(124.94490876,348.94818657)(124.78490975,349.29818817)
\curveto(124.73490897,349.40818611)(124.69490901,349.518186)(124.66490975,349.62818817)
\curveto(124.63490907,349.74818577)(124.59490911,349.86818565)(124.54490975,349.98818817)
\curveto(124.53490917,350.02818549)(124.52990918,350.06318546)(124.52990975,350.09318817)
\curveto(124.52990918,350.13318539)(124.52490918,350.17318535)(124.51490975,350.21318817)
\curveto(124.47490923,350.33318519)(124.44990926,350.46318506)(124.43990975,350.60318817)
\lineto(124.40990975,351.02318817)
\curveto(124.4099093,351.07318445)(124.4049093,351.12818439)(124.39490975,351.18818817)
\curveto(124.39490931,351.24818427)(124.39990931,351.30318422)(124.40990975,351.35318817)
\lineto(124.40990975,351.53318817)
\lineto(124.45490975,351.89318817)
\curveto(124.49490921,352.06318346)(124.52990918,352.22818329)(124.55990975,352.38818817)
\curveto(124.58990912,352.54818297)(124.63490907,352.69818282)(124.69490975,352.83818817)
\curveto(125.12490858,353.87818164)(125.85490785,354.61318091)(126.88490975,355.04318817)
\curveto(127.02490668,355.10318042)(127.16490654,355.14318038)(127.30490975,355.16318817)
\curveto(127.45490625,355.19318033)(127.6099061,355.22818029)(127.76990975,355.26818817)
\curveto(127.84990586,355.27818024)(127.92490578,355.28318024)(127.99490975,355.28318817)
\curveto(128.06490564,355.28318024)(128.13990557,355.28818023)(128.21990975,355.29818817)
\curveto(128.72990498,355.30818021)(129.16490454,355.24818027)(129.52490975,355.11818817)
\curveto(129.89490381,354.99818052)(130.22490348,354.83818068)(130.51490975,354.63818817)
\curveto(130.6049031,354.57818094)(130.69490301,354.50818101)(130.78490975,354.42818817)
\curveto(130.87490283,354.35818116)(130.95490275,354.28318124)(131.02490975,354.20318817)
\curveto(131.05490265,354.15318137)(131.09490261,354.11318141)(131.14490975,354.08318817)
\curveto(131.22490248,353.97318155)(131.29990241,353.85818166)(131.36990975,353.73818817)
\curveto(131.43990227,353.62818189)(131.51490219,353.51318201)(131.59490975,353.39318817)
\curveto(131.64490206,353.30318222)(131.68490202,353.20818231)(131.71490975,353.10818817)
\curveto(131.75490195,353.0181825)(131.79490191,352.9181826)(131.83490975,352.80818817)
\curveto(131.88490182,352.67818284)(131.92490178,352.54318298)(131.95490975,352.40318817)
\curveto(131.98490172,352.26318326)(132.01990169,352.1231834)(132.05990975,351.98318817)
\curveto(132.07990163,351.90318362)(132.08490162,351.81318371)(132.07490975,351.71318817)
\curveto(132.07490163,351.6231839)(132.08490162,351.53818398)(132.10490975,351.45818817)
\lineto(132.10490975,351.29318817)
\moveto(129.85490975,352.17818817)
\curveto(129.92490378,352.27818324)(129.92990378,352.39818312)(129.86990975,352.53818817)
\curveto(129.81990389,352.68818283)(129.77990393,352.79818272)(129.74990975,352.86818817)
\curveto(129.6099041,353.13818238)(129.42490428,353.34318218)(129.19490975,353.48318817)
\curveto(128.96490474,353.63318189)(128.64490506,353.71318181)(128.23490975,353.72318817)
\curveto(128.2049055,353.70318182)(128.16990554,353.69818182)(128.12990975,353.70818817)
\curveto(128.08990562,353.7181818)(128.05490565,353.7181818)(128.02490975,353.70818817)
\curveto(127.97490573,353.68818183)(127.91990579,353.67318185)(127.85990975,353.66318817)
\curveto(127.79990591,353.66318186)(127.74490596,353.65318187)(127.69490975,353.63318817)
\curveto(127.25490645,353.49318203)(126.92990678,353.2181823)(126.71990975,352.80818817)
\curveto(126.69990701,352.76818275)(126.67490703,352.71318281)(126.64490975,352.64318817)
\curveto(126.62490708,352.58318294)(126.6099071,352.518183)(126.59990975,352.44818817)
\curveto(126.58990712,352.38818313)(126.58990712,352.32818319)(126.59990975,352.26818817)
\curveto(126.61990709,352.20818331)(126.65490705,352.15818336)(126.70490975,352.11818817)
\curveto(126.78490692,352.06818345)(126.89490681,352.04318348)(127.03490975,352.04318817)
\lineto(127.43990975,352.04318817)
\lineto(129.10490975,352.04318817)
\lineto(129.53990975,352.04318817)
\curveto(129.69990401,352.05318347)(129.8049039,352.09818342)(129.85490975,352.17818817)
}
}
{
\newrgbcolor{curcolor}{0 0 0}
\pscustom[linestyle=none,fillstyle=solid,fillcolor=curcolor]
{
\newpath
\moveto(136.323191,355.29818817)
\curveto(137.0731865,355.3181802)(137.72318585,355.23318029)(138.273191,355.04318817)
\curveto(138.83318474,354.86318066)(139.25818431,354.54818097)(139.548191,354.09818817)
\curveto(139.61818395,353.98818153)(139.67818389,353.87318165)(139.728191,353.75318817)
\curveto(139.78818378,353.64318188)(139.83818373,353.518182)(139.878191,353.37818817)
\curveto(139.89818367,353.3181822)(139.90818366,353.25318227)(139.908191,353.18318817)
\curveto(139.90818366,353.11318241)(139.89818367,353.05318247)(139.878191,353.00318817)
\curveto(139.83818373,352.94318258)(139.78318379,352.90318262)(139.713191,352.88318817)
\curveto(139.66318391,352.86318266)(139.60318397,352.85318267)(139.533191,352.85318817)
\lineto(139.323191,352.85318817)
\lineto(138.663191,352.85318817)
\curveto(138.59318498,352.85318267)(138.52318505,352.84818267)(138.453191,352.83818817)
\curveto(138.38318519,352.83818268)(138.31818525,352.84818267)(138.258191,352.86818817)
\curveto(138.15818541,352.88818263)(138.08318549,352.92818259)(138.033191,352.98818817)
\curveto(137.98318559,353.04818247)(137.93818563,353.10818241)(137.898191,353.16818817)
\lineto(137.778191,353.37818817)
\curveto(137.74818582,353.45818206)(137.69818587,353.523182)(137.628191,353.57318817)
\curveto(137.52818604,353.65318187)(137.42818614,353.71318181)(137.328191,353.75318817)
\curveto(137.23818633,353.79318173)(137.12318645,353.82818169)(136.983191,353.85818817)
\curveto(136.91318666,353.87818164)(136.80818676,353.89318163)(136.668191,353.90318817)
\curveto(136.53818703,353.91318161)(136.43818713,353.90818161)(136.368191,353.88818817)
\lineto(136.263191,353.88818817)
\lineto(136.113191,353.85818817)
\curveto(136.0731875,353.85818166)(136.02818754,353.85318167)(135.978191,353.84318817)
\curveto(135.80818776,353.79318173)(135.6681879,353.7231818)(135.558191,353.63318817)
\curveto(135.45818811,353.55318197)(135.38818818,353.42818209)(135.348191,353.25818817)
\curveto(135.32818824,353.18818233)(135.32818824,353.1231824)(135.348191,353.06318817)
\curveto(135.3681882,353.00318252)(135.38818818,352.95318257)(135.408191,352.91318817)
\curveto(135.47818809,352.79318273)(135.55818801,352.69818282)(135.648191,352.62818817)
\curveto(135.74818782,352.55818296)(135.86318771,352.49818302)(135.993191,352.44818817)
\curveto(136.18318739,352.36818315)(136.38818718,352.29818322)(136.608191,352.23818817)
\lineto(137.298191,352.08818817)
\curveto(137.53818603,352.04818347)(137.7681858,351.99818352)(137.988191,351.93818817)
\curveto(138.21818535,351.88818363)(138.43318514,351.8231837)(138.633191,351.74318817)
\curveto(138.72318485,351.70318382)(138.80818476,351.66818385)(138.888191,351.63818817)
\curveto(138.97818459,351.6181839)(139.06318451,351.58318394)(139.143191,351.53318817)
\curveto(139.33318424,351.41318411)(139.50318407,351.28318424)(139.653191,351.14318817)
\curveto(139.81318376,351.00318452)(139.93818363,350.82818469)(140.028191,350.61818817)
\curveto(140.05818351,350.54818497)(140.08318349,350.47818504)(140.103191,350.40818817)
\curveto(140.12318345,350.33818518)(140.14318343,350.26318526)(140.163191,350.18318817)
\curveto(140.1731834,350.1231854)(140.17818339,350.02818549)(140.178191,349.89818817)
\curveto(140.18818338,349.77818574)(140.18818338,349.68318584)(140.178191,349.61318817)
\lineto(140.178191,349.53818817)
\curveto(140.15818341,349.47818604)(140.14318343,349.4181861)(140.133191,349.35818817)
\curveto(140.13318344,349.30818621)(140.12818344,349.25818626)(140.118191,349.20818817)
\curveto(140.04818352,348.90818661)(139.93818363,348.64318688)(139.788191,348.41318817)
\curveto(139.62818394,348.17318735)(139.43318414,347.97818754)(139.203191,347.82818817)
\curveto(138.9731846,347.67818784)(138.71318486,347.54818797)(138.423191,347.43818817)
\curveto(138.31318526,347.38818813)(138.19318538,347.35318817)(138.063191,347.33318817)
\curveto(137.94318563,347.31318821)(137.82318575,347.28818823)(137.703191,347.25818817)
\curveto(137.61318596,347.23818828)(137.51818605,347.22818829)(137.418191,347.22818817)
\curveto(137.32818624,347.2181883)(137.23818633,347.20318832)(137.148191,347.18318817)
\lineto(136.878191,347.18318817)
\curveto(136.81818675,347.16318836)(136.71318686,347.15318837)(136.563191,347.15318817)
\curveto(136.42318715,347.15318837)(136.32318725,347.16318836)(136.263191,347.18318817)
\curveto(136.23318734,347.18318834)(136.19818737,347.18818833)(136.158191,347.19818817)
\lineto(136.053191,347.19818817)
\curveto(135.93318764,347.2181883)(135.81318776,347.23318829)(135.693191,347.24318817)
\curveto(135.573188,347.25318827)(135.45818811,347.27318825)(135.348191,347.30318817)
\curveto(134.95818861,347.41318811)(134.61318896,347.53818798)(134.313191,347.67818817)
\curveto(134.01318956,347.82818769)(133.75818981,348.04818747)(133.548191,348.33818817)
\curveto(133.40819016,348.52818699)(133.28819028,348.74818677)(133.188191,348.99818817)
\curveto(133.1681904,349.05818646)(133.14819042,349.13818638)(133.128191,349.23818817)
\curveto(133.10819046,349.28818623)(133.09319048,349.35818616)(133.083191,349.44818817)
\curveto(133.0731905,349.53818598)(133.07819049,349.61318591)(133.098191,349.67318817)
\curveto(133.12819044,349.74318578)(133.17819039,349.79318573)(133.248191,349.82318817)
\curveto(133.29819027,349.84318568)(133.35819021,349.85318567)(133.428191,349.85318817)
\lineto(133.653191,349.85318817)
\lineto(134.358191,349.85318817)
\lineto(134.598191,349.85318817)
\curveto(134.67818889,349.85318567)(134.74818882,349.84318568)(134.808191,349.82318817)
\curveto(134.91818865,349.78318574)(134.98818858,349.7181858)(135.018191,349.62818817)
\curveto(135.05818851,349.53818598)(135.10318847,349.44318608)(135.153191,349.34318817)
\curveto(135.1731884,349.29318623)(135.20818836,349.22818629)(135.258191,349.14818817)
\curveto(135.31818825,349.06818645)(135.3681882,349.0181865)(135.408191,348.99818817)
\curveto(135.52818804,348.89818662)(135.64318793,348.8181867)(135.753191,348.75818817)
\curveto(135.86318771,348.70818681)(136.00318757,348.65818686)(136.173191,348.60818817)
\curveto(136.22318735,348.58818693)(136.2731873,348.57818694)(136.323191,348.57818817)
\curveto(136.3731872,348.58818693)(136.42318715,348.58818693)(136.473191,348.57818817)
\curveto(136.55318702,348.55818696)(136.63818693,348.54818697)(136.728191,348.54818817)
\curveto(136.82818674,348.55818696)(136.91318666,348.57318695)(136.983191,348.59318817)
\curveto(137.03318654,348.60318692)(137.07818649,348.60818691)(137.118191,348.60818817)
\curveto(137.1681864,348.60818691)(137.21818635,348.6181869)(137.268191,348.63818817)
\curveto(137.40818616,348.68818683)(137.53318604,348.74818677)(137.643191,348.81818817)
\curveto(137.76318581,348.88818663)(137.85818571,348.97818654)(137.928191,349.08818817)
\curveto(137.97818559,349.16818635)(138.01818555,349.29318623)(138.048191,349.46318817)
\curveto(138.0681855,349.53318599)(138.0681855,349.59818592)(138.048191,349.65818817)
\curveto(138.02818554,349.7181858)(138.00818556,349.76818575)(137.988191,349.80818817)
\curveto(137.91818565,349.94818557)(137.82818574,350.05318547)(137.718191,350.12318817)
\curveto(137.61818595,350.19318533)(137.49818607,350.25818526)(137.358191,350.31818817)
\curveto(137.1681864,350.39818512)(136.9681866,350.46318506)(136.758191,350.51318817)
\curveto(136.54818702,350.56318496)(136.33818723,350.6181849)(136.128191,350.67818817)
\curveto(136.04818752,350.69818482)(135.96318761,350.71318481)(135.873191,350.72318817)
\curveto(135.79318778,350.73318479)(135.71318786,350.74818477)(135.633191,350.76818817)
\curveto(135.31318826,350.85818466)(135.00818856,350.94318458)(134.718191,351.02318817)
\curveto(134.42818914,351.11318441)(134.16318941,351.24318428)(133.923191,351.41318817)
\curveto(133.64318993,351.61318391)(133.43819013,351.88318364)(133.308191,352.22318817)
\curveto(133.28819028,352.29318323)(133.2681903,352.38818313)(133.248191,352.50818817)
\curveto(133.22819034,352.57818294)(133.21319036,352.66318286)(133.203191,352.76318817)
\curveto(133.19319038,352.86318266)(133.19819037,352.95318257)(133.218191,353.03318817)
\curveto(133.23819033,353.08318244)(133.24319033,353.1231824)(133.233191,353.15318817)
\curveto(133.22319035,353.19318233)(133.22819034,353.23818228)(133.248191,353.28818817)
\curveto(133.2681903,353.39818212)(133.28819028,353.49818202)(133.308191,353.58818817)
\curveto(133.33819023,353.68818183)(133.3731902,353.78318174)(133.413191,353.87318817)
\curveto(133.54319003,354.16318136)(133.72318985,354.39818112)(133.953191,354.57818817)
\curveto(134.18318939,354.75818076)(134.44318913,354.90318062)(134.733191,355.01318817)
\curveto(134.84318873,355.06318046)(134.95818861,355.09818042)(135.078191,355.11818817)
\curveto(135.19818837,355.14818037)(135.32318825,355.17818034)(135.453191,355.20818817)
\curveto(135.51318806,355.22818029)(135.573188,355.23818028)(135.633191,355.23818817)
\lineto(135.813191,355.26818817)
\curveto(135.89318768,355.27818024)(135.97818759,355.28318024)(136.068191,355.28318817)
\curveto(136.15818741,355.28318024)(136.24318733,355.28818023)(136.323191,355.29818817)
}
}
{
\newrgbcolor{curcolor}{0 0 0}
\pscustom[linestyle=none,fillstyle=solid,fillcolor=curcolor]
{
\newpath
\moveto(142.45983162,357.39818817)
\lineto(143.46483162,357.39818817)
\curveto(143.61482864,357.39817812)(143.74482851,357.38817813)(143.85483162,357.36818817)
\curveto(143.97482828,357.35817816)(144.05982819,357.29817822)(144.10983162,357.18818817)
\curveto(144.12982812,357.13817838)(144.13982811,357.07817844)(144.13983162,357.00818817)
\lineto(144.13983162,356.79818817)
\lineto(144.13983162,356.12318817)
\curveto(144.13982811,356.07317945)(144.13482812,356.01317951)(144.12483162,355.94318817)
\curveto(144.12482813,355.88317964)(144.12982812,355.82817969)(144.13983162,355.77818817)
\lineto(144.13983162,355.61318817)
\curveto(144.13982811,355.53317999)(144.14482811,355.45818006)(144.15483162,355.38818817)
\curveto(144.16482809,355.32818019)(144.18982806,355.27318025)(144.22983162,355.22318817)
\curveto(144.29982795,355.13318039)(144.42482783,355.08318044)(144.60483162,355.07318817)
\lineto(145.14483162,355.07318817)
\lineto(145.32483162,355.07318817)
\curveto(145.38482687,355.07318045)(145.43982681,355.06318046)(145.48983162,355.04318817)
\curveto(145.59982665,354.99318053)(145.65982659,354.90318062)(145.66983162,354.77318817)
\curveto(145.68982656,354.64318088)(145.69982655,354.49818102)(145.69983162,354.33818817)
\lineto(145.69983162,354.12818817)
\curveto(145.70982654,354.05818146)(145.70482655,353.99818152)(145.68483162,353.94818817)
\curveto(145.63482662,353.78818173)(145.52982672,353.70318182)(145.36983162,353.69318817)
\curveto(145.20982704,353.68318184)(145.02982722,353.67818184)(144.82983162,353.67818817)
\lineto(144.69483162,353.67818817)
\curveto(144.6548276,353.68818183)(144.61982763,353.68818183)(144.58983162,353.67818817)
\curveto(144.5498277,353.66818185)(144.51482774,353.66318186)(144.48483162,353.66318817)
\curveto(144.4548278,353.67318185)(144.42482783,353.66818185)(144.39483162,353.64818817)
\curveto(144.31482794,353.62818189)(144.254828,353.58318194)(144.21483162,353.51318817)
\curveto(144.18482807,353.45318207)(144.15982809,353.37818214)(144.13983162,353.28818817)
\curveto(144.12982812,353.23818228)(144.12982812,353.18318234)(144.13983162,353.12318817)
\curveto(144.1498281,353.06318246)(144.1498281,353.00818251)(144.13983162,352.95818817)
\lineto(144.13983162,352.02818817)
\lineto(144.13983162,350.27318817)
\curveto(144.13982811,350.0231855)(144.14482811,349.80318572)(144.15483162,349.61318817)
\curveto(144.17482808,349.43318609)(144.23982801,349.27318625)(144.34983162,349.13318817)
\curveto(144.39982785,349.07318645)(144.46482779,349.02818649)(144.54483162,348.99818817)
\lineto(144.81483162,348.93818817)
\curveto(144.84482741,348.92818659)(144.87482738,348.9231866)(144.90483162,348.92318817)
\curveto(144.94482731,348.93318659)(144.97482728,348.93318659)(144.99483162,348.92318817)
\lineto(145.15983162,348.92318817)
\curveto(145.26982698,348.9231866)(145.36482689,348.9181866)(145.44483162,348.90818817)
\curveto(145.52482673,348.89818662)(145.58982666,348.85818666)(145.63983162,348.78818817)
\curveto(145.67982657,348.72818679)(145.69982655,348.64818687)(145.69983162,348.54818817)
\lineto(145.69983162,348.26318817)
\curveto(145.69982655,348.05318747)(145.69482656,347.85818766)(145.68483162,347.67818817)
\curveto(145.68482657,347.50818801)(145.60482665,347.39318813)(145.44483162,347.33318817)
\curveto(145.39482686,347.31318821)(145.3498269,347.30818821)(145.30983162,347.31818817)
\curveto(145.26982698,347.3181882)(145.22482703,347.30818821)(145.17483162,347.28818817)
\lineto(145.02483162,347.28818817)
\curveto(145.00482725,347.28818823)(144.97482728,347.29318823)(144.93483162,347.30318817)
\curveto(144.89482736,347.30318822)(144.85982739,347.29818822)(144.82983162,347.28818817)
\curveto(144.77982747,347.27818824)(144.72482753,347.27818824)(144.66483162,347.28818817)
\lineto(144.51483162,347.28818817)
\lineto(144.36483162,347.28818817)
\curveto(144.31482794,347.27818824)(144.26982798,347.27818824)(144.22983162,347.28818817)
\lineto(144.06483162,347.28818817)
\curveto(144.01482824,347.29818822)(143.95982829,347.30318822)(143.89983162,347.30318817)
\curveto(143.83982841,347.30318822)(143.78482847,347.30818821)(143.73483162,347.31818817)
\curveto(143.66482859,347.32818819)(143.59982865,347.33818818)(143.53983162,347.34818817)
\lineto(143.35983162,347.37818817)
\curveto(143.249829,347.40818811)(143.14482911,347.44318808)(143.04483162,347.48318817)
\curveto(142.94482931,347.523188)(142.8498294,347.56818795)(142.75983162,347.61818817)
\lineto(142.66983162,347.67818817)
\curveto(142.63982961,347.70818781)(142.60482965,347.73818778)(142.56483162,347.76818817)
\curveto(142.54482971,347.78818773)(142.51982973,347.80818771)(142.48983162,347.82818817)
\lineto(142.41483162,347.90318817)
\curveto(142.27482998,348.09318743)(142.16983008,348.30318722)(142.09983162,348.53318817)
\curveto(142.07983017,348.57318695)(142.06983018,348.60818691)(142.06983162,348.63818817)
\curveto(142.07983017,348.67818684)(142.07983017,348.7231868)(142.06983162,348.77318817)
\curveto(142.05983019,348.79318673)(142.0548302,348.8181867)(142.05483162,348.84818817)
\curveto(142.0548302,348.87818664)(142.0498302,348.90318662)(142.03983162,348.92318817)
\lineto(142.03983162,349.07318817)
\curveto(142.02983022,349.11318641)(142.02483023,349.15818636)(142.02483162,349.20818817)
\curveto(142.03483022,349.25818626)(142.03983021,349.30818621)(142.03983162,349.35818817)
\lineto(142.03983162,349.92818817)
\lineto(142.03983162,352.16318817)
\lineto(142.03983162,352.95818817)
\lineto(142.03983162,353.16818817)
\curveto(142.0498302,353.23818228)(142.04483021,353.30318222)(142.02483162,353.36318817)
\curveto(141.98483027,353.50318202)(141.91483034,353.59318193)(141.81483162,353.63318817)
\curveto(141.70483055,353.68318184)(141.56483069,353.69818182)(141.39483162,353.67818817)
\curveto(141.22483103,353.65818186)(141.07983117,353.67318185)(140.95983162,353.72318817)
\curveto(140.87983137,353.75318177)(140.82983142,353.79818172)(140.80983162,353.85818817)
\curveto(140.78983146,353.9181816)(140.76983148,353.99318153)(140.74983162,354.08318817)
\lineto(140.74983162,354.39818817)
\curveto(140.7498315,354.57818094)(140.75983149,354.7231808)(140.77983162,354.83318817)
\curveto(140.79983145,354.94318058)(140.88483137,355.0181805)(141.03483162,355.05818817)
\curveto(141.07483118,355.07818044)(141.11483114,355.08318044)(141.15483162,355.07318817)
\lineto(141.28983162,355.07318817)
\curveto(141.43983081,355.07318045)(141.57983067,355.07818044)(141.70983162,355.08818817)
\curveto(141.83983041,355.10818041)(141.92983032,355.16818035)(141.97983162,355.26818817)
\curveto(142.00983024,355.33818018)(142.02483023,355.4181801)(142.02483162,355.50818817)
\curveto(142.03483022,355.59817992)(142.03983021,355.68817983)(142.03983162,355.77818817)
\lineto(142.03983162,356.70818817)
\lineto(142.03983162,356.96318817)
\curveto(142.03983021,357.05317847)(142.0498302,357.12817839)(142.06983162,357.18818817)
\curveto(142.11983013,357.28817823)(142.19483006,357.35317817)(142.29483162,357.38318817)
\curveto(142.31482994,357.39317813)(142.33982991,357.39317813)(142.36983162,357.38318817)
\curveto(142.40982984,357.38317814)(142.43982981,357.38817813)(142.45983162,357.39818817)
}
}
{
\newrgbcolor{curcolor}{0 0 0}
\pscustom[linestyle=none,fillstyle=solid,fillcolor=curcolor]
{
\newpath
\moveto(153.73326912,347.94818817)
\curveto(153.75326127,347.83818768)(153.76326126,347.72818779)(153.76326912,347.61818817)
\curveto(153.77326125,347.50818801)(153.7232613,347.43318809)(153.61326912,347.39318817)
\curveto(153.55326147,347.36318816)(153.48326154,347.34818817)(153.40326912,347.34818817)
\lineto(153.16326912,347.34818817)
\lineto(152.35326912,347.34818817)
\lineto(152.08326912,347.34818817)
\curveto(152.00326302,347.35818816)(151.93826309,347.38318814)(151.88826912,347.42318817)
\curveto(151.81826321,347.46318806)(151.76326326,347.518188)(151.72326912,347.58818817)
\curveto(151.69326333,347.66818785)(151.64826338,347.73318779)(151.58826912,347.78318817)
\curveto(151.56826346,347.80318772)(151.54326348,347.8181877)(151.51326912,347.82818817)
\curveto(151.48326354,347.84818767)(151.44326358,347.85318767)(151.39326912,347.84318817)
\curveto(151.34326368,347.8231877)(151.29326373,347.79818772)(151.24326912,347.76818817)
\curveto(151.20326382,347.73818778)(151.15826387,347.71318781)(151.10826912,347.69318817)
\curveto(151.05826397,347.65318787)(151.00326402,347.6181879)(150.94326912,347.58818817)
\lineto(150.76326912,347.49818817)
\curveto(150.63326439,347.43818808)(150.49826453,347.38818813)(150.35826912,347.34818817)
\curveto(150.21826481,347.3181882)(150.07326495,347.28318824)(149.92326912,347.24318817)
\curveto(149.85326517,347.2231883)(149.78326524,347.21318831)(149.71326912,347.21318817)
\curveto(149.65326537,347.20318832)(149.58826544,347.19318833)(149.51826912,347.18318817)
\lineto(149.42826912,347.18318817)
\curveto(149.39826563,347.17318835)(149.36826566,347.16818835)(149.33826912,347.16818817)
\lineto(149.17326912,347.16818817)
\curveto(149.07326595,347.14818837)(148.97326605,347.14818837)(148.87326912,347.16818817)
\lineto(148.73826912,347.16818817)
\curveto(148.66826636,347.18818833)(148.59826643,347.19818832)(148.52826912,347.19818817)
\curveto(148.46826656,347.18818833)(148.40826662,347.19318833)(148.34826912,347.21318817)
\curveto(148.24826678,347.23318829)(148.15326687,347.25318827)(148.06326912,347.27318817)
\curveto(147.97326705,347.28318824)(147.88826714,347.30818821)(147.80826912,347.34818817)
\curveto(147.51826751,347.45818806)(147.26826776,347.59818792)(147.05826912,347.76818817)
\curveto(146.85826817,347.94818757)(146.69826833,348.18318734)(146.57826912,348.47318817)
\curveto(146.54826848,348.54318698)(146.51826851,348.6181869)(146.48826912,348.69818817)
\curveto(146.46826856,348.77818674)(146.44826858,348.86318666)(146.42826912,348.95318817)
\curveto(146.40826862,349.00318652)(146.39826863,349.05318647)(146.39826912,349.10318817)
\curveto(146.40826862,349.15318637)(146.40826862,349.20318632)(146.39826912,349.25318817)
\curveto(146.38826864,349.28318624)(146.37826865,349.34318618)(146.36826912,349.43318817)
\curveto(146.36826866,349.53318599)(146.37326865,349.60318592)(146.38326912,349.64318817)
\curveto(146.40326862,349.74318578)(146.41326861,349.82818569)(146.41326912,349.89818817)
\lineto(146.50326912,350.22818817)
\curveto(146.53326849,350.34818517)(146.57326845,350.45318507)(146.62326912,350.54318817)
\curveto(146.79326823,350.83318469)(146.98826804,351.05318447)(147.20826912,351.20318817)
\curveto(147.4282676,351.35318417)(147.70826732,351.48318404)(148.04826912,351.59318817)
\curveto(148.17826685,351.64318388)(148.31326671,351.67818384)(148.45326912,351.69818817)
\curveto(148.59326643,351.7181838)(148.73326629,351.74318378)(148.87326912,351.77318817)
\curveto(148.95326607,351.79318373)(149.03826599,351.80318372)(149.12826912,351.80318817)
\curveto(149.21826581,351.81318371)(149.30826572,351.82818369)(149.39826912,351.84818817)
\curveto(149.46826556,351.86818365)(149.53826549,351.87318365)(149.60826912,351.86318817)
\curveto(149.67826535,351.86318366)(149.75326527,351.87318365)(149.83326912,351.89318817)
\curveto(149.90326512,351.91318361)(149.97326505,351.9231836)(150.04326912,351.92318817)
\curveto(150.11326491,351.9231836)(150.18826484,351.93318359)(150.26826912,351.95318817)
\curveto(150.47826455,352.00318352)(150.66826436,352.04318348)(150.83826912,352.07318817)
\curveto(151.01826401,352.11318341)(151.17826385,352.20318332)(151.31826912,352.34318817)
\curveto(151.40826362,352.43318309)(151.46826356,352.53318299)(151.49826912,352.64318817)
\curveto(151.50826352,352.67318285)(151.50826352,352.69818282)(151.49826912,352.71818817)
\curveto(151.49826353,352.73818278)(151.50326352,352.75818276)(151.51326912,352.77818817)
\curveto(151.5232635,352.79818272)(151.5282635,352.82818269)(151.52826912,352.86818817)
\lineto(151.52826912,352.95818817)
\lineto(151.49826912,353.07818817)
\curveto(151.49826353,353.1181824)(151.49326353,353.15318237)(151.48326912,353.18318817)
\curveto(151.38326364,353.48318204)(151.17326385,353.68818183)(150.85326912,353.79818817)
\curveto(150.76326426,353.82818169)(150.65326437,353.84818167)(150.52326912,353.85818817)
\curveto(150.40326462,353.87818164)(150.27826475,353.88318164)(150.14826912,353.87318817)
\curveto(150.01826501,353.87318165)(149.89326513,353.86318166)(149.77326912,353.84318817)
\curveto(149.65326537,353.8231817)(149.54826548,353.79818172)(149.45826912,353.76818817)
\curveto(149.39826563,353.74818177)(149.33826569,353.7181818)(149.27826912,353.67818817)
\curveto(149.2282658,353.64818187)(149.17826585,353.61318191)(149.12826912,353.57318817)
\curveto(149.07826595,353.53318199)(149.023266,353.47818204)(148.96326912,353.40818817)
\curveto(148.91326611,353.33818218)(148.87826615,353.27318225)(148.85826912,353.21318817)
\curveto(148.80826622,353.11318241)(148.76326626,353.0181825)(148.72326912,352.92818817)
\curveto(148.69326633,352.83818268)(148.6232664,352.77818274)(148.51326912,352.74818817)
\curveto(148.43326659,352.72818279)(148.34826668,352.7181828)(148.25826912,352.71818817)
\lineto(147.98826912,352.71818817)
\lineto(147.41826912,352.71818817)
\curveto(147.36826766,352.7181828)(147.31826771,352.71318281)(147.26826912,352.70318817)
\curveto(147.21826781,352.70318282)(147.17326785,352.70818281)(147.13326912,352.71818817)
\lineto(146.99826912,352.71818817)
\curveto(146.97826805,352.72818279)(146.95326807,352.73318279)(146.92326912,352.73318817)
\curveto(146.89326813,352.73318279)(146.86826816,352.74318278)(146.84826912,352.76318817)
\curveto(146.76826826,352.78318274)(146.71326831,352.84818267)(146.68326912,352.95818817)
\curveto(146.67326835,353.00818251)(146.67326835,353.05818246)(146.68326912,353.10818817)
\curveto(146.69326833,353.15818236)(146.70326832,353.20318232)(146.71326912,353.24318817)
\curveto(146.74326828,353.35318217)(146.77326825,353.45318207)(146.80326912,353.54318817)
\curveto(146.84326818,353.64318188)(146.88826814,353.73318179)(146.93826912,353.81318817)
\lineto(147.02826912,353.96318817)
\lineto(147.11826912,354.11318817)
\curveto(147.19826783,354.2231813)(147.29826773,354.32818119)(147.41826912,354.42818817)
\curveto(147.43826759,354.43818108)(147.46826756,354.46318106)(147.50826912,354.50318817)
\curveto(147.55826747,354.54318098)(147.60326742,354.57818094)(147.64326912,354.60818817)
\curveto(147.68326734,354.63818088)(147.7282673,354.66818085)(147.77826912,354.69818817)
\curveto(147.94826708,354.80818071)(148.1282669,354.89318063)(148.31826912,354.95318817)
\curveto(148.50826652,355.0231805)(148.70326632,355.08818043)(148.90326912,355.14818817)
\curveto(149.023266,355.17818034)(149.14826588,355.19818032)(149.27826912,355.20818817)
\curveto(149.40826562,355.2181803)(149.53826549,355.23818028)(149.66826912,355.26818817)
\curveto(149.70826532,355.27818024)(149.76826526,355.27818024)(149.84826912,355.26818817)
\curveto(149.93826509,355.25818026)(149.99326503,355.26318026)(150.01326912,355.28318817)
\curveto(150.4232646,355.29318023)(150.81326421,355.27818024)(151.18326912,355.23818817)
\curveto(151.56326346,355.19818032)(151.90326312,355.1231804)(152.20326912,355.01318817)
\curveto(152.51326251,354.90318062)(152.77826225,354.75318077)(152.99826912,354.56318817)
\curveto(153.21826181,354.38318114)(153.38826164,354.14818137)(153.50826912,353.85818817)
\curveto(153.57826145,353.68818183)(153.61826141,353.49318203)(153.62826912,353.27318817)
\curveto(153.63826139,353.05318247)(153.64326138,352.82818269)(153.64326912,352.59818817)
\lineto(153.64326912,349.25318817)
\lineto(153.64326912,348.66818817)
\curveto(153.64326138,348.47818704)(153.66326136,348.30318722)(153.70326912,348.14318817)
\curveto(153.71326131,348.11318741)(153.71826131,348.07818744)(153.71826912,348.03818817)
\curveto(153.71826131,348.00818751)(153.7232613,347.97818754)(153.73326912,347.94818817)
\moveto(151.52826912,350.25818817)
\curveto(151.53826349,350.30818521)(151.54326348,350.36318516)(151.54326912,350.42318817)
\curveto(151.54326348,350.49318503)(151.53826349,350.55318497)(151.52826912,350.60318817)
\curveto(151.50826352,350.66318486)(151.49826353,350.7181848)(151.49826912,350.76818817)
\curveto(151.49826353,350.8181847)(151.47826355,350.85818466)(151.43826912,350.88818817)
\curveto(151.38826364,350.92818459)(151.31326371,350.94818457)(151.21326912,350.94818817)
\curveto(151.17326385,350.93818458)(151.13826389,350.92818459)(151.10826912,350.91818817)
\curveto(151.07826395,350.9181846)(151.04326398,350.91318461)(151.00326912,350.90318817)
\curveto(150.93326409,350.88318464)(150.85826417,350.86818465)(150.77826912,350.85818817)
\curveto(150.69826433,350.84818467)(150.61826441,350.83318469)(150.53826912,350.81318817)
\curveto(150.50826452,350.80318472)(150.46326456,350.79818472)(150.40326912,350.79818817)
\curveto(150.27326475,350.76818475)(150.14326488,350.74818477)(150.01326912,350.73818817)
\curveto(149.88326514,350.72818479)(149.75826527,350.70318482)(149.63826912,350.66318817)
\curveto(149.55826547,350.64318488)(149.48326554,350.6231849)(149.41326912,350.60318817)
\curveto(149.34326568,350.59318493)(149.27326575,350.57318495)(149.20326912,350.54318817)
\curveto(148.99326603,350.45318507)(148.81326621,350.3181852)(148.66326912,350.13818817)
\curveto(148.5232665,349.95818556)(148.47326655,349.70818581)(148.51326912,349.38818817)
\curveto(148.53326649,349.2181863)(148.58826644,349.07818644)(148.67826912,348.96818817)
\curveto(148.74826628,348.85818666)(148.85326617,348.76818675)(148.99326912,348.69818817)
\curveto(149.13326589,348.63818688)(149.28326574,348.59318693)(149.44326912,348.56318817)
\curveto(149.61326541,348.53318699)(149.78826524,348.523187)(149.96826912,348.53318817)
\curveto(150.15826487,348.55318697)(150.33326469,348.58818693)(150.49326912,348.63818817)
\curveto(150.75326427,348.7181868)(150.95826407,348.84318668)(151.10826912,349.01318817)
\curveto(151.25826377,349.19318633)(151.37326365,349.41318611)(151.45326912,349.67318817)
\curveto(151.47326355,349.74318578)(151.48326354,349.81318571)(151.48326912,349.88318817)
\curveto(151.49326353,349.96318556)(151.50826352,350.04318548)(151.52826912,350.12318817)
\lineto(151.52826912,350.25818817)
}
}
{
\newrgbcolor{curcolor}{0 0 0}
\pscustom[linestyle=none,fillstyle=solid,fillcolor=curcolor]
{
\newpath
\moveto(163.09655037,351.60818817)
\curveto(163.11654177,351.54818397)(163.12654176,351.44318408)(163.12655037,351.29318817)
\curveto(163.12654176,351.15318437)(163.12154177,351.05318447)(163.11155037,350.99318817)
\curveto(163.11154178,350.94318458)(163.10654178,350.89818462)(163.09655037,350.85818817)
\lineto(163.09655037,350.73818817)
\curveto(163.07654181,350.65818486)(163.06654182,350.57818494)(163.06655037,350.49818817)
\curveto(163.06654182,350.42818509)(163.05654183,350.35318517)(163.03655037,350.27318817)
\curveto(163.03654185,350.23318529)(163.02654186,350.16318536)(163.00655037,350.06318817)
\curveto(162.97654191,349.94318558)(162.94654194,349.8181857)(162.91655037,349.68818817)
\curveto(162.89654199,349.56818595)(162.86154203,349.45318607)(162.81155037,349.34318817)
\curveto(162.63154226,348.89318663)(162.40654248,348.50318702)(162.13655037,348.17318817)
\curveto(161.86654302,347.84318768)(161.51154338,347.58318794)(161.07155037,347.39318817)
\curveto(160.98154391,347.35318817)(160.886544,347.3231882)(160.78655037,347.30318817)
\curveto(160.69654419,347.27318825)(160.59654429,347.24318828)(160.48655037,347.21318817)
\curveto(160.42654446,347.19318833)(160.36154453,347.18318834)(160.29155037,347.18318817)
\curveto(160.23154466,347.18318834)(160.17154472,347.17818834)(160.11155037,347.16818817)
\lineto(159.97655037,347.16818817)
\curveto(159.91654497,347.14818837)(159.83654505,347.14318838)(159.73655037,347.15318817)
\curveto(159.63654525,347.15318837)(159.55654533,347.16318836)(159.49655037,347.18318817)
\lineto(159.40655037,347.18318817)
\curveto(159.35654553,347.19318833)(159.30154559,347.20318832)(159.24155037,347.21318817)
\curveto(159.18154571,347.21318831)(159.12154577,347.2181883)(159.06155037,347.22818817)
\curveto(158.87154602,347.27818824)(158.69654619,347.32818819)(158.53655037,347.37818817)
\curveto(158.37654651,347.42818809)(158.22654666,347.49818802)(158.08655037,347.58818817)
\lineto(157.90655037,347.70818817)
\curveto(157.85654703,347.74818777)(157.80654708,347.79318773)(157.75655037,347.84318817)
\lineto(157.66655037,347.90318817)
\curveto(157.63654725,347.9231876)(157.60654728,347.93818758)(157.57655037,347.94818817)
\curveto(157.4865474,347.97818754)(157.43154746,347.95818756)(157.41155037,347.88818817)
\curveto(157.36154753,347.8181877)(157.32654756,347.73318779)(157.30655037,347.63318817)
\curveto(157.29654759,347.54318798)(157.26154763,347.47318805)(157.20155037,347.42318817)
\curveto(157.14154775,347.38318814)(157.07154782,347.35818816)(156.99155037,347.34818817)
\lineto(156.72155037,347.34818817)
\lineto(156.00155037,347.34818817)
\lineto(155.77655037,347.34818817)
\curveto(155.70654918,347.33818818)(155.64154925,347.34318818)(155.58155037,347.36318817)
\curveto(155.44154945,347.41318811)(155.36154953,347.50318802)(155.34155037,347.63318817)
\curveto(155.33154956,347.77318775)(155.32654956,347.92818759)(155.32655037,348.09818817)
\lineto(155.32655037,357.24818817)
\lineto(155.32655037,357.59318817)
\curveto(155.32654956,357.71317781)(155.35154954,357.80817771)(155.40155037,357.87818817)
\curveto(155.44154945,357.94817757)(155.51154938,357.99317753)(155.61155037,358.01318817)
\curveto(155.63154926,358.0231775)(155.65154924,358.0231775)(155.67155037,358.01318817)
\curveto(155.70154919,358.01317751)(155.72654916,358.0181775)(155.74655037,358.02818817)
\lineto(156.69155037,358.02818817)
\curveto(156.87154802,358.02817749)(157.02654786,358.0181775)(157.15655037,357.99818817)
\curveto(157.2865476,357.98817753)(157.37154752,357.91317761)(157.41155037,357.77318817)
\curveto(157.44154745,357.67317785)(157.45154744,357.53817798)(157.44155037,357.36818817)
\curveto(157.43154746,357.20817831)(157.42654746,357.06817845)(157.42655037,356.94818817)
\lineto(157.42655037,355.31318817)
\lineto(157.42655037,354.98318817)
\curveto(157.42654746,354.87318065)(157.43654745,354.77818074)(157.45655037,354.69818817)
\curveto(157.46654742,354.64818087)(157.47654741,354.60318092)(157.48655037,354.56318817)
\curveto(157.49654739,354.53318099)(157.52154737,354.51318101)(157.56155037,354.50318817)
\curveto(157.58154731,354.48318104)(157.60654728,354.47318105)(157.63655037,354.47318817)
\curveto(157.67654721,354.47318105)(157.70654718,354.47818104)(157.72655037,354.48818817)
\curveto(157.79654709,354.52818099)(157.86154703,354.56818095)(157.92155037,354.60818817)
\curveto(157.98154691,354.65818086)(158.04654684,354.70818081)(158.11655037,354.75818817)
\curveto(158.24654664,354.84818067)(158.38154651,354.9231806)(158.52155037,354.98318817)
\curveto(158.66154623,355.05318047)(158.81654607,355.11318041)(158.98655037,355.16318817)
\curveto(159.06654582,355.19318033)(159.14654574,355.20818031)(159.22655037,355.20818817)
\curveto(159.30654558,355.2181803)(159.3865455,355.23318029)(159.46655037,355.25318817)
\curveto(159.53654535,355.27318025)(159.61154528,355.28318024)(159.69155037,355.28318817)
\lineto(159.93155037,355.28318817)
\lineto(160.08155037,355.28318817)
\curveto(160.11154478,355.27318025)(160.14654474,355.26818025)(160.18655037,355.26818817)
\curveto(160.22654466,355.27818024)(160.26654462,355.27818024)(160.30655037,355.26818817)
\curveto(160.41654447,355.23818028)(160.51654437,355.21318031)(160.60655037,355.19318817)
\curveto(160.70654418,355.18318034)(160.80154409,355.15818036)(160.89155037,355.11818817)
\curveto(161.35154354,354.92818059)(161.72654316,354.68318084)(162.01655037,354.38318817)
\curveto(162.30654258,354.08318144)(162.55154234,353.70818181)(162.75155037,353.25818817)
\curveto(162.80154209,353.13818238)(162.84154205,353.01318251)(162.87155037,352.88318817)
\curveto(162.91154198,352.75318277)(162.95154194,352.6181829)(162.99155037,352.47818817)
\curveto(163.01154188,352.40818311)(163.02154187,352.33818318)(163.02155037,352.26818817)
\curveto(163.03154186,352.20818331)(163.04654184,352.13818338)(163.06655037,352.05818817)
\curveto(163.0865418,352.00818351)(163.0915418,351.95318357)(163.08155037,351.89318817)
\curveto(163.08154181,351.83318369)(163.0865418,351.77318375)(163.09655037,351.71318817)
\lineto(163.09655037,351.60818817)
\moveto(160.87655037,350.19818817)
\curveto(160.90654398,350.29818522)(160.93154396,350.4231851)(160.95155037,350.57318817)
\curveto(160.98154391,350.7231848)(160.99654389,350.87318465)(160.99655037,351.02318817)
\curveto(161.00654388,351.18318434)(161.00654388,351.33818418)(160.99655037,351.48818817)
\curveto(160.99654389,351.64818387)(160.98154391,351.78318374)(160.95155037,351.89318817)
\curveto(160.92154397,351.99318353)(160.90154399,352.08818343)(160.89155037,352.17818817)
\curveto(160.88154401,352.26818325)(160.85654403,352.35318317)(160.81655037,352.43318817)
\curveto(160.67654421,352.78318274)(160.47654441,353.07818244)(160.21655037,353.31818817)
\curveto(159.96654492,353.56818195)(159.59654529,353.69318183)(159.10655037,353.69318817)
\curveto(159.06654582,353.69318183)(159.03154586,353.68818183)(159.00155037,353.67818817)
\lineto(158.89655037,353.67818817)
\curveto(158.82654606,353.65818186)(158.76154613,353.63818188)(158.70155037,353.61818817)
\curveto(158.64154625,353.60818191)(158.58154631,353.59318193)(158.52155037,353.57318817)
\curveto(158.23154666,353.44318208)(158.01154688,353.25818226)(157.86155037,353.01818817)
\curveto(157.71154718,352.78818273)(157.5865473,352.523183)(157.48655037,352.22318817)
\curveto(157.45654743,352.14318338)(157.43654745,352.05818346)(157.42655037,351.96818817)
\curveto(157.42654746,351.88818363)(157.41654747,351.80818371)(157.39655037,351.72818817)
\curveto(157.3865475,351.69818382)(157.38154751,351.64818387)(157.38155037,351.57818817)
\curveto(157.37154752,351.53818398)(157.36654752,351.49818402)(157.36655037,351.45818817)
\curveto(157.37654751,351.4181841)(157.37654751,351.37818414)(157.36655037,351.33818817)
\curveto(157.34654754,351.25818426)(157.34154755,351.14818437)(157.35155037,351.00818817)
\curveto(157.36154753,350.86818465)(157.37654751,350.76818475)(157.39655037,350.70818817)
\curveto(157.41654747,350.6181849)(157.42654746,350.53318499)(157.42655037,350.45318817)
\curveto(157.43654745,350.37318515)(157.45654743,350.29318523)(157.48655037,350.21318817)
\curveto(157.57654731,349.93318559)(157.68154721,349.68818583)(157.80155037,349.47818817)
\curveto(157.93154696,349.27818624)(158.11154678,349.10818641)(158.34155037,348.96818817)
\curveto(158.50154639,348.86818665)(158.66654622,348.79818672)(158.83655037,348.75818817)
\curveto(158.85654603,348.75818676)(158.87654601,348.75318677)(158.89655037,348.74318817)
\lineto(158.98655037,348.74318817)
\curveto(159.01654587,348.73318679)(159.06654582,348.7231868)(159.13655037,348.71318817)
\curveto(159.20654568,348.71318681)(159.26654562,348.7181868)(159.31655037,348.72818817)
\curveto(159.41654547,348.74818677)(159.50654538,348.76318676)(159.58655037,348.77318817)
\curveto(159.67654521,348.79318673)(159.76154513,348.8181867)(159.84155037,348.84818817)
\curveto(160.12154477,348.97818654)(160.33654455,349.15818636)(160.48655037,349.38818817)
\curveto(160.64654424,349.6181859)(160.77654411,349.88818563)(160.87655037,350.19818817)
}
}
{
\newrgbcolor{curcolor}{0 0 0}
\pscustom[linestyle=none,fillstyle=solid,fillcolor=curcolor]
{
\newpath
\moveto(164.97647225,358.04318817)
\lineto(166.07147225,358.04318817)
\curveto(166.17146976,358.04317748)(166.26646967,358.03817748)(166.35647225,358.02818817)
\curveto(166.44646949,358.0181775)(166.51646942,357.98817753)(166.56647225,357.93818817)
\curveto(166.62646931,357.86817765)(166.65646928,357.77317775)(166.65647225,357.65318817)
\curveto(166.66646927,357.54317798)(166.67146926,357.42817809)(166.67147225,357.30818817)
\lineto(166.67147225,355.97318817)
\lineto(166.67147225,350.58818817)
\lineto(166.67147225,348.29318817)
\lineto(166.67147225,347.87318817)
\curveto(166.68146925,347.7231878)(166.66146927,347.60818791)(166.61147225,347.52818817)
\curveto(166.56146937,347.44818807)(166.47146946,347.39318813)(166.34147225,347.36318817)
\curveto(166.28146965,347.34318818)(166.21146972,347.33818818)(166.13147225,347.34818817)
\curveto(166.06146987,347.35818816)(165.99146994,347.36318816)(165.92147225,347.36318817)
\lineto(165.20147225,347.36318817)
\curveto(165.09147084,347.36318816)(164.99147094,347.36818815)(164.90147225,347.37818817)
\curveto(164.81147112,347.38818813)(164.7364712,347.4181881)(164.67647225,347.46818817)
\curveto(164.61647132,347.518188)(164.58147135,347.59318793)(164.57147225,347.69318817)
\lineto(164.57147225,348.02318817)
\lineto(164.57147225,349.35818817)
\lineto(164.57147225,354.98318817)
\lineto(164.57147225,357.02318817)
\curveto(164.57147136,357.15317837)(164.56647137,357.30817821)(164.55647225,357.48818817)
\curveto(164.55647138,357.66817785)(164.58147135,357.79817772)(164.63147225,357.87818817)
\curveto(164.65147128,357.9181776)(164.67647126,357.94817757)(164.70647225,357.96818817)
\lineto(164.82647225,358.02818817)
\curveto(164.84647109,358.02817749)(164.87147106,358.02817749)(164.90147225,358.02818817)
\curveto(164.931471,358.03817748)(164.95647098,358.04317748)(164.97647225,358.04318817)
}
}
{
\newrgbcolor{curcolor}{0 0 0}
\pscustom[linestyle=none,fillstyle=solid,fillcolor=curcolor]
{
\newpath
\moveto(175.69865975,351.29318817)
\curveto(175.71865158,351.21318431)(175.71865158,351.1231844)(175.69865975,351.02318817)
\curveto(175.67865162,350.9231846)(175.64365166,350.85818466)(175.59365975,350.82818817)
\curveto(175.54365176,350.78818473)(175.46865183,350.75818476)(175.36865975,350.73818817)
\curveto(175.27865202,350.72818479)(175.17365213,350.7181848)(175.05365975,350.70818817)
\lineto(174.70865975,350.70818817)
\curveto(174.5986527,350.7181848)(174.4986528,350.7231848)(174.40865975,350.72318817)
\lineto(170.74865975,350.72318817)
\lineto(170.53865975,350.72318817)
\curveto(170.47865682,350.7231848)(170.42365688,350.71318481)(170.37365975,350.69318817)
\curveto(170.29365701,350.65318487)(170.24365706,350.61318491)(170.22365975,350.57318817)
\curveto(170.2036571,350.55318497)(170.18365712,350.51318501)(170.16365975,350.45318817)
\curveto(170.14365716,350.40318512)(170.13865716,350.35318517)(170.14865975,350.30318817)
\curveto(170.16865713,350.24318528)(170.17865712,350.18318534)(170.17865975,350.12318817)
\curveto(170.18865711,350.07318545)(170.2036571,350.0181855)(170.22365975,349.95818817)
\curveto(170.303657,349.7181858)(170.3986569,349.518186)(170.50865975,349.35818817)
\curveto(170.62865667,349.20818631)(170.78865651,349.07318645)(170.98865975,348.95318817)
\curveto(171.06865623,348.90318662)(171.14865615,348.86818665)(171.22865975,348.84818817)
\curveto(171.31865598,348.83818668)(171.40865589,348.8181867)(171.49865975,348.78818817)
\curveto(171.57865572,348.76818675)(171.68865561,348.75318677)(171.82865975,348.74318817)
\curveto(171.96865533,348.73318679)(172.08865521,348.73818678)(172.18865975,348.75818817)
\lineto(172.32365975,348.75818817)
\curveto(172.42365488,348.77818674)(172.51365479,348.79818672)(172.59365975,348.81818817)
\curveto(172.68365462,348.84818667)(172.76865453,348.87818664)(172.84865975,348.90818817)
\curveto(172.94865435,348.95818656)(173.05865424,349.0231865)(173.17865975,349.10318817)
\curveto(173.30865399,349.18318634)(173.4036539,349.26318626)(173.46365975,349.34318817)
\curveto(173.51365379,349.41318611)(173.56365374,349.47818604)(173.61365975,349.53818817)
\curveto(173.67365363,349.60818591)(173.74365356,349.65818586)(173.82365975,349.68818817)
\curveto(173.92365338,349.73818578)(174.04865325,349.75818576)(174.19865975,349.74818817)
\lineto(174.63365975,349.74818817)
\lineto(174.81365975,349.74818817)
\curveto(174.88365242,349.75818576)(174.94365236,349.75318577)(174.99365975,349.73318817)
\lineto(175.14365975,349.73318817)
\curveto(175.24365206,349.71318581)(175.31365199,349.68818583)(175.35365975,349.65818817)
\curveto(175.39365191,349.63818588)(175.41365189,349.59318593)(175.41365975,349.52318817)
\curveto(175.42365188,349.45318607)(175.41865188,349.39318613)(175.39865975,349.34318817)
\curveto(175.34865195,349.20318632)(175.29365201,349.07818644)(175.23365975,348.96818817)
\curveto(175.17365213,348.85818666)(175.1036522,348.74818677)(175.02365975,348.63818817)
\curveto(174.8036525,348.30818721)(174.55365275,348.04318748)(174.27365975,347.84318817)
\curveto(173.99365331,347.64318788)(173.64365366,347.47318805)(173.22365975,347.33318817)
\curveto(173.11365419,347.29318823)(173.0036543,347.26818825)(172.89365975,347.25818817)
\curveto(172.78365452,347.24818827)(172.66865463,347.22818829)(172.54865975,347.19818817)
\curveto(172.50865479,347.18818833)(172.46365484,347.18818833)(172.41365975,347.19818817)
\curveto(172.37365493,347.19818832)(172.33365497,347.19318833)(172.29365975,347.18318817)
\lineto(172.12865975,347.18318817)
\curveto(172.07865522,347.16318836)(172.01865528,347.15818836)(171.94865975,347.16818817)
\curveto(171.88865541,347.16818835)(171.83365547,347.17318835)(171.78365975,347.18318817)
\curveto(171.7036556,347.19318833)(171.63365567,347.19318833)(171.57365975,347.18318817)
\curveto(171.51365579,347.17318835)(171.44865585,347.17818834)(171.37865975,347.19818817)
\curveto(171.32865597,347.2181883)(171.27365603,347.22818829)(171.21365975,347.22818817)
\curveto(171.15365615,347.22818829)(171.0986562,347.23818828)(171.04865975,347.25818817)
\curveto(170.93865636,347.27818824)(170.82865647,347.30318822)(170.71865975,347.33318817)
\curveto(170.60865669,347.35318817)(170.50865679,347.38818813)(170.41865975,347.43818817)
\curveto(170.30865699,347.47818804)(170.2036571,347.51318801)(170.10365975,347.54318817)
\curveto(170.01365729,347.58318794)(169.92865737,347.62818789)(169.84865975,347.67818817)
\curveto(169.52865777,347.87818764)(169.24365806,348.10818741)(168.99365975,348.36818817)
\curveto(168.74365856,348.63818688)(168.53865876,348.94818657)(168.37865975,349.29818817)
\curveto(168.32865897,349.40818611)(168.28865901,349.518186)(168.25865975,349.62818817)
\curveto(168.22865907,349.74818577)(168.18865911,349.86818565)(168.13865975,349.98818817)
\curveto(168.12865917,350.02818549)(168.12365918,350.06318546)(168.12365975,350.09318817)
\curveto(168.12365918,350.13318539)(168.11865918,350.17318535)(168.10865975,350.21318817)
\curveto(168.06865923,350.33318519)(168.04365926,350.46318506)(168.03365975,350.60318817)
\lineto(168.00365975,351.02318817)
\curveto(168.0036593,351.07318445)(167.9986593,351.12818439)(167.98865975,351.18818817)
\curveto(167.98865931,351.24818427)(167.99365931,351.30318422)(168.00365975,351.35318817)
\lineto(168.00365975,351.53318817)
\lineto(168.04865975,351.89318817)
\curveto(168.08865921,352.06318346)(168.12365918,352.22818329)(168.15365975,352.38818817)
\curveto(168.18365912,352.54818297)(168.22865907,352.69818282)(168.28865975,352.83818817)
\curveto(168.71865858,353.87818164)(169.44865785,354.61318091)(170.47865975,355.04318817)
\curveto(170.61865668,355.10318042)(170.75865654,355.14318038)(170.89865975,355.16318817)
\curveto(171.04865625,355.19318033)(171.2036561,355.22818029)(171.36365975,355.26818817)
\curveto(171.44365586,355.27818024)(171.51865578,355.28318024)(171.58865975,355.28318817)
\curveto(171.65865564,355.28318024)(171.73365557,355.28818023)(171.81365975,355.29818817)
\curveto(172.32365498,355.30818021)(172.75865454,355.24818027)(173.11865975,355.11818817)
\curveto(173.48865381,354.99818052)(173.81865348,354.83818068)(174.10865975,354.63818817)
\curveto(174.1986531,354.57818094)(174.28865301,354.50818101)(174.37865975,354.42818817)
\curveto(174.46865283,354.35818116)(174.54865275,354.28318124)(174.61865975,354.20318817)
\curveto(174.64865265,354.15318137)(174.68865261,354.11318141)(174.73865975,354.08318817)
\curveto(174.81865248,353.97318155)(174.89365241,353.85818166)(174.96365975,353.73818817)
\curveto(175.03365227,353.62818189)(175.10865219,353.51318201)(175.18865975,353.39318817)
\curveto(175.23865206,353.30318222)(175.27865202,353.20818231)(175.30865975,353.10818817)
\curveto(175.34865195,353.0181825)(175.38865191,352.9181826)(175.42865975,352.80818817)
\curveto(175.47865182,352.67818284)(175.51865178,352.54318298)(175.54865975,352.40318817)
\curveto(175.57865172,352.26318326)(175.61365169,352.1231834)(175.65365975,351.98318817)
\curveto(175.67365163,351.90318362)(175.67865162,351.81318371)(175.66865975,351.71318817)
\curveto(175.66865163,351.6231839)(175.67865162,351.53818398)(175.69865975,351.45818817)
\lineto(175.69865975,351.29318817)
\moveto(173.44865975,352.17818817)
\curveto(173.51865378,352.27818324)(173.52365378,352.39818312)(173.46365975,352.53818817)
\curveto(173.41365389,352.68818283)(173.37365393,352.79818272)(173.34365975,352.86818817)
\curveto(173.2036541,353.13818238)(173.01865428,353.34318218)(172.78865975,353.48318817)
\curveto(172.55865474,353.63318189)(172.23865506,353.71318181)(171.82865975,353.72318817)
\curveto(171.7986555,353.70318182)(171.76365554,353.69818182)(171.72365975,353.70818817)
\curveto(171.68365562,353.7181818)(171.64865565,353.7181818)(171.61865975,353.70818817)
\curveto(171.56865573,353.68818183)(171.51365579,353.67318185)(171.45365975,353.66318817)
\curveto(171.39365591,353.66318186)(171.33865596,353.65318187)(171.28865975,353.63318817)
\curveto(170.84865645,353.49318203)(170.52365678,353.2181823)(170.31365975,352.80818817)
\curveto(170.29365701,352.76818275)(170.26865703,352.71318281)(170.23865975,352.64318817)
\curveto(170.21865708,352.58318294)(170.2036571,352.518183)(170.19365975,352.44818817)
\curveto(170.18365712,352.38818313)(170.18365712,352.32818319)(170.19365975,352.26818817)
\curveto(170.21365709,352.20818331)(170.24865705,352.15818336)(170.29865975,352.11818817)
\curveto(170.37865692,352.06818345)(170.48865681,352.04318348)(170.62865975,352.04318817)
\lineto(171.03365975,352.04318817)
\lineto(172.69865975,352.04318817)
\lineto(173.13365975,352.04318817)
\curveto(173.29365401,352.05318347)(173.3986539,352.09818342)(173.44865975,352.17818817)
}
}
{
\newrgbcolor{curcolor}{0 0 0}
\pscustom[linestyle=none,fillstyle=solid,fillcolor=curcolor]
{
\newpath
\moveto(180.516941,355.29818817)
\curveto(181.32693584,355.3181802)(182.00193516,355.19818032)(182.541941,354.93818817)
\curveto(183.09193407,354.67818084)(183.52693364,354.30818121)(183.846941,353.82818817)
\curveto(184.00693316,353.58818193)(184.12693304,353.31318221)(184.206941,353.00318817)
\curveto(184.22693294,352.95318257)(184.24193292,352.88818263)(184.251941,352.80818817)
\curveto(184.27193289,352.72818279)(184.27193289,352.65818286)(184.251941,352.59818817)
\curveto(184.21193295,352.48818303)(184.14193302,352.4231831)(184.041941,352.40318817)
\curveto(183.94193322,352.39318313)(183.82193334,352.38818313)(183.681941,352.38818817)
\lineto(182.901941,352.38818817)
\lineto(182.616941,352.38818817)
\curveto(182.52693464,352.38818313)(182.45193471,352.40818311)(182.391941,352.44818817)
\curveto(182.31193485,352.48818303)(182.25693491,352.54818297)(182.226941,352.62818817)
\curveto(182.19693497,352.7181828)(182.15693501,352.80818271)(182.106941,352.89818817)
\curveto(182.04693512,353.00818251)(181.98193518,353.10818241)(181.911941,353.19818817)
\curveto(181.84193532,353.28818223)(181.7619354,353.36818215)(181.671941,353.43818817)
\curveto(181.53193563,353.52818199)(181.37693579,353.59818192)(181.206941,353.64818817)
\curveto(181.14693602,353.66818185)(181.08693608,353.67818184)(181.026941,353.67818817)
\curveto(180.9669362,353.67818184)(180.91193625,353.68818183)(180.861941,353.70818817)
\lineto(180.711941,353.70818817)
\curveto(180.51193665,353.70818181)(180.35193681,353.68818183)(180.231941,353.64818817)
\curveto(179.94193722,353.55818196)(179.70693746,353.4181821)(179.526941,353.22818817)
\curveto(179.34693782,353.04818247)(179.20193796,352.82818269)(179.091941,352.56818817)
\curveto(179.04193812,352.45818306)(179.00193816,352.33818318)(178.971941,352.20818817)
\curveto(178.95193821,352.08818343)(178.92693824,351.95818356)(178.896941,351.81818817)
\curveto(178.88693828,351.77818374)(178.88193828,351.73818378)(178.881941,351.69818817)
\curveto(178.88193828,351.65818386)(178.87693829,351.6181839)(178.866941,351.57818817)
\curveto(178.84693832,351.47818404)(178.83693833,351.33818418)(178.836941,351.15818817)
\curveto(178.84693832,350.97818454)(178.8619383,350.83818468)(178.881941,350.73818817)
\curveto(178.88193828,350.65818486)(178.88693828,350.60318492)(178.896941,350.57318817)
\curveto(178.91693825,350.50318502)(178.92693824,350.43318509)(178.926941,350.36318817)
\curveto(178.93693823,350.29318523)(178.95193821,350.2231853)(178.971941,350.15318817)
\curveto(179.05193811,349.9231856)(179.14693802,349.71318581)(179.256941,349.52318817)
\curveto(179.3669378,349.33318619)(179.50693766,349.17318635)(179.676941,349.04318817)
\curveto(179.71693745,349.01318651)(179.77693739,348.97818654)(179.856941,348.93818817)
\curveto(179.9669372,348.86818665)(180.07693709,348.8231867)(180.186941,348.80318817)
\curveto(180.30693686,348.78318674)(180.45193671,348.76318676)(180.621941,348.74318817)
\lineto(180.711941,348.74318817)
\curveto(180.75193641,348.74318678)(180.78193638,348.74818677)(180.801941,348.75818817)
\lineto(180.936941,348.75818817)
\curveto(181.00693616,348.77818674)(181.07193609,348.79318673)(181.131941,348.80318817)
\curveto(181.20193596,348.8231867)(181.2669359,348.84318668)(181.326941,348.86318817)
\curveto(181.62693554,348.99318653)(181.85693531,349.18318634)(182.016941,349.43318817)
\curveto(182.05693511,349.48318604)(182.09193507,349.53818598)(182.121941,349.59818817)
\curveto(182.15193501,349.66818585)(182.17693499,349.72818579)(182.196941,349.77818817)
\curveto(182.23693493,349.88818563)(182.27193489,349.98318554)(182.301941,350.06318817)
\curveto(182.33193483,350.15318537)(182.40193476,350.2231853)(182.511941,350.27318817)
\curveto(182.60193456,350.31318521)(182.74693442,350.32818519)(182.946941,350.31818817)
\lineto(183.441941,350.31818817)
\lineto(183.651941,350.31818817)
\curveto(183.73193343,350.32818519)(183.79693337,350.3231852)(183.846941,350.30318817)
\lineto(183.966941,350.30318817)
\lineto(184.086941,350.27318817)
\curveto(184.12693304,350.27318525)(184.15693301,350.26318526)(184.176941,350.24318817)
\curveto(184.22693294,350.20318532)(184.25693291,350.14318538)(184.266941,350.06318817)
\curveto(184.28693288,349.99318553)(184.28693288,349.9181856)(184.266941,349.83818817)
\curveto(184.17693299,349.50818601)(184.0669331,349.21318631)(183.936941,348.95318817)
\curveto(183.52693364,348.18318734)(182.87193429,347.64818787)(181.971941,347.34818817)
\curveto(181.87193529,347.3181882)(181.7669354,347.29818822)(181.656941,347.28818817)
\curveto(181.54693562,347.26818825)(181.43693573,347.24318828)(181.326941,347.21318817)
\curveto(181.2669359,347.20318832)(181.20693596,347.19818832)(181.146941,347.19818817)
\curveto(181.08693608,347.19818832)(181.02693614,347.19318833)(180.966941,347.18318817)
\lineto(180.801941,347.18318817)
\curveto(180.75193641,347.16318836)(180.67693649,347.15818836)(180.576941,347.16818817)
\curveto(180.47693669,347.16818835)(180.40193676,347.17318835)(180.351941,347.18318817)
\curveto(180.27193689,347.20318832)(180.19693697,347.21318831)(180.126941,347.21318817)
\curveto(180.0669371,347.20318832)(180.00193716,347.20818831)(179.931941,347.22818817)
\lineto(179.781941,347.25818817)
\curveto(179.73193743,347.25818826)(179.68193748,347.26318826)(179.631941,347.27318817)
\curveto(179.52193764,347.30318822)(179.41693775,347.33318819)(179.316941,347.36318817)
\curveto(179.21693795,347.39318813)(179.12193804,347.42818809)(179.031941,347.46818817)
\curveto(178.5619386,347.66818785)(178.166939,347.9231876)(177.846941,348.23318817)
\curveto(177.52693964,348.55318697)(177.2669399,348.94818657)(177.066941,349.41818817)
\curveto(177.01694015,349.50818601)(176.97694019,349.60318592)(176.946941,349.70318817)
\lineto(176.856941,350.03318817)
\curveto(176.84694032,350.07318545)(176.84194032,350.10818541)(176.841941,350.13818817)
\curveto(176.84194032,350.17818534)(176.83194033,350.2231853)(176.811941,350.27318817)
\curveto(176.79194037,350.34318518)(176.78194038,350.41318511)(176.781941,350.48318817)
\curveto(176.78194038,350.56318496)(176.77194039,350.63818488)(176.751941,350.70818817)
\lineto(176.751941,350.96318817)
\curveto(176.73194043,351.01318451)(176.72194044,351.06818445)(176.721941,351.12818817)
\curveto(176.72194044,351.19818432)(176.73194043,351.25818426)(176.751941,351.30818817)
\curveto(176.7619404,351.35818416)(176.7619404,351.40318412)(176.751941,351.44318817)
\curveto(176.74194042,351.48318404)(176.74194042,351.523184)(176.751941,351.56318817)
\curveto(176.77194039,351.63318389)(176.77694039,351.69818382)(176.766941,351.75818817)
\curveto(176.7669404,351.8181837)(176.77694039,351.87818364)(176.796941,351.93818817)
\curveto(176.84694032,352.1181834)(176.88694028,352.28818323)(176.916941,352.44818817)
\curveto(176.94694022,352.6181829)(176.99194017,352.78318274)(177.051941,352.94318817)
\curveto(177.27193989,353.45318207)(177.54693962,353.87818164)(177.876941,354.21818817)
\curveto(178.21693895,354.55818096)(178.64693852,354.83318069)(179.166941,355.04318817)
\curveto(179.30693786,355.10318042)(179.45193771,355.14318038)(179.601941,355.16318817)
\curveto(179.75193741,355.19318033)(179.90693726,355.22818029)(180.066941,355.26818817)
\curveto(180.14693702,355.27818024)(180.22193694,355.28318024)(180.291941,355.28318817)
\curveto(180.3619368,355.28318024)(180.43693673,355.28818023)(180.516941,355.29818817)
}
}
{
\newrgbcolor{curcolor}{0 0 0}
\pscustom[linestyle=none,fillstyle=solid,fillcolor=curcolor]
{
\newpath
\moveto(187.66022225,357.93818817)
\curveto(187.7302193,357.85817766)(187.76521926,357.73817778)(187.76522225,357.57818817)
\lineto(187.76522225,357.11318817)
\lineto(187.76522225,356.70818817)
\curveto(187.76521926,356.56817895)(187.7302193,356.47317905)(187.66022225,356.42318817)
\curveto(187.60021943,356.37317915)(187.52021951,356.34317918)(187.42022225,356.33318817)
\curveto(187.3302197,356.3231792)(187.2302198,356.3181792)(187.12022225,356.31818817)
\lineto(186.28022225,356.31818817)
\curveto(186.17022086,356.3181792)(186.07022096,356.3231792)(185.98022225,356.33318817)
\curveto(185.90022113,356.34317918)(185.8302212,356.37317915)(185.77022225,356.42318817)
\curveto(185.7302213,356.45317907)(185.70022133,356.50817901)(185.68022225,356.58818817)
\curveto(185.67022136,356.67817884)(185.66022137,356.77317875)(185.65022225,356.87318817)
\lineto(185.65022225,357.20318817)
\curveto(185.66022137,357.31317821)(185.66522136,357.40817811)(185.66522225,357.48818817)
\lineto(185.66522225,357.69818817)
\curveto(185.67522135,357.76817775)(185.69522133,357.82817769)(185.72522225,357.87818817)
\curveto(185.74522128,357.9181776)(185.77022126,357.94817757)(185.80022225,357.96818817)
\lineto(185.92022225,358.02818817)
\curveto(185.94022109,358.02817749)(185.96522106,358.02817749)(185.99522225,358.02818817)
\curveto(186.025221,358.03817748)(186.05022098,358.04317748)(186.07022225,358.04318817)
\lineto(187.16522225,358.04318817)
\curveto(187.26521976,358.04317748)(187.36021967,358.03817748)(187.45022225,358.02818817)
\curveto(187.54021949,358.0181775)(187.61021942,357.98817753)(187.66022225,357.93818817)
\moveto(187.76522225,348.17318817)
\curveto(187.76521926,347.97318755)(187.76021927,347.80318772)(187.75022225,347.66318817)
\curveto(187.74021929,347.523188)(187.65021938,347.42818809)(187.48022225,347.37818817)
\curveto(187.42021961,347.35818816)(187.35521967,347.34818817)(187.28522225,347.34818817)
\curveto(187.21521981,347.35818816)(187.14021989,347.36318816)(187.06022225,347.36318817)
\lineto(186.22022225,347.36318817)
\curveto(186.1302209,347.36318816)(186.04022099,347.36818815)(185.95022225,347.37818817)
\curveto(185.87022116,347.38818813)(185.81022122,347.4181881)(185.77022225,347.46818817)
\curveto(185.71022132,347.53818798)(185.67522135,347.6231879)(185.66522225,347.72318817)
\lineto(185.66522225,348.06818817)
\lineto(185.66522225,354.39818817)
\lineto(185.66522225,354.69818817)
\curveto(185.66522136,354.79818072)(185.68522134,354.87818064)(185.72522225,354.93818817)
\curveto(185.78522124,355.00818051)(185.87022116,355.05318047)(185.98022225,355.07318817)
\curveto(186.00022103,355.08318044)(186.025221,355.08318044)(186.05522225,355.07318817)
\curveto(186.09522093,355.07318045)(186.1252209,355.07818044)(186.14522225,355.08818817)
\lineto(186.89522225,355.08818817)
\lineto(187.09022225,355.08818817)
\curveto(187.17021986,355.09818042)(187.23521979,355.09818042)(187.28522225,355.08818817)
\lineto(187.40522225,355.08818817)
\curveto(187.46521956,355.06818045)(187.52021951,355.05318047)(187.57022225,355.04318817)
\curveto(187.62021941,355.03318049)(187.66021937,355.00318052)(187.69022225,354.95318817)
\curveto(187.7302193,354.90318062)(187.75021928,354.83318069)(187.75022225,354.74318817)
\curveto(187.76021927,354.65318087)(187.76521926,354.55818096)(187.76522225,354.45818817)
\lineto(187.76522225,348.17318817)
}
}
{
\newrgbcolor{curcolor}{0 0 0}
\pscustom[linestyle=none,fillstyle=solid,fillcolor=curcolor]
{
\newpath
\moveto(196.79240975,351.29318817)
\curveto(196.81240158,351.21318431)(196.81240158,351.1231844)(196.79240975,351.02318817)
\curveto(196.77240162,350.9231846)(196.73740166,350.85818466)(196.68740975,350.82818817)
\curveto(196.63740176,350.78818473)(196.56240183,350.75818476)(196.46240975,350.73818817)
\curveto(196.37240202,350.72818479)(196.26740213,350.7181848)(196.14740975,350.70818817)
\lineto(195.80240975,350.70818817)
\curveto(195.6924027,350.7181848)(195.5924028,350.7231848)(195.50240975,350.72318817)
\lineto(191.84240975,350.72318817)
\lineto(191.63240975,350.72318817)
\curveto(191.57240682,350.7231848)(191.51740688,350.71318481)(191.46740975,350.69318817)
\curveto(191.38740701,350.65318487)(191.33740706,350.61318491)(191.31740975,350.57318817)
\curveto(191.2974071,350.55318497)(191.27740712,350.51318501)(191.25740975,350.45318817)
\curveto(191.23740716,350.40318512)(191.23240716,350.35318517)(191.24240975,350.30318817)
\curveto(191.26240713,350.24318528)(191.27240712,350.18318534)(191.27240975,350.12318817)
\curveto(191.28240711,350.07318545)(191.2974071,350.0181855)(191.31740975,349.95818817)
\curveto(191.397407,349.7181858)(191.4924069,349.518186)(191.60240975,349.35818817)
\curveto(191.72240667,349.20818631)(191.88240651,349.07318645)(192.08240975,348.95318817)
\curveto(192.16240623,348.90318662)(192.24240615,348.86818665)(192.32240975,348.84818817)
\curveto(192.41240598,348.83818668)(192.50240589,348.8181867)(192.59240975,348.78818817)
\curveto(192.67240572,348.76818675)(192.78240561,348.75318677)(192.92240975,348.74318817)
\curveto(193.06240533,348.73318679)(193.18240521,348.73818678)(193.28240975,348.75818817)
\lineto(193.41740975,348.75818817)
\curveto(193.51740488,348.77818674)(193.60740479,348.79818672)(193.68740975,348.81818817)
\curveto(193.77740462,348.84818667)(193.86240453,348.87818664)(193.94240975,348.90818817)
\curveto(194.04240435,348.95818656)(194.15240424,349.0231865)(194.27240975,349.10318817)
\curveto(194.40240399,349.18318634)(194.4974039,349.26318626)(194.55740975,349.34318817)
\curveto(194.60740379,349.41318611)(194.65740374,349.47818604)(194.70740975,349.53818817)
\curveto(194.76740363,349.60818591)(194.83740356,349.65818586)(194.91740975,349.68818817)
\curveto(195.01740338,349.73818578)(195.14240325,349.75818576)(195.29240975,349.74818817)
\lineto(195.72740975,349.74818817)
\lineto(195.90740975,349.74818817)
\curveto(195.97740242,349.75818576)(196.03740236,349.75318577)(196.08740975,349.73318817)
\lineto(196.23740975,349.73318817)
\curveto(196.33740206,349.71318581)(196.40740199,349.68818583)(196.44740975,349.65818817)
\curveto(196.48740191,349.63818588)(196.50740189,349.59318593)(196.50740975,349.52318817)
\curveto(196.51740188,349.45318607)(196.51240188,349.39318613)(196.49240975,349.34318817)
\curveto(196.44240195,349.20318632)(196.38740201,349.07818644)(196.32740975,348.96818817)
\curveto(196.26740213,348.85818666)(196.1974022,348.74818677)(196.11740975,348.63818817)
\curveto(195.8974025,348.30818721)(195.64740275,348.04318748)(195.36740975,347.84318817)
\curveto(195.08740331,347.64318788)(194.73740366,347.47318805)(194.31740975,347.33318817)
\curveto(194.20740419,347.29318823)(194.0974043,347.26818825)(193.98740975,347.25818817)
\curveto(193.87740452,347.24818827)(193.76240463,347.22818829)(193.64240975,347.19818817)
\curveto(193.60240479,347.18818833)(193.55740484,347.18818833)(193.50740975,347.19818817)
\curveto(193.46740493,347.19818832)(193.42740497,347.19318833)(193.38740975,347.18318817)
\lineto(193.22240975,347.18318817)
\curveto(193.17240522,347.16318836)(193.11240528,347.15818836)(193.04240975,347.16818817)
\curveto(192.98240541,347.16818835)(192.92740547,347.17318835)(192.87740975,347.18318817)
\curveto(192.7974056,347.19318833)(192.72740567,347.19318833)(192.66740975,347.18318817)
\curveto(192.60740579,347.17318835)(192.54240585,347.17818834)(192.47240975,347.19818817)
\curveto(192.42240597,347.2181883)(192.36740603,347.22818829)(192.30740975,347.22818817)
\curveto(192.24740615,347.22818829)(192.1924062,347.23818828)(192.14240975,347.25818817)
\curveto(192.03240636,347.27818824)(191.92240647,347.30318822)(191.81240975,347.33318817)
\curveto(191.70240669,347.35318817)(191.60240679,347.38818813)(191.51240975,347.43818817)
\curveto(191.40240699,347.47818804)(191.2974071,347.51318801)(191.19740975,347.54318817)
\curveto(191.10740729,347.58318794)(191.02240737,347.62818789)(190.94240975,347.67818817)
\curveto(190.62240777,347.87818764)(190.33740806,348.10818741)(190.08740975,348.36818817)
\curveto(189.83740856,348.63818688)(189.63240876,348.94818657)(189.47240975,349.29818817)
\curveto(189.42240897,349.40818611)(189.38240901,349.518186)(189.35240975,349.62818817)
\curveto(189.32240907,349.74818577)(189.28240911,349.86818565)(189.23240975,349.98818817)
\curveto(189.22240917,350.02818549)(189.21740918,350.06318546)(189.21740975,350.09318817)
\curveto(189.21740918,350.13318539)(189.21240918,350.17318535)(189.20240975,350.21318817)
\curveto(189.16240923,350.33318519)(189.13740926,350.46318506)(189.12740975,350.60318817)
\lineto(189.09740975,351.02318817)
\curveto(189.0974093,351.07318445)(189.0924093,351.12818439)(189.08240975,351.18818817)
\curveto(189.08240931,351.24818427)(189.08740931,351.30318422)(189.09740975,351.35318817)
\lineto(189.09740975,351.53318817)
\lineto(189.14240975,351.89318817)
\curveto(189.18240921,352.06318346)(189.21740918,352.22818329)(189.24740975,352.38818817)
\curveto(189.27740912,352.54818297)(189.32240907,352.69818282)(189.38240975,352.83818817)
\curveto(189.81240858,353.87818164)(190.54240785,354.61318091)(191.57240975,355.04318817)
\curveto(191.71240668,355.10318042)(191.85240654,355.14318038)(191.99240975,355.16318817)
\curveto(192.14240625,355.19318033)(192.2974061,355.22818029)(192.45740975,355.26818817)
\curveto(192.53740586,355.27818024)(192.61240578,355.28318024)(192.68240975,355.28318817)
\curveto(192.75240564,355.28318024)(192.82740557,355.28818023)(192.90740975,355.29818817)
\curveto(193.41740498,355.30818021)(193.85240454,355.24818027)(194.21240975,355.11818817)
\curveto(194.58240381,354.99818052)(194.91240348,354.83818068)(195.20240975,354.63818817)
\curveto(195.2924031,354.57818094)(195.38240301,354.50818101)(195.47240975,354.42818817)
\curveto(195.56240283,354.35818116)(195.64240275,354.28318124)(195.71240975,354.20318817)
\curveto(195.74240265,354.15318137)(195.78240261,354.11318141)(195.83240975,354.08318817)
\curveto(195.91240248,353.97318155)(195.98740241,353.85818166)(196.05740975,353.73818817)
\curveto(196.12740227,353.62818189)(196.20240219,353.51318201)(196.28240975,353.39318817)
\curveto(196.33240206,353.30318222)(196.37240202,353.20818231)(196.40240975,353.10818817)
\curveto(196.44240195,353.0181825)(196.48240191,352.9181826)(196.52240975,352.80818817)
\curveto(196.57240182,352.67818284)(196.61240178,352.54318298)(196.64240975,352.40318817)
\curveto(196.67240172,352.26318326)(196.70740169,352.1231834)(196.74740975,351.98318817)
\curveto(196.76740163,351.90318362)(196.77240162,351.81318371)(196.76240975,351.71318817)
\curveto(196.76240163,351.6231839)(196.77240162,351.53818398)(196.79240975,351.45818817)
\lineto(196.79240975,351.29318817)
\moveto(194.54240975,352.17818817)
\curveto(194.61240378,352.27818324)(194.61740378,352.39818312)(194.55740975,352.53818817)
\curveto(194.50740389,352.68818283)(194.46740393,352.79818272)(194.43740975,352.86818817)
\curveto(194.2974041,353.13818238)(194.11240428,353.34318218)(193.88240975,353.48318817)
\curveto(193.65240474,353.63318189)(193.33240506,353.71318181)(192.92240975,353.72318817)
\curveto(192.8924055,353.70318182)(192.85740554,353.69818182)(192.81740975,353.70818817)
\curveto(192.77740562,353.7181818)(192.74240565,353.7181818)(192.71240975,353.70818817)
\curveto(192.66240573,353.68818183)(192.60740579,353.67318185)(192.54740975,353.66318817)
\curveto(192.48740591,353.66318186)(192.43240596,353.65318187)(192.38240975,353.63318817)
\curveto(191.94240645,353.49318203)(191.61740678,353.2181823)(191.40740975,352.80818817)
\curveto(191.38740701,352.76818275)(191.36240703,352.71318281)(191.33240975,352.64318817)
\curveto(191.31240708,352.58318294)(191.2974071,352.518183)(191.28740975,352.44818817)
\curveto(191.27740712,352.38818313)(191.27740712,352.32818319)(191.28740975,352.26818817)
\curveto(191.30740709,352.20818331)(191.34240705,352.15818336)(191.39240975,352.11818817)
\curveto(191.47240692,352.06818345)(191.58240681,352.04318348)(191.72240975,352.04318817)
\lineto(192.12740975,352.04318817)
\lineto(193.79240975,352.04318817)
\lineto(194.22740975,352.04318817)
\curveto(194.38740401,352.05318347)(194.4924039,352.09818342)(194.54240975,352.17818817)
}
}
{
\newrgbcolor{curcolor}{0 0 0}
\pscustom[linestyle=none,fillstyle=solid,fillcolor=curcolor]
{
\newpath
\moveto(202.465691,355.28318817)
\curveto(202.57568568,355.28318024)(202.67068559,355.27318025)(202.750691,355.25318817)
\curveto(202.84068542,355.23318029)(202.91068535,355.18818033)(202.960691,355.11818817)
\curveto(203.02068524,355.03818048)(203.05068521,354.89818062)(203.050691,354.69818817)
\lineto(203.050691,354.18818817)
\lineto(203.050691,353.81318817)
\curveto(203.0606852,353.67318185)(203.04568521,353.56318196)(203.005691,353.48318817)
\curveto(202.96568529,353.41318211)(202.90568535,353.36818215)(202.825691,353.34818817)
\curveto(202.7556855,353.32818219)(202.67068559,353.3181822)(202.570691,353.31818817)
\curveto(202.48068578,353.3181822)(202.38068588,353.3231822)(202.270691,353.33318817)
\curveto(202.17068609,353.34318218)(202.07568618,353.33818218)(201.985691,353.31818817)
\curveto(201.91568634,353.29818222)(201.84568641,353.28318224)(201.775691,353.27318817)
\curveto(201.70568655,353.27318225)(201.64068662,353.26318226)(201.580691,353.24318817)
\curveto(201.42068684,353.19318233)(201.260687,353.1181824)(201.100691,353.01818817)
\curveto(200.94068732,352.92818259)(200.81568744,352.8231827)(200.725691,352.70318817)
\curveto(200.67568758,352.6231829)(200.62068764,352.53818298)(200.560691,352.44818817)
\curveto(200.51068775,352.36818315)(200.4606878,352.28318324)(200.410691,352.19318817)
\curveto(200.38068788,352.11318341)(200.35068791,352.02818349)(200.320691,351.93818817)
\lineto(200.260691,351.69818817)
\curveto(200.24068802,351.62818389)(200.23068803,351.55318397)(200.230691,351.47318817)
\curveto(200.23068803,351.40318412)(200.22068804,351.33318419)(200.200691,351.26318817)
\curveto(200.19068807,351.2231843)(200.18568807,351.18318434)(200.185691,351.14318817)
\curveto(200.19568806,351.11318441)(200.19568806,351.08318444)(200.185691,351.05318817)
\lineto(200.185691,350.81318817)
\curveto(200.16568809,350.74318478)(200.1606881,350.66318486)(200.170691,350.57318817)
\curveto(200.18068808,350.49318503)(200.18568807,350.41318511)(200.185691,350.33318817)
\lineto(200.185691,349.37318817)
\lineto(200.185691,348.09818817)
\curveto(200.18568807,347.96818755)(200.18068808,347.84818767)(200.170691,347.73818817)
\curveto(200.1606881,347.62818789)(200.13068813,347.53818798)(200.080691,347.46818817)
\curveto(200.0606882,347.43818808)(200.02568823,347.41318811)(199.975691,347.39318817)
\curveto(199.93568832,347.38318814)(199.89068837,347.37318815)(199.840691,347.36318817)
\lineto(199.765691,347.36318817)
\curveto(199.71568854,347.35318817)(199.6606886,347.34818817)(199.600691,347.34818817)
\lineto(199.435691,347.34818817)
\lineto(198.790691,347.34818817)
\curveto(198.73068953,347.35818816)(198.66568959,347.36318816)(198.595691,347.36318817)
\lineto(198.400691,347.36318817)
\curveto(198.35068991,347.38318814)(198.30068996,347.39818812)(198.250691,347.40818817)
\curveto(198.20069006,347.42818809)(198.16569009,347.46318806)(198.145691,347.51318817)
\curveto(198.10569015,347.56318796)(198.08069018,347.63318789)(198.070691,347.72318817)
\lineto(198.070691,348.02318817)
\lineto(198.070691,349.04318817)
\lineto(198.070691,353.27318817)
\lineto(198.070691,354.38318817)
\lineto(198.070691,354.66818817)
\curveto(198.07069019,354.76818075)(198.09069017,354.84818067)(198.130691,354.90818817)
\curveto(198.18069008,354.98818053)(198.25569,355.03818048)(198.355691,355.05818817)
\curveto(198.4556898,355.07818044)(198.57568968,355.08818043)(198.715691,355.08818817)
\lineto(199.480691,355.08818817)
\curveto(199.60068866,355.08818043)(199.70568855,355.07818044)(199.795691,355.05818817)
\curveto(199.88568837,355.04818047)(199.9556883,355.00318052)(200.005691,354.92318817)
\curveto(200.03568822,354.87318065)(200.05068821,354.80318072)(200.050691,354.71318817)
\lineto(200.080691,354.44318817)
\curveto(200.09068817,354.36318116)(200.10568815,354.28818123)(200.125691,354.21818817)
\curveto(200.1556881,354.14818137)(200.20568805,354.11318141)(200.275691,354.11318817)
\curveto(200.29568796,354.13318139)(200.31568794,354.14318138)(200.335691,354.14318817)
\curveto(200.3556879,354.14318138)(200.37568788,354.15318137)(200.395691,354.17318817)
\curveto(200.4556878,354.2231813)(200.50568775,354.27818124)(200.545691,354.33818817)
\curveto(200.59568766,354.40818111)(200.6556876,354.46818105)(200.725691,354.51818817)
\curveto(200.76568749,354.54818097)(200.80068746,354.57818094)(200.830691,354.60818817)
\curveto(200.8606874,354.64818087)(200.89568736,354.68318084)(200.935691,354.71318817)
\lineto(201.205691,354.89318817)
\curveto(201.30568695,354.95318057)(201.40568685,355.00818051)(201.505691,355.05818817)
\curveto(201.60568665,355.09818042)(201.70568655,355.13318039)(201.805691,355.16318817)
\lineto(202.135691,355.25318817)
\curveto(202.16568609,355.26318026)(202.22068604,355.26318026)(202.300691,355.25318817)
\curveto(202.39068587,355.25318027)(202.44568581,355.26318026)(202.465691,355.28318817)
}
}
{
\newrgbcolor{curcolor}{0 0 0}
\pscustom[linestyle=none,fillstyle=solid,fillcolor=curcolor]
{
\newpath
\moveto(211.37709725,351.53318817)
\curveto(211.39708868,351.47318405)(211.40708867,351.38818413)(211.40709725,351.27818817)
\curveto(211.40708867,351.16818435)(211.39708868,351.08318444)(211.37709725,351.02318817)
\lineto(211.37709725,350.87318817)
\curveto(211.35708872,350.79318473)(211.34708873,350.71318481)(211.34709725,350.63318817)
\curveto(211.35708872,350.55318497)(211.35208872,350.47318505)(211.33209725,350.39318817)
\curveto(211.31208876,350.3231852)(211.29708878,350.25818526)(211.28709725,350.19818817)
\curveto(211.2770888,350.13818538)(211.26708881,350.07318545)(211.25709725,350.00318817)
\curveto(211.21708886,349.89318563)(211.18208889,349.77818574)(211.15209725,349.65818817)
\curveto(211.12208895,349.54818597)(211.08208899,349.44318608)(211.03209725,349.34318817)
\curveto(210.82208925,348.86318666)(210.54708953,348.47318705)(210.20709725,348.17318817)
\curveto(209.86709021,347.87318765)(209.45709062,347.6231879)(208.97709725,347.42318817)
\curveto(208.85709122,347.37318815)(208.73209134,347.33818818)(208.60209725,347.31818817)
\curveto(208.48209159,347.28818823)(208.35709172,347.25818826)(208.22709725,347.22818817)
\curveto(208.1770919,347.20818831)(208.12209195,347.19818832)(208.06209725,347.19818817)
\curveto(208.00209207,347.19818832)(207.94709213,347.19318833)(207.89709725,347.18318817)
\lineto(207.79209725,347.18318817)
\curveto(207.76209231,347.17318835)(207.73209234,347.16818835)(207.70209725,347.16818817)
\curveto(207.65209242,347.15818836)(207.5720925,347.15318837)(207.46209725,347.15318817)
\curveto(207.35209272,347.14318838)(207.26709281,347.14818837)(207.20709725,347.16818817)
\lineto(207.05709725,347.16818817)
\curveto(207.00709307,347.17818834)(206.95209312,347.18318834)(206.89209725,347.18318817)
\curveto(206.84209323,347.17318835)(206.79209328,347.17818834)(206.74209725,347.19818817)
\curveto(206.70209337,347.20818831)(206.66209341,347.21318831)(206.62209725,347.21318817)
\curveto(206.59209348,347.21318831)(206.55209352,347.2181883)(206.50209725,347.22818817)
\curveto(206.40209367,347.25818826)(206.30209377,347.28318824)(206.20209725,347.30318817)
\curveto(206.10209397,347.3231882)(206.00709407,347.35318817)(205.91709725,347.39318817)
\curveto(205.79709428,347.43318809)(205.68209439,347.47318805)(205.57209725,347.51318817)
\curveto(205.4720946,347.55318797)(205.36709471,347.60318792)(205.25709725,347.66318817)
\curveto(204.90709517,347.87318765)(204.60709547,348.1181874)(204.35709725,348.39818817)
\curveto(204.10709597,348.67818684)(203.89709618,349.01318651)(203.72709725,349.40318817)
\curveto(203.6770964,349.49318603)(203.63709644,349.58818593)(203.60709725,349.68818817)
\curveto(203.58709649,349.78818573)(203.56209651,349.89318563)(203.53209725,350.00318817)
\curveto(203.51209656,350.05318547)(203.50209657,350.09818542)(203.50209725,350.13818817)
\curveto(203.50209657,350.17818534)(203.49209658,350.2231853)(203.47209725,350.27318817)
\curveto(203.45209662,350.35318517)(203.44209663,350.43318509)(203.44209725,350.51318817)
\curveto(203.44209663,350.60318492)(203.43209664,350.68818483)(203.41209725,350.76818817)
\curveto(203.40209667,350.8181847)(203.39709668,350.86318466)(203.39709725,350.90318817)
\lineto(203.39709725,351.03818817)
\curveto(203.3770967,351.09818442)(203.36709671,351.18318434)(203.36709725,351.29318817)
\curveto(203.3770967,351.40318412)(203.39209668,351.48818403)(203.41209725,351.54818817)
\lineto(203.41209725,351.65318817)
\curveto(203.42209665,351.70318382)(203.42209665,351.75318377)(203.41209725,351.80318817)
\curveto(203.41209666,351.86318366)(203.42209665,351.9181836)(203.44209725,351.96818817)
\curveto(203.45209662,352.0181835)(203.45709662,352.06318346)(203.45709725,352.10318817)
\curveto(203.45709662,352.15318337)(203.46709661,352.20318332)(203.48709725,352.25318817)
\curveto(203.52709655,352.38318314)(203.56209651,352.50818301)(203.59209725,352.62818817)
\curveto(203.62209645,352.75818276)(203.66209641,352.88318264)(203.71209725,353.00318817)
\curveto(203.89209618,353.41318211)(204.10709597,353.75318177)(204.35709725,354.02318817)
\curveto(204.60709547,354.30318122)(204.91209516,354.55818096)(205.27209725,354.78818817)
\curveto(205.3720947,354.83818068)(205.4770946,354.88318064)(205.58709725,354.92318817)
\curveto(205.69709438,354.96318056)(205.80709427,355.00818051)(205.91709725,355.05818817)
\curveto(206.04709403,355.10818041)(206.18209389,355.14318038)(206.32209725,355.16318817)
\curveto(206.46209361,355.18318034)(206.60709347,355.21318031)(206.75709725,355.25318817)
\curveto(206.83709324,355.26318026)(206.91209316,355.26818025)(206.98209725,355.26818817)
\curveto(207.05209302,355.26818025)(207.12209295,355.27318025)(207.19209725,355.28318817)
\curveto(207.7720923,355.29318023)(208.2720918,355.23318029)(208.69209725,355.10318817)
\curveto(209.12209095,354.97318055)(209.50209057,354.79318073)(209.83209725,354.56318817)
\curveto(209.94209013,354.48318104)(210.05209002,354.39318113)(210.16209725,354.29318817)
\curveto(210.28208979,354.20318132)(210.38208969,354.10318142)(210.46209725,353.99318817)
\curveto(210.54208953,353.89318163)(210.61208946,353.79318173)(210.67209725,353.69318817)
\curveto(210.74208933,353.59318193)(210.81208926,353.48818203)(210.88209725,353.37818817)
\curveto(210.95208912,353.26818225)(211.00708907,353.14818237)(211.04709725,353.01818817)
\curveto(211.08708899,352.89818262)(211.13208894,352.76818275)(211.18209725,352.62818817)
\curveto(211.21208886,352.54818297)(211.23708884,352.46318306)(211.25709725,352.37318817)
\lineto(211.31709725,352.10318817)
\curveto(211.32708875,352.06318346)(211.33208874,352.0231835)(211.33209725,351.98318817)
\curveto(211.33208874,351.94318358)(211.33708874,351.90318362)(211.34709725,351.86318817)
\curveto(211.36708871,351.81318371)(211.3720887,351.75818376)(211.36209725,351.69818817)
\curveto(211.35208872,351.63818388)(211.35708872,351.58318394)(211.37709725,351.53318817)
\moveto(209.27709725,350.99318817)
\curveto(209.28709079,351.04318448)(209.29209078,351.11318441)(209.29209725,351.20318817)
\curveto(209.29209078,351.30318422)(209.28709079,351.37818414)(209.27709725,351.42818817)
\lineto(209.27709725,351.54818817)
\curveto(209.25709082,351.59818392)(209.24709083,351.65318387)(209.24709725,351.71318817)
\curveto(209.24709083,351.77318375)(209.24209083,351.82818369)(209.23209725,351.87818817)
\curveto(209.23209084,351.9181836)(209.22709085,351.94818357)(209.21709725,351.96818817)
\lineto(209.15709725,352.20818817)
\curveto(209.14709093,352.29818322)(209.12709095,352.38318314)(209.09709725,352.46318817)
\curveto(208.98709109,352.7231828)(208.85709122,352.94318258)(208.70709725,353.12318817)
\curveto(208.55709152,353.31318221)(208.35709172,353.46318206)(208.10709725,353.57318817)
\curveto(208.04709203,353.59318193)(207.98709209,353.60818191)(207.92709725,353.61818817)
\curveto(207.86709221,353.63818188)(207.80209227,353.65818186)(207.73209725,353.67818817)
\curveto(207.65209242,353.69818182)(207.56709251,353.70318182)(207.47709725,353.69318817)
\lineto(207.20709725,353.69318817)
\curveto(207.1770929,353.67318185)(207.14209293,353.66318186)(207.10209725,353.66318817)
\curveto(207.06209301,353.67318185)(207.02709305,353.67318185)(206.99709725,353.66318817)
\lineto(206.78709725,353.60318817)
\curveto(206.72709335,353.59318193)(206.6720934,353.57318195)(206.62209725,353.54318817)
\curveto(206.3720937,353.43318209)(206.16709391,353.27318225)(206.00709725,353.06318817)
\curveto(205.85709422,352.86318266)(205.73709434,352.62818289)(205.64709725,352.35818817)
\curveto(205.61709446,352.25818326)(205.59209448,352.15318337)(205.57209725,352.04318817)
\curveto(205.56209451,351.93318359)(205.54709453,351.8231837)(205.52709725,351.71318817)
\curveto(205.51709456,351.66318386)(205.51209456,351.61318391)(205.51209725,351.56318817)
\lineto(205.51209725,351.41318817)
\curveto(205.49209458,351.34318418)(205.48209459,351.23818428)(205.48209725,351.09818817)
\curveto(205.49209458,350.95818456)(205.50709457,350.85318467)(205.52709725,350.78318817)
\lineto(205.52709725,350.64818817)
\curveto(205.54709453,350.56818495)(205.56209451,350.48818503)(205.57209725,350.40818817)
\curveto(205.58209449,350.33818518)(205.59709448,350.26318526)(205.61709725,350.18318817)
\curveto(205.71709436,349.88318564)(205.82209425,349.63818588)(205.93209725,349.44818817)
\curveto(206.05209402,349.26818625)(206.23709384,349.10318642)(206.48709725,348.95318817)
\curveto(206.55709352,348.90318662)(206.63209344,348.86318666)(206.71209725,348.83318817)
\curveto(206.80209327,348.80318672)(206.89209318,348.77818674)(206.98209725,348.75818817)
\curveto(207.02209305,348.74818677)(207.05709302,348.74318678)(207.08709725,348.74318817)
\curveto(207.11709296,348.75318677)(207.15209292,348.75318677)(207.19209725,348.74318817)
\lineto(207.31209725,348.71318817)
\curveto(207.36209271,348.71318681)(207.40709267,348.7181868)(207.44709725,348.72818817)
\lineto(207.56709725,348.72818817)
\curveto(207.64709243,348.74818677)(207.72709235,348.76318676)(207.80709725,348.77318817)
\curveto(207.88709219,348.78318674)(207.96209211,348.80318672)(208.03209725,348.83318817)
\curveto(208.29209178,348.93318659)(208.50209157,349.06818645)(208.66209725,349.23818817)
\curveto(208.82209125,349.40818611)(208.95709112,349.6181859)(209.06709725,349.86818817)
\curveto(209.10709097,349.96818555)(209.13709094,350.06818545)(209.15709725,350.16818817)
\curveto(209.1770909,350.26818525)(209.20209087,350.37318515)(209.23209725,350.48318817)
\curveto(209.24209083,350.523185)(209.24709083,350.55818496)(209.24709725,350.58818817)
\curveto(209.24709083,350.62818489)(209.25209082,350.66818485)(209.26209725,350.70818817)
\lineto(209.26209725,350.84318817)
\curveto(209.26209081,350.89318463)(209.26709081,350.94318458)(209.27709725,350.99318817)
}
}
{
\newrgbcolor{curcolor}{0 0 0}
\pscustom[linestyle=none,fillstyle=solid,fillcolor=curcolor]
{
\newpath
\moveto(217.20201912,355.28318817)
\curveto(217.80201332,355.30318022)(218.30201282,355.2181803)(218.70201912,355.02818817)
\curveto(219.10201202,354.83818068)(219.4170117,354.55818096)(219.64701912,354.18818817)
\curveto(219.7170114,354.07818144)(219.77201135,353.95818156)(219.81201912,353.82818817)
\curveto(219.85201127,353.70818181)(219.89201123,353.58318194)(219.93201912,353.45318817)
\curveto(219.95201117,353.37318215)(219.96201116,353.29818222)(219.96201912,353.22818817)
\curveto(219.97201115,353.15818236)(219.98701113,353.08818243)(220.00701912,353.01818817)
\curveto(220.00701111,352.95818256)(220.01201111,352.9181826)(220.02201912,352.89818817)
\curveto(220.04201108,352.75818276)(220.05201107,352.61318291)(220.05201912,352.46318817)
\lineto(220.05201912,352.02818817)
\lineto(220.05201912,350.69318817)
\lineto(220.05201912,348.26318817)
\curveto(220.05201107,348.07318745)(220.04701107,347.88818763)(220.03701912,347.70818817)
\curveto(220.03701108,347.53818798)(219.96701115,347.42818809)(219.82701912,347.37818817)
\curveto(219.76701135,347.35818816)(219.69701142,347.34818817)(219.61701912,347.34818817)
\lineto(219.37701912,347.34818817)
\lineto(218.56701912,347.34818817)
\curveto(218.44701267,347.34818817)(218.33701278,347.35318817)(218.23701912,347.36318817)
\curveto(218.14701297,347.38318814)(218.07701304,347.42818809)(218.02701912,347.49818817)
\curveto(217.98701313,347.55818796)(217.96201316,347.63318789)(217.95201912,347.72318817)
\lineto(217.95201912,348.03818817)
\lineto(217.95201912,349.08818817)
\lineto(217.95201912,351.32318817)
\curveto(217.95201317,351.69318383)(217.93701318,352.03318349)(217.90701912,352.34318817)
\curveto(217.87701324,352.66318286)(217.78701333,352.93318259)(217.63701912,353.15318817)
\curveto(217.49701362,353.35318217)(217.29201383,353.49318203)(217.02201912,353.57318817)
\curveto(216.97201415,353.59318193)(216.9170142,353.60318192)(216.85701912,353.60318817)
\curveto(216.80701431,353.60318192)(216.75201437,353.61318191)(216.69201912,353.63318817)
\curveto(216.64201448,353.64318188)(216.57701454,353.64318188)(216.49701912,353.63318817)
\curveto(216.42701469,353.63318189)(216.37201475,353.62818189)(216.33201912,353.61818817)
\curveto(216.29201483,353.60818191)(216.25701486,353.60318192)(216.22701912,353.60318817)
\curveto(216.19701492,353.60318192)(216.16701495,353.59818192)(216.13701912,353.58818817)
\curveto(215.90701521,353.52818199)(215.7220154,353.44818207)(215.58201912,353.34818817)
\curveto(215.26201586,353.1181824)(215.07201605,352.78318274)(215.01201912,352.34318817)
\curveto(214.95201617,351.90318362)(214.9220162,351.40818411)(214.92201912,350.85818817)
\lineto(214.92201912,348.98318817)
\lineto(214.92201912,348.06818817)
\lineto(214.92201912,347.79818817)
\curveto(214.9220162,347.70818781)(214.90701621,347.63318789)(214.87701912,347.57318817)
\curveto(214.82701629,347.46318806)(214.74701637,347.39818812)(214.63701912,347.37818817)
\curveto(214.52701659,347.35818816)(214.39201673,347.34818817)(214.23201912,347.34818817)
\lineto(213.48201912,347.34818817)
\curveto(213.37201775,347.34818817)(213.26201786,347.35318817)(213.15201912,347.36318817)
\curveto(213.04201808,347.37318815)(212.96201816,347.40818811)(212.91201912,347.46818817)
\curveto(212.84201828,347.55818796)(212.80701831,347.68818783)(212.80701912,347.85818817)
\curveto(212.8170183,348.02818749)(212.8220183,348.18818733)(212.82201912,348.33818817)
\lineto(212.82201912,350.37818817)
\lineto(212.82201912,353.67818817)
\lineto(212.82201912,354.44318817)
\lineto(212.82201912,354.74318817)
\curveto(212.83201829,354.83318069)(212.86201826,354.90818061)(212.91201912,354.96818817)
\curveto(212.93201819,354.99818052)(212.96201816,355.0181805)(213.00201912,355.02818817)
\curveto(213.05201807,355.04818047)(213.10201802,355.06318046)(213.15201912,355.07318817)
\lineto(213.22701912,355.07318817)
\curveto(213.27701784,355.08318044)(213.32701779,355.08818043)(213.37701912,355.08818817)
\lineto(213.54201912,355.08818817)
\lineto(214.17201912,355.08818817)
\curveto(214.25201687,355.08818043)(214.32701679,355.08318044)(214.39701912,355.07318817)
\curveto(214.47701664,355.07318045)(214.54701657,355.06318046)(214.60701912,355.04318817)
\curveto(214.67701644,355.01318051)(214.7220164,354.96818055)(214.74201912,354.90818817)
\curveto(214.77201635,354.84818067)(214.79701632,354.77818074)(214.81701912,354.69818817)
\curveto(214.82701629,354.65818086)(214.82701629,354.6231809)(214.81701912,354.59318817)
\curveto(214.8170163,354.56318096)(214.82701629,354.53318099)(214.84701912,354.50318817)
\curveto(214.86701625,354.45318107)(214.88201624,354.4231811)(214.89201912,354.41318817)
\curveto(214.91201621,354.40318112)(214.93701618,354.38818113)(214.96701912,354.36818817)
\curveto(215.07701604,354.35818116)(215.16701595,354.39318113)(215.23701912,354.47318817)
\curveto(215.30701581,354.56318096)(215.38201574,354.63318089)(215.46201912,354.68318817)
\curveto(215.73201539,354.88318064)(216.03201509,355.04318048)(216.36201912,355.16318817)
\curveto(216.45201467,355.19318033)(216.54201458,355.21318031)(216.63201912,355.22318817)
\curveto(216.73201439,355.23318029)(216.83701428,355.24818027)(216.94701912,355.26818817)
\curveto(216.97701414,355.27818024)(217.0220141,355.27818024)(217.08201912,355.26818817)
\curveto(217.14201398,355.26818025)(217.18201394,355.27318025)(217.20201912,355.28318817)
}
}
{
\newrgbcolor{curcolor}{0 0 0}
\pscustom[linestyle=none,fillstyle=solid,fillcolor=curcolor]
{
\newpath
\moveto(26.5466285,336.32318817)
\lineto(27.6716285,336.32318817)
\curveto(27.78162606,336.32318045)(27.88162596,336.31818045)(27.9716285,336.30818817)
\curveto(28.06162578,336.29818047)(28.12662572,336.26318051)(28.1666285,336.20318817)
\curveto(28.21662563,336.14318063)(28.2466256,336.05818071)(28.2566285,335.94818817)
\curveto(28.26662558,335.84818092)(28.27162557,335.74318103)(28.2716285,335.63318817)
\lineto(28.2716285,334.58318817)
\lineto(28.2716285,332.34818817)
\curveto(28.27162557,331.98818478)(28.28662556,331.64818512)(28.3166285,331.32818817)
\curveto(28.3466255,331.00818576)(28.43662541,330.74318603)(28.5866285,330.53318817)
\curveto(28.72662512,330.32318645)(28.95162489,330.1731866)(29.2616285,330.08318817)
\curveto(29.31162453,330.0731867)(29.35162449,330.0681867)(29.3816285,330.06818817)
\curveto(29.42162442,330.0681867)(29.46662438,330.06318671)(29.5166285,330.05318817)
\curveto(29.56662428,330.04318673)(29.62162422,330.03818673)(29.6816285,330.03818817)
\curveto(29.7416241,330.03818673)(29.78662406,330.04318673)(29.8166285,330.05318817)
\curveto(29.86662398,330.0731867)(29.90662394,330.07818669)(29.9366285,330.06818817)
\curveto(29.97662387,330.05818671)(30.01662383,330.06318671)(30.0566285,330.08318817)
\curveto(30.26662358,330.13318664)(30.43162341,330.19818657)(30.5516285,330.27818817)
\curveto(30.73162311,330.38818638)(30.87162297,330.52818624)(30.9716285,330.69818817)
\curveto(31.08162276,330.87818589)(31.15662269,331.0731857)(31.1966285,331.28318817)
\curveto(31.2466226,331.50318527)(31.27662257,331.74318503)(31.2866285,332.00318817)
\curveto(31.29662255,332.2731845)(31.30162254,332.55318422)(31.3016285,332.84318817)
\lineto(31.3016285,334.65818817)
\lineto(31.3016285,335.63318817)
\lineto(31.3016285,335.90318817)
\curveto(31.30162254,336.00318077)(31.32162252,336.08318069)(31.3616285,336.14318817)
\curveto(31.41162243,336.23318054)(31.48662236,336.28318049)(31.5866285,336.29318817)
\curveto(31.68662216,336.31318046)(31.80662204,336.32318045)(31.9466285,336.32318817)
\lineto(32.7416285,336.32318817)
\lineto(33.0266285,336.32318817)
\curveto(33.11662073,336.32318045)(33.19162065,336.30318047)(33.2516285,336.26318817)
\curveto(33.33162051,336.21318056)(33.37662047,336.13818063)(33.3866285,336.03818817)
\curveto(33.39662045,335.93818083)(33.40162044,335.82318095)(33.4016285,335.69318817)
\lineto(33.4016285,334.55318817)
\lineto(33.4016285,330.33818817)
\lineto(33.4016285,329.27318817)
\lineto(33.4016285,328.97318817)
\curveto(33.40162044,328.8731879)(33.38162046,328.79818797)(33.3416285,328.74818817)
\curveto(33.29162055,328.6681881)(33.21662063,328.62318815)(33.1166285,328.61318817)
\curveto(33.01662083,328.60318817)(32.91162093,328.59818817)(32.8016285,328.59818817)
\lineto(31.9916285,328.59818817)
\curveto(31.88162196,328.59818817)(31.78162206,328.60318817)(31.6916285,328.61318817)
\curveto(31.61162223,328.62318815)(31.5466223,328.66318811)(31.4966285,328.73318817)
\curveto(31.47662237,328.76318801)(31.45662239,328.80818796)(31.4366285,328.86818817)
\curveto(31.42662242,328.92818784)(31.41162243,328.98818778)(31.3916285,329.04818817)
\curveto(31.38162246,329.10818766)(31.36662248,329.16318761)(31.3466285,329.21318817)
\curveto(31.32662252,329.26318751)(31.29662255,329.29318748)(31.2566285,329.30318817)
\curveto(31.23662261,329.32318745)(31.21162263,329.32818744)(31.1816285,329.31818817)
\curveto(31.15162269,329.30818746)(31.12662272,329.29818747)(31.1066285,329.28818817)
\curveto(31.03662281,329.24818752)(30.97662287,329.20318757)(30.9266285,329.15318817)
\curveto(30.87662297,329.10318767)(30.82162302,329.05818771)(30.7616285,329.01818817)
\curveto(30.72162312,328.98818778)(30.68162316,328.95318782)(30.6416285,328.91318817)
\curveto(30.61162323,328.88318789)(30.57162327,328.85318792)(30.5216285,328.82318817)
\curveto(30.29162355,328.68318809)(30.02162382,328.5731882)(29.7116285,328.49318817)
\curveto(29.6416242,328.4731883)(29.57162427,328.46318831)(29.5016285,328.46318817)
\curveto(29.43162441,328.45318832)(29.35662449,328.43818833)(29.2766285,328.41818817)
\curveto(29.23662461,328.40818836)(29.19162465,328.40818836)(29.1416285,328.41818817)
\curveto(29.10162474,328.41818835)(29.06162478,328.41318836)(29.0216285,328.40318817)
\curveto(28.99162485,328.39318838)(28.92662492,328.39318838)(28.8266285,328.40318817)
\curveto(28.73662511,328.40318837)(28.67662517,328.40818836)(28.6466285,328.41818817)
\curveto(28.59662525,328.41818835)(28.5466253,328.42318835)(28.4966285,328.43318817)
\lineto(28.3466285,328.43318817)
\curveto(28.22662562,328.46318831)(28.11162573,328.48818828)(28.0016285,328.50818817)
\curveto(27.89162595,328.52818824)(27.78162606,328.55818821)(27.6716285,328.59818817)
\curveto(27.62162622,328.61818815)(27.57662627,328.63318814)(27.5366285,328.64318817)
\curveto(27.50662634,328.66318811)(27.46662638,328.68318809)(27.4166285,328.70318817)
\curveto(27.06662678,328.89318788)(26.78662706,329.15818761)(26.5766285,329.49818817)
\curveto(26.4466274,329.70818706)(26.35162749,329.95818681)(26.2916285,330.24818817)
\curveto(26.23162761,330.54818622)(26.19162765,330.86318591)(26.1716285,331.19318817)
\curveto(26.16162768,331.53318524)(26.15662769,331.87818489)(26.1566285,332.22818817)
\curveto(26.16662768,332.58818418)(26.17162767,332.94318383)(26.1716285,333.29318817)
\lineto(26.1716285,335.33318817)
\curveto(26.17162767,335.46318131)(26.16662768,335.61318116)(26.1566285,335.78318817)
\curveto(26.15662769,335.96318081)(26.18162766,336.09318068)(26.2316285,336.17318817)
\curveto(26.26162758,336.22318055)(26.32162752,336.2681805)(26.4116285,336.30818817)
\curveto(26.47162737,336.30818046)(26.51662733,336.31318046)(26.5466285,336.32318817)
}
}
{
\newrgbcolor{curcolor}{0 0 0}
\pscustom[linestyle=none,fillstyle=solid,fillcolor=curcolor]
{
\newpath
\moveto(39.4578785,336.53318817)
\curveto(40.05787269,336.55318022)(40.55787219,336.4681803)(40.9578785,336.27818817)
\curveto(41.35787139,336.08818068)(41.67287108,335.80818096)(41.9028785,335.43818817)
\curveto(41.97287078,335.32818144)(42.02787072,335.20818156)(42.0678785,335.07818817)
\curveto(42.10787064,334.95818181)(42.1478706,334.83318194)(42.1878785,334.70318817)
\curveto(42.20787054,334.62318215)(42.21787053,334.54818222)(42.2178785,334.47818817)
\curveto(42.22787052,334.40818236)(42.24287051,334.33818243)(42.2628785,334.26818817)
\curveto(42.26287049,334.20818256)(42.26787048,334.1681826)(42.2778785,334.14818817)
\curveto(42.29787045,334.00818276)(42.30787044,333.86318291)(42.3078785,333.71318817)
\lineto(42.3078785,333.27818817)
\lineto(42.3078785,331.94318817)
\lineto(42.3078785,329.51318817)
\curveto(42.30787044,329.32318745)(42.30287045,329.13818763)(42.2928785,328.95818817)
\curveto(42.29287046,328.78818798)(42.22287053,328.67818809)(42.0828785,328.62818817)
\curveto(42.02287073,328.60818816)(41.9528708,328.59818817)(41.8728785,328.59818817)
\lineto(41.6328785,328.59818817)
\lineto(40.8228785,328.59818817)
\curveto(40.70287205,328.59818817)(40.59287216,328.60318817)(40.4928785,328.61318817)
\curveto(40.40287235,328.63318814)(40.33287242,328.67818809)(40.2828785,328.74818817)
\curveto(40.24287251,328.80818796)(40.21787253,328.88318789)(40.2078785,328.97318817)
\lineto(40.2078785,329.28818817)
\lineto(40.2078785,330.33818817)
\lineto(40.2078785,332.57318817)
\curveto(40.20787254,332.94318383)(40.19287256,333.28318349)(40.1628785,333.59318817)
\curveto(40.13287262,333.91318286)(40.04287271,334.18318259)(39.8928785,334.40318817)
\curveto(39.752873,334.60318217)(39.5478732,334.74318203)(39.2778785,334.82318817)
\curveto(39.22787352,334.84318193)(39.17287358,334.85318192)(39.1128785,334.85318817)
\curveto(39.06287369,334.85318192)(39.00787374,334.86318191)(38.9478785,334.88318817)
\curveto(38.89787385,334.89318188)(38.83287392,334.89318188)(38.7528785,334.88318817)
\curveto(38.68287407,334.88318189)(38.62787412,334.87818189)(38.5878785,334.86818817)
\curveto(38.5478742,334.85818191)(38.51287424,334.85318192)(38.4828785,334.85318817)
\curveto(38.4528743,334.85318192)(38.42287433,334.84818192)(38.3928785,334.83818817)
\curveto(38.16287459,334.77818199)(37.97787477,334.69818207)(37.8378785,334.59818817)
\curveto(37.51787523,334.3681824)(37.32787542,334.03318274)(37.2678785,333.59318817)
\curveto(37.20787554,333.15318362)(37.17787557,332.65818411)(37.1778785,332.10818817)
\lineto(37.1778785,330.23318817)
\lineto(37.1778785,329.31818817)
\lineto(37.1778785,329.04818817)
\curveto(37.17787557,328.95818781)(37.16287559,328.88318789)(37.1328785,328.82318817)
\curveto(37.08287567,328.71318806)(37.00287575,328.64818812)(36.8928785,328.62818817)
\curveto(36.78287597,328.60818816)(36.6478761,328.59818817)(36.4878785,328.59818817)
\lineto(35.7378785,328.59818817)
\curveto(35.62787712,328.59818817)(35.51787723,328.60318817)(35.4078785,328.61318817)
\curveto(35.29787745,328.62318815)(35.21787753,328.65818811)(35.1678785,328.71818817)
\curveto(35.09787765,328.80818796)(35.06287769,328.93818783)(35.0628785,329.10818817)
\curveto(35.07287768,329.27818749)(35.07787767,329.43818733)(35.0778785,329.58818817)
\lineto(35.0778785,331.62818817)
\lineto(35.0778785,334.92818817)
\lineto(35.0778785,335.69318817)
\lineto(35.0778785,335.99318817)
\curveto(35.08787766,336.08318069)(35.11787763,336.15818061)(35.1678785,336.21818817)
\curveto(35.18787756,336.24818052)(35.21787753,336.2681805)(35.2578785,336.27818817)
\curveto(35.30787744,336.29818047)(35.35787739,336.31318046)(35.4078785,336.32318817)
\lineto(35.4828785,336.32318817)
\curveto(35.53287722,336.33318044)(35.58287717,336.33818043)(35.6328785,336.33818817)
\lineto(35.7978785,336.33818817)
\lineto(36.4278785,336.33818817)
\curveto(36.50787624,336.33818043)(36.58287617,336.33318044)(36.6528785,336.32318817)
\curveto(36.73287602,336.32318045)(36.80287595,336.31318046)(36.8628785,336.29318817)
\curveto(36.93287582,336.26318051)(36.97787577,336.21818055)(36.9978785,336.15818817)
\curveto(37.02787572,336.09818067)(37.0528757,336.02818074)(37.0728785,335.94818817)
\curveto(37.08287567,335.90818086)(37.08287567,335.8731809)(37.0728785,335.84318817)
\curveto(37.07287568,335.81318096)(37.08287567,335.78318099)(37.1028785,335.75318817)
\curveto(37.12287563,335.70318107)(37.13787561,335.6731811)(37.1478785,335.66318817)
\curveto(37.16787558,335.65318112)(37.19287556,335.63818113)(37.2228785,335.61818817)
\curveto(37.33287542,335.60818116)(37.42287533,335.64318113)(37.4928785,335.72318817)
\curveto(37.56287519,335.81318096)(37.63787511,335.88318089)(37.7178785,335.93318817)
\curveto(37.98787476,336.13318064)(38.28787446,336.29318048)(38.6178785,336.41318817)
\curveto(38.70787404,336.44318033)(38.79787395,336.46318031)(38.8878785,336.47318817)
\curveto(38.98787376,336.48318029)(39.09287366,336.49818027)(39.2028785,336.51818817)
\curveto(39.23287352,336.52818024)(39.27787347,336.52818024)(39.3378785,336.51818817)
\curveto(39.39787335,336.51818025)(39.43787331,336.52318025)(39.4578785,336.53318817)
}
}
{
\newrgbcolor{curcolor}{0 0 0}
\pscustom[linestyle=none,fillstyle=solid,fillcolor=curcolor]
{
\newpath
\moveto(50.9891285,329.19818817)
\curveto(51.00912065,329.08818768)(51.01912064,328.97818779)(51.0191285,328.86818817)
\curveto(51.02912063,328.75818801)(50.97912068,328.68318809)(50.8691285,328.64318817)
\curveto(50.80912085,328.61318816)(50.73912092,328.59818817)(50.6591285,328.59818817)
\lineto(50.4191285,328.59818817)
\lineto(49.6091285,328.59818817)
\lineto(49.3391285,328.59818817)
\curveto(49.2591224,328.60818816)(49.19412246,328.63318814)(49.1441285,328.67318817)
\curveto(49.07412258,328.71318806)(49.01912264,328.768188)(48.9791285,328.83818817)
\curveto(48.94912271,328.91818785)(48.90412275,328.98318779)(48.8441285,329.03318817)
\curveto(48.82412283,329.05318772)(48.79912286,329.0681877)(48.7691285,329.07818817)
\curveto(48.73912292,329.09818767)(48.69912296,329.10318767)(48.6491285,329.09318817)
\curveto(48.59912306,329.0731877)(48.54912311,329.04818772)(48.4991285,329.01818817)
\curveto(48.4591232,328.98818778)(48.41412324,328.96318781)(48.3641285,328.94318817)
\curveto(48.31412334,328.90318787)(48.2591234,328.8681879)(48.1991285,328.83818817)
\lineto(48.0191285,328.74818817)
\curveto(47.88912377,328.68818808)(47.7541239,328.63818813)(47.6141285,328.59818817)
\curveto(47.47412418,328.5681882)(47.32912433,328.53318824)(47.1791285,328.49318817)
\curveto(47.10912455,328.4731883)(47.03912462,328.46318831)(46.9691285,328.46318817)
\curveto(46.90912475,328.45318832)(46.84412481,328.44318833)(46.7741285,328.43318817)
\lineto(46.6841285,328.43318817)
\curveto(46.654125,328.42318835)(46.62412503,328.41818835)(46.5941285,328.41818817)
\lineto(46.4291285,328.41818817)
\curveto(46.32912533,328.39818837)(46.22912543,328.39818837)(46.1291285,328.41818817)
\lineto(45.9941285,328.41818817)
\curveto(45.92412573,328.43818833)(45.8541258,328.44818832)(45.7841285,328.44818817)
\curveto(45.72412593,328.43818833)(45.66412599,328.44318833)(45.6041285,328.46318817)
\curveto(45.50412615,328.48318829)(45.40912625,328.50318827)(45.3191285,328.52318817)
\curveto(45.22912643,328.53318824)(45.14412651,328.55818821)(45.0641285,328.59818817)
\curveto(44.77412688,328.70818806)(44.52412713,328.84818792)(44.3141285,329.01818817)
\curveto(44.11412754,329.19818757)(43.9541277,329.43318734)(43.8341285,329.72318817)
\curveto(43.80412785,329.79318698)(43.77412788,329.8681869)(43.7441285,329.94818817)
\curveto(43.72412793,330.02818674)(43.70412795,330.11318666)(43.6841285,330.20318817)
\curveto(43.66412799,330.25318652)(43.654128,330.30318647)(43.6541285,330.35318817)
\curveto(43.66412799,330.40318637)(43.66412799,330.45318632)(43.6541285,330.50318817)
\curveto(43.64412801,330.53318624)(43.63412802,330.59318618)(43.6241285,330.68318817)
\curveto(43.62412803,330.78318599)(43.62912803,330.85318592)(43.6391285,330.89318817)
\curveto(43.659128,330.99318578)(43.66912799,331.07818569)(43.6691285,331.14818817)
\lineto(43.7591285,331.47818817)
\curveto(43.78912787,331.59818517)(43.82912783,331.70318507)(43.8791285,331.79318817)
\curveto(44.04912761,332.08318469)(44.24412741,332.30318447)(44.4641285,332.45318817)
\curveto(44.68412697,332.60318417)(44.96412669,332.73318404)(45.3041285,332.84318817)
\curveto(45.43412622,332.89318388)(45.56912609,332.92818384)(45.7091285,332.94818817)
\curveto(45.84912581,332.9681838)(45.98912567,332.99318378)(46.1291285,333.02318817)
\curveto(46.20912545,333.04318373)(46.29412536,333.05318372)(46.3841285,333.05318817)
\curveto(46.47412518,333.06318371)(46.56412509,333.07818369)(46.6541285,333.09818817)
\curveto(46.72412493,333.11818365)(46.79412486,333.12318365)(46.8641285,333.11318817)
\curveto(46.93412472,333.11318366)(47.00912465,333.12318365)(47.0891285,333.14318817)
\curveto(47.1591245,333.16318361)(47.22912443,333.1731836)(47.2991285,333.17318817)
\curveto(47.36912429,333.1731836)(47.44412421,333.18318359)(47.5241285,333.20318817)
\curveto(47.73412392,333.25318352)(47.92412373,333.29318348)(48.0941285,333.32318817)
\curveto(48.27412338,333.36318341)(48.43412322,333.45318332)(48.5741285,333.59318817)
\curveto(48.66412299,333.68318309)(48.72412293,333.78318299)(48.7541285,333.89318817)
\curveto(48.76412289,333.92318285)(48.76412289,333.94818282)(48.7541285,333.96818817)
\curveto(48.7541229,333.98818278)(48.7591229,334.00818276)(48.7691285,334.02818817)
\curveto(48.77912288,334.04818272)(48.78412287,334.07818269)(48.7841285,334.11818817)
\lineto(48.7841285,334.20818817)
\lineto(48.7541285,334.32818817)
\curveto(48.7541229,334.3681824)(48.74912291,334.40318237)(48.7391285,334.43318817)
\curveto(48.63912302,334.73318204)(48.42912323,334.93818183)(48.1091285,335.04818817)
\curveto(48.01912364,335.07818169)(47.90912375,335.09818167)(47.7791285,335.10818817)
\curveto(47.659124,335.12818164)(47.53412412,335.13318164)(47.4041285,335.12318817)
\curveto(47.27412438,335.12318165)(47.14912451,335.11318166)(47.0291285,335.09318817)
\curveto(46.90912475,335.0731817)(46.80412485,335.04818172)(46.7141285,335.01818817)
\curveto(46.654125,334.99818177)(46.59412506,334.9681818)(46.5341285,334.92818817)
\curveto(46.48412517,334.89818187)(46.43412522,334.86318191)(46.3841285,334.82318817)
\curveto(46.33412532,334.78318199)(46.27912538,334.72818204)(46.2191285,334.65818817)
\curveto(46.16912549,334.58818218)(46.13412552,334.52318225)(46.1141285,334.46318817)
\curveto(46.06412559,334.36318241)(46.01912564,334.2681825)(45.9791285,334.17818817)
\curveto(45.94912571,334.08818268)(45.87912578,334.02818274)(45.7691285,333.99818817)
\curveto(45.68912597,333.97818279)(45.60412605,333.9681828)(45.5141285,333.96818817)
\lineto(45.2441285,333.96818817)
\lineto(44.6741285,333.96818817)
\curveto(44.62412703,333.9681828)(44.57412708,333.96318281)(44.5241285,333.95318817)
\curveto(44.47412718,333.95318282)(44.42912723,333.95818281)(44.3891285,333.96818817)
\lineto(44.2541285,333.96818817)
\curveto(44.23412742,333.97818279)(44.20912745,333.98318279)(44.1791285,333.98318817)
\curveto(44.14912751,333.98318279)(44.12412753,333.99318278)(44.1041285,334.01318817)
\curveto(44.02412763,334.03318274)(43.96912769,334.09818267)(43.9391285,334.20818817)
\curveto(43.92912773,334.25818251)(43.92912773,334.30818246)(43.9391285,334.35818817)
\curveto(43.94912771,334.40818236)(43.9591277,334.45318232)(43.9691285,334.49318817)
\curveto(43.99912766,334.60318217)(44.02912763,334.70318207)(44.0591285,334.79318817)
\curveto(44.09912756,334.89318188)(44.14412751,334.98318179)(44.1941285,335.06318817)
\lineto(44.2841285,335.21318817)
\lineto(44.3741285,335.36318817)
\curveto(44.4541272,335.4731813)(44.5541271,335.57818119)(44.6741285,335.67818817)
\curveto(44.69412696,335.68818108)(44.72412693,335.71318106)(44.7641285,335.75318817)
\curveto(44.81412684,335.79318098)(44.8591268,335.82818094)(44.8991285,335.85818817)
\curveto(44.93912672,335.88818088)(44.98412667,335.91818085)(45.0341285,335.94818817)
\curveto(45.20412645,336.05818071)(45.38412627,336.14318063)(45.5741285,336.20318817)
\curveto(45.76412589,336.2731805)(45.9591257,336.33818043)(46.1591285,336.39818817)
\curveto(46.27912538,336.42818034)(46.40412525,336.44818032)(46.5341285,336.45818817)
\curveto(46.66412499,336.4681803)(46.79412486,336.48818028)(46.9241285,336.51818817)
\curveto(46.96412469,336.52818024)(47.02412463,336.52818024)(47.1041285,336.51818817)
\curveto(47.19412446,336.50818026)(47.24912441,336.51318026)(47.2691285,336.53318817)
\curveto(47.67912398,336.54318023)(48.06912359,336.52818024)(48.4391285,336.48818817)
\curveto(48.81912284,336.44818032)(49.1591225,336.3731804)(49.4591285,336.26318817)
\curveto(49.76912189,336.15318062)(50.03412162,336.00318077)(50.2541285,335.81318817)
\curveto(50.47412118,335.63318114)(50.64412101,335.39818137)(50.7641285,335.10818817)
\curveto(50.83412082,334.93818183)(50.87412078,334.74318203)(50.8841285,334.52318817)
\curveto(50.89412076,334.30318247)(50.89912076,334.07818269)(50.8991285,333.84818817)
\lineto(50.8991285,330.50318817)
\lineto(50.8991285,329.91818817)
\curveto(50.89912076,329.72818704)(50.91912074,329.55318722)(50.9591285,329.39318817)
\curveto(50.96912069,329.36318741)(50.97412068,329.32818744)(50.9741285,329.28818817)
\curveto(50.97412068,329.25818751)(50.97912068,329.22818754)(50.9891285,329.19818817)
\moveto(48.7841285,331.50818817)
\curveto(48.79412286,331.55818521)(48.79912286,331.61318516)(48.7991285,331.67318817)
\curveto(48.79912286,331.74318503)(48.79412286,331.80318497)(48.7841285,331.85318817)
\curveto(48.76412289,331.91318486)(48.7541229,331.9681848)(48.7541285,332.01818817)
\curveto(48.7541229,332.0681847)(48.73412292,332.10818466)(48.6941285,332.13818817)
\curveto(48.64412301,332.17818459)(48.56912309,332.19818457)(48.4691285,332.19818817)
\curveto(48.42912323,332.18818458)(48.39412326,332.17818459)(48.3641285,332.16818817)
\curveto(48.33412332,332.1681846)(48.29912336,332.16318461)(48.2591285,332.15318817)
\curveto(48.18912347,332.13318464)(48.11412354,332.11818465)(48.0341285,332.10818817)
\curveto(47.9541237,332.09818467)(47.87412378,332.08318469)(47.7941285,332.06318817)
\curveto(47.76412389,332.05318472)(47.71912394,332.04818472)(47.6591285,332.04818817)
\curveto(47.52912413,332.01818475)(47.39912426,331.99818477)(47.2691285,331.98818817)
\curveto(47.13912452,331.97818479)(47.01412464,331.95318482)(46.8941285,331.91318817)
\curveto(46.81412484,331.89318488)(46.73912492,331.8731849)(46.6691285,331.85318817)
\curveto(46.59912506,331.84318493)(46.52912513,331.82318495)(46.4591285,331.79318817)
\curveto(46.24912541,331.70318507)(46.06912559,331.5681852)(45.9191285,331.38818817)
\curveto(45.77912588,331.20818556)(45.72912593,330.95818581)(45.7691285,330.63818817)
\curveto(45.78912587,330.4681863)(45.84412581,330.32818644)(45.9341285,330.21818817)
\curveto(46.00412565,330.10818666)(46.10912555,330.01818675)(46.2491285,329.94818817)
\curveto(46.38912527,329.88818688)(46.53912512,329.84318693)(46.6991285,329.81318817)
\curveto(46.86912479,329.78318699)(47.04412461,329.773187)(47.2241285,329.78318817)
\curveto(47.41412424,329.80318697)(47.58912407,329.83818693)(47.7491285,329.88818817)
\curveto(48.00912365,329.9681868)(48.21412344,330.09318668)(48.3641285,330.26318817)
\curveto(48.51412314,330.44318633)(48.62912303,330.66318611)(48.7091285,330.92318817)
\curveto(48.72912293,330.99318578)(48.73912292,331.06318571)(48.7391285,331.13318817)
\curveto(48.74912291,331.21318556)(48.76412289,331.29318548)(48.7841285,331.37318817)
\lineto(48.7841285,331.50818817)
}
}
{
\newrgbcolor{curcolor}{0 0 0}
\pscustom[linestyle=none,fillstyle=solid,fillcolor=curcolor]
{
}
}
{
\newrgbcolor{curcolor}{0 0 0}
\pscustom[linestyle=none,fillstyle=solid,fillcolor=curcolor]
{
\newpath
\moveto(59.682566,339.29318817)
\curveto(59.77256216,339.29317748)(59.87256206,339.29317748)(59.982566,339.29318817)
\curveto(60.10256183,339.29317748)(60.21756171,339.28817748)(60.327566,339.27818817)
\curveto(60.44756148,339.2681775)(60.55256138,339.24817752)(60.642566,339.21818817)
\curveto(60.7325612,339.19817757)(60.79256114,339.16317761)(60.822566,339.11318817)
\curveto(60.88256105,339.03317774)(60.91256102,338.91817785)(60.912566,338.76818817)
\lineto(60.912566,338.36318817)
\curveto(60.91256102,338.26317851)(60.90756102,338.16317861)(60.897566,338.06318817)
\curveto(60.89756103,337.96317881)(60.87756105,337.88817888)(60.837566,337.83818817)
\curveto(60.79756113,337.77817899)(60.74756118,337.73817903)(60.687566,337.71818817)
\curveto(60.6275613,337.70817906)(60.55756137,337.70317907)(60.477566,337.70318817)
\lineto(60.252566,337.70318817)
\curveto(60.18256175,337.71317906)(60.11256182,337.71317906)(60.042566,337.70318817)
\curveto(59.86256207,337.66317911)(59.72256221,337.61317916)(59.622566,337.55318817)
\curveto(59.52256241,337.50317927)(59.44256249,337.39317938)(59.382566,337.22318817)
\curveto(59.36256257,337.19317958)(59.35256258,337.16317961)(59.352566,337.13318817)
\curveto(59.36256257,337.11317966)(59.36256257,337.08817968)(59.352566,337.05818817)
\curveto(59.34256259,337.01817975)(59.3325626,336.95817981)(59.322566,336.87818817)
\curveto(59.31256262,336.79817997)(59.31256262,336.73318004)(59.322566,336.68318817)
\curveto(59.34256259,336.61318016)(59.36756256,336.55318022)(59.397566,336.50318817)
\curveto(59.4275625,336.45318032)(59.47256246,336.41318036)(59.532566,336.38318817)
\curveto(59.6325623,336.33318044)(59.75256218,336.31818045)(59.892566,336.33818817)
\curveto(60.0325619,336.35818041)(60.16256177,336.35818041)(60.282566,336.33818817)
\curveto(60.3325616,336.32818044)(60.37256156,336.32318045)(60.402566,336.32318817)
\curveto(60.44256149,336.33318044)(60.48256145,336.33318044)(60.522566,336.32318817)
\curveto(60.61256132,336.28318049)(60.67756125,336.23818053)(60.717566,336.18818817)
\curveto(60.73756119,336.15818061)(60.75256118,336.10818066)(60.762566,336.03818817)
\curveto(60.77256116,335.97818079)(60.78256115,335.90818086)(60.792566,335.82818817)
\curveto(60.80256113,335.75818101)(60.80256113,335.68318109)(60.792566,335.60318817)
\curveto(60.79256114,335.53318124)(60.78756114,335.47818129)(60.777566,335.43818817)
\curveto(60.76756116,335.39818137)(60.76756116,335.35818141)(60.777566,335.31818817)
\curveto(60.78756114,335.28818148)(60.78256115,335.25318152)(60.762566,335.21318817)
\curveto(60.74256119,335.09318168)(60.68256125,335.01818175)(60.582566,334.98818817)
\curveto(60.50256143,334.94818182)(60.40756152,334.92818184)(60.297566,334.92818817)
\curveto(60.18756174,334.93818183)(60.07756185,334.94318183)(59.967566,334.94318817)
\lineto(59.862566,334.94318817)
\curveto(59.82256211,334.94318183)(59.78756214,334.93818183)(59.757566,334.92818817)
\lineto(59.637566,334.92818817)
\curveto(59.46756246,334.88818188)(59.36256257,334.77818199)(59.322566,334.59818817)
\curveto(59.30256263,334.53818223)(59.29756263,334.4681823)(59.307566,334.38818817)
\curveto(59.31756261,334.30818246)(59.32256261,334.22818254)(59.322566,334.14818817)
\lineto(59.322566,333.23318817)
\lineto(59.322566,330.30818817)
\lineto(59.322566,329.60318817)
\lineto(59.322566,329.40818817)
\curveto(59.3325626,329.34818742)(59.3275626,329.29318748)(59.307566,329.24318817)
\lineto(59.307566,329.07818817)
\curveto(59.30756262,328.91818785)(59.28256265,328.80318797)(59.232566,328.73318817)
\curveto(59.21256272,328.70318807)(59.17756275,328.67818809)(59.127566,328.65818817)
\curveto(59.07756285,328.64818812)(59.0275629,328.63318814)(58.977566,328.61318817)
\lineto(58.902566,328.61318817)
\curveto(58.85256308,328.60318817)(58.79756313,328.59818817)(58.737566,328.59818817)
\curveto(58.67756325,328.60818816)(58.62256331,328.61318816)(58.572566,328.61318817)
\lineto(57.912566,328.61318817)
\curveto(57.84256409,328.61318816)(57.76756416,328.60818816)(57.687566,328.59818817)
\curveto(57.61756431,328.59818817)(57.55756437,328.60818816)(57.507566,328.62818817)
\curveto(57.38756454,328.65818811)(57.30756462,328.70818806)(57.267566,328.77818817)
\curveto(57.23756469,328.82818794)(57.21756471,328.89318788)(57.207566,328.97318817)
\lineto(57.207566,329.21318817)
\lineto(57.207566,329.99318817)
\lineto(57.207566,334.19318817)
\curveto(57.20756472,334.36318241)(57.19756473,334.50818226)(57.177566,334.62818817)
\curveto(57.15756477,334.75818201)(57.08756484,334.84818192)(56.967566,334.89818817)
\curveto(56.85756507,334.94818182)(56.72256521,334.95818181)(56.562566,334.92818817)
\curveto(56.40256553,334.90818186)(56.26756566,334.92318185)(56.157566,334.97318817)
\curveto(56.04756588,335.02318175)(55.97756595,335.10818166)(55.947566,335.22818817)
\curveto(55.927566,335.27818149)(55.92256601,335.33818143)(55.932566,335.40818817)
\lineto(55.932566,335.61818817)
\curveto(55.932566,335.79818097)(55.94256599,335.94818082)(55.962566,336.06818817)
\curveto(55.98256595,336.18818058)(56.06756586,336.2731805)(56.217566,336.32318817)
\curveto(56.29756563,336.34318043)(56.38256555,336.35318042)(56.472566,336.35318817)
\lineto(56.727566,336.35318817)
\curveto(56.81756511,336.35318042)(56.89756503,336.35818041)(56.967566,336.36818817)
\curveto(57.03756489,336.38818038)(57.09256484,336.42818034)(57.132566,336.48818817)
\curveto(57.20256473,336.58818018)(57.2275647,336.71318006)(57.207566,336.86318817)
\curveto(57.19756473,337.02317975)(57.20756472,337.1731796)(57.237566,337.31318817)
\curveto(57.24756468,337.35317942)(57.25256468,337.39317938)(57.252566,337.43318817)
\curveto(57.26256467,337.4731793)(57.27256466,337.51817925)(57.282566,337.56818817)
\curveto(57.32256461,337.70817906)(57.36256457,337.83317894)(57.402566,337.94318817)
\curveto(57.44256449,338.06317871)(57.49756443,338.1731786)(57.567566,338.27318817)
\curveto(57.70756422,338.51317826)(57.89256404,338.70317807)(58.122566,338.84318817)
\curveto(58.35256358,338.99317778)(58.61256332,339.10817766)(58.902566,339.18818817)
\curveto(58.98256295,339.21817755)(59.06756286,339.23317754)(59.157566,339.23318817)
\curveto(59.24756268,339.24317753)(59.33756259,339.25817751)(59.427566,339.27818817)
\curveto(59.45756247,339.28817748)(59.50256243,339.28817748)(59.562566,339.27818817)
\curveto(59.62256231,339.2681775)(59.66256227,339.2731775)(59.682566,339.29318817)
}
}
{
\newrgbcolor{curcolor}{0 0 0}
\pscustom[linestyle=none,fillstyle=solid,fillcolor=curcolor]
{
\newpath
\moveto(69.49233162,332.78318817)
\curveto(69.51232305,332.72318405)(69.52232304,332.63818413)(69.52233162,332.52818817)
\curveto(69.52232304,332.41818435)(69.51232305,332.33318444)(69.49233162,332.27318817)
\lineto(69.49233162,332.12318817)
\curveto(69.47232309,332.04318473)(69.4623231,331.96318481)(69.46233162,331.88318817)
\curveto(69.47232309,331.80318497)(69.4673231,331.72318505)(69.44733162,331.64318817)
\curveto(69.42732314,331.5731852)(69.41232315,331.50818526)(69.40233162,331.44818817)
\curveto(69.39232317,331.38818538)(69.38232318,331.32318545)(69.37233162,331.25318817)
\curveto(69.33232323,331.14318563)(69.29732327,331.02818574)(69.26733162,330.90818817)
\curveto(69.23732333,330.79818597)(69.19732337,330.69318608)(69.14733162,330.59318817)
\curveto(68.93732363,330.11318666)(68.6623239,329.72318705)(68.32233162,329.42318817)
\curveto(67.98232458,329.12318765)(67.57232499,328.8731879)(67.09233162,328.67318817)
\curveto(66.97232559,328.62318815)(66.84732572,328.58818818)(66.71733162,328.56818817)
\curveto(66.59732597,328.53818823)(66.47232609,328.50818826)(66.34233162,328.47818817)
\curveto(66.29232627,328.45818831)(66.23732633,328.44818832)(66.17733162,328.44818817)
\curveto(66.11732645,328.44818832)(66.0623265,328.44318833)(66.01233162,328.43318817)
\lineto(65.90733162,328.43318817)
\curveto(65.87732669,328.42318835)(65.84732672,328.41818835)(65.81733162,328.41818817)
\curveto(65.7673268,328.40818836)(65.68732688,328.40318837)(65.57733162,328.40318817)
\curveto(65.4673271,328.39318838)(65.38232718,328.39818837)(65.32233162,328.41818817)
\lineto(65.17233162,328.41818817)
\curveto(65.12232744,328.42818834)(65.0673275,328.43318834)(65.00733162,328.43318817)
\curveto(64.95732761,328.42318835)(64.90732766,328.42818834)(64.85733162,328.44818817)
\curveto(64.81732775,328.45818831)(64.77732779,328.46318831)(64.73733162,328.46318817)
\curveto(64.70732786,328.46318831)(64.6673279,328.4681883)(64.61733162,328.47818817)
\curveto(64.51732805,328.50818826)(64.41732815,328.53318824)(64.31733162,328.55318817)
\curveto(64.21732835,328.5731882)(64.12232844,328.60318817)(64.03233162,328.64318817)
\curveto(63.91232865,328.68318809)(63.79732877,328.72318805)(63.68733162,328.76318817)
\curveto(63.58732898,328.80318797)(63.48232908,328.85318792)(63.37233162,328.91318817)
\curveto(63.02232954,329.12318765)(62.72232984,329.3681874)(62.47233162,329.64818817)
\curveto(62.22233034,329.92818684)(62.01233055,330.26318651)(61.84233162,330.65318817)
\curveto(61.79233077,330.74318603)(61.75233081,330.83818593)(61.72233162,330.93818817)
\curveto(61.70233086,331.03818573)(61.67733089,331.14318563)(61.64733162,331.25318817)
\curveto(61.62733094,331.30318547)(61.61733095,331.34818542)(61.61733162,331.38818817)
\curveto(61.61733095,331.42818534)(61.60733096,331.4731853)(61.58733162,331.52318817)
\curveto(61.567331,331.60318517)(61.55733101,331.68318509)(61.55733162,331.76318817)
\curveto(61.55733101,331.85318492)(61.54733102,331.93818483)(61.52733162,332.01818817)
\curveto(61.51733105,332.0681847)(61.51233105,332.11318466)(61.51233162,332.15318817)
\lineto(61.51233162,332.28818817)
\curveto(61.49233107,332.34818442)(61.48233108,332.43318434)(61.48233162,332.54318817)
\curveto(61.49233107,332.65318412)(61.50733106,332.73818403)(61.52733162,332.79818817)
\lineto(61.52733162,332.90318817)
\curveto(61.53733103,332.95318382)(61.53733103,333.00318377)(61.52733162,333.05318817)
\curveto(61.52733104,333.11318366)(61.53733103,333.1681836)(61.55733162,333.21818817)
\curveto(61.567331,333.2681835)(61.57233099,333.31318346)(61.57233162,333.35318817)
\curveto(61.57233099,333.40318337)(61.58233098,333.45318332)(61.60233162,333.50318817)
\curveto(61.64233092,333.63318314)(61.67733089,333.75818301)(61.70733162,333.87818817)
\curveto(61.73733083,334.00818276)(61.77733079,334.13318264)(61.82733162,334.25318817)
\curveto(62.00733056,334.66318211)(62.22233034,335.00318177)(62.47233162,335.27318817)
\curveto(62.72232984,335.55318122)(63.02732954,335.80818096)(63.38733162,336.03818817)
\curveto(63.48732908,336.08818068)(63.59232897,336.13318064)(63.70233162,336.17318817)
\curveto(63.81232875,336.21318056)(63.92232864,336.25818051)(64.03233162,336.30818817)
\curveto(64.1623284,336.35818041)(64.29732827,336.39318038)(64.43733162,336.41318817)
\curveto(64.57732799,336.43318034)(64.72232784,336.46318031)(64.87233162,336.50318817)
\curveto(64.95232761,336.51318026)(65.02732754,336.51818025)(65.09733162,336.51818817)
\curveto(65.1673274,336.51818025)(65.23732733,336.52318025)(65.30733162,336.53318817)
\curveto(65.88732668,336.54318023)(66.38732618,336.48318029)(66.80733162,336.35318817)
\curveto(67.23732533,336.22318055)(67.61732495,336.04318073)(67.94733162,335.81318817)
\curveto(68.05732451,335.73318104)(68.1673244,335.64318113)(68.27733162,335.54318817)
\curveto(68.39732417,335.45318132)(68.49732407,335.35318142)(68.57733162,335.24318817)
\curveto(68.65732391,335.14318163)(68.72732384,335.04318173)(68.78733162,334.94318817)
\curveto(68.85732371,334.84318193)(68.92732364,334.73818203)(68.99733162,334.62818817)
\curveto(69.0673235,334.51818225)(69.12232344,334.39818237)(69.16233162,334.26818817)
\curveto(69.20232336,334.14818262)(69.24732332,334.01818275)(69.29733162,333.87818817)
\curveto(69.32732324,333.79818297)(69.35232321,333.71318306)(69.37233162,333.62318817)
\lineto(69.43233162,333.35318817)
\curveto(69.44232312,333.31318346)(69.44732312,333.2731835)(69.44733162,333.23318817)
\curveto(69.44732312,333.19318358)(69.45232311,333.15318362)(69.46233162,333.11318817)
\curveto(69.48232308,333.06318371)(69.48732308,333.00818376)(69.47733162,332.94818817)
\curveto(69.4673231,332.88818388)(69.47232309,332.83318394)(69.49233162,332.78318817)
\moveto(67.39233162,332.24318817)
\curveto(67.40232516,332.29318448)(67.40732516,332.36318441)(67.40733162,332.45318817)
\curveto(67.40732516,332.55318422)(67.40232516,332.62818414)(67.39233162,332.67818817)
\lineto(67.39233162,332.79818817)
\curveto(67.37232519,332.84818392)(67.3623252,332.90318387)(67.36233162,332.96318817)
\curveto(67.3623252,333.02318375)(67.35732521,333.07818369)(67.34733162,333.12818817)
\curveto(67.34732522,333.1681836)(67.34232522,333.19818357)(67.33233162,333.21818817)
\lineto(67.27233162,333.45818817)
\curveto(67.2623253,333.54818322)(67.24232532,333.63318314)(67.21233162,333.71318817)
\curveto(67.10232546,333.9731828)(66.97232559,334.19318258)(66.82233162,334.37318817)
\curveto(66.67232589,334.56318221)(66.47232609,334.71318206)(66.22233162,334.82318817)
\curveto(66.1623264,334.84318193)(66.10232646,334.85818191)(66.04233162,334.86818817)
\curveto(65.98232658,334.88818188)(65.91732665,334.90818186)(65.84733162,334.92818817)
\curveto(65.7673268,334.94818182)(65.68232688,334.95318182)(65.59233162,334.94318817)
\lineto(65.32233162,334.94318817)
\curveto(65.29232727,334.92318185)(65.25732731,334.91318186)(65.21733162,334.91318817)
\curveto(65.17732739,334.92318185)(65.14232742,334.92318185)(65.11233162,334.91318817)
\lineto(64.90233162,334.85318817)
\curveto(64.84232772,334.84318193)(64.78732778,334.82318195)(64.73733162,334.79318817)
\curveto(64.48732808,334.68318209)(64.28232828,334.52318225)(64.12233162,334.31318817)
\curveto(63.97232859,334.11318266)(63.85232871,333.87818289)(63.76233162,333.60818817)
\curveto(63.73232883,333.50818326)(63.70732886,333.40318337)(63.68733162,333.29318817)
\curveto(63.67732889,333.18318359)(63.6623289,333.0731837)(63.64233162,332.96318817)
\curveto(63.63232893,332.91318386)(63.62732894,332.86318391)(63.62733162,332.81318817)
\lineto(63.62733162,332.66318817)
\curveto(63.60732896,332.59318418)(63.59732897,332.48818428)(63.59733162,332.34818817)
\curveto(63.60732896,332.20818456)(63.62232894,332.10318467)(63.64233162,332.03318817)
\lineto(63.64233162,331.89818817)
\curveto(63.6623289,331.81818495)(63.67732889,331.73818503)(63.68733162,331.65818817)
\curveto(63.69732887,331.58818518)(63.71232885,331.51318526)(63.73233162,331.43318817)
\curveto(63.83232873,331.13318564)(63.93732863,330.88818588)(64.04733162,330.69818817)
\curveto(64.1673284,330.51818625)(64.35232821,330.35318642)(64.60233162,330.20318817)
\curveto(64.67232789,330.15318662)(64.74732782,330.11318666)(64.82733162,330.08318817)
\curveto(64.91732765,330.05318672)(65.00732756,330.02818674)(65.09733162,330.00818817)
\curveto(65.13732743,329.99818677)(65.17232739,329.99318678)(65.20233162,329.99318817)
\curveto(65.23232733,330.00318677)(65.2673273,330.00318677)(65.30733162,329.99318817)
\lineto(65.42733162,329.96318817)
\curveto(65.47732709,329.96318681)(65.52232704,329.9681868)(65.56233162,329.97818817)
\lineto(65.68233162,329.97818817)
\curveto(65.7623268,329.99818677)(65.84232672,330.01318676)(65.92233162,330.02318817)
\curveto(66.00232656,330.03318674)(66.07732649,330.05318672)(66.14733162,330.08318817)
\curveto(66.40732616,330.18318659)(66.61732595,330.31818645)(66.77733162,330.48818817)
\curveto(66.93732563,330.65818611)(67.07232549,330.8681859)(67.18233162,331.11818817)
\curveto(67.22232534,331.21818555)(67.25232531,331.31818545)(67.27233162,331.41818817)
\curveto(67.29232527,331.51818525)(67.31732525,331.62318515)(67.34733162,331.73318817)
\curveto(67.35732521,331.773185)(67.3623252,331.80818496)(67.36233162,331.83818817)
\curveto(67.3623252,331.87818489)(67.3673252,331.91818485)(67.37733162,331.95818817)
\lineto(67.37733162,332.09318817)
\curveto(67.37732519,332.14318463)(67.38232518,332.19318458)(67.39233162,332.24318817)
}
}
{
\newrgbcolor{curcolor}{0 0 0}
\pscustom[linestyle=none,fillstyle=solid,fillcolor=curcolor]
{
\newpath
\moveto(71.9422535,338.64818817)
\lineto(72.9472535,338.64818817)
\curveto(73.09725051,338.64817812)(73.22725038,338.63817813)(73.3372535,338.61818817)
\curveto(73.45725015,338.60817816)(73.54225007,338.54817822)(73.5922535,338.43818817)
\curveto(73.61225,338.38817838)(73.62224999,338.32817844)(73.6222535,338.25818817)
\lineto(73.6222535,338.04818817)
\lineto(73.6222535,337.37318817)
\curveto(73.62224999,337.32317945)(73.61724999,337.26317951)(73.6072535,337.19318817)
\curveto(73.60725,337.13317964)(73.61225,337.07817969)(73.6222535,337.02818817)
\lineto(73.6222535,336.86318817)
\curveto(73.62224999,336.78317999)(73.62724998,336.70818006)(73.6372535,336.63818817)
\curveto(73.64724996,336.57818019)(73.67224994,336.52318025)(73.7122535,336.47318817)
\curveto(73.78224983,336.38318039)(73.9072497,336.33318044)(74.0872535,336.32318817)
\lineto(74.6272535,336.32318817)
\lineto(74.8072535,336.32318817)
\curveto(74.86724874,336.32318045)(74.92224869,336.31318046)(74.9722535,336.29318817)
\curveto(75.08224853,336.24318053)(75.14224847,336.15318062)(75.1522535,336.02318817)
\curveto(75.17224844,335.89318088)(75.18224843,335.74818102)(75.1822535,335.58818817)
\lineto(75.1822535,335.37818817)
\curveto(75.19224842,335.30818146)(75.18724842,335.24818152)(75.1672535,335.19818817)
\curveto(75.11724849,335.03818173)(75.0122486,334.95318182)(74.8522535,334.94318817)
\curveto(74.69224892,334.93318184)(74.5122491,334.92818184)(74.3122535,334.92818817)
\lineto(74.1772535,334.92818817)
\curveto(74.13724947,334.93818183)(74.10224951,334.93818183)(74.0722535,334.92818817)
\curveto(74.03224958,334.91818185)(73.99724961,334.91318186)(73.9672535,334.91318817)
\curveto(73.93724967,334.92318185)(73.9072497,334.91818185)(73.8772535,334.89818817)
\curveto(73.79724981,334.87818189)(73.73724987,334.83318194)(73.6972535,334.76318817)
\curveto(73.66724994,334.70318207)(73.64224997,334.62818214)(73.6222535,334.53818817)
\curveto(73.61225,334.48818228)(73.61225,334.43318234)(73.6222535,334.37318817)
\curveto(73.63224998,334.31318246)(73.63224998,334.25818251)(73.6222535,334.20818817)
\lineto(73.6222535,333.27818817)
\lineto(73.6222535,331.52318817)
\curveto(73.62224999,331.2731855)(73.62724998,331.05318572)(73.6372535,330.86318817)
\curveto(73.65724995,330.68318609)(73.72224989,330.52318625)(73.8322535,330.38318817)
\curveto(73.88224973,330.32318645)(73.94724966,330.27818649)(74.0272535,330.24818817)
\lineto(74.2972535,330.18818817)
\curveto(74.32724928,330.17818659)(74.35724925,330.1731866)(74.3872535,330.17318817)
\curveto(74.42724918,330.18318659)(74.45724915,330.18318659)(74.4772535,330.17318817)
\lineto(74.6422535,330.17318817)
\curveto(74.75224886,330.1731866)(74.84724876,330.1681866)(74.9272535,330.15818817)
\curveto(75.0072486,330.14818662)(75.07224854,330.10818666)(75.1222535,330.03818817)
\curveto(75.16224845,329.97818679)(75.18224843,329.89818687)(75.1822535,329.79818817)
\lineto(75.1822535,329.51318817)
\curveto(75.18224843,329.30318747)(75.17724843,329.10818766)(75.1672535,328.92818817)
\curveto(75.16724844,328.75818801)(75.08724852,328.64318813)(74.9272535,328.58318817)
\curveto(74.87724873,328.56318821)(74.83224878,328.55818821)(74.7922535,328.56818817)
\curveto(74.75224886,328.5681882)(74.7072489,328.55818821)(74.6572535,328.53818817)
\lineto(74.5072535,328.53818817)
\curveto(74.48724912,328.53818823)(74.45724915,328.54318823)(74.4172535,328.55318817)
\curveto(74.37724923,328.55318822)(74.34224927,328.54818822)(74.3122535,328.53818817)
\curveto(74.26224935,328.52818824)(74.2072494,328.52818824)(74.1472535,328.53818817)
\lineto(73.9972535,328.53818817)
\lineto(73.8472535,328.53818817)
\curveto(73.79724981,328.52818824)(73.75224986,328.52818824)(73.7122535,328.53818817)
\lineto(73.5472535,328.53818817)
\curveto(73.49725011,328.54818822)(73.44225017,328.55318822)(73.3822535,328.55318817)
\curveto(73.32225029,328.55318822)(73.26725034,328.55818821)(73.2172535,328.56818817)
\curveto(73.14725046,328.57818819)(73.08225053,328.58818818)(73.0222535,328.59818817)
\lineto(72.8422535,328.62818817)
\curveto(72.73225088,328.65818811)(72.62725098,328.69318808)(72.5272535,328.73318817)
\curveto(72.42725118,328.773188)(72.33225128,328.81818795)(72.2422535,328.86818817)
\lineto(72.1522535,328.92818817)
\curveto(72.12225149,328.95818781)(72.08725152,328.98818778)(72.0472535,329.01818817)
\curveto(72.02725158,329.03818773)(72.00225161,329.05818771)(71.9722535,329.07818817)
\lineto(71.8972535,329.15318817)
\curveto(71.75725185,329.34318743)(71.65225196,329.55318722)(71.5822535,329.78318817)
\curveto(71.56225205,329.82318695)(71.55225206,329.85818691)(71.5522535,329.88818817)
\curveto(71.56225205,329.92818684)(71.56225205,329.9731868)(71.5522535,330.02318817)
\curveto(71.54225207,330.04318673)(71.53725207,330.0681867)(71.5372535,330.09818817)
\curveto(71.53725207,330.12818664)(71.53225208,330.15318662)(71.5222535,330.17318817)
\lineto(71.5222535,330.32318817)
\curveto(71.5122521,330.36318641)(71.5072521,330.40818636)(71.5072535,330.45818817)
\curveto(71.51725209,330.50818626)(71.52225209,330.55818621)(71.5222535,330.60818817)
\lineto(71.5222535,331.17818817)
\lineto(71.5222535,333.41318817)
\lineto(71.5222535,334.20818817)
\lineto(71.5222535,334.41818817)
\curveto(71.53225208,334.48818228)(71.52725208,334.55318222)(71.5072535,334.61318817)
\curveto(71.46725214,334.75318202)(71.39725221,334.84318193)(71.2972535,334.88318817)
\curveto(71.18725242,334.93318184)(71.04725256,334.94818182)(70.8772535,334.92818817)
\curveto(70.7072529,334.90818186)(70.56225305,334.92318185)(70.4422535,334.97318817)
\curveto(70.36225325,335.00318177)(70.3122533,335.04818172)(70.2922535,335.10818817)
\curveto(70.27225334,335.1681816)(70.25225336,335.24318153)(70.2322535,335.33318817)
\lineto(70.2322535,335.64818817)
\curveto(70.23225338,335.82818094)(70.24225337,335.9731808)(70.2622535,336.08318817)
\curveto(70.28225333,336.19318058)(70.36725324,336.2681805)(70.5172535,336.30818817)
\curveto(70.55725305,336.32818044)(70.59725301,336.33318044)(70.6372535,336.32318817)
\lineto(70.7722535,336.32318817)
\curveto(70.92225269,336.32318045)(71.06225255,336.32818044)(71.1922535,336.33818817)
\curveto(71.32225229,336.35818041)(71.4122522,336.41818035)(71.4622535,336.51818817)
\curveto(71.49225212,336.58818018)(71.5072521,336.6681801)(71.5072535,336.75818817)
\curveto(71.51725209,336.84817992)(71.52225209,336.93817983)(71.5222535,337.02818817)
\lineto(71.5222535,337.95818817)
\lineto(71.5222535,338.21318817)
\curveto(71.52225209,338.30317847)(71.53225208,338.37817839)(71.5522535,338.43818817)
\curveto(71.60225201,338.53817823)(71.67725193,338.60317817)(71.7772535,338.63318817)
\curveto(71.79725181,338.64317813)(71.82225179,338.64317813)(71.8522535,338.63318817)
\curveto(71.89225172,338.63317814)(71.92225169,338.63817813)(71.9422535,338.64818817)
}
}
{
\newrgbcolor{curcolor}{0 0 0}
\pscustom[linestyle=none,fillstyle=solid,fillcolor=curcolor]
{
\newpath
\moveto(83.935691,332.78318817)
\curveto(83.95568243,332.72318405)(83.96568242,332.63818413)(83.965691,332.52818817)
\curveto(83.96568242,332.41818435)(83.95568243,332.33318444)(83.935691,332.27318817)
\lineto(83.935691,332.12318817)
\curveto(83.91568247,332.04318473)(83.90568248,331.96318481)(83.905691,331.88318817)
\curveto(83.91568247,331.80318497)(83.91068247,331.72318505)(83.890691,331.64318817)
\curveto(83.87068251,331.5731852)(83.85568253,331.50818526)(83.845691,331.44818817)
\curveto(83.83568255,331.38818538)(83.82568256,331.32318545)(83.815691,331.25318817)
\curveto(83.77568261,331.14318563)(83.74068264,331.02818574)(83.710691,330.90818817)
\curveto(83.6806827,330.79818597)(83.64068274,330.69318608)(83.590691,330.59318817)
\curveto(83.380683,330.11318666)(83.10568328,329.72318705)(82.765691,329.42318817)
\curveto(82.42568396,329.12318765)(82.01568437,328.8731879)(81.535691,328.67318817)
\curveto(81.41568497,328.62318815)(81.29068509,328.58818818)(81.160691,328.56818817)
\curveto(81.04068534,328.53818823)(80.91568547,328.50818826)(80.785691,328.47818817)
\curveto(80.73568565,328.45818831)(80.6806857,328.44818832)(80.620691,328.44818817)
\curveto(80.56068582,328.44818832)(80.50568588,328.44318833)(80.455691,328.43318817)
\lineto(80.350691,328.43318817)
\curveto(80.32068606,328.42318835)(80.29068609,328.41818835)(80.260691,328.41818817)
\curveto(80.21068617,328.40818836)(80.13068625,328.40318837)(80.020691,328.40318817)
\curveto(79.91068647,328.39318838)(79.82568656,328.39818837)(79.765691,328.41818817)
\lineto(79.615691,328.41818817)
\curveto(79.56568682,328.42818834)(79.51068687,328.43318834)(79.450691,328.43318817)
\curveto(79.40068698,328.42318835)(79.35068703,328.42818834)(79.300691,328.44818817)
\curveto(79.26068712,328.45818831)(79.22068716,328.46318831)(79.180691,328.46318817)
\curveto(79.15068723,328.46318831)(79.11068727,328.4681883)(79.060691,328.47818817)
\curveto(78.96068742,328.50818826)(78.86068752,328.53318824)(78.760691,328.55318817)
\curveto(78.66068772,328.5731882)(78.56568782,328.60318817)(78.475691,328.64318817)
\curveto(78.35568803,328.68318809)(78.24068814,328.72318805)(78.130691,328.76318817)
\curveto(78.03068835,328.80318797)(77.92568846,328.85318792)(77.815691,328.91318817)
\curveto(77.46568892,329.12318765)(77.16568922,329.3681874)(76.915691,329.64818817)
\curveto(76.66568972,329.92818684)(76.45568993,330.26318651)(76.285691,330.65318817)
\curveto(76.23569015,330.74318603)(76.19569019,330.83818593)(76.165691,330.93818817)
\curveto(76.14569024,331.03818573)(76.12069026,331.14318563)(76.090691,331.25318817)
\curveto(76.07069031,331.30318547)(76.06069032,331.34818542)(76.060691,331.38818817)
\curveto(76.06069032,331.42818534)(76.05069033,331.4731853)(76.030691,331.52318817)
\curveto(76.01069037,331.60318517)(76.00069038,331.68318509)(76.000691,331.76318817)
\curveto(76.00069038,331.85318492)(75.99069039,331.93818483)(75.970691,332.01818817)
\curveto(75.96069042,332.0681847)(75.95569043,332.11318466)(75.955691,332.15318817)
\lineto(75.955691,332.28818817)
\curveto(75.93569045,332.34818442)(75.92569046,332.43318434)(75.925691,332.54318817)
\curveto(75.93569045,332.65318412)(75.95069043,332.73818403)(75.970691,332.79818817)
\lineto(75.970691,332.90318817)
\curveto(75.9806904,332.95318382)(75.9806904,333.00318377)(75.970691,333.05318817)
\curveto(75.97069041,333.11318366)(75.9806904,333.1681836)(76.000691,333.21818817)
\curveto(76.01069037,333.2681835)(76.01569037,333.31318346)(76.015691,333.35318817)
\curveto(76.01569037,333.40318337)(76.02569036,333.45318332)(76.045691,333.50318817)
\curveto(76.0856903,333.63318314)(76.12069026,333.75818301)(76.150691,333.87818817)
\curveto(76.1806902,334.00818276)(76.22069016,334.13318264)(76.270691,334.25318817)
\curveto(76.45068993,334.66318211)(76.66568972,335.00318177)(76.915691,335.27318817)
\curveto(77.16568922,335.55318122)(77.47068891,335.80818096)(77.830691,336.03818817)
\curveto(77.93068845,336.08818068)(78.03568835,336.13318064)(78.145691,336.17318817)
\curveto(78.25568813,336.21318056)(78.36568802,336.25818051)(78.475691,336.30818817)
\curveto(78.60568778,336.35818041)(78.74068764,336.39318038)(78.880691,336.41318817)
\curveto(79.02068736,336.43318034)(79.16568722,336.46318031)(79.315691,336.50318817)
\curveto(79.39568699,336.51318026)(79.47068691,336.51818025)(79.540691,336.51818817)
\curveto(79.61068677,336.51818025)(79.6806867,336.52318025)(79.750691,336.53318817)
\curveto(80.33068605,336.54318023)(80.83068555,336.48318029)(81.250691,336.35318817)
\curveto(81.6806847,336.22318055)(82.06068432,336.04318073)(82.390691,335.81318817)
\curveto(82.50068388,335.73318104)(82.61068377,335.64318113)(82.720691,335.54318817)
\curveto(82.84068354,335.45318132)(82.94068344,335.35318142)(83.020691,335.24318817)
\curveto(83.10068328,335.14318163)(83.17068321,335.04318173)(83.230691,334.94318817)
\curveto(83.30068308,334.84318193)(83.37068301,334.73818203)(83.440691,334.62818817)
\curveto(83.51068287,334.51818225)(83.56568282,334.39818237)(83.605691,334.26818817)
\curveto(83.64568274,334.14818262)(83.69068269,334.01818275)(83.740691,333.87818817)
\curveto(83.77068261,333.79818297)(83.79568259,333.71318306)(83.815691,333.62318817)
\lineto(83.875691,333.35318817)
\curveto(83.8856825,333.31318346)(83.89068249,333.2731835)(83.890691,333.23318817)
\curveto(83.89068249,333.19318358)(83.89568249,333.15318362)(83.905691,333.11318817)
\curveto(83.92568246,333.06318371)(83.93068245,333.00818376)(83.920691,332.94818817)
\curveto(83.91068247,332.88818388)(83.91568247,332.83318394)(83.935691,332.78318817)
\moveto(81.835691,332.24318817)
\curveto(81.84568454,332.29318448)(81.85068453,332.36318441)(81.850691,332.45318817)
\curveto(81.85068453,332.55318422)(81.84568454,332.62818414)(81.835691,332.67818817)
\lineto(81.835691,332.79818817)
\curveto(81.81568457,332.84818392)(81.80568458,332.90318387)(81.805691,332.96318817)
\curveto(81.80568458,333.02318375)(81.80068458,333.07818369)(81.790691,333.12818817)
\curveto(81.79068459,333.1681836)(81.7856846,333.19818357)(81.775691,333.21818817)
\lineto(81.715691,333.45818817)
\curveto(81.70568468,333.54818322)(81.6856847,333.63318314)(81.655691,333.71318817)
\curveto(81.54568484,333.9731828)(81.41568497,334.19318258)(81.265691,334.37318817)
\curveto(81.11568527,334.56318221)(80.91568547,334.71318206)(80.665691,334.82318817)
\curveto(80.60568578,334.84318193)(80.54568584,334.85818191)(80.485691,334.86818817)
\curveto(80.42568596,334.88818188)(80.36068602,334.90818186)(80.290691,334.92818817)
\curveto(80.21068617,334.94818182)(80.12568626,334.95318182)(80.035691,334.94318817)
\lineto(79.765691,334.94318817)
\curveto(79.73568665,334.92318185)(79.70068668,334.91318186)(79.660691,334.91318817)
\curveto(79.62068676,334.92318185)(79.5856868,334.92318185)(79.555691,334.91318817)
\lineto(79.345691,334.85318817)
\curveto(79.2856871,334.84318193)(79.23068715,334.82318195)(79.180691,334.79318817)
\curveto(78.93068745,334.68318209)(78.72568766,334.52318225)(78.565691,334.31318817)
\curveto(78.41568797,334.11318266)(78.29568809,333.87818289)(78.205691,333.60818817)
\curveto(78.17568821,333.50818326)(78.15068823,333.40318337)(78.130691,333.29318817)
\curveto(78.12068826,333.18318359)(78.10568828,333.0731837)(78.085691,332.96318817)
\curveto(78.07568831,332.91318386)(78.07068831,332.86318391)(78.070691,332.81318817)
\lineto(78.070691,332.66318817)
\curveto(78.05068833,332.59318418)(78.04068834,332.48818428)(78.040691,332.34818817)
\curveto(78.05068833,332.20818456)(78.06568832,332.10318467)(78.085691,332.03318817)
\lineto(78.085691,331.89818817)
\curveto(78.10568828,331.81818495)(78.12068826,331.73818503)(78.130691,331.65818817)
\curveto(78.14068824,331.58818518)(78.15568823,331.51318526)(78.175691,331.43318817)
\curveto(78.27568811,331.13318564)(78.380688,330.88818588)(78.490691,330.69818817)
\curveto(78.61068777,330.51818625)(78.79568759,330.35318642)(79.045691,330.20318817)
\curveto(79.11568727,330.15318662)(79.19068719,330.11318666)(79.270691,330.08318817)
\curveto(79.36068702,330.05318672)(79.45068693,330.02818674)(79.540691,330.00818817)
\curveto(79.5806868,329.99818677)(79.61568677,329.99318678)(79.645691,329.99318817)
\curveto(79.67568671,330.00318677)(79.71068667,330.00318677)(79.750691,329.99318817)
\lineto(79.870691,329.96318817)
\curveto(79.92068646,329.96318681)(79.96568642,329.9681868)(80.005691,329.97818817)
\lineto(80.125691,329.97818817)
\curveto(80.20568618,329.99818677)(80.2856861,330.01318676)(80.365691,330.02318817)
\curveto(80.44568594,330.03318674)(80.52068586,330.05318672)(80.590691,330.08318817)
\curveto(80.85068553,330.18318659)(81.06068532,330.31818645)(81.220691,330.48818817)
\curveto(81.380685,330.65818611)(81.51568487,330.8681859)(81.625691,331.11818817)
\curveto(81.66568472,331.21818555)(81.69568469,331.31818545)(81.715691,331.41818817)
\curveto(81.73568465,331.51818525)(81.76068462,331.62318515)(81.790691,331.73318817)
\curveto(81.80068458,331.773185)(81.80568458,331.80818496)(81.805691,331.83818817)
\curveto(81.80568458,331.87818489)(81.81068457,331.91818485)(81.820691,331.95818817)
\lineto(81.820691,332.09318817)
\curveto(81.82068456,332.14318463)(81.82568456,332.19318458)(81.835691,332.24318817)
}
}
{
\newrgbcolor{curcolor}{0 0 0}
\pscustom[linestyle=none,fillstyle=solid,fillcolor=curcolor]
{
\newpath
\moveto(92.68561287,336.24818817)
\curveto(92.75560467,336.19818057)(92.79060464,336.11318066)(92.79061287,335.99318817)
\curveto(92.80060463,335.88318089)(92.80560462,335.768181)(92.80561287,335.64818817)
\lineto(92.80561287,329.24318817)
\curveto(92.80560462,329.16318761)(92.80060463,329.08318769)(92.79061287,329.00318817)
\lineto(92.79061287,328.77818817)
\curveto(92.78060465,328.69818807)(92.77060466,328.62818814)(92.76061287,328.56818817)
\curveto(92.76060467,328.49818827)(92.75560467,328.42318835)(92.74561287,328.34318817)
\curveto(92.70560472,328.20318857)(92.67060476,328.0731887)(92.64061287,327.95318817)
\curveto(92.62060481,327.82318895)(92.58560484,327.70318907)(92.53561287,327.59318817)
\curveto(92.36560506,327.21318956)(92.14560528,326.89818987)(91.87561287,326.64818817)
\curveto(91.61560581,326.39819037)(91.29560613,326.19319058)(90.91561287,326.03318817)
\curveto(90.80560662,325.98319079)(90.69560673,325.94319083)(90.58561287,325.91318817)
\curveto(90.47560695,325.88319089)(90.36060707,325.85319092)(90.24061287,325.82318817)
\curveto(90.1306073,325.79319098)(90.02060741,325.773191)(89.91061287,325.76318817)
\curveto(89.80060763,325.75319102)(89.69060774,325.73819103)(89.58061287,325.71818817)
\lineto(89.46061287,325.71818817)
\curveto(89.42060801,325.70819106)(89.37560805,325.70319107)(89.32561287,325.70318817)
\curveto(89.28560814,325.69319108)(89.24060819,325.69319108)(89.19061287,325.70318817)
\curveto(89.14060829,325.70319107)(89.09060834,325.69819107)(89.04061287,325.68818817)
\curveto(88.99060844,325.67819109)(88.9256085,325.6731911)(88.84561287,325.67318817)
\curveto(88.76560866,325.6731911)(88.70060873,325.67819109)(88.65061287,325.68818817)
\lineto(88.51561287,325.68818817)
\curveto(88.47560895,325.68819108)(88.43560899,325.69319108)(88.39561287,325.70318817)
\curveto(88.31560911,325.72319105)(88.2306092,325.73319104)(88.14061287,325.73318817)
\curveto(88.06060937,325.73319104)(87.98560944,325.74319103)(87.91561287,325.76318817)
\curveto(87.89560953,325.773191)(87.87060956,325.77819099)(87.84061287,325.77818817)
\curveto(87.81060962,325.77819099)(87.78560964,325.78319099)(87.76561287,325.79318817)
\curveto(87.66560976,325.81319096)(87.56560986,325.83819093)(87.46561287,325.86818817)
\curveto(87.37561005,325.88819088)(87.28561014,325.91819085)(87.19561287,325.95818817)
\curveto(86.81561061,326.11819065)(86.47561095,326.32319045)(86.17561287,326.57318817)
\curveto(85.87561155,326.81318996)(85.65561177,327.13818963)(85.51561287,327.54818817)
\curveto(85.49561193,327.57818919)(85.48561194,327.60818916)(85.48561287,327.63818817)
\curveto(85.48561194,327.6681891)(85.48061195,327.69318908)(85.47061287,327.71318817)
\curveto(85.44061199,327.84318893)(85.45061198,327.94318883)(85.50061287,328.01318817)
\curveto(85.56061187,328.0731887)(85.64061179,328.11318866)(85.74061287,328.13318817)
\curveto(85.84061159,328.15318862)(85.95061148,328.16318861)(86.07061287,328.16318817)
\curveto(86.20061123,328.15318862)(86.32061111,328.14818862)(86.43061287,328.14818817)
\lineto(86.94061287,328.14818817)
\lineto(87.06061287,328.14818817)
\curveto(87.10061033,328.13818863)(87.14561028,328.13318864)(87.19561287,328.13318817)
\curveto(87.35561007,328.09318868)(87.45560997,328.04318873)(87.49561287,327.98318817)
\curveto(87.53560989,327.91318886)(87.59560983,327.82318895)(87.67561287,327.71318817)
\curveto(87.70560972,327.6731891)(87.75060968,327.62318915)(87.81061287,327.56318817)
\curveto(87.82060961,327.54318923)(87.8306096,327.52818924)(87.84061287,327.51818817)
\curveto(87.85060958,327.50818926)(87.86060957,327.49318928)(87.87061287,327.47318817)
\curveto(87.95060948,327.41318936)(88.03560939,327.35818941)(88.12561287,327.30818817)
\curveto(88.21560921,327.25818951)(88.31560911,327.21318956)(88.42561287,327.17318817)
\curveto(88.49560893,327.15318962)(88.56560886,327.14318963)(88.63561287,327.14318817)
\curveto(88.70560872,327.13318964)(88.78060865,327.11818965)(88.86061287,327.09818817)
\lineto(89.02561287,327.09818817)
\curveto(89.09560833,327.07818969)(89.18560824,327.07818969)(89.29561287,327.09818817)
\curveto(89.40560802,327.10818966)(89.49060794,327.12318965)(89.55061287,327.14318817)
\curveto(89.60060783,327.16318961)(89.64060779,327.1731896)(89.67061287,327.17318817)
\curveto(89.71060772,327.1731896)(89.75060768,327.18318959)(89.79061287,327.20318817)
\curveto(90.00060743,327.29318948)(90.17560725,327.41318936)(90.31561287,327.56318817)
\curveto(90.45560697,327.71318906)(90.57060686,327.88818888)(90.66061287,328.08818817)
\curveto(90.68060675,328.14818862)(90.69560673,328.20818856)(90.70561287,328.26818817)
\curveto(90.71560671,328.32818844)(90.7306067,328.39318838)(90.75061287,328.46318817)
\curveto(90.77060666,328.55318822)(90.78060665,328.64818812)(90.78061287,328.74818817)
\curveto(90.79060664,328.85818791)(90.79560663,328.9681878)(90.79561287,329.07818817)
\lineto(90.79561287,329.19818817)
\curveto(90.80560662,329.23818753)(90.80560662,329.2731875)(90.79561287,329.30318817)
\curveto(90.77560665,329.35318742)(90.76560666,329.39818737)(90.76561287,329.43818817)
\curveto(90.77560665,329.47818729)(90.77060666,329.51818725)(90.75061287,329.55818817)
\curveto(90.74060669,329.57818719)(90.7256067,329.59318718)(90.70561287,329.60318817)
\lineto(90.66061287,329.64818817)
\curveto(90.57060686,329.65818711)(90.49560693,329.63818713)(90.43561287,329.58818817)
\curveto(90.38560704,329.53818723)(90.33560709,329.49318728)(90.28561287,329.45318817)
\curveto(90.19560723,329.38318739)(90.10560732,329.31818745)(90.01561287,329.25818817)
\curveto(89.9256075,329.19818757)(89.8256076,329.14318763)(89.71561287,329.09318817)
\curveto(89.60560782,329.04318773)(89.49560793,329.00318777)(89.38561287,328.97318817)
\curveto(89.27560815,328.94318783)(89.16060827,328.91318786)(89.04061287,328.88318817)
\lineto(88.86061287,328.85318817)
\curveto(88.81060862,328.85318792)(88.76060867,328.84818792)(88.71061287,328.83818817)
\curveto(88.66060877,328.82818794)(88.58060885,328.82318795)(88.47061287,328.82318817)
\curveto(88.36060907,328.82318795)(88.28060915,328.82818794)(88.23061287,328.83818817)
\lineto(88.11061287,328.83818817)
\curveto(88.08060935,328.84818792)(88.04560938,328.85318792)(88.00561287,328.85318817)
\curveto(87.97560945,328.85318792)(87.94060949,328.85818791)(87.90061287,328.86818817)
\curveto(87.76060967,328.89818787)(87.6256098,328.92318785)(87.49561287,328.94318817)
\curveto(87.36561006,328.9731878)(87.24561018,329.01318776)(87.13561287,329.06318817)
\curveto(86.70561072,329.23318754)(86.35561107,329.4681873)(86.08561287,329.76818817)
\curveto(85.8256116,330.07818669)(85.60561182,330.44818632)(85.42561287,330.87818817)
\curveto(85.37561205,330.98818578)(85.34061209,331.10318567)(85.32061287,331.22318817)
\curveto(85.30061213,331.34318543)(85.27061216,331.46318531)(85.23061287,331.58318817)
\curveto(85.2306122,331.63318514)(85.2256122,331.6731851)(85.21561287,331.70318817)
\curveto(85.19561223,331.78318499)(85.18561224,331.8681849)(85.18561287,331.95818817)
\curveto(85.18561224,332.05818471)(85.17561225,332.14818462)(85.15561287,332.22818817)
\curveto(85.14561228,332.27818449)(85.14061229,332.32318445)(85.14061287,332.36318817)
\lineto(85.14061287,332.51318817)
\curveto(85.1306123,332.56318421)(85.1256123,332.62318415)(85.12561287,332.69318817)
\curveto(85.1256123,332.773184)(85.1306123,332.83818393)(85.14061287,332.88818817)
\lineto(85.14061287,333.03818817)
\curveto(85.15061228,333.07818369)(85.15061228,333.11818365)(85.14061287,333.15818817)
\curveto(85.14061229,333.19818357)(85.15061228,333.23818353)(85.17061287,333.27818817)
\curveto(85.19061224,333.37818339)(85.20561222,333.4731833)(85.21561287,333.56318817)
\curveto(85.2256122,333.66318311)(85.24061219,333.76318301)(85.26061287,333.86318817)
\curveto(85.32061211,334.06318271)(85.38061205,334.25318252)(85.44061287,334.43318817)
\curveto(85.51061192,334.61318216)(85.59561183,334.78318199)(85.69561287,334.94318817)
\curveto(85.74561168,335.04318173)(85.80061163,335.13318164)(85.86061287,335.21318817)
\lineto(86.07061287,335.48318817)
\curveto(86.10061133,335.53318124)(86.14061129,335.58318119)(86.19061287,335.63318817)
\curveto(86.25061118,335.68318109)(86.30561112,335.72818104)(86.35561287,335.76818817)
\lineto(86.44561287,335.85818817)
\curveto(86.49561093,335.89818087)(86.54561088,335.93318084)(86.59561287,335.96318817)
\curveto(86.64561078,336.00318077)(86.69561073,336.03818073)(86.74561287,336.06818817)
\curveto(86.87561055,336.14818062)(87.01061042,336.21818055)(87.15061287,336.27818817)
\curveto(87.29061014,336.33818043)(87.44560998,336.39318038)(87.61561287,336.44318817)
\curveto(87.69560973,336.4731803)(87.77560965,336.48818028)(87.85561287,336.48818817)
\curveto(87.94560948,336.49818027)(88.0306094,336.51318026)(88.11061287,336.53318817)
\curveto(88.15060928,336.54318023)(88.20560922,336.54318023)(88.27561287,336.53318817)
\curveto(88.34560908,336.52318025)(88.39060904,336.52818024)(88.41061287,336.54818817)
\curveto(88.7306087,336.55818021)(89.01560841,336.52818024)(89.26561287,336.45818817)
\curveto(89.5256079,336.38818038)(89.75560767,336.28818048)(89.95561287,336.15818817)
\curveto(89.98560744,336.13818063)(90.01560741,336.11318066)(90.04561287,336.08318817)
\curveto(90.07560735,336.06318071)(90.11060732,336.03818073)(90.15061287,336.00818817)
\curveto(90.21060722,335.95818081)(90.26560716,335.90818086)(90.31561287,335.85818817)
\curveto(90.36560706,335.80818096)(90.425607,335.76318101)(90.49561287,335.72318817)
\curveto(90.51560691,335.71318106)(90.54060689,335.70318107)(90.57061287,335.69318817)
\curveto(90.61060682,335.68318109)(90.64060679,335.68818108)(90.66061287,335.70818817)
\curveto(90.71060672,335.72818104)(90.74060669,335.76318101)(90.75061287,335.81318817)
\curveto(90.76060667,335.86318091)(90.77560665,335.91318086)(90.79561287,335.96318817)
\curveto(90.81560661,336.01318076)(90.8306066,336.06318071)(90.84061287,336.11318817)
\curveto(90.86060657,336.1731806)(90.89060654,336.22318055)(90.93061287,336.26318817)
\curveto(90.99060644,336.30318047)(91.06060637,336.32318045)(91.14061287,336.32318817)
\curveto(91.2306062,336.33318044)(91.32060611,336.33818043)(91.41061287,336.33818817)
\lineto(92.17561287,336.33818817)
\curveto(92.28560514,336.33818043)(92.38060505,336.33318044)(92.46061287,336.32318817)
\curveto(92.55060488,336.32318045)(92.6256048,336.29818047)(92.68561287,336.24818817)
\moveto(90.63061287,331.61318817)
\curveto(90.67060676,331.70318507)(90.70560672,331.81818495)(90.73561287,331.95818817)
\curveto(90.76560666,332.09818467)(90.78560664,332.24318453)(90.79561287,332.39318817)
\curveto(90.80560662,332.55318422)(90.80560662,332.70818406)(90.79561287,332.85818817)
\curveto(90.79560663,333.00818376)(90.78060665,333.14318363)(90.75061287,333.26318817)
\curveto(90.7306067,333.30318347)(90.72060671,333.33318344)(90.72061287,333.35318817)
\curveto(90.7306067,333.38318339)(90.7306067,333.41818335)(90.72061287,333.45818817)
\lineto(90.66061287,333.66818817)
\curveto(90.64060679,333.73818303)(90.61560681,333.80318297)(90.58561287,333.86318817)
\curveto(90.44560698,334.21318256)(90.24560718,334.48318229)(89.98561287,334.67318817)
\curveto(89.7256077,334.86318191)(89.34560808,334.95818181)(88.84561287,334.95818817)
\curveto(88.8256086,334.93818183)(88.79560863,334.92818184)(88.75561287,334.92818817)
\curveto(88.7256087,334.93818183)(88.69560873,334.93818183)(88.66561287,334.92818817)
\curveto(88.59560883,334.90818186)(88.5306089,334.88818188)(88.47061287,334.86818817)
\curveto(88.41060902,334.85818191)(88.35060908,334.84318193)(88.29061287,334.82318817)
\curveto(88.0306094,334.71318206)(87.8306096,334.54818222)(87.69061287,334.32818817)
\curveto(87.55060988,334.10818266)(87.43560999,333.86318291)(87.34561287,333.59318817)
\curveto(87.3256101,333.54318323)(87.31561011,333.50318327)(87.31561287,333.47318817)
\curveto(87.31561011,333.44318333)(87.31061012,333.40318337)(87.30061287,333.35318817)
\curveto(87.27061016,333.24318353)(87.25061018,333.08318369)(87.24061287,332.87318817)
\curveto(87.2306102,332.66318411)(87.24061019,332.49318428)(87.27061287,332.36318817)
\lineto(87.27061287,332.21318817)
\curveto(87.29061014,332.13318464)(87.30561012,332.05318472)(87.31561287,331.97318817)
\curveto(87.3256101,331.90318487)(87.34061009,331.82818494)(87.36061287,331.74818817)
\curveto(87.45060998,331.48818528)(87.56060987,331.25818551)(87.69061287,331.05818817)
\curveto(87.82060961,330.8681859)(88.00060943,330.71318606)(88.23061287,330.59318817)
\curveto(88.3306091,330.54318623)(88.47060896,330.49318628)(88.65061287,330.44318817)
\curveto(88.72060871,330.44318633)(88.77560865,330.43818633)(88.81561287,330.42818817)
\curveto(88.83560859,330.42818634)(88.86560856,330.42318635)(88.90561287,330.41318817)
\curveto(88.94560848,330.41318636)(88.97560845,330.41818635)(88.99561287,330.42818817)
\lineto(89.14561287,330.42818817)
\curveto(89.23560819,330.44818632)(89.32060811,330.46318631)(89.40061287,330.47318817)
\curveto(89.48060795,330.48318629)(89.56060787,330.50818626)(89.64061287,330.54818817)
\curveto(89.89060754,330.64818612)(90.09060734,330.78818598)(90.24061287,330.96818817)
\curveto(90.40060703,331.14818562)(90.5306069,331.36318541)(90.63061287,331.61318817)
}
}
{
\newrgbcolor{curcolor}{0 0 0}
\pscustom[linestyle=none,fillstyle=solid,fillcolor=curcolor]
{
\newpath
\moveto(98.93053475,336.53318817)
\curveto(99.04052943,336.53318024)(99.13552934,336.52318025)(99.21553475,336.50318817)
\curveto(99.30552917,336.48318029)(99.3755291,336.43818033)(99.42553475,336.36818817)
\curveto(99.48552899,336.28818048)(99.51552896,336.14818062)(99.51553475,335.94818817)
\lineto(99.51553475,335.43818817)
\lineto(99.51553475,335.06318817)
\curveto(99.52552895,334.92318185)(99.51052896,334.81318196)(99.47053475,334.73318817)
\curveto(99.43052904,334.66318211)(99.3705291,334.61818215)(99.29053475,334.59818817)
\curveto(99.22052925,334.57818219)(99.13552934,334.5681822)(99.03553475,334.56818817)
\curveto(98.94552953,334.5681822)(98.84552963,334.5731822)(98.73553475,334.58318817)
\curveto(98.63552984,334.59318218)(98.54052993,334.58818218)(98.45053475,334.56818817)
\curveto(98.38053009,334.54818222)(98.31053016,334.53318224)(98.24053475,334.52318817)
\curveto(98.1705303,334.52318225)(98.10553037,334.51318226)(98.04553475,334.49318817)
\curveto(97.88553059,334.44318233)(97.72553075,334.3681824)(97.56553475,334.26818817)
\curveto(97.40553107,334.17818259)(97.28053119,334.0731827)(97.19053475,333.95318817)
\curveto(97.14053133,333.8731829)(97.08553139,333.78818298)(97.02553475,333.69818817)
\curveto(96.9755315,333.61818315)(96.92553155,333.53318324)(96.87553475,333.44318817)
\curveto(96.84553163,333.36318341)(96.81553166,333.27818349)(96.78553475,333.18818817)
\lineto(96.72553475,332.94818817)
\curveto(96.70553177,332.87818389)(96.69553178,332.80318397)(96.69553475,332.72318817)
\curveto(96.69553178,332.65318412)(96.68553179,332.58318419)(96.66553475,332.51318817)
\curveto(96.65553182,332.4731843)(96.65053182,332.43318434)(96.65053475,332.39318817)
\curveto(96.66053181,332.36318441)(96.66053181,332.33318444)(96.65053475,332.30318817)
\lineto(96.65053475,332.06318817)
\curveto(96.63053184,331.99318478)(96.62553185,331.91318486)(96.63553475,331.82318817)
\curveto(96.64553183,331.74318503)(96.65053182,331.66318511)(96.65053475,331.58318817)
\lineto(96.65053475,330.62318817)
\lineto(96.65053475,329.34818817)
\curveto(96.65053182,329.21818755)(96.64553183,329.09818767)(96.63553475,328.98818817)
\curveto(96.62553185,328.87818789)(96.59553188,328.78818798)(96.54553475,328.71818817)
\curveto(96.52553195,328.68818808)(96.49053198,328.66318811)(96.44053475,328.64318817)
\curveto(96.40053207,328.63318814)(96.35553212,328.62318815)(96.30553475,328.61318817)
\lineto(96.23053475,328.61318817)
\curveto(96.18053229,328.60318817)(96.12553235,328.59818817)(96.06553475,328.59818817)
\lineto(95.90053475,328.59818817)
\lineto(95.25553475,328.59818817)
\curveto(95.19553328,328.60818816)(95.13053334,328.61318816)(95.06053475,328.61318817)
\lineto(94.86553475,328.61318817)
\curveto(94.81553366,328.63318814)(94.76553371,328.64818812)(94.71553475,328.65818817)
\curveto(94.66553381,328.67818809)(94.63053384,328.71318806)(94.61053475,328.76318817)
\curveto(94.5705339,328.81318796)(94.54553393,328.88318789)(94.53553475,328.97318817)
\lineto(94.53553475,329.27318817)
\lineto(94.53553475,330.29318817)
\lineto(94.53553475,334.52318817)
\lineto(94.53553475,335.63318817)
\lineto(94.53553475,335.91818817)
\curveto(94.53553394,336.01818075)(94.55553392,336.09818067)(94.59553475,336.15818817)
\curveto(94.64553383,336.23818053)(94.72053375,336.28818048)(94.82053475,336.30818817)
\curveto(94.92053355,336.32818044)(95.04053343,336.33818043)(95.18053475,336.33818817)
\lineto(95.94553475,336.33818817)
\curveto(96.06553241,336.33818043)(96.1705323,336.32818044)(96.26053475,336.30818817)
\curveto(96.35053212,336.29818047)(96.42053205,336.25318052)(96.47053475,336.17318817)
\curveto(96.50053197,336.12318065)(96.51553196,336.05318072)(96.51553475,335.96318817)
\lineto(96.54553475,335.69318817)
\curveto(96.55553192,335.61318116)(96.5705319,335.53818123)(96.59053475,335.46818817)
\curveto(96.62053185,335.39818137)(96.6705318,335.36318141)(96.74053475,335.36318817)
\curveto(96.76053171,335.38318139)(96.78053169,335.39318138)(96.80053475,335.39318817)
\curveto(96.82053165,335.39318138)(96.84053163,335.40318137)(96.86053475,335.42318817)
\curveto(96.92053155,335.4731813)(96.9705315,335.52818124)(97.01053475,335.58818817)
\curveto(97.06053141,335.65818111)(97.12053135,335.71818105)(97.19053475,335.76818817)
\curveto(97.23053124,335.79818097)(97.26553121,335.82818094)(97.29553475,335.85818817)
\curveto(97.32553115,335.89818087)(97.36053111,335.93318084)(97.40053475,335.96318817)
\lineto(97.67053475,336.14318817)
\curveto(97.7705307,336.20318057)(97.8705306,336.25818051)(97.97053475,336.30818817)
\curveto(98.0705304,336.34818042)(98.1705303,336.38318039)(98.27053475,336.41318817)
\lineto(98.60053475,336.50318817)
\curveto(98.63052984,336.51318026)(98.68552979,336.51318026)(98.76553475,336.50318817)
\curveto(98.85552962,336.50318027)(98.91052956,336.51318026)(98.93053475,336.53318817)
}
}
{
\newrgbcolor{curcolor}{0 0 0}
\pscustom[linestyle=none,fillstyle=solid,fillcolor=curcolor]
{
\newpath
\moveto(107.38561287,329.19818817)
\curveto(107.40560502,329.08818768)(107.41560501,328.97818779)(107.41561287,328.86818817)
\curveto(107.425605,328.75818801)(107.37560505,328.68318809)(107.26561287,328.64318817)
\curveto(107.20560522,328.61318816)(107.13560529,328.59818817)(107.05561287,328.59818817)
\lineto(106.81561287,328.59818817)
\lineto(106.00561287,328.59818817)
\lineto(105.73561287,328.59818817)
\curveto(105.65560677,328.60818816)(105.59060684,328.63318814)(105.54061287,328.67318817)
\curveto(105.47060696,328.71318806)(105.41560701,328.768188)(105.37561287,328.83818817)
\curveto(105.34560708,328.91818785)(105.30060713,328.98318779)(105.24061287,329.03318817)
\curveto(105.22060721,329.05318772)(105.19560723,329.0681877)(105.16561287,329.07818817)
\curveto(105.13560729,329.09818767)(105.09560733,329.10318767)(105.04561287,329.09318817)
\curveto(104.99560743,329.0731877)(104.94560748,329.04818772)(104.89561287,329.01818817)
\curveto(104.85560757,328.98818778)(104.81060762,328.96318781)(104.76061287,328.94318817)
\curveto(104.71060772,328.90318787)(104.65560777,328.8681879)(104.59561287,328.83818817)
\lineto(104.41561287,328.74818817)
\curveto(104.28560814,328.68818808)(104.15060828,328.63818813)(104.01061287,328.59818817)
\curveto(103.87060856,328.5681882)(103.7256087,328.53318824)(103.57561287,328.49318817)
\curveto(103.50560892,328.4731883)(103.43560899,328.46318831)(103.36561287,328.46318817)
\curveto(103.30560912,328.45318832)(103.24060919,328.44318833)(103.17061287,328.43318817)
\lineto(103.08061287,328.43318817)
\curveto(103.05060938,328.42318835)(103.02060941,328.41818835)(102.99061287,328.41818817)
\lineto(102.82561287,328.41818817)
\curveto(102.7256097,328.39818837)(102.6256098,328.39818837)(102.52561287,328.41818817)
\lineto(102.39061287,328.41818817)
\curveto(102.32061011,328.43818833)(102.25061018,328.44818832)(102.18061287,328.44818817)
\curveto(102.12061031,328.43818833)(102.06061037,328.44318833)(102.00061287,328.46318817)
\curveto(101.90061053,328.48318829)(101.80561062,328.50318827)(101.71561287,328.52318817)
\curveto(101.6256108,328.53318824)(101.54061089,328.55818821)(101.46061287,328.59818817)
\curveto(101.17061126,328.70818806)(100.92061151,328.84818792)(100.71061287,329.01818817)
\curveto(100.51061192,329.19818757)(100.35061208,329.43318734)(100.23061287,329.72318817)
\curveto(100.20061223,329.79318698)(100.17061226,329.8681869)(100.14061287,329.94818817)
\curveto(100.12061231,330.02818674)(100.10061233,330.11318666)(100.08061287,330.20318817)
\curveto(100.06061237,330.25318652)(100.05061238,330.30318647)(100.05061287,330.35318817)
\curveto(100.06061237,330.40318637)(100.06061237,330.45318632)(100.05061287,330.50318817)
\curveto(100.04061239,330.53318624)(100.0306124,330.59318618)(100.02061287,330.68318817)
\curveto(100.02061241,330.78318599)(100.0256124,330.85318592)(100.03561287,330.89318817)
\curveto(100.05561237,330.99318578)(100.06561236,331.07818569)(100.06561287,331.14818817)
\lineto(100.15561287,331.47818817)
\curveto(100.18561224,331.59818517)(100.2256122,331.70318507)(100.27561287,331.79318817)
\curveto(100.44561198,332.08318469)(100.64061179,332.30318447)(100.86061287,332.45318817)
\curveto(101.08061135,332.60318417)(101.36061107,332.73318404)(101.70061287,332.84318817)
\curveto(101.8306106,332.89318388)(101.96561046,332.92818384)(102.10561287,332.94818817)
\curveto(102.24561018,332.9681838)(102.38561004,332.99318378)(102.52561287,333.02318817)
\curveto(102.60560982,333.04318373)(102.69060974,333.05318372)(102.78061287,333.05318817)
\curveto(102.87060956,333.06318371)(102.96060947,333.07818369)(103.05061287,333.09818817)
\curveto(103.12060931,333.11818365)(103.19060924,333.12318365)(103.26061287,333.11318817)
\curveto(103.3306091,333.11318366)(103.40560902,333.12318365)(103.48561287,333.14318817)
\curveto(103.55560887,333.16318361)(103.6256088,333.1731836)(103.69561287,333.17318817)
\curveto(103.76560866,333.1731836)(103.84060859,333.18318359)(103.92061287,333.20318817)
\curveto(104.1306083,333.25318352)(104.32060811,333.29318348)(104.49061287,333.32318817)
\curveto(104.67060776,333.36318341)(104.8306076,333.45318332)(104.97061287,333.59318817)
\curveto(105.06060737,333.68318309)(105.12060731,333.78318299)(105.15061287,333.89318817)
\curveto(105.16060727,333.92318285)(105.16060727,333.94818282)(105.15061287,333.96818817)
\curveto(105.15060728,333.98818278)(105.15560727,334.00818276)(105.16561287,334.02818817)
\curveto(105.17560725,334.04818272)(105.18060725,334.07818269)(105.18061287,334.11818817)
\lineto(105.18061287,334.20818817)
\lineto(105.15061287,334.32818817)
\curveto(105.15060728,334.3681824)(105.14560728,334.40318237)(105.13561287,334.43318817)
\curveto(105.03560739,334.73318204)(104.8256076,334.93818183)(104.50561287,335.04818817)
\curveto(104.41560801,335.07818169)(104.30560812,335.09818167)(104.17561287,335.10818817)
\curveto(104.05560837,335.12818164)(103.9306085,335.13318164)(103.80061287,335.12318817)
\curveto(103.67060876,335.12318165)(103.54560888,335.11318166)(103.42561287,335.09318817)
\curveto(103.30560912,335.0731817)(103.20060923,335.04818172)(103.11061287,335.01818817)
\curveto(103.05060938,334.99818177)(102.99060944,334.9681818)(102.93061287,334.92818817)
\curveto(102.88060955,334.89818187)(102.8306096,334.86318191)(102.78061287,334.82318817)
\curveto(102.7306097,334.78318199)(102.67560975,334.72818204)(102.61561287,334.65818817)
\curveto(102.56560986,334.58818218)(102.5306099,334.52318225)(102.51061287,334.46318817)
\curveto(102.46060997,334.36318241)(102.41561001,334.2681825)(102.37561287,334.17818817)
\curveto(102.34561008,334.08818268)(102.27561015,334.02818274)(102.16561287,333.99818817)
\curveto(102.08561034,333.97818279)(102.00061043,333.9681828)(101.91061287,333.96818817)
\lineto(101.64061287,333.96818817)
\lineto(101.07061287,333.96818817)
\curveto(101.02061141,333.9681828)(100.97061146,333.96318281)(100.92061287,333.95318817)
\curveto(100.87061156,333.95318282)(100.8256116,333.95818281)(100.78561287,333.96818817)
\lineto(100.65061287,333.96818817)
\curveto(100.6306118,333.97818279)(100.60561182,333.98318279)(100.57561287,333.98318817)
\curveto(100.54561188,333.98318279)(100.52061191,333.99318278)(100.50061287,334.01318817)
\curveto(100.42061201,334.03318274)(100.36561206,334.09818267)(100.33561287,334.20818817)
\curveto(100.3256121,334.25818251)(100.3256121,334.30818246)(100.33561287,334.35818817)
\curveto(100.34561208,334.40818236)(100.35561207,334.45318232)(100.36561287,334.49318817)
\curveto(100.39561203,334.60318217)(100.425612,334.70318207)(100.45561287,334.79318817)
\curveto(100.49561193,334.89318188)(100.54061189,334.98318179)(100.59061287,335.06318817)
\lineto(100.68061287,335.21318817)
\lineto(100.77061287,335.36318817)
\curveto(100.85061158,335.4731813)(100.95061148,335.57818119)(101.07061287,335.67818817)
\curveto(101.09061134,335.68818108)(101.12061131,335.71318106)(101.16061287,335.75318817)
\curveto(101.21061122,335.79318098)(101.25561117,335.82818094)(101.29561287,335.85818817)
\curveto(101.33561109,335.88818088)(101.38061105,335.91818085)(101.43061287,335.94818817)
\curveto(101.60061083,336.05818071)(101.78061065,336.14318063)(101.97061287,336.20318817)
\curveto(102.16061027,336.2731805)(102.35561007,336.33818043)(102.55561287,336.39818817)
\curveto(102.67560975,336.42818034)(102.80060963,336.44818032)(102.93061287,336.45818817)
\curveto(103.06060937,336.4681803)(103.19060924,336.48818028)(103.32061287,336.51818817)
\curveto(103.36060907,336.52818024)(103.42060901,336.52818024)(103.50061287,336.51818817)
\curveto(103.59060884,336.50818026)(103.64560878,336.51318026)(103.66561287,336.53318817)
\curveto(104.07560835,336.54318023)(104.46560796,336.52818024)(104.83561287,336.48818817)
\curveto(105.21560721,336.44818032)(105.55560687,336.3731804)(105.85561287,336.26318817)
\curveto(106.16560626,336.15318062)(106.430606,336.00318077)(106.65061287,335.81318817)
\curveto(106.87060556,335.63318114)(107.04060539,335.39818137)(107.16061287,335.10818817)
\curveto(107.2306052,334.93818183)(107.27060516,334.74318203)(107.28061287,334.52318817)
\curveto(107.29060514,334.30318247)(107.29560513,334.07818269)(107.29561287,333.84818817)
\lineto(107.29561287,330.50318817)
\lineto(107.29561287,329.91818817)
\curveto(107.29560513,329.72818704)(107.31560511,329.55318722)(107.35561287,329.39318817)
\curveto(107.36560506,329.36318741)(107.37060506,329.32818744)(107.37061287,329.28818817)
\curveto(107.37060506,329.25818751)(107.37560505,329.22818754)(107.38561287,329.19818817)
\moveto(105.18061287,331.50818817)
\curveto(105.19060724,331.55818521)(105.19560723,331.61318516)(105.19561287,331.67318817)
\curveto(105.19560723,331.74318503)(105.19060724,331.80318497)(105.18061287,331.85318817)
\curveto(105.16060727,331.91318486)(105.15060728,331.9681848)(105.15061287,332.01818817)
\curveto(105.15060728,332.0681847)(105.1306073,332.10818466)(105.09061287,332.13818817)
\curveto(105.04060739,332.17818459)(104.96560746,332.19818457)(104.86561287,332.19818817)
\curveto(104.8256076,332.18818458)(104.79060764,332.17818459)(104.76061287,332.16818817)
\curveto(104.7306077,332.1681846)(104.69560773,332.16318461)(104.65561287,332.15318817)
\curveto(104.58560784,332.13318464)(104.51060792,332.11818465)(104.43061287,332.10818817)
\curveto(104.35060808,332.09818467)(104.27060816,332.08318469)(104.19061287,332.06318817)
\curveto(104.16060827,332.05318472)(104.11560831,332.04818472)(104.05561287,332.04818817)
\curveto(103.9256085,332.01818475)(103.79560863,331.99818477)(103.66561287,331.98818817)
\curveto(103.53560889,331.97818479)(103.41060902,331.95318482)(103.29061287,331.91318817)
\curveto(103.21060922,331.89318488)(103.13560929,331.8731849)(103.06561287,331.85318817)
\curveto(102.99560943,331.84318493)(102.9256095,331.82318495)(102.85561287,331.79318817)
\curveto(102.64560978,331.70318507)(102.46560996,331.5681852)(102.31561287,331.38818817)
\curveto(102.17561025,331.20818556)(102.1256103,330.95818581)(102.16561287,330.63818817)
\curveto(102.18561024,330.4681863)(102.24061019,330.32818644)(102.33061287,330.21818817)
\curveto(102.40061003,330.10818666)(102.50560992,330.01818675)(102.64561287,329.94818817)
\curveto(102.78560964,329.88818688)(102.93560949,329.84318693)(103.09561287,329.81318817)
\curveto(103.26560916,329.78318699)(103.44060899,329.773187)(103.62061287,329.78318817)
\curveto(103.81060862,329.80318697)(103.98560844,329.83818693)(104.14561287,329.88818817)
\curveto(104.40560802,329.9681868)(104.61060782,330.09318668)(104.76061287,330.26318817)
\curveto(104.91060752,330.44318633)(105.0256074,330.66318611)(105.10561287,330.92318817)
\curveto(105.1256073,330.99318578)(105.13560729,331.06318571)(105.13561287,331.13318817)
\curveto(105.14560728,331.21318556)(105.16060727,331.29318548)(105.18061287,331.37318817)
\lineto(105.18061287,331.50818817)
}
}
{
\newrgbcolor{curcolor}{0 0 0}
\pscustom[linestyle=none,fillstyle=solid,fillcolor=curcolor]
{
\newpath
\moveto(111.91889412,339.29318817)
\curveto(112.00889028,339.29317748)(112.10889018,339.29317748)(112.21889412,339.29318817)
\curveto(112.33888995,339.29317748)(112.45388984,339.28817748)(112.56389412,339.27818817)
\curveto(112.68388961,339.2681775)(112.7888895,339.24817752)(112.87889412,339.21818817)
\curveto(112.96888932,339.19817757)(113.02888926,339.16317761)(113.05889412,339.11318817)
\curveto(113.11888917,339.03317774)(113.14888914,338.91817785)(113.14889412,338.76818817)
\lineto(113.14889412,338.36318817)
\curveto(113.14888914,338.26317851)(113.14388915,338.16317861)(113.13389412,338.06318817)
\curveto(113.13388916,337.96317881)(113.11388918,337.88817888)(113.07389412,337.83818817)
\curveto(113.03388926,337.77817899)(112.98388931,337.73817903)(112.92389412,337.71818817)
\curveto(112.86388943,337.70817906)(112.7938895,337.70317907)(112.71389412,337.70318817)
\lineto(112.48889412,337.70318817)
\curveto(112.41888987,337.71317906)(112.34888994,337.71317906)(112.27889412,337.70318817)
\curveto(112.09889019,337.66317911)(111.95889033,337.61317916)(111.85889412,337.55318817)
\curveto(111.75889053,337.50317927)(111.67889061,337.39317938)(111.61889412,337.22318817)
\curveto(111.59889069,337.19317958)(111.5888907,337.16317961)(111.58889412,337.13318817)
\curveto(111.59889069,337.11317966)(111.59889069,337.08817968)(111.58889412,337.05818817)
\curveto(111.57889071,337.01817975)(111.56889072,336.95817981)(111.55889412,336.87818817)
\curveto(111.54889074,336.79817997)(111.54889074,336.73318004)(111.55889412,336.68318817)
\curveto(111.57889071,336.61318016)(111.60389069,336.55318022)(111.63389412,336.50318817)
\curveto(111.66389063,336.45318032)(111.70889058,336.41318036)(111.76889412,336.38318817)
\curveto(111.86889042,336.33318044)(111.9888903,336.31818045)(112.12889412,336.33818817)
\curveto(112.26889002,336.35818041)(112.39888989,336.35818041)(112.51889412,336.33818817)
\curveto(112.56888972,336.32818044)(112.60888968,336.32318045)(112.63889412,336.32318817)
\curveto(112.67888961,336.33318044)(112.71888957,336.33318044)(112.75889412,336.32318817)
\curveto(112.84888944,336.28318049)(112.91388938,336.23818053)(112.95389412,336.18818817)
\curveto(112.97388932,336.15818061)(112.9888893,336.10818066)(112.99889412,336.03818817)
\curveto(113.00888928,335.97818079)(113.01888927,335.90818086)(113.02889412,335.82818817)
\curveto(113.03888925,335.75818101)(113.03888925,335.68318109)(113.02889412,335.60318817)
\curveto(113.02888926,335.53318124)(113.02388927,335.47818129)(113.01389412,335.43818817)
\curveto(113.00388929,335.39818137)(113.00388929,335.35818141)(113.01389412,335.31818817)
\curveto(113.02388927,335.28818148)(113.01888927,335.25318152)(112.99889412,335.21318817)
\curveto(112.97888931,335.09318168)(112.91888937,335.01818175)(112.81889412,334.98818817)
\curveto(112.73888955,334.94818182)(112.64388965,334.92818184)(112.53389412,334.92818817)
\curveto(112.42388987,334.93818183)(112.31388998,334.94318183)(112.20389412,334.94318817)
\lineto(112.09889412,334.94318817)
\curveto(112.05889023,334.94318183)(112.02389027,334.93818183)(111.99389412,334.92818817)
\lineto(111.87389412,334.92818817)
\curveto(111.70389059,334.88818188)(111.59889069,334.77818199)(111.55889412,334.59818817)
\curveto(111.53889075,334.53818223)(111.53389076,334.4681823)(111.54389412,334.38818817)
\curveto(111.55389074,334.30818246)(111.55889073,334.22818254)(111.55889412,334.14818817)
\lineto(111.55889412,333.23318817)
\lineto(111.55889412,330.30818817)
\lineto(111.55889412,329.60318817)
\lineto(111.55889412,329.40818817)
\curveto(111.56889072,329.34818742)(111.56389073,329.29318748)(111.54389412,329.24318817)
\lineto(111.54389412,329.07818817)
\curveto(111.54389075,328.91818785)(111.51889077,328.80318797)(111.46889412,328.73318817)
\curveto(111.44889084,328.70318807)(111.41389088,328.67818809)(111.36389412,328.65818817)
\curveto(111.31389098,328.64818812)(111.26389103,328.63318814)(111.21389412,328.61318817)
\lineto(111.13889412,328.61318817)
\curveto(111.0888912,328.60318817)(111.03389126,328.59818817)(110.97389412,328.59818817)
\curveto(110.91389138,328.60818816)(110.85889143,328.61318816)(110.80889412,328.61318817)
\lineto(110.14889412,328.61318817)
\curveto(110.07889221,328.61318816)(110.00389229,328.60818816)(109.92389412,328.59818817)
\curveto(109.85389244,328.59818817)(109.7938925,328.60818816)(109.74389412,328.62818817)
\curveto(109.62389267,328.65818811)(109.54389275,328.70818806)(109.50389412,328.77818817)
\curveto(109.47389282,328.82818794)(109.45389284,328.89318788)(109.44389412,328.97318817)
\lineto(109.44389412,329.21318817)
\lineto(109.44389412,329.99318817)
\lineto(109.44389412,334.19318817)
\curveto(109.44389285,334.36318241)(109.43389286,334.50818226)(109.41389412,334.62818817)
\curveto(109.3938929,334.75818201)(109.32389297,334.84818192)(109.20389412,334.89818817)
\curveto(109.0938932,334.94818182)(108.95889333,334.95818181)(108.79889412,334.92818817)
\curveto(108.63889365,334.90818186)(108.50389379,334.92318185)(108.39389412,334.97318817)
\curveto(108.28389401,335.02318175)(108.21389408,335.10818166)(108.18389412,335.22818817)
\curveto(108.16389413,335.27818149)(108.15889413,335.33818143)(108.16889412,335.40818817)
\lineto(108.16889412,335.61818817)
\curveto(108.16889412,335.79818097)(108.17889411,335.94818082)(108.19889412,336.06818817)
\curveto(108.21889407,336.18818058)(108.30389399,336.2731805)(108.45389412,336.32318817)
\curveto(108.53389376,336.34318043)(108.61889367,336.35318042)(108.70889412,336.35318817)
\lineto(108.96389412,336.35318817)
\curveto(109.05389324,336.35318042)(109.13389316,336.35818041)(109.20389412,336.36818817)
\curveto(109.27389302,336.38818038)(109.32889296,336.42818034)(109.36889412,336.48818817)
\curveto(109.43889285,336.58818018)(109.46389283,336.71318006)(109.44389412,336.86318817)
\curveto(109.43389286,337.02317975)(109.44389285,337.1731796)(109.47389412,337.31318817)
\curveto(109.48389281,337.35317942)(109.4888928,337.39317938)(109.48889412,337.43318817)
\curveto(109.49889279,337.4731793)(109.50889278,337.51817925)(109.51889412,337.56818817)
\curveto(109.55889273,337.70817906)(109.59889269,337.83317894)(109.63889412,337.94318817)
\curveto(109.67889261,338.06317871)(109.73389256,338.1731786)(109.80389412,338.27318817)
\curveto(109.94389235,338.51317826)(110.12889216,338.70317807)(110.35889412,338.84318817)
\curveto(110.5888917,338.99317778)(110.84889144,339.10817766)(111.13889412,339.18818817)
\curveto(111.21889107,339.21817755)(111.30389099,339.23317754)(111.39389412,339.23318817)
\curveto(111.48389081,339.24317753)(111.57389072,339.25817751)(111.66389412,339.27818817)
\curveto(111.6938906,339.28817748)(111.73889055,339.28817748)(111.79889412,339.27818817)
\curveto(111.85889043,339.2681775)(111.89889039,339.2731775)(111.91889412,339.29318817)
}
}
{
\newrgbcolor{curcolor}{0 0 0}
\pscustom[linestyle=none,fillstyle=solid,fillcolor=curcolor]
{
\newpath
\moveto(116.16365975,335.70818817)
\curveto(116.16365676,335.80818096)(116.15865677,335.90318087)(116.14865975,335.99318817)
\curveto(116.14865678,336.08318069)(116.1286568,336.15318062)(116.08865975,336.20318817)
\curveto(116.05865687,336.25318052)(116.01865691,336.28318049)(115.96865975,336.29318817)
\curveto(115.91865701,336.30318047)(115.86365706,336.31818045)(115.80365975,336.33818817)
\lineto(115.68365975,336.33818817)
\curveto(115.63365729,336.34818042)(115.56865736,336.34818042)(115.48865975,336.33818817)
\lineto(115.29365975,336.33818817)
\lineto(114.54365975,336.33818817)
\curveto(114.5236584,336.32818044)(114.49365843,336.32318045)(114.45365975,336.32318817)
\curveto(114.4236585,336.33318044)(114.39865853,336.33318044)(114.37865975,336.32318817)
\curveto(114.26865866,336.30318047)(114.18365874,336.25818051)(114.12365975,336.18818817)
\curveto(114.08365884,336.12818064)(114.06365886,336.04818072)(114.06365975,335.94818817)
\lineto(114.06365975,335.64818817)
\lineto(114.06365975,329.31818817)
\lineto(114.06365975,328.97318817)
\curveto(114.07365885,328.8731879)(114.10865882,328.78818798)(114.16865975,328.71818817)
\curveto(114.20865872,328.6681881)(114.26865866,328.63818813)(114.34865975,328.62818817)
\curveto(114.43865849,328.61818815)(114.5286584,328.61318816)(114.61865975,328.61318817)
\lineto(115.45865975,328.61318817)
\curveto(115.53865739,328.61318816)(115.61365731,328.60818816)(115.68365975,328.59818817)
\curveto(115.75365717,328.59818817)(115.81865711,328.60818816)(115.87865975,328.62818817)
\curveto(116.04865688,328.67818809)(116.13865679,328.773188)(116.14865975,328.91318817)
\curveto(116.15865677,329.05318772)(116.16365676,329.22318755)(116.16365975,329.42318817)
\lineto(116.16365975,335.70818817)
\moveto(115.74365975,339.41318817)
\lineto(116.80865975,339.41318817)
\curveto(116.88865604,339.41317736)(116.98365594,339.41317736)(117.09365975,339.41318817)
\curveto(117.20365572,339.41317736)(117.28365564,339.39817737)(117.33365975,339.36818817)
\curveto(117.35365557,339.35817741)(117.36365556,339.34317743)(117.36365975,339.32318817)
\curveto(117.37365555,339.31317746)(117.38865554,339.30317747)(117.40865975,339.29318817)
\curveto(117.41865551,339.1731776)(117.36865556,339.0681777)(117.25865975,338.97818817)
\curveto(117.15865577,338.88817788)(117.07365585,338.80817796)(117.00365975,338.73818817)
\curveto(116.923656,338.6681781)(116.84365608,338.59317818)(116.76365975,338.51318817)
\curveto(116.69365623,338.44317833)(116.61865631,338.37817839)(116.53865975,338.31818817)
\curveto(116.49865643,338.28817848)(116.46365646,338.25317852)(116.43365975,338.21318817)
\curveto(116.41365651,338.18317859)(116.38365654,338.15817861)(116.34365975,338.13818817)
\curveto(116.3236566,338.10817866)(116.29865663,338.08317869)(116.26865975,338.06318817)
\lineto(116.11865975,337.91318817)
\lineto(115.96865975,337.79318817)
\lineto(115.92365975,337.74818817)
\curveto(115.923657,337.73817903)(115.91365701,337.72317905)(115.89365975,337.70318817)
\curveto(115.81365711,337.64317913)(115.73365719,337.57817919)(115.65365975,337.50818817)
\curveto(115.58365734,337.43817933)(115.49365743,337.38317939)(115.38365975,337.34318817)
\curveto(115.34365758,337.33317944)(115.30365762,337.32817944)(115.26365975,337.32818817)
\curveto(115.23365769,337.32817944)(115.19365773,337.32317945)(115.14365975,337.31318817)
\curveto(115.11365781,337.30317947)(115.07365785,337.29817947)(115.02365975,337.29818817)
\curveto(114.97365795,337.30817946)(114.928658,337.31317946)(114.88865975,337.31318817)
\lineto(114.54365975,337.31318817)
\curveto(114.4236585,337.31317946)(114.33365859,337.33817943)(114.27365975,337.38818817)
\curveto(114.21365871,337.42817934)(114.19865873,337.49817927)(114.22865975,337.59818817)
\curveto(114.24865868,337.67817909)(114.28365864,337.74817902)(114.33365975,337.80818817)
\curveto(114.38365854,337.87817889)(114.4286585,337.94817882)(114.46865975,338.01818817)
\curveto(114.56865836,338.15817861)(114.66365826,338.29317848)(114.75365975,338.42318817)
\curveto(114.84365808,338.55317822)(114.93365799,338.68817808)(115.02365975,338.82818817)
\curveto(115.07365785,338.90817786)(115.1236578,338.99317778)(115.17365975,339.08318817)
\curveto(115.23365769,339.1731776)(115.29865763,339.24317753)(115.36865975,339.29318817)
\curveto(115.40865752,339.32317745)(115.47865745,339.35817741)(115.57865975,339.39818817)
\curveto(115.59865733,339.40817736)(115.6236573,339.40817736)(115.65365975,339.39818817)
\curveto(115.69365723,339.39817737)(115.7236572,339.40317737)(115.74365975,339.41318817)
}
}
{
\newrgbcolor{curcolor}{0 0 0}
\pscustom[linestyle=none,fillstyle=solid,fillcolor=curcolor]
{
\newpath
\moveto(124.87584725,329.19818817)
\curveto(124.8958394,329.08818768)(124.90583939,328.97818779)(124.90584725,328.86818817)
\curveto(124.91583938,328.75818801)(124.86583943,328.68318809)(124.75584725,328.64318817)
\curveto(124.6958396,328.61318816)(124.62583967,328.59818817)(124.54584725,328.59818817)
\lineto(124.30584725,328.59818817)
\lineto(123.49584725,328.59818817)
\lineto(123.22584725,328.59818817)
\curveto(123.14584115,328.60818816)(123.08084121,328.63318814)(123.03084725,328.67318817)
\curveto(122.96084133,328.71318806)(122.90584139,328.768188)(122.86584725,328.83818817)
\curveto(122.83584146,328.91818785)(122.7908415,328.98318779)(122.73084725,329.03318817)
\curveto(122.71084158,329.05318772)(122.68584161,329.0681877)(122.65584725,329.07818817)
\curveto(122.62584167,329.09818767)(122.58584171,329.10318767)(122.53584725,329.09318817)
\curveto(122.48584181,329.0731877)(122.43584186,329.04818772)(122.38584725,329.01818817)
\curveto(122.34584195,328.98818778)(122.30084199,328.96318781)(122.25084725,328.94318817)
\curveto(122.20084209,328.90318787)(122.14584215,328.8681879)(122.08584725,328.83818817)
\lineto(121.90584725,328.74818817)
\curveto(121.77584252,328.68818808)(121.64084265,328.63818813)(121.50084725,328.59818817)
\curveto(121.36084293,328.5681882)(121.21584308,328.53318824)(121.06584725,328.49318817)
\curveto(120.9958433,328.4731883)(120.92584337,328.46318831)(120.85584725,328.46318817)
\curveto(120.7958435,328.45318832)(120.73084356,328.44318833)(120.66084725,328.43318817)
\lineto(120.57084725,328.43318817)
\curveto(120.54084375,328.42318835)(120.51084378,328.41818835)(120.48084725,328.41818817)
\lineto(120.31584725,328.41818817)
\curveto(120.21584408,328.39818837)(120.11584418,328.39818837)(120.01584725,328.41818817)
\lineto(119.88084725,328.41818817)
\curveto(119.81084448,328.43818833)(119.74084455,328.44818832)(119.67084725,328.44818817)
\curveto(119.61084468,328.43818833)(119.55084474,328.44318833)(119.49084725,328.46318817)
\curveto(119.3908449,328.48318829)(119.295845,328.50318827)(119.20584725,328.52318817)
\curveto(119.11584518,328.53318824)(119.03084526,328.55818821)(118.95084725,328.59818817)
\curveto(118.66084563,328.70818806)(118.41084588,328.84818792)(118.20084725,329.01818817)
\curveto(118.00084629,329.19818757)(117.84084645,329.43318734)(117.72084725,329.72318817)
\curveto(117.6908466,329.79318698)(117.66084663,329.8681869)(117.63084725,329.94818817)
\curveto(117.61084668,330.02818674)(117.5908467,330.11318666)(117.57084725,330.20318817)
\curveto(117.55084674,330.25318652)(117.54084675,330.30318647)(117.54084725,330.35318817)
\curveto(117.55084674,330.40318637)(117.55084674,330.45318632)(117.54084725,330.50318817)
\curveto(117.53084676,330.53318624)(117.52084677,330.59318618)(117.51084725,330.68318817)
\curveto(117.51084678,330.78318599)(117.51584678,330.85318592)(117.52584725,330.89318817)
\curveto(117.54584675,330.99318578)(117.55584674,331.07818569)(117.55584725,331.14818817)
\lineto(117.64584725,331.47818817)
\curveto(117.67584662,331.59818517)(117.71584658,331.70318507)(117.76584725,331.79318817)
\curveto(117.93584636,332.08318469)(118.13084616,332.30318447)(118.35084725,332.45318817)
\curveto(118.57084572,332.60318417)(118.85084544,332.73318404)(119.19084725,332.84318817)
\curveto(119.32084497,332.89318388)(119.45584484,332.92818384)(119.59584725,332.94818817)
\curveto(119.73584456,332.9681838)(119.87584442,332.99318378)(120.01584725,333.02318817)
\curveto(120.0958442,333.04318373)(120.18084411,333.05318372)(120.27084725,333.05318817)
\curveto(120.36084393,333.06318371)(120.45084384,333.07818369)(120.54084725,333.09818817)
\curveto(120.61084368,333.11818365)(120.68084361,333.12318365)(120.75084725,333.11318817)
\curveto(120.82084347,333.11318366)(120.8958434,333.12318365)(120.97584725,333.14318817)
\curveto(121.04584325,333.16318361)(121.11584318,333.1731836)(121.18584725,333.17318817)
\curveto(121.25584304,333.1731836)(121.33084296,333.18318359)(121.41084725,333.20318817)
\curveto(121.62084267,333.25318352)(121.81084248,333.29318348)(121.98084725,333.32318817)
\curveto(122.16084213,333.36318341)(122.32084197,333.45318332)(122.46084725,333.59318817)
\curveto(122.55084174,333.68318309)(122.61084168,333.78318299)(122.64084725,333.89318817)
\curveto(122.65084164,333.92318285)(122.65084164,333.94818282)(122.64084725,333.96818817)
\curveto(122.64084165,333.98818278)(122.64584165,334.00818276)(122.65584725,334.02818817)
\curveto(122.66584163,334.04818272)(122.67084162,334.07818269)(122.67084725,334.11818817)
\lineto(122.67084725,334.20818817)
\lineto(122.64084725,334.32818817)
\curveto(122.64084165,334.3681824)(122.63584166,334.40318237)(122.62584725,334.43318817)
\curveto(122.52584177,334.73318204)(122.31584198,334.93818183)(121.99584725,335.04818817)
\curveto(121.90584239,335.07818169)(121.7958425,335.09818167)(121.66584725,335.10818817)
\curveto(121.54584275,335.12818164)(121.42084287,335.13318164)(121.29084725,335.12318817)
\curveto(121.16084313,335.12318165)(121.03584326,335.11318166)(120.91584725,335.09318817)
\curveto(120.7958435,335.0731817)(120.6908436,335.04818172)(120.60084725,335.01818817)
\curveto(120.54084375,334.99818177)(120.48084381,334.9681818)(120.42084725,334.92818817)
\curveto(120.37084392,334.89818187)(120.32084397,334.86318191)(120.27084725,334.82318817)
\curveto(120.22084407,334.78318199)(120.16584413,334.72818204)(120.10584725,334.65818817)
\curveto(120.05584424,334.58818218)(120.02084427,334.52318225)(120.00084725,334.46318817)
\curveto(119.95084434,334.36318241)(119.90584439,334.2681825)(119.86584725,334.17818817)
\curveto(119.83584446,334.08818268)(119.76584453,334.02818274)(119.65584725,333.99818817)
\curveto(119.57584472,333.97818279)(119.4908448,333.9681828)(119.40084725,333.96818817)
\lineto(119.13084725,333.96818817)
\lineto(118.56084725,333.96818817)
\curveto(118.51084578,333.9681828)(118.46084583,333.96318281)(118.41084725,333.95318817)
\curveto(118.36084593,333.95318282)(118.31584598,333.95818281)(118.27584725,333.96818817)
\lineto(118.14084725,333.96818817)
\curveto(118.12084617,333.97818279)(118.0958462,333.98318279)(118.06584725,333.98318817)
\curveto(118.03584626,333.98318279)(118.01084628,333.99318278)(117.99084725,334.01318817)
\curveto(117.91084638,334.03318274)(117.85584644,334.09818267)(117.82584725,334.20818817)
\curveto(117.81584648,334.25818251)(117.81584648,334.30818246)(117.82584725,334.35818817)
\curveto(117.83584646,334.40818236)(117.84584645,334.45318232)(117.85584725,334.49318817)
\curveto(117.88584641,334.60318217)(117.91584638,334.70318207)(117.94584725,334.79318817)
\curveto(117.98584631,334.89318188)(118.03084626,334.98318179)(118.08084725,335.06318817)
\lineto(118.17084725,335.21318817)
\lineto(118.26084725,335.36318817)
\curveto(118.34084595,335.4731813)(118.44084585,335.57818119)(118.56084725,335.67818817)
\curveto(118.58084571,335.68818108)(118.61084568,335.71318106)(118.65084725,335.75318817)
\curveto(118.70084559,335.79318098)(118.74584555,335.82818094)(118.78584725,335.85818817)
\curveto(118.82584547,335.88818088)(118.87084542,335.91818085)(118.92084725,335.94818817)
\curveto(119.0908452,336.05818071)(119.27084502,336.14318063)(119.46084725,336.20318817)
\curveto(119.65084464,336.2731805)(119.84584445,336.33818043)(120.04584725,336.39818817)
\curveto(120.16584413,336.42818034)(120.290844,336.44818032)(120.42084725,336.45818817)
\curveto(120.55084374,336.4681803)(120.68084361,336.48818028)(120.81084725,336.51818817)
\curveto(120.85084344,336.52818024)(120.91084338,336.52818024)(120.99084725,336.51818817)
\curveto(121.08084321,336.50818026)(121.13584316,336.51318026)(121.15584725,336.53318817)
\curveto(121.56584273,336.54318023)(121.95584234,336.52818024)(122.32584725,336.48818817)
\curveto(122.70584159,336.44818032)(123.04584125,336.3731804)(123.34584725,336.26318817)
\curveto(123.65584064,336.15318062)(123.92084037,336.00318077)(124.14084725,335.81318817)
\curveto(124.36083993,335.63318114)(124.53083976,335.39818137)(124.65084725,335.10818817)
\curveto(124.72083957,334.93818183)(124.76083953,334.74318203)(124.77084725,334.52318817)
\curveto(124.78083951,334.30318247)(124.78583951,334.07818269)(124.78584725,333.84818817)
\lineto(124.78584725,330.50318817)
\lineto(124.78584725,329.91818817)
\curveto(124.78583951,329.72818704)(124.80583949,329.55318722)(124.84584725,329.39318817)
\curveto(124.85583944,329.36318741)(124.86083943,329.32818744)(124.86084725,329.28818817)
\curveto(124.86083943,329.25818751)(124.86583943,329.22818754)(124.87584725,329.19818817)
\moveto(122.67084725,331.50818817)
\curveto(122.68084161,331.55818521)(122.68584161,331.61318516)(122.68584725,331.67318817)
\curveto(122.68584161,331.74318503)(122.68084161,331.80318497)(122.67084725,331.85318817)
\curveto(122.65084164,331.91318486)(122.64084165,331.9681848)(122.64084725,332.01818817)
\curveto(122.64084165,332.0681847)(122.62084167,332.10818466)(122.58084725,332.13818817)
\curveto(122.53084176,332.17818459)(122.45584184,332.19818457)(122.35584725,332.19818817)
\curveto(122.31584198,332.18818458)(122.28084201,332.17818459)(122.25084725,332.16818817)
\curveto(122.22084207,332.1681846)(122.18584211,332.16318461)(122.14584725,332.15318817)
\curveto(122.07584222,332.13318464)(122.00084229,332.11818465)(121.92084725,332.10818817)
\curveto(121.84084245,332.09818467)(121.76084253,332.08318469)(121.68084725,332.06318817)
\curveto(121.65084264,332.05318472)(121.60584269,332.04818472)(121.54584725,332.04818817)
\curveto(121.41584288,332.01818475)(121.28584301,331.99818477)(121.15584725,331.98818817)
\curveto(121.02584327,331.97818479)(120.90084339,331.95318482)(120.78084725,331.91318817)
\curveto(120.70084359,331.89318488)(120.62584367,331.8731849)(120.55584725,331.85318817)
\curveto(120.48584381,331.84318493)(120.41584388,331.82318495)(120.34584725,331.79318817)
\curveto(120.13584416,331.70318507)(119.95584434,331.5681852)(119.80584725,331.38818817)
\curveto(119.66584463,331.20818556)(119.61584468,330.95818581)(119.65584725,330.63818817)
\curveto(119.67584462,330.4681863)(119.73084456,330.32818644)(119.82084725,330.21818817)
\curveto(119.8908444,330.10818666)(119.9958443,330.01818675)(120.13584725,329.94818817)
\curveto(120.27584402,329.88818688)(120.42584387,329.84318693)(120.58584725,329.81318817)
\curveto(120.75584354,329.78318699)(120.93084336,329.773187)(121.11084725,329.78318817)
\curveto(121.30084299,329.80318697)(121.47584282,329.83818693)(121.63584725,329.88818817)
\curveto(121.8958424,329.9681868)(122.10084219,330.09318668)(122.25084725,330.26318817)
\curveto(122.40084189,330.44318633)(122.51584178,330.66318611)(122.59584725,330.92318817)
\curveto(122.61584168,330.99318578)(122.62584167,331.06318571)(122.62584725,331.13318817)
\curveto(122.63584166,331.21318556)(122.65084164,331.29318548)(122.67084725,331.37318817)
\lineto(122.67084725,331.50818817)
}
}
{
\newrgbcolor{curcolor}{0 0 0}
\pscustom[linestyle=none,fillstyle=solid,fillcolor=curcolor]
{
}
}
{
\newrgbcolor{curcolor}{0 0 0}
\pscustom[linestyle=none,fillstyle=solid,fillcolor=curcolor]
{
\newpath
\moveto(138.48928475,332.55818817)
\curveto(138.49927607,332.49818427)(138.50427606,332.40818436)(138.50428475,332.28818817)
\curveto(138.50427606,332.1681846)(138.49427607,332.08318469)(138.47428475,332.03318817)
\lineto(138.47428475,331.83818817)
\curveto(138.44427612,331.72818504)(138.42427614,331.62318515)(138.41428475,331.52318817)
\curveto(138.41427615,331.42318535)(138.39927617,331.32318545)(138.36928475,331.22318817)
\curveto(138.34927622,331.13318564)(138.32927624,331.03818573)(138.30928475,330.93818817)
\curveto(138.28927628,330.84818592)(138.25927631,330.75818601)(138.21928475,330.66818817)
\curveto(138.14927642,330.49818627)(138.07927649,330.33818643)(138.00928475,330.18818817)
\curveto(137.93927663,330.04818672)(137.85927671,329.90818686)(137.76928475,329.76818817)
\curveto(137.70927686,329.67818709)(137.64427692,329.59318718)(137.57428475,329.51318817)
\curveto(137.51427705,329.44318733)(137.44427712,329.3681874)(137.36428475,329.28818817)
\lineto(137.25928475,329.18318817)
\curveto(137.20927736,329.13318764)(137.15427741,329.08818768)(137.09428475,329.04818817)
\lineto(136.94428475,328.92818817)
\curveto(136.8642777,328.8681879)(136.77427779,328.81318796)(136.67428475,328.76318817)
\curveto(136.58427798,328.72318805)(136.48927808,328.67818809)(136.38928475,328.62818817)
\curveto(136.28927828,328.57818819)(136.18427838,328.54318823)(136.07428475,328.52318817)
\curveto(135.97427859,328.50318827)(135.8692787,328.48318829)(135.75928475,328.46318817)
\curveto(135.69927887,328.44318833)(135.63427893,328.43318834)(135.56428475,328.43318817)
\curveto(135.50427906,328.43318834)(135.43927913,328.42318835)(135.36928475,328.40318817)
\lineto(135.23428475,328.40318817)
\curveto(135.15427941,328.38318839)(135.07927949,328.38318839)(135.00928475,328.40318817)
\lineto(134.85928475,328.40318817)
\curveto(134.79927977,328.42318835)(134.73427983,328.43318834)(134.66428475,328.43318817)
\curveto(134.60427996,328.42318835)(134.54428002,328.42818834)(134.48428475,328.44818817)
\curveto(134.32428024,328.49818827)(134.1692804,328.54318823)(134.01928475,328.58318817)
\curveto(133.87928069,328.62318815)(133.74928082,328.68318809)(133.62928475,328.76318817)
\curveto(133.55928101,328.80318797)(133.49428107,328.84318793)(133.43428475,328.88318817)
\curveto(133.37428119,328.93318784)(133.30928126,328.98318779)(133.23928475,329.03318817)
\lineto(133.05928475,329.16818817)
\curveto(132.97928159,329.22818754)(132.90928166,329.23318754)(132.84928475,329.18318817)
\curveto(132.79928177,329.15318762)(132.77428179,329.11318766)(132.77428475,329.06318817)
\curveto(132.77428179,329.02318775)(132.7642818,328.9731878)(132.74428475,328.91318817)
\curveto(132.72428184,328.81318796)(132.71428185,328.69818807)(132.71428475,328.56818817)
\curveto(132.72428184,328.43818833)(132.72928184,328.31818845)(132.72928475,328.20818817)
\lineto(132.72928475,326.67818817)
\curveto(132.72928184,326.54819022)(132.72428184,326.42319035)(132.71428475,326.30318817)
\curveto(132.71428185,326.1731906)(132.68928188,326.0681907)(132.63928475,325.98818817)
\curveto(132.60928196,325.94819082)(132.55428201,325.91819085)(132.47428475,325.89818817)
\curveto(132.39428217,325.87819089)(132.30428226,325.8681909)(132.20428475,325.86818817)
\curveto(132.10428246,325.85819091)(132.00428256,325.85819091)(131.90428475,325.86818817)
\lineto(131.64928475,325.86818817)
\lineto(131.24428475,325.86818817)
\lineto(131.13928475,325.86818817)
\curveto(131.09928347,325.8681909)(131.0642835,325.8731909)(131.03428475,325.88318817)
\lineto(130.91428475,325.88318817)
\curveto(130.74428382,325.93319084)(130.65428391,326.03319074)(130.64428475,326.18318817)
\curveto(130.63428393,326.32319045)(130.62928394,326.49319028)(130.62928475,326.69318817)
\lineto(130.62928475,335.49818817)
\curveto(130.62928394,335.60818116)(130.62428394,335.72318105)(130.61428475,335.84318817)
\curveto(130.61428395,335.9731808)(130.63928393,336.0731807)(130.68928475,336.14318817)
\curveto(130.72928384,336.21318056)(130.78428378,336.25818051)(130.85428475,336.27818817)
\curveto(130.90428366,336.29818047)(130.9642836,336.30818046)(131.03428475,336.30818817)
\lineto(131.25928475,336.30818817)
\lineto(131.97928475,336.30818817)
\lineto(132.26428475,336.30818817)
\curveto(132.35428221,336.30818046)(132.42928214,336.28318049)(132.48928475,336.23318817)
\curveto(132.55928201,336.18318059)(132.59428197,336.11818065)(132.59428475,336.03818817)
\curveto(132.60428196,335.9681808)(132.62928194,335.89318088)(132.66928475,335.81318817)
\curveto(132.67928189,335.78318099)(132.68928188,335.75818101)(132.69928475,335.73818817)
\curveto(132.71928185,335.72818104)(132.73928183,335.71318106)(132.75928475,335.69318817)
\curveto(132.8692817,335.68318109)(132.95928161,335.71318106)(133.02928475,335.78318817)
\curveto(133.09928147,335.85318092)(133.1692814,335.91318086)(133.23928475,335.96318817)
\curveto(133.3692812,336.05318072)(133.50428106,336.13318064)(133.64428475,336.20318817)
\curveto(133.78428078,336.28318049)(133.93928063,336.34818042)(134.10928475,336.39818817)
\curveto(134.18928038,336.42818034)(134.27428029,336.44818032)(134.36428475,336.45818817)
\curveto(134.4642801,336.4681803)(134.55928001,336.48318029)(134.64928475,336.50318817)
\curveto(134.68927988,336.51318026)(134.72927984,336.51318026)(134.76928475,336.50318817)
\curveto(134.81927975,336.49318028)(134.85927971,336.49818027)(134.88928475,336.51818817)
\curveto(135.45927911,336.53818023)(135.93927863,336.45818031)(136.32928475,336.27818817)
\curveto(136.72927784,336.10818066)(137.0692775,335.88318089)(137.34928475,335.60318817)
\curveto(137.39927717,335.55318122)(137.44427712,335.50318127)(137.48428475,335.45318817)
\curveto(137.52427704,335.41318136)(137.564277,335.3681814)(137.60428475,335.31818817)
\curveto(137.67427689,335.22818154)(137.73427683,335.13818163)(137.78428475,335.04818817)
\curveto(137.84427672,334.95818181)(137.89927667,334.8681819)(137.94928475,334.77818817)
\curveto(137.9692766,334.75818201)(137.97927659,334.73318204)(137.97928475,334.70318817)
\curveto(137.98927658,334.6731821)(138.00427656,334.63818213)(138.02428475,334.59818817)
\curveto(138.08427648,334.49818227)(138.13927643,334.37818239)(138.18928475,334.23818817)
\curveto(138.20927636,334.17818259)(138.22927634,334.11318266)(138.24928475,334.04318817)
\curveto(138.2692763,333.98318279)(138.28927628,333.91818285)(138.30928475,333.84818817)
\curveto(138.34927622,333.72818304)(138.37427619,333.60318317)(138.38428475,333.47318817)
\curveto(138.40427616,333.34318343)(138.42927614,333.20818356)(138.45928475,333.06818817)
\lineto(138.45928475,332.90318817)
\lineto(138.48928475,332.72318817)
\lineto(138.48928475,332.55818817)
\moveto(136.37428475,332.21318817)
\curveto(136.38427818,332.26318451)(136.38927818,332.32818444)(136.38928475,332.40818817)
\curveto(136.38927818,332.49818427)(136.38427818,332.5681842)(136.37428475,332.61818817)
\lineto(136.37428475,332.75318817)
\curveto(136.35427821,332.81318396)(136.34427822,332.87818389)(136.34428475,332.94818817)
\curveto(136.34427822,333.01818375)(136.33427823,333.08818368)(136.31428475,333.15818817)
\curveto(136.29427827,333.25818351)(136.27427829,333.35318342)(136.25428475,333.44318817)
\curveto(136.23427833,333.54318323)(136.20427836,333.63318314)(136.16428475,333.71318817)
\curveto(136.04427852,334.03318274)(135.88927868,334.28818248)(135.69928475,334.47818817)
\curveto(135.50927906,334.6681821)(135.23927933,334.80818196)(134.88928475,334.89818817)
\curveto(134.80927976,334.91818185)(134.71927985,334.92818184)(134.61928475,334.92818817)
\lineto(134.34928475,334.92818817)
\curveto(134.30928026,334.91818185)(134.27428029,334.91318186)(134.24428475,334.91318817)
\curveto(134.21428035,334.91318186)(134.17928039,334.90818186)(134.13928475,334.89818817)
\lineto(133.92928475,334.83818817)
\curveto(133.8692807,334.82818194)(133.80928076,334.80818196)(133.74928475,334.77818817)
\curveto(133.48928108,334.6681821)(133.28428128,334.49818227)(133.13428475,334.26818817)
\curveto(132.99428157,334.03818273)(132.87928169,333.78318299)(132.78928475,333.50318817)
\curveto(132.7692818,333.42318335)(132.75428181,333.33818343)(132.74428475,333.24818817)
\curveto(132.73428183,333.1681836)(132.71928185,333.08818368)(132.69928475,333.00818817)
\curveto(132.68928188,332.9681838)(132.68428188,332.90318387)(132.68428475,332.81318817)
\curveto(132.6642819,332.773184)(132.65928191,332.72318405)(132.66928475,332.66318817)
\curveto(132.67928189,332.61318416)(132.67928189,332.56318421)(132.66928475,332.51318817)
\curveto(132.64928192,332.45318432)(132.64928192,332.39818437)(132.66928475,332.34818817)
\lineto(132.66928475,332.16818817)
\lineto(132.66928475,332.03318817)
\curveto(132.6692819,331.99318478)(132.67928189,331.95318482)(132.69928475,331.91318817)
\curveto(132.69928187,331.84318493)(132.70428186,331.78818498)(132.71428475,331.74818817)
\lineto(132.74428475,331.56818817)
\curveto(132.75428181,331.50818526)(132.7692818,331.44818532)(132.78928475,331.38818817)
\curveto(132.87928169,331.09818567)(132.98428158,330.85818591)(133.10428475,330.66818817)
\curveto(133.23428133,330.48818628)(133.41428115,330.32818644)(133.64428475,330.18818817)
\curveto(133.78428078,330.10818666)(133.94928062,330.04318673)(134.13928475,329.99318817)
\curveto(134.17928039,329.98318679)(134.21428035,329.97818679)(134.24428475,329.97818817)
\curveto(134.27428029,329.98818678)(134.30928026,329.98818678)(134.34928475,329.97818817)
\curveto(134.38928018,329.9681868)(134.44928012,329.95818681)(134.52928475,329.94818817)
\curveto(134.60927996,329.94818682)(134.67427989,329.95318682)(134.72428475,329.96318817)
\curveto(134.80427976,329.98318679)(134.88427968,329.99818677)(134.96428475,330.00818817)
\curveto(135.05427951,330.02818674)(135.13927943,330.05318672)(135.21928475,330.08318817)
\curveto(135.45927911,330.18318659)(135.65427891,330.32318645)(135.80428475,330.50318817)
\curveto(135.95427861,330.68318609)(136.07927849,330.89318588)(136.17928475,331.13318817)
\curveto(136.22927834,331.25318552)(136.2642783,331.37818539)(136.28428475,331.50818817)
\curveto(136.30427826,331.63818513)(136.32927824,331.773185)(136.35928475,331.91318817)
\lineto(136.35928475,332.06318817)
\curveto(136.3692782,332.11318466)(136.37427819,332.16318461)(136.37428475,332.21318817)
}
}
{
\newrgbcolor{curcolor}{0 0 0}
\pscustom[linestyle=none,fillstyle=solid,fillcolor=curcolor]
{
\newpath
\moveto(147.13420662,332.54318817)
\curveto(147.15419846,332.46318431)(147.15419846,332.3731844)(147.13420662,332.27318817)
\curveto(147.1141985,332.1731846)(147.07919853,332.10818466)(147.02920662,332.07818817)
\curveto(146.97919863,332.03818473)(146.90419871,332.00818476)(146.80420662,331.98818817)
\curveto(146.7141989,331.97818479)(146.609199,331.9681848)(146.48920662,331.95818817)
\lineto(146.14420662,331.95818817)
\curveto(146.03419958,331.9681848)(145.93419968,331.9731848)(145.84420662,331.97318817)
\lineto(142.18420662,331.97318817)
\lineto(141.97420662,331.97318817)
\curveto(141.9142037,331.9731848)(141.85920375,331.96318481)(141.80920662,331.94318817)
\curveto(141.72920388,331.90318487)(141.67920393,331.86318491)(141.65920662,331.82318817)
\curveto(141.63920397,331.80318497)(141.61920399,331.76318501)(141.59920662,331.70318817)
\curveto(141.57920403,331.65318512)(141.57420404,331.60318517)(141.58420662,331.55318817)
\curveto(141.60420401,331.49318528)(141.614204,331.43318534)(141.61420662,331.37318817)
\curveto(141.62420399,331.32318545)(141.63920397,331.2681855)(141.65920662,331.20818817)
\curveto(141.73920387,330.9681858)(141.83420378,330.768186)(141.94420662,330.60818817)
\curveto(142.06420355,330.45818631)(142.22420339,330.32318645)(142.42420662,330.20318817)
\curveto(142.50420311,330.15318662)(142.58420303,330.11818665)(142.66420662,330.09818817)
\curveto(142.75420286,330.08818668)(142.84420277,330.0681867)(142.93420662,330.03818817)
\curveto(143.0142026,330.01818675)(143.12420249,330.00318677)(143.26420662,329.99318817)
\curveto(143.40420221,329.98318679)(143.52420209,329.98818678)(143.62420662,330.00818817)
\lineto(143.75920662,330.00818817)
\curveto(143.85920175,330.02818674)(143.94920166,330.04818672)(144.02920662,330.06818817)
\curveto(144.11920149,330.09818667)(144.20420141,330.12818664)(144.28420662,330.15818817)
\curveto(144.38420123,330.20818656)(144.49420112,330.2731865)(144.61420662,330.35318817)
\curveto(144.74420087,330.43318634)(144.83920077,330.51318626)(144.89920662,330.59318817)
\curveto(144.94920066,330.66318611)(144.99920061,330.72818604)(145.04920662,330.78818817)
\curveto(145.1092005,330.85818591)(145.17920043,330.90818586)(145.25920662,330.93818817)
\curveto(145.35920025,330.98818578)(145.48420013,331.00818576)(145.63420662,330.99818817)
\lineto(146.06920662,330.99818817)
\lineto(146.24920662,330.99818817)
\curveto(146.31919929,331.00818576)(146.37919923,331.00318577)(146.42920662,330.98318817)
\lineto(146.57920662,330.98318817)
\curveto(146.67919893,330.96318581)(146.74919886,330.93818583)(146.78920662,330.90818817)
\curveto(146.82919878,330.88818588)(146.84919876,330.84318593)(146.84920662,330.77318817)
\curveto(146.85919875,330.70318607)(146.85419876,330.64318613)(146.83420662,330.59318817)
\curveto(146.78419883,330.45318632)(146.72919888,330.32818644)(146.66920662,330.21818817)
\curveto(146.609199,330.10818666)(146.53919907,329.99818677)(146.45920662,329.88818817)
\curveto(146.23919937,329.55818721)(145.98919962,329.29318748)(145.70920662,329.09318817)
\curveto(145.42920018,328.89318788)(145.07920053,328.72318805)(144.65920662,328.58318817)
\curveto(144.54920106,328.54318823)(144.43920117,328.51818825)(144.32920662,328.50818817)
\curveto(144.21920139,328.49818827)(144.10420151,328.47818829)(143.98420662,328.44818817)
\curveto(143.94420167,328.43818833)(143.89920171,328.43818833)(143.84920662,328.44818817)
\curveto(143.8092018,328.44818832)(143.76920184,328.44318833)(143.72920662,328.43318817)
\lineto(143.56420662,328.43318817)
\curveto(143.5142021,328.41318836)(143.45420216,328.40818836)(143.38420662,328.41818817)
\curveto(143.32420229,328.41818835)(143.26920234,328.42318835)(143.21920662,328.43318817)
\curveto(143.13920247,328.44318833)(143.06920254,328.44318833)(143.00920662,328.43318817)
\curveto(142.94920266,328.42318835)(142.88420273,328.42818834)(142.81420662,328.44818817)
\curveto(142.76420285,328.4681883)(142.7092029,328.47818829)(142.64920662,328.47818817)
\curveto(142.58920302,328.47818829)(142.53420308,328.48818828)(142.48420662,328.50818817)
\curveto(142.37420324,328.52818824)(142.26420335,328.55318822)(142.15420662,328.58318817)
\curveto(142.04420357,328.60318817)(141.94420367,328.63818813)(141.85420662,328.68818817)
\curveto(141.74420387,328.72818804)(141.63920397,328.76318801)(141.53920662,328.79318817)
\curveto(141.44920416,328.83318794)(141.36420425,328.87818789)(141.28420662,328.92818817)
\curveto(140.96420465,329.12818764)(140.67920493,329.35818741)(140.42920662,329.61818817)
\curveto(140.17920543,329.88818688)(139.97420564,330.19818657)(139.81420662,330.54818817)
\curveto(139.76420585,330.65818611)(139.72420589,330.768186)(139.69420662,330.87818817)
\curveto(139.66420595,330.99818577)(139.62420599,331.11818565)(139.57420662,331.23818817)
\curveto(139.56420605,331.27818549)(139.55920605,331.31318546)(139.55920662,331.34318817)
\curveto(139.55920605,331.38318539)(139.55420606,331.42318535)(139.54420662,331.46318817)
\curveto(139.50420611,331.58318519)(139.47920613,331.71318506)(139.46920662,331.85318817)
\lineto(139.43920662,332.27318817)
\curveto(139.43920617,332.32318445)(139.43420618,332.37818439)(139.42420662,332.43818817)
\curveto(139.42420619,332.49818427)(139.42920618,332.55318422)(139.43920662,332.60318817)
\lineto(139.43920662,332.78318817)
\lineto(139.48420662,333.14318817)
\curveto(139.52420609,333.31318346)(139.55920605,333.47818329)(139.58920662,333.63818817)
\curveto(139.61920599,333.79818297)(139.66420595,333.94818282)(139.72420662,334.08818817)
\curveto(140.15420546,335.12818164)(140.88420473,335.86318091)(141.91420662,336.29318817)
\curveto(142.05420356,336.35318042)(142.19420342,336.39318038)(142.33420662,336.41318817)
\curveto(142.48420313,336.44318033)(142.63920297,336.47818029)(142.79920662,336.51818817)
\curveto(142.87920273,336.52818024)(142.95420266,336.53318024)(143.02420662,336.53318817)
\curveto(143.09420252,336.53318024)(143.16920244,336.53818023)(143.24920662,336.54818817)
\curveto(143.75920185,336.55818021)(144.19420142,336.49818027)(144.55420662,336.36818817)
\curveto(144.92420069,336.24818052)(145.25420036,336.08818068)(145.54420662,335.88818817)
\curveto(145.63419998,335.82818094)(145.72419989,335.75818101)(145.81420662,335.67818817)
\curveto(145.90419971,335.60818116)(145.98419963,335.53318124)(146.05420662,335.45318817)
\curveto(146.08419953,335.40318137)(146.12419949,335.36318141)(146.17420662,335.33318817)
\curveto(146.25419936,335.22318155)(146.32919928,335.10818166)(146.39920662,334.98818817)
\curveto(146.46919914,334.87818189)(146.54419907,334.76318201)(146.62420662,334.64318817)
\curveto(146.67419894,334.55318222)(146.7141989,334.45818231)(146.74420662,334.35818817)
\curveto(146.78419883,334.2681825)(146.82419879,334.1681826)(146.86420662,334.05818817)
\curveto(146.9141987,333.92818284)(146.95419866,333.79318298)(146.98420662,333.65318817)
\curveto(147.0141986,333.51318326)(147.04919856,333.3731834)(147.08920662,333.23318817)
\curveto(147.1091985,333.15318362)(147.1141985,333.06318371)(147.10420662,332.96318817)
\curveto(147.10419851,332.8731839)(147.1141985,332.78818398)(147.13420662,332.70818817)
\lineto(147.13420662,332.54318817)
\moveto(144.88420662,333.42818817)
\curveto(144.95420066,333.52818324)(144.95920065,333.64818312)(144.89920662,333.78818817)
\curveto(144.84920076,333.93818283)(144.8092008,334.04818272)(144.77920662,334.11818817)
\curveto(144.63920097,334.38818238)(144.45420116,334.59318218)(144.22420662,334.73318817)
\curveto(143.99420162,334.88318189)(143.67420194,334.96318181)(143.26420662,334.97318817)
\curveto(143.23420238,334.95318182)(143.19920241,334.94818182)(143.15920662,334.95818817)
\curveto(143.11920249,334.9681818)(143.08420253,334.9681818)(143.05420662,334.95818817)
\curveto(143.00420261,334.93818183)(142.94920266,334.92318185)(142.88920662,334.91318817)
\curveto(142.82920278,334.91318186)(142.77420284,334.90318187)(142.72420662,334.88318817)
\curveto(142.28420333,334.74318203)(141.95920365,334.4681823)(141.74920662,334.05818817)
\curveto(141.72920388,334.01818275)(141.70420391,333.96318281)(141.67420662,333.89318817)
\curveto(141.65420396,333.83318294)(141.63920397,333.768183)(141.62920662,333.69818817)
\curveto(141.61920399,333.63818313)(141.61920399,333.57818319)(141.62920662,333.51818817)
\curveto(141.64920396,333.45818331)(141.68420393,333.40818336)(141.73420662,333.36818817)
\curveto(141.8142038,333.31818345)(141.92420369,333.29318348)(142.06420662,333.29318817)
\lineto(142.46920662,333.29318817)
\lineto(144.13420662,333.29318817)
\lineto(144.56920662,333.29318817)
\curveto(144.72920088,333.30318347)(144.83420078,333.34818342)(144.88420662,333.42818817)
}
}
{
\newrgbcolor{curcolor}{0 0 0}
\pscustom[linestyle=none,fillstyle=solid,fillcolor=curcolor]
{
\newpath
\moveto(152.80748787,336.53318817)
\curveto(152.91748256,336.53318024)(153.01248246,336.52318025)(153.09248787,336.50318817)
\curveto(153.18248229,336.48318029)(153.25248222,336.43818033)(153.30248787,336.36818817)
\curveto(153.36248211,336.28818048)(153.39248208,336.14818062)(153.39248787,335.94818817)
\lineto(153.39248787,335.43818817)
\lineto(153.39248787,335.06318817)
\curveto(153.40248207,334.92318185)(153.38748209,334.81318196)(153.34748787,334.73318817)
\curveto(153.30748217,334.66318211)(153.24748223,334.61818215)(153.16748787,334.59818817)
\curveto(153.09748238,334.57818219)(153.01248246,334.5681822)(152.91248787,334.56818817)
\curveto(152.82248265,334.5681822)(152.72248275,334.5731822)(152.61248787,334.58318817)
\curveto(152.51248296,334.59318218)(152.41748306,334.58818218)(152.32748787,334.56818817)
\curveto(152.25748322,334.54818222)(152.18748329,334.53318224)(152.11748787,334.52318817)
\curveto(152.04748343,334.52318225)(151.98248349,334.51318226)(151.92248787,334.49318817)
\curveto(151.76248371,334.44318233)(151.60248387,334.3681824)(151.44248787,334.26818817)
\curveto(151.28248419,334.17818259)(151.15748432,334.0731827)(151.06748787,333.95318817)
\curveto(151.01748446,333.8731829)(150.96248451,333.78818298)(150.90248787,333.69818817)
\curveto(150.85248462,333.61818315)(150.80248467,333.53318324)(150.75248787,333.44318817)
\curveto(150.72248475,333.36318341)(150.69248478,333.27818349)(150.66248787,333.18818817)
\lineto(150.60248787,332.94818817)
\curveto(150.58248489,332.87818389)(150.5724849,332.80318397)(150.57248787,332.72318817)
\curveto(150.5724849,332.65318412)(150.56248491,332.58318419)(150.54248787,332.51318817)
\curveto(150.53248494,332.4731843)(150.52748495,332.43318434)(150.52748787,332.39318817)
\curveto(150.53748494,332.36318441)(150.53748494,332.33318444)(150.52748787,332.30318817)
\lineto(150.52748787,332.06318817)
\curveto(150.50748497,331.99318478)(150.50248497,331.91318486)(150.51248787,331.82318817)
\curveto(150.52248495,331.74318503)(150.52748495,331.66318511)(150.52748787,331.58318817)
\lineto(150.52748787,330.62318817)
\lineto(150.52748787,329.34818817)
\curveto(150.52748495,329.21818755)(150.52248495,329.09818767)(150.51248787,328.98818817)
\curveto(150.50248497,328.87818789)(150.472485,328.78818798)(150.42248787,328.71818817)
\curveto(150.40248507,328.68818808)(150.36748511,328.66318811)(150.31748787,328.64318817)
\curveto(150.2774852,328.63318814)(150.23248524,328.62318815)(150.18248787,328.61318817)
\lineto(150.10748787,328.61318817)
\curveto(150.05748542,328.60318817)(150.00248547,328.59818817)(149.94248787,328.59818817)
\lineto(149.77748787,328.59818817)
\lineto(149.13248787,328.59818817)
\curveto(149.0724864,328.60818816)(149.00748647,328.61318816)(148.93748787,328.61318817)
\lineto(148.74248787,328.61318817)
\curveto(148.69248678,328.63318814)(148.64248683,328.64818812)(148.59248787,328.65818817)
\curveto(148.54248693,328.67818809)(148.50748697,328.71318806)(148.48748787,328.76318817)
\curveto(148.44748703,328.81318796)(148.42248705,328.88318789)(148.41248787,328.97318817)
\lineto(148.41248787,329.27318817)
\lineto(148.41248787,330.29318817)
\lineto(148.41248787,334.52318817)
\lineto(148.41248787,335.63318817)
\lineto(148.41248787,335.91818817)
\curveto(148.41248706,336.01818075)(148.43248704,336.09818067)(148.47248787,336.15818817)
\curveto(148.52248695,336.23818053)(148.59748688,336.28818048)(148.69748787,336.30818817)
\curveto(148.79748668,336.32818044)(148.91748656,336.33818043)(149.05748787,336.33818817)
\lineto(149.82248787,336.33818817)
\curveto(149.94248553,336.33818043)(150.04748543,336.32818044)(150.13748787,336.30818817)
\curveto(150.22748525,336.29818047)(150.29748518,336.25318052)(150.34748787,336.17318817)
\curveto(150.3774851,336.12318065)(150.39248508,336.05318072)(150.39248787,335.96318817)
\lineto(150.42248787,335.69318817)
\curveto(150.43248504,335.61318116)(150.44748503,335.53818123)(150.46748787,335.46818817)
\curveto(150.49748498,335.39818137)(150.54748493,335.36318141)(150.61748787,335.36318817)
\curveto(150.63748484,335.38318139)(150.65748482,335.39318138)(150.67748787,335.39318817)
\curveto(150.69748478,335.39318138)(150.71748476,335.40318137)(150.73748787,335.42318817)
\curveto(150.79748468,335.4731813)(150.84748463,335.52818124)(150.88748787,335.58818817)
\curveto(150.93748454,335.65818111)(150.99748448,335.71818105)(151.06748787,335.76818817)
\curveto(151.10748437,335.79818097)(151.14248433,335.82818094)(151.17248787,335.85818817)
\curveto(151.20248427,335.89818087)(151.23748424,335.93318084)(151.27748787,335.96318817)
\lineto(151.54748787,336.14318817)
\curveto(151.64748383,336.20318057)(151.74748373,336.25818051)(151.84748787,336.30818817)
\curveto(151.94748353,336.34818042)(152.04748343,336.38318039)(152.14748787,336.41318817)
\lineto(152.47748787,336.50318817)
\curveto(152.50748297,336.51318026)(152.56248291,336.51318026)(152.64248787,336.50318817)
\curveto(152.73248274,336.50318027)(152.78748269,336.51318026)(152.80748787,336.53318817)
}
}
{
\newrgbcolor{curcolor}{0 0 0}
\pscustom[linestyle=none,fillstyle=solid,fillcolor=curcolor]
{
\newpath
\moveto(157.182566,336.54818817)
\curveto(157.9325615,336.5681802)(158.58256085,336.48318029)(159.132566,336.29318817)
\curveto(159.69255974,336.11318066)(160.11755931,335.79818097)(160.407566,335.34818817)
\curveto(160.47755895,335.23818153)(160.53755889,335.12318165)(160.587566,335.00318817)
\curveto(160.64755878,334.89318188)(160.69755873,334.768182)(160.737566,334.62818817)
\curveto(160.75755867,334.5681822)(160.76755866,334.50318227)(160.767566,334.43318817)
\curveto(160.76755866,334.36318241)(160.75755867,334.30318247)(160.737566,334.25318817)
\curveto(160.69755873,334.19318258)(160.64255879,334.15318262)(160.572566,334.13318817)
\curveto(160.52255891,334.11318266)(160.46255897,334.10318267)(160.392566,334.10318817)
\lineto(160.182566,334.10318817)
\lineto(159.522566,334.10318817)
\curveto(159.45255998,334.10318267)(159.38256005,334.09818267)(159.312566,334.08818817)
\curveto(159.24256019,334.08818268)(159.17756025,334.09818267)(159.117566,334.11818817)
\curveto(159.01756041,334.13818263)(158.94256049,334.17818259)(158.892566,334.23818817)
\curveto(158.84256059,334.29818247)(158.79756063,334.35818241)(158.757566,334.41818817)
\lineto(158.637566,334.62818817)
\curveto(158.60756082,334.70818206)(158.55756087,334.773182)(158.487566,334.82318817)
\curveto(158.38756104,334.90318187)(158.28756114,334.96318181)(158.187566,335.00318817)
\curveto(158.09756133,335.04318173)(157.98256145,335.07818169)(157.842566,335.10818817)
\curveto(157.77256166,335.12818164)(157.66756176,335.14318163)(157.527566,335.15318817)
\curveto(157.39756203,335.16318161)(157.29756213,335.15818161)(157.227566,335.13818817)
\lineto(157.122566,335.13818817)
\lineto(156.972566,335.10818817)
\curveto(156.9325625,335.10818166)(156.88756254,335.10318167)(156.837566,335.09318817)
\curveto(156.66756276,335.04318173)(156.5275629,334.9731818)(156.417566,334.88318817)
\curveto(156.31756311,334.80318197)(156.24756318,334.67818209)(156.207566,334.50818817)
\curveto(156.18756324,334.43818233)(156.18756324,334.3731824)(156.207566,334.31318817)
\curveto(156.2275632,334.25318252)(156.24756318,334.20318257)(156.267566,334.16318817)
\curveto(156.33756309,334.04318273)(156.41756301,333.94818282)(156.507566,333.87818817)
\curveto(156.60756282,333.80818296)(156.72256271,333.74818302)(156.852566,333.69818817)
\curveto(157.04256239,333.61818315)(157.24756218,333.54818322)(157.467566,333.48818817)
\lineto(158.157566,333.33818817)
\curveto(158.39756103,333.29818347)(158.6275608,333.24818352)(158.847566,333.18818817)
\curveto(159.07756035,333.13818363)(159.29256014,333.0731837)(159.492566,332.99318817)
\curveto(159.58255985,332.95318382)(159.66755976,332.91818385)(159.747566,332.88818817)
\curveto(159.83755959,332.8681839)(159.92255951,332.83318394)(160.002566,332.78318817)
\curveto(160.19255924,332.66318411)(160.36255907,332.53318424)(160.512566,332.39318817)
\curveto(160.67255876,332.25318452)(160.79755863,332.07818469)(160.887566,331.86818817)
\curveto(160.91755851,331.79818497)(160.94255849,331.72818504)(160.962566,331.65818817)
\curveto(160.98255845,331.58818518)(161.00255843,331.51318526)(161.022566,331.43318817)
\curveto(161.0325584,331.3731854)(161.03755839,331.27818549)(161.037566,331.14818817)
\curveto(161.04755838,331.02818574)(161.04755838,330.93318584)(161.037566,330.86318817)
\lineto(161.037566,330.78818817)
\curveto(161.01755841,330.72818604)(161.00255843,330.6681861)(160.992566,330.60818817)
\curveto(160.99255844,330.55818621)(160.98755844,330.50818626)(160.977566,330.45818817)
\curveto(160.90755852,330.15818661)(160.79755863,329.89318688)(160.647566,329.66318817)
\curveto(160.48755894,329.42318735)(160.29255914,329.22818754)(160.062566,329.07818817)
\curveto(159.8325596,328.92818784)(159.57255986,328.79818797)(159.282566,328.68818817)
\curveto(159.17256026,328.63818813)(159.05256038,328.60318817)(158.922566,328.58318817)
\curveto(158.80256063,328.56318821)(158.68256075,328.53818823)(158.562566,328.50818817)
\curveto(158.47256096,328.48818828)(158.37756105,328.47818829)(158.277566,328.47818817)
\curveto(158.18756124,328.4681883)(158.09756133,328.45318832)(158.007566,328.43318817)
\lineto(157.737566,328.43318817)
\curveto(157.67756175,328.41318836)(157.57256186,328.40318837)(157.422566,328.40318817)
\curveto(157.28256215,328.40318837)(157.18256225,328.41318836)(157.122566,328.43318817)
\curveto(157.09256234,328.43318834)(157.05756237,328.43818833)(157.017566,328.44818817)
\lineto(156.912566,328.44818817)
\curveto(156.79256264,328.4681883)(156.67256276,328.48318829)(156.552566,328.49318817)
\curveto(156.432563,328.50318827)(156.31756311,328.52318825)(156.207566,328.55318817)
\curveto(155.81756361,328.66318811)(155.47256396,328.78818798)(155.172566,328.92818817)
\curveto(154.87256456,329.07818769)(154.61756481,329.29818747)(154.407566,329.58818817)
\curveto(154.26756516,329.77818699)(154.14756528,329.99818677)(154.047566,330.24818817)
\curveto(154.0275654,330.30818646)(154.00756542,330.38818638)(153.987566,330.48818817)
\curveto(153.96756546,330.53818623)(153.95256548,330.60818616)(153.942566,330.69818817)
\curveto(153.9325655,330.78818598)(153.93756549,330.86318591)(153.957566,330.92318817)
\curveto(153.98756544,330.99318578)(154.03756539,331.04318573)(154.107566,331.07318817)
\curveto(154.15756527,331.09318568)(154.21756521,331.10318567)(154.287566,331.10318817)
\lineto(154.512566,331.10318817)
\lineto(155.217566,331.10318817)
\lineto(155.457566,331.10318817)
\curveto(155.53756389,331.10318567)(155.60756382,331.09318568)(155.667566,331.07318817)
\curveto(155.77756365,331.03318574)(155.84756358,330.9681858)(155.877566,330.87818817)
\curveto(155.91756351,330.78818598)(155.96256347,330.69318608)(156.012566,330.59318817)
\curveto(156.0325634,330.54318623)(156.06756336,330.47818629)(156.117566,330.39818817)
\curveto(156.17756325,330.31818645)(156.2275632,330.2681865)(156.267566,330.24818817)
\curveto(156.38756304,330.14818662)(156.50256293,330.0681867)(156.612566,330.00818817)
\curveto(156.72256271,329.95818681)(156.86256257,329.90818686)(157.032566,329.85818817)
\curveto(157.08256235,329.83818693)(157.1325623,329.82818694)(157.182566,329.82818817)
\curveto(157.2325622,329.83818693)(157.28256215,329.83818693)(157.332566,329.82818817)
\curveto(157.41256202,329.80818696)(157.49756193,329.79818697)(157.587566,329.79818817)
\curveto(157.68756174,329.80818696)(157.77256166,329.82318695)(157.842566,329.84318817)
\curveto(157.89256154,329.85318692)(157.93756149,329.85818691)(157.977566,329.85818817)
\curveto(158.0275614,329.85818691)(158.07756135,329.8681869)(158.127566,329.88818817)
\curveto(158.26756116,329.93818683)(158.39256104,329.99818677)(158.502566,330.06818817)
\curveto(158.62256081,330.13818663)(158.71756071,330.22818654)(158.787566,330.33818817)
\curveto(158.83756059,330.41818635)(158.87756055,330.54318623)(158.907566,330.71318817)
\curveto(158.9275605,330.78318599)(158.9275605,330.84818592)(158.907566,330.90818817)
\curveto(158.88756054,330.9681858)(158.86756056,331.01818575)(158.847566,331.05818817)
\curveto(158.77756065,331.19818557)(158.68756074,331.30318547)(158.577566,331.37318817)
\curveto(158.47756095,331.44318533)(158.35756107,331.50818526)(158.217566,331.56818817)
\curveto(158.0275614,331.64818512)(157.8275616,331.71318506)(157.617566,331.76318817)
\curveto(157.40756202,331.81318496)(157.19756223,331.8681849)(156.987566,331.92818817)
\curveto(156.90756252,331.94818482)(156.82256261,331.96318481)(156.732566,331.97318817)
\curveto(156.65256278,331.98318479)(156.57256286,331.99818477)(156.492566,332.01818817)
\curveto(156.17256326,332.10818466)(155.86756356,332.19318458)(155.577566,332.27318817)
\curveto(155.28756414,332.36318441)(155.02256441,332.49318428)(154.782566,332.66318817)
\curveto(154.50256493,332.86318391)(154.29756513,333.13318364)(154.167566,333.47318817)
\curveto(154.14756528,333.54318323)(154.1275653,333.63818313)(154.107566,333.75818817)
\curveto(154.08756534,333.82818294)(154.07256536,333.91318286)(154.062566,334.01318817)
\curveto(154.05256538,334.11318266)(154.05756537,334.20318257)(154.077566,334.28318817)
\curveto(154.09756533,334.33318244)(154.10256533,334.3731824)(154.092566,334.40318817)
\curveto(154.08256535,334.44318233)(154.08756534,334.48818228)(154.107566,334.53818817)
\curveto(154.1275653,334.64818212)(154.14756528,334.74818202)(154.167566,334.83818817)
\curveto(154.19756523,334.93818183)(154.2325652,335.03318174)(154.272566,335.12318817)
\curveto(154.40256503,335.41318136)(154.58256485,335.64818112)(154.812566,335.82818817)
\curveto(155.04256439,336.00818076)(155.30256413,336.15318062)(155.592566,336.26318817)
\curveto(155.70256373,336.31318046)(155.81756361,336.34818042)(155.937566,336.36818817)
\curveto(156.05756337,336.39818037)(156.18256325,336.42818034)(156.312566,336.45818817)
\curveto(156.37256306,336.47818029)(156.432563,336.48818028)(156.492566,336.48818817)
\lineto(156.672566,336.51818817)
\curveto(156.75256268,336.52818024)(156.83756259,336.53318024)(156.927566,336.53318817)
\curveto(157.01756241,336.53318024)(157.10256233,336.53818023)(157.182566,336.54818817)
}
}
{
\newrgbcolor{curcolor}{0 0 0}
\pscustom[linestyle=none,fillstyle=solid,fillcolor=curcolor]
{
\newpath
\moveto(170.03920662,332.78318817)
\curveto(170.05919805,332.72318405)(170.06919804,332.63818413)(170.06920662,332.52818817)
\curveto(170.06919804,332.41818435)(170.05919805,332.33318444)(170.03920662,332.27318817)
\lineto(170.03920662,332.12318817)
\curveto(170.01919809,332.04318473)(170.0091981,331.96318481)(170.00920662,331.88318817)
\curveto(170.01919809,331.80318497)(170.0141981,331.72318505)(169.99420662,331.64318817)
\curveto(169.97419814,331.5731852)(169.95919815,331.50818526)(169.94920662,331.44818817)
\curveto(169.93919817,331.38818538)(169.92919818,331.32318545)(169.91920662,331.25318817)
\curveto(169.87919823,331.14318563)(169.84419827,331.02818574)(169.81420662,330.90818817)
\curveto(169.78419833,330.79818597)(169.74419837,330.69318608)(169.69420662,330.59318817)
\curveto(169.48419863,330.11318666)(169.2091989,329.72318705)(168.86920662,329.42318817)
\curveto(168.52919958,329.12318765)(168.11919999,328.8731879)(167.63920662,328.67318817)
\curveto(167.51920059,328.62318815)(167.39420072,328.58818818)(167.26420662,328.56818817)
\curveto(167.14420097,328.53818823)(167.01920109,328.50818826)(166.88920662,328.47818817)
\curveto(166.83920127,328.45818831)(166.78420133,328.44818832)(166.72420662,328.44818817)
\curveto(166.66420145,328.44818832)(166.6092015,328.44318833)(166.55920662,328.43318817)
\lineto(166.45420662,328.43318817)
\curveto(166.42420169,328.42318835)(166.39420172,328.41818835)(166.36420662,328.41818817)
\curveto(166.3142018,328.40818836)(166.23420188,328.40318837)(166.12420662,328.40318817)
\curveto(166.0142021,328.39318838)(165.92920218,328.39818837)(165.86920662,328.41818817)
\lineto(165.71920662,328.41818817)
\curveto(165.66920244,328.42818834)(165.6142025,328.43318834)(165.55420662,328.43318817)
\curveto(165.50420261,328.42318835)(165.45420266,328.42818834)(165.40420662,328.44818817)
\curveto(165.36420275,328.45818831)(165.32420279,328.46318831)(165.28420662,328.46318817)
\curveto(165.25420286,328.46318831)(165.2142029,328.4681883)(165.16420662,328.47818817)
\curveto(165.06420305,328.50818826)(164.96420315,328.53318824)(164.86420662,328.55318817)
\curveto(164.76420335,328.5731882)(164.66920344,328.60318817)(164.57920662,328.64318817)
\curveto(164.45920365,328.68318809)(164.34420377,328.72318805)(164.23420662,328.76318817)
\curveto(164.13420398,328.80318797)(164.02920408,328.85318792)(163.91920662,328.91318817)
\curveto(163.56920454,329.12318765)(163.26920484,329.3681874)(163.01920662,329.64818817)
\curveto(162.76920534,329.92818684)(162.55920555,330.26318651)(162.38920662,330.65318817)
\curveto(162.33920577,330.74318603)(162.29920581,330.83818593)(162.26920662,330.93818817)
\curveto(162.24920586,331.03818573)(162.22420589,331.14318563)(162.19420662,331.25318817)
\curveto(162.17420594,331.30318547)(162.16420595,331.34818542)(162.16420662,331.38818817)
\curveto(162.16420595,331.42818534)(162.15420596,331.4731853)(162.13420662,331.52318817)
\curveto(162.114206,331.60318517)(162.10420601,331.68318509)(162.10420662,331.76318817)
\curveto(162.10420601,331.85318492)(162.09420602,331.93818483)(162.07420662,332.01818817)
\curveto(162.06420605,332.0681847)(162.05920605,332.11318466)(162.05920662,332.15318817)
\lineto(162.05920662,332.28818817)
\curveto(162.03920607,332.34818442)(162.02920608,332.43318434)(162.02920662,332.54318817)
\curveto(162.03920607,332.65318412)(162.05420606,332.73818403)(162.07420662,332.79818817)
\lineto(162.07420662,332.90318817)
\curveto(162.08420603,332.95318382)(162.08420603,333.00318377)(162.07420662,333.05318817)
\curveto(162.07420604,333.11318366)(162.08420603,333.1681836)(162.10420662,333.21818817)
\curveto(162.114206,333.2681835)(162.11920599,333.31318346)(162.11920662,333.35318817)
\curveto(162.11920599,333.40318337)(162.12920598,333.45318332)(162.14920662,333.50318817)
\curveto(162.18920592,333.63318314)(162.22420589,333.75818301)(162.25420662,333.87818817)
\curveto(162.28420583,334.00818276)(162.32420579,334.13318264)(162.37420662,334.25318817)
\curveto(162.55420556,334.66318211)(162.76920534,335.00318177)(163.01920662,335.27318817)
\curveto(163.26920484,335.55318122)(163.57420454,335.80818096)(163.93420662,336.03818817)
\curveto(164.03420408,336.08818068)(164.13920397,336.13318064)(164.24920662,336.17318817)
\curveto(164.35920375,336.21318056)(164.46920364,336.25818051)(164.57920662,336.30818817)
\curveto(164.7092034,336.35818041)(164.84420327,336.39318038)(164.98420662,336.41318817)
\curveto(165.12420299,336.43318034)(165.26920284,336.46318031)(165.41920662,336.50318817)
\curveto(165.49920261,336.51318026)(165.57420254,336.51818025)(165.64420662,336.51818817)
\curveto(165.7142024,336.51818025)(165.78420233,336.52318025)(165.85420662,336.53318817)
\curveto(166.43420168,336.54318023)(166.93420118,336.48318029)(167.35420662,336.35318817)
\curveto(167.78420033,336.22318055)(168.16419995,336.04318073)(168.49420662,335.81318817)
\curveto(168.60419951,335.73318104)(168.7141994,335.64318113)(168.82420662,335.54318817)
\curveto(168.94419917,335.45318132)(169.04419907,335.35318142)(169.12420662,335.24318817)
\curveto(169.20419891,335.14318163)(169.27419884,335.04318173)(169.33420662,334.94318817)
\curveto(169.40419871,334.84318193)(169.47419864,334.73818203)(169.54420662,334.62818817)
\curveto(169.6141985,334.51818225)(169.66919844,334.39818237)(169.70920662,334.26818817)
\curveto(169.74919836,334.14818262)(169.79419832,334.01818275)(169.84420662,333.87818817)
\curveto(169.87419824,333.79818297)(169.89919821,333.71318306)(169.91920662,333.62318817)
\lineto(169.97920662,333.35318817)
\curveto(169.98919812,333.31318346)(169.99419812,333.2731835)(169.99420662,333.23318817)
\curveto(169.99419812,333.19318358)(169.99919811,333.15318362)(170.00920662,333.11318817)
\curveto(170.02919808,333.06318371)(170.03419808,333.00818376)(170.02420662,332.94818817)
\curveto(170.0141981,332.88818388)(170.01919809,332.83318394)(170.03920662,332.78318817)
\moveto(167.93920662,332.24318817)
\curveto(167.94920016,332.29318448)(167.95420016,332.36318441)(167.95420662,332.45318817)
\curveto(167.95420016,332.55318422)(167.94920016,332.62818414)(167.93920662,332.67818817)
\lineto(167.93920662,332.79818817)
\curveto(167.91920019,332.84818392)(167.9092002,332.90318387)(167.90920662,332.96318817)
\curveto(167.9092002,333.02318375)(167.90420021,333.07818369)(167.89420662,333.12818817)
\curveto(167.89420022,333.1681836)(167.88920022,333.19818357)(167.87920662,333.21818817)
\lineto(167.81920662,333.45818817)
\curveto(167.8092003,333.54818322)(167.78920032,333.63318314)(167.75920662,333.71318817)
\curveto(167.64920046,333.9731828)(167.51920059,334.19318258)(167.36920662,334.37318817)
\curveto(167.21920089,334.56318221)(167.01920109,334.71318206)(166.76920662,334.82318817)
\curveto(166.7092014,334.84318193)(166.64920146,334.85818191)(166.58920662,334.86818817)
\curveto(166.52920158,334.88818188)(166.46420165,334.90818186)(166.39420662,334.92818817)
\curveto(166.3142018,334.94818182)(166.22920188,334.95318182)(166.13920662,334.94318817)
\lineto(165.86920662,334.94318817)
\curveto(165.83920227,334.92318185)(165.80420231,334.91318186)(165.76420662,334.91318817)
\curveto(165.72420239,334.92318185)(165.68920242,334.92318185)(165.65920662,334.91318817)
\lineto(165.44920662,334.85318817)
\curveto(165.38920272,334.84318193)(165.33420278,334.82318195)(165.28420662,334.79318817)
\curveto(165.03420308,334.68318209)(164.82920328,334.52318225)(164.66920662,334.31318817)
\curveto(164.51920359,334.11318266)(164.39920371,333.87818289)(164.30920662,333.60818817)
\curveto(164.27920383,333.50818326)(164.25420386,333.40318337)(164.23420662,333.29318817)
\curveto(164.22420389,333.18318359)(164.2092039,333.0731837)(164.18920662,332.96318817)
\curveto(164.17920393,332.91318386)(164.17420394,332.86318391)(164.17420662,332.81318817)
\lineto(164.17420662,332.66318817)
\curveto(164.15420396,332.59318418)(164.14420397,332.48818428)(164.14420662,332.34818817)
\curveto(164.15420396,332.20818456)(164.16920394,332.10318467)(164.18920662,332.03318817)
\lineto(164.18920662,331.89818817)
\curveto(164.2092039,331.81818495)(164.22420389,331.73818503)(164.23420662,331.65818817)
\curveto(164.24420387,331.58818518)(164.25920385,331.51318526)(164.27920662,331.43318817)
\curveto(164.37920373,331.13318564)(164.48420363,330.88818588)(164.59420662,330.69818817)
\curveto(164.7142034,330.51818625)(164.89920321,330.35318642)(165.14920662,330.20318817)
\curveto(165.21920289,330.15318662)(165.29420282,330.11318666)(165.37420662,330.08318817)
\curveto(165.46420265,330.05318672)(165.55420256,330.02818674)(165.64420662,330.00818817)
\curveto(165.68420243,329.99818677)(165.71920239,329.99318678)(165.74920662,329.99318817)
\curveto(165.77920233,330.00318677)(165.8142023,330.00318677)(165.85420662,329.99318817)
\lineto(165.97420662,329.96318817)
\curveto(166.02420209,329.96318681)(166.06920204,329.9681868)(166.10920662,329.97818817)
\lineto(166.22920662,329.97818817)
\curveto(166.3092018,329.99818677)(166.38920172,330.01318676)(166.46920662,330.02318817)
\curveto(166.54920156,330.03318674)(166.62420149,330.05318672)(166.69420662,330.08318817)
\curveto(166.95420116,330.18318659)(167.16420095,330.31818645)(167.32420662,330.48818817)
\curveto(167.48420063,330.65818611)(167.61920049,330.8681859)(167.72920662,331.11818817)
\curveto(167.76920034,331.21818555)(167.79920031,331.31818545)(167.81920662,331.41818817)
\curveto(167.83920027,331.51818525)(167.86420025,331.62318515)(167.89420662,331.73318817)
\curveto(167.90420021,331.773185)(167.9092002,331.80818496)(167.90920662,331.83818817)
\curveto(167.9092002,331.87818489)(167.9142002,331.91818485)(167.92420662,331.95818817)
\lineto(167.92420662,332.09318817)
\curveto(167.92420019,332.14318463)(167.92920018,332.19318458)(167.93920662,332.24318817)
}
}
{
\newrgbcolor{curcolor}{0 0 0}
\pscustom[linestyle=none,fillstyle=solid,fillcolor=curcolor]
{
\newpath
\moveto(175.8641285,336.53318817)
\curveto(176.46412269,336.55318022)(176.96412219,336.4681803)(177.3641285,336.27818817)
\curveto(177.76412139,336.08818068)(178.07912108,335.80818096)(178.3091285,335.43818817)
\curveto(178.37912078,335.32818144)(178.43412072,335.20818156)(178.4741285,335.07818817)
\curveto(178.51412064,334.95818181)(178.5541206,334.83318194)(178.5941285,334.70318817)
\curveto(178.61412054,334.62318215)(178.62412053,334.54818222)(178.6241285,334.47818817)
\curveto(178.63412052,334.40818236)(178.64912051,334.33818243)(178.6691285,334.26818817)
\curveto(178.66912049,334.20818256)(178.67412048,334.1681826)(178.6841285,334.14818817)
\curveto(178.70412045,334.00818276)(178.71412044,333.86318291)(178.7141285,333.71318817)
\lineto(178.7141285,333.27818817)
\lineto(178.7141285,331.94318817)
\lineto(178.7141285,329.51318817)
\curveto(178.71412044,329.32318745)(178.70912045,329.13818763)(178.6991285,328.95818817)
\curveto(178.69912046,328.78818798)(178.62912053,328.67818809)(178.4891285,328.62818817)
\curveto(178.42912073,328.60818816)(178.3591208,328.59818817)(178.2791285,328.59818817)
\lineto(178.0391285,328.59818817)
\lineto(177.2291285,328.59818817)
\curveto(177.10912205,328.59818817)(176.99912216,328.60318817)(176.8991285,328.61318817)
\curveto(176.80912235,328.63318814)(176.73912242,328.67818809)(176.6891285,328.74818817)
\curveto(176.64912251,328.80818796)(176.62412253,328.88318789)(176.6141285,328.97318817)
\lineto(176.6141285,329.28818817)
\lineto(176.6141285,330.33818817)
\lineto(176.6141285,332.57318817)
\curveto(176.61412254,332.94318383)(176.59912256,333.28318349)(176.5691285,333.59318817)
\curveto(176.53912262,333.91318286)(176.44912271,334.18318259)(176.2991285,334.40318817)
\curveto(176.159123,334.60318217)(175.9541232,334.74318203)(175.6841285,334.82318817)
\curveto(175.63412352,334.84318193)(175.57912358,334.85318192)(175.5191285,334.85318817)
\curveto(175.46912369,334.85318192)(175.41412374,334.86318191)(175.3541285,334.88318817)
\curveto(175.30412385,334.89318188)(175.23912392,334.89318188)(175.1591285,334.88318817)
\curveto(175.08912407,334.88318189)(175.03412412,334.87818189)(174.9941285,334.86818817)
\curveto(174.9541242,334.85818191)(174.91912424,334.85318192)(174.8891285,334.85318817)
\curveto(174.8591243,334.85318192)(174.82912433,334.84818192)(174.7991285,334.83818817)
\curveto(174.56912459,334.77818199)(174.38412477,334.69818207)(174.2441285,334.59818817)
\curveto(173.92412523,334.3681824)(173.73412542,334.03318274)(173.6741285,333.59318817)
\curveto(173.61412554,333.15318362)(173.58412557,332.65818411)(173.5841285,332.10818817)
\lineto(173.5841285,330.23318817)
\lineto(173.5841285,329.31818817)
\lineto(173.5841285,329.04818817)
\curveto(173.58412557,328.95818781)(173.56912559,328.88318789)(173.5391285,328.82318817)
\curveto(173.48912567,328.71318806)(173.40912575,328.64818812)(173.2991285,328.62818817)
\curveto(173.18912597,328.60818816)(173.0541261,328.59818817)(172.8941285,328.59818817)
\lineto(172.1441285,328.59818817)
\curveto(172.03412712,328.59818817)(171.92412723,328.60318817)(171.8141285,328.61318817)
\curveto(171.70412745,328.62318815)(171.62412753,328.65818811)(171.5741285,328.71818817)
\curveto(171.50412765,328.80818796)(171.46912769,328.93818783)(171.4691285,329.10818817)
\curveto(171.47912768,329.27818749)(171.48412767,329.43818733)(171.4841285,329.58818817)
\lineto(171.4841285,331.62818817)
\lineto(171.4841285,334.92818817)
\lineto(171.4841285,335.69318817)
\lineto(171.4841285,335.99318817)
\curveto(171.49412766,336.08318069)(171.52412763,336.15818061)(171.5741285,336.21818817)
\curveto(171.59412756,336.24818052)(171.62412753,336.2681805)(171.6641285,336.27818817)
\curveto(171.71412744,336.29818047)(171.76412739,336.31318046)(171.8141285,336.32318817)
\lineto(171.8891285,336.32318817)
\curveto(171.93912722,336.33318044)(171.98912717,336.33818043)(172.0391285,336.33818817)
\lineto(172.2041285,336.33818817)
\lineto(172.8341285,336.33818817)
\curveto(172.91412624,336.33818043)(172.98912617,336.33318044)(173.0591285,336.32318817)
\curveto(173.13912602,336.32318045)(173.20912595,336.31318046)(173.2691285,336.29318817)
\curveto(173.33912582,336.26318051)(173.38412577,336.21818055)(173.4041285,336.15818817)
\curveto(173.43412572,336.09818067)(173.4591257,336.02818074)(173.4791285,335.94818817)
\curveto(173.48912567,335.90818086)(173.48912567,335.8731809)(173.4791285,335.84318817)
\curveto(173.47912568,335.81318096)(173.48912567,335.78318099)(173.5091285,335.75318817)
\curveto(173.52912563,335.70318107)(173.54412561,335.6731811)(173.5541285,335.66318817)
\curveto(173.57412558,335.65318112)(173.59912556,335.63818113)(173.6291285,335.61818817)
\curveto(173.73912542,335.60818116)(173.82912533,335.64318113)(173.8991285,335.72318817)
\curveto(173.96912519,335.81318096)(174.04412511,335.88318089)(174.1241285,335.93318817)
\curveto(174.39412476,336.13318064)(174.69412446,336.29318048)(175.0241285,336.41318817)
\curveto(175.11412404,336.44318033)(175.20412395,336.46318031)(175.2941285,336.47318817)
\curveto(175.39412376,336.48318029)(175.49912366,336.49818027)(175.6091285,336.51818817)
\curveto(175.63912352,336.52818024)(175.68412347,336.52818024)(175.7441285,336.51818817)
\curveto(175.80412335,336.51818025)(175.84412331,336.52318025)(175.8641285,336.53318817)
}
}
{
\newrgbcolor{curcolor}{0 0 0}
\pscustom[linestyle=none,fillstyle=solid,fillcolor=curcolor]
{
\newpath
\moveto(187.3953785,329.19818817)
\curveto(187.41537065,329.08818768)(187.42537064,328.97818779)(187.4253785,328.86818817)
\curveto(187.43537063,328.75818801)(187.38537068,328.68318809)(187.2753785,328.64318817)
\curveto(187.21537085,328.61318816)(187.14537092,328.59818817)(187.0653785,328.59818817)
\lineto(186.8253785,328.59818817)
\lineto(186.0153785,328.59818817)
\lineto(185.7453785,328.59818817)
\curveto(185.6653724,328.60818816)(185.60037246,328.63318814)(185.5503785,328.67318817)
\curveto(185.48037258,328.71318806)(185.42537264,328.768188)(185.3853785,328.83818817)
\curveto(185.35537271,328.91818785)(185.31037275,328.98318779)(185.2503785,329.03318817)
\curveto(185.23037283,329.05318772)(185.20537286,329.0681877)(185.1753785,329.07818817)
\curveto(185.14537292,329.09818767)(185.10537296,329.10318767)(185.0553785,329.09318817)
\curveto(185.00537306,329.0731877)(184.95537311,329.04818772)(184.9053785,329.01818817)
\curveto(184.8653732,328.98818778)(184.82037324,328.96318781)(184.7703785,328.94318817)
\curveto(184.72037334,328.90318787)(184.6653734,328.8681879)(184.6053785,328.83818817)
\lineto(184.4253785,328.74818817)
\curveto(184.29537377,328.68818808)(184.1603739,328.63818813)(184.0203785,328.59818817)
\curveto(183.88037418,328.5681882)(183.73537433,328.53318824)(183.5853785,328.49318817)
\curveto(183.51537455,328.4731883)(183.44537462,328.46318831)(183.3753785,328.46318817)
\curveto(183.31537475,328.45318832)(183.25037481,328.44318833)(183.1803785,328.43318817)
\lineto(183.0903785,328.43318817)
\curveto(183.060375,328.42318835)(183.03037503,328.41818835)(183.0003785,328.41818817)
\lineto(182.8353785,328.41818817)
\curveto(182.73537533,328.39818837)(182.63537543,328.39818837)(182.5353785,328.41818817)
\lineto(182.4003785,328.41818817)
\curveto(182.33037573,328.43818833)(182.2603758,328.44818832)(182.1903785,328.44818817)
\curveto(182.13037593,328.43818833)(182.07037599,328.44318833)(182.0103785,328.46318817)
\curveto(181.91037615,328.48318829)(181.81537625,328.50318827)(181.7253785,328.52318817)
\curveto(181.63537643,328.53318824)(181.55037651,328.55818821)(181.4703785,328.59818817)
\curveto(181.18037688,328.70818806)(180.93037713,328.84818792)(180.7203785,329.01818817)
\curveto(180.52037754,329.19818757)(180.3603777,329.43318734)(180.2403785,329.72318817)
\curveto(180.21037785,329.79318698)(180.18037788,329.8681869)(180.1503785,329.94818817)
\curveto(180.13037793,330.02818674)(180.11037795,330.11318666)(180.0903785,330.20318817)
\curveto(180.07037799,330.25318652)(180.060378,330.30318647)(180.0603785,330.35318817)
\curveto(180.07037799,330.40318637)(180.07037799,330.45318632)(180.0603785,330.50318817)
\curveto(180.05037801,330.53318624)(180.04037802,330.59318618)(180.0303785,330.68318817)
\curveto(180.03037803,330.78318599)(180.03537803,330.85318592)(180.0453785,330.89318817)
\curveto(180.065378,330.99318578)(180.07537799,331.07818569)(180.0753785,331.14818817)
\lineto(180.1653785,331.47818817)
\curveto(180.19537787,331.59818517)(180.23537783,331.70318507)(180.2853785,331.79318817)
\curveto(180.45537761,332.08318469)(180.65037741,332.30318447)(180.8703785,332.45318817)
\curveto(181.09037697,332.60318417)(181.37037669,332.73318404)(181.7103785,332.84318817)
\curveto(181.84037622,332.89318388)(181.97537609,332.92818384)(182.1153785,332.94818817)
\curveto(182.25537581,332.9681838)(182.39537567,332.99318378)(182.5353785,333.02318817)
\curveto(182.61537545,333.04318373)(182.70037536,333.05318372)(182.7903785,333.05318817)
\curveto(182.88037518,333.06318371)(182.97037509,333.07818369)(183.0603785,333.09818817)
\curveto(183.13037493,333.11818365)(183.20037486,333.12318365)(183.2703785,333.11318817)
\curveto(183.34037472,333.11318366)(183.41537465,333.12318365)(183.4953785,333.14318817)
\curveto(183.5653745,333.16318361)(183.63537443,333.1731836)(183.7053785,333.17318817)
\curveto(183.77537429,333.1731836)(183.85037421,333.18318359)(183.9303785,333.20318817)
\curveto(184.14037392,333.25318352)(184.33037373,333.29318348)(184.5003785,333.32318817)
\curveto(184.68037338,333.36318341)(184.84037322,333.45318332)(184.9803785,333.59318817)
\curveto(185.07037299,333.68318309)(185.13037293,333.78318299)(185.1603785,333.89318817)
\curveto(185.17037289,333.92318285)(185.17037289,333.94818282)(185.1603785,333.96818817)
\curveto(185.1603729,333.98818278)(185.1653729,334.00818276)(185.1753785,334.02818817)
\curveto(185.18537288,334.04818272)(185.19037287,334.07818269)(185.1903785,334.11818817)
\lineto(185.1903785,334.20818817)
\lineto(185.1603785,334.32818817)
\curveto(185.1603729,334.3681824)(185.15537291,334.40318237)(185.1453785,334.43318817)
\curveto(185.04537302,334.73318204)(184.83537323,334.93818183)(184.5153785,335.04818817)
\curveto(184.42537364,335.07818169)(184.31537375,335.09818167)(184.1853785,335.10818817)
\curveto(184.065374,335.12818164)(183.94037412,335.13318164)(183.8103785,335.12318817)
\curveto(183.68037438,335.12318165)(183.55537451,335.11318166)(183.4353785,335.09318817)
\curveto(183.31537475,335.0731817)(183.21037485,335.04818172)(183.1203785,335.01818817)
\curveto(183.060375,334.99818177)(183.00037506,334.9681818)(182.9403785,334.92818817)
\curveto(182.89037517,334.89818187)(182.84037522,334.86318191)(182.7903785,334.82318817)
\curveto(182.74037532,334.78318199)(182.68537538,334.72818204)(182.6253785,334.65818817)
\curveto(182.57537549,334.58818218)(182.54037552,334.52318225)(182.5203785,334.46318817)
\curveto(182.47037559,334.36318241)(182.42537564,334.2681825)(182.3853785,334.17818817)
\curveto(182.35537571,334.08818268)(182.28537578,334.02818274)(182.1753785,333.99818817)
\curveto(182.09537597,333.97818279)(182.01037605,333.9681828)(181.9203785,333.96818817)
\lineto(181.6503785,333.96818817)
\lineto(181.0803785,333.96818817)
\curveto(181.03037703,333.9681828)(180.98037708,333.96318281)(180.9303785,333.95318817)
\curveto(180.88037718,333.95318282)(180.83537723,333.95818281)(180.7953785,333.96818817)
\lineto(180.6603785,333.96818817)
\curveto(180.64037742,333.97818279)(180.61537745,333.98318279)(180.5853785,333.98318817)
\curveto(180.55537751,333.98318279)(180.53037753,333.99318278)(180.5103785,334.01318817)
\curveto(180.43037763,334.03318274)(180.37537769,334.09818267)(180.3453785,334.20818817)
\curveto(180.33537773,334.25818251)(180.33537773,334.30818246)(180.3453785,334.35818817)
\curveto(180.35537771,334.40818236)(180.3653777,334.45318232)(180.3753785,334.49318817)
\curveto(180.40537766,334.60318217)(180.43537763,334.70318207)(180.4653785,334.79318817)
\curveto(180.50537756,334.89318188)(180.55037751,334.98318179)(180.6003785,335.06318817)
\lineto(180.6903785,335.21318817)
\lineto(180.7803785,335.36318817)
\curveto(180.8603772,335.4731813)(180.9603771,335.57818119)(181.0803785,335.67818817)
\curveto(181.10037696,335.68818108)(181.13037693,335.71318106)(181.1703785,335.75318817)
\curveto(181.22037684,335.79318098)(181.2653768,335.82818094)(181.3053785,335.85818817)
\curveto(181.34537672,335.88818088)(181.39037667,335.91818085)(181.4403785,335.94818817)
\curveto(181.61037645,336.05818071)(181.79037627,336.14318063)(181.9803785,336.20318817)
\curveto(182.17037589,336.2731805)(182.3653757,336.33818043)(182.5653785,336.39818817)
\curveto(182.68537538,336.42818034)(182.81037525,336.44818032)(182.9403785,336.45818817)
\curveto(183.07037499,336.4681803)(183.20037486,336.48818028)(183.3303785,336.51818817)
\curveto(183.37037469,336.52818024)(183.43037463,336.52818024)(183.5103785,336.51818817)
\curveto(183.60037446,336.50818026)(183.65537441,336.51318026)(183.6753785,336.53318817)
\curveto(184.08537398,336.54318023)(184.47537359,336.52818024)(184.8453785,336.48818817)
\curveto(185.22537284,336.44818032)(185.5653725,336.3731804)(185.8653785,336.26318817)
\curveto(186.17537189,336.15318062)(186.44037162,336.00318077)(186.6603785,335.81318817)
\curveto(186.88037118,335.63318114)(187.05037101,335.39818137)(187.1703785,335.10818817)
\curveto(187.24037082,334.93818183)(187.28037078,334.74318203)(187.2903785,334.52318817)
\curveto(187.30037076,334.30318247)(187.30537076,334.07818269)(187.3053785,333.84818817)
\lineto(187.3053785,330.50318817)
\lineto(187.3053785,329.91818817)
\curveto(187.30537076,329.72818704)(187.32537074,329.55318722)(187.3653785,329.39318817)
\curveto(187.37537069,329.36318741)(187.38037068,329.32818744)(187.3803785,329.28818817)
\curveto(187.38037068,329.25818751)(187.38537068,329.22818754)(187.3953785,329.19818817)
\moveto(185.1903785,331.50818817)
\curveto(185.20037286,331.55818521)(185.20537286,331.61318516)(185.2053785,331.67318817)
\curveto(185.20537286,331.74318503)(185.20037286,331.80318497)(185.1903785,331.85318817)
\curveto(185.17037289,331.91318486)(185.1603729,331.9681848)(185.1603785,332.01818817)
\curveto(185.1603729,332.0681847)(185.14037292,332.10818466)(185.1003785,332.13818817)
\curveto(185.05037301,332.17818459)(184.97537309,332.19818457)(184.8753785,332.19818817)
\curveto(184.83537323,332.18818458)(184.80037326,332.17818459)(184.7703785,332.16818817)
\curveto(184.74037332,332.1681846)(184.70537336,332.16318461)(184.6653785,332.15318817)
\curveto(184.59537347,332.13318464)(184.52037354,332.11818465)(184.4403785,332.10818817)
\curveto(184.3603737,332.09818467)(184.28037378,332.08318469)(184.2003785,332.06318817)
\curveto(184.17037389,332.05318472)(184.12537394,332.04818472)(184.0653785,332.04818817)
\curveto(183.93537413,332.01818475)(183.80537426,331.99818477)(183.6753785,331.98818817)
\curveto(183.54537452,331.97818479)(183.42037464,331.95318482)(183.3003785,331.91318817)
\curveto(183.22037484,331.89318488)(183.14537492,331.8731849)(183.0753785,331.85318817)
\curveto(183.00537506,331.84318493)(182.93537513,331.82318495)(182.8653785,331.79318817)
\curveto(182.65537541,331.70318507)(182.47537559,331.5681852)(182.3253785,331.38818817)
\curveto(182.18537588,331.20818556)(182.13537593,330.95818581)(182.1753785,330.63818817)
\curveto(182.19537587,330.4681863)(182.25037581,330.32818644)(182.3403785,330.21818817)
\curveto(182.41037565,330.10818666)(182.51537555,330.01818675)(182.6553785,329.94818817)
\curveto(182.79537527,329.88818688)(182.94537512,329.84318693)(183.1053785,329.81318817)
\curveto(183.27537479,329.78318699)(183.45037461,329.773187)(183.6303785,329.78318817)
\curveto(183.82037424,329.80318697)(183.99537407,329.83818693)(184.1553785,329.88818817)
\curveto(184.41537365,329.9681868)(184.62037344,330.09318668)(184.7703785,330.26318817)
\curveto(184.92037314,330.44318633)(185.03537303,330.66318611)(185.1153785,330.92318817)
\curveto(185.13537293,330.99318578)(185.14537292,331.06318571)(185.1453785,331.13318817)
\curveto(185.15537291,331.21318556)(185.17037289,331.29318548)(185.1903785,331.37318817)
\lineto(185.1903785,331.50818817)
}
}
{
\newrgbcolor{curcolor}{0 0 0}
\pscustom[linestyle=none,fillstyle=solid,fillcolor=curcolor]
{
\newpath
\moveto(189.46865975,339.29318817)
\lineto(190.56365975,339.29318817)
\curveto(190.66365726,339.29317748)(190.75865717,339.28817748)(190.84865975,339.27818817)
\curveto(190.93865699,339.2681775)(191.00865692,339.23817753)(191.05865975,339.18818817)
\curveto(191.11865681,339.11817765)(191.14865678,339.02317775)(191.14865975,338.90318817)
\curveto(191.15865677,338.79317798)(191.16365676,338.67817809)(191.16365975,338.55818817)
\lineto(191.16365975,337.22318817)
\lineto(191.16365975,331.83818817)
\lineto(191.16365975,329.54318817)
\lineto(191.16365975,329.12318817)
\curveto(191.17365675,328.9731878)(191.15365677,328.85818791)(191.10365975,328.77818817)
\curveto(191.05365687,328.69818807)(190.96365696,328.64318813)(190.83365975,328.61318817)
\curveto(190.77365715,328.59318818)(190.70365722,328.58818818)(190.62365975,328.59818817)
\curveto(190.55365737,328.60818816)(190.48365744,328.61318816)(190.41365975,328.61318817)
\lineto(189.69365975,328.61318817)
\curveto(189.58365834,328.61318816)(189.48365844,328.61818815)(189.39365975,328.62818817)
\curveto(189.30365862,328.63818813)(189.2286587,328.6681881)(189.16865975,328.71818817)
\curveto(189.10865882,328.768188)(189.07365885,328.84318793)(189.06365975,328.94318817)
\lineto(189.06365975,329.27318817)
\lineto(189.06365975,330.60818817)
\lineto(189.06365975,336.23318817)
\lineto(189.06365975,338.27318817)
\curveto(189.06365886,338.40317837)(189.05865887,338.55817821)(189.04865975,338.73818817)
\curveto(189.04865888,338.91817785)(189.07365885,339.04817772)(189.12365975,339.12818817)
\curveto(189.14365878,339.1681776)(189.16865876,339.19817757)(189.19865975,339.21818817)
\lineto(189.31865975,339.27818817)
\curveto(189.33865859,339.27817749)(189.36365856,339.27817749)(189.39365975,339.27818817)
\curveto(189.4236585,339.28817748)(189.44865848,339.29317748)(189.46865975,339.29318817)
}
}
{
\newrgbcolor{curcolor}{0.80000001 0.80000001 0.80000001}
\pscustom[linestyle=none,fillstyle=solid,fillcolor=curcolor]
{
\newpath
\moveto(242.05118898,1012.72623666)
\curveto(322.69324385,1003.45493389)(380.55067,930.56575594)(371.27936723,849.92370107)
\curveto(362.00806446,769.2816462)(289.11888651,711.42422004)(208.47683165,720.69552281)
\curveto(127.83477678,729.96682558)(69.97735062,802.85600353)(79.24865339,883.4980584)
\curveto(86.60533092,947.48664855)(134.76786875,999.23032194)(198.0655138,1011.14957127)
\lineto(225.26401031,866.71087974)
\closepath
}
}
{
\newrgbcolor{curcolor}{0.90196079 0.90196079 0.90196079}
\pscustom[linestyle=none,fillstyle=solid,fillcolor=curcolor]
{
\newpath
\moveto(225.26401319,1013.68806785)
\curveto(230.8661189,1013.68806774)(236.46364286,1013.36777744)(242.02918427,1012.72876483)
\lineto(225.26401031,866.71087974)
\closepath
}
}
{
\newrgbcolor{curcolor}{0.7019608 0.7019608 0.7019608}
\pscustom[linestyle=none,fillstyle=solid,fillcolor=curcolor]
{
\newpath
\moveto(198.05021893,1011.14669034)
\curveto(202.36113445,1011.95892904)(206.70625256,1012.57776143)(211.0725227,1013.00132886)
\lineto(225.26401031,866.71087974)
\closepath
}
}
{
\newrgbcolor{curcolor}{0.60000002 0.60000002 0.60000002}
\pscustom[linestyle=none,fillstyle=solid,fillcolor=curcolor]
{
\newpath
\moveto(211.05265659,1012.9994003)
\curveto(212.72032784,1013.161408)(214.39064748,1013.29488698)(216.062883,1013.39977871)
\lineto(225.26401031,866.71087974)
\closepath
}
}
{
\newrgbcolor{curcolor}{0.50196081 0.50196081 0.50196081}
\pscustom[linestyle=none,fillstyle=solid,fillcolor=curcolor]
{
\newpath
\moveto(216.02490555,1013.39739162)
\curveto(216.71365047,1013.44077247)(217.40269237,1013.47930194)(218.09198,1013.51297715)
\lineto(225.26401031,866.71087974)
\closepath
}
}
{
\newrgbcolor{curcolor}{0.40000001 0.40000001 0.40000001}
\pscustom[linestyle=none,fillstyle=solid,fillcolor=curcolor]
{
\newpath
\moveto(218.07995694,1013.51238927)
\curveto(219.51669975,1013.58269943)(220.95439734,1013.63191375)(222.39258498,1013.66001632)
\lineto(225.26401031,866.71087974)
\closepath
}
}
{
\newrgbcolor{curcolor}{0.3019608 0.3019608 0.3019608}
\pscustom[linestyle=none,fillstyle=solid,fillcolor=curcolor]
{
\newpath
\moveto(222.36431202,1013.65946113)
\curveto(223.33076149,1013.67853183)(224.29737558,1013.68806787)(225.26401319,1013.68806785)
\lineto(225.26401031,866.71087974)
\closepath
}
}
{
\newrgbcolor{curcolor}{0 0 0}
\pscustom[linestyle=none,fillstyle=solid,fillcolor=curcolor]
{
\newpath
\moveto(565.37181562,243.91410181)
\curveto(565.44181389,243.91409115)(565.52181381,243.91409115)(565.61181562,243.91410181)
\curveto(565.70181363,243.92409114)(565.78681354,243.92409114)(565.86681562,243.91410181)
\curveto(565.95681337,243.91409115)(566.03681329,243.90409116)(566.10681562,243.88410181)
\curveto(566.17681315,243.8640912)(566.2268131,243.83409123)(566.25681562,243.79410181)
\curveto(566.31681301,243.72409134)(566.34681298,243.62409144)(566.34681562,243.49410181)
\curveto(566.35681297,243.37409169)(566.36181297,243.24909181)(566.36181562,243.11910181)
\lineto(566.36181562,241.66410181)
\lineto(566.36181562,235.87410181)
\lineto(566.36181562,234.11910181)
\lineto(566.36181562,233.69910181)
\curveto(566.36181297,233.5591015)(566.33681299,233.44910161)(566.28681562,233.36910181)
\curveto(566.24681308,233.31910174)(566.19681313,233.28910177)(566.13681562,233.27910181)
\curveto(566.08681324,233.26910179)(566.02181331,233.25410181)(565.94181562,233.23410181)
\lineto(565.65681562,233.23410181)
\curveto(565.51681381,233.23410183)(565.38681394,233.23910182)(565.26681562,233.24910181)
\curveto(565.14681418,233.2591018)(565.06181427,233.30910175)(565.01181562,233.39910181)
\curveto(564.97181436,233.4591016)(564.95181438,233.53910152)(564.95181562,233.63910181)
\lineto(564.95181562,233.96910181)
\lineto(564.95181562,235.16910181)
\lineto(564.95181562,241.43910181)
\lineto(564.95181562,243.05910181)
\curveto(564.95181438,243.16909189)(564.94681438,243.28909177)(564.93681562,243.41910181)
\curveto(564.93681439,243.5590915)(564.96181437,243.66909139)(565.01181562,243.74910181)
\curveto(565.05181428,243.81909124)(565.1318142,243.86909119)(565.25181562,243.89910181)
\curveto(565.27181406,243.90909115)(565.29181404,243.90909115)(565.31181562,243.89910181)
\curveto(565.331814,243.89909116)(565.35181398,243.90409116)(565.37181562,243.91410181)
}
}
{
\newrgbcolor{curcolor}{0 0 0}
\pscustom[linestyle=none,fillstyle=solid,fillcolor=curcolor]
{
\newpath
\moveto(572.2083,241.13910181)
\curveto(572.83829476,241.1590939)(573.34329426,241.07409399)(573.7233,240.88410181)
\curveto(574.1032935,240.69409437)(574.40829319,240.40909465)(574.6383,240.02910181)
\curveto(574.6982929,239.92909513)(574.74329286,239.81909524)(574.7733,239.69910181)
\curveto(574.81329279,239.58909547)(574.84829275,239.47409559)(574.8783,239.35410181)
\curveto(574.92829267,239.1640959)(574.95829264,238.9590961)(574.9683,238.73910181)
\curveto(574.97829262,238.51909654)(574.98329262,238.29409677)(574.9833,238.06410181)
\lineto(574.9833,236.45910181)
\lineto(574.9833,234.11910181)
\curveto(574.98329262,233.94910111)(574.97829262,233.77910128)(574.9683,233.60910181)
\curveto(574.96829263,233.43910162)(574.9032927,233.32910173)(574.7733,233.27910181)
\curveto(574.72329288,233.2591018)(574.66829293,233.24910181)(574.6083,233.24910181)
\curveto(574.55829304,233.23910182)(574.5032931,233.23410183)(574.4433,233.23410181)
\curveto(574.31329329,233.23410183)(574.18829341,233.23910182)(574.0683,233.24910181)
\curveto(573.94829365,233.24910181)(573.86329374,233.28910177)(573.8133,233.36910181)
\curveto(573.76329384,233.43910162)(573.73829386,233.52910153)(573.7383,233.63910181)
\lineto(573.7383,233.96910181)
\lineto(573.7383,235.25910181)
\lineto(573.7383,237.70410181)
\curveto(573.73829386,237.97409709)(573.73329387,238.23909682)(573.7233,238.49910181)
\curveto(573.71329389,238.76909629)(573.66829393,238.99909606)(573.5883,239.18910181)
\curveto(573.50829409,239.38909567)(573.38829421,239.54909551)(573.2283,239.66910181)
\curveto(573.06829453,239.79909526)(572.88329472,239.89909516)(572.6733,239.96910181)
\curveto(572.61329499,239.98909507)(572.54829505,239.99909506)(572.4783,239.99910181)
\curveto(572.41829518,240.00909505)(572.35829524,240.02409504)(572.2983,240.04410181)
\curveto(572.24829535,240.05409501)(572.16829543,240.05409501)(572.0583,240.04410181)
\curveto(571.95829564,240.04409502)(571.88829571,240.03909502)(571.8483,240.02910181)
\curveto(571.80829579,240.00909505)(571.77329583,239.99909506)(571.7433,239.99910181)
\curveto(571.71329589,240.00909505)(571.67829592,240.00909505)(571.6383,239.99910181)
\curveto(571.50829609,239.96909509)(571.38329622,239.93409513)(571.2633,239.89410181)
\curveto(571.15329645,239.8640952)(571.04829655,239.81909524)(570.9483,239.75910181)
\curveto(570.90829669,239.73909532)(570.87329673,239.71909534)(570.8433,239.69910181)
\curveto(570.81329679,239.67909538)(570.77829682,239.6590954)(570.7383,239.63910181)
\curveto(570.38829721,239.38909567)(570.13329747,239.01409605)(569.9733,238.51410181)
\curveto(569.94329766,238.43409663)(569.92329768,238.34909671)(569.9133,238.25910181)
\curveto(569.9032977,238.17909688)(569.88829771,238.09909696)(569.8683,238.01910181)
\curveto(569.84829775,237.96909709)(569.84329776,237.91909714)(569.8533,237.86910181)
\curveto(569.86329774,237.82909723)(569.85829774,237.78909727)(569.8383,237.74910181)
\lineto(569.8383,237.43410181)
\curveto(569.82829777,237.40409766)(569.82329778,237.36909769)(569.8233,237.32910181)
\curveto(569.83329777,237.28909777)(569.83829776,237.24409782)(569.8383,237.19410181)
\lineto(569.8383,236.74410181)
\lineto(569.8383,235.30410181)
\lineto(569.8383,233.98410181)
\lineto(569.8383,233.63910181)
\curveto(569.83829776,233.52910153)(569.81329779,233.43910162)(569.7633,233.36910181)
\curveto(569.71329789,233.28910177)(569.62329798,233.24910181)(569.4933,233.24910181)
\curveto(569.37329823,233.23910182)(569.24829835,233.23410183)(569.1183,233.23410181)
\curveto(569.03829856,233.23410183)(568.96329864,233.23910182)(568.8933,233.24910181)
\curveto(568.82329878,233.2591018)(568.76329884,233.28410178)(568.7133,233.32410181)
\curveto(568.63329897,233.37410169)(568.59329901,233.46910159)(568.5933,233.60910181)
\lineto(568.5933,234.01410181)
\lineto(568.5933,235.78410181)
\lineto(568.5933,239.41410181)
\lineto(568.5933,240.32910181)
\lineto(568.5933,240.59910181)
\curveto(568.59329901,240.68909437)(568.61329899,240.7590943)(568.6533,240.80910181)
\curveto(568.68329892,240.86909419)(568.73329887,240.90909415)(568.8033,240.92910181)
\curveto(568.84329876,240.93909412)(568.8982987,240.94909411)(568.9683,240.95910181)
\curveto(569.04829855,240.96909409)(569.12829847,240.97409409)(569.2083,240.97410181)
\curveto(569.28829831,240.97409409)(569.36329824,240.96909409)(569.4333,240.95910181)
\curveto(569.51329809,240.94909411)(569.56829803,240.93409413)(569.5983,240.91410181)
\curveto(569.70829789,240.84409422)(569.75829784,240.75409431)(569.7483,240.64410181)
\curveto(569.73829786,240.54409452)(569.75329785,240.42909463)(569.7933,240.29910181)
\curveto(569.81329779,240.23909482)(569.85329775,240.18909487)(569.9133,240.14910181)
\curveto(570.03329757,240.13909492)(570.12829747,240.18409488)(570.1983,240.28410181)
\curveto(570.27829732,240.38409468)(570.35829724,240.4640946)(570.4383,240.52410181)
\curveto(570.57829702,240.62409444)(570.71829688,240.71409435)(570.8583,240.79410181)
\curveto(571.00829659,240.88409418)(571.17829642,240.9590941)(571.3683,241.01910181)
\curveto(571.44829615,241.04909401)(571.53329607,241.06909399)(571.6233,241.07910181)
\curveto(571.72329588,241.08909397)(571.81829578,241.10409396)(571.9083,241.12410181)
\curveto(571.95829564,241.13409393)(572.00829559,241.13909392)(572.0583,241.13910181)
\lineto(572.2083,241.13910181)
}
}
{
\newrgbcolor{curcolor}{0 0 0}
\pscustom[linestyle=none,fillstyle=solid,fillcolor=curcolor]
{
\newpath
\moveto(576.64290937,240.98910181)
\lineto(577.12290937,240.98910181)
\curveto(577.29290803,240.98909407)(577.4229079,240.9590941)(577.51290937,240.89910181)
\curveto(577.58290774,240.84909421)(577.6279077,240.78409428)(577.64790937,240.70410181)
\curveto(577.67790765,240.63409443)(577.70790762,240.5590945)(577.73790937,240.47910181)
\curveto(577.79790753,240.33909472)(577.84790748,240.19909486)(577.88790937,240.05910181)
\curveto(577.9279074,239.91909514)(577.97290735,239.77909528)(578.02290937,239.63910181)
\curveto(578.2229071,239.09909596)(578.40790692,238.55409651)(578.57790937,238.00410181)
\curveto(578.74790658,237.4640976)(578.93290639,236.92409814)(579.13290937,236.38410181)
\curveto(579.20290612,236.20409886)(579.26290606,236.01909904)(579.31290937,235.82910181)
\curveto(579.36290596,235.64909941)(579.4279059,235.46909959)(579.50790937,235.28910181)
\curveto(579.5279058,235.21909984)(579.55290577,235.14409992)(579.58290937,235.06410181)
\curveto(579.61290571,234.98410008)(579.66290566,234.93410013)(579.73290937,234.91410181)
\curveto(579.81290551,234.89410017)(579.87290545,234.92910013)(579.91290937,235.01910181)
\curveto(579.96290536,235.10909995)(579.99790533,235.17909988)(580.01790937,235.22910181)
\curveto(580.09790523,235.41909964)(580.16290516,235.60909945)(580.21290937,235.79910181)
\curveto(580.27290505,235.99909906)(580.33790499,236.19909886)(580.40790937,236.39910181)
\curveto(580.53790479,236.77909828)(580.66290466,237.15409791)(580.78290937,237.52410181)
\curveto(580.90290442,237.90409716)(581.0279043,238.28409678)(581.15790937,238.66410181)
\curveto(581.20790412,238.83409623)(581.25790407,238.99909606)(581.30790937,239.15910181)
\curveto(581.35790397,239.32909573)(581.41790391,239.49409557)(581.48790937,239.65410181)
\curveto(581.53790379,239.79409527)(581.58290374,239.93409513)(581.62290937,240.07410181)
\curveto(581.66290366,240.21409485)(581.70790362,240.35409471)(581.75790937,240.49410181)
\curveto(581.77790355,240.5640945)(581.80290352,240.63409443)(581.83290937,240.70410181)
\curveto(581.86290346,240.77409429)(581.90290342,240.83409423)(581.95290937,240.88410181)
\curveto(582.03290329,240.93409413)(582.1229032,240.9640941)(582.22290937,240.97410181)
\curveto(582.322903,240.98409408)(582.44290288,240.98909407)(582.58290937,240.98910181)
\curveto(582.65290267,240.98909407)(582.71790261,240.98409408)(582.77790937,240.97410181)
\curveto(582.83790249,240.97409409)(582.89290243,240.9640941)(582.94290937,240.94410181)
\curveto(583.03290229,240.90409416)(583.07790225,240.83909422)(583.07790937,240.74910181)
\curveto(583.08790224,240.6590944)(583.07290225,240.56909449)(583.03290937,240.47910181)
\curveto(582.97290235,240.30909475)(582.91290241,240.13409493)(582.85290937,239.95410181)
\curveto(582.79290253,239.77409529)(582.7229026,239.59909546)(582.64290937,239.42910181)
\curveto(582.6229027,239.37909568)(582.60790272,239.32909573)(582.59790937,239.27910181)
\curveto(582.58790274,239.23909582)(582.57290275,239.19409587)(582.55290937,239.14410181)
\curveto(582.47290285,238.97409609)(582.40790292,238.79909626)(582.35790937,238.61910181)
\curveto(582.30790302,238.43909662)(582.24290308,238.2590968)(582.16290937,238.07910181)
\curveto(582.11290321,237.94909711)(582.06290326,237.81409725)(582.01290937,237.67410181)
\curveto(581.97290335,237.54409752)(581.9229034,237.41409765)(581.86290937,237.28410181)
\curveto(581.69290363,236.87409819)(581.53790379,236.4590986)(581.39790937,236.03910181)
\curveto(581.26790406,235.61909944)(581.11790421,235.20409986)(580.94790937,234.79410181)
\curveto(580.88790444,234.63410043)(580.83290449,234.47410059)(580.78290937,234.31410181)
\curveto(580.73290459,234.15410091)(580.67290465,233.99410107)(580.60290937,233.83410181)
\curveto(580.55290477,233.72410134)(580.50790482,233.61910144)(580.46790937,233.51910181)
\curveto(580.43790489,233.42910163)(580.36790496,233.3591017)(580.25790937,233.30910181)
\curveto(580.19790513,233.27910178)(580.1279052,233.2641018)(580.04790937,233.26410181)
\lineto(579.82290937,233.26410181)
\lineto(579.35790937,233.26410181)
\curveto(579.20790612,233.27410179)(579.09790623,233.32410174)(579.02790937,233.41410181)
\curveto(578.95790637,233.49410157)(578.90790642,233.58910147)(578.87790937,233.69910181)
\curveto(578.84790648,233.81910124)(578.80790652,233.93410113)(578.75790937,234.04410181)
\curveto(578.69790663,234.18410088)(578.63790669,234.32910073)(578.57790937,234.47910181)
\curveto(578.5279068,234.63910042)(578.47790685,234.78910027)(578.42790937,234.92910181)
\curveto(578.40790692,234.97910008)(578.39290693,235.01910004)(578.38290937,235.04910181)
\curveto(578.37290695,235.08909997)(578.35790697,235.13409993)(578.33790937,235.18410181)
\curveto(578.13790719,235.6640994)(577.95290737,236.14909891)(577.78290937,236.63910181)
\curveto(577.6229077,237.12909793)(577.44290788,237.61409745)(577.24290937,238.09410181)
\curveto(577.18290814,238.25409681)(577.1229082,238.40909665)(577.06290937,238.55910181)
\curveto(577.01290831,238.71909634)(576.95790837,238.87909618)(576.89790937,239.03910181)
\lineto(576.83790937,239.18910181)
\curveto(576.8279085,239.24909581)(576.81290851,239.30409576)(576.79290937,239.35410181)
\curveto(576.71290861,239.52409554)(576.64290868,239.69409537)(576.58290937,239.86410181)
\curveto(576.53290879,240.03409503)(576.47290885,240.20409486)(576.40290937,240.37410181)
\curveto(576.38290894,240.43409463)(576.35790897,240.51409455)(576.32790937,240.61410181)
\curveto(576.29790903,240.71409435)(576.30290902,240.79909426)(576.34290937,240.86910181)
\curveto(576.39290893,240.91909414)(576.45290887,240.95409411)(576.52290937,240.97410181)
\curveto(576.59290873,240.97409409)(576.63290869,240.97909408)(576.64290937,240.98910181)
}
}
{
\newrgbcolor{curcolor}{0 0 0}
\pscustom[linestyle=none,fillstyle=solid,fillcolor=curcolor]
{
\newpath
\moveto(584.65290937,242.48910181)
\curveto(584.57290825,242.54909251)(584.5279083,242.65409241)(584.51790937,242.80410181)
\lineto(584.51790937,243.26910181)
\lineto(584.51790937,243.52410181)
\curveto(584.51790831,243.61409145)(584.53290829,243.68909137)(584.56290937,243.74910181)
\curveto(584.60290822,243.82909123)(584.68290814,243.88909117)(584.80290937,243.92910181)
\curveto(584.822908,243.93909112)(584.84290798,243.93909112)(584.86290937,243.92910181)
\curveto(584.89290793,243.92909113)(584.91790791,243.93409113)(584.93790937,243.94410181)
\curveto(585.10790772,243.94409112)(585.26790756,243.93909112)(585.41790937,243.92910181)
\curveto(585.56790726,243.91909114)(585.66790716,243.8590912)(585.71790937,243.74910181)
\curveto(585.74790708,243.68909137)(585.76290706,243.61409145)(585.76290937,243.52410181)
\lineto(585.76290937,243.26910181)
\curveto(585.76290706,243.08909197)(585.75790707,242.91909214)(585.74790937,242.75910181)
\curveto(585.74790708,242.59909246)(585.68290714,242.49409257)(585.55290937,242.44410181)
\curveto(585.50290732,242.42409264)(585.44790738,242.41409265)(585.38790937,242.41410181)
\lineto(585.22290937,242.41410181)
\lineto(584.90790937,242.41410181)
\curveto(584.80790802,242.41409265)(584.7229081,242.43909262)(584.65290937,242.48910181)
\moveto(585.76290937,233.98410181)
\lineto(585.76290937,233.66910181)
\curveto(585.77290705,233.56910149)(585.75290707,233.48910157)(585.70290937,233.42910181)
\curveto(585.67290715,233.36910169)(585.6279072,233.32910173)(585.56790937,233.30910181)
\curveto(585.50790732,233.29910176)(585.43790739,233.28410178)(585.35790937,233.26410181)
\lineto(585.13290937,233.26410181)
\curveto(585.00290782,233.2641018)(584.88790794,233.26910179)(584.78790937,233.27910181)
\curveto(584.69790813,233.29910176)(584.6279082,233.34910171)(584.57790937,233.42910181)
\curveto(584.53790829,233.48910157)(584.51790831,233.5641015)(584.51790937,233.65410181)
\lineto(584.51790937,233.93910181)
\lineto(584.51790937,240.28410181)
\lineto(584.51790937,240.59910181)
\curveto(584.51790831,240.70909435)(584.54290828,240.79409427)(584.59290937,240.85410181)
\curveto(584.6229082,240.90409416)(584.66290816,240.93409413)(584.71290937,240.94410181)
\curveto(584.76290806,240.95409411)(584.81790801,240.96909409)(584.87790937,240.98910181)
\curveto(584.89790793,240.98909407)(584.91790791,240.98409408)(584.93790937,240.97410181)
\curveto(584.96790786,240.97409409)(584.99290783,240.97909408)(585.01290937,240.98910181)
\curveto(585.14290768,240.98909407)(585.27290755,240.98409408)(585.40290937,240.97410181)
\curveto(585.54290728,240.97409409)(585.63790719,240.93409413)(585.68790937,240.85410181)
\curveto(585.73790709,240.79409427)(585.76290706,240.71409435)(585.76290937,240.61410181)
\lineto(585.76290937,240.32910181)
\lineto(585.76290937,233.98410181)
}
}
{
\newrgbcolor{curcolor}{0 0 0}
\pscustom[linestyle=none,fillstyle=solid,fillcolor=curcolor]
{
\newpath
\moveto(588.65275312,243.32910181)
\curveto(588.80275111,243.32909173)(588.95275096,243.32409174)(589.10275312,243.31410181)
\curveto(589.25275066,243.31409175)(589.35775056,243.27409179)(589.41775312,243.19410181)
\curveto(589.46775045,243.13409193)(589.49275042,243.04909201)(589.49275312,242.93910181)
\curveto(589.50275041,242.83909222)(589.50775041,242.73409233)(589.50775312,242.62410181)
\lineto(589.50775312,241.75410181)
\curveto(589.50775041,241.67409339)(589.50275041,241.58909347)(589.49275312,241.49910181)
\curveto(589.49275042,241.41909364)(589.50275041,241.34909371)(589.52275312,241.28910181)
\curveto(589.56275035,241.14909391)(589.65275026,241.059094)(589.79275312,241.01910181)
\curveto(589.84275007,241.00909405)(589.88775003,241.00409406)(589.92775312,241.00410181)
\lineto(590.07775312,241.00410181)
\lineto(590.48275312,241.00410181)
\curveto(590.64274927,241.01409405)(590.75774916,241.00409406)(590.82775312,240.97410181)
\curveto(590.917749,240.91409415)(590.97774894,240.85409421)(591.00775312,240.79410181)
\curveto(591.02774889,240.75409431)(591.03774888,240.70909435)(591.03775312,240.65910181)
\lineto(591.03775312,240.50910181)
\curveto(591.03774888,240.39909466)(591.03274888,240.29409477)(591.02275312,240.19410181)
\curveto(591.0127489,240.10409496)(590.97774894,240.03409503)(590.91775312,239.98410181)
\curveto(590.85774906,239.93409513)(590.77274914,239.90409516)(590.66275312,239.89410181)
\lineto(590.33275312,239.89410181)
\curveto(590.22274969,239.90409516)(590.1127498,239.90909515)(590.00275312,239.90910181)
\curveto(589.89275002,239.90909515)(589.79775012,239.89409517)(589.71775312,239.86410181)
\curveto(589.64775027,239.83409523)(589.59775032,239.78409528)(589.56775312,239.71410181)
\curveto(589.53775038,239.64409542)(589.5177504,239.5590955)(589.50775312,239.45910181)
\curveto(589.49775042,239.36909569)(589.49275042,239.26909579)(589.49275312,239.15910181)
\curveto(589.50275041,239.059096)(589.50775041,238.9590961)(589.50775312,238.85910181)
\lineto(589.50775312,235.88910181)
\curveto(589.50775041,235.66909939)(589.50275041,235.43409963)(589.49275312,235.18410181)
\curveto(589.49275042,234.94410012)(589.53775038,234.7591003)(589.62775312,234.62910181)
\curveto(589.67775024,234.54910051)(589.74275017,234.49410057)(589.82275312,234.46410181)
\curveto(589.90275001,234.43410063)(589.99774992,234.40910065)(590.10775312,234.38910181)
\curveto(590.13774978,234.37910068)(590.16774975,234.37410069)(590.19775312,234.37410181)
\curveto(590.23774968,234.38410068)(590.27274964,234.38410068)(590.30275312,234.37410181)
\lineto(590.49775312,234.37410181)
\curveto(590.59774932,234.37410069)(590.68774923,234.3641007)(590.76775312,234.34410181)
\curveto(590.85774906,234.33410073)(590.92274899,234.29910076)(590.96275312,234.23910181)
\curveto(590.98274893,234.20910085)(590.99774892,234.15410091)(591.00775312,234.07410181)
\curveto(591.02774889,234.00410106)(591.03774888,233.92910113)(591.03775312,233.84910181)
\curveto(591.04774887,233.76910129)(591.04774887,233.68910137)(591.03775312,233.60910181)
\curveto(591.02774889,233.53910152)(591.00774891,233.48410158)(590.97775312,233.44410181)
\curveto(590.93774898,233.37410169)(590.86274905,233.32410174)(590.75275312,233.29410181)
\curveto(590.67274924,233.27410179)(590.58274933,233.2641018)(590.48275312,233.26410181)
\curveto(590.38274953,233.27410179)(590.29274962,233.27910178)(590.21275312,233.27910181)
\curveto(590.15274976,233.27910178)(590.09274982,233.27410179)(590.03275312,233.26410181)
\curveto(589.97274994,233.2641018)(589.91775,233.26910179)(589.86775312,233.27910181)
\lineto(589.68775312,233.27910181)
\curveto(589.63775028,233.28910177)(589.58775033,233.29410177)(589.53775312,233.29410181)
\curveto(589.49775042,233.30410176)(589.45275046,233.30910175)(589.40275312,233.30910181)
\curveto(589.20275071,233.3591017)(589.02775089,233.41410165)(588.87775312,233.47410181)
\curveto(588.73775118,233.53410153)(588.6177513,233.63910142)(588.51775312,233.78910181)
\curveto(588.37775154,233.98910107)(588.29775162,234.23910082)(588.27775312,234.53910181)
\curveto(588.25775166,234.84910021)(588.24775167,235.17909988)(588.24775312,235.52910181)
\lineto(588.24775312,239.45910181)
\curveto(588.2177517,239.58909547)(588.18775173,239.68409538)(588.15775312,239.74410181)
\curveto(588.13775178,239.80409526)(588.06775185,239.85409521)(587.94775312,239.89410181)
\curveto(587.90775201,239.90409516)(587.86775205,239.90409516)(587.82775312,239.89410181)
\curveto(587.78775213,239.88409518)(587.74775217,239.88909517)(587.70775312,239.90910181)
\lineto(587.46775312,239.90910181)
\curveto(587.33775258,239.90909515)(587.22775269,239.91909514)(587.13775312,239.93910181)
\curveto(587.05775286,239.96909509)(587.00275291,240.02909503)(586.97275312,240.11910181)
\curveto(586.95275296,240.1590949)(586.93775298,240.20409486)(586.92775312,240.25410181)
\lineto(586.92775312,240.40410181)
\curveto(586.92775299,240.54409452)(586.93775298,240.6590944)(586.95775312,240.74910181)
\curveto(586.97775294,240.84909421)(587.03775288,240.92409414)(587.13775312,240.97410181)
\curveto(587.24775267,241.01409405)(587.38775253,241.02409404)(587.55775312,241.00410181)
\curveto(587.73775218,240.98409408)(587.88775203,240.99409407)(588.00775312,241.03410181)
\curveto(588.09775182,241.08409398)(588.16775175,241.15409391)(588.21775312,241.24410181)
\curveto(588.23775168,241.30409376)(588.24775167,241.37909368)(588.24775312,241.46910181)
\lineto(588.24775312,241.72410181)
\lineto(588.24775312,242.65410181)
\lineto(588.24775312,242.89410181)
\curveto(588.24775167,242.98409208)(588.25775166,243.059092)(588.27775312,243.11910181)
\curveto(588.3177516,243.19909186)(588.39275152,243.2640918)(588.50275312,243.31410181)
\curveto(588.53275138,243.31409175)(588.55775136,243.31409175)(588.57775312,243.31410181)
\curveto(588.60775131,243.32409174)(588.63275128,243.32909173)(588.65275312,243.32910181)
}
}
{
\newrgbcolor{curcolor}{0 0 0}
\pscustom[linestyle=none,fillstyle=solid,fillcolor=curcolor]
{
\newpath
\moveto(599.30955,233.81910181)
\curveto(599.33954217,233.6591014)(599.32454218,233.52410154)(599.26455,233.41410181)
\curveto(599.2045423,233.31410175)(599.12454238,233.23910182)(599.02455,233.18910181)
\curveto(598.97454253,233.16910189)(598.91954259,233.1591019)(598.85955,233.15910181)
\curveto(598.8095427,233.1591019)(598.75454275,233.14910191)(598.69455,233.12910181)
\curveto(598.47454303,233.07910198)(598.25454325,233.09410197)(598.03455,233.17410181)
\curveto(597.82454368,233.24410182)(597.67954383,233.33410173)(597.59955,233.44410181)
\curveto(597.54954396,233.51410155)(597.504544,233.59410147)(597.46455,233.68410181)
\curveto(597.42454408,233.78410128)(597.37454413,233.8641012)(597.31455,233.92410181)
\curveto(597.29454421,233.94410112)(597.26954424,233.9641011)(597.23955,233.98410181)
\curveto(597.21954429,234.00410106)(597.18954432,234.00910105)(597.14955,233.99910181)
\curveto(597.03954447,233.96910109)(596.93454457,233.91410115)(596.83455,233.83410181)
\curveto(596.74454476,233.75410131)(596.65454485,233.68410138)(596.56455,233.62410181)
\curveto(596.43454507,233.54410152)(596.29454521,233.46910159)(596.14455,233.39910181)
\curveto(595.99454551,233.33910172)(595.83454567,233.28410178)(595.66455,233.23410181)
\curveto(595.56454594,233.20410186)(595.45454605,233.18410188)(595.33455,233.17410181)
\curveto(595.22454628,233.1641019)(595.11454639,233.14910191)(595.00455,233.12910181)
\curveto(594.95454655,233.11910194)(594.9095466,233.11410195)(594.86955,233.11410181)
\lineto(594.76455,233.11410181)
\curveto(594.65454685,233.09410197)(594.54954696,233.09410197)(594.44955,233.11410181)
\lineto(594.31455,233.11410181)
\curveto(594.26454724,233.12410194)(594.21454729,233.12910193)(594.16455,233.12910181)
\curveto(594.11454739,233.12910193)(594.06954744,233.13910192)(594.02955,233.15910181)
\curveto(593.98954752,233.16910189)(593.95454755,233.17410189)(593.92455,233.17410181)
\curveto(593.9045476,233.1641019)(593.87954763,233.1641019)(593.84955,233.17410181)
\lineto(593.60955,233.23410181)
\curveto(593.52954798,233.24410182)(593.45454805,233.2641018)(593.38455,233.29410181)
\curveto(593.08454842,233.42410164)(592.83954867,233.56910149)(592.64955,233.72910181)
\curveto(592.46954904,233.89910116)(592.31954919,234.13410093)(592.19955,234.43410181)
\curveto(592.1095494,234.65410041)(592.06454944,234.91910014)(592.06455,235.22910181)
\lineto(592.06455,235.54410181)
\curveto(592.07454943,235.59409947)(592.07954943,235.64409942)(592.07955,235.69410181)
\lineto(592.10955,235.87410181)
\lineto(592.22955,236.20410181)
\curveto(592.26954924,236.31409875)(592.31954919,236.41409865)(592.37955,236.50410181)
\curveto(592.55954895,236.79409827)(592.8045487,237.00909805)(593.11455,237.14910181)
\curveto(593.42454808,237.28909777)(593.76454774,237.41409765)(594.13455,237.52410181)
\curveto(594.27454723,237.5640975)(594.41954709,237.59409747)(594.56955,237.61410181)
\curveto(594.71954679,237.63409743)(594.86954664,237.6590974)(595.01955,237.68910181)
\curveto(595.08954642,237.70909735)(595.15454635,237.71909734)(595.21455,237.71910181)
\curveto(595.28454622,237.71909734)(595.35954615,237.72909733)(595.43955,237.74910181)
\curveto(595.509546,237.76909729)(595.57954593,237.77909728)(595.64955,237.77910181)
\curveto(595.71954579,237.78909727)(595.79454571,237.80409726)(595.87455,237.82410181)
\curveto(596.12454538,237.88409718)(596.35954515,237.93409713)(596.57955,237.97410181)
\curveto(596.79954471,238.02409704)(596.97454453,238.13909692)(597.10455,238.31910181)
\curveto(597.16454434,238.39909666)(597.21454429,238.49909656)(597.25455,238.61910181)
\curveto(597.29454421,238.74909631)(597.29454421,238.88909617)(597.25455,239.03910181)
\curveto(597.19454431,239.27909578)(597.1045444,239.46909559)(596.98455,239.60910181)
\curveto(596.87454463,239.74909531)(596.71454479,239.8590952)(596.50455,239.93910181)
\curveto(596.38454512,239.98909507)(596.23954527,240.02409504)(596.06955,240.04410181)
\curveto(595.9095456,240.064095)(595.73954577,240.07409499)(595.55955,240.07410181)
\curveto(595.37954613,240.07409499)(595.2045463,240.064095)(595.03455,240.04410181)
\curveto(594.86454664,240.02409504)(594.71954679,239.99409507)(594.59955,239.95410181)
\curveto(594.42954708,239.89409517)(594.26454724,239.80909525)(594.10455,239.69910181)
\curveto(594.02454748,239.63909542)(593.94954756,239.5590955)(593.87955,239.45910181)
\curveto(593.81954769,239.36909569)(593.76454774,239.26909579)(593.71455,239.15910181)
\curveto(593.68454782,239.07909598)(593.65454785,238.99409607)(593.62455,238.90410181)
\curveto(593.6045479,238.81409625)(593.55954795,238.74409632)(593.48955,238.69410181)
\curveto(593.44954806,238.6640964)(593.37954813,238.63909642)(593.27955,238.61910181)
\curveto(593.18954832,238.60909645)(593.09454841,238.60409646)(592.99455,238.60410181)
\curveto(592.89454861,238.60409646)(592.79454871,238.60909645)(592.69455,238.61910181)
\curveto(592.6045489,238.63909642)(592.53954897,238.6640964)(592.49955,238.69410181)
\curveto(592.45954905,238.72409634)(592.42954908,238.77409629)(592.40955,238.84410181)
\curveto(592.38954912,238.91409615)(592.38954912,238.98909607)(592.40955,239.06910181)
\curveto(592.43954907,239.19909586)(592.46954904,239.31909574)(592.49955,239.42910181)
\curveto(592.53954897,239.54909551)(592.58454892,239.6640954)(592.63455,239.77410181)
\curveto(592.82454868,240.12409494)(593.06454844,240.39409467)(593.35455,240.58410181)
\curveto(593.64454786,240.78409428)(594.0045475,240.94409412)(594.43455,241.06410181)
\curveto(594.53454697,241.08409398)(594.63454687,241.09909396)(594.73455,241.10910181)
\curveto(594.84454666,241.11909394)(594.95454655,241.13409393)(595.06455,241.15410181)
\curveto(595.1045464,241.1640939)(595.16954634,241.1640939)(595.25955,241.15410181)
\curveto(595.34954616,241.15409391)(595.4045461,241.1640939)(595.42455,241.18410181)
\curveto(596.12454538,241.19409387)(596.73454477,241.11409395)(597.25455,240.94410181)
\curveto(597.77454373,240.77409429)(598.13954337,240.44909461)(598.34955,239.96910181)
\curveto(598.43954307,239.76909529)(598.48954302,239.53409553)(598.49955,239.26410181)
\curveto(598.51954299,239.00409606)(598.52954298,238.72909633)(598.52955,238.43910181)
\lineto(598.52955,235.12410181)
\curveto(598.52954298,234.98410008)(598.53454297,234.84910021)(598.54455,234.71910181)
\curveto(598.55454295,234.58910047)(598.58454292,234.48410058)(598.63455,234.40410181)
\curveto(598.68454282,234.33410073)(598.74954276,234.28410078)(598.82955,234.25410181)
\curveto(598.91954259,234.21410085)(599.0045425,234.18410088)(599.08455,234.16410181)
\curveto(599.16454234,234.15410091)(599.22454228,234.10910095)(599.26455,234.02910181)
\curveto(599.28454222,233.99910106)(599.29454221,233.96910109)(599.29455,233.93910181)
\curveto(599.29454221,233.90910115)(599.29954221,233.86910119)(599.30955,233.81910181)
\moveto(597.16455,235.48410181)
\curveto(597.22454428,235.62409944)(597.25454425,235.78409928)(597.25455,235.96410181)
\curveto(597.26454424,236.15409891)(597.26954424,236.34909871)(597.26955,236.54910181)
\curveto(597.26954424,236.6590984)(597.26454424,236.7590983)(597.25455,236.84910181)
\curveto(597.24454426,236.93909812)(597.2045443,237.00909805)(597.13455,237.05910181)
\curveto(597.1045444,237.07909798)(597.03454447,237.08909797)(596.92455,237.08910181)
\curveto(596.9045446,237.06909799)(596.86954464,237.059098)(596.81955,237.05910181)
\curveto(596.76954474,237.059098)(596.72454478,237.04909801)(596.68455,237.02910181)
\curveto(596.6045449,237.00909805)(596.51454499,236.98909807)(596.41455,236.96910181)
\lineto(596.11455,236.90910181)
\curveto(596.08454542,236.90909815)(596.04954546,236.90409816)(596.00955,236.89410181)
\lineto(595.90455,236.89410181)
\curveto(595.75454575,236.85409821)(595.58954592,236.82909823)(595.40955,236.81910181)
\curveto(595.23954627,236.81909824)(595.07954643,236.79909826)(594.92955,236.75910181)
\curveto(594.84954666,236.73909832)(594.77454673,236.71909834)(594.70455,236.69910181)
\curveto(594.64454686,236.68909837)(594.57454693,236.67409839)(594.49455,236.65410181)
\curveto(594.33454717,236.60409846)(594.18454732,236.53909852)(594.04455,236.45910181)
\curveto(593.9045476,236.38909867)(593.78454772,236.29909876)(593.68455,236.18910181)
\curveto(593.58454792,236.07909898)(593.509548,235.94409912)(593.45955,235.78410181)
\curveto(593.4095481,235.63409943)(593.38954812,235.44909961)(593.39955,235.22910181)
\curveto(593.39954811,235.12909993)(593.41454809,235.03410003)(593.44455,234.94410181)
\curveto(593.48454802,234.8641002)(593.52954798,234.78910027)(593.57955,234.71910181)
\curveto(593.65954785,234.60910045)(593.76454774,234.51410055)(593.89455,234.43410181)
\curveto(594.02454748,234.3641007)(594.16454734,234.30410076)(594.31455,234.25410181)
\curveto(594.36454714,234.24410082)(594.41454709,234.23910082)(594.46455,234.23910181)
\curveto(594.51454699,234.23910082)(594.56454694,234.23410083)(594.61455,234.22410181)
\curveto(594.68454682,234.20410086)(594.76954674,234.18910087)(594.86955,234.17910181)
\curveto(594.97954653,234.17910088)(595.06954644,234.18910087)(595.13955,234.20910181)
\curveto(595.19954631,234.22910083)(595.25954625,234.23410083)(595.31955,234.22410181)
\curveto(595.37954613,234.22410084)(595.43954607,234.23410083)(595.49955,234.25410181)
\curveto(595.57954593,234.27410079)(595.65454585,234.28910077)(595.72455,234.29910181)
\curveto(595.8045457,234.30910075)(595.87954563,234.32910073)(595.94955,234.35910181)
\curveto(596.23954527,234.47910058)(596.48454502,234.62410044)(596.68455,234.79410181)
\curveto(596.89454461,234.9641001)(597.05454445,235.19409987)(597.16455,235.48410181)
}
}
{
\newrgbcolor{curcolor}{0 0 0}
\pscustom[linestyle=none,fillstyle=solid,fillcolor=curcolor]
{
\newpath
\moveto(607.44119062,234.07410181)
\lineto(607.44119062,233.68410181)
\curveto(607.44118275,233.5641015)(607.41618277,233.4641016)(607.36619062,233.38410181)
\curveto(607.31618287,233.31410175)(607.23118296,233.27410179)(607.11119062,233.26410181)
\lineto(606.76619062,233.26410181)
\curveto(606.70618348,233.2641018)(606.64618354,233.2591018)(606.58619062,233.24910181)
\curveto(606.53618365,233.24910181)(606.4911837,233.2591018)(606.45119062,233.27910181)
\curveto(606.36118383,233.29910176)(606.30118389,233.33910172)(606.27119062,233.39910181)
\curveto(606.23118396,233.44910161)(606.20618398,233.50910155)(606.19619062,233.57910181)
\curveto(606.19618399,233.64910141)(606.18118401,233.71910134)(606.15119062,233.78910181)
\curveto(606.14118405,233.80910125)(606.12618406,233.82410124)(606.10619062,233.83410181)
\curveto(606.09618409,233.85410121)(606.08118411,233.87410119)(606.06119062,233.89410181)
\curveto(605.96118423,233.90410116)(605.88118431,233.88410118)(605.82119062,233.83410181)
\curveto(605.77118442,233.78410128)(605.71618447,233.73410133)(605.65619062,233.68410181)
\curveto(605.45618473,233.53410153)(605.25618493,233.41910164)(605.05619062,233.33910181)
\curveto(604.87618531,233.2591018)(604.66618552,233.19910186)(604.42619062,233.15910181)
\curveto(604.19618599,233.11910194)(603.95618623,233.09910196)(603.70619062,233.09910181)
\curveto(603.46618672,233.08910197)(603.22618696,233.10410196)(602.98619062,233.14410181)
\curveto(602.74618744,233.17410189)(602.53618765,233.22910183)(602.35619062,233.30910181)
\curveto(601.83618835,233.52910153)(601.41618877,233.82410124)(601.09619062,234.19410181)
\curveto(600.77618941,234.57410049)(600.52618966,235.04410002)(600.34619062,235.60410181)
\curveto(600.30618988,235.69409937)(600.27618991,235.78409928)(600.25619062,235.87410181)
\curveto(600.24618994,235.97409909)(600.22618996,236.07409899)(600.19619062,236.17410181)
\curveto(600.18619,236.22409884)(600.18119001,236.27409879)(600.18119062,236.32410181)
\curveto(600.18119001,236.37409869)(600.17619001,236.42409864)(600.16619062,236.47410181)
\curveto(600.14619004,236.52409854)(600.13619005,236.57409849)(600.13619062,236.62410181)
\curveto(600.14619004,236.68409838)(600.14619004,236.73909832)(600.13619062,236.78910181)
\lineto(600.13619062,236.93910181)
\curveto(600.11619007,236.98909807)(600.10619008,237.05409801)(600.10619062,237.13410181)
\curveto(600.10619008,237.21409785)(600.11619007,237.27909778)(600.13619062,237.32910181)
\lineto(600.13619062,237.49410181)
\curveto(600.15619003,237.5640975)(600.16119003,237.63409743)(600.15119062,237.70410181)
\curveto(600.15119004,237.78409728)(600.16119003,237.8590972)(600.18119062,237.92910181)
\curveto(600.19119,237.97909708)(600.19618999,238.02409704)(600.19619062,238.06410181)
\curveto(600.19618999,238.10409696)(600.20118999,238.14909691)(600.21119062,238.19910181)
\curveto(600.24118995,238.29909676)(600.26618992,238.39409667)(600.28619062,238.48410181)
\curveto(600.30618988,238.58409648)(600.33118986,238.67909638)(600.36119062,238.76910181)
\curveto(600.4911897,239.14909591)(600.65618953,239.48909557)(600.85619062,239.78910181)
\curveto(601.06618912,240.09909496)(601.31618887,240.35409471)(601.60619062,240.55410181)
\curveto(601.77618841,240.67409439)(601.95118824,240.77409429)(602.13119062,240.85410181)
\curveto(602.32118787,240.93409413)(602.52618766,241.00409406)(602.74619062,241.06410181)
\curveto(602.81618737,241.07409399)(602.88118731,241.08409398)(602.94119062,241.09410181)
\curveto(603.01118718,241.10409396)(603.08118711,241.11909394)(603.15119062,241.13910181)
\lineto(603.30119062,241.13910181)
\curveto(603.38118681,241.1590939)(603.49618669,241.16909389)(603.64619062,241.16910181)
\curveto(603.80618638,241.16909389)(603.92618626,241.1590939)(604.00619062,241.13910181)
\curveto(604.04618614,241.12909393)(604.10118609,241.12409394)(604.17119062,241.12410181)
\curveto(604.28118591,241.09409397)(604.3911858,241.06909399)(604.50119062,241.04910181)
\curveto(604.61118558,241.03909402)(604.71618547,241.00909405)(604.81619062,240.95910181)
\curveto(604.96618522,240.89909416)(605.10618508,240.83409423)(605.23619062,240.76410181)
\curveto(605.37618481,240.69409437)(605.50618468,240.61409445)(605.62619062,240.52410181)
\curveto(605.6861845,240.47409459)(605.74618444,240.41909464)(605.80619062,240.35910181)
\curveto(605.87618431,240.30909475)(605.96618422,240.29409477)(606.07619062,240.31410181)
\curveto(606.09618409,240.34409472)(606.11118408,240.36909469)(606.12119062,240.38910181)
\curveto(606.14118405,240.40909465)(606.15618403,240.43909462)(606.16619062,240.47910181)
\curveto(606.19618399,240.56909449)(606.20618398,240.68409438)(606.19619062,240.82410181)
\lineto(606.19619062,241.19910181)
\lineto(606.19619062,242.92410181)
\lineto(606.19619062,243.38910181)
\curveto(606.19618399,243.56909149)(606.22118397,243.69909136)(606.27119062,243.77910181)
\curveto(606.31118388,243.84909121)(606.37118382,243.89409117)(606.45119062,243.91410181)
\curveto(606.47118372,243.91409115)(606.49618369,243.91409115)(606.52619062,243.91410181)
\curveto(606.55618363,243.92409114)(606.58118361,243.92909113)(606.60119062,243.92910181)
\curveto(606.74118345,243.93909112)(606.8861833,243.93909112)(607.03619062,243.92910181)
\curveto(607.19618299,243.92909113)(607.30618288,243.88909117)(607.36619062,243.80910181)
\curveto(607.41618277,243.72909133)(607.44118275,243.62909143)(607.44119062,243.50910181)
\lineto(607.44119062,243.13410181)
\lineto(607.44119062,234.07410181)
\moveto(606.22619062,236.90910181)
\curveto(606.24618394,236.9590981)(606.25618393,237.02409804)(606.25619062,237.10410181)
\curveto(606.25618393,237.19409787)(606.24618394,237.2640978)(606.22619062,237.31410181)
\lineto(606.22619062,237.53910181)
\curveto(606.20618398,237.62909743)(606.191184,237.71909734)(606.18119062,237.80910181)
\curveto(606.17118402,237.90909715)(606.15118404,237.99909706)(606.12119062,238.07910181)
\curveto(606.10118409,238.1590969)(606.08118411,238.23409683)(606.06119062,238.30410181)
\curveto(606.05118414,238.37409669)(606.03118416,238.44409662)(606.00119062,238.51410181)
\curveto(605.88118431,238.81409625)(605.72618446,239.07909598)(605.53619062,239.30910181)
\curveto(605.34618484,239.53909552)(605.10618508,239.71909534)(604.81619062,239.84910181)
\curveto(604.71618547,239.89909516)(604.61118558,239.93409513)(604.50119062,239.95410181)
\curveto(604.40118579,239.98409508)(604.2911859,240.00909505)(604.17119062,240.02910181)
\curveto(604.0911861,240.04909501)(604.00118619,240.059095)(603.90119062,240.05910181)
\lineto(603.63119062,240.05910181)
\curveto(603.58118661,240.04909501)(603.53618665,240.03909502)(603.49619062,240.02910181)
\lineto(603.36119062,240.02910181)
\curveto(603.28118691,240.00909505)(603.19618699,239.98909507)(603.10619062,239.96910181)
\curveto(603.02618716,239.94909511)(602.94618724,239.92409514)(602.86619062,239.89410181)
\curveto(602.54618764,239.75409531)(602.2861879,239.54909551)(602.08619062,239.27910181)
\curveto(601.89618829,239.01909604)(601.74118845,238.71409635)(601.62119062,238.36410181)
\curveto(601.58118861,238.25409681)(601.55118864,238.13909692)(601.53119062,238.01910181)
\curveto(601.52118867,237.90909715)(601.50618868,237.79909726)(601.48619062,237.68910181)
\curveto(601.4861887,237.64909741)(601.48118871,237.60909745)(601.47119062,237.56910181)
\lineto(601.47119062,237.46410181)
\curveto(601.45118874,237.41409765)(601.44118875,237.3590977)(601.44119062,237.29910181)
\curveto(601.45118874,237.23909782)(601.45618873,237.18409788)(601.45619062,237.13410181)
\lineto(601.45619062,236.80410181)
\curveto(601.45618873,236.70409836)(601.46618872,236.60909845)(601.48619062,236.51910181)
\curveto(601.49618869,236.48909857)(601.50118869,236.43909862)(601.50119062,236.36910181)
\curveto(601.52118867,236.29909876)(601.53618865,236.22909883)(601.54619062,236.15910181)
\lineto(601.60619062,235.94910181)
\curveto(601.71618847,235.59909946)(601.86618832,235.29909976)(602.05619062,235.04910181)
\curveto(602.24618794,234.79910026)(602.4861877,234.59410047)(602.77619062,234.43410181)
\curveto(602.86618732,234.38410068)(602.95618723,234.34410072)(603.04619062,234.31410181)
\curveto(603.13618705,234.28410078)(603.23618695,234.25410081)(603.34619062,234.22410181)
\curveto(603.39618679,234.20410086)(603.44618674,234.19910086)(603.49619062,234.20910181)
\curveto(603.55618663,234.21910084)(603.61118658,234.21410085)(603.66119062,234.19410181)
\curveto(603.70118649,234.18410088)(603.74118645,234.17910088)(603.78119062,234.17910181)
\lineto(603.91619062,234.17910181)
\lineto(604.05119062,234.17910181)
\curveto(604.08118611,234.18910087)(604.13118606,234.19410087)(604.20119062,234.19410181)
\curveto(604.28118591,234.21410085)(604.36118583,234.22910083)(604.44119062,234.23910181)
\curveto(604.52118567,234.2591008)(604.59618559,234.28410078)(604.66619062,234.31410181)
\curveto(604.99618519,234.45410061)(605.26118493,234.62910043)(605.46119062,234.83910181)
\curveto(605.67118452,235.0591)(605.84618434,235.33409973)(605.98619062,235.66410181)
\curveto(606.03618415,235.77409929)(606.07118412,235.88409918)(606.09119062,235.99410181)
\curveto(606.11118408,236.10409896)(606.13618405,236.21409885)(606.16619062,236.32410181)
\curveto(606.186184,236.3640987)(606.19618399,236.39909866)(606.19619062,236.42910181)
\curveto(606.19618399,236.46909859)(606.20118399,236.50909855)(606.21119062,236.54910181)
\curveto(606.22118397,236.60909845)(606.22118397,236.66909839)(606.21119062,236.72910181)
\curveto(606.21118398,236.78909827)(606.21618397,236.84909821)(606.22619062,236.90910181)
}
}
{
\newrgbcolor{curcolor}{0 0 0}
\pscustom[linestyle=none,fillstyle=solid,fillcolor=curcolor]
{
\newpath
\moveto(616.51244062,237.46410181)
\curveto(616.53243256,237.40409766)(616.54243255,237.30909775)(616.54244062,237.17910181)
\curveto(616.54243255,237.059098)(616.53743256,236.97409809)(616.52744062,236.92410181)
\lineto(616.52744062,236.77410181)
\curveto(616.51743258,236.69409837)(616.50743259,236.61909844)(616.49744062,236.54910181)
\curveto(616.4974326,236.48909857)(616.4924326,236.41909864)(616.48244062,236.33910181)
\curveto(616.46243263,236.27909878)(616.44743265,236.21909884)(616.43744062,236.15910181)
\curveto(616.43743266,236.09909896)(616.42743267,236.03909902)(616.40744062,235.97910181)
\curveto(616.36743273,235.84909921)(616.33243276,235.71909934)(616.30244062,235.58910181)
\curveto(616.27243282,235.4590996)(616.23243286,235.33909972)(616.18244062,235.22910181)
\curveto(615.97243312,234.74910031)(615.6924334,234.34410072)(615.34244062,234.01410181)
\curveto(614.9924341,233.69410137)(614.56243453,233.44910161)(614.05244062,233.27910181)
\curveto(613.94243515,233.23910182)(613.82243527,233.20910185)(613.69244062,233.18910181)
\curveto(613.57243552,233.16910189)(613.44743565,233.14910191)(613.31744062,233.12910181)
\curveto(613.25743584,233.11910194)(613.1924359,233.11410195)(613.12244062,233.11410181)
\curveto(613.06243603,233.10410196)(613.00243609,233.09910196)(612.94244062,233.09910181)
\curveto(612.90243619,233.08910197)(612.84243625,233.08410198)(612.76244062,233.08410181)
\curveto(612.6924364,233.08410198)(612.64243645,233.08910197)(612.61244062,233.09910181)
\curveto(612.57243652,233.10910195)(612.53243656,233.11410195)(612.49244062,233.11410181)
\curveto(612.45243664,233.10410196)(612.41743668,233.10410196)(612.38744062,233.11410181)
\lineto(612.29744062,233.11410181)
\lineto(611.93744062,233.15910181)
\curveto(611.7974373,233.19910186)(611.66243743,233.23910182)(611.53244062,233.27910181)
\curveto(611.40243769,233.31910174)(611.27743782,233.3641017)(611.15744062,233.41410181)
\curveto(610.70743839,233.61410145)(610.33743876,233.87410119)(610.04744062,234.19410181)
\curveto(609.75743934,234.51410055)(609.51743958,234.90410016)(609.32744062,235.36410181)
\curveto(609.27743982,235.4640996)(609.23743986,235.5640995)(609.20744062,235.66410181)
\curveto(609.18743991,235.7640993)(609.16743993,235.86909919)(609.14744062,235.97910181)
\curveto(609.12743997,236.01909904)(609.11743998,236.04909901)(609.11744062,236.06910181)
\curveto(609.12743997,236.09909896)(609.12743997,236.13409893)(609.11744062,236.17410181)
\curveto(609.09744,236.25409881)(609.08244001,236.33409873)(609.07244062,236.41410181)
\curveto(609.07244002,236.50409856)(609.06244003,236.58909847)(609.04244062,236.66910181)
\lineto(609.04244062,236.78910181)
\curveto(609.04244005,236.82909823)(609.03744006,236.87409819)(609.02744062,236.92410181)
\curveto(609.01744008,236.97409809)(609.01244008,237.059098)(609.01244062,237.17910181)
\curveto(609.01244008,237.30909775)(609.02244007,237.40409766)(609.04244062,237.46410181)
\curveto(609.06244003,237.53409753)(609.06744003,237.60409746)(609.05744062,237.67410181)
\curveto(609.04744005,237.74409732)(609.05244004,237.81409725)(609.07244062,237.88410181)
\curveto(609.08244001,237.93409713)(609.08744001,237.97409709)(609.08744062,238.00410181)
\curveto(609.09744,238.04409702)(609.10743999,238.08909697)(609.11744062,238.13910181)
\curveto(609.14743995,238.2590968)(609.17243992,238.37909668)(609.19244062,238.49910181)
\curveto(609.22243987,238.61909644)(609.26243983,238.73409633)(609.31244062,238.84410181)
\curveto(609.46243963,239.21409585)(609.64243945,239.54409552)(609.85244062,239.83410181)
\curveto(610.07243902,240.13409493)(610.33743876,240.38409468)(610.64744062,240.58410181)
\curveto(610.76743833,240.6640944)(610.8924382,240.72909433)(611.02244062,240.77910181)
\curveto(611.15243794,240.83909422)(611.28743781,240.89909416)(611.42744062,240.95910181)
\curveto(611.54743755,241.00909405)(611.67743742,241.03909402)(611.81744062,241.04910181)
\curveto(611.95743714,241.06909399)(612.097437,241.09909396)(612.23744062,241.13910181)
\lineto(612.43244062,241.13910181)
\curveto(612.50243659,241.14909391)(612.56743653,241.1590939)(612.62744062,241.16910181)
\curveto(613.51743558,241.17909388)(614.25743484,240.99409407)(614.84744062,240.61410181)
\curveto(615.43743366,240.23409483)(615.86243323,239.73909532)(616.12244062,239.12910181)
\curveto(616.17243292,239.02909603)(616.21243288,238.92909613)(616.24244062,238.82910181)
\curveto(616.27243282,238.72909633)(616.30743279,238.62409644)(616.34744062,238.51410181)
\curveto(616.37743272,238.40409666)(616.40243269,238.28409678)(616.42244062,238.15410181)
\curveto(616.44243265,238.03409703)(616.46743263,237.90909715)(616.49744062,237.77910181)
\curveto(616.50743259,237.72909733)(616.50743259,237.67409739)(616.49744062,237.61410181)
\curveto(616.4974326,237.5640975)(616.50243259,237.51409755)(616.51244062,237.46410181)
\moveto(615.17744062,236.60910181)
\curveto(615.1974339,236.67909838)(615.20243389,236.7590983)(615.19244062,236.84910181)
\lineto(615.19244062,237.10410181)
\curveto(615.1924339,237.49409757)(615.15743394,237.82409724)(615.08744062,238.09410181)
\curveto(615.05743404,238.17409689)(615.03243406,238.25409681)(615.01244062,238.33410181)
\curveto(614.9924341,238.41409665)(614.96743413,238.48909657)(614.93744062,238.55910181)
\curveto(614.65743444,239.20909585)(614.21243488,239.6590954)(613.60244062,239.90910181)
\curveto(613.53243556,239.93909512)(613.45743564,239.9590951)(613.37744062,239.96910181)
\lineto(613.13744062,240.02910181)
\curveto(613.05743604,240.04909501)(612.97243612,240.059095)(612.88244062,240.05910181)
\lineto(612.61244062,240.05910181)
\lineto(612.34244062,240.01410181)
\curveto(612.24243685,239.99409507)(612.14743695,239.96909509)(612.05744062,239.93910181)
\curveto(611.97743712,239.91909514)(611.8974372,239.88909517)(611.81744062,239.84910181)
\curveto(611.74743735,239.82909523)(611.68243741,239.79909526)(611.62244062,239.75910181)
\curveto(611.56243753,239.71909534)(611.50743759,239.67909538)(611.45744062,239.63910181)
\curveto(611.21743788,239.46909559)(611.02243807,239.2640958)(610.87244062,239.02410181)
\curveto(610.72243837,238.78409628)(610.5924385,238.50409656)(610.48244062,238.18410181)
\curveto(610.45243864,238.08409698)(610.43243866,237.97909708)(610.42244062,237.86910181)
\curveto(610.41243868,237.76909729)(610.3974387,237.6640974)(610.37744062,237.55410181)
\curveto(610.36743873,237.51409755)(610.36243873,237.44909761)(610.36244062,237.35910181)
\curveto(610.35243874,237.32909773)(610.34743875,237.29409777)(610.34744062,237.25410181)
\curveto(610.35743874,237.21409785)(610.36243873,237.16909789)(610.36244062,237.11910181)
\lineto(610.36244062,236.81910181)
\curveto(610.36243873,236.71909834)(610.37243872,236.62909843)(610.39244062,236.54910181)
\lineto(610.42244062,236.36910181)
\curveto(610.44243865,236.26909879)(610.45743864,236.16909889)(610.46744062,236.06910181)
\curveto(610.48743861,235.97909908)(610.51743858,235.89409917)(610.55744062,235.81410181)
\curveto(610.65743844,235.57409949)(610.77243832,235.34909971)(610.90244062,235.13910181)
\curveto(611.04243805,234.92910013)(611.21243788,234.75410031)(611.41244062,234.61410181)
\curveto(611.46243763,234.58410048)(611.50743759,234.5591005)(611.54744062,234.53910181)
\curveto(611.58743751,234.51910054)(611.63243746,234.49410057)(611.68244062,234.46410181)
\curveto(611.76243733,234.41410065)(611.84743725,234.36910069)(611.93744062,234.32910181)
\curveto(612.03743706,234.29910076)(612.14243695,234.26910079)(612.25244062,234.23910181)
\curveto(612.30243679,234.21910084)(612.34743675,234.20910085)(612.38744062,234.20910181)
\curveto(612.43743666,234.21910084)(612.48743661,234.21910084)(612.53744062,234.20910181)
\curveto(612.56743653,234.19910086)(612.62743647,234.18910087)(612.71744062,234.17910181)
\curveto(612.81743628,234.16910089)(612.8924362,234.17410089)(612.94244062,234.19410181)
\curveto(612.98243611,234.20410086)(613.02243607,234.20410086)(613.06244062,234.19410181)
\curveto(613.10243599,234.19410087)(613.14243595,234.20410086)(613.18244062,234.22410181)
\curveto(613.26243583,234.24410082)(613.34243575,234.2591008)(613.42244062,234.26910181)
\curveto(613.50243559,234.28910077)(613.57743552,234.31410075)(613.64744062,234.34410181)
\curveto(613.98743511,234.48410058)(614.26243483,234.67910038)(614.47244062,234.92910181)
\curveto(614.68243441,235.17909988)(614.85743424,235.47409959)(614.99744062,235.81410181)
\curveto(615.04743405,235.93409913)(615.07743402,236.059099)(615.08744062,236.18910181)
\curveto(615.10743399,236.32909873)(615.13743396,236.46909859)(615.17744062,236.60910181)
}
}
{
\newrgbcolor{curcolor}{0.90196079 0.90196079 0.90196079}
\pscustom[linestyle=none,fillstyle=solid,fillcolor=curcolor]
{
\newpath
\moveto(545.29360762,243.97413844)
\lineto(560.29360762,243.97413844)
\lineto(560.29360762,228.97413844)
\lineto(545.29360762,228.97413844)
\closepath
}
}
{
\newrgbcolor{curcolor}{0 0 0}
\pscustom[linestyle=none,fillstyle=solid,fillcolor=curcolor]
{
\newpath
\moveto(565.28181562,220.90839625)
\lineto(570.18681562,220.90839625)
\lineto(571.47681562,220.90839625)
\curveto(571.58680774,220.90838555)(571.69680763,220.90838555)(571.80681562,220.90839625)
\curveto(571.91680741,220.91838554)(572.00680732,220.89838556)(572.07681562,220.84839625)
\curveto(572.10680722,220.82838563)(572.1318072,220.80338566)(572.15181562,220.77339625)
\curveto(572.17180716,220.74338572)(572.19180714,220.71338575)(572.21181562,220.68339625)
\curveto(572.2318071,220.61338585)(572.24180709,220.49838596)(572.24181562,220.33839625)
\curveto(572.24180709,220.18838627)(572.2318071,220.07338639)(572.21181562,219.99339625)
\curveto(572.17180716,219.85338661)(572.08680724,219.77338669)(571.95681562,219.75339625)
\curveto(571.8268075,219.74338672)(571.67180766,219.73838672)(571.49181562,219.73839625)
\lineto(569.99181562,219.73839625)
\lineto(567.47181562,219.73839625)
\lineto(566.90181562,219.73839625)
\curveto(566.69181264,219.74838671)(566.53681279,219.72338674)(566.43681562,219.66339625)
\curveto(566.33681299,219.60338686)(566.28181305,219.49838696)(566.27181562,219.34839625)
\lineto(566.27181562,218.88339625)
\lineto(566.27181562,217.35339625)
\curveto(566.27181306,217.24338922)(566.26681306,217.11338935)(566.25681562,216.96339625)
\curveto(566.25681307,216.81338965)(566.26681306,216.69338977)(566.28681562,216.60339625)
\curveto(566.31681301,216.48338998)(566.37681295,216.40339006)(566.46681562,216.36339625)
\curveto(566.50681282,216.34339012)(566.57681275,216.32339014)(566.67681562,216.30339625)
\lineto(566.82681562,216.30339625)
\curveto(566.86681246,216.29339017)(566.90681242,216.28839017)(566.94681562,216.28839625)
\curveto(566.99681233,216.29839016)(567.04681228,216.30339016)(567.09681562,216.30339625)
\lineto(567.60681562,216.30339625)
\lineto(570.54681562,216.30339625)
\lineto(570.84681562,216.30339625)
\curveto(570.95680837,216.31339015)(571.06680826,216.31339015)(571.17681562,216.30339625)
\curveto(571.29680803,216.30339016)(571.40180793,216.29339017)(571.49181562,216.27339625)
\curveto(571.59180774,216.2633902)(571.66680766,216.24339022)(571.71681562,216.21339625)
\curveto(571.74680758,216.19339027)(571.77180756,216.14839031)(571.79181562,216.07839625)
\curveto(571.81180752,216.00839045)(571.8268075,215.93339053)(571.83681562,215.85339625)
\curveto(571.84680748,215.77339069)(571.84680748,215.68839077)(571.83681562,215.59839625)
\curveto(571.83680749,215.51839094)(571.8268075,215.44839101)(571.80681562,215.38839625)
\curveto(571.78680754,215.29839116)(571.74180759,215.23339123)(571.67181562,215.19339625)
\curveto(571.65180768,215.17339129)(571.62180771,215.1583913)(571.58181562,215.14839625)
\curveto(571.55180778,215.14839131)(571.52180781,215.14339132)(571.49181562,215.13339625)
\lineto(571.40181562,215.13339625)
\curveto(571.35180798,215.12339134)(571.30180803,215.11839134)(571.25181562,215.11839625)
\curveto(571.20180813,215.12839133)(571.15180818,215.13339133)(571.10181562,215.13339625)
\lineto(570.54681562,215.13339625)
\lineto(567.38181562,215.13339625)
\lineto(567.02181562,215.13339625)
\curveto(566.91181242,215.14339132)(566.80681252,215.13839132)(566.70681562,215.11839625)
\curveto(566.60681272,215.10839135)(566.51681281,215.08339138)(566.43681562,215.04339625)
\curveto(566.36681296,215.00339146)(566.31681301,214.93339153)(566.28681562,214.83339625)
\curveto(566.26681306,214.77339169)(566.25681307,214.70339176)(566.25681562,214.62339625)
\curveto(566.26681306,214.54339192)(566.27181306,214.463392)(566.27181562,214.38339625)
\lineto(566.27181562,213.54339625)
\lineto(566.27181562,212.11839625)
\curveto(566.27181306,211.97839448)(566.27681305,211.84839461)(566.28681562,211.72839625)
\curveto(566.29681303,211.61839484)(566.33681299,211.53839492)(566.40681562,211.48839625)
\curveto(566.47681285,211.43839502)(566.55681277,211.40839505)(566.64681562,211.39839625)
\lineto(566.94681562,211.39839625)
\lineto(567.90681562,211.39839625)
\lineto(570.68181562,211.39839625)
\lineto(571.53681562,211.39839625)
\lineto(571.77681562,211.39839625)
\curveto(571.85680747,211.40839505)(571.9268074,211.40339506)(571.98681562,211.38339625)
\curveto(572.10680722,211.34339512)(572.18680714,211.28839517)(572.22681562,211.21839625)
\curveto(572.24680708,211.18839527)(572.26180707,211.13839532)(572.27181562,211.06839625)
\curveto(572.28180705,210.99839546)(572.28680704,210.92339554)(572.28681562,210.84339625)
\curveto(572.29680703,210.77339569)(572.29680703,210.69839576)(572.28681562,210.61839625)
\curveto(572.27680705,210.54839591)(572.26680706,210.49339597)(572.25681562,210.45339625)
\curveto(572.21680711,210.37339609)(572.17180716,210.31839614)(572.12181562,210.28839625)
\curveto(572.06180727,210.24839621)(571.98180735,210.22839623)(571.88181562,210.22839625)
\lineto(571.61181562,210.22839625)
\lineto(570.56181562,210.22839625)
\lineto(566.57181562,210.22839625)
\lineto(565.52181562,210.22839625)
\curveto(565.38181395,210.22839623)(565.26181407,210.23339623)(565.16181562,210.24339625)
\curveto(565.06181427,210.2633962)(564.98681434,210.31339615)(564.93681562,210.39339625)
\curveto(564.89681443,210.45339601)(564.87681445,210.52839593)(564.87681562,210.61839625)
\lineto(564.87681562,210.90339625)
\lineto(564.87681562,211.95339625)
\lineto(564.87681562,215.97339625)
\lineto(564.87681562,219.33339625)
\lineto(564.87681562,220.26339625)
\lineto(564.87681562,220.53339625)
\curveto(564.87681445,220.62338584)(564.89681443,220.69338577)(564.93681562,220.74339625)
\curveto(564.97681435,220.81338565)(565.05181428,220.8633856)(565.16181562,220.89339625)
\curveto(565.18181415,220.90338556)(565.20181413,220.90338556)(565.22181562,220.89339625)
\curveto(565.24181409,220.89338557)(565.26181407,220.89838556)(565.28181562,220.90839625)
}
}
{
\newrgbcolor{curcolor}{0 0 0}
\pscustom[linestyle=none,fillstyle=solid,fillcolor=curcolor]
{
\newpath
\moveto(576.2217375,218.13339625)
\curveto(576.94173343,218.14338832)(577.54673283,218.0583884)(578.0367375,217.87839625)
\curveto(578.52673185,217.70838875)(578.90673147,217.40338906)(579.1767375,216.96339625)
\curveto(579.24673113,216.85338961)(579.30173107,216.73838972)(579.3417375,216.61839625)
\curveto(579.38173099,216.50838995)(579.42173095,216.38339008)(579.4617375,216.24339625)
\curveto(579.48173089,216.17339029)(579.48673089,216.09839036)(579.4767375,216.01839625)
\curveto(579.46673091,215.94839051)(579.45173092,215.89339057)(579.4317375,215.85339625)
\curveto(579.41173096,215.83339063)(579.38673099,215.81339065)(579.3567375,215.79339625)
\curveto(579.32673105,215.78339068)(579.30173107,215.76839069)(579.2817375,215.74839625)
\curveto(579.23173114,215.72839073)(579.18173119,215.72339074)(579.1317375,215.73339625)
\curveto(579.08173129,215.74339072)(579.03173134,215.74339072)(578.9817375,215.73339625)
\curveto(578.90173147,215.71339075)(578.79673158,215.70839075)(578.6667375,215.71839625)
\curveto(578.53673184,215.73839072)(578.44673193,215.7633907)(578.3967375,215.79339625)
\curveto(578.31673206,215.84339062)(578.26173211,215.90839055)(578.2317375,215.98839625)
\curveto(578.21173216,216.07839038)(578.1767322,216.1633903)(578.1267375,216.24339625)
\curveto(578.03673234,216.40339006)(577.91173246,216.54838991)(577.7517375,216.67839625)
\curveto(577.64173273,216.7583897)(577.52173285,216.81838964)(577.3917375,216.85839625)
\curveto(577.26173311,216.89838956)(577.12173325,216.93838952)(576.9717375,216.97839625)
\curveto(576.92173345,216.99838946)(576.8717335,217.00338946)(576.8217375,216.99339625)
\curveto(576.7717336,216.99338947)(576.72173365,216.99838946)(576.6717375,217.00839625)
\curveto(576.61173376,217.02838943)(576.53673384,217.03838942)(576.4467375,217.03839625)
\curveto(576.35673402,217.03838942)(576.28173409,217.02838943)(576.2217375,217.00839625)
\lineto(576.1317375,217.00839625)
\lineto(575.9817375,216.97839625)
\curveto(575.93173444,216.97838948)(575.88173449,216.97338949)(575.8317375,216.96339625)
\curveto(575.5717348,216.90338956)(575.35673502,216.81838964)(575.1867375,216.70839625)
\curveto(575.01673536,216.59838986)(574.90173547,216.41339005)(574.8417375,216.15339625)
\curveto(574.82173555,216.08339038)(574.81673556,216.01339045)(574.8267375,215.94339625)
\curveto(574.84673553,215.87339059)(574.86673551,215.81339065)(574.8867375,215.76339625)
\curveto(574.94673543,215.61339085)(575.01673536,215.50339096)(575.0967375,215.43339625)
\curveto(575.18673519,215.37339109)(575.29673508,215.30339116)(575.4267375,215.22339625)
\curveto(575.58673479,215.12339134)(575.76673461,215.04839141)(575.9667375,214.99839625)
\curveto(576.16673421,214.9583915)(576.36673401,214.90839155)(576.5667375,214.84839625)
\curveto(576.69673368,214.80839165)(576.82673355,214.77839168)(576.9567375,214.75839625)
\curveto(577.08673329,214.73839172)(577.21673316,214.70839175)(577.3467375,214.66839625)
\curveto(577.55673282,214.60839185)(577.76173261,214.54839191)(577.9617375,214.48839625)
\curveto(578.16173221,214.43839202)(578.36173201,214.37339209)(578.5617375,214.29339625)
\lineto(578.7117375,214.23339625)
\curveto(578.76173161,214.21339225)(578.81173156,214.18839227)(578.8617375,214.15839625)
\curveto(579.06173131,214.03839242)(579.23673114,213.90339256)(579.3867375,213.75339625)
\curveto(579.53673084,213.60339286)(579.66173071,213.41339305)(579.7617375,213.18339625)
\curveto(579.78173059,213.11339335)(579.80173057,213.01839344)(579.8217375,212.89839625)
\curveto(579.84173053,212.82839363)(579.85173052,212.75339371)(579.8517375,212.67339625)
\curveto(579.86173051,212.60339386)(579.86673051,212.52339394)(579.8667375,212.43339625)
\lineto(579.8667375,212.28339625)
\curveto(579.84673053,212.21339425)(579.83673054,212.14339432)(579.8367375,212.07339625)
\curveto(579.83673054,212.00339446)(579.82673055,211.93339453)(579.8067375,211.86339625)
\curveto(579.7767306,211.75339471)(579.74173063,211.64839481)(579.7017375,211.54839625)
\curveto(579.66173071,211.44839501)(579.61673076,211.3583951)(579.5667375,211.27839625)
\curveto(579.40673097,211.01839544)(579.20173117,210.80839565)(578.9517375,210.64839625)
\curveto(578.70173167,210.49839596)(578.42173195,210.36839609)(578.1117375,210.25839625)
\curveto(578.02173235,210.22839623)(577.92673245,210.20839625)(577.8267375,210.19839625)
\curveto(577.73673264,210.17839628)(577.64673273,210.15339631)(577.5567375,210.12339625)
\curveto(577.45673292,210.10339636)(577.35673302,210.09339637)(577.2567375,210.09339625)
\curveto(577.15673322,210.09339637)(577.05673332,210.08339638)(576.9567375,210.06339625)
\lineto(576.8067375,210.06339625)
\curveto(576.75673362,210.05339641)(576.68673369,210.04839641)(576.5967375,210.04839625)
\curveto(576.50673387,210.04839641)(576.43673394,210.05339641)(576.3867375,210.06339625)
\lineto(576.2217375,210.06339625)
\curveto(576.16173421,210.08339638)(576.09673428,210.09339637)(576.0267375,210.09339625)
\curveto(575.95673442,210.08339638)(575.89673448,210.08839637)(575.8467375,210.10839625)
\curveto(575.79673458,210.11839634)(575.73173464,210.12339634)(575.6517375,210.12339625)
\lineto(575.4117375,210.18339625)
\curveto(575.34173503,210.19339627)(575.26673511,210.21339625)(575.1867375,210.24339625)
\curveto(574.8767355,210.34339612)(574.60673577,210.46839599)(574.3767375,210.61839625)
\curveto(574.14673623,210.76839569)(573.94673643,210.9633955)(573.7767375,211.20339625)
\curveto(573.68673669,211.33339513)(573.61173676,211.46839499)(573.5517375,211.60839625)
\curveto(573.49173688,211.74839471)(573.43673694,211.90339456)(573.3867375,212.07339625)
\curveto(573.36673701,212.13339433)(573.35673702,212.20339426)(573.3567375,212.28339625)
\curveto(573.36673701,212.37339409)(573.38173699,212.44339402)(573.4017375,212.49339625)
\curveto(573.43173694,212.53339393)(573.48173689,212.57339389)(573.5517375,212.61339625)
\curveto(573.60173677,212.63339383)(573.6717367,212.64339382)(573.7617375,212.64339625)
\curveto(573.85173652,212.65339381)(573.94173643,212.65339381)(574.0317375,212.64339625)
\curveto(574.12173625,212.63339383)(574.20673617,212.61839384)(574.2867375,212.59839625)
\curveto(574.376736,212.58839387)(574.43673594,212.57339389)(574.4667375,212.55339625)
\curveto(574.53673584,212.50339396)(574.58173579,212.42839403)(574.6017375,212.32839625)
\curveto(574.63173574,212.23839422)(574.66673571,212.15339431)(574.7067375,212.07339625)
\curveto(574.80673557,211.85339461)(574.94173543,211.68339478)(575.1117375,211.56339625)
\curveto(575.23173514,211.47339499)(575.36673501,211.40339506)(575.5167375,211.35339625)
\curveto(575.66673471,211.30339516)(575.82673455,211.25339521)(575.9967375,211.20339625)
\lineto(576.3117375,211.15839625)
\lineto(576.4017375,211.15839625)
\curveto(576.4717339,211.13839532)(576.56173381,211.12839533)(576.6717375,211.12839625)
\curveto(576.79173358,211.12839533)(576.89173348,211.13839532)(576.9717375,211.15839625)
\curveto(577.04173333,211.1583953)(577.09673328,211.1633953)(577.1367375,211.17339625)
\curveto(577.19673318,211.18339528)(577.25673312,211.18839527)(577.3167375,211.18839625)
\curveto(577.376733,211.19839526)(577.43173294,211.20839525)(577.4817375,211.21839625)
\curveto(577.7717326,211.29839516)(578.00173237,211.40339506)(578.1717375,211.53339625)
\curveto(578.34173203,211.6633948)(578.46173191,211.88339458)(578.5317375,212.19339625)
\curveto(578.55173182,212.24339422)(578.55673182,212.29839416)(578.5467375,212.35839625)
\curveto(578.53673184,212.41839404)(578.52673185,212.463394)(578.5167375,212.49339625)
\curveto(578.46673191,212.68339378)(578.39673198,212.82339364)(578.3067375,212.91339625)
\curveto(578.21673216,213.01339345)(578.10173227,213.10339336)(577.9617375,213.18339625)
\curveto(577.8717325,213.24339322)(577.7717326,213.29339317)(577.6617375,213.33339625)
\lineto(577.3317375,213.45339625)
\curveto(577.30173307,213.463393)(577.2717331,213.46839299)(577.2417375,213.46839625)
\curveto(577.22173315,213.46839299)(577.19673318,213.47839298)(577.1667375,213.49839625)
\curveto(576.82673355,213.60839285)(576.4717339,213.68839277)(576.1017375,213.73839625)
\curveto(575.74173463,213.79839266)(575.40173497,213.89339257)(575.0817375,214.02339625)
\curveto(574.98173539,214.0633924)(574.88673549,214.09839236)(574.7967375,214.12839625)
\curveto(574.70673567,214.1583923)(574.62173575,214.19839226)(574.5417375,214.24839625)
\curveto(574.35173602,214.3583921)(574.1767362,214.48339198)(574.0167375,214.62339625)
\curveto(573.85673652,214.7633917)(573.73173664,214.93839152)(573.6417375,215.14839625)
\curveto(573.61173676,215.21839124)(573.58673679,215.28839117)(573.5667375,215.35839625)
\curveto(573.55673682,215.42839103)(573.54173683,215.50339096)(573.5217375,215.58339625)
\curveto(573.49173688,215.70339076)(573.48173689,215.83839062)(573.4917375,215.98839625)
\curveto(573.50173687,216.14839031)(573.51673686,216.28339018)(573.5367375,216.39339625)
\curveto(573.55673682,216.44339002)(573.56673681,216.48338998)(573.5667375,216.51339625)
\curveto(573.5767368,216.55338991)(573.59173678,216.59338987)(573.6117375,216.63339625)
\curveto(573.70173667,216.8633896)(573.82173655,217.0633894)(573.9717375,217.23339625)
\curveto(574.13173624,217.40338906)(574.31173606,217.55338891)(574.5117375,217.68339625)
\curveto(574.66173571,217.77338869)(574.82673555,217.84338862)(575.0067375,217.89339625)
\curveto(575.18673519,217.95338851)(575.376735,218.00838845)(575.5767375,218.05839625)
\curveto(575.64673473,218.06838839)(575.71173466,218.07838838)(575.7717375,218.08839625)
\curveto(575.84173453,218.09838836)(575.91673446,218.10838835)(575.9967375,218.11839625)
\curveto(576.02673435,218.12838833)(576.06673431,218.12838833)(576.1167375,218.11839625)
\curveto(576.16673421,218.10838835)(576.20173417,218.11338835)(576.2217375,218.13339625)
}
}
{
\newrgbcolor{curcolor}{0 0 0}
\pscustom[linestyle=none,fillstyle=solid,fillcolor=curcolor]
{
\newpath
\moveto(582.2367375,220.29339625)
\curveto(582.38673549,220.29338617)(582.53673534,220.28838617)(582.6867375,220.27839625)
\curveto(582.83673504,220.27838618)(582.94173493,220.23838622)(583.0017375,220.15839625)
\curveto(583.05173482,220.09838636)(583.0767348,220.01338645)(583.0767375,219.90339625)
\curveto(583.08673479,219.80338666)(583.09173478,219.69838676)(583.0917375,219.58839625)
\lineto(583.0917375,218.71839625)
\curveto(583.09173478,218.63838782)(583.08673479,218.55338791)(583.0767375,218.46339625)
\curveto(583.0767348,218.38338808)(583.08673479,218.31338815)(583.1067375,218.25339625)
\curveto(583.14673473,218.11338835)(583.23673464,218.02338844)(583.3767375,217.98339625)
\curveto(583.42673445,217.97338849)(583.4717344,217.96838849)(583.5117375,217.96839625)
\lineto(583.6617375,217.96839625)
\lineto(584.0667375,217.96839625)
\curveto(584.22673365,217.97838848)(584.34173353,217.96838849)(584.4117375,217.93839625)
\curveto(584.50173337,217.87838858)(584.56173331,217.81838864)(584.5917375,217.75839625)
\curveto(584.61173326,217.71838874)(584.62173325,217.67338879)(584.6217375,217.62339625)
\lineto(584.6217375,217.47339625)
\curveto(584.62173325,217.3633891)(584.61673326,217.2583892)(584.6067375,217.15839625)
\curveto(584.59673328,217.06838939)(584.56173331,216.99838946)(584.5017375,216.94839625)
\curveto(584.44173343,216.89838956)(584.35673352,216.86838959)(584.2467375,216.85839625)
\lineto(583.9167375,216.85839625)
\curveto(583.80673407,216.86838959)(583.69673418,216.87338959)(583.5867375,216.87339625)
\curveto(583.4767344,216.87338959)(583.38173449,216.8583896)(583.3017375,216.82839625)
\curveto(583.23173464,216.79838966)(583.18173469,216.74838971)(583.1517375,216.67839625)
\curveto(583.12173475,216.60838985)(583.10173477,216.52338994)(583.0917375,216.42339625)
\curveto(583.08173479,216.33339013)(583.0767348,216.23339023)(583.0767375,216.12339625)
\curveto(583.08673479,216.02339044)(583.09173478,215.92339054)(583.0917375,215.82339625)
\lineto(583.0917375,212.85339625)
\curveto(583.09173478,212.63339383)(583.08673479,212.39839406)(583.0767375,212.14839625)
\curveto(583.0767348,211.90839455)(583.12173475,211.72339474)(583.2117375,211.59339625)
\curveto(583.26173461,211.51339495)(583.32673455,211.458395)(583.4067375,211.42839625)
\curveto(583.48673439,211.39839506)(583.58173429,211.37339509)(583.6917375,211.35339625)
\curveto(583.72173415,211.34339512)(583.75173412,211.33839512)(583.7817375,211.33839625)
\curveto(583.82173405,211.34839511)(583.85673402,211.34839511)(583.8867375,211.33839625)
\lineto(584.0817375,211.33839625)
\curveto(584.18173369,211.33839512)(584.2717336,211.32839513)(584.3517375,211.30839625)
\curveto(584.44173343,211.29839516)(584.50673337,211.2633952)(584.5467375,211.20339625)
\curveto(584.56673331,211.17339529)(584.58173329,211.11839534)(584.5917375,211.03839625)
\curveto(584.61173326,210.96839549)(584.62173325,210.89339557)(584.6217375,210.81339625)
\curveto(584.63173324,210.73339573)(584.63173324,210.65339581)(584.6217375,210.57339625)
\curveto(584.61173326,210.50339596)(584.59173328,210.44839601)(584.5617375,210.40839625)
\curveto(584.52173335,210.33839612)(584.44673343,210.28839617)(584.3367375,210.25839625)
\curveto(584.25673362,210.23839622)(584.16673371,210.22839623)(584.0667375,210.22839625)
\curveto(583.96673391,210.23839622)(583.876734,210.24339622)(583.7967375,210.24339625)
\curveto(583.73673414,210.24339622)(583.6767342,210.23839622)(583.6167375,210.22839625)
\curveto(583.55673432,210.22839623)(583.50173437,210.23339623)(583.4517375,210.24339625)
\lineto(583.2717375,210.24339625)
\curveto(583.22173465,210.25339621)(583.1717347,210.2583962)(583.1217375,210.25839625)
\curveto(583.08173479,210.26839619)(583.03673484,210.27339619)(582.9867375,210.27339625)
\curveto(582.78673509,210.32339614)(582.61173526,210.37839608)(582.4617375,210.43839625)
\curveto(582.32173555,210.49839596)(582.20173567,210.60339586)(582.1017375,210.75339625)
\curveto(581.96173591,210.95339551)(581.88173599,211.20339526)(581.8617375,211.50339625)
\curveto(581.84173603,211.81339465)(581.83173604,212.14339432)(581.8317375,212.49339625)
\lineto(581.8317375,216.42339625)
\curveto(581.80173607,216.55338991)(581.7717361,216.64838981)(581.7417375,216.70839625)
\curveto(581.72173615,216.76838969)(581.65173622,216.81838964)(581.5317375,216.85839625)
\curveto(581.49173638,216.86838959)(581.45173642,216.86838959)(581.4117375,216.85839625)
\curveto(581.3717365,216.84838961)(581.33173654,216.85338961)(581.2917375,216.87339625)
\lineto(581.0517375,216.87339625)
\curveto(580.92173695,216.87338959)(580.81173706,216.88338958)(580.7217375,216.90339625)
\curveto(580.64173723,216.93338953)(580.58673729,216.99338947)(580.5567375,217.08339625)
\curveto(580.53673734,217.12338934)(580.52173735,217.16838929)(580.5117375,217.21839625)
\lineto(580.5117375,217.36839625)
\curveto(580.51173736,217.50838895)(580.52173735,217.62338884)(580.5417375,217.71339625)
\curveto(580.56173731,217.81338865)(580.62173725,217.88838857)(580.7217375,217.93839625)
\curveto(580.83173704,217.97838848)(580.9717369,217.98838847)(581.1417375,217.96839625)
\curveto(581.32173655,217.94838851)(581.4717364,217.9583885)(581.5917375,217.99839625)
\curveto(581.68173619,218.04838841)(581.75173612,218.11838834)(581.8017375,218.20839625)
\curveto(581.82173605,218.26838819)(581.83173604,218.34338812)(581.8317375,218.43339625)
\lineto(581.8317375,218.68839625)
\lineto(581.8317375,219.61839625)
\lineto(581.8317375,219.85839625)
\curveto(581.83173604,219.94838651)(581.84173603,220.02338644)(581.8617375,220.08339625)
\curveto(581.90173597,220.1633863)(581.9767359,220.22838623)(582.0867375,220.27839625)
\curveto(582.11673576,220.27838618)(582.14173573,220.27838618)(582.1617375,220.27839625)
\curveto(582.19173568,220.28838617)(582.21673566,220.29338617)(582.2367375,220.29339625)
}
}
{
\newrgbcolor{curcolor}{0 0 0}
\pscustom[linestyle=none,fillstyle=solid,fillcolor=curcolor]
{
\newpath
\moveto(586.47353437,217.95339625)
\lineto(586.90853437,217.95339625)
\curveto(587.05853241,217.95338851)(587.1635323,217.91338855)(587.22353437,217.83339625)
\curveto(587.27353219,217.75338871)(587.29853217,217.65338881)(587.29853437,217.53339625)
\curveto(587.30853216,217.41338905)(587.31353215,217.29338917)(587.31353437,217.17339625)
\lineto(587.31353437,215.74839625)
\lineto(587.31353437,213.48339625)
\lineto(587.31353437,212.79339625)
\curveto(587.31353215,212.5633939)(587.33853213,212.3633941)(587.38853437,212.19339625)
\curveto(587.54853192,211.74339472)(587.84853162,211.42839503)(588.28853437,211.24839625)
\curveto(588.50853096,211.1583953)(588.77353069,211.12339534)(589.08353437,211.14339625)
\curveto(589.39353007,211.17339529)(589.64352982,211.22839523)(589.83353437,211.30839625)
\curveto(590.1635293,211.44839501)(590.42352904,211.62339484)(590.61353437,211.83339625)
\curveto(590.81352865,212.05339441)(590.9685285,212.33839412)(591.07853437,212.68839625)
\curveto(591.10852836,212.76839369)(591.12852834,212.84839361)(591.13853437,212.92839625)
\curveto(591.14852832,213.00839345)(591.1635283,213.09339337)(591.18353437,213.18339625)
\curveto(591.19352827,213.23339323)(591.19352827,213.27839318)(591.18353437,213.31839625)
\curveto(591.18352828,213.3583931)(591.19352827,213.40339306)(591.21353437,213.45339625)
\lineto(591.21353437,213.76839625)
\curveto(591.23352823,213.84839261)(591.23852823,213.93839252)(591.22853437,214.03839625)
\curveto(591.21852825,214.14839231)(591.21352825,214.24839221)(591.21353437,214.33839625)
\lineto(591.21353437,215.50839625)
\lineto(591.21353437,217.09839625)
\curveto(591.21352825,217.21838924)(591.20852826,217.34338912)(591.19853437,217.47339625)
\curveto(591.19852827,217.61338885)(591.22352824,217.72338874)(591.27353437,217.80339625)
\curveto(591.31352815,217.85338861)(591.35852811,217.88338858)(591.40853437,217.89339625)
\curveto(591.468528,217.91338855)(591.53852793,217.93338853)(591.61853437,217.95339625)
\lineto(591.84353437,217.95339625)
\curveto(591.9635275,217.95338851)(592.0685274,217.94838851)(592.15853437,217.93839625)
\curveto(592.25852721,217.92838853)(592.33352713,217.88338858)(592.38353437,217.80339625)
\curveto(592.43352703,217.75338871)(592.45852701,217.67838878)(592.45853437,217.57839625)
\lineto(592.45853437,217.29339625)
\lineto(592.45853437,216.27339625)
\lineto(592.45853437,212.23839625)
\lineto(592.45853437,210.88839625)
\curveto(592.45852701,210.76839569)(592.45352701,210.65339581)(592.44353437,210.54339625)
\curveto(592.44352702,210.44339602)(592.40852706,210.36839609)(592.33853437,210.31839625)
\curveto(592.29852717,210.28839617)(592.23852723,210.2633962)(592.15853437,210.24339625)
\curveto(592.07852739,210.23339623)(591.98852748,210.22339624)(591.88853437,210.21339625)
\curveto(591.79852767,210.21339625)(591.70852776,210.21839624)(591.61853437,210.22839625)
\curveto(591.53852793,210.23839622)(591.47852799,210.2583962)(591.43853437,210.28839625)
\curveto(591.38852808,210.32839613)(591.34352812,210.39339607)(591.30353437,210.48339625)
\curveto(591.29352817,210.52339594)(591.28352818,210.57839588)(591.27353437,210.64839625)
\curveto(591.27352819,210.71839574)(591.2685282,210.78339568)(591.25853437,210.84339625)
\curveto(591.24852822,210.91339555)(591.22852824,210.96839549)(591.19853437,211.00839625)
\curveto(591.1685283,211.04839541)(591.12352834,211.0633954)(591.06353437,211.05339625)
\curveto(590.98352848,211.03339543)(590.90352856,210.97339549)(590.82353437,210.87339625)
\curveto(590.74352872,210.78339568)(590.6685288,210.71339575)(590.59853437,210.66339625)
\curveto(590.37852909,210.50339596)(590.12852934,210.3633961)(589.84853437,210.24339625)
\curveto(589.73852973,210.19339627)(589.62352984,210.1633963)(589.50353437,210.15339625)
\curveto(589.39353007,210.13339633)(589.27853019,210.10839635)(589.15853437,210.07839625)
\curveto(589.10853036,210.06839639)(589.05353041,210.06839639)(588.99353437,210.07839625)
\curveto(588.94353052,210.08839637)(588.89353057,210.08339638)(588.84353437,210.06339625)
\curveto(588.74353072,210.04339642)(588.65353081,210.04339642)(588.57353437,210.06339625)
\lineto(588.42353437,210.06339625)
\curveto(588.37353109,210.08339638)(588.31353115,210.09339637)(588.24353437,210.09339625)
\curveto(588.18353128,210.09339637)(588.12853134,210.09839636)(588.07853437,210.10839625)
\curveto(588.03853143,210.12839633)(587.99853147,210.13839632)(587.95853437,210.13839625)
\curveto(587.92853154,210.12839633)(587.88853158,210.13339633)(587.83853437,210.15339625)
\lineto(587.59853437,210.21339625)
\curveto(587.52853194,210.23339623)(587.45353201,210.2633962)(587.37353437,210.30339625)
\curveto(587.11353235,210.41339605)(586.89353257,210.5583959)(586.71353437,210.73839625)
\curveto(586.54353292,210.92839553)(586.40353306,211.15339531)(586.29353437,211.41339625)
\curveto(586.25353321,211.50339496)(586.22353324,211.59339487)(586.20353437,211.68339625)
\lineto(586.14353437,211.98339625)
\curveto(586.12353334,212.04339442)(586.11353335,212.09839436)(586.11353437,212.14839625)
\curveto(586.12353334,212.20839425)(586.11853335,212.27339419)(586.09853437,212.34339625)
\curveto(586.08853338,212.3633941)(586.08353338,212.38839407)(586.08353437,212.41839625)
\curveto(586.08353338,212.458394)(586.07853339,212.49339397)(586.06853437,212.52339625)
\lineto(586.06853437,212.67339625)
\curveto(586.05853341,212.71339375)(586.05353341,212.7583937)(586.05353437,212.80839625)
\curveto(586.0635334,212.86839359)(586.0685334,212.92339354)(586.06853437,212.97339625)
\lineto(586.06853437,213.57339625)
\lineto(586.06853437,216.33339625)
\lineto(586.06853437,217.29339625)
\lineto(586.06853437,217.56339625)
\curveto(586.0685334,217.65338881)(586.08853338,217.72838873)(586.12853437,217.78839625)
\curveto(586.1685333,217.8583886)(586.24353322,217.90838855)(586.35353437,217.93839625)
\curveto(586.37353309,217.94838851)(586.39353307,217.94838851)(586.41353437,217.93839625)
\curveto(586.43353303,217.93838852)(586.45353301,217.94338852)(586.47353437,217.95339625)
}
}
{
\newrgbcolor{curcolor}{0 0 0}
\pscustom[linestyle=none,fillstyle=solid,fillcolor=curcolor]
{
\newpath
\moveto(601.31814375,211.03839625)
\lineto(601.31814375,210.64839625)
\curveto(601.31813587,210.52839593)(601.2931359,210.42839603)(601.24314375,210.34839625)
\curveto(601.193136,210.27839618)(601.10813608,210.23839622)(600.98814375,210.22839625)
\lineto(600.64314375,210.22839625)
\curveto(600.58313661,210.22839623)(600.52313667,210.22339624)(600.46314375,210.21339625)
\curveto(600.41313678,210.21339625)(600.36813682,210.22339624)(600.32814375,210.24339625)
\curveto(600.23813695,210.2633962)(600.17813701,210.30339616)(600.14814375,210.36339625)
\curveto(600.10813708,210.41339605)(600.08313711,210.47339599)(600.07314375,210.54339625)
\curveto(600.07313712,210.61339585)(600.05813713,210.68339578)(600.02814375,210.75339625)
\curveto(600.01813717,210.77339569)(600.00313719,210.78839567)(599.98314375,210.79839625)
\curveto(599.97313722,210.81839564)(599.95813723,210.83839562)(599.93814375,210.85839625)
\curveto(599.83813735,210.86839559)(599.75813743,210.84839561)(599.69814375,210.79839625)
\curveto(599.64813754,210.74839571)(599.5931376,210.69839576)(599.53314375,210.64839625)
\curveto(599.33313786,210.49839596)(599.13313806,210.38339608)(598.93314375,210.30339625)
\curveto(598.75313844,210.22339624)(598.54313865,210.1633963)(598.30314375,210.12339625)
\curveto(598.07313912,210.08339638)(597.83313936,210.0633964)(597.58314375,210.06339625)
\curveto(597.34313985,210.05339641)(597.10314009,210.06839639)(596.86314375,210.10839625)
\curveto(596.62314057,210.13839632)(596.41314078,210.19339627)(596.23314375,210.27339625)
\curveto(595.71314148,210.49339597)(595.2931419,210.78839567)(594.97314375,211.15839625)
\curveto(594.65314254,211.53839492)(594.40314279,212.00839445)(594.22314375,212.56839625)
\curveto(594.18314301,212.6583938)(594.15314304,212.74839371)(594.13314375,212.83839625)
\curveto(594.12314307,212.93839352)(594.10314309,213.03839342)(594.07314375,213.13839625)
\curveto(594.06314313,213.18839327)(594.05814313,213.23839322)(594.05814375,213.28839625)
\curveto(594.05814313,213.33839312)(594.05314314,213.38839307)(594.04314375,213.43839625)
\curveto(594.02314317,213.48839297)(594.01314318,213.53839292)(594.01314375,213.58839625)
\curveto(594.02314317,213.64839281)(594.02314317,213.70339276)(594.01314375,213.75339625)
\lineto(594.01314375,213.90339625)
\curveto(593.9931432,213.95339251)(593.98314321,214.01839244)(593.98314375,214.09839625)
\curveto(593.98314321,214.17839228)(593.9931432,214.24339222)(594.01314375,214.29339625)
\lineto(594.01314375,214.45839625)
\curveto(594.03314316,214.52839193)(594.03814315,214.59839186)(594.02814375,214.66839625)
\curveto(594.02814316,214.74839171)(594.03814315,214.82339164)(594.05814375,214.89339625)
\curveto(594.06814312,214.94339152)(594.07314312,214.98839147)(594.07314375,215.02839625)
\curveto(594.07314312,215.06839139)(594.07814311,215.11339135)(594.08814375,215.16339625)
\curveto(594.11814307,215.2633912)(594.14314305,215.3583911)(594.16314375,215.44839625)
\curveto(594.18314301,215.54839091)(594.20814298,215.64339082)(594.23814375,215.73339625)
\curveto(594.36814282,216.11339035)(594.53314266,216.45339001)(594.73314375,216.75339625)
\curveto(594.94314225,217.0633894)(595.193142,217.31838914)(595.48314375,217.51839625)
\curveto(595.65314154,217.63838882)(595.82814136,217.73838872)(596.00814375,217.81839625)
\curveto(596.19814099,217.89838856)(596.40314079,217.96838849)(596.62314375,218.02839625)
\curveto(596.6931405,218.03838842)(596.75814043,218.04838841)(596.81814375,218.05839625)
\curveto(596.8881403,218.06838839)(596.95814023,218.08338838)(597.02814375,218.10339625)
\lineto(597.17814375,218.10339625)
\curveto(597.25813993,218.12338834)(597.37313982,218.13338833)(597.52314375,218.13339625)
\curveto(597.68313951,218.13338833)(597.80313939,218.12338834)(597.88314375,218.10339625)
\curveto(597.92313927,218.09338837)(597.97813921,218.08838837)(598.04814375,218.08839625)
\curveto(598.15813903,218.0583884)(598.26813892,218.03338843)(598.37814375,218.01339625)
\curveto(598.4881387,218.00338846)(598.5931386,217.97338849)(598.69314375,217.92339625)
\curveto(598.84313835,217.8633886)(598.98313821,217.79838866)(599.11314375,217.72839625)
\curveto(599.25313794,217.6583888)(599.38313781,217.57838888)(599.50314375,217.48839625)
\curveto(599.56313763,217.43838902)(599.62313757,217.38338908)(599.68314375,217.32339625)
\curveto(599.75313744,217.27338919)(599.84313735,217.2583892)(599.95314375,217.27839625)
\curveto(599.97313722,217.30838915)(599.9881372,217.33338913)(599.99814375,217.35339625)
\curveto(600.01813717,217.37338909)(600.03313716,217.40338906)(600.04314375,217.44339625)
\curveto(600.07313712,217.53338893)(600.08313711,217.64838881)(600.07314375,217.78839625)
\lineto(600.07314375,218.16339625)
\lineto(600.07314375,219.88839625)
\lineto(600.07314375,220.35339625)
\curveto(600.07313712,220.53338593)(600.09813709,220.6633858)(600.14814375,220.74339625)
\curveto(600.188137,220.81338565)(600.24813694,220.8583856)(600.32814375,220.87839625)
\curveto(600.34813684,220.87838558)(600.37313682,220.87838558)(600.40314375,220.87839625)
\curveto(600.43313676,220.88838557)(600.45813673,220.89338557)(600.47814375,220.89339625)
\curveto(600.61813657,220.90338556)(600.76313643,220.90338556)(600.91314375,220.89339625)
\curveto(601.07313612,220.89338557)(601.18313601,220.85338561)(601.24314375,220.77339625)
\curveto(601.2931359,220.69338577)(601.31813587,220.59338587)(601.31814375,220.47339625)
\lineto(601.31814375,220.09839625)
\lineto(601.31814375,211.03839625)
\moveto(600.10314375,213.87339625)
\curveto(600.12313707,213.92339254)(600.13313706,213.98839247)(600.13314375,214.06839625)
\curveto(600.13313706,214.1583923)(600.12313707,214.22839223)(600.10314375,214.27839625)
\lineto(600.10314375,214.50339625)
\curveto(600.08313711,214.59339187)(600.06813712,214.68339178)(600.05814375,214.77339625)
\curveto(600.04813714,214.87339159)(600.02813716,214.9633915)(599.99814375,215.04339625)
\curveto(599.97813721,215.12339134)(599.95813723,215.19839126)(599.93814375,215.26839625)
\curveto(599.92813726,215.33839112)(599.90813728,215.40839105)(599.87814375,215.47839625)
\curveto(599.75813743,215.77839068)(599.60313759,216.04339042)(599.41314375,216.27339625)
\curveto(599.22313797,216.50338996)(598.98313821,216.68338978)(598.69314375,216.81339625)
\curveto(598.5931386,216.8633896)(598.4881387,216.89838956)(598.37814375,216.91839625)
\curveto(598.27813891,216.94838951)(598.16813902,216.97338949)(598.04814375,216.99339625)
\curveto(597.96813922,217.01338945)(597.87813931,217.02338944)(597.77814375,217.02339625)
\lineto(597.50814375,217.02339625)
\curveto(597.45813973,217.01338945)(597.41313978,217.00338946)(597.37314375,216.99339625)
\lineto(597.23814375,216.99339625)
\curveto(597.15814003,216.97338949)(597.07314012,216.95338951)(596.98314375,216.93339625)
\curveto(596.90314029,216.91338955)(596.82314037,216.88838957)(596.74314375,216.85839625)
\curveto(596.42314077,216.71838974)(596.16314103,216.51338995)(595.96314375,216.24339625)
\curveto(595.77314142,215.98339048)(595.61814157,215.67839078)(595.49814375,215.32839625)
\curveto(595.45814173,215.21839124)(595.42814176,215.10339136)(595.40814375,214.98339625)
\curveto(595.39814179,214.87339159)(595.38314181,214.7633917)(595.36314375,214.65339625)
\curveto(595.36314183,214.61339185)(595.35814183,214.57339189)(595.34814375,214.53339625)
\lineto(595.34814375,214.42839625)
\curveto(595.32814186,214.37839208)(595.31814187,214.32339214)(595.31814375,214.26339625)
\curveto(595.32814186,214.20339226)(595.33314186,214.14839231)(595.33314375,214.09839625)
\lineto(595.33314375,213.76839625)
\curveto(595.33314186,213.66839279)(595.34314185,213.57339289)(595.36314375,213.48339625)
\curveto(595.37314182,213.45339301)(595.37814181,213.40339306)(595.37814375,213.33339625)
\curveto(595.39814179,213.2633932)(595.41314178,213.19339327)(595.42314375,213.12339625)
\lineto(595.48314375,212.91339625)
\curveto(595.5931416,212.5633939)(595.74314145,212.2633942)(595.93314375,212.01339625)
\curveto(596.12314107,211.7633947)(596.36314083,211.5583949)(596.65314375,211.39839625)
\curveto(596.74314045,211.34839511)(596.83314036,211.30839515)(596.92314375,211.27839625)
\curveto(597.01314018,211.24839521)(597.11314008,211.21839524)(597.22314375,211.18839625)
\curveto(597.27313992,211.16839529)(597.32313987,211.1633953)(597.37314375,211.17339625)
\curveto(597.43313976,211.18339528)(597.4881397,211.17839528)(597.53814375,211.15839625)
\curveto(597.57813961,211.14839531)(597.61813957,211.14339532)(597.65814375,211.14339625)
\lineto(597.79314375,211.14339625)
\lineto(597.92814375,211.14339625)
\curveto(597.95813923,211.15339531)(598.00813918,211.1583953)(598.07814375,211.15839625)
\curveto(598.15813903,211.17839528)(598.23813895,211.19339527)(598.31814375,211.20339625)
\curveto(598.39813879,211.22339524)(598.47313872,211.24839521)(598.54314375,211.27839625)
\curveto(598.87313832,211.41839504)(599.13813805,211.59339487)(599.33814375,211.80339625)
\curveto(599.54813764,212.02339444)(599.72313747,212.29839416)(599.86314375,212.62839625)
\curveto(599.91313728,212.73839372)(599.94813724,212.84839361)(599.96814375,212.95839625)
\curveto(599.9881372,213.06839339)(600.01313718,213.17839328)(600.04314375,213.28839625)
\curveto(600.06313713,213.32839313)(600.07313712,213.3633931)(600.07314375,213.39339625)
\curveto(600.07313712,213.43339303)(600.07813711,213.47339299)(600.08814375,213.51339625)
\curveto(600.09813709,213.57339289)(600.09813709,213.63339283)(600.08814375,213.69339625)
\curveto(600.0881371,213.75339271)(600.0931371,213.81339265)(600.10314375,213.87339625)
}
}
{
\newrgbcolor{curcolor}{0 0 0}
\pscustom[linestyle=none,fillstyle=solid,fillcolor=curcolor]
{
\newpath
\moveto(603.54939375,219.45339625)
\curveto(603.46939263,219.51338695)(603.42439267,219.61838684)(603.41439375,219.76839625)
\lineto(603.41439375,220.23339625)
\lineto(603.41439375,220.48839625)
\curveto(603.41439268,220.57838588)(603.42939267,220.65338581)(603.45939375,220.71339625)
\curveto(603.4993926,220.79338567)(603.57939252,220.85338561)(603.69939375,220.89339625)
\curveto(603.71939238,220.90338556)(603.73939236,220.90338556)(603.75939375,220.89339625)
\curveto(603.78939231,220.89338557)(603.81439228,220.89838556)(603.83439375,220.90839625)
\curveto(604.00439209,220.90838555)(604.16439193,220.90338556)(604.31439375,220.89339625)
\curveto(604.46439163,220.88338558)(604.56439153,220.82338564)(604.61439375,220.71339625)
\curveto(604.64439145,220.65338581)(604.65939144,220.57838588)(604.65939375,220.48839625)
\lineto(604.65939375,220.23339625)
\curveto(604.65939144,220.05338641)(604.65439144,219.88338658)(604.64439375,219.72339625)
\curveto(604.64439145,219.5633869)(604.57939152,219.458387)(604.44939375,219.40839625)
\curveto(604.3993917,219.38838707)(604.34439175,219.37838708)(604.28439375,219.37839625)
\lineto(604.11939375,219.37839625)
\lineto(603.80439375,219.37839625)
\curveto(603.70439239,219.37838708)(603.61939248,219.40338706)(603.54939375,219.45339625)
\moveto(604.65939375,210.94839625)
\lineto(604.65939375,210.63339625)
\curveto(604.66939143,210.53339593)(604.64939145,210.45339601)(604.59939375,210.39339625)
\curveto(604.56939153,210.33339613)(604.52439157,210.29339617)(604.46439375,210.27339625)
\curveto(604.40439169,210.2633962)(604.33439176,210.24839621)(604.25439375,210.22839625)
\lineto(604.02939375,210.22839625)
\curveto(603.8993922,210.22839623)(603.78439231,210.23339623)(603.68439375,210.24339625)
\curveto(603.5943925,210.2633962)(603.52439257,210.31339615)(603.47439375,210.39339625)
\curveto(603.43439266,210.45339601)(603.41439268,210.52839593)(603.41439375,210.61839625)
\lineto(603.41439375,210.90339625)
\lineto(603.41439375,217.24839625)
\lineto(603.41439375,217.56339625)
\curveto(603.41439268,217.67338879)(603.43939266,217.7583887)(603.48939375,217.81839625)
\curveto(603.51939258,217.86838859)(603.55939254,217.89838856)(603.60939375,217.90839625)
\curveto(603.65939244,217.91838854)(603.71439238,217.93338853)(603.77439375,217.95339625)
\curveto(603.7943923,217.95338851)(603.81439228,217.94838851)(603.83439375,217.93839625)
\curveto(603.86439223,217.93838852)(603.88939221,217.94338852)(603.90939375,217.95339625)
\curveto(604.03939206,217.95338851)(604.16939193,217.94838851)(604.29939375,217.93839625)
\curveto(604.43939166,217.93838852)(604.53439156,217.89838856)(604.58439375,217.81839625)
\curveto(604.63439146,217.7583887)(604.65939144,217.67838878)(604.65939375,217.57839625)
\lineto(604.65939375,217.29339625)
\lineto(604.65939375,210.94839625)
}
}
{
\newrgbcolor{curcolor}{0 0 0}
\pscustom[linestyle=none,fillstyle=solid,fillcolor=curcolor]
{
\newpath
\moveto(613.4892375,210.78339625)
\curveto(613.51922967,210.62339584)(613.50422968,210.48839597)(613.4442375,210.37839625)
\curveto(613.3842298,210.27839618)(613.30422988,210.20339626)(613.2042375,210.15339625)
\curveto(613.15423003,210.13339633)(613.09923009,210.12339634)(613.0392375,210.12339625)
\curveto(612.9892302,210.12339634)(612.93423025,210.11339635)(612.8742375,210.09339625)
\curveto(612.65423053,210.04339642)(612.43423075,210.0583964)(612.2142375,210.13839625)
\curveto(612.00423118,210.20839625)(611.85923133,210.29839616)(611.7792375,210.40839625)
\curveto(611.72923146,210.47839598)(611.6842315,210.5583959)(611.6442375,210.64839625)
\curveto(611.60423158,210.74839571)(611.55423163,210.82839563)(611.4942375,210.88839625)
\curveto(611.47423171,210.90839555)(611.44923174,210.92839553)(611.4192375,210.94839625)
\curveto(611.39923179,210.96839549)(611.36923182,210.97339549)(611.3292375,210.96339625)
\curveto(611.21923197,210.93339553)(611.11423207,210.87839558)(611.0142375,210.79839625)
\curveto(610.92423226,210.71839574)(610.83423235,210.64839581)(610.7442375,210.58839625)
\curveto(610.61423257,210.50839595)(610.47423271,210.43339603)(610.3242375,210.36339625)
\curveto(610.17423301,210.30339616)(610.01423317,210.24839621)(609.8442375,210.19839625)
\curveto(609.74423344,210.16839629)(609.63423355,210.14839631)(609.5142375,210.13839625)
\curveto(609.40423378,210.12839633)(609.29423389,210.11339635)(609.1842375,210.09339625)
\curveto(609.13423405,210.08339638)(609.0892341,210.07839638)(609.0492375,210.07839625)
\lineto(608.9442375,210.07839625)
\curveto(608.83423435,210.0583964)(608.72923446,210.0583964)(608.6292375,210.07839625)
\lineto(608.4942375,210.07839625)
\curveto(608.44423474,210.08839637)(608.39423479,210.09339637)(608.3442375,210.09339625)
\curveto(608.29423489,210.09339637)(608.24923494,210.10339636)(608.2092375,210.12339625)
\curveto(608.16923502,210.13339633)(608.13423505,210.13839632)(608.1042375,210.13839625)
\curveto(608.0842351,210.12839633)(608.05923513,210.12839633)(608.0292375,210.13839625)
\lineto(607.7892375,210.19839625)
\curveto(607.70923548,210.20839625)(607.63423555,210.22839623)(607.5642375,210.25839625)
\curveto(607.26423592,210.38839607)(607.01923617,210.53339593)(606.8292375,210.69339625)
\curveto(606.64923654,210.8633956)(606.49923669,211.09839536)(606.3792375,211.39839625)
\curveto(606.2892369,211.61839484)(606.24423694,211.88339458)(606.2442375,212.19339625)
\lineto(606.2442375,212.50839625)
\curveto(606.25423693,212.5583939)(606.25923693,212.60839385)(606.2592375,212.65839625)
\lineto(606.2892375,212.83839625)
\lineto(606.4092375,213.16839625)
\curveto(606.44923674,213.27839318)(606.49923669,213.37839308)(606.5592375,213.46839625)
\curveto(606.73923645,213.7583927)(606.9842362,213.97339249)(607.2942375,214.11339625)
\curveto(607.60423558,214.25339221)(607.94423524,214.37839208)(608.3142375,214.48839625)
\curveto(608.45423473,214.52839193)(608.59923459,214.5583919)(608.7492375,214.57839625)
\curveto(608.89923429,214.59839186)(609.04923414,214.62339184)(609.1992375,214.65339625)
\curveto(609.26923392,214.67339179)(609.33423385,214.68339178)(609.3942375,214.68339625)
\curveto(609.46423372,214.68339178)(609.53923365,214.69339177)(609.6192375,214.71339625)
\curveto(609.6892335,214.73339173)(609.75923343,214.74339172)(609.8292375,214.74339625)
\curveto(609.89923329,214.75339171)(609.97423321,214.76839169)(610.0542375,214.78839625)
\curveto(610.30423288,214.84839161)(610.53923265,214.89839156)(610.7592375,214.93839625)
\curveto(610.97923221,214.98839147)(611.15423203,215.10339136)(611.2842375,215.28339625)
\curveto(611.34423184,215.3633911)(611.39423179,215.463391)(611.4342375,215.58339625)
\curveto(611.47423171,215.71339075)(611.47423171,215.85339061)(611.4342375,216.00339625)
\curveto(611.37423181,216.24339022)(611.2842319,216.43339003)(611.1642375,216.57339625)
\curveto(611.05423213,216.71338975)(610.89423229,216.82338964)(610.6842375,216.90339625)
\curveto(610.56423262,216.95338951)(610.41923277,216.98838947)(610.2492375,217.00839625)
\curveto(610.0892331,217.02838943)(609.91923327,217.03838942)(609.7392375,217.03839625)
\curveto(609.55923363,217.03838942)(609.3842338,217.02838943)(609.2142375,217.00839625)
\curveto(609.04423414,216.98838947)(608.89923429,216.9583895)(608.7792375,216.91839625)
\curveto(608.60923458,216.8583896)(608.44423474,216.77338969)(608.2842375,216.66339625)
\curveto(608.20423498,216.60338986)(608.12923506,216.52338994)(608.0592375,216.42339625)
\curveto(607.99923519,216.33339013)(607.94423524,216.23339023)(607.8942375,216.12339625)
\curveto(607.86423532,216.04339042)(607.83423535,215.9583905)(607.8042375,215.86839625)
\curveto(607.7842354,215.77839068)(607.73923545,215.70839075)(607.6692375,215.65839625)
\curveto(607.62923556,215.62839083)(607.55923563,215.60339086)(607.4592375,215.58339625)
\curveto(607.36923582,215.57339089)(607.27423591,215.56839089)(607.1742375,215.56839625)
\curveto(607.07423611,215.56839089)(606.97423621,215.57339089)(606.8742375,215.58339625)
\curveto(606.7842364,215.60339086)(606.71923647,215.62839083)(606.6792375,215.65839625)
\curveto(606.63923655,215.68839077)(606.60923658,215.73839072)(606.5892375,215.80839625)
\curveto(606.56923662,215.87839058)(606.56923662,215.95339051)(606.5892375,216.03339625)
\curveto(606.61923657,216.1633903)(606.64923654,216.28339018)(606.6792375,216.39339625)
\curveto(606.71923647,216.51338995)(606.76423642,216.62838983)(606.8142375,216.73839625)
\curveto(607.00423618,217.08838937)(607.24423594,217.3583891)(607.5342375,217.54839625)
\curveto(607.82423536,217.74838871)(608.184235,217.90838855)(608.6142375,218.02839625)
\curveto(608.71423447,218.04838841)(608.81423437,218.0633884)(608.9142375,218.07339625)
\curveto(609.02423416,218.08338838)(609.13423405,218.09838836)(609.2442375,218.11839625)
\curveto(609.2842339,218.12838833)(609.34923384,218.12838833)(609.4392375,218.11839625)
\curveto(609.52923366,218.11838834)(609.5842336,218.12838833)(609.6042375,218.14839625)
\curveto(610.30423288,218.1583883)(610.91423227,218.07838838)(611.4342375,217.90839625)
\curveto(611.95423123,217.73838872)(612.31923087,217.41338905)(612.5292375,216.93339625)
\curveto(612.61923057,216.73338973)(612.66923052,216.49838996)(612.6792375,216.22839625)
\curveto(612.69923049,215.96839049)(612.70923048,215.69339077)(612.7092375,215.40339625)
\lineto(612.7092375,212.08839625)
\curveto(612.70923048,211.94839451)(612.71423047,211.81339465)(612.7242375,211.68339625)
\curveto(612.73423045,211.55339491)(612.76423042,211.44839501)(612.8142375,211.36839625)
\curveto(612.86423032,211.29839516)(612.92923026,211.24839521)(613.0092375,211.21839625)
\curveto(613.09923009,211.17839528)(613.18423,211.14839531)(613.2642375,211.12839625)
\curveto(613.34422984,211.11839534)(613.40422978,211.07339539)(613.4442375,210.99339625)
\curveto(613.46422972,210.9633955)(613.47422971,210.93339553)(613.4742375,210.90339625)
\curveto(613.47422971,210.87339559)(613.47922971,210.83339563)(613.4892375,210.78339625)
\moveto(611.3442375,212.44839625)
\curveto(611.40423178,212.58839387)(611.43423175,212.74839371)(611.4342375,212.92839625)
\curveto(611.44423174,213.11839334)(611.44923174,213.31339315)(611.4492375,213.51339625)
\curveto(611.44923174,213.62339284)(611.44423174,213.72339274)(611.4342375,213.81339625)
\curveto(611.42423176,213.90339256)(611.3842318,213.97339249)(611.3142375,214.02339625)
\curveto(611.2842319,214.04339242)(611.21423197,214.05339241)(611.1042375,214.05339625)
\curveto(611.0842321,214.03339243)(611.04923214,214.02339244)(610.9992375,214.02339625)
\curveto(610.94923224,214.02339244)(610.90423228,214.01339245)(610.8642375,213.99339625)
\curveto(610.7842324,213.97339249)(610.69423249,213.95339251)(610.5942375,213.93339625)
\lineto(610.2942375,213.87339625)
\curveto(610.26423292,213.87339259)(610.22923296,213.86839259)(610.1892375,213.85839625)
\lineto(610.0842375,213.85839625)
\curveto(609.93423325,213.81839264)(609.76923342,213.79339267)(609.5892375,213.78339625)
\curveto(609.41923377,213.78339268)(609.25923393,213.7633927)(609.1092375,213.72339625)
\curveto(609.02923416,213.70339276)(608.95423423,213.68339278)(608.8842375,213.66339625)
\curveto(608.82423436,213.65339281)(608.75423443,213.63839282)(608.6742375,213.61839625)
\curveto(608.51423467,213.56839289)(608.36423482,213.50339296)(608.2242375,213.42339625)
\curveto(608.0842351,213.35339311)(607.96423522,213.2633932)(607.8642375,213.15339625)
\curveto(607.76423542,213.04339342)(607.6892355,212.90839355)(607.6392375,212.74839625)
\curveto(607.5892356,212.59839386)(607.56923562,212.41339405)(607.5792375,212.19339625)
\curveto(607.57923561,212.09339437)(607.59423559,211.99839446)(607.6242375,211.90839625)
\curveto(607.66423552,211.82839463)(607.70923548,211.75339471)(607.7592375,211.68339625)
\curveto(607.83923535,211.57339489)(607.94423524,211.47839498)(608.0742375,211.39839625)
\curveto(608.20423498,211.32839513)(608.34423484,211.26839519)(608.4942375,211.21839625)
\curveto(608.54423464,211.20839525)(608.59423459,211.20339526)(608.6442375,211.20339625)
\curveto(608.69423449,211.20339526)(608.74423444,211.19839526)(608.7942375,211.18839625)
\curveto(608.86423432,211.16839529)(608.94923424,211.15339531)(609.0492375,211.14339625)
\curveto(609.15923403,211.14339532)(609.24923394,211.15339531)(609.3192375,211.17339625)
\curveto(609.37923381,211.19339527)(609.43923375,211.19839526)(609.4992375,211.18839625)
\curveto(609.55923363,211.18839527)(609.61923357,211.19839526)(609.6792375,211.21839625)
\curveto(609.75923343,211.23839522)(609.83423335,211.25339521)(609.9042375,211.26339625)
\curveto(609.9842332,211.27339519)(610.05923313,211.29339517)(610.1292375,211.32339625)
\curveto(610.41923277,211.44339502)(610.66423252,211.58839487)(610.8642375,211.75839625)
\curveto(611.07423211,211.92839453)(611.23423195,212.1583943)(611.3442375,212.44839625)
}
}
{
\newrgbcolor{curcolor}{0 0 0}
\pscustom[linestyle=none,fillstyle=solid,fillcolor=curcolor]
{
\newpath
\moveto(618.35087812,218.10339625)
\curveto(618.98087289,218.12338834)(619.48587238,218.03838842)(619.86587812,217.84839625)
\curveto(620.24587162,217.6583888)(620.55087132,217.37338909)(620.78087812,216.99339625)
\curveto(620.84087103,216.89338957)(620.88587098,216.78338968)(620.91587812,216.66339625)
\curveto(620.95587091,216.55338991)(620.99087088,216.43839002)(621.02087812,216.31839625)
\curveto(621.0708708,216.12839033)(621.10087077,215.92339054)(621.11087812,215.70339625)
\curveto(621.12087075,215.48339098)(621.12587074,215.2583912)(621.12587812,215.02839625)
\lineto(621.12587812,213.42339625)
\lineto(621.12587812,211.08339625)
\curveto(621.12587074,210.91339555)(621.12087075,210.74339572)(621.11087812,210.57339625)
\curveto(621.11087076,210.40339606)(621.04587082,210.29339617)(620.91587812,210.24339625)
\curveto(620.865871,210.22339624)(620.81087106,210.21339625)(620.75087812,210.21339625)
\curveto(620.70087117,210.20339626)(620.64587122,210.19839626)(620.58587812,210.19839625)
\curveto(620.45587141,210.19839626)(620.33087154,210.20339626)(620.21087812,210.21339625)
\curveto(620.09087178,210.21339625)(620.00587186,210.25339621)(619.95587812,210.33339625)
\curveto(619.90587196,210.40339606)(619.88087199,210.49339597)(619.88087812,210.60339625)
\lineto(619.88087812,210.93339625)
\lineto(619.88087812,212.22339625)
\lineto(619.88087812,214.66839625)
\curveto(619.88087199,214.93839152)(619.87587199,215.20339126)(619.86587812,215.46339625)
\curveto(619.85587201,215.73339073)(619.81087206,215.9633905)(619.73087812,216.15339625)
\curveto(619.65087222,216.35339011)(619.53087234,216.51338995)(619.37087812,216.63339625)
\curveto(619.21087266,216.7633897)(619.02587284,216.8633896)(618.81587812,216.93339625)
\curveto(618.75587311,216.95338951)(618.69087318,216.9633895)(618.62087812,216.96339625)
\curveto(618.56087331,216.97338949)(618.50087337,216.98838947)(618.44087812,217.00839625)
\curveto(618.39087348,217.01838944)(618.31087356,217.01838944)(618.20087812,217.00839625)
\curveto(618.10087377,217.00838945)(618.03087384,217.00338946)(617.99087812,216.99339625)
\curveto(617.95087392,216.97338949)(617.91587395,216.9633895)(617.88587812,216.96339625)
\curveto(617.85587401,216.97338949)(617.82087405,216.97338949)(617.78087812,216.96339625)
\curveto(617.65087422,216.93338953)(617.52587434,216.89838956)(617.40587812,216.85839625)
\curveto(617.29587457,216.82838963)(617.19087468,216.78338968)(617.09087812,216.72339625)
\curveto(617.05087482,216.70338976)(617.01587485,216.68338978)(616.98587812,216.66339625)
\curveto(616.95587491,216.64338982)(616.92087495,216.62338984)(616.88087812,216.60339625)
\curveto(616.53087534,216.35339011)(616.27587559,215.97839048)(616.11587812,215.47839625)
\curveto(616.08587578,215.39839106)(616.0658758,215.31339115)(616.05587812,215.22339625)
\curveto(616.04587582,215.14339132)(616.03087584,215.0633914)(616.01087812,214.98339625)
\curveto(615.99087588,214.93339153)(615.98587588,214.88339158)(615.99587812,214.83339625)
\curveto(616.00587586,214.79339167)(616.00087587,214.75339171)(615.98087812,214.71339625)
\lineto(615.98087812,214.39839625)
\curveto(615.9708759,214.36839209)(615.9658759,214.33339213)(615.96587812,214.29339625)
\curveto(615.97587589,214.25339221)(615.98087589,214.20839225)(615.98087812,214.15839625)
\lineto(615.98087812,213.70839625)
\lineto(615.98087812,212.26839625)
\lineto(615.98087812,210.94839625)
\lineto(615.98087812,210.60339625)
\curveto(615.98087589,210.49339597)(615.95587591,210.40339606)(615.90587812,210.33339625)
\curveto(615.85587601,210.25339621)(615.7658761,210.21339625)(615.63587812,210.21339625)
\curveto(615.51587635,210.20339626)(615.39087648,210.19839626)(615.26087812,210.19839625)
\curveto(615.18087669,210.19839626)(615.10587676,210.20339626)(615.03587812,210.21339625)
\curveto(614.9658769,210.22339624)(614.90587696,210.24839621)(614.85587812,210.28839625)
\curveto(614.77587709,210.33839612)(614.73587713,210.43339603)(614.73587812,210.57339625)
\lineto(614.73587812,210.97839625)
\lineto(614.73587812,212.74839625)
\lineto(614.73587812,216.37839625)
\lineto(614.73587812,217.29339625)
\lineto(614.73587812,217.56339625)
\curveto(614.73587713,217.65338881)(614.75587711,217.72338874)(614.79587812,217.77339625)
\curveto(614.82587704,217.83338863)(614.87587699,217.87338859)(614.94587812,217.89339625)
\curveto(614.98587688,217.90338856)(615.04087683,217.91338855)(615.11087812,217.92339625)
\curveto(615.19087668,217.93338853)(615.2708766,217.93838852)(615.35087812,217.93839625)
\curveto(615.43087644,217.93838852)(615.50587636,217.93338853)(615.57587812,217.92339625)
\curveto(615.65587621,217.91338855)(615.71087616,217.89838856)(615.74087812,217.87839625)
\curveto(615.85087602,217.80838865)(615.90087597,217.71838874)(615.89087812,217.60839625)
\curveto(615.88087599,217.50838895)(615.89587597,217.39338907)(615.93587812,217.26339625)
\curveto(615.95587591,217.20338926)(615.99587587,217.15338931)(616.05587812,217.11339625)
\curveto(616.17587569,217.10338936)(616.2708756,217.14838931)(616.34087812,217.24839625)
\curveto(616.42087545,217.34838911)(616.50087537,217.42838903)(616.58087812,217.48839625)
\curveto(616.72087515,217.58838887)(616.86087501,217.67838878)(617.00087812,217.75839625)
\curveto(617.15087472,217.84838861)(617.32087455,217.92338854)(617.51087812,217.98339625)
\curveto(617.59087428,218.01338845)(617.67587419,218.03338843)(617.76587812,218.04339625)
\curveto(617.865874,218.05338841)(617.96087391,218.06838839)(618.05087812,218.08839625)
\curveto(618.10087377,218.09838836)(618.15087372,218.10338836)(618.20087812,218.10339625)
\lineto(618.35087812,218.10339625)
}
}
{
\newrgbcolor{curcolor}{0 0 0}
\pscustom[linestyle=none,fillstyle=solid,fillcolor=curcolor]
{
\newpath
\moveto(623.9554875,220.29339625)
\curveto(624.10548549,220.29338617)(624.25548534,220.28838617)(624.4054875,220.27839625)
\curveto(624.55548504,220.27838618)(624.66048493,220.23838622)(624.7204875,220.15839625)
\curveto(624.77048482,220.09838636)(624.7954848,220.01338645)(624.7954875,219.90339625)
\curveto(624.80548479,219.80338666)(624.81048478,219.69838676)(624.8104875,219.58839625)
\lineto(624.8104875,218.71839625)
\curveto(624.81048478,218.63838782)(624.80548479,218.55338791)(624.7954875,218.46339625)
\curveto(624.7954848,218.38338808)(624.80548479,218.31338815)(624.8254875,218.25339625)
\curveto(624.86548473,218.11338835)(624.95548464,218.02338844)(625.0954875,217.98339625)
\curveto(625.14548445,217.97338849)(625.1904844,217.96838849)(625.2304875,217.96839625)
\lineto(625.3804875,217.96839625)
\lineto(625.7854875,217.96839625)
\curveto(625.94548365,217.97838848)(626.06048353,217.96838849)(626.1304875,217.93839625)
\curveto(626.22048337,217.87838858)(626.28048331,217.81838864)(626.3104875,217.75839625)
\curveto(626.33048326,217.71838874)(626.34048325,217.67338879)(626.3404875,217.62339625)
\lineto(626.3404875,217.47339625)
\curveto(626.34048325,217.3633891)(626.33548326,217.2583892)(626.3254875,217.15839625)
\curveto(626.31548328,217.06838939)(626.28048331,216.99838946)(626.2204875,216.94839625)
\curveto(626.16048343,216.89838956)(626.07548352,216.86838959)(625.9654875,216.85839625)
\lineto(625.6354875,216.85839625)
\curveto(625.52548407,216.86838959)(625.41548418,216.87338959)(625.3054875,216.87339625)
\curveto(625.1954844,216.87338959)(625.10048449,216.8583896)(625.0204875,216.82839625)
\curveto(624.95048464,216.79838966)(624.90048469,216.74838971)(624.8704875,216.67839625)
\curveto(624.84048475,216.60838985)(624.82048477,216.52338994)(624.8104875,216.42339625)
\curveto(624.80048479,216.33339013)(624.7954848,216.23339023)(624.7954875,216.12339625)
\curveto(624.80548479,216.02339044)(624.81048478,215.92339054)(624.8104875,215.82339625)
\lineto(624.8104875,212.85339625)
\curveto(624.81048478,212.63339383)(624.80548479,212.39839406)(624.7954875,212.14839625)
\curveto(624.7954848,211.90839455)(624.84048475,211.72339474)(624.9304875,211.59339625)
\curveto(624.98048461,211.51339495)(625.04548455,211.458395)(625.1254875,211.42839625)
\curveto(625.20548439,211.39839506)(625.30048429,211.37339509)(625.4104875,211.35339625)
\curveto(625.44048415,211.34339512)(625.47048412,211.33839512)(625.5004875,211.33839625)
\curveto(625.54048405,211.34839511)(625.57548402,211.34839511)(625.6054875,211.33839625)
\lineto(625.8004875,211.33839625)
\curveto(625.90048369,211.33839512)(625.9904836,211.32839513)(626.0704875,211.30839625)
\curveto(626.16048343,211.29839516)(626.22548337,211.2633952)(626.2654875,211.20339625)
\curveto(626.28548331,211.17339529)(626.30048329,211.11839534)(626.3104875,211.03839625)
\curveto(626.33048326,210.96839549)(626.34048325,210.89339557)(626.3404875,210.81339625)
\curveto(626.35048324,210.73339573)(626.35048324,210.65339581)(626.3404875,210.57339625)
\curveto(626.33048326,210.50339596)(626.31048328,210.44839601)(626.2804875,210.40839625)
\curveto(626.24048335,210.33839612)(626.16548343,210.28839617)(626.0554875,210.25839625)
\curveto(625.97548362,210.23839622)(625.88548371,210.22839623)(625.7854875,210.22839625)
\curveto(625.68548391,210.23839622)(625.595484,210.24339622)(625.5154875,210.24339625)
\curveto(625.45548414,210.24339622)(625.3954842,210.23839622)(625.3354875,210.22839625)
\curveto(625.27548432,210.22839623)(625.22048437,210.23339623)(625.1704875,210.24339625)
\lineto(624.9904875,210.24339625)
\curveto(624.94048465,210.25339621)(624.8904847,210.2583962)(624.8404875,210.25839625)
\curveto(624.80048479,210.26839619)(624.75548484,210.27339619)(624.7054875,210.27339625)
\curveto(624.50548509,210.32339614)(624.33048526,210.37839608)(624.1804875,210.43839625)
\curveto(624.04048555,210.49839596)(623.92048567,210.60339586)(623.8204875,210.75339625)
\curveto(623.68048591,210.95339551)(623.60048599,211.20339526)(623.5804875,211.50339625)
\curveto(623.56048603,211.81339465)(623.55048604,212.14339432)(623.5504875,212.49339625)
\lineto(623.5504875,216.42339625)
\curveto(623.52048607,216.55338991)(623.4904861,216.64838981)(623.4604875,216.70839625)
\curveto(623.44048615,216.76838969)(623.37048622,216.81838964)(623.2504875,216.85839625)
\curveto(623.21048638,216.86838959)(623.17048642,216.86838959)(623.1304875,216.85839625)
\curveto(623.0904865,216.84838961)(623.05048654,216.85338961)(623.0104875,216.87339625)
\lineto(622.7704875,216.87339625)
\curveto(622.64048695,216.87338959)(622.53048706,216.88338958)(622.4404875,216.90339625)
\curveto(622.36048723,216.93338953)(622.30548729,216.99338947)(622.2754875,217.08339625)
\curveto(622.25548734,217.12338934)(622.24048735,217.16838929)(622.2304875,217.21839625)
\lineto(622.2304875,217.36839625)
\curveto(622.23048736,217.50838895)(622.24048735,217.62338884)(622.2604875,217.71339625)
\curveto(622.28048731,217.81338865)(622.34048725,217.88838857)(622.4404875,217.93839625)
\curveto(622.55048704,217.97838848)(622.6904869,217.98838847)(622.8604875,217.96839625)
\curveto(623.04048655,217.94838851)(623.1904864,217.9583885)(623.3104875,217.99839625)
\curveto(623.40048619,218.04838841)(623.47048612,218.11838834)(623.5204875,218.20839625)
\curveto(623.54048605,218.26838819)(623.55048604,218.34338812)(623.5504875,218.43339625)
\lineto(623.5504875,218.68839625)
\lineto(623.5504875,219.61839625)
\lineto(623.5504875,219.85839625)
\curveto(623.55048604,219.94838651)(623.56048603,220.02338644)(623.5804875,220.08339625)
\curveto(623.62048597,220.1633863)(623.6954859,220.22838623)(623.8054875,220.27839625)
\curveto(623.83548576,220.27838618)(623.86048573,220.27838618)(623.8804875,220.27839625)
\curveto(623.91048568,220.28838617)(623.93548566,220.29338617)(623.9554875,220.29339625)
}
}
{
\newrgbcolor{curcolor}{0 0 0}
\pscustom[linestyle=none,fillstyle=solid,fillcolor=curcolor]
{
\newpath
\moveto(634.47728437,214.39839625)
\curveto(634.49727669,214.29839216)(634.49727669,214.18339228)(634.47728437,214.05339625)
\curveto(634.46727672,213.93339253)(634.43727675,213.84839261)(634.38728437,213.79839625)
\curveto(634.33727685,213.7583927)(634.26227692,213.72839273)(634.16228437,213.70839625)
\curveto(634.07227711,213.69839276)(633.96727722,213.69339277)(633.84728437,213.69339625)
\lineto(633.48728437,213.69339625)
\curveto(633.36727782,213.70339276)(633.26227792,213.70839275)(633.17228437,213.70839625)
\lineto(629.33228437,213.70839625)
\curveto(629.25228193,213.70839275)(629.17228201,213.70339276)(629.09228437,213.69339625)
\curveto(629.01228217,213.69339277)(628.94728224,213.67839278)(628.89728437,213.64839625)
\curveto(628.85728233,213.62839283)(628.81728237,213.58839287)(628.77728437,213.52839625)
\curveto(628.75728243,213.49839296)(628.73728245,213.45339301)(628.71728437,213.39339625)
\curveto(628.69728249,213.34339312)(628.69728249,213.29339317)(628.71728437,213.24339625)
\curveto(628.72728246,213.19339327)(628.73228245,213.14839331)(628.73228437,213.10839625)
\curveto(628.73228245,213.06839339)(628.73728245,213.02839343)(628.74728437,212.98839625)
\curveto(628.76728242,212.90839355)(628.7872824,212.82339364)(628.80728437,212.73339625)
\curveto(628.82728236,212.65339381)(628.85728233,212.57339389)(628.89728437,212.49339625)
\curveto(629.12728206,211.95339451)(629.50728168,211.56839489)(630.03728437,211.33839625)
\curveto(630.09728109,211.30839515)(630.16228102,211.28339518)(630.23228437,211.26339625)
\lineto(630.44228437,211.20339625)
\curveto(630.47228071,211.19339527)(630.52228066,211.18839527)(630.59228437,211.18839625)
\curveto(630.73228045,211.14839531)(630.91728027,211.12839533)(631.14728437,211.12839625)
\curveto(631.37727981,211.12839533)(631.56227962,211.14839531)(631.70228437,211.18839625)
\curveto(631.84227934,211.22839523)(631.96727922,211.26839519)(632.07728437,211.30839625)
\curveto(632.19727899,211.3583951)(632.30727888,211.41839504)(632.40728437,211.48839625)
\curveto(632.51727867,211.5583949)(632.61227857,211.63839482)(632.69228437,211.72839625)
\curveto(632.77227841,211.82839463)(632.84227834,211.93339453)(632.90228437,212.04339625)
\curveto(632.96227822,212.14339432)(633.01227817,212.24839421)(633.05228437,212.35839625)
\curveto(633.10227808,212.46839399)(633.182278,212.54839391)(633.29228437,212.59839625)
\curveto(633.33227785,212.61839384)(633.39727779,212.63339383)(633.48728437,212.64339625)
\curveto(633.57727761,212.65339381)(633.66727752,212.65339381)(633.75728437,212.64339625)
\curveto(633.84727734,212.64339382)(633.93227725,212.63839382)(634.01228437,212.62839625)
\curveto(634.09227709,212.61839384)(634.14727704,212.59839386)(634.17728437,212.56839625)
\curveto(634.27727691,212.49839396)(634.30227688,212.38339408)(634.25228437,212.22339625)
\curveto(634.17227701,211.95339451)(634.06727712,211.71339475)(633.93728437,211.50339625)
\curveto(633.73727745,211.18339528)(633.50727768,210.91839554)(633.24728437,210.70839625)
\curveto(632.99727819,210.50839595)(632.67727851,210.34339612)(632.28728437,210.21339625)
\curveto(632.187279,210.17339629)(632.0872791,210.14839631)(631.98728437,210.13839625)
\curveto(631.8872793,210.11839634)(631.7822794,210.09839636)(631.67228437,210.07839625)
\curveto(631.62227956,210.06839639)(631.57227961,210.0633964)(631.52228437,210.06339625)
\curveto(631.4822797,210.0633964)(631.43727975,210.0583964)(631.38728437,210.04839625)
\lineto(631.23728437,210.04839625)
\curveto(631.18728,210.03839642)(631.12728006,210.03339643)(631.05728437,210.03339625)
\curveto(630.99728019,210.03339643)(630.94728024,210.03839642)(630.90728437,210.04839625)
\lineto(630.77228437,210.04839625)
\curveto(630.72228046,210.0583964)(630.67728051,210.0633964)(630.63728437,210.06339625)
\curveto(630.59728059,210.0633964)(630.55728063,210.06839639)(630.51728437,210.07839625)
\curveto(630.46728072,210.08839637)(630.41228077,210.09839636)(630.35228437,210.10839625)
\curveto(630.29228089,210.10839635)(630.23728095,210.11339635)(630.18728437,210.12339625)
\curveto(630.09728109,210.14339632)(630.00728118,210.16839629)(629.91728437,210.19839625)
\curveto(629.82728136,210.21839624)(629.74228144,210.24339622)(629.66228437,210.27339625)
\curveto(629.62228156,210.29339617)(629.5872816,210.30339616)(629.55728437,210.30339625)
\curveto(629.52728166,210.31339615)(629.49228169,210.32839613)(629.45228437,210.34839625)
\curveto(629.30228188,210.41839604)(629.14228204,210.50339596)(628.97228437,210.60339625)
\curveto(628.6822825,210.79339567)(628.43228275,211.02339544)(628.22228437,211.29339625)
\curveto(628.02228316,211.57339489)(627.85228333,211.88339458)(627.71228437,212.22339625)
\curveto(627.66228352,212.33339413)(627.62228356,212.44839401)(627.59228437,212.56839625)
\curveto(627.57228361,212.68839377)(627.54228364,212.80839365)(627.50228437,212.92839625)
\curveto(627.49228369,212.96839349)(627.4872837,213.00339346)(627.48728437,213.03339625)
\curveto(627.4872837,213.0633934)(627.4822837,213.10339336)(627.47228437,213.15339625)
\curveto(627.45228373,213.23339323)(627.43728375,213.31839314)(627.42728437,213.40839625)
\curveto(627.41728377,213.49839296)(627.40228378,213.58839287)(627.38228437,213.67839625)
\lineto(627.38228437,213.88839625)
\curveto(627.37228381,213.92839253)(627.36228382,213.98339248)(627.35228437,214.05339625)
\curveto(627.35228383,214.13339233)(627.35728383,214.19839226)(627.36728437,214.24839625)
\lineto(627.36728437,214.41339625)
\curveto(627.3872838,214.463392)(627.39228379,214.51339195)(627.38228437,214.56339625)
\curveto(627.3822838,214.62339184)(627.3872838,214.67839178)(627.39728437,214.72839625)
\curveto(627.43728375,214.88839157)(627.46728372,215.04839141)(627.48728437,215.20839625)
\curveto(627.51728367,215.36839109)(627.56228362,215.51839094)(627.62228437,215.65839625)
\curveto(627.67228351,215.76839069)(627.71728347,215.87839058)(627.75728437,215.98839625)
\curveto(627.80728338,216.10839035)(627.86228332,216.22339024)(627.92228437,216.33339625)
\curveto(628.14228304,216.68338978)(628.39228279,216.98338948)(628.67228437,217.23339625)
\curveto(628.95228223,217.49338897)(629.29728189,217.70838875)(629.70728437,217.87839625)
\curveto(629.82728136,217.92838853)(629.94728124,217.9633885)(630.06728437,217.98339625)
\curveto(630.19728099,218.01338845)(630.33228085,218.04338842)(630.47228437,218.07339625)
\curveto(630.52228066,218.08338838)(630.56728062,218.08838837)(630.60728437,218.08839625)
\curveto(630.64728054,218.09838836)(630.69228049,218.10338836)(630.74228437,218.10339625)
\curveto(630.76228042,218.11338835)(630.7872804,218.11338835)(630.81728437,218.10339625)
\curveto(630.84728034,218.09338837)(630.87228031,218.09838836)(630.89228437,218.11839625)
\curveto(631.31227987,218.12838833)(631.67727951,218.08338838)(631.98728437,217.98339625)
\curveto(632.29727889,217.89338857)(632.57727861,217.76838869)(632.82728437,217.60839625)
\curveto(632.87727831,217.58838887)(632.91727827,217.5583889)(632.94728437,217.51839625)
\curveto(632.97727821,217.48838897)(633.01227817,217.463389)(633.05228437,217.44339625)
\curveto(633.13227805,217.38338908)(633.21227797,217.31338915)(633.29228437,217.23339625)
\curveto(633.3822778,217.15338931)(633.45727773,217.07338939)(633.51728437,216.99339625)
\curveto(633.67727751,216.78338968)(633.81227737,216.58338988)(633.92228437,216.39339625)
\curveto(633.99227719,216.28339018)(634.04727714,216.1633903)(634.08728437,216.03339625)
\curveto(634.12727706,215.90339056)(634.17227701,215.77339069)(634.22228437,215.64339625)
\curveto(634.27227691,215.51339095)(634.30727688,215.37839108)(634.32728437,215.23839625)
\curveto(634.35727683,215.09839136)(634.39227679,214.9583915)(634.43228437,214.81839625)
\curveto(634.44227674,214.74839171)(634.44727674,214.67839178)(634.44728437,214.60839625)
\lineto(634.47728437,214.39839625)
\moveto(633.02228437,214.90839625)
\curveto(633.05227813,214.94839151)(633.07727811,214.99839146)(633.09728437,215.05839625)
\curveto(633.11727807,215.12839133)(633.11727807,215.19839126)(633.09728437,215.26839625)
\curveto(633.03727815,215.48839097)(632.95227823,215.69339077)(632.84228437,215.88339625)
\curveto(632.70227848,216.11339035)(632.54727864,216.30839015)(632.37728437,216.46839625)
\curveto(632.20727898,216.62838983)(631.9872792,216.7633897)(631.71728437,216.87339625)
\curveto(631.64727954,216.89338957)(631.57727961,216.90838955)(631.50728437,216.91839625)
\curveto(631.43727975,216.93838952)(631.36227982,216.9583895)(631.28228437,216.97839625)
\curveto(631.20227998,216.99838946)(631.11728007,217.00838945)(631.02728437,217.00839625)
\lineto(630.77228437,217.00839625)
\curveto(630.74228044,216.98838947)(630.70728048,216.97838948)(630.66728437,216.97839625)
\curveto(630.62728056,216.98838947)(630.59228059,216.98838947)(630.56228437,216.97839625)
\lineto(630.32228437,216.91839625)
\curveto(630.25228093,216.90838955)(630.182281,216.89338957)(630.11228437,216.87339625)
\curveto(629.82228136,216.75338971)(629.5872816,216.60338986)(629.40728437,216.42339625)
\curveto(629.23728195,216.24339022)(629.0822821,216.01839044)(628.94228437,215.74839625)
\curveto(628.91228227,215.69839076)(628.8822823,215.63339083)(628.85228437,215.55339625)
\curveto(628.82228236,215.48339098)(628.79728239,215.40339106)(628.77728437,215.31339625)
\curveto(628.75728243,215.22339124)(628.75228243,215.13839132)(628.76228437,215.05839625)
\curveto(628.77228241,214.97839148)(628.80728238,214.91839154)(628.86728437,214.87839625)
\curveto(628.94728224,214.81839164)(629.0822821,214.78839167)(629.27228437,214.78839625)
\curveto(629.47228171,214.79839166)(629.64228154,214.80339166)(629.78228437,214.80339625)
\lineto(632.06228437,214.80339625)
\curveto(632.21227897,214.80339166)(632.39227879,214.79839166)(632.60228437,214.78839625)
\curveto(632.81227837,214.78839167)(632.95227823,214.82839163)(633.02228437,214.90839625)
}
}
{
\newrgbcolor{curcolor}{0.80000001 0.80000001 0.80000001}
\pscustom[linestyle=none,fillstyle=solid,fillcolor=curcolor]
{
\newpath
\moveto(545.29360762,220.93843287)
\lineto(560.29360762,220.93843287)
\lineto(560.29360762,205.93843287)
\lineto(545.29360762,205.93843287)
\closepath
}
}
{
\newrgbcolor{curcolor}{0 0 0}
\pscustom[linestyle=none,fillstyle=solid,fillcolor=curcolor]
{
\newpath
\moveto(573.23181562,188.15121851)
\curveto(573.25180608,188.10121777)(573.27680605,188.04121783)(573.30681562,187.97121851)
\curveto(573.33680599,187.90121797)(573.35680597,187.82621804)(573.36681562,187.74621851)
\curveto(573.38680594,187.67621819)(573.38680594,187.60621826)(573.36681562,187.53621851)
\curveto(573.35680597,187.47621839)(573.31680601,187.43121844)(573.24681562,187.40121851)
\curveto(573.19680613,187.38121849)(573.13680619,187.3712185)(573.06681562,187.37121851)
\lineto(572.85681562,187.37121851)
\lineto(572.40681562,187.37121851)
\curveto(572.25680707,187.3712185)(572.13680719,187.39621847)(572.04681562,187.44621851)
\curveto(571.94680738,187.50621836)(571.87180746,187.61121826)(571.82181562,187.76121851)
\curveto(571.78180755,187.91121796)(571.73680759,188.04621782)(571.68681562,188.16621851)
\curveto(571.57680775,188.42621744)(571.47680785,188.69621717)(571.38681562,188.97621851)
\curveto(571.29680803,189.25621661)(571.19680813,189.53121634)(571.08681562,189.80121851)
\curveto(571.05680827,189.89121598)(571.0268083,189.97621589)(570.99681562,190.05621851)
\curveto(570.97680835,190.13621573)(570.94680838,190.21121566)(570.90681562,190.28121851)
\curveto(570.87680845,190.35121552)(570.8318085,190.41121546)(570.77181562,190.46121851)
\curveto(570.71180862,190.51121536)(570.6318087,190.55121532)(570.53181562,190.58121851)
\curveto(570.48180885,190.60121527)(570.42180891,190.60621526)(570.35181562,190.59621851)
\lineto(570.15681562,190.59621851)
\lineto(567.32181562,190.59621851)
\lineto(567.02181562,190.59621851)
\curveto(566.91181242,190.60621526)(566.80681252,190.60621526)(566.70681562,190.59621851)
\curveto(566.60681272,190.58621528)(566.51181282,190.5712153)(566.42181562,190.55121851)
\curveto(566.34181299,190.53121534)(566.28181305,190.49121538)(566.24181562,190.43121851)
\curveto(566.16181317,190.33121554)(566.10181323,190.21621565)(566.06181562,190.08621851)
\curveto(566.0318133,189.9662159)(565.99181334,189.84121603)(565.94181562,189.71121851)
\curveto(565.84181349,189.48121639)(565.74681358,189.24121663)(565.65681562,188.99121851)
\curveto(565.57681375,188.74121713)(565.48681384,188.50121737)(565.38681562,188.27121851)
\curveto(565.36681396,188.21121766)(565.34181399,188.14121773)(565.31181562,188.06121851)
\curveto(565.29181404,187.99121788)(565.26681406,187.91621795)(565.23681562,187.83621851)
\curveto(565.20681412,187.75621811)(565.17181416,187.68121819)(565.13181562,187.61121851)
\curveto(565.10181423,187.55121832)(565.06681426,187.50621836)(565.02681562,187.47621851)
\curveto(564.94681438,187.41621845)(564.83681449,187.38121849)(564.69681562,187.37121851)
\lineto(564.27681562,187.37121851)
\lineto(564.03681562,187.37121851)
\curveto(563.96681536,187.38121849)(563.90681542,187.40621846)(563.85681562,187.44621851)
\curveto(563.80681552,187.47621839)(563.77681555,187.52121835)(563.76681562,187.58121851)
\curveto(563.76681556,187.64121823)(563.77181556,187.70121817)(563.78181562,187.76121851)
\curveto(563.80181553,187.83121804)(563.82181551,187.89621797)(563.84181562,187.95621851)
\curveto(563.87181546,188.02621784)(563.89681543,188.07621779)(563.91681562,188.10621851)
\curveto(564.05681527,188.42621744)(564.18181515,188.74121713)(564.29181562,189.05121851)
\curveto(564.40181493,189.3712165)(564.52181481,189.69121618)(564.65181562,190.01121851)
\curveto(564.74181459,190.23121564)(564.8268145,190.44621542)(564.90681562,190.65621851)
\curveto(564.98681434,190.87621499)(565.07181426,191.09621477)(565.16181562,191.31621851)
\curveto(565.46181387,192.03621383)(565.74681358,192.76121311)(566.01681562,193.49121851)
\curveto(566.28681304,194.23121164)(566.57181276,194.9662109)(566.87181562,195.69621851)
\curveto(566.98181235,195.95620991)(567.08181225,196.22120965)(567.17181562,196.49121851)
\curveto(567.27181206,196.76120911)(567.37681195,197.02620884)(567.48681562,197.28621851)
\curveto(567.53681179,197.39620847)(567.58181175,197.51620835)(567.62181562,197.64621851)
\curveto(567.67181166,197.78620808)(567.74181159,197.88620798)(567.83181562,197.94621851)
\curveto(567.87181146,197.98620788)(567.93681139,198.01620785)(568.02681562,198.03621851)
\curveto(568.04681128,198.04620782)(568.06681126,198.04620782)(568.08681562,198.03621851)
\curveto(568.11681121,198.03620783)(568.14181119,198.04120783)(568.16181562,198.05121851)
\curveto(568.34181099,198.05120782)(568.55181078,198.05120782)(568.79181562,198.05121851)
\curveto(569.0318103,198.06120781)(569.20681012,198.02620784)(569.31681562,197.94621851)
\curveto(569.39680993,197.88620798)(569.45680987,197.78620808)(569.49681562,197.64621851)
\curveto(569.54680978,197.51620835)(569.59680973,197.39620847)(569.64681562,197.28621851)
\curveto(569.74680958,197.05620881)(569.83680949,196.82620904)(569.91681562,196.59621851)
\curveto(569.99680933,196.3662095)(570.08680924,196.13620973)(570.18681562,195.90621851)
\curveto(570.26680906,195.70621016)(570.34180899,195.50121037)(570.41181562,195.29121851)
\curveto(570.49180884,195.08121079)(570.57680875,194.87621099)(570.66681562,194.67621851)
\curveto(570.96680836,193.94621192)(571.25180808,193.20621266)(571.52181562,192.45621851)
\curveto(571.80180753,191.71621415)(572.09680723,190.98121489)(572.40681562,190.25121851)
\curveto(572.44680688,190.16121571)(572.47680685,190.07621579)(572.49681562,189.99621851)
\curveto(572.5268068,189.91621595)(572.55680677,189.83121604)(572.58681562,189.74121851)
\curveto(572.69680663,189.48121639)(572.80180653,189.21621665)(572.90181562,188.94621851)
\curveto(573.01180632,188.67621719)(573.12180621,188.41121746)(573.23181562,188.15121851)
\moveto(570.02181562,191.79621851)
\curveto(570.11180922,191.82621404)(570.16680916,191.87621399)(570.18681562,191.94621851)
\curveto(570.21680911,192.01621385)(570.22180911,192.09121378)(570.20181562,192.17121851)
\curveto(570.19180914,192.26121361)(570.16680916,192.34621352)(570.12681562,192.42621851)
\curveto(570.09680923,192.51621335)(570.06680926,192.59121328)(570.03681562,192.65121851)
\curveto(570.01680931,192.69121318)(570.00680932,192.72621314)(570.00681562,192.75621851)
\curveto(570.00680932,192.78621308)(569.99680933,192.82121305)(569.97681562,192.86121851)
\lineto(569.88681562,193.10121851)
\curveto(569.86680946,193.19121268)(569.83680949,193.28121259)(569.79681562,193.37121851)
\curveto(569.64680968,193.73121214)(569.51180982,194.09621177)(569.39181562,194.46621851)
\curveto(569.28181005,194.84621102)(569.15181018,195.21621065)(569.00181562,195.57621851)
\curveto(568.95181038,195.68621018)(568.90681042,195.79621007)(568.86681562,195.90621851)
\curveto(568.83681049,196.01620985)(568.79681053,196.12120975)(568.74681562,196.22121851)
\curveto(568.7268106,196.2712096)(568.70181063,196.31620955)(568.67181562,196.35621851)
\curveto(568.65181068,196.40620946)(568.60181073,196.43120944)(568.52181562,196.43121851)
\curveto(568.50181083,196.41120946)(568.48181085,196.39620947)(568.46181562,196.38621851)
\curveto(568.44181089,196.37620949)(568.42181091,196.36120951)(568.40181562,196.34121851)
\curveto(568.36181097,196.29120958)(568.331811,196.23620963)(568.31181562,196.17621851)
\curveto(568.29181104,196.12620974)(568.27181106,196.0712098)(568.25181562,196.01121851)
\curveto(568.20181113,195.90120997)(568.16181117,195.79121008)(568.13181562,195.68121851)
\curveto(568.10181123,195.5712103)(568.06181127,195.46121041)(568.01181562,195.35121851)
\curveto(567.84181149,194.96121091)(567.69181164,194.5662113)(567.56181562,194.16621851)
\curveto(567.44181189,193.7662121)(567.30181203,193.37621249)(567.14181562,192.99621851)
\lineto(567.08181562,192.84621851)
\curveto(567.07181226,192.79621307)(567.05681227,192.74621312)(567.03681562,192.69621851)
\lineto(566.94681562,192.45621851)
\curveto(566.91681241,192.37621349)(566.89181244,192.29621357)(566.87181562,192.21621851)
\curveto(566.85181248,192.1662137)(566.84181249,192.11121376)(566.84181562,192.05121851)
\curveto(566.85181248,191.99121388)(566.86681246,191.94121393)(566.88681562,191.90121851)
\curveto(566.93681239,191.82121405)(567.04181229,191.77621409)(567.20181562,191.76621851)
\lineto(567.65181562,191.76621851)
\lineto(569.25681562,191.76621851)
\curveto(569.36680996,191.7662141)(569.50180983,191.76121411)(569.66181562,191.75121851)
\curveto(569.82180951,191.75121412)(569.94180939,191.7662141)(570.02181562,191.79621851)
}
}
{
\newrgbcolor{curcolor}{0 0 0}
\pscustom[linestyle=none,fillstyle=solid,fillcolor=curcolor]
{
\newpath
\moveto(574.81337812,195.09621851)
\lineto(575.24837812,195.09621851)
\curveto(575.39837616,195.09621077)(575.50337605,195.05621081)(575.56337812,194.97621851)
\curveto(575.61337594,194.89621097)(575.63837592,194.79621107)(575.63837812,194.67621851)
\curveto(575.64837591,194.55621131)(575.6533759,194.43621143)(575.65337812,194.31621851)
\lineto(575.65337812,192.89121851)
\lineto(575.65337812,190.62621851)
\lineto(575.65337812,189.93621851)
\curveto(575.6533759,189.70621616)(575.67837588,189.50621636)(575.72837812,189.33621851)
\curveto(575.88837567,188.88621698)(576.18837537,188.5712173)(576.62837812,188.39121851)
\curveto(576.84837471,188.30121757)(577.11337444,188.2662176)(577.42337812,188.28621851)
\curveto(577.73337382,188.31621755)(577.98337357,188.3712175)(578.17337812,188.45121851)
\curveto(578.50337305,188.59121728)(578.76337279,188.7662171)(578.95337812,188.97621851)
\curveto(579.1533724,189.19621667)(579.30837225,189.48121639)(579.41837812,189.83121851)
\curveto(579.44837211,189.91121596)(579.46837209,189.99121588)(579.47837812,190.07121851)
\curveto(579.48837207,190.15121572)(579.50337205,190.23621563)(579.52337812,190.32621851)
\curveto(579.53337202,190.37621549)(579.53337202,190.42121545)(579.52337812,190.46121851)
\curveto(579.52337203,190.50121537)(579.53337202,190.54621532)(579.55337812,190.59621851)
\lineto(579.55337812,190.91121851)
\curveto(579.57337198,190.99121488)(579.57837198,191.08121479)(579.56837812,191.18121851)
\curveto(579.558372,191.29121458)(579.553372,191.39121448)(579.55337812,191.48121851)
\lineto(579.55337812,192.65121851)
\lineto(579.55337812,194.24121851)
\curveto(579.553372,194.36121151)(579.54837201,194.48621138)(579.53837812,194.61621851)
\curveto(579.53837202,194.75621111)(579.56337199,194.866211)(579.61337812,194.94621851)
\curveto(579.6533719,194.99621087)(579.69837186,195.02621084)(579.74837812,195.03621851)
\curveto(579.80837175,195.05621081)(579.87837168,195.07621079)(579.95837812,195.09621851)
\lineto(580.18337812,195.09621851)
\curveto(580.30337125,195.09621077)(580.40837115,195.09121078)(580.49837812,195.08121851)
\curveto(580.59837096,195.0712108)(580.67337088,195.02621084)(580.72337812,194.94621851)
\curveto(580.77337078,194.89621097)(580.79837076,194.82121105)(580.79837812,194.72121851)
\lineto(580.79837812,194.43621851)
\lineto(580.79837812,193.41621851)
\lineto(580.79837812,189.38121851)
\lineto(580.79837812,188.03121851)
\curveto(580.79837076,187.91121796)(580.79337076,187.79621807)(580.78337812,187.68621851)
\curveto(580.78337077,187.58621828)(580.74837081,187.51121836)(580.67837812,187.46121851)
\curveto(580.63837092,187.43121844)(580.57837098,187.40621846)(580.49837812,187.38621851)
\curveto(580.41837114,187.37621849)(580.32837123,187.3662185)(580.22837812,187.35621851)
\curveto(580.13837142,187.35621851)(580.04837151,187.36121851)(579.95837812,187.37121851)
\curveto(579.87837168,187.38121849)(579.81837174,187.40121847)(579.77837812,187.43121851)
\curveto(579.72837183,187.4712184)(579.68337187,187.53621833)(579.64337812,187.62621851)
\curveto(579.63337192,187.6662182)(579.62337193,187.72121815)(579.61337812,187.79121851)
\curveto(579.61337194,187.86121801)(579.60837195,187.92621794)(579.59837812,187.98621851)
\curveto(579.58837197,188.05621781)(579.56837199,188.11121776)(579.53837812,188.15121851)
\curveto(579.50837205,188.19121768)(579.46337209,188.20621766)(579.40337812,188.19621851)
\curveto(579.32337223,188.17621769)(579.24337231,188.11621775)(579.16337812,188.01621851)
\curveto(579.08337247,187.92621794)(579.00837255,187.85621801)(578.93837812,187.80621851)
\curveto(578.71837284,187.64621822)(578.46837309,187.50621836)(578.18837812,187.38621851)
\curveto(578.07837348,187.33621853)(577.96337359,187.30621856)(577.84337812,187.29621851)
\curveto(577.73337382,187.27621859)(577.61837394,187.25121862)(577.49837812,187.22121851)
\curveto(577.44837411,187.21121866)(577.39337416,187.21121866)(577.33337812,187.22121851)
\curveto(577.28337427,187.23121864)(577.23337432,187.22621864)(577.18337812,187.20621851)
\curveto(577.08337447,187.18621868)(576.99337456,187.18621868)(576.91337812,187.20621851)
\lineto(576.76337812,187.20621851)
\curveto(576.71337484,187.22621864)(576.6533749,187.23621863)(576.58337812,187.23621851)
\curveto(576.52337503,187.23621863)(576.46837509,187.24121863)(576.41837812,187.25121851)
\curveto(576.37837518,187.2712186)(576.33837522,187.28121859)(576.29837812,187.28121851)
\curveto(576.26837529,187.2712186)(576.22837533,187.27621859)(576.17837812,187.29621851)
\lineto(575.93837812,187.35621851)
\curveto(575.86837569,187.37621849)(575.79337576,187.40621846)(575.71337812,187.44621851)
\curveto(575.4533761,187.55621831)(575.23337632,187.70121817)(575.05337812,187.88121851)
\curveto(574.88337667,188.0712178)(574.74337681,188.29621757)(574.63337812,188.55621851)
\curveto(574.59337696,188.64621722)(574.56337699,188.73621713)(574.54337812,188.82621851)
\lineto(574.48337812,189.12621851)
\curveto(574.46337709,189.18621668)(574.4533771,189.24121663)(574.45337812,189.29121851)
\curveto(574.46337709,189.35121652)(574.4583771,189.41621645)(574.43837812,189.48621851)
\curveto(574.42837713,189.50621636)(574.42337713,189.53121634)(574.42337812,189.56121851)
\curveto(574.42337713,189.60121627)(574.41837714,189.63621623)(574.40837812,189.66621851)
\lineto(574.40837812,189.81621851)
\curveto(574.39837716,189.85621601)(574.39337716,189.90121597)(574.39337812,189.95121851)
\curveto(574.40337715,190.01121586)(574.40837715,190.0662158)(574.40837812,190.11621851)
\lineto(574.40837812,190.71621851)
\lineto(574.40837812,193.47621851)
\lineto(574.40837812,194.43621851)
\lineto(574.40837812,194.70621851)
\curveto(574.40837715,194.79621107)(574.42837713,194.871211)(574.46837812,194.93121851)
\curveto(574.50837705,195.00121087)(574.58337697,195.05121082)(574.69337812,195.08121851)
\curveto(574.71337684,195.09121078)(574.73337682,195.09121078)(574.75337812,195.08121851)
\curveto(574.77337678,195.08121079)(574.79337676,195.08621078)(574.81337812,195.09621851)
}
}
{
\newrgbcolor{curcolor}{0 0 0}
\pscustom[linestyle=none,fillstyle=solid,fillcolor=curcolor]
{
\newpath
\moveto(582.7129875,195.09621851)
\lineto(583.2379875,195.09621851)
\curveto(583.43798584,195.10621076)(583.58798569,195.08621078)(583.6879875,195.03621851)
\curveto(583.80798547,194.98621088)(583.90298538,194.90621096)(583.9729875,194.79621851)
\curveto(584.05298523,194.68621118)(584.12798515,194.57621129)(584.1979875,194.46621851)
\curveto(584.32798495,194.2662116)(584.45798482,194.0712118)(584.5879875,193.88121851)
\curveto(584.71798456,193.70121217)(584.85298443,193.51121236)(584.9929875,193.31121851)
\curveto(585.04298424,193.23121264)(585.09298419,193.15621271)(585.1429875,193.08621851)
\curveto(585.20298408,193.01621285)(585.25798402,192.94621292)(585.3079875,192.87621851)
\curveto(585.34798393,192.81621305)(585.38798389,192.76121311)(585.4279875,192.71121851)
\curveto(585.46798381,192.66121321)(585.52798375,192.62621324)(585.6079875,192.60621851)
\curveto(585.65798362,192.58621328)(585.69798358,192.58621328)(585.7279875,192.60621851)
\curveto(585.76798351,192.63621323)(585.79798348,192.66121321)(585.8179875,192.68121851)
\curveto(585.89798338,192.73121314)(585.96298332,192.80121307)(586.0129875,192.89121851)
\curveto(586.07298321,192.98121289)(586.12798315,193.0662128)(586.1779875,193.14621851)
\curveto(586.32798295,193.34621252)(586.4779828,193.55121232)(586.6279875,193.76121851)
\lineto(587.0779875,194.39121851)
\curveto(587.15798212,194.50121137)(587.23798204,194.61621125)(587.3179875,194.73621851)
\curveto(587.39798188,194.85621101)(587.49298179,194.95121092)(587.6029875,195.02121851)
\curveto(587.6829816,195.0712108)(587.7779815,195.09621077)(587.8879875,195.09621851)
\lineto(588.2329875,195.09621851)
\lineto(588.3679875,195.09621851)
\curveto(588.41798086,195.09621077)(588.46798081,195.09121078)(588.5179875,195.08121851)
\lineto(588.5929875,195.08121851)
\curveto(588.71298057,195.06121081)(588.79298049,195.02121085)(588.8329875,194.96121851)
\curveto(588.85298043,194.91121096)(588.84798043,194.85621101)(588.8179875,194.79621851)
\curveto(588.79798048,194.74621112)(588.7779805,194.70621116)(588.7579875,194.67621851)
\lineto(588.5479875,194.37621851)
\curveto(588.4779808,194.28621158)(588.40298088,194.19121168)(588.3229875,194.09121851)
\curveto(588.09298119,193.7712121)(587.85798142,193.45621241)(587.6179875,193.14621851)
\curveto(587.38798189,192.84621302)(587.15798212,192.53621333)(586.9279875,192.21621851)
\curveto(586.8779824,192.13621373)(586.82298246,192.05621381)(586.7629875,191.97621851)
\curveto(586.70298258,191.90621396)(586.64798263,191.82621404)(586.5979875,191.73621851)
\curveto(586.5779827,191.70621416)(586.55798272,191.6662142)(586.5379875,191.61621851)
\curveto(586.51798276,191.57621429)(586.51798276,191.52621434)(586.5379875,191.46621851)
\curveto(586.55798272,191.37621449)(586.58798269,191.30121457)(586.6279875,191.24121851)
\curveto(586.6779826,191.18121469)(586.72798255,191.11621475)(586.7779875,191.04621851)
\lineto(586.9579875,190.77621851)
\curveto(587.02798225,190.68621518)(587.09298219,190.59621527)(587.1529875,190.50621851)
\lineto(587.8429875,189.54621851)
\lineto(588.5329875,188.58621851)
\curveto(588.61298067,188.47621739)(588.69298059,188.36121751)(588.7729875,188.24121851)
\lineto(589.0129875,187.91121851)
\curveto(589.06298022,187.84121803)(589.10298018,187.77621809)(589.1329875,187.71621851)
\curveto(589.17298011,187.6662182)(589.1829801,187.58621828)(589.1629875,187.47621851)
\curveto(589.14298014,187.4662184)(589.12298016,187.45121842)(589.1029875,187.43121851)
\curveto(589.09298019,187.42121845)(589.0779802,187.41121846)(589.0579875,187.40121851)
\curveto(589.00798027,187.38121849)(588.94298034,187.3712185)(588.8629875,187.37121851)
\lineto(588.6229875,187.37121851)
\lineto(588.1129875,187.37121851)
\curveto(587.97298131,187.38121849)(587.84798143,187.42621844)(587.7379875,187.50621851)
\curveto(587.68798159,187.53621833)(587.64798163,187.5712183)(587.6179875,187.61121851)
\curveto(587.59798168,187.66121821)(587.57298171,187.71121816)(587.5429875,187.76121851)
\lineto(587.3929875,187.97121851)
\curveto(587.34298194,188.04121783)(587.29298199,188.11621775)(587.2429875,188.19621851)
\lineto(586.2979875,189.59121851)
\curveto(586.24798303,189.6712162)(586.19798308,189.74621612)(586.1479875,189.81621851)
\curveto(586.09798318,189.88621598)(586.04798323,189.96121591)(585.9979875,190.04121851)
\curveto(585.94798333,190.11121576)(585.89798338,190.1712157)(585.8479875,190.22121851)
\curveto(585.80798347,190.28121559)(585.74798353,190.32121555)(585.6679875,190.34121851)
\curveto(585.61798366,190.36121551)(585.56798371,190.35121552)(585.5179875,190.31121851)
\curveto(585.4779838,190.28121559)(585.44798383,190.25621561)(585.4279875,190.23621851)
\curveto(585.34798393,190.15621571)(585.277984,190.0662158)(585.2179875,189.96621851)
\curveto(585.15798412,189.866216)(585.09798418,189.7712161)(585.0379875,189.68121851)
\curveto(584.86798441,189.42121645)(584.69298459,189.16121671)(584.5129875,188.90121851)
\curveto(584.34298494,188.65121722)(584.16798511,188.40121747)(583.9879875,188.15121851)
\curveto(583.93798534,188.0712178)(583.8829854,187.99121788)(583.8229875,187.91121851)
\lineto(583.6729875,187.67121851)
\curveto(583.65298563,187.64121823)(583.62798565,187.60621826)(583.5979875,187.56621851)
\curveto(583.5779857,187.53621833)(583.55298573,187.51121836)(583.5229875,187.49121851)
\curveto(583.42298586,187.42121845)(583.30298598,187.38121849)(583.1629875,187.37121851)
\lineto(582.7129875,187.37121851)
\lineto(582.4879875,187.37121851)
\curveto(582.41798686,187.3712185)(582.35798692,187.38121849)(582.3079875,187.40121851)
\curveto(582.277987,187.42121845)(582.25298703,187.43621843)(582.2329875,187.44621851)
\curveto(582.22298706,187.4662184)(582.20798707,187.48621838)(582.1879875,187.50621851)
\curveto(582.1779871,187.61621825)(582.19298709,187.70121817)(582.2329875,187.76121851)
\curveto(582.282987,187.82121805)(582.33298695,187.88621798)(582.3829875,187.95621851)
\curveto(582.46298682,188.0662178)(582.53798674,188.1662177)(582.6079875,188.25621851)
\curveto(582.6779866,188.35621751)(582.74798653,188.46121741)(582.8179875,188.57121851)
\curveto(583.03798624,188.871217)(583.25298603,189.1712167)(583.4629875,189.47121851)
\lineto(584.0929875,190.37121851)
\curveto(584.16298512,190.46121541)(584.22798505,190.55121532)(584.2879875,190.64121851)
\curveto(584.35798492,190.73121514)(584.42298486,190.82621504)(584.4829875,190.92621851)
\curveto(584.53298475,190.99621487)(584.5829847,191.06121481)(584.6329875,191.12121851)
\curveto(584.6829846,191.19121468)(584.71798456,191.28121459)(584.7379875,191.39121851)
\curveto(584.75798452,191.44121443)(584.75298453,191.49121438)(584.7229875,191.54121851)
\curveto(584.70298458,191.59121428)(584.6829846,191.63121424)(584.6629875,191.66121851)
\curveto(584.61298467,191.75121412)(584.55798472,191.83621403)(584.4979875,191.91621851)
\lineto(584.3179875,192.15621851)
\curveto(584.08798519,192.47621339)(583.85298543,192.79621307)(583.6129875,193.11621851)
\lineto(582.9229875,194.07621851)
\curveto(582.84298644,194.18621168)(582.76298652,194.28621158)(582.6829875,194.37621851)
\curveto(582.61298667,194.4662114)(582.54298674,194.5662113)(582.4729875,194.67621851)
\curveto(582.45298683,194.70621116)(582.43298685,194.74621112)(582.4129875,194.79621851)
\curveto(582.39298689,194.85621101)(582.39298689,194.90621096)(582.4129875,194.94621851)
\curveto(582.43298685,194.99621087)(582.46298682,195.02621084)(582.5029875,195.03621851)
\curveto(582.54298674,195.05621081)(582.58798669,195.0712108)(582.6379875,195.08121851)
\curveto(582.65798662,195.09121078)(582.67298661,195.09121078)(582.6829875,195.08121851)
\curveto(582.69298659,195.08121079)(582.70298658,195.08621078)(582.7129875,195.09621851)
}
}
{
\newrgbcolor{curcolor}{0 0 0}
\pscustom[linestyle=none,fillstyle=solid,fillcolor=curcolor]
{
\newpath
\moveto(590.74665937,196.59621851)
\curveto(590.66665825,196.65620921)(590.6216583,196.76120911)(590.61165937,196.91121851)
\lineto(590.61165937,197.37621851)
\lineto(590.61165937,197.63121851)
\curveto(590.61165831,197.72120815)(590.62665829,197.79620807)(590.65665937,197.85621851)
\curveto(590.69665822,197.93620793)(590.77665814,197.99620787)(590.89665937,198.03621851)
\curveto(590.916658,198.04620782)(590.93665798,198.04620782)(590.95665937,198.03621851)
\curveto(590.98665793,198.03620783)(591.01165791,198.04120783)(591.03165937,198.05121851)
\curveto(591.20165772,198.05120782)(591.36165756,198.04620782)(591.51165937,198.03621851)
\curveto(591.66165726,198.02620784)(591.76165716,197.9662079)(591.81165937,197.85621851)
\curveto(591.84165708,197.79620807)(591.85665706,197.72120815)(591.85665937,197.63121851)
\lineto(591.85665937,197.37621851)
\curveto(591.85665706,197.19620867)(591.85165707,197.02620884)(591.84165937,196.86621851)
\curveto(591.84165708,196.70620916)(591.77665714,196.60120927)(591.64665937,196.55121851)
\curveto(591.59665732,196.53120934)(591.54165738,196.52120935)(591.48165937,196.52121851)
\lineto(591.31665937,196.52121851)
\lineto(591.00165937,196.52121851)
\curveto(590.90165802,196.52120935)(590.8166581,196.54620932)(590.74665937,196.59621851)
\moveto(591.85665937,188.09121851)
\lineto(591.85665937,187.77621851)
\curveto(591.86665705,187.67621819)(591.84665707,187.59621827)(591.79665937,187.53621851)
\curveto(591.76665715,187.47621839)(591.7216572,187.43621843)(591.66165937,187.41621851)
\curveto(591.60165732,187.40621846)(591.53165739,187.39121848)(591.45165937,187.37121851)
\lineto(591.22665937,187.37121851)
\curveto(591.09665782,187.3712185)(590.98165794,187.37621849)(590.88165937,187.38621851)
\curveto(590.79165813,187.40621846)(590.7216582,187.45621841)(590.67165937,187.53621851)
\curveto(590.63165829,187.59621827)(590.61165831,187.6712182)(590.61165937,187.76121851)
\lineto(590.61165937,188.04621851)
\lineto(590.61165937,194.39121851)
\lineto(590.61165937,194.70621851)
\curveto(590.61165831,194.81621105)(590.63665828,194.90121097)(590.68665937,194.96121851)
\curveto(590.7166582,195.01121086)(590.75665816,195.04121083)(590.80665937,195.05121851)
\curveto(590.85665806,195.06121081)(590.91165801,195.07621079)(590.97165937,195.09621851)
\curveto(590.99165793,195.09621077)(591.01165791,195.09121078)(591.03165937,195.08121851)
\curveto(591.06165786,195.08121079)(591.08665783,195.08621078)(591.10665937,195.09621851)
\curveto(591.23665768,195.09621077)(591.36665755,195.09121078)(591.49665937,195.08121851)
\curveto(591.63665728,195.08121079)(591.73165719,195.04121083)(591.78165937,194.96121851)
\curveto(591.83165709,194.90121097)(591.85665706,194.82121105)(591.85665937,194.72121851)
\lineto(591.85665937,194.43621851)
\lineto(591.85665937,188.09121851)
}
}
{
\newrgbcolor{curcolor}{0 0 0}
\pscustom[linestyle=none,fillstyle=solid,fillcolor=curcolor]
{
\newpath
\moveto(594.37150312,198.05121851)
\curveto(594.50150151,198.05120782)(594.63650137,198.05120782)(594.77650312,198.05121851)
\curveto(594.92650108,198.05120782)(595.03650097,198.01620785)(595.10650312,197.94621851)
\curveto(595.15650085,197.87620799)(595.18150083,197.78120809)(595.18150312,197.66121851)
\curveto(595.19150082,197.55120832)(595.19650081,197.43620843)(595.19650312,197.31621851)
\lineto(595.19650312,195.98121851)
\lineto(595.19650312,189.90621851)
\lineto(595.19650312,188.22621851)
\lineto(595.19650312,187.83621851)
\curveto(595.19650081,187.69621817)(595.17150084,187.58621828)(595.12150312,187.50621851)
\curveto(595.09150092,187.45621841)(595.04650096,187.42621844)(594.98650312,187.41621851)
\curveto(594.93650107,187.40621846)(594.87150114,187.39121848)(594.79150312,187.37121851)
\lineto(594.58150312,187.37121851)
\lineto(594.26650312,187.37121851)
\curveto(594.16650184,187.38121849)(594.09150192,187.41621845)(594.04150312,187.47621851)
\curveto(593.99150202,187.55621831)(593.96150205,187.65621821)(593.95150312,187.77621851)
\lineto(593.95150312,188.15121851)
\lineto(593.95150312,189.53121851)
\lineto(593.95150312,195.77121851)
\lineto(593.95150312,197.24121851)
\curveto(593.95150206,197.35120852)(593.94650206,197.4662084)(593.93650312,197.58621851)
\curveto(593.93650207,197.71620815)(593.96150205,197.81620805)(594.01150312,197.88621851)
\curveto(594.05150196,197.94620792)(594.12650188,197.99620787)(594.23650312,198.03621851)
\curveto(594.25650175,198.04620782)(594.27650173,198.04620782)(594.29650312,198.03621851)
\curveto(594.32650168,198.03620783)(594.35150166,198.04120783)(594.37150312,198.05121851)
}
}
{
\newrgbcolor{curcolor}{0 0 0}
\pscustom[linestyle=none,fillstyle=solid,fillcolor=curcolor]
{
\newpath
\moveto(597.42634687,196.59621851)
\curveto(597.34634575,196.65620921)(597.3013458,196.76120911)(597.29134687,196.91121851)
\lineto(597.29134687,197.37621851)
\lineto(597.29134687,197.63121851)
\curveto(597.29134581,197.72120815)(597.30634579,197.79620807)(597.33634687,197.85621851)
\curveto(597.37634572,197.93620793)(597.45634564,197.99620787)(597.57634687,198.03621851)
\curveto(597.5963455,198.04620782)(597.61634548,198.04620782)(597.63634687,198.03621851)
\curveto(597.66634543,198.03620783)(597.69134541,198.04120783)(597.71134687,198.05121851)
\curveto(597.88134522,198.05120782)(598.04134506,198.04620782)(598.19134687,198.03621851)
\curveto(598.34134476,198.02620784)(598.44134466,197.9662079)(598.49134687,197.85621851)
\curveto(598.52134458,197.79620807)(598.53634456,197.72120815)(598.53634687,197.63121851)
\lineto(598.53634687,197.37621851)
\curveto(598.53634456,197.19620867)(598.53134457,197.02620884)(598.52134687,196.86621851)
\curveto(598.52134458,196.70620916)(598.45634464,196.60120927)(598.32634687,196.55121851)
\curveto(598.27634482,196.53120934)(598.22134488,196.52120935)(598.16134687,196.52121851)
\lineto(597.99634687,196.52121851)
\lineto(597.68134687,196.52121851)
\curveto(597.58134552,196.52120935)(597.4963456,196.54620932)(597.42634687,196.59621851)
\moveto(598.53634687,188.09121851)
\lineto(598.53634687,187.77621851)
\curveto(598.54634455,187.67621819)(598.52634457,187.59621827)(598.47634687,187.53621851)
\curveto(598.44634465,187.47621839)(598.4013447,187.43621843)(598.34134687,187.41621851)
\curveto(598.28134482,187.40621846)(598.21134489,187.39121848)(598.13134687,187.37121851)
\lineto(597.90634687,187.37121851)
\curveto(597.77634532,187.3712185)(597.66134544,187.37621849)(597.56134687,187.38621851)
\curveto(597.47134563,187.40621846)(597.4013457,187.45621841)(597.35134687,187.53621851)
\curveto(597.31134579,187.59621827)(597.29134581,187.6712182)(597.29134687,187.76121851)
\lineto(597.29134687,188.04621851)
\lineto(597.29134687,194.39121851)
\lineto(597.29134687,194.70621851)
\curveto(597.29134581,194.81621105)(597.31634578,194.90121097)(597.36634687,194.96121851)
\curveto(597.3963457,195.01121086)(597.43634566,195.04121083)(597.48634687,195.05121851)
\curveto(597.53634556,195.06121081)(597.59134551,195.07621079)(597.65134687,195.09621851)
\curveto(597.67134543,195.09621077)(597.69134541,195.09121078)(597.71134687,195.08121851)
\curveto(597.74134536,195.08121079)(597.76634533,195.08621078)(597.78634687,195.09621851)
\curveto(597.91634518,195.09621077)(598.04634505,195.09121078)(598.17634687,195.08121851)
\curveto(598.31634478,195.08121079)(598.41134469,195.04121083)(598.46134687,194.96121851)
\curveto(598.51134459,194.90121097)(598.53634456,194.82121105)(598.53634687,194.72121851)
\lineto(598.53634687,194.43621851)
\lineto(598.53634687,188.09121851)
}
}
{
\newrgbcolor{curcolor}{0 0 0}
\pscustom[linestyle=none,fillstyle=solid,fillcolor=curcolor]
{
\newpath
\moveto(607.36619062,187.92621851)
\curveto(607.39618279,187.7662181)(607.38118281,187.63121824)(607.32119062,187.52121851)
\curveto(607.26118293,187.42121845)(607.18118301,187.34621852)(607.08119062,187.29621851)
\curveto(607.03118316,187.27621859)(606.97618321,187.2662186)(606.91619062,187.26621851)
\curveto(606.86618332,187.2662186)(606.81118338,187.25621861)(606.75119062,187.23621851)
\curveto(606.53118366,187.18621868)(606.31118388,187.20121867)(606.09119062,187.28121851)
\curveto(605.88118431,187.35121852)(605.73618445,187.44121843)(605.65619062,187.55121851)
\curveto(605.60618458,187.62121825)(605.56118463,187.70121817)(605.52119062,187.79121851)
\curveto(605.48118471,187.89121798)(605.43118476,187.9712179)(605.37119062,188.03121851)
\curveto(605.35118484,188.05121782)(605.32618486,188.0712178)(605.29619062,188.09121851)
\curveto(605.27618491,188.11121776)(605.24618494,188.11621775)(605.20619062,188.10621851)
\curveto(605.09618509,188.07621779)(604.9911852,188.02121785)(604.89119062,187.94121851)
\curveto(604.80118539,187.86121801)(604.71118548,187.79121808)(604.62119062,187.73121851)
\curveto(604.4911857,187.65121822)(604.35118584,187.57621829)(604.20119062,187.50621851)
\curveto(604.05118614,187.44621842)(603.8911863,187.39121848)(603.72119062,187.34121851)
\curveto(603.62118657,187.31121856)(603.51118668,187.29121858)(603.39119062,187.28121851)
\curveto(603.28118691,187.2712186)(603.17118702,187.25621861)(603.06119062,187.23621851)
\curveto(603.01118718,187.22621864)(602.96618722,187.22121865)(602.92619062,187.22121851)
\lineto(602.82119062,187.22121851)
\curveto(602.71118748,187.20121867)(602.60618758,187.20121867)(602.50619062,187.22121851)
\lineto(602.37119062,187.22121851)
\curveto(602.32118787,187.23121864)(602.27118792,187.23621863)(602.22119062,187.23621851)
\curveto(602.17118802,187.23621863)(602.12618806,187.24621862)(602.08619062,187.26621851)
\curveto(602.04618814,187.27621859)(602.01118818,187.28121859)(601.98119062,187.28121851)
\curveto(601.96118823,187.2712186)(601.93618825,187.2712186)(601.90619062,187.28121851)
\lineto(601.66619062,187.34121851)
\curveto(601.5861886,187.35121852)(601.51118868,187.3712185)(601.44119062,187.40121851)
\curveto(601.14118905,187.53121834)(600.89618929,187.67621819)(600.70619062,187.83621851)
\curveto(600.52618966,188.00621786)(600.37618981,188.24121763)(600.25619062,188.54121851)
\curveto(600.16619002,188.76121711)(600.12119007,189.02621684)(600.12119062,189.33621851)
\lineto(600.12119062,189.65121851)
\curveto(600.13119006,189.70121617)(600.13619005,189.75121612)(600.13619062,189.80121851)
\lineto(600.16619062,189.98121851)
\lineto(600.28619062,190.31121851)
\curveto(600.32618986,190.42121545)(600.37618981,190.52121535)(600.43619062,190.61121851)
\curveto(600.61618957,190.90121497)(600.86118933,191.11621475)(601.17119062,191.25621851)
\curveto(601.48118871,191.39621447)(601.82118837,191.52121435)(602.19119062,191.63121851)
\curveto(602.33118786,191.6712142)(602.47618771,191.70121417)(602.62619062,191.72121851)
\curveto(602.77618741,191.74121413)(602.92618726,191.7662141)(603.07619062,191.79621851)
\curveto(603.14618704,191.81621405)(603.21118698,191.82621404)(603.27119062,191.82621851)
\curveto(603.34118685,191.82621404)(603.41618677,191.83621403)(603.49619062,191.85621851)
\curveto(603.56618662,191.87621399)(603.63618655,191.88621398)(603.70619062,191.88621851)
\curveto(603.77618641,191.89621397)(603.85118634,191.91121396)(603.93119062,191.93121851)
\curveto(604.18118601,191.99121388)(604.41618577,192.04121383)(604.63619062,192.08121851)
\curveto(604.85618533,192.13121374)(605.03118516,192.24621362)(605.16119062,192.42621851)
\curveto(605.22118497,192.50621336)(605.27118492,192.60621326)(605.31119062,192.72621851)
\curveto(605.35118484,192.85621301)(605.35118484,192.99621287)(605.31119062,193.14621851)
\curveto(605.25118494,193.38621248)(605.16118503,193.57621229)(605.04119062,193.71621851)
\curveto(604.93118526,193.85621201)(604.77118542,193.9662119)(604.56119062,194.04621851)
\curveto(604.44118575,194.09621177)(604.29618589,194.13121174)(604.12619062,194.15121851)
\curveto(603.96618622,194.1712117)(603.79618639,194.18121169)(603.61619062,194.18121851)
\curveto(603.43618675,194.18121169)(603.26118693,194.1712117)(603.09119062,194.15121851)
\curveto(602.92118727,194.13121174)(602.77618741,194.10121177)(602.65619062,194.06121851)
\curveto(602.4861877,194.00121187)(602.32118787,193.91621195)(602.16119062,193.80621851)
\curveto(602.08118811,193.74621212)(602.00618818,193.6662122)(601.93619062,193.56621851)
\curveto(601.87618831,193.47621239)(601.82118837,193.37621249)(601.77119062,193.26621851)
\curveto(601.74118845,193.18621268)(601.71118848,193.10121277)(601.68119062,193.01121851)
\curveto(601.66118853,192.92121295)(601.61618857,192.85121302)(601.54619062,192.80121851)
\curveto(601.50618868,192.7712131)(601.43618875,192.74621312)(601.33619062,192.72621851)
\curveto(601.24618894,192.71621315)(601.15118904,192.71121316)(601.05119062,192.71121851)
\curveto(600.95118924,192.71121316)(600.85118934,192.71621315)(600.75119062,192.72621851)
\curveto(600.66118953,192.74621312)(600.59618959,192.7712131)(600.55619062,192.80121851)
\curveto(600.51618967,192.83121304)(600.4861897,192.88121299)(600.46619062,192.95121851)
\curveto(600.44618974,193.02121285)(600.44618974,193.09621277)(600.46619062,193.17621851)
\curveto(600.49618969,193.30621256)(600.52618966,193.42621244)(600.55619062,193.53621851)
\curveto(600.59618959,193.65621221)(600.64118955,193.7712121)(600.69119062,193.88121851)
\curveto(600.88118931,194.23121164)(601.12118907,194.50121137)(601.41119062,194.69121851)
\curveto(601.70118849,194.89121098)(602.06118813,195.05121082)(602.49119062,195.17121851)
\curveto(602.5911876,195.19121068)(602.6911875,195.20621066)(602.79119062,195.21621851)
\curveto(602.90118729,195.22621064)(603.01118718,195.24121063)(603.12119062,195.26121851)
\curveto(603.16118703,195.2712106)(603.22618696,195.2712106)(603.31619062,195.26121851)
\curveto(603.40618678,195.26121061)(603.46118673,195.2712106)(603.48119062,195.29121851)
\curveto(604.18118601,195.30121057)(604.7911854,195.22121065)(605.31119062,195.05121851)
\curveto(605.83118436,194.88121099)(606.19618399,194.55621131)(606.40619062,194.07621851)
\curveto(606.49618369,193.87621199)(606.54618364,193.64121223)(606.55619062,193.37121851)
\curveto(606.57618361,193.11121276)(606.5861836,192.83621303)(606.58619062,192.54621851)
\lineto(606.58619062,189.23121851)
\curveto(606.5861836,189.09121678)(606.5911836,188.95621691)(606.60119062,188.82621851)
\curveto(606.61118358,188.69621717)(606.64118355,188.59121728)(606.69119062,188.51121851)
\curveto(606.74118345,188.44121743)(606.80618338,188.39121748)(606.88619062,188.36121851)
\curveto(606.97618321,188.32121755)(607.06118313,188.29121758)(607.14119062,188.27121851)
\curveto(607.22118297,188.26121761)(607.28118291,188.21621765)(607.32119062,188.13621851)
\curveto(607.34118285,188.10621776)(607.35118284,188.07621779)(607.35119062,188.04621851)
\curveto(607.35118284,188.01621785)(607.35618283,187.97621789)(607.36619062,187.92621851)
\moveto(605.22119062,189.59121851)
\curveto(605.28118491,189.73121614)(605.31118488,189.89121598)(605.31119062,190.07121851)
\curveto(605.32118487,190.26121561)(605.32618486,190.45621541)(605.32619062,190.65621851)
\curveto(605.32618486,190.7662151)(605.32118487,190.866215)(605.31119062,190.95621851)
\curveto(605.30118489,191.04621482)(605.26118493,191.11621475)(605.19119062,191.16621851)
\curveto(605.16118503,191.18621468)(605.0911851,191.19621467)(604.98119062,191.19621851)
\curveto(604.96118523,191.17621469)(604.92618526,191.1662147)(604.87619062,191.16621851)
\curveto(604.82618536,191.1662147)(604.78118541,191.15621471)(604.74119062,191.13621851)
\curveto(604.66118553,191.11621475)(604.57118562,191.09621477)(604.47119062,191.07621851)
\lineto(604.17119062,191.01621851)
\curveto(604.14118605,191.01621485)(604.10618608,191.01121486)(604.06619062,191.00121851)
\lineto(603.96119062,191.00121851)
\curveto(603.81118638,190.96121491)(603.64618654,190.93621493)(603.46619062,190.92621851)
\curveto(603.29618689,190.92621494)(603.13618705,190.90621496)(602.98619062,190.86621851)
\curveto(602.90618728,190.84621502)(602.83118736,190.82621504)(602.76119062,190.80621851)
\curveto(602.70118749,190.79621507)(602.63118756,190.78121509)(602.55119062,190.76121851)
\curveto(602.3911878,190.71121516)(602.24118795,190.64621522)(602.10119062,190.56621851)
\curveto(601.96118823,190.49621537)(601.84118835,190.40621546)(601.74119062,190.29621851)
\curveto(601.64118855,190.18621568)(601.56618862,190.05121582)(601.51619062,189.89121851)
\curveto(601.46618872,189.74121613)(601.44618874,189.55621631)(601.45619062,189.33621851)
\curveto(601.45618873,189.23621663)(601.47118872,189.14121673)(601.50119062,189.05121851)
\curveto(601.54118865,188.9712169)(601.5861886,188.89621697)(601.63619062,188.82621851)
\curveto(601.71618847,188.71621715)(601.82118837,188.62121725)(601.95119062,188.54121851)
\curveto(602.08118811,188.4712174)(602.22118797,188.41121746)(602.37119062,188.36121851)
\curveto(602.42118777,188.35121752)(602.47118772,188.34621752)(602.52119062,188.34621851)
\curveto(602.57118762,188.34621752)(602.62118757,188.34121753)(602.67119062,188.33121851)
\curveto(602.74118745,188.31121756)(602.82618736,188.29621757)(602.92619062,188.28621851)
\curveto(603.03618715,188.28621758)(603.12618706,188.29621757)(603.19619062,188.31621851)
\curveto(603.25618693,188.33621753)(603.31618687,188.34121753)(603.37619062,188.33121851)
\curveto(603.43618675,188.33121754)(603.49618669,188.34121753)(603.55619062,188.36121851)
\curveto(603.63618655,188.38121749)(603.71118648,188.39621747)(603.78119062,188.40621851)
\curveto(603.86118633,188.41621745)(603.93618625,188.43621743)(604.00619062,188.46621851)
\curveto(604.29618589,188.58621728)(604.54118565,188.73121714)(604.74119062,188.90121851)
\curveto(604.95118524,189.0712168)(605.11118508,189.30121657)(605.22119062,189.59121851)
}
}
{
\newrgbcolor{curcolor}{0 0 0}
\pscustom[linestyle=none,fillstyle=solid,fillcolor=curcolor]
{
\newpath
\moveto(612.18283125,195.27621851)
\curveto(612.41282646,195.27621059)(612.54282633,195.21621065)(612.57283125,195.09621851)
\curveto(612.60282627,194.98621088)(612.61782625,194.82121105)(612.61783125,194.60121851)
\lineto(612.61783125,194.31621851)
\curveto(612.61782625,194.22621164)(612.59282628,194.15121172)(612.54283125,194.09121851)
\curveto(612.48282639,194.01121186)(612.39782647,193.9662119)(612.28783125,193.95621851)
\curveto(612.17782669,193.95621191)(612.0678268,193.94121193)(611.95783125,193.91121851)
\curveto(611.81782705,193.88121199)(611.68282719,193.85121202)(611.55283125,193.82121851)
\curveto(611.43282744,193.79121208)(611.31782755,193.75121212)(611.20783125,193.70121851)
\curveto(610.91782795,193.5712123)(610.68282819,193.39121248)(610.50283125,193.16121851)
\curveto(610.32282855,192.94121293)(610.1678287,192.68621318)(610.03783125,192.39621851)
\curveto(609.99782887,192.28621358)(609.9678289,192.1712137)(609.94783125,192.05121851)
\curveto(609.92782894,191.94121393)(609.90282897,191.82621404)(609.87283125,191.70621851)
\curveto(609.86282901,191.65621421)(609.85782901,191.60621426)(609.85783125,191.55621851)
\curveto(609.867829,191.50621436)(609.867829,191.45621441)(609.85783125,191.40621851)
\curveto(609.82782904,191.28621458)(609.81282906,191.14621472)(609.81283125,190.98621851)
\curveto(609.82282905,190.83621503)(609.82782904,190.69121518)(609.82783125,190.55121851)
\lineto(609.82783125,188.70621851)
\lineto(609.82783125,188.36121851)
\curveto(609.82782904,188.24121763)(609.82282905,188.12621774)(609.81283125,188.01621851)
\curveto(609.80282907,187.90621796)(609.79782907,187.81121806)(609.79783125,187.73121851)
\curveto(609.80782906,187.65121822)(609.78782908,187.58121829)(609.73783125,187.52121851)
\curveto(609.68782918,187.45121842)(609.60782926,187.41121846)(609.49783125,187.40121851)
\curveto(609.39782947,187.39121848)(609.28782958,187.38621848)(609.16783125,187.38621851)
\lineto(608.89783125,187.38621851)
\curveto(608.84783002,187.40621846)(608.79783007,187.42121845)(608.74783125,187.43121851)
\curveto(608.70783016,187.45121842)(608.67783019,187.47621839)(608.65783125,187.50621851)
\curveto(608.60783026,187.57621829)(608.57783029,187.66121821)(608.56783125,187.76121851)
\lineto(608.56783125,188.09121851)
\lineto(608.56783125,189.24621851)
\lineto(608.56783125,193.40121851)
\lineto(608.56783125,194.43621851)
\lineto(608.56783125,194.73621851)
\curveto(608.57783029,194.83621103)(608.60783026,194.92121095)(608.65783125,194.99121851)
\curveto(608.68783018,195.03121084)(608.73783013,195.06121081)(608.80783125,195.08121851)
\curveto(608.88782998,195.10121077)(608.9728299,195.11121076)(609.06283125,195.11121851)
\curveto(609.15282972,195.12121075)(609.24282963,195.12121075)(609.33283125,195.11121851)
\curveto(609.42282945,195.10121077)(609.49282938,195.08621078)(609.54283125,195.06621851)
\curveto(609.62282925,195.03621083)(609.6728292,194.97621089)(609.69283125,194.88621851)
\curveto(609.72282915,194.80621106)(609.73782913,194.71621115)(609.73783125,194.61621851)
\lineto(609.73783125,194.31621851)
\curveto(609.73782913,194.21621165)(609.75782911,194.12621174)(609.79783125,194.04621851)
\curveto(609.80782906,194.02621184)(609.81782905,194.01121186)(609.82783125,194.00121851)
\lineto(609.87283125,193.95621851)
\curveto(609.98282889,193.95621191)(610.0728288,194.00121187)(610.14283125,194.09121851)
\curveto(610.21282866,194.19121168)(610.2728286,194.2712116)(610.32283125,194.33121851)
\lineto(610.41283125,194.42121851)
\curveto(610.50282837,194.53121134)(610.62782824,194.64621122)(610.78783125,194.76621851)
\curveto(610.94782792,194.88621098)(611.09782777,194.97621089)(611.23783125,195.03621851)
\curveto(611.32782754,195.08621078)(611.42282745,195.12121075)(611.52283125,195.14121851)
\curveto(611.62282725,195.1712107)(611.72782714,195.20121067)(611.83783125,195.23121851)
\curveto(611.89782697,195.24121063)(611.95782691,195.24621062)(612.01783125,195.24621851)
\curveto(612.07782679,195.25621061)(612.13282674,195.2662106)(612.18283125,195.27621851)
}
}
{
\newrgbcolor{curcolor}{0.7019608 0.7019608 0.7019608}
\pscustom[linestyle=none,fillstyle=solid,fillcolor=curcolor]
{
\newpath
\moveto(545.29360762,198.08125513)
\lineto(560.29360762,198.08125513)
\lineto(560.29360762,183.08125513)
\lineto(545.29360762,183.08125513)
\closepath
}
}
{
\newrgbcolor{curcolor}{0 0 0}
\pscustom[linestyle=none,fillstyle=solid,fillcolor=curcolor]
{
\newpath
\moveto(573.66681562,169.93045191)
\lineto(573.66681562,169.66045191)
\curveto(573.67680565,169.57044666)(573.67180566,169.49044674)(573.65181562,169.42045191)
\lineto(573.65181562,169.27045191)
\curveto(573.64180569,169.24044699)(573.63680569,169.20544703)(573.63681562,169.16545191)
\curveto(573.64680568,169.12544711)(573.64680568,169.09544714)(573.63681562,169.07545191)
\curveto(573.6268057,169.02544721)(573.62180571,168.97044726)(573.62181562,168.91045191)
\curveto(573.62180571,168.86044737)(573.61680571,168.81044742)(573.60681562,168.76045191)
\curveto(573.57680575,168.62044761)(573.55680577,168.47044776)(573.54681562,168.31045191)
\curveto(573.53680579,168.16044807)(573.50680582,168.01544822)(573.45681562,167.87545191)
\curveto(573.4268059,167.75544848)(573.39180594,167.6304486)(573.35181562,167.50045191)
\curveto(573.32180601,167.38044885)(573.28180605,167.26044897)(573.23181562,167.14045191)
\curveto(573.06180627,166.71044952)(572.84680648,166.32044991)(572.58681562,165.97045191)
\curveto(572.33680699,165.6304506)(572.02180731,165.34045089)(571.64181562,165.10045191)
\curveto(571.45180788,164.98045125)(571.24680808,164.87545136)(571.02681562,164.78545191)
\curveto(570.81680851,164.70545153)(570.58680874,164.62545161)(570.33681562,164.54545191)
\curveto(570.2268091,164.50545173)(570.10680922,164.47545176)(569.97681562,164.45545191)
\curveto(569.85680947,164.44545179)(569.73680959,164.42545181)(569.61681562,164.39545191)
\curveto(569.50680982,164.37545186)(569.39680993,164.36545187)(569.28681562,164.36545191)
\curveto(569.18681014,164.36545187)(569.08681024,164.35545188)(568.98681562,164.33545191)
\lineto(568.77681562,164.33545191)
\curveto(568.74681058,164.32545191)(568.71181062,164.32045191)(568.67181562,164.32045191)
\curveto(568.6318107,164.3304519)(568.59181074,164.3354519)(568.55181562,164.33545191)
\lineto(565.55181562,164.33545191)
\curveto(565.40181393,164.3354519)(565.26681406,164.34045189)(565.14681562,164.35045191)
\curveto(565.03681429,164.37045186)(564.96181437,164.4354518)(564.92181562,164.54545191)
\curveto(564.88181445,164.62545161)(564.86181447,164.74045149)(564.86181562,164.89045191)
\curveto(564.87181446,165.04045119)(564.87681445,165.17545106)(564.87681562,165.29545191)
\lineto(564.87681562,174.16045191)
\curveto(564.87681445,174.28044195)(564.87181446,174.40544183)(564.86181562,174.53545191)
\curveto(564.86181447,174.67544156)(564.88681444,174.78544145)(564.93681562,174.86545191)
\curveto(564.97681435,174.9354413)(565.05181428,174.98044125)(565.16181562,175.00045191)
\curveto(565.18181415,175.01044122)(565.20181413,175.01044122)(565.22181562,175.00045191)
\curveto(565.24181409,175.00044123)(565.26181407,175.00544123)(565.28181562,175.01545191)
\lineto(568.53681562,175.01545191)
\curveto(568.58681074,175.01544122)(568.6318107,175.01544122)(568.67181562,175.01545191)
\curveto(568.72181061,175.02544121)(568.76681056,175.02544121)(568.80681562,175.01545191)
\curveto(568.85681047,174.99544124)(568.90681042,174.99044124)(568.95681562,175.00045191)
\curveto(569.01681031,175.01044122)(569.07181026,175.01044122)(569.12181562,175.00045191)
\curveto(569.17181016,174.99044124)(569.2268101,174.98544125)(569.28681562,174.98545191)
\curveto(569.34680998,174.98544125)(569.40180993,174.98044125)(569.45181562,174.97045191)
\curveto(569.50180983,174.96044127)(569.54680978,174.95544128)(569.58681562,174.95545191)
\curveto(569.63680969,174.95544128)(569.68680964,174.95044128)(569.73681562,174.94045191)
\curveto(569.84680948,174.92044131)(569.95180938,174.90044133)(570.05181562,174.88045191)
\curveto(570.15180918,174.87044136)(570.25180908,174.85044138)(570.35181562,174.82045191)
\curveto(570.57180876,174.75044148)(570.78180855,174.68044155)(570.98181562,174.61045191)
\curveto(571.18180815,174.55044168)(571.36680796,174.46544177)(571.53681562,174.35545191)
\curveto(571.67680765,174.27544196)(571.80180753,174.19544204)(571.91181562,174.11545191)
\curveto(571.94180739,174.09544214)(571.97180736,174.07044216)(572.00181562,174.04045191)
\curveto(572.0318073,174.02044221)(572.06180727,174.00044223)(572.09181562,173.98045191)
\curveto(572.15180718,173.9304423)(572.20680712,173.88044235)(572.25681562,173.83045191)
\curveto(572.30680702,173.78044245)(572.35680697,173.7304425)(572.40681562,173.68045191)
\curveto(572.45680687,173.6304426)(572.49680683,173.59544264)(572.52681562,173.57545191)
\curveto(572.56680676,173.51544272)(572.60680672,173.46044277)(572.64681562,173.41045191)
\curveto(572.69680663,173.36044287)(572.74180659,173.30544293)(572.78181562,173.24545191)
\curveto(572.8318065,173.18544305)(572.87180646,173.12044311)(572.90181562,173.05045191)
\curveto(572.94180639,172.99044324)(572.98680634,172.92544331)(573.03681562,172.85545191)
\curveto(573.05680627,172.81544342)(573.07180626,172.78044345)(573.08181562,172.75045191)
\curveto(573.09180624,172.72044351)(573.10680622,172.68544355)(573.12681562,172.64545191)
\curveto(573.16680616,172.56544367)(573.20180613,172.48544375)(573.23181562,172.40545191)
\curveto(573.26180607,172.3354439)(573.29680603,172.26044397)(573.33681562,172.18045191)
\curveto(573.37680595,172.07044416)(573.40680592,171.95544428)(573.42681562,171.83545191)
\curveto(573.45680587,171.72544451)(573.48680584,171.61544462)(573.51681562,171.50545191)
\curveto(573.53680579,171.44544479)(573.54680578,171.38544485)(573.54681562,171.32545191)
\curveto(573.54680578,171.27544496)(573.55680577,171.22044501)(573.57681562,171.16045191)
\curveto(573.6268057,170.98044525)(573.65180568,170.78044545)(573.65181562,170.56045191)
\curveto(573.66180567,170.35044588)(573.66680566,170.14044609)(573.66681562,169.93045191)
\moveto(572.24181562,169.15045191)
\curveto(572.26180707,169.25044698)(572.27180706,169.35544688)(572.27181562,169.46545191)
\lineto(572.27181562,169.81045191)
\lineto(572.27181562,170.03545191)
\curveto(572.28180705,170.11544612)(572.27680705,170.19044604)(572.25681562,170.26045191)
\curveto(572.25680707,170.29044594)(572.25180708,170.32044591)(572.24181562,170.35045191)
\lineto(572.24181562,170.45545191)
\curveto(572.22180711,170.56544567)(572.20680712,170.67544556)(572.19681562,170.78545191)
\curveto(572.19680713,170.89544534)(572.18180715,171.00544523)(572.15181562,171.11545191)
\curveto(572.1318072,171.19544504)(572.11180722,171.27044496)(572.09181562,171.34045191)
\curveto(572.08180725,171.42044481)(572.06680726,171.50044473)(572.04681562,171.58045191)
\curveto(571.93680739,171.94044429)(571.79680753,172.25544398)(571.62681562,172.52545191)
\curveto(571.34680798,172.97544326)(570.9318084,173.31544292)(570.38181562,173.54545191)
\curveto(570.29180904,173.59544264)(570.19680913,173.6304426)(570.09681562,173.65045191)
\curveto(569.99680933,173.68044255)(569.89180944,173.71044252)(569.78181562,173.74045191)
\curveto(569.67180966,173.77044246)(569.55680977,173.78544245)(569.43681562,173.78545191)
\curveto(569.32681,173.79544244)(569.21681011,173.81044242)(569.10681562,173.83045191)
\lineto(568.79181562,173.83045191)
\curveto(568.76181057,173.84044239)(568.7268106,173.84544239)(568.68681562,173.84545191)
\lineto(568.56681562,173.84545191)
\lineto(566.73681562,173.84545191)
\curveto(566.71681261,173.8354424)(566.69181264,173.8304424)(566.66181562,173.83045191)
\curveto(566.6318127,173.84044239)(566.60681272,173.84044239)(566.58681562,173.83045191)
\lineto(566.43681562,173.77045191)
\curveto(566.39681293,173.75044248)(566.36681296,173.72044251)(566.34681562,173.68045191)
\curveto(566.326813,173.64044259)(566.30681302,173.57044266)(566.28681562,173.47045191)
\lineto(566.28681562,173.35045191)
\curveto(566.27681305,173.31044292)(566.27181306,173.26544297)(566.27181562,173.21545191)
\lineto(566.27181562,173.08045191)
\lineto(566.27181562,166.27045191)
\lineto(566.27181562,166.12045191)
\curveto(566.27181306,166.08045015)(566.27681305,166.04045019)(566.28681562,166.00045191)
\lineto(566.28681562,165.88045191)
\curveto(566.30681302,165.78045045)(566.326813,165.71045052)(566.34681562,165.67045191)
\curveto(566.4268129,165.55045068)(566.57681275,165.49045074)(566.79681562,165.49045191)
\curveto(567.01681231,165.50045073)(567.2268121,165.50545073)(567.42681562,165.50545191)
\lineto(568.29681562,165.50545191)
\curveto(568.36681096,165.50545073)(568.44181089,165.50045073)(568.52181562,165.49045191)
\curveto(568.60181073,165.49045074)(568.67181066,165.50045073)(568.73181562,165.52045191)
\lineto(568.89681562,165.52045191)
\curveto(568.94681038,165.5304507)(569.00181033,165.5304507)(569.06181562,165.52045191)
\curveto(569.12181021,165.52045071)(569.18181015,165.52545071)(569.24181562,165.53545191)
\curveto(569.30181003,165.55545068)(569.36180997,165.56545067)(569.42181562,165.56545191)
\curveto(569.48180985,165.57545066)(569.54680978,165.59045064)(569.61681562,165.61045191)
\curveto(569.7268096,165.64045059)(569.8318095,165.67045056)(569.93181562,165.70045191)
\curveto(570.04180929,165.7304505)(570.15180918,165.77045046)(570.26181562,165.82045191)
\curveto(570.6318087,165.98045025)(570.94680838,166.19545004)(571.20681562,166.46545191)
\curveto(571.47680785,166.74544949)(571.69680763,167.07544916)(571.86681562,167.45545191)
\curveto(571.91680741,167.56544867)(571.95680737,167.68044855)(571.98681562,167.80045191)
\lineto(572.10681562,168.19045191)
\curveto(572.13680719,168.30044793)(572.15680717,168.41544782)(572.16681562,168.53545191)
\curveto(572.18680714,168.66544757)(572.20680712,168.79044744)(572.22681562,168.91045191)
\curveto(572.23680709,168.96044727)(572.24180709,169.00044723)(572.24181562,169.03045191)
\lineto(572.24181562,169.15045191)
}
}
{
\newrgbcolor{curcolor}{0 0 0}
\pscustom[linestyle=none,fillstyle=solid,fillcolor=curcolor]
{
\newpath
\moveto(582.29369062,168.53545191)
\curveto(582.31368256,168.47544776)(582.32368255,168.38044785)(582.32369062,168.25045191)
\curveto(582.32368255,168.1304481)(582.31868256,168.04544819)(582.30869062,167.99545191)
\lineto(582.30869062,167.84545191)
\curveto(582.29868258,167.76544847)(582.28868259,167.69044854)(582.27869062,167.62045191)
\curveto(582.2786826,167.56044867)(582.2736826,167.49044874)(582.26369062,167.41045191)
\curveto(582.24368263,167.35044888)(582.22868265,167.29044894)(582.21869062,167.23045191)
\curveto(582.21868266,167.17044906)(582.20868267,167.11044912)(582.18869062,167.05045191)
\curveto(582.14868273,166.92044931)(582.11368276,166.79044944)(582.08369062,166.66045191)
\curveto(582.05368282,166.5304497)(582.01368286,166.41044982)(581.96369062,166.30045191)
\curveto(581.75368312,165.82045041)(581.4736834,165.41545082)(581.12369062,165.08545191)
\curveto(580.7736841,164.76545147)(580.34368453,164.52045171)(579.83369062,164.35045191)
\curveto(579.72368515,164.31045192)(579.60368527,164.28045195)(579.47369062,164.26045191)
\curveto(579.35368552,164.24045199)(579.22868565,164.22045201)(579.09869062,164.20045191)
\curveto(579.03868584,164.19045204)(578.9736859,164.18545205)(578.90369062,164.18545191)
\curveto(578.84368603,164.17545206)(578.78368609,164.17045206)(578.72369062,164.17045191)
\curveto(578.68368619,164.16045207)(578.62368625,164.15545208)(578.54369062,164.15545191)
\curveto(578.4736864,164.15545208)(578.42368645,164.16045207)(578.39369062,164.17045191)
\curveto(578.35368652,164.18045205)(578.31368656,164.18545205)(578.27369062,164.18545191)
\curveto(578.23368664,164.17545206)(578.19868668,164.17545206)(578.16869062,164.18545191)
\lineto(578.07869062,164.18545191)
\lineto(577.71869062,164.23045191)
\curveto(577.5786873,164.27045196)(577.44368743,164.31045192)(577.31369062,164.35045191)
\curveto(577.18368769,164.39045184)(577.05868782,164.4354518)(576.93869062,164.48545191)
\curveto(576.48868839,164.68545155)(576.11868876,164.94545129)(575.82869062,165.26545191)
\curveto(575.53868934,165.58545065)(575.29868958,165.97545026)(575.10869062,166.43545191)
\curveto(575.05868982,166.5354497)(575.01868986,166.6354496)(574.98869062,166.73545191)
\curveto(574.96868991,166.8354494)(574.94868993,166.94044929)(574.92869062,167.05045191)
\curveto(574.90868997,167.09044914)(574.89868998,167.12044911)(574.89869062,167.14045191)
\curveto(574.90868997,167.17044906)(574.90868997,167.20544903)(574.89869062,167.24545191)
\curveto(574.87869,167.32544891)(574.86369001,167.40544883)(574.85369062,167.48545191)
\curveto(574.85369002,167.57544866)(574.84369003,167.66044857)(574.82369062,167.74045191)
\lineto(574.82369062,167.86045191)
\curveto(574.82369005,167.90044833)(574.81869006,167.94544829)(574.80869062,167.99545191)
\curveto(574.79869008,168.04544819)(574.79369008,168.1304481)(574.79369062,168.25045191)
\curveto(574.79369008,168.38044785)(574.80369007,168.47544776)(574.82369062,168.53545191)
\curveto(574.84369003,168.60544763)(574.84869003,168.67544756)(574.83869062,168.74545191)
\curveto(574.82869005,168.81544742)(574.83369004,168.88544735)(574.85369062,168.95545191)
\curveto(574.86369001,169.00544723)(574.86869001,169.04544719)(574.86869062,169.07545191)
\curveto(574.87869,169.11544712)(574.88868999,169.16044707)(574.89869062,169.21045191)
\curveto(574.92868995,169.3304469)(574.95368992,169.45044678)(574.97369062,169.57045191)
\curveto(575.00368987,169.69044654)(575.04368983,169.80544643)(575.09369062,169.91545191)
\curveto(575.24368963,170.28544595)(575.42368945,170.61544562)(575.63369062,170.90545191)
\curveto(575.85368902,171.20544503)(576.11868876,171.45544478)(576.42869062,171.65545191)
\curveto(576.54868833,171.7354445)(576.6736882,171.80044443)(576.80369062,171.85045191)
\curveto(576.93368794,171.91044432)(577.06868781,171.97044426)(577.20869062,172.03045191)
\curveto(577.32868755,172.08044415)(577.45868742,172.11044412)(577.59869062,172.12045191)
\curveto(577.73868714,172.14044409)(577.878687,172.17044406)(578.01869062,172.21045191)
\lineto(578.21369062,172.21045191)
\curveto(578.28368659,172.22044401)(578.34868653,172.230444)(578.40869062,172.24045191)
\curveto(579.29868558,172.25044398)(580.03868484,172.06544417)(580.62869062,171.68545191)
\curveto(581.21868366,171.30544493)(581.64368323,170.81044542)(581.90369062,170.20045191)
\curveto(581.95368292,170.10044613)(581.99368288,170.00044623)(582.02369062,169.90045191)
\curveto(582.05368282,169.80044643)(582.08868279,169.69544654)(582.12869062,169.58545191)
\curveto(582.15868272,169.47544676)(582.18368269,169.35544688)(582.20369062,169.22545191)
\curveto(582.22368265,169.10544713)(582.24868263,168.98044725)(582.27869062,168.85045191)
\curveto(582.28868259,168.80044743)(582.28868259,168.74544749)(582.27869062,168.68545191)
\curveto(582.2786826,168.6354476)(582.28368259,168.58544765)(582.29369062,168.53545191)
\moveto(580.95869062,167.68045191)
\curveto(580.9786839,167.75044848)(580.98368389,167.8304484)(580.97369062,167.92045191)
\lineto(580.97369062,168.17545191)
\curveto(580.9736839,168.56544767)(580.93868394,168.89544734)(580.86869062,169.16545191)
\curveto(580.83868404,169.24544699)(580.81368406,169.32544691)(580.79369062,169.40545191)
\curveto(580.7736841,169.48544675)(580.74868413,169.56044667)(580.71869062,169.63045191)
\curveto(580.43868444,170.28044595)(579.99368488,170.7304455)(579.38369062,170.98045191)
\curveto(579.31368556,171.01044522)(579.23868564,171.0304452)(579.15869062,171.04045191)
\lineto(578.91869062,171.10045191)
\curveto(578.83868604,171.12044511)(578.75368612,171.1304451)(578.66369062,171.13045191)
\lineto(578.39369062,171.13045191)
\lineto(578.12369062,171.08545191)
\curveto(578.02368685,171.06544517)(577.92868695,171.04044519)(577.83869062,171.01045191)
\curveto(577.75868712,170.99044524)(577.6786872,170.96044527)(577.59869062,170.92045191)
\curveto(577.52868735,170.90044533)(577.46368741,170.87044536)(577.40369062,170.83045191)
\curveto(577.34368753,170.79044544)(577.28868759,170.75044548)(577.23869062,170.71045191)
\curveto(576.99868788,170.54044569)(576.80368807,170.3354459)(576.65369062,170.09545191)
\curveto(576.50368837,169.85544638)(576.3736885,169.57544666)(576.26369062,169.25545191)
\curveto(576.23368864,169.15544708)(576.21368866,169.05044718)(576.20369062,168.94045191)
\curveto(576.19368868,168.84044739)(576.1786887,168.7354475)(576.15869062,168.62545191)
\curveto(576.14868873,168.58544765)(576.14368873,168.52044771)(576.14369062,168.43045191)
\curveto(576.13368874,168.40044783)(576.12868875,168.36544787)(576.12869062,168.32545191)
\curveto(576.13868874,168.28544795)(576.14368873,168.24044799)(576.14369062,168.19045191)
\lineto(576.14369062,167.89045191)
\curveto(576.14368873,167.79044844)(576.15368872,167.70044853)(576.17369062,167.62045191)
\lineto(576.20369062,167.44045191)
\curveto(576.22368865,167.34044889)(576.23868864,167.24044899)(576.24869062,167.14045191)
\curveto(576.26868861,167.05044918)(576.29868858,166.96544927)(576.33869062,166.88545191)
\curveto(576.43868844,166.64544959)(576.55368832,166.42044981)(576.68369062,166.21045191)
\curveto(576.82368805,166.00045023)(576.99368788,165.82545041)(577.19369062,165.68545191)
\curveto(577.24368763,165.65545058)(577.28868759,165.6304506)(577.32869062,165.61045191)
\curveto(577.36868751,165.59045064)(577.41368746,165.56545067)(577.46369062,165.53545191)
\curveto(577.54368733,165.48545075)(577.62868725,165.44045079)(577.71869062,165.40045191)
\curveto(577.81868706,165.37045086)(577.92368695,165.34045089)(578.03369062,165.31045191)
\curveto(578.08368679,165.29045094)(578.12868675,165.28045095)(578.16869062,165.28045191)
\curveto(578.21868666,165.29045094)(578.26868661,165.29045094)(578.31869062,165.28045191)
\curveto(578.34868653,165.27045096)(578.40868647,165.26045097)(578.49869062,165.25045191)
\curveto(578.59868628,165.24045099)(578.6736862,165.24545099)(578.72369062,165.26545191)
\curveto(578.76368611,165.27545096)(578.80368607,165.27545096)(578.84369062,165.26545191)
\curveto(578.88368599,165.26545097)(578.92368595,165.27545096)(578.96369062,165.29545191)
\curveto(579.04368583,165.31545092)(579.12368575,165.3304509)(579.20369062,165.34045191)
\curveto(579.28368559,165.36045087)(579.35868552,165.38545085)(579.42869062,165.41545191)
\curveto(579.76868511,165.55545068)(580.04368483,165.75045048)(580.25369062,166.00045191)
\curveto(580.46368441,166.25044998)(580.63868424,166.54544969)(580.77869062,166.88545191)
\curveto(580.82868405,167.00544923)(580.85868402,167.1304491)(580.86869062,167.26045191)
\curveto(580.88868399,167.40044883)(580.91868396,167.54044869)(580.95869062,167.68045191)
}
}
{
\newrgbcolor{curcolor}{0 0 0}
\pscustom[linestyle=none,fillstyle=solid,fillcolor=curcolor]
{
\newpath
\moveto(586.91697187,172.24045191)
\curveto(587.65696708,172.25044398)(588.27196647,172.14044409)(588.76197187,171.91045191)
\curveto(589.26196548,171.69044454)(589.65696508,171.35544488)(589.94697187,170.90545191)
\curveto(590.07696466,170.70544553)(590.18696455,170.46044577)(590.27697187,170.17045191)
\curveto(590.29696444,170.12044611)(590.31196443,170.05544618)(590.32197187,169.97545191)
\curveto(590.33196441,169.89544634)(590.32696441,169.82544641)(590.30697187,169.76545191)
\curveto(590.27696446,169.71544652)(590.22696451,169.67044656)(590.15697187,169.63045191)
\curveto(590.12696461,169.61044662)(590.09696464,169.60044663)(590.06697187,169.60045191)
\curveto(590.0369647,169.61044662)(590.00196474,169.61044662)(589.96197187,169.60045191)
\curveto(589.92196482,169.59044664)(589.88196486,169.58544665)(589.84197187,169.58545191)
\curveto(589.80196494,169.59544664)(589.76196498,169.60044663)(589.72197187,169.60045191)
\lineto(589.40697187,169.60045191)
\curveto(589.30696543,169.61044662)(589.22196552,169.64044659)(589.15197187,169.69045191)
\curveto(589.07196567,169.75044648)(589.01696572,169.8354464)(588.98697187,169.94545191)
\curveto(588.95696578,170.05544618)(588.91696582,170.15044608)(588.86697187,170.23045191)
\curveto(588.71696602,170.49044574)(588.52196622,170.69544554)(588.28197187,170.84545191)
\curveto(588.20196654,170.89544534)(588.11696662,170.9354453)(588.02697187,170.96545191)
\curveto(587.9369668,171.00544523)(587.8419669,171.04044519)(587.74197187,171.07045191)
\curveto(587.60196714,171.11044512)(587.41696732,171.1304451)(587.18697187,171.13045191)
\curveto(586.95696778,171.14044509)(586.76696797,171.12044511)(586.61697187,171.07045191)
\curveto(586.54696819,171.05044518)(586.48196826,171.0354452)(586.42197187,171.02545191)
\curveto(586.36196838,171.01544522)(586.29696844,171.00044523)(586.22697187,170.98045191)
\curveto(585.96696877,170.87044536)(585.736969,170.72044551)(585.53697187,170.53045191)
\curveto(585.3369694,170.34044589)(585.18196956,170.11544612)(585.07197187,169.85545191)
\curveto(585.03196971,169.76544647)(584.99696974,169.67044656)(584.96697187,169.57045191)
\curveto(584.9369698,169.48044675)(584.90696983,169.38044685)(584.87697187,169.27045191)
\lineto(584.78697187,168.86545191)
\curveto(584.77696996,168.81544742)(584.77196997,168.76044747)(584.77197187,168.70045191)
\curveto(584.78196996,168.64044759)(584.77696996,168.58544765)(584.75697187,168.53545191)
\lineto(584.75697187,168.41545191)
\curveto(584.74696999,168.37544786)(584.73697,168.31044792)(584.72697187,168.22045191)
\curveto(584.72697001,168.1304481)(584.73697,168.06544817)(584.75697187,168.02545191)
\curveto(584.76696997,167.97544826)(584.76696997,167.92544831)(584.75697187,167.87545191)
\curveto(584.74696999,167.82544841)(584.74696999,167.77544846)(584.75697187,167.72545191)
\curveto(584.76696997,167.68544855)(584.77196997,167.61544862)(584.77197187,167.51545191)
\curveto(584.79196995,167.4354488)(584.80696993,167.35044888)(584.81697187,167.26045191)
\curveto(584.8369699,167.17044906)(584.85696988,167.08544915)(584.87697187,167.00545191)
\curveto(584.98696975,166.68544955)(585.11196963,166.40544983)(585.25197187,166.16545191)
\curveto(585.40196934,165.9354503)(585.60696913,165.7354505)(585.86697187,165.56545191)
\curveto(585.95696878,165.51545072)(586.04696869,165.47045076)(586.13697187,165.43045191)
\curveto(586.2369685,165.39045084)(586.3419684,165.35045088)(586.45197187,165.31045191)
\curveto(586.50196824,165.30045093)(586.5419682,165.29545094)(586.57197187,165.29545191)
\curveto(586.60196814,165.29545094)(586.6419681,165.29045094)(586.69197187,165.28045191)
\curveto(586.72196802,165.27045096)(586.77196797,165.26545097)(586.84197187,165.26545191)
\lineto(587.00697187,165.26545191)
\curveto(587.00696773,165.25545098)(587.02696771,165.25045098)(587.06697187,165.25045191)
\curveto(587.08696765,165.26045097)(587.11196763,165.26045097)(587.14197187,165.25045191)
\curveto(587.17196757,165.25045098)(587.20196754,165.25545098)(587.23197187,165.26545191)
\curveto(587.30196744,165.28545095)(587.36696737,165.29045094)(587.42697187,165.28045191)
\curveto(587.49696724,165.28045095)(587.56696717,165.29045094)(587.63697187,165.31045191)
\curveto(587.89696684,165.39045084)(588.12196662,165.49045074)(588.31197187,165.61045191)
\curveto(588.50196624,165.74045049)(588.66196608,165.90545033)(588.79197187,166.10545191)
\curveto(588.8419659,166.18545005)(588.88696585,166.27044996)(588.92697187,166.36045191)
\lineto(589.04697187,166.63045191)
\curveto(589.06696567,166.71044952)(589.08696565,166.78544945)(589.10697187,166.85545191)
\curveto(589.1369656,166.9354493)(589.18696555,167.00044923)(589.25697187,167.05045191)
\curveto(589.28696545,167.08044915)(589.34696539,167.10044913)(589.43697187,167.11045191)
\curveto(589.52696521,167.1304491)(589.62196512,167.14044909)(589.72197187,167.14045191)
\curveto(589.83196491,167.15044908)(589.93196481,167.15044908)(590.02197187,167.14045191)
\curveto(590.12196462,167.1304491)(590.19196455,167.11044912)(590.23197187,167.08045191)
\curveto(590.29196445,167.04044919)(590.32696441,166.98044925)(590.33697187,166.90045191)
\curveto(590.35696438,166.82044941)(590.35696438,166.7354495)(590.33697187,166.64545191)
\curveto(590.28696445,166.49544974)(590.2369645,166.35044988)(590.18697187,166.21045191)
\curveto(590.14696459,166.08045015)(590.09196465,165.95045028)(590.02197187,165.82045191)
\curveto(589.87196487,165.52045071)(589.68196506,165.25545098)(589.45197187,165.02545191)
\curveto(589.23196551,164.79545144)(588.96196578,164.61045162)(588.64197187,164.47045191)
\curveto(588.56196618,164.4304518)(588.47696626,164.39545184)(588.38697187,164.36545191)
\curveto(588.29696644,164.34545189)(588.20196654,164.32045191)(588.10197187,164.29045191)
\curveto(587.99196675,164.25045198)(587.88196686,164.230452)(587.77197187,164.23045191)
\curveto(587.66196708,164.22045201)(587.55196719,164.20545203)(587.44197187,164.18545191)
\curveto(587.40196734,164.16545207)(587.36196738,164.16045207)(587.32197187,164.17045191)
\curveto(587.28196746,164.18045205)(587.2419675,164.18045205)(587.20197187,164.17045191)
\lineto(587.06697187,164.17045191)
\lineto(586.82697187,164.17045191)
\curveto(586.75696798,164.16045207)(586.69196805,164.16545207)(586.63197187,164.18545191)
\lineto(586.55697187,164.18545191)
\lineto(586.19697187,164.23045191)
\curveto(586.06696867,164.27045196)(585.9419688,164.30545193)(585.82197187,164.33545191)
\curveto(585.70196904,164.36545187)(585.58696915,164.40545183)(585.47697187,164.45545191)
\curveto(585.11696962,164.61545162)(584.81696992,164.80545143)(584.57697187,165.02545191)
\curveto(584.34697039,165.24545099)(584.13197061,165.51545072)(583.93197187,165.83545191)
\curveto(583.88197086,165.91545032)(583.8369709,166.00545023)(583.79697187,166.10545191)
\lineto(583.67697187,166.40545191)
\curveto(583.62697111,166.51544972)(583.59197115,166.6304496)(583.57197187,166.75045191)
\curveto(583.55197119,166.87044936)(583.52697121,166.99044924)(583.49697187,167.11045191)
\curveto(583.48697125,167.15044908)(583.48197126,167.19044904)(583.48197187,167.23045191)
\curveto(583.48197126,167.27044896)(583.47697126,167.31044892)(583.46697187,167.35045191)
\curveto(583.44697129,167.41044882)(583.4369713,167.47544876)(583.43697187,167.54545191)
\curveto(583.44697129,167.61544862)(583.4419713,167.68044855)(583.42197187,167.74045191)
\lineto(583.42197187,167.89045191)
\curveto(583.41197133,167.94044829)(583.40697133,168.01044822)(583.40697187,168.10045191)
\curveto(583.40697133,168.19044804)(583.41197133,168.26044797)(583.42197187,168.31045191)
\curveto(583.43197131,168.36044787)(583.43197131,168.40544783)(583.42197187,168.44545191)
\curveto(583.42197132,168.48544775)(583.42697131,168.52544771)(583.43697187,168.56545191)
\curveto(583.45697128,168.6354476)(583.46197128,168.70544753)(583.45197187,168.77545191)
\curveto(583.45197129,168.84544739)(583.46197128,168.91044732)(583.48197187,168.97045191)
\curveto(583.52197122,169.14044709)(583.55697118,169.31044692)(583.58697187,169.48045191)
\curveto(583.61697112,169.65044658)(583.66197108,169.81044642)(583.72197187,169.96045191)
\curveto(583.93197081,170.48044575)(584.18697055,170.90044533)(584.48697187,171.22045191)
\curveto(584.78696995,171.54044469)(585.19696954,171.80544443)(585.71697187,172.01545191)
\curveto(585.82696891,172.06544417)(585.94696879,172.10044413)(586.07697187,172.12045191)
\curveto(586.20696853,172.14044409)(586.3419684,172.16544407)(586.48197187,172.19545191)
\curveto(586.55196819,172.20544403)(586.62196812,172.21044402)(586.69197187,172.21045191)
\curveto(586.76196798,172.22044401)(586.8369679,172.230444)(586.91697187,172.24045191)
}
}
{
\newrgbcolor{curcolor}{0 0 0}
\pscustom[linestyle=none,fillstyle=solid,fillcolor=curcolor]
{
\newpath
\moveto(598.5886125,168.50545191)
\curveto(598.60860481,168.40544783)(598.60860481,168.29044794)(598.5886125,168.16045191)
\curveto(598.57860484,168.04044819)(598.54860487,167.95544828)(598.4986125,167.90545191)
\curveto(598.44860497,167.86544837)(598.37360505,167.8354484)(598.2736125,167.81545191)
\curveto(598.18360524,167.80544843)(598.07860534,167.80044843)(597.9586125,167.80045191)
\lineto(597.5986125,167.80045191)
\curveto(597.47860594,167.81044842)(597.37360605,167.81544842)(597.2836125,167.81545191)
\lineto(593.4436125,167.81545191)
\curveto(593.36361006,167.81544842)(593.28361014,167.81044842)(593.2036125,167.80045191)
\curveto(593.1236103,167.80044843)(593.05861036,167.78544845)(593.0086125,167.75545191)
\curveto(592.96861045,167.7354485)(592.92861049,167.69544854)(592.8886125,167.63545191)
\curveto(592.86861055,167.60544863)(592.84861057,167.56044867)(592.8286125,167.50045191)
\curveto(592.80861061,167.45044878)(592.80861061,167.40044883)(592.8286125,167.35045191)
\curveto(592.83861058,167.30044893)(592.84361058,167.25544898)(592.8436125,167.21545191)
\curveto(592.84361058,167.17544906)(592.84861057,167.1354491)(592.8586125,167.09545191)
\curveto(592.87861054,167.01544922)(592.89861052,166.9304493)(592.9186125,166.84045191)
\curveto(592.93861048,166.76044947)(592.96861045,166.68044955)(593.0086125,166.60045191)
\curveto(593.23861018,166.06045017)(593.6186098,165.67545056)(594.1486125,165.44545191)
\curveto(594.20860921,165.41545082)(594.27360915,165.39045084)(594.3436125,165.37045191)
\lineto(594.5536125,165.31045191)
\curveto(594.58360884,165.30045093)(594.63360879,165.29545094)(594.7036125,165.29545191)
\curveto(594.84360858,165.25545098)(595.02860839,165.235451)(595.2586125,165.23545191)
\curveto(595.48860793,165.235451)(595.67360775,165.25545098)(595.8136125,165.29545191)
\curveto(595.95360747,165.3354509)(596.07860734,165.37545086)(596.1886125,165.41545191)
\curveto(596.30860711,165.46545077)(596.418607,165.52545071)(596.5186125,165.59545191)
\curveto(596.62860679,165.66545057)(596.7236067,165.74545049)(596.8036125,165.83545191)
\curveto(596.88360654,165.9354503)(596.95360647,166.04045019)(597.0136125,166.15045191)
\curveto(597.07360635,166.25044998)(597.1236063,166.35544988)(597.1636125,166.46545191)
\curveto(597.21360621,166.57544966)(597.29360613,166.65544958)(597.4036125,166.70545191)
\curveto(597.44360598,166.72544951)(597.50860591,166.74044949)(597.5986125,166.75045191)
\curveto(597.68860573,166.76044947)(597.77860564,166.76044947)(597.8686125,166.75045191)
\curveto(597.95860546,166.75044948)(598.04360538,166.74544949)(598.1236125,166.73545191)
\curveto(598.20360522,166.72544951)(598.25860516,166.70544953)(598.2886125,166.67545191)
\curveto(598.38860503,166.60544963)(598.41360501,166.49044974)(598.3636125,166.33045191)
\curveto(598.28360514,166.06045017)(598.17860524,165.82045041)(598.0486125,165.61045191)
\curveto(597.84860557,165.29045094)(597.6186058,165.02545121)(597.3586125,164.81545191)
\curveto(597.10860631,164.61545162)(596.78860663,164.45045178)(596.3986125,164.32045191)
\curveto(596.29860712,164.28045195)(596.19860722,164.25545198)(596.0986125,164.24545191)
\curveto(595.99860742,164.22545201)(595.89360753,164.20545203)(595.7836125,164.18545191)
\curveto(595.73360769,164.17545206)(595.68360774,164.17045206)(595.6336125,164.17045191)
\curveto(595.59360783,164.17045206)(595.54860787,164.16545207)(595.4986125,164.15545191)
\lineto(595.3486125,164.15545191)
\curveto(595.29860812,164.14545209)(595.23860818,164.14045209)(595.1686125,164.14045191)
\curveto(595.10860831,164.14045209)(595.05860836,164.14545209)(595.0186125,164.15545191)
\lineto(594.8836125,164.15545191)
\curveto(594.83360859,164.16545207)(594.78860863,164.17045206)(594.7486125,164.17045191)
\curveto(594.70860871,164.17045206)(594.66860875,164.17545206)(594.6286125,164.18545191)
\curveto(594.57860884,164.19545204)(594.5236089,164.20545203)(594.4636125,164.21545191)
\curveto(594.40360902,164.21545202)(594.34860907,164.22045201)(594.2986125,164.23045191)
\curveto(594.20860921,164.25045198)(594.1186093,164.27545196)(594.0286125,164.30545191)
\curveto(593.93860948,164.32545191)(593.85360957,164.35045188)(593.7736125,164.38045191)
\curveto(593.73360969,164.40045183)(593.69860972,164.41045182)(593.6686125,164.41045191)
\curveto(593.63860978,164.42045181)(593.60360982,164.4354518)(593.5636125,164.45545191)
\curveto(593.41361001,164.52545171)(593.25361017,164.61045162)(593.0836125,164.71045191)
\curveto(592.79361063,164.90045133)(592.54361088,165.1304511)(592.3336125,165.40045191)
\curveto(592.13361129,165.68045055)(591.96361146,165.99045024)(591.8236125,166.33045191)
\curveto(591.77361165,166.44044979)(591.73361169,166.55544968)(591.7036125,166.67545191)
\curveto(591.68361174,166.79544944)(591.65361177,166.91544932)(591.6136125,167.03545191)
\curveto(591.60361182,167.07544916)(591.59861182,167.11044912)(591.5986125,167.14045191)
\curveto(591.59861182,167.17044906)(591.59361183,167.21044902)(591.5836125,167.26045191)
\curveto(591.56361186,167.34044889)(591.54861187,167.42544881)(591.5386125,167.51545191)
\curveto(591.52861189,167.60544863)(591.51361191,167.69544854)(591.4936125,167.78545191)
\lineto(591.4936125,167.99545191)
\curveto(591.48361194,168.0354482)(591.47361195,168.09044814)(591.4636125,168.16045191)
\curveto(591.46361196,168.24044799)(591.46861195,168.30544793)(591.4786125,168.35545191)
\lineto(591.4786125,168.52045191)
\curveto(591.49861192,168.57044766)(591.50361192,168.62044761)(591.4936125,168.67045191)
\curveto(591.49361193,168.7304475)(591.49861192,168.78544745)(591.5086125,168.83545191)
\curveto(591.54861187,168.99544724)(591.57861184,169.15544708)(591.5986125,169.31545191)
\curveto(591.62861179,169.47544676)(591.67361175,169.62544661)(591.7336125,169.76545191)
\curveto(591.78361164,169.87544636)(591.82861159,169.98544625)(591.8686125,170.09545191)
\curveto(591.9186115,170.21544602)(591.97361145,170.3304459)(592.0336125,170.44045191)
\curveto(592.25361117,170.79044544)(592.50361092,171.09044514)(592.7836125,171.34045191)
\curveto(593.06361036,171.60044463)(593.40861001,171.81544442)(593.8186125,171.98545191)
\curveto(593.93860948,172.0354442)(594.05860936,172.07044416)(594.1786125,172.09045191)
\curveto(594.30860911,172.12044411)(594.44360898,172.15044408)(594.5836125,172.18045191)
\curveto(594.63360879,172.19044404)(594.67860874,172.19544404)(594.7186125,172.19545191)
\curveto(594.75860866,172.20544403)(594.80360862,172.21044402)(594.8536125,172.21045191)
\curveto(594.87360855,172.22044401)(594.89860852,172.22044401)(594.9286125,172.21045191)
\curveto(594.95860846,172.20044403)(594.98360844,172.20544403)(595.0036125,172.22545191)
\curveto(595.423608,172.235444)(595.78860763,172.19044404)(596.0986125,172.09045191)
\curveto(596.40860701,172.00044423)(596.68860673,171.87544436)(596.9386125,171.71545191)
\curveto(596.98860643,171.69544454)(597.02860639,171.66544457)(597.0586125,171.62545191)
\curveto(597.08860633,171.59544464)(597.1236063,171.57044466)(597.1636125,171.55045191)
\curveto(597.24360618,171.49044474)(597.3236061,171.42044481)(597.4036125,171.34045191)
\curveto(597.49360593,171.26044497)(597.56860585,171.18044505)(597.6286125,171.10045191)
\curveto(597.78860563,170.89044534)(597.9236055,170.69044554)(598.0336125,170.50045191)
\curveto(598.10360532,170.39044584)(598.15860526,170.27044596)(598.1986125,170.14045191)
\curveto(598.23860518,170.01044622)(598.28360514,169.88044635)(598.3336125,169.75045191)
\curveto(598.38360504,169.62044661)(598.418605,169.48544675)(598.4386125,169.34545191)
\curveto(598.46860495,169.20544703)(598.50360492,169.06544717)(598.5436125,168.92545191)
\curveto(598.55360487,168.85544738)(598.55860486,168.78544745)(598.5586125,168.71545191)
\lineto(598.5886125,168.50545191)
\moveto(597.1336125,169.01545191)
\curveto(597.16360626,169.05544718)(597.18860623,169.10544713)(597.2086125,169.16545191)
\curveto(597.22860619,169.235447)(597.22860619,169.30544693)(597.2086125,169.37545191)
\curveto(597.14860627,169.59544664)(597.06360636,169.80044643)(596.9536125,169.99045191)
\curveto(596.81360661,170.22044601)(596.65860676,170.41544582)(596.4886125,170.57545191)
\curveto(596.3186071,170.7354455)(596.09860732,170.87044536)(595.8286125,170.98045191)
\curveto(595.75860766,171.00044523)(595.68860773,171.01544522)(595.6186125,171.02545191)
\curveto(595.54860787,171.04544519)(595.47360795,171.06544517)(595.3936125,171.08545191)
\curveto(595.31360811,171.10544513)(595.22860819,171.11544512)(595.1386125,171.11545191)
\lineto(594.8836125,171.11545191)
\curveto(594.85360857,171.09544514)(594.8186086,171.08544515)(594.7786125,171.08545191)
\curveto(594.73860868,171.09544514)(594.70360872,171.09544514)(594.6736125,171.08545191)
\lineto(594.4336125,171.02545191)
\curveto(594.36360906,171.01544522)(594.29360913,171.00044523)(594.2236125,170.98045191)
\curveto(593.93360949,170.86044537)(593.69860972,170.71044552)(593.5186125,170.53045191)
\curveto(593.34861007,170.35044588)(593.19361023,170.12544611)(593.0536125,169.85545191)
\curveto(593.0236104,169.80544643)(592.99361043,169.74044649)(592.9636125,169.66045191)
\curveto(592.93361049,169.59044664)(592.90861051,169.51044672)(592.8886125,169.42045191)
\curveto(592.86861055,169.3304469)(592.86361056,169.24544699)(592.8736125,169.16545191)
\curveto(592.88361054,169.08544715)(592.9186105,169.02544721)(592.9786125,168.98545191)
\curveto(593.05861036,168.92544731)(593.19361023,168.89544734)(593.3836125,168.89545191)
\curveto(593.58360984,168.90544733)(593.75360967,168.91044732)(593.8936125,168.91045191)
\lineto(596.1736125,168.91045191)
\curveto(596.3236071,168.91044732)(596.50360692,168.90544733)(596.7136125,168.89545191)
\curveto(596.9236065,168.89544734)(597.06360636,168.9354473)(597.1336125,169.01545191)
}
}
{
\newrgbcolor{curcolor}{0 0 0}
\pscustom[linestyle=none,fillstyle=solid,fillcolor=curcolor]
{
\newpath
\moveto(603.58525312,172.21045191)
\curveto(604.21524789,172.230444)(604.72024738,172.14544409)(605.10025312,171.95545191)
\curveto(605.48024662,171.76544447)(605.78524632,171.48044475)(606.01525312,171.10045191)
\curveto(606.07524603,171.00044523)(606.12024598,170.89044534)(606.15025312,170.77045191)
\curveto(606.19024591,170.66044557)(606.22524588,170.54544569)(606.25525312,170.42545191)
\curveto(606.3052458,170.235446)(606.33524577,170.0304462)(606.34525312,169.81045191)
\curveto(606.35524575,169.59044664)(606.36024574,169.36544687)(606.36025312,169.13545191)
\lineto(606.36025312,167.53045191)
\lineto(606.36025312,165.19045191)
\curveto(606.36024574,165.02045121)(606.35524575,164.85045138)(606.34525312,164.68045191)
\curveto(606.34524576,164.51045172)(606.28024582,164.40045183)(606.15025312,164.35045191)
\curveto(606.100246,164.3304519)(606.04524606,164.32045191)(605.98525312,164.32045191)
\curveto(605.93524617,164.31045192)(605.88024622,164.30545193)(605.82025312,164.30545191)
\curveto(605.69024641,164.30545193)(605.56524654,164.31045192)(605.44525312,164.32045191)
\curveto(605.32524678,164.32045191)(605.24024686,164.36045187)(605.19025312,164.44045191)
\curveto(605.14024696,164.51045172)(605.11524699,164.60045163)(605.11525312,164.71045191)
\lineto(605.11525312,165.04045191)
\lineto(605.11525312,166.33045191)
\lineto(605.11525312,168.77545191)
\curveto(605.11524699,169.04544719)(605.11024699,169.31044692)(605.10025312,169.57045191)
\curveto(605.09024701,169.84044639)(605.04524706,170.07044616)(604.96525312,170.26045191)
\curveto(604.88524722,170.46044577)(604.76524734,170.62044561)(604.60525312,170.74045191)
\curveto(604.44524766,170.87044536)(604.26024784,170.97044526)(604.05025312,171.04045191)
\curveto(603.99024811,171.06044517)(603.92524818,171.07044516)(603.85525312,171.07045191)
\curveto(603.79524831,171.08044515)(603.73524837,171.09544514)(603.67525312,171.11545191)
\curveto(603.62524848,171.12544511)(603.54524856,171.12544511)(603.43525312,171.11545191)
\curveto(603.33524877,171.11544512)(603.26524884,171.11044512)(603.22525312,171.10045191)
\curveto(603.18524892,171.08044515)(603.15024895,171.07044516)(603.12025312,171.07045191)
\curveto(603.09024901,171.08044515)(603.05524905,171.08044515)(603.01525312,171.07045191)
\curveto(602.88524922,171.04044519)(602.76024934,171.00544523)(602.64025312,170.96545191)
\curveto(602.53024957,170.9354453)(602.42524968,170.89044534)(602.32525312,170.83045191)
\curveto(602.28524982,170.81044542)(602.25024985,170.79044544)(602.22025312,170.77045191)
\curveto(602.19024991,170.75044548)(602.15524995,170.7304455)(602.11525312,170.71045191)
\curveto(601.76525034,170.46044577)(601.51025059,170.08544615)(601.35025312,169.58545191)
\curveto(601.32025078,169.50544673)(601.3002508,169.42044681)(601.29025312,169.33045191)
\curveto(601.28025082,169.25044698)(601.26525084,169.17044706)(601.24525312,169.09045191)
\curveto(601.22525088,169.04044719)(601.22025088,168.99044724)(601.23025312,168.94045191)
\curveto(601.24025086,168.90044733)(601.23525087,168.86044737)(601.21525312,168.82045191)
\lineto(601.21525312,168.50545191)
\curveto(601.2052509,168.47544776)(601.2002509,168.44044779)(601.20025312,168.40045191)
\curveto(601.21025089,168.36044787)(601.21525089,168.31544792)(601.21525312,168.26545191)
\lineto(601.21525312,167.81545191)
\lineto(601.21525312,166.37545191)
\lineto(601.21525312,165.05545191)
\lineto(601.21525312,164.71045191)
\curveto(601.21525089,164.60045163)(601.19025091,164.51045172)(601.14025312,164.44045191)
\curveto(601.09025101,164.36045187)(601.0002511,164.32045191)(600.87025312,164.32045191)
\curveto(600.75025135,164.31045192)(600.62525148,164.30545193)(600.49525312,164.30545191)
\curveto(600.41525169,164.30545193)(600.34025176,164.31045192)(600.27025312,164.32045191)
\curveto(600.2002519,164.3304519)(600.14025196,164.35545188)(600.09025312,164.39545191)
\curveto(600.01025209,164.44545179)(599.97025213,164.54045169)(599.97025312,164.68045191)
\lineto(599.97025312,165.08545191)
\lineto(599.97025312,166.85545191)
\lineto(599.97025312,170.48545191)
\lineto(599.97025312,171.40045191)
\lineto(599.97025312,171.67045191)
\curveto(599.97025213,171.76044447)(599.99025211,171.8304444)(600.03025312,171.88045191)
\curveto(600.06025204,171.94044429)(600.11025199,171.98044425)(600.18025312,172.00045191)
\curveto(600.22025188,172.01044422)(600.27525183,172.02044421)(600.34525312,172.03045191)
\curveto(600.42525168,172.04044419)(600.5052516,172.04544419)(600.58525312,172.04545191)
\curveto(600.66525144,172.04544419)(600.74025136,172.04044419)(600.81025312,172.03045191)
\curveto(600.89025121,172.02044421)(600.94525116,172.00544423)(600.97525312,171.98545191)
\curveto(601.08525102,171.91544432)(601.13525097,171.82544441)(601.12525312,171.71545191)
\curveto(601.11525099,171.61544462)(601.13025097,171.50044473)(601.17025312,171.37045191)
\curveto(601.19025091,171.31044492)(601.23025087,171.26044497)(601.29025312,171.22045191)
\curveto(601.41025069,171.21044502)(601.5052506,171.25544498)(601.57525312,171.35545191)
\curveto(601.65525045,171.45544478)(601.73525037,171.5354447)(601.81525312,171.59545191)
\curveto(601.95525015,171.69544454)(602.09525001,171.78544445)(602.23525312,171.86545191)
\curveto(602.38524972,171.95544428)(602.55524955,172.0304442)(602.74525312,172.09045191)
\curveto(602.82524928,172.12044411)(602.91024919,172.14044409)(603.00025312,172.15045191)
\curveto(603.100249,172.16044407)(603.19524891,172.17544406)(603.28525312,172.19545191)
\curveto(603.33524877,172.20544403)(603.38524872,172.21044402)(603.43525312,172.21045191)
\lineto(603.58525312,172.21045191)
}
}
{
\newrgbcolor{curcolor}{0 0 0}
\pscustom[linestyle=none,fillstyle=solid,fillcolor=curcolor]
{
\newpath
\moveto(609.1898625,174.40045191)
\curveto(609.33986049,174.40044183)(609.48986034,174.39544184)(609.6398625,174.38545191)
\curveto(609.78986004,174.38544185)(609.89485993,174.34544189)(609.9548625,174.26545191)
\curveto(610.00485982,174.20544203)(610.0298598,174.12044211)(610.0298625,174.01045191)
\curveto(610.03985979,173.91044232)(610.04485978,173.80544243)(610.0448625,173.69545191)
\lineto(610.0448625,172.82545191)
\curveto(610.04485978,172.74544349)(610.03985979,172.66044357)(610.0298625,172.57045191)
\curveto(610.0298598,172.49044374)(610.03985979,172.42044381)(610.0598625,172.36045191)
\curveto(610.09985973,172.22044401)(610.18985964,172.1304441)(610.3298625,172.09045191)
\curveto(610.37985945,172.08044415)(610.4248594,172.07544416)(610.4648625,172.07545191)
\lineto(610.6148625,172.07545191)
\lineto(611.0198625,172.07545191)
\curveto(611.17985865,172.08544415)(611.29485853,172.07544416)(611.3648625,172.04545191)
\curveto(611.45485837,171.98544425)(611.51485831,171.92544431)(611.5448625,171.86545191)
\curveto(611.56485826,171.82544441)(611.57485825,171.78044445)(611.5748625,171.73045191)
\lineto(611.5748625,171.58045191)
\curveto(611.57485825,171.47044476)(611.56985826,171.36544487)(611.5598625,171.26545191)
\curveto(611.54985828,171.17544506)(611.51485831,171.10544513)(611.4548625,171.05545191)
\curveto(611.39485843,171.00544523)(611.30985852,170.97544526)(611.1998625,170.96545191)
\lineto(610.8698625,170.96545191)
\curveto(610.75985907,170.97544526)(610.64985918,170.98044525)(610.5398625,170.98045191)
\curveto(610.4298594,170.98044525)(610.33485949,170.96544527)(610.2548625,170.93545191)
\curveto(610.18485964,170.90544533)(610.13485969,170.85544538)(610.1048625,170.78545191)
\curveto(610.07485975,170.71544552)(610.05485977,170.6304456)(610.0448625,170.53045191)
\curveto(610.03485979,170.44044579)(610.0298598,170.34044589)(610.0298625,170.23045191)
\curveto(610.03985979,170.1304461)(610.04485978,170.0304462)(610.0448625,169.93045191)
\lineto(610.0448625,166.96045191)
\curveto(610.04485978,166.74044949)(610.03985979,166.50544973)(610.0298625,166.25545191)
\curveto(610.0298598,166.01545022)(610.07485975,165.8304504)(610.1648625,165.70045191)
\curveto(610.21485961,165.62045061)(610.27985955,165.56545067)(610.3598625,165.53545191)
\curveto(610.43985939,165.50545073)(610.53485929,165.48045075)(610.6448625,165.46045191)
\curveto(610.67485915,165.45045078)(610.70485912,165.44545079)(610.7348625,165.44545191)
\curveto(610.77485905,165.45545078)(610.80985902,165.45545078)(610.8398625,165.44545191)
\lineto(611.0348625,165.44545191)
\curveto(611.13485869,165.44545079)(611.2248586,165.4354508)(611.3048625,165.41545191)
\curveto(611.39485843,165.40545083)(611.45985837,165.37045086)(611.4998625,165.31045191)
\curveto(611.51985831,165.28045095)(611.53485829,165.22545101)(611.5448625,165.14545191)
\curveto(611.56485826,165.07545116)(611.57485825,165.00045123)(611.5748625,164.92045191)
\curveto(611.58485824,164.84045139)(611.58485824,164.76045147)(611.5748625,164.68045191)
\curveto(611.56485826,164.61045162)(611.54485828,164.55545168)(611.5148625,164.51545191)
\curveto(611.47485835,164.44545179)(611.39985843,164.39545184)(611.2898625,164.36545191)
\curveto(611.20985862,164.34545189)(611.11985871,164.3354519)(611.0198625,164.33545191)
\curveto(610.91985891,164.34545189)(610.829859,164.35045188)(610.7498625,164.35045191)
\curveto(610.68985914,164.35045188)(610.6298592,164.34545189)(610.5698625,164.33545191)
\curveto(610.50985932,164.3354519)(610.45485937,164.34045189)(610.4048625,164.35045191)
\lineto(610.2248625,164.35045191)
\curveto(610.17485965,164.36045187)(610.1248597,164.36545187)(610.0748625,164.36545191)
\curveto(610.03485979,164.37545186)(609.98985984,164.38045185)(609.9398625,164.38045191)
\curveto(609.73986009,164.4304518)(609.56486026,164.48545175)(609.4148625,164.54545191)
\curveto(609.27486055,164.60545163)(609.15486067,164.71045152)(609.0548625,164.86045191)
\curveto(608.91486091,165.06045117)(608.83486099,165.31045092)(608.8148625,165.61045191)
\curveto(608.79486103,165.92045031)(608.78486104,166.25044998)(608.7848625,166.60045191)
\lineto(608.7848625,170.53045191)
\curveto(608.75486107,170.66044557)(608.7248611,170.75544548)(608.6948625,170.81545191)
\curveto(608.67486115,170.87544536)(608.60486122,170.92544531)(608.4848625,170.96545191)
\curveto(608.44486138,170.97544526)(608.40486142,170.97544526)(608.3648625,170.96545191)
\curveto(608.3248615,170.95544528)(608.28486154,170.96044527)(608.2448625,170.98045191)
\lineto(608.0048625,170.98045191)
\curveto(607.87486195,170.98044525)(607.76486206,170.99044524)(607.6748625,171.01045191)
\curveto(607.59486223,171.04044519)(607.53986229,171.10044513)(607.5098625,171.19045191)
\curveto(607.48986234,171.230445)(607.47486235,171.27544496)(607.4648625,171.32545191)
\lineto(607.4648625,171.47545191)
\curveto(607.46486236,171.61544462)(607.47486235,171.7304445)(607.4948625,171.82045191)
\curveto(607.51486231,171.92044431)(607.57486225,171.99544424)(607.6748625,172.04545191)
\curveto(607.78486204,172.08544415)(607.9248619,172.09544414)(608.0948625,172.07545191)
\curveto(608.27486155,172.05544418)(608.4248614,172.06544417)(608.5448625,172.10545191)
\curveto(608.63486119,172.15544408)(608.70486112,172.22544401)(608.7548625,172.31545191)
\curveto(608.77486105,172.37544386)(608.78486104,172.45044378)(608.7848625,172.54045191)
\lineto(608.7848625,172.79545191)
\lineto(608.7848625,173.72545191)
\lineto(608.7848625,173.96545191)
\curveto(608.78486104,174.05544218)(608.79486103,174.1304421)(608.8148625,174.19045191)
\curveto(608.85486097,174.27044196)(608.9298609,174.3354419)(609.0398625,174.38545191)
\curveto(609.06986076,174.38544185)(609.09486073,174.38544185)(609.1148625,174.38545191)
\curveto(609.14486068,174.39544184)(609.16986066,174.40044183)(609.1898625,174.40045191)
}
}
{
\newrgbcolor{curcolor}{0 0 0}
\pscustom[linestyle=none,fillstyle=solid,fillcolor=curcolor]
{
\newpath
\moveto(619.71165937,168.50545191)
\curveto(619.73165169,168.40544783)(619.73165169,168.29044794)(619.71165937,168.16045191)
\curveto(619.70165172,168.04044819)(619.67165175,167.95544828)(619.62165937,167.90545191)
\curveto(619.57165185,167.86544837)(619.49665192,167.8354484)(619.39665937,167.81545191)
\curveto(619.30665211,167.80544843)(619.20165222,167.80044843)(619.08165937,167.80045191)
\lineto(618.72165937,167.80045191)
\curveto(618.60165282,167.81044842)(618.49665292,167.81544842)(618.40665937,167.81545191)
\lineto(614.56665937,167.81545191)
\curveto(614.48665693,167.81544842)(614.40665701,167.81044842)(614.32665937,167.80045191)
\curveto(614.24665717,167.80044843)(614.18165724,167.78544845)(614.13165937,167.75545191)
\curveto(614.09165733,167.7354485)(614.05165737,167.69544854)(614.01165937,167.63545191)
\curveto(613.99165743,167.60544863)(613.97165745,167.56044867)(613.95165937,167.50045191)
\curveto(613.93165749,167.45044878)(613.93165749,167.40044883)(613.95165937,167.35045191)
\curveto(613.96165746,167.30044893)(613.96665745,167.25544898)(613.96665937,167.21545191)
\curveto(613.96665745,167.17544906)(613.97165745,167.1354491)(613.98165937,167.09545191)
\curveto(614.00165742,167.01544922)(614.0216574,166.9304493)(614.04165937,166.84045191)
\curveto(614.06165736,166.76044947)(614.09165733,166.68044955)(614.13165937,166.60045191)
\curveto(614.36165706,166.06045017)(614.74165668,165.67545056)(615.27165937,165.44545191)
\curveto(615.33165609,165.41545082)(615.39665602,165.39045084)(615.46665937,165.37045191)
\lineto(615.67665937,165.31045191)
\curveto(615.70665571,165.30045093)(615.75665566,165.29545094)(615.82665937,165.29545191)
\curveto(615.96665545,165.25545098)(616.15165527,165.235451)(616.38165937,165.23545191)
\curveto(616.61165481,165.235451)(616.79665462,165.25545098)(616.93665937,165.29545191)
\curveto(617.07665434,165.3354509)(617.20165422,165.37545086)(617.31165937,165.41545191)
\curveto(617.43165399,165.46545077)(617.54165388,165.52545071)(617.64165937,165.59545191)
\curveto(617.75165367,165.66545057)(617.84665357,165.74545049)(617.92665937,165.83545191)
\curveto(618.00665341,165.9354503)(618.07665334,166.04045019)(618.13665937,166.15045191)
\curveto(618.19665322,166.25044998)(618.24665317,166.35544988)(618.28665937,166.46545191)
\curveto(618.33665308,166.57544966)(618.416653,166.65544958)(618.52665937,166.70545191)
\curveto(618.56665285,166.72544951)(618.63165279,166.74044949)(618.72165937,166.75045191)
\curveto(618.81165261,166.76044947)(618.90165252,166.76044947)(618.99165937,166.75045191)
\curveto(619.08165234,166.75044948)(619.16665225,166.74544949)(619.24665937,166.73545191)
\curveto(619.32665209,166.72544951)(619.38165204,166.70544953)(619.41165937,166.67545191)
\curveto(619.51165191,166.60544963)(619.53665188,166.49044974)(619.48665937,166.33045191)
\curveto(619.40665201,166.06045017)(619.30165212,165.82045041)(619.17165937,165.61045191)
\curveto(618.97165245,165.29045094)(618.74165268,165.02545121)(618.48165937,164.81545191)
\curveto(618.23165319,164.61545162)(617.91165351,164.45045178)(617.52165937,164.32045191)
\curveto(617.421654,164.28045195)(617.3216541,164.25545198)(617.22165937,164.24545191)
\curveto(617.1216543,164.22545201)(617.0166544,164.20545203)(616.90665937,164.18545191)
\curveto(616.85665456,164.17545206)(616.80665461,164.17045206)(616.75665937,164.17045191)
\curveto(616.7166547,164.17045206)(616.67165475,164.16545207)(616.62165937,164.15545191)
\lineto(616.47165937,164.15545191)
\curveto(616.421655,164.14545209)(616.36165506,164.14045209)(616.29165937,164.14045191)
\curveto(616.23165519,164.14045209)(616.18165524,164.14545209)(616.14165937,164.15545191)
\lineto(616.00665937,164.15545191)
\curveto(615.95665546,164.16545207)(615.91165551,164.17045206)(615.87165937,164.17045191)
\curveto(615.83165559,164.17045206)(615.79165563,164.17545206)(615.75165937,164.18545191)
\curveto(615.70165572,164.19545204)(615.64665577,164.20545203)(615.58665937,164.21545191)
\curveto(615.52665589,164.21545202)(615.47165595,164.22045201)(615.42165937,164.23045191)
\curveto(615.33165609,164.25045198)(615.24165618,164.27545196)(615.15165937,164.30545191)
\curveto(615.06165636,164.32545191)(614.97665644,164.35045188)(614.89665937,164.38045191)
\curveto(614.85665656,164.40045183)(614.8216566,164.41045182)(614.79165937,164.41045191)
\curveto(614.76165666,164.42045181)(614.72665669,164.4354518)(614.68665937,164.45545191)
\curveto(614.53665688,164.52545171)(614.37665704,164.61045162)(614.20665937,164.71045191)
\curveto(613.9166575,164.90045133)(613.66665775,165.1304511)(613.45665937,165.40045191)
\curveto(613.25665816,165.68045055)(613.08665833,165.99045024)(612.94665937,166.33045191)
\curveto(612.89665852,166.44044979)(612.85665856,166.55544968)(612.82665937,166.67545191)
\curveto(612.80665861,166.79544944)(612.77665864,166.91544932)(612.73665937,167.03545191)
\curveto(612.72665869,167.07544916)(612.7216587,167.11044912)(612.72165937,167.14045191)
\curveto(612.7216587,167.17044906)(612.7166587,167.21044902)(612.70665937,167.26045191)
\curveto(612.68665873,167.34044889)(612.67165875,167.42544881)(612.66165937,167.51545191)
\curveto(612.65165877,167.60544863)(612.63665878,167.69544854)(612.61665937,167.78545191)
\lineto(612.61665937,167.99545191)
\curveto(612.60665881,168.0354482)(612.59665882,168.09044814)(612.58665937,168.16045191)
\curveto(612.58665883,168.24044799)(612.59165883,168.30544793)(612.60165937,168.35545191)
\lineto(612.60165937,168.52045191)
\curveto(612.6216588,168.57044766)(612.62665879,168.62044761)(612.61665937,168.67045191)
\curveto(612.6166588,168.7304475)(612.6216588,168.78544745)(612.63165937,168.83545191)
\curveto(612.67165875,168.99544724)(612.70165872,169.15544708)(612.72165937,169.31545191)
\curveto(612.75165867,169.47544676)(612.79665862,169.62544661)(612.85665937,169.76545191)
\curveto(612.90665851,169.87544636)(612.95165847,169.98544625)(612.99165937,170.09545191)
\curveto(613.04165838,170.21544602)(613.09665832,170.3304459)(613.15665937,170.44045191)
\curveto(613.37665804,170.79044544)(613.62665779,171.09044514)(613.90665937,171.34045191)
\curveto(614.18665723,171.60044463)(614.53165689,171.81544442)(614.94165937,171.98545191)
\curveto(615.06165636,172.0354442)(615.18165624,172.07044416)(615.30165937,172.09045191)
\curveto(615.43165599,172.12044411)(615.56665585,172.15044408)(615.70665937,172.18045191)
\curveto(615.75665566,172.19044404)(615.80165562,172.19544404)(615.84165937,172.19545191)
\curveto(615.88165554,172.20544403)(615.92665549,172.21044402)(615.97665937,172.21045191)
\curveto(615.99665542,172.22044401)(616.0216554,172.22044401)(616.05165937,172.21045191)
\curveto(616.08165534,172.20044403)(616.10665531,172.20544403)(616.12665937,172.22545191)
\curveto(616.54665487,172.235444)(616.91165451,172.19044404)(617.22165937,172.09045191)
\curveto(617.53165389,172.00044423)(617.81165361,171.87544436)(618.06165937,171.71545191)
\curveto(618.11165331,171.69544454)(618.15165327,171.66544457)(618.18165937,171.62545191)
\curveto(618.21165321,171.59544464)(618.24665317,171.57044466)(618.28665937,171.55045191)
\curveto(618.36665305,171.49044474)(618.44665297,171.42044481)(618.52665937,171.34045191)
\curveto(618.6166528,171.26044497)(618.69165273,171.18044505)(618.75165937,171.10045191)
\curveto(618.91165251,170.89044534)(619.04665237,170.69044554)(619.15665937,170.50045191)
\curveto(619.22665219,170.39044584)(619.28165214,170.27044596)(619.32165937,170.14045191)
\curveto(619.36165206,170.01044622)(619.40665201,169.88044635)(619.45665937,169.75045191)
\curveto(619.50665191,169.62044661)(619.54165188,169.48544675)(619.56165937,169.34545191)
\curveto(619.59165183,169.20544703)(619.62665179,169.06544717)(619.66665937,168.92545191)
\curveto(619.67665174,168.85544738)(619.68165174,168.78544745)(619.68165937,168.71545191)
\lineto(619.71165937,168.50545191)
\moveto(618.25665937,169.01545191)
\curveto(618.28665313,169.05544718)(618.31165311,169.10544713)(618.33165937,169.16545191)
\curveto(618.35165307,169.235447)(618.35165307,169.30544693)(618.33165937,169.37545191)
\curveto(618.27165315,169.59544664)(618.18665323,169.80044643)(618.07665937,169.99045191)
\curveto(617.93665348,170.22044601)(617.78165364,170.41544582)(617.61165937,170.57545191)
\curveto(617.44165398,170.7354455)(617.2216542,170.87044536)(616.95165937,170.98045191)
\curveto(616.88165454,171.00044523)(616.81165461,171.01544522)(616.74165937,171.02545191)
\curveto(616.67165475,171.04544519)(616.59665482,171.06544517)(616.51665937,171.08545191)
\curveto(616.43665498,171.10544513)(616.35165507,171.11544512)(616.26165937,171.11545191)
\lineto(616.00665937,171.11545191)
\curveto(615.97665544,171.09544514)(615.94165548,171.08544515)(615.90165937,171.08545191)
\curveto(615.86165556,171.09544514)(615.82665559,171.09544514)(615.79665937,171.08545191)
\lineto(615.55665937,171.02545191)
\curveto(615.48665593,171.01544522)(615.416656,171.00044523)(615.34665937,170.98045191)
\curveto(615.05665636,170.86044537)(614.8216566,170.71044552)(614.64165937,170.53045191)
\curveto(614.47165695,170.35044588)(614.3166571,170.12544611)(614.17665937,169.85545191)
\curveto(614.14665727,169.80544643)(614.1166573,169.74044649)(614.08665937,169.66045191)
\curveto(614.05665736,169.59044664)(614.03165739,169.51044672)(614.01165937,169.42045191)
\curveto(613.99165743,169.3304469)(613.98665743,169.24544699)(613.99665937,169.16545191)
\curveto(614.00665741,169.08544715)(614.04165738,169.02544721)(614.10165937,168.98545191)
\curveto(614.18165724,168.92544731)(614.3166571,168.89544734)(614.50665937,168.89545191)
\curveto(614.70665671,168.90544733)(614.87665654,168.91044732)(615.01665937,168.91045191)
\lineto(617.29665937,168.91045191)
\curveto(617.44665397,168.91044732)(617.62665379,168.90544733)(617.83665937,168.89545191)
\curveto(618.04665337,168.89544734)(618.18665323,168.9354473)(618.25665937,169.01545191)
}
}
{
\newrgbcolor{curcolor}{0.60000002 0.60000002 0.60000002}
\pscustom[linestyle=none,fillstyle=solid,fillcolor=curcolor]
{
\newpath
\moveto(545.29360762,175.04548853)
\lineto(560.29360762,175.04548853)
\lineto(560.29360762,160.04548853)
\lineto(545.29360762,160.04548853)
\closepath
}
}
{
\newrgbcolor{curcolor}{0 0 0}
\pscustom[linestyle=none,fillstyle=solid,fillcolor=curcolor]
{
\newpath
\moveto(565.34181562,151.97974635)
\lineto(566.25681562,151.97974635)
\curveto(566.35681297,151.97973565)(566.45181288,151.97973565)(566.54181562,151.97974635)
\curveto(566.6318127,151.97973565)(566.70681262,151.95973567)(566.76681562,151.91974635)
\curveto(566.85681247,151.85973577)(566.91681241,151.77973585)(566.94681562,151.67974635)
\curveto(566.98681234,151.57973605)(567.0318123,151.47473616)(567.08181562,151.36474635)
\curveto(567.16181217,151.17473646)(567.2318121,150.98473665)(567.29181562,150.79474635)
\curveto(567.36181197,150.60473703)(567.43681189,150.41473722)(567.51681562,150.22474635)
\curveto(567.58681174,150.04473759)(567.65181168,149.85973777)(567.71181562,149.66974635)
\curveto(567.77181156,149.48973814)(567.84181149,149.30973832)(567.92181562,149.12974635)
\curveto(567.98181135,148.98973864)(568.03681129,148.84473879)(568.08681562,148.69474635)
\curveto(568.13681119,148.54473909)(568.19181114,148.39973923)(568.25181562,148.25974635)
\curveto(568.4318109,147.80973982)(568.60181073,147.35474028)(568.76181562,146.89474635)
\curveto(568.92181041,146.44474119)(569.09181024,145.99474164)(569.27181562,145.54474635)
\curveto(569.29181004,145.49474214)(569.30681002,145.44474219)(569.31681562,145.39474635)
\lineto(569.37681562,145.24474635)
\curveto(569.46680986,145.02474261)(569.55180978,144.79974283)(569.63181562,144.56974635)
\curveto(569.71180962,144.34974328)(569.79680953,144.1297435)(569.88681562,143.90974635)
\curveto(569.9268094,143.81974381)(569.96680936,143.70974392)(570.00681562,143.57974635)
\curveto(570.04680928,143.45974417)(570.11180922,143.38974424)(570.20181562,143.36974635)
\curveto(570.24180909,143.35974427)(570.27180906,143.35974427)(570.29181562,143.36974635)
\lineto(570.35181562,143.42974635)
\curveto(570.40180893,143.47974415)(570.43680889,143.5347441)(570.45681562,143.59474635)
\curveto(570.48680884,143.65474398)(570.51680881,143.71974391)(570.54681562,143.78974635)
\lineto(570.78681562,144.41974635)
\curveto(570.86680846,144.63974299)(570.94680838,144.85474278)(571.02681562,145.06474635)
\lineto(571.08681562,145.21474635)
\lineto(571.14681562,145.39474635)
\curveto(571.2268081,145.58474205)(571.29680803,145.77474186)(571.35681562,145.96474635)
\curveto(571.4268079,146.16474147)(571.50180783,146.36474127)(571.58181562,146.56474635)
\curveto(571.82180751,147.14474049)(572.04180729,147.7297399)(572.24181562,148.31974635)
\curveto(572.45180688,148.90973872)(572.67680665,149.49473814)(572.91681562,150.07474635)
\curveto(572.99680633,150.27473736)(573.07180626,150.47973715)(573.14181562,150.68974635)
\curveto(573.22180611,150.89973673)(573.30180603,151.10473653)(573.38181562,151.30474635)
\curveto(573.42180591,151.38473625)(573.45680587,151.48473615)(573.48681562,151.60474635)
\curveto(573.5268058,151.72473591)(573.58180575,151.80973582)(573.65181562,151.85974635)
\curveto(573.71180562,151.89973573)(573.78680554,151.9297357)(573.87681562,151.94974635)
\curveto(573.97680535,151.96973566)(574.08680524,151.97973565)(574.20681562,151.97974635)
\curveto(574.326805,151.98973564)(574.44680488,151.98973564)(574.56681562,151.97974635)
\curveto(574.68680464,151.97973565)(574.79680453,151.97973565)(574.89681562,151.97974635)
\curveto(574.98680434,151.97973565)(575.07680425,151.97973565)(575.16681562,151.97974635)
\curveto(575.26680406,151.97973565)(575.34180399,151.95973567)(575.39181562,151.91974635)
\curveto(575.48180385,151.86973576)(575.5318038,151.77973585)(575.54181562,151.64974635)
\curveto(575.55180378,151.51973611)(575.55680377,151.37973625)(575.55681562,151.22974635)
\lineto(575.55681562,149.57974635)
\lineto(575.55681562,143.30974635)
\lineto(575.55681562,142.04974635)
\curveto(575.55680377,141.93974569)(575.55680377,141.8297458)(575.55681562,141.71974635)
\curveto(575.56680376,141.60974602)(575.54680378,141.52474611)(575.49681562,141.46474635)
\curveto(575.46680386,141.40474623)(575.42180391,141.36474627)(575.36181562,141.34474635)
\curveto(575.30180403,141.3347463)(575.2318041,141.31974631)(575.15181562,141.29974635)
\lineto(574.91181562,141.29974635)
\lineto(574.55181562,141.29974635)
\curveto(574.44180489,141.30974632)(574.36180497,141.35474628)(574.31181562,141.43474635)
\curveto(574.29180504,141.46474617)(574.27680505,141.49474614)(574.26681562,141.52474635)
\curveto(574.26680506,141.56474607)(574.25680507,141.60974602)(574.23681562,141.65974635)
\lineto(574.23681562,141.82474635)
\curveto(574.2268051,141.88474575)(574.22180511,141.95474568)(574.22181562,142.03474635)
\curveto(574.2318051,142.11474552)(574.23680509,142.18974544)(574.23681562,142.25974635)
\lineto(574.23681562,143.09974635)
\lineto(574.23681562,147.52474635)
\curveto(574.23680509,147.77473986)(574.23680509,148.02473961)(574.23681562,148.27474635)
\curveto(574.23680509,148.5347391)(574.2318051,148.78473885)(574.22181562,149.02474635)
\curveto(574.22180511,149.12473851)(574.21680511,149.2347384)(574.20681562,149.35474635)
\curveto(574.19680513,149.47473816)(574.14180519,149.5347381)(574.04181562,149.53474635)
\lineto(574.04181562,149.51974635)
\curveto(573.97180536,149.49973813)(573.91180542,149.4347382)(573.86181562,149.32474635)
\curveto(573.82180551,149.21473842)(573.78680554,149.11973851)(573.75681562,149.03974635)
\curveto(573.68680564,148.86973876)(573.62180571,148.69473894)(573.56181562,148.51474635)
\curveto(573.50180583,148.34473929)(573.4318059,148.17473946)(573.35181562,148.00474635)
\curveto(573.331806,147.95473968)(573.31680601,147.90973972)(573.30681562,147.86974635)
\curveto(573.29680603,147.8297398)(573.28180605,147.78473985)(573.26181562,147.73474635)
\curveto(573.18180615,147.55474008)(573.11180622,147.36974026)(573.05181562,147.17974635)
\curveto(573.00180633,146.99974063)(572.93680639,146.81974081)(572.85681562,146.63974635)
\curveto(572.78680654,146.48974114)(572.7268066,146.3347413)(572.67681562,146.17474635)
\curveto(572.6268067,146.02474161)(572.57180676,145.87474176)(572.51181562,145.72474635)
\curveto(572.31180702,145.25474238)(572.1318072,144.77974285)(571.97181562,144.29974635)
\curveto(571.81180752,143.8297438)(571.63680769,143.36474427)(571.44681562,142.90474635)
\curveto(571.36680796,142.72474491)(571.29680803,142.54474509)(571.23681562,142.36474635)
\curveto(571.17680815,142.18474545)(571.11180822,142.00474563)(571.04181562,141.82474635)
\curveto(570.99180834,141.71474592)(570.94180839,141.60974602)(570.89181562,141.50974635)
\curveto(570.85180848,141.41974621)(570.76680856,141.35474628)(570.63681562,141.31474635)
\curveto(570.61680871,141.30474633)(570.59180874,141.29974633)(570.56181562,141.29974635)
\curveto(570.54180879,141.30974632)(570.51680881,141.30974632)(570.48681562,141.29974635)
\curveto(570.45680887,141.28974634)(570.42180891,141.28474635)(570.38181562,141.28474635)
\curveto(570.34180899,141.29474634)(570.30180903,141.29974633)(570.26181562,141.29974635)
\lineto(569.96181562,141.29974635)
\curveto(569.86180947,141.29974633)(569.78180955,141.32474631)(569.72181562,141.37474635)
\curveto(569.64180969,141.42474621)(569.58180975,141.49474614)(569.54181562,141.58474635)
\curveto(569.51180982,141.68474595)(569.47180986,141.78474585)(569.42181562,141.88474635)
\curveto(569.34180999,142.08474555)(569.26181007,142.28974534)(569.18181562,142.49974635)
\curveto(569.11181022,142.71974491)(569.03681029,142.9297447)(568.95681562,143.12974635)
\curveto(568.87681045,143.30974432)(568.80681052,143.48974414)(568.74681562,143.66974635)
\curveto(568.69681063,143.85974377)(568.6318107,144.04474359)(568.55181562,144.22474635)
\curveto(568.32181101,144.78474285)(568.10681122,145.34974228)(567.90681562,145.91974635)
\curveto(567.70681162,146.48974114)(567.49181184,147.05474058)(567.26181562,147.61474635)
\lineto(567.02181562,148.24474635)
\curveto(566.95181238,148.46473917)(566.87681245,148.67473896)(566.79681562,148.87474635)
\curveto(566.74681258,148.98473865)(566.70181263,149.08973854)(566.66181562,149.18974635)
\curveto(566.6318127,149.29973833)(566.58181275,149.39473824)(566.51181562,149.47474635)
\curveto(566.50181283,149.49473814)(566.49181284,149.50473813)(566.48181562,149.50474635)
\lineto(566.45181562,149.53474635)
\lineto(566.37681562,149.53474635)
\lineto(566.34681562,149.50474635)
\curveto(566.33681299,149.50473813)(566.326813,149.49973813)(566.31681562,149.48974635)
\curveto(566.29681303,149.43973819)(566.28681304,149.38473825)(566.28681562,149.32474635)
\curveto(566.28681304,149.26473837)(566.27681305,149.20473843)(566.25681562,149.14474635)
\lineto(566.25681562,148.97974635)
\curveto(566.23681309,148.91973871)(566.2318131,148.85473878)(566.24181562,148.78474635)
\curveto(566.25181308,148.71473892)(566.25681307,148.64473899)(566.25681562,148.57474635)
\lineto(566.25681562,147.76474635)
\lineto(566.25681562,143.20474635)
\lineto(566.25681562,142.01974635)
\curveto(566.25681307,141.90974572)(566.25181308,141.79974583)(566.24181562,141.68974635)
\curveto(566.24181309,141.57974605)(566.21681311,141.49474614)(566.16681562,141.43474635)
\curveto(566.11681321,141.35474628)(566.0268133,141.30974632)(565.89681562,141.29974635)
\lineto(565.50681562,141.29974635)
\lineto(565.31181562,141.29974635)
\curveto(565.26181407,141.29974633)(565.21181412,141.30974632)(565.16181562,141.32974635)
\curveto(565.0318143,141.36974626)(564.95681437,141.45474618)(564.93681562,141.58474635)
\curveto(564.9268144,141.71474592)(564.92181441,141.86474577)(564.92181562,142.03474635)
\lineto(564.92181562,143.77474635)
\lineto(564.92181562,149.77474635)
\lineto(564.92181562,151.18474635)
\curveto(564.92181441,151.29473634)(564.91681441,151.40973622)(564.90681562,151.52974635)
\curveto(564.90681442,151.64973598)(564.9318144,151.74473589)(564.98181562,151.81474635)
\curveto(565.02181431,151.87473576)(565.09681423,151.92473571)(565.20681562,151.96474635)
\curveto(565.2268141,151.97473566)(565.24681408,151.97473566)(565.26681562,151.96474635)
\curveto(565.29681403,151.96473567)(565.32181401,151.96973566)(565.34181562,151.97974635)
}
}
{
\newrgbcolor{curcolor}{0 0 0}
\pscustom[linestyle=none,fillstyle=solid,fillcolor=curcolor]
{
\newpath
\moveto(584.783925,145.49974635)
\curveto(584.80391694,145.43974219)(584.81391693,145.34474229)(584.813925,145.21474635)
\curveto(584.81391693,145.09474254)(584.80891693,145.00974262)(584.798925,144.95974635)
\lineto(584.798925,144.80974635)
\curveto(584.78891695,144.7297429)(584.77891696,144.65474298)(584.768925,144.58474635)
\curveto(584.76891697,144.52474311)(584.76391698,144.45474318)(584.753925,144.37474635)
\curveto(584.73391701,144.31474332)(584.71891702,144.25474338)(584.708925,144.19474635)
\curveto(584.70891703,144.1347435)(584.69891704,144.07474356)(584.678925,144.01474635)
\curveto(584.6389171,143.88474375)(584.60391714,143.75474388)(584.573925,143.62474635)
\curveto(584.5439172,143.49474414)(584.50391724,143.37474426)(584.453925,143.26474635)
\curveto(584.2439175,142.78474485)(583.96391778,142.37974525)(583.613925,142.04974635)
\curveto(583.26391848,141.7297459)(582.83391891,141.48474615)(582.323925,141.31474635)
\curveto(582.21391953,141.27474636)(582.09391965,141.24474639)(581.963925,141.22474635)
\curveto(581.8439199,141.20474643)(581.71892002,141.18474645)(581.588925,141.16474635)
\curveto(581.52892021,141.15474648)(581.46392028,141.14974648)(581.393925,141.14974635)
\curveto(581.33392041,141.13974649)(581.27392047,141.1347465)(581.213925,141.13474635)
\curveto(581.17392057,141.12474651)(581.11392063,141.11974651)(581.033925,141.11974635)
\curveto(580.96392078,141.11974651)(580.91392083,141.12474651)(580.883925,141.13474635)
\curveto(580.8439209,141.14474649)(580.80392094,141.14974648)(580.763925,141.14974635)
\curveto(580.72392102,141.13974649)(580.68892105,141.13974649)(580.658925,141.14974635)
\lineto(580.568925,141.14974635)
\lineto(580.208925,141.19474635)
\curveto(580.06892167,141.2347464)(579.93392181,141.27474636)(579.803925,141.31474635)
\curveto(579.67392207,141.35474628)(579.54892219,141.39974623)(579.428925,141.44974635)
\curveto(578.97892276,141.64974598)(578.60892313,141.90974572)(578.318925,142.22974635)
\curveto(578.02892371,142.54974508)(577.78892395,142.93974469)(577.598925,143.39974635)
\curveto(577.54892419,143.49974413)(577.50892423,143.59974403)(577.478925,143.69974635)
\curveto(577.45892428,143.79974383)(577.4389243,143.90474373)(577.418925,144.01474635)
\curveto(577.39892434,144.05474358)(577.38892435,144.08474355)(577.388925,144.10474635)
\curveto(577.39892434,144.1347435)(577.39892434,144.16974346)(577.388925,144.20974635)
\curveto(577.36892437,144.28974334)(577.35392439,144.36974326)(577.343925,144.44974635)
\curveto(577.3439244,144.53974309)(577.33392441,144.62474301)(577.313925,144.70474635)
\lineto(577.313925,144.82474635)
\curveto(577.31392443,144.86474277)(577.30892443,144.90974272)(577.298925,144.95974635)
\curveto(577.28892445,145.00974262)(577.28392446,145.09474254)(577.283925,145.21474635)
\curveto(577.28392446,145.34474229)(577.29392445,145.43974219)(577.313925,145.49974635)
\curveto(577.33392441,145.56974206)(577.3389244,145.63974199)(577.328925,145.70974635)
\curveto(577.31892442,145.77974185)(577.32392442,145.84974178)(577.343925,145.91974635)
\curveto(577.35392439,145.96974166)(577.35892438,146.00974162)(577.358925,146.03974635)
\curveto(577.36892437,146.07974155)(577.37892436,146.12474151)(577.388925,146.17474635)
\curveto(577.41892432,146.29474134)(577.4439243,146.41474122)(577.463925,146.53474635)
\curveto(577.49392425,146.65474098)(577.53392421,146.76974086)(577.583925,146.87974635)
\curveto(577.73392401,147.24974038)(577.91392383,147.57974005)(578.123925,147.86974635)
\curveto(578.3439234,148.16973946)(578.60892313,148.41973921)(578.918925,148.61974635)
\curveto(579.0389227,148.69973893)(579.16392258,148.76473887)(579.293925,148.81474635)
\curveto(579.42392232,148.87473876)(579.55892218,148.9347387)(579.698925,148.99474635)
\curveto(579.81892192,149.04473859)(579.94892179,149.07473856)(580.088925,149.08474635)
\curveto(580.22892151,149.10473853)(580.36892137,149.1347385)(580.508925,149.17474635)
\lineto(580.703925,149.17474635)
\curveto(580.77392097,149.18473845)(580.8389209,149.19473844)(580.898925,149.20474635)
\curveto(581.78891995,149.21473842)(582.52891921,149.0297386)(583.118925,148.64974635)
\curveto(583.70891803,148.26973936)(584.13391761,147.77473986)(584.393925,147.16474635)
\curveto(584.4439173,147.06474057)(584.48391726,146.96474067)(584.513925,146.86474635)
\curveto(584.5439172,146.76474087)(584.57891716,146.65974097)(584.618925,146.54974635)
\curveto(584.64891709,146.43974119)(584.67391707,146.31974131)(584.693925,146.18974635)
\curveto(584.71391703,146.06974156)(584.738917,145.94474169)(584.768925,145.81474635)
\curveto(584.77891696,145.76474187)(584.77891696,145.70974192)(584.768925,145.64974635)
\curveto(584.76891697,145.59974203)(584.77391697,145.54974208)(584.783925,145.49974635)
\moveto(583.448925,144.64474635)
\curveto(583.46891827,144.71474292)(583.47391827,144.79474284)(583.463925,144.88474635)
\lineto(583.463925,145.13974635)
\curveto(583.46391828,145.5297421)(583.42891831,145.85974177)(583.358925,146.12974635)
\curveto(583.32891841,146.20974142)(583.30391844,146.28974134)(583.283925,146.36974635)
\curveto(583.26391848,146.44974118)(583.2389185,146.52474111)(583.208925,146.59474635)
\curveto(582.92891881,147.24474039)(582.48391926,147.69473994)(581.873925,147.94474635)
\curveto(581.80391994,147.97473966)(581.72892001,147.99473964)(581.648925,148.00474635)
\lineto(581.408925,148.06474635)
\curveto(581.32892041,148.08473955)(581.2439205,148.09473954)(581.153925,148.09474635)
\lineto(580.883925,148.09474635)
\lineto(580.613925,148.04974635)
\curveto(580.51392123,148.0297396)(580.41892132,148.00473963)(580.328925,147.97474635)
\curveto(580.24892149,147.95473968)(580.16892157,147.92473971)(580.088925,147.88474635)
\curveto(580.01892172,147.86473977)(579.95392179,147.8347398)(579.893925,147.79474635)
\curveto(579.83392191,147.75473988)(579.77892196,147.71473992)(579.728925,147.67474635)
\curveto(579.48892225,147.50474013)(579.29392245,147.29974033)(579.143925,147.05974635)
\curveto(578.99392275,146.81974081)(578.86392288,146.53974109)(578.753925,146.21974635)
\curveto(578.72392302,146.11974151)(578.70392304,146.01474162)(578.693925,145.90474635)
\curveto(578.68392306,145.80474183)(578.66892307,145.69974193)(578.648925,145.58974635)
\curveto(578.6389231,145.54974208)(578.63392311,145.48474215)(578.633925,145.39474635)
\curveto(578.62392312,145.36474227)(578.61892312,145.3297423)(578.618925,145.28974635)
\curveto(578.62892311,145.24974238)(578.63392311,145.20474243)(578.633925,145.15474635)
\lineto(578.633925,144.85474635)
\curveto(578.63392311,144.75474288)(578.6439231,144.66474297)(578.663925,144.58474635)
\lineto(578.693925,144.40474635)
\curveto(578.71392303,144.30474333)(578.72892301,144.20474343)(578.738925,144.10474635)
\curveto(578.75892298,144.01474362)(578.78892295,143.9297437)(578.828925,143.84974635)
\curveto(578.92892281,143.60974402)(579.0439227,143.38474425)(579.173925,143.17474635)
\curveto(579.31392243,142.96474467)(579.48392226,142.78974484)(579.683925,142.64974635)
\curveto(579.73392201,142.61974501)(579.77892196,142.59474504)(579.818925,142.57474635)
\curveto(579.85892188,142.55474508)(579.90392184,142.5297451)(579.953925,142.49974635)
\curveto(580.03392171,142.44974518)(580.11892162,142.40474523)(580.208925,142.36474635)
\curveto(580.30892143,142.3347453)(580.41392133,142.30474533)(580.523925,142.27474635)
\curveto(580.57392117,142.25474538)(580.61892112,142.24474539)(580.658925,142.24474635)
\curveto(580.70892103,142.25474538)(580.75892098,142.25474538)(580.808925,142.24474635)
\curveto(580.8389209,142.2347454)(580.89892084,142.22474541)(580.988925,142.21474635)
\curveto(581.08892065,142.20474543)(581.16392058,142.20974542)(581.213925,142.22974635)
\curveto(581.25392049,142.23974539)(581.29392045,142.23974539)(581.333925,142.22974635)
\curveto(581.37392037,142.2297454)(581.41392033,142.23974539)(581.453925,142.25974635)
\curveto(581.53392021,142.27974535)(581.61392013,142.29474534)(581.693925,142.30474635)
\curveto(581.77391997,142.32474531)(581.84891989,142.34974528)(581.918925,142.37974635)
\curveto(582.25891948,142.51974511)(582.53391921,142.71474492)(582.743925,142.96474635)
\curveto(582.95391879,143.21474442)(583.12891861,143.50974412)(583.268925,143.84974635)
\curveto(583.31891842,143.96974366)(583.34891839,144.09474354)(583.358925,144.22474635)
\curveto(583.37891836,144.36474327)(583.40891833,144.50474313)(583.448925,144.64474635)
}
}
{
\newrgbcolor{curcolor}{0 0 0}
\pscustom[linestyle=none,fillstyle=solid,fillcolor=curcolor]
{
\newpath
\moveto(593.23220625,142.10974635)
\lineto(593.23220625,141.71974635)
\curveto(593.23219837,141.59974603)(593.2071984,141.49974613)(593.15720625,141.41974635)
\curveto(593.1071985,141.34974628)(593.02219858,141.30974632)(592.90220625,141.29974635)
\lineto(592.55720625,141.29974635)
\curveto(592.49719911,141.29974633)(592.43719917,141.29474634)(592.37720625,141.28474635)
\curveto(592.32719928,141.28474635)(592.28219932,141.29474634)(592.24220625,141.31474635)
\curveto(592.15219945,141.3347463)(592.09219951,141.37474626)(592.06220625,141.43474635)
\curveto(592.02219958,141.48474615)(591.99719961,141.54474609)(591.98720625,141.61474635)
\curveto(591.98719962,141.68474595)(591.97219963,141.75474588)(591.94220625,141.82474635)
\curveto(591.93219967,141.84474579)(591.91719969,141.85974577)(591.89720625,141.86974635)
\curveto(591.88719972,141.88974574)(591.87219973,141.90974572)(591.85220625,141.92974635)
\curveto(591.75219985,141.93974569)(591.67219993,141.91974571)(591.61220625,141.86974635)
\curveto(591.56220004,141.81974581)(591.5072001,141.76974586)(591.44720625,141.71974635)
\curveto(591.24720036,141.56974606)(591.04720056,141.45474618)(590.84720625,141.37474635)
\curveto(590.66720094,141.29474634)(590.45720115,141.2347464)(590.21720625,141.19474635)
\curveto(589.98720162,141.15474648)(589.74720186,141.1347465)(589.49720625,141.13474635)
\curveto(589.25720235,141.12474651)(589.01720259,141.13974649)(588.77720625,141.17974635)
\curveto(588.53720307,141.20974642)(588.32720328,141.26474637)(588.14720625,141.34474635)
\curveto(587.62720398,141.56474607)(587.2072044,141.85974577)(586.88720625,142.22974635)
\curveto(586.56720504,142.60974502)(586.31720529,143.07974455)(586.13720625,143.63974635)
\curveto(586.09720551,143.7297439)(586.06720554,143.81974381)(586.04720625,143.90974635)
\curveto(586.03720557,144.00974362)(586.01720559,144.10974352)(585.98720625,144.20974635)
\curveto(585.97720563,144.25974337)(585.97220563,144.30974332)(585.97220625,144.35974635)
\curveto(585.97220563,144.40974322)(585.96720564,144.45974317)(585.95720625,144.50974635)
\curveto(585.93720567,144.55974307)(585.92720568,144.60974302)(585.92720625,144.65974635)
\curveto(585.93720567,144.71974291)(585.93720567,144.77474286)(585.92720625,144.82474635)
\lineto(585.92720625,144.97474635)
\curveto(585.9072057,145.02474261)(585.89720571,145.08974254)(585.89720625,145.16974635)
\curveto(585.89720571,145.24974238)(585.9072057,145.31474232)(585.92720625,145.36474635)
\lineto(585.92720625,145.52974635)
\curveto(585.94720566,145.59974203)(585.95220565,145.66974196)(585.94220625,145.73974635)
\curveto(585.94220566,145.81974181)(585.95220565,145.89474174)(585.97220625,145.96474635)
\curveto(585.98220562,146.01474162)(585.98720562,146.05974157)(585.98720625,146.09974635)
\curveto(585.98720562,146.13974149)(585.99220561,146.18474145)(586.00220625,146.23474635)
\curveto(586.03220557,146.3347413)(586.05720555,146.4297412)(586.07720625,146.51974635)
\curveto(586.09720551,146.61974101)(586.12220548,146.71474092)(586.15220625,146.80474635)
\curveto(586.28220532,147.18474045)(586.44720516,147.52474011)(586.64720625,147.82474635)
\curveto(586.85720475,148.1347395)(587.1072045,148.38973924)(587.39720625,148.58974635)
\curveto(587.56720404,148.70973892)(587.74220386,148.80973882)(587.92220625,148.88974635)
\curveto(588.11220349,148.96973866)(588.31720329,149.03973859)(588.53720625,149.09974635)
\curveto(588.607203,149.10973852)(588.67220293,149.11973851)(588.73220625,149.12974635)
\curveto(588.8022028,149.13973849)(588.87220273,149.15473848)(588.94220625,149.17474635)
\lineto(589.09220625,149.17474635)
\curveto(589.17220243,149.19473844)(589.28720232,149.20473843)(589.43720625,149.20474635)
\curveto(589.59720201,149.20473843)(589.71720189,149.19473844)(589.79720625,149.17474635)
\curveto(589.83720177,149.16473847)(589.89220171,149.15973847)(589.96220625,149.15974635)
\curveto(590.07220153,149.1297385)(590.18220142,149.10473853)(590.29220625,149.08474635)
\curveto(590.4022012,149.07473856)(590.5072011,149.04473859)(590.60720625,148.99474635)
\curveto(590.75720085,148.9347387)(590.89720071,148.86973876)(591.02720625,148.79974635)
\curveto(591.16720044,148.7297389)(591.29720031,148.64973898)(591.41720625,148.55974635)
\curveto(591.47720013,148.50973912)(591.53720007,148.45473918)(591.59720625,148.39474635)
\curveto(591.66719994,148.34473929)(591.75719985,148.3297393)(591.86720625,148.34974635)
\curveto(591.88719972,148.37973925)(591.9021997,148.40473923)(591.91220625,148.42474635)
\curveto(591.93219967,148.44473919)(591.94719966,148.47473916)(591.95720625,148.51474635)
\curveto(591.98719962,148.60473903)(591.99719961,148.71973891)(591.98720625,148.85974635)
\lineto(591.98720625,149.23474635)
\lineto(591.98720625,150.95974635)
\lineto(591.98720625,151.42474635)
\curveto(591.98719962,151.60473603)(592.01219959,151.7347359)(592.06220625,151.81474635)
\curveto(592.1021995,151.88473575)(592.16219944,151.9297357)(592.24220625,151.94974635)
\curveto(592.26219934,151.94973568)(592.28719932,151.94973568)(592.31720625,151.94974635)
\curveto(592.34719926,151.95973567)(592.37219923,151.96473567)(592.39220625,151.96474635)
\curveto(592.53219907,151.97473566)(592.67719893,151.97473566)(592.82720625,151.96474635)
\curveto(592.98719862,151.96473567)(593.09719851,151.92473571)(593.15720625,151.84474635)
\curveto(593.2071984,151.76473587)(593.23219837,151.66473597)(593.23220625,151.54474635)
\lineto(593.23220625,151.16974635)
\lineto(593.23220625,142.10974635)
\moveto(592.01720625,144.94474635)
\curveto(592.03719957,144.99474264)(592.04719956,145.05974257)(592.04720625,145.13974635)
\curveto(592.04719956,145.2297424)(592.03719957,145.29974233)(592.01720625,145.34974635)
\lineto(592.01720625,145.57474635)
\curveto(591.99719961,145.66474197)(591.98219962,145.75474188)(591.97220625,145.84474635)
\curveto(591.96219964,145.94474169)(591.94219966,146.0347416)(591.91220625,146.11474635)
\curveto(591.89219971,146.19474144)(591.87219973,146.26974136)(591.85220625,146.33974635)
\curveto(591.84219976,146.40974122)(591.82219978,146.47974115)(591.79220625,146.54974635)
\curveto(591.67219993,146.84974078)(591.51720009,147.11474052)(591.32720625,147.34474635)
\curveto(591.13720047,147.57474006)(590.89720071,147.75473988)(590.60720625,147.88474635)
\curveto(590.5072011,147.9347397)(590.4022012,147.96973966)(590.29220625,147.98974635)
\curveto(590.19220141,148.01973961)(590.08220152,148.04473959)(589.96220625,148.06474635)
\curveto(589.88220172,148.08473955)(589.79220181,148.09473954)(589.69220625,148.09474635)
\lineto(589.42220625,148.09474635)
\curveto(589.37220223,148.08473955)(589.32720228,148.07473956)(589.28720625,148.06474635)
\lineto(589.15220625,148.06474635)
\curveto(589.07220253,148.04473959)(588.98720262,148.02473961)(588.89720625,148.00474635)
\curveto(588.81720279,147.98473965)(588.73720287,147.95973967)(588.65720625,147.92974635)
\curveto(588.33720327,147.78973984)(588.07720353,147.58474005)(587.87720625,147.31474635)
\curveto(587.68720392,147.05474058)(587.53220407,146.74974088)(587.41220625,146.39974635)
\curveto(587.37220423,146.28974134)(587.34220426,146.17474146)(587.32220625,146.05474635)
\curveto(587.31220429,145.94474169)(587.29720431,145.8347418)(587.27720625,145.72474635)
\curveto(587.27720433,145.68474195)(587.27220433,145.64474199)(587.26220625,145.60474635)
\lineto(587.26220625,145.49974635)
\curveto(587.24220436,145.44974218)(587.23220437,145.39474224)(587.23220625,145.33474635)
\curveto(587.24220436,145.27474236)(587.24720436,145.21974241)(587.24720625,145.16974635)
\lineto(587.24720625,144.83974635)
\curveto(587.24720436,144.73974289)(587.25720435,144.64474299)(587.27720625,144.55474635)
\curveto(587.28720432,144.52474311)(587.29220431,144.47474316)(587.29220625,144.40474635)
\curveto(587.31220429,144.3347433)(587.32720428,144.26474337)(587.33720625,144.19474635)
\lineto(587.39720625,143.98474635)
\curveto(587.5072041,143.634744)(587.65720395,143.3347443)(587.84720625,143.08474635)
\curveto(588.03720357,142.8347448)(588.27720333,142.629745)(588.56720625,142.46974635)
\curveto(588.65720295,142.41974521)(588.74720286,142.37974525)(588.83720625,142.34974635)
\curveto(588.92720268,142.31974531)(589.02720258,142.28974534)(589.13720625,142.25974635)
\curveto(589.18720242,142.23974539)(589.23720237,142.2347454)(589.28720625,142.24474635)
\curveto(589.34720226,142.25474538)(589.4022022,142.24974538)(589.45220625,142.22974635)
\curveto(589.49220211,142.21974541)(589.53220207,142.21474542)(589.57220625,142.21474635)
\lineto(589.70720625,142.21474635)
\lineto(589.84220625,142.21474635)
\curveto(589.87220173,142.22474541)(589.92220168,142.2297454)(589.99220625,142.22974635)
\curveto(590.07220153,142.24974538)(590.15220145,142.26474537)(590.23220625,142.27474635)
\curveto(590.31220129,142.29474534)(590.38720122,142.31974531)(590.45720625,142.34974635)
\curveto(590.78720082,142.48974514)(591.05220055,142.66474497)(591.25220625,142.87474635)
\curveto(591.46220014,143.09474454)(591.63719997,143.36974426)(591.77720625,143.69974635)
\curveto(591.82719978,143.80974382)(591.86219974,143.91974371)(591.88220625,144.02974635)
\curveto(591.9021997,144.13974349)(591.92719968,144.24974338)(591.95720625,144.35974635)
\curveto(591.97719963,144.39974323)(591.98719962,144.4347432)(591.98720625,144.46474635)
\curveto(591.98719962,144.50474313)(591.99219961,144.54474309)(592.00220625,144.58474635)
\curveto(592.01219959,144.64474299)(592.01219959,144.70474293)(592.00220625,144.76474635)
\curveto(592.0021996,144.82474281)(592.0071996,144.88474275)(592.01720625,144.94474635)
}
}
{
\newrgbcolor{curcolor}{0 0 0}
\pscustom[linestyle=none,fillstyle=solid,fillcolor=curcolor]
{
\newpath
\moveto(601.92845625,145.46974635)
\curveto(601.94844856,145.36974226)(601.94844856,145.25474238)(601.92845625,145.12474635)
\curveto(601.91844859,145.00474263)(601.88844862,144.91974271)(601.83845625,144.86974635)
\curveto(601.78844872,144.8297428)(601.7134488,144.79974283)(601.61345625,144.77974635)
\curveto(601.52344899,144.76974286)(601.41844909,144.76474287)(601.29845625,144.76474635)
\lineto(600.93845625,144.76474635)
\curveto(600.81844969,144.77474286)(600.7134498,144.77974285)(600.62345625,144.77974635)
\lineto(596.78345625,144.77974635)
\curveto(596.70345381,144.77974285)(596.62345389,144.77474286)(596.54345625,144.76474635)
\curveto(596.46345405,144.76474287)(596.39845411,144.74974288)(596.34845625,144.71974635)
\curveto(596.3084542,144.69974293)(596.26845424,144.65974297)(596.22845625,144.59974635)
\curveto(596.2084543,144.56974306)(596.18845432,144.52474311)(596.16845625,144.46474635)
\curveto(596.14845436,144.41474322)(596.14845436,144.36474327)(596.16845625,144.31474635)
\curveto(596.17845433,144.26474337)(596.18345433,144.21974341)(596.18345625,144.17974635)
\curveto(596.18345433,144.13974349)(596.18845432,144.09974353)(596.19845625,144.05974635)
\curveto(596.21845429,143.97974365)(596.23845427,143.89474374)(596.25845625,143.80474635)
\curveto(596.27845423,143.72474391)(596.3084542,143.64474399)(596.34845625,143.56474635)
\curveto(596.57845393,143.02474461)(596.95845355,142.63974499)(597.48845625,142.40974635)
\curveto(597.54845296,142.37974525)(597.6134529,142.35474528)(597.68345625,142.33474635)
\lineto(597.89345625,142.27474635)
\curveto(597.92345259,142.26474537)(597.97345254,142.25974537)(598.04345625,142.25974635)
\curveto(598.18345233,142.21974541)(598.36845214,142.19974543)(598.59845625,142.19974635)
\curveto(598.82845168,142.19974543)(599.0134515,142.21974541)(599.15345625,142.25974635)
\curveto(599.29345122,142.29974533)(599.41845109,142.33974529)(599.52845625,142.37974635)
\curveto(599.64845086,142.4297452)(599.75845075,142.48974514)(599.85845625,142.55974635)
\curveto(599.96845054,142.629745)(600.06345045,142.70974492)(600.14345625,142.79974635)
\curveto(600.22345029,142.89974473)(600.29345022,143.00474463)(600.35345625,143.11474635)
\curveto(600.4134501,143.21474442)(600.46345005,143.31974431)(600.50345625,143.42974635)
\curveto(600.55344996,143.53974409)(600.63344988,143.61974401)(600.74345625,143.66974635)
\curveto(600.78344973,143.68974394)(600.84844966,143.70474393)(600.93845625,143.71474635)
\curveto(601.02844948,143.72474391)(601.11844939,143.72474391)(601.20845625,143.71474635)
\curveto(601.29844921,143.71474392)(601.38344913,143.70974392)(601.46345625,143.69974635)
\curveto(601.54344897,143.68974394)(601.59844891,143.66974396)(601.62845625,143.63974635)
\curveto(601.72844878,143.56974406)(601.75344876,143.45474418)(601.70345625,143.29474635)
\curveto(601.62344889,143.02474461)(601.51844899,142.78474485)(601.38845625,142.57474635)
\curveto(601.18844932,142.25474538)(600.95844955,141.98974564)(600.69845625,141.77974635)
\curveto(600.44845006,141.57974605)(600.12845038,141.41474622)(599.73845625,141.28474635)
\curveto(599.63845087,141.24474639)(599.53845097,141.21974641)(599.43845625,141.20974635)
\curveto(599.33845117,141.18974644)(599.23345128,141.16974646)(599.12345625,141.14974635)
\curveto(599.07345144,141.13974649)(599.02345149,141.1347465)(598.97345625,141.13474635)
\curveto(598.93345158,141.1347465)(598.88845162,141.1297465)(598.83845625,141.11974635)
\lineto(598.68845625,141.11974635)
\curveto(598.63845187,141.10974652)(598.57845193,141.10474653)(598.50845625,141.10474635)
\curveto(598.44845206,141.10474653)(598.39845211,141.10974652)(598.35845625,141.11974635)
\lineto(598.22345625,141.11974635)
\curveto(598.17345234,141.1297465)(598.12845238,141.1347465)(598.08845625,141.13474635)
\curveto(598.04845246,141.1347465)(598.0084525,141.13974649)(597.96845625,141.14974635)
\curveto(597.91845259,141.15974647)(597.86345265,141.16974646)(597.80345625,141.17974635)
\curveto(597.74345277,141.17974645)(597.68845282,141.18474645)(597.63845625,141.19474635)
\curveto(597.54845296,141.21474642)(597.45845305,141.23974639)(597.36845625,141.26974635)
\curveto(597.27845323,141.28974634)(597.19345332,141.31474632)(597.11345625,141.34474635)
\curveto(597.07345344,141.36474627)(597.03845347,141.37474626)(597.00845625,141.37474635)
\curveto(596.97845353,141.38474625)(596.94345357,141.39974623)(596.90345625,141.41974635)
\curveto(596.75345376,141.48974614)(596.59345392,141.57474606)(596.42345625,141.67474635)
\curveto(596.13345438,141.86474577)(595.88345463,142.09474554)(595.67345625,142.36474635)
\curveto(595.47345504,142.64474499)(595.30345521,142.95474468)(595.16345625,143.29474635)
\curveto(595.1134554,143.40474423)(595.07345544,143.51974411)(595.04345625,143.63974635)
\curveto(595.02345549,143.75974387)(594.99345552,143.87974375)(594.95345625,143.99974635)
\curveto(594.94345557,144.03974359)(594.93845557,144.07474356)(594.93845625,144.10474635)
\curveto(594.93845557,144.1347435)(594.93345558,144.17474346)(594.92345625,144.22474635)
\curveto(594.90345561,144.30474333)(594.88845562,144.38974324)(594.87845625,144.47974635)
\curveto(594.86845564,144.56974306)(594.85345566,144.65974297)(594.83345625,144.74974635)
\lineto(594.83345625,144.95974635)
\curveto(594.82345569,144.99974263)(594.8134557,145.05474258)(594.80345625,145.12474635)
\curveto(594.80345571,145.20474243)(594.8084557,145.26974236)(594.81845625,145.31974635)
\lineto(594.81845625,145.48474635)
\curveto(594.83845567,145.5347421)(594.84345567,145.58474205)(594.83345625,145.63474635)
\curveto(594.83345568,145.69474194)(594.83845567,145.74974188)(594.84845625,145.79974635)
\curveto(594.88845562,145.95974167)(594.91845559,146.11974151)(594.93845625,146.27974635)
\curveto(594.96845554,146.43974119)(595.0134555,146.58974104)(595.07345625,146.72974635)
\curveto(595.12345539,146.83974079)(595.16845534,146.94974068)(595.20845625,147.05974635)
\curveto(595.25845525,147.17974045)(595.3134552,147.29474034)(595.37345625,147.40474635)
\curveto(595.59345492,147.75473988)(595.84345467,148.05473958)(596.12345625,148.30474635)
\curveto(596.40345411,148.56473907)(596.74845376,148.77973885)(597.15845625,148.94974635)
\curveto(597.27845323,148.99973863)(597.39845311,149.0347386)(597.51845625,149.05474635)
\curveto(597.64845286,149.08473855)(597.78345273,149.11473852)(597.92345625,149.14474635)
\curveto(597.97345254,149.15473848)(598.01845249,149.15973847)(598.05845625,149.15974635)
\curveto(598.09845241,149.16973846)(598.14345237,149.17473846)(598.19345625,149.17474635)
\curveto(598.2134523,149.18473845)(598.23845227,149.18473845)(598.26845625,149.17474635)
\curveto(598.29845221,149.16473847)(598.32345219,149.16973846)(598.34345625,149.18974635)
\curveto(598.76345175,149.19973843)(599.12845138,149.15473848)(599.43845625,149.05474635)
\curveto(599.74845076,148.96473867)(600.02845048,148.83973879)(600.27845625,148.67974635)
\curveto(600.32845018,148.65973897)(600.36845014,148.629739)(600.39845625,148.58974635)
\curveto(600.42845008,148.55973907)(600.46345005,148.5347391)(600.50345625,148.51474635)
\curveto(600.58344993,148.45473918)(600.66344985,148.38473925)(600.74345625,148.30474635)
\curveto(600.83344968,148.22473941)(600.9084496,148.14473949)(600.96845625,148.06474635)
\curveto(601.12844938,147.85473978)(601.26344925,147.65473998)(601.37345625,147.46474635)
\curveto(601.44344907,147.35474028)(601.49844901,147.2347404)(601.53845625,147.10474635)
\curveto(601.57844893,146.97474066)(601.62344889,146.84474079)(601.67345625,146.71474635)
\curveto(601.72344879,146.58474105)(601.75844875,146.44974118)(601.77845625,146.30974635)
\curveto(601.8084487,146.16974146)(601.84344867,146.0297416)(601.88345625,145.88974635)
\curveto(601.89344862,145.81974181)(601.89844861,145.74974188)(601.89845625,145.67974635)
\lineto(601.92845625,145.46974635)
\moveto(600.47345625,145.97974635)
\curveto(600.50345001,146.01974161)(600.52844998,146.06974156)(600.54845625,146.12974635)
\curveto(600.56844994,146.19974143)(600.56844994,146.26974136)(600.54845625,146.33974635)
\curveto(600.48845002,146.55974107)(600.40345011,146.76474087)(600.29345625,146.95474635)
\curveto(600.15345036,147.18474045)(599.99845051,147.37974025)(599.82845625,147.53974635)
\curveto(599.65845085,147.69973993)(599.43845107,147.8347398)(599.16845625,147.94474635)
\curveto(599.09845141,147.96473967)(599.02845148,147.97973965)(598.95845625,147.98974635)
\curveto(598.88845162,148.00973962)(598.8134517,148.0297396)(598.73345625,148.04974635)
\curveto(598.65345186,148.06973956)(598.56845194,148.07973955)(598.47845625,148.07974635)
\lineto(598.22345625,148.07974635)
\curveto(598.19345232,148.05973957)(598.15845235,148.04973958)(598.11845625,148.04974635)
\curveto(598.07845243,148.05973957)(598.04345247,148.05973957)(598.01345625,148.04974635)
\lineto(597.77345625,147.98974635)
\curveto(597.70345281,147.97973965)(597.63345288,147.96473967)(597.56345625,147.94474635)
\curveto(597.27345324,147.82473981)(597.03845347,147.67473996)(596.85845625,147.49474635)
\curveto(596.68845382,147.31474032)(596.53345398,147.08974054)(596.39345625,146.81974635)
\curveto(596.36345415,146.76974086)(596.33345418,146.70474093)(596.30345625,146.62474635)
\curveto(596.27345424,146.55474108)(596.24845426,146.47474116)(596.22845625,146.38474635)
\curveto(596.2084543,146.29474134)(596.20345431,146.20974142)(596.21345625,146.12974635)
\curveto(596.22345429,146.04974158)(596.25845425,145.98974164)(596.31845625,145.94974635)
\curveto(596.39845411,145.88974174)(596.53345398,145.85974177)(596.72345625,145.85974635)
\curveto(596.92345359,145.86974176)(597.09345342,145.87474176)(597.23345625,145.87474635)
\lineto(599.51345625,145.87474635)
\curveto(599.66345085,145.87474176)(599.84345067,145.86974176)(600.05345625,145.85974635)
\curveto(600.26345025,145.85974177)(600.40345011,145.89974173)(600.47345625,145.97974635)
}
}
{
\newrgbcolor{curcolor}{0 0 0}
\pscustom[linestyle=none,fillstyle=solid,fillcolor=curcolor]
{
\newpath
\moveto(606.88009687,149.20474635)
\curveto(607.11009208,149.20473843)(607.24009195,149.14473849)(607.27009687,149.02474635)
\curveto(607.30009189,148.91473872)(607.31509188,148.74973888)(607.31509687,148.52974635)
\lineto(607.31509687,148.24474635)
\curveto(607.31509188,148.15473948)(607.2900919,148.07973955)(607.24009687,148.01974635)
\curveto(607.18009201,147.93973969)(607.0950921,147.89473974)(606.98509687,147.88474635)
\curveto(606.87509232,147.88473975)(606.76509243,147.86973976)(606.65509687,147.83974635)
\curveto(606.51509268,147.80973982)(606.38009281,147.77973985)(606.25009687,147.74974635)
\curveto(606.13009306,147.71973991)(606.01509318,147.67973995)(605.90509687,147.62974635)
\curveto(605.61509358,147.49974013)(605.38009381,147.31974031)(605.20009687,147.08974635)
\curveto(605.02009417,146.86974076)(604.86509433,146.61474102)(604.73509687,146.32474635)
\curveto(604.6950945,146.21474142)(604.66509453,146.09974153)(604.64509687,145.97974635)
\curveto(604.62509457,145.86974176)(604.60009459,145.75474188)(604.57009687,145.63474635)
\curveto(604.56009463,145.58474205)(604.55509464,145.5347421)(604.55509687,145.48474635)
\curveto(604.56509463,145.4347422)(604.56509463,145.38474225)(604.55509687,145.33474635)
\curveto(604.52509467,145.21474242)(604.51009468,145.07474256)(604.51009687,144.91474635)
\curveto(604.52009467,144.76474287)(604.52509467,144.61974301)(604.52509687,144.47974635)
\lineto(604.52509687,142.63474635)
\lineto(604.52509687,142.28974635)
\curveto(604.52509467,142.16974546)(604.52009467,142.05474558)(604.51009687,141.94474635)
\curveto(604.50009469,141.8347458)(604.4950947,141.73974589)(604.49509687,141.65974635)
\curveto(604.50509469,141.57974605)(604.48509471,141.50974612)(604.43509687,141.44974635)
\curveto(604.38509481,141.37974625)(604.30509489,141.33974629)(604.19509687,141.32974635)
\curveto(604.0950951,141.31974631)(603.98509521,141.31474632)(603.86509687,141.31474635)
\lineto(603.59509687,141.31474635)
\curveto(603.54509565,141.3347463)(603.4950957,141.34974628)(603.44509687,141.35974635)
\curveto(603.40509579,141.37974625)(603.37509582,141.40474623)(603.35509687,141.43474635)
\curveto(603.30509589,141.50474613)(603.27509592,141.58974604)(603.26509687,141.68974635)
\lineto(603.26509687,142.01974635)
\lineto(603.26509687,143.17474635)
\lineto(603.26509687,147.32974635)
\lineto(603.26509687,148.36474635)
\lineto(603.26509687,148.66474635)
\curveto(603.27509592,148.76473887)(603.30509589,148.84973878)(603.35509687,148.91974635)
\curveto(603.38509581,148.95973867)(603.43509576,148.98973864)(603.50509687,149.00974635)
\curveto(603.58509561,149.0297386)(603.67009552,149.03973859)(603.76009687,149.03974635)
\curveto(603.85009534,149.04973858)(603.94009525,149.04973858)(604.03009687,149.03974635)
\curveto(604.12009507,149.0297386)(604.190095,149.01473862)(604.24009687,148.99474635)
\curveto(604.32009487,148.96473867)(604.37009482,148.90473873)(604.39009687,148.81474635)
\curveto(604.42009477,148.7347389)(604.43509476,148.64473899)(604.43509687,148.54474635)
\lineto(604.43509687,148.24474635)
\curveto(604.43509476,148.14473949)(604.45509474,148.05473958)(604.49509687,147.97474635)
\curveto(604.50509469,147.95473968)(604.51509468,147.93973969)(604.52509687,147.92974635)
\lineto(604.57009687,147.88474635)
\curveto(604.68009451,147.88473975)(604.77009442,147.9297397)(604.84009687,148.01974635)
\curveto(604.91009428,148.11973951)(604.97009422,148.19973943)(605.02009687,148.25974635)
\lineto(605.11009687,148.34974635)
\curveto(605.20009399,148.45973917)(605.32509387,148.57473906)(605.48509687,148.69474635)
\curveto(605.64509355,148.81473882)(605.7950934,148.90473873)(605.93509687,148.96474635)
\curveto(606.02509317,149.01473862)(606.12009307,149.04973858)(606.22009687,149.06974635)
\curveto(606.32009287,149.09973853)(606.42509277,149.1297385)(606.53509687,149.15974635)
\curveto(606.5950926,149.16973846)(606.65509254,149.17473846)(606.71509687,149.17474635)
\curveto(606.77509242,149.18473845)(606.83009236,149.19473844)(606.88009687,149.20474635)
}
}
{
\newrgbcolor{curcolor}{0 0 0}
\pscustom[linestyle=none,fillstyle=solid,fillcolor=curcolor]
{
\newpath
\moveto(615.1298625,141.85474635)
\curveto(615.15985467,141.69474594)(615.14485468,141.55974607)(615.0848625,141.44974635)
\curveto(615.0248548,141.34974628)(614.94485488,141.27474636)(614.8448625,141.22474635)
\curveto(614.79485503,141.20474643)(614.73985509,141.19474644)(614.6798625,141.19474635)
\curveto(614.6298552,141.19474644)(614.57485525,141.18474645)(614.5148625,141.16474635)
\curveto(614.29485553,141.11474652)(614.07485575,141.1297465)(613.8548625,141.20974635)
\curveto(613.64485618,141.27974635)(613.49985633,141.36974626)(613.4198625,141.47974635)
\curveto(613.36985646,141.54974608)(613.3248565,141.629746)(613.2848625,141.71974635)
\curveto(613.24485658,141.81974581)(613.19485663,141.89974573)(613.1348625,141.95974635)
\curveto(613.11485671,141.97974565)(613.08985674,141.99974563)(613.0598625,142.01974635)
\curveto(613.03985679,142.03974559)(613.00985682,142.04474559)(612.9698625,142.03474635)
\curveto(612.85985697,142.00474563)(612.75485707,141.94974568)(612.6548625,141.86974635)
\curveto(612.56485726,141.78974584)(612.47485735,141.71974591)(612.3848625,141.65974635)
\curveto(612.25485757,141.57974605)(612.11485771,141.50474613)(611.9648625,141.43474635)
\curveto(611.81485801,141.37474626)(611.65485817,141.31974631)(611.4848625,141.26974635)
\curveto(611.38485844,141.23974639)(611.27485855,141.21974641)(611.1548625,141.20974635)
\curveto(611.04485878,141.19974643)(610.93485889,141.18474645)(610.8248625,141.16474635)
\curveto(610.77485905,141.15474648)(610.7298591,141.14974648)(610.6898625,141.14974635)
\lineto(610.5848625,141.14974635)
\curveto(610.47485935,141.1297465)(610.36985946,141.1297465)(610.2698625,141.14974635)
\lineto(610.1348625,141.14974635)
\curveto(610.08485974,141.15974647)(610.03485979,141.16474647)(609.9848625,141.16474635)
\curveto(609.93485989,141.16474647)(609.88985994,141.17474646)(609.8498625,141.19474635)
\curveto(609.80986002,141.20474643)(609.77486005,141.20974642)(609.7448625,141.20974635)
\curveto(609.7248601,141.19974643)(609.69986013,141.19974643)(609.6698625,141.20974635)
\lineto(609.4298625,141.26974635)
\curveto(609.34986048,141.27974635)(609.27486055,141.29974633)(609.2048625,141.32974635)
\curveto(608.90486092,141.45974617)(608.65986117,141.60474603)(608.4698625,141.76474635)
\curveto(608.28986154,141.9347457)(608.13986169,142.16974546)(608.0198625,142.46974635)
\curveto(607.9298619,142.68974494)(607.88486194,142.95474468)(607.8848625,143.26474635)
\lineto(607.8848625,143.57974635)
\curveto(607.89486193,143.629744)(607.89986193,143.67974395)(607.8998625,143.72974635)
\lineto(607.9298625,143.90974635)
\lineto(608.0498625,144.23974635)
\curveto(608.08986174,144.34974328)(608.13986169,144.44974318)(608.1998625,144.53974635)
\curveto(608.37986145,144.8297428)(608.6248612,145.04474259)(608.9348625,145.18474635)
\curveto(609.24486058,145.32474231)(609.58486024,145.44974218)(609.9548625,145.55974635)
\curveto(610.09485973,145.59974203)(610.23985959,145.629742)(610.3898625,145.64974635)
\curveto(610.53985929,145.66974196)(610.68985914,145.69474194)(610.8398625,145.72474635)
\curveto(610.90985892,145.74474189)(610.97485885,145.75474188)(611.0348625,145.75474635)
\curveto(611.10485872,145.75474188)(611.17985865,145.76474187)(611.2598625,145.78474635)
\curveto(611.3298585,145.80474183)(611.39985843,145.81474182)(611.4698625,145.81474635)
\curveto(611.53985829,145.82474181)(611.61485821,145.83974179)(611.6948625,145.85974635)
\curveto(611.94485788,145.91974171)(612.17985765,145.96974166)(612.3998625,146.00974635)
\curveto(612.61985721,146.05974157)(612.79485703,146.17474146)(612.9248625,146.35474635)
\curveto(612.98485684,146.4347412)(613.03485679,146.5347411)(613.0748625,146.65474635)
\curveto(613.11485671,146.78474085)(613.11485671,146.92474071)(613.0748625,147.07474635)
\curveto(613.01485681,147.31474032)(612.9248569,147.50474013)(612.8048625,147.64474635)
\curveto(612.69485713,147.78473985)(612.53485729,147.89473974)(612.3248625,147.97474635)
\curveto(612.20485762,148.02473961)(612.05985777,148.05973957)(611.8898625,148.07974635)
\curveto(611.7298581,148.09973953)(611.55985827,148.10973952)(611.3798625,148.10974635)
\curveto(611.19985863,148.10973952)(611.0248588,148.09973953)(610.8548625,148.07974635)
\curveto(610.68485914,148.05973957)(610.53985929,148.0297396)(610.4198625,147.98974635)
\curveto(610.24985958,147.9297397)(610.08485974,147.84473979)(609.9248625,147.73474635)
\curveto(609.84485998,147.67473996)(609.76986006,147.59474004)(609.6998625,147.49474635)
\curveto(609.63986019,147.40474023)(609.58486024,147.30474033)(609.5348625,147.19474635)
\curveto(609.50486032,147.11474052)(609.47486035,147.0297406)(609.4448625,146.93974635)
\curveto(609.4248604,146.84974078)(609.37986045,146.77974085)(609.3098625,146.72974635)
\curveto(609.26986056,146.69974093)(609.19986063,146.67474096)(609.0998625,146.65474635)
\curveto(609.00986082,146.64474099)(608.91486091,146.63974099)(608.8148625,146.63974635)
\curveto(608.71486111,146.63974099)(608.61486121,146.64474099)(608.5148625,146.65474635)
\curveto(608.4248614,146.67474096)(608.35986147,146.69974093)(608.3198625,146.72974635)
\curveto(608.27986155,146.75974087)(608.24986158,146.80974082)(608.2298625,146.87974635)
\curveto(608.20986162,146.94974068)(608.20986162,147.02474061)(608.2298625,147.10474635)
\curveto(608.25986157,147.2347404)(608.28986154,147.35474028)(608.3198625,147.46474635)
\curveto(608.35986147,147.58474005)(608.40486142,147.69973993)(608.4548625,147.80974635)
\curveto(608.64486118,148.15973947)(608.88486094,148.4297392)(609.1748625,148.61974635)
\curveto(609.46486036,148.81973881)(609.82486,148.97973865)(610.2548625,149.09974635)
\curveto(610.35485947,149.11973851)(610.45485937,149.1347385)(610.5548625,149.14474635)
\curveto(610.66485916,149.15473848)(610.77485905,149.16973846)(610.8848625,149.18974635)
\curveto(610.9248589,149.19973843)(610.98985884,149.19973843)(611.0798625,149.18974635)
\curveto(611.16985866,149.18973844)(611.2248586,149.19973843)(611.2448625,149.21974635)
\curveto(611.94485788,149.2297384)(612.55485727,149.14973848)(613.0748625,148.97974635)
\curveto(613.59485623,148.80973882)(613.95985587,148.48473915)(614.1698625,148.00474635)
\curveto(614.25985557,147.80473983)(614.30985552,147.56974006)(614.3198625,147.29974635)
\curveto(614.33985549,147.03974059)(614.34985548,146.76474087)(614.3498625,146.47474635)
\lineto(614.3498625,143.15974635)
\curveto(614.34985548,143.01974461)(614.35485547,142.88474475)(614.3648625,142.75474635)
\curveto(614.37485545,142.62474501)(614.40485542,142.51974511)(614.4548625,142.43974635)
\curveto(614.50485532,142.36974526)(614.56985526,142.31974531)(614.6498625,142.28974635)
\curveto(614.73985509,142.24974538)(614.824855,142.21974541)(614.9048625,142.19974635)
\curveto(614.98485484,142.18974544)(615.04485478,142.14474549)(615.0848625,142.06474635)
\curveto(615.10485472,142.0347456)(615.11485471,142.00474563)(615.1148625,141.97474635)
\curveto(615.11485471,141.94474569)(615.11985471,141.90474573)(615.1298625,141.85474635)
\moveto(612.9848625,143.51974635)
\curveto(613.04485678,143.65974397)(613.07485675,143.81974381)(613.0748625,143.99974635)
\curveto(613.08485674,144.18974344)(613.08985674,144.38474325)(613.0898625,144.58474635)
\curveto(613.08985674,144.69474294)(613.08485674,144.79474284)(613.0748625,144.88474635)
\curveto(613.06485676,144.97474266)(613.0248568,145.04474259)(612.9548625,145.09474635)
\curveto(612.9248569,145.11474252)(612.85485697,145.12474251)(612.7448625,145.12474635)
\curveto(612.7248571,145.10474253)(612.68985714,145.09474254)(612.6398625,145.09474635)
\curveto(612.58985724,145.09474254)(612.54485728,145.08474255)(612.5048625,145.06474635)
\curveto(612.4248574,145.04474259)(612.33485749,145.02474261)(612.2348625,145.00474635)
\lineto(611.9348625,144.94474635)
\curveto(611.90485792,144.94474269)(611.86985796,144.93974269)(611.8298625,144.92974635)
\lineto(611.7248625,144.92974635)
\curveto(611.57485825,144.88974274)(611.40985842,144.86474277)(611.2298625,144.85474635)
\curveto(611.05985877,144.85474278)(610.89985893,144.8347428)(610.7498625,144.79474635)
\curveto(610.66985916,144.77474286)(610.59485923,144.75474288)(610.5248625,144.73474635)
\curveto(610.46485936,144.72474291)(610.39485943,144.70974292)(610.3148625,144.68974635)
\curveto(610.15485967,144.63974299)(610.00485982,144.57474306)(609.8648625,144.49474635)
\curveto(609.7248601,144.42474321)(609.60486022,144.3347433)(609.5048625,144.22474635)
\curveto(609.40486042,144.11474352)(609.3298605,143.97974365)(609.2798625,143.81974635)
\curveto(609.2298606,143.66974396)(609.20986062,143.48474415)(609.2198625,143.26474635)
\curveto(609.21986061,143.16474447)(609.23486059,143.06974456)(609.2648625,142.97974635)
\curveto(609.30486052,142.89974473)(609.34986048,142.82474481)(609.3998625,142.75474635)
\curveto(609.47986035,142.64474499)(609.58486024,142.54974508)(609.7148625,142.46974635)
\curveto(609.84485998,142.39974523)(609.98485984,142.33974529)(610.1348625,142.28974635)
\curveto(610.18485964,142.27974535)(610.23485959,142.27474536)(610.2848625,142.27474635)
\curveto(610.33485949,142.27474536)(610.38485944,142.26974536)(610.4348625,142.25974635)
\curveto(610.50485932,142.23974539)(610.58985924,142.22474541)(610.6898625,142.21474635)
\curveto(610.79985903,142.21474542)(610.88985894,142.22474541)(610.9598625,142.24474635)
\curveto(611.01985881,142.26474537)(611.07985875,142.26974536)(611.1398625,142.25974635)
\curveto(611.19985863,142.25974537)(611.25985857,142.26974536)(611.3198625,142.28974635)
\curveto(611.39985843,142.30974532)(611.47485835,142.32474531)(611.5448625,142.33474635)
\curveto(611.6248582,142.34474529)(611.69985813,142.36474527)(611.7698625,142.39474635)
\curveto(612.05985777,142.51474512)(612.30485752,142.65974497)(612.5048625,142.82974635)
\curveto(612.71485711,142.99974463)(612.87485695,143.2297444)(612.9848625,143.51974635)
}
}
{
\newrgbcolor{curcolor}{0 0 0}
\pscustom[linestyle=none,fillstyle=solid,fillcolor=curcolor]
{
\newpath
\moveto(623.26150312,142.10974635)
\lineto(623.26150312,141.71974635)
\curveto(623.26149525,141.59974603)(623.23649527,141.49974613)(623.18650312,141.41974635)
\curveto(623.13649537,141.34974628)(623.05149546,141.30974632)(622.93150312,141.29974635)
\lineto(622.58650312,141.29974635)
\curveto(622.52649598,141.29974633)(622.46649604,141.29474634)(622.40650312,141.28474635)
\curveto(622.35649615,141.28474635)(622.3114962,141.29474634)(622.27150312,141.31474635)
\curveto(622.18149633,141.3347463)(622.12149639,141.37474626)(622.09150312,141.43474635)
\curveto(622.05149646,141.48474615)(622.02649648,141.54474609)(622.01650312,141.61474635)
\curveto(622.01649649,141.68474595)(622.00149651,141.75474588)(621.97150312,141.82474635)
\curveto(621.96149655,141.84474579)(621.94649656,141.85974577)(621.92650312,141.86974635)
\curveto(621.91649659,141.88974574)(621.90149661,141.90974572)(621.88150312,141.92974635)
\curveto(621.78149673,141.93974569)(621.70149681,141.91974571)(621.64150312,141.86974635)
\curveto(621.59149692,141.81974581)(621.53649697,141.76974586)(621.47650312,141.71974635)
\curveto(621.27649723,141.56974606)(621.07649743,141.45474618)(620.87650312,141.37474635)
\curveto(620.69649781,141.29474634)(620.48649802,141.2347464)(620.24650312,141.19474635)
\curveto(620.01649849,141.15474648)(619.77649873,141.1347465)(619.52650312,141.13474635)
\curveto(619.28649922,141.12474651)(619.04649946,141.13974649)(618.80650312,141.17974635)
\curveto(618.56649994,141.20974642)(618.35650015,141.26474637)(618.17650312,141.34474635)
\curveto(617.65650085,141.56474607)(617.23650127,141.85974577)(616.91650312,142.22974635)
\curveto(616.59650191,142.60974502)(616.34650216,143.07974455)(616.16650312,143.63974635)
\curveto(616.12650238,143.7297439)(616.09650241,143.81974381)(616.07650312,143.90974635)
\curveto(616.06650244,144.00974362)(616.04650246,144.10974352)(616.01650312,144.20974635)
\curveto(616.0065025,144.25974337)(616.00150251,144.30974332)(616.00150312,144.35974635)
\curveto(616.00150251,144.40974322)(615.99650251,144.45974317)(615.98650312,144.50974635)
\curveto(615.96650254,144.55974307)(615.95650255,144.60974302)(615.95650312,144.65974635)
\curveto(615.96650254,144.71974291)(615.96650254,144.77474286)(615.95650312,144.82474635)
\lineto(615.95650312,144.97474635)
\curveto(615.93650257,145.02474261)(615.92650258,145.08974254)(615.92650312,145.16974635)
\curveto(615.92650258,145.24974238)(615.93650257,145.31474232)(615.95650312,145.36474635)
\lineto(615.95650312,145.52974635)
\curveto(615.97650253,145.59974203)(615.98150253,145.66974196)(615.97150312,145.73974635)
\curveto(615.97150254,145.81974181)(615.98150253,145.89474174)(616.00150312,145.96474635)
\curveto(616.0115025,146.01474162)(616.01650249,146.05974157)(616.01650312,146.09974635)
\curveto(616.01650249,146.13974149)(616.02150249,146.18474145)(616.03150312,146.23474635)
\curveto(616.06150245,146.3347413)(616.08650242,146.4297412)(616.10650312,146.51974635)
\curveto(616.12650238,146.61974101)(616.15150236,146.71474092)(616.18150312,146.80474635)
\curveto(616.3115022,147.18474045)(616.47650203,147.52474011)(616.67650312,147.82474635)
\curveto(616.88650162,148.1347395)(617.13650137,148.38973924)(617.42650312,148.58974635)
\curveto(617.59650091,148.70973892)(617.77150074,148.80973882)(617.95150312,148.88974635)
\curveto(618.14150037,148.96973866)(618.34650016,149.03973859)(618.56650312,149.09974635)
\curveto(618.63649987,149.10973852)(618.70149981,149.11973851)(618.76150312,149.12974635)
\curveto(618.83149968,149.13973849)(618.90149961,149.15473848)(618.97150312,149.17474635)
\lineto(619.12150312,149.17474635)
\curveto(619.20149931,149.19473844)(619.31649919,149.20473843)(619.46650312,149.20474635)
\curveto(619.62649888,149.20473843)(619.74649876,149.19473844)(619.82650312,149.17474635)
\curveto(619.86649864,149.16473847)(619.92149859,149.15973847)(619.99150312,149.15974635)
\curveto(620.10149841,149.1297385)(620.2114983,149.10473853)(620.32150312,149.08474635)
\curveto(620.43149808,149.07473856)(620.53649797,149.04473859)(620.63650312,148.99474635)
\curveto(620.78649772,148.9347387)(620.92649758,148.86973876)(621.05650312,148.79974635)
\curveto(621.19649731,148.7297389)(621.32649718,148.64973898)(621.44650312,148.55974635)
\curveto(621.506497,148.50973912)(621.56649694,148.45473918)(621.62650312,148.39474635)
\curveto(621.69649681,148.34473929)(621.78649672,148.3297393)(621.89650312,148.34974635)
\curveto(621.91649659,148.37973925)(621.93149658,148.40473923)(621.94150312,148.42474635)
\curveto(621.96149655,148.44473919)(621.97649653,148.47473916)(621.98650312,148.51474635)
\curveto(622.01649649,148.60473903)(622.02649648,148.71973891)(622.01650312,148.85974635)
\lineto(622.01650312,149.23474635)
\lineto(622.01650312,150.95974635)
\lineto(622.01650312,151.42474635)
\curveto(622.01649649,151.60473603)(622.04149647,151.7347359)(622.09150312,151.81474635)
\curveto(622.13149638,151.88473575)(622.19149632,151.9297357)(622.27150312,151.94974635)
\curveto(622.29149622,151.94973568)(622.31649619,151.94973568)(622.34650312,151.94974635)
\curveto(622.37649613,151.95973567)(622.40149611,151.96473567)(622.42150312,151.96474635)
\curveto(622.56149595,151.97473566)(622.7064958,151.97473566)(622.85650312,151.96474635)
\curveto(623.01649549,151.96473567)(623.12649538,151.92473571)(623.18650312,151.84474635)
\curveto(623.23649527,151.76473587)(623.26149525,151.66473597)(623.26150312,151.54474635)
\lineto(623.26150312,151.16974635)
\lineto(623.26150312,142.10974635)
\moveto(622.04650312,144.94474635)
\curveto(622.06649644,144.99474264)(622.07649643,145.05974257)(622.07650312,145.13974635)
\curveto(622.07649643,145.2297424)(622.06649644,145.29974233)(622.04650312,145.34974635)
\lineto(622.04650312,145.57474635)
\curveto(622.02649648,145.66474197)(622.0114965,145.75474188)(622.00150312,145.84474635)
\curveto(621.99149652,145.94474169)(621.97149654,146.0347416)(621.94150312,146.11474635)
\curveto(621.92149659,146.19474144)(621.90149661,146.26974136)(621.88150312,146.33974635)
\curveto(621.87149664,146.40974122)(621.85149666,146.47974115)(621.82150312,146.54974635)
\curveto(621.70149681,146.84974078)(621.54649696,147.11474052)(621.35650312,147.34474635)
\curveto(621.16649734,147.57474006)(620.92649758,147.75473988)(620.63650312,147.88474635)
\curveto(620.53649797,147.9347397)(620.43149808,147.96973966)(620.32150312,147.98974635)
\curveto(620.22149829,148.01973961)(620.1114984,148.04473959)(619.99150312,148.06474635)
\curveto(619.9114986,148.08473955)(619.82149869,148.09473954)(619.72150312,148.09474635)
\lineto(619.45150312,148.09474635)
\curveto(619.40149911,148.08473955)(619.35649915,148.07473956)(619.31650312,148.06474635)
\lineto(619.18150312,148.06474635)
\curveto(619.10149941,148.04473959)(619.01649949,148.02473961)(618.92650312,148.00474635)
\curveto(618.84649966,147.98473965)(618.76649974,147.95973967)(618.68650312,147.92974635)
\curveto(618.36650014,147.78973984)(618.1065004,147.58474005)(617.90650312,147.31474635)
\curveto(617.71650079,147.05474058)(617.56150095,146.74974088)(617.44150312,146.39974635)
\curveto(617.40150111,146.28974134)(617.37150114,146.17474146)(617.35150312,146.05474635)
\curveto(617.34150117,145.94474169)(617.32650118,145.8347418)(617.30650312,145.72474635)
\curveto(617.3065012,145.68474195)(617.30150121,145.64474199)(617.29150312,145.60474635)
\lineto(617.29150312,145.49974635)
\curveto(617.27150124,145.44974218)(617.26150125,145.39474224)(617.26150312,145.33474635)
\curveto(617.27150124,145.27474236)(617.27650123,145.21974241)(617.27650312,145.16974635)
\lineto(617.27650312,144.83974635)
\curveto(617.27650123,144.73974289)(617.28650122,144.64474299)(617.30650312,144.55474635)
\curveto(617.31650119,144.52474311)(617.32150119,144.47474316)(617.32150312,144.40474635)
\curveto(617.34150117,144.3347433)(617.35650115,144.26474337)(617.36650312,144.19474635)
\lineto(617.42650312,143.98474635)
\curveto(617.53650097,143.634744)(617.68650082,143.3347443)(617.87650312,143.08474635)
\curveto(618.06650044,142.8347448)(618.3065002,142.629745)(618.59650312,142.46974635)
\curveto(618.68649982,142.41974521)(618.77649973,142.37974525)(618.86650312,142.34974635)
\curveto(618.95649955,142.31974531)(619.05649945,142.28974534)(619.16650312,142.25974635)
\curveto(619.21649929,142.23974539)(619.26649924,142.2347454)(619.31650312,142.24474635)
\curveto(619.37649913,142.25474538)(619.43149908,142.24974538)(619.48150312,142.22974635)
\curveto(619.52149899,142.21974541)(619.56149895,142.21474542)(619.60150312,142.21474635)
\lineto(619.73650312,142.21474635)
\lineto(619.87150312,142.21474635)
\curveto(619.90149861,142.22474541)(619.95149856,142.2297454)(620.02150312,142.22974635)
\curveto(620.10149841,142.24974538)(620.18149833,142.26474537)(620.26150312,142.27474635)
\curveto(620.34149817,142.29474534)(620.41649809,142.31974531)(620.48650312,142.34974635)
\curveto(620.81649769,142.48974514)(621.08149743,142.66474497)(621.28150312,142.87474635)
\curveto(621.49149702,143.09474454)(621.66649684,143.36974426)(621.80650312,143.69974635)
\curveto(621.85649665,143.80974382)(621.89149662,143.91974371)(621.91150312,144.02974635)
\curveto(621.93149658,144.13974349)(621.95649655,144.24974338)(621.98650312,144.35974635)
\curveto(622.0064965,144.39974323)(622.01649649,144.4347432)(622.01650312,144.46474635)
\curveto(622.01649649,144.50474313)(622.02149649,144.54474309)(622.03150312,144.58474635)
\curveto(622.04149647,144.64474299)(622.04149647,144.70474293)(622.03150312,144.76474635)
\curveto(622.03149648,144.82474281)(622.03649647,144.88474275)(622.04650312,144.94474635)
}
}
{
\newrgbcolor{curcolor}{0 0 0}
\pscustom[linestyle=none,fillstyle=solid,fillcolor=curcolor]
{
\newpath
\moveto(632.33275312,145.49974635)
\curveto(632.35274506,145.43974219)(632.36274505,145.34474229)(632.36275312,145.21474635)
\curveto(632.36274505,145.09474254)(632.35774506,145.00974262)(632.34775312,144.95974635)
\lineto(632.34775312,144.80974635)
\curveto(632.33774508,144.7297429)(632.32774509,144.65474298)(632.31775312,144.58474635)
\curveto(632.3177451,144.52474311)(632.3127451,144.45474318)(632.30275312,144.37474635)
\curveto(632.28274513,144.31474332)(632.26774515,144.25474338)(632.25775312,144.19474635)
\curveto(632.25774516,144.1347435)(632.24774517,144.07474356)(632.22775312,144.01474635)
\curveto(632.18774523,143.88474375)(632.15274526,143.75474388)(632.12275312,143.62474635)
\curveto(632.09274532,143.49474414)(632.05274536,143.37474426)(632.00275312,143.26474635)
\curveto(631.79274562,142.78474485)(631.5127459,142.37974525)(631.16275312,142.04974635)
\curveto(630.8127466,141.7297459)(630.38274703,141.48474615)(629.87275312,141.31474635)
\curveto(629.76274765,141.27474636)(629.64274777,141.24474639)(629.51275312,141.22474635)
\curveto(629.39274802,141.20474643)(629.26774815,141.18474645)(629.13775312,141.16474635)
\curveto(629.07774834,141.15474648)(629.0127484,141.14974648)(628.94275312,141.14974635)
\curveto(628.88274853,141.13974649)(628.82274859,141.1347465)(628.76275312,141.13474635)
\curveto(628.72274869,141.12474651)(628.66274875,141.11974651)(628.58275312,141.11974635)
\curveto(628.5127489,141.11974651)(628.46274895,141.12474651)(628.43275312,141.13474635)
\curveto(628.39274902,141.14474649)(628.35274906,141.14974648)(628.31275312,141.14974635)
\curveto(628.27274914,141.13974649)(628.23774918,141.13974649)(628.20775312,141.14974635)
\lineto(628.11775312,141.14974635)
\lineto(627.75775312,141.19474635)
\curveto(627.6177498,141.2347464)(627.48274993,141.27474636)(627.35275312,141.31474635)
\curveto(627.22275019,141.35474628)(627.09775032,141.39974623)(626.97775312,141.44974635)
\curveto(626.52775089,141.64974598)(626.15775126,141.90974572)(625.86775312,142.22974635)
\curveto(625.57775184,142.54974508)(625.33775208,142.93974469)(625.14775312,143.39974635)
\curveto(625.09775232,143.49974413)(625.05775236,143.59974403)(625.02775312,143.69974635)
\curveto(625.00775241,143.79974383)(624.98775243,143.90474373)(624.96775312,144.01474635)
\curveto(624.94775247,144.05474358)(624.93775248,144.08474355)(624.93775312,144.10474635)
\curveto(624.94775247,144.1347435)(624.94775247,144.16974346)(624.93775312,144.20974635)
\curveto(624.9177525,144.28974334)(624.90275251,144.36974326)(624.89275312,144.44974635)
\curveto(624.89275252,144.53974309)(624.88275253,144.62474301)(624.86275312,144.70474635)
\lineto(624.86275312,144.82474635)
\curveto(624.86275255,144.86474277)(624.85775256,144.90974272)(624.84775312,144.95974635)
\curveto(624.83775258,145.00974262)(624.83275258,145.09474254)(624.83275312,145.21474635)
\curveto(624.83275258,145.34474229)(624.84275257,145.43974219)(624.86275312,145.49974635)
\curveto(624.88275253,145.56974206)(624.88775253,145.63974199)(624.87775312,145.70974635)
\curveto(624.86775255,145.77974185)(624.87275254,145.84974178)(624.89275312,145.91974635)
\curveto(624.90275251,145.96974166)(624.90775251,146.00974162)(624.90775312,146.03974635)
\curveto(624.9177525,146.07974155)(624.92775249,146.12474151)(624.93775312,146.17474635)
\curveto(624.96775245,146.29474134)(624.99275242,146.41474122)(625.01275312,146.53474635)
\curveto(625.04275237,146.65474098)(625.08275233,146.76974086)(625.13275312,146.87974635)
\curveto(625.28275213,147.24974038)(625.46275195,147.57974005)(625.67275312,147.86974635)
\curveto(625.89275152,148.16973946)(626.15775126,148.41973921)(626.46775312,148.61974635)
\curveto(626.58775083,148.69973893)(626.7127507,148.76473887)(626.84275312,148.81474635)
\curveto(626.97275044,148.87473876)(627.10775031,148.9347387)(627.24775312,148.99474635)
\curveto(627.36775005,149.04473859)(627.49774992,149.07473856)(627.63775312,149.08474635)
\curveto(627.77774964,149.10473853)(627.9177495,149.1347385)(628.05775312,149.17474635)
\lineto(628.25275312,149.17474635)
\curveto(628.32274909,149.18473845)(628.38774903,149.19473844)(628.44775312,149.20474635)
\curveto(629.33774808,149.21473842)(630.07774734,149.0297386)(630.66775312,148.64974635)
\curveto(631.25774616,148.26973936)(631.68274573,147.77473986)(631.94275312,147.16474635)
\curveto(631.99274542,147.06474057)(632.03274538,146.96474067)(632.06275312,146.86474635)
\curveto(632.09274532,146.76474087)(632.12774529,146.65974097)(632.16775312,146.54974635)
\curveto(632.19774522,146.43974119)(632.22274519,146.31974131)(632.24275312,146.18974635)
\curveto(632.26274515,146.06974156)(632.28774513,145.94474169)(632.31775312,145.81474635)
\curveto(632.32774509,145.76474187)(632.32774509,145.70974192)(632.31775312,145.64974635)
\curveto(632.3177451,145.59974203)(632.32274509,145.54974208)(632.33275312,145.49974635)
\moveto(630.99775312,144.64474635)
\curveto(631.0177464,144.71474292)(631.02274639,144.79474284)(631.01275312,144.88474635)
\lineto(631.01275312,145.13974635)
\curveto(631.0127464,145.5297421)(630.97774644,145.85974177)(630.90775312,146.12974635)
\curveto(630.87774654,146.20974142)(630.85274656,146.28974134)(630.83275312,146.36974635)
\curveto(630.8127466,146.44974118)(630.78774663,146.52474111)(630.75775312,146.59474635)
\curveto(630.47774694,147.24474039)(630.03274738,147.69473994)(629.42275312,147.94474635)
\curveto(629.35274806,147.97473966)(629.27774814,147.99473964)(629.19775312,148.00474635)
\lineto(628.95775312,148.06474635)
\curveto(628.87774854,148.08473955)(628.79274862,148.09473954)(628.70275312,148.09474635)
\lineto(628.43275312,148.09474635)
\lineto(628.16275312,148.04974635)
\curveto(628.06274935,148.0297396)(627.96774945,148.00473963)(627.87775312,147.97474635)
\curveto(627.79774962,147.95473968)(627.7177497,147.92473971)(627.63775312,147.88474635)
\curveto(627.56774985,147.86473977)(627.50274991,147.8347398)(627.44275312,147.79474635)
\curveto(627.38275003,147.75473988)(627.32775009,147.71473992)(627.27775312,147.67474635)
\curveto(627.03775038,147.50474013)(626.84275057,147.29974033)(626.69275312,147.05974635)
\curveto(626.54275087,146.81974081)(626.412751,146.53974109)(626.30275312,146.21974635)
\curveto(626.27275114,146.11974151)(626.25275116,146.01474162)(626.24275312,145.90474635)
\curveto(626.23275118,145.80474183)(626.2177512,145.69974193)(626.19775312,145.58974635)
\curveto(626.18775123,145.54974208)(626.18275123,145.48474215)(626.18275312,145.39474635)
\curveto(626.17275124,145.36474227)(626.16775125,145.3297423)(626.16775312,145.28974635)
\curveto(626.17775124,145.24974238)(626.18275123,145.20474243)(626.18275312,145.15474635)
\lineto(626.18275312,144.85474635)
\curveto(626.18275123,144.75474288)(626.19275122,144.66474297)(626.21275312,144.58474635)
\lineto(626.24275312,144.40474635)
\curveto(626.26275115,144.30474333)(626.27775114,144.20474343)(626.28775312,144.10474635)
\curveto(626.30775111,144.01474362)(626.33775108,143.9297437)(626.37775312,143.84974635)
\curveto(626.47775094,143.60974402)(626.59275082,143.38474425)(626.72275312,143.17474635)
\curveto(626.86275055,142.96474467)(627.03275038,142.78974484)(627.23275312,142.64974635)
\curveto(627.28275013,142.61974501)(627.32775009,142.59474504)(627.36775312,142.57474635)
\curveto(627.40775001,142.55474508)(627.45274996,142.5297451)(627.50275312,142.49974635)
\curveto(627.58274983,142.44974518)(627.66774975,142.40474523)(627.75775312,142.36474635)
\curveto(627.85774956,142.3347453)(627.96274945,142.30474533)(628.07275312,142.27474635)
\curveto(628.12274929,142.25474538)(628.16774925,142.24474539)(628.20775312,142.24474635)
\curveto(628.25774916,142.25474538)(628.30774911,142.25474538)(628.35775312,142.24474635)
\curveto(628.38774903,142.2347454)(628.44774897,142.22474541)(628.53775312,142.21474635)
\curveto(628.63774878,142.20474543)(628.7127487,142.20974542)(628.76275312,142.22974635)
\curveto(628.80274861,142.23974539)(628.84274857,142.23974539)(628.88275312,142.22974635)
\curveto(628.92274849,142.2297454)(628.96274845,142.23974539)(629.00275312,142.25974635)
\curveto(629.08274833,142.27974535)(629.16274825,142.29474534)(629.24275312,142.30474635)
\curveto(629.32274809,142.32474531)(629.39774802,142.34974528)(629.46775312,142.37974635)
\curveto(629.80774761,142.51974511)(630.08274733,142.71474492)(630.29275312,142.96474635)
\curveto(630.50274691,143.21474442)(630.67774674,143.50974412)(630.81775312,143.84974635)
\curveto(630.86774655,143.96974366)(630.89774652,144.09474354)(630.90775312,144.22474635)
\curveto(630.92774649,144.36474327)(630.95774646,144.50474313)(630.99775312,144.64474635)
}
}
{
\newrgbcolor{curcolor}{0 0 0}
\pscustom[linestyle=none,fillstyle=solid,fillcolor=curcolor]
{
\newpath
\moveto(637.46603437,149.20474635)
\curveto(637.69602958,149.20473843)(637.82602945,149.14473849)(637.85603437,149.02474635)
\curveto(637.88602939,148.91473872)(637.90102938,148.74973888)(637.90103437,148.52974635)
\lineto(637.90103437,148.24474635)
\curveto(637.90102938,148.15473948)(637.8760294,148.07973955)(637.82603437,148.01974635)
\curveto(637.76602951,147.93973969)(637.6810296,147.89473974)(637.57103437,147.88474635)
\curveto(637.46102982,147.88473975)(637.35102993,147.86973976)(637.24103437,147.83974635)
\curveto(637.10103018,147.80973982)(636.96603031,147.77973985)(636.83603437,147.74974635)
\curveto(636.71603056,147.71973991)(636.60103068,147.67973995)(636.49103437,147.62974635)
\curveto(636.20103108,147.49974013)(635.96603131,147.31974031)(635.78603437,147.08974635)
\curveto(635.60603167,146.86974076)(635.45103183,146.61474102)(635.32103437,146.32474635)
\curveto(635.281032,146.21474142)(635.25103203,146.09974153)(635.23103437,145.97974635)
\curveto(635.21103207,145.86974176)(635.18603209,145.75474188)(635.15603437,145.63474635)
\curveto(635.14603213,145.58474205)(635.14103214,145.5347421)(635.14103437,145.48474635)
\curveto(635.15103213,145.4347422)(635.15103213,145.38474225)(635.14103437,145.33474635)
\curveto(635.11103217,145.21474242)(635.09603218,145.07474256)(635.09603437,144.91474635)
\curveto(635.10603217,144.76474287)(635.11103217,144.61974301)(635.11103437,144.47974635)
\lineto(635.11103437,142.63474635)
\lineto(635.11103437,142.28974635)
\curveto(635.11103217,142.16974546)(635.10603217,142.05474558)(635.09603437,141.94474635)
\curveto(635.08603219,141.8347458)(635.0810322,141.73974589)(635.08103437,141.65974635)
\curveto(635.09103219,141.57974605)(635.07103221,141.50974612)(635.02103437,141.44974635)
\curveto(634.97103231,141.37974625)(634.89103239,141.33974629)(634.78103437,141.32974635)
\curveto(634.6810326,141.31974631)(634.57103271,141.31474632)(634.45103437,141.31474635)
\lineto(634.18103437,141.31474635)
\curveto(634.13103315,141.3347463)(634.0810332,141.34974628)(634.03103437,141.35974635)
\curveto(633.99103329,141.37974625)(633.96103332,141.40474623)(633.94103437,141.43474635)
\curveto(633.89103339,141.50474613)(633.86103342,141.58974604)(633.85103437,141.68974635)
\lineto(633.85103437,142.01974635)
\lineto(633.85103437,143.17474635)
\lineto(633.85103437,147.32974635)
\lineto(633.85103437,148.36474635)
\lineto(633.85103437,148.66474635)
\curveto(633.86103342,148.76473887)(633.89103339,148.84973878)(633.94103437,148.91974635)
\curveto(633.97103331,148.95973867)(634.02103326,148.98973864)(634.09103437,149.00974635)
\curveto(634.17103311,149.0297386)(634.25603302,149.03973859)(634.34603437,149.03974635)
\curveto(634.43603284,149.04973858)(634.52603275,149.04973858)(634.61603437,149.03974635)
\curveto(634.70603257,149.0297386)(634.7760325,149.01473862)(634.82603437,148.99474635)
\curveto(634.90603237,148.96473867)(634.95603232,148.90473873)(634.97603437,148.81474635)
\curveto(635.00603227,148.7347389)(635.02103226,148.64473899)(635.02103437,148.54474635)
\lineto(635.02103437,148.24474635)
\curveto(635.02103226,148.14473949)(635.04103224,148.05473958)(635.08103437,147.97474635)
\curveto(635.09103219,147.95473968)(635.10103218,147.93973969)(635.11103437,147.92974635)
\lineto(635.15603437,147.88474635)
\curveto(635.26603201,147.88473975)(635.35603192,147.9297397)(635.42603437,148.01974635)
\curveto(635.49603178,148.11973951)(635.55603172,148.19973943)(635.60603437,148.25974635)
\lineto(635.69603437,148.34974635)
\curveto(635.78603149,148.45973917)(635.91103137,148.57473906)(636.07103437,148.69474635)
\curveto(636.23103105,148.81473882)(636.3810309,148.90473873)(636.52103437,148.96474635)
\curveto(636.61103067,149.01473862)(636.70603057,149.04973858)(636.80603437,149.06974635)
\curveto(636.90603037,149.09973853)(637.01103027,149.1297385)(637.12103437,149.15974635)
\curveto(637.1810301,149.16973846)(637.24103004,149.17473846)(637.30103437,149.17474635)
\curveto(637.36102992,149.18473845)(637.41602986,149.19473844)(637.46603437,149.20474635)
}
}
{
\newrgbcolor{curcolor}{0.50196081 0.50196081 0.50196081}
\pscustom[linestyle=none,fillstyle=solid,fillcolor=curcolor]
{
\newpath
\moveto(545.29360762,152.00978297)
\lineto(560.29360762,152.00978297)
\lineto(560.29360762,137.00978297)
\lineto(545.29360762,137.00978297)
\closepath
}
}
{
\newrgbcolor{curcolor}{0 0 0}
\pscustom[linestyle=none,fillstyle=solid,fillcolor=curcolor]
{
\newpath
\moveto(573.66681562,124.03756861)
\lineto(573.66681562,123.76756861)
\curveto(573.67680565,123.67756336)(573.67180566,123.59756344)(573.65181562,123.52756861)
\lineto(573.65181562,123.37756861)
\curveto(573.64180569,123.34756369)(573.63680569,123.31256373)(573.63681562,123.27256861)
\curveto(573.64680568,123.23256381)(573.64680568,123.20256384)(573.63681562,123.18256861)
\curveto(573.6268057,123.13256391)(573.62180571,123.07756396)(573.62181562,123.01756861)
\curveto(573.62180571,122.96756407)(573.61680571,122.91756412)(573.60681562,122.86756861)
\curveto(573.57680575,122.72756431)(573.55680577,122.57756446)(573.54681562,122.41756861)
\curveto(573.53680579,122.26756477)(573.50680582,122.12256492)(573.45681562,121.98256861)
\curveto(573.4268059,121.86256518)(573.39180594,121.7375653)(573.35181562,121.60756861)
\curveto(573.32180601,121.48756555)(573.28180605,121.36756567)(573.23181562,121.24756861)
\curveto(573.06180627,120.81756622)(572.84680648,120.42756661)(572.58681562,120.07756861)
\curveto(572.33680699,119.7375673)(572.02180731,119.44756759)(571.64181562,119.20756861)
\curveto(571.45180788,119.08756795)(571.24680808,118.98256806)(571.02681562,118.89256861)
\curveto(570.81680851,118.81256823)(570.58680874,118.73256831)(570.33681562,118.65256861)
\curveto(570.2268091,118.61256843)(570.10680922,118.58256846)(569.97681562,118.56256861)
\curveto(569.85680947,118.55256849)(569.73680959,118.53256851)(569.61681562,118.50256861)
\curveto(569.50680982,118.48256856)(569.39680993,118.47256857)(569.28681562,118.47256861)
\curveto(569.18681014,118.47256857)(569.08681024,118.46256858)(568.98681562,118.44256861)
\lineto(568.77681562,118.44256861)
\curveto(568.74681058,118.43256861)(568.71181062,118.42756861)(568.67181562,118.42756861)
\curveto(568.6318107,118.4375686)(568.59181074,118.4425686)(568.55181562,118.44256861)
\lineto(565.55181562,118.44256861)
\curveto(565.40181393,118.4425686)(565.26681406,118.44756859)(565.14681562,118.45756861)
\curveto(565.03681429,118.47756856)(564.96181437,118.5425685)(564.92181562,118.65256861)
\curveto(564.88181445,118.73256831)(564.86181447,118.84756819)(564.86181562,118.99756861)
\curveto(564.87181446,119.14756789)(564.87681445,119.28256776)(564.87681562,119.40256861)
\lineto(564.87681562,128.26756861)
\curveto(564.87681445,128.38755865)(564.87181446,128.51255853)(564.86181562,128.64256861)
\curveto(564.86181447,128.78255826)(564.88681444,128.89255815)(564.93681562,128.97256861)
\curveto(564.97681435,129.042558)(565.05181428,129.08755795)(565.16181562,129.10756861)
\curveto(565.18181415,129.11755792)(565.20181413,129.11755792)(565.22181562,129.10756861)
\curveto(565.24181409,129.10755793)(565.26181407,129.11255793)(565.28181562,129.12256861)
\lineto(568.53681562,129.12256861)
\curveto(568.58681074,129.12255792)(568.6318107,129.12255792)(568.67181562,129.12256861)
\curveto(568.72181061,129.13255791)(568.76681056,129.13255791)(568.80681562,129.12256861)
\curveto(568.85681047,129.10255794)(568.90681042,129.09755794)(568.95681562,129.10756861)
\curveto(569.01681031,129.11755792)(569.07181026,129.11755792)(569.12181562,129.10756861)
\curveto(569.17181016,129.09755794)(569.2268101,129.09255795)(569.28681562,129.09256861)
\curveto(569.34680998,129.09255795)(569.40180993,129.08755795)(569.45181562,129.07756861)
\curveto(569.50180983,129.06755797)(569.54680978,129.06255798)(569.58681562,129.06256861)
\curveto(569.63680969,129.06255798)(569.68680964,129.05755798)(569.73681562,129.04756861)
\curveto(569.84680948,129.02755801)(569.95180938,129.00755803)(570.05181562,128.98756861)
\curveto(570.15180918,128.97755806)(570.25180908,128.95755808)(570.35181562,128.92756861)
\curveto(570.57180876,128.85755818)(570.78180855,128.78755825)(570.98181562,128.71756861)
\curveto(571.18180815,128.65755838)(571.36680796,128.57255847)(571.53681562,128.46256861)
\curveto(571.67680765,128.38255866)(571.80180753,128.30255874)(571.91181562,128.22256861)
\curveto(571.94180739,128.20255884)(571.97180736,128.17755886)(572.00181562,128.14756861)
\curveto(572.0318073,128.12755891)(572.06180727,128.10755893)(572.09181562,128.08756861)
\curveto(572.15180718,128.037559)(572.20680712,127.98755905)(572.25681562,127.93756861)
\curveto(572.30680702,127.88755915)(572.35680697,127.8375592)(572.40681562,127.78756861)
\curveto(572.45680687,127.7375593)(572.49680683,127.70255934)(572.52681562,127.68256861)
\curveto(572.56680676,127.62255942)(572.60680672,127.56755947)(572.64681562,127.51756861)
\curveto(572.69680663,127.46755957)(572.74180659,127.41255963)(572.78181562,127.35256861)
\curveto(572.8318065,127.29255975)(572.87180646,127.22755981)(572.90181562,127.15756861)
\curveto(572.94180639,127.09755994)(572.98680634,127.03256001)(573.03681562,126.96256861)
\curveto(573.05680627,126.92256012)(573.07180626,126.88756015)(573.08181562,126.85756861)
\curveto(573.09180624,126.82756021)(573.10680622,126.79256025)(573.12681562,126.75256861)
\curveto(573.16680616,126.67256037)(573.20180613,126.59256045)(573.23181562,126.51256861)
\curveto(573.26180607,126.4425606)(573.29680603,126.36756067)(573.33681562,126.28756861)
\curveto(573.37680595,126.17756086)(573.40680592,126.06256098)(573.42681562,125.94256861)
\curveto(573.45680587,125.83256121)(573.48680584,125.72256132)(573.51681562,125.61256861)
\curveto(573.53680579,125.55256149)(573.54680578,125.49256155)(573.54681562,125.43256861)
\curveto(573.54680578,125.38256166)(573.55680577,125.32756171)(573.57681562,125.26756861)
\curveto(573.6268057,125.08756195)(573.65180568,124.88756215)(573.65181562,124.66756861)
\curveto(573.66180567,124.45756258)(573.66680566,124.24756279)(573.66681562,124.03756861)
\moveto(572.24181562,123.25756861)
\curveto(572.26180707,123.35756368)(572.27180706,123.46256358)(572.27181562,123.57256861)
\lineto(572.27181562,123.91756861)
\lineto(572.27181562,124.14256861)
\curveto(572.28180705,124.22256282)(572.27680705,124.29756274)(572.25681562,124.36756861)
\curveto(572.25680707,124.39756264)(572.25180708,124.42756261)(572.24181562,124.45756861)
\lineto(572.24181562,124.56256861)
\curveto(572.22180711,124.67256237)(572.20680712,124.78256226)(572.19681562,124.89256861)
\curveto(572.19680713,125.00256204)(572.18180715,125.11256193)(572.15181562,125.22256861)
\curveto(572.1318072,125.30256174)(572.11180722,125.37756166)(572.09181562,125.44756861)
\curveto(572.08180725,125.52756151)(572.06680726,125.60756143)(572.04681562,125.68756861)
\curveto(571.93680739,126.04756099)(571.79680753,126.36256068)(571.62681562,126.63256861)
\curveto(571.34680798,127.08255996)(570.9318084,127.42255962)(570.38181562,127.65256861)
\curveto(570.29180904,127.70255934)(570.19680913,127.7375593)(570.09681562,127.75756861)
\curveto(569.99680933,127.78755925)(569.89180944,127.81755922)(569.78181562,127.84756861)
\curveto(569.67180966,127.87755916)(569.55680977,127.89255915)(569.43681562,127.89256861)
\curveto(569.32681,127.90255914)(569.21681011,127.91755912)(569.10681562,127.93756861)
\lineto(568.79181562,127.93756861)
\curveto(568.76181057,127.94755909)(568.7268106,127.95255909)(568.68681562,127.95256861)
\lineto(568.56681562,127.95256861)
\lineto(566.73681562,127.95256861)
\curveto(566.71681261,127.9425591)(566.69181264,127.9375591)(566.66181562,127.93756861)
\curveto(566.6318127,127.94755909)(566.60681272,127.94755909)(566.58681562,127.93756861)
\lineto(566.43681562,127.87756861)
\curveto(566.39681293,127.85755918)(566.36681296,127.82755921)(566.34681562,127.78756861)
\curveto(566.326813,127.74755929)(566.30681302,127.67755936)(566.28681562,127.57756861)
\lineto(566.28681562,127.45756861)
\curveto(566.27681305,127.41755962)(566.27181306,127.37255967)(566.27181562,127.32256861)
\lineto(566.27181562,127.18756861)
\lineto(566.27181562,120.37756861)
\lineto(566.27181562,120.22756861)
\curveto(566.27181306,120.18756685)(566.27681305,120.14756689)(566.28681562,120.10756861)
\lineto(566.28681562,119.98756861)
\curveto(566.30681302,119.88756715)(566.326813,119.81756722)(566.34681562,119.77756861)
\curveto(566.4268129,119.65756738)(566.57681275,119.59756744)(566.79681562,119.59756861)
\curveto(567.01681231,119.60756743)(567.2268121,119.61256743)(567.42681562,119.61256861)
\lineto(568.29681562,119.61256861)
\curveto(568.36681096,119.61256743)(568.44181089,119.60756743)(568.52181562,119.59756861)
\curveto(568.60181073,119.59756744)(568.67181066,119.60756743)(568.73181562,119.62756861)
\lineto(568.89681562,119.62756861)
\curveto(568.94681038,119.6375674)(569.00181033,119.6375674)(569.06181562,119.62756861)
\curveto(569.12181021,119.62756741)(569.18181015,119.63256741)(569.24181562,119.64256861)
\curveto(569.30181003,119.66256738)(569.36180997,119.67256737)(569.42181562,119.67256861)
\curveto(569.48180985,119.68256736)(569.54680978,119.69756734)(569.61681562,119.71756861)
\curveto(569.7268096,119.74756729)(569.8318095,119.77756726)(569.93181562,119.80756861)
\curveto(570.04180929,119.8375672)(570.15180918,119.87756716)(570.26181562,119.92756861)
\curveto(570.6318087,120.08756695)(570.94680838,120.30256674)(571.20681562,120.57256861)
\curveto(571.47680785,120.85256619)(571.69680763,121.18256586)(571.86681562,121.56256861)
\curveto(571.91680741,121.67256537)(571.95680737,121.78756525)(571.98681562,121.90756861)
\lineto(572.10681562,122.29756861)
\curveto(572.13680719,122.40756463)(572.15680717,122.52256452)(572.16681562,122.64256861)
\curveto(572.18680714,122.77256427)(572.20680712,122.89756414)(572.22681562,123.01756861)
\curveto(572.23680709,123.06756397)(572.24180709,123.10756393)(572.24181562,123.13756861)
\lineto(572.24181562,123.25756861)
}
}
{
\newrgbcolor{curcolor}{0 0 0}
\pscustom[linestyle=none,fillstyle=solid,fillcolor=curcolor]
{
\newpath
\moveto(581.91869062,122.61256861)
\curveto(581.93868294,122.51256453)(581.93868294,122.39756464)(581.91869062,122.26756861)
\curveto(581.90868297,122.14756489)(581.878683,122.06256498)(581.82869062,122.01256861)
\curveto(581.7786831,121.97256507)(581.70368317,121.9425651)(581.60369062,121.92256861)
\curveto(581.51368336,121.91256513)(581.40868347,121.90756513)(581.28869062,121.90756861)
\lineto(580.92869062,121.90756861)
\curveto(580.80868407,121.91756512)(580.70368417,121.92256512)(580.61369062,121.92256861)
\lineto(576.77369062,121.92256861)
\curveto(576.69368818,121.92256512)(576.61368826,121.91756512)(576.53369062,121.90756861)
\curveto(576.45368842,121.90756513)(576.38868849,121.89256515)(576.33869062,121.86256861)
\curveto(576.29868858,121.8425652)(576.25868862,121.80256524)(576.21869062,121.74256861)
\curveto(576.19868868,121.71256533)(576.1786887,121.66756537)(576.15869062,121.60756861)
\curveto(576.13868874,121.55756548)(576.13868874,121.50756553)(576.15869062,121.45756861)
\curveto(576.16868871,121.40756563)(576.1736887,121.36256568)(576.17369062,121.32256861)
\curveto(576.1736887,121.28256576)(576.1786887,121.2425658)(576.18869062,121.20256861)
\curveto(576.20868867,121.12256592)(576.22868865,121.037566)(576.24869062,120.94756861)
\curveto(576.26868861,120.86756617)(576.29868858,120.78756625)(576.33869062,120.70756861)
\curveto(576.56868831,120.16756687)(576.94868793,119.78256726)(577.47869062,119.55256861)
\curveto(577.53868734,119.52256752)(577.60368727,119.49756754)(577.67369062,119.47756861)
\lineto(577.88369062,119.41756861)
\curveto(577.91368696,119.40756763)(577.96368691,119.40256764)(578.03369062,119.40256861)
\curveto(578.1736867,119.36256768)(578.35868652,119.3425677)(578.58869062,119.34256861)
\curveto(578.81868606,119.3425677)(579.00368587,119.36256768)(579.14369062,119.40256861)
\curveto(579.28368559,119.4425676)(579.40868547,119.48256756)(579.51869062,119.52256861)
\curveto(579.63868524,119.57256747)(579.74868513,119.63256741)(579.84869062,119.70256861)
\curveto(579.95868492,119.77256727)(580.05368482,119.85256719)(580.13369062,119.94256861)
\curveto(580.21368466,120.042567)(580.28368459,120.14756689)(580.34369062,120.25756861)
\curveto(580.40368447,120.35756668)(580.45368442,120.46256658)(580.49369062,120.57256861)
\curveto(580.54368433,120.68256636)(580.62368425,120.76256628)(580.73369062,120.81256861)
\curveto(580.7736841,120.83256621)(580.83868404,120.84756619)(580.92869062,120.85756861)
\curveto(581.01868386,120.86756617)(581.10868377,120.86756617)(581.19869062,120.85756861)
\curveto(581.28868359,120.85756618)(581.3736835,120.85256619)(581.45369062,120.84256861)
\curveto(581.53368334,120.83256621)(581.58868329,120.81256623)(581.61869062,120.78256861)
\curveto(581.71868316,120.71256633)(581.74368313,120.59756644)(581.69369062,120.43756861)
\curveto(581.61368326,120.16756687)(581.50868337,119.92756711)(581.37869062,119.71756861)
\curveto(581.1786837,119.39756764)(580.94868393,119.13256791)(580.68869062,118.92256861)
\curveto(580.43868444,118.72256832)(580.11868476,118.55756848)(579.72869062,118.42756861)
\curveto(579.62868525,118.38756865)(579.52868535,118.36256868)(579.42869062,118.35256861)
\curveto(579.32868555,118.33256871)(579.22368565,118.31256873)(579.11369062,118.29256861)
\curveto(579.06368581,118.28256876)(579.01368586,118.27756876)(578.96369062,118.27756861)
\curveto(578.92368595,118.27756876)(578.878686,118.27256877)(578.82869062,118.26256861)
\lineto(578.67869062,118.26256861)
\curveto(578.62868625,118.25256879)(578.56868631,118.24756879)(578.49869062,118.24756861)
\curveto(578.43868644,118.24756879)(578.38868649,118.25256879)(578.34869062,118.26256861)
\lineto(578.21369062,118.26256861)
\curveto(578.16368671,118.27256877)(578.11868676,118.27756876)(578.07869062,118.27756861)
\curveto(578.03868684,118.27756876)(577.99868688,118.28256876)(577.95869062,118.29256861)
\curveto(577.90868697,118.30256874)(577.85368702,118.31256873)(577.79369062,118.32256861)
\curveto(577.73368714,118.32256872)(577.6786872,118.32756871)(577.62869062,118.33756861)
\curveto(577.53868734,118.35756868)(577.44868743,118.38256866)(577.35869062,118.41256861)
\curveto(577.26868761,118.43256861)(577.18368769,118.45756858)(577.10369062,118.48756861)
\curveto(577.06368781,118.50756853)(577.02868785,118.51756852)(576.99869062,118.51756861)
\curveto(576.96868791,118.52756851)(576.93368794,118.5425685)(576.89369062,118.56256861)
\curveto(576.74368813,118.63256841)(576.58368829,118.71756832)(576.41369062,118.81756861)
\curveto(576.12368875,119.00756803)(575.873689,119.2375678)(575.66369062,119.50756861)
\curveto(575.46368941,119.78756725)(575.29368958,120.09756694)(575.15369062,120.43756861)
\curveto(575.10368977,120.54756649)(575.06368981,120.66256638)(575.03369062,120.78256861)
\curveto(575.01368986,120.90256614)(574.98368989,121.02256602)(574.94369062,121.14256861)
\curveto(574.93368994,121.18256586)(574.92868995,121.21756582)(574.92869062,121.24756861)
\curveto(574.92868995,121.27756576)(574.92368995,121.31756572)(574.91369062,121.36756861)
\curveto(574.89368998,121.44756559)(574.87869,121.53256551)(574.86869062,121.62256861)
\curveto(574.85869002,121.71256533)(574.84369003,121.80256524)(574.82369062,121.89256861)
\lineto(574.82369062,122.10256861)
\curveto(574.81369006,122.1425649)(574.80369007,122.19756484)(574.79369062,122.26756861)
\curveto(574.79369008,122.34756469)(574.79869008,122.41256463)(574.80869062,122.46256861)
\lineto(574.80869062,122.62756861)
\curveto(574.82869005,122.67756436)(574.83369004,122.72756431)(574.82369062,122.77756861)
\curveto(574.82369005,122.8375642)(574.82869005,122.89256415)(574.83869062,122.94256861)
\curveto(574.87869,123.10256394)(574.90868997,123.26256378)(574.92869062,123.42256861)
\curveto(574.95868992,123.58256346)(575.00368987,123.73256331)(575.06369062,123.87256861)
\curveto(575.11368976,123.98256306)(575.15868972,124.09256295)(575.19869062,124.20256861)
\curveto(575.24868963,124.32256272)(575.30368957,124.4375626)(575.36369062,124.54756861)
\curveto(575.58368929,124.89756214)(575.83368904,125.19756184)(576.11369062,125.44756861)
\curveto(576.39368848,125.70756133)(576.73868814,125.92256112)(577.14869062,126.09256861)
\curveto(577.26868761,126.1425609)(577.38868749,126.17756086)(577.50869062,126.19756861)
\curveto(577.63868724,126.22756081)(577.7736871,126.25756078)(577.91369062,126.28756861)
\curveto(577.96368691,126.29756074)(578.00868687,126.30256074)(578.04869062,126.30256861)
\curveto(578.08868679,126.31256073)(578.13368674,126.31756072)(578.18369062,126.31756861)
\curveto(578.20368667,126.32756071)(578.22868665,126.32756071)(578.25869062,126.31756861)
\curveto(578.28868659,126.30756073)(578.31368656,126.31256073)(578.33369062,126.33256861)
\curveto(578.75368612,126.3425607)(579.11868576,126.29756074)(579.42869062,126.19756861)
\curveto(579.73868514,126.10756093)(580.01868486,125.98256106)(580.26869062,125.82256861)
\curveto(580.31868456,125.80256124)(580.35868452,125.77256127)(580.38869062,125.73256861)
\curveto(580.41868446,125.70256134)(580.45368442,125.67756136)(580.49369062,125.65756861)
\curveto(580.5736843,125.59756144)(580.65368422,125.52756151)(580.73369062,125.44756861)
\curveto(580.82368405,125.36756167)(580.89868398,125.28756175)(580.95869062,125.20756861)
\curveto(581.11868376,124.99756204)(581.25368362,124.79756224)(581.36369062,124.60756861)
\curveto(581.43368344,124.49756254)(581.48868339,124.37756266)(581.52869062,124.24756861)
\curveto(581.56868331,124.11756292)(581.61368326,123.98756305)(581.66369062,123.85756861)
\curveto(581.71368316,123.72756331)(581.74868313,123.59256345)(581.76869062,123.45256861)
\curveto(581.79868308,123.31256373)(581.83368304,123.17256387)(581.87369062,123.03256861)
\curveto(581.88368299,122.96256408)(581.88868299,122.89256415)(581.88869062,122.82256861)
\lineto(581.91869062,122.61256861)
\moveto(580.46369062,123.12256861)
\curveto(580.49368438,123.16256388)(580.51868436,123.21256383)(580.53869062,123.27256861)
\curveto(580.55868432,123.3425637)(580.55868432,123.41256363)(580.53869062,123.48256861)
\curveto(580.4786844,123.70256334)(580.39368448,123.90756313)(580.28369062,124.09756861)
\curveto(580.14368473,124.32756271)(579.98868489,124.52256252)(579.81869062,124.68256861)
\curveto(579.64868523,124.8425622)(579.42868545,124.97756206)(579.15869062,125.08756861)
\curveto(579.08868579,125.10756193)(579.01868586,125.12256192)(578.94869062,125.13256861)
\curveto(578.878686,125.15256189)(578.80368607,125.17256187)(578.72369062,125.19256861)
\curveto(578.64368623,125.21256183)(578.55868632,125.22256182)(578.46869062,125.22256861)
\lineto(578.21369062,125.22256861)
\curveto(578.18368669,125.20256184)(578.14868673,125.19256185)(578.10869062,125.19256861)
\curveto(578.06868681,125.20256184)(578.03368684,125.20256184)(578.00369062,125.19256861)
\lineto(577.76369062,125.13256861)
\curveto(577.69368718,125.12256192)(577.62368725,125.10756193)(577.55369062,125.08756861)
\curveto(577.26368761,124.96756207)(577.02868785,124.81756222)(576.84869062,124.63756861)
\curveto(576.6786882,124.45756258)(576.52368835,124.23256281)(576.38369062,123.96256861)
\curveto(576.35368852,123.91256313)(576.32368855,123.84756319)(576.29369062,123.76756861)
\curveto(576.26368861,123.69756334)(576.23868864,123.61756342)(576.21869062,123.52756861)
\curveto(576.19868868,123.4375636)(576.19368868,123.35256369)(576.20369062,123.27256861)
\curveto(576.21368866,123.19256385)(576.24868863,123.13256391)(576.30869062,123.09256861)
\curveto(576.38868849,123.03256401)(576.52368835,123.00256404)(576.71369062,123.00256861)
\curveto(576.91368796,123.01256403)(577.08368779,123.01756402)(577.22369062,123.01756861)
\lineto(579.50369062,123.01756861)
\curveto(579.65368522,123.01756402)(579.83368504,123.01256403)(580.04369062,123.00256861)
\curveto(580.25368462,123.00256404)(580.39368448,123.042564)(580.46369062,123.12256861)
}
}
{
\newrgbcolor{curcolor}{0 0 0}
\pscustom[linestyle=none,fillstyle=solid,fillcolor=curcolor]
{
\newpath
\moveto(585.65533125,126.34756861)
\curveto(586.37532718,126.35756068)(586.98032658,126.27256077)(587.47033125,126.09256861)
\curveto(587.9603256,125.92256112)(588.34032522,125.61756142)(588.61033125,125.17756861)
\curveto(588.68032488,125.06756197)(588.73532482,124.95256209)(588.77533125,124.83256861)
\curveto(588.81532474,124.72256232)(588.8553247,124.59756244)(588.89533125,124.45756861)
\curveto(588.91532464,124.38756265)(588.92032464,124.31256273)(588.91033125,124.23256861)
\curveto(588.90032466,124.16256288)(588.88532467,124.10756293)(588.86533125,124.06756861)
\curveto(588.84532471,124.04756299)(588.82032474,124.02756301)(588.79033125,124.00756861)
\curveto(588.7603248,123.99756304)(588.73532482,123.98256306)(588.71533125,123.96256861)
\curveto(588.66532489,123.9425631)(588.61532494,123.9375631)(588.56533125,123.94756861)
\curveto(588.51532504,123.95756308)(588.46532509,123.95756308)(588.41533125,123.94756861)
\curveto(588.33532522,123.92756311)(588.23032533,123.92256312)(588.10033125,123.93256861)
\curveto(587.97032559,123.95256309)(587.88032568,123.97756306)(587.83033125,124.00756861)
\curveto(587.75032581,124.05756298)(587.69532586,124.12256292)(587.66533125,124.20256861)
\curveto(587.64532591,124.29256275)(587.61032595,124.37756266)(587.56033125,124.45756861)
\curveto(587.47032609,124.61756242)(587.34532621,124.76256228)(587.18533125,124.89256861)
\curveto(587.07532648,124.97256207)(586.9553266,125.03256201)(586.82533125,125.07256861)
\curveto(586.69532686,125.11256193)(586.555327,125.15256189)(586.40533125,125.19256861)
\curveto(586.3553272,125.21256183)(586.30532725,125.21756182)(586.25533125,125.20756861)
\curveto(586.20532735,125.20756183)(586.1553274,125.21256183)(586.10533125,125.22256861)
\curveto(586.04532751,125.2425618)(585.97032759,125.25256179)(585.88033125,125.25256861)
\curveto(585.79032777,125.25256179)(585.71532784,125.2425618)(585.65533125,125.22256861)
\lineto(585.56533125,125.22256861)
\lineto(585.41533125,125.19256861)
\curveto(585.36532819,125.19256185)(585.31532824,125.18756185)(585.26533125,125.17756861)
\curveto(585.00532855,125.11756192)(584.79032877,125.03256201)(584.62033125,124.92256861)
\curveto(584.45032911,124.81256223)(584.33532922,124.62756241)(584.27533125,124.36756861)
\curveto(584.2553293,124.29756274)(584.25032931,124.22756281)(584.26033125,124.15756861)
\curveto(584.28032928,124.08756295)(584.30032926,124.02756301)(584.32033125,123.97756861)
\curveto(584.38032918,123.82756321)(584.45032911,123.71756332)(584.53033125,123.64756861)
\curveto(584.62032894,123.58756345)(584.73032883,123.51756352)(584.86033125,123.43756861)
\curveto(585.02032854,123.3375637)(585.20032836,123.26256378)(585.40033125,123.21256861)
\curveto(585.60032796,123.17256387)(585.80032776,123.12256392)(586.00033125,123.06256861)
\curveto(586.13032743,123.02256402)(586.2603273,122.99256405)(586.39033125,122.97256861)
\curveto(586.52032704,122.95256409)(586.65032691,122.92256412)(586.78033125,122.88256861)
\curveto(586.99032657,122.82256422)(587.19532636,122.76256428)(587.39533125,122.70256861)
\curveto(587.59532596,122.65256439)(587.79532576,122.58756445)(587.99533125,122.50756861)
\lineto(588.14533125,122.44756861)
\curveto(588.19532536,122.42756461)(588.24532531,122.40256464)(588.29533125,122.37256861)
\curveto(588.49532506,122.25256479)(588.67032489,122.11756492)(588.82033125,121.96756861)
\curveto(588.97032459,121.81756522)(589.09532446,121.62756541)(589.19533125,121.39756861)
\curveto(589.21532434,121.32756571)(589.23532432,121.23256581)(589.25533125,121.11256861)
\curveto(589.27532428,121.042566)(589.28532427,120.96756607)(589.28533125,120.88756861)
\curveto(589.29532426,120.81756622)(589.30032426,120.7375663)(589.30033125,120.64756861)
\lineto(589.30033125,120.49756861)
\curveto(589.28032428,120.42756661)(589.27032429,120.35756668)(589.27033125,120.28756861)
\curveto(589.27032429,120.21756682)(589.2603243,120.14756689)(589.24033125,120.07756861)
\curveto(589.21032435,119.96756707)(589.17532438,119.86256718)(589.13533125,119.76256861)
\curveto(589.09532446,119.66256738)(589.05032451,119.57256747)(589.00033125,119.49256861)
\curveto(588.84032472,119.23256781)(588.63532492,119.02256802)(588.38533125,118.86256861)
\curveto(588.13532542,118.71256833)(587.8553257,118.58256846)(587.54533125,118.47256861)
\curveto(587.4553261,118.4425686)(587.3603262,118.42256862)(587.26033125,118.41256861)
\curveto(587.17032639,118.39256865)(587.08032648,118.36756867)(586.99033125,118.33756861)
\curveto(586.89032667,118.31756872)(586.79032677,118.30756873)(586.69033125,118.30756861)
\curveto(586.59032697,118.30756873)(586.49032707,118.29756874)(586.39033125,118.27756861)
\lineto(586.24033125,118.27756861)
\curveto(586.19032737,118.26756877)(586.12032744,118.26256878)(586.03033125,118.26256861)
\curveto(585.94032762,118.26256878)(585.87032769,118.26756877)(585.82033125,118.27756861)
\lineto(585.65533125,118.27756861)
\curveto(585.59532796,118.29756874)(585.53032803,118.30756873)(585.46033125,118.30756861)
\curveto(585.39032817,118.29756874)(585.33032823,118.30256874)(585.28033125,118.32256861)
\curveto(585.23032833,118.33256871)(585.16532839,118.3375687)(585.08533125,118.33756861)
\lineto(584.84533125,118.39756861)
\curveto(584.77532878,118.40756863)(584.70032886,118.42756861)(584.62033125,118.45756861)
\curveto(584.31032925,118.55756848)(584.04032952,118.68256836)(583.81033125,118.83256861)
\curveto(583.58032998,118.98256806)(583.38033018,119.17756786)(583.21033125,119.41756861)
\curveto(583.12033044,119.54756749)(583.04533051,119.68256736)(582.98533125,119.82256861)
\curveto(582.92533063,119.96256708)(582.87033069,120.11756692)(582.82033125,120.28756861)
\curveto(582.80033076,120.34756669)(582.79033077,120.41756662)(582.79033125,120.49756861)
\curveto(582.80033076,120.58756645)(582.81533074,120.65756638)(582.83533125,120.70756861)
\curveto(582.86533069,120.74756629)(582.91533064,120.78756625)(582.98533125,120.82756861)
\curveto(583.03533052,120.84756619)(583.10533045,120.85756618)(583.19533125,120.85756861)
\curveto(583.28533027,120.86756617)(583.37533018,120.86756617)(583.46533125,120.85756861)
\curveto(583.55533,120.84756619)(583.64032992,120.83256621)(583.72033125,120.81256861)
\curveto(583.81032975,120.80256624)(583.87032969,120.78756625)(583.90033125,120.76756861)
\curveto(583.97032959,120.71756632)(584.01532954,120.6425664)(584.03533125,120.54256861)
\curveto(584.06532949,120.45256659)(584.10032946,120.36756667)(584.14033125,120.28756861)
\curveto(584.24032932,120.06756697)(584.37532918,119.89756714)(584.54533125,119.77756861)
\curveto(584.66532889,119.68756735)(584.80032876,119.61756742)(584.95033125,119.56756861)
\curveto(585.10032846,119.51756752)(585.2603283,119.46756757)(585.43033125,119.41756861)
\lineto(585.74533125,119.37256861)
\lineto(585.83533125,119.37256861)
\curveto(585.90532765,119.35256769)(585.99532756,119.3425677)(586.10533125,119.34256861)
\curveto(586.22532733,119.3425677)(586.32532723,119.35256769)(586.40533125,119.37256861)
\curveto(586.47532708,119.37256767)(586.53032703,119.37756766)(586.57033125,119.38756861)
\curveto(586.63032693,119.39756764)(586.69032687,119.40256764)(586.75033125,119.40256861)
\curveto(586.81032675,119.41256763)(586.86532669,119.42256762)(586.91533125,119.43256861)
\curveto(587.20532635,119.51256753)(587.43532612,119.61756742)(587.60533125,119.74756861)
\curveto(587.77532578,119.87756716)(587.89532566,120.09756694)(587.96533125,120.40756861)
\curveto(587.98532557,120.45756658)(587.99032557,120.51256653)(587.98033125,120.57256861)
\curveto(587.97032559,120.63256641)(587.9603256,120.67756636)(587.95033125,120.70756861)
\curveto(587.90032566,120.89756614)(587.83032573,121.037566)(587.74033125,121.12756861)
\curveto(587.65032591,121.22756581)(587.53532602,121.31756572)(587.39533125,121.39756861)
\curveto(587.30532625,121.45756558)(587.20532635,121.50756553)(587.09533125,121.54756861)
\lineto(586.76533125,121.66756861)
\curveto(586.73532682,121.67756536)(586.70532685,121.68256536)(586.67533125,121.68256861)
\curveto(586.6553269,121.68256536)(586.63032693,121.69256535)(586.60033125,121.71256861)
\curveto(586.2603273,121.82256522)(585.90532765,121.90256514)(585.53533125,121.95256861)
\curveto(585.17532838,122.01256503)(584.83532872,122.10756493)(584.51533125,122.23756861)
\curveto(584.41532914,122.27756476)(584.32032924,122.31256473)(584.23033125,122.34256861)
\curveto(584.14032942,122.37256467)(584.0553295,122.41256463)(583.97533125,122.46256861)
\curveto(583.78532977,122.57256447)(583.61032995,122.69756434)(583.45033125,122.83756861)
\curveto(583.29033027,122.97756406)(583.16533039,123.15256389)(583.07533125,123.36256861)
\curveto(583.04533051,123.43256361)(583.02033054,123.50256354)(583.00033125,123.57256861)
\curveto(582.99033057,123.6425634)(582.97533058,123.71756332)(582.95533125,123.79756861)
\curveto(582.92533063,123.91756312)(582.91533064,124.05256299)(582.92533125,124.20256861)
\curveto(582.93533062,124.36256268)(582.95033061,124.49756254)(582.97033125,124.60756861)
\curveto(582.99033057,124.65756238)(583.00033056,124.69756234)(583.00033125,124.72756861)
\curveto(583.01033055,124.76756227)(583.02533053,124.80756223)(583.04533125,124.84756861)
\curveto(583.13533042,125.07756196)(583.2553303,125.27756176)(583.40533125,125.44756861)
\curveto(583.56532999,125.61756142)(583.74532981,125.76756127)(583.94533125,125.89756861)
\curveto(584.09532946,125.98756105)(584.2603293,126.05756098)(584.44033125,126.10756861)
\curveto(584.62032894,126.16756087)(584.81032875,126.22256082)(585.01033125,126.27256861)
\curveto(585.08032848,126.28256076)(585.14532841,126.29256075)(585.20533125,126.30256861)
\curveto(585.27532828,126.31256073)(585.35032821,126.32256072)(585.43033125,126.33256861)
\curveto(585.4603281,126.3425607)(585.50032806,126.3425607)(585.55033125,126.33256861)
\curveto(585.60032796,126.32256072)(585.63532792,126.32756071)(585.65533125,126.34756861)
}
}
{
\newrgbcolor{curcolor}{0 0 0}
\pscustom[linestyle=none,fillstyle=solid,fillcolor=curcolor]
{
\newpath
\moveto(597.61033125,118.99756861)
\curveto(597.64032342,118.8375682)(597.62532343,118.70256834)(597.56533125,118.59256861)
\curveto(597.50532355,118.49256855)(597.42532363,118.41756862)(597.32533125,118.36756861)
\curveto(597.27532378,118.34756869)(597.22032384,118.3375687)(597.16033125,118.33756861)
\curveto(597.11032395,118.3375687)(597.055324,118.32756871)(596.99533125,118.30756861)
\curveto(596.77532428,118.25756878)(596.5553245,118.27256877)(596.33533125,118.35256861)
\curveto(596.12532493,118.42256862)(595.98032508,118.51256853)(595.90033125,118.62256861)
\curveto(595.85032521,118.69256835)(595.80532525,118.77256827)(595.76533125,118.86256861)
\curveto(595.72532533,118.96256808)(595.67532538,119.042568)(595.61533125,119.10256861)
\curveto(595.59532546,119.12256792)(595.57032549,119.1425679)(595.54033125,119.16256861)
\curveto(595.52032554,119.18256786)(595.49032557,119.18756785)(595.45033125,119.17756861)
\curveto(595.34032572,119.14756789)(595.23532582,119.09256795)(595.13533125,119.01256861)
\curveto(595.04532601,118.93256811)(594.9553261,118.86256818)(594.86533125,118.80256861)
\curveto(594.73532632,118.72256832)(594.59532646,118.64756839)(594.44533125,118.57756861)
\curveto(594.29532676,118.51756852)(594.13532692,118.46256858)(593.96533125,118.41256861)
\curveto(593.86532719,118.38256866)(593.7553273,118.36256868)(593.63533125,118.35256861)
\curveto(593.52532753,118.3425687)(593.41532764,118.32756871)(593.30533125,118.30756861)
\curveto(593.2553278,118.29756874)(593.21032785,118.29256875)(593.17033125,118.29256861)
\lineto(593.06533125,118.29256861)
\curveto(592.9553281,118.27256877)(592.85032821,118.27256877)(592.75033125,118.29256861)
\lineto(592.61533125,118.29256861)
\curveto(592.56532849,118.30256874)(592.51532854,118.30756873)(592.46533125,118.30756861)
\curveto(592.41532864,118.30756873)(592.37032869,118.31756872)(592.33033125,118.33756861)
\curveto(592.29032877,118.34756869)(592.2553288,118.35256869)(592.22533125,118.35256861)
\curveto(592.20532885,118.3425687)(592.18032888,118.3425687)(592.15033125,118.35256861)
\lineto(591.91033125,118.41256861)
\curveto(591.83032923,118.42256862)(591.7553293,118.4425686)(591.68533125,118.47256861)
\curveto(591.38532967,118.60256844)(591.14032992,118.74756829)(590.95033125,118.90756861)
\curveto(590.77033029,119.07756796)(590.62033044,119.31256773)(590.50033125,119.61256861)
\curveto(590.41033065,119.83256721)(590.36533069,120.09756694)(590.36533125,120.40756861)
\lineto(590.36533125,120.72256861)
\curveto(590.37533068,120.77256627)(590.38033068,120.82256622)(590.38033125,120.87256861)
\lineto(590.41033125,121.05256861)
\lineto(590.53033125,121.38256861)
\curveto(590.57033049,121.49256555)(590.62033044,121.59256545)(590.68033125,121.68256861)
\curveto(590.8603302,121.97256507)(591.10532995,122.18756485)(591.41533125,122.32756861)
\curveto(591.72532933,122.46756457)(592.06532899,122.59256445)(592.43533125,122.70256861)
\curveto(592.57532848,122.7425643)(592.72032834,122.77256427)(592.87033125,122.79256861)
\curveto(593.02032804,122.81256423)(593.17032789,122.8375642)(593.32033125,122.86756861)
\curveto(593.39032767,122.88756415)(593.4553276,122.89756414)(593.51533125,122.89756861)
\curveto(593.58532747,122.89756414)(593.6603274,122.90756413)(593.74033125,122.92756861)
\curveto(593.81032725,122.94756409)(593.88032718,122.95756408)(593.95033125,122.95756861)
\curveto(594.02032704,122.96756407)(594.09532696,122.98256406)(594.17533125,123.00256861)
\curveto(594.42532663,123.06256398)(594.6603264,123.11256393)(594.88033125,123.15256861)
\curveto(595.10032596,123.20256384)(595.27532578,123.31756372)(595.40533125,123.49756861)
\curveto(595.46532559,123.57756346)(595.51532554,123.67756336)(595.55533125,123.79756861)
\curveto(595.59532546,123.92756311)(595.59532546,124.06756297)(595.55533125,124.21756861)
\curveto(595.49532556,124.45756258)(595.40532565,124.64756239)(595.28533125,124.78756861)
\curveto(595.17532588,124.92756211)(595.01532604,125.037562)(594.80533125,125.11756861)
\curveto(594.68532637,125.16756187)(594.54032652,125.20256184)(594.37033125,125.22256861)
\curveto(594.21032685,125.2425618)(594.04032702,125.25256179)(593.86033125,125.25256861)
\curveto(593.68032738,125.25256179)(593.50532755,125.2425618)(593.33533125,125.22256861)
\curveto(593.16532789,125.20256184)(593.02032804,125.17256187)(592.90033125,125.13256861)
\curveto(592.73032833,125.07256197)(592.56532849,124.98756205)(592.40533125,124.87756861)
\curveto(592.32532873,124.81756222)(592.25032881,124.7375623)(592.18033125,124.63756861)
\curveto(592.12032894,124.54756249)(592.06532899,124.44756259)(592.01533125,124.33756861)
\curveto(591.98532907,124.25756278)(591.9553291,124.17256287)(591.92533125,124.08256861)
\curveto(591.90532915,123.99256305)(591.8603292,123.92256312)(591.79033125,123.87256861)
\curveto(591.75032931,123.8425632)(591.68032938,123.81756322)(591.58033125,123.79756861)
\curveto(591.49032957,123.78756325)(591.39532966,123.78256326)(591.29533125,123.78256861)
\curveto(591.19532986,123.78256326)(591.09532996,123.78756325)(590.99533125,123.79756861)
\curveto(590.90533015,123.81756322)(590.84033022,123.8425632)(590.80033125,123.87256861)
\curveto(590.7603303,123.90256314)(590.73033033,123.95256309)(590.71033125,124.02256861)
\curveto(590.69033037,124.09256295)(590.69033037,124.16756287)(590.71033125,124.24756861)
\curveto(590.74033032,124.37756266)(590.77033029,124.49756254)(590.80033125,124.60756861)
\curveto(590.84033022,124.72756231)(590.88533017,124.8425622)(590.93533125,124.95256861)
\curveto(591.12532993,125.30256174)(591.36532969,125.57256147)(591.65533125,125.76256861)
\curveto(591.94532911,125.96256108)(592.30532875,126.12256092)(592.73533125,126.24256861)
\curveto(592.83532822,126.26256078)(592.93532812,126.27756076)(593.03533125,126.28756861)
\curveto(593.14532791,126.29756074)(593.2553278,126.31256073)(593.36533125,126.33256861)
\curveto(593.40532765,126.3425607)(593.47032759,126.3425607)(593.56033125,126.33256861)
\curveto(593.65032741,126.33256071)(593.70532735,126.3425607)(593.72533125,126.36256861)
\curveto(594.42532663,126.37256067)(595.03532602,126.29256075)(595.55533125,126.12256861)
\curveto(596.07532498,125.95256109)(596.44032462,125.62756141)(596.65033125,125.14756861)
\curveto(596.74032432,124.94756209)(596.79032427,124.71256233)(596.80033125,124.44256861)
\curveto(596.82032424,124.18256286)(596.83032423,123.90756313)(596.83033125,123.61756861)
\lineto(596.83033125,120.30256861)
\curveto(596.83032423,120.16256688)(596.83532422,120.02756701)(596.84533125,119.89756861)
\curveto(596.8553242,119.76756727)(596.88532417,119.66256738)(596.93533125,119.58256861)
\curveto(596.98532407,119.51256753)(597.05032401,119.46256758)(597.13033125,119.43256861)
\curveto(597.22032384,119.39256765)(597.30532375,119.36256768)(597.38533125,119.34256861)
\curveto(597.46532359,119.33256771)(597.52532353,119.28756775)(597.56533125,119.20756861)
\curveto(597.58532347,119.17756786)(597.59532346,119.14756789)(597.59533125,119.11756861)
\curveto(597.59532346,119.08756795)(597.60032346,119.04756799)(597.61033125,118.99756861)
\moveto(595.46533125,120.66256861)
\curveto(595.52532553,120.80256624)(595.5553255,120.96256608)(595.55533125,121.14256861)
\curveto(595.56532549,121.33256571)(595.57032549,121.52756551)(595.57033125,121.72756861)
\curveto(595.57032549,121.8375652)(595.56532549,121.9375651)(595.55533125,122.02756861)
\curveto(595.54532551,122.11756492)(595.50532555,122.18756485)(595.43533125,122.23756861)
\curveto(595.40532565,122.25756478)(595.33532572,122.26756477)(595.22533125,122.26756861)
\curveto(595.20532585,122.24756479)(595.17032589,122.2375648)(595.12033125,122.23756861)
\curveto(595.07032599,122.2375648)(595.02532603,122.22756481)(594.98533125,122.20756861)
\curveto(594.90532615,122.18756485)(594.81532624,122.16756487)(594.71533125,122.14756861)
\lineto(594.41533125,122.08756861)
\curveto(594.38532667,122.08756495)(594.35032671,122.08256496)(594.31033125,122.07256861)
\lineto(594.20533125,122.07256861)
\curveto(594.055327,122.03256501)(593.89032717,122.00756503)(593.71033125,121.99756861)
\curveto(593.54032752,121.99756504)(593.38032768,121.97756506)(593.23033125,121.93756861)
\curveto(593.15032791,121.91756512)(593.07532798,121.89756514)(593.00533125,121.87756861)
\curveto(592.94532811,121.86756517)(592.87532818,121.85256519)(592.79533125,121.83256861)
\curveto(592.63532842,121.78256526)(592.48532857,121.71756532)(592.34533125,121.63756861)
\curveto(592.20532885,121.56756547)(592.08532897,121.47756556)(591.98533125,121.36756861)
\curveto(591.88532917,121.25756578)(591.81032925,121.12256592)(591.76033125,120.96256861)
\curveto(591.71032935,120.81256623)(591.69032937,120.62756641)(591.70033125,120.40756861)
\curveto(591.70032936,120.30756673)(591.71532934,120.21256683)(591.74533125,120.12256861)
\curveto(591.78532927,120.042567)(591.83032923,119.96756707)(591.88033125,119.89756861)
\curveto(591.9603291,119.78756725)(592.06532899,119.69256735)(592.19533125,119.61256861)
\curveto(592.32532873,119.5425675)(592.46532859,119.48256756)(592.61533125,119.43256861)
\curveto(592.66532839,119.42256762)(592.71532834,119.41756762)(592.76533125,119.41756861)
\curveto(592.81532824,119.41756762)(592.86532819,119.41256763)(592.91533125,119.40256861)
\curveto(592.98532807,119.38256766)(593.07032799,119.36756767)(593.17033125,119.35756861)
\curveto(593.28032778,119.35756768)(593.37032769,119.36756767)(593.44033125,119.38756861)
\curveto(593.50032756,119.40756763)(593.5603275,119.41256763)(593.62033125,119.40256861)
\curveto(593.68032738,119.40256764)(593.74032732,119.41256763)(593.80033125,119.43256861)
\curveto(593.88032718,119.45256759)(593.9553271,119.46756757)(594.02533125,119.47756861)
\curveto(594.10532695,119.48756755)(594.18032688,119.50756753)(594.25033125,119.53756861)
\curveto(594.54032652,119.65756738)(594.78532627,119.80256724)(594.98533125,119.97256861)
\curveto(595.19532586,120.1425669)(595.3553257,120.37256667)(595.46533125,120.66256861)
}
}
{
\newrgbcolor{curcolor}{0 0 0}
\pscustom[linestyle=none,fillstyle=solid,fillcolor=curcolor]
{
\newpath
\moveto(602.42697187,126.34756861)
\curveto(602.65696708,126.34756069)(602.78696695,126.28756075)(602.81697187,126.16756861)
\curveto(602.84696689,126.05756098)(602.86196688,125.89256115)(602.86197187,125.67256861)
\lineto(602.86197187,125.38756861)
\curveto(602.86196688,125.29756174)(602.8369669,125.22256182)(602.78697187,125.16256861)
\curveto(602.72696701,125.08256196)(602.6419671,125.037562)(602.53197187,125.02756861)
\curveto(602.42196732,125.02756201)(602.31196743,125.01256203)(602.20197187,124.98256861)
\curveto(602.06196768,124.95256209)(601.92696781,124.92256212)(601.79697187,124.89256861)
\curveto(601.67696806,124.86256218)(601.56196818,124.82256222)(601.45197187,124.77256861)
\curveto(601.16196858,124.6425624)(600.92696881,124.46256258)(600.74697187,124.23256861)
\curveto(600.56696917,124.01256303)(600.41196933,123.75756328)(600.28197187,123.46756861)
\curveto(600.2419695,123.35756368)(600.21196953,123.2425638)(600.19197187,123.12256861)
\curveto(600.17196957,123.01256403)(600.14696959,122.89756414)(600.11697187,122.77756861)
\curveto(600.10696963,122.72756431)(600.10196964,122.67756436)(600.10197187,122.62756861)
\curveto(600.11196963,122.57756446)(600.11196963,122.52756451)(600.10197187,122.47756861)
\curveto(600.07196967,122.35756468)(600.05696968,122.21756482)(600.05697187,122.05756861)
\curveto(600.06696967,121.90756513)(600.07196967,121.76256528)(600.07197187,121.62256861)
\lineto(600.07197187,119.77756861)
\lineto(600.07197187,119.43256861)
\curveto(600.07196967,119.31256773)(600.06696967,119.19756784)(600.05697187,119.08756861)
\curveto(600.04696969,118.97756806)(600.0419697,118.88256816)(600.04197187,118.80256861)
\curveto(600.05196969,118.72256832)(600.03196971,118.65256839)(599.98197187,118.59256861)
\curveto(599.93196981,118.52256852)(599.85196989,118.48256856)(599.74197187,118.47256861)
\curveto(599.6419701,118.46256858)(599.53197021,118.45756858)(599.41197187,118.45756861)
\lineto(599.14197187,118.45756861)
\curveto(599.09197065,118.47756856)(599.0419707,118.49256855)(598.99197187,118.50256861)
\curveto(598.95197079,118.52256852)(598.92197082,118.54756849)(598.90197187,118.57756861)
\curveto(598.85197089,118.64756839)(598.82197092,118.73256831)(598.81197187,118.83256861)
\lineto(598.81197187,119.16256861)
\lineto(598.81197187,120.31756861)
\lineto(598.81197187,124.47256861)
\lineto(598.81197187,125.50756861)
\lineto(598.81197187,125.80756861)
\curveto(598.82197092,125.90756113)(598.85197089,125.99256105)(598.90197187,126.06256861)
\curveto(598.93197081,126.10256094)(598.98197076,126.13256091)(599.05197187,126.15256861)
\curveto(599.13197061,126.17256087)(599.21697052,126.18256086)(599.30697187,126.18256861)
\curveto(599.39697034,126.19256085)(599.48697025,126.19256085)(599.57697187,126.18256861)
\curveto(599.66697007,126.17256087)(599.73697,126.15756088)(599.78697187,126.13756861)
\curveto(599.86696987,126.10756093)(599.91696982,126.04756099)(599.93697187,125.95756861)
\curveto(599.96696977,125.87756116)(599.98196976,125.78756125)(599.98197187,125.68756861)
\lineto(599.98197187,125.38756861)
\curveto(599.98196976,125.28756175)(600.00196974,125.19756184)(600.04197187,125.11756861)
\curveto(600.05196969,125.09756194)(600.06196968,125.08256196)(600.07197187,125.07256861)
\lineto(600.11697187,125.02756861)
\curveto(600.22696951,125.02756201)(600.31696942,125.07256197)(600.38697187,125.16256861)
\curveto(600.45696928,125.26256178)(600.51696922,125.3425617)(600.56697187,125.40256861)
\lineto(600.65697187,125.49256861)
\curveto(600.74696899,125.60256144)(600.87196887,125.71756132)(601.03197187,125.83756861)
\curveto(601.19196855,125.95756108)(601.3419684,126.04756099)(601.48197187,126.10756861)
\curveto(601.57196817,126.15756088)(601.66696807,126.19256085)(601.76697187,126.21256861)
\curveto(601.86696787,126.2425608)(601.97196777,126.27256077)(602.08197187,126.30256861)
\curveto(602.1419676,126.31256073)(602.20196754,126.31756072)(602.26197187,126.31756861)
\curveto(602.32196742,126.32756071)(602.37696736,126.3375607)(602.42697187,126.34756861)
}
}
{
\newrgbcolor{curcolor}{0 0 0}
\pscustom[linestyle=none,fillstyle=solid,fillcolor=curcolor]
{
\newpath
\moveto(607.4367375,126.34756861)
\curveto(607.66673271,126.34756069)(607.79673258,126.28756075)(607.8267375,126.16756861)
\curveto(607.85673252,126.05756098)(607.8717325,125.89256115)(607.8717375,125.67256861)
\lineto(607.8717375,125.38756861)
\curveto(607.8717325,125.29756174)(607.84673253,125.22256182)(607.7967375,125.16256861)
\curveto(607.73673264,125.08256196)(607.65173272,125.037562)(607.5417375,125.02756861)
\curveto(607.43173294,125.02756201)(607.32173305,125.01256203)(607.2117375,124.98256861)
\curveto(607.0717333,124.95256209)(606.93673344,124.92256212)(606.8067375,124.89256861)
\curveto(606.68673369,124.86256218)(606.5717338,124.82256222)(606.4617375,124.77256861)
\curveto(606.1717342,124.6425624)(605.93673444,124.46256258)(605.7567375,124.23256861)
\curveto(605.5767348,124.01256303)(605.42173495,123.75756328)(605.2917375,123.46756861)
\curveto(605.25173512,123.35756368)(605.22173515,123.2425638)(605.2017375,123.12256861)
\curveto(605.18173519,123.01256403)(605.15673522,122.89756414)(605.1267375,122.77756861)
\curveto(605.11673526,122.72756431)(605.11173526,122.67756436)(605.1117375,122.62756861)
\curveto(605.12173525,122.57756446)(605.12173525,122.52756451)(605.1117375,122.47756861)
\curveto(605.08173529,122.35756468)(605.06673531,122.21756482)(605.0667375,122.05756861)
\curveto(605.0767353,121.90756513)(605.08173529,121.76256528)(605.0817375,121.62256861)
\lineto(605.0817375,119.77756861)
\lineto(605.0817375,119.43256861)
\curveto(605.08173529,119.31256773)(605.0767353,119.19756784)(605.0667375,119.08756861)
\curveto(605.05673532,118.97756806)(605.05173532,118.88256816)(605.0517375,118.80256861)
\curveto(605.06173531,118.72256832)(605.04173533,118.65256839)(604.9917375,118.59256861)
\curveto(604.94173543,118.52256852)(604.86173551,118.48256856)(604.7517375,118.47256861)
\curveto(604.65173572,118.46256858)(604.54173583,118.45756858)(604.4217375,118.45756861)
\lineto(604.1517375,118.45756861)
\curveto(604.10173627,118.47756856)(604.05173632,118.49256855)(604.0017375,118.50256861)
\curveto(603.96173641,118.52256852)(603.93173644,118.54756849)(603.9117375,118.57756861)
\curveto(603.86173651,118.64756839)(603.83173654,118.73256831)(603.8217375,118.83256861)
\lineto(603.8217375,119.16256861)
\lineto(603.8217375,120.31756861)
\lineto(603.8217375,124.47256861)
\lineto(603.8217375,125.50756861)
\lineto(603.8217375,125.80756861)
\curveto(603.83173654,125.90756113)(603.86173651,125.99256105)(603.9117375,126.06256861)
\curveto(603.94173643,126.10256094)(603.99173638,126.13256091)(604.0617375,126.15256861)
\curveto(604.14173623,126.17256087)(604.22673615,126.18256086)(604.3167375,126.18256861)
\curveto(604.40673597,126.19256085)(604.49673588,126.19256085)(604.5867375,126.18256861)
\curveto(604.6767357,126.17256087)(604.74673563,126.15756088)(604.7967375,126.13756861)
\curveto(604.8767355,126.10756093)(604.92673545,126.04756099)(604.9467375,125.95756861)
\curveto(604.9767354,125.87756116)(604.99173538,125.78756125)(604.9917375,125.68756861)
\lineto(604.9917375,125.38756861)
\curveto(604.99173538,125.28756175)(605.01173536,125.19756184)(605.0517375,125.11756861)
\curveto(605.06173531,125.09756194)(605.0717353,125.08256196)(605.0817375,125.07256861)
\lineto(605.1267375,125.02756861)
\curveto(605.23673514,125.02756201)(605.32673505,125.07256197)(605.3967375,125.16256861)
\curveto(605.46673491,125.26256178)(605.52673485,125.3425617)(605.5767375,125.40256861)
\lineto(605.6667375,125.49256861)
\curveto(605.75673462,125.60256144)(605.88173449,125.71756132)(606.0417375,125.83756861)
\curveto(606.20173417,125.95756108)(606.35173402,126.04756099)(606.4917375,126.10756861)
\curveto(606.58173379,126.15756088)(606.6767337,126.19256085)(606.7767375,126.21256861)
\curveto(606.8767335,126.2425608)(606.98173339,126.27256077)(607.0917375,126.30256861)
\curveto(607.15173322,126.31256073)(607.21173316,126.31756072)(607.2717375,126.31756861)
\curveto(607.33173304,126.32756071)(607.38673299,126.3375607)(607.4367375,126.34756861)
}
}
{
\newrgbcolor{curcolor}{0 0 0}
\pscustom[linestyle=none,fillstyle=solid,fillcolor=curcolor]
{
\newpath
\moveto(615.92650312,122.64256861)
\curveto(615.94649506,122.58256446)(615.95649505,122.48756455)(615.95650312,122.35756861)
\curveto(615.95649505,122.2375648)(615.95149506,122.15256489)(615.94150312,122.10256861)
\lineto(615.94150312,121.95256861)
\curveto(615.93149508,121.87256517)(615.92149509,121.79756524)(615.91150312,121.72756861)
\curveto(615.9114951,121.66756537)(615.9064951,121.59756544)(615.89650312,121.51756861)
\curveto(615.87649513,121.45756558)(615.86149515,121.39756564)(615.85150312,121.33756861)
\curveto(615.85149516,121.27756576)(615.84149517,121.21756582)(615.82150312,121.15756861)
\curveto(615.78149523,121.02756601)(615.74649526,120.89756614)(615.71650312,120.76756861)
\curveto(615.68649532,120.6375664)(615.64649536,120.51756652)(615.59650312,120.40756861)
\curveto(615.38649562,119.92756711)(615.1064959,119.52256752)(614.75650312,119.19256861)
\curveto(614.4064966,118.87256817)(613.97649703,118.62756841)(613.46650312,118.45756861)
\curveto(613.35649765,118.41756862)(613.23649777,118.38756865)(613.10650312,118.36756861)
\curveto(612.98649802,118.34756869)(612.86149815,118.32756871)(612.73150312,118.30756861)
\curveto(612.67149834,118.29756874)(612.6064984,118.29256875)(612.53650312,118.29256861)
\curveto(612.47649853,118.28256876)(612.41649859,118.27756876)(612.35650312,118.27756861)
\curveto(612.31649869,118.26756877)(612.25649875,118.26256878)(612.17650312,118.26256861)
\curveto(612.1064989,118.26256878)(612.05649895,118.26756877)(612.02650312,118.27756861)
\curveto(611.98649902,118.28756875)(611.94649906,118.29256875)(611.90650312,118.29256861)
\curveto(611.86649914,118.28256876)(611.83149918,118.28256876)(611.80150312,118.29256861)
\lineto(611.71150312,118.29256861)
\lineto(611.35150312,118.33756861)
\curveto(611.2114998,118.37756866)(611.07649993,118.41756862)(610.94650312,118.45756861)
\curveto(610.81650019,118.49756854)(610.69150032,118.5425685)(610.57150312,118.59256861)
\curveto(610.12150089,118.79256825)(609.75150126,119.05256799)(609.46150312,119.37256861)
\curveto(609.17150184,119.69256735)(608.93150208,120.08256696)(608.74150312,120.54256861)
\curveto(608.69150232,120.6425664)(608.65150236,120.7425663)(608.62150312,120.84256861)
\curveto(608.60150241,120.9425661)(608.58150243,121.04756599)(608.56150312,121.15756861)
\curveto(608.54150247,121.19756584)(608.53150248,121.22756581)(608.53150312,121.24756861)
\curveto(608.54150247,121.27756576)(608.54150247,121.31256573)(608.53150312,121.35256861)
\curveto(608.5115025,121.43256561)(608.49650251,121.51256553)(608.48650312,121.59256861)
\curveto(608.48650252,121.68256536)(608.47650253,121.76756527)(608.45650312,121.84756861)
\lineto(608.45650312,121.96756861)
\curveto(608.45650255,122.00756503)(608.45150256,122.05256499)(608.44150312,122.10256861)
\curveto(608.43150258,122.15256489)(608.42650258,122.2375648)(608.42650312,122.35756861)
\curveto(608.42650258,122.48756455)(608.43650257,122.58256446)(608.45650312,122.64256861)
\curveto(608.47650253,122.71256433)(608.48150253,122.78256426)(608.47150312,122.85256861)
\curveto(608.46150255,122.92256412)(608.46650254,122.99256405)(608.48650312,123.06256861)
\curveto(608.49650251,123.11256393)(608.50150251,123.15256389)(608.50150312,123.18256861)
\curveto(608.5115025,123.22256382)(608.52150249,123.26756377)(608.53150312,123.31756861)
\curveto(608.56150245,123.4375636)(608.58650242,123.55756348)(608.60650312,123.67756861)
\curveto(608.63650237,123.79756324)(608.67650233,123.91256313)(608.72650312,124.02256861)
\curveto(608.87650213,124.39256265)(609.05650195,124.72256232)(609.26650312,125.01256861)
\curveto(609.48650152,125.31256173)(609.75150126,125.56256148)(610.06150312,125.76256861)
\curveto(610.18150083,125.8425612)(610.3065007,125.90756113)(610.43650312,125.95756861)
\curveto(610.56650044,126.01756102)(610.70150031,126.07756096)(610.84150312,126.13756861)
\curveto(610.96150005,126.18756085)(611.09149992,126.21756082)(611.23150312,126.22756861)
\curveto(611.37149964,126.24756079)(611.5114995,126.27756076)(611.65150312,126.31756861)
\lineto(611.84650312,126.31756861)
\curveto(611.91649909,126.32756071)(611.98149903,126.3375607)(612.04150312,126.34756861)
\curveto(612.93149808,126.35756068)(613.67149734,126.17256087)(614.26150312,125.79256861)
\curveto(614.85149616,125.41256163)(615.27649573,124.91756212)(615.53650312,124.30756861)
\curveto(615.58649542,124.20756283)(615.62649538,124.10756293)(615.65650312,124.00756861)
\curveto(615.68649532,123.90756313)(615.72149529,123.80256324)(615.76150312,123.69256861)
\curveto(615.79149522,123.58256346)(615.81649519,123.46256358)(615.83650312,123.33256861)
\curveto(615.85649515,123.21256383)(615.88149513,123.08756395)(615.91150312,122.95756861)
\curveto(615.92149509,122.90756413)(615.92149509,122.85256419)(615.91150312,122.79256861)
\curveto(615.9114951,122.7425643)(615.91649509,122.69256435)(615.92650312,122.64256861)
\moveto(614.59150312,121.78756861)
\curveto(614.6114964,121.85756518)(614.61649639,121.9375651)(614.60650312,122.02756861)
\lineto(614.60650312,122.28256861)
\curveto(614.6064964,122.67256437)(614.57149644,123.00256404)(614.50150312,123.27256861)
\curveto(614.47149654,123.35256369)(614.44649656,123.43256361)(614.42650312,123.51256861)
\curveto(614.4064966,123.59256345)(614.38149663,123.66756337)(614.35150312,123.73756861)
\curveto(614.07149694,124.38756265)(613.62649738,124.8375622)(613.01650312,125.08756861)
\curveto(612.94649806,125.11756192)(612.87149814,125.1375619)(612.79150312,125.14756861)
\lineto(612.55150312,125.20756861)
\curveto(612.47149854,125.22756181)(612.38649862,125.2375618)(612.29650312,125.23756861)
\lineto(612.02650312,125.23756861)
\lineto(611.75650312,125.19256861)
\curveto(611.65649935,125.17256187)(611.56149945,125.14756189)(611.47150312,125.11756861)
\curveto(611.39149962,125.09756194)(611.3114997,125.06756197)(611.23150312,125.02756861)
\curveto(611.16149985,125.00756203)(611.09649991,124.97756206)(611.03650312,124.93756861)
\curveto(610.97650003,124.89756214)(610.92150009,124.85756218)(610.87150312,124.81756861)
\curveto(610.63150038,124.64756239)(610.43650057,124.4425626)(610.28650312,124.20256861)
\curveto(610.13650087,123.96256308)(610.006501,123.68256336)(609.89650312,123.36256861)
\curveto(609.86650114,123.26256378)(609.84650116,123.15756388)(609.83650312,123.04756861)
\curveto(609.82650118,122.94756409)(609.8115012,122.8425642)(609.79150312,122.73256861)
\curveto(609.78150123,122.69256435)(609.77650123,122.62756441)(609.77650312,122.53756861)
\curveto(609.76650124,122.50756453)(609.76150125,122.47256457)(609.76150312,122.43256861)
\curveto(609.77150124,122.39256465)(609.77650123,122.34756469)(609.77650312,122.29756861)
\lineto(609.77650312,121.99756861)
\curveto(609.77650123,121.89756514)(609.78650122,121.80756523)(609.80650312,121.72756861)
\lineto(609.83650312,121.54756861)
\curveto(609.85650115,121.44756559)(609.87150114,121.34756569)(609.88150312,121.24756861)
\curveto(609.90150111,121.15756588)(609.93150108,121.07256597)(609.97150312,120.99256861)
\curveto(610.07150094,120.75256629)(610.18650082,120.52756651)(610.31650312,120.31756861)
\curveto(610.45650055,120.10756693)(610.62650038,119.93256711)(610.82650312,119.79256861)
\curveto(610.87650013,119.76256728)(610.92150009,119.7375673)(610.96150312,119.71756861)
\curveto(611.00150001,119.69756734)(611.04649996,119.67256737)(611.09650312,119.64256861)
\curveto(611.17649983,119.59256745)(611.26149975,119.54756749)(611.35150312,119.50756861)
\curveto(611.45149956,119.47756756)(611.55649945,119.44756759)(611.66650312,119.41756861)
\curveto(611.71649929,119.39756764)(611.76149925,119.38756765)(611.80150312,119.38756861)
\curveto(611.85149916,119.39756764)(611.90149911,119.39756764)(611.95150312,119.38756861)
\curveto(611.98149903,119.37756766)(612.04149897,119.36756767)(612.13150312,119.35756861)
\curveto(612.23149878,119.34756769)(612.3064987,119.35256769)(612.35650312,119.37256861)
\curveto(612.39649861,119.38256766)(612.43649857,119.38256766)(612.47650312,119.37256861)
\curveto(612.51649849,119.37256767)(612.55649845,119.38256766)(612.59650312,119.40256861)
\curveto(612.67649833,119.42256762)(612.75649825,119.4375676)(612.83650312,119.44756861)
\curveto(612.91649809,119.46756757)(612.99149802,119.49256755)(613.06150312,119.52256861)
\curveto(613.40149761,119.66256738)(613.67649733,119.85756718)(613.88650312,120.10756861)
\curveto(614.09649691,120.35756668)(614.27149674,120.65256639)(614.41150312,120.99256861)
\curveto(614.46149655,121.11256593)(614.49149652,121.2375658)(614.50150312,121.36756861)
\curveto(614.52149649,121.50756553)(614.55149646,121.64756539)(614.59150312,121.78756861)
}
}
{
\newrgbcolor{curcolor}{0 0 0}
\pscustom[linestyle=none,fillstyle=solid,fillcolor=curcolor]
{
\newpath
\moveto(617.98478437,129.12256861)
\curveto(618.11478276,129.12255792)(618.24978262,129.12255792)(618.38978437,129.12256861)
\curveto(618.53978233,129.12255792)(618.64978222,129.08755795)(618.71978437,129.01756861)
\curveto(618.7697821,128.94755809)(618.79478208,128.85255819)(618.79478437,128.73256861)
\curveto(618.80478207,128.62255842)(618.80978206,128.50755853)(618.80978437,128.38756861)
\lineto(618.80978437,127.05256861)
\lineto(618.80978437,120.97756861)
\lineto(618.80978437,119.29756861)
\lineto(618.80978437,118.90756861)
\curveto(618.80978206,118.76756827)(618.78478209,118.65756838)(618.73478437,118.57756861)
\curveto(618.70478217,118.52756851)(618.65978221,118.49756854)(618.59978437,118.48756861)
\curveto(618.54978232,118.47756856)(618.48478239,118.46256858)(618.40478437,118.44256861)
\lineto(618.19478437,118.44256861)
\lineto(617.87978437,118.44256861)
\curveto(617.77978309,118.45256859)(617.70478317,118.48756855)(617.65478437,118.54756861)
\curveto(617.60478327,118.62756841)(617.5747833,118.72756831)(617.56478437,118.84756861)
\lineto(617.56478437,119.22256861)
\lineto(617.56478437,120.60256861)
\lineto(617.56478437,126.84256861)
\lineto(617.56478437,128.31256861)
\curveto(617.56478331,128.42255862)(617.55978331,128.5375585)(617.54978437,128.65756861)
\curveto(617.54978332,128.78755825)(617.5747833,128.88755815)(617.62478437,128.95756861)
\curveto(617.66478321,129.01755802)(617.73978313,129.06755797)(617.84978437,129.10756861)
\curveto(617.869783,129.11755792)(617.88978298,129.11755792)(617.90978437,129.10756861)
\curveto(617.93978293,129.10755793)(617.96478291,129.11255793)(617.98478437,129.12256861)
}
}
{
\newrgbcolor{curcolor}{0 0 0}
\pscustom[linestyle=none,fillstyle=solid,fillcolor=curcolor]
{
\newpath
\moveto(621.32462812,129.12256861)
\curveto(621.45462651,129.12255792)(621.58962637,129.12255792)(621.72962812,129.12256861)
\curveto(621.87962608,129.12255792)(621.98962597,129.08755795)(622.05962812,129.01756861)
\curveto(622.10962585,128.94755809)(622.13462583,128.85255819)(622.13462812,128.73256861)
\curveto(622.14462582,128.62255842)(622.14962581,128.50755853)(622.14962812,128.38756861)
\lineto(622.14962812,127.05256861)
\lineto(622.14962812,120.97756861)
\lineto(622.14962812,119.29756861)
\lineto(622.14962812,118.90756861)
\curveto(622.14962581,118.76756827)(622.12462584,118.65756838)(622.07462812,118.57756861)
\curveto(622.04462592,118.52756851)(621.99962596,118.49756854)(621.93962812,118.48756861)
\curveto(621.88962607,118.47756856)(621.82462614,118.46256858)(621.74462812,118.44256861)
\lineto(621.53462812,118.44256861)
\lineto(621.21962812,118.44256861)
\curveto(621.11962684,118.45256859)(621.04462692,118.48756855)(620.99462812,118.54756861)
\curveto(620.94462702,118.62756841)(620.91462705,118.72756831)(620.90462812,118.84756861)
\lineto(620.90462812,119.22256861)
\lineto(620.90462812,120.60256861)
\lineto(620.90462812,126.84256861)
\lineto(620.90462812,128.31256861)
\curveto(620.90462706,128.42255862)(620.89962706,128.5375585)(620.88962812,128.65756861)
\curveto(620.88962707,128.78755825)(620.91462705,128.88755815)(620.96462812,128.95756861)
\curveto(621.00462696,129.01755802)(621.07962688,129.06755797)(621.18962812,129.10756861)
\curveto(621.20962675,129.11755792)(621.22962673,129.11755792)(621.24962812,129.10756861)
\curveto(621.27962668,129.10755793)(621.30462666,129.11255793)(621.32462812,129.12256861)
}
}
{
\newrgbcolor{curcolor}{0 0 0}
\pscustom[linestyle=none,fillstyle=solid,fillcolor=curcolor]
{
\newpath
\moveto(630.97947187,118.99756861)
\curveto(631.00946404,118.8375682)(630.99446406,118.70256834)(630.93447187,118.59256861)
\curveto(630.87446418,118.49256855)(630.79446426,118.41756862)(630.69447187,118.36756861)
\curveto(630.64446441,118.34756869)(630.58946446,118.3375687)(630.52947187,118.33756861)
\curveto(630.47946457,118.3375687)(630.42446463,118.32756871)(630.36447187,118.30756861)
\curveto(630.14446491,118.25756878)(629.92446513,118.27256877)(629.70447187,118.35256861)
\curveto(629.49446556,118.42256862)(629.3494657,118.51256853)(629.26947187,118.62256861)
\curveto(629.21946583,118.69256835)(629.17446588,118.77256827)(629.13447187,118.86256861)
\curveto(629.09446596,118.96256808)(629.04446601,119.042568)(628.98447187,119.10256861)
\curveto(628.96446609,119.12256792)(628.93946611,119.1425679)(628.90947187,119.16256861)
\curveto(628.88946616,119.18256786)(628.85946619,119.18756785)(628.81947187,119.17756861)
\curveto(628.70946634,119.14756789)(628.60446645,119.09256795)(628.50447187,119.01256861)
\curveto(628.41446664,118.93256811)(628.32446673,118.86256818)(628.23447187,118.80256861)
\curveto(628.10446695,118.72256832)(627.96446709,118.64756839)(627.81447187,118.57756861)
\curveto(627.66446739,118.51756852)(627.50446755,118.46256858)(627.33447187,118.41256861)
\curveto(627.23446782,118.38256866)(627.12446793,118.36256868)(627.00447187,118.35256861)
\curveto(626.89446816,118.3425687)(626.78446827,118.32756871)(626.67447187,118.30756861)
\curveto(626.62446843,118.29756874)(626.57946847,118.29256875)(626.53947187,118.29256861)
\lineto(626.43447187,118.29256861)
\curveto(626.32446873,118.27256877)(626.21946883,118.27256877)(626.11947187,118.29256861)
\lineto(625.98447187,118.29256861)
\curveto(625.93446912,118.30256874)(625.88446917,118.30756873)(625.83447187,118.30756861)
\curveto(625.78446927,118.30756873)(625.73946931,118.31756872)(625.69947187,118.33756861)
\curveto(625.65946939,118.34756869)(625.62446943,118.35256869)(625.59447187,118.35256861)
\curveto(625.57446948,118.3425687)(625.5494695,118.3425687)(625.51947187,118.35256861)
\lineto(625.27947187,118.41256861)
\curveto(625.19946985,118.42256862)(625.12446993,118.4425686)(625.05447187,118.47256861)
\curveto(624.7544703,118.60256844)(624.50947054,118.74756829)(624.31947187,118.90756861)
\curveto(624.13947091,119.07756796)(623.98947106,119.31256773)(623.86947187,119.61256861)
\curveto(623.77947127,119.83256721)(623.73447132,120.09756694)(623.73447187,120.40756861)
\lineto(623.73447187,120.72256861)
\curveto(623.74447131,120.77256627)(623.7494713,120.82256622)(623.74947187,120.87256861)
\lineto(623.77947187,121.05256861)
\lineto(623.89947187,121.38256861)
\curveto(623.93947111,121.49256555)(623.98947106,121.59256545)(624.04947187,121.68256861)
\curveto(624.22947082,121.97256507)(624.47447058,122.18756485)(624.78447187,122.32756861)
\curveto(625.09446996,122.46756457)(625.43446962,122.59256445)(625.80447187,122.70256861)
\curveto(625.94446911,122.7425643)(626.08946896,122.77256427)(626.23947187,122.79256861)
\curveto(626.38946866,122.81256423)(626.53946851,122.8375642)(626.68947187,122.86756861)
\curveto(626.75946829,122.88756415)(626.82446823,122.89756414)(626.88447187,122.89756861)
\curveto(626.9544681,122.89756414)(627.02946802,122.90756413)(627.10947187,122.92756861)
\curveto(627.17946787,122.94756409)(627.2494678,122.95756408)(627.31947187,122.95756861)
\curveto(627.38946766,122.96756407)(627.46446759,122.98256406)(627.54447187,123.00256861)
\curveto(627.79446726,123.06256398)(628.02946702,123.11256393)(628.24947187,123.15256861)
\curveto(628.46946658,123.20256384)(628.64446641,123.31756372)(628.77447187,123.49756861)
\curveto(628.83446622,123.57756346)(628.88446617,123.67756336)(628.92447187,123.79756861)
\curveto(628.96446609,123.92756311)(628.96446609,124.06756297)(628.92447187,124.21756861)
\curveto(628.86446619,124.45756258)(628.77446628,124.64756239)(628.65447187,124.78756861)
\curveto(628.54446651,124.92756211)(628.38446667,125.037562)(628.17447187,125.11756861)
\curveto(628.054467,125.16756187)(627.90946714,125.20256184)(627.73947187,125.22256861)
\curveto(627.57946747,125.2425618)(627.40946764,125.25256179)(627.22947187,125.25256861)
\curveto(627.049468,125.25256179)(626.87446818,125.2425618)(626.70447187,125.22256861)
\curveto(626.53446852,125.20256184)(626.38946866,125.17256187)(626.26947187,125.13256861)
\curveto(626.09946895,125.07256197)(625.93446912,124.98756205)(625.77447187,124.87756861)
\curveto(625.69446936,124.81756222)(625.61946943,124.7375623)(625.54947187,124.63756861)
\curveto(625.48946956,124.54756249)(625.43446962,124.44756259)(625.38447187,124.33756861)
\curveto(625.3544697,124.25756278)(625.32446973,124.17256287)(625.29447187,124.08256861)
\curveto(625.27446978,123.99256305)(625.22946982,123.92256312)(625.15947187,123.87256861)
\curveto(625.11946993,123.8425632)(625.04947,123.81756322)(624.94947187,123.79756861)
\curveto(624.85947019,123.78756325)(624.76447029,123.78256326)(624.66447187,123.78256861)
\curveto(624.56447049,123.78256326)(624.46447059,123.78756325)(624.36447187,123.79756861)
\curveto(624.27447078,123.81756322)(624.20947084,123.8425632)(624.16947187,123.87256861)
\curveto(624.12947092,123.90256314)(624.09947095,123.95256309)(624.07947187,124.02256861)
\curveto(624.05947099,124.09256295)(624.05947099,124.16756287)(624.07947187,124.24756861)
\curveto(624.10947094,124.37756266)(624.13947091,124.49756254)(624.16947187,124.60756861)
\curveto(624.20947084,124.72756231)(624.2544708,124.8425622)(624.30447187,124.95256861)
\curveto(624.49447056,125.30256174)(624.73447032,125.57256147)(625.02447187,125.76256861)
\curveto(625.31446974,125.96256108)(625.67446938,126.12256092)(626.10447187,126.24256861)
\curveto(626.20446885,126.26256078)(626.30446875,126.27756076)(626.40447187,126.28756861)
\curveto(626.51446854,126.29756074)(626.62446843,126.31256073)(626.73447187,126.33256861)
\curveto(626.77446828,126.3425607)(626.83946821,126.3425607)(626.92947187,126.33256861)
\curveto(627.01946803,126.33256071)(627.07446798,126.3425607)(627.09447187,126.36256861)
\curveto(627.79446726,126.37256067)(628.40446665,126.29256075)(628.92447187,126.12256861)
\curveto(629.44446561,125.95256109)(629.80946524,125.62756141)(630.01947187,125.14756861)
\curveto(630.10946494,124.94756209)(630.15946489,124.71256233)(630.16947187,124.44256861)
\curveto(630.18946486,124.18256286)(630.19946485,123.90756313)(630.19947187,123.61756861)
\lineto(630.19947187,120.30256861)
\curveto(630.19946485,120.16256688)(630.20446485,120.02756701)(630.21447187,119.89756861)
\curveto(630.22446483,119.76756727)(630.2544648,119.66256738)(630.30447187,119.58256861)
\curveto(630.3544647,119.51256753)(630.41946463,119.46256758)(630.49947187,119.43256861)
\curveto(630.58946446,119.39256765)(630.67446438,119.36256768)(630.75447187,119.34256861)
\curveto(630.83446422,119.33256771)(630.89446416,119.28756775)(630.93447187,119.20756861)
\curveto(630.9544641,119.17756786)(630.96446409,119.14756789)(630.96447187,119.11756861)
\curveto(630.96446409,119.08756795)(630.96946408,119.04756799)(630.97947187,118.99756861)
\moveto(628.83447187,120.66256861)
\curveto(628.89446616,120.80256624)(628.92446613,120.96256608)(628.92447187,121.14256861)
\curveto(628.93446612,121.33256571)(628.93946611,121.52756551)(628.93947187,121.72756861)
\curveto(628.93946611,121.8375652)(628.93446612,121.9375651)(628.92447187,122.02756861)
\curveto(628.91446614,122.11756492)(628.87446618,122.18756485)(628.80447187,122.23756861)
\curveto(628.77446628,122.25756478)(628.70446635,122.26756477)(628.59447187,122.26756861)
\curveto(628.57446648,122.24756479)(628.53946651,122.2375648)(628.48947187,122.23756861)
\curveto(628.43946661,122.2375648)(628.39446666,122.22756481)(628.35447187,122.20756861)
\curveto(628.27446678,122.18756485)(628.18446687,122.16756487)(628.08447187,122.14756861)
\lineto(627.78447187,122.08756861)
\curveto(627.7544673,122.08756495)(627.71946733,122.08256496)(627.67947187,122.07256861)
\lineto(627.57447187,122.07256861)
\curveto(627.42446763,122.03256501)(627.25946779,122.00756503)(627.07947187,121.99756861)
\curveto(626.90946814,121.99756504)(626.7494683,121.97756506)(626.59947187,121.93756861)
\curveto(626.51946853,121.91756512)(626.44446861,121.89756514)(626.37447187,121.87756861)
\curveto(626.31446874,121.86756517)(626.24446881,121.85256519)(626.16447187,121.83256861)
\curveto(626.00446905,121.78256526)(625.8544692,121.71756532)(625.71447187,121.63756861)
\curveto(625.57446948,121.56756547)(625.4544696,121.47756556)(625.35447187,121.36756861)
\curveto(625.2544698,121.25756578)(625.17946987,121.12256592)(625.12947187,120.96256861)
\curveto(625.07946997,120.81256623)(625.05946999,120.62756641)(625.06947187,120.40756861)
\curveto(625.06946998,120.30756673)(625.08446997,120.21256683)(625.11447187,120.12256861)
\curveto(625.1544699,120.042567)(625.19946985,119.96756707)(625.24947187,119.89756861)
\curveto(625.32946972,119.78756725)(625.43446962,119.69256735)(625.56447187,119.61256861)
\curveto(625.69446936,119.5425675)(625.83446922,119.48256756)(625.98447187,119.43256861)
\curveto(626.03446902,119.42256762)(626.08446897,119.41756762)(626.13447187,119.41756861)
\curveto(626.18446887,119.41756762)(626.23446882,119.41256763)(626.28447187,119.40256861)
\curveto(626.3544687,119.38256766)(626.43946861,119.36756767)(626.53947187,119.35756861)
\curveto(626.6494684,119.35756768)(626.73946831,119.36756767)(626.80947187,119.38756861)
\curveto(626.86946818,119.40756763)(626.92946812,119.41256763)(626.98947187,119.40256861)
\curveto(627.049468,119.40256764)(627.10946794,119.41256763)(627.16947187,119.43256861)
\curveto(627.2494678,119.45256759)(627.32446773,119.46756757)(627.39447187,119.47756861)
\curveto(627.47446758,119.48756755)(627.5494675,119.50756753)(627.61947187,119.53756861)
\curveto(627.90946714,119.65756738)(628.1544669,119.80256724)(628.35447187,119.97256861)
\curveto(628.56446649,120.1425669)(628.72446633,120.37256667)(628.83447187,120.66256861)
}
}
{
\newrgbcolor{curcolor}{0 0 0}
\pscustom[linestyle=none,fillstyle=solid,fillcolor=curcolor]
{
\newpath
\moveto(639.1111125,119.25256861)
\lineto(639.1111125,118.86256861)
\curveto(639.11110462,118.7425683)(639.08610465,118.6425684)(639.0361125,118.56256861)
\curveto(638.98610475,118.49256855)(638.90110483,118.45256859)(638.7811125,118.44256861)
\lineto(638.4361125,118.44256861)
\curveto(638.37610536,118.4425686)(638.31610542,118.4375686)(638.2561125,118.42756861)
\curveto(638.20610553,118.42756861)(638.16110557,118.4375686)(638.1211125,118.45756861)
\curveto(638.0311057,118.47756856)(637.97110576,118.51756852)(637.9411125,118.57756861)
\curveto(637.90110583,118.62756841)(637.87610586,118.68756835)(637.8661125,118.75756861)
\curveto(637.86610587,118.82756821)(637.85110588,118.89756814)(637.8211125,118.96756861)
\curveto(637.81110592,118.98756805)(637.79610594,119.00256804)(637.7761125,119.01256861)
\curveto(637.76610597,119.03256801)(637.75110598,119.05256799)(637.7311125,119.07256861)
\curveto(637.6311061,119.08256796)(637.55110618,119.06256798)(637.4911125,119.01256861)
\curveto(637.44110629,118.96256808)(637.38610635,118.91256813)(637.3261125,118.86256861)
\curveto(637.12610661,118.71256833)(636.92610681,118.59756844)(636.7261125,118.51756861)
\curveto(636.54610719,118.4375686)(636.3361074,118.37756866)(636.0961125,118.33756861)
\curveto(635.86610787,118.29756874)(635.62610811,118.27756876)(635.3761125,118.27756861)
\curveto(635.1361086,118.26756877)(634.89610884,118.28256876)(634.6561125,118.32256861)
\curveto(634.41610932,118.35256869)(634.20610953,118.40756863)(634.0261125,118.48756861)
\curveto(633.50611023,118.70756833)(633.08611065,119.00256804)(632.7661125,119.37256861)
\curveto(632.44611129,119.75256729)(632.19611154,120.22256682)(632.0161125,120.78256861)
\curveto(631.97611176,120.87256617)(631.94611179,120.96256608)(631.9261125,121.05256861)
\curveto(631.91611182,121.15256589)(631.89611184,121.25256579)(631.8661125,121.35256861)
\curveto(631.85611188,121.40256564)(631.85111188,121.45256559)(631.8511125,121.50256861)
\curveto(631.85111188,121.55256549)(631.84611189,121.60256544)(631.8361125,121.65256861)
\curveto(631.81611192,121.70256534)(631.80611193,121.75256529)(631.8061125,121.80256861)
\curveto(631.81611192,121.86256518)(631.81611192,121.91756512)(631.8061125,121.96756861)
\lineto(631.8061125,122.11756861)
\curveto(631.78611195,122.16756487)(631.77611196,122.23256481)(631.7761125,122.31256861)
\curveto(631.77611196,122.39256465)(631.78611195,122.45756458)(631.8061125,122.50756861)
\lineto(631.8061125,122.67256861)
\curveto(631.82611191,122.7425643)(631.8311119,122.81256423)(631.8211125,122.88256861)
\curveto(631.82111191,122.96256408)(631.8311119,123.037564)(631.8511125,123.10756861)
\curveto(631.86111187,123.15756388)(631.86611187,123.20256384)(631.8661125,123.24256861)
\curveto(631.86611187,123.28256376)(631.87111186,123.32756371)(631.8811125,123.37756861)
\curveto(631.91111182,123.47756356)(631.9361118,123.57256347)(631.9561125,123.66256861)
\curveto(631.97611176,123.76256328)(632.00111173,123.85756318)(632.0311125,123.94756861)
\curveto(632.16111157,124.32756271)(632.32611141,124.66756237)(632.5261125,124.96756861)
\curveto(632.736111,125.27756176)(632.98611075,125.53256151)(633.2761125,125.73256861)
\curveto(633.44611029,125.85256119)(633.62111011,125.95256109)(633.8011125,126.03256861)
\curveto(633.99110974,126.11256093)(634.19610954,126.18256086)(634.4161125,126.24256861)
\curveto(634.48610925,126.25256079)(634.55110918,126.26256078)(634.6111125,126.27256861)
\curveto(634.68110905,126.28256076)(634.75110898,126.29756074)(634.8211125,126.31756861)
\lineto(634.9711125,126.31756861)
\curveto(635.05110868,126.3375607)(635.16610857,126.34756069)(635.3161125,126.34756861)
\curveto(635.47610826,126.34756069)(635.59610814,126.3375607)(635.6761125,126.31756861)
\curveto(635.71610802,126.30756073)(635.77110796,126.30256074)(635.8411125,126.30256861)
\curveto(635.95110778,126.27256077)(636.06110767,126.24756079)(636.1711125,126.22756861)
\curveto(636.28110745,126.21756082)(636.38610735,126.18756085)(636.4861125,126.13756861)
\curveto(636.6361071,126.07756096)(636.77610696,126.01256103)(636.9061125,125.94256861)
\curveto(637.04610669,125.87256117)(637.17610656,125.79256125)(637.2961125,125.70256861)
\curveto(637.35610638,125.65256139)(637.41610632,125.59756144)(637.4761125,125.53756861)
\curveto(637.54610619,125.48756155)(637.6361061,125.47256157)(637.7461125,125.49256861)
\curveto(637.76610597,125.52256152)(637.78110595,125.54756149)(637.7911125,125.56756861)
\curveto(637.81110592,125.58756145)(637.82610591,125.61756142)(637.8361125,125.65756861)
\curveto(637.86610587,125.74756129)(637.87610586,125.86256118)(637.8661125,126.00256861)
\lineto(637.8661125,126.37756861)
\lineto(637.8661125,128.10256861)
\lineto(637.8661125,128.56756861)
\curveto(637.86610587,128.74755829)(637.89110584,128.87755816)(637.9411125,128.95756861)
\curveto(637.98110575,129.02755801)(638.04110569,129.07255797)(638.1211125,129.09256861)
\curveto(638.14110559,129.09255795)(638.16610557,129.09255795)(638.1961125,129.09256861)
\curveto(638.22610551,129.10255794)(638.25110548,129.10755793)(638.2711125,129.10756861)
\curveto(638.41110532,129.11755792)(638.55610518,129.11755792)(638.7061125,129.10756861)
\curveto(638.86610487,129.10755793)(638.97610476,129.06755797)(639.0361125,128.98756861)
\curveto(639.08610465,128.90755813)(639.11110462,128.80755823)(639.1111125,128.68756861)
\lineto(639.1111125,128.31256861)
\lineto(639.1111125,119.25256861)
\moveto(637.8961125,122.08756861)
\curveto(637.91610582,122.1375649)(637.92610581,122.20256484)(637.9261125,122.28256861)
\curveto(637.92610581,122.37256467)(637.91610582,122.4425646)(637.8961125,122.49256861)
\lineto(637.8961125,122.71756861)
\curveto(637.87610586,122.80756423)(637.86110587,122.89756414)(637.8511125,122.98756861)
\curveto(637.84110589,123.08756395)(637.82110591,123.17756386)(637.7911125,123.25756861)
\curveto(637.77110596,123.3375637)(637.75110598,123.41256363)(637.7311125,123.48256861)
\curveto(637.72110601,123.55256349)(637.70110603,123.62256342)(637.6711125,123.69256861)
\curveto(637.55110618,123.99256305)(637.39610634,124.25756278)(637.2061125,124.48756861)
\curveto(637.01610672,124.71756232)(636.77610696,124.89756214)(636.4861125,125.02756861)
\curveto(636.38610735,125.07756196)(636.28110745,125.11256193)(636.1711125,125.13256861)
\curveto(636.07110766,125.16256188)(635.96110777,125.18756185)(635.8411125,125.20756861)
\curveto(635.76110797,125.22756181)(635.67110806,125.2375618)(635.5711125,125.23756861)
\lineto(635.3011125,125.23756861)
\curveto(635.25110848,125.22756181)(635.20610853,125.21756182)(635.1661125,125.20756861)
\lineto(635.0311125,125.20756861)
\curveto(634.95110878,125.18756185)(634.86610887,125.16756187)(634.7761125,125.14756861)
\curveto(634.69610904,125.12756191)(634.61610912,125.10256194)(634.5361125,125.07256861)
\curveto(634.21610952,124.93256211)(633.95610978,124.72756231)(633.7561125,124.45756861)
\curveto(633.56611017,124.19756284)(633.41111032,123.89256315)(633.2911125,123.54256861)
\curveto(633.25111048,123.43256361)(633.22111051,123.31756372)(633.2011125,123.19756861)
\curveto(633.19111054,123.08756395)(633.17611056,122.97756406)(633.1561125,122.86756861)
\curveto(633.15611058,122.82756421)(633.15111058,122.78756425)(633.1411125,122.74756861)
\lineto(633.1411125,122.64256861)
\curveto(633.12111061,122.59256445)(633.11111062,122.5375645)(633.1111125,122.47756861)
\curveto(633.12111061,122.41756462)(633.12611061,122.36256468)(633.1261125,122.31256861)
\lineto(633.1261125,121.98256861)
\curveto(633.12611061,121.88256516)(633.1361106,121.78756525)(633.1561125,121.69756861)
\curveto(633.16611057,121.66756537)(633.17111056,121.61756542)(633.1711125,121.54756861)
\curveto(633.19111054,121.47756556)(633.20611053,121.40756563)(633.2161125,121.33756861)
\lineto(633.2761125,121.12756861)
\curveto(633.38611035,120.77756626)(633.5361102,120.47756656)(633.7261125,120.22756861)
\curveto(633.91610982,119.97756706)(634.15610958,119.77256727)(634.4461125,119.61256861)
\curveto(634.5361092,119.56256748)(634.62610911,119.52256752)(634.7161125,119.49256861)
\curveto(634.80610893,119.46256758)(634.90610883,119.43256761)(635.0161125,119.40256861)
\curveto(635.06610867,119.38256766)(635.11610862,119.37756766)(635.1661125,119.38756861)
\curveto(635.22610851,119.39756764)(635.28110845,119.39256765)(635.3311125,119.37256861)
\curveto(635.37110836,119.36256768)(635.41110832,119.35756768)(635.4511125,119.35756861)
\lineto(635.5861125,119.35756861)
\lineto(635.7211125,119.35756861)
\curveto(635.75110798,119.36756767)(635.80110793,119.37256767)(635.8711125,119.37256861)
\curveto(635.95110778,119.39256765)(636.0311077,119.40756763)(636.1111125,119.41756861)
\curveto(636.19110754,119.4375676)(636.26610747,119.46256758)(636.3361125,119.49256861)
\curveto(636.66610707,119.63256741)(636.9311068,119.80756723)(637.1311125,120.01756861)
\curveto(637.34110639,120.2375668)(637.51610622,120.51256653)(637.6561125,120.84256861)
\curveto(637.70610603,120.95256609)(637.74110599,121.06256598)(637.7611125,121.17256861)
\curveto(637.78110595,121.28256576)(637.80610593,121.39256565)(637.8361125,121.50256861)
\curveto(637.85610588,121.5425655)(637.86610587,121.57756546)(637.8661125,121.60756861)
\curveto(637.86610587,121.64756539)(637.87110586,121.68756535)(637.8811125,121.72756861)
\curveto(637.89110584,121.78756525)(637.89110584,121.84756519)(637.8811125,121.90756861)
\curveto(637.88110585,121.96756507)(637.88610585,122.02756501)(637.8961125,122.08756861)
}
}
{
\newrgbcolor{curcolor}{0 0 0}
\pscustom[linestyle=none,fillstyle=solid,fillcolor=curcolor]
{
\newpath
\moveto(648.1823625,122.64256861)
\curveto(648.20235444,122.58256446)(648.21235443,122.48756455)(648.2123625,122.35756861)
\curveto(648.21235443,122.2375648)(648.20735443,122.15256489)(648.1973625,122.10256861)
\lineto(648.1973625,121.95256861)
\curveto(648.18735445,121.87256517)(648.17735446,121.79756524)(648.1673625,121.72756861)
\curveto(648.16735447,121.66756537)(648.16235448,121.59756544)(648.1523625,121.51756861)
\curveto(648.13235451,121.45756558)(648.11735452,121.39756564)(648.1073625,121.33756861)
\curveto(648.10735453,121.27756576)(648.09735454,121.21756582)(648.0773625,121.15756861)
\curveto(648.0373546,121.02756601)(648.00235464,120.89756614)(647.9723625,120.76756861)
\curveto(647.9423547,120.6375664)(647.90235474,120.51756652)(647.8523625,120.40756861)
\curveto(647.642355,119.92756711)(647.36235528,119.52256752)(647.0123625,119.19256861)
\curveto(646.66235598,118.87256817)(646.23235641,118.62756841)(645.7223625,118.45756861)
\curveto(645.61235703,118.41756862)(645.49235715,118.38756865)(645.3623625,118.36756861)
\curveto(645.2423574,118.34756869)(645.11735752,118.32756871)(644.9873625,118.30756861)
\curveto(644.92735771,118.29756874)(644.86235778,118.29256875)(644.7923625,118.29256861)
\curveto(644.73235791,118.28256876)(644.67235797,118.27756876)(644.6123625,118.27756861)
\curveto(644.57235807,118.26756877)(644.51235813,118.26256878)(644.4323625,118.26256861)
\curveto(644.36235828,118.26256878)(644.31235833,118.26756877)(644.2823625,118.27756861)
\curveto(644.2423584,118.28756875)(644.20235844,118.29256875)(644.1623625,118.29256861)
\curveto(644.12235852,118.28256876)(644.08735855,118.28256876)(644.0573625,118.29256861)
\lineto(643.9673625,118.29256861)
\lineto(643.6073625,118.33756861)
\curveto(643.46735917,118.37756866)(643.33235931,118.41756862)(643.2023625,118.45756861)
\curveto(643.07235957,118.49756854)(642.94735969,118.5425685)(642.8273625,118.59256861)
\curveto(642.37736026,118.79256825)(642.00736063,119.05256799)(641.7173625,119.37256861)
\curveto(641.42736121,119.69256735)(641.18736145,120.08256696)(640.9973625,120.54256861)
\curveto(640.94736169,120.6425664)(640.90736173,120.7425663)(640.8773625,120.84256861)
\curveto(640.85736178,120.9425661)(640.8373618,121.04756599)(640.8173625,121.15756861)
\curveto(640.79736184,121.19756584)(640.78736185,121.22756581)(640.7873625,121.24756861)
\curveto(640.79736184,121.27756576)(640.79736184,121.31256573)(640.7873625,121.35256861)
\curveto(640.76736187,121.43256561)(640.75236189,121.51256553)(640.7423625,121.59256861)
\curveto(640.7423619,121.68256536)(640.73236191,121.76756527)(640.7123625,121.84756861)
\lineto(640.7123625,121.96756861)
\curveto(640.71236193,122.00756503)(640.70736193,122.05256499)(640.6973625,122.10256861)
\curveto(640.68736195,122.15256489)(640.68236196,122.2375648)(640.6823625,122.35756861)
\curveto(640.68236196,122.48756455)(640.69236195,122.58256446)(640.7123625,122.64256861)
\curveto(640.73236191,122.71256433)(640.7373619,122.78256426)(640.7273625,122.85256861)
\curveto(640.71736192,122.92256412)(640.72236192,122.99256405)(640.7423625,123.06256861)
\curveto(640.75236189,123.11256393)(640.75736188,123.15256389)(640.7573625,123.18256861)
\curveto(640.76736187,123.22256382)(640.77736186,123.26756377)(640.7873625,123.31756861)
\curveto(640.81736182,123.4375636)(640.8423618,123.55756348)(640.8623625,123.67756861)
\curveto(640.89236175,123.79756324)(640.93236171,123.91256313)(640.9823625,124.02256861)
\curveto(641.13236151,124.39256265)(641.31236133,124.72256232)(641.5223625,125.01256861)
\curveto(641.7423609,125.31256173)(642.00736063,125.56256148)(642.3173625,125.76256861)
\curveto(642.4373602,125.8425612)(642.56236008,125.90756113)(642.6923625,125.95756861)
\curveto(642.82235982,126.01756102)(642.95735968,126.07756096)(643.0973625,126.13756861)
\curveto(643.21735942,126.18756085)(643.34735929,126.21756082)(643.4873625,126.22756861)
\curveto(643.62735901,126.24756079)(643.76735887,126.27756076)(643.9073625,126.31756861)
\lineto(644.1023625,126.31756861)
\curveto(644.17235847,126.32756071)(644.2373584,126.3375607)(644.2973625,126.34756861)
\curveto(645.18735745,126.35756068)(645.92735671,126.17256087)(646.5173625,125.79256861)
\curveto(647.10735553,125.41256163)(647.53235511,124.91756212)(647.7923625,124.30756861)
\curveto(647.8423548,124.20756283)(647.88235476,124.10756293)(647.9123625,124.00756861)
\curveto(647.9423547,123.90756313)(647.97735466,123.80256324)(648.0173625,123.69256861)
\curveto(648.04735459,123.58256346)(648.07235457,123.46256358)(648.0923625,123.33256861)
\curveto(648.11235453,123.21256383)(648.1373545,123.08756395)(648.1673625,122.95756861)
\curveto(648.17735446,122.90756413)(648.17735446,122.85256419)(648.1673625,122.79256861)
\curveto(648.16735447,122.7425643)(648.17235447,122.69256435)(648.1823625,122.64256861)
\moveto(646.8473625,121.78756861)
\curveto(646.86735577,121.85756518)(646.87235577,121.9375651)(646.8623625,122.02756861)
\lineto(646.8623625,122.28256861)
\curveto(646.86235578,122.67256437)(646.82735581,123.00256404)(646.7573625,123.27256861)
\curveto(646.72735591,123.35256369)(646.70235594,123.43256361)(646.6823625,123.51256861)
\curveto(646.66235598,123.59256345)(646.637356,123.66756337)(646.6073625,123.73756861)
\curveto(646.32735631,124.38756265)(645.88235676,124.8375622)(645.2723625,125.08756861)
\curveto(645.20235744,125.11756192)(645.12735751,125.1375619)(645.0473625,125.14756861)
\lineto(644.8073625,125.20756861)
\curveto(644.72735791,125.22756181)(644.642358,125.2375618)(644.5523625,125.23756861)
\lineto(644.2823625,125.23756861)
\lineto(644.0123625,125.19256861)
\curveto(643.91235873,125.17256187)(643.81735882,125.14756189)(643.7273625,125.11756861)
\curveto(643.64735899,125.09756194)(643.56735907,125.06756197)(643.4873625,125.02756861)
\curveto(643.41735922,125.00756203)(643.35235929,124.97756206)(643.2923625,124.93756861)
\curveto(643.23235941,124.89756214)(643.17735946,124.85756218)(643.1273625,124.81756861)
\curveto(642.88735975,124.64756239)(642.69235995,124.4425626)(642.5423625,124.20256861)
\curveto(642.39236025,123.96256308)(642.26236038,123.68256336)(642.1523625,123.36256861)
\curveto(642.12236052,123.26256378)(642.10236054,123.15756388)(642.0923625,123.04756861)
\curveto(642.08236056,122.94756409)(642.06736057,122.8425642)(642.0473625,122.73256861)
\curveto(642.0373606,122.69256435)(642.03236061,122.62756441)(642.0323625,122.53756861)
\curveto(642.02236062,122.50756453)(642.01736062,122.47256457)(642.0173625,122.43256861)
\curveto(642.02736061,122.39256465)(642.03236061,122.34756469)(642.0323625,122.29756861)
\lineto(642.0323625,121.99756861)
\curveto(642.03236061,121.89756514)(642.0423606,121.80756523)(642.0623625,121.72756861)
\lineto(642.0923625,121.54756861)
\curveto(642.11236053,121.44756559)(642.12736051,121.34756569)(642.1373625,121.24756861)
\curveto(642.15736048,121.15756588)(642.18736045,121.07256597)(642.2273625,120.99256861)
\curveto(642.32736031,120.75256629)(642.4423602,120.52756651)(642.5723625,120.31756861)
\curveto(642.71235993,120.10756693)(642.88235976,119.93256711)(643.0823625,119.79256861)
\curveto(643.13235951,119.76256728)(643.17735946,119.7375673)(643.2173625,119.71756861)
\curveto(643.25735938,119.69756734)(643.30235934,119.67256737)(643.3523625,119.64256861)
\curveto(643.43235921,119.59256745)(643.51735912,119.54756749)(643.6073625,119.50756861)
\curveto(643.70735893,119.47756756)(643.81235883,119.44756759)(643.9223625,119.41756861)
\curveto(643.97235867,119.39756764)(644.01735862,119.38756765)(644.0573625,119.38756861)
\curveto(644.10735853,119.39756764)(644.15735848,119.39756764)(644.2073625,119.38756861)
\curveto(644.2373584,119.37756766)(644.29735834,119.36756767)(644.3873625,119.35756861)
\curveto(644.48735815,119.34756769)(644.56235808,119.35256769)(644.6123625,119.37256861)
\curveto(644.65235799,119.38256766)(644.69235795,119.38256766)(644.7323625,119.37256861)
\curveto(644.77235787,119.37256767)(644.81235783,119.38256766)(644.8523625,119.40256861)
\curveto(644.93235771,119.42256762)(645.01235763,119.4375676)(645.0923625,119.44756861)
\curveto(645.17235747,119.46756757)(645.24735739,119.49256755)(645.3173625,119.52256861)
\curveto(645.65735698,119.66256738)(645.93235671,119.85756718)(646.1423625,120.10756861)
\curveto(646.35235629,120.35756668)(646.52735611,120.65256639)(646.6673625,120.99256861)
\curveto(646.71735592,121.11256593)(646.74735589,121.2375658)(646.7573625,121.36756861)
\curveto(646.77735586,121.50756553)(646.80735583,121.64756539)(646.8473625,121.78756861)
}
}
{
\newrgbcolor{curcolor}{0 0 0}
\pscustom[linestyle=none,fillstyle=solid,fillcolor=curcolor]
{
\newpath
\moveto(653.31564375,126.34756861)
\curveto(653.54563896,126.34756069)(653.67563883,126.28756075)(653.70564375,126.16756861)
\curveto(653.73563877,126.05756098)(653.75063875,125.89256115)(653.75064375,125.67256861)
\lineto(653.75064375,125.38756861)
\curveto(653.75063875,125.29756174)(653.72563878,125.22256182)(653.67564375,125.16256861)
\curveto(653.61563889,125.08256196)(653.53063897,125.037562)(653.42064375,125.02756861)
\curveto(653.31063919,125.02756201)(653.2006393,125.01256203)(653.09064375,124.98256861)
\curveto(652.95063955,124.95256209)(652.81563969,124.92256212)(652.68564375,124.89256861)
\curveto(652.56563994,124.86256218)(652.45064005,124.82256222)(652.34064375,124.77256861)
\curveto(652.05064045,124.6425624)(651.81564069,124.46256258)(651.63564375,124.23256861)
\curveto(651.45564105,124.01256303)(651.3006412,123.75756328)(651.17064375,123.46756861)
\curveto(651.13064137,123.35756368)(651.1006414,123.2425638)(651.08064375,123.12256861)
\curveto(651.06064144,123.01256403)(651.03564147,122.89756414)(651.00564375,122.77756861)
\curveto(650.99564151,122.72756431)(650.99064151,122.67756436)(650.99064375,122.62756861)
\curveto(651.0006415,122.57756446)(651.0006415,122.52756451)(650.99064375,122.47756861)
\curveto(650.96064154,122.35756468)(650.94564156,122.21756482)(650.94564375,122.05756861)
\curveto(650.95564155,121.90756513)(650.96064154,121.76256528)(650.96064375,121.62256861)
\lineto(650.96064375,119.77756861)
\lineto(650.96064375,119.43256861)
\curveto(650.96064154,119.31256773)(650.95564155,119.19756784)(650.94564375,119.08756861)
\curveto(650.93564157,118.97756806)(650.93064157,118.88256816)(650.93064375,118.80256861)
\curveto(650.94064156,118.72256832)(650.92064158,118.65256839)(650.87064375,118.59256861)
\curveto(650.82064168,118.52256852)(650.74064176,118.48256856)(650.63064375,118.47256861)
\curveto(650.53064197,118.46256858)(650.42064208,118.45756858)(650.30064375,118.45756861)
\lineto(650.03064375,118.45756861)
\curveto(649.98064252,118.47756856)(649.93064257,118.49256855)(649.88064375,118.50256861)
\curveto(649.84064266,118.52256852)(649.81064269,118.54756849)(649.79064375,118.57756861)
\curveto(649.74064276,118.64756839)(649.71064279,118.73256831)(649.70064375,118.83256861)
\lineto(649.70064375,119.16256861)
\lineto(649.70064375,120.31756861)
\lineto(649.70064375,124.47256861)
\lineto(649.70064375,125.50756861)
\lineto(649.70064375,125.80756861)
\curveto(649.71064279,125.90756113)(649.74064276,125.99256105)(649.79064375,126.06256861)
\curveto(649.82064268,126.10256094)(649.87064263,126.13256091)(649.94064375,126.15256861)
\curveto(650.02064248,126.17256087)(650.1056424,126.18256086)(650.19564375,126.18256861)
\curveto(650.28564222,126.19256085)(650.37564213,126.19256085)(650.46564375,126.18256861)
\curveto(650.55564195,126.17256087)(650.62564188,126.15756088)(650.67564375,126.13756861)
\curveto(650.75564175,126.10756093)(650.8056417,126.04756099)(650.82564375,125.95756861)
\curveto(650.85564165,125.87756116)(650.87064163,125.78756125)(650.87064375,125.68756861)
\lineto(650.87064375,125.38756861)
\curveto(650.87064163,125.28756175)(650.89064161,125.19756184)(650.93064375,125.11756861)
\curveto(650.94064156,125.09756194)(650.95064155,125.08256196)(650.96064375,125.07256861)
\lineto(651.00564375,125.02756861)
\curveto(651.11564139,125.02756201)(651.2056413,125.07256197)(651.27564375,125.16256861)
\curveto(651.34564116,125.26256178)(651.4056411,125.3425617)(651.45564375,125.40256861)
\lineto(651.54564375,125.49256861)
\curveto(651.63564087,125.60256144)(651.76064074,125.71756132)(651.92064375,125.83756861)
\curveto(652.08064042,125.95756108)(652.23064027,126.04756099)(652.37064375,126.10756861)
\curveto(652.46064004,126.15756088)(652.55563995,126.19256085)(652.65564375,126.21256861)
\curveto(652.75563975,126.2425608)(652.86063964,126.27256077)(652.97064375,126.30256861)
\curveto(653.03063947,126.31256073)(653.09063941,126.31756072)(653.15064375,126.31756861)
\curveto(653.21063929,126.32756071)(653.26563924,126.3375607)(653.31564375,126.34756861)
}
}
{
\newrgbcolor{curcolor}{0.40000001 0.40000001 0.40000001}
\pscustom[linestyle=none,fillstyle=solid,fillcolor=curcolor]
{
\newpath
\moveto(545.29360762,129.15260523)
\lineto(560.29360762,129.15260523)
\lineto(560.29360762,114.15260523)
\lineto(545.29360762,114.15260523)
\closepath
}
}
{
\newrgbcolor{curcolor}{0 0 0}
\pscustom[linestyle=none,fillstyle=solid,fillcolor=curcolor]
{
\newpath
\moveto(573.23181562,96.00839625)
\curveto(573.25180608,95.9583955)(573.27680605,95.89839556)(573.30681562,95.82839625)
\curveto(573.33680599,95.7583957)(573.35680597,95.68339578)(573.36681562,95.60339625)
\curveto(573.38680594,95.53339593)(573.38680594,95.463396)(573.36681562,95.39339625)
\curveto(573.35680597,95.33339613)(573.31680601,95.28839617)(573.24681562,95.25839625)
\curveto(573.19680613,95.23839622)(573.13680619,95.22839623)(573.06681562,95.22839625)
\lineto(572.85681562,95.22839625)
\lineto(572.40681562,95.22839625)
\curveto(572.25680707,95.22839623)(572.13680719,95.25339621)(572.04681562,95.30339625)
\curveto(571.94680738,95.3633961)(571.87180746,95.46839599)(571.82181562,95.61839625)
\curveto(571.78180755,95.76839569)(571.73680759,95.90339556)(571.68681562,96.02339625)
\curveto(571.57680775,96.28339518)(571.47680785,96.55339491)(571.38681562,96.83339625)
\curveto(571.29680803,97.11339435)(571.19680813,97.38839407)(571.08681562,97.65839625)
\curveto(571.05680827,97.74839371)(571.0268083,97.83339363)(570.99681562,97.91339625)
\curveto(570.97680835,97.99339347)(570.94680838,98.06839339)(570.90681562,98.13839625)
\curveto(570.87680845,98.20839325)(570.8318085,98.26839319)(570.77181562,98.31839625)
\curveto(570.71180862,98.36839309)(570.6318087,98.40839305)(570.53181562,98.43839625)
\curveto(570.48180885,98.458393)(570.42180891,98.463393)(570.35181562,98.45339625)
\lineto(570.15681562,98.45339625)
\lineto(567.32181562,98.45339625)
\lineto(567.02181562,98.45339625)
\curveto(566.91181242,98.463393)(566.80681252,98.463393)(566.70681562,98.45339625)
\curveto(566.60681272,98.44339302)(566.51181282,98.42839303)(566.42181562,98.40839625)
\curveto(566.34181299,98.38839307)(566.28181305,98.34839311)(566.24181562,98.28839625)
\curveto(566.16181317,98.18839327)(566.10181323,98.07339339)(566.06181562,97.94339625)
\curveto(566.0318133,97.82339364)(565.99181334,97.69839376)(565.94181562,97.56839625)
\curveto(565.84181349,97.33839412)(565.74681358,97.09839436)(565.65681562,96.84839625)
\curveto(565.57681375,96.59839486)(565.48681384,96.3583951)(565.38681562,96.12839625)
\curveto(565.36681396,96.06839539)(565.34181399,95.99839546)(565.31181562,95.91839625)
\curveto(565.29181404,95.84839561)(565.26681406,95.77339569)(565.23681562,95.69339625)
\curveto(565.20681412,95.61339585)(565.17181416,95.53839592)(565.13181562,95.46839625)
\curveto(565.10181423,95.40839605)(565.06681426,95.3633961)(565.02681562,95.33339625)
\curveto(564.94681438,95.27339619)(564.83681449,95.23839622)(564.69681562,95.22839625)
\lineto(564.27681562,95.22839625)
\lineto(564.03681562,95.22839625)
\curveto(563.96681536,95.23839622)(563.90681542,95.2633962)(563.85681562,95.30339625)
\curveto(563.80681552,95.33339613)(563.77681555,95.37839608)(563.76681562,95.43839625)
\curveto(563.76681556,95.49839596)(563.77181556,95.5583959)(563.78181562,95.61839625)
\curveto(563.80181553,95.68839577)(563.82181551,95.75339571)(563.84181562,95.81339625)
\curveto(563.87181546,95.88339558)(563.89681543,95.93339553)(563.91681562,95.96339625)
\curveto(564.05681527,96.28339518)(564.18181515,96.59839486)(564.29181562,96.90839625)
\curveto(564.40181493,97.22839423)(564.52181481,97.54839391)(564.65181562,97.86839625)
\curveto(564.74181459,98.08839337)(564.8268145,98.30339316)(564.90681562,98.51339625)
\curveto(564.98681434,98.73339273)(565.07181426,98.95339251)(565.16181562,99.17339625)
\curveto(565.46181387,99.89339157)(565.74681358,100.61839084)(566.01681562,101.34839625)
\curveto(566.28681304,102.08838937)(566.57181276,102.82338864)(566.87181562,103.55339625)
\curveto(566.98181235,103.81338765)(567.08181225,104.07838738)(567.17181562,104.34839625)
\curveto(567.27181206,104.61838684)(567.37681195,104.88338658)(567.48681562,105.14339625)
\curveto(567.53681179,105.25338621)(567.58181175,105.37338609)(567.62181562,105.50339625)
\curveto(567.67181166,105.64338582)(567.74181159,105.74338572)(567.83181562,105.80339625)
\curveto(567.87181146,105.84338562)(567.93681139,105.87338559)(568.02681562,105.89339625)
\curveto(568.04681128,105.90338556)(568.06681126,105.90338556)(568.08681562,105.89339625)
\curveto(568.11681121,105.89338557)(568.14181119,105.89838556)(568.16181562,105.90839625)
\curveto(568.34181099,105.90838555)(568.55181078,105.90838555)(568.79181562,105.90839625)
\curveto(569.0318103,105.91838554)(569.20681012,105.88338558)(569.31681562,105.80339625)
\curveto(569.39680993,105.74338572)(569.45680987,105.64338582)(569.49681562,105.50339625)
\curveto(569.54680978,105.37338609)(569.59680973,105.25338621)(569.64681562,105.14339625)
\curveto(569.74680958,104.91338655)(569.83680949,104.68338678)(569.91681562,104.45339625)
\curveto(569.99680933,104.22338724)(570.08680924,103.99338747)(570.18681562,103.76339625)
\curveto(570.26680906,103.5633879)(570.34180899,103.3583881)(570.41181562,103.14839625)
\curveto(570.49180884,102.93838852)(570.57680875,102.73338873)(570.66681562,102.53339625)
\curveto(570.96680836,101.80338966)(571.25180808,101.0633904)(571.52181562,100.31339625)
\curveto(571.80180753,99.57339189)(572.09680723,98.83839262)(572.40681562,98.10839625)
\curveto(572.44680688,98.01839344)(572.47680685,97.93339353)(572.49681562,97.85339625)
\curveto(572.5268068,97.77339369)(572.55680677,97.68839377)(572.58681562,97.59839625)
\curveto(572.69680663,97.33839412)(572.80180653,97.07339439)(572.90181562,96.80339625)
\curveto(573.01180632,96.53339493)(573.12180621,96.26839519)(573.23181562,96.00839625)
\moveto(570.02181562,99.65339625)
\curveto(570.11180922,99.68339178)(570.16680916,99.73339173)(570.18681562,99.80339625)
\curveto(570.21680911,99.87339159)(570.22180911,99.94839151)(570.20181562,100.02839625)
\curveto(570.19180914,100.11839134)(570.16680916,100.20339126)(570.12681562,100.28339625)
\curveto(570.09680923,100.37339109)(570.06680926,100.44839101)(570.03681562,100.50839625)
\curveto(570.01680931,100.54839091)(570.00680932,100.58339088)(570.00681562,100.61339625)
\curveto(570.00680932,100.64339082)(569.99680933,100.67839078)(569.97681562,100.71839625)
\lineto(569.88681562,100.95839625)
\curveto(569.86680946,101.04839041)(569.83680949,101.13839032)(569.79681562,101.22839625)
\curveto(569.64680968,101.58838987)(569.51180982,101.95338951)(569.39181562,102.32339625)
\curveto(569.28181005,102.70338876)(569.15181018,103.07338839)(569.00181562,103.43339625)
\curveto(568.95181038,103.54338792)(568.90681042,103.65338781)(568.86681562,103.76339625)
\curveto(568.83681049,103.87338759)(568.79681053,103.97838748)(568.74681562,104.07839625)
\curveto(568.7268106,104.12838733)(568.70181063,104.17338729)(568.67181562,104.21339625)
\curveto(568.65181068,104.2633872)(568.60181073,104.28838717)(568.52181562,104.28839625)
\curveto(568.50181083,104.26838719)(568.48181085,104.25338721)(568.46181562,104.24339625)
\curveto(568.44181089,104.23338723)(568.42181091,104.21838724)(568.40181562,104.19839625)
\curveto(568.36181097,104.14838731)(568.331811,104.09338737)(568.31181562,104.03339625)
\curveto(568.29181104,103.98338748)(568.27181106,103.92838753)(568.25181562,103.86839625)
\curveto(568.20181113,103.7583877)(568.16181117,103.64838781)(568.13181562,103.53839625)
\curveto(568.10181123,103.42838803)(568.06181127,103.31838814)(568.01181562,103.20839625)
\curveto(567.84181149,102.81838864)(567.69181164,102.42338904)(567.56181562,102.02339625)
\curveto(567.44181189,101.62338984)(567.30181203,101.23339023)(567.14181562,100.85339625)
\lineto(567.08181562,100.70339625)
\curveto(567.07181226,100.65339081)(567.05681227,100.60339086)(567.03681562,100.55339625)
\lineto(566.94681562,100.31339625)
\curveto(566.91681241,100.23339123)(566.89181244,100.15339131)(566.87181562,100.07339625)
\curveto(566.85181248,100.02339144)(566.84181249,99.96839149)(566.84181562,99.90839625)
\curveto(566.85181248,99.84839161)(566.86681246,99.79839166)(566.88681562,99.75839625)
\curveto(566.93681239,99.67839178)(567.04181229,99.63339183)(567.20181562,99.62339625)
\lineto(567.65181562,99.62339625)
\lineto(569.25681562,99.62339625)
\curveto(569.36680996,99.62339184)(569.50180983,99.61839184)(569.66181562,99.60839625)
\curveto(569.82180951,99.60839185)(569.94180939,99.62339184)(570.02181562,99.65339625)
}
}
{
\newrgbcolor{curcolor}{0 0 0}
\pscustom[linestyle=none,fillstyle=solid,fillcolor=curcolor]
{
\newpath
\moveto(581.30837812,96.03839625)
\lineto(581.30837812,95.64839625)
\curveto(581.30837025,95.52839593)(581.28337027,95.42839603)(581.23337812,95.34839625)
\curveto(581.18337037,95.27839618)(581.09837046,95.23839622)(580.97837812,95.22839625)
\lineto(580.63337812,95.22839625)
\curveto(580.57337098,95.22839623)(580.51337104,95.22339624)(580.45337812,95.21339625)
\curveto(580.40337115,95.21339625)(580.3583712,95.22339624)(580.31837812,95.24339625)
\curveto(580.22837133,95.2633962)(580.16837139,95.30339616)(580.13837812,95.36339625)
\curveto(580.09837146,95.41339605)(580.07337148,95.47339599)(580.06337812,95.54339625)
\curveto(580.06337149,95.61339585)(580.04837151,95.68339578)(580.01837812,95.75339625)
\curveto(580.00837155,95.77339569)(579.99337156,95.78839567)(579.97337812,95.79839625)
\curveto(579.96337159,95.81839564)(579.94837161,95.83839562)(579.92837812,95.85839625)
\curveto(579.82837173,95.86839559)(579.74837181,95.84839561)(579.68837812,95.79839625)
\curveto(579.63837192,95.74839571)(579.58337197,95.69839576)(579.52337812,95.64839625)
\curveto(579.32337223,95.49839596)(579.12337243,95.38339608)(578.92337812,95.30339625)
\curveto(578.74337281,95.22339624)(578.53337302,95.1633963)(578.29337812,95.12339625)
\curveto(578.06337349,95.08339638)(577.82337373,95.0633964)(577.57337812,95.06339625)
\curveto(577.33337422,95.05339641)(577.09337446,95.06839639)(576.85337812,95.10839625)
\curveto(576.61337494,95.13839632)(576.40337515,95.19339627)(576.22337812,95.27339625)
\curveto(575.70337585,95.49339597)(575.28337627,95.78839567)(574.96337812,96.15839625)
\curveto(574.64337691,96.53839492)(574.39337716,97.00839445)(574.21337812,97.56839625)
\curveto(574.17337738,97.6583938)(574.14337741,97.74839371)(574.12337812,97.83839625)
\curveto(574.11337744,97.93839352)(574.09337746,98.03839342)(574.06337812,98.13839625)
\curveto(574.0533775,98.18839327)(574.04837751,98.23839322)(574.04837812,98.28839625)
\curveto(574.04837751,98.33839312)(574.04337751,98.38839307)(574.03337812,98.43839625)
\curveto(574.01337754,98.48839297)(574.00337755,98.53839292)(574.00337812,98.58839625)
\curveto(574.01337754,98.64839281)(574.01337754,98.70339276)(574.00337812,98.75339625)
\lineto(574.00337812,98.90339625)
\curveto(573.98337757,98.95339251)(573.97337758,99.01839244)(573.97337812,99.09839625)
\curveto(573.97337758,99.17839228)(573.98337757,99.24339222)(574.00337812,99.29339625)
\lineto(574.00337812,99.45839625)
\curveto(574.02337753,99.52839193)(574.02837753,99.59839186)(574.01837812,99.66839625)
\curveto(574.01837754,99.74839171)(574.02837753,99.82339164)(574.04837812,99.89339625)
\curveto(574.0583775,99.94339152)(574.06337749,99.98839147)(574.06337812,100.02839625)
\curveto(574.06337749,100.06839139)(574.06837749,100.11339135)(574.07837812,100.16339625)
\curveto(574.10837745,100.2633912)(574.13337742,100.3583911)(574.15337812,100.44839625)
\curveto(574.17337738,100.54839091)(574.19837736,100.64339082)(574.22837812,100.73339625)
\curveto(574.3583772,101.11339035)(574.52337703,101.45339001)(574.72337812,101.75339625)
\curveto(574.93337662,102.0633894)(575.18337637,102.31838914)(575.47337812,102.51839625)
\curveto(575.64337591,102.63838882)(575.81837574,102.73838872)(575.99837812,102.81839625)
\curveto(576.18837537,102.89838856)(576.39337516,102.96838849)(576.61337812,103.02839625)
\curveto(576.68337487,103.03838842)(576.74837481,103.04838841)(576.80837812,103.05839625)
\curveto(576.87837468,103.06838839)(576.94837461,103.08338838)(577.01837812,103.10339625)
\lineto(577.16837812,103.10339625)
\curveto(577.24837431,103.12338834)(577.36337419,103.13338833)(577.51337812,103.13339625)
\curveto(577.67337388,103.13338833)(577.79337376,103.12338834)(577.87337812,103.10339625)
\curveto(577.91337364,103.09338837)(577.96837359,103.08838837)(578.03837812,103.08839625)
\curveto(578.14837341,103.0583884)(578.2583733,103.03338843)(578.36837812,103.01339625)
\curveto(578.47837308,103.00338846)(578.58337297,102.97338849)(578.68337812,102.92339625)
\curveto(578.83337272,102.8633886)(578.97337258,102.79838866)(579.10337812,102.72839625)
\curveto(579.24337231,102.6583888)(579.37337218,102.57838888)(579.49337812,102.48839625)
\curveto(579.553372,102.43838902)(579.61337194,102.38338908)(579.67337812,102.32339625)
\curveto(579.74337181,102.27338919)(579.83337172,102.2583892)(579.94337812,102.27839625)
\curveto(579.96337159,102.30838915)(579.97837158,102.33338913)(579.98837812,102.35339625)
\curveto(580.00837155,102.37338909)(580.02337153,102.40338906)(580.03337812,102.44339625)
\curveto(580.06337149,102.53338893)(580.07337148,102.64838881)(580.06337812,102.78839625)
\lineto(580.06337812,103.16339625)
\lineto(580.06337812,104.88839625)
\lineto(580.06337812,105.35339625)
\curveto(580.06337149,105.53338593)(580.08837147,105.6633858)(580.13837812,105.74339625)
\curveto(580.17837138,105.81338565)(580.23837132,105.8583856)(580.31837812,105.87839625)
\curveto(580.33837122,105.87838558)(580.36337119,105.87838558)(580.39337812,105.87839625)
\curveto(580.42337113,105.88838557)(580.44837111,105.89338557)(580.46837812,105.89339625)
\curveto(580.60837095,105.90338556)(580.7533708,105.90338556)(580.90337812,105.89339625)
\curveto(581.06337049,105.89338557)(581.17337038,105.85338561)(581.23337812,105.77339625)
\curveto(581.28337027,105.69338577)(581.30837025,105.59338587)(581.30837812,105.47339625)
\lineto(581.30837812,105.09839625)
\lineto(581.30837812,96.03839625)
\moveto(580.09337812,98.87339625)
\curveto(580.11337144,98.92339254)(580.12337143,98.98839247)(580.12337812,99.06839625)
\curveto(580.12337143,99.1583923)(580.11337144,99.22839223)(580.09337812,99.27839625)
\lineto(580.09337812,99.50339625)
\curveto(580.07337148,99.59339187)(580.0583715,99.68339178)(580.04837812,99.77339625)
\curveto(580.03837152,99.87339159)(580.01837154,99.9633915)(579.98837812,100.04339625)
\curveto(579.96837159,100.12339134)(579.94837161,100.19839126)(579.92837812,100.26839625)
\curveto(579.91837164,100.33839112)(579.89837166,100.40839105)(579.86837812,100.47839625)
\curveto(579.74837181,100.77839068)(579.59337196,101.04339042)(579.40337812,101.27339625)
\curveto(579.21337234,101.50338996)(578.97337258,101.68338978)(578.68337812,101.81339625)
\curveto(578.58337297,101.8633896)(578.47837308,101.89838956)(578.36837812,101.91839625)
\curveto(578.26837329,101.94838951)(578.1583734,101.97338949)(578.03837812,101.99339625)
\curveto(577.9583736,102.01338945)(577.86837369,102.02338944)(577.76837812,102.02339625)
\lineto(577.49837812,102.02339625)
\curveto(577.44837411,102.01338945)(577.40337415,102.00338946)(577.36337812,101.99339625)
\lineto(577.22837812,101.99339625)
\curveto(577.14837441,101.97338949)(577.06337449,101.95338951)(576.97337812,101.93339625)
\curveto(576.89337466,101.91338955)(576.81337474,101.88838957)(576.73337812,101.85839625)
\curveto(576.41337514,101.71838974)(576.1533754,101.51338995)(575.95337812,101.24339625)
\curveto(575.76337579,100.98339048)(575.60837595,100.67839078)(575.48837812,100.32839625)
\curveto(575.44837611,100.21839124)(575.41837614,100.10339136)(575.39837812,99.98339625)
\curveto(575.38837617,99.87339159)(575.37337618,99.7633917)(575.35337812,99.65339625)
\curveto(575.3533762,99.61339185)(575.34837621,99.57339189)(575.33837812,99.53339625)
\lineto(575.33837812,99.42839625)
\curveto(575.31837624,99.37839208)(575.30837625,99.32339214)(575.30837812,99.26339625)
\curveto(575.31837624,99.20339226)(575.32337623,99.14839231)(575.32337812,99.09839625)
\lineto(575.32337812,98.76839625)
\curveto(575.32337623,98.66839279)(575.33337622,98.57339289)(575.35337812,98.48339625)
\curveto(575.36337619,98.45339301)(575.36837619,98.40339306)(575.36837812,98.33339625)
\curveto(575.38837617,98.2633932)(575.40337615,98.19339327)(575.41337812,98.12339625)
\lineto(575.47337812,97.91339625)
\curveto(575.58337597,97.5633939)(575.73337582,97.2633942)(575.92337812,97.01339625)
\curveto(576.11337544,96.7633947)(576.3533752,96.5583949)(576.64337812,96.39839625)
\curveto(576.73337482,96.34839511)(576.82337473,96.30839515)(576.91337812,96.27839625)
\curveto(577.00337455,96.24839521)(577.10337445,96.21839524)(577.21337812,96.18839625)
\curveto(577.26337429,96.16839529)(577.31337424,96.1633953)(577.36337812,96.17339625)
\curveto(577.42337413,96.18339528)(577.47837408,96.17839528)(577.52837812,96.15839625)
\curveto(577.56837399,96.14839531)(577.60837395,96.14339532)(577.64837812,96.14339625)
\lineto(577.78337812,96.14339625)
\lineto(577.91837812,96.14339625)
\curveto(577.94837361,96.15339531)(577.99837356,96.1583953)(578.06837812,96.15839625)
\curveto(578.14837341,96.17839528)(578.22837333,96.19339527)(578.30837812,96.20339625)
\curveto(578.38837317,96.22339524)(578.46337309,96.24839521)(578.53337812,96.27839625)
\curveto(578.86337269,96.41839504)(579.12837243,96.59339487)(579.32837812,96.80339625)
\curveto(579.53837202,97.02339444)(579.71337184,97.29839416)(579.85337812,97.62839625)
\curveto(579.90337165,97.73839372)(579.93837162,97.84839361)(579.95837812,97.95839625)
\curveto(579.97837158,98.06839339)(580.00337155,98.17839328)(580.03337812,98.28839625)
\curveto(580.0533715,98.32839313)(580.06337149,98.3633931)(580.06337812,98.39339625)
\curveto(580.06337149,98.43339303)(580.06837149,98.47339299)(580.07837812,98.51339625)
\curveto(580.08837147,98.57339289)(580.08837147,98.63339283)(580.07837812,98.69339625)
\curveto(580.07837148,98.75339271)(580.08337147,98.81339265)(580.09337812,98.87339625)
}
}
{
\newrgbcolor{curcolor}{0 0 0}
\pscustom[linestyle=none,fillstyle=solid,fillcolor=curcolor]
{
\newpath
\moveto(586.94462812,103.13339625)
\curveto(587.32462314,103.14338832)(587.64462282,103.10338836)(587.90462812,103.01339625)
\curveto(588.17462229,102.92338854)(588.41962204,102.79338867)(588.63962812,102.62339625)
\curveto(588.71962174,102.57338889)(588.78462168,102.50338896)(588.83462812,102.41339625)
\curveto(588.89462157,102.33338913)(588.9596215,102.2583892)(589.02962812,102.18839625)
\curveto(589.04962141,102.16838929)(589.07962138,102.14338932)(589.11962812,102.11339625)
\curveto(589.1596213,102.08338938)(589.20962125,102.07338939)(589.26962812,102.08339625)
\curveto(589.36962109,102.11338935)(589.45462101,102.17338929)(589.52462812,102.26339625)
\curveto(589.60462086,102.3633891)(589.68462078,102.43838902)(589.76462812,102.48839625)
\curveto(589.90462056,102.59838886)(590.04962041,102.69338877)(590.19962812,102.77339625)
\curveto(590.34962011,102.8633886)(590.51461995,102.93838852)(590.69462812,102.99839625)
\curveto(590.77461969,103.02838843)(590.8596196,103.04838841)(590.94962812,103.05839625)
\curveto(591.04961941,103.07838838)(591.14461932,103.09838836)(591.23462812,103.11839625)
\curveto(591.28461918,103.12838833)(591.32961913,103.13338833)(591.36962812,103.13339625)
\lineto(591.51962812,103.13339625)
\curveto(591.56961889,103.15338831)(591.63961882,103.1583883)(591.72962812,103.14839625)
\curveto(591.81961864,103.14838831)(591.88461858,103.14338832)(591.92462812,103.13339625)
\curveto(591.97461849,103.12338834)(592.04961841,103.11838834)(592.14962812,103.11839625)
\curveto(592.23961822,103.09838836)(592.32461814,103.07838838)(592.40462812,103.05839625)
\curveto(592.49461797,103.04838841)(592.57961788,103.02838843)(592.65962812,102.99839625)
\curveto(592.70961775,102.97838848)(592.75461771,102.9633885)(592.79462812,102.95339625)
\curveto(592.84461762,102.95338851)(592.89461757,102.94338852)(592.94462812,102.92339625)
\curveto(593.44461702,102.70338876)(593.78961667,102.3633891)(593.97962812,101.90339625)
\curveto(594.01961644,101.82338964)(594.04961641,101.73338973)(594.06962812,101.63339625)
\curveto(594.08961637,101.54338992)(594.10961635,101.44339002)(594.12962812,101.33339625)
\curveto(594.14961631,101.30339016)(594.15461631,101.26839019)(594.14462812,101.22839625)
\curveto(594.14461632,101.19839026)(594.14961631,101.16839029)(594.15962812,101.13839625)
\lineto(594.15962812,101.00339625)
\curveto(594.16961629,100.9633905)(594.16961629,100.91839054)(594.15962812,100.86839625)
\curveto(594.1596163,100.81839064)(594.1596163,100.76839069)(594.15962812,100.71839625)
\lineto(594.15962812,100.13339625)
\lineto(594.15962812,99.17339625)
\lineto(594.15962812,96.32339625)
\curveto(594.1596163,96.1633953)(594.1596163,95.97339549)(594.15962812,95.75339625)
\curveto(594.16961629,95.53339593)(594.12961633,95.38839607)(594.03962812,95.31839625)
\curveto(593.99961646,95.28839617)(593.93461653,95.2633962)(593.84462812,95.24339625)
\curveto(593.75461671,95.23339623)(593.6596168,95.22839623)(593.55962812,95.22839625)
\curveto(593.459617,95.22839623)(593.3596171,95.23339623)(593.25962812,95.24339625)
\curveto(593.16961729,95.25339621)(593.10461736,95.27339619)(593.06462812,95.30339625)
\curveto(593.00461746,95.33339613)(592.9646175,95.39339607)(592.94462812,95.48339625)
\curveto(592.92461754,95.54339592)(592.91961754,95.60339586)(592.92962812,95.66339625)
\curveto(592.93961752,95.73339573)(592.93461753,95.79839566)(592.91462812,95.85839625)
\curveto(592.90461756,95.90839555)(592.89961756,95.9633955)(592.89962812,96.02339625)
\curveto(592.90961755,96.09339537)(592.91461755,96.1583953)(592.91462812,96.21839625)
\lineto(592.91462812,96.89339625)
\lineto(592.91462812,99.75839625)
\curveto(592.91461755,100.08839137)(592.90461756,100.39839106)(592.88462812,100.68839625)
\curveto(592.87461759,100.98839047)(592.80461766,101.23839022)(592.67462812,101.43839625)
\curveto(592.52461794,101.67838978)(592.29461817,101.85338961)(591.98462812,101.96339625)
\curveto(591.92461854,101.98338948)(591.8596186,101.99338947)(591.78962812,101.99339625)
\curveto(591.72961873,102.00338946)(591.6646188,102.01838944)(591.59462812,102.03839625)
\curveto(591.55461891,102.04838941)(591.48961897,102.04838941)(591.39962812,102.03839625)
\curveto(591.30961915,102.03838942)(591.24961921,102.03338943)(591.21962812,102.02339625)
\curveto(591.16961929,102.01338945)(591.11961934,102.00838945)(591.06962812,102.00839625)
\curveto(591.01961944,102.01838944)(590.96961949,102.01338945)(590.91962812,101.99339625)
\curveto(590.77961968,101.9633895)(590.64461982,101.92338954)(590.51462812,101.87339625)
\curveto(589.99462047,101.65338981)(589.64462082,101.26839019)(589.46462812,100.71839625)
\curveto(589.41462105,100.54839091)(589.38462108,100.35339111)(589.37462812,100.13339625)
\lineto(589.37462812,99.45839625)
\lineto(589.37462812,97.49339625)
\lineto(589.37462812,96.03839625)
\lineto(589.37462812,95.66339625)
\curveto(589.37462109,95.54339592)(589.34962111,95.44839601)(589.29962812,95.37839625)
\curveto(589.24962121,95.29839616)(589.1646213,95.25339621)(589.04462812,95.24339625)
\curveto(588.92462154,95.23339623)(588.79962166,95.22839623)(588.66962812,95.22839625)
\curveto(588.49962196,95.22839623)(588.37462209,95.24839621)(588.29462812,95.28839625)
\curveto(588.20462226,95.33839612)(588.14962231,95.41839604)(588.12962812,95.52839625)
\curveto(588.11962234,95.64839581)(588.11462235,95.77839568)(588.11462812,95.91839625)
\lineto(588.11462812,97.34339625)
\lineto(588.11462812,99.81839625)
\curveto(588.11462235,100.13839132)(588.10462236,100.43339103)(588.08462812,100.70339625)
\curveto(588.0646224,100.98339048)(587.99462247,101.22339024)(587.87462812,101.42339625)
\curveto(587.7646227,101.60338986)(587.63962282,101.73338973)(587.49962812,101.81339625)
\curveto(587.3596231,101.90338956)(587.16962329,101.97338949)(586.92962812,102.02339625)
\curveto(586.88962357,102.03338943)(586.84462362,102.03838942)(586.79462812,102.03839625)
\lineto(586.65962812,102.03839625)
\curveto(586.43962402,102.03838942)(586.24462422,102.01338945)(586.07462812,101.96339625)
\curveto(585.91462455,101.91338955)(585.76962469,101.84838961)(585.63962812,101.76839625)
\curveto(585.12962533,101.45839)(584.78962567,100.99339047)(584.61962812,100.37339625)
\curveto(584.57962588,100.24339122)(584.5596259,100.09339137)(584.55962812,99.92339625)
\curveto(584.56962589,99.7633917)(584.57462589,99.60339186)(584.57462812,99.44339625)
\lineto(584.57462812,97.74839625)
\lineto(584.57462812,96.09839625)
\lineto(584.57462812,95.69339625)
\curveto(584.57462589,95.55339591)(584.54462592,95.44339602)(584.48462812,95.36339625)
\curveto(584.43462603,95.29339617)(584.3596261,95.25339621)(584.25962812,95.24339625)
\curveto(584.1596263,95.23339623)(584.05462641,95.22839623)(583.94462812,95.22839625)
\lineto(583.71962812,95.22839625)
\curveto(583.6596268,95.24839621)(583.59962686,95.2633962)(583.53962812,95.27339625)
\curveto(583.48962697,95.28339618)(583.44462702,95.31339615)(583.40462812,95.36339625)
\curveto(583.35462711,95.42339604)(583.32962713,95.49839596)(583.32962812,95.58839625)
\lineto(583.32962812,95.90339625)
\lineto(583.32962812,96.87839625)
\lineto(583.32962812,101.16839625)
\lineto(583.32962812,102.27839625)
\lineto(583.32962812,102.56339625)
\curveto(583.32962713,102.6633888)(583.34962711,102.74338872)(583.38962812,102.80339625)
\curveto(583.41962704,102.8633886)(583.464627,102.90338856)(583.52462812,102.92339625)
\curveto(583.60462686,102.95338851)(583.72962673,102.96838849)(583.89962812,102.96839625)
\curveto(584.07962638,102.96838849)(584.20962625,102.95338851)(584.28962812,102.92339625)
\curveto(584.36962609,102.88338858)(584.42462604,102.83338863)(584.45462812,102.77339625)
\curveto(584.47462599,102.72338874)(584.48462598,102.6633888)(584.48462812,102.59339625)
\curveto(584.49462597,102.52338894)(584.50462596,102.458389)(584.51462812,102.39839625)
\curveto(584.52462594,102.33838912)(584.54462592,102.28838917)(584.57462812,102.24839625)
\curveto(584.60462586,102.20838925)(584.65462581,102.18838927)(584.72462812,102.18839625)
\curveto(584.74462572,102.20838925)(584.7646257,102.21838924)(584.78462812,102.21839625)
\curveto(584.81462565,102.21838924)(584.83962562,102.22838923)(584.85962812,102.24839625)
\curveto(584.91962554,102.29838916)(584.97462549,102.34838911)(585.02462812,102.39839625)
\lineto(585.20462812,102.54839625)
\curveto(585.42462504,102.70838875)(585.67462479,102.84838861)(585.95462812,102.96839625)
\curveto(586.05462441,103.00838845)(586.15462431,103.03338843)(586.25462812,103.04339625)
\curveto(586.35462411,103.0633884)(586.459624,103.08838837)(586.56962812,103.11839625)
\lineto(586.74962812,103.11839625)
\curveto(586.81962364,103.12838833)(586.88462358,103.13338833)(586.94462812,103.13339625)
}
}
{
\newrgbcolor{curcolor}{0 0 0}
\pscustom[linestyle=none,fillstyle=solid,fillcolor=curcolor]
{
\newpath
\moveto(596.3423625,104.45339625)
\curveto(596.26236138,104.51338695)(596.21736142,104.61838684)(596.2073625,104.76839625)
\lineto(596.2073625,105.23339625)
\lineto(596.2073625,105.48839625)
\curveto(596.20736143,105.57838588)(596.22236142,105.65338581)(596.2523625,105.71339625)
\curveto(596.29236135,105.79338567)(596.37236127,105.85338561)(596.4923625,105.89339625)
\curveto(596.51236113,105.90338556)(596.53236111,105.90338556)(596.5523625,105.89339625)
\curveto(596.58236106,105.89338557)(596.60736103,105.89838556)(596.6273625,105.90839625)
\curveto(596.79736084,105.90838555)(596.95736068,105.90338556)(597.1073625,105.89339625)
\curveto(597.25736038,105.88338558)(597.35736028,105.82338564)(597.4073625,105.71339625)
\curveto(597.4373602,105.65338581)(597.45236019,105.57838588)(597.4523625,105.48839625)
\lineto(597.4523625,105.23339625)
\curveto(597.45236019,105.05338641)(597.44736019,104.88338658)(597.4373625,104.72339625)
\curveto(597.4373602,104.5633869)(597.37236027,104.458387)(597.2423625,104.40839625)
\curveto(597.19236045,104.38838707)(597.1373605,104.37838708)(597.0773625,104.37839625)
\lineto(596.9123625,104.37839625)
\lineto(596.5973625,104.37839625)
\curveto(596.49736114,104.37838708)(596.41236123,104.40338706)(596.3423625,104.45339625)
\moveto(597.4523625,95.94839625)
\lineto(597.4523625,95.63339625)
\curveto(597.46236018,95.53339593)(597.4423602,95.45339601)(597.3923625,95.39339625)
\curveto(597.36236028,95.33339613)(597.31736032,95.29339617)(597.2573625,95.27339625)
\curveto(597.19736044,95.2633962)(597.12736051,95.24839621)(597.0473625,95.22839625)
\lineto(596.8223625,95.22839625)
\curveto(596.69236095,95.22839623)(596.57736106,95.23339623)(596.4773625,95.24339625)
\curveto(596.38736125,95.2633962)(596.31736132,95.31339615)(596.2673625,95.39339625)
\curveto(596.22736141,95.45339601)(596.20736143,95.52839593)(596.2073625,95.61839625)
\lineto(596.2073625,95.90339625)
\lineto(596.2073625,102.24839625)
\lineto(596.2073625,102.56339625)
\curveto(596.20736143,102.67338879)(596.23236141,102.7583887)(596.2823625,102.81839625)
\curveto(596.31236133,102.86838859)(596.35236129,102.89838856)(596.4023625,102.90839625)
\curveto(596.45236119,102.91838854)(596.50736113,102.93338853)(596.5673625,102.95339625)
\curveto(596.58736105,102.95338851)(596.60736103,102.94838851)(596.6273625,102.93839625)
\curveto(596.65736098,102.93838852)(596.68236096,102.94338852)(596.7023625,102.95339625)
\curveto(596.83236081,102.95338851)(596.96236068,102.94838851)(597.0923625,102.93839625)
\curveto(597.23236041,102.93838852)(597.32736031,102.89838856)(597.3773625,102.81839625)
\curveto(597.42736021,102.7583887)(597.45236019,102.67838878)(597.4523625,102.57839625)
\lineto(597.4523625,102.29339625)
\lineto(597.4523625,95.94839625)
}
}
{
\newrgbcolor{curcolor}{0 0 0}
\pscustom[linestyle=none,fillstyle=solid,fillcolor=curcolor]
{
\newpath
\moveto(603.08720625,103.10339625)
\curveto(603.71720101,103.12338834)(604.22220051,103.03838842)(604.60220625,102.84839625)
\curveto(604.98219975,102.6583888)(605.28719944,102.37338909)(605.51720625,101.99339625)
\curveto(605.57719915,101.89338957)(605.62219911,101.78338968)(605.65220625,101.66339625)
\curveto(605.69219904,101.55338991)(605.727199,101.43839002)(605.75720625,101.31839625)
\curveto(605.80719892,101.12839033)(605.83719889,100.92339054)(605.84720625,100.70339625)
\curveto(605.85719887,100.48339098)(605.86219887,100.2583912)(605.86220625,100.02839625)
\lineto(605.86220625,98.42339625)
\lineto(605.86220625,96.08339625)
\curveto(605.86219887,95.91339555)(605.85719887,95.74339572)(605.84720625,95.57339625)
\curveto(605.84719888,95.40339606)(605.78219895,95.29339617)(605.65220625,95.24339625)
\curveto(605.60219913,95.22339624)(605.54719918,95.21339625)(605.48720625,95.21339625)
\curveto(605.43719929,95.20339626)(605.38219935,95.19839626)(605.32220625,95.19839625)
\curveto(605.19219954,95.19839626)(605.06719966,95.20339626)(604.94720625,95.21339625)
\curveto(604.8271999,95.21339625)(604.74219999,95.25339621)(604.69220625,95.33339625)
\curveto(604.64220009,95.40339606)(604.61720011,95.49339597)(604.61720625,95.60339625)
\lineto(604.61720625,95.93339625)
\lineto(604.61720625,97.22339625)
\lineto(604.61720625,99.66839625)
\curveto(604.61720011,99.93839152)(604.61220012,100.20339126)(604.60220625,100.46339625)
\curveto(604.59220014,100.73339073)(604.54720018,100.9633905)(604.46720625,101.15339625)
\curveto(604.38720034,101.35339011)(604.26720046,101.51338995)(604.10720625,101.63339625)
\curveto(603.94720078,101.7633897)(603.76220097,101.8633896)(603.55220625,101.93339625)
\curveto(603.49220124,101.95338951)(603.4272013,101.9633895)(603.35720625,101.96339625)
\curveto(603.29720143,101.97338949)(603.23720149,101.98838947)(603.17720625,102.00839625)
\curveto(603.1272016,102.01838944)(603.04720168,102.01838944)(602.93720625,102.00839625)
\curveto(602.83720189,102.00838945)(602.76720196,102.00338946)(602.72720625,101.99339625)
\curveto(602.68720204,101.97338949)(602.65220208,101.9633895)(602.62220625,101.96339625)
\curveto(602.59220214,101.97338949)(602.55720217,101.97338949)(602.51720625,101.96339625)
\curveto(602.38720234,101.93338953)(602.26220247,101.89838956)(602.14220625,101.85839625)
\curveto(602.0322027,101.82838963)(601.9272028,101.78338968)(601.82720625,101.72339625)
\curveto(601.78720294,101.70338976)(601.75220298,101.68338978)(601.72220625,101.66339625)
\curveto(601.69220304,101.64338982)(601.65720307,101.62338984)(601.61720625,101.60339625)
\curveto(601.26720346,101.35339011)(601.01220372,100.97839048)(600.85220625,100.47839625)
\curveto(600.82220391,100.39839106)(600.80220393,100.31339115)(600.79220625,100.22339625)
\curveto(600.78220395,100.14339132)(600.76720396,100.0633914)(600.74720625,99.98339625)
\curveto(600.727204,99.93339153)(600.72220401,99.88339158)(600.73220625,99.83339625)
\curveto(600.74220399,99.79339167)(600.73720399,99.75339171)(600.71720625,99.71339625)
\lineto(600.71720625,99.39839625)
\curveto(600.70720402,99.36839209)(600.70220403,99.33339213)(600.70220625,99.29339625)
\curveto(600.71220402,99.25339221)(600.71720401,99.20839225)(600.71720625,99.15839625)
\lineto(600.71720625,98.70839625)
\lineto(600.71720625,97.26839625)
\lineto(600.71720625,95.94839625)
\lineto(600.71720625,95.60339625)
\curveto(600.71720401,95.49339597)(600.69220404,95.40339606)(600.64220625,95.33339625)
\curveto(600.59220414,95.25339621)(600.50220423,95.21339625)(600.37220625,95.21339625)
\curveto(600.25220448,95.20339626)(600.1272046,95.19839626)(599.99720625,95.19839625)
\curveto(599.91720481,95.19839626)(599.84220489,95.20339626)(599.77220625,95.21339625)
\curveto(599.70220503,95.22339624)(599.64220509,95.24839621)(599.59220625,95.28839625)
\curveto(599.51220522,95.33839612)(599.47220526,95.43339603)(599.47220625,95.57339625)
\lineto(599.47220625,95.97839625)
\lineto(599.47220625,97.74839625)
\lineto(599.47220625,101.37839625)
\lineto(599.47220625,102.29339625)
\lineto(599.47220625,102.56339625)
\curveto(599.47220526,102.65338881)(599.49220524,102.72338874)(599.53220625,102.77339625)
\curveto(599.56220517,102.83338863)(599.61220512,102.87338859)(599.68220625,102.89339625)
\curveto(599.72220501,102.90338856)(599.77720495,102.91338855)(599.84720625,102.92339625)
\curveto(599.9272048,102.93338853)(600.00720472,102.93838852)(600.08720625,102.93839625)
\curveto(600.16720456,102.93838852)(600.24220449,102.93338853)(600.31220625,102.92339625)
\curveto(600.39220434,102.91338855)(600.44720428,102.89838856)(600.47720625,102.87839625)
\curveto(600.58720414,102.80838865)(600.63720409,102.71838874)(600.62720625,102.60839625)
\curveto(600.61720411,102.50838895)(600.6322041,102.39338907)(600.67220625,102.26339625)
\curveto(600.69220404,102.20338926)(600.732204,102.15338931)(600.79220625,102.11339625)
\curveto(600.91220382,102.10338936)(601.00720372,102.14838931)(601.07720625,102.24839625)
\curveto(601.15720357,102.34838911)(601.23720349,102.42838903)(601.31720625,102.48839625)
\curveto(601.45720327,102.58838887)(601.59720313,102.67838878)(601.73720625,102.75839625)
\curveto(601.88720284,102.84838861)(602.05720267,102.92338854)(602.24720625,102.98339625)
\curveto(602.3272024,103.01338845)(602.41220232,103.03338843)(602.50220625,103.04339625)
\curveto(602.60220213,103.05338841)(602.69720203,103.06838839)(602.78720625,103.08839625)
\curveto(602.83720189,103.09838836)(602.88720184,103.10338836)(602.93720625,103.10339625)
\lineto(603.08720625,103.10339625)
}
}
{
\newrgbcolor{curcolor}{0 0 0}
\pscustom[linestyle=none,fillstyle=solid,fillcolor=curcolor]
{
\newpath
\moveto(608.03181562,104.45339625)
\curveto(607.9518145,104.51338695)(607.90681455,104.61838684)(607.89681562,104.76839625)
\lineto(607.89681562,105.23339625)
\lineto(607.89681562,105.48839625)
\curveto(607.89681456,105.57838588)(607.91181454,105.65338581)(607.94181562,105.71339625)
\curveto(607.98181447,105.79338567)(608.06181439,105.85338561)(608.18181562,105.89339625)
\curveto(608.20181425,105.90338556)(608.22181423,105.90338556)(608.24181562,105.89339625)
\curveto(608.27181418,105.89338557)(608.29681416,105.89838556)(608.31681562,105.90839625)
\curveto(608.48681397,105.90838555)(608.64681381,105.90338556)(608.79681562,105.89339625)
\curveto(608.94681351,105.88338558)(609.04681341,105.82338564)(609.09681562,105.71339625)
\curveto(609.12681333,105.65338581)(609.14181331,105.57838588)(609.14181562,105.48839625)
\lineto(609.14181562,105.23339625)
\curveto(609.14181331,105.05338641)(609.13681332,104.88338658)(609.12681562,104.72339625)
\curveto(609.12681333,104.5633869)(609.06181339,104.458387)(608.93181562,104.40839625)
\curveto(608.88181357,104.38838707)(608.82681363,104.37838708)(608.76681562,104.37839625)
\lineto(608.60181562,104.37839625)
\lineto(608.28681562,104.37839625)
\curveto(608.18681427,104.37838708)(608.10181435,104.40338706)(608.03181562,104.45339625)
\moveto(609.14181562,95.94839625)
\lineto(609.14181562,95.63339625)
\curveto(609.1518133,95.53339593)(609.13181332,95.45339601)(609.08181562,95.39339625)
\curveto(609.0518134,95.33339613)(609.00681345,95.29339617)(608.94681562,95.27339625)
\curveto(608.88681357,95.2633962)(608.81681364,95.24839621)(608.73681562,95.22839625)
\lineto(608.51181562,95.22839625)
\curveto(608.38181407,95.22839623)(608.26681419,95.23339623)(608.16681562,95.24339625)
\curveto(608.07681438,95.2633962)(608.00681445,95.31339615)(607.95681562,95.39339625)
\curveto(607.91681454,95.45339601)(607.89681456,95.52839593)(607.89681562,95.61839625)
\lineto(607.89681562,95.90339625)
\lineto(607.89681562,102.24839625)
\lineto(607.89681562,102.56339625)
\curveto(607.89681456,102.67338879)(607.92181453,102.7583887)(607.97181562,102.81839625)
\curveto(608.00181445,102.86838859)(608.04181441,102.89838856)(608.09181562,102.90839625)
\curveto(608.14181431,102.91838854)(608.19681426,102.93338853)(608.25681562,102.95339625)
\curveto(608.27681418,102.95338851)(608.29681416,102.94838851)(608.31681562,102.93839625)
\curveto(608.34681411,102.93838852)(608.37181408,102.94338852)(608.39181562,102.95339625)
\curveto(608.52181393,102.95338851)(608.6518138,102.94838851)(608.78181562,102.93839625)
\curveto(608.92181353,102.93838852)(609.01681344,102.89838856)(609.06681562,102.81839625)
\curveto(609.11681334,102.7583887)(609.14181331,102.67838878)(609.14181562,102.57839625)
\lineto(609.14181562,102.29339625)
\lineto(609.14181562,95.94839625)
}
}
{
\newrgbcolor{curcolor}{0 0 0}
\pscustom[linestyle=none,fillstyle=solid,fillcolor=curcolor]
{
\newpath
\moveto(613.51665937,103.13339625)
\curveto(614.23665531,103.14338832)(614.8416547,103.0583884)(615.33165937,102.87839625)
\curveto(615.82165372,102.70838875)(616.20165334,102.40338906)(616.47165937,101.96339625)
\curveto(616.541653,101.85338961)(616.59665295,101.73838972)(616.63665937,101.61839625)
\curveto(616.67665287,101.50838995)(616.71665283,101.38339008)(616.75665937,101.24339625)
\curveto(616.77665277,101.17339029)(616.78165276,101.09839036)(616.77165937,101.01839625)
\curveto(616.76165278,100.94839051)(616.7466528,100.89339057)(616.72665937,100.85339625)
\curveto(616.70665284,100.83339063)(616.68165286,100.81339065)(616.65165937,100.79339625)
\curveto(616.62165292,100.78339068)(616.59665295,100.76839069)(616.57665937,100.74839625)
\curveto(616.52665302,100.72839073)(616.47665307,100.72339074)(616.42665937,100.73339625)
\curveto(616.37665317,100.74339072)(616.32665322,100.74339072)(616.27665937,100.73339625)
\curveto(616.19665335,100.71339075)(616.09165345,100.70839075)(615.96165937,100.71839625)
\curveto(615.83165371,100.73839072)(615.7416538,100.7633907)(615.69165937,100.79339625)
\curveto(615.61165393,100.84339062)(615.55665399,100.90839055)(615.52665937,100.98839625)
\curveto(615.50665404,101.07839038)(615.47165407,101.1633903)(615.42165937,101.24339625)
\curveto(615.33165421,101.40339006)(615.20665434,101.54838991)(615.04665937,101.67839625)
\curveto(614.93665461,101.7583897)(614.81665473,101.81838964)(614.68665937,101.85839625)
\curveto(614.55665499,101.89838956)(614.41665513,101.93838952)(614.26665937,101.97839625)
\curveto(614.21665533,101.99838946)(614.16665538,102.00338946)(614.11665937,101.99339625)
\curveto(614.06665548,101.99338947)(614.01665553,101.99838946)(613.96665937,102.00839625)
\curveto(613.90665564,102.02838943)(613.83165571,102.03838942)(613.74165937,102.03839625)
\curveto(613.65165589,102.03838942)(613.57665597,102.02838943)(613.51665937,102.00839625)
\lineto(613.42665937,102.00839625)
\lineto(613.27665937,101.97839625)
\curveto(613.22665632,101.97838948)(613.17665637,101.97338949)(613.12665937,101.96339625)
\curveto(612.86665668,101.90338956)(612.65165689,101.81838964)(612.48165937,101.70839625)
\curveto(612.31165723,101.59838986)(612.19665735,101.41339005)(612.13665937,101.15339625)
\curveto(612.11665743,101.08339038)(612.11165743,101.01339045)(612.12165937,100.94339625)
\curveto(612.1416574,100.87339059)(612.16165738,100.81339065)(612.18165937,100.76339625)
\curveto(612.2416573,100.61339085)(612.31165723,100.50339096)(612.39165937,100.43339625)
\curveto(612.48165706,100.37339109)(612.59165695,100.30339116)(612.72165937,100.22339625)
\curveto(612.88165666,100.12339134)(613.06165648,100.04839141)(613.26165937,99.99839625)
\curveto(613.46165608,99.9583915)(613.66165588,99.90839155)(613.86165937,99.84839625)
\curveto(613.99165555,99.80839165)(614.12165542,99.77839168)(614.25165937,99.75839625)
\curveto(614.38165516,99.73839172)(614.51165503,99.70839175)(614.64165937,99.66839625)
\curveto(614.85165469,99.60839185)(615.05665449,99.54839191)(615.25665937,99.48839625)
\curveto(615.45665409,99.43839202)(615.65665389,99.37339209)(615.85665937,99.29339625)
\lineto(616.00665937,99.23339625)
\curveto(616.05665349,99.21339225)(616.10665344,99.18839227)(616.15665937,99.15839625)
\curveto(616.35665319,99.03839242)(616.53165301,98.90339256)(616.68165937,98.75339625)
\curveto(616.83165271,98.60339286)(616.95665259,98.41339305)(617.05665937,98.18339625)
\curveto(617.07665247,98.11339335)(617.09665245,98.01839344)(617.11665937,97.89839625)
\curveto(617.13665241,97.82839363)(617.1466524,97.75339371)(617.14665937,97.67339625)
\curveto(617.15665239,97.60339386)(617.16165238,97.52339394)(617.16165937,97.43339625)
\lineto(617.16165937,97.28339625)
\curveto(617.1416524,97.21339425)(617.13165241,97.14339432)(617.13165937,97.07339625)
\curveto(617.13165241,97.00339446)(617.12165242,96.93339453)(617.10165937,96.86339625)
\curveto(617.07165247,96.75339471)(617.03665251,96.64839481)(616.99665937,96.54839625)
\curveto(616.95665259,96.44839501)(616.91165263,96.3583951)(616.86165937,96.27839625)
\curveto(616.70165284,96.01839544)(616.49665305,95.80839565)(616.24665937,95.64839625)
\curveto(615.99665355,95.49839596)(615.71665383,95.36839609)(615.40665937,95.25839625)
\curveto(615.31665423,95.22839623)(615.22165432,95.20839625)(615.12165937,95.19839625)
\curveto(615.03165451,95.17839628)(614.9416546,95.15339631)(614.85165937,95.12339625)
\curveto(614.75165479,95.10339636)(614.65165489,95.09339637)(614.55165937,95.09339625)
\curveto(614.45165509,95.09339637)(614.35165519,95.08339638)(614.25165937,95.06339625)
\lineto(614.10165937,95.06339625)
\curveto(614.05165549,95.05339641)(613.98165556,95.04839641)(613.89165937,95.04839625)
\curveto(613.80165574,95.04839641)(613.73165581,95.05339641)(613.68165937,95.06339625)
\lineto(613.51665937,95.06339625)
\curveto(613.45665609,95.08339638)(613.39165615,95.09339637)(613.32165937,95.09339625)
\curveto(613.25165629,95.08339638)(613.19165635,95.08839637)(613.14165937,95.10839625)
\curveto(613.09165645,95.11839634)(613.02665652,95.12339634)(612.94665937,95.12339625)
\lineto(612.70665937,95.18339625)
\curveto(612.63665691,95.19339627)(612.56165698,95.21339625)(612.48165937,95.24339625)
\curveto(612.17165737,95.34339612)(611.90165764,95.46839599)(611.67165937,95.61839625)
\curveto(611.4416581,95.76839569)(611.2416583,95.9633955)(611.07165937,96.20339625)
\curveto(610.98165856,96.33339513)(610.90665864,96.46839499)(610.84665937,96.60839625)
\curveto(610.78665876,96.74839471)(610.73165881,96.90339456)(610.68165937,97.07339625)
\curveto(610.66165888,97.13339433)(610.65165889,97.20339426)(610.65165937,97.28339625)
\curveto(610.66165888,97.37339409)(610.67665887,97.44339402)(610.69665937,97.49339625)
\curveto(610.72665882,97.53339393)(610.77665877,97.57339389)(610.84665937,97.61339625)
\curveto(610.89665865,97.63339383)(610.96665858,97.64339382)(611.05665937,97.64339625)
\curveto(611.1466584,97.65339381)(611.23665831,97.65339381)(611.32665937,97.64339625)
\curveto(611.41665813,97.63339383)(611.50165804,97.61839384)(611.58165937,97.59839625)
\curveto(611.67165787,97.58839387)(611.73165781,97.57339389)(611.76165937,97.55339625)
\curveto(611.83165771,97.50339396)(611.87665767,97.42839403)(611.89665937,97.32839625)
\curveto(611.92665762,97.23839422)(611.96165758,97.15339431)(612.00165937,97.07339625)
\curveto(612.10165744,96.85339461)(612.23665731,96.68339478)(612.40665937,96.56339625)
\curveto(612.52665702,96.47339499)(612.66165688,96.40339506)(612.81165937,96.35339625)
\curveto(612.96165658,96.30339516)(613.12165642,96.25339521)(613.29165937,96.20339625)
\lineto(613.60665937,96.15839625)
\lineto(613.69665937,96.15839625)
\curveto(613.76665578,96.13839532)(613.85665569,96.12839533)(613.96665937,96.12839625)
\curveto(614.08665546,96.12839533)(614.18665536,96.13839532)(614.26665937,96.15839625)
\curveto(614.33665521,96.1583953)(614.39165515,96.1633953)(614.43165937,96.17339625)
\curveto(614.49165505,96.18339528)(614.55165499,96.18839527)(614.61165937,96.18839625)
\curveto(614.67165487,96.19839526)(614.72665482,96.20839525)(614.77665937,96.21839625)
\curveto(615.06665448,96.29839516)(615.29665425,96.40339506)(615.46665937,96.53339625)
\curveto(615.63665391,96.6633948)(615.75665379,96.88339458)(615.82665937,97.19339625)
\curveto(615.8466537,97.24339422)(615.85165369,97.29839416)(615.84165937,97.35839625)
\curveto(615.83165371,97.41839404)(615.82165372,97.463394)(615.81165937,97.49339625)
\curveto(615.76165378,97.68339378)(615.69165385,97.82339364)(615.60165937,97.91339625)
\curveto(615.51165403,98.01339345)(615.39665415,98.10339336)(615.25665937,98.18339625)
\curveto(615.16665438,98.24339322)(615.06665448,98.29339317)(614.95665937,98.33339625)
\lineto(614.62665937,98.45339625)
\curveto(614.59665495,98.463393)(614.56665498,98.46839299)(614.53665937,98.46839625)
\curveto(614.51665503,98.46839299)(614.49165505,98.47839298)(614.46165937,98.49839625)
\curveto(614.12165542,98.60839285)(613.76665578,98.68839277)(613.39665937,98.73839625)
\curveto(613.03665651,98.79839266)(612.69665685,98.89339257)(612.37665937,99.02339625)
\curveto(612.27665727,99.0633924)(612.18165736,99.09839236)(612.09165937,99.12839625)
\curveto(612.00165754,99.1583923)(611.91665763,99.19839226)(611.83665937,99.24839625)
\curveto(611.6466579,99.3583921)(611.47165807,99.48339198)(611.31165937,99.62339625)
\curveto(611.15165839,99.7633917)(611.02665852,99.93839152)(610.93665937,100.14839625)
\curveto(610.90665864,100.21839124)(610.88165866,100.28839117)(610.86165937,100.35839625)
\curveto(610.85165869,100.42839103)(610.83665871,100.50339096)(610.81665937,100.58339625)
\curveto(610.78665876,100.70339076)(610.77665877,100.83839062)(610.78665937,100.98839625)
\curveto(610.79665875,101.14839031)(610.81165873,101.28339018)(610.83165937,101.39339625)
\curveto(610.85165869,101.44339002)(610.86165868,101.48338998)(610.86165937,101.51339625)
\curveto(610.87165867,101.55338991)(610.88665866,101.59338987)(610.90665937,101.63339625)
\curveto(610.99665855,101.8633896)(611.11665843,102.0633894)(611.26665937,102.23339625)
\curveto(611.42665812,102.40338906)(611.60665794,102.55338891)(611.80665937,102.68339625)
\curveto(611.95665759,102.77338869)(612.12165742,102.84338862)(612.30165937,102.89339625)
\curveto(612.48165706,102.95338851)(612.67165687,103.00838845)(612.87165937,103.05839625)
\curveto(612.9416566,103.06838839)(613.00665654,103.07838838)(613.06665937,103.08839625)
\curveto(613.13665641,103.09838836)(613.21165633,103.10838835)(613.29165937,103.11839625)
\curveto(613.32165622,103.12838833)(613.36165618,103.12838833)(613.41165937,103.11839625)
\curveto(613.46165608,103.10838835)(613.49665605,103.11338835)(613.51665937,103.13339625)
}
}
{
\newrgbcolor{curcolor}{0 0 0}
\pscustom[linestyle=none,fillstyle=solid,fillcolor=curcolor]
{
\newpath
\moveto(619.53165937,105.29339625)
\curveto(619.68165736,105.29338617)(619.83165721,105.28838617)(619.98165937,105.27839625)
\curveto(620.13165691,105.27838618)(620.23665681,105.23838622)(620.29665937,105.15839625)
\curveto(620.3466567,105.09838636)(620.37165667,105.01338645)(620.37165937,104.90339625)
\curveto(620.38165666,104.80338666)(620.38665666,104.69838676)(620.38665937,104.58839625)
\lineto(620.38665937,103.71839625)
\curveto(620.38665666,103.63838782)(620.38165666,103.55338791)(620.37165937,103.46339625)
\curveto(620.37165667,103.38338808)(620.38165666,103.31338815)(620.40165937,103.25339625)
\curveto(620.4416566,103.11338835)(620.53165651,103.02338844)(620.67165937,102.98339625)
\curveto(620.72165632,102.97338849)(620.76665628,102.96838849)(620.80665937,102.96839625)
\lineto(620.95665937,102.96839625)
\lineto(621.36165937,102.96839625)
\curveto(621.52165552,102.97838848)(621.63665541,102.96838849)(621.70665937,102.93839625)
\curveto(621.79665525,102.87838858)(621.85665519,102.81838864)(621.88665937,102.75839625)
\curveto(621.90665514,102.71838874)(621.91665513,102.67338879)(621.91665937,102.62339625)
\lineto(621.91665937,102.47339625)
\curveto(621.91665513,102.3633891)(621.91165513,102.2583892)(621.90165937,102.15839625)
\curveto(621.89165515,102.06838939)(621.85665519,101.99838946)(621.79665937,101.94839625)
\curveto(621.73665531,101.89838956)(621.65165539,101.86838959)(621.54165937,101.85839625)
\lineto(621.21165937,101.85839625)
\curveto(621.10165594,101.86838959)(620.99165605,101.87338959)(620.88165937,101.87339625)
\curveto(620.77165627,101.87338959)(620.67665637,101.8583896)(620.59665937,101.82839625)
\curveto(620.52665652,101.79838966)(620.47665657,101.74838971)(620.44665937,101.67839625)
\curveto(620.41665663,101.60838985)(620.39665665,101.52338994)(620.38665937,101.42339625)
\curveto(620.37665667,101.33339013)(620.37165667,101.23339023)(620.37165937,101.12339625)
\curveto(620.38165666,101.02339044)(620.38665666,100.92339054)(620.38665937,100.82339625)
\lineto(620.38665937,97.85339625)
\curveto(620.38665666,97.63339383)(620.38165666,97.39839406)(620.37165937,97.14839625)
\curveto(620.37165667,96.90839455)(620.41665663,96.72339474)(620.50665937,96.59339625)
\curveto(620.55665649,96.51339495)(620.62165642,96.458395)(620.70165937,96.42839625)
\curveto(620.78165626,96.39839506)(620.87665617,96.37339509)(620.98665937,96.35339625)
\curveto(621.01665603,96.34339512)(621.046656,96.33839512)(621.07665937,96.33839625)
\curveto(621.11665593,96.34839511)(621.15165589,96.34839511)(621.18165937,96.33839625)
\lineto(621.37665937,96.33839625)
\curveto(621.47665557,96.33839512)(621.56665548,96.32839513)(621.64665937,96.30839625)
\curveto(621.73665531,96.29839516)(621.80165524,96.2633952)(621.84165937,96.20339625)
\curveto(621.86165518,96.17339529)(621.87665517,96.11839534)(621.88665937,96.03839625)
\curveto(621.90665514,95.96839549)(621.91665513,95.89339557)(621.91665937,95.81339625)
\curveto(621.92665512,95.73339573)(621.92665512,95.65339581)(621.91665937,95.57339625)
\curveto(621.90665514,95.50339596)(621.88665516,95.44839601)(621.85665937,95.40839625)
\curveto(621.81665523,95.33839612)(621.7416553,95.28839617)(621.63165937,95.25839625)
\curveto(621.55165549,95.23839622)(621.46165558,95.22839623)(621.36165937,95.22839625)
\curveto(621.26165578,95.23839622)(621.17165587,95.24339622)(621.09165937,95.24339625)
\curveto(621.03165601,95.24339622)(620.97165607,95.23839622)(620.91165937,95.22839625)
\curveto(620.85165619,95.22839623)(620.79665625,95.23339623)(620.74665937,95.24339625)
\lineto(620.56665937,95.24339625)
\curveto(620.51665653,95.25339621)(620.46665658,95.2583962)(620.41665937,95.25839625)
\curveto(620.37665667,95.26839619)(620.33165671,95.27339619)(620.28165937,95.27339625)
\curveto(620.08165696,95.32339614)(619.90665714,95.37839608)(619.75665937,95.43839625)
\curveto(619.61665743,95.49839596)(619.49665755,95.60339586)(619.39665937,95.75339625)
\curveto(619.25665779,95.95339551)(619.17665787,96.20339526)(619.15665937,96.50339625)
\curveto(619.13665791,96.81339465)(619.12665792,97.14339432)(619.12665937,97.49339625)
\lineto(619.12665937,101.42339625)
\curveto(619.09665795,101.55338991)(619.06665798,101.64838981)(619.03665937,101.70839625)
\curveto(619.01665803,101.76838969)(618.9466581,101.81838964)(618.82665937,101.85839625)
\curveto(618.78665826,101.86838959)(618.7466583,101.86838959)(618.70665937,101.85839625)
\curveto(618.66665838,101.84838961)(618.62665842,101.85338961)(618.58665937,101.87339625)
\lineto(618.34665937,101.87339625)
\curveto(618.21665883,101.87338959)(618.10665894,101.88338958)(618.01665937,101.90339625)
\curveto(617.93665911,101.93338953)(617.88165916,101.99338947)(617.85165937,102.08339625)
\curveto(617.83165921,102.12338934)(617.81665923,102.16838929)(617.80665937,102.21839625)
\lineto(617.80665937,102.36839625)
\curveto(617.80665924,102.50838895)(617.81665923,102.62338884)(617.83665937,102.71339625)
\curveto(617.85665919,102.81338865)(617.91665913,102.88838857)(618.01665937,102.93839625)
\curveto(618.12665892,102.97838848)(618.26665878,102.98838847)(618.43665937,102.96839625)
\curveto(618.61665843,102.94838851)(618.76665828,102.9583885)(618.88665937,102.99839625)
\curveto(618.97665807,103.04838841)(619.046658,103.11838834)(619.09665937,103.20839625)
\curveto(619.11665793,103.26838819)(619.12665792,103.34338812)(619.12665937,103.43339625)
\lineto(619.12665937,103.68839625)
\lineto(619.12665937,104.61839625)
\lineto(619.12665937,104.85839625)
\curveto(619.12665792,104.94838651)(619.13665791,105.02338644)(619.15665937,105.08339625)
\curveto(619.19665785,105.1633863)(619.27165777,105.22838623)(619.38165937,105.27839625)
\curveto(619.41165763,105.27838618)(619.43665761,105.27838618)(619.45665937,105.27839625)
\curveto(619.48665756,105.28838617)(619.51165753,105.29338617)(619.53165937,105.29339625)
}
}
{
\newrgbcolor{curcolor}{0 0 0}
\pscustom[linestyle=none,fillstyle=solid,fillcolor=curcolor]
{
\newpath
\moveto(626.94845625,103.13339625)
\curveto(627.17845146,103.13338833)(627.30845133,103.07338839)(627.33845625,102.95339625)
\curveto(627.36845127,102.84338862)(627.38345125,102.67838878)(627.38345625,102.45839625)
\lineto(627.38345625,102.17339625)
\curveto(627.38345125,102.08338938)(627.35845128,102.00838945)(627.30845625,101.94839625)
\curveto(627.24845139,101.86838959)(627.16345147,101.82338964)(627.05345625,101.81339625)
\curveto(626.94345169,101.81338965)(626.8334518,101.79838966)(626.72345625,101.76839625)
\curveto(626.58345205,101.73838972)(626.44845219,101.70838975)(626.31845625,101.67839625)
\curveto(626.19845244,101.64838981)(626.08345255,101.60838985)(625.97345625,101.55839625)
\curveto(625.68345295,101.42839003)(625.44845319,101.24839021)(625.26845625,101.01839625)
\curveto(625.08845355,100.79839066)(624.9334537,100.54339092)(624.80345625,100.25339625)
\curveto(624.76345387,100.14339132)(624.7334539,100.02839143)(624.71345625,99.90839625)
\curveto(624.69345394,99.79839166)(624.66845397,99.68339178)(624.63845625,99.56339625)
\curveto(624.62845401,99.51339195)(624.62345401,99.463392)(624.62345625,99.41339625)
\curveto(624.633454,99.3633921)(624.633454,99.31339215)(624.62345625,99.26339625)
\curveto(624.59345404,99.14339232)(624.57845406,99.00339246)(624.57845625,98.84339625)
\curveto(624.58845405,98.69339277)(624.59345404,98.54839291)(624.59345625,98.40839625)
\lineto(624.59345625,96.56339625)
\lineto(624.59345625,96.21839625)
\curveto(624.59345404,96.09839536)(624.58845405,95.98339548)(624.57845625,95.87339625)
\curveto(624.56845407,95.7633957)(624.56345407,95.66839579)(624.56345625,95.58839625)
\curveto(624.57345406,95.50839595)(624.55345408,95.43839602)(624.50345625,95.37839625)
\curveto(624.45345418,95.30839615)(624.37345426,95.26839619)(624.26345625,95.25839625)
\curveto(624.16345447,95.24839621)(624.05345458,95.24339622)(623.93345625,95.24339625)
\lineto(623.66345625,95.24339625)
\curveto(623.61345502,95.2633962)(623.56345507,95.27839618)(623.51345625,95.28839625)
\curveto(623.47345516,95.30839615)(623.44345519,95.33339613)(623.42345625,95.36339625)
\curveto(623.37345526,95.43339603)(623.34345529,95.51839594)(623.33345625,95.61839625)
\lineto(623.33345625,95.94839625)
\lineto(623.33345625,97.10339625)
\lineto(623.33345625,101.25839625)
\lineto(623.33345625,102.29339625)
\lineto(623.33345625,102.59339625)
\curveto(623.34345529,102.69338877)(623.37345526,102.77838868)(623.42345625,102.84839625)
\curveto(623.45345518,102.88838857)(623.50345513,102.91838854)(623.57345625,102.93839625)
\curveto(623.65345498,102.9583885)(623.7384549,102.96838849)(623.82845625,102.96839625)
\curveto(623.91845472,102.97838848)(624.00845463,102.97838848)(624.09845625,102.96839625)
\curveto(624.18845445,102.9583885)(624.25845438,102.94338852)(624.30845625,102.92339625)
\curveto(624.38845425,102.89338857)(624.4384542,102.83338863)(624.45845625,102.74339625)
\curveto(624.48845415,102.6633888)(624.50345413,102.57338889)(624.50345625,102.47339625)
\lineto(624.50345625,102.17339625)
\curveto(624.50345413,102.07338939)(624.52345411,101.98338948)(624.56345625,101.90339625)
\curveto(624.57345406,101.88338958)(624.58345405,101.86838959)(624.59345625,101.85839625)
\lineto(624.63845625,101.81339625)
\curveto(624.74845389,101.81338965)(624.8384538,101.8583896)(624.90845625,101.94839625)
\curveto(624.97845366,102.04838941)(625.0384536,102.12838933)(625.08845625,102.18839625)
\lineto(625.17845625,102.27839625)
\curveto(625.26845337,102.38838907)(625.39345324,102.50338896)(625.55345625,102.62339625)
\curveto(625.71345292,102.74338872)(625.86345277,102.83338863)(626.00345625,102.89339625)
\curveto(626.09345254,102.94338852)(626.18845245,102.97838848)(626.28845625,102.99839625)
\curveto(626.38845225,103.02838843)(626.49345214,103.0583884)(626.60345625,103.08839625)
\curveto(626.66345197,103.09838836)(626.72345191,103.10338836)(626.78345625,103.10339625)
\curveto(626.84345179,103.11338835)(626.89845174,103.12338834)(626.94845625,103.13339625)
}
}
{
\newrgbcolor{curcolor}{0 0 0}
\pscustom[linestyle=none,fillstyle=solid,fillcolor=curcolor]
{
\newpath
\moveto(635.19822187,95.78339625)
\curveto(635.22821404,95.62339584)(635.21321406,95.48839597)(635.15322187,95.37839625)
\curveto(635.09321418,95.27839618)(635.01321426,95.20339626)(634.91322187,95.15339625)
\curveto(634.86321441,95.13339633)(634.80821446,95.12339634)(634.74822187,95.12339625)
\curveto(634.69821457,95.12339634)(634.64321463,95.11339635)(634.58322187,95.09339625)
\curveto(634.36321491,95.04339642)(634.14321513,95.0583964)(633.92322187,95.13839625)
\curveto(633.71321556,95.20839625)(633.5682157,95.29839616)(633.48822187,95.40839625)
\curveto(633.43821583,95.47839598)(633.39321588,95.5583959)(633.35322187,95.64839625)
\curveto(633.31321596,95.74839571)(633.26321601,95.82839563)(633.20322187,95.88839625)
\curveto(633.18321609,95.90839555)(633.15821611,95.92839553)(633.12822187,95.94839625)
\curveto(633.10821616,95.96839549)(633.07821619,95.97339549)(633.03822187,95.96339625)
\curveto(632.92821634,95.93339553)(632.82321645,95.87839558)(632.72322187,95.79839625)
\curveto(632.63321664,95.71839574)(632.54321673,95.64839581)(632.45322187,95.58839625)
\curveto(632.32321695,95.50839595)(632.18321709,95.43339603)(632.03322187,95.36339625)
\curveto(631.88321739,95.30339616)(631.72321755,95.24839621)(631.55322187,95.19839625)
\curveto(631.45321782,95.16839629)(631.34321793,95.14839631)(631.22322187,95.13839625)
\curveto(631.11321816,95.12839633)(631.00321827,95.11339635)(630.89322187,95.09339625)
\curveto(630.84321843,95.08339638)(630.79821847,95.07839638)(630.75822187,95.07839625)
\lineto(630.65322187,95.07839625)
\curveto(630.54321873,95.0583964)(630.43821883,95.0583964)(630.33822187,95.07839625)
\lineto(630.20322187,95.07839625)
\curveto(630.15321912,95.08839637)(630.10321917,95.09339637)(630.05322187,95.09339625)
\curveto(630.00321927,95.09339637)(629.95821931,95.10339636)(629.91822187,95.12339625)
\curveto(629.87821939,95.13339633)(629.84321943,95.13839632)(629.81322187,95.13839625)
\curveto(629.79321948,95.12839633)(629.7682195,95.12839633)(629.73822187,95.13839625)
\lineto(629.49822187,95.19839625)
\curveto(629.41821985,95.20839625)(629.34321993,95.22839623)(629.27322187,95.25839625)
\curveto(628.9732203,95.38839607)(628.72822054,95.53339593)(628.53822187,95.69339625)
\curveto(628.35822091,95.8633956)(628.20822106,96.09839536)(628.08822187,96.39839625)
\curveto(627.99822127,96.61839484)(627.95322132,96.88339458)(627.95322187,97.19339625)
\lineto(627.95322187,97.50839625)
\curveto(627.96322131,97.5583939)(627.9682213,97.60839385)(627.96822187,97.65839625)
\lineto(627.99822187,97.83839625)
\lineto(628.11822187,98.16839625)
\curveto(628.15822111,98.27839318)(628.20822106,98.37839308)(628.26822187,98.46839625)
\curveto(628.44822082,98.7583927)(628.69322058,98.97339249)(629.00322187,99.11339625)
\curveto(629.31321996,99.25339221)(629.65321962,99.37839208)(630.02322187,99.48839625)
\curveto(630.16321911,99.52839193)(630.30821896,99.5583919)(630.45822187,99.57839625)
\curveto(630.60821866,99.59839186)(630.75821851,99.62339184)(630.90822187,99.65339625)
\curveto(630.97821829,99.67339179)(631.04321823,99.68339178)(631.10322187,99.68339625)
\curveto(631.1732181,99.68339178)(631.24821802,99.69339177)(631.32822187,99.71339625)
\curveto(631.39821787,99.73339173)(631.4682178,99.74339172)(631.53822187,99.74339625)
\curveto(631.60821766,99.75339171)(631.68321759,99.76839169)(631.76322187,99.78839625)
\curveto(632.01321726,99.84839161)(632.24821702,99.89839156)(632.46822187,99.93839625)
\curveto(632.68821658,99.98839147)(632.86321641,100.10339136)(632.99322187,100.28339625)
\curveto(633.05321622,100.3633911)(633.10321617,100.463391)(633.14322187,100.58339625)
\curveto(633.18321609,100.71339075)(633.18321609,100.85339061)(633.14322187,101.00339625)
\curveto(633.08321619,101.24339022)(632.99321628,101.43339003)(632.87322187,101.57339625)
\curveto(632.76321651,101.71338975)(632.60321667,101.82338964)(632.39322187,101.90339625)
\curveto(632.273217,101.95338951)(632.12821714,101.98838947)(631.95822187,102.00839625)
\curveto(631.79821747,102.02838943)(631.62821764,102.03838942)(631.44822187,102.03839625)
\curveto(631.268218,102.03838942)(631.09321818,102.02838943)(630.92322187,102.00839625)
\curveto(630.75321852,101.98838947)(630.60821866,101.9583895)(630.48822187,101.91839625)
\curveto(630.31821895,101.8583896)(630.15321912,101.77338969)(629.99322187,101.66339625)
\curveto(629.91321936,101.60338986)(629.83821943,101.52338994)(629.76822187,101.42339625)
\curveto(629.70821956,101.33339013)(629.65321962,101.23339023)(629.60322187,101.12339625)
\curveto(629.5732197,101.04339042)(629.54321973,100.9583905)(629.51322187,100.86839625)
\curveto(629.49321978,100.77839068)(629.44821982,100.70839075)(629.37822187,100.65839625)
\curveto(629.33821993,100.62839083)(629.26822,100.60339086)(629.16822187,100.58339625)
\curveto(629.07822019,100.57339089)(628.98322029,100.56839089)(628.88322187,100.56839625)
\curveto(628.78322049,100.56839089)(628.68322059,100.57339089)(628.58322187,100.58339625)
\curveto(628.49322078,100.60339086)(628.42822084,100.62839083)(628.38822187,100.65839625)
\curveto(628.34822092,100.68839077)(628.31822095,100.73839072)(628.29822187,100.80839625)
\curveto(628.27822099,100.87839058)(628.27822099,100.95339051)(628.29822187,101.03339625)
\curveto(628.32822094,101.1633903)(628.35822091,101.28339018)(628.38822187,101.39339625)
\curveto(628.42822084,101.51338995)(628.4732208,101.62838983)(628.52322187,101.73839625)
\curveto(628.71322056,102.08838937)(628.95322032,102.3583891)(629.24322187,102.54839625)
\curveto(629.53321974,102.74838871)(629.89321938,102.90838855)(630.32322187,103.02839625)
\curveto(630.42321885,103.04838841)(630.52321875,103.0633884)(630.62322187,103.07339625)
\curveto(630.73321854,103.08338838)(630.84321843,103.09838836)(630.95322187,103.11839625)
\curveto(630.99321828,103.12838833)(631.05821821,103.12838833)(631.14822187,103.11839625)
\curveto(631.23821803,103.11838834)(631.29321798,103.12838833)(631.31322187,103.14839625)
\curveto(632.01321726,103.1583883)(632.62321665,103.07838838)(633.14322187,102.90839625)
\curveto(633.66321561,102.73838872)(634.02821524,102.41338905)(634.23822187,101.93339625)
\curveto(634.32821494,101.73338973)(634.37821489,101.49838996)(634.38822187,101.22839625)
\curveto(634.40821486,100.96839049)(634.41821485,100.69339077)(634.41822187,100.40339625)
\lineto(634.41822187,97.08839625)
\curveto(634.41821485,96.94839451)(634.42321485,96.81339465)(634.43322187,96.68339625)
\curveto(634.44321483,96.55339491)(634.4732148,96.44839501)(634.52322187,96.36839625)
\curveto(634.5732147,96.29839516)(634.63821463,96.24839521)(634.71822187,96.21839625)
\curveto(634.80821446,96.17839528)(634.89321438,96.14839531)(634.97322187,96.12839625)
\curveto(635.05321422,96.11839534)(635.11321416,96.07339539)(635.15322187,95.99339625)
\curveto(635.1732141,95.9633955)(635.18321409,95.93339553)(635.18322187,95.90339625)
\curveto(635.18321409,95.87339559)(635.18821408,95.83339563)(635.19822187,95.78339625)
\moveto(633.05322187,97.44839625)
\curveto(633.11321616,97.58839387)(633.14321613,97.74839371)(633.14322187,97.92839625)
\curveto(633.15321612,98.11839334)(633.15821611,98.31339315)(633.15822187,98.51339625)
\curveto(633.15821611,98.62339284)(633.15321612,98.72339274)(633.14322187,98.81339625)
\curveto(633.13321614,98.90339256)(633.09321618,98.97339249)(633.02322187,99.02339625)
\curveto(632.99321628,99.04339242)(632.92321635,99.05339241)(632.81322187,99.05339625)
\curveto(632.79321648,99.03339243)(632.75821651,99.02339244)(632.70822187,99.02339625)
\curveto(632.65821661,99.02339244)(632.61321666,99.01339245)(632.57322187,98.99339625)
\curveto(632.49321678,98.97339249)(632.40321687,98.95339251)(632.30322187,98.93339625)
\lineto(632.00322187,98.87339625)
\curveto(631.9732173,98.87339259)(631.93821733,98.86839259)(631.89822187,98.85839625)
\lineto(631.79322187,98.85839625)
\curveto(631.64321763,98.81839264)(631.47821779,98.79339267)(631.29822187,98.78339625)
\curveto(631.12821814,98.78339268)(630.9682183,98.7633927)(630.81822187,98.72339625)
\curveto(630.73821853,98.70339276)(630.66321861,98.68339278)(630.59322187,98.66339625)
\curveto(630.53321874,98.65339281)(630.46321881,98.63839282)(630.38322187,98.61839625)
\curveto(630.22321905,98.56839289)(630.0732192,98.50339296)(629.93322187,98.42339625)
\curveto(629.79321948,98.35339311)(629.6732196,98.2633932)(629.57322187,98.15339625)
\curveto(629.4732198,98.04339342)(629.39821987,97.90839355)(629.34822187,97.74839625)
\curveto(629.29821997,97.59839386)(629.27821999,97.41339405)(629.28822187,97.19339625)
\curveto(629.28821998,97.09339437)(629.30321997,96.99839446)(629.33322187,96.90839625)
\curveto(629.3732199,96.82839463)(629.41821985,96.75339471)(629.46822187,96.68339625)
\curveto(629.54821972,96.57339489)(629.65321962,96.47839498)(629.78322187,96.39839625)
\curveto(629.91321936,96.32839513)(630.05321922,96.26839519)(630.20322187,96.21839625)
\curveto(630.25321902,96.20839525)(630.30321897,96.20339526)(630.35322187,96.20339625)
\curveto(630.40321887,96.20339526)(630.45321882,96.19839526)(630.50322187,96.18839625)
\curveto(630.5732187,96.16839529)(630.65821861,96.15339531)(630.75822187,96.14339625)
\curveto(630.8682184,96.14339532)(630.95821831,96.15339531)(631.02822187,96.17339625)
\curveto(631.08821818,96.19339527)(631.14821812,96.19839526)(631.20822187,96.18839625)
\curveto(631.268218,96.18839527)(631.32821794,96.19839526)(631.38822187,96.21839625)
\curveto(631.4682178,96.23839522)(631.54321773,96.25339521)(631.61322187,96.26339625)
\curveto(631.69321758,96.27339519)(631.7682175,96.29339517)(631.83822187,96.32339625)
\curveto(632.12821714,96.44339502)(632.3732169,96.58839487)(632.57322187,96.75839625)
\curveto(632.78321649,96.92839453)(632.94321633,97.1583943)(633.05322187,97.44839625)
}
}
{
\newrgbcolor{curcolor}{0 0 0}
\pscustom[linestyle=none,fillstyle=solid,fillcolor=curcolor]
{
\newpath
\moveto(643.3298625,96.03839625)
\lineto(643.3298625,95.64839625)
\curveto(643.32985462,95.52839593)(643.30485465,95.42839603)(643.2548625,95.34839625)
\curveto(643.20485475,95.27839618)(643.11985483,95.23839622)(642.9998625,95.22839625)
\lineto(642.6548625,95.22839625)
\curveto(642.59485536,95.22839623)(642.53485542,95.22339624)(642.4748625,95.21339625)
\curveto(642.42485553,95.21339625)(642.37985557,95.22339624)(642.3398625,95.24339625)
\curveto(642.2498557,95.2633962)(642.18985576,95.30339616)(642.1598625,95.36339625)
\curveto(642.11985583,95.41339605)(642.09485586,95.47339599)(642.0848625,95.54339625)
\curveto(642.08485587,95.61339585)(642.06985588,95.68339578)(642.0398625,95.75339625)
\curveto(642.02985592,95.77339569)(642.01485594,95.78839567)(641.9948625,95.79839625)
\curveto(641.98485597,95.81839564)(641.96985598,95.83839562)(641.9498625,95.85839625)
\curveto(641.8498561,95.86839559)(641.76985618,95.84839561)(641.7098625,95.79839625)
\curveto(641.65985629,95.74839571)(641.60485635,95.69839576)(641.5448625,95.64839625)
\curveto(641.34485661,95.49839596)(641.14485681,95.38339608)(640.9448625,95.30339625)
\curveto(640.76485719,95.22339624)(640.5548574,95.1633963)(640.3148625,95.12339625)
\curveto(640.08485787,95.08339638)(639.84485811,95.0633964)(639.5948625,95.06339625)
\curveto(639.3548586,95.05339641)(639.11485884,95.06839639)(638.8748625,95.10839625)
\curveto(638.63485932,95.13839632)(638.42485953,95.19339627)(638.2448625,95.27339625)
\curveto(637.72486023,95.49339597)(637.30486065,95.78839567)(636.9848625,96.15839625)
\curveto(636.66486129,96.53839492)(636.41486154,97.00839445)(636.2348625,97.56839625)
\curveto(636.19486176,97.6583938)(636.16486179,97.74839371)(636.1448625,97.83839625)
\curveto(636.13486182,97.93839352)(636.11486184,98.03839342)(636.0848625,98.13839625)
\curveto(636.07486188,98.18839327)(636.06986188,98.23839322)(636.0698625,98.28839625)
\curveto(636.06986188,98.33839312)(636.06486189,98.38839307)(636.0548625,98.43839625)
\curveto(636.03486192,98.48839297)(636.02486193,98.53839292)(636.0248625,98.58839625)
\curveto(636.03486192,98.64839281)(636.03486192,98.70339276)(636.0248625,98.75339625)
\lineto(636.0248625,98.90339625)
\curveto(636.00486195,98.95339251)(635.99486196,99.01839244)(635.9948625,99.09839625)
\curveto(635.99486196,99.17839228)(636.00486195,99.24339222)(636.0248625,99.29339625)
\lineto(636.0248625,99.45839625)
\curveto(636.04486191,99.52839193)(636.0498619,99.59839186)(636.0398625,99.66839625)
\curveto(636.03986191,99.74839171)(636.0498619,99.82339164)(636.0698625,99.89339625)
\curveto(636.07986187,99.94339152)(636.08486187,99.98839147)(636.0848625,100.02839625)
\curveto(636.08486187,100.06839139)(636.08986186,100.11339135)(636.0998625,100.16339625)
\curveto(636.12986182,100.2633912)(636.1548618,100.3583911)(636.1748625,100.44839625)
\curveto(636.19486176,100.54839091)(636.21986173,100.64339082)(636.2498625,100.73339625)
\curveto(636.37986157,101.11339035)(636.54486141,101.45339001)(636.7448625,101.75339625)
\curveto(636.954861,102.0633894)(637.20486075,102.31838914)(637.4948625,102.51839625)
\curveto(637.66486029,102.63838882)(637.83986011,102.73838872)(638.0198625,102.81839625)
\curveto(638.20985974,102.89838856)(638.41485954,102.96838849)(638.6348625,103.02839625)
\curveto(638.70485925,103.03838842)(638.76985918,103.04838841)(638.8298625,103.05839625)
\curveto(638.89985905,103.06838839)(638.96985898,103.08338838)(639.0398625,103.10339625)
\lineto(639.1898625,103.10339625)
\curveto(639.26985868,103.12338834)(639.38485857,103.13338833)(639.5348625,103.13339625)
\curveto(639.69485826,103.13338833)(639.81485814,103.12338834)(639.8948625,103.10339625)
\curveto(639.93485802,103.09338837)(639.98985796,103.08838837)(640.0598625,103.08839625)
\curveto(640.16985778,103.0583884)(640.27985767,103.03338843)(640.3898625,103.01339625)
\curveto(640.49985745,103.00338846)(640.60485735,102.97338849)(640.7048625,102.92339625)
\curveto(640.8548571,102.8633886)(640.99485696,102.79838866)(641.1248625,102.72839625)
\curveto(641.26485669,102.6583888)(641.39485656,102.57838888)(641.5148625,102.48839625)
\curveto(641.57485638,102.43838902)(641.63485632,102.38338908)(641.6948625,102.32339625)
\curveto(641.76485619,102.27338919)(641.8548561,102.2583892)(641.9648625,102.27839625)
\curveto(641.98485597,102.30838915)(641.99985595,102.33338913)(642.0098625,102.35339625)
\curveto(642.02985592,102.37338909)(642.04485591,102.40338906)(642.0548625,102.44339625)
\curveto(642.08485587,102.53338893)(642.09485586,102.64838881)(642.0848625,102.78839625)
\lineto(642.0848625,103.16339625)
\lineto(642.0848625,104.88839625)
\lineto(642.0848625,105.35339625)
\curveto(642.08485587,105.53338593)(642.10985584,105.6633858)(642.1598625,105.74339625)
\curveto(642.19985575,105.81338565)(642.25985569,105.8583856)(642.3398625,105.87839625)
\curveto(642.35985559,105.87838558)(642.38485557,105.87838558)(642.4148625,105.87839625)
\curveto(642.44485551,105.88838557)(642.46985548,105.89338557)(642.4898625,105.89339625)
\curveto(642.62985532,105.90338556)(642.77485518,105.90338556)(642.9248625,105.89339625)
\curveto(643.08485487,105.89338557)(643.19485476,105.85338561)(643.2548625,105.77339625)
\curveto(643.30485465,105.69338577)(643.32985462,105.59338587)(643.3298625,105.47339625)
\lineto(643.3298625,105.09839625)
\lineto(643.3298625,96.03839625)
\moveto(642.1148625,98.87339625)
\curveto(642.13485582,98.92339254)(642.14485581,98.98839247)(642.1448625,99.06839625)
\curveto(642.14485581,99.1583923)(642.13485582,99.22839223)(642.1148625,99.27839625)
\lineto(642.1148625,99.50339625)
\curveto(642.09485586,99.59339187)(642.07985587,99.68339178)(642.0698625,99.77339625)
\curveto(642.05985589,99.87339159)(642.03985591,99.9633915)(642.0098625,100.04339625)
\curveto(641.98985596,100.12339134)(641.96985598,100.19839126)(641.9498625,100.26839625)
\curveto(641.93985601,100.33839112)(641.91985603,100.40839105)(641.8898625,100.47839625)
\curveto(641.76985618,100.77839068)(641.61485634,101.04339042)(641.4248625,101.27339625)
\curveto(641.23485672,101.50338996)(640.99485696,101.68338978)(640.7048625,101.81339625)
\curveto(640.60485735,101.8633896)(640.49985745,101.89838956)(640.3898625,101.91839625)
\curveto(640.28985766,101.94838951)(640.17985777,101.97338949)(640.0598625,101.99339625)
\curveto(639.97985797,102.01338945)(639.88985806,102.02338944)(639.7898625,102.02339625)
\lineto(639.5198625,102.02339625)
\curveto(639.46985848,102.01338945)(639.42485853,102.00338946)(639.3848625,101.99339625)
\lineto(639.2498625,101.99339625)
\curveto(639.16985878,101.97338949)(639.08485887,101.95338951)(638.9948625,101.93339625)
\curveto(638.91485904,101.91338955)(638.83485912,101.88838957)(638.7548625,101.85839625)
\curveto(638.43485952,101.71838974)(638.17485978,101.51338995)(637.9748625,101.24339625)
\curveto(637.78486017,100.98339048)(637.62986032,100.67839078)(637.5098625,100.32839625)
\curveto(637.46986048,100.21839124)(637.43986051,100.10339136)(637.4198625,99.98339625)
\curveto(637.40986054,99.87339159)(637.39486056,99.7633917)(637.3748625,99.65339625)
\curveto(637.37486058,99.61339185)(637.36986058,99.57339189)(637.3598625,99.53339625)
\lineto(637.3598625,99.42839625)
\curveto(637.33986061,99.37839208)(637.32986062,99.32339214)(637.3298625,99.26339625)
\curveto(637.33986061,99.20339226)(637.34486061,99.14839231)(637.3448625,99.09839625)
\lineto(637.3448625,98.76839625)
\curveto(637.34486061,98.66839279)(637.3548606,98.57339289)(637.3748625,98.48339625)
\curveto(637.38486057,98.45339301)(637.38986056,98.40339306)(637.3898625,98.33339625)
\curveto(637.40986054,98.2633932)(637.42486053,98.19339327)(637.4348625,98.12339625)
\lineto(637.4948625,97.91339625)
\curveto(637.60486035,97.5633939)(637.7548602,97.2633942)(637.9448625,97.01339625)
\curveto(638.13485982,96.7633947)(638.37485958,96.5583949)(638.6648625,96.39839625)
\curveto(638.7548592,96.34839511)(638.84485911,96.30839515)(638.9348625,96.27839625)
\curveto(639.02485893,96.24839521)(639.12485883,96.21839524)(639.2348625,96.18839625)
\curveto(639.28485867,96.16839529)(639.33485862,96.1633953)(639.3848625,96.17339625)
\curveto(639.44485851,96.18339528)(639.49985845,96.17839528)(639.5498625,96.15839625)
\curveto(639.58985836,96.14839531)(639.62985832,96.14339532)(639.6698625,96.14339625)
\lineto(639.8048625,96.14339625)
\lineto(639.9398625,96.14339625)
\curveto(639.96985798,96.15339531)(640.01985793,96.1583953)(640.0898625,96.15839625)
\curveto(640.16985778,96.17839528)(640.2498577,96.19339527)(640.3298625,96.20339625)
\curveto(640.40985754,96.22339524)(640.48485747,96.24839521)(640.5548625,96.27839625)
\curveto(640.88485707,96.41839504)(641.1498568,96.59339487)(641.3498625,96.80339625)
\curveto(641.55985639,97.02339444)(641.73485622,97.29839416)(641.8748625,97.62839625)
\curveto(641.92485603,97.73839372)(641.95985599,97.84839361)(641.9798625,97.95839625)
\curveto(641.99985595,98.06839339)(642.02485593,98.17839328)(642.0548625,98.28839625)
\curveto(642.07485588,98.32839313)(642.08485587,98.3633931)(642.0848625,98.39339625)
\curveto(642.08485587,98.43339303)(642.08985586,98.47339299)(642.0998625,98.51339625)
\curveto(642.10985584,98.57339289)(642.10985584,98.63339283)(642.0998625,98.69339625)
\curveto(642.09985585,98.75339271)(642.10485585,98.81339265)(642.1148625,98.87339625)
}
}
{
\newrgbcolor{curcolor}{0 0 0}
\pscustom[linestyle=none,fillstyle=solid,fillcolor=curcolor]
{
\newpath
\moveto(652.4011125,99.42839625)
\curveto(652.42110444,99.36839209)(652.43110443,99.27339219)(652.4311125,99.14339625)
\curveto(652.43110443,99.02339244)(652.42610443,98.93839252)(652.4161125,98.88839625)
\lineto(652.4161125,98.73839625)
\curveto(652.40610445,98.6583928)(652.39610446,98.58339288)(652.3861125,98.51339625)
\curveto(652.38610447,98.45339301)(652.38110448,98.38339308)(652.3711125,98.30339625)
\curveto(652.35110451,98.24339322)(652.33610452,98.18339328)(652.3261125,98.12339625)
\curveto(652.32610453,98.0633934)(652.31610454,98.00339346)(652.2961125,97.94339625)
\curveto(652.2561046,97.81339365)(652.22110464,97.68339378)(652.1911125,97.55339625)
\curveto(652.1611047,97.42339404)(652.12110474,97.30339416)(652.0711125,97.19339625)
\curveto(651.861105,96.71339475)(651.58110528,96.30839515)(651.2311125,95.97839625)
\curveto(650.88110598,95.6583958)(650.45110641,95.41339605)(649.9411125,95.24339625)
\curveto(649.83110703,95.20339626)(649.71110715,95.17339629)(649.5811125,95.15339625)
\curveto(649.4611074,95.13339633)(649.33610752,95.11339635)(649.2061125,95.09339625)
\curveto(649.14610771,95.08339638)(649.08110778,95.07839638)(649.0111125,95.07839625)
\curveto(648.95110791,95.06839639)(648.89110797,95.0633964)(648.8311125,95.06339625)
\curveto(648.79110807,95.05339641)(648.73110813,95.04839641)(648.6511125,95.04839625)
\curveto(648.58110828,95.04839641)(648.53110833,95.05339641)(648.5011125,95.06339625)
\curveto(648.4611084,95.07339639)(648.42110844,95.07839638)(648.3811125,95.07839625)
\curveto(648.34110852,95.06839639)(648.30610855,95.06839639)(648.2761125,95.07839625)
\lineto(648.1861125,95.07839625)
\lineto(647.8261125,95.12339625)
\curveto(647.68610917,95.1633963)(647.55110931,95.20339626)(647.4211125,95.24339625)
\curveto(647.29110957,95.28339618)(647.16610969,95.32839613)(647.0461125,95.37839625)
\curveto(646.59611026,95.57839588)(646.22611063,95.83839562)(645.9361125,96.15839625)
\curveto(645.64611121,96.47839498)(645.40611145,96.86839459)(645.2161125,97.32839625)
\curveto(645.16611169,97.42839403)(645.12611173,97.52839393)(645.0961125,97.62839625)
\curveto(645.07611178,97.72839373)(645.0561118,97.83339363)(645.0361125,97.94339625)
\curveto(645.01611184,97.98339348)(645.00611185,98.01339345)(645.0061125,98.03339625)
\curveto(645.01611184,98.0633934)(645.01611184,98.09839336)(645.0061125,98.13839625)
\curveto(644.98611187,98.21839324)(644.97111189,98.29839316)(644.9611125,98.37839625)
\curveto(644.9611119,98.46839299)(644.95111191,98.55339291)(644.9311125,98.63339625)
\lineto(644.9311125,98.75339625)
\curveto(644.93111193,98.79339267)(644.92611193,98.83839262)(644.9161125,98.88839625)
\curveto(644.90611195,98.93839252)(644.90111196,99.02339244)(644.9011125,99.14339625)
\curveto(644.90111196,99.27339219)(644.91111195,99.36839209)(644.9311125,99.42839625)
\curveto(644.95111191,99.49839196)(644.9561119,99.56839189)(644.9461125,99.63839625)
\curveto(644.93611192,99.70839175)(644.94111192,99.77839168)(644.9611125,99.84839625)
\curveto(644.97111189,99.89839156)(644.97611188,99.93839152)(644.9761125,99.96839625)
\curveto(644.98611187,100.00839145)(644.99611186,100.05339141)(645.0061125,100.10339625)
\curveto(645.03611182,100.22339124)(645.0611118,100.34339112)(645.0811125,100.46339625)
\curveto(645.11111175,100.58339088)(645.15111171,100.69839076)(645.2011125,100.80839625)
\curveto(645.35111151,101.17839028)(645.53111133,101.50838995)(645.7411125,101.79839625)
\curveto(645.9611109,102.09838936)(646.22611063,102.34838911)(646.5361125,102.54839625)
\curveto(646.6561102,102.62838883)(646.78111008,102.69338877)(646.9111125,102.74339625)
\curveto(647.04110982,102.80338866)(647.17610968,102.8633886)(647.3161125,102.92339625)
\curveto(647.43610942,102.97338849)(647.56610929,103.00338846)(647.7061125,103.01339625)
\curveto(647.84610901,103.03338843)(647.98610887,103.0633884)(648.1261125,103.10339625)
\lineto(648.3211125,103.10339625)
\curveto(648.39110847,103.11338835)(648.4561084,103.12338834)(648.5161125,103.13339625)
\curveto(649.40610745,103.14338832)(650.14610671,102.9583885)(650.7361125,102.57839625)
\curveto(651.32610553,102.19838926)(651.75110511,101.70338976)(652.0111125,101.09339625)
\curveto(652.0611048,100.99339047)(652.10110476,100.89339057)(652.1311125,100.79339625)
\curveto(652.1611047,100.69339077)(652.19610466,100.58839087)(652.2361125,100.47839625)
\curveto(652.26610459,100.36839109)(652.29110457,100.24839121)(652.3111125,100.11839625)
\curveto(652.33110453,99.99839146)(652.3561045,99.87339159)(652.3861125,99.74339625)
\curveto(652.39610446,99.69339177)(652.39610446,99.63839182)(652.3861125,99.57839625)
\curveto(652.38610447,99.52839193)(652.39110447,99.47839198)(652.4011125,99.42839625)
\moveto(651.0661125,98.57339625)
\curveto(651.08610577,98.64339282)(651.09110577,98.72339274)(651.0811125,98.81339625)
\lineto(651.0811125,99.06839625)
\curveto(651.08110578,99.458392)(651.04610581,99.78839167)(650.9761125,100.05839625)
\curveto(650.94610591,100.13839132)(650.92110594,100.21839124)(650.9011125,100.29839625)
\curveto(650.88110598,100.37839108)(650.856106,100.45339101)(650.8261125,100.52339625)
\curveto(650.54610631,101.17339029)(650.10110676,101.62338984)(649.4911125,101.87339625)
\curveto(649.42110744,101.90338956)(649.34610751,101.92338954)(649.2661125,101.93339625)
\lineto(649.0261125,101.99339625)
\curveto(648.94610791,102.01338945)(648.861108,102.02338944)(648.7711125,102.02339625)
\lineto(648.5011125,102.02339625)
\lineto(648.2311125,101.97839625)
\curveto(648.13110873,101.9583895)(648.03610882,101.93338953)(647.9461125,101.90339625)
\curveto(647.86610899,101.88338958)(647.78610907,101.85338961)(647.7061125,101.81339625)
\curveto(647.63610922,101.79338967)(647.57110929,101.7633897)(647.5111125,101.72339625)
\curveto(647.45110941,101.68338978)(647.39610946,101.64338982)(647.3461125,101.60339625)
\curveto(647.10610975,101.43339003)(646.91110995,101.22839023)(646.7611125,100.98839625)
\curveto(646.61111025,100.74839071)(646.48111038,100.46839099)(646.3711125,100.14839625)
\curveto(646.34111052,100.04839141)(646.32111054,99.94339152)(646.3111125,99.83339625)
\curveto(646.30111056,99.73339173)(646.28611057,99.62839183)(646.2661125,99.51839625)
\curveto(646.2561106,99.47839198)(646.25111061,99.41339205)(646.2511125,99.32339625)
\curveto(646.24111062,99.29339217)(646.23611062,99.2583922)(646.2361125,99.21839625)
\curveto(646.24611061,99.17839228)(646.25111061,99.13339233)(646.2511125,99.08339625)
\lineto(646.2511125,98.78339625)
\curveto(646.25111061,98.68339278)(646.2611106,98.59339287)(646.2811125,98.51339625)
\lineto(646.3111125,98.33339625)
\curveto(646.33111053,98.23339323)(646.34611051,98.13339333)(646.3561125,98.03339625)
\curveto(646.37611048,97.94339352)(646.40611045,97.8583936)(646.4461125,97.77839625)
\curveto(646.54611031,97.53839392)(646.6611102,97.31339415)(646.7911125,97.10339625)
\curveto(646.93110993,96.89339457)(647.10110976,96.71839474)(647.3011125,96.57839625)
\curveto(647.35110951,96.54839491)(647.39610946,96.52339494)(647.4361125,96.50339625)
\curveto(647.47610938,96.48339498)(647.52110934,96.458395)(647.5711125,96.42839625)
\curveto(647.65110921,96.37839508)(647.73610912,96.33339513)(647.8261125,96.29339625)
\curveto(647.92610893,96.2633952)(648.03110883,96.23339523)(648.1411125,96.20339625)
\curveto(648.19110867,96.18339528)(648.23610862,96.17339529)(648.2761125,96.17339625)
\curveto(648.32610853,96.18339528)(648.37610848,96.18339528)(648.4261125,96.17339625)
\curveto(648.4561084,96.1633953)(648.51610834,96.15339531)(648.6061125,96.14339625)
\curveto(648.70610815,96.13339533)(648.78110808,96.13839532)(648.8311125,96.15839625)
\curveto(648.87110799,96.16839529)(648.91110795,96.16839529)(648.9511125,96.15839625)
\curveto(648.99110787,96.1583953)(649.03110783,96.16839529)(649.0711125,96.18839625)
\curveto(649.15110771,96.20839525)(649.23110763,96.22339524)(649.3111125,96.23339625)
\curveto(649.39110747,96.25339521)(649.46610739,96.27839518)(649.5361125,96.30839625)
\curveto(649.87610698,96.44839501)(650.15110671,96.64339482)(650.3611125,96.89339625)
\curveto(650.57110629,97.14339432)(650.74610611,97.43839402)(650.8861125,97.77839625)
\curveto(650.93610592,97.89839356)(650.96610589,98.02339344)(650.9761125,98.15339625)
\curveto(650.99610586,98.29339317)(651.02610583,98.43339303)(651.0661125,98.57339625)
}
}
{
\newrgbcolor{curcolor}{0 0 0}
\pscustom[linestyle=none,fillstyle=solid,fillcolor=curcolor]
{
\newpath
\moveto(657.53439375,103.13339625)
\curveto(657.76438896,103.13338833)(657.89438883,103.07338839)(657.92439375,102.95339625)
\curveto(657.95438877,102.84338862)(657.96938875,102.67838878)(657.96939375,102.45839625)
\lineto(657.96939375,102.17339625)
\curveto(657.96938875,102.08338938)(657.94438878,102.00838945)(657.89439375,101.94839625)
\curveto(657.83438889,101.86838959)(657.74938897,101.82338964)(657.63939375,101.81339625)
\curveto(657.52938919,101.81338965)(657.4193893,101.79838966)(657.30939375,101.76839625)
\curveto(657.16938955,101.73838972)(657.03438969,101.70838975)(656.90439375,101.67839625)
\curveto(656.78438994,101.64838981)(656.66939005,101.60838985)(656.55939375,101.55839625)
\curveto(656.26939045,101.42839003)(656.03439069,101.24839021)(655.85439375,101.01839625)
\curveto(655.67439105,100.79839066)(655.5193912,100.54339092)(655.38939375,100.25339625)
\curveto(655.34939137,100.14339132)(655.3193914,100.02839143)(655.29939375,99.90839625)
\curveto(655.27939144,99.79839166)(655.25439147,99.68339178)(655.22439375,99.56339625)
\curveto(655.21439151,99.51339195)(655.20939151,99.463392)(655.20939375,99.41339625)
\curveto(655.2193915,99.3633921)(655.2193915,99.31339215)(655.20939375,99.26339625)
\curveto(655.17939154,99.14339232)(655.16439156,99.00339246)(655.16439375,98.84339625)
\curveto(655.17439155,98.69339277)(655.17939154,98.54839291)(655.17939375,98.40839625)
\lineto(655.17939375,96.56339625)
\lineto(655.17939375,96.21839625)
\curveto(655.17939154,96.09839536)(655.17439155,95.98339548)(655.16439375,95.87339625)
\curveto(655.15439157,95.7633957)(655.14939157,95.66839579)(655.14939375,95.58839625)
\curveto(655.15939156,95.50839595)(655.13939158,95.43839602)(655.08939375,95.37839625)
\curveto(655.03939168,95.30839615)(654.95939176,95.26839619)(654.84939375,95.25839625)
\curveto(654.74939197,95.24839621)(654.63939208,95.24339622)(654.51939375,95.24339625)
\lineto(654.24939375,95.24339625)
\curveto(654.19939252,95.2633962)(654.14939257,95.27839618)(654.09939375,95.28839625)
\curveto(654.05939266,95.30839615)(654.02939269,95.33339613)(654.00939375,95.36339625)
\curveto(653.95939276,95.43339603)(653.92939279,95.51839594)(653.91939375,95.61839625)
\lineto(653.91939375,95.94839625)
\lineto(653.91939375,97.10339625)
\lineto(653.91939375,101.25839625)
\lineto(653.91939375,102.29339625)
\lineto(653.91939375,102.59339625)
\curveto(653.92939279,102.69338877)(653.95939276,102.77838868)(654.00939375,102.84839625)
\curveto(654.03939268,102.88838857)(654.08939263,102.91838854)(654.15939375,102.93839625)
\curveto(654.23939248,102.9583885)(654.3243924,102.96838849)(654.41439375,102.96839625)
\curveto(654.50439222,102.97838848)(654.59439213,102.97838848)(654.68439375,102.96839625)
\curveto(654.77439195,102.9583885)(654.84439188,102.94338852)(654.89439375,102.92339625)
\curveto(654.97439175,102.89338857)(655.0243917,102.83338863)(655.04439375,102.74339625)
\curveto(655.07439165,102.6633888)(655.08939163,102.57338889)(655.08939375,102.47339625)
\lineto(655.08939375,102.17339625)
\curveto(655.08939163,102.07338939)(655.10939161,101.98338948)(655.14939375,101.90339625)
\curveto(655.15939156,101.88338958)(655.16939155,101.86838959)(655.17939375,101.85839625)
\lineto(655.22439375,101.81339625)
\curveto(655.33439139,101.81338965)(655.4243913,101.8583896)(655.49439375,101.94839625)
\curveto(655.56439116,102.04838941)(655.6243911,102.12838933)(655.67439375,102.18839625)
\lineto(655.76439375,102.27839625)
\curveto(655.85439087,102.38838907)(655.97939074,102.50338896)(656.13939375,102.62339625)
\curveto(656.29939042,102.74338872)(656.44939027,102.83338863)(656.58939375,102.89339625)
\curveto(656.67939004,102.94338852)(656.77438995,102.97838848)(656.87439375,102.99839625)
\curveto(656.97438975,103.02838843)(657.07938964,103.0583884)(657.18939375,103.08839625)
\curveto(657.24938947,103.09838836)(657.30938941,103.10338836)(657.36939375,103.10339625)
\curveto(657.42938929,103.11338835)(657.48438924,103.12338834)(657.53439375,103.13339625)
}
}
{
\newrgbcolor{curcolor}{0.3019608 0.3019608 0.3019608}
\pscustom[linestyle=none,fillstyle=solid,fillcolor=curcolor]
{
\newpath
\moveto(545.29360762,105.93837183)
\lineto(560.29360762,105.93837183)
\lineto(560.29360762,90.93837183)
\lineto(545.29360762,90.93837183)
\closepath
}
}
{
\newrgbcolor{curcolor}{0.80000001 0.80000001 0.80000001}
\pscustom[linestyle=none,fillstyle=solid,fillcolor=curcolor]
{
\newpath
\moveto(655.60333808,999.6697026)
\curveto(729.0344683,965.07248908)(760.51557789,877.49821339)(725.91836437,804.06708317)
\curveto(691.32115086,730.63595295)(603.74687516,699.15484336)(530.31574494,733.75205688)
\curveto(456.88461472,768.34927039)(425.40350513,855.92354608)(460.00071865,929.3546763)
\curveto(468.8727997,948.18529509)(481.64358758,964.91657331)(497.46761825,978.44085072)
\lineto(592.95954151,866.71087974)
\closepath
}
}
{
\newrgbcolor{curcolor}{0.90196079 0.90196079 0.90196079}
\pscustom[linestyle=none,fillstyle=solid,fillcolor=curcolor]
{
\newpath
\moveto(592.95954439,1013.68806785)
\curveto(614.5842764,1013.68806742)(635.94260228,1008.91629224)(655.51119588,999.71307656)
\lineto(592.95954151,866.71087974)
\closepath
}
}
{
\newrgbcolor{curcolor}{0.7019608 0.7019608 0.7019608}
\pscustom[linestyle=none,fillstyle=solid,fillcolor=curcolor]
{
\newpath
\moveto(497.43918686,978.41654509)
\curveto(508.63489815,987.99008203)(521.20621588,995.82614083)(534.7317083,1001.66203114)
\lineto(592.95954151,866.71087974)
\closepath
}
}
{
\newrgbcolor{curcolor}{0.60000002 0.60000002 0.60000002}
\pscustom[linestyle=none,fillstyle=solid,fillcolor=curcolor]
{
\newpath
\moveto(534.39126395,1001.51462853)
\curveto(541.19030259,1004.46861153)(548.20308123,1006.90398238)(555.36956357,1008.79989192)
\lineto(592.95954151,866.71087974)
\closepath
}
}
{
\newrgbcolor{curcolor}{0.50196081 0.50196081 0.50196081}
\pscustom[linestyle=none,fillstyle=solid,fillcolor=curcolor]
{
\newpath
\moveto(555.25816314,1008.77037394)
\curveto(557.72660079,1009.42547624)(560.2116938,1010.0161296)(562.71091072,1010.54173234)
\lineto(592.95954151,866.71087974)
\closepath
}
}
{
\newrgbcolor{curcolor}{0.40000001 0.40000001 0.40000001}
\pscustom[linestyle=none,fillstyle=solid,fillcolor=curcolor]
{
\newpath
\moveto(562.63288694,1010.52530129)
\curveto(569.78777468,1012.03407772)(577.04528119,1013.00680917)(584.34494839,1013.43539238)
\lineto(592.95954151,866.71087974)
\closepath
}
}
{
\newrgbcolor{curcolor}{0.3019608 0.3019608 0.3019608}
\pscustom[linestyle=none,fillstyle=solid,fillcolor=curcolor]
{
\newpath
\moveto(584.27951012,1013.43153568)
\curveto(587.16969669,1013.60251985)(590.0643045,1013.6880679)(592.95954439,1013.68806785)
\lineto(592.95954151,866.71087974)
\closepath
}
}
{
\newrgbcolor{curcolor}{0.80000001 0.80000001 0.80000001}
\pscustom[linestyle=none,fillstyle=solid,fillcolor=curcolor]
{
\newpath
\moveto(337.24576523,612.4312744)
\curveto(389.71618383,550.49620162)(382.04356565,457.75222632)(320.10849287,405.28180772)
\curveto(258.17342009,352.81138911)(165.4294448,360.48400729)(112.95902619,422.41908007)
\curveto(81.79630987,459.20295097)(70.6540066,508.93065117)(83.14119419,555.49499582)
\lineto(225.10239571,517.42517724)
\closepath
}
}
{
\newrgbcolor{curcolor}{0.90196079 0.90196079 0.90196079}
\pscustom[linestyle=none,fillstyle=solid,fillcolor=curcolor]
{
\newpath
\moveto(225.10239859,664.40236535)
\curveto(268.29106261,664.4023645)(309.29392027,645.40697953)(337.22001446,612.46166176)
\lineto(225.10239571,517.42517724)
\closepath
}
}
{
\newrgbcolor{curcolor}{0.7019608 0.7019608 0.7019608}
\pscustom[linestyle=none,fillstyle=solid,fillcolor=curcolor]
{
\newpath
\moveto(83.14795409,555.52019435)
\curveto(90.9564631,584.61724221)(107.51783106,610.61195198)(130.58965993,629.98465751)
\lineto(225.10239571,517.42517724)
\closepath
}
}
{
\newrgbcolor{curcolor}{0.60000002 0.60000002 0.60000002}
\pscustom[linestyle=none,fillstyle=solid,fillcolor=curcolor]
{
\newpath
\moveto(130.52910536,629.93378392)
\curveto(140.37839787,638.21297275)(151.27220321,645.16299478)(162.93116175,650.60562435)
\lineto(225.10239571,517.42517724)
\closepath
}
}
{
\newrgbcolor{curcolor}{0.50196081 0.50196081 0.50196081}
\pscustom[linestyle=none,fillstyle=solid,fillcolor=curcolor]
{
\newpath
\moveto(162.81658301,650.5520767)
\curveto(166.70525148,652.37146014)(170.67176963,654.01958659)(174.7047261,655.49171454)
\lineto(225.10239571,517.42517724)
\closepath
}
}
{
\newrgbcolor{curcolor}{0.40000001 0.40000001 0.40000001}
\pscustom[linestyle=none,fillstyle=solid,fillcolor=curcolor]
{
\newpath
\moveto(174.7223438,655.49814416)
\curveto(186.5733537,659.8223396)(198.93177124,662.60278343)(211.49284518,663.77091157)
\lineto(225.10239571,517.42517724)
\closepath
}
}
{
\newrgbcolor{curcolor}{0.3019608 0.3019608 0.3019608}
\pscustom[linestyle=none,fillstyle=solid,fillcolor=curcolor]
{
\newpath
\moveto(211.47711309,663.7694477)
\curveto(216.0066328,664.19116552)(220.55328937,664.40236544)(225.10239859,664.40236535)
\lineto(225.10239571,517.42517724)
\closepath
}
}
{
\newrgbcolor{curcolor}{0.80000001 0.80000001 0.80000001}
\pscustom[linestyle=none,fillstyle=solid,fillcolor=curcolor]
{
\newpath
\moveto(737.59321856,498.90315155)
\curveto(727.36378624,418.3770267)(653.79187067,361.39029786)(573.26574582,371.61973019)
\curveto(524.78698765,377.77810645)(482.53951405,407.62887649)(460.54247204,451.26654067)
\lineto(591.78777151,517.42517724)
\closepath
}
}
{
\newrgbcolor{curcolor}{0.90196079 0.90196079 0.90196079}
\pscustom[linestyle=none,fillstyle=solid,fillcolor=curcolor]
{
\newpath
\moveto(591.78777439,664.40236535)
\curveto(672.96103396,664.40236375)(738.76496121,598.59843392)(738.76495962,517.42517435)
\curveto(738.7649595,511.16913812)(738.36553057,504.91948396)(737.56911023,498.71434867)
\lineto(591.78777151,517.42517724)
\closepath
}
}
{
\newrgbcolor{curcolor}{0.7019608 0.7019608 0.7019608}
\pscustom[linestyle=none,fillstyle=solid,fillcolor=curcolor]
{
\newpath
\moveto(460.67042249,451.01332128)
\curveto(445.4455484,481.07188691)(441.0107759,515.44892364)(448.10880638,548.38723147)
\lineto(591.78777151,517.42517724)
\closepath
}
}
{
\newrgbcolor{curcolor}{0.60000002 0.60000002 0.60000002}
\pscustom[linestyle=none,fillstyle=solid,fillcolor=curcolor]
{
\newpath
\moveto(448.0512875,548.11910152)
\curveto(453.29177242,572.65975233)(464.72822409,595.44999033)(481.27123689,614.31887879)
\lineto(591.78777151,517.42517724)
\closepath
}
}
{
\newrgbcolor{curcolor}{0.50196081 0.50196081 0.50196081}
\pscustom[linestyle=none,fillstyle=solid,fillcolor=curcolor]
{
\newpath
\moveto(480.99285422,614.00043352)
\curveto(486.49274358,620.31012206)(492.52157759,626.13845849)(499.01360535,631.42187859)
\lineto(591.78777151,517.42517724)
\closepath
}
}
{
\newrgbcolor{curcolor}{0.40000001 0.40000001 0.40000001}
\pscustom[linestyle=none,fillstyle=solid,fillcolor=curcolor]
{
\newpath
\moveto(498.93108467,631.35467093)
\curveto(518.36714403,647.1957672)(541.53268184,657.80117419)(566.22442315,662.16221135)
\lineto(591.78777151,517.42517724)
\closepath
}
}
{
\newrgbcolor{curcolor}{0.3019608 0.3019608 0.3019608}
\pscustom[linestyle=none,fillstyle=solid,fillcolor=curcolor]
{
\newpath
\moveto(566.22766526,662.16278394)
\curveto(574.6661256,663.65298367)(583.21874247,664.40236551)(591.78777439,664.40236535)
\lineto(591.78777151,517.42517724)
\closepath
}
}
{
\newrgbcolor{curcolor}{0.80000001 0.80000001 0.80000001}
\pscustom[linestyle=none,fillstyle=solid,fillcolor=curcolor]
{
\newpath
\moveto(271.70982689,311.34469046)
\curveto(348.69372779,285.60411798)(390.23465321,202.32947875)(364.49408073,125.34557785)
\curveto(338.75350826,48.36167696)(255.47886903,6.82075154)(178.49496813,32.56132401)
\curveto(154.66116138,40.53046851)(133.26880368,54.4777227)(116.36147388,73.07072696)
\lineto(225.10239751,171.95300724)
\closepath
}
}
{
\newrgbcolor{curcolor}{0.90196079 0.90196079 0.90196079}
\pscustom[linestyle=none,fillstyle=solid,fillcolor=curcolor]
{
\newpath
\moveto(225.10240039,318.93019535)
\curveto(240.93969708,318.93019504)(256.67289668,316.37051786)(271.69342017,311.35017519)
\lineto(225.10239751,171.95300724)
\closepath
}
}
{
\newrgbcolor{curcolor}{0.7019608 0.7019608 0.7019608}
\pscustom[linestyle=none,fillstyle=solid,fillcolor=curcolor]
{
\newpath
\moveto(116.49685574,72.92205187)
\curveto(83.09772861,109.55029889)(70.41138548,160.54621997)(82.75544301,208.55407784)
\lineto(225.10239751,171.95300724)
\closepath
}
}
{
\newrgbcolor{curcolor}{0.60000002 0.60000002 0.60000002}
\pscustom[linestyle=none,fillstyle=solid,fillcolor=curcolor]
{
\newpath
\moveto(82.711147,208.38136986)
\curveto(86.74771998,224.15953369)(93.37863416,239.15708723)(102.33284823,252.76099938)
\lineto(225.10239751,171.95300724)
\closepath
}
}
{
\newrgbcolor{curcolor}{0.50196081 0.50196081 0.50196081}
\pscustom[linestyle=none,fillstyle=solid,fillcolor=curcolor]
{
\newpath
\moveto(102.34455052,252.77877556)
\curveto(102.4127109,252.88229727)(102.4810022,252.98573272)(102.54942426,253.08908165)
\lineto(225.10239751,171.95300724)
\closepath
}
}
{
\newrgbcolor{curcolor}{0.40000001 0.40000001 0.40000001}
\pscustom[linestyle=none,fillstyle=solid,fillcolor=curcolor]
{
\newpath
\moveto(102.39720219,252.85868606)
\curveto(120.14708949,279.77896434)(146.25731169,300.11503808)(176.70425977,310.73309979)
\lineto(225.10239751,171.95300724)
\closepath
}
}
{
\newrgbcolor{curcolor}{0.3019608 0.3019608 0.3019608}
\pscustom[linestyle=none,fillstyle=solid,fillcolor=curcolor]
{
\newpath
\moveto(176.60385117,310.69804256)
\curveto(192.19276986,316.14717372)(208.58854511,318.93019567)(225.10240039,318.93019535)
\lineto(225.10239751,171.95300724)
\closepath
}
}
{
\newrgbcolor{curcolor}{0 0 0}
\pscustom[linestyle=none,fillstyle=solid,fillcolor=curcolor]
{
\newpath
\moveto(116.30596169,582.36980927)
\curveto(116.30595405,582.28980374)(116.31095405,582.20980382)(116.32096169,582.12980927)
\curveto(116.33095403,582.04980398)(116.32595403,581.97480405)(116.30596169,581.90480927)
\curveto(116.28595407,581.86480416)(116.28095408,581.81980421)(116.29096169,581.76980927)
\curveto(116.30095406,581.7298043)(116.30095406,581.68980434)(116.29096169,581.64980927)
\lineto(116.29096169,581.49980927)
\curveto(116.28095408,581.40980462)(116.27595408,581.31980471)(116.27596169,581.22980927)
\curveto(116.27595408,581.14980488)(116.27095409,581.06980496)(116.26096169,580.98980927)
\lineto(116.23096169,580.74980927)
\curveto(116.22095414,580.67980535)(116.21095415,580.60480542)(116.20096169,580.52480927)
\curveto(116.19095417,580.48480554)(116.18595417,580.44480558)(116.18596169,580.40480927)
\curveto(116.18595417,580.36480566)(116.18095418,580.31980571)(116.17096169,580.26980927)
\curveto(116.13095423,580.1298059)(116.10095426,579.98980604)(116.08096169,579.84980927)
\curveto(116.07095429,579.70980632)(116.04095432,579.57480645)(115.99096169,579.44480927)
\curveto(115.94095442,579.27480675)(115.88595447,579.10980692)(115.82596169,578.94980927)
\curveto(115.77595458,578.78980724)(115.71595464,578.63480739)(115.64596169,578.48480927)
\curveto(115.62595473,578.4248076)(115.59595476,578.36480766)(115.55596169,578.30480927)
\lineto(115.46596169,578.15480927)
\curveto(115.26595509,577.83480819)(115.05095531,577.56980846)(114.82096169,577.35980927)
\curveto(114.59095577,577.14980888)(114.29595606,576.96980906)(113.93596169,576.81980927)
\curveto(113.81595654,576.76980926)(113.68595667,576.73480929)(113.54596169,576.71480927)
\curveto(113.41595694,576.69480933)(113.28095708,576.66980936)(113.14096169,576.63980927)
\curveto(113.08095728,576.6298094)(113.02095734,576.6248094)(112.96096169,576.62480927)
\curveto(112.90095746,576.6248094)(112.83595752,576.61980941)(112.76596169,576.60980927)
\curveto(112.73595762,576.59980943)(112.68595767,576.59980943)(112.61596169,576.60980927)
\lineto(112.46596169,576.60980927)
\lineto(112.31596169,576.60980927)
\curveto(112.23595812,576.6298094)(112.15095821,576.64480938)(112.06096169,576.65480927)
\curveto(111.98095838,576.65480937)(111.90595845,576.66480936)(111.83596169,576.68480927)
\curveto(111.79595856,576.69480933)(111.7609586,576.69980933)(111.73096169,576.69980927)
\curveto(111.71095865,576.68980934)(111.68595867,576.69480933)(111.65596169,576.71480927)
\lineto(111.38596169,576.77480927)
\curveto(111.29595906,576.80480922)(111.21095915,576.83480919)(111.13096169,576.86480927)
\curveto(110.55095981,577.10480892)(110.11596024,577.47480855)(109.82596169,577.97480927)
\curveto(109.74596061,578.10480792)(109.68096068,578.23980779)(109.63096169,578.37980927)
\curveto(109.59096077,578.51980751)(109.54596081,578.66980736)(109.49596169,578.82980927)
\curveto(109.47596088,578.90980712)(109.47096089,578.98980704)(109.48096169,579.06980927)
\curveto(109.50096086,579.14980688)(109.53596082,579.20480682)(109.58596169,579.23480927)
\curveto(109.61596074,579.25480677)(109.67096069,579.26980676)(109.75096169,579.27980927)
\curveto(109.83096053,579.29980673)(109.91596044,579.30980672)(110.00596169,579.30980927)
\curveto(110.09596026,579.31980671)(110.18096018,579.31980671)(110.26096169,579.30980927)
\curveto(110.35096001,579.29980673)(110.42095994,579.28980674)(110.47096169,579.27980927)
\curveto(110.49095987,579.26980676)(110.51595984,579.25480677)(110.54596169,579.23480927)
\curveto(110.58595977,579.21480681)(110.61595974,579.19480683)(110.63596169,579.17480927)
\curveto(110.69595966,579.09480693)(110.74095962,578.99980703)(110.77096169,578.88980927)
\curveto(110.81095955,578.77980725)(110.8559595,578.67980735)(110.90596169,578.58980927)
\curveto(111.1559592,578.19980783)(111.52595883,577.9298081)(112.01596169,577.77980927)
\curveto(112.08595827,577.75980827)(112.1559582,577.74480828)(112.22596169,577.73480927)
\curveto(112.30595805,577.73480829)(112.38595797,577.7248083)(112.46596169,577.70480927)
\curveto(112.50595785,577.69480833)(112.5609578,577.68980834)(112.63096169,577.68980927)
\curveto(112.71095765,577.68980834)(112.76595759,577.69480833)(112.79596169,577.70480927)
\curveto(112.82595753,577.71480831)(112.8559575,577.71980831)(112.88596169,577.71980927)
\lineto(112.99096169,577.71980927)
\curveto(113.07095729,577.73980829)(113.14595721,577.75980827)(113.21596169,577.77980927)
\curveto(113.29595706,577.79980823)(113.37095699,577.8248082)(113.44096169,577.85480927)
\curveto(113.79095657,578.00480802)(114.0609563,578.21980781)(114.25096169,578.49980927)
\curveto(114.44095592,578.77980725)(114.59595576,579.10480692)(114.71596169,579.47480927)
\curveto(114.74595561,579.55480647)(114.76595559,579.6298064)(114.77596169,579.69980927)
\curveto(114.79595556,579.76980626)(114.81595554,579.84480618)(114.83596169,579.92480927)
\curveto(114.8559555,580.01480601)(114.87095549,580.10980592)(114.88096169,580.20980927)
\curveto(114.90095546,580.31980571)(114.92095544,580.4248056)(114.94096169,580.52480927)
\curveto(114.95095541,580.57480545)(114.9559554,580.6248054)(114.95596169,580.67480927)
\curveto(114.96595539,580.73480529)(114.97095539,580.78980524)(114.97096169,580.83980927)
\curveto(114.99095537,580.89980513)(115.00095536,580.97480505)(115.00096169,581.06480927)
\curveto(115.00095536,581.16480486)(114.99095537,581.24480478)(114.97096169,581.30480927)
\curveto(114.94095542,581.39480463)(114.89095547,581.43480459)(114.82096169,581.42480927)
\curveto(114.7609556,581.41480461)(114.70595565,581.38480464)(114.65596169,581.33480927)
\curveto(114.57595578,581.28480474)(114.50595585,581.2248048)(114.44596169,581.15480927)
\curveto(114.39595596,581.08480494)(114.33095603,581.024805)(114.25096169,580.97480927)
\curveto(114.09095627,580.86480516)(113.92595643,580.76480526)(113.75596169,580.67480927)
\curveto(113.58595677,580.59480543)(113.39095697,580.5248055)(113.17096169,580.46480927)
\curveto(113.07095729,580.43480559)(112.97095739,580.41980561)(112.87096169,580.41980927)
\curveto(112.78095758,580.41980561)(112.68095768,580.40980562)(112.57096169,580.38980927)
\lineto(112.42096169,580.38980927)
\curveto(112.37095799,580.40980562)(112.32095804,580.41480561)(112.27096169,580.40480927)
\curveto(112.23095813,580.39480563)(112.19095817,580.39480563)(112.15096169,580.40480927)
\curveto(112.12095824,580.41480561)(112.07595828,580.41980561)(112.01596169,580.41980927)
\curveto(111.9559584,580.4298056)(111.89095847,580.43980559)(111.82096169,580.44980927)
\lineto(111.64096169,580.47980927)
\curveto(111.19095917,580.59980543)(110.81095955,580.76480526)(110.50096169,580.97480927)
\curveto(110.23096013,581.16480486)(110.00096036,581.39480463)(109.81096169,581.66480927)
\curveto(109.63096073,581.94480408)(109.48596087,582.25980377)(109.37596169,582.60980927)
\lineto(109.31596169,582.81980927)
\curveto(109.30596105,582.89980313)(109.29096107,582.97980305)(109.27096169,583.05980927)
\curveto(109.2609611,583.08980294)(109.2559611,583.11980291)(109.25596169,583.14980927)
\curveto(109.2559611,583.17980285)(109.25096111,583.20980282)(109.24096169,583.23980927)
\curveto(109.23096113,583.29980273)(109.22596113,583.35980267)(109.22596169,583.41980927)
\curveto(109.22596113,583.48980254)(109.21596114,583.54980248)(109.19596169,583.59980927)
\lineto(109.19596169,583.77980927)
\curveto(109.18596117,583.8298022)(109.18096118,583.89980213)(109.18096169,583.98980927)
\curveto(109.18096118,584.07980195)(109.19096117,584.14980188)(109.21096169,584.19980927)
\lineto(109.21096169,584.36480927)
\curveto(109.23096113,584.44480158)(109.24096112,584.51980151)(109.24096169,584.58980927)
\curveto(109.25096111,584.65980137)(109.26596109,584.7298013)(109.28596169,584.79980927)
\curveto(109.34596101,584.99980103)(109.40596095,585.18980084)(109.46596169,585.36980927)
\curveto(109.53596082,585.54980048)(109.62596073,585.71980031)(109.73596169,585.87980927)
\curveto(109.77596058,585.94980008)(109.81596054,586.01480001)(109.85596169,586.07480927)
\lineto(110.00596169,586.25480927)
\curveto(110.02596033,586.26479976)(110.04596031,586.27979975)(110.06596169,586.29980927)
\curveto(110.1559602,586.4297996)(110.26596009,586.53979949)(110.39596169,586.62980927)
\curveto(110.6559597,586.8297992)(110.92095944,586.98479904)(111.19096169,587.09480927)
\curveto(111.27095909,587.13479889)(111.35095901,587.16479886)(111.43096169,587.18480927)
\curveto(111.52095884,587.21479881)(111.61095875,587.23979879)(111.70096169,587.25980927)
\curveto(111.80095856,587.28979874)(111.90095846,587.30979872)(112.00096169,587.31980927)
\curveto(112.10095826,587.3297987)(112.20595815,587.34479868)(112.31596169,587.36480927)
\curveto(112.34595801,587.37479865)(112.38595797,587.37479865)(112.43596169,587.36480927)
\curveto(112.49595786,587.35479867)(112.53595782,587.35979867)(112.55596169,587.37980927)
\curveto(113.27595708,587.39979863)(113.87595648,587.28479874)(114.35596169,587.03480927)
\curveto(114.83595552,586.78479924)(115.21095515,586.44479958)(115.48096169,586.01480927)
\curveto(115.57095479,585.87480015)(115.65095471,585.7298003)(115.72096169,585.57980927)
\curveto(115.79095457,585.4298006)(115.8609545,585.26980076)(115.93096169,585.09980927)
\curveto(115.98095438,584.95980107)(116.02095434,584.80980122)(116.05096169,584.64980927)
\curveto(116.08095428,584.48980154)(116.11595424,584.3298017)(116.15596169,584.16980927)
\curveto(116.17595418,584.11980191)(116.18595417,584.06480196)(116.18596169,584.00480927)
\curveto(116.18595417,583.95480207)(116.19095417,583.90480212)(116.20096169,583.85480927)
\curveto(116.22095414,583.79480223)(116.23095413,583.7298023)(116.23096169,583.65980927)
\curveto(116.23095413,583.59980243)(116.24095412,583.54480248)(116.26096169,583.49480927)
\lineto(116.26096169,583.32980927)
\curveto(116.28095408,583.27980275)(116.28595407,583.2298028)(116.27596169,583.17980927)
\curveto(116.26595409,583.1298029)(116.27095409,583.07980295)(116.29096169,583.02980927)
\curveto(116.29095407,583.00980302)(116.28595407,582.98480304)(116.27596169,582.95480927)
\curveto(116.27595408,582.9248031)(116.28095408,582.89980313)(116.29096169,582.87980927)
\curveto(116.30095406,582.84980318)(116.30095406,582.81480321)(116.29096169,582.77480927)
\curveto(116.29095407,582.73480329)(116.29595406,582.69480333)(116.30596169,582.65480927)
\curveto(116.31595404,582.61480341)(116.31595404,582.56980346)(116.30596169,582.51980927)
\lineto(116.30596169,582.36980927)
\moveto(114.80596169,583.67480927)
\curveto(114.81595554,583.7248023)(114.82095554,583.78480224)(114.82096169,583.85480927)
\curveto(114.82095554,583.9248021)(114.81595554,583.98480204)(114.80596169,584.03480927)
\curveto(114.79595556,584.08480194)(114.79095557,584.15980187)(114.79096169,584.25980927)
\curveto(114.77095559,584.33980169)(114.75095561,584.41480161)(114.73096169,584.48480927)
\curveto(114.72095564,584.55480147)(114.70595565,584.6248014)(114.68596169,584.69480927)
\curveto(114.54595581,585.1248009)(114.35095601,585.45980057)(114.10096169,585.69980927)
\curveto(113.8609565,585.93980009)(113.51595684,586.11979991)(113.06596169,586.23980927)
\curveto(112.97595738,586.25979977)(112.87595748,586.26979976)(112.76596169,586.26980927)
\lineto(112.43596169,586.26980927)
\curveto(112.41595794,586.24979978)(112.38095798,586.23979979)(112.33096169,586.23980927)
\curveto(112.28095808,586.24979978)(112.23595812,586.24979978)(112.19596169,586.23980927)
\curveto(112.11595824,586.21979981)(112.04095832,586.19979983)(111.97096169,586.17980927)
\lineto(111.76096169,586.11980927)
\curveto(111.47095889,585.98980004)(111.24095912,585.80980022)(111.07096169,585.57980927)
\curveto(110.90095946,585.35980067)(110.76595959,585.09980093)(110.66596169,584.79980927)
\curveto(110.63595972,584.70980132)(110.61095975,584.61480141)(110.59096169,584.51480927)
\curveto(110.58095978,584.4248016)(110.56595979,584.3298017)(110.54596169,584.22980927)
\lineto(110.54596169,584.09480927)
\curveto(110.51595984,583.98480204)(110.50595985,583.84480218)(110.51596169,583.67480927)
\curveto(110.53595982,583.51480251)(110.5559598,583.38480264)(110.57596169,583.28480927)
\curveto(110.59595976,583.2248028)(110.61095975,583.16480286)(110.62096169,583.10480927)
\curveto(110.63095973,583.05480297)(110.64595971,583.00480302)(110.66596169,582.95480927)
\curveto(110.74595961,582.75480327)(110.84095952,582.56480346)(110.95096169,582.38480927)
\curveto(111.07095929,582.20480382)(111.21095915,582.05980397)(111.37096169,581.94980927)
\curveto(111.42095894,581.89980413)(111.47595888,581.85980417)(111.53596169,581.82980927)
\curveto(111.59595876,581.79980423)(111.6559587,581.76480426)(111.71596169,581.72480927)
\curveto(111.86595849,581.64480438)(112.05095831,581.57980445)(112.27096169,581.52980927)
\curveto(112.32095804,581.50980452)(112.360958,581.50480452)(112.39096169,581.51480927)
\curveto(112.43095793,581.5248045)(112.47595788,581.51980451)(112.52596169,581.49980927)
\curveto(112.56595779,581.48980454)(112.62095774,581.48480454)(112.69096169,581.48480927)
\curveto(112.7609576,581.48480454)(112.82095754,581.48980454)(112.87096169,581.49980927)
\curveto(112.97095739,581.51980451)(113.06595729,581.53480449)(113.15596169,581.54480927)
\curveto(113.24595711,581.56480446)(113.33595702,581.59480443)(113.42596169,581.63480927)
\curveto(113.96595639,581.85480417)(114.360956,582.24980378)(114.61096169,582.81980927)
\curveto(114.6609557,582.91980311)(114.69595566,583.01980301)(114.71596169,583.11980927)
\curveto(114.73595562,583.2298028)(114.7609556,583.33980269)(114.79096169,583.44980927)
\curveto(114.79095557,583.54980248)(114.79595556,583.6248024)(114.80596169,583.67480927)
}
}
{
\newrgbcolor{curcolor}{0 0 0}
\pscustom[linestyle=none,fillstyle=solid,fillcolor=curcolor]
{
\newpath
\moveto(118.67057106,578.40980927)
\lineto(118.97057106,578.40980927)
\curveto(119.080569,578.41980761)(119.1855689,578.41980761)(119.28557106,578.40980927)
\curveto(119.39556869,578.40980762)(119.49556859,578.39980763)(119.58557106,578.37980927)
\curveto(119.67556841,578.36980766)(119.74556834,578.34480768)(119.79557106,578.30480927)
\curveto(119.81556827,578.28480774)(119.83056825,578.25480777)(119.84057106,578.21480927)
\curveto(119.86056822,578.17480785)(119.8805682,578.1298079)(119.90057106,578.07980927)
\lineto(119.90057106,578.00480927)
\curveto(119.91056817,577.95480807)(119.91056817,577.89980813)(119.90057106,577.83980927)
\lineto(119.90057106,577.68980927)
\lineto(119.90057106,577.20980927)
\curveto(119.90056818,577.03980899)(119.86056822,576.91980911)(119.78057106,576.84980927)
\curveto(119.71056837,576.79980923)(119.62056846,576.77480925)(119.51057106,576.77480927)
\lineto(119.18057106,576.77480927)
\lineto(118.73057106,576.77480927)
\curveto(118.5805695,576.77480925)(118.46556962,576.80480922)(118.38557106,576.86480927)
\curveto(118.34556974,576.89480913)(118.31556977,576.94480908)(118.29557106,577.01480927)
\curveto(118.27556981,577.09480893)(118.26056982,577.17980885)(118.25057106,577.26980927)
\lineto(118.25057106,577.55480927)
\curveto(118.26056982,577.65480837)(118.26556982,577.73980829)(118.26557106,577.80980927)
\lineto(118.26557106,578.00480927)
\curveto(118.26556982,578.06480796)(118.27556981,578.11980791)(118.29557106,578.16980927)
\curveto(118.33556975,578.27980775)(118.40556968,578.34980768)(118.50557106,578.37980927)
\curveto(118.53556955,578.37980765)(118.59056949,578.38980764)(118.67057106,578.40980927)
}
}
{
\newrgbcolor{curcolor}{0 0 0}
\pscustom[linestyle=none,fillstyle=solid,fillcolor=curcolor]
{
\newpath
\moveto(122.33572731,587.18480927)
\lineto(127.13572731,587.18480927)
\lineto(128.14072731,587.18480927)
\curveto(128.28072021,587.18479884)(128.40072009,587.17479885)(128.50072731,587.15480927)
\curveto(128.61071988,587.14479888)(128.6907198,587.09979893)(128.74072731,587.01980927)
\curveto(128.76071973,586.97979905)(128.77071972,586.9297991)(128.77072731,586.86980927)
\curveto(128.78071971,586.80979922)(128.78571971,586.74479928)(128.78572731,586.67480927)
\lineto(128.78572731,586.40480927)
\curveto(128.78571971,586.31479971)(128.77571972,586.23479979)(128.75572731,586.16480927)
\curveto(128.71571978,586.08479994)(128.67071982,586.01480001)(128.62072731,585.95480927)
\lineto(128.47072731,585.77480927)
\curveto(128.44072005,585.7248003)(128.40572009,585.68480034)(128.36572731,585.65480927)
\curveto(128.32572017,585.6248004)(128.28572021,585.58480044)(128.24572731,585.53480927)
\curveto(128.16572033,585.4248006)(128.08072041,585.31480071)(127.99072731,585.20480927)
\curveto(127.90072059,585.10480092)(127.81572068,584.99980103)(127.73572731,584.88980927)
\curveto(127.5957209,584.68980134)(127.45572104,584.47980155)(127.31572731,584.25980927)
\curveto(127.17572132,584.04980198)(127.03572146,583.83480219)(126.89572731,583.61480927)
\curveto(126.84572165,583.5248025)(126.7957217,583.4298026)(126.74572731,583.32980927)
\curveto(126.6957218,583.2298028)(126.64072185,583.13480289)(126.58072731,583.04480927)
\curveto(126.56072193,583.024803)(126.55072194,582.99980303)(126.55072731,582.96980927)
\curveto(126.55072194,582.93980309)(126.54072195,582.91480311)(126.52072731,582.89480927)
\curveto(126.45072204,582.79480323)(126.38572211,582.67980335)(126.32572731,582.54980927)
\curveto(126.26572223,582.4298036)(126.21072228,582.31480371)(126.16072731,582.20480927)
\curveto(126.06072243,581.97480405)(125.96572253,581.73980429)(125.87572731,581.49980927)
\curveto(125.78572271,581.25980477)(125.68572281,581.01980501)(125.57572731,580.77980927)
\curveto(125.55572294,580.7298053)(125.54072295,580.68480534)(125.53072731,580.64480927)
\curveto(125.53072296,580.60480542)(125.52072297,580.55980547)(125.50072731,580.50980927)
\curveto(125.45072304,580.38980564)(125.40572309,580.26480576)(125.36572731,580.13480927)
\curveto(125.33572316,580.01480601)(125.30072319,579.89480613)(125.26072731,579.77480927)
\curveto(125.18072331,579.54480648)(125.11572338,579.30480672)(125.06572731,579.05480927)
\curveto(125.02572347,578.81480721)(124.97572352,578.57480745)(124.91572731,578.33480927)
\curveto(124.87572362,578.18480784)(124.85072364,578.03480799)(124.84072731,577.88480927)
\curveto(124.83072366,577.73480829)(124.81072368,577.58480844)(124.78072731,577.43480927)
\curveto(124.77072372,577.39480863)(124.76572373,577.33480869)(124.76572731,577.25480927)
\curveto(124.73572376,577.13480889)(124.70572379,577.03480899)(124.67572731,576.95480927)
\curveto(124.64572385,576.87480915)(124.57572392,576.81980921)(124.46572731,576.78980927)
\curveto(124.41572408,576.76980926)(124.36072413,576.75980927)(124.30072731,576.75980927)
\lineto(124.10572731,576.75980927)
\curveto(123.96572453,576.75980927)(123.82572467,576.76480926)(123.68572731,576.77480927)
\curveto(123.55572494,576.78480924)(123.46072503,576.8298092)(123.40072731,576.90980927)
\curveto(123.36072513,576.96980906)(123.34072515,577.05480897)(123.34072731,577.16480927)
\curveto(123.35072514,577.27480875)(123.36572513,577.36980866)(123.38572731,577.44980927)
\lineto(123.38572731,577.52480927)
\curveto(123.3957251,577.55480847)(123.40072509,577.58480844)(123.40072731,577.61480927)
\curveto(123.42072507,577.69480833)(123.43072506,577.76980826)(123.43072731,577.83980927)
\curveto(123.43072506,577.90980812)(123.44072505,577.97980805)(123.46072731,578.04980927)
\curveto(123.51072498,578.23980779)(123.55072494,578.4248076)(123.58072731,578.60480927)
\curveto(123.61072488,578.79480723)(123.65072484,578.97480705)(123.70072731,579.14480927)
\curveto(123.72072477,579.19480683)(123.73072476,579.23480679)(123.73072731,579.26480927)
\curveto(123.73072476,579.29480673)(123.73572476,579.3298067)(123.74572731,579.36980927)
\curveto(123.84572465,579.66980636)(123.93572456,579.96480606)(124.01572731,580.25480927)
\curveto(124.10572439,580.54480548)(124.21072428,580.8248052)(124.33072731,581.09480927)
\curveto(124.5907239,581.67480435)(124.86072363,582.2248038)(125.14072731,582.74480927)
\curveto(125.42072307,583.27480275)(125.73072276,583.77980225)(126.07072731,584.25980927)
\curveto(126.21072228,584.45980157)(126.36072213,584.64980138)(126.52072731,584.82980927)
\curveto(126.68072181,585.01980101)(126.83072166,585.20980082)(126.97072731,585.39980927)
\curveto(127.01072148,585.44980058)(127.04572145,585.49480053)(127.07572731,585.53480927)
\curveto(127.11572138,585.58480044)(127.15072134,585.63480039)(127.18072731,585.68480927)
\curveto(127.1907213,585.70480032)(127.20072129,585.7298003)(127.21072731,585.75980927)
\curveto(127.23072126,585.78980024)(127.23072126,585.81980021)(127.21072731,585.84980927)
\curveto(127.1907213,585.90980012)(127.15572134,585.94480008)(127.10572731,585.95480927)
\curveto(127.05572144,585.97480005)(127.00572149,585.99480003)(126.95572731,586.01480927)
\lineto(126.85072731,586.01480927)
\curveto(126.81072168,586.0248)(126.76072173,586.0248)(126.70072731,586.01480927)
\lineto(126.55072731,586.01480927)
\lineto(125.95072731,586.01480927)
\lineto(123.31072731,586.01480927)
\lineto(122.57572731,586.01480927)
\lineto(122.33572731,586.01480927)
\curveto(122.26572623,586.0248)(122.20572629,586.03979999)(122.15572731,586.05980927)
\curveto(122.06572643,586.09979993)(122.00572649,586.15979987)(121.97572731,586.23980927)
\curveto(121.92572657,586.33979969)(121.91072658,586.48479954)(121.93072731,586.67480927)
\curveto(121.95072654,586.87479915)(121.98572651,587.00979902)(122.03572731,587.07980927)
\curveto(122.05572644,587.09979893)(122.08072641,587.11479891)(122.11072731,587.12480927)
\lineto(122.23072731,587.18480927)
\curveto(122.25072624,587.18479884)(122.26572623,587.17979885)(122.27572731,587.16980927)
\curveto(122.2957262,587.16979886)(122.31572618,587.17479885)(122.33572731,587.18480927)
}
}
{
\newrgbcolor{curcolor}{0 0 0}
\pscustom[linestyle=none,fillstyle=solid,fillcolor=curcolor]
{
\newpath
\moveto(140.03033669,585.29480927)
\curveto(139.83032639,585.00480102)(139.6203266,584.71980131)(139.40033669,584.43980927)
\curveto(139.19032703,584.15980187)(138.98532723,583.87480215)(138.78533669,583.58480927)
\curveto(138.18532803,582.73480329)(137.58032864,581.89480413)(136.97033669,581.06480927)
\curveto(136.36032986,580.24480578)(135.75533046,579.40980662)(135.15533669,578.55980927)
\lineto(134.64533669,577.83980927)
\lineto(134.13533669,577.14980927)
\curveto(134.05533216,577.03980899)(133.97533224,576.9248091)(133.89533669,576.80480927)
\curveto(133.8153324,576.68480934)(133.7203325,576.58980944)(133.61033669,576.51980927)
\curveto(133.57033265,576.49980953)(133.50533271,576.48480954)(133.41533669,576.47480927)
\curveto(133.33533288,576.45480957)(133.24533297,576.44480958)(133.14533669,576.44480927)
\curveto(133.04533317,576.44480958)(132.95033327,576.44980958)(132.86033669,576.45980927)
\curveto(132.78033344,576.46980956)(132.7203335,576.48980954)(132.68033669,576.51980927)
\curveto(132.65033357,576.53980949)(132.62533359,576.57480945)(132.60533669,576.62480927)
\curveto(132.59533362,576.66480936)(132.60033362,576.70980932)(132.62033669,576.75980927)
\curveto(132.66033356,576.83980919)(132.70533351,576.91480911)(132.75533669,576.98480927)
\curveto(132.8153334,577.06480896)(132.87033335,577.14480888)(132.92033669,577.22480927)
\curveto(133.16033306,577.56480846)(133.40533281,577.89980813)(133.65533669,578.22980927)
\curveto(133.90533231,578.55980747)(134.14533207,578.89480713)(134.37533669,579.23480927)
\curveto(134.53533168,579.45480657)(134.69533152,579.66980636)(134.85533669,579.87980927)
\curveto(135.0153312,580.08980594)(135.17533104,580.30480572)(135.33533669,580.52480927)
\curveto(135.69533052,581.04480498)(136.06033016,581.55480447)(136.43033669,582.05480927)
\curveto(136.80032942,582.55480347)(137.17032905,583.06480296)(137.54033669,583.58480927)
\curveto(137.68032854,583.78480224)(137.8203284,583.97980205)(137.96033669,584.16980927)
\curveto(138.11032811,584.35980167)(138.25532796,584.55480147)(138.39533669,584.75480927)
\curveto(138.60532761,585.05480097)(138.8203274,585.35480067)(139.04033669,585.65480927)
\lineto(139.70033669,586.55480927)
\lineto(139.88033669,586.82480927)
\lineto(140.09033669,587.09480927)
\lineto(140.21033669,587.27480927)
\curveto(140.26032596,587.33479869)(140.31032591,587.38979864)(140.36033669,587.43980927)
\curveto(140.43032579,587.48979854)(140.50532571,587.5247985)(140.58533669,587.54480927)
\curveto(140.60532561,587.55479847)(140.63032559,587.55479847)(140.66033669,587.54480927)
\curveto(140.70032552,587.54479848)(140.73032549,587.55479847)(140.75033669,587.57480927)
\curveto(140.87032535,587.57479845)(141.00532521,587.56979846)(141.15533669,587.55980927)
\curveto(141.30532491,587.55979847)(141.39532482,587.51479851)(141.42533669,587.42480927)
\curveto(141.44532477,587.39479863)(141.45032477,587.35979867)(141.44033669,587.31980927)
\curveto(141.43032479,587.27979875)(141.4153248,587.24979878)(141.39533669,587.22980927)
\curveto(141.35532486,587.14979888)(141.3153249,587.07979895)(141.27533669,587.01980927)
\curveto(141.23532498,586.95979907)(141.19032503,586.89979913)(141.14033669,586.83980927)
\lineto(140.57033669,586.05980927)
\curveto(140.39032583,585.80980022)(140.21032601,585.55480047)(140.03033669,585.29480927)
\moveto(133.17533669,581.39480927)
\curveto(133.12533309,581.41480461)(133.07533314,581.41980461)(133.02533669,581.40980927)
\curveto(132.97533324,581.39980463)(132.92533329,581.40480462)(132.87533669,581.42480927)
\curveto(132.76533345,581.44480458)(132.66033356,581.46480456)(132.56033669,581.48480927)
\curveto(132.47033375,581.51480451)(132.37533384,581.55480447)(132.27533669,581.60480927)
\curveto(131.94533427,581.74480428)(131.69033453,581.93980409)(131.51033669,582.18980927)
\curveto(131.33033489,582.44980358)(131.18533503,582.75980327)(131.07533669,583.11980927)
\curveto(131.04533517,583.19980283)(131.02533519,583.27980275)(131.01533669,583.35980927)
\curveto(131.00533521,583.44980258)(130.99033523,583.53480249)(130.97033669,583.61480927)
\curveto(130.96033526,583.66480236)(130.95533526,583.7298023)(130.95533669,583.80980927)
\curveto(130.94533527,583.83980219)(130.94033528,583.86980216)(130.94033669,583.89980927)
\curveto(130.94033528,583.93980209)(130.93533528,583.97480205)(130.92533669,584.00480927)
\lineto(130.92533669,584.15480927)
\curveto(130.9153353,584.20480182)(130.91033531,584.26480176)(130.91033669,584.33480927)
\curveto(130.91033531,584.41480161)(130.9153353,584.47980155)(130.92533669,584.52980927)
\lineto(130.92533669,584.69480927)
\curveto(130.94533527,584.74480128)(130.95033527,584.78980124)(130.94033669,584.82980927)
\curveto(130.94033528,584.87980115)(130.94533527,584.9248011)(130.95533669,584.96480927)
\curveto(130.96533525,585.00480102)(130.97033525,585.03980099)(130.97033669,585.06980927)
\curveto(130.97033525,585.10980092)(130.97533524,585.14980088)(130.98533669,585.18980927)
\curveto(131.0153352,585.29980073)(131.03533518,585.40980062)(131.04533669,585.51980927)
\curveto(131.06533515,585.63980039)(131.10033512,585.75480027)(131.15033669,585.86480927)
\curveto(131.29033493,586.20479982)(131.45033477,586.47979955)(131.63033669,586.68980927)
\curveto(131.8203344,586.90979912)(132.09033413,587.08979894)(132.44033669,587.22980927)
\curveto(132.5203337,587.25979877)(132.60533361,587.27979875)(132.69533669,587.28980927)
\curveto(132.78533343,587.30979872)(132.88033334,587.3297987)(132.98033669,587.34980927)
\curveto(133.01033321,587.35979867)(133.06533315,587.35979867)(133.14533669,587.34980927)
\curveto(133.22533299,587.34979868)(133.27533294,587.35979867)(133.29533669,587.37980927)
\curveto(133.85533236,587.38979864)(134.30533191,587.27979875)(134.64533669,587.04980927)
\curveto(134.99533122,586.81979921)(135.25533096,586.51479951)(135.42533669,586.13480927)
\curveto(135.46533075,586.04479998)(135.50033072,585.94980008)(135.53033669,585.84980927)
\curveto(135.56033066,585.74980028)(135.58533063,585.64980038)(135.60533669,585.54980927)
\curveto(135.62533059,585.51980051)(135.63033059,585.48980054)(135.62033669,585.45980927)
\curveto(135.6203306,585.4298006)(135.62533059,585.39980063)(135.63533669,585.36980927)
\curveto(135.66533055,585.25980077)(135.68533053,585.13480089)(135.69533669,584.99480927)
\curveto(135.70533051,584.86480116)(135.7153305,584.7298013)(135.72533669,584.58980927)
\lineto(135.72533669,584.42480927)
\curveto(135.73533048,584.36480166)(135.73533048,584.30980172)(135.72533669,584.25980927)
\curveto(135.7153305,584.20980182)(135.71033051,584.15980187)(135.71033669,584.10980927)
\lineto(135.71033669,583.97480927)
\curveto(135.70033052,583.93480209)(135.69533052,583.89480213)(135.69533669,583.85480927)
\curveto(135.70533051,583.81480221)(135.70033052,583.76980226)(135.68033669,583.71980927)
\curveto(135.66033056,583.60980242)(135.64033058,583.50480252)(135.62033669,583.40480927)
\curveto(135.61033061,583.30480272)(135.59033063,583.20480282)(135.56033669,583.10480927)
\curveto(135.43033079,582.74480328)(135.26533095,582.4298036)(135.06533669,582.15980927)
\curveto(134.86533135,581.88980414)(134.59033163,581.68480434)(134.24033669,581.54480927)
\curveto(134.16033206,581.51480451)(134.07533214,581.48980454)(133.98533669,581.46980927)
\lineto(133.71533669,581.40980927)
\curveto(133.66533255,581.39980463)(133.6203326,581.39480463)(133.58033669,581.39480927)
\curveto(133.54033268,581.40480462)(133.50033272,581.40480462)(133.46033669,581.39480927)
\curveto(133.36033286,581.37480465)(133.26533295,581.37480465)(133.17533669,581.39480927)
\moveto(132.33533669,582.78980927)
\curveto(132.37533384,582.71980331)(132.4153338,582.65480337)(132.45533669,582.59480927)
\curveto(132.49533372,582.54480348)(132.54533367,582.49480353)(132.60533669,582.44480927)
\lineto(132.75533669,582.32480927)
\curveto(132.8153334,582.29480373)(132.88033334,582.26980376)(132.95033669,582.24980927)
\curveto(132.99033323,582.2298038)(133.02533319,582.21980381)(133.05533669,582.21980927)
\curveto(133.09533312,582.2298038)(133.13533308,582.2248038)(133.17533669,582.20480927)
\curveto(133.20533301,582.20480382)(133.24533297,582.19980383)(133.29533669,582.18980927)
\curveto(133.34533287,582.18980384)(133.38533283,582.19480383)(133.41533669,582.20480927)
\lineto(133.64033669,582.24980927)
\curveto(133.89033233,582.3298037)(134.07533214,582.45480357)(134.19533669,582.62480927)
\curveto(134.27533194,582.7248033)(134.34533187,582.85480317)(134.40533669,583.01480927)
\curveto(134.48533173,583.19480283)(134.54533167,583.41980261)(134.58533669,583.68980927)
\curveto(134.62533159,583.96980206)(134.64033158,584.24980178)(134.63033669,584.52980927)
\curveto(134.6203316,584.81980121)(134.59033163,585.09480093)(134.54033669,585.35480927)
\curveto(134.49033173,585.61480041)(134.4153318,585.8248002)(134.31533669,585.98480927)
\curveto(134.19533202,586.18479984)(134.04533217,586.33479969)(133.86533669,586.43480927)
\curveto(133.78533243,586.48479954)(133.69533252,586.51479951)(133.59533669,586.52480927)
\curveto(133.49533272,586.54479948)(133.39033283,586.55479947)(133.28033669,586.55480927)
\curveto(133.26033296,586.54479948)(133.23533298,586.53979949)(133.20533669,586.53980927)
\curveto(133.18533303,586.54979948)(133.16533305,586.54979948)(133.14533669,586.53980927)
\curveto(133.09533312,586.5297995)(133.05033317,586.51979951)(133.01033669,586.50980927)
\curveto(132.97033325,586.50979952)(132.93033329,586.49979953)(132.89033669,586.47980927)
\curveto(132.71033351,586.39979963)(132.56033366,586.27979975)(132.44033669,586.11980927)
\curveto(132.33033389,585.95980007)(132.24033398,585.77980025)(132.17033669,585.57980927)
\curveto(132.11033411,585.38980064)(132.06533415,585.16480086)(132.03533669,584.90480927)
\curveto(132.0153342,584.64480138)(132.01033421,584.37980165)(132.02033669,584.10980927)
\curveto(132.03033419,583.84980218)(132.06033416,583.59980243)(132.11033669,583.35980927)
\curveto(132.17033405,583.1298029)(132.24533397,582.93980309)(132.33533669,582.78980927)
\moveto(143.13533669,579.80480927)
\curveto(143.14532307,579.75480627)(143.15032307,579.66480636)(143.15033669,579.53480927)
\curveto(143.15032307,579.40480662)(143.14032308,579.31480671)(143.12033669,579.26480927)
\curveto(143.10032312,579.21480681)(143.09532312,579.15980687)(143.10533669,579.09980927)
\curveto(143.1153231,579.04980698)(143.1153231,578.99980703)(143.10533669,578.94980927)
\curveto(143.06532315,578.80980722)(143.03532318,578.67480735)(143.01533669,578.54480927)
\curveto(143.00532321,578.41480761)(142.97532324,578.29480773)(142.92533669,578.18480927)
\curveto(142.78532343,577.83480819)(142.6203236,577.53980849)(142.43033669,577.29980927)
\curveto(142.24032398,577.06980896)(141.97032425,576.88480914)(141.62033669,576.74480927)
\curveto(141.54032468,576.71480931)(141.45532476,576.69480933)(141.36533669,576.68480927)
\curveto(141.27532494,576.66480936)(141.19032503,576.64480938)(141.11033669,576.62480927)
\curveto(141.06032516,576.61480941)(141.01032521,576.60980942)(140.96033669,576.60980927)
\curveto(140.91032531,576.60980942)(140.86032536,576.60480942)(140.81033669,576.59480927)
\curveto(140.78032544,576.58480944)(140.73032549,576.58480944)(140.66033669,576.59480927)
\curveto(140.59032563,576.59480943)(140.54032568,576.59980943)(140.51033669,576.60980927)
\curveto(140.45032577,576.6298094)(140.39032583,576.63980939)(140.33033669,576.63980927)
\curveto(140.28032594,576.6298094)(140.23032599,576.63480939)(140.18033669,576.65480927)
\curveto(140.09032613,576.67480935)(140.00032622,576.69980933)(139.91033669,576.72980927)
\curveto(139.83032639,576.74980928)(139.75032647,576.77980925)(139.67033669,576.81980927)
\curveto(139.35032687,576.95980907)(139.10032712,577.15480887)(138.92033669,577.40480927)
\curveto(138.74032748,577.66480836)(138.59032763,577.96980806)(138.47033669,578.31980927)
\curveto(138.45032777,578.39980763)(138.43532778,578.48480754)(138.42533669,578.57480927)
\curveto(138.4153278,578.66480736)(138.40032782,578.74980728)(138.38033669,578.82980927)
\curveto(138.37032785,578.85980717)(138.36532785,578.88980714)(138.36533669,578.91980927)
\lineto(138.36533669,579.02480927)
\curveto(138.34532787,579.10480692)(138.33532788,579.18480684)(138.33533669,579.26480927)
\lineto(138.33533669,579.39980927)
\curveto(138.3153279,579.49980653)(138.3153279,579.59980643)(138.33533669,579.69980927)
\lineto(138.33533669,579.87980927)
\curveto(138.34532787,579.9298061)(138.35032787,579.97480605)(138.35033669,580.01480927)
\curveto(138.35032787,580.06480596)(138.35532786,580.10980592)(138.36533669,580.14980927)
\curveto(138.37532784,580.18980584)(138.38032784,580.2248058)(138.38033669,580.25480927)
\curveto(138.38032784,580.29480573)(138.38532783,580.33480569)(138.39533669,580.37480927)
\lineto(138.45533669,580.70480927)
\curveto(138.47532774,580.8248052)(138.50532771,580.93480509)(138.54533669,581.03480927)
\curveto(138.68532753,581.36480466)(138.84532737,581.63980439)(139.02533669,581.85980927)
\curveto(139.215327,582.08980394)(139.47532674,582.27480375)(139.80533669,582.41480927)
\curveto(139.88532633,582.45480357)(139.97032625,582.47980355)(140.06033669,582.48980927)
\lineto(140.36033669,582.54980927)
\lineto(140.49533669,582.54980927)
\curveto(140.54532567,582.55980347)(140.59532562,582.56480346)(140.64533669,582.56480927)
\curveto(141.215325,582.58480344)(141.67532454,582.47980355)(142.02533669,582.24980927)
\curveto(142.38532383,582.029804)(142.65032357,581.7298043)(142.82033669,581.34980927)
\curveto(142.87032335,581.24980478)(142.91032331,581.14980488)(142.94033669,581.04980927)
\curveto(142.97032325,580.94980508)(143.00032322,580.84480518)(143.03033669,580.73480927)
\curveto(143.04032318,580.69480533)(143.04532317,580.65980537)(143.04533669,580.62980927)
\curveto(143.04532317,580.60980542)(143.05032317,580.57980545)(143.06033669,580.53980927)
\curveto(143.08032314,580.46980556)(143.09032313,580.39480563)(143.09033669,580.31480927)
\curveto(143.09032313,580.23480579)(143.10032312,580.15480587)(143.12033669,580.07480927)
\curveto(143.1203231,580.024806)(143.1203231,579.97980605)(143.12033669,579.93980927)
\curveto(143.1203231,579.89980613)(143.12532309,579.85480617)(143.13533669,579.80480927)
\moveto(142.02533669,579.36980927)
\curveto(142.03532418,579.41980661)(142.04032418,579.49480653)(142.04033669,579.59480927)
\curveto(142.05032417,579.69480633)(142.04532417,579.76980626)(142.02533669,579.81980927)
\curveto(142.00532421,579.87980615)(142.00032422,579.93480609)(142.01033669,579.98480927)
\curveto(142.03032419,580.04480598)(142.03032419,580.10480592)(142.01033669,580.16480927)
\curveto(142.00032422,580.19480583)(141.99532422,580.2298058)(141.99533669,580.26980927)
\curveto(141.99532422,580.30980572)(141.99032423,580.34980568)(141.98033669,580.38980927)
\curveto(141.96032426,580.46980556)(141.94032428,580.54480548)(141.92033669,580.61480927)
\curveto(141.91032431,580.69480533)(141.89532432,580.77480525)(141.87533669,580.85480927)
\curveto(141.84532437,580.91480511)(141.8203244,580.97480505)(141.80033669,581.03480927)
\curveto(141.78032444,581.09480493)(141.75032447,581.15480487)(141.71033669,581.21480927)
\curveto(141.61032461,581.38480464)(141.48032474,581.51980451)(141.32033669,581.61980927)
\curveto(141.24032498,581.66980436)(141.14532507,581.70480432)(141.03533669,581.72480927)
\curveto(140.92532529,581.74480428)(140.80032542,581.75480427)(140.66033669,581.75480927)
\curveto(140.64032558,581.74480428)(140.6153256,581.73980429)(140.58533669,581.73980927)
\curveto(140.55532566,581.74980428)(140.52532569,581.74980428)(140.49533669,581.73980927)
\lineto(140.34533669,581.67980927)
\curveto(140.29532592,581.66980436)(140.25032597,581.65480437)(140.21033669,581.63480927)
\curveto(140.0203262,581.5248045)(139.87532634,581.37980465)(139.77533669,581.19980927)
\curveto(139.68532653,581.01980501)(139.60532661,580.81480521)(139.53533669,580.58480927)
\curveto(139.49532672,580.45480557)(139.47532674,580.31980571)(139.47533669,580.17980927)
\curveto(139.47532674,580.04980598)(139.46532675,579.90480612)(139.44533669,579.74480927)
\curveto(139.43532678,579.69480633)(139.42532679,579.63480639)(139.41533669,579.56480927)
\curveto(139.4153268,579.49480653)(139.42532679,579.43480659)(139.44533669,579.38480927)
\lineto(139.44533669,579.21980927)
\lineto(139.44533669,579.03980927)
\curveto(139.45532676,578.98980704)(139.46532675,578.93480709)(139.47533669,578.87480927)
\curveto(139.48532673,578.8248072)(139.49032673,578.76980726)(139.49033669,578.70980927)
\curveto(139.50032672,578.64980738)(139.5153267,578.59480743)(139.53533669,578.54480927)
\curveto(139.58532663,578.35480767)(139.64532657,578.17980785)(139.71533669,578.01980927)
\curveto(139.78532643,577.85980817)(139.89032633,577.7298083)(140.03033669,577.62980927)
\curveto(140.16032606,577.5298085)(140.30032592,577.45980857)(140.45033669,577.41980927)
\curveto(140.48032574,577.40980862)(140.50532571,577.40480862)(140.52533669,577.40480927)
\curveto(140.55532566,577.41480861)(140.58532563,577.41480861)(140.61533669,577.40480927)
\curveto(140.63532558,577.40480862)(140.66532555,577.39980863)(140.70533669,577.38980927)
\curveto(140.74532547,577.38980864)(140.78032544,577.39480863)(140.81033669,577.40480927)
\curveto(140.85032537,577.41480861)(140.89032533,577.41980861)(140.93033669,577.41980927)
\curveto(140.97032525,577.41980861)(141.01032521,577.4298086)(141.05033669,577.44980927)
\curveto(141.29032493,577.5298085)(141.48532473,577.66480836)(141.63533669,577.85480927)
\curveto(141.75532446,578.03480799)(141.84532437,578.23980779)(141.90533669,578.46980927)
\curveto(141.92532429,578.53980749)(141.94032428,578.60980742)(141.95033669,578.67980927)
\curveto(141.96032426,578.75980727)(141.97532424,578.83980719)(141.99533669,578.91980927)
\curveto(141.99532422,578.97980705)(142.00032422,579.024807)(142.01033669,579.05480927)
\curveto(142.01032421,579.07480695)(142.01032421,579.09980693)(142.01033669,579.12980927)
\curveto(142.01032421,579.16980686)(142.0153242,579.19980683)(142.02533669,579.21980927)
\lineto(142.02533669,579.36980927)
}
}
{
\newrgbcolor{curcolor}{0 0 0}
\pscustom[linestyle=none,fillstyle=solid,fillcolor=curcolor]
{
\newpath
\moveto(237.08024086,398.23763153)
\curveto(237.11023314,398.11762732)(237.13523311,397.97762746)(237.15524086,397.81763153)
\curveto(237.17523307,397.65762778)(237.18523306,397.49262794)(237.18524086,397.32263153)
\curveto(237.18523306,397.15262828)(237.17523307,396.98762845)(237.15524086,396.82763153)
\curveto(237.13523311,396.66762877)(237.11023314,396.52762891)(237.08024086,396.40763153)
\curveto(237.04023321,396.26762917)(237.00523324,396.14262929)(236.97524086,396.03263153)
\curveto(236.9452333,395.92262951)(236.90523334,395.81262962)(236.85524086,395.70263153)
\curveto(236.58523366,395.06263037)(236.17023408,394.57763086)(235.61024086,394.24763153)
\curveto(235.53023472,394.18763125)(235.4452348,394.1376313)(235.35524086,394.09763153)
\curveto(235.26523498,394.06763137)(235.16523508,394.0326314)(235.05524086,393.99263153)
\curveto(234.9452353,393.94263149)(234.82523542,393.90763153)(234.69524086,393.88763153)
\curveto(234.57523567,393.85763158)(234.4452358,393.82763161)(234.30524086,393.79763153)
\curveto(234.245236,393.77763166)(234.18523606,393.77263166)(234.12524086,393.78263153)
\curveto(234.07523617,393.79263164)(234.01523623,393.78763165)(233.94524086,393.76763153)
\curveto(233.92523632,393.75763168)(233.90023635,393.75763168)(233.87024086,393.76763153)
\curveto(233.84023641,393.76763167)(233.81523643,393.76263167)(233.79524086,393.75263153)
\lineto(233.64524086,393.75263153)
\curveto(233.57523667,393.74263169)(233.52523672,393.74263169)(233.49524086,393.75263153)
\curveto(233.45523679,393.76263167)(233.41023684,393.76763167)(233.36024086,393.76763153)
\curveto(233.32023693,393.75763168)(233.28023697,393.75763168)(233.24024086,393.76763153)
\curveto(233.1502371,393.78763165)(233.06023719,393.80263163)(232.97024086,393.81263153)
\curveto(232.88023737,393.81263162)(232.79023746,393.82263161)(232.70024086,393.84263153)
\curveto(232.61023764,393.87263156)(232.52023773,393.89763154)(232.43024086,393.91763153)
\curveto(232.34023791,393.9376315)(232.25523799,393.96763147)(232.17524086,394.00763153)
\curveto(231.93523831,394.11763132)(231.71023854,394.24763119)(231.50024086,394.39763153)
\curveto(231.29023896,394.55763088)(231.11023914,394.7376307)(230.96024086,394.93763153)
\curveto(230.84023941,395.10763033)(230.73523951,395.28263015)(230.64524086,395.46263153)
\curveto(230.55523969,395.64262979)(230.46523978,395.8326296)(230.37524086,396.03263153)
\curveto(230.33523991,396.1326293)(230.30023995,396.2326292)(230.27024086,396.33263153)
\curveto(230.25024,396.44262899)(230.22524002,396.55262888)(230.19524086,396.66263153)
\curveto(230.15524009,396.80262863)(230.13024012,396.94262849)(230.12024086,397.08263153)
\curveto(230.11024014,397.22262821)(230.09024016,397.36262807)(230.06024086,397.50263153)
\curveto(230.0502402,397.61262782)(230.04024021,397.71262772)(230.03024086,397.80263153)
\curveto(230.03024022,397.90262753)(230.02024023,398.00262743)(230.00024086,398.10263153)
\lineto(230.00024086,398.19263153)
\curveto(230.01024024,398.22262721)(230.01024024,398.24762719)(230.00024086,398.26763153)
\lineto(230.00024086,398.47763153)
\curveto(229.98024027,398.5376269)(229.97024028,398.60262683)(229.97024086,398.67263153)
\curveto(229.98024027,398.75262668)(229.98524026,398.82762661)(229.98524086,398.89763153)
\lineto(229.98524086,399.04763153)
\curveto(229.98524026,399.09762634)(229.99024026,399.14762629)(230.00024086,399.19763153)
\lineto(230.00024086,399.57263153)
\curveto(230.01024024,399.60262583)(230.01024024,399.6376258)(230.00024086,399.67763153)
\curveto(230.00024025,399.71762572)(230.00524024,399.75762568)(230.01524086,399.79763153)
\curveto(230.03524021,399.90762553)(230.0502402,400.01762542)(230.06024086,400.12763153)
\curveto(230.07024018,400.24762519)(230.08024017,400.36262507)(230.09024086,400.47263153)
\curveto(230.13024012,400.62262481)(230.15524009,400.76762467)(230.16524086,400.90763153)
\curveto(230.18524006,401.05762438)(230.21524003,401.20262423)(230.25524086,401.34263153)
\curveto(230.3452399,401.64262379)(230.44023981,401.92762351)(230.54024086,402.19763153)
\curveto(230.64023961,402.46762297)(230.76523948,402.71762272)(230.91524086,402.94763153)
\curveto(231.11523913,403.26762217)(231.36023889,403.54762189)(231.65024086,403.78763153)
\curveto(231.94023831,404.02762141)(232.28023797,404.21262122)(232.67024086,404.34263153)
\curveto(232.78023747,404.38262105)(232.89023736,404.40762103)(233.00024086,404.41763153)
\curveto(233.12023713,404.437621)(233.24023701,404.46262097)(233.36024086,404.49263153)
\curveto(233.43023682,404.50262093)(233.49523675,404.50762093)(233.55524086,404.50763153)
\curveto(233.61523663,404.50762093)(233.68023657,404.51262092)(233.75024086,404.52263153)
\curveto(234.4502358,404.54262089)(235.02523522,404.42762101)(235.47524086,404.17763153)
\curveto(235.92523432,403.92762151)(236.27023398,403.57762186)(236.51024086,403.12763153)
\curveto(236.62023363,402.89762254)(236.72023353,402.62262281)(236.81024086,402.30263153)
\curveto(236.83023342,402.2326232)(236.83023342,402.15762328)(236.81024086,402.07763153)
\curveto(236.80023345,402.00762343)(236.77523347,401.95762348)(236.73524086,401.92763153)
\curveto(236.70523354,401.89762354)(236.6452336,401.87262356)(236.55524086,401.85263153)
\curveto(236.46523378,401.84262359)(236.36523388,401.8326236)(236.25524086,401.82263153)
\curveto(236.15523409,401.82262361)(236.05523419,401.82762361)(235.95524086,401.83763153)
\curveto(235.86523438,401.84762359)(235.80023445,401.86762357)(235.76024086,401.89763153)
\curveto(235.6502346,401.96762347)(235.57023468,402.07762336)(235.52024086,402.22763153)
\curveto(235.48023477,402.37762306)(235.42523482,402.50762293)(235.35524086,402.61763153)
\curveto(235.16523508,402.92762251)(234.88523536,403.15762228)(234.51524086,403.30763153)
\curveto(234.4452358,403.3376221)(234.37023588,403.35762208)(234.29024086,403.36763153)
\curveto(234.22023603,403.37762206)(234.1452361,403.39262204)(234.06524086,403.41263153)
\curveto(234.01523623,403.42262201)(233.9452363,403.42762201)(233.85524086,403.42763153)
\curveto(233.77523647,403.42762201)(233.71023654,403.42262201)(233.66024086,403.41263153)
\curveto(233.62023663,403.39262204)(233.58523666,403.38762205)(233.55524086,403.39763153)
\curveto(233.52523672,403.40762203)(233.49023676,403.40762203)(233.45024086,403.39763153)
\lineto(233.21024086,403.33763153)
\curveto(233.14023711,403.31762212)(233.07023718,403.29262214)(233.00024086,403.26263153)
\curveto(232.62023763,403.10262233)(232.33023792,402.89262254)(232.13024086,402.63263153)
\curveto(231.94023831,402.37262306)(231.76523848,402.05762338)(231.60524086,401.68763153)
\curveto(231.57523867,401.60762383)(231.5502387,401.52762391)(231.53024086,401.44763153)
\curveto(231.52023873,401.36762407)(231.50023875,401.28762415)(231.47024086,401.20763153)
\curveto(231.44023881,401.09762434)(231.41523883,400.98262445)(231.39524086,400.86263153)
\curveto(231.38523886,400.74262469)(231.36523888,400.62262481)(231.33524086,400.50263153)
\curveto(231.31523893,400.45262498)(231.30523894,400.40262503)(231.30524086,400.35263153)
\curveto(231.31523893,400.30262513)(231.31023894,400.25262518)(231.29024086,400.20263153)
\curveto(231.28023897,400.14262529)(231.28023897,400.06262537)(231.29024086,399.96263153)
\curveto(231.30023895,399.87262556)(231.31523893,399.81762562)(231.33524086,399.79763153)
\curveto(231.35523889,399.75762568)(231.38523886,399.7376257)(231.42524086,399.73763153)
\curveto(231.47523877,399.7376257)(231.52023873,399.74762569)(231.56024086,399.76763153)
\curveto(231.63023862,399.80762563)(231.69023856,399.85262558)(231.74024086,399.90263153)
\curveto(231.79023846,399.95262548)(231.8502384,400.00262543)(231.92024086,400.05263153)
\lineto(231.98024086,400.11263153)
\curveto(232.01023824,400.14262529)(232.04023821,400.16762527)(232.07024086,400.18763153)
\curveto(232.30023795,400.34762509)(232.57523767,400.48262495)(232.89524086,400.59263153)
\curveto(232.96523728,400.61262482)(233.03523721,400.62762481)(233.10524086,400.63763153)
\curveto(233.17523707,400.64762479)(233.250237,400.66262477)(233.33024086,400.68263153)
\curveto(233.37023688,400.68262475)(233.40523684,400.68762475)(233.43524086,400.69763153)
\curveto(233.46523678,400.70762473)(233.50023675,400.70762473)(233.54024086,400.69763153)
\curveto(233.59023666,400.69762474)(233.63023662,400.70762473)(233.66024086,400.72763153)
\lineto(233.82524086,400.72763153)
\lineto(233.91524086,400.72763153)
\curveto(233.96523628,400.7376247)(234.00523624,400.7376247)(234.03524086,400.72763153)
\curveto(234.08523616,400.71762472)(234.13523611,400.71262472)(234.18524086,400.71263153)
\curveto(234.245236,400.72262471)(234.30023595,400.72262471)(234.35024086,400.71263153)
\curveto(234.46023579,400.68262475)(234.56523568,400.66262477)(234.66524086,400.65263153)
\curveto(234.77523547,400.64262479)(234.88023537,400.61762482)(234.98024086,400.57763153)
\curveto(235.40023485,400.437625)(235.7452345,400.25262518)(236.01524086,400.02263153)
\curveto(236.28523396,399.80262563)(236.52523372,399.51762592)(236.73524086,399.16763153)
\curveto(236.81523343,399.02762641)(236.88023337,398.87762656)(236.93024086,398.71763153)
\curveto(236.98023327,398.56762687)(237.03023322,398.40762703)(237.08024086,398.23763153)
\moveto(235.83524086,396.93263153)
\curveto(235.8452344,396.98262845)(235.8502344,397.02762841)(235.85024086,397.06763153)
\lineto(235.85024086,397.21763153)
\curveto(235.8502344,397.52762791)(235.81023444,397.81262762)(235.73024086,398.07263153)
\curveto(235.71023454,398.1326273)(235.69023456,398.18762725)(235.67024086,398.23763153)
\curveto(235.66023459,398.29762714)(235.6452346,398.35262708)(235.62524086,398.40263153)
\curveto(235.40523484,398.89262654)(235.06023519,399.24262619)(234.59024086,399.45263153)
\curveto(234.51023574,399.48262595)(234.43023582,399.50762593)(234.35024086,399.52763153)
\lineto(234.11024086,399.58763153)
\curveto(234.03023622,399.60762583)(233.94023631,399.61762582)(233.84024086,399.61763153)
\lineto(233.52524086,399.61763153)
\curveto(233.50523674,399.59762584)(233.46523678,399.58762585)(233.40524086,399.58763153)
\curveto(233.35523689,399.59762584)(233.31023694,399.59762584)(233.27024086,399.58763153)
\lineto(233.03024086,399.52763153)
\curveto(232.96023729,399.51762592)(232.89023736,399.49762594)(232.82024086,399.46763153)
\curveto(232.22023803,399.20762623)(231.81523843,398.74262669)(231.60524086,398.07263153)
\curveto(231.57523867,397.99262744)(231.55523869,397.91262752)(231.54524086,397.83263153)
\curveto(231.53523871,397.75262768)(231.52023873,397.66762777)(231.50024086,397.57763153)
\lineto(231.50024086,397.42763153)
\curveto(231.49023876,397.38762805)(231.48523876,397.31762812)(231.48524086,397.21763153)
\curveto(231.48523876,396.98762845)(231.50523874,396.79262864)(231.54524086,396.63263153)
\curveto(231.56523868,396.56262887)(231.58023867,396.49762894)(231.59024086,396.43763153)
\curveto(231.60023865,396.37762906)(231.62023863,396.31262912)(231.65024086,396.24263153)
\curveto(231.76023849,395.96262947)(231.90523834,395.71762972)(232.08524086,395.50763153)
\curveto(232.26523798,395.30763013)(232.50023775,395.14763029)(232.79024086,395.02763153)
\lineto(233.03024086,394.93763153)
\lineto(233.27024086,394.87763153)
\curveto(233.32023693,394.85763058)(233.36023689,394.85263058)(233.39024086,394.86263153)
\curveto(233.43023682,394.87263056)(233.47523677,394.86763057)(233.52524086,394.84763153)
\curveto(233.55523669,394.8376306)(233.61023664,394.8326306)(233.69024086,394.83263153)
\curveto(233.77023648,394.8326306)(233.83023642,394.8376306)(233.87024086,394.84763153)
\curveto(233.98023627,394.86763057)(234.08523616,394.88263055)(234.18524086,394.89263153)
\curveto(234.28523596,394.90263053)(234.38023587,394.9326305)(234.47024086,394.98263153)
\curveto(235.00023525,395.18263025)(235.39023486,395.55762988)(235.64024086,396.10763153)
\curveto(235.68023457,396.20762923)(235.71023454,396.31262912)(235.73024086,396.42263153)
\lineto(235.82024086,396.75263153)
\curveto(235.82023443,396.8326286)(235.82523442,396.89262854)(235.83524086,396.93263153)
}
}
{
\newrgbcolor{curcolor}{0 0 0}
\pscustom[linestyle=none,fillstyle=solid,fillcolor=curcolor]
{
\newpath
\moveto(239.93985024,404.32763153)
\lineto(243.53985024,404.32763153)
\lineto(244.18485024,404.32763153)
\curveto(244.26484371,404.32762111)(244.33984363,404.32262111)(244.40985024,404.31263153)
\curveto(244.47984349,404.31262112)(244.53984343,404.30262113)(244.58985024,404.28263153)
\curveto(244.65984331,404.25262118)(244.71484326,404.19262124)(244.75485024,404.10263153)
\curveto(244.7748432,404.07262136)(244.78484319,404.0326214)(244.78485024,403.98263153)
\lineto(244.78485024,403.84763153)
\curveto(244.79484318,403.7376217)(244.78984318,403.6326218)(244.76985024,403.53263153)
\curveto(244.75984321,403.432622)(244.72484325,403.36262207)(244.66485024,403.32263153)
\curveto(244.5748434,403.25262218)(244.43984353,403.21762222)(244.25985024,403.21763153)
\curveto(244.07984389,403.22762221)(243.91484406,403.2326222)(243.76485024,403.23263153)
\lineto(241.76985024,403.23263153)
\lineto(241.27485024,403.23263153)
\lineto(241.13985024,403.23263153)
\curveto(241.09984687,403.2326222)(241.05984691,403.22762221)(241.01985024,403.21763153)
\lineto(240.80985024,403.21763153)
\curveto(240.69984727,403.18762225)(240.61984735,403.14762229)(240.56985024,403.09763153)
\curveto(240.51984745,403.05762238)(240.48484749,403.00262243)(240.46485024,402.93263153)
\curveto(240.44484753,402.87262256)(240.42984754,402.80262263)(240.41985024,402.72263153)
\curveto(240.40984756,402.64262279)(240.38984758,402.55262288)(240.35985024,402.45263153)
\curveto(240.30984766,402.25262318)(240.2698477,402.04762339)(240.23985024,401.83763153)
\curveto(240.20984776,401.62762381)(240.1698478,401.42262401)(240.11985024,401.22263153)
\curveto(240.09984787,401.15262428)(240.08984788,401.08262435)(240.08985024,401.01263153)
\curveto(240.08984788,400.95262448)(240.07984789,400.88762455)(240.05985024,400.81763153)
\curveto(240.04984792,400.78762465)(240.03984793,400.74762469)(240.02985024,400.69763153)
\curveto(240.02984794,400.65762478)(240.03484794,400.61762482)(240.04485024,400.57763153)
\curveto(240.06484791,400.52762491)(240.08984788,400.48262495)(240.11985024,400.44263153)
\curveto(240.15984781,400.41262502)(240.21984775,400.40762503)(240.29985024,400.42763153)
\curveto(240.35984761,400.44762499)(240.41984755,400.47262496)(240.47985024,400.50263153)
\curveto(240.53984743,400.54262489)(240.59984737,400.57762486)(240.65985024,400.60763153)
\curveto(240.71984725,400.62762481)(240.7698472,400.64262479)(240.80985024,400.65263153)
\curveto(240.99984697,400.7326247)(241.20484677,400.78762465)(241.42485024,400.81763153)
\curveto(241.65484632,400.84762459)(241.88484609,400.85762458)(242.11485024,400.84763153)
\curveto(242.35484562,400.84762459)(242.58484539,400.82262461)(242.80485024,400.77263153)
\curveto(243.02484495,400.7326247)(243.22484475,400.67262476)(243.40485024,400.59263153)
\curveto(243.45484452,400.57262486)(243.49984447,400.55262488)(243.53985024,400.53263153)
\curveto(243.58984438,400.51262492)(243.63984433,400.48762495)(243.68985024,400.45763153)
\curveto(244.03984393,400.24762519)(244.31984365,400.01762542)(244.52985024,399.76763153)
\curveto(244.74984322,399.51762592)(244.94484303,399.19262624)(245.11485024,398.79263153)
\curveto(245.16484281,398.68262675)(245.19984277,398.57262686)(245.21985024,398.46263153)
\curveto(245.23984273,398.35262708)(245.26484271,398.2376272)(245.29485024,398.11763153)
\curveto(245.30484267,398.08762735)(245.30984266,398.04262739)(245.30985024,397.98263153)
\curveto(245.32984264,397.92262751)(245.33984263,397.85262758)(245.33985024,397.77263153)
\curveto(245.33984263,397.70262773)(245.34984262,397.6376278)(245.36985024,397.57763153)
\lineto(245.36985024,397.41263153)
\curveto(245.37984259,397.36262807)(245.38484259,397.29262814)(245.38485024,397.20263153)
\curveto(245.38484259,397.11262832)(245.3748426,397.04262839)(245.35485024,396.99263153)
\curveto(245.33484264,396.9326285)(245.32984264,396.87262856)(245.33985024,396.81263153)
\curveto(245.34984262,396.76262867)(245.34484263,396.71262872)(245.32485024,396.66263153)
\curveto(245.28484269,396.50262893)(245.24984272,396.35262908)(245.21985024,396.21263153)
\curveto(245.18984278,396.07262936)(245.14484283,395.9376295)(245.08485024,395.80763153)
\curveto(244.92484305,395.43763)(244.70484327,395.10263033)(244.42485024,394.80263153)
\curveto(244.14484383,394.50263093)(243.82484415,394.27263116)(243.46485024,394.11263153)
\curveto(243.29484468,394.0326314)(243.09484488,393.95763148)(242.86485024,393.88763153)
\curveto(242.75484522,393.84763159)(242.63984533,393.82263161)(242.51985024,393.81263153)
\curveto(242.39984557,393.80263163)(242.27984569,393.78263165)(242.15985024,393.75263153)
\curveto(242.10984586,393.7326317)(242.05484592,393.7326317)(241.99485024,393.75263153)
\curveto(241.93484604,393.76263167)(241.8748461,393.75763168)(241.81485024,393.73763153)
\curveto(241.71484626,393.71763172)(241.61484636,393.71763172)(241.51485024,393.73763153)
\lineto(241.37985024,393.73763153)
\curveto(241.32984664,393.75763168)(241.2698467,393.76763167)(241.19985024,393.76763153)
\curveto(241.13984683,393.75763168)(241.08484689,393.76263167)(241.03485024,393.78263153)
\curveto(240.99484698,393.79263164)(240.95984701,393.79763164)(240.92985024,393.79763153)
\curveto(240.89984707,393.79763164)(240.86484711,393.80263163)(240.82485024,393.81263153)
\lineto(240.55485024,393.87263153)
\curveto(240.46484751,393.89263154)(240.37984759,393.92263151)(240.29985024,393.96263153)
\curveto(239.95984801,394.10263133)(239.6698483,394.25763118)(239.42985024,394.42763153)
\curveto(239.18984878,394.60763083)(238.969849,394.8376306)(238.76985024,395.11763153)
\curveto(238.61984935,395.34763009)(238.50484947,395.58762985)(238.42485024,395.83763153)
\curveto(238.40484957,395.88762955)(238.39484958,395.9326295)(238.39485024,395.97263153)
\curveto(238.39484958,396.02262941)(238.38484959,396.07262936)(238.36485024,396.12263153)
\curveto(238.34484963,396.18262925)(238.32984964,396.26262917)(238.31985024,396.36263153)
\curveto(238.31984965,396.46262897)(238.33984963,396.5376289)(238.37985024,396.58763153)
\curveto(238.42984954,396.66762877)(238.50984946,396.71262872)(238.61985024,396.72263153)
\curveto(238.72984924,396.7326287)(238.84484913,396.7376287)(238.96485024,396.73763153)
\lineto(239.12985024,396.73763153)
\curveto(239.18984878,396.7376287)(239.24484873,396.72762871)(239.29485024,396.70763153)
\curveto(239.38484859,396.68762875)(239.45484852,396.64762879)(239.50485024,396.58763153)
\curveto(239.5748484,396.49762894)(239.61984835,396.38762905)(239.63985024,396.25763153)
\curveto(239.6698483,396.1376293)(239.71484826,396.0326294)(239.77485024,395.94263153)
\curveto(239.96484801,395.60262983)(240.22484775,395.3326301)(240.55485024,395.13263153)
\curveto(240.65484732,395.07263036)(240.75984721,395.02263041)(240.86985024,394.98263153)
\curveto(240.98984698,394.95263048)(241.10984686,394.91763052)(241.22985024,394.87763153)
\curveto(241.39984657,394.82763061)(241.60484637,394.80763063)(241.84485024,394.81763153)
\curveto(242.09484588,394.8376306)(242.29484568,394.87263056)(242.44485024,394.92263153)
\curveto(242.81484516,395.04263039)(243.10484487,395.20263023)(243.31485024,395.40263153)
\curveto(243.53484444,395.61262982)(243.71484426,395.89262954)(243.85485024,396.24263153)
\curveto(243.90484407,396.34262909)(243.93484404,396.44762899)(243.94485024,396.55763153)
\curveto(243.96484401,396.66762877)(243.98984398,396.78262865)(244.01985024,396.90263153)
\lineto(244.01985024,397.00763153)
\curveto(244.02984394,397.04762839)(244.03484394,397.08762835)(244.03485024,397.12763153)
\curveto(244.04484393,397.15762828)(244.04484393,397.19262824)(244.03485024,397.23263153)
\lineto(244.03485024,397.35263153)
\curveto(244.03484394,397.61262782)(244.00484397,397.85762758)(243.94485024,398.08763153)
\curveto(243.83484414,398.437627)(243.67984429,398.7326267)(243.47985024,398.97263153)
\curveto(243.27984469,399.22262621)(243.01984495,399.41762602)(242.69985024,399.55763153)
\lineto(242.51985024,399.61763153)
\curveto(242.4698455,399.6376258)(242.40984556,399.65762578)(242.33985024,399.67763153)
\curveto(242.28984568,399.69762574)(242.22984574,399.70762573)(242.15985024,399.70763153)
\curveto(242.09984587,399.71762572)(242.03484594,399.7326257)(241.96485024,399.75263153)
\lineto(241.81485024,399.75263153)
\curveto(241.7748462,399.77262566)(241.71984625,399.78262565)(241.64985024,399.78263153)
\curveto(241.58984638,399.78262565)(241.53484644,399.77262566)(241.48485024,399.75263153)
\lineto(241.37985024,399.75263153)
\curveto(241.34984662,399.75262568)(241.31484666,399.74762569)(241.27485024,399.73763153)
\lineto(241.03485024,399.67763153)
\curveto(240.95484702,399.66762577)(240.8748471,399.64762579)(240.79485024,399.61763153)
\curveto(240.55484742,399.51762592)(240.32484765,399.38262605)(240.10485024,399.21263153)
\curveto(240.01484796,399.14262629)(239.92984804,399.06762637)(239.84985024,398.98763153)
\curveto(239.7698482,398.91762652)(239.6698483,398.86262657)(239.54985024,398.82263153)
\curveto(239.45984851,398.79262664)(239.31984865,398.78262665)(239.12985024,398.79263153)
\curveto(238.94984902,398.80262663)(238.82984914,398.82762661)(238.76985024,398.86763153)
\curveto(238.71984925,398.90762653)(238.67984929,398.96762647)(238.64985024,399.04763153)
\curveto(238.62984934,399.12762631)(238.62984934,399.21262622)(238.64985024,399.30263153)
\curveto(238.67984929,399.42262601)(238.69984927,399.54262589)(238.70985024,399.66263153)
\curveto(238.72984924,399.79262564)(238.75484922,399.91762552)(238.78485024,400.03763153)
\curveto(238.80484917,400.07762536)(238.80984916,400.11262532)(238.79985024,400.14263153)
\curveto(238.79984917,400.18262525)(238.80984916,400.22762521)(238.82985024,400.27763153)
\curveto(238.84984912,400.36762507)(238.86484911,400.45762498)(238.87485024,400.54763153)
\curveto(238.88484909,400.64762479)(238.90484907,400.74262469)(238.93485024,400.83263153)
\curveto(238.94484903,400.89262454)(238.94984902,400.95262448)(238.94985024,401.01263153)
\curveto(238.95984901,401.07262436)(238.974849,401.1326243)(238.99485024,401.19263153)
\curveto(239.04484893,401.39262404)(239.07984889,401.59762384)(239.09985024,401.80763153)
\curveto(239.12984884,402.02762341)(239.1698488,402.2376232)(239.21985024,402.43763153)
\curveto(239.24984872,402.5376229)(239.2698487,402.6376228)(239.27985024,402.73763153)
\curveto(239.28984868,402.8376226)(239.30484867,402.9376225)(239.32485024,403.03763153)
\curveto(239.33484864,403.06762237)(239.33984863,403.10762233)(239.33985024,403.15763153)
\curveto(239.3698486,403.26762217)(239.38984858,403.37262206)(239.39985024,403.47263153)
\curveto(239.41984855,403.58262185)(239.44484853,403.69262174)(239.47485024,403.80263153)
\curveto(239.49484848,403.88262155)(239.50984846,403.95262148)(239.51985024,404.01263153)
\curveto(239.52984844,404.08262135)(239.55484842,404.14262129)(239.59485024,404.19263153)
\curveto(239.61484836,404.22262121)(239.64484833,404.24262119)(239.68485024,404.25263153)
\curveto(239.72484825,404.27262116)(239.7698482,404.29262114)(239.81985024,404.31263153)
\curveto(239.87984809,404.31262112)(239.91984805,404.31762112)(239.93985024,404.32763153)
}
}
{
\newrgbcolor{curcolor}{0 0 0}
\pscustom[linestyle=none,fillstyle=solid,fillcolor=curcolor]
{
\newpath
\moveto(247.73445961,395.55263153)
\lineto(248.03445961,395.55263153)
\curveto(248.14445755,395.56262987)(248.24945745,395.56262987)(248.34945961,395.55263153)
\curveto(248.45945724,395.55262988)(248.55945714,395.54262989)(248.64945961,395.52263153)
\curveto(248.73945696,395.51262992)(248.80945689,395.48762995)(248.85945961,395.44763153)
\curveto(248.87945682,395.42763001)(248.8944568,395.39763004)(248.90445961,395.35763153)
\curveto(248.92445677,395.31763012)(248.94445675,395.27263016)(248.96445961,395.22263153)
\lineto(248.96445961,395.14763153)
\curveto(248.97445672,395.09763034)(248.97445672,395.04263039)(248.96445961,394.98263153)
\lineto(248.96445961,394.83263153)
\lineto(248.96445961,394.35263153)
\curveto(248.96445673,394.18263125)(248.92445677,394.06263137)(248.84445961,393.99263153)
\curveto(248.77445692,393.94263149)(248.68445701,393.91763152)(248.57445961,393.91763153)
\lineto(248.24445961,393.91763153)
\lineto(247.79445961,393.91763153)
\curveto(247.64445805,393.91763152)(247.52945817,393.94763149)(247.44945961,394.00763153)
\curveto(247.40945829,394.0376314)(247.37945832,394.08763135)(247.35945961,394.15763153)
\curveto(247.33945836,394.2376312)(247.32445837,394.32263111)(247.31445961,394.41263153)
\lineto(247.31445961,394.69763153)
\curveto(247.32445837,394.79763064)(247.32945837,394.88263055)(247.32945961,394.95263153)
\lineto(247.32945961,395.14763153)
\curveto(247.32945837,395.20763023)(247.33945836,395.26263017)(247.35945961,395.31263153)
\curveto(247.3994583,395.42263001)(247.46945823,395.49262994)(247.56945961,395.52263153)
\curveto(247.5994581,395.52262991)(247.65445804,395.5326299)(247.73445961,395.55263153)
}
}
{
\newrgbcolor{curcolor}{0 0 0}
\pscustom[linestyle=none,fillstyle=solid,fillcolor=curcolor]
{
\newpath
\moveto(254.05461586,404.52263153)
\curveto(255.68461042,404.55262088)(256.73460937,403.99762144)(257.20461586,402.85763153)
\curveto(257.3046088,402.62762281)(257.36960874,402.3376231)(257.39961586,401.98763153)
\curveto(257.43960867,401.64762379)(257.41460869,401.3376241)(257.32461586,401.05763153)
\curveto(257.23460887,400.79762464)(257.11460899,400.57262486)(256.96461586,400.38263153)
\curveto(256.94460916,400.34262509)(256.91960919,400.30762513)(256.88961586,400.27763153)
\curveto(256.85960925,400.25762518)(256.83460927,400.2326252)(256.81461586,400.20263153)
\lineto(256.72461586,400.08263153)
\curveto(256.69460941,400.05262538)(256.65960945,400.02762541)(256.61961586,400.00763153)
\curveto(256.56960954,399.95762548)(256.51460959,399.91262552)(256.45461586,399.87263153)
\curveto(256.4046097,399.8326256)(256.35960975,399.78262565)(256.31961586,399.72263153)
\curveto(256.27960983,399.68262575)(256.26460984,399.6326258)(256.27461586,399.57263153)
\curveto(256.28460982,399.52262591)(256.31460979,399.47762596)(256.36461586,399.43763153)
\curveto(256.41460969,399.39762604)(256.46960964,399.35762608)(256.52961586,399.31763153)
\curveto(256.59960951,399.28762615)(256.66460944,399.25762618)(256.72461586,399.22763153)
\curveto(256.78460932,399.19762624)(256.83460927,399.16762627)(256.87461586,399.13763153)
\curveto(257.19460891,398.91762652)(257.44960866,398.60762683)(257.63961586,398.20763153)
\curveto(257.67960843,398.11762732)(257.7096084,398.02262741)(257.72961586,397.92263153)
\curveto(257.75960835,397.8326276)(257.78460832,397.74262769)(257.80461586,397.65263153)
\curveto(257.81460829,397.60262783)(257.81960829,397.55262788)(257.81961586,397.50263153)
\curveto(257.82960828,397.46262797)(257.83960827,397.41762802)(257.84961586,397.36763153)
\curveto(257.85960825,397.31762812)(257.85960825,397.26762817)(257.84961586,397.21763153)
\curveto(257.83960827,397.16762827)(257.84460826,397.11762832)(257.86461586,397.06763153)
\curveto(257.87460823,397.01762842)(257.87960823,396.95762848)(257.87961586,396.88763153)
\curveto(257.87960823,396.81762862)(257.86960824,396.75762868)(257.84961586,396.70763153)
\lineto(257.84961586,396.48263153)
\lineto(257.78961586,396.24263153)
\curveto(257.77960833,396.17262926)(257.76460834,396.10262933)(257.74461586,396.03263153)
\curveto(257.71460839,395.94262949)(257.68460842,395.85762958)(257.65461586,395.77763153)
\curveto(257.63460847,395.69762974)(257.6046085,395.61762982)(257.56461586,395.53763153)
\curveto(257.54460856,395.47762996)(257.51460859,395.41763002)(257.47461586,395.35763153)
\curveto(257.44460866,395.30763013)(257.4096087,395.25763018)(257.36961586,395.20763153)
\curveto(257.16960894,394.89763054)(256.91960919,394.6376308)(256.61961586,394.42763153)
\curveto(256.31960979,394.22763121)(255.97461013,394.06263137)(255.58461586,393.93263153)
\curveto(255.46461064,393.89263154)(255.33461077,393.86763157)(255.19461586,393.85763153)
\curveto(255.06461104,393.8376316)(254.92961118,393.81263162)(254.78961586,393.78263153)
\curveto(254.71961139,393.77263166)(254.64961146,393.76763167)(254.57961586,393.76763153)
\curveto(254.51961159,393.76763167)(254.45461165,393.76263167)(254.38461586,393.75263153)
\curveto(254.34461176,393.74263169)(254.28461182,393.7376317)(254.20461586,393.73763153)
\curveto(254.13461197,393.7376317)(254.08461202,393.74263169)(254.05461586,393.75263153)
\curveto(254.0046121,393.76263167)(253.95961215,393.76763167)(253.91961586,393.76763153)
\lineto(253.79961586,393.76763153)
\curveto(253.69961241,393.78763165)(253.59961251,393.80263163)(253.49961586,393.81263153)
\curveto(253.39961271,393.82263161)(253.3046128,393.8376316)(253.21461586,393.85763153)
\curveto(253.104613,393.88763155)(252.99461311,393.91263152)(252.88461586,393.93263153)
\curveto(252.78461332,393.96263147)(252.67961343,394.00263143)(252.56961586,394.05263153)
\curveto(252.19961391,394.21263122)(251.88461422,394.41263102)(251.62461586,394.65263153)
\curveto(251.36461474,394.90263053)(251.15461495,395.21263022)(250.99461586,395.58263153)
\curveto(250.95461515,395.67262976)(250.91961519,395.76762967)(250.88961586,395.86763153)
\curveto(250.85961525,395.96762947)(250.82961528,396.07262936)(250.79961586,396.18263153)
\curveto(250.77961533,396.2326292)(250.76961534,396.28262915)(250.76961586,396.33263153)
\curveto(250.76961534,396.39262904)(250.75961535,396.45262898)(250.73961586,396.51263153)
\curveto(250.71961539,396.57262886)(250.7096154,396.65262878)(250.70961586,396.75263153)
\curveto(250.7096154,396.85262858)(250.72461538,396.92762851)(250.75461586,396.97763153)
\curveto(250.76461534,397.00762843)(250.77961533,397.0326284)(250.79961586,397.05263153)
\lineto(250.85961586,397.11263153)
\curveto(250.89961521,397.1326283)(250.95961515,397.14762829)(251.03961586,397.15763153)
\curveto(251.12961498,397.16762827)(251.21961489,397.17262826)(251.30961586,397.17263153)
\curveto(251.39961471,397.17262826)(251.48461462,397.16762827)(251.56461586,397.15763153)
\curveto(251.65461445,397.14762829)(251.71961439,397.1376283)(251.75961586,397.12763153)
\curveto(251.77961433,397.10762833)(251.79961431,397.09262834)(251.81961586,397.08263153)
\curveto(251.83961427,397.08262835)(251.85961425,397.07262836)(251.87961586,397.05263153)
\curveto(251.94961416,396.96262847)(251.98961412,396.84762859)(251.99961586,396.70763153)
\curveto(252.01961409,396.56762887)(252.04961406,396.44262899)(252.08961586,396.33263153)
\lineto(252.23961586,395.97263153)
\curveto(252.28961382,395.86262957)(252.35461375,395.75762968)(252.43461586,395.65763153)
\curveto(252.45461365,395.62762981)(252.47461363,395.60262983)(252.49461586,395.58263153)
\curveto(252.52461358,395.56262987)(252.54961356,395.5376299)(252.56961586,395.50763153)
\curveto(252.6096135,395.44762999)(252.64461346,395.40263003)(252.67461586,395.37263153)
\curveto(252.71461339,395.34263009)(252.74961336,395.31263012)(252.77961586,395.28263153)
\curveto(252.81961329,395.25263018)(252.86461324,395.22263021)(252.91461586,395.19263153)
\curveto(253.0046131,395.1326303)(253.09961301,395.08263035)(253.19961586,395.04263153)
\lineto(253.52961586,394.92263153)
\curveto(253.67961243,394.87263056)(253.87961223,394.84263059)(254.12961586,394.83263153)
\curveto(254.37961173,394.82263061)(254.58961152,394.84263059)(254.75961586,394.89263153)
\curveto(254.83961127,394.91263052)(254.9096112,394.92763051)(254.96961586,394.93763153)
\lineto(255.17961586,394.99763153)
\curveto(255.45961065,395.11763032)(255.69961041,395.26763017)(255.89961586,395.44763153)
\curveto(256.10961,395.62762981)(256.27460983,395.85762958)(256.39461586,396.13763153)
\curveto(256.42460968,396.20762923)(256.44460966,396.27762916)(256.45461586,396.34763153)
\lineto(256.51461586,396.58763153)
\curveto(256.55460955,396.72762871)(256.56460954,396.88762855)(256.54461586,397.06763153)
\curveto(256.52460958,397.25762818)(256.49460961,397.40762803)(256.45461586,397.51763153)
\curveto(256.32460978,397.89762754)(256.13960997,398.18762725)(255.89961586,398.38763153)
\curveto(255.66961044,398.58762685)(255.35961075,398.74762669)(254.96961586,398.86763153)
\curveto(254.85961125,398.89762654)(254.73961137,398.91762652)(254.60961586,398.92763153)
\curveto(254.48961162,398.9376265)(254.36461174,398.94262649)(254.23461586,398.94263153)
\curveto(254.07461203,398.94262649)(253.93461217,398.94762649)(253.81461586,398.95763153)
\curveto(253.69461241,398.96762647)(253.6096125,399.02762641)(253.55961586,399.13763153)
\curveto(253.53961257,399.16762627)(253.52961258,399.20262623)(253.52961586,399.24263153)
\lineto(253.52961586,399.37763153)
\curveto(253.51961259,399.47762596)(253.51961259,399.57262586)(253.52961586,399.66263153)
\curveto(253.54961256,399.75262568)(253.58961252,399.81762562)(253.64961586,399.85763153)
\curveto(253.68961242,399.88762555)(253.72961238,399.90762553)(253.76961586,399.91763153)
\curveto(253.81961229,399.92762551)(253.87461223,399.9376255)(253.93461586,399.94763153)
\curveto(253.95461215,399.95762548)(253.97961213,399.95762548)(254.00961586,399.94763153)
\curveto(254.03961207,399.94762549)(254.06461204,399.95262548)(254.08461586,399.96263153)
\lineto(254.21961586,399.96263153)
\curveto(254.32961178,399.98262545)(254.42961168,399.99262544)(254.51961586,399.99263153)
\curveto(254.61961149,400.00262543)(254.71461139,400.02262541)(254.80461586,400.05263153)
\curveto(255.12461098,400.16262527)(255.37961073,400.30762513)(255.56961586,400.48763153)
\curveto(255.75961035,400.66762477)(255.9096102,400.91762452)(256.01961586,401.23763153)
\curveto(256.04961006,401.3376241)(256.06961004,401.46262397)(256.07961586,401.61263153)
\curveto(256.09961001,401.77262366)(256.09461001,401.91762352)(256.06461586,402.04763153)
\curveto(256.04461006,402.11762332)(256.02461008,402.18262325)(256.00461586,402.24263153)
\curveto(255.99461011,402.31262312)(255.97461013,402.37762306)(255.94461586,402.43763153)
\curveto(255.84461026,402.67762276)(255.69961041,402.86762257)(255.50961586,403.00763153)
\curveto(255.31961079,403.14762229)(255.09461101,403.25762218)(254.83461586,403.33763153)
\curveto(254.77461133,403.35762208)(254.71461139,403.36762207)(254.65461586,403.36763153)
\curveto(254.59461151,403.36762207)(254.52961158,403.37762206)(254.45961586,403.39763153)
\curveto(254.37961173,403.41762202)(254.28461182,403.42762201)(254.17461586,403.42763153)
\curveto(254.06461204,403.42762201)(253.96961214,403.41762202)(253.88961586,403.39763153)
\curveto(253.83961227,403.37762206)(253.78961232,403.36762207)(253.73961586,403.36763153)
\curveto(253.69961241,403.36762207)(253.65461245,403.35762208)(253.60461586,403.33763153)
\curveto(253.42461268,403.28762215)(253.25461285,403.21262222)(253.09461586,403.11263153)
\curveto(252.94461316,403.02262241)(252.81461329,402.90762253)(252.70461586,402.76763153)
\curveto(252.61461349,402.64762279)(252.53461357,402.51762292)(252.46461586,402.37763153)
\curveto(252.39461371,402.2376232)(252.32961378,402.08262335)(252.26961586,401.91263153)
\curveto(252.23961387,401.80262363)(252.21961389,401.68262375)(252.20961586,401.55263153)
\curveto(252.19961391,401.432624)(252.16461394,401.3326241)(252.10461586,401.25263153)
\curveto(252.08461402,401.21262422)(252.02461408,401.17262426)(251.92461586,401.13263153)
\curveto(251.88461422,401.12262431)(251.82461428,401.11262432)(251.74461586,401.10263153)
\lineto(251.48961586,401.10263153)
\curveto(251.39961471,401.11262432)(251.31461479,401.12262431)(251.23461586,401.13263153)
\curveto(251.16461494,401.14262429)(251.11461499,401.15762428)(251.08461586,401.17763153)
\curveto(251.04461506,401.20762423)(251.0096151,401.26262417)(250.97961586,401.34263153)
\curveto(250.94961516,401.42262401)(250.94461516,401.50762393)(250.96461586,401.59763153)
\curveto(250.97461513,401.64762379)(250.97961513,401.69762374)(250.97961586,401.74763153)
\lineto(251.00961586,401.92763153)
\curveto(251.03961507,402.02762341)(251.06461504,402.12762331)(251.08461586,402.22763153)
\curveto(251.11461499,402.32762311)(251.14961496,402.41762302)(251.18961586,402.49763153)
\curveto(251.23961487,402.60762283)(251.28461482,402.71262272)(251.32461586,402.81263153)
\curveto(251.36461474,402.92262251)(251.41461469,403.02762241)(251.47461586,403.12763153)
\curveto(251.8046143,403.66762177)(252.27461383,404.06262137)(252.88461586,404.31263153)
\curveto(253.0046131,404.36262107)(253.12961298,404.39762104)(253.25961586,404.41763153)
\curveto(253.39961271,404.437621)(253.53961257,404.46262097)(253.67961586,404.49263153)
\curveto(253.73961237,404.50262093)(253.79961231,404.50762093)(253.85961586,404.50763153)
\curveto(253.92961218,404.50762093)(253.99461211,404.51262092)(254.05461586,404.52263153)
}
}
{
\newrgbcolor{curcolor}{0 0 0}
\pscustom[linestyle=none,fillstyle=solid,fillcolor=curcolor]
{
\newpath
\moveto(269.09422524,402.43763153)
\curveto(268.89421494,402.14762329)(268.68421515,401.86262357)(268.46422524,401.58263153)
\curveto(268.25421558,401.30262413)(268.04921578,401.01762442)(267.84922524,400.72763153)
\curveto(267.24921658,399.87762556)(266.64421719,399.0376264)(266.03422524,398.20763153)
\curveto(265.42421841,397.38762805)(264.81921901,396.55262888)(264.21922524,395.70263153)
\lineto(263.70922524,394.98263153)
\lineto(263.19922524,394.29263153)
\curveto(263.11922071,394.18263125)(263.03922079,394.06763137)(262.95922524,393.94763153)
\curveto(262.87922095,393.82763161)(262.78422105,393.7326317)(262.67422524,393.66263153)
\curveto(262.6342212,393.64263179)(262.56922126,393.62763181)(262.47922524,393.61763153)
\curveto(262.39922143,393.59763184)(262.30922152,393.58763185)(262.20922524,393.58763153)
\curveto(262.10922172,393.58763185)(262.01422182,393.59263184)(261.92422524,393.60263153)
\curveto(261.84422199,393.61263182)(261.78422205,393.6326318)(261.74422524,393.66263153)
\curveto(261.71422212,393.68263175)(261.68922214,393.71763172)(261.66922524,393.76763153)
\curveto(261.65922217,393.80763163)(261.66422217,393.85263158)(261.68422524,393.90263153)
\curveto(261.72422211,393.98263145)(261.76922206,394.05763138)(261.81922524,394.12763153)
\curveto(261.87922195,394.20763123)(261.9342219,394.28763115)(261.98422524,394.36763153)
\curveto(262.22422161,394.70763073)(262.46922136,395.04263039)(262.71922524,395.37263153)
\curveto(262.96922086,395.70262973)(263.20922062,396.0376294)(263.43922524,396.37763153)
\curveto(263.59922023,396.59762884)(263.75922007,396.81262862)(263.91922524,397.02263153)
\curveto(264.07921975,397.2326282)(264.23921959,397.44762799)(264.39922524,397.66763153)
\curveto(264.75921907,398.18762725)(265.12421871,398.69762674)(265.49422524,399.19763153)
\curveto(265.86421797,399.69762574)(266.2342176,400.20762523)(266.60422524,400.72763153)
\curveto(266.74421709,400.92762451)(266.88421695,401.12262431)(267.02422524,401.31263153)
\curveto(267.17421666,401.50262393)(267.31921651,401.69762374)(267.45922524,401.89763153)
\curveto(267.66921616,402.19762324)(267.88421595,402.49762294)(268.10422524,402.79763153)
\lineto(268.76422524,403.69763153)
\lineto(268.94422524,403.96763153)
\lineto(269.15422524,404.23763153)
\lineto(269.27422524,404.41763153)
\curveto(269.32421451,404.47762096)(269.37421446,404.5326209)(269.42422524,404.58263153)
\curveto(269.49421434,404.6326208)(269.56921426,404.66762077)(269.64922524,404.68763153)
\curveto(269.66921416,404.69762074)(269.69421414,404.69762074)(269.72422524,404.68763153)
\curveto(269.76421407,404.68762075)(269.79421404,404.69762074)(269.81422524,404.71763153)
\curveto(269.9342139,404.71762072)(270.06921376,404.71262072)(270.21922524,404.70263153)
\curveto(270.36921346,404.70262073)(270.45921337,404.65762078)(270.48922524,404.56763153)
\curveto(270.50921332,404.5376209)(270.51421332,404.50262093)(270.50422524,404.46263153)
\curveto(270.49421334,404.42262101)(270.47921335,404.39262104)(270.45922524,404.37263153)
\curveto(270.41921341,404.29262114)(270.37921345,404.22262121)(270.33922524,404.16263153)
\curveto(270.29921353,404.10262133)(270.25421358,404.04262139)(270.20422524,403.98263153)
\lineto(269.63422524,403.20263153)
\curveto(269.45421438,402.95262248)(269.27421456,402.69762274)(269.09422524,402.43763153)
\moveto(262.23922524,398.53763153)
\curveto(262.18922164,398.55762688)(262.13922169,398.56262687)(262.08922524,398.55263153)
\curveto(262.03922179,398.54262689)(261.98922184,398.54762689)(261.93922524,398.56763153)
\curveto(261.829222,398.58762685)(261.72422211,398.60762683)(261.62422524,398.62763153)
\curveto(261.5342223,398.65762678)(261.43922239,398.69762674)(261.33922524,398.74763153)
\curveto(261.00922282,398.88762655)(260.75422308,399.08262635)(260.57422524,399.33263153)
\curveto(260.39422344,399.59262584)(260.24922358,399.90262553)(260.13922524,400.26263153)
\curveto(260.10922372,400.34262509)(260.08922374,400.42262501)(260.07922524,400.50263153)
\curveto(260.06922376,400.59262484)(260.05422378,400.67762476)(260.03422524,400.75763153)
\curveto(260.02422381,400.80762463)(260.01922381,400.87262456)(260.01922524,400.95263153)
\curveto(260.00922382,400.98262445)(260.00422383,401.01262442)(260.00422524,401.04263153)
\curveto(260.00422383,401.08262435)(259.99922383,401.11762432)(259.98922524,401.14763153)
\lineto(259.98922524,401.29763153)
\curveto(259.97922385,401.34762409)(259.97422386,401.40762403)(259.97422524,401.47763153)
\curveto(259.97422386,401.55762388)(259.97922385,401.62262381)(259.98922524,401.67263153)
\lineto(259.98922524,401.83763153)
\curveto(260.00922382,401.88762355)(260.01422382,401.9326235)(260.00422524,401.97263153)
\curveto(260.00422383,402.02262341)(260.00922382,402.06762337)(260.01922524,402.10763153)
\curveto(260.0292238,402.14762329)(260.0342238,402.18262325)(260.03422524,402.21263153)
\curveto(260.0342238,402.25262318)(260.03922379,402.29262314)(260.04922524,402.33263153)
\curveto(260.07922375,402.44262299)(260.09922373,402.55262288)(260.10922524,402.66263153)
\curveto(260.1292237,402.78262265)(260.16422367,402.89762254)(260.21422524,403.00763153)
\curveto(260.35422348,403.34762209)(260.51422332,403.62262181)(260.69422524,403.83263153)
\curveto(260.88422295,404.05262138)(261.15422268,404.2326212)(261.50422524,404.37263153)
\curveto(261.58422225,404.40262103)(261.66922216,404.42262101)(261.75922524,404.43263153)
\curveto(261.84922198,404.45262098)(261.94422189,404.47262096)(262.04422524,404.49263153)
\curveto(262.07422176,404.50262093)(262.1292217,404.50262093)(262.20922524,404.49263153)
\curveto(262.28922154,404.49262094)(262.33922149,404.50262093)(262.35922524,404.52263153)
\curveto(262.91922091,404.5326209)(263.36922046,404.42262101)(263.70922524,404.19263153)
\curveto(264.05921977,403.96262147)(264.31921951,403.65762178)(264.48922524,403.27763153)
\curveto(264.5292193,403.18762225)(264.56421927,403.09262234)(264.59422524,402.99263153)
\curveto(264.62421921,402.89262254)(264.64921918,402.79262264)(264.66922524,402.69263153)
\curveto(264.68921914,402.66262277)(264.69421914,402.6326228)(264.68422524,402.60263153)
\curveto(264.68421915,402.57262286)(264.68921914,402.54262289)(264.69922524,402.51263153)
\curveto(264.7292191,402.40262303)(264.74921908,402.27762316)(264.75922524,402.13763153)
\curveto(264.76921906,402.00762343)(264.77921905,401.87262356)(264.78922524,401.73263153)
\lineto(264.78922524,401.56763153)
\curveto(264.79921903,401.50762393)(264.79921903,401.45262398)(264.78922524,401.40263153)
\curveto(264.77921905,401.35262408)(264.77421906,401.30262413)(264.77422524,401.25263153)
\lineto(264.77422524,401.11763153)
\curveto(264.76421907,401.07762436)(264.75921907,401.0376244)(264.75922524,400.99763153)
\curveto(264.76921906,400.95762448)(264.76421907,400.91262452)(264.74422524,400.86263153)
\curveto(264.72421911,400.75262468)(264.70421913,400.64762479)(264.68422524,400.54763153)
\curveto(264.67421916,400.44762499)(264.65421918,400.34762509)(264.62422524,400.24763153)
\curveto(264.49421934,399.88762555)(264.3292195,399.57262586)(264.12922524,399.30263153)
\curveto(263.9292199,399.0326264)(263.65422018,398.82762661)(263.30422524,398.68763153)
\curveto(263.22422061,398.65762678)(263.13922069,398.6326268)(263.04922524,398.61263153)
\lineto(262.77922524,398.55263153)
\curveto(262.7292211,398.54262689)(262.68422115,398.5376269)(262.64422524,398.53763153)
\curveto(262.60422123,398.54762689)(262.56422127,398.54762689)(262.52422524,398.53763153)
\curveto(262.42422141,398.51762692)(262.3292215,398.51762692)(262.23922524,398.53763153)
\moveto(261.39922524,399.93263153)
\curveto(261.43922239,399.86262557)(261.47922235,399.79762564)(261.51922524,399.73763153)
\curveto(261.55922227,399.68762575)(261.60922222,399.6376258)(261.66922524,399.58763153)
\lineto(261.81922524,399.46763153)
\curveto(261.87922195,399.437626)(261.94422189,399.41262602)(262.01422524,399.39263153)
\curveto(262.05422178,399.37262606)(262.08922174,399.36262607)(262.11922524,399.36263153)
\curveto(262.15922167,399.37262606)(262.19922163,399.36762607)(262.23922524,399.34763153)
\curveto(262.26922156,399.34762609)(262.30922152,399.34262609)(262.35922524,399.33263153)
\curveto(262.40922142,399.3326261)(262.44922138,399.3376261)(262.47922524,399.34763153)
\lineto(262.70422524,399.39263153)
\curveto(262.95422088,399.47262596)(263.13922069,399.59762584)(263.25922524,399.76763153)
\curveto(263.33922049,399.86762557)(263.40922042,399.99762544)(263.46922524,400.15763153)
\curveto(263.54922028,400.3376251)(263.60922022,400.56262487)(263.64922524,400.83263153)
\curveto(263.68922014,401.11262432)(263.70422013,401.39262404)(263.69422524,401.67263153)
\curveto(263.68422015,401.96262347)(263.65422018,402.2376232)(263.60422524,402.49763153)
\curveto(263.55422028,402.75762268)(263.47922035,402.96762247)(263.37922524,403.12763153)
\curveto(263.25922057,403.32762211)(263.10922072,403.47762196)(262.92922524,403.57763153)
\curveto(262.84922098,403.62762181)(262.75922107,403.65762178)(262.65922524,403.66763153)
\curveto(262.55922127,403.68762175)(262.45422138,403.69762174)(262.34422524,403.69763153)
\curveto(262.32422151,403.68762175)(262.29922153,403.68262175)(262.26922524,403.68263153)
\curveto(262.24922158,403.69262174)(262.2292216,403.69262174)(262.20922524,403.68263153)
\curveto(262.15922167,403.67262176)(262.11422172,403.66262177)(262.07422524,403.65263153)
\curveto(262.0342218,403.65262178)(261.99422184,403.64262179)(261.95422524,403.62263153)
\curveto(261.77422206,403.54262189)(261.62422221,403.42262201)(261.50422524,403.26263153)
\curveto(261.39422244,403.10262233)(261.30422253,402.92262251)(261.23422524,402.72263153)
\curveto(261.17422266,402.5326229)(261.1292227,402.30762313)(261.09922524,402.04763153)
\curveto(261.07922275,401.78762365)(261.07422276,401.52262391)(261.08422524,401.25263153)
\curveto(261.09422274,400.99262444)(261.12422271,400.74262469)(261.17422524,400.50263153)
\curveto(261.2342226,400.27262516)(261.30922252,400.08262535)(261.39922524,399.93263153)
\moveto(272.19922524,396.94763153)
\curveto(272.20921162,396.89762854)(272.21421162,396.80762863)(272.21422524,396.67763153)
\curveto(272.21421162,396.54762889)(272.20421163,396.45762898)(272.18422524,396.40763153)
\curveto(272.16421167,396.35762908)(272.15921167,396.30262913)(272.16922524,396.24263153)
\curveto(272.17921165,396.19262924)(272.17921165,396.14262929)(272.16922524,396.09263153)
\curveto(272.1292117,395.95262948)(272.09921173,395.81762962)(272.07922524,395.68763153)
\curveto(272.06921176,395.55762988)(272.03921179,395.43763)(271.98922524,395.32763153)
\curveto(271.84921198,394.97763046)(271.68421215,394.68263075)(271.49422524,394.44263153)
\curveto(271.30421253,394.21263122)(271.0342128,394.02763141)(270.68422524,393.88763153)
\curveto(270.60421323,393.85763158)(270.51921331,393.8376316)(270.42922524,393.82763153)
\curveto(270.33921349,393.80763163)(270.25421358,393.78763165)(270.17422524,393.76763153)
\curveto(270.12421371,393.75763168)(270.07421376,393.75263168)(270.02422524,393.75263153)
\curveto(269.97421386,393.75263168)(269.92421391,393.74763169)(269.87422524,393.73763153)
\curveto(269.84421399,393.72763171)(269.79421404,393.72763171)(269.72422524,393.73763153)
\curveto(269.65421418,393.7376317)(269.60421423,393.74263169)(269.57422524,393.75263153)
\curveto(269.51421432,393.77263166)(269.45421438,393.78263165)(269.39422524,393.78263153)
\curveto(269.34421449,393.77263166)(269.29421454,393.77763166)(269.24422524,393.79763153)
\curveto(269.15421468,393.81763162)(269.06421477,393.84263159)(268.97422524,393.87263153)
\curveto(268.89421494,393.89263154)(268.81421502,393.92263151)(268.73422524,393.96263153)
\curveto(268.41421542,394.10263133)(268.16421567,394.29763114)(267.98422524,394.54763153)
\curveto(267.80421603,394.80763063)(267.65421618,395.11263032)(267.53422524,395.46263153)
\curveto(267.51421632,395.54262989)(267.49921633,395.62762981)(267.48922524,395.71763153)
\curveto(267.47921635,395.80762963)(267.46421637,395.89262954)(267.44422524,395.97263153)
\curveto(267.4342164,396.00262943)(267.4292164,396.0326294)(267.42922524,396.06263153)
\lineto(267.42922524,396.16763153)
\curveto(267.40921642,396.24762919)(267.39921643,396.32762911)(267.39922524,396.40763153)
\lineto(267.39922524,396.54263153)
\curveto(267.37921645,396.64262879)(267.37921645,396.74262869)(267.39922524,396.84263153)
\lineto(267.39922524,397.02263153)
\curveto(267.40921642,397.07262836)(267.41421642,397.11762832)(267.41422524,397.15763153)
\curveto(267.41421642,397.20762823)(267.41921641,397.25262818)(267.42922524,397.29263153)
\curveto(267.43921639,397.3326281)(267.44421639,397.36762807)(267.44422524,397.39763153)
\curveto(267.44421639,397.437628)(267.44921638,397.47762796)(267.45922524,397.51763153)
\lineto(267.51922524,397.84763153)
\curveto(267.53921629,397.96762747)(267.56921626,398.07762736)(267.60922524,398.17763153)
\curveto(267.74921608,398.50762693)(267.90921592,398.78262665)(268.08922524,399.00263153)
\curveto(268.27921555,399.2326262)(268.53921529,399.41762602)(268.86922524,399.55763153)
\curveto(268.94921488,399.59762584)(269.0342148,399.62262581)(269.12422524,399.63263153)
\lineto(269.42422524,399.69263153)
\lineto(269.55922524,399.69263153)
\curveto(269.60921422,399.70262573)(269.65921417,399.70762573)(269.70922524,399.70763153)
\curveto(270.27921355,399.72762571)(270.73921309,399.62262581)(271.08922524,399.39263153)
\curveto(271.44921238,399.17262626)(271.71421212,398.87262656)(271.88422524,398.49263153)
\curveto(271.9342119,398.39262704)(271.97421186,398.29262714)(272.00422524,398.19263153)
\curveto(272.0342118,398.09262734)(272.06421177,397.98762745)(272.09422524,397.87763153)
\curveto(272.10421173,397.8376276)(272.10921172,397.80262763)(272.10922524,397.77263153)
\curveto(272.10921172,397.75262768)(272.11421172,397.72262771)(272.12422524,397.68263153)
\curveto(272.14421169,397.61262782)(272.15421168,397.5376279)(272.15422524,397.45763153)
\curveto(272.15421168,397.37762806)(272.16421167,397.29762814)(272.18422524,397.21763153)
\curveto(272.18421165,397.16762827)(272.18421165,397.12262831)(272.18422524,397.08263153)
\curveto(272.18421165,397.04262839)(272.18921164,396.99762844)(272.19922524,396.94763153)
\moveto(271.08922524,396.51263153)
\curveto(271.09921273,396.56262887)(271.10421273,396.6376288)(271.10422524,396.73763153)
\curveto(271.11421272,396.8376286)(271.10921272,396.91262852)(271.08922524,396.96263153)
\curveto(271.06921276,397.02262841)(271.06421277,397.07762836)(271.07422524,397.12763153)
\curveto(271.09421274,397.18762825)(271.09421274,397.24762819)(271.07422524,397.30763153)
\curveto(271.06421277,397.3376281)(271.05921277,397.37262806)(271.05922524,397.41263153)
\curveto(271.05921277,397.45262798)(271.05421278,397.49262794)(271.04422524,397.53263153)
\curveto(271.02421281,397.61262782)(271.00421283,397.68762775)(270.98422524,397.75763153)
\curveto(270.97421286,397.8376276)(270.95921287,397.91762752)(270.93922524,397.99763153)
\curveto(270.90921292,398.05762738)(270.88421295,398.11762732)(270.86422524,398.17763153)
\curveto(270.84421299,398.2376272)(270.81421302,398.29762714)(270.77422524,398.35763153)
\curveto(270.67421316,398.52762691)(270.54421329,398.66262677)(270.38422524,398.76263153)
\curveto(270.30421353,398.81262662)(270.20921362,398.84762659)(270.09922524,398.86763153)
\curveto(269.98921384,398.88762655)(269.86421397,398.89762654)(269.72422524,398.89763153)
\curveto(269.70421413,398.88762655)(269.67921415,398.88262655)(269.64922524,398.88263153)
\curveto(269.61921421,398.89262654)(269.58921424,398.89262654)(269.55922524,398.88263153)
\lineto(269.40922524,398.82263153)
\curveto(269.35921447,398.81262662)(269.31421452,398.79762664)(269.27422524,398.77763153)
\curveto(269.08421475,398.66762677)(268.93921489,398.52262691)(268.83922524,398.34263153)
\curveto(268.74921508,398.16262727)(268.66921516,397.95762748)(268.59922524,397.72763153)
\curveto(268.55921527,397.59762784)(268.53921529,397.46262797)(268.53922524,397.32263153)
\curveto(268.53921529,397.19262824)(268.5292153,397.04762839)(268.50922524,396.88763153)
\curveto(268.49921533,396.8376286)(268.48921534,396.77762866)(268.47922524,396.70763153)
\curveto(268.47921535,396.6376288)(268.48921534,396.57762886)(268.50922524,396.52763153)
\lineto(268.50922524,396.36263153)
\lineto(268.50922524,396.18263153)
\curveto(268.51921531,396.1326293)(268.5292153,396.07762936)(268.53922524,396.01763153)
\curveto(268.54921528,395.96762947)(268.55421528,395.91262952)(268.55422524,395.85263153)
\curveto(268.56421527,395.79262964)(268.57921525,395.7376297)(268.59922524,395.68763153)
\curveto(268.64921518,395.49762994)(268.70921512,395.32263011)(268.77922524,395.16263153)
\curveto(268.84921498,395.00263043)(268.95421488,394.87263056)(269.09422524,394.77263153)
\curveto(269.22421461,394.67263076)(269.36421447,394.60263083)(269.51422524,394.56263153)
\curveto(269.54421429,394.55263088)(269.56921426,394.54763089)(269.58922524,394.54763153)
\curveto(269.61921421,394.55763088)(269.64921418,394.55763088)(269.67922524,394.54763153)
\curveto(269.69921413,394.54763089)(269.7292141,394.54263089)(269.76922524,394.53263153)
\curveto(269.80921402,394.5326309)(269.84421399,394.5376309)(269.87422524,394.54763153)
\curveto(269.91421392,394.55763088)(269.95421388,394.56263087)(269.99422524,394.56263153)
\curveto(270.0342138,394.56263087)(270.07421376,394.57263086)(270.11422524,394.59263153)
\curveto(270.35421348,394.67263076)(270.54921328,394.80763063)(270.69922524,394.99763153)
\curveto(270.81921301,395.17763026)(270.90921292,395.38263005)(270.96922524,395.61263153)
\curveto(270.98921284,395.68262975)(271.00421283,395.75262968)(271.01422524,395.82263153)
\curveto(271.02421281,395.90262953)(271.03921279,395.98262945)(271.05922524,396.06263153)
\curveto(271.05921277,396.12262931)(271.06421277,396.16762927)(271.07422524,396.19763153)
\curveto(271.07421276,396.21762922)(271.07421276,396.24262919)(271.07422524,396.27263153)
\curveto(271.07421276,396.31262912)(271.07921275,396.34262909)(271.08922524,396.36263153)
\lineto(271.08922524,396.51263153)
}
}
{
\newrgbcolor{curcolor}{0 0 0}
\pscustom[linestyle=none,fillstyle=solid,fillcolor=curcolor]
{
\newpath
\moveto(669.75884499,601.66551483)
\curveto(670.44884035,601.6755042)(671.04883975,601.55550432)(671.55884499,601.30551483)
\curveto(672.07883872,601.05550482)(672.47383833,600.72050516)(672.74384499,600.30051483)
\curveto(672.79383801,600.22050566)(672.83883796,600.13050575)(672.87884499,600.03051483)
\curveto(672.91883788,599.94050594)(672.96383784,599.84550603)(673.01384499,599.74551483)
\curveto(673.05383775,599.64550623)(673.08383772,599.54550633)(673.10384499,599.44551483)
\curveto(673.12383768,599.34550653)(673.14383766,599.24050664)(673.16384499,599.13051483)
\curveto(673.18383762,599.0805068)(673.18883761,599.03550684)(673.17884499,598.99551483)
\curveto(673.16883763,598.95550692)(673.17383763,598.91050697)(673.19384499,598.86051483)
\curveto(673.2038376,598.81050707)(673.20883759,598.72550715)(673.20884499,598.60551483)
\curveto(673.20883759,598.49550738)(673.2038376,598.41050747)(673.19384499,598.35051483)
\curveto(673.17383763,598.29050759)(673.16383764,598.23050765)(673.16384499,598.17051483)
\curveto(673.17383763,598.11050777)(673.16883763,598.05050783)(673.14884499,597.99051483)
\curveto(673.10883769,597.85050803)(673.07383773,597.71550816)(673.04384499,597.58551483)
\curveto(673.01383779,597.45550842)(672.97383783,597.33050855)(672.92384499,597.21051483)
\curveto(672.86383794,597.07050881)(672.79383801,596.94550893)(672.71384499,596.83551483)
\curveto(672.64383816,596.72550915)(672.56883823,596.61550926)(672.48884499,596.50551483)
\lineto(672.42884499,596.44551483)
\curveto(672.41883838,596.42550945)(672.4038384,596.40550947)(672.38384499,596.38551483)
\curveto(672.26383854,596.22550965)(672.12883867,596.0805098)(671.97884499,595.95051483)
\curveto(671.82883897,595.82051006)(671.66883913,595.69551018)(671.49884499,595.57551483)
\curveto(671.18883961,595.35551052)(670.89383991,595.15051073)(670.61384499,594.96051483)
\curveto(670.38384042,594.82051106)(670.15384065,594.68551119)(669.92384499,594.55551483)
\curveto(669.7038411,594.42551145)(669.48384132,594.29051159)(669.26384499,594.15051483)
\curveto(669.01384179,593.9805119)(668.77384203,593.80051208)(668.54384499,593.61051483)
\curveto(668.32384248,593.42051246)(668.13384267,593.19551268)(667.97384499,592.93551483)
\curveto(667.93384287,592.875513)(667.8988429,592.81551306)(667.86884499,592.75551483)
\curveto(667.83884296,592.70551317)(667.80884299,592.64051324)(667.77884499,592.56051483)
\curveto(667.75884304,592.49051339)(667.75384305,592.43051345)(667.76384499,592.38051483)
\curveto(667.78384302,592.31051357)(667.81884298,592.25551362)(667.86884499,592.21551483)
\curveto(667.91884288,592.18551369)(667.97884282,592.16551371)(668.04884499,592.15551483)
\lineto(668.28884499,592.15551483)
\lineto(669.03884499,592.15551483)
\lineto(671.84384499,592.15551483)
\lineto(672.50384499,592.15551483)
\curveto(672.59383821,592.15551372)(672.67883812,592.15051373)(672.75884499,592.14051483)
\curveto(672.83883796,592.14051374)(672.9038379,592.12051376)(672.95384499,592.08051483)
\curveto(673.0038378,592.04051384)(673.04383776,591.96551391)(673.07384499,591.85551483)
\curveto(673.11383769,591.75551412)(673.12383768,591.65551422)(673.10384499,591.55551483)
\lineto(673.10384499,591.42051483)
\curveto(673.08383772,591.35051453)(673.06383774,591.29051459)(673.04384499,591.24051483)
\curveto(673.02383778,591.19051469)(672.98883781,591.15051473)(672.93884499,591.12051483)
\curveto(672.88883791,591.0805148)(672.81883798,591.06051482)(672.72884499,591.06051483)
\lineto(672.45884499,591.06051483)
\lineto(671.55884499,591.06051483)
\lineto(668.04884499,591.06051483)
\lineto(666.98384499,591.06051483)
\curveto(666.9038439,591.06051482)(666.81384399,591.05551482)(666.71384499,591.04551483)
\curveto(666.61384419,591.04551483)(666.52884427,591.05551482)(666.45884499,591.07551483)
\curveto(666.24884455,591.14551473)(666.18384462,591.32551455)(666.26384499,591.61551483)
\curveto(666.27384453,591.65551422)(666.27384453,591.69051419)(666.26384499,591.72051483)
\curveto(666.26384454,591.76051412)(666.27384453,591.80551407)(666.29384499,591.85551483)
\curveto(666.31384449,591.93551394)(666.33384447,592.02051386)(666.35384499,592.11051483)
\curveto(666.37384443,592.20051368)(666.3988444,592.28551359)(666.42884499,592.36551483)
\curveto(666.58884421,592.85551302)(666.78884401,593.27051261)(667.02884499,593.61051483)
\curveto(667.20884359,593.86051202)(667.41384339,594.08551179)(667.64384499,594.28551483)
\curveto(667.87384293,594.49551138)(668.11384269,594.69051119)(668.36384499,594.87051483)
\curveto(668.62384218,595.05051083)(668.88884191,595.22051066)(669.15884499,595.38051483)
\curveto(669.43884136,595.55051033)(669.70884109,595.72551015)(669.96884499,595.90551483)
\curveto(670.07884072,595.98550989)(670.18384062,596.06050982)(670.28384499,596.13051483)
\curveto(670.39384041,596.20050968)(670.5038403,596.2755096)(670.61384499,596.35551483)
\curveto(670.65384015,596.38550949)(670.68884011,596.41550946)(670.71884499,596.44551483)
\curveto(670.75884004,596.48550939)(670.79884,596.51550936)(670.83884499,596.53551483)
\curveto(670.97883982,596.64550923)(671.1038397,596.77050911)(671.21384499,596.91051483)
\curveto(671.23383957,596.94050894)(671.25883954,596.96550891)(671.28884499,596.98551483)
\curveto(671.31883948,597.01550886)(671.34383946,597.04550883)(671.36384499,597.07551483)
\curveto(671.44383936,597.1755087)(671.50883929,597.2755086)(671.55884499,597.37551483)
\curveto(671.61883918,597.4755084)(671.67383913,597.58550829)(671.72384499,597.70551483)
\curveto(671.75383905,597.7755081)(671.77383903,597.85050803)(671.78384499,597.93051483)
\lineto(671.84384499,598.17051483)
\lineto(671.84384499,598.26051483)
\curveto(671.85383895,598.29050759)(671.85883894,598.32050756)(671.85884499,598.35051483)
\curveto(671.87883892,598.42050746)(671.88383892,598.51550736)(671.87384499,598.63551483)
\curveto(671.87383893,598.76550711)(671.86383894,598.86550701)(671.84384499,598.93551483)
\curveto(671.82383898,599.01550686)(671.803839,599.09050679)(671.78384499,599.16051483)
\curveto(671.77383903,599.24050664)(671.75383905,599.32050656)(671.72384499,599.40051483)
\curveto(671.61383919,599.64050624)(671.46383934,599.84050604)(671.27384499,600.00051483)
\curveto(671.09383971,600.17050571)(670.87383993,600.31050557)(670.61384499,600.42051483)
\curveto(670.54384026,600.44050544)(670.47384033,600.45550542)(670.40384499,600.46551483)
\curveto(670.33384047,600.48550539)(670.25884054,600.50550537)(670.17884499,600.52551483)
\curveto(670.0988407,600.54550533)(669.98884081,600.55550532)(669.84884499,600.55551483)
\curveto(669.71884108,600.55550532)(669.61384119,600.54550533)(669.53384499,600.52551483)
\curveto(669.47384133,600.51550536)(669.41884138,600.51050537)(669.36884499,600.51051483)
\curveto(669.31884148,600.51050537)(669.26884153,600.50050538)(669.21884499,600.48051483)
\curveto(669.11884168,600.44050544)(669.02384178,600.40050548)(668.93384499,600.36051483)
\curveto(668.85384195,600.32050556)(668.77384203,600.2755056)(668.69384499,600.22551483)
\curveto(668.66384214,600.20550567)(668.63384217,600.1805057)(668.60384499,600.15051483)
\curveto(668.58384222,600.12050576)(668.55884224,600.09550578)(668.52884499,600.07551483)
\lineto(668.45384499,600.00051483)
\curveto(668.42384238,599.9805059)(668.3988424,599.96050592)(668.37884499,599.94051483)
\lineto(668.22884499,599.73051483)
\curveto(668.18884261,599.67050621)(668.14384266,599.60550627)(668.09384499,599.53551483)
\curveto(668.03384277,599.44550643)(667.98384282,599.34050654)(667.94384499,599.22051483)
\curveto(667.91384289,599.11050677)(667.87884292,599.00050688)(667.83884499,598.89051483)
\curveto(667.798843,598.7805071)(667.77384303,598.63550724)(667.76384499,598.45551483)
\curveto(667.75384305,598.28550759)(667.72384308,598.16050772)(667.67384499,598.08051483)
\curveto(667.62384318,598.00050788)(667.54884325,597.95550792)(667.44884499,597.94551483)
\curveto(667.34884345,597.93550794)(667.23884356,597.93050795)(667.11884499,597.93051483)
\curveto(667.07884372,597.93050795)(667.03884376,597.92550795)(666.99884499,597.91551483)
\curveto(666.95884384,597.91550796)(666.92384388,597.92050796)(666.89384499,597.93051483)
\curveto(666.84384396,597.95050793)(666.79384401,597.96050792)(666.74384499,597.96051483)
\curveto(666.7038441,597.96050792)(666.66384414,597.97050791)(666.62384499,597.99051483)
\curveto(666.53384427,598.05050783)(666.48884431,598.18550769)(666.48884499,598.39551483)
\lineto(666.48884499,598.51551483)
\curveto(666.4988443,598.5755073)(666.5038443,598.63550724)(666.50384499,598.69551483)
\curveto(666.51384429,598.76550711)(666.52384428,598.83050705)(666.53384499,598.89051483)
\curveto(666.55384425,599.00050688)(666.57384423,599.10050678)(666.59384499,599.19051483)
\curveto(666.61384419,599.29050659)(666.64384416,599.38550649)(666.68384499,599.47551483)
\curveto(666.7038441,599.54550633)(666.72384408,599.60550627)(666.74384499,599.65551483)
\lineto(666.80384499,599.83551483)
\curveto(666.92384388,600.09550578)(667.07884372,600.34050554)(667.26884499,600.57051483)
\curveto(667.46884333,600.80050508)(667.68384312,600.98550489)(667.91384499,601.12551483)
\curveto(668.02384278,601.20550467)(668.13884266,601.27050461)(668.25884499,601.32051483)
\lineto(668.64884499,601.47051483)
\curveto(668.75884204,601.52050436)(668.87384193,601.55050433)(668.99384499,601.56051483)
\curveto(669.11384169,601.5805043)(669.23884156,601.60550427)(669.36884499,601.63551483)
\curveto(669.43884136,601.63550424)(669.5038413,601.63550424)(669.56384499,601.63551483)
\curveto(669.62384118,601.64550423)(669.68884111,601.65550422)(669.75884499,601.66551483)
}
}
{
\newrgbcolor{curcolor}{0 0 0}
\pscustom[linestyle=none,fillstyle=solid,fillcolor=curcolor]
{
\newpath
\moveto(675.31845436,601.47051483)
\lineto(680.11845436,601.47051483)
\lineto(681.12345436,601.47051483)
\curveto(681.26344726,601.47050441)(681.38344714,601.46050442)(681.48345436,601.44051483)
\curveto(681.59344693,601.43050445)(681.67344685,601.38550449)(681.72345436,601.30551483)
\curveto(681.74344678,601.26550461)(681.75344677,601.21550466)(681.75345436,601.15551483)
\curveto(681.76344676,601.09550478)(681.76844676,601.03050485)(681.76845436,600.96051483)
\lineto(681.76845436,600.69051483)
\curveto(681.76844676,600.60050528)(681.75844677,600.52050536)(681.73845436,600.45051483)
\curveto(681.69844683,600.37050551)(681.65344687,600.30050558)(681.60345436,600.24051483)
\lineto(681.45345436,600.06051483)
\curveto(681.4234471,600.01050587)(681.38844714,599.97050591)(681.34845436,599.94051483)
\curveto(681.30844722,599.91050597)(681.26844726,599.87050601)(681.22845436,599.82051483)
\curveto(681.14844738,599.71050617)(681.06344746,599.60050628)(680.97345436,599.49051483)
\curveto(680.88344764,599.39050649)(680.79844773,599.28550659)(680.71845436,599.17551483)
\curveto(680.57844795,598.9755069)(680.43844809,598.76550711)(680.29845436,598.54551483)
\curveto(680.15844837,598.33550754)(680.01844851,598.12050776)(679.87845436,597.90051483)
\curveto(679.8284487,597.81050807)(679.77844875,597.71550816)(679.72845436,597.61551483)
\curveto(679.67844885,597.51550836)(679.6234489,597.42050846)(679.56345436,597.33051483)
\curveto(679.54344898,597.31050857)(679.53344899,597.28550859)(679.53345436,597.25551483)
\curveto(679.53344899,597.22550865)(679.523449,597.20050868)(679.50345436,597.18051483)
\curveto(679.43344909,597.0805088)(679.36844916,596.96550891)(679.30845436,596.83551483)
\curveto(679.24844928,596.71550916)(679.19344933,596.60050928)(679.14345436,596.49051483)
\curveto(679.04344948,596.26050962)(678.94844958,596.02550985)(678.85845436,595.78551483)
\curveto(678.76844976,595.54551033)(678.66844986,595.30551057)(678.55845436,595.06551483)
\curveto(678.53844999,595.01551086)(678.52345,594.97051091)(678.51345436,594.93051483)
\curveto(678.51345001,594.89051099)(678.50345002,594.84551103)(678.48345436,594.79551483)
\curveto(678.43345009,594.6755112)(678.38845014,594.55051133)(678.34845436,594.42051483)
\curveto(678.31845021,594.30051158)(678.28345024,594.1805117)(678.24345436,594.06051483)
\curveto(678.16345036,593.83051205)(678.09845043,593.59051229)(678.04845436,593.34051483)
\curveto(678.00845052,593.10051278)(677.95845057,592.86051302)(677.89845436,592.62051483)
\curveto(677.85845067,592.47051341)(677.83345069,592.32051356)(677.82345436,592.17051483)
\curveto(677.81345071,592.02051386)(677.79345073,591.87051401)(677.76345436,591.72051483)
\curveto(677.75345077,591.6805142)(677.74845078,591.62051426)(677.74845436,591.54051483)
\curveto(677.71845081,591.42051446)(677.68845084,591.32051456)(677.65845436,591.24051483)
\curveto(677.6284509,591.16051472)(677.55845097,591.10551477)(677.44845436,591.07551483)
\curveto(677.39845113,591.05551482)(677.34345118,591.04551483)(677.28345436,591.04551483)
\lineto(677.08845436,591.04551483)
\curveto(676.94845158,591.04551483)(676.80845172,591.05051483)(676.66845436,591.06051483)
\curveto(676.53845199,591.07051481)(676.44345208,591.11551476)(676.38345436,591.19551483)
\curveto(676.34345218,591.25551462)(676.3234522,591.34051454)(676.32345436,591.45051483)
\curveto(676.33345219,591.56051432)(676.34845218,591.65551422)(676.36845436,591.73551483)
\lineto(676.36845436,591.81051483)
\curveto(676.37845215,591.84051404)(676.38345214,591.87051401)(676.38345436,591.90051483)
\curveto(676.40345212,591.9805139)(676.41345211,592.05551382)(676.41345436,592.12551483)
\curveto(676.41345211,592.19551368)(676.4234521,592.26551361)(676.44345436,592.33551483)
\curveto(676.49345203,592.52551335)(676.53345199,592.71051317)(676.56345436,592.89051483)
\curveto(676.59345193,593.0805128)(676.63345189,593.26051262)(676.68345436,593.43051483)
\curveto(676.70345182,593.4805124)(676.71345181,593.52051236)(676.71345436,593.55051483)
\curveto(676.71345181,593.5805123)(676.71845181,593.61551226)(676.72845436,593.65551483)
\curveto(676.8284517,593.95551192)(676.91845161,594.25051163)(676.99845436,594.54051483)
\curveto(677.08845144,594.83051105)(677.19345133,595.11051077)(677.31345436,595.38051483)
\curveto(677.57345095,595.96050992)(677.84345068,596.51050937)(678.12345436,597.03051483)
\curveto(678.40345012,597.56050832)(678.71344981,598.06550781)(679.05345436,598.54551483)
\curveto(679.19344933,598.74550713)(679.34344918,598.93550694)(679.50345436,599.11551483)
\curveto(679.66344886,599.30550657)(679.81344871,599.49550638)(679.95345436,599.68551483)
\curveto(679.99344853,599.73550614)(680.0284485,599.7805061)(680.05845436,599.82051483)
\curveto(680.09844843,599.87050601)(680.13344839,599.92050596)(680.16345436,599.97051483)
\curveto(680.17344835,599.99050589)(680.18344834,600.01550586)(680.19345436,600.04551483)
\curveto(680.21344831,600.0755058)(680.21344831,600.10550577)(680.19345436,600.13551483)
\curveto(680.17344835,600.19550568)(680.13844839,600.23050565)(680.08845436,600.24051483)
\curveto(680.03844849,600.26050562)(679.98844854,600.2805056)(679.93845436,600.30051483)
\lineto(679.83345436,600.30051483)
\curveto(679.79344873,600.31050557)(679.74344878,600.31050557)(679.68345436,600.30051483)
\lineto(679.53345436,600.30051483)
\lineto(678.93345436,600.30051483)
\lineto(676.29345436,600.30051483)
\lineto(675.55845436,600.30051483)
\lineto(675.31845436,600.30051483)
\curveto(675.24845328,600.31050557)(675.18845334,600.32550555)(675.13845436,600.34551483)
\curveto(675.04845348,600.38550549)(674.98845354,600.44550543)(674.95845436,600.52551483)
\curveto(674.90845362,600.62550525)(674.89345363,600.77050511)(674.91345436,600.96051483)
\curveto(674.93345359,601.16050472)(674.96845356,601.29550458)(675.01845436,601.36551483)
\curveto(675.03845349,601.38550449)(675.06345346,601.40050448)(675.09345436,601.41051483)
\lineto(675.21345436,601.47051483)
\curveto(675.23345329,601.47050441)(675.24845328,601.46550441)(675.25845436,601.45551483)
\curveto(675.27845325,601.45550442)(675.29845323,601.46050442)(675.31845436,601.47051483)
}
}
{
\newrgbcolor{curcolor}{0 0 0}
\pscustom[linestyle=none,fillstyle=solid,fillcolor=curcolor]
{
\newpath
\moveto(693.01306374,599.58051483)
\curveto(692.81305344,599.29050659)(692.60305365,599.00550687)(692.38306374,598.72551483)
\curveto(692.17305408,598.44550743)(691.96805428,598.16050772)(691.76806374,597.87051483)
\curveto(691.16805508,597.02050886)(690.56305569,596.1805097)(689.95306374,595.35051483)
\curveto(689.34305691,594.53051135)(688.73805751,593.69551218)(688.13806374,592.84551483)
\lineto(687.62806374,592.12551483)
\lineto(687.11806374,591.43551483)
\curveto(687.03805921,591.32551455)(686.95805929,591.21051467)(686.87806374,591.09051483)
\curveto(686.79805945,590.97051491)(686.70305955,590.875515)(686.59306374,590.80551483)
\curveto(686.5530597,590.78551509)(686.48805976,590.77051511)(686.39806374,590.76051483)
\curveto(686.31805993,590.74051514)(686.22806002,590.73051515)(686.12806374,590.73051483)
\curveto(686.02806022,590.73051515)(685.93306032,590.73551514)(685.84306374,590.74551483)
\curveto(685.76306049,590.75551512)(685.70306055,590.7755151)(685.66306374,590.80551483)
\curveto(685.63306062,590.82551505)(685.60806064,590.86051502)(685.58806374,590.91051483)
\curveto(685.57806067,590.95051493)(685.58306067,590.99551488)(685.60306374,591.04551483)
\curveto(685.64306061,591.12551475)(685.68806056,591.20051468)(685.73806374,591.27051483)
\curveto(685.79806045,591.35051453)(685.8530604,591.43051445)(685.90306374,591.51051483)
\curveto(686.14306011,591.85051403)(686.38805986,592.18551369)(686.63806374,592.51551483)
\curveto(686.88805936,592.84551303)(687.12805912,593.1805127)(687.35806374,593.52051483)
\curveto(687.51805873,593.74051214)(687.67805857,593.95551192)(687.83806374,594.16551483)
\curveto(687.99805825,594.3755115)(688.15805809,594.59051129)(688.31806374,594.81051483)
\curveto(688.67805757,595.33051055)(689.04305721,595.84051004)(689.41306374,596.34051483)
\curveto(689.78305647,596.84050904)(690.1530561,597.35050853)(690.52306374,597.87051483)
\curveto(690.66305559,598.07050781)(690.80305545,598.26550761)(690.94306374,598.45551483)
\curveto(691.09305516,598.64550723)(691.23805501,598.84050704)(691.37806374,599.04051483)
\curveto(691.58805466,599.34050654)(691.80305445,599.64050624)(692.02306374,599.94051483)
\lineto(692.68306374,600.84051483)
\lineto(692.86306374,601.11051483)
\lineto(693.07306374,601.38051483)
\lineto(693.19306374,601.56051483)
\curveto(693.24305301,601.62050426)(693.29305296,601.6755042)(693.34306374,601.72551483)
\curveto(693.41305284,601.7755041)(693.48805276,601.81050407)(693.56806374,601.83051483)
\curveto(693.58805266,601.84050404)(693.61305264,601.84050404)(693.64306374,601.83051483)
\curveto(693.68305257,601.83050405)(693.71305254,601.84050404)(693.73306374,601.86051483)
\curveto(693.8530524,601.86050402)(693.98805226,601.85550402)(694.13806374,601.84551483)
\curveto(694.28805196,601.84550403)(694.37805187,601.80050408)(694.40806374,601.71051483)
\curveto(694.42805182,601.6805042)(694.43305182,601.64550423)(694.42306374,601.60551483)
\curveto(694.41305184,601.56550431)(694.39805185,601.53550434)(694.37806374,601.51551483)
\curveto(694.33805191,601.43550444)(694.29805195,601.36550451)(694.25806374,601.30551483)
\curveto(694.21805203,601.24550463)(694.17305208,601.18550469)(694.12306374,601.12551483)
\lineto(693.55306374,600.34551483)
\curveto(693.37305288,600.09550578)(693.19305306,599.84050604)(693.01306374,599.58051483)
\moveto(686.15806374,595.68051483)
\curveto(686.10806014,595.70051018)(686.05806019,595.70551017)(686.00806374,595.69551483)
\curveto(685.95806029,595.68551019)(685.90806034,595.69051019)(685.85806374,595.71051483)
\curveto(685.7480605,595.73051015)(685.64306061,595.75051013)(685.54306374,595.77051483)
\curveto(685.4530608,595.80051008)(685.35806089,595.84051004)(685.25806374,595.89051483)
\curveto(684.92806132,596.03050985)(684.67306158,596.22550965)(684.49306374,596.47551483)
\curveto(684.31306194,596.73550914)(684.16806208,597.04550883)(684.05806374,597.40551483)
\curveto(684.02806222,597.48550839)(684.00806224,597.56550831)(683.99806374,597.64551483)
\curveto(683.98806226,597.73550814)(683.97306228,597.82050806)(683.95306374,597.90051483)
\curveto(683.94306231,597.95050793)(683.93806231,598.01550786)(683.93806374,598.09551483)
\curveto(683.92806232,598.12550775)(683.92306233,598.15550772)(683.92306374,598.18551483)
\curveto(683.92306233,598.22550765)(683.91806233,598.26050762)(683.90806374,598.29051483)
\lineto(683.90806374,598.44051483)
\curveto(683.89806235,598.49050739)(683.89306236,598.55050733)(683.89306374,598.62051483)
\curveto(683.89306236,598.70050718)(683.89806235,598.76550711)(683.90806374,598.81551483)
\lineto(683.90806374,598.98051483)
\curveto(683.92806232,599.03050685)(683.93306232,599.0755068)(683.92306374,599.11551483)
\curveto(683.92306233,599.16550671)(683.92806232,599.21050667)(683.93806374,599.25051483)
\curveto(683.9480623,599.29050659)(683.9530623,599.32550655)(683.95306374,599.35551483)
\curveto(683.9530623,599.39550648)(683.95806229,599.43550644)(683.96806374,599.47551483)
\curveto(683.99806225,599.58550629)(684.01806223,599.69550618)(684.02806374,599.80551483)
\curveto(684.0480622,599.92550595)(684.08306217,600.04050584)(684.13306374,600.15051483)
\curveto(684.27306198,600.49050539)(684.43306182,600.76550511)(684.61306374,600.97551483)
\curveto(684.80306145,601.19550468)(685.07306118,601.3755045)(685.42306374,601.51551483)
\curveto(685.50306075,601.54550433)(685.58806066,601.56550431)(685.67806374,601.57551483)
\curveto(685.76806048,601.59550428)(685.86306039,601.61550426)(685.96306374,601.63551483)
\curveto(685.99306026,601.64550423)(686.0480602,601.64550423)(686.12806374,601.63551483)
\curveto(686.20806004,601.63550424)(686.25805999,601.64550423)(686.27806374,601.66551483)
\curveto(686.83805941,601.6755042)(687.28805896,601.56550431)(687.62806374,601.33551483)
\curveto(687.97805827,601.10550477)(688.23805801,600.80050508)(688.40806374,600.42051483)
\curveto(688.4480578,600.33050555)(688.48305777,600.23550564)(688.51306374,600.13551483)
\curveto(688.54305771,600.03550584)(688.56805768,599.93550594)(688.58806374,599.83551483)
\curveto(688.60805764,599.80550607)(688.61305764,599.7755061)(688.60306374,599.74551483)
\curveto(688.60305765,599.71550616)(688.60805764,599.68550619)(688.61806374,599.65551483)
\curveto(688.6480576,599.54550633)(688.66805758,599.42050646)(688.67806374,599.28051483)
\curveto(688.68805756,599.15050673)(688.69805755,599.01550686)(688.70806374,598.87551483)
\lineto(688.70806374,598.71051483)
\curveto(688.71805753,598.65050723)(688.71805753,598.59550728)(688.70806374,598.54551483)
\curveto(688.69805755,598.49550738)(688.69305756,598.44550743)(688.69306374,598.39551483)
\lineto(688.69306374,598.26051483)
\curveto(688.68305757,598.22050766)(688.67805757,598.1805077)(688.67806374,598.14051483)
\curveto(688.68805756,598.10050778)(688.68305757,598.05550782)(688.66306374,598.00551483)
\curveto(688.64305761,597.89550798)(688.62305763,597.79050809)(688.60306374,597.69051483)
\curveto(688.59305766,597.59050829)(688.57305768,597.49050839)(688.54306374,597.39051483)
\curveto(688.41305784,597.03050885)(688.248058,596.71550916)(688.04806374,596.44551483)
\curveto(687.8480584,596.1755097)(687.57305868,595.97050991)(687.22306374,595.83051483)
\curveto(687.14305911,595.80051008)(687.05805919,595.7755101)(686.96806374,595.75551483)
\lineto(686.69806374,595.69551483)
\curveto(686.6480596,595.68551019)(686.60305965,595.6805102)(686.56306374,595.68051483)
\curveto(686.52305973,595.69051019)(686.48305977,595.69051019)(686.44306374,595.68051483)
\curveto(686.34305991,595.66051022)(686.24806,595.66051022)(686.15806374,595.68051483)
\moveto(685.31806374,597.07551483)
\curveto(685.35806089,597.00550887)(685.39806085,596.94050894)(685.43806374,596.88051483)
\curveto(685.47806077,596.83050905)(685.52806072,596.7805091)(685.58806374,596.73051483)
\lineto(685.73806374,596.61051483)
\curveto(685.79806045,596.5805093)(685.86306039,596.55550932)(685.93306374,596.53551483)
\curveto(685.97306028,596.51550936)(686.00806024,596.50550937)(686.03806374,596.50551483)
\curveto(686.07806017,596.51550936)(686.11806013,596.51050937)(686.15806374,596.49051483)
\curveto(686.18806006,596.49050939)(686.22806002,596.48550939)(686.27806374,596.47551483)
\curveto(686.32805992,596.4755094)(686.36805988,596.4805094)(686.39806374,596.49051483)
\lineto(686.62306374,596.53551483)
\curveto(686.87305938,596.61550926)(687.05805919,596.74050914)(687.17806374,596.91051483)
\curveto(687.25805899,597.01050887)(687.32805892,597.14050874)(687.38806374,597.30051483)
\curveto(687.46805878,597.4805084)(687.52805872,597.70550817)(687.56806374,597.97551483)
\curveto(687.60805864,598.25550762)(687.62305863,598.53550734)(687.61306374,598.81551483)
\curveto(687.60305865,599.10550677)(687.57305868,599.3805065)(687.52306374,599.64051483)
\curveto(687.47305878,599.90050598)(687.39805885,600.11050577)(687.29806374,600.27051483)
\curveto(687.17805907,600.47050541)(687.02805922,600.62050526)(686.84806374,600.72051483)
\curveto(686.76805948,600.77050511)(686.67805957,600.80050508)(686.57806374,600.81051483)
\curveto(686.47805977,600.83050505)(686.37305988,600.84050504)(686.26306374,600.84051483)
\curveto(686.24306001,600.83050505)(686.21806003,600.82550505)(686.18806374,600.82551483)
\curveto(686.16806008,600.83550504)(686.1480601,600.83550504)(686.12806374,600.82551483)
\curveto(686.07806017,600.81550506)(686.03306022,600.80550507)(685.99306374,600.79551483)
\curveto(685.9530603,600.79550508)(685.91306034,600.78550509)(685.87306374,600.76551483)
\curveto(685.69306056,600.68550519)(685.54306071,600.56550531)(685.42306374,600.40551483)
\curveto(685.31306094,600.24550563)(685.22306103,600.06550581)(685.15306374,599.86551483)
\curveto(685.09306116,599.6755062)(685.0480612,599.45050643)(685.01806374,599.19051483)
\curveto(684.99806125,598.93050695)(684.99306126,598.66550721)(685.00306374,598.39551483)
\curveto(685.01306124,598.13550774)(685.04306121,597.88550799)(685.09306374,597.64551483)
\curveto(685.1530611,597.41550846)(685.22806102,597.22550865)(685.31806374,597.07551483)
\moveto(696.11806374,594.09051483)
\curveto(696.12805012,594.04051184)(696.13305012,593.95051193)(696.13306374,593.82051483)
\curveto(696.13305012,593.69051219)(696.12305013,593.60051228)(696.10306374,593.55051483)
\curveto(696.08305017,593.50051238)(696.07805017,593.44551243)(696.08806374,593.38551483)
\curveto(696.09805015,593.33551254)(696.09805015,593.28551259)(696.08806374,593.23551483)
\curveto(696.0480502,593.09551278)(696.01805023,592.96051292)(695.99806374,592.83051483)
\curveto(695.98805026,592.70051318)(695.95805029,592.5805133)(695.90806374,592.47051483)
\curveto(695.76805048,592.12051376)(695.60305065,591.82551405)(695.41306374,591.58551483)
\curveto(695.22305103,591.35551452)(694.9530513,591.17051471)(694.60306374,591.03051483)
\curveto(694.52305173,591.00051488)(694.43805181,590.9805149)(694.34806374,590.97051483)
\curveto(694.25805199,590.95051493)(694.17305208,590.93051495)(694.09306374,590.91051483)
\curveto(694.04305221,590.90051498)(693.99305226,590.89551498)(693.94306374,590.89551483)
\curveto(693.89305236,590.89551498)(693.84305241,590.89051499)(693.79306374,590.88051483)
\curveto(693.76305249,590.87051501)(693.71305254,590.87051501)(693.64306374,590.88051483)
\curveto(693.57305268,590.880515)(693.52305273,590.88551499)(693.49306374,590.89551483)
\curveto(693.43305282,590.91551496)(693.37305288,590.92551495)(693.31306374,590.92551483)
\curveto(693.26305299,590.91551496)(693.21305304,590.92051496)(693.16306374,590.94051483)
\curveto(693.07305318,590.96051492)(692.98305327,590.98551489)(692.89306374,591.01551483)
\curveto(692.81305344,591.03551484)(692.73305352,591.06551481)(692.65306374,591.10551483)
\curveto(692.33305392,591.24551463)(692.08305417,591.44051444)(691.90306374,591.69051483)
\curveto(691.72305453,591.95051393)(691.57305468,592.25551362)(691.45306374,592.60551483)
\curveto(691.43305482,592.68551319)(691.41805483,592.77051311)(691.40806374,592.86051483)
\curveto(691.39805485,592.95051293)(691.38305487,593.03551284)(691.36306374,593.11551483)
\curveto(691.3530549,593.14551273)(691.3480549,593.1755127)(691.34806374,593.20551483)
\lineto(691.34806374,593.31051483)
\curveto(691.32805492,593.39051249)(691.31805493,593.47051241)(691.31806374,593.55051483)
\lineto(691.31806374,593.68551483)
\curveto(691.29805495,593.78551209)(691.29805495,593.88551199)(691.31806374,593.98551483)
\lineto(691.31806374,594.16551483)
\curveto(691.32805492,594.21551166)(691.33305492,594.26051162)(691.33306374,594.30051483)
\curveto(691.33305492,594.35051153)(691.33805491,594.39551148)(691.34806374,594.43551483)
\curveto(691.35805489,594.4755114)(691.36305489,594.51051137)(691.36306374,594.54051483)
\curveto(691.36305489,594.5805113)(691.36805488,594.62051126)(691.37806374,594.66051483)
\lineto(691.43806374,594.99051483)
\curveto(691.45805479,595.11051077)(691.48805476,595.22051066)(691.52806374,595.32051483)
\curveto(691.66805458,595.65051023)(691.82805442,595.92550995)(692.00806374,596.14551483)
\curveto(692.19805405,596.3755095)(692.45805379,596.56050932)(692.78806374,596.70051483)
\curveto(692.86805338,596.74050914)(692.9530533,596.76550911)(693.04306374,596.77551483)
\lineto(693.34306374,596.83551483)
\lineto(693.47806374,596.83551483)
\curveto(693.52805272,596.84550903)(693.57805267,596.85050903)(693.62806374,596.85051483)
\curveto(694.19805205,596.87050901)(694.65805159,596.76550911)(695.00806374,596.53551483)
\curveto(695.36805088,596.31550956)(695.63305062,596.01550986)(695.80306374,595.63551483)
\curveto(695.8530504,595.53551034)(695.89305036,595.43551044)(695.92306374,595.33551483)
\curveto(695.9530503,595.23551064)(695.98305027,595.13051075)(696.01306374,595.02051483)
\curveto(696.02305023,594.9805109)(696.02805022,594.94551093)(696.02806374,594.91551483)
\curveto(696.02805022,594.89551098)(696.03305022,594.86551101)(696.04306374,594.82551483)
\curveto(696.06305019,594.75551112)(696.07305018,594.6805112)(696.07306374,594.60051483)
\curveto(696.07305018,594.52051136)(696.08305017,594.44051144)(696.10306374,594.36051483)
\curveto(696.10305015,594.31051157)(696.10305015,594.26551161)(696.10306374,594.22551483)
\curveto(696.10305015,594.18551169)(696.10805014,594.14051174)(696.11806374,594.09051483)
\moveto(695.00806374,593.65551483)
\curveto(695.01805123,593.70551217)(695.02305123,593.7805121)(695.02306374,593.88051483)
\curveto(695.03305122,593.9805119)(695.02805122,594.05551182)(695.00806374,594.10551483)
\curveto(694.98805126,594.16551171)(694.98305127,594.22051166)(694.99306374,594.27051483)
\curveto(695.01305124,594.33051155)(695.01305124,594.39051149)(694.99306374,594.45051483)
\curveto(694.98305127,594.4805114)(694.97805127,594.51551136)(694.97806374,594.55551483)
\curveto(694.97805127,594.59551128)(694.97305128,594.63551124)(694.96306374,594.67551483)
\curveto(694.94305131,594.75551112)(694.92305133,594.83051105)(694.90306374,594.90051483)
\curveto(694.89305136,594.9805109)(694.87805137,595.06051082)(694.85806374,595.14051483)
\curveto(694.82805142,595.20051068)(694.80305145,595.26051062)(694.78306374,595.32051483)
\curveto(694.76305149,595.3805105)(694.73305152,595.44051044)(694.69306374,595.50051483)
\curveto(694.59305166,595.67051021)(694.46305179,595.80551007)(694.30306374,595.90551483)
\curveto(694.22305203,595.95550992)(694.12805212,595.99050989)(694.01806374,596.01051483)
\curveto(693.90805234,596.03050985)(693.78305247,596.04050984)(693.64306374,596.04051483)
\curveto(693.62305263,596.03050985)(693.59805265,596.02550985)(693.56806374,596.02551483)
\curveto(693.53805271,596.03550984)(693.50805274,596.03550984)(693.47806374,596.02551483)
\lineto(693.32806374,595.96551483)
\curveto(693.27805297,595.95550992)(693.23305302,595.94050994)(693.19306374,595.92051483)
\curveto(693.00305325,595.81051007)(692.85805339,595.66551021)(692.75806374,595.48551483)
\curveto(692.66805358,595.30551057)(692.58805366,595.10051078)(692.51806374,594.87051483)
\curveto(692.47805377,594.74051114)(692.45805379,594.60551127)(692.45806374,594.46551483)
\curveto(692.45805379,594.33551154)(692.4480538,594.19051169)(692.42806374,594.03051483)
\curveto(692.41805383,593.9805119)(692.40805384,593.92051196)(692.39806374,593.85051483)
\curveto(692.39805385,593.7805121)(692.40805384,593.72051216)(692.42806374,593.67051483)
\lineto(692.42806374,593.50551483)
\lineto(692.42806374,593.32551483)
\curveto(692.43805381,593.2755126)(692.4480538,593.22051266)(692.45806374,593.16051483)
\curveto(692.46805378,593.11051277)(692.47305378,593.05551282)(692.47306374,592.99551483)
\curveto(692.48305377,592.93551294)(692.49805375,592.880513)(692.51806374,592.83051483)
\curveto(692.56805368,592.64051324)(692.62805362,592.46551341)(692.69806374,592.30551483)
\curveto(692.76805348,592.14551373)(692.87305338,592.01551386)(693.01306374,591.91551483)
\curveto(693.14305311,591.81551406)(693.28305297,591.74551413)(693.43306374,591.70551483)
\curveto(693.46305279,591.69551418)(693.48805276,591.69051419)(693.50806374,591.69051483)
\curveto(693.53805271,591.70051418)(693.56805268,591.70051418)(693.59806374,591.69051483)
\curveto(693.61805263,591.69051419)(693.6480526,591.68551419)(693.68806374,591.67551483)
\curveto(693.72805252,591.6755142)(693.76305249,591.6805142)(693.79306374,591.69051483)
\curveto(693.83305242,591.70051418)(693.87305238,591.70551417)(693.91306374,591.70551483)
\curveto(693.9530523,591.70551417)(693.99305226,591.71551416)(694.03306374,591.73551483)
\curveto(694.27305198,591.81551406)(694.46805178,591.95051393)(694.61806374,592.14051483)
\curveto(694.73805151,592.32051356)(694.82805142,592.52551335)(694.88806374,592.75551483)
\curveto(694.90805134,592.82551305)(694.92305133,592.89551298)(694.93306374,592.96551483)
\curveto(694.94305131,593.04551283)(694.95805129,593.12551275)(694.97806374,593.20551483)
\curveto(694.97805127,593.26551261)(694.98305127,593.31051257)(694.99306374,593.34051483)
\curveto(694.99305126,593.36051252)(694.99305126,593.38551249)(694.99306374,593.41551483)
\curveto(694.99305126,593.45551242)(694.99805125,593.48551239)(695.00806374,593.50551483)
\lineto(695.00806374,593.65551483)
}
}
{
\newrgbcolor{curcolor}{0 0 0}
\pscustom[linestyle=none,fillstyle=solid,fillcolor=curcolor]
{
\newpath
\moveto(460.56881447,508.0941037)
\curveto(460.66880961,508.09409308)(460.76380952,508.08409309)(460.85381447,508.0641037)
\curveto(460.94380934,508.05409312)(461.00880927,508.02409315)(461.04881447,507.9741037)
\curveto(461.10880917,507.89409328)(461.13880914,507.78909339)(461.13881447,507.6591037)
\lineto(461.13881447,507.2691037)
\lineto(461.13881447,505.7691037)
\lineto(461.13881447,499.3791037)
\lineto(461.13881447,498.2091037)
\lineto(461.13881447,497.8941037)
\curveto(461.14880913,497.79410338)(461.13380915,497.71410346)(461.09381447,497.6541037)
\curveto(461.04380924,497.5741036)(460.96880931,497.52410365)(460.86881447,497.5041037)
\curveto(460.7788095,497.49410368)(460.66880961,497.48910369)(460.53881447,497.4891037)
\lineto(460.31381447,497.4891037)
\curveto(460.23381005,497.50910367)(460.16381012,497.52410365)(460.10381447,497.5341037)
\curveto(460.04381024,497.55410362)(459.99381029,497.59410358)(459.95381447,497.6541037)
\curveto(459.91381037,497.71410346)(459.89381039,497.78910339)(459.89381447,497.8791037)
\lineto(459.89381447,498.1791037)
\lineto(459.89381447,499.2741037)
\lineto(459.89381447,504.6141037)
\curveto(459.87381041,504.70409647)(459.85881042,504.7790964)(459.84881447,504.8391037)
\curveto(459.84881043,504.90909627)(459.81881046,504.96909621)(459.75881447,505.0191037)
\curveto(459.68881059,505.06909611)(459.59881068,505.09409608)(459.48881447,505.0941037)
\curveto(459.38881089,505.10409607)(459.278811,505.10909607)(459.15881447,505.1091037)
\lineto(458.01881447,505.1091037)
\lineto(457.52381447,505.1091037)
\curveto(457.36381292,505.11909606)(457.25381303,505.179096)(457.19381447,505.2891037)
\curveto(457.17381311,505.31909586)(457.16381312,505.34909583)(457.16381447,505.3791037)
\curveto(457.16381312,505.41909576)(457.15881312,505.46409571)(457.14881447,505.5141037)
\curveto(457.12881315,505.63409554)(457.13381315,505.74409543)(457.16381447,505.8441037)
\curveto(457.20381308,505.94409523)(457.25881302,506.01409516)(457.32881447,506.0541037)
\curveto(457.40881287,506.10409507)(457.52881275,506.12909505)(457.68881447,506.1291037)
\curveto(457.84881243,506.12909505)(457.9838123,506.14409503)(458.09381447,506.1741037)
\curveto(458.14381214,506.18409499)(458.19881208,506.18909499)(458.25881447,506.1891037)
\curveto(458.31881196,506.19909498)(458.3788119,506.21409496)(458.43881447,506.2341037)
\curveto(458.58881169,506.28409489)(458.73381155,506.33409484)(458.87381447,506.3841037)
\curveto(459.01381127,506.44409473)(459.14881113,506.51409466)(459.27881447,506.5941037)
\curveto(459.41881086,506.68409449)(459.53881074,506.78909439)(459.63881447,506.9091037)
\curveto(459.73881054,507.02909415)(459.83381045,507.15909402)(459.92381447,507.2991037)
\curveto(459.9838103,507.39909378)(460.02881025,507.50909367)(460.05881447,507.6291037)
\curveto(460.09881018,507.74909343)(460.14881013,507.85409332)(460.20881447,507.9441037)
\curveto(460.25881002,508.00409317)(460.32880995,508.04409313)(460.41881447,508.0641037)
\curveto(460.43880984,508.0740931)(460.46380982,508.0790931)(460.49381447,508.0791037)
\curveto(460.52380976,508.0790931)(460.54880973,508.08409309)(460.56881447,508.0941037)
}
}
{
\newrgbcolor{curcolor}{0 0 0}
\pscustom[linestyle=none,fillstyle=solid,fillcolor=curcolor]
{
\newpath
\moveto(471.85842384,502.5741037)
\lineto(471.85842384,502.3191037)
\curveto(471.86841614,502.23909894)(471.86341614,502.16409901)(471.84342384,502.0941037)
\lineto(471.84342384,501.8541037)
\lineto(471.84342384,501.6891037)
\curveto(471.82341618,501.58909959)(471.81341619,501.48409969)(471.81342384,501.3741037)
\curveto(471.81341619,501.2740999)(471.8034162,501.1741)(471.78342384,501.0741037)
\lineto(471.78342384,500.9241037)
\curveto(471.75341625,500.78410039)(471.73341627,500.64410053)(471.72342384,500.5041037)
\curveto(471.71341629,500.3741008)(471.68841632,500.24410093)(471.64842384,500.1141037)
\curveto(471.62841638,500.03410114)(471.6084164,499.94910123)(471.58842384,499.8591037)
\lineto(471.52842384,499.6191037)
\lineto(471.40842384,499.3191037)
\curveto(471.37841663,499.22910195)(471.34341666,499.13910204)(471.30342384,499.0491037)
\curveto(471.2034168,498.82910235)(471.06841694,498.61410256)(470.89842384,498.4041037)
\curveto(470.73841727,498.19410298)(470.56341744,498.02410315)(470.37342384,497.8941037)
\curveto(470.32341768,497.85410332)(470.26341774,497.81410336)(470.19342384,497.7741037)
\curveto(470.13341787,497.74410343)(470.07341793,497.70910347)(470.01342384,497.6691037)
\curveto(469.93341807,497.61910356)(469.83841817,497.5791036)(469.72842384,497.5491037)
\curveto(469.61841839,497.51910366)(469.51341849,497.48910369)(469.41342384,497.4591037)
\curveto(469.3034187,497.41910376)(469.19341881,497.39410378)(469.08342384,497.3841037)
\curveto(468.97341903,497.3741038)(468.85841915,497.35910382)(468.73842384,497.3391037)
\curveto(468.69841931,497.32910385)(468.65341935,497.32910385)(468.60342384,497.3391037)
\curveto(468.56341944,497.33910384)(468.52341948,497.33410384)(468.48342384,497.3241037)
\curveto(468.44341956,497.31410386)(468.38841962,497.30910387)(468.31842384,497.3091037)
\curveto(468.24841976,497.30910387)(468.19841981,497.31410386)(468.16842384,497.3241037)
\curveto(468.11841989,497.34410383)(468.07341993,497.34910383)(468.03342384,497.3391037)
\curveto(467.99342001,497.32910385)(467.95842005,497.32910385)(467.92842384,497.3391037)
\lineto(467.83842384,497.3391037)
\curveto(467.77842023,497.35910382)(467.71342029,497.3741038)(467.64342384,497.3841037)
\curveto(467.58342042,497.38410379)(467.51842049,497.38910379)(467.44842384,497.3991037)
\curveto(467.27842073,497.44910373)(467.11842089,497.49910368)(466.96842384,497.5491037)
\curveto(466.81842119,497.59910358)(466.67342133,497.66410351)(466.53342384,497.7441037)
\curveto(466.48342152,497.78410339)(466.42842158,497.81410336)(466.36842384,497.8341037)
\curveto(466.31842169,497.86410331)(466.26842174,497.89910328)(466.21842384,497.9391037)
\curveto(465.97842203,498.11910306)(465.77842223,498.33910284)(465.61842384,498.5991037)
\curveto(465.45842255,498.85910232)(465.31842269,499.14410203)(465.19842384,499.4541037)
\curveto(465.13842287,499.59410158)(465.09342291,499.73410144)(465.06342384,499.8741037)
\curveto(465.03342297,500.02410115)(464.99842301,500.179101)(464.95842384,500.3391037)
\curveto(464.93842307,500.44910073)(464.92342308,500.55910062)(464.91342384,500.6691037)
\curveto(464.9034231,500.7791004)(464.88842312,500.88910029)(464.86842384,500.9991037)
\curveto(464.85842315,501.03910014)(464.85342315,501.0791001)(464.85342384,501.1191037)
\curveto(464.86342314,501.15910002)(464.86342314,501.19909998)(464.85342384,501.2391037)
\curveto(464.84342316,501.28909989)(464.83842317,501.33909984)(464.83842384,501.3891037)
\lineto(464.83842384,501.5541037)
\curveto(464.81842319,501.60409957)(464.81342319,501.65409952)(464.82342384,501.7041037)
\curveto(464.83342317,501.76409941)(464.83342317,501.81909936)(464.82342384,501.8691037)
\curveto(464.81342319,501.90909927)(464.81342319,501.95409922)(464.82342384,502.0041037)
\curveto(464.83342317,502.05409912)(464.82842318,502.10409907)(464.80842384,502.1541037)
\curveto(464.78842322,502.22409895)(464.78342322,502.29909888)(464.79342384,502.3791037)
\curveto(464.8034232,502.46909871)(464.8084232,502.55409862)(464.80842384,502.6341037)
\curveto(464.8084232,502.72409845)(464.8034232,502.82409835)(464.79342384,502.9341037)
\curveto(464.78342322,503.05409812)(464.78842322,503.15409802)(464.80842384,503.2341037)
\lineto(464.80842384,503.5191037)
\lineto(464.85342384,504.1491037)
\curveto(464.86342314,504.24909693)(464.87342313,504.34409683)(464.88342384,504.4341037)
\lineto(464.91342384,504.7341037)
\curveto(464.93342307,504.78409639)(464.93842307,504.83409634)(464.92842384,504.8841037)
\curveto(464.92842308,504.94409623)(464.93842307,504.99909618)(464.95842384,505.0491037)
\curveto(465.008423,505.21909596)(465.04842296,505.38409579)(465.07842384,505.5441037)
\curveto(465.1084229,505.71409546)(465.15842285,505.8740953)(465.22842384,506.0241037)
\curveto(465.41842259,506.48409469)(465.63842237,506.85909432)(465.88842384,507.1491037)
\curveto(466.14842186,507.43909374)(466.5084215,507.68409349)(466.96842384,507.8841037)
\curveto(467.09842091,507.93409324)(467.22842078,507.96909321)(467.35842384,507.9891037)
\curveto(467.49842051,508.00909317)(467.63842037,508.03409314)(467.77842384,508.0641037)
\curveto(467.84842016,508.0740931)(467.91342009,508.0790931)(467.97342384,508.0791037)
\curveto(468.03341997,508.0790931)(468.09841991,508.08409309)(468.16842384,508.0941037)
\curveto(468.99841901,508.11409306)(469.66841834,507.96409321)(470.17842384,507.6441037)
\curveto(470.68841732,507.33409384)(471.06841694,506.89409428)(471.31842384,506.3241037)
\curveto(471.36841664,506.20409497)(471.41341659,506.0790951)(471.45342384,505.9491037)
\curveto(471.49341651,505.81909536)(471.53841647,505.68409549)(471.58842384,505.5441037)
\curveto(471.6084164,505.46409571)(471.62341638,505.3790958)(471.63342384,505.2891037)
\lineto(471.69342384,505.0491037)
\curveto(471.72341628,504.93909624)(471.73841627,504.82909635)(471.73842384,504.7191037)
\curveto(471.74841626,504.60909657)(471.76341624,504.49909668)(471.78342384,504.3891037)
\curveto(471.8034162,504.33909684)(471.8084162,504.29409688)(471.79842384,504.2541037)
\curveto(471.79841621,504.21409696)(471.8034162,504.174097)(471.81342384,504.1341037)
\curveto(471.82341618,504.08409709)(471.82341618,504.02909715)(471.81342384,503.9691037)
\curveto(471.81341619,503.91909726)(471.81841619,503.86909731)(471.82842384,503.8191037)
\lineto(471.82842384,503.6841037)
\curveto(471.84841616,503.62409755)(471.84841616,503.55409762)(471.82842384,503.4741037)
\curveto(471.81841619,503.40409777)(471.82341618,503.33909784)(471.84342384,503.2791037)
\curveto(471.85341615,503.24909793)(471.85841615,503.20909797)(471.85842384,503.1591037)
\lineto(471.85842384,503.0391037)
\lineto(471.85842384,502.5741037)
\moveto(470.31342384,500.2491037)
\curveto(470.41341759,500.56910061)(470.47341753,500.93410024)(470.49342384,501.3441037)
\curveto(470.51341749,501.75409942)(470.52341748,502.16409901)(470.52342384,502.5741037)
\curveto(470.52341748,503.00409817)(470.51341749,503.42409775)(470.49342384,503.8341037)
\curveto(470.47341753,504.24409693)(470.42841758,504.62909655)(470.35842384,504.9891037)
\curveto(470.28841772,505.34909583)(470.17841783,505.66909551)(470.02842384,505.9491037)
\curveto(469.88841812,506.23909494)(469.69341831,506.4740947)(469.44342384,506.6541037)
\curveto(469.28341872,506.76409441)(469.1034189,506.84409433)(468.90342384,506.8941037)
\curveto(468.7034193,506.95409422)(468.45841955,506.98409419)(468.16842384,506.9841037)
\curveto(468.14841986,506.96409421)(468.11341989,506.95409422)(468.06342384,506.9541037)
\curveto(468.01341999,506.96409421)(467.97342003,506.96409421)(467.94342384,506.9541037)
\curveto(467.86342014,506.93409424)(467.78842022,506.91409426)(467.71842384,506.8941037)
\curveto(467.65842035,506.88409429)(467.59342041,506.86409431)(467.52342384,506.8341037)
\curveto(467.25342075,506.71409446)(467.03342097,506.54409463)(466.86342384,506.3241037)
\curveto(466.7034213,506.11409506)(466.56842144,505.86909531)(466.45842384,505.5891037)
\curveto(466.4084216,505.4790957)(466.36842164,505.35909582)(466.33842384,505.2291037)
\curveto(466.31842169,505.10909607)(466.29342171,504.98409619)(466.26342384,504.8541037)
\curveto(466.24342176,504.80409637)(466.23342177,504.74909643)(466.23342384,504.6891037)
\curveto(466.23342177,504.63909654)(466.22842178,504.58909659)(466.21842384,504.5391037)
\curveto(466.2084218,504.44909673)(466.19842181,504.35409682)(466.18842384,504.2541037)
\curveto(466.17842183,504.16409701)(466.16842184,504.06909711)(466.15842384,503.9691037)
\curveto(466.15842185,503.88909729)(466.15342185,503.80409737)(466.14342384,503.7141037)
\lineto(466.14342384,503.4741037)
\lineto(466.14342384,503.2941037)
\curveto(466.13342187,503.26409791)(466.12842188,503.22909795)(466.12842384,503.1891037)
\lineto(466.12842384,503.0541037)
\lineto(466.12842384,502.6041037)
\curveto(466.12842188,502.52409865)(466.12342188,502.43909874)(466.11342384,502.3491037)
\curveto(466.11342189,502.26909891)(466.12342188,502.19409898)(466.14342384,502.1241037)
\lineto(466.14342384,501.8541037)
\curveto(466.14342186,501.83409934)(466.13842187,501.80409937)(466.12842384,501.7641037)
\curveto(466.12842188,501.73409944)(466.13342187,501.70909947)(466.14342384,501.6891037)
\curveto(466.15342185,501.58909959)(466.15842185,501.48909969)(466.15842384,501.3891037)
\curveto(466.16842184,501.29909988)(466.17842183,501.19909998)(466.18842384,501.0891037)
\curveto(466.21842179,500.96910021)(466.23342177,500.84410033)(466.23342384,500.7141037)
\curveto(466.24342176,500.59410058)(466.26842174,500.4791007)(466.30842384,500.3691037)
\curveto(466.38842162,500.06910111)(466.47342153,499.80410137)(466.56342384,499.5741037)
\curveto(466.66342134,499.34410183)(466.8084212,499.12910205)(466.99842384,498.9291037)
\curveto(467.2084208,498.72910245)(467.47342053,498.5791026)(467.79342384,498.4791037)
\curveto(467.83342017,498.45910272)(467.86842014,498.44910273)(467.89842384,498.4491037)
\curveto(467.93842007,498.45910272)(467.98342002,498.45410272)(468.03342384,498.4341037)
\curveto(468.07341993,498.42410275)(468.14341986,498.41410276)(468.24342384,498.4041037)
\curveto(468.35341965,498.39410278)(468.43841957,498.39910278)(468.49842384,498.4191037)
\curveto(468.56841944,498.43910274)(468.63841937,498.44910273)(468.70842384,498.4491037)
\curveto(468.77841923,498.45910272)(468.84341916,498.4741027)(468.90342384,498.4941037)
\curveto(469.1034189,498.55410262)(469.28341872,498.63910254)(469.44342384,498.7491037)
\curveto(469.47341853,498.76910241)(469.49841851,498.78910239)(469.51842384,498.8091037)
\lineto(469.57842384,498.8691037)
\curveto(469.61841839,498.88910229)(469.66841834,498.92910225)(469.72842384,498.9891037)
\curveto(469.82841818,499.12910205)(469.91341809,499.25910192)(469.98342384,499.3791037)
\curveto(470.05341795,499.49910168)(470.12341788,499.64410153)(470.19342384,499.8141037)
\curveto(470.22341778,499.88410129)(470.24341776,499.95410122)(470.25342384,500.0241037)
\curveto(470.27341773,500.09410108)(470.29341771,500.16910101)(470.31342384,500.2491037)
}
}
{
\newrgbcolor{curcolor}{0 0 0}
\pscustom[linestyle=none,fillstyle=solid,fillcolor=curcolor]
{
\newpath
\moveto(474.16303322,499.1241037)
\lineto(474.46303322,499.1241037)
\curveto(474.57303116,499.13410204)(474.67803105,499.13410204)(474.77803322,499.1241037)
\curveto(474.88803084,499.12410205)(474.98803074,499.11410206)(475.07803322,499.0941037)
\curveto(475.16803056,499.08410209)(475.23803049,499.05910212)(475.28803322,499.0191037)
\curveto(475.30803042,498.99910218)(475.32303041,498.96910221)(475.33303322,498.9291037)
\curveto(475.35303038,498.88910229)(475.37303036,498.84410233)(475.39303322,498.7941037)
\lineto(475.39303322,498.7191037)
\curveto(475.40303033,498.66910251)(475.40303033,498.61410256)(475.39303322,498.5541037)
\lineto(475.39303322,498.4041037)
\lineto(475.39303322,497.9241037)
\curveto(475.39303034,497.75410342)(475.35303038,497.63410354)(475.27303322,497.5641037)
\curveto(475.20303053,497.51410366)(475.11303062,497.48910369)(475.00303322,497.4891037)
\lineto(474.67303322,497.4891037)
\lineto(474.22303322,497.4891037)
\curveto(474.07303166,497.48910369)(473.95803177,497.51910366)(473.87803322,497.5791037)
\curveto(473.83803189,497.60910357)(473.80803192,497.65910352)(473.78803322,497.7291037)
\curveto(473.76803196,497.80910337)(473.75303198,497.89410328)(473.74303322,497.9841037)
\lineto(473.74303322,498.2691037)
\curveto(473.75303198,498.36910281)(473.75803197,498.45410272)(473.75803322,498.5241037)
\lineto(473.75803322,498.7191037)
\curveto(473.75803197,498.7791024)(473.76803196,498.83410234)(473.78803322,498.8841037)
\curveto(473.8280319,498.99410218)(473.89803183,499.06410211)(473.99803322,499.0941037)
\curveto(474.0280317,499.09410208)(474.08303165,499.10410207)(474.16303322,499.1241037)
}
}
{
\newrgbcolor{curcolor}{0 0 0}
\pscustom[linestyle=none,fillstyle=solid,fillcolor=curcolor]
{
\newpath
\moveto(484.38318947,500.5641037)
\curveto(484.39318175,500.52410065)(484.39318175,500.4741007)(484.38318947,500.4141037)
\curveto(484.38318176,500.35410082)(484.37818176,500.30410087)(484.36818947,500.2641037)
\curveto(484.36818177,500.22410095)(484.36318178,500.18410099)(484.35318947,500.1441037)
\lineto(484.35318947,500.0391037)
\curveto(484.33318181,499.95910122)(484.31818182,499.8791013)(484.30818947,499.7991037)
\curveto(484.29818184,499.71910146)(484.27818186,499.64410153)(484.24818947,499.5741037)
\curveto(484.22818191,499.49410168)(484.20818193,499.41910176)(484.18818947,499.3491037)
\curveto(484.16818197,499.2791019)(484.138182,499.20410197)(484.09818947,499.1241037)
\curveto(483.91818222,498.70410247)(483.66318248,498.36410281)(483.33318947,498.1041037)
\curveto(483.00318314,497.84410333)(482.61318353,497.63910354)(482.16318947,497.4891037)
\curveto(482.0431841,497.44910373)(481.91818422,497.42410375)(481.78818947,497.4141037)
\curveto(481.66818447,497.39410378)(481.5431846,497.36910381)(481.41318947,497.3391037)
\curveto(481.35318479,497.32910385)(481.28818485,497.32410385)(481.21818947,497.3241037)
\curveto(481.15818498,497.32410385)(481.09318505,497.31910386)(481.02318947,497.3091037)
\lineto(480.90318947,497.3091037)
\lineto(480.70818947,497.3091037)
\curveto(480.64818549,497.29910388)(480.59318555,497.30410387)(480.54318947,497.3241037)
\curveto(480.47318567,497.34410383)(480.40818573,497.34910383)(480.34818947,497.3391037)
\curveto(480.28818585,497.32910385)(480.22818591,497.33410384)(480.16818947,497.3541037)
\curveto(480.11818602,497.36410381)(480.07318607,497.36910381)(480.03318947,497.3691037)
\curveto(479.99318615,497.36910381)(479.94818619,497.3791038)(479.89818947,497.3991037)
\curveto(479.81818632,497.41910376)(479.7431864,497.43910374)(479.67318947,497.4591037)
\curveto(479.60318654,497.46910371)(479.53318661,497.48410369)(479.46318947,497.5041037)
\curveto(478.98318716,497.6741035)(478.58318756,497.88410329)(478.26318947,498.1341037)
\curveto(477.95318819,498.39410278)(477.70318844,498.74910243)(477.51318947,499.1991037)
\curveto(477.48318866,499.25910192)(477.45818868,499.31910186)(477.43818947,499.3791037)
\curveto(477.42818871,499.44910173)(477.41318873,499.52410165)(477.39318947,499.6041037)
\curveto(477.37318877,499.66410151)(477.35818878,499.72910145)(477.34818947,499.7991037)
\curveto(477.3381888,499.86910131)(477.32318882,499.93910124)(477.30318947,500.0091037)
\curveto(477.29318885,500.05910112)(477.28818885,500.09910108)(477.28818947,500.1291037)
\lineto(477.28818947,500.2491037)
\curveto(477.27818886,500.28910089)(477.26818887,500.33910084)(477.25818947,500.3991037)
\curveto(477.25818888,500.45910072)(477.26318888,500.50910067)(477.27318947,500.5491037)
\lineto(477.27318947,500.6841037)
\curveto(477.28318886,500.73410044)(477.28818885,500.78410039)(477.28818947,500.8341037)
\curveto(477.30818883,500.93410024)(477.32318882,501.02910015)(477.33318947,501.1191037)
\curveto(477.3431888,501.21909996)(477.36318878,501.31409986)(477.39318947,501.4041037)
\curveto(477.4431887,501.55409962)(477.49818864,501.69409948)(477.55818947,501.8241037)
\curveto(477.61818852,501.95409922)(477.68818845,502.0740991)(477.76818947,502.1841037)
\curveto(477.79818834,502.23409894)(477.82818831,502.2740989)(477.85818947,502.3041037)
\curveto(477.89818824,502.33409884)(477.93318821,502.36909881)(477.96318947,502.4091037)
\curveto(478.02318812,502.48909869)(478.09318805,502.55909862)(478.17318947,502.6191037)
\curveto(478.23318791,502.66909851)(478.29318785,502.71409846)(478.35318947,502.7541037)
\lineto(478.56318947,502.9041037)
\curveto(478.61318753,502.94409823)(478.66318748,502.9790982)(478.71318947,503.0091037)
\curveto(478.76318738,503.04909813)(478.79818734,503.10409807)(478.81818947,503.1741037)
\curveto(478.81818732,503.20409797)(478.80818733,503.22909795)(478.78818947,503.2491037)
\curveto(478.77818736,503.2790979)(478.76818737,503.30409787)(478.75818947,503.3241037)
\curveto(478.71818742,503.3740978)(478.66818747,503.41909776)(478.60818947,503.4591037)
\curveto(478.55818758,503.50909767)(478.50818763,503.55409762)(478.45818947,503.5941037)
\curveto(478.41818772,503.62409755)(478.36818777,503.6790975)(478.30818947,503.7591037)
\curveto(478.28818785,503.78909739)(478.25818788,503.81409736)(478.21818947,503.8341037)
\curveto(478.18818795,503.86409731)(478.16318798,503.89909728)(478.14318947,503.9391037)
\curveto(477.97318817,504.14909703)(477.8431883,504.39409678)(477.75318947,504.6741037)
\curveto(477.73318841,504.75409642)(477.71818842,504.83409634)(477.70818947,504.9141037)
\curveto(477.69818844,504.99409618)(477.68318846,505.0740961)(477.66318947,505.1541037)
\curveto(477.6431885,505.20409597)(477.63318851,505.26909591)(477.63318947,505.3491037)
\curveto(477.63318851,505.43909574)(477.6431885,505.50909567)(477.66318947,505.5591037)
\curveto(477.66318848,505.65909552)(477.66818847,505.72909545)(477.67818947,505.7691037)
\curveto(477.69818844,505.84909533)(477.71318843,505.91909526)(477.72318947,505.9791037)
\curveto(477.73318841,506.04909513)(477.74818839,506.11909506)(477.76818947,506.1891037)
\curveto(477.91818822,506.61909456)(478.13318801,506.96409421)(478.41318947,507.2241037)
\curveto(478.70318744,507.48409369)(479.05318709,507.69909348)(479.46318947,507.8691037)
\curveto(479.57318657,507.91909326)(479.68818645,507.94909323)(479.80818947,507.9591037)
\curveto(479.9381862,507.9790932)(480.06818607,508.00909317)(480.19818947,508.0491037)
\curveto(480.27818586,508.04909313)(480.34818579,508.04909313)(480.40818947,508.0491037)
\curveto(480.47818566,508.05909312)(480.55318559,508.06909311)(480.63318947,508.0791037)
\curveto(481.42318472,508.09909308)(482.07818406,507.96909321)(482.59818947,507.6891037)
\curveto(483.12818301,507.40909377)(483.50818263,506.99909418)(483.73818947,506.4591037)
\curveto(483.84818229,506.22909495)(483.91818222,505.94409523)(483.94818947,505.6041037)
\curveto(483.98818215,505.2740959)(483.95818218,504.96909621)(483.85818947,504.6891037)
\curveto(483.81818232,504.55909662)(483.76818237,504.43909674)(483.70818947,504.3291037)
\curveto(483.65818248,504.21909696)(483.59818254,504.11409706)(483.52818947,504.0141037)
\curveto(483.50818263,503.9740972)(483.47818266,503.93909724)(483.43818947,503.9091037)
\lineto(483.34818947,503.8191037)
\curveto(483.29818284,503.72909745)(483.2381829,503.66409751)(483.16818947,503.6241037)
\curveto(483.11818302,503.5740976)(483.06318308,503.52409765)(483.00318947,503.4741037)
\curveto(482.95318319,503.43409774)(482.90818323,503.38909779)(482.86818947,503.3391037)
\curveto(482.84818329,503.31909786)(482.82818331,503.29409788)(482.80818947,503.2641037)
\curveto(482.79818334,503.24409793)(482.79818334,503.21909796)(482.80818947,503.1891037)
\curveto(482.81818332,503.13909804)(482.84818329,503.08909809)(482.89818947,503.0391037)
\curveto(482.94818319,502.99909818)(483.00318314,502.95909822)(483.06318947,502.9191037)
\lineto(483.24318947,502.7991037)
\curveto(483.30318284,502.76909841)(483.35318279,502.73909844)(483.39318947,502.7091037)
\curveto(483.72318242,502.46909871)(483.97318217,502.15909902)(484.14318947,501.7791037)
\curveto(484.18318196,501.69909948)(484.21318193,501.61409956)(484.23318947,501.5241037)
\curveto(484.26318188,501.43409974)(484.28818185,501.34409983)(484.30818947,501.2541037)
\curveto(484.31818182,501.20409997)(484.32818181,501.14910003)(484.33818947,501.0891037)
\lineto(484.36818947,500.9391037)
\curveto(484.37818176,500.8791003)(484.37818176,500.81410036)(484.36818947,500.7441037)
\curveto(484.35818178,500.68410049)(484.36318178,500.62410055)(484.38318947,500.5641037)
\moveto(478.99818947,505.6041037)
\curveto(478.96818717,505.49409568)(478.96318718,505.35409582)(478.98318947,505.1841037)
\curveto(479.00318714,505.02409615)(479.02818711,504.89909628)(479.05818947,504.8091037)
\curveto(479.16818697,504.48909669)(479.31818682,504.24409693)(479.50818947,504.0741037)
\curveto(479.69818644,503.91409726)(479.96318618,503.78409739)(480.30318947,503.6841037)
\curveto(480.43318571,503.65409752)(480.59818554,503.62909755)(480.79818947,503.6091037)
\curveto(480.99818514,503.59909758)(481.16818497,503.61409756)(481.30818947,503.6541037)
\curveto(481.59818454,503.73409744)(481.8381843,503.84409733)(482.02818947,503.9841037)
\curveto(482.22818391,504.13409704)(482.38318376,504.33409684)(482.49318947,504.5841037)
\curveto(482.51318363,504.63409654)(482.52318362,504.6790965)(482.52318947,504.7191037)
\curveto(482.53318361,504.75909642)(482.54818359,504.80409637)(482.56818947,504.8541037)
\curveto(482.59818354,504.96409621)(482.61818352,505.10409607)(482.62818947,505.2741037)
\curveto(482.6381835,505.44409573)(482.62818351,505.58909559)(482.59818947,505.7091037)
\curveto(482.57818356,505.79909538)(482.55318359,505.88409529)(482.52318947,505.9641037)
\curveto(482.50318364,506.04409513)(482.46818367,506.12409505)(482.41818947,506.2041037)
\curveto(482.24818389,506.4740947)(482.02318412,506.66909451)(481.74318947,506.7891037)
\curveto(481.47318467,506.90909427)(481.11318503,506.96909421)(480.66318947,506.9691037)
\curveto(480.6431855,506.94909423)(480.61318553,506.94409423)(480.57318947,506.9541037)
\curveto(480.53318561,506.96409421)(480.49818564,506.96409421)(480.46818947,506.9541037)
\curveto(480.41818572,506.93409424)(480.36318578,506.91909426)(480.30318947,506.9091037)
\curveto(480.25318589,506.90909427)(480.20318594,506.89909428)(480.15318947,506.8791037)
\curveto(479.91318623,506.78909439)(479.70318644,506.6740945)(479.52318947,506.5341037)
\curveto(479.3431868,506.40409477)(479.20318694,506.22409495)(479.10318947,505.9941037)
\curveto(479.08318706,505.93409524)(479.06318708,505.86909531)(479.04318947,505.7991037)
\curveto(479.03318711,505.73909544)(479.01818712,505.6740955)(478.99818947,505.6041037)
\moveto(483.01818947,500.0691037)
\curveto(483.06818307,500.25910092)(483.07318307,500.46410071)(483.03318947,500.6841037)
\curveto(483.00318314,500.90410027)(482.95818318,501.08410009)(482.89818947,501.2241037)
\curveto(482.72818341,501.59409958)(482.46818367,501.89909928)(482.11818947,502.1391037)
\curveto(481.77818436,502.3790988)(481.3431848,502.49909868)(480.81318947,502.4991037)
\curveto(480.78318536,502.4790987)(480.7431854,502.4740987)(480.69318947,502.4841037)
\curveto(480.6431855,502.50409867)(480.60318554,502.50909867)(480.57318947,502.4991037)
\lineto(480.30318947,502.4391037)
\curveto(480.22318592,502.42909875)(480.143186,502.41409876)(480.06318947,502.3941037)
\curveto(479.76318638,502.28409889)(479.49818664,502.13909904)(479.26818947,501.9591037)
\curveto(479.04818709,501.7790994)(478.87818726,501.54909963)(478.75818947,501.2691037)
\curveto(478.72818741,501.18909999)(478.70318744,501.10910007)(478.68318947,501.0291037)
\curveto(478.66318748,500.94910023)(478.6431875,500.86410031)(478.62318947,500.7741037)
\curveto(478.59318755,500.65410052)(478.58318756,500.50410067)(478.59318947,500.3241037)
\curveto(478.61318753,500.14410103)(478.6381875,500.00410117)(478.66818947,499.9041037)
\curveto(478.68818745,499.85410132)(478.69818744,499.80910137)(478.69818947,499.7691037)
\curveto(478.70818743,499.73910144)(478.72318742,499.69910148)(478.74318947,499.6491037)
\curveto(478.8431873,499.42910175)(478.97318717,499.22910195)(479.13318947,499.0491037)
\curveto(479.30318684,498.86910231)(479.49818664,498.73410244)(479.71818947,498.6441037)
\curveto(479.78818635,498.60410257)(479.88318626,498.56910261)(480.00318947,498.5391037)
\curveto(480.22318592,498.44910273)(480.47818566,498.40410277)(480.76818947,498.4041037)
\lineto(481.05318947,498.4041037)
\curveto(481.15318499,498.42410275)(481.24818489,498.43910274)(481.33818947,498.4491037)
\curveto(481.42818471,498.45910272)(481.51818462,498.4791027)(481.60818947,498.5091037)
\curveto(481.86818427,498.58910259)(482.10818403,498.71910246)(482.32818947,498.8991037)
\curveto(482.55818358,499.08910209)(482.72818341,499.30410187)(482.83818947,499.5441037)
\curveto(482.87818326,499.62410155)(482.90818323,499.70410147)(482.92818947,499.7841037)
\curveto(482.95818318,499.8741013)(482.98818315,499.96910121)(483.01818947,500.0691037)
}
}
{
\newrgbcolor{curcolor}{0 0 0}
\pscustom[linestyle=none,fillstyle=solid,fillcolor=curcolor]
{
\newpath
\moveto(495.52279884,506.0091037)
\curveto(495.32278854,505.71909546)(495.11278875,505.43409574)(494.89279884,505.1541037)
\curveto(494.68278918,504.8740963)(494.47778939,504.58909659)(494.27779884,504.2991037)
\curveto(493.67779019,503.44909773)(493.07279079,502.60909857)(492.46279884,501.7791037)
\curveto(491.85279201,500.95910022)(491.24779262,500.12410105)(490.64779884,499.2741037)
\lineto(490.13779884,498.5541037)
\lineto(489.62779884,497.8641037)
\curveto(489.54779432,497.75410342)(489.4677944,497.63910354)(489.38779884,497.5191037)
\curveto(489.30779456,497.39910378)(489.21279465,497.30410387)(489.10279884,497.2341037)
\curveto(489.0627948,497.21410396)(488.99779487,497.19910398)(488.90779884,497.1891037)
\curveto(488.82779504,497.16910401)(488.73779513,497.15910402)(488.63779884,497.1591037)
\curveto(488.53779533,497.15910402)(488.44279542,497.16410401)(488.35279884,497.1741037)
\curveto(488.27279559,497.18410399)(488.21279565,497.20410397)(488.17279884,497.2341037)
\curveto(488.14279572,497.25410392)(488.11779575,497.28910389)(488.09779884,497.3391037)
\curveto(488.08779578,497.3791038)(488.09279577,497.42410375)(488.11279884,497.4741037)
\curveto(488.15279571,497.55410362)(488.19779567,497.62910355)(488.24779884,497.6991037)
\curveto(488.30779556,497.7791034)(488.3627955,497.85910332)(488.41279884,497.9391037)
\curveto(488.65279521,498.2791029)(488.89779497,498.61410256)(489.14779884,498.9441037)
\curveto(489.39779447,499.2741019)(489.63779423,499.60910157)(489.86779884,499.9491037)
\curveto(490.02779384,500.16910101)(490.18779368,500.38410079)(490.34779884,500.5941037)
\curveto(490.50779336,500.80410037)(490.6677932,501.01910016)(490.82779884,501.2391037)
\curveto(491.18779268,501.75909942)(491.55279231,502.26909891)(491.92279884,502.7691037)
\curveto(492.29279157,503.26909791)(492.6627912,503.7790974)(493.03279884,504.2991037)
\curveto(493.17279069,504.49909668)(493.31279055,504.69409648)(493.45279884,504.8841037)
\curveto(493.60279026,505.0740961)(493.74779012,505.26909591)(493.88779884,505.4691037)
\curveto(494.09778977,505.76909541)(494.31278955,506.06909511)(494.53279884,506.3691037)
\lineto(495.19279884,507.2691037)
\lineto(495.37279884,507.5391037)
\lineto(495.58279884,507.8091037)
\lineto(495.70279884,507.9891037)
\curveto(495.75278811,508.04909313)(495.80278806,508.10409307)(495.85279884,508.1541037)
\curveto(495.92278794,508.20409297)(495.99778787,508.23909294)(496.07779884,508.2591037)
\curveto(496.09778777,508.26909291)(496.12278774,508.26909291)(496.15279884,508.2591037)
\curveto(496.19278767,508.25909292)(496.22278764,508.26909291)(496.24279884,508.2891037)
\curveto(496.3627875,508.28909289)(496.49778737,508.28409289)(496.64779884,508.2741037)
\curveto(496.79778707,508.2740929)(496.88778698,508.22909295)(496.91779884,508.1391037)
\curveto(496.93778693,508.10909307)(496.94278692,508.0740931)(496.93279884,508.0341037)
\curveto(496.92278694,507.99409318)(496.90778696,507.96409321)(496.88779884,507.9441037)
\curveto(496.84778702,507.86409331)(496.80778706,507.79409338)(496.76779884,507.7341037)
\curveto(496.72778714,507.6740935)(496.68278718,507.61409356)(496.63279884,507.5541037)
\lineto(496.06279884,506.7741037)
\curveto(495.88278798,506.52409465)(495.70278816,506.26909491)(495.52279884,506.0091037)
\moveto(488.66779884,502.1091037)
\curveto(488.61779525,502.12909905)(488.5677953,502.13409904)(488.51779884,502.1241037)
\curveto(488.4677954,502.11409906)(488.41779545,502.11909906)(488.36779884,502.1391037)
\curveto(488.25779561,502.15909902)(488.15279571,502.179099)(488.05279884,502.1991037)
\curveto(487.9627959,502.22909895)(487.867796,502.26909891)(487.76779884,502.3191037)
\curveto(487.43779643,502.45909872)(487.18279668,502.65409852)(487.00279884,502.9041037)
\curveto(486.82279704,503.16409801)(486.67779719,503.4740977)(486.56779884,503.8341037)
\curveto(486.53779733,503.91409726)(486.51779735,503.99409718)(486.50779884,504.0741037)
\curveto(486.49779737,504.16409701)(486.48279738,504.24909693)(486.46279884,504.3291037)
\curveto(486.45279741,504.3790968)(486.44779742,504.44409673)(486.44779884,504.5241037)
\curveto(486.43779743,504.55409662)(486.43279743,504.58409659)(486.43279884,504.6141037)
\curveto(486.43279743,504.65409652)(486.42779744,504.68909649)(486.41779884,504.7191037)
\lineto(486.41779884,504.8691037)
\curveto(486.40779746,504.91909626)(486.40279746,504.9790962)(486.40279884,505.0491037)
\curveto(486.40279746,505.12909605)(486.40779746,505.19409598)(486.41779884,505.2441037)
\lineto(486.41779884,505.4091037)
\curveto(486.43779743,505.45909572)(486.44279742,505.50409567)(486.43279884,505.5441037)
\curveto(486.43279743,505.59409558)(486.43779743,505.63909554)(486.44779884,505.6791037)
\curveto(486.45779741,505.71909546)(486.4627974,505.75409542)(486.46279884,505.7841037)
\curveto(486.4627974,505.82409535)(486.4677974,505.86409531)(486.47779884,505.9041037)
\curveto(486.50779736,506.01409516)(486.52779734,506.12409505)(486.53779884,506.2341037)
\curveto(486.55779731,506.35409482)(486.59279727,506.46909471)(486.64279884,506.5791037)
\curveto(486.78279708,506.91909426)(486.94279692,507.19409398)(487.12279884,507.4041037)
\curveto(487.31279655,507.62409355)(487.58279628,507.80409337)(487.93279884,507.9441037)
\curveto(488.01279585,507.9740932)(488.09779577,507.99409318)(488.18779884,508.0041037)
\curveto(488.27779559,508.02409315)(488.37279549,508.04409313)(488.47279884,508.0641037)
\curveto(488.50279536,508.0740931)(488.55779531,508.0740931)(488.63779884,508.0641037)
\curveto(488.71779515,508.06409311)(488.7677951,508.0740931)(488.78779884,508.0941037)
\curveto(489.34779452,508.10409307)(489.79779407,507.99409318)(490.13779884,507.7641037)
\curveto(490.48779338,507.53409364)(490.74779312,507.22909395)(490.91779884,506.8491037)
\curveto(490.95779291,506.75909442)(490.99279287,506.66409451)(491.02279884,506.5641037)
\curveto(491.05279281,506.46409471)(491.07779279,506.36409481)(491.09779884,506.2641037)
\curveto(491.11779275,506.23409494)(491.12279274,506.20409497)(491.11279884,506.1741037)
\curveto(491.11279275,506.14409503)(491.11779275,506.11409506)(491.12779884,506.0841037)
\curveto(491.15779271,505.9740952)(491.17779269,505.84909533)(491.18779884,505.7091037)
\curveto(491.19779267,505.5790956)(491.20779266,505.44409573)(491.21779884,505.3041037)
\lineto(491.21779884,505.1391037)
\curveto(491.22779264,505.0790961)(491.22779264,505.02409615)(491.21779884,504.9741037)
\curveto(491.20779266,504.92409625)(491.20279266,504.8740963)(491.20279884,504.8241037)
\lineto(491.20279884,504.6891037)
\curveto(491.19279267,504.64909653)(491.18779268,504.60909657)(491.18779884,504.5691037)
\curveto(491.19779267,504.52909665)(491.19279267,504.48409669)(491.17279884,504.4341037)
\curveto(491.15279271,504.32409685)(491.13279273,504.21909696)(491.11279884,504.1191037)
\curveto(491.10279276,504.01909716)(491.08279278,503.91909726)(491.05279884,503.8191037)
\curveto(490.92279294,503.45909772)(490.75779311,503.14409803)(490.55779884,502.8741037)
\curveto(490.35779351,502.60409857)(490.08279378,502.39909878)(489.73279884,502.2591037)
\curveto(489.65279421,502.22909895)(489.5677943,502.20409897)(489.47779884,502.1841037)
\lineto(489.20779884,502.1241037)
\curveto(489.15779471,502.11409906)(489.11279475,502.10909907)(489.07279884,502.1091037)
\curveto(489.03279483,502.11909906)(488.99279487,502.11909906)(488.95279884,502.1091037)
\curveto(488.85279501,502.08909909)(488.75779511,502.08909909)(488.66779884,502.1091037)
\moveto(487.82779884,503.5041037)
\curveto(487.867796,503.43409774)(487.90779596,503.36909781)(487.94779884,503.3091037)
\curveto(487.98779588,503.25909792)(488.03779583,503.20909797)(488.09779884,503.1591037)
\lineto(488.24779884,503.0391037)
\curveto(488.30779556,503.00909817)(488.37279549,502.98409819)(488.44279884,502.9641037)
\curveto(488.48279538,502.94409823)(488.51779535,502.93409824)(488.54779884,502.9341037)
\curveto(488.58779528,502.94409823)(488.62779524,502.93909824)(488.66779884,502.9191037)
\curveto(488.69779517,502.91909826)(488.73779513,502.91409826)(488.78779884,502.9041037)
\curveto(488.83779503,502.90409827)(488.87779499,502.90909827)(488.90779884,502.9191037)
\lineto(489.13279884,502.9641037)
\curveto(489.38279448,503.04409813)(489.5677943,503.16909801)(489.68779884,503.3391037)
\curveto(489.7677941,503.43909774)(489.83779403,503.56909761)(489.89779884,503.7291037)
\curveto(489.97779389,503.90909727)(490.03779383,504.13409704)(490.07779884,504.4041037)
\curveto(490.11779375,504.68409649)(490.13279373,504.96409621)(490.12279884,505.2441037)
\curveto(490.11279375,505.53409564)(490.08279378,505.80909537)(490.03279884,506.0691037)
\curveto(489.98279388,506.32909485)(489.90779396,506.53909464)(489.80779884,506.6991037)
\curveto(489.68779418,506.89909428)(489.53779433,507.04909413)(489.35779884,507.1491037)
\curveto(489.27779459,507.19909398)(489.18779468,507.22909395)(489.08779884,507.2391037)
\curveto(488.98779488,507.25909392)(488.88279498,507.26909391)(488.77279884,507.2691037)
\curveto(488.75279511,507.25909392)(488.72779514,507.25409392)(488.69779884,507.2541037)
\curveto(488.67779519,507.26409391)(488.65779521,507.26409391)(488.63779884,507.2541037)
\curveto(488.58779528,507.24409393)(488.54279532,507.23409394)(488.50279884,507.2241037)
\curveto(488.4627954,507.22409395)(488.42279544,507.21409396)(488.38279884,507.1941037)
\curveto(488.20279566,507.11409406)(488.05279581,506.99409418)(487.93279884,506.8341037)
\curveto(487.82279604,506.6740945)(487.73279613,506.49409468)(487.66279884,506.2941037)
\curveto(487.60279626,506.10409507)(487.55779631,505.8790953)(487.52779884,505.6191037)
\curveto(487.50779636,505.35909582)(487.50279636,505.09409608)(487.51279884,504.8241037)
\curveto(487.52279634,504.56409661)(487.55279631,504.31409686)(487.60279884,504.0741037)
\curveto(487.6627962,503.84409733)(487.73779613,503.65409752)(487.82779884,503.5041037)
\moveto(498.62779884,500.5191037)
\curveto(498.63778523,500.46910071)(498.64278522,500.3791008)(498.64279884,500.2491037)
\curveto(498.64278522,500.11910106)(498.63278523,500.02910115)(498.61279884,499.9791037)
\curveto(498.59278527,499.92910125)(498.58778528,499.8741013)(498.59779884,499.8141037)
\curveto(498.60778526,499.76410141)(498.60778526,499.71410146)(498.59779884,499.6641037)
\curveto(498.55778531,499.52410165)(498.52778534,499.38910179)(498.50779884,499.2591037)
\curveto(498.49778537,499.12910205)(498.4677854,499.00910217)(498.41779884,498.8991037)
\curveto(498.27778559,498.54910263)(498.11278575,498.25410292)(497.92279884,498.0141037)
\curveto(497.73278613,497.78410339)(497.4627864,497.59910358)(497.11279884,497.4591037)
\curveto(497.03278683,497.42910375)(496.94778692,497.40910377)(496.85779884,497.3991037)
\curveto(496.7677871,497.3791038)(496.68278718,497.35910382)(496.60279884,497.3391037)
\curveto(496.55278731,497.32910385)(496.50278736,497.32410385)(496.45279884,497.3241037)
\curveto(496.40278746,497.32410385)(496.35278751,497.31910386)(496.30279884,497.3091037)
\curveto(496.27278759,497.29910388)(496.22278764,497.29910388)(496.15279884,497.3091037)
\curveto(496.08278778,497.30910387)(496.03278783,497.31410386)(496.00279884,497.3241037)
\curveto(495.94278792,497.34410383)(495.88278798,497.35410382)(495.82279884,497.3541037)
\curveto(495.77278809,497.34410383)(495.72278814,497.34910383)(495.67279884,497.3691037)
\curveto(495.58278828,497.38910379)(495.49278837,497.41410376)(495.40279884,497.4441037)
\curveto(495.32278854,497.46410371)(495.24278862,497.49410368)(495.16279884,497.5341037)
\curveto(494.84278902,497.6741035)(494.59278927,497.86910331)(494.41279884,498.1191037)
\curveto(494.23278963,498.3791028)(494.08278978,498.68410249)(493.96279884,499.0341037)
\curveto(493.94278992,499.11410206)(493.92778994,499.19910198)(493.91779884,499.2891037)
\curveto(493.90778996,499.3791018)(493.89278997,499.46410171)(493.87279884,499.5441037)
\curveto(493.86279,499.5741016)(493.85779001,499.60410157)(493.85779884,499.6341037)
\lineto(493.85779884,499.7391037)
\curveto(493.83779003,499.81910136)(493.82779004,499.89910128)(493.82779884,499.9791037)
\lineto(493.82779884,500.1141037)
\curveto(493.80779006,500.21410096)(493.80779006,500.31410086)(493.82779884,500.4141037)
\lineto(493.82779884,500.5941037)
\curveto(493.83779003,500.64410053)(493.84279002,500.68910049)(493.84279884,500.7291037)
\curveto(493.84279002,500.7791004)(493.84779002,500.82410035)(493.85779884,500.8641037)
\curveto(493.86779,500.90410027)(493.87278999,500.93910024)(493.87279884,500.9691037)
\curveto(493.87278999,501.00910017)(493.87778999,501.04910013)(493.88779884,501.0891037)
\lineto(493.94779884,501.4191037)
\curveto(493.9677899,501.53909964)(493.99778987,501.64909953)(494.03779884,501.7491037)
\curveto(494.17778969,502.0790991)(494.33778953,502.35409882)(494.51779884,502.5741037)
\curveto(494.70778916,502.80409837)(494.9677889,502.98909819)(495.29779884,503.1291037)
\curveto(495.37778849,503.16909801)(495.4627884,503.19409798)(495.55279884,503.2041037)
\lineto(495.85279884,503.2641037)
\lineto(495.98779884,503.2641037)
\curveto(496.03778783,503.2740979)(496.08778778,503.2790979)(496.13779884,503.2791037)
\curveto(496.70778716,503.29909788)(497.1677867,503.19409798)(497.51779884,502.9641037)
\curveto(497.87778599,502.74409843)(498.14278572,502.44409873)(498.31279884,502.0641037)
\curveto(498.3627855,501.96409921)(498.40278546,501.86409931)(498.43279884,501.7641037)
\curveto(498.4627854,501.66409951)(498.49278537,501.55909962)(498.52279884,501.4491037)
\curveto(498.53278533,501.40909977)(498.53778533,501.3740998)(498.53779884,501.3441037)
\curveto(498.53778533,501.32409985)(498.54278532,501.29409988)(498.55279884,501.2541037)
\curveto(498.57278529,501.18409999)(498.58278528,501.10910007)(498.58279884,501.0291037)
\curveto(498.58278528,500.94910023)(498.59278527,500.86910031)(498.61279884,500.7891037)
\curveto(498.61278525,500.73910044)(498.61278525,500.69410048)(498.61279884,500.6541037)
\curveto(498.61278525,500.61410056)(498.61778525,500.56910061)(498.62779884,500.5191037)
\moveto(497.51779884,500.0841037)
\curveto(497.52778634,500.13410104)(497.53278633,500.20910097)(497.53279884,500.3091037)
\curveto(497.54278632,500.40910077)(497.53778633,500.48410069)(497.51779884,500.5341037)
\curveto(497.49778637,500.59410058)(497.49278637,500.64910053)(497.50279884,500.6991037)
\curveto(497.52278634,500.75910042)(497.52278634,500.81910036)(497.50279884,500.8791037)
\curveto(497.49278637,500.90910027)(497.48778638,500.94410023)(497.48779884,500.9841037)
\curveto(497.48778638,501.02410015)(497.48278638,501.06410011)(497.47279884,501.1041037)
\curveto(497.45278641,501.18409999)(497.43278643,501.25909992)(497.41279884,501.3291037)
\curveto(497.40278646,501.40909977)(497.38778648,501.48909969)(497.36779884,501.5691037)
\curveto(497.33778653,501.62909955)(497.31278655,501.68909949)(497.29279884,501.7491037)
\curveto(497.27278659,501.80909937)(497.24278662,501.86909931)(497.20279884,501.9291037)
\curveto(497.10278676,502.09909908)(496.97278689,502.23409894)(496.81279884,502.3341037)
\curveto(496.73278713,502.38409879)(496.63778723,502.41909876)(496.52779884,502.4391037)
\curveto(496.41778745,502.45909872)(496.29278757,502.46909871)(496.15279884,502.4691037)
\curveto(496.13278773,502.45909872)(496.10778776,502.45409872)(496.07779884,502.4541037)
\curveto(496.04778782,502.46409871)(496.01778785,502.46409871)(495.98779884,502.4541037)
\lineto(495.83779884,502.3941037)
\curveto(495.78778808,502.38409879)(495.74278812,502.36909881)(495.70279884,502.3491037)
\curveto(495.51278835,502.23909894)(495.3677885,502.09409908)(495.26779884,501.9141037)
\curveto(495.17778869,501.73409944)(495.09778877,501.52909965)(495.02779884,501.2991037)
\curveto(494.98778888,501.16910001)(494.9677889,501.03410014)(494.96779884,500.8941037)
\curveto(494.9677889,500.76410041)(494.95778891,500.61910056)(494.93779884,500.4591037)
\curveto(494.92778894,500.40910077)(494.91778895,500.34910083)(494.90779884,500.2791037)
\curveto(494.90778896,500.20910097)(494.91778895,500.14910103)(494.93779884,500.0991037)
\lineto(494.93779884,499.9341037)
\lineto(494.93779884,499.7541037)
\curveto(494.94778892,499.70410147)(494.95778891,499.64910153)(494.96779884,499.5891037)
\curveto(494.97778889,499.53910164)(494.98278888,499.48410169)(494.98279884,499.4241037)
\curveto(494.99278887,499.36410181)(495.00778886,499.30910187)(495.02779884,499.2591037)
\curveto(495.07778879,499.06910211)(495.13778873,498.89410228)(495.20779884,498.7341037)
\curveto(495.27778859,498.5741026)(495.38278848,498.44410273)(495.52279884,498.3441037)
\curveto(495.65278821,498.24410293)(495.79278807,498.174103)(495.94279884,498.1341037)
\curveto(495.97278789,498.12410305)(495.99778787,498.11910306)(496.01779884,498.1191037)
\curveto(496.04778782,498.12910305)(496.07778779,498.12910305)(496.10779884,498.1191037)
\curveto(496.12778774,498.11910306)(496.15778771,498.11410306)(496.19779884,498.1041037)
\curveto(496.23778763,498.10410307)(496.27278759,498.10910307)(496.30279884,498.1191037)
\curveto(496.34278752,498.12910305)(496.38278748,498.13410304)(496.42279884,498.1341037)
\curveto(496.4627874,498.13410304)(496.50278736,498.14410303)(496.54279884,498.1641037)
\curveto(496.78278708,498.24410293)(496.97778689,498.3791028)(497.12779884,498.5691037)
\curveto(497.24778662,498.74910243)(497.33778653,498.95410222)(497.39779884,499.1841037)
\curveto(497.41778645,499.25410192)(497.43278643,499.32410185)(497.44279884,499.3941037)
\curveto(497.45278641,499.4741017)(497.4677864,499.55410162)(497.48779884,499.6341037)
\curveto(497.48778638,499.69410148)(497.49278637,499.73910144)(497.50279884,499.7691037)
\curveto(497.50278636,499.78910139)(497.50278636,499.81410136)(497.50279884,499.8441037)
\curveto(497.50278636,499.88410129)(497.50778636,499.91410126)(497.51779884,499.9341037)
\lineto(497.51779884,500.0841037)
}
}
{
\newrgbcolor{curcolor}{0 0 0}
\pscustom[linestyle=none,fillstyle=solid,fillcolor=curcolor]
{
\newpath
\moveto(477.09519509,566.99260102)
\curveto(477.10518737,566.95259797)(477.10518737,566.90259802)(477.09519509,566.84260102)
\curveto(477.09518738,566.78259814)(477.09018738,566.73259819)(477.08019509,566.69260102)
\curveto(477.08018739,566.65259827)(477.0751874,566.61259831)(477.06519509,566.57260102)
\lineto(477.06519509,566.46760102)
\curveto(477.04518743,566.38759853)(477.03018744,566.30759861)(477.02019509,566.22760102)
\curveto(477.01018746,566.14759877)(476.99018748,566.07259885)(476.96019509,566.00260102)
\curveto(476.94018753,565.922599)(476.92018755,565.84759907)(476.90019509,565.77760102)
\curveto(476.88018759,565.70759921)(476.85018762,565.63259929)(476.81019509,565.55260102)
\curveto(476.63018784,565.13259979)(476.3751881,564.79260013)(476.04519509,564.53260102)
\curveto(475.71518876,564.27260065)(475.32518915,564.06760085)(474.87519509,563.91760102)
\curveto(474.75518972,563.87760104)(474.63018984,563.85260107)(474.50019509,563.84260102)
\curveto(474.38019009,563.8226011)(474.25519022,563.79760112)(474.12519509,563.76760102)
\curveto(474.06519041,563.75760116)(474.00019047,563.75260117)(473.93019509,563.75260102)
\curveto(473.8701906,563.75260117)(473.80519067,563.74760117)(473.73519509,563.73760102)
\lineto(473.61519509,563.73760102)
\lineto(473.42019509,563.73760102)
\curveto(473.36019111,563.72760119)(473.30519117,563.73260119)(473.25519509,563.75260102)
\curveto(473.18519129,563.77260115)(473.12019135,563.77760114)(473.06019509,563.76760102)
\curveto(473.00019147,563.75760116)(472.94019153,563.76260116)(472.88019509,563.78260102)
\curveto(472.83019164,563.79260113)(472.78519169,563.79760112)(472.74519509,563.79760102)
\curveto(472.70519177,563.79760112)(472.66019181,563.80760111)(472.61019509,563.82760102)
\curveto(472.53019194,563.84760107)(472.45519202,563.86760105)(472.38519509,563.88760102)
\curveto(472.31519216,563.89760102)(472.24519223,563.91260101)(472.17519509,563.93260102)
\curveto(471.69519278,564.10260082)(471.29519318,564.31260061)(470.97519509,564.56260102)
\curveto(470.66519381,564.8226001)(470.41519406,565.17759974)(470.22519509,565.62760102)
\curveto(470.19519428,565.68759923)(470.1701943,565.74759917)(470.15019509,565.80760102)
\curveto(470.14019433,565.87759904)(470.12519435,565.95259897)(470.10519509,566.03260102)
\curveto(470.08519439,566.09259883)(470.0701944,566.15759876)(470.06019509,566.22760102)
\curveto(470.05019442,566.29759862)(470.03519444,566.36759855)(470.01519509,566.43760102)
\curveto(470.00519447,566.48759843)(470.00019447,566.52759839)(470.00019509,566.55760102)
\lineto(470.00019509,566.67760102)
\curveto(469.99019448,566.7175982)(469.98019449,566.76759815)(469.97019509,566.82760102)
\curveto(469.9701945,566.88759803)(469.9751945,566.93759798)(469.98519509,566.97760102)
\lineto(469.98519509,567.11260102)
\curveto(469.99519448,567.16259776)(470.00019447,567.21259771)(470.00019509,567.26260102)
\curveto(470.02019445,567.36259756)(470.03519444,567.45759746)(470.04519509,567.54760102)
\curveto(470.05519442,567.64759727)(470.0751944,567.74259718)(470.10519509,567.83260102)
\curveto(470.15519432,567.98259694)(470.21019426,568.1225968)(470.27019509,568.25260102)
\curveto(470.33019414,568.38259654)(470.40019407,568.50259642)(470.48019509,568.61260102)
\curveto(470.51019396,568.66259626)(470.54019393,568.70259622)(470.57019509,568.73260102)
\curveto(470.61019386,568.76259616)(470.64519383,568.79759612)(470.67519509,568.83760102)
\curveto(470.73519374,568.917596)(470.80519367,568.98759593)(470.88519509,569.04760102)
\curveto(470.94519353,569.09759582)(471.00519347,569.14259578)(471.06519509,569.18260102)
\lineto(471.27519509,569.33260102)
\curveto(471.32519315,569.37259555)(471.3751931,569.40759551)(471.42519509,569.43760102)
\curveto(471.475193,569.47759544)(471.51019296,569.53259539)(471.53019509,569.60260102)
\curveto(471.53019294,569.63259529)(471.52019295,569.65759526)(471.50019509,569.67760102)
\curveto(471.49019298,569.70759521)(471.48019299,569.73259519)(471.47019509,569.75260102)
\curveto(471.43019304,569.80259512)(471.38019309,569.84759507)(471.32019509,569.88760102)
\curveto(471.2701932,569.93759498)(471.22019325,569.98259494)(471.17019509,570.02260102)
\curveto(471.13019334,570.05259487)(471.08019339,570.10759481)(471.02019509,570.18760102)
\curveto(471.00019347,570.2175947)(470.9701935,570.24259468)(470.93019509,570.26260102)
\curveto(470.90019357,570.29259463)(470.8751936,570.32759459)(470.85519509,570.36760102)
\curveto(470.68519379,570.57759434)(470.55519392,570.8225941)(470.46519509,571.10260102)
\curveto(470.44519403,571.18259374)(470.43019404,571.26259366)(470.42019509,571.34260102)
\curveto(470.41019406,571.4225935)(470.39519408,571.50259342)(470.37519509,571.58260102)
\curveto(470.35519412,571.63259329)(470.34519413,571.69759322)(470.34519509,571.77760102)
\curveto(470.34519413,571.86759305)(470.35519412,571.93759298)(470.37519509,571.98760102)
\curveto(470.3751941,572.08759283)(470.38019409,572.15759276)(470.39019509,572.19760102)
\curveto(470.41019406,572.27759264)(470.42519405,572.34759257)(470.43519509,572.40760102)
\curveto(470.44519403,572.47759244)(470.46019401,572.54759237)(470.48019509,572.61760102)
\curveto(470.63019384,573.04759187)(470.84519363,573.39259153)(471.12519509,573.65260102)
\curveto(471.41519306,573.91259101)(471.76519271,574.12759079)(472.17519509,574.29760102)
\curveto(472.28519219,574.34759057)(472.40019207,574.37759054)(472.52019509,574.38760102)
\curveto(472.65019182,574.40759051)(472.78019169,574.43759048)(472.91019509,574.47760102)
\curveto(472.99019148,574.47759044)(473.06019141,574.47759044)(473.12019509,574.47760102)
\curveto(473.19019128,574.48759043)(473.26519121,574.49759042)(473.34519509,574.50760102)
\curveto(474.13519034,574.52759039)(474.79018968,574.39759052)(475.31019509,574.11760102)
\curveto(475.84018863,573.83759108)(476.22018825,573.42759149)(476.45019509,572.88760102)
\curveto(476.56018791,572.65759226)(476.63018784,572.37259255)(476.66019509,572.03260102)
\curveto(476.70018777,571.70259322)(476.6701878,571.39759352)(476.57019509,571.11760102)
\curveto(476.53018794,570.98759393)(476.48018799,570.86759405)(476.42019509,570.75760102)
\curveto(476.3701881,570.64759427)(476.31018816,570.54259438)(476.24019509,570.44260102)
\curveto(476.22018825,570.40259452)(476.19018828,570.36759455)(476.15019509,570.33760102)
\lineto(476.06019509,570.24760102)
\curveto(476.01018846,570.15759476)(475.95018852,570.09259483)(475.88019509,570.05260102)
\curveto(475.83018864,570.00259492)(475.7751887,569.95259497)(475.71519509,569.90260102)
\curveto(475.66518881,569.86259506)(475.62018885,569.8175951)(475.58019509,569.76760102)
\curveto(475.56018891,569.74759517)(475.54018893,569.7225952)(475.52019509,569.69260102)
\curveto(475.51018896,569.67259525)(475.51018896,569.64759527)(475.52019509,569.61760102)
\curveto(475.53018894,569.56759535)(475.56018891,569.5175954)(475.61019509,569.46760102)
\curveto(475.66018881,569.42759549)(475.71518876,569.38759553)(475.77519509,569.34760102)
\lineto(475.95519509,569.22760102)
\curveto(476.01518846,569.19759572)(476.06518841,569.16759575)(476.10519509,569.13760102)
\curveto(476.43518804,568.89759602)(476.68518779,568.58759633)(476.85519509,568.20760102)
\curveto(476.89518758,568.12759679)(476.92518755,568.04259688)(476.94519509,567.95260102)
\curveto(476.9751875,567.86259706)(477.00018747,567.77259715)(477.02019509,567.68260102)
\curveto(477.03018744,567.63259729)(477.04018743,567.57759734)(477.05019509,567.51760102)
\lineto(477.08019509,567.36760102)
\curveto(477.09018738,567.30759761)(477.09018738,567.24259768)(477.08019509,567.17260102)
\curveto(477.0701874,567.11259781)(477.0751874,567.05259787)(477.09519509,566.99260102)
\moveto(471.71019509,572.03260102)
\curveto(471.68019279,571.922593)(471.6751928,571.78259314)(471.69519509,571.61260102)
\curveto(471.71519276,571.45259347)(471.74019273,571.32759359)(471.77019509,571.23760102)
\curveto(471.88019259,570.917594)(472.03019244,570.67259425)(472.22019509,570.50260102)
\curveto(472.41019206,570.34259458)(472.6751918,570.21259471)(473.01519509,570.11260102)
\curveto(473.14519133,570.08259484)(473.31019116,570.05759486)(473.51019509,570.03760102)
\curveto(473.71019076,570.02759489)(473.88019059,570.04259488)(474.02019509,570.08260102)
\curveto(474.31019016,570.16259476)(474.55018992,570.27259465)(474.74019509,570.41260102)
\curveto(474.94018953,570.56259436)(475.09518938,570.76259416)(475.20519509,571.01260102)
\curveto(475.22518925,571.06259386)(475.23518924,571.10759381)(475.23519509,571.14760102)
\curveto(475.24518923,571.18759373)(475.26018921,571.23259369)(475.28019509,571.28260102)
\curveto(475.31018916,571.39259353)(475.33018914,571.53259339)(475.34019509,571.70260102)
\curveto(475.35018912,571.87259305)(475.34018913,572.0175929)(475.31019509,572.13760102)
\curveto(475.29018918,572.22759269)(475.26518921,572.31259261)(475.23519509,572.39260102)
\curveto(475.21518926,572.47259245)(475.18018929,572.55259237)(475.13019509,572.63260102)
\curveto(474.96018951,572.90259202)(474.73518974,573.09759182)(474.45519509,573.21760102)
\curveto(474.18519029,573.33759158)(473.82519065,573.39759152)(473.37519509,573.39760102)
\curveto(473.35519112,573.37759154)(473.32519115,573.37259155)(473.28519509,573.38260102)
\curveto(473.24519123,573.39259153)(473.21019126,573.39259153)(473.18019509,573.38260102)
\curveto(473.13019134,573.36259156)(473.0751914,573.34759157)(473.01519509,573.33760102)
\curveto(472.96519151,573.33759158)(472.91519156,573.32759159)(472.86519509,573.30760102)
\curveto(472.62519185,573.2175917)(472.41519206,573.10259182)(472.23519509,572.96260102)
\curveto(472.05519242,572.83259209)(471.91519256,572.65259227)(471.81519509,572.42260102)
\curveto(471.79519268,572.36259256)(471.7751927,572.29759262)(471.75519509,572.22760102)
\curveto(471.74519273,572.16759275)(471.73019274,572.10259282)(471.71019509,572.03260102)
\moveto(475.73019509,566.49760102)
\curveto(475.78018869,566.68759823)(475.78518869,566.89259803)(475.74519509,567.11260102)
\curveto(475.71518876,567.33259759)(475.6701888,567.51259741)(475.61019509,567.65260102)
\curveto(475.44018903,568.0225969)(475.18018929,568.32759659)(474.83019509,568.56760102)
\curveto(474.49018998,568.80759611)(474.05519042,568.92759599)(473.52519509,568.92760102)
\curveto(473.49519098,568.90759601)(473.45519102,568.90259602)(473.40519509,568.91260102)
\curveto(473.35519112,568.93259599)(473.31519116,568.93759598)(473.28519509,568.92760102)
\lineto(473.01519509,568.86760102)
\curveto(472.93519154,568.85759606)(472.85519162,568.84259608)(472.77519509,568.82260102)
\curveto(472.475192,568.71259621)(472.21019226,568.56759635)(471.98019509,568.38760102)
\curveto(471.76019271,568.20759671)(471.59019288,567.97759694)(471.47019509,567.69760102)
\curveto(471.44019303,567.6175973)(471.41519306,567.53759738)(471.39519509,567.45760102)
\curveto(471.3751931,567.37759754)(471.35519312,567.29259763)(471.33519509,567.20260102)
\curveto(471.30519317,567.08259784)(471.29519318,566.93259799)(471.30519509,566.75260102)
\curveto(471.32519315,566.57259835)(471.35019312,566.43259849)(471.38019509,566.33260102)
\curveto(471.40019307,566.28259864)(471.41019306,566.23759868)(471.41019509,566.19760102)
\curveto(471.42019305,566.16759875)(471.43519304,566.12759879)(471.45519509,566.07760102)
\curveto(471.55519292,565.85759906)(471.68519279,565.65759926)(471.84519509,565.47760102)
\curveto(472.01519246,565.29759962)(472.21019226,565.16259976)(472.43019509,565.07260102)
\curveto(472.50019197,565.03259989)(472.59519188,564.99759992)(472.71519509,564.96760102)
\curveto(472.93519154,564.87760004)(473.19019128,564.83260009)(473.48019509,564.83260102)
\lineto(473.76519509,564.83260102)
\curveto(473.86519061,564.85260007)(473.96019051,564.86760005)(474.05019509,564.87760102)
\curveto(474.14019033,564.88760003)(474.23019024,564.90760001)(474.32019509,564.93760102)
\curveto(474.58018989,565.0175999)(474.82018965,565.14759977)(475.04019509,565.32760102)
\curveto(475.2701892,565.5175994)(475.44018903,565.73259919)(475.55019509,565.97260102)
\curveto(475.59018888,566.05259887)(475.62018885,566.13259879)(475.64019509,566.21260102)
\curveto(475.6701888,566.30259862)(475.70018877,566.39759852)(475.73019509,566.49760102)
}
}
{
\newrgbcolor{curcolor}{0 0 0}
\pscustom[linestyle=none,fillstyle=solid,fillcolor=curcolor]
{
\newpath
\moveto(479.38480446,565.55260102)
\lineto(479.68480446,565.55260102)
\curveto(479.7948024,565.56259936)(479.8998023,565.56259936)(479.99980446,565.55260102)
\curveto(480.10980209,565.55259937)(480.20980199,565.54259938)(480.29980446,565.52260102)
\curveto(480.38980181,565.51259941)(480.45980174,565.48759943)(480.50980446,565.44760102)
\curveto(480.52980167,565.42759949)(480.54480165,565.39759952)(480.55480446,565.35760102)
\curveto(480.57480162,565.3175996)(480.5948016,565.27259965)(480.61480446,565.22260102)
\lineto(480.61480446,565.14760102)
\curveto(480.62480157,565.09759982)(480.62480157,565.04259988)(480.61480446,564.98260102)
\lineto(480.61480446,564.83260102)
\lineto(480.61480446,564.35260102)
\curveto(480.61480158,564.18260074)(480.57480162,564.06260086)(480.49480446,563.99260102)
\curveto(480.42480177,563.94260098)(480.33480186,563.917601)(480.22480446,563.91760102)
\lineto(479.89480446,563.91760102)
\lineto(479.44480446,563.91760102)
\curveto(479.2948029,563.917601)(479.17980302,563.94760097)(479.09980446,564.00760102)
\curveto(479.05980314,564.03760088)(479.02980317,564.08760083)(479.00980446,564.15760102)
\curveto(478.98980321,564.23760068)(478.97480322,564.3226006)(478.96480446,564.41260102)
\lineto(478.96480446,564.69760102)
\curveto(478.97480322,564.79760012)(478.97980322,564.88260004)(478.97980446,564.95260102)
\lineto(478.97980446,565.14760102)
\curveto(478.97980322,565.20759971)(478.98980321,565.26259966)(479.00980446,565.31260102)
\curveto(479.04980315,565.4225995)(479.11980308,565.49259943)(479.21980446,565.52260102)
\curveto(479.24980295,565.5225994)(479.30480289,565.53259939)(479.38480446,565.55260102)
}
}
{
\newrgbcolor{curcolor}{0 0 0}
\pscustom[linestyle=none,fillstyle=solid,fillcolor=curcolor]
{
\newpath
\moveto(486.64996071,574.52260102)
\curveto(486.74995586,574.5225904)(486.84495576,574.51259041)(486.93496071,574.49260102)
\curveto(487.02495558,574.48259044)(487.08995552,574.45259047)(487.12996071,574.40260102)
\curveto(487.18995542,574.3225906)(487.21995539,574.2175907)(487.21996071,574.08760102)
\lineto(487.21996071,573.69760102)
\lineto(487.21996071,572.19760102)
\lineto(487.21996071,565.80760102)
\lineto(487.21996071,564.63760102)
\lineto(487.21996071,564.32260102)
\curveto(487.22995538,564.2226007)(487.21495539,564.14260078)(487.17496071,564.08260102)
\curveto(487.12495548,564.00260092)(487.04995556,563.95260097)(486.94996071,563.93260102)
\curveto(486.85995575,563.922601)(486.74995586,563.917601)(486.61996071,563.91760102)
\lineto(486.39496071,563.91760102)
\curveto(486.31495629,563.93760098)(486.24495636,563.95260097)(486.18496071,563.96260102)
\curveto(486.12495648,563.98260094)(486.07495653,564.0226009)(486.03496071,564.08260102)
\curveto(485.99495661,564.14260078)(485.97495663,564.2176007)(485.97496071,564.30760102)
\lineto(485.97496071,564.60760102)
\lineto(485.97496071,565.70260102)
\lineto(485.97496071,571.04260102)
\curveto(485.95495665,571.13259379)(485.93995667,571.20759371)(485.92996071,571.26760102)
\curveto(485.92995668,571.33759358)(485.89995671,571.39759352)(485.83996071,571.44760102)
\curveto(485.76995684,571.49759342)(485.67995693,571.5225934)(485.56996071,571.52260102)
\curveto(485.46995714,571.53259339)(485.35995725,571.53759338)(485.23996071,571.53760102)
\lineto(484.09996071,571.53760102)
\lineto(483.60496071,571.53760102)
\curveto(483.44495916,571.54759337)(483.33495927,571.60759331)(483.27496071,571.71760102)
\curveto(483.25495935,571.74759317)(483.24495936,571.77759314)(483.24496071,571.80760102)
\curveto(483.24495936,571.84759307)(483.23995937,571.89259303)(483.22996071,571.94260102)
\curveto(483.2099594,572.06259286)(483.21495939,572.17259275)(483.24496071,572.27260102)
\curveto(483.28495932,572.37259255)(483.33995927,572.44259248)(483.40996071,572.48260102)
\curveto(483.48995912,572.53259239)(483.609959,572.55759236)(483.76996071,572.55760102)
\curveto(483.92995868,572.55759236)(484.06495854,572.57259235)(484.17496071,572.60260102)
\curveto(484.22495838,572.61259231)(484.27995833,572.6175923)(484.33996071,572.61760102)
\curveto(484.39995821,572.62759229)(484.45995815,572.64259228)(484.51996071,572.66260102)
\curveto(484.66995794,572.71259221)(484.81495779,572.76259216)(484.95496071,572.81260102)
\curveto(485.09495751,572.87259205)(485.22995738,572.94259198)(485.35996071,573.02260102)
\curveto(485.49995711,573.11259181)(485.61995699,573.2175917)(485.71996071,573.33760102)
\curveto(485.81995679,573.45759146)(485.91495669,573.58759133)(486.00496071,573.72760102)
\curveto(486.06495654,573.82759109)(486.1099565,573.93759098)(486.13996071,574.05760102)
\curveto(486.17995643,574.17759074)(486.22995638,574.28259064)(486.28996071,574.37260102)
\curveto(486.33995627,574.43259049)(486.4099562,574.47259045)(486.49996071,574.49260102)
\curveto(486.51995609,574.50259042)(486.54495606,574.50759041)(486.57496071,574.50760102)
\curveto(486.604956,574.50759041)(486.62995598,574.51259041)(486.64996071,574.52260102)
}
}
{
\newrgbcolor{curcolor}{0 0 0}
\pscustom[linestyle=none,fillstyle=solid,fillcolor=curcolor]
{
\newpath
\moveto(500.74457009,572.43760102)
\curveto(500.54455979,572.14759277)(500.33456,571.86259306)(500.11457009,571.58260102)
\curveto(499.90456043,571.30259362)(499.69956063,571.0175939)(499.49957009,570.72760102)
\curveto(498.89956143,569.87759504)(498.29456204,569.03759588)(497.68457009,568.20760102)
\curveto(497.07456326,567.38759753)(496.46956386,566.55259837)(495.86957009,565.70260102)
\lineto(495.35957009,564.98260102)
\lineto(494.84957009,564.29260102)
\curveto(494.76956556,564.18260074)(494.68956564,564.06760085)(494.60957009,563.94760102)
\curveto(494.5295658,563.82760109)(494.4345659,563.73260119)(494.32457009,563.66260102)
\curveto(494.28456605,563.64260128)(494.21956611,563.62760129)(494.12957009,563.61760102)
\curveto(494.04956628,563.59760132)(493.95956637,563.58760133)(493.85957009,563.58760102)
\curveto(493.75956657,563.58760133)(493.66456667,563.59260133)(493.57457009,563.60260102)
\curveto(493.49456684,563.61260131)(493.4345669,563.63260129)(493.39457009,563.66260102)
\curveto(493.36456697,563.68260124)(493.33956699,563.7176012)(493.31957009,563.76760102)
\curveto(493.30956702,563.80760111)(493.31456702,563.85260107)(493.33457009,563.90260102)
\curveto(493.37456696,563.98260094)(493.41956691,564.05760086)(493.46957009,564.12760102)
\curveto(493.5295668,564.20760071)(493.58456675,564.28760063)(493.63457009,564.36760102)
\curveto(493.87456646,564.70760021)(494.11956621,565.04259988)(494.36957009,565.37260102)
\curveto(494.61956571,565.70259922)(494.85956547,566.03759888)(495.08957009,566.37760102)
\curveto(495.24956508,566.59759832)(495.40956492,566.81259811)(495.56957009,567.02260102)
\curveto(495.7295646,567.23259769)(495.88956444,567.44759747)(496.04957009,567.66760102)
\curveto(496.40956392,568.18759673)(496.77456356,568.69759622)(497.14457009,569.19760102)
\curveto(497.51456282,569.69759522)(497.88456245,570.20759471)(498.25457009,570.72760102)
\curveto(498.39456194,570.92759399)(498.5345618,571.1225938)(498.67457009,571.31260102)
\curveto(498.82456151,571.50259342)(498.96956136,571.69759322)(499.10957009,571.89760102)
\curveto(499.31956101,572.19759272)(499.5345608,572.49759242)(499.75457009,572.79760102)
\lineto(500.41457009,573.69760102)
\lineto(500.59457009,573.96760102)
\lineto(500.80457009,574.23760102)
\lineto(500.92457009,574.41760102)
\curveto(500.97455936,574.47759044)(501.02455931,574.53259039)(501.07457009,574.58260102)
\curveto(501.14455919,574.63259029)(501.21955911,574.66759025)(501.29957009,574.68760102)
\curveto(501.31955901,574.69759022)(501.34455899,574.69759022)(501.37457009,574.68760102)
\curveto(501.41455892,574.68759023)(501.44455889,574.69759022)(501.46457009,574.71760102)
\curveto(501.58455875,574.7175902)(501.71955861,574.71259021)(501.86957009,574.70260102)
\curveto(502.01955831,574.70259022)(502.10955822,574.65759026)(502.13957009,574.56760102)
\curveto(502.15955817,574.53759038)(502.16455817,574.50259042)(502.15457009,574.46260102)
\curveto(502.14455819,574.4225905)(502.1295582,574.39259053)(502.10957009,574.37260102)
\curveto(502.06955826,574.29259063)(502.0295583,574.2225907)(501.98957009,574.16260102)
\curveto(501.94955838,574.10259082)(501.90455843,574.04259088)(501.85457009,573.98260102)
\lineto(501.28457009,573.20260102)
\curveto(501.10455923,572.95259197)(500.92455941,572.69759222)(500.74457009,572.43760102)
\moveto(493.88957009,568.53760102)
\curveto(493.83956649,568.55759636)(493.78956654,568.56259636)(493.73957009,568.55260102)
\curveto(493.68956664,568.54259638)(493.63956669,568.54759637)(493.58957009,568.56760102)
\curveto(493.47956685,568.58759633)(493.37456696,568.60759631)(493.27457009,568.62760102)
\curveto(493.18456715,568.65759626)(493.08956724,568.69759622)(492.98957009,568.74760102)
\curveto(492.65956767,568.88759603)(492.40456793,569.08259584)(492.22457009,569.33260102)
\curveto(492.04456829,569.59259533)(491.89956843,569.90259502)(491.78957009,570.26260102)
\curveto(491.75956857,570.34259458)(491.73956859,570.4225945)(491.72957009,570.50260102)
\curveto(491.71956861,570.59259433)(491.70456863,570.67759424)(491.68457009,570.75760102)
\curveto(491.67456866,570.80759411)(491.66956866,570.87259405)(491.66957009,570.95260102)
\curveto(491.65956867,570.98259394)(491.65456868,571.01259391)(491.65457009,571.04260102)
\curveto(491.65456868,571.08259384)(491.64956868,571.1175938)(491.63957009,571.14760102)
\lineto(491.63957009,571.29760102)
\curveto(491.6295687,571.34759357)(491.62456871,571.40759351)(491.62457009,571.47760102)
\curveto(491.62456871,571.55759336)(491.6295687,571.6225933)(491.63957009,571.67260102)
\lineto(491.63957009,571.83760102)
\curveto(491.65956867,571.88759303)(491.66456867,571.93259299)(491.65457009,571.97260102)
\curveto(491.65456868,572.0225929)(491.65956867,572.06759285)(491.66957009,572.10760102)
\curveto(491.67956865,572.14759277)(491.68456865,572.18259274)(491.68457009,572.21260102)
\curveto(491.68456865,572.25259267)(491.68956864,572.29259263)(491.69957009,572.33260102)
\curveto(491.7295686,572.44259248)(491.74956858,572.55259237)(491.75957009,572.66260102)
\curveto(491.77956855,572.78259214)(491.81456852,572.89759202)(491.86457009,573.00760102)
\curveto(492.00456833,573.34759157)(492.16456817,573.6225913)(492.34457009,573.83260102)
\curveto(492.5345678,574.05259087)(492.80456753,574.23259069)(493.15457009,574.37260102)
\curveto(493.2345671,574.40259052)(493.31956701,574.4225905)(493.40957009,574.43260102)
\curveto(493.49956683,574.45259047)(493.59456674,574.47259045)(493.69457009,574.49260102)
\curveto(493.72456661,574.50259042)(493.77956655,574.50259042)(493.85957009,574.49260102)
\curveto(493.93956639,574.49259043)(493.98956634,574.50259042)(494.00957009,574.52260102)
\curveto(494.56956576,574.53259039)(495.01956531,574.4225905)(495.35957009,574.19260102)
\curveto(495.70956462,573.96259096)(495.96956436,573.65759126)(496.13957009,573.27760102)
\curveto(496.17956415,573.18759173)(496.21456412,573.09259183)(496.24457009,572.99260102)
\curveto(496.27456406,572.89259203)(496.29956403,572.79259213)(496.31957009,572.69260102)
\curveto(496.33956399,572.66259226)(496.34456399,572.63259229)(496.33457009,572.60260102)
\curveto(496.334564,572.57259235)(496.33956399,572.54259238)(496.34957009,572.51260102)
\curveto(496.37956395,572.40259252)(496.39956393,572.27759264)(496.40957009,572.13760102)
\curveto(496.41956391,572.00759291)(496.4295639,571.87259305)(496.43957009,571.73260102)
\lineto(496.43957009,571.56760102)
\curveto(496.44956388,571.50759341)(496.44956388,571.45259347)(496.43957009,571.40260102)
\curveto(496.4295639,571.35259357)(496.42456391,571.30259362)(496.42457009,571.25260102)
\lineto(496.42457009,571.11760102)
\curveto(496.41456392,571.07759384)(496.40956392,571.03759388)(496.40957009,570.99760102)
\curveto(496.41956391,570.95759396)(496.41456392,570.91259401)(496.39457009,570.86260102)
\curveto(496.37456396,570.75259417)(496.35456398,570.64759427)(496.33457009,570.54760102)
\curveto(496.32456401,570.44759447)(496.30456403,570.34759457)(496.27457009,570.24760102)
\curveto(496.14456419,569.88759503)(495.97956435,569.57259535)(495.77957009,569.30260102)
\curveto(495.57956475,569.03259589)(495.30456503,568.82759609)(494.95457009,568.68760102)
\curveto(494.87456546,568.65759626)(494.78956554,568.63259629)(494.69957009,568.61260102)
\lineto(494.42957009,568.55260102)
\curveto(494.37956595,568.54259638)(494.334566,568.53759638)(494.29457009,568.53760102)
\curveto(494.25456608,568.54759637)(494.21456612,568.54759637)(494.17457009,568.53760102)
\curveto(494.07456626,568.5175964)(493.97956635,568.5175964)(493.88957009,568.53760102)
\moveto(493.04957009,569.93260102)
\curveto(493.08956724,569.86259506)(493.1295672,569.79759512)(493.16957009,569.73760102)
\curveto(493.20956712,569.68759523)(493.25956707,569.63759528)(493.31957009,569.58760102)
\lineto(493.46957009,569.46760102)
\curveto(493.5295668,569.43759548)(493.59456674,569.41259551)(493.66457009,569.39260102)
\curveto(493.70456663,569.37259555)(493.73956659,569.36259556)(493.76957009,569.36260102)
\curveto(493.80956652,569.37259555)(493.84956648,569.36759555)(493.88957009,569.34760102)
\curveto(493.91956641,569.34759557)(493.95956637,569.34259558)(494.00957009,569.33260102)
\curveto(494.05956627,569.33259559)(494.09956623,569.33759558)(494.12957009,569.34760102)
\lineto(494.35457009,569.39260102)
\curveto(494.60456573,569.47259545)(494.78956554,569.59759532)(494.90957009,569.76760102)
\curveto(494.98956534,569.86759505)(495.05956527,569.99759492)(495.11957009,570.15760102)
\curveto(495.19956513,570.33759458)(495.25956507,570.56259436)(495.29957009,570.83260102)
\curveto(495.33956499,571.11259381)(495.35456498,571.39259353)(495.34457009,571.67260102)
\curveto(495.334565,571.96259296)(495.30456503,572.23759268)(495.25457009,572.49760102)
\curveto(495.20456513,572.75759216)(495.1295652,572.96759195)(495.02957009,573.12760102)
\curveto(494.90956542,573.32759159)(494.75956557,573.47759144)(494.57957009,573.57760102)
\curveto(494.49956583,573.62759129)(494.40956592,573.65759126)(494.30957009,573.66760102)
\curveto(494.20956612,573.68759123)(494.10456623,573.69759122)(493.99457009,573.69760102)
\curveto(493.97456636,573.68759123)(493.94956638,573.68259124)(493.91957009,573.68260102)
\curveto(493.89956643,573.69259123)(493.87956645,573.69259123)(493.85957009,573.68260102)
\curveto(493.80956652,573.67259125)(493.76456657,573.66259126)(493.72457009,573.65260102)
\curveto(493.68456665,573.65259127)(493.64456669,573.64259128)(493.60457009,573.62260102)
\curveto(493.42456691,573.54259138)(493.27456706,573.4225915)(493.15457009,573.26260102)
\curveto(493.04456729,573.10259182)(492.95456738,572.922592)(492.88457009,572.72260102)
\curveto(492.82456751,572.53259239)(492.77956755,572.30759261)(492.74957009,572.04760102)
\curveto(492.7295676,571.78759313)(492.72456761,571.5225934)(492.73457009,571.25260102)
\curveto(492.74456759,570.99259393)(492.77456756,570.74259418)(492.82457009,570.50260102)
\curveto(492.88456745,570.27259465)(492.95956737,570.08259484)(493.04957009,569.93260102)
\moveto(503.84957009,566.94760102)
\curveto(503.85955647,566.89759802)(503.86455647,566.80759811)(503.86457009,566.67760102)
\curveto(503.86455647,566.54759837)(503.85455648,566.45759846)(503.83457009,566.40760102)
\curveto(503.81455652,566.35759856)(503.80955652,566.30259862)(503.81957009,566.24260102)
\curveto(503.8295565,566.19259873)(503.8295565,566.14259878)(503.81957009,566.09260102)
\curveto(503.77955655,565.95259897)(503.74955658,565.8175991)(503.72957009,565.68760102)
\curveto(503.71955661,565.55759936)(503.68955664,565.43759948)(503.63957009,565.32760102)
\curveto(503.49955683,564.97759994)(503.334557,564.68260024)(503.14457009,564.44260102)
\curveto(502.95455738,564.21260071)(502.68455765,564.02760089)(502.33457009,563.88760102)
\curveto(502.25455808,563.85760106)(502.16955816,563.83760108)(502.07957009,563.82760102)
\curveto(501.98955834,563.80760111)(501.90455843,563.78760113)(501.82457009,563.76760102)
\curveto(501.77455856,563.75760116)(501.72455861,563.75260117)(501.67457009,563.75260102)
\curveto(501.62455871,563.75260117)(501.57455876,563.74760117)(501.52457009,563.73760102)
\curveto(501.49455884,563.72760119)(501.44455889,563.72760119)(501.37457009,563.73760102)
\curveto(501.30455903,563.73760118)(501.25455908,563.74260118)(501.22457009,563.75260102)
\curveto(501.16455917,563.77260115)(501.10455923,563.78260114)(501.04457009,563.78260102)
\curveto(500.99455934,563.77260115)(500.94455939,563.77760114)(500.89457009,563.79760102)
\curveto(500.80455953,563.8176011)(500.71455962,563.84260108)(500.62457009,563.87260102)
\curveto(500.54455979,563.89260103)(500.46455987,563.922601)(500.38457009,563.96260102)
\curveto(500.06456027,564.10260082)(499.81456052,564.29760062)(499.63457009,564.54760102)
\curveto(499.45456088,564.80760011)(499.30456103,565.11259981)(499.18457009,565.46260102)
\curveto(499.16456117,565.54259938)(499.14956118,565.62759929)(499.13957009,565.71760102)
\curveto(499.1295612,565.80759911)(499.11456122,565.89259903)(499.09457009,565.97260102)
\curveto(499.08456125,566.00259892)(499.07956125,566.03259889)(499.07957009,566.06260102)
\lineto(499.07957009,566.16760102)
\curveto(499.05956127,566.24759867)(499.04956128,566.32759859)(499.04957009,566.40760102)
\lineto(499.04957009,566.54260102)
\curveto(499.0295613,566.64259828)(499.0295613,566.74259818)(499.04957009,566.84260102)
\lineto(499.04957009,567.02260102)
\curveto(499.05956127,567.07259785)(499.06456127,567.1175978)(499.06457009,567.15760102)
\curveto(499.06456127,567.20759771)(499.06956126,567.25259767)(499.07957009,567.29260102)
\curveto(499.08956124,567.33259759)(499.09456124,567.36759755)(499.09457009,567.39760102)
\curveto(499.09456124,567.43759748)(499.09956123,567.47759744)(499.10957009,567.51760102)
\lineto(499.16957009,567.84760102)
\curveto(499.18956114,567.96759695)(499.21956111,568.07759684)(499.25957009,568.17760102)
\curveto(499.39956093,568.50759641)(499.55956077,568.78259614)(499.73957009,569.00260102)
\curveto(499.9295604,569.23259569)(500.18956014,569.4175955)(500.51957009,569.55760102)
\curveto(500.59955973,569.59759532)(500.68455965,569.6225953)(500.77457009,569.63260102)
\lineto(501.07457009,569.69260102)
\lineto(501.20957009,569.69260102)
\curveto(501.25955907,569.70259522)(501.30955902,569.70759521)(501.35957009,569.70760102)
\curveto(501.9295584,569.72759519)(502.38955794,569.6225953)(502.73957009,569.39260102)
\curveto(503.09955723,569.17259575)(503.36455697,568.87259605)(503.53457009,568.49260102)
\curveto(503.58455675,568.39259653)(503.62455671,568.29259663)(503.65457009,568.19260102)
\curveto(503.68455665,568.09259683)(503.71455662,567.98759693)(503.74457009,567.87760102)
\curveto(503.75455658,567.83759708)(503.75955657,567.80259712)(503.75957009,567.77260102)
\curveto(503.75955657,567.75259717)(503.76455657,567.7225972)(503.77457009,567.68260102)
\curveto(503.79455654,567.61259731)(503.80455653,567.53759738)(503.80457009,567.45760102)
\curveto(503.80455653,567.37759754)(503.81455652,567.29759762)(503.83457009,567.21760102)
\curveto(503.8345565,567.16759775)(503.8345565,567.1225978)(503.83457009,567.08260102)
\curveto(503.8345565,567.04259788)(503.83955649,566.99759792)(503.84957009,566.94760102)
\moveto(502.73957009,566.51260102)
\curveto(502.74955758,566.56259836)(502.75455758,566.63759828)(502.75457009,566.73760102)
\curveto(502.76455757,566.83759808)(502.75955757,566.91259801)(502.73957009,566.96260102)
\curveto(502.71955761,567.0225979)(502.71455762,567.07759784)(502.72457009,567.12760102)
\curveto(502.74455759,567.18759773)(502.74455759,567.24759767)(502.72457009,567.30760102)
\curveto(502.71455762,567.33759758)(502.70955762,567.37259755)(502.70957009,567.41260102)
\curveto(502.70955762,567.45259747)(502.70455763,567.49259743)(502.69457009,567.53260102)
\curveto(502.67455766,567.61259731)(502.65455768,567.68759723)(502.63457009,567.75760102)
\curveto(502.62455771,567.83759708)(502.60955772,567.917597)(502.58957009,567.99760102)
\curveto(502.55955777,568.05759686)(502.5345578,568.1175968)(502.51457009,568.17760102)
\curveto(502.49455784,568.23759668)(502.46455787,568.29759662)(502.42457009,568.35760102)
\curveto(502.32455801,568.52759639)(502.19455814,568.66259626)(502.03457009,568.76260102)
\curveto(501.95455838,568.81259611)(501.85955847,568.84759607)(501.74957009,568.86760102)
\curveto(501.63955869,568.88759603)(501.51455882,568.89759602)(501.37457009,568.89760102)
\curveto(501.35455898,568.88759603)(501.329559,568.88259604)(501.29957009,568.88260102)
\curveto(501.26955906,568.89259603)(501.23955909,568.89259603)(501.20957009,568.88260102)
\lineto(501.05957009,568.82260102)
\curveto(501.00955932,568.81259611)(500.96455937,568.79759612)(500.92457009,568.77760102)
\curveto(500.7345596,568.66759625)(500.58955974,568.5225964)(500.48957009,568.34260102)
\curveto(500.39955993,568.16259676)(500.31956001,567.95759696)(500.24957009,567.72760102)
\curveto(500.20956012,567.59759732)(500.18956014,567.46259746)(500.18957009,567.32260102)
\curveto(500.18956014,567.19259773)(500.17956015,567.04759787)(500.15957009,566.88760102)
\curveto(500.14956018,566.83759808)(500.13956019,566.77759814)(500.12957009,566.70760102)
\curveto(500.1295602,566.63759828)(500.13956019,566.57759834)(500.15957009,566.52760102)
\lineto(500.15957009,566.36260102)
\lineto(500.15957009,566.18260102)
\curveto(500.16956016,566.13259879)(500.17956015,566.07759884)(500.18957009,566.01760102)
\curveto(500.19956013,565.96759895)(500.20456013,565.91259901)(500.20457009,565.85260102)
\curveto(500.21456012,565.79259913)(500.2295601,565.73759918)(500.24957009,565.68760102)
\curveto(500.29956003,565.49759942)(500.35955997,565.3225996)(500.42957009,565.16260102)
\curveto(500.49955983,565.00259992)(500.60455973,564.87260005)(500.74457009,564.77260102)
\curveto(500.87455946,564.67260025)(501.01455932,564.60260032)(501.16457009,564.56260102)
\curveto(501.19455914,564.55260037)(501.21955911,564.54760037)(501.23957009,564.54760102)
\curveto(501.26955906,564.55760036)(501.29955903,564.55760036)(501.32957009,564.54760102)
\curveto(501.34955898,564.54760037)(501.37955895,564.54260038)(501.41957009,564.53260102)
\curveto(501.45955887,564.53260039)(501.49455884,564.53760038)(501.52457009,564.54760102)
\curveto(501.56455877,564.55760036)(501.60455873,564.56260036)(501.64457009,564.56260102)
\curveto(501.68455865,564.56260036)(501.72455861,564.57260035)(501.76457009,564.59260102)
\curveto(502.00455833,564.67260025)(502.19955813,564.80760011)(502.34957009,564.99760102)
\curveto(502.46955786,565.17759974)(502.55955777,565.38259954)(502.61957009,565.61260102)
\curveto(502.63955769,565.68259924)(502.65455768,565.75259917)(502.66457009,565.82260102)
\curveto(502.67455766,565.90259902)(502.68955764,565.98259894)(502.70957009,566.06260102)
\curveto(502.70955762,566.1225988)(502.71455762,566.16759875)(502.72457009,566.19760102)
\curveto(502.72455761,566.2175987)(502.72455761,566.24259868)(502.72457009,566.27260102)
\curveto(502.72455761,566.31259861)(502.7295576,566.34259858)(502.73957009,566.36260102)
\lineto(502.73957009,566.51260102)
}
}
{
\newrgbcolor{curcolor}{0 0 0}
\pscustom[linestyle=none,fillstyle=solid,fillcolor=curcolor]
{
\newpath
\moveto(597.67448952,395.26980927)
\curveto(597.74448187,395.21980581)(597.78448183,395.14980588)(597.79448952,395.05980927)
\curveto(597.8144818,394.96980606)(597.82448179,394.86480616)(597.82448952,394.74480927)
\curveto(597.82448179,394.69480633)(597.8194818,394.64480638)(597.80948952,394.59480927)
\curveto(597.80948181,394.54480648)(597.79948182,394.49980653)(597.77948952,394.45980927)
\curveto(597.74948187,394.36980666)(597.68948193,394.30980672)(597.59948952,394.27980927)
\curveto(597.5194821,394.25980677)(597.42448219,394.24980678)(597.31448952,394.24980927)
\lineto(596.99948952,394.24980927)
\curveto(596.88948273,394.25980677)(596.78448283,394.24980678)(596.68448952,394.21980927)
\curveto(596.54448307,394.18980684)(596.45448316,394.10980692)(596.41448952,393.97980927)
\curveto(596.39448322,393.90980712)(596.38448323,393.8248072)(596.38448952,393.72480927)
\lineto(596.38448952,393.45480927)
\lineto(596.38448952,392.50980927)
\lineto(596.38448952,392.17980927)
\curveto(596.38448323,392.06980896)(596.36448325,391.98480904)(596.32448952,391.92480927)
\curveto(596.28448333,391.86480916)(596.23448338,391.8248092)(596.17448952,391.80480927)
\curveto(596.12448349,391.79480923)(596.05948356,391.77980925)(595.97948952,391.75980927)
\lineto(595.78448952,391.75980927)
\curveto(595.66448395,391.75980927)(595.55948406,391.76480926)(595.46948952,391.77480927)
\curveto(595.37948424,391.79480923)(595.30948431,391.84480918)(595.25948952,391.92480927)
\curveto(595.22948439,391.97480905)(595.2144844,392.04480898)(595.21448952,392.13480927)
\lineto(595.21448952,392.43480927)
\lineto(595.21448952,393.46980927)
\curveto(595.2144844,393.6298074)(595.20448441,393.77480725)(595.18448952,393.90480927)
\curveto(595.17448444,394.04480698)(595.1194845,394.13980689)(595.01948952,394.18980927)
\curveto(594.96948465,394.20980682)(594.89948472,394.2248068)(594.80948952,394.23480927)
\curveto(594.72948489,394.24480678)(594.63948498,394.24980678)(594.53948952,394.24980927)
\lineto(594.25448952,394.24980927)
\lineto(594.01448952,394.24980927)
\lineto(591.74948952,394.24980927)
\curveto(591.65948796,394.24980678)(591.55448806,394.24480678)(591.43448952,394.23480927)
\lineto(591.10448952,394.23480927)
\curveto(590.99448862,394.23480679)(590.89448872,394.24480678)(590.80448952,394.26480927)
\curveto(590.7144889,394.28480674)(590.65448896,394.31980671)(590.62448952,394.36980927)
\curveto(590.57448904,394.43980659)(590.54948907,394.53480649)(590.54948952,394.65480927)
\lineto(590.54948952,394.99980927)
\lineto(590.54948952,395.26980927)
\curveto(590.58948903,395.43980559)(590.64448897,395.57980545)(590.71448952,395.68980927)
\curveto(590.78448883,395.79980523)(590.86448875,395.91480511)(590.95448952,396.03480927)
\lineto(591.31448952,396.57480927)
\curveto(591.75448786,397.20480382)(592.18948743,397.8248032)(592.61948952,398.43480927)
\lineto(593.93948952,400.29480927)
\curveto(594.09948552,400.5248005)(594.25448536,400.74480028)(594.40448952,400.95480927)
\curveto(594.55448506,401.17479985)(594.70948491,401.39979963)(594.86948952,401.62980927)
\curveto(594.9194847,401.69979933)(594.96948465,401.76479926)(595.01948952,401.82480927)
\curveto(595.06948455,401.89479913)(595.1194845,401.96979906)(595.16948952,402.04980927)
\lineto(595.22948952,402.13980927)
\curveto(595.25948436,402.17979885)(595.28948433,402.20979882)(595.31948952,402.22980927)
\curveto(595.35948426,402.25979877)(595.39948422,402.27979875)(595.43948952,402.28980927)
\curveto(595.47948414,402.30979872)(595.52448409,402.3297987)(595.57448952,402.34980927)
\curveto(595.59448402,402.34979868)(595.614484,402.34479868)(595.63448952,402.33480927)
\curveto(595.66448395,402.33479869)(595.68948393,402.34479868)(595.70948952,402.36480927)
\curveto(595.83948378,402.36479866)(595.95948366,402.35979867)(596.06948952,402.34980927)
\curveto(596.17948344,402.33979869)(596.25948336,402.29479873)(596.30948952,402.21480927)
\curveto(596.34948327,402.16479886)(596.36948325,402.09479893)(596.36948952,402.00480927)
\curveto(596.37948324,401.91479911)(596.38448323,401.81979921)(596.38448952,401.71980927)
\lineto(596.38448952,396.25980927)
\curveto(596.38448323,396.18980484)(596.37948324,396.11480491)(596.36948952,396.03480927)
\curveto(596.36948325,395.96480506)(596.37448324,395.89480513)(596.38448952,395.82480927)
\lineto(596.38448952,395.71980927)
\curveto(596.40448321,395.66980536)(596.4194832,395.61480541)(596.42948952,395.55480927)
\curveto(596.43948318,395.50480552)(596.46448315,395.46480556)(596.50448952,395.43480927)
\curveto(596.57448304,395.38480564)(596.65948296,395.35480567)(596.75948952,395.34480927)
\lineto(597.08948952,395.34480927)
\curveto(597.19948242,395.34480568)(597.30448231,395.33980569)(597.40448952,395.32980927)
\curveto(597.5144821,395.3298057)(597.60448201,395.30980572)(597.67448952,395.26980927)
\moveto(595.10948952,395.46480927)
\curveto(595.18948443,395.57480545)(595.22448439,395.74480528)(595.21448952,395.97480927)
\lineto(595.21448952,396.58980927)
\lineto(595.21448952,399.06480927)
\lineto(595.21448952,399.37980927)
\curveto(595.22448439,399.49980153)(595.2194844,399.59980143)(595.19948952,399.67980927)
\lineto(595.19948952,399.82980927)
\curveto(595.19948442,399.91980111)(595.18448443,400.00480102)(595.15448952,400.08480927)
\curveto(595.14448447,400.10480092)(595.13448448,400.11480091)(595.12448952,400.11480927)
\lineto(595.07948952,400.15980927)
\curveto(595.05948456,400.16980086)(595.02948459,400.17480085)(594.98948952,400.17480927)
\curveto(594.96948465,400.15480087)(594.94948467,400.13980089)(594.92948952,400.12980927)
\curveto(594.9194847,400.1298009)(594.90448471,400.1248009)(594.88448952,400.11480927)
\curveto(594.82448479,400.06480096)(594.76448485,399.99480103)(594.70448952,399.90480927)
\curveto(594.64448497,399.81480121)(594.58948503,399.73480129)(594.53948952,399.66480927)
\curveto(594.43948518,399.5248015)(594.34448527,399.37980165)(594.25448952,399.22980927)
\curveto(594.16448545,399.08980194)(594.06948555,398.94980208)(593.96948952,398.80980927)
\lineto(593.42948952,398.02980927)
\curveto(593.25948636,397.76980326)(593.08448653,397.50980352)(592.90448952,397.24980927)
\curveto(592.82448679,397.13980389)(592.74948687,397.03480399)(592.67948952,396.93480927)
\lineto(592.46948952,396.63480927)
\curveto(592.4194872,396.55480447)(592.36948725,396.47980455)(592.31948952,396.40980927)
\curveto(592.27948734,396.33980469)(592.23448738,396.26480476)(592.18448952,396.18480927)
\curveto(592.13448748,396.1248049)(592.08448753,396.05980497)(592.03448952,395.98980927)
\curveto(591.99448762,395.9298051)(591.95448766,395.85980517)(591.91448952,395.77980927)
\curveto(591.87448774,395.71980531)(591.84948777,395.64980538)(591.83948952,395.56980927)
\curveto(591.82948779,395.49980553)(591.86448775,395.44480558)(591.94448952,395.40480927)
\curveto(592.0144876,395.35480567)(592.12448749,395.3298057)(592.27448952,395.32980927)
\curveto(592.43448718,395.33980569)(592.56948705,395.34480568)(592.67948952,395.34480927)
\lineto(594.35948952,395.34480927)
\lineto(594.79448952,395.34480927)
\curveto(594.94448467,395.34480568)(595.04948457,395.38480564)(595.10948952,395.46480927)
}
}
{
\newrgbcolor{curcolor}{0 0 0}
\pscustom[linestyle=none,fillstyle=solid,fillcolor=curcolor]
{
\newpath
\moveto(606.14409889,396.85980927)
\lineto(606.14409889,396.60480927)
\curveto(606.15409119,396.5248045)(606.14909119,396.44980458)(606.12909889,396.37980927)
\lineto(606.12909889,396.13980927)
\lineto(606.12909889,395.97480927)
\curveto(606.10909123,395.87480515)(606.09909124,395.76980526)(606.09909889,395.65980927)
\curveto(606.09909124,395.55980547)(606.08909125,395.45980557)(606.06909889,395.35980927)
\lineto(606.06909889,395.20980927)
\curveto(606.0390913,395.06980596)(606.01909132,394.9298061)(606.00909889,394.78980927)
\curveto(605.99909134,394.65980637)(605.97409137,394.5298065)(605.93409889,394.39980927)
\curveto(605.91409143,394.31980671)(605.89409145,394.23480679)(605.87409889,394.14480927)
\lineto(605.81409889,393.90480927)
\lineto(605.69409889,393.60480927)
\curveto(605.66409168,393.51480751)(605.62909171,393.4248076)(605.58909889,393.33480927)
\curveto(605.48909185,393.11480791)(605.35409199,392.89980813)(605.18409889,392.68980927)
\curveto(605.02409232,392.47980855)(604.84909249,392.30980872)(604.65909889,392.17980927)
\curveto(604.60909273,392.13980889)(604.54909279,392.09980893)(604.47909889,392.05980927)
\curveto(604.41909292,392.029809)(604.35909298,391.99480903)(604.29909889,391.95480927)
\curveto(604.21909312,391.90480912)(604.12409322,391.86480916)(604.01409889,391.83480927)
\curveto(603.90409344,391.80480922)(603.79909354,391.77480925)(603.69909889,391.74480927)
\curveto(603.58909375,391.70480932)(603.47909386,391.67980935)(603.36909889,391.66980927)
\curveto(603.25909408,391.65980937)(603.1440942,391.64480938)(603.02409889,391.62480927)
\curveto(602.98409436,391.61480941)(602.9390944,391.61480941)(602.88909889,391.62480927)
\curveto(602.84909449,391.6248094)(602.80909453,391.61980941)(602.76909889,391.60980927)
\curveto(602.72909461,391.59980943)(602.67409467,391.59480943)(602.60409889,391.59480927)
\curveto(602.53409481,391.59480943)(602.48409486,391.59980943)(602.45409889,391.60980927)
\curveto(602.40409494,391.6298094)(602.35909498,391.63480939)(602.31909889,391.62480927)
\curveto(602.27909506,391.61480941)(602.2440951,391.61480941)(602.21409889,391.62480927)
\lineto(602.12409889,391.62480927)
\curveto(602.06409528,391.64480938)(601.99909534,391.65980937)(601.92909889,391.66980927)
\curveto(601.86909547,391.66980936)(601.80409554,391.67480935)(601.73409889,391.68480927)
\curveto(601.56409578,391.73480929)(601.40409594,391.78480924)(601.25409889,391.83480927)
\curveto(601.10409624,391.88480914)(600.95909638,391.94980908)(600.81909889,392.02980927)
\curveto(600.76909657,392.06980896)(600.71409663,392.09980893)(600.65409889,392.11980927)
\curveto(600.60409674,392.14980888)(600.55409679,392.18480884)(600.50409889,392.22480927)
\curveto(600.26409708,392.40480862)(600.06409728,392.6248084)(599.90409889,392.88480927)
\curveto(599.7440976,393.14480788)(599.60409774,393.4298076)(599.48409889,393.73980927)
\curveto(599.42409792,393.87980715)(599.37909796,394.01980701)(599.34909889,394.15980927)
\curveto(599.31909802,394.30980672)(599.28409806,394.46480656)(599.24409889,394.62480927)
\curveto(599.22409812,394.73480629)(599.20909813,394.84480618)(599.19909889,394.95480927)
\curveto(599.18909815,395.06480596)(599.17409817,395.17480585)(599.15409889,395.28480927)
\curveto(599.1440982,395.3248057)(599.1390982,395.36480566)(599.13909889,395.40480927)
\curveto(599.14909819,395.44480558)(599.14909819,395.48480554)(599.13909889,395.52480927)
\curveto(599.12909821,395.57480545)(599.12409822,395.6248054)(599.12409889,395.67480927)
\lineto(599.12409889,395.83980927)
\curveto(599.10409824,395.88980514)(599.09909824,395.93980509)(599.10909889,395.98980927)
\curveto(599.11909822,396.04980498)(599.11909822,396.10480492)(599.10909889,396.15480927)
\curveto(599.09909824,396.19480483)(599.09909824,396.23980479)(599.10909889,396.28980927)
\curveto(599.11909822,396.33980469)(599.11409823,396.38980464)(599.09409889,396.43980927)
\curveto(599.07409827,396.50980452)(599.06909827,396.58480444)(599.07909889,396.66480927)
\curveto(599.08909825,396.75480427)(599.09409825,396.83980419)(599.09409889,396.91980927)
\curveto(599.09409825,397.00980402)(599.08909825,397.10980392)(599.07909889,397.21980927)
\curveto(599.06909827,397.33980369)(599.07409827,397.43980359)(599.09409889,397.51980927)
\lineto(599.09409889,397.80480927)
\lineto(599.13909889,398.43480927)
\curveto(599.14909819,398.53480249)(599.15909818,398.6298024)(599.16909889,398.71980927)
\lineto(599.19909889,399.01980927)
\curveto(599.21909812,399.06980196)(599.22409812,399.11980191)(599.21409889,399.16980927)
\curveto(599.21409813,399.2298018)(599.22409812,399.28480174)(599.24409889,399.33480927)
\curveto(599.29409805,399.50480152)(599.33409801,399.66980136)(599.36409889,399.82980927)
\curveto(599.39409795,399.99980103)(599.4440979,400.15980087)(599.51409889,400.30980927)
\curveto(599.70409764,400.76980026)(599.92409742,401.14479988)(600.17409889,401.43480927)
\curveto(600.43409691,401.7247993)(600.79409655,401.96979906)(601.25409889,402.16980927)
\curveto(601.38409596,402.21979881)(601.51409583,402.25479877)(601.64409889,402.27480927)
\curveto(601.78409556,402.29479873)(601.92409542,402.31979871)(602.06409889,402.34980927)
\curveto(602.13409521,402.35979867)(602.19909514,402.36479866)(602.25909889,402.36480927)
\curveto(602.31909502,402.36479866)(602.38409496,402.36979866)(602.45409889,402.37980927)
\curveto(603.28409406,402.39979863)(603.95409339,402.24979878)(604.46409889,401.92980927)
\curveto(604.97409237,401.61979941)(605.35409199,401.17979985)(605.60409889,400.60980927)
\curveto(605.65409169,400.48980054)(605.69909164,400.36480066)(605.73909889,400.23480927)
\curveto(605.77909156,400.10480092)(605.82409152,399.96980106)(605.87409889,399.82980927)
\curveto(605.89409145,399.74980128)(605.90909143,399.66480136)(605.91909889,399.57480927)
\lineto(605.97909889,399.33480927)
\curveto(606.00909133,399.2248018)(606.02409132,399.11480191)(606.02409889,399.00480927)
\curveto(606.03409131,398.89480213)(606.04909129,398.78480224)(606.06909889,398.67480927)
\curveto(606.08909125,398.6248024)(606.09409125,398.57980245)(606.08409889,398.53980927)
\curveto(606.08409126,398.49980253)(606.08909125,398.45980257)(606.09909889,398.41980927)
\curveto(606.10909123,398.36980266)(606.10909123,398.31480271)(606.09909889,398.25480927)
\curveto(606.09909124,398.20480282)(606.10409124,398.15480287)(606.11409889,398.10480927)
\lineto(606.11409889,397.96980927)
\curveto(606.13409121,397.90980312)(606.13409121,397.83980319)(606.11409889,397.75980927)
\curveto(606.10409124,397.68980334)(606.10909123,397.6248034)(606.12909889,397.56480927)
\curveto(606.1390912,397.53480349)(606.1440912,397.49480353)(606.14409889,397.44480927)
\lineto(606.14409889,397.32480927)
\lineto(606.14409889,396.85980927)
\moveto(604.59909889,394.53480927)
\curveto(604.69909264,394.85480617)(604.75909258,395.21980581)(604.77909889,395.62980927)
\curveto(604.79909254,396.03980499)(604.80909253,396.44980458)(604.80909889,396.85980927)
\curveto(604.80909253,397.28980374)(604.79909254,397.70980332)(604.77909889,398.11980927)
\curveto(604.75909258,398.5298025)(604.71409263,398.91480211)(604.64409889,399.27480927)
\curveto(604.57409277,399.63480139)(604.46409288,399.95480107)(604.31409889,400.23480927)
\curveto(604.17409317,400.5248005)(603.97909336,400.75980027)(603.72909889,400.93980927)
\curveto(603.56909377,401.04979998)(603.38909395,401.1297999)(603.18909889,401.17980927)
\curveto(602.98909435,401.23979979)(602.7440946,401.26979976)(602.45409889,401.26980927)
\curveto(602.43409491,401.24979978)(602.39909494,401.23979979)(602.34909889,401.23980927)
\curveto(602.29909504,401.24979978)(602.25909508,401.24979978)(602.22909889,401.23980927)
\curveto(602.14909519,401.21979981)(602.07409527,401.19979983)(602.00409889,401.17980927)
\curveto(601.9440954,401.16979986)(601.87909546,401.14979988)(601.80909889,401.11980927)
\curveto(601.5390958,400.99980003)(601.31909602,400.8298002)(601.14909889,400.60980927)
\curveto(600.98909635,400.39980063)(600.85409649,400.15480087)(600.74409889,399.87480927)
\curveto(600.69409665,399.76480126)(600.65409669,399.64480138)(600.62409889,399.51480927)
\curveto(600.60409674,399.39480163)(600.57909676,399.26980176)(600.54909889,399.13980927)
\curveto(600.52909681,399.08980194)(600.51909682,399.03480199)(600.51909889,398.97480927)
\curveto(600.51909682,398.9248021)(600.51409683,398.87480215)(600.50409889,398.82480927)
\curveto(600.49409685,398.73480229)(600.48409686,398.63980239)(600.47409889,398.53980927)
\curveto(600.46409688,398.44980258)(600.45409689,398.35480267)(600.44409889,398.25480927)
\curveto(600.4440969,398.17480285)(600.4390969,398.08980294)(600.42909889,397.99980927)
\lineto(600.42909889,397.75980927)
\lineto(600.42909889,397.57980927)
\curveto(600.41909692,397.54980348)(600.41409693,397.51480351)(600.41409889,397.47480927)
\lineto(600.41409889,397.33980927)
\lineto(600.41409889,396.88980927)
\curveto(600.41409693,396.80980422)(600.40909693,396.7248043)(600.39909889,396.63480927)
\curveto(600.39909694,396.55480447)(600.40909693,396.47980455)(600.42909889,396.40980927)
\lineto(600.42909889,396.13980927)
\curveto(600.42909691,396.11980491)(600.42409692,396.08980494)(600.41409889,396.04980927)
\curveto(600.41409693,396.01980501)(600.41909692,395.99480503)(600.42909889,395.97480927)
\curveto(600.4390969,395.87480515)(600.4440969,395.77480525)(600.44409889,395.67480927)
\curveto(600.45409689,395.58480544)(600.46409688,395.48480554)(600.47409889,395.37480927)
\curveto(600.50409684,395.25480577)(600.51909682,395.1298059)(600.51909889,394.99980927)
\curveto(600.52909681,394.87980615)(600.55409679,394.76480626)(600.59409889,394.65480927)
\curveto(600.67409667,394.35480667)(600.75909658,394.08980694)(600.84909889,393.85980927)
\curveto(600.94909639,393.6298074)(601.09409625,393.41480761)(601.28409889,393.21480927)
\curveto(601.49409585,393.01480801)(601.75909558,392.86480816)(602.07909889,392.76480927)
\curveto(602.11909522,392.74480828)(602.15409519,392.73480829)(602.18409889,392.73480927)
\curveto(602.22409512,392.74480828)(602.26909507,392.73980829)(602.31909889,392.71980927)
\curveto(602.35909498,392.70980832)(602.42909491,392.69980833)(602.52909889,392.68980927)
\curveto(602.6390947,392.67980835)(602.72409462,392.68480834)(602.78409889,392.70480927)
\curveto(602.85409449,392.7248083)(602.92409442,392.73480829)(602.99409889,392.73480927)
\curveto(603.06409428,392.74480828)(603.12909421,392.75980827)(603.18909889,392.77980927)
\curveto(603.38909395,392.83980819)(603.56909377,392.9248081)(603.72909889,393.03480927)
\curveto(603.75909358,393.05480797)(603.78409356,393.07480795)(603.80409889,393.09480927)
\lineto(603.86409889,393.15480927)
\curveto(603.90409344,393.17480785)(603.95409339,393.21480781)(604.01409889,393.27480927)
\curveto(604.11409323,393.41480761)(604.19909314,393.54480748)(604.26909889,393.66480927)
\curveto(604.339093,393.78480724)(604.40909293,393.9298071)(604.47909889,394.09980927)
\curveto(604.50909283,394.16980686)(604.52909281,394.23980679)(604.53909889,394.30980927)
\curveto(604.55909278,394.37980665)(604.57909276,394.45480657)(604.59909889,394.53480927)
}
}
{
\newrgbcolor{curcolor}{0 0 0}
\pscustom[linestyle=none,fillstyle=solid,fillcolor=curcolor]
{
\newpath
\moveto(608.44870827,393.40980927)
\lineto(608.74870827,393.40980927)
\curveto(608.85870621,393.41980761)(608.9637061,393.41980761)(609.06370827,393.40980927)
\curveto(609.17370589,393.40980762)(609.27370579,393.39980763)(609.36370827,393.37980927)
\curveto(609.45370561,393.36980766)(609.52370554,393.34480768)(609.57370827,393.30480927)
\curveto(609.59370547,393.28480774)(609.60870546,393.25480777)(609.61870827,393.21480927)
\curveto(609.63870543,393.17480785)(609.65870541,393.1298079)(609.67870827,393.07980927)
\lineto(609.67870827,393.00480927)
\curveto(609.68870538,392.95480807)(609.68870538,392.89980813)(609.67870827,392.83980927)
\lineto(609.67870827,392.68980927)
\lineto(609.67870827,392.20980927)
\curveto(609.67870539,392.03980899)(609.63870543,391.91980911)(609.55870827,391.84980927)
\curveto(609.48870558,391.79980923)(609.39870567,391.77480925)(609.28870827,391.77480927)
\lineto(608.95870827,391.77480927)
\lineto(608.50870827,391.77480927)
\curveto(608.35870671,391.77480925)(608.24370682,391.80480922)(608.16370827,391.86480927)
\curveto(608.12370694,391.89480913)(608.09370697,391.94480908)(608.07370827,392.01480927)
\curveto(608.05370701,392.09480893)(608.03870703,392.17980885)(608.02870827,392.26980927)
\lineto(608.02870827,392.55480927)
\curveto(608.03870703,392.65480837)(608.04370702,392.73980829)(608.04370827,392.80980927)
\lineto(608.04370827,393.00480927)
\curveto(608.04370702,393.06480796)(608.05370701,393.11980791)(608.07370827,393.16980927)
\curveto(608.11370695,393.27980775)(608.18370688,393.34980768)(608.28370827,393.37980927)
\curveto(608.31370675,393.37980765)(608.3687067,393.38980764)(608.44870827,393.40980927)
}
}
{
\newrgbcolor{curcolor}{0 0 0}
\pscustom[linestyle=none,fillstyle=solid,fillcolor=curcolor]
{
\newpath
\moveto(613.16386452,402.18480927)
\lineto(616.76386452,402.18480927)
\lineto(617.40886452,402.18480927)
\curveto(617.48885799,402.18479884)(617.56385791,402.17979885)(617.63386452,402.16980927)
\curveto(617.70385777,402.16979886)(617.76385771,402.15979887)(617.81386452,402.13980927)
\curveto(617.88385759,402.10979892)(617.93885754,402.04979898)(617.97886452,401.95980927)
\curveto(617.99885748,401.9297991)(618.00885747,401.88979914)(618.00886452,401.83980927)
\lineto(618.00886452,401.70480927)
\curveto(618.01885746,401.59479943)(618.01385746,401.48979954)(617.99386452,401.38980927)
\curveto(617.98385749,401.28979974)(617.94885753,401.21979981)(617.88886452,401.17980927)
\curveto(617.79885768,401.10979992)(617.66385781,401.07479995)(617.48386452,401.07480927)
\curveto(617.30385817,401.08479994)(617.13885834,401.08979994)(616.98886452,401.08980927)
\lineto(614.99386452,401.08980927)
\lineto(614.49886452,401.08980927)
\lineto(614.36386452,401.08980927)
\curveto(614.32386115,401.08979994)(614.28386119,401.08479994)(614.24386452,401.07480927)
\lineto(614.03386452,401.07480927)
\curveto(613.92386155,401.04479998)(613.84386163,401.00480002)(613.79386452,400.95480927)
\curveto(613.74386173,400.91480011)(613.70886177,400.85980017)(613.68886452,400.78980927)
\curveto(613.66886181,400.7298003)(613.65386182,400.65980037)(613.64386452,400.57980927)
\curveto(613.63386184,400.49980053)(613.61386186,400.40980062)(613.58386452,400.30980927)
\curveto(613.53386194,400.10980092)(613.49386198,399.90480112)(613.46386452,399.69480927)
\curveto(613.43386204,399.48480154)(613.39386208,399.27980175)(613.34386452,399.07980927)
\curveto(613.32386215,399.00980202)(613.31386216,398.93980209)(613.31386452,398.86980927)
\curveto(613.31386216,398.80980222)(613.30386217,398.74480228)(613.28386452,398.67480927)
\curveto(613.2738622,398.64480238)(613.26386221,398.60480242)(613.25386452,398.55480927)
\curveto(613.25386222,398.51480251)(613.25886222,398.47480255)(613.26886452,398.43480927)
\curveto(613.28886219,398.38480264)(613.31386216,398.33980269)(613.34386452,398.29980927)
\curveto(613.38386209,398.26980276)(613.44386203,398.26480276)(613.52386452,398.28480927)
\curveto(613.58386189,398.30480272)(613.64386183,398.3298027)(613.70386452,398.35980927)
\curveto(613.76386171,398.39980263)(613.82386165,398.43480259)(613.88386452,398.46480927)
\curveto(613.94386153,398.48480254)(613.99386148,398.49980253)(614.03386452,398.50980927)
\curveto(614.22386125,398.58980244)(614.42886105,398.64480238)(614.64886452,398.67480927)
\curveto(614.8788606,398.70480232)(615.10886037,398.71480231)(615.33886452,398.70480927)
\curveto(615.5788599,398.70480232)(615.80885967,398.67980235)(616.02886452,398.62980927)
\curveto(616.24885923,398.58980244)(616.44885903,398.5298025)(616.62886452,398.44980927)
\curveto(616.6788588,398.4298026)(616.72385875,398.40980262)(616.76386452,398.38980927)
\curveto(616.81385866,398.36980266)(616.86385861,398.34480268)(616.91386452,398.31480927)
\curveto(617.26385821,398.10480292)(617.54385793,397.87480315)(617.75386452,397.62480927)
\curveto(617.9738575,397.37480365)(618.16885731,397.04980398)(618.33886452,396.64980927)
\curveto(618.38885709,396.53980449)(618.42385705,396.4298046)(618.44386452,396.31980927)
\curveto(618.46385701,396.20980482)(618.48885699,396.09480493)(618.51886452,395.97480927)
\curveto(618.52885695,395.94480508)(618.53385694,395.89980513)(618.53386452,395.83980927)
\curveto(618.55385692,395.77980525)(618.56385691,395.70980532)(618.56386452,395.62980927)
\curveto(618.56385691,395.55980547)(618.5738569,395.49480553)(618.59386452,395.43480927)
\lineto(618.59386452,395.26980927)
\curveto(618.60385687,395.21980581)(618.60885687,395.14980588)(618.60886452,395.05980927)
\curveto(618.60885687,394.96980606)(618.59885688,394.89980613)(618.57886452,394.84980927)
\curveto(618.55885692,394.78980624)(618.55385692,394.7298063)(618.56386452,394.66980927)
\curveto(618.5738569,394.61980641)(618.56885691,394.56980646)(618.54886452,394.51980927)
\curveto(618.50885697,394.35980667)(618.473857,394.20980682)(618.44386452,394.06980927)
\curveto(618.41385706,393.9298071)(618.36885711,393.79480723)(618.30886452,393.66480927)
\curveto(618.14885733,393.29480773)(617.92885755,392.95980807)(617.64886452,392.65980927)
\curveto(617.36885811,392.35980867)(617.04885843,392.1298089)(616.68886452,391.96980927)
\curveto(616.51885896,391.88980914)(616.31885916,391.81480921)(616.08886452,391.74480927)
\curveto(615.9788595,391.70480932)(615.86385961,391.67980935)(615.74386452,391.66980927)
\curveto(615.62385985,391.65980937)(615.50385997,391.63980939)(615.38386452,391.60980927)
\curveto(615.33386014,391.58980944)(615.2788602,391.58980944)(615.21886452,391.60980927)
\curveto(615.15886032,391.61980941)(615.09886038,391.61480941)(615.03886452,391.59480927)
\curveto(614.93886054,391.57480945)(614.83886064,391.57480945)(614.73886452,391.59480927)
\lineto(614.60386452,391.59480927)
\curveto(614.55386092,391.61480941)(614.49386098,391.6248094)(614.42386452,391.62480927)
\curveto(614.36386111,391.61480941)(614.30886117,391.61980941)(614.25886452,391.63980927)
\curveto(614.21886126,391.64980938)(614.18386129,391.65480937)(614.15386452,391.65480927)
\curveto(614.12386135,391.65480937)(614.08886139,391.65980937)(614.04886452,391.66980927)
\lineto(613.77886452,391.72980927)
\curveto(613.68886179,391.74980928)(613.60386187,391.77980925)(613.52386452,391.81980927)
\curveto(613.18386229,391.95980907)(612.89386258,392.11480891)(612.65386452,392.28480927)
\curveto(612.41386306,392.46480856)(612.19386328,392.69480833)(611.99386452,392.97480927)
\curveto(611.84386363,393.20480782)(611.72886375,393.44480758)(611.64886452,393.69480927)
\curveto(611.62886385,393.74480728)(611.61886386,393.78980724)(611.61886452,393.82980927)
\curveto(611.61886386,393.87980715)(611.60886387,393.9298071)(611.58886452,393.97980927)
\curveto(611.56886391,394.03980699)(611.55386392,394.11980691)(611.54386452,394.21980927)
\curveto(611.54386393,394.31980671)(611.56386391,394.39480663)(611.60386452,394.44480927)
\curveto(611.65386382,394.5248065)(611.73386374,394.56980646)(611.84386452,394.57980927)
\curveto(611.95386352,394.58980644)(612.06886341,394.59480643)(612.18886452,394.59480927)
\lineto(612.35386452,394.59480927)
\curveto(612.41386306,394.59480643)(612.46886301,394.58480644)(612.51886452,394.56480927)
\curveto(612.60886287,394.54480648)(612.6788628,394.50480652)(612.72886452,394.44480927)
\curveto(612.79886268,394.35480667)(612.84386263,394.24480678)(612.86386452,394.11480927)
\curveto(612.89386258,393.99480703)(612.93886254,393.88980714)(612.99886452,393.79980927)
\curveto(613.18886229,393.45980757)(613.44886203,393.18980784)(613.77886452,392.98980927)
\curveto(613.8788616,392.9298081)(613.98386149,392.87980815)(614.09386452,392.83980927)
\curveto(614.21386126,392.80980822)(614.33386114,392.77480825)(614.45386452,392.73480927)
\curveto(614.62386085,392.68480834)(614.82886065,392.66480836)(615.06886452,392.67480927)
\curveto(615.31886016,392.69480833)(615.51885996,392.7298083)(615.66886452,392.77980927)
\curveto(616.03885944,392.89980813)(616.32885915,393.05980797)(616.53886452,393.25980927)
\curveto(616.75885872,393.46980756)(616.93885854,393.74980728)(617.07886452,394.09980927)
\curveto(617.12885835,394.19980683)(617.15885832,394.30480672)(617.16886452,394.41480927)
\curveto(617.18885829,394.5248065)(617.21385826,394.63980639)(617.24386452,394.75980927)
\lineto(617.24386452,394.86480927)
\curveto(617.25385822,394.90480612)(617.25885822,394.94480608)(617.25886452,394.98480927)
\curveto(617.26885821,395.01480601)(617.26885821,395.04980598)(617.25886452,395.08980927)
\lineto(617.25886452,395.20980927)
\curveto(617.25885822,395.46980556)(617.22885825,395.71480531)(617.16886452,395.94480927)
\curveto(617.05885842,396.29480473)(616.90385857,396.58980444)(616.70386452,396.82980927)
\curveto(616.50385897,397.07980395)(616.24385923,397.27480375)(615.92386452,397.41480927)
\lineto(615.74386452,397.47480927)
\curveto(615.69385978,397.49480353)(615.63385984,397.51480351)(615.56386452,397.53480927)
\curveto(615.51385996,397.55480347)(615.45386002,397.56480346)(615.38386452,397.56480927)
\curveto(615.32386015,397.57480345)(615.25886022,397.58980344)(615.18886452,397.60980927)
\lineto(615.03886452,397.60980927)
\curveto(614.99886048,397.6298034)(614.94386053,397.63980339)(614.87386452,397.63980927)
\curveto(614.81386066,397.63980339)(614.75886072,397.6298034)(614.70886452,397.60980927)
\lineto(614.60386452,397.60980927)
\curveto(614.5738609,397.60980342)(614.53886094,397.60480342)(614.49886452,397.59480927)
\lineto(614.25886452,397.53480927)
\curveto(614.1788613,397.5248035)(614.09886138,397.50480352)(614.01886452,397.47480927)
\curveto(613.7788617,397.37480365)(613.54886193,397.23980379)(613.32886452,397.06980927)
\curveto(613.23886224,396.99980403)(613.15386232,396.9248041)(613.07386452,396.84480927)
\curveto(612.99386248,396.77480425)(612.89386258,396.71980431)(612.77386452,396.67980927)
\curveto(612.68386279,396.64980438)(612.54386293,396.63980439)(612.35386452,396.64980927)
\curveto(612.1738633,396.65980437)(612.05386342,396.68480434)(611.99386452,396.72480927)
\curveto(611.94386353,396.76480426)(611.90386357,396.8248042)(611.87386452,396.90480927)
\curveto(611.85386362,396.98480404)(611.85386362,397.06980396)(611.87386452,397.15980927)
\curveto(611.90386357,397.27980375)(611.92386355,397.39980363)(611.93386452,397.51980927)
\curveto(611.95386352,397.64980338)(611.9788635,397.77480325)(612.00886452,397.89480927)
\curveto(612.02886345,397.93480309)(612.03386344,397.96980306)(612.02386452,397.99980927)
\curveto(612.02386345,398.03980299)(612.03386344,398.08480294)(612.05386452,398.13480927)
\curveto(612.0738634,398.2248028)(612.08886339,398.31480271)(612.09886452,398.40480927)
\curveto(612.10886337,398.50480252)(612.12886335,398.59980243)(612.15886452,398.68980927)
\curveto(612.16886331,398.74980228)(612.1738633,398.80980222)(612.17386452,398.86980927)
\curveto(612.18386329,398.9298021)(612.19886328,398.98980204)(612.21886452,399.04980927)
\curveto(612.26886321,399.24980178)(612.30386317,399.45480157)(612.32386452,399.66480927)
\curveto(612.35386312,399.88480114)(612.39386308,400.09480093)(612.44386452,400.29480927)
\curveto(612.473863,400.39480063)(612.49386298,400.49480053)(612.50386452,400.59480927)
\curveto(612.51386296,400.69480033)(612.52886295,400.79480023)(612.54886452,400.89480927)
\curveto(612.55886292,400.9248001)(612.56386291,400.96480006)(612.56386452,401.01480927)
\curveto(612.59386288,401.1247999)(612.61386286,401.2297998)(612.62386452,401.32980927)
\curveto(612.64386283,401.43979959)(612.66886281,401.54979948)(612.69886452,401.65980927)
\curveto(612.71886276,401.73979929)(612.73386274,401.80979922)(612.74386452,401.86980927)
\curveto(612.75386272,401.93979909)(612.7788627,401.99979903)(612.81886452,402.04980927)
\curveto(612.83886264,402.07979895)(612.86886261,402.09979893)(612.90886452,402.10980927)
\curveto(612.94886253,402.1297989)(612.99386248,402.14979888)(613.04386452,402.16980927)
\curveto(613.10386237,402.16979886)(613.14386233,402.17479885)(613.16386452,402.18480927)
}
}
{
\newrgbcolor{curcolor}{0 0 0}
\pscustom[linestyle=none,fillstyle=solid,fillcolor=curcolor]
{
\newpath
\moveto(629.80847389,400.29480927)
\curveto(629.60846359,400.00480102)(629.3984638,399.71980131)(629.17847389,399.43980927)
\curveto(628.96846423,399.15980187)(628.76346444,398.87480215)(628.56347389,398.58480927)
\curveto(627.96346524,397.73480329)(627.35846584,396.89480413)(626.74847389,396.06480927)
\curveto(626.13846706,395.24480578)(625.53346767,394.40980662)(624.93347389,393.55980927)
\lineto(624.42347389,392.83980927)
\lineto(623.91347389,392.14980927)
\curveto(623.83346937,392.03980899)(623.75346945,391.9248091)(623.67347389,391.80480927)
\curveto(623.59346961,391.68480934)(623.4984697,391.58980944)(623.38847389,391.51980927)
\curveto(623.34846985,391.49980953)(623.28346992,391.48480954)(623.19347389,391.47480927)
\curveto(623.11347009,391.45480957)(623.02347018,391.44480958)(622.92347389,391.44480927)
\curveto(622.82347038,391.44480958)(622.72847047,391.44980958)(622.63847389,391.45980927)
\curveto(622.55847064,391.46980956)(622.4984707,391.48980954)(622.45847389,391.51980927)
\curveto(622.42847077,391.53980949)(622.4034708,391.57480945)(622.38347389,391.62480927)
\curveto(622.37347083,391.66480936)(622.37847082,391.70980932)(622.39847389,391.75980927)
\curveto(622.43847076,391.83980919)(622.48347072,391.91480911)(622.53347389,391.98480927)
\curveto(622.59347061,392.06480896)(622.64847055,392.14480888)(622.69847389,392.22480927)
\curveto(622.93847026,392.56480846)(623.18347002,392.89980813)(623.43347389,393.22980927)
\curveto(623.68346952,393.55980747)(623.92346928,393.89480713)(624.15347389,394.23480927)
\curveto(624.31346889,394.45480657)(624.47346873,394.66980636)(624.63347389,394.87980927)
\curveto(624.79346841,395.08980594)(624.95346825,395.30480572)(625.11347389,395.52480927)
\curveto(625.47346773,396.04480498)(625.83846736,396.55480447)(626.20847389,397.05480927)
\curveto(626.57846662,397.55480347)(626.94846625,398.06480296)(627.31847389,398.58480927)
\curveto(627.45846574,398.78480224)(627.5984656,398.97980205)(627.73847389,399.16980927)
\curveto(627.88846531,399.35980167)(628.03346517,399.55480147)(628.17347389,399.75480927)
\curveto(628.38346482,400.05480097)(628.5984646,400.35480067)(628.81847389,400.65480927)
\lineto(629.47847389,401.55480927)
\lineto(629.65847389,401.82480927)
\lineto(629.86847389,402.09480927)
\lineto(629.98847389,402.27480927)
\curveto(630.03846316,402.33479869)(630.08846311,402.38979864)(630.13847389,402.43980927)
\curveto(630.20846299,402.48979854)(630.28346292,402.5247985)(630.36347389,402.54480927)
\curveto(630.38346282,402.55479847)(630.40846279,402.55479847)(630.43847389,402.54480927)
\curveto(630.47846272,402.54479848)(630.50846269,402.55479847)(630.52847389,402.57480927)
\curveto(630.64846255,402.57479845)(630.78346242,402.56979846)(630.93347389,402.55980927)
\curveto(631.08346212,402.55979847)(631.17346203,402.51479851)(631.20347389,402.42480927)
\curveto(631.22346198,402.39479863)(631.22846197,402.35979867)(631.21847389,402.31980927)
\curveto(631.20846199,402.27979875)(631.19346201,402.24979878)(631.17347389,402.22980927)
\curveto(631.13346207,402.14979888)(631.09346211,402.07979895)(631.05347389,402.01980927)
\curveto(631.01346219,401.95979907)(630.96846223,401.89979913)(630.91847389,401.83980927)
\lineto(630.34847389,401.05980927)
\curveto(630.16846303,400.80980022)(629.98846321,400.55480047)(629.80847389,400.29480927)
\moveto(622.95347389,396.39480927)
\curveto(622.9034703,396.41480461)(622.85347035,396.41980461)(622.80347389,396.40980927)
\curveto(622.75347045,396.39980463)(622.7034705,396.40480462)(622.65347389,396.42480927)
\curveto(622.54347066,396.44480458)(622.43847076,396.46480456)(622.33847389,396.48480927)
\curveto(622.24847095,396.51480451)(622.15347105,396.55480447)(622.05347389,396.60480927)
\curveto(621.72347148,396.74480428)(621.46847173,396.93980409)(621.28847389,397.18980927)
\curveto(621.10847209,397.44980358)(620.96347224,397.75980327)(620.85347389,398.11980927)
\curveto(620.82347238,398.19980283)(620.8034724,398.27980275)(620.79347389,398.35980927)
\curveto(620.78347242,398.44980258)(620.76847243,398.53480249)(620.74847389,398.61480927)
\curveto(620.73847246,398.66480236)(620.73347247,398.7298023)(620.73347389,398.80980927)
\curveto(620.72347248,398.83980219)(620.71847248,398.86980216)(620.71847389,398.89980927)
\curveto(620.71847248,398.93980209)(620.71347249,398.97480205)(620.70347389,399.00480927)
\lineto(620.70347389,399.15480927)
\curveto(620.69347251,399.20480182)(620.68847251,399.26480176)(620.68847389,399.33480927)
\curveto(620.68847251,399.41480161)(620.69347251,399.47980155)(620.70347389,399.52980927)
\lineto(620.70347389,399.69480927)
\curveto(620.72347248,399.74480128)(620.72847247,399.78980124)(620.71847389,399.82980927)
\curveto(620.71847248,399.87980115)(620.72347248,399.9248011)(620.73347389,399.96480927)
\curveto(620.74347246,400.00480102)(620.74847245,400.03980099)(620.74847389,400.06980927)
\curveto(620.74847245,400.10980092)(620.75347245,400.14980088)(620.76347389,400.18980927)
\curveto(620.79347241,400.29980073)(620.81347239,400.40980062)(620.82347389,400.51980927)
\curveto(620.84347236,400.63980039)(620.87847232,400.75480027)(620.92847389,400.86480927)
\curveto(621.06847213,401.20479982)(621.22847197,401.47979955)(621.40847389,401.68980927)
\curveto(621.5984716,401.90979912)(621.86847133,402.08979894)(622.21847389,402.22980927)
\curveto(622.2984709,402.25979877)(622.38347082,402.27979875)(622.47347389,402.28980927)
\curveto(622.56347064,402.30979872)(622.65847054,402.3297987)(622.75847389,402.34980927)
\curveto(622.78847041,402.35979867)(622.84347036,402.35979867)(622.92347389,402.34980927)
\curveto(623.0034702,402.34979868)(623.05347015,402.35979867)(623.07347389,402.37980927)
\curveto(623.63346957,402.38979864)(624.08346912,402.27979875)(624.42347389,402.04980927)
\curveto(624.77346843,401.81979921)(625.03346817,401.51479951)(625.20347389,401.13480927)
\curveto(625.24346796,401.04479998)(625.27846792,400.94980008)(625.30847389,400.84980927)
\curveto(625.33846786,400.74980028)(625.36346784,400.64980038)(625.38347389,400.54980927)
\curveto(625.4034678,400.51980051)(625.40846779,400.48980054)(625.39847389,400.45980927)
\curveto(625.3984678,400.4298006)(625.4034678,400.39980063)(625.41347389,400.36980927)
\curveto(625.44346776,400.25980077)(625.46346774,400.13480089)(625.47347389,399.99480927)
\curveto(625.48346772,399.86480116)(625.49346771,399.7298013)(625.50347389,399.58980927)
\lineto(625.50347389,399.42480927)
\curveto(625.51346769,399.36480166)(625.51346769,399.30980172)(625.50347389,399.25980927)
\curveto(625.49346771,399.20980182)(625.48846771,399.15980187)(625.48847389,399.10980927)
\lineto(625.48847389,398.97480927)
\curveto(625.47846772,398.93480209)(625.47346773,398.89480213)(625.47347389,398.85480927)
\curveto(625.48346772,398.81480221)(625.47846772,398.76980226)(625.45847389,398.71980927)
\curveto(625.43846776,398.60980242)(625.41846778,398.50480252)(625.39847389,398.40480927)
\curveto(625.38846781,398.30480272)(625.36846783,398.20480282)(625.33847389,398.10480927)
\curveto(625.20846799,397.74480328)(625.04346816,397.4298036)(624.84347389,397.15980927)
\curveto(624.64346856,396.88980414)(624.36846883,396.68480434)(624.01847389,396.54480927)
\curveto(623.93846926,396.51480451)(623.85346935,396.48980454)(623.76347389,396.46980927)
\lineto(623.49347389,396.40980927)
\curveto(623.44346976,396.39980463)(623.3984698,396.39480463)(623.35847389,396.39480927)
\curveto(623.31846988,396.40480462)(623.27846992,396.40480462)(623.23847389,396.39480927)
\curveto(623.13847006,396.37480465)(623.04347016,396.37480465)(622.95347389,396.39480927)
\moveto(622.11347389,397.78980927)
\curveto(622.15347105,397.71980331)(622.19347101,397.65480337)(622.23347389,397.59480927)
\curveto(622.27347093,397.54480348)(622.32347088,397.49480353)(622.38347389,397.44480927)
\lineto(622.53347389,397.32480927)
\curveto(622.59347061,397.29480373)(622.65847054,397.26980376)(622.72847389,397.24980927)
\curveto(622.76847043,397.2298038)(622.8034704,397.21980381)(622.83347389,397.21980927)
\curveto(622.87347033,397.2298038)(622.91347029,397.2248038)(622.95347389,397.20480927)
\curveto(622.98347022,397.20480382)(623.02347018,397.19980383)(623.07347389,397.18980927)
\curveto(623.12347008,397.18980384)(623.16347004,397.19480383)(623.19347389,397.20480927)
\lineto(623.41847389,397.24980927)
\curveto(623.66846953,397.3298037)(623.85346935,397.45480357)(623.97347389,397.62480927)
\curveto(624.05346915,397.7248033)(624.12346908,397.85480317)(624.18347389,398.01480927)
\curveto(624.26346894,398.19480283)(624.32346888,398.41980261)(624.36347389,398.68980927)
\curveto(624.4034688,398.96980206)(624.41846878,399.24980178)(624.40847389,399.52980927)
\curveto(624.3984688,399.81980121)(624.36846883,400.09480093)(624.31847389,400.35480927)
\curveto(624.26846893,400.61480041)(624.19346901,400.8248002)(624.09347389,400.98480927)
\curveto(623.97346923,401.18479984)(623.82346938,401.33479969)(623.64347389,401.43480927)
\curveto(623.56346964,401.48479954)(623.47346973,401.51479951)(623.37347389,401.52480927)
\curveto(623.27346993,401.54479948)(623.16847003,401.55479947)(623.05847389,401.55480927)
\curveto(623.03847016,401.54479948)(623.01347019,401.53979949)(622.98347389,401.53980927)
\curveto(622.96347024,401.54979948)(622.94347026,401.54979948)(622.92347389,401.53980927)
\curveto(622.87347033,401.5297995)(622.82847037,401.51979951)(622.78847389,401.50980927)
\curveto(622.74847045,401.50979952)(622.70847049,401.49979953)(622.66847389,401.47980927)
\curveto(622.48847071,401.39979963)(622.33847086,401.27979975)(622.21847389,401.11980927)
\curveto(622.10847109,400.95980007)(622.01847118,400.77980025)(621.94847389,400.57980927)
\curveto(621.88847131,400.38980064)(621.84347136,400.16480086)(621.81347389,399.90480927)
\curveto(621.79347141,399.64480138)(621.78847141,399.37980165)(621.79847389,399.10980927)
\curveto(621.80847139,398.84980218)(621.83847136,398.59980243)(621.88847389,398.35980927)
\curveto(621.94847125,398.1298029)(622.02347118,397.93980309)(622.11347389,397.78980927)
\moveto(632.91347389,394.80480927)
\curveto(632.92346028,394.75480627)(632.92846027,394.66480636)(632.92847389,394.53480927)
\curveto(632.92846027,394.40480662)(632.91846028,394.31480671)(632.89847389,394.26480927)
\curveto(632.87846032,394.21480681)(632.87346033,394.15980687)(632.88347389,394.09980927)
\curveto(632.89346031,394.04980698)(632.89346031,393.99980703)(632.88347389,393.94980927)
\curveto(632.84346036,393.80980722)(632.81346039,393.67480735)(632.79347389,393.54480927)
\curveto(632.78346042,393.41480761)(632.75346045,393.29480773)(632.70347389,393.18480927)
\curveto(632.56346064,392.83480819)(632.3984608,392.53980849)(632.20847389,392.29980927)
\curveto(632.01846118,392.06980896)(631.74846145,391.88480914)(631.39847389,391.74480927)
\curveto(631.31846188,391.71480931)(631.23346197,391.69480933)(631.14347389,391.68480927)
\curveto(631.05346215,391.66480936)(630.96846223,391.64480938)(630.88847389,391.62480927)
\curveto(630.83846236,391.61480941)(630.78846241,391.60980942)(630.73847389,391.60980927)
\curveto(630.68846251,391.60980942)(630.63846256,391.60480942)(630.58847389,391.59480927)
\curveto(630.55846264,391.58480944)(630.50846269,391.58480944)(630.43847389,391.59480927)
\curveto(630.36846283,391.59480943)(630.31846288,391.59980943)(630.28847389,391.60980927)
\curveto(630.22846297,391.6298094)(630.16846303,391.63980939)(630.10847389,391.63980927)
\curveto(630.05846314,391.6298094)(630.00846319,391.63480939)(629.95847389,391.65480927)
\curveto(629.86846333,391.67480935)(629.77846342,391.69980933)(629.68847389,391.72980927)
\curveto(629.60846359,391.74980928)(629.52846367,391.77980925)(629.44847389,391.81980927)
\curveto(629.12846407,391.95980907)(628.87846432,392.15480887)(628.69847389,392.40480927)
\curveto(628.51846468,392.66480836)(628.36846483,392.96980806)(628.24847389,393.31980927)
\curveto(628.22846497,393.39980763)(628.21346499,393.48480754)(628.20347389,393.57480927)
\curveto(628.19346501,393.66480736)(628.17846502,393.74980728)(628.15847389,393.82980927)
\curveto(628.14846505,393.85980717)(628.14346506,393.88980714)(628.14347389,393.91980927)
\lineto(628.14347389,394.02480927)
\curveto(628.12346508,394.10480692)(628.11346509,394.18480684)(628.11347389,394.26480927)
\lineto(628.11347389,394.39980927)
\curveto(628.09346511,394.49980653)(628.09346511,394.59980643)(628.11347389,394.69980927)
\lineto(628.11347389,394.87980927)
\curveto(628.12346508,394.9298061)(628.12846507,394.97480605)(628.12847389,395.01480927)
\curveto(628.12846507,395.06480596)(628.13346507,395.10980592)(628.14347389,395.14980927)
\curveto(628.15346505,395.18980584)(628.15846504,395.2248058)(628.15847389,395.25480927)
\curveto(628.15846504,395.29480573)(628.16346504,395.33480569)(628.17347389,395.37480927)
\lineto(628.23347389,395.70480927)
\curveto(628.25346495,395.8248052)(628.28346492,395.93480509)(628.32347389,396.03480927)
\curveto(628.46346474,396.36480466)(628.62346458,396.63980439)(628.80347389,396.85980927)
\curveto(628.99346421,397.08980394)(629.25346395,397.27480375)(629.58347389,397.41480927)
\curveto(629.66346354,397.45480357)(629.74846345,397.47980355)(629.83847389,397.48980927)
\lineto(630.13847389,397.54980927)
\lineto(630.27347389,397.54980927)
\curveto(630.32346288,397.55980347)(630.37346283,397.56480346)(630.42347389,397.56480927)
\curveto(630.99346221,397.58480344)(631.45346175,397.47980355)(631.80347389,397.24980927)
\curveto(632.16346104,397.029804)(632.42846077,396.7298043)(632.59847389,396.34980927)
\curveto(632.64846055,396.24980478)(632.68846051,396.14980488)(632.71847389,396.04980927)
\curveto(632.74846045,395.94980508)(632.77846042,395.84480518)(632.80847389,395.73480927)
\curveto(632.81846038,395.69480533)(632.82346038,395.65980537)(632.82347389,395.62980927)
\curveto(632.82346038,395.60980542)(632.82846037,395.57980545)(632.83847389,395.53980927)
\curveto(632.85846034,395.46980556)(632.86846033,395.39480563)(632.86847389,395.31480927)
\curveto(632.86846033,395.23480579)(632.87846032,395.15480587)(632.89847389,395.07480927)
\curveto(632.8984603,395.024806)(632.8984603,394.97980605)(632.89847389,394.93980927)
\curveto(632.8984603,394.89980613)(632.9034603,394.85480617)(632.91347389,394.80480927)
\moveto(631.80347389,394.36980927)
\curveto(631.81346139,394.41980661)(631.81846138,394.49480653)(631.81847389,394.59480927)
\curveto(631.82846137,394.69480633)(631.82346138,394.76980626)(631.80347389,394.81980927)
\curveto(631.78346142,394.87980615)(631.77846142,394.93480609)(631.78847389,394.98480927)
\curveto(631.80846139,395.04480598)(631.80846139,395.10480592)(631.78847389,395.16480927)
\curveto(631.77846142,395.19480583)(631.77346143,395.2298058)(631.77347389,395.26980927)
\curveto(631.77346143,395.30980572)(631.76846143,395.34980568)(631.75847389,395.38980927)
\curveto(631.73846146,395.46980556)(631.71846148,395.54480548)(631.69847389,395.61480927)
\curveto(631.68846151,395.69480533)(631.67346153,395.77480525)(631.65347389,395.85480927)
\curveto(631.62346158,395.91480511)(631.5984616,395.97480505)(631.57847389,396.03480927)
\curveto(631.55846164,396.09480493)(631.52846167,396.15480487)(631.48847389,396.21480927)
\curveto(631.38846181,396.38480464)(631.25846194,396.51980451)(631.09847389,396.61980927)
\curveto(631.01846218,396.66980436)(630.92346228,396.70480432)(630.81347389,396.72480927)
\curveto(630.7034625,396.74480428)(630.57846262,396.75480427)(630.43847389,396.75480927)
\curveto(630.41846278,396.74480428)(630.39346281,396.73980429)(630.36347389,396.73980927)
\curveto(630.33346287,396.74980428)(630.3034629,396.74980428)(630.27347389,396.73980927)
\lineto(630.12347389,396.67980927)
\curveto(630.07346313,396.66980436)(630.02846317,396.65480437)(629.98847389,396.63480927)
\curveto(629.7984634,396.5248045)(629.65346355,396.37980465)(629.55347389,396.19980927)
\curveto(629.46346374,396.01980501)(629.38346382,395.81480521)(629.31347389,395.58480927)
\curveto(629.27346393,395.45480557)(629.25346395,395.31980571)(629.25347389,395.17980927)
\curveto(629.25346395,395.04980598)(629.24346396,394.90480612)(629.22347389,394.74480927)
\curveto(629.21346399,394.69480633)(629.203464,394.63480639)(629.19347389,394.56480927)
\curveto(629.19346401,394.49480653)(629.203464,394.43480659)(629.22347389,394.38480927)
\lineto(629.22347389,394.21980927)
\lineto(629.22347389,394.03980927)
\curveto(629.23346397,393.98980704)(629.24346396,393.93480709)(629.25347389,393.87480927)
\curveto(629.26346394,393.8248072)(629.26846393,393.76980726)(629.26847389,393.70980927)
\curveto(629.27846392,393.64980738)(629.29346391,393.59480743)(629.31347389,393.54480927)
\curveto(629.36346384,393.35480767)(629.42346378,393.17980785)(629.49347389,393.01980927)
\curveto(629.56346364,392.85980817)(629.66846353,392.7298083)(629.80847389,392.62980927)
\curveto(629.93846326,392.5298085)(630.07846312,392.45980857)(630.22847389,392.41980927)
\curveto(630.25846294,392.40980862)(630.28346292,392.40480862)(630.30347389,392.40480927)
\curveto(630.33346287,392.41480861)(630.36346284,392.41480861)(630.39347389,392.40480927)
\curveto(630.41346279,392.40480862)(630.44346276,392.39980863)(630.48347389,392.38980927)
\curveto(630.52346268,392.38980864)(630.55846264,392.39480863)(630.58847389,392.40480927)
\curveto(630.62846257,392.41480861)(630.66846253,392.41980861)(630.70847389,392.41980927)
\curveto(630.74846245,392.41980861)(630.78846241,392.4298086)(630.82847389,392.44980927)
\curveto(631.06846213,392.5298085)(631.26346194,392.66480836)(631.41347389,392.85480927)
\curveto(631.53346167,393.03480799)(631.62346158,393.23980779)(631.68347389,393.46980927)
\curveto(631.7034615,393.53980749)(631.71846148,393.60980742)(631.72847389,393.67980927)
\curveto(631.73846146,393.75980727)(631.75346145,393.83980719)(631.77347389,393.91980927)
\curveto(631.77346143,393.97980705)(631.77846142,394.024807)(631.78847389,394.05480927)
\curveto(631.78846141,394.07480695)(631.78846141,394.09980693)(631.78847389,394.12980927)
\curveto(631.78846141,394.16980686)(631.79346141,394.19980683)(631.80347389,394.21980927)
\lineto(631.80347389,394.36980927)
}
}
{
\newrgbcolor{curcolor}{0 0 0}
\pscustom[linestyle=none,fillstyle=solid,fillcolor=curcolor]
{
\newpath
\moveto(215.59168251,742.36980927)
\curveto(215.59167488,742.28980374)(215.59667487,742.20980382)(215.60668251,742.12980927)
\curveto(215.61667485,742.04980398)(215.61167486,741.97480405)(215.59168251,741.90480927)
\curveto(215.5716749,741.86480416)(215.5666749,741.81980421)(215.57668251,741.76980927)
\curveto(215.58667488,741.7298043)(215.58667488,741.68980434)(215.57668251,741.64980927)
\lineto(215.57668251,741.49980927)
\curveto(215.5666749,741.40980462)(215.56167491,741.31980471)(215.56168251,741.22980927)
\curveto(215.56167491,741.14980488)(215.55667491,741.06980496)(215.54668251,740.98980927)
\lineto(215.51668251,740.74980927)
\curveto(215.50667496,740.67980535)(215.49667497,740.60480542)(215.48668251,740.52480927)
\curveto(215.47667499,740.48480554)(215.471675,740.44480558)(215.47168251,740.40480927)
\curveto(215.471675,740.36480566)(215.466675,740.31980571)(215.45668251,740.26980927)
\curveto(215.41667505,740.1298059)(215.38667508,739.98980604)(215.36668251,739.84980927)
\curveto(215.35667511,739.70980632)(215.32667514,739.57480645)(215.27668251,739.44480927)
\curveto(215.22667524,739.27480675)(215.1716753,739.10980692)(215.11168251,738.94980927)
\curveto(215.06167541,738.78980724)(215.00167547,738.63480739)(214.93168251,738.48480927)
\curveto(214.91167556,738.4248076)(214.88167559,738.36480766)(214.84168251,738.30480927)
\lineto(214.75168251,738.15480927)
\curveto(214.55167592,737.83480819)(214.33667613,737.56980846)(214.10668251,737.35980927)
\curveto(213.87667659,737.14980888)(213.58167689,736.96980906)(213.22168251,736.81980927)
\curveto(213.10167737,736.76980926)(212.9716775,736.73480929)(212.83168251,736.71480927)
\curveto(212.70167777,736.69480933)(212.5666779,736.66980936)(212.42668251,736.63980927)
\curveto(212.3666781,736.6298094)(212.30667816,736.6248094)(212.24668251,736.62480927)
\curveto(212.18667828,736.6248094)(212.12167835,736.61980941)(212.05168251,736.60980927)
\curveto(212.02167845,736.59980943)(211.9716785,736.59980943)(211.90168251,736.60980927)
\lineto(211.75168251,736.60980927)
\lineto(211.60168251,736.60980927)
\curveto(211.52167895,736.6298094)(211.43667903,736.64480938)(211.34668251,736.65480927)
\curveto(211.2666792,736.65480937)(211.19167928,736.66480936)(211.12168251,736.68480927)
\curveto(211.08167939,736.69480933)(211.04667942,736.69980933)(211.01668251,736.69980927)
\curveto(210.99667947,736.68980934)(210.9716795,736.69480933)(210.94168251,736.71480927)
\lineto(210.67168251,736.77480927)
\curveto(210.58167989,736.80480922)(210.49667997,736.83480919)(210.41668251,736.86480927)
\curveto(209.83668063,737.10480892)(209.40168107,737.47480855)(209.11168251,737.97480927)
\curveto(209.03168144,738.10480792)(208.9666815,738.23980779)(208.91668251,738.37980927)
\curveto(208.87668159,738.51980751)(208.83168164,738.66980736)(208.78168251,738.82980927)
\curveto(208.76168171,738.90980712)(208.75668171,738.98980704)(208.76668251,739.06980927)
\curveto(208.78668168,739.14980688)(208.82168165,739.20480682)(208.87168251,739.23480927)
\curveto(208.90168157,739.25480677)(208.95668151,739.26980676)(209.03668251,739.27980927)
\curveto(209.11668135,739.29980673)(209.20168127,739.30980672)(209.29168251,739.30980927)
\curveto(209.38168109,739.31980671)(209.466681,739.31980671)(209.54668251,739.30980927)
\curveto(209.63668083,739.29980673)(209.70668076,739.28980674)(209.75668251,739.27980927)
\curveto(209.77668069,739.26980676)(209.80168067,739.25480677)(209.83168251,739.23480927)
\curveto(209.8716806,739.21480681)(209.90168057,739.19480683)(209.92168251,739.17480927)
\curveto(209.98168049,739.09480693)(210.02668044,738.99980703)(210.05668251,738.88980927)
\curveto(210.09668037,738.77980725)(210.14168033,738.67980735)(210.19168251,738.58980927)
\curveto(210.44168003,738.19980783)(210.81167966,737.9298081)(211.30168251,737.77980927)
\curveto(211.3716791,737.75980827)(211.44167903,737.74480828)(211.51168251,737.73480927)
\curveto(211.59167888,737.73480829)(211.6716788,737.7248083)(211.75168251,737.70480927)
\curveto(211.79167868,737.69480833)(211.84667862,737.68980834)(211.91668251,737.68980927)
\curveto(211.99667847,737.68980834)(212.05167842,737.69480833)(212.08168251,737.70480927)
\curveto(212.11167836,737.71480831)(212.14167833,737.71980831)(212.17168251,737.71980927)
\lineto(212.27668251,737.71980927)
\curveto(212.35667811,737.73980829)(212.43167804,737.75980827)(212.50168251,737.77980927)
\curveto(212.58167789,737.79980823)(212.65667781,737.8248082)(212.72668251,737.85480927)
\curveto(213.07667739,738.00480802)(213.34667712,738.21980781)(213.53668251,738.49980927)
\curveto(213.72667674,738.77980725)(213.88167659,739.10480692)(214.00168251,739.47480927)
\curveto(214.03167644,739.55480647)(214.05167642,739.6298064)(214.06168251,739.69980927)
\curveto(214.08167639,739.76980626)(214.10167637,739.84480618)(214.12168251,739.92480927)
\curveto(214.14167633,740.01480601)(214.15667631,740.10980592)(214.16668251,740.20980927)
\curveto(214.18667628,740.31980571)(214.20667626,740.4248056)(214.22668251,740.52480927)
\curveto(214.23667623,740.57480545)(214.24167623,740.6248054)(214.24168251,740.67480927)
\curveto(214.25167622,740.73480529)(214.25667621,740.78980524)(214.25668251,740.83980927)
\curveto(214.27667619,740.89980513)(214.28667618,740.97480505)(214.28668251,741.06480927)
\curveto(214.28667618,741.16480486)(214.27667619,741.24480478)(214.25668251,741.30480927)
\curveto(214.22667624,741.39480463)(214.17667629,741.43480459)(214.10668251,741.42480927)
\curveto(214.04667642,741.41480461)(213.99167648,741.38480464)(213.94168251,741.33480927)
\curveto(213.86167661,741.28480474)(213.79167668,741.2248048)(213.73168251,741.15480927)
\curveto(213.68167679,741.08480494)(213.61667685,741.024805)(213.53668251,740.97480927)
\curveto(213.37667709,740.86480516)(213.21167726,740.76480526)(213.04168251,740.67480927)
\curveto(212.8716776,740.59480543)(212.67667779,740.5248055)(212.45668251,740.46480927)
\curveto(212.35667811,740.43480559)(212.25667821,740.41980561)(212.15668251,740.41980927)
\curveto(212.0666784,740.41980561)(211.9666785,740.40980562)(211.85668251,740.38980927)
\lineto(211.70668251,740.38980927)
\curveto(211.65667881,740.40980562)(211.60667886,740.41480561)(211.55668251,740.40480927)
\curveto(211.51667895,740.39480563)(211.47667899,740.39480563)(211.43668251,740.40480927)
\curveto(211.40667906,740.41480561)(211.36167911,740.41980561)(211.30168251,740.41980927)
\curveto(211.24167923,740.4298056)(211.17667929,740.43980559)(211.10668251,740.44980927)
\lineto(210.92668251,740.47980927)
\curveto(210.47667999,740.59980543)(210.09668037,740.76480526)(209.78668251,740.97480927)
\curveto(209.51668095,741.16480486)(209.28668118,741.39480463)(209.09668251,741.66480927)
\curveto(208.91668155,741.94480408)(208.7716817,742.25980377)(208.66168251,742.60980927)
\lineto(208.60168251,742.81980927)
\curveto(208.59168188,742.89980313)(208.57668189,742.97980305)(208.55668251,743.05980927)
\curveto(208.54668192,743.08980294)(208.54168193,743.11980291)(208.54168251,743.14980927)
\curveto(208.54168193,743.17980285)(208.53668193,743.20980282)(208.52668251,743.23980927)
\curveto(208.51668195,743.29980273)(208.51168196,743.35980267)(208.51168251,743.41980927)
\curveto(208.51168196,743.48980254)(208.50168197,743.54980248)(208.48168251,743.59980927)
\lineto(208.48168251,743.77980927)
\curveto(208.471682,743.8298022)(208.466682,743.89980213)(208.46668251,743.98980927)
\curveto(208.466682,744.07980195)(208.47668199,744.14980188)(208.49668251,744.19980927)
\lineto(208.49668251,744.36480927)
\curveto(208.51668195,744.44480158)(208.52668194,744.51980151)(208.52668251,744.58980927)
\curveto(208.53668193,744.65980137)(208.55168192,744.7298013)(208.57168251,744.79980927)
\curveto(208.63168184,744.99980103)(208.69168178,745.18980084)(208.75168251,745.36980927)
\curveto(208.82168165,745.54980048)(208.91168156,745.71980031)(209.02168251,745.87980927)
\curveto(209.06168141,745.94980008)(209.10168137,746.01480001)(209.14168251,746.07480927)
\lineto(209.29168251,746.25480927)
\curveto(209.31168116,746.26479976)(209.33168114,746.27979975)(209.35168251,746.29980927)
\curveto(209.44168103,746.4297996)(209.55168092,746.53979949)(209.68168251,746.62980927)
\curveto(209.94168053,746.8297992)(210.20668026,746.98479904)(210.47668251,747.09480927)
\curveto(210.55667991,747.13479889)(210.63667983,747.16479886)(210.71668251,747.18480927)
\curveto(210.80667966,747.21479881)(210.89667957,747.23979879)(210.98668251,747.25980927)
\curveto(211.08667938,747.28979874)(211.18667928,747.30979872)(211.28668251,747.31980927)
\curveto(211.38667908,747.3297987)(211.49167898,747.34479868)(211.60168251,747.36480927)
\curveto(211.63167884,747.37479865)(211.6716788,747.37479865)(211.72168251,747.36480927)
\curveto(211.78167869,747.35479867)(211.82167865,747.35979867)(211.84168251,747.37980927)
\curveto(212.56167791,747.39979863)(213.16167731,747.28479874)(213.64168251,747.03480927)
\curveto(214.12167635,746.78479924)(214.49667597,746.44479958)(214.76668251,746.01480927)
\curveto(214.85667561,745.87480015)(214.93667553,745.7298003)(215.00668251,745.57980927)
\curveto(215.07667539,745.4298006)(215.14667532,745.26980076)(215.21668251,745.09980927)
\curveto(215.2666752,744.95980107)(215.30667516,744.80980122)(215.33668251,744.64980927)
\curveto(215.3666751,744.48980154)(215.40167507,744.3298017)(215.44168251,744.16980927)
\curveto(215.46167501,744.11980191)(215.471675,744.06480196)(215.47168251,744.00480927)
\curveto(215.471675,743.95480207)(215.47667499,743.90480212)(215.48668251,743.85480927)
\curveto(215.50667496,743.79480223)(215.51667495,743.7298023)(215.51668251,743.65980927)
\curveto(215.51667495,743.59980243)(215.52667494,743.54480248)(215.54668251,743.49480927)
\lineto(215.54668251,743.32980927)
\curveto(215.5666749,743.27980275)(215.5716749,743.2298028)(215.56168251,743.17980927)
\curveto(215.55167492,743.1298029)(215.55667491,743.07980295)(215.57668251,743.02980927)
\curveto(215.57667489,743.00980302)(215.5716749,742.98480304)(215.56168251,742.95480927)
\curveto(215.56167491,742.9248031)(215.5666749,742.89980313)(215.57668251,742.87980927)
\curveto(215.58667488,742.84980318)(215.58667488,742.81480321)(215.57668251,742.77480927)
\curveto(215.57667489,742.73480329)(215.58167489,742.69480333)(215.59168251,742.65480927)
\curveto(215.60167487,742.61480341)(215.60167487,742.56980346)(215.59168251,742.51980927)
\lineto(215.59168251,742.36980927)
\moveto(214.09168251,743.67480927)
\curveto(214.10167637,743.7248023)(214.10667636,743.78480224)(214.10668251,743.85480927)
\curveto(214.10667636,743.9248021)(214.10167637,743.98480204)(214.09168251,744.03480927)
\curveto(214.08167639,744.08480194)(214.07667639,744.15980187)(214.07668251,744.25980927)
\curveto(214.05667641,744.33980169)(214.03667643,744.41480161)(214.01668251,744.48480927)
\curveto(214.00667646,744.55480147)(213.99167648,744.6248014)(213.97168251,744.69480927)
\curveto(213.83167664,745.1248009)(213.63667683,745.45980057)(213.38668251,745.69980927)
\curveto(213.14667732,745.93980009)(212.80167767,746.11979991)(212.35168251,746.23980927)
\curveto(212.26167821,746.25979977)(212.16167831,746.26979976)(212.05168251,746.26980927)
\lineto(211.72168251,746.26980927)
\curveto(211.70167877,746.24979978)(211.6666788,746.23979979)(211.61668251,746.23980927)
\curveto(211.5666789,746.24979978)(211.52167895,746.24979978)(211.48168251,746.23980927)
\curveto(211.40167907,746.21979981)(211.32667914,746.19979983)(211.25668251,746.17980927)
\lineto(211.04668251,746.11980927)
\curveto(210.75667971,745.98980004)(210.52667994,745.80980022)(210.35668251,745.57980927)
\curveto(210.18668028,745.35980067)(210.05168042,745.09980093)(209.95168251,744.79980927)
\curveto(209.92168055,744.70980132)(209.89668057,744.61480141)(209.87668251,744.51480927)
\curveto(209.8666806,744.4248016)(209.85168062,744.3298017)(209.83168251,744.22980927)
\lineto(209.83168251,744.09480927)
\curveto(209.80168067,743.98480204)(209.79168068,743.84480218)(209.80168251,743.67480927)
\curveto(209.82168065,743.51480251)(209.84168063,743.38480264)(209.86168251,743.28480927)
\curveto(209.88168059,743.2248028)(209.89668057,743.16480286)(209.90668251,743.10480927)
\curveto(209.91668055,743.05480297)(209.93168054,743.00480302)(209.95168251,742.95480927)
\curveto(210.03168044,742.75480327)(210.12668034,742.56480346)(210.23668251,742.38480927)
\curveto(210.35668011,742.20480382)(210.49667997,742.05980397)(210.65668251,741.94980927)
\curveto(210.70667976,741.89980413)(210.76167971,741.85980417)(210.82168251,741.82980927)
\curveto(210.88167959,741.79980423)(210.94167953,741.76480426)(211.00168251,741.72480927)
\curveto(211.15167932,741.64480438)(211.33667913,741.57980445)(211.55668251,741.52980927)
\curveto(211.60667886,741.50980452)(211.64667882,741.50480452)(211.67668251,741.51480927)
\curveto(211.71667875,741.5248045)(211.76167871,741.51980451)(211.81168251,741.49980927)
\curveto(211.85167862,741.48980454)(211.90667856,741.48480454)(211.97668251,741.48480927)
\curveto(212.04667842,741.48480454)(212.10667836,741.48980454)(212.15668251,741.49980927)
\curveto(212.25667821,741.51980451)(212.35167812,741.53480449)(212.44168251,741.54480927)
\curveto(212.53167794,741.56480446)(212.62167785,741.59480443)(212.71168251,741.63480927)
\curveto(213.25167722,741.85480417)(213.64667682,742.24980378)(213.89668251,742.81980927)
\curveto(213.94667652,742.91980311)(213.98167649,743.01980301)(214.00168251,743.11980927)
\curveto(214.02167645,743.2298028)(214.04667642,743.33980269)(214.07668251,743.44980927)
\curveto(214.07667639,743.54980248)(214.08167639,743.6248024)(214.09168251,743.67480927)
}
}
{
\newrgbcolor{curcolor}{0 0 0}
\pscustom[linestyle=none,fillstyle=solid,fillcolor=curcolor]
{
\newpath
\moveto(218.51129189,747.18480927)
\lineto(222.11129189,747.18480927)
\lineto(222.75629189,747.18480927)
\curveto(222.83628536,747.18479884)(222.91128528,747.17979885)(222.98129189,747.16980927)
\curveto(223.05128514,747.16979886)(223.11128508,747.15979887)(223.16129189,747.13980927)
\curveto(223.23128496,747.10979892)(223.28628491,747.04979898)(223.32629189,746.95980927)
\curveto(223.34628485,746.9297991)(223.35628484,746.88979914)(223.35629189,746.83980927)
\lineto(223.35629189,746.70480927)
\curveto(223.36628483,746.59479943)(223.36128483,746.48979954)(223.34129189,746.38980927)
\curveto(223.33128486,746.28979974)(223.2962849,746.21979981)(223.23629189,746.17980927)
\curveto(223.14628505,746.10979992)(223.01128518,746.07479995)(222.83129189,746.07480927)
\curveto(222.65128554,746.08479994)(222.48628571,746.08979994)(222.33629189,746.08980927)
\lineto(220.34129189,746.08980927)
\lineto(219.84629189,746.08980927)
\lineto(219.71129189,746.08980927)
\curveto(219.67128852,746.08979994)(219.63128856,746.08479994)(219.59129189,746.07480927)
\lineto(219.38129189,746.07480927)
\curveto(219.27128892,746.04479998)(219.191289,746.00480002)(219.14129189,745.95480927)
\curveto(219.0912891,745.91480011)(219.05628914,745.85980017)(219.03629189,745.78980927)
\curveto(219.01628918,745.7298003)(219.00128919,745.65980037)(218.99129189,745.57980927)
\curveto(218.98128921,745.49980053)(218.96128923,745.40980062)(218.93129189,745.30980927)
\curveto(218.88128931,745.10980092)(218.84128935,744.90480112)(218.81129189,744.69480927)
\curveto(218.78128941,744.48480154)(218.74128945,744.27980175)(218.69129189,744.07980927)
\curveto(218.67128952,744.00980202)(218.66128953,743.93980209)(218.66129189,743.86980927)
\curveto(218.66128953,743.80980222)(218.65128954,743.74480228)(218.63129189,743.67480927)
\curveto(218.62128957,743.64480238)(218.61128958,743.60480242)(218.60129189,743.55480927)
\curveto(218.60128959,743.51480251)(218.60628959,743.47480255)(218.61629189,743.43480927)
\curveto(218.63628956,743.38480264)(218.66128953,743.33980269)(218.69129189,743.29980927)
\curveto(218.73128946,743.26980276)(218.7912894,743.26480276)(218.87129189,743.28480927)
\curveto(218.93128926,743.30480272)(218.9912892,743.3298027)(219.05129189,743.35980927)
\curveto(219.11128908,743.39980263)(219.17128902,743.43480259)(219.23129189,743.46480927)
\curveto(219.2912889,743.48480254)(219.34128885,743.49980253)(219.38129189,743.50980927)
\curveto(219.57128862,743.58980244)(219.77628842,743.64480238)(219.99629189,743.67480927)
\curveto(220.22628797,743.70480232)(220.45628774,743.71480231)(220.68629189,743.70480927)
\curveto(220.92628727,743.70480232)(221.15628704,743.67980235)(221.37629189,743.62980927)
\curveto(221.5962866,743.58980244)(221.7962864,743.5298025)(221.97629189,743.44980927)
\curveto(222.02628617,743.4298026)(222.07128612,743.40980262)(222.11129189,743.38980927)
\curveto(222.16128603,743.36980266)(222.21128598,743.34480268)(222.26129189,743.31480927)
\curveto(222.61128558,743.10480292)(222.8912853,742.87480315)(223.10129189,742.62480927)
\curveto(223.32128487,742.37480365)(223.51628468,742.04980398)(223.68629189,741.64980927)
\curveto(223.73628446,741.53980449)(223.77128442,741.4298046)(223.79129189,741.31980927)
\curveto(223.81128438,741.20980482)(223.83628436,741.09480493)(223.86629189,740.97480927)
\curveto(223.87628432,740.94480508)(223.88128431,740.89980513)(223.88129189,740.83980927)
\curveto(223.90128429,740.77980525)(223.91128428,740.70980532)(223.91129189,740.62980927)
\curveto(223.91128428,740.55980547)(223.92128427,740.49480553)(223.94129189,740.43480927)
\lineto(223.94129189,740.26980927)
\curveto(223.95128424,740.21980581)(223.95628424,740.14980588)(223.95629189,740.05980927)
\curveto(223.95628424,739.96980606)(223.94628425,739.89980613)(223.92629189,739.84980927)
\curveto(223.90628429,739.78980624)(223.90128429,739.7298063)(223.91129189,739.66980927)
\curveto(223.92128427,739.61980641)(223.91628428,739.56980646)(223.89629189,739.51980927)
\curveto(223.85628434,739.35980667)(223.82128437,739.20980682)(223.79129189,739.06980927)
\curveto(223.76128443,738.9298071)(223.71628448,738.79480723)(223.65629189,738.66480927)
\curveto(223.4962847,738.29480773)(223.27628492,737.95980807)(222.99629189,737.65980927)
\curveto(222.71628548,737.35980867)(222.3962858,737.1298089)(222.03629189,736.96980927)
\curveto(221.86628633,736.88980914)(221.66628653,736.81480921)(221.43629189,736.74480927)
\curveto(221.32628687,736.70480932)(221.21128698,736.67980935)(221.09129189,736.66980927)
\curveto(220.97128722,736.65980937)(220.85128734,736.63980939)(220.73129189,736.60980927)
\curveto(220.68128751,736.58980944)(220.62628757,736.58980944)(220.56629189,736.60980927)
\curveto(220.50628769,736.61980941)(220.44628775,736.61480941)(220.38629189,736.59480927)
\curveto(220.28628791,736.57480945)(220.18628801,736.57480945)(220.08629189,736.59480927)
\lineto(219.95129189,736.59480927)
\curveto(219.90128829,736.61480941)(219.84128835,736.6248094)(219.77129189,736.62480927)
\curveto(219.71128848,736.61480941)(219.65628854,736.61980941)(219.60629189,736.63980927)
\curveto(219.56628863,736.64980938)(219.53128866,736.65480937)(219.50129189,736.65480927)
\curveto(219.47128872,736.65480937)(219.43628876,736.65980937)(219.39629189,736.66980927)
\lineto(219.12629189,736.72980927)
\curveto(219.03628916,736.74980928)(218.95128924,736.77980925)(218.87129189,736.81980927)
\curveto(218.53128966,736.95980907)(218.24128995,737.11480891)(218.00129189,737.28480927)
\curveto(217.76129043,737.46480856)(217.54129065,737.69480833)(217.34129189,737.97480927)
\curveto(217.191291,738.20480782)(217.07629112,738.44480758)(216.99629189,738.69480927)
\curveto(216.97629122,738.74480728)(216.96629123,738.78980724)(216.96629189,738.82980927)
\curveto(216.96629123,738.87980715)(216.95629124,738.9298071)(216.93629189,738.97980927)
\curveto(216.91629128,739.03980699)(216.90129129,739.11980691)(216.89129189,739.21980927)
\curveto(216.8912913,739.31980671)(216.91129128,739.39480663)(216.95129189,739.44480927)
\curveto(217.00129119,739.5248065)(217.08129111,739.56980646)(217.19129189,739.57980927)
\curveto(217.30129089,739.58980644)(217.41629078,739.59480643)(217.53629189,739.59480927)
\lineto(217.70129189,739.59480927)
\curveto(217.76129043,739.59480643)(217.81629038,739.58480644)(217.86629189,739.56480927)
\curveto(217.95629024,739.54480648)(218.02629017,739.50480652)(218.07629189,739.44480927)
\curveto(218.14629005,739.35480667)(218.19129,739.24480678)(218.21129189,739.11480927)
\curveto(218.24128995,738.99480703)(218.28628991,738.88980714)(218.34629189,738.79980927)
\curveto(218.53628966,738.45980757)(218.7962894,738.18980784)(219.12629189,737.98980927)
\curveto(219.22628897,737.9298081)(219.33128886,737.87980815)(219.44129189,737.83980927)
\curveto(219.56128863,737.80980822)(219.68128851,737.77480825)(219.80129189,737.73480927)
\curveto(219.97128822,737.68480834)(220.17628802,737.66480836)(220.41629189,737.67480927)
\curveto(220.66628753,737.69480833)(220.86628733,737.7298083)(221.01629189,737.77980927)
\curveto(221.38628681,737.89980813)(221.67628652,738.05980797)(221.88629189,738.25980927)
\curveto(222.10628609,738.46980756)(222.28628591,738.74980728)(222.42629189,739.09980927)
\curveto(222.47628572,739.19980683)(222.50628569,739.30480672)(222.51629189,739.41480927)
\curveto(222.53628566,739.5248065)(222.56128563,739.63980639)(222.59129189,739.75980927)
\lineto(222.59129189,739.86480927)
\curveto(222.60128559,739.90480612)(222.60628559,739.94480608)(222.60629189,739.98480927)
\curveto(222.61628558,740.01480601)(222.61628558,740.04980598)(222.60629189,740.08980927)
\lineto(222.60629189,740.20980927)
\curveto(222.60628559,740.46980556)(222.57628562,740.71480531)(222.51629189,740.94480927)
\curveto(222.40628579,741.29480473)(222.25128594,741.58980444)(222.05129189,741.82980927)
\curveto(221.85128634,742.07980395)(221.5912866,742.27480375)(221.27129189,742.41480927)
\lineto(221.09129189,742.47480927)
\curveto(221.04128715,742.49480353)(220.98128721,742.51480351)(220.91129189,742.53480927)
\curveto(220.86128733,742.55480347)(220.80128739,742.56480346)(220.73129189,742.56480927)
\curveto(220.67128752,742.57480345)(220.60628759,742.58980344)(220.53629189,742.60980927)
\lineto(220.38629189,742.60980927)
\curveto(220.34628785,742.6298034)(220.2912879,742.63980339)(220.22129189,742.63980927)
\curveto(220.16128803,742.63980339)(220.10628809,742.6298034)(220.05629189,742.60980927)
\lineto(219.95129189,742.60980927)
\curveto(219.92128827,742.60980342)(219.88628831,742.60480342)(219.84629189,742.59480927)
\lineto(219.60629189,742.53480927)
\curveto(219.52628867,742.5248035)(219.44628875,742.50480352)(219.36629189,742.47480927)
\curveto(219.12628907,742.37480365)(218.8962893,742.23980379)(218.67629189,742.06980927)
\curveto(218.58628961,741.99980403)(218.50128969,741.9248041)(218.42129189,741.84480927)
\curveto(218.34128985,741.77480425)(218.24128995,741.71980431)(218.12129189,741.67980927)
\curveto(218.03129016,741.64980438)(217.8912903,741.63980439)(217.70129189,741.64980927)
\curveto(217.52129067,741.65980437)(217.40129079,741.68480434)(217.34129189,741.72480927)
\curveto(217.2912909,741.76480426)(217.25129094,741.8248042)(217.22129189,741.90480927)
\curveto(217.20129099,741.98480404)(217.20129099,742.06980396)(217.22129189,742.15980927)
\curveto(217.25129094,742.27980375)(217.27129092,742.39980363)(217.28129189,742.51980927)
\curveto(217.30129089,742.64980338)(217.32629087,742.77480325)(217.35629189,742.89480927)
\curveto(217.37629082,742.93480309)(217.38129081,742.96980306)(217.37129189,742.99980927)
\curveto(217.37129082,743.03980299)(217.38129081,743.08480294)(217.40129189,743.13480927)
\curveto(217.42129077,743.2248028)(217.43629076,743.31480271)(217.44629189,743.40480927)
\curveto(217.45629074,743.50480252)(217.47629072,743.59980243)(217.50629189,743.68980927)
\curveto(217.51629068,743.74980228)(217.52129067,743.80980222)(217.52129189,743.86980927)
\curveto(217.53129066,743.9298021)(217.54629065,743.98980204)(217.56629189,744.04980927)
\curveto(217.61629058,744.24980178)(217.65129054,744.45480157)(217.67129189,744.66480927)
\curveto(217.70129049,744.88480114)(217.74129045,745.09480093)(217.79129189,745.29480927)
\curveto(217.82129037,745.39480063)(217.84129035,745.49480053)(217.85129189,745.59480927)
\curveto(217.86129033,745.69480033)(217.87629032,745.79480023)(217.89629189,745.89480927)
\curveto(217.90629029,745.9248001)(217.91129028,745.96480006)(217.91129189,746.01480927)
\curveto(217.94129025,746.1247999)(217.96129023,746.2297998)(217.97129189,746.32980927)
\curveto(217.9912902,746.43979959)(218.01629018,746.54979948)(218.04629189,746.65980927)
\curveto(218.06629013,746.73979929)(218.08129011,746.80979922)(218.09129189,746.86980927)
\curveto(218.10129009,746.93979909)(218.12629007,746.99979903)(218.16629189,747.04980927)
\curveto(218.18629001,747.07979895)(218.21628998,747.09979893)(218.25629189,747.10980927)
\curveto(218.2962899,747.1297989)(218.34128985,747.14979888)(218.39129189,747.16980927)
\curveto(218.45128974,747.16979886)(218.4912897,747.17479885)(218.51129189,747.18480927)
}
}
{
\newrgbcolor{curcolor}{0 0 0}
\pscustom[linestyle=none,fillstyle=solid,fillcolor=curcolor]
{
\newpath
\moveto(226.30590126,738.40980927)
\lineto(226.60590126,738.40980927)
\curveto(226.7158992,738.41980761)(226.8208991,738.41980761)(226.92090126,738.40980927)
\curveto(227.03089889,738.40980762)(227.13089879,738.39980763)(227.22090126,738.37980927)
\curveto(227.31089861,738.36980766)(227.38089854,738.34480768)(227.43090126,738.30480927)
\curveto(227.45089847,738.28480774)(227.46589845,738.25480777)(227.47590126,738.21480927)
\curveto(227.49589842,738.17480785)(227.5158984,738.1298079)(227.53590126,738.07980927)
\lineto(227.53590126,738.00480927)
\curveto(227.54589837,737.95480807)(227.54589837,737.89980813)(227.53590126,737.83980927)
\lineto(227.53590126,737.68980927)
\lineto(227.53590126,737.20980927)
\curveto(227.53589838,737.03980899)(227.49589842,736.91980911)(227.41590126,736.84980927)
\curveto(227.34589857,736.79980923)(227.25589866,736.77480925)(227.14590126,736.77480927)
\lineto(226.81590126,736.77480927)
\lineto(226.36590126,736.77480927)
\curveto(226.2158997,736.77480925)(226.10089982,736.80480922)(226.02090126,736.86480927)
\curveto(225.98089994,736.89480913)(225.95089997,736.94480908)(225.93090126,737.01480927)
\curveto(225.91090001,737.09480893)(225.89590002,737.17980885)(225.88590126,737.26980927)
\lineto(225.88590126,737.55480927)
\curveto(225.89590002,737.65480837)(225.90090002,737.73980829)(225.90090126,737.80980927)
\lineto(225.90090126,738.00480927)
\curveto(225.90090002,738.06480796)(225.91090001,738.11980791)(225.93090126,738.16980927)
\curveto(225.97089995,738.27980775)(226.04089988,738.34980768)(226.14090126,738.37980927)
\curveto(226.17089975,738.37980765)(226.22589969,738.38980764)(226.30590126,738.40980927)
}
}
{
\newrgbcolor{curcolor}{0 0 0}
\pscustom[linestyle=none,fillstyle=solid,fillcolor=curcolor]
{
\newpath
\moveto(236.39105751,740.26980927)
\curveto(236.46104987,740.21980581)(236.50104983,740.14980588)(236.51105751,740.05980927)
\curveto(236.5310498,739.96980606)(236.54104979,739.86480616)(236.54105751,739.74480927)
\curveto(236.54104979,739.69480633)(236.53604979,739.64480638)(236.52605751,739.59480927)
\curveto(236.5260498,739.54480648)(236.51604981,739.49980653)(236.49605751,739.45980927)
\curveto(236.46604986,739.36980666)(236.40604992,739.30980672)(236.31605751,739.27980927)
\curveto(236.23605009,739.25980677)(236.14105019,739.24980678)(236.03105751,739.24980927)
\lineto(235.71605751,739.24980927)
\curveto(235.60605072,739.25980677)(235.50105083,739.24980678)(235.40105751,739.21980927)
\curveto(235.26105107,739.18980684)(235.17105116,739.10980692)(235.13105751,738.97980927)
\curveto(235.11105122,738.90980712)(235.10105123,738.8248072)(235.10105751,738.72480927)
\lineto(235.10105751,738.45480927)
\lineto(235.10105751,737.50980927)
\lineto(235.10105751,737.17980927)
\curveto(235.10105123,737.06980896)(235.08105125,736.98480904)(235.04105751,736.92480927)
\curveto(235.00105133,736.86480916)(234.95105138,736.8248092)(234.89105751,736.80480927)
\curveto(234.84105149,736.79480923)(234.77605155,736.77980925)(234.69605751,736.75980927)
\lineto(234.50105751,736.75980927)
\curveto(234.38105195,736.75980927)(234.27605205,736.76480926)(234.18605751,736.77480927)
\curveto(234.09605223,736.79480923)(234.0260523,736.84480918)(233.97605751,736.92480927)
\curveto(233.94605238,736.97480905)(233.9310524,737.04480898)(233.93105751,737.13480927)
\lineto(233.93105751,737.43480927)
\lineto(233.93105751,738.46980927)
\curveto(233.9310524,738.6298074)(233.92105241,738.77480725)(233.90105751,738.90480927)
\curveto(233.89105244,739.04480698)(233.83605249,739.13980689)(233.73605751,739.18980927)
\curveto(233.68605264,739.20980682)(233.61605271,739.2248068)(233.52605751,739.23480927)
\curveto(233.44605288,739.24480678)(233.35605297,739.24980678)(233.25605751,739.24980927)
\lineto(232.97105751,739.24980927)
\lineto(232.73105751,739.24980927)
\lineto(230.46605751,739.24980927)
\curveto(230.37605595,739.24980678)(230.27105606,739.24480678)(230.15105751,739.23480927)
\lineto(229.82105751,739.23480927)
\curveto(229.71105662,739.23480679)(229.61105672,739.24480678)(229.52105751,739.26480927)
\curveto(229.4310569,739.28480674)(229.37105696,739.31980671)(229.34105751,739.36980927)
\curveto(229.29105704,739.43980659)(229.26605706,739.53480649)(229.26605751,739.65480927)
\lineto(229.26605751,739.99980927)
\lineto(229.26605751,740.26980927)
\curveto(229.30605702,740.43980559)(229.36105697,740.57980545)(229.43105751,740.68980927)
\curveto(229.50105683,740.79980523)(229.58105675,740.91480511)(229.67105751,741.03480927)
\lineto(230.03105751,741.57480927)
\curveto(230.47105586,742.20480382)(230.90605542,742.8248032)(231.33605751,743.43480927)
\lineto(232.65605751,745.29480927)
\curveto(232.81605351,745.5248005)(232.97105336,745.74480028)(233.12105751,745.95480927)
\curveto(233.27105306,746.17479985)(233.4260529,746.39979963)(233.58605751,746.62980927)
\curveto(233.63605269,746.69979933)(233.68605264,746.76479926)(233.73605751,746.82480927)
\curveto(233.78605254,746.89479913)(233.83605249,746.96979906)(233.88605751,747.04980927)
\lineto(233.94605751,747.13980927)
\curveto(233.97605235,747.17979885)(234.00605232,747.20979882)(234.03605751,747.22980927)
\curveto(234.07605225,747.25979877)(234.11605221,747.27979875)(234.15605751,747.28980927)
\curveto(234.19605213,747.30979872)(234.24105209,747.3297987)(234.29105751,747.34980927)
\curveto(234.31105202,747.34979868)(234.331052,747.34479868)(234.35105751,747.33480927)
\curveto(234.38105195,747.33479869)(234.40605192,747.34479868)(234.42605751,747.36480927)
\curveto(234.55605177,747.36479866)(234.67605165,747.35979867)(234.78605751,747.34980927)
\curveto(234.89605143,747.33979869)(234.97605135,747.29479873)(235.02605751,747.21480927)
\curveto(235.06605126,747.16479886)(235.08605124,747.09479893)(235.08605751,747.00480927)
\curveto(235.09605123,746.91479911)(235.10105123,746.81979921)(235.10105751,746.71980927)
\lineto(235.10105751,741.25980927)
\curveto(235.10105123,741.18980484)(235.09605123,741.11480491)(235.08605751,741.03480927)
\curveto(235.08605124,740.96480506)(235.09105124,740.89480513)(235.10105751,740.82480927)
\lineto(235.10105751,740.71980927)
\curveto(235.12105121,740.66980536)(235.13605119,740.61480541)(235.14605751,740.55480927)
\curveto(235.15605117,740.50480552)(235.18105115,740.46480556)(235.22105751,740.43480927)
\curveto(235.29105104,740.38480564)(235.37605095,740.35480567)(235.47605751,740.34480927)
\lineto(235.80605751,740.34480927)
\curveto(235.91605041,740.34480568)(236.02105031,740.33980569)(236.12105751,740.32980927)
\curveto(236.2310501,740.3298057)(236.32105001,740.30980572)(236.39105751,740.26980927)
\moveto(233.82605751,740.46480927)
\curveto(233.90605242,740.57480545)(233.94105239,740.74480528)(233.93105751,740.97480927)
\lineto(233.93105751,741.58980927)
\lineto(233.93105751,744.06480927)
\lineto(233.93105751,744.37980927)
\curveto(233.94105239,744.49980153)(233.93605239,744.59980143)(233.91605751,744.67980927)
\lineto(233.91605751,744.82980927)
\curveto(233.91605241,744.91980111)(233.90105243,745.00480102)(233.87105751,745.08480927)
\curveto(233.86105247,745.10480092)(233.85105248,745.11480091)(233.84105751,745.11480927)
\lineto(233.79605751,745.15980927)
\curveto(233.77605255,745.16980086)(233.74605258,745.17480085)(233.70605751,745.17480927)
\curveto(233.68605264,745.15480087)(233.66605266,745.13980089)(233.64605751,745.12980927)
\curveto(233.63605269,745.1298009)(233.62105271,745.1248009)(233.60105751,745.11480927)
\curveto(233.54105279,745.06480096)(233.48105285,744.99480103)(233.42105751,744.90480927)
\curveto(233.36105297,744.81480121)(233.30605302,744.73480129)(233.25605751,744.66480927)
\curveto(233.15605317,744.5248015)(233.06105327,744.37980165)(232.97105751,744.22980927)
\curveto(232.88105345,744.08980194)(232.78605354,743.94980208)(232.68605751,743.80980927)
\lineto(232.14605751,743.02980927)
\curveto(231.97605435,742.76980326)(231.80105453,742.50980352)(231.62105751,742.24980927)
\curveto(231.54105479,742.13980389)(231.46605486,742.03480399)(231.39605751,741.93480927)
\lineto(231.18605751,741.63480927)
\curveto(231.13605519,741.55480447)(231.08605524,741.47980455)(231.03605751,741.40980927)
\curveto(230.99605533,741.33980469)(230.95105538,741.26480476)(230.90105751,741.18480927)
\curveto(230.85105548,741.1248049)(230.80105553,741.05980497)(230.75105751,740.98980927)
\curveto(230.71105562,740.9298051)(230.67105566,740.85980517)(230.63105751,740.77980927)
\curveto(230.59105574,740.71980531)(230.56605576,740.64980538)(230.55605751,740.56980927)
\curveto(230.54605578,740.49980553)(230.58105575,740.44480558)(230.66105751,740.40480927)
\curveto(230.7310556,740.35480567)(230.84105549,740.3298057)(230.99105751,740.32980927)
\curveto(231.15105518,740.33980569)(231.28605504,740.34480568)(231.39605751,740.34480927)
\lineto(233.07605751,740.34480927)
\lineto(233.51105751,740.34480927)
\curveto(233.66105267,740.34480568)(233.76605256,740.38480564)(233.82605751,740.46480927)
}
}
{
\newrgbcolor{curcolor}{0 0 0}
\pscustom[linestyle=none,fillstyle=solid,fillcolor=curcolor]
{
\newpath
\moveto(247.66566689,745.29480927)
\curveto(247.46565659,745.00480102)(247.2556568,744.71980131)(247.03566689,744.43980927)
\curveto(246.82565723,744.15980187)(246.62065743,743.87480215)(246.42066689,743.58480927)
\curveto(245.82065823,742.73480329)(245.21565884,741.89480413)(244.60566689,741.06480927)
\curveto(243.99566006,740.24480578)(243.39066066,739.40980662)(242.79066689,738.55980927)
\lineto(242.28066689,737.83980927)
\lineto(241.77066689,737.14980927)
\curveto(241.69066236,737.03980899)(241.61066244,736.9248091)(241.53066689,736.80480927)
\curveto(241.4506626,736.68480934)(241.3556627,736.58980944)(241.24566689,736.51980927)
\curveto(241.20566285,736.49980953)(241.14066291,736.48480954)(241.05066689,736.47480927)
\curveto(240.97066308,736.45480957)(240.88066317,736.44480958)(240.78066689,736.44480927)
\curveto(240.68066337,736.44480958)(240.58566347,736.44980958)(240.49566689,736.45980927)
\curveto(240.41566364,736.46980956)(240.3556637,736.48980954)(240.31566689,736.51980927)
\curveto(240.28566377,736.53980949)(240.26066379,736.57480945)(240.24066689,736.62480927)
\curveto(240.23066382,736.66480936)(240.23566382,736.70980932)(240.25566689,736.75980927)
\curveto(240.29566376,736.83980919)(240.34066371,736.91480911)(240.39066689,736.98480927)
\curveto(240.4506636,737.06480896)(240.50566355,737.14480888)(240.55566689,737.22480927)
\curveto(240.79566326,737.56480846)(241.04066301,737.89980813)(241.29066689,738.22980927)
\curveto(241.54066251,738.55980747)(241.78066227,738.89480713)(242.01066689,739.23480927)
\curveto(242.17066188,739.45480657)(242.33066172,739.66980636)(242.49066689,739.87980927)
\curveto(242.6506614,740.08980594)(242.81066124,740.30480572)(242.97066689,740.52480927)
\curveto(243.33066072,741.04480498)(243.69566036,741.55480447)(244.06566689,742.05480927)
\curveto(244.43565962,742.55480347)(244.80565925,743.06480296)(245.17566689,743.58480927)
\curveto(245.31565874,743.78480224)(245.4556586,743.97980205)(245.59566689,744.16980927)
\curveto(245.74565831,744.35980167)(245.89065816,744.55480147)(246.03066689,744.75480927)
\curveto(246.24065781,745.05480097)(246.4556576,745.35480067)(246.67566689,745.65480927)
\lineto(247.33566689,746.55480927)
\lineto(247.51566689,746.82480927)
\lineto(247.72566689,747.09480927)
\lineto(247.84566689,747.27480927)
\curveto(247.89565616,747.33479869)(247.94565611,747.38979864)(247.99566689,747.43980927)
\curveto(248.06565599,747.48979854)(248.14065591,747.5247985)(248.22066689,747.54480927)
\curveto(248.24065581,747.55479847)(248.26565579,747.55479847)(248.29566689,747.54480927)
\curveto(248.33565572,747.54479848)(248.36565569,747.55479847)(248.38566689,747.57480927)
\curveto(248.50565555,747.57479845)(248.64065541,747.56979846)(248.79066689,747.55980927)
\curveto(248.94065511,747.55979847)(249.03065502,747.51479851)(249.06066689,747.42480927)
\curveto(249.08065497,747.39479863)(249.08565497,747.35979867)(249.07566689,747.31980927)
\curveto(249.06565499,747.27979875)(249.050655,747.24979878)(249.03066689,747.22980927)
\curveto(248.99065506,747.14979888)(248.9506551,747.07979895)(248.91066689,747.01980927)
\curveto(248.87065518,746.95979907)(248.82565523,746.89979913)(248.77566689,746.83980927)
\lineto(248.20566689,746.05980927)
\curveto(248.02565603,745.80980022)(247.84565621,745.55480047)(247.66566689,745.29480927)
\moveto(240.81066689,741.39480927)
\curveto(240.76066329,741.41480461)(240.71066334,741.41980461)(240.66066689,741.40980927)
\curveto(240.61066344,741.39980463)(240.56066349,741.40480462)(240.51066689,741.42480927)
\curveto(240.40066365,741.44480458)(240.29566376,741.46480456)(240.19566689,741.48480927)
\curveto(240.10566395,741.51480451)(240.01066404,741.55480447)(239.91066689,741.60480927)
\curveto(239.58066447,741.74480428)(239.32566473,741.93980409)(239.14566689,742.18980927)
\curveto(238.96566509,742.44980358)(238.82066523,742.75980327)(238.71066689,743.11980927)
\curveto(238.68066537,743.19980283)(238.66066539,743.27980275)(238.65066689,743.35980927)
\curveto(238.64066541,743.44980258)(238.62566543,743.53480249)(238.60566689,743.61480927)
\curveto(238.59566546,743.66480236)(238.59066546,743.7298023)(238.59066689,743.80980927)
\curveto(238.58066547,743.83980219)(238.57566548,743.86980216)(238.57566689,743.89980927)
\curveto(238.57566548,743.93980209)(238.57066548,743.97480205)(238.56066689,744.00480927)
\lineto(238.56066689,744.15480927)
\curveto(238.5506655,744.20480182)(238.54566551,744.26480176)(238.54566689,744.33480927)
\curveto(238.54566551,744.41480161)(238.5506655,744.47980155)(238.56066689,744.52980927)
\lineto(238.56066689,744.69480927)
\curveto(238.58066547,744.74480128)(238.58566547,744.78980124)(238.57566689,744.82980927)
\curveto(238.57566548,744.87980115)(238.58066547,744.9248011)(238.59066689,744.96480927)
\curveto(238.60066545,745.00480102)(238.60566545,745.03980099)(238.60566689,745.06980927)
\curveto(238.60566545,745.10980092)(238.61066544,745.14980088)(238.62066689,745.18980927)
\curveto(238.6506654,745.29980073)(238.67066538,745.40980062)(238.68066689,745.51980927)
\curveto(238.70066535,745.63980039)(238.73566532,745.75480027)(238.78566689,745.86480927)
\curveto(238.92566513,746.20479982)(239.08566497,746.47979955)(239.26566689,746.68980927)
\curveto(239.4556646,746.90979912)(239.72566433,747.08979894)(240.07566689,747.22980927)
\curveto(240.1556639,747.25979877)(240.24066381,747.27979875)(240.33066689,747.28980927)
\curveto(240.42066363,747.30979872)(240.51566354,747.3297987)(240.61566689,747.34980927)
\curveto(240.64566341,747.35979867)(240.70066335,747.35979867)(240.78066689,747.34980927)
\curveto(240.86066319,747.34979868)(240.91066314,747.35979867)(240.93066689,747.37980927)
\curveto(241.49066256,747.38979864)(241.94066211,747.27979875)(242.28066689,747.04980927)
\curveto(242.63066142,746.81979921)(242.89066116,746.51479951)(243.06066689,746.13480927)
\curveto(243.10066095,746.04479998)(243.13566092,745.94980008)(243.16566689,745.84980927)
\curveto(243.19566086,745.74980028)(243.22066083,745.64980038)(243.24066689,745.54980927)
\curveto(243.26066079,745.51980051)(243.26566079,745.48980054)(243.25566689,745.45980927)
\curveto(243.2556608,745.4298006)(243.26066079,745.39980063)(243.27066689,745.36980927)
\curveto(243.30066075,745.25980077)(243.32066073,745.13480089)(243.33066689,744.99480927)
\curveto(243.34066071,744.86480116)(243.3506607,744.7298013)(243.36066689,744.58980927)
\lineto(243.36066689,744.42480927)
\curveto(243.37066068,744.36480166)(243.37066068,744.30980172)(243.36066689,744.25980927)
\curveto(243.3506607,744.20980182)(243.34566071,744.15980187)(243.34566689,744.10980927)
\lineto(243.34566689,743.97480927)
\curveto(243.33566072,743.93480209)(243.33066072,743.89480213)(243.33066689,743.85480927)
\curveto(243.34066071,743.81480221)(243.33566072,743.76980226)(243.31566689,743.71980927)
\curveto(243.29566076,743.60980242)(243.27566078,743.50480252)(243.25566689,743.40480927)
\curveto(243.24566081,743.30480272)(243.22566083,743.20480282)(243.19566689,743.10480927)
\curveto(243.06566099,742.74480328)(242.90066115,742.4298036)(242.70066689,742.15980927)
\curveto(242.50066155,741.88980414)(242.22566183,741.68480434)(241.87566689,741.54480927)
\curveto(241.79566226,741.51480451)(241.71066234,741.48980454)(241.62066689,741.46980927)
\lineto(241.35066689,741.40980927)
\curveto(241.30066275,741.39980463)(241.2556628,741.39480463)(241.21566689,741.39480927)
\curveto(241.17566288,741.40480462)(241.13566292,741.40480462)(241.09566689,741.39480927)
\curveto(240.99566306,741.37480465)(240.90066315,741.37480465)(240.81066689,741.39480927)
\moveto(239.97066689,742.78980927)
\curveto(240.01066404,742.71980331)(240.050664,742.65480337)(240.09066689,742.59480927)
\curveto(240.13066392,742.54480348)(240.18066387,742.49480353)(240.24066689,742.44480927)
\lineto(240.39066689,742.32480927)
\curveto(240.4506636,742.29480373)(240.51566354,742.26980376)(240.58566689,742.24980927)
\curveto(240.62566343,742.2298038)(240.66066339,742.21980381)(240.69066689,742.21980927)
\curveto(240.73066332,742.2298038)(240.77066328,742.2248038)(240.81066689,742.20480927)
\curveto(240.84066321,742.20480382)(240.88066317,742.19980383)(240.93066689,742.18980927)
\curveto(240.98066307,742.18980384)(241.02066303,742.19480383)(241.05066689,742.20480927)
\lineto(241.27566689,742.24980927)
\curveto(241.52566253,742.3298037)(241.71066234,742.45480357)(241.83066689,742.62480927)
\curveto(241.91066214,742.7248033)(241.98066207,742.85480317)(242.04066689,743.01480927)
\curveto(242.12066193,743.19480283)(242.18066187,743.41980261)(242.22066689,743.68980927)
\curveto(242.26066179,743.96980206)(242.27566178,744.24980178)(242.26566689,744.52980927)
\curveto(242.2556618,744.81980121)(242.22566183,745.09480093)(242.17566689,745.35480927)
\curveto(242.12566193,745.61480041)(242.050662,745.8248002)(241.95066689,745.98480927)
\curveto(241.83066222,746.18479984)(241.68066237,746.33479969)(241.50066689,746.43480927)
\curveto(241.42066263,746.48479954)(241.33066272,746.51479951)(241.23066689,746.52480927)
\curveto(241.13066292,746.54479948)(241.02566303,746.55479947)(240.91566689,746.55480927)
\curveto(240.89566316,746.54479948)(240.87066318,746.53979949)(240.84066689,746.53980927)
\curveto(240.82066323,746.54979948)(240.80066325,746.54979948)(240.78066689,746.53980927)
\curveto(240.73066332,746.5297995)(240.68566337,746.51979951)(240.64566689,746.50980927)
\curveto(240.60566345,746.50979952)(240.56566349,746.49979953)(240.52566689,746.47980927)
\curveto(240.34566371,746.39979963)(240.19566386,746.27979975)(240.07566689,746.11980927)
\curveto(239.96566409,745.95980007)(239.87566418,745.77980025)(239.80566689,745.57980927)
\curveto(239.74566431,745.38980064)(239.70066435,745.16480086)(239.67066689,744.90480927)
\curveto(239.6506644,744.64480138)(239.64566441,744.37980165)(239.65566689,744.10980927)
\curveto(239.66566439,743.84980218)(239.69566436,743.59980243)(239.74566689,743.35980927)
\curveto(239.80566425,743.1298029)(239.88066417,742.93980309)(239.97066689,742.78980927)
\moveto(250.77066689,739.80480927)
\curveto(250.78065327,739.75480627)(250.78565327,739.66480636)(250.78566689,739.53480927)
\curveto(250.78565327,739.40480662)(250.77565328,739.31480671)(250.75566689,739.26480927)
\curveto(250.73565332,739.21480681)(250.73065332,739.15980687)(250.74066689,739.09980927)
\curveto(250.7506533,739.04980698)(250.7506533,738.99980703)(250.74066689,738.94980927)
\curveto(250.70065335,738.80980722)(250.67065338,738.67480735)(250.65066689,738.54480927)
\curveto(250.64065341,738.41480761)(250.61065344,738.29480773)(250.56066689,738.18480927)
\curveto(250.42065363,737.83480819)(250.2556538,737.53980849)(250.06566689,737.29980927)
\curveto(249.87565418,737.06980896)(249.60565445,736.88480914)(249.25566689,736.74480927)
\curveto(249.17565488,736.71480931)(249.09065496,736.69480933)(249.00066689,736.68480927)
\curveto(248.91065514,736.66480936)(248.82565523,736.64480938)(248.74566689,736.62480927)
\curveto(248.69565536,736.61480941)(248.64565541,736.60980942)(248.59566689,736.60980927)
\curveto(248.54565551,736.60980942)(248.49565556,736.60480942)(248.44566689,736.59480927)
\curveto(248.41565564,736.58480944)(248.36565569,736.58480944)(248.29566689,736.59480927)
\curveto(248.22565583,736.59480943)(248.17565588,736.59980943)(248.14566689,736.60980927)
\curveto(248.08565597,736.6298094)(248.02565603,736.63980939)(247.96566689,736.63980927)
\curveto(247.91565614,736.6298094)(247.86565619,736.63480939)(247.81566689,736.65480927)
\curveto(247.72565633,736.67480935)(247.63565642,736.69980933)(247.54566689,736.72980927)
\curveto(247.46565659,736.74980928)(247.38565667,736.77980925)(247.30566689,736.81980927)
\curveto(246.98565707,736.95980907)(246.73565732,737.15480887)(246.55566689,737.40480927)
\curveto(246.37565768,737.66480836)(246.22565783,737.96980806)(246.10566689,738.31980927)
\curveto(246.08565797,738.39980763)(246.07065798,738.48480754)(246.06066689,738.57480927)
\curveto(246.050658,738.66480736)(246.03565802,738.74980728)(246.01566689,738.82980927)
\curveto(246.00565805,738.85980717)(246.00065805,738.88980714)(246.00066689,738.91980927)
\lineto(246.00066689,739.02480927)
\curveto(245.98065807,739.10480692)(245.97065808,739.18480684)(245.97066689,739.26480927)
\lineto(245.97066689,739.39980927)
\curveto(245.9506581,739.49980653)(245.9506581,739.59980643)(245.97066689,739.69980927)
\lineto(245.97066689,739.87980927)
\curveto(245.98065807,739.9298061)(245.98565807,739.97480605)(245.98566689,740.01480927)
\curveto(245.98565807,740.06480596)(245.99065806,740.10980592)(246.00066689,740.14980927)
\curveto(246.01065804,740.18980584)(246.01565804,740.2248058)(246.01566689,740.25480927)
\curveto(246.01565804,740.29480573)(246.02065803,740.33480569)(246.03066689,740.37480927)
\lineto(246.09066689,740.70480927)
\curveto(246.11065794,740.8248052)(246.14065791,740.93480509)(246.18066689,741.03480927)
\curveto(246.32065773,741.36480466)(246.48065757,741.63980439)(246.66066689,741.85980927)
\curveto(246.8506572,742.08980394)(247.11065694,742.27480375)(247.44066689,742.41480927)
\curveto(247.52065653,742.45480357)(247.60565645,742.47980355)(247.69566689,742.48980927)
\lineto(247.99566689,742.54980927)
\lineto(248.13066689,742.54980927)
\curveto(248.18065587,742.55980347)(248.23065582,742.56480346)(248.28066689,742.56480927)
\curveto(248.8506552,742.58480344)(249.31065474,742.47980355)(249.66066689,742.24980927)
\curveto(250.02065403,742.029804)(250.28565377,741.7298043)(250.45566689,741.34980927)
\curveto(250.50565355,741.24980478)(250.54565351,741.14980488)(250.57566689,741.04980927)
\curveto(250.60565345,740.94980508)(250.63565342,740.84480518)(250.66566689,740.73480927)
\curveto(250.67565338,740.69480533)(250.68065337,740.65980537)(250.68066689,740.62980927)
\curveto(250.68065337,740.60980542)(250.68565337,740.57980545)(250.69566689,740.53980927)
\curveto(250.71565334,740.46980556)(250.72565333,740.39480563)(250.72566689,740.31480927)
\curveto(250.72565333,740.23480579)(250.73565332,740.15480587)(250.75566689,740.07480927)
\curveto(250.7556533,740.024806)(250.7556533,739.97980605)(250.75566689,739.93980927)
\curveto(250.7556533,739.89980613)(250.76065329,739.85480617)(250.77066689,739.80480927)
\moveto(249.66066689,739.36980927)
\curveto(249.67065438,739.41980661)(249.67565438,739.49480653)(249.67566689,739.59480927)
\curveto(249.68565437,739.69480633)(249.68065437,739.76980626)(249.66066689,739.81980927)
\curveto(249.64065441,739.87980615)(249.63565442,739.93480609)(249.64566689,739.98480927)
\curveto(249.66565439,740.04480598)(249.66565439,740.10480592)(249.64566689,740.16480927)
\curveto(249.63565442,740.19480583)(249.63065442,740.2298058)(249.63066689,740.26980927)
\curveto(249.63065442,740.30980572)(249.62565443,740.34980568)(249.61566689,740.38980927)
\curveto(249.59565446,740.46980556)(249.57565448,740.54480548)(249.55566689,740.61480927)
\curveto(249.54565451,740.69480533)(249.53065452,740.77480525)(249.51066689,740.85480927)
\curveto(249.48065457,740.91480511)(249.4556546,740.97480505)(249.43566689,741.03480927)
\curveto(249.41565464,741.09480493)(249.38565467,741.15480487)(249.34566689,741.21480927)
\curveto(249.24565481,741.38480464)(249.11565494,741.51980451)(248.95566689,741.61980927)
\curveto(248.87565518,741.66980436)(248.78065527,741.70480432)(248.67066689,741.72480927)
\curveto(248.56065549,741.74480428)(248.43565562,741.75480427)(248.29566689,741.75480927)
\curveto(248.27565578,741.74480428)(248.2506558,741.73980429)(248.22066689,741.73980927)
\curveto(248.19065586,741.74980428)(248.16065589,741.74980428)(248.13066689,741.73980927)
\lineto(247.98066689,741.67980927)
\curveto(247.93065612,741.66980436)(247.88565617,741.65480437)(247.84566689,741.63480927)
\curveto(247.6556564,741.5248045)(247.51065654,741.37980465)(247.41066689,741.19980927)
\curveto(247.32065673,741.01980501)(247.24065681,740.81480521)(247.17066689,740.58480927)
\curveto(247.13065692,740.45480557)(247.11065694,740.31980571)(247.11066689,740.17980927)
\curveto(247.11065694,740.04980598)(247.10065695,739.90480612)(247.08066689,739.74480927)
\curveto(247.07065698,739.69480633)(247.06065699,739.63480639)(247.05066689,739.56480927)
\curveto(247.050657,739.49480653)(247.06065699,739.43480659)(247.08066689,739.38480927)
\lineto(247.08066689,739.21980927)
\lineto(247.08066689,739.03980927)
\curveto(247.09065696,738.98980704)(247.10065695,738.93480709)(247.11066689,738.87480927)
\curveto(247.12065693,738.8248072)(247.12565693,738.76980726)(247.12566689,738.70980927)
\curveto(247.13565692,738.64980738)(247.1506569,738.59480743)(247.17066689,738.54480927)
\curveto(247.22065683,738.35480767)(247.28065677,738.17980785)(247.35066689,738.01980927)
\curveto(247.42065663,737.85980817)(247.52565653,737.7298083)(247.66566689,737.62980927)
\curveto(247.79565626,737.5298085)(247.93565612,737.45980857)(248.08566689,737.41980927)
\curveto(248.11565594,737.40980862)(248.14065591,737.40480862)(248.16066689,737.40480927)
\curveto(248.19065586,737.41480861)(248.22065583,737.41480861)(248.25066689,737.40480927)
\curveto(248.27065578,737.40480862)(248.30065575,737.39980863)(248.34066689,737.38980927)
\curveto(248.38065567,737.38980864)(248.41565564,737.39480863)(248.44566689,737.40480927)
\curveto(248.48565557,737.41480861)(248.52565553,737.41980861)(248.56566689,737.41980927)
\curveto(248.60565545,737.41980861)(248.64565541,737.4298086)(248.68566689,737.44980927)
\curveto(248.92565513,737.5298085)(249.12065493,737.66480836)(249.27066689,737.85480927)
\curveto(249.39065466,738.03480799)(249.48065457,738.23980779)(249.54066689,738.46980927)
\curveto(249.56065449,738.53980749)(249.57565448,738.60980742)(249.58566689,738.67980927)
\curveto(249.59565446,738.75980727)(249.61065444,738.83980719)(249.63066689,738.91980927)
\curveto(249.63065442,738.97980705)(249.63565442,739.024807)(249.64566689,739.05480927)
\curveto(249.64565441,739.07480695)(249.64565441,739.09980693)(249.64566689,739.12980927)
\curveto(249.64565441,739.16980686)(249.6506544,739.19980683)(249.66066689,739.21980927)
\lineto(249.66066689,739.36980927)
}
}
{
\newrgbcolor{curcolor}{0 0 0}
\pscustom[linestyle=none,fillstyle=solid,fillcolor=curcolor]
{
\newpath
\moveto(604.11166725,994.32763916)
\lineto(608.91166725,994.32763916)
\lineto(609.91666725,994.32763916)
\curveto(610.05666015,994.32762874)(610.17666003,994.31762875)(610.27666725,994.29763916)
\curveto(610.38665982,994.28762878)(610.46665974,994.24262882)(610.51666725,994.16263916)
\curveto(610.53665967,994.12262894)(610.54665966,994.07262899)(610.54666725,994.01263916)
\curveto(610.55665965,993.95262911)(610.56165965,993.88762918)(610.56166725,993.81763916)
\lineto(610.56166725,993.54763916)
\curveto(610.56165965,993.45762961)(610.55165966,993.37762969)(610.53166725,993.30763916)
\curveto(610.49165972,993.22762984)(610.44665976,993.15762991)(610.39666725,993.09763916)
\lineto(610.24666725,992.91763916)
\curveto(610.21665999,992.8676302)(610.18166003,992.82763024)(610.14166725,992.79763916)
\curveto(610.10166011,992.7676303)(610.06166015,992.72763034)(610.02166725,992.67763916)
\curveto(609.94166027,992.5676305)(609.85666035,992.45763061)(609.76666725,992.34763916)
\curveto(609.67666053,992.24763082)(609.59166062,992.14263092)(609.51166725,992.03263916)
\curveto(609.37166084,991.83263123)(609.23166098,991.62263144)(609.09166725,991.40263916)
\curveto(608.95166126,991.19263187)(608.8116614,990.97763209)(608.67166725,990.75763916)
\curveto(608.62166159,990.6676324)(608.57166164,990.57263249)(608.52166725,990.47263916)
\curveto(608.47166174,990.37263269)(608.41666179,990.27763279)(608.35666725,990.18763916)
\curveto(608.33666187,990.1676329)(608.32666188,990.14263292)(608.32666725,990.11263916)
\curveto(608.32666188,990.08263298)(608.31666189,990.05763301)(608.29666725,990.03763916)
\curveto(608.22666198,989.93763313)(608.16166205,989.82263324)(608.10166725,989.69263916)
\curveto(608.04166217,989.57263349)(607.98666222,989.45763361)(607.93666725,989.34763916)
\curveto(607.83666237,989.11763395)(607.74166247,988.88263418)(607.65166725,988.64263916)
\curveto(607.56166265,988.40263466)(607.46166275,988.1626349)(607.35166725,987.92263916)
\curveto(607.33166288,987.87263519)(607.31666289,987.82763524)(607.30666725,987.78763916)
\curveto(607.3066629,987.74763532)(607.29666291,987.70263536)(607.27666725,987.65263916)
\curveto(607.22666298,987.53263553)(607.18166303,987.40763566)(607.14166725,987.27763916)
\curveto(607.1116631,987.15763591)(607.07666313,987.03763603)(607.03666725,986.91763916)
\curveto(606.95666325,986.68763638)(606.89166332,986.44763662)(606.84166725,986.19763916)
\curveto(606.80166341,985.95763711)(606.75166346,985.71763735)(606.69166725,985.47763916)
\curveto(606.65166356,985.32763774)(606.62666358,985.17763789)(606.61666725,985.02763916)
\curveto(606.6066636,984.87763819)(606.58666362,984.72763834)(606.55666725,984.57763916)
\curveto(606.54666366,984.53763853)(606.54166367,984.47763859)(606.54166725,984.39763916)
\curveto(606.5116637,984.27763879)(606.48166373,984.17763889)(606.45166725,984.09763916)
\curveto(606.42166379,984.01763905)(606.35166386,983.9626391)(606.24166725,983.93263916)
\curveto(606.19166402,983.91263915)(606.13666407,983.90263916)(606.07666725,983.90263916)
\lineto(605.88166725,983.90263916)
\curveto(605.74166447,983.90263916)(605.60166461,983.90763916)(605.46166725,983.91763916)
\curveto(605.33166488,983.92763914)(605.23666497,983.97263909)(605.17666725,984.05263916)
\curveto(605.13666507,984.11263895)(605.11666509,984.19763887)(605.11666725,984.30763916)
\curveto(605.12666508,984.41763865)(605.14166507,984.51263855)(605.16166725,984.59263916)
\lineto(605.16166725,984.66763916)
\curveto(605.17166504,984.69763837)(605.17666503,984.72763834)(605.17666725,984.75763916)
\curveto(605.19666501,984.83763823)(605.206665,984.91263815)(605.20666725,984.98263916)
\curveto(605.206665,985.05263801)(605.21666499,985.12263794)(605.23666725,985.19263916)
\curveto(605.28666492,985.38263768)(605.32666488,985.5676375)(605.35666725,985.74763916)
\curveto(605.38666482,985.93763713)(605.42666478,986.11763695)(605.47666725,986.28763916)
\curveto(605.49666471,986.33763673)(605.5066647,986.37763669)(605.50666725,986.40763916)
\curveto(605.5066647,986.43763663)(605.5116647,986.47263659)(605.52166725,986.51263916)
\curveto(605.62166459,986.81263625)(605.7116645,987.10763596)(605.79166725,987.39763916)
\curveto(605.88166433,987.68763538)(605.98666422,987.9676351)(606.10666725,988.23763916)
\curveto(606.36666384,988.81763425)(606.63666357,989.3676337)(606.91666725,989.88763916)
\curveto(607.19666301,990.41763265)(607.5066627,990.92263214)(607.84666725,991.40263916)
\curveto(607.98666222,991.60263146)(608.13666207,991.79263127)(608.29666725,991.97263916)
\curveto(608.45666175,992.1626309)(608.6066616,992.35263071)(608.74666725,992.54263916)
\curveto(608.78666142,992.59263047)(608.82166139,992.63763043)(608.85166725,992.67763916)
\curveto(608.89166132,992.72763034)(608.92666128,992.77763029)(608.95666725,992.82763916)
\curveto(608.96666124,992.84763022)(608.97666123,992.87263019)(608.98666725,992.90263916)
\curveto(609.0066612,992.93263013)(609.0066612,992.9626301)(608.98666725,992.99263916)
\curveto(608.96666124,993.05263001)(608.93166128,993.08762998)(608.88166725,993.09763916)
\curveto(608.83166138,993.11762995)(608.78166143,993.13762993)(608.73166725,993.15763916)
\lineto(608.62666725,993.15763916)
\curveto(608.58666162,993.1676299)(608.53666167,993.1676299)(608.47666725,993.15763916)
\lineto(608.32666725,993.15763916)
\lineto(607.72666725,993.15763916)
\lineto(605.08666725,993.15763916)
\lineto(604.35166725,993.15763916)
\lineto(604.11166725,993.15763916)
\curveto(604.04166617,993.1676299)(603.98166623,993.18262988)(603.93166725,993.20263916)
\curveto(603.84166637,993.24262982)(603.78166643,993.30262976)(603.75166725,993.38263916)
\curveto(603.70166651,993.48262958)(603.68666652,993.62762944)(603.70666725,993.81763916)
\curveto(603.72666648,994.01762905)(603.76166645,994.15262891)(603.81166725,994.22263916)
\curveto(603.83166638,994.24262882)(603.85666635,994.25762881)(603.88666725,994.26763916)
\lineto(604.00666725,994.32763916)
\curveto(604.02666618,994.32762874)(604.04166617,994.32262874)(604.05166725,994.31263916)
\curveto(604.07166614,994.31262875)(604.09166612,994.31762875)(604.11166725,994.32763916)
}
}
{
\newrgbcolor{curcolor}{0 0 0}
\pscustom[linestyle=none,fillstyle=solid,fillcolor=curcolor]
{
\newpath
\moveto(612.95627663,985.55263916)
\lineto(613.25627663,985.55263916)
\curveto(613.36627457,985.5626375)(613.47127446,985.5626375)(613.57127663,985.55263916)
\curveto(613.68127425,985.55263751)(613.78127415,985.54263752)(613.87127663,985.52263916)
\curveto(613.96127397,985.51263755)(614.0312739,985.48763758)(614.08127663,985.44763916)
\curveto(614.10127383,985.42763764)(614.11627382,985.39763767)(614.12627663,985.35763916)
\curveto(614.14627379,985.31763775)(614.16627377,985.27263779)(614.18627663,985.22263916)
\lineto(614.18627663,985.14763916)
\curveto(614.19627374,985.09763797)(614.19627374,985.04263802)(614.18627663,984.98263916)
\lineto(614.18627663,984.83263916)
\lineto(614.18627663,984.35263916)
\curveto(614.18627375,984.18263888)(614.14627379,984.062639)(614.06627663,983.99263916)
\curveto(613.99627394,983.94263912)(613.90627403,983.91763915)(613.79627663,983.91763916)
\lineto(613.46627663,983.91763916)
\lineto(613.01627663,983.91763916)
\curveto(612.86627507,983.91763915)(612.75127518,983.94763912)(612.67127663,984.00763916)
\curveto(612.6312753,984.03763903)(612.60127533,984.08763898)(612.58127663,984.15763916)
\curveto(612.56127537,984.23763883)(612.54627539,984.32263874)(612.53627663,984.41263916)
\lineto(612.53627663,984.69763916)
\curveto(612.54627539,984.79763827)(612.55127538,984.88263818)(612.55127663,984.95263916)
\lineto(612.55127663,985.14763916)
\curveto(612.55127538,985.20763786)(612.56127537,985.2626378)(612.58127663,985.31263916)
\curveto(612.62127531,985.42263764)(612.69127524,985.49263757)(612.79127663,985.52263916)
\curveto(612.82127511,985.52263754)(612.87627506,985.53263753)(612.95627663,985.55263916)
}
}
{
\newrgbcolor{curcolor}{0 0 0}
\pscustom[linestyle=none,fillstyle=solid,fillcolor=curcolor]
{
\newpath
\moveto(620.22143288,994.52263916)
\curveto(620.32142802,994.52262854)(620.41642793,994.51262855)(620.50643288,994.49263916)
\curveto(620.59642775,994.48262858)(620.66142768,994.45262861)(620.70143288,994.40263916)
\curveto(620.76142758,994.32262874)(620.79142755,994.21762885)(620.79143288,994.08763916)
\lineto(620.79143288,993.69763916)
\lineto(620.79143288,992.19763916)
\lineto(620.79143288,985.80763916)
\lineto(620.79143288,984.63763916)
\lineto(620.79143288,984.32263916)
\curveto(620.80142754,984.22263884)(620.78642756,984.14263892)(620.74643288,984.08263916)
\curveto(620.69642765,984.00263906)(620.62142772,983.95263911)(620.52143288,983.93263916)
\curveto(620.43142791,983.92263914)(620.32142802,983.91763915)(620.19143288,983.91763916)
\lineto(619.96643288,983.91763916)
\curveto(619.88642846,983.93763913)(619.81642853,983.95263911)(619.75643288,983.96263916)
\curveto(619.69642865,983.98263908)(619.6464287,984.02263904)(619.60643288,984.08263916)
\curveto(619.56642878,984.14263892)(619.5464288,984.21763885)(619.54643288,984.30763916)
\lineto(619.54643288,984.60763916)
\lineto(619.54643288,985.70263916)
\lineto(619.54643288,991.04263916)
\curveto(619.52642882,991.13263193)(619.51142883,991.20763186)(619.50143288,991.26763916)
\curveto(619.50142884,991.33763173)(619.47142887,991.39763167)(619.41143288,991.44763916)
\curveto(619.341429,991.49763157)(619.25142909,991.52263154)(619.14143288,991.52263916)
\curveto(619.0414293,991.53263153)(618.93142941,991.53763153)(618.81143288,991.53763916)
\lineto(617.67143288,991.53763916)
\lineto(617.17643288,991.53763916)
\curveto(617.01643133,991.54763152)(616.90643144,991.60763146)(616.84643288,991.71763916)
\curveto(616.82643152,991.74763132)(616.81643153,991.77763129)(616.81643288,991.80763916)
\curveto(616.81643153,991.84763122)(616.81143153,991.89263117)(616.80143288,991.94263916)
\curveto(616.78143156,992.062631)(616.78643156,992.17263089)(616.81643288,992.27263916)
\curveto(616.85643149,992.37263069)(616.91143143,992.44263062)(616.98143288,992.48263916)
\curveto(617.06143128,992.53263053)(617.18143116,992.55763051)(617.34143288,992.55763916)
\curveto(617.50143084,992.55763051)(617.63643071,992.57263049)(617.74643288,992.60263916)
\curveto(617.79643055,992.61263045)(617.85143049,992.61763045)(617.91143288,992.61763916)
\curveto(617.97143037,992.62763044)(618.03143031,992.64263042)(618.09143288,992.66263916)
\curveto(618.2414301,992.71263035)(618.38642996,992.7626303)(618.52643288,992.81263916)
\curveto(618.66642968,992.87263019)(618.80142954,992.94263012)(618.93143288,993.02263916)
\curveto(619.07142927,993.11262995)(619.19142915,993.21762985)(619.29143288,993.33763916)
\curveto(619.39142895,993.45762961)(619.48642886,993.58762948)(619.57643288,993.72763916)
\curveto(619.63642871,993.82762924)(619.68142866,993.93762913)(619.71143288,994.05763916)
\curveto(619.75142859,994.17762889)(619.80142854,994.28262878)(619.86143288,994.37263916)
\curveto(619.91142843,994.43262863)(619.98142836,994.47262859)(620.07143288,994.49263916)
\curveto(620.09142825,994.50262856)(620.11642823,994.50762856)(620.14643288,994.50763916)
\curveto(620.17642817,994.50762856)(620.20142814,994.51262855)(620.22143288,994.52263916)
}
}
{
\newrgbcolor{curcolor}{0 0 0}
\pscustom[linestyle=none,fillstyle=solid,fillcolor=curcolor]
{
\newpath
\moveto(634.31604225,992.43763916)
\curveto(634.11603195,992.14763092)(633.90603216,991.8626312)(633.68604225,991.58263916)
\curveto(633.47603259,991.30263176)(633.2710328,991.01763205)(633.07104225,990.72763916)
\curveto(632.4710336,989.87763319)(631.8660342,989.03763403)(631.25604225,988.20763916)
\curveto(630.64603542,987.38763568)(630.04103603,986.55263651)(629.44104225,985.70263916)
\lineto(628.93104225,984.98263916)
\lineto(628.42104225,984.29263916)
\curveto(628.34103773,984.18263888)(628.26103781,984.067639)(628.18104225,983.94763916)
\curveto(628.10103797,983.82763924)(628.00603806,983.73263933)(627.89604225,983.66263916)
\curveto(627.85603821,983.64263942)(627.79103828,983.62763944)(627.70104225,983.61763916)
\curveto(627.62103845,983.59763947)(627.53103854,983.58763948)(627.43104225,983.58763916)
\curveto(627.33103874,983.58763948)(627.23603883,983.59263947)(627.14604225,983.60263916)
\curveto(627.066039,983.61263945)(627.00603906,983.63263943)(626.96604225,983.66263916)
\curveto(626.93603913,983.68263938)(626.91103916,983.71763935)(626.89104225,983.76763916)
\curveto(626.88103919,983.80763926)(626.88603918,983.85263921)(626.90604225,983.90263916)
\curveto(626.94603912,983.98263908)(626.99103908,984.05763901)(627.04104225,984.12763916)
\curveto(627.10103897,984.20763886)(627.15603891,984.28763878)(627.20604225,984.36763916)
\curveto(627.44603862,984.70763836)(627.69103838,985.04263802)(627.94104225,985.37263916)
\curveto(628.19103788,985.70263736)(628.43103764,986.03763703)(628.66104225,986.37763916)
\curveto(628.82103725,986.59763647)(628.98103709,986.81263625)(629.14104225,987.02263916)
\curveto(629.30103677,987.23263583)(629.46103661,987.44763562)(629.62104225,987.66763916)
\curveto(629.98103609,988.18763488)(630.34603572,988.69763437)(630.71604225,989.19763916)
\curveto(631.08603498,989.69763337)(631.45603461,990.20763286)(631.82604225,990.72763916)
\curveto(631.9660341,990.92763214)(632.10603396,991.12263194)(632.24604225,991.31263916)
\curveto(632.39603367,991.50263156)(632.54103353,991.69763137)(632.68104225,991.89763916)
\curveto(632.89103318,992.19763087)(633.10603296,992.49763057)(633.32604225,992.79763916)
\lineto(633.98604225,993.69763916)
\lineto(634.16604225,993.96763916)
\lineto(634.37604225,994.23763916)
\lineto(634.49604225,994.41763916)
\curveto(634.54603152,994.47762859)(634.59603147,994.53262853)(634.64604225,994.58263916)
\curveto(634.71603135,994.63262843)(634.79103128,994.6676284)(634.87104225,994.68763916)
\curveto(634.89103118,994.69762837)(634.91603115,994.69762837)(634.94604225,994.68763916)
\curveto(634.98603108,994.68762838)(635.01603105,994.69762837)(635.03604225,994.71763916)
\curveto(635.15603091,994.71762835)(635.29103078,994.71262835)(635.44104225,994.70263916)
\curveto(635.59103048,994.70262836)(635.68103039,994.65762841)(635.71104225,994.56763916)
\curveto(635.73103034,994.53762853)(635.73603033,994.50262856)(635.72604225,994.46263916)
\curveto(635.71603035,994.42262864)(635.70103037,994.39262867)(635.68104225,994.37263916)
\curveto(635.64103043,994.29262877)(635.60103047,994.22262884)(635.56104225,994.16263916)
\curveto(635.52103055,994.10262896)(635.47603059,994.04262902)(635.42604225,993.98263916)
\lineto(634.85604225,993.20263916)
\curveto(634.67603139,992.95263011)(634.49603157,992.69763037)(634.31604225,992.43763916)
\moveto(627.46104225,988.53763916)
\curveto(627.41103866,988.55763451)(627.36103871,988.5626345)(627.31104225,988.55263916)
\curveto(627.26103881,988.54263452)(627.21103886,988.54763452)(627.16104225,988.56763916)
\curveto(627.05103902,988.58763448)(626.94603912,988.60763446)(626.84604225,988.62763916)
\curveto(626.75603931,988.65763441)(626.66103941,988.69763437)(626.56104225,988.74763916)
\curveto(626.23103984,988.88763418)(625.97604009,989.08263398)(625.79604225,989.33263916)
\curveto(625.61604045,989.59263347)(625.4710406,989.90263316)(625.36104225,990.26263916)
\curveto(625.33104074,990.34263272)(625.31104076,990.42263264)(625.30104225,990.50263916)
\curveto(625.29104078,990.59263247)(625.27604079,990.67763239)(625.25604225,990.75763916)
\curveto(625.24604082,990.80763226)(625.24104083,990.87263219)(625.24104225,990.95263916)
\curveto(625.23104084,990.98263208)(625.22604084,991.01263205)(625.22604225,991.04263916)
\curveto(625.22604084,991.08263198)(625.22104085,991.11763195)(625.21104225,991.14763916)
\lineto(625.21104225,991.29763916)
\curveto(625.20104087,991.34763172)(625.19604087,991.40763166)(625.19604225,991.47763916)
\curveto(625.19604087,991.55763151)(625.20104087,991.62263144)(625.21104225,991.67263916)
\lineto(625.21104225,991.83763916)
\curveto(625.23104084,991.88763118)(625.23604083,991.93263113)(625.22604225,991.97263916)
\curveto(625.22604084,992.02263104)(625.23104084,992.067631)(625.24104225,992.10763916)
\curveto(625.25104082,992.14763092)(625.25604081,992.18263088)(625.25604225,992.21263916)
\curveto(625.25604081,992.25263081)(625.26104081,992.29263077)(625.27104225,992.33263916)
\curveto(625.30104077,992.44263062)(625.32104075,992.55263051)(625.33104225,992.66263916)
\curveto(625.35104072,992.78263028)(625.38604068,992.89763017)(625.43604225,993.00763916)
\curveto(625.57604049,993.34762972)(625.73604033,993.62262944)(625.91604225,993.83263916)
\curveto(626.10603996,994.05262901)(626.37603969,994.23262883)(626.72604225,994.37263916)
\curveto(626.80603926,994.40262866)(626.89103918,994.42262864)(626.98104225,994.43263916)
\curveto(627.071039,994.45262861)(627.1660389,994.47262859)(627.26604225,994.49263916)
\curveto(627.29603877,994.50262856)(627.35103872,994.50262856)(627.43104225,994.49263916)
\curveto(627.51103856,994.49262857)(627.56103851,994.50262856)(627.58104225,994.52263916)
\curveto(628.14103793,994.53262853)(628.59103748,994.42262864)(628.93104225,994.19263916)
\curveto(629.28103679,993.9626291)(629.54103653,993.65762941)(629.71104225,993.27763916)
\curveto(629.75103632,993.18762988)(629.78603628,993.09262997)(629.81604225,992.99263916)
\curveto(629.84603622,992.89263017)(629.8710362,992.79263027)(629.89104225,992.69263916)
\curveto(629.91103616,992.6626304)(629.91603615,992.63263043)(629.90604225,992.60263916)
\curveto(629.90603616,992.57263049)(629.91103616,992.54263052)(629.92104225,992.51263916)
\curveto(629.95103612,992.40263066)(629.9710361,992.27763079)(629.98104225,992.13763916)
\curveto(629.99103608,992.00763106)(630.00103607,991.87263119)(630.01104225,991.73263916)
\lineto(630.01104225,991.56763916)
\curveto(630.02103605,991.50763156)(630.02103605,991.45263161)(630.01104225,991.40263916)
\curveto(630.00103607,991.35263171)(629.99603607,991.30263176)(629.99604225,991.25263916)
\lineto(629.99604225,991.11763916)
\curveto(629.98603608,991.07763199)(629.98103609,991.03763203)(629.98104225,990.99763916)
\curveto(629.99103608,990.95763211)(629.98603608,990.91263215)(629.96604225,990.86263916)
\curveto(629.94603612,990.75263231)(629.92603614,990.64763242)(629.90604225,990.54763916)
\curveto(629.89603617,990.44763262)(629.87603619,990.34763272)(629.84604225,990.24763916)
\curveto(629.71603635,989.88763318)(629.55103652,989.57263349)(629.35104225,989.30263916)
\curveto(629.15103692,989.03263403)(628.87603719,988.82763424)(628.52604225,988.68763916)
\curveto(628.44603762,988.65763441)(628.36103771,988.63263443)(628.27104225,988.61263916)
\lineto(628.00104225,988.55263916)
\curveto(627.95103812,988.54263452)(627.90603816,988.53763453)(627.86604225,988.53763916)
\curveto(627.82603824,988.54763452)(627.78603828,988.54763452)(627.74604225,988.53763916)
\curveto(627.64603842,988.51763455)(627.55103852,988.51763455)(627.46104225,988.53763916)
\moveto(626.62104225,989.93263916)
\curveto(626.66103941,989.8626332)(626.70103937,989.79763327)(626.74104225,989.73763916)
\curveto(626.78103929,989.68763338)(626.83103924,989.63763343)(626.89104225,989.58763916)
\lineto(627.04104225,989.46763916)
\curveto(627.10103897,989.43763363)(627.1660389,989.41263365)(627.23604225,989.39263916)
\curveto(627.27603879,989.37263369)(627.31103876,989.3626337)(627.34104225,989.36263916)
\curveto(627.38103869,989.37263369)(627.42103865,989.3676337)(627.46104225,989.34763916)
\curveto(627.49103858,989.34763372)(627.53103854,989.34263372)(627.58104225,989.33263916)
\curveto(627.63103844,989.33263373)(627.6710384,989.33763373)(627.70104225,989.34763916)
\lineto(627.92604225,989.39263916)
\curveto(628.17603789,989.47263359)(628.36103771,989.59763347)(628.48104225,989.76763916)
\curveto(628.56103751,989.8676332)(628.63103744,989.99763307)(628.69104225,990.15763916)
\curveto(628.7710373,990.33763273)(628.83103724,990.5626325)(628.87104225,990.83263916)
\curveto(628.91103716,991.11263195)(628.92603714,991.39263167)(628.91604225,991.67263916)
\curveto(628.90603716,991.9626311)(628.87603719,992.23763083)(628.82604225,992.49763916)
\curveto(628.77603729,992.75763031)(628.70103737,992.9676301)(628.60104225,993.12763916)
\curveto(628.48103759,993.32762974)(628.33103774,993.47762959)(628.15104225,993.57763916)
\curveto(628.071038,993.62762944)(627.98103809,993.65762941)(627.88104225,993.66763916)
\curveto(627.78103829,993.68762938)(627.67603839,993.69762937)(627.56604225,993.69763916)
\curveto(627.54603852,993.68762938)(627.52103855,993.68262938)(627.49104225,993.68263916)
\curveto(627.4710386,993.69262937)(627.45103862,993.69262937)(627.43104225,993.68263916)
\curveto(627.38103869,993.67262939)(627.33603873,993.6626294)(627.29604225,993.65263916)
\curveto(627.25603881,993.65262941)(627.21603885,993.64262942)(627.17604225,993.62263916)
\curveto(626.99603907,993.54262952)(626.84603922,993.42262964)(626.72604225,993.26263916)
\curveto(626.61603945,993.10262996)(626.52603954,992.92263014)(626.45604225,992.72263916)
\curveto(626.39603967,992.53263053)(626.35103972,992.30763076)(626.32104225,992.04763916)
\curveto(626.30103977,991.78763128)(626.29603977,991.52263154)(626.30604225,991.25263916)
\curveto(626.31603975,990.99263207)(626.34603972,990.74263232)(626.39604225,990.50263916)
\curveto(626.45603961,990.27263279)(626.53103954,990.08263298)(626.62104225,989.93263916)
\moveto(637.42104225,986.94763916)
\curveto(637.43102864,986.89763617)(637.43602863,986.80763626)(637.43604225,986.67763916)
\curveto(637.43602863,986.54763652)(637.42602864,986.45763661)(637.40604225,986.40763916)
\curveto(637.38602868,986.35763671)(637.38102869,986.30263676)(637.39104225,986.24263916)
\curveto(637.40102867,986.19263687)(637.40102867,986.14263692)(637.39104225,986.09263916)
\curveto(637.35102872,985.95263711)(637.32102875,985.81763725)(637.30104225,985.68763916)
\curveto(637.29102878,985.55763751)(637.26102881,985.43763763)(637.21104225,985.32763916)
\curveto(637.071029,984.97763809)(636.90602916,984.68263838)(636.71604225,984.44263916)
\curveto(636.52602954,984.21263885)(636.25602981,984.02763904)(635.90604225,983.88763916)
\curveto(635.82603024,983.85763921)(635.74103033,983.83763923)(635.65104225,983.82763916)
\curveto(635.56103051,983.80763926)(635.47603059,983.78763928)(635.39604225,983.76763916)
\curveto(635.34603072,983.75763931)(635.29603077,983.75263931)(635.24604225,983.75263916)
\curveto(635.19603087,983.75263931)(635.14603092,983.74763932)(635.09604225,983.73763916)
\curveto(635.066031,983.72763934)(635.01603105,983.72763934)(634.94604225,983.73763916)
\curveto(634.87603119,983.73763933)(634.82603124,983.74263932)(634.79604225,983.75263916)
\curveto(634.73603133,983.77263929)(634.67603139,983.78263928)(634.61604225,983.78263916)
\curveto(634.5660315,983.77263929)(634.51603155,983.77763929)(634.46604225,983.79763916)
\curveto(634.37603169,983.81763925)(634.28603178,983.84263922)(634.19604225,983.87263916)
\curveto(634.11603195,983.89263917)(634.03603203,983.92263914)(633.95604225,983.96263916)
\curveto(633.63603243,984.10263896)(633.38603268,984.29763877)(633.20604225,984.54763916)
\curveto(633.02603304,984.80763826)(632.87603319,985.11263795)(632.75604225,985.46263916)
\curveto(632.73603333,985.54263752)(632.72103335,985.62763744)(632.71104225,985.71763916)
\curveto(632.70103337,985.80763726)(632.68603338,985.89263717)(632.66604225,985.97263916)
\curveto(632.65603341,986.00263706)(632.65103342,986.03263703)(632.65104225,986.06263916)
\lineto(632.65104225,986.16763916)
\curveto(632.63103344,986.24763682)(632.62103345,986.32763674)(632.62104225,986.40763916)
\lineto(632.62104225,986.54263916)
\curveto(632.60103347,986.64263642)(632.60103347,986.74263632)(632.62104225,986.84263916)
\lineto(632.62104225,987.02263916)
\curveto(632.63103344,987.07263599)(632.63603343,987.11763595)(632.63604225,987.15763916)
\curveto(632.63603343,987.20763586)(632.64103343,987.25263581)(632.65104225,987.29263916)
\curveto(632.66103341,987.33263573)(632.6660334,987.3676357)(632.66604225,987.39763916)
\curveto(632.6660334,987.43763563)(632.6710334,987.47763559)(632.68104225,987.51763916)
\lineto(632.74104225,987.84763916)
\curveto(632.76103331,987.9676351)(632.79103328,988.07763499)(632.83104225,988.17763916)
\curveto(632.9710331,988.50763456)(633.13103294,988.78263428)(633.31104225,989.00263916)
\curveto(633.50103257,989.23263383)(633.76103231,989.41763365)(634.09104225,989.55763916)
\curveto(634.1710319,989.59763347)(634.25603181,989.62263344)(634.34604225,989.63263916)
\lineto(634.64604225,989.69263916)
\lineto(634.78104225,989.69263916)
\curveto(634.83103124,989.70263336)(634.88103119,989.70763336)(634.93104225,989.70763916)
\curveto(635.50103057,989.72763334)(635.96103011,989.62263344)(636.31104225,989.39263916)
\curveto(636.6710294,989.17263389)(636.93602913,988.87263419)(637.10604225,988.49263916)
\curveto(637.15602891,988.39263467)(637.19602887,988.29263477)(637.22604225,988.19263916)
\curveto(637.25602881,988.09263497)(637.28602878,987.98763508)(637.31604225,987.87763916)
\curveto(637.32602874,987.83763523)(637.33102874,987.80263526)(637.33104225,987.77263916)
\curveto(637.33102874,987.75263531)(637.33602873,987.72263534)(637.34604225,987.68263916)
\curveto(637.3660287,987.61263545)(637.37602869,987.53763553)(637.37604225,987.45763916)
\curveto(637.37602869,987.37763569)(637.38602868,987.29763577)(637.40604225,987.21763916)
\curveto(637.40602866,987.1676359)(637.40602866,987.12263594)(637.40604225,987.08263916)
\curveto(637.40602866,987.04263602)(637.41102866,986.99763607)(637.42104225,986.94763916)
\moveto(636.31104225,986.51263916)
\curveto(636.32102975,986.5626365)(636.32602974,986.63763643)(636.32604225,986.73763916)
\curveto(636.33602973,986.83763623)(636.33102974,986.91263615)(636.31104225,986.96263916)
\curveto(636.29102978,987.02263604)(636.28602978,987.07763599)(636.29604225,987.12763916)
\curveto(636.31602975,987.18763588)(636.31602975,987.24763582)(636.29604225,987.30763916)
\curveto(636.28602978,987.33763573)(636.28102979,987.37263569)(636.28104225,987.41263916)
\curveto(636.28102979,987.45263561)(636.27602979,987.49263557)(636.26604225,987.53263916)
\curveto(636.24602982,987.61263545)(636.22602984,987.68763538)(636.20604225,987.75763916)
\curveto(636.19602987,987.83763523)(636.18102989,987.91763515)(636.16104225,987.99763916)
\curveto(636.13102994,988.05763501)(636.10602996,988.11763495)(636.08604225,988.17763916)
\curveto(636.06603,988.23763483)(636.03603003,988.29763477)(635.99604225,988.35763916)
\curveto(635.89603017,988.52763454)(635.7660303,988.6626344)(635.60604225,988.76263916)
\curveto(635.52603054,988.81263425)(635.43103064,988.84763422)(635.32104225,988.86763916)
\curveto(635.21103086,988.88763418)(635.08603098,988.89763417)(634.94604225,988.89763916)
\curveto(634.92603114,988.88763418)(634.90103117,988.88263418)(634.87104225,988.88263916)
\curveto(634.84103123,988.89263417)(634.81103126,988.89263417)(634.78104225,988.88263916)
\lineto(634.63104225,988.82263916)
\curveto(634.58103149,988.81263425)(634.53603153,988.79763427)(634.49604225,988.77763916)
\curveto(634.30603176,988.6676344)(634.16103191,988.52263454)(634.06104225,988.34263916)
\curveto(633.9710321,988.1626349)(633.89103218,987.95763511)(633.82104225,987.72763916)
\curveto(633.78103229,987.59763547)(633.76103231,987.4626356)(633.76104225,987.32263916)
\curveto(633.76103231,987.19263587)(633.75103232,987.04763602)(633.73104225,986.88763916)
\curveto(633.72103235,986.83763623)(633.71103236,986.77763629)(633.70104225,986.70763916)
\curveto(633.70103237,986.63763643)(633.71103236,986.57763649)(633.73104225,986.52763916)
\lineto(633.73104225,986.36263916)
\lineto(633.73104225,986.18263916)
\curveto(633.74103233,986.13263693)(633.75103232,986.07763699)(633.76104225,986.01763916)
\curveto(633.7710323,985.9676371)(633.77603229,985.91263715)(633.77604225,985.85263916)
\curveto(633.78603228,985.79263727)(633.80103227,985.73763733)(633.82104225,985.68763916)
\curveto(633.8710322,985.49763757)(633.93103214,985.32263774)(634.00104225,985.16263916)
\curveto(634.071032,985.00263806)(634.17603189,984.87263819)(634.31604225,984.77263916)
\curveto(634.44603162,984.67263839)(634.58603148,984.60263846)(634.73604225,984.56263916)
\curveto(634.7660313,984.55263851)(634.79103128,984.54763852)(634.81104225,984.54763916)
\curveto(634.84103123,984.55763851)(634.8710312,984.55763851)(634.90104225,984.54763916)
\curveto(634.92103115,984.54763852)(634.95103112,984.54263852)(634.99104225,984.53263916)
\curveto(635.03103104,984.53263853)(635.066031,984.53763853)(635.09604225,984.54763916)
\curveto(635.13603093,984.55763851)(635.17603089,984.5626385)(635.21604225,984.56263916)
\curveto(635.25603081,984.5626385)(635.29603077,984.57263849)(635.33604225,984.59263916)
\curveto(635.57603049,984.67263839)(635.7710303,984.80763826)(635.92104225,984.99763916)
\curveto(636.04103003,985.17763789)(636.13102994,985.38263768)(636.19104225,985.61263916)
\curveto(636.21102986,985.68263738)(636.22602984,985.75263731)(636.23604225,985.82263916)
\curveto(636.24602982,985.90263716)(636.26102981,985.98263708)(636.28104225,986.06263916)
\curveto(636.28102979,986.12263694)(636.28602978,986.1676369)(636.29604225,986.19763916)
\curveto(636.29602977,986.21763685)(636.29602977,986.24263682)(636.29604225,986.27263916)
\curveto(636.29602977,986.31263675)(636.30102977,986.34263672)(636.31104225,986.36263916)
\lineto(636.31104225,986.51263916)
}
}
{
\newrgbcolor{curcolor}{0 0 0}
\pscustom[linestyle=none,fillstyle=solid,fillcolor=curcolor]
{
\newpath
\moveto(592.80955055,744.13548432)
\curveto(592.81954283,744.09548127)(592.81954283,744.04548132)(592.80955055,743.98548432)
\curveto(592.80954284,743.92548144)(592.80454285,743.87548149)(592.79455055,743.83548432)
\curveto(592.79454286,743.79548157)(592.78954286,743.75548161)(592.77955055,743.71548432)
\lineto(592.77955055,743.61048432)
\curveto(592.75954289,743.53048183)(592.74454291,743.45048191)(592.73455055,743.37048432)
\curveto(592.72454293,743.29048207)(592.70454295,743.21548215)(592.67455055,743.14548432)
\curveto(592.654543,743.0654823)(592.63454302,742.99048237)(592.61455055,742.92048432)
\curveto(592.59454306,742.85048251)(592.56454309,742.77548259)(592.52455055,742.69548432)
\curveto(592.34454331,742.27548309)(592.08954356,741.93548343)(591.75955055,741.67548432)
\curveto(591.42954422,741.41548395)(591.03954461,741.21048415)(590.58955055,741.06048432)
\curveto(590.46954518,741.02048434)(590.34454531,740.99548437)(590.21455055,740.98548432)
\curveto(590.09454556,740.9654844)(589.96954568,740.94048442)(589.83955055,740.91048432)
\curveto(589.77954587,740.90048446)(589.71454594,740.89548447)(589.64455055,740.89548432)
\curveto(589.58454607,740.89548447)(589.51954613,740.89048447)(589.44955055,740.88048432)
\lineto(589.32955055,740.88048432)
\lineto(589.13455055,740.88048432)
\curveto(589.07454658,740.87048449)(589.01954663,740.87548449)(588.96955055,740.89548432)
\curveto(588.89954675,740.91548445)(588.83454682,740.92048444)(588.77455055,740.91048432)
\curveto(588.71454694,740.90048446)(588.654547,740.90548446)(588.59455055,740.92548432)
\curveto(588.54454711,740.93548443)(588.49954715,740.94048442)(588.45955055,740.94048432)
\curveto(588.41954723,740.94048442)(588.37454728,740.95048441)(588.32455055,740.97048432)
\curveto(588.24454741,740.99048437)(588.16954748,741.01048435)(588.09955055,741.03048432)
\curveto(588.02954762,741.04048432)(587.95954769,741.05548431)(587.88955055,741.07548432)
\curveto(587.40954824,741.24548412)(587.00954864,741.45548391)(586.68955055,741.70548432)
\curveto(586.37954927,741.9654834)(586.12954952,742.32048304)(585.93955055,742.77048432)
\curveto(585.90954974,742.83048253)(585.88454977,742.89048247)(585.86455055,742.95048432)
\curveto(585.8545498,743.02048234)(585.83954981,743.09548227)(585.81955055,743.17548432)
\curveto(585.79954985,743.23548213)(585.78454987,743.30048206)(585.77455055,743.37048432)
\curveto(585.76454989,743.44048192)(585.7495499,743.51048185)(585.72955055,743.58048432)
\curveto(585.71954993,743.63048173)(585.71454994,743.67048169)(585.71455055,743.70048432)
\lineto(585.71455055,743.82048432)
\curveto(585.70454995,743.8604815)(585.69454996,743.91048145)(585.68455055,743.97048432)
\curveto(585.68454997,744.03048133)(585.68954996,744.08048128)(585.69955055,744.12048432)
\lineto(585.69955055,744.25548432)
\curveto(585.70954994,744.30548106)(585.71454994,744.35548101)(585.71455055,744.40548432)
\curveto(585.73454992,744.50548086)(585.7495499,744.60048076)(585.75955055,744.69048432)
\curveto(585.76954988,744.79048057)(585.78954986,744.88548048)(585.81955055,744.97548432)
\curveto(585.86954978,745.12548024)(585.92454973,745.2654801)(585.98455055,745.39548432)
\curveto(586.04454961,745.52547984)(586.11454954,745.64547972)(586.19455055,745.75548432)
\curveto(586.22454943,745.80547956)(586.2545494,745.84547952)(586.28455055,745.87548432)
\curveto(586.32454933,745.90547946)(586.35954929,745.94047942)(586.38955055,745.98048432)
\curveto(586.4495492,746.0604793)(586.51954913,746.13047923)(586.59955055,746.19048432)
\curveto(586.65954899,746.24047912)(586.71954893,746.28547908)(586.77955055,746.32548432)
\lineto(586.98955055,746.47548432)
\curveto(587.03954861,746.51547885)(587.08954856,746.55047881)(587.13955055,746.58048432)
\curveto(587.18954846,746.62047874)(587.22454843,746.67547869)(587.24455055,746.74548432)
\curveto(587.24454841,746.77547859)(587.23454842,746.80047856)(587.21455055,746.82048432)
\curveto(587.20454845,746.85047851)(587.19454846,746.87547849)(587.18455055,746.89548432)
\curveto(587.14454851,746.94547842)(587.09454856,746.99047837)(587.03455055,747.03048432)
\curveto(586.98454867,747.08047828)(586.93454872,747.12547824)(586.88455055,747.16548432)
\curveto(586.84454881,747.19547817)(586.79454886,747.25047811)(586.73455055,747.33048432)
\curveto(586.71454894,747.360478)(586.68454897,747.38547798)(586.64455055,747.40548432)
\curveto(586.61454904,747.43547793)(586.58954906,747.47047789)(586.56955055,747.51048432)
\curveto(586.39954925,747.72047764)(586.26954938,747.9654774)(586.17955055,748.24548432)
\curveto(586.15954949,748.32547704)(586.14454951,748.40547696)(586.13455055,748.48548432)
\curveto(586.12454953,748.5654768)(586.10954954,748.64547672)(586.08955055,748.72548432)
\curveto(586.06954958,748.77547659)(586.05954959,748.84047652)(586.05955055,748.92048432)
\curveto(586.05954959,749.01047635)(586.06954958,749.08047628)(586.08955055,749.13048432)
\curveto(586.08954956,749.23047613)(586.09454956,749.30047606)(586.10455055,749.34048432)
\curveto(586.12454953,749.42047594)(586.13954951,749.49047587)(586.14955055,749.55048432)
\curveto(586.15954949,749.62047574)(586.17454948,749.69047567)(586.19455055,749.76048432)
\curveto(586.34454931,750.19047517)(586.55954909,750.53547483)(586.83955055,750.79548432)
\curveto(587.12954852,751.05547431)(587.47954817,751.27047409)(587.88955055,751.44048432)
\curveto(587.99954765,751.49047387)(588.11454754,751.52047384)(588.23455055,751.53048432)
\curveto(588.36454729,751.55047381)(588.49454716,751.58047378)(588.62455055,751.62048432)
\curveto(588.70454695,751.62047374)(588.77454688,751.62047374)(588.83455055,751.62048432)
\curveto(588.90454675,751.63047373)(588.97954667,751.64047372)(589.05955055,751.65048432)
\curveto(589.8495458,751.67047369)(590.50454515,751.54047382)(591.02455055,751.26048432)
\curveto(591.5545441,750.98047438)(591.93454372,750.57047479)(592.16455055,750.03048432)
\curveto(592.27454338,749.80047556)(592.34454331,749.51547585)(592.37455055,749.17548432)
\curveto(592.41454324,748.84547652)(592.38454327,748.54047682)(592.28455055,748.26048432)
\curveto(592.24454341,748.13047723)(592.19454346,748.01047735)(592.13455055,747.90048432)
\curveto(592.08454357,747.79047757)(592.02454363,747.68547768)(591.95455055,747.58548432)
\curveto(591.93454372,747.54547782)(591.90454375,747.51047785)(591.86455055,747.48048432)
\lineto(591.77455055,747.39048432)
\curveto(591.72454393,747.30047806)(591.66454399,747.23547813)(591.59455055,747.19548432)
\curveto(591.54454411,747.14547822)(591.48954416,747.09547827)(591.42955055,747.04548432)
\curveto(591.37954427,747.00547836)(591.33454432,746.9604784)(591.29455055,746.91048432)
\curveto(591.27454438,746.89047847)(591.2545444,746.8654785)(591.23455055,746.83548432)
\curveto(591.22454443,746.81547855)(591.22454443,746.79047857)(591.23455055,746.76048432)
\curveto(591.24454441,746.71047865)(591.27454438,746.6604787)(591.32455055,746.61048432)
\curveto(591.37454428,746.57047879)(591.42954422,746.53047883)(591.48955055,746.49048432)
\lineto(591.66955055,746.37048432)
\curveto(591.72954392,746.34047902)(591.77954387,746.31047905)(591.81955055,746.28048432)
\curveto(592.1495435,746.04047932)(592.39954325,745.73047963)(592.56955055,745.35048432)
\curveto(592.60954304,745.27048009)(592.63954301,745.18548018)(592.65955055,745.09548432)
\curveto(592.68954296,745.00548036)(592.71454294,744.91548045)(592.73455055,744.82548432)
\curveto(592.74454291,744.77548059)(592.7545429,744.72048064)(592.76455055,744.66048432)
\lineto(592.79455055,744.51048432)
\curveto(592.80454285,744.45048091)(592.80454285,744.38548098)(592.79455055,744.31548432)
\curveto(592.78454287,744.25548111)(592.78954286,744.19548117)(592.80955055,744.13548432)
\moveto(587.42455055,749.17548432)
\curveto(587.39454826,749.0654763)(587.38954826,748.92547644)(587.40955055,748.75548432)
\curveto(587.42954822,748.59547677)(587.4545482,748.47047689)(587.48455055,748.38048432)
\curveto(587.59454806,748.0604773)(587.74454791,747.81547755)(587.93455055,747.64548432)
\curveto(588.12454753,747.48547788)(588.38954726,747.35547801)(588.72955055,747.25548432)
\curveto(588.85954679,747.22547814)(589.02454663,747.20047816)(589.22455055,747.18048432)
\curveto(589.42454623,747.17047819)(589.59454606,747.18547818)(589.73455055,747.22548432)
\curveto(590.02454563,747.30547806)(590.26454539,747.41547795)(590.45455055,747.55548432)
\curveto(590.654545,747.70547766)(590.80954484,747.90547746)(590.91955055,748.15548432)
\curveto(590.93954471,748.20547716)(590.9495447,748.25047711)(590.94955055,748.29048432)
\curveto(590.95954469,748.33047703)(590.97454468,748.37547699)(590.99455055,748.42548432)
\curveto(591.02454463,748.53547683)(591.04454461,748.67547669)(591.05455055,748.84548432)
\curveto(591.06454459,749.01547635)(591.0545446,749.1604762)(591.02455055,749.28048432)
\curveto(591.00454465,749.37047599)(590.97954467,749.45547591)(590.94955055,749.53548432)
\curveto(590.92954472,749.61547575)(590.89454476,749.69547567)(590.84455055,749.77548432)
\curveto(590.67454498,750.04547532)(590.4495452,750.24047512)(590.16955055,750.36048432)
\curveto(589.89954575,750.48047488)(589.53954611,750.54047482)(589.08955055,750.54048432)
\curveto(589.06954658,750.52047484)(589.03954661,750.51547485)(588.99955055,750.52548432)
\curveto(588.95954669,750.53547483)(588.92454673,750.53547483)(588.89455055,750.52548432)
\curveto(588.84454681,750.50547486)(588.78954686,750.49047487)(588.72955055,750.48048432)
\curveto(588.67954697,750.48047488)(588.62954702,750.47047489)(588.57955055,750.45048432)
\curveto(588.33954731,750.360475)(588.12954752,750.24547512)(587.94955055,750.10548432)
\curveto(587.76954788,749.97547539)(587.62954802,749.79547557)(587.52955055,749.56548432)
\curveto(587.50954814,749.50547586)(587.48954816,749.44047592)(587.46955055,749.37048432)
\curveto(587.45954819,749.31047605)(587.44454821,749.24547612)(587.42455055,749.17548432)
\moveto(591.44455055,743.64048432)
\curveto(591.49454416,743.83048153)(591.49954415,744.03548133)(591.45955055,744.25548432)
\curveto(591.42954422,744.47548089)(591.38454427,744.65548071)(591.32455055,744.79548432)
\curveto(591.1545445,745.1654802)(590.89454476,745.47047989)(590.54455055,745.71048432)
\curveto(590.20454545,745.95047941)(589.76954588,746.07047929)(589.23955055,746.07048432)
\curveto(589.20954644,746.05047931)(589.16954648,746.04547932)(589.11955055,746.05548432)
\curveto(589.06954658,746.07547929)(589.02954662,746.08047928)(588.99955055,746.07048432)
\lineto(588.72955055,746.01048432)
\curveto(588.649547,746.00047936)(588.56954708,745.98547938)(588.48955055,745.96548432)
\curveto(588.18954746,745.85547951)(587.92454773,745.71047965)(587.69455055,745.53048432)
\curveto(587.47454818,745.35048001)(587.30454835,745.12048024)(587.18455055,744.84048432)
\curveto(587.1545485,744.7604806)(587.12954852,744.68048068)(587.10955055,744.60048432)
\curveto(587.08954856,744.52048084)(587.06954858,744.43548093)(587.04955055,744.34548432)
\curveto(587.01954863,744.22548114)(587.00954864,744.07548129)(587.01955055,743.89548432)
\curveto(587.03954861,743.71548165)(587.06454859,743.57548179)(587.09455055,743.47548432)
\curveto(587.11454854,743.42548194)(587.12454853,743.38048198)(587.12455055,743.34048432)
\curveto(587.13454852,743.31048205)(587.1495485,743.27048209)(587.16955055,743.22048432)
\curveto(587.26954838,743.00048236)(587.39954825,742.80048256)(587.55955055,742.62048432)
\curveto(587.72954792,742.44048292)(587.92454773,742.30548306)(588.14455055,742.21548432)
\curveto(588.21454744,742.17548319)(588.30954734,742.14048322)(588.42955055,742.11048432)
\curveto(588.649547,742.02048334)(588.90454675,741.97548339)(589.19455055,741.97548432)
\lineto(589.47955055,741.97548432)
\curveto(589.57954607,741.99548337)(589.67454598,742.01048335)(589.76455055,742.02048432)
\curveto(589.8545458,742.03048333)(589.94454571,742.05048331)(590.03455055,742.08048432)
\curveto(590.29454536,742.1604832)(590.53454512,742.29048307)(590.75455055,742.47048432)
\curveto(590.98454467,742.6604827)(591.1545445,742.87548249)(591.26455055,743.11548432)
\curveto(591.30454435,743.19548217)(591.33454432,743.27548209)(591.35455055,743.35548432)
\curveto(591.38454427,743.44548192)(591.41454424,743.54048182)(591.44455055,743.64048432)
}
}
{
\newrgbcolor{curcolor}{0 0 0}
\pscustom[linestyle=none,fillstyle=solid,fillcolor=curcolor]
{
\newpath
\moveto(598.20415993,751.66548432)
\curveto(598.30415507,751.6654737)(598.39915498,751.65547371)(598.48915993,751.63548432)
\curveto(598.5791548,751.62547374)(598.64415473,751.59547377)(598.68415993,751.54548432)
\curveto(598.74415463,751.4654739)(598.7741546,751.360474)(598.77415993,751.23048432)
\lineto(598.77415993,750.84048432)
\lineto(598.77415993,749.34048432)
\lineto(598.77415993,742.95048432)
\lineto(598.77415993,741.78048432)
\lineto(598.77415993,741.46548432)
\curveto(598.78415459,741.365484)(598.76915461,741.28548408)(598.72915993,741.22548432)
\curveto(598.6791547,741.14548422)(598.60415477,741.09548427)(598.50415993,741.07548432)
\curveto(598.41415496,741.0654843)(598.30415507,741.0604843)(598.17415993,741.06048432)
\lineto(597.94915993,741.06048432)
\curveto(597.86915551,741.08048428)(597.79915558,741.09548427)(597.73915993,741.10548432)
\curveto(597.6791557,741.12548424)(597.62915575,741.1654842)(597.58915993,741.22548432)
\curveto(597.54915583,741.28548408)(597.52915585,741.360484)(597.52915993,741.45048432)
\lineto(597.52915993,741.75048432)
\lineto(597.52915993,742.84548432)
\lineto(597.52915993,748.18548432)
\curveto(597.50915587,748.27547709)(597.49415588,748.35047701)(597.48415993,748.41048432)
\curveto(597.48415589,748.48047688)(597.45415592,748.54047682)(597.39415993,748.59048432)
\curveto(597.32415605,748.64047672)(597.23415614,748.6654767)(597.12415993,748.66548432)
\curveto(597.02415635,748.67547669)(596.91415646,748.68047668)(596.79415993,748.68048432)
\lineto(595.65415993,748.68048432)
\lineto(595.15915993,748.68048432)
\curveto(594.99915838,748.69047667)(594.88915849,748.75047661)(594.82915993,748.86048432)
\curveto(594.80915857,748.89047647)(594.79915858,748.92047644)(594.79915993,748.95048432)
\curveto(594.79915858,748.99047637)(594.79415858,749.03547633)(594.78415993,749.08548432)
\curveto(594.76415861,749.20547616)(594.76915861,749.31547605)(594.79915993,749.41548432)
\curveto(594.83915854,749.51547585)(594.89415848,749.58547578)(594.96415993,749.62548432)
\curveto(595.04415833,749.67547569)(595.16415821,749.70047566)(595.32415993,749.70048432)
\curveto(595.48415789,749.70047566)(595.61915776,749.71547565)(595.72915993,749.74548432)
\curveto(595.7791576,749.75547561)(595.83415754,749.7604756)(595.89415993,749.76048432)
\curveto(595.95415742,749.77047559)(596.01415736,749.78547558)(596.07415993,749.80548432)
\curveto(596.22415715,749.85547551)(596.36915701,749.90547546)(596.50915993,749.95548432)
\curveto(596.64915673,750.01547535)(596.78415659,750.08547528)(596.91415993,750.16548432)
\curveto(597.05415632,750.25547511)(597.1741562,750.360475)(597.27415993,750.48048432)
\curveto(597.374156,750.60047476)(597.46915591,750.73047463)(597.55915993,750.87048432)
\curveto(597.61915576,750.97047439)(597.66415571,751.08047428)(597.69415993,751.20048432)
\curveto(597.73415564,751.32047404)(597.78415559,751.42547394)(597.84415993,751.51548432)
\curveto(597.89415548,751.57547379)(597.96415541,751.61547375)(598.05415993,751.63548432)
\curveto(598.0741553,751.64547372)(598.09915528,751.65047371)(598.12915993,751.65048432)
\curveto(598.15915522,751.65047371)(598.18415519,751.65547371)(598.20415993,751.66548432)
}
}
{
\newrgbcolor{curcolor}{0 0 0}
\pscustom[linestyle=none,fillstyle=solid,fillcolor=curcolor]
{
\newpath
\moveto(603.4487693,742.69548432)
\lineto(603.7487693,742.69548432)
\curveto(603.85876724,742.70548266)(603.96376714,742.70548266)(604.0637693,742.69548432)
\curveto(604.17376693,742.69548267)(604.27376683,742.68548268)(604.3637693,742.66548432)
\curveto(604.45376665,742.65548271)(604.52376658,742.63048273)(604.5737693,742.59048432)
\curveto(604.59376651,742.57048279)(604.60876649,742.54048282)(604.6187693,742.50048432)
\curveto(604.63876646,742.4604829)(604.65876644,742.41548295)(604.6787693,742.36548432)
\lineto(604.6787693,742.29048432)
\curveto(604.68876641,742.24048312)(604.68876641,742.18548318)(604.6787693,742.12548432)
\lineto(604.6787693,741.97548432)
\lineto(604.6787693,741.49548432)
\curveto(604.67876642,741.32548404)(604.63876646,741.20548416)(604.5587693,741.13548432)
\curveto(604.48876661,741.08548428)(604.3987667,741.0604843)(604.2887693,741.06048432)
\lineto(603.9587693,741.06048432)
\lineto(603.5087693,741.06048432)
\curveto(603.35876774,741.0604843)(603.24376786,741.09048427)(603.1637693,741.15048432)
\curveto(603.12376798,741.18048418)(603.09376801,741.23048413)(603.0737693,741.30048432)
\curveto(603.05376805,741.38048398)(603.03876806,741.4654839)(603.0287693,741.55548432)
\lineto(603.0287693,741.84048432)
\curveto(603.03876806,741.94048342)(603.04376806,742.02548334)(603.0437693,742.09548432)
\lineto(603.0437693,742.29048432)
\curveto(603.04376806,742.35048301)(603.05376805,742.40548296)(603.0737693,742.45548432)
\curveto(603.11376799,742.5654828)(603.18376792,742.63548273)(603.2837693,742.66548432)
\curveto(603.31376779,742.6654827)(603.36876773,742.67548269)(603.4487693,742.69548432)
}
}
{
\newrgbcolor{curcolor}{0 0 0}
\pscustom[linestyle=none,fillstyle=solid,fillcolor=curcolor]
{
\newpath
\moveto(607.11392555,751.47048432)
\lineto(611.91392555,751.47048432)
\lineto(612.91892555,751.47048432)
\curveto(613.05891845,751.47047389)(613.17891833,751.4604739)(613.27892555,751.44048432)
\curveto(613.38891812,751.43047393)(613.46891804,751.38547398)(613.51892555,751.30548432)
\curveto(613.53891797,751.2654741)(613.54891796,751.21547415)(613.54892555,751.15548432)
\curveto(613.55891795,751.09547427)(613.56391795,751.03047433)(613.56392555,750.96048432)
\lineto(613.56392555,750.69048432)
\curveto(613.56391795,750.60047476)(613.55391796,750.52047484)(613.53392555,750.45048432)
\curveto(613.49391802,750.37047499)(613.44891806,750.30047506)(613.39892555,750.24048432)
\lineto(613.24892555,750.06048432)
\curveto(613.21891829,750.01047535)(613.18391833,749.97047539)(613.14392555,749.94048432)
\curveto(613.10391841,749.91047545)(613.06391845,749.87047549)(613.02392555,749.82048432)
\curveto(612.94391857,749.71047565)(612.85891865,749.60047576)(612.76892555,749.49048432)
\curveto(612.67891883,749.39047597)(612.59391892,749.28547608)(612.51392555,749.17548432)
\curveto(612.37391914,748.97547639)(612.23391928,748.7654766)(612.09392555,748.54548432)
\curveto(611.95391956,748.33547703)(611.8139197,748.12047724)(611.67392555,747.90048432)
\curveto(611.62391989,747.81047755)(611.57391994,747.71547765)(611.52392555,747.61548432)
\curveto(611.47392004,747.51547785)(611.41892009,747.42047794)(611.35892555,747.33048432)
\curveto(611.33892017,747.31047805)(611.32892018,747.28547808)(611.32892555,747.25548432)
\curveto(611.32892018,747.22547814)(611.31892019,747.20047816)(611.29892555,747.18048432)
\curveto(611.22892028,747.08047828)(611.16392035,746.9654784)(611.10392555,746.83548432)
\curveto(611.04392047,746.71547865)(610.98892052,746.60047876)(610.93892555,746.49048432)
\curveto(610.83892067,746.2604791)(610.74392077,746.02547934)(610.65392555,745.78548432)
\curveto(610.56392095,745.54547982)(610.46392105,745.30548006)(610.35392555,745.06548432)
\curveto(610.33392118,745.01548035)(610.31892119,744.97048039)(610.30892555,744.93048432)
\curveto(610.3089212,744.89048047)(610.29892121,744.84548052)(610.27892555,744.79548432)
\curveto(610.22892128,744.67548069)(610.18392133,744.55048081)(610.14392555,744.42048432)
\curveto(610.1139214,744.30048106)(610.07892143,744.18048118)(610.03892555,744.06048432)
\curveto(609.95892155,743.83048153)(609.89392162,743.59048177)(609.84392555,743.34048432)
\curveto(609.80392171,743.10048226)(609.75392176,742.8604825)(609.69392555,742.62048432)
\curveto(609.65392186,742.47048289)(609.62892188,742.32048304)(609.61892555,742.17048432)
\curveto(609.6089219,742.02048334)(609.58892192,741.87048349)(609.55892555,741.72048432)
\curveto(609.54892196,741.68048368)(609.54392197,741.62048374)(609.54392555,741.54048432)
\curveto(609.513922,741.42048394)(609.48392203,741.32048404)(609.45392555,741.24048432)
\curveto(609.42392209,741.1604842)(609.35392216,741.10548426)(609.24392555,741.07548432)
\curveto(609.19392232,741.05548431)(609.13892237,741.04548432)(609.07892555,741.04548432)
\lineto(608.88392555,741.04548432)
\curveto(608.74392277,741.04548432)(608.60392291,741.05048431)(608.46392555,741.06048432)
\curveto(608.33392318,741.07048429)(608.23892327,741.11548425)(608.17892555,741.19548432)
\curveto(608.13892337,741.25548411)(608.11892339,741.34048402)(608.11892555,741.45048432)
\curveto(608.12892338,741.5604838)(608.14392337,741.65548371)(608.16392555,741.73548432)
\lineto(608.16392555,741.81048432)
\curveto(608.17392334,741.84048352)(608.17892333,741.87048349)(608.17892555,741.90048432)
\curveto(608.19892331,741.98048338)(608.2089233,742.05548331)(608.20892555,742.12548432)
\curveto(608.2089233,742.19548317)(608.21892329,742.2654831)(608.23892555,742.33548432)
\curveto(608.28892322,742.52548284)(608.32892318,742.71048265)(608.35892555,742.89048432)
\curveto(608.38892312,743.08048228)(608.42892308,743.2604821)(608.47892555,743.43048432)
\curveto(608.49892301,743.48048188)(608.508923,743.52048184)(608.50892555,743.55048432)
\curveto(608.508923,743.58048178)(608.513923,743.61548175)(608.52392555,743.65548432)
\curveto(608.62392289,743.95548141)(608.7139228,744.25048111)(608.79392555,744.54048432)
\curveto(608.88392263,744.83048053)(608.98892252,745.11048025)(609.10892555,745.38048432)
\curveto(609.36892214,745.9604794)(609.63892187,746.51047885)(609.91892555,747.03048432)
\curveto(610.19892131,747.5604778)(610.508921,748.0654773)(610.84892555,748.54548432)
\curveto(610.98892052,748.74547662)(611.13892037,748.93547643)(611.29892555,749.11548432)
\curveto(611.45892005,749.30547606)(611.6089199,749.49547587)(611.74892555,749.68548432)
\curveto(611.78891972,749.73547563)(611.82391969,749.78047558)(611.85392555,749.82048432)
\curveto(611.89391962,749.87047549)(611.92891958,749.92047544)(611.95892555,749.97048432)
\curveto(611.96891954,749.99047537)(611.97891953,750.01547535)(611.98892555,750.04548432)
\curveto(612.0089195,750.07547529)(612.0089195,750.10547526)(611.98892555,750.13548432)
\curveto(611.96891954,750.19547517)(611.93391958,750.23047513)(611.88392555,750.24048432)
\curveto(611.83391968,750.2604751)(611.78391973,750.28047508)(611.73392555,750.30048432)
\lineto(611.62892555,750.30048432)
\curveto(611.58891992,750.31047505)(611.53891997,750.31047505)(611.47892555,750.30048432)
\lineto(611.32892555,750.30048432)
\lineto(610.72892555,750.30048432)
\lineto(608.08892555,750.30048432)
\lineto(607.35392555,750.30048432)
\lineto(607.11392555,750.30048432)
\curveto(607.04392447,750.31047505)(606.98392453,750.32547504)(606.93392555,750.34548432)
\curveto(606.84392467,750.38547498)(606.78392473,750.44547492)(606.75392555,750.52548432)
\curveto(606.70392481,750.62547474)(606.68892482,750.77047459)(606.70892555,750.96048432)
\curveto(606.72892478,751.1604742)(606.76392475,751.29547407)(606.81392555,751.36548432)
\curveto(606.83392468,751.38547398)(606.85892465,751.40047396)(606.88892555,751.41048432)
\lineto(607.00892555,751.47048432)
\curveto(607.02892448,751.47047389)(607.04392447,751.4654739)(607.05392555,751.45548432)
\curveto(607.07392444,751.45547391)(607.09392442,751.4604739)(607.11392555,751.47048432)
}
}
{
\newrgbcolor{curcolor}{0 0 0}
\pscustom[linestyle=none,fillstyle=solid,fillcolor=curcolor]
{
\newpath
\moveto(624.80853493,749.58048432)
\curveto(624.60852463,749.29047607)(624.39852484,749.00547636)(624.17853493,748.72548432)
\curveto(623.96852527,748.44547692)(623.76352547,748.1604772)(623.56353493,747.87048432)
\curveto(622.96352627,747.02047834)(622.35852688,746.18047918)(621.74853493,745.35048432)
\curveto(621.1385281,744.53048083)(620.5335287,743.69548167)(619.93353493,742.84548432)
\lineto(619.42353493,742.12548432)
\lineto(618.91353493,741.43548432)
\curveto(618.8335304,741.32548404)(618.75353048,741.21048415)(618.67353493,741.09048432)
\curveto(618.59353064,740.97048439)(618.49853074,740.87548449)(618.38853493,740.80548432)
\curveto(618.34853089,740.78548458)(618.28353095,740.77048459)(618.19353493,740.76048432)
\curveto(618.11353112,740.74048462)(618.02353121,740.73048463)(617.92353493,740.73048432)
\curveto(617.82353141,740.73048463)(617.72853151,740.73548463)(617.63853493,740.74548432)
\curveto(617.55853168,740.75548461)(617.49853174,740.77548459)(617.45853493,740.80548432)
\curveto(617.42853181,740.82548454)(617.40353183,740.8604845)(617.38353493,740.91048432)
\curveto(617.37353186,740.95048441)(617.37853186,740.99548437)(617.39853493,741.04548432)
\curveto(617.4385318,741.12548424)(617.48353175,741.20048416)(617.53353493,741.27048432)
\curveto(617.59353164,741.35048401)(617.64853159,741.43048393)(617.69853493,741.51048432)
\curveto(617.9385313,741.85048351)(618.18353105,742.18548318)(618.43353493,742.51548432)
\curveto(618.68353055,742.84548252)(618.92353031,743.18048218)(619.15353493,743.52048432)
\curveto(619.31352992,743.74048162)(619.47352976,743.95548141)(619.63353493,744.16548432)
\curveto(619.79352944,744.37548099)(619.95352928,744.59048077)(620.11353493,744.81048432)
\curveto(620.47352876,745.33048003)(620.8385284,745.84047952)(621.20853493,746.34048432)
\curveto(621.57852766,746.84047852)(621.94852729,747.35047801)(622.31853493,747.87048432)
\curveto(622.45852678,748.07047729)(622.59852664,748.2654771)(622.73853493,748.45548432)
\curveto(622.88852635,748.64547672)(623.0335262,748.84047652)(623.17353493,749.04048432)
\curveto(623.38352585,749.34047602)(623.59852564,749.64047572)(623.81853493,749.94048432)
\lineto(624.47853493,750.84048432)
\lineto(624.65853493,751.11048432)
\lineto(624.86853493,751.38048432)
\lineto(624.98853493,751.56048432)
\curveto(625.0385242,751.62047374)(625.08852415,751.67547369)(625.13853493,751.72548432)
\curveto(625.20852403,751.77547359)(625.28352395,751.81047355)(625.36353493,751.83048432)
\curveto(625.38352385,751.84047352)(625.40852383,751.84047352)(625.43853493,751.83048432)
\curveto(625.47852376,751.83047353)(625.50852373,751.84047352)(625.52853493,751.86048432)
\curveto(625.64852359,751.8604735)(625.78352345,751.85547351)(625.93353493,751.84548432)
\curveto(626.08352315,751.84547352)(626.17352306,751.80047356)(626.20353493,751.71048432)
\curveto(626.22352301,751.68047368)(626.22852301,751.64547372)(626.21853493,751.60548432)
\curveto(626.20852303,751.5654738)(626.19352304,751.53547383)(626.17353493,751.51548432)
\curveto(626.1335231,751.43547393)(626.09352314,751.365474)(626.05353493,751.30548432)
\curveto(626.01352322,751.24547412)(625.96852327,751.18547418)(625.91853493,751.12548432)
\lineto(625.34853493,750.34548432)
\curveto(625.16852407,750.09547527)(624.98852425,749.84047552)(624.80853493,749.58048432)
\moveto(617.95353493,745.68048432)
\curveto(617.90353133,745.70047966)(617.85353138,745.70547966)(617.80353493,745.69548432)
\curveto(617.75353148,745.68547968)(617.70353153,745.69047967)(617.65353493,745.71048432)
\curveto(617.54353169,745.73047963)(617.4385318,745.75047961)(617.33853493,745.77048432)
\curveto(617.24853199,745.80047956)(617.15353208,745.84047952)(617.05353493,745.89048432)
\curveto(616.72353251,746.03047933)(616.46853277,746.22547914)(616.28853493,746.47548432)
\curveto(616.10853313,746.73547863)(615.96353327,747.04547832)(615.85353493,747.40548432)
\curveto(615.82353341,747.48547788)(615.80353343,747.5654778)(615.79353493,747.64548432)
\curveto(615.78353345,747.73547763)(615.76853347,747.82047754)(615.74853493,747.90048432)
\curveto(615.7385335,747.95047741)(615.7335335,748.01547735)(615.73353493,748.09548432)
\curveto(615.72353351,748.12547724)(615.71853352,748.15547721)(615.71853493,748.18548432)
\curveto(615.71853352,748.22547714)(615.71353352,748.2604771)(615.70353493,748.29048432)
\lineto(615.70353493,748.44048432)
\curveto(615.69353354,748.49047687)(615.68853355,748.55047681)(615.68853493,748.62048432)
\curveto(615.68853355,748.70047666)(615.69353354,748.7654766)(615.70353493,748.81548432)
\lineto(615.70353493,748.98048432)
\curveto(615.72353351,749.03047633)(615.72853351,749.07547629)(615.71853493,749.11548432)
\curveto(615.71853352,749.1654762)(615.72353351,749.21047615)(615.73353493,749.25048432)
\curveto(615.74353349,749.29047607)(615.74853349,749.32547604)(615.74853493,749.35548432)
\curveto(615.74853349,749.39547597)(615.75353348,749.43547593)(615.76353493,749.47548432)
\curveto(615.79353344,749.58547578)(615.81353342,749.69547567)(615.82353493,749.80548432)
\curveto(615.84353339,749.92547544)(615.87853336,750.04047532)(615.92853493,750.15048432)
\curveto(616.06853317,750.49047487)(616.22853301,750.7654746)(616.40853493,750.97548432)
\curveto(616.59853264,751.19547417)(616.86853237,751.37547399)(617.21853493,751.51548432)
\curveto(617.29853194,751.54547382)(617.38353185,751.5654738)(617.47353493,751.57548432)
\curveto(617.56353167,751.59547377)(617.65853158,751.61547375)(617.75853493,751.63548432)
\curveto(617.78853145,751.64547372)(617.84353139,751.64547372)(617.92353493,751.63548432)
\curveto(618.00353123,751.63547373)(618.05353118,751.64547372)(618.07353493,751.66548432)
\curveto(618.6335306,751.67547369)(619.08353015,751.5654738)(619.42353493,751.33548432)
\curveto(619.77352946,751.10547426)(620.0335292,750.80047456)(620.20353493,750.42048432)
\curveto(620.24352899,750.33047503)(620.27852896,750.23547513)(620.30853493,750.13548432)
\curveto(620.3385289,750.03547533)(620.36352887,749.93547543)(620.38353493,749.83548432)
\curveto(620.40352883,749.80547556)(620.40852883,749.77547559)(620.39853493,749.74548432)
\curveto(620.39852884,749.71547565)(620.40352883,749.68547568)(620.41353493,749.65548432)
\curveto(620.44352879,749.54547582)(620.46352877,749.42047594)(620.47353493,749.28048432)
\curveto(620.48352875,749.15047621)(620.49352874,749.01547635)(620.50353493,748.87548432)
\lineto(620.50353493,748.71048432)
\curveto(620.51352872,748.65047671)(620.51352872,748.59547677)(620.50353493,748.54548432)
\curveto(620.49352874,748.49547687)(620.48852875,748.44547692)(620.48853493,748.39548432)
\lineto(620.48853493,748.26048432)
\curveto(620.47852876,748.22047714)(620.47352876,748.18047718)(620.47353493,748.14048432)
\curveto(620.48352875,748.10047726)(620.47852876,748.05547731)(620.45853493,748.00548432)
\curveto(620.4385288,747.89547747)(620.41852882,747.79047757)(620.39853493,747.69048432)
\curveto(620.38852885,747.59047777)(620.36852887,747.49047787)(620.33853493,747.39048432)
\curveto(620.20852903,747.03047833)(620.04352919,746.71547865)(619.84353493,746.44548432)
\curveto(619.64352959,746.17547919)(619.36852987,745.97047939)(619.01853493,745.83048432)
\curveto(618.9385303,745.80047956)(618.85353038,745.77547959)(618.76353493,745.75548432)
\lineto(618.49353493,745.69548432)
\curveto(618.44353079,745.68547968)(618.39853084,745.68047968)(618.35853493,745.68048432)
\curveto(618.31853092,745.69047967)(618.27853096,745.69047967)(618.23853493,745.68048432)
\curveto(618.1385311,745.6604797)(618.04353119,745.6604797)(617.95353493,745.68048432)
\moveto(617.11353493,747.07548432)
\curveto(617.15353208,747.00547836)(617.19353204,746.94047842)(617.23353493,746.88048432)
\curveto(617.27353196,746.83047853)(617.32353191,746.78047858)(617.38353493,746.73048432)
\lineto(617.53353493,746.61048432)
\curveto(617.59353164,746.58047878)(617.65853158,746.55547881)(617.72853493,746.53548432)
\curveto(617.76853147,746.51547885)(617.80353143,746.50547886)(617.83353493,746.50548432)
\curveto(617.87353136,746.51547885)(617.91353132,746.51047885)(617.95353493,746.49048432)
\curveto(617.98353125,746.49047887)(618.02353121,746.48547888)(618.07353493,746.47548432)
\curveto(618.12353111,746.47547889)(618.16353107,746.48047888)(618.19353493,746.49048432)
\lineto(618.41853493,746.53548432)
\curveto(618.66853057,746.61547875)(618.85353038,746.74047862)(618.97353493,746.91048432)
\curveto(619.05353018,747.01047835)(619.12353011,747.14047822)(619.18353493,747.30048432)
\curveto(619.26352997,747.48047788)(619.32352991,747.70547766)(619.36353493,747.97548432)
\curveto(619.40352983,748.25547711)(619.41852982,748.53547683)(619.40853493,748.81548432)
\curveto(619.39852984,749.10547626)(619.36852987,749.38047598)(619.31853493,749.64048432)
\curveto(619.26852997,749.90047546)(619.19353004,750.11047525)(619.09353493,750.27048432)
\curveto(618.97353026,750.47047489)(618.82353041,750.62047474)(618.64353493,750.72048432)
\curveto(618.56353067,750.77047459)(618.47353076,750.80047456)(618.37353493,750.81048432)
\curveto(618.27353096,750.83047453)(618.16853107,750.84047452)(618.05853493,750.84048432)
\curveto(618.0385312,750.83047453)(618.01353122,750.82547454)(617.98353493,750.82548432)
\curveto(617.96353127,750.83547453)(617.94353129,750.83547453)(617.92353493,750.82548432)
\curveto(617.87353136,750.81547455)(617.82853141,750.80547456)(617.78853493,750.79548432)
\curveto(617.74853149,750.79547457)(617.70853153,750.78547458)(617.66853493,750.76548432)
\curveto(617.48853175,750.68547468)(617.3385319,750.5654748)(617.21853493,750.40548432)
\curveto(617.10853213,750.24547512)(617.01853222,750.0654753)(616.94853493,749.86548432)
\curveto(616.88853235,749.67547569)(616.84353239,749.45047591)(616.81353493,749.19048432)
\curveto(616.79353244,748.93047643)(616.78853245,748.6654767)(616.79853493,748.39548432)
\curveto(616.80853243,748.13547723)(616.8385324,747.88547748)(616.88853493,747.64548432)
\curveto(616.94853229,747.41547795)(617.02353221,747.22547814)(617.11353493,747.07548432)
\moveto(627.91353493,744.09048432)
\curveto(627.92352131,744.04048132)(627.92852131,743.95048141)(627.92853493,743.82048432)
\curveto(627.92852131,743.69048167)(627.91852132,743.60048176)(627.89853493,743.55048432)
\curveto(627.87852136,743.50048186)(627.87352136,743.44548192)(627.88353493,743.38548432)
\curveto(627.89352134,743.33548203)(627.89352134,743.28548208)(627.88353493,743.23548432)
\curveto(627.84352139,743.09548227)(627.81352142,742.9604824)(627.79353493,742.83048432)
\curveto(627.78352145,742.70048266)(627.75352148,742.58048278)(627.70353493,742.47048432)
\curveto(627.56352167,742.12048324)(627.39852184,741.82548354)(627.20853493,741.58548432)
\curveto(627.01852222,741.35548401)(626.74852249,741.17048419)(626.39853493,741.03048432)
\curveto(626.31852292,741.00048436)(626.233523,740.98048438)(626.14353493,740.97048432)
\curveto(626.05352318,740.95048441)(625.96852327,740.93048443)(625.88853493,740.91048432)
\curveto(625.8385234,740.90048446)(625.78852345,740.89548447)(625.73853493,740.89548432)
\curveto(625.68852355,740.89548447)(625.6385236,740.89048447)(625.58853493,740.88048432)
\curveto(625.55852368,740.87048449)(625.50852373,740.87048449)(625.43853493,740.88048432)
\curveto(625.36852387,740.88048448)(625.31852392,740.88548448)(625.28853493,740.89548432)
\curveto(625.22852401,740.91548445)(625.16852407,740.92548444)(625.10853493,740.92548432)
\curveto(625.05852418,740.91548445)(625.00852423,740.92048444)(624.95853493,740.94048432)
\curveto(624.86852437,740.9604844)(624.77852446,740.98548438)(624.68853493,741.01548432)
\curveto(624.60852463,741.03548433)(624.52852471,741.0654843)(624.44853493,741.10548432)
\curveto(624.12852511,741.24548412)(623.87852536,741.44048392)(623.69853493,741.69048432)
\curveto(623.51852572,741.95048341)(623.36852587,742.25548311)(623.24853493,742.60548432)
\curveto(623.22852601,742.68548268)(623.21352602,742.77048259)(623.20353493,742.86048432)
\curveto(623.19352604,742.95048241)(623.17852606,743.03548233)(623.15853493,743.11548432)
\curveto(623.14852609,743.14548222)(623.14352609,743.17548219)(623.14353493,743.20548432)
\lineto(623.14353493,743.31048432)
\curveto(623.12352611,743.39048197)(623.11352612,743.47048189)(623.11353493,743.55048432)
\lineto(623.11353493,743.68548432)
\curveto(623.09352614,743.78548158)(623.09352614,743.88548148)(623.11353493,743.98548432)
\lineto(623.11353493,744.16548432)
\curveto(623.12352611,744.21548115)(623.12852611,744.2604811)(623.12853493,744.30048432)
\curveto(623.12852611,744.35048101)(623.1335261,744.39548097)(623.14353493,744.43548432)
\curveto(623.15352608,744.47548089)(623.15852608,744.51048085)(623.15853493,744.54048432)
\curveto(623.15852608,744.58048078)(623.16352607,744.62048074)(623.17353493,744.66048432)
\lineto(623.23353493,744.99048432)
\curveto(623.25352598,745.11048025)(623.28352595,745.22048014)(623.32353493,745.32048432)
\curveto(623.46352577,745.65047971)(623.62352561,745.92547944)(623.80353493,746.14548432)
\curveto(623.99352524,746.37547899)(624.25352498,746.5604788)(624.58353493,746.70048432)
\curveto(624.66352457,746.74047862)(624.74852449,746.7654786)(624.83853493,746.77548432)
\lineto(625.13853493,746.83548432)
\lineto(625.27353493,746.83548432)
\curveto(625.32352391,746.84547852)(625.37352386,746.85047851)(625.42353493,746.85048432)
\curveto(625.99352324,746.87047849)(626.45352278,746.7654786)(626.80353493,746.53548432)
\curveto(627.16352207,746.31547905)(627.42852181,746.01547935)(627.59853493,745.63548432)
\curveto(627.64852159,745.53547983)(627.68852155,745.43547993)(627.71853493,745.33548432)
\curveto(627.74852149,745.23548013)(627.77852146,745.13048023)(627.80853493,745.02048432)
\curveto(627.81852142,744.98048038)(627.82352141,744.94548042)(627.82353493,744.91548432)
\curveto(627.82352141,744.89548047)(627.82852141,744.8654805)(627.83853493,744.82548432)
\curveto(627.85852138,744.75548061)(627.86852137,744.68048068)(627.86853493,744.60048432)
\curveto(627.86852137,744.52048084)(627.87852136,744.44048092)(627.89853493,744.36048432)
\curveto(627.89852134,744.31048105)(627.89852134,744.2654811)(627.89853493,744.22548432)
\curveto(627.89852134,744.18548118)(627.90352133,744.14048122)(627.91353493,744.09048432)
\moveto(626.80353493,743.65548432)
\curveto(626.81352242,743.70548166)(626.81852242,743.78048158)(626.81853493,743.88048432)
\curveto(626.82852241,743.98048138)(626.82352241,744.05548131)(626.80353493,744.10548432)
\curveto(626.78352245,744.1654812)(626.77852246,744.22048114)(626.78853493,744.27048432)
\curveto(626.80852243,744.33048103)(626.80852243,744.39048097)(626.78853493,744.45048432)
\curveto(626.77852246,744.48048088)(626.77352246,744.51548085)(626.77353493,744.55548432)
\curveto(626.77352246,744.59548077)(626.76852247,744.63548073)(626.75853493,744.67548432)
\curveto(626.7385225,744.75548061)(626.71852252,744.83048053)(626.69853493,744.90048432)
\curveto(626.68852255,744.98048038)(626.67352256,745.0604803)(626.65353493,745.14048432)
\curveto(626.62352261,745.20048016)(626.59852264,745.2604801)(626.57853493,745.32048432)
\curveto(626.55852268,745.38047998)(626.52852271,745.44047992)(626.48853493,745.50048432)
\curveto(626.38852285,745.67047969)(626.25852298,745.80547956)(626.09853493,745.90548432)
\curveto(626.01852322,745.95547941)(625.92352331,745.99047937)(625.81353493,746.01048432)
\curveto(625.70352353,746.03047933)(625.57852366,746.04047932)(625.43853493,746.04048432)
\curveto(625.41852382,746.03047933)(625.39352384,746.02547934)(625.36353493,746.02548432)
\curveto(625.3335239,746.03547933)(625.30352393,746.03547933)(625.27353493,746.02548432)
\lineto(625.12353493,745.96548432)
\curveto(625.07352416,745.95547941)(625.02852421,745.94047942)(624.98853493,745.92048432)
\curveto(624.79852444,745.81047955)(624.65352458,745.6654797)(624.55353493,745.48548432)
\curveto(624.46352477,745.30548006)(624.38352485,745.10048026)(624.31353493,744.87048432)
\curveto(624.27352496,744.74048062)(624.25352498,744.60548076)(624.25353493,744.46548432)
\curveto(624.25352498,744.33548103)(624.24352499,744.19048117)(624.22353493,744.03048432)
\curveto(624.21352502,743.98048138)(624.20352503,743.92048144)(624.19353493,743.85048432)
\curveto(624.19352504,743.78048158)(624.20352503,743.72048164)(624.22353493,743.67048432)
\lineto(624.22353493,743.50548432)
\lineto(624.22353493,743.32548432)
\curveto(624.233525,743.27548209)(624.24352499,743.22048214)(624.25353493,743.16048432)
\curveto(624.26352497,743.11048225)(624.26852497,743.05548231)(624.26853493,742.99548432)
\curveto(624.27852496,742.93548243)(624.29352494,742.88048248)(624.31353493,742.83048432)
\curveto(624.36352487,742.64048272)(624.42352481,742.4654829)(624.49353493,742.30548432)
\curveto(624.56352467,742.14548322)(624.66852457,742.01548335)(624.80853493,741.91548432)
\curveto(624.9385243,741.81548355)(625.07852416,741.74548362)(625.22853493,741.70548432)
\curveto(625.25852398,741.69548367)(625.28352395,741.69048367)(625.30353493,741.69048432)
\curveto(625.3335239,741.70048366)(625.36352387,741.70048366)(625.39353493,741.69048432)
\curveto(625.41352382,741.69048367)(625.44352379,741.68548368)(625.48353493,741.67548432)
\curveto(625.52352371,741.67548369)(625.55852368,741.68048368)(625.58853493,741.69048432)
\curveto(625.62852361,741.70048366)(625.66852357,741.70548366)(625.70853493,741.70548432)
\curveto(625.74852349,741.70548366)(625.78852345,741.71548365)(625.82853493,741.73548432)
\curveto(626.06852317,741.81548355)(626.26352297,741.95048341)(626.41353493,742.14048432)
\curveto(626.5335227,742.32048304)(626.62352261,742.52548284)(626.68353493,742.75548432)
\curveto(626.70352253,742.82548254)(626.71852252,742.89548247)(626.72853493,742.96548432)
\curveto(626.7385225,743.04548232)(626.75352248,743.12548224)(626.77353493,743.20548432)
\curveto(626.77352246,743.2654821)(626.77852246,743.31048205)(626.78853493,743.34048432)
\curveto(626.78852245,743.360482)(626.78852245,743.38548198)(626.78853493,743.41548432)
\curveto(626.78852245,743.45548191)(626.79352244,743.48548188)(626.80353493,743.50548432)
\lineto(626.80353493,743.65548432)
}
}
{
\newrgbcolor{curcolor}{0 0 0}
\pscustom[linestyle=none,fillstyle=solid,fillcolor=curcolor]
{
\newpath
\moveto(259.14024086,630.23692597)
\curveto(259.24023601,630.23691535)(259.33523591,630.22691536)(259.42524086,630.20692597)
\curveto(259.51523573,630.19691539)(259.58023567,630.16691542)(259.62024086,630.11692597)
\curveto(259.68023557,630.03691555)(259.71023554,629.93191565)(259.71024086,629.80192597)
\lineto(259.71024086,629.41192597)
\lineto(259.71024086,627.91192597)
\lineto(259.71024086,621.52192597)
\lineto(259.71024086,620.35192597)
\lineto(259.71024086,620.03692597)
\curveto(259.72023553,619.93692565)(259.70523554,619.85692573)(259.66524086,619.79692597)
\curveto(259.61523563,619.71692587)(259.54023571,619.66692592)(259.44024086,619.64692597)
\curveto(259.3502359,619.63692595)(259.24023601,619.63192595)(259.11024086,619.63192597)
\lineto(258.88524086,619.63192597)
\curveto(258.80523644,619.65192593)(258.73523651,619.66692592)(258.67524086,619.67692597)
\curveto(258.61523663,619.69692589)(258.56523668,619.73692585)(258.52524086,619.79692597)
\curveto(258.48523676,619.85692573)(258.46523678,619.93192565)(258.46524086,620.02192597)
\lineto(258.46524086,620.32192597)
\lineto(258.46524086,621.41692597)
\lineto(258.46524086,626.75692597)
\curveto(258.4452368,626.84691874)(258.43023682,626.92191866)(258.42024086,626.98192597)
\curveto(258.42023683,627.05191853)(258.39023686,627.11191847)(258.33024086,627.16192597)
\curveto(258.26023699,627.21191837)(258.17023708,627.23691835)(258.06024086,627.23692597)
\curveto(257.96023729,627.24691834)(257.8502374,627.25191833)(257.73024086,627.25192597)
\lineto(256.59024086,627.25192597)
\lineto(256.09524086,627.25192597)
\curveto(255.93523931,627.26191832)(255.82523942,627.32191826)(255.76524086,627.43192597)
\curveto(255.7452395,627.46191812)(255.73523951,627.49191809)(255.73524086,627.52192597)
\curveto(255.73523951,627.56191802)(255.73023952,627.60691798)(255.72024086,627.65692597)
\curveto(255.70023955,627.77691781)(255.70523954,627.8869177)(255.73524086,627.98692597)
\curveto(255.77523947,628.0869175)(255.83023942,628.15691743)(255.90024086,628.19692597)
\curveto(255.98023927,628.24691734)(256.10023915,628.27191731)(256.26024086,628.27192597)
\curveto(256.42023883,628.27191731)(256.55523869,628.2869173)(256.66524086,628.31692597)
\curveto(256.71523853,628.32691726)(256.77023848,628.33191725)(256.83024086,628.33192597)
\curveto(256.89023836,628.34191724)(256.9502383,628.35691723)(257.01024086,628.37692597)
\curveto(257.16023809,628.42691716)(257.30523794,628.47691711)(257.44524086,628.52692597)
\curveto(257.58523766,628.586917)(257.72023753,628.65691693)(257.85024086,628.73692597)
\curveto(257.99023726,628.82691676)(258.11023714,628.93191665)(258.21024086,629.05192597)
\curveto(258.31023694,629.17191641)(258.40523684,629.30191628)(258.49524086,629.44192597)
\curveto(258.55523669,629.54191604)(258.60023665,629.65191593)(258.63024086,629.77192597)
\curveto(258.67023658,629.89191569)(258.72023653,629.99691559)(258.78024086,630.08692597)
\curveto(258.83023642,630.14691544)(258.90023635,630.1869154)(258.99024086,630.20692597)
\curveto(259.01023624,630.21691537)(259.03523621,630.22191536)(259.06524086,630.22192597)
\curveto(259.09523615,630.22191536)(259.12023613,630.22691536)(259.14024086,630.23692597)
}
}
{
\newrgbcolor{curcolor}{0 0 0}
\pscustom[linestyle=none,fillstyle=solid,fillcolor=curcolor]
{
\newpath
\moveto(266.54485024,630.23692597)
\curveto(268.1748448,630.26691532)(269.22484375,629.71191587)(269.69485024,628.57192597)
\curveto(269.79484318,628.34191724)(269.85984311,628.05191753)(269.88985024,627.70192597)
\curveto(269.92984304,627.36191822)(269.90484307,627.05191853)(269.81485024,626.77192597)
\curveto(269.72484325,626.51191907)(269.60484337,626.2869193)(269.45485024,626.09692597)
\curveto(269.43484354,626.05691953)(269.40984356,626.02191956)(269.37985024,625.99192597)
\curveto(269.34984362,625.97191961)(269.32484365,625.94691964)(269.30485024,625.91692597)
\lineto(269.21485024,625.79692597)
\curveto(269.18484379,625.76691982)(269.14984382,625.74191984)(269.10985024,625.72192597)
\curveto(269.05984391,625.67191991)(269.00484397,625.62691996)(268.94485024,625.58692597)
\curveto(268.89484408,625.54692004)(268.84984412,625.49692009)(268.80985024,625.43692597)
\curveto(268.7698442,625.39692019)(268.75484422,625.34692024)(268.76485024,625.28692597)
\curveto(268.7748442,625.23692035)(268.80484417,625.19192039)(268.85485024,625.15192597)
\curveto(268.90484407,625.11192047)(268.95984401,625.07192051)(269.01985024,625.03192597)
\curveto(269.08984388,625.00192058)(269.15484382,624.97192061)(269.21485024,624.94192597)
\curveto(269.2748437,624.91192067)(269.32484365,624.8819207)(269.36485024,624.85192597)
\curveto(269.68484329,624.63192095)(269.93984303,624.32192126)(270.12985024,623.92192597)
\curveto(270.1698428,623.83192175)(270.19984277,623.73692185)(270.21985024,623.63692597)
\curveto(270.24984272,623.54692204)(270.2748427,623.45692213)(270.29485024,623.36692597)
\curveto(270.30484267,623.31692227)(270.30984266,623.26692232)(270.30985024,623.21692597)
\curveto(270.31984265,623.17692241)(270.32984264,623.13192245)(270.33985024,623.08192597)
\curveto(270.34984262,623.03192255)(270.34984262,622.9819226)(270.33985024,622.93192597)
\curveto(270.32984264,622.8819227)(270.33484264,622.83192275)(270.35485024,622.78192597)
\curveto(270.36484261,622.73192285)(270.3698426,622.67192291)(270.36985024,622.60192597)
\curveto(270.3698426,622.53192305)(270.35984261,622.47192311)(270.33985024,622.42192597)
\lineto(270.33985024,622.19692597)
\lineto(270.27985024,621.95692597)
\curveto(270.2698427,621.8869237)(270.25484272,621.81692377)(270.23485024,621.74692597)
\curveto(270.20484277,621.65692393)(270.1748428,621.57192401)(270.14485024,621.49192597)
\curveto(270.12484285,621.41192417)(270.09484288,621.33192425)(270.05485024,621.25192597)
\curveto(270.03484294,621.19192439)(270.00484297,621.13192445)(269.96485024,621.07192597)
\curveto(269.93484304,621.02192456)(269.89984307,620.97192461)(269.85985024,620.92192597)
\curveto(269.65984331,620.61192497)(269.40984356,620.35192523)(269.10985024,620.14192597)
\curveto(268.80984416,619.94192564)(268.46484451,619.77692581)(268.07485024,619.64692597)
\curveto(267.95484502,619.60692598)(267.82484515,619.581926)(267.68485024,619.57192597)
\curveto(267.55484542,619.55192603)(267.41984555,619.52692606)(267.27985024,619.49692597)
\curveto(267.20984576,619.4869261)(267.13984583,619.4819261)(267.06985024,619.48192597)
\curveto(267.00984596,619.4819261)(266.94484603,619.47692611)(266.87485024,619.46692597)
\curveto(266.83484614,619.45692613)(266.7748462,619.45192613)(266.69485024,619.45192597)
\curveto(266.62484635,619.45192613)(266.5748464,619.45692613)(266.54485024,619.46692597)
\curveto(266.49484648,619.47692611)(266.44984652,619.4819261)(266.40985024,619.48192597)
\lineto(266.28985024,619.48192597)
\curveto(266.18984678,619.50192608)(266.08984688,619.51692607)(265.98985024,619.52692597)
\curveto(265.88984708,619.53692605)(265.79484718,619.55192603)(265.70485024,619.57192597)
\curveto(265.59484738,619.60192598)(265.48484749,619.62692596)(265.37485024,619.64692597)
\curveto(265.2748477,619.67692591)(265.1698478,619.71692587)(265.05985024,619.76692597)
\curveto(264.68984828,619.92692566)(264.3748486,620.12692546)(264.11485024,620.36692597)
\curveto(263.85484912,620.61692497)(263.64484933,620.92692466)(263.48485024,621.29692597)
\curveto(263.44484953,621.3869242)(263.40984956,621.4819241)(263.37985024,621.58192597)
\curveto(263.34984962,621.6819239)(263.31984965,621.7869238)(263.28985024,621.89692597)
\curveto(263.2698497,621.94692364)(263.25984971,621.99692359)(263.25985024,622.04692597)
\curveto(263.25984971,622.10692348)(263.24984972,622.16692342)(263.22985024,622.22692597)
\curveto(263.20984976,622.2869233)(263.19984977,622.36692322)(263.19985024,622.46692597)
\curveto(263.19984977,622.56692302)(263.21484976,622.64192294)(263.24485024,622.69192597)
\curveto(263.25484972,622.72192286)(263.2698497,622.74692284)(263.28985024,622.76692597)
\lineto(263.34985024,622.82692597)
\curveto(263.38984958,622.84692274)(263.44984952,622.86192272)(263.52985024,622.87192597)
\curveto(263.61984935,622.8819227)(263.70984926,622.8869227)(263.79985024,622.88692597)
\curveto(263.88984908,622.8869227)(263.974849,622.8819227)(264.05485024,622.87192597)
\curveto(264.14484883,622.86192272)(264.20984876,622.85192273)(264.24985024,622.84192597)
\curveto(264.2698487,622.82192276)(264.28984868,622.80692278)(264.30985024,622.79692597)
\curveto(264.32984864,622.79692279)(264.34984862,622.7869228)(264.36985024,622.76692597)
\curveto(264.43984853,622.67692291)(264.47984849,622.56192302)(264.48985024,622.42192597)
\curveto(264.50984846,622.2819233)(264.53984843,622.15692343)(264.57985024,622.04692597)
\lineto(264.72985024,621.68692597)
\curveto(264.77984819,621.57692401)(264.84484813,621.47192411)(264.92485024,621.37192597)
\curveto(264.94484803,621.34192424)(264.96484801,621.31692427)(264.98485024,621.29692597)
\curveto(265.01484796,621.27692431)(265.03984793,621.25192433)(265.05985024,621.22192597)
\curveto(265.09984787,621.16192442)(265.13484784,621.11692447)(265.16485024,621.08692597)
\curveto(265.20484777,621.05692453)(265.23984773,621.02692456)(265.26985024,620.99692597)
\curveto(265.30984766,620.96692462)(265.35484762,620.93692465)(265.40485024,620.90692597)
\curveto(265.49484748,620.84692474)(265.58984738,620.79692479)(265.68985024,620.75692597)
\lineto(266.01985024,620.63692597)
\curveto(266.1698468,620.586925)(266.3698466,620.55692503)(266.61985024,620.54692597)
\curveto(266.8698461,620.53692505)(267.07984589,620.55692503)(267.24985024,620.60692597)
\curveto(267.32984564,620.62692496)(267.39984557,620.64192494)(267.45985024,620.65192597)
\lineto(267.66985024,620.71192597)
\curveto(267.94984502,620.83192475)(268.18984478,620.9819246)(268.38985024,621.16192597)
\curveto(268.59984437,621.34192424)(268.76484421,621.57192401)(268.88485024,621.85192597)
\curveto(268.91484406,621.92192366)(268.93484404,621.99192359)(268.94485024,622.06192597)
\lineto(269.00485024,622.30192597)
\curveto(269.04484393,622.44192314)(269.05484392,622.60192298)(269.03485024,622.78192597)
\curveto(269.01484396,622.97192261)(268.98484399,623.12192246)(268.94485024,623.23192597)
\curveto(268.81484416,623.61192197)(268.62984434,623.90192168)(268.38985024,624.10192597)
\curveto(268.15984481,624.30192128)(267.84984512,624.46192112)(267.45985024,624.58192597)
\curveto(267.34984562,624.61192097)(267.22984574,624.63192095)(267.09985024,624.64192597)
\curveto(266.97984599,624.65192093)(266.85484612,624.65692093)(266.72485024,624.65692597)
\curveto(266.56484641,624.65692093)(266.42484655,624.66192092)(266.30485024,624.67192597)
\curveto(266.18484679,624.6819209)(266.09984687,624.74192084)(266.04985024,624.85192597)
\curveto(266.02984694,624.8819207)(266.01984695,624.91692067)(266.01985024,624.95692597)
\lineto(266.01985024,625.09192597)
\curveto(266.00984696,625.19192039)(266.00984696,625.2869203)(266.01985024,625.37692597)
\curveto(266.03984693,625.46692012)(266.07984689,625.53192005)(266.13985024,625.57192597)
\curveto(266.17984679,625.60191998)(266.21984675,625.62191996)(266.25985024,625.63192597)
\curveto(266.30984666,625.64191994)(266.36484661,625.65191993)(266.42485024,625.66192597)
\curveto(266.44484653,625.67191991)(266.4698465,625.67191991)(266.49985024,625.66192597)
\curveto(266.52984644,625.66191992)(266.55484642,625.66691992)(266.57485024,625.67692597)
\lineto(266.70985024,625.67692597)
\curveto(266.81984615,625.69691989)(266.91984605,625.70691988)(267.00985024,625.70692597)
\curveto(267.10984586,625.71691987)(267.20484577,625.73691985)(267.29485024,625.76692597)
\curveto(267.61484536,625.87691971)(267.8698451,626.02191956)(268.05985024,626.20192597)
\curveto(268.24984472,626.3819192)(268.39984457,626.63191895)(268.50985024,626.95192597)
\curveto(268.53984443,627.05191853)(268.55984441,627.17691841)(268.56985024,627.32692597)
\curveto(268.58984438,627.4869181)(268.58484439,627.63191795)(268.55485024,627.76192597)
\curveto(268.53484444,627.83191775)(268.51484446,627.89691769)(268.49485024,627.95692597)
\curveto(268.48484449,628.02691756)(268.46484451,628.09191749)(268.43485024,628.15192597)
\curveto(268.33484464,628.39191719)(268.18984478,628.581917)(267.99985024,628.72192597)
\curveto(267.80984516,628.86191672)(267.58484539,628.97191661)(267.32485024,629.05192597)
\curveto(267.26484571,629.07191651)(267.20484577,629.0819165)(267.14485024,629.08192597)
\curveto(267.08484589,629.0819165)(267.01984595,629.09191649)(266.94985024,629.11192597)
\curveto(266.8698461,629.13191645)(266.7748462,629.14191644)(266.66485024,629.14192597)
\curveto(266.55484642,629.14191644)(266.45984651,629.13191645)(266.37985024,629.11192597)
\curveto(266.32984664,629.09191649)(266.27984669,629.0819165)(266.22985024,629.08192597)
\curveto(266.18984678,629.0819165)(266.14484683,629.07191651)(266.09485024,629.05192597)
\curveto(265.91484706,629.00191658)(265.74484723,628.92691666)(265.58485024,628.82692597)
\curveto(265.43484754,628.73691685)(265.30484767,628.62191696)(265.19485024,628.48192597)
\curveto(265.10484787,628.36191722)(265.02484795,628.23191735)(264.95485024,628.09192597)
\curveto(264.88484809,627.95191763)(264.81984815,627.79691779)(264.75985024,627.62692597)
\curveto(264.72984824,627.51691807)(264.70984826,627.39691819)(264.69985024,627.26692597)
\curveto(264.68984828,627.14691844)(264.65484832,627.04691854)(264.59485024,626.96692597)
\curveto(264.5748484,626.92691866)(264.51484846,626.8869187)(264.41485024,626.84692597)
\curveto(264.3748486,626.83691875)(264.31484866,626.82691876)(264.23485024,626.81692597)
\lineto(263.97985024,626.81692597)
\curveto(263.88984908,626.82691876)(263.80484917,626.83691875)(263.72485024,626.84692597)
\curveto(263.65484932,626.85691873)(263.60484937,626.87191871)(263.57485024,626.89192597)
\curveto(263.53484944,626.92191866)(263.49984947,626.97691861)(263.46985024,627.05692597)
\curveto(263.43984953,627.13691845)(263.43484954,627.22191836)(263.45485024,627.31192597)
\curveto(263.46484951,627.36191822)(263.4698495,627.41191817)(263.46985024,627.46192597)
\lineto(263.49985024,627.64192597)
\curveto(263.52984944,627.74191784)(263.55484942,627.84191774)(263.57485024,627.94192597)
\curveto(263.60484937,628.04191754)(263.63984933,628.13191745)(263.67985024,628.21192597)
\curveto(263.72984924,628.32191726)(263.7748492,628.42691716)(263.81485024,628.52692597)
\curveto(263.85484912,628.63691695)(263.90484907,628.74191684)(263.96485024,628.84192597)
\curveto(264.29484868,629.3819162)(264.76484821,629.77691581)(265.37485024,630.02692597)
\curveto(265.49484748,630.07691551)(265.61984735,630.11191547)(265.74985024,630.13192597)
\curveto(265.88984708,630.15191543)(266.02984694,630.17691541)(266.16985024,630.20692597)
\curveto(266.22984674,630.21691537)(266.28984668,630.22191536)(266.34985024,630.22192597)
\curveto(266.41984655,630.22191536)(266.48484649,630.22691536)(266.54485024,630.23692597)
}
}
{
\newrgbcolor{curcolor}{0 0 0}
\pscustom[linestyle=none,fillstyle=solid,fillcolor=curcolor]
{
\newpath
\moveto(272.73445961,621.26692597)
\lineto(273.03445961,621.26692597)
\curveto(273.14445755,621.27692431)(273.24945745,621.27692431)(273.34945961,621.26692597)
\curveto(273.45945724,621.26692432)(273.55945714,621.25692433)(273.64945961,621.23692597)
\curveto(273.73945696,621.22692436)(273.80945689,621.20192438)(273.85945961,621.16192597)
\curveto(273.87945682,621.14192444)(273.8944568,621.11192447)(273.90445961,621.07192597)
\curveto(273.92445677,621.03192455)(273.94445675,620.9869246)(273.96445961,620.93692597)
\lineto(273.96445961,620.86192597)
\curveto(273.97445672,620.81192477)(273.97445672,620.75692483)(273.96445961,620.69692597)
\lineto(273.96445961,620.54692597)
\lineto(273.96445961,620.06692597)
\curveto(273.96445673,619.89692569)(273.92445677,619.77692581)(273.84445961,619.70692597)
\curveto(273.77445692,619.65692593)(273.68445701,619.63192595)(273.57445961,619.63192597)
\lineto(273.24445961,619.63192597)
\lineto(272.79445961,619.63192597)
\curveto(272.64445805,619.63192595)(272.52945817,619.66192592)(272.44945961,619.72192597)
\curveto(272.40945829,619.75192583)(272.37945832,619.80192578)(272.35945961,619.87192597)
\curveto(272.33945836,619.95192563)(272.32445837,620.03692555)(272.31445961,620.12692597)
\lineto(272.31445961,620.41192597)
\curveto(272.32445837,620.51192507)(272.32945837,620.59692499)(272.32945961,620.66692597)
\lineto(272.32945961,620.86192597)
\curveto(272.32945837,620.92192466)(272.33945836,620.97692461)(272.35945961,621.02692597)
\curveto(272.3994583,621.13692445)(272.46945823,621.20692438)(272.56945961,621.23692597)
\curveto(272.5994581,621.23692435)(272.65445804,621.24692434)(272.73445961,621.26692597)
}
}
{
\newrgbcolor{curcolor}{0 0 0}
\pscustom[linestyle=none,fillstyle=solid,fillcolor=curcolor]
{
\newpath
\moveto(282.87961586,625.22692597)
\curveto(282.87960823,625.14692044)(282.88460822,625.06692052)(282.89461586,624.98692597)
\curveto(282.9046082,624.90692068)(282.89960821,624.83192075)(282.87961586,624.76192597)
\curveto(282.85960825,624.72192086)(282.85460825,624.67692091)(282.86461586,624.62692597)
\curveto(282.87460823,624.586921)(282.87460823,624.54692104)(282.86461586,624.50692597)
\lineto(282.86461586,624.35692597)
\curveto(282.85460825,624.26692132)(282.84960826,624.17692141)(282.84961586,624.08692597)
\curveto(282.84960826,624.00692158)(282.84460826,623.92692166)(282.83461586,623.84692597)
\lineto(282.80461586,623.60692597)
\curveto(282.79460831,623.53692205)(282.78460832,623.46192212)(282.77461586,623.38192597)
\curveto(282.76460834,623.34192224)(282.75960835,623.30192228)(282.75961586,623.26192597)
\curveto(282.75960835,623.22192236)(282.75460835,623.17692241)(282.74461586,623.12692597)
\curveto(282.7046084,622.9869226)(282.67460843,622.84692274)(282.65461586,622.70692597)
\curveto(282.64460846,622.56692302)(282.61460849,622.43192315)(282.56461586,622.30192597)
\curveto(282.51460859,622.13192345)(282.45960865,621.96692362)(282.39961586,621.80692597)
\curveto(282.34960876,621.64692394)(282.28960882,621.49192409)(282.21961586,621.34192597)
\curveto(282.19960891,621.2819243)(282.16960894,621.22192436)(282.12961586,621.16192597)
\lineto(282.03961586,621.01192597)
\curveto(281.83960927,620.69192489)(281.62460948,620.42692516)(281.39461586,620.21692597)
\curveto(281.16460994,620.00692558)(280.86961024,619.82692576)(280.50961586,619.67692597)
\curveto(280.38961072,619.62692596)(280.25961085,619.59192599)(280.11961586,619.57192597)
\curveto(279.98961112,619.55192603)(279.85461125,619.52692606)(279.71461586,619.49692597)
\curveto(279.65461145,619.4869261)(279.59461151,619.4819261)(279.53461586,619.48192597)
\curveto(279.47461163,619.4819261)(279.4096117,619.47692611)(279.33961586,619.46692597)
\curveto(279.3096118,619.45692613)(279.25961185,619.45692613)(279.18961586,619.46692597)
\lineto(279.03961586,619.46692597)
\lineto(278.88961586,619.46692597)
\curveto(278.8096123,619.4869261)(278.72461238,619.50192608)(278.63461586,619.51192597)
\curveto(278.55461255,619.51192607)(278.47961263,619.52192606)(278.40961586,619.54192597)
\curveto(278.36961274,619.55192603)(278.33461277,619.55692603)(278.30461586,619.55692597)
\curveto(278.28461282,619.54692604)(278.25961285,619.55192603)(278.22961586,619.57192597)
\lineto(277.95961586,619.63192597)
\curveto(277.86961324,619.66192592)(277.78461332,619.69192589)(277.70461586,619.72192597)
\curveto(277.12461398,619.96192562)(276.68961442,620.33192525)(276.39961586,620.83192597)
\curveto(276.31961479,620.96192462)(276.25461485,621.09692449)(276.20461586,621.23692597)
\curveto(276.16461494,621.37692421)(276.11961499,621.52692406)(276.06961586,621.68692597)
\curveto(276.04961506,621.76692382)(276.04461506,621.84692374)(276.05461586,621.92692597)
\curveto(276.07461503,622.00692358)(276.109615,622.06192352)(276.15961586,622.09192597)
\curveto(276.18961492,622.11192347)(276.24461486,622.12692346)(276.32461586,622.13692597)
\curveto(276.4046147,622.15692343)(276.48961462,622.16692342)(276.57961586,622.16692597)
\curveto(276.66961444,622.17692341)(276.75461435,622.17692341)(276.83461586,622.16692597)
\curveto(276.92461418,622.15692343)(276.99461411,622.14692344)(277.04461586,622.13692597)
\curveto(277.06461404,622.12692346)(277.08961402,622.11192347)(277.11961586,622.09192597)
\curveto(277.15961395,622.07192351)(277.18961392,622.05192353)(277.20961586,622.03192597)
\curveto(277.26961384,621.95192363)(277.31461379,621.85692373)(277.34461586,621.74692597)
\curveto(277.38461372,621.63692395)(277.42961368,621.53692405)(277.47961586,621.44692597)
\curveto(277.72961338,621.05692453)(278.09961301,620.7869248)(278.58961586,620.63692597)
\curveto(278.65961245,620.61692497)(278.72961238,620.60192498)(278.79961586,620.59192597)
\curveto(278.87961223,620.59192499)(278.95961215,620.581925)(279.03961586,620.56192597)
\curveto(279.07961203,620.55192503)(279.13461197,620.54692504)(279.20461586,620.54692597)
\curveto(279.28461182,620.54692504)(279.33961177,620.55192503)(279.36961586,620.56192597)
\curveto(279.39961171,620.57192501)(279.42961168,620.57692501)(279.45961586,620.57692597)
\lineto(279.56461586,620.57692597)
\curveto(279.64461146,620.59692499)(279.71961139,620.61692497)(279.78961586,620.63692597)
\curveto(279.86961124,620.65692493)(279.94461116,620.6819249)(280.01461586,620.71192597)
\curveto(280.36461074,620.86192472)(280.63461047,621.07692451)(280.82461586,621.35692597)
\curveto(281.01461009,621.63692395)(281.16960994,621.96192362)(281.28961586,622.33192597)
\curveto(281.31960979,622.41192317)(281.33960977,622.4869231)(281.34961586,622.55692597)
\curveto(281.36960974,622.62692296)(281.38960972,622.70192288)(281.40961586,622.78192597)
\curveto(281.42960968,622.87192271)(281.44460966,622.96692262)(281.45461586,623.06692597)
\curveto(281.47460963,623.17692241)(281.49460961,623.2819223)(281.51461586,623.38192597)
\curveto(281.52460958,623.43192215)(281.52960958,623.4819221)(281.52961586,623.53192597)
\curveto(281.53960957,623.59192199)(281.54460956,623.64692194)(281.54461586,623.69692597)
\curveto(281.56460954,623.75692183)(281.57460953,623.83192175)(281.57461586,623.92192597)
\curveto(281.57460953,624.02192156)(281.56460954,624.10192148)(281.54461586,624.16192597)
\curveto(281.51460959,624.25192133)(281.46460964,624.29192129)(281.39461586,624.28192597)
\curveto(281.33460977,624.27192131)(281.27960983,624.24192134)(281.22961586,624.19192597)
\curveto(281.14960996,624.14192144)(281.07961003,624.0819215)(281.01961586,624.01192597)
\curveto(280.96961014,623.94192164)(280.9046102,623.8819217)(280.82461586,623.83192597)
\curveto(280.66461044,623.72192186)(280.49961061,623.62192196)(280.32961586,623.53192597)
\curveto(280.15961095,623.45192213)(279.96461114,623.3819222)(279.74461586,623.32192597)
\curveto(279.64461146,623.29192229)(279.54461156,623.27692231)(279.44461586,623.27692597)
\curveto(279.35461175,623.27692231)(279.25461185,623.26692232)(279.14461586,623.24692597)
\lineto(278.99461586,623.24692597)
\curveto(278.94461216,623.26692232)(278.89461221,623.27192231)(278.84461586,623.26192597)
\curveto(278.8046123,623.25192233)(278.76461234,623.25192233)(278.72461586,623.26192597)
\curveto(278.69461241,623.27192231)(278.64961246,623.27692231)(278.58961586,623.27692597)
\curveto(278.52961258,623.2869223)(278.46461264,623.29692229)(278.39461586,623.30692597)
\lineto(278.21461586,623.33692597)
\curveto(277.76461334,623.45692213)(277.38461372,623.62192196)(277.07461586,623.83192597)
\curveto(276.8046143,624.02192156)(276.57461453,624.25192133)(276.38461586,624.52192597)
\curveto(276.2046149,624.80192078)(276.05961505,625.11692047)(275.94961586,625.46692597)
\lineto(275.88961586,625.67692597)
\curveto(275.87961523,625.75691983)(275.86461524,625.83691975)(275.84461586,625.91692597)
\curveto(275.83461527,625.94691964)(275.82961528,625.97691961)(275.82961586,626.00692597)
\curveto(275.82961528,626.03691955)(275.82461528,626.06691952)(275.81461586,626.09692597)
\curveto(275.8046153,626.15691943)(275.79961531,626.21691937)(275.79961586,626.27692597)
\curveto(275.79961531,626.34691924)(275.78961532,626.40691918)(275.76961586,626.45692597)
\lineto(275.76961586,626.63692597)
\curveto(275.75961535,626.6869189)(275.75461535,626.75691883)(275.75461586,626.84692597)
\curveto(275.75461535,626.93691865)(275.76461534,627.00691858)(275.78461586,627.05692597)
\lineto(275.78461586,627.22192597)
\curveto(275.8046153,627.30191828)(275.81461529,627.37691821)(275.81461586,627.44692597)
\curveto(275.82461528,627.51691807)(275.83961527,627.586918)(275.85961586,627.65692597)
\curveto(275.91961519,627.85691773)(275.97961513,628.04691754)(276.03961586,628.22692597)
\curveto(276.109615,628.40691718)(276.19961491,628.57691701)(276.30961586,628.73692597)
\curveto(276.34961476,628.80691678)(276.38961472,628.87191671)(276.42961586,628.93192597)
\lineto(276.57961586,629.11192597)
\curveto(276.59961451,629.12191646)(276.61961449,629.13691645)(276.63961586,629.15692597)
\curveto(276.72961438,629.2869163)(276.83961427,629.39691619)(276.96961586,629.48692597)
\curveto(277.22961388,629.6869159)(277.49461361,629.84191574)(277.76461586,629.95192597)
\curveto(277.84461326,629.99191559)(277.92461318,630.02191556)(278.00461586,630.04192597)
\curveto(278.09461301,630.07191551)(278.18461292,630.09691549)(278.27461586,630.11692597)
\curveto(278.37461273,630.14691544)(278.47461263,630.16691542)(278.57461586,630.17692597)
\curveto(278.67461243,630.1869154)(278.77961233,630.20191538)(278.88961586,630.22192597)
\curveto(278.91961219,630.23191535)(278.95961215,630.23191535)(279.00961586,630.22192597)
\curveto(279.06961204,630.21191537)(279.109612,630.21691537)(279.12961586,630.23692597)
\curveto(279.84961126,630.25691533)(280.44961066,630.14191544)(280.92961586,629.89192597)
\curveto(281.4096097,629.64191594)(281.78460932,629.30191628)(282.05461586,628.87192597)
\curveto(282.14460896,628.73191685)(282.22460888,628.586917)(282.29461586,628.43692597)
\curveto(282.36460874,628.2869173)(282.43460867,628.12691746)(282.50461586,627.95692597)
\curveto(282.55460855,627.81691777)(282.59460851,627.66691792)(282.62461586,627.50692597)
\curveto(282.65460845,627.34691824)(282.68960842,627.1869184)(282.72961586,627.02692597)
\curveto(282.74960836,626.97691861)(282.75960835,626.92191866)(282.75961586,626.86192597)
\curveto(282.75960835,626.81191877)(282.76460834,626.76191882)(282.77461586,626.71192597)
\curveto(282.79460831,626.65191893)(282.8046083,626.586919)(282.80461586,626.51692597)
\curveto(282.8046083,626.45691913)(282.81460829,626.40191918)(282.83461586,626.35192597)
\lineto(282.83461586,626.18692597)
\curveto(282.85460825,626.13691945)(282.85960825,626.0869195)(282.84961586,626.03692597)
\curveto(282.83960827,625.9869196)(282.84460826,625.93691965)(282.86461586,625.88692597)
\curveto(282.86460824,625.86691972)(282.85960825,625.84191974)(282.84961586,625.81192597)
\curveto(282.84960826,625.7819198)(282.85460825,625.75691983)(282.86461586,625.73692597)
\curveto(282.87460823,625.70691988)(282.87460823,625.67191991)(282.86461586,625.63192597)
\curveto(282.86460824,625.59191999)(282.86960824,625.55192003)(282.87961586,625.51192597)
\curveto(282.88960822,625.47192011)(282.88960822,625.42692016)(282.87961586,625.37692597)
\lineto(282.87961586,625.22692597)
\moveto(281.37961586,626.53192597)
\curveto(281.38960972,626.581919)(281.39460971,626.64191894)(281.39461586,626.71192597)
\curveto(281.39460971,626.7819188)(281.38960972,626.84191874)(281.37961586,626.89192597)
\curveto(281.36960974,626.94191864)(281.36460974,627.01691857)(281.36461586,627.11692597)
\curveto(281.34460976,627.19691839)(281.32460978,627.27191831)(281.30461586,627.34192597)
\curveto(281.29460981,627.41191817)(281.27960983,627.4819181)(281.25961586,627.55192597)
\curveto(281.11960999,627.9819176)(280.92461018,628.31691727)(280.67461586,628.55692597)
\curveto(280.43461067,628.79691679)(280.08961102,628.97691661)(279.63961586,629.09692597)
\curveto(279.54961156,629.11691647)(279.44961166,629.12691646)(279.33961586,629.12692597)
\lineto(279.00961586,629.12692597)
\curveto(278.98961212,629.10691648)(278.95461215,629.09691649)(278.90461586,629.09692597)
\curveto(278.85461225,629.10691648)(278.8096123,629.10691648)(278.76961586,629.09692597)
\curveto(278.68961242,629.07691651)(278.61461249,629.05691653)(278.54461586,629.03692597)
\lineto(278.33461586,628.97692597)
\curveto(278.04461306,628.84691674)(277.81461329,628.66691692)(277.64461586,628.43692597)
\curveto(277.47461363,628.21691737)(277.33961377,627.95691763)(277.23961586,627.65692597)
\curveto(277.2096139,627.56691802)(277.18461392,627.47191811)(277.16461586,627.37192597)
\curveto(277.15461395,627.2819183)(277.13961397,627.1869184)(277.11961586,627.08692597)
\lineto(277.11961586,626.95192597)
\curveto(277.08961402,626.84191874)(277.07961403,626.70191888)(277.08961586,626.53192597)
\curveto(277.109614,626.37191921)(277.12961398,626.24191934)(277.14961586,626.14192597)
\curveto(277.16961394,626.0819195)(277.18461392,626.02191956)(277.19461586,625.96192597)
\curveto(277.2046139,625.91191967)(277.21961389,625.86191972)(277.23961586,625.81192597)
\curveto(277.31961379,625.61191997)(277.41461369,625.42192016)(277.52461586,625.24192597)
\curveto(277.64461346,625.06192052)(277.78461332,624.91692067)(277.94461586,624.80692597)
\curveto(277.99461311,624.75692083)(278.04961306,624.71692087)(278.10961586,624.68692597)
\curveto(278.16961294,624.65692093)(278.22961288,624.62192096)(278.28961586,624.58192597)
\curveto(278.43961267,624.50192108)(278.62461248,624.43692115)(278.84461586,624.38692597)
\curveto(278.89461221,624.36692122)(278.93461217,624.36192122)(278.96461586,624.37192597)
\curveto(279.0046121,624.3819212)(279.04961206,624.37692121)(279.09961586,624.35692597)
\curveto(279.13961197,624.34692124)(279.19461191,624.34192124)(279.26461586,624.34192597)
\curveto(279.33461177,624.34192124)(279.39461171,624.34692124)(279.44461586,624.35692597)
\curveto(279.54461156,624.37692121)(279.63961147,624.39192119)(279.72961586,624.40192597)
\curveto(279.81961129,624.42192116)(279.9096112,624.45192113)(279.99961586,624.49192597)
\curveto(280.53961057,624.71192087)(280.93461017,625.10692048)(281.18461586,625.67692597)
\curveto(281.23460987,625.77691981)(281.26960984,625.87691971)(281.28961586,625.97692597)
\curveto(281.3096098,626.0869195)(281.33460977,626.19691939)(281.36461586,626.30692597)
\curveto(281.36460974,626.40691918)(281.36960974,626.4819191)(281.37961586,626.53192597)
}
}
{
\newrgbcolor{curcolor}{0 0 0}
\pscustom[linestyle=none,fillstyle=solid,fillcolor=curcolor]
{
\newpath
\moveto(294.09422524,628.15192597)
\curveto(293.89421494,627.86191772)(293.68421515,627.57691801)(293.46422524,627.29692597)
\curveto(293.25421558,627.01691857)(293.04921578,626.73191885)(292.84922524,626.44192597)
\curveto(292.24921658,625.59191999)(291.64421719,624.75192083)(291.03422524,623.92192597)
\curveto(290.42421841,623.10192248)(289.81921901,622.26692332)(289.21922524,621.41692597)
\lineto(288.70922524,620.69692597)
\lineto(288.19922524,620.00692597)
\curveto(288.11922071,619.89692569)(288.03922079,619.7819258)(287.95922524,619.66192597)
\curveto(287.87922095,619.54192604)(287.78422105,619.44692614)(287.67422524,619.37692597)
\curveto(287.6342212,619.35692623)(287.56922126,619.34192624)(287.47922524,619.33192597)
\curveto(287.39922143,619.31192627)(287.30922152,619.30192628)(287.20922524,619.30192597)
\curveto(287.10922172,619.30192628)(287.01422182,619.30692628)(286.92422524,619.31692597)
\curveto(286.84422199,619.32692626)(286.78422205,619.34692624)(286.74422524,619.37692597)
\curveto(286.71422212,619.39692619)(286.68922214,619.43192615)(286.66922524,619.48192597)
\curveto(286.65922217,619.52192606)(286.66422217,619.56692602)(286.68422524,619.61692597)
\curveto(286.72422211,619.69692589)(286.76922206,619.77192581)(286.81922524,619.84192597)
\curveto(286.87922195,619.92192566)(286.9342219,620.00192558)(286.98422524,620.08192597)
\curveto(287.22422161,620.42192516)(287.46922136,620.75692483)(287.71922524,621.08692597)
\curveto(287.96922086,621.41692417)(288.20922062,621.75192383)(288.43922524,622.09192597)
\curveto(288.59922023,622.31192327)(288.75922007,622.52692306)(288.91922524,622.73692597)
\curveto(289.07921975,622.94692264)(289.23921959,623.16192242)(289.39922524,623.38192597)
\curveto(289.75921907,623.90192168)(290.12421871,624.41192117)(290.49422524,624.91192597)
\curveto(290.86421797,625.41192017)(291.2342176,625.92191966)(291.60422524,626.44192597)
\curveto(291.74421709,626.64191894)(291.88421695,626.83691875)(292.02422524,627.02692597)
\curveto(292.17421666,627.21691837)(292.31921651,627.41191817)(292.45922524,627.61192597)
\curveto(292.66921616,627.91191767)(292.88421595,628.21191737)(293.10422524,628.51192597)
\lineto(293.76422524,629.41192597)
\lineto(293.94422524,629.68192597)
\lineto(294.15422524,629.95192597)
\lineto(294.27422524,630.13192597)
\curveto(294.32421451,630.19191539)(294.37421446,630.24691534)(294.42422524,630.29692597)
\curveto(294.49421434,630.34691524)(294.56921426,630.3819152)(294.64922524,630.40192597)
\curveto(294.66921416,630.41191517)(294.69421414,630.41191517)(294.72422524,630.40192597)
\curveto(294.76421407,630.40191518)(294.79421404,630.41191517)(294.81422524,630.43192597)
\curveto(294.9342139,630.43191515)(295.06921376,630.42691516)(295.21922524,630.41692597)
\curveto(295.36921346,630.41691517)(295.45921337,630.37191521)(295.48922524,630.28192597)
\curveto(295.50921332,630.25191533)(295.51421332,630.21691537)(295.50422524,630.17692597)
\curveto(295.49421334,630.13691545)(295.47921335,630.10691548)(295.45922524,630.08692597)
\curveto(295.41921341,630.00691558)(295.37921345,629.93691565)(295.33922524,629.87692597)
\curveto(295.29921353,629.81691577)(295.25421358,629.75691583)(295.20422524,629.69692597)
\lineto(294.63422524,628.91692597)
\curveto(294.45421438,628.66691692)(294.27421456,628.41191717)(294.09422524,628.15192597)
\moveto(287.23922524,624.25192597)
\curveto(287.18922164,624.27192131)(287.13922169,624.27692131)(287.08922524,624.26692597)
\curveto(287.03922179,624.25692133)(286.98922184,624.26192132)(286.93922524,624.28192597)
\curveto(286.829222,624.30192128)(286.72422211,624.32192126)(286.62422524,624.34192597)
\curveto(286.5342223,624.37192121)(286.43922239,624.41192117)(286.33922524,624.46192597)
\curveto(286.00922282,624.60192098)(285.75422308,624.79692079)(285.57422524,625.04692597)
\curveto(285.39422344,625.30692028)(285.24922358,625.61691997)(285.13922524,625.97692597)
\curveto(285.10922372,626.05691953)(285.08922374,626.13691945)(285.07922524,626.21692597)
\curveto(285.06922376,626.30691928)(285.05422378,626.39191919)(285.03422524,626.47192597)
\curveto(285.02422381,626.52191906)(285.01922381,626.586919)(285.01922524,626.66692597)
\curveto(285.00922382,626.69691889)(285.00422383,626.72691886)(285.00422524,626.75692597)
\curveto(285.00422383,626.79691879)(284.99922383,626.83191875)(284.98922524,626.86192597)
\lineto(284.98922524,627.01192597)
\curveto(284.97922385,627.06191852)(284.97422386,627.12191846)(284.97422524,627.19192597)
\curveto(284.97422386,627.27191831)(284.97922385,627.33691825)(284.98922524,627.38692597)
\lineto(284.98922524,627.55192597)
\curveto(285.00922382,627.60191798)(285.01422382,627.64691794)(285.00422524,627.68692597)
\curveto(285.00422383,627.73691785)(285.00922382,627.7819178)(285.01922524,627.82192597)
\curveto(285.0292238,627.86191772)(285.0342238,627.89691769)(285.03422524,627.92692597)
\curveto(285.0342238,627.96691762)(285.03922379,628.00691758)(285.04922524,628.04692597)
\curveto(285.07922375,628.15691743)(285.09922373,628.26691732)(285.10922524,628.37692597)
\curveto(285.1292237,628.49691709)(285.16422367,628.61191697)(285.21422524,628.72192597)
\curveto(285.35422348,629.06191652)(285.51422332,629.33691625)(285.69422524,629.54692597)
\curveto(285.88422295,629.76691582)(286.15422268,629.94691564)(286.50422524,630.08692597)
\curveto(286.58422225,630.11691547)(286.66922216,630.13691545)(286.75922524,630.14692597)
\curveto(286.84922198,630.16691542)(286.94422189,630.1869154)(287.04422524,630.20692597)
\curveto(287.07422176,630.21691537)(287.1292217,630.21691537)(287.20922524,630.20692597)
\curveto(287.28922154,630.20691538)(287.33922149,630.21691537)(287.35922524,630.23692597)
\curveto(287.91922091,630.24691534)(288.36922046,630.13691545)(288.70922524,629.90692597)
\curveto(289.05921977,629.67691591)(289.31921951,629.37191621)(289.48922524,628.99192597)
\curveto(289.5292193,628.90191668)(289.56421927,628.80691678)(289.59422524,628.70692597)
\curveto(289.62421921,628.60691698)(289.64921918,628.50691708)(289.66922524,628.40692597)
\curveto(289.68921914,628.37691721)(289.69421914,628.34691724)(289.68422524,628.31692597)
\curveto(289.68421915,628.2869173)(289.68921914,628.25691733)(289.69922524,628.22692597)
\curveto(289.7292191,628.11691747)(289.74921908,627.99191759)(289.75922524,627.85192597)
\curveto(289.76921906,627.72191786)(289.77921905,627.586918)(289.78922524,627.44692597)
\lineto(289.78922524,627.28192597)
\curveto(289.79921903,627.22191836)(289.79921903,627.16691842)(289.78922524,627.11692597)
\curveto(289.77921905,627.06691852)(289.77421906,627.01691857)(289.77422524,626.96692597)
\lineto(289.77422524,626.83192597)
\curveto(289.76421907,626.79191879)(289.75921907,626.75191883)(289.75922524,626.71192597)
\curveto(289.76921906,626.67191891)(289.76421907,626.62691896)(289.74422524,626.57692597)
\curveto(289.72421911,626.46691912)(289.70421913,626.36191922)(289.68422524,626.26192597)
\curveto(289.67421916,626.16191942)(289.65421918,626.06191952)(289.62422524,625.96192597)
\curveto(289.49421934,625.60191998)(289.3292195,625.2869203)(289.12922524,625.01692597)
\curveto(288.9292199,624.74692084)(288.65422018,624.54192104)(288.30422524,624.40192597)
\curveto(288.22422061,624.37192121)(288.13922069,624.34692124)(288.04922524,624.32692597)
\lineto(287.77922524,624.26692597)
\curveto(287.7292211,624.25692133)(287.68422115,624.25192133)(287.64422524,624.25192597)
\curveto(287.60422123,624.26192132)(287.56422127,624.26192132)(287.52422524,624.25192597)
\curveto(287.42422141,624.23192135)(287.3292215,624.23192135)(287.23922524,624.25192597)
\moveto(286.39922524,625.64692597)
\curveto(286.43922239,625.57692001)(286.47922235,625.51192007)(286.51922524,625.45192597)
\curveto(286.55922227,625.40192018)(286.60922222,625.35192023)(286.66922524,625.30192597)
\lineto(286.81922524,625.18192597)
\curveto(286.87922195,625.15192043)(286.94422189,625.12692046)(287.01422524,625.10692597)
\curveto(287.05422178,625.0869205)(287.08922174,625.07692051)(287.11922524,625.07692597)
\curveto(287.15922167,625.0869205)(287.19922163,625.0819205)(287.23922524,625.06192597)
\curveto(287.26922156,625.06192052)(287.30922152,625.05692053)(287.35922524,625.04692597)
\curveto(287.40922142,625.04692054)(287.44922138,625.05192053)(287.47922524,625.06192597)
\lineto(287.70422524,625.10692597)
\curveto(287.95422088,625.1869204)(288.13922069,625.31192027)(288.25922524,625.48192597)
\curveto(288.33922049,625.58192)(288.40922042,625.71191987)(288.46922524,625.87192597)
\curveto(288.54922028,626.05191953)(288.60922022,626.27691931)(288.64922524,626.54692597)
\curveto(288.68922014,626.82691876)(288.70422013,627.10691848)(288.69422524,627.38692597)
\curveto(288.68422015,627.67691791)(288.65422018,627.95191763)(288.60422524,628.21192597)
\curveto(288.55422028,628.47191711)(288.47922035,628.6819169)(288.37922524,628.84192597)
\curveto(288.25922057,629.04191654)(288.10922072,629.19191639)(287.92922524,629.29192597)
\curveto(287.84922098,629.34191624)(287.75922107,629.37191621)(287.65922524,629.38192597)
\curveto(287.55922127,629.40191618)(287.45422138,629.41191617)(287.34422524,629.41192597)
\curveto(287.32422151,629.40191618)(287.29922153,629.39691619)(287.26922524,629.39692597)
\curveto(287.24922158,629.40691618)(287.2292216,629.40691618)(287.20922524,629.39692597)
\curveto(287.15922167,629.3869162)(287.11422172,629.37691621)(287.07422524,629.36692597)
\curveto(287.0342218,629.36691622)(286.99422184,629.35691623)(286.95422524,629.33692597)
\curveto(286.77422206,629.25691633)(286.62422221,629.13691645)(286.50422524,628.97692597)
\curveto(286.39422244,628.81691677)(286.30422253,628.63691695)(286.23422524,628.43692597)
\curveto(286.17422266,628.24691734)(286.1292227,628.02191756)(286.09922524,627.76192597)
\curveto(286.07922275,627.50191808)(286.07422276,627.23691835)(286.08422524,626.96692597)
\curveto(286.09422274,626.70691888)(286.12422271,626.45691913)(286.17422524,626.21692597)
\curveto(286.2342226,625.9869196)(286.30922252,625.79691979)(286.39922524,625.64692597)
\moveto(297.19922524,622.66192597)
\curveto(297.20921162,622.61192297)(297.21421162,622.52192306)(297.21422524,622.39192597)
\curveto(297.21421162,622.26192332)(297.20421163,622.17192341)(297.18422524,622.12192597)
\curveto(297.16421167,622.07192351)(297.15921167,622.01692357)(297.16922524,621.95692597)
\curveto(297.17921165,621.90692368)(297.17921165,621.85692373)(297.16922524,621.80692597)
\curveto(297.1292117,621.66692392)(297.09921173,621.53192405)(297.07922524,621.40192597)
\curveto(297.06921176,621.27192431)(297.03921179,621.15192443)(296.98922524,621.04192597)
\curveto(296.84921198,620.69192489)(296.68421215,620.39692519)(296.49422524,620.15692597)
\curveto(296.30421253,619.92692566)(296.0342128,619.74192584)(295.68422524,619.60192597)
\curveto(295.60421323,619.57192601)(295.51921331,619.55192603)(295.42922524,619.54192597)
\curveto(295.33921349,619.52192606)(295.25421358,619.50192608)(295.17422524,619.48192597)
\curveto(295.12421371,619.47192611)(295.07421376,619.46692612)(295.02422524,619.46692597)
\curveto(294.97421386,619.46692612)(294.92421391,619.46192612)(294.87422524,619.45192597)
\curveto(294.84421399,619.44192614)(294.79421404,619.44192614)(294.72422524,619.45192597)
\curveto(294.65421418,619.45192613)(294.60421423,619.45692613)(294.57422524,619.46692597)
\curveto(294.51421432,619.4869261)(294.45421438,619.49692609)(294.39422524,619.49692597)
\curveto(294.34421449,619.4869261)(294.29421454,619.49192609)(294.24422524,619.51192597)
\curveto(294.15421468,619.53192605)(294.06421477,619.55692603)(293.97422524,619.58692597)
\curveto(293.89421494,619.60692598)(293.81421502,619.63692595)(293.73422524,619.67692597)
\curveto(293.41421542,619.81692577)(293.16421567,620.01192557)(292.98422524,620.26192597)
\curveto(292.80421603,620.52192506)(292.65421618,620.82692476)(292.53422524,621.17692597)
\curveto(292.51421632,621.25692433)(292.49921633,621.34192424)(292.48922524,621.43192597)
\curveto(292.47921635,621.52192406)(292.46421637,621.60692398)(292.44422524,621.68692597)
\curveto(292.4342164,621.71692387)(292.4292164,621.74692384)(292.42922524,621.77692597)
\lineto(292.42922524,621.88192597)
\curveto(292.40921642,621.96192362)(292.39921643,622.04192354)(292.39922524,622.12192597)
\lineto(292.39922524,622.25692597)
\curveto(292.37921645,622.35692323)(292.37921645,622.45692313)(292.39922524,622.55692597)
\lineto(292.39922524,622.73692597)
\curveto(292.40921642,622.7869228)(292.41421642,622.83192275)(292.41422524,622.87192597)
\curveto(292.41421642,622.92192266)(292.41921641,622.96692262)(292.42922524,623.00692597)
\curveto(292.43921639,623.04692254)(292.44421639,623.0819225)(292.44422524,623.11192597)
\curveto(292.44421639,623.15192243)(292.44921638,623.19192239)(292.45922524,623.23192597)
\lineto(292.51922524,623.56192597)
\curveto(292.53921629,623.6819219)(292.56921626,623.79192179)(292.60922524,623.89192597)
\curveto(292.74921608,624.22192136)(292.90921592,624.49692109)(293.08922524,624.71692597)
\curveto(293.27921555,624.94692064)(293.53921529,625.13192045)(293.86922524,625.27192597)
\curveto(293.94921488,625.31192027)(294.0342148,625.33692025)(294.12422524,625.34692597)
\lineto(294.42422524,625.40692597)
\lineto(294.55922524,625.40692597)
\curveto(294.60921422,625.41692017)(294.65921417,625.42192016)(294.70922524,625.42192597)
\curveto(295.27921355,625.44192014)(295.73921309,625.33692025)(296.08922524,625.10692597)
\curveto(296.44921238,624.8869207)(296.71421212,624.586921)(296.88422524,624.20692597)
\curveto(296.9342119,624.10692148)(296.97421186,624.00692158)(297.00422524,623.90692597)
\curveto(297.0342118,623.80692178)(297.06421177,623.70192188)(297.09422524,623.59192597)
\curveto(297.10421173,623.55192203)(297.10921172,623.51692207)(297.10922524,623.48692597)
\curveto(297.10921172,623.46692212)(297.11421172,623.43692215)(297.12422524,623.39692597)
\curveto(297.14421169,623.32692226)(297.15421168,623.25192233)(297.15422524,623.17192597)
\curveto(297.15421168,623.09192249)(297.16421167,623.01192257)(297.18422524,622.93192597)
\curveto(297.18421165,622.8819227)(297.18421165,622.83692275)(297.18422524,622.79692597)
\curveto(297.18421165,622.75692283)(297.18921164,622.71192287)(297.19922524,622.66192597)
\moveto(296.08922524,622.22692597)
\curveto(296.09921273,622.27692331)(296.10421273,622.35192323)(296.10422524,622.45192597)
\curveto(296.11421272,622.55192303)(296.10921272,622.62692296)(296.08922524,622.67692597)
\curveto(296.06921276,622.73692285)(296.06421277,622.79192279)(296.07422524,622.84192597)
\curveto(296.09421274,622.90192268)(296.09421274,622.96192262)(296.07422524,623.02192597)
\curveto(296.06421277,623.05192253)(296.05921277,623.0869225)(296.05922524,623.12692597)
\curveto(296.05921277,623.16692242)(296.05421278,623.20692238)(296.04422524,623.24692597)
\curveto(296.02421281,623.32692226)(296.00421283,623.40192218)(295.98422524,623.47192597)
\curveto(295.97421286,623.55192203)(295.95921287,623.63192195)(295.93922524,623.71192597)
\curveto(295.90921292,623.77192181)(295.88421295,623.83192175)(295.86422524,623.89192597)
\curveto(295.84421299,623.95192163)(295.81421302,624.01192157)(295.77422524,624.07192597)
\curveto(295.67421316,624.24192134)(295.54421329,624.37692121)(295.38422524,624.47692597)
\curveto(295.30421353,624.52692106)(295.20921362,624.56192102)(295.09922524,624.58192597)
\curveto(294.98921384,624.60192098)(294.86421397,624.61192097)(294.72422524,624.61192597)
\curveto(294.70421413,624.60192098)(294.67921415,624.59692099)(294.64922524,624.59692597)
\curveto(294.61921421,624.60692098)(294.58921424,624.60692098)(294.55922524,624.59692597)
\lineto(294.40922524,624.53692597)
\curveto(294.35921447,624.52692106)(294.31421452,624.51192107)(294.27422524,624.49192597)
\curveto(294.08421475,624.3819212)(293.93921489,624.23692135)(293.83922524,624.05692597)
\curveto(293.74921508,623.87692171)(293.66921516,623.67192191)(293.59922524,623.44192597)
\curveto(293.55921527,623.31192227)(293.53921529,623.17692241)(293.53922524,623.03692597)
\curveto(293.53921529,622.90692268)(293.5292153,622.76192282)(293.50922524,622.60192597)
\curveto(293.49921533,622.55192303)(293.48921534,622.49192309)(293.47922524,622.42192597)
\curveto(293.47921535,622.35192323)(293.48921534,622.29192329)(293.50922524,622.24192597)
\lineto(293.50922524,622.07692597)
\lineto(293.50922524,621.89692597)
\curveto(293.51921531,621.84692374)(293.5292153,621.79192379)(293.53922524,621.73192597)
\curveto(293.54921528,621.6819239)(293.55421528,621.62692396)(293.55422524,621.56692597)
\curveto(293.56421527,621.50692408)(293.57921525,621.45192413)(293.59922524,621.40192597)
\curveto(293.64921518,621.21192437)(293.70921512,621.03692455)(293.77922524,620.87692597)
\curveto(293.84921498,620.71692487)(293.95421488,620.586925)(294.09422524,620.48692597)
\curveto(294.22421461,620.3869252)(294.36421447,620.31692527)(294.51422524,620.27692597)
\curveto(294.54421429,620.26692532)(294.56921426,620.26192532)(294.58922524,620.26192597)
\curveto(294.61921421,620.27192531)(294.64921418,620.27192531)(294.67922524,620.26192597)
\curveto(294.69921413,620.26192532)(294.7292141,620.25692533)(294.76922524,620.24692597)
\curveto(294.80921402,620.24692534)(294.84421399,620.25192533)(294.87422524,620.26192597)
\curveto(294.91421392,620.27192531)(294.95421388,620.27692531)(294.99422524,620.27692597)
\curveto(295.0342138,620.27692531)(295.07421376,620.2869253)(295.11422524,620.30692597)
\curveto(295.35421348,620.3869252)(295.54921328,620.52192506)(295.69922524,620.71192597)
\curveto(295.81921301,620.89192469)(295.90921292,621.09692449)(295.96922524,621.32692597)
\curveto(295.98921284,621.39692419)(296.00421283,621.46692412)(296.01422524,621.53692597)
\curveto(296.02421281,621.61692397)(296.03921279,621.69692389)(296.05922524,621.77692597)
\curveto(296.05921277,621.83692375)(296.06421277,621.8819237)(296.07422524,621.91192597)
\curveto(296.07421276,621.93192365)(296.07421276,621.95692363)(296.07422524,621.98692597)
\curveto(296.07421276,622.02692356)(296.07921275,622.05692353)(296.08922524,622.07692597)
\lineto(296.08922524,622.22692597)
}
}
{
\newrgbcolor{curcolor}{0 0 0}
\pscustom[linestyle=none,fillstyle=solid,fillcolor=curcolor]
{
\newpath
\moveto(530.66666725,625.5641037)
\curveto(530.67665953,625.52410065)(530.67665953,625.4741007)(530.66666725,625.4141037)
\curveto(530.66665954,625.35410082)(530.66165955,625.30410087)(530.65166725,625.2641037)
\curveto(530.65165956,625.22410095)(530.64665956,625.18410099)(530.63666725,625.1441037)
\lineto(530.63666725,625.0391037)
\curveto(530.61665959,624.95910122)(530.60165961,624.8791013)(530.59166725,624.7991037)
\curveto(530.58165963,624.71910146)(530.56165965,624.64410153)(530.53166725,624.5741037)
\curveto(530.5116597,624.49410168)(530.49165972,624.41910176)(530.47166725,624.3491037)
\curveto(530.45165976,624.2791019)(530.42165979,624.20410197)(530.38166725,624.1241037)
\curveto(530.20166001,623.70410247)(529.94666026,623.36410281)(529.61666725,623.1041037)
\curveto(529.28666092,622.84410333)(528.89666131,622.63910354)(528.44666725,622.4891037)
\curveto(528.32666188,622.44910373)(528.20166201,622.42410375)(528.07166725,622.4141037)
\curveto(527.95166226,622.39410378)(527.82666238,622.36910381)(527.69666725,622.3391037)
\curveto(527.63666257,622.32910385)(527.57166264,622.32410385)(527.50166725,622.3241037)
\curveto(527.44166277,622.32410385)(527.37666283,622.31910386)(527.30666725,622.3091037)
\lineto(527.18666725,622.3091037)
\lineto(526.99166725,622.3091037)
\curveto(526.93166328,622.29910388)(526.87666333,622.30410387)(526.82666725,622.3241037)
\curveto(526.75666345,622.34410383)(526.69166352,622.34910383)(526.63166725,622.3391037)
\curveto(526.57166364,622.32910385)(526.5116637,622.33410384)(526.45166725,622.3541037)
\curveto(526.40166381,622.36410381)(526.35666385,622.36910381)(526.31666725,622.3691037)
\curveto(526.27666393,622.36910381)(526.23166398,622.3791038)(526.18166725,622.3991037)
\curveto(526.10166411,622.41910376)(526.02666418,622.43910374)(525.95666725,622.4591037)
\curveto(525.88666432,622.46910371)(525.81666439,622.48410369)(525.74666725,622.5041037)
\curveto(525.26666494,622.6741035)(524.86666534,622.88410329)(524.54666725,623.1341037)
\curveto(524.23666597,623.39410278)(523.98666622,623.74910243)(523.79666725,624.1991037)
\curveto(523.76666644,624.25910192)(523.74166647,624.31910186)(523.72166725,624.3791037)
\curveto(523.7116665,624.44910173)(523.69666651,624.52410165)(523.67666725,624.6041037)
\curveto(523.65666655,624.66410151)(523.64166657,624.72910145)(523.63166725,624.7991037)
\curveto(523.62166659,624.86910131)(523.6066666,624.93910124)(523.58666725,625.0091037)
\curveto(523.57666663,625.05910112)(523.57166664,625.09910108)(523.57166725,625.1291037)
\lineto(523.57166725,625.2491037)
\curveto(523.56166665,625.28910089)(523.55166666,625.33910084)(523.54166725,625.3991037)
\curveto(523.54166667,625.45910072)(523.54666666,625.50910067)(523.55666725,625.5491037)
\lineto(523.55666725,625.6841037)
\curveto(523.56666664,625.73410044)(523.57166664,625.78410039)(523.57166725,625.8341037)
\curveto(523.59166662,625.93410024)(523.6066666,626.02910015)(523.61666725,626.1191037)
\curveto(523.62666658,626.21909996)(523.64666656,626.31409986)(523.67666725,626.4041037)
\curveto(523.72666648,626.55409962)(523.78166643,626.69409948)(523.84166725,626.8241037)
\curveto(523.90166631,626.95409922)(523.97166624,627.0740991)(524.05166725,627.1841037)
\curveto(524.08166613,627.23409894)(524.1116661,627.2740989)(524.14166725,627.3041037)
\curveto(524.18166603,627.33409884)(524.21666599,627.36909881)(524.24666725,627.4091037)
\curveto(524.3066659,627.48909869)(524.37666583,627.55909862)(524.45666725,627.6191037)
\curveto(524.51666569,627.66909851)(524.57666563,627.71409846)(524.63666725,627.7541037)
\lineto(524.84666725,627.9041037)
\curveto(524.89666531,627.94409823)(524.94666526,627.9790982)(524.99666725,628.0091037)
\curveto(525.04666516,628.04909813)(525.08166513,628.10409807)(525.10166725,628.1741037)
\curveto(525.10166511,628.20409797)(525.09166512,628.22909795)(525.07166725,628.2491037)
\curveto(525.06166515,628.2790979)(525.05166516,628.30409787)(525.04166725,628.3241037)
\curveto(525.00166521,628.3740978)(524.95166526,628.41909776)(524.89166725,628.4591037)
\curveto(524.84166537,628.50909767)(524.79166542,628.55409762)(524.74166725,628.5941037)
\curveto(524.70166551,628.62409755)(524.65166556,628.6790975)(524.59166725,628.7591037)
\curveto(524.57166564,628.78909739)(524.54166567,628.81409736)(524.50166725,628.8341037)
\curveto(524.47166574,628.86409731)(524.44666576,628.89909728)(524.42666725,628.9391037)
\curveto(524.25666595,629.14909703)(524.12666608,629.39409678)(524.03666725,629.6741037)
\curveto(524.01666619,629.75409642)(524.00166621,629.83409634)(523.99166725,629.9141037)
\curveto(523.98166623,629.99409618)(523.96666624,630.0740961)(523.94666725,630.1541037)
\curveto(523.92666628,630.20409597)(523.91666629,630.26909591)(523.91666725,630.3491037)
\curveto(523.91666629,630.43909574)(523.92666628,630.50909567)(523.94666725,630.5591037)
\curveto(523.94666626,630.65909552)(523.95166626,630.72909545)(523.96166725,630.7691037)
\curveto(523.98166623,630.84909533)(523.99666621,630.91909526)(524.00666725,630.9791037)
\curveto(524.01666619,631.04909513)(524.03166618,631.11909506)(524.05166725,631.1891037)
\curveto(524.20166601,631.61909456)(524.41666579,631.96409421)(524.69666725,632.2241037)
\curveto(524.98666522,632.48409369)(525.33666487,632.69909348)(525.74666725,632.8691037)
\curveto(525.85666435,632.91909326)(525.97166424,632.94909323)(526.09166725,632.9591037)
\curveto(526.22166399,632.9790932)(526.35166386,633.00909317)(526.48166725,633.0491037)
\curveto(526.56166365,633.04909313)(526.63166358,633.04909313)(526.69166725,633.0491037)
\curveto(526.76166345,633.05909312)(526.83666337,633.06909311)(526.91666725,633.0791037)
\curveto(527.7066625,633.09909308)(528.36166185,632.96909321)(528.88166725,632.6891037)
\curveto(529.4116608,632.40909377)(529.79166042,631.99909418)(530.02166725,631.4591037)
\curveto(530.13166008,631.22909495)(530.20166001,630.94409523)(530.23166725,630.6041037)
\curveto(530.27165994,630.2740959)(530.24165997,629.96909621)(530.14166725,629.6891037)
\curveto(530.10166011,629.55909662)(530.05166016,629.43909674)(529.99166725,629.3291037)
\curveto(529.94166027,629.21909696)(529.88166033,629.11409706)(529.81166725,629.0141037)
\curveto(529.79166042,628.9740972)(529.76166045,628.93909724)(529.72166725,628.9091037)
\lineto(529.63166725,628.8191037)
\curveto(529.58166063,628.72909745)(529.52166069,628.66409751)(529.45166725,628.6241037)
\curveto(529.40166081,628.5740976)(529.34666086,628.52409765)(529.28666725,628.4741037)
\curveto(529.23666097,628.43409774)(529.19166102,628.38909779)(529.15166725,628.3391037)
\curveto(529.13166108,628.31909786)(529.1116611,628.29409788)(529.09166725,628.2641037)
\curveto(529.08166113,628.24409793)(529.08166113,628.21909796)(529.09166725,628.1891037)
\curveto(529.10166111,628.13909804)(529.13166108,628.08909809)(529.18166725,628.0391037)
\curveto(529.23166098,627.99909818)(529.28666092,627.95909822)(529.34666725,627.9191037)
\lineto(529.52666725,627.7991037)
\curveto(529.58666062,627.76909841)(529.63666057,627.73909844)(529.67666725,627.7091037)
\curveto(530.0066602,627.46909871)(530.25665995,627.15909902)(530.42666725,626.7791037)
\curveto(530.46665974,626.69909948)(530.49665971,626.61409956)(530.51666725,626.5241037)
\curveto(530.54665966,626.43409974)(530.57165964,626.34409983)(530.59166725,626.2541037)
\curveto(530.60165961,626.20409997)(530.6116596,626.14910003)(530.62166725,626.0891037)
\lineto(530.65166725,625.9391037)
\curveto(530.66165955,625.8791003)(530.66165955,625.81410036)(530.65166725,625.7441037)
\curveto(530.64165957,625.68410049)(530.64665956,625.62410055)(530.66666725,625.5641037)
\moveto(525.28166725,630.6041037)
\curveto(525.25166496,630.49409568)(525.24666496,630.35409582)(525.26666725,630.1841037)
\curveto(525.28666492,630.02409615)(525.3116649,629.89909628)(525.34166725,629.8091037)
\curveto(525.45166476,629.48909669)(525.60166461,629.24409693)(525.79166725,629.0741037)
\curveto(525.98166423,628.91409726)(526.24666396,628.78409739)(526.58666725,628.6841037)
\curveto(526.71666349,628.65409752)(526.88166333,628.62909755)(527.08166725,628.6091037)
\curveto(527.28166293,628.59909758)(527.45166276,628.61409756)(527.59166725,628.6541037)
\curveto(527.88166233,628.73409744)(528.12166209,628.84409733)(528.31166725,628.9841037)
\curveto(528.5116617,629.13409704)(528.66666154,629.33409684)(528.77666725,629.5841037)
\curveto(528.79666141,629.63409654)(528.8066614,629.6790965)(528.80666725,629.7191037)
\curveto(528.81666139,629.75909642)(528.83166138,629.80409637)(528.85166725,629.8541037)
\curveto(528.88166133,629.96409621)(528.90166131,630.10409607)(528.91166725,630.2741037)
\curveto(528.92166129,630.44409573)(528.9116613,630.58909559)(528.88166725,630.7091037)
\curveto(528.86166135,630.79909538)(528.83666137,630.88409529)(528.80666725,630.9641037)
\curveto(528.78666142,631.04409513)(528.75166146,631.12409505)(528.70166725,631.2041037)
\curveto(528.53166168,631.4740947)(528.3066619,631.66909451)(528.02666725,631.7891037)
\curveto(527.75666245,631.90909427)(527.39666281,631.96909421)(526.94666725,631.9691037)
\curveto(526.92666328,631.94909423)(526.89666331,631.94409423)(526.85666725,631.9541037)
\curveto(526.81666339,631.96409421)(526.78166343,631.96409421)(526.75166725,631.9541037)
\curveto(526.70166351,631.93409424)(526.64666356,631.91909426)(526.58666725,631.9091037)
\curveto(526.53666367,631.90909427)(526.48666372,631.89909428)(526.43666725,631.8791037)
\curveto(526.19666401,631.78909439)(525.98666422,631.6740945)(525.80666725,631.5341037)
\curveto(525.62666458,631.40409477)(525.48666472,631.22409495)(525.38666725,630.9941037)
\curveto(525.36666484,630.93409524)(525.34666486,630.86909531)(525.32666725,630.7991037)
\curveto(525.31666489,630.73909544)(525.30166491,630.6740955)(525.28166725,630.6041037)
\moveto(529.30166725,625.0691037)
\curveto(529.35166086,625.25910092)(529.35666085,625.46410071)(529.31666725,625.6841037)
\curveto(529.28666092,625.90410027)(529.24166097,626.08410009)(529.18166725,626.2241037)
\curveto(529.0116612,626.59409958)(528.75166146,626.89909928)(528.40166725,627.1391037)
\curveto(528.06166215,627.3790988)(527.62666258,627.49909868)(527.09666725,627.4991037)
\curveto(527.06666314,627.4790987)(527.02666318,627.4740987)(526.97666725,627.4841037)
\curveto(526.92666328,627.50409867)(526.88666332,627.50909867)(526.85666725,627.4991037)
\lineto(526.58666725,627.4391037)
\curveto(526.5066637,627.42909875)(526.42666378,627.41409876)(526.34666725,627.3941037)
\curveto(526.04666416,627.28409889)(525.78166443,627.13909904)(525.55166725,626.9591037)
\curveto(525.33166488,626.7790994)(525.16166505,626.54909963)(525.04166725,626.2691037)
\curveto(525.0116652,626.18909999)(524.98666522,626.10910007)(524.96666725,626.0291037)
\curveto(524.94666526,625.94910023)(524.92666528,625.86410031)(524.90666725,625.7741037)
\curveto(524.87666533,625.65410052)(524.86666534,625.50410067)(524.87666725,625.3241037)
\curveto(524.89666531,625.14410103)(524.92166529,625.00410117)(524.95166725,624.9041037)
\curveto(524.97166524,624.85410132)(524.98166523,624.80910137)(524.98166725,624.7691037)
\curveto(524.99166522,624.73910144)(525.0066652,624.69910148)(525.02666725,624.6491037)
\curveto(525.12666508,624.42910175)(525.25666495,624.22910195)(525.41666725,624.0491037)
\curveto(525.58666462,623.86910231)(525.78166443,623.73410244)(526.00166725,623.6441037)
\curveto(526.07166414,623.60410257)(526.16666404,623.56910261)(526.28666725,623.5391037)
\curveto(526.5066637,623.44910273)(526.76166345,623.40410277)(527.05166725,623.4041037)
\lineto(527.33666725,623.4041037)
\curveto(527.43666277,623.42410275)(527.53166268,623.43910274)(527.62166725,623.4491037)
\curveto(527.7116625,623.45910272)(527.80166241,623.4791027)(527.89166725,623.5091037)
\curveto(528.15166206,623.58910259)(528.39166182,623.71910246)(528.61166725,623.8991037)
\curveto(528.84166137,624.08910209)(529.0116612,624.30410187)(529.12166725,624.5441037)
\curveto(529.16166105,624.62410155)(529.19166102,624.70410147)(529.21166725,624.7841037)
\curveto(529.24166097,624.8741013)(529.27166094,624.96910121)(529.30166725,625.0691037)
}
}
{
\newrgbcolor{curcolor}{0 0 0}
\pscustom[linestyle=none,fillstyle=solid,fillcolor=curcolor]
{
\newpath
\moveto(532.95627663,624.1241037)
\lineto(533.25627663,624.1241037)
\curveto(533.36627457,624.13410204)(533.47127446,624.13410204)(533.57127663,624.1241037)
\curveto(533.68127425,624.12410205)(533.78127415,624.11410206)(533.87127663,624.0941037)
\curveto(533.96127397,624.08410209)(534.0312739,624.05910212)(534.08127663,624.0191037)
\curveto(534.10127383,623.99910218)(534.11627382,623.96910221)(534.12627663,623.9291037)
\curveto(534.14627379,623.88910229)(534.16627377,623.84410233)(534.18627663,623.7941037)
\lineto(534.18627663,623.7191037)
\curveto(534.19627374,623.66910251)(534.19627374,623.61410256)(534.18627663,623.5541037)
\lineto(534.18627663,623.4041037)
\lineto(534.18627663,622.9241037)
\curveto(534.18627375,622.75410342)(534.14627379,622.63410354)(534.06627663,622.5641037)
\curveto(533.99627394,622.51410366)(533.90627403,622.48910369)(533.79627663,622.4891037)
\lineto(533.46627663,622.4891037)
\lineto(533.01627663,622.4891037)
\curveto(532.86627507,622.48910369)(532.75127518,622.51910366)(532.67127663,622.5791037)
\curveto(532.6312753,622.60910357)(532.60127533,622.65910352)(532.58127663,622.7291037)
\curveto(532.56127537,622.80910337)(532.54627539,622.89410328)(532.53627663,622.9841037)
\lineto(532.53627663,623.2691037)
\curveto(532.54627539,623.36910281)(532.55127538,623.45410272)(532.55127663,623.5241037)
\lineto(532.55127663,623.7191037)
\curveto(532.55127538,623.7791024)(532.56127537,623.83410234)(532.58127663,623.8841037)
\curveto(532.62127531,623.99410218)(532.69127524,624.06410211)(532.79127663,624.0941037)
\curveto(532.82127511,624.09410208)(532.87627506,624.10410207)(532.95627663,624.1241037)
}
}
{
\newrgbcolor{curcolor}{0 0 0}
\pscustom[linestyle=none,fillstyle=solid,fillcolor=curcolor]
{
\newpath
\moveto(540.22143288,633.0941037)
\curveto(540.32142802,633.09409308)(540.41642793,633.08409309)(540.50643288,633.0641037)
\curveto(540.59642775,633.05409312)(540.66142768,633.02409315)(540.70143288,632.9741037)
\curveto(540.76142758,632.89409328)(540.79142755,632.78909339)(540.79143288,632.6591037)
\lineto(540.79143288,632.2691037)
\lineto(540.79143288,630.7691037)
\lineto(540.79143288,624.3791037)
\lineto(540.79143288,623.2091037)
\lineto(540.79143288,622.8941037)
\curveto(540.80142754,622.79410338)(540.78642756,622.71410346)(540.74643288,622.6541037)
\curveto(540.69642765,622.5741036)(540.62142772,622.52410365)(540.52143288,622.5041037)
\curveto(540.43142791,622.49410368)(540.32142802,622.48910369)(540.19143288,622.4891037)
\lineto(539.96643288,622.4891037)
\curveto(539.88642846,622.50910367)(539.81642853,622.52410365)(539.75643288,622.5341037)
\curveto(539.69642865,622.55410362)(539.6464287,622.59410358)(539.60643288,622.6541037)
\curveto(539.56642878,622.71410346)(539.5464288,622.78910339)(539.54643288,622.8791037)
\lineto(539.54643288,623.1791037)
\lineto(539.54643288,624.2741037)
\lineto(539.54643288,629.6141037)
\curveto(539.52642882,629.70409647)(539.51142883,629.7790964)(539.50143288,629.8391037)
\curveto(539.50142884,629.90909627)(539.47142887,629.96909621)(539.41143288,630.0191037)
\curveto(539.341429,630.06909611)(539.25142909,630.09409608)(539.14143288,630.0941037)
\curveto(539.0414293,630.10409607)(538.93142941,630.10909607)(538.81143288,630.1091037)
\lineto(537.67143288,630.1091037)
\lineto(537.17643288,630.1091037)
\curveto(537.01643133,630.11909606)(536.90643144,630.179096)(536.84643288,630.2891037)
\curveto(536.82643152,630.31909586)(536.81643153,630.34909583)(536.81643288,630.3791037)
\curveto(536.81643153,630.41909576)(536.81143153,630.46409571)(536.80143288,630.5141037)
\curveto(536.78143156,630.63409554)(536.78643156,630.74409543)(536.81643288,630.8441037)
\curveto(536.85643149,630.94409523)(536.91143143,631.01409516)(536.98143288,631.0541037)
\curveto(537.06143128,631.10409507)(537.18143116,631.12909505)(537.34143288,631.1291037)
\curveto(537.50143084,631.12909505)(537.63643071,631.14409503)(537.74643288,631.1741037)
\curveto(537.79643055,631.18409499)(537.85143049,631.18909499)(537.91143288,631.1891037)
\curveto(537.97143037,631.19909498)(538.03143031,631.21409496)(538.09143288,631.2341037)
\curveto(538.2414301,631.28409489)(538.38642996,631.33409484)(538.52643288,631.3841037)
\curveto(538.66642968,631.44409473)(538.80142954,631.51409466)(538.93143288,631.5941037)
\curveto(539.07142927,631.68409449)(539.19142915,631.78909439)(539.29143288,631.9091037)
\curveto(539.39142895,632.02909415)(539.48642886,632.15909402)(539.57643288,632.2991037)
\curveto(539.63642871,632.39909378)(539.68142866,632.50909367)(539.71143288,632.6291037)
\curveto(539.75142859,632.74909343)(539.80142854,632.85409332)(539.86143288,632.9441037)
\curveto(539.91142843,633.00409317)(539.98142836,633.04409313)(540.07143288,633.0641037)
\curveto(540.09142825,633.0740931)(540.11642823,633.0790931)(540.14643288,633.0791037)
\curveto(540.17642817,633.0790931)(540.20142814,633.08409309)(540.22143288,633.0941037)
}
}
{
\newrgbcolor{curcolor}{0 0 0}
\pscustom[linestyle=none,fillstyle=solid,fillcolor=curcolor]
{
\newpath
\moveto(554.31604225,631.0091037)
\curveto(554.11603195,630.71909546)(553.90603216,630.43409574)(553.68604225,630.1541037)
\curveto(553.47603259,629.8740963)(553.2710328,629.58909659)(553.07104225,629.2991037)
\curveto(552.4710336,628.44909773)(551.8660342,627.60909857)(551.25604225,626.7791037)
\curveto(550.64603542,625.95910022)(550.04103603,625.12410105)(549.44104225,624.2741037)
\lineto(548.93104225,623.5541037)
\lineto(548.42104225,622.8641037)
\curveto(548.34103773,622.75410342)(548.26103781,622.63910354)(548.18104225,622.5191037)
\curveto(548.10103797,622.39910378)(548.00603806,622.30410387)(547.89604225,622.2341037)
\curveto(547.85603821,622.21410396)(547.79103828,622.19910398)(547.70104225,622.1891037)
\curveto(547.62103845,622.16910401)(547.53103854,622.15910402)(547.43104225,622.1591037)
\curveto(547.33103874,622.15910402)(547.23603883,622.16410401)(547.14604225,622.1741037)
\curveto(547.066039,622.18410399)(547.00603906,622.20410397)(546.96604225,622.2341037)
\curveto(546.93603913,622.25410392)(546.91103916,622.28910389)(546.89104225,622.3391037)
\curveto(546.88103919,622.3791038)(546.88603918,622.42410375)(546.90604225,622.4741037)
\curveto(546.94603912,622.55410362)(546.99103908,622.62910355)(547.04104225,622.6991037)
\curveto(547.10103897,622.7791034)(547.15603891,622.85910332)(547.20604225,622.9391037)
\curveto(547.44603862,623.2791029)(547.69103838,623.61410256)(547.94104225,623.9441037)
\curveto(548.19103788,624.2741019)(548.43103764,624.60910157)(548.66104225,624.9491037)
\curveto(548.82103725,625.16910101)(548.98103709,625.38410079)(549.14104225,625.5941037)
\curveto(549.30103677,625.80410037)(549.46103661,626.01910016)(549.62104225,626.2391037)
\curveto(549.98103609,626.75909942)(550.34603572,627.26909891)(550.71604225,627.7691037)
\curveto(551.08603498,628.26909791)(551.45603461,628.7790974)(551.82604225,629.2991037)
\curveto(551.9660341,629.49909668)(552.10603396,629.69409648)(552.24604225,629.8841037)
\curveto(552.39603367,630.0740961)(552.54103353,630.26909591)(552.68104225,630.4691037)
\curveto(552.89103318,630.76909541)(553.10603296,631.06909511)(553.32604225,631.3691037)
\lineto(553.98604225,632.2691037)
\lineto(554.16604225,632.5391037)
\lineto(554.37604225,632.8091037)
\lineto(554.49604225,632.9891037)
\curveto(554.54603152,633.04909313)(554.59603147,633.10409307)(554.64604225,633.1541037)
\curveto(554.71603135,633.20409297)(554.79103128,633.23909294)(554.87104225,633.2591037)
\curveto(554.89103118,633.26909291)(554.91603115,633.26909291)(554.94604225,633.2591037)
\curveto(554.98603108,633.25909292)(555.01603105,633.26909291)(555.03604225,633.2891037)
\curveto(555.15603091,633.28909289)(555.29103078,633.28409289)(555.44104225,633.2741037)
\curveto(555.59103048,633.2740929)(555.68103039,633.22909295)(555.71104225,633.1391037)
\curveto(555.73103034,633.10909307)(555.73603033,633.0740931)(555.72604225,633.0341037)
\curveto(555.71603035,632.99409318)(555.70103037,632.96409321)(555.68104225,632.9441037)
\curveto(555.64103043,632.86409331)(555.60103047,632.79409338)(555.56104225,632.7341037)
\curveto(555.52103055,632.6740935)(555.47603059,632.61409356)(555.42604225,632.5541037)
\lineto(554.85604225,631.7741037)
\curveto(554.67603139,631.52409465)(554.49603157,631.26909491)(554.31604225,631.0091037)
\moveto(547.46104225,627.1091037)
\curveto(547.41103866,627.12909905)(547.36103871,627.13409904)(547.31104225,627.1241037)
\curveto(547.26103881,627.11409906)(547.21103886,627.11909906)(547.16104225,627.1391037)
\curveto(547.05103902,627.15909902)(546.94603912,627.179099)(546.84604225,627.1991037)
\curveto(546.75603931,627.22909895)(546.66103941,627.26909891)(546.56104225,627.3191037)
\curveto(546.23103984,627.45909872)(545.97604009,627.65409852)(545.79604225,627.9041037)
\curveto(545.61604045,628.16409801)(545.4710406,628.4740977)(545.36104225,628.8341037)
\curveto(545.33104074,628.91409726)(545.31104076,628.99409718)(545.30104225,629.0741037)
\curveto(545.29104078,629.16409701)(545.27604079,629.24909693)(545.25604225,629.3291037)
\curveto(545.24604082,629.3790968)(545.24104083,629.44409673)(545.24104225,629.5241037)
\curveto(545.23104084,629.55409662)(545.22604084,629.58409659)(545.22604225,629.6141037)
\curveto(545.22604084,629.65409652)(545.22104085,629.68909649)(545.21104225,629.7191037)
\lineto(545.21104225,629.8691037)
\curveto(545.20104087,629.91909626)(545.19604087,629.9790962)(545.19604225,630.0491037)
\curveto(545.19604087,630.12909605)(545.20104087,630.19409598)(545.21104225,630.2441037)
\lineto(545.21104225,630.4091037)
\curveto(545.23104084,630.45909572)(545.23604083,630.50409567)(545.22604225,630.5441037)
\curveto(545.22604084,630.59409558)(545.23104084,630.63909554)(545.24104225,630.6791037)
\curveto(545.25104082,630.71909546)(545.25604081,630.75409542)(545.25604225,630.7841037)
\curveto(545.25604081,630.82409535)(545.26104081,630.86409531)(545.27104225,630.9041037)
\curveto(545.30104077,631.01409516)(545.32104075,631.12409505)(545.33104225,631.2341037)
\curveto(545.35104072,631.35409482)(545.38604068,631.46909471)(545.43604225,631.5791037)
\curveto(545.57604049,631.91909426)(545.73604033,632.19409398)(545.91604225,632.4041037)
\curveto(546.10603996,632.62409355)(546.37603969,632.80409337)(546.72604225,632.9441037)
\curveto(546.80603926,632.9740932)(546.89103918,632.99409318)(546.98104225,633.0041037)
\curveto(547.071039,633.02409315)(547.1660389,633.04409313)(547.26604225,633.0641037)
\curveto(547.29603877,633.0740931)(547.35103872,633.0740931)(547.43104225,633.0641037)
\curveto(547.51103856,633.06409311)(547.56103851,633.0740931)(547.58104225,633.0941037)
\curveto(548.14103793,633.10409307)(548.59103748,632.99409318)(548.93104225,632.7641037)
\curveto(549.28103679,632.53409364)(549.54103653,632.22909395)(549.71104225,631.8491037)
\curveto(549.75103632,631.75909442)(549.78603628,631.66409451)(549.81604225,631.5641037)
\curveto(549.84603622,631.46409471)(549.8710362,631.36409481)(549.89104225,631.2641037)
\curveto(549.91103616,631.23409494)(549.91603615,631.20409497)(549.90604225,631.1741037)
\curveto(549.90603616,631.14409503)(549.91103616,631.11409506)(549.92104225,631.0841037)
\curveto(549.95103612,630.9740952)(549.9710361,630.84909533)(549.98104225,630.7091037)
\curveto(549.99103608,630.5790956)(550.00103607,630.44409573)(550.01104225,630.3041037)
\lineto(550.01104225,630.1391037)
\curveto(550.02103605,630.0790961)(550.02103605,630.02409615)(550.01104225,629.9741037)
\curveto(550.00103607,629.92409625)(549.99603607,629.8740963)(549.99604225,629.8241037)
\lineto(549.99604225,629.6891037)
\curveto(549.98603608,629.64909653)(549.98103609,629.60909657)(549.98104225,629.5691037)
\curveto(549.99103608,629.52909665)(549.98603608,629.48409669)(549.96604225,629.4341037)
\curveto(549.94603612,629.32409685)(549.92603614,629.21909696)(549.90604225,629.1191037)
\curveto(549.89603617,629.01909716)(549.87603619,628.91909726)(549.84604225,628.8191037)
\curveto(549.71603635,628.45909772)(549.55103652,628.14409803)(549.35104225,627.8741037)
\curveto(549.15103692,627.60409857)(548.87603719,627.39909878)(548.52604225,627.2591037)
\curveto(548.44603762,627.22909895)(548.36103771,627.20409897)(548.27104225,627.1841037)
\lineto(548.00104225,627.1241037)
\curveto(547.95103812,627.11409906)(547.90603816,627.10909907)(547.86604225,627.1091037)
\curveto(547.82603824,627.11909906)(547.78603828,627.11909906)(547.74604225,627.1091037)
\curveto(547.64603842,627.08909909)(547.55103852,627.08909909)(547.46104225,627.1091037)
\moveto(546.62104225,628.5041037)
\curveto(546.66103941,628.43409774)(546.70103937,628.36909781)(546.74104225,628.3091037)
\curveto(546.78103929,628.25909792)(546.83103924,628.20909797)(546.89104225,628.1591037)
\lineto(547.04104225,628.0391037)
\curveto(547.10103897,628.00909817)(547.1660389,627.98409819)(547.23604225,627.9641037)
\curveto(547.27603879,627.94409823)(547.31103876,627.93409824)(547.34104225,627.9341037)
\curveto(547.38103869,627.94409823)(547.42103865,627.93909824)(547.46104225,627.9191037)
\curveto(547.49103858,627.91909826)(547.53103854,627.91409826)(547.58104225,627.9041037)
\curveto(547.63103844,627.90409827)(547.6710384,627.90909827)(547.70104225,627.9191037)
\lineto(547.92604225,627.9641037)
\curveto(548.17603789,628.04409813)(548.36103771,628.16909801)(548.48104225,628.3391037)
\curveto(548.56103751,628.43909774)(548.63103744,628.56909761)(548.69104225,628.7291037)
\curveto(548.7710373,628.90909727)(548.83103724,629.13409704)(548.87104225,629.4041037)
\curveto(548.91103716,629.68409649)(548.92603714,629.96409621)(548.91604225,630.2441037)
\curveto(548.90603716,630.53409564)(548.87603719,630.80909537)(548.82604225,631.0691037)
\curveto(548.77603729,631.32909485)(548.70103737,631.53909464)(548.60104225,631.6991037)
\curveto(548.48103759,631.89909428)(548.33103774,632.04909413)(548.15104225,632.1491037)
\curveto(548.071038,632.19909398)(547.98103809,632.22909395)(547.88104225,632.2391037)
\curveto(547.78103829,632.25909392)(547.67603839,632.26909391)(547.56604225,632.2691037)
\curveto(547.54603852,632.25909392)(547.52103855,632.25409392)(547.49104225,632.2541037)
\curveto(547.4710386,632.26409391)(547.45103862,632.26409391)(547.43104225,632.2541037)
\curveto(547.38103869,632.24409393)(547.33603873,632.23409394)(547.29604225,632.2241037)
\curveto(547.25603881,632.22409395)(547.21603885,632.21409396)(547.17604225,632.1941037)
\curveto(546.99603907,632.11409406)(546.84603922,631.99409418)(546.72604225,631.8341037)
\curveto(546.61603945,631.6740945)(546.52603954,631.49409468)(546.45604225,631.2941037)
\curveto(546.39603967,631.10409507)(546.35103972,630.8790953)(546.32104225,630.6191037)
\curveto(546.30103977,630.35909582)(546.29603977,630.09409608)(546.30604225,629.8241037)
\curveto(546.31603975,629.56409661)(546.34603972,629.31409686)(546.39604225,629.0741037)
\curveto(546.45603961,628.84409733)(546.53103954,628.65409752)(546.62104225,628.5041037)
\moveto(557.42104225,625.5191037)
\curveto(557.43102864,625.46910071)(557.43602863,625.3791008)(557.43604225,625.2491037)
\curveto(557.43602863,625.11910106)(557.42602864,625.02910115)(557.40604225,624.9791037)
\curveto(557.38602868,624.92910125)(557.38102869,624.8741013)(557.39104225,624.8141037)
\curveto(557.40102867,624.76410141)(557.40102867,624.71410146)(557.39104225,624.6641037)
\curveto(557.35102872,624.52410165)(557.32102875,624.38910179)(557.30104225,624.2591037)
\curveto(557.29102878,624.12910205)(557.26102881,624.00910217)(557.21104225,623.8991037)
\curveto(557.071029,623.54910263)(556.90602916,623.25410292)(556.71604225,623.0141037)
\curveto(556.52602954,622.78410339)(556.25602981,622.59910358)(555.90604225,622.4591037)
\curveto(555.82603024,622.42910375)(555.74103033,622.40910377)(555.65104225,622.3991037)
\curveto(555.56103051,622.3791038)(555.47603059,622.35910382)(555.39604225,622.3391037)
\curveto(555.34603072,622.32910385)(555.29603077,622.32410385)(555.24604225,622.3241037)
\curveto(555.19603087,622.32410385)(555.14603092,622.31910386)(555.09604225,622.3091037)
\curveto(555.066031,622.29910388)(555.01603105,622.29910388)(554.94604225,622.3091037)
\curveto(554.87603119,622.30910387)(554.82603124,622.31410386)(554.79604225,622.3241037)
\curveto(554.73603133,622.34410383)(554.67603139,622.35410382)(554.61604225,622.3541037)
\curveto(554.5660315,622.34410383)(554.51603155,622.34910383)(554.46604225,622.3691037)
\curveto(554.37603169,622.38910379)(554.28603178,622.41410376)(554.19604225,622.4441037)
\curveto(554.11603195,622.46410371)(554.03603203,622.49410368)(553.95604225,622.5341037)
\curveto(553.63603243,622.6741035)(553.38603268,622.86910331)(553.20604225,623.1191037)
\curveto(553.02603304,623.3791028)(552.87603319,623.68410249)(552.75604225,624.0341037)
\curveto(552.73603333,624.11410206)(552.72103335,624.19910198)(552.71104225,624.2891037)
\curveto(552.70103337,624.3791018)(552.68603338,624.46410171)(552.66604225,624.5441037)
\curveto(552.65603341,624.5741016)(552.65103342,624.60410157)(552.65104225,624.6341037)
\lineto(552.65104225,624.7391037)
\curveto(552.63103344,624.81910136)(552.62103345,624.89910128)(552.62104225,624.9791037)
\lineto(552.62104225,625.1141037)
\curveto(552.60103347,625.21410096)(552.60103347,625.31410086)(552.62104225,625.4141037)
\lineto(552.62104225,625.5941037)
\curveto(552.63103344,625.64410053)(552.63603343,625.68910049)(552.63604225,625.7291037)
\curveto(552.63603343,625.7791004)(552.64103343,625.82410035)(552.65104225,625.8641037)
\curveto(552.66103341,625.90410027)(552.6660334,625.93910024)(552.66604225,625.9691037)
\curveto(552.6660334,626.00910017)(552.6710334,626.04910013)(552.68104225,626.0891037)
\lineto(552.74104225,626.4191037)
\curveto(552.76103331,626.53909964)(552.79103328,626.64909953)(552.83104225,626.7491037)
\curveto(552.9710331,627.0790991)(553.13103294,627.35409882)(553.31104225,627.5741037)
\curveto(553.50103257,627.80409837)(553.76103231,627.98909819)(554.09104225,628.1291037)
\curveto(554.1710319,628.16909801)(554.25603181,628.19409798)(554.34604225,628.2041037)
\lineto(554.64604225,628.2641037)
\lineto(554.78104225,628.2641037)
\curveto(554.83103124,628.2740979)(554.88103119,628.2790979)(554.93104225,628.2791037)
\curveto(555.50103057,628.29909788)(555.96103011,628.19409798)(556.31104225,627.9641037)
\curveto(556.6710294,627.74409843)(556.93602913,627.44409873)(557.10604225,627.0641037)
\curveto(557.15602891,626.96409921)(557.19602887,626.86409931)(557.22604225,626.7641037)
\curveto(557.25602881,626.66409951)(557.28602878,626.55909962)(557.31604225,626.4491037)
\curveto(557.32602874,626.40909977)(557.33102874,626.3740998)(557.33104225,626.3441037)
\curveto(557.33102874,626.32409985)(557.33602873,626.29409988)(557.34604225,626.2541037)
\curveto(557.3660287,626.18409999)(557.37602869,626.10910007)(557.37604225,626.0291037)
\curveto(557.37602869,625.94910023)(557.38602868,625.86910031)(557.40604225,625.7891037)
\curveto(557.40602866,625.73910044)(557.40602866,625.69410048)(557.40604225,625.6541037)
\curveto(557.40602866,625.61410056)(557.41102866,625.56910061)(557.42104225,625.5191037)
\moveto(556.31104225,625.0841037)
\curveto(556.32102975,625.13410104)(556.32602974,625.20910097)(556.32604225,625.3091037)
\curveto(556.33602973,625.40910077)(556.33102974,625.48410069)(556.31104225,625.5341037)
\curveto(556.29102978,625.59410058)(556.28602978,625.64910053)(556.29604225,625.6991037)
\curveto(556.31602975,625.75910042)(556.31602975,625.81910036)(556.29604225,625.8791037)
\curveto(556.28602978,625.90910027)(556.28102979,625.94410023)(556.28104225,625.9841037)
\curveto(556.28102979,626.02410015)(556.27602979,626.06410011)(556.26604225,626.1041037)
\curveto(556.24602982,626.18409999)(556.22602984,626.25909992)(556.20604225,626.3291037)
\curveto(556.19602987,626.40909977)(556.18102989,626.48909969)(556.16104225,626.5691037)
\curveto(556.13102994,626.62909955)(556.10602996,626.68909949)(556.08604225,626.7491037)
\curveto(556.06603,626.80909937)(556.03603003,626.86909931)(555.99604225,626.9291037)
\curveto(555.89603017,627.09909908)(555.7660303,627.23409894)(555.60604225,627.3341037)
\curveto(555.52603054,627.38409879)(555.43103064,627.41909876)(555.32104225,627.4391037)
\curveto(555.21103086,627.45909872)(555.08603098,627.46909871)(554.94604225,627.4691037)
\curveto(554.92603114,627.45909872)(554.90103117,627.45409872)(554.87104225,627.4541037)
\curveto(554.84103123,627.46409871)(554.81103126,627.46409871)(554.78104225,627.4541037)
\lineto(554.63104225,627.3941037)
\curveto(554.58103149,627.38409879)(554.53603153,627.36909881)(554.49604225,627.3491037)
\curveto(554.30603176,627.23909894)(554.16103191,627.09409908)(554.06104225,626.9141037)
\curveto(553.9710321,626.73409944)(553.89103218,626.52909965)(553.82104225,626.2991037)
\curveto(553.78103229,626.16910001)(553.76103231,626.03410014)(553.76104225,625.8941037)
\curveto(553.76103231,625.76410041)(553.75103232,625.61910056)(553.73104225,625.4591037)
\curveto(553.72103235,625.40910077)(553.71103236,625.34910083)(553.70104225,625.2791037)
\curveto(553.70103237,625.20910097)(553.71103236,625.14910103)(553.73104225,625.0991037)
\lineto(553.73104225,624.9341037)
\lineto(553.73104225,624.7541037)
\curveto(553.74103233,624.70410147)(553.75103232,624.64910153)(553.76104225,624.5891037)
\curveto(553.7710323,624.53910164)(553.77603229,624.48410169)(553.77604225,624.4241037)
\curveto(553.78603228,624.36410181)(553.80103227,624.30910187)(553.82104225,624.2591037)
\curveto(553.8710322,624.06910211)(553.93103214,623.89410228)(554.00104225,623.7341037)
\curveto(554.071032,623.5741026)(554.17603189,623.44410273)(554.31604225,623.3441037)
\curveto(554.44603162,623.24410293)(554.58603148,623.174103)(554.73604225,623.1341037)
\curveto(554.7660313,623.12410305)(554.79103128,623.11910306)(554.81104225,623.1191037)
\curveto(554.84103123,623.12910305)(554.8710312,623.12910305)(554.90104225,623.1191037)
\curveto(554.92103115,623.11910306)(554.95103112,623.11410306)(554.99104225,623.1041037)
\curveto(555.03103104,623.10410307)(555.066031,623.10910307)(555.09604225,623.1191037)
\curveto(555.13603093,623.12910305)(555.17603089,623.13410304)(555.21604225,623.1341037)
\curveto(555.25603081,623.13410304)(555.29603077,623.14410303)(555.33604225,623.1641037)
\curveto(555.57603049,623.24410293)(555.7710303,623.3791028)(555.92104225,623.5691037)
\curveto(556.04103003,623.74910243)(556.13102994,623.95410222)(556.19104225,624.1841037)
\curveto(556.21102986,624.25410192)(556.22602984,624.32410185)(556.23604225,624.3941037)
\curveto(556.24602982,624.4741017)(556.26102981,624.55410162)(556.28104225,624.6341037)
\curveto(556.28102979,624.69410148)(556.28602978,624.73910144)(556.29604225,624.7691037)
\curveto(556.29602977,624.78910139)(556.29602977,624.81410136)(556.29604225,624.8441037)
\curveto(556.29602977,624.88410129)(556.30102977,624.91410126)(556.31104225,624.9341037)
\lineto(556.31104225,625.0841037)
}
}
{
\newrgbcolor{curcolor}{0 0 0}
\pscustom[linestyle=none,fillstyle=solid,fillcolor=curcolor]
{
\newpath
\moveto(231.08548012,301.11339813)
\lineto(234.68548012,301.11339813)
\lineto(235.33048012,301.11339813)
\curveto(235.41047359,301.11338771)(235.48547351,301.10838771)(235.55548012,301.09839813)
\curveto(235.62547337,301.09838772)(235.68547331,301.08838773)(235.73548012,301.06839813)
\curveto(235.80547319,301.03838778)(235.86047314,300.97838784)(235.90048012,300.88839813)
\curveto(235.92047308,300.85838796)(235.93047307,300.818388)(235.93048012,300.76839813)
\lineto(235.93048012,300.63339813)
\curveto(235.94047306,300.5233883)(235.93547306,300.4183884)(235.91548012,300.31839813)
\curveto(235.90547309,300.2183886)(235.87047313,300.14838867)(235.81048012,300.10839813)
\curveto(235.72047328,300.03838878)(235.58547341,300.00338882)(235.40548012,300.00339813)
\curveto(235.22547377,300.01338881)(235.06047394,300.0183888)(234.91048012,300.01839813)
\lineto(232.91548012,300.01839813)
\lineto(232.42048012,300.01839813)
\lineto(232.28548012,300.01839813)
\curveto(232.24547675,300.0183888)(232.20547679,300.01338881)(232.16548012,300.00339813)
\lineto(231.95548012,300.00339813)
\curveto(231.84547715,299.97338885)(231.76547723,299.93338889)(231.71548012,299.88339813)
\curveto(231.66547733,299.84338898)(231.63047737,299.78838903)(231.61048012,299.71839813)
\curveto(231.59047741,299.65838916)(231.57547742,299.58838923)(231.56548012,299.50839813)
\curveto(231.55547744,299.42838939)(231.53547746,299.33838948)(231.50548012,299.23839813)
\curveto(231.45547754,299.03838978)(231.41547758,298.83338999)(231.38548012,298.62339813)
\curveto(231.35547764,298.41339041)(231.31547768,298.20839061)(231.26548012,298.00839813)
\curveto(231.24547775,297.93839088)(231.23547776,297.86839095)(231.23548012,297.79839813)
\curveto(231.23547776,297.73839108)(231.22547777,297.67339115)(231.20548012,297.60339813)
\curveto(231.1954778,297.57339125)(231.18547781,297.53339129)(231.17548012,297.48339813)
\curveto(231.17547782,297.44339138)(231.18047782,297.40339142)(231.19048012,297.36339813)
\curveto(231.21047779,297.31339151)(231.23547776,297.26839155)(231.26548012,297.22839813)
\curveto(231.30547769,297.19839162)(231.36547763,297.19339163)(231.44548012,297.21339813)
\curveto(231.50547749,297.23339159)(231.56547743,297.25839156)(231.62548012,297.28839813)
\curveto(231.68547731,297.32839149)(231.74547725,297.36339146)(231.80548012,297.39339813)
\curveto(231.86547713,297.41339141)(231.91547708,297.42839139)(231.95548012,297.43839813)
\curveto(232.14547685,297.5183913)(232.35047665,297.57339125)(232.57048012,297.60339813)
\curveto(232.8004762,297.63339119)(233.03047597,297.64339118)(233.26048012,297.63339813)
\curveto(233.5004755,297.63339119)(233.73047527,297.60839121)(233.95048012,297.55839813)
\curveto(234.17047483,297.5183913)(234.37047463,297.45839136)(234.55048012,297.37839813)
\curveto(234.6004744,297.35839146)(234.64547435,297.33839148)(234.68548012,297.31839813)
\curveto(234.73547426,297.29839152)(234.78547421,297.27339155)(234.83548012,297.24339813)
\curveto(235.18547381,297.03339179)(235.46547353,296.80339202)(235.67548012,296.55339813)
\curveto(235.8954731,296.30339252)(236.09047291,295.97839284)(236.26048012,295.57839813)
\curveto(236.31047269,295.46839335)(236.34547265,295.35839346)(236.36548012,295.24839813)
\curveto(236.38547261,295.13839368)(236.41047259,295.0233938)(236.44048012,294.90339813)
\curveto(236.45047255,294.87339395)(236.45547254,294.82839399)(236.45548012,294.76839813)
\curveto(236.47547252,294.70839411)(236.48547251,294.63839418)(236.48548012,294.55839813)
\curveto(236.48547251,294.48839433)(236.4954725,294.4233944)(236.51548012,294.36339813)
\lineto(236.51548012,294.19839813)
\curveto(236.52547247,294.14839467)(236.53047247,294.07839474)(236.53048012,293.98839813)
\curveto(236.53047247,293.89839492)(236.52047248,293.82839499)(236.50048012,293.77839813)
\curveto(236.48047252,293.7183951)(236.47547252,293.65839516)(236.48548012,293.59839813)
\curveto(236.4954725,293.54839527)(236.49047251,293.49839532)(236.47048012,293.44839813)
\curveto(236.43047257,293.28839553)(236.3954726,293.13839568)(236.36548012,292.99839813)
\curveto(236.33547266,292.85839596)(236.29047271,292.7233961)(236.23048012,292.59339813)
\curveto(236.07047293,292.2233966)(235.85047315,291.88839693)(235.57048012,291.58839813)
\curveto(235.29047371,291.28839753)(234.97047403,291.05839776)(234.61048012,290.89839813)
\curveto(234.44047456,290.818398)(234.24047476,290.74339808)(234.01048012,290.67339813)
\curveto(233.9004751,290.63339819)(233.78547521,290.60839821)(233.66548012,290.59839813)
\curveto(233.54547545,290.58839823)(233.42547557,290.56839825)(233.30548012,290.53839813)
\curveto(233.25547574,290.5183983)(233.2004758,290.5183983)(233.14048012,290.53839813)
\curveto(233.08047592,290.54839827)(233.02047598,290.54339828)(232.96048012,290.52339813)
\curveto(232.86047614,290.50339832)(232.76047624,290.50339832)(232.66048012,290.52339813)
\lineto(232.52548012,290.52339813)
\curveto(232.47547652,290.54339828)(232.41547658,290.55339827)(232.34548012,290.55339813)
\curveto(232.28547671,290.54339828)(232.23047677,290.54839827)(232.18048012,290.56839813)
\curveto(232.14047686,290.57839824)(232.10547689,290.58339824)(232.07548012,290.58339813)
\curveto(232.04547695,290.58339824)(232.01047699,290.58839823)(231.97048012,290.59839813)
\lineto(231.70048012,290.65839813)
\curveto(231.61047739,290.67839814)(231.52547747,290.70839811)(231.44548012,290.74839813)
\curveto(231.10547789,290.88839793)(230.81547818,291.04339778)(230.57548012,291.21339813)
\curveto(230.33547866,291.39339743)(230.11547888,291.6233972)(229.91548012,291.90339813)
\curveto(229.76547923,292.13339669)(229.65047935,292.37339645)(229.57048012,292.62339813)
\curveto(229.55047945,292.67339615)(229.54047946,292.7183961)(229.54048012,292.75839813)
\curveto(229.54047946,292.80839601)(229.53047947,292.85839596)(229.51048012,292.90839813)
\curveto(229.49047951,292.96839585)(229.47547952,293.04839577)(229.46548012,293.14839813)
\curveto(229.46547953,293.24839557)(229.48547951,293.3233955)(229.52548012,293.37339813)
\curveto(229.57547942,293.45339537)(229.65547934,293.49839532)(229.76548012,293.50839813)
\curveto(229.87547912,293.5183953)(229.99047901,293.5233953)(230.11048012,293.52339813)
\lineto(230.27548012,293.52339813)
\curveto(230.33547866,293.5233953)(230.39047861,293.51339531)(230.44048012,293.49339813)
\curveto(230.53047847,293.47339535)(230.6004784,293.43339539)(230.65048012,293.37339813)
\curveto(230.72047828,293.28339554)(230.76547823,293.17339565)(230.78548012,293.04339813)
\curveto(230.81547818,292.9233959)(230.86047814,292.818396)(230.92048012,292.72839813)
\curveto(231.11047789,292.38839643)(231.37047763,292.1183967)(231.70048012,291.91839813)
\curveto(231.8004772,291.85839696)(231.90547709,291.80839701)(232.01548012,291.76839813)
\curveto(232.13547686,291.73839708)(232.25547674,291.70339712)(232.37548012,291.66339813)
\curveto(232.54547645,291.61339721)(232.75047625,291.59339723)(232.99048012,291.60339813)
\curveto(233.24047576,291.6233972)(233.44047556,291.65839716)(233.59048012,291.70839813)
\curveto(233.96047504,291.82839699)(234.25047475,291.98839683)(234.46048012,292.18839813)
\curveto(234.68047432,292.39839642)(234.86047414,292.67839614)(235.00048012,293.02839813)
\curveto(235.05047395,293.12839569)(235.08047392,293.23339559)(235.09048012,293.34339813)
\curveto(235.11047389,293.45339537)(235.13547386,293.56839525)(235.16548012,293.68839813)
\lineto(235.16548012,293.79339813)
\curveto(235.17547382,293.83339499)(235.18047382,293.87339495)(235.18048012,293.91339813)
\curveto(235.19047381,293.94339488)(235.19047381,293.97839484)(235.18048012,294.01839813)
\lineto(235.18048012,294.13839813)
\curveto(235.18047382,294.39839442)(235.15047385,294.64339418)(235.09048012,294.87339813)
\curveto(234.98047402,295.2233936)(234.82547417,295.5183933)(234.62548012,295.75839813)
\curveto(234.42547457,296.00839281)(234.16547483,296.20339262)(233.84548012,296.34339813)
\lineto(233.66548012,296.40339813)
\curveto(233.61547538,296.4233924)(233.55547544,296.44339238)(233.48548012,296.46339813)
\curveto(233.43547556,296.48339234)(233.37547562,296.49339233)(233.30548012,296.49339813)
\curveto(233.24547575,296.50339232)(233.18047582,296.5183923)(233.11048012,296.53839813)
\lineto(232.96048012,296.53839813)
\curveto(232.92047608,296.55839226)(232.86547613,296.56839225)(232.79548012,296.56839813)
\curveto(232.73547626,296.56839225)(232.68047632,296.55839226)(232.63048012,296.53839813)
\lineto(232.52548012,296.53839813)
\curveto(232.4954765,296.53839228)(232.46047654,296.53339229)(232.42048012,296.52339813)
\lineto(232.18048012,296.46339813)
\curveto(232.1004769,296.45339237)(232.02047698,296.43339239)(231.94048012,296.40339813)
\curveto(231.7004773,296.30339252)(231.47047753,296.16839265)(231.25048012,295.99839813)
\curveto(231.16047784,295.92839289)(231.07547792,295.85339297)(230.99548012,295.77339813)
\curveto(230.91547808,295.70339312)(230.81547818,295.64839317)(230.69548012,295.60839813)
\curveto(230.60547839,295.57839324)(230.46547853,295.56839325)(230.27548012,295.57839813)
\curveto(230.0954789,295.58839323)(229.97547902,295.61339321)(229.91548012,295.65339813)
\curveto(229.86547913,295.69339313)(229.82547917,295.75339307)(229.79548012,295.83339813)
\curveto(229.77547922,295.91339291)(229.77547922,295.99839282)(229.79548012,296.08839813)
\curveto(229.82547917,296.20839261)(229.84547915,296.32839249)(229.85548012,296.44839813)
\curveto(229.87547912,296.57839224)(229.9004791,296.70339212)(229.93048012,296.82339813)
\curveto(229.95047905,296.86339196)(229.95547904,296.89839192)(229.94548012,296.92839813)
\curveto(229.94547905,296.96839185)(229.95547904,297.01339181)(229.97548012,297.06339813)
\curveto(229.995479,297.15339167)(230.01047899,297.24339158)(230.02048012,297.33339813)
\curveto(230.03047897,297.43339139)(230.05047895,297.52839129)(230.08048012,297.61839813)
\curveto(230.09047891,297.67839114)(230.0954789,297.73839108)(230.09548012,297.79839813)
\curveto(230.10547889,297.85839096)(230.12047888,297.9183909)(230.14048012,297.97839813)
\curveto(230.19047881,298.17839064)(230.22547877,298.38339044)(230.24548012,298.59339813)
\curveto(230.27547872,298.81339001)(230.31547868,299.0233898)(230.36548012,299.22339813)
\curveto(230.3954786,299.3233895)(230.41547858,299.4233894)(230.42548012,299.52339813)
\curveto(230.43547856,299.6233892)(230.45047855,299.7233891)(230.47048012,299.82339813)
\curveto(230.48047852,299.85338897)(230.48547851,299.89338893)(230.48548012,299.94339813)
\curveto(230.51547848,300.05338877)(230.53547846,300.15838866)(230.54548012,300.25839813)
\curveto(230.56547843,300.36838845)(230.59047841,300.47838834)(230.62048012,300.58839813)
\curveto(230.64047836,300.66838815)(230.65547834,300.73838808)(230.66548012,300.79839813)
\curveto(230.67547832,300.86838795)(230.7004783,300.92838789)(230.74048012,300.97839813)
\curveto(230.76047824,301.00838781)(230.79047821,301.02838779)(230.83048012,301.03839813)
\curveto(230.87047813,301.05838776)(230.91547808,301.07838774)(230.96548012,301.09839813)
\curveto(231.02547797,301.09838772)(231.06547793,301.10338772)(231.08548012,301.11339813)
}
}
{
\newrgbcolor{curcolor}{0 0 0}
\pscustom[linestyle=none,fillstyle=solid,fillcolor=curcolor]
{
\newpath
\moveto(238.88008949,292.33839813)
\lineto(239.18008949,292.33839813)
\curveto(239.29008743,292.34839647)(239.39508733,292.34839647)(239.49508949,292.33839813)
\curveto(239.60508712,292.33839648)(239.70508702,292.32839649)(239.79508949,292.30839813)
\curveto(239.88508684,292.29839652)(239.95508677,292.27339655)(240.00508949,292.23339813)
\curveto(240.0250867,292.21339661)(240.04008668,292.18339664)(240.05008949,292.14339813)
\curveto(240.07008665,292.10339672)(240.09008663,292.05839676)(240.11008949,292.00839813)
\lineto(240.11008949,291.93339813)
\curveto(240.1200866,291.88339694)(240.1200866,291.82839699)(240.11008949,291.76839813)
\lineto(240.11008949,291.61839813)
\lineto(240.11008949,291.13839813)
\curveto(240.11008661,290.96839785)(240.07008665,290.84839797)(239.99008949,290.77839813)
\curveto(239.9200868,290.72839809)(239.83008689,290.70339812)(239.72008949,290.70339813)
\lineto(239.39008949,290.70339813)
\lineto(238.94008949,290.70339813)
\curveto(238.79008793,290.70339812)(238.67508805,290.73339809)(238.59508949,290.79339813)
\curveto(238.55508817,290.823398)(238.5250882,290.87339795)(238.50508949,290.94339813)
\curveto(238.48508824,291.0233978)(238.47008825,291.10839771)(238.46008949,291.19839813)
\lineto(238.46008949,291.48339813)
\curveto(238.47008825,291.58339724)(238.47508825,291.66839715)(238.47508949,291.73839813)
\lineto(238.47508949,291.93339813)
\curveto(238.47508825,291.99339683)(238.48508824,292.04839677)(238.50508949,292.09839813)
\curveto(238.54508818,292.20839661)(238.61508811,292.27839654)(238.71508949,292.30839813)
\curveto(238.74508798,292.30839651)(238.80008792,292.3183965)(238.88008949,292.33839813)
}
}
{
\newrgbcolor{curcolor}{0 0 0}
\pscustom[linestyle=none,fillstyle=solid,fillcolor=curcolor]
{
\newpath
\moveto(245.20024574,301.30839813)
\curveto(246.8302403,301.33838748)(247.88023925,300.78338804)(248.35024574,299.64339813)
\curveto(248.45023868,299.41338941)(248.51523862,299.1233897)(248.54524574,298.77339813)
\curveto(248.58523855,298.43339039)(248.56023857,298.1233907)(248.47024574,297.84339813)
\curveto(248.38023875,297.58339124)(248.26023887,297.35839146)(248.11024574,297.16839813)
\curveto(248.09023904,297.12839169)(248.06523907,297.09339173)(248.03524574,297.06339813)
\curveto(248.00523913,297.04339178)(247.98023915,297.0183918)(247.96024574,296.98839813)
\lineto(247.87024574,296.86839813)
\curveto(247.84023929,296.83839198)(247.80523933,296.81339201)(247.76524574,296.79339813)
\curveto(247.71523942,296.74339208)(247.66023947,296.69839212)(247.60024574,296.65839813)
\curveto(247.55023958,296.6183922)(247.50523963,296.56839225)(247.46524574,296.50839813)
\curveto(247.42523971,296.46839235)(247.41023972,296.4183924)(247.42024574,296.35839813)
\curveto(247.4302397,296.30839251)(247.46023967,296.26339256)(247.51024574,296.22339813)
\curveto(247.56023957,296.18339264)(247.61523952,296.14339268)(247.67524574,296.10339813)
\curveto(247.74523939,296.07339275)(247.81023932,296.04339278)(247.87024574,296.01339813)
\curveto(247.9302392,295.98339284)(247.98023915,295.95339287)(248.02024574,295.92339813)
\curveto(248.34023879,295.70339312)(248.59523854,295.39339343)(248.78524574,294.99339813)
\curveto(248.82523831,294.90339392)(248.85523828,294.80839401)(248.87524574,294.70839813)
\curveto(248.90523823,294.6183942)(248.9302382,294.52839429)(248.95024574,294.43839813)
\curveto(248.96023817,294.38839443)(248.96523817,294.33839448)(248.96524574,294.28839813)
\curveto(248.97523816,294.24839457)(248.98523815,294.20339462)(248.99524574,294.15339813)
\curveto(249.00523813,294.10339472)(249.00523813,294.05339477)(248.99524574,294.00339813)
\curveto(248.98523815,293.95339487)(248.99023814,293.90339492)(249.01024574,293.85339813)
\curveto(249.02023811,293.80339502)(249.02523811,293.74339508)(249.02524574,293.67339813)
\curveto(249.02523811,293.60339522)(249.01523812,293.54339528)(248.99524574,293.49339813)
\lineto(248.99524574,293.26839813)
\lineto(248.93524574,293.02839813)
\curveto(248.92523821,292.95839586)(248.91023822,292.88839593)(248.89024574,292.81839813)
\curveto(248.86023827,292.72839609)(248.8302383,292.64339618)(248.80024574,292.56339813)
\curveto(248.78023835,292.48339634)(248.75023838,292.40339642)(248.71024574,292.32339813)
\curveto(248.69023844,292.26339656)(248.66023847,292.20339662)(248.62024574,292.14339813)
\curveto(248.59023854,292.09339673)(248.55523858,292.04339678)(248.51524574,291.99339813)
\curveto(248.31523882,291.68339714)(248.06523907,291.4233974)(247.76524574,291.21339813)
\curveto(247.46523967,291.01339781)(247.12024001,290.84839797)(246.73024574,290.71839813)
\curveto(246.61024052,290.67839814)(246.48024065,290.65339817)(246.34024574,290.64339813)
\curveto(246.21024092,290.6233982)(246.07524106,290.59839822)(245.93524574,290.56839813)
\curveto(245.86524127,290.55839826)(245.79524134,290.55339827)(245.72524574,290.55339813)
\curveto(245.66524147,290.55339827)(245.60024153,290.54839827)(245.53024574,290.53839813)
\curveto(245.49024164,290.52839829)(245.4302417,290.5233983)(245.35024574,290.52339813)
\curveto(245.28024185,290.5233983)(245.2302419,290.52839829)(245.20024574,290.53839813)
\curveto(245.15024198,290.54839827)(245.10524203,290.55339827)(245.06524574,290.55339813)
\lineto(244.94524574,290.55339813)
\curveto(244.84524229,290.57339825)(244.74524239,290.58839823)(244.64524574,290.59839813)
\curveto(244.54524259,290.60839821)(244.45024268,290.6233982)(244.36024574,290.64339813)
\curveto(244.25024288,290.67339815)(244.14024299,290.69839812)(244.03024574,290.71839813)
\curveto(243.9302432,290.74839807)(243.82524331,290.78839803)(243.71524574,290.83839813)
\curveto(243.34524379,290.99839782)(243.0302441,291.19839762)(242.77024574,291.43839813)
\curveto(242.51024462,291.68839713)(242.30024483,291.99839682)(242.14024574,292.36839813)
\curveto(242.10024503,292.45839636)(242.06524507,292.55339627)(242.03524574,292.65339813)
\curveto(242.00524513,292.75339607)(241.97524516,292.85839596)(241.94524574,292.96839813)
\curveto(241.92524521,293.0183958)(241.91524522,293.06839575)(241.91524574,293.11839813)
\curveto(241.91524522,293.17839564)(241.90524523,293.23839558)(241.88524574,293.29839813)
\curveto(241.86524527,293.35839546)(241.85524528,293.43839538)(241.85524574,293.53839813)
\curveto(241.85524528,293.63839518)(241.87024526,293.71339511)(241.90024574,293.76339813)
\curveto(241.91024522,293.79339503)(241.92524521,293.818395)(241.94524574,293.83839813)
\lineto(242.00524574,293.89839813)
\curveto(242.04524509,293.9183949)(242.10524503,293.93339489)(242.18524574,293.94339813)
\curveto(242.27524486,293.95339487)(242.36524477,293.95839486)(242.45524574,293.95839813)
\curveto(242.54524459,293.95839486)(242.6302445,293.95339487)(242.71024574,293.94339813)
\curveto(242.80024433,293.93339489)(242.86524427,293.9233949)(242.90524574,293.91339813)
\curveto(242.92524421,293.89339493)(242.94524419,293.87839494)(242.96524574,293.86839813)
\curveto(242.98524415,293.86839495)(243.00524413,293.85839496)(243.02524574,293.83839813)
\curveto(243.09524404,293.74839507)(243.135244,293.63339519)(243.14524574,293.49339813)
\curveto(243.16524397,293.35339547)(243.19524394,293.22839559)(243.23524574,293.11839813)
\lineto(243.38524574,292.75839813)
\curveto(243.4352437,292.64839617)(243.50024363,292.54339628)(243.58024574,292.44339813)
\curveto(243.60024353,292.41339641)(243.62024351,292.38839643)(243.64024574,292.36839813)
\curveto(243.67024346,292.34839647)(243.69524344,292.3233965)(243.71524574,292.29339813)
\curveto(243.75524338,292.23339659)(243.79024334,292.18839663)(243.82024574,292.15839813)
\curveto(243.86024327,292.12839669)(243.89524324,292.09839672)(243.92524574,292.06839813)
\curveto(243.96524317,292.03839678)(244.01024312,292.00839681)(244.06024574,291.97839813)
\curveto(244.15024298,291.9183969)(244.24524289,291.86839695)(244.34524574,291.82839813)
\lineto(244.67524574,291.70839813)
\curveto(244.82524231,291.65839716)(245.02524211,291.62839719)(245.27524574,291.61839813)
\curveto(245.52524161,291.60839721)(245.7352414,291.62839719)(245.90524574,291.67839813)
\curveto(245.98524115,291.69839712)(246.05524108,291.71339711)(246.11524574,291.72339813)
\lineto(246.32524574,291.78339813)
\curveto(246.60524053,291.90339692)(246.84524029,292.05339677)(247.04524574,292.23339813)
\curveto(247.25523988,292.41339641)(247.42023971,292.64339618)(247.54024574,292.92339813)
\curveto(247.57023956,292.99339583)(247.59023954,293.06339576)(247.60024574,293.13339813)
\lineto(247.66024574,293.37339813)
\curveto(247.70023943,293.51339531)(247.71023942,293.67339515)(247.69024574,293.85339813)
\curveto(247.67023946,294.04339478)(247.64023949,294.19339463)(247.60024574,294.30339813)
\curveto(247.47023966,294.68339414)(247.28523985,294.97339385)(247.04524574,295.17339813)
\curveto(246.81524032,295.37339345)(246.50524063,295.53339329)(246.11524574,295.65339813)
\curveto(246.00524113,295.68339314)(245.88524125,295.70339312)(245.75524574,295.71339813)
\curveto(245.6352415,295.7233931)(245.51024162,295.72839309)(245.38024574,295.72839813)
\curveto(245.22024191,295.72839309)(245.08024205,295.73339309)(244.96024574,295.74339813)
\curveto(244.84024229,295.75339307)(244.75524238,295.81339301)(244.70524574,295.92339813)
\curveto(244.68524245,295.95339287)(244.67524246,295.98839283)(244.67524574,296.02839813)
\lineto(244.67524574,296.16339813)
\curveto(244.66524247,296.26339256)(244.66524247,296.35839246)(244.67524574,296.44839813)
\curveto(244.69524244,296.53839228)(244.7352424,296.60339222)(244.79524574,296.64339813)
\curveto(244.8352423,296.67339215)(244.87524226,296.69339213)(244.91524574,296.70339813)
\curveto(244.96524217,296.71339211)(245.02024211,296.7233921)(245.08024574,296.73339813)
\curveto(245.10024203,296.74339208)(245.12524201,296.74339208)(245.15524574,296.73339813)
\curveto(245.18524195,296.73339209)(245.21024192,296.73839208)(245.23024574,296.74839813)
\lineto(245.36524574,296.74839813)
\curveto(245.47524166,296.76839205)(245.57524156,296.77839204)(245.66524574,296.77839813)
\curveto(245.76524137,296.78839203)(245.86024127,296.80839201)(245.95024574,296.83839813)
\curveto(246.27024086,296.94839187)(246.52524061,297.09339173)(246.71524574,297.27339813)
\curveto(246.90524023,297.45339137)(247.05524008,297.70339112)(247.16524574,298.02339813)
\curveto(247.19523994,298.1233907)(247.21523992,298.24839057)(247.22524574,298.39839813)
\curveto(247.24523989,298.55839026)(247.24023989,298.70339012)(247.21024574,298.83339813)
\curveto(247.19023994,298.90338992)(247.17023996,298.96838985)(247.15024574,299.02839813)
\curveto(247.14023999,299.09838972)(247.12024001,299.16338966)(247.09024574,299.22339813)
\curveto(246.99024014,299.46338936)(246.84524029,299.65338917)(246.65524574,299.79339813)
\curveto(246.46524067,299.93338889)(246.24024089,300.04338878)(245.98024574,300.12339813)
\curveto(245.92024121,300.14338868)(245.86024127,300.15338867)(245.80024574,300.15339813)
\curveto(245.74024139,300.15338867)(245.67524146,300.16338866)(245.60524574,300.18339813)
\curveto(245.52524161,300.20338862)(245.4302417,300.21338861)(245.32024574,300.21339813)
\curveto(245.21024192,300.21338861)(245.11524202,300.20338862)(245.03524574,300.18339813)
\curveto(244.98524215,300.16338866)(244.9352422,300.15338867)(244.88524574,300.15339813)
\curveto(244.84524229,300.15338867)(244.80024233,300.14338868)(244.75024574,300.12339813)
\curveto(244.57024256,300.07338875)(244.40024273,299.99838882)(244.24024574,299.89839813)
\curveto(244.09024304,299.80838901)(243.96024317,299.69338913)(243.85024574,299.55339813)
\curveto(243.76024337,299.43338939)(243.68024345,299.30338952)(243.61024574,299.16339813)
\curveto(243.54024359,299.0233898)(243.47524366,298.86838995)(243.41524574,298.69839813)
\curveto(243.38524375,298.58839023)(243.36524377,298.46839035)(243.35524574,298.33839813)
\curveto(243.34524379,298.2183906)(243.31024382,298.1183907)(243.25024574,298.03839813)
\curveto(243.2302439,297.99839082)(243.17024396,297.95839086)(243.07024574,297.91839813)
\curveto(243.0302441,297.90839091)(242.97024416,297.89839092)(242.89024574,297.88839813)
\lineto(242.63524574,297.88839813)
\curveto(242.54524459,297.89839092)(242.46024467,297.90839091)(242.38024574,297.91839813)
\curveto(242.31024482,297.92839089)(242.26024487,297.94339088)(242.23024574,297.96339813)
\curveto(242.19024494,297.99339083)(242.15524498,298.04839077)(242.12524574,298.12839813)
\curveto(242.09524504,298.20839061)(242.09024504,298.29339053)(242.11024574,298.38339813)
\curveto(242.12024501,298.43339039)(242.12524501,298.48339034)(242.12524574,298.53339813)
\lineto(242.15524574,298.71339813)
\curveto(242.18524495,298.81339001)(242.21024492,298.91338991)(242.23024574,299.01339813)
\curveto(242.26024487,299.11338971)(242.29524484,299.20338962)(242.33524574,299.28339813)
\curveto(242.38524475,299.39338943)(242.4302447,299.49838932)(242.47024574,299.59839813)
\curveto(242.51024462,299.70838911)(242.56024457,299.81338901)(242.62024574,299.91339813)
\curveto(242.95024418,300.45338837)(243.42024371,300.84838797)(244.03024574,301.09839813)
\curveto(244.15024298,301.14838767)(244.27524286,301.18338764)(244.40524574,301.20339813)
\curveto(244.54524259,301.2233876)(244.68524245,301.24838757)(244.82524574,301.27839813)
\curveto(244.88524225,301.28838753)(244.94524219,301.29338753)(245.00524574,301.29339813)
\curveto(245.07524206,301.29338753)(245.14024199,301.29838752)(245.20024574,301.30839813)
}
}
{
\newrgbcolor{curcolor}{0 0 0}
\pscustom[linestyle=none,fillstyle=solid,fillcolor=curcolor]
{
\newpath
\moveto(260.23985512,299.22339813)
\curveto(260.03984482,298.93338989)(259.82984503,298.64839017)(259.60985512,298.36839813)
\curveto(259.39984546,298.08839073)(259.19484566,297.80339102)(258.99485512,297.51339813)
\curveto(258.39484646,296.66339216)(257.78984707,295.823393)(257.17985512,294.99339813)
\curveto(256.56984829,294.17339465)(255.96484889,293.33839548)(255.36485512,292.48839813)
\lineto(254.85485512,291.76839813)
\lineto(254.34485512,291.07839813)
\curveto(254.26485059,290.96839785)(254.18485067,290.85339797)(254.10485512,290.73339813)
\curveto(254.02485083,290.61339821)(253.92985093,290.5183983)(253.81985512,290.44839813)
\curveto(253.77985108,290.42839839)(253.71485114,290.41339841)(253.62485512,290.40339813)
\curveto(253.54485131,290.38339844)(253.4548514,290.37339845)(253.35485512,290.37339813)
\curveto(253.2548516,290.37339845)(253.1598517,290.37839844)(253.06985512,290.38839813)
\curveto(252.98985187,290.39839842)(252.92985193,290.4183984)(252.88985512,290.44839813)
\curveto(252.859852,290.46839835)(252.83485202,290.50339832)(252.81485512,290.55339813)
\curveto(252.80485205,290.59339823)(252.80985205,290.63839818)(252.82985512,290.68839813)
\curveto(252.86985199,290.76839805)(252.91485194,290.84339798)(252.96485512,290.91339813)
\curveto(253.02485183,290.99339783)(253.07985178,291.07339775)(253.12985512,291.15339813)
\curveto(253.36985149,291.49339733)(253.61485124,291.82839699)(253.86485512,292.15839813)
\curveto(254.11485074,292.48839633)(254.3548505,292.823396)(254.58485512,293.16339813)
\curveto(254.74485011,293.38339544)(254.90484995,293.59839522)(255.06485512,293.80839813)
\curveto(255.22484963,294.0183948)(255.38484947,294.23339459)(255.54485512,294.45339813)
\curveto(255.90484895,294.97339385)(256.26984859,295.48339334)(256.63985512,295.98339813)
\curveto(257.00984785,296.48339234)(257.37984748,296.99339183)(257.74985512,297.51339813)
\curveto(257.88984697,297.71339111)(258.02984683,297.90839091)(258.16985512,298.09839813)
\curveto(258.31984654,298.28839053)(258.46484639,298.48339034)(258.60485512,298.68339813)
\curveto(258.81484604,298.98338984)(259.02984583,299.28338954)(259.24985512,299.58339813)
\lineto(259.90985512,300.48339813)
\lineto(260.08985512,300.75339813)
\lineto(260.29985512,301.02339813)
\lineto(260.41985512,301.20339813)
\curveto(260.46984439,301.26338756)(260.51984434,301.3183875)(260.56985512,301.36839813)
\curveto(260.63984422,301.4183874)(260.71484414,301.45338737)(260.79485512,301.47339813)
\curveto(260.81484404,301.48338734)(260.83984402,301.48338734)(260.86985512,301.47339813)
\curveto(260.90984395,301.47338735)(260.93984392,301.48338734)(260.95985512,301.50339813)
\curveto(261.07984378,301.50338732)(261.21484364,301.49838732)(261.36485512,301.48839813)
\curveto(261.51484334,301.48838733)(261.60484325,301.44338738)(261.63485512,301.35339813)
\curveto(261.6548432,301.3233875)(261.6598432,301.28838753)(261.64985512,301.24839813)
\curveto(261.63984322,301.20838761)(261.62484323,301.17838764)(261.60485512,301.15839813)
\curveto(261.56484329,301.07838774)(261.52484333,301.00838781)(261.48485512,300.94839813)
\curveto(261.44484341,300.88838793)(261.39984346,300.82838799)(261.34985512,300.76839813)
\lineto(260.77985512,299.98839813)
\curveto(260.59984426,299.73838908)(260.41984444,299.48338934)(260.23985512,299.22339813)
\moveto(253.38485512,295.32339813)
\curveto(253.33485152,295.34339348)(253.28485157,295.34839347)(253.23485512,295.33839813)
\curveto(253.18485167,295.32839349)(253.13485172,295.33339349)(253.08485512,295.35339813)
\curveto(252.97485188,295.37339345)(252.86985199,295.39339343)(252.76985512,295.41339813)
\curveto(252.67985218,295.44339338)(252.58485227,295.48339334)(252.48485512,295.53339813)
\curveto(252.1548527,295.67339315)(251.89985296,295.86839295)(251.71985512,296.11839813)
\curveto(251.53985332,296.37839244)(251.39485346,296.68839213)(251.28485512,297.04839813)
\curveto(251.2548536,297.12839169)(251.23485362,297.20839161)(251.22485512,297.28839813)
\curveto(251.21485364,297.37839144)(251.19985366,297.46339136)(251.17985512,297.54339813)
\curveto(251.16985369,297.59339123)(251.16485369,297.65839116)(251.16485512,297.73839813)
\curveto(251.1548537,297.76839105)(251.14985371,297.79839102)(251.14985512,297.82839813)
\curveto(251.14985371,297.86839095)(251.14485371,297.90339092)(251.13485512,297.93339813)
\lineto(251.13485512,298.08339813)
\curveto(251.12485373,298.13339069)(251.11985374,298.19339063)(251.11985512,298.26339813)
\curveto(251.11985374,298.34339048)(251.12485373,298.40839041)(251.13485512,298.45839813)
\lineto(251.13485512,298.62339813)
\curveto(251.1548537,298.67339015)(251.1598537,298.7183901)(251.14985512,298.75839813)
\curveto(251.14985371,298.80839001)(251.1548537,298.85338997)(251.16485512,298.89339813)
\curveto(251.17485368,298.93338989)(251.17985368,298.96838985)(251.17985512,298.99839813)
\curveto(251.17985368,299.03838978)(251.18485367,299.07838974)(251.19485512,299.11839813)
\curveto(251.22485363,299.22838959)(251.24485361,299.33838948)(251.25485512,299.44839813)
\curveto(251.27485358,299.56838925)(251.30985355,299.68338914)(251.35985512,299.79339813)
\curveto(251.49985336,300.13338869)(251.6598532,300.40838841)(251.83985512,300.61839813)
\curveto(252.02985283,300.83838798)(252.29985256,301.0183878)(252.64985512,301.15839813)
\curveto(252.72985213,301.18838763)(252.81485204,301.20838761)(252.90485512,301.21839813)
\curveto(252.99485186,301.23838758)(253.08985177,301.25838756)(253.18985512,301.27839813)
\curveto(253.21985164,301.28838753)(253.27485158,301.28838753)(253.35485512,301.27839813)
\curveto(253.43485142,301.27838754)(253.48485137,301.28838753)(253.50485512,301.30839813)
\curveto(254.06485079,301.3183875)(254.51485034,301.20838761)(254.85485512,300.97839813)
\curveto(255.20484965,300.74838807)(255.46484939,300.44338838)(255.63485512,300.06339813)
\curveto(255.67484918,299.97338885)(255.70984915,299.87838894)(255.73985512,299.77839813)
\curveto(255.76984909,299.67838914)(255.79484906,299.57838924)(255.81485512,299.47839813)
\curveto(255.83484902,299.44838937)(255.83984902,299.4183894)(255.82985512,299.38839813)
\curveto(255.82984903,299.35838946)(255.83484902,299.32838949)(255.84485512,299.29839813)
\curveto(255.87484898,299.18838963)(255.89484896,299.06338976)(255.90485512,298.92339813)
\curveto(255.91484894,298.79339003)(255.92484893,298.65839016)(255.93485512,298.51839813)
\lineto(255.93485512,298.35339813)
\curveto(255.94484891,298.29339053)(255.94484891,298.23839058)(255.93485512,298.18839813)
\curveto(255.92484893,298.13839068)(255.91984894,298.08839073)(255.91985512,298.03839813)
\lineto(255.91985512,297.90339813)
\curveto(255.90984895,297.86339096)(255.90484895,297.823391)(255.90485512,297.78339813)
\curveto(255.91484894,297.74339108)(255.90984895,297.69839112)(255.88985512,297.64839813)
\curveto(255.86984899,297.53839128)(255.84984901,297.43339139)(255.82985512,297.33339813)
\curveto(255.81984904,297.23339159)(255.79984906,297.13339169)(255.76985512,297.03339813)
\curveto(255.63984922,296.67339215)(255.47484938,296.35839246)(255.27485512,296.08839813)
\curveto(255.07484978,295.818393)(254.79985006,295.61339321)(254.44985512,295.47339813)
\curveto(254.36985049,295.44339338)(254.28485057,295.4183934)(254.19485512,295.39839813)
\lineto(253.92485512,295.33839813)
\curveto(253.87485098,295.32839349)(253.82985103,295.3233935)(253.78985512,295.32339813)
\curveto(253.74985111,295.33339349)(253.70985115,295.33339349)(253.66985512,295.32339813)
\curveto(253.56985129,295.30339352)(253.47485138,295.30339352)(253.38485512,295.32339813)
\moveto(252.54485512,296.71839813)
\curveto(252.58485227,296.64839217)(252.62485223,296.58339224)(252.66485512,296.52339813)
\curveto(252.70485215,296.47339235)(252.7548521,296.4233924)(252.81485512,296.37339813)
\lineto(252.96485512,296.25339813)
\curveto(253.02485183,296.2233926)(253.08985177,296.19839262)(253.15985512,296.17839813)
\curveto(253.19985166,296.15839266)(253.23485162,296.14839267)(253.26485512,296.14839813)
\curveto(253.30485155,296.15839266)(253.34485151,296.15339267)(253.38485512,296.13339813)
\curveto(253.41485144,296.13339269)(253.4548514,296.12839269)(253.50485512,296.11839813)
\curveto(253.5548513,296.1183927)(253.59485126,296.1233927)(253.62485512,296.13339813)
\lineto(253.84985512,296.17839813)
\curveto(254.09985076,296.25839256)(254.28485057,296.38339244)(254.40485512,296.55339813)
\curveto(254.48485037,296.65339217)(254.5548503,296.78339204)(254.61485512,296.94339813)
\curveto(254.69485016,297.1233917)(254.7548501,297.34839147)(254.79485512,297.61839813)
\curveto(254.83485002,297.89839092)(254.84985001,298.17839064)(254.83985512,298.45839813)
\curveto(254.82985003,298.74839007)(254.79985006,299.0233898)(254.74985512,299.28339813)
\curveto(254.69985016,299.54338928)(254.62485023,299.75338907)(254.52485512,299.91339813)
\curveto(254.40485045,300.11338871)(254.2548506,300.26338856)(254.07485512,300.36339813)
\curveto(253.99485086,300.41338841)(253.90485095,300.44338838)(253.80485512,300.45339813)
\curveto(253.70485115,300.47338835)(253.59985126,300.48338834)(253.48985512,300.48339813)
\curveto(253.46985139,300.47338835)(253.44485141,300.46838835)(253.41485512,300.46839813)
\curveto(253.39485146,300.47838834)(253.37485148,300.47838834)(253.35485512,300.46839813)
\curveto(253.30485155,300.45838836)(253.2598516,300.44838837)(253.21985512,300.43839813)
\curveto(253.17985168,300.43838838)(253.13985172,300.42838839)(253.09985512,300.40839813)
\curveto(252.91985194,300.32838849)(252.76985209,300.20838861)(252.64985512,300.04839813)
\curveto(252.53985232,299.88838893)(252.44985241,299.70838911)(252.37985512,299.50839813)
\curveto(252.31985254,299.3183895)(252.27485258,299.09338973)(252.24485512,298.83339813)
\curveto(252.22485263,298.57339025)(252.21985264,298.30839051)(252.22985512,298.03839813)
\curveto(252.23985262,297.77839104)(252.26985259,297.52839129)(252.31985512,297.28839813)
\curveto(252.37985248,297.05839176)(252.4548524,296.86839195)(252.54485512,296.71839813)
\moveto(263.34485512,293.73339813)
\curveto(263.3548415,293.68339514)(263.3598415,293.59339523)(263.35985512,293.46339813)
\curveto(263.3598415,293.33339549)(263.34984151,293.24339558)(263.32985512,293.19339813)
\curveto(263.30984155,293.14339568)(263.30484155,293.08839573)(263.31485512,293.02839813)
\curveto(263.32484153,292.97839584)(263.32484153,292.92839589)(263.31485512,292.87839813)
\curveto(263.27484158,292.73839608)(263.24484161,292.60339622)(263.22485512,292.47339813)
\curveto(263.21484164,292.34339648)(263.18484167,292.2233966)(263.13485512,292.11339813)
\curveto(262.99484186,291.76339706)(262.82984203,291.46839735)(262.63985512,291.22839813)
\curveto(262.44984241,290.99839782)(262.17984268,290.81339801)(261.82985512,290.67339813)
\curveto(261.74984311,290.64339818)(261.66484319,290.6233982)(261.57485512,290.61339813)
\curveto(261.48484337,290.59339823)(261.39984346,290.57339825)(261.31985512,290.55339813)
\curveto(261.26984359,290.54339828)(261.21984364,290.53839828)(261.16985512,290.53839813)
\curveto(261.11984374,290.53839828)(261.06984379,290.53339829)(261.01985512,290.52339813)
\curveto(260.98984387,290.51339831)(260.93984392,290.51339831)(260.86985512,290.52339813)
\curveto(260.79984406,290.5233983)(260.74984411,290.52839829)(260.71985512,290.53839813)
\curveto(260.6598442,290.55839826)(260.59984426,290.56839825)(260.53985512,290.56839813)
\curveto(260.48984437,290.55839826)(260.43984442,290.56339826)(260.38985512,290.58339813)
\curveto(260.29984456,290.60339822)(260.20984465,290.62839819)(260.11985512,290.65839813)
\curveto(260.03984482,290.67839814)(259.9598449,290.70839811)(259.87985512,290.74839813)
\curveto(259.5598453,290.88839793)(259.30984555,291.08339774)(259.12985512,291.33339813)
\curveto(258.94984591,291.59339723)(258.79984606,291.89839692)(258.67985512,292.24839813)
\curveto(258.6598462,292.32839649)(258.64484621,292.41339641)(258.63485512,292.50339813)
\curveto(258.62484623,292.59339623)(258.60984625,292.67839614)(258.58985512,292.75839813)
\curveto(258.57984628,292.78839603)(258.57484628,292.818396)(258.57485512,292.84839813)
\lineto(258.57485512,292.95339813)
\curveto(258.5548463,293.03339579)(258.54484631,293.11339571)(258.54485512,293.19339813)
\lineto(258.54485512,293.32839813)
\curveto(258.52484633,293.42839539)(258.52484633,293.52839529)(258.54485512,293.62839813)
\lineto(258.54485512,293.80839813)
\curveto(258.5548463,293.85839496)(258.5598463,293.90339492)(258.55985512,293.94339813)
\curveto(258.5598463,293.99339483)(258.56484629,294.03839478)(258.57485512,294.07839813)
\curveto(258.58484627,294.1183947)(258.58984627,294.15339467)(258.58985512,294.18339813)
\curveto(258.58984627,294.2233946)(258.59484626,294.26339456)(258.60485512,294.30339813)
\lineto(258.66485512,294.63339813)
\curveto(258.68484617,294.75339407)(258.71484614,294.86339396)(258.75485512,294.96339813)
\curveto(258.89484596,295.29339353)(259.0548458,295.56839325)(259.23485512,295.78839813)
\curveto(259.42484543,296.0183928)(259.68484517,296.20339262)(260.01485512,296.34339813)
\curveto(260.09484476,296.38339244)(260.17984468,296.40839241)(260.26985512,296.41839813)
\lineto(260.56985512,296.47839813)
\lineto(260.70485512,296.47839813)
\curveto(260.7548441,296.48839233)(260.80484405,296.49339233)(260.85485512,296.49339813)
\curveto(261.42484343,296.51339231)(261.88484297,296.40839241)(262.23485512,296.17839813)
\curveto(262.59484226,295.95839286)(262.859842,295.65839316)(263.02985512,295.27839813)
\curveto(263.07984178,295.17839364)(263.11984174,295.07839374)(263.14985512,294.97839813)
\curveto(263.17984168,294.87839394)(263.20984165,294.77339405)(263.23985512,294.66339813)
\curveto(263.24984161,294.6233942)(263.2548416,294.58839423)(263.25485512,294.55839813)
\curveto(263.2548416,294.53839428)(263.2598416,294.50839431)(263.26985512,294.46839813)
\curveto(263.28984157,294.39839442)(263.29984156,294.3233945)(263.29985512,294.24339813)
\curveto(263.29984156,294.16339466)(263.30984155,294.08339474)(263.32985512,294.00339813)
\curveto(263.32984153,293.95339487)(263.32984153,293.90839491)(263.32985512,293.86839813)
\curveto(263.32984153,293.82839499)(263.33484152,293.78339504)(263.34485512,293.73339813)
\moveto(262.23485512,293.29839813)
\curveto(262.24484261,293.34839547)(262.24984261,293.4233954)(262.24985512,293.52339813)
\curveto(262.2598426,293.6233952)(262.2548426,293.69839512)(262.23485512,293.74839813)
\curveto(262.21484264,293.80839501)(262.20984265,293.86339496)(262.21985512,293.91339813)
\curveto(262.23984262,293.97339485)(262.23984262,294.03339479)(262.21985512,294.09339813)
\curveto(262.20984265,294.1233947)(262.20484265,294.15839466)(262.20485512,294.19839813)
\curveto(262.20484265,294.23839458)(262.19984266,294.27839454)(262.18985512,294.31839813)
\curveto(262.16984269,294.39839442)(262.14984271,294.47339435)(262.12985512,294.54339813)
\curveto(262.11984274,294.6233942)(262.10484275,294.70339412)(262.08485512,294.78339813)
\curveto(262.0548428,294.84339398)(262.02984283,294.90339392)(262.00985512,294.96339813)
\curveto(261.98984287,295.0233938)(261.9598429,295.08339374)(261.91985512,295.14339813)
\curveto(261.81984304,295.31339351)(261.68984317,295.44839337)(261.52985512,295.54839813)
\curveto(261.44984341,295.59839322)(261.3548435,295.63339319)(261.24485512,295.65339813)
\curveto(261.13484372,295.67339315)(261.00984385,295.68339314)(260.86985512,295.68339813)
\curveto(260.84984401,295.67339315)(260.82484403,295.66839315)(260.79485512,295.66839813)
\curveto(260.76484409,295.67839314)(260.73484412,295.67839314)(260.70485512,295.66839813)
\lineto(260.55485512,295.60839813)
\curveto(260.50484435,295.59839322)(260.4598444,295.58339324)(260.41985512,295.56339813)
\curveto(260.22984463,295.45339337)(260.08484477,295.30839351)(259.98485512,295.12839813)
\curveto(259.89484496,294.94839387)(259.81484504,294.74339408)(259.74485512,294.51339813)
\curveto(259.70484515,294.38339444)(259.68484517,294.24839457)(259.68485512,294.10839813)
\curveto(259.68484517,293.97839484)(259.67484518,293.83339499)(259.65485512,293.67339813)
\curveto(259.64484521,293.6233952)(259.63484522,293.56339526)(259.62485512,293.49339813)
\curveto(259.62484523,293.4233954)(259.63484522,293.36339546)(259.65485512,293.31339813)
\lineto(259.65485512,293.14839813)
\lineto(259.65485512,292.96839813)
\curveto(259.66484519,292.9183959)(259.67484518,292.86339596)(259.68485512,292.80339813)
\curveto(259.69484516,292.75339607)(259.69984516,292.69839612)(259.69985512,292.63839813)
\curveto(259.70984515,292.57839624)(259.72484513,292.5233963)(259.74485512,292.47339813)
\curveto(259.79484506,292.28339654)(259.854845,292.10839671)(259.92485512,291.94839813)
\curveto(259.99484486,291.78839703)(260.09984476,291.65839716)(260.23985512,291.55839813)
\curveto(260.36984449,291.45839736)(260.50984435,291.38839743)(260.65985512,291.34839813)
\curveto(260.68984417,291.33839748)(260.71484414,291.33339749)(260.73485512,291.33339813)
\curveto(260.76484409,291.34339748)(260.79484406,291.34339748)(260.82485512,291.33339813)
\curveto(260.84484401,291.33339749)(260.87484398,291.32839749)(260.91485512,291.31839813)
\curveto(260.9548439,291.3183975)(260.98984387,291.3233975)(261.01985512,291.33339813)
\curveto(261.0598438,291.34339748)(261.09984376,291.34839747)(261.13985512,291.34839813)
\curveto(261.17984368,291.34839747)(261.21984364,291.35839746)(261.25985512,291.37839813)
\curveto(261.49984336,291.45839736)(261.69484316,291.59339723)(261.84485512,291.78339813)
\curveto(261.96484289,291.96339686)(262.0548428,292.16839665)(262.11485512,292.39839813)
\curveto(262.13484272,292.46839635)(262.14984271,292.53839628)(262.15985512,292.60839813)
\curveto(262.16984269,292.68839613)(262.18484267,292.76839605)(262.20485512,292.84839813)
\curveto(262.20484265,292.90839591)(262.20984265,292.95339587)(262.21985512,292.98339813)
\curveto(262.21984264,293.00339582)(262.21984264,293.02839579)(262.21985512,293.05839813)
\curveto(262.21984264,293.09839572)(262.22484263,293.12839569)(262.23485512,293.14839813)
\lineto(262.23485512,293.29839813)
}
}
{
\newrgbcolor{curcolor}{0 0 0}
\pscustom[linestyle=none,fillstyle=solid,fillcolor=curcolor]
{
\newpath
\moveto(188.35086098,300.90410858)
\lineto(191.95086098,300.90410858)
\lineto(192.59586098,300.90410858)
\curveto(192.67585445,300.90409816)(192.75085437,300.89909816)(192.82086098,300.88910858)
\curveto(192.89085423,300.88909817)(192.95085417,300.87909818)(193.00086098,300.85910858)
\curveto(193.07085405,300.82909823)(193.125854,300.76909829)(193.16586098,300.67910858)
\curveto(193.18585394,300.64909841)(193.19585393,300.60909845)(193.19586098,300.55910858)
\lineto(193.19586098,300.42410858)
\curveto(193.20585392,300.31409875)(193.20085392,300.20909885)(193.18086098,300.10910858)
\curveto(193.17085395,300.00909905)(193.13585399,299.93909912)(193.07586098,299.89910858)
\curveto(192.98585414,299.82909923)(192.85085427,299.79409927)(192.67086098,299.79410858)
\curveto(192.49085463,299.80409926)(192.3258548,299.80909925)(192.17586098,299.80910858)
\lineto(190.18086098,299.80910858)
\lineto(189.68586098,299.80910858)
\lineto(189.55086098,299.80910858)
\curveto(189.51085761,299.80909925)(189.47085765,299.80409926)(189.43086098,299.79410858)
\lineto(189.22086098,299.79410858)
\curveto(189.11085801,299.7640993)(189.03085809,299.72409934)(188.98086098,299.67410858)
\curveto(188.93085819,299.63409943)(188.89585823,299.57909948)(188.87586098,299.50910858)
\curveto(188.85585827,299.44909961)(188.84085828,299.37909968)(188.83086098,299.29910858)
\curveto(188.8208583,299.21909984)(188.80085832,299.12909993)(188.77086098,299.02910858)
\curveto(188.7208584,298.82910023)(188.68085844,298.62410044)(188.65086098,298.41410858)
\curveto(188.6208585,298.20410086)(188.58085854,297.99910106)(188.53086098,297.79910858)
\curveto(188.51085861,297.72910133)(188.50085862,297.6591014)(188.50086098,297.58910858)
\curveto(188.50085862,297.52910153)(188.49085863,297.4641016)(188.47086098,297.39410858)
\curveto(188.46085866,297.3641017)(188.45085867,297.32410174)(188.44086098,297.27410858)
\curveto(188.44085868,297.23410183)(188.44585868,297.19410187)(188.45586098,297.15410858)
\curveto(188.47585865,297.10410196)(188.50085862,297.059102)(188.53086098,297.01910858)
\curveto(188.57085855,296.98910207)(188.63085849,296.98410208)(188.71086098,297.00410858)
\curveto(188.77085835,297.02410204)(188.83085829,297.04910201)(188.89086098,297.07910858)
\curveto(188.95085817,297.11910194)(189.01085811,297.15410191)(189.07086098,297.18410858)
\curveto(189.13085799,297.20410186)(189.18085794,297.21910184)(189.22086098,297.22910858)
\curveto(189.41085771,297.30910175)(189.61585751,297.3641017)(189.83586098,297.39410858)
\curveto(190.06585706,297.42410164)(190.29585683,297.43410163)(190.52586098,297.42410858)
\curveto(190.76585636,297.42410164)(190.99585613,297.39910166)(191.21586098,297.34910858)
\curveto(191.43585569,297.30910175)(191.63585549,297.24910181)(191.81586098,297.16910858)
\curveto(191.86585526,297.14910191)(191.91085521,297.12910193)(191.95086098,297.10910858)
\curveto(192.00085512,297.08910197)(192.05085507,297.064102)(192.10086098,297.03410858)
\curveto(192.45085467,296.82410224)(192.73085439,296.59410247)(192.94086098,296.34410858)
\curveto(193.16085396,296.09410297)(193.35585377,295.76910329)(193.52586098,295.36910858)
\curveto(193.57585355,295.2591038)(193.61085351,295.14910391)(193.63086098,295.03910858)
\curveto(193.65085347,294.92910413)(193.67585345,294.81410425)(193.70586098,294.69410858)
\curveto(193.71585341,294.6641044)(193.7208534,294.61910444)(193.72086098,294.55910858)
\curveto(193.74085338,294.49910456)(193.75085337,294.42910463)(193.75086098,294.34910858)
\curveto(193.75085337,294.27910478)(193.76085336,294.21410485)(193.78086098,294.15410858)
\lineto(193.78086098,293.98910858)
\curveto(193.79085333,293.93910512)(193.79585333,293.86910519)(193.79586098,293.77910858)
\curveto(193.79585333,293.68910537)(193.78585334,293.61910544)(193.76586098,293.56910858)
\curveto(193.74585338,293.50910555)(193.74085338,293.44910561)(193.75086098,293.38910858)
\curveto(193.76085336,293.33910572)(193.75585337,293.28910577)(193.73586098,293.23910858)
\curveto(193.69585343,293.07910598)(193.66085346,292.92910613)(193.63086098,292.78910858)
\curveto(193.60085352,292.64910641)(193.55585357,292.51410655)(193.49586098,292.38410858)
\curveto(193.33585379,292.01410705)(193.11585401,291.67910738)(192.83586098,291.37910858)
\curveto(192.55585457,291.07910798)(192.23585489,290.84910821)(191.87586098,290.68910858)
\curveto(191.70585542,290.60910845)(191.50585562,290.53410853)(191.27586098,290.46410858)
\curveto(191.16585596,290.42410864)(191.05085607,290.39910866)(190.93086098,290.38910858)
\curveto(190.81085631,290.37910868)(190.69085643,290.3591087)(190.57086098,290.32910858)
\curveto(190.5208566,290.30910875)(190.46585666,290.30910875)(190.40586098,290.32910858)
\curveto(190.34585678,290.33910872)(190.28585684,290.33410873)(190.22586098,290.31410858)
\curveto(190.125857,290.29410877)(190.0258571,290.29410877)(189.92586098,290.31410858)
\lineto(189.79086098,290.31410858)
\curveto(189.74085738,290.33410873)(189.68085744,290.34410872)(189.61086098,290.34410858)
\curveto(189.55085757,290.33410873)(189.49585763,290.33910872)(189.44586098,290.35910858)
\curveto(189.40585772,290.36910869)(189.37085775,290.37410869)(189.34086098,290.37410858)
\curveto(189.31085781,290.37410869)(189.27585785,290.37910868)(189.23586098,290.38910858)
\lineto(188.96586098,290.44910858)
\curveto(188.87585825,290.46910859)(188.79085833,290.49910856)(188.71086098,290.53910858)
\curveto(188.37085875,290.67910838)(188.08085904,290.83410823)(187.84086098,291.00410858)
\curveto(187.60085952,291.18410788)(187.38085974,291.41410765)(187.18086098,291.69410858)
\curveto(187.03086009,291.92410714)(186.91586021,292.1641069)(186.83586098,292.41410858)
\curveto(186.81586031,292.4641066)(186.80586032,292.50910655)(186.80586098,292.54910858)
\curveto(186.80586032,292.59910646)(186.79586033,292.64910641)(186.77586098,292.69910858)
\curveto(186.75586037,292.7591063)(186.74086038,292.83910622)(186.73086098,292.93910858)
\curveto(186.73086039,293.03910602)(186.75086037,293.11410595)(186.79086098,293.16410858)
\curveto(186.84086028,293.24410582)(186.9208602,293.28910577)(187.03086098,293.29910858)
\curveto(187.14085998,293.30910575)(187.25585987,293.31410575)(187.37586098,293.31410858)
\lineto(187.54086098,293.31410858)
\curveto(187.60085952,293.31410575)(187.65585947,293.30410576)(187.70586098,293.28410858)
\curveto(187.79585933,293.2641058)(187.86585926,293.22410584)(187.91586098,293.16410858)
\curveto(187.98585914,293.07410599)(188.03085909,292.9641061)(188.05086098,292.83410858)
\curveto(188.08085904,292.71410635)(188.125859,292.60910645)(188.18586098,292.51910858)
\curveto(188.37585875,292.17910688)(188.63585849,291.90910715)(188.96586098,291.70910858)
\curveto(189.06585806,291.64910741)(189.17085795,291.59910746)(189.28086098,291.55910858)
\curveto(189.40085772,291.52910753)(189.5208576,291.49410757)(189.64086098,291.45410858)
\curveto(189.81085731,291.40410766)(190.01585711,291.38410768)(190.25586098,291.39410858)
\curveto(190.50585662,291.41410765)(190.70585642,291.44910761)(190.85586098,291.49910858)
\curveto(191.2258559,291.61910744)(191.51585561,291.77910728)(191.72586098,291.97910858)
\curveto(191.94585518,292.18910687)(192.125855,292.46910659)(192.26586098,292.81910858)
\curveto(192.31585481,292.91910614)(192.34585478,293.02410604)(192.35586098,293.13410858)
\curveto(192.37585475,293.24410582)(192.40085472,293.3591057)(192.43086098,293.47910858)
\lineto(192.43086098,293.58410858)
\curveto(192.44085468,293.62410544)(192.44585468,293.6641054)(192.44586098,293.70410858)
\curveto(192.45585467,293.73410533)(192.45585467,293.76910529)(192.44586098,293.80910858)
\lineto(192.44586098,293.92910858)
\curveto(192.44585468,294.18910487)(192.41585471,294.43410463)(192.35586098,294.66410858)
\curveto(192.24585488,295.01410405)(192.09085503,295.30910375)(191.89086098,295.54910858)
\curveto(191.69085543,295.79910326)(191.43085569,295.99410307)(191.11086098,296.13410858)
\lineto(190.93086098,296.19410858)
\curveto(190.88085624,296.21410285)(190.8208563,296.23410283)(190.75086098,296.25410858)
\curveto(190.70085642,296.27410279)(190.64085648,296.28410278)(190.57086098,296.28410858)
\curveto(190.51085661,296.29410277)(190.44585668,296.30910275)(190.37586098,296.32910858)
\lineto(190.22586098,296.32910858)
\curveto(190.18585694,296.34910271)(190.13085699,296.3591027)(190.06086098,296.35910858)
\curveto(190.00085712,296.3591027)(189.94585718,296.34910271)(189.89586098,296.32910858)
\lineto(189.79086098,296.32910858)
\curveto(189.76085736,296.32910273)(189.7258574,296.32410274)(189.68586098,296.31410858)
\lineto(189.44586098,296.25410858)
\curveto(189.36585776,296.24410282)(189.28585784,296.22410284)(189.20586098,296.19410858)
\curveto(188.96585816,296.09410297)(188.73585839,295.9591031)(188.51586098,295.78910858)
\curveto(188.4258587,295.71910334)(188.34085878,295.64410342)(188.26086098,295.56410858)
\curveto(188.18085894,295.49410357)(188.08085904,295.43910362)(187.96086098,295.39910858)
\curveto(187.87085925,295.36910369)(187.73085939,295.3591037)(187.54086098,295.36910858)
\curveto(187.36085976,295.37910368)(187.24085988,295.40410366)(187.18086098,295.44410858)
\curveto(187.13085999,295.48410358)(187.09086003,295.54410352)(187.06086098,295.62410858)
\curveto(187.04086008,295.70410336)(187.04086008,295.78910327)(187.06086098,295.87910858)
\curveto(187.09086003,295.99910306)(187.11086001,296.11910294)(187.12086098,296.23910858)
\curveto(187.14085998,296.36910269)(187.16585996,296.49410257)(187.19586098,296.61410858)
\curveto(187.21585991,296.65410241)(187.2208599,296.68910237)(187.21086098,296.71910858)
\curveto(187.21085991,296.7591023)(187.2208599,296.80410226)(187.24086098,296.85410858)
\curveto(187.26085986,296.94410212)(187.27585985,297.03410203)(187.28586098,297.12410858)
\curveto(187.29585983,297.22410184)(187.31585981,297.31910174)(187.34586098,297.40910858)
\curveto(187.35585977,297.46910159)(187.36085976,297.52910153)(187.36086098,297.58910858)
\curveto(187.37085975,297.64910141)(187.38585974,297.70910135)(187.40586098,297.76910858)
\curveto(187.45585967,297.96910109)(187.49085963,298.17410089)(187.51086098,298.38410858)
\curveto(187.54085958,298.60410046)(187.58085954,298.81410025)(187.63086098,299.01410858)
\curveto(187.66085946,299.11409995)(187.68085944,299.21409985)(187.69086098,299.31410858)
\curveto(187.70085942,299.41409965)(187.71585941,299.51409955)(187.73586098,299.61410858)
\curveto(187.74585938,299.64409942)(187.75085937,299.68409938)(187.75086098,299.73410858)
\curveto(187.78085934,299.84409922)(187.80085932,299.94909911)(187.81086098,300.04910858)
\curveto(187.83085929,300.1590989)(187.85585927,300.26909879)(187.88586098,300.37910858)
\curveto(187.90585922,300.4590986)(187.9208592,300.52909853)(187.93086098,300.58910858)
\curveto(187.94085918,300.6590984)(187.96585916,300.71909834)(188.00586098,300.76910858)
\curveto(188.0258591,300.79909826)(188.05585907,300.81909824)(188.09586098,300.82910858)
\curveto(188.13585899,300.84909821)(188.18085894,300.86909819)(188.23086098,300.88910858)
\curveto(188.29085883,300.88909817)(188.33085879,300.89409817)(188.35086098,300.90410858)
}
}
{
\newrgbcolor{curcolor}{0 0 0}
\pscustom[linestyle=none,fillstyle=solid,fillcolor=curcolor]
{
\newpath
\moveto(196.14547035,292.12910858)
\lineto(196.44547035,292.12910858)
\curveto(196.55546829,292.13910692)(196.66046819,292.13910692)(196.76047035,292.12910858)
\curveto(196.87046798,292.12910693)(196.97046788,292.11910694)(197.06047035,292.09910858)
\curveto(197.1504677,292.08910697)(197.22046763,292.064107)(197.27047035,292.02410858)
\curveto(197.29046756,292.00410706)(197.30546754,291.97410709)(197.31547035,291.93410858)
\curveto(197.33546751,291.89410717)(197.35546749,291.84910721)(197.37547035,291.79910858)
\lineto(197.37547035,291.72410858)
\curveto(197.38546746,291.67410739)(197.38546746,291.61910744)(197.37547035,291.55910858)
\lineto(197.37547035,291.40910858)
\lineto(197.37547035,290.92910858)
\curveto(197.37546747,290.7591083)(197.33546751,290.63910842)(197.25547035,290.56910858)
\curveto(197.18546766,290.51910854)(197.09546775,290.49410857)(196.98547035,290.49410858)
\lineto(196.65547035,290.49410858)
\lineto(196.20547035,290.49410858)
\curveto(196.05546879,290.49410857)(195.94046891,290.52410854)(195.86047035,290.58410858)
\curveto(195.82046903,290.61410845)(195.79046906,290.6641084)(195.77047035,290.73410858)
\curveto(195.7504691,290.81410825)(195.73546911,290.89910816)(195.72547035,290.98910858)
\lineto(195.72547035,291.27410858)
\curveto(195.73546911,291.37410769)(195.74046911,291.4591076)(195.74047035,291.52910858)
\lineto(195.74047035,291.72410858)
\curveto(195.74046911,291.78410728)(195.7504691,291.83910722)(195.77047035,291.88910858)
\curveto(195.81046904,291.99910706)(195.88046897,292.06910699)(195.98047035,292.09910858)
\curveto(196.01046884,292.09910696)(196.06546878,292.10910695)(196.14547035,292.12910858)
}
}
{
\newrgbcolor{curcolor}{0 0 0}
\pscustom[linestyle=none,fillstyle=solid,fillcolor=curcolor]
{
\newpath
\moveto(202.4656266,301.09910858)
\curveto(204.09562116,301.12909793)(205.14562011,300.57409849)(205.6156266,299.43410858)
\curveto(205.71561954,299.20409986)(205.78061948,298.91410015)(205.8106266,298.56410858)
\curveto(205.85061941,298.22410084)(205.82561943,297.91410115)(205.7356266,297.63410858)
\curveto(205.64561961,297.37410169)(205.52561973,297.14910191)(205.3756266,296.95910858)
\curveto(205.3556199,296.91910214)(205.33061993,296.88410218)(205.3006266,296.85410858)
\curveto(205.27061999,296.83410223)(205.24562001,296.80910225)(205.2256266,296.77910858)
\lineto(205.1356266,296.65910858)
\curveto(205.10562015,296.62910243)(205.07062019,296.60410246)(205.0306266,296.58410858)
\curveto(204.98062028,296.53410253)(204.92562033,296.48910257)(204.8656266,296.44910858)
\curveto(204.81562044,296.40910265)(204.77062049,296.3591027)(204.7306266,296.29910858)
\curveto(204.69062057,296.2591028)(204.67562058,296.20910285)(204.6856266,296.14910858)
\curveto(204.69562056,296.09910296)(204.72562053,296.05410301)(204.7756266,296.01410858)
\curveto(204.82562043,295.97410309)(204.88062038,295.93410313)(204.9406266,295.89410858)
\curveto(205.01062025,295.8641032)(205.07562018,295.83410323)(205.1356266,295.80410858)
\curveto(205.19562006,295.77410329)(205.24562001,295.74410332)(205.2856266,295.71410858)
\curveto(205.60561965,295.49410357)(205.8606194,295.18410388)(206.0506266,294.78410858)
\curveto(206.09061917,294.69410437)(206.12061914,294.59910446)(206.1406266,294.49910858)
\curveto(206.17061909,294.40910465)(206.19561906,294.31910474)(206.2156266,294.22910858)
\curveto(206.22561903,294.17910488)(206.23061903,294.12910493)(206.2306266,294.07910858)
\curveto(206.24061902,294.03910502)(206.25061901,293.99410507)(206.2606266,293.94410858)
\curveto(206.27061899,293.89410517)(206.27061899,293.84410522)(206.2606266,293.79410858)
\curveto(206.25061901,293.74410532)(206.255619,293.69410537)(206.2756266,293.64410858)
\curveto(206.28561897,293.59410547)(206.29061897,293.53410553)(206.2906266,293.46410858)
\curveto(206.29061897,293.39410567)(206.28061898,293.33410573)(206.2606266,293.28410858)
\lineto(206.2606266,293.05910858)
\lineto(206.2006266,292.81910858)
\curveto(206.19061907,292.74910631)(206.17561908,292.67910638)(206.1556266,292.60910858)
\curveto(206.12561913,292.51910654)(206.09561916,292.43410663)(206.0656266,292.35410858)
\curveto(206.04561921,292.27410679)(206.01561924,292.19410687)(205.9756266,292.11410858)
\curveto(205.9556193,292.05410701)(205.92561933,291.99410707)(205.8856266,291.93410858)
\curveto(205.8556194,291.88410718)(205.82061944,291.83410723)(205.7806266,291.78410858)
\curveto(205.58061968,291.47410759)(205.33061993,291.21410785)(205.0306266,291.00410858)
\curveto(204.73062053,290.80410826)(204.38562087,290.63910842)(203.9956266,290.50910858)
\curveto(203.87562138,290.46910859)(203.74562151,290.44410862)(203.6056266,290.43410858)
\curveto(203.47562178,290.41410865)(203.34062192,290.38910867)(203.2006266,290.35910858)
\curveto(203.13062213,290.34910871)(203.0606222,290.34410872)(202.9906266,290.34410858)
\curveto(202.93062233,290.34410872)(202.86562239,290.33910872)(202.7956266,290.32910858)
\curveto(202.7556225,290.31910874)(202.69562256,290.31410875)(202.6156266,290.31410858)
\curveto(202.54562271,290.31410875)(202.49562276,290.31910874)(202.4656266,290.32910858)
\curveto(202.41562284,290.33910872)(202.37062289,290.34410872)(202.3306266,290.34410858)
\lineto(202.2106266,290.34410858)
\curveto(202.11062315,290.3641087)(202.01062325,290.37910868)(201.9106266,290.38910858)
\curveto(201.81062345,290.39910866)(201.71562354,290.41410865)(201.6256266,290.43410858)
\curveto(201.51562374,290.4641086)(201.40562385,290.48910857)(201.2956266,290.50910858)
\curveto(201.19562406,290.53910852)(201.09062417,290.57910848)(200.9806266,290.62910858)
\curveto(200.61062465,290.78910827)(200.29562496,290.98910807)(200.0356266,291.22910858)
\curveto(199.77562548,291.47910758)(199.56562569,291.78910727)(199.4056266,292.15910858)
\curveto(199.36562589,292.24910681)(199.33062593,292.34410672)(199.3006266,292.44410858)
\curveto(199.27062599,292.54410652)(199.24062602,292.64910641)(199.2106266,292.75910858)
\curveto(199.19062607,292.80910625)(199.18062608,292.8591062)(199.1806266,292.90910858)
\curveto(199.18062608,292.96910609)(199.17062609,293.02910603)(199.1506266,293.08910858)
\curveto(199.13062613,293.14910591)(199.12062614,293.22910583)(199.1206266,293.32910858)
\curveto(199.12062614,293.42910563)(199.13562612,293.50410556)(199.1656266,293.55410858)
\curveto(199.17562608,293.58410548)(199.19062607,293.60910545)(199.2106266,293.62910858)
\lineto(199.2706266,293.68910858)
\curveto(199.31062595,293.70910535)(199.37062589,293.72410534)(199.4506266,293.73410858)
\curveto(199.54062572,293.74410532)(199.63062563,293.74910531)(199.7206266,293.74910858)
\curveto(199.81062545,293.74910531)(199.89562536,293.74410532)(199.9756266,293.73410858)
\curveto(200.06562519,293.72410534)(200.13062513,293.71410535)(200.1706266,293.70410858)
\curveto(200.19062507,293.68410538)(200.21062505,293.66910539)(200.2306266,293.65910858)
\curveto(200.25062501,293.6591054)(200.27062499,293.64910541)(200.2906266,293.62910858)
\curveto(200.3606249,293.53910552)(200.40062486,293.42410564)(200.4106266,293.28410858)
\curveto(200.43062483,293.14410592)(200.4606248,293.01910604)(200.5006266,292.90910858)
\lineto(200.6506266,292.54910858)
\curveto(200.70062456,292.43910662)(200.76562449,292.33410673)(200.8456266,292.23410858)
\curveto(200.86562439,292.20410686)(200.88562437,292.17910688)(200.9056266,292.15910858)
\curveto(200.93562432,292.13910692)(200.9606243,292.11410695)(200.9806266,292.08410858)
\curveto(201.02062424,292.02410704)(201.0556242,291.97910708)(201.0856266,291.94910858)
\curveto(201.12562413,291.91910714)(201.1606241,291.88910717)(201.1906266,291.85910858)
\curveto(201.23062403,291.82910723)(201.27562398,291.79910726)(201.3256266,291.76910858)
\curveto(201.41562384,291.70910735)(201.51062375,291.6591074)(201.6106266,291.61910858)
\lineto(201.9406266,291.49910858)
\curveto(202.09062317,291.44910761)(202.29062297,291.41910764)(202.5406266,291.40910858)
\curveto(202.79062247,291.39910766)(203.00062226,291.41910764)(203.1706266,291.46910858)
\curveto(203.25062201,291.48910757)(203.32062194,291.50410756)(203.3806266,291.51410858)
\lineto(203.5906266,291.57410858)
\curveto(203.87062139,291.69410737)(204.11062115,291.84410722)(204.3106266,292.02410858)
\curveto(204.52062074,292.20410686)(204.68562057,292.43410663)(204.8056266,292.71410858)
\curveto(204.83562042,292.78410628)(204.8556204,292.85410621)(204.8656266,292.92410858)
\lineto(204.9256266,293.16410858)
\curveto(204.96562029,293.30410576)(204.97562028,293.4641056)(204.9556266,293.64410858)
\curveto(204.93562032,293.83410523)(204.90562035,293.98410508)(204.8656266,294.09410858)
\curveto(204.73562052,294.47410459)(204.55062071,294.7641043)(204.3106266,294.96410858)
\curveto(204.08062118,295.1641039)(203.77062149,295.32410374)(203.3806266,295.44410858)
\curveto(203.27062199,295.47410359)(203.15062211,295.49410357)(203.0206266,295.50410858)
\curveto(202.90062236,295.51410355)(202.77562248,295.51910354)(202.6456266,295.51910858)
\curveto(202.48562277,295.51910354)(202.34562291,295.52410354)(202.2256266,295.53410858)
\curveto(202.10562315,295.54410352)(202.02062324,295.60410346)(201.9706266,295.71410858)
\curveto(201.95062331,295.74410332)(201.94062332,295.77910328)(201.9406266,295.81910858)
\lineto(201.9406266,295.95410858)
\curveto(201.93062333,296.05410301)(201.93062333,296.14910291)(201.9406266,296.23910858)
\curveto(201.9606233,296.32910273)(202.00062326,296.39410267)(202.0606266,296.43410858)
\curveto(202.10062316,296.4641026)(202.14062312,296.48410258)(202.1806266,296.49410858)
\curveto(202.23062303,296.50410256)(202.28562297,296.51410255)(202.3456266,296.52410858)
\curveto(202.36562289,296.53410253)(202.39062287,296.53410253)(202.4206266,296.52410858)
\curveto(202.45062281,296.52410254)(202.47562278,296.52910253)(202.4956266,296.53910858)
\lineto(202.6306266,296.53910858)
\curveto(202.74062252,296.5591025)(202.84062242,296.56910249)(202.9306266,296.56910858)
\curveto(203.03062223,296.57910248)(203.12562213,296.59910246)(203.2156266,296.62910858)
\curveto(203.53562172,296.73910232)(203.79062147,296.88410218)(203.9806266,297.06410858)
\curveto(204.17062109,297.24410182)(204.32062094,297.49410157)(204.4306266,297.81410858)
\curveto(204.4606208,297.91410115)(204.48062078,298.03910102)(204.4906266,298.18910858)
\curveto(204.51062075,298.34910071)(204.50562075,298.49410057)(204.4756266,298.62410858)
\curveto(204.4556208,298.69410037)(204.43562082,298.7591003)(204.4156266,298.81910858)
\curveto(204.40562085,298.88910017)(204.38562087,298.95410011)(204.3556266,299.01410858)
\curveto(204.255621,299.25409981)(204.11062115,299.44409962)(203.9206266,299.58410858)
\curveto(203.73062153,299.72409934)(203.50562175,299.83409923)(203.2456266,299.91410858)
\curveto(203.18562207,299.93409913)(203.12562213,299.94409912)(203.0656266,299.94410858)
\curveto(203.00562225,299.94409912)(202.94062232,299.95409911)(202.8706266,299.97410858)
\curveto(202.79062247,299.99409907)(202.69562256,300.00409906)(202.5856266,300.00410858)
\curveto(202.47562278,300.00409906)(202.38062288,299.99409907)(202.3006266,299.97410858)
\curveto(202.25062301,299.95409911)(202.20062306,299.94409912)(202.1506266,299.94410858)
\curveto(202.11062315,299.94409912)(202.06562319,299.93409913)(202.0156266,299.91410858)
\curveto(201.83562342,299.8640992)(201.66562359,299.78909927)(201.5056266,299.68910858)
\curveto(201.3556239,299.59909946)(201.22562403,299.48409958)(201.1156266,299.34410858)
\curveto(201.02562423,299.22409984)(200.94562431,299.09409997)(200.8756266,298.95410858)
\curveto(200.80562445,298.81410025)(200.74062452,298.6591004)(200.6806266,298.48910858)
\curveto(200.65062461,298.37910068)(200.63062463,298.2591008)(200.6206266,298.12910858)
\curveto(200.61062465,298.00910105)(200.57562468,297.90910115)(200.5156266,297.82910858)
\curveto(200.49562476,297.78910127)(200.43562482,297.74910131)(200.3356266,297.70910858)
\curveto(200.29562496,297.69910136)(200.23562502,297.68910137)(200.1556266,297.67910858)
\lineto(199.9006266,297.67910858)
\curveto(199.81062545,297.68910137)(199.72562553,297.69910136)(199.6456266,297.70910858)
\curveto(199.57562568,297.71910134)(199.52562573,297.73410133)(199.4956266,297.75410858)
\curveto(199.4556258,297.78410128)(199.42062584,297.83910122)(199.3906266,297.91910858)
\curveto(199.3606259,297.99910106)(199.3556259,298.08410098)(199.3756266,298.17410858)
\curveto(199.38562587,298.22410084)(199.39062587,298.27410079)(199.3906266,298.32410858)
\lineto(199.4206266,298.50410858)
\curveto(199.45062581,298.60410046)(199.47562578,298.70410036)(199.4956266,298.80410858)
\curveto(199.52562573,298.90410016)(199.5606257,298.99410007)(199.6006266,299.07410858)
\curveto(199.65062561,299.18409988)(199.69562556,299.28909977)(199.7356266,299.38910858)
\curveto(199.77562548,299.49909956)(199.82562543,299.60409946)(199.8856266,299.70410858)
\curveto(200.21562504,300.24409882)(200.68562457,300.63909842)(201.2956266,300.88910858)
\curveto(201.41562384,300.93909812)(201.54062372,300.97409809)(201.6706266,300.99410858)
\curveto(201.81062345,301.01409805)(201.95062331,301.03909802)(202.0906266,301.06910858)
\curveto(202.15062311,301.07909798)(202.21062305,301.08409798)(202.2706266,301.08410858)
\curveto(202.34062292,301.08409798)(202.40562285,301.08909797)(202.4656266,301.09910858)
}
}
{
\newrgbcolor{curcolor}{0 0 0}
\pscustom[linestyle=none,fillstyle=solid,fillcolor=curcolor]
{
\newpath
\moveto(217.50523598,299.01410858)
\curveto(217.30522568,298.72410034)(217.09522589,298.43910062)(216.87523598,298.15910858)
\curveto(216.66522632,297.87910118)(216.46022652,297.59410147)(216.26023598,297.30410858)
\curveto(215.66022732,296.45410261)(215.05522793,295.61410345)(214.44523598,294.78410858)
\curveto(213.83522915,293.9641051)(213.23022975,293.12910593)(212.63023598,292.27910858)
\lineto(212.12023598,291.55910858)
\lineto(211.61023598,290.86910858)
\curveto(211.53023145,290.7591083)(211.45023153,290.64410842)(211.37023598,290.52410858)
\curveto(211.29023169,290.40410866)(211.19523179,290.30910875)(211.08523598,290.23910858)
\curveto(211.04523194,290.21910884)(210.980232,290.20410886)(210.89023598,290.19410858)
\curveto(210.81023217,290.17410889)(210.72023226,290.1641089)(210.62023598,290.16410858)
\curveto(210.52023246,290.1641089)(210.42523256,290.16910889)(210.33523598,290.17910858)
\curveto(210.25523273,290.18910887)(210.19523279,290.20910885)(210.15523598,290.23910858)
\curveto(210.12523286,290.2591088)(210.10023288,290.29410877)(210.08023598,290.34410858)
\curveto(210.07023291,290.38410868)(210.07523291,290.42910863)(210.09523598,290.47910858)
\curveto(210.13523285,290.5591085)(210.1802328,290.63410843)(210.23023598,290.70410858)
\curveto(210.29023269,290.78410828)(210.34523264,290.8641082)(210.39523598,290.94410858)
\curveto(210.63523235,291.28410778)(210.8802321,291.61910744)(211.13023598,291.94910858)
\curveto(211.3802316,292.27910678)(211.62023136,292.61410645)(211.85023598,292.95410858)
\curveto(212.01023097,293.17410589)(212.17023081,293.38910567)(212.33023598,293.59910858)
\curveto(212.49023049,293.80910525)(212.65023033,294.02410504)(212.81023598,294.24410858)
\curveto(213.17022981,294.7641043)(213.53522945,295.27410379)(213.90523598,295.77410858)
\curveto(214.27522871,296.27410279)(214.64522834,296.78410228)(215.01523598,297.30410858)
\curveto(215.15522783,297.50410156)(215.29522769,297.69910136)(215.43523598,297.88910858)
\curveto(215.5852274,298.07910098)(215.73022725,298.27410079)(215.87023598,298.47410858)
\curveto(216.0802269,298.77410029)(216.29522669,299.07409999)(216.51523598,299.37410858)
\lineto(217.17523598,300.27410858)
\lineto(217.35523598,300.54410858)
\lineto(217.56523598,300.81410858)
\lineto(217.68523598,300.99410858)
\curveto(217.73522525,301.05409801)(217.7852252,301.10909795)(217.83523598,301.15910858)
\curveto(217.90522508,301.20909785)(217.980225,301.24409782)(218.06023598,301.26410858)
\curveto(218.0802249,301.27409779)(218.10522488,301.27409779)(218.13523598,301.26410858)
\curveto(218.17522481,301.2640978)(218.20522478,301.27409779)(218.22523598,301.29410858)
\curveto(218.34522464,301.29409777)(218.4802245,301.28909777)(218.63023598,301.27910858)
\curveto(218.7802242,301.27909778)(218.87022411,301.23409783)(218.90023598,301.14410858)
\curveto(218.92022406,301.11409795)(218.92522406,301.07909798)(218.91523598,301.03910858)
\curveto(218.90522408,300.99909806)(218.89022409,300.96909809)(218.87023598,300.94910858)
\curveto(218.83022415,300.86909819)(218.79022419,300.79909826)(218.75023598,300.73910858)
\curveto(218.71022427,300.67909838)(218.66522432,300.61909844)(218.61523598,300.55910858)
\lineto(218.04523598,299.77910858)
\curveto(217.86522512,299.52909953)(217.6852253,299.27409979)(217.50523598,299.01410858)
\moveto(210.65023598,295.11410858)
\curveto(210.60023238,295.13410393)(210.55023243,295.13910392)(210.50023598,295.12910858)
\curveto(210.45023253,295.11910394)(210.40023258,295.12410394)(210.35023598,295.14410858)
\curveto(210.24023274,295.1641039)(210.13523285,295.18410388)(210.03523598,295.20410858)
\curveto(209.94523304,295.23410383)(209.85023313,295.27410379)(209.75023598,295.32410858)
\curveto(209.42023356,295.4641036)(209.16523382,295.6591034)(208.98523598,295.90910858)
\curveto(208.80523418,296.16910289)(208.66023432,296.47910258)(208.55023598,296.83910858)
\curveto(208.52023446,296.91910214)(208.50023448,296.99910206)(208.49023598,297.07910858)
\curveto(208.4802345,297.16910189)(208.46523452,297.25410181)(208.44523598,297.33410858)
\curveto(208.43523455,297.38410168)(208.43023455,297.44910161)(208.43023598,297.52910858)
\curveto(208.42023456,297.5591015)(208.41523457,297.58910147)(208.41523598,297.61910858)
\curveto(208.41523457,297.6591014)(208.41023457,297.69410137)(208.40023598,297.72410858)
\lineto(208.40023598,297.87410858)
\curveto(208.39023459,297.92410114)(208.3852346,297.98410108)(208.38523598,298.05410858)
\curveto(208.3852346,298.13410093)(208.39023459,298.19910086)(208.40023598,298.24910858)
\lineto(208.40023598,298.41410858)
\curveto(208.42023456,298.4641006)(208.42523456,298.50910055)(208.41523598,298.54910858)
\curveto(208.41523457,298.59910046)(208.42023456,298.64410042)(208.43023598,298.68410858)
\curveto(208.44023454,298.72410034)(208.44523454,298.7591003)(208.44523598,298.78910858)
\curveto(208.44523454,298.82910023)(208.45023453,298.86910019)(208.46023598,298.90910858)
\curveto(208.49023449,299.01910004)(208.51023447,299.12909993)(208.52023598,299.23910858)
\curveto(208.54023444,299.3590997)(208.57523441,299.47409959)(208.62523598,299.58410858)
\curveto(208.76523422,299.92409914)(208.92523406,300.19909886)(209.10523598,300.40910858)
\curveto(209.29523369,300.62909843)(209.56523342,300.80909825)(209.91523598,300.94910858)
\curveto(209.99523299,300.97909808)(210.0802329,300.99909806)(210.17023598,301.00910858)
\curveto(210.26023272,301.02909803)(210.35523263,301.04909801)(210.45523598,301.06910858)
\curveto(210.4852325,301.07909798)(210.54023244,301.07909798)(210.62023598,301.06910858)
\curveto(210.70023228,301.06909799)(210.75023223,301.07909798)(210.77023598,301.09910858)
\curveto(211.33023165,301.10909795)(211.7802312,300.99909806)(212.12023598,300.76910858)
\curveto(212.47023051,300.53909852)(212.73023025,300.23409883)(212.90023598,299.85410858)
\curveto(212.94023004,299.7640993)(212.97523001,299.66909939)(213.00523598,299.56910858)
\curveto(213.03522995,299.46909959)(213.06022992,299.36909969)(213.08023598,299.26910858)
\curveto(213.10022988,299.23909982)(213.10522988,299.20909985)(213.09523598,299.17910858)
\curveto(213.09522989,299.14909991)(213.10022988,299.11909994)(213.11023598,299.08910858)
\curveto(213.14022984,298.97910008)(213.16022982,298.85410021)(213.17023598,298.71410858)
\curveto(213.1802298,298.58410048)(213.19022979,298.44910061)(213.20023598,298.30910858)
\lineto(213.20023598,298.14410858)
\curveto(213.21022977,298.08410098)(213.21022977,298.02910103)(213.20023598,297.97910858)
\curveto(213.19022979,297.92910113)(213.1852298,297.87910118)(213.18523598,297.82910858)
\lineto(213.18523598,297.69410858)
\curveto(213.17522981,297.65410141)(213.17022981,297.61410145)(213.17023598,297.57410858)
\curveto(213.1802298,297.53410153)(213.17522981,297.48910157)(213.15523598,297.43910858)
\curveto(213.13522985,297.32910173)(213.11522987,297.22410184)(213.09523598,297.12410858)
\curveto(213.0852299,297.02410204)(213.06522992,296.92410214)(213.03523598,296.82410858)
\curveto(212.90523008,296.4641026)(212.74023024,296.14910291)(212.54023598,295.87910858)
\curveto(212.34023064,295.60910345)(212.06523092,295.40410366)(211.71523598,295.26410858)
\curveto(211.63523135,295.23410383)(211.55023143,295.20910385)(211.46023598,295.18910858)
\lineto(211.19023598,295.12910858)
\curveto(211.14023184,295.11910394)(211.09523189,295.11410395)(211.05523598,295.11410858)
\curveto(211.01523197,295.12410394)(210.97523201,295.12410394)(210.93523598,295.11410858)
\curveto(210.83523215,295.09410397)(210.74023224,295.09410397)(210.65023598,295.11410858)
\moveto(209.81023598,296.50910858)
\curveto(209.85023313,296.43910262)(209.89023309,296.37410269)(209.93023598,296.31410858)
\curveto(209.97023301,296.2641028)(210.02023296,296.21410285)(210.08023598,296.16410858)
\lineto(210.23023598,296.04410858)
\curveto(210.29023269,296.01410305)(210.35523263,295.98910307)(210.42523598,295.96910858)
\curveto(210.46523252,295.94910311)(210.50023248,295.93910312)(210.53023598,295.93910858)
\curveto(210.57023241,295.94910311)(210.61023237,295.94410312)(210.65023598,295.92410858)
\curveto(210.6802323,295.92410314)(210.72023226,295.91910314)(210.77023598,295.90910858)
\curveto(210.82023216,295.90910315)(210.86023212,295.91410315)(210.89023598,295.92410858)
\lineto(211.11523598,295.96910858)
\curveto(211.36523162,296.04910301)(211.55023143,296.17410289)(211.67023598,296.34410858)
\curveto(211.75023123,296.44410262)(211.82023116,296.57410249)(211.88023598,296.73410858)
\curveto(211.96023102,296.91410215)(212.02023096,297.13910192)(212.06023598,297.40910858)
\curveto(212.10023088,297.68910137)(212.11523087,297.96910109)(212.10523598,298.24910858)
\curveto(212.09523089,298.53910052)(212.06523092,298.81410025)(212.01523598,299.07410858)
\curveto(211.96523102,299.33409973)(211.89023109,299.54409952)(211.79023598,299.70410858)
\curveto(211.67023131,299.90409916)(211.52023146,300.05409901)(211.34023598,300.15410858)
\curveto(211.26023172,300.20409886)(211.17023181,300.23409883)(211.07023598,300.24410858)
\curveto(210.97023201,300.2640988)(210.86523212,300.27409879)(210.75523598,300.27410858)
\curveto(210.73523225,300.2640988)(210.71023227,300.2590988)(210.68023598,300.25910858)
\curveto(210.66023232,300.26909879)(210.64023234,300.26909879)(210.62023598,300.25910858)
\curveto(210.57023241,300.24909881)(210.52523246,300.23909882)(210.48523598,300.22910858)
\curveto(210.44523254,300.22909883)(210.40523258,300.21909884)(210.36523598,300.19910858)
\curveto(210.1852328,300.11909894)(210.03523295,299.99909906)(209.91523598,299.83910858)
\curveto(209.80523318,299.67909938)(209.71523327,299.49909956)(209.64523598,299.29910858)
\curveto(209.5852334,299.10909995)(209.54023344,298.88410018)(209.51023598,298.62410858)
\curveto(209.49023349,298.3641007)(209.4852335,298.09910096)(209.49523598,297.82910858)
\curveto(209.50523348,297.56910149)(209.53523345,297.31910174)(209.58523598,297.07910858)
\curveto(209.64523334,296.84910221)(209.72023326,296.6591024)(209.81023598,296.50910858)
\moveto(220.61023598,293.52410858)
\curveto(220.62022236,293.47410559)(220.62522236,293.38410568)(220.62523598,293.25410858)
\curveto(220.62522236,293.12410594)(220.61522237,293.03410603)(220.59523598,292.98410858)
\curveto(220.57522241,292.93410613)(220.57022241,292.87910618)(220.58023598,292.81910858)
\curveto(220.59022239,292.76910629)(220.59022239,292.71910634)(220.58023598,292.66910858)
\curveto(220.54022244,292.52910653)(220.51022247,292.39410667)(220.49023598,292.26410858)
\curveto(220.4802225,292.13410693)(220.45022253,292.01410705)(220.40023598,291.90410858)
\curveto(220.26022272,291.55410751)(220.09522289,291.2591078)(219.90523598,291.01910858)
\curveto(219.71522327,290.78910827)(219.44522354,290.60410846)(219.09523598,290.46410858)
\curveto(219.01522397,290.43410863)(218.93022405,290.41410865)(218.84023598,290.40410858)
\curveto(218.75022423,290.38410868)(218.66522432,290.3641087)(218.58523598,290.34410858)
\curveto(218.53522445,290.33410873)(218.4852245,290.32910873)(218.43523598,290.32910858)
\curveto(218.3852246,290.32910873)(218.33522465,290.32410874)(218.28523598,290.31410858)
\curveto(218.25522473,290.30410876)(218.20522478,290.30410876)(218.13523598,290.31410858)
\curveto(218.06522492,290.31410875)(218.01522497,290.31910874)(217.98523598,290.32910858)
\curveto(217.92522506,290.34910871)(217.86522512,290.3591087)(217.80523598,290.35910858)
\curveto(217.75522523,290.34910871)(217.70522528,290.35410871)(217.65523598,290.37410858)
\curveto(217.56522542,290.39410867)(217.47522551,290.41910864)(217.38523598,290.44910858)
\curveto(217.30522568,290.46910859)(217.22522576,290.49910856)(217.14523598,290.53910858)
\curveto(216.82522616,290.67910838)(216.57522641,290.87410819)(216.39523598,291.12410858)
\curveto(216.21522677,291.38410768)(216.06522692,291.68910737)(215.94523598,292.03910858)
\curveto(215.92522706,292.11910694)(215.91022707,292.20410686)(215.90023598,292.29410858)
\curveto(215.89022709,292.38410668)(215.87522711,292.46910659)(215.85523598,292.54910858)
\curveto(215.84522714,292.57910648)(215.84022714,292.60910645)(215.84023598,292.63910858)
\lineto(215.84023598,292.74410858)
\curveto(215.82022716,292.82410624)(215.81022717,292.90410616)(215.81023598,292.98410858)
\lineto(215.81023598,293.11910858)
\curveto(215.79022719,293.21910584)(215.79022719,293.31910574)(215.81023598,293.41910858)
\lineto(215.81023598,293.59910858)
\curveto(215.82022716,293.64910541)(215.82522716,293.69410537)(215.82523598,293.73410858)
\curveto(215.82522716,293.78410528)(215.83022715,293.82910523)(215.84023598,293.86910858)
\curveto(215.85022713,293.90910515)(215.85522713,293.94410512)(215.85523598,293.97410858)
\curveto(215.85522713,294.01410505)(215.86022712,294.05410501)(215.87023598,294.09410858)
\lineto(215.93023598,294.42410858)
\curveto(215.95022703,294.54410452)(215.980227,294.65410441)(216.02023598,294.75410858)
\curveto(216.16022682,295.08410398)(216.32022666,295.3591037)(216.50023598,295.57910858)
\curveto(216.69022629,295.80910325)(216.95022603,295.99410307)(217.28023598,296.13410858)
\curveto(217.36022562,296.17410289)(217.44522554,296.19910286)(217.53523598,296.20910858)
\lineto(217.83523598,296.26910858)
\lineto(217.97023598,296.26910858)
\curveto(218.02022496,296.27910278)(218.07022491,296.28410278)(218.12023598,296.28410858)
\curveto(218.69022429,296.30410276)(219.15022383,296.19910286)(219.50023598,295.96910858)
\curveto(219.86022312,295.74910331)(220.12522286,295.44910361)(220.29523598,295.06910858)
\curveto(220.34522264,294.96910409)(220.3852226,294.86910419)(220.41523598,294.76910858)
\curveto(220.44522254,294.66910439)(220.47522251,294.5641045)(220.50523598,294.45410858)
\curveto(220.51522247,294.41410465)(220.52022246,294.37910468)(220.52023598,294.34910858)
\curveto(220.52022246,294.32910473)(220.52522246,294.29910476)(220.53523598,294.25910858)
\curveto(220.55522243,294.18910487)(220.56522242,294.11410495)(220.56523598,294.03410858)
\curveto(220.56522242,293.95410511)(220.57522241,293.87410519)(220.59523598,293.79410858)
\curveto(220.59522239,293.74410532)(220.59522239,293.69910536)(220.59523598,293.65910858)
\curveto(220.59522239,293.61910544)(220.60022238,293.57410549)(220.61023598,293.52410858)
\moveto(219.50023598,293.08910858)
\curveto(219.51022347,293.13910592)(219.51522347,293.21410585)(219.51523598,293.31410858)
\curveto(219.52522346,293.41410565)(219.52022346,293.48910557)(219.50023598,293.53910858)
\curveto(219.4802235,293.59910546)(219.47522351,293.65410541)(219.48523598,293.70410858)
\curveto(219.50522348,293.7641053)(219.50522348,293.82410524)(219.48523598,293.88410858)
\curveto(219.47522351,293.91410515)(219.47022351,293.94910511)(219.47023598,293.98910858)
\curveto(219.47022351,294.02910503)(219.46522352,294.06910499)(219.45523598,294.10910858)
\curveto(219.43522355,294.18910487)(219.41522357,294.2641048)(219.39523598,294.33410858)
\curveto(219.3852236,294.41410465)(219.37022361,294.49410457)(219.35023598,294.57410858)
\curveto(219.32022366,294.63410443)(219.29522369,294.69410437)(219.27523598,294.75410858)
\curveto(219.25522373,294.81410425)(219.22522376,294.87410419)(219.18523598,294.93410858)
\curveto(219.0852239,295.10410396)(218.95522403,295.23910382)(218.79523598,295.33910858)
\curveto(218.71522427,295.38910367)(218.62022436,295.42410364)(218.51023598,295.44410858)
\curveto(218.40022458,295.4641036)(218.27522471,295.47410359)(218.13523598,295.47410858)
\curveto(218.11522487,295.4641036)(218.09022489,295.4591036)(218.06023598,295.45910858)
\curveto(218.03022495,295.46910359)(218.00022498,295.46910359)(217.97023598,295.45910858)
\lineto(217.82023598,295.39910858)
\curveto(217.77022521,295.38910367)(217.72522526,295.37410369)(217.68523598,295.35410858)
\curveto(217.49522549,295.24410382)(217.35022563,295.09910396)(217.25023598,294.91910858)
\curveto(217.16022582,294.73910432)(217.0802259,294.53410453)(217.01023598,294.30410858)
\curveto(216.97022601,294.17410489)(216.95022603,294.03910502)(216.95023598,293.89910858)
\curveto(216.95022603,293.76910529)(216.94022604,293.62410544)(216.92023598,293.46410858)
\curveto(216.91022607,293.41410565)(216.90022608,293.35410571)(216.89023598,293.28410858)
\curveto(216.89022609,293.21410585)(216.90022608,293.15410591)(216.92023598,293.10410858)
\lineto(216.92023598,292.93910858)
\lineto(216.92023598,292.75910858)
\curveto(216.93022605,292.70910635)(216.94022604,292.65410641)(216.95023598,292.59410858)
\curveto(216.96022602,292.54410652)(216.96522602,292.48910657)(216.96523598,292.42910858)
\curveto(216.97522601,292.36910669)(216.99022599,292.31410675)(217.01023598,292.26410858)
\curveto(217.06022592,292.07410699)(217.12022586,291.89910716)(217.19023598,291.73910858)
\curveto(217.26022572,291.57910748)(217.36522562,291.44910761)(217.50523598,291.34910858)
\curveto(217.63522535,291.24910781)(217.77522521,291.17910788)(217.92523598,291.13910858)
\curveto(217.95522503,291.12910793)(217.980225,291.12410794)(218.00023598,291.12410858)
\curveto(218.03022495,291.13410793)(218.06022492,291.13410793)(218.09023598,291.12410858)
\curveto(218.11022487,291.12410794)(218.14022484,291.11910794)(218.18023598,291.10910858)
\curveto(218.22022476,291.10910795)(218.25522473,291.11410795)(218.28523598,291.12410858)
\curveto(218.32522466,291.13410793)(218.36522462,291.13910792)(218.40523598,291.13910858)
\curveto(218.44522454,291.13910792)(218.4852245,291.14910791)(218.52523598,291.16910858)
\curveto(218.76522422,291.24910781)(218.96022402,291.38410768)(219.11023598,291.57410858)
\curveto(219.23022375,291.75410731)(219.32022366,291.9591071)(219.38023598,292.18910858)
\curveto(219.40022358,292.2591068)(219.41522357,292.32910673)(219.42523598,292.39910858)
\curveto(219.43522355,292.47910658)(219.45022353,292.5591065)(219.47023598,292.63910858)
\curveto(219.47022351,292.69910636)(219.47522351,292.74410632)(219.48523598,292.77410858)
\curveto(219.4852235,292.79410627)(219.4852235,292.81910624)(219.48523598,292.84910858)
\curveto(219.4852235,292.88910617)(219.49022349,292.91910614)(219.50023598,292.93910858)
\lineto(219.50023598,293.08910858)
}
}
{
\newrgbcolor{curcolor}{0 0 0}
\pscustom[linestyle=none,fillstyle=solid,fillcolor=curcolor]
{
\newpath
\moveto(137.03356667,270.59404267)
\curveto(137.13356181,270.59403205)(137.22856172,270.58403206)(137.31856667,270.56404267)
\curveto(137.40856154,270.55403209)(137.47356147,270.52403212)(137.51356667,270.47404267)
\curveto(137.57356137,270.39403225)(137.60356134,270.28903235)(137.60356667,270.15904267)
\lineto(137.60356667,269.76904267)
\lineto(137.60356667,268.26904267)
\lineto(137.60356667,261.87904267)
\lineto(137.60356667,260.70904267)
\lineto(137.60356667,260.39404267)
\curveto(137.61356133,260.29404235)(137.59856135,260.21404243)(137.55856667,260.15404267)
\curveto(137.50856144,260.07404257)(137.43356151,260.02404262)(137.33356667,260.00404267)
\curveto(137.2435617,259.99404265)(137.13356181,259.98904265)(137.00356667,259.98904267)
\lineto(136.77856667,259.98904267)
\curveto(136.69856225,260.00904263)(136.62856232,260.02404262)(136.56856667,260.03404267)
\curveto(136.50856244,260.05404259)(136.45856249,260.09404255)(136.41856667,260.15404267)
\curveto(136.37856257,260.21404243)(136.35856259,260.28904235)(136.35856667,260.37904267)
\lineto(136.35856667,260.67904267)
\lineto(136.35856667,261.77404267)
\lineto(136.35856667,267.11404267)
\curveto(136.33856261,267.20403544)(136.32356262,267.27903536)(136.31356667,267.33904267)
\curveto(136.31356263,267.40903523)(136.28356266,267.46903517)(136.22356667,267.51904267)
\curveto(136.15356279,267.56903507)(136.06356288,267.59403505)(135.95356667,267.59404267)
\curveto(135.85356309,267.60403504)(135.7435632,267.60903503)(135.62356667,267.60904267)
\lineto(134.48356667,267.60904267)
\lineto(133.98856667,267.60904267)
\curveto(133.82856512,267.61903502)(133.71856523,267.67903496)(133.65856667,267.78904267)
\curveto(133.63856531,267.81903482)(133.62856532,267.84903479)(133.62856667,267.87904267)
\curveto(133.62856532,267.91903472)(133.62356532,267.96403468)(133.61356667,268.01404267)
\curveto(133.59356535,268.13403451)(133.59856535,268.2440344)(133.62856667,268.34404267)
\curveto(133.66856528,268.4440342)(133.72356522,268.51403413)(133.79356667,268.55404267)
\curveto(133.87356507,268.60403404)(133.99356495,268.62903401)(134.15356667,268.62904267)
\curveto(134.31356463,268.62903401)(134.4485645,268.644034)(134.55856667,268.67404267)
\curveto(134.60856434,268.68403396)(134.66356428,268.68903395)(134.72356667,268.68904267)
\curveto(134.78356416,268.69903394)(134.8435641,268.71403393)(134.90356667,268.73404267)
\curveto(135.05356389,268.78403386)(135.19856375,268.83403381)(135.33856667,268.88404267)
\curveto(135.47856347,268.9440337)(135.61356333,269.01403363)(135.74356667,269.09404267)
\curveto(135.88356306,269.18403346)(136.00356294,269.28903335)(136.10356667,269.40904267)
\curveto(136.20356274,269.52903311)(136.29856265,269.65903298)(136.38856667,269.79904267)
\curveto(136.4485625,269.89903274)(136.49356245,270.00903263)(136.52356667,270.12904267)
\curveto(136.56356238,270.24903239)(136.61356233,270.35403229)(136.67356667,270.44404267)
\curveto(136.72356222,270.50403214)(136.79356215,270.5440321)(136.88356667,270.56404267)
\curveto(136.90356204,270.57403207)(136.92856202,270.57903206)(136.95856667,270.57904267)
\curveto(136.98856196,270.57903206)(137.01356193,270.58403206)(137.03356667,270.59404267)
}
}
{
\newrgbcolor{curcolor}{0 0 0}
\pscustom[linestyle=none,fillstyle=solid,fillcolor=curcolor]
{
\newpath
\moveto(148.32317604,265.07404267)
\lineto(148.32317604,264.81904267)
\curveto(148.33316834,264.7390379)(148.32816834,264.66403798)(148.30817604,264.59404267)
\lineto(148.30817604,264.35404267)
\lineto(148.30817604,264.18904267)
\curveto(148.28816838,264.08903855)(148.27816839,263.98403866)(148.27817604,263.87404267)
\curveto(148.27816839,263.77403887)(148.2681684,263.67403897)(148.24817604,263.57404267)
\lineto(148.24817604,263.42404267)
\curveto(148.21816845,263.28403936)(148.19816847,263.1440395)(148.18817604,263.00404267)
\curveto(148.17816849,262.87403977)(148.15316852,262.7440399)(148.11317604,262.61404267)
\curveto(148.09316858,262.53404011)(148.0731686,262.44904019)(148.05317604,262.35904267)
\lineto(147.99317604,262.11904267)
\lineto(147.87317604,261.81904267)
\curveto(147.84316883,261.72904091)(147.80816886,261.639041)(147.76817604,261.54904267)
\curveto(147.668169,261.32904131)(147.53316914,261.11404153)(147.36317604,260.90404267)
\curveto(147.20316947,260.69404195)(147.02816964,260.52404212)(146.83817604,260.39404267)
\curveto(146.78816988,260.35404229)(146.72816994,260.31404233)(146.65817604,260.27404267)
\curveto(146.59817007,260.2440424)(146.53817013,260.20904243)(146.47817604,260.16904267)
\curveto(146.39817027,260.11904252)(146.30317037,260.07904256)(146.19317604,260.04904267)
\curveto(146.08317059,260.01904262)(145.97817069,259.98904265)(145.87817604,259.95904267)
\curveto(145.7681709,259.91904272)(145.65817101,259.89404275)(145.54817604,259.88404267)
\curveto(145.43817123,259.87404277)(145.32317135,259.85904278)(145.20317604,259.83904267)
\curveto(145.16317151,259.82904281)(145.11817155,259.82904281)(145.06817604,259.83904267)
\curveto(145.02817164,259.8390428)(144.98817168,259.83404281)(144.94817604,259.82404267)
\curveto(144.90817176,259.81404283)(144.85317182,259.80904283)(144.78317604,259.80904267)
\curveto(144.71317196,259.80904283)(144.66317201,259.81404283)(144.63317604,259.82404267)
\curveto(144.58317209,259.8440428)(144.53817213,259.84904279)(144.49817604,259.83904267)
\curveto(144.45817221,259.82904281)(144.42317225,259.82904281)(144.39317604,259.83904267)
\lineto(144.30317604,259.83904267)
\curveto(144.24317243,259.85904278)(144.17817249,259.87404277)(144.10817604,259.88404267)
\curveto(144.04817262,259.88404276)(143.98317269,259.88904275)(143.91317604,259.89904267)
\curveto(143.74317293,259.94904269)(143.58317309,259.99904264)(143.43317604,260.04904267)
\curveto(143.28317339,260.09904254)(143.13817353,260.16404248)(142.99817604,260.24404267)
\curveto(142.94817372,260.28404236)(142.89317378,260.31404233)(142.83317604,260.33404267)
\curveto(142.78317389,260.36404228)(142.73317394,260.39904224)(142.68317604,260.43904267)
\curveto(142.44317423,260.61904202)(142.24317443,260.8390418)(142.08317604,261.09904267)
\curveto(141.92317475,261.35904128)(141.78317489,261.644041)(141.66317604,261.95404267)
\curveto(141.60317507,262.09404055)(141.55817511,262.23404041)(141.52817604,262.37404267)
\curveto(141.49817517,262.52404012)(141.46317521,262.67903996)(141.42317604,262.83904267)
\curveto(141.40317527,262.94903969)(141.38817528,263.05903958)(141.37817604,263.16904267)
\curveto(141.3681753,263.27903936)(141.35317532,263.38903925)(141.33317604,263.49904267)
\curveto(141.32317535,263.5390391)(141.31817535,263.57903906)(141.31817604,263.61904267)
\curveto(141.32817534,263.65903898)(141.32817534,263.69903894)(141.31817604,263.73904267)
\curveto(141.30817536,263.78903885)(141.30317537,263.8390388)(141.30317604,263.88904267)
\lineto(141.30317604,264.05404267)
\curveto(141.28317539,264.10403854)(141.27817539,264.15403849)(141.28817604,264.20404267)
\curveto(141.29817537,264.26403838)(141.29817537,264.31903832)(141.28817604,264.36904267)
\curveto(141.27817539,264.40903823)(141.27817539,264.45403819)(141.28817604,264.50404267)
\curveto(141.29817537,264.55403809)(141.29317538,264.60403804)(141.27317604,264.65404267)
\curveto(141.25317542,264.72403792)(141.24817542,264.79903784)(141.25817604,264.87904267)
\curveto(141.2681754,264.96903767)(141.2731754,265.05403759)(141.27317604,265.13404267)
\curveto(141.2731754,265.22403742)(141.2681754,265.32403732)(141.25817604,265.43404267)
\curveto(141.24817542,265.55403709)(141.25317542,265.65403699)(141.27317604,265.73404267)
\lineto(141.27317604,266.01904267)
\lineto(141.31817604,266.64904267)
\curveto(141.32817534,266.74903589)(141.33817533,266.8440358)(141.34817604,266.93404267)
\lineto(141.37817604,267.23404267)
\curveto(141.39817527,267.28403536)(141.40317527,267.33403531)(141.39317604,267.38404267)
\curveto(141.39317528,267.4440352)(141.40317527,267.49903514)(141.42317604,267.54904267)
\curveto(141.4731752,267.71903492)(141.51317516,267.88403476)(141.54317604,268.04404267)
\curveto(141.5731751,268.21403443)(141.62317505,268.37403427)(141.69317604,268.52404267)
\curveto(141.88317479,268.98403366)(142.10317457,269.35903328)(142.35317604,269.64904267)
\curveto(142.61317406,269.9390327)(142.9731737,270.18403246)(143.43317604,270.38404267)
\curveto(143.56317311,270.43403221)(143.69317298,270.46903217)(143.82317604,270.48904267)
\curveto(143.96317271,270.50903213)(144.10317257,270.53403211)(144.24317604,270.56404267)
\curveto(144.31317236,270.57403207)(144.37817229,270.57903206)(144.43817604,270.57904267)
\curveto(144.49817217,270.57903206)(144.56317211,270.58403206)(144.63317604,270.59404267)
\curveto(145.46317121,270.61403203)(146.13317054,270.46403218)(146.64317604,270.14404267)
\curveto(147.15316952,269.83403281)(147.53316914,269.39403325)(147.78317604,268.82404267)
\curveto(147.83316884,268.70403394)(147.87816879,268.57903406)(147.91817604,268.44904267)
\curveto(147.95816871,268.31903432)(148.00316867,268.18403446)(148.05317604,268.04404267)
\curveto(148.0731686,267.96403468)(148.08816858,267.87903476)(148.09817604,267.78904267)
\lineto(148.15817604,267.54904267)
\curveto(148.18816848,267.4390352)(148.20316847,267.32903531)(148.20317604,267.21904267)
\curveto(148.21316846,267.10903553)(148.22816844,266.99903564)(148.24817604,266.88904267)
\curveto(148.2681684,266.8390358)(148.2731684,266.79403585)(148.26317604,266.75404267)
\curveto(148.26316841,266.71403593)(148.2681684,266.67403597)(148.27817604,266.63404267)
\curveto(148.28816838,266.58403606)(148.28816838,266.52903611)(148.27817604,266.46904267)
\curveto(148.27816839,266.41903622)(148.28316839,266.36903627)(148.29317604,266.31904267)
\lineto(148.29317604,266.18404267)
\curveto(148.31316836,266.12403652)(148.31316836,266.05403659)(148.29317604,265.97404267)
\curveto(148.28316839,265.90403674)(148.28816838,265.8390368)(148.30817604,265.77904267)
\curveto(148.31816835,265.74903689)(148.32316835,265.70903693)(148.32317604,265.65904267)
\lineto(148.32317604,265.53904267)
\lineto(148.32317604,265.07404267)
\moveto(146.77817604,262.74904267)
\curveto(146.87816979,263.06903957)(146.93816973,263.43403921)(146.95817604,263.84404267)
\curveto(146.97816969,264.25403839)(146.98816968,264.66403798)(146.98817604,265.07404267)
\curveto(146.98816968,265.50403714)(146.97816969,265.92403672)(146.95817604,266.33404267)
\curveto(146.93816973,266.7440359)(146.89316978,267.12903551)(146.82317604,267.48904267)
\curveto(146.75316992,267.84903479)(146.64317003,268.16903447)(146.49317604,268.44904267)
\curveto(146.35317032,268.7390339)(146.15817051,268.97403367)(145.90817604,269.15404267)
\curveto(145.74817092,269.26403338)(145.5681711,269.3440333)(145.36817604,269.39404267)
\curveto(145.1681715,269.45403319)(144.92317175,269.48403316)(144.63317604,269.48404267)
\curveto(144.61317206,269.46403318)(144.57817209,269.45403319)(144.52817604,269.45404267)
\curveto(144.47817219,269.46403318)(144.43817223,269.46403318)(144.40817604,269.45404267)
\curveto(144.32817234,269.43403321)(144.25317242,269.41403323)(144.18317604,269.39404267)
\curveto(144.12317255,269.38403326)(144.05817261,269.36403328)(143.98817604,269.33404267)
\curveto(143.71817295,269.21403343)(143.49817317,269.0440336)(143.32817604,268.82404267)
\curveto(143.1681735,268.61403403)(143.03317364,268.36903427)(142.92317604,268.08904267)
\curveto(142.8731738,267.97903466)(142.83317384,267.85903478)(142.80317604,267.72904267)
\curveto(142.78317389,267.60903503)(142.75817391,267.48403516)(142.72817604,267.35404267)
\curveto(142.70817396,267.30403534)(142.69817397,267.24903539)(142.69817604,267.18904267)
\curveto(142.69817397,267.1390355)(142.69317398,267.08903555)(142.68317604,267.03904267)
\curveto(142.673174,266.94903569)(142.66317401,266.85403579)(142.65317604,266.75404267)
\curveto(142.64317403,266.66403598)(142.63317404,266.56903607)(142.62317604,266.46904267)
\curveto(142.62317405,266.38903625)(142.61817405,266.30403634)(142.60817604,266.21404267)
\lineto(142.60817604,265.97404267)
\lineto(142.60817604,265.79404267)
\curveto(142.59817407,265.76403688)(142.59317408,265.72903691)(142.59317604,265.68904267)
\lineto(142.59317604,265.55404267)
\lineto(142.59317604,265.10404267)
\curveto(142.59317408,265.02403762)(142.58817408,264.9390377)(142.57817604,264.84904267)
\curveto(142.57817409,264.76903787)(142.58817408,264.69403795)(142.60817604,264.62404267)
\lineto(142.60817604,264.35404267)
\curveto(142.60817406,264.33403831)(142.60317407,264.30403834)(142.59317604,264.26404267)
\curveto(142.59317408,264.23403841)(142.59817407,264.20903843)(142.60817604,264.18904267)
\curveto(142.61817405,264.08903855)(142.62317405,263.98903865)(142.62317604,263.88904267)
\curveto(142.63317404,263.79903884)(142.64317403,263.69903894)(142.65317604,263.58904267)
\curveto(142.68317399,263.46903917)(142.69817397,263.3440393)(142.69817604,263.21404267)
\curveto(142.70817396,263.09403955)(142.73317394,262.97903966)(142.77317604,262.86904267)
\curveto(142.85317382,262.56904007)(142.93817373,262.30404034)(143.02817604,262.07404267)
\curveto(143.12817354,261.8440408)(143.2731734,261.62904101)(143.46317604,261.42904267)
\curveto(143.673173,261.22904141)(143.93817273,261.07904156)(144.25817604,260.97904267)
\curveto(144.29817237,260.95904168)(144.33317234,260.94904169)(144.36317604,260.94904267)
\curveto(144.40317227,260.95904168)(144.44817222,260.95404169)(144.49817604,260.93404267)
\curveto(144.53817213,260.92404172)(144.60817206,260.91404173)(144.70817604,260.90404267)
\curveto(144.81817185,260.89404175)(144.90317177,260.89904174)(144.96317604,260.91904267)
\curveto(145.03317164,260.9390417)(145.10317157,260.94904169)(145.17317604,260.94904267)
\curveto(145.24317143,260.95904168)(145.30817136,260.97404167)(145.36817604,260.99404267)
\curveto(145.5681711,261.05404159)(145.74817092,261.1390415)(145.90817604,261.24904267)
\curveto(145.93817073,261.26904137)(145.96317071,261.28904135)(145.98317604,261.30904267)
\lineto(146.04317604,261.36904267)
\curveto(146.08317059,261.38904125)(146.13317054,261.42904121)(146.19317604,261.48904267)
\curveto(146.29317038,261.62904101)(146.37817029,261.75904088)(146.44817604,261.87904267)
\curveto(146.51817015,261.99904064)(146.58817008,262.1440405)(146.65817604,262.31404267)
\curveto(146.68816998,262.38404026)(146.70816996,262.45404019)(146.71817604,262.52404267)
\curveto(146.73816993,262.59404005)(146.75816991,262.66903997)(146.77817604,262.74904267)
}
}
{
\newrgbcolor{curcolor}{0 0 0}
\pscustom[linestyle=none,fillstyle=solid,fillcolor=curcolor]
{
\newpath
\moveto(150.62778542,261.62404267)
\lineto(150.92778542,261.62404267)
\curveto(151.03778336,261.63404101)(151.14278325,261.63404101)(151.24278542,261.62404267)
\curveto(151.35278304,261.62404102)(151.45278294,261.61404103)(151.54278542,261.59404267)
\curveto(151.63278276,261.58404106)(151.70278269,261.55904108)(151.75278542,261.51904267)
\curveto(151.77278262,261.49904114)(151.78778261,261.46904117)(151.79778542,261.42904267)
\curveto(151.81778258,261.38904125)(151.83778256,261.3440413)(151.85778542,261.29404267)
\lineto(151.85778542,261.21904267)
\curveto(151.86778253,261.16904147)(151.86778253,261.11404153)(151.85778542,261.05404267)
\lineto(151.85778542,260.90404267)
\lineto(151.85778542,260.42404267)
\curveto(151.85778254,260.25404239)(151.81778258,260.13404251)(151.73778542,260.06404267)
\curveto(151.66778273,260.01404263)(151.57778282,259.98904265)(151.46778542,259.98904267)
\lineto(151.13778542,259.98904267)
\lineto(150.68778542,259.98904267)
\curveto(150.53778386,259.98904265)(150.42278397,260.01904262)(150.34278542,260.07904267)
\curveto(150.30278409,260.10904253)(150.27278412,260.15904248)(150.25278542,260.22904267)
\curveto(150.23278416,260.30904233)(150.21778418,260.39404225)(150.20778542,260.48404267)
\lineto(150.20778542,260.76904267)
\curveto(150.21778418,260.86904177)(150.22278417,260.95404169)(150.22278542,261.02404267)
\lineto(150.22278542,261.21904267)
\curveto(150.22278417,261.27904136)(150.23278416,261.33404131)(150.25278542,261.38404267)
\curveto(150.2927841,261.49404115)(150.36278403,261.56404108)(150.46278542,261.59404267)
\curveto(150.4927839,261.59404105)(150.54778385,261.60404104)(150.62778542,261.62404267)
}
}
{
\newrgbcolor{curcolor}{0 0 0}
\pscustom[linestyle=none,fillstyle=solid,fillcolor=curcolor]
{
\newpath
\moveto(155.34294167,270.39904267)
\lineto(158.94294167,270.39904267)
\lineto(159.58794167,270.39904267)
\curveto(159.66793514,270.39903224)(159.74293506,270.39403225)(159.81294167,270.38404267)
\curveto(159.88293492,270.38403226)(159.94293486,270.37403227)(159.99294167,270.35404267)
\curveto(160.06293474,270.32403232)(160.11793469,270.26403238)(160.15794167,270.17404267)
\curveto(160.17793463,270.1440325)(160.18793462,270.10403254)(160.18794167,270.05404267)
\lineto(160.18794167,269.91904267)
\curveto(160.19793461,269.80903283)(160.19293461,269.70403294)(160.17294167,269.60404267)
\curveto(160.16293464,269.50403314)(160.12793468,269.43403321)(160.06794167,269.39404267)
\curveto(159.97793483,269.32403332)(159.84293496,269.28903335)(159.66294167,269.28904267)
\curveto(159.48293532,269.29903334)(159.31793549,269.30403334)(159.16794167,269.30404267)
\lineto(157.17294167,269.30404267)
\lineto(156.67794167,269.30404267)
\lineto(156.54294167,269.30404267)
\curveto(156.5029383,269.30403334)(156.46293834,269.29903334)(156.42294167,269.28904267)
\lineto(156.21294167,269.28904267)
\curveto(156.1029387,269.25903338)(156.02293878,269.21903342)(155.97294167,269.16904267)
\curveto(155.92293888,269.12903351)(155.88793892,269.07403357)(155.86794167,269.00404267)
\curveto(155.84793896,268.9440337)(155.83293897,268.87403377)(155.82294167,268.79404267)
\curveto(155.81293899,268.71403393)(155.79293901,268.62403402)(155.76294167,268.52404267)
\curveto(155.71293909,268.32403432)(155.67293913,268.11903452)(155.64294167,267.90904267)
\curveto(155.61293919,267.69903494)(155.57293923,267.49403515)(155.52294167,267.29404267)
\curveto(155.5029393,267.22403542)(155.49293931,267.15403549)(155.49294167,267.08404267)
\curveto(155.49293931,267.02403562)(155.48293932,266.95903568)(155.46294167,266.88904267)
\curveto(155.45293935,266.85903578)(155.44293936,266.81903582)(155.43294167,266.76904267)
\curveto(155.43293937,266.72903591)(155.43793937,266.68903595)(155.44794167,266.64904267)
\curveto(155.46793934,266.59903604)(155.49293931,266.55403609)(155.52294167,266.51404267)
\curveto(155.56293924,266.48403616)(155.62293918,266.47903616)(155.70294167,266.49904267)
\curveto(155.76293904,266.51903612)(155.82293898,266.5440361)(155.88294167,266.57404267)
\curveto(155.94293886,266.61403603)(156.0029388,266.64903599)(156.06294167,266.67904267)
\curveto(156.12293868,266.69903594)(156.17293863,266.71403593)(156.21294167,266.72404267)
\curveto(156.4029384,266.80403584)(156.6079382,266.85903578)(156.82794167,266.88904267)
\curveto(157.05793775,266.91903572)(157.28793752,266.92903571)(157.51794167,266.91904267)
\curveto(157.75793705,266.91903572)(157.98793682,266.89403575)(158.20794167,266.84404267)
\curveto(158.42793638,266.80403584)(158.62793618,266.7440359)(158.80794167,266.66404267)
\curveto(158.85793595,266.644036)(158.9029359,266.62403602)(158.94294167,266.60404267)
\curveto(158.99293581,266.58403606)(159.04293576,266.55903608)(159.09294167,266.52904267)
\curveto(159.44293536,266.31903632)(159.72293508,266.08903655)(159.93294167,265.83904267)
\curveto(160.15293465,265.58903705)(160.34793446,265.26403738)(160.51794167,264.86404267)
\curveto(160.56793424,264.75403789)(160.6029342,264.644038)(160.62294167,264.53404267)
\curveto(160.64293416,264.42403822)(160.66793414,264.30903833)(160.69794167,264.18904267)
\curveto(160.7079341,264.15903848)(160.71293409,264.11403853)(160.71294167,264.05404267)
\curveto(160.73293407,263.99403865)(160.74293406,263.92403872)(160.74294167,263.84404267)
\curveto(160.74293406,263.77403887)(160.75293405,263.70903893)(160.77294167,263.64904267)
\lineto(160.77294167,263.48404267)
\curveto(160.78293402,263.43403921)(160.78793402,263.36403928)(160.78794167,263.27404267)
\curveto(160.78793402,263.18403946)(160.77793403,263.11403953)(160.75794167,263.06404267)
\curveto(160.73793407,263.00403964)(160.73293407,262.9440397)(160.74294167,262.88404267)
\curveto(160.75293405,262.83403981)(160.74793406,262.78403986)(160.72794167,262.73404267)
\curveto(160.68793412,262.57404007)(160.65293415,262.42404022)(160.62294167,262.28404267)
\curveto(160.59293421,262.1440405)(160.54793426,262.00904063)(160.48794167,261.87904267)
\curveto(160.32793448,261.50904113)(160.1079347,261.17404147)(159.82794167,260.87404267)
\curveto(159.54793526,260.57404207)(159.22793558,260.3440423)(158.86794167,260.18404267)
\curveto(158.69793611,260.10404254)(158.49793631,260.02904261)(158.26794167,259.95904267)
\curveto(158.15793665,259.91904272)(158.04293676,259.89404275)(157.92294167,259.88404267)
\curveto(157.802937,259.87404277)(157.68293712,259.85404279)(157.56294167,259.82404267)
\curveto(157.51293729,259.80404284)(157.45793735,259.80404284)(157.39794167,259.82404267)
\curveto(157.33793747,259.83404281)(157.27793753,259.82904281)(157.21794167,259.80904267)
\curveto(157.11793769,259.78904285)(157.01793779,259.78904285)(156.91794167,259.80904267)
\lineto(156.78294167,259.80904267)
\curveto(156.73293807,259.82904281)(156.67293813,259.8390428)(156.60294167,259.83904267)
\curveto(156.54293826,259.82904281)(156.48793832,259.83404281)(156.43794167,259.85404267)
\curveto(156.39793841,259.86404278)(156.36293844,259.86904277)(156.33294167,259.86904267)
\curveto(156.3029385,259.86904277)(156.26793854,259.87404277)(156.22794167,259.88404267)
\lineto(155.95794167,259.94404267)
\curveto(155.86793894,259.96404268)(155.78293902,259.99404265)(155.70294167,260.03404267)
\curveto(155.36293944,260.17404247)(155.07293973,260.32904231)(154.83294167,260.49904267)
\curveto(154.59294021,260.67904196)(154.37294043,260.90904173)(154.17294167,261.18904267)
\curveto(154.02294078,261.41904122)(153.9079409,261.65904098)(153.82794167,261.90904267)
\curveto(153.807941,261.95904068)(153.79794101,262.00404064)(153.79794167,262.04404267)
\curveto(153.79794101,262.09404055)(153.78794102,262.1440405)(153.76794167,262.19404267)
\curveto(153.74794106,262.25404039)(153.73294107,262.33404031)(153.72294167,262.43404267)
\curveto(153.72294108,262.53404011)(153.74294106,262.60904003)(153.78294167,262.65904267)
\curveto(153.83294097,262.7390399)(153.91294089,262.78403986)(154.02294167,262.79404267)
\curveto(154.13294067,262.80403984)(154.24794056,262.80903983)(154.36794167,262.80904267)
\lineto(154.53294167,262.80904267)
\curveto(154.59294021,262.80903983)(154.64794016,262.79903984)(154.69794167,262.77904267)
\curveto(154.78794002,262.75903988)(154.85793995,262.71903992)(154.90794167,262.65904267)
\curveto(154.97793983,262.56904007)(155.02293978,262.45904018)(155.04294167,262.32904267)
\curveto(155.07293973,262.20904043)(155.11793969,262.10404054)(155.17794167,262.01404267)
\curveto(155.36793944,261.67404097)(155.62793918,261.40404124)(155.95794167,261.20404267)
\curveto(156.05793875,261.1440415)(156.16293864,261.09404155)(156.27294167,261.05404267)
\curveto(156.39293841,261.02404162)(156.51293829,260.98904165)(156.63294167,260.94904267)
\curveto(156.802938,260.89904174)(157.0079378,260.87904176)(157.24794167,260.88904267)
\curveto(157.49793731,260.90904173)(157.69793711,260.9440417)(157.84794167,260.99404267)
\curveto(158.21793659,261.11404153)(158.5079363,261.27404137)(158.71794167,261.47404267)
\curveto(158.93793587,261.68404096)(159.11793569,261.96404068)(159.25794167,262.31404267)
\curveto(159.3079355,262.41404023)(159.33793547,262.51904012)(159.34794167,262.62904267)
\curveto(159.36793544,262.7390399)(159.39293541,262.85403979)(159.42294167,262.97404267)
\lineto(159.42294167,263.07904267)
\curveto(159.43293537,263.11903952)(159.43793537,263.15903948)(159.43794167,263.19904267)
\curveto(159.44793536,263.22903941)(159.44793536,263.26403938)(159.43794167,263.30404267)
\lineto(159.43794167,263.42404267)
\curveto(159.43793537,263.68403896)(159.4079354,263.92903871)(159.34794167,264.15904267)
\curveto(159.23793557,264.50903813)(159.08293572,264.80403784)(158.88294167,265.04404267)
\curveto(158.68293612,265.29403735)(158.42293638,265.48903715)(158.10294167,265.62904267)
\lineto(157.92294167,265.68904267)
\curveto(157.87293693,265.70903693)(157.81293699,265.72903691)(157.74294167,265.74904267)
\curveto(157.69293711,265.76903687)(157.63293717,265.77903686)(157.56294167,265.77904267)
\curveto(157.5029373,265.78903685)(157.43793737,265.80403684)(157.36794167,265.82404267)
\lineto(157.21794167,265.82404267)
\curveto(157.17793763,265.8440368)(157.12293768,265.85403679)(157.05294167,265.85404267)
\curveto(156.99293781,265.85403679)(156.93793787,265.8440368)(156.88794167,265.82404267)
\lineto(156.78294167,265.82404267)
\curveto(156.75293805,265.82403682)(156.71793809,265.81903682)(156.67794167,265.80904267)
\lineto(156.43794167,265.74904267)
\curveto(156.35793845,265.7390369)(156.27793853,265.71903692)(156.19794167,265.68904267)
\curveto(155.95793885,265.58903705)(155.72793908,265.45403719)(155.50794167,265.28404267)
\curveto(155.41793939,265.21403743)(155.33293947,265.1390375)(155.25294167,265.05904267)
\curveto(155.17293963,264.98903765)(155.07293973,264.93403771)(154.95294167,264.89404267)
\curveto(154.86293994,264.86403778)(154.72294008,264.85403779)(154.53294167,264.86404267)
\curveto(154.35294045,264.87403777)(154.23294057,264.89903774)(154.17294167,264.93904267)
\curveto(154.12294068,264.97903766)(154.08294072,265.0390376)(154.05294167,265.11904267)
\curveto(154.03294077,265.19903744)(154.03294077,265.28403736)(154.05294167,265.37404267)
\curveto(154.08294072,265.49403715)(154.1029407,265.61403703)(154.11294167,265.73404267)
\curveto(154.13294067,265.86403678)(154.15794065,265.98903665)(154.18794167,266.10904267)
\curveto(154.2079406,266.14903649)(154.21294059,266.18403646)(154.20294167,266.21404267)
\curveto(154.2029406,266.25403639)(154.21294059,266.29903634)(154.23294167,266.34904267)
\curveto(154.25294055,266.4390362)(154.26794054,266.52903611)(154.27794167,266.61904267)
\curveto(154.28794052,266.71903592)(154.3079405,266.81403583)(154.33794167,266.90404267)
\curveto(154.34794046,266.96403568)(154.35294045,267.02403562)(154.35294167,267.08404267)
\curveto(154.36294044,267.1440355)(154.37794043,267.20403544)(154.39794167,267.26404267)
\curveto(154.44794036,267.46403518)(154.48294032,267.66903497)(154.50294167,267.87904267)
\curveto(154.53294027,268.09903454)(154.57294023,268.30903433)(154.62294167,268.50904267)
\curveto(154.65294015,268.60903403)(154.67294013,268.70903393)(154.68294167,268.80904267)
\curveto(154.69294011,268.90903373)(154.7079401,269.00903363)(154.72794167,269.10904267)
\curveto(154.73794007,269.1390335)(154.74294006,269.17903346)(154.74294167,269.22904267)
\curveto(154.77294003,269.3390333)(154.79294001,269.4440332)(154.80294167,269.54404267)
\curveto(154.82293998,269.65403299)(154.84793996,269.76403288)(154.87794167,269.87404267)
\curveto(154.89793991,269.95403269)(154.91293989,270.02403262)(154.92294167,270.08404267)
\curveto(154.93293987,270.15403249)(154.95793985,270.21403243)(154.99794167,270.26404267)
\curveto(155.01793979,270.29403235)(155.04793976,270.31403233)(155.08794167,270.32404267)
\curveto(155.12793968,270.3440323)(155.17293963,270.36403228)(155.22294167,270.38404267)
\curveto(155.28293952,270.38403226)(155.32293948,270.38903225)(155.34294167,270.39904267)
}
}
{
\newrgbcolor{curcolor}{0 0 0}
\pscustom[linestyle=none,fillstyle=solid,fillcolor=curcolor]
{
\newpath
\moveto(171.98755104,268.50904267)
\curveto(171.78754074,268.21903442)(171.57754095,267.93403471)(171.35755104,267.65404267)
\curveto(171.14754138,267.37403527)(170.94254159,267.08903555)(170.74255104,266.79904267)
\curveto(170.14254239,265.94903669)(169.53754299,265.10903753)(168.92755104,264.27904267)
\curveto(168.31754421,263.45903918)(167.71254482,262.62404002)(167.11255104,261.77404267)
\lineto(166.60255104,261.05404267)
\lineto(166.09255104,260.36404267)
\curveto(166.01254652,260.25404239)(165.9325466,260.1390425)(165.85255104,260.01904267)
\curveto(165.77254676,259.89904274)(165.67754685,259.80404284)(165.56755104,259.73404267)
\curveto(165.527547,259.71404293)(165.46254707,259.69904294)(165.37255104,259.68904267)
\curveto(165.29254724,259.66904297)(165.20254733,259.65904298)(165.10255104,259.65904267)
\curveto(165.00254753,259.65904298)(164.90754762,259.66404298)(164.81755104,259.67404267)
\curveto(164.73754779,259.68404296)(164.67754785,259.70404294)(164.63755104,259.73404267)
\curveto(164.60754792,259.75404289)(164.58254795,259.78904285)(164.56255104,259.83904267)
\curveto(164.55254798,259.87904276)(164.55754797,259.92404272)(164.57755104,259.97404267)
\curveto(164.61754791,260.05404259)(164.66254787,260.12904251)(164.71255104,260.19904267)
\curveto(164.77254776,260.27904236)(164.8275477,260.35904228)(164.87755104,260.43904267)
\curveto(165.11754741,260.77904186)(165.36254717,261.11404153)(165.61255104,261.44404267)
\curveto(165.86254667,261.77404087)(166.10254643,262.10904053)(166.33255104,262.44904267)
\curveto(166.49254604,262.66903997)(166.65254588,262.88403976)(166.81255104,263.09404267)
\curveto(166.97254556,263.30403934)(167.1325454,263.51903912)(167.29255104,263.73904267)
\curveto(167.65254488,264.25903838)(168.01754451,264.76903787)(168.38755104,265.26904267)
\curveto(168.75754377,265.76903687)(169.1275434,266.27903636)(169.49755104,266.79904267)
\curveto(169.63754289,266.99903564)(169.77754275,267.19403545)(169.91755104,267.38404267)
\curveto(170.06754246,267.57403507)(170.21254232,267.76903487)(170.35255104,267.96904267)
\curveto(170.56254197,268.26903437)(170.77754175,268.56903407)(170.99755104,268.86904267)
\lineto(171.65755104,269.76904267)
\lineto(171.83755104,270.03904267)
\lineto(172.04755104,270.30904267)
\lineto(172.16755104,270.48904267)
\curveto(172.21754031,270.54903209)(172.26754026,270.60403204)(172.31755104,270.65404267)
\curveto(172.38754014,270.70403194)(172.46254007,270.7390319)(172.54255104,270.75904267)
\curveto(172.56253997,270.76903187)(172.58753994,270.76903187)(172.61755104,270.75904267)
\curveto(172.65753987,270.75903188)(172.68753984,270.76903187)(172.70755104,270.78904267)
\curveto(172.8275397,270.78903185)(172.96253957,270.78403186)(173.11255104,270.77404267)
\curveto(173.26253927,270.77403187)(173.35253918,270.72903191)(173.38255104,270.63904267)
\curveto(173.40253913,270.60903203)(173.40753912,270.57403207)(173.39755104,270.53404267)
\curveto(173.38753914,270.49403215)(173.37253916,270.46403218)(173.35255104,270.44404267)
\curveto(173.31253922,270.36403228)(173.27253926,270.29403235)(173.23255104,270.23404267)
\curveto(173.19253934,270.17403247)(173.14753938,270.11403253)(173.09755104,270.05404267)
\lineto(172.52755104,269.27404267)
\curveto(172.34754018,269.02403362)(172.16754036,268.76903387)(171.98755104,268.50904267)
\moveto(165.13255104,264.60904267)
\curveto(165.08254745,264.62903801)(165.0325475,264.63403801)(164.98255104,264.62404267)
\curveto(164.9325476,264.61403803)(164.88254765,264.61903802)(164.83255104,264.63904267)
\curveto(164.72254781,264.65903798)(164.61754791,264.67903796)(164.51755104,264.69904267)
\curveto(164.4275481,264.72903791)(164.3325482,264.76903787)(164.23255104,264.81904267)
\curveto(163.90254863,264.95903768)(163.64754888,265.15403749)(163.46755104,265.40404267)
\curveto(163.28754924,265.66403698)(163.14254939,265.97403667)(163.03255104,266.33404267)
\curveto(163.00254953,266.41403623)(162.98254955,266.49403615)(162.97255104,266.57404267)
\curveto(162.96254957,266.66403598)(162.94754958,266.74903589)(162.92755104,266.82904267)
\curveto(162.91754961,266.87903576)(162.91254962,266.9440357)(162.91255104,267.02404267)
\curveto(162.90254963,267.05403559)(162.89754963,267.08403556)(162.89755104,267.11404267)
\curveto(162.89754963,267.15403549)(162.89254964,267.18903545)(162.88255104,267.21904267)
\lineto(162.88255104,267.36904267)
\curveto(162.87254966,267.41903522)(162.86754966,267.47903516)(162.86755104,267.54904267)
\curveto(162.86754966,267.62903501)(162.87254966,267.69403495)(162.88255104,267.74404267)
\lineto(162.88255104,267.90904267)
\curveto(162.90254963,267.95903468)(162.90754962,268.00403464)(162.89755104,268.04404267)
\curveto(162.89754963,268.09403455)(162.90254963,268.1390345)(162.91255104,268.17904267)
\curveto(162.92254961,268.21903442)(162.9275496,268.25403439)(162.92755104,268.28404267)
\curveto(162.9275496,268.32403432)(162.9325496,268.36403428)(162.94255104,268.40404267)
\curveto(162.97254956,268.51403413)(162.99254954,268.62403402)(163.00255104,268.73404267)
\curveto(163.02254951,268.85403379)(163.05754947,268.96903367)(163.10755104,269.07904267)
\curveto(163.24754928,269.41903322)(163.40754912,269.69403295)(163.58755104,269.90404267)
\curveto(163.77754875,270.12403252)(164.04754848,270.30403234)(164.39755104,270.44404267)
\curveto(164.47754805,270.47403217)(164.56254797,270.49403215)(164.65255104,270.50404267)
\curveto(164.74254779,270.52403212)(164.83754769,270.5440321)(164.93755104,270.56404267)
\curveto(164.96754756,270.57403207)(165.02254751,270.57403207)(165.10255104,270.56404267)
\curveto(165.18254735,270.56403208)(165.2325473,270.57403207)(165.25255104,270.59404267)
\curveto(165.81254672,270.60403204)(166.26254627,270.49403215)(166.60255104,270.26404267)
\curveto(166.95254558,270.03403261)(167.21254532,269.72903291)(167.38255104,269.34904267)
\curveto(167.42254511,269.25903338)(167.45754507,269.16403348)(167.48755104,269.06404267)
\curveto(167.51754501,268.96403368)(167.54254499,268.86403378)(167.56255104,268.76404267)
\curveto(167.58254495,268.73403391)(167.58754494,268.70403394)(167.57755104,268.67404267)
\curveto(167.57754495,268.644034)(167.58254495,268.61403403)(167.59255104,268.58404267)
\curveto(167.62254491,268.47403417)(167.64254489,268.34903429)(167.65255104,268.20904267)
\curveto(167.66254487,268.07903456)(167.67254486,267.9440347)(167.68255104,267.80404267)
\lineto(167.68255104,267.63904267)
\curveto(167.69254484,267.57903506)(167.69254484,267.52403512)(167.68255104,267.47404267)
\curveto(167.67254486,267.42403522)(167.66754486,267.37403527)(167.66755104,267.32404267)
\lineto(167.66755104,267.18904267)
\curveto(167.65754487,267.14903549)(167.65254488,267.10903553)(167.65255104,267.06904267)
\curveto(167.66254487,267.02903561)(167.65754487,266.98403566)(167.63755104,266.93404267)
\curveto(167.61754491,266.82403582)(167.59754493,266.71903592)(167.57755104,266.61904267)
\curveto(167.56754496,266.51903612)(167.54754498,266.41903622)(167.51755104,266.31904267)
\curveto(167.38754514,265.95903668)(167.22254531,265.644037)(167.02255104,265.37404267)
\curveto(166.82254571,265.10403754)(166.54754598,264.89903774)(166.19755104,264.75904267)
\curveto(166.11754641,264.72903791)(166.0325465,264.70403794)(165.94255104,264.68404267)
\lineto(165.67255104,264.62404267)
\curveto(165.62254691,264.61403803)(165.57754695,264.60903803)(165.53755104,264.60904267)
\curveto(165.49754703,264.61903802)(165.45754707,264.61903802)(165.41755104,264.60904267)
\curveto(165.31754721,264.58903805)(165.22254731,264.58903805)(165.13255104,264.60904267)
\moveto(164.29255104,266.00404267)
\curveto(164.3325482,265.93403671)(164.37254816,265.86903677)(164.41255104,265.80904267)
\curveto(164.45254808,265.75903688)(164.50254803,265.70903693)(164.56255104,265.65904267)
\lineto(164.71255104,265.53904267)
\curveto(164.77254776,265.50903713)(164.83754769,265.48403716)(164.90755104,265.46404267)
\curveto(164.94754758,265.4440372)(164.98254755,265.43403721)(165.01255104,265.43404267)
\curveto(165.05254748,265.4440372)(165.09254744,265.4390372)(165.13255104,265.41904267)
\curveto(165.16254737,265.41903722)(165.20254733,265.41403723)(165.25255104,265.40404267)
\curveto(165.30254723,265.40403724)(165.34254719,265.40903723)(165.37255104,265.41904267)
\lineto(165.59755104,265.46404267)
\curveto(165.84754668,265.5440371)(166.0325465,265.66903697)(166.15255104,265.83904267)
\curveto(166.2325463,265.9390367)(166.30254623,266.06903657)(166.36255104,266.22904267)
\curveto(166.44254609,266.40903623)(166.50254603,266.63403601)(166.54255104,266.90404267)
\curveto(166.58254595,267.18403546)(166.59754593,267.46403518)(166.58755104,267.74404267)
\curveto(166.57754595,268.03403461)(166.54754598,268.30903433)(166.49755104,268.56904267)
\curveto(166.44754608,268.82903381)(166.37254616,269.0390336)(166.27255104,269.19904267)
\curveto(166.15254638,269.39903324)(166.00254653,269.54903309)(165.82255104,269.64904267)
\curveto(165.74254679,269.69903294)(165.65254688,269.72903291)(165.55255104,269.73904267)
\curveto(165.45254708,269.75903288)(165.34754718,269.76903287)(165.23755104,269.76904267)
\curveto(165.21754731,269.75903288)(165.19254734,269.75403289)(165.16255104,269.75404267)
\curveto(165.14254739,269.76403288)(165.12254741,269.76403288)(165.10255104,269.75404267)
\curveto(165.05254748,269.7440329)(165.00754752,269.73403291)(164.96755104,269.72404267)
\curveto(164.9275476,269.72403292)(164.88754764,269.71403293)(164.84755104,269.69404267)
\curveto(164.66754786,269.61403303)(164.51754801,269.49403315)(164.39755104,269.33404267)
\curveto(164.28754824,269.17403347)(164.19754833,268.99403365)(164.12755104,268.79404267)
\curveto(164.06754846,268.60403404)(164.02254851,268.37903426)(163.99255104,268.11904267)
\curveto(163.97254856,267.85903478)(163.96754856,267.59403505)(163.97755104,267.32404267)
\curveto(163.98754854,267.06403558)(164.01754851,266.81403583)(164.06755104,266.57404267)
\curveto(164.1275484,266.3440363)(164.20254833,266.15403649)(164.29255104,266.00404267)
\moveto(175.09255104,263.01904267)
\curveto(175.10253743,262.96903967)(175.10753742,262.87903976)(175.10755104,262.74904267)
\curveto(175.10753742,262.61904002)(175.09753743,262.52904011)(175.07755104,262.47904267)
\curveto(175.05753747,262.42904021)(175.05253748,262.37404027)(175.06255104,262.31404267)
\curveto(175.07253746,262.26404038)(175.07253746,262.21404043)(175.06255104,262.16404267)
\curveto(175.02253751,262.02404062)(174.99253754,261.88904075)(174.97255104,261.75904267)
\curveto(174.96253757,261.62904101)(174.9325376,261.50904113)(174.88255104,261.39904267)
\curveto(174.74253779,261.04904159)(174.57753795,260.75404189)(174.38755104,260.51404267)
\curveto(174.19753833,260.28404236)(173.9275386,260.09904254)(173.57755104,259.95904267)
\curveto(173.49753903,259.92904271)(173.41253912,259.90904273)(173.32255104,259.89904267)
\curveto(173.2325393,259.87904276)(173.14753938,259.85904278)(173.06755104,259.83904267)
\curveto(173.01753951,259.82904281)(172.96753956,259.82404282)(172.91755104,259.82404267)
\curveto(172.86753966,259.82404282)(172.81753971,259.81904282)(172.76755104,259.80904267)
\curveto(172.73753979,259.79904284)(172.68753984,259.79904284)(172.61755104,259.80904267)
\curveto(172.54753998,259.80904283)(172.49754003,259.81404283)(172.46755104,259.82404267)
\curveto(172.40754012,259.8440428)(172.34754018,259.85404279)(172.28755104,259.85404267)
\curveto(172.23754029,259.8440428)(172.18754034,259.84904279)(172.13755104,259.86904267)
\curveto(172.04754048,259.88904275)(171.95754057,259.91404273)(171.86755104,259.94404267)
\curveto(171.78754074,259.96404268)(171.70754082,259.99404265)(171.62755104,260.03404267)
\curveto(171.30754122,260.17404247)(171.05754147,260.36904227)(170.87755104,260.61904267)
\curveto(170.69754183,260.87904176)(170.54754198,261.18404146)(170.42755104,261.53404267)
\curveto(170.40754212,261.61404103)(170.39254214,261.69904094)(170.38255104,261.78904267)
\curveto(170.37254216,261.87904076)(170.35754217,261.96404068)(170.33755104,262.04404267)
\curveto(170.3275422,262.07404057)(170.32254221,262.10404054)(170.32255104,262.13404267)
\lineto(170.32255104,262.23904267)
\curveto(170.30254223,262.31904032)(170.29254224,262.39904024)(170.29255104,262.47904267)
\lineto(170.29255104,262.61404267)
\curveto(170.27254226,262.71403993)(170.27254226,262.81403983)(170.29255104,262.91404267)
\lineto(170.29255104,263.09404267)
\curveto(170.30254223,263.1440395)(170.30754222,263.18903945)(170.30755104,263.22904267)
\curveto(170.30754222,263.27903936)(170.31254222,263.32403932)(170.32255104,263.36404267)
\curveto(170.3325422,263.40403924)(170.33754219,263.4390392)(170.33755104,263.46904267)
\curveto(170.33754219,263.50903913)(170.34254219,263.54903909)(170.35255104,263.58904267)
\lineto(170.41255104,263.91904267)
\curveto(170.4325421,264.0390386)(170.46254207,264.14903849)(170.50255104,264.24904267)
\curveto(170.64254189,264.57903806)(170.80254173,264.85403779)(170.98255104,265.07404267)
\curveto(171.17254136,265.30403734)(171.4325411,265.48903715)(171.76255104,265.62904267)
\curveto(171.84254069,265.66903697)(171.9275406,265.69403695)(172.01755104,265.70404267)
\lineto(172.31755104,265.76404267)
\lineto(172.45255104,265.76404267)
\curveto(172.50254003,265.77403687)(172.55253998,265.77903686)(172.60255104,265.77904267)
\curveto(173.17253936,265.79903684)(173.6325389,265.69403695)(173.98255104,265.46404267)
\curveto(174.34253819,265.2440374)(174.60753792,264.9440377)(174.77755104,264.56404267)
\curveto(174.8275377,264.46403818)(174.86753766,264.36403828)(174.89755104,264.26404267)
\curveto(174.9275376,264.16403848)(174.95753757,264.05903858)(174.98755104,263.94904267)
\curveto(174.99753753,263.90903873)(175.00253753,263.87403877)(175.00255104,263.84404267)
\curveto(175.00253753,263.82403882)(175.00753752,263.79403885)(175.01755104,263.75404267)
\curveto(175.03753749,263.68403896)(175.04753748,263.60903903)(175.04755104,263.52904267)
\curveto(175.04753748,263.44903919)(175.05753747,263.36903927)(175.07755104,263.28904267)
\curveto(175.07753745,263.2390394)(175.07753745,263.19403945)(175.07755104,263.15404267)
\curveto(175.07753745,263.11403953)(175.08253745,263.06903957)(175.09255104,263.01904267)
\moveto(173.98255104,262.58404267)
\curveto(173.99253854,262.63404001)(173.99753853,262.70903993)(173.99755104,262.80904267)
\curveto(174.00753852,262.90903973)(174.00253853,262.98403966)(173.98255104,263.03404267)
\curveto(173.96253857,263.09403955)(173.95753857,263.14903949)(173.96755104,263.19904267)
\curveto(173.98753854,263.25903938)(173.98753854,263.31903932)(173.96755104,263.37904267)
\curveto(173.95753857,263.40903923)(173.95253858,263.4440392)(173.95255104,263.48404267)
\curveto(173.95253858,263.52403912)(173.94753858,263.56403908)(173.93755104,263.60404267)
\curveto(173.91753861,263.68403896)(173.89753863,263.75903888)(173.87755104,263.82904267)
\curveto(173.86753866,263.90903873)(173.85253868,263.98903865)(173.83255104,264.06904267)
\curveto(173.80253873,264.12903851)(173.77753875,264.18903845)(173.75755104,264.24904267)
\curveto(173.73753879,264.30903833)(173.70753882,264.36903827)(173.66755104,264.42904267)
\curveto(173.56753896,264.59903804)(173.43753909,264.73403791)(173.27755104,264.83404267)
\curveto(173.19753933,264.88403776)(173.10253943,264.91903772)(172.99255104,264.93904267)
\curveto(172.88253965,264.95903768)(172.75753977,264.96903767)(172.61755104,264.96904267)
\curveto(172.59753993,264.95903768)(172.57253996,264.95403769)(172.54255104,264.95404267)
\curveto(172.51254002,264.96403768)(172.48254005,264.96403768)(172.45255104,264.95404267)
\lineto(172.30255104,264.89404267)
\curveto(172.25254028,264.88403776)(172.20754032,264.86903777)(172.16755104,264.84904267)
\curveto(171.97754055,264.7390379)(171.8325407,264.59403805)(171.73255104,264.41404267)
\curveto(171.64254089,264.23403841)(171.56254097,264.02903861)(171.49255104,263.79904267)
\curveto(171.45254108,263.66903897)(171.4325411,263.53403911)(171.43255104,263.39404267)
\curveto(171.4325411,263.26403938)(171.42254111,263.11903952)(171.40255104,262.95904267)
\curveto(171.39254114,262.90903973)(171.38254115,262.84903979)(171.37255104,262.77904267)
\curveto(171.37254116,262.70903993)(171.38254115,262.64903999)(171.40255104,262.59904267)
\lineto(171.40255104,262.43404267)
\lineto(171.40255104,262.25404267)
\curveto(171.41254112,262.20404044)(171.42254111,262.14904049)(171.43255104,262.08904267)
\curveto(171.44254109,262.0390406)(171.44754108,261.98404066)(171.44755104,261.92404267)
\curveto(171.45754107,261.86404078)(171.47254106,261.80904083)(171.49255104,261.75904267)
\curveto(171.54254099,261.56904107)(171.60254093,261.39404125)(171.67255104,261.23404267)
\curveto(171.74254079,261.07404157)(171.84754068,260.9440417)(171.98755104,260.84404267)
\curveto(172.11754041,260.7440419)(172.25754027,260.67404197)(172.40755104,260.63404267)
\curveto(172.43754009,260.62404202)(172.46254007,260.61904202)(172.48255104,260.61904267)
\curveto(172.51254002,260.62904201)(172.54253999,260.62904201)(172.57255104,260.61904267)
\curveto(172.59253994,260.61904202)(172.62253991,260.61404203)(172.66255104,260.60404267)
\curveto(172.70253983,260.60404204)(172.73753979,260.60904203)(172.76755104,260.61904267)
\curveto(172.80753972,260.62904201)(172.84753968,260.63404201)(172.88755104,260.63404267)
\curveto(172.9275396,260.63404201)(172.96753956,260.644042)(173.00755104,260.66404267)
\curveto(173.24753928,260.7440419)(173.44253909,260.87904176)(173.59255104,261.06904267)
\curveto(173.71253882,261.24904139)(173.80253873,261.45404119)(173.86255104,261.68404267)
\curveto(173.88253865,261.75404089)(173.89753863,261.82404082)(173.90755104,261.89404267)
\curveto(173.91753861,261.97404067)(173.9325386,262.05404059)(173.95255104,262.13404267)
\curveto(173.95253858,262.19404045)(173.95753857,262.2390404)(173.96755104,262.26904267)
\curveto(173.96753856,262.28904035)(173.96753856,262.31404033)(173.96755104,262.34404267)
\curveto(173.96753856,262.38404026)(173.97253856,262.41404023)(173.98255104,262.43404267)
\lineto(173.98255104,262.58404267)
}
}
{
\newrgbcolor{curcolor}{0 0 0}
\pscustom[linestyle=none,fillstyle=solid,fillcolor=curcolor]
{
\newpath
\moveto(103.05499306,225.3991037)
\lineto(106.65499306,225.3991037)
\lineto(107.29999306,225.3991037)
\curveto(107.37998653,225.39909328)(107.45498645,225.39409328)(107.52499306,225.3841037)
\curveto(107.59498631,225.38409329)(107.65498625,225.3740933)(107.70499306,225.3541037)
\curveto(107.77498613,225.32409335)(107.82998608,225.26409341)(107.86999306,225.1741037)
\curveto(107.88998602,225.14409353)(107.89998601,225.10409357)(107.89999306,225.0541037)
\lineto(107.89999306,224.9191037)
\curveto(107.909986,224.80909387)(107.904986,224.70409397)(107.88499306,224.6041037)
\curveto(107.87498603,224.50409417)(107.83998607,224.43409424)(107.77999306,224.3941037)
\curveto(107.68998622,224.32409435)(107.55498635,224.28909439)(107.37499306,224.2891037)
\curveto(107.19498671,224.29909438)(107.02998688,224.30409437)(106.87999306,224.3041037)
\lineto(104.88499306,224.3041037)
\lineto(104.38999306,224.3041037)
\lineto(104.25499306,224.3041037)
\curveto(104.21498969,224.30409437)(104.17498973,224.29909438)(104.13499306,224.2891037)
\lineto(103.92499306,224.2891037)
\curveto(103.81499009,224.25909442)(103.73499017,224.21909446)(103.68499306,224.1691037)
\curveto(103.63499027,224.12909455)(103.59999031,224.0740946)(103.57999306,224.0041037)
\curveto(103.55999035,223.94409473)(103.54499036,223.8740948)(103.53499306,223.7941037)
\curveto(103.52499038,223.71409496)(103.5049904,223.62409505)(103.47499306,223.5241037)
\curveto(103.42499048,223.32409535)(103.38499052,223.11909556)(103.35499306,222.9091037)
\curveto(103.32499058,222.69909598)(103.28499062,222.49409618)(103.23499306,222.2941037)
\curveto(103.21499069,222.22409645)(103.2049907,222.15409652)(103.20499306,222.0841037)
\curveto(103.2049907,222.02409665)(103.19499071,221.95909672)(103.17499306,221.8891037)
\curveto(103.16499074,221.85909682)(103.15499075,221.81909686)(103.14499306,221.7691037)
\curveto(103.14499076,221.72909695)(103.14999076,221.68909699)(103.15999306,221.6491037)
\curveto(103.17999073,221.59909708)(103.2049907,221.55409712)(103.23499306,221.5141037)
\curveto(103.27499063,221.48409719)(103.33499057,221.4790972)(103.41499306,221.4991037)
\curveto(103.47499043,221.51909716)(103.53499037,221.54409713)(103.59499306,221.5741037)
\curveto(103.65499025,221.61409706)(103.71499019,221.64909703)(103.77499306,221.6791037)
\curveto(103.83499007,221.69909698)(103.88499002,221.71409696)(103.92499306,221.7241037)
\curveto(104.11498979,221.80409687)(104.31998959,221.85909682)(104.53999306,221.8891037)
\curveto(104.76998914,221.91909676)(104.99998891,221.92909675)(105.22999306,221.9191037)
\curveto(105.46998844,221.91909676)(105.69998821,221.89409678)(105.91999306,221.8441037)
\curveto(106.13998777,221.80409687)(106.33998757,221.74409693)(106.51999306,221.6641037)
\curveto(106.56998734,221.64409703)(106.61498729,221.62409705)(106.65499306,221.6041037)
\curveto(106.7049872,221.58409709)(106.75498715,221.55909712)(106.80499306,221.5291037)
\curveto(107.15498675,221.31909736)(107.43498647,221.08909759)(107.64499306,220.8391037)
\curveto(107.86498604,220.58909809)(108.05998585,220.26409841)(108.22999306,219.8641037)
\curveto(108.27998563,219.75409892)(108.31498559,219.64409903)(108.33499306,219.5341037)
\curveto(108.35498555,219.42409925)(108.37998553,219.30909937)(108.40999306,219.1891037)
\curveto(108.41998549,219.15909952)(108.42498548,219.11409956)(108.42499306,219.0541037)
\curveto(108.44498546,218.99409968)(108.45498545,218.92409975)(108.45499306,218.8441037)
\curveto(108.45498545,218.7740999)(108.46498544,218.70909997)(108.48499306,218.6491037)
\lineto(108.48499306,218.4841037)
\curveto(108.49498541,218.43410024)(108.49998541,218.36410031)(108.49999306,218.2741037)
\curveto(108.49998541,218.18410049)(108.48998542,218.11410056)(108.46999306,218.0641037)
\curveto(108.44998546,218.00410067)(108.44498546,217.94410073)(108.45499306,217.8841037)
\curveto(108.46498544,217.83410084)(108.45998545,217.78410089)(108.43999306,217.7341037)
\curveto(108.39998551,217.5741011)(108.36498554,217.42410125)(108.33499306,217.2841037)
\curveto(108.3049856,217.14410153)(108.25998565,217.00910167)(108.19999306,216.8791037)
\curveto(108.03998587,216.50910217)(107.81998609,216.1741025)(107.53999306,215.8741037)
\curveto(107.25998665,215.5741031)(106.93998697,215.34410333)(106.57999306,215.1841037)
\curveto(106.4099875,215.10410357)(106.2099877,215.02910365)(105.97999306,214.9591037)
\curveto(105.86998804,214.91910376)(105.75498815,214.89410378)(105.63499306,214.8841037)
\curveto(105.51498839,214.8741038)(105.39498851,214.85410382)(105.27499306,214.8241037)
\curveto(105.22498868,214.80410387)(105.16998874,214.80410387)(105.10999306,214.8241037)
\curveto(105.04998886,214.83410384)(104.98998892,214.82910385)(104.92999306,214.8091037)
\curveto(104.82998908,214.78910389)(104.72998918,214.78910389)(104.62999306,214.8091037)
\lineto(104.49499306,214.8091037)
\curveto(104.44498946,214.82910385)(104.38498952,214.83910384)(104.31499306,214.8391037)
\curveto(104.25498965,214.82910385)(104.19998971,214.83410384)(104.14999306,214.8541037)
\curveto(104.1099898,214.86410381)(104.07498983,214.86910381)(104.04499306,214.8691037)
\curveto(104.01498989,214.86910381)(103.97998993,214.8741038)(103.93999306,214.8841037)
\lineto(103.66999306,214.9441037)
\curveto(103.57999033,214.96410371)(103.49499041,214.99410368)(103.41499306,215.0341037)
\curveto(103.07499083,215.1741035)(102.78499112,215.32910335)(102.54499306,215.4991037)
\curveto(102.3049916,215.679103)(102.08499182,215.90910277)(101.88499306,216.1891037)
\curveto(101.73499217,216.41910226)(101.61999229,216.65910202)(101.53999306,216.9091037)
\curveto(101.51999239,216.95910172)(101.5099924,217.00410167)(101.50999306,217.0441037)
\curveto(101.5099924,217.09410158)(101.49999241,217.14410153)(101.47999306,217.1941037)
\curveto(101.45999245,217.25410142)(101.44499246,217.33410134)(101.43499306,217.4341037)
\curveto(101.43499247,217.53410114)(101.45499245,217.60910107)(101.49499306,217.6591037)
\curveto(101.54499236,217.73910094)(101.62499228,217.78410089)(101.73499306,217.7941037)
\curveto(101.84499206,217.80410087)(101.95999195,217.80910087)(102.07999306,217.8091037)
\lineto(102.24499306,217.8091037)
\curveto(102.3049916,217.80910087)(102.35999155,217.79910088)(102.40999306,217.7791037)
\curveto(102.49999141,217.75910092)(102.56999134,217.71910096)(102.61999306,217.6591037)
\curveto(102.68999122,217.56910111)(102.73499117,217.45910122)(102.75499306,217.3291037)
\curveto(102.78499112,217.20910147)(102.82999108,217.10410157)(102.88999306,217.0141037)
\curveto(103.07999083,216.674102)(103.33999057,216.40410227)(103.66999306,216.2041037)
\curveto(103.76999014,216.14410253)(103.87499003,216.09410258)(103.98499306,216.0541037)
\curveto(104.1049898,216.02410265)(104.22498968,215.98910269)(104.34499306,215.9491037)
\curveto(104.51498939,215.89910278)(104.71998919,215.8791028)(104.95999306,215.8891037)
\curveto(105.2099887,215.90910277)(105.4099885,215.94410273)(105.55999306,215.9941037)
\curveto(105.92998798,216.11410256)(106.21998769,216.2741024)(106.42999306,216.4741037)
\curveto(106.64998726,216.68410199)(106.82998708,216.96410171)(106.96999306,217.3141037)
\curveto(107.01998689,217.41410126)(107.04998686,217.51910116)(107.05999306,217.6291037)
\curveto(107.07998683,217.73910094)(107.1049868,217.85410082)(107.13499306,217.9741037)
\lineto(107.13499306,218.0791037)
\curveto(107.14498676,218.11910056)(107.14998676,218.15910052)(107.14999306,218.1991037)
\curveto(107.15998675,218.22910045)(107.15998675,218.26410041)(107.14999306,218.3041037)
\lineto(107.14999306,218.4241037)
\curveto(107.14998676,218.68409999)(107.11998679,218.92909975)(107.05999306,219.1591037)
\curveto(106.94998696,219.50909917)(106.79498711,219.80409887)(106.59499306,220.0441037)
\curveto(106.39498751,220.29409838)(106.13498777,220.48909819)(105.81499306,220.6291037)
\lineto(105.63499306,220.6891037)
\curveto(105.58498832,220.70909797)(105.52498838,220.72909795)(105.45499306,220.7491037)
\curveto(105.4049885,220.76909791)(105.34498856,220.7790979)(105.27499306,220.7791037)
\curveto(105.21498869,220.78909789)(105.14998876,220.80409787)(105.07999306,220.8241037)
\lineto(104.92999306,220.8241037)
\curveto(104.88998902,220.84409783)(104.83498907,220.85409782)(104.76499306,220.8541037)
\curveto(104.7049892,220.85409782)(104.64998926,220.84409783)(104.59999306,220.8241037)
\lineto(104.49499306,220.8241037)
\curveto(104.46498944,220.82409785)(104.42998948,220.81909786)(104.38999306,220.8091037)
\lineto(104.14999306,220.7491037)
\curveto(104.06998984,220.73909794)(103.98998992,220.71909796)(103.90999306,220.6891037)
\curveto(103.66999024,220.58909809)(103.43999047,220.45409822)(103.21999306,220.2841037)
\curveto(103.12999078,220.21409846)(103.04499086,220.13909854)(102.96499306,220.0591037)
\curveto(102.88499102,219.98909869)(102.78499112,219.93409874)(102.66499306,219.8941037)
\curveto(102.57499133,219.86409881)(102.43499147,219.85409882)(102.24499306,219.8641037)
\curveto(102.06499184,219.8740988)(101.94499196,219.89909878)(101.88499306,219.9391037)
\curveto(101.83499207,219.9790987)(101.79499211,220.03909864)(101.76499306,220.1191037)
\curveto(101.74499216,220.19909848)(101.74499216,220.28409839)(101.76499306,220.3741037)
\curveto(101.79499211,220.49409818)(101.81499209,220.61409806)(101.82499306,220.7341037)
\curveto(101.84499206,220.86409781)(101.86999204,220.98909769)(101.89999306,221.1091037)
\curveto(101.91999199,221.14909753)(101.92499198,221.18409749)(101.91499306,221.2141037)
\curveto(101.91499199,221.25409742)(101.92499198,221.29909738)(101.94499306,221.3491037)
\curveto(101.96499194,221.43909724)(101.97999193,221.52909715)(101.98999306,221.6191037)
\curveto(101.99999191,221.71909696)(102.01999189,221.81409686)(102.04999306,221.9041037)
\curveto(102.05999185,221.96409671)(102.06499184,222.02409665)(102.06499306,222.0841037)
\curveto(102.07499183,222.14409653)(102.08999182,222.20409647)(102.10999306,222.2641037)
\curveto(102.15999175,222.46409621)(102.19499171,222.66909601)(102.21499306,222.8791037)
\curveto(102.24499166,223.09909558)(102.28499162,223.30909537)(102.33499306,223.5091037)
\curveto(102.36499154,223.60909507)(102.38499152,223.70909497)(102.39499306,223.8091037)
\curveto(102.4049915,223.90909477)(102.41999149,224.00909467)(102.43999306,224.1091037)
\curveto(102.44999146,224.13909454)(102.45499145,224.1790945)(102.45499306,224.2291037)
\curveto(102.48499142,224.33909434)(102.5049914,224.44409423)(102.51499306,224.5441037)
\curveto(102.53499137,224.65409402)(102.55999135,224.76409391)(102.58999306,224.8741037)
\curveto(102.6099913,224.95409372)(102.62499128,225.02409365)(102.63499306,225.0841037)
\curveto(102.64499126,225.15409352)(102.66999124,225.21409346)(102.70999306,225.2641037)
\curveto(102.72999118,225.29409338)(102.75999115,225.31409336)(102.79999306,225.3241037)
\curveto(102.83999107,225.34409333)(102.88499102,225.36409331)(102.93499306,225.3841037)
\curveto(102.99499091,225.38409329)(103.03499087,225.38909329)(103.05499306,225.3991037)
}
}
{
\newrgbcolor{curcolor}{0 0 0}
\pscustom[linestyle=none,fillstyle=solid,fillcolor=curcolor]
{
\newpath
\moveto(110.84960243,216.6241037)
\lineto(111.14960243,216.6241037)
\curveto(111.25960037,216.63410204)(111.36460027,216.63410204)(111.46460243,216.6241037)
\curveto(111.57460006,216.62410205)(111.67459996,216.61410206)(111.76460243,216.5941037)
\curveto(111.85459978,216.58410209)(111.92459971,216.55910212)(111.97460243,216.5191037)
\curveto(111.99459964,216.49910218)(112.00959962,216.46910221)(112.01960243,216.4291037)
\curveto(112.03959959,216.38910229)(112.05959957,216.34410233)(112.07960243,216.2941037)
\lineto(112.07960243,216.2191037)
\curveto(112.08959954,216.16910251)(112.08959954,216.11410256)(112.07960243,216.0541037)
\lineto(112.07960243,215.9041037)
\lineto(112.07960243,215.4241037)
\curveto(112.07959955,215.25410342)(112.03959959,215.13410354)(111.95960243,215.0641037)
\curveto(111.88959974,215.01410366)(111.79959983,214.98910369)(111.68960243,214.9891037)
\lineto(111.35960243,214.9891037)
\lineto(110.90960243,214.9891037)
\curveto(110.75960087,214.98910369)(110.64460099,215.01910366)(110.56460243,215.0791037)
\curveto(110.52460111,215.10910357)(110.49460114,215.15910352)(110.47460243,215.2291037)
\curveto(110.45460118,215.30910337)(110.43960119,215.39410328)(110.42960243,215.4841037)
\lineto(110.42960243,215.7691037)
\curveto(110.43960119,215.86910281)(110.44460119,215.95410272)(110.44460243,216.0241037)
\lineto(110.44460243,216.2191037)
\curveto(110.44460119,216.2791024)(110.45460118,216.33410234)(110.47460243,216.3841037)
\curveto(110.51460112,216.49410218)(110.58460105,216.56410211)(110.68460243,216.5941037)
\curveto(110.71460092,216.59410208)(110.76960086,216.60410207)(110.84960243,216.6241037)
}
}
{
\newrgbcolor{curcolor}{0 0 0}
\pscustom[linestyle=none,fillstyle=solid,fillcolor=curcolor]
{
\newpath
\moveto(117.16975868,225.5941037)
\curveto(118.79975324,225.62409305)(119.84975219,225.06909361)(120.31975868,223.9291037)
\curveto(120.41975162,223.69909498)(120.48475156,223.40909527)(120.51475868,223.0591037)
\curveto(120.55475149,222.71909596)(120.52975151,222.40909627)(120.43975868,222.1291037)
\curveto(120.34975169,221.86909681)(120.22975181,221.64409703)(120.07975868,221.4541037)
\curveto(120.05975198,221.41409726)(120.03475201,221.3790973)(120.00475868,221.3491037)
\curveto(119.97475207,221.32909735)(119.94975209,221.30409737)(119.92975868,221.2741037)
\lineto(119.83975868,221.1541037)
\curveto(119.80975223,221.12409755)(119.77475227,221.09909758)(119.73475868,221.0791037)
\curveto(119.68475236,221.02909765)(119.62975241,220.98409769)(119.56975868,220.9441037)
\curveto(119.51975252,220.90409777)(119.47475257,220.85409782)(119.43475868,220.7941037)
\curveto(119.39475265,220.75409792)(119.37975266,220.70409797)(119.38975868,220.6441037)
\curveto(119.39975264,220.59409808)(119.42975261,220.54909813)(119.47975868,220.5091037)
\curveto(119.52975251,220.46909821)(119.58475246,220.42909825)(119.64475868,220.3891037)
\curveto(119.71475233,220.35909832)(119.77975226,220.32909835)(119.83975868,220.2991037)
\curveto(119.89975214,220.26909841)(119.94975209,220.23909844)(119.98975868,220.2091037)
\curveto(120.30975173,219.98909869)(120.56475148,219.679099)(120.75475868,219.2791037)
\curveto(120.79475125,219.18909949)(120.82475122,219.09409958)(120.84475868,218.9941037)
\curveto(120.87475117,218.90409977)(120.89975114,218.81409986)(120.91975868,218.7241037)
\curveto(120.92975111,218.6741)(120.93475111,218.62410005)(120.93475868,218.5741037)
\curveto(120.9447511,218.53410014)(120.95475109,218.48910019)(120.96475868,218.4391037)
\curveto(120.97475107,218.38910029)(120.97475107,218.33910034)(120.96475868,218.2891037)
\curveto(120.95475109,218.23910044)(120.95975108,218.18910049)(120.97975868,218.1391037)
\curveto(120.98975105,218.08910059)(120.99475105,218.02910065)(120.99475868,217.9591037)
\curveto(120.99475105,217.88910079)(120.98475106,217.82910085)(120.96475868,217.7791037)
\lineto(120.96475868,217.5541037)
\lineto(120.90475868,217.3141037)
\curveto(120.89475115,217.24410143)(120.87975116,217.1741015)(120.85975868,217.1041037)
\curveto(120.82975121,217.01410166)(120.79975124,216.92910175)(120.76975868,216.8491037)
\curveto(120.74975129,216.76910191)(120.71975132,216.68910199)(120.67975868,216.6091037)
\curveto(120.65975138,216.54910213)(120.62975141,216.48910219)(120.58975868,216.4291037)
\curveto(120.55975148,216.3791023)(120.52475152,216.32910235)(120.48475868,216.2791037)
\curveto(120.28475176,215.96910271)(120.03475201,215.70910297)(119.73475868,215.4991037)
\curveto(119.43475261,215.29910338)(119.08975295,215.13410354)(118.69975868,215.0041037)
\curveto(118.57975346,214.96410371)(118.44975359,214.93910374)(118.30975868,214.9291037)
\curveto(118.17975386,214.90910377)(118.044754,214.88410379)(117.90475868,214.8541037)
\curveto(117.83475421,214.84410383)(117.76475428,214.83910384)(117.69475868,214.8391037)
\curveto(117.63475441,214.83910384)(117.56975447,214.83410384)(117.49975868,214.8241037)
\curveto(117.45975458,214.81410386)(117.39975464,214.80910387)(117.31975868,214.8091037)
\curveto(117.24975479,214.80910387)(117.19975484,214.81410386)(117.16975868,214.8241037)
\curveto(117.11975492,214.83410384)(117.07475497,214.83910384)(117.03475868,214.8391037)
\lineto(116.91475868,214.8391037)
\curveto(116.81475523,214.85910382)(116.71475533,214.8741038)(116.61475868,214.8841037)
\curveto(116.51475553,214.89410378)(116.41975562,214.90910377)(116.32975868,214.9291037)
\curveto(116.21975582,214.95910372)(116.10975593,214.98410369)(115.99975868,215.0041037)
\curveto(115.89975614,215.03410364)(115.79475625,215.0741036)(115.68475868,215.1241037)
\curveto(115.31475673,215.28410339)(114.99975704,215.48410319)(114.73975868,215.7241037)
\curveto(114.47975756,215.9741027)(114.26975777,216.28410239)(114.10975868,216.6541037)
\curveto(114.06975797,216.74410193)(114.03475801,216.83910184)(114.00475868,216.9391037)
\curveto(113.97475807,217.03910164)(113.9447581,217.14410153)(113.91475868,217.2541037)
\curveto(113.89475815,217.30410137)(113.88475816,217.35410132)(113.88475868,217.4041037)
\curveto(113.88475816,217.46410121)(113.87475817,217.52410115)(113.85475868,217.5841037)
\curveto(113.83475821,217.64410103)(113.82475822,217.72410095)(113.82475868,217.8241037)
\curveto(113.82475822,217.92410075)(113.8397582,217.99910068)(113.86975868,218.0491037)
\curveto(113.87975816,218.0791006)(113.89475815,218.10410057)(113.91475868,218.1241037)
\lineto(113.97475868,218.1841037)
\curveto(114.01475803,218.20410047)(114.07475797,218.21910046)(114.15475868,218.2291037)
\curveto(114.2447578,218.23910044)(114.33475771,218.24410043)(114.42475868,218.2441037)
\curveto(114.51475753,218.24410043)(114.59975744,218.23910044)(114.67975868,218.2291037)
\curveto(114.76975727,218.21910046)(114.83475721,218.20910047)(114.87475868,218.1991037)
\curveto(114.89475715,218.1791005)(114.91475713,218.16410051)(114.93475868,218.1541037)
\curveto(114.95475709,218.15410052)(114.97475707,218.14410053)(114.99475868,218.1241037)
\curveto(115.06475698,218.03410064)(115.10475694,217.91910076)(115.11475868,217.7791037)
\curveto(115.13475691,217.63910104)(115.16475688,217.51410116)(115.20475868,217.4041037)
\lineto(115.35475868,217.0441037)
\curveto(115.40475664,216.93410174)(115.46975657,216.82910185)(115.54975868,216.7291037)
\curveto(115.56975647,216.69910198)(115.58975645,216.674102)(115.60975868,216.6541037)
\curveto(115.6397564,216.63410204)(115.66475638,216.60910207)(115.68475868,216.5791037)
\curveto(115.72475632,216.51910216)(115.75975628,216.4741022)(115.78975868,216.4441037)
\curveto(115.82975621,216.41410226)(115.86475618,216.38410229)(115.89475868,216.3541037)
\curveto(115.93475611,216.32410235)(115.97975606,216.29410238)(116.02975868,216.2641037)
\curveto(116.11975592,216.20410247)(116.21475583,216.15410252)(116.31475868,216.1141037)
\lineto(116.64475868,215.9941037)
\curveto(116.79475525,215.94410273)(116.99475505,215.91410276)(117.24475868,215.9041037)
\curveto(117.49475455,215.89410278)(117.70475434,215.91410276)(117.87475868,215.9641037)
\curveto(117.95475409,215.98410269)(118.02475402,215.99910268)(118.08475868,216.0091037)
\lineto(118.29475868,216.0691037)
\curveto(118.57475347,216.18910249)(118.81475323,216.33910234)(119.01475868,216.5191037)
\curveto(119.22475282,216.69910198)(119.38975265,216.92910175)(119.50975868,217.2091037)
\curveto(119.5397525,217.2791014)(119.55975248,217.34910133)(119.56975868,217.4191037)
\lineto(119.62975868,217.6591037)
\curveto(119.66975237,217.79910088)(119.67975236,217.95910072)(119.65975868,218.1391037)
\curveto(119.6397524,218.32910035)(119.60975243,218.4791002)(119.56975868,218.5891037)
\curveto(119.4397526,218.96909971)(119.25475279,219.25909942)(119.01475868,219.4591037)
\curveto(118.78475326,219.65909902)(118.47475357,219.81909886)(118.08475868,219.9391037)
\curveto(117.97475407,219.96909871)(117.85475419,219.98909869)(117.72475868,219.9991037)
\curveto(117.60475444,220.00909867)(117.47975456,220.01409866)(117.34975868,220.0141037)
\curveto(117.18975485,220.01409866)(117.04975499,220.01909866)(116.92975868,220.0291037)
\curveto(116.80975523,220.03909864)(116.72475532,220.09909858)(116.67475868,220.2091037)
\curveto(116.65475539,220.23909844)(116.6447554,220.2740984)(116.64475868,220.3141037)
\lineto(116.64475868,220.4491037)
\curveto(116.63475541,220.54909813)(116.63475541,220.64409803)(116.64475868,220.7341037)
\curveto(116.66475538,220.82409785)(116.70475534,220.88909779)(116.76475868,220.9291037)
\curveto(116.80475524,220.95909772)(116.8447552,220.9790977)(116.88475868,220.9891037)
\curveto(116.93475511,220.99909768)(116.98975505,221.00909767)(117.04975868,221.0191037)
\curveto(117.06975497,221.02909765)(117.09475495,221.02909765)(117.12475868,221.0191037)
\curveto(117.15475489,221.01909766)(117.17975486,221.02409765)(117.19975868,221.0341037)
\lineto(117.33475868,221.0341037)
\curveto(117.4447546,221.05409762)(117.5447545,221.06409761)(117.63475868,221.0641037)
\curveto(117.73475431,221.0740976)(117.82975421,221.09409758)(117.91975868,221.1241037)
\curveto(118.2397538,221.23409744)(118.49475355,221.3790973)(118.68475868,221.5591037)
\curveto(118.87475317,221.73909694)(119.02475302,221.98909669)(119.13475868,222.3091037)
\curveto(119.16475288,222.40909627)(119.18475286,222.53409614)(119.19475868,222.6841037)
\curveto(119.21475283,222.84409583)(119.20975283,222.98909569)(119.17975868,223.1191037)
\curveto(119.15975288,223.18909549)(119.1397529,223.25409542)(119.11975868,223.3141037)
\curveto(119.10975293,223.38409529)(119.08975295,223.44909523)(119.05975868,223.5091037)
\curveto(118.95975308,223.74909493)(118.81475323,223.93909474)(118.62475868,224.0791037)
\curveto(118.43475361,224.21909446)(118.20975383,224.32909435)(117.94975868,224.4091037)
\curveto(117.88975415,224.42909425)(117.82975421,224.43909424)(117.76975868,224.4391037)
\curveto(117.70975433,224.43909424)(117.6447544,224.44909423)(117.57475868,224.4691037)
\curveto(117.49475455,224.48909419)(117.39975464,224.49909418)(117.28975868,224.4991037)
\curveto(117.17975486,224.49909418)(117.08475496,224.48909419)(117.00475868,224.4691037)
\curveto(116.95475509,224.44909423)(116.90475514,224.43909424)(116.85475868,224.4391037)
\curveto(116.81475523,224.43909424)(116.76975527,224.42909425)(116.71975868,224.4091037)
\curveto(116.5397555,224.35909432)(116.36975567,224.28409439)(116.20975868,224.1841037)
\curveto(116.05975598,224.09409458)(115.92975611,223.9790947)(115.81975868,223.8391037)
\curveto(115.72975631,223.71909496)(115.64975639,223.58909509)(115.57975868,223.4491037)
\curveto(115.50975653,223.30909537)(115.4447566,223.15409552)(115.38475868,222.9841037)
\curveto(115.35475669,222.8740958)(115.33475671,222.75409592)(115.32475868,222.6241037)
\curveto(115.31475673,222.50409617)(115.27975676,222.40409627)(115.21975868,222.3241037)
\curveto(115.19975684,222.28409639)(115.1397569,222.24409643)(115.03975868,222.2041037)
\curveto(114.99975704,222.19409648)(114.9397571,222.18409649)(114.85975868,222.1741037)
\lineto(114.60475868,222.1741037)
\curveto(114.51475753,222.18409649)(114.42975761,222.19409648)(114.34975868,222.2041037)
\curveto(114.27975776,222.21409646)(114.22975781,222.22909645)(114.19975868,222.2491037)
\curveto(114.15975788,222.2790964)(114.12475792,222.33409634)(114.09475868,222.4141037)
\curveto(114.06475798,222.49409618)(114.05975798,222.5790961)(114.07975868,222.6691037)
\curveto(114.08975795,222.71909596)(114.09475795,222.76909591)(114.09475868,222.8191037)
\lineto(114.12475868,222.9991037)
\curveto(114.15475789,223.09909558)(114.17975786,223.19909548)(114.19975868,223.2991037)
\curveto(114.22975781,223.39909528)(114.26475778,223.48909519)(114.30475868,223.5691037)
\curveto(114.35475769,223.679095)(114.39975764,223.78409489)(114.43975868,223.8841037)
\curveto(114.47975756,223.99409468)(114.52975751,224.09909458)(114.58975868,224.1991037)
\curveto(114.91975712,224.73909394)(115.38975665,225.13409354)(115.99975868,225.3841037)
\curveto(116.11975592,225.43409324)(116.2447558,225.46909321)(116.37475868,225.4891037)
\curveto(116.51475553,225.50909317)(116.65475539,225.53409314)(116.79475868,225.5641037)
\curveto(116.85475519,225.5740931)(116.91475513,225.5790931)(116.97475868,225.5791037)
\curveto(117.044755,225.5790931)(117.10975493,225.58409309)(117.16975868,225.5941037)
}
}
{
\newrgbcolor{curcolor}{0 0 0}
\pscustom[linestyle=none,fillstyle=solid,fillcolor=curcolor]
{
\newpath
\moveto(132.20936806,223.5091037)
\curveto(132.00935776,223.21909546)(131.79935797,222.93409574)(131.57936806,222.6541037)
\curveto(131.3693584,222.3740963)(131.1643586,222.08909659)(130.96436806,221.7991037)
\curveto(130.3643594,220.94909773)(129.75936001,220.10909857)(129.14936806,219.2791037)
\curveto(128.53936123,218.45910022)(127.93436183,217.62410105)(127.33436806,216.7741037)
\lineto(126.82436806,216.0541037)
\lineto(126.31436806,215.3641037)
\curveto(126.23436353,215.25410342)(126.15436361,215.13910354)(126.07436806,215.0191037)
\curveto(125.99436377,214.89910378)(125.89936387,214.80410387)(125.78936806,214.7341037)
\curveto(125.74936402,214.71410396)(125.68436408,214.69910398)(125.59436806,214.6891037)
\curveto(125.51436425,214.66910401)(125.42436434,214.65910402)(125.32436806,214.6591037)
\curveto(125.22436454,214.65910402)(125.12936464,214.66410401)(125.03936806,214.6741037)
\curveto(124.95936481,214.68410399)(124.89936487,214.70410397)(124.85936806,214.7341037)
\curveto(124.82936494,214.75410392)(124.80436496,214.78910389)(124.78436806,214.8391037)
\curveto(124.77436499,214.8791038)(124.77936499,214.92410375)(124.79936806,214.9741037)
\curveto(124.83936493,215.05410362)(124.88436488,215.12910355)(124.93436806,215.1991037)
\curveto(124.99436477,215.2791034)(125.04936472,215.35910332)(125.09936806,215.4391037)
\curveto(125.33936443,215.7791029)(125.58436418,216.11410256)(125.83436806,216.4441037)
\curveto(126.08436368,216.7741019)(126.32436344,217.10910157)(126.55436806,217.4491037)
\curveto(126.71436305,217.66910101)(126.87436289,217.88410079)(127.03436806,218.0941037)
\curveto(127.19436257,218.30410037)(127.35436241,218.51910016)(127.51436806,218.7391037)
\curveto(127.87436189,219.25909942)(128.23936153,219.76909891)(128.60936806,220.2691037)
\curveto(128.97936079,220.76909791)(129.34936042,221.2790974)(129.71936806,221.7991037)
\curveto(129.85935991,221.99909668)(129.99935977,222.19409648)(130.13936806,222.3841037)
\curveto(130.28935948,222.5740961)(130.43435933,222.76909591)(130.57436806,222.9691037)
\curveto(130.78435898,223.26909541)(130.99935877,223.56909511)(131.21936806,223.8691037)
\lineto(131.87936806,224.7691037)
\lineto(132.05936806,225.0391037)
\lineto(132.26936806,225.3091037)
\lineto(132.38936806,225.4891037)
\curveto(132.43935733,225.54909313)(132.48935728,225.60409307)(132.53936806,225.6541037)
\curveto(132.60935716,225.70409297)(132.68435708,225.73909294)(132.76436806,225.7591037)
\curveto(132.78435698,225.76909291)(132.80935696,225.76909291)(132.83936806,225.7591037)
\curveto(132.87935689,225.75909292)(132.90935686,225.76909291)(132.92936806,225.7891037)
\curveto(133.04935672,225.78909289)(133.18435658,225.78409289)(133.33436806,225.7741037)
\curveto(133.48435628,225.7740929)(133.57435619,225.72909295)(133.60436806,225.6391037)
\curveto(133.62435614,225.60909307)(133.62935614,225.5740931)(133.61936806,225.5341037)
\curveto(133.60935616,225.49409318)(133.59435617,225.46409321)(133.57436806,225.4441037)
\curveto(133.53435623,225.36409331)(133.49435627,225.29409338)(133.45436806,225.2341037)
\curveto(133.41435635,225.1740935)(133.3693564,225.11409356)(133.31936806,225.0541037)
\lineto(132.74936806,224.2741037)
\curveto(132.5693572,224.02409465)(132.38935738,223.76909491)(132.20936806,223.5091037)
\moveto(125.35436806,219.6091037)
\curveto(125.30436446,219.62909905)(125.25436451,219.63409904)(125.20436806,219.6241037)
\curveto(125.15436461,219.61409906)(125.10436466,219.61909906)(125.05436806,219.6391037)
\curveto(124.94436482,219.65909902)(124.83936493,219.679099)(124.73936806,219.6991037)
\curveto(124.64936512,219.72909895)(124.55436521,219.76909891)(124.45436806,219.8191037)
\curveto(124.12436564,219.95909872)(123.8693659,220.15409852)(123.68936806,220.4041037)
\curveto(123.50936626,220.66409801)(123.3643664,220.9740977)(123.25436806,221.3341037)
\curveto(123.22436654,221.41409726)(123.20436656,221.49409718)(123.19436806,221.5741037)
\curveto(123.18436658,221.66409701)(123.1693666,221.74909693)(123.14936806,221.8291037)
\curveto(123.13936663,221.8790968)(123.13436663,221.94409673)(123.13436806,222.0241037)
\curveto(123.12436664,222.05409662)(123.11936665,222.08409659)(123.11936806,222.1141037)
\curveto(123.11936665,222.15409652)(123.11436665,222.18909649)(123.10436806,222.2191037)
\lineto(123.10436806,222.3691037)
\curveto(123.09436667,222.41909626)(123.08936668,222.4790962)(123.08936806,222.5491037)
\curveto(123.08936668,222.62909605)(123.09436667,222.69409598)(123.10436806,222.7441037)
\lineto(123.10436806,222.9091037)
\curveto(123.12436664,222.95909572)(123.12936664,223.00409567)(123.11936806,223.0441037)
\curveto(123.11936665,223.09409558)(123.12436664,223.13909554)(123.13436806,223.1791037)
\curveto(123.14436662,223.21909546)(123.14936662,223.25409542)(123.14936806,223.2841037)
\curveto(123.14936662,223.32409535)(123.15436661,223.36409531)(123.16436806,223.4041037)
\curveto(123.19436657,223.51409516)(123.21436655,223.62409505)(123.22436806,223.7341037)
\curveto(123.24436652,223.85409482)(123.27936649,223.96909471)(123.32936806,224.0791037)
\curveto(123.4693663,224.41909426)(123.62936614,224.69409398)(123.80936806,224.9041037)
\curveto(123.99936577,225.12409355)(124.2693655,225.30409337)(124.61936806,225.4441037)
\curveto(124.69936507,225.4740932)(124.78436498,225.49409318)(124.87436806,225.5041037)
\curveto(124.9643648,225.52409315)(125.05936471,225.54409313)(125.15936806,225.5641037)
\curveto(125.18936458,225.5740931)(125.24436452,225.5740931)(125.32436806,225.5641037)
\curveto(125.40436436,225.56409311)(125.45436431,225.5740931)(125.47436806,225.5941037)
\curveto(126.03436373,225.60409307)(126.48436328,225.49409318)(126.82436806,225.2641037)
\curveto(127.17436259,225.03409364)(127.43436233,224.72909395)(127.60436806,224.3491037)
\curveto(127.64436212,224.25909442)(127.67936209,224.16409451)(127.70936806,224.0641037)
\curveto(127.73936203,223.96409471)(127.764362,223.86409481)(127.78436806,223.7641037)
\curveto(127.80436196,223.73409494)(127.80936196,223.70409497)(127.79936806,223.6741037)
\curveto(127.79936197,223.64409503)(127.80436196,223.61409506)(127.81436806,223.5841037)
\curveto(127.84436192,223.4740952)(127.8643619,223.34909533)(127.87436806,223.2091037)
\curveto(127.88436188,223.0790956)(127.89436187,222.94409573)(127.90436806,222.8041037)
\lineto(127.90436806,222.6391037)
\curveto(127.91436185,222.5790961)(127.91436185,222.52409615)(127.90436806,222.4741037)
\curveto(127.89436187,222.42409625)(127.88936188,222.3740963)(127.88936806,222.3241037)
\lineto(127.88936806,222.1891037)
\curveto(127.87936189,222.14909653)(127.87436189,222.10909657)(127.87436806,222.0691037)
\curveto(127.88436188,222.02909665)(127.87936189,221.98409669)(127.85936806,221.9341037)
\curveto(127.83936193,221.82409685)(127.81936195,221.71909696)(127.79936806,221.6191037)
\curveto(127.78936198,221.51909716)(127.769362,221.41909726)(127.73936806,221.3191037)
\curveto(127.60936216,220.95909772)(127.44436232,220.64409803)(127.24436806,220.3741037)
\curveto(127.04436272,220.10409857)(126.769363,219.89909878)(126.41936806,219.7591037)
\curveto(126.33936343,219.72909895)(126.25436351,219.70409897)(126.16436806,219.6841037)
\lineto(125.89436806,219.6241037)
\curveto(125.84436392,219.61409906)(125.79936397,219.60909907)(125.75936806,219.6091037)
\curveto(125.71936405,219.61909906)(125.67936409,219.61909906)(125.63936806,219.6091037)
\curveto(125.53936423,219.58909909)(125.44436432,219.58909909)(125.35436806,219.6091037)
\moveto(124.51436806,221.0041037)
\curveto(124.55436521,220.93409774)(124.59436517,220.86909781)(124.63436806,220.8091037)
\curveto(124.67436509,220.75909792)(124.72436504,220.70909797)(124.78436806,220.6591037)
\lineto(124.93436806,220.5391037)
\curveto(124.99436477,220.50909817)(125.05936471,220.48409819)(125.12936806,220.4641037)
\curveto(125.1693646,220.44409823)(125.20436456,220.43409824)(125.23436806,220.4341037)
\curveto(125.27436449,220.44409823)(125.31436445,220.43909824)(125.35436806,220.4191037)
\curveto(125.38436438,220.41909826)(125.42436434,220.41409826)(125.47436806,220.4041037)
\curveto(125.52436424,220.40409827)(125.5643642,220.40909827)(125.59436806,220.4191037)
\lineto(125.81936806,220.4641037)
\curveto(126.0693637,220.54409813)(126.25436351,220.66909801)(126.37436806,220.8391037)
\curveto(126.45436331,220.93909774)(126.52436324,221.06909761)(126.58436806,221.2291037)
\curveto(126.6643631,221.40909727)(126.72436304,221.63409704)(126.76436806,221.9041037)
\curveto(126.80436296,222.18409649)(126.81936295,222.46409621)(126.80936806,222.7441037)
\curveto(126.79936297,223.03409564)(126.769363,223.30909537)(126.71936806,223.5691037)
\curveto(126.6693631,223.82909485)(126.59436317,224.03909464)(126.49436806,224.1991037)
\curveto(126.37436339,224.39909428)(126.22436354,224.54909413)(126.04436806,224.6491037)
\curveto(125.9643638,224.69909398)(125.87436389,224.72909395)(125.77436806,224.7391037)
\curveto(125.67436409,224.75909392)(125.5693642,224.76909391)(125.45936806,224.7691037)
\curveto(125.43936433,224.75909392)(125.41436435,224.75409392)(125.38436806,224.7541037)
\curveto(125.3643644,224.76409391)(125.34436442,224.76409391)(125.32436806,224.7541037)
\curveto(125.27436449,224.74409393)(125.22936454,224.73409394)(125.18936806,224.7241037)
\curveto(125.14936462,224.72409395)(125.10936466,224.71409396)(125.06936806,224.6941037)
\curveto(124.88936488,224.61409406)(124.73936503,224.49409418)(124.61936806,224.3341037)
\curveto(124.50936526,224.1740945)(124.41936535,223.99409468)(124.34936806,223.7941037)
\curveto(124.28936548,223.60409507)(124.24436552,223.3790953)(124.21436806,223.1191037)
\curveto(124.19436557,222.85909582)(124.18936558,222.59409608)(124.19936806,222.3241037)
\curveto(124.20936556,222.06409661)(124.23936553,221.81409686)(124.28936806,221.5741037)
\curveto(124.34936542,221.34409733)(124.42436534,221.15409752)(124.51436806,221.0041037)
\moveto(135.31436806,218.0191037)
\curveto(135.32435444,217.96910071)(135.32935444,217.8791008)(135.32936806,217.7491037)
\curveto(135.32935444,217.61910106)(135.31935445,217.52910115)(135.29936806,217.4791037)
\curveto(135.27935449,217.42910125)(135.27435449,217.3741013)(135.28436806,217.3141037)
\curveto(135.29435447,217.26410141)(135.29435447,217.21410146)(135.28436806,217.1641037)
\curveto(135.24435452,217.02410165)(135.21435455,216.88910179)(135.19436806,216.7591037)
\curveto(135.18435458,216.62910205)(135.15435461,216.50910217)(135.10436806,216.3991037)
\curveto(134.9643548,216.04910263)(134.79935497,215.75410292)(134.60936806,215.5141037)
\curveto(134.41935535,215.28410339)(134.14935562,215.09910358)(133.79936806,214.9591037)
\curveto(133.71935605,214.92910375)(133.63435613,214.90910377)(133.54436806,214.8991037)
\curveto(133.45435631,214.8791038)(133.3693564,214.85910382)(133.28936806,214.8391037)
\curveto(133.23935653,214.82910385)(133.18935658,214.82410385)(133.13936806,214.8241037)
\curveto(133.08935668,214.82410385)(133.03935673,214.81910386)(132.98936806,214.8091037)
\curveto(132.95935681,214.79910388)(132.90935686,214.79910388)(132.83936806,214.8091037)
\curveto(132.769357,214.80910387)(132.71935705,214.81410386)(132.68936806,214.8241037)
\curveto(132.62935714,214.84410383)(132.5693572,214.85410382)(132.50936806,214.8541037)
\curveto(132.45935731,214.84410383)(132.40935736,214.84910383)(132.35936806,214.8691037)
\curveto(132.2693575,214.88910379)(132.17935759,214.91410376)(132.08936806,214.9441037)
\curveto(132.00935776,214.96410371)(131.92935784,214.99410368)(131.84936806,215.0341037)
\curveto(131.52935824,215.1741035)(131.27935849,215.36910331)(131.09936806,215.6191037)
\curveto(130.91935885,215.8791028)(130.769359,216.18410249)(130.64936806,216.5341037)
\curveto(130.62935914,216.61410206)(130.61435915,216.69910198)(130.60436806,216.7891037)
\curveto(130.59435917,216.8791018)(130.57935919,216.96410171)(130.55936806,217.0441037)
\curveto(130.54935922,217.0741016)(130.54435922,217.10410157)(130.54436806,217.1341037)
\lineto(130.54436806,217.2391037)
\curveto(130.52435924,217.31910136)(130.51435925,217.39910128)(130.51436806,217.4791037)
\lineto(130.51436806,217.6141037)
\curveto(130.49435927,217.71410096)(130.49435927,217.81410086)(130.51436806,217.9141037)
\lineto(130.51436806,218.0941037)
\curveto(130.52435924,218.14410053)(130.52935924,218.18910049)(130.52936806,218.2291037)
\curveto(130.52935924,218.2791004)(130.53435923,218.32410035)(130.54436806,218.3641037)
\curveto(130.55435921,218.40410027)(130.55935921,218.43910024)(130.55936806,218.4691037)
\curveto(130.55935921,218.50910017)(130.5643592,218.54910013)(130.57436806,218.5891037)
\lineto(130.63436806,218.9191037)
\curveto(130.65435911,219.03909964)(130.68435908,219.14909953)(130.72436806,219.2491037)
\curveto(130.8643589,219.5790991)(131.02435874,219.85409882)(131.20436806,220.0741037)
\curveto(131.39435837,220.30409837)(131.65435811,220.48909819)(131.98436806,220.6291037)
\curveto(132.0643577,220.66909801)(132.14935762,220.69409798)(132.23936806,220.7041037)
\lineto(132.53936806,220.7641037)
\lineto(132.67436806,220.7641037)
\curveto(132.72435704,220.7740979)(132.77435699,220.7790979)(132.82436806,220.7791037)
\curveto(133.39435637,220.79909788)(133.85435591,220.69409798)(134.20436806,220.4641037)
\curveto(134.5643552,220.24409843)(134.82935494,219.94409873)(134.99936806,219.5641037)
\curveto(135.04935472,219.46409921)(135.08935468,219.36409931)(135.11936806,219.2641037)
\curveto(135.14935462,219.16409951)(135.17935459,219.05909962)(135.20936806,218.9491037)
\curveto(135.21935455,218.90909977)(135.22435454,218.8740998)(135.22436806,218.8441037)
\curveto(135.22435454,218.82409985)(135.22935454,218.79409988)(135.23936806,218.7541037)
\curveto(135.25935451,218.68409999)(135.2693545,218.60910007)(135.26936806,218.5291037)
\curveto(135.2693545,218.44910023)(135.27935449,218.36910031)(135.29936806,218.2891037)
\curveto(135.29935447,218.23910044)(135.29935447,218.19410048)(135.29936806,218.1541037)
\curveto(135.29935447,218.11410056)(135.30435446,218.06910061)(135.31436806,218.0191037)
\moveto(134.20436806,217.5841037)
\curveto(134.21435555,217.63410104)(134.21935555,217.70910097)(134.21936806,217.8091037)
\curveto(134.22935554,217.90910077)(134.22435554,217.98410069)(134.20436806,218.0341037)
\curveto(134.18435558,218.09410058)(134.17935559,218.14910053)(134.18936806,218.1991037)
\curveto(134.20935556,218.25910042)(134.20935556,218.31910036)(134.18936806,218.3791037)
\curveto(134.17935559,218.40910027)(134.17435559,218.44410023)(134.17436806,218.4841037)
\curveto(134.17435559,218.52410015)(134.1693556,218.56410011)(134.15936806,218.6041037)
\curveto(134.13935563,218.68409999)(134.11935565,218.75909992)(134.09936806,218.8291037)
\curveto(134.08935568,218.90909977)(134.07435569,218.98909969)(134.05436806,219.0691037)
\curveto(134.02435574,219.12909955)(133.99935577,219.18909949)(133.97936806,219.2491037)
\curveto(133.95935581,219.30909937)(133.92935584,219.36909931)(133.88936806,219.4291037)
\curveto(133.78935598,219.59909908)(133.65935611,219.73409894)(133.49936806,219.8341037)
\curveto(133.41935635,219.88409879)(133.32435644,219.91909876)(133.21436806,219.9391037)
\curveto(133.10435666,219.95909872)(132.97935679,219.96909871)(132.83936806,219.9691037)
\curveto(132.81935695,219.95909872)(132.79435697,219.95409872)(132.76436806,219.9541037)
\curveto(132.73435703,219.96409871)(132.70435706,219.96409871)(132.67436806,219.9541037)
\lineto(132.52436806,219.8941037)
\curveto(132.47435729,219.88409879)(132.42935734,219.86909881)(132.38936806,219.8491037)
\curveto(132.19935757,219.73909894)(132.05435771,219.59409908)(131.95436806,219.4141037)
\curveto(131.8643579,219.23409944)(131.78435798,219.02909965)(131.71436806,218.7991037)
\curveto(131.67435809,218.66910001)(131.65435811,218.53410014)(131.65436806,218.3941037)
\curveto(131.65435811,218.26410041)(131.64435812,218.11910056)(131.62436806,217.9591037)
\curveto(131.61435815,217.90910077)(131.60435816,217.84910083)(131.59436806,217.7791037)
\curveto(131.59435817,217.70910097)(131.60435816,217.64910103)(131.62436806,217.5991037)
\lineto(131.62436806,217.4341037)
\lineto(131.62436806,217.2541037)
\curveto(131.63435813,217.20410147)(131.64435812,217.14910153)(131.65436806,217.0891037)
\curveto(131.6643581,217.03910164)(131.6693581,216.98410169)(131.66936806,216.9241037)
\curveto(131.67935809,216.86410181)(131.69435807,216.80910187)(131.71436806,216.7591037)
\curveto(131.764358,216.56910211)(131.82435794,216.39410228)(131.89436806,216.2341037)
\curveto(131.9643578,216.0741026)(132.0693577,215.94410273)(132.20936806,215.8441037)
\curveto(132.33935743,215.74410293)(132.47935729,215.674103)(132.62936806,215.6341037)
\curveto(132.65935711,215.62410305)(132.68435708,215.61910306)(132.70436806,215.6191037)
\curveto(132.73435703,215.62910305)(132.764357,215.62910305)(132.79436806,215.6191037)
\curveto(132.81435695,215.61910306)(132.84435692,215.61410306)(132.88436806,215.6041037)
\curveto(132.92435684,215.60410307)(132.95935681,215.60910307)(132.98936806,215.6191037)
\curveto(133.02935674,215.62910305)(133.0693567,215.63410304)(133.10936806,215.6341037)
\curveto(133.14935662,215.63410304)(133.18935658,215.64410303)(133.22936806,215.6641037)
\curveto(133.4693563,215.74410293)(133.6643561,215.8791028)(133.81436806,216.0691037)
\curveto(133.93435583,216.24910243)(134.02435574,216.45410222)(134.08436806,216.6841037)
\curveto(134.10435566,216.75410192)(134.11935565,216.82410185)(134.12936806,216.8941037)
\curveto(134.13935563,216.9741017)(134.15435561,217.05410162)(134.17436806,217.1341037)
\curveto(134.17435559,217.19410148)(134.17935559,217.23910144)(134.18936806,217.2691037)
\curveto(134.18935558,217.28910139)(134.18935558,217.31410136)(134.18936806,217.3441037)
\curveto(134.18935558,217.38410129)(134.19435557,217.41410126)(134.20436806,217.4341037)
\lineto(134.20436806,217.5841037)
}
}
{
\newrgbcolor{curcolor}{0 0 0}
\pscustom[linestyle=none,fillstyle=solid,fillcolor=curcolor]
{
\newpath
\moveto(98.46214028,150.5941037)
\curveto(98.56213542,150.59409308)(98.65713533,150.58409309)(98.74714028,150.5641037)
\curveto(98.83713515,150.55409312)(98.90213508,150.52409315)(98.94214028,150.4741037)
\curveto(99.00213498,150.39409328)(99.03213495,150.28909339)(99.03214028,150.1591037)
\lineto(99.03214028,149.7691037)
\lineto(99.03214028,148.2691037)
\lineto(99.03214028,141.8791037)
\lineto(99.03214028,140.7091037)
\lineto(99.03214028,140.3941037)
\curveto(99.04213494,140.29410338)(99.02713496,140.21410346)(98.98714028,140.1541037)
\curveto(98.93713505,140.0741036)(98.86213512,140.02410365)(98.76214028,140.0041037)
\curveto(98.67213531,139.99410368)(98.56213542,139.98910369)(98.43214028,139.9891037)
\lineto(98.20714028,139.9891037)
\curveto(98.12713586,140.00910367)(98.05713593,140.02410365)(97.99714028,140.0341037)
\curveto(97.93713605,140.05410362)(97.8871361,140.09410358)(97.84714028,140.1541037)
\curveto(97.80713618,140.21410346)(97.7871362,140.28910339)(97.78714028,140.3791037)
\lineto(97.78714028,140.6791037)
\lineto(97.78714028,141.7741037)
\lineto(97.78714028,147.1141037)
\curveto(97.76713622,147.20409647)(97.75213623,147.2790964)(97.74214028,147.3391037)
\curveto(97.74213624,147.40909627)(97.71213627,147.46909621)(97.65214028,147.5191037)
\curveto(97.5821364,147.56909611)(97.49213649,147.59409608)(97.38214028,147.5941037)
\curveto(97.2821367,147.60409607)(97.17213681,147.60909607)(97.05214028,147.6091037)
\lineto(95.91214028,147.6091037)
\lineto(95.41714028,147.6091037)
\curveto(95.25713873,147.61909606)(95.14713884,147.679096)(95.08714028,147.7891037)
\curveto(95.06713892,147.81909586)(95.05713893,147.84909583)(95.05714028,147.8791037)
\curveto(95.05713893,147.91909576)(95.05213893,147.96409571)(95.04214028,148.0141037)
\curveto(95.02213896,148.13409554)(95.02713896,148.24409543)(95.05714028,148.3441037)
\curveto(95.09713889,148.44409523)(95.15213883,148.51409516)(95.22214028,148.5541037)
\curveto(95.30213868,148.60409507)(95.42213856,148.62909505)(95.58214028,148.6291037)
\curveto(95.74213824,148.62909505)(95.87713811,148.64409503)(95.98714028,148.6741037)
\curveto(96.03713795,148.68409499)(96.09213789,148.68909499)(96.15214028,148.6891037)
\curveto(96.21213777,148.69909498)(96.27213771,148.71409496)(96.33214028,148.7341037)
\curveto(96.4821375,148.78409489)(96.62713736,148.83409484)(96.76714028,148.8841037)
\curveto(96.90713708,148.94409473)(97.04213694,149.01409466)(97.17214028,149.0941037)
\curveto(97.31213667,149.18409449)(97.43213655,149.28909439)(97.53214028,149.4091037)
\curveto(97.63213635,149.52909415)(97.72713626,149.65909402)(97.81714028,149.7991037)
\curveto(97.87713611,149.89909378)(97.92213606,150.00909367)(97.95214028,150.1291037)
\curveto(97.99213599,150.24909343)(98.04213594,150.35409332)(98.10214028,150.4441037)
\curveto(98.15213583,150.50409317)(98.22213576,150.54409313)(98.31214028,150.5641037)
\curveto(98.33213565,150.5740931)(98.35713563,150.5790931)(98.38714028,150.5791037)
\curveto(98.41713557,150.5790931)(98.44213554,150.58409309)(98.46214028,150.5941037)
}
}
{
\newrgbcolor{curcolor}{0 0 0}
\pscustom[linestyle=none,fillstyle=solid,fillcolor=curcolor]
{
\newpath
\moveto(104.26174965,150.3991037)
\lineto(107.86174965,150.3991037)
\lineto(108.50674965,150.3991037)
\curveto(108.58674312,150.39909328)(108.66174305,150.39409328)(108.73174965,150.3841037)
\curveto(108.80174291,150.38409329)(108.86174285,150.3740933)(108.91174965,150.3541037)
\curveto(108.98174273,150.32409335)(109.03674267,150.26409341)(109.07674965,150.1741037)
\curveto(109.09674261,150.14409353)(109.1067426,150.10409357)(109.10674965,150.0541037)
\lineto(109.10674965,149.9191037)
\curveto(109.11674259,149.80909387)(109.1117426,149.70409397)(109.09174965,149.6041037)
\curveto(109.08174263,149.50409417)(109.04674266,149.43409424)(108.98674965,149.3941037)
\curveto(108.89674281,149.32409435)(108.76174295,149.28909439)(108.58174965,149.2891037)
\curveto(108.40174331,149.29909438)(108.23674347,149.30409437)(108.08674965,149.3041037)
\lineto(106.09174965,149.3041037)
\lineto(105.59674965,149.3041037)
\lineto(105.46174965,149.3041037)
\curveto(105.42174629,149.30409437)(105.38174633,149.29909438)(105.34174965,149.2891037)
\lineto(105.13174965,149.2891037)
\curveto(105.02174669,149.25909442)(104.94174677,149.21909446)(104.89174965,149.1691037)
\curveto(104.84174687,149.12909455)(104.8067469,149.0740946)(104.78674965,149.0041037)
\curveto(104.76674694,148.94409473)(104.75174696,148.8740948)(104.74174965,148.7941037)
\curveto(104.73174698,148.71409496)(104.711747,148.62409505)(104.68174965,148.5241037)
\curveto(104.63174708,148.32409535)(104.59174712,148.11909556)(104.56174965,147.9091037)
\curveto(104.53174718,147.69909598)(104.49174722,147.49409618)(104.44174965,147.2941037)
\curveto(104.42174729,147.22409645)(104.4117473,147.15409652)(104.41174965,147.0841037)
\curveto(104.4117473,147.02409665)(104.40174731,146.95909672)(104.38174965,146.8891037)
\curveto(104.37174734,146.85909682)(104.36174735,146.81909686)(104.35174965,146.7691037)
\curveto(104.35174736,146.72909695)(104.35674735,146.68909699)(104.36674965,146.6491037)
\curveto(104.38674732,146.59909708)(104.4117473,146.55409712)(104.44174965,146.5141037)
\curveto(104.48174723,146.48409719)(104.54174717,146.4790972)(104.62174965,146.4991037)
\curveto(104.68174703,146.51909716)(104.74174697,146.54409713)(104.80174965,146.5741037)
\curveto(104.86174685,146.61409706)(104.92174679,146.64909703)(104.98174965,146.6791037)
\curveto(105.04174667,146.69909698)(105.09174662,146.71409696)(105.13174965,146.7241037)
\curveto(105.32174639,146.80409687)(105.52674618,146.85909682)(105.74674965,146.8891037)
\curveto(105.97674573,146.91909676)(106.2067455,146.92909675)(106.43674965,146.9191037)
\curveto(106.67674503,146.91909676)(106.9067448,146.89409678)(107.12674965,146.8441037)
\curveto(107.34674436,146.80409687)(107.54674416,146.74409693)(107.72674965,146.6641037)
\curveto(107.77674393,146.64409703)(107.82174389,146.62409705)(107.86174965,146.6041037)
\curveto(107.9117438,146.58409709)(107.96174375,146.55909712)(108.01174965,146.5291037)
\curveto(108.36174335,146.31909736)(108.64174307,146.08909759)(108.85174965,145.8391037)
\curveto(109.07174264,145.58909809)(109.26674244,145.26409841)(109.43674965,144.8641037)
\curveto(109.48674222,144.75409892)(109.52174219,144.64409903)(109.54174965,144.5341037)
\curveto(109.56174215,144.42409925)(109.58674212,144.30909937)(109.61674965,144.1891037)
\curveto(109.62674208,144.15909952)(109.63174208,144.11409956)(109.63174965,144.0541037)
\curveto(109.65174206,143.99409968)(109.66174205,143.92409975)(109.66174965,143.8441037)
\curveto(109.66174205,143.7740999)(109.67174204,143.70909997)(109.69174965,143.6491037)
\lineto(109.69174965,143.4841037)
\curveto(109.70174201,143.43410024)(109.706742,143.36410031)(109.70674965,143.2741037)
\curveto(109.706742,143.18410049)(109.69674201,143.11410056)(109.67674965,143.0641037)
\curveto(109.65674205,143.00410067)(109.65174206,142.94410073)(109.66174965,142.8841037)
\curveto(109.67174204,142.83410084)(109.66674204,142.78410089)(109.64674965,142.7341037)
\curveto(109.6067421,142.5741011)(109.57174214,142.42410125)(109.54174965,142.2841037)
\curveto(109.5117422,142.14410153)(109.46674224,142.00910167)(109.40674965,141.8791037)
\curveto(109.24674246,141.50910217)(109.02674268,141.1741025)(108.74674965,140.8741037)
\curveto(108.46674324,140.5741031)(108.14674356,140.34410333)(107.78674965,140.1841037)
\curveto(107.61674409,140.10410357)(107.41674429,140.02910365)(107.18674965,139.9591037)
\curveto(107.07674463,139.91910376)(106.96174475,139.89410378)(106.84174965,139.8841037)
\curveto(106.72174499,139.8741038)(106.60174511,139.85410382)(106.48174965,139.8241037)
\curveto(106.43174528,139.80410387)(106.37674533,139.80410387)(106.31674965,139.8241037)
\curveto(106.25674545,139.83410384)(106.19674551,139.82910385)(106.13674965,139.8091037)
\curveto(106.03674567,139.78910389)(105.93674577,139.78910389)(105.83674965,139.8091037)
\lineto(105.70174965,139.8091037)
\curveto(105.65174606,139.82910385)(105.59174612,139.83910384)(105.52174965,139.8391037)
\curveto(105.46174625,139.82910385)(105.4067463,139.83410384)(105.35674965,139.8541037)
\curveto(105.31674639,139.86410381)(105.28174643,139.86910381)(105.25174965,139.8691037)
\curveto(105.22174649,139.86910381)(105.18674652,139.8741038)(105.14674965,139.8841037)
\lineto(104.87674965,139.9441037)
\curveto(104.78674692,139.96410371)(104.70174701,139.99410368)(104.62174965,140.0341037)
\curveto(104.28174743,140.1741035)(103.99174772,140.32910335)(103.75174965,140.4991037)
\curveto(103.5117482,140.679103)(103.29174842,140.90910277)(103.09174965,141.1891037)
\curveto(102.94174877,141.41910226)(102.82674888,141.65910202)(102.74674965,141.9091037)
\curveto(102.72674898,141.95910172)(102.71674899,142.00410167)(102.71674965,142.0441037)
\curveto(102.71674899,142.09410158)(102.706749,142.14410153)(102.68674965,142.1941037)
\curveto(102.66674904,142.25410142)(102.65174906,142.33410134)(102.64174965,142.4341037)
\curveto(102.64174907,142.53410114)(102.66174905,142.60910107)(102.70174965,142.6591037)
\curveto(102.75174896,142.73910094)(102.83174888,142.78410089)(102.94174965,142.7941037)
\curveto(103.05174866,142.80410087)(103.16674854,142.80910087)(103.28674965,142.8091037)
\lineto(103.45174965,142.8091037)
\curveto(103.5117482,142.80910087)(103.56674814,142.79910088)(103.61674965,142.7791037)
\curveto(103.706748,142.75910092)(103.77674793,142.71910096)(103.82674965,142.6591037)
\curveto(103.89674781,142.56910111)(103.94174777,142.45910122)(103.96174965,142.3291037)
\curveto(103.99174772,142.20910147)(104.03674767,142.10410157)(104.09674965,142.0141037)
\curveto(104.28674742,141.674102)(104.54674716,141.40410227)(104.87674965,141.2041037)
\curveto(104.97674673,141.14410253)(105.08174663,141.09410258)(105.19174965,141.0541037)
\curveto(105.3117464,141.02410265)(105.43174628,140.98910269)(105.55174965,140.9491037)
\curveto(105.72174599,140.89910278)(105.92674578,140.8791028)(106.16674965,140.8891037)
\curveto(106.41674529,140.90910277)(106.61674509,140.94410273)(106.76674965,140.9941037)
\curveto(107.13674457,141.11410256)(107.42674428,141.2741024)(107.63674965,141.4741037)
\curveto(107.85674385,141.68410199)(108.03674367,141.96410171)(108.17674965,142.3141037)
\curveto(108.22674348,142.41410126)(108.25674345,142.51910116)(108.26674965,142.6291037)
\curveto(108.28674342,142.73910094)(108.3117434,142.85410082)(108.34174965,142.9741037)
\lineto(108.34174965,143.0791037)
\curveto(108.35174336,143.11910056)(108.35674335,143.15910052)(108.35674965,143.1991037)
\curveto(108.36674334,143.22910045)(108.36674334,143.26410041)(108.35674965,143.3041037)
\lineto(108.35674965,143.4241037)
\curveto(108.35674335,143.68409999)(108.32674338,143.92909975)(108.26674965,144.1591037)
\curveto(108.15674355,144.50909917)(108.00174371,144.80409887)(107.80174965,145.0441037)
\curveto(107.60174411,145.29409838)(107.34174437,145.48909819)(107.02174965,145.6291037)
\lineto(106.84174965,145.6891037)
\curveto(106.79174492,145.70909797)(106.73174498,145.72909795)(106.66174965,145.7491037)
\curveto(106.6117451,145.76909791)(106.55174516,145.7790979)(106.48174965,145.7791037)
\curveto(106.42174529,145.78909789)(106.35674535,145.80409787)(106.28674965,145.8241037)
\lineto(106.13674965,145.8241037)
\curveto(106.09674561,145.84409783)(106.04174567,145.85409782)(105.97174965,145.8541037)
\curveto(105.9117458,145.85409782)(105.85674585,145.84409783)(105.80674965,145.8241037)
\lineto(105.70174965,145.8241037)
\curveto(105.67174604,145.82409785)(105.63674607,145.81909786)(105.59674965,145.8091037)
\lineto(105.35674965,145.7491037)
\curveto(105.27674643,145.73909794)(105.19674651,145.71909796)(105.11674965,145.6891037)
\curveto(104.87674683,145.58909809)(104.64674706,145.45409822)(104.42674965,145.2841037)
\curveto(104.33674737,145.21409846)(104.25174746,145.13909854)(104.17174965,145.0591037)
\curveto(104.09174762,144.98909869)(103.99174772,144.93409874)(103.87174965,144.8941037)
\curveto(103.78174793,144.86409881)(103.64174807,144.85409882)(103.45174965,144.8641037)
\curveto(103.27174844,144.8740988)(103.15174856,144.89909878)(103.09174965,144.9391037)
\curveto(103.04174867,144.9790987)(103.00174871,145.03909864)(102.97174965,145.1191037)
\curveto(102.95174876,145.19909848)(102.95174876,145.28409839)(102.97174965,145.3741037)
\curveto(103.00174871,145.49409818)(103.02174869,145.61409806)(103.03174965,145.7341037)
\curveto(103.05174866,145.86409781)(103.07674863,145.98909769)(103.10674965,146.1091037)
\curveto(103.12674858,146.14909753)(103.13174858,146.18409749)(103.12174965,146.2141037)
\curveto(103.12174859,146.25409742)(103.13174858,146.29909738)(103.15174965,146.3491037)
\curveto(103.17174854,146.43909724)(103.18674852,146.52909715)(103.19674965,146.6191037)
\curveto(103.2067485,146.71909696)(103.22674848,146.81409686)(103.25674965,146.9041037)
\curveto(103.26674844,146.96409671)(103.27174844,147.02409665)(103.27174965,147.0841037)
\curveto(103.28174843,147.14409653)(103.29674841,147.20409647)(103.31674965,147.2641037)
\curveto(103.36674834,147.46409621)(103.40174831,147.66909601)(103.42174965,147.8791037)
\curveto(103.45174826,148.09909558)(103.49174822,148.30909537)(103.54174965,148.5091037)
\curveto(103.57174814,148.60909507)(103.59174812,148.70909497)(103.60174965,148.8091037)
\curveto(103.6117481,148.90909477)(103.62674808,149.00909467)(103.64674965,149.1091037)
\curveto(103.65674805,149.13909454)(103.66174805,149.1790945)(103.66174965,149.2291037)
\curveto(103.69174802,149.33909434)(103.711748,149.44409423)(103.72174965,149.5441037)
\curveto(103.74174797,149.65409402)(103.76674794,149.76409391)(103.79674965,149.8741037)
\curveto(103.81674789,149.95409372)(103.83174788,150.02409365)(103.84174965,150.0841037)
\curveto(103.85174786,150.15409352)(103.87674783,150.21409346)(103.91674965,150.2641037)
\curveto(103.93674777,150.29409338)(103.96674774,150.31409336)(104.00674965,150.3241037)
\curveto(104.04674766,150.34409333)(104.09174762,150.36409331)(104.14174965,150.3841037)
\curveto(104.20174751,150.38409329)(104.24174747,150.38909329)(104.26174965,150.3991037)
}
}
{
\newrgbcolor{curcolor}{0 0 0}
\pscustom[linestyle=none,fillstyle=solid,fillcolor=curcolor]
{
\newpath
\moveto(112.05635903,141.6241037)
\lineto(112.35635903,141.6241037)
\curveto(112.46635697,141.63410204)(112.57135686,141.63410204)(112.67135903,141.6241037)
\curveto(112.78135665,141.62410205)(112.88135655,141.61410206)(112.97135903,141.5941037)
\curveto(113.06135637,141.58410209)(113.1313563,141.55910212)(113.18135903,141.5191037)
\curveto(113.20135623,141.49910218)(113.21635622,141.46910221)(113.22635903,141.4291037)
\curveto(113.24635619,141.38910229)(113.26635617,141.34410233)(113.28635903,141.2941037)
\lineto(113.28635903,141.2191037)
\curveto(113.29635614,141.16910251)(113.29635614,141.11410256)(113.28635903,141.0541037)
\lineto(113.28635903,140.9041037)
\lineto(113.28635903,140.4241037)
\curveto(113.28635615,140.25410342)(113.24635619,140.13410354)(113.16635903,140.0641037)
\curveto(113.09635634,140.01410366)(113.00635643,139.98910369)(112.89635903,139.9891037)
\lineto(112.56635903,139.9891037)
\lineto(112.11635903,139.9891037)
\curveto(111.96635747,139.98910369)(111.85135758,140.01910366)(111.77135903,140.0791037)
\curveto(111.7313577,140.10910357)(111.70135773,140.15910352)(111.68135903,140.2291037)
\curveto(111.66135777,140.30910337)(111.64635779,140.39410328)(111.63635903,140.4841037)
\lineto(111.63635903,140.7691037)
\curveto(111.64635779,140.86910281)(111.65135778,140.95410272)(111.65135903,141.0241037)
\lineto(111.65135903,141.2191037)
\curveto(111.65135778,141.2791024)(111.66135777,141.33410234)(111.68135903,141.3841037)
\curveto(111.72135771,141.49410218)(111.79135764,141.56410211)(111.89135903,141.5941037)
\curveto(111.92135751,141.59410208)(111.97635746,141.60410207)(112.05635903,141.6241037)
}
}
{
\newrgbcolor{curcolor}{0 0 0}
\pscustom[linestyle=none,fillstyle=solid,fillcolor=curcolor]
{
\newpath
\moveto(122.27651528,143.0641037)
\curveto(122.28650756,143.02410065)(122.28650756,142.9741007)(122.27651528,142.9141037)
\curveto(122.27650757,142.85410082)(122.27150757,142.80410087)(122.26151528,142.7641037)
\curveto(122.26150758,142.72410095)(122.25650759,142.68410099)(122.24651528,142.6441037)
\lineto(122.24651528,142.5391037)
\curveto(122.22650762,142.45910122)(122.21150763,142.3791013)(122.20151528,142.2991037)
\curveto(122.19150765,142.21910146)(122.17150767,142.14410153)(122.14151528,142.0741037)
\curveto(122.12150772,141.99410168)(122.10150774,141.91910176)(122.08151528,141.8491037)
\curveto(122.06150778,141.7791019)(122.03150781,141.70410197)(121.99151528,141.6241037)
\curveto(121.81150803,141.20410247)(121.55650829,140.86410281)(121.22651528,140.6041037)
\curveto(120.89650895,140.34410333)(120.50650934,140.13910354)(120.05651528,139.9891037)
\curveto(119.93650991,139.94910373)(119.81151003,139.92410375)(119.68151528,139.9141037)
\curveto(119.56151028,139.89410378)(119.43651041,139.86910381)(119.30651528,139.8391037)
\curveto(119.2465106,139.82910385)(119.18151066,139.82410385)(119.11151528,139.8241037)
\curveto(119.05151079,139.82410385)(118.98651086,139.81910386)(118.91651528,139.8091037)
\lineto(118.79651528,139.8091037)
\lineto(118.60151528,139.8091037)
\curveto(118.5415113,139.79910388)(118.48651136,139.80410387)(118.43651528,139.8241037)
\curveto(118.36651148,139.84410383)(118.30151154,139.84910383)(118.24151528,139.8391037)
\curveto(118.18151166,139.82910385)(118.12151172,139.83410384)(118.06151528,139.8541037)
\curveto(118.01151183,139.86410381)(117.96651188,139.86910381)(117.92651528,139.8691037)
\curveto(117.88651196,139.86910381)(117.841512,139.8791038)(117.79151528,139.8991037)
\curveto(117.71151213,139.91910376)(117.63651221,139.93910374)(117.56651528,139.9591037)
\curveto(117.49651235,139.96910371)(117.42651242,139.98410369)(117.35651528,140.0041037)
\curveto(116.87651297,140.1741035)(116.47651337,140.38410329)(116.15651528,140.6341037)
\curveto(115.846514,140.89410278)(115.59651425,141.24910243)(115.40651528,141.6991037)
\curveto(115.37651447,141.75910192)(115.35151449,141.81910186)(115.33151528,141.8791037)
\curveto(115.32151452,141.94910173)(115.30651454,142.02410165)(115.28651528,142.1041037)
\curveto(115.26651458,142.16410151)(115.25151459,142.22910145)(115.24151528,142.2991037)
\curveto(115.23151461,142.36910131)(115.21651463,142.43910124)(115.19651528,142.5091037)
\curveto(115.18651466,142.55910112)(115.18151466,142.59910108)(115.18151528,142.6291037)
\lineto(115.18151528,142.7491037)
\curveto(115.17151467,142.78910089)(115.16151468,142.83910084)(115.15151528,142.8991037)
\curveto(115.15151469,142.95910072)(115.15651469,143.00910067)(115.16651528,143.0491037)
\lineto(115.16651528,143.1841037)
\curveto(115.17651467,143.23410044)(115.18151466,143.28410039)(115.18151528,143.3341037)
\curveto(115.20151464,143.43410024)(115.21651463,143.52910015)(115.22651528,143.6191037)
\curveto(115.23651461,143.71909996)(115.25651459,143.81409986)(115.28651528,143.9041037)
\curveto(115.33651451,144.05409962)(115.39151445,144.19409948)(115.45151528,144.3241037)
\curveto(115.51151433,144.45409922)(115.58151426,144.5740991)(115.66151528,144.6841037)
\curveto(115.69151415,144.73409894)(115.72151412,144.7740989)(115.75151528,144.8041037)
\curveto(115.79151405,144.83409884)(115.82651402,144.86909881)(115.85651528,144.9091037)
\curveto(115.91651393,144.98909869)(115.98651386,145.05909862)(116.06651528,145.1191037)
\curveto(116.12651372,145.16909851)(116.18651366,145.21409846)(116.24651528,145.2541037)
\lineto(116.45651528,145.4041037)
\curveto(116.50651334,145.44409823)(116.55651329,145.4790982)(116.60651528,145.5091037)
\curveto(116.65651319,145.54909813)(116.69151315,145.60409807)(116.71151528,145.6741037)
\curveto(116.71151313,145.70409797)(116.70151314,145.72909795)(116.68151528,145.7491037)
\curveto(116.67151317,145.7790979)(116.66151318,145.80409787)(116.65151528,145.8241037)
\curveto(116.61151323,145.8740978)(116.56151328,145.91909776)(116.50151528,145.9591037)
\curveto(116.45151339,146.00909767)(116.40151344,146.05409762)(116.35151528,146.0941037)
\curveto(116.31151353,146.12409755)(116.26151358,146.1790975)(116.20151528,146.2591037)
\curveto(116.18151366,146.28909739)(116.15151369,146.31409736)(116.11151528,146.3341037)
\curveto(116.08151376,146.36409731)(116.05651379,146.39909728)(116.03651528,146.4391037)
\curveto(115.86651398,146.64909703)(115.73651411,146.89409678)(115.64651528,147.1741037)
\curveto(115.62651422,147.25409642)(115.61151423,147.33409634)(115.60151528,147.4141037)
\curveto(115.59151425,147.49409618)(115.57651427,147.5740961)(115.55651528,147.6541037)
\curveto(115.53651431,147.70409597)(115.52651432,147.76909591)(115.52651528,147.8491037)
\curveto(115.52651432,147.93909574)(115.53651431,148.00909567)(115.55651528,148.0591037)
\curveto(115.55651429,148.15909552)(115.56151428,148.22909545)(115.57151528,148.2691037)
\curveto(115.59151425,148.34909533)(115.60651424,148.41909526)(115.61651528,148.4791037)
\curveto(115.62651422,148.54909513)(115.6415142,148.61909506)(115.66151528,148.6891037)
\curveto(115.81151403,149.11909456)(116.02651382,149.46409421)(116.30651528,149.7241037)
\curveto(116.59651325,149.98409369)(116.9465129,150.19909348)(117.35651528,150.3691037)
\curveto(117.46651238,150.41909326)(117.58151226,150.44909323)(117.70151528,150.4591037)
\curveto(117.83151201,150.4790932)(117.96151188,150.50909317)(118.09151528,150.5491037)
\curveto(118.17151167,150.54909313)(118.2415116,150.54909313)(118.30151528,150.5491037)
\curveto(118.37151147,150.55909312)(118.4465114,150.56909311)(118.52651528,150.5791037)
\curveto(119.31651053,150.59909308)(119.97150987,150.46909321)(120.49151528,150.1891037)
\curveto(121.02150882,149.90909377)(121.40150844,149.49909418)(121.63151528,148.9591037)
\curveto(121.7415081,148.72909495)(121.81150803,148.44409523)(121.84151528,148.1041037)
\curveto(121.88150796,147.7740959)(121.85150799,147.46909621)(121.75151528,147.1891037)
\curveto(121.71150813,147.05909662)(121.66150818,146.93909674)(121.60151528,146.8291037)
\curveto(121.55150829,146.71909696)(121.49150835,146.61409706)(121.42151528,146.5141037)
\curveto(121.40150844,146.4740972)(121.37150847,146.43909724)(121.33151528,146.4091037)
\lineto(121.24151528,146.3191037)
\curveto(121.19150865,146.22909745)(121.13150871,146.16409751)(121.06151528,146.1241037)
\curveto(121.01150883,146.0740976)(120.95650889,146.02409765)(120.89651528,145.9741037)
\curveto(120.846509,145.93409774)(120.80150904,145.88909779)(120.76151528,145.8391037)
\curveto(120.7415091,145.81909786)(120.72150912,145.79409788)(120.70151528,145.7641037)
\curveto(120.69150915,145.74409793)(120.69150915,145.71909796)(120.70151528,145.6891037)
\curveto(120.71150913,145.63909804)(120.7415091,145.58909809)(120.79151528,145.5391037)
\curveto(120.841509,145.49909818)(120.89650895,145.45909822)(120.95651528,145.4191037)
\lineto(121.13651528,145.2991037)
\curveto(121.19650865,145.26909841)(121.2465086,145.23909844)(121.28651528,145.2091037)
\curveto(121.61650823,144.96909871)(121.86650798,144.65909902)(122.03651528,144.2791037)
\curveto(122.07650777,144.19909948)(122.10650774,144.11409956)(122.12651528,144.0241037)
\curveto(122.15650769,143.93409974)(122.18150766,143.84409983)(122.20151528,143.7541037)
\curveto(122.21150763,143.70409997)(122.22150762,143.64910003)(122.23151528,143.5891037)
\lineto(122.26151528,143.4391037)
\curveto(122.27150757,143.3791003)(122.27150757,143.31410036)(122.26151528,143.2441037)
\curveto(122.25150759,143.18410049)(122.25650759,143.12410055)(122.27651528,143.0641037)
\moveto(116.89151528,148.1041037)
\curveto(116.86151298,147.99409568)(116.85651299,147.85409582)(116.87651528,147.6841037)
\curveto(116.89651295,147.52409615)(116.92151292,147.39909628)(116.95151528,147.3091037)
\curveto(117.06151278,146.98909669)(117.21151263,146.74409693)(117.40151528,146.5741037)
\curveto(117.59151225,146.41409726)(117.85651199,146.28409739)(118.19651528,146.1841037)
\curveto(118.32651152,146.15409752)(118.49151135,146.12909755)(118.69151528,146.1091037)
\curveto(118.89151095,146.09909758)(119.06151078,146.11409756)(119.20151528,146.1541037)
\curveto(119.49151035,146.23409744)(119.73151011,146.34409733)(119.92151528,146.4841037)
\curveto(120.12150972,146.63409704)(120.27650957,146.83409684)(120.38651528,147.0841037)
\curveto(120.40650944,147.13409654)(120.41650943,147.1790965)(120.41651528,147.2191037)
\curveto(120.42650942,147.25909642)(120.4415094,147.30409637)(120.46151528,147.3541037)
\curveto(120.49150935,147.46409621)(120.51150933,147.60409607)(120.52151528,147.7741037)
\curveto(120.53150931,147.94409573)(120.52150932,148.08909559)(120.49151528,148.2091037)
\curveto(120.47150937,148.29909538)(120.4465094,148.38409529)(120.41651528,148.4641037)
\curveto(120.39650945,148.54409513)(120.36150948,148.62409505)(120.31151528,148.7041037)
\curveto(120.1415097,148.9740947)(119.91650993,149.16909451)(119.63651528,149.2891037)
\curveto(119.36651048,149.40909427)(119.00651084,149.46909421)(118.55651528,149.4691037)
\curveto(118.53651131,149.44909423)(118.50651134,149.44409423)(118.46651528,149.4541037)
\curveto(118.42651142,149.46409421)(118.39151145,149.46409421)(118.36151528,149.4541037)
\curveto(118.31151153,149.43409424)(118.25651159,149.41909426)(118.19651528,149.4091037)
\curveto(118.1465117,149.40909427)(118.09651175,149.39909428)(118.04651528,149.3791037)
\curveto(117.80651204,149.28909439)(117.59651225,149.1740945)(117.41651528,149.0341037)
\curveto(117.23651261,148.90409477)(117.09651275,148.72409495)(116.99651528,148.4941037)
\curveto(116.97651287,148.43409524)(116.95651289,148.36909531)(116.93651528,148.2991037)
\curveto(116.92651292,148.23909544)(116.91151293,148.1740955)(116.89151528,148.1041037)
\moveto(120.91151528,142.5691037)
\curveto(120.96150888,142.75910092)(120.96650888,142.96410071)(120.92651528,143.1841037)
\curveto(120.89650895,143.40410027)(120.85150899,143.58410009)(120.79151528,143.7241037)
\curveto(120.62150922,144.09409958)(120.36150948,144.39909928)(120.01151528,144.6391037)
\curveto(119.67151017,144.8790988)(119.23651061,144.99909868)(118.70651528,144.9991037)
\curveto(118.67651117,144.9790987)(118.63651121,144.9740987)(118.58651528,144.9841037)
\curveto(118.53651131,145.00409867)(118.49651135,145.00909867)(118.46651528,144.9991037)
\lineto(118.19651528,144.9391037)
\curveto(118.11651173,144.92909875)(118.03651181,144.91409876)(117.95651528,144.8941037)
\curveto(117.65651219,144.78409889)(117.39151245,144.63909904)(117.16151528,144.4591037)
\curveto(116.9415129,144.2790994)(116.77151307,144.04909963)(116.65151528,143.7691037)
\curveto(116.62151322,143.68909999)(116.59651325,143.60910007)(116.57651528,143.5291037)
\curveto(116.55651329,143.44910023)(116.53651331,143.36410031)(116.51651528,143.2741037)
\curveto(116.48651336,143.15410052)(116.47651337,143.00410067)(116.48651528,142.8241037)
\curveto(116.50651334,142.64410103)(116.53151331,142.50410117)(116.56151528,142.4041037)
\curveto(116.58151326,142.35410132)(116.59151325,142.30910137)(116.59151528,142.2691037)
\curveto(116.60151324,142.23910144)(116.61651323,142.19910148)(116.63651528,142.1491037)
\curveto(116.73651311,141.92910175)(116.86651298,141.72910195)(117.02651528,141.5491037)
\curveto(117.19651265,141.36910231)(117.39151245,141.23410244)(117.61151528,141.1441037)
\curveto(117.68151216,141.10410257)(117.77651207,141.06910261)(117.89651528,141.0391037)
\curveto(118.11651173,140.94910273)(118.37151147,140.90410277)(118.66151528,140.9041037)
\lineto(118.94651528,140.9041037)
\curveto(119.0465108,140.92410275)(119.1415107,140.93910274)(119.23151528,140.9491037)
\curveto(119.32151052,140.95910272)(119.41151043,140.9791027)(119.50151528,141.0091037)
\curveto(119.76151008,141.08910259)(120.00150984,141.21910246)(120.22151528,141.3991037)
\curveto(120.45150939,141.58910209)(120.62150922,141.80410187)(120.73151528,142.0441037)
\curveto(120.77150907,142.12410155)(120.80150904,142.20410147)(120.82151528,142.2841037)
\curveto(120.85150899,142.3741013)(120.88150896,142.46910121)(120.91151528,142.5691037)
}
}
{
\newrgbcolor{curcolor}{0 0 0}
\pscustom[linestyle=none,fillstyle=solid,fillcolor=curcolor]
{
\newpath
\moveto(133.41612465,148.5091037)
\curveto(133.21611435,148.21909546)(133.00611456,147.93409574)(132.78612465,147.6541037)
\curveto(132.57611499,147.3740963)(132.3711152,147.08909659)(132.17112465,146.7991037)
\curveto(131.571116,145.94909773)(130.9661166,145.10909857)(130.35612465,144.2791037)
\curveto(129.74611782,143.45910022)(129.14111843,142.62410105)(128.54112465,141.7741037)
\lineto(128.03112465,141.0541037)
\lineto(127.52112465,140.3641037)
\curveto(127.44112013,140.25410342)(127.36112021,140.13910354)(127.28112465,140.0191037)
\curveto(127.20112037,139.89910378)(127.10612046,139.80410387)(126.99612465,139.7341037)
\curveto(126.95612061,139.71410396)(126.89112068,139.69910398)(126.80112465,139.6891037)
\curveto(126.72112085,139.66910401)(126.63112094,139.65910402)(126.53112465,139.6591037)
\curveto(126.43112114,139.65910402)(126.33612123,139.66410401)(126.24612465,139.6741037)
\curveto(126.1661214,139.68410399)(126.10612146,139.70410397)(126.06612465,139.7341037)
\curveto(126.03612153,139.75410392)(126.01112156,139.78910389)(125.99112465,139.8391037)
\curveto(125.98112159,139.8791038)(125.98612158,139.92410375)(126.00612465,139.9741037)
\curveto(126.04612152,140.05410362)(126.09112148,140.12910355)(126.14112465,140.1991037)
\curveto(126.20112137,140.2791034)(126.25612131,140.35910332)(126.30612465,140.4391037)
\curveto(126.54612102,140.7791029)(126.79112078,141.11410256)(127.04112465,141.4441037)
\curveto(127.29112028,141.7741019)(127.53112004,142.10910157)(127.76112465,142.4491037)
\curveto(127.92111965,142.66910101)(128.08111949,142.88410079)(128.24112465,143.0941037)
\curveto(128.40111917,143.30410037)(128.56111901,143.51910016)(128.72112465,143.7391037)
\curveto(129.08111849,144.25909942)(129.44611812,144.76909891)(129.81612465,145.2691037)
\curveto(130.18611738,145.76909791)(130.55611701,146.2790974)(130.92612465,146.7991037)
\curveto(131.0661165,146.99909668)(131.20611636,147.19409648)(131.34612465,147.3841037)
\curveto(131.49611607,147.5740961)(131.64111593,147.76909591)(131.78112465,147.9691037)
\curveto(131.99111558,148.26909541)(132.20611536,148.56909511)(132.42612465,148.8691037)
\lineto(133.08612465,149.7691037)
\lineto(133.26612465,150.0391037)
\lineto(133.47612465,150.3091037)
\lineto(133.59612465,150.4891037)
\curveto(133.64611392,150.54909313)(133.69611387,150.60409307)(133.74612465,150.6541037)
\curveto(133.81611375,150.70409297)(133.89111368,150.73909294)(133.97112465,150.7591037)
\curveto(133.99111358,150.76909291)(134.01611355,150.76909291)(134.04612465,150.7591037)
\curveto(134.08611348,150.75909292)(134.11611345,150.76909291)(134.13612465,150.7891037)
\curveto(134.25611331,150.78909289)(134.39111318,150.78409289)(134.54112465,150.7741037)
\curveto(134.69111288,150.7740929)(134.78111279,150.72909295)(134.81112465,150.6391037)
\curveto(134.83111274,150.60909307)(134.83611273,150.5740931)(134.82612465,150.5341037)
\curveto(134.81611275,150.49409318)(134.80111277,150.46409321)(134.78112465,150.4441037)
\curveto(134.74111283,150.36409331)(134.70111287,150.29409338)(134.66112465,150.2341037)
\curveto(134.62111295,150.1740935)(134.57611299,150.11409356)(134.52612465,150.0541037)
\lineto(133.95612465,149.2741037)
\curveto(133.77611379,149.02409465)(133.59611397,148.76909491)(133.41612465,148.5091037)
\moveto(126.56112465,144.6091037)
\curveto(126.51112106,144.62909905)(126.46112111,144.63409904)(126.41112465,144.6241037)
\curveto(126.36112121,144.61409906)(126.31112126,144.61909906)(126.26112465,144.6391037)
\curveto(126.15112142,144.65909902)(126.04612152,144.679099)(125.94612465,144.6991037)
\curveto(125.85612171,144.72909895)(125.76112181,144.76909891)(125.66112465,144.8191037)
\curveto(125.33112224,144.95909872)(125.07612249,145.15409852)(124.89612465,145.4041037)
\curveto(124.71612285,145.66409801)(124.571123,145.9740977)(124.46112465,146.3341037)
\curveto(124.43112314,146.41409726)(124.41112316,146.49409718)(124.40112465,146.5741037)
\curveto(124.39112318,146.66409701)(124.37612319,146.74909693)(124.35612465,146.8291037)
\curveto(124.34612322,146.8790968)(124.34112323,146.94409673)(124.34112465,147.0241037)
\curveto(124.33112324,147.05409662)(124.32612324,147.08409659)(124.32612465,147.1141037)
\curveto(124.32612324,147.15409652)(124.32112325,147.18909649)(124.31112465,147.2191037)
\lineto(124.31112465,147.3691037)
\curveto(124.30112327,147.41909626)(124.29612327,147.4790962)(124.29612465,147.5491037)
\curveto(124.29612327,147.62909605)(124.30112327,147.69409598)(124.31112465,147.7441037)
\lineto(124.31112465,147.9091037)
\curveto(124.33112324,147.95909572)(124.33612323,148.00409567)(124.32612465,148.0441037)
\curveto(124.32612324,148.09409558)(124.33112324,148.13909554)(124.34112465,148.1791037)
\curveto(124.35112322,148.21909546)(124.35612321,148.25409542)(124.35612465,148.2841037)
\curveto(124.35612321,148.32409535)(124.36112321,148.36409531)(124.37112465,148.4041037)
\curveto(124.40112317,148.51409516)(124.42112315,148.62409505)(124.43112465,148.7341037)
\curveto(124.45112312,148.85409482)(124.48612308,148.96909471)(124.53612465,149.0791037)
\curveto(124.67612289,149.41909426)(124.83612273,149.69409398)(125.01612465,149.9041037)
\curveto(125.20612236,150.12409355)(125.47612209,150.30409337)(125.82612465,150.4441037)
\curveto(125.90612166,150.4740932)(125.99112158,150.49409318)(126.08112465,150.5041037)
\curveto(126.1711214,150.52409315)(126.2661213,150.54409313)(126.36612465,150.5641037)
\curveto(126.39612117,150.5740931)(126.45112112,150.5740931)(126.53112465,150.5641037)
\curveto(126.61112096,150.56409311)(126.66112091,150.5740931)(126.68112465,150.5941037)
\curveto(127.24112033,150.60409307)(127.69111988,150.49409318)(128.03112465,150.2641037)
\curveto(128.38111919,150.03409364)(128.64111893,149.72909395)(128.81112465,149.3491037)
\curveto(128.85111872,149.25909442)(128.88611868,149.16409451)(128.91612465,149.0641037)
\curveto(128.94611862,148.96409471)(128.9711186,148.86409481)(128.99112465,148.7641037)
\curveto(129.01111856,148.73409494)(129.01611855,148.70409497)(129.00612465,148.6741037)
\curveto(129.00611856,148.64409503)(129.01111856,148.61409506)(129.02112465,148.5841037)
\curveto(129.05111852,148.4740952)(129.0711185,148.34909533)(129.08112465,148.2091037)
\curveto(129.09111848,148.0790956)(129.10111847,147.94409573)(129.11112465,147.8041037)
\lineto(129.11112465,147.6391037)
\curveto(129.12111845,147.5790961)(129.12111845,147.52409615)(129.11112465,147.4741037)
\curveto(129.10111847,147.42409625)(129.09611847,147.3740963)(129.09612465,147.3241037)
\lineto(129.09612465,147.1891037)
\curveto(129.08611848,147.14909653)(129.08111849,147.10909657)(129.08112465,147.0691037)
\curveto(129.09111848,147.02909665)(129.08611848,146.98409669)(129.06612465,146.9341037)
\curveto(129.04611852,146.82409685)(129.02611854,146.71909696)(129.00612465,146.6191037)
\curveto(128.99611857,146.51909716)(128.97611859,146.41909726)(128.94612465,146.3191037)
\curveto(128.81611875,145.95909772)(128.65111892,145.64409803)(128.45112465,145.3741037)
\curveto(128.25111932,145.10409857)(127.97611959,144.89909878)(127.62612465,144.7591037)
\curveto(127.54612002,144.72909895)(127.46112011,144.70409897)(127.37112465,144.6841037)
\lineto(127.10112465,144.6241037)
\curveto(127.05112052,144.61409906)(127.00612056,144.60909907)(126.96612465,144.6091037)
\curveto(126.92612064,144.61909906)(126.88612068,144.61909906)(126.84612465,144.6091037)
\curveto(126.74612082,144.58909909)(126.65112092,144.58909909)(126.56112465,144.6091037)
\moveto(125.72112465,146.0041037)
\curveto(125.76112181,145.93409774)(125.80112177,145.86909781)(125.84112465,145.8091037)
\curveto(125.88112169,145.75909792)(125.93112164,145.70909797)(125.99112465,145.6591037)
\lineto(126.14112465,145.5391037)
\curveto(126.20112137,145.50909817)(126.2661213,145.48409819)(126.33612465,145.4641037)
\curveto(126.37612119,145.44409823)(126.41112116,145.43409824)(126.44112465,145.4341037)
\curveto(126.48112109,145.44409823)(126.52112105,145.43909824)(126.56112465,145.4191037)
\curveto(126.59112098,145.41909826)(126.63112094,145.41409826)(126.68112465,145.4041037)
\curveto(126.73112084,145.40409827)(126.7711208,145.40909827)(126.80112465,145.4191037)
\lineto(127.02612465,145.4641037)
\curveto(127.27612029,145.54409813)(127.46112011,145.66909801)(127.58112465,145.8391037)
\curveto(127.66111991,145.93909774)(127.73111984,146.06909761)(127.79112465,146.2291037)
\curveto(127.8711197,146.40909727)(127.93111964,146.63409704)(127.97112465,146.9041037)
\curveto(128.01111956,147.18409649)(128.02611954,147.46409621)(128.01612465,147.7441037)
\curveto(128.00611956,148.03409564)(127.97611959,148.30909537)(127.92612465,148.5691037)
\curveto(127.87611969,148.82909485)(127.80111977,149.03909464)(127.70112465,149.1991037)
\curveto(127.58111999,149.39909428)(127.43112014,149.54909413)(127.25112465,149.6491037)
\curveto(127.1711204,149.69909398)(127.08112049,149.72909395)(126.98112465,149.7391037)
\curveto(126.88112069,149.75909392)(126.77612079,149.76909391)(126.66612465,149.7691037)
\curveto(126.64612092,149.75909392)(126.62112095,149.75409392)(126.59112465,149.7541037)
\curveto(126.571121,149.76409391)(126.55112102,149.76409391)(126.53112465,149.7541037)
\curveto(126.48112109,149.74409393)(126.43612113,149.73409394)(126.39612465,149.7241037)
\curveto(126.35612121,149.72409395)(126.31612125,149.71409396)(126.27612465,149.6941037)
\curveto(126.09612147,149.61409406)(125.94612162,149.49409418)(125.82612465,149.3341037)
\curveto(125.71612185,149.1740945)(125.62612194,148.99409468)(125.55612465,148.7941037)
\curveto(125.49612207,148.60409507)(125.45112212,148.3790953)(125.42112465,148.1191037)
\curveto(125.40112217,147.85909582)(125.39612217,147.59409608)(125.40612465,147.3241037)
\curveto(125.41612215,147.06409661)(125.44612212,146.81409686)(125.49612465,146.5741037)
\curveto(125.55612201,146.34409733)(125.63112194,146.15409752)(125.72112465,146.0041037)
\moveto(136.52112465,143.0191037)
\curveto(136.53111104,142.96910071)(136.53611103,142.8791008)(136.53612465,142.7491037)
\curveto(136.53611103,142.61910106)(136.52611104,142.52910115)(136.50612465,142.4791037)
\curveto(136.48611108,142.42910125)(136.48111109,142.3741013)(136.49112465,142.3141037)
\curveto(136.50111107,142.26410141)(136.50111107,142.21410146)(136.49112465,142.1641037)
\curveto(136.45111112,142.02410165)(136.42111115,141.88910179)(136.40112465,141.7591037)
\curveto(136.39111118,141.62910205)(136.36111121,141.50910217)(136.31112465,141.3991037)
\curveto(136.1711114,141.04910263)(136.00611156,140.75410292)(135.81612465,140.5141037)
\curveto(135.62611194,140.28410339)(135.35611221,140.09910358)(135.00612465,139.9591037)
\curveto(134.92611264,139.92910375)(134.84111273,139.90910377)(134.75112465,139.8991037)
\curveto(134.66111291,139.8791038)(134.57611299,139.85910382)(134.49612465,139.8391037)
\curveto(134.44611312,139.82910385)(134.39611317,139.82410385)(134.34612465,139.8241037)
\curveto(134.29611327,139.82410385)(134.24611332,139.81910386)(134.19612465,139.8091037)
\curveto(134.1661134,139.79910388)(134.11611345,139.79910388)(134.04612465,139.8091037)
\curveto(133.97611359,139.80910387)(133.92611364,139.81410386)(133.89612465,139.8241037)
\curveto(133.83611373,139.84410383)(133.77611379,139.85410382)(133.71612465,139.8541037)
\curveto(133.6661139,139.84410383)(133.61611395,139.84910383)(133.56612465,139.8691037)
\curveto(133.47611409,139.88910379)(133.38611418,139.91410376)(133.29612465,139.9441037)
\curveto(133.21611435,139.96410371)(133.13611443,139.99410368)(133.05612465,140.0341037)
\curveto(132.73611483,140.1741035)(132.48611508,140.36910331)(132.30612465,140.6191037)
\curveto(132.12611544,140.8791028)(131.97611559,141.18410249)(131.85612465,141.5341037)
\curveto(131.83611573,141.61410206)(131.82111575,141.69910198)(131.81112465,141.7891037)
\curveto(131.80111577,141.8791018)(131.78611578,141.96410171)(131.76612465,142.0441037)
\curveto(131.75611581,142.0741016)(131.75111582,142.10410157)(131.75112465,142.1341037)
\lineto(131.75112465,142.2391037)
\curveto(131.73111584,142.31910136)(131.72111585,142.39910128)(131.72112465,142.4791037)
\lineto(131.72112465,142.6141037)
\curveto(131.70111587,142.71410096)(131.70111587,142.81410086)(131.72112465,142.9141037)
\lineto(131.72112465,143.0941037)
\curveto(131.73111584,143.14410053)(131.73611583,143.18910049)(131.73612465,143.2291037)
\curveto(131.73611583,143.2791004)(131.74111583,143.32410035)(131.75112465,143.3641037)
\curveto(131.76111581,143.40410027)(131.7661158,143.43910024)(131.76612465,143.4691037)
\curveto(131.7661158,143.50910017)(131.7711158,143.54910013)(131.78112465,143.5891037)
\lineto(131.84112465,143.9191037)
\curveto(131.86111571,144.03909964)(131.89111568,144.14909953)(131.93112465,144.2491037)
\curveto(132.0711155,144.5790991)(132.23111534,144.85409882)(132.41112465,145.0741037)
\curveto(132.60111497,145.30409837)(132.86111471,145.48909819)(133.19112465,145.6291037)
\curveto(133.2711143,145.66909801)(133.35611421,145.69409798)(133.44612465,145.7041037)
\lineto(133.74612465,145.7641037)
\lineto(133.88112465,145.7641037)
\curveto(133.93111364,145.7740979)(133.98111359,145.7790979)(134.03112465,145.7791037)
\curveto(134.60111297,145.79909788)(135.06111251,145.69409798)(135.41112465,145.4641037)
\curveto(135.7711118,145.24409843)(136.03611153,144.94409873)(136.20612465,144.5641037)
\curveto(136.25611131,144.46409921)(136.29611127,144.36409931)(136.32612465,144.2641037)
\curveto(136.35611121,144.16409951)(136.38611118,144.05909962)(136.41612465,143.9491037)
\curveto(136.42611114,143.90909977)(136.43111114,143.8740998)(136.43112465,143.8441037)
\curveto(136.43111114,143.82409985)(136.43611113,143.79409988)(136.44612465,143.7541037)
\curveto(136.4661111,143.68409999)(136.47611109,143.60910007)(136.47612465,143.5291037)
\curveto(136.47611109,143.44910023)(136.48611108,143.36910031)(136.50612465,143.2891037)
\curveto(136.50611106,143.23910044)(136.50611106,143.19410048)(136.50612465,143.1541037)
\curveto(136.50611106,143.11410056)(136.51111106,143.06910061)(136.52112465,143.0191037)
\moveto(135.41112465,142.5841037)
\curveto(135.42111215,142.63410104)(135.42611214,142.70910097)(135.42612465,142.8091037)
\curveto(135.43611213,142.90910077)(135.43111214,142.98410069)(135.41112465,143.0341037)
\curveto(135.39111218,143.09410058)(135.38611218,143.14910053)(135.39612465,143.1991037)
\curveto(135.41611215,143.25910042)(135.41611215,143.31910036)(135.39612465,143.3791037)
\curveto(135.38611218,143.40910027)(135.38111219,143.44410023)(135.38112465,143.4841037)
\curveto(135.38111219,143.52410015)(135.37611219,143.56410011)(135.36612465,143.6041037)
\curveto(135.34611222,143.68409999)(135.32611224,143.75909992)(135.30612465,143.8291037)
\curveto(135.29611227,143.90909977)(135.28111229,143.98909969)(135.26112465,144.0691037)
\curveto(135.23111234,144.12909955)(135.20611236,144.18909949)(135.18612465,144.2491037)
\curveto(135.1661124,144.30909937)(135.13611243,144.36909931)(135.09612465,144.4291037)
\curveto(134.99611257,144.59909908)(134.8661127,144.73409894)(134.70612465,144.8341037)
\curveto(134.62611294,144.88409879)(134.53111304,144.91909876)(134.42112465,144.9391037)
\curveto(134.31111326,144.95909872)(134.18611338,144.96909871)(134.04612465,144.9691037)
\curveto(134.02611354,144.95909872)(134.00111357,144.95409872)(133.97112465,144.9541037)
\curveto(133.94111363,144.96409871)(133.91111366,144.96409871)(133.88112465,144.9541037)
\lineto(133.73112465,144.8941037)
\curveto(133.68111389,144.88409879)(133.63611393,144.86909881)(133.59612465,144.8491037)
\curveto(133.40611416,144.73909894)(133.26111431,144.59409908)(133.16112465,144.4141037)
\curveto(133.0711145,144.23409944)(132.99111458,144.02909965)(132.92112465,143.7991037)
\curveto(132.88111469,143.66910001)(132.86111471,143.53410014)(132.86112465,143.3941037)
\curveto(132.86111471,143.26410041)(132.85111472,143.11910056)(132.83112465,142.9591037)
\curveto(132.82111475,142.90910077)(132.81111476,142.84910083)(132.80112465,142.7791037)
\curveto(132.80111477,142.70910097)(132.81111476,142.64910103)(132.83112465,142.5991037)
\lineto(132.83112465,142.4341037)
\lineto(132.83112465,142.2541037)
\curveto(132.84111473,142.20410147)(132.85111472,142.14910153)(132.86112465,142.0891037)
\curveto(132.8711147,142.03910164)(132.87611469,141.98410169)(132.87612465,141.9241037)
\curveto(132.88611468,141.86410181)(132.90111467,141.80910187)(132.92112465,141.7591037)
\curveto(132.9711146,141.56910211)(133.03111454,141.39410228)(133.10112465,141.2341037)
\curveto(133.1711144,141.0741026)(133.27611429,140.94410273)(133.41612465,140.8441037)
\curveto(133.54611402,140.74410293)(133.68611388,140.674103)(133.83612465,140.6341037)
\curveto(133.8661137,140.62410305)(133.89111368,140.61910306)(133.91112465,140.6191037)
\curveto(133.94111363,140.62910305)(133.9711136,140.62910305)(134.00112465,140.6191037)
\curveto(134.02111355,140.61910306)(134.05111352,140.61410306)(134.09112465,140.6041037)
\curveto(134.13111344,140.60410307)(134.1661134,140.60910307)(134.19612465,140.6191037)
\curveto(134.23611333,140.62910305)(134.27611329,140.63410304)(134.31612465,140.6341037)
\curveto(134.35611321,140.63410304)(134.39611317,140.64410303)(134.43612465,140.6641037)
\curveto(134.67611289,140.74410293)(134.8711127,140.8791028)(135.02112465,141.0691037)
\curveto(135.14111243,141.24910243)(135.23111234,141.45410222)(135.29112465,141.6841037)
\curveto(135.31111226,141.75410192)(135.32611224,141.82410185)(135.33612465,141.8941037)
\curveto(135.34611222,141.9741017)(135.36111221,142.05410162)(135.38112465,142.1341037)
\curveto(135.38111219,142.19410148)(135.38611218,142.23910144)(135.39612465,142.2691037)
\curveto(135.39611217,142.28910139)(135.39611217,142.31410136)(135.39612465,142.3441037)
\curveto(135.39611217,142.38410129)(135.40111217,142.41410126)(135.41112465,142.4341037)
\lineto(135.41112465,142.5841037)
}
}
{
\newrgbcolor{curcolor}{0 0 0}
\pscustom[linestyle=none,fillstyle=solid,fillcolor=curcolor]
{
\newpath
\moveto(298.76927223,117.54192597)
\lineto(302.36927223,117.54192597)
\lineto(303.01427223,117.54192597)
\curveto(303.0942657,117.54191554)(303.16926563,117.53691555)(303.23927223,117.52692597)
\curveto(303.30926549,117.52691556)(303.36926543,117.51691557)(303.41927223,117.49692597)
\curveto(303.48926531,117.46691562)(303.54426525,117.40691568)(303.58427223,117.31692597)
\curveto(303.60426519,117.2869158)(303.61426518,117.24691584)(303.61427223,117.19692597)
\lineto(303.61427223,117.06192597)
\curveto(303.62426517,116.95191613)(303.61926518,116.84691624)(303.59927223,116.74692597)
\curveto(303.58926521,116.64691644)(303.55426524,116.57691651)(303.49427223,116.53692597)
\curveto(303.40426539,116.46691662)(303.26926553,116.43191665)(303.08927223,116.43192597)
\curveto(302.90926589,116.44191664)(302.74426605,116.44691664)(302.59427223,116.44692597)
\lineto(300.59927223,116.44692597)
\lineto(300.10427223,116.44692597)
\lineto(299.96927223,116.44692597)
\curveto(299.92926887,116.44691664)(299.88926891,116.44191664)(299.84927223,116.43192597)
\lineto(299.63927223,116.43192597)
\curveto(299.52926927,116.40191668)(299.44926935,116.36191672)(299.39927223,116.31192597)
\curveto(299.34926945,116.27191681)(299.31426948,116.21691687)(299.29427223,116.14692597)
\curveto(299.27426952,116.086917)(299.25926954,116.01691707)(299.24927223,115.93692597)
\curveto(299.23926956,115.85691723)(299.21926958,115.76691732)(299.18927223,115.66692597)
\curveto(299.13926966,115.46691762)(299.0992697,115.26191782)(299.06927223,115.05192597)
\curveto(299.03926976,114.84191824)(298.9992698,114.63691845)(298.94927223,114.43692597)
\curveto(298.92926987,114.36691872)(298.91926988,114.29691879)(298.91927223,114.22692597)
\curveto(298.91926988,114.16691892)(298.90926989,114.10191898)(298.88927223,114.03192597)
\curveto(298.87926992,114.00191908)(298.86926993,113.96191912)(298.85927223,113.91192597)
\curveto(298.85926994,113.87191921)(298.86426993,113.83191925)(298.87427223,113.79192597)
\curveto(298.8942699,113.74191934)(298.91926988,113.69691939)(298.94927223,113.65692597)
\curveto(298.98926981,113.62691946)(299.04926975,113.62191946)(299.12927223,113.64192597)
\curveto(299.18926961,113.66191942)(299.24926955,113.6869194)(299.30927223,113.71692597)
\curveto(299.36926943,113.75691933)(299.42926937,113.79191929)(299.48927223,113.82192597)
\curveto(299.54926925,113.84191924)(299.5992692,113.85691923)(299.63927223,113.86692597)
\curveto(299.82926897,113.94691914)(300.03426876,114.00191908)(300.25427223,114.03192597)
\curveto(300.48426831,114.06191902)(300.71426808,114.07191901)(300.94427223,114.06192597)
\curveto(301.18426761,114.06191902)(301.41426738,114.03691905)(301.63427223,113.98692597)
\curveto(301.85426694,113.94691914)(302.05426674,113.8869192)(302.23427223,113.80692597)
\curveto(302.28426651,113.7869193)(302.32926647,113.76691932)(302.36927223,113.74692597)
\curveto(302.41926638,113.72691936)(302.46926633,113.70191938)(302.51927223,113.67192597)
\curveto(302.86926593,113.46191962)(303.14926565,113.23191985)(303.35927223,112.98192597)
\curveto(303.57926522,112.73192035)(303.77426502,112.40692068)(303.94427223,112.00692597)
\curveto(303.9942648,111.89692119)(304.02926477,111.7869213)(304.04927223,111.67692597)
\curveto(304.06926473,111.56692152)(304.0942647,111.45192163)(304.12427223,111.33192597)
\curveto(304.13426466,111.30192178)(304.13926466,111.25692183)(304.13927223,111.19692597)
\curveto(304.15926464,111.13692195)(304.16926463,111.06692202)(304.16927223,110.98692597)
\curveto(304.16926463,110.91692217)(304.17926462,110.85192223)(304.19927223,110.79192597)
\lineto(304.19927223,110.62692597)
\curveto(304.20926459,110.57692251)(304.21426458,110.50692258)(304.21427223,110.41692597)
\curveto(304.21426458,110.32692276)(304.20426459,110.25692283)(304.18427223,110.20692597)
\curveto(304.16426463,110.14692294)(304.15926464,110.086923)(304.16927223,110.02692597)
\curveto(304.17926462,109.97692311)(304.17426462,109.92692316)(304.15427223,109.87692597)
\curveto(304.11426468,109.71692337)(304.07926472,109.56692352)(304.04927223,109.42692597)
\curveto(304.01926478,109.2869238)(303.97426482,109.15192393)(303.91427223,109.02192597)
\curveto(303.75426504,108.65192443)(303.53426526,108.31692477)(303.25427223,108.01692597)
\curveto(302.97426582,107.71692537)(302.65426614,107.4869256)(302.29427223,107.32692597)
\curveto(302.12426667,107.24692584)(301.92426687,107.17192591)(301.69427223,107.10192597)
\curveto(301.58426721,107.06192602)(301.46926733,107.03692605)(301.34927223,107.02692597)
\curveto(301.22926757,107.01692607)(301.10926769,106.99692609)(300.98927223,106.96692597)
\curveto(300.93926786,106.94692614)(300.88426791,106.94692614)(300.82427223,106.96692597)
\curveto(300.76426803,106.97692611)(300.70426809,106.97192611)(300.64427223,106.95192597)
\curveto(300.54426825,106.93192615)(300.44426835,106.93192615)(300.34427223,106.95192597)
\lineto(300.20927223,106.95192597)
\curveto(300.15926864,106.97192611)(300.0992687,106.9819261)(300.02927223,106.98192597)
\curveto(299.96926883,106.97192611)(299.91426888,106.97692611)(299.86427223,106.99692597)
\curveto(299.82426897,107.00692608)(299.78926901,107.01192607)(299.75927223,107.01192597)
\curveto(299.72926907,107.01192607)(299.6942691,107.01692607)(299.65427223,107.02692597)
\lineto(299.38427223,107.08692597)
\curveto(299.2942695,107.10692598)(299.20926959,107.13692595)(299.12927223,107.17692597)
\curveto(298.78927001,107.31692577)(298.4992703,107.47192561)(298.25927223,107.64192597)
\curveto(298.01927078,107.82192526)(297.799271,108.05192503)(297.59927223,108.33192597)
\curveto(297.44927135,108.56192452)(297.33427146,108.80192428)(297.25427223,109.05192597)
\curveto(297.23427156,109.10192398)(297.22427157,109.14692394)(297.22427223,109.18692597)
\curveto(297.22427157,109.23692385)(297.21427158,109.2869238)(297.19427223,109.33692597)
\curveto(297.17427162,109.39692369)(297.15927164,109.47692361)(297.14927223,109.57692597)
\curveto(297.14927165,109.67692341)(297.16927163,109.75192333)(297.20927223,109.80192597)
\curveto(297.25927154,109.8819232)(297.33927146,109.92692316)(297.44927223,109.93692597)
\curveto(297.55927124,109.94692314)(297.67427112,109.95192313)(297.79427223,109.95192597)
\lineto(297.95927223,109.95192597)
\curveto(298.01927078,109.95192313)(298.07427072,109.94192314)(298.12427223,109.92192597)
\curveto(298.21427058,109.90192318)(298.28427051,109.86192322)(298.33427223,109.80192597)
\curveto(298.40427039,109.71192337)(298.44927035,109.60192348)(298.46927223,109.47192597)
\curveto(298.4992703,109.35192373)(298.54427025,109.24692384)(298.60427223,109.15692597)
\curveto(298.79427,108.81692427)(299.05426974,108.54692454)(299.38427223,108.34692597)
\curveto(299.48426931,108.2869248)(299.58926921,108.23692485)(299.69927223,108.19692597)
\curveto(299.81926898,108.16692492)(299.93926886,108.13192495)(300.05927223,108.09192597)
\curveto(300.22926857,108.04192504)(300.43426836,108.02192506)(300.67427223,108.03192597)
\curveto(300.92426787,108.05192503)(301.12426767,108.086925)(301.27427223,108.13692597)
\curveto(301.64426715,108.25692483)(301.93426686,108.41692467)(302.14427223,108.61692597)
\curveto(302.36426643,108.82692426)(302.54426625,109.10692398)(302.68427223,109.45692597)
\curveto(302.73426606,109.55692353)(302.76426603,109.66192342)(302.77427223,109.77192597)
\curveto(302.794266,109.8819232)(302.81926598,109.99692309)(302.84927223,110.11692597)
\lineto(302.84927223,110.22192597)
\curveto(302.85926594,110.26192282)(302.86426593,110.30192278)(302.86427223,110.34192597)
\curveto(302.87426592,110.37192271)(302.87426592,110.40692268)(302.86427223,110.44692597)
\lineto(302.86427223,110.56692597)
\curveto(302.86426593,110.82692226)(302.83426596,111.07192201)(302.77427223,111.30192597)
\curveto(302.66426613,111.65192143)(302.50926629,111.94692114)(302.30927223,112.18692597)
\curveto(302.10926669,112.43692065)(301.84926695,112.63192045)(301.52927223,112.77192597)
\lineto(301.34927223,112.83192597)
\curveto(301.2992675,112.85192023)(301.23926756,112.87192021)(301.16927223,112.89192597)
\curveto(301.11926768,112.91192017)(301.05926774,112.92192016)(300.98927223,112.92192597)
\curveto(300.92926787,112.93192015)(300.86426793,112.94692014)(300.79427223,112.96692597)
\lineto(300.64427223,112.96692597)
\curveto(300.60426819,112.9869201)(300.54926825,112.99692009)(300.47927223,112.99692597)
\curveto(300.41926838,112.99692009)(300.36426843,112.9869201)(300.31427223,112.96692597)
\lineto(300.20927223,112.96692597)
\curveto(300.17926862,112.96692012)(300.14426865,112.96192012)(300.10427223,112.95192597)
\lineto(299.86427223,112.89192597)
\curveto(299.78426901,112.8819202)(299.70426909,112.86192022)(299.62427223,112.83192597)
\curveto(299.38426941,112.73192035)(299.15426964,112.59692049)(298.93427223,112.42692597)
\curveto(298.84426995,112.35692073)(298.75927004,112.2819208)(298.67927223,112.20192597)
\curveto(298.5992702,112.13192095)(298.4992703,112.07692101)(298.37927223,112.03692597)
\curveto(298.28927051,112.00692108)(298.14927065,111.99692109)(297.95927223,112.00692597)
\curveto(297.77927102,112.01692107)(297.65927114,112.04192104)(297.59927223,112.08192597)
\curveto(297.54927125,112.12192096)(297.50927129,112.1819209)(297.47927223,112.26192597)
\curveto(297.45927134,112.34192074)(297.45927134,112.42692066)(297.47927223,112.51692597)
\curveto(297.50927129,112.63692045)(297.52927127,112.75692033)(297.53927223,112.87692597)
\curveto(297.55927124,113.00692008)(297.58427121,113.13191995)(297.61427223,113.25192597)
\curveto(297.63427116,113.29191979)(297.63927116,113.32691976)(297.62927223,113.35692597)
\curveto(297.62927117,113.39691969)(297.63927116,113.44191964)(297.65927223,113.49192597)
\curveto(297.67927112,113.5819195)(297.6942711,113.67191941)(297.70427223,113.76192597)
\curveto(297.71427108,113.86191922)(297.73427106,113.95691913)(297.76427223,114.04692597)
\curveto(297.77427102,114.10691898)(297.77927102,114.16691892)(297.77927223,114.22692597)
\curveto(297.78927101,114.2869188)(297.80427099,114.34691874)(297.82427223,114.40692597)
\curveto(297.87427092,114.60691848)(297.90927089,114.81191827)(297.92927223,115.02192597)
\curveto(297.95927084,115.24191784)(297.9992708,115.45191763)(298.04927223,115.65192597)
\curveto(298.07927072,115.75191733)(298.0992707,115.85191723)(298.10927223,115.95192597)
\curveto(298.11927068,116.05191703)(298.13427066,116.15191693)(298.15427223,116.25192597)
\curveto(298.16427063,116.2819168)(298.16927063,116.32191676)(298.16927223,116.37192597)
\curveto(298.1992706,116.4819166)(298.21927058,116.5869165)(298.22927223,116.68692597)
\curveto(298.24927055,116.79691629)(298.27427052,116.90691618)(298.30427223,117.01692597)
\curveto(298.32427047,117.09691599)(298.33927046,117.16691592)(298.34927223,117.22692597)
\curveto(298.35927044,117.29691579)(298.38427041,117.35691573)(298.42427223,117.40692597)
\curveto(298.44427035,117.43691565)(298.47427032,117.45691563)(298.51427223,117.46692597)
\curveto(298.55427024,117.4869156)(298.5992702,117.50691558)(298.64927223,117.52692597)
\curveto(298.70927009,117.52691556)(298.74927005,117.53191555)(298.76927223,117.54192597)
}
}
{
\newrgbcolor{curcolor}{0 0 0}
\pscustom[linestyle=none,fillstyle=solid,fillcolor=curcolor]
{
\newpath
\moveto(306.06888161,117.54192597)
\lineto(310.86888161,117.54192597)
\lineto(311.87388161,117.54192597)
\curveto(312.01387451,117.54191554)(312.13387439,117.53191555)(312.23388161,117.51192597)
\curveto(312.34387418,117.50191558)(312.4238741,117.45691563)(312.47388161,117.37692597)
\curveto(312.49387403,117.33691575)(312.50387402,117.2869158)(312.50388161,117.22692597)
\curveto(312.51387401,117.16691592)(312.518874,117.10191598)(312.51888161,117.03192597)
\lineto(312.51888161,116.76192597)
\curveto(312.518874,116.67191641)(312.50887401,116.59191649)(312.48888161,116.52192597)
\curveto(312.44887407,116.44191664)(312.40387412,116.37191671)(312.35388161,116.31192597)
\lineto(312.20388161,116.13192597)
\curveto(312.17387435,116.081917)(312.13887438,116.04191704)(312.09888161,116.01192597)
\curveto(312.05887446,115.9819171)(312.0188745,115.94191714)(311.97888161,115.89192597)
\curveto(311.89887462,115.7819173)(311.81387471,115.67191741)(311.72388161,115.56192597)
\curveto(311.63387489,115.46191762)(311.54887497,115.35691773)(311.46888161,115.24692597)
\curveto(311.32887519,115.04691804)(311.18887533,114.83691825)(311.04888161,114.61692597)
\curveto(310.90887561,114.40691868)(310.76887575,114.19191889)(310.62888161,113.97192597)
\curveto(310.57887594,113.8819192)(310.52887599,113.7869193)(310.47888161,113.68692597)
\curveto(310.42887609,113.5869195)(310.37387615,113.49191959)(310.31388161,113.40192597)
\curveto(310.29387623,113.3819197)(310.28387624,113.35691973)(310.28388161,113.32692597)
\curveto(310.28387624,113.29691979)(310.27387625,113.27191981)(310.25388161,113.25192597)
\curveto(310.18387634,113.15191993)(310.1188764,113.03692005)(310.05888161,112.90692597)
\curveto(309.99887652,112.7869203)(309.94387658,112.67192041)(309.89388161,112.56192597)
\curveto(309.79387673,112.33192075)(309.69887682,112.09692099)(309.60888161,111.85692597)
\curveto(309.518877,111.61692147)(309.4188771,111.37692171)(309.30888161,111.13692597)
\curveto(309.28887723,111.086922)(309.27387725,111.04192204)(309.26388161,111.00192597)
\curveto(309.26387726,110.96192212)(309.25387727,110.91692217)(309.23388161,110.86692597)
\curveto(309.18387734,110.74692234)(309.13887738,110.62192246)(309.09888161,110.49192597)
\curveto(309.06887745,110.37192271)(309.03387749,110.25192283)(308.99388161,110.13192597)
\curveto(308.91387761,109.90192318)(308.84887767,109.66192342)(308.79888161,109.41192597)
\curveto(308.75887776,109.17192391)(308.70887781,108.93192415)(308.64888161,108.69192597)
\curveto(308.60887791,108.54192454)(308.58387794,108.39192469)(308.57388161,108.24192597)
\curveto(308.56387796,108.09192499)(308.54387798,107.94192514)(308.51388161,107.79192597)
\curveto(308.50387802,107.75192533)(308.49887802,107.69192539)(308.49888161,107.61192597)
\curveto(308.46887805,107.49192559)(308.43887808,107.39192569)(308.40888161,107.31192597)
\curveto(308.37887814,107.23192585)(308.30887821,107.17692591)(308.19888161,107.14692597)
\curveto(308.14887837,107.12692596)(308.09387843,107.11692597)(308.03388161,107.11692597)
\lineto(307.83888161,107.11692597)
\curveto(307.69887882,107.11692597)(307.55887896,107.12192596)(307.41888161,107.13192597)
\curveto(307.28887923,107.14192594)(307.19387933,107.1869259)(307.13388161,107.26692597)
\curveto(307.09387943,107.32692576)(307.07387945,107.41192567)(307.07388161,107.52192597)
\curveto(307.08387944,107.63192545)(307.09887942,107.72692536)(307.11888161,107.80692597)
\lineto(307.11888161,107.88192597)
\curveto(307.12887939,107.91192517)(307.13387939,107.94192514)(307.13388161,107.97192597)
\curveto(307.15387937,108.05192503)(307.16387936,108.12692496)(307.16388161,108.19692597)
\curveto(307.16387936,108.26692482)(307.17387935,108.33692475)(307.19388161,108.40692597)
\curveto(307.24387928,108.59692449)(307.28387924,108.7819243)(307.31388161,108.96192597)
\curveto(307.34387918,109.15192393)(307.38387914,109.33192375)(307.43388161,109.50192597)
\curveto(307.45387907,109.55192353)(307.46387906,109.59192349)(307.46388161,109.62192597)
\curveto(307.46387906,109.65192343)(307.46887905,109.6869234)(307.47888161,109.72692597)
\curveto(307.57887894,110.02692306)(307.66887885,110.32192276)(307.74888161,110.61192597)
\curveto(307.83887868,110.90192218)(307.94387858,111.1819219)(308.06388161,111.45192597)
\curveto(308.3238782,112.03192105)(308.59387793,112.5819205)(308.87388161,113.10192597)
\curveto(309.15387737,113.63191945)(309.46387706,114.13691895)(309.80388161,114.61692597)
\curveto(309.94387658,114.81691827)(310.09387643,115.00691808)(310.25388161,115.18692597)
\curveto(310.41387611,115.37691771)(310.56387596,115.56691752)(310.70388161,115.75692597)
\curveto(310.74387578,115.80691728)(310.77887574,115.85191723)(310.80888161,115.89192597)
\curveto(310.84887567,115.94191714)(310.88387564,115.99191709)(310.91388161,116.04192597)
\curveto(310.9238756,116.06191702)(310.93387559,116.086917)(310.94388161,116.11692597)
\curveto(310.96387556,116.14691694)(310.96387556,116.17691691)(310.94388161,116.20692597)
\curveto(310.9238756,116.26691682)(310.88887563,116.30191678)(310.83888161,116.31192597)
\curveto(310.78887573,116.33191675)(310.73887578,116.35191673)(310.68888161,116.37192597)
\lineto(310.58388161,116.37192597)
\curveto(310.54387598,116.3819167)(310.49387603,116.3819167)(310.43388161,116.37192597)
\lineto(310.28388161,116.37192597)
\lineto(309.68388161,116.37192597)
\lineto(307.04388161,116.37192597)
\lineto(306.30888161,116.37192597)
\lineto(306.06888161,116.37192597)
\curveto(305.99888052,116.3819167)(305.93888058,116.39691669)(305.88888161,116.41692597)
\curveto(305.79888072,116.45691663)(305.73888078,116.51691657)(305.70888161,116.59692597)
\curveto(305.65888086,116.69691639)(305.64388088,116.84191624)(305.66388161,117.03192597)
\curveto(305.68388084,117.23191585)(305.7188808,117.36691572)(305.76888161,117.43692597)
\curveto(305.78888073,117.45691563)(305.81388071,117.47191561)(305.84388161,117.48192597)
\lineto(305.96388161,117.54192597)
\curveto(305.98388054,117.54191554)(305.99888052,117.53691555)(306.00888161,117.52692597)
\curveto(306.02888049,117.52691556)(306.04888047,117.53191555)(306.06888161,117.54192597)
}
}
{
\newrgbcolor{curcolor}{0 0 0}
\pscustom[linestyle=none,fillstyle=solid,fillcolor=curcolor]
{
\newpath
\moveto(314.91349098,108.76692597)
\lineto(315.21349098,108.76692597)
\curveto(315.32348892,108.77692431)(315.42848882,108.77692431)(315.52849098,108.76692597)
\curveto(315.63848861,108.76692432)(315.73848851,108.75692433)(315.82849098,108.73692597)
\curveto(315.91848833,108.72692436)(315.98848826,108.70192438)(316.03849098,108.66192597)
\curveto(316.05848819,108.64192444)(316.07348817,108.61192447)(316.08349098,108.57192597)
\curveto(316.10348814,108.53192455)(316.12348812,108.4869246)(316.14349098,108.43692597)
\lineto(316.14349098,108.36192597)
\curveto(316.15348809,108.31192477)(316.15348809,108.25692483)(316.14349098,108.19692597)
\lineto(316.14349098,108.04692597)
\lineto(316.14349098,107.56692597)
\curveto(316.1434881,107.39692569)(316.10348814,107.27692581)(316.02349098,107.20692597)
\curveto(315.95348829,107.15692593)(315.86348838,107.13192595)(315.75349098,107.13192597)
\lineto(315.42349098,107.13192597)
\lineto(314.97349098,107.13192597)
\curveto(314.82348942,107.13192595)(314.70848954,107.16192592)(314.62849098,107.22192597)
\curveto(314.58848966,107.25192583)(314.55848969,107.30192578)(314.53849098,107.37192597)
\curveto(314.51848973,107.45192563)(314.50348974,107.53692555)(314.49349098,107.62692597)
\lineto(314.49349098,107.91192597)
\curveto(314.50348974,108.01192507)(314.50848974,108.09692499)(314.50849098,108.16692597)
\lineto(314.50849098,108.36192597)
\curveto(314.50848974,108.42192466)(314.51848973,108.47692461)(314.53849098,108.52692597)
\curveto(314.57848967,108.63692445)(314.6484896,108.70692438)(314.74849098,108.73692597)
\curveto(314.77848947,108.73692435)(314.83348941,108.74692434)(314.91349098,108.76692597)
}
}
{
\newrgbcolor{curcolor}{0 0 0}
\pscustom[linestyle=none,fillstyle=solid,fillcolor=curcolor]
{
\newpath
\moveto(325.05864723,112.72692597)
\curveto(325.0586396,112.64692044)(325.06363959,112.56692052)(325.07364723,112.48692597)
\curveto(325.08363957,112.40692068)(325.07863958,112.33192075)(325.05864723,112.26192597)
\curveto(325.03863962,112.22192086)(325.03363962,112.17692091)(325.04364723,112.12692597)
\curveto(325.0536396,112.086921)(325.0536396,112.04692104)(325.04364723,112.00692597)
\lineto(325.04364723,111.85692597)
\curveto(325.03363962,111.76692132)(325.02863963,111.67692141)(325.02864723,111.58692597)
\curveto(325.02863963,111.50692158)(325.02363963,111.42692166)(325.01364723,111.34692597)
\lineto(324.98364723,111.10692597)
\curveto(324.97363968,111.03692205)(324.96363969,110.96192212)(324.95364723,110.88192597)
\curveto(324.94363971,110.84192224)(324.93863972,110.80192228)(324.93864723,110.76192597)
\curveto(324.93863972,110.72192236)(324.93363972,110.67692241)(324.92364723,110.62692597)
\curveto(324.88363977,110.4869226)(324.8536398,110.34692274)(324.83364723,110.20692597)
\curveto(324.82363983,110.06692302)(324.79363986,109.93192315)(324.74364723,109.80192597)
\curveto(324.69363996,109.63192345)(324.63864002,109.46692362)(324.57864723,109.30692597)
\curveto(324.52864013,109.14692394)(324.46864019,108.99192409)(324.39864723,108.84192597)
\curveto(324.37864028,108.7819243)(324.34864031,108.72192436)(324.30864723,108.66192597)
\lineto(324.21864723,108.51192597)
\curveto(324.01864064,108.19192489)(323.80364085,107.92692516)(323.57364723,107.71692597)
\curveto(323.34364131,107.50692558)(323.04864161,107.32692576)(322.68864723,107.17692597)
\curveto(322.56864209,107.12692596)(322.43864222,107.09192599)(322.29864723,107.07192597)
\curveto(322.16864249,107.05192603)(322.03364262,107.02692606)(321.89364723,106.99692597)
\curveto(321.83364282,106.9869261)(321.77364288,106.9819261)(321.71364723,106.98192597)
\curveto(321.653643,106.9819261)(321.58864307,106.97692611)(321.51864723,106.96692597)
\curveto(321.48864317,106.95692613)(321.43864322,106.95692613)(321.36864723,106.96692597)
\lineto(321.21864723,106.96692597)
\lineto(321.06864723,106.96692597)
\curveto(320.98864367,106.9869261)(320.90364375,107.00192608)(320.81364723,107.01192597)
\curveto(320.73364392,107.01192607)(320.658644,107.02192606)(320.58864723,107.04192597)
\curveto(320.54864411,107.05192603)(320.51364414,107.05692603)(320.48364723,107.05692597)
\curveto(320.46364419,107.04692604)(320.43864422,107.05192603)(320.40864723,107.07192597)
\lineto(320.13864723,107.13192597)
\curveto(320.04864461,107.16192592)(319.96364469,107.19192589)(319.88364723,107.22192597)
\curveto(319.30364535,107.46192562)(318.86864579,107.83192525)(318.57864723,108.33192597)
\curveto(318.49864616,108.46192462)(318.43364622,108.59692449)(318.38364723,108.73692597)
\curveto(318.34364631,108.87692421)(318.29864636,109.02692406)(318.24864723,109.18692597)
\curveto(318.22864643,109.26692382)(318.22364643,109.34692374)(318.23364723,109.42692597)
\curveto(318.2536464,109.50692358)(318.28864637,109.56192352)(318.33864723,109.59192597)
\curveto(318.36864629,109.61192347)(318.42364623,109.62692346)(318.50364723,109.63692597)
\curveto(318.58364607,109.65692343)(318.66864599,109.66692342)(318.75864723,109.66692597)
\curveto(318.84864581,109.67692341)(318.93364572,109.67692341)(319.01364723,109.66692597)
\curveto(319.10364555,109.65692343)(319.17364548,109.64692344)(319.22364723,109.63692597)
\curveto(319.24364541,109.62692346)(319.26864539,109.61192347)(319.29864723,109.59192597)
\curveto(319.33864532,109.57192351)(319.36864529,109.55192353)(319.38864723,109.53192597)
\curveto(319.44864521,109.45192363)(319.49364516,109.35692373)(319.52364723,109.24692597)
\curveto(319.56364509,109.13692395)(319.60864505,109.03692405)(319.65864723,108.94692597)
\curveto(319.90864475,108.55692453)(320.27864438,108.2869248)(320.76864723,108.13692597)
\curveto(320.83864382,108.11692497)(320.90864375,108.10192498)(320.97864723,108.09192597)
\curveto(321.0586436,108.09192499)(321.13864352,108.081925)(321.21864723,108.06192597)
\curveto(321.2586434,108.05192503)(321.31364334,108.04692504)(321.38364723,108.04692597)
\curveto(321.46364319,108.04692504)(321.51864314,108.05192503)(321.54864723,108.06192597)
\curveto(321.57864308,108.07192501)(321.60864305,108.07692501)(321.63864723,108.07692597)
\lineto(321.74364723,108.07692597)
\curveto(321.82364283,108.09692499)(321.89864276,108.11692497)(321.96864723,108.13692597)
\curveto(322.04864261,108.15692493)(322.12364253,108.1819249)(322.19364723,108.21192597)
\curveto(322.54364211,108.36192472)(322.81364184,108.57692451)(323.00364723,108.85692597)
\curveto(323.19364146,109.13692395)(323.34864131,109.46192362)(323.46864723,109.83192597)
\curveto(323.49864116,109.91192317)(323.51864114,109.9869231)(323.52864723,110.05692597)
\curveto(323.54864111,110.12692296)(323.56864109,110.20192288)(323.58864723,110.28192597)
\curveto(323.60864105,110.37192271)(323.62364103,110.46692262)(323.63364723,110.56692597)
\curveto(323.653641,110.67692241)(323.67364098,110.7819223)(323.69364723,110.88192597)
\curveto(323.70364095,110.93192215)(323.70864095,110.9819221)(323.70864723,111.03192597)
\curveto(323.71864094,111.09192199)(323.72364093,111.14692194)(323.72364723,111.19692597)
\curveto(323.74364091,111.25692183)(323.7536409,111.33192175)(323.75364723,111.42192597)
\curveto(323.7536409,111.52192156)(323.74364091,111.60192148)(323.72364723,111.66192597)
\curveto(323.69364096,111.75192133)(323.64364101,111.79192129)(323.57364723,111.78192597)
\curveto(323.51364114,111.77192131)(323.4586412,111.74192134)(323.40864723,111.69192597)
\curveto(323.32864133,111.64192144)(323.2586414,111.5819215)(323.19864723,111.51192597)
\curveto(323.14864151,111.44192164)(323.08364157,111.3819217)(323.00364723,111.33192597)
\curveto(322.84364181,111.22192186)(322.67864198,111.12192196)(322.50864723,111.03192597)
\curveto(322.33864232,110.95192213)(322.14364251,110.8819222)(321.92364723,110.82192597)
\curveto(321.82364283,110.79192229)(321.72364293,110.77692231)(321.62364723,110.77692597)
\curveto(321.53364312,110.77692231)(321.43364322,110.76692232)(321.32364723,110.74692597)
\lineto(321.17364723,110.74692597)
\curveto(321.12364353,110.76692232)(321.07364358,110.77192231)(321.02364723,110.76192597)
\curveto(320.98364367,110.75192233)(320.94364371,110.75192233)(320.90364723,110.76192597)
\curveto(320.87364378,110.77192231)(320.82864383,110.77692231)(320.76864723,110.77692597)
\curveto(320.70864395,110.7869223)(320.64364401,110.79692229)(320.57364723,110.80692597)
\lineto(320.39364723,110.83692597)
\curveto(319.94364471,110.95692213)(319.56364509,111.12192196)(319.25364723,111.33192597)
\curveto(318.98364567,111.52192156)(318.7536459,111.75192133)(318.56364723,112.02192597)
\curveto(318.38364627,112.30192078)(318.23864642,112.61692047)(318.12864723,112.96692597)
\lineto(318.06864723,113.17692597)
\curveto(318.0586466,113.25691983)(318.04364661,113.33691975)(318.02364723,113.41692597)
\curveto(318.01364664,113.44691964)(318.00864665,113.47691961)(318.00864723,113.50692597)
\curveto(318.00864665,113.53691955)(318.00364665,113.56691952)(317.99364723,113.59692597)
\curveto(317.98364667,113.65691943)(317.97864668,113.71691937)(317.97864723,113.77692597)
\curveto(317.97864668,113.84691924)(317.96864669,113.90691918)(317.94864723,113.95692597)
\lineto(317.94864723,114.13692597)
\curveto(317.93864672,114.1869189)(317.93364672,114.25691883)(317.93364723,114.34692597)
\curveto(317.93364672,114.43691865)(317.94364671,114.50691858)(317.96364723,114.55692597)
\lineto(317.96364723,114.72192597)
\curveto(317.98364667,114.80191828)(317.99364666,114.87691821)(317.99364723,114.94692597)
\curveto(318.00364665,115.01691807)(318.01864664,115.086918)(318.03864723,115.15692597)
\curveto(318.09864656,115.35691773)(318.1586465,115.54691754)(318.21864723,115.72692597)
\curveto(318.28864637,115.90691718)(318.37864628,116.07691701)(318.48864723,116.23692597)
\curveto(318.52864613,116.30691678)(318.56864609,116.37191671)(318.60864723,116.43192597)
\lineto(318.75864723,116.61192597)
\curveto(318.77864588,116.62191646)(318.79864586,116.63691645)(318.81864723,116.65692597)
\curveto(318.90864575,116.7869163)(319.01864564,116.89691619)(319.14864723,116.98692597)
\curveto(319.40864525,117.1869159)(319.67364498,117.34191574)(319.94364723,117.45192597)
\curveto(320.02364463,117.49191559)(320.10364455,117.52191556)(320.18364723,117.54192597)
\curveto(320.27364438,117.57191551)(320.36364429,117.59691549)(320.45364723,117.61692597)
\curveto(320.5536441,117.64691544)(320.653644,117.66691542)(320.75364723,117.67692597)
\curveto(320.8536438,117.6869154)(320.9586437,117.70191538)(321.06864723,117.72192597)
\curveto(321.09864356,117.73191535)(321.13864352,117.73191535)(321.18864723,117.72192597)
\curveto(321.24864341,117.71191537)(321.28864337,117.71691537)(321.30864723,117.73692597)
\curveto(322.02864263,117.75691533)(322.62864203,117.64191544)(323.10864723,117.39192597)
\curveto(323.58864107,117.14191594)(323.96364069,116.80191628)(324.23364723,116.37192597)
\curveto(324.32364033,116.23191685)(324.40364025,116.086917)(324.47364723,115.93692597)
\curveto(324.54364011,115.7869173)(324.61364004,115.62691746)(324.68364723,115.45692597)
\curveto(324.73363992,115.31691777)(324.77363988,115.16691792)(324.80364723,115.00692597)
\curveto(324.83363982,114.84691824)(324.86863979,114.6869184)(324.90864723,114.52692597)
\curveto(324.92863973,114.47691861)(324.93863972,114.42191866)(324.93864723,114.36192597)
\curveto(324.93863972,114.31191877)(324.94363971,114.26191882)(324.95364723,114.21192597)
\curveto(324.97363968,114.15191893)(324.98363967,114.086919)(324.98364723,114.01692597)
\curveto(324.98363967,113.95691913)(324.99363966,113.90191918)(325.01364723,113.85192597)
\lineto(325.01364723,113.68692597)
\curveto(325.03363962,113.63691945)(325.03863962,113.5869195)(325.02864723,113.53692597)
\curveto(325.01863964,113.4869196)(325.02363963,113.43691965)(325.04364723,113.38692597)
\curveto(325.04363961,113.36691972)(325.03863962,113.34191974)(325.02864723,113.31192597)
\curveto(325.02863963,113.2819198)(325.03363962,113.25691983)(325.04364723,113.23692597)
\curveto(325.0536396,113.20691988)(325.0536396,113.17191991)(325.04364723,113.13192597)
\curveto(325.04363961,113.09191999)(325.04863961,113.05192003)(325.05864723,113.01192597)
\curveto(325.06863959,112.97192011)(325.06863959,112.92692016)(325.05864723,112.87692597)
\lineto(325.05864723,112.72692597)
\moveto(323.55864723,114.03192597)
\curveto(323.56864109,114.081919)(323.57364108,114.14191894)(323.57364723,114.21192597)
\curveto(323.57364108,114.2819188)(323.56864109,114.34191874)(323.55864723,114.39192597)
\curveto(323.54864111,114.44191864)(323.54364111,114.51691857)(323.54364723,114.61692597)
\curveto(323.52364113,114.69691839)(323.50364115,114.77191831)(323.48364723,114.84192597)
\curveto(323.47364118,114.91191817)(323.4586412,114.9819181)(323.43864723,115.05192597)
\curveto(323.29864136,115.4819176)(323.10364155,115.81691727)(322.85364723,116.05692597)
\curveto(322.61364204,116.29691679)(322.26864239,116.47691661)(321.81864723,116.59692597)
\curveto(321.72864293,116.61691647)(321.62864303,116.62691646)(321.51864723,116.62692597)
\lineto(321.18864723,116.62692597)
\curveto(321.16864349,116.60691648)(321.13364352,116.59691649)(321.08364723,116.59692597)
\curveto(321.03364362,116.60691648)(320.98864367,116.60691648)(320.94864723,116.59692597)
\curveto(320.86864379,116.57691651)(320.79364386,116.55691653)(320.72364723,116.53692597)
\lineto(320.51364723,116.47692597)
\curveto(320.22364443,116.34691674)(319.99364466,116.16691692)(319.82364723,115.93692597)
\curveto(319.653645,115.71691737)(319.51864514,115.45691763)(319.41864723,115.15692597)
\curveto(319.38864527,115.06691802)(319.36364529,114.97191811)(319.34364723,114.87192597)
\curveto(319.33364532,114.7819183)(319.31864534,114.6869184)(319.29864723,114.58692597)
\lineto(319.29864723,114.45192597)
\curveto(319.26864539,114.34191874)(319.2586454,114.20191888)(319.26864723,114.03192597)
\curveto(319.28864537,113.87191921)(319.30864535,113.74191934)(319.32864723,113.64192597)
\curveto(319.34864531,113.5819195)(319.36364529,113.52191956)(319.37364723,113.46192597)
\curveto(319.38364527,113.41191967)(319.39864526,113.36191972)(319.41864723,113.31192597)
\curveto(319.49864516,113.11191997)(319.59364506,112.92192016)(319.70364723,112.74192597)
\curveto(319.82364483,112.56192052)(319.96364469,112.41692067)(320.12364723,112.30692597)
\curveto(320.17364448,112.25692083)(320.22864443,112.21692087)(320.28864723,112.18692597)
\curveto(320.34864431,112.15692093)(320.40864425,112.12192096)(320.46864723,112.08192597)
\curveto(320.61864404,112.00192108)(320.80364385,111.93692115)(321.02364723,111.88692597)
\curveto(321.07364358,111.86692122)(321.11364354,111.86192122)(321.14364723,111.87192597)
\curveto(321.18364347,111.8819212)(321.22864343,111.87692121)(321.27864723,111.85692597)
\curveto(321.31864334,111.84692124)(321.37364328,111.84192124)(321.44364723,111.84192597)
\curveto(321.51364314,111.84192124)(321.57364308,111.84692124)(321.62364723,111.85692597)
\curveto(321.72364293,111.87692121)(321.81864284,111.89192119)(321.90864723,111.90192597)
\curveto(321.99864266,111.92192116)(322.08864257,111.95192113)(322.17864723,111.99192597)
\curveto(322.71864194,112.21192087)(323.11364154,112.60692048)(323.36364723,113.17692597)
\curveto(323.41364124,113.27691981)(323.44864121,113.37691971)(323.46864723,113.47692597)
\curveto(323.48864117,113.5869195)(323.51364114,113.69691939)(323.54364723,113.80692597)
\curveto(323.54364111,113.90691918)(323.54864111,113.9819191)(323.55864723,114.03192597)
}
}
{
\newrgbcolor{curcolor}{0 0 0}
\pscustom[linestyle=none,fillstyle=solid,fillcolor=curcolor]
{
\newpath
\moveto(336.27325661,115.65192597)
\curveto(336.07324631,115.36191772)(335.86324652,115.07691801)(335.64325661,114.79692597)
\curveto(335.43324695,114.51691857)(335.22824715,114.23191885)(335.02825661,113.94192597)
\curveto(334.42824795,113.09191999)(333.82324856,112.25192083)(333.21325661,111.42192597)
\curveto(332.60324978,110.60192248)(331.99825038,109.76692332)(331.39825661,108.91692597)
\lineto(330.88825661,108.19692597)
\lineto(330.37825661,107.50692597)
\curveto(330.29825208,107.39692569)(330.21825216,107.2819258)(330.13825661,107.16192597)
\curveto(330.05825232,107.04192604)(329.96325242,106.94692614)(329.85325661,106.87692597)
\curveto(329.81325257,106.85692623)(329.74825263,106.84192624)(329.65825661,106.83192597)
\curveto(329.5782528,106.81192627)(329.48825289,106.80192628)(329.38825661,106.80192597)
\curveto(329.28825309,106.80192628)(329.19325319,106.80692628)(329.10325661,106.81692597)
\curveto(329.02325336,106.82692626)(328.96325342,106.84692624)(328.92325661,106.87692597)
\curveto(328.89325349,106.89692619)(328.86825351,106.93192615)(328.84825661,106.98192597)
\curveto(328.83825354,107.02192606)(328.84325354,107.06692602)(328.86325661,107.11692597)
\curveto(328.90325348,107.19692589)(328.94825343,107.27192581)(328.99825661,107.34192597)
\curveto(329.05825332,107.42192566)(329.11325327,107.50192558)(329.16325661,107.58192597)
\curveto(329.40325298,107.92192516)(329.64825273,108.25692483)(329.89825661,108.58692597)
\curveto(330.14825223,108.91692417)(330.38825199,109.25192383)(330.61825661,109.59192597)
\curveto(330.7782516,109.81192327)(330.93825144,110.02692306)(331.09825661,110.23692597)
\curveto(331.25825112,110.44692264)(331.41825096,110.66192242)(331.57825661,110.88192597)
\curveto(331.93825044,111.40192168)(332.30325008,111.91192117)(332.67325661,112.41192597)
\curveto(333.04324934,112.91192017)(333.41324897,113.42191966)(333.78325661,113.94192597)
\curveto(333.92324846,114.14191894)(334.06324832,114.33691875)(334.20325661,114.52692597)
\curveto(334.35324803,114.71691837)(334.49824788,114.91191817)(334.63825661,115.11192597)
\curveto(334.84824753,115.41191767)(335.06324732,115.71191737)(335.28325661,116.01192597)
\lineto(335.94325661,116.91192597)
\lineto(336.12325661,117.18192597)
\lineto(336.33325661,117.45192597)
\lineto(336.45325661,117.63192597)
\curveto(336.50324588,117.69191539)(336.55324583,117.74691534)(336.60325661,117.79692597)
\curveto(336.67324571,117.84691524)(336.74824563,117.8819152)(336.82825661,117.90192597)
\curveto(336.84824553,117.91191517)(336.87324551,117.91191517)(336.90325661,117.90192597)
\curveto(336.94324544,117.90191518)(336.97324541,117.91191517)(336.99325661,117.93192597)
\curveto(337.11324527,117.93191515)(337.24824513,117.92691516)(337.39825661,117.91692597)
\curveto(337.54824483,117.91691517)(337.63824474,117.87191521)(337.66825661,117.78192597)
\curveto(337.68824469,117.75191533)(337.69324469,117.71691537)(337.68325661,117.67692597)
\curveto(337.67324471,117.63691545)(337.65824472,117.60691548)(337.63825661,117.58692597)
\curveto(337.59824478,117.50691558)(337.55824482,117.43691565)(337.51825661,117.37692597)
\curveto(337.4782449,117.31691577)(337.43324495,117.25691583)(337.38325661,117.19692597)
\lineto(336.81325661,116.41692597)
\curveto(336.63324575,116.16691692)(336.45324593,115.91191717)(336.27325661,115.65192597)
\moveto(329.41825661,111.75192597)
\curveto(329.36825301,111.77192131)(329.31825306,111.77692131)(329.26825661,111.76692597)
\curveto(329.21825316,111.75692133)(329.16825321,111.76192132)(329.11825661,111.78192597)
\curveto(329.00825337,111.80192128)(328.90325348,111.82192126)(328.80325661,111.84192597)
\curveto(328.71325367,111.87192121)(328.61825376,111.91192117)(328.51825661,111.96192597)
\curveto(328.18825419,112.10192098)(327.93325445,112.29692079)(327.75325661,112.54692597)
\curveto(327.57325481,112.80692028)(327.42825495,113.11691997)(327.31825661,113.47692597)
\curveto(327.28825509,113.55691953)(327.26825511,113.63691945)(327.25825661,113.71692597)
\curveto(327.24825513,113.80691928)(327.23325515,113.89191919)(327.21325661,113.97192597)
\curveto(327.20325518,114.02191906)(327.19825518,114.086919)(327.19825661,114.16692597)
\curveto(327.18825519,114.19691889)(327.1832552,114.22691886)(327.18325661,114.25692597)
\curveto(327.1832552,114.29691879)(327.1782552,114.33191875)(327.16825661,114.36192597)
\lineto(327.16825661,114.51192597)
\curveto(327.15825522,114.56191852)(327.15325523,114.62191846)(327.15325661,114.69192597)
\curveto(327.15325523,114.77191831)(327.15825522,114.83691825)(327.16825661,114.88692597)
\lineto(327.16825661,115.05192597)
\curveto(327.18825519,115.10191798)(327.19325519,115.14691794)(327.18325661,115.18692597)
\curveto(327.1832552,115.23691785)(327.18825519,115.2819178)(327.19825661,115.32192597)
\curveto(327.20825517,115.36191772)(327.21325517,115.39691769)(327.21325661,115.42692597)
\curveto(327.21325517,115.46691762)(327.21825516,115.50691758)(327.22825661,115.54692597)
\curveto(327.25825512,115.65691743)(327.2782551,115.76691732)(327.28825661,115.87692597)
\curveto(327.30825507,115.99691709)(327.34325504,116.11191697)(327.39325661,116.22192597)
\curveto(327.53325485,116.56191652)(327.69325469,116.83691625)(327.87325661,117.04692597)
\curveto(328.06325432,117.26691582)(328.33325405,117.44691564)(328.68325661,117.58692597)
\curveto(328.76325362,117.61691547)(328.84825353,117.63691545)(328.93825661,117.64692597)
\curveto(329.02825335,117.66691542)(329.12325326,117.6869154)(329.22325661,117.70692597)
\curveto(329.25325313,117.71691537)(329.30825307,117.71691537)(329.38825661,117.70692597)
\curveto(329.46825291,117.70691538)(329.51825286,117.71691537)(329.53825661,117.73692597)
\curveto(330.09825228,117.74691534)(330.54825183,117.63691545)(330.88825661,117.40692597)
\curveto(331.23825114,117.17691591)(331.49825088,116.87191621)(331.66825661,116.49192597)
\curveto(331.70825067,116.40191668)(331.74325064,116.30691678)(331.77325661,116.20692597)
\curveto(331.80325058,116.10691698)(331.82825055,116.00691708)(331.84825661,115.90692597)
\curveto(331.86825051,115.87691721)(331.87325051,115.84691724)(331.86325661,115.81692597)
\curveto(331.86325052,115.7869173)(331.86825051,115.75691733)(331.87825661,115.72692597)
\curveto(331.90825047,115.61691747)(331.92825045,115.49191759)(331.93825661,115.35192597)
\curveto(331.94825043,115.22191786)(331.95825042,115.086918)(331.96825661,114.94692597)
\lineto(331.96825661,114.78192597)
\curveto(331.9782504,114.72191836)(331.9782504,114.66691842)(331.96825661,114.61692597)
\curveto(331.95825042,114.56691852)(331.95325043,114.51691857)(331.95325661,114.46692597)
\lineto(331.95325661,114.33192597)
\curveto(331.94325044,114.29191879)(331.93825044,114.25191883)(331.93825661,114.21192597)
\curveto(331.94825043,114.17191891)(331.94325044,114.12691896)(331.92325661,114.07692597)
\curveto(331.90325048,113.96691912)(331.8832505,113.86191922)(331.86325661,113.76192597)
\curveto(331.85325053,113.66191942)(331.83325055,113.56191952)(331.80325661,113.46192597)
\curveto(331.67325071,113.10191998)(331.50825087,112.7869203)(331.30825661,112.51692597)
\curveto(331.10825127,112.24692084)(330.83325155,112.04192104)(330.48325661,111.90192597)
\curveto(330.40325198,111.87192121)(330.31825206,111.84692124)(330.22825661,111.82692597)
\lineto(329.95825661,111.76692597)
\curveto(329.90825247,111.75692133)(329.86325252,111.75192133)(329.82325661,111.75192597)
\curveto(329.7832526,111.76192132)(329.74325264,111.76192132)(329.70325661,111.75192597)
\curveto(329.60325278,111.73192135)(329.50825287,111.73192135)(329.41825661,111.75192597)
\moveto(328.57825661,113.14692597)
\curveto(328.61825376,113.07692001)(328.65825372,113.01192007)(328.69825661,112.95192597)
\curveto(328.73825364,112.90192018)(328.78825359,112.85192023)(328.84825661,112.80192597)
\lineto(328.99825661,112.68192597)
\curveto(329.05825332,112.65192043)(329.12325326,112.62692046)(329.19325661,112.60692597)
\curveto(329.23325315,112.5869205)(329.26825311,112.57692051)(329.29825661,112.57692597)
\curveto(329.33825304,112.5869205)(329.378253,112.5819205)(329.41825661,112.56192597)
\curveto(329.44825293,112.56192052)(329.48825289,112.55692053)(329.53825661,112.54692597)
\curveto(329.58825279,112.54692054)(329.62825275,112.55192053)(329.65825661,112.56192597)
\lineto(329.88325661,112.60692597)
\curveto(330.13325225,112.6869204)(330.31825206,112.81192027)(330.43825661,112.98192597)
\curveto(330.51825186,113.08192)(330.58825179,113.21191987)(330.64825661,113.37192597)
\curveto(330.72825165,113.55191953)(330.78825159,113.77691931)(330.82825661,114.04692597)
\curveto(330.86825151,114.32691876)(330.8832515,114.60691848)(330.87325661,114.88692597)
\curveto(330.86325152,115.17691791)(330.83325155,115.45191763)(330.78325661,115.71192597)
\curveto(330.73325165,115.97191711)(330.65825172,116.1819169)(330.55825661,116.34192597)
\curveto(330.43825194,116.54191654)(330.28825209,116.69191639)(330.10825661,116.79192597)
\curveto(330.02825235,116.84191624)(329.93825244,116.87191621)(329.83825661,116.88192597)
\curveto(329.73825264,116.90191618)(329.63325275,116.91191617)(329.52325661,116.91192597)
\curveto(329.50325288,116.90191618)(329.4782529,116.89691619)(329.44825661,116.89692597)
\curveto(329.42825295,116.90691618)(329.40825297,116.90691618)(329.38825661,116.89692597)
\curveto(329.33825304,116.8869162)(329.29325309,116.87691621)(329.25325661,116.86692597)
\curveto(329.21325317,116.86691622)(329.17325321,116.85691623)(329.13325661,116.83692597)
\curveto(328.95325343,116.75691633)(328.80325358,116.63691645)(328.68325661,116.47692597)
\curveto(328.57325381,116.31691677)(328.4832539,116.13691695)(328.41325661,115.93692597)
\curveto(328.35325403,115.74691734)(328.30825407,115.52191756)(328.27825661,115.26192597)
\curveto(328.25825412,115.00191808)(328.25325413,114.73691835)(328.26325661,114.46692597)
\curveto(328.27325411,114.20691888)(328.30325408,113.95691913)(328.35325661,113.71692597)
\curveto(328.41325397,113.4869196)(328.48825389,113.29691979)(328.57825661,113.14692597)
\moveto(339.37825661,110.16192597)
\curveto(339.38824299,110.11192297)(339.39324299,110.02192306)(339.39325661,109.89192597)
\curveto(339.39324299,109.76192332)(339.383243,109.67192341)(339.36325661,109.62192597)
\curveto(339.34324304,109.57192351)(339.33824304,109.51692357)(339.34825661,109.45692597)
\curveto(339.35824302,109.40692368)(339.35824302,109.35692373)(339.34825661,109.30692597)
\curveto(339.30824307,109.16692392)(339.2782431,109.03192405)(339.25825661,108.90192597)
\curveto(339.24824313,108.77192431)(339.21824316,108.65192443)(339.16825661,108.54192597)
\curveto(339.02824335,108.19192489)(338.86324352,107.89692519)(338.67325661,107.65692597)
\curveto(338.4832439,107.42692566)(338.21324417,107.24192584)(337.86325661,107.10192597)
\curveto(337.7832446,107.07192601)(337.69824468,107.05192603)(337.60825661,107.04192597)
\curveto(337.51824486,107.02192606)(337.43324495,107.00192608)(337.35325661,106.98192597)
\curveto(337.30324508,106.97192611)(337.25324513,106.96692612)(337.20325661,106.96692597)
\curveto(337.15324523,106.96692612)(337.10324528,106.96192612)(337.05325661,106.95192597)
\curveto(337.02324536,106.94192614)(336.97324541,106.94192614)(336.90325661,106.95192597)
\curveto(336.83324555,106.95192613)(336.7832456,106.95692613)(336.75325661,106.96692597)
\curveto(336.69324569,106.9869261)(336.63324575,106.99692609)(336.57325661,106.99692597)
\curveto(336.52324586,106.9869261)(336.47324591,106.99192609)(336.42325661,107.01192597)
\curveto(336.33324605,107.03192605)(336.24324614,107.05692603)(336.15325661,107.08692597)
\curveto(336.07324631,107.10692598)(335.99324639,107.13692595)(335.91325661,107.17692597)
\curveto(335.59324679,107.31692577)(335.34324704,107.51192557)(335.16325661,107.76192597)
\curveto(334.9832474,108.02192506)(334.83324755,108.32692476)(334.71325661,108.67692597)
\curveto(334.69324769,108.75692433)(334.6782477,108.84192424)(334.66825661,108.93192597)
\curveto(334.65824772,109.02192406)(334.64324774,109.10692398)(334.62325661,109.18692597)
\curveto(334.61324777,109.21692387)(334.60824777,109.24692384)(334.60825661,109.27692597)
\lineto(334.60825661,109.38192597)
\curveto(334.58824779,109.46192362)(334.5782478,109.54192354)(334.57825661,109.62192597)
\lineto(334.57825661,109.75692597)
\curveto(334.55824782,109.85692323)(334.55824782,109.95692313)(334.57825661,110.05692597)
\lineto(334.57825661,110.23692597)
\curveto(334.58824779,110.2869228)(334.59324779,110.33192275)(334.59325661,110.37192597)
\curveto(334.59324779,110.42192266)(334.59824778,110.46692262)(334.60825661,110.50692597)
\curveto(334.61824776,110.54692254)(334.62324776,110.5819225)(334.62325661,110.61192597)
\curveto(334.62324776,110.65192243)(334.62824775,110.69192239)(334.63825661,110.73192597)
\lineto(334.69825661,111.06192597)
\curveto(334.71824766,111.1819219)(334.74824763,111.29192179)(334.78825661,111.39192597)
\curveto(334.92824745,111.72192136)(335.08824729,111.99692109)(335.26825661,112.21692597)
\curveto(335.45824692,112.44692064)(335.71824666,112.63192045)(336.04825661,112.77192597)
\curveto(336.12824625,112.81192027)(336.21324617,112.83692025)(336.30325661,112.84692597)
\lineto(336.60325661,112.90692597)
\lineto(336.73825661,112.90692597)
\curveto(336.78824559,112.91692017)(336.83824554,112.92192016)(336.88825661,112.92192597)
\curveto(337.45824492,112.94192014)(337.91824446,112.83692025)(338.26825661,112.60692597)
\curveto(338.62824375,112.3869207)(338.89324349,112.086921)(339.06325661,111.70692597)
\curveto(339.11324327,111.60692148)(339.15324323,111.50692158)(339.18325661,111.40692597)
\curveto(339.21324317,111.30692178)(339.24324314,111.20192188)(339.27325661,111.09192597)
\curveto(339.2832431,111.05192203)(339.28824309,111.01692207)(339.28825661,110.98692597)
\curveto(339.28824309,110.96692212)(339.29324309,110.93692215)(339.30325661,110.89692597)
\curveto(339.32324306,110.82692226)(339.33324305,110.75192233)(339.33325661,110.67192597)
\curveto(339.33324305,110.59192249)(339.34324304,110.51192257)(339.36325661,110.43192597)
\curveto(339.36324302,110.3819227)(339.36324302,110.33692275)(339.36325661,110.29692597)
\curveto(339.36324302,110.25692283)(339.36824301,110.21192287)(339.37825661,110.16192597)
\moveto(338.26825661,109.72692597)
\curveto(338.2782441,109.77692331)(338.2832441,109.85192323)(338.28325661,109.95192597)
\curveto(338.29324409,110.05192303)(338.28824409,110.12692296)(338.26825661,110.17692597)
\curveto(338.24824413,110.23692285)(338.24324414,110.29192279)(338.25325661,110.34192597)
\curveto(338.27324411,110.40192268)(338.27324411,110.46192262)(338.25325661,110.52192597)
\curveto(338.24324414,110.55192253)(338.23824414,110.5869225)(338.23825661,110.62692597)
\curveto(338.23824414,110.66692242)(338.23324415,110.70692238)(338.22325661,110.74692597)
\curveto(338.20324418,110.82692226)(338.1832442,110.90192218)(338.16325661,110.97192597)
\curveto(338.15324423,111.05192203)(338.13824424,111.13192195)(338.11825661,111.21192597)
\curveto(338.08824429,111.27192181)(338.06324432,111.33192175)(338.04325661,111.39192597)
\curveto(338.02324436,111.45192163)(337.99324439,111.51192157)(337.95325661,111.57192597)
\curveto(337.85324453,111.74192134)(337.72324466,111.87692121)(337.56325661,111.97692597)
\curveto(337.4832449,112.02692106)(337.38824499,112.06192102)(337.27825661,112.08192597)
\curveto(337.16824521,112.10192098)(337.04324534,112.11192097)(336.90325661,112.11192597)
\curveto(336.8832455,112.10192098)(336.85824552,112.09692099)(336.82825661,112.09692597)
\curveto(336.79824558,112.10692098)(336.76824561,112.10692098)(336.73825661,112.09692597)
\lineto(336.58825661,112.03692597)
\curveto(336.53824584,112.02692106)(336.49324589,112.01192107)(336.45325661,111.99192597)
\curveto(336.26324612,111.8819212)(336.11824626,111.73692135)(336.01825661,111.55692597)
\curveto(335.92824645,111.37692171)(335.84824653,111.17192191)(335.77825661,110.94192597)
\curveto(335.73824664,110.81192227)(335.71824666,110.67692241)(335.71825661,110.53692597)
\curveto(335.71824666,110.40692268)(335.70824667,110.26192282)(335.68825661,110.10192597)
\curveto(335.6782467,110.05192303)(335.66824671,109.99192309)(335.65825661,109.92192597)
\curveto(335.65824672,109.85192323)(335.66824671,109.79192329)(335.68825661,109.74192597)
\lineto(335.68825661,109.57692597)
\lineto(335.68825661,109.39692597)
\curveto(335.69824668,109.34692374)(335.70824667,109.29192379)(335.71825661,109.23192597)
\curveto(335.72824665,109.1819239)(335.73324665,109.12692396)(335.73325661,109.06692597)
\curveto(335.74324664,109.00692408)(335.75824662,108.95192413)(335.77825661,108.90192597)
\curveto(335.82824655,108.71192437)(335.88824649,108.53692455)(335.95825661,108.37692597)
\curveto(336.02824635,108.21692487)(336.13324625,108.086925)(336.27325661,107.98692597)
\curveto(336.40324598,107.8869252)(336.54324584,107.81692527)(336.69325661,107.77692597)
\curveto(336.72324566,107.76692532)(336.74824563,107.76192532)(336.76825661,107.76192597)
\curveto(336.79824558,107.77192531)(336.82824555,107.77192531)(336.85825661,107.76192597)
\curveto(336.8782455,107.76192532)(336.90824547,107.75692533)(336.94825661,107.74692597)
\curveto(336.98824539,107.74692534)(337.02324536,107.75192533)(337.05325661,107.76192597)
\curveto(337.09324529,107.77192531)(337.13324525,107.77692531)(337.17325661,107.77692597)
\curveto(337.21324517,107.77692531)(337.25324513,107.7869253)(337.29325661,107.80692597)
\curveto(337.53324485,107.8869252)(337.72824465,108.02192506)(337.87825661,108.21192597)
\curveto(337.99824438,108.39192469)(338.08824429,108.59692449)(338.14825661,108.82692597)
\curveto(338.16824421,108.89692419)(338.1832442,108.96692412)(338.19325661,109.03692597)
\curveto(338.20324418,109.11692397)(338.21824416,109.19692389)(338.23825661,109.27692597)
\curveto(338.23824414,109.33692375)(338.24324414,109.3819237)(338.25325661,109.41192597)
\curveto(338.25324413,109.43192365)(338.25324413,109.45692363)(338.25325661,109.48692597)
\curveto(338.25324413,109.52692356)(338.25824412,109.55692353)(338.26825661,109.57692597)
\lineto(338.26825661,109.72692597)
}
}
\end{pspicture}

\caption{Gráficas circulares de los diferentes niveles de intención de los
usuarios clasificados por rol}
\label{usuarios_pie_1}
\end{figure}

\subsection{Valoraciones}
El segundo aspecto a analizar esta referido a las valoraciones que poseen los
usuarios del sistema, puede verse en el cuadro \ref{usuarios_tabla_2}, el conteo
de la puntuación por cada rol establecido en el sistema, según el tipo de
valoración.

\begin{table}
\centering
\begin{tabular}{l|c c c c}
$Rol$ & $Actividad$ & $Participacion$ & $Sociabilidad$ & $Popularidad$ \\
\hline
$Invitado     $ & $ 2$ & $ 0$ & $25$ & $ 1$ \\
$Estudiante   $ & $ 3$ & $10$ & $58$ & $ 0$ \\
$Auxiliar     $ & $ 7$ & $ 0$ & $12$ & $ 0$ \\
$Docente      $ & $ 4$ & $ 0$ & $ 6$ & $ 0$ \\
$Moderador    $ & $27$ & $ 0$ & $ 5$ & $ 3$ \\
$Desarrollador$ & $61$ & $54$ & $96$ & $19$ \\
$Administrador$ & $ 0$ & $ 2$ & $10$ & $ 0$ \\
\end{tabular}
\caption{Valoraciones de los usuarios clasificados por rol}
\label{usuarios_tabla_2}
\end{table}

La figura \ref{usuarios_bars_2}, muestra el diagrama de barras referente al
cuadro \ref{usuarios_tabla_2}, puede notarse una gran diferencia entre las
variables \emph{actividad} y \emph{sociabilidad}, en contraste con las variables
\emph{participación} y \emph{popularidad}.

\begin{figure}
\centering
%LaTeX with PSTricks extensions
%%Creator: inkscape 0.48.5
%%Please note this file requires PSTricks extensions
\psset{xunit=.5pt,yunit=.5pt,runit=.5pt}
\begin{pspicture}(856,429)
{
\newrgbcolor{curcolor}{0 0 0}
\pscustom[linestyle=none,fillstyle=solid,fillcolor=curcolor]
{
\newpath
\moveto(365.45475342,12.82530273)
\curveto(365.48474374,12.69530247)(365.44474378,12.59530257)(365.33475342,12.52530273)
\curveto(365.28474394,12.49530267)(365.21974401,12.47530269)(365.13975342,12.46530273)
\lineto(364.89975342,12.46530273)
\lineto(364.41975342,12.46530273)
\curveto(364.25974497,12.4653027)(364.14474508,12.50030267)(364.07475342,12.57030273)
\curveto(364.00474522,12.62030255)(363.96474526,12.69530247)(363.95475342,12.79530273)
\lineto(363.95475342,13.12530273)
\lineto(363.95475342,13.23030273)
\curveto(363.96474526,13.2703019)(363.97474525,13.30530186)(363.98475342,13.33530273)
\curveto(363.97474525,13.38530178)(363.97974525,13.43030174)(363.99975342,13.47030273)
\curveto(364.01974521,13.51030166)(364.0247452,13.55030162)(364.01475342,13.59030273)
\lineto(364.04475342,13.77030273)
\lineto(364.07475342,13.95030273)
\lineto(364.16475342,14.62530273)
\curveto(364.16474506,14.69530047)(364.16974506,14.7653004)(364.17975342,14.83530273)
\curveto(364.18974504,14.90530026)(364.19474503,14.98030019)(364.19475342,15.06030273)
\curveto(364.18474504,15.24029993)(364.18474504,15.42029975)(364.19475342,15.60030273)
\curveto(364.20474502,15.78029939)(364.18974504,15.95029922)(364.14975342,16.11030273)
\curveto(364.03974519,16.53029864)(363.77974545,16.81029836)(363.36975342,16.95030273)
\curveto(363.24974598,17.00029817)(363.10974612,17.02529814)(362.94975342,17.02530273)
\curveto(362.79974643,17.03529813)(362.63974659,17.04029813)(362.46975342,17.04030273)
\lineto(359.70975342,17.04030273)
\curveto(359.63974959,17.02029815)(359.57474965,17.00029817)(359.51475342,16.98030273)
\curveto(359.45474977,16.9702982)(359.39974983,16.94029823)(359.34975342,16.89030273)
\curveto(359.25974997,16.79029838)(359.19475003,16.62529854)(359.15475342,16.39530273)
\curveto(359.11475011,16.17529899)(359.07975015,15.98029919)(359.04975342,15.81030273)
\lineto(358.61475342,13.63530273)
\curveto(358.58475064,13.49530167)(358.55475067,13.32030185)(358.52475342,13.11030273)
\curveto(358.49475073,12.91030226)(358.44475078,12.76030241)(358.37475342,12.66030273)
\curveto(358.34475088,12.59030258)(358.29475093,12.54530262)(358.22475342,12.52530273)
\curveto(358.18475104,12.50530266)(358.14475108,12.49530267)(358.10475342,12.49530273)
\curveto(358.07475115,12.49530267)(358.0297512,12.48530268)(357.96975342,12.46530273)
\curveto(357.9297513,12.45530271)(357.88475134,12.45030272)(357.83475342,12.45030273)
\curveto(357.78475144,12.46030271)(357.73475149,12.4653027)(357.68475342,12.46530273)
\lineto(357.36975342,12.46530273)
\curveto(357.26975196,12.47530269)(357.18975204,12.50530266)(357.12975342,12.55530273)
\curveto(357.05975217,12.60530256)(357.03475219,12.69530247)(357.05475342,12.82530273)
\curveto(357.08475214,12.9653022)(357.11475211,13.10030207)(357.14475342,13.23030273)
\lineto(358.95975342,22.35030273)
\curveto(358.97975025,22.46029271)(358.99975023,22.57529259)(359.01975342,22.69530273)
\curveto(359.03975019,22.81529235)(359.08475014,22.91029226)(359.15475342,22.98030273)
\curveto(359.20475002,23.04029213)(359.28974994,23.09029208)(359.40975342,23.13030273)
\curveto(359.4297498,23.14029203)(359.44974978,23.14029203)(359.46975342,23.13030273)
\curveto(359.48974974,23.13029204)(359.50974972,23.13529203)(359.52975342,23.14530273)
\lineto(363.87975342,23.14530273)
\curveto(363.94974528,23.14529202)(364.0247452,23.14529202)(364.10475342,23.14530273)
\curveto(364.18474504,23.15529201)(364.25474497,23.15529201)(364.31475342,23.14530273)
\lineto(364.47975342,23.14530273)
\curveto(364.53974469,23.13529203)(364.59474463,23.12529204)(364.64475342,23.11530273)
\curveto(364.70474452,23.11529205)(364.76974446,23.11029206)(364.83975342,23.10030273)
\curveto(364.91974431,23.08029209)(364.99974423,23.0652921)(365.07975342,23.05530273)
\curveto(365.15974407,23.04529212)(365.23974399,23.03029214)(365.31975342,23.01030273)
\curveto(365.49974373,22.95029222)(365.65974357,22.88529228)(365.79975342,22.81530273)
\curveto(365.94974328,22.74529242)(366.08474314,22.66029251)(366.20475342,22.56030273)
\curveto(366.4247428,22.39029278)(366.58474264,22.18029299)(366.68475342,21.93030273)
\curveto(366.79474243,21.69029348)(366.86474236,21.40529376)(366.89475342,21.07530273)
\curveto(366.90474232,20.99529417)(366.89974233,20.91029426)(366.87975342,20.82030273)
\curveto(366.86974236,20.74029443)(366.86974236,20.66029451)(366.87975342,20.58030273)
\lineto(366.84975342,20.43030273)
\curveto(366.84974238,20.38029479)(366.83974239,20.32029485)(366.81975342,20.25030273)
\curveto(366.79974243,20.19029498)(366.77974245,20.13529503)(366.75975342,20.08530273)
\lineto(366.72975342,19.92030273)
\curveto(366.68974254,19.84029533)(366.65974257,19.7652954)(366.63975342,19.69530273)
\curveto(366.61974261,19.62529554)(366.59474263,19.55529561)(366.56475342,19.48530273)
\curveto(366.48474274,19.33529583)(366.40974282,19.19029598)(366.33975342,19.05030273)
\curveto(366.26974296,18.92029625)(366.17974305,18.79529637)(366.06975342,18.67530273)
\curveto(366.0297432,18.62529654)(365.98974324,18.58029659)(365.94975342,18.54030273)
\curveto(365.90974332,18.50029667)(365.86974336,18.45529671)(365.82975342,18.40530273)
\curveto(365.80974342,18.39529677)(365.79474343,18.38529678)(365.78475342,18.37530273)
\curveto(365.77474345,18.37529679)(365.76474346,18.3702968)(365.75475342,18.36030273)
\curveto(365.73474349,18.34029683)(365.70474352,18.31529685)(365.66475342,18.28530273)
\lineto(365.58975342,18.21030273)
\curveto(365.49974373,18.15029702)(365.41474381,18.09029708)(365.33475342,18.03030273)
\curveto(365.25474397,17.98029719)(365.16974406,17.93029724)(365.07975342,17.88030273)
\curveto(365.01974421,17.85029732)(364.96474426,17.81529735)(364.91475342,17.77530273)
\curveto(364.87474435,17.74529742)(364.83974439,17.70029747)(364.80975342,17.64030273)
\curveto(364.78974444,17.58029759)(364.80474442,17.53029764)(364.85475342,17.49030273)
\curveto(364.90474432,17.45029772)(364.94474428,17.42029775)(364.97475342,17.40030273)
\curveto(365.07474415,17.33029784)(365.16474406,17.25529791)(365.24475342,17.17530273)
\curveto(365.3247439,17.09529807)(365.38474384,17.00029817)(365.42475342,16.89030273)
\curveto(365.51474371,16.75029842)(365.56474366,16.59029858)(365.57475342,16.41030273)
\curveto(365.59474363,16.24029893)(365.60974362,16.05529911)(365.61975342,15.85530273)
\lineto(365.58975342,15.61530273)
\curveto(365.58974364,15.54529962)(365.58474364,15.4702997)(365.57475342,15.39030273)
\curveto(365.58474364,15.32029985)(365.57474365,15.25029992)(365.54475342,15.18030273)
\curveto(365.5247437,15.11030006)(365.51974371,15.04030013)(365.52975342,14.97030273)
\lineto(365.49975342,14.83530273)
\curveto(365.50974372,14.7653004)(365.49974373,14.69030048)(365.46975342,14.61030273)
\curveto(365.43974379,14.53030064)(365.4297438,14.45030072)(365.43975342,14.37030273)
\curveto(365.43974379,14.33030084)(365.43474379,14.29030088)(365.42475342,14.25030273)
\curveto(365.41474381,14.22030095)(365.40974382,14.18030099)(365.40975342,14.13030273)
\curveto(365.40974382,14.03030114)(365.39974383,13.92530124)(365.37975342,13.81530273)
\curveto(365.36974386,13.71530145)(365.37474385,13.62030155)(365.39475342,13.53030273)
\curveto(365.39474383,13.4703017)(365.38974384,13.41030176)(365.37975342,13.35030273)
\curveto(365.37974385,13.30030187)(365.38474384,13.24530192)(365.39475342,13.18530273)
\lineto(365.45475342,12.82530273)
\moveto(364.88475342,19.05030273)
\curveto(364.97474425,19.16029601)(365.04974418,19.27529589)(365.10975342,19.39530273)
\curveto(365.16974406,19.51529565)(365.229744,19.64529552)(365.28975342,19.78530273)
\lineto(365.31975342,19.92030273)
\curveto(365.37974385,20.06029511)(365.41474381,20.21029496)(365.42475342,20.37030273)
\curveto(365.43474379,20.54029463)(365.4297438,20.68029449)(365.40975342,20.79030273)
\curveto(365.34974388,21.29029388)(365.10474412,21.63529353)(364.67475342,21.82530273)
\curveto(364.49474473,21.90529326)(364.26974496,21.95029322)(363.99975342,21.96030273)
\curveto(363.73974549,21.9702932)(363.46974576,21.97529319)(363.18975342,21.97530273)
\lineto(360.71475342,21.97530273)
\curveto(360.69474853,21.9652932)(360.66974856,21.96029321)(360.63975342,21.96030273)
\curveto(360.61974861,21.96029321)(360.59474863,21.95529321)(360.56475342,21.94530273)
\curveto(360.43474879,21.91529325)(360.33974889,21.85029332)(360.27975342,21.75030273)
\curveto(360.21974901,21.66029351)(360.17474905,21.53529363)(360.14475342,21.37530273)
\curveto(360.1247491,21.21529395)(360.09974913,21.0702941)(360.06975342,20.94030273)
\lineto(359.72475342,19.21530273)
\curveto(359.69474953,19.0652961)(359.65974957,18.90529626)(359.61975342,18.73530273)
\curveto(359.58974964,18.57529659)(359.59974963,18.45029672)(359.64975342,18.36030273)
\curveto(359.68974954,18.29029688)(359.75474947,18.24529692)(359.84475342,18.22530273)
\curveto(359.94474928,18.21529695)(360.05474917,18.21029696)(360.17475342,18.21030273)
\lineto(361.10475342,18.21030273)
\curveto(361.49474773,18.21029696)(361.87474735,18.20529696)(362.24475342,18.19530273)
\curveto(362.61474661,18.19529697)(362.95974627,18.21529695)(363.27975342,18.25530273)
\curveto(363.60974562,18.30529686)(363.90974532,18.39029678)(364.17975342,18.51030273)
\curveto(364.44974478,18.63029654)(364.68474454,18.81029636)(364.88475342,19.05030273)
}
}
{
\newrgbcolor{curcolor}{0 0 0}
\pscustom[linestyle=none,fillstyle=solid,fillcolor=curcolor]
{
\newpath
\moveto(374.97295654,16.65030273)
\curveto(374.98294765,16.59029858)(374.97294766,16.49529867)(374.94295654,16.36530273)
\curveto(374.92294771,16.24529892)(374.90294773,16.16029901)(374.88295654,16.11030273)
\lineto(374.85295654,15.96030273)
\curveto(374.82294781,15.88029929)(374.79794784,15.80529936)(374.77795654,15.73530273)
\curveto(374.76794787,15.67529949)(374.74794789,15.60529956)(374.71795654,15.52530273)
\curveto(374.68794795,15.4652997)(374.66294797,15.40529976)(374.64295654,15.34530273)
\curveto(374.632948,15.28529988)(374.60794803,15.22529994)(374.56795654,15.16530273)
\lineto(374.38795654,14.77530273)
\curveto(374.3379483,14.64530052)(374.27294836,14.52530064)(374.19295654,14.41530273)
\curveto(373.89294874,13.93530123)(373.5329491,13.53030164)(373.11295654,13.20030273)
\curveto(372.70294993,12.88030229)(372.22295041,12.63530253)(371.67295654,12.46530273)
\curveto(371.56295107,12.42530274)(371.44295119,12.39530277)(371.31295654,12.37530273)
\curveto(371.18295145,12.35530281)(371.04795159,12.33530283)(370.90795654,12.31530273)
\curveto(370.84795179,12.30530286)(370.78295185,12.30030287)(370.71295654,12.30030273)
\curveto(370.65295198,12.29030288)(370.59295204,12.28530288)(370.53295654,12.28530273)
\curveto(370.49295214,12.27530289)(370.4329522,12.2703029)(370.35295654,12.27030273)
\curveto(370.28295235,12.2703029)(370.2329524,12.27530289)(370.20295654,12.28530273)
\curveto(370.16295247,12.29530287)(370.12295251,12.30030287)(370.08295654,12.30030273)
\curveto(370.04295259,12.29030288)(370.00795263,12.29030288)(369.97795654,12.30030273)
\lineto(369.88795654,12.30030273)
\lineto(369.54295654,12.34530273)
\lineto(369.15295654,12.46530273)
\curveto(369.0329536,12.50530266)(368.91795372,12.55030262)(368.80795654,12.60030273)
\curveto(368.39795424,12.80030237)(368.07795456,13.06030211)(367.84795654,13.38030273)
\curveto(367.62795501,13.70030147)(367.46795517,14.09030108)(367.36795654,14.55030273)
\curveto(367.3379553,14.65030052)(367.31795532,14.75030042)(367.30795654,14.85030273)
\lineto(367.30795654,15.16530273)
\curveto(367.29795534,15.20529996)(367.29795534,15.23529993)(367.30795654,15.25530273)
\curveto(367.31795532,15.28529988)(367.32295531,15.32029985)(367.32295654,15.36030273)
\curveto(367.32295531,15.44029973)(367.32795531,15.52029965)(367.33795654,15.60030273)
\curveto(367.34795529,15.69029948)(367.35295528,15.77529939)(367.35295654,15.85530273)
\curveto(367.36295527,15.90529926)(367.36795527,15.94529922)(367.36795654,15.97530273)
\curveto(367.37795526,16.01529915)(367.38295525,16.06029911)(367.38295654,16.11030273)
\curveto(367.38295525,16.16029901)(367.39295524,16.24529892)(367.41295654,16.36530273)
\curveto(367.44295519,16.49529867)(367.47295516,16.59029858)(367.50295654,16.65030273)
\curveto(367.54295509,16.72029845)(367.56295507,16.79029838)(367.56295654,16.86030273)
\curveto(367.56295507,16.93029824)(367.58295505,17.00029817)(367.62295654,17.07030273)
\curveto(367.64295499,17.12029805)(367.65795498,17.16029801)(367.66795654,17.19030273)
\curveto(367.67795496,17.23029794)(367.69295494,17.27529789)(367.71295654,17.32530273)
\curveto(367.77295486,17.44529772)(367.82295481,17.5652976)(367.86295654,17.68530273)
\curveto(367.91295472,17.80529736)(367.97795466,17.92029725)(368.05795654,18.03030273)
\curveto(368.27795436,18.40029677)(368.52295411,18.73029644)(368.79295654,19.02030273)
\curveto(369.07295356,19.32029585)(369.38795325,19.5702956)(369.73795654,19.77030273)
\curveto(369.86795277,19.85029532)(370.00295263,19.91529525)(370.14295654,19.96530273)
\lineto(370.59295654,20.14530273)
\curveto(370.72295191,20.19529497)(370.85795178,20.22529494)(370.99795654,20.23530273)
\curveto(371.1379515,20.25529491)(371.28295135,20.28529488)(371.43295654,20.32530273)
\lineto(371.62795654,20.32530273)
\lineto(371.83795654,20.35530273)
\curveto(372.72794991,20.3652948)(373.42794921,20.18029499)(373.93795654,19.80030273)
\curveto(374.45794818,19.42029575)(374.78294785,18.92529624)(374.91295654,18.31530273)
\curveto(374.94294769,18.21529695)(374.96294767,18.11529705)(374.97295654,18.01530273)
\curveto(374.98294765,17.91529725)(374.99794764,17.81029736)(375.01795654,17.70030273)
\curveto(375.02794761,17.59029758)(375.02794761,17.4702977)(375.01795654,17.34030273)
\lineto(375.01795654,16.96530273)
\curveto(375.01794762,16.91529825)(375.00794763,16.86029831)(374.98795654,16.80030273)
\curveto(374.97794766,16.75029842)(374.97294766,16.70029847)(374.97295654,16.65030273)
\moveto(373.47295654,15.79530273)
\curveto(373.50294913,15.8652993)(373.52294911,15.94529922)(373.53295654,16.03530273)
\curveto(373.55294908,16.12529904)(373.56794907,16.21029896)(373.57795654,16.29030273)
\curveto(373.65794898,16.68029849)(373.69294894,17.01029816)(373.68295654,17.28030273)
\curveto(373.66294897,17.36029781)(373.64794899,17.44029773)(373.63795654,17.52030273)
\curveto(373.637949,17.60029757)(373.632949,17.67529749)(373.62295654,17.74530273)
\curveto(373.47294916,18.39529677)(373.11794952,18.84529632)(372.55795654,19.09530273)
\curveto(372.48795015,19.12529604)(372.41295022,19.14529602)(372.33295654,19.15530273)
\curveto(372.26295037,19.17529599)(372.18795045,19.19529597)(372.10795654,19.21530273)
\curveto(372.0379506,19.23529593)(371.95795068,19.24529592)(371.86795654,19.24530273)
\lineto(371.59795654,19.24530273)
\lineto(371.31295654,19.20030273)
\curveto(371.21295142,19.18029599)(371.11795152,19.15529601)(371.02795654,19.12530273)
\curveto(370.9379517,19.10529606)(370.84795179,19.07529609)(370.75795654,19.03530273)
\curveto(370.68795195,19.01529615)(370.61795202,18.98529618)(370.54795654,18.94530273)
\curveto(370.47795216,18.90529626)(370.41295222,18.8652963)(370.35295654,18.82530273)
\curveto(370.08295255,18.65529651)(369.84795279,18.45029672)(369.64795654,18.21030273)
\curveto(369.44795319,17.9702972)(369.26295337,17.69029748)(369.09295654,17.37030273)
\curveto(369.04295359,17.2702979)(369.00295363,17.165298)(368.97295654,17.05530273)
\curveto(368.94295369,16.95529821)(368.90295373,16.85029832)(368.85295654,16.74030273)
\curveto(368.84295379,16.70029847)(368.82795381,16.63529853)(368.80795654,16.54530273)
\curveto(368.78795385,16.51529865)(368.77795386,16.48029869)(368.77795654,16.44030273)
\curveto(368.77795386,16.40029877)(368.77295386,16.35529881)(368.76295654,16.30530273)
\lineto(368.70295654,16.00530273)
\curveto(368.68295395,15.90529926)(368.67295396,15.81529935)(368.67295654,15.73530273)
\lineto(368.67295654,15.55530273)
\curveto(368.67295396,15.45529971)(368.66795397,15.35529981)(368.65795654,15.25530273)
\curveto(368.65795398,15.1653)(368.66795397,15.08030009)(368.68795654,15.00030273)
\curveto(368.7379539,14.76030041)(368.80795383,14.53530063)(368.89795654,14.32530273)
\curveto(368.99795364,14.11530105)(369.1329535,13.94030123)(369.30295654,13.80030273)
\curveto(369.35295328,13.7703014)(369.39295324,13.74530142)(369.42295654,13.72530273)
\curveto(369.46295317,13.70530146)(369.50295313,13.68030149)(369.54295654,13.65030273)
\curveto(369.61295302,13.60030157)(369.69295294,13.55530161)(369.78295654,13.51530273)
\curveto(369.87295276,13.48530168)(369.96795267,13.45530171)(370.06795654,13.42530273)
\curveto(370.11795252,13.40530176)(370.16295247,13.39530177)(370.20295654,13.39530273)
\curveto(370.25295238,13.40530176)(370.30295233,13.40530176)(370.35295654,13.39530273)
\curveto(370.38295225,13.38530178)(370.44295219,13.37530179)(370.53295654,13.36530273)
\curveto(370.62295201,13.35530181)(370.69795194,13.36030181)(370.75795654,13.38030273)
\curveto(370.79795184,13.39030178)(370.8379518,13.39030178)(370.87795654,13.38030273)
\curveto(370.91795172,13.38030179)(370.95795168,13.39030178)(370.99795654,13.41030273)
\curveto(371.07795156,13.43030174)(371.15795148,13.44530172)(371.23795654,13.45530273)
\curveto(371.32795131,13.47530169)(371.41295122,13.50030167)(371.49295654,13.53030273)
\curveto(371.85295078,13.6703015)(372.16295047,13.8653013)(372.42295654,14.11530273)
\curveto(372.68294995,14.3653008)(372.91794972,14.66030051)(373.12795654,15.00030273)
\curveto(373.20794943,15.12030005)(373.26794937,15.24529992)(373.30795654,15.37530273)
\curveto(373.34794929,15.51529965)(373.40294923,15.65529951)(373.47295654,15.79530273)
}
}
{
\newrgbcolor{curcolor}{0 0 0}
\pscustom[linestyle=none,fillstyle=solid,fillcolor=curcolor]
{
\newpath
\moveto(378.33623779,23.13030273)
\curveto(378.46623403,23.13029204)(378.6012339,23.13029204)(378.74123779,23.13030273)
\curveto(378.89123361,23.13029204)(378.99123351,23.09529207)(379.04123779,23.02530273)
\curveto(379.08123342,22.95529221)(379.09123341,22.86029231)(379.07123779,22.74030273)
\curveto(379.05123345,22.63029254)(379.03123347,22.51529265)(379.01123779,22.39530273)
\lineto(378.74123779,21.06030273)
\lineto(377.52623779,14.98530273)
\lineto(377.19623779,13.30530273)
\curveto(377.16623533,13.18530198)(377.13623536,13.05530211)(377.10623779,12.91530273)
\curveto(377.08623541,12.77530239)(377.04123546,12.6653025)(376.97123779,12.58530273)
\curveto(376.93123557,12.53530263)(376.88123562,12.50530266)(376.82123779,12.49530273)
\curveto(376.77123573,12.48530268)(376.7012358,12.4703027)(376.61123779,12.45030273)
\lineto(376.40123779,12.45030273)
\lineto(376.08623779,12.45030273)
\curveto(375.98623651,12.46030271)(375.92123658,12.49530267)(375.89123779,12.55530273)
\curveto(375.85123665,12.63530253)(375.84123666,12.73530243)(375.86123779,12.85530273)
\curveto(375.88123662,12.97530219)(375.90623659,13.10030207)(375.93623779,13.23030273)
\lineto(376.20623779,14.61030273)
\lineto(377.45123779,20.85030273)
\lineto(377.75123779,22.32030273)
\curveto(377.77123473,22.43029274)(377.79123471,22.54529262)(377.81123779,22.66530273)
\curveto(377.83123467,22.79529237)(377.87123463,22.89529227)(377.93123779,22.96530273)
\curveto(377.99123451,23.02529214)(378.07623442,23.07529209)(378.18623779,23.11530273)
\curveto(378.21623428,23.12529204)(378.24123426,23.12529204)(378.26123779,23.11530273)
\curveto(378.28123422,23.11529205)(378.30623419,23.12029205)(378.33623779,23.13030273)
}
}
{
\newrgbcolor{curcolor}{0 0 0}
\pscustom[linestyle=none,fillstyle=solid,fillcolor=curcolor]
{
\newpath
\moveto(386.55108154,16.62030273)
\curveto(386.55107304,16.52029865)(386.53107306,16.40529876)(386.49108154,16.27530273)
\curveto(386.45107314,16.15529901)(386.40107319,16.0702991)(386.34108154,16.02030273)
\curveto(386.28107331,15.98029919)(386.20107339,15.95029922)(386.10108154,15.93030273)
\curveto(386.00107359,15.92029925)(385.8910737,15.91529925)(385.77108154,15.91530273)
\lineto(385.41108154,15.91530273)
\curveto(385.30107429,15.92529924)(385.20107439,15.93029924)(385.11108154,15.93030273)
\lineto(381.27108154,15.93030273)
\curveto(381.1910784,15.93029924)(381.10607848,15.92529924)(381.01608154,15.91530273)
\curveto(380.93607865,15.91529925)(380.87107872,15.90029927)(380.82108154,15.87030273)
\curveto(380.77107882,15.85029932)(380.72107887,15.81029936)(380.67108154,15.75030273)
\lineto(380.58108154,15.61530273)
\curveto(380.55107904,15.5652996)(380.54107905,15.51529965)(380.55108154,15.46530273)
\curveto(380.55107904,15.41529975)(380.54607904,15.3702998)(380.53608154,15.33030273)
\lineto(380.53608154,15.21030273)
\lineto(380.53608154,14.95530273)
\curveto(380.54607904,14.87530029)(380.56107903,14.79530037)(380.58108154,14.71530273)
\curveto(380.71107888,14.17530099)(381.01607857,13.79030138)(381.49608154,13.56030273)
\curveto(381.54607804,13.53030164)(381.60607798,13.50530166)(381.67608154,13.48530273)
\curveto(381.74607784,13.4653017)(381.81107778,13.44530172)(381.87108154,13.42530273)
\curveto(381.90107769,13.41530175)(381.95107764,13.41030176)(382.02108154,13.41030273)
\curveto(382.15107744,13.3703018)(382.33107726,13.35030182)(382.56108154,13.35030273)
\curveto(382.7910768,13.35030182)(382.98107661,13.3703018)(383.13108154,13.41030273)
\curveto(383.28107631,13.45030172)(383.41607617,13.49030168)(383.53608154,13.53030273)
\curveto(383.66607592,13.58030159)(383.7860758,13.64030153)(383.89608154,13.71030273)
\curveto(384.01607557,13.78030139)(384.12607546,13.86030131)(384.22608154,13.95030273)
\curveto(384.32607526,14.05030112)(384.41607517,14.15530101)(384.49608154,14.26530273)
\curveto(384.57607501,14.3653008)(384.65107494,14.4703007)(384.72108154,14.58030273)
\curveto(384.7910748,14.69030048)(384.8860747,14.7703004)(385.00608154,14.82030273)
\curveto(385.04607454,14.84030033)(385.11107448,14.85530031)(385.20108154,14.86530273)
\curveto(385.30107429,14.87530029)(385.3910742,14.87530029)(385.47108154,14.86530273)
\curveto(385.56107403,14.8653003)(385.64607394,14.86030031)(385.72608154,14.85030273)
\curveto(385.80607378,14.84030033)(385.85607373,14.82030035)(385.87608154,14.79030273)
\curveto(385.96607362,14.72030045)(385.97107362,14.60530056)(385.89108154,14.44530273)
\curveto(385.75107384,14.17530099)(385.59607399,13.93530123)(385.42608154,13.72530273)
\curveto(385.16607442,13.40530176)(384.8860747,13.14030203)(384.58608154,12.93030273)
\curveto(384.29607529,12.73030244)(383.94107565,12.5653026)(383.52108154,12.43530273)
\curveto(383.41107618,12.39530277)(383.30607628,12.3703028)(383.20608154,12.36030273)
\curveto(383.10607648,12.34030283)(382.99607659,12.32030285)(382.87608154,12.30030273)
\curveto(382.82607676,12.29030288)(382.77607681,12.28530288)(382.72608154,12.28530273)
\curveto(382.6860769,12.28530288)(382.64107695,12.28030289)(382.59108154,12.27030273)
\lineto(382.44108154,12.27030273)
\curveto(382.3910772,12.26030291)(382.33107726,12.25530291)(382.26108154,12.25530273)
\curveto(382.20107739,12.25530291)(382.15107744,12.26030291)(382.11108154,12.27030273)
\lineto(381.97608154,12.27030273)
\curveto(381.92607766,12.28030289)(381.88107771,12.28530288)(381.84108154,12.28530273)
\curveto(381.80107779,12.28530288)(381.76107783,12.29030288)(381.72108154,12.30030273)
\curveto(381.67107792,12.31030286)(381.61607797,12.32030285)(381.55608154,12.33030273)
\curveto(381.50607808,12.33030284)(381.45607813,12.33530283)(381.40608154,12.34530273)
\curveto(381.31607827,12.3653028)(381.22607836,12.39030278)(381.13608154,12.42030273)
\curveto(381.05607853,12.44030273)(380.98107861,12.4653027)(380.91108154,12.49530273)
\curveto(380.87107872,12.51530265)(380.83607875,12.52530264)(380.80608154,12.52530273)
\curveto(380.77607881,12.53530263)(380.74607884,12.55030262)(380.71608154,12.57030273)
\curveto(380.57607901,12.64030253)(380.43107916,12.72530244)(380.28108154,12.82530273)
\curveto(380.03107956,13.01530215)(379.83107976,13.24530192)(379.68108154,13.51530273)
\curveto(379.53108006,13.79530137)(379.42108017,14.10530106)(379.35108154,14.44530273)
\curveto(379.32108027,14.55530061)(379.30608028,14.6703005)(379.30608154,14.79030273)
\curveto(379.30608028,14.91030026)(379.29608029,15.03030014)(379.27608154,15.15030273)
\lineto(379.27608154,15.25530273)
\curveto(379.2860803,15.28529988)(379.2910803,15.32529984)(379.29108154,15.37530273)
\lineto(379.29108154,15.63030273)
\curveto(379.30108029,15.72029945)(379.30608028,15.81029936)(379.30608154,15.90030273)
\lineto(379.35108154,16.11030273)
\curveto(379.35108024,16.15029902)(379.35608023,16.20529896)(379.36608154,16.27530273)
\curveto(379.37608021,16.35529881)(379.3910802,16.42029875)(379.41108154,16.47030273)
\lineto(379.44108154,16.63530273)
\curveto(379.47108012,16.68529848)(379.4860801,16.73529843)(379.48608154,16.78530273)
\curveto(379.49608009,16.84529832)(379.51108008,16.90029827)(379.53108154,16.95030273)
\curveto(379.60107999,17.11029806)(379.66607992,17.2702979)(379.72608154,17.43030273)
\curveto(379.7860798,17.59029758)(379.86107973,17.74029743)(379.95108154,17.88030273)
\curveto(380.02107957,17.99029718)(380.0860795,18.10029707)(380.14608154,18.21030273)
\curveto(380.21607937,18.33029684)(380.29607929,18.44529672)(380.38608154,18.55530273)
\curveto(380.67607891,18.90529626)(380.9860786,19.20529596)(381.31608154,19.45530273)
\curveto(381.64607794,19.71529545)(382.03107756,19.93029524)(382.47108154,20.10030273)
\curveto(382.60107699,20.15029502)(382.73107686,20.18529498)(382.86108154,20.20530273)
\curveto(382.9910766,20.23529493)(383.13107646,20.2652949)(383.28108154,20.29530273)
\curveto(383.33107626,20.30529486)(383.37607621,20.31029486)(383.41608154,20.31030273)
\curveto(383.45607613,20.32029485)(383.50107609,20.32529484)(383.55108154,20.32530273)
\curveto(383.57107602,20.33529483)(383.59607599,20.33529483)(383.62608154,20.32530273)
\curveto(383.65607593,20.31529485)(383.68107591,20.32029485)(383.70108154,20.34030273)
\curveto(384.13107546,20.35029482)(384.4910751,20.30529486)(384.78108154,20.20530273)
\curveto(385.07107452,20.11529505)(385.32607426,19.99029518)(385.54608154,19.83030273)
\curveto(385.586074,19.81029536)(385.61607397,19.78029539)(385.63608154,19.74030273)
\curveto(385.66607392,19.71029546)(385.69607389,19.68529548)(385.72608154,19.66530273)
\curveto(385.79607379,19.60529556)(385.86607372,19.53529563)(385.93608154,19.45530273)
\curveto(386.00607358,19.37529579)(386.06107353,19.29529587)(386.10108154,19.21530273)
\curveto(386.22107337,19.00529616)(386.31607327,18.80529636)(386.38608154,18.61530273)
\curveto(386.43607315,18.50529666)(386.46607312,18.38529678)(386.47608154,18.25530273)
\lineto(386.53608154,17.86530273)
\curveto(386.56607302,17.73529743)(386.57607301,17.60029757)(386.56608154,17.46030273)
\curveto(386.56607302,17.32029785)(386.57107302,17.18029799)(386.58108154,17.04030273)
\curveto(386.58107301,16.9702982)(386.57607301,16.90029827)(386.56608154,16.83030273)
\curveto(386.55607303,16.76029841)(386.55107304,16.69029848)(386.55108154,16.62030273)
\moveto(385.20108154,17.13030273)
\curveto(385.23107436,17.170298)(385.26107433,17.22029795)(385.29108154,17.28030273)
\curveto(385.33107426,17.35029782)(385.34607424,17.42029775)(385.33608154,17.49030273)
\curveto(385.32607426,17.71029746)(385.2860743,17.91529725)(385.21608154,18.10530273)
\curveto(385.11607447,18.33529683)(384.99607459,18.53029664)(384.85608154,18.69030273)
\curveto(384.72607486,18.85029632)(384.53607505,18.98529618)(384.28608154,19.09530273)
\curveto(384.21607537,19.11529605)(384.14607544,19.13029604)(384.07608154,19.14030273)
\curveto(384.01607557,19.16029601)(383.94607564,19.18029599)(383.86608154,19.20030273)
\curveto(383.79607579,19.22029595)(383.71607587,19.23029594)(383.62608154,19.23030273)
\lineto(383.37108154,19.23030273)
\curveto(383.33107626,19.21029596)(383.2910763,19.20029597)(383.25108154,19.20030273)
\curveto(383.21107638,19.21029596)(383.17607641,19.21029596)(383.14608154,19.20030273)
\lineto(382.90608154,19.14030273)
\curveto(382.82607676,19.13029604)(382.75107684,19.11529605)(382.68108154,19.09530273)
\curveto(382.36107723,18.97529619)(382.09607749,18.82529634)(381.88608154,18.64530273)
\curveto(381.67607791,18.4652967)(381.47607811,18.24029693)(381.28608154,17.97030273)
\curveto(381.24607834,17.92029725)(381.20107839,17.85529731)(381.15108154,17.77530273)
\curveto(381.11107848,17.70529746)(381.07107852,17.62529754)(381.03108154,17.53530273)
\curveto(380.9910786,17.44529772)(380.96607862,17.36029781)(380.95608154,17.28030273)
\curveto(380.95607863,17.20029797)(380.98107861,17.14029803)(381.03108154,17.10030273)
\curveto(381.10107849,17.04029813)(381.23107836,17.01029816)(381.42108154,17.01030273)
\curveto(381.62107797,17.02029815)(381.7910778,17.02529814)(381.93108154,17.02530273)
\lineto(384.21108154,17.02530273)
\curveto(384.36107523,17.02529814)(384.54107505,17.02029815)(384.75108154,17.01030273)
\curveto(384.96107463,17.01029816)(385.11107448,17.05029812)(385.20108154,17.13030273)
}
}
{
\newrgbcolor{curcolor}{0 0 0}
\pscustom[linestyle=none,fillstyle=solid,fillcolor=curcolor]
{
\newpath
\moveto(391.03772217,20.35530273)
\curveto(391.75771651,20.3652948)(392.34271593,20.28029489)(392.79272217,20.10030273)
\curveto(393.25271502,19.93029524)(393.5727147,19.62529554)(393.75272217,19.18530273)
\curveto(393.80271447,19.07529609)(393.83271444,18.96029621)(393.84272217,18.84030273)
\curveto(393.86271441,18.73029644)(393.87771439,18.60529656)(393.88772217,18.46530273)
\curveto(393.89771437,18.39529677)(393.88771438,18.32029685)(393.85772217,18.24030273)
\curveto(393.83771443,18.170297)(393.81271446,18.11529705)(393.78272217,18.07530273)
\curveto(393.76271451,18.05529711)(393.73271454,18.03529713)(393.69272217,18.01530273)
\curveto(393.66271461,18.00529716)(393.63771463,17.99029718)(393.61772217,17.97030273)
\curveto(393.55771471,17.95029722)(393.50271477,17.94529722)(393.45272217,17.95530273)
\curveto(393.41271486,17.9652972)(393.3677149,17.9652972)(393.31772217,17.95530273)
\curveto(393.22771504,17.93529723)(393.11771515,17.93029724)(392.98772217,17.94030273)
\curveto(392.8677154,17.96029721)(392.78271549,17.98529718)(392.73272217,18.01530273)
\curveto(392.66271561,18.0652971)(392.62271565,18.13029704)(392.61272217,18.21030273)
\curveto(392.61271566,18.30029687)(392.59271568,18.38529678)(392.55272217,18.46530273)
\curveto(392.50271577,18.62529654)(392.40771586,18.7702964)(392.26772217,18.90030273)
\curveto(392.17771609,18.98029619)(392.0677162,19.04029613)(391.93772217,19.08030273)
\curveto(391.81771645,19.12029605)(391.68771658,19.16029601)(391.54772217,19.20030273)
\curveto(391.50771676,19.22029595)(391.45771681,19.22529594)(391.39772217,19.21530273)
\curveto(391.34771692,19.21529595)(391.30271697,19.22029595)(391.26272217,19.23030273)
\curveto(391.20271707,19.25029592)(391.12771714,19.26029591)(391.03772217,19.26030273)
\curveto(390.94771732,19.26029591)(390.8727174,19.25029592)(390.81272217,19.23030273)
\lineto(390.72272217,19.23030273)
\curveto(390.66271761,19.22029595)(390.60771766,19.21029596)(390.55772217,19.20030273)
\curveto(390.50771776,19.20029597)(390.45771781,19.19529597)(390.40772217,19.18530273)
\curveto(390.13771813,19.12529604)(389.90271837,19.04029613)(389.70272217,18.93030273)
\curveto(389.51271876,18.82029635)(389.36271891,18.63529653)(389.25272217,18.37530273)
\curveto(389.22271905,18.30529686)(389.20771906,18.23529693)(389.20772217,18.16530273)
\curveto(389.20771906,18.09529707)(389.21271906,18.03529713)(389.22272217,17.98530273)
\curveto(389.25271902,17.83529733)(389.30271897,17.72529744)(389.37272217,17.65530273)
\curveto(389.44271883,17.59529757)(389.53771873,17.52529764)(389.65772217,17.44530273)
\curveto(389.79771847,17.34529782)(389.96271831,17.2702979)(390.15272217,17.22030273)
\curveto(390.34271793,17.18029799)(390.53271774,17.13029804)(390.72272217,17.07030273)
\curveto(390.84271743,17.03029814)(390.96271731,17.00029817)(391.08272217,16.98030273)
\curveto(391.21271706,16.96029821)(391.33771693,16.93029824)(391.45772217,16.89030273)
\curveto(391.65771661,16.83029834)(391.85271642,16.7702984)(392.04272217,16.71030273)
\curveto(392.23271604,16.66029851)(392.41771585,16.59529857)(392.59772217,16.51530273)
\curveto(392.64771562,16.49529867)(392.69271558,16.47529869)(392.73272217,16.45530273)
\curveto(392.78271549,16.43529873)(392.83271544,16.41029876)(392.88272217,16.38030273)
\curveto(393.05271522,16.26029891)(393.19771507,16.12529904)(393.31772217,15.97530273)
\curveto(393.43771483,15.82529934)(393.52771474,15.63529953)(393.58772217,15.40530273)
\lineto(393.58772217,15.12030273)
\curveto(393.58771468,15.05030012)(393.58271469,14.97530019)(393.57272217,14.89530273)
\curveto(393.56271471,14.82530034)(393.55271472,14.74530042)(393.54272217,14.65530273)
\lineto(393.51272217,14.50530273)
\curveto(393.4727148,14.43530073)(393.44271483,14.3653008)(393.42272217,14.29530273)
\curveto(393.41271486,14.22530094)(393.39271488,14.15530101)(393.36272217,14.08530273)
\curveto(393.31271496,13.97530119)(393.25771501,13.8703013)(393.19772217,13.77030273)
\curveto(393.13771513,13.6703015)(393.0727152,13.58030159)(393.00272217,13.50030273)
\curveto(392.79271548,13.24030193)(392.54771572,13.03030214)(392.26772217,12.87030273)
\curveto(391.98771628,12.72030245)(391.68271659,12.59030258)(391.35272217,12.48030273)
\curveto(391.25271702,12.45030272)(391.15271712,12.43030274)(391.05272217,12.42030273)
\curveto(390.95271732,12.40030277)(390.85771741,12.37530279)(390.76772217,12.34530273)
\curveto(390.65771761,12.32530284)(390.55271772,12.31530285)(390.45272217,12.31530273)
\curveto(390.35271792,12.31530285)(390.25271802,12.30530286)(390.15272217,12.28530273)
\lineto(390.00272217,12.28530273)
\curveto(389.95271832,12.27530289)(389.88271839,12.2703029)(389.79272217,12.27030273)
\curveto(389.70271857,12.2703029)(389.63271864,12.27530289)(389.58272217,12.28530273)
\lineto(389.41772217,12.28530273)
\curveto(389.35771891,12.30530286)(389.29271898,12.31530285)(389.22272217,12.31530273)
\curveto(389.15271912,12.30530286)(389.09771917,12.31030286)(389.05772217,12.33030273)
\curveto(389.00771926,12.34030283)(388.94271933,12.34530282)(388.86272217,12.34530273)
\curveto(388.78271949,12.3653028)(388.70771956,12.38530278)(388.63772217,12.40530273)
\curveto(388.5677197,12.41530275)(388.49271978,12.43530273)(388.41272217,12.46530273)
\curveto(388.12272015,12.5653026)(387.87772039,12.69030248)(387.67772217,12.84030273)
\curveto(387.47772079,12.99030218)(387.31772095,13.18530198)(387.19772217,13.42530273)
\curveto(387.13772113,13.55530161)(387.08772118,13.69030148)(387.04772217,13.83030273)
\curveto(387.01772125,13.9703012)(386.99772127,14.12530104)(386.98772217,14.29530273)
\curveto(386.97772129,14.35530081)(386.98272129,14.42530074)(387.00272217,14.50530273)
\curveto(387.02272125,14.59530057)(387.04772122,14.6653005)(387.07772217,14.71530273)
\curveto(387.11772115,14.75530041)(387.17772109,14.79530037)(387.25772217,14.83530273)
\curveto(387.30772096,14.85530031)(387.37772089,14.8653003)(387.46772217,14.86530273)
\curveto(387.5677207,14.87530029)(387.65772061,14.87530029)(387.73772217,14.86530273)
\curveto(387.82772044,14.85530031)(387.91272036,14.84030033)(387.99272217,14.82030273)
\curveto(388.08272019,14.81030036)(388.13772013,14.79530037)(388.15772217,14.77530273)
\curveto(388.21772005,14.72530044)(388.24772002,14.65030052)(388.24772217,14.55030273)
\curveto(388.25772001,14.46030071)(388.27771999,14.37530079)(388.30772217,14.29530273)
\curveto(388.35771991,14.07530109)(388.45771981,13.90530126)(388.60772217,13.78530273)
\curveto(388.70771956,13.69530147)(388.82771944,13.62530154)(388.96772217,13.57530273)
\curveto(389.10771916,13.52530164)(389.25771901,13.47530169)(389.41772217,13.42530273)
\lineto(389.73272217,13.38030273)
\lineto(389.82272217,13.38030273)
\curveto(389.88271839,13.36030181)(389.9677183,13.35030182)(390.07772217,13.35030273)
\curveto(390.19771807,13.35030182)(390.30271797,13.36030181)(390.39272217,13.38030273)
\curveto(390.46271781,13.38030179)(390.51771775,13.38530178)(390.55772217,13.39530273)
\curveto(390.61771765,13.40530176)(390.67771759,13.41030176)(390.73772217,13.41030273)
\curveto(390.79771747,13.42030175)(390.85271742,13.43030174)(390.90272217,13.44030273)
\curveto(391.21271706,13.52030165)(391.46271681,13.62530154)(391.65272217,13.75530273)
\curveto(391.85271642,13.88530128)(392.01771625,14.10530106)(392.14772217,14.41530273)
\curveto(392.17771609,14.4653007)(392.19271608,14.52030065)(392.19272217,14.58030273)
\curveto(392.20271607,14.64030053)(392.20271607,14.68530048)(392.19272217,14.71530273)
\curveto(392.18271609,14.90530026)(392.14271613,15.04530012)(392.07272217,15.13530273)
\curveto(392.00271627,15.23529993)(391.90771636,15.32529984)(391.78772217,15.40530273)
\curveto(391.70771656,15.4652997)(391.61271666,15.51529965)(391.50272217,15.55530273)
\lineto(391.20272217,15.67530273)
\curveto(391.1727171,15.68529948)(391.14271713,15.69029948)(391.11272217,15.69030273)
\curveto(391.09271718,15.69029948)(391.0727172,15.70029947)(391.05272217,15.72030273)
\curveto(390.73271754,15.83029934)(390.39271788,15.91029926)(390.03272217,15.96030273)
\curveto(389.68271859,16.02029915)(389.36271891,16.11529905)(389.07272217,16.24530273)
\curveto(388.98271929,16.28529888)(388.89271938,16.32029885)(388.80272217,16.35030273)
\curveto(388.72271955,16.38029879)(388.64771962,16.42029875)(388.57772217,16.47030273)
\curveto(388.40771986,16.58029859)(388.25772001,16.70529846)(388.12772217,16.84530273)
\curveto(387.99772027,16.98529818)(387.90772036,17.16029801)(387.85772217,17.37030273)
\curveto(387.83772043,17.44029773)(387.82772044,17.51029766)(387.82772217,17.58030273)
\lineto(387.82772217,17.80530273)
\curveto(387.81772045,17.92529724)(387.83272044,18.06029711)(387.87272217,18.21030273)
\curveto(387.91272036,18.3702968)(387.95272032,18.50529666)(387.99272217,18.61530273)
\curveto(388.02272025,18.6652965)(388.04272023,18.70529646)(388.05272217,18.73530273)
\curveto(388.0727202,18.77529639)(388.09772017,18.81529635)(388.12772217,18.85530273)
\curveto(388.25772001,19.08529608)(388.41771985,19.28529588)(388.60772217,19.45530273)
\curveto(388.79771947,19.62529554)(389.00771926,19.77529539)(389.23772217,19.90530273)
\curveto(389.39771887,19.99529517)(389.5727187,20.0652951)(389.76272217,20.11530273)
\curveto(389.96271831,20.17529499)(390.1677181,20.23029494)(390.37772217,20.28030273)
\curveto(390.44771782,20.29029488)(390.51271776,20.30029487)(390.57272217,20.31030273)
\curveto(390.64271763,20.32029485)(390.71771755,20.33029484)(390.79772217,20.34030273)
\curveto(390.83771743,20.35029482)(390.87771739,20.35029482)(390.91772217,20.34030273)
\curveto(390.9677173,20.33029484)(391.00771726,20.33529483)(391.03772217,20.35530273)
}
}
{
\newrgbcolor{curcolor}{0 0 0}
\pscustom[linestyle=none,fillstyle=solid,fillcolor=curcolor]
{
}
}
{
\newrgbcolor{curcolor}{0 0 0}
\pscustom[linestyle=none,fillstyle=solid,fillcolor=curcolor]
{
\newpath
\moveto(405.80287842,13.26030273)
\lineto(405.71287842,12.87030273)
\curveto(405.69287049,12.75030242)(405.65287053,12.65030252)(405.59287842,12.57030273)
\curveto(405.52287066,12.50030267)(405.42787075,12.46030271)(405.30787842,12.45030273)
\lineto(404.96287842,12.45030273)
\curveto(404.90287128,12.45030272)(404.84287134,12.44530272)(404.78287842,12.43530273)
\curveto(404.73287145,12.43530273)(404.68787149,12.44530272)(404.64787842,12.46530273)
\curveto(404.56787161,12.48530268)(404.51787166,12.52530264)(404.49787842,12.58530273)
\curveto(404.46787171,12.63530253)(404.45787172,12.69530247)(404.46787842,12.76530273)
\curveto(404.4778717,12.83530233)(404.47287171,12.90530226)(404.45287842,12.97530273)
\curveto(404.45287173,12.99530217)(404.44287174,13.01030216)(404.42287842,13.02030273)
\lineto(404.39287842,13.08030273)
\curveto(404.29287189,13.09030208)(404.20787197,13.0703021)(404.13787842,13.02030273)
\curveto(404.0778721,12.9703022)(404.01287217,12.92030225)(403.94287842,12.87030273)
\curveto(403.71287247,12.72030245)(403.48787269,12.60530256)(403.26787842,12.52530273)
\curveto(403.0778731,12.44530272)(402.85787332,12.38530278)(402.60787842,12.34530273)
\curveto(402.36787381,12.30530286)(402.12287406,12.28530288)(401.87287842,12.28530273)
\curveto(401.63287455,12.27530289)(401.39287479,12.29030288)(401.15287842,12.33030273)
\curveto(400.92287526,12.36030281)(400.72787545,12.41530275)(400.56787842,12.49530273)
\curveto(400.08787609,12.71530245)(399.72287646,13.01030216)(399.47287842,13.38030273)
\curveto(399.23287695,13.76030141)(399.0778771,14.23030094)(399.00787842,14.79030273)
\curveto(398.98787719,14.88030029)(398.9778772,14.9703002)(398.97787842,15.06030273)
\curveto(398.98787719,15.16030001)(398.98787719,15.26029991)(398.97787842,15.36030273)
\curveto(398.9778772,15.41029976)(398.9828772,15.46029971)(398.99287842,15.51030273)
\curveto(399.00287718,15.56029961)(399.00787717,15.61029956)(399.00787842,15.66030273)
\curveto(398.99787718,15.71029946)(398.99787718,15.76029941)(399.00787842,15.81030273)
\curveto(399.02787715,15.8702993)(399.03787714,15.92529924)(399.03787842,15.97530273)
\lineto(399.06787842,16.12530273)
\curveto(399.05787712,16.17529899)(399.05787712,16.24029893)(399.06787842,16.32030273)
\curveto(399.08787709,16.40029877)(399.11287707,16.4652987)(399.14287842,16.51530273)
\lineto(399.18787842,16.68030273)
\curveto(399.21787696,16.75029842)(399.23787694,16.82029835)(399.24787842,16.89030273)
\curveto(399.25787692,16.9702982)(399.2778769,17.04529812)(399.30787842,17.11530273)
\curveto(399.32787685,17.165298)(399.34287684,17.21029796)(399.35287842,17.25030273)
\curveto(399.36287682,17.29029788)(399.3778768,17.33529783)(399.39787842,17.38530273)
\curveto(399.44787673,17.48529768)(399.49287669,17.58029759)(399.53287842,17.67030273)
\curveto(399.57287661,17.7702974)(399.61787656,17.8652973)(399.66787842,17.95530273)
\curveto(399.86787631,18.33529683)(400.09787608,18.67529649)(400.35787842,18.97530273)
\curveto(400.62787555,19.28529588)(400.92787525,19.54029563)(401.25787842,19.74030273)
\curveto(401.45787472,19.86029531)(401.65787452,19.96029521)(401.85787842,20.04030273)
\curveto(402.05787412,20.12029505)(402.27287391,20.19029498)(402.50287842,20.25030273)
\lineto(402.71287842,20.28030273)
\curveto(402.7828734,20.29029488)(402.85287333,20.30529486)(402.92287842,20.32530273)
\lineto(403.07287842,20.32530273)
\curveto(403.16287302,20.34529482)(403.2828729,20.35529481)(403.43287842,20.35530273)
\curveto(403.59287259,20.35529481)(403.70787247,20.34529482)(403.77787842,20.32530273)
\curveto(403.81787236,20.31529485)(403.87287231,20.31029486)(403.94287842,20.31030273)
\curveto(404.04287214,20.28029489)(404.14787203,20.25529491)(404.25787842,20.23530273)
\curveto(404.36787181,20.22529494)(404.46787171,20.19529497)(404.55787842,20.14530273)
\curveto(404.69787148,20.08529508)(404.82787135,20.02029515)(404.94787842,19.95030273)
\curveto(405.06787111,19.88029529)(405.177871,19.80029537)(405.27787842,19.71030273)
\curveto(405.32787085,19.66029551)(405.3778708,19.60529556)(405.42787842,19.54530273)
\curveto(405.48787069,19.49529567)(405.57287061,19.48029569)(405.68287842,19.50030273)
\lineto(405.75787842,19.57530273)
\curveto(405.7778704,19.59529557)(405.79287039,19.62529554)(405.80287842,19.66530273)
\curveto(405.85287033,19.75529541)(405.88787029,19.8702953)(405.90787842,20.01030273)
\curveto(405.93787024,20.15029502)(405.96287022,20.27529489)(405.98287842,20.38530273)
\lineto(406.32787842,22.11030273)
\curveto(406.35786982,22.25029292)(406.38786979,22.40529276)(406.41787842,22.57530273)
\curveto(406.45786972,22.75529241)(406.50786967,22.88529228)(406.56787842,22.96530273)
\curveto(406.62786955,23.03529213)(406.69786948,23.08029209)(406.77787842,23.10030273)
\curveto(406.79786938,23.10029207)(406.82286936,23.10029207)(406.85287842,23.10030273)
\curveto(406.8828693,23.11029206)(406.90786927,23.11529205)(406.92787842,23.11530273)
\curveto(407.0778691,23.12529204)(407.22786895,23.12529204)(407.37787842,23.11530273)
\curveto(407.52786865,23.11529205)(407.62786855,23.07529209)(407.67787842,22.99530273)
\curveto(407.70786847,22.91529225)(407.70786847,22.81529235)(407.67787842,22.69530273)
\curveto(407.65786852,22.57529259)(407.63786854,22.45029272)(407.61787842,22.32030273)
\lineto(405.80287842,13.26030273)
\moveto(405.15787842,16.09530273)
\curveto(405.18787099,16.14529902)(405.20787097,16.21029896)(405.21787842,16.29030273)
\curveto(405.23787094,16.38029879)(405.24287094,16.45029872)(405.23287842,16.50030273)
\lineto(405.27787842,16.72530273)
\curveto(405.2778709,16.81529835)(405.2828709,16.90529826)(405.29287842,16.99530273)
\curveto(405.30287088,17.09529807)(405.29787088,17.18529798)(405.27787842,17.26530273)
\lineto(405.27787842,17.49030273)
\curveto(405.2778709,17.56029761)(405.26787091,17.63029754)(405.24787842,17.70030273)
\curveto(405.18787099,18.00029717)(405.0828711,18.2652969)(404.93287842,18.49530273)
\curveto(404.79287139,18.72529644)(404.59287159,18.90529626)(404.33287842,19.03530273)
\curveto(404.24287194,19.08529608)(404.14787203,19.12029605)(404.04787842,19.14030273)
\curveto(403.94787223,19.170296)(403.83787234,19.19529597)(403.71787842,19.21530273)
\curveto(403.64787253,19.23529593)(403.56287262,19.24529592)(403.46287842,19.24530273)
\lineto(403.19287842,19.24530273)
\lineto(403.04287842,19.21530273)
\lineto(402.90787842,19.21530273)
\curveto(402.82787335,19.19529597)(402.74287344,19.17529599)(402.65287842,19.15530273)
\curveto(402.56287362,19.13529603)(402.4778737,19.11029606)(402.39787842,19.08030273)
\curveto(402.04787413,18.94029623)(401.74787443,18.73529643)(401.49787842,18.46530273)
\curveto(401.24787493,18.20529696)(401.02787515,17.90029727)(400.83787842,17.55030273)
\curveto(400.7778754,17.44029773)(400.72787545,17.32529784)(400.68787842,17.20530273)
\lineto(400.56787842,16.87530273)
\lineto(400.53787842,16.75530273)
\curveto(400.52787565,16.72529844)(400.51787566,16.69029848)(400.50787842,16.65030273)
\curveto(400.4778757,16.60029857)(400.45787572,16.54529862)(400.44787842,16.48530273)
\curveto(400.44787573,16.42529874)(400.44287574,16.3702988)(400.43287842,16.32030273)
\curveto(400.41287577,16.21029896)(400.38787579,16.10029907)(400.35787842,15.99030273)
\curveto(400.33787584,15.89029928)(400.33287585,15.79529937)(400.34287842,15.70530273)
\curveto(400.34287584,15.67529949)(400.33787584,15.62529954)(400.32787842,15.55530273)
\lineto(400.32787842,15.34530273)
\curveto(400.32787585,15.27529989)(400.33287585,15.20529996)(400.34287842,15.13530273)
\curveto(400.3828758,14.78530038)(400.47287571,14.48530068)(400.61287842,14.23530273)
\curveto(400.75287543,13.98530118)(400.95287523,13.78030139)(401.21287842,13.62030273)
\curveto(401.29287489,13.5703016)(401.37287481,13.53030164)(401.45287842,13.50030273)
\curveto(401.54287464,13.4703017)(401.63787454,13.44030173)(401.73787842,13.41030273)
\curveto(401.78787439,13.39030178)(401.83787434,13.38530178)(401.88787842,13.39530273)
\curveto(401.94787423,13.40530176)(402.00287418,13.40030177)(402.05287842,13.38030273)
\curveto(402.0828741,13.3703018)(402.11787406,13.3653018)(402.15787842,13.36530273)
\lineto(402.29287842,13.36530273)
\lineto(402.42787842,13.36530273)
\curveto(402.46787371,13.37530179)(402.52287366,13.38030179)(402.59287842,13.38030273)
\curveto(402.67287351,13.40030177)(402.75287343,13.41530175)(402.83287842,13.42530273)
\curveto(402.92287326,13.44530172)(403.00287318,13.4703017)(403.07287842,13.50030273)
\curveto(403.43287275,13.64030153)(403.73787244,13.81530135)(403.98787842,14.02530273)
\curveto(404.23787194,14.24530092)(404.46287172,14.52030065)(404.66287842,14.85030273)
\curveto(404.73287145,14.96030021)(404.78787139,15.0703001)(404.82787842,15.18030273)
\lineto(404.97787842,15.51030273)
\curveto(405.00787117,15.55029962)(405.02287116,15.58529958)(405.02287842,15.61530273)
\curveto(405.03287115,15.65529951)(405.04787113,15.69529947)(405.06787842,15.73530273)
\curveto(405.08787109,15.79529937)(405.10287108,15.85529931)(405.11287842,15.91530273)
\curveto(405.12287106,15.97529919)(405.13787104,16.03529913)(405.15787842,16.09530273)
}
}
{
\newrgbcolor{curcolor}{0 0 0}
\pscustom[linestyle=none,fillstyle=solid,fillcolor=curcolor]
{
\newpath
\moveto(415.17412842,16.62030273)
\curveto(415.17411991,16.52029865)(415.15411993,16.40529876)(415.11412842,16.27530273)
\curveto(415.07412001,16.15529901)(415.02412006,16.0702991)(414.96412842,16.02030273)
\curveto(414.90412018,15.98029919)(414.82412026,15.95029922)(414.72412842,15.93030273)
\curveto(414.62412046,15.92029925)(414.51412057,15.91529925)(414.39412842,15.91530273)
\lineto(414.03412842,15.91530273)
\curveto(413.92412116,15.92529924)(413.82412126,15.93029924)(413.73412842,15.93030273)
\lineto(409.89412842,15.93030273)
\curveto(409.81412527,15.93029924)(409.72912536,15.92529924)(409.63912842,15.91530273)
\curveto(409.55912553,15.91529925)(409.49412559,15.90029927)(409.44412842,15.87030273)
\curveto(409.39412569,15.85029932)(409.34412574,15.81029936)(409.29412842,15.75030273)
\lineto(409.20412842,15.61530273)
\curveto(409.17412591,15.5652996)(409.16412592,15.51529965)(409.17412842,15.46530273)
\curveto(409.17412591,15.41529975)(409.16912592,15.3702998)(409.15912842,15.33030273)
\lineto(409.15912842,15.21030273)
\lineto(409.15912842,14.95530273)
\curveto(409.16912592,14.87530029)(409.1841259,14.79530037)(409.20412842,14.71530273)
\curveto(409.33412575,14.17530099)(409.63912545,13.79030138)(410.11912842,13.56030273)
\curveto(410.16912492,13.53030164)(410.22912486,13.50530166)(410.29912842,13.48530273)
\curveto(410.36912472,13.4653017)(410.43412465,13.44530172)(410.49412842,13.42530273)
\curveto(410.52412456,13.41530175)(410.57412451,13.41030176)(410.64412842,13.41030273)
\curveto(410.77412431,13.3703018)(410.95412413,13.35030182)(411.18412842,13.35030273)
\curveto(411.41412367,13.35030182)(411.60412348,13.3703018)(411.75412842,13.41030273)
\curveto(411.90412318,13.45030172)(412.03912305,13.49030168)(412.15912842,13.53030273)
\curveto(412.2891228,13.58030159)(412.40912268,13.64030153)(412.51912842,13.71030273)
\curveto(412.63912245,13.78030139)(412.74912234,13.86030131)(412.84912842,13.95030273)
\curveto(412.94912214,14.05030112)(413.03912205,14.15530101)(413.11912842,14.26530273)
\curveto(413.19912189,14.3653008)(413.27412181,14.4703007)(413.34412842,14.58030273)
\curveto(413.41412167,14.69030048)(413.50912158,14.7703004)(413.62912842,14.82030273)
\curveto(413.66912142,14.84030033)(413.73412135,14.85530031)(413.82412842,14.86530273)
\curveto(413.92412116,14.87530029)(414.01412107,14.87530029)(414.09412842,14.86530273)
\curveto(414.1841209,14.8653003)(414.26912082,14.86030031)(414.34912842,14.85030273)
\curveto(414.42912066,14.84030033)(414.47912061,14.82030035)(414.49912842,14.79030273)
\curveto(414.5891205,14.72030045)(414.59412049,14.60530056)(414.51412842,14.44530273)
\curveto(414.37412071,14.17530099)(414.21912087,13.93530123)(414.04912842,13.72530273)
\curveto(413.7891213,13.40530176)(413.50912158,13.14030203)(413.20912842,12.93030273)
\curveto(412.91912217,12.73030244)(412.56412252,12.5653026)(412.14412842,12.43530273)
\curveto(412.03412305,12.39530277)(411.92912316,12.3703028)(411.82912842,12.36030273)
\curveto(411.72912336,12.34030283)(411.61912347,12.32030285)(411.49912842,12.30030273)
\curveto(411.44912364,12.29030288)(411.39912369,12.28530288)(411.34912842,12.28530273)
\curveto(411.30912378,12.28530288)(411.26412382,12.28030289)(411.21412842,12.27030273)
\lineto(411.06412842,12.27030273)
\curveto(411.01412407,12.26030291)(410.95412413,12.25530291)(410.88412842,12.25530273)
\curveto(410.82412426,12.25530291)(410.77412431,12.26030291)(410.73412842,12.27030273)
\lineto(410.59912842,12.27030273)
\curveto(410.54912454,12.28030289)(410.50412458,12.28530288)(410.46412842,12.28530273)
\curveto(410.42412466,12.28530288)(410.3841247,12.29030288)(410.34412842,12.30030273)
\curveto(410.29412479,12.31030286)(410.23912485,12.32030285)(410.17912842,12.33030273)
\curveto(410.12912496,12.33030284)(410.07912501,12.33530283)(410.02912842,12.34530273)
\curveto(409.93912515,12.3653028)(409.84912524,12.39030278)(409.75912842,12.42030273)
\curveto(409.67912541,12.44030273)(409.60412548,12.4653027)(409.53412842,12.49530273)
\curveto(409.49412559,12.51530265)(409.45912563,12.52530264)(409.42912842,12.52530273)
\curveto(409.39912569,12.53530263)(409.36912572,12.55030262)(409.33912842,12.57030273)
\curveto(409.19912589,12.64030253)(409.05412603,12.72530244)(408.90412842,12.82530273)
\curveto(408.65412643,13.01530215)(408.45412663,13.24530192)(408.30412842,13.51530273)
\curveto(408.15412693,13.79530137)(408.04412704,14.10530106)(407.97412842,14.44530273)
\curveto(407.94412714,14.55530061)(407.92912716,14.6703005)(407.92912842,14.79030273)
\curveto(407.92912716,14.91030026)(407.91912717,15.03030014)(407.89912842,15.15030273)
\lineto(407.89912842,15.25530273)
\curveto(407.90912718,15.28529988)(407.91412717,15.32529984)(407.91412842,15.37530273)
\lineto(407.91412842,15.63030273)
\curveto(407.92412716,15.72029945)(407.92912716,15.81029936)(407.92912842,15.90030273)
\lineto(407.97412842,16.11030273)
\curveto(407.97412711,16.15029902)(407.97912711,16.20529896)(407.98912842,16.27530273)
\curveto(407.99912709,16.35529881)(408.01412707,16.42029875)(408.03412842,16.47030273)
\lineto(408.06412842,16.63530273)
\curveto(408.09412699,16.68529848)(408.10912698,16.73529843)(408.10912842,16.78530273)
\curveto(408.11912697,16.84529832)(408.13412695,16.90029827)(408.15412842,16.95030273)
\curveto(408.22412686,17.11029806)(408.2891268,17.2702979)(408.34912842,17.43030273)
\curveto(408.40912668,17.59029758)(408.4841266,17.74029743)(408.57412842,17.88030273)
\curveto(408.64412644,17.99029718)(408.70912638,18.10029707)(408.76912842,18.21030273)
\curveto(408.83912625,18.33029684)(408.91912617,18.44529672)(409.00912842,18.55530273)
\curveto(409.29912579,18.90529626)(409.60912548,19.20529596)(409.93912842,19.45530273)
\curveto(410.26912482,19.71529545)(410.65412443,19.93029524)(411.09412842,20.10030273)
\curveto(411.22412386,20.15029502)(411.35412373,20.18529498)(411.48412842,20.20530273)
\curveto(411.61412347,20.23529493)(411.75412333,20.2652949)(411.90412842,20.29530273)
\curveto(411.95412313,20.30529486)(411.99912309,20.31029486)(412.03912842,20.31030273)
\curveto(412.07912301,20.32029485)(412.12412296,20.32529484)(412.17412842,20.32530273)
\curveto(412.19412289,20.33529483)(412.21912287,20.33529483)(412.24912842,20.32530273)
\curveto(412.27912281,20.31529485)(412.30412278,20.32029485)(412.32412842,20.34030273)
\curveto(412.75412233,20.35029482)(413.11412197,20.30529486)(413.40412842,20.20530273)
\curveto(413.69412139,20.11529505)(413.94912114,19.99029518)(414.16912842,19.83030273)
\curveto(414.20912088,19.81029536)(414.23912085,19.78029539)(414.25912842,19.74030273)
\curveto(414.2891208,19.71029546)(414.31912077,19.68529548)(414.34912842,19.66530273)
\curveto(414.41912067,19.60529556)(414.4891206,19.53529563)(414.55912842,19.45530273)
\curveto(414.62912046,19.37529579)(414.6841204,19.29529587)(414.72412842,19.21530273)
\curveto(414.84412024,19.00529616)(414.93912015,18.80529636)(415.00912842,18.61530273)
\curveto(415.05912003,18.50529666)(415.08912,18.38529678)(415.09912842,18.25530273)
\lineto(415.15912842,17.86530273)
\curveto(415.1891199,17.73529743)(415.19911989,17.60029757)(415.18912842,17.46030273)
\curveto(415.1891199,17.32029785)(415.19411989,17.18029799)(415.20412842,17.04030273)
\curveto(415.20411988,16.9702982)(415.19911989,16.90029827)(415.18912842,16.83030273)
\curveto(415.17911991,16.76029841)(415.17411991,16.69029848)(415.17412842,16.62030273)
\moveto(413.82412842,17.13030273)
\curveto(413.85412123,17.170298)(413.8841212,17.22029795)(413.91412842,17.28030273)
\curveto(413.95412113,17.35029782)(413.96912112,17.42029775)(413.95912842,17.49030273)
\curveto(413.94912114,17.71029746)(413.90912118,17.91529725)(413.83912842,18.10530273)
\curveto(413.73912135,18.33529683)(413.61912147,18.53029664)(413.47912842,18.69030273)
\curveto(413.34912174,18.85029632)(413.15912193,18.98529618)(412.90912842,19.09530273)
\curveto(412.83912225,19.11529605)(412.76912232,19.13029604)(412.69912842,19.14030273)
\curveto(412.63912245,19.16029601)(412.56912252,19.18029599)(412.48912842,19.20030273)
\curveto(412.41912267,19.22029595)(412.33912275,19.23029594)(412.24912842,19.23030273)
\lineto(411.99412842,19.23030273)
\curveto(411.95412313,19.21029596)(411.91412317,19.20029597)(411.87412842,19.20030273)
\curveto(411.83412325,19.21029596)(411.79912329,19.21029596)(411.76912842,19.20030273)
\lineto(411.52912842,19.14030273)
\curveto(411.44912364,19.13029604)(411.37412371,19.11529605)(411.30412842,19.09530273)
\curveto(410.9841241,18.97529619)(410.71912437,18.82529634)(410.50912842,18.64530273)
\curveto(410.29912479,18.4652967)(410.09912499,18.24029693)(409.90912842,17.97030273)
\curveto(409.86912522,17.92029725)(409.82412526,17.85529731)(409.77412842,17.77530273)
\curveto(409.73412535,17.70529746)(409.69412539,17.62529754)(409.65412842,17.53530273)
\curveto(409.61412547,17.44529772)(409.5891255,17.36029781)(409.57912842,17.28030273)
\curveto(409.57912551,17.20029797)(409.60412548,17.14029803)(409.65412842,17.10030273)
\curveto(409.72412536,17.04029813)(409.85412523,17.01029816)(410.04412842,17.01030273)
\curveto(410.24412484,17.02029815)(410.41412467,17.02529814)(410.55412842,17.02530273)
\lineto(412.83412842,17.02530273)
\curveto(412.9841221,17.02529814)(413.16412192,17.02029815)(413.37412842,17.01030273)
\curveto(413.5841215,17.01029816)(413.73412135,17.05029812)(413.82412842,17.13030273)
}
}
{
\newrgbcolor{curcolor}{0 0 0}
\pscustom[linestyle=none,fillstyle=solid,fillcolor=curcolor]
{
\newpath
\moveto(420.24576904,23.25030273)
\curveto(420.42576334,23.26029191)(420.61576315,23.26029191)(420.81576904,23.25030273)
\curveto(421.01576275,23.24029193)(421.14576262,23.18029199)(421.20576904,23.07030273)
\curveto(421.23576253,23.01029216)(421.24576252,22.93529223)(421.23576904,22.84530273)
\curveto(421.22576254,22.7652924)(421.21076256,22.67529249)(421.19076904,22.57530273)
\curveto(421.1707626,22.44529272)(421.12576264,22.34029283)(421.05576904,22.26030273)
\curveto(421.00576276,22.21029296)(420.94076283,22.17529299)(420.86076904,22.15530273)
\curveto(420.78076299,22.14529302)(420.69576307,22.14029303)(420.60576904,22.14030273)
\lineto(420.33576904,22.14030273)
\curveto(420.24576352,22.15029302)(420.16076361,22.15029302)(420.08076904,22.14030273)
\curveto(419.79076398,22.06029311)(419.58576418,21.93029324)(419.46576904,21.75030273)
\curveto(419.34576442,21.58029359)(419.25076452,21.32029385)(419.18076904,20.97030273)
\curveto(419.16076461,20.90029427)(419.13576463,20.82529434)(419.10576904,20.74530273)
\curveto(419.08576468,20.67529449)(419.08076469,20.61029456)(419.09076904,20.55030273)
\curveto(419.09076468,20.40029477)(419.13576463,20.29529487)(419.22576904,20.23530273)
\curveto(419.29576447,20.20529496)(419.39076438,20.19029498)(419.51076904,20.19030273)
\lineto(419.87076904,20.19030273)
\lineto(420.09576904,20.19030273)
\curveto(420.12576364,20.170295)(420.15576361,20.165295)(420.18576904,20.17530273)
\curveto(420.21576355,20.18529498)(420.24576352,20.18029499)(420.27576904,20.16030273)
\curveto(420.3657634,20.13029504)(420.41576335,20.0702951)(420.42576904,19.98030273)
\curveto(420.44576332,19.90029527)(420.44076333,19.79529537)(420.41076904,19.66530273)
\lineto(420.38076904,19.54530273)
\lineto(420.35076904,19.42530273)
\curveto(420.29076348,19.27529589)(420.20576356,19.17529599)(420.09576904,19.12530273)
\curveto(419.95576381,19.07529609)(419.78576398,19.06029611)(419.58576904,19.08030273)
\curveto(419.38576438,19.11029606)(419.21076456,19.10529606)(419.06076904,19.06530273)
\curveto(418.98076479,19.04529612)(418.91576485,19.00529616)(418.86576904,18.94530273)
\curveto(418.81576495,18.89529627)(418.770765,18.82529634)(418.73076904,18.73530273)
\curveto(418.70076507,18.6652965)(418.68076509,18.58529658)(418.67076904,18.49530273)
\curveto(418.66076511,18.40529676)(418.64576512,18.32029685)(418.62576904,18.24030273)
\lineto(418.43076904,17.25030273)
\lineto(417.80076904,14.07030273)
\lineto(417.65076904,13.32030273)
\curveto(417.64076613,13.26030191)(417.63076614,13.19530197)(417.62076904,13.12530273)
\curveto(417.61076616,13.05530211)(417.59076618,12.99530217)(417.56076904,12.94530273)
\lineto(417.53076904,12.82530273)
\lineto(417.47076904,12.70530273)
\curveto(417.46076631,12.6653025)(417.44076633,12.63030254)(417.41076904,12.60030273)
\curveto(417.35076642,12.53030264)(417.2657665,12.49030268)(417.15576904,12.48030273)
\curveto(417.05576671,12.4703027)(416.94576682,12.4653027)(416.82576904,12.46530273)
\lineto(416.54076904,12.46530273)
\curveto(416.50076727,12.48530268)(416.45576731,12.50030267)(416.40576904,12.51030273)
\curveto(416.3657674,12.53030264)(416.33576743,12.5653026)(416.31576904,12.61530273)
\curveto(416.30576746,12.64530252)(416.30076747,12.71030246)(416.30076904,12.81030273)
\lineto(416.31576904,12.91530273)
\curveto(416.30576746,12.9653022)(416.31076746,13.01530215)(416.33076904,13.06530273)
\curveto(416.35076742,13.12530204)(416.3657674,13.18030199)(416.37576904,13.23030273)
\lineto(416.49576904,13.83030273)
\lineto(417.30576904,17.92530273)
\curveto(417.32576644,18.03529713)(417.35076642,18.15029702)(417.38076904,18.27030273)
\curveto(417.41076636,18.39029678)(417.43076634,18.50029667)(417.44076904,18.60030273)
\curveto(417.46076631,18.71029646)(417.46076631,18.80529636)(417.44076904,18.88530273)
\curveto(417.43076634,18.9652962)(417.38576638,19.02029615)(417.30576904,19.05030273)
\curveto(417.25576651,19.08029609)(417.19076658,19.09529607)(417.11076904,19.09530273)
\lineto(416.88576904,19.09530273)
\lineto(416.64576904,19.09530273)
\curveto(416.57576719,19.09529607)(416.51076726,19.10529606)(416.45076904,19.12530273)
\curveto(416.3707674,19.165296)(416.32576744,19.25029592)(416.31576904,19.38030273)
\lineto(416.31576904,19.51530273)
\curveto(416.32576744,19.55529561)(416.33576743,19.60029557)(416.34576904,19.65030273)
\curveto(416.37576739,19.79029538)(416.41076736,19.90029527)(416.45076904,19.98030273)
\curveto(416.50076727,20.0702951)(416.58076719,20.13029504)(416.69076904,20.16030273)
\curveto(416.770767,20.19029498)(416.85576691,20.20029497)(416.94576904,20.19030273)
\lineto(417.21576904,20.19030273)
\curveto(417.31576645,20.19029498)(417.40576636,20.20029497)(417.48576904,20.22030273)
\curveto(417.5657662,20.24029493)(417.63576613,20.28029489)(417.69576904,20.34030273)
\curveto(417.78576598,20.42029475)(417.84576592,20.54529462)(417.87576904,20.71530273)
\curveto(417.90576586,20.88529428)(417.93576583,21.04529412)(417.96576904,21.19530273)
\curveto(418.00576576,21.39529377)(418.05576571,21.58029359)(418.11576904,21.75030273)
\curveto(418.17576559,21.93029324)(418.25076552,22.09029308)(418.34076904,22.23030273)
\curveto(418.49076528,22.4702927)(418.6707651,22.6652925)(418.88076904,22.81530273)
\curveto(419.10076467,22.9652922)(419.35076442,23.08029209)(419.63076904,23.16030273)
\curveto(419.69076408,23.18029199)(419.75576401,23.19029198)(419.82576904,23.19030273)
\curveto(419.89576387,23.20029197)(419.9657638,23.21529195)(420.03576904,23.23530273)
\curveto(420.05576371,23.24529192)(420.09076368,23.24529192)(420.14076904,23.23530273)
\curveto(420.19076358,23.23529193)(420.22576354,23.24029193)(420.24576904,23.25030273)
\moveto(422.19576904,21.67530273)
\curveto(422.25576151,21.62529354)(422.33576143,21.60029357)(422.43576904,21.60030273)
\lineto(422.75076904,21.60030273)
\lineto(422.91576904,21.60030273)
\curveto(422.97576079,21.60029357)(423.03576073,21.61029356)(423.09576904,21.63030273)
\curveto(423.23576053,21.68029349)(423.32076045,21.78529338)(423.35076904,21.94530273)
\curveto(423.39076038,22.10529306)(423.43076034,22.27529289)(423.47076904,22.45530273)
\curveto(423.48076029,22.54529262)(423.49576027,22.63029254)(423.51576904,22.71030273)
\curveto(423.53576023,22.80029237)(423.53576023,22.87529229)(423.51576904,22.93530273)
\curveto(423.48576028,23.04529212)(423.39576037,23.10529206)(423.24576904,23.11530273)
\curveto(423.10576066,23.12529204)(422.95076082,23.13029204)(422.78076904,23.13030273)
\curveto(422.75076102,23.12029205)(422.72576104,23.11529205)(422.70576904,23.11530273)
\curveto(422.68576108,23.12529204)(422.66076111,23.12529204)(422.63076904,23.11530273)
\curveto(422.51076126,23.07529209)(422.42076135,23.01529215)(422.36076904,22.93530273)
\curveto(422.32076145,22.87529229)(422.29076148,22.80029237)(422.27076904,22.71030273)
\curveto(422.25076152,22.62029255)(422.23576153,22.53529263)(422.22576904,22.45530273)
\curveto(422.19576157,22.30529286)(422.1657616,22.15029302)(422.13576904,21.99030273)
\curveto(422.10576166,21.84029333)(422.12576164,21.73529343)(422.19576904,21.67530273)
\moveto(422.87076904,19.51530273)
\curveto(422.89076088,19.61529555)(422.91076086,19.71029546)(422.93076904,19.80030273)
\curveto(422.95076082,19.90029527)(422.94076083,19.98029519)(422.90076904,20.04030273)
\curveto(422.8707609,20.12029505)(422.78576098,20.16029501)(422.64576904,20.16030273)
\curveto(422.51576125,20.170295)(422.38576138,20.17529499)(422.25576904,20.17530273)
\curveto(422.23576153,20.165295)(422.21076156,20.16029501)(422.18076904,20.16030273)
\curveto(422.16076161,20.170295)(422.14076163,20.17529499)(422.12076904,20.17530273)
\curveto(422.06076171,20.15529501)(422.00076177,20.14029503)(421.94076904,20.13030273)
\curveto(421.89076188,20.12029505)(421.84576192,20.09029508)(421.80576904,20.04030273)
\curveto(421.74576202,19.98029519)(421.70576206,19.89529527)(421.68576904,19.78530273)
\curveto(421.6657621,19.68529548)(421.64576212,19.58029559)(421.62576904,19.47030273)
\lineto(420.35076904,13.12530273)
\curveto(420.33076344,13.03530213)(420.31076346,12.94030223)(420.29076904,12.84030273)
\curveto(420.28076349,12.75030242)(420.28576348,12.67530249)(420.30576904,12.61530273)
\curveto(420.34576342,12.53530263)(420.41076336,12.48530268)(420.50076904,12.46530273)
\curveto(420.59076318,12.45530271)(420.70076307,12.45030272)(420.83076904,12.45030273)
\lineto(421.05576904,12.45030273)
\curveto(421.14576262,12.4703027)(421.22076255,12.48530268)(421.28076904,12.49530273)
\curveto(421.34076243,12.51530265)(421.39076238,12.55530261)(421.43076904,12.61530273)
\curveto(421.50076227,12.67530249)(421.54076223,12.75530241)(421.55076904,12.85530273)
\curveto(421.5707622,12.9653022)(421.59076218,13.0703021)(421.61076904,13.17030273)
\lineto(422.87076904,19.51530273)
}
}
{
\newrgbcolor{curcolor}{0 0 0}
\pscustom[linestyle=none,fillstyle=solid,fillcolor=curcolor]
{
\newpath
\moveto(428.66944092,20.32530273)
\curveto(429.3094341,20.34529482)(429.79943361,20.26029491)(430.13944092,20.07030273)
\curveto(430.47943293,19.88029529)(430.72443268,19.59529557)(430.87444092,19.21530273)
\curveto(430.91443249,19.11529605)(430.93943247,19.00529616)(430.94944092,18.88530273)
\curveto(430.96943244,18.77529639)(430.97943243,18.66029651)(430.97944092,18.54030273)
\curveto(430.99943241,18.35029682)(430.98943242,18.14529702)(430.94944092,17.92530273)
\curveto(430.91943249,17.70529746)(430.87943253,17.48029769)(430.82944092,17.25030273)
\lineto(430.51444092,15.64530273)
\lineto(430.04944092,13.30530273)
\lineto(429.92944092,12.79530273)
\curveto(429.88943352,12.62530254)(429.79943361,12.51530265)(429.65944092,12.46530273)
\curveto(429.6094338,12.44530272)(429.55443385,12.43530273)(429.49444092,12.43530273)
\curveto(429.44443396,12.42530274)(429.38943402,12.42030275)(429.32944092,12.42030273)
\curveto(429.19943421,12.42030275)(429.07443433,12.42530274)(428.95444092,12.43530273)
\curveto(428.83443457,12.43530273)(428.75943465,12.47530269)(428.72944092,12.55530273)
\curveto(428.68943472,12.62530254)(428.67943473,12.71530245)(428.69944092,12.82530273)
\curveto(428.71943469,12.93530223)(428.74443466,13.04530212)(428.77444092,13.15530273)
\lineto(429.02944092,14.44530273)
\lineto(429.50944092,16.89030273)
\curveto(429.56943384,17.16029801)(429.61943379,17.42529774)(429.65944092,17.68530273)
\curveto(429.69943371,17.95529721)(429.69943371,18.18529698)(429.65944092,18.37530273)
\curveto(429.61943379,18.57529659)(429.52943388,18.73529643)(429.38944092,18.85530273)
\curveto(429.25943415,18.98529618)(429.09943431,19.08529608)(428.90944092,19.15530273)
\curveto(428.84943456,19.17529599)(428.78443462,19.18529598)(428.71444092,19.18530273)
\curveto(428.65443475,19.19529597)(428.59943481,19.21029596)(428.54944092,19.23030273)
\curveto(428.49943491,19.24029593)(428.41943499,19.24029593)(428.30944092,19.23030273)
\curveto(428.2094352,19.23029594)(428.13443527,19.22529594)(428.08444092,19.21530273)
\curveto(428.04443536,19.19529597)(428.0094354,19.18529598)(427.97944092,19.18530273)
\curveto(427.94943546,19.19529597)(427.91443549,19.19529597)(427.87444092,19.18530273)
\curveto(427.73443567,19.15529601)(427.6044358,19.12029605)(427.48444092,19.08030273)
\curveto(427.36443604,19.05029612)(427.24943616,19.00529616)(427.13944092,18.94530273)
\curveto(427.08943632,18.92529624)(427.04943636,18.90529626)(427.01944092,18.88530273)
\curveto(426.98943642,18.8652963)(426.94943646,18.84529632)(426.89944092,18.82530273)
\curveto(426.49943691,18.57529659)(426.16943724,18.20029697)(425.90944092,17.70030273)
\curveto(425.86943754,17.62029755)(425.83443757,17.53529763)(425.80444092,17.44530273)
\lineto(425.71444092,17.20530273)
\curveto(425.68443772,17.15529801)(425.66943774,17.10529806)(425.66944092,17.05530273)
\curveto(425.66943774,17.01529815)(425.65443775,16.97529819)(425.62444092,16.93530273)
\lineto(425.56444092,16.62030273)
\curveto(425.54443786,16.59029858)(425.53443787,16.55529861)(425.53444092,16.51530273)
\curveto(425.53443787,16.47529869)(425.52943788,16.43029874)(425.51944092,16.38030273)
\lineto(425.42944092,15.93030273)
\lineto(425.12944092,14.49030273)
\lineto(424.87444092,13.17030273)
\curveto(424.85443855,13.06030211)(424.82943858,12.94530222)(424.79944092,12.82530273)
\curveto(424.77943863,12.71530245)(424.73943867,12.62530254)(424.67944092,12.55530273)
\curveto(424.6094388,12.47530269)(424.5094389,12.43530273)(424.37944092,12.43530273)
\curveto(424.25943915,12.42530274)(424.13443927,12.42030275)(424.00444092,12.42030273)
\curveto(423.92443948,12.42030275)(423.84943956,12.42530274)(423.77944092,12.43530273)
\curveto(423.7094397,12.44530272)(423.65443975,12.4703027)(423.61444092,12.51030273)
\curveto(423.54443986,12.56030261)(423.52443988,12.65530251)(423.55444092,12.79530273)
\curveto(423.58443982,12.93530223)(423.6094398,13.0703021)(423.62944092,13.20030273)
\lineto(423.98944092,14.97030273)
\lineto(424.70944092,18.60030273)
\lineto(424.88944092,19.51530273)
\lineto(424.94944092,19.78530273)
\curveto(424.96943844,19.87529529)(425.0044384,19.94529522)(425.05444092,19.99530273)
\curveto(425.09443831,20.05529511)(425.14943826,20.09529507)(425.21944092,20.11530273)
\curveto(425.26943814,20.12529504)(425.32943808,20.13529503)(425.39944092,20.14530273)
\curveto(425.47943793,20.15529501)(425.55943785,20.16029501)(425.63944092,20.16030273)
\curveto(425.71943769,20.16029501)(425.79443761,20.15529501)(425.86444092,20.14530273)
\curveto(425.94443746,20.13529503)(425.99443741,20.12029505)(426.01444092,20.10030273)
\curveto(426.11443729,20.03029514)(426.14943726,19.94029523)(426.11944092,19.83030273)
\curveto(426.08943732,19.73029544)(426.07943733,19.61529555)(426.08944092,19.48530273)
\curveto(426.09943731,19.42529574)(426.12943728,19.37529579)(426.17944092,19.33530273)
\curveto(426.29943711,19.32529584)(426.404437,19.3702958)(426.49444092,19.47030273)
\curveto(426.59443681,19.5702956)(426.68943672,19.65029552)(426.77944092,19.71030273)
\curveto(426.93943647,19.81029536)(427.09943631,19.90029527)(427.25944092,19.98030273)
\curveto(427.41943599,20.0702951)(427.6044358,20.14529502)(427.81444092,20.20530273)
\curveto(427.89443551,20.23529493)(427.98443542,20.25529491)(428.08444092,20.26530273)
\curveto(428.18443522,20.27529489)(428.27943513,20.29029488)(428.36944092,20.31030273)
\curveto(428.41943499,20.32029485)(428.46943494,20.32529484)(428.51944092,20.32530273)
\lineto(428.66944092,20.32530273)
}
}
{
\newrgbcolor{curcolor}{0 0 0}
\pscustom[linestyle=none,fillstyle=solid,fillcolor=curcolor]
{
\newpath
\moveto(433.88405029,21.67530273)
\curveto(433.81404732,21.73529343)(433.79404734,21.84029333)(433.82405029,21.99030273)
\curveto(433.85404728,22.15029302)(433.88404725,22.30529286)(433.91405029,22.45530273)
\curveto(433.92404721,22.53529263)(433.93904719,22.62029255)(433.95905029,22.71030273)
\curveto(433.97904715,22.80029237)(434.00904712,22.87529229)(434.04905029,22.93530273)
\curveto(434.10904702,23.01529215)(434.19904693,23.07529209)(434.31905029,23.11530273)
\curveto(434.34904678,23.12529204)(434.37404676,23.12529204)(434.39405029,23.11530273)
\curveto(434.41404672,23.11529205)(434.43904669,23.12029205)(434.46905029,23.13030273)
\curveto(434.63904649,23.13029204)(434.79404634,23.12529204)(434.93405029,23.11530273)
\curveto(435.08404605,23.10529206)(435.17404596,23.04529212)(435.20405029,22.93530273)
\curveto(435.22404591,22.87529229)(435.22404591,22.80029237)(435.20405029,22.71030273)
\curveto(435.18404595,22.63029254)(435.16904596,22.54529262)(435.15905029,22.45530273)
\curveto(435.11904601,22.27529289)(435.07904605,22.10529306)(435.03905029,21.94530273)
\curveto(435.00904612,21.78529338)(434.92404621,21.68029349)(434.78405029,21.63030273)
\curveto(434.72404641,21.61029356)(434.66404647,21.60029357)(434.60405029,21.60030273)
\lineto(434.43905029,21.60030273)
\lineto(434.12405029,21.60030273)
\curveto(434.02404711,21.60029357)(433.94404719,21.62529354)(433.88405029,21.67530273)
\moveto(433.29905029,13.17030273)
\curveto(433.27904785,13.0703021)(433.25904787,12.9653022)(433.23905029,12.85530273)
\curveto(433.2290479,12.75530241)(433.18904794,12.67530249)(433.11905029,12.61530273)
\curveto(433.07904805,12.55530261)(433.0290481,12.51530265)(432.96905029,12.49530273)
\curveto(432.90904822,12.48530268)(432.8340483,12.4703027)(432.74405029,12.45030273)
\lineto(432.51905029,12.45030273)
\curveto(432.38904874,12.45030272)(432.27904885,12.45530271)(432.18905029,12.46530273)
\curveto(432.09904903,12.48530268)(432.0340491,12.53530263)(431.99405029,12.61530273)
\curveto(431.97404916,12.67530249)(431.96904916,12.75030242)(431.97905029,12.84030273)
\curveto(431.99904913,12.94030223)(432.01904911,13.03530213)(432.03905029,13.12530273)
\lineto(433.31405029,19.47030273)
\curveto(433.3340478,19.58029559)(433.35404778,19.68529548)(433.37405029,19.78530273)
\curveto(433.39404774,19.89529527)(433.4340477,19.98029519)(433.49405029,20.04030273)
\curveto(433.5340476,20.09029508)(433.57904755,20.12029505)(433.62905029,20.13030273)
\curveto(433.68904744,20.14029503)(433.74904738,20.15529501)(433.80905029,20.17530273)
\curveto(433.8290473,20.17529499)(433.84904728,20.170295)(433.86905029,20.16030273)
\curveto(433.89904723,20.16029501)(433.92404721,20.165295)(433.94405029,20.17530273)
\curveto(434.07404706,20.17529499)(434.20404693,20.170295)(434.33405029,20.16030273)
\curveto(434.47404666,20.16029501)(434.55904657,20.12029505)(434.58905029,20.04030273)
\curveto(434.6290465,19.98029519)(434.63904649,19.90029527)(434.61905029,19.80030273)
\curveto(434.59904653,19.71029546)(434.57904655,19.61529555)(434.55905029,19.51530273)
\lineto(433.29905029,13.17030273)
}
}
{
\newrgbcolor{curcolor}{0 0 0}
\pscustom[linestyle=none,fillstyle=solid,fillcolor=curcolor]
{
\newpath
\moveto(442.21889404,13.26030273)
\lineto(442.12889404,12.87030273)
\curveto(442.10888611,12.75030242)(442.06888615,12.65030252)(442.00889404,12.57030273)
\curveto(441.93888628,12.50030267)(441.84388638,12.46030271)(441.72389404,12.45030273)
\lineto(441.37889404,12.45030273)
\curveto(441.3188869,12.45030272)(441.25888696,12.44530272)(441.19889404,12.43530273)
\curveto(441.14888707,12.43530273)(441.10388712,12.44530272)(441.06389404,12.46530273)
\curveto(440.98388724,12.48530268)(440.93388729,12.52530264)(440.91389404,12.58530273)
\curveto(440.88388734,12.63530253)(440.87388735,12.69530247)(440.88389404,12.76530273)
\curveto(440.89388733,12.83530233)(440.88888733,12.90530226)(440.86889404,12.97530273)
\curveto(440.86888735,12.99530217)(440.85888736,13.01030216)(440.83889404,13.02030273)
\lineto(440.80889404,13.08030273)
\curveto(440.70888751,13.09030208)(440.6238876,13.0703021)(440.55389404,13.02030273)
\curveto(440.49388773,12.9703022)(440.42888779,12.92030225)(440.35889404,12.87030273)
\curveto(440.12888809,12.72030245)(439.90388832,12.60530256)(439.68389404,12.52530273)
\curveto(439.49388873,12.44530272)(439.27388895,12.38530278)(439.02389404,12.34530273)
\curveto(438.78388944,12.30530286)(438.53888968,12.28530288)(438.28889404,12.28530273)
\curveto(438.04889017,12.27530289)(437.80889041,12.29030288)(437.56889404,12.33030273)
\curveto(437.33889088,12.36030281)(437.14389108,12.41530275)(436.98389404,12.49530273)
\curveto(436.50389172,12.71530245)(436.13889208,13.01030216)(435.88889404,13.38030273)
\curveto(435.64889257,13.76030141)(435.49389273,14.23030094)(435.42389404,14.79030273)
\curveto(435.40389282,14.88030029)(435.39389283,14.9703002)(435.39389404,15.06030273)
\curveto(435.40389282,15.16030001)(435.40389282,15.26029991)(435.39389404,15.36030273)
\curveto(435.39389283,15.41029976)(435.39889282,15.46029971)(435.40889404,15.51030273)
\curveto(435.4188928,15.56029961)(435.4238928,15.61029956)(435.42389404,15.66030273)
\curveto(435.41389281,15.71029946)(435.41389281,15.76029941)(435.42389404,15.81030273)
\curveto(435.44389278,15.8702993)(435.45389277,15.92529924)(435.45389404,15.97530273)
\lineto(435.48389404,16.12530273)
\curveto(435.47389275,16.17529899)(435.47389275,16.24029893)(435.48389404,16.32030273)
\curveto(435.50389272,16.40029877)(435.52889269,16.4652987)(435.55889404,16.51530273)
\lineto(435.60389404,16.68030273)
\curveto(435.63389259,16.75029842)(435.65389257,16.82029835)(435.66389404,16.89030273)
\curveto(435.67389255,16.9702982)(435.69389253,17.04529812)(435.72389404,17.11530273)
\curveto(435.74389248,17.165298)(435.75889246,17.21029796)(435.76889404,17.25030273)
\curveto(435.77889244,17.29029788)(435.79389243,17.33529783)(435.81389404,17.38530273)
\curveto(435.86389236,17.48529768)(435.90889231,17.58029759)(435.94889404,17.67030273)
\curveto(435.98889223,17.7702974)(436.03389219,17.8652973)(436.08389404,17.95530273)
\curveto(436.28389194,18.33529683)(436.51389171,18.67529649)(436.77389404,18.97530273)
\curveto(437.04389118,19.28529588)(437.34389088,19.54029563)(437.67389404,19.74030273)
\curveto(437.87389035,19.86029531)(438.07389015,19.96029521)(438.27389404,20.04030273)
\curveto(438.47388975,20.12029505)(438.68888953,20.19029498)(438.91889404,20.25030273)
\lineto(439.12889404,20.28030273)
\curveto(439.19888902,20.29029488)(439.26888895,20.30529486)(439.33889404,20.32530273)
\lineto(439.48889404,20.32530273)
\curveto(439.57888864,20.34529482)(439.69888852,20.35529481)(439.84889404,20.35530273)
\curveto(440.00888821,20.35529481)(440.1238881,20.34529482)(440.19389404,20.32530273)
\curveto(440.23388799,20.31529485)(440.28888793,20.31029486)(440.35889404,20.31030273)
\curveto(440.45888776,20.28029489)(440.56388766,20.25529491)(440.67389404,20.23530273)
\curveto(440.78388744,20.22529494)(440.88388734,20.19529497)(440.97389404,20.14530273)
\curveto(441.11388711,20.08529508)(441.24388698,20.02029515)(441.36389404,19.95030273)
\curveto(441.48388674,19.88029529)(441.59388663,19.80029537)(441.69389404,19.71030273)
\curveto(441.74388648,19.66029551)(441.79388643,19.60529556)(441.84389404,19.54530273)
\curveto(441.90388632,19.49529567)(441.98888623,19.48029569)(442.09889404,19.50030273)
\lineto(442.17389404,19.57530273)
\curveto(442.19388603,19.59529557)(442.20888601,19.62529554)(442.21889404,19.66530273)
\curveto(442.26888595,19.75529541)(442.30388592,19.8702953)(442.32389404,20.01030273)
\curveto(442.35388587,20.15029502)(442.37888584,20.27529489)(442.39889404,20.38530273)
\lineto(442.74389404,22.11030273)
\curveto(442.77388545,22.25029292)(442.80388542,22.40529276)(442.83389404,22.57530273)
\curveto(442.87388535,22.75529241)(442.9238853,22.88529228)(442.98389404,22.96530273)
\curveto(443.04388518,23.03529213)(443.11388511,23.08029209)(443.19389404,23.10030273)
\curveto(443.21388501,23.10029207)(443.23888498,23.10029207)(443.26889404,23.10030273)
\curveto(443.29888492,23.11029206)(443.3238849,23.11529205)(443.34389404,23.11530273)
\curveto(443.49388473,23.12529204)(443.64388458,23.12529204)(443.79389404,23.11530273)
\curveto(443.94388428,23.11529205)(444.04388418,23.07529209)(444.09389404,22.99530273)
\curveto(444.1238841,22.91529225)(444.1238841,22.81529235)(444.09389404,22.69530273)
\curveto(444.07388415,22.57529259)(444.05388417,22.45029272)(444.03389404,22.32030273)
\lineto(442.21889404,13.26030273)
\moveto(441.57389404,16.09530273)
\curveto(441.60388662,16.14529902)(441.6238866,16.21029896)(441.63389404,16.29030273)
\curveto(441.65388657,16.38029879)(441.65888656,16.45029872)(441.64889404,16.50030273)
\lineto(441.69389404,16.72530273)
\curveto(441.69388653,16.81529835)(441.69888652,16.90529826)(441.70889404,16.99530273)
\curveto(441.7188865,17.09529807)(441.71388651,17.18529798)(441.69389404,17.26530273)
\lineto(441.69389404,17.49030273)
\curveto(441.69388653,17.56029761)(441.68388654,17.63029754)(441.66389404,17.70030273)
\curveto(441.60388662,18.00029717)(441.49888672,18.2652969)(441.34889404,18.49530273)
\curveto(441.20888701,18.72529644)(441.00888721,18.90529626)(440.74889404,19.03530273)
\curveto(440.65888756,19.08529608)(440.56388766,19.12029605)(440.46389404,19.14030273)
\curveto(440.36388786,19.170296)(440.25388797,19.19529597)(440.13389404,19.21530273)
\curveto(440.06388816,19.23529593)(439.97888824,19.24529592)(439.87889404,19.24530273)
\lineto(439.60889404,19.24530273)
\lineto(439.45889404,19.21530273)
\lineto(439.32389404,19.21530273)
\curveto(439.24388898,19.19529597)(439.15888906,19.17529599)(439.06889404,19.15530273)
\curveto(438.97888924,19.13529603)(438.89388933,19.11029606)(438.81389404,19.08030273)
\curveto(438.46388976,18.94029623)(438.16389006,18.73529643)(437.91389404,18.46530273)
\curveto(437.66389056,18.20529696)(437.44389078,17.90029727)(437.25389404,17.55030273)
\curveto(437.19389103,17.44029773)(437.14389108,17.32529784)(437.10389404,17.20530273)
\lineto(436.98389404,16.87530273)
\lineto(436.95389404,16.75530273)
\curveto(436.94389128,16.72529844)(436.93389129,16.69029848)(436.92389404,16.65030273)
\curveto(436.89389133,16.60029857)(436.87389135,16.54529862)(436.86389404,16.48530273)
\curveto(436.86389136,16.42529874)(436.85889136,16.3702988)(436.84889404,16.32030273)
\curveto(436.82889139,16.21029896)(436.80389142,16.10029907)(436.77389404,15.99030273)
\curveto(436.75389147,15.89029928)(436.74889147,15.79529937)(436.75889404,15.70530273)
\curveto(436.75889146,15.67529949)(436.75389147,15.62529954)(436.74389404,15.55530273)
\lineto(436.74389404,15.34530273)
\curveto(436.74389148,15.27529989)(436.74889147,15.20529996)(436.75889404,15.13530273)
\curveto(436.79889142,14.78530038)(436.88889133,14.48530068)(437.02889404,14.23530273)
\curveto(437.16889105,13.98530118)(437.36889085,13.78030139)(437.62889404,13.62030273)
\curveto(437.70889051,13.5703016)(437.78889043,13.53030164)(437.86889404,13.50030273)
\curveto(437.95889026,13.4703017)(438.05389017,13.44030173)(438.15389404,13.41030273)
\curveto(438.20389002,13.39030178)(438.25388997,13.38530178)(438.30389404,13.39530273)
\curveto(438.36388986,13.40530176)(438.4188898,13.40030177)(438.46889404,13.38030273)
\curveto(438.49888972,13.3703018)(438.53388969,13.3653018)(438.57389404,13.36530273)
\lineto(438.70889404,13.36530273)
\lineto(438.84389404,13.36530273)
\curveto(438.88388934,13.37530179)(438.93888928,13.38030179)(439.00889404,13.38030273)
\curveto(439.08888913,13.40030177)(439.16888905,13.41530175)(439.24889404,13.42530273)
\curveto(439.33888888,13.44530172)(439.4188888,13.4703017)(439.48889404,13.50030273)
\curveto(439.84888837,13.64030153)(440.15388807,13.81530135)(440.40389404,14.02530273)
\curveto(440.65388757,14.24530092)(440.87888734,14.52030065)(441.07889404,14.85030273)
\curveto(441.14888707,14.96030021)(441.20388702,15.0703001)(441.24389404,15.18030273)
\lineto(441.39389404,15.51030273)
\curveto(441.4238868,15.55029962)(441.43888678,15.58529958)(441.43889404,15.61530273)
\curveto(441.44888677,15.65529951)(441.46388676,15.69529947)(441.48389404,15.73530273)
\curveto(441.50388672,15.79529937)(441.5188867,15.85529931)(441.52889404,15.91530273)
\curveto(441.53888668,15.97529919)(441.55388667,16.03529913)(441.57389404,16.09530273)
}
}
{
\newrgbcolor{curcolor}{0 0 0}
\pscustom[linestyle=none,fillstyle=solid,fillcolor=curcolor]
{
\newpath
\moveto(451.96514404,16.65030273)
\curveto(451.97513515,16.59029858)(451.96513516,16.49529867)(451.93514404,16.36530273)
\curveto(451.91513521,16.24529892)(451.89513523,16.16029901)(451.87514404,16.11030273)
\lineto(451.84514404,15.96030273)
\curveto(451.81513531,15.88029929)(451.79013534,15.80529936)(451.77014404,15.73530273)
\curveto(451.76013537,15.67529949)(451.74013539,15.60529956)(451.71014404,15.52530273)
\curveto(451.68013545,15.4652997)(451.65513547,15.40529976)(451.63514404,15.34530273)
\curveto(451.6251355,15.28529988)(451.60013553,15.22529994)(451.56014404,15.16530273)
\lineto(451.38014404,14.77530273)
\curveto(451.3301358,14.64530052)(451.26513586,14.52530064)(451.18514404,14.41530273)
\curveto(450.88513624,13.93530123)(450.5251366,13.53030164)(450.10514404,13.20030273)
\curveto(449.69513743,12.88030229)(449.21513791,12.63530253)(448.66514404,12.46530273)
\curveto(448.55513857,12.42530274)(448.43513869,12.39530277)(448.30514404,12.37530273)
\curveto(448.17513895,12.35530281)(448.04013909,12.33530283)(447.90014404,12.31530273)
\curveto(447.84013929,12.30530286)(447.77513935,12.30030287)(447.70514404,12.30030273)
\curveto(447.64513948,12.29030288)(447.58513954,12.28530288)(447.52514404,12.28530273)
\curveto(447.48513964,12.27530289)(447.4251397,12.2703029)(447.34514404,12.27030273)
\curveto(447.27513985,12.2703029)(447.2251399,12.27530289)(447.19514404,12.28530273)
\curveto(447.15513997,12.29530287)(447.11514001,12.30030287)(447.07514404,12.30030273)
\curveto(447.03514009,12.29030288)(447.00014013,12.29030288)(446.97014404,12.30030273)
\lineto(446.88014404,12.30030273)
\lineto(446.53514404,12.34530273)
\lineto(446.14514404,12.46530273)
\curveto(446.0251411,12.50530266)(445.91014122,12.55030262)(445.80014404,12.60030273)
\curveto(445.39014174,12.80030237)(445.07014206,13.06030211)(444.84014404,13.38030273)
\curveto(444.62014251,13.70030147)(444.46014267,14.09030108)(444.36014404,14.55030273)
\curveto(444.3301428,14.65030052)(444.31014282,14.75030042)(444.30014404,14.85030273)
\lineto(444.30014404,15.16530273)
\curveto(444.29014284,15.20529996)(444.29014284,15.23529993)(444.30014404,15.25530273)
\curveto(444.31014282,15.28529988)(444.31514281,15.32029985)(444.31514404,15.36030273)
\curveto(444.31514281,15.44029973)(444.32014281,15.52029965)(444.33014404,15.60030273)
\curveto(444.34014279,15.69029948)(444.34514278,15.77529939)(444.34514404,15.85530273)
\curveto(444.35514277,15.90529926)(444.36014277,15.94529922)(444.36014404,15.97530273)
\curveto(444.37014276,16.01529915)(444.37514275,16.06029911)(444.37514404,16.11030273)
\curveto(444.37514275,16.16029901)(444.38514274,16.24529892)(444.40514404,16.36530273)
\curveto(444.43514269,16.49529867)(444.46514266,16.59029858)(444.49514404,16.65030273)
\curveto(444.53514259,16.72029845)(444.55514257,16.79029838)(444.55514404,16.86030273)
\curveto(444.55514257,16.93029824)(444.57514255,17.00029817)(444.61514404,17.07030273)
\curveto(444.63514249,17.12029805)(444.65014248,17.16029801)(444.66014404,17.19030273)
\curveto(444.67014246,17.23029794)(444.68514244,17.27529789)(444.70514404,17.32530273)
\curveto(444.76514236,17.44529772)(444.81514231,17.5652976)(444.85514404,17.68530273)
\curveto(444.90514222,17.80529736)(444.97014216,17.92029725)(445.05014404,18.03030273)
\curveto(445.27014186,18.40029677)(445.51514161,18.73029644)(445.78514404,19.02030273)
\curveto(446.06514106,19.32029585)(446.38014075,19.5702956)(446.73014404,19.77030273)
\curveto(446.86014027,19.85029532)(446.99514013,19.91529525)(447.13514404,19.96530273)
\lineto(447.58514404,20.14530273)
\curveto(447.71513941,20.19529497)(447.85013928,20.22529494)(447.99014404,20.23530273)
\curveto(448.130139,20.25529491)(448.27513885,20.28529488)(448.42514404,20.32530273)
\lineto(448.62014404,20.32530273)
\lineto(448.83014404,20.35530273)
\curveto(449.72013741,20.3652948)(450.42013671,20.18029499)(450.93014404,19.80030273)
\curveto(451.45013568,19.42029575)(451.77513535,18.92529624)(451.90514404,18.31530273)
\curveto(451.93513519,18.21529695)(451.95513517,18.11529705)(451.96514404,18.01530273)
\curveto(451.97513515,17.91529725)(451.99013514,17.81029736)(452.01014404,17.70030273)
\curveto(452.02013511,17.59029758)(452.02013511,17.4702977)(452.01014404,17.34030273)
\lineto(452.01014404,16.96530273)
\curveto(452.01013512,16.91529825)(452.00013513,16.86029831)(451.98014404,16.80030273)
\curveto(451.97013516,16.75029842)(451.96513516,16.70029847)(451.96514404,16.65030273)
\moveto(450.46514404,15.79530273)
\curveto(450.49513663,15.8652993)(450.51513661,15.94529922)(450.52514404,16.03530273)
\curveto(450.54513658,16.12529904)(450.56013657,16.21029896)(450.57014404,16.29030273)
\curveto(450.65013648,16.68029849)(450.68513644,17.01029816)(450.67514404,17.28030273)
\curveto(450.65513647,17.36029781)(450.64013649,17.44029773)(450.63014404,17.52030273)
\curveto(450.6301365,17.60029757)(450.6251365,17.67529749)(450.61514404,17.74530273)
\curveto(450.46513666,18.39529677)(450.11013702,18.84529632)(449.55014404,19.09530273)
\curveto(449.48013765,19.12529604)(449.40513772,19.14529602)(449.32514404,19.15530273)
\curveto(449.25513787,19.17529599)(449.18013795,19.19529597)(449.10014404,19.21530273)
\curveto(449.0301381,19.23529593)(448.95013818,19.24529592)(448.86014404,19.24530273)
\lineto(448.59014404,19.24530273)
\lineto(448.30514404,19.20030273)
\curveto(448.20513892,19.18029599)(448.11013902,19.15529601)(448.02014404,19.12530273)
\curveto(447.9301392,19.10529606)(447.84013929,19.07529609)(447.75014404,19.03530273)
\curveto(447.68013945,19.01529615)(447.61013952,18.98529618)(447.54014404,18.94530273)
\curveto(447.47013966,18.90529626)(447.40513972,18.8652963)(447.34514404,18.82530273)
\curveto(447.07514005,18.65529651)(446.84014029,18.45029672)(446.64014404,18.21030273)
\curveto(446.44014069,17.9702972)(446.25514087,17.69029748)(446.08514404,17.37030273)
\curveto(446.03514109,17.2702979)(445.99514113,17.165298)(445.96514404,17.05530273)
\curveto(445.93514119,16.95529821)(445.89514123,16.85029832)(445.84514404,16.74030273)
\curveto(445.83514129,16.70029847)(445.82014131,16.63529853)(445.80014404,16.54530273)
\curveto(445.78014135,16.51529865)(445.77014136,16.48029869)(445.77014404,16.44030273)
\curveto(445.77014136,16.40029877)(445.76514136,16.35529881)(445.75514404,16.30530273)
\lineto(445.69514404,16.00530273)
\curveto(445.67514145,15.90529926)(445.66514146,15.81529935)(445.66514404,15.73530273)
\lineto(445.66514404,15.55530273)
\curveto(445.66514146,15.45529971)(445.66014147,15.35529981)(445.65014404,15.25530273)
\curveto(445.65014148,15.1653)(445.66014147,15.08030009)(445.68014404,15.00030273)
\curveto(445.7301414,14.76030041)(445.80014133,14.53530063)(445.89014404,14.32530273)
\curveto(445.99014114,14.11530105)(446.125141,13.94030123)(446.29514404,13.80030273)
\curveto(446.34514078,13.7703014)(446.38514074,13.74530142)(446.41514404,13.72530273)
\curveto(446.45514067,13.70530146)(446.49514063,13.68030149)(446.53514404,13.65030273)
\curveto(446.60514052,13.60030157)(446.68514044,13.55530161)(446.77514404,13.51530273)
\curveto(446.86514026,13.48530168)(446.96014017,13.45530171)(447.06014404,13.42530273)
\curveto(447.11014002,13.40530176)(447.15513997,13.39530177)(447.19514404,13.39530273)
\curveto(447.24513988,13.40530176)(447.29513983,13.40530176)(447.34514404,13.39530273)
\curveto(447.37513975,13.38530178)(447.43513969,13.37530179)(447.52514404,13.36530273)
\curveto(447.61513951,13.35530181)(447.69013944,13.36030181)(447.75014404,13.38030273)
\curveto(447.79013934,13.39030178)(447.8301393,13.39030178)(447.87014404,13.38030273)
\curveto(447.91013922,13.38030179)(447.95013918,13.39030178)(447.99014404,13.41030273)
\curveto(448.07013906,13.43030174)(448.15013898,13.44530172)(448.23014404,13.45530273)
\curveto(448.32013881,13.47530169)(448.40513872,13.50030167)(448.48514404,13.53030273)
\curveto(448.84513828,13.6703015)(449.15513797,13.8653013)(449.41514404,14.11530273)
\curveto(449.67513745,14.3653008)(449.91013722,14.66030051)(450.12014404,15.00030273)
\curveto(450.20013693,15.12030005)(450.26013687,15.24529992)(450.30014404,15.37530273)
\curveto(450.34013679,15.51529965)(450.39513673,15.65529951)(450.46514404,15.79530273)
}
}
{
\newrgbcolor{curcolor}{0 0 0}
\pscustom[linestyle=none,fillstyle=solid,fillcolor=curcolor]
{
\newpath
\moveto(456.63342529,20.35530273)
\curveto(457.35341964,20.3652948)(457.93841905,20.28029489)(458.38842529,20.10030273)
\curveto(458.84841814,19.93029524)(459.16841782,19.62529554)(459.34842529,19.18530273)
\curveto(459.39841759,19.07529609)(459.42841756,18.96029621)(459.43842529,18.84030273)
\curveto(459.45841753,18.73029644)(459.47341752,18.60529656)(459.48342529,18.46530273)
\curveto(459.4934175,18.39529677)(459.48341751,18.32029685)(459.45342529,18.24030273)
\curveto(459.43341756,18.170297)(459.40841758,18.11529705)(459.37842529,18.07530273)
\curveto(459.35841763,18.05529711)(459.32841766,18.03529713)(459.28842529,18.01530273)
\curveto(459.25841773,18.00529716)(459.23341776,17.99029718)(459.21342529,17.97030273)
\curveto(459.15341784,17.95029722)(459.09841789,17.94529722)(459.04842529,17.95530273)
\curveto(459.00841798,17.9652972)(458.96341803,17.9652972)(458.91342529,17.95530273)
\curveto(458.82341817,17.93529723)(458.71341828,17.93029724)(458.58342529,17.94030273)
\curveto(458.46341853,17.96029721)(458.37841861,17.98529718)(458.32842529,18.01530273)
\curveto(458.25841873,18.0652971)(458.21841877,18.13029704)(458.20842529,18.21030273)
\curveto(458.20841878,18.30029687)(458.1884188,18.38529678)(458.14842529,18.46530273)
\curveto(458.09841889,18.62529654)(458.00341899,18.7702964)(457.86342529,18.90030273)
\curveto(457.77341922,18.98029619)(457.66341933,19.04029613)(457.53342529,19.08030273)
\curveto(457.41341958,19.12029605)(457.28341971,19.16029601)(457.14342529,19.20030273)
\curveto(457.10341989,19.22029595)(457.05341994,19.22529594)(456.99342529,19.21530273)
\curveto(456.94342005,19.21529595)(456.89842009,19.22029595)(456.85842529,19.23030273)
\curveto(456.79842019,19.25029592)(456.72342027,19.26029591)(456.63342529,19.26030273)
\curveto(456.54342045,19.26029591)(456.46842052,19.25029592)(456.40842529,19.23030273)
\lineto(456.31842529,19.23030273)
\curveto(456.25842073,19.22029595)(456.20342079,19.21029596)(456.15342529,19.20030273)
\curveto(456.10342089,19.20029597)(456.05342094,19.19529597)(456.00342529,19.18530273)
\curveto(455.73342126,19.12529604)(455.49842149,19.04029613)(455.29842529,18.93030273)
\curveto(455.10842188,18.82029635)(454.95842203,18.63529653)(454.84842529,18.37530273)
\curveto(454.81842217,18.30529686)(454.80342219,18.23529693)(454.80342529,18.16530273)
\curveto(454.80342219,18.09529707)(454.80842218,18.03529713)(454.81842529,17.98530273)
\curveto(454.84842214,17.83529733)(454.89842209,17.72529744)(454.96842529,17.65530273)
\curveto(455.03842195,17.59529757)(455.13342186,17.52529764)(455.25342529,17.44530273)
\curveto(455.3934216,17.34529782)(455.55842143,17.2702979)(455.74842529,17.22030273)
\curveto(455.93842105,17.18029799)(456.12842086,17.13029804)(456.31842529,17.07030273)
\curveto(456.43842055,17.03029814)(456.55842043,17.00029817)(456.67842529,16.98030273)
\curveto(456.80842018,16.96029821)(456.93342006,16.93029824)(457.05342529,16.89030273)
\curveto(457.25341974,16.83029834)(457.44841954,16.7702984)(457.63842529,16.71030273)
\curveto(457.82841916,16.66029851)(458.01341898,16.59529857)(458.19342529,16.51530273)
\curveto(458.24341875,16.49529867)(458.2884187,16.47529869)(458.32842529,16.45530273)
\curveto(458.37841861,16.43529873)(458.42841856,16.41029876)(458.47842529,16.38030273)
\curveto(458.64841834,16.26029891)(458.7934182,16.12529904)(458.91342529,15.97530273)
\curveto(459.03341796,15.82529934)(459.12341787,15.63529953)(459.18342529,15.40530273)
\lineto(459.18342529,15.12030273)
\curveto(459.18341781,15.05030012)(459.17841781,14.97530019)(459.16842529,14.89530273)
\curveto(459.15841783,14.82530034)(459.14841784,14.74530042)(459.13842529,14.65530273)
\lineto(459.10842529,14.50530273)
\curveto(459.06841792,14.43530073)(459.03841795,14.3653008)(459.01842529,14.29530273)
\curveto(459.00841798,14.22530094)(458.988418,14.15530101)(458.95842529,14.08530273)
\curveto(458.90841808,13.97530119)(458.85341814,13.8703013)(458.79342529,13.77030273)
\curveto(458.73341826,13.6703015)(458.66841832,13.58030159)(458.59842529,13.50030273)
\curveto(458.3884186,13.24030193)(458.14341885,13.03030214)(457.86342529,12.87030273)
\curveto(457.58341941,12.72030245)(457.27841971,12.59030258)(456.94842529,12.48030273)
\curveto(456.84842014,12.45030272)(456.74842024,12.43030274)(456.64842529,12.42030273)
\curveto(456.54842044,12.40030277)(456.45342054,12.37530279)(456.36342529,12.34530273)
\curveto(456.25342074,12.32530284)(456.14842084,12.31530285)(456.04842529,12.31530273)
\curveto(455.94842104,12.31530285)(455.84842114,12.30530286)(455.74842529,12.28530273)
\lineto(455.59842529,12.28530273)
\curveto(455.54842144,12.27530289)(455.47842151,12.2703029)(455.38842529,12.27030273)
\curveto(455.29842169,12.2703029)(455.22842176,12.27530289)(455.17842529,12.28530273)
\lineto(455.01342529,12.28530273)
\curveto(454.95342204,12.30530286)(454.8884221,12.31530285)(454.81842529,12.31530273)
\curveto(454.74842224,12.30530286)(454.6934223,12.31030286)(454.65342529,12.33030273)
\curveto(454.60342239,12.34030283)(454.53842245,12.34530282)(454.45842529,12.34530273)
\curveto(454.37842261,12.3653028)(454.30342269,12.38530278)(454.23342529,12.40530273)
\curveto(454.16342283,12.41530275)(454.0884229,12.43530273)(454.00842529,12.46530273)
\curveto(453.71842327,12.5653026)(453.47342352,12.69030248)(453.27342529,12.84030273)
\curveto(453.07342392,12.99030218)(452.91342408,13.18530198)(452.79342529,13.42530273)
\curveto(452.73342426,13.55530161)(452.68342431,13.69030148)(452.64342529,13.83030273)
\curveto(452.61342438,13.9703012)(452.5934244,14.12530104)(452.58342529,14.29530273)
\curveto(452.57342442,14.35530081)(452.57842441,14.42530074)(452.59842529,14.50530273)
\curveto(452.61842437,14.59530057)(452.64342435,14.6653005)(452.67342529,14.71530273)
\curveto(452.71342428,14.75530041)(452.77342422,14.79530037)(452.85342529,14.83530273)
\curveto(452.90342409,14.85530031)(452.97342402,14.8653003)(453.06342529,14.86530273)
\curveto(453.16342383,14.87530029)(453.25342374,14.87530029)(453.33342529,14.86530273)
\curveto(453.42342357,14.85530031)(453.50842348,14.84030033)(453.58842529,14.82030273)
\curveto(453.67842331,14.81030036)(453.73342326,14.79530037)(453.75342529,14.77530273)
\curveto(453.81342318,14.72530044)(453.84342315,14.65030052)(453.84342529,14.55030273)
\curveto(453.85342314,14.46030071)(453.87342312,14.37530079)(453.90342529,14.29530273)
\curveto(453.95342304,14.07530109)(454.05342294,13.90530126)(454.20342529,13.78530273)
\curveto(454.30342269,13.69530147)(454.42342257,13.62530154)(454.56342529,13.57530273)
\curveto(454.70342229,13.52530164)(454.85342214,13.47530169)(455.01342529,13.42530273)
\lineto(455.32842529,13.38030273)
\lineto(455.41842529,13.38030273)
\curveto(455.47842151,13.36030181)(455.56342143,13.35030182)(455.67342529,13.35030273)
\curveto(455.7934212,13.35030182)(455.89842109,13.36030181)(455.98842529,13.38030273)
\curveto(456.05842093,13.38030179)(456.11342088,13.38530178)(456.15342529,13.39530273)
\curveto(456.21342078,13.40530176)(456.27342072,13.41030176)(456.33342529,13.41030273)
\curveto(456.3934206,13.42030175)(456.44842054,13.43030174)(456.49842529,13.44030273)
\curveto(456.80842018,13.52030165)(457.05841993,13.62530154)(457.24842529,13.75530273)
\curveto(457.44841954,13.88530128)(457.61341938,14.10530106)(457.74342529,14.41530273)
\curveto(457.77341922,14.4653007)(457.7884192,14.52030065)(457.78842529,14.58030273)
\curveto(457.79841919,14.64030053)(457.79841919,14.68530048)(457.78842529,14.71530273)
\curveto(457.77841921,14.90530026)(457.73841925,15.04530012)(457.66842529,15.13530273)
\curveto(457.59841939,15.23529993)(457.50341949,15.32529984)(457.38342529,15.40530273)
\curveto(457.30341969,15.4652997)(457.20841978,15.51529965)(457.09842529,15.55530273)
\lineto(456.79842529,15.67530273)
\curveto(456.76842022,15.68529948)(456.73842025,15.69029948)(456.70842529,15.69030273)
\curveto(456.6884203,15.69029948)(456.66842032,15.70029947)(456.64842529,15.72030273)
\curveto(456.32842066,15.83029934)(455.988421,15.91029926)(455.62842529,15.96030273)
\curveto(455.27842171,16.02029915)(454.95842203,16.11529905)(454.66842529,16.24530273)
\curveto(454.57842241,16.28529888)(454.4884225,16.32029885)(454.39842529,16.35030273)
\curveto(454.31842267,16.38029879)(454.24342275,16.42029875)(454.17342529,16.47030273)
\curveto(454.00342299,16.58029859)(453.85342314,16.70529846)(453.72342529,16.84530273)
\curveto(453.5934234,16.98529818)(453.50342349,17.16029801)(453.45342529,17.37030273)
\curveto(453.43342356,17.44029773)(453.42342357,17.51029766)(453.42342529,17.58030273)
\lineto(453.42342529,17.80530273)
\curveto(453.41342358,17.92529724)(453.42842356,18.06029711)(453.46842529,18.21030273)
\curveto(453.50842348,18.3702968)(453.54842344,18.50529666)(453.58842529,18.61530273)
\curveto(453.61842337,18.6652965)(453.63842335,18.70529646)(453.64842529,18.73530273)
\curveto(453.66842332,18.77529639)(453.6934233,18.81529635)(453.72342529,18.85530273)
\curveto(453.85342314,19.08529608)(454.01342298,19.28529588)(454.20342529,19.45530273)
\curveto(454.3934226,19.62529554)(454.60342239,19.77529539)(454.83342529,19.90530273)
\curveto(454.993422,19.99529517)(455.16842182,20.0652951)(455.35842529,20.11530273)
\curveto(455.55842143,20.17529499)(455.76342123,20.23029494)(455.97342529,20.28030273)
\curveto(456.04342095,20.29029488)(456.10842088,20.30029487)(456.16842529,20.31030273)
\curveto(456.23842075,20.32029485)(456.31342068,20.33029484)(456.39342529,20.34030273)
\curveto(456.43342056,20.35029482)(456.47342052,20.35029482)(456.51342529,20.34030273)
\curveto(456.56342043,20.33029484)(456.60342039,20.33529483)(456.63342529,20.35530273)
}
}
{
\newrgbcolor{curcolor}{0 0 0}
\pscustom[linewidth=1,linecolor=curcolor]
{
\newpath
\moveto(86.01786,82.52252)
\lineto(723.99776,82.52252)
}
}
{
\newrgbcolor{curcolor}{0 0 0}
\pscustom[linewidth=1,linecolor=curcolor]
{
\newpath
\moveto(86.01786,156.51623)
\lineto(723.99776,156.51623)
}
}
{
\newrgbcolor{curcolor}{0 0 0}
\pscustom[linewidth=1,linecolor=curcolor]
{
\newpath
\moveto(86.01786,230.55822)
\lineto(723.99776,230.55822)
}
}
{
\newrgbcolor{curcolor}{0 0 0}
\pscustom[linewidth=1,linecolor=curcolor]
{
\newpath
\moveto(86.01786,305.62335)
\lineto(723.99776,305.62335)
}
}
{
\newrgbcolor{curcolor}{0 0 0}
\pscustom[linewidth=1,linecolor=curcolor]
{
\newpath
\moveto(86.01786,379.589017)
\lineto(723.99776,379.589017)
}
}
{
\newrgbcolor{curcolor}{0 0 0}
\pscustom[linestyle=none,fillstyle=solid,fillcolor=curcolor]
{
\newpath
\moveto(87.44285278,415.81210297)
\lineto(88.73285278,415.81210297)
\curveto(88.84284996,415.81209229)(88.94784985,415.80709229)(89.04785278,415.79710297)
\curveto(89.14784965,415.7970923)(89.22284958,415.76209234)(89.27285278,415.69210297)
\curveto(89.32284948,415.62209248)(89.34784945,415.53209257)(89.34785278,415.42210297)
\curveto(89.35784944,415.31209279)(89.36284944,415.19209291)(89.36285278,415.06210297)
\lineto(89.36285278,413.75710297)
\lineto(89.36285278,408.55210297)
\lineto(89.36285278,406.09210297)
\lineto(89.36285278,405.65710297)
\curveto(89.37284943,405.4971026)(89.35284945,405.37710272)(89.30285278,405.29710297)
\curveto(89.26284954,405.22710287)(89.17284963,405.17210293)(89.03285278,405.13210297)
\curveto(88.96284984,405.11210299)(88.88784991,405.10710299)(88.80785278,405.11710297)
\curveto(88.72785007,405.12710297)(88.64785015,405.13210297)(88.56785278,405.13210297)
\lineto(87.68285278,405.13210297)
\curveto(87.57285123,405.13210297)(87.46785133,405.13710296)(87.36785278,405.14710297)
\curveto(87.27785152,405.15710294)(87.2028516,405.18710291)(87.14285278,405.23710297)
\curveto(87.09285171,405.28710281)(87.06285174,405.36210274)(87.05285278,405.46210297)
\curveto(87.04285176,405.56210254)(87.03785176,405.66710243)(87.03785278,405.77710297)
\lineto(87.03785278,407.08210297)
\lineto(87.03785278,412.55710297)
\lineto(87.03785278,414.74710297)
\curveto(87.03785176,414.88709321)(87.03285177,415.05209305)(87.02285278,415.24210297)
\curveto(87.02285178,415.43209267)(87.04785175,415.56709253)(87.09785278,415.64710297)
\curveto(87.13785166,415.70709239)(87.2028516,415.75709234)(87.29285278,415.79710297)
\curveto(87.32285148,415.7970923)(87.34785145,415.7970923)(87.36785278,415.79710297)
\curveto(87.3978514,415.80709229)(87.42285138,415.81209229)(87.44285278,415.81210297)
}
}
{
\newrgbcolor{curcolor}{0 0 0}
\pscustom[linestyle=none,fillstyle=solid,fillcolor=curcolor]
{
\newpath
\moveto(95.64668091,413.06710297)
\curveto(96.2466751,413.08709501)(96.7466746,413.0020951)(97.14668091,412.81210297)
\curveto(97.5466738,412.62209548)(97.86167349,412.34209576)(98.09168091,411.97210297)
\curveto(98.16167319,411.86209624)(98.21667313,411.74209636)(98.25668091,411.61210297)
\curveto(98.29667305,411.49209661)(98.33667301,411.36709673)(98.37668091,411.23710297)
\curveto(98.39667295,411.15709694)(98.40667294,411.08209702)(98.40668091,411.01210297)
\curveto(98.41667293,410.94209716)(98.43167292,410.87209723)(98.45168091,410.80210297)
\curveto(98.4516729,410.74209736)(98.45667289,410.7020974)(98.46668091,410.68210297)
\curveto(98.48667286,410.54209756)(98.49667285,410.3970977)(98.49668091,410.24710297)
\lineto(98.49668091,409.81210297)
\lineto(98.49668091,408.47710297)
\lineto(98.49668091,406.04710297)
\curveto(98.49667285,405.85710224)(98.49167286,405.67210243)(98.48168091,405.49210297)
\curveto(98.48167287,405.32210278)(98.41167294,405.21210289)(98.27168091,405.16210297)
\curveto(98.21167314,405.14210296)(98.14167321,405.13210297)(98.06168091,405.13210297)
\lineto(97.82168091,405.13210297)
\lineto(97.01168091,405.13210297)
\curveto(96.89167446,405.13210297)(96.78167457,405.13710296)(96.68168091,405.14710297)
\curveto(96.59167476,405.16710293)(96.52167483,405.21210289)(96.47168091,405.28210297)
\curveto(96.43167492,405.34210276)(96.40667494,405.41710268)(96.39668091,405.50710297)
\lineto(96.39668091,405.82210297)
\lineto(96.39668091,406.87210297)
\lineto(96.39668091,409.10710297)
\curveto(96.39667495,409.47709862)(96.38167497,409.81709828)(96.35168091,410.12710297)
\curveto(96.32167503,410.44709765)(96.23167512,410.71709738)(96.08168091,410.93710297)
\curveto(95.94167541,411.13709696)(95.73667561,411.27709682)(95.46668091,411.35710297)
\curveto(95.41667593,411.37709672)(95.36167599,411.38709671)(95.30168091,411.38710297)
\curveto(95.2516761,411.38709671)(95.19667615,411.3970967)(95.13668091,411.41710297)
\curveto(95.08667626,411.42709667)(95.02167633,411.42709667)(94.94168091,411.41710297)
\curveto(94.87167648,411.41709668)(94.81667653,411.41209669)(94.77668091,411.40210297)
\curveto(94.73667661,411.39209671)(94.70167665,411.38709671)(94.67168091,411.38710297)
\curveto(94.64167671,411.38709671)(94.61167674,411.38209672)(94.58168091,411.37210297)
\curveto(94.351677,411.31209679)(94.16667718,411.23209687)(94.02668091,411.13210297)
\curveto(93.70667764,410.9020972)(93.51667783,410.56709753)(93.45668091,410.12710297)
\curveto(93.39667795,409.68709841)(93.36667798,409.19209891)(93.36668091,408.64210297)
\lineto(93.36668091,406.76710297)
\lineto(93.36668091,405.85210297)
\lineto(93.36668091,405.58210297)
\curveto(93.36667798,405.49210261)(93.351678,405.41710268)(93.32168091,405.35710297)
\curveto(93.27167808,405.24710285)(93.19167816,405.18210292)(93.08168091,405.16210297)
\curveto(92.97167838,405.14210296)(92.83667851,405.13210297)(92.67668091,405.13210297)
\lineto(91.92668091,405.13210297)
\curveto(91.81667953,405.13210297)(91.70667964,405.13710296)(91.59668091,405.14710297)
\curveto(91.48667986,405.15710294)(91.40667994,405.19210291)(91.35668091,405.25210297)
\curveto(91.28668006,405.34210276)(91.2516801,405.47210263)(91.25168091,405.64210297)
\curveto(91.26168009,405.81210229)(91.26668008,405.97210213)(91.26668091,406.12210297)
\lineto(91.26668091,408.16210297)
\lineto(91.26668091,411.46210297)
\lineto(91.26668091,412.22710297)
\lineto(91.26668091,412.52710297)
\curveto(91.27668007,412.61709548)(91.30668004,412.69209541)(91.35668091,412.75210297)
\curveto(91.37667997,412.78209532)(91.40667994,412.8020953)(91.44668091,412.81210297)
\curveto(91.49667985,412.83209527)(91.5466798,412.84709525)(91.59668091,412.85710297)
\lineto(91.67168091,412.85710297)
\curveto(91.72167963,412.86709523)(91.77167958,412.87209523)(91.82168091,412.87210297)
\lineto(91.98668091,412.87210297)
\lineto(92.61668091,412.87210297)
\curveto(92.69667865,412.87209523)(92.77167858,412.86709523)(92.84168091,412.85710297)
\curveto(92.92167843,412.85709524)(92.99167836,412.84709525)(93.05168091,412.82710297)
\curveto(93.12167823,412.7970953)(93.16667818,412.75209535)(93.18668091,412.69210297)
\curveto(93.21667813,412.63209547)(93.24167811,412.56209554)(93.26168091,412.48210297)
\curveto(93.27167808,412.44209566)(93.27167808,412.40709569)(93.26168091,412.37710297)
\curveto(93.26167809,412.34709575)(93.27167808,412.31709578)(93.29168091,412.28710297)
\curveto(93.31167804,412.23709586)(93.32667802,412.20709589)(93.33668091,412.19710297)
\curveto(93.35667799,412.18709591)(93.38167797,412.17209593)(93.41168091,412.15210297)
\curveto(93.52167783,412.14209596)(93.61167774,412.17709592)(93.68168091,412.25710297)
\curveto(93.7516776,412.34709575)(93.82667752,412.41709568)(93.90668091,412.46710297)
\curveto(94.17667717,412.66709543)(94.47667687,412.82709527)(94.80668091,412.94710297)
\curveto(94.89667645,412.97709512)(94.98667636,412.9970951)(95.07668091,413.00710297)
\curveto(95.17667617,413.01709508)(95.28167607,413.03209507)(95.39168091,413.05210297)
\curveto(95.42167593,413.06209504)(95.46667588,413.06209504)(95.52668091,413.05210297)
\curveto(95.58667576,413.05209505)(95.62667572,413.05709504)(95.64668091,413.06710297)
}
}
{
\newrgbcolor{curcolor}{0 0 0}
\pscustom[linestyle=none,fillstyle=solid,fillcolor=curcolor]
{
\newpath
\moveto(107.71793091,405.98710297)
\lineto(107.71793091,405.56710297)
\curveto(107.71792254,405.43710266)(107.68792257,405.33210277)(107.62793091,405.25210297)
\curveto(107.57792268,405.2021029)(107.51292274,405.16710293)(107.43293091,405.14710297)
\curveto(107.3529229,405.13710296)(107.26292299,405.13210297)(107.16293091,405.13210297)
\lineto(106.33793091,405.13210297)
\lineto(106.05293091,405.13210297)
\curveto(105.97292428,405.14210296)(105.90792435,405.16710293)(105.85793091,405.20710297)
\curveto(105.78792447,405.25710284)(105.74792451,405.32210278)(105.73793091,405.40210297)
\curveto(105.72792453,405.48210262)(105.70792455,405.56210254)(105.67793091,405.64210297)
\curveto(105.6579246,405.66210244)(105.63792462,405.67710242)(105.61793091,405.68710297)
\curveto(105.60792465,405.70710239)(105.59292466,405.72710237)(105.57293091,405.74710297)
\curveto(105.46292479,405.74710235)(105.38292487,405.72210238)(105.33293091,405.67210297)
\lineto(105.18293091,405.52210297)
\curveto(105.11292514,405.47210263)(105.04792521,405.42710267)(104.98793091,405.38710297)
\curveto(104.92792533,405.35710274)(104.86292539,405.31710278)(104.79293091,405.26710297)
\curveto(104.7529255,405.24710285)(104.70792555,405.22710287)(104.65793091,405.20710297)
\curveto(104.61792564,405.18710291)(104.57292568,405.16710293)(104.52293091,405.14710297)
\curveto(104.38292587,405.097103)(104.23292602,405.05210305)(104.07293091,405.01210297)
\curveto(104.02292623,404.99210311)(103.97792628,404.98210312)(103.93793091,404.98210297)
\curveto(103.89792636,404.98210312)(103.8579264,404.97710312)(103.81793091,404.96710297)
\lineto(103.68293091,404.96710297)
\curveto(103.6529266,404.95710314)(103.61292664,404.95210315)(103.56293091,404.95210297)
\lineto(103.42793091,404.95210297)
\curveto(103.36792689,404.93210317)(103.27792698,404.92710317)(103.15793091,404.93710297)
\curveto(103.03792722,404.93710316)(102.9529273,404.94710315)(102.90293091,404.96710297)
\curveto(102.83292742,404.98710311)(102.76792749,404.9971031)(102.70793091,404.99710297)
\curveto(102.6579276,404.98710311)(102.60292765,404.99210311)(102.54293091,405.01210297)
\lineto(102.18293091,405.13210297)
\curveto(102.07292818,405.16210294)(101.96292829,405.2021029)(101.85293091,405.25210297)
\curveto(101.50292875,405.4021027)(101.18792907,405.63210247)(100.90793091,405.94210297)
\curveto(100.63792962,406.26210184)(100.42292983,406.5971015)(100.26293091,406.94710297)
\curveto(100.21293004,407.05710104)(100.17293008,407.16210094)(100.14293091,407.26210297)
\curveto(100.11293014,407.37210073)(100.07793018,407.48210062)(100.03793091,407.59210297)
\curveto(100.02793023,407.63210047)(100.02293023,407.66710043)(100.02293091,407.69710297)
\curveto(100.02293023,407.73710036)(100.01293024,407.78210032)(99.99293091,407.83210297)
\curveto(99.97293028,407.91210019)(99.9529303,407.9971001)(99.93293091,408.08710297)
\curveto(99.92293033,408.18709991)(99.90793035,408.28709981)(99.88793091,408.38710297)
\curveto(99.87793038,408.41709968)(99.87293038,408.45209965)(99.87293091,408.49210297)
\curveto(99.88293037,408.53209957)(99.88293037,408.56709953)(99.87293091,408.59710297)
\lineto(99.87293091,408.73210297)
\curveto(99.87293038,408.78209932)(99.86793039,408.83209927)(99.85793091,408.88210297)
\curveto(99.84793041,408.93209917)(99.84293041,408.98709911)(99.84293091,409.04710297)
\curveto(99.84293041,409.11709898)(99.84793041,409.17209893)(99.85793091,409.21210297)
\curveto(99.86793039,409.26209884)(99.87293038,409.30709879)(99.87293091,409.34710297)
\lineto(99.87293091,409.49710297)
\curveto(99.88293037,409.54709855)(99.88293037,409.59209851)(99.87293091,409.63210297)
\curveto(99.87293038,409.68209842)(99.88293037,409.73209837)(99.90293091,409.78210297)
\curveto(99.92293033,409.89209821)(99.93793032,409.9970981)(99.94793091,410.09710297)
\curveto(99.96793029,410.1970979)(99.99293026,410.2970978)(100.02293091,410.39710297)
\curveto(100.06293019,410.51709758)(100.09793016,410.63209747)(100.12793091,410.74210297)
\curveto(100.1579301,410.85209725)(100.19793006,410.96209714)(100.24793091,411.07210297)
\curveto(100.38792987,411.37209673)(100.56292969,411.65709644)(100.77293091,411.92710297)
\curveto(100.79292946,411.95709614)(100.81792944,411.98209612)(100.84793091,412.00210297)
\curveto(100.88792937,412.03209607)(100.91792934,412.06209604)(100.93793091,412.09210297)
\curveto(100.97792928,412.14209596)(101.01792924,412.18709591)(101.05793091,412.22710297)
\curveto(101.09792916,412.26709583)(101.14292911,412.30709579)(101.19293091,412.34710297)
\curveto(101.23292902,412.36709573)(101.26792899,412.39209571)(101.29793091,412.42210297)
\curveto(101.32792893,412.46209564)(101.36292889,412.49209561)(101.40293091,412.51210297)
\curveto(101.6529286,412.68209542)(101.94292831,412.82209528)(102.27293091,412.93210297)
\curveto(102.34292791,412.95209515)(102.41292784,412.96709513)(102.48293091,412.97710297)
\curveto(102.56292769,412.98709511)(102.64292761,413.0020951)(102.72293091,413.02210297)
\curveto(102.79292746,413.04209506)(102.88292737,413.05209505)(102.99293091,413.05210297)
\curveto(103.10292715,413.06209504)(103.21292704,413.06709503)(103.32293091,413.06710297)
\curveto(103.43292682,413.06709503)(103.53792672,413.06209504)(103.63793091,413.05210297)
\curveto(103.74792651,413.04209506)(103.83792642,413.02709507)(103.90793091,413.00710297)
\curveto(104.0579262,412.95709514)(104.20292605,412.91209519)(104.34293091,412.87210297)
\curveto(104.48292577,412.83209527)(104.61292564,412.77709532)(104.73293091,412.70710297)
\curveto(104.80292545,412.65709544)(104.86792539,412.60709549)(104.92793091,412.55710297)
\curveto(104.98792527,412.51709558)(105.0529252,412.47209563)(105.12293091,412.42210297)
\curveto(105.16292509,412.39209571)(105.21792504,412.35209575)(105.28793091,412.30210297)
\curveto(105.36792489,412.25209585)(105.44292481,412.25209585)(105.51293091,412.30210297)
\curveto(105.5529247,412.32209578)(105.57292468,412.35709574)(105.57293091,412.40710297)
\curveto(105.57292468,412.45709564)(105.58292467,412.50709559)(105.60293091,412.55710297)
\lineto(105.60293091,412.70710297)
\curveto(105.61292464,412.73709536)(105.61792464,412.77209533)(105.61793091,412.81210297)
\lineto(105.61793091,412.93210297)
\lineto(105.61793091,414.97210297)
\curveto(105.61792464,415.08209302)(105.61292464,415.2020929)(105.60293091,415.33210297)
\curveto(105.60292465,415.47209263)(105.62792463,415.57709252)(105.67793091,415.64710297)
\curveto(105.71792454,415.72709237)(105.79292446,415.77709232)(105.90293091,415.79710297)
\curveto(105.92292433,415.80709229)(105.94292431,415.80709229)(105.96293091,415.79710297)
\curveto(105.98292427,415.7970923)(106.00292425,415.8020923)(106.02293091,415.81210297)
\lineto(107.08793091,415.81210297)
\curveto(107.20792305,415.81209229)(107.31792294,415.80709229)(107.41793091,415.79710297)
\curveto(107.51792274,415.78709231)(107.59292266,415.74709235)(107.64293091,415.67710297)
\curveto(107.69292256,415.5970925)(107.71792254,415.49209261)(107.71793091,415.36210297)
\lineto(107.71793091,415.00210297)
\lineto(107.71793091,405.98710297)
\moveto(105.67793091,408.92710297)
\curveto(105.68792457,408.96709913)(105.68792457,409.00709909)(105.67793091,409.04710297)
\lineto(105.67793091,409.18210297)
\curveto(105.67792458,409.28209882)(105.67292458,409.38209872)(105.66293091,409.48210297)
\curveto(105.6529246,409.58209852)(105.63792462,409.67209843)(105.61793091,409.75210297)
\curveto(105.59792466,409.86209824)(105.57792468,409.96209814)(105.55793091,410.05210297)
\curveto(105.54792471,410.14209796)(105.52292473,410.22709787)(105.48293091,410.30710297)
\curveto(105.34292491,410.66709743)(105.13792512,410.95209715)(104.86793091,411.16210297)
\curveto(104.60792565,411.37209673)(104.22792603,411.47709662)(103.72793091,411.47710297)
\curveto(103.66792659,411.47709662)(103.58792667,411.46709663)(103.48793091,411.44710297)
\curveto(103.40792685,411.42709667)(103.33292692,411.40709669)(103.26293091,411.38710297)
\curveto(103.20292705,411.37709672)(103.14292711,411.35709674)(103.08293091,411.32710297)
\curveto(102.81292744,411.21709688)(102.60292765,411.04709705)(102.45293091,410.81710297)
\curveto(102.30292795,410.58709751)(102.18292807,410.32709777)(102.09293091,410.03710297)
\curveto(102.06292819,409.93709816)(102.04292821,409.83709826)(102.03293091,409.73710297)
\curveto(102.02292823,409.63709846)(102.00292825,409.53209857)(101.97293091,409.42210297)
\lineto(101.97293091,409.21210297)
\curveto(101.9529283,409.12209898)(101.94792831,408.9970991)(101.95793091,408.83710297)
\curveto(101.96792829,408.68709941)(101.98292827,408.57709952)(102.00293091,408.50710297)
\lineto(102.00293091,408.41710297)
\curveto(102.01292824,408.3970997)(102.01792824,408.37709972)(102.01793091,408.35710297)
\curveto(102.03792822,408.27709982)(102.0529282,408.2020999)(102.06293091,408.13210297)
\curveto(102.08292817,408.06210004)(102.10292815,407.98710011)(102.12293091,407.90710297)
\curveto(102.29292796,407.38710071)(102.58292767,407.0021011)(102.99293091,406.75210297)
\curveto(103.12292713,406.66210144)(103.30292695,406.59210151)(103.53293091,406.54210297)
\curveto(103.57292668,406.53210157)(103.63292662,406.52710157)(103.71293091,406.52710297)
\curveto(103.74292651,406.51710158)(103.78792647,406.50710159)(103.84793091,406.49710297)
\curveto(103.91792634,406.4971016)(103.97292628,406.5021016)(104.01293091,406.51210297)
\curveto(104.09292616,406.53210157)(104.17292608,406.54710155)(104.25293091,406.55710297)
\curveto(104.33292592,406.56710153)(104.41292584,406.58710151)(104.49293091,406.61710297)
\curveto(104.74292551,406.72710137)(104.94292531,406.86710123)(105.09293091,407.03710297)
\curveto(105.24292501,407.20710089)(105.37292488,407.42210068)(105.48293091,407.68210297)
\curveto(105.52292473,407.77210033)(105.5529247,407.86210024)(105.57293091,407.95210297)
\curveto(105.59292466,408.05210005)(105.61292464,408.15709994)(105.63293091,408.26710297)
\curveto(105.64292461,408.31709978)(105.64292461,408.36209974)(105.63293091,408.40210297)
\curveto(105.63292462,408.45209965)(105.64292461,408.5020996)(105.66293091,408.55210297)
\curveto(105.67292458,408.58209952)(105.67792458,408.61709948)(105.67793091,408.65710297)
\lineto(105.67793091,408.79210297)
\lineto(105.67793091,408.92710297)
}
}
{
\newrgbcolor{curcolor}{0 0 0}
\pscustom[linestyle=none,fillstyle=solid,fillcolor=curcolor]
{
\newpath
\moveto(111.39785278,415.72210297)
\curveto(111.46784983,415.64209246)(111.5028498,415.52209258)(111.50285278,415.36210297)
\lineto(111.50285278,414.89710297)
\lineto(111.50285278,414.49210297)
\curveto(111.5028498,414.35209375)(111.46784983,414.25709384)(111.39785278,414.20710297)
\curveto(111.33784996,414.15709394)(111.25785004,414.12709397)(111.15785278,414.11710297)
\curveto(111.06785023,414.10709399)(110.96785033,414.102094)(110.85785278,414.10210297)
\lineto(110.01785278,414.10210297)
\curveto(109.90785139,414.102094)(109.80785149,414.10709399)(109.71785278,414.11710297)
\curveto(109.63785166,414.12709397)(109.56785173,414.15709394)(109.50785278,414.20710297)
\curveto(109.46785183,414.23709386)(109.43785186,414.29209381)(109.41785278,414.37210297)
\curveto(109.40785189,414.46209364)(109.3978519,414.55709354)(109.38785278,414.65710297)
\lineto(109.38785278,414.98710297)
\curveto(109.3978519,415.097093)(109.4028519,415.19209291)(109.40285278,415.27210297)
\lineto(109.40285278,415.48210297)
\curveto(109.41285189,415.55209255)(109.43285187,415.61209249)(109.46285278,415.66210297)
\curveto(109.48285182,415.7020924)(109.50785179,415.73209237)(109.53785278,415.75210297)
\lineto(109.65785278,415.81210297)
\curveto(109.67785162,415.81209229)(109.7028516,415.81209229)(109.73285278,415.81210297)
\curveto(109.76285154,415.82209228)(109.78785151,415.82709227)(109.80785278,415.82710297)
\lineto(110.90285278,415.82710297)
\curveto(111.0028503,415.82709227)(111.0978502,415.82209228)(111.18785278,415.81210297)
\curveto(111.27785002,415.8020923)(111.34784995,415.77209233)(111.39785278,415.72210297)
\moveto(111.50285278,405.95710297)
\curveto(111.5028498,405.75710234)(111.4978498,405.58710251)(111.48785278,405.44710297)
\curveto(111.47784982,405.30710279)(111.38784991,405.21210289)(111.21785278,405.16210297)
\curveto(111.15785014,405.14210296)(111.09285021,405.13210297)(111.02285278,405.13210297)
\curveto(110.95285035,405.14210296)(110.87785042,405.14710295)(110.79785278,405.14710297)
\lineto(109.95785278,405.14710297)
\curveto(109.86785143,405.14710295)(109.77785152,405.15210295)(109.68785278,405.16210297)
\curveto(109.60785169,405.17210293)(109.54785175,405.2021029)(109.50785278,405.25210297)
\curveto(109.44785185,405.32210278)(109.41285189,405.40710269)(109.40285278,405.50710297)
\lineto(109.40285278,405.85210297)
\lineto(109.40285278,412.18210297)
\lineto(109.40285278,412.48210297)
\curveto(109.4028519,412.58209552)(109.42285188,412.66209544)(109.46285278,412.72210297)
\curveto(109.52285178,412.79209531)(109.60785169,412.83709526)(109.71785278,412.85710297)
\curveto(109.73785156,412.86709523)(109.76285154,412.86709523)(109.79285278,412.85710297)
\curveto(109.83285147,412.85709524)(109.86285144,412.86209524)(109.88285278,412.87210297)
\lineto(110.63285278,412.87210297)
\lineto(110.82785278,412.87210297)
\curveto(110.90785039,412.88209522)(110.97285033,412.88209522)(111.02285278,412.87210297)
\lineto(111.14285278,412.87210297)
\curveto(111.2028501,412.85209525)(111.25785004,412.83709526)(111.30785278,412.82710297)
\curveto(111.35784994,412.81709528)(111.3978499,412.78709531)(111.42785278,412.73710297)
\curveto(111.46784983,412.68709541)(111.48784981,412.61709548)(111.48785278,412.52710297)
\curveto(111.4978498,412.43709566)(111.5028498,412.34209576)(111.50285278,412.24210297)
\lineto(111.50285278,405.95710297)
}
}
{
\newrgbcolor{curcolor}{0 0 0}
\pscustom[linestyle=none,fillstyle=solid,fillcolor=curcolor]
{
\newpath
\moveto(116.73504028,413.08210297)
\curveto(117.54503512,413.102095)(118.22003445,412.98209512)(118.76004028,412.72210297)
\curveto(119.31003336,412.46209564)(119.74503292,412.09209601)(120.06504028,411.61210297)
\curveto(120.22503244,411.37209673)(120.34503232,411.097097)(120.42504028,410.78710297)
\curveto(120.44503222,410.73709736)(120.46003221,410.67209743)(120.47004028,410.59210297)
\curveto(120.49003218,410.51209759)(120.49003218,410.44209766)(120.47004028,410.38210297)
\curveto(120.43003224,410.27209783)(120.36003231,410.20709789)(120.26004028,410.18710297)
\curveto(120.16003251,410.17709792)(120.04003263,410.17209793)(119.90004028,410.17210297)
\lineto(119.12004028,410.17210297)
\lineto(118.83504028,410.17210297)
\curveto(118.74503392,410.17209793)(118.670034,410.19209791)(118.61004028,410.23210297)
\curveto(118.53003414,410.27209783)(118.47503419,410.33209777)(118.44504028,410.41210297)
\curveto(118.41503425,410.5020976)(118.37503429,410.59209751)(118.32504028,410.68210297)
\curveto(118.2650344,410.79209731)(118.20003447,410.89209721)(118.13004028,410.98210297)
\curveto(118.06003461,411.07209703)(117.98003469,411.15209695)(117.89004028,411.22210297)
\curveto(117.75003492,411.31209679)(117.59503507,411.38209672)(117.42504028,411.43210297)
\curveto(117.3650353,411.45209665)(117.30503536,411.46209664)(117.24504028,411.46210297)
\curveto(117.18503548,411.46209664)(117.13003554,411.47209663)(117.08004028,411.49210297)
\lineto(116.93004028,411.49210297)
\curveto(116.73003594,411.49209661)(116.5700361,411.47209663)(116.45004028,411.43210297)
\curveto(116.16003651,411.34209676)(115.92503674,411.2020969)(115.74504028,411.01210297)
\curveto(115.5650371,410.83209727)(115.42003725,410.61209749)(115.31004028,410.35210297)
\curveto(115.26003741,410.24209786)(115.22003745,410.12209798)(115.19004028,409.99210297)
\curveto(115.1700375,409.87209823)(115.14503752,409.74209836)(115.11504028,409.60210297)
\curveto(115.10503756,409.56209854)(115.10003757,409.52209858)(115.10004028,409.48210297)
\curveto(115.10003757,409.44209866)(115.09503757,409.4020987)(115.08504028,409.36210297)
\curveto(115.0650376,409.26209884)(115.05503761,409.12209898)(115.05504028,408.94210297)
\curveto(115.0650376,408.76209934)(115.08003759,408.62209948)(115.10004028,408.52210297)
\curveto(115.10003757,408.44209966)(115.10503756,408.38709971)(115.11504028,408.35710297)
\curveto(115.13503753,408.28709981)(115.14503752,408.21709988)(115.14504028,408.14710297)
\curveto(115.15503751,408.07710002)(115.1700375,408.00710009)(115.19004028,407.93710297)
\curveto(115.2700374,407.70710039)(115.3650373,407.4971006)(115.47504028,407.30710297)
\curveto(115.58503708,407.11710098)(115.72503694,406.95710114)(115.89504028,406.82710297)
\curveto(115.93503673,406.7971013)(115.99503667,406.76210134)(116.07504028,406.72210297)
\curveto(116.18503648,406.65210145)(116.29503637,406.60710149)(116.40504028,406.58710297)
\curveto(116.52503614,406.56710153)(116.670036,406.54710155)(116.84004028,406.52710297)
\lineto(116.93004028,406.52710297)
\curveto(116.9700357,406.52710157)(117.00003567,406.53210157)(117.02004028,406.54210297)
\lineto(117.15504028,406.54210297)
\curveto(117.22503544,406.56210154)(117.29003538,406.57710152)(117.35004028,406.58710297)
\curveto(117.42003525,406.60710149)(117.48503518,406.62710147)(117.54504028,406.64710297)
\curveto(117.84503482,406.77710132)(118.07503459,406.96710113)(118.23504028,407.21710297)
\curveto(118.27503439,407.26710083)(118.31003436,407.32210078)(118.34004028,407.38210297)
\curveto(118.3700343,407.45210065)(118.39503427,407.51210059)(118.41504028,407.56210297)
\curveto(118.45503421,407.67210043)(118.49003418,407.76710033)(118.52004028,407.84710297)
\curveto(118.55003412,407.93710016)(118.62003405,408.00710009)(118.73004028,408.05710297)
\curveto(118.82003385,408.0971)(118.9650337,408.11209999)(119.16504028,408.10210297)
\lineto(119.66004028,408.10210297)
\lineto(119.87004028,408.10210297)
\curveto(119.95003272,408.11209999)(120.01503265,408.10709999)(120.06504028,408.08710297)
\lineto(120.18504028,408.08710297)
\lineto(120.30504028,408.05710297)
\curveto(120.34503232,408.05710004)(120.37503229,408.04710005)(120.39504028,408.02710297)
\curveto(120.44503222,407.98710011)(120.47503219,407.92710017)(120.48504028,407.84710297)
\curveto(120.50503216,407.77710032)(120.50503216,407.7021004)(120.48504028,407.62210297)
\curveto(120.39503227,407.29210081)(120.28503238,406.9971011)(120.15504028,406.73710297)
\curveto(119.74503292,405.96710213)(119.09003358,405.43210267)(118.19004028,405.13210297)
\curveto(118.09003458,405.102103)(117.98503468,405.08210302)(117.87504028,405.07210297)
\curveto(117.7650349,405.05210305)(117.65503501,405.02710307)(117.54504028,404.99710297)
\curveto(117.48503518,404.98710311)(117.42503524,404.98210312)(117.36504028,404.98210297)
\curveto(117.30503536,404.98210312)(117.24503542,404.97710312)(117.18504028,404.96710297)
\lineto(117.02004028,404.96710297)
\curveto(116.9700357,404.94710315)(116.89503577,404.94210316)(116.79504028,404.95210297)
\curveto(116.69503597,404.95210315)(116.62003605,404.95710314)(116.57004028,404.96710297)
\curveto(116.49003618,404.98710311)(116.41503625,404.9971031)(116.34504028,404.99710297)
\curveto(116.28503638,404.98710311)(116.22003645,404.99210311)(116.15004028,405.01210297)
\lineto(116.00004028,405.04210297)
\curveto(115.95003672,405.04210306)(115.90003677,405.04710305)(115.85004028,405.05710297)
\curveto(115.74003693,405.08710301)(115.63503703,405.11710298)(115.53504028,405.14710297)
\curveto(115.43503723,405.17710292)(115.34003733,405.21210289)(115.25004028,405.25210297)
\curveto(114.78003789,405.45210265)(114.38503828,405.70710239)(114.06504028,406.01710297)
\curveto(113.74503892,406.33710176)(113.48503918,406.73210137)(113.28504028,407.20210297)
\curveto(113.23503943,407.29210081)(113.19503947,407.38710071)(113.16504028,407.48710297)
\lineto(113.07504028,407.81710297)
\curveto(113.0650396,407.85710024)(113.06003961,407.89210021)(113.06004028,407.92210297)
\curveto(113.06003961,407.96210014)(113.05003962,408.00710009)(113.03004028,408.05710297)
\curveto(113.01003966,408.12709997)(113.00003967,408.1970999)(113.00004028,408.26710297)
\curveto(113.00003967,408.34709975)(112.99003968,408.42209968)(112.97004028,408.49210297)
\lineto(112.97004028,408.74710297)
\curveto(112.95003972,408.7970993)(112.94003973,408.85209925)(112.94004028,408.91210297)
\curveto(112.94003973,408.98209912)(112.95003972,409.04209906)(112.97004028,409.09210297)
\curveto(112.98003969,409.14209896)(112.98003969,409.18709891)(112.97004028,409.22710297)
\curveto(112.96003971,409.26709883)(112.96003971,409.30709879)(112.97004028,409.34710297)
\curveto(112.99003968,409.41709868)(112.99503967,409.48209862)(112.98504028,409.54210297)
\curveto(112.98503968,409.6020985)(112.99503967,409.66209844)(113.01504028,409.72210297)
\curveto(113.0650396,409.9020982)(113.10503956,410.07209803)(113.13504028,410.23210297)
\curveto(113.1650395,410.4020977)(113.21003946,410.56709753)(113.27004028,410.72710297)
\curveto(113.49003918,411.23709686)(113.7650389,411.66209644)(114.09504028,412.00210297)
\curveto(114.43503823,412.34209576)(114.8650378,412.61709548)(115.38504028,412.82710297)
\curveto(115.52503714,412.88709521)(115.670037,412.92709517)(115.82004028,412.94710297)
\curveto(115.9700367,412.97709512)(116.12503654,413.01209509)(116.28504028,413.05210297)
\curveto(116.3650363,413.06209504)(116.44003623,413.06709503)(116.51004028,413.06710297)
\curveto(116.58003609,413.06709503)(116.65503601,413.07209503)(116.73504028,413.08210297)
}
}
{
\newrgbcolor{curcolor}{0 0 0}
\pscustom[linestyle=none,fillstyle=solid,fillcolor=curcolor]
{
\newpath
\moveto(128.82832153,405.73210297)
\curveto(128.84831368,405.62210248)(128.85831367,405.51210259)(128.85832153,405.40210297)
\curveto(128.86831366,405.29210281)(128.81831371,405.21710288)(128.70832153,405.17710297)
\curveto(128.64831388,405.14710295)(128.57831395,405.13210297)(128.49832153,405.13210297)
\lineto(128.25832153,405.13210297)
\lineto(127.44832153,405.13210297)
\lineto(127.17832153,405.13210297)
\curveto(127.09831543,405.14210296)(127.0333155,405.16710293)(126.98332153,405.20710297)
\curveto(126.91331562,405.24710285)(126.85831567,405.3021028)(126.81832153,405.37210297)
\curveto(126.78831574,405.45210265)(126.74331579,405.51710258)(126.68332153,405.56710297)
\curveto(126.66331587,405.58710251)(126.63831589,405.6021025)(126.60832153,405.61210297)
\curveto(126.57831595,405.63210247)(126.53831599,405.63710246)(126.48832153,405.62710297)
\curveto(126.43831609,405.60710249)(126.38831614,405.58210252)(126.33832153,405.55210297)
\curveto(126.29831623,405.52210258)(126.25331628,405.4971026)(126.20332153,405.47710297)
\curveto(126.15331638,405.43710266)(126.09831643,405.4021027)(126.03832153,405.37210297)
\lineto(125.85832153,405.28210297)
\curveto(125.7283168,405.22210288)(125.59331694,405.17210293)(125.45332153,405.13210297)
\curveto(125.31331722,405.102103)(125.16831736,405.06710303)(125.01832153,405.02710297)
\curveto(124.94831758,405.00710309)(124.87831765,404.9971031)(124.80832153,404.99710297)
\curveto(124.74831778,404.98710311)(124.68331785,404.97710312)(124.61332153,404.96710297)
\lineto(124.52332153,404.96710297)
\curveto(124.49331804,404.95710314)(124.46331807,404.95210315)(124.43332153,404.95210297)
\lineto(124.26832153,404.95210297)
\curveto(124.16831836,404.93210317)(124.06831846,404.93210317)(123.96832153,404.95210297)
\lineto(123.83332153,404.95210297)
\curveto(123.76331877,404.97210313)(123.69331884,404.98210312)(123.62332153,404.98210297)
\curveto(123.56331897,404.97210313)(123.50331903,404.97710312)(123.44332153,404.99710297)
\curveto(123.34331919,405.01710308)(123.24831928,405.03710306)(123.15832153,405.05710297)
\curveto(123.06831946,405.06710303)(122.98331955,405.09210301)(122.90332153,405.13210297)
\curveto(122.61331992,405.24210286)(122.36332017,405.38210272)(122.15332153,405.55210297)
\curveto(121.95332058,405.73210237)(121.79332074,405.96710213)(121.67332153,406.25710297)
\curveto(121.64332089,406.32710177)(121.61332092,406.4021017)(121.58332153,406.48210297)
\curveto(121.56332097,406.56210154)(121.54332099,406.64710145)(121.52332153,406.73710297)
\curveto(121.50332103,406.78710131)(121.49332104,406.83710126)(121.49332153,406.88710297)
\curveto(121.50332103,406.93710116)(121.50332103,406.98710111)(121.49332153,407.03710297)
\curveto(121.48332105,407.06710103)(121.47332106,407.12710097)(121.46332153,407.21710297)
\curveto(121.46332107,407.31710078)(121.46832106,407.38710071)(121.47832153,407.42710297)
\curveto(121.49832103,407.52710057)(121.50832102,407.61210049)(121.50832153,407.68210297)
\lineto(121.59832153,408.01210297)
\curveto(121.6283209,408.13209997)(121.66832086,408.23709986)(121.71832153,408.32710297)
\curveto(121.88832064,408.61709948)(122.08332045,408.83709926)(122.30332153,408.98710297)
\curveto(122.52332001,409.13709896)(122.80331973,409.26709883)(123.14332153,409.37710297)
\curveto(123.27331926,409.42709867)(123.40831912,409.46209864)(123.54832153,409.48210297)
\curveto(123.68831884,409.5020986)(123.8283187,409.52709857)(123.96832153,409.55710297)
\curveto(124.04831848,409.57709852)(124.1333184,409.58709851)(124.22332153,409.58710297)
\curveto(124.31331822,409.5970985)(124.40331813,409.61209849)(124.49332153,409.63210297)
\curveto(124.56331797,409.65209845)(124.6333179,409.65709844)(124.70332153,409.64710297)
\curveto(124.77331776,409.64709845)(124.84831768,409.65709844)(124.92832153,409.67710297)
\curveto(124.99831753,409.6970984)(125.06831746,409.70709839)(125.13832153,409.70710297)
\curveto(125.20831732,409.70709839)(125.28331725,409.71709838)(125.36332153,409.73710297)
\curveto(125.57331696,409.78709831)(125.76331677,409.82709827)(125.93332153,409.85710297)
\curveto(126.11331642,409.8970982)(126.27331626,409.98709811)(126.41332153,410.12710297)
\curveto(126.50331603,410.21709788)(126.56331597,410.31709778)(126.59332153,410.42710297)
\curveto(126.60331593,410.45709764)(126.60331593,410.48209762)(126.59332153,410.50210297)
\curveto(126.59331594,410.52209758)(126.59831593,410.54209756)(126.60832153,410.56210297)
\curveto(126.61831591,410.58209752)(126.62331591,410.61209749)(126.62332153,410.65210297)
\lineto(126.62332153,410.74210297)
\lineto(126.59332153,410.86210297)
\curveto(126.59331594,410.9020972)(126.58831594,410.93709716)(126.57832153,410.96710297)
\curveto(126.47831605,411.26709683)(126.26831626,411.47209663)(125.94832153,411.58210297)
\curveto(125.85831667,411.61209649)(125.74831678,411.63209647)(125.61832153,411.64210297)
\curveto(125.49831703,411.66209644)(125.37331716,411.66709643)(125.24332153,411.65710297)
\curveto(125.11331742,411.65709644)(124.98831754,411.64709645)(124.86832153,411.62710297)
\curveto(124.74831778,411.60709649)(124.64331789,411.58209652)(124.55332153,411.55210297)
\curveto(124.49331804,411.53209657)(124.4333181,411.5020966)(124.37332153,411.46210297)
\curveto(124.32331821,411.43209667)(124.27331826,411.3970967)(124.22332153,411.35710297)
\curveto(124.17331836,411.31709678)(124.11831841,411.26209684)(124.05832153,411.19210297)
\curveto(124.00831852,411.12209698)(123.97331856,411.05709704)(123.95332153,410.99710297)
\curveto(123.90331863,410.8970972)(123.85831867,410.8020973)(123.81832153,410.71210297)
\curveto(123.78831874,410.62209748)(123.71831881,410.56209754)(123.60832153,410.53210297)
\curveto(123.528319,410.51209759)(123.44331909,410.5020976)(123.35332153,410.50210297)
\lineto(123.08332153,410.50210297)
\lineto(122.51332153,410.50210297)
\curveto(122.46332007,410.5020976)(122.41332012,410.4970976)(122.36332153,410.48710297)
\curveto(122.31332022,410.48709761)(122.26832026,410.49209761)(122.22832153,410.50210297)
\lineto(122.09332153,410.50210297)
\curveto(122.07332046,410.51209759)(122.04832048,410.51709758)(122.01832153,410.51710297)
\curveto(121.98832054,410.51709758)(121.96332057,410.52709757)(121.94332153,410.54710297)
\curveto(121.86332067,410.56709753)(121.80832072,410.63209747)(121.77832153,410.74210297)
\curveto(121.76832076,410.79209731)(121.76832076,410.84209726)(121.77832153,410.89210297)
\curveto(121.78832074,410.94209716)(121.79832073,410.98709711)(121.80832153,411.02710297)
\curveto(121.83832069,411.13709696)(121.86832066,411.23709686)(121.89832153,411.32710297)
\curveto(121.93832059,411.42709667)(121.98332055,411.51709658)(122.03332153,411.59710297)
\lineto(122.12332153,411.74710297)
\lineto(122.21332153,411.89710297)
\curveto(122.29332024,412.00709609)(122.39332014,412.11209599)(122.51332153,412.21210297)
\curveto(122.53332,412.22209588)(122.56331997,412.24709585)(122.60332153,412.28710297)
\curveto(122.65331988,412.32709577)(122.69831983,412.36209574)(122.73832153,412.39210297)
\curveto(122.77831975,412.42209568)(122.82331971,412.45209565)(122.87332153,412.48210297)
\curveto(123.04331949,412.59209551)(123.22331931,412.67709542)(123.41332153,412.73710297)
\curveto(123.60331893,412.80709529)(123.79831873,412.87209523)(123.99832153,412.93210297)
\curveto(124.11831841,412.96209514)(124.24331829,412.98209512)(124.37332153,412.99210297)
\curveto(124.50331803,413.0020951)(124.6333179,413.02209508)(124.76332153,413.05210297)
\curveto(124.80331773,413.06209504)(124.86331767,413.06209504)(124.94332153,413.05210297)
\curveto(125.0333175,413.04209506)(125.08831744,413.04709505)(125.10832153,413.06710297)
\curveto(125.51831701,413.07709502)(125.90831662,413.06209504)(126.27832153,413.02210297)
\curveto(126.65831587,412.98209512)(126.99831553,412.90709519)(127.29832153,412.79710297)
\curveto(127.60831492,412.68709541)(127.87331466,412.53709556)(128.09332153,412.34710297)
\curveto(128.31331422,412.16709593)(128.48331405,411.93209617)(128.60332153,411.64210297)
\curveto(128.67331386,411.47209663)(128.71331382,411.27709682)(128.72332153,411.05710297)
\curveto(128.7333138,410.83709726)(128.73831379,410.61209749)(128.73832153,410.38210297)
\lineto(128.73832153,407.03710297)
\lineto(128.73832153,406.45210297)
\curveto(128.73831379,406.26210184)(128.75831377,406.08710201)(128.79832153,405.92710297)
\curveto(128.80831372,405.8971022)(128.81331372,405.86210224)(128.81332153,405.82210297)
\curveto(128.81331372,405.79210231)(128.81831371,405.76210234)(128.82832153,405.73210297)
\moveto(126.62332153,408.04210297)
\curveto(126.6333159,408.09210001)(126.63831589,408.14709995)(126.63832153,408.20710297)
\curveto(126.63831589,408.27709982)(126.6333159,408.33709976)(126.62332153,408.38710297)
\curveto(126.60331593,408.44709965)(126.59331594,408.5020996)(126.59332153,408.55210297)
\curveto(126.59331594,408.6020995)(126.57331596,408.64209946)(126.53332153,408.67210297)
\curveto(126.48331605,408.71209939)(126.40831612,408.73209937)(126.30832153,408.73210297)
\curveto(126.26831626,408.72209938)(126.2333163,408.71209939)(126.20332153,408.70210297)
\curveto(126.17331636,408.7020994)(126.13831639,408.6970994)(126.09832153,408.68710297)
\curveto(126.0283165,408.66709943)(125.95331658,408.65209945)(125.87332153,408.64210297)
\curveto(125.79331674,408.63209947)(125.71331682,408.61709948)(125.63332153,408.59710297)
\curveto(125.60331693,408.58709951)(125.55831697,408.58209952)(125.49832153,408.58210297)
\curveto(125.36831716,408.55209955)(125.23831729,408.53209957)(125.10832153,408.52210297)
\curveto(124.97831755,408.51209959)(124.85331768,408.48709961)(124.73332153,408.44710297)
\curveto(124.65331788,408.42709967)(124.57831795,408.40709969)(124.50832153,408.38710297)
\curveto(124.43831809,408.37709972)(124.36831816,408.35709974)(124.29832153,408.32710297)
\curveto(124.08831844,408.23709986)(123.90831862,408.1021)(123.75832153,407.92210297)
\curveto(123.61831891,407.74210036)(123.56831896,407.49210061)(123.60832153,407.17210297)
\curveto(123.6283189,407.0021011)(123.68331885,406.86210124)(123.77332153,406.75210297)
\curveto(123.84331869,406.64210146)(123.94831858,406.55210155)(124.08832153,406.48210297)
\curveto(124.2283183,406.42210168)(124.37831815,406.37710172)(124.53832153,406.34710297)
\curveto(124.70831782,406.31710178)(124.88331765,406.30710179)(125.06332153,406.31710297)
\curveto(125.25331728,406.33710176)(125.4283171,406.37210173)(125.58832153,406.42210297)
\curveto(125.84831668,406.5021016)(126.05331648,406.62710147)(126.20332153,406.79710297)
\curveto(126.35331618,406.97710112)(126.46831606,407.1971009)(126.54832153,407.45710297)
\curveto(126.56831596,407.52710057)(126.57831595,407.5971005)(126.57832153,407.66710297)
\curveto(126.58831594,407.74710035)(126.60331593,407.82710027)(126.62332153,407.90710297)
\lineto(126.62332153,408.04210297)
}
}
{
\newrgbcolor{curcolor}{0 0 0}
\pscustom[linestyle=none,fillstyle=solid,fillcolor=curcolor]
{
\newpath
\moveto(137.98160278,405.98710297)
\lineto(137.98160278,405.56710297)
\curveto(137.98159441,405.43710266)(137.95159444,405.33210277)(137.89160278,405.25210297)
\curveto(137.84159455,405.2021029)(137.77659462,405.16710293)(137.69660278,405.14710297)
\curveto(137.61659478,405.13710296)(137.52659487,405.13210297)(137.42660278,405.13210297)
\lineto(136.60160278,405.13210297)
\lineto(136.31660278,405.13210297)
\curveto(136.23659616,405.14210296)(136.17159622,405.16710293)(136.12160278,405.20710297)
\curveto(136.05159634,405.25710284)(136.01159638,405.32210278)(136.00160278,405.40210297)
\curveto(135.9915964,405.48210262)(135.97159642,405.56210254)(135.94160278,405.64210297)
\curveto(135.92159647,405.66210244)(135.90159649,405.67710242)(135.88160278,405.68710297)
\curveto(135.87159652,405.70710239)(135.85659654,405.72710237)(135.83660278,405.74710297)
\curveto(135.72659667,405.74710235)(135.64659675,405.72210238)(135.59660278,405.67210297)
\lineto(135.44660278,405.52210297)
\curveto(135.37659702,405.47210263)(135.31159708,405.42710267)(135.25160278,405.38710297)
\curveto(135.1915972,405.35710274)(135.12659727,405.31710278)(135.05660278,405.26710297)
\curveto(135.01659738,405.24710285)(134.97159742,405.22710287)(134.92160278,405.20710297)
\curveto(134.88159751,405.18710291)(134.83659756,405.16710293)(134.78660278,405.14710297)
\curveto(134.64659775,405.097103)(134.4965979,405.05210305)(134.33660278,405.01210297)
\curveto(134.28659811,404.99210311)(134.24159815,404.98210312)(134.20160278,404.98210297)
\curveto(134.16159823,404.98210312)(134.12159827,404.97710312)(134.08160278,404.96710297)
\lineto(133.94660278,404.96710297)
\curveto(133.91659848,404.95710314)(133.87659852,404.95210315)(133.82660278,404.95210297)
\lineto(133.69160278,404.95210297)
\curveto(133.63159876,404.93210317)(133.54159885,404.92710317)(133.42160278,404.93710297)
\curveto(133.30159909,404.93710316)(133.21659918,404.94710315)(133.16660278,404.96710297)
\curveto(133.0965993,404.98710311)(133.03159936,404.9971031)(132.97160278,404.99710297)
\curveto(132.92159947,404.98710311)(132.86659953,404.99210311)(132.80660278,405.01210297)
\lineto(132.44660278,405.13210297)
\curveto(132.33660006,405.16210294)(132.22660017,405.2021029)(132.11660278,405.25210297)
\curveto(131.76660063,405.4021027)(131.45160094,405.63210247)(131.17160278,405.94210297)
\curveto(130.90160149,406.26210184)(130.68660171,406.5971015)(130.52660278,406.94710297)
\curveto(130.47660192,407.05710104)(130.43660196,407.16210094)(130.40660278,407.26210297)
\curveto(130.37660202,407.37210073)(130.34160205,407.48210062)(130.30160278,407.59210297)
\curveto(130.2916021,407.63210047)(130.28660211,407.66710043)(130.28660278,407.69710297)
\curveto(130.28660211,407.73710036)(130.27660212,407.78210032)(130.25660278,407.83210297)
\curveto(130.23660216,407.91210019)(130.21660218,407.9971001)(130.19660278,408.08710297)
\curveto(130.18660221,408.18709991)(130.17160222,408.28709981)(130.15160278,408.38710297)
\curveto(130.14160225,408.41709968)(130.13660226,408.45209965)(130.13660278,408.49210297)
\curveto(130.14660225,408.53209957)(130.14660225,408.56709953)(130.13660278,408.59710297)
\lineto(130.13660278,408.73210297)
\curveto(130.13660226,408.78209932)(130.13160226,408.83209927)(130.12160278,408.88210297)
\curveto(130.11160228,408.93209917)(130.10660229,408.98709911)(130.10660278,409.04710297)
\curveto(130.10660229,409.11709898)(130.11160228,409.17209893)(130.12160278,409.21210297)
\curveto(130.13160226,409.26209884)(130.13660226,409.30709879)(130.13660278,409.34710297)
\lineto(130.13660278,409.49710297)
\curveto(130.14660225,409.54709855)(130.14660225,409.59209851)(130.13660278,409.63210297)
\curveto(130.13660226,409.68209842)(130.14660225,409.73209837)(130.16660278,409.78210297)
\curveto(130.18660221,409.89209821)(130.20160219,409.9970981)(130.21160278,410.09710297)
\curveto(130.23160216,410.1970979)(130.25660214,410.2970978)(130.28660278,410.39710297)
\curveto(130.32660207,410.51709758)(130.36160203,410.63209747)(130.39160278,410.74210297)
\curveto(130.42160197,410.85209725)(130.46160193,410.96209714)(130.51160278,411.07210297)
\curveto(130.65160174,411.37209673)(130.82660157,411.65709644)(131.03660278,411.92710297)
\curveto(131.05660134,411.95709614)(131.08160131,411.98209612)(131.11160278,412.00210297)
\curveto(131.15160124,412.03209607)(131.18160121,412.06209604)(131.20160278,412.09210297)
\curveto(131.24160115,412.14209596)(131.28160111,412.18709591)(131.32160278,412.22710297)
\curveto(131.36160103,412.26709583)(131.40660099,412.30709579)(131.45660278,412.34710297)
\curveto(131.4966009,412.36709573)(131.53160086,412.39209571)(131.56160278,412.42210297)
\curveto(131.5916008,412.46209564)(131.62660077,412.49209561)(131.66660278,412.51210297)
\curveto(131.91660048,412.68209542)(132.20660019,412.82209528)(132.53660278,412.93210297)
\curveto(132.60659979,412.95209515)(132.67659972,412.96709513)(132.74660278,412.97710297)
\curveto(132.82659957,412.98709511)(132.90659949,413.0020951)(132.98660278,413.02210297)
\curveto(133.05659934,413.04209506)(133.14659925,413.05209505)(133.25660278,413.05210297)
\curveto(133.36659903,413.06209504)(133.47659892,413.06709503)(133.58660278,413.06710297)
\curveto(133.6965987,413.06709503)(133.80159859,413.06209504)(133.90160278,413.05210297)
\curveto(134.01159838,413.04209506)(134.10159829,413.02709507)(134.17160278,413.00710297)
\curveto(134.32159807,412.95709514)(134.46659793,412.91209519)(134.60660278,412.87210297)
\curveto(134.74659765,412.83209527)(134.87659752,412.77709532)(134.99660278,412.70710297)
\curveto(135.06659733,412.65709544)(135.13159726,412.60709549)(135.19160278,412.55710297)
\curveto(135.25159714,412.51709558)(135.31659708,412.47209563)(135.38660278,412.42210297)
\curveto(135.42659697,412.39209571)(135.48159691,412.35209575)(135.55160278,412.30210297)
\curveto(135.63159676,412.25209585)(135.70659669,412.25209585)(135.77660278,412.30210297)
\curveto(135.81659658,412.32209578)(135.83659656,412.35709574)(135.83660278,412.40710297)
\curveto(135.83659656,412.45709564)(135.84659655,412.50709559)(135.86660278,412.55710297)
\lineto(135.86660278,412.70710297)
\curveto(135.87659652,412.73709536)(135.88159651,412.77209533)(135.88160278,412.81210297)
\lineto(135.88160278,412.93210297)
\lineto(135.88160278,414.97210297)
\curveto(135.88159651,415.08209302)(135.87659652,415.2020929)(135.86660278,415.33210297)
\curveto(135.86659653,415.47209263)(135.8915965,415.57709252)(135.94160278,415.64710297)
\curveto(135.98159641,415.72709237)(136.05659634,415.77709232)(136.16660278,415.79710297)
\curveto(136.18659621,415.80709229)(136.20659619,415.80709229)(136.22660278,415.79710297)
\curveto(136.24659615,415.7970923)(136.26659613,415.8020923)(136.28660278,415.81210297)
\lineto(137.35160278,415.81210297)
\curveto(137.47159492,415.81209229)(137.58159481,415.80709229)(137.68160278,415.79710297)
\curveto(137.78159461,415.78709231)(137.85659454,415.74709235)(137.90660278,415.67710297)
\curveto(137.95659444,415.5970925)(137.98159441,415.49209261)(137.98160278,415.36210297)
\lineto(137.98160278,415.00210297)
\lineto(137.98160278,405.98710297)
\moveto(135.94160278,408.92710297)
\curveto(135.95159644,408.96709913)(135.95159644,409.00709909)(135.94160278,409.04710297)
\lineto(135.94160278,409.18210297)
\curveto(135.94159645,409.28209882)(135.93659646,409.38209872)(135.92660278,409.48210297)
\curveto(135.91659648,409.58209852)(135.90159649,409.67209843)(135.88160278,409.75210297)
\curveto(135.86159653,409.86209824)(135.84159655,409.96209814)(135.82160278,410.05210297)
\curveto(135.81159658,410.14209796)(135.78659661,410.22709787)(135.74660278,410.30710297)
\curveto(135.60659679,410.66709743)(135.40159699,410.95209715)(135.13160278,411.16210297)
\curveto(134.87159752,411.37209673)(134.4915979,411.47709662)(133.99160278,411.47710297)
\curveto(133.93159846,411.47709662)(133.85159854,411.46709663)(133.75160278,411.44710297)
\curveto(133.67159872,411.42709667)(133.5965988,411.40709669)(133.52660278,411.38710297)
\curveto(133.46659893,411.37709672)(133.40659899,411.35709674)(133.34660278,411.32710297)
\curveto(133.07659932,411.21709688)(132.86659953,411.04709705)(132.71660278,410.81710297)
\curveto(132.56659983,410.58709751)(132.44659995,410.32709777)(132.35660278,410.03710297)
\curveto(132.32660007,409.93709816)(132.30660009,409.83709826)(132.29660278,409.73710297)
\curveto(132.28660011,409.63709846)(132.26660013,409.53209857)(132.23660278,409.42210297)
\lineto(132.23660278,409.21210297)
\curveto(132.21660018,409.12209898)(132.21160018,408.9970991)(132.22160278,408.83710297)
\curveto(132.23160016,408.68709941)(132.24660015,408.57709952)(132.26660278,408.50710297)
\lineto(132.26660278,408.41710297)
\curveto(132.27660012,408.3970997)(132.28160011,408.37709972)(132.28160278,408.35710297)
\curveto(132.30160009,408.27709982)(132.31660008,408.2020999)(132.32660278,408.13210297)
\curveto(132.34660005,408.06210004)(132.36660003,407.98710011)(132.38660278,407.90710297)
\curveto(132.55659984,407.38710071)(132.84659955,407.0021011)(133.25660278,406.75210297)
\curveto(133.38659901,406.66210144)(133.56659883,406.59210151)(133.79660278,406.54210297)
\curveto(133.83659856,406.53210157)(133.8965985,406.52710157)(133.97660278,406.52710297)
\curveto(134.00659839,406.51710158)(134.05159834,406.50710159)(134.11160278,406.49710297)
\curveto(134.18159821,406.4971016)(134.23659816,406.5021016)(134.27660278,406.51210297)
\curveto(134.35659804,406.53210157)(134.43659796,406.54710155)(134.51660278,406.55710297)
\curveto(134.5965978,406.56710153)(134.67659772,406.58710151)(134.75660278,406.61710297)
\curveto(135.00659739,406.72710137)(135.20659719,406.86710123)(135.35660278,407.03710297)
\curveto(135.50659689,407.20710089)(135.63659676,407.42210068)(135.74660278,407.68210297)
\curveto(135.78659661,407.77210033)(135.81659658,407.86210024)(135.83660278,407.95210297)
\curveto(135.85659654,408.05210005)(135.87659652,408.15709994)(135.89660278,408.26710297)
\curveto(135.90659649,408.31709978)(135.90659649,408.36209974)(135.89660278,408.40210297)
\curveto(135.8965965,408.45209965)(135.90659649,408.5020996)(135.92660278,408.55210297)
\curveto(135.93659646,408.58209952)(135.94159645,408.61709948)(135.94160278,408.65710297)
\lineto(135.94160278,408.79210297)
\lineto(135.94160278,408.92710297)
}
}
{
\newrgbcolor{curcolor}{0 0 0}
\pscustom[linestyle=none,fillstyle=solid,fillcolor=curcolor]
{
\newpath
\moveto(147.33152466,409.31710297)
\curveto(147.35151609,409.25709884)(147.36151608,409.17209893)(147.36152466,409.06210297)
\curveto(147.36151608,408.95209915)(147.35151609,408.86709923)(147.33152466,408.80710297)
\lineto(147.33152466,408.65710297)
\curveto(147.31151613,408.57709952)(147.30151614,408.4970996)(147.30152466,408.41710297)
\curveto(147.31151613,408.33709976)(147.30651613,408.25709984)(147.28652466,408.17710297)
\curveto(147.26651617,408.10709999)(147.25151619,408.04210006)(147.24152466,407.98210297)
\curveto(147.23151621,407.92210018)(147.22151622,407.85710024)(147.21152466,407.78710297)
\curveto(147.17151627,407.67710042)(147.1365163,407.56210054)(147.10652466,407.44210297)
\curveto(147.07651636,407.33210077)(147.0365164,407.22710087)(146.98652466,407.12710297)
\curveto(146.77651666,406.64710145)(146.50151694,406.25710184)(146.16152466,405.95710297)
\curveto(145.82151762,405.65710244)(145.41151803,405.40710269)(144.93152466,405.20710297)
\curveto(144.81151863,405.15710294)(144.68651875,405.12210298)(144.55652466,405.10210297)
\curveto(144.436519,405.07210303)(144.31151913,405.04210306)(144.18152466,405.01210297)
\curveto(144.13151931,404.99210311)(144.07651936,404.98210312)(144.01652466,404.98210297)
\curveto(143.95651948,404.98210312)(143.90151954,404.97710312)(143.85152466,404.96710297)
\lineto(143.74652466,404.96710297)
\curveto(143.71651972,404.95710314)(143.68651975,404.95210315)(143.65652466,404.95210297)
\curveto(143.60651983,404.94210316)(143.52651991,404.93710316)(143.41652466,404.93710297)
\curveto(143.30652013,404.92710317)(143.22152022,404.93210317)(143.16152466,404.95210297)
\lineto(143.01152466,404.95210297)
\curveto(142.96152048,404.96210314)(142.90652053,404.96710313)(142.84652466,404.96710297)
\curveto(142.79652064,404.95710314)(142.74652069,404.96210314)(142.69652466,404.98210297)
\curveto(142.65652078,404.99210311)(142.61652082,404.9971031)(142.57652466,404.99710297)
\curveto(142.54652089,404.9971031)(142.50652093,405.0021031)(142.45652466,405.01210297)
\curveto(142.35652108,405.04210306)(142.25652118,405.06710303)(142.15652466,405.08710297)
\curveto(142.05652138,405.10710299)(141.96152148,405.13710296)(141.87152466,405.17710297)
\curveto(141.75152169,405.21710288)(141.6365218,405.25710284)(141.52652466,405.29710297)
\curveto(141.42652201,405.33710276)(141.32152212,405.38710271)(141.21152466,405.44710297)
\curveto(140.86152258,405.65710244)(140.56152288,405.9021022)(140.31152466,406.18210297)
\curveto(140.06152338,406.46210164)(139.85152359,406.7971013)(139.68152466,407.18710297)
\curveto(139.63152381,407.27710082)(139.59152385,407.37210073)(139.56152466,407.47210297)
\curveto(139.5415239,407.57210053)(139.51652392,407.67710042)(139.48652466,407.78710297)
\curveto(139.46652397,407.83710026)(139.45652398,407.88210022)(139.45652466,407.92210297)
\curveto(139.45652398,407.96210014)(139.44652399,408.00710009)(139.42652466,408.05710297)
\curveto(139.40652403,408.13709996)(139.39652404,408.21709988)(139.39652466,408.29710297)
\curveto(139.39652404,408.38709971)(139.38652405,408.47209963)(139.36652466,408.55210297)
\curveto(139.35652408,408.6020995)(139.35152409,408.64709945)(139.35152466,408.68710297)
\lineto(139.35152466,408.82210297)
\curveto(139.33152411,408.88209922)(139.32152412,408.96709913)(139.32152466,409.07710297)
\curveto(139.33152411,409.18709891)(139.34652409,409.27209883)(139.36652466,409.33210297)
\lineto(139.36652466,409.43710297)
\curveto(139.37652406,409.48709861)(139.37652406,409.53709856)(139.36652466,409.58710297)
\curveto(139.36652407,409.64709845)(139.37652406,409.7020984)(139.39652466,409.75210297)
\curveto(139.40652403,409.8020983)(139.41152403,409.84709825)(139.41152466,409.88710297)
\curveto(139.41152403,409.93709816)(139.42152402,409.98709811)(139.44152466,410.03710297)
\curveto(139.48152396,410.16709793)(139.51652392,410.29209781)(139.54652466,410.41210297)
\curveto(139.57652386,410.54209756)(139.61652382,410.66709743)(139.66652466,410.78710297)
\curveto(139.84652359,411.1970969)(140.06152338,411.53709656)(140.31152466,411.80710297)
\curveto(140.56152288,412.08709601)(140.86652257,412.34209576)(141.22652466,412.57210297)
\curveto(141.32652211,412.62209548)(141.43152201,412.66709543)(141.54152466,412.70710297)
\curveto(141.65152179,412.74709535)(141.76152168,412.79209531)(141.87152466,412.84210297)
\curveto(142.00152144,412.89209521)(142.1365213,412.92709517)(142.27652466,412.94710297)
\curveto(142.41652102,412.96709513)(142.56152088,412.9970951)(142.71152466,413.03710297)
\curveto(142.79152065,413.04709505)(142.86652057,413.05209505)(142.93652466,413.05210297)
\curveto(143.00652043,413.05209505)(143.07652036,413.05709504)(143.14652466,413.06710297)
\curveto(143.72651971,413.07709502)(144.22651921,413.01709508)(144.64652466,412.88710297)
\curveto(145.07651836,412.75709534)(145.45651798,412.57709552)(145.78652466,412.34710297)
\curveto(145.89651754,412.26709583)(146.00651743,412.17709592)(146.11652466,412.07710297)
\curveto(146.2365172,411.98709611)(146.3365171,411.88709621)(146.41652466,411.77710297)
\curveto(146.49651694,411.67709642)(146.56651687,411.57709652)(146.62652466,411.47710297)
\curveto(146.69651674,411.37709672)(146.76651667,411.27209683)(146.83652466,411.16210297)
\curveto(146.90651653,411.05209705)(146.96151648,410.93209717)(147.00152466,410.80210297)
\curveto(147.0415164,410.68209742)(147.08651635,410.55209755)(147.13652466,410.41210297)
\curveto(147.16651627,410.33209777)(147.19151625,410.24709785)(147.21152466,410.15710297)
\lineto(147.27152466,409.88710297)
\curveto(147.28151616,409.84709825)(147.28651615,409.80709829)(147.28652466,409.76710297)
\curveto(147.28651615,409.72709837)(147.29151615,409.68709841)(147.30152466,409.64710297)
\curveto(147.32151612,409.5970985)(147.32651611,409.54209856)(147.31652466,409.48210297)
\curveto(147.30651613,409.42209868)(147.31151613,409.36709873)(147.33152466,409.31710297)
\moveto(145.23152466,408.77710297)
\curveto(145.2415182,408.82709927)(145.24651819,408.8970992)(145.24652466,408.98710297)
\curveto(145.24651819,409.08709901)(145.2415182,409.16209894)(145.23152466,409.21210297)
\lineto(145.23152466,409.33210297)
\curveto(145.21151823,409.38209872)(145.20151824,409.43709866)(145.20152466,409.49710297)
\curveto(145.20151824,409.55709854)(145.19651824,409.61209849)(145.18652466,409.66210297)
\curveto(145.18651825,409.7020984)(145.18151826,409.73209837)(145.17152466,409.75210297)
\lineto(145.11152466,409.99210297)
\curveto(145.10151834,410.08209802)(145.08151836,410.16709793)(145.05152466,410.24710297)
\curveto(144.9415185,410.50709759)(144.81151863,410.72709737)(144.66152466,410.90710297)
\curveto(144.51151893,411.097097)(144.31151913,411.24709685)(144.06152466,411.35710297)
\curveto(144.00151944,411.37709672)(143.9415195,411.39209671)(143.88152466,411.40210297)
\curveto(143.82151962,411.42209668)(143.75651968,411.44209666)(143.68652466,411.46210297)
\curveto(143.60651983,411.48209662)(143.52151992,411.48709661)(143.43152466,411.47710297)
\lineto(143.16152466,411.47710297)
\curveto(143.13152031,411.45709664)(143.09652034,411.44709665)(143.05652466,411.44710297)
\curveto(143.01652042,411.45709664)(142.98152046,411.45709664)(142.95152466,411.44710297)
\lineto(142.74152466,411.38710297)
\curveto(142.68152076,411.37709672)(142.62652081,411.35709674)(142.57652466,411.32710297)
\curveto(142.32652111,411.21709688)(142.12152132,411.05709704)(141.96152466,410.84710297)
\curveto(141.81152163,410.64709745)(141.69152175,410.41209769)(141.60152466,410.14210297)
\curveto(141.57152187,410.04209806)(141.54652189,409.93709816)(141.52652466,409.82710297)
\curveto(141.51652192,409.71709838)(141.50152194,409.60709849)(141.48152466,409.49710297)
\curveto(141.47152197,409.44709865)(141.46652197,409.3970987)(141.46652466,409.34710297)
\lineto(141.46652466,409.19710297)
\curveto(141.44652199,409.12709897)(141.436522,409.02209908)(141.43652466,408.88210297)
\curveto(141.44652199,408.74209936)(141.46152198,408.63709946)(141.48152466,408.56710297)
\lineto(141.48152466,408.43210297)
\curveto(141.50152194,408.35209975)(141.51652192,408.27209983)(141.52652466,408.19210297)
\curveto(141.5365219,408.12209998)(141.55152189,408.04710005)(141.57152466,407.96710297)
\curveto(141.67152177,407.66710043)(141.77652166,407.42210068)(141.88652466,407.23210297)
\curveto(142.00652143,407.05210105)(142.19152125,406.88710121)(142.44152466,406.73710297)
\curveto(142.51152093,406.68710141)(142.58652085,406.64710145)(142.66652466,406.61710297)
\curveto(142.75652068,406.58710151)(142.84652059,406.56210154)(142.93652466,406.54210297)
\curveto(142.97652046,406.53210157)(143.01152043,406.52710157)(143.04152466,406.52710297)
\curveto(143.07152037,406.53710156)(143.10652033,406.53710156)(143.14652466,406.52710297)
\lineto(143.26652466,406.49710297)
\curveto(143.31652012,406.4971016)(143.36152008,406.5021016)(143.40152466,406.51210297)
\lineto(143.52152466,406.51210297)
\curveto(143.60151984,406.53210157)(143.68151976,406.54710155)(143.76152466,406.55710297)
\curveto(143.8415196,406.56710153)(143.91651952,406.58710151)(143.98652466,406.61710297)
\curveto(144.24651919,406.71710138)(144.45651898,406.85210125)(144.61652466,407.02210297)
\curveto(144.77651866,407.19210091)(144.91151853,407.4021007)(145.02152466,407.65210297)
\curveto(145.06151838,407.75210035)(145.09151835,407.85210025)(145.11152466,407.95210297)
\curveto(145.13151831,408.05210005)(145.15651828,408.15709994)(145.18652466,408.26710297)
\curveto(145.19651824,408.30709979)(145.20151824,408.34209976)(145.20152466,408.37210297)
\curveto(145.20151824,408.41209969)(145.20651823,408.45209965)(145.21652466,408.49210297)
\lineto(145.21652466,408.62710297)
\curveto(145.21651822,408.67709942)(145.22151822,408.72709937)(145.23152466,408.77710297)
}
}
{
\newrgbcolor{curcolor}{0 0 0}
\pscustom[linestyle=none,fillstyle=solid,fillcolor=curcolor]
{
\newpath
\moveto(153.15644653,413.06710297)
\curveto(153.26644122,413.06709503)(153.36144112,413.05709504)(153.44144653,413.03710297)
\curveto(153.53144095,413.01709508)(153.60144088,412.97209513)(153.65144653,412.90210297)
\curveto(153.71144077,412.82209528)(153.74144074,412.68209542)(153.74144653,412.48210297)
\lineto(153.74144653,411.97210297)
\lineto(153.74144653,411.59710297)
\curveto(153.75144073,411.45709664)(153.73644075,411.34709675)(153.69644653,411.26710297)
\curveto(153.65644083,411.1970969)(153.59644089,411.15209695)(153.51644653,411.13210297)
\curveto(153.44644104,411.11209699)(153.36144112,411.102097)(153.26144653,411.10210297)
\curveto(153.17144131,411.102097)(153.07144141,411.10709699)(152.96144653,411.11710297)
\curveto(152.86144162,411.12709697)(152.76644172,411.12209698)(152.67644653,411.10210297)
\curveto(152.60644188,411.08209702)(152.53644195,411.06709703)(152.46644653,411.05710297)
\curveto(152.39644209,411.05709704)(152.33144215,411.04709705)(152.27144653,411.02710297)
\curveto(152.11144237,410.97709712)(151.95144253,410.9020972)(151.79144653,410.80210297)
\curveto(151.63144285,410.71209739)(151.50644298,410.60709749)(151.41644653,410.48710297)
\curveto(151.36644312,410.40709769)(151.31144317,410.32209778)(151.25144653,410.23210297)
\curveto(151.20144328,410.15209795)(151.15144333,410.06709803)(151.10144653,409.97710297)
\curveto(151.07144341,409.8970982)(151.04144344,409.81209829)(151.01144653,409.72210297)
\lineto(150.95144653,409.48210297)
\curveto(150.93144355,409.41209869)(150.92144356,409.33709876)(150.92144653,409.25710297)
\curveto(150.92144356,409.18709891)(150.91144357,409.11709898)(150.89144653,409.04710297)
\curveto(150.8814436,409.00709909)(150.87644361,408.96709913)(150.87644653,408.92710297)
\curveto(150.8864436,408.8970992)(150.8864436,408.86709923)(150.87644653,408.83710297)
\lineto(150.87644653,408.59710297)
\curveto(150.85644363,408.52709957)(150.85144363,408.44709965)(150.86144653,408.35710297)
\curveto(150.87144361,408.27709982)(150.87644361,408.1970999)(150.87644653,408.11710297)
\lineto(150.87644653,407.15710297)
\lineto(150.87644653,405.88210297)
\curveto(150.87644361,405.75210235)(150.87144361,405.63210247)(150.86144653,405.52210297)
\curveto(150.85144363,405.41210269)(150.82144366,405.32210278)(150.77144653,405.25210297)
\curveto(150.75144373,405.22210288)(150.71644377,405.1971029)(150.66644653,405.17710297)
\curveto(150.62644386,405.16710293)(150.5814439,405.15710294)(150.53144653,405.14710297)
\lineto(150.45644653,405.14710297)
\curveto(150.40644408,405.13710296)(150.35144413,405.13210297)(150.29144653,405.13210297)
\lineto(150.12644653,405.13210297)
\lineto(149.48144653,405.13210297)
\curveto(149.42144506,405.14210296)(149.35644513,405.14710295)(149.28644653,405.14710297)
\lineto(149.09144653,405.14710297)
\curveto(149.04144544,405.16710293)(148.99144549,405.18210292)(148.94144653,405.19210297)
\curveto(148.89144559,405.21210289)(148.85644563,405.24710285)(148.83644653,405.29710297)
\curveto(148.79644569,405.34710275)(148.77144571,405.41710268)(148.76144653,405.50710297)
\lineto(148.76144653,405.80710297)
\lineto(148.76144653,406.82710297)
\lineto(148.76144653,411.05710297)
\lineto(148.76144653,412.16710297)
\lineto(148.76144653,412.45210297)
\curveto(148.76144572,412.55209555)(148.7814457,412.63209547)(148.82144653,412.69210297)
\curveto(148.87144561,412.77209533)(148.94644554,412.82209528)(149.04644653,412.84210297)
\curveto(149.14644534,412.86209524)(149.26644522,412.87209523)(149.40644653,412.87210297)
\lineto(150.17144653,412.87210297)
\curveto(150.29144419,412.87209523)(150.39644409,412.86209524)(150.48644653,412.84210297)
\curveto(150.57644391,412.83209527)(150.64644384,412.78709531)(150.69644653,412.70710297)
\curveto(150.72644376,412.65709544)(150.74144374,412.58709551)(150.74144653,412.49710297)
\lineto(150.77144653,412.22710297)
\curveto(150.7814437,412.14709595)(150.79644369,412.07209603)(150.81644653,412.00210297)
\curveto(150.84644364,411.93209617)(150.89644359,411.8970962)(150.96644653,411.89710297)
\curveto(150.9864435,411.91709618)(151.00644348,411.92709617)(151.02644653,411.92710297)
\curveto(151.04644344,411.92709617)(151.06644342,411.93709616)(151.08644653,411.95710297)
\curveto(151.14644334,412.00709609)(151.19644329,412.06209604)(151.23644653,412.12210297)
\curveto(151.2864432,412.19209591)(151.34644314,412.25209585)(151.41644653,412.30210297)
\curveto(151.45644303,412.33209577)(151.49144299,412.36209574)(151.52144653,412.39210297)
\curveto(151.55144293,412.43209567)(151.5864429,412.46709563)(151.62644653,412.49710297)
\lineto(151.89644653,412.67710297)
\curveto(151.99644249,412.73709536)(152.09644239,412.79209531)(152.19644653,412.84210297)
\curveto(152.29644219,412.88209522)(152.39644209,412.91709518)(152.49644653,412.94710297)
\lineto(152.82644653,413.03710297)
\curveto(152.85644163,413.04709505)(152.91144157,413.04709505)(152.99144653,413.03710297)
\curveto(153.0814414,413.03709506)(153.13644135,413.04709505)(153.15644653,413.06710297)
}
}
{
\newrgbcolor{curcolor}{0 0 0}
\pscustom[linestyle=none,fillstyle=solid,fillcolor=curcolor]
{
\newpath
\moveto(161.66285278,409.07710297)
\curveto(161.68284462,408.9970991)(161.68284462,408.90709919)(161.66285278,408.80710297)
\curveto(161.64284466,408.70709939)(161.60784469,408.64209946)(161.55785278,408.61210297)
\curveto(161.50784479,408.57209953)(161.43284487,408.54209956)(161.33285278,408.52210297)
\curveto(161.24284506,408.51209959)(161.13784516,408.5020996)(161.01785278,408.49210297)
\lineto(160.67285278,408.49210297)
\curveto(160.56284574,408.5020996)(160.46284584,408.50709959)(160.37285278,408.50710297)
\lineto(156.71285278,408.50710297)
\lineto(156.50285278,408.50710297)
\curveto(156.44284986,408.50709959)(156.38784991,408.4970996)(156.33785278,408.47710297)
\curveto(156.25785004,408.43709966)(156.20785009,408.3970997)(156.18785278,408.35710297)
\curveto(156.16785013,408.33709976)(156.14785015,408.2970998)(156.12785278,408.23710297)
\curveto(156.10785019,408.18709991)(156.1028502,408.13709996)(156.11285278,408.08710297)
\curveto(156.13285017,408.02710007)(156.14285016,407.96710013)(156.14285278,407.90710297)
\curveto(156.15285015,407.85710024)(156.16785013,407.8021003)(156.18785278,407.74210297)
\curveto(156.26785003,407.5021006)(156.36284994,407.3021008)(156.47285278,407.14210297)
\curveto(156.59284971,406.99210111)(156.75284955,406.85710124)(156.95285278,406.73710297)
\curveto(157.03284927,406.68710141)(157.11284919,406.65210145)(157.19285278,406.63210297)
\curveto(157.28284902,406.62210148)(157.37284893,406.6021015)(157.46285278,406.57210297)
\curveto(157.54284876,406.55210155)(157.65284865,406.53710156)(157.79285278,406.52710297)
\curveto(157.93284837,406.51710158)(158.05284825,406.52210158)(158.15285278,406.54210297)
\lineto(158.28785278,406.54210297)
\curveto(158.38784791,406.56210154)(158.47784782,406.58210152)(158.55785278,406.60210297)
\curveto(158.64784765,406.63210147)(158.73284757,406.66210144)(158.81285278,406.69210297)
\curveto(158.91284739,406.74210136)(159.02284728,406.80710129)(159.14285278,406.88710297)
\curveto(159.27284703,406.96710113)(159.36784693,407.04710105)(159.42785278,407.12710297)
\curveto(159.47784682,407.1971009)(159.52784677,407.26210084)(159.57785278,407.32210297)
\curveto(159.63784666,407.39210071)(159.70784659,407.44210066)(159.78785278,407.47210297)
\curveto(159.88784641,407.52210058)(160.01284629,407.54210056)(160.16285278,407.53210297)
\lineto(160.59785278,407.53210297)
\lineto(160.77785278,407.53210297)
\curveto(160.84784545,407.54210056)(160.90784539,407.53710056)(160.95785278,407.51710297)
\lineto(161.10785278,407.51710297)
\curveto(161.20784509,407.4971006)(161.27784502,407.47210063)(161.31785278,407.44210297)
\curveto(161.35784494,407.42210068)(161.37784492,407.37710072)(161.37785278,407.30710297)
\curveto(161.38784491,407.23710086)(161.38284492,407.17710092)(161.36285278,407.12710297)
\curveto(161.31284499,406.98710111)(161.25784504,406.86210124)(161.19785278,406.75210297)
\curveto(161.13784516,406.64210146)(161.06784523,406.53210157)(160.98785278,406.42210297)
\curveto(160.76784553,406.09210201)(160.51784578,405.82710227)(160.23785278,405.62710297)
\curveto(159.95784634,405.42710267)(159.60784669,405.25710284)(159.18785278,405.11710297)
\curveto(159.07784722,405.07710302)(158.96784733,405.05210305)(158.85785278,405.04210297)
\curveto(158.74784755,405.03210307)(158.63284767,405.01210309)(158.51285278,404.98210297)
\curveto(158.47284783,404.97210313)(158.42784787,404.97210313)(158.37785278,404.98210297)
\curveto(158.33784796,404.98210312)(158.297848,404.97710312)(158.25785278,404.96710297)
\lineto(158.09285278,404.96710297)
\curveto(158.04284826,404.94710315)(157.98284832,404.94210316)(157.91285278,404.95210297)
\curveto(157.85284845,404.95210315)(157.7978485,404.95710314)(157.74785278,404.96710297)
\curveto(157.66784863,404.97710312)(157.5978487,404.97710312)(157.53785278,404.96710297)
\curveto(157.47784882,404.95710314)(157.41284889,404.96210314)(157.34285278,404.98210297)
\curveto(157.29284901,405.0021031)(157.23784906,405.01210309)(157.17785278,405.01210297)
\curveto(157.11784918,405.01210309)(157.06284924,405.02210308)(157.01285278,405.04210297)
\curveto(156.9028494,405.06210304)(156.79284951,405.08710301)(156.68285278,405.11710297)
\curveto(156.57284973,405.13710296)(156.47284983,405.17210293)(156.38285278,405.22210297)
\curveto(156.27285003,405.26210284)(156.16785013,405.2971028)(156.06785278,405.32710297)
\curveto(155.97785032,405.36710273)(155.89285041,405.41210269)(155.81285278,405.46210297)
\curveto(155.49285081,405.66210244)(155.20785109,405.89210221)(154.95785278,406.15210297)
\curveto(154.70785159,406.42210168)(154.5028518,406.73210137)(154.34285278,407.08210297)
\curveto(154.29285201,407.19210091)(154.25285205,407.3021008)(154.22285278,407.41210297)
\curveto(154.19285211,407.53210057)(154.15285215,407.65210045)(154.10285278,407.77210297)
\curveto(154.09285221,407.81210029)(154.08785221,407.84710025)(154.08785278,407.87710297)
\curveto(154.08785221,407.91710018)(154.08285222,407.95710014)(154.07285278,407.99710297)
\curveto(154.03285227,408.11709998)(154.00785229,408.24709985)(153.99785278,408.38710297)
\lineto(153.96785278,408.80710297)
\curveto(153.96785233,408.85709924)(153.96285234,408.91209919)(153.95285278,408.97210297)
\curveto(153.95285235,409.03209907)(153.95785234,409.08709901)(153.96785278,409.13710297)
\lineto(153.96785278,409.31710297)
\lineto(154.01285278,409.67710297)
\curveto(154.05285225,409.84709825)(154.08785221,410.01209809)(154.11785278,410.17210297)
\curveto(154.14785215,410.33209777)(154.19285211,410.48209762)(154.25285278,410.62210297)
\curveto(154.68285162,411.66209644)(155.41285089,412.3970957)(156.44285278,412.82710297)
\curveto(156.58284972,412.88709521)(156.72284958,412.92709517)(156.86285278,412.94710297)
\curveto(157.01284929,412.97709512)(157.16784913,413.01209509)(157.32785278,413.05210297)
\curveto(157.40784889,413.06209504)(157.48284882,413.06709503)(157.55285278,413.06710297)
\curveto(157.62284868,413.06709503)(157.6978486,413.07209503)(157.77785278,413.08210297)
\curveto(158.28784801,413.09209501)(158.72284758,413.03209507)(159.08285278,412.90210297)
\curveto(159.45284685,412.78209532)(159.78284652,412.62209548)(160.07285278,412.42210297)
\curveto(160.16284614,412.36209574)(160.25284605,412.29209581)(160.34285278,412.21210297)
\curveto(160.43284587,412.14209596)(160.51284579,412.06709603)(160.58285278,411.98710297)
\curveto(160.61284569,411.93709616)(160.65284565,411.8970962)(160.70285278,411.86710297)
\curveto(160.78284552,411.75709634)(160.85784544,411.64209646)(160.92785278,411.52210297)
\curveto(160.9978453,411.41209669)(161.07284523,411.2970968)(161.15285278,411.17710297)
\curveto(161.2028451,411.08709701)(161.24284506,410.99209711)(161.27285278,410.89210297)
\curveto(161.31284499,410.8020973)(161.35284495,410.7020974)(161.39285278,410.59210297)
\curveto(161.44284486,410.46209764)(161.48284482,410.32709777)(161.51285278,410.18710297)
\curveto(161.54284476,410.04709805)(161.57784472,409.90709819)(161.61785278,409.76710297)
\curveto(161.63784466,409.68709841)(161.64284466,409.5970985)(161.63285278,409.49710297)
\curveto(161.63284467,409.40709869)(161.64284466,409.32209878)(161.66285278,409.24210297)
\lineto(161.66285278,409.07710297)
\moveto(159.41285278,409.96210297)
\curveto(159.48284682,410.06209804)(159.48784681,410.18209792)(159.42785278,410.32210297)
\curveto(159.37784692,410.47209763)(159.33784696,410.58209752)(159.30785278,410.65210297)
\curveto(159.16784713,410.92209718)(158.98284732,411.12709697)(158.75285278,411.26710297)
\curveto(158.52284778,411.41709668)(158.2028481,411.4970966)(157.79285278,411.50710297)
\curveto(157.76284854,411.48709661)(157.72784857,411.48209662)(157.68785278,411.49210297)
\curveto(157.64784865,411.5020966)(157.61284869,411.5020966)(157.58285278,411.49210297)
\curveto(157.53284877,411.47209663)(157.47784882,411.45709664)(157.41785278,411.44710297)
\curveto(157.35784894,411.44709665)(157.302849,411.43709666)(157.25285278,411.41710297)
\curveto(156.81284949,411.27709682)(156.48784981,411.0020971)(156.27785278,410.59210297)
\curveto(156.25785004,410.55209755)(156.23285007,410.4970976)(156.20285278,410.42710297)
\curveto(156.18285012,410.36709773)(156.16785013,410.3020978)(156.15785278,410.23210297)
\curveto(156.14785015,410.17209793)(156.14785015,410.11209799)(156.15785278,410.05210297)
\curveto(156.17785012,409.99209811)(156.21285009,409.94209816)(156.26285278,409.90210297)
\curveto(156.34284996,409.85209825)(156.45284985,409.82709827)(156.59285278,409.82710297)
\lineto(156.99785278,409.82710297)
\lineto(158.66285278,409.82710297)
\lineto(159.09785278,409.82710297)
\curveto(159.25784704,409.83709826)(159.36284694,409.88209822)(159.41285278,409.96210297)
}
}
{
\newrgbcolor{curcolor}{0 0 0}
\pscustom[linestyle=none,fillstyle=solid,fillcolor=curcolor]
{
\newpath
\moveto(165.88113403,413.08210297)
\curveto(166.63112953,413.102095)(167.28112888,413.01709508)(167.83113403,412.82710297)
\curveto(168.39112777,412.64709545)(168.81612735,412.33209577)(169.10613403,411.88210297)
\curveto(169.17612699,411.77209633)(169.23612693,411.65709644)(169.28613403,411.53710297)
\curveto(169.34612682,411.42709667)(169.39612677,411.3020968)(169.43613403,411.16210297)
\curveto(169.45612671,411.102097)(169.4661267,411.03709706)(169.46613403,410.96710297)
\curveto(169.4661267,410.8970972)(169.45612671,410.83709726)(169.43613403,410.78710297)
\curveto(169.39612677,410.72709737)(169.34112682,410.68709741)(169.27113403,410.66710297)
\curveto(169.22112694,410.64709745)(169.161127,410.63709746)(169.09113403,410.63710297)
\lineto(168.88113403,410.63710297)
\lineto(168.22113403,410.63710297)
\curveto(168.15112801,410.63709746)(168.08112808,410.63209747)(168.01113403,410.62210297)
\curveto(167.94112822,410.62209748)(167.87612829,410.63209747)(167.81613403,410.65210297)
\curveto(167.71612845,410.67209743)(167.64112852,410.71209739)(167.59113403,410.77210297)
\curveto(167.54112862,410.83209727)(167.49612867,410.89209721)(167.45613403,410.95210297)
\lineto(167.33613403,411.16210297)
\curveto(167.30612886,411.24209686)(167.25612891,411.30709679)(167.18613403,411.35710297)
\curveto(167.08612908,411.43709666)(166.98612918,411.4970966)(166.88613403,411.53710297)
\curveto(166.79612937,411.57709652)(166.68112948,411.61209649)(166.54113403,411.64210297)
\curveto(166.47112969,411.66209644)(166.3661298,411.67709642)(166.22613403,411.68710297)
\curveto(166.09613007,411.6970964)(165.99613017,411.69209641)(165.92613403,411.67210297)
\lineto(165.82113403,411.67210297)
\lineto(165.67113403,411.64210297)
\curveto(165.63113053,411.64209646)(165.58613058,411.63709646)(165.53613403,411.62710297)
\curveto(165.3661308,411.57709652)(165.22613094,411.50709659)(165.11613403,411.41710297)
\curveto(165.01613115,411.33709676)(164.94613122,411.21209689)(164.90613403,411.04210297)
\curveto(164.88613128,410.97209713)(164.88613128,410.90709719)(164.90613403,410.84710297)
\curveto(164.92613124,410.78709731)(164.94613122,410.73709736)(164.96613403,410.69710297)
\curveto(165.03613113,410.57709752)(165.11613105,410.48209762)(165.20613403,410.41210297)
\curveto(165.30613086,410.34209776)(165.42113074,410.28209782)(165.55113403,410.23210297)
\curveto(165.74113042,410.15209795)(165.94613022,410.08209802)(166.16613403,410.02210297)
\lineto(166.85613403,409.87210297)
\curveto(167.09612907,409.83209827)(167.32612884,409.78209832)(167.54613403,409.72210297)
\curveto(167.77612839,409.67209843)(167.99112817,409.60709849)(168.19113403,409.52710297)
\curveto(168.28112788,409.48709861)(168.3661278,409.45209865)(168.44613403,409.42210297)
\curveto(168.53612763,409.4020987)(168.62112754,409.36709873)(168.70113403,409.31710297)
\curveto(168.89112727,409.1970989)(169.0611271,409.06709903)(169.21113403,408.92710297)
\curveto(169.37112679,408.78709931)(169.49612667,408.61209949)(169.58613403,408.40210297)
\curveto(169.61612655,408.33209977)(169.64112652,408.26209984)(169.66113403,408.19210297)
\curveto(169.68112648,408.12209998)(169.70112646,408.04710005)(169.72113403,407.96710297)
\curveto(169.73112643,407.90710019)(169.73612643,407.81210029)(169.73613403,407.68210297)
\curveto(169.74612642,407.56210054)(169.74612642,407.46710063)(169.73613403,407.39710297)
\lineto(169.73613403,407.32210297)
\curveto(169.71612645,407.26210084)(169.70112646,407.2021009)(169.69113403,407.14210297)
\curveto(169.69112647,407.09210101)(169.68612648,407.04210106)(169.67613403,406.99210297)
\curveto(169.60612656,406.69210141)(169.49612667,406.42710167)(169.34613403,406.19710297)
\curveto(169.18612698,405.95710214)(168.99112717,405.76210234)(168.76113403,405.61210297)
\curveto(168.53112763,405.46210264)(168.27112789,405.33210277)(167.98113403,405.22210297)
\curveto(167.87112829,405.17210293)(167.75112841,405.13710296)(167.62113403,405.11710297)
\curveto(167.50112866,405.097103)(167.38112878,405.07210303)(167.26113403,405.04210297)
\curveto(167.17112899,405.02210308)(167.07612909,405.01210309)(166.97613403,405.01210297)
\curveto(166.88612928,405.0021031)(166.79612937,404.98710311)(166.70613403,404.96710297)
\lineto(166.43613403,404.96710297)
\curveto(166.37612979,404.94710315)(166.27112989,404.93710316)(166.12113403,404.93710297)
\curveto(165.98113018,404.93710316)(165.88113028,404.94710315)(165.82113403,404.96710297)
\curveto(165.79113037,404.96710313)(165.75613041,404.97210313)(165.71613403,404.98210297)
\lineto(165.61113403,404.98210297)
\curveto(165.49113067,405.0021031)(165.37113079,405.01710308)(165.25113403,405.02710297)
\curveto(165.13113103,405.03710306)(165.01613115,405.05710304)(164.90613403,405.08710297)
\curveto(164.51613165,405.1971029)(164.17113199,405.32210278)(163.87113403,405.46210297)
\curveto(163.57113259,405.61210249)(163.31613285,405.83210227)(163.10613403,406.12210297)
\curveto(162.9661332,406.31210179)(162.84613332,406.53210157)(162.74613403,406.78210297)
\curveto(162.72613344,406.84210126)(162.70613346,406.92210118)(162.68613403,407.02210297)
\curveto(162.6661335,407.07210103)(162.65113351,407.14210096)(162.64113403,407.23210297)
\curveto(162.63113353,407.32210078)(162.63613353,407.3971007)(162.65613403,407.45710297)
\curveto(162.68613348,407.52710057)(162.73613343,407.57710052)(162.80613403,407.60710297)
\curveto(162.85613331,407.62710047)(162.91613325,407.63710046)(162.98613403,407.63710297)
\lineto(163.21113403,407.63710297)
\lineto(163.91613403,407.63710297)
\lineto(164.15613403,407.63710297)
\curveto(164.23613193,407.63710046)(164.30613186,407.62710047)(164.36613403,407.60710297)
\curveto(164.47613169,407.56710053)(164.54613162,407.5021006)(164.57613403,407.41210297)
\curveto(164.61613155,407.32210078)(164.6611315,407.22710087)(164.71113403,407.12710297)
\curveto(164.73113143,407.07710102)(164.7661314,407.01210109)(164.81613403,406.93210297)
\curveto(164.87613129,406.85210125)(164.92613124,406.8021013)(164.96613403,406.78210297)
\curveto(165.08613108,406.68210142)(165.20113096,406.6021015)(165.31113403,406.54210297)
\curveto(165.42113074,406.49210161)(165.5611306,406.44210166)(165.73113403,406.39210297)
\curveto(165.78113038,406.37210173)(165.83113033,406.36210174)(165.88113403,406.36210297)
\curveto(165.93113023,406.37210173)(165.98113018,406.37210173)(166.03113403,406.36210297)
\curveto(166.11113005,406.34210176)(166.19612997,406.33210177)(166.28613403,406.33210297)
\curveto(166.38612978,406.34210176)(166.47112969,406.35710174)(166.54113403,406.37710297)
\curveto(166.59112957,406.38710171)(166.63612953,406.39210171)(166.67613403,406.39210297)
\curveto(166.72612944,406.39210171)(166.77612939,406.4021017)(166.82613403,406.42210297)
\curveto(166.9661292,406.47210163)(167.09112907,406.53210157)(167.20113403,406.60210297)
\curveto(167.32112884,406.67210143)(167.41612875,406.76210134)(167.48613403,406.87210297)
\curveto(167.53612863,406.95210115)(167.57612859,407.07710102)(167.60613403,407.24710297)
\curveto(167.62612854,407.31710078)(167.62612854,407.38210072)(167.60613403,407.44210297)
\curveto(167.58612858,407.5021006)(167.5661286,407.55210055)(167.54613403,407.59210297)
\curveto(167.47612869,407.73210037)(167.38612878,407.83710026)(167.27613403,407.90710297)
\curveto(167.17612899,407.97710012)(167.05612911,408.04210006)(166.91613403,408.10210297)
\curveto(166.72612944,408.18209992)(166.52612964,408.24709985)(166.31613403,408.29710297)
\curveto(166.10613006,408.34709975)(165.89613027,408.4020997)(165.68613403,408.46210297)
\curveto(165.60613056,408.48209962)(165.52113064,408.4970996)(165.43113403,408.50710297)
\curveto(165.35113081,408.51709958)(165.27113089,408.53209957)(165.19113403,408.55210297)
\curveto(164.87113129,408.64209946)(164.5661316,408.72709937)(164.27613403,408.80710297)
\curveto(163.98613218,408.8970992)(163.72113244,409.02709907)(163.48113403,409.19710297)
\curveto(163.20113296,409.3970987)(162.99613317,409.66709843)(162.86613403,410.00710297)
\curveto(162.84613332,410.07709802)(162.82613334,410.17209793)(162.80613403,410.29210297)
\curveto(162.78613338,410.36209774)(162.77113339,410.44709765)(162.76113403,410.54710297)
\curveto(162.75113341,410.64709745)(162.75613341,410.73709736)(162.77613403,410.81710297)
\curveto(162.79613337,410.86709723)(162.80113336,410.90709719)(162.79113403,410.93710297)
\curveto(162.78113338,410.97709712)(162.78613338,411.02209708)(162.80613403,411.07210297)
\curveto(162.82613334,411.18209692)(162.84613332,411.28209682)(162.86613403,411.37210297)
\curveto(162.89613327,411.47209663)(162.93113323,411.56709653)(162.97113403,411.65710297)
\curveto(163.10113306,411.94709615)(163.28113288,412.18209592)(163.51113403,412.36210297)
\curveto(163.74113242,412.54209556)(164.00113216,412.68709541)(164.29113403,412.79710297)
\curveto(164.40113176,412.84709525)(164.51613165,412.88209522)(164.63613403,412.90210297)
\curveto(164.75613141,412.93209517)(164.88113128,412.96209514)(165.01113403,412.99210297)
\curveto(165.07113109,413.01209509)(165.13113103,413.02209508)(165.19113403,413.02210297)
\lineto(165.37113403,413.05210297)
\curveto(165.45113071,413.06209504)(165.53613063,413.06709503)(165.62613403,413.06710297)
\curveto(165.71613045,413.06709503)(165.80113036,413.07209503)(165.88113403,413.08210297)
}
}
{
\newrgbcolor{curcolor}{0 0 0}
\pscustom[linestyle=none,fillstyle=solid,fillcolor=curcolor]
{
}
}
{
\newrgbcolor{curcolor}{0 0 0}
\pscustom[linestyle=none,fillstyle=solid,fillcolor=curcolor]
{
\newpath
\moveto(182.71793091,405.98710297)
\lineto(182.71793091,405.56710297)
\curveto(182.71792254,405.43710266)(182.68792257,405.33210277)(182.62793091,405.25210297)
\curveto(182.57792268,405.2021029)(182.51292274,405.16710293)(182.43293091,405.14710297)
\curveto(182.3529229,405.13710296)(182.26292299,405.13210297)(182.16293091,405.13210297)
\lineto(181.33793091,405.13210297)
\lineto(181.05293091,405.13210297)
\curveto(180.97292428,405.14210296)(180.90792435,405.16710293)(180.85793091,405.20710297)
\curveto(180.78792447,405.25710284)(180.74792451,405.32210278)(180.73793091,405.40210297)
\curveto(180.72792453,405.48210262)(180.70792455,405.56210254)(180.67793091,405.64210297)
\curveto(180.6579246,405.66210244)(180.63792462,405.67710242)(180.61793091,405.68710297)
\curveto(180.60792465,405.70710239)(180.59292466,405.72710237)(180.57293091,405.74710297)
\curveto(180.46292479,405.74710235)(180.38292487,405.72210238)(180.33293091,405.67210297)
\lineto(180.18293091,405.52210297)
\curveto(180.11292514,405.47210263)(180.04792521,405.42710267)(179.98793091,405.38710297)
\curveto(179.92792533,405.35710274)(179.86292539,405.31710278)(179.79293091,405.26710297)
\curveto(179.7529255,405.24710285)(179.70792555,405.22710287)(179.65793091,405.20710297)
\curveto(179.61792564,405.18710291)(179.57292568,405.16710293)(179.52293091,405.14710297)
\curveto(179.38292587,405.097103)(179.23292602,405.05210305)(179.07293091,405.01210297)
\curveto(179.02292623,404.99210311)(178.97792628,404.98210312)(178.93793091,404.98210297)
\curveto(178.89792636,404.98210312)(178.8579264,404.97710312)(178.81793091,404.96710297)
\lineto(178.68293091,404.96710297)
\curveto(178.6529266,404.95710314)(178.61292664,404.95210315)(178.56293091,404.95210297)
\lineto(178.42793091,404.95210297)
\curveto(178.36792689,404.93210317)(178.27792698,404.92710317)(178.15793091,404.93710297)
\curveto(178.03792722,404.93710316)(177.9529273,404.94710315)(177.90293091,404.96710297)
\curveto(177.83292742,404.98710311)(177.76792749,404.9971031)(177.70793091,404.99710297)
\curveto(177.6579276,404.98710311)(177.60292765,404.99210311)(177.54293091,405.01210297)
\lineto(177.18293091,405.13210297)
\curveto(177.07292818,405.16210294)(176.96292829,405.2021029)(176.85293091,405.25210297)
\curveto(176.50292875,405.4021027)(176.18792907,405.63210247)(175.90793091,405.94210297)
\curveto(175.63792962,406.26210184)(175.42292983,406.5971015)(175.26293091,406.94710297)
\curveto(175.21293004,407.05710104)(175.17293008,407.16210094)(175.14293091,407.26210297)
\curveto(175.11293014,407.37210073)(175.07793018,407.48210062)(175.03793091,407.59210297)
\curveto(175.02793023,407.63210047)(175.02293023,407.66710043)(175.02293091,407.69710297)
\curveto(175.02293023,407.73710036)(175.01293024,407.78210032)(174.99293091,407.83210297)
\curveto(174.97293028,407.91210019)(174.9529303,407.9971001)(174.93293091,408.08710297)
\curveto(174.92293033,408.18709991)(174.90793035,408.28709981)(174.88793091,408.38710297)
\curveto(174.87793038,408.41709968)(174.87293038,408.45209965)(174.87293091,408.49210297)
\curveto(174.88293037,408.53209957)(174.88293037,408.56709953)(174.87293091,408.59710297)
\lineto(174.87293091,408.73210297)
\curveto(174.87293038,408.78209932)(174.86793039,408.83209927)(174.85793091,408.88210297)
\curveto(174.84793041,408.93209917)(174.84293041,408.98709911)(174.84293091,409.04710297)
\curveto(174.84293041,409.11709898)(174.84793041,409.17209893)(174.85793091,409.21210297)
\curveto(174.86793039,409.26209884)(174.87293038,409.30709879)(174.87293091,409.34710297)
\lineto(174.87293091,409.49710297)
\curveto(174.88293037,409.54709855)(174.88293037,409.59209851)(174.87293091,409.63210297)
\curveto(174.87293038,409.68209842)(174.88293037,409.73209837)(174.90293091,409.78210297)
\curveto(174.92293033,409.89209821)(174.93793032,409.9970981)(174.94793091,410.09710297)
\curveto(174.96793029,410.1970979)(174.99293026,410.2970978)(175.02293091,410.39710297)
\curveto(175.06293019,410.51709758)(175.09793016,410.63209747)(175.12793091,410.74210297)
\curveto(175.1579301,410.85209725)(175.19793006,410.96209714)(175.24793091,411.07210297)
\curveto(175.38792987,411.37209673)(175.56292969,411.65709644)(175.77293091,411.92710297)
\curveto(175.79292946,411.95709614)(175.81792944,411.98209612)(175.84793091,412.00210297)
\curveto(175.88792937,412.03209607)(175.91792934,412.06209604)(175.93793091,412.09210297)
\curveto(175.97792928,412.14209596)(176.01792924,412.18709591)(176.05793091,412.22710297)
\curveto(176.09792916,412.26709583)(176.14292911,412.30709579)(176.19293091,412.34710297)
\curveto(176.23292902,412.36709573)(176.26792899,412.39209571)(176.29793091,412.42210297)
\curveto(176.32792893,412.46209564)(176.36292889,412.49209561)(176.40293091,412.51210297)
\curveto(176.6529286,412.68209542)(176.94292831,412.82209528)(177.27293091,412.93210297)
\curveto(177.34292791,412.95209515)(177.41292784,412.96709513)(177.48293091,412.97710297)
\curveto(177.56292769,412.98709511)(177.64292761,413.0020951)(177.72293091,413.02210297)
\curveto(177.79292746,413.04209506)(177.88292737,413.05209505)(177.99293091,413.05210297)
\curveto(178.10292715,413.06209504)(178.21292704,413.06709503)(178.32293091,413.06710297)
\curveto(178.43292682,413.06709503)(178.53792672,413.06209504)(178.63793091,413.05210297)
\curveto(178.74792651,413.04209506)(178.83792642,413.02709507)(178.90793091,413.00710297)
\curveto(179.0579262,412.95709514)(179.20292605,412.91209519)(179.34293091,412.87210297)
\curveto(179.48292577,412.83209527)(179.61292564,412.77709532)(179.73293091,412.70710297)
\curveto(179.80292545,412.65709544)(179.86792539,412.60709549)(179.92793091,412.55710297)
\curveto(179.98792527,412.51709558)(180.0529252,412.47209563)(180.12293091,412.42210297)
\curveto(180.16292509,412.39209571)(180.21792504,412.35209575)(180.28793091,412.30210297)
\curveto(180.36792489,412.25209585)(180.44292481,412.25209585)(180.51293091,412.30210297)
\curveto(180.5529247,412.32209578)(180.57292468,412.35709574)(180.57293091,412.40710297)
\curveto(180.57292468,412.45709564)(180.58292467,412.50709559)(180.60293091,412.55710297)
\lineto(180.60293091,412.70710297)
\curveto(180.61292464,412.73709536)(180.61792464,412.77209533)(180.61793091,412.81210297)
\lineto(180.61793091,412.93210297)
\lineto(180.61793091,414.97210297)
\curveto(180.61792464,415.08209302)(180.61292464,415.2020929)(180.60293091,415.33210297)
\curveto(180.60292465,415.47209263)(180.62792463,415.57709252)(180.67793091,415.64710297)
\curveto(180.71792454,415.72709237)(180.79292446,415.77709232)(180.90293091,415.79710297)
\curveto(180.92292433,415.80709229)(180.94292431,415.80709229)(180.96293091,415.79710297)
\curveto(180.98292427,415.7970923)(181.00292425,415.8020923)(181.02293091,415.81210297)
\lineto(182.08793091,415.81210297)
\curveto(182.20792305,415.81209229)(182.31792294,415.80709229)(182.41793091,415.79710297)
\curveto(182.51792274,415.78709231)(182.59292266,415.74709235)(182.64293091,415.67710297)
\curveto(182.69292256,415.5970925)(182.71792254,415.49209261)(182.71793091,415.36210297)
\lineto(182.71793091,415.00210297)
\lineto(182.71793091,405.98710297)
\moveto(180.67793091,408.92710297)
\curveto(180.68792457,408.96709913)(180.68792457,409.00709909)(180.67793091,409.04710297)
\lineto(180.67793091,409.18210297)
\curveto(180.67792458,409.28209882)(180.67292458,409.38209872)(180.66293091,409.48210297)
\curveto(180.6529246,409.58209852)(180.63792462,409.67209843)(180.61793091,409.75210297)
\curveto(180.59792466,409.86209824)(180.57792468,409.96209814)(180.55793091,410.05210297)
\curveto(180.54792471,410.14209796)(180.52292473,410.22709787)(180.48293091,410.30710297)
\curveto(180.34292491,410.66709743)(180.13792512,410.95209715)(179.86793091,411.16210297)
\curveto(179.60792565,411.37209673)(179.22792603,411.47709662)(178.72793091,411.47710297)
\curveto(178.66792659,411.47709662)(178.58792667,411.46709663)(178.48793091,411.44710297)
\curveto(178.40792685,411.42709667)(178.33292692,411.40709669)(178.26293091,411.38710297)
\curveto(178.20292705,411.37709672)(178.14292711,411.35709674)(178.08293091,411.32710297)
\curveto(177.81292744,411.21709688)(177.60292765,411.04709705)(177.45293091,410.81710297)
\curveto(177.30292795,410.58709751)(177.18292807,410.32709777)(177.09293091,410.03710297)
\curveto(177.06292819,409.93709816)(177.04292821,409.83709826)(177.03293091,409.73710297)
\curveto(177.02292823,409.63709846)(177.00292825,409.53209857)(176.97293091,409.42210297)
\lineto(176.97293091,409.21210297)
\curveto(176.9529283,409.12209898)(176.94792831,408.9970991)(176.95793091,408.83710297)
\curveto(176.96792829,408.68709941)(176.98292827,408.57709952)(177.00293091,408.50710297)
\lineto(177.00293091,408.41710297)
\curveto(177.01292824,408.3970997)(177.01792824,408.37709972)(177.01793091,408.35710297)
\curveto(177.03792822,408.27709982)(177.0529282,408.2020999)(177.06293091,408.13210297)
\curveto(177.08292817,408.06210004)(177.10292815,407.98710011)(177.12293091,407.90710297)
\curveto(177.29292796,407.38710071)(177.58292767,407.0021011)(177.99293091,406.75210297)
\curveto(178.12292713,406.66210144)(178.30292695,406.59210151)(178.53293091,406.54210297)
\curveto(178.57292668,406.53210157)(178.63292662,406.52710157)(178.71293091,406.52710297)
\curveto(178.74292651,406.51710158)(178.78792647,406.50710159)(178.84793091,406.49710297)
\curveto(178.91792634,406.4971016)(178.97292628,406.5021016)(179.01293091,406.51210297)
\curveto(179.09292616,406.53210157)(179.17292608,406.54710155)(179.25293091,406.55710297)
\curveto(179.33292592,406.56710153)(179.41292584,406.58710151)(179.49293091,406.61710297)
\curveto(179.74292551,406.72710137)(179.94292531,406.86710123)(180.09293091,407.03710297)
\curveto(180.24292501,407.20710089)(180.37292488,407.42210068)(180.48293091,407.68210297)
\curveto(180.52292473,407.77210033)(180.5529247,407.86210024)(180.57293091,407.95210297)
\curveto(180.59292466,408.05210005)(180.61292464,408.15709994)(180.63293091,408.26710297)
\curveto(180.64292461,408.31709978)(180.64292461,408.36209974)(180.63293091,408.40210297)
\curveto(180.63292462,408.45209965)(180.64292461,408.5020996)(180.66293091,408.55210297)
\curveto(180.67292458,408.58209952)(180.67792458,408.61709948)(180.67793091,408.65710297)
\lineto(180.67793091,408.79210297)
\lineto(180.67793091,408.92710297)
}
}
{
\newrgbcolor{curcolor}{0 0 0}
\pscustom[linestyle=none,fillstyle=solid,fillcolor=curcolor]
{
\newpath
\moveto(191.66285278,409.07710297)
\curveto(191.68284462,408.9970991)(191.68284462,408.90709919)(191.66285278,408.80710297)
\curveto(191.64284466,408.70709939)(191.60784469,408.64209946)(191.55785278,408.61210297)
\curveto(191.50784479,408.57209953)(191.43284487,408.54209956)(191.33285278,408.52210297)
\curveto(191.24284506,408.51209959)(191.13784516,408.5020996)(191.01785278,408.49210297)
\lineto(190.67285278,408.49210297)
\curveto(190.56284574,408.5020996)(190.46284584,408.50709959)(190.37285278,408.50710297)
\lineto(186.71285278,408.50710297)
\lineto(186.50285278,408.50710297)
\curveto(186.44284986,408.50709959)(186.38784991,408.4970996)(186.33785278,408.47710297)
\curveto(186.25785004,408.43709966)(186.20785009,408.3970997)(186.18785278,408.35710297)
\curveto(186.16785013,408.33709976)(186.14785015,408.2970998)(186.12785278,408.23710297)
\curveto(186.10785019,408.18709991)(186.1028502,408.13709996)(186.11285278,408.08710297)
\curveto(186.13285017,408.02710007)(186.14285016,407.96710013)(186.14285278,407.90710297)
\curveto(186.15285015,407.85710024)(186.16785013,407.8021003)(186.18785278,407.74210297)
\curveto(186.26785003,407.5021006)(186.36284994,407.3021008)(186.47285278,407.14210297)
\curveto(186.59284971,406.99210111)(186.75284955,406.85710124)(186.95285278,406.73710297)
\curveto(187.03284927,406.68710141)(187.11284919,406.65210145)(187.19285278,406.63210297)
\curveto(187.28284902,406.62210148)(187.37284893,406.6021015)(187.46285278,406.57210297)
\curveto(187.54284876,406.55210155)(187.65284865,406.53710156)(187.79285278,406.52710297)
\curveto(187.93284837,406.51710158)(188.05284825,406.52210158)(188.15285278,406.54210297)
\lineto(188.28785278,406.54210297)
\curveto(188.38784791,406.56210154)(188.47784782,406.58210152)(188.55785278,406.60210297)
\curveto(188.64784765,406.63210147)(188.73284757,406.66210144)(188.81285278,406.69210297)
\curveto(188.91284739,406.74210136)(189.02284728,406.80710129)(189.14285278,406.88710297)
\curveto(189.27284703,406.96710113)(189.36784693,407.04710105)(189.42785278,407.12710297)
\curveto(189.47784682,407.1971009)(189.52784677,407.26210084)(189.57785278,407.32210297)
\curveto(189.63784666,407.39210071)(189.70784659,407.44210066)(189.78785278,407.47210297)
\curveto(189.88784641,407.52210058)(190.01284629,407.54210056)(190.16285278,407.53210297)
\lineto(190.59785278,407.53210297)
\lineto(190.77785278,407.53210297)
\curveto(190.84784545,407.54210056)(190.90784539,407.53710056)(190.95785278,407.51710297)
\lineto(191.10785278,407.51710297)
\curveto(191.20784509,407.4971006)(191.27784502,407.47210063)(191.31785278,407.44210297)
\curveto(191.35784494,407.42210068)(191.37784492,407.37710072)(191.37785278,407.30710297)
\curveto(191.38784491,407.23710086)(191.38284492,407.17710092)(191.36285278,407.12710297)
\curveto(191.31284499,406.98710111)(191.25784504,406.86210124)(191.19785278,406.75210297)
\curveto(191.13784516,406.64210146)(191.06784523,406.53210157)(190.98785278,406.42210297)
\curveto(190.76784553,406.09210201)(190.51784578,405.82710227)(190.23785278,405.62710297)
\curveto(189.95784634,405.42710267)(189.60784669,405.25710284)(189.18785278,405.11710297)
\curveto(189.07784722,405.07710302)(188.96784733,405.05210305)(188.85785278,405.04210297)
\curveto(188.74784755,405.03210307)(188.63284767,405.01210309)(188.51285278,404.98210297)
\curveto(188.47284783,404.97210313)(188.42784787,404.97210313)(188.37785278,404.98210297)
\curveto(188.33784796,404.98210312)(188.297848,404.97710312)(188.25785278,404.96710297)
\lineto(188.09285278,404.96710297)
\curveto(188.04284826,404.94710315)(187.98284832,404.94210316)(187.91285278,404.95210297)
\curveto(187.85284845,404.95210315)(187.7978485,404.95710314)(187.74785278,404.96710297)
\curveto(187.66784863,404.97710312)(187.5978487,404.97710312)(187.53785278,404.96710297)
\curveto(187.47784882,404.95710314)(187.41284889,404.96210314)(187.34285278,404.98210297)
\curveto(187.29284901,405.0021031)(187.23784906,405.01210309)(187.17785278,405.01210297)
\curveto(187.11784918,405.01210309)(187.06284924,405.02210308)(187.01285278,405.04210297)
\curveto(186.9028494,405.06210304)(186.79284951,405.08710301)(186.68285278,405.11710297)
\curveto(186.57284973,405.13710296)(186.47284983,405.17210293)(186.38285278,405.22210297)
\curveto(186.27285003,405.26210284)(186.16785013,405.2971028)(186.06785278,405.32710297)
\curveto(185.97785032,405.36710273)(185.89285041,405.41210269)(185.81285278,405.46210297)
\curveto(185.49285081,405.66210244)(185.20785109,405.89210221)(184.95785278,406.15210297)
\curveto(184.70785159,406.42210168)(184.5028518,406.73210137)(184.34285278,407.08210297)
\curveto(184.29285201,407.19210091)(184.25285205,407.3021008)(184.22285278,407.41210297)
\curveto(184.19285211,407.53210057)(184.15285215,407.65210045)(184.10285278,407.77210297)
\curveto(184.09285221,407.81210029)(184.08785221,407.84710025)(184.08785278,407.87710297)
\curveto(184.08785221,407.91710018)(184.08285222,407.95710014)(184.07285278,407.99710297)
\curveto(184.03285227,408.11709998)(184.00785229,408.24709985)(183.99785278,408.38710297)
\lineto(183.96785278,408.80710297)
\curveto(183.96785233,408.85709924)(183.96285234,408.91209919)(183.95285278,408.97210297)
\curveto(183.95285235,409.03209907)(183.95785234,409.08709901)(183.96785278,409.13710297)
\lineto(183.96785278,409.31710297)
\lineto(184.01285278,409.67710297)
\curveto(184.05285225,409.84709825)(184.08785221,410.01209809)(184.11785278,410.17210297)
\curveto(184.14785215,410.33209777)(184.19285211,410.48209762)(184.25285278,410.62210297)
\curveto(184.68285162,411.66209644)(185.41285089,412.3970957)(186.44285278,412.82710297)
\curveto(186.58284972,412.88709521)(186.72284958,412.92709517)(186.86285278,412.94710297)
\curveto(187.01284929,412.97709512)(187.16784913,413.01209509)(187.32785278,413.05210297)
\curveto(187.40784889,413.06209504)(187.48284882,413.06709503)(187.55285278,413.06710297)
\curveto(187.62284868,413.06709503)(187.6978486,413.07209503)(187.77785278,413.08210297)
\curveto(188.28784801,413.09209501)(188.72284758,413.03209507)(189.08285278,412.90210297)
\curveto(189.45284685,412.78209532)(189.78284652,412.62209548)(190.07285278,412.42210297)
\curveto(190.16284614,412.36209574)(190.25284605,412.29209581)(190.34285278,412.21210297)
\curveto(190.43284587,412.14209596)(190.51284579,412.06709603)(190.58285278,411.98710297)
\curveto(190.61284569,411.93709616)(190.65284565,411.8970962)(190.70285278,411.86710297)
\curveto(190.78284552,411.75709634)(190.85784544,411.64209646)(190.92785278,411.52210297)
\curveto(190.9978453,411.41209669)(191.07284523,411.2970968)(191.15285278,411.17710297)
\curveto(191.2028451,411.08709701)(191.24284506,410.99209711)(191.27285278,410.89210297)
\curveto(191.31284499,410.8020973)(191.35284495,410.7020974)(191.39285278,410.59210297)
\curveto(191.44284486,410.46209764)(191.48284482,410.32709777)(191.51285278,410.18710297)
\curveto(191.54284476,410.04709805)(191.57784472,409.90709819)(191.61785278,409.76710297)
\curveto(191.63784466,409.68709841)(191.64284466,409.5970985)(191.63285278,409.49710297)
\curveto(191.63284467,409.40709869)(191.64284466,409.32209878)(191.66285278,409.24210297)
\lineto(191.66285278,409.07710297)
\moveto(189.41285278,409.96210297)
\curveto(189.48284682,410.06209804)(189.48784681,410.18209792)(189.42785278,410.32210297)
\curveto(189.37784692,410.47209763)(189.33784696,410.58209752)(189.30785278,410.65210297)
\curveto(189.16784713,410.92209718)(188.98284732,411.12709697)(188.75285278,411.26710297)
\curveto(188.52284778,411.41709668)(188.2028481,411.4970966)(187.79285278,411.50710297)
\curveto(187.76284854,411.48709661)(187.72784857,411.48209662)(187.68785278,411.49210297)
\curveto(187.64784865,411.5020966)(187.61284869,411.5020966)(187.58285278,411.49210297)
\curveto(187.53284877,411.47209663)(187.47784882,411.45709664)(187.41785278,411.44710297)
\curveto(187.35784894,411.44709665)(187.302849,411.43709666)(187.25285278,411.41710297)
\curveto(186.81284949,411.27709682)(186.48784981,411.0020971)(186.27785278,410.59210297)
\curveto(186.25785004,410.55209755)(186.23285007,410.4970976)(186.20285278,410.42710297)
\curveto(186.18285012,410.36709773)(186.16785013,410.3020978)(186.15785278,410.23210297)
\curveto(186.14785015,410.17209793)(186.14785015,410.11209799)(186.15785278,410.05210297)
\curveto(186.17785012,409.99209811)(186.21285009,409.94209816)(186.26285278,409.90210297)
\curveto(186.34284996,409.85209825)(186.45284985,409.82709827)(186.59285278,409.82710297)
\lineto(186.99785278,409.82710297)
\lineto(188.66285278,409.82710297)
\lineto(189.09785278,409.82710297)
\curveto(189.25784704,409.83709826)(189.36284694,409.88209822)(189.41285278,409.96210297)
}
}
{
\newrgbcolor{curcolor}{0 0 0}
\pscustom[linestyle=none,fillstyle=solid,fillcolor=curcolor]
{
}
}
{
\newrgbcolor{curcolor}{0 0 0}
\pscustom[linestyle=none,fillstyle=solid,fillcolor=curcolor]
{
\newpath
\moveto(199.17129028,415.72210297)
\curveto(199.24128733,415.64209246)(199.2762873,415.52209258)(199.27629028,415.36210297)
\lineto(199.27629028,414.89710297)
\lineto(199.27629028,414.49210297)
\curveto(199.2762873,414.35209375)(199.24128733,414.25709384)(199.17129028,414.20710297)
\curveto(199.11128746,414.15709394)(199.03128754,414.12709397)(198.93129028,414.11710297)
\curveto(198.84128773,414.10709399)(198.74128783,414.102094)(198.63129028,414.10210297)
\lineto(197.79129028,414.10210297)
\curveto(197.68128889,414.102094)(197.58128899,414.10709399)(197.49129028,414.11710297)
\curveto(197.41128916,414.12709397)(197.34128923,414.15709394)(197.28129028,414.20710297)
\curveto(197.24128933,414.23709386)(197.21128936,414.29209381)(197.19129028,414.37210297)
\curveto(197.18128939,414.46209364)(197.1712894,414.55709354)(197.16129028,414.65710297)
\lineto(197.16129028,414.98710297)
\curveto(197.1712894,415.097093)(197.1762894,415.19209291)(197.17629028,415.27210297)
\lineto(197.17629028,415.48210297)
\curveto(197.18628939,415.55209255)(197.20628937,415.61209249)(197.23629028,415.66210297)
\curveto(197.25628932,415.7020924)(197.28128929,415.73209237)(197.31129028,415.75210297)
\lineto(197.43129028,415.81210297)
\curveto(197.45128912,415.81209229)(197.4762891,415.81209229)(197.50629028,415.81210297)
\curveto(197.53628904,415.82209228)(197.56128901,415.82709227)(197.58129028,415.82710297)
\lineto(198.67629028,415.82710297)
\curveto(198.7762878,415.82709227)(198.8712877,415.82209228)(198.96129028,415.81210297)
\curveto(199.05128752,415.8020923)(199.12128745,415.77209233)(199.17129028,415.72210297)
\moveto(199.27629028,405.95710297)
\curveto(199.2762873,405.75710234)(199.2712873,405.58710251)(199.26129028,405.44710297)
\curveto(199.25128732,405.30710279)(199.16128741,405.21210289)(198.99129028,405.16210297)
\curveto(198.93128764,405.14210296)(198.86628771,405.13210297)(198.79629028,405.13210297)
\curveto(198.72628785,405.14210296)(198.65128792,405.14710295)(198.57129028,405.14710297)
\lineto(197.73129028,405.14710297)
\curveto(197.64128893,405.14710295)(197.55128902,405.15210295)(197.46129028,405.16210297)
\curveto(197.38128919,405.17210293)(197.32128925,405.2021029)(197.28129028,405.25210297)
\curveto(197.22128935,405.32210278)(197.18628939,405.40710269)(197.17629028,405.50710297)
\lineto(197.17629028,405.85210297)
\lineto(197.17629028,412.18210297)
\lineto(197.17629028,412.48210297)
\curveto(197.1762894,412.58209552)(197.19628938,412.66209544)(197.23629028,412.72210297)
\curveto(197.29628928,412.79209531)(197.38128919,412.83709526)(197.49129028,412.85710297)
\curveto(197.51128906,412.86709523)(197.53628904,412.86709523)(197.56629028,412.85710297)
\curveto(197.60628897,412.85709524)(197.63628894,412.86209524)(197.65629028,412.87210297)
\lineto(198.40629028,412.87210297)
\lineto(198.60129028,412.87210297)
\curveto(198.68128789,412.88209522)(198.74628783,412.88209522)(198.79629028,412.87210297)
\lineto(198.91629028,412.87210297)
\curveto(198.9762876,412.85209525)(199.03128754,412.83709526)(199.08129028,412.82710297)
\curveto(199.13128744,412.81709528)(199.1712874,412.78709531)(199.20129028,412.73710297)
\curveto(199.24128733,412.68709541)(199.26128731,412.61709548)(199.26129028,412.52710297)
\curveto(199.2712873,412.43709566)(199.2762873,412.34209576)(199.27629028,412.24210297)
\lineto(199.27629028,405.95710297)
}
}
{
\newrgbcolor{curcolor}{0 0 0}
\pscustom[linestyle=none,fillstyle=solid,fillcolor=curcolor]
{
\newpath
\moveto(205.36347778,413.06710297)
\curveto(205.96347198,413.08709501)(206.46347148,413.0020951)(206.86347778,412.81210297)
\curveto(207.26347068,412.62209548)(207.57847036,412.34209576)(207.80847778,411.97210297)
\curveto(207.87847006,411.86209624)(207.93347001,411.74209636)(207.97347778,411.61210297)
\curveto(208.01346993,411.49209661)(208.05346989,411.36709673)(208.09347778,411.23710297)
\curveto(208.11346983,411.15709694)(208.12346982,411.08209702)(208.12347778,411.01210297)
\curveto(208.13346981,410.94209716)(208.14846979,410.87209723)(208.16847778,410.80210297)
\curveto(208.16846977,410.74209736)(208.17346977,410.7020974)(208.18347778,410.68210297)
\curveto(208.20346974,410.54209756)(208.21346973,410.3970977)(208.21347778,410.24710297)
\lineto(208.21347778,409.81210297)
\lineto(208.21347778,408.47710297)
\lineto(208.21347778,406.04710297)
\curveto(208.21346973,405.85710224)(208.20846973,405.67210243)(208.19847778,405.49210297)
\curveto(208.19846974,405.32210278)(208.12846981,405.21210289)(207.98847778,405.16210297)
\curveto(207.92847001,405.14210296)(207.85847008,405.13210297)(207.77847778,405.13210297)
\lineto(207.53847778,405.13210297)
\lineto(206.72847778,405.13210297)
\curveto(206.60847133,405.13210297)(206.49847144,405.13710296)(206.39847778,405.14710297)
\curveto(206.30847163,405.16710293)(206.2384717,405.21210289)(206.18847778,405.28210297)
\curveto(206.14847179,405.34210276)(206.12347182,405.41710268)(206.11347778,405.50710297)
\lineto(206.11347778,405.82210297)
\lineto(206.11347778,406.87210297)
\lineto(206.11347778,409.10710297)
\curveto(206.11347183,409.47709862)(206.09847184,409.81709828)(206.06847778,410.12710297)
\curveto(206.0384719,410.44709765)(205.94847199,410.71709738)(205.79847778,410.93710297)
\curveto(205.65847228,411.13709696)(205.45347249,411.27709682)(205.18347778,411.35710297)
\curveto(205.13347281,411.37709672)(205.07847286,411.38709671)(205.01847778,411.38710297)
\curveto(204.96847297,411.38709671)(204.91347303,411.3970967)(204.85347778,411.41710297)
\curveto(204.80347314,411.42709667)(204.7384732,411.42709667)(204.65847778,411.41710297)
\curveto(204.58847335,411.41709668)(204.53347341,411.41209669)(204.49347778,411.40210297)
\curveto(204.45347349,411.39209671)(204.41847352,411.38709671)(204.38847778,411.38710297)
\curveto(204.35847358,411.38709671)(204.32847361,411.38209672)(204.29847778,411.37210297)
\curveto(204.06847387,411.31209679)(203.88347406,411.23209687)(203.74347778,411.13210297)
\curveto(203.42347452,410.9020972)(203.23347471,410.56709753)(203.17347778,410.12710297)
\curveto(203.11347483,409.68709841)(203.08347486,409.19209891)(203.08347778,408.64210297)
\lineto(203.08347778,406.76710297)
\lineto(203.08347778,405.85210297)
\lineto(203.08347778,405.58210297)
\curveto(203.08347486,405.49210261)(203.06847487,405.41710268)(203.03847778,405.35710297)
\curveto(202.98847495,405.24710285)(202.90847503,405.18210292)(202.79847778,405.16210297)
\curveto(202.68847525,405.14210296)(202.55347539,405.13210297)(202.39347778,405.13210297)
\lineto(201.64347778,405.13210297)
\curveto(201.53347641,405.13210297)(201.42347652,405.13710296)(201.31347778,405.14710297)
\curveto(201.20347674,405.15710294)(201.12347682,405.19210291)(201.07347778,405.25210297)
\curveto(201.00347694,405.34210276)(200.96847697,405.47210263)(200.96847778,405.64210297)
\curveto(200.97847696,405.81210229)(200.98347696,405.97210213)(200.98347778,406.12210297)
\lineto(200.98347778,408.16210297)
\lineto(200.98347778,411.46210297)
\lineto(200.98347778,412.22710297)
\lineto(200.98347778,412.52710297)
\curveto(200.99347695,412.61709548)(201.02347692,412.69209541)(201.07347778,412.75210297)
\curveto(201.09347685,412.78209532)(201.12347682,412.8020953)(201.16347778,412.81210297)
\curveto(201.21347673,412.83209527)(201.26347668,412.84709525)(201.31347778,412.85710297)
\lineto(201.38847778,412.85710297)
\curveto(201.4384765,412.86709523)(201.48847645,412.87209523)(201.53847778,412.87210297)
\lineto(201.70347778,412.87210297)
\lineto(202.33347778,412.87210297)
\curveto(202.41347553,412.87209523)(202.48847545,412.86709523)(202.55847778,412.85710297)
\curveto(202.6384753,412.85709524)(202.70847523,412.84709525)(202.76847778,412.82710297)
\curveto(202.8384751,412.7970953)(202.88347506,412.75209535)(202.90347778,412.69210297)
\curveto(202.93347501,412.63209547)(202.95847498,412.56209554)(202.97847778,412.48210297)
\curveto(202.98847495,412.44209566)(202.98847495,412.40709569)(202.97847778,412.37710297)
\curveto(202.97847496,412.34709575)(202.98847495,412.31709578)(203.00847778,412.28710297)
\curveto(203.02847491,412.23709586)(203.0434749,412.20709589)(203.05347778,412.19710297)
\curveto(203.07347487,412.18709591)(203.09847484,412.17209593)(203.12847778,412.15210297)
\curveto(203.2384747,412.14209596)(203.32847461,412.17709592)(203.39847778,412.25710297)
\curveto(203.46847447,412.34709575)(203.5434744,412.41709568)(203.62347778,412.46710297)
\curveto(203.89347405,412.66709543)(204.19347375,412.82709527)(204.52347778,412.94710297)
\curveto(204.61347333,412.97709512)(204.70347324,412.9970951)(204.79347778,413.00710297)
\curveto(204.89347305,413.01709508)(204.99847294,413.03209507)(205.10847778,413.05210297)
\curveto(205.1384728,413.06209504)(205.18347276,413.06209504)(205.24347778,413.05210297)
\curveto(205.30347264,413.05209505)(205.3434726,413.05709504)(205.36347778,413.06710297)
}
}
{
\newrgbcolor{curcolor}{0 0 0}
\pscustom[linestyle=none,fillstyle=solid,fillcolor=curcolor]
{
\newpath
\moveto(210.89472778,415.18210297)
\lineto(211.89972778,415.18210297)
\curveto(212.0497248,415.18209292)(212.17972467,415.17209293)(212.28972778,415.15210297)
\curveto(212.40972444,415.14209296)(212.49472435,415.08209302)(212.54472778,414.97210297)
\curveto(212.56472428,414.92209318)(212.57472427,414.86209324)(212.57472778,414.79210297)
\lineto(212.57472778,414.58210297)
\lineto(212.57472778,413.90710297)
\curveto(212.57472427,413.85709424)(212.56972428,413.7970943)(212.55972778,413.72710297)
\curveto(212.55972429,413.66709443)(212.56472428,413.61209449)(212.57472778,413.56210297)
\lineto(212.57472778,413.39710297)
\curveto(212.57472427,413.31709478)(212.57972427,413.24209486)(212.58972778,413.17210297)
\curveto(212.59972425,413.11209499)(212.62472422,413.05709504)(212.66472778,413.00710297)
\curveto(212.73472411,412.91709518)(212.85972399,412.86709523)(213.03972778,412.85710297)
\lineto(213.57972778,412.85710297)
\lineto(213.75972778,412.85710297)
\curveto(213.81972303,412.85709524)(213.87472297,412.84709525)(213.92472778,412.82710297)
\curveto(214.03472281,412.77709532)(214.09472275,412.68709541)(214.10472778,412.55710297)
\curveto(214.12472272,412.42709567)(214.13472271,412.28209582)(214.13472778,412.12210297)
\lineto(214.13472778,411.91210297)
\curveto(214.1447227,411.84209626)(214.13972271,411.78209632)(214.11972778,411.73210297)
\curveto(214.06972278,411.57209653)(213.96472288,411.48709661)(213.80472778,411.47710297)
\curveto(213.6447232,411.46709663)(213.46472338,411.46209664)(213.26472778,411.46210297)
\lineto(213.12972778,411.46210297)
\curveto(213.08972376,411.47209663)(213.05472379,411.47209663)(213.02472778,411.46210297)
\curveto(212.98472386,411.45209665)(212.9497239,411.44709665)(212.91972778,411.44710297)
\curveto(212.88972396,411.45709664)(212.85972399,411.45209665)(212.82972778,411.43210297)
\curveto(212.7497241,411.41209669)(212.68972416,411.36709673)(212.64972778,411.29710297)
\curveto(212.61972423,411.23709686)(212.59472425,411.16209694)(212.57472778,411.07210297)
\curveto(212.56472428,411.02209708)(212.56472428,410.96709713)(212.57472778,410.90710297)
\curveto(212.58472426,410.84709725)(212.58472426,410.79209731)(212.57472778,410.74210297)
\lineto(212.57472778,409.81210297)
\lineto(212.57472778,408.05710297)
\curveto(212.57472427,407.80710029)(212.57972427,407.58710051)(212.58972778,407.39710297)
\curveto(212.60972424,407.21710088)(212.67472417,407.05710104)(212.78472778,406.91710297)
\curveto(212.83472401,406.85710124)(212.89972395,406.81210129)(212.97972778,406.78210297)
\lineto(213.24972778,406.72210297)
\curveto(213.27972357,406.71210139)(213.30972354,406.70710139)(213.33972778,406.70710297)
\curveto(213.37972347,406.71710138)(213.40972344,406.71710138)(213.42972778,406.70710297)
\lineto(213.59472778,406.70710297)
\curveto(213.70472314,406.70710139)(213.79972305,406.7021014)(213.87972778,406.69210297)
\curveto(213.95972289,406.68210142)(214.02472282,406.64210146)(214.07472778,406.57210297)
\curveto(214.11472273,406.51210159)(214.13472271,406.43210167)(214.13472778,406.33210297)
\lineto(214.13472778,406.04710297)
\curveto(214.13472271,405.83710226)(214.12972272,405.64210246)(214.11972778,405.46210297)
\curveto(214.11972273,405.29210281)(214.03972281,405.17710292)(213.87972778,405.11710297)
\curveto(213.82972302,405.097103)(213.78472306,405.09210301)(213.74472778,405.10210297)
\curveto(213.70472314,405.102103)(213.65972319,405.09210301)(213.60972778,405.07210297)
\lineto(213.45972778,405.07210297)
\curveto(213.43972341,405.07210303)(213.40972344,405.07710302)(213.36972778,405.08710297)
\curveto(213.32972352,405.08710301)(213.29472355,405.08210302)(213.26472778,405.07210297)
\curveto(213.21472363,405.06210304)(213.15972369,405.06210304)(213.09972778,405.07210297)
\lineto(212.94972778,405.07210297)
\lineto(212.79972778,405.07210297)
\curveto(212.7497241,405.06210304)(212.70472414,405.06210304)(212.66472778,405.07210297)
\lineto(212.49972778,405.07210297)
\curveto(212.4497244,405.08210302)(212.39472445,405.08710301)(212.33472778,405.08710297)
\curveto(212.27472457,405.08710301)(212.21972463,405.09210301)(212.16972778,405.10210297)
\curveto(212.09972475,405.11210299)(212.03472481,405.12210298)(211.97472778,405.13210297)
\lineto(211.79472778,405.16210297)
\curveto(211.68472516,405.19210291)(211.57972527,405.22710287)(211.47972778,405.26710297)
\curveto(211.37972547,405.30710279)(211.28472556,405.35210275)(211.19472778,405.40210297)
\lineto(211.10472778,405.46210297)
\curveto(211.07472577,405.49210261)(211.03972581,405.52210258)(210.99972778,405.55210297)
\curveto(210.97972587,405.57210253)(210.95472589,405.59210251)(210.92472778,405.61210297)
\lineto(210.84972778,405.68710297)
\curveto(210.70972614,405.87710222)(210.60472624,406.08710201)(210.53472778,406.31710297)
\curveto(210.51472633,406.35710174)(210.50472634,406.39210171)(210.50472778,406.42210297)
\curveto(210.51472633,406.46210164)(210.51472633,406.50710159)(210.50472778,406.55710297)
\curveto(210.49472635,406.57710152)(210.48972636,406.6021015)(210.48972778,406.63210297)
\curveto(210.48972636,406.66210144)(210.48472636,406.68710141)(210.47472778,406.70710297)
\lineto(210.47472778,406.85710297)
\curveto(210.46472638,406.8971012)(210.45972639,406.94210116)(210.45972778,406.99210297)
\curveto(210.46972638,407.04210106)(210.47472637,407.09210101)(210.47472778,407.14210297)
\lineto(210.47472778,407.71210297)
\lineto(210.47472778,409.94710297)
\lineto(210.47472778,410.74210297)
\lineto(210.47472778,410.95210297)
\curveto(210.48472636,411.02209708)(210.47972637,411.08709701)(210.45972778,411.14710297)
\curveto(210.41972643,411.28709681)(210.3497265,411.37709672)(210.24972778,411.41710297)
\curveto(210.13972671,411.46709663)(209.99972685,411.48209662)(209.82972778,411.46210297)
\curveto(209.65972719,411.44209666)(209.51472733,411.45709664)(209.39472778,411.50710297)
\curveto(209.31472753,411.53709656)(209.26472758,411.58209652)(209.24472778,411.64210297)
\curveto(209.22472762,411.7020964)(209.20472764,411.77709632)(209.18472778,411.86710297)
\lineto(209.18472778,412.18210297)
\curveto(209.18472766,412.36209574)(209.19472765,412.50709559)(209.21472778,412.61710297)
\curveto(209.23472761,412.72709537)(209.31972753,412.8020953)(209.46972778,412.84210297)
\curveto(209.50972734,412.86209524)(209.5497273,412.86709523)(209.58972778,412.85710297)
\lineto(209.72472778,412.85710297)
\curveto(209.87472697,412.85709524)(210.01472683,412.86209524)(210.14472778,412.87210297)
\curveto(210.27472657,412.89209521)(210.36472648,412.95209515)(210.41472778,413.05210297)
\curveto(210.4447264,413.12209498)(210.45972639,413.2020949)(210.45972778,413.29210297)
\curveto(210.46972638,413.38209472)(210.47472637,413.47209463)(210.47472778,413.56210297)
\lineto(210.47472778,414.49210297)
\lineto(210.47472778,414.74710297)
\curveto(210.47472637,414.83709326)(210.48472636,414.91209319)(210.50472778,414.97210297)
\curveto(210.55472629,415.07209303)(210.62972622,415.13709296)(210.72972778,415.16710297)
\curveto(210.7497261,415.17709292)(210.77472607,415.17709292)(210.80472778,415.16710297)
\curveto(210.844726,415.16709293)(210.87472597,415.17209293)(210.89472778,415.18210297)
}
}
{
\newrgbcolor{curcolor}{0 0 0}
\pscustom[linestyle=none,fillstyle=solid,fillcolor=curcolor]
{
\newpath
\moveto(222.48316528,409.07710297)
\curveto(222.50315712,408.9970991)(222.50315712,408.90709919)(222.48316528,408.80710297)
\curveto(222.46315716,408.70709939)(222.42815719,408.64209946)(222.37816528,408.61210297)
\curveto(222.32815729,408.57209953)(222.25315737,408.54209956)(222.15316528,408.52210297)
\curveto(222.06315756,408.51209959)(221.95815766,408.5020996)(221.83816528,408.49210297)
\lineto(221.49316528,408.49210297)
\curveto(221.38315824,408.5020996)(221.28315834,408.50709959)(221.19316528,408.50710297)
\lineto(217.53316528,408.50710297)
\lineto(217.32316528,408.50710297)
\curveto(217.26316236,408.50709959)(217.20816241,408.4970996)(217.15816528,408.47710297)
\curveto(217.07816254,408.43709966)(217.02816259,408.3970997)(217.00816528,408.35710297)
\curveto(216.98816263,408.33709976)(216.96816265,408.2970998)(216.94816528,408.23710297)
\curveto(216.92816269,408.18709991)(216.9231627,408.13709996)(216.93316528,408.08710297)
\curveto(216.95316267,408.02710007)(216.96316266,407.96710013)(216.96316528,407.90710297)
\curveto(216.97316265,407.85710024)(216.98816263,407.8021003)(217.00816528,407.74210297)
\curveto(217.08816253,407.5021006)(217.18316244,407.3021008)(217.29316528,407.14210297)
\curveto(217.41316221,406.99210111)(217.57316205,406.85710124)(217.77316528,406.73710297)
\curveto(217.85316177,406.68710141)(217.93316169,406.65210145)(218.01316528,406.63210297)
\curveto(218.10316152,406.62210148)(218.19316143,406.6021015)(218.28316528,406.57210297)
\curveto(218.36316126,406.55210155)(218.47316115,406.53710156)(218.61316528,406.52710297)
\curveto(218.75316087,406.51710158)(218.87316075,406.52210158)(218.97316528,406.54210297)
\lineto(219.10816528,406.54210297)
\curveto(219.20816041,406.56210154)(219.29816032,406.58210152)(219.37816528,406.60210297)
\curveto(219.46816015,406.63210147)(219.55316007,406.66210144)(219.63316528,406.69210297)
\curveto(219.73315989,406.74210136)(219.84315978,406.80710129)(219.96316528,406.88710297)
\curveto(220.09315953,406.96710113)(220.18815943,407.04710105)(220.24816528,407.12710297)
\curveto(220.29815932,407.1971009)(220.34815927,407.26210084)(220.39816528,407.32210297)
\curveto(220.45815916,407.39210071)(220.52815909,407.44210066)(220.60816528,407.47210297)
\curveto(220.70815891,407.52210058)(220.83315879,407.54210056)(220.98316528,407.53210297)
\lineto(221.41816528,407.53210297)
\lineto(221.59816528,407.53210297)
\curveto(221.66815795,407.54210056)(221.72815789,407.53710056)(221.77816528,407.51710297)
\lineto(221.92816528,407.51710297)
\curveto(222.02815759,407.4971006)(222.09815752,407.47210063)(222.13816528,407.44210297)
\curveto(222.17815744,407.42210068)(222.19815742,407.37710072)(222.19816528,407.30710297)
\curveto(222.20815741,407.23710086)(222.20315742,407.17710092)(222.18316528,407.12710297)
\curveto(222.13315749,406.98710111)(222.07815754,406.86210124)(222.01816528,406.75210297)
\curveto(221.95815766,406.64210146)(221.88815773,406.53210157)(221.80816528,406.42210297)
\curveto(221.58815803,406.09210201)(221.33815828,405.82710227)(221.05816528,405.62710297)
\curveto(220.77815884,405.42710267)(220.42815919,405.25710284)(220.00816528,405.11710297)
\curveto(219.89815972,405.07710302)(219.78815983,405.05210305)(219.67816528,405.04210297)
\curveto(219.56816005,405.03210307)(219.45316017,405.01210309)(219.33316528,404.98210297)
\curveto(219.29316033,404.97210313)(219.24816037,404.97210313)(219.19816528,404.98210297)
\curveto(219.15816046,404.98210312)(219.1181605,404.97710312)(219.07816528,404.96710297)
\lineto(218.91316528,404.96710297)
\curveto(218.86316076,404.94710315)(218.80316082,404.94210316)(218.73316528,404.95210297)
\curveto(218.67316095,404.95210315)(218.618161,404.95710314)(218.56816528,404.96710297)
\curveto(218.48816113,404.97710312)(218.4181612,404.97710312)(218.35816528,404.96710297)
\curveto(218.29816132,404.95710314)(218.23316139,404.96210314)(218.16316528,404.98210297)
\curveto(218.11316151,405.0021031)(218.05816156,405.01210309)(217.99816528,405.01210297)
\curveto(217.93816168,405.01210309)(217.88316174,405.02210308)(217.83316528,405.04210297)
\curveto(217.7231619,405.06210304)(217.61316201,405.08710301)(217.50316528,405.11710297)
\curveto(217.39316223,405.13710296)(217.29316233,405.17210293)(217.20316528,405.22210297)
\curveto(217.09316253,405.26210284)(216.98816263,405.2971028)(216.88816528,405.32710297)
\curveto(216.79816282,405.36710273)(216.71316291,405.41210269)(216.63316528,405.46210297)
\curveto(216.31316331,405.66210244)(216.02816359,405.89210221)(215.77816528,406.15210297)
\curveto(215.52816409,406.42210168)(215.3231643,406.73210137)(215.16316528,407.08210297)
\curveto(215.11316451,407.19210091)(215.07316455,407.3021008)(215.04316528,407.41210297)
\curveto(215.01316461,407.53210057)(214.97316465,407.65210045)(214.92316528,407.77210297)
\curveto(214.91316471,407.81210029)(214.90816471,407.84710025)(214.90816528,407.87710297)
\curveto(214.90816471,407.91710018)(214.90316472,407.95710014)(214.89316528,407.99710297)
\curveto(214.85316477,408.11709998)(214.82816479,408.24709985)(214.81816528,408.38710297)
\lineto(214.78816528,408.80710297)
\curveto(214.78816483,408.85709924)(214.78316484,408.91209919)(214.77316528,408.97210297)
\curveto(214.77316485,409.03209907)(214.77816484,409.08709901)(214.78816528,409.13710297)
\lineto(214.78816528,409.31710297)
\lineto(214.83316528,409.67710297)
\curveto(214.87316475,409.84709825)(214.90816471,410.01209809)(214.93816528,410.17210297)
\curveto(214.96816465,410.33209777)(215.01316461,410.48209762)(215.07316528,410.62210297)
\curveto(215.50316412,411.66209644)(216.23316339,412.3970957)(217.26316528,412.82710297)
\curveto(217.40316222,412.88709521)(217.54316208,412.92709517)(217.68316528,412.94710297)
\curveto(217.83316179,412.97709512)(217.98816163,413.01209509)(218.14816528,413.05210297)
\curveto(218.22816139,413.06209504)(218.30316132,413.06709503)(218.37316528,413.06710297)
\curveto(218.44316118,413.06709503)(218.5181611,413.07209503)(218.59816528,413.08210297)
\curveto(219.10816051,413.09209501)(219.54316008,413.03209507)(219.90316528,412.90210297)
\curveto(220.27315935,412.78209532)(220.60315902,412.62209548)(220.89316528,412.42210297)
\curveto(220.98315864,412.36209574)(221.07315855,412.29209581)(221.16316528,412.21210297)
\curveto(221.25315837,412.14209596)(221.33315829,412.06709603)(221.40316528,411.98710297)
\curveto(221.43315819,411.93709616)(221.47315815,411.8970962)(221.52316528,411.86710297)
\curveto(221.60315802,411.75709634)(221.67815794,411.64209646)(221.74816528,411.52210297)
\curveto(221.8181578,411.41209669)(221.89315773,411.2970968)(221.97316528,411.17710297)
\curveto(222.0231576,411.08709701)(222.06315756,410.99209711)(222.09316528,410.89210297)
\curveto(222.13315749,410.8020973)(222.17315745,410.7020974)(222.21316528,410.59210297)
\curveto(222.26315736,410.46209764)(222.30315732,410.32709777)(222.33316528,410.18710297)
\curveto(222.36315726,410.04709805)(222.39815722,409.90709819)(222.43816528,409.76710297)
\curveto(222.45815716,409.68709841)(222.46315716,409.5970985)(222.45316528,409.49710297)
\curveto(222.45315717,409.40709869)(222.46315716,409.32209878)(222.48316528,409.24210297)
\lineto(222.48316528,409.07710297)
\moveto(220.23316528,409.96210297)
\curveto(220.30315932,410.06209804)(220.30815931,410.18209792)(220.24816528,410.32210297)
\curveto(220.19815942,410.47209763)(220.15815946,410.58209752)(220.12816528,410.65210297)
\curveto(219.98815963,410.92209718)(219.80315982,411.12709697)(219.57316528,411.26710297)
\curveto(219.34316028,411.41709668)(219.0231606,411.4970966)(218.61316528,411.50710297)
\curveto(218.58316104,411.48709661)(218.54816107,411.48209662)(218.50816528,411.49210297)
\curveto(218.46816115,411.5020966)(218.43316119,411.5020966)(218.40316528,411.49210297)
\curveto(218.35316127,411.47209663)(218.29816132,411.45709664)(218.23816528,411.44710297)
\curveto(218.17816144,411.44709665)(218.1231615,411.43709666)(218.07316528,411.41710297)
\curveto(217.63316199,411.27709682)(217.30816231,411.0020971)(217.09816528,410.59210297)
\curveto(217.07816254,410.55209755)(217.05316257,410.4970976)(217.02316528,410.42710297)
\curveto(217.00316262,410.36709773)(216.98816263,410.3020978)(216.97816528,410.23210297)
\curveto(216.96816265,410.17209793)(216.96816265,410.11209799)(216.97816528,410.05210297)
\curveto(216.99816262,409.99209811)(217.03316259,409.94209816)(217.08316528,409.90210297)
\curveto(217.16316246,409.85209825)(217.27316235,409.82709827)(217.41316528,409.82710297)
\lineto(217.81816528,409.82710297)
\lineto(219.48316528,409.82710297)
\lineto(219.91816528,409.82710297)
\curveto(220.07815954,409.83709826)(220.18315944,409.88209822)(220.23316528,409.96210297)
}
}
{
\newrgbcolor{curcolor}{0 0 0}
\pscustom[linestyle=none,fillstyle=solid,fillcolor=curcolor]
{
\newpath
\moveto(228.15644653,413.06710297)
\curveto(228.26644122,413.06709503)(228.36144112,413.05709504)(228.44144653,413.03710297)
\curveto(228.53144095,413.01709508)(228.60144088,412.97209513)(228.65144653,412.90210297)
\curveto(228.71144077,412.82209528)(228.74144074,412.68209542)(228.74144653,412.48210297)
\lineto(228.74144653,411.97210297)
\lineto(228.74144653,411.59710297)
\curveto(228.75144073,411.45709664)(228.73644075,411.34709675)(228.69644653,411.26710297)
\curveto(228.65644083,411.1970969)(228.59644089,411.15209695)(228.51644653,411.13210297)
\curveto(228.44644104,411.11209699)(228.36144112,411.102097)(228.26144653,411.10210297)
\curveto(228.17144131,411.102097)(228.07144141,411.10709699)(227.96144653,411.11710297)
\curveto(227.86144162,411.12709697)(227.76644172,411.12209698)(227.67644653,411.10210297)
\curveto(227.60644188,411.08209702)(227.53644195,411.06709703)(227.46644653,411.05710297)
\curveto(227.39644209,411.05709704)(227.33144215,411.04709705)(227.27144653,411.02710297)
\curveto(227.11144237,410.97709712)(226.95144253,410.9020972)(226.79144653,410.80210297)
\curveto(226.63144285,410.71209739)(226.50644298,410.60709749)(226.41644653,410.48710297)
\curveto(226.36644312,410.40709769)(226.31144317,410.32209778)(226.25144653,410.23210297)
\curveto(226.20144328,410.15209795)(226.15144333,410.06709803)(226.10144653,409.97710297)
\curveto(226.07144341,409.8970982)(226.04144344,409.81209829)(226.01144653,409.72210297)
\lineto(225.95144653,409.48210297)
\curveto(225.93144355,409.41209869)(225.92144356,409.33709876)(225.92144653,409.25710297)
\curveto(225.92144356,409.18709891)(225.91144357,409.11709898)(225.89144653,409.04710297)
\curveto(225.8814436,409.00709909)(225.87644361,408.96709913)(225.87644653,408.92710297)
\curveto(225.8864436,408.8970992)(225.8864436,408.86709923)(225.87644653,408.83710297)
\lineto(225.87644653,408.59710297)
\curveto(225.85644363,408.52709957)(225.85144363,408.44709965)(225.86144653,408.35710297)
\curveto(225.87144361,408.27709982)(225.87644361,408.1970999)(225.87644653,408.11710297)
\lineto(225.87644653,407.15710297)
\lineto(225.87644653,405.88210297)
\curveto(225.87644361,405.75210235)(225.87144361,405.63210247)(225.86144653,405.52210297)
\curveto(225.85144363,405.41210269)(225.82144366,405.32210278)(225.77144653,405.25210297)
\curveto(225.75144373,405.22210288)(225.71644377,405.1971029)(225.66644653,405.17710297)
\curveto(225.62644386,405.16710293)(225.5814439,405.15710294)(225.53144653,405.14710297)
\lineto(225.45644653,405.14710297)
\curveto(225.40644408,405.13710296)(225.35144413,405.13210297)(225.29144653,405.13210297)
\lineto(225.12644653,405.13210297)
\lineto(224.48144653,405.13210297)
\curveto(224.42144506,405.14210296)(224.35644513,405.14710295)(224.28644653,405.14710297)
\lineto(224.09144653,405.14710297)
\curveto(224.04144544,405.16710293)(223.99144549,405.18210292)(223.94144653,405.19210297)
\curveto(223.89144559,405.21210289)(223.85644563,405.24710285)(223.83644653,405.29710297)
\curveto(223.79644569,405.34710275)(223.77144571,405.41710268)(223.76144653,405.50710297)
\lineto(223.76144653,405.80710297)
\lineto(223.76144653,406.82710297)
\lineto(223.76144653,411.05710297)
\lineto(223.76144653,412.16710297)
\lineto(223.76144653,412.45210297)
\curveto(223.76144572,412.55209555)(223.7814457,412.63209547)(223.82144653,412.69210297)
\curveto(223.87144561,412.77209533)(223.94644554,412.82209528)(224.04644653,412.84210297)
\curveto(224.14644534,412.86209524)(224.26644522,412.87209523)(224.40644653,412.87210297)
\lineto(225.17144653,412.87210297)
\curveto(225.29144419,412.87209523)(225.39644409,412.86209524)(225.48644653,412.84210297)
\curveto(225.57644391,412.83209527)(225.64644384,412.78709531)(225.69644653,412.70710297)
\curveto(225.72644376,412.65709544)(225.74144374,412.58709551)(225.74144653,412.49710297)
\lineto(225.77144653,412.22710297)
\curveto(225.7814437,412.14709595)(225.79644369,412.07209603)(225.81644653,412.00210297)
\curveto(225.84644364,411.93209617)(225.89644359,411.8970962)(225.96644653,411.89710297)
\curveto(225.9864435,411.91709618)(226.00644348,411.92709617)(226.02644653,411.92710297)
\curveto(226.04644344,411.92709617)(226.06644342,411.93709616)(226.08644653,411.95710297)
\curveto(226.14644334,412.00709609)(226.19644329,412.06209604)(226.23644653,412.12210297)
\curveto(226.2864432,412.19209591)(226.34644314,412.25209585)(226.41644653,412.30210297)
\curveto(226.45644303,412.33209577)(226.49144299,412.36209574)(226.52144653,412.39210297)
\curveto(226.55144293,412.43209567)(226.5864429,412.46709563)(226.62644653,412.49710297)
\lineto(226.89644653,412.67710297)
\curveto(226.99644249,412.73709536)(227.09644239,412.79209531)(227.19644653,412.84210297)
\curveto(227.29644219,412.88209522)(227.39644209,412.91709518)(227.49644653,412.94710297)
\lineto(227.82644653,413.03710297)
\curveto(227.85644163,413.04709505)(227.91144157,413.04709505)(227.99144653,413.03710297)
\curveto(228.0814414,413.03709506)(228.13644135,413.04709505)(228.15644653,413.06710297)
}
}
{
\newrgbcolor{curcolor}{0 0 0}
\pscustom[linestyle=none,fillstyle=solid,fillcolor=curcolor]
{
\newpath
\moveto(236.61152466,405.73210297)
\curveto(236.63151681,405.62210248)(236.6415168,405.51210259)(236.64152466,405.40210297)
\curveto(236.65151679,405.29210281)(236.60151684,405.21710288)(236.49152466,405.17710297)
\curveto(236.43151701,405.14710295)(236.36151708,405.13210297)(236.28152466,405.13210297)
\lineto(236.04152466,405.13210297)
\lineto(235.23152466,405.13210297)
\lineto(234.96152466,405.13210297)
\curveto(234.88151856,405.14210296)(234.81651862,405.16710293)(234.76652466,405.20710297)
\curveto(234.69651874,405.24710285)(234.6415188,405.3021028)(234.60152466,405.37210297)
\curveto(234.57151887,405.45210265)(234.52651891,405.51710258)(234.46652466,405.56710297)
\curveto(234.44651899,405.58710251)(234.42151902,405.6021025)(234.39152466,405.61210297)
\curveto(234.36151908,405.63210247)(234.32151912,405.63710246)(234.27152466,405.62710297)
\curveto(234.22151922,405.60710249)(234.17151927,405.58210252)(234.12152466,405.55210297)
\curveto(234.08151936,405.52210258)(234.0365194,405.4971026)(233.98652466,405.47710297)
\curveto(233.9365195,405.43710266)(233.88151956,405.4021027)(233.82152466,405.37210297)
\lineto(233.64152466,405.28210297)
\curveto(233.51151993,405.22210288)(233.37652006,405.17210293)(233.23652466,405.13210297)
\curveto(233.09652034,405.102103)(232.95152049,405.06710303)(232.80152466,405.02710297)
\curveto(232.73152071,405.00710309)(232.66152078,404.9971031)(232.59152466,404.99710297)
\curveto(232.53152091,404.98710311)(232.46652097,404.97710312)(232.39652466,404.96710297)
\lineto(232.30652466,404.96710297)
\curveto(232.27652116,404.95710314)(232.24652119,404.95210315)(232.21652466,404.95210297)
\lineto(232.05152466,404.95210297)
\curveto(231.95152149,404.93210317)(231.85152159,404.93210317)(231.75152466,404.95210297)
\lineto(231.61652466,404.95210297)
\curveto(231.54652189,404.97210313)(231.47652196,404.98210312)(231.40652466,404.98210297)
\curveto(231.34652209,404.97210313)(231.28652215,404.97710312)(231.22652466,404.99710297)
\curveto(231.12652231,405.01710308)(231.03152241,405.03710306)(230.94152466,405.05710297)
\curveto(230.85152259,405.06710303)(230.76652267,405.09210301)(230.68652466,405.13210297)
\curveto(230.39652304,405.24210286)(230.14652329,405.38210272)(229.93652466,405.55210297)
\curveto(229.7365237,405.73210237)(229.57652386,405.96710213)(229.45652466,406.25710297)
\curveto(229.42652401,406.32710177)(229.39652404,406.4021017)(229.36652466,406.48210297)
\curveto(229.34652409,406.56210154)(229.32652411,406.64710145)(229.30652466,406.73710297)
\curveto(229.28652415,406.78710131)(229.27652416,406.83710126)(229.27652466,406.88710297)
\curveto(229.28652415,406.93710116)(229.28652415,406.98710111)(229.27652466,407.03710297)
\curveto(229.26652417,407.06710103)(229.25652418,407.12710097)(229.24652466,407.21710297)
\curveto(229.24652419,407.31710078)(229.25152419,407.38710071)(229.26152466,407.42710297)
\curveto(229.28152416,407.52710057)(229.29152415,407.61210049)(229.29152466,407.68210297)
\lineto(229.38152466,408.01210297)
\curveto(229.41152403,408.13209997)(229.45152399,408.23709986)(229.50152466,408.32710297)
\curveto(229.67152377,408.61709948)(229.86652357,408.83709926)(230.08652466,408.98710297)
\curveto(230.30652313,409.13709896)(230.58652285,409.26709883)(230.92652466,409.37710297)
\curveto(231.05652238,409.42709867)(231.19152225,409.46209864)(231.33152466,409.48210297)
\curveto(231.47152197,409.5020986)(231.61152183,409.52709857)(231.75152466,409.55710297)
\curveto(231.83152161,409.57709852)(231.91652152,409.58709851)(232.00652466,409.58710297)
\curveto(232.09652134,409.5970985)(232.18652125,409.61209849)(232.27652466,409.63210297)
\curveto(232.34652109,409.65209845)(232.41652102,409.65709844)(232.48652466,409.64710297)
\curveto(232.55652088,409.64709845)(232.63152081,409.65709844)(232.71152466,409.67710297)
\curveto(232.78152066,409.6970984)(232.85152059,409.70709839)(232.92152466,409.70710297)
\curveto(232.99152045,409.70709839)(233.06652037,409.71709838)(233.14652466,409.73710297)
\curveto(233.35652008,409.78709831)(233.54651989,409.82709827)(233.71652466,409.85710297)
\curveto(233.89651954,409.8970982)(234.05651938,409.98709811)(234.19652466,410.12710297)
\curveto(234.28651915,410.21709788)(234.34651909,410.31709778)(234.37652466,410.42710297)
\curveto(234.38651905,410.45709764)(234.38651905,410.48209762)(234.37652466,410.50210297)
\curveto(234.37651906,410.52209758)(234.38151906,410.54209756)(234.39152466,410.56210297)
\curveto(234.40151904,410.58209752)(234.40651903,410.61209749)(234.40652466,410.65210297)
\lineto(234.40652466,410.74210297)
\lineto(234.37652466,410.86210297)
\curveto(234.37651906,410.9020972)(234.37151907,410.93709716)(234.36152466,410.96710297)
\curveto(234.26151918,411.26709683)(234.05151939,411.47209663)(233.73152466,411.58210297)
\curveto(233.6415198,411.61209649)(233.53151991,411.63209647)(233.40152466,411.64210297)
\curveto(233.28152016,411.66209644)(233.15652028,411.66709643)(233.02652466,411.65710297)
\curveto(232.89652054,411.65709644)(232.77152067,411.64709645)(232.65152466,411.62710297)
\curveto(232.53152091,411.60709649)(232.42652101,411.58209652)(232.33652466,411.55210297)
\curveto(232.27652116,411.53209657)(232.21652122,411.5020966)(232.15652466,411.46210297)
\curveto(232.10652133,411.43209667)(232.05652138,411.3970967)(232.00652466,411.35710297)
\curveto(231.95652148,411.31709678)(231.90152154,411.26209684)(231.84152466,411.19210297)
\curveto(231.79152165,411.12209698)(231.75652168,411.05709704)(231.73652466,410.99710297)
\curveto(231.68652175,410.8970972)(231.6415218,410.8020973)(231.60152466,410.71210297)
\curveto(231.57152187,410.62209748)(231.50152194,410.56209754)(231.39152466,410.53210297)
\curveto(231.31152213,410.51209759)(231.22652221,410.5020976)(231.13652466,410.50210297)
\lineto(230.86652466,410.50210297)
\lineto(230.29652466,410.50210297)
\curveto(230.24652319,410.5020976)(230.19652324,410.4970976)(230.14652466,410.48710297)
\curveto(230.09652334,410.48709761)(230.05152339,410.49209761)(230.01152466,410.50210297)
\lineto(229.87652466,410.50210297)
\curveto(229.85652358,410.51209759)(229.83152361,410.51709758)(229.80152466,410.51710297)
\curveto(229.77152367,410.51709758)(229.74652369,410.52709757)(229.72652466,410.54710297)
\curveto(229.64652379,410.56709753)(229.59152385,410.63209747)(229.56152466,410.74210297)
\curveto(229.55152389,410.79209731)(229.55152389,410.84209726)(229.56152466,410.89210297)
\curveto(229.57152387,410.94209716)(229.58152386,410.98709711)(229.59152466,411.02710297)
\curveto(229.62152382,411.13709696)(229.65152379,411.23709686)(229.68152466,411.32710297)
\curveto(229.72152372,411.42709667)(229.76652367,411.51709658)(229.81652466,411.59710297)
\lineto(229.90652466,411.74710297)
\lineto(229.99652466,411.89710297)
\curveto(230.07652336,412.00709609)(230.17652326,412.11209599)(230.29652466,412.21210297)
\curveto(230.31652312,412.22209588)(230.34652309,412.24709585)(230.38652466,412.28710297)
\curveto(230.436523,412.32709577)(230.48152296,412.36209574)(230.52152466,412.39210297)
\curveto(230.56152288,412.42209568)(230.60652283,412.45209565)(230.65652466,412.48210297)
\curveto(230.82652261,412.59209551)(231.00652243,412.67709542)(231.19652466,412.73710297)
\curveto(231.38652205,412.80709529)(231.58152186,412.87209523)(231.78152466,412.93210297)
\curveto(231.90152154,412.96209514)(232.02652141,412.98209512)(232.15652466,412.99210297)
\curveto(232.28652115,413.0020951)(232.41652102,413.02209508)(232.54652466,413.05210297)
\curveto(232.58652085,413.06209504)(232.64652079,413.06209504)(232.72652466,413.05210297)
\curveto(232.81652062,413.04209506)(232.87152057,413.04709505)(232.89152466,413.06710297)
\curveto(233.30152014,413.07709502)(233.69151975,413.06209504)(234.06152466,413.02210297)
\curveto(234.441519,412.98209512)(234.78151866,412.90709519)(235.08152466,412.79710297)
\curveto(235.39151805,412.68709541)(235.65651778,412.53709556)(235.87652466,412.34710297)
\curveto(236.09651734,412.16709593)(236.26651717,411.93209617)(236.38652466,411.64210297)
\curveto(236.45651698,411.47209663)(236.49651694,411.27709682)(236.50652466,411.05710297)
\curveto(236.51651692,410.83709726)(236.52151692,410.61209749)(236.52152466,410.38210297)
\lineto(236.52152466,407.03710297)
\lineto(236.52152466,406.45210297)
\curveto(236.52151692,406.26210184)(236.5415169,406.08710201)(236.58152466,405.92710297)
\curveto(236.59151685,405.8971022)(236.59651684,405.86210224)(236.59652466,405.82210297)
\curveto(236.59651684,405.79210231)(236.60151684,405.76210234)(236.61152466,405.73210297)
\moveto(234.40652466,408.04210297)
\curveto(234.41651902,408.09210001)(234.42151902,408.14709995)(234.42152466,408.20710297)
\curveto(234.42151902,408.27709982)(234.41651902,408.33709976)(234.40652466,408.38710297)
\curveto(234.38651905,408.44709965)(234.37651906,408.5020996)(234.37652466,408.55210297)
\curveto(234.37651906,408.6020995)(234.35651908,408.64209946)(234.31652466,408.67210297)
\curveto(234.26651917,408.71209939)(234.19151925,408.73209937)(234.09152466,408.73210297)
\curveto(234.05151939,408.72209938)(234.01651942,408.71209939)(233.98652466,408.70210297)
\curveto(233.95651948,408.7020994)(233.92151952,408.6970994)(233.88152466,408.68710297)
\curveto(233.81151963,408.66709943)(233.7365197,408.65209945)(233.65652466,408.64210297)
\curveto(233.57651986,408.63209947)(233.49651994,408.61709948)(233.41652466,408.59710297)
\curveto(233.38652005,408.58709951)(233.3415201,408.58209952)(233.28152466,408.58210297)
\curveto(233.15152029,408.55209955)(233.02152042,408.53209957)(232.89152466,408.52210297)
\curveto(232.76152068,408.51209959)(232.6365208,408.48709961)(232.51652466,408.44710297)
\curveto(232.436521,408.42709967)(232.36152108,408.40709969)(232.29152466,408.38710297)
\curveto(232.22152122,408.37709972)(232.15152129,408.35709974)(232.08152466,408.32710297)
\curveto(231.87152157,408.23709986)(231.69152175,408.1021)(231.54152466,407.92210297)
\curveto(231.40152204,407.74210036)(231.35152209,407.49210061)(231.39152466,407.17210297)
\curveto(231.41152203,407.0021011)(231.46652197,406.86210124)(231.55652466,406.75210297)
\curveto(231.62652181,406.64210146)(231.73152171,406.55210155)(231.87152466,406.48210297)
\curveto(232.01152143,406.42210168)(232.16152128,406.37710172)(232.32152466,406.34710297)
\curveto(232.49152095,406.31710178)(232.66652077,406.30710179)(232.84652466,406.31710297)
\curveto(233.0365204,406.33710176)(233.21152023,406.37210173)(233.37152466,406.42210297)
\curveto(233.63151981,406.5021016)(233.8365196,406.62710147)(233.98652466,406.79710297)
\curveto(234.1365193,406.97710112)(234.25151919,407.1971009)(234.33152466,407.45710297)
\curveto(234.35151909,407.52710057)(234.36151908,407.5971005)(234.36152466,407.66710297)
\curveto(234.37151907,407.74710035)(234.38651905,407.82710027)(234.40652466,407.90710297)
\lineto(234.40652466,408.04210297)
}
}
{
\newrgbcolor{curcolor}{0 0 0}
\pscustom[linestyle=none,fillstyle=solid,fillcolor=curcolor]
{
\newpath
\moveto(241.74480591,413.08210297)
\curveto(242.55480075,413.102095)(243.22980007,412.98209512)(243.76980591,412.72210297)
\curveto(244.31979898,412.46209564)(244.75479855,412.09209601)(245.07480591,411.61210297)
\curveto(245.23479807,411.37209673)(245.35479795,411.097097)(245.43480591,410.78710297)
\curveto(245.45479785,410.73709736)(245.46979783,410.67209743)(245.47980591,410.59210297)
\curveto(245.4997978,410.51209759)(245.4997978,410.44209766)(245.47980591,410.38210297)
\curveto(245.43979786,410.27209783)(245.36979793,410.20709789)(245.26980591,410.18710297)
\curveto(245.16979813,410.17709792)(245.04979825,410.17209793)(244.90980591,410.17210297)
\lineto(244.12980591,410.17210297)
\lineto(243.84480591,410.17210297)
\curveto(243.75479955,410.17209793)(243.67979962,410.19209791)(243.61980591,410.23210297)
\curveto(243.53979976,410.27209783)(243.48479982,410.33209777)(243.45480591,410.41210297)
\curveto(243.42479988,410.5020976)(243.38479992,410.59209751)(243.33480591,410.68210297)
\curveto(243.27480003,410.79209731)(243.20980009,410.89209721)(243.13980591,410.98210297)
\curveto(243.06980023,411.07209703)(242.98980031,411.15209695)(242.89980591,411.22210297)
\curveto(242.75980054,411.31209679)(242.6048007,411.38209672)(242.43480591,411.43210297)
\curveto(242.37480093,411.45209665)(242.31480099,411.46209664)(242.25480591,411.46210297)
\curveto(242.19480111,411.46209664)(242.13980116,411.47209663)(242.08980591,411.49210297)
\lineto(241.93980591,411.49210297)
\curveto(241.73980156,411.49209661)(241.57980172,411.47209663)(241.45980591,411.43210297)
\curveto(241.16980213,411.34209676)(240.93480237,411.2020969)(240.75480591,411.01210297)
\curveto(240.57480273,410.83209727)(240.42980287,410.61209749)(240.31980591,410.35210297)
\curveto(240.26980303,410.24209786)(240.22980307,410.12209798)(240.19980591,409.99210297)
\curveto(240.17980312,409.87209823)(240.15480315,409.74209836)(240.12480591,409.60210297)
\curveto(240.11480319,409.56209854)(240.10980319,409.52209858)(240.10980591,409.48210297)
\curveto(240.10980319,409.44209866)(240.1048032,409.4020987)(240.09480591,409.36210297)
\curveto(240.07480323,409.26209884)(240.06480324,409.12209898)(240.06480591,408.94210297)
\curveto(240.07480323,408.76209934)(240.08980321,408.62209948)(240.10980591,408.52210297)
\curveto(240.10980319,408.44209966)(240.11480319,408.38709971)(240.12480591,408.35710297)
\curveto(240.14480316,408.28709981)(240.15480315,408.21709988)(240.15480591,408.14710297)
\curveto(240.16480314,408.07710002)(240.17980312,408.00710009)(240.19980591,407.93710297)
\curveto(240.27980302,407.70710039)(240.37480293,407.4971006)(240.48480591,407.30710297)
\curveto(240.59480271,407.11710098)(240.73480257,406.95710114)(240.90480591,406.82710297)
\curveto(240.94480236,406.7971013)(241.0048023,406.76210134)(241.08480591,406.72210297)
\curveto(241.19480211,406.65210145)(241.304802,406.60710149)(241.41480591,406.58710297)
\curveto(241.53480177,406.56710153)(241.67980162,406.54710155)(241.84980591,406.52710297)
\lineto(241.93980591,406.52710297)
\curveto(241.97980132,406.52710157)(242.00980129,406.53210157)(242.02980591,406.54210297)
\lineto(242.16480591,406.54210297)
\curveto(242.23480107,406.56210154)(242.299801,406.57710152)(242.35980591,406.58710297)
\curveto(242.42980087,406.60710149)(242.49480081,406.62710147)(242.55480591,406.64710297)
\curveto(242.85480045,406.77710132)(243.08480022,406.96710113)(243.24480591,407.21710297)
\curveto(243.28480002,407.26710083)(243.31979998,407.32210078)(243.34980591,407.38210297)
\curveto(243.37979992,407.45210065)(243.4047999,407.51210059)(243.42480591,407.56210297)
\curveto(243.46479984,407.67210043)(243.4997998,407.76710033)(243.52980591,407.84710297)
\curveto(243.55979974,407.93710016)(243.62979967,408.00710009)(243.73980591,408.05710297)
\curveto(243.82979947,408.0971)(243.97479933,408.11209999)(244.17480591,408.10210297)
\lineto(244.66980591,408.10210297)
\lineto(244.87980591,408.10210297)
\curveto(244.95979834,408.11209999)(245.02479828,408.10709999)(245.07480591,408.08710297)
\lineto(245.19480591,408.08710297)
\lineto(245.31480591,408.05710297)
\curveto(245.35479795,408.05710004)(245.38479792,408.04710005)(245.40480591,408.02710297)
\curveto(245.45479785,407.98710011)(245.48479782,407.92710017)(245.49480591,407.84710297)
\curveto(245.51479779,407.77710032)(245.51479779,407.7021004)(245.49480591,407.62210297)
\curveto(245.4047979,407.29210081)(245.29479801,406.9971011)(245.16480591,406.73710297)
\curveto(244.75479855,405.96710213)(244.0997992,405.43210267)(243.19980591,405.13210297)
\curveto(243.0998002,405.102103)(242.99480031,405.08210302)(242.88480591,405.07210297)
\curveto(242.77480053,405.05210305)(242.66480064,405.02710307)(242.55480591,404.99710297)
\curveto(242.49480081,404.98710311)(242.43480087,404.98210312)(242.37480591,404.98210297)
\curveto(242.31480099,404.98210312)(242.25480105,404.97710312)(242.19480591,404.96710297)
\lineto(242.02980591,404.96710297)
\curveto(241.97980132,404.94710315)(241.9048014,404.94210316)(241.80480591,404.95210297)
\curveto(241.7048016,404.95210315)(241.62980167,404.95710314)(241.57980591,404.96710297)
\curveto(241.4998018,404.98710311)(241.42480188,404.9971031)(241.35480591,404.99710297)
\curveto(241.29480201,404.98710311)(241.22980207,404.99210311)(241.15980591,405.01210297)
\lineto(241.00980591,405.04210297)
\curveto(240.95980234,405.04210306)(240.90980239,405.04710305)(240.85980591,405.05710297)
\curveto(240.74980255,405.08710301)(240.64480266,405.11710298)(240.54480591,405.14710297)
\curveto(240.44480286,405.17710292)(240.34980295,405.21210289)(240.25980591,405.25210297)
\curveto(239.78980351,405.45210265)(239.39480391,405.70710239)(239.07480591,406.01710297)
\curveto(238.75480455,406.33710176)(238.49480481,406.73210137)(238.29480591,407.20210297)
\curveto(238.24480506,407.29210081)(238.2048051,407.38710071)(238.17480591,407.48710297)
\lineto(238.08480591,407.81710297)
\curveto(238.07480523,407.85710024)(238.06980523,407.89210021)(238.06980591,407.92210297)
\curveto(238.06980523,407.96210014)(238.05980524,408.00710009)(238.03980591,408.05710297)
\curveto(238.01980528,408.12709997)(238.00980529,408.1970999)(238.00980591,408.26710297)
\curveto(238.00980529,408.34709975)(237.9998053,408.42209968)(237.97980591,408.49210297)
\lineto(237.97980591,408.74710297)
\curveto(237.95980534,408.7970993)(237.94980535,408.85209925)(237.94980591,408.91210297)
\curveto(237.94980535,408.98209912)(237.95980534,409.04209906)(237.97980591,409.09210297)
\curveto(237.98980531,409.14209896)(237.98980531,409.18709891)(237.97980591,409.22710297)
\curveto(237.96980533,409.26709883)(237.96980533,409.30709879)(237.97980591,409.34710297)
\curveto(237.9998053,409.41709868)(238.0048053,409.48209862)(237.99480591,409.54210297)
\curveto(237.99480531,409.6020985)(238.0048053,409.66209844)(238.02480591,409.72210297)
\curveto(238.07480523,409.9020982)(238.11480519,410.07209803)(238.14480591,410.23210297)
\curveto(238.17480513,410.4020977)(238.21980508,410.56709753)(238.27980591,410.72710297)
\curveto(238.4998048,411.23709686)(238.77480453,411.66209644)(239.10480591,412.00210297)
\curveto(239.44480386,412.34209576)(239.87480343,412.61709548)(240.39480591,412.82710297)
\curveto(240.53480277,412.88709521)(240.67980262,412.92709517)(240.82980591,412.94710297)
\curveto(240.97980232,412.97709512)(241.13480217,413.01209509)(241.29480591,413.05210297)
\curveto(241.37480193,413.06209504)(241.44980185,413.06709503)(241.51980591,413.06710297)
\curveto(241.58980171,413.06709503)(241.66480164,413.07209503)(241.74480591,413.08210297)
}
}
{
\newrgbcolor{curcolor}{0 0 0}
\pscustom[linestyle=none,fillstyle=solid,fillcolor=curcolor]
{
\newpath
\moveto(250.35808716,413.08210297)
\curveto(251.168082,413.102095)(251.84308132,412.98209512)(252.38308716,412.72210297)
\curveto(252.93308023,412.46209564)(253.3680798,412.09209601)(253.68808716,411.61210297)
\curveto(253.84807932,411.37209673)(253.9680792,411.097097)(254.04808716,410.78710297)
\curveto(254.0680791,410.73709736)(254.08307908,410.67209743)(254.09308716,410.59210297)
\curveto(254.11307905,410.51209759)(254.11307905,410.44209766)(254.09308716,410.38210297)
\curveto(254.05307911,410.27209783)(253.98307918,410.20709789)(253.88308716,410.18710297)
\curveto(253.78307938,410.17709792)(253.6630795,410.17209793)(253.52308716,410.17210297)
\lineto(252.74308716,410.17210297)
\lineto(252.45808716,410.17210297)
\curveto(252.3680808,410.17209793)(252.29308087,410.19209791)(252.23308716,410.23210297)
\curveto(252.15308101,410.27209783)(252.09808107,410.33209777)(252.06808716,410.41210297)
\curveto(252.03808113,410.5020976)(251.99808117,410.59209751)(251.94808716,410.68210297)
\curveto(251.88808128,410.79209731)(251.82308134,410.89209721)(251.75308716,410.98210297)
\curveto(251.68308148,411.07209703)(251.60308156,411.15209695)(251.51308716,411.22210297)
\curveto(251.37308179,411.31209679)(251.21808195,411.38209672)(251.04808716,411.43210297)
\curveto(250.98808218,411.45209665)(250.92808224,411.46209664)(250.86808716,411.46210297)
\curveto(250.80808236,411.46209664)(250.75308241,411.47209663)(250.70308716,411.49210297)
\lineto(250.55308716,411.49210297)
\curveto(250.35308281,411.49209661)(250.19308297,411.47209663)(250.07308716,411.43210297)
\curveto(249.78308338,411.34209676)(249.54808362,411.2020969)(249.36808716,411.01210297)
\curveto(249.18808398,410.83209727)(249.04308412,410.61209749)(248.93308716,410.35210297)
\curveto(248.88308428,410.24209786)(248.84308432,410.12209798)(248.81308716,409.99210297)
\curveto(248.79308437,409.87209823)(248.7680844,409.74209836)(248.73808716,409.60210297)
\curveto(248.72808444,409.56209854)(248.72308444,409.52209858)(248.72308716,409.48210297)
\curveto(248.72308444,409.44209866)(248.71808445,409.4020987)(248.70808716,409.36210297)
\curveto(248.68808448,409.26209884)(248.67808449,409.12209898)(248.67808716,408.94210297)
\curveto(248.68808448,408.76209934)(248.70308446,408.62209948)(248.72308716,408.52210297)
\curveto(248.72308444,408.44209966)(248.72808444,408.38709971)(248.73808716,408.35710297)
\curveto(248.75808441,408.28709981)(248.7680844,408.21709988)(248.76808716,408.14710297)
\curveto(248.77808439,408.07710002)(248.79308437,408.00710009)(248.81308716,407.93710297)
\curveto(248.89308427,407.70710039)(248.98808418,407.4971006)(249.09808716,407.30710297)
\curveto(249.20808396,407.11710098)(249.34808382,406.95710114)(249.51808716,406.82710297)
\curveto(249.55808361,406.7971013)(249.61808355,406.76210134)(249.69808716,406.72210297)
\curveto(249.80808336,406.65210145)(249.91808325,406.60710149)(250.02808716,406.58710297)
\curveto(250.14808302,406.56710153)(250.29308287,406.54710155)(250.46308716,406.52710297)
\lineto(250.55308716,406.52710297)
\curveto(250.59308257,406.52710157)(250.62308254,406.53210157)(250.64308716,406.54210297)
\lineto(250.77808716,406.54210297)
\curveto(250.84808232,406.56210154)(250.91308225,406.57710152)(250.97308716,406.58710297)
\curveto(251.04308212,406.60710149)(251.10808206,406.62710147)(251.16808716,406.64710297)
\curveto(251.4680817,406.77710132)(251.69808147,406.96710113)(251.85808716,407.21710297)
\curveto(251.89808127,407.26710083)(251.93308123,407.32210078)(251.96308716,407.38210297)
\curveto(251.99308117,407.45210065)(252.01808115,407.51210059)(252.03808716,407.56210297)
\curveto(252.07808109,407.67210043)(252.11308105,407.76710033)(252.14308716,407.84710297)
\curveto(252.17308099,407.93710016)(252.24308092,408.00710009)(252.35308716,408.05710297)
\curveto(252.44308072,408.0971)(252.58808058,408.11209999)(252.78808716,408.10210297)
\lineto(253.28308716,408.10210297)
\lineto(253.49308716,408.10210297)
\curveto(253.57307959,408.11209999)(253.63807953,408.10709999)(253.68808716,408.08710297)
\lineto(253.80808716,408.08710297)
\lineto(253.92808716,408.05710297)
\curveto(253.9680792,408.05710004)(253.99807917,408.04710005)(254.01808716,408.02710297)
\curveto(254.0680791,407.98710011)(254.09807907,407.92710017)(254.10808716,407.84710297)
\curveto(254.12807904,407.77710032)(254.12807904,407.7021004)(254.10808716,407.62210297)
\curveto(254.01807915,407.29210081)(253.90807926,406.9971011)(253.77808716,406.73710297)
\curveto(253.3680798,405.96710213)(252.71308045,405.43210267)(251.81308716,405.13210297)
\curveto(251.71308145,405.102103)(251.60808156,405.08210302)(251.49808716,405.07210297)
\curveto(251.38808178,405.05210305)(251.27808189,405.02710307)(251.16808716,404.99710297)
\curveto(251.10808206,404.98710311)(251.04808212,404.98210312)(250.98808716,404.98210297)
\curveto(250.92808224,404.98210312)(250.8680823,404.97710312)(250.80808716,404.96710297)
\lineto(250.64308716,404.96710297)
\curveto(250.59308257,404.94710315)(250.51808265,404.94210316)(250.41808716,404.95210297)
\curveto(250.31808285,404.95210315)(250.24308292,404.95710314)(250.19308716,404.96710297)
\curveto(250.11308305,404.98710311)(250.03808313,404.9971031)(249.96808716,404.99710297)
\curveto(249.90808326,404.98710311)(249.84308332,404.99210311)(249.77308716,405.01210297)
\lineto(249.62308716,405.04210297)
\curveto(249.57308359,405.04210306)(249.52308364,405.04710305)(249.47308716,405.05710297)
\curveto(249.3630838,405.08710301)(249.25808391,405.11710298)(249.15808716,405.14710297)
\curveto(249.05808411,405.17710292)(248.9630842,405.21210289)(248.87308716,405.25210297)
\curveto(248.40308476,405.45210265)(248.00808516,405.70710239)(247.68808716,406.01710297)
\curveto(247.3680858,406.33710176)(247.10808606,406.73210137)(246.90808716,407.20210297)
\curveto(246.85808631,407.29210081)(246.81808635,407.38710071)(246.78808716,407.48710297)
\lineto(246.69808716,407.81710297)
\curveto(246.68808648,407.85710024)(246.68308648,407.89210021)(246.68308716,407.92210297)
\curveto(246.68308648,407.96210014)(246.67308649,408.00710009)(246.65308716,408.05710297)
\curveto(246.63308653,408.12709997)(246.62308654,408.1970999)(246.62308716,408.26710297)
\curveto(246.62308654,408.34709975)(246.61308655,408.42209968)(246.59308716,408.49210297)
\lineto(246.59308716,408.74710297)
\curveto(246.57308659,408.7970993)(246.5630866,408.85209925)(246.56308716,408.91210297)
\curveto(246.5630866,408.98209912)(246.57308659,409.04209906)(246.59308716,409.09210297)
\curveto(246.60308656,409.14209896)(246.60308656,409.18709891)(246.59308716,409.22710297)
\curveto(246.58308658,409.26709883)(246.58308658,409.30709879)(246.59308716,409.34710297)
\curveto(246.61308655,409.41709868)(246.61808655,409.48209862)(246.60808716,409.54210297)
\curveto(246.60808656,409.6020985)(246.61808655,409.66209844)(246.63808716,409.72210297)
\curveto(246.68808648,409.9020982)(246.72808644,410.07209803)(246.75808716,410.23210297)
\curveto(246.78808638,410.4020977)(246.83308633,410.56709753)(246.89308716,410.72710297)
\curveto(247.11308605,411.23709686)(247.38808578,411.66209644)(247.71808716,412.00210297)
\curveto(248.05808511,412.34209576)(248.48808468,412.61709548)(249.00808716,412.82710297)
\curveto(249.14808402,412.88709521)(249.29308387,412.92709517)(249.44308716,412.94710297)
\curveto(249.59308357,412.97709512)(249.74808342,413.01209509)(249.90808716,413.05210297)
\curveto(249.98808318,413.06209504)(250.0630831,413.06709503)(250.13308716,413.06710297)
\curveto(250.20308296,413.06709503)(250.27808289,413.07209503)(250.35808716,413.08210297)
}
}
{
\newrgbcolor{curcolor}{0 0 0}
\pscustom[linestyle=none,fillstyle=solid,fillcolor=curcolor]
{
\newpath
\moveto(257.50136841,415.72210297)
\curveto(257.57136546,415.64209246)(257.60636542,415.52209258)(257.60636841,415.36210297)
\lineto(257.60636841,414.89710297)
\lineto(257.60636841,414.49210297)
\curveto(257.60636542,414.35209375)(257.57136546,414.25709384)(257.50136841,414.20710297)
\curveto(257.44136559,414.15709394)(257.36136567,414.12709397)(257.26136841,414.11710297)
\curveto(257.17136586,414.10709399)(257.07136596,414.102094)(256.96136841,414.10210297)
\lineto(256.12136841,414.10210297)
\curveto(256.01136702,414.102094)(255.91136712,414.10709399)(255.82136841,414.11710297)
\curveto(255.74136729,414.12709397)(255.67136736,414.15709394)(255.61136841,414.20710297)
\curveto(255.57136746,414.23709386)(255.54136749,414.29209381)(255.52136841,414.37210297)
\curveto(255.51136752,414.46209364)(255.50136753,414.55709354)(255.49136841,414.65710297)
\lineto(255.49136841,414.98710297)
\curveto(255.50136753,415.097093)(255.50636752,415.19209291)(255.50636841,415.27210297)
\lineto(255.50636841,415.48210297)
\curveto(255.51636751,415.55209255)(255.53636749,415.61209249)(255.56636841,415.66210297)
\curveto(255.58636744,415.7020924)(255.61136742,415.73209237)(255.64136841,415.75210297)
\lineto(255.76136841,415.81210297)
\curveto(255.78136725,415.81209229)(255.80636722,415.81209229)(255.83636841,415.81210297)
\curveto(255.86636716,415.82209228)(255.89136714,415.82709227)(255.91136841,415.82710297)
\lineto(257.00636841,415.82710297)
\curveto(257.10636592,415.82709227)(257.20136583,415.82209228)(257.29136841,415.81210297)
\curveto(257.38136565,415.8020923)(257.45136558,415.77209233)(257.50136841,415.72210297)
\moveto(257.60636841,405.95710297)
\curveto(257.60636542,405.75710234)(257.60136543,405.58710251)(257.59136841,405.44710297)
\curveto(257.58136545,405.30710279)(257.49136554,405.21210289)(257.32136841,405.16210297)
\curveto(257.26136577,405.14210296)(257.19636583,405.13210297)(257.12636841,405.13210297)
\curveto(257.05636597,405.14210296)(256.98136605,405.14710295)(256.90136841,405.14710297)
\lineto(256.06136841,405.14710297)
\curveto(255.97136706,405.14710295)(255.88136715,405.15210295)(255.79136841,405.16210297)
\curveto(255.71136732,405.17210293)(255.65136738,405.2021029)(255.61136841,405.25210297)
\curveto(255.55136748,405.32210278)(255.51636751,405.40710269)(255.50636841,405.50710297)
\lineto(255.50636841,405.85210297)
\lineto(255.50636841,412.18210297)
\lineto(255.50636841,412.48210297)
\curveto(255.50636752,412.58209552)(255.5263675,412.66209544)(255.56636841,412.72210297)
\curveto(255.6263674,412.79209531)(255.71136732,412.83709526)(255.82136841,412.85710297)
\curveto(255.84136719,412.86709523)(255.86636716,412.86709523)(255.89636841,412.85710297)
\curveto(255.93636709,412.85709524)(255.96636706,412.86209524)(255.98636841,412.87210297)
\lineto(256.73636841,412.87210297)
\lineto(256.93136841,412.87210297)
\curveto(257.01136602,412.88209522)(257.07636595,412.88209522)(257.12636841,412.87210297)
\lineto(257.24636841,412.87210297)
\curveto(257.30636572,412.85209525)(257.36136567,412.83709526)(257.41136841,412.82710297)
\curveto(257.46136557,412.81709528)(257.50136553,412.78709531)(257.53136841,412.73710297)
\curveto(257.57136546,412.68709541)(257.59136544,412.61709548)(257.59136841,412.52710297)
\curveto(257.60136543,412.43709566)(257.60636542,412.34209576)(257.60636841,412.24210297)
\lineto(257.60636841,405.95710297)
}
}
{
\newrgbcolor{curcolor}{0 0 0}
\pscustom[linestyle=none,fillstyle=solid,fillcolor=curcolor]
{
\newpath
\moveto(267.03855591,409.31710297)
\curveto(267.01854738,409.36709873)(267.01354738,409.42209868)(267.02355591,409.48210297)
\curveto(267.03354736,409.54209856)(267.02854737,409.5970985)(267.00855591,409.64710297)
\curveto(266.9985474,409.68709841)(266.9935474,409.72709837)(266.99355591,409.76710297)
\curveto(266.9935474,409.80709829)(266.98854741,409.84709825)(266.97855591,409.88710297)
\lineto(266.91855591,410.15710297)
\curveto(266.8985475,410.24709785)(266.87354752,410.33209777)(266.84355591,410.41210297)
\curveto(266.7935476,410.55209755)(266.74854765,410.68209742)(266.70855591,410.80210297)
\curveto(266.66854773,410.93209717)(266.61354778,411.05209705)(266.54355591,411.16210297)
\curveto(266.47354792,411.27209683)(266.40354799,411.37709672)(266.33355591,411.47710297)
\curveto(266.27354812,411.57709652)(266.20354819,411.67709642)(266.12355591,411.77710297)
\curveto(266.04354835,411.88709621)(265.94354845,411.98709611)(265.82355591,412.07710297)
\curveto(265.71354868,412.17709592)(265.60354879,412.26709583)(265.49355591,412.34710297)
\curveto(265.16354923,412.57709552)(264.78354961,412.75709534)(264.35355591,412.88710297)
\curveto(263.93355046,413.01709508)(263.43355096,413.07709502)(262.85355591,413.06710297)
\curveto(262.78355161,413.05709504)(262.71355168,413.05209505)(262.64355591,413.05210297)
\curveto(262.57355182,413.05209505)(262.4985519,413.04709505)(262.41855591,413.03710297)
\curveto(262.26855213,412.9970951)(262.12355227,412.96709513)(261.98355591,412.94710297)
\curveto(261.84355255,412.92709517)(261.70855269,412.89209521)(261.57855591,412.84210297)
\curveto(261.46855293,412.79209531)(261.35855304,412.74709535)(261.24855591,412.70710297)
\curveto(261.13855326,412.66709543)(261.03355336,412.62209548)(260.93355591,412.57210297)
\curveto(260.57355382,412.34209576)(260.26855413,412.08709601)(260.01855591,411.80710297)
\curveto(259.76855463,411.53709656)(259.55355484,411.1970969)(259.37355591,410.78710297)
\curveto(259.32355507,410.66709743)(259.28355511,410.54209756)(259.25355591,410.41210297)
\curveto(259.22355517,410.29209781)(259.18855521,410.16709793)(259.14855591,410.03710297)
\curveto(259.12855527,409.98709811)(259.11855528,409.93709816)(259.11855591,409.88710297)
\curveto(259.11855528,409.84709825)(259.11355528,409.8020983)(259.10355591,409.75210297)
\curveto(259.08355531,409.7020984)(259.07355532,409.64709845)(259.07355591,409.58710297)
\curveto(259.08355531,409.53709856)(259.08355531,409.48709861)(259.07355591,409.43710297)
\lineto(259.07355591,409.33210297)
\curveto(259.05355534,409.27209883)(259.03855536,409.18709891)(259.02855591,409.07710297)
\curveto(259.02855537,408.96709913)(259.03855536,408.88209922)(259.05855591,408.82210297)
\lineto(259.05855591,408.68710297)
\curveto(259.05855534,408.64709945)(259.06355533,408.6020995)(259.07355591,408.55210297)
\curveto(259.0935553,408.47209963)(259.10355529,408.38709971)(259.10355591,408.29710297)
\curveto(259.10355529,408.21709988)(259.11355528,408.13709996)(259.13355591,408.05710297)
\curveto(259.15355524,408.00710009)(259.16355523,407.96210014)(259.16355591,407.92210297)
\curveto(259.16355523,407.88210022)(259.17355522,407.83710026)(259.19355591,407.78710297)
\curveto(259.22355517,407.67710042)(259.24855515,407.57210053)(259.26855591,407.47210297)
\curveto(259.2985551,407.37210073)(259.33855506,407.27710082)(259.38855591,407.18710297)
\curveto(259.55855484,406.7971013)(259.76855463,406.46210164)(260.01855591,406.18210297)
\curveto(260.26855413,405.9021022)(260.56855383,405.65710244)(260.91855591,405.44710297)
\curveto(261.02855337,405.38710271)(261.13355326,405.33710276)(261.23355591,405.29710297)
\curveto(261.34355305,405.25710284)(261.45855294,405.21710288)(261.57855591,405.17710297)
\curveto(261.66855273,405.13710296)(261.76355263,405.10710299)(261.86355591,405.08710297)
\curveto(261.96355243,405.06710303)(262.06355233,405.04210306)(262.16355591,405.01210297)
\curveto(262.21355218,405.0021031)(262.25355214,404.9971031)(262.28355591,404.99710297)
\curveto(262.32355207,404.9971031)(262.36355203,404.99210311)(262.40355591,404.98210297)
\curveto(262.45355194,404.96210314)(262.50355189,404.95710314)(262.55355591,404.96710297)
\curveto(262.61355178,404.96710313)(262.66855173,404.96210314)(262.71855591,404.95210297)
\lineto(262.86855591,404.95210297)
\curveto(262.92855147,404.93210317)(263.01355138,404.92710317)(263.12355591,404.93710297)
\curveto(263.23355116,404.93710316)(263.31355108,404.94210316)(263.36355591,404.95210297)
\curveto(263.393551,404.95210315)(263.42355097,404.95710314)(263.45355591,404.96710297)
\lineto(263.55855591,404.96710297)
\curveto(263.60855079,404.97710312)(263.66355073,404.98210312)(263.72355591,404.98210297)
\curveto(263.78355061,404.98210312)(263.83855056,404.99210311)(263.88855591,405.01210297)
\curveto(264.01855038,405.04210306)(264.14355025,405.07210303)(264.26355591,405.10210297)
\curveto(264.39355,405.12210298)(264.51854988,405.15710294)(264.63855591,405.20710297)
\curveto(265.11854928,405.40710269)(265.52854887,405.65710244)(265.86855591,405.95710297)
\curveto(266.20854819,406.25710184)(266.48354791,406.64710145)(266.69355591,407.12710297)
\curveto(266.74354765,407.22710087)(266.78354761,407.33210077)(266.81355591,407.44210297)
\curveto(266.84354755,407.56210054)(266.87854752,407.67710042)(266.91855591,407.78710297)
\curveto(266.92854747,407.85710024)(266.93854746,407.92210018)(266.94855591,407.98210297)
\curveto(266.95854744,408.04210006)(266.97354742,408.10709999)(266.99355591,408.17710297)
\curveto(267.01354738,408.25709984)(267.01854738,408.33709976)(267.00855591,408.41710297)
\curveto(267.00854739,408.4970996)(267.01854738,408.57709952)(267.03855591,408.65710297)
\lineto(267.03855591,408.80710297)
\curveto(267.05854734,408.86709923)(267.06854733,408.95209915)(267.06855591,409.06210297)
\curveto(267.06854733,409.17209893)(267.05854734,409.25709884)(267.03855591,409.31710297)
\moveto(264.93855591,408.77710297)
\curveto(264.92854947,408.72709937)(264.92354947,408.67709942)(264.92355591,408.62710297)
\lineto(264.92355591,408.49210297)
\curveto(264.91354948,408.45209965)(264.90854949,408.41209969)(264.90855591,408.37210297)
\curveto(264.90854949,408.34209976)(264.90354949,408.30709979)(264.89355591,408.26710297)
\curveto(264.86354953,408.15709994)(264.83854956,408.05210005)(264.81855591,407.95210297)
\curveto(264.7985496,407.85210025)(264.76854963,407.75210035)(264.72855591,407.65210297)
\curveto(264.61854978,407.4021007)(264.48354991,407.19210091)(264.32355591,407.02210297)
\curveto(264.16355023,406.85210125)(263.95355044,406.71710138)(263.69355591,406.61710297)
\curveto(263.62355077,406.58710151)(263.54855085,406.56710153)(263.46855591,406.55710297)
\curveto(263.38855101,406.54710155)(263.30855109,406.53210157)(263.22855591,406.51210297)
\lineto(263.10855591,406.51210297)
\curveto(263.06855133,406.5021016)(263.02355137,406.4971016)(262.97355591,406.49710297)
\lineto(262.85355591,406.52710297)
\curveto(262.81355158,406.53710156)(262.77855162,406.53710156)(262.74855591,406.52710297)
\curveto(262.71855168,406.52710157)(262.68355171,406.53210157)(262.64355591,406.54210297)
\curveto(262.55355184,406.56210154)(262.46355193,406.58710151)(262.37355591,406.61710297)
\curveto(262.2935521,406.64710145)(262.21855218,406.68710141)(262.14855591,406.73710297)
\curveto(261.8985525,406.88710121)(261.71355268,407.05210105)(261.59355591,407.23210297)
\curveto(261.48355291,407.42210068)(261.37855302,407.66710043)(261.27855591,407.96710297)
\curveto(261.25855314,408.04710005)(261.24355315,408.12209998)(261.23355591,408.19210297)
\curveto(261.22355317,408.27209983)(261.20855319,408.35209975)(261.18855591,408.43210297)
\lineto(261.18855591,408.56710297)
\curveto(261.16855323,408.63709946)(261.15355324,408.74209936)(261.14355591,408.88210297)
\curveto(261.14355325,409.02209908)(261.15355324,409.12709897)(261.17355591,409.19710297)
\lineto(261.17355591,409.34710297)
\curveto(261.17355322,409.3970987)(261.17855322,409.44709865)(261.18855591,409.49710297)
\curveto(261.20855319,409.60709849)(261.22355317,409.71709838)(261.23355591,409.82710297)
\curveto(261.25355314,409.93709816)(261.27855312,410.04209806)(261.30855591,410.14210297)
\curveto(261.398553,410.41209769)(261.51855288,410.64709745)(261.66855591,410.84710297)
\curveto(261.82855257,411.05709704)(262.03355236,411.21709688)(262.28355591,411.32710297)
\curveto(262.33355206,411.35709674)(262.38855201,411.37709672)(262.44855591,411.38710297)
\lineto(262.65855591,411.44710297)
\curveto(262.68855171,411.45709664)(262.72355167,411.45709664)(262.76355591,411.44710297)
\curveto(262.80355159,411.44709665)(262.83855156,411.45709664)(262.86855591,411.47710297)
\lineto(263.13855591,411.47710297)
\curveto(263.22855117,411.48709661)(263.31355108,411.48209662)(263.39355591,411.46210297)
\curveto(263.46355093,411.44209666)(263.52855087,411.42209668)(263.58855591,411.40210297)
\curveto(263.64855075,411.39209671)(263.70855069,411.37709672)(263.76855591,411.35710297)
\curveto(264.01855038,411.24709685)(264.21855018,411.097097)(264.36855591,410.90710297)
\curveto(264.51854988,410.72709737)(264.64854975,410.50709759)(264.75855591,410.24710297)
\curveto(264.78854961,410.16709793)(264.80854959,410.08209802)(264.81855591,409.99210297)
\lineto(264.87855591,409.75210297)
\curveto(264.88854951,409.73209837)(264.8935495,409.7020984)(264.89355591,409.66210297)
\curveto(264.90354949,409.61209849)(264.90854949,409.55709854)(264.90855591,409.49710297)
\curveto(264.90854949,409.43709866)(264.91854948,409.38209872)(264.93855591,409.33210297)
\lineto(264.93855591,409.21210297)
\curveto(264.94854945,409.16209894)(264.95354944,409.08709901)(264.95355591,408.98710297)
\curveto(264.95354944,408.8970992)(264.94854945,408.82709927)(264.93855591,408.77710297)
\moveto(263.70855591,415.94710297)
\lineto(264.77355591,415.94710297)
\curveto(264.85354954,415.94709215)(264.94854945,415.94709215)(265.05855591,415.94710297)
\curveto(265.16854923,415.94709215)(265.24854915,415.93209217)(265.29855591,415.90210297)
\curveto(265.31854908,415.89209221)(265.32854907,415.87709222)(265.32855591,415.85710297)
\curveto(265.33854906,415.84709225)(265.35354904,415.83709226)(265.37355591,415.82710297)
\curveto(265.38354901,415.70709239)(265.33354906,415.6020925)(265.22355591,415.51210297)
\curveto(265.12354927,415.42209268)(265.03854936,415.34209276)(264.96855591,415.27210297)
\curveto(264.88854951,415.2020929)(264.80854959,415.12709297)(264.72855591,415.04710297)
\curveto(264.65854974,414.97709312)(264.58354981,414.91209319)(264.50355591,414.85210297)
\curveto(264.46354993,414.82209328)(264.42854997,414.78709331)(264.39855591,414.74710297)
\curveto(264.37855002,414.71709338)(264.34855005,414.69209341)(264.30855591,414.67210297)
\curveto(264.28855011,414.64209346)(264.26355013,414.61709348)(264.23355591,414.59710297)
\lineto(264.08355591,414.44710297)
\lineto(263.93355591,414.32710297)
\lineto(263.88855591,414.28210297)
\curveto(263.88855051,414.27209383)(263.87855052,414.25709384)(263.85855591,414.23710297)
\curveto(263.77855062,414.17709392)(263.6985507,414.11209399)(263.61855591,414.04210297)
\curveto(263.54855085,413.97209413)(263.45855094,413.91709418)(263.34855591,413.87710297)
\curveto(263.30855109,413.86709423)(263.26855113,413.86209424)(263.22855591,413.86210297)
\curveto(263.1985512,413.86209424)(263.15855124,413.85709424)(263.10855591,413.84710297)
\curveto(263.07855132,413.83709426)(263.03855136,413.83209427)(262.98855591,413.83210297)
\curveto(262.93855146,413.84209426)(262.8935515,413.84709425)(262.85355591,413.84710297)
\lineto(262.50855591,413.84710297)
\curveto(262.38855201,413.84709425)(262.2985521,413.87209423)(262.23855591,413.92210297)
\curveto(262.17855222,413.96209414)(262.16355223,414.03209407)(262.19355591,414.13210297)
\curveto(262.21355218,414.21209389)(262.24855215,414.28209382)(262.29855591,414.34210297)
\curveto(262.34855205,414.41209369)(262.393552,414.48209362)(262.43355591,414.55210297)
\curveto(262.53355186,414.69209341)(262.62855177,414.82709327)(262.71855591,414.95710297)
\curveto(262.80855159,415.08709301)(262.8985515,415.22209288)(262.98855591,415.36210297)
\curveto(263.03855136,415.44209266)(263.08855131,415.52709257)(263.13855591,415.61710297)
\curveto(263.1985512,415.70709239)(263.26355113,415.77709232)(263.33355591,415.82710297)
\curveto(263.37355102,415.85709224)(263.44355095,415.89209221)(263.54355591,415.93210297)
\curveto(263.56355083,415.94209216)(263.58855081,415.94209216)(263.61855591,415.93210297)
\curveto(263.65855074,415.93209217)(263.68855071,415.93709216)(263.70855591,415.94710297)
}
}
{
\newrgbcolor{curcolor}{0 0 0}
\pscustom[linestyle=none,fillstyle=solid,fillcolor=curcolor]
{
\newpath
\moveto(272.86347778,413.06710297)
\curveto(273.46347198,413.08709501)(273.96347148,413.0020951)(274.36347778,412.81210297)
\curveto(274.76347068,412.62209548)(275.07847036,412.34209576)(275.30847778,411.97210297)
\curveto(275.37847006,411.86209624)(275.43347001,411.74209636)(275.47347778,411.61210297)
\curveto(275.51346993,411.49209661)(275.55346989,411.36709673)(275.59347778,411.23710297)
\curveto(275.61346983,411.15709694)(275.62346982,411.08209702)(275.62347778,411.01210297)
\curveto(275.63346981,410.94209716)(275.64846979,410.87209723)(275.66847778,410.80210297)
\curveto(275.66846977,410.74209736)(275.67346977,410.7020974)(275.68347778,410.68210297)
\curveto(275.70346974,410.54209756)(275.71346973,410.3970977)(275.71347778,410.24710297)
\lineto(275.71347778,409.81210297)
\lineto(275.71347778,408.47710297)
\lineto(275.71347778,406.04710297)
\curveto(275.71346973,405.85710224)(275.70846973,405.67210243)(275.69847778,405.49210297)
\curveto(275.69846974,405.32210278)(275.62846981,405.21210289)(275.48847778,405.16210297)
\curveto(275.42847001,405.14210296)(275.35847008,405.13210297)(275.27847778,405.13210297)
\lineto(275.03847778,405.13210297)
\lineto(274.22847778,405.13210297)
\curveto(274.10847133,405.13210297)(273.99847144,405.13710296)(273.89847778,405.14710297)
\curveto(273.80847163,405.16710293)(273.7384717,405.21210289)(273.68847778,405.28210297)
\curveto(273.64847179,405.34210276)(273.62347182,405.41710268)(273.61347778,405.50710297)
\lineto(273.61347778,405.82210297)
\lineto(273.61347778,406.87210297)
\lineto(273.61347778,409.10710297)
\curveto(273.61347183,409.47709862)(273.59847184,409.81709828)(273.56847778,410.12710297)
\curveto(273.5384719,410.44709765)(273.44847199,410.71709738)(273.29847778,410.93710297)
\curveto(273.15847228,411.13709696)(272.95347249,411.27709682)(272.68347778,411.35710297)
\curveto(272.63347281,411.37709672)(272.57847286,411.38709671)(272.51847778,411.38710297)
\curveto(272.46847297,411.38709671)(272.41347303,411.3970967)(272.35347778,411.41710297)
\curveto(272.30347314,411.42709667)(272.2384732,411.42709667)(272.15847778,411.41710297)
\curveto(272.08847335,411.41709668)(272.03347341,411.41209669)(271.99347778,411.40210297)
\curveto(271.95347349,411.39209671)(271.91847352,411.38709671)(271.88847778,411.38710297)
\curveto(271.85847358,411.38709671)(271.82847361,411.38209672)(271.79847778,411.37210297)
\curveto(271.56847387,411.31209679)(271.38347406,411.23209687)(271.24347778,411.13210297)
\curveto(270.92347452,410.9020972)(270.73347471,410.56709753)(270.67347778,410.12710297)
\curveto(270.61347483,409.68709841)(270.58347486,409.19209891)(270.58347778,408.64210297)
\lineto(270.58347778,406.76710297)
\lineto(270.58347778,405.85210297)
\lineto(270.58347778,405.58210297)
\curveto(270.58347486,405.49210261)(270.56847487,405.41710268)(270.53847778,405.35710297)
\curveto(270.48847495,405.24710285)(270.40847503,405.18210292)(270.29847778,405.16210297)
\curveto(270.18847525,405.14210296)(270.05347539,405.13210297)(269.89347778,405.13210297)
\lineto(269.14347778,405.13210297)
\curveto(269.03347641,405.13210297)(268.92347652,405.13710296)(268.81347778,405.14710297)
\curveto(268.70347674,405.15710294)(268.62347682,405.19210291)(268.57347778,405.25210297)
\curveto(268.50347694,405.34210276)(268.46847697,405.47210263)(268.46847778,405.64210297)
\curveto(268.47847696,405.81210229)(268.48347696,405.97210213)(268.48347778,406.12210297)
\lineto(268.48347778,408.16210297)
\lineto(268.48347778,411.46210297)
\lineto(268.48347778,412.22710297)
\lineto(268.48347778,412.52710297)
\curveto(268.49347695,412.61709548)(268.52347692,412.69209541)(268.57347778,412.75210297)
\curveto(268.59347685,412.78209532)(268.62347682,412.8020953)(268.66347778,412.81210297)
\curveto(268.71347673,412.83209527)(268.76347668,412.84709525)(268.81347778,412.85710297)
\lineto(268.88847778,412.85710297)
\curveto(268.9384765,412.86709523)(268.98847645,412.87209523)(269.03847778,412.87210297)
\lineto(269.20347778,412.87210297)
\lineto(269.83347778,412.87210297)
\curveto(269.91347553,412.87209523)(269.98847545,412.86709523)(270.05847778,412.85710297)
\curveto(270.1384753,412.85709524)(270.20847523,412.84709525)(270.26847778,412.82710297)
\curveto(270.3384751,412.7970953)(270.38347506,412.75209535)(270.40347778,412.69210297)
\curveto(270.43347501,412.63209547)(270.45847498,412.56209554)(270.47847778,412.48210297)
\curveto(270.48847495,412.44209566)(270.48847495,412.40709569)(270.47847778,412.37710297)
\curveto(270.47847496,412.34709575)(270.48847495,412.31709578)(270.50847778,412.28710297)
\curveto(270.52847491,412.23709586)(270.5434749,412.20709589)(270.55347778,412.19710297)
\curveto(270.57347487,412.18709591)(270.59847484,412.17209593)(270.62847778,412.15210297)
\curveto(270.7384747,412.14209596)(270.82847461,412.17709592)(270.89847778,412.25710297)
\curveto(270.96847447,412.34709575)(271.0434744,412.41709568)(271.12347778,412.46710297)
\curveto(271.39347405,412.66709543)(271.69347375,412.82709527)(272.02347778,412.94710297)
\curveto(272.11347333,412.97709512)(272.20347324,412.9970951)(272.29347778,413.00710297)
\curveto(272.39347305,413.01709508)(272.49847294,413.03209507)(272.60847778,413.05210297)
\curveto(272.6384728,413.06209504)(272.68347276,413.06209504)(272.74347778,413.05210297)
\curveto(272.80347264,413.05209505)(272.8434726,413.05709504)(272.86347778,413.06710297)
}
}
{
\newrgbcolor{curcolor}{0 0 0}
\pscustom[linestyle=none,fillstyle=solid,fillcolor=curcolor]
{
}
}
{
\newrgbcolor{curcolor}{0 0 0}
\pscustom[linestyle=none,fillstyle=solid,fillcolor=curcolor]
{
\newpath
\moveto(288.86988403,409.07710297)
\curveto(288.88987587,408.9970991)(288.88987587,408.90709919)(288.86988403,408.80710297)
\curveto(288.84987591,408.70709939)(288.81487594,408.64209946)(288.76488403,408.61210297)
\curveto(288.71487604,408.57209953)(288.63987612,408.54209956)(288.53988403,408.52210297)
\curveto(288.44987631,408.51209959)(288.34487641,408.5020996)(288.22488403,408.49210297)
\lineto(287.87988403,408.49210297)
\curveto(287.76987699,408.5020996)(287.66987709,408.50709959)(287.57988403,408.50710297)
\lineto(283.91988403,408.50710297)
\lineto(283.70988403,408.50710297)
\curveto(283.64988111,408.50709959)(283.59488116,408.4970996)(283.54488403,408.47710297)
\curveto(283.46488129,408.43709966)(283.41488134,408.3970997)(283.39488403,408.35710297)
\curveto(283.37488138,408.33709976)(283.3548814,408.2970998)(283.33488403,408.23710297)
\curveto(283.31488144,408.18709991)(283.30988145,408.13709996)(283.31988403,408.08710297)
\curveto(283.33988142,408.02710007)(283.34988141,407.96710013)(283.34988403,407.90710297)
\curveto(283.3598814,407.85710024)(283.37488138,407.8021003)(283.39488403,407.74210297)
\curveto(283.47488128,407.5021006)(283.56988119,407.3021008)(283.67988403,407.14210297)
\curveto(283.79988096,406.99210111)(283.9598808,406.85710124)(284.15988403,406.73710297)
\curveto(284.23988052,406.68710141)(284.31988044,406.65210145)(284.39988403,406.63210297)
\curveto(284.48988027,406.62210148)(284.57988018,406.6021015)(284.66988403,406.57210297)
\curveto(284.74988001,406.55210155)(284.8598799,406.53710156)(284.99988403,406.52710297)
\curveto(285.13987962,406.51710158)(285.2598795,406.52210158)(285.35988403,406.54210297)
\lineto(285.49488403,406.54210297)
\curveto(285.59487916,406.56210154)(285.68487907,406.58210152)(285.76488403,406.60210297)
\curveto(285.8548789,406.63210147)(285.93987882,406.66210144)(286.01988403,406.69210297)
\curveto(286.11987864,406.74210136)(286.22987853,406.80710129)(286.34988403,406.88710297)
\curveto(286.47987828,406.96710113)(286.57487818,407.04710105)(286.63488403,407.12710297)
\curveto(286.68487807,407.1971009)(286.73487802,407.26210084)(286.78488403,407.32210297)
\curveto(286.84487791,407.39210071)(286.91487784,407.44210066)(286.99488403,407.47210297)
\curveto(287.09487766,407.52210058)(287.21987754,407.54210056)(287.36988403,407.53210297)
\lineto(287.80488403,407.53210297)
\lineto(287.98488403,407.53210297)
\curveto(288.0548767,407.54210056)(288.11487664,407.53710056)(288.16488403,407.51710297)
\lineto(288.31488403,407.51710297)
\curveto(288.41487634,407.4971006)(288.48487627,407.47210063)(288.52488403,407.44210297)
\curveto(288.56487619,407.42210068)(288.58487617,407.37710072)(288.58488403,407.30710297)
\curveto(288.59487616,407.23710086)(288.58987617,407.17710092)(288.56988403,407.12710297)
\curveto(288.51987624,406.98710111)(288.46487629,406.86210124)(288.40488403,406.75210297)
\curveto(288.34487641,406.64210146)(288.27487648,406.53210157)(288.19488403,406.42210297)
\curveto(287.97487678,406.09210201)(287.72487703,405.82710227)(287.44488403,405.62710297)
\curveto(287.16487759,405.42710267)(286.81487794,405.25710284)(286.39488403,405.11710297)
\curveto(286.28487847,405.07710302)(286.17487858,405.05210305)(286.06488403,405.04210297)
\curveto(285.9548788,405.03210307)(285.83987892,405.01210309)(285.71988403,404.98210297)
\curveto(285.67987908,404.97210313)(285.63487912,404.97210313)(285.58488403,404.98210297)
\curveto(285.54487921,404.98210312)(285.50487925,404.97710312)(285.46488403,404.96710297)
\lineto(285.29988403,404.96710297)
\curveto(285.24987951,404.94710315)(285.18987957,404.94210316)(285.11988403,404.95210297)
\curveto(285.0598797,404.95210315)(285.00487975,404.95710314)(284.95488403,404.96710297)
\curveto(284.87487988,404.97710312)(284.80487995,404.97710312)(284.74488403,404.96710297)
\curveto(284.68488007,404.95710314)(284.61988014,404.96210314)(284.54988403,404.98210297)
\curveto(284.49988026,405.0021031)(284.44488031,405.01210309)(284.38488403,405.01210297)
\curveto(284.32488043,405.01210309)(284.26988049,405.02210308)(284.21988403,405.04210297)
\curveto(284.10988065,405.06210304)(283.99988076,405.08710301)(283.88988403,405.11710297)
\curveto(283.77988098,405.13710296)(283.67988108,405.17210293)(283.58988403,405.22210297)
\curveto(283.47988128,405.26210284)(283.37488138,405.2971028)(283.27488403,405.32710297)
\curveto(283.18488157,405.36710273)(283.09988166,405.41210269)(283.01988403,405.46210297)
\curveto(282.69988206,405.66210244)(282.41488234,405.89210221)(282.16488403,406.15210297)
\curveto(281.91488284,406.42210168)(281.70988305,406.73210137)(281.54988403,407.08210297)
\curveto(281.49988326,407.19210091)(281.4598833,407.3021008)(281.42988403,407.41210297)
\curveto(281.39988336,407.53210057)(281.3598834,407.65210045)(281.30988403,407.77210297)
\curveto(281.29988346,407.81210029)(281.29488346,407.84710025)(281.29488403,407.87710297)
\curveto(281.29488346,407.91710018)(281.28988347,407.95710014)(281.27988403,407.99710297)
\curveto(281.23988352,408.11709998)(281.21488354,408.24709985)(281.20488403,408.38710297)
\lineto(281.17488403,408.80710297)
\curveto(281.17488358,408.85709924)(281.16988359,408.91209919)(281.15988403,408.97210297)
\curveto(281.1598836,409.03209907)(281.16488359,409.08709901)(281.17488403,409.13710297)
\lineto(281.17488403,409.31710297)
\lineto(281.21988403,409.67710297)
\curveto(281.2598835,409.84709825)(281.29488346,410.01209809)(281.32488403,410.17210297)
\curveto(281.3548834,410.33209777)(281.39988336,410.48209762)(281.45988403,410.62210297)
\curveto(281.88988287,411.66209644)(282.61988214,412.3970957)(283.64988403,412.82710297)
\curveto(283.78988097,412.88709521)(283.92988083,412.92709517)(284.06988403,412.94710297)
\curveto(284.21988054,412.97709512)(284.37488038,413.01209509)(284.53488403,413.05210297)
\curveto(284.61488014,413.06209504)(284.68988007,413.06709503)(284.75988403,413.06710297)
\curveto(284.82987993,413.06709503)(284.90487985,413.07209503)(284.98488403,413.08210297)
\curveto(285.49487926,413.09209501)(285.92987883,413.03209507)(286.28988403,412.90210297)
\curveto(286.6598781,412.78209532)(286.98987777,412.62209548)(287.27988403,412.42210297)
\curveto(287.36987739,412.36209574)(287.4598773,412.29209581)(287.54988403,412.21210297)
\curveto(287.63987712,412.14209596)(287.71987704,412.06709603)(287.78988403,411.98710297)
\curveto(287.81987694,411.93709616)(287.8598769,411.8970962)(287.90988403,411.86710297)
\curveto(287.98987677,411.75709634)(288.06487669,411.64209646)(288.13488403,411.52210297)
\curveto(288.20487655,411.41209669)(288.27987648,411.2970968)(288.35988403,411.17710297)
\curveto(288.40987635,411.08709701)(288.44987631,410.99209711)(288.47988403,410.89210297)
\curveto(288.51987624,410.8020973)(288.5598762,410.7020974)(288.59988403,410.59210297)
\curveto(288.64987611,410.46209764)(288.68987607,410.32709777)(288.71988403,410.18710297)
\curveto(288.74987601,410.04709805)(288.78487597,409.90709819)(288.82488403,409.76710297)
\curveto(288.84487591,409.68709841)(288.84987591,409.5970985)(288.83988403,409.49710297)
\curveto(288.83987592,409.40709869)(288.84987591,409.32209878)(288.86988403,409.24210297)
\lineto(288.86988403,409.07710297)
\moveto(286.61988403,409.96210297)
\curveto(286.68987807,410.06209804)(286.69487806,410.18209792)(286.63488403,410.32210297)
\curveto(286.58487817,410.47209763)(286.54487821,410.58209752)(286.51488403,410.65210297)
\curveto(286.37487838,410.92209718)(286.18987857,411.12709697)(285.95988403,411.26710297)
\curveto(285.72987903,411.41709668)(285.40987935,411.4970966)(284.99988403,411.50710297)
\curveto(284.96987979,411.48709661)(284.93487982,411.48209662)(284.89488403,411.49210297)
\curveto(284.8548799,411.5020966)(284.81987994,411.5020966)(284.78988403,411.49210297)
\curveto(284.73988002,411.47209663)(284.68488007,411.45709664)(284.62488403,411.44710297)
\curveto(284.56488019,411.44709665)(284.50988025,411.43709666)(284.45988403,411.41710297)
\curveto(284.01988074,411.27709682)(283.69488106,411.0020971)(283.48488403,410.59210297)
\curveto(283.46488129,410.55209755)(283.43988132,410.4970976)(283.40988403,410.42710297)
\curveto(283.38988137,410.36709773)(283.37488138,410.3020978)(283.36488403,410.23210297)
\curveto(283.3548814,410.17209793)(283.3548814,410.11209799)(283.36488403,410.05210297)
\curveto(283.38488137,409.99209811)(283.41988134,409.94209816)(283.46988403,409.90210297)
\curveto(283.54988121,409.85209825)(283.6598811,409.82709827)(283.79988403,409.82710297)
\lineto(284.20488403,409.82710297)
\lineto(285.86988403,409.82710297)
\lineto(286.30488403,409.82710297)
\curveto(286.46487829,409.83709826)(286.56987819,409.88209822)(286.61988403,409.96210297)
}
}
{
\newrgbcolor{curcolor}{0 0 0}
\pscustom[linestyle=none,fillstyle=solid,fillcolor=curcolor]
{
\newpath
\moveto(294.54316528,413.06710297)
\curveto(295.14315948,413.08709501)(295.64315898,413.0020951)(296.04316528,412.81210297)
\curveto(296.44315818,412.62209548)(296.75815786,412.34209576)(296.98816528,411.97210297)
\curveto(297.05815756,411.86209624)(297.11315751,411.74209636)(297.15316528,411.61210297)
\curveto(297.19315743,411.49209661)(297.23315739,411.36709673)(297.27316528,411.23710297)
\curveto(297.29315733,411.15709694)(297.30315732,411.08209702)(297.30316528,411.01210297)
\curveto(297.31315731,410.94209716)(297.32815729,410.87209723)(297.34816528,410.80210297)
\curveto(297.34815727,410.74209736)(297.35315727,410.7020974)(297.36316528,410.68210297)
\curveto(297.38315724,410.54209756)(297.39315723,410.3970977)(297.39316528,410.24710297)
\lineto(297.39316528,409.81210297)
\lineto(297.39316528,408.47710297)
\lineto(297.39316528,406.04710297)
\curveto(297.39315723,405.85710224)(297.38815723,405.67210243)(297.37816528,405.49210297)
\curveto(297.37815724,405.32210278)(297.30815731,405.21210289)(297.16816528,405.16210297)
\curveto(297.10815751,405.14210296)(297.03815758,405.13210297)(296.95816528,405.13210297)
\lineto(296.71816528,405.13210297)
\lineto(295.90816528,405.13210297)
\curveto(295.78815883,405.13210297)(295.67815894,405.13710296)(295.57816528,405.14710297)
\curveto(295.48815913,405.16710293)(295.4181592,405.21210289)(295.36816528,405.28210297)
\curveto(295.32815929,405.34210276)(295.30315932,405.41710268)(295.29316528,405.50710297)
\lineto(295.29316528,405.82210297)
\lineto(295.29316528,406.87210297)
\lineto(295.29316528,409.10710297)
\curveto(295.29315933,409.47709862)(295.27815934,409.81709828)(295.24816528,410.12710297)
\curveto(295.2181594,410.44709765)(295.12815949,410.71709738)(294.97816528,410.93710297)
\curveto(294.83815978,411.13709696)(294.63315999,411.27709682)(294.36316528,411.35710297)
\curveto(294.31316031,411.37709672)(294.25816036,411.38709671)(294.19816528,411.38710297)
\curveto(294.14816047,411.38709671)(294.09316053,411.3970967)(294.03316528,411.41710297)
\curveto(293.98316064,411.42709667)(293.9181607,411.42709667)(293.83816528,411.41710297)
\curveto(293.76816085,411.41709668)(293.71316091,411.41209669)(293.67316528,411.40210297)
\curveto(293.63316099,411.39209671)(293.59816102,411.38709671)(293.56816528,411.38710297)
\curveto(293.53816108,411.38709671)(293.50816111,411.38209672)(293.47816528,411.37210297)
\curveto(293.24816137,411.31209679)(293.06316156,411.23209687)(292.92316528,411.13210297)
\curveto(292.60316202,410.9020972)(292.41316221,410.56709753)(292.35316528,410.12710297)
\curveto(292.29316233,409.68709841)(292.26316236,409.19209891)(292.26316528,408.64210297)
\lineto(292.26316528,406.76710297)
\lineto(292.26316528,405.85210297)
\lineto(292.26316528,405.58210297)
\curveto(292.26316236,405.49210261)(292.24816237,405.41710268)(292.21816528,405.35710297)
\curveto(292.16816245,405.24710285)(292.08816253,405.18210292)(291.97816528,405.16210297)
\curveto(291.86816275,405.14210296)(291.73316289,405.13210297)(291.57316528,405.13210297)
\lineto(290.82316528,405.13210297)
\curveto(290.71316391,405.13210297)(290.60316402,405.13710296)(290.49316528,405.14710297)
\curveto(290.38316424,405.15710294)(290.30316432,405.19210291)(290.25316528,405.25210297)
\curveto(290.18316444,405.34210276)(290.14816447,405.47210263)(290.14816528,405.64210297)
\curveto(290.15816446,405.81210229)(290.16316446,405.97210213)(290.16316528,406.12210297)
\lineto(290.16316528,408.16210297)
\lineto(290.16316528,411.46210297)
\lineto(290.16316528,412.22710297)
\lineto(290.16316528,412.52710297)
\curveto(290.17316445,412.61709548)(290.20316442,412.69209541)(290.25316528,412.75210297)
\curveto(290.27316435,412.78209532)(290.30316432,412.8020953)(290.34316528,412.81210297)
\curveto(290.39316423,412.83209527)(290.44316418,412.84709525)(290.49316528,412.85710297)
\lineto(290.56816528,412.85710297)
\curveto(290.618164,412.86709523)(290.66816395,412.87209523)(290.71816528,412.87210297)
\lineto(290.88316528,412.87210297)
\lineto(291.51316528,412.87210297)
\curveto(291.59316303,412.87209523)(291.66816295,412.86709523)(291.73816528,412.85710297)
\curveto(291.8181628,412.85709524)(291.88816273,412.84709525)(291.94816528,412.82710297)
\curveto(292.0181626,412.7970953)(292.06316256,412.75209535)(292.08316528,412.69210297)
\curveto(292.11316251,412.63209547)(292.13816248,412.56209554)(292.15816528,412.48210297)
\curveto(292.16816245,412.44209566)(292.16816245,412.40709569)(292.15816528,412.37710297)
\curveto(292.15816246,412.34709575)(292.16816245,412.31709578)(292.18816528,412.28710297)
\curveto(292.20816241,412.23709586)(292.2231624,412.20709589)(292.23316528,412.19710297)
\curveto(292.25316237,412.18709591)(292.27816234,412.17209593)(292.30816528,412.15210297)
\curveto(292.4181622,412.14209596)(292.50816211,412.17709592)(292.57816528,412.25710297)
\curveto(292.64816197,412.34709575)(292.7231619,412.41709568)(292.80316528,412.46710297)
\curveto(293.07316155,412.66709543)(293.37316125,412.82709527)(293.70316528,412.94710297)
\curveto(293.79316083,412.97709512)(293.88316074,412.9970951)(293.97316528,413.00710297)
\curveto(294.07316055,413.01709508)(294.17816044,413.03209507)(294.28816528,413.05210297)
\curveto(294.3181603,413.06209504)(294.36316026,413.06209504)(294.42316528,413.05210297)
\curveto(294.48316014,413.05209505)(294.5231601,413.05709504)(294.54316528,413.06710297)
}
}
{
\newrgbcolor{curcolor}{0 0 0}
\pscustom[linestyle=none,fillstyle=solid,fillcolor=curcolor]
{
}
}
{
\newrgbcolor{curcolor}{0 0 0}
\pscustom[linestyle=none,fillstyle=solid,fillcolor=curcolor]
{
\newpath
\moveto(310.54957153,409.07710297)
\curveto(310.56956337,408.9970991)(310.56956337,408.90709919)(310.54957153,408.80710297)
\curveto(310.52956341,408.70709939)(310.49456344,408.64209946)(310.44457153,408.61210297)
\curveto(310.39456354,408.57209953)(310.31956362,408.54209956)(310.21957153,408.52210297)
\curveto(310.12956381,408.51209959)(310.02456391,408.5020996)(309.90457153,408.49210297)
\lineto(309.55957153,408.49210297)
\curveto(309.44956449,408.5020996)(309.34956459,408.50709959)(309.25957153,408.50710297)
\lineto(305.59957153,408.50710297)
\lineto(305.38957153,408.50710297)
\curveto(305.32956861,408.50709959)(305.27456866,408.4970996)(305.22457153,408.47710297)
\curveto(305.14456879,408.43709966)(305.09456884,408.3970997)(305.07457153,408.35710297)
\curveto(305.05456888,408.33709976)(305.0345689,408.2970998)(305.01457153,408.23710297)
\curveto(304.99456894,408.18709991)(304.98956895,408.13709996)(304.99957153,408.08710297)
\curveto(305.01956892,408.02710007)(305.02956891,407.96710013)(305.02957153,407.90710297)
\curveto(305.0395689,407.85710024)(305.05456888,407.8021003)(305.07457153,407.74210297)
\curveto(305.15456878,407.5021006)(305.24956869,407.3021008)(305.35957153,407.14210297)
\curveto(305.47956846,406.99210111)(305.6395683,406.85710124)(305.83957153,406.73710297)
\curveto(305.91956802,406.68710141)(305.99956794,406.65210145)(306.07957153,406.63210297)
\curveto(306.16956777,406.62210148)(306.25956768,406.6021015)(306.34957153,406.57210297)
\curveto(306.42956751,406.55210155)(306.5395674,406.53710156)(306.67957153,406.52710297)
\curveto(306.81956712,406.51710158)(306.939567,406.52210158)(307.03957153,406.54210297)
\lineto(307.17457153,406.54210297)
\curveto(307.27456666,406.56210154)(307.36456657,406.58210152)(307.44457153,406.60210297)
\curveto(307.5345664,406.63210147)(307.61956632,406.66210144)(307.69957153,406.69210297)
\curveto(307.79956614,406.74210136)(307.90956603,406.80710129)(308.02957153,406.88710297)
\curveto(308.15956578,406.96710113)(308.25456568,407.04710105)(308.31457153,407.12710297)
\curveto(308.36456557,407.1971009)(308.41456552,407.26210084)(308.46457153,407.32210297)
\curveto(308.52456541,407.39210071)(308.59456534,407.44210066)(308.67457153,407.47210297)
\curveto(308.77456516,407.52210058)(308.89956504,407.54210056)(309.04957153,407.53210297)
\lineto(309.48457153,407.53210297)
\lineto(309.66457153,407.53210297)
\curveto(309.7345642,407.54210056)(309.79456414,407.53710056)(309.84457153,407.51710297)
\lineto(309.99457153,407.51710297)
\curveto(310.09456384,407.4971006)(310.16456377,407.47210063)(310.20457153,407.44210297)
\curveto(310.24456369,407.42210068)(310.26456367,407.37710072)(310.26457153,407.30710297)
\curveto(310.27456366,407.23710086)(310.26956367,407.17710092)(310.24957153,407.12710297)
\curveto(310.19956374,406.98710111)(310.14456379,406.86210124)(310.08457153,406.75210297)
\curveto(310.02456391,406.64210146)(309.95456398,406.53210157)(309.87457153,406.42210297)
\curveto(309.65456428,406.09210201)(309.40456453,405.82710227)(309.12457153,405.62710297)
\curveto(308.84456509,405.42710267)(308.49456544,405.25710284)(308.07457153,405.11710297)
\curveto(307.96456597,405.07710302)(307.85456608,405.05210305)(307.74457153,405.04210297)
\curveto(307.6345663,405.03210307)(307.51956642,405.01210309)(307.39957153,404.98210297)
\curveto(307.35956658,404.97210313)(307.31456662,404.97210313)(307.26457153,404.98210297)
\curveto(307.22456671,404.98210312)(307.18456675,404.97710312)(307.14457153,404.96710297)
\lineto(306.97957153,404.96710297)
\curveto(306.92956701,404.94710315)(306.86956707,404.94210316)(306.79957153,404.95210297)
\curveto(306.7395672,404.95210315)(306.68456725,404.95710314)(306.63457153,404.96710297)
\curveto(306.55456738,404.97710312)(306.48456745,404.97710312)(306.42457153,404.96710297)
\curveto(306.36456757,404.95710314)(306.29956764,404.96210314)(306.22957153,404.98210297)
\curveto(306.17956776,405.0021031)(306.12456781,405.01210309)(306.06457153,405.01210297)
\curveto(306.00456793,405.01210309)(305.94956799,405.02210308)(305.89957153,405.04210297)
\curveto(305.78956815,405.06210304)(305.67956826,405.08710301)(305.56957153,405.11710297)
\curveto(305.45956848,405.13710296)(305.35956858,405.17210293)(305.26957153,405.22210297)
\curveto(305.15956878,405.26210284)(305.05456888,405.2971028)(304.95457153,405.32710297)
\curveto(304.86456907,405.36710273)(304.77956916,405.41210269)(304.69957153,405.46210297)
\curveto(304.37956956,405.66210244)(304.09456984,405.89210221)(303.84457153,406.15210297)
\curveto(303.59457034,406.42210168)(303.38957055,406.73210137)(303.22957153,407.08210297)
\curveto(303.17957076,407.19210091)(303.1395708,407.3021008)(303.10957153,407.41210297)
\curveto(303.07957086,407.53210057)(303.0395709,407.65210045)(302.98957153,407.77210297)
\curveto(302.97957096,407.81210029)(302.97457096,407.84710025)(302.97457153,407.87710297)
\curveto(302.97457096,407.91710018)(302.96957097,407.95710014)(302.95957153,407.99710297)
\curveto(302.91957102,408.11709998)(302.89457104,408.24709985)(302.88457153,408.38710297)
\lineto(302.85457153,408.80710297)
\curveto(302.85457108,408.85709924)(302.84957109,408.91209919)(302.83957153,408.97210297)
\curveto(302.8395711,409.03209907)(302.84457109,409.08709901)(302.85457153,409.13710297)
\lineto(302.85457153,409.31710297)
\lineto(302.89957153,409.67710297)
\curveto(302.939571,409.84709825)(302.97457096,410.01209809)(303.00457153,410.17210297)
\curveto(303.0345709,410.33209777)(303.07957086,410.48209762)(303.13957153,410.62210297)
\curveto(303.56957037,411.66209644)(304.29956964,412.3970957)(305.32957153,412.82710297)
\curveto(305.46956847,412.88709521)(305.60956833,412.92709517)(305.74957153,412.94710297)
\curveto(305.89956804,412.97709512)(306.05456788,413.01209509)(306.21457153,413.05210297)
\curveto(306.29456764,413.06209504)(306.36956757,413.06709503)(306.43957153,413.06710297)
\curveto(306.50956743,413.06709503)(306.58456735,413.07209503)(306.66457153,413.08210297)
\curveto(307.17456676,413.09209501)(307.60956633,413.03209507)(307.96957153,412.90210297)
\curveto(308.3395656,412.78209532)(308.66956527,412.62209548)(308.95957153,412.42210297)
\curveto(309.04956489,412.36209574)(309.1395648,412.29209581)(309.22957153,412.21210297)
\curveto(309.31956462,412.14209596)(309.39956454,412.06709603)(309.46957153,411.98710297)
\curveto(309.49956444,411.93709616)(309.5395644,411.8970962)(309.58957153,411.86710297)
\curveto(309.66956427,411.75709634)(309.74456419,411.64209646)(309.81457153,411.52210297)
\curveto(309.88456405,411.41209669)(309.95956398,411.2970968)(310.03957153,411.17710297)
\curveto(310.08956385,411.08709701)(310.12956381,410.99209711)(310.15957153,410.89210297)
\curveto(310.19956374,410.8020973)(310.2395637,410.7020974)(310.27957153,410.59210297)
\curveto(310.32956361,410.46209764)(310.36956357,410.32709777)(310.39957153,410.18710297)
\curveto(310.42956351,410.04709805)(310.46456347,409.90709819)(310.50457153,409.76710297)
\curveto(310.52456341,409.68709841)(310.52956341,409.5970985)(310.51957153,409.49710297)
\curveto(310.51956342,409.40709869)(310.52956341,409.32209878)(310.54957153,409.24210297)
\lineto(310.54957153,409.07710297)
\moveto(308.29957153,409.96210297)
\curveto(308.36956557,410.06209804)(308.37456556,410.18209792)(308.31457153,410.32210297)
\curveto(308.26456567,410.47209763)(308.22456571,410.58209752)(308.19457153,410.65210297)
\curveto(308.05456588,410.92209718)(307.86956607,411.12709697)(307.63957153,411.26710297)
\curveto(307.40956653,411.41709668)(307.08956685,411.4970966)(306.67957153,411.50710297)
\curveto(306.64956729,411.48709661)(306.61456732,411.48209662)(306.57457153,411.49210297)
\curveto(306.5345674,411.5020966)(306.49956744,411.5020966)(306.46957153,411.49210297)
\curveto(306.41956752,411.47209663)(306.36456757,411.45709664)(306.30457153,411.44710297)
\curveto(306.24456769,411.44709665)(306.18956775,411.43709666)(306.13957153,411.41710297)
\curveto(305.69956824,411.27709682)(305.37456856,411.0020971)(305.16457153,410.59210297)
\curveto(305.14456879,410.55209755)(305.11956882,410.4970976)(305.08957153,410.42710297)
\curveto(305.06956887,410.36709773)(305.05456888,410.3020978)(305.04457153,410.23210297)
\curveto(305.0345689,410.17209793)(305.0345689,410.11209799)(305.04457153,410.05210297)
\curveto(305.06456887,409.99209811)(305.09956884,409.94209816)(305.14957153,409.90210297)
\curveto(305.22956871,409.85209825)(305.3395686,409.82709827)(305.47957153,409.82710297)
\lineto(305.88457153,409.82710297)
\lineto(307.54957153,409.82710297)
\lineto(307.98457153,409.82710297)
\curveto(308.14456579,409.83709826)(308.24956569,409.88209822)(308.29957153,409.96210297)
}
}
{
\newrgbcolor{curcolor}{0 0 0}
\pscustom[linestyle=none,fillstyle=solid,fillcolor=curcolor]
{
\newpath
\moveto(312.30785278,415.82710297)
\lineto(313.40285278,415.82710297)
\curveto(313.5028503,415.82709227)(313.5978502,415.82209228)(313.68785278,415.81210297)
\curveto(313.77785002,415.8020923)(313.84784995,415.77209233)(313.89785278,415.72210297)
\curveto(313.95784984,415.65209245)(313.98784981,415.55709254)(313.98785278,415.43710297)
\curveto(313.9978498,415.32709277)(314.0028498,415.21209289)(314.00285278,415.09210297)
\lineto(314.00285278,413.75710297)
\lineto(314.00285278,408.37210297)
\lineto(314.00285278,406.07710297)
\lineto(314.00285278,405.65710297)
\curveto(314.01284979,405.50710259)(313.99284981,405.39210271)(313.94285278,405.31210297)
\curveto(313.89284991,405.23210287)(313.80285,405.17710292)(313.67285278,405.14710297)
\curveto(313.61285019,405.12710297)(313.54285026,405.12210298)(313.46285278,405.13210297)
\curveto(313.39285041,405.14210296)(313.32285048,405.14710295)(313.25285278,405.14710297)
\lineto(312.53285278,405.14710297)
\curveto(312.42285138,405.14710295)(312.32285148,405.15210295)(312.23285278,405.16210297)
\curveto(312.14285166,405.17210293)(312.06785173,405.2021029)(312.00785278,405.25210297)
\curveto(311.94785185,405.3021028)(311.91285189,405.37710272)(311.90285278,405.47710297)
\lineto(311.90285278,405.80710297)
\lineto(311.90285278,407.14210297)
\lineto(311.90285278,412.76710297)
\lineto(311.90285278,414.80710297)
\curveto(311.9028519,414.93709316)(311.8978519,415.09209301)(311.88785278,415.27210297)
\curveto(311.88785191,415.45209265)(311.91285189,415.58209252)(311.96285278,415.66210297)
\curveto(311.98285182,415.7020924)(312.00785179,415.73209237)(312.03785278,415.75210297)
\lineto(312.15785278,415.81210297)
\curveto(312.17785162,415.81209229)(312.2028516,415.81209229)(312.23285278,415.81210297)
\curveto(312.26285154,415.82209228)(312.28785151,415.82709227)(312.30785278,415.82710297)
}
}
{
\newrgbcolor{curcolor}{0 0 0}
\pscustom[linestyle=none,fillstyle=solid,fillcolor=curcolor]
{
}
}
{
\newrgbcolor{curcolor}{0 0 0}
\pscustom[linestyle=none,fillstyle=solid,fillcolor=curcolor]
{
\newpath
\moveto(322.79519653,413.08210297)
\curveto(323.54519203,413.102095)(324.19519138,413.01709508)(324.74519653,412.82710297)
\curveto(325.30519027,412.64709545)(325.73018985,412.33209577)(326.02019653,411.88210297)
\curveto(326.09018949,411.77209633)(326.15018943,411.65709644)(326.20019653,411.53710297)
\curveto(326.26018932,411.42709667)(326.31018927,411.3020968)(326.35019653,411.16210297)
\curveto(326.37018921,411.102097)(326.3801892,411.03709706)(326.38019653,410.96710297)
\curveto(326.3801892,410.8970972)(326.37018921,410.83709726)(326.35019653,410.78710297)
\curveto(326.31018927,410.72709737)(326.25518932,410.68709741)(326.18519653,410.66710297)
\curveto(326.13518944,410.64709745)(326.0751895,410.63709746)(326.00519653,410.63710297)
\lineto(325.79519653,410.63710297)
\lineto(325.13519653,410.63710297)
\curveto(325.06519051,410.63709746)(324.99519058,410.63209747)(324.92519653,410.62210297)
\curveto(324.85519072,410.62209748)(324.79019079,410.63209747)(324.73019653,410.65210297)
\curveto(324.63019095,410.67209743)(324.55519102,410.71209739)(324.50519653,410.77210297)
\curveto(324.45519112,410.83209727)(324.41019117,410.89209721)(324.37019653,410.95210297)
\lineto(324.25019653,411.16210297)
\curveto(324.22019136,411.24209686)(324.17019141,411.30709679)(324.10019653,411.35710297)
\curveto(324.00019158,411.43709666)(323.90019168,411.4970966)(323.80019653,411.53710297)
\curveto(323.71019187,411.57709652)(323.59519198,411.61209649)(323.45519653,411.64210297)
\curveto(323.38519219,411.66209644)(323.2801923,411.67709642)(323.14019653,411.68710297)
\curveto(323.01019257,411.6970964)(322.91019267,411.69209641)(322.84019653,411.67210297)
\lineto(322.73519653,411.67210297)
\lineto(322.58519653,411.64210297)
\curveto(322.54519303,411.64209646)(322.50019308,411.63709646)(322.45019653,411.62710297)
\curveto(322.2801933,411.57709652)(322.14019344,411.50709659)(322.03019653,411.41710297)
\curveto(321.93019365,411.33709676)(321.86019372,411.21209689)(321.82019653,411.04210297)
\curveto(321.80019378,410.97209713)(321.80019378,410.90709719)(321.82019653,410.84710297)
\curveto(321.84019374,410.78709731)(321.86019372,410.73709736)(321.88019653,410.69710297)
\curveto(321.95019363,410.57709752)(322.03019355,410.48209762)(322.12019653,410.41210297)
\curveto(322.22019336,410.34209776)(322.33519324,410.28209782)(322.46519653,410.23210297)
\curveto(322.65519292,410.15209795)(322.86019272,410.08209802)(323.08019653,410.02210297)
\lineto(323.77019653,409.87210297)
\curveto(324.01019157,409.83209827)(324.24019134,409.78209832)(324.46019653,409.72210297)
\curveto(324.69019089,409.67209843)(324.90519067,409.60709849)(325.10519653,409.52710297)
\curveto(325.19519038,409.48709861)(325.2801903,409.45209865)(325.36019653,409.42210297)
\curveto(325.45019013,409.4020987)(325.53519004,409.36709873)(325.61519653,409.31710297)
\curveto(325.80518977,409.1970989)(325.9751896,409.06709903)(326.12519653,408.92710297)
\curveto(326.28518929,408.78709931)(326.41018917,408.61209949)(326.50019653,408.40210297)
\curveto(326.53018905,408.33209977)(326.55518902,408.26209984)(326.57519653,408.19210297)
\curveto(326.59518898,408.12209998)(326.61518896,408.04710005)(326.63519653,407.96710297)
\curveto(326.64518893,407.90710019)(326.65018893,407.81210029)(326.65019653,407.68210297)
\curveto(326.66018892,407.56210054)(326.66018892,407.46710063)(326.65019653,407.39710297)
\lineto(326.65019653,407.32210297)
\curveto(326.63018895,407.26210084)(326.61518896,407.2021009)(326.60519653,407.14210297)
\curveto(326.60518897,407.09210101)(326.60018898,407.04210106)(326.59019653,406.99210297)
\curveto(326.52018906,406.69210141)(326.41018917,406.42710167)(326.26019653,406.19710297)
\curveto(326.10018948,405.95710214)(325.90518967,405.76210234)(325.67519653,405.61210297)
\curveto(325.44519013,405.46210264)(325.18519039,405.33210277)(324.89519653,405.22210297)
\curveto(324.78519079,405.17210293)(324.66519091,405.13710296)(324.53519653,405.11710297)
\curveto(324.41519116,405.097103)(324.29519128,405.07210303)(324.17519653,405.04210297)
\curveto(324.08519149,405.02210308)(323.99019159,405.01210309)(323.89019653,405.01210297)
\curveto(323.80019178,405.0021031)(323.71019187,404.98710311)(323.62019653,404.96710297)
\lineto(323.35019653,404.96710297)
\curveto(323.29019229,404.94710315)(323.18519239,404.93710316)(323.03519653,404.93710297)
\curveto(322.89519268,404.93710316)(322.79519278,404.94710315)(322.73519653,404.96710297)
\curveto(322.70519287,404.96710313)(322.67019291,404.97210313)(322.63019653,404.98210297)
\lineto(322.52519653,404.98210297)
\curveto(322.40519317,405.0021031)(322.28519329,405.01710308)(322.16519653,405.02710297)
\curveto(322.04519353,405.03710306)(321.93019365,405.05710304)(321.82019653,405.08710297)
\curveto(321.43019415,405.1971029)(321.08519449,405.32210278)(320.78519653,405.46210297)
\curveto(320.48519509,405.61210249)(320.23019535,405.83210227)(320.02019653,406.12210297)
\curveto(319.8801957,406.31210179)(319.76019582,406.53210157)(319.66019653,406.78210297)
\curveto(319.64019594,406.84210126)(319.62019596,406.92210118)(319.60019653,407.02210297)
\curveto(319.580196,407.07210103)(319.56519601,407.14210096)(319.55519653,407.23210297)
\curveto(319.54519603,407.32210078)(319.55019603,407.3971007)(319.57019653,407.45710297)
\curveto(319.60019598,407.52710057)(319.65019593,407.57710052)(319.72019653,407.60710297)
\curveto(319.77019581,407.62710047)(319.83019575,407.63710046)(319.90019653,407.63710297)
\lineto(320.12519653,407.63710297)
\lineto(320.83019653,407.63710297)
\lineto(321.07019653,407.63710297)
\curveto(321.15019443,407.63710046)(321.22019436,407.62710047)(321.28019653,407.60710297)
\curveto(321.39019419,407.56710053)(321.46019412,407.5021006)(321.49019653,407.41210297)
\curveto(321.53019405,407.32210078)(321.575194,407.22710087)(321.62519653,407.12710297)
\curveto(321.64519393,407.07710102)(321.6801939,407.01210109)(321.73019653,406.93210297)
\curveto(321.79019379,406.85210125)(321.84019374,406.8021013)(321.88019653,406.78210297)
\curveto(322.00019358,406.68210142)(322.11519346,406.6021015)(322.22519653,406.54210297)
\curveto(322.33519324,406.49210161)(322.4751931,406.44210166)(322.64519653,406.39210297)
\curveto(322.69519288,406.37210173)(322.74519283,406.36210174)(322.79519653,406.36210297)
\curveto(322.84519273,406.37210173)(322.89519268,406.37210173)(322.94519653,406.36210297)
\curveto(323.02519255,406.34210176)(323.11019247,406.33210177)(323.20019653,406.33210297)
\curveto(323.30019228,406.34210176)(323.38519219,406.35710174)(323.45519653,406.37710297)
\curveto(323.50519207,406.38710171)(323.55019203,406.39210171)(323.59019653,406.39210297)
\curveto(323.64019194,406.39210171)(323.69019189,406.4021017)(323.74019653,406.42210297)
\curveto(323.8801917,406.47210163)(324.00519157,406.53210157)(324.11519653,406.60210297)
\curveto(324.23519134,406.67210143)(324.33019125,406.76210134)(324.40019653,406.87210297)
\curveto(324.45019113,406.95210115)(324.49019109,407.07710102)(324.52019653,407.24710297)
\curveto(324.54019104,407.31710078)(324.54019104,407.38210072)(324.52019653,407.44210297)
\curveto(324.50019108,407.5021006)(324.4801911,407.55210055)(324.46019653,407.59210297)
\curveto(324.39019119,407.73210037)(324.30019128,407.83710026)(324.19019653,407.90710297)
\curveto(324.09019149,407.97710012)(323.97019161,408.04210006)(323.83019653,408.10210297)
\curveto(323.64019194,408.18209992)(323.44019214,408.24709985)(323.23019653,408.29710297)
\curveto(323.02019256,408.34709975)(322.81019277,408.4020997)(322.60019653,408.46210297)
\curveto(322.52019306,408.48209962)(322.43519314,408.4970996)(322.34519653,408.50710297)
\curveto(322.26519331,408.51709958)(322.18519339,408.53209957)(322.10519653,408.55210297)
\curveto(321.78519379,408.64209946)(321.4801941,408.72709937)(321.19019653,408.80710297)
\curveto(320.90019468,408.8970992)(320.63519494,409.02709907)(320.39519653,409.19710297)
\curveto(320.11519546,409.3970987)(319.91019567,409.66709843)(319.78019653,410.00710297)
\curveto(319.76019582,410.07709802)(319.74019584,410.17209793)(319.72019653,410.29210297)
\curveto(319.70019588,410.36209774)(319.68519589,410.44709765)(319.67519653,410.54710297)
\curveto(319.66519591,410.64709745)(319.67019591,410.73709736)(319.69019653,410.81710297)
\curveto(319.71019587,410.86709723)(319.71519586,410.90709719)(319.70519653,410.93710297)
\curveto(319.69519588,410.97709712)(319.70019588,411.02209708)(319.72019653,411.07210297)
\curveto(319.74019584,411.18209692)(319.76019582,411.28209682)(319.78019653,411.37210297)
\curveto(319.81019577,411.47209663)(319.84519573,411.56709653)(319.88519653,411.65710297)
\curveto(320.01519556,411.94709615)(320.19519538,412.18209592)(320.42519653,412.36210297)
\curveto(320.65519492,412.54209556)(320.91519466,412.68709541)(321.20519653,412.79710297)
\curveto(321.31519426,412.84709525)(321.43019415,412.88209522)(321.55019653,412.90210297)
\curveto(321.67019391,412.93209517)(321.79519378,412.96209514)(321.92519653,412.99210297)
\curveto(321.98519359,413.01209509)(322.04519353,413.02209508)(322.10519653,413.02210297)
\lineto(322.28519653,413.05210297)
\curveto(322.36519321,413.06209504)(322.45019313,413.06709503)(322.54019653,413.06710297)
\curveto(322.63019295,413.06709503)(322.71519286,413.07209503)(322.79519653,413.08210297)
}
}
{
\newrgbcolor{curcolor}{0 0 0}
\pscustom[linestyle=none,fillstyle=solid,fillcolor=curcolor]
{
\newpath
\moveto(329.98183716,415.72210297)
\curveto(330.05183421,415.64209246)(330.08683417,415.52209258)(330.08683716,415.36210297)
\lineto(330.08683716,414.89710297)
\lineto(330.08683716,414.49210297)
\curveto(330.08683417,414.35209375)(330.05183421,414.25709384)(329.98183716,414.20710297)
\curveto(329.92183434,414.15709394)(329.84183442,414.12709397)(329.74183716,414.11710297)
\curveto(329.65183461,414.10709399)(329.55183471,414.102094)(329.44183716,414.10210297)
\lineto(328.60183716,414.10210297)
\curveto(328.49183577,414.102094)(328.39183587,414.10709399)(328.30183716,414.11710297)
\curveto(328.22183604,414.12709397)(328.15183611,414.15709394)(328.09183716,414.20710297)
\curveto(328.05183621,414.23709386)(328.02183624,414.29209381)(328.00183716,414.37210297)
\curveto(327.99183627,414.46209364)(327.98183628,414.55709354)(327.97183716,414.65710297)
\lineto(327.97183716,414.98710297)
\curveto(327.98183628,415.097093)(327.98683627,415.19209291)(327.98683716,415.27210297)
\lineto(327.98683716,415.48210297)
\curveto(327.99683626,415.55209255)(328.01683624,415.61209249)(328.04683716,415.66210297)
\curveto(328.06683619,415.7020924)(328.09183617,415.73209237)(328.12183716,415.75210297)
\lineto(328.24183716,415.81210297)
\curveto(328.261836,415.81209229)(328.28683597,415.81209229)(328.31683716,415.81210297)
\curveto(328.34683591,415.82209228)(328.37183589,415.82709227)(328.39183716,415.82710297)
\lineto(329.48683716,415.82710297)
\curveto(329.58683467,415.82709227)(329.68183458,415.82209228)(329.77183716,415.81210297)
\curveto(329.8618344,415.8020923)(329.93183433,415.77209233)(329.98183716,415.72210297)
\moveto(330.08683716,405.95710297)
\curveto(330.08683417,405.75710234)(330.08183418,405.58710251)(330.07183716,405.44710297)
\curveto(330.0618342,405.30710279)(329.97183429,405.21210289)(329.80183716,405.16210297)
\curveto(329.74183452,405.14210296)(329.67683458,405.13210297)(329.60683716,405.13210297)
\curveto(329.53683472,405.14210296)(329.4618348,405.14710295)(329.38183716,405.14710297)
\lineto(328.54183716,405.14710297)
\curveto(328.45183581,405.14710295)(328.3618359,405.15210295)(328.27183716,405.16210297)
\curveto(328.19183607,405.17210293)(328.13183613,405.2021029)(328.09183716,405.25210297)
\curveto(328.03183623,405.32210278)(327.99683626,405.40710269)(327.98683716,405.50710297)
\lineto(327.98683716,405.85210297)
\lineto(327.98683716,412.18210297)
\lineto(327.98683716,412.48210297)
\curveto(327.98683627,412.58209552)(328.00683625,412.66209544)(328.04683716,412.72210297)
\curveto(328.10683615,412.79209531)(328.19183607,412.83709526)(328.30183716,412.85710297)
\curveto(328.32183594,412.86709523)(328.34683591,412.86709523)(328.37683716,412.85710297)
\curveto(328.41683584,412.85709524)(328.44683581,412.86209524)(328.46683716,412.87210297)
\lineto(329.21683716,412.87210297)
\lineto(329.41183716,412.87210297)
\curveto(329.49183477,412.88209522)(329.5568347,412.88209522)(329.60683716,412.87210297)
\lineto(329.72683716,412.87210297)
\curveto(329.78683447,412.85209525)(329.84183442,412.83709526)(329.89183716,412.82710297)
\curveto(329.94183432,412.81709528)(329.98183428,412.78709531)(330.01183716,412.73710297)
\curveto(330.05183421,412.68709541)(330.07183419,412.61709548)(330.07183716,412.52710297)
\curveto(330.08183418,412.43709566)(330.08683417,412.34209576)(330.08683716,412.24210297)
\lineto(330.08683716,405.95710297)
}
}
{
\newrgbcolor{curcolor}{0 0 0}
\pscustom[linestyle=none,fillstyle=solid,fillcolor=curcolor]
{
\newpath
\moveto(334.71902466,413.08210297)
\curveto(335.46902016,413.102095)(336.11901951,413.01709508)(336.66902466,412.82710297)
\curveto(337.2290184,412.64709545)(337.65401797,412.33209577)(337.94402466,411.88210297)
\curveto(338.01401761,411.77209633)(338.07401755,411.65709644)(338.12402466,411.53710297)
\curveto(338.18401744,411.42709667)(338.23401739,411.3020968)(338.27402466,411.16210297)
\curveto(338.29401733,411.102097)(338.30401732,411.03709706)(338.30402466,410.96710297)
\curveto(338.30401732,410.8970972)(338.29401733,410.83709726)(338.27402466,410.78710297)
\curveto(338.23401739,410.72709737)(338.17901745,410.68709741)(338.10902466,410.66710297)
\curveto(338.05901757,410.64709745)(337.99901763,410.63709746)(337.92902466,410.63710297)
\lineto(337.71902466,410.63710297)
\lineto(337.05902466,410.63710297)
\curveto(336.98901864,410.63709746)(336.91901871,410.63209747)(336.84902466,410.62210297)
\curveto(336.77901885,410.62209748)(336.71401891,410.63209747)(336.65402466,410.65210297)
\curveto(336.55401907,410.67209743)(336.47901915,410.71209739)(336.42902466,410.77210297)
\curveto(336.37901925,410.83209727)(336.33401929,410.89209721)(336.29402466,410.95210297)
\lineto(336.17402466,411.16210297)
\curveto(336.14401948,411.24209686)(336.09401953,411.30709679)(336.02402466,411.35710297)
\curveto(335.9240197,411.43709666)(335.8240198,411.4970966)(335.72402466,411.53710297)
\curveto(335.63401999,411.57709652)(335.51902011,411.61209649)(335.37902466,411.64210297)
\curveto(335.30902032,411.66209644)(335.20402042,411.67709642)(335.06402466,411.68710297)
\curveto(334.93402069,411.6970964)(334.83402079,411.69209641)(334.76402466,411.67210297)
\lineto(334.65902466,411.67210297)
\lineto(334.50902466,411.64210297)
\curveto(334.46902116,411.64209646)(334.4240212,411.63709646)(334.37402466,411.62710297)
\curveto(334.20402142,411.57709652)(334.06402156,411.50709659)(333.95402466,411.41710297)
\curveto(333.85402177,411.33709676)(333.78402184,411.21209689)(333.74402466,411.04210297)
\curveto(333.7240219,410.97209713)(333.7240219,410.90709719)(333.74402466,410.84710297)
\curveto(333.76402186,410.78709731)(333.78402184,410.73709736)(333.80402466,410.69710297)
\curveto(333.87402175,410.57709752)(333.95402167,410.48209762)(334.04402466,410.41210297)
\curveto(334.14402148,410.34209776)(334.25902137,410.28209782)(334.38902466,410.23210297)
\curveto(334.57902105,410.15209795)(334.78402084,410.08209802)(335.00402466,410.02210297)
\lineto(335.69402466,409.87210297)
\curveto(335.93401969,409.83209827)(336.16401946,409.78209832)(336.38402466,409.72210297)
\curveto(336.61401901,409.67209843)(336.8290188,409.60709849)(337.02902466,409.52710297)
\curveto(337.11901851,409.48709861)(337.20401842,409.45209865)(337.28402466,409.42210297)
\curveto(337.37401825,409.4020987)(337.45901817,409.36709873)(337.53902466,409.31710297)
\curveto(337.7290179,409.1970989)(337.89901773,409.06709903)(338.04902466,408.92710297)
\curveto(338.20901742,408.78709931)(338.33401729,408.61209949)(338.42402466,408.40210297)
\curveto(338.45401717,408.33209977)(338.47901715,408.26209984)(338.49902466,408.19210297)
\curveto(338.51901711,408.12209998)(338.53901709,408.04710005)(338.55902466,407.96710297)
\curveto(338.56901706,407.90710019)(338.57401705,407.81210029)(338.57402466,407.68210297)
\curveto(338.58401704,407.56210054)(338.58401704,407.46710063)(338.57402466,407.39710297)
\lineto(338.57402466,407.32210297)
\curveto(338.55401707,407.26210084)(338.53901709,407.2021009)(338.52902466,407.14210297)
\curveto(338.5290171,407.09210101)(338.5240171,407.04210106)(338.51402466,406.99210297)
\curveto(338.44401718,406.69210141)(338.33401729,406.42710167)(338.18402466,406.19710297)
\curveto(338.0240176,405.95710214)(337.8290178,405.76210234)(337.59902466,405.61210297)
\curveto(337.36901826,405.46210264)(337.10901852,405.33210277)(336.81902466,405.22210297)
\curveto(336.70901892,405.17210293)(336.58901904,405.13710296)(336.45902466,405.11710297)
\curveto(336.33901929,405.097103)(336.21901941,405.07210303)(336.09902466,405.04210297)
\curveto(336.00901962,405.02210308)(335.91401971,405.01210309)(335.81402466,405.01210297)
\curveto(335.7240199,405.0021031)(335.63401999,404.98710311)(335.54402466,404.96710297)
\lineto(335.27402466,404.96710297)
\curveto(335.21402041,404.94710315)(335.10902052,404.93710316)(334.95902466,404.93710297)
\curveto(334.81902081,404.93710316)(334.71902091,404.94710315)(334.65902466,404.96710297)
\curveto(334.629021,404.96710313)(334.59402103,404.97210313)(334.55402466,404.98210297)
\lineto(334.44902466,404.98210297)
\curveto(334.3290213,405.0021031)(334.20902142,405.01710308)(334.08902466,405.02710297)
\curveto(333.96902166,405.03710306)(333.85402177,405.05710304)(333.74402466,405.08710297)
\curveto(333.35402227,405.1971029)(333.00902262,405.32210278)(332.70902466,405.46210297)
\curveto(332.40902322,405.61210249)(332.15402347,405.83210227)(331.94402466,406.12210297)
\curveto(331.80402382,406.31210179)(331.68402394,406.53210157)(331.58402466,406.78210297)
\curveto(331.56402406,406.84210126)(331.54402408,406.92210118)(331.52402466,407.02210297)
\curveto(331.50402412,407.07210103)(331.48902414,407.14210096)(331.47902466,407.23210297)
\curveto(331.46902416,407.32210078)(331.47402415,407.3971007)(331.49402466,407.45710297)
\curveto(331.5240241,407.52710057)(331.57402405,407.57710052)(331.64402466,407.60710297)
\curveto(331.69402393,407.62710047)(331.75402387,407.63710046)(331.82402466,407.63710297)
\lineto(332.04902466,407.63710297)
\lineto(332.75402466,407.63710297)
\lineto(332.99402466,407.63710297)
\curveto(333.07402255,407.63710046)(333.14402248,407.62710047)(333.20402466,407.60710297)
\curveto(333.31402231,407.56710053)(333.38402224,407.5021006)(333.41402466,407.41210297)
\curveto(333.45402217,407.32210078)(333.49902213,407.22710087)(333.54902466,407.12710297)
\curveto(333.56902206,407.07710102)(333.60402202,407.01210109)(333.65402466,406.93210297)
\curveto(333.71402191,406.85210125)(333.76402186,406.8021013)(333.80402466,406.78210297)
\curveto(333.9240217,406.68210142)(334.03902159,406.6021015)(334.14902466,406.54210297)
\curveto(334.25902137,406.49210161)(334.39902123,406.44210166)(334.56902466,406.39210297)
\curveto(334.61902101,406.37210173)(334.66902096,406.36210174)(334.71902466,406.36210297)
\curveto(334.76902086,406.37210173)(334.81902081,406.37210173)(334.86902466,406.36210297)
\curveto(334.94902068,406.34210176)(335.03402059,406.33210177)(335.12402466,406.33210297)
\curveto(335.2240204,406.34210176)(335.30902032,406.35710174)(335.37902466,406.37710297)
\curveto(335.4290202,406.38710171)(335.47402015,406.39210171)(335.51402466,406.39210297)
\curveto(335.56402006,406.39210171)(335.61402001,406.4021017)(335.66402466,406.42210297)
\curveto(335.80401982,406.47210163)(335.9290197,406.53210157)(336.03902466,406.60210297)
\curveto(336.15901947,406.67210143)(336.25401937,406.76210134)(336.32402466,406.87210297)
\curveto(336.37401925,406.95210115)(336.41401921,407.07710102)(336.44402466,407.24710297)
\curveto(336.46401916,407.31710078)(336.46401916,407.38210072)(336.44402466,407.44210297)
\curveto(336.4240192,407.5021006)(336.40401922,407.55210055)(336.38402466,407.59210297)
\curveto(336.31401931,407.73210037)(336.2240194,407.83710026)(336.11402466,407.90710297)
\curveto(336.01401961,407.97710012)(335.89401973,408.04210006)(335.75402466,408.10210297)
\curveto(335.56402006,408.18209992)(335.36402026,408.24709985)(335.15402466,408.29710297)
\curveto(334.94402068,408.34709975)(334.73402089,408.4020997)(334.52402466,408.46210297)
\curveto(334.44402118,408.48209962)(334.35902127,408.4970996)(334.26902466,408.50710297)
\curveto(334.18902144,408.51709958)(334.10902152,408.53209957)(334.02902466,408.55210297)
\curveto(333.70902192,408.64209946)(333.40402222,408.72709937)(333.11402466,408.80710297)
\curveto(332.8240228,408.8970992)(332.55902307,409.02709907)(332.31902466,409.19710297)
\curveto(332.03902359,409.3970987)(331.83402379,409.66709843)(331.70402466,410.00710297)
\curveto(331.68402394,410.07709802)(331.66402396,410.17209793)(331.64402466,410.29210297)
\curveto(331.624024,410.36209774)(331.60902402,410.44709765)(331.59902466,410.54710297)
\curveto(331.58902404,410.64709745)(331.59402403,410.73709736)(331.61402466,410.81710297)
\curveto(331.63402399,410.86709723)(331.63902399,410.90709719)(331.62902466,410.93710297)
\curveto(331.61902401,410.97709712)(331.624024,411.02209708)(331.64402466,411.07210297)
\curveto(331.66402396,411.18209692)(331.68402394,411.28209682)(331.70402466,411.37210297)
\curveto(331.73402389,411.47209663)(331.76902386,411.56709653)(331.80902466,411.65710297)
\curveto(331.93902369,411.94709615)(332.11902351,412.18209592)(332.34902466,412.36210297)
\curveto(332.57902305,412.54209556)(332.83902279,412.68709541)(333.12902466,412.79710297)
\curveto(333.23902239,412.84709525)(333.35402227,412.88209522)(333.47402466,412.90210297)
\curveto(333.59402203,412.93209517)(333.71902191,412.96209514)(333.84902466,412.99210297)
\curveto(333.90902172,413.01209509)(333.96902166,413.02209508)(334.02902466,413.02210297)
\lineto(334.20902466,413.05210297)
\curveto(334.28902134,413.06209504)(334.37402125,413.06709503)(334.46402466,413.06710297)
\curveto(334.55402107,413.06709503)(334.63902099,413.07209503)(334.71902466,413.08210297)
}
}
{
\newrgbcolor{curcolor}{0 0 0}
\pscustom[linestyle=none,fillstyle=solid,fillcolor=curcolor]
{
\newpath
\moveto(340.85566528,415.18210297)
\lineto(341.86066528,415.18210297)
\curveto(342.0106623,415.18209292)(342.14066217,415.17209293)(342.25066528,415.15210297)
\curveto(342.37066194,415.14209296)(342.45566185,415.08209302)(342.50566528,414.97210297)
\curveto(342.52566178,414.92209318)(342.53566177,414.86209324)(342.53566528,414.79210297)
\lineto(342.53566528,414.58210297)
\lineto(342.53566528,413.90710297)
\curveto(342.53566177,413.85709424)(342.53066178,413.7970943)(342.52066528,413.72710297)
\curveto(342.52066179,413.66709443)(342.52566178,413.61209449)(342.53566528,413.56210297)
\lineto(342.53566528,413.39710297)
\curveto(342.53566177,413.31709478)(342.54066177,413.24209486)(342.55066528,413.17210297)
\curveto(342.56066175,413.11209499)(342.58566172,413.05709504)(342.62566528,413.00710297)
\curveto(342.69566161,412.91709518)(342.82066149,412.86709523)(343.00066528,412.85710297)
\lineto(343.54066528,412.85710297)
\lineto(343.72066528,412.85710297)
\curveto(343.78066053,412.85709524)(343.83566047,412.84709525)(343.88566528,412.82710297)
\curveto(343.99566031,412.77709532)(344.05566025,412.68709541)(344.06566528,412.55710297)
\curveto(344.08566022,412.42709567)(344.09566021,412.28209582)(344.09566528,412.12210297)
\lineto(344.09566528,411.91210297)
\curveto(344.1056602,411.84209626)(344.10066021,411.78209632)(344.08066528,411.73210297)
\curveto(344.03066028,411.57209653)(343.92566038,411.48709661)(343.76566528,411.47710297)
\curveto(343.6056607,411.46709663)(343.42566088,411.46209664)(343.22566528,411.46210297)
\lineto(343.09066528,411.46210297)
\curveto(343.05066126,411.47209663)(343.01566129,411.47209663)(342.98566528,411.46210297)
\curveto(342.94566136,411.45209665)(342.9106614,411.44709665)(342.88066528,411.44710297)
\curveto(342.85066146,411.45709664)(342.82066149,411.45209665)(342.79066528,411.43210297)
\curveto(342.7106616,411.41209669)(342.65066166,411.36709673)(342.61066528,411.29710297)
\curveto(342.58066173,411.23709686)(342.55566175,411.16209694)(342.53566528,411.07210297)
\curveto(342.52566178,411.02209708)(342.52566178,410.96709713)(342.53566528,410.90710297)
\curveto(342.54566176,410.84709725)(342.54566176,410.79209731)(342.53566528,410.74210297)
\lineto(342.53566528,409.81210297)
\lineto(342.53566528,408.05710297)
\curveto(342.53566177,407.80710029)(342.54066177,407.58710051)(342.55066528,407.39710297)
\curveto(342.57066174,407.21710088)(342.63566167,407.05710104)(342.74566528,406.91710297)
\curveto(342.79566151,406.85710124)(342.86066145,406.81210129)(342.94066528,406.78210297)
\lineto(343.21066528,406.72210297)
\curveto(343.24066107,406.71210139)(343.27066104,406.70710139)(343.30066528,406.70710297)
\curveto(343.34066097,406.71710138)(343.37066094,406.71710138)(343.39066528,406.70710297)
\lineto(343.55566528,406.70710297)
\curveto(343.66566064,406.70710139)(343.76066055,406.7021014)(343.84066528,406.69210297)
\curveto(343.92066039,406.68210142)(343.98566032,406.64210146)(344.03566528,406.57210297)
\curveto(344.07566023,406.51210159)(344.09566021,406.43210167)(344.09566528,406.33210297)
\lineto(344.09566528,406.04710297)
\curveto(344.09566021,405.83710226)(344.09066022,405.64210246)(344.08066528,405.46210297)
\curveto(344.08066023,405.29210281)(344.00066031,405.17710292)(343.84066528,405.11710297)
\curveto(343.79066052,405.097103)(343.74566056,405.09210301)(343.70566528,405.10210297)
\curveto(343.66566064,405.102103)(343.62066069,405.09210301)(343.57066528,405.07210297)
\lineto(343.42066528,405.07210297)
\curveto(343.40066091,405.07210303)(343.37066094,405.07710302)(343.33066528,405.08710297)
\curveto(343.29066102,405.08710301)(343.25566105,405.08210302)(343.22566528,405.07210297)
\curveto(343.17566113,405.06210304)(343.12066119,405.06210304)(343.06066528,405.07210297)
\lineto(342.91066528,405.07210297)
\lineto(342.76066528,405.07210297)
\curveto(342.7106616,405.06210304)(342.66566164,405.06210304)(342.62566528,405.07210297)
\lineto(342.46066528,405.07210297)
\curveto(342.4106619,405.08210302)(342.35566195,405.08710301)(342.29566528,405.08710297)
\curveto(342.23566207,405.08710301)(342.18066213,405.09210301)(342.13066528,405.10210297)
\curveto(342.06066225,405.11210299)(341.99566231,405.12210298)(341.93566528,405.13210297)
\lineto(341.75566528,405.16210297)
\curveto(341.64566266,405.19210291)(341.54066277,405.22710287)(341.44066528,405.26710297)
\curveto(341.34066297,405.30710279)(341.24566306,405.35210275)(341.15566528,405.40210297)
\lineto(341.06566528,405.46210297)
\curveto(341.03566327,405.49210261)(341.00066331,405.52210258)(340.96066528,405.55210297)
\curveto(340.94066337,405.57210253)(340.91566339,405.59210251)(340.88566528,405.61210297)
\lineto(340.81066528,405.68710297)
\curveto(340.67066364,405.87710222)(340.56566374,406.08710201)(340.49566528,406.31710297)
\curveto(340.47566383,406.35710174)(340.46566384,406.39210171)(340.46566528,406.42210297)
\curveto(340.47566383,406.46210164)(340.47566383,406.50710159)(340.46566528,406.55710297)
\curveto(340.45566385,406.57710152)(340.45066386,406.6021015)(340.45066528,406.63210297)
\curveto(340.45066386,406.66210144)(340.44566386,406.68710141)(340.43566528,406.70710297)
\lineto(340.43566528,406.85710297)
\curveto(340.42566388,406.8971012)(340.42066389,406.94210116)(340.42066528,406.99210297)
\curveto(340.43066388,407.04210106)(340.43566387,407.09210101)(340.43566528,407.14210297)
\lineto(340.43566528,407.71210297)
\lineto(340.43566528,409.94710297)
\lineto(340.43566528,410.74210297)
\lineto(340.43566528,410.95210297)
\curveto(340.44566386,411.02209708)(340.44066387,411.08709701)(340.42066528,411.14710297)
\curveto(340.38066393,411.28709681)(340.310664,411.37709672)(340.21066528,411.41710297)
\curveto(340.10066421,411.46709663)(339.96066435,411.48209662)(339.79066528,411.46210297)
\curveto(339.62066469,411.44209666)(339.47566483,411.45709664)(339.35566528,411.50710297)
\curveto(339.27566503,411.53709656)(339.22566508,411.58209652)(339.20566528,411.64210297)
\curveto(339.18566512,411.7020964)(339.16566514,411.77709632)(339.14566528,411.86710297)
\lineto(339.14566528,412.18210297)
\curveto(339.14566516,412.36209574)(339.15566515,412.50709559)(339.17566528,412.61710297)
\curveto(339.19566511,412.72709537)(339.28066503,412.8020953)(339.43066528,412.84210297)
\curveto(339.47066484,412.86209524)(339.5106648,412.86709523)(339.55066528,412.85710297)
\lineto(339.68566528,412.85710297)
\curveto(339.83566447,412.85709524)(339.97566433,412.86209524)(340.10566528,412.87210297)
\curveto(340.23566407,412.89209521)(340.32566398,412.95209515)(340.37566528,413.05210297)
\curveto(340.4056639,413.12209498)(340.42066389,413.2020949)(340.42066528,413.29210297)
\curveto(340.43066388,413.38209472)(340.43566387,413.47209463)(340.43566528,413.56210297)
\lineto(340.43566528,414.49210297)
\lineto(340.43566528,414.74710297)
\curveto(340.43566387,414.83709326)(340.44566386,414.91209319)(340.46566528,414.97210297)
\curveto(340.51566379,415.07209303)(340.59066372,415.13709296)(340.69066528,415.16710297)
\curveto(340.7106636,415.17709292)(340.73566357,415.17709292)(340.76566528,415.16710297)
\curveto(340.8056635,415.16709293)(340.83566347,415.17209293)(340.85566528,415.18210297)
}
}
{
\newrgbcolor{curcolor}{0 0 0}
\pscustom[linestyle=none,fillstyle=solid,fillcolor=curcolor]
{
\newpath
\moveto(352.44410278,409.07710297)
\curveto(352.46409462,408.9970991)(352.46409462,408.90709919)(352.44410278,408.80710297)
\curveto(352.42409466,408.70709939)(352.38909469,408.64209946)(352.33910278,408.61210297)
\curveto(352.28909479,408.57209953)(352.21409487,408.54209956)(352.11410278,408.52210297)
\curveto(352.02409506,408.51209959)(351.91909516,408.5020996)(351.79910278,408.49210297)
\lineto(351.45410278,408.49210297)
\curveto(351.34409574,408.5020996)(351.24409584,408.50709959)(351.15410278,408.50710297)
\lineto(347.49410278,408.50710297)
\lineto(347.28410278,408.50710297)
\curveto(347.22409986,408.50709959)(347.16909991,408.4970996)(347.11910278,408.47710297)
\curveto(347.03910004,408.43709966)(346.98910009,408.3970997)(346.96910278,408.35710297)
\curveto(346.94910013,408.33709976)(346.92910015,408.2970998)(346.90910278,408.23710297)
\curveto(346.88910019,408.18709991)(346.8841002,408.13709996)(346.89410278,408.08710297)
\curveto(346.91410017,408.02710007)(346.92410016,407.96710013)(346.92410278,407.90710297)
\curveto(346.93410015,407.85710024)(346.94910013,407.8021003)(346.96910278,407.74210297)
\curveto(347.04910003,407.5021006)(347.14409994,407.3021008)(347.25410278,407.14210297)
\curveto(347.37409971,406.99210111)(347.53409955,406.85710124)(347.73410278,406.73710297)
\curveto(347.81409927,406.68710141)(347.89409919,406.65210145)(347.97410278,406.63210297)
\curveto(348.06409902,406.62210148)(348.15409893,406.6021015)(348.24410278,406.57210297)
\curveto(348.32409876,406.55210155)(348.43409865,406.53710156)(348.57410278,406.52710297)
\curveto(348.71409837,406.51710158)(348.83409825,406.52210158)(348.93410278,406.54210297)
\lineto(349.06910278,406.54210297)
\curveto(349.16909791,406.56210154)(349.25909782,406.58210152)(349.33910278,406.60210297)
\curveto(349.42909765,406.63210147)(349.51409757,406.66210144)(349.59410278,406.69210297)
\curveto(349.69409739,406.74210136)(349.80409728,406.80710129)(349.92410278,406.88710297)
\curveto(350.05409703,406.96710113)(350.14909693,407.04710105)(350.20910278,407.12710297)
\curveto(350.25909682,407.1971009)(350.30909677,407.26210084)(350.35910278,407.32210297)
\curveto(350.41909666,407.39210071)(350.48909659,407.44210066)(350.56910278,407.47210297)
\curveto(350.66909641,407.52210058)(350.79409629,407.54210056)(350.94410278,407.53210297)
\lineto(351.37910278,407.53210297)
\lineto(351.55910278,407.53210297)
\curveto(351.62909545,407.54210056)(351.68909539,407.53710056)(351.73910278,407.51710297)
\lineto(351.88910278,407.51710297)
\curveto(351.98909509,407.4971006)(352.05909502,407.47210063)(352.09910278,407.44210297)
\curveto(352.13909494,407.42210068)(352.15909492,407.37710072)(352.15910278,407.30710297)
\curveto(352.16909491,407.23710086)(352.16409492,407.17710092)(352.14410278,407.12710297)
\curveto(352.09409499,406.98710111)(352.03909504,406.86210124)(351.97910278,406.75210297)
\curveto(351.91909516,406.64210146)(351.84909523,406.53210157)(351.76910278,406.42210297)
\curveto(351.54909553,406.09210201)(351.29909578,405.82710227)(351.01910278,405.62710297)
\curveto(350.73909634,405.42710267)(350.38909669,405.25710284)(349.96910278,405.11710297)
\curveto(349.85909722,405.07710302)(349.74909733,405.05210305)(349.63910278,405.04210297)
\curveto(349.52909755,405.03210307)(349.41409767,405.01210309)(349.29410278,404.98210297)
\curveto(349.25409783,404.97210313)(349.20909787,404.97210313)(349.15910278,404.98210297)
\curveto(349.11909796,404.98210312)(349.079098,404.97710312)(349.03910278,404.96710297)
\lineto(348.87410278,404.96710297)
\curveto(348.82409826,404.94710315)(348.76409832,404.94210316)(348.69410278,404.95210297)
\curveto(348.63409845,404.95210315)(348.5790985,404.95710314)(348.52910278,404.96710297)
\curveto(348.44909863,404.97710312)(348.3790987,404.97710312)(348.31910278,404.96710297)
\curveto(348.25909882,404.95710314)(348.19409889,404.96210314)(348.12410278,404.98210297)
\curveto(348.07409901,405.0021031)(348.01909906,405.01210309)(347.95910278,405.01210297)
\curveto(347.89909918,405.01210309)(347.84409924,405.02210308)(347.79410278,405.04210297)
\curveto(347.6840994,405.06210304)(347.57409951,405.08710301)(347.46410278,405.11710297)
\curveto(347.35409973,405.13710296)(347.25409983,405.17210293)(347.16410278,405.22210297)
\curveto(347.05410003,405.26210284)(346.94910013,405.2971028)(346.84910278,405.32710297)
\curveto(346.75910032,405.36710273)(346.67410041,405.41210269)(346.59410278,405.46210297)
\curveto(346.27410081,405.66210244)(345.98910109,405.89210221)(345.73910278,406.15210297)
\curveto(345.48910159,406.42210168)(345.2841018,406.73210137)(345.12410278,407.08210297)
\curveto(345.07410201,407.19210091)(345.03410205,407.3021008)(345.00410278,407.41210297)
\curveto(344.97410211,407.53210057)(344.93410215,407.65210045)(344.88410278,407.77210297)
\curveto(344.87410221,407.81210029)(344.86910221,407.84710025)(344.86910278,407.87710297)
\curveto(344.86910221,407.91710018)(344.86410222,407.95710014)(344.85410278,407.99710297)
\curveto(344.81410227,408.11709998)(344.78910229,408.24709985)(344.77910278,408.38710297)
\lineto(344.74910278,408.80710297)
\curveto(344.74910233,408.85709924)(344.74410234,408.91209919)(344.73410278,408.97210297)
\curveto(344.73410235,409.03209907)(344.73910234,409.08709901)(344.74910278,409.13710297)
\lineto(344.74910278,409.31710297)
\lineto(344.79410278,409.67710297)
\curveto(344.83410225,409.84709825)(344.86910221,410.01209809)(344.89910278,410.17210297)
\curveto(344.92910215,410.33209777)(344.97410211,410.48209762)(345.03410278,410.62210297)
\curveto(345.46410162,411.66209644)(346.19410089,412.3970957)(347.22410278,412.82710297)
\curveto(347.36409972,412.88709521)(347.50409958,412.92709517)(347.64410278,412.94710297)
\curveto(347.79409929,412.97709512)(347.94909913,413.01209509)(348.10910278,413.05210297)
\curveto(348.18909889,413.06209504)(348.26409882,413.06709503)(348.33410278,413.06710297)
\curveto(348.40409868,413.06709503)(348.4790986,413.07209503)(348.55910278,413.08210297)
\curveto(349.06909801,413.09209501)(349.50409758,413.03209507)(349.86410278,412.90210297)
\curveto(350.23409685,412.78209532)(350.56409652,412.62209548)(350.85410278,412.42210297)
\curveto(350.94409614,412.36209574)(351.03409605,412.29209581)(351.12410278,412.21210297)
\curveto(351.21409587,412.14209596)(351.29409579,412.06709603)(351.36410278,411.98710297)
\curveto(351.39409569,411.93709616)(351.43409565,411.8970962)(351.48410278,411.86710297)
\curveto(351.56409552,411.75709634)(351.63909544,411.64209646)(351.70910278,411.52210297)
\curveto(351.7790953,411.41209669)(351.85409523,411.2970968)(351.93410278,411.17710297)
\curveto(351.9840951,411.08709701)(352.02409506,410.99209711)(352.05410278,410.89210297)
\curveto(352.09409499,410.8020973)(352.13409495,410.7020974)(352.17410278,410.59210297)
\curveto(352.22409486,410.46209764)(352.26409482,410.32709777)(352.29410278,410.18710297)
\curveto(352.32409476,410.04709805)(352.35909472,409.90709819)(352.39910278,409.76710297)
\curveto(352.41909466,409.68709841)(352.42409466,409.5970985)(352.41410278,409.49710297)
\curveto(352.41409467,409.40709869)(352.42409466,409.32209878)(352.44410278,409.24210297)
\lineto(352.44410278,409.07710297)
\moveto(350.19410278,409.96210297)
\curveto(350.26409682,410.06209804)(350.26909681,410.18209792)(350.20910278,410.32210297)
\curveto(350.15909692,410.47209763)(350.11909696,410.58209752)(350.08910278,410.65210297)
\curveto(349.94909713,410.92209718)(349.76409732,411.12709697)(349.53410278,411.26710297)
\curveto(349.30409778,411.41709668)(348.9840981,411.4970966)(348.57410278,411.50710297)
\curveto(348.54409854,411.48709661)(348.50909857,411.48209662)(348.46910278,411.49210297)
\curveto(348.42909865,411.5020966)(348.39409869,411.5020966)(348.36410278,411.49210297)
\curveto(348.31409877,411.47209663)(348.25909882,411.45709664)(348.19910278,411.44710297)
\curveto(348.13909894,411.44709665)(348.084099,411.43709666)(348.03410278,411.41710297)
\curveto(347.59409949,411.27709682)(347.26909981,411.0020971)(347.05910278,410.59210297)
\curveto(347.03910004,410.55209755)(347.01410007,410.4970976)(346.98410278,410.42710297)
\curveto(346.96410012,410.36709773)(346.94910013,410.3020978)(346.93910278,410.23210297)
\curveto(346.92910015,410.17209793)(346.92910015,410.11209799)(346.93910278,410.05210297)
\curveto(346.95910012,409.99209811)(346.99410009,409.94209816)(347.04410278,409.90210297)
\curveto(347.12409996,409.85209825)(347.23409985,409.82709827)(347.37410278,409.82710297)
\lineto(347.77910278,409.82710297)
\lineto(349.44410278,409.82710297)
\lineto(349.87910278,409.82710297)
\curveto(350.03909704,409.83709826)(350.14409694,409.88209822)(350.19410278,409.96210297)
}
}
{
\newrgbcolor{curcolor}{0 0 0}
\pscustom[linestyle=none,fillstyle=solid,fillcolor=curcolor]
{
\newpath
\moveto(358.14738403,413.06710297)
\curveto(358.51737843,413.07709502)(358.8423781,413.03709506)(359.12238403,412.94710297)
\curveto(359.40237754,412.85709524)(359.6473773,412.73209537)(359.85738403,412.57210297)
\curveto(359.93737701,412.51209559)(360.00737694,412.44209566)(360.06738403,412.36210297)
\curveto(360.13737681,412.28209582)(360.21237673,412.2020959)(360.29238403,412.12210297)
\curveto(360.31237663,412.102096)(360.3423766,412.07209603)(360.38238403,412.03210297)
\curveto(360.43237651,412.0020961)(360.48237646,411.9970961)(360.53238403,412.01710297)
\curveto(360.6423763,412.04709605)(360.7473762,412.11709598)(360.84738403,412.22710297)
\curveto(360.947376,412.34709575)(361.0423759,412.43709566)(361.13238403,412.49710297)
\curveto(361.27237567,412.60709549)(361.42237552,412.6970954)(361.58238403,412.76710297)
\curveto(361.7423752,412.84709525)(361.92237502,412.92209518)(362.12238403,412.99210297)
\curveto(362.20237474,413.01209509)(362.29737465,413.02709507)(362.40738403,413.03710297)
\curveto(362.52737442,413.05709504)(362.6473743,413.06709503)(362.76738403,413.06710297)
\curveto(362.89737405,413.07709502)(363.01737393,413.07709502)(363.12738403,413.06710297)
\curveto(363.2473737,413.05709504)(363.35237359,413.04209506)(363.44238403,413.02210297)
\curveto(363.49237345,413.01209509)(363.53737341,413.00709509)(363.57738403,413.00710297)
\curveto(363.61737333,413.00709509)(363.66237328,412.9970951)(363.71238403,412.97710297)
\curveto(363.85237309,412.93709516)(363.98737296,412.8970952)(364.11738403,412.85710297)
\curveto(364.2473727,412.81709528)(364.36737258,412.76209534)(364.47738403,412.69210297)
\curveto(364.89737205,412.43209567)(365.21237173,412.05209605)(365.42238403,411.55210297)
\curveto(365.46237148,411.46209664)(365.49237145,411.36709673)(365.51238403,411.26710297)
\curveto(365.53237141,411.17709692)(365.55237139,411.08709701)(365.57238403,410.99710297)
\curveto(365.58237136,410.92709717)(365.58737136,410.86209724)(365.58738403,410.80210297)
\curveto(365.59737135,410.74209736)(365.60737134,410.68209742)(365.61738403,410.62210297)
\lineto(365.61738403,410.47210297)
\curveto(365.62737132,410.41209769)(365.62737132,410.34209776)(365.61738403,410.26210297)
\curveto(365.61737133,410.18209792)(365.61737133,410.10709799)(365.61738403,410.03710297)
\lineto(365.61738403,409.16710297)
\lineto(365.61738403,406.24210297)
\curveto(365.61737133,406.16210194)(365.61737133,406.06710203)(365.61738403,405.95710297)
\curveto(365.62737132,405.85710224)(365.62737132,405.75710234)(365.61738403,405.65710297)
\curveto(365.61737133,405.56710253)(365.60737134,405.47710262)(365.58738403,405.38710297)
\curveto(365.56737138,405.30710279)(365.53737141,405.25210285)(365.49738403,405.22210297)
\curveto(365.43737151,405.17210293)(365.35737159,405.14210296)(365.25738403,405.13210297)
\lineto(364.95738403,405.13210297)
\lineto(364.16238403,405.13210297)
\curveto(364.02237292,405.13210297)(363.89737305,405.14210296)(363.78738403,405.16210297)
\curveto(363.67737327,405.18210292)(363.60237334,405.23710286)(363.56238403,405.32710297)
\curveto(363.53237341,405.3971027)(363.51737343,405.47210263)(363.51738403,405.55210297)
\curveto(363.51737343,405.64210246)(363.51737343,405.72710237)(363.51738403,405.80710297)
\lineto(363.51738403,406.64710297)
\lineto(363.51738403,408.67210297)
\lineto(363.51738403,409.30210297)
\curveto(363.51737343,409.35209875)(363.51737343,409.40709869)(363.51738403,409.46710297)
\curveto(363.52737342,409.52709857)(363.52237342,409.58209852)(363.50238403,409.63210297)
\lineto(363.50238403,409.75210297)
\curveto(363.50237344,409.81209829)(363.50237344,409.87209823)(363.50238403,409.93210297)
\curveto(363.50237344,409.99209811)(363.49737345,410.05209805)(363.48738403,410.11210297)
\curveto(363.47737347,410.15209795)(363.47237347,410.19209791)(363.47238403,410.23210297)
\curveto(363.47237347,410.28209782)(363.46737348,410.32709777)(363.45738403,410.36710297)
\curveto(363.41737353,410.51709758)(363.37237357,410.64709745)(363.32238403,410.75710297)
\curveto(363.28237366,410.87709722)(363.21737373,410.98209712)(363.12738403,411.07210297)
\curveto(362.98737396,411.21209689)(362.81737413,411.31209679)(362.61738403,411.37210297)
\curveto(362.57737437,411.38209672)(362.5423744,411.38209672)(362.51238403,411.37210297)
\curveto(362.48237446,411.37209673)(362.4473745,411.38209672)(362.40738403,411.40210297)
\curveto(362.36737458,411.41209669)(362.31737463,411.41709668)(362.25738403,411.41710297)
\curveto(362.20737474,411.42709667)(362.15737479,411.42709667)(362.10738403,411.41710297)
\curveto(362.0473749,411.3970967)(361.98737496,411.38709671)(361.92738403,411.38710297)
\curveto(361.86737508,411.38709671)(361.80737514,411.37709672)(361.74738403,411.35710297)
\curveto(361.45737549,411.25709684)(361.2473757,411.10709699)(361.11738403,410.90710297)
\curveto(360.947376,410.67709742)(360.8423761,410.38709771)(360.80238403,410.03710297)
\curveto(360.77237617,409.6970984)(360.75737619,409.32209878)(360.75738403,408.91210297)
\lineto(360.75738403,406.93210297)
\lineto(360.75738403,405.82210297)
\lineto(360.75738403,405.52210297)
\curveto(360.75737619,405.42210268)(360.73237621,405.34210276)(360.68238403,405.28210297)
\curveto(360.63237631,405.21210289)(360.55737639,405.16710293)(360.45738403,405.14710297)
\curveto(360.36737658,405.13710296)(360.26237668,405.13210297)(360.14238403,405.13210297)
\lineto(359.33238403,405.13210297)
\lineto(359.06238403,405.13210297)
\curveto(358.98237796,405.14210296)(358.91237803,405.15710294)(358.85238403,405.17710297)
\curveto(358.75237819,405.22710287)(358.69237825,405.30710279)(358.67238403,405.41710297)
\curveto(358.66237828,405.52710257)(358.65737829,405.65210245)(358.65738403,405.79210297)
\lineto(358.65738403,407.06710297)
\lineto(358.65738403,409.42210297)
\curveto(358.65737829,409.71209839)(358.6473783,409.98709811)(358.62738403,410.24710297)
\curveto(358.60737834,410.50709759)(358.5423784,410.72209738)(358.43238403,410.89210297)
\curveto(358.35237859,411.03209707)(358.2473787,411.13709696)(358.11738403,411.20710297)
\curveto(357.99737895,411.27709682)(357.8473791,411.33709676)(357.66738403,411.38710297)
\curveto(357.62737932,411.3970967)(357.58737936,411.3970967)(357.54738403,411.38710297)
\curveto(357.50737944,411.38709671)(357.46237948,411.39209671)(357.41238403,411.40210297)
\curveto(357.30237964,411.42209668)(357.19737975,411.41209669)(357.09738403,411.37210297)
\curveto(357.07737987,411.37209673)(357.05737989,411.36709673)(357.03738403,411.35710297)
\lineto(356.97738403,411.35710297)
\curveto(356.81738013,411.30709679)(356.66238028,411.22209688)(356.51238403,411.10210297)
\curveto(356.35238059,410.98209712)(356.22738072,410.84209726)(356.13738403,410.68210297)
\curveto(356.05738089,410.53209757)(355.99738095,410.35709774)(355.95738403,410.15710297)
\curveto(355.92738102,409.96709813)(355.90738104,409.75709834)(355.89738403,409.52710297)
\lineto(355.89738403,408.77710297)
\lineto(355.89738403,406.75210297)
\lineto(355.89738403,405.83710297)
\lineto(355.89738403,405.56710297)
\curveto(355.89738105,405.47710262)(355.88238106,405.3971027)(355.85238403,405.32710297)
\curveto(355.81238113,405.23710286)(355.73738121,405.18210292)(355.62738403,405.16210297)
\curveto(355.51738143,405.14210296)(355.39238155,405.13210297)(355.25238403,405.13210297)
\lineto(354.47238403,405.13210297)
\lineto(354.17238403,405.13210297)
\curveto(354.08238286,405.14210296)(354.00738294,405.16710293)(353.94738403,405.20710297)
\curveto(353.85738309,405.25710284)(353.80738314,405.34710275)(353.79738403,405.47710297)
\lineto(353.79738403,405.91210297)
\lineto(353.79738403,407.66710297)
\lineto(353.79738403,411.32710297)
\lineto(353.79738403,412.22710297)
\lineto(353.79738403,412.51210297)
\curveto(353.80738314,412.6020955)(353.83238311,412.67709542)(353.87238403,412.73710297)
\curveto(353.92238302,412.7970953)(354.00238294,412.83709526)(354.11238403,412.85710297)
\lineto(354.20238403,412.85710297)
\curveto(354.25238269,412.86709523)(354.30238264,412.87209523)(354.35238403,412.87210297)
\lineto(354.51738403,412.87210297)
\lineto(355.13238403,412.87210297)
\curveto(355.21238173,412.87209523)(355.28738166,412.86709523)(355.35738403,412.85710297)
\curveto(355.43738151,412.85709524)(355.50738144,412.84709525)(355.56738403,412.82710297)
\curveto(355.6473813,412.7970953)(355.69738125,412.74709535)(355.71738403,412.67710297)
\curveto(355.7473812,412.60709549)(355.77238117,412.52709557)(355.79238403,412.43710297)
\curveto(355.80238114,412.40709569)(355.80238114,412.37709572)(355.79238403,412.34710297)
\curveto(355.79238115,412.32709577)(355.80238114,412.30709579)(355.82238403,412.28710297)
\curveto(355.83238111,412.25709584)(355.8423811,412.23209587)(355.85238403,412.21210297)
\curveto(355.87238107,412.2020959)(355.89238105,412.18709591)(355.91238403,412.16710297)
\curveto(356.03238091,412.15709594)(356.13238081,412.19209591)(356.21238403,412.27210297)
\curveto(356.29238065,412.36209574)(356.36738058,412.43209567)(356.43738403,412.48210297)
\curveto(356.57738037,412.58209552)(356.71738023,412.67209543)(356.85738403,412.75210297)
\curveto(357.00737994,412.83209527)(357.16737978,412.8970952)(357.33738403,412.94710297)
\curveto(357.42737952,412.97709512)(357.51737943,412.9970951)(357.60738403,413.00710297)
\curveto(357.69737925,413.01709508)(357.79237915,413.03209507)(357.89238403,413.05210297)
\curveto(357.92237902,413.06209504)(357.96737898,413.06209504)(358.02738403,413.05210297)
\curveto(358.08737886,413.05209505)(358.12737882,413.05709504)(358.14738403,413.06710297)
}
}
{
\newrgbcolor{curcolor}{0 0 0}
\pscustom[linestyle=none,fillstyle=solid,fillcolor=curcolor]
{
\newpath
\moveto(374.33613403,405.73210297)
\curveto(374.35612618,405.62210248)(374.36612617,405.51210259)(374.36613403,405.40210297)
\curveto(374.37612616,405.29210281)(374.32612621,405.21710288)(374.21613403,405.17710297)
\curveto(374.15612638,405.14710295)(374.08612645,405.13210297)(374.00613403,405.13210297)
\lineto(373.76613403,405.13210297)
\lineto(372.95613403,405.13210297)
\lineto(372.68613403,405.13210297)
\curveto(372.60612793,405.14210296)(372.541128,405.16710293)(372.49113403,405.20710297)
\curveto(372.42112812,405.24710285)(372.36612817,405.3021028)(372.32613403,405.37210297)
\curveto(372.29612824,405.45210265)(372.25112829,405.51710258)(372.19113403,405.56710297)
\curveto(372.17112837,405.58710251)(372.14612839,405.6021025)(372.11613403,405.61210297)
\curveto(372.08612845,405.63210247)(372.04612849,405.63710246)(371.99613403,405.62710297)
\curveto(371.94612859,405.60710249)(371.89612864,405.58210252)(371.84613403,405.55210297)
\curveto(371.80612873,405.52210258)(371.76112878,405.4971026)(371.71113403,405.47710297)
\curveto(371.66112888,405.43710266)(371.60612893,405.4021027)(371.54613403,405.37210297)
\lineto(371.36613403,405.28210297)
\curveto(371.2361293,405.22210288)(371.10112944,405.17210293)(370.96113403,405.13210297)
\curveto(370.82112972,405.102103)(370.67612986,405.06710303)(370.52613403,405.02710297)
\curveto(370.45613008,405.00710309)(370.38613015,404.9971031)(370.31613403,404.99710297)
\curveto(370.25613028,404.98710311)(370.19113035,404.97710312)(370.12113403,404.96710297)
\lineto(370.03113403,404.96710297)
\curveto(370.00113054,404.95710314)(369.97113057,404.95210315)(369.94113403,404.95210297)
\lineto(369.77613403,404.95210297)
\curveto(369.67613086,404.93210317)(369.57613096,404.93210317)(369.47613403,404.95210297)
\lineto(369.34113403,404.95210297)
\curveto(369.27113127,404.97210313)(369.20113134,404.98210312)(369.13113403,404.98210297)
\curveto(369.07113147,404.97210313)(369.01113153,404.97710312)(368.95113403,404.99710297)
\curveto(368.85113169,405.01710308)(368.75613178,405.03710306)(368.66613403,405.05710297)
\curveto(368.57613196,405.06710303)(368.49113205,405.09210301)(368.41113403,405.13210297)
\curveto(368.12113242,405.24210286)(367.87113267,405.38210272)(367.66113403,405.55210297)
\curveto(367.46113308,405.73210237)(367.30113324,405.96710213)(367.18113403,406.25710297)
\curveto(367.15113339,406.32710177)(367.12113342,406.4021017)(367.09113403,406.48210297)
\curveto(367.07113347,406.56210154)(367.05113349,406.64710145)(367.03113403,406.73710297)
\curveto(367.01113353,406.78710131)(367.00113354,406.83710126)(367.00113403,406.88710297)
\curveto(367.01113353,406.93710116)(367.01113353,406.98710111)(367.00113403,407.03710297)
\curveto(366.99113355,407.06710103)(366.98113356,407.12710097)(366.97113403,407.21710297)
\curveto(366.97113357,407.31710078)(366.97613356,407.38710071)(366.98613403,407.42710297)
\curveto(367.00613353,407.52710057)(367.01613352,407.61210049)(367.01613403,407.68210297)
\lineto(367.10613403,408.01210297)
\curveto(367.1361334,408.13209997)(367.17613336,408.23709986)(367.22613403,408.32710297)
\curveto(367.39613314,408.61709948)(367.59113295,408.83709926)(367.81113403,408.98710297)
\curveto(368.03113251,409.13709896)(368.31113223,409.26709883)(368.65113403,409.37710297)
\curveto(368.78113176,409.42709867)(368.91613162,409.46209864)(369.05613403,409.48210297)
\curveto(369.19613134,409.5020986)(369.3361312,409.52709857)(369.47613403,409.55710297)
\curveto(369.55613098,409.57709852)(369.6411309,409.58709851)(369.73113403,409.58710297)
\curveto(369.82113072,409.5970985)(369.91113063,409.61209849)(370.00113403,409.63210297)
\curveto(370.07113047,409.65209845)(370.1411304,409.65709844)(370.21113403,409.64710297)
\curveto(370.28113026,409.64709845)(370.35613018,409.65709844)(370.43613403,409.67710297)
\curveto(370.50613003,409.6970984)(370.57612996,409.70709839)(370.64613403,409.70710297)
\curveto(370.71612982,409.70709839)(370.79112975,409.71709838)(370.87113403,409.73710297)
\curveto(371.08112946,409.78709831)(371.27112927,409.82709827)(371.44113403,409.85710297)
\curveto(371.62112892,409.8970982)(371.78112876,409.98709811)(371.92113403,410.12710297)
\curveto(372.01112853,410.21709788)(372.07112847,410.31709778)(372.10113403,410.42710297)
\curveto(372.11112843,410.45709764)(372.11112843,410.48209762)(372.10113403,410.50210297)
\curveto(372.10112844,410.52209758)(372.10612843,410.54209756)(372.11613403,410.56210297)
\curveto(372.12612841,410.58209752)(372.13112841,410.61209749)(372.13113403,410.65210297)
\lineto(372.13113403,410.74210297)
\lineto(372.10113403,410.86210297)
\curveto(372.10112844,410.9020972)(372.09612844,410.93709716)(372.08613403,410.96710297)
\curveto(371.98612855,411.26709683)(371.77612876,411.47209663)(371.45613403,411.58210297)
\curveto(371.36612917,411.61209649)(371.25612928,411.63209647)(371.12613403,411.64210297)
\curveto(371.00612953,411.66209644)(370.88112966,411.66709643)(370.75113403,411.65710297)
\curveto(370.62112992,411.65709644)(370.49613004,411.64709645)(370.37613403,411.62710297)
\curveto(370.25613028,411.60709649)(370.15113039,411.58209652)(370.06113403,411.55210297)
\curveto(370.00113054,411.53209657)(369.9411306,411.5020966)(369.88113403,411.46210297)
\curveto(369.83113071,411.43209667)(369.78113076,411.3970967)(369.73113403,411.35710297)
\curveto(369.68113086,411.31709678)(369.62613091,411.26209684)(369.56613403,411.19210297)
\curveto(369.51613102,411.12209698)(369.48113106,411.05709704)(369.46113403,410.99710297)
\curveto(369.41113113,410.8970972)(369.36613117,410.8020973)(369.32613403,410.71210297)
\curveto(369.29613124,410.62209748)(369.22613131,410.56209754)(369.11613403,410.53210297)
\curveto(369.0361315,410.51209759)(368.95113159,410.5020976)(368.86113403,410.50210297)
\lineto(368.59113403,410.50210297)
\lineto(368.02113403,410.50210297)
\curveto(367.97113257,410.5020976)(367.92113262,410.4970976)(367.87113403,410.48710297)
\curveto(367.82113272,410.48709761)(367.77613276,410.49209761)(367.73613403,410.50210297)
\lineto(367.60113403,410.50210297)
\curveto(367.58113296,410.51209759)(367.55613298,410.51709758)(367.52613403,410.51710297)
\curveto(367.49613304,410.51709758)(367.47113307,410.52709757)(367.45113403,410.54710297)
\curveto(367.37113317,410.56709753)(367.31613322,410.63209747)(367.28613403,410.74210297)
\curveto(367.27613326,410.79209731)(367.27613326,410.84209726)(367.28613403,410.89210297)
\curveto(367.29613324,410.94209716)(367.30613323,410.98709711)(367.31613403,411.02710297)
\curveto(367.34613319,411.13709696)(367.37613316,411.23709686)(367.40613403,411.32710297)
\curveto(367.44613309,411.42709667)(367.49113305,411.51709658)(367.54113403,411.59710297)
\lineto(367.63113403,411.74710297)
\lineto(367.72113403,411.89710297)
\curveto(367.80113274,412.00709609)(367.90113264,412.11209599)(368.02113403,412.21210297)
\curveto(368.0411325,412.22209588)(368.07113247,412.24709585)(368.11113403,412.28710297)
\curveto(368.16113238,412.32709577)(368.20613233,412.36209574)(368.24613403,412.39210297)
\curveto(368.28613225,412.42209568)(368.33113221,412.45209565)(368.38113403,412.48210297)
\curveto(368.55113199,412.59209551)(368.73113181,412.67709542)(368.92113403,412.73710297)
\curveto(369.11113143,412.80709529)(369.30613123,412.87209523)(369.50613403,412.93210297)
\curveto(369.62613091,412.96209514)(369.75113079,412.98209512)(369.88113403,412.99210297)
\curveto(370.01113053,413.0020951)(370.1411304,413.02209508)(370.27113403,413.05210297)
\curveto(370.31113023,413.06209504)(370.37113017,413.06209504)(370.45113403,413.05210297)
\curveto(370.54113,413.04209506)(370.59612994,413.04709505)(370.61613403,413.06710297)
\curveto(371.02612951,413.07709502)(371.41612912,413.06209504)(371.78613403,413.02210297)
\curveto(372.16612837,412.98209512)(372.50612803,412.90709519)(372.80613403,412.79710297)
\curveto(373.11612742,412.68709541)(373.38112716,412.53709556)(373.60113403,412.34710297)
\curveto(373.82112672,412.16709593)(373.99112655,411.93209617)(374.11113403,411.64210297)
\curveto(374.18112636,411.47209663)(374.22112632,411.27709682)(374.23113403,411.05710297)
\curveto(374.2411263,410.83709726)(374.24612629,410.61209749)(374.24613403,410.38210297)
\lineto(374.24613403,407.03710297)
\lineto(374.24613403,406.45210297)
\curveto(374.24612629,406.26210184)(374.26612627,406.08710201)(374.30613403,405.92710297)
\curveto(374.31612622,405.8971022)(374.32112622,405.86210224)(374.32113403,405.82210297)
\curveto(374.32112622,405.79210231)(374.32612621,405.76210234)(374.33613403,405.73210297)
\moveto(372.13113403,408.04210297)
\curveto(372.1411284,408.09210001)(372.14612839,408.14709995)(372.14613403,408.20710297)
\curveto(372.14612839,408.27709982)(372.1411284,408.33709976)(372.13113403,408.38710297)
\curveto(372.11112843,408.44709965)(372.10112844,408.5020996)(372.10113403,408.55210297)
\curveto(372.10112844,408.6020995)(372.08112846,408.64209946)(372.04113403,408.67210297)
\curveto(371.99112855,408.71209939)(371.91612862,408.73209937)(371.81613403,408.73210297)
\curveto(371.77612876,408.72209938)(371.7411288,408.71209939)(371.71113403,408.70210297)
\curveto(371.68112886,408.7020994)(371.64612889,408.6970994)(371.60613403,408.68710297)
\curveto(371.536129,408.66709943)(371.46112908,408.65209945)(371.38113403,408.64210297)
\curveto(371.30112924,408.63209947)(371.22112932,408.61709948)(371.14113403,408.59710297)
\curveto(371.11112943,408.58709951)(371.06612947,408.58209952)(371.00613403,408.58210297)
\curveto(370.87612966,408.55209955)(370.74612979,408.53209957)(370.61613403,408.52210297)
\curveto(370.48613005,408.51209959)(370.36113018,408.48709961)(370.24113403,408.44710297)
\curveto(370.16113038,408.42709967)(370.08613045,408.40709969)(370.01613403,408.38710297)
\curveto(369.94613059,408.37709972)(369.87613066,408.35709974)(369.80613403,408.32710297)
\curveto(369.59613094,408.23709986)(369.41613112,408.1021)(369.26613403,407.92210297)
\curveto(369.12613141,407.74210036)(369.07613146,407.49210061)(369.11613403,407.17210297)
\curveto(369.1361314,407.0021011)(369.19113135,406.86210124)(369.28113403,406.75210297)
\curveto(369.35113119,406.64210146)(369.45613108,406.55210155)(369.59613403,406.48210297)
\curveto(369.7361308,406.42210168)(369.88613065,406.37710172)(370.04613403,406.34710297)
\curveto(370.21613032,406.31710178)(370.39113015,406.30710179)(370.57113403,406.31710297)
\curveto(370.76112978,406.33710176)(370.9361296,406.37210173)(371.09613403,406.42210297)
\curveto(371.35612918,406.5021016)(371.56112898,406.62710147)(371.71113403,406.79710297)
\curveto(371.86112868,406.97710112)(371.97612856,407.1971009)(372.05613403,407.45710297)
\curveto(372.07612846,407.52710057)(372.08612845,407.5971005)(372.08613403,407.66710297)
\curveto(372.09612844,407.74710035)(372.11112843,407.82710027)(372.13113403,407.90710297)
\lineto(372.13113403,408.04210297)
}
}
{
\newrgbcolor{curcolor}{0 0 0}
\pscustom[linestyle=none,fillstyle=solid,fillcolor=curcolor]
{
\newpath
\moveto(15.91581787,173.85068481)
\curveto(15.99582565,173.85067413)(16.10582554,173.83567415)(16.24581787,173.80568481)
\curveto(16.37582527,173.7856742)(16.47582517,173.75567423)(16.54581787,173.71568481)
\curveto(16.61582503,173.6856743)(16.68082497,173.67067431)(16.74081787,173.67068481)
\curveto(16.80082485,173.67067431)(16.86582478,173.65067433)(16.93581787,173.61068481)
\curveto(16.99582465,173.5806744)(17.06082459,173.55067443)(17.13081787,173.52068481)
\curveto(17.19082446,173.50067448)(17.2508244,173.47567451)(17.31081787,173.44568481)
\curveto(17.39082426,173.41567457)(17.46582418,173.3806746)(17.53581787,173.34068481)
\curveto(17.60582404,173.30067468)(17.67582397,173.25567473)(17.74581787,173.20568481)
\curveto(17.77582387,173.1856748)(17.80582384,173.17067481)(17.83581787,173.16068481)
\curveto(17.85582379,173.15067483)(17.87582377,173.13067485)(17.89581787,173.10068481)
\curveto(18.09582355,172.95067503)(18.27582337,172.79567519)(18.43581787,172.63568481)
\curveto(18.47582317,172.60567538)(18.51582313,172.56567542)(18.55581787,172.51568481)
\curveto(18.59582305,172.46567552)(18.62582302,172.42067556)(18.64581787,172.38068481)
\curveto(18.66582298,172.34067564)(18.69582295,172.30067568)(18.73581787,172.26068481)
\curveto(18.76582288,172.23067575)(18.79082286,172.19567579)(18.81081787,172.15568481)
\lineto(18.99081787,171.81068481)
\curveto(19.07082258,171.6806763)(19.13082252,171.53567645)(19.17081787,171.37568481)
\curveto(19.21082244,171.22567676)(19.2508224,171.07067691)(19.29081787,170.91068481)
\curveto(19.31082234,170.82067716)(19.32582232,170.73567725)(19.33581787,170.65568481)
\curveto(19.33582231,170.57567741)(19.3458223,170.49567749)(19.36581787,170.41568481)
\curveto(19.37582227,170.37567761)(19.38082227,170.33567765)(19.38081787,170.29568481)
\curveto(19.37082228,170.26567772)(19.37082228,170.23567775)(19.38081787,170.20568481)
\curveto(19.39082226,170.14567784)(19.39082226,170.09567789)(19.38081787,170.05568481)
\curveto(19.37082228,170.01567797)(19.37082228,169.97067801)(19.38081787,169.92068481)
\lineto(19.38081787,167.65568481)
\lineto(19.38081787,167.16068481)
\curveto(19.37082228,166.99068099)(19.40082225,166.85068113)(19.47081787,166.74068481)
\curveto(19.5508221,166.62068136)(19.69582195,166.54068144)(19.90581787,166.50068481)
\curveto(20.11582153,166.46068152)(20.31082134,166.42568156)(20.49081787,166.39568481)
\lineto(22.69581787,165.94568481)
\curveto(22.82581882,165.92568206)(22.99081866,165.89568209)(23.19081787,165.85568481)
\curveto(23.38081827,165.82568216)(23.51581813,165.77568221)(23.59581787,165.70568481)
\curveto(23.65581799,165.65568233)(23.69581795,165.60068238)(23.71581787,165.54068481)
\curveto(23.72581792,165.49068249)(23.74081791,165.42068256)(23.76081787,165.33068481)
\lineto(23.76081787,165.06068481)
\curveto(23.76081789,164.91068307)(23.75581789,164.77568321)(23.74581787,164.65568481)
\curveto(23.73581791,164.54568344)(23.68581796,164.47068351)(23.59581787,164.43068481)
\curveto(23.53581811,164.41068357)(23.45581819,164.41068357)(23.35581787,164.43068481)
\curveto(23.2458184,164.45068353)(23.14081851,164.47068351)(23.04081787,164.49068481)
\lineto(13.93581787,166.30568481)
\curveto(13.82582782,166.32568166)(13.70582794,166.34568164)(13.57581787,166.36568481)
\curveto(13.43582821,166.39568159)(13.32582832,166.44068154)(13.24581787,166.50068481)
\curveto(13.18582846,166.56068142)(13.13582851,166.64568134)(13.09581787,166.75568481)
\curveto(13.08582856,166.7856812)(13.08582856,166.80568118)(13.09581787,166.81568481)
\curveto(13.09582855,166.83568115)(13.09082856,166.86068112)(13.08081787,166.89068481)
\lineto(13.08081787,170.29568481)
\curveto(13.08082857,170.67567731)(13.08582856,171.04567694)(13.09581787,171.40568481)
\curveto(13.09582855,171.76567622)(13.14082851,172.0856759)(13.23081787,172.36568481)
\curveto(13.38082827,172.7856752)(13.57582807,173.11067487)(13.81581787,173.34068481)
\curveto(14.05582759,173.57067441)(14.38582726,173.73567425)(14.80581787,173.83568481)
\curveto(14.91582673,173.85567413)(15.03082662,173.86567412)(15.15081787,173.86568481)
\curveto(15.27082638,173.87567411)(15.39582625,173.8856741)(15.52581787,173.89568481)
\curveto(15.59582605,173.90567408)(15.66082599,173.89567409)(15.72081787,173.86568481)
\curveto(15.78082587,173.84567414)(15.8458258,173.84067414)(15.91581787,173.85068481)
\moveto(16.45581787,172.33568481)
\curveto(16.31582533,172.39567559)(16.15582549,172.43067555)(15.97581787,172.44068481)
\curveto(15.78582586,172.45067553)(15.63582601,172.45067553)(15.52581787,172.44068481)
\curveto(15.2458264,172.40067558)(15.02582662,172.31067567)(14.86581787,172.17068481)
\curveto(14.69582695,172.04067594)(14.55582709,171.85567613)(14.44581787,171.61568481)
\curveto(14.35582729,171.41567657)(14.30082735,171.17567681)(14.28081787,170.89568481)
\curveto(14.26082739,170.61567737)(14.2508274,170.32067766)(14.25081787,170.01068481)
\lineto(14.25081787,168.07568481)
\curveto(14.26082739,168.05567993)(14.26582738,168.03067995)(14.26581787,168.00068481)
\curveto(14.26582738,167.98068)(14.27082738,167.95568003)(14.28081787,167.92568481)
\curveto(14.31082734,167.79568019)(14.37582727,167.70068028)(14.47581787,167.64068481)
\curveto(14.57582707,167.5806804)(14.71082694,167.53568045)(14.88081787,167.50568481)
\curveto(15.04082661,167.47568051)(15.19082646,167.45068053)(15.33081787,167.43068481)
\lineto(17.13081787,167.07068481)
\curveto(17.28082437,167.04068094)(17.4458242,167.00068098)(17.62581787,166.95068481)
\curveto(17.80582384,166.91068107)(17.9458237,166.91068107)(18.04581787,166.95068481)
\curveto(18.12582352,166.99068099)(18.17582347,167.06068092)(18.19581787,167.16068481)
\curveto(18.20582344,167.26068072)(18.21082344,167.37568061)(18.21081787,167.50568481)
\lineto(18.21081787,169.53068481)
\lineto(18.21081787,169.99568481)
\curveto(18.20082345,170.15567783)(18.18082347,170.30067768)(18.15081787,170.43068481)
\curveto(18.08082357,170.71067727)(18.00082365,170.96067702)(17.91081787,171.18068481)
\curveto(17.81082384,171.41067657)(17.66082399,171.61567637)(17.46081787,171.79568481)
\curveto(17.34082431,171.90567608)(17.21582443,171.99567599)(17.08581787,172.06568481)
\curveto(16.95582469,172.13567585)(16.81082484,172.20567578)(16.65081787,172.27568481)
\curveto(16.61082504,172.29567569)(16.5458251,172.31567567)(16.45581787,172.33568481)
}
}
{
\newrgbcolor{curcolor}{0 0 0}
\pscustom[linestyle=none,fillstyle=solid,fillcolor=curcolor]
{
\newpath
\moveto(16.03581787,175.82224731)
\lineto(16.03581787,176.25724731)
\curveto(16.03582561,176.4072438)(16.07582557,176.5072437)(16.15581787,176.55724731)
\curveto(16.23582541,176.58724362)(16.33582531,176.59224362)(16.45581787,176.57224731)
\lineto(16.81581787,176.51224731)
\lineto(18.24081787,176.22724731)
\lineto(20.50581787,175.77724731)
\curveto(20.72582092,175.72724448)(20.95582069,175.67724453)(21.19581787,175.62724731)
\curveto(21.42582022,175.58724462)(21.62582002,175.57224464)(21.79581787,175.58224731)
\curveto(22.2458194,175.65224456)(22.56081909,175.89224432)(22.74081787,176.30224731)
\curveto(22.83081882,176.50224371)(22.86581878,176.75724345)(22.84581787,177.06724731)
\curveto(22.81581883,177.38724282)(22.76081889,177.65224256)(22.68081787,177.86224731)
\curveto(22.54081911,178.212242)(22.36581928,178.5072417)(22.15581787,178.74724731)
\curveto(21.93581971,178.98724122)(21.65082,179.19724101)(21.30081787,179.37724731)
\curveto(21.22082043,179.42724078)(21.14082051,179.46224075)(21.06081787,179.48224731)
\curveto(20.98082067,179.5122407)(20.89582075,179.54724066)(20.80581787,179.58724731)
\curveto(20.75582089,179.6072406)(20.71082094,179.61724059)(20.67081787,179.61724731)
\curveto(20.63082102,179.61724059)(20.58582106,179.63224058)(20.53581787,179.66224731)
\lineto(20.22081787,179.72224731)
\curveto(20.14082151,179.76224045)(20.0508216,179.78724042)(19.95081787,179.79724731)
\curveto(19.84082181,179.8072404)(19.74082191,179.82224039)(19.65081787,179.84224731)
\lineto(18.48081787,180.08224731)
\lineto(16.89081787,180.39724731)
\curveto(16.77082488,180.41723979)(16.645825,180.43723977)(16.51581787,180.45724731)
\curveto(16.37582527,180.48723972)(16.26582538,180.53223968)(16.18581787,180.59224731)
\curveto(16.13582551,180.64223957)(16.10582554,180.69723951)(16.09581787,180.75724731)
\curveto(16.07582557,180.81723939)(16.05582559,180.88723932)(16.03581787,180.96724731)
\lineto(16.03581787,181.19224731)
\curveto(16.03582561,181.3122389)(16.04082561,181.41723879)(16.05081787,181.50724731)
\curveto(16.06082559,181.6072386)(16.10582554,181.67223854)(16.18581787,181.70224731)
\curveto(16.23582541,181.74223847)(16.31082534,181.75223846)(16.41081787,181.73224731)
\curveto(16.50082515,181.7122385)(16.59582505,181.69223852)(16.69581787,181.67224731)
\lineto(17.71581787,181.47724731)
\lineto(21.75081787,180.66724731)
\lineto(23.10081787,180.39724731)
\curveto(23.22081843,180.37723983)(23.33581831,180.35223986)(23.44581787,180.32224731)
\curveto(23.5458181,180.29223992)(23.62081803,180.23723997)(23.67081787,180.15724731)
\curveto(23.70081795,180.11724009)(23.72581792,180.05224016)(23.74581787,179.96224731)
\curveto(23.75581789,179.88224033)(23.76581788,179.79224042)(23.77581787,179.69224731)
\curveto(23.77581787,179.60224061)(23.77081788,179.5122407)(23.76081787,179.42224731)
\curveto(23.7508179,179.34224087)(23.73081792,179.28724092)(23.70081787,179.25724731)
\curveto(23.66081799,179.21724099)(23.59581805,179.18724102)(23.50581787,179.16724731)
\curveto(23.46581818,179.15724105)(23.41081824,179.15724105)(23.34081787,179.16724731)
\curveto(23.27081838,179.17724103)(23.20581844,179.18224103)(23.14581787,179.18224731)
\curveto(23.07581857,179.19224102)(23.02081863,179.18224103)(22.98081787,179.15224731)
\curveto(22.94081871,179.13224108)(22.92581872,179.09224112)(22.93581787,179.03224731)
\curveto(22.95581869,178.95224126)(23.01581863,178.86224135)(23.11581787,178.76224731)
\curveto(23.20581844,178.66224155)(23.27581837,178.57224164)(23.32581787,178.49224731)
\curveto(23.48581816,178.24224197)(23.62581802,177.96224225)(23.74581787,177.65224731)
\curveto(23.79581785,177.53224268)(23.82581782,177.4122428)(23.83581787,177.29224731)
\curveto(23.85581779,177.18224303)(23.88081777,177.06224315)(23.91081787,176.93224731)
\curveto(23.92081773,176.88224333)(23.92081773,176.82724338)(23.91081787,176.76724731)
\curveto(23.90081775,176.71724349)(23.90581774,176.66724354)(23.92581787,176.61724731)
\curveto(23.9458177,176.51724369)(23.9458177,176.42724378)(23.92581787,176.34724731)
\lineto(23.92581787,176.19724731)
\curveto(23.90581774,176.14724406)(23.89581775,176.08724412)(23.89581787,176.01724731)
\curveto(23.89581775,175.95724425)(23.89081776,175.9072443)(23.88081787,175.86724731)
\curveto(23.86081779,175.82724438)(23.8508178,175.78724442)(23.85081787,175.74724731)
\curveto(23.86081779,175.71724449)(23.85581779,175.67724453)(23.83581787,175.62724731)
\curveto(23.81581783,175.55724465)(23.79581785,175.48224473)(23.77581787,175.40224731)
\curveto(23.75581789,175.33224488)(23.72581792,175.26224495)(23.68581787,175.19224731)
\curveto(23.57581807,174.95224526)(23.43081822,174.76224545)(23.25081787,174.62224731)
\curveto(23.06081859,174.49224572)(22.83581881,174.39724581)(22.57581787,174.33724731)
\curveto(22.48581916,174.31724589)(22.39581925,174.3072459)(22.30581787,174.30724731)
\lineto(22.00581787,174.30724731)
\curveto(21.9458197,174.29724591)(21.89081976,174.29724591)(21.84081787,174.30724731)
\curveto(21.78081987,174.32724588)(21.71581993,174.33224588)(21.64581787,174.32224731)
\lineto(21.57081787,174.32224731)
\curveto(21.53082012,174.33224588)(21.49582015,174.33724587)(21.46581787,174.33724731)
\lineto(21.31581787,174.36724731)
\curveto(21.27582037,174.36724584)(21.23082042,174.37224584)(21.18081787,174.38224731)
\curveto(21.12082053,174.40224581)(21.06582058,174.41724579)(21.01581787,174.42724731)
\lineto(20.41581787,174.54724731)
\lineto(17.65581787,175.10224731)
\lineto(16.69581787,175.28224731)
\lineto(16.42581787,175.34224731)
\curveto(16.33582531,175.36224485)(16.26082539,175.39724481)(16.20081787,175.44724731)
\curveto(16.13082552,175.49724471)(16.08082557,175.58224463)(16.05081787,175.70224731)
\curveto(16.04082561,175.72224449)(16.04082561,175.74224447)(16.05081787,175.76224731)
\curveto(16.0508256,175.78224443)(16.0458256,175.80224441)(16.03581787,175.82224731)
}
}
{
\newrgbcolor{curcolor}{0 0 0}
\pscustom[linestyle=none,fillstyle=solid,fillcolor=curcolor]
{
\newpath
\moveto(15.88581787,187.42685669)
\curveto(15.86582578,188.06684987)(15.9508257,188.55684938)(16.14081787,188.89685669)
\curveto(16.33082532,189.2368487)(16.61582503,189.48184845)(16.99581787,189.63185669)
\curveto(17.09582455,189.67184826)(17.20582444,189.69684824)(17.32581787,189.70685669)
\curveto(17.43582421,189.72684821)(17.5508241,189.7368482)(17.67081787,189.73685669)
\curveto(17.86082379,189.75684818)(18.06582358,189.74684819)(18.28581787,189.70685669)
\curveto(18.50582314,189.67684826)(18.73082292,189.6368483)(18.96081787,189.58685669)
\lineto(20.56581787,189.27185669)
\lineto(22.90581787,188.80685669)
\lineto(23.41581787,188.68685669)
\curveto(23.58581806,188.64684929)(23.69581795,188.55684938)(23.74581787,188.41685669)
\curveto(23.76581788,188.36684957)(23.77581787,188.31184962)(23.77581787,188.25185669)
\curveto(23.78581786,188.20184973)(23.79081786,188.14684979)(23.79081787,188.08685669)
\curveto(23.79081786,187.95684998)(23.78581786,187.8318501)(23.77581787,187.71185669)
\curveto(23.77581787,187.59185034)(23.73581791,187.51685042)(23.65581787,187.48685669)
\curveto(23.58581806,187.44685049)(23.49581815,187.4368505)(23.38581787,187.45685669)
\curveto(23.27581837,187.47685046)(23.16581848,187.50185043)(23.05581787,187.53185669)
\lineto(21.76581787,187.78685669)
\lineto(19.32081787,188.26685669)
\curveto(19.0508226,188.32684961)(18.78582286,188.37684956)(18.52581787,188.41685669)
\curveto(18.25582339,188.45684948)(18.02582362,188.45684948)(17.83581787,188.41685669)
\curveto(17.63582401,188.37684956)(17.47582417,188.28684965)(17.35581787,188.14685669)
\curveto(17.22582442,188.01684992)(17.12582452,187.85685008)(17.05581787,187.66685669)
\curveto(17.03582461,187.60685033)(17.02582462,187.54185039)(17.02581787,187.47185669)
\curveto(17.01582463,187.41185052)(17.00082465,187.35685058)(16.98081787,187.30685669)
\curveto(16.97082468,187.25685068)(16.97082468,187.17685076)(16.98081787,187.06685669)
\curveto(16.98082467,186.96685097)(16.98582466,186.89185104)(16.99581787,186.84185669)
\curveto(17.01582463,186.80185113)(17.02582462,186.76685117)(17.02581787,186.73685669)
\curveto(17.01582463,186.70685123)(17.01582463,186.67185126)(17.02581787,186.63185669)
\curveto(17.05582459,186.49185144)(17.09082456,186.36185157)(17.13081787,186.24185669)
\curveto(17.16082449,186.12185181)(17.20582444,186.00685193)(17.26581787,185.89685669)
\curveto(17.28582436,185.84685209)(17.30582434,185.80685213)(17.32581787,185.77685669)
\curveto(17.3458243,185.74685219)(17.36582428,185.70685223)(17.38581787,185.65685669)
\curveto(17.63582401,185.25685268)(18.01082364,184.92685301)(18.51081787,184.66685669)
\curveto(18.59082306,184.62685331)(18.67582297,184.59185334)(18.76581787,184.56185669)
\lineto(19.00581787,184.47185669)
\curveto(19.05582259,184.44185349)(19.10582254,184.42685351)(19.15581787,184.42685669)
\curveto(19.19582245,184.42685351)(19.23582241,184.41185352)(19.27581787,184.38185669)
\lineto(19.59081787,184.32185669)
\curveto(19.62082203,184.30185363)(19.65582199,184.29185364)(19.69581787,184.29185669)
\curveto(19.73582191,184.29185364)(19.78082187,184.28685365)(19.83081787,184.27685669)
\lineto(20.28081787,184.18685669)
\lineto(21.72081787,183.88685669)
\lineto(23.04081787,183.63185669)
\curveto(23.1508185,183.61185432)(23.26581838,183.58685435)(23.38581787,183.55685669)
\curveto(23.49581815,183.5368544)(23.58581806,183.49685444)(23.65581787,183.43685669)
\curveto(23.73581791,183.36685457)(23.77581787,183.26685467)(23.77581787,183.13685669)
\curveto(23.78581786,183.01685492)(23.79081786,182.89185504)(23.79081787,182.76185669)
\curveto(23.79081786,182.68185525)(23.78581786,182.60685533)(23.77581787,182.53685669)
\curveto(23.76581788,182.46685547)(23.74081791,182.41185552)(23.70081787,182.37185669)
\curveto(23.650818,182.30185563)(23.55581809,182.28185565)(23.41581787,182.31185669)
\curveto(23.27581837,182.34185559)(23.14081851,182.36685557)(23.01081787,182.38685669)
\lineto(21.24081787,182.74685669)
\lineto(17.61081787,183.46685669)
\lineto(16.69581787,183.64685669)
\lineto(16.42581787,183.70685669)
\curveto(16.33582531,183.72685421)(16.26582538,183.76185417)(16.21581787,183.81185669)
\curveto(16.15582549,183.85185408)(16.11582553,183.90685403)(16.09581787,183.97685669)
\curveto(16.08582556,184.02685391)(16.07582557,184.08685385)(16.06581787,184.15685669)
\curveto(16.05582559,184.2368537)(16.0508256,184.31685362)(16.05081787,184.39685669)
\curveto(16.0508256,184.47685346)(16.05582559,184.55185338)(16.06581787,184.62185669)
\curveto(16.07582557,184.70185323)(16.09082556,184.75185318)(16.11081787,184.77185669)
\curveto(16.18082547,184.87185306)(16.27082538,184.90685303)(16.38081787,184.87685669)
\curveto(16.48082517,184.84685309)(16.59582505,184.8368531)(16.72581787,184.84685669)
\curveto(16.78582486,184.85685308)(16.83582481,184.88685305)(16.87581787,184.93685669)
\curveto(16.88582476,185.05685288)(16.84082481,185.16185277)(16.74081787,185.25185669)
\curveto(16.64082501,185.35185258)(16.56082509,185.44685249)(16.50081787,185.53685669)
\curveto(16.40082525,185.69685224)(16.31082534,185.85685208)(16.23081787,186.01685669)
\curveto(16.14082551,186.17685176)(16.06582558,186.36185157)(16.00581787,186.57185669)
\curveto(15.97582567,186.65185128)(15.95582569,186.74185119)(15.94581787,186.84185669)
\curveto(15.93582571,186.94185099)(15.92082573,187.0368509)(15.90081787,187.12685669)
\curveto(15.89082576,187.17685076)(15.88582576,187.22685071)(15.88581787,187.27685669)
\lineto(15.88581787,187.42685669)
}
}
{
\newrgbcolor{curcolor}{0 0 0}
\pscustom[linestyle=none,fillstyle=solid,fillcolor=curcolor]
{
\newpath
\moveto(13.69581787,193.46646606)
\curveto(13.69582795,193.61646204)(13.70082795,193.76646189)(13.71081787,193.91646606)
\curveto(13.71082794,194.06646159)(13.7508279,194.16646149)(13.83081787,194.21646606)
\curveto(13.89082776,194.24646141)(13.97582767,194.25146141)(14.08581787,194.23146606)
\curveto(14.18582746,194.22146144)(14.29082736,194.20646145)(14.40081787,194.18646606)
\lineto(15.27081787,194.00646606)
\curveto(15.3508263,193.99646166)(15.43582621,193.97646168)(15.52581787,193.94646606)
\curveto(15.60582604,193.92646173)(15.67582597,193.92146174)(15.73581787,193.93146606)
\curveto(15.87582577,193.94146172)(15.96582568,194.01646164)(16.00581787,194.15646606)
\curveto(16.01582563,194.19646146)(16.02082563,194.23646142)(16.02081787,194.27646606)
\lineto(16.02081787,194.42646606)
\lineto(16.02081787,194.83146606)
\curveto(16.01082564,195.00146066)(16.02082563,195.11646054)(16.05081787,195.17646606)
\curveto(16.11082554,195.2564604)(16.17082548,195.30646035)(16.23081787,195.32646606)
\curveto(16.27082538,195.33646032)(16.31582533,195.33646032)(16.36581787,195.32646606)
\lineto(16.51581787,195.29646606)
\curveto(16.62582502,195.27646038)(16.73082492,195.25146041)(16.83081787,195.22146606)
\curveto(16.92082473,195.19146047)(16.99082466,195.14146052)(17.04081787,195.07146606)
\curveto(17.09082456,195.00146066)(17.12082453,194.91146075)(17.13081787,194.80146606)
\lineto(17.13081787,194.47146606)
\curveto(17.12082453,194.3614613)(17.11582453,194.25146141)(17.11581787,194.14146606)
\curveto(17.11582453,194.03146163)(17.13082452,193.93146173)(17.16081787,193.84146606)
\curveto(17.19082446,193.77146189)(17.24082441,193.71146195)(17.31081787,193.66146606)
\curveto(17.38082427,193.62146204)(17.46582418,193.58646207)(17.56581787,193.55646606)
\curveto(17.65582399,193.52646213)(17.75582389,193.50146216)(17.86581787,193.48146606)
\curveto(17.96582368,193.47146219)(18.06582358,193.4564622)(18.16581787,193.43646606)
\lineto(21.13581787,192.83646606)
\curveto(21.35582029,192.79646286)(21.59082006,192.74646291)(21.84081787,192.68646606)
\curveto(22.08081957,192.63646302)(22.26581938,192.64146302)(22.39581787,192.70146606)
\curveto(22.47581917,192.74146292)(22.53081912,192.79646286)(22.56081787,192.86646606)
\curveto(22.59081906,192.94646271)(22.61581903,193.03646262)(22.63581787,193.13646606)
\curveto(22.645819,193.16646249)(22.650819,193.19646246)(22.65081787,193.22646606)
\curveto(22.64081901,193.26646239)(22.64081901,193.30146236)(22.65081787,193.33146606)
\lineto(22.65081787,193.52646606)
\curveto(22.650819,193.62646203)(22.66081899,193.71646194)(22.68081787,193.79646606)
\curveto(22.69081896,193.87646178)(22.72581892,193.93146173)(22.78581787,193.96146606)
\curveto(22.81581883,193.98146168)(22.87081878,193.99146167)(22.95081787,193.99146606)
\curveto(23.02081863,193.99146167)(23.09581855,193.98146168)(23.17581787,193.96146606)
\curveto(23.25581839,193.95146171)(23.33581831,193.93146173)(23.41581787,193.90146606)
\curveto(23.48581816,193.88146178)(23.54081811,193.8564618)(23.58081787,193.82646606)
\curveto(23.650818,193.76646189)(23.70081795,193.68146198)(23.73081787,193.57146606)
\curveto(23.7508179,193.48146218)(23.76081789,193.38646227)(23.76081787,193.28646606)
\curveto(23.7508179,193.18646247)(23.7458179,193.09646256)(23.74581787,193.01646606)
\curveto(23.7458179,192.9564627)(23.7508179,192.89646276)(23.76081787,192.83646606)
\curveto(23.76081789,192.77646288)(23.75581789,192.72146294)(23.74581787,192.67146606)
\lineto(23.74581787,192.49146606)
\curveto(23.73581791,192.45146321)(23.73081792,192.40646325)(23.73081787,192.35646606)
\curveto(23.72081793,192.31646334)(23.71581793,192.27146339)(23.71581787,192.22146606)
\curveto(23.66581798,192.03146363)(23.61081804,191.86646379)(23.55081787,191.72646606)
\curveto(23.49081816,191.59646406)(23.38581826,191.49646416)(23.23581787,191.42646606)
\curveto(23.03581861,191.32646433)(22.78581886,191.29646436)(22.48581787,191.33646606)
\curveto(22.17581947,191.37646428)(21.8458198,191.43146423)(21.49581787,191.50146606)
\lineto(17.56581787,192.29646606)
\curveto(17.43582421,192.28646337)(17.34082431,192.27646338)(17.28081787,192.26646606)
\curveto(17.22082443,192.2564634)(17.17082448,192.19646346)(17.13081787,192.08646606)
\curveto(17.12082453,192.04646361)(17.12082453,192.00146366)(17.13081787,191.95146606)
\curveto(17.14082451,191.91146375)(17.13582451,191.87646378)(17.11581787,191.84646606)
\lineto(17.11581787,191.60646606)
\curveto(17.11582453,191.47646418)(17.10582454,191.37146429)(17.08581787,191.29146606)
\curveto(17.05582459,191.21146445)(16.99582465,191.16646449)(16.90581787,191.15646606)
\curveto(16.86582478,191.13646452)(16.82082483,191.12646453)(16.77081787,191.12646606)
\lineto(16.62081787,191.15646606)
\curveto(16.48082517,191.18646447)(16.36582528,191.22146444)(16.27581787,191.26146606)
\curveto(16.17582547,191.30146436)(16.10082555,191.37646428)(16.05081787,191.48646606)
\curveto(16.01082564,191.60646405)(16.00082565,191.75146391)(16.02081787,191.92146606)
\curveto(16.04082561,192.09146357)(16.03082562,192.24146342)(15.99081787,192.37146606)
\curveto(15.94082571,192.47146319)(15.87082578,192.5564631)(15.78081787,192.62646606)
\curveto(15.72082593,192.656463)(15.645826,192.67646298)(15.55581787,192.68646606)
\curveto(15.46582618,192.70646295)(15.38082627,192.72646293)(15.30081787,192.74646606)
\lineto(14.37081787,192.92646606)
\curveto(14.29082736,192.94646271)(14.21082744,192.9614627)(14.13081787,192.97146606)
\curveto(14.04082761,192.99146267)(13.96582768,193.02146264)(13.90581787,193.06146606)
\curveto(13.82582782,193.11146255)(13.76082789,193.19646246)(13.71081787,193.31646606)
\curveto(13.71082794,193.34646231)(13.71082794,193.37146229)(13.71081787,193.39146606)
\curveto(13.70082795,193.42146224)(13.69582795,193.44646221)(13.69581787,193.46646606)
}
}
{
\newrgbcolor{curcolor}{0 0 0}
\pscustom[linestyle=none,fillstyle=solid,fillcolor=curcolor]
{
\newpath
\moveto(23.20581787,202.23326294)
\curveto(23.36581828,202.22325503)(23.50081815,202.17825507)(23.61081787,202.09826294)
\curveto(23.71081794,202.01825523)(23.78581786,201.92325533)(23.83581787,201.81326294)
\curveto(23.85581779,201.76325549)(23.86581778,201.70825554)(23.86581787,201.64826294)
\curveto(23.86581778,201.59825565)(23.87581777,201.53825571)(23.89581787,201.46826294)
\curveto(23.9458177,201.23825601)(23.93081772,201.02325623)(23.85081787,200.82326294)
\curveto(23.78081787,200.62325663)(23.69081796,200.49825675)(23.58081787,200.44826294)
\curveto(23.51081814,200.40825684)(23.43081822,200.37825687)(23.34081787,200.35826294)
\curveto(23.24081841,200.33825691)(23.16081849,200.30325695)(23.10081787,200.25326294)
\lineto(23.04081787,200.19326294)
\curveto(23.02081863,200.17325708)(23.01581863,200.14325711)(23.02581787,200.10326294)
\curveto(23.05581859,199.98325727)(23.11081854,199.86825738)(23.19081787,199.75826294)
\curveto(23.27081838,199.6482576)(23.34081831,199.54325771)(23.40081787,199.44326294)
\curveto(23.48081817,199.29325796)(23.55581809,199.13825811)(23.62581787,198.97826294)
\curveto(23.68581796,198.81825843)(23.74081791,198.6482586)(23.79081787,198.46826294)
\curveto(23.82081783,198.35825889)(23.84081781,198.24325901)(23.85081787,198.12326294)
\curveto(23.86081779,198.01325924)(23.87581777,197.89825935)(23.89581787,197.77826294)
\curveto(23.90581774,197.72825952)(23.91081774,197.68325957)(23.91081787,197.64326294)
\lineto(23.91081787,197.53826294)
\curveto(23.93081772,197.42825982)(23.93081772,197.32325993)(23.91081787,197.22326294)
\lineto(23.91081787,197.08826294)
\curveto(23.90081775,197.03826021)(23.89581775,196.98826026)(23.89581787,196.93826294)
\curveto(23.89581775,196.88826036)(23.88581776,196.8482604)(23.86581787,196.81826294)
\curveto(23.85581779,196.77826047)(23.8508178,196.74326051)(23.85081787,196.71326294)
\curveto(23.86081779,196.69326056)(23.86081779,196.66826058)(23.85081787,196.63826294)
\lineto(23.79081787,196.39826294)
\curveto(23.78081787,196.32826092)(23.76081789,196.26326099)(23.73081787,196.20326294)
\curveto(23.60081805,195.92326133)(23.45581819,195.70826154)(23.29581787,195.55826294)
\curveto(23.12581852,195.40826184)(22.89081876,195.30326195)(22.59081787,195.24326294)
\curveto(22.37081928,195.19326206)(22.10581954,195.19826205)(21.79581787,195.25826294)
\lineto(21.48081787,195.33326294)
\curveto(21.43082022,195.3532619)(21.38082027,195.36826188)(21.33081787,195.37826294)
\lineto(21.15081787,195.43826294)
\lineto(20.82081787,195.61826294)
\curveto(20.71082094,195.68826156)(20.61082104,195.75826149)(20.52081787,195.82826294)
\curveto(20.23082142,196.06826118)(20.01582163,196.35826089)(19.87581787,196.69826294)
\curveto(19.73582191,197.03826021)(19.61082204,197.40325985)(19.50081787,197.79326294)
\curveto(19.46082219,197.94325931)(19.43082222,198.09325916)(19.41081787,198.24326294)
\curveto(19.39082226,198.40325885)(19.36582228,198.55825869)(19.33581787,198.70826294)
\curveto(19.31582233,198.78825846)(19.30582234,198.85825839)(19.30581787,198.91826294)
\curveto(19.30582234,198.98825826)(19.29582235,199.06325819)(19.27581787,199.14326294)
\curveto(19.25582239,199.21325804)(19.2458224,199.28325797)(19.24581787,199.35326294)
\curveto(19.23582241,199.43325782)(19.22082243,199.51325774)(19.20081787,199.59326294)
\curveto(19.14082251,199.8532574)(19.09082256,200.09825715)(19.05081787,200.32826294)
\curveto(19.00082265,200.55825669)(18.88582276,200.75825649)(18.70581787,200.92826294)
\curveto(18.62582302,200.99825625)(18.52582312,201.06325619)(18.40581787,201.12326294)
\curveto(18.27582337,201.19325606)(18.13582351,201.22325603)(17.98581787,201.21326294)
\curveto(17.7458239,201.20325605)(17.55582409,201.1532561)(17.41581787,201.06326294)
\curveto(17.27582437,200.98325627)(17.16582448,200.84325641)(17.08581787,200.64326294)
\curveto(17.03582461,200.53325672)(17.00082465,200.39825685)(16.98081787,200.23826294)
\curveto(16.96082469,200.07825717)(16.9508247,199.90825734)(16.95081787,199.72826294)
\curveto(16.9508247,199.5482577)(16.96082469,199.36825788)(16.98081787,199.18826294)
\curveto(17.00082465,199.01825823)(17.03082462,198.86825838)(17.07081787,198.73826294)
\curveto(17.13082452,198.55825869)(17.21582443,198.37825887)(17.32581787,198.19826294)
\curveto(17.38582426,198.10825914)(17.46582418,198.01825923)(17.56581787,197.92826294)
\curveto(17.65582399,197.8482594)(17.75582389,197.77325948)(17.86581787,197.70326294)
\curveto(17.9458237,197.6532596)(18.03082362,197.60825964)(18.12081787,197.56826294)
\curveto(18.21082344,197.52825972)(18.28082337,197.46825978)(18.33081787,197.38826294)
\curveto(18.36082329,197.33825991)(18.38582326,197.26325999)(18.40581787,197.16326294)
\curveto(18.41582323,197.06326019)(18.42082323,196.96326029)(18.42081787,196.86326294)
\curveto(18.42082323,196.76326049)(18.41582323,196.66826058)(18.40581787,196.57826294)
\curveto(18.38582326,196.48826076)(18.36082329,196.42826082)(18.33081787,196.39826294)
\curveto(18.30082335,196.35826089)(18.2508234,196.33326092)(18.18081787,196.32326294)
\curveto(18.11082354,196.32326093)(18.03582361,196.34326091)(17.95581787,196.38326294)
\curveto(17.82582382,196.43326082)(17.70582394,196.48826076)(17.59581787,196.54826294)
\curveto(17.47582417,196.60826064)(17.36082429,196.67326058)(17.25081787,196.74326294)
\curveto(16.90082475,197.00326025)(16.63082502,197.29825995)(16.44081787,197.62826294)
\curveto(16.24082541,197.95825929)(16.08082557,198.3482589)(15.96081787,198.79826294)
\curveto(15.94082571,198.90825834)(15.92582572,199.01325824)(15.91581787,199.11326294)
\curveto(15.90582574,199.22325803)(15.89082576,199.33325792)(15.87081787,199.44326294)
\curveto(15.86082579,199.49325776)(15.86082579,199.55825769)(15.87081787,199.63826294)
\curveto(15.87082578,199.72825752)(15.86082579,199.78825746)(15.84081787,199.81826294)
\curveto(15.83082582,200.51825673)(15.91082574,201.10825614)(16.08081787,201.58826294)
\curveto(16.2508254,202.07825517)(16.57582507,202.38325487)(17.05581787,202.50326294)
\curveto(17.25582439,202.5532547)(17.49082416,202.55825469)(17.76081787,202.51826294)
\curveto(18.02082363,202.47825477)(18.29582335,202.42825482)(18.58581787,202.36826294)
\lineto(21.90081787,201.70826294)
\curveto(22.04081961,201.67825557)(22.17581947,201.6532556)(22.30581787,201.63326294)
\curveto(22.43581921,201.62325563)(22.54081911,201.63325562)(22.62081787,201.66326294)
\curveto(22.69081896,201.70325555)(22.74081891,201.75825549)(22.77081787,201.82826294)
\curveto(22.81081884,201.91825533)(22.84081881,201.99825525)(22.86081787,202.06826294)
\curveto(22.87081878,202.1482551)(22.91581873,202.19825505)(22.99581787,202.21826294)
\curveto(23.02581862,202.23825501)(23.05581859,202.24325501)(23.08581787,202.23326294)
\lineto(23.20581787,202.23326294)
\moveto(21.54081787,200.41826294)
\curveto(21.40082025,200.50825674)(21.24082041,200.57325668)(21.06081787,200.61326294)
\curveto(20.87082078,200.6532566)(20.67582097,200.69325656)(20.47581787,200.73326294)
\curveto(20.36582128,200.7532565)(20.26582138,200.76825648)(20.17581787,200.77826294)
\curveto(20.08582156,200.78825646)(20.01582163,200.76325649)(19.96581787,200.70326294)
\curveto(19.9458217,200.67325658)(19.93582171,200.60325665)(19.93581787,200.49326294)
\curveto(19.95582169,200.47325678)(19.96582168,200.43825681)(19.96581787,200.38826294)
\curveto(19.96582168,200.33825691)(19.97582167,200.28825696)(19.99581787,200.23826294)
\curveto(20.01582163,200.15825709)(20.03582161,200.06325719)(20.05581787,199.95326294)
\lineto(20.11581787,199.65326294)
\curveto(20.11582153,199.62325763)(20.12082153,199.58825766)(20.13081787,199.54826294)
\lineto(20.13081787,199.44326294)
\curveto(20.17082148,199.28325797)(20.19582145,199.11325814)(20.20581787,198.93326294)
\curveto(20.20582144,198.76325849)(20.22582142,198.59825865)(20.26581787,198.43826294)
\curveto(20.28582136,198.3482589)(20.30582134,198.26825898)(20.32581787,198.19826294)
\curveto(20.33582131,198.13825911)(20.3508213,198.06325919)(20.37081787,197.97326294)
\curveto(20.42082123,197.80325945)(20.48582116,197.63825961)(20.56581787,197.47826294)
\curveto(20.63582101,197.32825992)(20.72582092,197.19326006)(20.83581787,197.07326294)
\curveto(20.9458207,196.9532603)(21.08082057,196.8532604)(21.24081787,196.77326294)
\curveto(21.39082026,196.69326056)(21.57582007,196.63326062)(21.79581787,196.59326294)
\curveto(21.89581975,196.57326068)(21.99081966,196.57326068)(22.08081787,196.59326294)
\curveto(22.16081949,196.61326064)(22.23581941,196.64326061)(22.30581787,196.68326294)
\curveto(22.41581923,196.73326052)(22.51081914,196.81326044)(22.59081787,196.92326294)
\curveto(22.66081899,197.04326021)(22.72081893,197.17326008)(22.77081787,197.31326294)
\curveto(22.78081887,197.36325989)(22.78581886,197.41325984)(22.78581787,197.46326294)
\curveto(22.78581886,197.51325974)(22.79081886,197.56325969)(22.80081787,197.61326294)
\curveto(22.82081883,197.68325957)(22.83581881,197.76825948)(22.84581787,197.86826294)
\curveto(22.8458188,197.96825928)(22.83581881,198.05825919)(22.81581787,198.13826294)
\curveto(22.79581885,198.19825905)(22.79081886,198.25825899)(22.80081787,198.31826294)
\curveto(22.80081885,198.37825887)(22.79081886,198.43825881)(22.77081787,198.49826294)
\curveto(22.7508189,198.58825866)(22.73581891,198.66825858)(22.72581787,198.73826294)
\curveto(22.71581893,198.81825843)(22.69581895,198.89825835)(22.66581787,198.97826294)
\curveto(22.5458191,199.28825796)(22.40081925,199.56325769)(22.23081787,199.80326294)
\curveto(22.06081959,200.04325721)(21.83081982,200.248257)(21.54081787,200.41826294)
}
}
{
\newrgbcolor{curcolor}{0 0 0}
\pscustom[linestyle=none,fillstyle=solid,fillcolor=curcolor]
{
\newpath
\moveto(14.50581787,205.39990356)
\curveto(14.45582719,205.3299006)(14.35582729,205.30990062)(14.20581787,205.33990356)
\curveto(14.0458276,205.36990056)(13.90082775,205.39490054)(13.77081787,205.41490356)
\curveto(13.68082797,205.4349005)(13.59582805,205.45490048)(13.51581787,205.47490356)
\curveto(13.42582822,205.49490044)(13.3458283,205.52490041)(13.27581787,205.56490356)
\curveto(13.22582842,205.59490034)(13.18582846,205.6349003)(13.15581787,205.68490356)
\curveto(13.12582852,205.7349002)(13.10082855,205.79490014)(13.08081787,205.86490356)
\curveto(13.08082857,205.88490005)(13.08582856,205.89990003)(13.09581787,205.90990356)
\curveto(13.09582855,205.9299)(13.08582856,205.95489998)(13.06581787,205.98490356)
\curveto(13.06582858,206.15489978)(13.07082858,206.31489962)(13.08081787,206.46490356)
\curveto(13.09082856,206.61489932)(13.1508285,206.70489923)(13.26081787,206.73490356)
\curveto(13.33082832,206.74489919)(13.41082824,206.73989919)(13.50081787,206.71990356)
\curveto(13.58082807,206.70989922)(13.66582798,206.69989923)(13.75581787,206.68990356)
\curveto(13.93582771,206.64989928)(14.10582754,206.60989932)(14.26581787,206.56990356)
\curveto(14.41582723,206.5298994)(14.51582713,206.43989949)(14.56581787,206.29990356)
\curveto(14.58582706,206.23989969)(14.60082705,206.17989975)(14.61081787,206.11990356)
\lineto(14.61081787,205.96990356)
\curveto(14.61082704,205.84990008)(14.60582704,205.73990019)(14.59581787,205.63990356)
\curveto(14.58582706,205.53990039)(14.55582709,205.45990047)(14.50581787,205.39990356)
\moveto(24.42081787,204.54490356)
\curveto(24.49081716,204.5349014)(24.56581708,204.52490141)(24.64581787,204.51490356)
\curveto(24.72581692,204.50490143)(24.79581685,204.48490145)(24.85581787,204.45490356)
\lineto(24.97581787,204.42490356)
\curveto(25.03581661,204.40490153)(25.09081656,204.38490155)(25.14081787,204.36490356)
\curveto(25.20081645,204.35490158)(25.26081639,204.3349016)(25.32081787,204.30490356)
\curveto(25.49081616,204.2349017)(25.64081601,204.15990177)(25.77081787,204.07990356)
\curveto(25.90081575,203.99990193)(26.02081563,203.90490203)(26.13081787,203.79490356)
\curveto(26.2508154,203.68490225)(26.3508153,203.54990238)(26.43081787,203.38990356)
\curveto(26.51081514,203.23990269)(26.57581507,203.08490285)(26.62581787,202.92490356)
\curveto(26.645815,202.85490308)(26.65581499,202.78990314)(26.65581787,202.72990356)
\curveto(26.65581499,202.66990326)(26.66581498,202.59990333)(26.68581787,202.51990356)
\curveto(26.70581494,202.45990347)(26.71581493,202.36990356)(26.71581787,202.24990356)
\curveto(26.72581492,202.1299038)(26.72081493,202.03990389)(26.70081787,201.97990356)
\curveto(26.68081497,201.91990401)(26.66081499,201.86990406)(26.64081787,201.82990356)
\curveto(26.63081502,201.77990415)(26.60581504,201.74490419)(26.56581787,201.72490356)
\curveto(26.52581512,201.71490422)(26.47081518,201.70990422)(26.40081787,201.70990356)
\curveto(26.33081532,201.69990423)(26.25581539,201.69990423)(26.17581787,201.70990356)
\curveto(26.10581554,201.71990421)(26.03081562,201.73990419)(25.95081787,201.76990356)
\curveto(25.88081577,201.78990414)(25.82581582,201.81490412)(25.78581787,201.84490356)
\curveto(25.70581594,201.90490403)(25.66081599,201.96990396)(25.65081787,202.03990356)
\curveto(25.64081601,202.10990382)(25.62581602,202.19990373)(25.60581787,202.30990356)
\curveto(25.60581604,202.3299036)(25.60081605,202.34990358)(25.59081787,202.36990356)
\lineto(25.59081787,202.42990356)
\curveto(25.57081608,202.54990338)(25.53581611,202.65490328)(25.48581787,202.74490356)
\curveto(25.40581624,202.90490303)(25.25581639,203.0299029)(25.03581787,203.11990356)
\curveto(24.99581665,203.13990279)(24.95581669,203.15490278)(24.91581787,203.16490356)
\curveto(24.87581677,203.18490275)(24.83081682,203.20490273)(24.78081787,203.22490356)
\curveto(24.72081693,203.24490269)(24.65581699,203.25490268)(24.58581787,203.25490356)
\curveto(24.51581713,203.26490267)(24.45581719,203.27990265)(24.40581787,203.29990356)
\lineto(16.83081787,204.81490356)
\curveto(16.72082493,204.8349011)(16.60582504,204.85490108)(16.48581787,204.87490356)
\curveto(16.35582529,204.90490103)(16.2508254,204.94990098)(16.17081787,205.00990356)
\curveto(16.12082553,205.05990087)(16.07582557,205.14490079)(16.03581787,205.26490356)
\curveto(16.03582561,205.28490065)(16.04082561,205.30490063)(16.05081787,205.32490356)
\curveto(16.0508256,205.35490058)(16.04082561,205.37990055)(16.02081787,205.39990356)
\curveto(16.02082563,205.54990038)(16.02582562,205.69490024)(16.03581787,205.83490356)
\curveto(16.03582561,205.98489995)(16.08082557,206.07989985)(16.17081787,206.11990356)
\curveto(16.2508254,206.15989977)(16.37082528,206.16489977)(16.53081787,206.13490356)
\curveto(16.68082497,206.10489983)(16.82582482,206.07489986)(16.96581787,206.04490356)
\lineto(24.42081787,204.54490356)
}
}
{
\newrgbcolor{curcolor}{0 0 0}
\pscustom[linestyle=none,fillstyle=solid,fillcolor=curcolor]
{
\newpath
\moveto(19.59081787,214.21474731)
\curveto(19.69082196,214.21473881)(19.80582184,214.19473883)(19.93581787,214.15474731)
\curveto(20.05582159,214.11473891)(20.14082151,214.06473896)(20.19081787,214.00474731)
\curveto(20.23082142,213.94473908)(20.26082139,213.86473916)(20.28081787,213.76474731)
\curveto(20.29082136,213.66473936)(20.29582135,213.55473947)(20.29581787,213.43474731)
\lineto(20.29581787,213.07474731)
\curveto(20.28582136,212.96474006)(20.28082137,212.86474016)(20.28081787,212.77474731)
\lineto(20.28081787,208.93474731)
\curveto(20.28082137,208.85474417)(20.28582136,208.76974425)(20.29581787,208.67974731)
\curveto(20.29582135,208.59974442)(20.31082134,208.53474449)(20.34081787,208.48474731)
\curveto(20.36082129,208.43474459)(20.40082125,208.38474464)(20.46081787,208.33474731)
\lineto(20.59581787,208.24474731)
\curveto(20.645821,208.21474481)(20.69582095,208.20474482)(20.74581787,208.21474731)
\curveto(20.79582085,208.21474481)(20.84082081,208.20974481)(20.88081787,208.19974731)
\lineto(21.00081787,208.19974731)
\lineto(21.25581787,208.19974731)
\curveto(21.33582031,208.20974481)(21.41582023,208.2247448)(21.49581787,208.24474731)
\curveto(22.03581961,208.37474465)(22.42081923,208.67974434)(22.65081787,209.15974731)
\curveto(22.68081897,209.20974381)(22.70581894,209.26974375)(22.72581787,209.33974731)
\curveto(22.7458189,209.40974361)(22.76581888,209.47474355)(22.78581787,209.53474731)
\curveto(22.79581885,209.56474346)(22.80081885,209.61474341)(22.80081787,209.68474731)
\curveto(22.84081881,209.81474321)(22.86081879,209.99474303)(22.86081787,210.22474731)
\curveto(22.86081879,210.45474257)(22.84081881,210.64474238)(22.80081787,210.79474731)
\curveto(22.76081889,210.94474208)(22.72081893,211.07974194)(22.68081787,211.19974731)
\curveto(22.63081902,211.32974169)(22.57081908,211.44974157)(22.50081787,211.55974731)
\curveto(22.43081922,211.67974134)(22.3508193,211.78974123)(22.26081787,211.88974731)
\curveto(22.16081949,211.98974103)(22.05581959,212.07974094)(21.94581787,212.15974731)
\curveto(21.8458198,212.23974078)(21.74081991,212.31474071)(21.63081787,212.38474731)
\curveto(21.52082013,212.45474057)(21.44082021,212.54974047)(21.39081787,212.66974731)
\curveto(21.37082028,212.70974031)(21.35582029,212.77474025)(21.34581787,212.86474731)
\curveto(21.33582031,212.96474006)(21.33582031,213.05473997)(21.34581787,213.13474731)
\curveto(21.3458203,213.2247398)(21.3508203,213.30973971)(21.36081787,213.38974731)
\curveto(21.37082028,213.46973955)(21.39082026,213.5197395)(21.42081787,213.53974731)
\curveto(21.49082016,213.62973939)(21.60582004,213.63473939)(21.76581787,213.55474731)
\curveto(22.03581961,213.41473961)(22.27581937,213.25973976)(22.48581787,213.08974731)
\curveto(22.80581884,212.82974019)(23.07081858,212.54974047)(23.28081787,212.24974731)
\curveto(23.48081817,211.95974106)(23.645818,211.60474142)(23.77581787,211.18474731)
\curveto(23.81581783,211.07474195)(23.84081781,210.96974205)(23.85081787,210.86974731)
\curveto(23.87081778,210.76974225)(23.89081776,210.65974236)(23.91081787,210.53974731)
\curveto(23.92081773,210.48974253)(23.92581772,210.43974258)(23.92581787,210.38974731)
\curveto(23.92581772,210.34974267)(23.93081772,210.30474272)(23.94081787,210.25474731)
\lineto(23.94081787,210.10474731)
\curveto(23.9508177,210.05474297)(23.95581769,209.99474303)(23.95581787,209.92474731)
\curveto(23.95581769,209.86474316)(23.9508177,209.81474321)(23.94081787,209.77474731)
\lineto(23.94081787,209.63974731)
\curveto(23.93081772,209.58974343)(23.92581772,209.54474348)(23.92581787,209.50474731)
\curveto(23.92581772,209.46474356)(23.92081773,209.4247436)(23.91081787,209.38474731)
\curveto(23.90081775,209.33474369)(23.89081776,209.27974374)(23.88081787,209.21974731)
\curveto(23.88081777,209.16974385)(23.87581777,209.1197439)(23.86581787,209.06974731)
\curveto(23.8458178,208.97974404)(23.82081783,208.88974413)(23.79081787,208.79974731)
\curveto(23.77081788,208.7197443)(23.7458179,208.64474438)(23.71581787,208.57474731)
\curveto(23.69581795,208.53474449)(23.68581796,208.49974452)(23.68581787,208.46974731)
\curveto(23.67581797,208.43974458)(23.66081799,208.40974461)(23.64081787,208.37974731)
\curveto(23.57081808,208.23974478)(23.48581816,208.09474493)(23.38581787,207.94474731)
\curveto(23.19581845,207.69474533)(22.96581868,207.49474553)(22.69581787,207.34474731)
\curveto(22.41581923,207.19474583)(22.10581954,207.08474594)(21.76581787,207.01474731)
\curveto(21.65581999,206.98474604)(21.54082011,206.96974605)(21.42081787,206.96974731)
\curveto(21.30082035,206.96974605)(21.18082047,206.95974606)(21.06081787,206.93974731)
\lineto(20.95581787,206.93974731)
\curveto(20.92582072,206.94974607)(20.88582076,206.95474607)(20.83581787,206.95474731)
\lineto(20.58081787,206.95474731)
\curveto(20.49082116,206.96474606)(20.40082125,206.96974605)(20.31081787,206.96974731)
\lineto(20.10081787,207.01474731)
\curveto(20.06082159,207.01474601)(20.00582164,207.019746)(19.93581787,207.02974731)
\curveto(19.85582179,207.03974598)(19.79082186,207.05474597)(19.74081787,207.07474731)
\lineto(19.57581787,207.10474731)
\curveto(19.52582212,207.13474589)(19.47582217,207.14974587)(19.42581787,207.14974731)
\curveto(19.36582228,207.15974586)(19.31082234,207.17474585)(19.26081787,207.19474731)
\curveto(19.10082255,207.26474576)(18.94082271,207.32974569)(18.78081787,207.38974731)
\curveto(18.62082303,207.44974557)(18.47082318,207.5247455)(18.33081787,207.61474731)
\curveto(18.22082343,207.68474534)(18.11082354,207.74974527)(18.00081787,207.80974731)
\curveto(17.88082377,207.87974514)(17.76582388,207.95974506)(17.65581787,208.04974731)
\curveto(17.30582434,208.33974468)(17.00582464,208.64974437)(16.75581787,208.97974731)
\curveto(16.49582515,209.30974371)(16.28082537,209.69474333)(16.11081787,210.13474731)
\curveto(16.06082559,210.26474276)(16.02582562,210.39474263)(16.00581787,210.52474731)
\curveto(15.97582567,210.65474237)(15.9458257,210.79474223)(15.91581787,210.94474731)
\curveto(15.90582574,210.99474203)(15.90082575,211.03974198)(15.90081787,211.07974731)
\curveto(15.89082576,211.1197419)(15.88582576,211.16474186)(15.88581787,211.21474731)
\curveto(15.87582577,211.23474179)(15.87582577,211.25974176)(15.88581787,211.28974731)
\curveto(15.89582575,211.3197417)(15.89082576,211.34474168)(15.87081787,211.36474731)
\curveto(15.86082579,211.79474123)(15.90582574,212.15474087)(16.00581787,212.44474731)
\curveto(16.09582555,212.73474029)(16.22082543,212.98974003)(16.38081787,213.20974731)
\curveto(16.40082525,213.24973977)(16.43082522,213.27973974)(16.47081787,213.29974731)
\curveto(16.50082515,213.32973969)(16.52582512,213.35973966)(16.54581787,213.38974731)
\curveto(16.60582504,213.45973956)(16.67582497,213.52973949)(16.75581787,213.59974731)
\curveto(16.83582481,213.66973935)(16.91582473,213.7247393)(16.99581787,213.76474731)
\curveto(17.20582444,213.88473914)(17.40582424,213.97973904)(17.59581787,214.04974731)
\curveto(17.70582394,214.09973892)(17.82582382,214.12973889)(17.95581787,214.13974731)
\lineto(18.34581787,214.19974731)
\curveto(18.47582317,214.22973879)(18.61082304,214.23973878)(18.75081787,214.22974731)
\curveto(18.89082276,214.22973879)(19.03082262,214.23473879)(19.17081787,214.24474731)
\curveto(19.24082241,214.24473878)(19.31082234,214.23973878)(19.38081787,214.22974731)
\curveto(19.4508222,214.2197388)(19.52082213,214.21473881)(19.59081787,214.21474731)
\moveto(19.08081787,212.86474731)
\curveto(19.04082261,212.89474013)(18.99082266,212.9247401)(18.93081787,212.95474731)
\curveto(18.86082279,212.99474003)(18.79082286,213.00974001)(18.72081787,212.99974731)
\curveto(18.50082315,212.98974003)(18.29582335,212.94974007)(18.10581787,212.87974731)
\curveto(17.87582377,212.77974024)(17.68082397,212.65974036)(17.52081787,212.51974731)
\curveto(17.36082429,212.38974063)(17.22582442,212.19974082)(17.11581787,211.94974731)
\curveto(17.09582455,211.87974114)(17.08082457,211.80974121)(17.07081787,211.73974731)
\curveto(17.0508246,211.67974134)(17.03082462,211.60974141)(17.01081787,211.52974731)
\curveto(16.99082466,211.45974156)(16.98082467,211.37974164)(16.98081787,211.28974731)
\lineto(16.98081787,211.03474731)
\curveto(17.00082465,210.99474203)(17.01082464,210.95474207)(17.01081787,210.91474731)
\curveto(17.00082465,210.87474215)(17.00082465,210.83974218)(17.01081787,210.80974731)
\lineto(17.07081787,210.56974731)
\curveto(17.08082457,210.48974253)(17.09582455,210.41474261)(17.11581787,210.34474731)
\curveto(17.23582441,210.024743)(17.38582426,209.75974326)(17.56581787,209.54974731)
\curveto(17.7458239,209.33974368)(17.97082368,209.13974388)(18.24081787,208.94974731)
\curveto(18.29082336,208.90974411)(18.35582329,208.86474416)(18.43581787,208.81474731)
\curveto(18.50582314,208.77474425)(18.58582306,208.73474429)(18.67581787,208.69474731)
\curveto(18.76582288,208.65474437)(18.8508228,208.62974439)(18.93081787,208.61974731)
\curveto(19.01082264,208.6197444)(19.07082258,208.64474438)(19.11081787,208.69474731)
\curveto(19.17082248,208.76474426)(19.20082245,208.89474413)(19.20081787,209.08474731)
\curveto(19.19082246,209.28474374)(19.18582246,209.45474357)(19.18581787,209.59474731)
\lineto(19.18581787,211.87474731)
\curveto(19.18582246,212.024741)(19.19082246,212.20474082)(19.20081787,212.41474731)
\curveto(19.20082245,212.6247404)(19.16082249,212.77474025)(19.08081787,212.86474731)
}
}
{
\newrgbcolor{curcolor}{0 0 0}
\pscustom[linestyle=none,fillstyle=solid,fillcolor=curcolor]
{
}
}
{
\newrgbcolor{curcolor}{0 0 0}
\pscustom[linestyle=none,fillstyle=solid,fillcolor=curcolor]
{
\newpath
\moveto(23.20581787,225.84654419)
\curveto(23.36581828,225.83653628)(23.50081815,225.79153632)(23.61081787,225.71154419)
\curveto(23.71081794,225.63153648)(23.78581786,225.53653658)(23.83581787,225.42654419)
\curveto(23.85581779,225.37653674)(23.86581778,225.32153679)(23.86581787,225.26154419)
\curveto(23.86581778,225.2115369)(23.87581777,225.15153696)(23.89581787,225.08154419)
\curveto(23.9458177,224.85153726)(23.93081772,224.63653748)(23.85081787,224.43654419)
\curveto(23.78081787,224.23653788)(23.69081796,224.111538)(23.58081787,224.06154419)
\curveto(23.51081814,224.02153809)(23.43081822,223.99153812)(23.34081787,223.97154419)
\curveto(23.24081841,223.95153816)(23.16081849,223.9165382)(23.10081787,223.86654419)
\lineto(23.04081787,223.80654419)
\curveto(23.02081863,223.78653833)(23.01581863,223.75653836)(23.02581787,223.71654419)
\curveto(23.05581859,223.59653852)(23.11081854,223.48153863)(23.19081787,223.37154419)
\curveto(23.27081838,223.26153885)(23.34081831,223.15653896)(23.40081787,223.05654419)
\curveto(23.48081817,222.90653921)(23.55581809,222.75153936)(23.62581787,222.59154419)
\curveto(23.68581796,222.43153968)(23.74081791,222.26153985)(23.79081787,222.08154419)
\curveto(23.82081783,221.97154014)(23.84081781,221.85654026)(23.85081787,221.73654419)
\curveto(23.86081779,221.62654049)(23.87581777,221.5115406)(23.89581787,221.39154419)
\curveto(23.90581774,221.34154077)(23.91081774,221.29654082)(23.91081787,221.25654419)
\lineto(23.91081787,221.15154419)
\curveto(23.93081772,221.04154107)(23.93081772,220.93654118)(23.91081787,220.83654419)
\lineto(23.91081787,220.70154419)
\curveto(23.90081775,220.65154146)(23.89581775,220.60154151)(23.89581787,220.55154419)
\curveto(23.89581775,220.50154161)(23.88581776,220.46154165)(23.86581787,220.43154419)
\curveto(23.85581779,220.39154172)(23.8508178,220.35654176)(23.85081787,220.32654419)
\curveto(23.86081779,220.30654181)(23.86081779,220.28154183)(23.85081787,220.25154419)
\lineto(23.79081787,220.01154419)
\curveto(23.78081787,219.94154217)(23.76081789,219.87654224)(23.73081787,219.81654419)
\curveto(23.60081805,219.53654258)(23.45581819,219.32154279)(23.29581787,219.17154419)
\curveto(23.12581852,219.02154309)(22.89081876,218.9165432)(22.59081787,218.85654419)
\curveto(22.37081928,218.80654331)(22.10581954,218.8115433)(21.79581787,218.87154419)
\lineto(21.48081787,218.94654419)
\curveto(21.43082022,218.96654315)(21.38082027,218.98154313)(21.33081787,218.99154419)
\lineto(21.15081787,219.05154419)
\lineto(20.82081787,219.23154419)
\curveto(20.71082094,219.30154281)(20.61082104,219.37154274)(20.52081787,219.44154419)
\curveto(20.23082142,219.68154243)(20.01582163,219.97154214)(19.87581787,220.31154419)
\curveto(19.73582191,220.65154146)(19.61082204,221.0165411)(19.50081787,221.40654419)
\curveto(19.46082219,221.55654056)(19.43082222,221.70654041)(19.41081787,221.85654419)
\curveto(19.39082226,222.0165401)(19.36582228,222.17153994)(19.33581787,222.32154419)
\curveto(19.31582233,222.40153971)(19.30582234,222.47153964)(19.30581787,222.53154419)
\curveto(19.30582234,222.60153951)(19.29582235,222.67653944)(19.27581787,222.75654419)
\curveto(19.25582239,222.82653929)(19.2458224,222.89653922)(19.24581787,222.96654419)
\curveto(19.23582241,223.04653907)(19.22082243,223.12653899)(19.20081787,223.20654419)
\curveto(19.14082251,223.46653865)(19.09082256,223.7115384)(19.05081787,223.94154419)
\curveto(19.00082265,224.17153794)(18.88582276,224.37153774)(18.70581787,224.54154419)
\curveto(18.62582302,224.6115375)(18.52582312,224.67653744)(18.40581787,224.73654419)
\curveto(18.27582337,224.80653731)(18.13582351,224.83653728)(17.98581787,224.82654419)
\curveto(17.7458239,224.8165373)(17.55582409,224.76653735)(17.41581787,224.67654419)
\curveto(17.27582437,224.59653752)(17.16582448,224.45653766)(17.08581787,224.25654419)
\curveto(17.03582461,224.14653797)(17.00082465,224.0115381)(16.98081787,223.85154419)
\curveto(16.96082469,223.69153842)(16.9508247,223.52153859)(16.95081787,223.34154419)
\curveto(16.9508247,223.16153895)(16.96082469,222.98153913)(16.98081787,222.80154419)
\curveto(17.00082465,222.63153948)(17.03082462,222.48153963)(17.07081787,222.35154419)
\curveto(17.13082452,222.17153994)(17.21582443,221.99154012)(17.32581787,221.81154419)
\curveto(17.38582426,221.72154039)(17.46582418,221.63154048)(17.56581787,221.54154419)
\curveto(17.65582399,221.46154065)(17.75582389,221.38654073)(17.86581787,221.31654419)
\curveto(17.9458237,221.26654085)(18.03082362,221.22154089)(18.12081787,221.18154419)
\curveto(18.21082344,221.14154097)(18.28082337,221.08154103)(18.33081787,221.00154419)
\curveto(18.36082329,220.95154116)(18.38582326,220.87654124)(18.40581787,220.77654419)
\curveto(18.41582323,220.67654144)(18.42082323,220.57654154)(18.42081787,220.47654419)
\curveto(18.42082323,220.37654174)(18.41582323,220.28154183)(18.40581787,220.19154419)
\curveto(18.38582326,220.10154201)(18.36082329,220.04154207)(18.33081787,220.01154419)
\curveto(18.30082335,219.97154214)(18.2508234,219.94654217)(18.18081787,219.93654419)
\curveto(18.11082354,219.93654218)(18.03582361,219.95654216)(17.95581787,219.99654419)
\curveto(17.82582382,220.04654207)(17.70582394,220.10154201)(17.59581787,220.16154419)
\curveto(17.47582417,220.22154189)(17.36082429,220.28654183)(17.25081787,220.35654419)
\curveto(16.90082475,220.6165415)(16.63082502,220.9115412)(16.44081787,221.24154419)
\curveto(16.24082541,221.57154054)(16.08082557,221.96154015)(15.96081787,222.41154419)
\curveto(15.94082571,222.52153959)(15.92582572,222.62653949)(15.91581787,222.72654419)
\curveto(15.90582574,222.83653928)(15.89082576,222.94653917)(15.87081787,223.05654419)
\curveto(15.86082579,223.10653901)(15.86082579,223.17153894)(15.87081787,223.25154419)
\curveto(15.87082578,223.34153877)(15.86082579,223.40153871)(15.84081787,223.43154419)
\curveto(15.83082582,224.13153798)(15.91082574,224.72153739)(16.08081787,225.20154419)
\curveto(16.2508254,225.69153642)(16.57582507,225.99653612)(17.05581787,226.11654419)
\curveto(17.25582439,226.16653595)(17.49082416,226.17153594)(17.76081787,226.13154419)
\curveto(18.02082363,226.09153602)(18.29582335,226.04153607)(18.58581787,225.98154419)
\lineto(21.90081787,225.32154419)
\curveto(22.04081961,225.29153682)(22.17581947,225.26653685)(22.30581787,225.24654419)
\curveto(22.43581921,225.23653688)(22.54081911,225.24653687)(22.62081787,225.27654419)
\curveto(22.69081896,225.3165368)(22.74081891,225.37153674)(22.77081787,225.44154419)
\curveto(22.81081884,225.53153658)(22.84081881,225.6115365)(22.86081787,225.68154419)
\curveto(22.87081878,225.76153635)(22.91581873,225.8115363)(22.99581787,225.83154419)
\curveto(23.02581862,225.85153626)(23.05581859,225.85653626)(23.08581787,225.84654419)
\lineto(23.20581787,225.84654419)
\moveto(21.54081787,224.03154419)
\curveto(21.40082025,224.12153799)(21.24082041,224.18653793)(21.06081787,224.22654419)
\curveto(20.87082078,224.26653785)(20.67582097,224.30653781)(20.47581787,224.34654419)
\curveto(20.36582128,224.36653775)(20.26582138,224.38153773)(20.17581787,224.39154419)
\curveto(20.08582156,224.40153771)(20.01582163,224.37653774)(19.96581787,224.31654419)
\curveto(19.9458217,224.28653783)(19.93582171,224.2165379)(19.93581787,224.10654419)
\curveto(19.95582169,224.08653803)(19.96582168,224.05153806)(19.96581787,224.00154419)
\curveto(19.96582168,223.95153816)(19.97582167,223.90153821)(19.99581787,223.85154419)
\curveto(20.01582163,223.77153834)(20.03582161,223.67653844)(20.05581787,223.56654419)
\lineto(20.11581787,223.26654419)
\curveto(20.11582153,223.23653888)(20.12082153,223.20153891)(20.13081787,223.16154419)
\lineto(20.13081787,223.05654419)
\curveto(20.17082148,222.89653922)(20.19582145,222.72653939)(20.20581787,222.54654419)
\curveto(20.20582144,222.37653974)(20.22582142,222.2115399)(20.26581787,222.05154419)
\curveto(20.28582136,221.96154015)(20.30582134,221.88154023)(20.32581787,221.81154419)
\curveto(20.33582131,221.75154036)(20.3508213,221.67654044)(20.37081787,221.58654419)
\curveto(20.42082123,221.4165407)(20.48582116,221.25154086)(20.56581787,221.09154419)
\curveto(20.63582101,220.94154117)(20.72582092,220.80654131)(20.83581787,220.68654419)
\curveto(20.9458207,220.56654155)(21.08082057,220.46654165)(21.24081787,220.38654419)
\curveto(21.39082026,220.30654181)(21.57582007,220.24654187)(21.79581787,220.20654419)
\curveto(21.89581975,220.18654193)(21.99081966,220.18654193)(22.08081787,220.20654419)
\curveto(22.16081949,220.22654189)(22.23581941,220.25654186)(22.30581787,220.29654419)
\curveto(22.41581923,220.34654177)(22.51081914,220.42654169)(22.59081787,220.53654419)
\curveto(22.66081899,220.65654146)(22.72081893,220.78654133)(22.77081787,220.92654419)
\curveto(22.78081887,220.97654114)(22.78581886,221.02654109)(22.78581787,221.07654419)
\curveto(22.78581886,221.12654099)(22.79081886,221.17654094)(22.80081787,221.22654419)
\curveto(22.82081883,221.29654082)(22.83581881,221.38154073)(22.84581787,221.48154419)
\curveto(22.8458188,221.58154053)(22.83581881,221.67154044)(22.81581787,221.75154419)
\curveto(22.79581885,221.8115403)(22.79081886,221.87154024)(22.80081787,221.93154419)
\curveto(22.80081885,221.99154012)(22.79081886,222.05154006)(22.77081787,222.11154419)
\curveto(22.7508189,222.20153991)(22.73581891,222.28153983)(22.72581787,222.35154419)
\curveto(22.71581893,222.43153968)(22.69581895,222.5115396)(22.66581787,222.59154419)
\curveto(22.5458191,222.90153921)(22.40081925,223.17653894)(22.23081787,223.41654419)
\curveto(22.06081959,223.65653846)(21.83081982,223.86153825)(21.54081787,224.03154419)
}
}
{
\newrgbcolor{curcolor}{0 0 0}
\pscustom[linestyle=none,fillstyle=solid,fillcolor=curcolor]
{
\newpath
\moveto(15.85581787,231.62318481)
\curveto(15.8458258,232.36317843)(15.95582569,232.95817784)(16.18581787,233.40818481)
\curveto(16.40582524,233.85817694)(16.74082491,234.18317661)(17.19081787,234.38318481)
\curveto(17.39082426,234.47317632)(17.63582401,234.53317626)(17.92581787,234.56318481)
\curveto(17.97582367,234.57317622)(18.04082361,234.57317622)(18.12081787,234.56318481)
\curveto(18.20082345,234.56317623)(18.27082338,234.54817625)(18.33081787,234.51818481)
\curveto(18.38082327,234.47817632)(18.42582322,234.41817638)(18.46581787,234.33818481)
\curveto(18.48582316,234.2981765)(18.49582315,234.26317653)(18.49581787,234.23318481)
\curveto(18.48582316,234.21317658)(18.48582316,234.17817662)(18.49581787,234.12818481)
\curveto(18.50582314,234.08817671)(18.51082314,234.04817675)(18.51081787,234.00818481)
\curveto(18.50082315,233.96817683)(18.49582315,233.92817687)(18.49581787,233.88818481)
\lineto(18.49581787,233.57318481)
\curveto(18.48582316,233.48317731)(18.45582319,233.40817739)(18.40581787,233.34818481)
\curveto(18.3458233,233.27817752)(18.26082339,233.23817756)(18.15081787,233.22818481)
\curveto(18.04082361,233.21817758)(17.9458237,233.1981776)(17.86581787,233.16818481)
\curveto(17.60582404,233.06817773)(17.40082425,232.91317788)(17.25081787,232.70318481)
\curveto(17.20082445,232.63317816)(17.16082449,232.55317824)(17.13081787,232.46318481)
\curveto(17.09082456,232.38317841)(17.05582459,232.2981785)(17.02581787,232.20818481)
\curveto(16.98582466,232.07817872)(16.96582468,231.8981789)(16.96581787,231.66818481)
\curveto(16.95582469,231.43817936)(16.97582467,231.24317955)(17.02581787,231.08318481)
\curveto(17.0458246,231.01317978)(17.06082459,230.94317985)(17.07081787,230.87318481)
\curveto(17.08082457,230.81317998)(17.09582455,230.74818005)(17.11581787,230.67818481)
\curveto(17.22582442,230.3981804)(17.37582427,230.13818066)(17.56581787,229.89818481)
\curveto(17.75582389,229.65818114)(17.98082367,229.45818134)(18.24081787,229.29818481)
\curveto(18.33082332,229.23818156)(18.42582322,229.18318161)(18.52581787,229.13318481)
\curveto(18.61582303,229.08318171)(18.71582293,229.03318176)(18.82581787,228.98318481)
\lineto(19.23081787,228.81818481)
\curveto(19.28082237,228.798182)(19.33582231,228.78318201)(19.39581787,228.77318481)
\curveto(19.45582219,228.76318203)(19.51082214,228.74318205)(19.56081787,228.71318481)
\lineto(19.68081787,228.69818481)
\curveto(19.72082193,228.67818212)(19.78582186,228.65318214)(19.87581787,228.62318481)
\curveto(19.96582168,228.60318219)(20.03082162,228.5981822)(20.07081787,228.60818481)
\curveto(20.12082153,228.60818219)(20.17082148,228.5981822)(20.22081787,228.57818481)
\curveto(20.27082138,228.55818224)(20.32082133,228.54818225)(20.37081787,228.54818481)
\curveto(20.41082124,228.55818224)(20.48082117,228.55318224)(20.58081787,228.53318481)
\curveto(20.66082099,228.53318226)(20.7458209,228.52818227)(20.83581787,228.51818481)
\curveto(20.92582072,228.51818228)(21.01082064,228.52318227)(21.09081787,228.53318481)
\curveto(21.41082024,228.57318222)(21.69081996,228.64318215)(21.93081787,228.74318481)
\curveto(22.16081949,228.84318195)(22.36081929,229.00818179)(22.53081787,229.23818481)
\curveto(22.58081907,229.31818148)(22.62581902,229.3981814)(22.66581787,229.47818481)
\curveto(22.70581894,229.56818123)(22.7458189,229.66318113)(22.78581787,229.76318481)
\curveto(22.79581885,229.81318098)(22.80081885,229.85318094)(22.80081787,229.88318481)
\curveto(22.80081885,229.91318088)(22.80581884,229.95318084)(22.81581787,230.00318481)
\curveto(22.82581882,230.03318076)(22.83081882,230.08318071)(22.83081787,230.15318481)
\lineto(22.83081787,230.31818481)
\curveto(22.84081881,230.30818049)(22.8458188,230.32318047)(22.84581787,230.36318481)
\curveto(22.83581881,230.3931804)(22.83581881,230.41818038)(22.84581787,230.43818481)
\curveto(22.8458188,230.46818033)(22.84081881,230.50318029)(22.83081787,230.54318481)
\curveto(22.81081884,230.61318018)(22.80581884,230.67818012)(22.81581787,230.73818481)
\curveto(22.81581883,230.80817999)(22.80581884,230.87817992)(22.78581787,230.94818481)
\curveto(22.70581894,231.22817957)(22.60581904,231.47317932)(22.48581787,231.68318481)
\curveto(22.35581929,231.90317889)(22.19081946,232.0981787)(21.99081787,232.26818481)
\curveto(21.91081974,232.32817847)(21.82581982,232.38817841)(21.73581787,232.44818481)
\lineto(21.46581787,232.62818481)
\curveto(21.38582026,232.65817814)(21.31082034,232.6931781)(21.24081787,232.73318481)
\curveto(21.16082049,232.77317802)(21.09582055,232.83317796)(21.04581787,232.91318481)
\curveto(21.01582063,232.95317784)(20.99582065,233.01817778)(20.98581787,233.10818481)
\curveto(20.96582068,233.20817759)(20.95582069,233.30817749)(20.95581787,233.40818481)
\curveto(20.9458207,233.51817728)(20.9458207,233.61817718)(20.95581787,233.70818481)
\curveto(20.96582068,233.798177)(20.98582066,233.86317693)(21.01581787,233.90318481)
\curveto(21.05582059,233.95317684)(21.11582053,233.97817682)(21.19581787,233.97818481)
\curveto(21.27582037,233.97817682)(21.36082029,233.95817684)(21.45081787,233.91818481)
\curveto(21.60082005,233.83817696)(21.7458199,233.76317703)(21.88581787,233.69318481)
\curveto(22.01581963,233.62317717)(22.1458195,233.53817726)(22.27581787,233.43818481)
\curveto(22.57581907,233.22817757)(22.84081881,232.98817781)(23.07081787,232.71818481)
\curveto(23.30081835,232.44817835)(23.48581816,232.13817866)(23.62581787,231.78818481)
\curveto(23.66581798,231.6981791)(23.70081795,231.60317919)(23.73081787,231.50318481)
\curveto(23.7508179,231.41317938)(23.77581787,231.31817948)(23.80581787,231.21818481)
\curveto(23.8458178,231.0981797)(23.86581778,230.98317981)(23.86581787,230.87318481)
\curveto(23.87581777,230.76318003)(23.89081776,230.64818015)(23.91081787,230.52818481)
\curveto(23.93081772,230.48818031)(23.93581771,230.44818035)(23.92581787,230.40818481)
\curveto(23.91581773,230.36818043)(23.91581773,230.32818047)(23.92581787,230.28818481)
\lineto(23.92581787,230.15318481)
\lineto(23.92581787,229.91318481)
\curveto(23.93581771,229.84318095)(23.93081772,229.77818102)(23.91081787,229.71818481)
\lineto(23.91081787,229.64318481)
\lineto(23.86581787,229.29818481)
\curveto(23.82581782,229.17818162)(23.79081786,229.05818174)(23.76081787,228.93818481)
\curveto(23.73081792,228.82818197)(23.69081796,228.72318207)(23.64081787,228.62318481)
\curveto(23.48081817,228.2931825)(23.29081836,228.03318276)(23.07081787,227.84318481)
\curveto(22.8508188,227.65318314)(22.58081907,227.48818331)(22.26081787,227.34818481)
\curveto(22.18081947,227.31818348)(22.09081956,227.2931835)(21.99081787,227.27318481)
\lineto(21.69081787,227.21318481)
\curveto(21.58082007,227.18318361)(21.46582018,227.16818363)(21.34581787,227.16818481)
\curveto(21.22582042,227.17818362)(21.10582054,227.17818362)(20.98581787,227.16818481)
\curveto(20.9458207,227.16818363)(20.90582074,227.17318362)(20.86581787,227.18318481)
\curveto(20.82582082,227.1931836)(20.78582086,227.1931836)(20.74581787,227.18318481)
\curveto(20.68582096,227.18318361)(20.62082103,227.18818361)(20.55081787,227.19818481)
\curveto(20.48082117,227.21818358)(20.41582123,227.22818357)(20.35581787,227.22818481)
\lineto(20.20581787,227.25818481)
\curveto(20.15582149,227.25818354)(20.08582156,227.26318353)(19.99581787,227.27318481)
\curveto(19.90582174,227.2931835)(19.83582181,227.31318348)(19.78581787,227.33318481)
\curveto(19.73582191,227.35318344)(19.69082196,227.36318343)(19.65081787,227.36318481)
\curveto(19.61082204,227.37318342)(19.57082208,227.38818341)(19.53081787,227.40818481)
\curveto(19.46082219,227.43818336)(19.39082226,227.45818334)(19.32081787,227.46818481)
\curveto(19.2508224,227.47818332)(19.18582246,227.4981833)(19.12581787,227.52818481)
\curveto(18.95582269,227.5981832)(18.78582286,227.66318313)(18.61581787,227.72318481)
\curveto(18.4458232,227.793183)(18.28582336,227.87318292)(18.13581787,227.96318481)
\curveto(17.61582403,228.28318251)(17.19582445,228.62318217)(16.87581787,228.98318481)
\curveto(16.55582509,229.34318145)(16.29082536,229.80818099)(16.08081787,230.37818481)
\curveto(16.03082562,230.4981803)(15.99582565,230.62318017)(15.97581787,230.75318481)
\curveto(15.95582569,230.88317991)(15.93082572,231.02317977)(15.90081787,231.17318481)
\curveto(15.89082576,231.24317955)(15.88582576,231.31317948)(15.88581787,231.38318481)
\curveto(15.87582577,231.45317934)(15.86582578,231.53317926)(15.85581787,231.62318481)
}
}
{
\newrgbcolor{curcolor}{0 0 0}
\pscustom[linestyle=none,fillstyle=solid,fillcolor=curcolor]
{
\newpath
\moveto(16.03581787,236.96482544)
\lineto(16.03581787,237.39982544)
\curveto(16.03582561,237.54982193)(16.07582557,237.64982183)(16.15581787,237.69982544)
\curveto(16.23582541,237.72982175)(16.33582531,237.73482174)(16.45581787,237.71482544)
\lineto(16.81581787,237.65482544)
\lineto(18.24081787,237.36982544)
\lineto(20.50581787,236.91982544)
\curveto(20.72582092,236.86982261)(20.95582069,236.81982266)(21.19581787,236.76982544)
\curveto(21.42582022,236.72982275)(21.62582002,236.71482276)(21.79581787,236.72482544)
\curveto(22.2458194,236.79482268)(22.56081909,237.03482244)(22.74081787,237.44482544)
\curveto(22.83081882,237.64482183)(22.86581878,237.89982158)(22.84581787,238.20982544)
\curveto(22.81581883,238.52982095)(22.76081889,238.79482068)(22.68081787,239.00482544)
\curveto(22.54081911,239.35482012)(22.36581928,239.64981983)(22.15581787,239.88982544)
\curveto(21.93581971,240.12981935)(21.65082,240.33981914)(21.30081787,240.51982544)
\curveto(21.22082043,240.56981891)(21.14082051,240.60481887)(21.06081787,240.62482544)
\curveto(20.98082067,240.65481882)(20.89582075,240.68981879)(20.80581787,240.72982544)
\curveto(20.75582089,240.74981873)(20.71082094,240.75981872)(20.67081787,240.75982544)
\curveto(20.63082102,240.75981872)(20.58582106,240.7748187)(20.53581787,240.80482544)
\lineto(20.22081787,240.86482544)
\curveto(20.14082151,240.90481857)(20.0508216,240.92981855)(19.95081787,240.93982544)
\curveto(19.84082181,240.94981853)(19.74082191,240.96481851)(19.65081787,240.98482544)
\lineto(18.48081787,241.22482544)
\lineto(16.89081787,241.53982544)
\curveto(16.77082488,241.55981792)(16.645825,241.5798179)(16.51581787,241.59982544)
\curveto(16.37582527,241.62981785)(16.26582538,241.6748178)(16.18581787,241.73482544)
\curveto(16.13582551,241.78481769)(16.10582554,241.83981764)(16.09581787,241.89982544)
\curveto(16.07582557,241.95981752)(16.05582559,242.02981745)(16.03581787,242.10982544)
\lineto(16.03581787,242.33482544)
\curveto(16.03582561,242.45481702)(16.04082561,242.55981692)(16.05081787,242.64982544)
\curveto(16.06082559,242.74981673)(16.10582554,242.81481666)(16.18581787,242.84482544)
\curveto(16.23582541,242.88481659)(16.31082534,242.89481658)(16.41081787,242.87482544)
\curveto(16.50082515,242.85481662)(16.59582505,242.83481664)(16.69581787,242.81482544)
\lineto(17.71581787,242.61982544)
\lineto(21.75081787,241.80982544)
\lineto(23.10081787,241.53982544)
\curveto(23.22081843,241.51981796)(23.33581831,241.49481798)(23.44581787,241.46482544)
\curveto(23.5458181,241.43481804)(23.62081803,241.3798181)(23.67081787,241.29982544)
\curveto(23.70081795,241.25981822)(23.72581792,241.19481828)(23.74581787,241.10482544)
\curveto(23.75581789,241.02481845)(23.76581788,240.93481854)(23.77581787,240.83482544)
\curveto(23.77581787,240.74481873)(23.77081788,240.65481882)(23.76081787,240.56482544)
\curveto(23.7508179,240.48481899)(23.73081792,240.42981905)(23.70081787,240.39982544)
\curveto(23.66081799,240.35981912)(23.59581805,240.32981915)(23.50581787,240.30982544)
\curveto(23.46581818,240.29981918)(23.41081824,240.29981918)(23.34081787,240.30982544)
\curveto(23.27081838,240.31981916)(23.20581844,240.32481915)(23.14581787,240.32482544)
\curveto(23.07581857,240.33481914)(23.02081863,240.32481915)(22.98081787,240.29482544)
\curveto(22.94081871,240.2748192)(22.92581872,240.23481924)(22.93581787,240.17482544)
\curveto(22.95581869,240.09481938)(23.01581863,240.00481947)(23.11581787,239.90482544)
\curveto(23.20581844,239.80481967)(23.27581837,239.71481976)(23.32581787,239.63482544)
\curveto(23.48581816,239.38482009)(23.62581802,239.10482037)(23.74581787,238.79482544)
\curveto(23.79581785,238.6748208)(23.82581782,238.55482092)(23.83581787,238.43482544)
\curveto(23.85581779,238.32482115)(23.88081777,238.20482127)(23.91081787,238.07482544)
\curveto(23.92081773,238.02482145)(23.92081773,237.96982151)(23.91081787,237.90982544)
\curveto(23.90081775,237.85982162)(23.90581774,237.80982167)(23.92581787,237.75982544)
\curveto(23.9458177,237.65982182)(23.9458177,237.56982191)(23.92581787,237.48982544)
\lineto(23.92581787,237.33982544)
\curveto(23.90581774,237.28982219)(23.89581775,237.22982225)(23.89581787,237.15982544)
\curveto(23.89581775,237.09982238)(23.89081776,237.04982243)(23.88081787,237.00982544)
\curveto(23.86081779,236.96982251)(23.8508178,236.92982255)(23.85081787,236.88982544)
\curveto(23.86081779,236.85982262)(23.85581779,236.81982266)(23.83581787,236.76982544)
\curveto(23.81581783,236.69982278)(23.79581785,236.62482285)(23.77581787,236.54482544)
\curveto(23.75581789,236.474823)(23.72581792,236.40482307)(23.68581787,236.33482544)
\curveto(23.57581807,236.09482338)(23.43081822,235.90482357)(23.25081787,235.76482544)
\curveto(23.06081859,235.63482384)(22.83581881,235.53982394)(22.57581787,235.47982544)
\curveto(22.48581916,235.45982402)(22.39581925,235.44982403)(22.30581787,235.44982544)
\lineto(22.00581787,235.44982544)
\curveto(21.9458197,235.43982404)(21.89081976,235.43982404)(21.84081787,235.44982544)
\curveto(21.78081987,235.46982401)(21.71581993,235.474824)(21.64581787,235.46482544)
\lineto(21.57081787,235.46482544)
\curveto(21.53082012,235.474824)(21.49582015,235.479824)(21.46581787,235.47982544)
\lineto(21.31581787,235.50982544)
\curveto(21.27582037,235.50982397)(21.23082042,235.51482396)(21.18081787,235.52482544)
\curveto(21.12082053,235.54482393)(21.06582058,235.55982392)(21.01581787,235.56982544)
\lineto(20.41581787,235.68982544)
\lineto(17.65581787,236.24482544)
\lineto(16.69581787,236.42482544)
\lineto(16.42581787,236.48482544)
\curveto(16.33582531,236.50482297)(16.26082539,236.53982294)(16.20081787,236.58982544)
\curveto(16.13082552,236.63982284)(16.08082557,236.72482275)(16.05081787,236.84482544)
\curveto(16.04082561,236.86482261)(16.04082561,236.88482259)(16.05081787,236.90482544)
\curveto(16.0508256,236.92482255)(16.0458256,236.94482253)(16.03581787,236.96482544)
}
}
{
\newrgbcolor{curcolor}{0 0 0}
\pscustom[linestyle=none,fillstyle=solid,fillcolor=curcolor]
{
\newpath
\moveto(15.85581787,248.58443481)
\curveto(15.8458258,248.96442824)(15.88582576,249.27442793)(15.97581787,249.51443481)
\curveto(16.06582558,249.76442744)(16.19582545,249.98442722)(16.36581787,250.17443481)
\curveto(16.41582523,250.24442696)(16.48582516,250.29442691)(16.57581787,250.32443481)
\curveto(16.65582499,250.36442684)(16.73082492,250.41442679)(16.80081787,250.47443481)
\curveto(16.82082483,250.49442671)(16.8458248,250.51942668)(16.87581787,250.54943481)
\curveto(16.90582474,250.57942662)(16.91582473,250.62942657)(16.90581787,250.69943481)
\curveto(16.87582477,250.7994264)(16.81582483,250.89442631)(16.72581787,250.98443481)
\curveto(16.62582502,251.08442612)(16.5508251,251.17942602)(16.50081787,251.26943481)
\curveto(16.39082526,251.42942577)(16.29582535,251.59442561)(16.21581787,251.76443481)
\curveto(16.12582552,251.93442527)(16.0508256,252.11442509)(15.99081787,252.30443481)
\curveto(15.96082569,252.38442482)(15.94082571,252.47442473)(15.93081787,252.57443481)
\curveto(15.91082574,252.67442453)(15.89082576,252.76942443)(15.87081787,252.85943481)
\curveto(15.86082579,252.91942428)(15.85582579,252.96942423)(15.85581787,253.00943481)
\lineto(15.85581787,253.15943481)
\curveto(15.83582581,253.20942399)(15.83082582,253.27942392)(15.84081787,253.36943481)
\curveto(15.84082581,253.45942374)(15.8458258,253.52442368)(15.85581787,253.56443481)
\curveto(15.86582578,253.6044236)(15.87082578,253.67442353)(15.87081787,253.77443481)
\curveto(15.89082576,253.86442334)(15.91082574,253.94942325)(15.93081787,254.02943481)
\curveto(15.94082571,254.10942309)(15.96082569,254.18942301)(15.99081787,254.26943481)
\curveto(16.01082564,254.30942289)(16.02582562,254.34942285)(16.03581787,254.38943481)
\curveto(16.03582561,254.43942276)(16.0458256,254.48942271)(16.06581787,254.53943481)
\curveto(16.28582536,254.98942221)(16.62582502,255.26442194)(17.08581787,255.36443481)
\curveto(17.16582448,255.39442181)(17.25582439,255.40942179)(17.35581787,255.40943481)
\curveto(17.4458242,255.40942179)(17.5458241,255.40942179)(17.65581787,255.40943481)
\curveto(17.68582396,255.41942178)(17.72082393,255.41942178)(17.76081787,255.40943481)
\curveto(17.79082386,255.3994218)(17.82082383,255.39442181)(17.85081787,255.39443481)
\lineto(17.98581787,255.36443481)
\curveto(18.02582362,255.36442184)(18.07082358,255.35442185)(18.12081787,255.33443481)
\curveto(18.17082348,255.32442188)(18.22082343,255.31442189)(18.27081787,255.30443481)
\lineto(18.85581787,255.19943481)
\lineto(19.81581787,255.00443481)
\lineto(22.66581787,254.43443481)
\curveto(22.82581882,254.4044228)(23.01581863,254.36942283)(23.23581787,254.32943481)
\curveto(23.45581819,254.28942291)(23.60081805,254.21442299)(23.67081787,254.10443481)
\curveto(23.70081795,254.06442314)(23.72581792,253.99442321)(23.74581787,253.89443481)
\curveto(23.75581789,253.8044234)(23.76081789,253.70942349)(23.76081787,253.60943481)
\curveto(23.76081789,253.50942369)(23.75581789,253.41442379)(23.74581787,253.32443481)
\curveto(23.73581791,253.23442397)(23.71581793,253.16942403)(23.68581787,253.12943481)
\curveto(23.65581799,253.07942412)(23.59581805,253.05442415)(23.50581787,253.05443481)
\curveto(23.4458182,253.04442416)(23.38581826,253.04942415)(23.32581787,253.06943481)
\curveto(23.25581839,253.08942411)(23.19081846,253.0994241)(23.13081787,253.09943481)
\curveto(23.08081857,253.0994241)(23.02581862,253.1044241)(22.96581787,253.11443481)
\curveto(22.89581875,253.13442407)(22.83081882,253.14942405)(22.77081787,253.15943481)
\lineto(22.09581787,253.29443481)
\lineto(19.23081787,253.87943481)
\curveto(18.90082275,253.93942326)(18.59082306,253.98942321)(18.30081787,254.02943481)
\curveto(18.00082365,254.07942312)(17.7508239,254.05942314)(17.55081787,253.96943481)
\curveto(17.31082434,253.86942333)(17.13582451,253.67442353)(17.02581787,253.38443481)
\curveto(17.00582464,253.32442388)(16.99582465,253.26442394)(16.99581787,253.20443481)
\curveto(16.98582466,253.14442406)(16.97082468,253.07942412)(16.95081787,253.00943481)
\curveto(16.94082471,252.96942423)(16.94082471,252.9044243)(16.95081787,252.81443481)
\curveto(16.9508247,252.72442448)(16.95582469,252.66442454)(16.96581787,252.63443481)
\curveto(16.97582467,252.58442462)(16.98082467,252.53442467)(16.98081787,252.48443481)
\curveto(16.97082468,252.43442477)(16.97582467,252.37942482)(16.99581787,252.31943481)
\curveto(17.02582462,252.17942502)(17.06582458,252.03942516)(17.11581787,251.89943481)
\curveto(17.33582431,251.32942587)(17.72082393,250.8994263)(18.27081787,250.60943481)
\curveto(18.44082321,250.52942667)(18.63582301,250.46442674)(18.85581787,250.41443481)
\curveto(19.07582257,250.36442684)(19.30082235,250.31942688)(19.53081787,250.27943481)
\lineto(21.49581787,249.87443481)
\lineto(22.95081787,249.58943481)
\curveto(23.07081858,249.56942763)(23.19581845,249.54442766)(23.32581787,249.51443481)
\curveto(23.4458182,249.49442771)(23.54081811,249.44942775)(23.61081787,249.37943481)
\curveto(23.69081796,249.31942788)(23.73581791,249.22442798)(23.74581787,249.09443481)
\curveto(23.75581789,248.97442823)(23.76081789,248.84942835)(23.76081787,248.71943481)
\curveto(23.76081789,248.54942865)(23.74081791,248.42942877)(23.70081787,248.35943481)
\curveto(23.650818,248.27942892)(23.57081808,248.24442896)(23.46081787,248.25443481)
\curveto(23.34081831,248.26442894)(23.21081844,248.27942892)(23.07081787,248.29943481)
\lineto(21.64581787,248.58443481)
\lineto(19.17081787,249.07943481)
\curveto(18.8508228,249.14942805)(18.55582309,249.199428)(18.28581787,249.22943481)
\curveto(18.00582364,249.26942793)(17.76582388,249.24942795)(17.56581787,249.16943481)
\curveto(17.38582426,249.08942811)(17.25582439,248.98942821)(17.17581787,248.86943481)
\curveto(17.08582456,248.74942845)(17.01582463,248.57442863)(16.96581787,248.34443481)
\curveto(16.95582469,248.3044289)(16.9508247,248.25942894)(16.95081787,248.20943481)
\lineto(16.95081787,248.07443481)
\curveto(16.9508247,247.85442935)(16.97582467,247.65442955)(17.02581787,247.47443481)
\curveto(17.07582457,247.3044299)(17.14082451,247.14443006)(17.22081787,246.99443481)
\curveto(17.53082412,246.42443078)(17.99582365,245.99443121)(18.61581787,245.70443481)
\curveto(18.7458229,245.63443157)(18.89582275,245.58443162)(19.06581787,245.55443481)
\curveto(19.22582242,245.53443167)(19.38582226,245.5044317)(19.54581787,245.46443481)
\lineto(21.24081787,245.13443481)
\lineto(22.89081787,244.80443481)
\curveto(23.02081863,244.77443243)(23.15581849,244.74443246)(23.29581787,244.71443481)
\curveto(23.43581821,244.69443251)(23.5458181,244.64443256)(23.62581787,244.56443481)
\curveto(23.69581795,244.5044327)(23.73581791,244.41943278)(23.74581787,244.30943481)
\curveto(23.75581789,244.20943299)(23.76081789,244.1044331)(23.76081787,243.99443481)
\lineto(23.76081787,243.76943481)
\curveto(23.74081791,243.71943348)(23.72581792,243.66443354)(23.71581787,243.60443481)
\curveto(23.70581794,243.55443365)(23.67581797,243.51443369)(23.62581787,243.48443481)
\curveto(23.56581808,243.44443376)(23.49081816,243.43443377)(23.40081787,243.45443481)
\curveto(23.30081835,243.47443373)(23.19581845,243.49443371)(23.08581787,243.51443481)
\lineto(22.11081787,243.70943481)
\lineto(17.82081787,244.56443481)
\lineto(16.71081787,244.78943481)
\curveto(16.61082504,244.80943239)(16.51582513,244.82943237)(16.42581787,244.84943481)
\curveto(16.32582532,244.86943233)(16.2458254,244.9044323)(16.18581787,244.95443481)
\curveto(16.12582552,244.99443221)(16.08582556,245.04943215)(16.06581787,245.11943481)
\curveto(16.03582561,245.199432)(16.02082563,245.32443188)(16.02081787,245.49443481)
\curveto(16.02082563,245.67443153)(16.03582561,245.8044314)(16.06581787,245.88443481)
\curveto(16.10582554,245.95443125)(16.15582549,245.9994312)(16.21581787,246.01943481)
\curveto(16.26582538,246.02943117)(16.32582532,246.02943117)(16.39581787,246.01943481)
\curveto(16.46582518,246.00943119)(16.53082512,246.0044312)(16.59081787,246.00443481)
\curveto(16.650825,246.0044312)(16.70082495,246.01443119)(16.74081787,246.03443481)
\curveto(16.78082487,246.05443115)(16.80082485,246.0994311)(16.80081787,246.16943481)
\curveto(16.78082487,246.18943101)(16.77082488,246.20943099)(16.77081787,246.22943481)
\curveto(16.77082488,246.25943094)(16.76082489,246.28943091)(16.74081787,246.31943481)
\curveto(16.69082496,246.38943081)(16.64082501,246.45443075)(16.59081787,246.51443481)
\lineto(16.44081787,246.72443481)
\curveto(16.28082537,246.97443023)(16.14082551,247.24942995)(16.02081787,247.54943481)
\curveto(15.98082567,247.65942954)(15.95582569,247.76442944)(15.94581787,247.86443481)
\curveto(15.92582572,247.97442923)(15.90082575,248.08442912)(15.87081787,248.19443481)
\lineto(15.87081787,248.38943481)
\curveto(15.86082579,248.45942874)(15.85582579,248.52442868)(15.85581787,248.58443481)
}
}
{
\newrgbcolor{curcolor}{0 0 0}
\pscustom[linestyle=none,fillstyle=solid,fillcolor=curcolor]
{
\newpath
\moveto(16.03581787,258.11716919)
\lineto(16.03581787,258.55216919)
\curveto(16.03582561,258.70216568)(16.07582557,258.80216558)(16.15581787,258.85216919)
\curveto(16.23582541,258.8821655)(16.33582531,258.88716549)(16.45581787,258.86716919)
\lineto(16.81581787,258.80716919)
\lineto(18.24081787,258.52216919)
\lineto(20.50581787,258.07216919)
\curveto(20.72582092,258.02216636)(20.95582069,257.97216641)(21.19581787,257.92216919)
\curveto(21.42582022,257.8821665)(21.62582002,257.86716651)(21.79581787,257.87716919)
\curveto(22.2458194,257.94716643)(22.56081909,258.18716619)(22.74081787,258.59716919)
\curveto(22.83081882,258.79716558)(22.86581878,259.05216533)(22.84581787,259.36216919)
\curveto(22.81581883,259.6821647)(22.76081889,259.94716443)(22.68081787,260.15716919)
\curveto(22.54081911,260.50716387)(22.36581928,260.80216358)(22.15581787,261.04216919)
\curveto(21.93581971,261.2821631)(21.65082,261.49216289)(21.30081787,261.67216919)
\curveto(21.22082043,261.72216266)(21.14082051,261.75716262)(21.06081787,261.77716919)
\curveto(20.98082067,261.80716257)(20.89582075,261.84216254)(20.80581787,261.88216919)
\curveto(20.75582089,261.90216248)(20.71082094,261.91216247)(20.67081787,261.91216919)
\curveto(20.63082102,261.91216247)(20.58582106,261.92716245)(20.53581787,261.95716919)
\lineto(20.22081787,262.01716919)
\curveto(20.14082151,262.05716232)(20.0508216,262.0821623)(19.95081787,262.09216919)
\curveto(19.84082181,262.10216228)(19.74082191,262.11716226)(19.65081787,262.13716919)
\lineto(18.48081787,262.37716919)
\lineto(16.89081787,262.69216919)
\curveto(16.77082488,262.71216167)(16.645825,262.73216165)(16.51581787,262.75216919)
\curveto(16.37582527,262.7821616)(16.26582538,262.82716155)(16.18581787,262.88716919)
\curveto(16.13582551,262.93716144)(16.10582554,262.99216139)(16.09581787,263.05216919)
\curveto(16.07582557,263.11216127)(16.05582559,263.1821612)(16.03581787,263.26216919)
\lineto(16.03581787,263.48716919)
\curveto(16.03582561,263.60716077)(16.04082561,263.71216067)(16.05081787,263.80216919)
\curveto(16.06082559,263.90216048)(16.10582554,263.96716041)(16.18581787,263.99716919)
\curveto(16.23582541,264.03716034)(16.31082534,264.04716033)(16.41081787,264.02716919)
\curveto(16.50082515,264.00716037)(16.59582505,263.98716039)(16.69581787,263.96716919)
\lineto(17.71581787,263.77216919)
\lineto(21.75081787,262.96216919)
\lineto(23.10081787,262.69216919)
\curveto(23.22081843,262.67216171)(23.33581831,262.64716173)(23.44581787,262.61716919)
\curveto(23.5458181,262.58716179)(23.62081803,262.53216185)(23.67081787,262.45216919)
\curveto(23.70081795,262.41216197)(23.72581792,262.34716203)(23.74581787,262.25716919)
\curveto(23.75581789,262.1771622)(23.76581788,262.08716229)(23.77581787,261.98716919)
\curveto(23.77581787,261.89716248)(23.77081788,261.80716257)(23.76081787,261.71716919)
\curveto(23.7508179,261.63716274)(23.73081792,261.5821628)(23.70081787,261.55216919)
\curveto(23.66081799,261.51216287)(23.59581805,261.4821629)(23.50581787,261.46216919)
\curveto(23.46581818,261.45216293)(23.41081824,261.45216293)(23.34081787,261.46216919)
\curveto(23.27081838,261.47216291)(23.20581844,261.4771629)(23.14581787,261.47716919)
\curveto(23.07581857,261.48716289)(23.02081863,261.4771629)(22.98081787,261.44716919)
\curveto(22.94081871,261.42716295)(22.92581872,261.38716299)(22.93581787,261.32716919)
\curveto(22.95581869,261.24716313)(23.01581863,261.15716322)(23.11581787,261.05716919)
\curveto(23.20581844,260.95716342)(23.27581837,260.86716351)(23.32581787,260.78716919)
\curveto(23.48581816,260.53716384)(23.62581802,260.25716412)(23.74581787,259.94716919)
\curveto(23.79581785,259.82716455)(23.82581782,259.70716467)(23.83581787,259.58716919)
\curveto(23.85581779,259.4771649)(23.88081777,259.35716502)(23.91081787,259.22716919)
\curveto(23.92081773,259.1771652)(23.92081773,259.12216526)(23.91081787,259.06216919)
\curveto(23.90081775,259.01216537)(23.90581774,258.96216542)(23.92581787,258.91216919)
\curveto(23.9458177,258.81216557)(23.9458177,258.72216566)(23.92581787,258.64216919)
\lineto(23.92581787,258.49216919)
\curveto(23.90581774,258.44216594)(23.89581775,258.382166)(23.89581787,258.31216919)
\curveto(23.89581775,258.25216613)(23.89081776,258.20216618)(23.88081787,258.16216919)
\curveto(23.86081779,258.12216626)(23.8508178,258.0821663)(23.85081787,258.04216919)
\curveto(23.86081779,258.01216637)(23.85581779,257.97216641)(23.83581787,257.92216919)
\curveto(23.81581783,257.85216653)(23.79581785,257.7771666)(23.77581787,257.69716919)
\curveto(23.75581789,257.62716675)(23.72581792,257.55716682)(23.68581787,257.48716919)
\curveto(23.57581807,257.24716713)(23.43081822,257.05716732)(23.25081787,256.91716919)
\curveto(23.06081859,256.78716759)(22.83581881,256.69216769)(22.57581787,256.63216919)
\curveto(22.48581916,256.61216777)(22.39581925,256.60216778)(22.30581787,256.60216919)
\lineto(22.00581787,256.60216919)
\curveto(21.9458197,256.59216779)(21.89081976,256.59216779)(21.84081787,256.60216919)
\curveto(21.78081987,256.62216776)(21.71581993,256.62716775)(21.64581787,256.61716919)
\lineto(21.57081787,256.61716919)
\curveto(21.53082012,256.62716775)(21.49582015,256.63216775)(21.46581787,256.63216919)
\lineto(21.31581787,256.66216919)
\curveto(21.27582037,256.66216772)(21.23082042,256.66716771)(21.18081787,256.67716919)
\curveto(21.12082053,256.69716768)(21.06582058,256.71216767)(21.01581787,256.72216919)
\lineto(20.41581787,256.84216919)
\lineto(17.65581787,257.39716919)
\lineto(16.69581787,257.57716919)
\lineto(16.42581787,257.63716919)
\curveto(16.33582531,257.65716672)(16.26082539,257.69216669)(16.20081787,257.74216919)
\curveto(16.13082552,257.79216659)(16.08082557,257.8771665)(16.05081787,257.99716919)
\curveto(16.04082561,258.01716636)(16.04082561,258.03716634)(16.05081787,258.05716919)
\curveto(16.0508256,258.0771663)(16.0458256,258.09716628)(16.03581787,258.11716919)
}
}
{
\newrgbcolor{curcolor}{0 0 0}
\pscustom[linestyle=none,fillstyle=solid,fillcolor=curcolor]
{
\newpath
\moveto(13.08081787,267.17177856)
\curveto(13.08082857,267.3017748)(13.08082857,267.43677467)(13.08081787,267.57677856)
\curveto(13.08082857,267.72677438)(13.11582853,267.82677428)(13.18581787,267.87677856)
\curveto(13.25582839,267.91677419)(13.3508283,267.92677418)(13.47081787,267.90677856)
\curveto(13.58082807,267.88677422)(13.69582795,267.86677424)(13.81581787,267.84677856)
\lineto(15.15081787,267.57677856)
\lineto(21.22581787,266.36177856)
\lineto(22.90581787,266.03177856)
\curveto(23.02581862,266.0017761)(23.15581849,265.97177613)(23.29581787,265.94177856)
\curveto(23.43581821,265.92177618)(23.5458181,265.87677623)(23.62581787,265.80677856)
\curveto(23.67581797,265.76677634)(23.70581794,265.71677639)(23.71581787,265.65677856)
\curveto(23.72581792,265.6067765)(23.74081791,265.53677657)(23.76081787,265.44677856)
\lineto(23.76081787,265.23677856)
\lineto(23.76081787,264.92177856)
\curveto(23.7508179,264.82177728)(23.71581793,264.75677735)(23.65581787,264.72677856)
\curveto(23.57581807,264.68677742)(23.47581817,264.67677743)(23.35581787,264.69677856)
\curveto(23.23581841,264.71677739)(23.11081854,264.74177736)(22.98081787,264.77177856)
\lineto(21.60081787,265.04177856)
\lineto(15.36081787,266.28677856)
\lineto(13.89081787,266.58677856)
\curveto(13.78082787,266.6067755)(13.66582798,266.62677548)(13.54581787,266.64677856)
\curveto(13.41582823,266.66677544)(13.31582833,266.7067754)(13.24581787,266.76677856)
\curveto(13.18582846,266.82677528)(13.13582851,266.91177519)(13.09581787,267.02177856)
\curveto(13.08582856,267.05177505)(13.08582856,267.07677503)(13.09581787,267.09677856)
\curveto(13.09582855,267.11677499)(13.09082856,267.14177496)(13.08081787,267.17177856)
}
}
{
\newrgbcolor{curcolor}{0 0 0}
\pscustom[linestyle=none,fillstyle=solid,fillcolor=curcolor]
{
\newpath
\moveto(23.20581787,274.80162231)
\curveto(23.36581828,274.7916144)(23.50081815,274.74661445)(23.61081787,274.66662231)
\curveto(23.71081794,274.58661461)(23.78581786,274.4916147)(23.83581787,274.38162231)
\curveto(23.85581779,274.33161486)(23.86581778,274.27661492)(23.86581787,274.21662231)
\curveto(23.86581778,274.16661503)(23.87581777,274.10661509)(23.89581787,274.03662231)
\curveto(23.9458177,273.80661539)(23.93081772,273.5916156)(23.85081787,273.39162231)
\curveto(23.78081787,273.191616)(23.69081796,273.06661613)(23.58081787,273.01662231)
\curveto(23.51081814,272.97661622)(23.43081822,272.94661625)(23.34081787,272.92662231)
\curveto(23.24081841,272.90661629)(23.16081849,272.87161632)(23.10081787,272.82162231)
\lineto(23.04081787,272.76162231)
\curveto(23.02081863,272.74161645)(23.01581863,272.71161648)(23.02581787,272.67162231)
\curveto(23.05581859,272.55161664)(23.11081854,272.43661676)(23.19081787,272.32662231)
\curveto(23.27081838,272.21661698)(23.34081831,272.11161708)(23.40081787,272.01162231)
\curveto(23.48081817,271.86161733)(23.55581809,271.70661749)(23.62581787,271.54662231)
\curveto(23.68581796,271.38661781)(23.74081791,271.21661798)(23.79081787,271.03662231)
\curveto(23.82081783,270.92661827)(23.84081781,270.81161838)(23.85081787,270.69162231)
\curveto(23.86081779,270.58161861)(23.87581777,270.46661873)(23.89581787,270.34662231)
\curveto(23.90581774,270.2966189)(23.91081774,270.25161894)(23.91081787,270.21162231)
\lineto(23.91081787,270.10662231)
\curveto(23.93081772,269.9966192)(23.93081772,269.8916193)(23.91081787,269.79162231)
\lineto(23.91081787,269.65662231)
\curveto(23.90081775,269.60661959)(23.89581775,269.55661964)(23.89581787,269.50662231)
\curveto(23.89581775,269.45661974)(23.88581776,269.41661978)(23.86581787,269.38662231)
\curveto(23.85581779,269.34661985)(23.8508178,269.31161988)(23.85081787,269.28162231)
\curveto(23.86081779,269.26161993)(23.86081779,269.23661996)(23.85081787,269.20662231)
\lineto(23.79081787,268.96662231)
\curveto(23.78081787,268.8966203)(23.76081789,268.83162036)(23.73081787,268.77162231)
\curveto(23.60081805,268.4916207)(23.45581819,268.27662092)(23.29581787,268.12662231)
\curveto(23.12581852,267.97662122)(22.89081876,267.87162132)(22.59081787,267.81162231)
\curveto(22.37081928,267.76162143)(22.10581954,267.76662143)(21.79581787,267.82662231)
\lineto(21.48081787,267.90162231)
\curveto(21.43082022,267.92162127)(21.38082027,267.93662126)(21.33081787,267.94662231)
\lineto(21.15081787,268.00662231)
\lineto(20.82081787,268.18662231)
\curveto(20.71082094,268.25662094)(20.61082104,268.32662087)(20.52081787,268.39662231)
\curveto(20.23082142,268.63662056)(20.01582163,268.92662027)(19.87581787,269.26662231)
\curveto(19.73582191,269.60661959)(19.61082204,269.97161922)(19.50081787,270.36162231)
\curveto(19.46082219,270.51161868)(19.43082222,270.66161853)(19.41081787,270.81162231)
\curveto(19.39082226,270.97161822)(19.36582228,271.12661807)(19.33581787,271.27662231)
\curveto(19.31582233,271.35661784)(19.30582234,271.42661777)(19.30581787,271.48662231)
\curveto(19.30582234,271.55661764)(19.29582235,271.63161756)(19.27581787,271.71162231)
\curveto(19.25582239,271.78161741)(19.2458224,271.85161734)(19.24581787,271.92162231)
\curveto(19.23582241,272.00161719)(19.22082243,272.08161711)(19.20081787,272.16162231)
\curveto(19.14082251,272.42161677)(19.09082256,272.66661653)(19.05081787,272.89662231)
\curveto(19.00082265,273.12661607)(18.88582276,273.32661587)(18.70581787,273.49662231)
\curveto(18.62582302,273.56661563)(18.52582312,273.63161556)(18.40581787,273.69162231)
\curveto(18.27582337,273.76161543)(18.13582351,273.7916154)(17.98581787,273.78162231)
\curveto(17.7458239,273.77161542)(17.55582409,273.72161547)(17.41581787,273.63162231)
\curveto(17.27582437,273.55161564)(17.16582448,273.41161578)(17.08581787,273.21162231)
\curveto(17.03582461,273.10161609)(17.00082465,272.96661623)(16.98081787,272.80662231)
\curveto(16.96082469,272.64661655)(16.9508247,272.47661672)(16.95081787,272.29662231)
\curveto(16.9508247,272.11661708)(16.96082469,271.93661726)(16.98081787,271.75662231)
\curveto(17.00082465,271.58661761)(17.03082462,271.43661776)(17.07081787,271.30662231)
\curveto(17.13082452,271.12661807)(17.21582443,270.94661825)(17.32581787,270.76662231)
\curveto(17.38582426,270.67661852)(17.46582418,270.58661861)(17.56581787,270.49662231)
\curveto(17.65582399,270.41661878)(17.75582389,270.34161885)(17.86581787,270.27162231)
\curveto(17.9458237,270.22161897)(18.03082362,270.17661902)(18.12081787,270.13662231)
\curveto(18.21082344,270.0966191)(18.28082337,270.03661916)(18.33081787,269.95662231)
\curveto(18.36082329,269.90661929)(18.38582326,269.83161936)(18.40581787,269.73162231)
\curveto(18.41582323,269.63161956)(18.42082323,269.53161966)(18.42081787,269.43162231)
\curveto(18.42082323,269.33161986)(18.41582323,269.23661996)(18.40581787,269.14662231)
\curveto(18.38582326,269.05662014)(18.36082329,268.9966202)(18.33081787,268.96662231)
\curveto(18.30082335,268.92662027)(18.2508234,268.90162029)(18.18081787,268.89162231)
\curveto(18.11082354,268.8916203)(18.03582361,268.91162028)(17.95581787,268.95162231)
\curveto(17.82582382,269.00162019)(17.70582394,269.05662014)(17.59581787,269.11662231)
\curveto(17.47582417,269.17662002)(17.36082429,269.24161995)(17.25081787,269.31162231)
\curveto(16.90082475,269.57161962)(16.63082502,269.86661933)(16.44081787,270.19662231)
\curveto(16.24082541,270.52661867)(16.08082557,270.91661828)(15.96081787,271.36662231)
\curveto(15.94082571,271.47661772)(15.92582572,271.58161761)(15.91581787,271.68162231)
\curveto(15.90582574,271.7916174)(15.89082576,271.90161729)(15.87081787,272.01162231)
\curveto(15.86082579,272.06161713)(15.86082579,272.12661707)(15.87081787,272.20662231)
\curveto(15.87082578,272.2966169)(15.86082579,272.35661684)(15.84081787,272.38662231)
\curveto(15.83082582,273.08661611)(15.91082574,273.67661552)(16.08081787,274.15662231)
\curveto(16.2508254,274.64661455)(16.57582507,274.95161424)(17.05581787,275.07162231)
\curveto(17.25582439,275.12161407)(17.49082416,275.12661407)(17.76081787,275.08662231)
\curveto(18.02082363,275.04661415)(18.29582335,274.9966142)(18.58581787,274.93662231)
\lineto(21.90081787,274.27662231)
\curveto(22.04081961,274.24661495)(22.17581947,274.22161497)(22.30581787,274.20162231)
\curveto(22.43581921,274.191615)(22.54081911,274.20161499)(22.62081787,274.23162231)
\curveto(22.69081896,274.27161492)(22.74081891,274.32661487)(22.77081787,274.39662231)
\curveto(22.81081884,274.48661471)(22.84081881,274.56661463)(22.86081787,274.63662231)
\curveto(22.87081878,274.71661448)(22.91581873,274.76661443)(22.99581787,274.78662231)
\curveto(23.02581862,274.80661439)(23.05581859,274.81161438)(23.08581787,274.80162231)
\lineto(23.20581787,274.80162231)
\moveto(21.54081787,272.98662231)
\curveto(21.40082025,273.07661612)(21.24082041,273.14161605)(21.06081787,273.18162231)
\curveto(20.87082078,273.22161597)(20.67582097,273.26161593)(20.47581787,273.30162231)
\curveto(20.36582128,273.32161587)(20.26582138,273.33661586)(20.17581787,273.34662231)
\curveto(20.08582156,273.35661584)(20.01582163,273.33161586)(19.96581787,273.27162231)
\curveto(19.9458217,273.24161595)(19.93582171,273.17161602)(19.93581787,273.06162231)
\curveto(19.95582169,273.04161615)(19.96582168,273.00661619)(19.96581787,272.95662231)
\curveto(19.96582168,272.90661629)(19.97582167,272.85661634)(19.99581787,272.80662231)
\curveto(20.01582163,272.72661647)(20.03582161,272.63161656)(20.05581787,272.52162231)
\lineto(20.11581787,272.22162231)
\curveto(20.11582153,272.191617)(20.12082153,272.15661704)(20.13081787,272.11662231)
\lineto(20.13081787,272.01162231)
\curveto(20.17082148,271.85161734)(20.19582145,271.68161751)(20.20581787,271.50162231)
\curveto(20.20582144,271.33161786)(20.22582142,271.16661803)(20.26581787,271.00662231)
\curveto(20.28582136,270.91661828)(20.30582134,270.83661836)(20.32581787,270.76662231)
\curveto(20.33582131,270.70661849)(20.3508213,270.63161856)(20.37081787,270.54162231)
\curveto(20.42082123,270.37161882)(20.48582116,270.20661899)(20.56581787,270.04662231)
\curveto(20.63582101,269.8966193)(20.72582092,269.76161943)(20.83581787,269.64162231)
\curveto(20.9458207,269.52161967)(21.08082057,269.42161977)(21.24081787,269.34162231)
\curveto(21.39082026,269.26161993)(21.57582007,269.20161999)(21.79581787,269.16162231)
\curveto(21.89581975,269.14162005)(21.99081966,269.14162005)(22.08081787,269.16162231)
\curveto(22.16081949,269.18162001)(22.23581941,269.21161998)(22.30581787,269.25162231)
\curveto(22.41581923,269.30161989)(22.51081914,269.38161981)(22.59081787,269.49162231)
\curveto(22.66081899,269.61161958)(22.72081893,269.74161945)(22.77081787,269.88162231)
\curveto(22.78081887,269.93161926)(22.78581886,269.98161921)(22.78581787,270.03162231)
\curveto(22.78581886,270.08161911)(22.79081886,270.13161906)(22.80081787,270.18162231)
\curveto(22.82081883,270.25161894)(22.83581881,270.33661886)(22.84581787,270.43662231)
\curveto(22.8458188,270.53661866)(22.83581881,270.62661857)(22.81581787,270.70662231)
\curveto(22.79581885,270.76661843)(22.79081886,270.82661837)(22.80081787,270.88662231)
\curveto(22.80081885,270.94661825)(22.79081886,271.00661819)(22.77081787,271.06662231)
\curveto(22.7508189,271.15661804)(22.73581891,271.23661796)(22.72581787,271.30662231)
\curveto(22.71581893,271.38661781)(22.69581895,271.46661773)(22.66581787,271.54662231)
\curveto(22.5458191,271.85661734)(22.40081925,272.13161706)(22.23081787,272.37162231)
\curveto(22.06081959,272.61161658)(21.83081982,272.81661638)(21.54081787,272.98662231)
}
}
{
\newrgbcolor{curcolor}{0 0 0}
\pscustom[linestyle=none,fillstyle=solid,fillcolor=curcolor]
{
\newpath
\moveto(22.95081787,282.97826294)
\lineto(23.34081787,282.88826294)
\curveto(23.46081819,282.86825501)(23.56081809,282.82825505)(23.64081787,282.76826294)
\curveto(23.71081794,282.69825518)(23.7508179,282.60325527)(23.76081787,282.48326294)
\lineto(23.76081787,282.13826294)
\curveto(23.76081789,282.0782558)(23.76581788,282.01825586)(23.77581787,281.95826294)
\curveto(23.77581787,281.90825597)(23.76581788,281.86325601)(23.74581787,281.82326294)
\curveto(23.72581792,281.74325613)(23.68581796,281.69325618)(23.62581787,281.67326294)
\curveto(23.57581807,281.64325623)(23.51581813,281.63325624)(23.44581787,281.64326294)
\curveto(23.37581827,281.65325622)(23.30581834,281.64825623)(23.23581787,281.62826294)
\curveto(23.21581843,281.62825625)(23.20081845,281.61825626)(23.19081787,281.59826294)
\lineto(23.13081787,281.56826294)
\curveto(23.12081853,281.46825641)(23.14081851,281.38325649)(23.19081787,281.31326294)
\curveto(23.24081841,281.25325662)(23.29081836,281.18825669)(23.34081787,281.11826294)
\curveto(23.49081816,280.88825699)(23.60581804,280.66325721)(23.68581787,280.44326294)
\curveto(23.76581788,280.25325762)(23.82581782,280.03325784)(23.86581787,279.78326294)
\curveto(23.90581774,279.54325833)(23.92581772,279.29825858)(23.92581787,279.04826294)
\curveto(23.93581771,278.80825907)(23.92081773,278.56825931)(23.88081787,278.32826294)
\curveto(23.8508178,278.09825978)(23.79581785,277.90325997)(23.71581787,277.74326294)
\curveto(23.49581815,277.26326061)(23.20081845,276.89826098)(22.83081787,276.64826294)
\curveto(22.4508192,276.40826147)(21.98081967,276.25326162)(21.42081787,276.18326294)
\curveto(21.33082032,276.16326171)(21.24082041,276.15326172)(21.15081787,276.15326294)
\curveto(21.0508206,276.16326171)(20.9508207,276.16326171)(20.85081787,276.15326294)
\curveto(20.80082085,276.15326172)(20.7508209,276.15826172)(20.70081787,276.16826294)
\curveto(20.650821,276.1782617)(20.60082105,276.18326169)(20.55081787,276.18326294)
\curveto(20.50082115,276.1732617)(20.4508212,276.1732617)(20.40081787,276.18326294)
\curveto(20.34082131,276.20326167)(20.28582136,276.21326166)(20.23581787,276.21326294)
\lineto(20.08581787,276.24326294)
\curveto(20.03582161,276.23326164)(19.97082168,276.23326164)(19.89081787,276.24326294)
\curveto(19.81082184,276.26326161)(19.7458219,276.28826159)(19.69581787,276.31826294)
\lineto(19.53081787,276.36326294)
\curveto(19.46082219,276.39326148)(19.39082226,276.41326146)(19.32081787,276.42326294)
\curveto(19.24082241,276.43326144)(19.16582248,276.45326142)(19.09581787,276.48326294)
\curveto(19.0458226,276.50326137)(19.00082265,276.51826136)(18.96081787,276.52826294)
\curveto(18.92082273,276.53826134)(18.87582277,276.55326132)(18.82581787,276.57326294)
\curveto(18.72582292,276.62326125)(18.63082302,276.66826121)(18.54081787,276.70826294)
\curveto(18.44082321,276.74826113)(18.3458233,276.79326108)(18.25581787,276.84326294)
\curveto(17.87582377,277.04326083)(17.53582411,277.2732606)(17.23581787,277.53326294)
\curveto(16.92582472,277.80326007)(16.67082498,278.10325977)(16.47081787,278.43326294)
\curveto(16.3508253,278.63325924)(16.2508254,278.83325904)(16.17081787,279.03326294)
\curveto(16.09082556,279.23325864)(16.02082563,279.44825843)(15.96081787,279.67826294)
\lineto(15.93081787,279.88826294)
\curveto(15.92082573,279.95825792)(15.90582574,280.02825785)(15.88581787,280.09826294)
\lineto(15.88581787,280.24826294)
\curveto(15.86582578,280.33825754)(15.85582579,280.45825742)(15.85581787,280.60826294)
\curveto(15.85582579,280.76825711)(15.86582578,280.88325699)(15.88581787,280.95326294)
\curveto(15.89582575,280.99325688)(15.90082575,281.04825683)(15.90081787,281.11826294)
\curveto(15.93082572,281.21825666)(15.95582569,281.32325655)(15.97581787,281.43326294)
\curveto(15.98582566,281.54325633)(16.01582563,281.64325623)(16.06581787,281.73326294)
\curveto(16.12582552,281.873256)(16.19082546,282.00325587)(16.26081787,282.12326294)
\curveto(16.33082532,282.24325563)(16.41082524,282.35325552)(16.50081787,282.45326294)
\curveto(16.5508251,282.50325537)(16.60582504,282.55325532)(16.66581787,282.60326294)
\curveto(16.71582493,282.66325521)(16.73082492,282.74825513)(16.71081787,282.85826294)
\lineto(16.63581787,282.93326294)
\curveto(16.61582503,282.95325492)(16.58582506,282.96825491)(16.54581787,282.97826294)
\curveto(16.45582519,283.02825485)(16.34082531,283.06325481)(16.20081787,283.08326294)
\curveto(16.06082559,283.11325476)(15.93582571,283.13825474)(15.82581787,283.15826294)
\lineto(14.10081787,283.50326294)
\curveto(13.96082769,283.53325434)(13.80582784,283.56325431)(13.63581787,283.59326294)
\curveto(13.45582819,283.63325424)(13.32582832,283.68325419)(13.24581787,283.74326294)
\curveto(13.17582847,283.80325407)(13.13082852,283.873254)(13.11081787,283.95326294)
\curveto(13.11082854,283.9732539)(13.11082854,283.99825388)(13.11081787,284.02826294)
\curveto(13.10082855,284.05825382)(13.09582855,284.08325379)(13.09581787,284.10326294)
\curveto(13.08582856,284.25325362)(13.08582856,284.40325347)(13.09581787,284.55326294)
\curveto(13.09582855,284.70325317)(13.13582851,284.80325307)(13.21581787,284.85326294)
\curveto(13.29582835,284.88325299)(13.39582825,284.88325299)(13.51581787,284.85326294)
\curveto(13.63582801,284.83325304)(13.76082789,284.81325306)(13.89081787,284.79326294)
\lineto(22.95081787,282.97826294)
\moveto(20.11581787,282.33326294)
\curveto(20.06582158,282.36325551)(20.00082165,282.38325549)(19.92081787,282.39326294)
\curveto(19.83082182,282.41325546)(19.76082189,282.41825546)(19.71081787,282.40826294)
\lineto(19.48581787,282.45326294)
\curveto(19.39582225,282.45325542)(19.30582234,282.45825542)(19.21581787,282.46826294)
\curveto(19.11582253,282.4782554)(19.02582262,282.4732554)(18.94581787,282.45326294)
\lineto(18.72081787,282.45326294)
\curveto(18.650823,282.45325542)(18.58082307,282.44325543)(18.51081787,282.42326294)
\curveto(18.21082344,282.36325551)(17.9458237,282.25825562)(17.71581787,282.10826294)
\curveto(17.48582416,281.96825591)(17.30582434,281.76825611)(17.17581787,281.50826294)
\curveto(17.12582452,281.41825646)(17.09082456,281.32325655)(17.07081787,281.22326294)
\curveto(17.04082461,281.12325675)(17.01582463,281.01325686)(16.99581787,280.89326294)
\curveto(16.97582467,280.82325705)(16.96582468,280.73825714)(16.96581787,280.63826294)
\lineto(16.96581787,280.36826294)
\lineto(16.99581787,280.21826294)
\lineto(16.99581787,280.08326294)
\curveto(17.01582463,280.00325787)(17.03582461,279.91825796)(17.05581787,279.82826294)
\curveto(17.07582457,279.73825814)(17.10082455,279.65325822)(17.13081787,279.57326294)
\curveto(17.27082438,279.22325865)(17.47582417,278.92325895)(17.74581787,278.67326294)
\curveto(18.00582364,278.42325945)(18.31082334,278.20325967)(18.66081787,278.01326294)
\curveto(18.77082288,277.95325992)(18.88582276,277.90325997)(19.00581787,277.86326294)
\lineto(19.33581787,277.74326294)
\lineto(19.45581787,277.71326294)
\curveto(19.48582216,277.70326017)(19.52082213,277.69326018)(19.56081787,277.68326294)
\curveto(19.61082204,277.65326022)(19.66582198,277.63326024)(19.72581787,277.62326294)
\curveto(19.78582186,277.62326025)(19.84082181,277.61826026)(19.89081787,277.60826294)
\curveto(20.00082165,277.58826029)(20.11082154,277.56326031)(20.22081787,277.53326294)
\curveto(20.32082133,277.51326036)(20.41582123,277.50826037)(20.50581787,277.51826294)
\curveto(20.53582111,277.51826036)(20.58582106,277.51326036)(20.65581787,277.50326294)
\lineto(20.86581787,277.50326294)
\curveto(20.93582071,277.50326037)(21.00582064,277.50826037)(21.07581787,277.51826294)
\curveto(21.42582022,277.55826032)(21.72581992,277.64826023)(21.97581787,277.78826294)
\curveto(22.22581942,277.92825995)(22.43081922,278.12825975)(22.59081787,278.38826294)
\curveto(22.64081901,278.46825941)(22.68081897,278.54825933)(22.71081787,278.62826294)
\curveto(22.74081891,278.71825916)(22.77081888,278.81325906)(22.80081787,278.91326294)
\curveto(22.82081883,278.96325891)(22.82581882,279.01325886)(22.81581787,279.06326294)
\curveto(22.80581884,279.12325875)(22.81081884,279.1782587)(22.83081787,279.22826294)
\curveto(22.84081881,279.25825862)(22.8458188,279.29325858)(22.84581787,279.33326294)
\lineto(22.84581787,279.46826294)
\lineto(22.84581787,279.60326294)
\curveto(22.83581881,279.64325823)(22.83081882,279.69825818)(22.83081787,279.76826294)
\curveto(22.81081884,279.84825803)(22.79581885,279.92825795)(22.78581787,280.00826294)
\curveto(22.76581888,280.09825778)(22.74081891,280.1782577)(22.71081787,280.24826294)
\curveto(22.57081908,280.60825727)(22.39581925,280.91325696)(22.18581787,281.16326294)
\curveto(21.96581968,281.41325646)(21.69081996,281.63825624)(21.36081787,281.83826294)
\curveto(21.2508204,281.90825597)(21.14082051,281.96325591)(21.03081787,282.00326294)
\lineto(20.70081787,282.15326294)
\curveto(20.66082099,282.18325569)(20.62582102,282.19825568)(20.59581787,282.19826294)
\curveto(20.55582109,282.20825567)(20.51582113,282.22325565)(20.47581787,282.24326294)
\curveto(20.41582123,282.26325561)(20.35582129,282.2782556)(20.29581787,282.28826294)
\curveto(20.23582141,282.29825558)(20.17582147,282.31325556)(20.11581787,282.33326294)
}
}
{
\newrgbcolor{curcolor}{0 0 0}
\pscustom[linestyle=none,fillstyle=solid,fillcolor=curcolor]
{
\newpath
\moveto(19.56081787,292.72451294)
\curveto(19.62082203,292.73450405)(19.71582193,292.72450406)(19.84581787,292.69451294)
\curveto(19.96582168,292.67450411)(20.0508216,292.65450413)(20.10081787,292.63451294)
\lineto(20.25081787,292.60451294)
\curveto(20.33082132,292.57450421)(20.40582124,292.54950423)(20.47581787,292.52951294)
\curveto(20.53582111,292.51950426)(20.60582104,292.49950428)(20.68581787,292.46951294)
\curveto(20.7458209,292.43950434)(20.80582084,292.41450437)(20.86581787,292.39451294)
\curveto(20.92582072,292.3845044)(20.98582066,292.35950442)(21.04581787,292.31951294)
\lineto(21.43581787,292.13951294)
\curveto(21.56582008,292.08950469)(21.68581996,292.02450476)(21.79581787,291.94451294)
\curveto(22.27581937,291.64450514)(22.68081897,291.2845055)(23.01081787,290.86451294)
\curveto(23.33081832,290.45450633)(23.57581807,289.97450681)(23.74581787,289.42451294)
\curveto(23.78581786,289.31450747)(23.81581783,289.19450759)(23.83581787,289.06451294)
\curveto(23.85581779,288.93450785)(23.87581777,288.79950798)(23.89581787,288.65951294)
\curveto(23.90581774,288.59950818)(23.91081774,288.53450825)(23.91081787,288.46451294)
\curveto(23.92081773,288.40450838)(23.92581772,288.34450844)(23.92581787,288.28451294)
\curveto(23.93581771,288.24450854)(23.94081771,288.1845086)(23.94081787,288.10451294)
\curveto(23.94081771,288.03450875)(23.93581771,287.9845088)(23.92581787,287.95451294)
\curveto(23.91581773,287.91450887)(23.91081774,287.87450891)(23.91081787,287.83451294)
\curveto(23.92081773,287.79450899)(23.92081773,287.75950902)(23.91081787,287.72951294)
\lineto(23.91081787,287.63951294)
\lineto(23.86581787,287.29451294)
\lineto(23.74581787,286.90451294)
\curveto(23.70581794,286.78451)(23.66081799,286.66951011)(23.61081787,286.55951294)
\curveto(23.41081824,286.14951063)(23.1508185,285.82951095)(22.83081787,285.59951294)
\curveto(22.51081914,285.3795114)(22.12081953,285.21951156)(21.66081787,285.11951294)
\curveto(21.56082009,285.08951169)(21.46082019,285.06951171)(21.36081787,285.05951294)
\lineto(21.04581787,285.05951294)
\curveto(21.00582064,285.04951173)(20.97582067,285.04951173)(20.95581787,285.05951294)
\curveto(20.92582072,285.06951171)(20.89082076,285.07451171)(20.85081787,285.07451294)
\curveto(20.77082088,285.07451171)(20.69082096,285.0795117)(20.61081787,285.08951294)
\curveto(20.52082113,285.09951168)(20.43582121,285.10451168)(20.35581787,285.10451294)
\curveto(20.30582134,285.11451167)(20.26582138,285.11951166)(20.23581787,285.11951294)
\curveto(20.19582145,285.12951165)(20.1508215,285.13451165)(20.10081787,285.13451294)
\curveto(20.0508216,285.13451165)(19.96582168,285.14451164)(19.84581787,285.16451294)
\curveto(19.71582193,285.19451159)(19.62082203,285.22451156)(19.56081787,285.25451294)
\curveto(19.49082216,285.29451149)(19.42082223,285.31451147)(19.35081787,285.31451294)
\curveto(19.28082237,285.31451147)(19.21082244,285.33451145)(19.14081787,285.37451294)
\curveto(19.09082256,285.39451139)(19.0508226,285.40951137)(19.02081787,285.41951294)
\curveto(18.98082267,285.42951135)(18.93582271,285.44451134)(18.88581787,285.46451294)
\curveto(18.76582288,285.52451126)(18.645823,285.57451121)(18.52581787,285.61451294)
\curveto(18.40582324,285.66451112)(18.29082336,285.72951105)(18.18081787,285.80951294)
\curveto(17.81082384,286.02951075)(17.48082417,286.27451051)(17.19081787,286.54451294)
\curveto(16.89082476,286.82450996)(16.64082501,287.13950964)(16.44081787,287.48951294)
\curveto(16.36082529,287.61950916)(16.29582535,287.75450903)(16.24581787,287.89451294)
\lineto(16.06581787,288.34451294)
\curveto(16.01582563,288.47450831)(15.98582566,288.60950817)(15.97581787,288.74951294)
\curveto(15.95582569,288.88950789)(15.92582572,289.03450775)(15.88581787,289.18451294)
\lineto(15.88581787,289.37951294)
\lineto(15.85581787,289.58951294)
\curveto(15.8458258,290.4795063)(16.03082562,291.1795056)(16.41081787,291.68951294)
\curveto(16.79082486,292.20950457)(17.28582436,292.53450425)(17.89581787,292.66451294)
\curveto(17.99582365,292.69450409)(18.09582355,292.71450407)(18.19581787,292.72451294)
\curveto(18.29582335,292.73450405)(18.40082325,292.74950403)(18.51081787,292.76951294)
\curveto(18.62082303,292.779504)(18.74082291,292.779504)(18.87081787,292.76951294)
\lineto(19.24581787,292.76951294)
\curveto(19.29582235,292.76950401)(19.3508223,292.75950402)(19.41081787,292.73951294)
\curveto(19.46082219,292.72950405)(19.51082214,292.72450406)(19.56081787,292.72451294)
\moveto(20.41581787,291.22451294)
\curveto(20.3458213,291.25450553)(20.26582138,291.27450551)(20.17581787,291.28451294)
\curveto(20.08582156,291.30450548)(20.00082165,291.31950546)(19.92081787,291.32951294)
\curveto(19.53082212,291.40950537)(19.20082245,291.44450534)(18.93081787,291.43451294)
\curveto(18.8508228,291.41450537)(18.77082288,291.39950538)(18.69081787,291.38951294)
\curveto(18.61082304,291.38950539)(18.53582311,291.3845054)(18.46581787,291.37451294)
\curveto(17.81582383,291.22450556)(17.36582428,290.86950591)(17.11581787,290.30951294)
\curveto(17.08582456,290.23950654)(17.06582458,290.16450662)(17.05581787,290.08451294)
\curveto(17.03582461,290.01450677)(17.01582463,289.93950684)(16.99581787,289.85951294)
\curveto(16.97582467,289.78950699)(16.96582468,289.70950707)(16.96581787,289.61951294)
\lineto(16.96581787,289.34951294)
\lineto(17.01081787,289.06451294)
\curveto(17.03082462,288.96450782)(17.05582459,288.86950791)(17.08581787,288.77951294)
\curveto(17.10582454,288.68950809)(17.13582451,288.59950818)(17.17581787,288.50951294)
\curveto(17.19582445,288.43950834)(17.22582442,288.36950841)(17.26581787,288.29951294)
\curveto(17.30582434,288.22950855)(17.3458243,288.16450862)(17.38581787,288.10451294)
\curveto(17.55582409,287.83450895)(17.76082389,287.59950918)(18.00081787,287.39951294)
\curveto(18.24082341,287.19950958)(18.52082313,287.01450977)(18.84081787,286.84451294)
\curveto(18.94082271,286.79450999)(19.0458226,286.75451003)(19.15581787,286.72451294)
\curveto(19.25582239,286.69451009)(19.36082229,286.65451013)(19.47081787,286.60451294)
\curveto(19.51082214,286.59451019)(19.57582207,286.5795102)(19.66581787,286.55951294)
\curveto(19.69582195,286.53951024)(19.73082192,286.52951025)(19.77081787,286.52951294)
\curveto(19.81082184,286.52951025)(19.85582179,286.52451026)(19.90581787,286.51451294)
\lineto(20.20581787,286.45451294)
\curveto(20.30582134,286.43451035)(20.39582125,286.42451036)(20.47581787,286.42451294)
\lineto(20.65581787,286.42451294)
\curveto(20.75582089,286.42451036)(20.85582079,286.41951036)(20.95581787,286.40951294)
\curveto(21.0458206,286.40951037)(21.13082052,286.41951036)(21.21081787,286.43951294)
\curveto(21.4508202,286.48951029)(21.67581997,286.55951022)(21.88581787,286.64951294)
\curveto(22.09581955,286.74951003)(22.27081938,286.8845099)(22.41081787,287.05451294)
\curveto(22.44081921,287.10450968)(22.46581918,287.14450964)(22.48581787,287.17451294)
\curveto(22.50581914,287.21450957)(22.53081912,287.25450953)(22.56081787,287.29451294)
\curveto(22.61081904,287.36450942)(22.65581899,287.44450934)(22.69581787,287.53451294)
\curveto(22.72581892,287.62450916)(22.75581889,287.71950906)(22.78581787,287.81951294)
\curveto(22.80581884,287.86950891)(22.81581883,287.91450887)(22.81581787,287.95451294)
\curveto(22.80581884,288.00450878)(22.80581884,288.05450873)(22.81581787,288.10451294)
\curveto(22.82581882,288.13450865)(22.83581881,288.19450859)(22.84581787,288.28451294)
\curveto(22.85581879,288.37450841)(22.8508188,288.44950833)(22.83081787,288.50951294)
\curveto(22.82081883,288.54950823)(22.82081883,288.58950819)(22.83081787,288.62951294)
\curveto(22.83081882,288.66950811)(22.82081883,288.70950807)(22.80081787,288.74951294)
\curveto(22.78081887,288.82950795)(22.76581888,288.90950787)(22.75581787,288.98951294)
\curveto(22.73581891,289.0795077)(22.71081894,289.16450762)(22.68081787,289.24451294)
\curveto(22.54081911,289.60450718)(22.3458193,289.91450687)(22.09581787,290.17451294)
\curveto(21.8458198,290.43450635)(21.5508201,290.66950611)(21.21081787,290.87951294)
\curveto(21.09082056,290.95950582)(20.96582068,291.01950576)(20.83581787,291.05951294)
\curveto(20.69582095,291.09950568)(20.55582109,291.15450563)(20.41581787,291.22451294)
}
}
{
\newrgbcolor{curcolor}{0 0 0}
\pscustom[linestyle=none,fillstyle=solid,fillcolor=curcolor]
{
\newpath
\moveto(108.45952515,72.65651611)
\curveto(108.52952341,72.65650545)(108.60952333,72.65650545)(108.69952515,72.65651611)
\curveto(108.78952315,72.66650544)(108.87452307,72.66650544)(108.95452515,72.65651611)
\curveto(109.0445229,72.65650545)(109.12452282,72.64650546)(109.19452515,72.62651611)
\curveto(109.26452268,72.6065055)(109.31452263,72.57650553)(109.34452515,72.53651611)
\curveto(109.40452254,72.46650564)(109.43452251,72.36650574)(109.43452515,72.23651611)
\curveto(109.4445225,72.11650599)(109.44952249,71.99150611)(109.44952515,71.86151611)
\lineto(109.44952515,70.40651611)
\lineto(109.44952515,64.61651611)
\lineto(109.44952515,62.86151611)
\lineto(109.44952515,62.44151611)
\curveto(109.44952249,62.3015158)(109.42452252,62.19151591)(109.37452515,62.11151611)
\curveto(109.33452261,62.06151604)(109.28452266,62.03151607)(109.22452515,62.02151611)
\curveto(109.17452277,62.01151609)(109.10952283,61.99651611)(109.02952515,61.97651611)
\lineto(108.74452515,61.97651611)
\curveto(108.60452334,61.97651613)(108.47452347,61.98151612)(108.35452515,61.99151611)
\curveto(108.23452371,62.0015161)(108.14952379,62.05151605)(108.09952515,62.14151611)
\curveto(108.05952388,62.2015159)(108.0395239,62.28151582)(108.03952515,62.38151611)
\lineto(108.03952515,62.71151611)
\lineto(108.03952515,63.91151611)
\lineto(108.03952515,70.18151611)
\lineto(108.03952515,71.80151611)
\curveto(108.0395239,71.91150619)(108.03452391,72.03150607)(108.02452515,72.16151611)
\curveto(108.02452392,72.3015058)(108.04952389,72.41150569)(108.09952515,72.49151611)
\curveto(108.1395238,72.56150554)(108.21952372,72.61150549)(108.33952515,72.64151611)
\curveto(108.35952358,72.65150545)(108.37952356,72.65150545)(108.39952515,72.64151611)
\curveto(108.41952352,72.64150546)(108.4395235,72.64650546)(108.45952515,72.65651611)
}
}
{
\newrgbcolor{curcolor}{0 0 0}
\pscustom[linestyle=none,fillstyle=solid,fillcolor=curcolor]
{
\newpath
\moveto(115.29600952,69.88151611)
\curveto(115.92600429,69.9015082)(116.43100378,69.81650829)(116.81100952,69.62651611)
\curveto(117.19100302,69.43650867)(117.49600272,69.15150895)(117.72600952,68.77151611)
\curveto(117.78600243,68.67150943)(117.83100238,68.56150954)(117.86100952,68.44151611)
\curveto(117.90100231,68.33150977)(117.93600228,68.21650989)(117.96600952,68.09651611)
\curveto(118.0160022,67.9065102)(118.04600217,67.7015104)(118.05600952,67.48151611)
\curveto(118.06600215,67.26151084)(118.07100214,67.03651107)(118.07100952,66.80651611)
\lineto(118.07100952,65.20151611)
\lineto(118.07100952,62.86151611)
\curveto(118.07100214,62.69151541)(118.06600215,62.52151558)(118.05600952,62.35151611)
\curveto(118.05600216,62.18151592)(117.99100222,62.07151603)(117.86100952,62.02151611)
\curveto(117.8110024,62.0015161)(117.75600246,61.99151611)(117.69600952,61.99151611)
\curveto(117.64600257,61.98151612)(117.59100262,61.97651613)(117.53100952,61.97651611)
\curveto(117.40100281,61.97651613)(117.27600294,61.98151612)(117.15600952,61.99151611)
\curveto(117.03600318,61.99151611)(116.95100326,62.03151607)(116.90100952,62.11151611)
\curveto(116.85100336,62.18151592)(116.82600339,62.27151583)(116.82600952,62.38151611)
\lineto(116.82600952,62.71151611)
\lineto(116.82600952,64.00151611)
\lineto(116.82600952,66.44651611)
\curveto(116.82600339,66.71651139)(116.82100339,66.98151112)(116.81100952,67.24151611)
\curveto(116.80100341,67.51151059)(116.75600346,67.74151036)(116.67600952,67.93151611)
\curveto(116.59600362,68.13150997)(116.47600374,68.29150981)(116.31600952,68.41151611)
\curveto(116.15600406,68.54150956)(115.97100424,68.64150946)(115.76100952,68.71151611)
\curveto(115.70100451,68.73150937)(115.63600458,68.74150936)(115.56600952,68.74151611)
\curveto(115.50600471,68.75150935)(115.44600477,68.76650934)(115.38600952,68.78651611)
\curveto(115.33600488,68.79650931)(115.25600496,68.79650931)(115.14600952,68.78651611)
\curveto(115.04600517,68.78650932)(114.97600524,68.78150932)(114.93600952,68.77151611)
\curveto(114.89600532,68.75150935)(114.86100535,68.74150936)(114.83100952,68.74151611)
\curveto(114.80100541,68.75150935)(114.76600545,68.75150935)(114.72600952,68.74151611)
\curveto(114.59600562,68.71150939)(114.47100574,68.67650943)(114.35100952,68.63651611)
\curveto(114.24100597,68.6065095)(114.13600608,68.56150954)(114.03600952,68.50151611)
\curveto(113.99600622,68.48150962)(113.96100625,68.46150964)(113.93100952,68.44151611)
\curveto(113.90100631,68.42150968)(113.86600635,68.4015097)(113.82600952,68.38151611)
\curveto(113.47600674,68.13150997)(113.22100699,67.75651035)(113.06100952,67.25651611)
\curveto(113.03100718,67.17651093)(113.0110072,67.09151101)(113.00100952,67.00151611)
\curveto(112.99100722,66.92151118)(112.97600724,66.84151126)(112.95600952,66.76151611)
\curveto(112.93600728,66.71151139)(112.93100728,66.66151144)(112.94100952,66.61151611)
\curveto(112.95100726,66.57151153)(112.94600727,66.53151157)(112.92600952,66.49151611)
\lineto(112.92600952,66.17651611)
\curveto(112.9160073,66.14651196)(112.9110073,66.11151199)(112.91100952,66.07151611)
\curveto(112.92100729,66.03151207)(112.92600729,65.98651212)(112.92600952,65.93651611)
\lineto(112.92600952,65.48651611)
\lineto(112.92600952,64.04651611)
\lineto(112.92600952,62.72651611)
\lineto(112.92600952,62.38151611)
\curveto(112.92600729,62.27151583)(112.90100731,62.18151592)(112.85100952,62.11151611)
\curveto(112.80100741,62.03151607)(112.7110075,61.99151611)(112.58100952,61.99151611)
\curveto(112.46100775,61.98151612)(112.33600788,61.97651613)(112.20600952,61.97651611)
\curveto(112.12600809,61.97651613)(112.05100816,61.98151612)(111.98100952,61.99151611)
\curveto(111.9110083,62.0015161)(111.85100836,62.02651608)(111.80100952,62.06651611)
\curveto(111.72100849,62.11651599)(111.68100853,62.21151589)(111.68100952,62.35151611)
\lineto(111.68100952,62.75651611)
\lineto(111.68100952,64.52651611)
\lineto(111.68100952,68.15651611)
\lineto(111.68100952,69.07151611)
\lineto(111.68100952,69.34151611)
\curveto(111.68100853,69.43150867)(111.70100851,69.5015086)(111.74100952,69.55151611)
\curveto(111.77100844,69.61150849)(111.82100839,69.65150845)(111.89100952,69.67151611)
\curveto(111.93100828,69.68150842)(111.98600823,69.69150841)(112.05600952,69.70151611)
\curveto(112.13600808,69.71150839)(112.216008,69.71650839)(112.29600952,69.71651611)
\curveto(112.37600784,69.71650839)(112.45100776,69.71150839)(112.52100952,69.70151611)
\curveto(112.60100761,69.69150841)(112.65600756,69.67650843)(112.68600952,69.65651611)
\curveto(112.79600742,69.58650852)(112.84600737,69.49650861)(112.83600952,69.38651611)
\curveto(112.82600739,69.28650882)(112.84100737,69.17150893)(112.88100952,69.04151611)
\curveto(112.90100731,68.98150912)(112.94100727,68.93150917)(113.00100952,68.89151611)
\curveto(113.12100709,68.88150922)(113.216007,68.92650918)(113.28600952,69.02651611)
\curveto(113.36600685,69.12650898)(113.44600677,69.2065089)(113.52600952,69.26651611)
\curveto(113.66600655,69.36650874)(113.80600641,69.45650865)(113.94600952,69.53651611)
\curveto(114.09600612,69.62650848)(114.26600595,69.7015084)(114.45600952,69.76151611)
\curveto(114.53600568,69.79150831)(114.62100559,69.81150829)(114.71100952,69.82151611)
\curveto(114.8110054,69.83150827)(114.90600531,69.84650826)(114.99600952,69.86651611)
\curveto(115.04600517,69.87650823)(115.09600512,69.88150822)(115.14600952,69.88151611)
\lineto(115.29600952,69.88151611)
}
}
{
\newrgbcolor{curcolor}{0 0 0}
\pscustom[linestyle=none,fillstyle=solid,fillcolor=curcolor]
{
\newpath
\moveto(119.7306189,69.73151611)
\lineto(120.2106189,69.73151611)
\curveto(120.38061756,69.73150837)(120.51061743,69.7015084)(120.6006189,69.64151611)
\curveto(120.67061727,69.59150851)(120.71561722,69.52650858)(120.7356189,69.44651611)
\curveto(120.76561717,69.37650873)(120.79561714,69.3015088)(120.8256189,69.22151611)
\curveto(120.88561705,69.08150902)(120.935617,68.94150916)(120.9756189,68.80151611)
\curveto(121.01561692,68.66150944)(121.06061688,68.52150958)(121.1106189,68.38151611)
\curveto(121.31061663,67.84151026)(121.49561644,67.29651081)(121.6656189,66.74651611)
\curveto(121.8356161,66.2065119)(122.02061592,65.66651244)(122.2206189,65.12651611)
\curveto(122.29061565,64.94651316)(122.35061559,64.76151334)(122.4006189,64.57151611)
\curveto(122.45061549,64.39151371)(122.51561542,64.21151389)(122.5956189,64.03151611)
\curveto(122.61561532,63.96151414)(122.6406153,63.88651422)(122.6706189,63.80651611)
\curveto(122.70061524,63.72651438)(122.75061519,63.67651443)(122.8206189,63.65651611)
\curveto(122.90061504,63.63651447)(122.96061498,63.67151443)(123.0006189,63.76151611)
\curveto(123.05061489,63.85151425)(123.08561485,63.92151418)(123.1056189,63.97151611)
\curveto(123.18561475,64.16151394)(123.25061469,64.35151375)(123.3006189,64.54151611)
\curveto(123.36061458,64.74151336)(123.42561451,64.94151316)(123.4956189,65.14151611)
\curveto(123.62561431,65.52151258)(123.75061419,65.89651221)(123.8706189,66.26651611)
\curveto(123.99061395,66.64651146)(124.11561382,67.02651108)(124.2456189,67.40651611)
\curveto(124.29561364,67.57651053)(124.34561359,67.74151036)(124.3956189,67.90151611)
\curveto(124.44561349,68.07151003)(124.50561343,68.23650987)(124.5756189,68.39651611)
\curveto(124.62561331,68.53650957)(124.67061327,68.67650943)(124.7106189,68.81651611)
\curveto(124.75061319,68.95650915)(124.79561314,69.09650901)(124.8456189,69.23651611)
\curveto(124.86561307,69.3065088)(124.89061305,69.37650873)(124.9206189,69.44651611)
\curveto(124.95061299,69.51650859)(124.99061295,69.57650853)(125.0406189,69.62651611)
\curveto(125.12061282,69.67650843)(125.21061273,69.7065084)(125.3106189,69.71651611)
\curveto(125.41061253,69.72650838)(125.53061241,69.73150837)(125.6706189,69.73151611)
\curveto(125.7406122,69.73150837)(125.80561213,69.72650838)(125.8656189,69.71651611)
\curveto(125.92561201,69.71650839)(125.98061196,69.7065084)(126.0306189,69.68651611)
\curveto(126.12061182,69.64650846)(126.16561177,69.58150852)(126.1656189,69.49151611)
\curveto(126.17561176,69.4015087)(126.16061178,69.31150879)(126.1206189,69.22151611)
\curveto(126.06061188,69.05150905)(126.00061194,68.87650923)(125.9406189,68.69651611)
\curveto(125.88061206,68.51650959)(125.81061213,68.34150976)(125.7306189,68.17151611)
\curveto(125.71061223,68.12150998)(125.69561224,68.07151003)(125.6856189,68.02151611)
\curveto(125.67561226,67.98151012)(125.66061228,67.93651017)(125.6406189,67.88651611)
\curveto(125.56061238,67.71651039)(125.49561244,67.54151056)(125.4456189,67.36151611)
\curveto(125.39561254,67.18151092)(125.33061261,67.0015111)(125.2506189,66.82151611)
\curveto(125.20061274,66.69151141)(125.15061279,66.55651155)(125.1006189,66.41651611)
\curveto(125.06061288,66.28651182)(125.01061293,66.15651195)(124.9506189,66.02651611)
\curveto(124.78061316,65.61651249)(124.62561331,65.2015129)(124.4856189,64.78151611)
\curveto(124.35561358,64.36151374)(124.20561373,63.94651416)(124.0356189,63.53651611)
\curveto(123.97561396,63.37651473)(123.92061402,63.21651489)(123.8706189,63.05651611)
\curveto(123.82061412,62.89651521)(123.76061418,62.73651537)(123.6906189,62.57651611)
\curveto(123.6406143,62.46651564)(123.59561434,62.36151574)(123.5556189,62.26151611)
\curveto(123.52561441,62.17151593)(123.45561448,62.101516)(123.3456189,62.05151611)
\curveto(123.28561465,62.02151608)(123.21561472,62.0065161)(123.1356189,62.00651611)
\lineto(122.9106189,62.00651611)
\lineto(122.4456189,62.00651611)
\curveto(122.29561564,62.01651609)(122.18561575,62.06651604)(122.1156189,62.15651611)
\curveto(122.04561589,62.23651587)(121.99561594,62.33151577)(121.9656189,62.44151611)
\curveto(121.935616,62.56151554)(121.89561604,62.67651543)(121.8456189,62.78651611)
\curveto(121.78561615,62.92651518)(121.72561621,63.07151503)(121.6656189,63.22151611)
\curveto(121.61561632,63.38151472)(121.56561637,63.53151457)(121.5156189,63.67151611)
\curveto(121.49561644,63.72151438)(121.48061646,63.76151434)(121.4706189,63.79151611)
\curveto(121.46061648,63.83151427)(121.44561649,63.87651423)(121.4256189,63.92651611)
\curveto(121.22561671,64.4065137)(121.0406169,64.89151321)(120.8706189,65.38151611)
\curveto(120.71061723,65.87151223)(120.53061741,66.35651175)(120.3306189,66.83651611)
\curveto(120.27061767,66.99651111)(120.21061773,67.15151095)(120.1506189,67.30151611)
\curveto(120.10061784,67.46151064)(120.04561789,67.62151048)(119.9856189,67.78151611)
\lineto(119.9256189,67.93151611)
\curveto(119.91561802,67.99151011)(119.90061804,68.04651006)(119.8806189,68.09651611)
\curveto(119.80061814,68.26650984)(119.73061821,68.43650967)(119.6706189,68.60651611)
\curveto(119.62061832,68.77650933)(119.56061838,68.94650916)(119.4906189,69.11651611)
\curveto(119.47061847,69.17650893)(119.44561849,69.25650885)(119.4156189,69.35651611)
\curveto(119.38561855,69.45650865)(119.39061855,69.54150856)(119.4306189,69.61151611)
\curveto(119.48061846,69.66150844)(119.5406184,69.69650841)(119.6106189,69.71651611)
\curveto(119.68061826,69.71650839)(119.72061822,69.72150838)(119.7306189,69.73151611)
}
}
{
\newrgbcolor{curcolor}{0 0 0}
\pscustom[linestyle=none,fillstyle=solid,fillcolor=curcolor]
{
\newpath
\moveto(127.7406189,71.23151611)
\curveto(127.66061778,71.29150681)(127.61561782,71.39650671)(127.6056189,71.54651611)
\lineto(127.6056189,72.01151611)
\lineto(127.6056189,72.26651611)
\curveto(127.60561783,72.35650575)(127.62061782,72.43150567)(127.6506189,72.49151611)
\curveto(127.69061775,72.57150553)(127.77061767,72.63150547)(127.8906189,72.67151611)
\curveto(127.91061753,72.68150542)(127.93061751,72.68150542)(127.9506189,72.67151611)
\curveto(127.98061746,72.67150543)(128.00561743,72.67650543)(128.0256189,72.68651611)
\curveto(128.19561724,72.68650542)(128.35561708,72.68150542)(128.5056189,72.67151611)
\curveto(128.65561678,72.66150544)(128.75561668,72.6015055)(128.8056189,72.49151611)
\curveto(128.8356166,72.43150567)(128.85061659,72.35650575)(128.8506189,72.26651611)
\lineto(128.8506189,72.01151611)
\curveto(128.85061659,71.83150627)(128.84561659,71.66150644)(128.8356189,71.50151611)
\curveto(128.8356166,71.34150676)(128.77061667,71.23650687)(128.6406189,71.18651611)
\curveto(128.59061685,71.16650694)(128.5356169,71.15650695)(128.4756189,71.15651611)
\lineto(128.3106189,71.15651611)
\lineto(127.9956189,71.15651611)
\curveto(127.89561754,71.15650695)(127.81061763,71.18150692)(127.7406189,71.23151611)
\moveto(128.8506189,62.72651611)
\lineto(128.8506189,62.41151611)
\curveto(128.86061658,62.31151579)(128.8406166,62.23151587)(128.7906189,62.17151611)
\curveto(128.76061668,62.11151599)(128.71561672,62.07151603)(128.6556189,62.05151611)
\curveto(128.59561684,62.04151606)(128.52561691,62.02651608)(128.4456189,62.00651611)
\lineto(128.2206189,62.00651611)
\curveto(128.09061735,62.0065161)(127.97561746,62.01151609)(127.8756189,62.02151611)
\curveto(127.78561765,62.04151606)(127.71561772,62.09151601)(127.6656189,62.17151611)
\curveto(127.62561781,62.23151587)(127.60561783,62.3065158)(127.6056189,62.39651611)
\lineto(127.6056189,62.68151611)
\lineto(127.6056189,69.02651611)
\lineto(127.6056189,69.34151611)
\curveto(127.60561783,69.45150865)(127.63061781,69.53650857)(127.6806189,69.59651611)
\curveto(127.71061773,69.64650846)(127.75061769,69.67650843)(127.8006189,69.68651611)
\curveto(127.85061759,69.69650841)(127.90561753,69.71150839)(127.9656189,69.73151611)
\curveto(127.98561745,69.73150837)(128.00561743,69.72650838)(128.0256189,69.71651611)
\curveto(128.05561738,69.71650839)(128.08061736,69.72150838)(128.1006189,69.73151611)
\curveto(128.23061721,69.73150837)(128.36061708,69.72650838)(128.4906189,69.71651611)
\curveto(128.63061681,69.71650839)(128.72561671,69.67650843)(128.7756189,69.59651611)
\curveto(128.82561661,69.53650857)(128.85061659,69.45650865)(128.8506189,69.35651611)
\lineto(128.8506189,69.07151611)
\lineto(128.8506189,62.72651611)
}
}
{
\newrgbcolor{curcolor}{0 0 0}
\pscustom[linestyle=none,fillstyle=solid,fillcolor=curcolor]
{
\newpath
\moveto(131.74046265,72.07151611)
\curveto(131.89046064,72.07150603)(132.04046049,72.06650604)(132.19046265,72.05651611)
\curveto(132.34046019,72.05650605)(132.44546008,72.01650609)(132.50546265,71.93651611)
\curveto(132.55545997,71.87650623)(132.58045995,71.79150631)(132.58046265,71.68151611)
\curveto(132.59045994,71.58150652)(132.59545993,71.47650663)(132.59546265,71.36651611)
\lineto(132.59546265,70.49651611)
\curveto(132.59545993,70.41650769)(132.59045994,70.33150777)(132.58046265,70.24151611)
\curveto(132.58045995,70.16150794)(132.59045994,70.09150801)(132.61046265,70.03151611)
\curveto(132.65045988,69.89150821)(132.74045979,69.8015083)(132.88046265,69.76151611)
\curveto(132.9304596,69.75150835)(132.97545955,69.74650836)(133.01546265,69.74651611)
\lineto(133.16546265,69.74651611)
\lineto(133.57046265,69.74651611)
\curveto(133.7304588,69.75650835)(133.84545868,69.74650836)(133.91546265,69.71651611)
\curveto(134.00545852,69.65650845)(134.06545846,69.59650851)(134.09546265,69.53651611)
\curveto(134.11545841,69.49650861)(134.1254584,69.45150865)(134.12546265,69.40151611)
\lineto(134.12546265,69.25151611)
\curveto(134.1254584,69.14150896)(134.12045841,69.03650907)(134.11046265,68.93651611)
\curveto(134.10045843,68.84650926)(134.06545846,68.77650933)(134.00546265,68.72651611)
\curveto(133.94545858,68.67650943)(133.86045867,68.64650946)(133.75046265,68.63651611)
\lineto(133.42046265,68.63651611)
\curveto(133.31045922,68.64650946)(133.20045933,68.65150945)(133.09046265,68.65151611)
\curveto(132.98045955,68.65150945)(132.88545964,68.63650947)(132.80546265,68.60651611)
\curveto(132.73545979,68.57650953)(132.68545984,68.52650958)(132.65546265,68.45651611)
\curveto(132.6254599,68.38650972)(132.60545992,68.3015098)(132.59546265,68.20151611)
\curveto(132.58545994,68.11150999)(132.58045995,68.01151009)(132.58046265,67.90151611)
\curveto(132.59045994,67.8015103)(132.59545993,67.7015104)(132.59546265,67.60151611)
\lineto(132.59546265,64.63151611)
\curveto(132.59545993,64.41151369)(132.59045994,64.17651393)(132.58046265,63.92651611)
\curveto(132.58045995,63.68651442)(132.6254599,63.5015146)(132.71546265,63.37151611)
\curveto(132.76545976,63.29151481)(132.8304597,63.23651487)(132.91046265,63.20651611)
\curveto(132.99045954,63.17651493)(133.08545944,63.15151495)(133.19546265,63.13151611)
\curveto(133.2254593,63.12151498)(133.25545927,63.11651499)(133.28546265,63.11651611)
\curveto(133.3254592,63.12651498)(133.36045917,63.12651498)(133.39046265,63.11651611)
\lineto(133.58546265,63.11651611)
\curveto(133.68545884,63.11651499)(133.77545875,63.106515)(133.85546265,63.08651611)
\curveto(133.94545858,63.07651503)(134.01045852,63.04151506)(134.05046265,62.98151611)
\curveto(134.07045846,62.95151515)(134.08545844,62.89651521)(134.09546265,62.81651611)
\curveto(134.11545841,62.74651536)(134.1254584,62.67151543)(134.12546265,62.59151611)
\curveto(134.13545839,62.51151559)(134.13545839,62.43151567)(134.12546265,62.35151611)
\curveto(134.11545841,62.28151582)(134.09545843,62.22651588)(134.06546265,62.18651611)
\curveto(134.0254585,62.11651599)(133.95045858,62.06651604)(133.84046265,62.03651611)
\curveto(133.76045877,62.01651609)(133.67045886,62.0065161)(133.57046265,62.00651611)
\curveto(133.47045906,62.01651609)(133.38045915,62.02151608)(133.30046265,62.02151611)
\curveto(133.24045929,62.02151608)(133.18045935,62.01651609)(133.12046265,62.00651611)
\curveto(133.06045947,62.0065161)(133.00545952,62.01151609)(132.95546265,62.02151611)
\lineto(132.77546265,62.02151611)
\curveto(132.7254598,62.03151607)(132.67545985,62.03651607)(132.62546265,62.03651611)
\curveto(132.58545994,62.04651606)(132.54045999,62.05151605)(132.49046265,62.05151611)
\curveto(132.29046024,62.101516)(132.11546041,62.15651595)(131.96546265,62.21651611)
\curveto(131.8254607,62.27651583)(131.70546082,62.38151572)(131.60546265,62.53151611)
\curveto(131.46546106,62.73151537)(131.38546114,62.98151512)(131.36546265,63.28151611)
\curveto(131.34546118,63.59151451)(131.33546119,63.92151418)(131.33546265,64.27151611)
\lineto(131.33546265,68.20151611)
\curveto(131.30546122,68.33150977)(131.27546125,68.42650968)(131.24546265,68.48651611)
\curveto(131.2254613,68.54650956)(131.15546137,68.59650951)(131.03546265,68.63651611)
\curveto(130.99546153,68.64650946)(130.95546157,68.64650946)(130.91546265,68.63651611)
\curveto(130.87546165,68.62650948)(130.83546169,68.63150947)(130.79546265,68.65151611)
\lineto(130.55546265,68.65151611)
\curveto(130.4254621,68.65150945)(130.31546221,68.66150944)(130.22546265,68.68151611)
\curveto(130.14546238,68.71150939)(130.09046244,68.77150933)(130.06046265,68.86151611)
\curveto(130.04046249,68.9015092)(130.0254625,68.94650916)(130.01546265,68.99651611)
\lineto(130.01546265,69.14651611)
\curveto(130.01546251,69.28650882)(130.0254625,69.4015087)(130.04546265,69.49151611)
\curveto(130.06546246,69.59150851)(130.1254624,69.66650844)(130.22546265,69.71651611)
\curveto(130.33546219,69.75650835)(130.47546205,69.76650834)(130.64546265,69.74651611)
\curveto(130.8254617,69.72650838)(130.97546155,69.73650837)(131.09546265,69.77651611)
\curveto(131.18546134,69.82650828)(131.25546127,69.89650821)(131.30546265,69.98651611)
\curveto(131.3254612,70.04650806)(131.33546119,70.12150798)(131.33546265,70.21151611)
\lineto(131.33546265,70.46651611)
\lineto(131.33546265,71.39651611)
\lineto(131.33546265,71.63651611)
\curveto(131.33546119,71.72650638)(131.34546118,71.8015063)(131.36546265,71.86151611)
\curveto(131.40546112,71.94150616)(131.48046105,72.0065061)(131.59046265,72.05651611)
\curveto(131.62046091,72.05650605)(131.64546088,72.05650605)(131.66546265,72.05651611)
\curveto(131.69546083,72.06650604)(131.72046081,72.07150603)(131.74046265,72.07151611)
}
}
{
\newrgbcolor{curcolor}{0 0 0}
\pscustom[linestyle=none,fillstyle=solid,fillcolor=curcolor]
{
\newpath
\moveto(142.39725952,62.56151611)
\curveto(142.42725169,62.4015157)(142.41225171,62.26651584)(142.35225952,62.15651611)
\curveto(142.29225183,62.05651605)(142.21225191,61.98151612)(142.11225952,61.93151611)
\curveto(142.06225206,61.91151619)(142.00725211,61.9015162)(141.94725952,61.90151611)
\curveto(141.89725222,61.9015162)(141.84225228,61.89151621)(141.78225952,61.87151611)
\curveto(141.56225256,61.82151628)(141.34225278,61.83651627)(141.12225952,61.91651611)
\curveto(140.91225321,61.98651612)(140.76725335,62.07651603)(140.68725952,62.18651611)
\curveto(140.63725348,62.25651585)(140.59225353,62.33651577)(140.55225952,62.42651611)
\curveto(140.51225361,62.52651558)(140.46225366,62.6065155)(140.40225952,62.66651611)
\curveto(140.38225374,62.68651542)(140.35725376,62.7065154)(140.32725952,62.72651611)
\curveto(140.30725381,62.74651536)(140.27725384,62.75151535)(140.23725952,62.74151611)
\curveto(140.12725399,62.71151539)(140.0222541,62.65651545)(139.92225952,62.57651611)
\curveto(139.83225429,62.49651561)(139.74225438,62.42651568)(139.65225952,62.36651611)
\curveto(139.5222546,62.28651582)(139.38225474,62.21151589)(139.23225952,62.14151611)
\curveto(139.08225504,62.08151602)(138.9222552,62.02651608)(138.75225952,61.97651611)
\curveto(138.65225547,61.94651616)(138.54225558,61.92651618)(138.42225952,61.91651611)
\curveto(138.31225581,61.9065162)(138.20225592,61.89151621)(138.09225952,61.87151611)
\curveto(138.04225608,61.86151624)(137.99725612,61.85651625)(137.95725952,61.85651611)
\lineto(137.85225952,61.85651611)
\curveto(137.74225638,61.83651627)(137.63725648,61.83651627)(137.53725952,61.85651611)
\lineto(137.40225952,61.85651611)
\curveto(137.35225677,61.86651624)(137.30225682,61.87151623)(137.25225952,61.87151611)
\curveto(137.20225692,61.87151623)(137.15725696,61.88151622)(137.11725952,61.90151611)
\curveto(137.07725704,61.91151619)(137.04225708,61.91651619)(137.01225952,61.91651611)
\curveto(136.99225713,61.9065162)(136.96725715,61.9065162)(136.93725952,61.91651611)
\lineto(136.69725952,61.97651611)
\curveto(136.6172575,61.98651612)(136.54225758,62.0065161)(136.47225952,62.03651611)
\curveto(136.17225795,62.16651594)(135.92725819,62.31151579)(135.73725952,62.47151611)
\curveto(135.55725856,62.64151546)(135.40725871,62.87651523)(135.28725952,63.17651611)
\curveto(135.19725892,63.39651471)(135.15225897,63.66151444)(135.15225952,63.97151611)
\lineto(135.15225952,64.28651611)
\curveto(135.16225896,64.33651377)(135.16725895,64.38651372)(135.16725952,64.43651611)
\lineto(135.19725952,64.61651611)
\lineto(135.31725952,64.94651611)
\curveto(135.35725876,65.05651305)(135.40725871,65.15651295)(135.46725952,65.24651611)
\curveto(135.64725847,65.53651257)(135.89225823,65.75151235)(136.20225952,65.89151611)
\curveto(136.51225761,66.03151207)(136.85225727,66.15651195)(137.22225952,66.26651611)
\curveto(137.36225676,66.3065118)(137.50725661,66.33651177)(137.65725952,66.35651611)
\curveto(137.80725631,66.37651173)(137.95725616,66.4015117)(138.10725952,66.43151611)
\curveto(138.17725594,66.45151165)(138.24225588,66.46151164)(138.30225952,66.46151611)
\curveto(138.37225575,66.46151164)(138.44725567,66.47151163)(138.52725952,66.49151611)
\curveto(138.59725552,66.51151159)(138.66725545,66.52151158)(138.73725952,66.52151611)
\curveto(138.80725531,66.53151157)(138.88225524,66.54651156)(138.96225952,66.56651611)
\curveto(139.21225491,66.62651148)(139.44725467,66.67651143)(139.66725952,66.71651611)
\curveto(139.88725423,66.76651134)(140.06225406,66.88151122)(140.19225952,67.06151611)
\curveto(140.25225387,67.14151096)(140.30225382,67.24151086)(140.34225952,67.36151611)
\curveto(140.38225374,67.49151061)(140.38225374,67.63151047)(140.34225952,67.78151611)
\curveto(140.28225384,68.02151008)(140.19225393,68.21150989)(140.07225952,68.35151611)
\curveto(139.96225416,68.49150961)(139.80225432,68.6015095)(139.59225952,68.68151611)
\curveto(139.47225465,68.73150937)(139.32725479,68.76650934)(139.15725952,68.78651611)
\curveto(138.99725512,68.8065093)(138.82725529,68.81650929)(138.64725952,68.81651611)
\curveto(138.46725565,68.81650929)(138.29225583,68.8065093)(138.12225952,68.78651611)
\curveto(137.95225617,68.76650934)(137.80725631,68.73650937)(137.68725952,68.69651611)
\curveto(137.5172566,68.63650947)(137.35225677,68.55150955)(137.19225952,68.44151611)
\curveto(137.11225701,68.38150972)(137.03725708,68.3015098)(136.96725952,68.20151611)
\curveto(136.90725721,68.11150999)(136.85225727,68.01151009)(136.80225952,67.90151611)
\curveto(136.77225735,67.82151028)(136.74225738,67.73651037)(136.71225952,67.64651611)
\curveto(136.69225743,67.55651055)(136.64725747,67.48651062)(136.57725952,67.43651611)
\curveto(136.53725758,67.4065107)(136.46725765,67.38151072)(136.36725952,67.36151611)
\curveto(136.27725784,67.35151075)(136.18225794,67.34651076)(136.08225952,67.34651611)
\curveto(135.98225814,67.34651076)(135.88225824,67.35151075)(135.78225952,67.36151611)
\curveto(135.69225843,67.38151072)(135.62725849,67.4065107)(135.58725952,67.43651611)
\curveto(135.54725857,67.46651064)(135.5172586,67.51651059)(135.49725952,67.58651611)
\curveto(135.47725864,67.65651045)(135.47725864,67.73151037)(135.49725952,67.81151611)
\curveto(135.52725859,67.94151016)(135.55725856,68.06151004)(135.58725952,68.17151611)
\curveto(135.62725849,68.29150981)(135.67225845,68.4065097)(135.72225952,68.51651611)
\curveto(135.91225821,68.86650924)(136.15225797,69.13650897)(136.44225952,69.32651611)
\curveto(136.73225739,69.52650858)(137.09225703,69.68650842)(137.52225952,69.80651611)
\curveto(137.6222565,69.82650828)(137.7222564,69.84150826)(137.82225952,69.85151611)
\curveto(137.93225619,69.86150824)(138.04225608,69.87650823)(138.15225952,69.89651611)
\curveto(138.19225593,69.9065082)(138.25725586,69.9065082)(138.34725952,69.89651611)
\curveto(138.43725568,69.89650821)(138.49225563,69.9065082)(138.51225952,69.92651611)
\curveto(139.21225491,69.93650817)(139.8222543,69.85650825)(140.34225952,69.68651611)
\curveto(140.86225326,69.51650859)(141.22725289,69.19150891)(141.43725952,68.71151611)
\curveto(141.52725259,68.51150959)(141.57725254,68.27650983)(141.58725952,68.00651611)
\curveto(141.60725251,67.74651036)(141.6172525,67.47151063)(141.61725952,67.18151611)
\lineto(141.61725952,63.86651611)
\curveto(141.6172525,63.72651438)(141.6222525,63.59151451)(141.63225952,63.46151611)
\curveto(141.64225248,63.33151477)(141.67225245,63.22651488)(141.72225952,63.14651611)
\curveto(141.77225235,63.07651503)(141.83725228,63.02651508)(141.91725952,62.99651611)
\curveto(142.00725211,62.95651515)(142.09225203,62.92651518)(142.17225952,62.90651611)
\curveto(142.25225187,62.89651521)(142.31225181,62.85151525)(142.35225952,62.77151611)
\curveto(142.37225175,62.74151536)(142.38225174,62.71151539)(142.38225952,62.68151611)
\curveto(142.38225174,62.65151545)(142.38725173,62.61151549)(142.39725952,62.56151611)
\moveto(140.25225952,64.22651611)
\curveto(140.31225381,64.36651374)(140.34225378,64.52651358)(140.34225952,64.70651611)
\curveto(140.35225377,64.89651321)(140.35725376,65.09151301)(140.35725952,65.29151611)
\curveto(140.35725376,65.4015127)(140.35225377,65.5015126)(140.34225952,65.59151611)
\curveto(140.33225379,65.68151242)(140.29225383,65.75151235)(140.22225952,65.80151611)
\curveto(140.19225393,65.82151228)(140.122254,65.83151227)(140.01225952,65.83151611)
\curveto(139.99225413,65.81151229)(139.95725416,65.8015123)(139.90725952,65.80151611)
\curveto(139.85725426,65.8015123)(139.81225431,65.79151231)(139.77225952,65.77151611)
\curveto(139.69225443,65.75151235)(139.60225452,65.73151237)(139.50225952,65.71151611)
\lineto(139.20225952,65.65151611)
\curveto(139.17225495,65.65151245)(139.13725498,65.64651246)(139.09725952,65.63651611)
\lineto(138.99225952,65.63651611)
\curveto(138.84225528,65.59651251)(138.67725544,65.57151253)(138.49725952,65.56151611)
\curveto(138.32725579,65.56151254)(138.16725595,65.54151256)(138.01725952,65.50151611)
\curveto(137.93725618,65.48151262)(137.86225626,65.46151264)(137.79225952,65.44151611)
\curveto(137.73225639,65.43151267)(137.66225646,65.41651269)(137.58225952,65.39651611)
\curveto(137.4222567,65.34651276)(137.27225685,65.28151282)(137.13225952,65.20151611)
\curveto(136.99225713,65.13151297)(136.87225725,65.04151306)(136.77225952,64.93151611)
\curveto(136.67225745,64.82151328)(136.59725752,64.68651342)(136.54725952,64.52651611)
\curveto(136.49725762,64.37651373)(136.47725764,64.19151391)(136.48725952,63.97151611)
\curveto(136.48725763,63.87151423)(136.50225762,63.77651433)(136.53225952,63.68651611)
\curveto(136.57225755,63.6065145)(136.6172575,63.53151457)(136.66725952,63.46151611)
\curveto(136.74725737,63.35151475)(136.85225727,63.25651485)(136.98225952,63.17651611)
\curveto(137.11225701,63.106515)(137.25225687,63.04651506)(137.40225952,62.99651611)
\curveto(137.45225667,62.98651512)(137.50225662,62.98151512)(137.55225952,62.98151611)
\curveto(137.60225652,62.98151512)(137.65225647,62.97651513)(137.70225952,62.96651611)
\curveto(137.77225635,62.94651516)(137.85725626,62.93151517)(137.95725952,62.92151611)
\curveto(138.06725605,62.92151518)(138.15725596,62.93151517)(138.22725952,62.95151611)
\curveto(138.28725583,62.97151513)(138.34725577,62.97651513)(138.40725952,62.96651611)
\curveto(138.46725565,62.96651514)(138.52725559,62.97651513)(138.58725952,62.99651611)
\curveto(138.66725545,63.01651509)(138.74225538,63.03151507)(138.81225952,63.04151611)
\curveto(138.89225523,63.05151505)(138.96725515,63.07151503)(139.03725952,63.10151611)
\curveto(139.32725479,63.22151488)(139.57225455,63.36651474)(139.77225952,63.53651611)
\curveto(139.98225414,63.7065144)(140.14225398,63.93651417)(140.25225952,64.22651611)
}
}
{
\newrgbcolor{curcolor}{0 0 0}
\pscustom[linestyle=none,fillstyle=solid,fillcolor=curcolor]
{
\newpath
\moveto(150.52890015,62.81651611)
\lineto(150.52890015,62.42651611)
\curveto(150.52889227,62.3065158)(150.5038923,62.2065159)(150.45390015,62.12651611)
\curveto(150.4038924,62.05651605)(150.31889248,62.01651609)(150.19890015,62.00651611)
\lineto(149.85390015,62.00651611)
\curveto(149.79389301,62.0065161)(149.73389307,62.0015161)(149.67390015,61.99151611)
\curveto(149.62389318,61.99151611)(149.57889322,62.0015161)(149.53890015,62.02151611)
\curveto(149.44889335,62.04151606)(149.38889341,62.08151602)(149.35890015,62.14151611)
\curveto(149.31889348,62.19151591)(149.29389351,62.25151585)(149.28390015,62.32151611)
\curveto(149.28389352,62.39151571)(149.26889353,62.46151564)(149.23890015,62.53151611)
\curveto(149.22889357,62.55151555)(149.21389359,62.56651554)(149.19390015,62.57651611)
\curveto(149.18389362,62.59651551)(149.16889363,62.61651549)(149.14890015,62.63651611)
\curveto(149.04889375,62.64651546)(148.96889383,62.62651548)(148.90890015,62.57651611)
\curveto(148.85889394,62.52651558)(148.803894,62.47651563)(148.74390015,62.42651611)
\curveto(148.54389426,62.27651583)(148.34389446,62.16151594)(148.14390015,62.08151611)
\curveto(147.96389484,62.0015161)(147.75389505,61.94151616)(147.51390015,61.90151611)
\curveto(147.28389552,61.86151624)(147.04389576,61.84151626)(146.79390015,61.84151611)
\curveto(146.55389625,61.83151627)(146.31389649,61.84651626)(146.07390015,61.88651611)
\curveto(145.83389697,61.91651619)(145.62389718,61.97151613)(145.44390015,62.05151611)
\curveto(144.92389788,62.27151583)(144.5038983,62.56651554)(144.18390015,62.93651611)
\curveto(143.86389894,63.31651479)(143.61389919,63.78651432)(143.43390015,64.34651611)
\curveto(143.39389941,64.43651367)(143.36389944,64.52651358)(143.34390015,64.61651611)
\curveto(143.33389947,64.71651339)(143.31389949,64.81651329)(143.28390015,64.91651611)
\curveto(143.27389953,64.96651314)(143.26889953,65.01651309)(143.26890015,65.06651611)
\curveto(143.26889953,65.11651299)(143.26389954,65.16651294)(143.25390015,65.21651611)
\curveto(143.23389957,65.26651284)(143.22389958,65.31651279)(143.22390015,65.36651611)
\curveto(143.23389957,65.42651268)(143.23389957,65.48151262)(143.22390015,65.53151611)
\lineto(143.22390015,65.68151611)
\curveto(143.2038996,65.73151237)(143.19389961,65.79651231)(143.19390015,65.87651611)
\curveto(143.19389961,65.95651215)(143.2038996,66.02151208)(143.22390015,66.07151611)
\lineto(143.22390015,66.23651611)
\curveto(143.24389956,66.3065118)(143.24889955,66.37651173)(143.23890015,66.44651611)
\curveto(143.23889956,66.52651158)(143.24889955,66.6015115)(143.26890015,66.67151611)
\curveto(143.27889952,66.72151138)(143.28389952,66.76651134)(143.28390015,66.80651611)
\curveto(143.28389952,66.84651126)(143.28889951,66.89151121)(143.29890015,66.94151611)
\curveto(143.32889947,67.04151106)(143.35389945,67.13651097)(143.37390015,67.22651611)
\curveto(143.39389941,67.32651078)(143.41889938,67.42151068)(143.44890015,67.51151611)
\curveto(143.57889922,67.89151021)(143.74389906,68.23150987)(143.94390015,68.53151611)
\curveto(144.15389865,68.84150926)(144.4038984,69.09650901)(144.69390015,69.29651611)
\curveto(144.86389794,69.41650869)(145.03889776,69.51650859)(145.21890015,69.59651611)
\curveto(145.40889739,69.67650843)(145.61389719,69.74650836)(145.83390015,69.80651611)
\curveto(145.9038969,69.81650829)(145.96889683,69.82650828)(146.02890015,69.83651611)
\curveto(146.0988967,69.84650826)(146.16889663,69.86150824)(146.23890015,69.88151611)
\lineto(146.38890015,69.88151611)
\curveto(146.46889633,69.9015082)(146.58389622,69.91150819)(146.73390015,69.91151611)
\curveto(146.89389591,69.91150819)(147.01389579,69.9015082)(147.09390015,69.88151611)
\curveto(147.13389567,69.87150823)(147.18889561,69.86650824)(147.25890015,69.86651611)
\curveto(147.36889543,69.83650827)(147.47889532,69.81150829)(147.58890015,69.79151611)
\curveto(147.6988951,69.78150832)(147.803895,69.75150835)(147.90390015,69.70151611)
\curveto(148.05389475,69.64150846)(148.19389461,69.57650853)(148.32390015,69.50651611)
\curveto(148.46389434,69.43650867)(148.59389421,69.35650875)(148.71390015,69.26651611)
\curveto(148.77389403,69.21650889)(148.83389397,69.16150894)(148.89390015,69.10151611)
\curveto(148.96389384,69.05150905)(149.05389375,69.03650907)(149.16390015,69.05651611)
\curveto(149.18389362,69.08650902)(149.1988936,69.11150899)(149.20890015,69.13151611)
\curveto(149.22889357,69.15150895)(149.24389356,69.18150892)(149.25390015,69.22151611)
\curveto(149.28389352,69.31150879)(149.29389351,69.42650868)(149.28390015,69.56651611)
\lineto(149.28390015,69.94151611)
\lineto(149.28390015,71.66651611)
\lineto(149.28390015,72.13151611)
\curveto(149.28389352,72.31150579)(149.30889349,72.44150566)(149.35890015,72.52151611)
\curveto(149.3988934,72.59150551)(149.45889334,72.63650547)(149.53890015,72.65651611)
\curveto(149.55889324,72.65650545)(149.58389322,72.65650545)(149.61390015,72.65651611)
\curveto(149.64389316,72.66650544)(149.66889313,72.67150543)(149.68890015,72.67151611)
\curveto(149.82889297,72.68150542)(149.97389283,72.68150542)(150.12390015,72.67151611)
\curveto(150.28389252,72.67150543)(150.39389241,72.63150547)(150.45390015,72.55151611)
\curveto(150.5038923,72.47150563)(150.52889227,72.37150573)(150.52890015,72.25151611)
\lineto(150.52890015,71.87651611)
\lineto(150.52890015,62.81651611)
\moveto(149.31390015,65.65151611)
\curveto(149.33389347,65.7015124)(149.34389346,65.76651234)(149.34390015,65.84651611)
\curveto(149.34389346,65.93651217)(149.33389347,66.0065121)(149.31390015,66.05651611)
\lineto(149.31390015,66.28151611)
\curveto(149.29389351,66.37151173)(149.27889352,66.46151164)(149.26890015,66.55151611)
\curveto(149.25889354,66.65151145)(149.23889356,66.74151136)(149.20890015,66.82151611)
\curveto(149.18889361,66.9015112)(149.16889363,66.97651113)(149.14890015,67.04651611)
\curveto(149.13889366,67.11651099)(149.11889368,67.18651092)(149.08890015,67.25651611)
\curveto(148.96889383,67.55651055)(148.81389399,67.82151028)(148.62390015,68.05151611)
\curveto(148.43389437,68.28150982)(148.19389461,68.46150964)(147.90390015,68.59151611)
\curveto(147.803895,68.64150946)(147.6988951,68.67650943)(147.58890015,68.69651611)
\curveto(147.48889531,68.72650938)(147.37889542,68.75150935)(147.25890015,68.77151611)
\curveto(147.17889562,68.79150931)(147.08889571,68.8015093)(146.98890015,68.80151611)
\lineto(146.71890015,68.80151611)
\curveto(146.66889613,68.79150931)(146.62389618,68.78150932)(146.58390015,68.77151611)
\lineto(146.44890015,68.77151611)
\curveto(146.36889643,68.75150935)(146.28389652,68.73150937)(146.19390015,68.71151611)
\curveto(146.11389669,68.69150941)(146.03389677,68.66650944)(145.95390015,68.63651611)
\curveto(145.63389717,68.49650961)(145.37389743,68.29150981)(145.17390015,68.02151611)
\curveto(144.98389782,67.76151034)(144.82889797,67.45651065)(144.70890015,67.10651611)
\curveto(144.66889813,66.99651111)(144.63889816,66.88151122)(144.61890015,66.76151611)
\curveto(144.60889819,66.65151145)(144.59389821,66.54151156)(144.57390015,66.43151611)
\curveto(144.57389823,66.39151171)(144.56889823,66.35151175)(144.55890015,66.31151611)
\lineto(144.55890015,66.20651611)
\curveto(144.53889826,66.15651195)(144.52889827,66.101512)(144.52890015,66.04151611)
\curveto(144.53889826,65.98151212)(144.54389826,65.92651218)(144.54390015,65.87651611)
\lineto(144.54390015,65.54651611)
\curveto(144.54389826,65.44651266)(144.55389825,65.35151275)(144.57390015,65.26151611)
\curveto(144.58389822,65.23151287)(144.58889821,65.18151292)(144.58890015,65.11151611)
\curveto(144.60889819,65.04151306)(144.62389818,64.97151313)(144.63390015,64.90151611)
\lineto(144.69390015,64.69151611)
\curveto(144.803898,64.34151376)(144.95389785,64.04151406)(145.14390015,63.79151611)
\curveto(145.33389747,63.54151456)(145.57389723,63.33651477)(145.86390015,63.17651611)
\curveto(145.95389685,63.12651498)(146.04389676,63.08651502)(146.13390015,63.05651611)
\curveto(146.22389658,63.02651508)(146.32389648,62.99651511)(146.43390015,62.96651611)
\curveto(146.48389632,62.94651516)(146.53389627,62.94151516)(146.58390015,62.95151611)
\curveto(146.64389616,62.96151514)(146.6988961,62.95651515)(146.74890015,62.93651611)
\curveto(146.78889601,62.92651518)(146.82889597,62.92151518)(146.86890015,62.92151611)
\lineto(147.00390015,62.92151611)
\lineto(147.13890015,62.92151611)
\curveto(147.16889563,62.93151517)(147.21889558,62.93651517)(147.28890015,62.93651611)
\curveto(147.36889543,62.95651515)(147.44889535,62.97151513)(147.52890015,62.98151611)
\curveto(147.60889519,63.0015151)(147.68389512,63.02651508)(147.75390015,63.05651611)
\curveto(148.08389472,63.19651491)(148.34889445,63.37151473)(148.54890015,63.58151611)
\curveto(148.75889404,63.8015143)(148.93389387,64.07651403)(149.07390015,64.40651611)
\curveto(149.12389368,64.51651359)(149.15889364,64.62651348)(149.17890015,64.73651611)
\curveto(149.1988936,64.84651326)(149.22389358,64.95651315)(149.25390015,65.06651611)
\curveto(149.27389353,65.106513)(149.28389352,65.14151296)(149.28390015,65.17151611)
\curveto(149.28389352,65.21151289)(149.28889351,65.25151285)(149.29890015,65.29151611)
\curveto(149.30889349,65.35151275)(149.30889349,65.41151269)(149.29890015,65.47151611)
\curveto(149.2988935,65.53151257)(149.3038935,65.59151251)(149.31390015,65.65151611)
}
}
{
\newrgbcolor{curcolor}{0 0 0}
\pscustom[linestyle=none,fillstyle=solid,fillcolor=curcolor]
{
\newpath
\moveto(159.60015015,66.20651611)
\curveto(159.62014209,66.14651196)(159.63014208,66.05151205)(159.63015015,65.92151611)
\curveto(159.63014208,65.8015123)(159.62514208,65.71651239)(159.61515015,65.66651611)
\lineto(159.61515015,65.51651611)
\curveto(159.6051421,65.43651267)(159.59514211,65.36151274)(159.58515015,65.29151611)
\curveto(159.58514212,65.23151287)(159.58014213,65.16151294)(159.57015015,65.08151611)
\curveto(159.55014216,65.02151308)(159.53514217,64.96151314)(159.52515015,64.90151611)
\curveto(159.52514218,64.84151326)(159.51514219,64.78151332)(159.49515015,64.72151611)
\curveto(159.45514225,64.59151351)(159.42014229,64.46151364)(159.39015015,64.33151611)
\curveto(159.36014235,64.2015139)(159.32014239,64.08151402)(159.27015015,63.97151611)
\curveto(159.06014265,63.49151461)(158.78014293,63.08651502)(158.43015015,62.75651611)
\curveto(158.08014363,62.43651567)(157.65014406,62.19151591)(157.14015015,62.02151611)
\curveto(157.03014468,61.98151612)(156.9101448,61.95151615)(156.78015015,61.93151611)
\curveto(156.66014505,61.91151619)(156.53514517,61.89151621)(156.40515015,61.87151611)
\curveto(156.34514536,61.86151624)(156.28014543,61.85651625)(156.21015015,61.85651611)
\curveto(156.15014556,61.84651626)(156.09014562,61.84151626)(156.03015015,61.84151611)
\curveto(155.99014572,61.83151627)(155.93014578,61.82651628)(155.85015015,61.82651611)
\curveto(155.78014593,61.82651628)(155.73014598,61.83151627)(155.70015015,61.84151611)
\curveto(155.66014605,61.85151625)(155.62014609,61.85651625)(155.58015015,61.85651611)
\curveto(155.54014617,61.84651626)(155.5051462,61.84651626)(155.47515015,61.85651611)
\lineto(155.38515015,61.85651611)
\lineto(155.02515015,61.90151611)
\curveto(154.88514682,61.94151616)(154.75014696,61.98151612)(154.62015015,62.02151611)
\curveto(154.49014722,62.06151604)(154.36514734,62.106516)(154.24515015,62.15651611)
\curveto(153.79514791,62.35651575)(153.42514828,62.61651549)(153.13515015,62.93651611)
\curveto(152.84514886,63.25651485)(152.6051491,63.64651446)(152.41515015,64.10651611)
\curveto(152.36514934,64.2065139)(152.32514938,64.3065138)(152.29515015,64.40651611)
\curveto(152.27514943,64.5065136)(152.25514945,64.61151349)(152.23515015,64.72151611)
\curveto(152.21514949,64.76151334)(152.2051495,64.79151331)(152.20515015,64.81151611)
\curveto(152.21514949,64.84151326)(152.21514949,64.87651323)(152.20515015,64.91651611)
\curveto(152.18514952,64.99651311)(152.17014954,65.07651303)(152.16015015,65.15651611)
\curveto(152.16014955,65.24651286)(152.15014956,65.33151277)(152.13015015,65.41151611)
\lineto(152.13015015,65.53151611)
\curveto(152.13014958,65.57151253)(152.12514958,65.61651249)(152.11515015,65.66651611)
\curveto(152.1051496,65.71651239)(152.10014961,65.8015123)(152.10015015,65.92151611)
\curveto(152.10014961,66.05151205)(152.1101496,66.14651196)(152.13015015,66.20651611)
\curveto(152.15014956,66.27651183)(152.15514955,66.34651176)(152.14515015,66.41651611)
\curveto(152.13514957,66.48651162)(152.14014957,66.55651155)(152.16015015,66.62651611)
\curveto(152.17014954,66.67651143)(152.17514953,66.71651139)(152.17515015,66.74651611)
\curveto(152.18514952,66.78651132)(152.19514951,66.83151127)(152.20515015,66.88151611)
\curveto(152.23514947,67.0015111)(152.26014945,67.12151098)(152.28015015,67.24151611)
\curveto(152.3101494,67.36151074)(152.35014936,67.47651063)(152.40015015,67.58651611)
\curveto(152.55014916,67.95651015)(152.73014898,68.28650982)(152.94015015,68.57651611)
\curveto(153.16014855,68.87650923)(153.42514828,69.12650898)(153.73515015,69.32651611)
\curveto(153.85514785,69.4065087)(153.98014773,69.47150863)(154.11015015,69.52151611)
\curveto(154.24014747,69.58150852)(154.37514733,69.64150846)(154.51515015,69.70151611)
\curveto(154.63514707,69.75150835)(154.76514694,69.78150832)(154.90515015,69.79151611)
\curveto(155.04514666,69.81150829)(155.18514652,69.84150826)(155.32515015,69.88151611)
\lineto(155.52015015,69.88151611)
\curveto(155.59014612,69.89150821)(155.65514605,69.9015082)(155.71515015,69.91151611)
\curveto(156.6051451,69.92150818)(157.34514436,69.73650837)(157.93515015,69.35651611)
\curveto(158.52514318,68.97650913)(158.95014276,68.48150962)(159.21015015,67.87151611)
\curveto(159.26014245,67.77151033)(159.30014241,67.67151043)(159.33015015,67.57151611)
\curveto(159.36014235,67.47151063)(159.39514231,67.36651074)(159.43515015,67.25651611)
\curveto(159.46514224,67.14651096)(159.49014222,67.02651108)(159.51015015,66.89651611)
\curveto(159.53014218,66.77651133)(159.55514215,66.65151145)(159.58515015,66.52151611)
\curveto(159.59514211,66.47151163)(159.59514211,66.41651169)(159.58515015,66.35651611)
\curveto(159.58514212,66.3065118)(159.59014212,66.25651185)(159.60015015,66.20651611)
\moveto(158.26515015,65.35151611)
\curveto(158.28514342,65.42151268)(158.29014342,65.5015126)(158.28015015,65.59151611)
\lineto(158.28015015,65.84651611)
\curveto(158.28014343,66.23651187)(158.24514346,66.56651154)(158.17515015,66.83651611)
\curveto(158.14514356,66.91651119)(158.12014359,66.99651111)(158.10015015,67.07651611)
\curveto(158.08014363,67.15651095)(158.05514365,67.23151087)(158.02515015,67.30151611)
\curveto(157.74514396,67.95151015)(157.30014441,68.4015097)(156.69015015,68.65151611)
\curveto(156.62014509,68.68150942)(156.54514516,68.7015094)(156.46515015,68.71151611)
\lineto(156.22515015,68.77151611)
\curveto(156.14514556,68.79150931)(156.06014565,68.8015093)(155.97015015,68.80151611)
\lineto(155.70015015,68.80151611)
\lineto(155.43015015,68.75651611)
\curveto(155.33014638,68.73650937)(155.23514647,68.71150939)(155.14515015,68.68151611)
\curveto(155.06514664,68.66150944)(154.98514672,68.63150947)(154.90515015,68.59151611)
\curveto(154.83514687,68.57150953)(154.77014694,68.54150956)(154.71015015,68.50151611)
\curveto(154.65014706,68.46150964)(154.59514711,68.42150968)(154.54515015,68.38151611)
\curveto(154.3051474,68.21150989)(154.1101476,68.0065101)(153.96015015,67.76651611)
\curveto(153.8101479,67.52651058)(153.68014803,67.24651086)(153.57015015,66.92651611)
\curveto(153.54014817,66.82651128)(153.52014819,66.72151138)(153.51015015,66.61151611)
\curveto(153.50014821,66.51151159)(153.48514822,66.4065117)(153.46515015,66.29651611)
\curveto(153.45514825,66.25651185)(153.45014826,66.19151191)(153.45015015,66.10151611)
\curveto(153.44014827,66.07151203)(153.43514827,66.03651207)(153.43515015,65.99651611)
\curveto(153.44514826,65.95651215)(153.45014826,65.91151219)(153.45015015,65.86151611)
\lineto(153.45015015,65.56151611)
\curveto(153.45014826,65.46151264)(153.46014825,65.37151273)(153.48015015,65.29151611)
\lineto(153.51015015,65.11151611)
\curveto(153.53014818,65.01151309)(153.54514816,64.91151319)(153.55515015,64.81151611)
\curveto(153.57514813,64.72151338)(153.6051481,64.63651347)(153.64515015,64.55651611)
\curveto(153.74514796,64.31651379)(153.86014785,64.09151401)(153.99015015,63.88151611)
\curveto(154.13014758,63.67151443)(154.30014741,63.49651461)(154.50015015,63.35651611)
\curveto(154.55014716,63.32651478)(154.59514711,63.3015148)(154.63515015,63.28151611)
\curveto(154.67514703,63.26151484)(154.72014699,63.23651487)(154.77015015,63.20651611)
\curveto(154.85014686,63.15651495)(154.93514677,63.11151499)(155.02515015,63.07151611)
\curveto(155.12514658,63.04151506)(155.23014648,63.01151509)(155.34015015,62.98151611)
\curveto(155.39014632,62.96151514)(155.43514627,62.95151515)(155.47515015,62.95151611)
\curveto(155.52514618,62.96151514)(155.57514613,62.96151514)(155.62515015,62.95151611)
\curveto(155.65514605,62.94151516)(155.71514599,62.93151517)(155.80515015,62.92151611)
\curveto(155.9051458,62.91151519)(155.98014573,62.91651519)(156.03015015,62.93651611)
\curveto(156.07014564,62.94651516)(156.1101456,62.94651516)(156.15015015,62.93651611)
\curveto(156.19014552,62.93651517)(156.23014548,62.94651516)(156.27015015,62.96651611)
\curveto(156.35014536,62.98651512)(156.43014528,63.0015151)(156.51015015,63.01151611)
\curveto(156.59014512,63.03151507)(156.66514504,63.05651505)(156.73515015,63.08651611)
\curveto(157.07514463,63.22651488)(157.35014436,63.42151468)(157.56015015,63.67151611)
\curveto(157.77014394,63.92151418)(157.94514376,64.21651389)(158.08515015,64.55651611)
\curveto(158.13514357,64.67651343)(158.16514354,64.8015133)(158.17515015,64.93151611)
\curveto(158.19514351,65.07151303)(158.22514348,65.21151289)(158.26515015,65.35151611)
}
}
{
\newrgbcolor{curcolor}{0 0 0}
\pscustom[linestyle=none,fillstyle=solid,fillcolor=curcolor]
{
\newpath
\moveto(300.29141846,62.78651611)
\curveto(300.31140891,62.73651537)(300.33640889,62.67651543)(300.36641846,62.60651611)
\curveto(300.39640883,62.53651557)(300.41640881,62.46151564)(300.42641846,62.38151611)
\curveto(300.44640878,62.31151579)(300.44640878,62.24151586)(300.42641846,62.17151611)
\curveto(300.41640881,62.11151599)(300.37640885,62.06651604)(300.30641846,62.03651611)
\curveto(300.25640897,62.01651609)(300.19640903,62.0065161)(300.12641846,62.00651611)
\lineto(299.91641846,62.00651611)
\lineto(299.46641846,62.00651611)
\curveto(299.31640991,62.0065161)(299.19641003,62.03151607)(299.10641846,62.08151611)
\curveto(299.00641022,62.14151596)(298.93141029,62.24651586)(298.88141846,62.39651611)
\curveto(298.84141038,62.54651556)(298.79641043,62.68151542)(298.74641846,62.80151611)
\curveto(298.63641059,63.06151504)(298.53641069,63.33151477)(298.44641846,63.61151611)
\curveto(298.35641087,63.89151421)(298.25641097,64.16651394)(298.14641846,64.43651611)
\curveto(298.11641111,64.52651358)(298.08641114,64.61151349)(298.05641846,64.69151611)
\curveto(298.03641119,64.77151333)(298.00641122,64.84651326)(297.96641846,64.91651611)
\curveto(297.93641129,64.98651312)(297.89141133,65.04651306)(297.83141846,65.09651611)
\curveto(297.77141145,65.14651296)(297.69141153,65.18651292)(297.59141846,65.21651611)
\curveto(297.54141168,65.23651287)(297.48141174,65.24151286)(297.41141846,65.23151611)
\lineto(297.21641846,65.23151611)
\lineto(294.38141846,65.23151611)
\lineto(294.08141846,65.23151611)
\curveto(293.97141525,65.24151286)(293.86641536,65.24151286)(293.76641846,65.23151611)
\curveto(293.66641556,65.22151288)(293.57141565,65.2065129)(293.48141846,65.18651611)
\curveto(293.40141582,65.16651294)(293.34141588,65.12651298)(293.30141846,65.06651611)
\curveto(293.221416,64.96651314)(293.16141606,64.85151325)(293.12141846,64.72151611)
\curveto(293.09141613,64.6015135)(293.05141617,64.47651363)(293.00141846,64.34651611)
\curveto(292.90141632,64.11651399)(292.80641642,63.87651423)(292.71641846,63.62651611)
\curveto(292.63641659,63.37651473)(292.54641668,63.13651497)(292.44641846,62.90651611)
\curveto(292.4264168,62.84651526)(292.40141682,62.77651533)(292.37141846,62.69651611)
\curveto(292.35141687,62.62651548)(292.3264169,62.55151555)(292.29641846,62.47151611)
\curveto(292.26641696,62.39151571)(292.23141699,62.31651579)(292.19141846,62.24651611)
\curveto(292.16141706,62.18651592)(292.1264171,62.14151596)(292.08641846,62.11151611)
\curveto(292.00641722,62.05151605)(291.89641733,62.01651609)(291.75641846,62.00651611)
\lineto(291.33641846,62.00651611)
\lineto(291.09641846,62.00651611)
\curveto(291.0264182,62.01651609)(290.96641826,62.04151606)(290.91641846,62.08151611)
\curveto(290.86641836,62.11151599)(290.83641839,62.15651595)(290.82641846,62.21651611)
\curveto(290.8264184,62.27651583)(290.83141839,62.33651577)(290.84141846,62.39651611)
\curveto(290.86141836,62.46651564)(290.88141834,62.53151557)(290.90141846,62.59151611)
\curveto(290.93141829,62.66151544)(290.95641827,62.71151539)(290.97641846,62.74151611)
\curveto(291.11641811,63.06151504)(291.24141798,63.37651473)(291.35141846,63.68651611)
\curveto(291.46141776,64.0065141)(291.58141764,64.32651378)(291.71141846,64.64651611)
\curveto(291.80141742,64.86651324)(291.88641734,65.08151302)(291.96641846,65.29151611)
\curveto(292.04641718,65.51151259)(292.13141709,65.73151237)(292.22141846,65.95151611)
\curveto(292.5214167,66.67151143)(292.80641642,67.39651071)(293.07641846,68.12651611)
\curveto(293.34641588,68.86650924)(293.63141559,69.6015085)(293.93141846,70.33151611)
\curveto(294.04141518,70.59150751)(294.14141508,70.85650725)(294.23141846,71.12651611)
\curveto(294.33141489,71.39650671)(294.43641479,71.66150644)(294.54641846,71.92151611)
\curveto(294.59641463,72.03150607)(294.64141458,72.15150595)(294.68141846,72.28151611)
\curveto(294.73141449,72.42150568)(294.80141442,72.52150558)(294.89141846,72.58151611)
\curveto(294.93141429,72.62150548)(294.99641423,72.65150545)(295.08641846,72.67151611)
\curveto(295.10641412,72.68150542)(295.1264141,72.68150542)(295.14641846,72.67151611)
\curveto(295.17641405,72.67150543)(295.20141402,72.67650543)(295.22141846,72.68651611)
\curveto(295.40141382,72.68650542)(295.61141361,72.68650542)(295.85141846,72.68651611)
\curveto(296.09141313,72.69650541)(296.26641296,72.66150544)(296.37641846,72.58151611)
\curveto(296.45641277,72.52150558)(296.51641271,72.42150568)(296.55641846,72.28151611)
\curveto(296.60641262,72.15150595)(296.65641257,72.03150607)(296.70641846,71.92151611)
\curveto(296.80641242,71.69150641)(296.89641233,71.46150664)(296.97641846,71.23151611)
\curveto(297.05641217,71.0015071)(297.14641208,70.77150733)(297.24641846,70.54151611)
\curveto(297.3264119,70.34150776)(297.40141182,70.13650797)(297.47141846,69.92651611)
\curveto(297.55141167,69.71650839)(297.63641159,69.51150859)(297.72641846,69.31151611)
\curveto(298.0264112,68.58150952)(298.31141091,67.84151026)(298.58141846,67.09151611)
\curveto(298.86141036,66.35151175)(299.15641007,65.61651249)(299.46641846,64.88651611)
\curveto(299.50640972,64.79651331)(299.53640969,64.71151339)(299.55641846,64.63151611)
\curveto(299.58640964,64.55151355)(299.61640961,64.46651364)(299.64641846,64.37651611)
\curveto(299.75640947,64.11651399)(299.86140936,63.85151425)(299.96141846,63.58151611)
\curveto(300.07140915,63.31151479)(300.18140904,63.04651506)(300.29141846,62.78651611)
\moveto(297.08141846,66.43151611)
\curveto(297.17141205,66.46151164)(297.226412,66.51151159)(297.24641846,66.58151611)
\curveto(297.27641195,66.65151145)(297.28141194,66.72651138)(297.26141846,66.80651611)
\curveto(297.25141197,66.89651121)(297.226412,66.98151112)(297.18641846,67.06151611)
\curveto(297.15641207,67.15151095)(297.1264121,67.22651088)(297.09641846,67.28651611)
\curveto(297.07641215,67.32651078)(297.06641216,67.36151074)(297.06641846,67.39151611)
\curveto(297.06641216,67.42151068)(297.05641217,67.45651065)(297.03641846,67.49651611)
\lineto(296.94641846,67.73651611)
\curveto(296.9264123,67.82651028)(296.89641233,67.91651019)(296.85641846,68.00651611)
\curveto(296.70641252,68.36650974)(296.57141265,68.73150937)(296.45141846,69.10151611)
\curveto(296.34141288,69.48150862)(296.21141301,69.85150825)(296.06141846,70.21151611)
\curveto(296.01141321,70.32150778)(295.96641326,70.43150767)(295.92641846,70.54151611)
\curveto(295.89641333,70.65150745)(295.85641337,70.75650735)(295.80641846,70.85651611)
\curveto(295.78641344,70.9065072)(295.76141346,70.95150715)(295.73141846,70.99151611)
\curveto(295.71141351,71.04150706)(295.66141356,71.06650704)(295.58141846,71.06651611)
\curveto(295.56141366,71.04650706)(295.54141368,71.03150707)(295.52141846,71.02151611)
\curveto(295.50141372,71.01150709)(295.48141374,70.99650711)(295.46141846,70.97651611)
\curveto(295.4214138,70.92650718)(295.39141383,70.87150723)(295.37141846,70.81151611)
\curveto(295.35141387,70.76150734)(295.33141389,70.7065074)(295.31141846,70.64651611)
\curveto(295.26141396,70.53650757)(295.221414,70.42650768)(295.19141846,70.31651611)
\curveto(295.16141406,70.2065079)(295.1214141,70.09650801)(295.07141846,69.98651611)
\curveto(294.90141432,69.59650851)(294.75141447,69.2015089)(294.62141846,68.80151611)
\curveto(294.50141472,68.4015097)(294.36141486,68.01151009)(294.20141846,67.63151611)
\lineto(294.14141846,67.48151611)
\curveto(294.13141509,67.43151067)(294.11641511,67.38151072)(294.09641846,67.33151611)
\lineto(294.00641846,67.09151611)
\curveto(293.97641525,67.01151109)(293.95141527,66.93151117)(293.93141846,66.85151611)
\curveto(293.91141531,66.8015113)(293.90141532,66.74651136)(293.90141846,66.68651611)
\curveto(293.91141531,66.62651148)(293.9264153,66.57651153)(293.94641846,66.53651611)
\curveto(293.99641523,66.45651165)(294.10141512,66.41151169)(294.26141846,66.40151611)
\lineto(294.71141846,66.40151611)
\lineto(296.31641846,66.40151611)
\curveto(296.4264128,66.4015117)(296.56141266,66.39651171)(296.72141846,66.38651611)
\curveto(296.88141234,66.38651172)(297.00141222,66.4015117)(297.08141846,66.43151611)
}
}
{
\newrgbcolor{curcolor}{0 0 0}
\pscustom[linestyle=none,fillstyle=solid,fillcolor=curcolor]
{
\newpath
\moveto(301.87298096,69.73151611)
\lineto(302.30798096,69.73151611)
\curveto(302.45797899,69.73150837)(302.56297889,69.69150841)(302.62298096,69.61151611)
\curveto(302.67297878,69.53150857)(302.69797875,69.43150867)(302.69798096,69.31151611)
\curveto(302.70797874,69.19150891)(302.71297874,69.07150903)(302.71298096,68.95151611)
\lineto(302.71298096,67.52651611)
\lineto(302.71298096,65.26151611)
\lineto(302.71298096,64.57151611)
\curveto(302.71297874,64.34151376)(302.73797871,64.14151396)(302.78798096,63.97151611)
\curveto(302.9479785,63.52151458)(303.2479782,63.2065149)(303.68798096,63.02651611)
\curveto(303.90797754,62.93651517)(304.17297728,62.9015152)(304.48298096,62.92151611)
\curveto(304.79297666,62.95151515)(305.04297641,63.0065151)(305.23298096,63.08651611)
\curveto(305.56297589,63.22651488)(305.82297563,63.4015147)(306.01298096,63.61151611)
\curveto(306.21297524,63.83151427)(306.36797508,64.11651399)(306.47798096,64.46651611)
\curveto(306.50797494,64.54651356)(306.52797492,64.62651348)(306.53798096,64.70651611)
\curveto(306.5479749,64.78651332)(306.56297489,64.87151323)(306.58298096,64.96151611)
\curveto(306.59297486,65.01151309)(306.59297486,65.05651305)(306.58298096,65.09651611)
\curveto(306.58297487,65.13651297)(306.59297486,65.18151292)(306.61298096,65.23151611)
\lineto(306.61298096,65.54651611)
\curveto(306.63297482,65.62651248)(306.63797481,65.71651239)(306.62798096,65.81651611)
\curveto(306.61797483,65.92651218)(306.61297484,66.02651208)(306.61298096,66.11651611)
\lineto(306.61298096,67.28651611)
\lineto(306.61298096,68.87651611)
\curveto(306.61297484,68.99650911)(306.60797484,69.12150898)(306.59798096,69.25151611)
\curveto(306.59797485,69.39150871)(306.62297483,69.5015086)(306.67298096,69.58151611)
\curveto(306.71297474,69.63150847)(306.75797469,69.66150844)(306.80798096,69.67151611)
\curveto(306.86797458,69.69150841)(306.93797451,69.71150839)(307.01798096,69.73151611)
\lineto(307.24298096,69.73151611)
\curveto(307.36297409,69.73150837)(307.46797398,69.72650838)(307.55798096,69.71651611)
\curveto(307.65797379,69.7065084)(307.73297372,69.66150844)(307.78298096,69.58151611)
\curveto(307.83297362,69.53150857)(307.85797359,69.45650865)(307.85798096,69.35651611)
\lineto(307.85798096,69.07151611)
\lineto(307.85798096,68.05151611)
\lineto(307.85798096,64.01651611)
\lineto(307.85798096,62.66651611)
\curveto(307.85797359,62.54651556)(307.8529736,62.43151567)(307.84298096,62.32151611)
\curveto(307.84297361,62.22151588)(307.80797364,62.14651596)(307.73798096,62.09651611)
\curveto(307.69797375,62.06651604)(307.63797381,62.04151606)(307.55798096,62.02151611)
\curveto(307.47797397,62.01151609)(307.38797406,62.0015161)(307.28798096,61.99151611)
\curveto(307.19797425,61.99151611)(307.10797434,61.99651611)(307.01798096,62.00651611)
\curveto(306.93797451,62.01651609)(306.87797457,62.03651607)(306.83798096,62.06651611)
\curveto(306.78797466,62.106516)(306.74297471,62.17151593)(306.70298096,62.26151611)
\curveto(306.69297476,62.3015158)(306.68297477,62.35651575)(306.67298096,62.42651611)
\curveto(306.67297478,62.49651561)(306.66797478,62.56151554)(306.65798096,62.62151611)
\curveto(306.6479748,62.69151541)(306.62797482,62.74651536)(306.59798096,62.78651611)
\curveto(306.56797488,62.82651528)(306.52297493,62.84151526)(306.46298096,62.83151611)
\curveto(306.38297507,62.81151529)(306.30297515,62.75151535)(306.22298096,62.65151611)
\curveto(306.14297531,62.56151554)(306.06797538,62.49151561)(305.99798096,62.44151611)
\curveto(305.77797567,62.28151582)(305.52797592,62.14151596)(305.24798096,62.02151611)
\curveto(305.13797631,61.97151613)(305.02297643,61.94151616)(304.90298096,61.93151611)
\curveto(304.79297666,61.91151619)(304.67797677,61.88651622)(304.55798096,61.85651611)
\curveto(304.50797694,61.84651626)(304.452977,61.84651626)(304.39298096,61.85651611)
\curveto(304.34297711,61.86651624)(304.29297716,61.86151624)(304.24298096,61.84151611)
\curveto(304.14297731,61.82151628)(304.0529774,61.82151628)(303.97298096,61.84151611)
\lineto(303.82298096,61.84151611)
\curveto(303.77297768,61.86151624)(303.71297774,61.87151623)(303.64298096,61.87151611)
\curveto(303.58297787,61.87151623)(303.52797792,61.87651623)(303.47798096,61.88651611)
\curveto(303.43797801,61.9065162)(303.39797805,61.91651619)(303.35798096,61.91651611)
\curveto(303.32797812,61.9065162)(303.28797816,61.91151619)(303.23798096,61.93151611)
\lineto(302.99798096,61.99151611)
\curveto(302.92797852,62.01151609)(302.8529786,62.04151606)(302.77298096,62.08151611)
\curveto(302.51297894,62.19151591)(302.29297916,62.33651577)(302.11298096,62.51651611)
\curveto(301.94297951,62.7065154)(301.80297965,62.93151517)(301.69298096,63.19151611)
\curveto(301.6529798,63.28151482)(301.62297983,63.37151473)(301.60298096,63.46151611)
\lineto(301.54298096,63.76151611)
\curveto(301.52297993,63.82151428)(301.51297994,63.87651423)(301.51298096,63.92651611)
\curveto(301.52297993,63.98651412)(301.51797993,64.05151405)(301.49798096,64.12151611)
\curveto(301.48797996,64.14151396)(301.48297997,64.16651394)(301.48298096,64.19651611)
\curveto(301.48297997,64.23651387)(301.47797997,64.27151383)(301.46798096,64.30151611)
\lineto(301.46798096,64.45151611)
\curveto(301.45797999,64.49151361)(301.45298,64.53651357)(301.45298096,64.58651611)
\curveto(301.46297999,64.64651346)(301.46797998,64.7015134)(301.46798096,64.75151611)
\lineto(301.46798096,65.35151611)
\lineto(301.46798096,68.11151611)
\lineto(301.46798096,69.07151611)
\lineto(301.46798096,69.34151611)
\curveto(301.46797998,69.43150867)(301.48797996,69.5065086)(301.52798096,69.56651611)
\curveto(301.56797988,69.63650847)(301.64297981,69.68650842)(301.75298096,69.71651611)
\curveto(301.77297968,69.72650838)(301.79297966,69.72650838)(301.81298096,69.71651611)
\curveto(301.83297962,69.71650839)(301.8529796,69.72150838)(301.87298096,69.73151611)
}
}
{
\newrgbcolor{curcolor}{0 0 0}
\pscustom[linestyle=none,fillstyle=solid,fillcolor=curcolor]
{
\newpath
\moveto(309.77259033,69.73151611)
\lineto(310.29759033,69.73151611)
\curveto(310.49758868,69.74150836)(310.64758853,69.72150838)(310.74759033,69.67151611)
\curveto(310.86758831,69.62150848)(310.96258821,69.54150856)(311.03259033,69.43151611)
\curveto(311.11258806,69.32150878)(311.18758799,69.21150889)(311.25759033,69.10151611)
\curveto(311.38758779,68.9015092)(311.51758766,68.7065094)(311.64759033,68.51651611)
\curveto(311.7775874,68.33650977)(311.91258726,68.14650996)(312.05259033,67.94651611)
\curveto(312.10258707,67.86651024)(312.15258702,67.79151031)(312.20259033,67.72151611)
\curveto(312.26258691,67.65151045)(312.31758686,67.58151052)(312.36759033,67.51151611)
\curveto(312.40758677,67.45151065)(312.44758673,67.39651071)(312.48759033,67.34651611)
\curveto(312.52758665,67.29651081)(312.58758659,67.26151084)(312.66759033,67.24151611)
\curveto(312.71758646,67.22151088)(312.75758642,67.22151088)(312.78759033,67.24151611)
\curveto(312.82758635,67.27151083)(312.85758632,67.29651081)(312.87759033,67.31651611)
\curveto(312.95758622,67.36651074)(313.02258615,67.43651067)(313.07259033,67.52651611)
\curveto(313.13258604,67.61651049)(313.18758599,67.7015104)(313.23759033,67.78151611)
\curveto(313.38758579,67.98151012)(313.53758564,68.18650992)(313.68759033,68.39651611)
\lineto(314.13759033,69.02651611)
\curveto(314.21758496,69.13650897)(314.29758488,69.25150885)(314.37759033,69.37151611)
\curveto(314.45758472,69.49150861)(314.55258462,69.58650852)(314.66259033,69.65651611)
\curveto(314.74258443,69.7065084)(314.83758434,69.73150837)(314.94759033,69.73151611)
\lineto(315.29259033,69.73151611)
\lineto(315.42759033,69.73151611)
\curveto(315.4775837,69.73150837)(315.52758365,69.72650838)(315.57759033,69.71651611)
\lineto(315.65259033,69.71651611)
\curveto(315.7725834,69.69650841)(315.85258332,69.65650845)(315.89259033,69.59651611)
\curveto(315.91258326,69.54650856)(315.90758327,69.49150861)(315.87759033,69.43151611)
\curveto(315.85758332,69.38150872)(315.83758334,69.34150876)(315.81759033,69.31151611)
\lineto(315.60759033,69.01151611)
\curveto(315.53758364,68.92150918)(315.46258371,68.82650928)(315.38259033,68.72651611)
\curveto(315.15258402,68.4065097)(314.91758426,68.09151001)(314.67759033,67.78151611)
\curveto(314.44758473,67.48151062)(314.21758496,67.17151093)(313.98759033,66.85151611)
\curveto(313.93758524,66.77151133)(313.88258529,66.69151141)(313.82259033,66.61151611)
\curveto(313.76258541,66.54151156)(313.70758547,66.46151164)(313.65759033,66.37151611)
\curveto(313.63758554,66.34151176)(313.61758556,66.3015118)(313.59759033,66.25151611)
\curveto(313.5775856,66.21151189)(313.5775856,66.16151194)(313.59759033,66.10151611)
\curveto(313.61758556,66.01151209)(313.64758553,65.93651217)(313.68759033,65.87651611)
\curveto(313.73758544,65.81651229)(313.78758539,65.75151235)(313.83759033,65.68151611)
\lineto(314.01759033,65.41151611)
\curveto(314.08758509,65.32151278)(314.15258502,65.23151287)(314.21259033,65.14151611)
\lineto(314.90259033,64.18151611)
\lineto(315.59259033,63.22151611)
\curveto(315.6725835,63.11151499)(315.75258342,62.99651511)(315.83259033,62.87651611)
\lineto(316.07259033,62.54651611)
\curveto(316.12258305,62.47651563)(316.16258301,62.41151569)(316.19259033,62.35151611)
\curveto(316.23258294,62.3015158)(316.24258293,62.22151588)(316.22259033,62.11151611)
\curveto(316.20258297,62.101516)(316.18258299,62.08651602)(316.16259033,62.06651611)
\curveto(316.15258302,62.05651605)(316.13758304,62.04651606)(316.11759033,62.03651611)
\curveto(316.06758311,62.01651609)(316.00258317,62.0065161)(315.92259033,62.00651611)
\lineto(315.68259033,62.00651611)
\lineto(315.17259033,62.00651611)
\curveto(315.03258414,62.01651609)(314.90758427,62.06151604)(314.79759033,62.14151611)
\curveto(314.74758443,62.17151593)(314.70758447,62.2065159)(314.67759033,62.24651611)
\curveto(314.65758452,62.29651581)(314.63258454,62.34651576)(314.60259033,62.39651611)
\lineto(314.45259033,62.60651611)
\curveto(314.40258477,62.67651543)(314.35258482,62.75151535)(314.30259033,62.83151611)
\lineto(313.35759033,64.22651611)
\curveto(313.30758587,64.3065138)(313.25758592,64.38151372)(313.20759033,64.45151611)
\curveto(313.15758602,64.52151358)(313.10758607,64.59651351)(313.05759033,64.67651611)
\curveto(313.00758617,64.74651336)(312.95758622,64.8065133)(312.90759033,64.85651611)
\curveto(312.86758631,64.91651319)(312.80758637,64.95651315)(312.72759033,64.97651611)
\curveto(312.6775865,64.99651311)(312.62758655,64.98651312)(312.57759033,64.94651611)
\curveto(312.53758664,64.91651319)(312.50758667,64.89151321)(312.48759033,64.87151611)
\curveto(312.40758677,64.79151331)(312.33758684,64.7015134)(312.27759033,64.60151611)
\curveto(312.21758696,64.5015136)(312.15758702,64.4065137)(312.09759033,64.31651611)
\curveto(311.92758725,64.05651405)(311.75258742,63.79651431)(311.57259033,63.53651611)
\curveto(311.40258777,63.28651482)(311.22758795,63.03651507)(311.04759033,62.78651611)
\curveto(310.99758818,62.7065154)(310.94258823,62.62651548)(310.88259033,62.54651611)
\lineto(310.73259033,62.30651611)
\curveto(310.71258846,62.27651583)(310.68758849,62.24151586)(310.65759033,62.20151611)
\curveto(310.63758854,62.17151593)(310.61258856,62.14651596)(310.58259033,62.12651611)
\curveto(310.48258869,62.05651605)(310.36258881,62.01651609)(310.22259033,62.00651611)
\lineto(309.77259033,62.00651611)
\lineto(309.54759033,62.00651611)
\curveto(309.4775897,62.0065161)(309.41758976,62.01651609)(309.36759033,62.03651611)
\curveto(309.33758984,62.05651605)(309.31258986,62.07151603)(309.29259033,62.08151611)
\curveto(309.28258989,62.101516)(309.26758991,62.12151598)(309.24759033,62.14151611)
\curveto(309.23758994,62.25151585)(309.25258992,62.33651577)(309.29259033,62.39651611)
\curveto(309.34258983,62.45651565)(309.39258978,62.52151558)(309.44259033,62.59151611)
\curveto(309.52258965,62.7015154)(309.59758958,62.8015153)(309.66759033,62.89151611)
\curveto(309.73758944,62.99151511)(309.80758937,63.09651501)(309.87759033,63.20651611)
\curveto(310.09758908,63.5065146)(310.31258886,63.8065143)(310.52259033,64.10651611)
\lineto(311.15259033,65.00651611)
\curveto(311.22258795,65.09651301)(311.28758789,65.18651292)(311.34759033,65.27651611)
\curveto(311.41758776,65.36651274)(311.48258769,65.46151264)(311.54259033,65.56151611)
\curveto(311.59258758,65.63151247)(311.64258753,65.69651241)(311.69259033,65.75651611)
\curveto(311.74258743,65.82651228)(311.7775874,65.91651219)(311.79759033,66.02651611)
\curveto(311.81758736,66.07651203)(311.81258736,66.12651198)(311.78259033,66.17651611)
\curveto(311.76258741,66.22651188)(311.74258743,66.26651184)(311.72259033,66.29651611)
\curveto(311.6725875,66.38651172)(311.61758756,66.47151163)(311.55759033,66.55151611)
\lineto(311.37759033,66.79151611)
\curveto(311.14758803,67.11151099)(310.91258826,67.43151067)(310.67259033,67.75151611)
\lineto(309.98259033,68.71151611)
\curveto(309.90258927,68.82150928)(309.82258935,68.92150918)(309.74259033,69.01151611)
\curveto(309.6725895,69.101509)(309.60258957,69.2015089)(309.53259033,69.31151611)
\curveto(309.51258966,69.34150876)(309.49258968,69.38150872)(309.47259033,69.43151611)
\curveto(309.45258972,69.49150861)(309.45258972,69.54150856)(309.47259033,69.58151611)
\curveto(309.49258968,69.63150847)(309.52258965,69.66150844)(309.56259033,69.67151611)
\curveto(309.60258957,69.69150841)(309.64758953,69.7065084)(309.69759033,69.71651611)
\curveto(309.71758946,69.72650838)(309.73258944,69.72650838)(309.74259033,69.71651611)
\curveto(309.75258942,69.71650839)(309.76258941,69.72150838)(309.77259033,69.73151611)
}
}
{
\newrgbcolor{curcolor}{0 0 0}
\pscustom[linestyle=none,fillstyle=solid,fillcolor=curcolor]
{
\newpath
\moveto(317.80626221,71.23151611)
\curveto(317.72626109,71.29150681)(317.68126113,71.39650671)(317.67126221,71.54651611)
\lineto(317.67126221,72.01151611)
\lineto(317.67126221,72.26651611)
\curveto(317.67126114,72.35650575)(317.68626113,72.43150567)(317.71626221,72.49151611)
\curveto(317.75626106,72.57150553)(317.83626098,72.63150547)(317.95626221,72.67151611)
\curveto(317.97626084,72.68150542)(317.99626082,72.68150542)(318.01626221,72.67151611)
\curveto(318.04626077,72.67150543)(318.07126074,72.67650543)(318.09126221,72.68651611)
\curveto(318.26126055,72.68650542)(318.42126039,72.68150542)(318.57126221,72.67151611)
\curveto(318.72126009,72.66150544)(318.82125999,72.6015055)(318.87126221,72.49151611)
\curveto(318.90125991,72.43150567)(318.9162599,72.35650575)(318.91626221,72.26651611)
\lineto(318.91626221,72.01151611)
\curveto(318.9162599,71.83150627)(318.9112599,71.66150644)(318.90126221,71.50151611)
\curveto(318.90125991,71.34150676)(318.83625998,71.23650687)(318.70626221,71.18651611)
\curveto(318.65626016,71.16650694)(318.60126021,71.15650695)(318.54126221,71.15651611)
\lineto(318.37626221,71.15651611)
\lineto(318.06126221,71.15651611)
\curveto(317.96126085,71.15650695)(317.87626094,71.18150692)(317.80626221,71.23151611)
\moveto(318.91626221,62.72651611)
\lineto(318.91626221,62.41151611)
\curveto(318.92625989,62.31151579)(318.90625991,62.23151587)(318.85626221,62.17151611)
\curveto(318.82625999,62.11151599)(318.78126003,62.07151603)(318.72126221,62.05151611)
\curveto(318.66126015,62.04151606)(318.59126022,62.02651608)(318.51126221,62.00651611)
\lineto(318.28626221,62.00651611)
\curveto(318.15626066,62.0065161)(318.04126077,62.01151609)(317.94126221,62.02151611)
\curveto(317.85126096,62.04151606)(317.78126103,62.09151601)(317.73126221,62.17151611)
\curveto(317.69126112,62.23151587)(317.67126114,62.3065158)(317.67126221,62.39651611)
\lineto(317.67126221,62.68151611)
\lineto(317.67126221,69.02651611)
\lineto(317.67126221,69.34151611)
\curveto(317.67126114,69.45150865)(317.69626112,69.53650857)(317.74626221,69.59651611)
\curveto(317.77626104,69.64650846)(317.816261,69.67650843)(317.86626221,69.68651611)
\curveto(317.9162609,69.69650841)(317.97126084,69.71150839)(318.03126221,69.73151611)
\curveto(318.05126076,69.73150837)(318.07126074,69.72650838)(318.09126221,69.71651611)
\curveto(318.12126069,69.71650839)(318.14626067,69.72150838)(318.16626221,69.73151611)
\curveto(318.29626052,69.73150837)(318.42626039,69.72650838)(318.55626221,69.71651611)
\curveto(318.69626012,69.71650839)(318.79126002,69.67650843)(318.84126221,69.59651611)
\curveto(318.89125992,69.53650857)(318.9162599,69.45650865)(318.91626221,69.35651611)
\lineto(318.91626221,69.07151611)
\lineto(318.91626221,62.72651611)
}
}
{
\newrgbcolor{curcolor}{0 0 0}
\pscustom[linestyle=none,fillstyle=solid,fillcolor=curcolor]
{
\newpath
\moveto(321.43110596,72.68651611)
\curveto(321.56110434,72.68650542)(321.69610421,72.68650542)(321.83610596,72.68651611)
\curveto(321.98610392,72.68650542)(322.09610381,72.65150545)(322.16610596,72.58151611)
\curveto(322.21610369,72.51150559)(322.24110366,72.41650569)(322.24110596,72.29651611)
\curveto(322.25110365,72.18650592)(322.25610365,72.07150603)(322.25610596,71.95151611)
\lineto(322.25610596,70.61651611)
\lineto(322.25610596,64.54151611)
\lineto(322.25610596,62.86151611)
\lineto(322.25610596,62.47151611)
\curveto(322.25610365,62.33151577)(322.23110367,62.22151588)(322.18110596,62.14151611)
\curveto(322.15110375,62.09151601)(322.1061038,62.06151604)(322.04610596,62.05151611)
\curveto(321.99610391,62.04151606)(321.93110397,62.02651608)(321.85110596,62.00651611)
\lineto(321.64110596,62.00651611)
\lineto(321.32610596,62.00651611)
\curveto(321.22610468,62.01651609)(321.15110475,62.05151605)(321.10110596,62.11151611)
\curveto(321.05110485,62.19151591)(321.02110488,62.29151581)(321.01110596,62.41151611)
\lineto(321.01110596,62.78651611)
\lineto(321.01110596,64.16651611)
\lineto(321.01110596,70.40651611)
\lineto(321.01110596,71.87651611)
\curveto(321.01110489,71.98650612)(321.0061049,72.101506)(320.99610596,72.22151611)
\curveto(320.99610491,72.35150575)(321.02110488,72.45150565)(321.07110596,72.52151611)
\curveto(321.11110479,72.58150552)(321.18610472,72.63150547)(321.29610596,72.67151611)
\curveto(321.31610459,72.68150542)(321.33610457,72.68150542)(321.35610596,72.67151611)
\curveto(321.38610452,72.67150543)(321.41110449,72.67650543)(321.43110596,72.68651611)
}
}
{
\newrgbcolor{curcolor}{0 0 0}
\pscustom[linestyle=none,fillstyle=solid,fillcolor=curcolor]
{
\newpath
\moveto(324.48594971,71.23151611)
\curveto(324.40594859,71.29150681)(324.36094863,71.39650671)(324.35094971,71.54651611)
\lineto(324.35094971,72.01151611)
\lineto(324.35094971,72.26651611)
\curveto(324.35094864,72.35650575)(324.36594863,72.43150567)(324.39594971,72.49151611)
\curveto(324.43594856,72.57150553)(324.51594848,72.63150547)(324.63594971,72.67151611)
\curveto(324.65594834,72.68150542)(324.67594832,72.68150542)(324.69594971,72.67151611)
\curveto(324.72594827,72.67150543)(324.75094824,72.67650543)(324.77094971,72.68651611)
\curveto(324.94094805,72.68650542)(325.10094789,72.68150542)(325.25094971,72.67151611)
\curveto(325.40094759,72.66150544)(325.50094749,72.6015055)(325.55094971,72.49151611)
\curveto(325.58094741,72.43150567)(325.5959474,72.35650575)(325.59594971,72.26651611)
\lineto(325.59594971,72.01151611)
\curveto(325.5959474,71.83150627)(325.5909474,71.66150644)(325.58094971,71.50151611)
\curveto(325.58094741,71.34150676)(325.51594748,71.23650687)(325.38594971,71.18651611)
\curveto(325.33594766,71.16650694)(325.28094771,71.15650695)(325.22094971,71.15651611)
\lineto(325.05594971,71.15651611)
\lineto(324.74094971,71.15651611)
\curveto(324.64094835,71.15650695)(324.55594844,71.18150692)(324.48594971,71.23151611)
\moveto(325.59594971,62.72651611)
\lineto(325.59594971,62.41151611)
\curveto(325.60594739,62.31151579)(325.58594741,62.23151587)(325.53594971,62.17151611)
\curveto(325.50594749,62.11151599)(325.46094753,62.07151603)(325.40094971,62.05151611)
\curveto(325.34094765,62.04151606)(325.27094772,62.02651608)(325.19094971,62.00651611)
\lineto(324.96594971,62.00651611)
\curveto(324.83594816,62.0065161)(324.72094827,62.01151609)(324.62094971,62.02151611)
\curveto(324.53094846,62.04151606)(324.46094853,62.09151601)(324.41094971,62.17151611)
\curveto(324.37094862,62.23151587)(324.35094864,62.3065158)(324.35094971,62.39651611)
\lineto(324.35094971,62.68151611)
\lineto(324.35094971,69.02651611)
\lineto(324.35094971,69.34151611)
\curveto(324.35094864,69.45150865)(324.37594862,69.53650857)(324.42594971,69.59651611)
\curveto(324.45594854,69.64650846)(324.4959485,69.67650843)(324.54594971,69.68651611)
\curveto(324.5959484,69.69650841)(324.65094834,69.71150839)(324.71094971,69.73151611)
\curveto(324.73094826,69.73150837)(324.75094824,69.72650838)(324.77094971,69.71651611)
\curveto(324.80094819,69.71650839)(324.82594817,69.72150838)(324.84594971,69.73151611)
\curveto(324.97594802,69.73150837)(325.10594789,69.72650838)(325.23594971,69.71651611)
\curveto(325.37594762,69.71650839)(325.47094752,69.67650843)(325.52094971,69.59651611)
\curveto(325.57094742,69.53650857)(325.5959474,69.45650865)(325.59594971,69.35651611)
\lineto(325.59594971,69.07151611)
\lineto(325.59594971,62.72651611)
}
}
{
\newrgbcolor{curcolor}{0 0 0}
\pscustom[linestyle=none,fillstyle=solid,fillcolor=curcolor]
{
\newpath
\moveto(334.42579346,62.56151611)
\curveto(334.45578563,62.4015157)(334.44078564,62.26651584)(334.38079346,62.15651611)
\curveto(334.32078576,62.05651605)(334.24078584,61.98151612)(334.14079346,61.93151611)
\curveto(334.09078599,61.91151619)(334.03578605,61.9015162)(333.97579346,61.90151611)
\curveto(333.92578616,61.9015162)(333.87078621,61.89151621)(333.81079346,61.87151611)
\curveto(333.59078649,61.82151628)(333.37078671,61.83651627)(333.15079346,61.91651611)
\curveto(332.94078714,61.98651612)(332.79578729,62.07651603)(332.71579346,62.18651611)
\curveto(332.66578742,62.25651585)(332.62078746,62.33651577)(332.58079346,62.42651611)
\curveto(332.54078754,62.52651558)(332.49078759,62.6065155)(332.43079346,62.66651611)
\curveto(332.41078767,62.68651542)(332.3857877,62.7065154)(332.35579346,62.72651611)
\curveto(332.33578775,62.74651536)(332.30578778,62.75151535)(332.26579346,62.74151611)
\curveto(332.15578793,62.71151539)(332.05078803,62.65651545)(331.95079346,62.57651611)
\curveto(331.86078822,62.49651561)(331.77078831,62.42651568)(331.68079346,62.36651611)
\curveto(331.55078853,62.28651582)(331.41078867,62.21151589)(331.26079346,62.14151611)
\curveto(331.11078897,62.08151602)(330.95078913,62.02651608)(330.78079346,61.97651611)
\curveto(330.6807894,61.94651616)(330.57078951,61.92651618)(330.45079346,61.91651611)
\curveto(330.34078974,61.9065162)(330.23078985,61.89151621)(330.12079346,61.87151611)
\curveto(330.07079001,61.86151624)(330.02579006,61.85651625)(329.98579346,61.85651611)
\lineto(329.88079346,61.85651611)
\curveto(329.77079031,61.83651627)(329.66579042,61.83651627)(329.56579346,61.85651611)
\lineto(329.43079346,61.85651611)
\curveto(329.3807907,61.86651624)(329.33079075,61.87151623)(329.28079346,61.87151611)
\curveto(329.23079085,61.87151623)(329.1857909,61.88151622)(329.14579346,61.90151611)
\curveto(329.10579098,61.91151619)(329.07079101,61.91651619)(329.04079346,61.91651611)
\curveto(329.02079106,61.9065162)(328.99579109,61.9065162)(328.96579346,61.91651611)
\lineto(328.72579346,61.97651611)
\curveto(328.64579144,61.98651612)(328.57079151,62.0065161)(328.50079346,62.03651611)
\curveto(328.20079188,62.16651594)(327.95579213,62.31151579)(327.76579346,62.47151611)
\curveto(327.5857925,62.64151546)(327.43579265,62.87651523)(327.31579346,63.17651611)
\curveto(327.22579286,63.39651471)(327.1807929,63.66151444)(327.18079346,63.97151611)
\lineto(327.18079346,64.28651611)
\curveto(327.19079289,64.33651377)(327.19579289,64.38651372)(327.19579346,64.43651611)
\lineto(327.22579346,64.61651611)
\lineto(327.34579346,64.94651611)
\curveto(327.3857927,65.05651305)(327.43579265,65.15651295)(327.49579346,65.24651611)
\curveto(327.67579241,65.53651257)(327.92079216,65.75151235)(328.23079346,65.89151611)
\curveto(328.54079154,66.03151207)(328.8807912,66.15651195)(329.25079346,66.26651611)
\curveto(329.39079069,66.3065118)(329.53579055,66.33651177)(329.68579346,66.35651611)
\curveto(329.83579025,66.37651173)(329.9857901,66.4015117)(330.13579346,66.43151611)
\curveto(330.20578988,66.45151165)(330.27078981,66.46151164)(330.33079346,66.46151611)
\curveto(330.40078968,66.46151164)(330.47578961,66.47151163)(330.55579346,66.49151611)
\curveto(330.62578946,66.51151159)(330.69578939,66.52151158)(330.76579346,66.52151611)
\curveto(330.83578925,66.53151157)(330.91078917,66.54651156)(330.99079346,66.56651611)
\curveto(331.24078884,66.62651148)(331.47578861,66.67651143)(331.69579346,66.71651611)
\curveto(331.91578817,66.76651134)(332.09078799,66.88151122)(332.22079346,67.06151611)
\curveto(332.2807878,67.14151096)(332.33078775,67.24151086)(332.37079346,67.36151611)
\curveto(332.41078767,67.49151061)(332.41078767,67.63151047)(332.37079346,67.78151611)
\curveto(332.31078777,68.02151008)(332.22078786,68.21150989)(332.10079346,68.35151611)
\curveto(331.99078809,68.49150961)(331.83078825,68.6015095)(331.62079346,68.68151611)
\curveto(331.50078858,68.73150937)(331.35578873,68.76650934)(331.18579346,68.78651611)
\curveto(331.02578906,68.8065093)(330.85578923,68.81650929)(330.67579346,68.81651611)
\curveto(330.49578959,68.81650929)(330.32078976,68.8065093)(330.15079346,68.78651611)
\curveto(329.9807901,68.76650934)(329.83579025,68.73650937)(329.71579346,68.69651611)
\curveto(329.54579054,68.63650947)(329.3807907,68.55150955)(329.22079346,68.44151611)
\curveto(329.14079094,68.38150972)(329.06579102,68.3015098)(328.99579346,68.20151611)
\curveto(328.93579115,68.11150999)(328.8807912,68.01151009)(328.83079346,67.90151611)
\curveto(328.80079128,67.82151028)(328.77079131,67.73651037)(328.74079346,67.64651611)
\curveto(328.72079136,67.55651055)(328.67579141,67.48651062)(328.60579346,67.43651611)
\curveto(328.56579152,67.4065107)(328.49579159,67.38151072)(328.39579346,67.36151611)
\curveto(328.30579178,67.35151075)(328.21079187,67.34651076)(328.11079346,67.34651611)
\curveto(328.01079207,67.34651076)(327.91079217,67.35151075)(327.81079346,67.36151611)
\curveto(327.72079236,67.38151072)(327.65579243,67.4065107)(327.61579346,67.43651611)
\curveto(327.57579251,67.46651064)(327.54579254,67.51651059)(327.52579346,67.58651611)
\curveto(327.50579258,67.65651045)(327.50579258,67.73151037)(327.52579346,67.81151611)
\curveto(327.55579253,67.94151016)(327.5857925,68.06151004)(327.61579346,68.17151611)
\curveto(327.65579243,68.29150981)(327.70079238,68.4065097)(327.75079346,68.51651611)
\curveto(327.94079214,68.86650924)(328.1807919,69.13650897)(328.47079346,69.32651611)
\curveto(328.76079132,69.52650858)(329.12079096,69.68650842)(329.55079346,69.80651611)
\curveto(329.65079043,69.82650828)(329.75079033,69.84150826)(329.85079346,69.85151611)
\curveto(329.96079012,69.86150824)(330.07079001,69.87650823)(330.18079346,69.89651611)
\curveto(330.22078986,69.9065082)(330.2857898,69.9065082)(330.37579346,69.89651611)
\curveto(330.46578962,69.89650821)(330.52078956,69.9065082)(330.54079346,69.92651611)
\curveto(331.24078884,69.93650817)(331.85078823,69.85650825)(332.37079346,69.68651611)
\curveto(332.89078719,69.51650859)(333.25578683,69.19150891)(333.46579346,68.71151611)
\curveto(333.55578653,68.51150959)(333.60578648,68.27650983)(333.61579346,68.00651611)
\curveto(333.63578645,67.74651036)(333.64578644,67.47151063)(333.64579346,67.18151611)
\lineto(333.64579346,63.86651611)
\curveto(333.64578644,63.72651438)(333.65078643,63.59151451)(333.66079346,63.46151611)
\curveto(333.67078641,63.33151477)(333.70078638,63.22651488)(333.75079346,63.14651611)
\curveto(333.80078628,63.07651503)(333.86578622,63.02651508)(333.94579346,62.99651611)
\curveto(334.03578605,62.95651515)(334.12078596,62.92651518)(334.20079346,62.90651611)
\curveto(334.2807858,62.89651521)(334.34078574,62.85151525)(334.38079346,62.77151611)
\curveto(334.40078568,62.74151536)(334.41078567,62.71151539)(334.41079346,62.68151611)
\curveto(334.41078567,62.65151545)(334.41578567,62.61151549)(334.42579346,62.56151611)
\moveto(332.28079346,64.22651611)
\curveto(332.34078774,64.36651374)(332.37078771,64.52651358)(332.37079346,64.70651611)
\curveto(332.3807877,64.89651321)(332.3857877,65.09151301)(332.38579346,65.29151611)
\curveto(332.3857877,65.4015127)(332.3807877,65.5015126)(332.37079346,65.59151611)
\curveto(332.36078772,65.68151242)(332.32078776,65.75151235)(332.25079346,65.80151611)
\curveto(332.22078786,65.82151228)(332.15078793,65.83151227)(332.04079346,65.83151611)
\curveto(332.02078806,65.81151229)(331.9857881,65.8015123)(331.93579346,65.80151611)
\curveto(331.8857882,65.8015123)(331.84078824,65.79151231)(331.80079346,65.77151611)
\curveto(331.72078836,65.75151235)(331.63078845,65.73151237)(331.53079346,65.71151611)
\lineto(331.23079346,65.65151611)
\curveto(331.20078888,65.65151245)(331.16578892,65.64651246)(331.12579346,65.63651611)
\lineto(331.02079346,65.63651611)
\curveto(330.87078921,65.59651251)(330.70578938,65.57151253)(330.52579346,65.56151611)
\curveto(330.35578973,65.56151254)(330.19578989,65.54151256)(330.04579346,65.50151611)
\curveto(329.96579012,65.48151262)(329.89079019,65.46151264)(329.82079346,65.44151611)
\curveto(329.76079032,65.43151267)(329.69079039,65.41651269)(329.61079346,65.39651611)
\curveto(329.45079063,65.34651276)(329.30079078,65.28151282)(329.16079346,65.20151611)
\curveto(329.02079106,65.13151297)(328.90079118,65.04151306)(328.80079346,64.93151611)
\curveto(328.70079138,64.82151328)(328.62579146,64.68651342)(328.57579346,64.52651611)
\curveto(328.52579156,64.37651373)(328.50579158,64.19151391)(328.51579346,63.97151611)
\curveto(328.51579157,63.87151423)(328.53079155,63.77651433)(328.56079346,63.68651611)
\curveto(328.60079148,63.6065145)(328.64579144,63.53151457)(328.69579346,63.46151611)
\curveto(328.77579131,63.35151475)(328.8807912,63.25651485)(329.01079346,63.17651611)
\curveto(329.14079094,63.106515)(329.2807908,63.04651506)(329.43079346,62.99651611)
\curveto(329.4807906,62.98651512)(329.53079055,62.98151512)(329.58079346,62.98151611)
\curveto(329.63079045,62.98151512)(329.6807904,62.97651513)(329.73079346,62.96651611)
\curveto(329.80079028,62.94651516)(329.8857902,62.93151517)(329.98579346,62.92151611)
\curveto(330.09578999,62.92151518)(330.1857899,62.93151517)(330.25579346,62.95151611)
\curveto(330.31578977,62.97151513)(330.37578971,62.97651513)(330.43579346,62.96651611)
\curveto(330.49578959,62.96651514)(330.55578953,62.97651513)(330.61579346,62.99651611)
\curveto(330.69578939,63.01651509)(330.77078931,63.03151507)(330.84079346,63.04151611)
\curveto(330.92078916,63.05151505)(330.99578909,63.07151503)(331.06579346,63.10151611)
\curveto(331.35578873,63.22151488)(331.60078848,63.36651474)(331.80079346,63.53651611)
\curveto(332.01078807,63.7065144)(332.17078791,63.93651417)(332.28079346,64.22651611)
}
}
{
\newrgbcolor{curcolor}{0 0 0}
\pscustom[linestyle=none,fillstyle=solid,fillcolor=curcolor]
{
\newpath
\moveto(339.24243408,69.91151611)
\curveto(339.47242929,69.91150819)(339.60242916,69.85150825)(339.63243408,69.73151611)
\curveto(339.6624291,69.62150848)(339.67742909,69.45650865)(339.67743408,69.23651611)
\lineto(339.67743408,68.95151611)
\curveto(339.67742909,68.86150924)(339.65242911,68.78650932)(339.60243408,68.72651611)
\curveto(339.54242922,68.64650946)(339.45742931,68.6015095)(339.34743408,68.59151611)
\curveto(339.23742953,68.59150951)(339.12742964,68.57650953)(339.01743408,68.54651611)
\curveto(338.87742989,68.51650959)(338.74243002,68.48650962)(338.61243408,68.45651611)
\curveto(338.49243027,68.42650968)(338.37743039,68.38650972)(338.26743408,68.33651611)
\curveto(337.97743079,68.2065099)(337.74243102,68.02651008)(337.56243408,67.79651611)
\curveto(337.38243138,67.57651053)(337.22743154,67.32151078)(337.09743408,67.03151611)
\curveto(337.05743171,66.92151118)(337.02743174,66.8065113)(337.00743408,66.68651611)
\curveto(336.98743178,66.57651153)(336.9624318,66.46151164)(336.93243408,66.34151611)
\curveto(336.92243184,66.29151181)(336.91743185,66.24151186)(336.91743408,66.19151611)
\curveto(336.92743184,66.14151196)(336.92743184,66.09151201)(336.91743408,66.04151611)
\curveto(336.88743188,65.92151218)(336.87243189,65.78151232)(336.87243408,65.62151611)
\curveto(336.88243188,65.47151263)(336.88743188,65.32651278)(336.88743408,65.18651611)
\lineto(336.88743408,63.34151611)
\lineto(336.88743408,62.99651611)
\curveto(336.88743188,62.87651523)(336.88243188,62.76151534)(336.87243408,62.65151611)
\curveto(336.8624319,62.54151556)(336.85743191,62.44651566)(336.85743408,62.36651611)
\curveto(336.8674319,62.28651582)(336.84743192,62.21651589)(336.79743408,62.15651611)
\curveto(336.74743202,62.08651602)(336.6674321,62.04651606)(336.55743408,62.03651611)
\curveto(336.45743231,62.02651608)(336.34743242,62.02151608)(336.22743408,62.02151611)
\lineto(335.95743408,62.02151611)
\curveto(335.90743286,62.04151606)(335.85743291,62.05651605)(335.80743408,62.06651611)
\curveto(335.767433,62.08651602)(335.73743303,62.11151599)(335.71743408,62.14151611)
\curveto(335.6674331,62.21151589)(335.63743313,62.29651581)(335.62743408,62.39651611)
\lineto(335.62743408,62.72651611)
\lineto(335.62743408,63.88151611)
\lineto(335.62743408,68.03651611)
\lineto(335.62743408,69.07151611)
\lineto(335.62743408,69.37151611)
\curveto(335.63743313,69.47150863)(335.6674331,69.55650855)(335.71743408,69.62651611)
\curveto(335.74743302,69.66650844)(335.79743297,69.69650841)(335.86743408,69.71651611)
\curveto(335.94743282,69.73650837)(336.03243273,69.74650836)(336.12243408,69.74651611)
\curveto(336.21243255,69.75650835)(336.30243246,69.75650835)(336.39243408,69.74651611)
\curveto(336.48243228,69.73650837)(336.55243221,69.72150838)(336.60243408,69.70151611)
\curveto(336.68243208,69.67150843)(336.73243203,69.61150849)(336.75243408,69.52151611)
\curveto(336.78243198,69.44150866)(336.79743197,69.35150875)(336.79743408,69.25151611)
\lineto(336.79743408,68.95151611)
\curveto(336.79743197,68.85150925)(336.81743195,68.76150934)(336.85743408,68.68151611)
\curveto(336.8674319,68.66150944)(336.87743189,68.64650946)(336.88743408,68.63651611)
\lineto(336.93243408,68.59151611)
\curveto(337.04243172,68.59150951)(337.13243163,68.63650947)(337.20243408,68.72651611)
\curveto(337.27243149,68.82650928)(337.33243143,68.9065092)(337.38243408,68.96651611)
\lineto(337.47243408,69.05651611)
\curveto(337.5624312,69.16650894)(337.68743108,69.28150882)(337.84743408,69.40151611)
\curveto(338.00743076,69.52150858)(338.15743061,69.61150849)(338.29743408,69.67151611)
\curveto(338.38743038,69.72150838)(338.48243028,69.75650835)(338.58243408,69.77651611)
\curveto(338.68243008,69.8065083)(338.78742998,69.83650827)(338.89743408,69.86651611)
\curveto(338.95742981,69.87650823)(339.01742975,69.88150822)(339.07743408,69.88151611)
\curveto(339.13742963,69.89150821)(339.19242957,69.9015082)(339.24243408,69.91151611)
}
}
{
\newrgbcolor{curcolor}{0 0 0}
\pscustom[linestyle=none,fillstyle=solid,fillcolor=curcolor]
{
\newpath
\moveto(463.28160645,72.68651611)
\lineto(464.19660645,72.68651611)
\curveto(464.2966038,72.68650542)(464.3916037,72.68650542)(464.48160645,72.68651611)
\curveto(464.57160352,72.68650542)(464.64660345,72.66650544)(464.70660645,72.62651611)
\curveto(464.7966033,72.56650554)(464.85660324,72.48650562)(464.88660645,72.38651611)
\curveto(464.92660317,72.28650582)(464.97160312,72.18150592)(465.02160645,72.07151611)
\curveto(465.10160299,71.88150622)(465.17160292,71.69150641)(465.23160645,71.50151611)
\curveto(465.30160279,71.31150679)(465.37660272,71.12150698)(465.45660645,70.93151611)
\curveto(465.52660257,70.75150735)(465.5916025,70.56650754)(465.65160645,70.37651611)
\curveto(465.71160238,70.19650791)(465.78160231,70.01650809)(465.86160645,69.83651611)
\curveto(465.92160217,69.69650841)(465.97660212,69.55150855)(466.02660645,69.40151611)
\curveto(466.07660202,69.25150885)(466.13160196,69.106509)(466.19160645,68.96651611)
\curveto(466.37160172,68.51650959)(466.54160155,68.06151004)(466.70160645,67.60151611)
\curveto(466.86160123,67.15151095)(467.03160106,66.7015114)(467.21160645,66.25151611)
\curveto(467.23160086,66.2015119)(467.24660085,66.15151195)(467.25660645,66.10151611)
\lineto(467.31660645,65.95151611)
\curveto(467.40660069,65.73151237)(467.4916006,65.5065126)(467.57160645,65.27651611)
\curveto(467.65160044,65.05651305)(467.73660036,64.83651327)(467.82660645,64.61651611)
\curveto(467.86660023,64.52651358)(467.90660019,64.41651369)(467.94660645,64.28651611)
\curveto(467.98660011,64.16651394)(468.05160004,64.09651401)(468.14160645,64.07651611)
\curveto(468.18159991,64.06651404)(468.21159988,64.06651404)(468.23160645,64.07651611)
\lineto(468.29160645,64.13651611)
\curveto(468.34159975,64.18651392)(468.37659972,64.24151386)(468.39660645,64.30151611)
\curveto(468.42659967,64.36151374)(468.45659964,64.42651368)(468.48660645,64.49651611)
\lineto(468.72660645,65.12651611)
\curveto(468.80659929,65.34651276)(468.88659921,65.56151254)(468.96660645,65.77151611)
\lineto(469.02660645,65.92151611)
\lineto(469.08660645,66.10151611)
\curveto(469.16659893,66.29151181)(469.23659886,66.48151162)(469.29660645,66.67151611)
\curveto(469.36659873,66.87151123)(469.44159865,67.07151103)(469.52160645,67.27151611)
\curveto(469.76159833,67.85151025)(469.98159811,68.43650967)(470.18160645,69.02651611)
\curveto(470.3915977,69.61650849)(470.61659748,70.2015079)(470.85660645,70.78151611)
\curveto(470.93659716,70.98150712)(471.01159708,71.18650692)(471.08160645,71.39651611)
\curveto(471.16159693,71.6065065)(471.24159685,71.81150629)(471.32160645,72.01151611)
\curveto(471.36159673,72.09150601)(471.3965967,72.19150591)(471.42660645,72.31151611)
\curveto(471.46659663,72.43150567)(471.52159657,72.51650559)(471.59160645,72.56651611)
\curveto(471.65159644,72.6065055)(471.72659637,72.63650547)(471.81660645,72.65651611)
\curveto(471.91659618,72.67650543)(472.02659607,72.68650542)(472.14660645,72.68651611)
\curveto(472.26659583,72.69650541)(472.38659571,72.69650541)(472.50660645,72.68651611)
\curveto(472.62659547,72.68650542)(472.73659536,72.68650542)(472.83660645,72.68651611)
\curveto(472.92659517,72.68650542)(473.01659508,72.68650542)(473.10660645,72.68651611)
\curveto(473.20659489,72.68650542)(473.28159481,72.66650544)(473.33160645,72.62651611)
\curveto(473.42159467,72.57650553)(473.47159462,72.48650562)(473.48160645,72.35651611)
\curveto(473.4915946,72.22650588)(473.4965946,72.08650602)(473.49660645,71.93651611)
\lineto(473.49660645,70.28651611)
\lineto(473.49660645,64.01651611)
\lineto(473.49660645,62.75651611)
\curveto(473.4965946,62.64651546)(473.4965946,62.53651557)(473.49660645,62.42651611)
\curveto(473.50659459,62.31651579)(473.48659461,62.23151587)(473.43660645,62.17151611)
\curveto(473.40659469,62.11151599)(473.36159473,62.07151603)(473.30160645,62.05151611)
\curveto(473.24159485,62.04151606)(473.17159492,62.02651608)(473.09160645,62.00651611)
\lineto(472.85160645,62.00651611)
\lineto(472.49160645,62.00651611)
\curveto(472.38159571,62.01651609)(472.30159579,62.06151604)(472.25160645,62.14151611)
\curveto(472.23159586,62.17151593)(472.21659588,62.2015159)(472.20660645,62.23151611)
\curveto(472.20659589,62.27151583)(472.1965959,62.31651579)(472.17660645,62.36651611)
\lineto(472.17660645,62.53151611)
\curveto(472.16659593,62.59151551)(472.16159593,62.66151544)(472.16160645,62.74151611)
\curveto(472.17159592,62.82151528)(472.17659592,62.89651521)(472.17660645,62.96651611)
\lineto(472.17660645,63.80651611)
\lineto(472.17660645,68.23151611)
\curveto(472.17659592,68.48150962)(472.17659592,68.73150937)(472.17660645,68.98151611)
\curveto(472.17659592,69.24150886)(472.17159592,69.49150861)(472.16160645,69.73151611)
\curveto(472.16159593,69.83150827)(472.15659594,69.94150816)(472.14660645,70.06151611)
\curveto(472.13659596,70.18150792)(472.08159601,70.24150786)(471.98160645,70.24151611)
\lineto(471.98160645,70.22651611)
\curveto(471.91159618,70.2065079)(471.85159624,70.14150796)(471.80160645,70.03151611)
\curveto(471.76159633,69.92150818)(471.72659637,69.82650828)(471.69660645,69.74651611)
\curveto(471.62659647,69.57650853)(471.56159653,69.4015087)(471.50160645,69.22151611)
\curveto(471.44159665,69.05150905)(471.37159672,68.88150922)(471.29160645,68.71151611)
\curveto(471.27159682,68.66150944)(471.25659684,68.61650949)(471.24660645,68.57651611)
\curveto(471.23659686,68.53650957)(471.22159687,68.49150961)(471.20160645,68.44151611)
\curveto(471.12159697,68.26150984)(471.05159704,68.07651003)(470.99160645,67.88651611)
\curveto(470.94159715,67.7065104)(470.87659722,67.52651058)(470.79660645,67.34651611)
\curveto(470.72659737,67.19651091)(470.66659743,67.04151106)(470.61660645,66.88151611)
\curveto(470.56659753,66.73151137)(470.51159758,66.58151152)(470.45160645,66.43151611)
\curveto(470.25159784,65.96151214)(470.07159802,65.48651262)(469.91160645,65.00651611)
\curveto(469.75159834,64.53651357)(469.57659852,64.07151403)(469.38660645,63.61151611)
\curveto(469.30659879,63.43151467)(469.23659886,63.25151485)(469.17660645,63.07151611)
\curveto(469.11659898,62.89151521)(469.05159904,62.71151539)(468.98160645,62.53151611)
\curveto(468.93159916,62.42151568)(468.88159921,62.31651579)(468.83160645,62.21651611)
\curveto(468.7915993,62.12651598)(468.70659939,62.06151604)(468.57660645,62.02151611)
\curveto(468.55659954,62.01151609)(468.53159956,62.0065161)(468.50160645,62.00651611)
\curveto(468.48159961,62.01651609)(468.45659964,62.01651609)(468.42660645,62.00651611)
\curveto(468.3965997,61.99651611)(468.36159973,61.99151611)(468.32160645,61.99151611)
\curveto(468.28159981,62.0015161)(468.24159985,62.0065161)(468.20160645,62.00651611)
\lineto(467.90160645,62.00651611)
\curveto(467.80160029,62.0065161)(467.72160037,62.03151607)(467.66160645,62.08151611)
\curveto(467.58160051,62.13151597)(467.52160057,62.2015159)(467.48160645,62.29151611)
\curveto(467.45160064,62.39151571)(467.41160068,62.49151561)(467.36160645,62.59151611)
\curveto(467.28160081,62.79151531)(467.20160089,62.99651511)(467.12160645,63.20651611)
\curveto(467.05160104,63.42651468)(466.97660112,63.63651447)(466.89660645,63.83651611)
\curveto(466.81660128,64.01651409)(466.74660135,64.19651391)(466.68660645,64.37651611)
\curveto(466.63660146,64.56651354)(466.57160152,64.75151335)(466.49160645,64.93151611)
\curveto(466.26160183,65.49151261)(466.04660205,66.05651205)(465.84660645,66.62651611)
\curveto(465.64660245,67.19651091)(465.43160266,67.76151034)(465.20160645,68.32151611)
\lineto(464.96160645,68.95151611)
\curveto(464.8916032,69.17150893)(464.81660328,69.38150872)(464.73660645,69.58151611)
\curveto(464.68660341,69.69150841)(464.64160345,69.79650831)(464.60160645,69.89651611)
\curveto(464.57160352,70.0065081)(464.52160357,70.101508)(464.45160645,70.18151611)
\curveto(464.44160365,70.2015079)(464.43160366,70.21150789)(464.42160645,70.21151611)
\lineto(464.39160645,70.24151611)
\lineto(464.31660645,70.24151611)
\lineto(464.28660645,70.21151611)
\curveto(464.27660382,70.21150789)(464.26660383,70.2065079)(464.25660645,70.19651611)
\curveto(464.23660386,70.14650796)(464.22660387,70.09150801)(464.22660645,70.03151611)
\curveto(464.22660387,69.97150813)(464.21660388,69.91150819)(464.19660645,69.85151611)
\lineto(464.19660645,69.68651611)
\curveto(464.17660392,69.62650848)(464.17160392,69.56150854)(464.18160645,69.49151611)
\curveto(464.1916039,69.42150868)(464.1966039,69.35150875)(464.19660645,69.28151611)
\lineto(464.19660645,68.47151611)
\lineto(464.19660645,63.91151611)
\lineto(464.19660645,62.72651611)
\curveto(464.1966039,62.61651549)(464.1916039,62.5065156)(464.18160645,62.39651611)
\curveto(464.18160391,62.28651582)(464.15660394,62.2015159)(464.10660645,62.14151611)
\curveto(464.05660404,62.06151604)(463.96660413,62.01651609)(463.83660645,62.00651611)
\lineto(463.44660645,62.00651611)
\lineto(463.25160645,62.00651611)
\curveto(463.20160489,62.0065161)(463.15160494,62.01651609)(463.10160645,62.03651611)
\curveto(462.97160512,62.07651603)(462.8966052,62.16151594)(462.87660645,62.29151611)
\curveto(462.86660523,62.42151568)(462.86160523,62.57151553)(462.86160645,62.74151611)
\lineto(462.86160645,64.48151611)
\lineto(462.86160645,70.48151611)
\lineto(462.86160645,71.89151611)
\curveto(462.86160523,72.0015061)(462.85660524,72.11650599)(462.84660645,72.23651611)
\curveto(462.84660525,72.35650575)(462.87160522,72.45150565)(462.92160645,72.52151611)
\curveto(462.96160513,72.58150552)(463.03660506,72.63150547)(463.14660645,72.67151611)
\curveto(463.16660493,72.68150542)(463.18660491,72.68150542)(463.20660645,72.67151611)
\curveto(463.23660486,72.67150543)(463.26160483,72.67650543)(463.28160645,72.68651611)
}
}
{
\newrgbcolor{curcolor}{0 0 0}
\pscustom[linestyle=none,fillstyle=solid,fillcolor=curcolor]
{
\newpath
\moveto(482.72371582,66.20651611)
\curveto(482.74370776,66.14651196)(482.75370775,66.05151205)(482.75371582,65.92151611)
\curveto(482.75370775,65.8015123)(482.74870776,65.71651239)(482.73871582,65.66651611)
\lineto(482.73871582,65.51651611)
\curveto(482.72870778,65.43651267)(482.71870779,65.36151274)(482.70871582,65.29151611)
\curveto(482.7087078,65.23151287)(482.7037078,65.16151294)(482.69371582,65.08151611)
\curveto(482.67370783,65.02151308)(482.65870785,64.96151314)(482.64871582,64.90151611)
\curveto(482.64870786,64.84151326)(482.63870787,64.78151332)(482.61871582,64.72151611)
\curveto(482.57870793,64.59151351)(482.54370796,64.46151364)(482.51371582,64.33151611)
\curveto(482.48370802,64.2015139)(482.44370806,64.08151402)(482.39371582,63.97151611)
\curveto(482.18370832,63.49151461)(481.9037086,63.08651502)(481.55371582,62.75651611)
\curveto(481.2037093,62.43651567)(480.77370973,62.19151591)(480.26371582,62.02151611)
\curveto(480.15371035,61.98151612)(480.03371047,61.95151615)(479.90371582,61.93151611)
\curveto(479.78371072,61.91151619)(479.65871085,61.89151621)(479.52871582,61.87151611)
\curveto(479.46871104,61.86151624)(479.4037111,61.85651625)(479.33371582,61.85651611)
\curveto(479.27371123,61.84651626)(479.21371129,61.84151626)(479.15371582,61.84151611)
\curveto(479.11371139,61.83151627)(479.05371145,61.82651628)(478.97371582,61.82651611)
\curveto(478.9037116,61.82651628)(478.85371165,61.83151627)(478.82371582,61.84151611)
\curveto(478.78371172,61.85151625)(478.74371176,61.85651625)(478.70371582,61.85651611)
\curveto(478.66371184,61.84651626)(478.62871188,61.84651626)(478.59871582,61.85651611)
\lineto(478.50871582,61.85651611)
\lineto(478.14871582,61.90151611)
\curveto(478.0087125,61.94151616)(477.87371263,61.98151612)(477.74371582,62.02151611)
\curveto(477.61371289,62.06151604)(477.48871302,62.106516)(477.36871582,62.15651611)
\curveto(476.91871359,62.35651575)(476.54871396,62.61651549)(476.25871582,62.93651611)
\curveto(475.96871454,63.25651485)(475.72871478,63.64651446)(475.53871582,64.10651611)
\curveto(475.48871502,64.2065139)(475.44871506,64.3065138)(475.41871582,64.40651611)
\curveto(475.39871511,64.5065136)(475.37871513,64.61151349)(475.35871582,64.72151611)
\curveto(475.33871517,64.76151334)(475.32871518,64.79151331)(475.32871582,64.81151611)
\curveto(475.33871517,64.84151326)(475.33871517,64.87651323)(475.32871582,64.91651611)
\curveto(475.3087152,64.99651311)(475.29371521,65.07651303)(475.28371582,65.15651611)
\curveto(475.28371522,65.24651286)(475.27371523,65.33151277)(475.25371582,65.41151611)
\lineto(475.25371582,65.53151611)
\curveto(475.25371525,65.57151253)(475.24871526,65.61651249)(475.23871582,65.66651611)
\curveto(475.22871528,65.71651239)(475.22371528,65.8015123)(475.22371582,65.92151611)
\curveto(475.22371528,66.05151205)(475.23371527,66.14651196)(475.25371582,66.20651611)
\curveto(475.27371523,66.27651183)(475.27871523,66.34651176)(475.26871582,66.41651611)
\curveto(475.25871525,66.48651162)(475.26371524,66.55651155)(475.28371582,66.62651611)
\curveto(475.29371521,66.67651143)(475.29871521,66.71651139)(475.29871582,66.74651611)
\curveto(475.3087152,66.78651132)(475.31871519,66.83151127)(475.32871582,66.88151611)
\curveto(475.35871515,67.0015111)(475.38371512,67.12151098)(475.40371582,67.24151611)
\curveto(475.43371507,67.36151074)(475.47371503,67.47651063)(475.52371582,67.58651611)
\curveto(475.67371483,67.95651015)(475.85371465,68.28650982)(476.06371582,68.57651611)
\curveto(476.28371422,68.87650923)(476.54871396,69.12650898)(476.85871582,69.32651611)
\curveto(476.97871353,69.4065087)(477.1037134,69.47150863)(477.23371582,69.52151611)
\curveto(477.36371314,69.58150852)(477.49871301,69.64150846)(477.63871582,69.70151611)
\curveto(477.75871275,69.75150835)(477.88871262,69.78150832)(478.02871582,69.79151611)
\curveto(478.16871234,69.81150829)(478.3087122,69.84150826)(478.44871582,69.88151611)
\lineto(478.64371582,69.88151611)
\curveto(478.71371179,69.89150821)(478.77871173,69.9015082)(478.83871582,69.91151611)
\curveto(479.72871078,69.92150818)(480.46871004,69.73650837)(481.05871582,69.35651611)
\curveto(481.64870886,68.97650913)(482.07370843,68.48150962)(482.33371582,67.87151611)
\curveto(482.38370812,67.77151033)(482.42370808,67.67151043)(482.45371582,67.57151611)
\curveto(482.48370802,67.47151063)(482.51870799,67.36651074)(482.55871582,67.25651611)
\curveto(482.58870792,67.14651096)(482.61370789,67.02651108)(482.63371582,66.89651611)
\curveto(482.65370785,66.77651133)(482.67870783,66.65151145)(482.70871582,66.52151611)
\curveto(482.71870779,66.47151163)(482.71870779,66.41651169)(482.70871582,66.35651611)
\curveto(482.7087078,66.3065118)(482.71370779,66.25651185)(482.72371582,66.20651611)
\moveto(481.38871582,65.35151611)
\curveto(481.4087091,65.42151268)(481.41370909,65.5015126)(481.40371582,65.59151611)
\lineto(481.40371582,65.84651611)
\curveto(481.4037091,66.23651187)(481.36870914,66.56651154)(481.29871582,66.83651611)
\curveto(481.26870924,66.91651119)(481.24370926,66.99651111)(481.22371582,67.07651611)
\curveto(481.2037093,67.15651095)(481.17870933,67.23151087)(481.14871582,67.30151611)
\curveto(480.86870964,67.95151015)(480.42371008,68.4015097)(479.81371582,68.65151611)
\curveto(479.74371076,68.68150942)(479.66871084,68.7015094)(479.58871582,68.71151611)
\lineto(479.34871582,68.77151611)
\curveto(479.26871124,68.79150931)(479.18371132,68.8015093)(479.09371582,68.80151611)
\lineto(478.82371582,68.80151611)
\lineto(478.55371582,68.75651611)
\curveto(478.45371205,68.73650937)(478.35871215,68.71150939)(478.26871582,68.68151611)
\curveto(478.18871232,68.66150944)(478.1087124,68.63150947)(478.02871582,68.59151611)
\curveto(477.95871255,68.57150953)(477.89371261,68.54150956)(477.83371582,68.50151611)
\curveto(477.77371273,68.46150964)(477.71871279,68.42150968)(477.66871582,68.38151611)
\curveto(477.42871308,68.21150989)(477.23371327,68.0065101)(477.08371582,67.76651611)
\curveto(476.93371357,67.52651058)(476.8037137,67.24651086)(476.69371582,66.92651611)
\curveto(476.66371384,66.82651128)(476.64371386,66.72151138)(476.63371582,66.61151611)
\curveto(476.62371388,66.51151159)(476.6087139,66.4065117)(476.58871582,66.29651611)
\curveto(476.57871393,66.25651185)(476.57371393,66.19151191)(476.57371582,66.10151611)
\curveto(476.56371394,66.07151203)(476.55871395,66.03651207)(476.55871582,65.99651611)
\curveto(476.56871394,65.95651215)(476.57371393,65.91151219)(476.57371582,65.86151611)
\lineto(476.57371582,65.56151611)
\curveto(476.57371393,65.46151264)(476.58371392,65.37151273)(476.60371582,65.29151611)
\lineto(476.63371582,65.11151611)
\curveto(476.65371385,65.01151309)(476.66871384,64.91151319)(476.67871582,64.81151611)
\curveto(476.69871381,64.72151338)(476.72871378,64.63651347)(476.76871582,64.55651611)
\curveto(476.86871364,64.31651379)(476.98371352,64.09151401)(477.11371582,63.88151611)
\curveto(477.25371325,63.67151443)(477.42371308,63.49651461)(477.62371582,63.35651611)
\curveto(477.67371283,63.32651478)(477.71871279,63.3015148)(477.75871582,63.28151611)
\curveto(477.79871271,63.26151484)(477.84371266,63.23651487)(477.89371582,63.20651611)
\curveto(477.97371253,63.15651495)(478.05871245,63.11151499)(478.14871582,63.07151611)
\curveto(478.24871226,63.04151506)(478.35371215,63.01151509)(478.46371582,62.98151611)
\curveto(478.51371199,62.96151514)(478.55871195,62.95151515)(478.59871582,62.95151611)
\curveto(478.64871186,62.96151514)(478.69871181,62.96151514)(478.74871582,62.95151611)
\curveto(478.77871173,62.94151516)(478.83871167,62.93151517)(478.92871582,62.92151611)
\curveto(479.02871148,62.91151519)(479.1037114,62.91651519)(479.15371582,62.93651611)
\curveto(479.19371131,62.94651516)(479.23371127,62.94651516)(479.27371582,62.93651611)
\curveto(479.31371119,62.93651517)(479.35371115,62.94651516)(479.39371582,62.96651611)
\curveto(479.47371103,62.98651512)(479.55371095,63.0015151)(479.63371582,63.01151611)
\curveto(479.71371079,63.03151507)(479.78871072,63.05651505)(479.85871582,63.08651611)
\curveto(480.19871031,63.22651488)(480.47371003,63.42151468)(480.68371582,63.67151611)
\curveto(480.89370961,63.92151418)(481.06870944,64.21651389)(481.20871582,64.55651611)
\curveto(481.25870925,64.67651343)(481.28870922,64.8015133)(481.29871582,64.93151611)
\curveto(481.31870919,65.07151303)(481.34870916,65.21151289)(481.38871582,65.35151611)
}
}
{
\newrgbcolor{curcolor}{0 0 0}
\pscustom[linestyle=none,fillstyle=solid,fillcolor=curcolor]
{
\newpath
\moveto(491.17199707,62.81651611)
\lineto(491.17199707,62.42651611)
\curveto(491.1719892,62.3065158)(491.14698922,62.2065159)(491.09699707,62.12651611)
\curveto(491.04698932,62.05651605)(490.96198941,62.01651609)(490.84199707,62.00651611)
\lineto(490.49699707,62.00651611)
\curveto(490.43698993,62.0065161)(490.37698999,62.0015161)(490.31699707,61.99151611)
\curveto(490.2669901,61.99151611)(490.22199015,62.0015161)(490.18199707,62.02151611)
\curveto(490.09199028,62.04151606)(490.03199034,62.08151602)(490.00199707,62.14151611)
\curveto(489.96199041,62.19151591)(489.93699043,62.25151585)(489.92699707,62.32151611)
\curveto(489.92699044,62.39151571)(489.91199046,62.46151564)(489.88199707,62.53151611)
\curveto(489.8719905,62.55151555)(489.85699051,62.56651554)(489.83699707,62.57651611)
\curveto(489.82699054,62.59651551)(489.81199056,62.61651549)(489.79199707,62.63651611)
\curveto(489.69199068,62.64651546)(489.61199076,62.62651548)(489.55199707,62.57651611)
\curveto(489.50199087,62.52651558)(489.44699092,62.47651563)(489.38699707,62.42651611)
\curveto(489.18699118,62.27651583)(488.98699138,62.16151594)(488.78699707,62.08151611)
\curveto(488.60699176,62.0015161)(488.39699197,61.94151616)(488.15699707,61.90151611)
\curveto(487.92699244,61.86151624)(487.68699268,61.84151626)(487.43699707,61.84151611)
\curveto(487.19699317,61.83151627)(486.95699341,61.84651626)(486.71699707,61.88651611)
\curveto(486.47699389,61.91651619)(486.2669941,61.97151613)(486.08699707,62.05151611)
\curveto(485.5669948,62.27151583)(485.14699522,62.56651554)(484.82699707,62.93651611)
\curveto(484.50699586,63.31651479)(484.25699611,63.78651432)(484.07699707,64.34651611)
\curveto(484.03699633,64.43651367)(484.00699636,64.52651358)(483.98699707,64.61651611)
\curveto(483.97699639,64.71651339)(483.95699641,64.81651329)(483.92699707,64.91651611)
\curveto(483.91699645,64.96651314)(483.91199646,65.01651309)(483.91199707,65.06651611)
\curveto(483.91199646,65.11651299)(483.90699646,65.16651294)(483.89699707,65.21651611)
\curveto(483.87699649,65.26651284)(483.8669965,65.31651279)(483.86699707,65.36651611)
\curveto(483.87699649,65.42651268)(483.87699649,65.48151262)(483.86699707,65.53151611)
\lineto(483.86699707,65.68151611)
\curveto(483.84699652,65.73151237)(483.83699653,65.79651231)(483.83699707,65.87651611)
\curveto(483.83699653,65.95651215)(483.84699652,66.02151208)(483.86699707,66.07151611)
\lineto(483.86699707,66.23651611)
\curveto(483.88699648,66.3065118)(483.89199648,66.37651173)(483.88199707,66.44651611)
\curveto(483.88199649,66.52651158)(483.89199648,66.6015115)(483.91199707,66.67151611)
\curveto(483.92199645,66.72151138)(483.92699644,66.76651134)(483.92699707,66.80651611)
\curveto(483.92699644,66.84651126)(483.93199644,66.89151121)(483.94199707,66.94151611)
\curveto(483.9719964,67.04151106)(483.99699637,67.13651097)(484.01699707,67.22651611)
\curveto(484.03699633,67.32651078)(484.06199631,67.42151068)(484.09199707,67.51151611)
\curveto(484.22199615,67.89151021)(484.38699598,68.23150987)(484.58699707,68.53151611)
\curveto(484.79699557,68.84150926)(485.04699532,69.09650901)(485.33699707,69.29651611)
\curveto(485.50699486,69.41650869)(485.68199469,69.51650859)(485.86199707,69.59651611)
\curveto(486.05199432,69.67650843)(486.25699411,69.74650836)(486.47699707,69.80651611)
\curveto(486.54699382,69.81650829)(486.61199376,69.82650828)(486.67199707,69.83651611)
\curveto(486.74199363,69.84650826)(486.81199356,69.86150824)(486.88199707,69.88151611)
\lineto(487.03199707,69.88151611)
\curveto(487.11199326,69.9015082)(487.22699314,69.91150819)(487.37699707,69.91151611)
\curveto(487.53699283,69.91150819)(487.65699271,69.9015082)(487.73699707,69.88151611)
\curveto(487.77699259,69.87150823)(487.83199254,69.86650824)(487.90199707,69.86651611)
\curveto(488.01199236,69.83650827)(488.12199225,69.81150829)(488.23199707,69.79151611)
\curveto(488.34199203,69.78150832)(488.44699192,69.75150835)(488.54699707,69.70151611)
\curveto(488.69699167,69.64150846)(488.83699153,69.57650853)(488.96699707,69.50651611)
\curveto(489.10699126,69.43650867)(489.23699113,69.35650875)(489.35699707,69.26651611)
\curveto(489.41699095,69.21650889)(489.47699089,69.16150894)(489.53699707,69.10151611)
\curveto(489.60699076,69.05150905)(489.69699067,69.03650907)(489.80699707,69.05651611)
\curveto(489.82699054,69.08650902)(489.84199053,69.11150899)(489.85199707,69.13151611)
\curveto(489.8719905,69.15150895)(489.88699048,69.18150892)(489.89699707,69.22151611)
\curveto(489.92699044,69.31150879)(489.93699043,69.42650868)(489.92699707,69.56651611)
\lineto(489.92699707,69.94151611)
\lineto(489.92699707,71.66651611)
\lineto(489.92699707,72.13151611)
\curveto(489.92699044,72.31150579)(489.95199042,72.44150566)(490.00199707,72.52151611)
\curveto(490.04199033,72.59150551)(490.10199027,72.63650547)(490.18199707,72.65651611)
\curveto(490.20199017,72.65650545)(490.22699014,72.65650545)(490.25699707,72.65651611)
\curveto(490.28699008,72.66650544)(490.31199006,72.67150543)(490.33199707,72.67151611)
\curveto(490.4719899,72.68150542)(490.61698975,72.68150542)(490.76699707,72.67151611)
\curveto(490.92698944,72.67150543)(491.03698933,72.63150547)(491.09699707,72.55151611)
\curveto(491.14698922,72.47150563)(491.1719892,72.37150573)(491.17199707,72.25151611)
\lineto(491.17199707,71.87651611)
\lineto(491.17199707,62.81651611)
\moveto(489.95699707,65.65151611)
\curveto(489.97699039,65.7015124)(489.98699038,65.76651234)(489.98699707,65.84651611)
\curveto(489.98699038,65.93651217)(489.97699039,66.0065121)(489.95699707,66.05651611)
\lineto(489.95699707,66.28151611)
\curveto(489.93699043,66.37151173)(489.92199045,66.46151164)(489.91199707,66.55151611)
\curveto(489.90199047,66.65151145)(489.88199049,66.74151136)(489.85199707,66.82151611)
\curveto(489.83199054,66.9015112)(489.81199056,66.97651113)(489.79199707,67.04651611)
\curveto(489.78199059,67.11651099)(489.76199061,67.18651092)(489.73199707,67.25651611)
\curveto(489.61199076,67.55651055)(489.45699091,67.82151028)(489.26699707,68.05151611)
\curveto(489.07699129,68.28150982)(488.83699153,68.46150964)(488.54699707,68.59151611)
\curveto(488.44699192,68.64150946)(488.34199203,68.67650943)(488.23199707,68.69651611)
\curveto(488.13199224,68.72650938)(488.02199235,68.75150935)(487.90199707,68.77151611)
\curveto(487.82199255,68.79150931)(487.73199264,68.8015093)(487.63199707,68.80151611)
\lineto(487.36199707,68.80151611)
\curveto(487.31199306,68.79150931)(487.2669931,68.78150932)(487.22699707,68.77151611)
\lineto(487.09199707,68.77151611)
\curveto(487.01199336,68.75150935)(486.92699344,68.73150937)(486.83699707,68.71151611)
\curveto(486.75699361,68.69150941)(486.67699369,68.66650944)(486.59699707,68.63651611)
\curveto(486.27699409,68.49650961)(486.01699435,68.29150981)(485.81699707,68.02151611)
\curveto(485.62699474,67.76151034)(485.4719949,67.45651065)(485.35199707,67.10651611)
\curveto(485.31199506,66.99651111)(485.28199509,66.88151122)(485.26199707,66.76151611)
\curveto(485.25199512,66.65151145)(485.23699513,66.54151156)(485.21699707,66.43151611)
\curveto(485.21699515,66.39151171)(485.21199516,66.35151175)(485.20199707,66.31151611)
\lineto(485.20199707,66.20651611)
\curveto(485.18199519,66.15651195)(485.1719952,66.101512)(485.17199707,66.04151611)
\curveto(485.18199519,65.98151212)(485.18699518,65.92651218)(485.18699707,65.87651611)
\lineto(485.18699707,65.54651611)
\curveto(485.18699518,65.44651266)(485.19699517,65.35151275)(485.21699707,65.26151611)
\curveto(485.22699514,65.23151287)(485.23199514,65.18151292)(485.23199707,65.11151611)
\curveto(485.25199512,65.04151306)(485.2669951,64.97151313)(485.27699707,64.90151611)
\lineto(485.33699707,64.69151611)
\curveto(485.44699492,64.34151376)(485.59699477,64.04151406)(485.78699707,63.79151611)
\curveto(485.97699439,63.54151456)(486.21699415,63.33651477)(486.50699707,63.17651611)
\curveto(486.59699377,63.12651498)(486.68699368,63.08651502)(486.77699707,63.05651611)
\curveto(486.8669935,63.02651508)(486.9669934,62.99651511)(487.07699707,62.96651611)
\curveto(487.12699324,62.94651516)(487.17699319,62.94151516)(487.22699707,62.95151611)
\curveto(487.28699308,62.96151514)(487.34199303,62.95651515)(487.39199707,62.93651611)
\curveto(487.43199294,62.92651518)(487.4719929,62.92151518)(487.51199707,62.92151611)
\lineto(487.64699707,62.92151611)
\lineto(487.78199707,62.92151611)
\curveto(487.81199256,62.93151517)(487.86199251,62.93651517)(487.93199707,62.93651611)
\curveto(488.01199236,62.95651515)(488.09199228,62.97151513)(488.17199707,62.98151611)
\curveto(488.25199212,63.0015151)(488.32699204,63.02651508)(488.39699707,63.05651611)
\curveto(488.72699164,63.19651491)(488.99199138,63.37151473)(489.19199707,63.58151611)
\curveto(489.40199097,63.8015143)(489.57699079,64.07651403)(489.71699707,64.40651611)
\curveto(489.7669906,64.51651359)(489.80199057,64.62651348)(489.82199707,64.73651611)
\curveto(489.84199053,64.84651326)(489.8669905,64.95651315)(489.89699707,65.06651611)
\curveto(489.91699045,65.106513)(489.92699044,65.14151296)(489.92699707,65.17151611)
\curveto(489.92699044,65.21151289)(489.93199044,65.25151285)(489.94199707,65.29151611)
\curveto(489.95199042,65.35151275)(489.95199042,65.41151269)(489.94199707,65.47151611)
\curveto(489.94199043,65.53151257)(489.94699042,65.59151251)(489.95699707,65.65151611)
}
}
{
\newrgbcolor{curcolor}{0 0 0}
\pscustom[linestyle=none,fillstyle=solid,fillcolor=curcolor]
{
\newpath
\moveto(499.86824707,66.17651611)
\curveto(499.88823939,66.07651203)(499.88823939,65.96151214)(499.86824707,65.83151611)
\curveto(499.85823942,65.71151239)(499.82823945,65.62651248)(499.77824707,65.57651611)
\curveto(499.72823955,65.53651257)(499.65323962,65.5065126)(499.55324707,65.48651611)
\curveto(499.46323981,65.47651263)(499.35823992,65.47151263)(499.23824707,65.47151611)
\lineto(498.87824707,65.47151611)
\curveto(498.75824052,65.48151262)(498.65324062,65.48651262)(498.56324707,65.48651611)
\lineto(494.72324707,65.48651611)
\curveto(494.64324463,65.48651262)(494.56324471,65.48151262)(494.48324707,65.47151611)
\curveto(494.40324487,65.47151263)(494.33824494,65.45651265)(494.28824707,65.42651611)
\curveto(494.24824503,65.4065127)(494.20824507,65.36651274)(494.16824707,65.30651611)
\curveto(494.14824513,65.27651283)(494.12824515,65.23151287)(494.10824707,65.17151611)
\curveto(494.08824519,65.12151298)(494.08824519,65.07151303)(494.10824707,65.02151611)
\curveto(494.11824516,64.97151313)(494.12324515,64.92651318)(494.12324707,64.88651611)
\curveto(494.12324515,64.84651326)(494.12824515,64.8065133)(494.13824707,64.76651611)
\curveto(494.15824512,64.68651342)(494.1782451,64.6015135)(494.19824707,64.51151611)
\curveto(494.21824506,64.43151367)(494.24824503,64.35151375)(494.28824707,64.27151611)
\curveto(494.51824476,63.73151437)(494.89824438,63.34651476)(495.42824707,63.11651611)
\curveto(495.48824379,63.08651502)(495.55324372,63.06151504)(495.62324707,63.04151611)
\lineto(495.83324707,62.98151611)
\curveto(495.86324341,62.97151513)(495.91324336,62.96651514)(495.98324707,62.96651611)
\curveto(496.12324315,62.92651518)(496.30824297,62.9065152)(496.53824707,62.90651611)
\curveto(496.76824251,62.9065152)(496.95324232,62.92651518)(497.09324707,62.96651611)
\curveto(497.23324204,63.0065151)(497.35824192,63.04651506)(497.46824707,63.08651611)
\curveto(497.58824169,63.13651497)(497.69824158,63.19651491)(497.79824707,63.26651611)
\curveto(497.90824137,63.33651477)(498.00324127,63.41651469)(498.08324707,63.50651611)
\curveto(498.16324111,63.6065145)(498.23324104,63.71151439)(498.29324707,63.82151611)
\curveto(498.35324092,63.92151418)(498.40324087,64.02651408)(498.44324707,64.13651611)
\curveto(498.49324078,64.24651386)(498.5732407,64.32651378)(498.68324707,64.37651611)
\curveto(498.72324055,64.39651371)(498.78824049,64.41151369)(498.87824707,64.42151611)
\curveto(498.96824031,64.43151367)(499.05824022,64.43151367)(499.14824707,64.42151611)
\curveto(499.23824004,64.42151368)(499.32323995,64.41651369)(499.40324707,64.40651611)
\curveto(499.48323979,64.39651371)(499.53823974,64.37651373)(499.56824707,64.34651611)
\curveto(499.66823961,64.27651383)(499.69323958,64.16151394)(499.64324707,64.00151611)
\curveto(499.56323971,63.73151437)(499.45823982,63.49151461)(499.32824707,63.28151611)
\curveto(499.12824015,62.96151514)(498.89824038,62.69651541)(498.63824707,62.48651611)
\curveto(498.38824089,62.28651582)(498.06824121,62.12151598)(497.67824707,61.99151611)
\curveto(497.5782417,61.95151615)(497.4782418,61.92651618)(497.37824707,61.91651611)
\curveto(497.278242,61.89651621)(497.1732421,61.87651623)(497.06324707,61.85651611)
\curveto(497.01324226,61.84651626)(496.96324231,61.84151626)(496.91324707,61.84151611)
\curveto(496.8732424,61.84151626)(496.82824245,61.83651627)(496.77824707,61.82651611)
\lineto(496.62824707,61.82651611)
\curveto(496.5782427,61.81651629)(496.51824276,61.81151629)(496.44824707,61.81151611)
\curveto(496.38824289,61.81151629)(496.33824294,61.81651629)(496.29824707,61.82651611)
\lineto(496.16324707,61.82651611)
\curveto(496.11324316,61.83651627)(496.06824321,61.84151626)(496.02824707,61.84151611)
\curveto(495.98824329,61.84151626)(495.94824333,61.84651626)(495.90824707,61.85651611)
\curveto(495.85824342,61.86651624)(495.80324347,61.87651623)(495.74324707,61.88651611)
\curveto(495.68324359,61.88651622)(495.62824365,61.89151621)(495.57824707,61.90151611)
\curveto(495.48824379,61.92151618)(495.39824388,61.94651616)(495.30824707,61.97651611)
\curveto(495.21824406,61.99651611)(495.13324414,62.02151608)(495.05324707,62.05151611)
\curveto(495.01324426,62.07151603)(494.9782443,62.08151602)(494.94824707,62.08151611)
\curveto(494.91824436,62.09151601)(494.88324439,62.106516)(494.84324707,62.12651611)
\curveto(494.69324458,62.19651591)(494.53324474,62.28151582)(494.36324707,62.38151611)
\curveto(494.0732452,62.57151553)(493.82324545,62.8015153)(493.61324707,63.07151611)
\curveto(493.41324586,63.35151475)(493.24324603,63.66151444)(493.10324707,64.00151611)
\curveto(493.05324622,64.11151399)(493.01324626,64.22651388)(492.98324707,64.34651611)
\curveto(492.96324631,64.46651364)(492.93324634,64.58651352)(492.89324707,64.70651611)
\curveto(492.88324639,64.74651336)(492.8782464,64.78151332)(492.87824707,64.81151611)
\curveto(492.8782464,64.84151326)(492.8732464,64.88151322)(492.86324707,64.93151611)
\curveto(492.84324643,65.01151309)(492.82824645,65.09651301)(492.81824707,65.18651611)
\curveto(492.80824647,65.27651283)(492.79324648,65.36651274)(492.77324707,65.45651611)
\lineto(492.77324707,65.66651611)
\curveto(492.76324651,65.7065124)(492.75324652,65.76151234)(492.74324707,65.83151611)
\curveto(492.74324653,65.91151219)(492.74824653,65.97651213)(492.75824707,66.02651611)
\lineto(492.75824707,66.19151611)
\curveto(492.7782465,66.24151186)(492.78324649,66.29151181)(492.77324707,66.34151611)
\curveto(492.7732465,66.4015117)(492.7782465,66.45651165)(492.78824707,66.50651611)
\curveto(492.82824645,66.66651144)(492.85824642,66.82651128)(492.87824707,66.98651611)
\curveto(492.90824637,67.14651096)(492.95324632,67.29651081)(493.01324707,67.43651611)
\curveto(493.06324621,67.54651056)(493.10824617,67.65651045)(493.14824707,67.76651611)
\curveto(493.19824608,67.88651022)(493.25324602,68.0015101)(493.31324707,68.11151611)
\curveto(493.53324574,68.46150964)(493.78324549,68.76150934)(494.06324707,69.01151611)
\curveto(494.34324493,69.27150883)(494.68824459,69.48650862)(495.09824707,69.65651611)
\curveto(495.21824406,69.7065084)(495.33824394,69.74150836)(495.45824707,69.76151611)
\curveto(495.58824369,69.79150831)(495.72324355,69.82150828)(495.86324707,69.85151611)
\curveto(495.91324336,69.86150824)(495.95824332,69.86650824)(495.99824707,69.86651611)
\curveto(496.03824324,69.87650823)(496.08324319,69.88150822)(496.13324707,69.88151611)
\curveto(496.15324312,69.89150821)(496.1782431,69.89150821)(496.20824707,69.88151611)
\curveto(496.23824304,69.87150823)(496.26324301,69.87650823)(496.28324707,69.89651611)
\curveto(496.70324257,69.9065082)(497.06824221,69.86150824)(497.37824707,69.76151611)
\curveto(497.68824159,69.67150843)(497.96824131,69.54650856)(498.21824707,69.38651611)
\curveto(498.26824101,69.36650874)(498.30824097,69.33650877)(498.33824707,69.29651611)
\curveto(498.36824091,69.26650884)(498.40324087,69.24150886)(498.44324707,69.22151611)
\curveto(498.52324075,69.16150894)(498.60324067,69.09150901)(498.68324707,69.01151611)
\curveto(498.7732405,68.93150917)(498.84824043,68.85150925)(498.90824707,68.77151611)
\curveto(499.06824021,68.56150954)(499.20324007,68.36150974)(499.31324707,68.17151611)
\curveto(499.38323989,68.06151004)(499.43823984,67.94151016)(499.47824707,67.81151611)
\curveto(499.51823976,67.68151042)(499.56323971,67.55151055)(499.61324707,67.42151611)
\curveto(499.66323961,67.29151081)(499.69823958,67.15651095)(499.71824707,67.01651611)
\curveto(499.74823953,66.87651123)(499.78323949,66.73651137)(499.82324707,66.59651611)
\curveto(499.83323944,66.52651158)(499.83823944,66.45651165)(499.83824707,66.38651611)
\lineto(499.86824707,66.17651611)
\moveto(498.41324707,66.68651611)
\curveto(498.44324083,66.72651138)(498.46824081,66.77651133)(498.48824707,66.83651611)
\curveto(498.50824077,66.9065112)(498.50824077,66.97651113)(498.48824707,67.04651611)
\curveto(498.42824085,67.26651084)(498.34324093,67.47151063)(498.23324707,67.66151611)
\curveto(498.09324118,67.89151021)(497.93824134,68.08651002)(497.76824707,68.24651611)
\curveto(497.59824168,68.4065097)(497.3782419,68.54150956)(497.10824707,68.65151611)
\curveto(497.03824224,68.67150943)(496.96824231,68.68650942)(496.89824707,68.69651611)
\curveto(496.82824245,68.71650939)(496.75324252,68.73650937)(496.67324707,68.75651611)
\curveto(496.59324268,68.77650933)(496.50824277,68.78650932)(496.41824707,68.78651611)
\lineto(496.16324707,68.78651611)
\curveto(496.13324314,68.76650934)(496.09824318,68.75650935)(496.05824707,68.75651611)
\curveto(496.01824326,68.76650934)(495.98324329,68.76650934)(495.95324707,68.75651611)
\lineto(495.71324707,68.69651611)
\curveto(495.64324363,68.68650942)(495.5732437,68.67150943)(495.50324707,68.65151611)
\curveto(495.21324406,68.53150957)(494.9782443,68.38150972)(494.79824707,68.20151611)
\curveto(494.62824465,68.02151008)(494.4732448,67.79651031)(494.33324707,67.52651611)
\curveto(494.30324497,67.47651063)(494.273245,67.41151069)(494.24324707,67.33151611)
\curveto(494.21324506,67.26151084)(494.18824509,67.18151092)(494.16824707,67.09151611)
\curveto(494.14824513,67.0015111)(494.14324513,66.91651119)(494.15324707,66.83651611)
\curveto(494.16324511,66.75651135)(494.19824508,66.69651141)(494.25824707,66.65651611)
\curveto(494.33824494,66.59651151)(494.4732448,66.56651154)(494.66324707,66.56651611)
\curveto(494.86324441,66.57651153)(495.03324424,66.58151152)(495.17324707,66.58151611)
\lineto(497.45324707,66.58151611)
\curveto(497.60324167,66.58151152)(497.78324149,66.57651153)(497.99324707,66.56651611)
\curveto(498.20324107,66.56651154)(498.34324093,66.6065115)(498.41324707,66.68651611)
}
}
{
\newrgbcolor{curcolor}{0 0 0}
\pscustom[linestyle=none,fillstyle=solid,fillcolor=curcolor]
{
\newpath
\moveto(504.8198877,69.91151611)
\curveto(505.04988291,69.91150819)(505.17988278,69.85150825)(505.2098877,69.73151611)
\curveto(505.23988272,69.62150848)(505.2548827,69.45650865)(505.2548877,69.23651611)
\lineto(505.2548877,68.95151611)
\curveto(505.2548827,68.86150924)(505.22988273,68.78650932)(505.1798877,68.72651611)
\curveto(505.11988284,68.64650946)(505.03488292,68.6015095)(504.9248877,68.59151611)
\curveto(504.81488314,68.59150951)(504.70488325,68.57650953)(504.5948877,68.54651611)
\curveto(504.4548835,68.51650959)(504.31988364,68.48650962)(504.1898877,68.45651611)
\curveto(504.06988389,68.42650968)(503.954884,68.38650972)(503.8448877,68.33651611)
\curveto(503.5548844,68.2065099)(503.31988464,68.02651008)(503.1398877,67.79651611)
\curveto(502.959885,67.57651053)(502.80488515,67.32151078)(502.6748877,67.03151611)
\curveto(502.63488532,66.92151118)(502.60488535,66.8065113)(502.5848877,66.68651611)
\curveto(502.56488539,66.57651153)(502.53988542,66.46151164)(502.5098877,66.34151611)
\curveto(502.49988546,66.29151181)(502.49488546,66.24151186)(502.4948877,66.19151611)
\curveto(502.50488545,66.14151196)(502.50488545,66.09151201)(502.4948877,66.04151611)
\curveto(502.46488549,65.92151218)(502.44988551,65.78151232)(502.4498877,65.62151611)
\curveto(502.4598855,65.47151263)(502.46488549,65.32651278)(502.4648877,65.18651611)
\lineto(502.4648877,63.34151611)
\lineto(502.4648877,62.99651611)
\curveto(502.46488549,62.87651523)(502.4598855,62.76151534)(502.4498877,62.65151611)
\curveto(502.43988552,62.54151556)(502.43488552,62.44651566)(502.4348877,62.36651611)
\curveto(502.44488551,62.28651582)(502.42488553,62.21651589)(502.3748877,62.15651611)
\curveto(502.32488563,62.08651602)(502.24488571,62.04651606)(502.1348877,62.03651611)
\curveto(502.03488592,62.02651608)(501.92488603,62.02151608)(501.8048877,62.02151611)
\lineto(501.5348877,62.02151611)
\curveto(501.48488647,62.04151606)(501.43488652,62.05651605)(501.3848877,62.06651611)
\curveto(501.34488661,62.08651602)(501.31488664,62.11151599)(501.2948877,62.14151611)
\curveto(501.24488671,62.21151589)(501.21488674,62.29651581)(501.2048877,62.39651611)
\lineto(501.2048877,62.72651611)
\lineto(501.2048877,63.88151611)
\lineto(501.2048877,68.03651611)
\lineto(501.2048877,69.07151611)
\lineto(501.2048877,69.37151611)
\curveto(501.21488674,69.47150863)(501.24488671,69.55650855)(501.2948877,69.62651611)
\curveto(501.32488663,69.66650844)(501.37488658,69.69650841)(501.4448877,69.71651611)
\curveto(501.52488643,69.73650837)(501.60988635,69.74650836)(501.6998877,69.74651611)
\curveto(501.78988617,69.75650835)(501.87988608,69.75650835)(501.9698877,69.74651611)
\curveto(502.0598859,69.73650837)(502.12988583,69.72150838)(502.1798877,69.70151611)
\curveto(502.2598857,69.67150843)(502.30988565,69.61150849)(502.3298877,69.52151611)
\curveto(502.3598856,69.44150866)(502.37488558,69.35150875)(502.3748877,69.25151611)
\lineto(502.3748877,68.95151611)
\curveto(502.37488558,68.85150925)(502.39488556,68.76150934)(502.4348877,68.68151611)
\curveto(502.44488551,68.66150944)(502.4548855,68.64650946)(502.4648877,68.63651611)
\lineto(502.5098877,68.59151611)
\curveto(502.61988534,68.59150951)(502.70988525,68.63650947)(502.7798877,68.72651611)
\curveto(502.84988511,68.82650928)(502.90988505,68.9065092)(502.9598877,68.96651611)
\lineto(503.0498877,69.05651611)
\curveto(503.13988482,69.16650894)(503.26488469,69.28150882)(503.4248877,69.40151611)
\curveto(503.58488437,69.52150858)(503.73488422,69.61150849)(503.8748877,69.67151611)
\curveto(503.96488399,69.72150838)(504.0598839,69.75650835)(504.1598877,69.77651611)
\curveto(504.2598837,69.8065083)(504.36488359,69.83650827)(504.4748877,69.86651611)
\curveto(504.53488342,69.87650823)(504.59488336,69.88150822)(504.6548877,69.88151611)
\curveto(504.71488324,69.89150821)(504.76988319,69.9015082)(504.8198877,69.91151611)
}
}
{
\newrgbcolor{curcolor}{0 0 0}
\pscustom[linestyle=none,fillstyle=solid,fillcolor=curcolor]
{
\newpath
\moveto(513.06965332,62.56151611)
\curveto(513.09964549,62.4015157)(513.08464551,62.26651584)(513.02465332,62.15651611)
\curveto(512.96464563,62.05651605)(512.88464571,61.98151612)(512.78465332,61.93151611)
\curveto(512.73464586,61.91151619)(512.67964591,61.9015162)(512.61965332,61.90151611)
\curveto(512.56964602,61.9015162)(512.51464608,61.89151621)(512.45465332,61.87151611)
\curveto(512.23464636,61.82151628)(512.01464658,61.83651627)(511.79465332,61.91651611)
\curveto(511.58464701,61.98651612)(511.43964715,62.07651603)(511.35965332,62.18651611)
\curveto(511.30964728,62.25651585)(511.26464733,62.33651577)(511.22465332,62.42651611)
\curveto(511.18464741,62.52651558)(511.13464746,62.6065155)(511.07465332,62.66651611)
\curveto(511.05464754,62.68651542)(511.02964756,62.7065154)(510.99965332,62.72651611)
\curveto(510.97964761,62.74651536)(510.94964764,62.75151535)(510.90965332,62.74151611)
\curveto(510.79964779,62.71151539)(510.6946479,62.65651545)(510.59465332,62.57651611)
\curveto(510.50464809,62.49651561)(510.41464818,62.42651568)(510.32465332,62.36651611)
\curveto(510.1946484,62.28651582)(510.05464854,62.21151589)(509.90465332,62.14151611)
\curveto(509.75464884,62.08151602)(509.594649,62.02651608)(509.42465332,61.97651611)
\curveto(509.32464927,61.94651616)(509.21464938,61.92651618)(509.09465332,61.91651611)
\curveto(508.98464961,61.9065162)(508.87464972,61.89151621)(508.76465332,61.87151611)
\curveto(508.71464988,61.86151624)(508.66964992,61.85651625)(508.62965332,61.85651611)
\lineto(508.52465332,61.85651611)
\curveto(508.41465018,61.83651627)(508.30965028,61.83651627)(508.20965332,61.85651611)
\lineto(508.07465332,61.85651611)
\curveto(508.02465057,61.86651624)(507.97465062,61.87151623)(507.92465332,61.87151611)
\curveto(507.87465072,61.87151623)(507.82965076,61.88151622)(507.78965332,61.90151611)
\curveto(507.74965084,61.91151619)(507.71465088,61.91651619)(507.68465332,61.91651611)
\curveto(507.66465093,61.9065162)(507.63965095,61.9065162)(507.60965332,61.91651611)
\lineto(507.36965332,61.97651611)
\curveto(507.2896513,61.98651612)(507.21465138,62.0065161)(507.14465332,62.03651611)
\curveto(506.84465175,62.16651594)(506.59965199,62.31151579)(506.40965332,62.47151611)
\curveto(506.22965236,62.64151546)(506.07965251,62.87651523)(505.95965332,63.17651611)
\curveto(505.86965272,63.39651471)(505.82465277,63.66151444)(505.82465332,63.97151611)
\lineto(505.82465332,64.28651611)
\curveto(505.83465276,64.33651377)(505.83965275,64.38651372)(505.83965332,64.43651611)
\lineto(505.86965332,64.61651611)
\lineto(505.98965332,64.94651611)
\curveto(506.02965256,65.05651305)(506.07965251,65.15651295)(506.13965332,65.24651611)
\curveto(506.31965227,65.53651257)(506.56465203,65.75151235)(506.87465332,65.89151611)
\curveto(507.18465141,66.03151207)(507.52465107,66.15651195)(507.89465332,66.26651611)
\curveto(508.03465056,66.3065118)(508.17965041,66.33651177)(508.32965332,66.35651611)
\curveto(508.47965011,66.37651173)(508.62964996,66.4015117)(508.77965332,66.43151611)
\curveto(508.84964974,66.45151165)(508.91464968,66.46151164)(508.97465332,66.46151611)
\curveto(509.04464955,66.46151164)(509.11964947,66.47151163)(509.19965332,66.49151611)
\curveto(509.26964932,66.51151159)(509.33964925,66.52151158)(509.40965332,66.52151611)
\curveto(509.47964911,66.53151157)(509.55464904,66.54651156)(509.63465332,66.56651611)
\curveto(509.88464871,66.62651148)(510.11964847,66.67651143)(510.33965332,66.71651611)
\curveto(510.55964803,66.76651134)(510.73464786,66.88151122)(510.86465332,67.06151611)
\curveto(510.92464767,67.14151096)(510.97464762,67.24151086)(511.01465332,67.36151611)
\curveto(511.05464754,67.49151061)(511.05464754,67.63151047)(511.01465332,67.78151611)
\curveto(510.95464764,68.02151008)(510.86464773,68.21150989)(510.74465332,68.35151611)
\curveto(510.63464796,68.49150961)(510.47464812,68.6015095)(510.26465332,68.68151611)
\curveto(510.14464845,68.73150937)(509.99964859,68.76650934)(509.82965332,68.78651611)
\curveto(509.66964892,68.8065093)(509.49964909,68.81650929)(509.31965332,68.81651611)
\curveto(509.13964945,68.81650929)(508.96464963,68.8065093)(508.79465332,68.78651611)
\curveto(508.62464997,68.76650934)(508.47965011,68.73650937)(508.35965332,68.69651611)
\curveto(508.1896504,68.63650947)(508.02465057,68.55150955)(507.86465332,68.44151611)
\curveto(507.78465081,68.38150972)(507.70965088,68.3015098)(507.63965332,68.20151611)
\curveto(507.57965101,68.11150999)(507.52465107,68.01151009)(507.47465332,67.90151611)
\curveto(507.44465115,67.82151028)(507.41465118,67.73651037)(507.38465332,67.64651611)
\curveto(507.36465123,67.55651055)(507.31965127,67.48651062)(507.24965332,67.43651611)
\curveto(507.20965138,67.4065107)(507.13965145,67.38151072)(507.03965332,67.36151611)
\curveto(506.94965164,67.35151075)(506.85465174,67.34651076)(506.75465332,67.34651611)
\curveto(506.65465194,67.34651076)(506.55465204,67.35151075)(506.45465332,67.36151611)
\curveto(506.36465223,67.38151072)(506.29965229,67.4065107)(506.25965332,67.43651611)
\curveto(506.21965237,67.46651064)(506.1896524,67.51651059)(506.16965332,67.58651611)
\curveto(506.14965244,67.65651045)(506.14965244,67.73151037)(506.16965332,67.81151611)
\curveto(506.19965239,67.94151016)(506.22965236,68.06151004)(506.25965332,68.17151611)
\curveto(506.29965229,68.29150981)(506.34465225,68.4065097)(506.39465332,68.51651611)
\curveto(506.58465201,68.86650924)(506.82465177,69.13650897)(507.11465332,69.32651611)
\curveto(507.40465119,69.52650858)(507.76465083,69.68650842)(508.19465332,69.80651611)
\curveto(508.2946503,69.82650828)(508.3946502,69.84150826)(508.49465332,69.85151611)
\curveto(508.60464999,69.86150824)(508.71464988,69.87650823)(508.82465332,69.89651611)
\curveto(508.86464973,69.9065082)(508.92964966,69.9065082)(509.01965332,69.89651611)
\curveto(509.10964948,69.89650821)(509.16464943,69.9065082)(509.18465332,69.92651611)
\curveto(509.88464871,69.93650817)(510.4946481,69.85650825)(511.01465332,69.68651611)
\curveto(511.53464706,69.51650859)(511.89964669,69.19150891)(512.10965332,68.71151611)
\curveto(512.19964639,68.51150959)(512.24964634,68.27650983)(512.25965332,68.00651611)
\curveto(512.27964631,67.74651036)(512.2896463,67.47151063)(512.28965332,67.18151611)
\lineto(512.28965332,63.86651611)
\curveto(512.2896463,63.72651438)(512.2946463,63.59151451)(512.30465332,63.46151611)
\curveto(512.31464628,63.33151477)(512.34464625,63.22651488)(512.39465332,63.14651611)
\curveto(512.44464615,63.07651503)(512.50964608,63.02651508)(512.58965332,62.99651611)
\curveto(512.67964591,62.95651515)(512.76464583,62.92651518)(512.84465332,62.90651611)
\curveto(512.92464567,62.89651521)(512.98464561,62.85151525)(513.02465332,62.77151611)
\curveto(513.04464555,62.74151536)(513.05464554,62.71151539)(513.05465332,62.68151611)
\curveto(513.05464554,62.65151545)(513.05964553,62.61151549)(513.06965332,62.56151611)
\moveto(510.92465332,64.22651611)
\curveto(510.98464761,64.36651374)(511.01464758,64.52651358)(511.01465332,64.70651611)
\curveto(511.02464757,64.89651321)(511.02964756,65.09151301)(511.02965332,65.29151611)
\curveto(511.02964756,65.4015127)(511.02464757,65.5015126)(511.01465332,65.59151611)
\curveto(511.00464759,65.68151242)(510.96464763,65.75151235)(510.89465332,65.80151611)
\curveto(510.86464773,65.82151228)(510.7946478,65.83151227)(510.68465332,65.83151611)
\curveto(510.66464793,65.81151229)(510.62964796,65.8015123)(510.57965332,65.80151611)
\curveto(510.52964806,65.8015123)(510.48464811,65.79151231)(510.44465332,65.77151611)
\curveto(510.36464823,65.75151235)(510.27464832,65.73151237)(510.17465332,65.71151611)
\lineto(509.87465332,65.65151611)
\curveto(509.84464875,65.65151245)(509.80964878,65.64651246)(509.76965332,65.63651611)
\lineto(509.66465332,65.63651611)
\curveto(509.51464908,65.59651251)(509.34964924,65.57151253)(509.16965332,65.56151611)
\curveto(508.99964959,65.56151254)(508.83964975,65.54151256)(508.68965332,65.50151611)
\curveto(508.60964998,65.48151262)(508.53465006,65.46151264)(508.46465332,65.44151611)
\curveto(508.40465019,65.43151267)(508.33465026,65.41651269)(508.25465332,65.39651611)
\curveto(508.0946505,65.34651276)(507.94465065,65.28151282)(507.80465332,65.20151611)
\curveto(507.66465093,65.13151297)(507.54465105,65.04151306)(507.44465332,64.93151611)
\curveto(507.34465125,64.82151328)(507.26965132,64.68651342)(507.21965332,64.52651611)
\curveto(507.16965142,64.37651373)(507.14965144,64.19151391)(507.15965332,63.97151611)
\curveto(507.15965143,63.87151423)(507.17465142,63.77651433)(507.20465332,63.68651611)
\curveto(507.24465135,63.6065145)(507.2896513,63.53151457)(507.33965332,63.46151611)
\curveto(507.41965117,63.35151475)(507.52465107,63.25651485)(507.65465332,63.17651611)
\curveto(507.78465081,63.106515)(507.92465067,63.04651506)(508.07465332,62.99651611)
\curveto(508.12465047,62.98651512)(508.17465042,62.98151512)(508.22465332,62.98151611)
\curveto(508.27465032,62.98151512)(508.32465027,62.97651513)(508.37465332,62.96651611)
\curveto(508.44465015,62.94651516)(508.52965006,62.93151517)(508.62965332,62.92151611)
\curveto(508.73964985,62.92151518)(508.82964976,62.93151517)(508.89965332,62.95151611)
\curveto(508.95964963,62.97151513)(509.01964957,62.97651513)(509.07965332,62.96651611)
\curveto(509.13964945,62.96651514)(509.19964939,62.97651513)(509.25965332,62.99651611)
\curveto(509.33964925,63.01651509)(509.41464918,63.03151507)(509.48465332,63.04151611)
\curveto(509.56464903,63.05151505)(509.63964895,63.07151503)(509.70965332,63.10151611)
\curveto(509.99964859,63.22151488)(510.24464835,63.36651474)(510.44465332,63.53651611)
\curveto(510.65464794,63.7065144)(510.81464778,63.93651417)(510.92465332,64.22651611)
}
}
{
\newrgbcolor{curcolor}{0 0 0}
\pscustom[linestyle=none,fillstyle=solid,fillcolor=curcolor]
{
\newpath
\moveto(521.20129395,62.81651611)
\lineto(521.20129395,62.42651611)
\curveto(521.20128607,62.3065158)(521.1762861,62.2065159)(521.12629395,62.12651611)
\curveto(521.0762862,62.05651605)(520.99128628,62.01651609)(520.87129395,62.00651611)
\lineto(520.52629395,62.00651611)
\curveto(520.46628681,62.0065161)(520.40628687,62.0015161)(520.34629395,61.99151611)
\curveto(520.29628698,61.99151611)(520.25128702,62.0015161)(520.21129395,62.02151611)
\curveto(520.12128715,62.04151606)(520.06128721,62.08151602)(520.03129395,62.14151611)
\curveto(519.99128728,62.19151591)(519.96628731,62.25151585)(519.95629395,62.32151611)
\curveto(519.95628732,62.39151571)(519.94128733,62.46151564)(519.91129395,62.53151611)
\curveto(519.90128737,62.55151555)(519.88628739,62.56651554)(519.86629395,62.57651611)
\curveto(519.85628742,62.59651551)(519.84128743,62.61651549)(519.82129395,62.63651611)
\curveto(519.72128755,62.64651546)(519.64128763,62.62651548)(519.58129395,62.57651611)
\curveto(519.53128774,62.52651558)(519.4762878,62.47651563)(519.41629395,62.42651611)
\curveto(519.21628806,62.27651583)(519.01628826,62.16151594)(518.81629395,62.08151611)
\curveto(518.63628864,62.0015161)(518.42628885,61.94151616)(518.18629395,61.90151611)
\curveto(517.95628932,61.86151624)(517.71628956,61.84151626)(517.46629395,61.84151611)
\curveto(517.22629005,61.83151627)(516.98629029,61.84651626)(516.74629395,61.88651611)
\curveto(516.50629077,61.91651619)(516.29629098,61.97151613)(516.11629395,62.05151611)
\curveto(515.59629168,62.27151583)(515.1762921,62.56651554)(514.85629395,62.93651611)
\curveto(514.53629274,63.31651479)(514.28629299,63.78651432)(514.10629395,64.34651611)
\curveto(514.06629321,64.43651367)(514.03629324,64.52651358)(514.01629395,64.61651611)
\curveto(514.00629327,64.71651339)(513.98629329,64.81651329)(513.95629395,64.91651611)
\curveto(513.94629333,64.96651314)(513.94129333,65.01651309)(513.94129395,65.06651611)
\curveto(513.94129333,65.11651299)(513.93629334,65.16651294)(513.92629395,65.21651611)
\curveto(513.90629337,65.26651284)(513.89629338,65.31651279)(513.89629395,65.36651611)
\curveto(513.90629337,65.42651268)(513.90629337,65.48151262)(513.89629395,65.53151611)
\lineto(513.89629395,65.68151611)
\curveto(513.8762934,65.73151237)(513.86629341,65.79651231)(513.86629395,65.87651611)
\curveto(513.86629341,65.95651215)(513.8762934,66.02151208)(513.89629395,66.07151611)
\lineto(513.89629395,66.23651611)
\curveto(513.91629336,66.3065118)(513.92129335,66.37651173)(513.91129395,66.44651611)
\curveto(513.91129336,66.52651158)(513.92129335,66.6015115)(513.94129395,66.67151611)
\curveto(513.95129332,66.72151138)(513.95629332,66.76651134)(513.95629395,66.80651611)
\curveto(513.95629332,66.84651126)(513.96129331,66.89151121)(513.97129395,66.94151611)
\curveto(514.00129327,67.04151106)(514.02629325,67.13651097)(514.04629395,67.22651611)
\curveto(514.06629321,67.32651078)(514.09129318,67.42151068)(514.12129395,67.51151611)
\curveto(514.25129302,67.89151021)(514.41629286,68.23150987)(514.61629395,68.53151611)
\curveto(514.82629245,68.84150926)(515.0762922,69.09650901)(515.36629395,69.29651611)
\curveto(515.53629174,69.41650869)(515.71129156,69.51650859)(515.89129395,69.59651611)
\curveto(516.08129119,69.67650843)(516.28629099,69.74650836)(516.50629395,69.80651611)
\curveto(516.5762907,69.81650829)(516.64129063,69.82650828)(516.70129395,69.83651611)
\curveto(516.7712905,69.84650826)(516.84129043,69.86150824)(516.91129395,69.88151611)
\lineto(517.06129395,69.88151611)
\curveto(517.14129013,69.9015082)(517.25629002,69.91150819)(517.40629395,69.91151611)
\curveto(517.56628971,69.91150819)(517.68628959,69.9015082)(517.76629395,69.88151611)
\curveto(517.80628947,69.87150823)(517.86128941,69.86650824)(517.93129395,69.86651611)
\curveto(518.04128923,69.83650827)(518.15128912,69.81150829)(518.26129395,69.79151611)
\curveto(518.3712889,69.78150832)(518.4762888,69.75150835)(518.57629395,69.70151611)
\curveto(518.72628855,69.64150846)(518.86628841,69.57650853)(518.99629395,69.50651611)
\curveto(519.13628814,69.43650867)(519.26628801,69.35650875)(519.38629395,69.26651611)
\curveto(519.44628783,69.21650889)(519.50628777,69.16150894)(519.56629395,69.10151611)
\curveto(519.63628764,69.05150905)(519.72628755,69.03650907)(519.83629395,69.05651611)
\curveto(519.85628742,69.08650902)(519.8712874,69.11150899)(519.88129395,69.13151611)
\curveto(519.90128737,69.15150895)(519.91628736,69.18150892)(519.92629395,69.22151611)
\curveto(519.95628732,69.31150879)(519.96628731,69.42650868)(519.95629395,69.56651611)
\lineto(519.95629395,69.94151611)
\lineto(519.95629395,71.66651611)
\lineto(519.95629395,72.13151611)
\curveto(519.95628732,72.31150579)(519.98128729,72.44150566)(520.03129395,72.52151611)
\curveto(520.0712872,72.59150551)(520.13128714,72.63650547)(520.21129395,72.65651611)
\curveto(520.23128704,72.65650545)(520.25628702,72.65650545)(520.28629395,72.65651611)
\curveto(520.31628696,72.66650544)(520.34128693,72.67150543)(520.36129395,72.67151611)
\curveto(520.50128677,72.68150542)(520.64628663,72.68150542)(520.79629395,72.67151611)
\curveto(520.95628632,72.67150543)(521.06628621,72.63150547)(521.12629395,72.55151611)
\curveto(521.1762861,72.47150563)(521.20128607,72.37150573)(521.20129395,72.25151611)
\lineto(521.20129395,71.87651611)
\lineto(521.20129395,62.81651611)
\moveto(519.98629395,65.65151611)
\curveto(520.00628727,65.7015124)(520.01628726,65.76651234)(520.01629395,65.84651611)
\curveto(520.01628726,65.93651217)(520.00628727,66.0065121)(519.98629395,66.05651611)
\lineto(519.98629395,66.28151611)
\curveto(519.96628731,66.37151173)(519.95128732,66.46151164)(519.94129395,66.55151611)
\curveto(519.93128734,66.65151145)(519.91128736,66.74151136)(519.88129395,66.82151611)
\curveto(519.86128741,66.9015112)(519.84128743,66.97651113)(519.82129395,67.04651611)
\curveto(519.81128746,67.11651099)(519.79128748,67.18651092)(519.76129395,67.25651611)
\curveto(519.64128763,67.55651055)(519.48628779,67.82151028)(519.29629395,68.05151611)
\curveto(519.10628817,68.28150982)(518.86628841,68.46150964)(518.57629395,68.59151611)
\curveto(518.4762888,68.64150946)(518.3712889,68.67650943)(518.26129395,68.69651611)
\curveto(518.16128911,68.72650938)(518.05128922,68.75150935)(517.93129395,68.77151611)
\curveto(517.85128942,68.79150931)(517.76128951,68.8015093)(517.66129395,68.80151611)
\lineto(517.39129395,68.80151611)
\curveto(517.34128993,68.79150931)(517.29628998,68.78150932)(517.25629395,68.77151611)
\lineto(517.12129395,68.77151611)
\curveto(517.04129023,68.75150935)(516.95629032,68.73150937)(516.86629395,68.71151611)
\curveto(516.78629049,68.69150941)(516.70629057,68.66650944)(516.62629395,68.63651611)
\curveto(516.30629097,68.49650961)(516.04629123,68.29150981)(515.84629395,68.02151611)
\curveto(515.65629162,67.76151034)(515.50129177,67.45651065)(515.38129395,67.10651611)
\curveto(515.34129193,66.99651111)(515.31129196,66.88151122)(515.29129395,66.76151611)
\curveto(515.28129199,66.65151145)(515.26629201,66.54151156)(515.24629395,66.43151611)
\curveto(515.24629203,66.39151171)(515.24129203,66.35151175)(515.23129395,66.31151611)
\lineto(515.23129395,66.20651611)
\curveto(515.21129206,66.15651195)(515.20129207,66.101512)(515.20129395,66.04151611)
\curveto(515.21129206,65.98151212)(515.21629206,65.92651218)(515.21629395,65.87651611)
\lineto(515.21629395,65.54651611)
\curveto(515.21629206,65.44651266)(515.22629205,65.35151275)(515.24629395,65.26151611)
\curveto(515.25629202,65.23151287)(515.26129201,65.18151292)(515.26129395,65.11151611)
\curveto(515.28129199,65.04151306)(515.29629198,64.97151313)(515.30629395,64.90151611)
\lineto(515.36629395,64.69151611)
\curveto(515.4762918,64.34151376)(515.62629165,64.04151406)(515.81629395,63.79151611)
\curveto(516.00629127,63.54151456)(516.24629103,63.33651477)(516.53629395,63.17651611)
\curveto(516.62629065,63.12651498)(516.71629056,63.08651502)(516.80629395,63.05651611)
\curveto(516.89629038,63.02651508)(516.99629028,62.99651511)(517.10629395,62.96651611)
\curveto(517.15629012,62.94651516)(517.20629007,62.94151516)(517.25629395,62.95151611)
\curveto(517.31628996,62.96151514)(517.3712899,62.95651515)(517.42129395,62.93651611)
\curveto(517.46128981,62.92651518)(517.50128977,62.92151518)(517.54129395,62.92151611)
\lineto(517.67629395,62.92151611)
\lineto(517.81129395,62.92151611)
\curveto(517.84128943,62.93151517)(517.89128938,62.93651517)(517.96129395,62.93651611)
\curveto(518.04128923,62.95651515)(518.12128915,62.97151513)(518.20129395,62.98151611)
\curveto(518.28128899,63.0015151)(518.35628892,63.02651508)(518.42629395,63.05651611)
\curveto(518.75628852,63.19651491)(519.02128825,63.37151473)(519.22129395,63.58151611)
\curveto(519.43128784,63.8015143)(519.60628767,64.07651403)(519.74629395,64.40651611)
\curveto(519.79628748,64.51651359)(519.83128744,64.62651348)(519.85129395,64.73651611)
\curveto(519.8712874,64.84651326)(519.89628738,64.95651315)(519.92629395,65.06651611)
\curveto(519.94628733,65.106513)(519.95628732,65.14151296)(519.95629395,65.17151611)
\curveto(519.95628732,65.21151289)(519.96128731,65.25151285)(519.97129395,65.29151611)
\curveto(519.98128729,65.35151275)(519.98128729,65.41151269)(519.97129395,65.47151611)
\curveto(519.9712873,65.53151257)(519.9762873,65.59151251)(519.98629395,65.65151611)
}
}
{
\newrgbcolor{curcolor}{0 0 0}
\pscustom[linestyle=none,fillstyle=solid,fillcolor=curcolor]
{
\newpath
\moveto(530.27254395,66.20651611)
\curveto(530.29253589,66.14651196)(530.30253588,66.05151205)(530.30254395,65.92151611)
\curveto(530.30253588,65.8015123)(530.29753588,65.71651239)(530.28754395,65.66651611)
\lineto(530.28754395,65.51651611)
\curveto(530.2775359,65.43651267)(530.26753591,65.36151274)(530.25754395,65.29151611)
\curveto(530.25753592,65.23151287)(530.25253593,65.16151294)(530.24254395,65.08151611)
\curveto(530.22253596,65.02151308)(530.20753597,64.96151314)(530.19754395,64.90151611)
\curveto(530.19753598,64.84151326)(530.18753599,64.78151332)(530.16754395,64.72151611)
\curveto(530.12753605,64.59151351)(530.09253609,64.46151364)(530.06254395,64.33151611)
\curveto(530.03253615,64.2015139)(529.99253619,64.08151402)(529.94254395,63.97151611)
\curveto(529.73253645,63.49151461)(529.45253673,63.08651502)(529.10254395,62.75651611)
\curveto(528.75253743,62.43651567)(528.32253786,62.19151591)(527.81254395,62.02151611)
\curveto(527.70253848,61.98151612)(527.5825386,61.95151615)(527.45254395,61.93151611)
\curveto(527.33253885,61.91151619)(527.20753897,61.89151621)(527.07754395,61.87151611)
\curveto(527.01753916,61.86151624)(526.95253923,61.85651625)(526.88254395,61.85651611)
\curveto(526.82253936,61.84651626)(526.76253942,61.84151626)(526.70254395,61.84151611)
\curveto(526.66253952,61.83151627)(526.60253958,61.82651628)(526.52254395,61.82651611)
\curveto(526.45253973,61.82651628)(526.40253978,61.83151627)(526.37254395,61.84151611)
\curveto(526.33253985,61.85151625)(526.29253989,61.85651625)(526.25254395,61.85651611)
\curveto(526.21253997,61.84651626)(526.17754,61.84651626)(526.14754395,61.85651611)
\lineto(526.05754395,61.85651611)
\lineto(525.69754395,61.90151611)
\curveto(525.55754062,61.94151616)(525.42254076,61.98151612)(525.29254395,62.02151611)
\curveto(525.16254102,62.06151604)(525.03754114,62.106516)(524.91754395,62.15651611)
\curveto(524.46754171,62.35651575)(524.09754208,62.61651549)(523.80754395,62.93651611)
\curveto(523.51754266,63.25651485)(523.2775429,63.64651446)(523.08754395,64.10651611)
\curveto(523.03754314,64.2065139)(522.99754318,64.3065138)(522.96754395,64.40651611)
\curveto(522.94754323,64.5065136)(522.92754325,64.61151349)(522.90754395,64.72151611)
\curveto(522.88754329,64.76151334)(522.8775433,64.79151331)(522.87754395,64.81151611)
\curveto(522.88754329,64.84151326)(522.88754329,64.87651323)(522.87754395,64.91651611)
\curveto(522.85754332,64.99651311)(522.84254334,65.07651303)(522.83254395,65.15651611)
\curveto(522.83254335,65.24651286)(522.82254336,65.33151277)(522.80254395,65.41151611)
\lineto(522.80254395,65.53151611)
\curveto(522.80254338,65.57151253)(522.79754338,65.61651249)(522.78754395,65.66651611)
\curveto(522.7775434,65.71651239)(522.77254341,65.8015123)(522.77254395,65.92151611)
\curveto(522.77254341,66.05151205)(522.7825434,66.14651196)(522.80254395,66.20651611)
\curveto(522.82254336,66.27651183)(522.82754335,66.34651176)(522.81754395,66.41651611)
\curveto(522.80754337,66.48651162)(522.81254337,66.55651155)(522.83254395,66.62651611)
\curveto(522.84254334,66.67651143)(522.84754333,66.71651139)(522.84754395,66.74651611)
\curveto(522.85754332,66.78651132)(522.86754331,66.83151127)(522.87754395,66.88151611)
\curveto(522.90754327,67.0015111)(522.93254325,67.12151098)(522.95254395,67.24151611)
\curveto(522.9825432,67.36151074)(523.02254316,67.47651063)(523.07254395,67.58651611)
\curveto(523.22254296,67.95651015)(523.40254278,68.28650982)(523.61254395,68.57651611)
\curveto(523.83254235,68.87650923)(524.09754208,69.12650898)(524.40754395,69.32651611)
\curveto(524.52754165,69.4065087)(524.65254153,69.47150863)(524.78254395,69.52151611)
\curveto(524.91254127,69.58150852)(525.04754113,69.64150846)(525.18754395,69.70151611)
\curveto(525.30754087,69.75150835)(525.43754074,69.78150832)(525.57754395,69.79151611)
\curveto(525.71754046,69.81150829)(525.85754032,69.84150826)(525.99754395,69.88151611)
\lineto(526.19254395,69.88151611)
\curveto(526.26253992,69.89150821)(526.32753985,69.9015082)(526.38754395,69.91151611)
\curveto(527.2775389,69.92150818)(528.01753816,69.73650837)(528.60754395,69.35651611)
\curveto(529.19753698,68.97650913)(529.62253656,68.48150962)(529.88254395,67.87151611)
\curveto(529.93253625,67.77151033)(529.97253621,67.67151043)(530.00254395,67.57151611)
\curveto(530.03253615,67.47151063)(530.06753611,67.36651074)(530.10754395,67.25651611)
\curveto(530.13753604,67.14651096)(530.16253602,67.02651108)(530.18254395,66.89651611)
\curveto(530.20253598,66.77651133)(530.22753595,66.65151145)(530.25754395,66.52151611)
\curveto(530.26753591,66.47151163)(530.26753591,66.41651169)(530.25754395,66.35651611)
\curveto(530.25753592,66.3065118)(530.26253592,66.25651185)(530.27254395,66.20651611)
\moveto(528.93754395,65.35151611)
\curveto(528.95753722,65.42151268)(528.96253722,65.5015126)(528.95254395,65.59151611)
\lineto(528.95254395,65.84651611)
\curveto(528.95253723,66.23651187)(528.91753726,66.56651154)(528.84754395,66.83651611)
\curveto(528.81753736,66.91651119)(528.79253739,66.99651111)(528.77254395,67.07651611)
\curveto(528.75253743,67.15651095)(528.72753745,67.23151087)(528.69754395,67.30151611)
\curveto(528.41753776,67.95151015)(527.97253821,68.4015097)(527.36254395,68.65151611)
\curveto(527.29253889,68.68150942)(527.21753896,68.7015094)(527.13754395,68.71151611)
\lineto(526.89754395,68.77151611)
\curveto(526.81753936,68.79150931)(526.73253945,68.8015093)(526.64254395,68.80151611)
\lineto(526.37254395,68.80151611)
\lineto(526.10254395,68.75651611)
\curveto(526.00254018,68.73650937)(525.90754027,68.71150939)(525.81754395,68.68151611)
\curveto(525.73754044,68.66150944)(525.65754052,68.63150947)(525.57754395,68.59151611)
\curveto(525.50754067,68.57150953)(525.44254074,68.54150956)(525.38254395,68.50151611)
\curveto(525.32254086,68.46150964)(525.26754091,68.42150968)(525.21754395,68.38151611)
\curveto(524.9775412,68.21150989)(524.7825414,68.0065101)(524.63254395,67.76651611)
\curveto(524.4825417,67.52651058)(524.35254183,67.24651086)(524.24254395,66.92651611)
\curveto(524.21254197,66.82651128)(524.19254199,66.72151138)(524.18254395,66.61151611)
\curveto(524.17254201,66.51151159)(524.15754202,66.4065117)(524.13754395,66.29651611)
\curveto(524.12754205,66.25651185)(524.12254206,66.19151191)(524.12254395,66.10151611)
\curveto(524.11254207,66.07151203)(524.10754207,66.03651207)(524.10754395,65.99651611)
\curveto(524.11754206,65.95651215)(524.12254206,65.91151219)(524.12254395,65.86151611)
\lineto(524.12254395,65.56151611)
\curveto(524.12254206,65.46151264)(524.13254205,65.37151273)(524.15254395,65.29151611)
\lineto(524.18254395,65.11151611)
\curveto(524.20254198,65.01151309)(524.21754196,64.91151319)(524.22754395,64.81151611)
\curveto(524.24754193,64.72151338)(524.2775419,64.63651347)(524.31754395,64.55651611)
\curveto(524.41754176,64.31651379)(524.53254165,64.09151401)(524.66254395,63.88151611)
\curveto(524.80254138,63.67151443)(524.97254121,63.49651461)(525.17254395,63.35651611)
\curveto(525.22254096,63.32651478)(525.26754091,63.3015148)(525.30754395,63.28151611)
\curveto(525.34754083,63.26151484)(525.39254079,63.23651487)(525.44254395,63.20651611)
\curveto(525.52254066,63.15651495)(525.60754057,63.11151499)(525.69754395,63.07151611)
\curveto(525.79754038,63.04151506)(525.90254028,63.01151509)(526.01254395,62.98151611)
\curveto(526.06254012,62.96151514)(526.10754007,62.95151515)(526.14754395,62.95151611)
\curveto(526.19753998,62.96151514)(526.24753993,62.96151514)(526.29754395,62.95151611)
\curveto(526.32753985,62.94151516)(526.38753979,62.93151517)(526.47754395,62.92151611)
\curveto(526.5775396,62.91151519)(526.65253953,62.91651519)(526.70254395,62.93651611)
\curveto(526.74253944,62.94651516)(526.7825394,62.94651516)(526.82254395,62.93651611)
\curveto(526.86253932,62.93651517)(526.90253928,62.94651516)(526.94254395,62.96651611)
\curveto(527.02253916,62.98651512)(527.10253908,63.0015151)(527.18254395,63.01151611)
\curveto(527.26253892,63.03151507)(527.33753884,63.05651505)(527.40754395,63.08651611)
\curveto(527.74753843,63.22651488)(528.02253816,63.42151468)(528.23254395,63.67151611)
\curveto(528.44253774,63.92151418)(528.61753756,64.21651389)(528.75754395,64.55651611)
\curveto(528.80753737,64.67651343)(528.83753734,64.8015133)(528.84754395,64.93151611)
\curveto(528.86753731,65.07151303)(528.89753728,65.21151289)(528.93754395,65.35151611)
}
}
{
\newrgbcolor{curcolor}{0 0 0}
\pscustom[linestyle=none,fillstyle=solid,fillcolor=curcolor]
{
\newpath
\moveto(535.4058252,69.91151611)
\curveto(535.63582041,69.91150819)(535.76582028,69.85150825)(535.7958252,69.73151611)
\curveto(535.82582022,69.62150848)(535.8408202,69.45650865)(535.8408252,69.23651611)
\lineto(535.8408252,68.95151611)
\curveto(535.8408202,68.86150924)(535.81582023,68.78650932)(535.7658252,68.72651611)
\curveto(535.70582034,68.64650946)(535.62082042,68.6015095)(535.5108252,68.59151611)
\curveto(535.40082064,68.59150951)(535.29082075,68.57650953)(535.1808252,68.54651611)
\curveto(535.040821,68.51650959)(534.90582114,68.48650962)(534.7758252,68.45651611)
\curveto(534.65582139,68.42650968)(534.5408215,68.38650972)(534.4308252,68.33651611)
\curveto(534.1408219,68.2065099)(533.90582214,68.02651008)(533.7258252,67.79651611)
\curveto(533.5458225,67.57651053)(533.39082265,67.32151078)(533.2608252,67.03151611)
\curveto(533.22082282,66.92151118)(533.19082285,66.8065113)(533.1708252,66.68651611)
\curveto(533.15082289,66.57651153)(533.12582292,66.46151164)(533.0958252,66.34151611)
\curveto(533.08582296,66.29151181)(533.08082296,66.24151186)(533.0808252,66.19151611)
\curveto(533.09082295,66.14151196)(533.09082295,66.09151201)(533.0808252,66.04151611)
\curveto(533.05082299,65.92151218)(533.03582301,65.78151232)(533.0358252,65.62151611)
\curveto(533.045823,65.47151263)(533.05082299,65.32651278)(533.0508252,65.18651611)
\lineto(533.0508252,63.34151611)
\lineto(533.0508252,62.99651611)
\curveto(533.05082299,62.87651523)(533.045823,62.76151534)(533.0358252,62.65151611)
\curveto(533.02582302,62.54151556)(533.02082302,62.44651566)(533.0208252,62.36651611)
\curveto(533.03082301,62.28651582)(533.01082303,62.21651589)(532.9608252,62.15651611)
\curveto(532.91082313,62.08651602)(532.83082321,62.04651606)(532.7208252,62.03651611)
\curveto(532.62082342,62.02651608)(532.51082353,62.02151608)(532.3908252,62.02151611)
\lineto(532.1208252,62.02151611)
\curveto(532.07082397,62.04151606)(532.02082402,62.05651605)(531.9708252,62.06651611)
\curveto(531.93082411,62.08651602)(531.90082414,62.11151599)(531.8808252,62.14151611)
\curveto(531.83082421,62.21151589)(531.80082424,62.29651581)(531.7908252,62.39651611)
\lineto(531.7908252,62.72651611)
\lineto(531.7908252,63.88151611)
\lineto(531.7908252,68.03651611)
\lineto(531.7908252,69.07151611)
\lineto(531.7908252,69.37151611)
\curveto(531.80082424,69.47150863)(531.83082421,69.55650855)(531.8808252,69.62651611)
\curveto(531.91082413,69.66650844)(531.96082408,69.69650841)(532.0308252,69.71651611)
\curveto(532.11082393,69.73650837)(532.19582385,69.74650836)(532.2858252,69.74651611)
\curveto(532.37582367,69.75650835)(532.46582358,69.75650835)(532.5558252,69.74651611)
\curveto(532.6458234,69.73650837)(532.71582333,69.72150838)(532.7658252,69.70151611)
\curveto(532.8458232,69.67150843)(532.89582315,69.61150849)(532.9158252,69.52151611)
\curveto(532.9458231,69.44150866)(532.96082308,69.35150875)(532.9608252,69.25151611)
\lineto(532.9608252,68.95151611)
\curveto(532.96082308,68.85150925)(532.98082306,68.76150934)(533.0208252,68.68151611)
\curveto(533.03082301,68.66150944)(533.040823,68.64650946)(533.0508252,68.63651611)
\lineto(533.0958252,68.59151611)
\curveto(533.20582284,68.59150951)(533.29582275,68.63650947)(533.3658252,68.72651611)
\curveto(533.43582261,68.82650928)(533.49582255,68.9065092)(533.5458252,68.96651611)
\lineto(533.6358252,69.05651611)
\curveto(533.72582232,69.16650894)(533.85082219,69.28150882)(534.0108252,69.40151611)
\curveto(534.17082187,69.52150858)(534.32082172,69.61150849)(534.4608252,69.67151611)
\curveto(534.55082149,69.72150838)(534.6458214,69.75650835)(534.7458252,69.77651611)
\curveto(534.8458212,69.8065083)(534.95082109,69.83650827)(535.0608252,69.86651611)
\curveto(535.12082092,69.87650823)(535.18082086,69.88150822)(535.2408252,69.88151611)
\curveto(535.30082074,69.89150821)(535.35582069,69.9015082)(535.4058252,69.91151611)
}
}
{
\newrgbcolor{curcolor}{0 0 0}
\pscustom[linestyle=none,fillstyle=solid,fillcolor=curcolor]
{
\newpath
\moveto(645.2480835,62.78651611)
\curveto(645.26807395,62.73651537)(645.29307393,62.67651543)(645.3230835,62.60651611)
\curveto(645.35307387,62.53651557)(645.37307385,62.46151564)(645.3830835,62.38151611)
\curveto(645.40307382,62.31151579)(645.40307382,62.24151586)(645.3830835,62.17151611)
\curveto(645.37307385,62.11151599)(645.33307389,62.06651604)(645.2630835,62.03651611)
\curveto(645.21307401,62.01651609)(645.15307407,62.0065161)(645.0830835,62.00651611)
\lineto(644.8730835,62.00651611)
\lineto(644.4230835,62.00651611)
\curveto(644.27307495,62.0065161)(644.15307507,62.03151607)(644.0630835,62.08151611)
\curveto(643.96307526,62.14151596)(643.88807533,62.24651586)(643.8380835,62.39651611)
\curveto(643.79807542,62.54651556)(643.75307547,62.68151542)(643.7030835,62.80151611)
\curveto(643.59307563,63.06151504)(643.49307573,63.33151477)(643.4030835,63.61151611)
\curveto(643.31307591,63.89151421)(643.21307601,64.16651394)(643.1030835,64.43651611)
\curveto(643.07307615,64.52651358)(643.04307618,64.61151349)(643.0130835,64.69151611)
\curveto(642.99307623,64.77151333)(642.96307626,64.84651326)(642.9230835,64.91651611)
\curveto(642.89307633,64.98651312)(642.84807637,65.04651306)(642.7880835,65.09651611)
\curveto(642.72807649,65.14651296)(642.64807657,65.18651292)(642.5480835,65.21651611)
\curveto(642.49807672,65.23651287)(642.43807678,65.24151286)(642.3680835,65.23151611)
\lineto(642.1730835,65.23151611)
\lineto(639.3380835,65.23151611)
\lineto(639.0380835,65.23151611)
\curveto(638.92808029,65.24151286)(638.8230804,65.24151286)(638.7230835,65.23151611)
\curveto(638.6230806,65.22151288)(638.52808069,65.2065129)(638.4380835,65.18651611)
\curveto(638.35808086,65.16651294)(638.29808092,65.12651298)(638.2580835,65.06651611)
\curveto(638.17808104,64.96651314)(638.1180811,64.85151325)(638.0780835,64.72151611)
\curveto(638.04808117,64.6015135)(638.00808121,64.47651363)(637.9580835,64.34651611)
\curveto(637.85808136,64.11651399)(637.76308146,63.87651423)(637.6730835,63.62651611)
\curveto(637.59308163,63.37651473)(637.50308172,63.13651497)(637.4030835,62.90651611)
\curveto(637.38308184,62.84651526)(637.35808186,62.77651533)(637.3280835,62.69651611)
\curveto(637.30808191,62.62651548)(637.28308194,62.55151555)(637.2530835,62.47151611)
\curveto(637.223082,62.39151571)(637.18808203,62.31651579)(637.1480835,62.24651611)
\curveto(637.1180821,62.18651592)(637.08308214,62.14151596)(637.0430835,62.11151611)
\curveto(636.96308226,62.05151605)(636.85308237,62.01651609)(636.7130835,62.00651611)
\lineto(636.2930835,62.00651611)
\lineto(636.0530835,62.00651611)
\curveto(635.98308324,62.01651609)(635.9230833,62.04151606)(635.8730835,62.08151611)
\curveto(635.8230834,62.11151599)(635.79308343,62.15651595)(635.7830835,62.21651611)
\curveto(635.78308344,62.27651583)(635.78808343,62.33651577)(635.7980835,62.39651611)
\curveto(635.8180834,62.46651564)(635.83808338,62.53151557)(635.8580835,62.59151611)
\curveto(635.88808333,62.66151544)(635.91308331,62.71151539)(635.9330835,62.74151611)
\curveto(636.07308315,63.06151504)(636.19808302,63.37651473)(636.3080835,63.68651611)
\curveto(636.4180828,64.0065141)(636.53808268,64.32651378)(636.6680835,64.64651611)
\curveto(636.75808246,64.86651324)(636.84308238,65.08151302)(636.9230835,65.29151611)
\curveto(637.00308222,65.51151259)(637.08808213,65.73151237)(637.1780835,65.95151611)
\curveto(637.47808174,66.67151143)(637.76308146,67.39651071)(638.0330835,68.12651611)
\curveto(638.30308092,68.86650924)(638.58808063,69.6015085)(638.8880835,70.33151611)
\curveto(638.99808022,70.59150751)(639.09808012,70.85650725)(639.1880835,71.12651611)
\curveto(639.28807993,71.39650671)(639.39307983,71.66150644)(639.5030835,71.92151611)
\curveto(639.55307967,72.03150607)(639.59807962,72.15150595)(639.6380835,72.28151611)
\curveto(639.68807953,72.42150568)(639.75807946,72.52150558)(639.8480835,72.58151611)
\curveto(639.88807933,72.62150548)(639.95307927,72.65150545)(640.0430835,72.67151611)
\curveto(640.06307916,72.68150542)(640.08307914,72.68150542)(640.1030835,72.67151611)
\curveto(640.13307909,72.67150543)(640.15807906,72.67650543)(640.1780835,72.68651611)
\curveto(640.35807886,72.68650542)(640.56807865,72.68650542)(640.8080835,72.68651611)
\curveto(641.04807817,72.69650541)(641.223078,72.66150544)(641.3330835,72.58151611)
\curveto(641.41307781,72.52150558)(641.47307775,72.42150568)(641.5130835,72.28151611)
\curveto(641.56307766,72.15150595)(641.61307761,72.03150607)(641.6630835,71.92151611)
\curveto(641.76307746,71.69150641)(641.85307737,71.46150664)(641.9330835,71.23151611)
\curveto(642.01307721,71.0015071)(642.10307712,70.77150733)(642.2030835,70.54151611)
\curveto(642.28307694,70.34150776)(642.35807686,70.13650797)(642.4280835,69.92651611)
\curveto(642.50807671,69.71650839)(642.59307663,69.51150859)(642.6830835,69.31151611)
\curveto(642.98307624,68.58150952)(643.26807595,67.84151026)(643.5380835,67.09151611)
\curveto(643.8180754,66.35151175)(644.11307511,65.61651249)(644.4230835,64.88651611)
\curveto(644.46307476,64.79651331)(644.49307473,64.71151339)(644.5130835,64.63151611)
\curveto(644.54307468,64.55151355)(644.57307465,64.46651364)(644.6030835,64.37651611)
\curveto(644.71307451,64.11651399)(644.8180744,63.85151425)(644.9180835,63.58151611)
\curveto(645.02807419,63.31151479)(645.13807408,63.04651506)(645.2480835,62.78651611)
\moveto(642.0380835,66.43151611)
\curveto(642.12807709,66.46151164)(642.18307704,66.51151159)(642.2030835,66.58151611)
\curveto(642.23307699,66.65151145)(642.23807698,66.72651138)(642.2180835,66.80651611)
\curveto(642.20807701,66.89651121)(642.18307704,66.98151112)(642.1430835,67.06151611)
\curveto(642.11307711,67.15151095)(642.08307714,67.22651088)(642.0530835,67.28651611)
\curveto(642.03307719,67.32651078)(642.0230772,67.36151074)(642.0230835,67.39151611)
\curveto(642.0230772,67.42151068)(642.01307721,67.45651065)(641.9930835,67.49651611)
\lineto(641.9030835,67.73651611)
\curveto(641.88307734,67.82651028)(641.85307737,67.91651019)(641.8130835,68.00651611)
\curveto(641.66307756,68.36650974)(641.52807769,68.73150937)(641.4080835,69.10151611)
\curveto(641.29807792,69.48150862)(641.16807805,69.85150825)(641.0180835,70.21151611)
\curveto(640.96807825,70.32150778)(640.9230783,70.43150767)(640.8830835,70.54151611)
\curveto(640.85307837,70.65150745)(640.81307841,70.75650735)(640.7630835,70.85651611)
\curveto(640.74307848,70.9065072)(640.7180785,70.95150715)(640.6880835,70.99151611)
\curveto(640.66807855,71.04150706)(640.6180786,71.06650704)(640.5380835,71.06651611)
\curveto(640.5180787,71.04650706)(640.49807872,71.03150707)(640.4780835,71.02151611)
\curveto(640.45807876,71.01150709)(640.43807878,70.99650711)(640.4180835,70.97651611)
\curveto(640.37807884,70.92650718)(640.34807887,70.87150723)(640.3280835,70.81151611)
\curveto(640.30807891,70.76150734)(640.28807893,70.7065074)(640.2680835,70.64651611)
\curveto(640.218079,70.53650757)(640.17807904,70.42650768)(640.1480835,70.31651611)
\curveto(640.1180791,70.2065079)(640.07807914,70.09650801)(640.0280835,69.98651611)
\curveto(639.85807936,69.59650851)(639.70807951,69.2015089)(639.5780835,68.80151611)
\curveto(639.45807976,68.4015097)(639.3180799,68.01151009)(639.1580835,67.63151611)
\lineto(639.0980835,67.48151611)
\curveto(639.08808013,67.43151067)(639.07308015,67.38151072)(639.0530835,67.33151611)
\lineto(638.9630835,67.09151611)
\curveto(638.93308029,67.01151109)(638.90808031,66.93151117)(638.8880835,66.85151611)
\curveto(638.86808035,66.8015113)(638.85808036,66.74651136)(638.8580835,66.68651611)
\curveto(638.86808035,66.62651148)(638.88308034,66.57651153)(638.9030835,66.53651611)
\curveto(638.95308027,66.45651165)(639.05808016,66.41151169)(639.2180835,66.40151611)
\lineto(639.6680835,66.40151611)
\lineto(641.2730835,66.40151611)
\curveto(641.38307784,66.4015117)(641.5180777,66.39651171)(641.6780835,66.38651611)
\curveto(641.83807738,66.38651172)(641.95807726,66.4015117)(642.0380835,66.43151611)
}
}
{
\newrgbcolor{curcolor}{0 0 0}
\pscustom[linestyle=none,fillstyle=solid,fillcolor=curcolor]
{
\newpath
\moveto(653.324646,62.81651611)
\lineto(653.324646,62.42651611)
\curveto(653.32463812,62.3065158)(653.29963815,62.2065159)(653.249646,62.12651611)
\curveto(653.19963825,62.05651605)(653.11463833,62.01651609)(652.994646,62.00651611)
\lineto(652.649646,62.00651611)
\curveto(652.58963886,62.0065161)(652.52963892,62.0015161)(652.469646,61.99151611)
\curveto(652.41963903,61.99151611)(652.37463907,62.0015161)(652.334646,62.02151611)
\curveto(652.2446392,62.04151606)(652.18463926,62.08151602)(652.154646,62.14151611)
\curveto(652.11463933,62.19151591)(652.08963936,62.25151585)(652.079646,62.32151611)
\curveto(652.07963937,62.39151571)(652.06463938,62.46151564)(652.034646,62.53151611)
\curveto(652.02463942,62.55151555)(652.00963944,62.56651554)(651.989646,62.57651611)
\curveto(651.97963947,62.59651551)(651.96463948,62.61651549)(651.944646,62.63651611)
\curveto(651.8446396,62.64651546)(651.76463968,62.62651548)(651.704646,62.57651611)
\curveto(651.65463979,62.52651558)(651.59963985,62.47651563)(651.539646,62.42651611)
\curveto(651.33964011,62.27651583)(651.13964031,62.16151594)(650.939646,62.08151611)
\curveto(650.75964069,62.0015161)(650.5496409,61.94151616)(650.309646,61.90151611)
\curveto(650.07964137,61.86151624)(649.83964161,61.84151626)(649.589646,61.84151611)
\curveto(649.3496421,61.83151627)(649.10964234,61.84651626)(648.869646,61.88651611)
\curveto(648.62964282,61.91651619)(648.41964303,61.97151613)(648.239646,62.05151611)
\curveto(647.71964373,62.27151583)(647.29964415,62.56651554)(646.979646,62.93651611)
\curveto(646.65964479,63.31651479)(646.40964504,63.78651432)(646.229646,64.34651611)
\curveto(646.18964526,64.43651367)(646.15964529,64.52651358)(646.139646,64.61651611)
\curveto(646.12964532,64.71651339)(646.10964534,64.81651329)(646.079646,64.91651611)
\curveto(646.06964538,64.96651314)(646.06464538,65.01651309)(646.064646,65.06651611)
\curveto(646.06464538,65.11651299)(646.05964539,65.16651294)(646.049646,65.21651611)
\curveto(646.02964542,65.26651284)(646.01964543,65.31651279)(646.019646,65.36651611)
\curveto(646.02964542,65.42651268)(646.02964542,65.48151262)(646.019646,65.53151611)
\lineto(646.019646,65.68151611)
\curveto(645.99964545,65.73151237)(645.98964546,65.79651231)(645.989646,65.87651611)
\curveto(645.98964546,65.95651215)(645.99964545,66.02151208)(646.019646,66.07151611)
\lineto(646.019646,66.23651611)
\curveto(646.03964541,66.3065118)(646.0446454,66.37651173)(646.034646,66.44651611)
\curveto(646.03464541,66.52651158)(646.0446454,66.6015115)(646.064646,66.67151611)
\curveto(646.07464537,66.72151138)(646.07964537,66.76651134)(646.079646,66.80651611)
\curveto(646.07964537,66.84651126)(646.08464536,66.89151121)(646.094646,66.94151611)
\curveto(646.12464532,67.04151106)(646.1496453,67.13651097)(646.169646,67.22651611)
\curveto(646.18964526,67.32651078)(646.21464523,67.42151068)(646.244646,67.51151611)
\curveto(646.37464507,67.89151021)(646.53964491,68.23150987)(646.739646,68.53151611)
\curveto(646.9496445,68.84150926)(647.19964425,69.09650901)(647.489646,69.29651611)
\curveto(647.65964379,69.41650869)(647.83464361,69.51650859)(648.014646,69.59651611)
\curveto(648.20464324,69.67650843)(648.40964304,69.74650836)(648.629646,69.80651611)
\curveto(648.69964275,69.81650829)(648.76464268,69.82650828)(648.824646,69.83651611)
\curveto(648.89464255,69.84650826)(648.96464248,69.86150824)(649.034646,69.88151611)
\lineto(649.184646,69.88151611)
\curveto(649.26464218,69.9015082)(649.37964207,69.91150819)(649.529646,69.91151611)
\curveto(649.68964176,69.91150819)(649.80964164,69.9015082)(649.889646,69.88151611)
\curveto(649.92964152,69.87150823)(649.98464146,69.86650824)(650.054646,69.86651611)
\curveto(650.16464128,69.83650827)(650.27464117,69.81150829)(650.384646,69.79151611)
\curveto(650.49464095,69.78150832)(650.59964085,69.75150835)(650.699646,69.70151611)
\curveto(650.8496406,69.64150846)(650.98964046,69.57650853)(651.119646,69.50651611)
\curveto(651.25964019,69.43650867)(651.38964006,69.35650875)(651.509646,69.26651611)
\curveto(651.56963988,69.21650889)(651.62963982,69.16150894)(651.689646,69.10151611)
\curveto(651.75963969,69.05150905)(651.8496396,69.03650907)(651.959646,69.05651611)
\curveto(651.97963947,69.08650902)(651.99463945,69.11150899)(652.004646,69.13151611)
\curveto(652.02463942,69.15150895)(652.03963941,69.18150892)(652.049646,69.22151611)
\curveto(652.07963937,69.31150879)(652.08963936,69.42650868)(652.079646,69.56651611)
\lineto(652.079646,69.94151611)
\lineto(652.079646,71.66651611)
\lineto(652.079646,72.13151611)
\curveto(652.07963937,72.31150579)(652.10463934,72.44150566)(652.154646,72.52151611)
\curveto(652.19463925,72.59150551)(652.25463919,72.63650547)(652.334646,72.65651611)
\curveto(652.35463909,72.65650545)(652.37963907,72.65650545)(652.409646,72.65651611)
\curveto(652.43963901,72.66650544)(652.46463898,72.67150543)(652.484646,72.67151611)
\curveto(652.62463882,72.68150542)(652.76963868,72.68150542)(652.919646,72.67151611)
\curveto(653.07963837,72.67150543)(653.18963826,72.63150547)(653.249646,72.55151611)
\curveto(653.29963815,72.47150563)(653.32463812,72.37150573)(653.324646,72.25151611)
\lineto(653.324646,71.87651611)
\lineto(653.324646,62.81651611)
\moveto(652.109646,65.65151611)
\curveto(652.12963932,65.7015124)(652.13963931,65.76651234)(652.139646,65.84651611)
\curveto(652.13963931,65.93651217)(652.12963932,66.0065121)(652.109646,66.05651611)
\lineto(652.109646,66.28151611)
\curveto(652.08963936,66.37151173)(652.07463937,66.46151164)(652.064646,66.55151611)
\curveto(652.05463939,66.65151145)(652.03463941,66.74151136)(652.004646,66.82151611)
\curveto(651.98463946,66.9015112)(651.96463948,66.97651113)(651.944646,67.04651611)
\curveto(651.93463951,67.11651099)(651.91463953,67.18651092)(651.884646,67.25651611)
\curveto(651.76463968,67.55651055)(651.60963984,67.82151028)(651.419646,68.05151611)
\curveto(651.22964022,68.28150982)(650.98964046,68.46150964)(650.699646,68.59151611)
\curveto(650.59964085,68.64150946)(650.49464095,68.67650943)(650.384646,68.69651611)
\curveto(650.28464116,68.72650938)(650.17464127,68.75150935)(650.054646,68.77151611)
\curveto(649.97464147,68.79150931)(649.88464156,68.8015093)(649.784646,68.80151611)
\lineto(649.514646,68.80151611)
\curveto(649.46464198,68.79150931)(649.41964203,68.78150932)(649.379646,68.77151611)
\lineto(649.244646,68.77151611)
\curveto(649.16464228,68.75150935)(649.07964237,68.73150937)(648.989646,68.71151611)
\curveto(648.90964254,68.69150941)(648.82964262,68.66650944)(648.749646,68.63651611)
\curveto(648.42964302,68.49650961)(648.16964328,68.29150981)(647.969646,68.02151611)
\curveto(647.77964367,67.76151034)(647.62464382,67.45651065)(647.504646,67.10651611)
\curveto(647.46464398,66.99651111)(647.43464401,66.88151122)(647.414646,66.76151611)
\curveto(647.40464404,66.65151145)(647.38964406,66.54151156)(647.369646,66.43151611)
\curveto(647.36964408,66.39151171)(647.36464408,66.35151175)(647.354646,66.31151611)
\lineto(647.354646,66.20651611)
\curveto(647.33464411,66.15651195)(647.32464412,66.101512)(647.324646,66.04151611)
\curveto(647.33464411,65.98151212)(647.33964411,65.92651218)(647.339646,65.87651611)
\lineto(647.339646,65.54651611)
\curveto(647.33964411,65.44651266)(647.3496441,65.35151275)(647.369646,65.26151611)
\curveto(647.37964407,65.23151287)(647.38464406,65.18151292)(647.384646,65.11151611)
\curveto(647.40464404,65.04151306)(647.41964403,64.97151313)(647.429646,64.90151611)
\lineto(647.489646,64.69151611)
\curveto(647.59964385,64.34151376)(647.7496437,64.04151406)(647.939646,63.79151611)
\curveto(648.12964332,63.54151456)(648.36964308,63.33651477)(648.659646,63.17651611)
\curveto(648.7496427,63.12651498)(648.83964261,63.08651502)(648.929646,63.05651611)
\curveto(649.01964243,63.02651508)(649.11964233,62.99651511)(649.229646,62.96651611)
\curveto(649.27964217,62.94651516)(649.32964212,62.94151516)(649.379646,62.95151611)
\curveto(649.43964201,62.96151514)(649.49464195,62.95651515)(649.544646,62.93651611)
\curveto(649.58464186,62.92651518)(649.62464182,62.92151518)(649.664646,62.92151611)
\lineto(649.799646,62.92151611)
\lineto(649.934646,62.92151611)
\curveto(649.96464148,62.93151517)(650.01464143,62.93651517)(650.084646,62.93651611)
\curveto(650.16464128,62.95651515)(650.2446412,62.97151513)(650.324646,62.98151611)
\curveto(650.40464104,63.0015151)(650.47964097,63.02651508)(650.549646,63.05651611)
\curveto(650.87964057,63.19651491)(651.1446403,63.37151473)(651.344646,63.58151611)
\curveto(651.55463989,63.8015143)(651.72963972,64.07651403)(651.869646,64.40651611)
\curveto(651.91963953,64.51651359)(651.95463949,64.62651348)(651.974646,64.73651611)
\curveto(651.99463945,64.84651326)(652.01963943,64.95651315)(652.049646,65.06651611)
\curveto(652.06963938,65.106513)(652.07963937,65.14151296)(652.079646,65.17151611)
\curveto(652.07963937,65.21151289)(652.08463936,65.25151285)(652.094646,65.29151611)
\curveto(652.10463934,65.35151275)(652.10463934,65.41151269)(652.094646,65.47151611)
\curveto(652.09463935,65.53151257)(652.09963935,65.59151251)(652.109646,65.65151611)
}
}
{
\newrgbcolor{curcolor}{0 0 0}
\pscustom[linestyle=none,fillstyle=solid,fillcolor=curcolor]
{
\newpath
\moveto(658.960896,69.91151611)
\curveto(659.34089101,69.92150818)(659.66089069,69.88150822)(659.920896,69.79151611)
\curveto(660.19089016,69.7015084)(660.43588992,69.57150853)(660.655896,69.40151611)
\curveto(660.73588962,69.35150875)(660.80088955,69.28150882)(660.850896,69.19151611)
\curveto(660.91088944,69.11150899)(660.97588938,69.03650907)(661.045896,68.96651611)
\curveto(661.06588929,68.94650916)(661.09588926,68.92150918)(661.135896,68.89151611)
\curveto(661.17588918,68.86150924)(661.22588913,68.85150925)(661.285896,68.86151611)
\curveto(661.38588897,68.89150921)(661.47088888,68.95150915)(661.540896,69.04151611)
\curveto(661.62088873,69.14150896)(661.70088865,69.21650889)(661.780896,69.26651611)
\curveto(661.92088843,69.37650873)(662.06588829,69.47150863)(662.215896,69.55151611)
\curveto(662.36588799,69.64150846)(662.53088782,69.71650839)(662.710896,69.77651611)
\curveto(662.79088756,69.8065083)(662.87588748,69.82650828)(662.965896,69.83651611)
\curveto(663.06588729,69.85650825)(663.16088719,69.87650823)(663.250896,69.89651611)
\curveto(663.30088705,69.9065082)(663.34588701,69.91150819)(663.385896,69.91151611)
\lineto(663.535896,69.91151611)
\curveto(663.58588677,69.93150817)(663.6558867,69.93650817)(663.745896,69.92651611)
\curveto(663.83588652,69.92650818)(663.90088645,69.92150818)(663.940896,69.91151611)
\curveto(663.99088636,69.9015082)(664.06588629,69.89650821)(664.165896,69.89651611)
\curveto(664.2558861,69.87650823)(664.34088601,69.85650825)(664.420896,69.83651611)
\curveto(664.51088584,69.82650828)(664.59588576,69.8065083)(664.675896,69.77651611)
\curveto(664.72588563,69.75650835)(664.77088558,69.74150836)(664.810896,69.73151611)
\curveto(664.86088549,69.73150837)(664.91088544,69.72150838)(664.960896,69.70151611)
\curveto(665.46088489,69.48150862)(665.80588455,69.14150896)(665.995896,68.68151611)
\curveto(666.03588432,68.6015095)(666.06588429,68.51150959)(666.085896,68.41151611)
\curveto(666.10588425,68.32150978)(666.12588423,68.22150988)(666.145896,68.11151611)
\curveto(666.16588419,68.08151002)(666.17088418,68.04651006)(666.160896,68.00651611)
\curveto(666.16088419,67.97651013)(666.16588419,67.94651016)(666.175896,67.91651611)
\lineto(666.175896,67.78151611)
\curveto(666.18588417,67.74151036)(666.18588417,67.69651041)(666.175896,67.64651611)
\curveto(666.17588418,67.59651051)(666.17588418,67.54651056)(666.175896,67.49651611)
\lineto(666.175896,66.91151611)
\lineto(666.175896,65.95151611)
\lineto(666.175896,63.10151611)
\curveto(666.17588418,62.94151516)(666.17588418,62.75151535)(666.175896,62.53151611)
\curveto(666.18588417,62.31151579)(666.14588421,62.16651594)(666.055896,62.09651611)
\curveto(666.01588434,62.06651604)(665.9508844,62.04151606)(665.860896,62.02151611)
\curveto(665.77088458,62.01151609)(665.67588468,62.0065161)(665.575896,62.00651611)
\curveto(665.47588488,62.0065161)(665.37588498,62.01151609)(665.275896,62.02151611)
\curveto(665.18588517,62.03151607)(665.12088523,62.05151605)(665.080896,62.08151611)
\curveto(665.02088533,62.11151599)(664.98088537,62.17151593)(664.960896,62.26151611)
\curveto(664.94088541,62.32151578)(664.93588542,62.38151572)(664.945896,62.44151611)
\curveto(664.9558854,62.51151559)(664.9508854,62.57651553)(664.930896,62.63651611)
\curveto(664.92088543,62.68651542)(664.91588544,62.74151536)(664.915896,62.80151611)
\curveto(664.92588543,62.87151523)(664.93088542,62.93651517)(664.930896,62.99651611)
\lineto(664.930896,63.67151611)
\lineto(664.930896,66.53651611)
\curveto(664.93088542,66.86651124)(664.92088543,67.17651093)(664.900896,67.46651611)
\curveto(664.89088546,67.76651034)(664.82088553,68.01651009)(664.690896,68.21651611)
\curveto(664.54088581,68.45650965)(664.31088604,68.63150947)(664.000896,68.74151611)
\curveto(663.94088641,68.76150934)(663.87588648,68.77150933)(663.805896,68.77151611)
\curveto(663.74588661,68.78150932)(663.68088667,68.79650931)(663.610896,68.81651611)
\curveto(663.57088678,68.82650928)(663.50588685,68.82650928)(663.415896,68.81651611)
\curveto(663.32588703,68.81650929)(663.26588709,68.81150929)(663.235896,68.80151611)
\curveto(663.18588717,68.79150931)(663.13588722,68.78650932)(663.085896,68.78651611)
\curveto(663.03588732,68.79650931)(662.98588737,68.79150931)(662.935896,68.77151611)
\curveto(662.79588756,68.74150936)(662.66088769,68.7015094)(662.530896,68.65151611)
\curveto(662.01088834,68.43150967)(661.66088869,68.04651006)(661.480896,67.49651611)
\curveto(661.43088892,67.32651078)(661.40088895,67.13151097)(661.390896,66.91151611)
\lineto(661.390896,66.23651611)
\lineto(661.390896,64.27151611)
\lineto(661.390896,62.81651611)
\lineto(661.390896,62.44151611)
\curveto(661.39088896,62.32151578)(661.36588899,62.22651588)(661.315896,62.15651611)
\curveto(661.26588909,62.07651603)(661.18088917,62.03151607)(661.060896,62.02151611)
\curveto(660.94088941,62.01151609)(660.81588954,62.0065161)(660.685896,62.00651611)
\curveto(660.51588984,62.0065161)(660.39088996,62.02651608)(660.310896,62.06651611)
\curveto(660.22089013,62.11651599)(660.16589019,62.19651591)(660.145896,62.30651611)
\curveto(660.13589022,62.42651568)(660.13089022,62.55651555)(660.130896,62.69651611)
\lineto(660.130896,64.12151611)
\lineto(660.130896,66.59651611)
\curveto(660.13089022,66.91651119)(660.12089023,67.21151089)(660.100896,67.48151611)
\curveto(660.08089027,67.76151034)(660.01089034,68.0015101)(659.890896,68.20151611)
\curveto(659.78089057,68.38150972)(659.6558907,68.51150959)(659.515896,68.59151611)
\curveto(659.37589098,68.68150942)(659.18589117,68.75150935)(658.945896,68.80151611)
\curveto(658.90589145,68.81150929)(658.86089149,68.81650929)(658.810896,68.81651611)
\lineto(658.675896,68.81651611)
\curveto(658.4558919,68.81650929)(658.26089209,68.79150931)(658.090896,68.74151611)
\curveto(657.93089242,68.69150941)(657.78589257,68.62650948)(657.655896,68.54651611)
\curveto(657.14589321,68.23650987)(656.80589355,67.77151033)(656.635896,67.15151611)
\curveto(656.59589376,67.02151108)(656.57589378,66.87151123)(656.575896,66.70151611)
\curveto(656.58589377,66.54151156)(656.59089376,66.38151172)(656.590896,66.22151611)
\lineto(656.590896,64.52651611)
\lineto(656.590896,62.87651611)
\lineto(656.590896,62.47151611)
\curveto(656.59089376,62.33151577)(656.56089379,62.22151588)(656.500896,62.14151611)
\curveto(656.4508939,62.07151603)(656.37589398,62.03151607)(656.275896,62.02151611)
\curveto(656.17589418,62.01151609)(656.07089428,62.0065161)(655.960896,62.00651611)
\lineto(655.735896,62.00651611)
\curveto(655.67589468,62.02651608)(655.61589474,62.04151606)(655.555896,62.05151611)
\curveto(655.50589485,62.06151604)(655.46089489,62.09151601)(655.420896,62.14151611)
\curveto(655.37089498,62.2015159)(655.34589501,62.27651583)(655.345896,62.36651611)
\lineto(655.345896,62.68151611)
\lineto(655.345896,63.65651611)
\lineto(655.345896,67.94651611)
\lineto(655.345896,69.05651611)
\lineto(655.345896,69.34151611)
\curveto(655.34589501,69.44150866)(655.36589499,69.52150858)(655.405896,69.58151611)
\curveto(655.43589492,69.64150846)(655.48089487,69.68150842)(655.540896,69.70151611)
\curveto(655.62089473,69.73150837)(655.74589461,69.74650836)(655.915896,69.74651611)
\curveto(656.09589426,69.74650836)(656.22589413,69.73150837)(656.305896,69.70151611)
\curveto(656.38589397,69.66150844)(656.44089391,69.61150849)(656.470896,69.55151611)
\curveto(656.49089386,69.5015086)(656.50089385,69.44150866)(656.500896,69.37151611)
\curveto(656.51089384,69.3015088)(656.52089383,69.23650887)(656.530896,69.17651611)
\curveto(656.54089381,69.11650899)(656.56089379,69.06650904)(656.590896,69.02651611)
\curveto(656.62089373,68.98650912)(656.67089368,68.96650914)(656.740896,68.96651611)
\curveto(656.76089359,68.98650912)(656.78089357,68.99650911)(656.800896,68.99651611)
\curveto(656.83089352,68.99650911)(656.8558935,69.0065091)(656.875896,69.02651611)
\curveto(656.93589342,69.07650903)(656.99089336,69.12650898)(657.040896,69.17651611)
\lineto(657.220896,69.32651611)
\curveto(657.44089291,69.48650862)(657.69089266,69.62650848)(657.970896,69.74651611)
\curveto(658.07089228,69.78650832)(658.17089218,69.81150829)(658.270896,69.82151611)
\curveto(658.37089198,69.84150826)(658.47589188,69.86650824)(658.585896,69.89651611)
\lineto(658.765896,69.89651611)
\curveto(658.83589152,69.9065082)(658.90089145,69.91150819)(658.960896,69.91151611)
}
}
{
\newrgbcolor{curcolor}{0 0 0}
\pscustom[linestyle=none,fillstyle=solid,fillcolor=curcolor]
{
\newpath
\moveto(668.35863037,71.23151611)
\curveto(668.27862925,71.29150681)(668.2336293,71.39650671)(668.22363037,71.54651611)
\lineto(668.22363037,72.01151611)
\lineto(668.22363037,72.26651611)
\curveto(668.22362931,72.35650575)(668.23862929,72.43150567)(668.26863037,72.49151611)
\curveto(668.30862922,72.57150553)(668.38862914,72.63150547)(668.50863037,72.67151611)
\curveto(668.528629,72.68150542)(668.54862898,72.68150542)(668.56863037,72.67151611)
\curveto(668.59862893,72.67150543)(668.62362891,72.67650543)(668.64363037,72.68651611)
\curveto(668.81362872,72.68650542)(668.97362856,72.68150542)(669.12363037,72.67151611)
\curveto(669.27362826,72.66150544)(669.37362816,72.6015055)(669.42363037,72.49151611)
\curveto(669.45362808,72.43150567)(669.46862806,72.35650575)(669.46863037,72.26651611)
\lineto(669.46863037,72.01151611)
\curveto(669.46862806,71.83150627)(669.46362807,71.66150644)(669.45363037,71.50151611)
\curveto(669.45362808,71.34150676)(669.38862814,71.23650687)(669.25863037,71.18651611)
\curveto(669.20862832,71.16650694)(669.15362838,71.15650695)(669.09363037,71.15651611)
\lineto(668.92863037,71.15651611)
\lineto(668.61363037,71.15651611)
\curveto(668.51362902,71.15650695)(668.4286291,71.18150692)(668.35863037,71.23151611)
\moveto(669.46863037,62.72651611)
\lineto(669.46863037,62.41151611)
\curveto(669.47862805,62.31151579)(669.45862807,62.23151587)(669.40863037,62.17151611)
\curveto(669.37862815,62.11151599)(669.3336282,62.07151603)(669.27363037,62.05151611)
\curveto(669.21362832,62.04151606)(669.14362839,62.02651608)(669.06363037,62.00651611)
\lineto(668.83863037,62.00651611)
\curveto(668.70862882,62.0065161)(668.59362894,62.01151609)(668.49363037,62.02151611)
\curveto(668.40362913,62.04151606)(668.3336292,62.09151601)(668.28363037,62.17151611)
\curveto(668.24362929,62.23151587)(668.22362931,62.3065158)(668.22363037,62.39651611)
\lineto(668.22363037,62.68151611)
\lineto(668.22363037,69.02651611)
\lineto(668.22363037,69.34151611)
\curveto(668.22362931,69.45150865)(668.24862928,69.53650857)(668.29863037,69.59651611)
\curveto(668.3286292,69.64650846)(668.36862916,69.67650843)(668.41863037,69.68651611)
\curveto(668.46862906,69.69650841)(668.52362901,69.71150839)(668.58363037,69.73151611)
\curveto(668.60362893,69.73150837)(668.62362891,69.72650838)(668.64363037,69.71651611)
\curveto(668.67362886,69.71650839)(668.69862883,69.72150838)(668.71863037,69.73151611)
\curveto(668.84862868,69.73150837)(668.97862855,69.72650838)(669.10863037,69.71651611)
\curveto(669.24862828,69.71650839)(669.34362819,69.67650843)(669.39363037,69.59651611)
\curveto(669.44362809,69.53650857)(669.46862806,69.45650865)(669.46863037,69.35651611)
\lineto(669.46863037,69.07151611)
\lineto(669.46863037,62.72651611)
}
}
{
\newrgbcolor{curcolor}{0 0 0}
\pscustom[linestyle=none,fillstyle=solid,fillcolor=curcolor]
{
\newpath
\moveto(675.10347412,69.88151611)
\curveto(675.73346889,69.9015082)(676.23846838,69.81650829)(676.61847412,69.62651611)
\curveto(676.99846762,69.43650867)(677.30346732,69.15150895)(677.53347412,68.77151611)
\curveto(677.59346703,68.67150943)(677.63846698,68.56150954)(677.66847412,68.44151611)
\curveto(677.70846691,68.33150977)(677.74346688,68.21650989)(677.77347412,68.09651611)
\curveto(677.8234668,67.9065102)(677.85346677,67.7015104)(677.86347412,67.48151611)
\curveto(677.87346675,67.26151084)(677.87846674,67.03651107)(677.87847412,66.80651611)
\lineto(677.87847412,65.20151611)
\lineto(677.87847412,62.86151611)
\curveto(677.87846674,62.69151541)(677.87346675,62.52151558)(677.86347412,62.35151611)
\curveto(677.86346676,62.18151592)(677.79846682,62.07151603)(677.66847412,62.02151611)
\curveto(677.618467,62.0015161)(677.56346706,61.99151611)(677.50347412,61.99151611)
\curveto(677.45346717,61.98151612)(677.39846722,61.97651613)(677.33847412,61.97651611)
\curveto(677.20846741,61.97651613)(677.08346754,61.98151612)(676.96347412,61.99151611)
\curveto(676.84346778,61.99151611)(676.75846786,62.03151607)(676.70847412,62.11151611)
\curveto(676.65846796,62.18151592)(676.63346799,62.27151583)(676.63347412,62.38151611)
\lineto(676.63347412,62.71151611)
\lineto(676.63347412,64.00151611)
\lineto(676.63347412,66.44651611)
\curveto(676.63346799,66.71651139)(676.62846799,66.98151112)(676.61847412,67.24151611)
\curveto(676.60846801,67.51151059)(676.56346806,67.74151036)(676.48347412,67.93151611)
\curveto(676.40346822,68.13150997)(676.28346834,68.29150981)(676.12347412,68.41151611)
\curveto(675.96346866,68.54150956)(675.77846884,68.64150946)(675.56847412,68.71151611)
\curveto(675.50846911,68.73150937)(675.44346918,68.74150936)(675.37347412,68.74151611)
\curveto(675.31346931,68.75150935)(675.25346937,68.76650934)(675.19347412,68.78651611)
\curveto(675.14346948,68.79650931)(675.06346956,68.79650931)(674.95347412,68.78651611)
\curveto(674.85346977,68.78650932)(674.78346984,68.78150932)(674.74347412,68.77151611)
\curveto(674.70346992,68.75150935)(674.66846995,68.74150936)(674.63847412,68.74151611)
\curveto(674.60847001,68.75150935)(674.57347005,68.75150935)(674.53347412,68.74151611)
\curveto(674.40347022,68.71150939)(674.27847034,68.67650943)(674.15847412,68.63651611)
\curveto(674.04847057,68.6065095)(673.94347068,68.56150954)(673.84347412,68.50151611)
\curveto(673.80347082,68.48150962)(673.76847085,68.46150964)(673.73847412,68.44151611)
\curveto(673.70847091,68.42150968)(673.67347095,68.4015097)(673.63347412,68.38151611)
\curveto(673.28347134,68.13150997)(673.02847159,67.75651035)(672.86847412,67.25651611)
\curveto(672.83847178,67.17651093)(672.8184718,67.09151101)(672.80847412,67.00151611)
\curveto(672.79847182,66.92151118)(672.78347184,66.84151126)(672.76347412,66.76151611)
\curveto(672.74347188,66.71151139)(672.73847188,66.66151144)(672.74847412,66.61151611)
\curveto(672.75847186,66.57151153)(672.75347187,66.53151157)(672.73347412,66.49151611)
\lineto(672.73347412,66.17651611)
\curveto(672.7234719,66.14651196)(672.7184719,66.11151199)(672.71847412,66.07151611)
\curveto(672.72847189,66.03151207)(672.73347189,65.98651212)(672.73347412,65.93651611)
\lineto(672.73347412,65.48651611)
\lineto(672.73347412,64.04651611)
\lineto(672.73347412,62.72651611)
\lineto(672.73347412,62.38151611)
\curveto(672.73347189,62.27151583)(672.70847191,62.18151592)(672.65847412,62.11151611)
\curveto(672.60847201,62.03151607)(672.5184721,61.99151611)(672.38847412,61.99151611)
\curveto(672.26847235,61.98151612)(672.14347248,61.97651613)(672.01347412,61.97651611)
\curveto(671.93347269,61.97651613)(671.85847276,61.98151612)(671.78847412,61.99151611)
\curveto(671.7184729,62.0015161)(671.65847296,62.02651608)(671.60847412,62.06651611)
\curveto(671.52847309,62.11651599)(671.48847313,62.21151589)(671.48847412,62.35151611)
\lineto(671.48847412,62.75651611)
\lineto(671.48847412,64.52651611)
\lineto(671.48847412,68.15651611)
\lineto(671.48847412,69.07151611)
\lineto(671.48847412,69.34151611)
\curveto(671.48847313,69.43150867)(671.50847311,69.5015086)(671.54847412,69.55151611)
\curveto(671.57847304,69.61150849)(671.62847299,69.65150845)(671.69847412,69.67151611)
\curveto(671.73847288,69.68150842)(671.79347283,69.69150841)(671.86347412,69.70151611)
\curveto(671.94347268,69.71150839)(672.0234726,69.71650839)(672.10347412,69.71651611)
\curveto(672.18347244,69.71650839)(672.25847236,69.71150839)(672.32847412,69.70151611)
\curveto(672.40847221,69.69150841)(672.46347216,69.67650843)(672.49347412,69.65651611)
\curveto(672.60347202,69.58650852)(672.65347197,69.49650861)(672.64347412,69.38651611)
\curveto(672.63347199,69.28650882)(672.64847197,69.17150893)(672.68847412,69.04151611)
\curveto(672.70847191,68.98150912)(672.74847187,68.93150917)(672.80847412,68.89151611)
\curveto(672.92847169,68.88150922)(673.0234716,68.92650918)(673.09347412,69.02651611)
\curveto(673.17347145,69.12650898)(673.25347137,69.2065089)(673.33347412,69.26651611)
\curveto(673.47347115,69.36650874)(673.61347101,69.45650865)(673.75347412,69.53651611)
\curveto(673.90347072,69.62650848)(674.07347055,69.7015084)(674.26347412,69.76151611)
\curveto(674.34347028,69.79150831)(674.42847019,69.81150829)(674.51847412,69.82151611)
\curveto(674.61847,69.83150827)(674.71346991,69.84650826)(674.80347412,69.86651611)
\curveto(674.85346977,69.87650823)(674.90346972,69.88150822)(674.95347412,69.88151611)
\lineto(675.10347412,69.88151611)
}
}
{
\newrgbcolor{curcolor}{0 0 0}
\pscustom[linestyle=none,fillstyle=solid,fillcolor=curcolor]
{
\newpath
\moveto(680.0480835,71.23151611)
\curveto(679.96808238,71.29150681)(679.92308242,71.39650671)(679.9130835,71.54651611)
\lineto(679.9130835,72.01151611)
\lineto(679.9130835,72.26651611)
\curveto(679.91308243,72.35650575)(679.92808242,72.43150567)(679.9580835,72.49151611)
\curveto(679.99808235,72.57150553)(680.07808227,72.63150547)(680.1980835,72.67151611)
\curveto(680.21808213,72.68150542)(680.23808211,72.68150542)(680.2580835,72.67151611)
\curveto(680.28808206,72.67150543)(680.31308203,72.67650543)(680.3330835,72.68651611)
\curveto(680.50308184,72.68650542)(680.66308168,72.68150542)(680.8130835,72.67151611)
\curveto(680.96308138,72.66150544)(681.06308128,72.6015055)(681.1130835,72.49151611)
\curveto(681.1430812,72.43150567)(681.15808119,72.35650575)(681.1580835,72.26651611)
\lineto(681.1580835,72.01151611)
\curveto(681.15808119,71.83150627)(681.15308119,71.66150644)(681.1430835,71.50151611)
\curveto(681.1430812,71.34150676)(681.07808127,71.23650687)(680.9480835,71.18651611)
\curveto(680.89808145,71.16650694)(680.8430815,71.15650695)(680.7830835,71.15651611)
\lineto(680.6180835,71.15651611)
\lineto(680.3030835,71.15651611)
\curveto(680.20308214,71.15650695)(680.11808223,71.18150692)(680.0480835,71.23151611)
\moveto(681.1580835,62.72651611)
\lineto(681.1580835,62.41151611)
\curveto(681.16808118,62.31151579)(681.1480812,62.23151587)(681.0980835,62.17151611)
\curveto(681.06808128,62.11151599)(681.02308132,62.07151603)(680.9630835,62.05151611)
\curveto(680.90308144,62.04151606)(680.83308151,62.02651608)(680.7530835,62.00651611)
\lineto(680.5280835,62.00651611)
\curveto(680.39808195,62.0065161)(680.28308206,62.01151609)(680.1830835,62.02151611)
\curveto(680.09308225,62.04151606)(680.02308232,62.09151601)(679.9730835,62.17151611)
\curveto(679.93308241,62.23151587)(679.91308243,62.3065158)(679.9130835,62.39651611)
\lineto(679.9130835,62.68151611)
\lineto(679.9130835,69.02651611)
\lineto(679.9130835,69.34151611)
\curveto(679.91308243,69.45150865)(679.93808241,69.53650857)(679.9880835,69.59651611)
\curveto(680.01808233,69.64650846)(680.05808229,69.67650843)(680.1080835,69.68651611)
\curveto(680.15808219,69.69650841)(680.21308213,69.71150839)(680.2730835,69.73151611)
\curveto(680.29308205,69.73150837)(680.31308203,69.72650838)(680.3330835,69.71651611)
\curveto(680.36308198,69.71650839)(680.38808196,69.72150838)(680.4080835,69.73151611)
\curveto(680.53808181,69.73150837)(680.66808168,69.72650838)(680.7980835,69.71651611)
\curveto(680.93808141,69.71650839)(681.03308131,69.67650843)(681.0830835,69.59651611)
\curveto(681.13308121,69.53650857)(681.15808119,69.45650865)(681.1580835,69.35651611)
\lineto(681.1580835,69.07151611)
\lineto(681.1580835,62.72651611)
}
}
{
\newrgbcolor{curcolor}{0 0 0}
\pscustom[linestyle=none,fillstyle=solid,fillcolor=curcolor]
{
\newpath
\moveto(685.53292725,69.91151611)
\curveto(686.25292318,69.92150818)(686.85792258,69.83650827)(687.34792725,69.65651611)
\curveto(687.8379216,69.48650862)(688.21792122,69.18150892)(688.48792725,68.74151611)
\curveto(688.55792088,68.63150947)(688.61292082,68.51650959)(688.65292725,68.39651611)
\curveto(688.69292074,68.28650982)(688.7329207,68.16150994)(688.77292725,68.02151611)
\curveto(688.79292064,67.95151015)(688.79792064,67.87651023)(688.78792725,67.79651611)
\curveto(688.77792066,67.72651038)(688.76292067,67.67151043)(688.74292725,67.63151611)
\curveto(688.72292071,67.61151049)(688.69792074,67.59151051)(688.66792725,67.57151611)
\curveto(688.6379208,67.56151054)(688.61292082,67.54651056)(688.59292725,67.52651611)
\curveto(688.54292089,67.5065106)(688.49292094,67.5015106)(688.44292725,67.51151611)
\curveto(688.39292104,67.52151058)(688.34292109,67.52151058)(688.29292725,67.51151611)
\curveto(688.21292122,67.49151061)(688.10792133,67.48651062)(687.97792725,67.49651611)
\curveto(687.84792159,67.51651059)(687.75792168,67.54151056)(687.70792725,67.57151611)
\curveto(687.62792181,67.62151048)(687.57292186,67.68651042)(687.54292725,67.76651611)
\curveto(687.52292191,67.85651025)(687.48792195,67.94151016)(687.43792725,68.02151611)
\curveto(687.34792209,68.18150992)(687.22292221,68.32650978)(687.06292725,68.45651611)
\curveto(686.95292248,68.53650957)(686.8329226,68.59650951)(686.70292725,68.63651611)
\curveto(686.57292286,68.67650943)(686.432923,68.71650939)(686.28292725,68.75651611)
\curveto(686.2329232,68.77650933)(686.18292325,68.78150932)(686.13292725,68.77151611)
\curveto(686.08292335,68.77150933)(686.0329234,68.77650933)(685.98292725,68.78651611)
\curveto(685.92292351,68.8065093)(685.84792359,68.81650929)(685.75792725,68.81651611)
\curveto(685.66792377,68.81650929)(685.59292384,68.8065093)(685.53292725,68.78651611)
\lineto(685.44292725,68.78651611)
\lineto(685.29292725,68.75651611)
\curveto(685.24292419,68.75650935)(685.19292424,68.75150935)(685.14292725,68.74151611)
\curveto(684.88292455,68.68150942)(684.66792477,68.59650951)(684.49792725,68.48651611)
\curveto(684.32792511,68.37650973)(684.21292522,68.19150991)(684.15292725,67.93151611)
\curveto(684.1329253,67.86151024)(684.12792531,67.79151031)(684.13792725,67.72151611)
\curveto(684.15792528,67.65151045)(684.17792526,67.59151051)(684.19792725,67.54151611)
\curveto(684.25792518,67.39151071)(684.32792511,67.28151082)(684.40792725,67.21151611)
\curveto(684.49792494,67.15151095)(684.60792483,67.08151102)(684.73792725,67.00151611)
\curveto(684.89792454,66.9015112)(685.07792436,66.82651128)(685.27792725,66.77651611)
\curveto(685.47792396,66.73651137)(685.67792376,66.68651142)(685.87792725,66.62651611)
\curveto(686.00792343,66.58651152)(686.1379233,66.55651155)(686.26792725,66.53651611)
\curveto(686.39792304,66.51651159)(686.52792291,66.48651162)(686.65792725,66.44651611)
\curveto(686.86792257,66.38651172)(687.07292236,66.32651178)(687.27292725,66.26651611)
\curveto(687.47292196,66.21651189)(687.67292176,66.15151195)(687.87292725,66.07151611)
\lineto(688.02292725,66.01151611)
\curveto(688.07292136,65.99151211)(688.12292131,65.96651214)(688.17292725,65.93651611)
\curveto(688.37292106,65.81651229)(688.54792089,65.68151242)(688.69792725,65.53151611)
\curveto(688.84792059,65.38151272)(688.97292046,65.19151291)(689.07292725,64.96151611)
\curveto(689.09292034,64.89151321)(689.11292032,64.79651331)(689.13292725,64.67651611)
\curveto(689.15292028,64.6065135)(689.16292027,64.53151357)(689.16292725,64.45151611)
\curveto(689.17292026,64.38151372)(689.17792026,64.3015138)(689.17792725,64.21151611)
\lineto(689.17792725,64.06151611)
\curveto(689.15792028,63.99151411)(689.14792029,63.92151418)(689.14792725,63.85151611)
\curveto(689.14792029,63.78151432)(689.1379203,63.71151439)(689.11792725,63.64151611)
\curveto(689.08792035,63.53151457)(689.05292038,63.42651468)(689.01292725,63.32651611)
\curveto(688.97292046,63.22651488)(688.92792051,63.13651497)(688.87792725,63.05651611)
\curveto(688.71792072,62.79651531)(688.51292092,62.58651552)(688.26292725,62.42651611)
\curveto(688.01292142,62.27651583)(687.7329217,62.14651596)(687.42292725,62.03651611)
\curveto(687.3329221,62.0065161)(687.2379222,61.98651612)(687.13792725,61.97651611)
\curveto(687.04792239,61.95651615)(686.95792248,61.93151617)(686.86792725,61.90151611)
\curveto(686.76792267,61.88151622)(686.66792277,61.87151623)(686.56792725,61.87151611)
\curveto(686.46792297,61.87151623)(686.36792307,61.86151624)(686.26792725,61.84151611)
\lineto(686.11792725,61.84151611)
\curveto(686.06792337,61.83151627)(685.99792344,61.82651628)(685.90792725,61.82651611)
\curveto(685.81792362,61.82651628)(685.74792369,61.83151627)(685.69792725,61.84151611)
\lineto(685.53292725,61.84151611)
\curveto(685.47292396,61.86151624)(685.40792403,61.87151623)(685.33792725,61.87151611)
\curveto(685.26792417,61.86151624)(685.20792423,61.86651624)(685.15792725,61.88651611)
\curveto(685.10792433,61.89651621)(685.04292439,61.9015162)(684.96292725,61.90151611)
\lineto(684.72292725,61.96151611)
\curveto(684.65292478,61.97151613)(684.57792486,61.99151611)(684.49792725,62.02151611)
\curveto(684.18792525,62.12151598)(683.91792552,62.24651586)(683.68792725,62.39651611)
\curveto(683.45792598,62.54651556)(683.25792618,62.74151536)(683.08792725,62.98151611)
\curveto(682.99792644,63.11151499)(682.92292651,63.24651486)(682.86292725,63.38651611)
\curveto(682.80292663,63.52651458)(682.74792669,63.68151442)(682.69792725,63.85151611)
\curveto(682.67792676,63.91151419)(682.66792677,63.98151412)(682.66792725,64.06151611)
\curveto(682.67792676,64.15151395)(682.69292674,64.22151388)(682.71292725,64.27151611)
\curveto(682.74292669,64.31151379)(682.79292664,64.35151375)(682.86292725,64.39151611)
\curveto(682.91292652,64.41151369)(682.98292645,64.42151368)(683.07292725,64.42151611)
\curveto(683.16292627,64.43151367)(683.25292618,64.43151367)(683.34292725,64.42151611)
\curveto(683.432926,64.41151369)(683.51792592,64.39651371)(683.59792725,64.37651611)
\curveto(683.68792575,64.36651374)(683.74792569,64.35151375)(683.77792725,64.33151611)
\curveto(683.84792559,64.28151382)(683.89292554,64.2065139)(683.91292725,64.10651611)
\curveto(683.94292549,64.01651409)(683.97792546,63.93151417)(684.01792725,63.85151611)
\curveto(684.11792532,63.63151447)(684.25292518,63.46151464)(684.42292725,63.34151611)
\curveto(684.54292489,63.25151485)(684.67792476,63.18151492)(684.82792725,63.13151611)
\curveto(684.97792446,63.08151502)(685.1379243,63.03151507)(685.30792725,62.98151611)
\lineto(685.62292725,62.93651611)
\lineto(685.71292725,62.93651611)
\curveto(685.78292365,62.91651519)(685.87292356,62.9065152)(685.98292725,62.90651611)
\curveto(686.10292333,62.9065152)(686.20292323,62.91651519)(686.28292725,62.93651611)
\curveto(686.35292308,62.93651517)(686.40792303,62.94151516)(686.44792725,62.95151611)
\curveto(686.50792293,62.96151514)(686.56792287,62.96651514)(686.62792725,62.96651611)
\curveto(686.68792275,62.97651513)(686.74292269,62.98651512)(686.79292725,62.99651611)
\curveto(687.08292235,63.07651503)(687.31292212,63.18151492)(687.48292725,63.31151611)
\curveto(687.65292178,63.44151466)(687.77292166,63.66151444)(687.84292725,63.97151611)
\curveto(687.86292157,64.02151408)(687.86792157,64.07651403)(687.85792725,64.13651611)
\curveto(687.84792159,64.19651391)(687.8379216,64.24151386)(687.82792725,64.27151611)
\curveto(687.77792166,64.46151364)(687.70792173,64.6015135)(687.61792725,64.69151611)
\curveto(687.52792191,64.79151331)(687.41292202,64.88151322)(687.27292725,64.96151611)
\curveto(687.18292225,65.02151308)(687.08292235,65.07151303)(686.97292725,65.11151611)
\lineto(686.64292725,65.23151611)
\curveto(686.61292282,65.24151286)(686.58292285,65.24651286)(686.55292725,65.24651611)
\curveto(686.5329229,65.24651286)(686.50792293,65.25651285)(686.47792725,65.27651611)
\curveto(686.1379233,65.38651272)(685.78292365,65.46651264)(685.41292725,65.51651611)
\curveto(685.05292438,65.57651253)(684.71292472,65.67151243)(684.39292725,65.80151611)
\curveto(684.29292514,65.84151226)(684.19792524,65.87651223)(684.10792725,65.90651611)
\curveto(684.01792542,65.93651217)(683.9329255,65.97651213)(683.85292725,66.02651611)
\curveto(683.66292577,66.13651197)(683.48792595,66.26151184)(683.32792725,66.40151611)
\curveto(683.16792627,66.54151156)(683.04292639,66.71651139)(682.95292725,66.92651611)
\curveto(682.92292651,66.99651111)(682.89792654,67.06651104)(682.87792725,67.13651611)
\curveto(682.86792657,67.2065109)(682.85292658,67.28151082)(682.83292725,67.36151611)
\curveto(682.80292663,67.48151062)(682.79292664,67.61651049)(682.80292725,67.76651611)
\curveto(682.81292662,67.92651018)(682.82792661,68.06151004)(682.84792725,68.17151611)
\curveto(682.86792657,68.22150988)(682.87792656,68.26150984)(682.87792725,68.29151611)
\curveto(682.88792655,68.33150977)(682.90292653,68.37150973)(682.92292725,68.41151611)
\curveto(683.01292642,68.64150946)(683.1329263,68.84150926)(683.28292725,69.01151611)
\curveto(683.44292599,69.18150892)(683.62292581,69.33150877)(683.82292725,69.46151611)
\curveto(683.97292546,69.55150855)(684.1379253,69.62150848)(684.31792725,69.67151611)
\curveto(684.49792494,69.73150837)(684.68792475,69.78650832)(684.88792725,69.83651611)
\curveto(684.95792448,69.84650826)(685.02292441,69.85650825)(685.08292725,69.86651611)
\curveto(685.15292428,69.87650823)(685.22792421,69.88650822)(685.30792725,69.89651611)
\curveto(685.3379241,69.9065082)(685.37792406,69.9065082)(685.42792725,69.89651611)
\curveto(685.47792396,69.88650822)(685.51292392,69.89150821)(685.53292725,69.91151611)
}
}
{
\newrgbcolor{curcolor}{0 0 0}
\pscustom[linestyle=none,fillstyle=solid,fillcolor=curcolor]
{
\newpath
\moveto(691.54792725,72.07151611)
\curveto(691.69792524,72.07150603)(691.84792509,72.06650604)(691.99792725,72.05651611)
\curveto(692.14792479,72.05650605)(692.25292468,72.01650609)(692.31292725,71.93651611)
\curveto(692.36292457,71.87650623)(692.38792455,71.79150631)(692.38792725,71.68151611)
\curveto(692.39792454,71.58150652)(692.40292453,71.47650663)(692.40292725,71.36651611)
\lineto(692.40292725,70.49651611)
\curveto(692.40292453,70.41650769)(692.39792454,70.33150777)(692.38792725,70.24151611)
\curveto(692.38792455,70.16150794)(692.39792454,70.09150801)(692.41792725,70.03151611)
\curveto(692.45792448,69.89150821)(692.54792439,69.8015083)(692.68792725,69.76151611)
\curveto(692.7379242,69.75150835)(692.78292415,69.74650836)(692.82292725,69.74651611)
\lineto(692.97292725,69.74651611)
\lineto(693.37792725,69.74651611)
\curveto(693.5379234,69.75650835)(693.65292328,69.74650836)(693.72292725,69.71651611)
\curveto(693.81292312,69.65650845)(693.87292306,69.59650851)(693.90292725,69.53651611)
\curveto(693.92292301,69.49650861)(693.932923,69.45150865)(693.93292725,69.40151611)
\lineto(693.93292725,69.25151611)
\curveto(693.932923,69.14150896)(693.92792301,69.03650907)(693.91792725,68.93651611)
\curveto(693.90792303,68.84650926)(693.87292306,68.77650933)(693.81292725,68.72651611)
\curveto(693.75292318,68.67650943)(693.66792327,68.64650946)(693.55792725,68.63651611)
\lineto(693.22792725,68.63651611)
\curveto(693.11792382,68.64650946)(693.00792393,68.65150945)(692.89792725,68.65151611)
\curveto(692.78792415,68.65150945)(692.69292424,68.63650947)(692.61292725,68.60651611)
\curveto(692.54292439,68.57650953)(692.49292444,68.52650958)(692.46292725,68.45651611)
\curveto(692.4329245,68.38650972)(692.41292452,68.3015098)(692.40292725,68.20151611)
\curveto(692.39292454,68.11150999)(692.38792455,68.01151009)(692.38792725,67.90151611)
\curveto(692.39792454,67.8015103)(692.40292453,67.7015104)(692.40292725,67.60151611)
\lineto(692.40292725,64.63151611)
\curveto(692.40292453,64.41151369)(692.39792454,64.17651393)(692.38792725,63.92651611)
\curveto(692.38792455,63.68651442)(692.4329245,63.5015146)(692.52292725,63.37151611)
\curveto(692.57292436,63.29151481)(692.6379243,63.23651487)(692.71792725,63.20651611)
\curveto(692.79792414,63.17651493)(692.89292404,63.15151495)(693.00292725,63.13151611)
\curveto(693.0329239,63.12151498)(693.06292387,63.11651499)(693.09292725,63.11651611)
\curveto(693.1329238,63.12651498)(693.16792377,63.12651498)(693.19792725,63.11651611)
\lineto(693.39292725,63.11651611)
\curveto(693.49292344,63.11651499)(693.58292335,63.106515)(693.66292725,63.08651611)
\curveto(693.75292318,63.07651503)(693.81792312,63.04151506)(693.85792725,62.98151611)
\curveto(693.87792306,62.95151515)(693.89292304,62.89651521)(693.90292725,62.81651611)
\curveto(693.92292301,62.74651536)(693.932923,62.67151543)(693.93292725,62.59151611)
\curveto(693.94292299,62.51151559)(693.94292299,62.43151567)(693.93292725,62.35151611)
\curveto(693.92292301,62.28151582)(693.90292303,62.22651588)(693.87292725,62.18651611)
\curveto(693.8329231,62.11651599)(693.75792318,62.06651604)(693.64792725,62.03651611)
\curveto(693.56792337,62.01651609)(693.47792346,62.0065161)(693.37792725,62.00651611)
\curveto(693.27792366,62.01651609)(693.18792375,62.02151608)(693.10792725,62.02151611)
\curveto(693.04792389,62.02151608)(692.98792395,62.01651609)(692.92792725,62.00651611)
\curveto(692.86792407,62.0065161)(692.81292412,62.01151609)(692.76292725,62.02151611)
\lineto(692.58292725,62.02151611)
\curveto(692.5329244,62.03151607)(692.48292445,62.03651607)(692.43292725,62.03651611)
\curveto(692.39292454,62.04651606)(692.34792459,62.05151605)(692.29792725,62.05151611)
\curveto(692.09792484,62.101516)(691.92292501,62.15651595)(691.77292725,62.21651611)
\curveto(691.6329253,62.27651583)(691.51292542,62.38151572)(691.41292725,62.53151611)
\curveto(691.27292566,62.73151537)(691.19292574,62.98151512)(691.17292725,63.28151611)
\curveto(691.15292578,63.59151451)(691.14292579,63.92151418)(691.14292725,64.27151611)
\lineto(691.14292725,68.20151611)
\curveto(691.11292582,68.33150977)(691.08292585,68.42650968)(691.05292725,68.48651611)
\curveto(691.0329259,68.54650956)(690.96292597,68.59650951)(690.84292725,68.63651611)
\curveto(690.80292613,68.64650946)(690.76292617,68.64650946)(690.72292725,68.63651611)
\curveto(690.68292625,68.62650948)(690.64292629,68.63150947)(690.60292725,68.65151611)
\lineto(690.36292725,68.65151611)
\curveto(690.2329267,68.65150945)(690.12292681,68.66150944)(690.03292725,68.68151611)
\curveto(689.95292698,68.71150939)(689.89792704,68.77150933)(689.86792725,68.86151611)
\curveto(689.84792709,68.9015092)(689.8329271,68.94650916)(689.82292725,68.99651611)
\lineto(689.82292725,69.14651611)
\curveto(689.82292711,69.28650882)(689.8329271,69.4015087)(689.85292725,69.49151611)
\curveto(689.87292706,69.59150851)(689.932927,69.66650844)(690.03292725,69.71651611)
\curveto(690.14292679,69.75650835)(690.28292665,69.76650834)(690.45292725,69.74651611)
\curveto(690.6329263,69.72650838)(690.78292615,69.73650837)(690.90292725,69.77651611)
\curveto(690.99292594,69.82650828)(691.06292587,69.89650821)(691.11292725,69.98651611)
\curveto(691.1329258,70.04650806)(691.14292579,70.12150798)(691.14292725,70.21151611)
\lineto(691.14292725,70.46651611)
\lineto(691.14292725,71.39651611)
\lineto(691.14292725,71.63651611)
\curveto(691.14292579,71.72650638)(691.15292578,71.8015063)(691.17292725,71.86151611)
\curveto(691.21292572,71.94150616)(691.28792565,72.0065061)(691.39792725,72.05651611)
\curveto(691.42792551,72.05650605)(691.45292548,72.05650605)(691.47292725,72.05651611)
\curveto(691.50292543,72.06650604)(691.52792541,72.07150603)(691.54792725,72.07151611)
}
}
{
\newrgbcolor{curcolor}{0 0 0}
\pscustom[linestyle=none,fillstyle=solid,fillcolor=curcolor]
{
\newpath
\moveto(698.96472412,69.91151611)
\curveto(699.19471933,69.91150819)(699.3247192,69.85150825)(699.35472412,69.73151611)
\curveto(699.38471914,69.62150848)(699.39971913,69.45650865)(699.39972412,69.23651611)
\lineto(699.39972412,68.95151611)
\curveto(699.39971913,68.86150924)(699.37471915,68.78650932)(699.32472412,68.72651611)
\curveto(699.26471926,68.64650946)(699.17971935,68.6015095)(699.06972412,68.59151611)
\curveto(698.95971957,68.59150951)(698.84971968,68.57650953)(698.73972412,68.54651611)
\curveto(698.59971993,68.51650959)(698.46472006,68.48650962)(698.33472412,68.45651611)
\curveto(698.21472031,68.42650968)(698.09972043,68.38650972)(697.98972412,68.33651611)
\curveto(697.69972083,68.2065099)(697.46472106,68.02651008)(697.28472412,67.79651611)
\curveto(697.10472142,67.57651053)(696.94972158,67.32151078)(696.81972412,67.03151611)
\curveto(696.77972175,66.92151118)(696.74972178,66.8065113)(696.72972412,66.68651611)
\curveto(696.70972182,66.57651153)(696.68472184,66.46151164)(696.65472412,66.34151611)
\curveto(696.64472188,66.29151181)(696.63972189,66.24151186)(696.63972412,66.19151611)
\curveto(696.64972188,66.14151196)(696.64972188,66.09151201)(696.63972412,66.04151611)
\curveto(696.60972192,65.92151218)(696.59472193,65.78151232)(696.59472412,65.62151611)
\curveto(696.60472192,65.47151263)(696.60972192,65.32651278)(696.60972412,65.18651611)
\lineto(696.60972412,63.34151611)
\lineto(696.60972412,62.99651611)
\curveto(696.60972192,62.87651523)(696.60472192,62.76151534)(696.59472412,62.65151611)
\curveto(696.58472194,62.54151556)(696.57972195,62.44651566)(696.57972412,62.36651611)
\curveto(696.58972194,62.28651582)(696.56972196,62.21651589)(696.51972412,62.15651611)
\curveto(696.46972206,62.08651602)(696.38972214,62.04651606)(696.27972412,62.03651611)
\curveto(696.17972235,62.02651608)(696.06972246,62.02151608)(695.94972412,62.02151611)
\lineto(695.67972412,62.02151611)
\curveto(695.6297229,62.04151606)(695.57972295,62.05651605)(695.52972412,62.06651611)
\curveto(695.48972304,62.08651602)(695.45972307,62.11151599)(695.43972412,62.14151611)
\curveto(695.38972314,62.21151589)(695.35972317,62.29651581)(695.34972412,62.39651611)
\lineto(695.34972412,62.72651611)
\lineto(695.34972412,63.88151611)
\lineto(695.34972412,68.03651611)
\lineto(695.34972412,69.07151611)
\lineto(695.34972412,69.37151611)
\curveto(695.35972317,69.47150863)(695.38972314,69.55650855)(695.43972412,69.62651611)
\curveto(695.46972306,69.66650844)(695.51972301,69.69650841)(695.58972412,69.71651611)
\curveto(695.66972286,69.73650837)(695.75472277,69.74650836)(695.84472412,69.74651611)
\curveto(695.93472259,69.75650835)(696.0247225,69.75650835)(696.11472412,69.74651611)
\curveto(696.20472232,69.73650837)(696.27472225,69.72150838)(696.32472412,69.70151611)
\curveto(696.40472212,69.67150843)(696.45472207,69.61150849)(696.47472412,69.52151611)
\curveto(696.50472202,69.44150866)(696.51972201,69.35150875)(696.51972412,69.25151611)
\lineto(696.51972412,68.95151611)
\curveto(696.51972201,68.85150925)(696.53972199,68.76150934)(696.57972412,68.68151611)
\curveto(696.58972194,68.66150944)(696.59972193,68.64650946)(696.60972412,68.63651611)
\lineto(696.65472412,68.59151611)
\curveto(696.76472176,68.59150951)(696.85472167,68.63650947)(696.92472412,68.72651611)
\curveto(696.99472153,68.82650928)(697.05472147,68.9065092)(697.10472412,68.96651611)
\lineto(697.19472412,69.05651611)
\curveto(697.28472124,69.16650894)(697.40972112,69.28150882)(697.56972412,69.40151611)
\curveto(697.7297208,69.52150858)(697.87972065,69.61150849)(698.01972412,69.67151611)
\curveto(698.10972042,69.72150838)(698.20472032,69.75650835)(698.30472412,69.77651611)
\curveto(698.40472012,69.8065083)(698.50972002,69.83650827)(698.61972412,69.86651611)
\curveto(698.67971985,69.87650823)(698.73971979,69.88150822)(698.79972412,69.88151611)
\curveto(698.85971967,69.89150821)(698.91471961,69.9015082)(698.96472412,69.91151611)
}
}
{
\newrgbcolor{curcolor}{0 0 0}
\pscustom[linestyle=none,fillstyle=solid,fillcolor=curcolor]
{
\newpath
\moveto(707.21448975,62.56151611)
\curveto(707.24448192,62.4015157)(707.22948193,62.26651584)(707.16948975,62.15651611)
\curveto(707.10948205,62.05651605)(707.02948213,61.98151612)(706.92948975,61.93151611)
\curveto(706.87948228,61.91151619)(706.82448234,61.9015162)(706.76448975,61.90151611)
\curveto(706.71448245,61.9015162)(706.6594825,61.89151621)(706.59948975,61.87151611)
\curveto(706.37948278,61.82151628)(706.159483,61.83651627)(705.93948975,61.91651611)
\curveto(705.72948343,61.98651612)(705.58448358,62.07651603)(705.50448975,62.18651611)
\curveto(705.45448371,62.25651585)(705.40948375,62.33651577)(705.36948975,62.42651611)
\curveto(705.32948383,62.52651558)(705.27948388,62.6065155)(705.21948975,62.66651611)
\curveto(705.19948396,62.68651542)(705.17448399,62.7065154)(705.14448975,62.72651611)
\curveto(705.12448404,62.74651536)(705.09448407,62.75151535)(705.05448975,62.74151611)
\curveto(704.94448422,62.71151539)(704.83948432,62.65651545)(704.73948975,62.57651611)
\curveto(704.64948451,62.49651561)(704.5594846,62.42651568)(704.46948975,62.36651611)
\curveto(704.33948482,62.28651582)(704.19948496,62.21151589)(704.04948975,62.14151611)
\curveto(703.89948526,62.08151602)(703.73948542,62.02651608)(703.56948975,61.97651611)
\curveto(703.46948569,61.94651616)(703.3594858,61.92651618)(703.23948975,61.91651611)
\curveto(703.12948603,61.9065162)(703.01948614,61.89151621)(702.90948975,61.87151611)
\curveto(702.8594863,61.86151624)(702.81448635,61.85651625)(702.77448975,61.85651611)
\lineto(702.66948975,61.85651611)
\curveto(702.5594866,61.83651627)(702.45448671,61.83651627)(702.35448975,61.85651611)
\lineto(702.21948975,61.85651611)
\curveto(702.16948699,61.86651624)(702.11948704,61.87151623)(702.06948975,61.87151611)
\curveto(702.01948714,61.87151623)(701.97448719,61.88151622)(701.93448975,61.90151611)
\curveto(701.89448727,61.91151619)(701.8594873,61.91651619)(701.82948975,61.91651611)
\curveto(701.80948735,61.9065162)(701.78448738,61.9065162)(701.75448975,61.91651611)
\lineto(701.51448975,61.97651611)
\curveto(701.43448773,61.98651612)(701.3594878,62.0065161)(701.28948975,62.03651611)
\curveto(700.98948817,62.16651594)(700.74448842,62.31151579)(700.55448975,62.47151611)
\curveto(700.37448879,62.64151546)(700.22448894,62.87651523)(700.10448975,63.17651611)
\curveto(700.01448915,63.39651471)(699.96948919,63.66151444)(699.96948975,63.97151611)
\lineto(699.96948975,64.28651611)
\curveto(699.97948918,64.33651377)(699.98448918,64.38651372)(699.98448975,64.43651611)
\lineto(700.01448975,64.61651611)
\lineto(700.13448975,64.94651611)
\curveto(700.17448899,65.05651305)(700.22448894,65.15651295)(700.28448975,65.24651611)
\curveto(700.4644887,65.53651257)(700.70948845,65.75151235)(701.01948975,65.89151611)
\curveto(701.32948783,66.03151207)(701.66948749,66.15651195)(702.03948975,66.26651611)
\curveto(702.17948698,66.3065118)(702.32448684,66.33651177)(702.47448975,66.35651611)
\curveto(702.62448654,66.37651173)(702.77448639,66.4015117)(702.92448975,66.43151611)
\curveto(702.99448617,66.45151165)(703.0594861,66.46151164)(703.11948975,66.46151611)
\curveto(703.18948597,66.46151164)(703.2644859,66.47151163)(703.34448975,66.49151611)
\curveto(703.41448575,66.51151159)(703.48448568,66.52151158)(703.55448975,66.52151611)
\curveto(703.62448554,66.53151157)(703.69948546,66.54651156)(703.77948975,66.56651611)
\curveto(704.02948513,66.62651148)(704.2644849,66.67651143)(704.48448975,66.71651611)
\curveto(704.70448446,66.76651134)(704.87948428,66.88151122)(705.00948975,67.06151611)
\curveto(705.06948409,67.14151096)(705.11948404,67.24151086)(705.15948975,67.36151611)
\curveto(705.19948396,67.49151061)(705.19948396,67.63151047)(705.15948975,67.78151611)
\curveto(705.09948406,68.02151008)(705.00948415,68.21150989)(704.88948975,68.35151611)
\curveto(704.77948438,68.49150961)(704.61948454,68.6015095)(704.40948975,68.68151611)
\curveto(704.28948487,68.73150937)(704.14448502,68.76650934)(703.97448975,68.78651611)
\curveto(703.81448535,68.8065093)(703.64448552,68.81650929)(703.46448975,68.81651611)
\curveto(703.28448588,68.81650929)(703.10948605,68.8065093)(702.93948975,68.78651611)
\curveto(702.76948639,68.76650934)(702.62448654,68.73650937)(702.50448975,68.69651611)
\curveto(702.33448683,68.63650947)(702.16948699,68.55150955)(702.00948975,68.44151611)
\curveto(701.92948723,68.38150972)(701.85448731,68.3015098)(701.78448975,68.20151611)
\curveto(701.72448744,68.11150999)(701.66948749,68.01151009)(701.61948975,67.90151611)
\curveto(701.58948757,67.82151028)(701.5594876,67.73651037)(701.52948975,67.64651611)
\curveto(701.50948765,67.55651055)(701.4644877,67.48651062)(701.39448975,67.43651611)
\curveto(701.35448781,67.4065107)(701.28448788,67.38151072)(701.18448975,67.36151611)
\curveto(701.09448807,67.35151075)(700.99948816,67.34651076)(700.89948975,67.34651611)
\curveto(700.79948836,67.34651076)(700.69948846,67.35151075)(700.59948975,67.36151611)
\curveto(700.50948865,67.38151072)(700.44448872,67.4065107)(700.40448975,67.43651611)
\curveto(700.3644888,67.46651064)(700.33448883,67.51651059)(700.31448975,67.58651611)
\curveto(700.29448887,67.65651045)(700.29448887,67.73151037)(700.31448975,67.81151611)
\curveto(700.34448882,67.94151016)(700.37448879,68.06151004)(700.40448975,68.17151611)
\curveto(700.44448872,68.29150981)(700.48948867,68.4065097)(700.53948975,68.51651611)
\curveto(700.72948843,68.86650924)(700.96948819,69.13650897)(701.25948975,69.32651611)
\curveto(701.54948761,69.52650858)(701.90948725,69.68650842)(702.33948975,69.80651611)
\curveto(702.43948672,69.82650828)(702.53948662,69.84150826)(702.63948975,69.85151611)
\curveto(702.74948641,69.86150824)(702.8594863,69.87650823)(702.96948975,69.89651611)
\curveto(703.00948615,69.9065082)(703.07448609,69.9065082)(703.16448975,69.89651611)
\curveto(703.25448591,69.89650821)(703.30948585,69.9065082)(703.32948975,69.92651611)
\curveto(704.02948513,69.93650817)(704.63948452,69.85650825)(705.15948975,69.68651611)
\curveto(705.67948348,69.51650859)(706.04448312,69.19150891)(706.25448975,68.71151611)
\curveto(706.34448282,68.51150959)(706.39448277,68.27650983)(706.40448975,68.00651611)
\curveto(706.42448274,67.74651036)(706.43448273,67.47151063)(706.43448975,67.18151611)
\lineto(706.43448975,63.86651611)
\curveto(706.43448273,63.72651438)(706.43948272,63.59151451)(706.44948975,63.46151611)
\curveto(706.4594827,63.33151477)(706.48948267,63.22651488)(706.53948975,63.14651611)
\curveto(706.58948257,63.07651503)(706.65448251,63.02651508)(706.73448975,62.99651611)
\curveto(706.82448234,62.95651515)(706.90948225,62.92651518)(706.98948975,62.90651611)
\curveto(707.06948209,62.89651521)(707.12948203,62.85151525)(707.16948975,62.77151611)
\curveto(707.18948197,62.74151536)(707.19948196,62.71151539)(707.19948975,62.68151611)
\curveto(707.19948196,62.65151545)(707.20448196,62.61151549)(707.21448975,62.56151611)
\moveto(705.06948975,64.22651611)
\curveto(705.12948403,64.36651374)(705.159484,64.52651358)(705.15948975,64.70651611)
\curveto(705.16948399,64.89651321)(705.17448399,65.09151301)(705.17448975,65.29151611)
\curveto(705.17448399,65.4015127)(705.16948399,65.5015126)(705.15948975,65.59151611)
\curveto(705.14948401,65.68151242)(705.10948405,65.75151235)(705.03948975,65.80151611)
\curveto(705.00948415,65.82151228)(704.93948422,65.83151227)(704.82948975,65.83151611)
\curveto(704.80948435,65.81151229)(704.77448439,65.8015123)(704.72448975,65.80151611)
\curveto(704.67448449,65.8015123)(704.62948453,65.79151231)(704.58948975,65.77151611)
\curveto(704.50948465,65.75151235)(704.41948474,65.73151237)(704.31948975,65.71151611)
\lineto(704.01948975,65.65151611)
\curveto(703.98948517,65.65151245)(703.95448521,65.64651246)(703.91448975,65.63651611)
\lineto(703.80948975,65.63651611)
\curveto(703.6594855,65.59651251)(703.49448567,65.57151253)(703.31448975,65.56151611)
\curveto(703.14448602,65.56151254)(702.98448618,65.54151256)(702.83448975,65.50151611)
\curveto(702.75448641,65.48151262)(702.67948648,65.46151264)(702.60948975,65.44151611)
\curveto(702.54948661,65.43151267)(702.47948668,65.41651269)(702.39948975,65.39651611)
\curveto(702.23948692,65.34651276)(702.08948707,65.28151282)(701.94948975,65.20151611)
\curveto(701.80948735,65.13151297)(701.68948747,65.04151306)(701.58948975,64.93151611)
\curveto(701.48948767,64.82151328)(701.41448775,64.68651342)(701.36448975,64.52651611)
\curveto(701.31448785,64.37651373)(701.29448787,64.19151391)(701.30448975,63.97151611)
\curveto(701.30448786,63.87151423)(701.31948784,63.77651433)(701.34948975,63.68651611)
\curveto(701.38948777,63.6065145)(701.43448773,63.53151457)(701.48448975,63.46151611)
\curveto(701.5644876,63.35151475)(701.66948749,63.25651485)(701.79948975,63.17651611)
\curveto(701.92948723,63.106515)(702.06948709,63.04651506)(702.21948975,62.99651611)
\curveto(702.26948689,62.98651512)(702.31948684,62.98151512)(702.36948975,62.98151611)
\curveto(702.41948674,62.98151512)(702.46948669,62.97651513)(702.51948975,62.96651611)
\curveto(702.58948657,62.94651516)(702.67448649,62.93151517)(702.77448975,62.92151611)
\curveto(702.88448628,62.92151518)(702.97448619,62.93151517)(703.04448975,62.95151611)
\curveto(703.10448606,62.97151513)(703.164486,62.97651513)(703.22448975,62.96651611)
\curveto(703.28448588,62.96651514)(703.34448582,62.97651513)(703.40448975,62.99651611)
\curveto(703.48448568,63.01651509)(703.5594856,63.03151507)(703.62948975,63.04151611)
\curveto(703.70948545,63.05151505)(703.78448538,63.07151503)(703.85448975,63.10151611)
\curveto(704.14448502,63.22151488)(704.38948477,63.36651474)(704.58948975,63.53651611)
\curveto(704.79948436,63.7065144)(704.9594842,63.93651417)(705.06948975,64.22651611)
}
}
{
\newrgbcolor{curcolor}{0 0 0}
\pscustom[linestyle=none,fillstyle=solid,fillcolor=curcolor]
{
\newpath
\moveto(715.34613037,62.81651611)
\lineto(715.34613037,62.42651611)
\curveto(715.3461225,62.3065158)(715.32112252,62.2065159)(715.27113037,62.12651611)
\curveto(715.22112262,62.05651605)(715.13612271,62.01651609)(715.01613037,62.00651611)
\lineto(714.67113037,62.00651611)
\curveto(714.61112323,62.0065161)(714.55112329,62.0015161)(714.49113037,61.99151611)
\curveto(714.4411234,61.99151611)(714.39612345,62.0015161)(714.35613037,62.02151611)
\curveto(714.26612358,62.04151606)(714.20612364,62.08151602)(714.17613037,62.14151611)
\curveto(714.13612371,62.19151591)(714.11112373,62.25151585)(714.10113037,62.32151611)
\curveto(714.10112374,62.39151571)(714.08612376,62.46151564)(714.05613037,62.53151611)
\curveto(714.0461238,62.55151555)(714.03112381,62.56651554)(714.01113037,62.57651611)
\curveto(714.00112384,62.59651551)(713.98612386,62.61651549)(713.96613037,62.63651611)
\curveto(713.86612398,62.64651546)(713.78612406,62.62651548)(713.72613037,62.57651611)
\curveto(713.67612417,62.52651558)(713.62112422,62.47651563)(713.56113037,62.42651611)
\curveto(713.36112448,62.27651583)(713.16112468,62.16151594)(712.96113037,62.08151611)
\curveto(712.78112506,62.0015161)(712.57112527,61.94151616)(712.33113037,61.90151611)
\curveto(712.10112574,61.86151624)(711.86112598,61.84151626)(711.61113037,61.84151611)
\curveto(711.37112647,61.83151627)(711.13112671,61.84651626)(710.89113037,61.88651611)
\curveto(710.65112719,61.91651619)(710.4411274,61.97151613)(710.26113037,62.05151611)
\curveto(709.7411281,62.27151583)(709.32112852,62.56651554)(709.00113037,62.93651611)
\curveto(708.68112916,63.31651479)(708.43112941,63.78651432)(708.25113037,64.34651611)
\curveto(708.21112963,64.43651367)(708.18112966,64.52651358)(708.16113037,64.61651611)
\curveto(708.15112969,64.71651339)(708.13112971,64.81651329)(708.10113037,64.91651611)
\curveto(708.09112975,64.96651314)(708.08612976,65.01651309)(708.08613037,65.06651611)
\curveto(708.08612976,65.11651299)(708.08112976,65.16651294)(708.07113037,65.21651611)
\curveto(708.05112979,65.26651284)(708.0411298,65.31651279)(708.04113037,65.36651611)
\curveto(708.05112979,65.42651268)(708.05112979,65.48151262)(708.04113037,65.53151611)
\lineto(708.04113037,65.68151611)
\curveto(708.02112982,65.73151237)(708.01112983,65.79651231)(708.01113037,65.87651611)
\curveto(708.01112983,65.95651215)(708.02112982,66.02151208)(708.04113037,66.07151611)
\lineto(708.04113037,66.23651611)
\curveto(708.06112978,66.3065118)(708.06612978,66.37651173)(708.05613037,66.44651611)
\curveto(708.05612979,66.52651158)(708.06612978,66.6015115)(708.08613037,66.67151611)
\curveto(708.09612975,66.72151138)(708.10112974,66.76651134)(708.10113037,66.80651611)
\curveto(708.10112974,66.84651126)(708.10612974,66.89151121)(708.11613037,66.94151611)
\curveto(708.1461297,67.04151106)(708.17112967,67.13651097)(708.19113037,67.22651611)
\curveto(708.21112963,67.32651078)(708.23612961,67.42151068)(708.26613037,67.51151611)
\curveto(708.39612945,67.89151021)(708.56112928,68.23150987)(708.76113037,68.53151611)
\curveto(708.97112887,68.84150926)(709.22112862,69.09650901)(709.51113037,69.29651611)
\curveto(709.68112816,69.41650869)(709.85612799,69.51650859)(710.03613037,69.59651611)
\curveto(710.22612762,69.67650843)(710.43112741,69.74650836)(710.65113037,69.80651611)
\curveto(710.72112712,69.81650829)(710.78612706,69.82650828)(710.84613037,69.83651611)
\curveto(710.91612693,69.84650826)(710.98612686,69.86150824)(711.05613037,69.88151611)
\lineto(711.20613037,69.88151611)
\curveto(711.28612656,69.9015082)(711.40112644,69.91150819)(711.55113037,69.91151611)
\curveto(711.71112613,69.91150819)(711.83112601,69.9015082)(711.91113037,69.88151611)
\curveto(711.95112589,69.87150823)(712.00612584,69.86650824)(712.07613037,69.86651611)
\curveto(712.18612566,69.83650827)(712.29612555,69.81150829)(712.40613037,69.79151611)
\curveto(712.51612533,69.78150832)(712.62112522,69.75150835)(712.72113037,69.70151611)
\curveto(712.87112497,69.64150846)(713.01112483,69.57650853)(713.14113037,69.50651611)
\curveto(713.28112456,69.43650867)(713.41112443,69.35650875)(713.53113037,69.26651611)
\curveto(713.59112425,69.21650889)(713.65112419,69.16150894)(713.71113037,69.10151611)
\curveto(713.78112406,69.05150905)(713.87112397,69.03650907)(713.98113037,69.05651611)
\curveto(714.00112384,69.08650902)(714.01612383,69.11150899)(714.02613037,69.13151611)
\curveto(714.0461238,69.15150895)(714.06112378,69.18150892)(714.07113037,69.22151611)
\curveto(714.10112374,69.31150879)(714.11112373,69.42650868)(714.10113037,69.56651611)
\lineto(714.10113037,69.94151611)
\lineto(714.10113037,71.66651611)
\lineto(714.10113037,72.13151611)
\curveto(714.10112374,72.31150579)(714.12612372,72.44150566)(714.17613037,72.52151611)
\curveto(714.21612363,72.59150551)(714.27612357,72.63650547)(714.35613037,72.65651611)
\curveto(714.37612347,72.65650545)(714.40112344,72.65650545)(714.43113037,72.65651611)
\curveto(714.46112338,72.66650544)(714.48612336,72.67150543)(714.50613037,72.67151611)
\curveto(714.6461232,72.68150542)(714.79112305,72.68150542)(714.94113037,72.67151611)
\curveto(715.10112274,72.67150543)(715.21112263,72.63150547)(715.27113037,72.55151611)
\curveto(715.32112252,72.47150563)(715.3461225,72.37150573)(715.34613037,72.25151611)
\lineto(715.34613037,71.87651611)
\lineto(715.34613037,62.81651611)
\moveto(714.13113037,65.65151611)
\curveto(714.15112369,65.7015124)(714.16112368,65.76651234)(714.16113037,65.84651611)
\curveto(714.16112368,65.93651217)(714.15112369,66.0065121)(714.13113037,66.05651611)
\lineto(714.13113037,66.28151611)
\curveto(714.11112373,66.37151173)(714.09612375,66.46151164)(714.08613037,66.55151611)
\curveto(714.07612377,66.65151145)(714.05612379,66.74151136)(714.02613037,66.82151611)
\curveto(714.00612384,66.9015112)(713.98612386,66.97651113)(713.96613037,67.04651611)
\curveto(713.95612389,67.11651099)(713.93612391,67.18651092)(713.90613037,67.25651611)
\curveto(713.78612406,67.55651055)(713.63112421,67.82151028)(713.44113037,68.05151611)
\curveto(713.25112459,68.28150982)(713.01112483,68.46150964)(712.72113037,68.59151611)
\curveto(712.62112522,68.64150946)(712.51612533,68.67650943)(712.40613037,68.69651611)
\curveto(712.30612554,68.72650938)(712.19612565,68.75150935)(712.07613037,68.77151611)
\curveto(711.99612585,68.79150931)(711.90612594,68.8015093)(711.80613037,68.80151611)
\lineto(711.53613037,68.80151611)
\curveto(711.48612636,68.79150931)(711.4411264,68.78150932)(711.40113037,68.77151611)
\lineto(711.26613037,68.77151611)
\curveto(711.18612666,68.75150935)(711.10112674,68.73150937)(711.01113037,68.71151611)
\curveto(710.93112691,68.69150941)(710.85112699,68.66650944)(710.77113037,68.63651611)
\curveto(710.45112739,68.49650961)(710.19112765,68.29150981)(709.99113037,68.02151611)
\curveto(709.80112804,67.76151034)(709.6461282,67.45651065)(709.52613037,67.10651611)
\curveto(709.48612836,66.99651111)(709.45612839,66.88151122)(709.43613037,66.76151611)
\curveto(709.42612842,66.65151145)(709.41112843,66.54151156)(709.39113037,66.43151611)
\curveto(709.39112845,66.39151171)(709.38612846,66.35151175)(709.37613037,66.31151611)
\lineto(709.37613037,66.20651611)
\curveto(709.35612849,66.15651195)(709.3461285,66.101512)(709.34613037,66.04151611)
\curveto(709.35612849,65.98151212)(709.36112848,65.92651218)(709.36113037,65.87651611)
\lineto(709.36113037,65.54651611)
\curveto(709.36112848,65.44651266)(709.37112847,65.35151275)(709.39113037,65.26151611)
\curveto(709.40112844,65.23151287)(709.40612844,65.18151292)(709.40613037,65.11151611)
\curveto(709.42612842,65.04151306)(709.4411284,64.97151313)(709.45113037,64.90151611)
\lineto(709.51113037,64.69151611)
\curveto(709.62112822,64.34151376)(709.77112807,64.04151406)(709.96113037,63.79151611)
\curveto(710.15112769,63.54151456)(710.39112745,63.33651477)(710.68113037,63.17651611)
\curveto(710.77112707,63.12651498)(710.86112698,63.08651502)(710.95113037,63.05651611)
\curveto(711.0411268,63.02651508)(711.1411267,62.99651511)(711.25113037,62.96651611)
\curveto(711.30112654,62.94651516)(711.35112649,62.94151516)(711.40113037,62.95151611)
\curveto(711.46112638,62.96151514)(711.51612633,62.95651515)(711.56613037,62.93651611)
\curveto(711.60612624,62.92651518)(711.6461262,62.92151518)(711.68613037,62.92151611)
\lineto(711.82113037,62.92151611)
\lineto(711.95613037,62.92151611)
\curveto(711.98612586,62.93151517)(712.03612581,62.93651517)(712.10613037,62.93651611)
\curveto(712.18612566,62.95651515)(712.26612558,62.97151513)(712.34613037,62.98151611)
\curveto(712.42612542,63.0015151)(712.50112534,63.02651508)(712.57113037,63.05651611)
\curveto(712.90112494,63.19651491)(713.16612468,63.37151473)(713.36613037,63.58151611)
\curveto(713.57612427,63.8015143)(713.75112409,64.07651403)(713.89113037,64.40651611)
\curveto(713.9411239,64.51651359)(713.97612387,64.62651348)(713.99613037,64.73651611)
\curveto(714.01612383,64.84651326)(714.0411238,64.95651315)(714.07113037,65.06651611)
\curveto(714.09112375,65.106513)(714.10112374,65.14151296)(714.10113037,65.17151611)
\curveto(714.10112374,65.21151289)(714.10612374,65.25151285)(714.11613037,65.29151611)
\curveto(714.12612372,65.35151275)(714.12612372,65.41151269)(714.11613037,65.47151611)
\curveto(714.11612373,65.53151257)(714.12112372,65.59151251)(714.13113037,65.65151611)
}
}
{
\newrgbcolor{curcolor}{0 0 0}
\pscustom[linestyle=none,fillstyle=solid,fillcolor=curcolor]
{
\newpath
\moveto(724.41738037,66.20651611)
\curveto(724.43737231,66.14651196)(724.4473723,66.05151205)(724.44738037,65.92151611)
\curveto(724.4473723,65.8015123)(724.44237231,65.71651239)(724.43238037,65.66651611)
\lineto(724.43238037,65.51651611)
\curveto(724.42237233,65.43651267)(724.41237234,65.36151274)(724.40238037,65.29151611)
\curveto(724.40237235,65.23151287)(724.39737235,65.16151294)(724.38738037,65.08151611)
\curveto(724.36737238,65.02151308)(724.3523724,64.96151314)(724.34238037,64.90151611)
\curveto(724.34237241,64.84151326)(724.33237242,64.78151332)(724.31238037,64.72151611)
\curveto(724.27237248,64.59151351)(724.23737251,64.46151364)(724.20738037,64.33151611)
\curveto(724.17737257,64.2015139)(724.13737261,64.08151402)(724.08738037,63.97151611)
\curveto(723.87737287,63.49151461)(723.59737315,63.08651502)(723.24738037,62.75651611)
\curveto(722.89737385,62.43651567)(722.46737428,62.19151591)(721.95738037,62.02151611)
\curveto(721.8473749,61.98151612)(721.72737502,61.95151615)(721.59738037,61.93151611)
\curveto(721.47737527,61.91151619)(721.3523754,61.89151621)(721.22238037,61.87151611)
\curveto(721.16237559,61.86151624)(721.09737565,61.85651625)(721.02738037,61.85651611)
\curveto(720.96737578,61.84651626)(720.90737584,61.84151626)(720.84738037,61.84151611)
\curveto(720.80737594,61.83151627)(720.747376,61.82651628)(720.66738037,61.82651611)
\curveto(720.59737615,61.82651628)(720.5473762,61.83151627)(720.51738037,61.84151611)
\curveto(720.47737627,61.85151625)(720.43737631,61.85651625)(720.39738037,61.85651611)
\curveto(720.35737639,61.84651626)(720.32237643,61.84651626)(720.29238037,61.85651611)
\lineto(720.20238037,61.85651611)
\lineto(719.84238037,61.90151611)
\curveto(719.70237705,61.94151616)(719.56737718,61.98151612)(719.43738037,62.02151611)
\curveto(719.30737744,62.06151604)(719.18237757,62.106516)(719.06238037,62.15651611)
\curveto(718.61237814,62.35651575)(718.24237851,62.61651549)(717.95238037,62.93651611)
\curveto(717.66237909,63.25651485)(717.42237933,63.64651446)(717.23238037,64.10651611)
\curveto(717.18237957,64.2065139)(717.14237961,64.3065138)(717.11238037,64.40651611)
\curveto(717.09237966,64.5065136)(717.07237968,64.61151349)(717.05238037,64.72151611)
\curveto(717.03237972,64.76151334)(717.02237973,64.79151331)(717.02238037,64.81151611)
\curveto(717.03237972,64.84151326)(717.03237972,64.87651323)(717.02238037,64.91651611)
\curveto(717.00237975,64.99651311)(716.98737976,65.07651303)(716.97738037,65.15651611)
\curveto(716.97737977,65.24651286)(716.96737978,65.33151277)(716.94738037,65.41151611)
\lineto(716.94738037,65.53151611)
\curveto(716.9473798,65.57151253)(716.94237981,65.61651249)(716.93238037,65.66651611)
\curveto(716.92237983,65.71651239)(716.91737983,65.8015123)(716.91738037,65.92151611)
\curveto(716.91737983,66.05151205)(716.92737982,66.14651196)(716.94738037,66.20651611)
\curveto(716.96737978,66.27651183)(716.97237978,66.34651176)(716.96238037,66.41651611)
\curveto(716.9523798,66.48651162)(716.95737979,66.55651155)(716.97738037,66.62651611)
\curveto(716.98737976,66.67651143)(716.99237976,66.71651139)(716.99238037,66.74651611)
\curveto(717.00237975,66.78651132)(717.01237974,66.83151127)(717.02238037,66.88151611)
\curveto(717.0523797,67.0015111)(717.07737967,67.12151098)(717.09738037,67.24151611)
\curveto(717.12737962,67.36151074)(717.16737958,67.47651063)(717.21738037,67.58651611)
\curveto(717.36737938,67.95651015)(717.5473792,68.28650982)(717.75738037,68.57651611)
\curveto(717.97737877,68.87650923)(718.24237851,69.12650898)(718.55238037,69.32651611)
\curveto(718.67237808,69.4065087)(718.79737795,69.47150863)(718.92738037,69.52151611)
\curveto(719.05737769,69.58150852)(719.19237756,69.64150846)(719.33238037,69.70151611)
\curveto(719.4523773,69.75150835)(719.58237717,69.78150832)(719.72238037,69.79151611)
\curveto(719.86237689,69.81150829)(720.00237675,69.84150826)(720.14238037,69.88151611)
\lineto(720.33738037,69.88151611)
\curveto(720.40737634,69.89150821)(720.47237628,69.9015082)(720.53238037,69.91151611)
\curveto(721.42237533,69.92150818)(722.16237459,69.73650837)(722.75238037,69.35651611)
\curveto(723.34237341,68.97650913)(723.76737298,68.48150962)(724.02738037,67.87151611)
\curveto(724.07737267,67.77151033)(724.11737263,67.67151043)(724.14738037,67.57151611)
\curveto(724.17737257,67.47151063)(724.21237254,67.36651074)(724.25238037,67.25651611)
\curveto(724.28237247,67.14651096)(724.30737244,67.02651108)(724.32738037,66.89651611)
\curveto(724.3473724,66.77651133)(724.37237238,66.65151145)(724.40238037,66.52151611)
\curveto(724.41237234,66.47151163)(724.41237234,66.41651169)(724.40238037,66.35651611)
\curveto(724.40237235,66.3065118)(724.40737234,66.25651185)(724.41738037,66.20651611)
\moveto(723.08238037,65.35151611)
\curveto(723.10237365,65.42151268)(723.10737364,65.5015126)(723.09738037,65.59151611)
\lineto(723.09738037,65.84651611)
\curveto(723.09737365,66.23651187)(723.06237369,66.56651154)(722.99238037,66.83651611)
\curveto(722.96237379,66.91651119)(722.93737381,66.99651111)(722.91738037,67.07651611)
\curveto(722.89737385,67.15651095)(722.87237388,67.23151087)(722.84238037,67.30151611)
\curveto(722.56237419,67.95151015)(722.11737463,68.4015097)(721.50738037,68.65151611)
\curveto(721.43737531,68.68150942)(721.36237539,68.7015094)(721.28238037,68.71151611)
\lineto(721.04238037,68.77151611)
\curveto(720.96237579,68.79150931)(720.87737587,68.8015093)(720.78738037,68.80151611)
\lineto(720.51738037,68.80151611)
\lineto(720.24738037,68.75651611)
\curveto(720.1473766,68.73650937)(720.0523767,68.71150939)(719.96238037,68.68151611)
\curveto(719.88237687,68.66150944)(719.80237695,68.63150947)(719.72238037,68.59151611)
\curveto(719.6523771,68.57150953)(719.58737716,68.54150956)(719.52738037,68.50151611)
\curveto(719.46737728,68.46150964)(719.41237734,68.42150968)(719.36238037,68.38151611)
\curveto(719.12237763,68.21150989)(718.92737782,68.0065101)(718.77738037,67.76651611)
\curveto(718.62737812,67.52651058)(718.49737825,67.24651086)(718.38738037,66.92651611)
\curveto(718.35737839,66.82651128)(718.33737841,66.72151138)(718.32738037,66.61151611)
\curveto(718.31737843,66.51151159)(718.30237845,66.4065117)(718.28238037,66.29651611)
\curveto(718.27237848,66.25651185)(718.26737848,66.19151191)(718.26738037,66.10151611)
\curveto(718.25737849,66.07151203)(718.2523785,66.03651207)(718.25238037,65.99651611)
\curveto(718.26237849,65.95651215)(718.26737848,65.91151219)(718.26738037,65.86151611)
\lineto(718.26738037,65.56151611)
\curveto(718.26737848,65.46151264)(718.27737847,65.37151273)(718.29738037,65.29151611)
\lineto(718.32738037,65.11151611)
\curveto(718.3473784,65.01151309)(718.36237839,64.91151319)(718.37238037,64.81151611)
\curveto(718.39237836,64.72151338)(718.42237833,64.63651347)(718.46238037,64.55651611)
\curveto(718.56237819,64.31651379)(718.67737807,64.09151401)(718.80738037,63.88151611)
\curveto(718.9473778,63.67151443)(719.11737763,63.49651461)(719.31738037,63.35651611)
\curveto(719.36737738,63.32651478)(719.41237734,63.3015148)(719.45238037,63.28151611)
\curveto(719.49237726,63.26151484)(719.53737721,63.23651487)(719.58738037,63.20651611)
\curveto(719.66737708,63.15651495)(719.752377,63.11151499)(719.84238037,63.07151611)
\curveto(719.94237681,63.04151506)(720.0473767,63.01151509)(720.15738037,62.98151611)
\curveto(720.20737654,62.96151514)(720.2523765,62.95151515)(720.29238037,62.95151611)
\curveto(720.34237641,62.96151514)(720.39237636,62.96151514)(720.44238037,62.95151611)
\curveto(720.47237628,62.94151516)(720.53237622,62.93151517)(720.62238037,62.92151611)
\curveto(720.72237603,62.91151519)(720.79737595,62.91651519)(720.84738037,62.93651611)
\curveto(720.88737586,62.94651516)(720.92737582,62.94651516)(720.96738037,62.93651611)
\curveto(721.00737574,62.93651517)(721.0473757,62.94651516)(721.08738037,62.96651611)
\curveto(721.16737558,62.98651512)(721.2473755,63.0015151)(721.32738037,63.01151611)
\curveto(721.40737534,63.03151507)(721.48237527,63.05651505)(721.55238037,63.08651611)
\curveto(721.89237486,63.22651488)(722.16737458,63.42151468)(722.37738037,63.67151611)
\curveto(722.58737416,63.92151418)(722.76237399,64.21651389)(722.90238037,64.55651611)
\curveto(722.9523738,64.67651343)(722.98237377,64.8015133)(722.99238037,64.93151611)
\curveto(723.01237374,65.07151303)(723.04237371,65.21151289)(723.08238037,65.35151611)
}
}
{
\newrgbcolor{curcolor}{0 0 0}
\pscustom[linestyle=none,fillstyle=solid,fillcolor=curcolor]
{
\newpath
\moveto(729.55066162,69.91151611)
\curveto(729.78065683,69.91150819)(729.9106567,69.85150825)(729.94066162,69.73151611)
\curveto(729.97065664,69.62150848)(729.98565663,69.45650865)(729.98566162,69.23651611)
\lineto(729.98566162,68.95151611)
\curveto(729.98565663,68.86150924)(729.96065665,68.78650932)(729.91066162,68.72651611)
\curveto(729.85065676,68.64650946)(729.76565685,68.6015095)(729.65566162,68.59151611)
\curveto(729.54565707,68.59150951)(729.43565718,68.57650953)(729.32566162,68.54651611)
\curveto(729.18565743,68.51650959)(729.05065756,68.48650962)(728.92066162,68.45651611)
\curveto(728.80065781,68.42650968)(728.68565793,68.38650972)(728.57566162,68.33651611)
\curveto(728.28565833,68.2065099)(728.05065856,68.02651008)(727.87066162,67.79651611)
\curveto(727.69065892,67.57651053)(727.53565908,67.32151078)(727.40566162,67.03151611)
\curveto(727.36565925,66.92151118)(727.33565928,66.8065113)(727.31566162,66.68651611)
\curveto(727.29565932,66.57651153)(727.27065934,66.46151164)(727.24066162,66.34151611)
\curveto(727.23065938,66.29151181)(727.22565939,66.24151186)(727.22566162,66.19151611)
\curveto(727.23565938,66.14151196)(727.23565938,66.09151201)(727.22566162,66.04151611)
\curveto(727.19565942,65.92151218)(727.18065943,65.78151232)(727.18066162,65.62151611)
\curveto(727.19065942,65.47151263)(727.19565942,65.32651278)(727.19566162,65.18651611)
\lineto(727.19566162,63.34151611)
\lineto(727.19566162,62.99651611)
\curveto(727.19565942,62.87651523)(727.19065942,62.76151534)(727.18066162,62.65151611)
\curveto(727.17065944,62.54151556)(727.16565945,62.44651566)(727.16566162,62.36651611)
\curveto(727.17565944,62.28651582)(727.15565946,62.21651589)(727.10566162,62.15651611)
\curveto(727.05565956,62.08651602)(726.97565964,62.04651606)(726.86566162,62.03651611)
\curveto(726.76565985,62.02651608)(726.65565996,62.02151608)(726.53566162,62.02151611)
\lineto(726.26566162,62.02151611)
\curveto(726.2156604,62.04151606)(726.16566045,62.05651605)(726.11566162,62.06651611)
\curveto(726.07566054,62.08651602)(726.04566057,62.11151599)(726.02566162,62.14151611)
\curveto(725.97566064,62.21151589)(725.94566067,62.29651581)(725.93566162,62.39651611)
\lineto(725.93566162,62.72651611)
\lineto(725.93566162,63.88151611)
\lineto(725.93566162,68.03651611)
\lineto(725.93566162,69.07151611)
\lineto(725.93566162,69.37151611)
\curveto(725.94566067,69.47150863)(725.97566064,69.55650855)(726.02566162,69.62651611)
\curveto(726.05566056,69.66650844)(726.10566051,69.69650841)(726.17566162,69.71651611)
\curveto(726.25566036,69.73650837)(726.34066027,69.74650836)(726.43066162,69.74651611)
\curveto(726.52066009,69.75650835)(726.61066,69.75650835)(726.70066162,69.74651611)
\curveto(726.79065982,69.73650837)(726.86065975,69.72150838)(726.91066162,69.70151611)
\curveto(726.99065962,69.67150843)(727.04065957,69.61150849)(727.06066162,69.52151611)
\curveto(727.09065952,69.44150866)(727.10565951,69.35150875)(727.10566162,69.25151611)
\lineto(727.10566162,68.95151611)
\curveto(727.10565951,68.85150925)(727.12565949,68.76150934)(727.16566162,68.68151611)
\curveto(727.17565944,68.66150944)(727.18565943,68.64650946)(727.19566162,68.63651611)
\lineto(727.24066162,68.59151611)
\curveto(727.35065926,68.59150951)(727.44065917,68.63650947)(727.51066162,68.72651611)
\curveto(727.58065903,68.82650928)(727.64065897,68.9065092)(727.69066162,68.96651611)
\lineto(727.78066162,69.05651611)
\curveto(727.87065874,69.16650894)(727.99565862,69.28150882)(728.15566162,69.40151611)
\curveto(728.3156583,69.52150858)(728.46565815,69.61150849)(728.60566162,69.67151611)
\curveto(728.69565792,69.72150838)(728.79065782,69.75650835)(728.89066162,69.77651611)
\curveto(728.99065762,69.8065083)(729.09565752,69.83650827)(729.20566162,69.86651611)
\curveto(729.26565735,69.87650823)(729.32565729,69.88150822)(729.38566162,69.88151611)
\curveto(729.44565717,69.89150821)(729.50065711,69.9015082)(729.55066162,69.91151611)
}
}
{
\newrgbcolor{curcolor}{0 0 0}
\pscustom[linestyle=none,fillstyle=solid,fillcolor=curcolor]
{
\newpath
\moveto(191.47920532,54.82939941)
\lineto(196.38420532,54.82939941)
\lineto(197.67420532,54.82939941)
\curveto(197.78419744,54.82938872)(197.89419733,54.82938872)(198.00420532,54.82939941)
\curveto(198.11419711,54.83938871)(198.20419702,54.81938873)(198.27420532,54.76939941)
\curveto(198.30419692,54.7493888)(198.3291969,54.72438882)(198.34920532,54.69439941)
\curveto(198.36919686,54.66438888)(198.38919684,54.63438891)(198.40920532,54.60439941)
\curveto(198.4291968,54.53438901)(198.43919679,54.41938913)(198.43920532,54.25939941)
\curveto(198.43919679,54.10938944)(198.4291968,53.99438955)(198.40920532,53.91439941)
\curveto(198.36919686,53.77438977)(198.28419694,53.69438985)(198.15420532,53.67439941)
\curveto(198.0241972,53.66438988)(197.86919736,53.65938989)(197.68920532,53.65939941)
\lineto(196.18920532,53.65939941)
\lineto(193.66920532,53.65939941)
\lineto(193.09920532,53.65939941)
\curveto(192.88920234,53.66938988)(192.73420249,53.6443899)(192.63420532,53.58439941)
\curveto(192.53420269,53.52439002)(192.47920275,53.41939013)(192.46920532,53.26939941)
\lineto(192.46920532,52.80439941)
\lineto(192.46920532,51.27439941)
\curveto(192.46920276,51.16439238)(192.46420276,51.03439251)(192.45420532,50.88439941)
\curveto(192.45420277,50.73439281)(192.46420276,50.61439293)(192.48420532,50.52439941)
\curveto(192.51420271,50.40439314)(192.57420265,50.32439322)(192.66420532,50.28439941)
\curveto(192.70420252,50.26439328)(192.77420245,50.2443933)(192.87420532,50.22439941)
\lineto(193.02420532,50.22439941)
\curveto(193.06420216,50.21439333)(193.10420212,50.20939334)(193.14420532,50.20939941)
\curveto(193.19420203,50.21939333)(193.24420198,50.22439332)(193.29420532,50.22439941)
\lineto(193.80420532,50.22439941)
\lineto(196.74420532,50.22439941)
\lineto(197.04420532,50.22439941)
\curveto(197.15419807,50.23439331)(197.26419796,50.23439331)(197.37420532,50.22439941)
\curveto(197.49419773,50.22439332)(197.59919763,50.21439333)(197.68920532,50.19439941)
\curveto(197.78919744,50.18439336)(197.86419736,50.16439338)(197.91420532,50.13439941)
\curveto(197.94419728,50.11439343)(197.96919726,50.06939348)(197.98920532,49.99939941)
\curveto(198.00919722,49.92939362)(198.0241972,49.85439369)(198.03420532,49.77439941)
\curveto(198.04419718,49.69439385)(198.04419718,49.60939394)(198.03420532,49.51939941)
\curveto(198.03419719,49.43939411)(198.0241972,49.36939418)(198.00420532,49.30939941)
\curveto(197.98419724,49.21939433)(197.93919729,49.15439439)(197.86920532,49.11439941)
\curveto(197.84919738,49.09439445)(197.81919741,49.07939447)(197.77920532,49.06939941)
\curveto(197.74919748,49.06939448)(197.71919751,49.06439448)(197.68920532,49.05439941)
\lineto(197.59920532,49.05439941)
\curveto(197.54919768,49.0443945)(197.49919773,49.03939451)(197.44920532,49.03939941)
\curveto(197.39919783,49.0493945)(197.34919788,49.05439449)(197.29920532,49.05439941)
\lineto(196.74420532,49.05439941)
\lineto(193.57920532,49.05439941)
\lineto(193.21920532,49.05439941)
\curveto(193.10920212,49.06439448)(193.00420222,49.05939449)(192.90420532,49.03939941)
\curveto(192.80420242,49.02939452)(192.71420251,49.00439454)(192.63420532,48.96439941)
\curveto(192.56420266,48.92439462)(192.51420271,48.85439469)(192.48420532,48.75439941)
\curveto(192.46420276,48.69439485)(192.45420277,48.62439492)(192.45420532,48.54439941)
\curveto(192.46420276,48.46439508)(192.46920276,48.38439516)(192.46920532,48.30439941)
\lineto(192.46920532,47.46439941)
\lineto(192.46920532,46.03939941)
\curveto(192.46920276,45.89939765)(192.47420275,45.76939778)(192.48420532,45.64939941)
\curveto(192.49420273,45.53939801)(192.53420269,45.45939809)(192.60420532,45.40939941)
\curveto(192.67420255,45.35939819)(192.75420247,45.32939822)(192.84420532,45.31939941)
\lineto(193.14420532,45.31939941)
\lineto(194.10420532,45.31939941)
\lineto(196.87920532,45.31939941)
\lineto(197.73420532,45.31939941)
\lineto(197.97420532,45.31939941)
\curveto(198.05419717,45.32939822)(198.1241971,45.32439822)(198.18420532,45.30439941)
\curveto(198.30419692,45.26439828)(198.38419684,45.20939834)(198.42420532,45.13939941)
\curveto(198.44419678,45.10939844)(198.45919677,45.05939849)(198.46920532,44.98939941)
\curveto(198.47919675,44.91939863)(198.48419674,44.8443987)(198.48420532,44.76439941)
\curveto(198.49419673,44.69439885)(198.49419673,44.61939893)(198.48420532,44.53939941)
\curveto(198.47419675,44.46939908)(198.46419676,44.41439913)(198.45420532,44.37439941)
\curveto(198.41419681,44.29439925)(198.36919686,44.23939931)(198.31920532,44.20939941)
\curveto(198.25919697,44.16939938)(198.17919705,44.1493994)(198.07920532,44.14939941)
\lineto(197.80920532,44.14939941)
\lineto(196.75920532,44.14939941)
\lineto(192.76920532,44.14939941)
\lineto(191.71920532,44.14939941)
\curveto(191.57920365,44.1493994)(191.45920377,44.15439939)(191.35920532,44.16439941)
\curveto(191.25920397,44.18439936)(191.18420404,44.23439931)(191.13420532,44.31439941)
\curveto(191.09420413,44.37439917)(191.07420415,44.4493991)(191.07420532,44.53939941)
\lineto(191.07420532,44.82439941)
\lineto(191.07420532,45.87439941)
\lineto(191.07420532,49.89439941)
\lineto(191.07420532,53.25439941)
\lineto(191.07420532,54.18439941)
\lineto(191.07420532,54.45439941)
\curveto(191.07420415,54.544389)(191.09420413,54.61438893)(191.13420532,54.66439941)
\curveto(191.17420405,54.73438881)(191.24920398,54.78438876)(191.35920532,54.81439941)
\curveto(191.37920385,54.82438872)(191.39920383,54.82438872)(191.41920532,54.81439941)
\curveto(191.43920379,54.81438873)(191.45920377,54.81938873)(191.47920532,54.82939941)
}
}
{
\newrgbcolor{curcolor}{0 0 0}
\pscustom[linestyle=none,fillstyle=solid,fillcolor=curcolor]
{
\newpath
\moveto(202.4191272,52.05439941)
\curveto(203.13912313,52.06439148)(203.74412253,51.97939157)(204.2341272,51.79939941)
\curveto(204.72412155,51.62939192)(205.10412117,51.32439222)(205.3741272,50.88439941)
\curveto(205.44412083,50.77439277)(205.49912077,50.65939289)(205.5391272,50.53939941)
\curveto(205.57912069,50.42939312)(205.61912065,50.30439324)(205.6591272,50.16439941)
\curveto(205.67912059,50.09439345)(205.68412059,50.01939353)(205.6741272,49.93939941)
\curveto(205.66412061,49.86939368)(205.64912062,49.81439373)(205.6291272,49.77439941)
\curveto(205.60912066,49.75439379)(205.58412069,49.73439381)(205.5541272,49.71439941)
\curveto(205.52412075,49.70439384)(205.49912077,49.68939386)(205.4791272,49.66939941)
\curveto(205.42912084,49.6493939)(205.37912089,49.6443939)(205.3291272,49.65439941)
\curveto(205.27912099,49.66439388)(205.22912104,49.66439388)(205.1791272,49.65439941)
\curveto(205.09912117,49.63439391)(204.99412128,49.62939392)(204.8641272,49.63939941)
\curveto(204.73412154,49.65939389)(204.64412163,49.68439386)(204.5941272,49.71439941)
\curveto(204.51412176,49.76439378)(204.45912181,49.82939372)(204.4291272,49.90939941)
\curveto(204.40912186,49.99939355)(204.3741219,50.08439346)(204.3241272,50.16439941)
\curveto(204.23412204,50.32439322)(204.10912216,50.46939308)(203.9491272,50.59939941)
\curveto(203.83912243,50.67939287)(203.71912255,50.73939281)(203.5891272,50.77939941)
\curveto(203.45912281,50.81939273)(203.31912295,50.85939269)(203.1691272,50.89939941)
\curveto(203.11912315,50.91939263)(203.0691232,50.92439262)(203.0191272,50.91439941)
\curveto(202.9691233,50.91439263)(202.91912335,50.91939263)(202.8691272,50.92939941)
\curveto(202.80912346,50.9493926)(202.73412354,50.95939259)(202.6441272,50.95939941)
\curveto(202.55412372,50.95939259)(202.47912379,50.9493926)(202.4191272,50.92939941)
\lineto(202.3291272,50.92939941)
\lineto(202.1791272,50.89939941)
\curveto(202.12912414,50.89939265)(202.07912419,50.89439265)(202.0291272,50.88439941)
\curveto(201.7691245,50.82439272)(201.55412472,50.73939281)(201.3841272,50.62939941)
\curveto(201.21412506,50.51939303)(201.09912517,50.33439321)(201.0391272,50.07439941)
\curveto(201.01912525,50.00439354)(201.01412526,49.93439361)(201.0241272,49.86439941)
\curveto(201.04412523,49.79439375)(201.06412521,49.73439381)(201.0841272,49.68439941)
\curveto(201.14412513,49.53439401)(201.21412506,49.42439412)(201.2941272,49.35439941)
\curveto(201.38412489,49.29439425)(201.49412478,49.22439432)(201.6241272,49.14439941)
\curveto(201.78412449,49.0443945)(201.96412431,48.96939458)(202.1641272,48.91939941)
\curveto(202.36412391,48.87939467)(202.56412371,48.82939472)(202.7641272,48.76939941)
\curveto(202.89412338,48.72939482)(203.02412325,48.69939485)(203.1541272,48.67939941)
\curveto(203.28412299,48.65939489)(203.41412286,48.62939492)(203.5441272,48.58939941)
\curveto(203.75412252,48.52939502)(203.95912231,48.46939508)(204.1591272,48.40939941)
\curveto(204.35912191,48.35939519)(204.55912171,48.29439525)(204.7591272,48.21439941)
\lineto(204.9091272,48.15439941)
\curveto(204.95912131,48.13439541)(205.00912126,48.10939544)(205.0591272,48.07939941)
\curveto(205.25912101,47.95939559)(205.43412084,47.82439572)(205.5841272,47.67439941)
\curveto(205.73412054,47.52439602)(205.85912041,47.33439621)(205.9591272,47.10439941)
\curveto(205.97912029,47.03439651)(205.99912027,46.93939661)(206.0191272,46.81939941)
\curveto(206.03912023,46.7493968)(206.04912022,46.67439687)(206.0491272,46.59439941)
\curveto(206.05912021,46.52439702)(206.06412021,46.4443971)(206.0641272,46.35439941)
\lineto(206.0641272,46.20439941)
\curveto(206.04412023,46.13439741)(206.03412024,46.06439748)(206.0341272,45.99439941)
\curveto(206.03412024,45.92439762)(206.02412025,45.85439769)(206.0041272,45.78439941)
\curveto(205.9741203,45.67439787)(205.93912033,45.56939798)(205.8991272,45.46939941)
\curveto(205.85912041,45.36939818)(205.81412046,45.27939827)(205.7641272,45.19939941)
\curveto(205.60412067,44.93939861)(205.39912087,44.72939882)(205.1491272,44.56939941)
\curveto(204.89912137,44.41939913)(204.61912165,44.28939926)(204.3091272,44.17939941)
\curveto(204.21912205,44.1493994)(204.12412215,44.12939942)(204.0241272,44.11939941)
\curveto(203.93412234,44.09939945)(203.84412243,44.07439947)(203.7541272,44.04439941)
\curveto(203.65412262,44.02439952)(203.55412272,44.01439953)(203.4541272,44.01439941)
\curveto(203.35412292,44.01439953)(203.25412302,44.00439954)(203.1541272,43.98439941)
\lineto(203.0041272,43.98439941)
\curveto(202.95412332,43.97439957)(202.88412339,43.96939958)(202.7941272,43.96939941)
\curveto(202.70412357,43.96939958)(202.63412364,43.97439957)(202.5841272,43.98439941)
\lineto(202.4191272,43.98439941)
\curveto(202.35912391,44.00439954)(202.29412398,44.01439953)(202.2241272,44.01439941)
\curveto(202.15412412,44.00439954)(202.09412418,44.00939954)(202.0441272,44.02939941)
\curveto(201.99412428,44.03939951)(201.92912434,44.0443995)(201.8491272,44.04439941)
\lineto(201.6091272,44.10439941)
\curveto(201.53912473,44.11439943)(201.46412481,44.13439941)(201.3841272,44.16439941)
\curveto(201.0741252,44.26439928)(200.80412547,44.38939916)(200.5741272,44.53939941)
\curveto(200.34412593,44.68939886)(200.14412613,44.88439866)(199.9741272,45.12439941)
\curveto(199.88412639,45.25439829)(199.80912646,45.38939816)(199.7491272,45.52939941)
\curveto(199.68912658,45.66939788)(199.63412664,45.82439772)(199.5841272,45.99439941)
\curveto(199.56412671,46.05439749)(199.55412672,46.12439742)(199.5541272,46.20439941)
\curveto(199.56412671,46.29439725)(199.57912669,46.36439718)(199.5991272,46.41439941)
\curveto(199.62912664,46.45439709)(199.67912659,46.49439705)(199.7491272,46.53439941)
\curveto(199.79912647,46.55439699)(199.8691264,46.56439698)(199.9591272,46.56439941)
\curveto(200.04912622,46.57439697)(200.13912613,46.57439697)(200.2291272,46.56439941)
\curveto(200.31912595,46.55439699)(200.40412587,46.53939701)(200.4841272,46.51939941)
\curveto(200.5741257,46.50939704)(200.63412564,46.49439705)(200.6641272,46.47439941)
\curveto(200.73412554,46.42439712)(200.77912549,46.3493972)(200.7991272,46.24939941)
\curveto(200.82912544,46.15939739)(200.86412541,46.07439747)(200.9041272,45.99439941)
\curveto(201.00412527,45.77439777)(201.13912513,45.60439794)(201.3091272,45.48439941)
\curveto(201.42912484,45.39439815)(201.56412471,45.32439822)(201.7141272,45.27439941)
\curveto(201.86412441,45.22439832)(202.02412425,45.17439837)(202.1941272,45.12439941)
\lineto(202.5091272,45.07939941)
\lineto(202.5991272,45.07939941)
\curveto(202.6691236,45.05939849)(202.75912351,45.0493985)(202.8691272,45.04939941)
\curveto(202.98912328,45.0493985)(203.08912318,45.05939849)(203.1691272,45.07939941)
\curveto(203.23912303,45.07939847)(203.29412298,45.08439846)(203.3341272,45.09439941)
\curveto(203.39412288,45.10439844)(203.45412282,45.10939844)(203.5141272,45.10939941)
\curveto(203.5741227,45.11939843)(203.62912264,45.12939842)(203.6791272,45.13939941)
\curveto(203.9691223,45.21939833)(204.19912207,45.32439822)(204.3691272,45.45439941)
\curveto(204.53912173,45.58439796)(204.65912161,45.80439774)(204.7291272,46.11439941)
\curveto(204.74912152,46.16439738)(204.75412152,46.21939733)(204.7441272,46.27939941)
\curveto(204.73412154,46.33939721)(204.72412155,46.38439716)(204.7141272,46.41439941)
\curveto(204.66412161,46.60439694)(204.59412168,46.7443968)(204.5041272,46.83439941)
\curveto(204.41412186,46.93439661)(204.29912197,47.02439652)(204.1591272,47.10439941)
\curveto(204.0691222,47.16439638)(203.9691223,47.21439633)(203.8591272,47.25439941)
\lineto(203.5291272,47.37439941)
\curveto(203.49912277,47.38439616)(203.4691228,47.38939616)(203.4391272,47.38939941)
\curveto(203.41912285,47.38939616)(203.39412288,47.39939615)(203.3641272,47.41939941)
\curveto(203.02412325,47.52939602)(202.6691236,47.60939594)(202.2991272,47.65939941)
\curveto(201.93912433,47.71939583)(201.59912467,47.81439573)(201.2791272,47.94439941)
\curveto(201.17912509,47.98439556)(201.08412519,48.01939553)(200.9941272,48.04939941)
\curveto(200.90412537,48.07939547)(200.81912545,48.11939543)(200.7391272,48.16939941)
\curveto(200.54912572,48.27939527)(200.3741259,48.40439514)(200.2141272,48.54439941)
\curveto(200.05412622,48.68439486)(199.92912634,48.85939469)(199.8391272,49.06939941)
\curveto(199.80912646,49.13939441)(199.78412649,49.20939434)(199.7641272,49.27939941)
\curveto(199.75412652,49.3493942)(199.73912653,49.42439412)(199.7191272,49.50439941)
\curveto(199.68912658,49.62439392)(199.67912659,49.75939379)(199.6891272,49.90939941)
\curveto(199.69912657,50.06939348)(199.71412656,50.20439334)(199.7341272,50.31439941)
\curveto(199.75412652,50.36439318)(199.76412651,50.40439314)(199.7641272,50.43439941)
\curveto(199.7741265,50.47439307)(199.78912648,50.51439303)(199.8091272,50.55439941)
\curveto(199.89912637,50.78439276)(200.01912625,50.98439256)(200.1691272,51.15439941)
\curveto(200.32912594,51.32439222)(200.50912576,51.47439207)(200.7091272,51.60439941)
\curveto(200.85912541,51.69439185)(201.02412525,51.76439178)(201.2041272,51.81439941)
\curveto(201.38412489,51.87439167)(201.5741247,51.92939162)(201.7741272,51.97939941)
\curveto(201.84412443,51.98939156)(201.90912436,51.99939155)(201.9691272,52.00939941)
\curveto(202.03912423,52.01939153)(202.11412416,52.02939152)(202.1941272,52.03939941)
\curveto(202.22412405,52.0493915)(202.26412401,52.0493915)(202.3141272,52.03939941)
\curveto(202.36412391,52.02939152)(202.39912387,52.03439151)(202.4191272,52.05439941)
}
}
{
\newrgbcolor{curcolor}{0 0 0}
\pscustom[linestyle=none,fillstyle=solid,fillcolor=curcolor]
{
\newpath
\moveto(208.4341272,54.21439941)
\curveto(208.58412519,54.21438933)(208.73412504,54.20938934)(208.8841272,54.19939941)
\curveto(209.03412474,54.19938935)(209.13912463,54.15938939)(209.1991272,54.07939941)
\curveto(209.24912452,54.01938953)(209.2741245,53.93438961)(209.2741272,53.82439941)
\curveto(209.28412449,53.72438982)(209.28912448,53.61938993)(209.2891272,53.50939941)
\lineto(209.2891272,52.63939941)
\curveto(209.28912448,52.55939099)(209.28412449,52.47439107)(209.2741272,52.38439941)
\curveto(209.2741245,52.30439124)(209.28412449,52.23439131)(209.3041272,52.17439941)
\curveto(209.34412443,52.03439151)(209.43412434,51.9443916)(209.5741272,51.90439941)
\curveto(209.62412415,51.89439165)(209.6691241,51.88939166)(209.7091272,51.88939941)
\lineto(209.8591272,51.88939941)
\lineto(210.2641272,51.88939941)
\curveto(210.42412335,51.89939165)(210.53912323,51.88939166)(210.6091272,51.85939941)
\curveto(210.69912307,51.79939175)(210.75912301,51.73939181)(210.7891272,51.67939941)
\curveto(210.80912296,51.63939191)(210.81912295,51.59439195)(210.8191272,51.54439941)
\lineto(210.8191272,51.39439941)
\curveto(210.81912295,51.28439226)(210.81412296,51.17939237)(210.8041272,51.07939941)
\curveto(210.79412298,50.98939256)(210.75912301,50.91939263)(210.6991272,50.86939941)
\curveto(210.63912313,50.81939273)(210.55412322,50.78939276)(210.4441272,50.77939941)
\lineto(210.1141272,50.77939941)
\curveto(210.00412377,50.78939276)(209.89412388,50.79439275)(209.7841272,50.79439941)
\curveto(209.6741241,50.79439275)(209.57912419,50.77939277)(209.4991272,50.74939941)
\curveto(209.42912434,50.71939283)(209.37912439,50.66939288)(209.3491272,50.59939941)
\curveto(209.31912445,50.52939302)(209.29912447,50.4443931)(209.2891272,50.34439941)
\curveto(209.27912449,50.25439329)(209.2741245,50.15439339)(209.2741272,50.04439941)
\curveto(209.28412449,49.9443936)(209.28912448,49.8443937)(209.2891272,49.74439941)
\lineto(209.2891272,46.77439941)
\curveto(209.28912448,46.55439699)(209.28412449,46.31939723)(209.2741272,46.06939941)
\curveto(209.2741245,45.82939772)(209.31912445,45.6443979)(209.4091272,45.51439941)
\curveto(209.45912431,45.43439811)(209.52412425,45.37939817)(209.6041272,45.34939941)
\curveto(209.68412409,45.31939823)(209.77912399,45.29439825)(209.8891272,45.27439941)
\curveto(209.91912385,45.26439828)(209.94912382,45.25939829)(209.9791272,45.25939941)
\curveto(210.01912375,45.26939828)(210.05412372,45.26939828)(210.0841272,45.25939941)
\lineto(210.2791272,45.25939941)
\curveto(210.37912339,45.25939829)(210.4691233,45.2493983)(210.5491272,45.22939941)
\curveto(210.63912313,45.21939833)(210.70412307,45.18439836)(210.7441272,45.12439941)
\curveto(210.76412301,45.09439845)(210.77912299,45.03939851)(210.7891272,44.95939941)
\curveto(210.80912296,44.88939866)(210.81912295,44.81439873)(210.8191272,44.73439941)
\curveto(210.82912294,44.65439889)(210.82912294,44.57439897)(210.8191272,44.49439941)
\curveto(210.80912296,44.42439912)(210.78912298,44.36939918)(210.7591272,44.32939941)
\curveto(210.71912305,44.25939929)(210.64412313,44.20939934)(210.5341272,44.17939941)
\curveto(210.45412332,44.15939939)(210.36412341,44.1493994)(210.2641272,44.14939941)
\curveto(210.16412361,44.15939939)(210.0741237,44.16439938)(209.9941272,44.16439941)
\curveto(209.93412384,44.16439938)(209.8741239,44.15939939)(209.8141272,44.14939941)
\curveto(209.75412402,44.1493994)(209.69912407,44.15439939)(209.6491272,44.16439941)
\lineto(209.4691272,44.16439941)
\curveto(209.41912435,44.17439937)(209.3691244,44.17939937)(209.3191272,44.17939941)
\curveto(209.27912449,44.18939936)(209.23412454,44.19439935)(209.1841272,44.19439941)
\curveto(208.98412479,44.2443993)(208.80912496,44.29939925)(208.6591272,44.35939941)
\curveto(208.51912525,44.41939913)(208.39912537,44.52439902)(208.2991272,44.67439941)
\curveto(208.15912561,44.87439867)(208.07912569,45.12439842)(208.0591272,45.42439941)
\curveto(208.03912573,45.73439781)(208.02912574,46.06439748)(208.0291272,46.41439941)
\lineto(208.0291272,50.34439941)
\curveto(207.99912577,50.47439307)(207.9691258,50.56939298)(207.9391272,50.62939941)
\curveto(207.91912585,50.68939286)(207.84912592,50.73939281)(207.7291272,50.77939941)
\curveto(207.68912608,50.78939276)(207.64912612,50.78939276)(207.6091272,50.77939941)
\curveto(207.5691262,50.76939278)(207.52912624,50.77439277)(207.4891272,50.79439941)
\lineto(207.2491272,50.79439941)
\curveto(207.11912665,50.79439275)(207.00912676,50.80439274)(206.9191272,50.82439941)
\curveto(206.83912693,50.85439269)(206.78412699,50.91439263)(206.7541272,51.00439941)
\curveto(206.73412704,51.0443925)(206.71912705,51.08939246)(206.7091272,51.13939941)
\lineto(206.7091272,51.28939941)
\curveto(206.70912706,51.42939212)(206.71912705,51.544392)(206.7391272,51.63439941)
\curveto(206.75912701,51.73439181)(206.81912695,51.80939174)(206.9191272,51.85939941)
\curveto(207.02912674,51.89939165)(207.1691266,51.90939164)(207.3391272,51.88939941)
\curveto(207.51912625,51.86939168)(207.6691261,51.87939167)(207.7891272,51.91939941)
\curveto(207.87912589,51.96939158)(207.94912582,52.03939151)(207.9991272,52.12939941)
\curveto(208.01912575,52.18939136)(208.02912574,52.26439128)(208.0291272,52.35439941)
\lineto(208.0291272,52.60939941)
\lineto(208.0291272,53.53939941)
\lineto(208.0291272,53.77939941)
\curveto(208.02912574,53.86938968)(208.03912573,53.9443896)(208.0591272,54.00439941)
\curveto(208.09912567,54.08438946)(208.1741256,54.1493894)(208.2841272,54.19939941)
\curveto(208.31412546,54.19938935)(208.33912543,54.19938935)(208.3591272,54.19939941)
\curveto(208.38912538,54.20938934)(208.41412536,54.21438933)(208.4341272,54.21439941)
}
}
{
\newrgbcolor{curcolor}{0 0 0}
\pscustom[linestyle=none,fillstyle=solid,fillcolor=curcolor]
{
\newpath
\moveto(212.67092407,51.87439941)
\lineto(213.10592407,51.87439941)
\curveto(213.25592211,51.87439167)(213.360922,51.83439171)(213.42092407,51.75439941)
\curveto(213.47092189,51.67439187)(213.49592187,51.57439197)(213.49592407,51.45439941)
\curveto(213.50592186,51.33439221)(213.51092185,51.21439233)(213.51092407,51.09439941)
\lineto(213.51092407,49.66939941)
\lineto(213.51092407,47.40439941)
\lineto(213.51092407,46.71439941)
\curveto(213.51092185,46.48439706)(213.53592183,46.28439726)(213.58592407,46.11439941)
\curveto(213.74592162,45.66439788)(214.04592132,45.3493982)(214.48592407,45.16939941)
\curveto(214.70592066,45.07939847)(214.97092039,45.0443985)(215.28092407,45.06439941)
\curveto(215.59091977,45.09439845)(215.84091952,45.1493984)(216.03092407,45.22939941)
\curveto(216.360919,45.36939818)(216.62091874,45.544398)(216.81092407,45.75439941)
\curveto(217.01091835,45.97439757)(217.1659182,46.25939729)(217.27592407,46.60939941)
\curveto(217.30591806,46.68939686)(217.32591804,46.76939678)(217.33592407,46.84939941)
\curveto(217.34591802,46.92939662)(217.360918,47.01439653)(217.38092407,47.10439941)
\curveto(217.39091797,47.15439639)(217.39091797,47.19939635)(217.38092407,47.23939941)
\curveto(217.38091798,47.27939627)(217.39091797,47.32439622)(217.41092407,47.37439941)
\lineto(217.41092407,47.68939941)
\curveto(217.43091793,47.76939578)(217.43591793,47.85939569)(217.42592407,47.95939941)
\curveto(217.41591795,48.06939548)(217.41091795,48.16939538)(217.41092407,48.25939941)
\lineto(217.41092407,49.42939941)
\lineto(217.41092407,51.01939941)
\curveto(217.41091795,51.13939241)(217.40591796,51.26439228)(217.39592407,51.39439941)
\curveto(217.39591797,51.53439201)(217.42091794,51.6443919)(217.47092407,51.72439941)
\curveto(217.51091785,51.77439177)(217.55591781,51.80439174)(217.60592407,51.81439941)
\curveto(217.6659177,51.83439171)(217.73591763,51.85439169)(217.81592407,51.87439941)
\lineto(218.04092407,51.87439941)
\curveto(218.1609172,51.87439167)(218.2659171,51.86939168)(218.35592407,51.85939941)
\curveto(218.45591691,51.8493917)(218.53091683,51.80439174)(218.58092407,51.72439941)
\curveto(218.63091673,51.67439187)(218.65591671,51.59939195)(218.65592407,51.49939941)
\lineto(218.65592407,51.21439941)
\lineto(218.65592407,50.19439941)
\lineto(218.65592407,46.15939941)
\lineto(218.65592407,44.80939941)
\curveto(218.65591671,44.68939886)(218.65091671,44.57439897)(218.64092407,44.46439941)
\curveto(218.64091672,44.36439918)(218.60591676,44.28939926)(218.53592407,44.23939941)
\curveto(218.49591687,44.20939934)(218.43591693,44.18439936)(218.35592407,44.16439941)
\curveto(218.27591709,44.15439939)(218.18591718,44.1443994)(218.08592407,44.13439941)
\curveto(217.99591737,44.13439941)(217.90591746,44.13939941)(217.81592407,44.14939941)
\curveto(217.73591763,44.15939939)(217.67591769,44.17939937)(217.63592407,44.20939941)
\curveto(217.58591778,44.2493993)(217.54091782,44.31439923)(217.50092407,44.40439941)
\curveto(217.49091787,44.4443991)(217.48091788,44.49939905)(217.47092407,44.56939941)
\curveto(217.47091789,44.63939891)(217.4659179,44.70439884)(217.45592407,44.76439941)
\curveto(217.44591792,44.83439871)(217.42591794,44.88939866)(217.39592407,44.92939941)
\curveto(217.365918,44.96939858)(217.32091804,44.98439856)(217.26092407,44.97439941)
\curveto(217.18091818,44.95439859)(217.10091826,44.89439865)(217.02092407,44.79439941)
\curveto(216.94091842,44.70439884)(216.8659185,44.63439891)(216.79592407,44.58439941)
\curveto(216.57591879,44.42439912)(216.32591904,44.28439926)(216.04592407,44.16439941)
\curveto(215.93591943,44.11439943)(215.82091954,44.08439946)(215.70092407,44.07439941)
\curveto(215.59091977,44.05439949)(215.47591989,44.02939952)(215.35592407,43.99939941)
\curveto(215.30592006,43.98939956)(215.25092011,43.98939956)(215.19092407,43.99939941)
\curveto(215.14092022,44.00939954)(215.09092027,44.00439954)(215.04092407,43.98439941)
\curveto(214.94092042,43.96439958)(214.85092051,43.96439958)(214.77092407,43.98439941)
\lineto(214.62092407,43.98439941)
\curveto(214.57092079,44.00439954)(214.51092085,44.01439953)(214.44092407,44.01439941)
\curveto(214.38092098,44.01439953)(214.32592104,44.01939953)(214.27592407,44.02939941)
\curveto(214.23592113,44.0493995)(214.19592117,44.05939949)(214.15592407,44.05939941)
\curveto(214.12592124,44.0493995)(214.08592128,44.05439949)(214.03592407,44.07439941)
\lineto(213.79592407,44.13439941)
\curveto(213.72592164,44.15439939)(213.65092171,44.18439936)(213.57092407,44.22439941)
\curveto(213.31092205,44.33439921)(213.09092227,44.47939907)(212.91092407,44.65939941)
\curveto(212.74092262,44.8493987)(212.60092276,45.07439847)(212.49092407,45.33439941)
\curveto(212.45092291,45.42439812)(212.42092294,45.51439803)(212.40092407,45.60439941)
\lineto(212.34092407,45.90439941)
\curveto(212.32092304,45.96439758)(212.31092305,46.01939753)(212.31092407,46.06939941)
\curveto(212.32092304,46.12939742)(212.31592305,46.19439735)(212.29592407,46.26439941)
\curveto(212.28592308,46.28439726)(212.28092308,46.30939724)(212.28092407,46.33939941)
\curveto(212.28092308,46.37939717)(212.27592309,46.41439713)(212.26592407,46.44439941)
\lineto(212.26592407,46.59439941)
\curveto(212.25592311,46.63439691)(212.25092311,46.67939687)(212.25092407,46.72939941)
\curveto(212.2609231,46.78939676)(212.2659231,46.8443967)(212.26592407,46.89439941)
\lineto(212.26592407,47.49439941)
\lineto(212.26592407,50.25439941)
\lineto(212.26592407,51.21439941)
\lineto(212.26592407,51.48439941)
\curveto(212.2659231,51.57439197)(212.28592308,51.6493919)(212.32592407,51.70939941)
\curveto(212.365923,51.77939177)(212.44092292,51.82939172)(212.55092407,51.85939941)
\curveto(212.57092279,51.86939168)(212.59092277,51.86939168)(212.61092407,51.85939941)
\curveto(212.63092273,51.85939169)(212.65092271,51.86439168)(212.67092407,51.87439941)
}
}
{
\newrgbcolor{curcolor}{0 0 0}
\pscustom[linestyle=none,fillstyle=solid,fillcolor=curcolor]
{
\newpath
\moveto(227.51553345,44.95939941)
\lineto(227.51553345,44.56939941)
\curveto(227.51552557,44.4493991)(227.4905256,44.3493992)(227.44053345,44.26939941)
\curveto(227.3905257,44.19939935)(227.30552578,44.15939939)(227.18553345,44.14939941)
\lineto(226.84053345,44.14939941)
\curveto(226.78052631,44.1493994)(226.72052637,44.1443994)(226.66053345,44.13439941)
\curveto(226.61052648,44.13439941)(226.56552652,44.1443994)(226.52553345,44.16439941)
\curveto(226.43552665,44.18439936)(226.37552671,44.22439932)(226.34553345,44.28439941)
\curveto(226.30552678,44.33439921)(226.28052681,44.39439915)(226.27053345,44.46439941)
\curveto(226.27052682,44.53439901)(226.25552683,44.60439894)(226.22553345,44.67439941)
\curveto(226.21552687,44.69439885)(226.20052689,44.70939884)(226.18053345,44.71939941)
\curveto(226.17052692,44.73939881)(226.15552693,44.75939879)(226.13553345,44.77939941)
\curveto(226.03552705,44.78939876)(225.95552713,44.76939878)(225.89553345,44.71939941)
\curveto(225.84552724,44.66939888)(225.7905273,44.61939893)(225.73053345,44.56939941)
\curveto(225.53052756,44.41939913)(225.33052776,44.30439924)(225.13053345,44.22439941)
\curveto(224.95052814,44.1443994)(224.74052835,44.08439946)(224.50053345,44.04439941)
\curveto(224.27052882,44.00439954)(224.03052906,43.98439956)(223.78053345,43.98439941)
\curveto(223.54052955,43.97439957)(223.30052979,43.98939956)(223.06053345,44.02939941)
\curveto(222.82053027,44.05939949)(222.61053048,44.11439943)(222.43053345,44.19439941)
\curveto(221.91053118,44.41439913)(221.4905316,44.70939884)(221.17053345,45.07939941)
\curveto(220.85053224,45.45939809)(220.60053249,45.92939762)(220.42053345,46.48939941)
\curveto(220.38053271,46.57939697)(220.35053274,46.66939688)(220.33053345,46.75939941)
\curveto(220.32053277,46.85939669)(220.30053279,46.95939659)(220.27053345,47.05939941)
\curveto(220.26053283,47.10939644)(220.25553283,47.15939639)(220.25553345,47.20939941)
\curveto(220.25553283,47.25939629)(220.25053284,47.30939624)(220.24053345,47.35939941)
\curveto(220.22053287,47.40939614)(220.21053288,47.45939609)(220.21053345,47.50939941)
\curveto(220.22053287,47.56939598)(220.22053287,47.62439592)(220.21053345,47.67439941)
\lineto(220.21053345,47.82439941)
\curveto(220.1905329,47.87439567)(220.18053291,47.93939561)(220.18053345,48.01939941)
\curveto(220.18053291,48.09939545)(220.1905329,48.16439538)(220.21053345,48.21439941)
\lineto(220.21053345,48.37939941)
\curveto(220.23053286,48.4493951)(220.23553285,48.51939503)(220.22553345,48.58939941)
\curveto(220.22553286,48.66939488)(220.23553285,48.7443948)(220.25553345,48.81439941)
\curveto(220.26553282,48.86439468)(220.27053282,48.90939464)(220.27053345,48.94939941)
\curveto(220.27053282,48.98939456)(220.27553281,49.03439451)(220.28553345,49.08439941)
\curveto(220.31553277,49.18439436)(220.34053275,49.27939427)(220.36053345,49.36939941)
\curveto(220.38053271,49.46939408)(220.40553268,49.56439398)(220.43553345,49.65439941)
\curveto(220.56553252,50.03439351)(220.73053236,50.37439317)(220.93053345,50.67439941)
\curveto(221.14053195,50.98439256)(221.3905317,51.23939231)(221.68053345,51.43939941)
\curveto(221.85053124,51.55939199)(222.02553106,51.65939189)(222.20553345,51.73939941)
\curveto(222.39553069,51.81939173)(222.60053049,51.88939166)(222.82053345,51.94939941)
\curveto(222.8905302,51.95939159)(222.95553013,51.96939158)(223.01553345,51.97939941)
\curveto(223.08553,51.98939156)(223.15552993,52.00439154)(223.22553345,52.02439941)
\lineto(223.37553345,52.02439941)
\curveto(223.45552963,52.0443915)(223.57052952,52.05439149)(223.72053345,52.05439941)
\curveto(223.88052921,52.05439149)(224.00052909,52.0443915)(224.08053345,52.02439941)
\curveto(224.12052897,52.01439153)(224.17552891,52.00939154)(224.24553345,52.00939941)
\curveto(224.35552873,51.97939157)(224.46552862,51.95439159)(224.57553345,51.93439941)
\curveto(224.6855284,51.92439162)(224.7905283,51.89439165)(224.89053345,51.84439941)
\curveto(225.04052805,51.78439176)(225.18052791,51.71939183)(225.31053345,51.64939941)
\curveto(225.45052764,51.57939197)(225.58052751,51.49939205)(225.70053345,51.40939941)
\curveto(225.76052733,51.35939219)(225.82052727,51.30439224)(225.88053345,51.24439941)
\curveto(225.95052714,51.19439235)(226.04052705,51.17939237)(226.15053345,51.19939941)
\curveto(226.17052692,51.22939232)(226.1855269,51.25439229)(226.19553345,51.27439941)
\curveto(226.21552687,51.29439225)(226.23052686,51.32439222)(226.24053345,51.36439941)
\curveto(226.27052682,51.45439209)(226.28052681,51.56939198)(226.27053345,51.70939941)
\lineto(226.27053345,52.08439941)
\lineto(226.27053345,53.80939941)
\lineto(226.27053345,54.27439941)
\curveto(226.27052682,54.45438909)(226.29552679,54.58438896)(226.34553345,54.66439941)
\curveto(226.3855267,54.73438881)(226.44552664,54.77938877)(226.52553345,54.79939941)
\curveto(226.54552654,54.79938875)(226.57052652,54.79938875)(226.60053345,54.79939941)
\curveto(226.63052646,54.80938874)(226.65552643,54.81438873)(226.67553345,54.81439941)
\curveto(226.81552627,54.82438872)(226.96052613,54.82438872)(227.11053345,54.81439941)
\curveto(227.27052582,54.81438873)(227.38052571,54.77438877)(227.44053345,54.69439941)
\curveto(227.4905256,54.61438893)(227.51552557,54.51438903)(227.51553345,54.39439941)
\lineto(227.51553345,54.01939941)
\lineto(227.51553345,44.95939941)
\moveto(226.30053345,47.79439941)
\curveto(226.32052677,47.8443957)(226.33052676,47.90939564)(226.33053345,47.98939941)
\curveto(226.33052676,48.07939547)(226.32052677,48.1493954)(226.30053345,48.19939941)
\lineto(226.30053345,48.42439941)
\curveto(226.28052681,48.51439503)(226.26552682,48.60439494)(226.25553345,48.69439941)
\curveto(226.24552684,48.79439475)(226.22552686,48.88439466)(226.19553345,48.96439941)
\curveto(226.17552691,49.0443945)(226.15552693,49.11939443)(226.13553345,49.18939941)
\curveto(226.12552696,49.25939429)(226.10552698,49.32939422)(226.07553345,49.39939941)
\curveto(225.95552713,49.69939385)(225.80052729,49.96439358)(225.61053345,50.19439941)
\curveto(225.42052767,50.42439312)(225.18052791,50.60439294)(224.89053345,50.73439941)
\curveto(224.7905283,50.78439276)(224.6855284,50.81939273)(224.57553345,50.83939941)
\curveto(224.47552861,50.86939268)(224.36552872,50.89439265)(224.24553345,50.91439941)
\curveto(224.16552892,50.93439261)(224.07552901,50.9443926)(223.97553345,50.94439941)
\lineto(223.70553345,50.94439941)
\curveto(223.65552943,50.93439261)(223.61052948,50.92439262)(223.57053345,50.91439941)
\lineto(223.43553345,50.91439941)
\curveto(223.35552973,50.89439265)(223.27052982,50.87439267)(223.18053345,50.85439941)
\curveto(223.10052999,50.83439271)(223.02053007,50.80939274)(222.94053345,50.77939941)
\curveto(222.62053047,50.63939291)(222.36053073,50.43439311)(222.16053345,50.16439941)
\curveto(221.97053112,49.90439364)(221.81553127,49.59939395)(221.69553345,49.24939941)
\curveto(221.65553143,49.13939441)(221.62553146,49.02439452)(221.60553345,48.90439941)
\curveto(221.59553149,48.79439475)(221.58053151,48.68439486)(221.56053345,48.57439941)
\curveto(221.56053153,48.53439501)(221.55553153,48.49439505)(221.54553345,48.45439941)
\lineto(221.54553345,48.34939941)
\curveto(221.52553156,48.29939525)(221.51553157,48.2443953)(221.51553345,48.18439941)
\curveto(221.52553156,48.12439542)(221.53053156,48.06939548)(221.53053345,48.01939941)
\lineto(221.53053345,47.68939941)
\curveto(221.53053156,47.58939596)(221.54053155,47.49439605)(221.56053345,47.40439941)
\curveto(221.57053152,47.37439617)(221.57553151,47.32439622)(221.57553345,47.25439941)
\curveto(221.59553149,47.18439636)(221.61053148,47.11439643)(221.62053345,47.04439941)
\lineto(221.68053345,46.83439941)
\curveto(221.7905313,46.48439706)(221.94053115,46.18439736)(222.13053345,45.93439941)
\curveto(222.32053077,45.68439786)(222.56053053,45.47939807)(222.85053345,45.31939941)
\curveto(222.94053015,45.26939828)(223.03053006,45.22939832)(223.12053345,45.19939941)
\curveto(223.21052988,45.16939838)(223.31052978,45.13939841)(223.42053345,45.10939941)
\curveto(223.47052962,45.08939846)(223.52052957,45.08439846)(223.57053345,45.09439941)
\curveto(223.63052946,45.10439844)(223.6855294,45.09939845)(223.73553345,45.07939941)
\curveto(223.77552931,45.06939848)(223.81552927,45.06439848)(223.85553345,45.06439941)
\lineto(223.99053345,45.06439941)
\lineto(224.12553345,45.06439941)
\curveto(224.15552893,45.07439847)(224.20552888,45.07939847)(224.27553345,45.07939941)
\curveto(224.35552873,45.09939845)(224.43552865,45.11439843)(224.51553345,45.12439941)
\curveto(224.59552849,45.1443984)(224.67052842,45.16939838)(224.74053345,45.19939941)
\curveto(225.07052802,45.33939821)(225.33552775,45.51439803)(225.53553345,45.72439941)
\curveto(225.74552734,45.9443976)(225.92052717,46.21939733)(226.06053345,46.54939941)
\curveto(226.11052698,46.65939689)(226.14552694,46.76939678)(226.16553345,46.87939941)
\curveto(226.1855269,46.98939656)(226.21052688,47.09939645)(226.24053345,47.20939941)
\curveto(226.26052683,47.2493963)(226.27052682,47.28439626)(226.27053345,47.31439941)
\curveto(226.27052682,47.35439619)(226.27552681,47.39439615)(226.28553345,47.43439941)
\curveto(226.29552679,47.49439605)(226.29552679,47.55439599)(226.28553345,47.61439941)
\curveto(226.2855268,47.67439587)(226.2905268,47.73439581)(226.30053345,47.79439941)
}
}
{
\newrgbcolor{curcolor}{0 0 0}
\pscustom[linestyle=none,fillstyle=solid,fillcolor=curcolor]
{
\newpath
\moveto(229.74678345,53.37439941)
\curveto(229.66678233,53.43439011)(229.62178237,53.53939001)(229.61178345,53.68939941)
\lineto(229.61178345,54.15439941)
\lineto(229.61178345,54.40939941)
\curveto(229.61178238,54.49938905)(229.62678237,54.57438897)(229.65678345,54.63439941)
\curveto(229.6967823,54.71438883)(229.77678222,54.77438877)(229.89678345,54.81439941)
\curveto(229.91678208,54.82438872)(229.93678206,54.82438872)(229.95678345,54.81439941)
\curveto(229.98678201,54.81438873)(230.01178198,54.81938873)(230.03178345,54.82939941)
\curveto(230.20178179,54.82938872)(230.36178163,54.82438872)(230.51178345,54.81439941)
\curveto(230.66178133,54.80438874)(230.76178123,54.7443888)(230.81178345,54.63439941)
\curveto(230.84178115,54.57438897)(230.85678114,54.49938905)(230.85678345,54.40939941)
\lineto(230.85678345,54.15439941)
\curveto(230.85678114,53.97438957)(230.85178114,53.80438974)(230.84178345,53.64439941)
\curveto(230.84178115,53.48439006)(230.77678122,53.37939017)(230.64678345,53.32939941)
\curveto(230.5967814,53.30939024)(230.54178145,53.29939025)(230.48178345,53.29939941)
\lineto(230.31678345,53.29939941)
\lineto(230.00178345,53.29939941)
\curveto(229.90178209,53.29939025)(229.81678218,53.32439022)(229.74678345,53.37439941)
\moveto(230.85678345,44.86939941)
\lineto(230.85678345,44.55439941)
\curveto(230.86678113,44.45439909)(230.84678115,44.37439917)(230.79678345,44.31439941)
\curveto(230.76678123,44.25439929)(230.72178127,44.21439933)(230.66178345,44.19439941)
\curveto(230.60178139,44.18439936)(230.53178146,44.16939938)(230.45178345,44.14939941)
\lineto(230.22678345,44.14939941)
\curveto(230.0967819,44.1493994)(229.98178201,44.15439939)(229.88178345,44.16439941)
\curveto(229.7917822,44.18439936)(229.72178227,44.23439931)(229.67178345,44.31439941)
\curveto(229.63178236,44.37439917)(229.61178238,44.4493991)(229.61178345,44.53939941)
\lineto(229.61178345,44.82439941)
\lineto(229.61178345,51.16939941)
\lineto(229.61178345,51.48439941)
\curveto(229.61178238,51.59439195)(229.63678236,51.67939187)(229.68678345,51.73939941)
\curveto(229.71678228,51.78939176)(229.75678224,51.81939173)(229.80678345,51.82939941)
\curveto(229.85678214,51.83939171)(229.91178208,51.85439169)(229.97178345,51.87439941)
\curveto(229.991782,51.87439167)(230.01178198,51.86939168)(230.03178345,51.85939941)
\curveto(230.06178193,51.85939169)(230.08678191,51.86439168)(230.10678345,51.87439941)
\curveto(230.23678176,51.87439167)(230.36678163,51.86939168)(230.49678345,51.85939941)
\curveto(230.63678136,51.85939169)(230.73178126,51.81939173)(230.78178345,51.73939941)
\curveto(230.83178116,51.67939187)(230.85678114,51.59939195)(230.85678345,51.49939941)
\lineto(230.85678345,51.21439941)
\lineto(230.85678345,44.86939941)
}
}
{
\newrgbcolor{curcolor}{0 0 0}
\pscustom[linestyle=none,fillstyle=solid,fillcolor=curcolor]
{
\newpath
\moveto(239.6866272,44.70439941)
\curveto(239.71661937,44.544399)(239.70161938,44.40939914)(239.6416272,44.29939941)
\curveto(239.5816195,44.19939935)(239.50161958,44.12439942)(239.4016272,44.07439941)
\curveto(239.35161973,44.05439949)(239.29661979,44.0443995)(239.2366272,44.04439941)
\curveto(239.1866199,44.0443995)(239.13161995,44.03439951)(239.0716272,44.01439941)
\curveto(238.85162023,43.96439958)(238.63162045,43.97939957)(238.4116272,44.05939941)
\curveto(238.20162088,44.12939942)(238.05662103,44.21939933)(237.9766272,44.32939941)
\curveto(237.92662116,44.39939915)(237.8816212,44.47939907)(237.8416272,44.56939941)
\curveto(237.80162128,44.66939888)(237.75162133,44.7493988)(237.6916272,44.80939941)
\curveto(237.67162141,44.82939872)(237.64662144,44.8493987)(237.6166272,44.86939941)
\curveto(237.59662149,44.88939866)(237.56662152,44.89439865)(237.5266272,44.88439941)
\curveto(237.41662167,44.85439869)(237.31162177,44.79939875)(237.2116272,44.71939941)
\curveto(237.12162196,44.63939891)(237.03162205,44.56939898)(236.9416272,44.50939941)
\curveto(236.81162227,44.42939912)(236.67162241,44.35439919)(236.5216272,44.28439941)
\curveto(236.37162271,44.22439932)(236.21162287,44.16939938)(236.0416272,44.11939941)
\curveto(235.94162314,44.08939946)(235.83162325,44.06939948)(235.7116272,44.05939941)
\curveto(235.60162348,44.0493995)(235.49162359,44.03439951)(235.3816272,44.01439941)
\curveto(235.33162375,44.00439954)(235.2866238,43.99939955)(235.2466272,43.99939941)
\lineto(235.1416272,43.99939941)
\curveto(235.03162405,43.97939957)(234.92662416,43.97939957)(234.8266272,43.99939941)
\lineto(234.6916272,43.99939941)
\curveto(234.64162444,44.00939954)(234.59162449,44.01439953)(234.5416272,44.01439941)
\curveto(234.49162459,44.01439953)(234.44662464,44.02439952)(234.4066272,44.04439941)
\curveto(234.36662472,44.05439949)(234.33162475,44.05939949)(234.3016272,44.05939941)
\curveto(234.2816248,44.0493995)(234.25662483,44.0493995)(234.2266272,44.05939941)
\lineto(233.9866272,44.11939941)
\curveto(233.90662518,44.12939942)(233.83162525,44.1493994)(233.7616272,44.17939941)
\curveto(233.46162562,44.30939924)(233.21662587,44.45439909)(233.0266272,44.61439941)
\curveto(232.84662624,44.78439876)(232.69662639,45.01939853)(232.5766272,45.31939941)
\curveto(232.4866266,45.53939801)(232.44162664,45.80439774)(232.4416272,46.11439941)
\lineto(232.4416272,46.42939941)
\curveto(232.45162663,46.47939707)(232.45662663,46.52939702)(232.4566272,46.57939941)
\lineto(232.4866272,46.75939941)
\lineto(232.6066272,47.08939941)
\curveto(232.64662644,47.19939635)(232.69662639,47.29939625)(232.7566272,47.38939941)
\curveto(232.93662615,47.67939587)(233.1816259,47.89439565)(233.4916272,48.03439941)
\curveto(233.80162528,48.17439537)(234.14162494,48.29939525)(234.5116272,48.40939941)
\curveto(234.65162443,48.4493951)(234.79662429,48.47939507)(234.9466272,48.49939941)
\curveto(235.09662399,48.51939503)(235.24662384,48.544395)(235.3966272,48.57439941)
\curveto(235.46662362,48.59439495)(235.53162355,48.60439494)(235.5916272,48.60439941)
\curveto(235.66162342,48.60439494)(235.73662335,48.61439493)(235.8166272,48.63439941)
\curveto(235.8866232,48.65439489)(235.95662313,48.66439488)(236.0266272,48.66439941)
\curveto(236.09662299,48.67439487)(236.17162291,48.68939486)(236.2516272,48.70939941)
\curveto(236.50162258,48.76939478)(236.73662235,48.81939473)(236.9566272,48.85939941)
\curveto(237.17662191,48.90939464)(237.35162173,49.02439452)(237.4816272,49.20439941)
\curveto(237.54162154,49.28439426)(237.59162149,49.38439416)(237.6316272,49.50439941)
\curveto(237.67162141,49.63439391)(237.67162141,49.77439377)(237.6316272,49.92439941)
\curveto(237.57162151,50.16439338)(237.4816216,50.35439319)(237.3616272,50.49439941)
\curveto(237.25162183,50.63439291)(237.09162199,50.7443928)(236.8816272,50.82439941)
\curveto(236.76162232,50.87439267)(236.61662247,50.90939264)(236.4466272,50.92939941)
\curveto(236.2866228,50.9493926)(236.11662297,50.95939259)(235.9366272,50.95939941)
\curveto(235.75662333,50.95939259)(235.5816235,50.9493926)(235.4116272,50.92939941)
\curveto(235.24162384,50.90939264)(235.09662399,50.87939267)(234.9766272,50.83939941)
\curveto(234.80662428,50.77939277)(234.64162444,50.69439285)(234.4816272,50.58439941)
\curveto(234.40162468,50.52439302)(234.32662476,50.4443931)(234.2566272,50.34439941)
\curveto(234.19662489,50.25439329)(234.14162494,50.15439339)(234.0916272,50.04439941)
\curveto(234.06162502,49.96439358)(234.03162505,49.87939367)(234.0016272,49.78939941)
\curveto(233.9816251,49.69939385)(233.93662515,49.62939392)(233.8666272,49.57939941)
\curveto(233.82662526,49.549394)(233.75662533,49.52439402)(233.6566272,49.50439941)
\curveto(233.56662552,49.49439405)(233.47162561,49.48939406)(233.3716272,49.48939941)
\curveto(233.27162581,49.48939406)(233.17162591,49.49439405)(233.0716272,49.50439941)
\curveto(232.9816261,49.52439402)(232.91662617,49.549394)(232.8766272,49.57939941)
\curveto(232.83662625,49.60939394)(232.80662628,49.65939389)(232.7866272,49.72939941)
\curveto(232.76662632,49.79939375)(232.76662632,49.87439367)(232.7866272,49.95439941)
\curveto(232.81662627,50.08439346)(232.84662624,50.20439334)(232.8766272,50.31439941)
\curveto(232.91662617,50.43439311)(232.96162612,50.549393)(233.0116272,50.65939941)
\curveto(233.20162588,51.00939254)(233.44162564,51.27939227)(233.7316272,51.46939941)
\curveto(234.02162506,51.66939188)(234.3816247,51.82939172)(234.8116272,51.94939941)
\curveto(234.91162417,51.96939158)(235.01162407,51.98439156)(235.1116272,51.99439941)
\curveto(235.22162386,52.00439154)(235.33162375,52.01939153)(235.4416272,52.03939941)
\curveto(235.4816236,52.0493915)(235.54662354,52.0493915)(235.6366272,52.03939941)
\curveto(235.72662336,52.03939151)(235.7816233,52.0493915)(235.8016272,52.06939941)
\curveto(236.50162258,52.07939147)(237.11162197,51.99939155)(237.6316272,51.82939941)
\curveto(238.15162093,51.65939189)(238.51662057,51.33439221)(238.7266272,50.85439941)
\curveto(238.81662027,50.65439289)(238.86662022,50.41939313)(238.8766272,50.14939941)
\curveto(238.89662019,49.88939366)(238.90662018,49.61439393)(238.9066272,49.32439941)
\lineto(238.9066272,46.00939941)
\curveto(238.90662018,45.86939768)(238.91162017,45.73439781)(238.9216272,45.60439941)
\curveto(238.93162015,45.47439807)(238.96162012,45.36939818)(239.0116272,45.28939941)
\curveto(239.06162002,45.21939833)(239.12661996,45.16939838)(239.2066272,45.13939941)
\curveto(239.29661979,45.09939845)(239.3816197,45.06939848)(239.4616272,45.04939941)
\curveto(239.54161954,45.03939851)(239.60161948,44.99439855)(239.6416272,44.91439941)
\curveto(239.66161942,44.88439866)(239.67161941,44.85439869)(239.6716272,44.82439941)
\curveto(239.67161941,44.79439875)(239.67661941,44.75439879)(239.6866272,44.70439941)
\moveto(237.5416272,46.36939941)
\curveto(237.60162148,46.50939704)(237.63162145,46.66939688)(237.6316272,46.84939941)
\curveto(237.64162144,47.03939651)(237.64662144,47.23439631)(237.6466272,47.43439941)
\curveto(237.64662144,47.544396)(237.64162144,47.6443959)(237.6316272,47.73439941)
\curveto(237.62162146,47.82439572)(237.5816215,47.89439565)(237.5116272,47.94439941)
\curveto(237.4816216,47.96439558)(237.41162167,47.97439557)(237.3016272,47.97439941)
\curveto(237.2816218,47.95439559)(237.24662184,47.9443956)(237.1966272,47.94439941)
\curveto(237.14662194,47.9443956)(237.10162198,47.93439561)(237.0616272,47.91439941)
\curveto(236.9816221,47.89439565)(236.89162219,47.87439567)(236.7916272,47.85439941)
\lineto(236.4916272,47.79439941)
\curveto(236.46162262,47.79439575)(236.42662266,47.78939576)(236.3866272,47.77939941)
\lineto(236.2816272,47.77939941)
\curveto(236.13162295,47.73939581)(235.96662312,47.71439583)(235.7866272,47.70439941)
\curveto(235.61662347,47.70439584)(235.45662363,47.68439586)(235.3066272,47.64439941)
\curveto(235.22662386,47.62439592)(235.15162393,47.60439594)(235.0816272,47.58439941)
\curveto(235.02162406,47.57439597)(234.95162413,47.55939599)(234.8716272,47.53939941)
\curveto(234.71162437,47.48939606)(234.56162452,47.42439612)(234.4216272,47.34439941)
\curveto(234.2816248,47.27439627)(234.16162492,47.18439636)(234.0616272,47.07439941)
\curveto(233.96162512,46.96439658)(233.8866252,46.82939672)(233.8366272,46.66939941)
\curveto(233.7866253,46.51939703)(233.76662532,46.33439721)(233.7766272,46.11439941)
\curveto(233.77662531,46.01439753)(233.79162529,45.91939763)(233.8216272,45.82939941)
\curveto(233.86162522,45.7493978)(233.90662518,45.67439787)(233.9566272,45.60439941)
\curveto(234.03662505,45.49439805)(234.14162494,45.39939815)(234.2716272,45.31939941)
\curveto(234.40162468,45.2493983)(234.54162454,45.18939836)(234.6916272,45.13939941)
\curveto(234.74162434,45.12939842)(234.79162429,45.12439842)(234.8416272,45.12439941)
\curveto(234.89162419,45.12439842)(234.94162414,45.11939843)(234.9916272,45.10939941)
\curveto(235.06162402,45.08939846)(235.14662394,45.07439847)(235.2466272,45.06439941)
\curveto(235.35662373,45.06439848)(235.44662364,45.07439847)(235.5166272,45.09439941)
\curveto(235.57662351,45.11439843)(235.63662345,45.11939843)(235.6966272,45.10939941)
\curveto(235.75662333,45.10939844)(235.81662327,45.11939843)(235.8766272,45.13939941)
\curveto(235.95662313,45.15939839)(236.03162305,45.17439837)(236.1016272,45.18439941)
\curveto(236.1816229,45.19439835)(236.25662283,45.21439833)(236.3266272,45.24439941)
\curveto(236.61662247,45.36439818)(236.86162222,45.50939804)(237.0616272,45.67939941)
\curveto(237.27162181,45.8493977)(237.43162165,46.07939747)(237.5416272,46.36939941)
}
}
{
\newrgbcolor{curcolor}{0 0 0}
\pscustom[linestyle=none,fillstyle=solid,fillcolor=curcolor]
{
\newpath
\moveto(244.54826782,52.02439941)
\curveto(245.17826259,52.0443915)(245.68326208,51.95939159)(246.06326782,51.76939941)
\curveto(246.44326132,51.57939197)(246.74826102,51.29439225)(246.97826782,50.91439941)
\curveto(247.03826073,50.81439273)(247.08326068,50.70439284)(247.11326782,50.58439941)
\curveto(247.15326061,50.47439307)(247.18826058,50.35939319)(247.21826782,50.23939941)
\curveto(247.2682605,50.0493935)(247.29826047,49.8443937)(247.30826782,49.62439941)
\curveto(247.31826045,49.40439414)(247.32326044,49.17939437)(247.32326782,48.94939941)
\lineto(247.32326782,47.34439941)
\lineto(247.32326782,45.00439941)
\curveto(247.32326044,44.83439871)(247.31826045,44.66439888)(247.30826782,44.49439941)
\curveto(247.30826046,44.32439922)(247.24326052,44.21439933)(247.11326782,44.16439941)
\curveto(247.0632607,44.1443994)(247.00826076,44.13439941)(246.94826782,44.13439941)
\curveto(246.89826087,44.12439942)(246.84326092,44.11939943)(246.78326782,44.11939941)
\curveto(246.65326111,44.11939943)(246.52826124,44.12439942)(246.40826782,44.13439941)
\curveto(246.28826148,44.13439941)(246.20326156,44.17439937)(246.15326782,44.25439941)
\curveto(246.10326166,44.32439922)(246.07826169,44.41439913)(246.07826782,44.52439941)
\lineto(246.07826782,44.85439941)
\lineto(246.07826782,46.14439941)
\lineto(246.07826782,48.58939941)
\curveto(246.07826169,48.85939469)(246.07326169,49.12439442)(246.06326782,49.38439941)
\curveto(246.05326171,49.65439389)(246.00826176,49.88439366)(245.92826782,50.07439941)
\curveto(245.84826192,50.27439327)(245.72826204,50.43439311)(245.56826782,50.55439941)
\curveto(245.40826236,50.68439286)(245.22326254,50.78439276)(245.01326782,50.85439941)
\curveto(244.95326281,50.87439267)(244.88826288,50.88439266)(244.81826782,50.88439941)
\curveto(244.75826301,50.89439265)(244.69826307,50.90939264)(244.63826782,50.92939941)
\curveto(244.58826318,50.93939261)(244.50826326,50.93939261)(244.39826782,50.92939941)
\curveto(244.29826347,50.92939262)(244.22826354,50.92439262)(244.18826782,50.91439941)
\curveto(244.14826362,50.89439265)(244.11326365,50.88439266)(244.08326782,50.88439941)
\curveto(244.05326371,50.89439265)(244.01826375,50.89439265)(243.97826782,50.88439941)
\curveto(243.84826392,50.85439269)(243.72326404,50.81939273)(243.60326782,50.77939941)
\curveto(243.49326427,50.7493928)(243.38826438,50.70439284)(243.28826782,50.64439941)
\curveto(243.24826452,50.62439292)(243.21326455,50.60439294)(243.18326782,50.58439941)
\curveto(243.15326461,50.56439298)(243.11826465,50.544393)(243.07826782,50.52439941)
\curveto(242.72826504,50.27439327)(242.47326529,49.89939365)(242.31326782,49.39939941)
\curveto(242.28326548,49.31939423)(242.2632655,49.23439431)(242.25326782,49.14439941)
\curveto(242.24326552,49.06439448)(242.22826554,48.98439456)(242.20826782,48.90439941)
\curveto(242.18826558,48.85439469)(242.18326558,48.80439474)(242.19326782,48.75439941)
\curveto(242.20326556,48.71439483)(242.19826557,48.67439487)(242.17826782,48.63439941)
\lineto(242.17826782,48.31939941)
\curveto(242.1682656,48.28939526)(242.1632656,48.25439529)(242.16326782,48.21439941)
\curveto(242.17326559,48.17439537)(242.17826559,48.12939542)(242.17826782,48.07939941)
\lineto(242.17826782,47.62939941)
\lineto(242.17826782,46.18939941)
\lineto(242.17826782,44.86939941)
\lineto(242.17826782,44.52439941)
\curveto(242.17826559,44.41439913)(242.15326561,44.32439922)(242.10326782,44.25439941)
\curveto(242.05326571,44.17439937)(241.9632658,44.13439941)(241.83326782,44.13439941)
\curveto(241.71326605,44.12439942)(241.58826618,44.11939943)(241.45826782,44.11939941)
\curveto(241.37826639,44.11939943)(241.30326646,44.12439942)(241.23326782,44.13439941)
\curveto(241.1632666,44.1443994)(241.10326666,44.16939938)(241.05326782,44.20939941)
\curveto(240.97326679,44.25939929)(240.93326683,44.35439919)(240.93326782,44.49439941)
\lineto(240.93326782,44.89939941)
\lineto(240.93326782,46.66939941)
\lineto(240.93326782,50.29939941)
\lineto(240.93326782,51.21439941)
\lineto(240.93326782,51.48439941)
\curveto(240.93326683,51.57439197)(240.95326681,51.6443919)(240.99326782,51.69439941)
\curveto(241.02326674,51.75439179)(241.07326669,51.79439175)(241.14326782,51.81439941)
\curveto(241.18326658,51.82439172)(241.23826653,51.83439171)(241.30826782,51.84439941)
\curveto(241.38826638,51.85439169)(241.4682663,51.85939169)(241.54826782,51.85939941)
\curveto(241.62826614,51.85939169)(241.70326606,51.85439169)(241.77326782,51.84439941)
\curveto(241.85326591,51.83439171)(241.90826586,51.81939173)(241.93826782,51.79939941)
\curveto(242.04826572,51.72939182)(242.09826567,51.63939191)(242.08826782,51.52939941)
\curveto(242.07826569,51.42939212)(242.09326567,51.31439223)(242.13326782,51.18439941)
\curveto(242.15326561,51.12439242)(242.19326557,51.07439247)(242.25326782,51.03439941)
\curveto(242.37326539,51.02439252)(242.4682653,51.06939248)(242.53826782,51.16939941)
\curveto(242.61826515,51.26939228)(242.69826507,51.3493922)(242.77826782,51.40939941)
\curveto(242.91826485,51.50939204)(243.05826471,51.59939195)(243.19826782,51.67939941)
\curveto(243.34826442,51.76939178)(243.51826425,51.8443917)(243.70826782,51.90439941)
\curveto(243.78826398,51.93439161)(243.87326389,51.95439159)(243.96326782,51.96439941)
\curveto(244.0632637,51.97439157)(244.15826361,51.98939156)(244.24826782,52.00939941)
\curveto(244.29826347,52.01939153)(244.34826342,52.02439152)(244.39826782,52.02439941)
\lineto(244.54826782,52.02439941)
}
}
{
\newrgbcolor{curcolor}{0 0 0}
\pscustom[linestyle=none,fillstyle=solid,fillcolor=curcolor]
{
\newpath
\moveto(250.1528772,54.21439941)
\curveto(250.30287519,54.21438933)(250.45287504,54.20938934)(250.6028772,54.19939941)
\curveto(250.75287474,54.19938935)(250.85787463,54.15938939)(250.9178772,54.07939941)
\curveto(250.96787452,54.01938953)(250.9928745,53.93438961)(250.9928772,53.82439941)
\curveto(251.00287449,53.72438982)(251.00787448,53.61938993)(251.0078772,53.50939941)
\lineto(251.0078772,52.63939941)
\curveto(251.00787448,52.55939099)(251.00287449,52.47439107)(250.9928772,52.38439941)
\curveto(250.9928745,52.30439124)(251.00287449,52.23439131)(251.0228772,52.17439941)
\curveto(251.06287443,52.03439151)(251.15287434,51.9443916)(251.2928772,51.90439941)
\curveto(251.34287415,51.89439165)(251.3878741,51.88939166)(251.4278772,51.88939941)
\lineto(251.5778772,51.88939941)
\lineto(251.9828772,51.88939941)
\curveto(252.14287335,51.89939165)(252.25787323,51.88939166)(252.3278772,51.85939941)
\curveto(252.41787307,51.79939175)(252.47787301,51.73939181)(252.5078772,51.67939941)
\curveto(252.52787296,51.63939191)(252.53787295,51.59439195)(252.5378772,51.54439941)
\lineto(252.5378772,51.39439941)
\curveto(252.53787295,51.28439226)(252.53287296,51.17939237)(252.5228772,51.07939941)
\curveto(252.51287298,50.98939256)(252.47787301,50.91939263)(252.4178772,50.86939941)
\curveto(252.35787313,50.81939273)(252.27287322,50.78939276)(252.1628772,50.77939941)
\lineto(251.8328772,50.77939941)
\curveto(251.72287377,50.78939276)(251.61287388,50.79439275)(251.5028772,50.79439941)
\curveto(251.3928741,50.79439275)(251.29787419,50.77939277)(251.2178772,50.74939941)
\curveto(251.14787434,50.71939283)(251.09787439,50.66939288)(251.0678772,50.59939941)
\curveto(251.03787445,50.52939302)(251.01787447,50.4443931)(251.0078772,50.34439941)
\curveto(250.99787449,50.25439329)(250.9928745,50.15439339)(250.9928772,50.04439941)
\curveto(251.00287449,49.9443936)(251.00787448,49.8443937)(251.0078772,49.74439941)
\lineto(251.0078772,46.77439941)
\curveto(251.00787448,46.55439699)(251.00287449,46.31939723)(250.9928772,46.06939941)
\curveto(250.9928745,45.82939772)(251.03787445,45.6443979)(251.1278772,45.51439941)
\curveto(251.17787431,45.43439811)(251.24287425,45.37939817)(251.3228772,45.34939941)
\curveto(251.40287409,45.31939823)(251.49787399,45.29439825)(251.6078772,45.27439941)
\curveto(251.63787385,45.26439828)(251.66787382,45.25939829)(251.6978772,45.25939941)
\curveto(251.73787375,45.26939828)(251.77287372,45.26939828)(251.8028772,45.25939941)
\lineto(251.9978772,45.25939941)
\curveto(252.09787339,45.25939829)(252.1878733,45.2493983)(252.2678772,45.22939941)
\curveto(252.35787313,45.21939833)(252.42287307,45.18439836)(252.4628772,45.12439941)
\curveto(252.48287301,45.09439845)(252.49787299,45.03939851)(252.5078772,44.95939941)
\curveto(252.52787296,44.88939866)(252.53787295,44.81439873)(252.5378772,44.73439941)
\curveto(252.54787294,44.65439889)(252.54787294,44.57439897)(252.5378772,44.49439941)
\curveto(252.52787296,44.42439912)(252.50787298,44.36939918)(252.4778772,44.32939941)
\curveto(252.43787305,44.25939929)(252.36287313,44.20939934)(252.2528772,44.17939941)
\curveto(252.17287332,44.15939939)(252.08287341,44.1493994)(251.9828772,44.14939941)
\curveto(251.88287361,44.15939939)(251.7928737,44.16439938)(251.7128772,44.16439941)
\curveto(251.65287384,44.16439938)(251.5928739,44.15939939)(251.5328772,44.14939941)
\curveto(251.47287402,44.1493994)(251.41787407,44.15439939)(251.3678772,44.16439941)
\lineto(251.1878772,44.16439941)
\curveto(251.13787435,44.17439937)(251.0878744,44.17939937)(251.0378772,44.17939941)
\curveto(250.99787449,44.18939936)(250.95287454,44.19439935)(250.9028772,44.19439941)
\curveto(250.70287479,44.2443993)(250.52787496,44.29939925)(250.3778772,44.35939941)
\curveto(250.23787525,44.41939913)(250.11787537,44.52439902)(250.0178772,44.67439941)
\curveto(249.87787561,44.87439867)(249.79787569,45.12439842)(249.7778772,45.42439941)
\curveto(249.75787573,45.73439781)(249.74787574,46.06439748)(249.7478772,46.41439941)
\lineto(249.7478772,50.34439941)
\curveto(249.71787577,50.47439307)(249.6878758,50.56939298)(249.6578772,50.62939941)
\curveto(249.63787585,50.68939286)(249.56787592,50.73939281)(249.4478772,50.77939941)
\curveto(249.40787608,50.78939276)(249.36787612,50.78939276)(249.3278772,50.77939941)
\curveto(249.2878762,50.76939278)(249.24787624,50.77439277)(249.2078772,50.79439941)
\lineto(248.9678772,50.79439941)
\curveto(248.83787665,50.79439275)(248.72787676,50.80439274)(248.6378772,50.82439941)
\curveto(248.55787693,50.85439269)(248.50287699,50.91439263)(248.4728772,51.00439941)
\curveto(248.45287704,51.0443925)(248.43787705,51.08939246)(248.4278772,51.13939941)
\lineto(248.4278772,51.28939941)
\curveto(248.42787706,51.42939212)(248.43787705,51.544392)(248.4578772,51.63439941)
\curveto(248.47787701,51.73439181)(248.53787695,51.80939174)(248.6378772,51.85939941)
\curveto(248.74787674,51.89939165)(248.8878766,51.90939164)(249.0578772,51.88939941)
\curveto(249.23787625,51.86939168)(249.3878761,51.87939167)(249.5078772,51.91939941)
\curveto(249.59787589,51.96939158)(249.66787582,52.03939151)(249.7178772,52.12939941)
\curveto(249.73787575,52.18939136)(249.74787574,52.26439128)(249.7478772,52.35439941)
\lineto(249.7478772,52.60939941)
\lineto(249.7478772,53.53939941)
\lineto(249.7478772,53.77939941)
\curveto(249.74787574,53.86938968)(249.75787573,53.9443896)(249.7778772,54.00439941)
\curveto(249.81787567,54.08438946)(249.8928756,54.1493894)(250.0028772,54.19939941)
\curveto(250.03287546,54.19938935)(250.05787543,54.19938935)(250.0778772,54.19939941)
\curveto(250.10787538,54.20938934)(250.13287536,54.21438933)(250.1528772,54.21439941)
}
}
{
\newrgbcolor{curcolor}{0 0 0}
\pscustom[linestyle=none,fillstyle=solid,fillcolor=curcolor]
{
\newpath
\moveto(260.67467407,48.31939941)
\curveto(260.69466639,48.21939533)(260.69466639,48.10439544)(260.67467407,47.97439941)
\curveto(260.66466642,47.85439569)(260.63466645,47.76939578)(260.58467407,47.71939941)
\curveto(260.53466655,47.67939587)(260.45966662,47.6493959)(260.35967407,47.62939941)
\curveto(260.26966681,47.61939593)(260.16466692,47.61439593)(260.04467407,47.61439941)
\lineto(259.68467407,47.61439941)
\curveto(259.56466752,47.62439592)(259.45966762,47.62939592)(259.36967407,47.62939941)
\lineto(255.52967407,47.62939941)
\curveto(255.44967163,47.62939592)(255.36967171,47.62439592)(255.28967407,47.61439941)
\curveto(255.20967187,47.61439593)(255.14467194,47.59939595)(255.09467407,47.56939941)
\curveto(255.05467203,47.549396)(255.01467207,47.50939604)(254.97467407,47.44939941)
\curveto(254.95467213,47.41939613)(254.93467215,47.37439617)(254.91467407,47.31439941)
\curveto(254.89467219,47.26439628)(254.89467219,47.21439633)(254.91467407,47.16439941)
\curveto(254.92467216,47.11439643)(254.92967215,47.06939648)(254.92967407,47.02939941)
\curveto(254.92967215,46.98939656)(254.93467215,46.9493966)(254.94467407,46.90939941)
\curveto(254.96467212,46.82939672)(254.9846721,46.7443968)(255.00467407,46.65439941)
\curveto(255.02467206,46.57439697)(255.05467203,46.49439705)(255.09467407,46.41439941)
\curveto(255.32467176,45.87439767)(255.70467138,45.48939806)(256.23467407,45.25939941)
\curveto(256.29467079,45.22939832)(256.35967072,45.20439834)(256.42967407,45.18439941)
\lineto(256.63967407,45.12439941)
\curveto(256.66967041,45.11439843)(256.71967036,45.10939844)(256.78967407,45.10939941)
\curveto(256.92967015,45.06939848)(257.11466997,45.0493985)(257.34467407,45.04939941)
\curveto(257.57466951,45.0493985)(257.75966932,45.06939848)(257.89967407,45.10939941)
\curveto(258.03966904,45.1493984)(258.16466892,45.18939836)(258.27467407,45.22939941)
\curveto(258.39466869,45.27939827)(258.50466858,45.33939821)(258.60467407,45.40939941)
\curveto(258.71466837,45.47939807)(258.80966827,45.55939799)(258.88967407,45.64939941)
\curveto(258.96966811,45.7493978)(259.03966804,45.85439769)(259.09967407,45.96439941)
\curveto(259.15966792,46.06439748)(259.20966787,46.16939738)(259.24967407,46.27939941)
\curveto(259.29966778,46.38939716)(259.3796677,46.46939708)(259.48967407,46.51939941)
\curveto(259.52966755,46.53939701)(259.59466749,46.55439699)(259.68467407,46.56439941)
\curveto(259.77466731,46.57439697)(259.86466722,46.57439697)(259.95467407,46.56439941)
\curveto(260.04466704,46.56439698)(260.12966695,46.55939699)(260.20967407,46.54939941)
\curveto(260.28966679,46.53939701)(260.34466674,46.51939703)(260.37467407,46.48939941)
\curveto(260.47466661,46.41939713)(260.49966658,46.30439724)(260.44967407,46.14439941)
\curveto(260.36966671,45.87439767)(260.26466682,45.63439791)(260.13467407,45.42439941)
\curveto(259.93466715,45.10439844)(259.70466738,44.83939871)(259.44467407,44.62939941)
\curveto(259.19466789,44.42939912)(258.87466821,44.26439928)(258.48467407,44.13439941)
\curveto(258.3846687,44.09439945)(258.2846688,44.06939948)(258.18467407,44.05939941)
\curveto(258.084669,44.03939951)(257.9796691,44.01939953)(257.86967407,43.99939941)
\curveto(257.81966926,43.98939956)(257.76966931,43.98439956)(257.71967407,43.98439941)
\curveto(257.6796694,43.98439956)(257.63466945,43.97939957)(257.58467407,43.96939941)
\lineto(257.43467407,43.96939941)
\curveto(257.3846697,43.95939959)(257.32466976,43.95439959)(257.25467407,43.95439941)
\curveto(257.19466989,43.95439959)(257.14466994,43.95939959)(257.10467407,43.96939941)
\lineto(256.96967407,43.96939941)
\curveto(256.91967016,43.97939957)(256.87467021,43.98439956)(256.83467407,43.98439941)
\curveto(256.79467029,43.98439956)(256.75467033,43.98939956)(256.71467407,43.99939941)
\curveto(256.66467042,44.00939954)(256.60967047,44.01939953)(256.54967407,44.02939941)
\curveto(256.48967059,44.02939952)(256.43467065,44.03439951)(256.38467407,44.04439941)
\curveto(256.29467079,44.06439948)(256.20467088,44.08939946)(256.11467407,44.11939941)
\curveto(256.02467106,44.13939941)(255.93967114,44.16439938)(255.85967407,44.19439941)
\curveto(255.81967126,44.21439933)(255.7846713,44.22439932)(255.75467407,44.22439941)
\curveto(255.72467136,44.23439931)(255.68967139,44.2493993)(255.64967407,44.26939941)
\curveto(255.49967158,44.33939921)(255.33967174,44.42439912)(255.16967407,44.52439941)
\curveto(254.8796722,44.71439883)(254.62967245,44.9443986)(254.41967407,45.21439941)
\curveto(254.21967286,45.49439805)(254.04967303,45.80439774)(253.90967407,46.14439941)
\curveto(253.85967322,46.25439729)(253.81967326,46.36939718)(253.78967407,46.48939941)
\curveto(253.76967331,46.60939694)(253.73967334,46.72939682)(253.69967407,46.84939941)
\curveto(253.68967339,46.88939666)(253.6846734,46.92439662)(253.68467407,46.95439941)
\curveto(253.6846734,46.98439656)(253.6796734,47.02439652)(253.66967407,47.07439941)
\curveto(253.64967343,47.15439639)(253.63467345,47.23939631)(253.62467407,47.32939941)
\curveto(253.61467347,47.41939613)(253.59967348,47.50939604)(253.57967407,47.59939941)
\lineto(253.57967407,47.80939941)
\curveto(253.56967351,47.8493957)(253.55967352,47.90439564)(253.54967407,47.97439941)
\curveto(253.54967353,48.05439549)(253.55467353,48.11939543)(253.56467407,48.16939941)
\lineto(253.56467407,48.33439941)
\curveto(253.5846735,48.38439516)(253.58967349,48.43439511)(253.57967407,48.48439941)
\curveto(253.5796735,48.544395)(253.5846735,48.59939495)(253.59467407,48.64939941)
\curveto(253.63467345,48.80939474)(253.66467342,48.96939458)(253.68467407,49.12939941)
\curveto(253.71467337,49.28939426)(253.75967332,49.43939411)(253.81967407,49.57939941)
\curveto(253.86967321,49.68939386)(253.91467317,49.79939375)(253.95467407,49.90939941)
\curveto(254.00467308,50.02939352)(254.05967302,50.1443934)(254.11967407,50.25439941)
\curveto(254.33967274,50.60439294)(254.58967249,50.90439264)(254.86967407,51.15439941)
\curveto(255.14967193,51.41439213)(255.49467159,51.62939192)(255.90467407,51.79939941)
\curveto(256.02467106,51.8493917)(256.14467094,51.88439166)(256.26467407,51.90439941)
\curveto(256.39467069,51.93439161)(256.52967055,51.96439158)(256.66967407,51.99439941)
\curveto(256.71967036,52.00439154)(256.76467032,52.00939154)(256.80467407,52.00939941)
\curveto(256.84467024,52.01939153)(256.88967019,52.02439152)(256.93967407,52.02439941)
\curveto(256.95967012,52.03439151)(256.9846701,52.03439151)(257.01467407,52.02439941)
\curveto(257.04467004,52.01439153)(257.06967001,52.01939153)(257.08967407,52.03939941)
\curveto(257.50966957,52.0493915)(257.87466921,52.00439154)(258.18467407,51.90439941)
\curveto(258.49466859,51.81439173)(258.77466831,51.68939186)(259.02467407,51.52939941)
\curveto(259.07466801,51.50939204)(259.11466797,51.47939207)(259.14467407,51.43939941)
\curveto(259.17466791,51.40939214)(259.20966787,51.38439216)(259.24967407,51.36439941)
\curveto(259.32966775,51.30439224)(259.40966767,51.23439231)(259.48967407,51.15439941)
\curveto(259.5796675,51.07439247)(259.65466743,50.99439255)(259.71467407,50.91439941)
\curveto(259.87466721,50.70439284)(260.00966707,50.50439304)(260.11967407,50.31439941)
\curveto(260.18966689,50.20439334)(260.24466684,50.08439346)(260.28467407,49.95439941)
\curveto(260.32466676,49.82439372)(260.36966671,49.69439385)(260.41967407,49.56439941)
\curveto(260.46966661,49.43439411)(260.50466658,49.29939425)(260.52467407,49.15939941)
\curveto(260.55466653,49.01939453)(260.58966649,48.87939467)(260.62967407,48.73939941)
\curveto(260.63966644,48.66939488)(260.64466644,48.59939495)(260.64467407,48.52939941)
\lineto(260.67467407,48.31939941)
\moveto(259.21967407,48.82939941)
\curveto(259.24966783,48.86939468)(259.27466781,48.91939463)(259.29467407,48.97939941)
\curveto(259.31466777,49.0493945)(259.31466777,49.11939443)(259.29467407,49.18939941)
\curveto(259.23466785,49.40939414)(259.14966793,49.61439393)(259.03967407,49.80439941)
\curveto(258.89966818,50.03439351)(258.74466834,50.22939332)(258.57467407,50.38939941)
\curveto(258.40466868,50.549393)(258.1846689,50.68439286)(257.91467407,50.79439941)
\curveto(257.84466924,50.81439273)(257.77466931,50.82939272)(257.70467407,50.83939941)
\curveto(257.63466945,50.85939269)(257.55966952,50.87939267)(257.47967407,50.89939941)
\curveto(257.39966968,50.91939263)(257.31466977,50.92939262)(257.22467407,50.92939941)
\lineto(256.96967407,50.92939941)
\curveto(256.93967014,50.90939264)(256.90467018,50.89939265)(256.86467407,50.89939941)
\curveto(256.82467026,50.90939264)(256.78967029,50.90939264)(256.75967407,50.89939941)
\lineto(256.51967407,50.83939941)
\curveto(256.44967063,50.82939272)(256.3796707,50.81439273)(256.30967407,50.79439941)
\curveto(256.01967106,50.67439287)(255.7846713,50.52439302)(255.60467407,50.34439941)
\curveto(255.43467165,50.16439338)(255.2796718,49.93939361)(255.13967407,49.66939941)
\curveto(255.10967197,49.61939393)(255.079672,49.55439399)(255.04967407,49.47439941)
\curveto(255.01967206,49.40439414)(254.99467209,49.32439422)(254.97467407,49.23439941)
\curveto(254.95467213,49.1443944)(254.94967213,49.05939449)(254.95967407,48.97939941)
\curveto(254.96967211,48.89939465)(255.00467208,48.83939471)(255.06467407,48.79939941)
\curveto(255.14467194,48.73939481)(255.2796718,48.70939484)(255.46967407,48.70939941)
\curveto(255.66967141,48.71939483)(255.83967124,48.72439482)(255.97967407,48.72439941)
\lineto(258.25967407,48.72439941)
\curveto(258.40966867,48.72439482)(258.58966849,48.71939483)(258.79967407,48.70939941)
\curveto(259.00966807,48.70939484)(259.14966793,48.7493948)(259.21967407,48.82939941)
}
}
{
\newrgbcolor{curcolor}{0 0 0}
\pscustom[linestyle=none,fillstyle=solid,fillcolor=curcolor]
{
\newpath
\moveto(388.67456909,49.72938477)
\lineto(388.67456909,49.45938477)
\curveto(388.68455912,49.36937952)(388.67955913,49.2893796)(388.65956909,49.21938477)
\lineto(388.65956909,49.06938477)
\curveto(388.64955916,49.03937985)(388.64455916,49.00437988)(388.64456909,48.96438477)
\curveto(388.65455915,48.92437996)(388.65455915,48.89437999)(388.64456909,48.87438477)
\curveto(388.63455917,48.82438006)(388.62955918,48.76938012)(388.62956909,48.70938477)
\curveto(388.62955918,48.65938023)(388.62455918,48.60938028)(388.61456909,48.55938477)
\curveto(388.58455922,48.41938047)(388.56455924,48.26938062)(388.55456909,48.10938477)
\curveto(388.54455926,47.95938093)(388.51455929,47.81438107)(388.46456909,47.67438477)
\curveto(388.43455937,47.55438133)(388.39955941,47.42938146)(388.35956909,47.29938477)
\curveto(388.32955948,47.17938171)(388.28955952,47.05938183)(388.23956909,46.93938477)
\curveto(388.06955974,46.50938238)(387.85455995,46.11938277)(387.59456909,45.76938477)
\curveto(387.34456046,45.42938346)(387.02956078,45.13938375)(386.64956909,44.89938477)
\curveto(386.45956135,44.77938411)(386.25456155,44.67438421)(386.03456909,44.58438477)
\curveto(385.82456198,44.50438438)(385.59456221,44.42438446)(385.34456909,44.34438477)
\curveto(385.23456257,44.30438458)(385.11456269,44.27438461)(384.98456909,44.25438477)
\curveto(384.86456294,44.24438464)(384.74456306,44.22438466)(384.62456909,44.19438477)
\curveto(384.51456329,44.17438471)(384.4045634,44.16438472)(384.29456909,44.16438477)
\curveto(384.19456361,44.16438472)(384.09456371,44.15438473)(383.99456909,44.13438477)
\lineto(383.78456909,44.13438477)
\curveto(383.75456405,44.12438476)(383.71956409,44.11938477)(383.67956909,44.11938477)
\curveto(383.63956417,44.12938476)(383.59956421,44.13438475)(383.55956909,44.13438477)
\lineto(380.55956909,44.13438477)
\curveto(380.4095674,44.13438475)(380.27456753,44.13938475)(380.15456909,44.14938477)
\curveto(380.04456776,44.16938472)(379.96956784,44.23438465)(379.92956909,44.34438477)
\curveto(379.88956792,44.42438446)(379.86956794,44.53938435)(379.86956909,44.68938477)
\curveto(379.87956793,44.83938405)(379.88456792,44.97438391)(379.88456909,45.09438477)
\lineto(379.88456909,53.95938477)
\curveto(379.88456792,54.07937481)(379.87956793,54.20437468)(379.86956909,54.33438477)
\curveto(379.86956794,54.47437441)(379.89456791,54.5843743)(379.94456909,54.66438477)
\curveto(379.98456782,54.73437415)(380.05956775,54.77937411)(380.16956909,54.79938477)
\curveto(380.18956762,54.80937408)(380.2095676,54.80937408)(380.22956909,54.79938477)
\curveto(380.24956756,54.79937409)(380.26956754,54.80437408)(380.28956909,54.81438477)
\lineto(383.54456909,54.81438477)
\curveto(383.59456421,54.81437407)(383.63956417,54.81437407)(383.67956909,54.81438477)
\curveto(383.72956408,54.82437406)(383.77456403,54.82437406)(383.81456909,54.81438477)
\curveto(383.86456394,54.79437409)(383.91456389,54.7893741)(383.96456909,54.79938477)
\curveto(384.02456378,54.80937408)(384.07956373,54.80937408)(384.12956909,54.79938477)
\curveto(384.17956363,54.7893741)(384.23456357,54.7843741)(384.29456909,54.78438477)
\curveto(384.35456345,54.7843741)(384.4095634,54.77937411)(384.45956909,54.76938477)
\curveto(384.5095633,54.75937413)(384.55456325,54.75437413)(384.59456909,54.75438477)
\curveto(384.64456316,54.75437413)(384.69456311,54.74937414)(384.74456909,54.73938477)
\curveto(384.85456295,54.71937417)(384.95956285,54.69937419)(385.05956909,54.67938477)
\curveto(385.15956265,54.66937422)(385.25956255,54.64937424)(385.35956909,54.61938477)
\curveto(385.57956223,54.54937434)(385.78956202,54.47937441)(385.98956909,54.40938477)
\curveto(386.18956162,54.34937454)(386.37456143,54.26437462)(386.54456909,54.15438477)
\curveto(386.68456112,54.07437481)(386.809561,53.99437489)(386.91956909,53.91438477)
\curveto(386.94956086,53.89437499)(386.97956083,53.86937502)(387.00956909,53.83938477)
\curveto(387.03956077,53.81937507)(387.06956074,53.79937509)(387.09956909,53.77938477)
\curveto(387.15956065,53.72937516)(387.21456059,53.67937521)(387.26456909,53.62938477)
\curveto(387.31456049,53.57937531)(387.36456044,53.52937536)(387.41456909,53.47938477)
\curveto(387.46456034,53.42937546)(387.5045603,53.39437549)(387.53456909,53.37438477)
\curveto(387.57456023,53.31437557)(387.61456019,53.25937563)(387.65456909,53.20938477)
\curveto(387.7045601,53.15937573)(387.74956006,53.10437578)(387.78956909,53.04438477)
\curveto(387.83955997,52.9843759)(387.87955993,52.91937597)(387.90956909,52.84938477)
\curveto(387.94955986,52.7893761)(387.99455981,52.72437616)(388.04456909,52.65438477)
\curveto(388.06455974,52.61437627)(388.07955973,52.57937631)(388.08956909,52.54938477)
\curveto(388.09955971,52.51937637)(388.11455969,52.4843764)(388.13456909,52.44438477)
\curveto(388.17455963,52.36437652)(388.2095596,52.2843766)(388.23956909,52.20438477)
\curveto(388.26955954,52.13437675)(388.3045595,52.05937683)(388.34456909,51.97938477)
\curveto(388.38455942,51.86937702)(388.41455939,51.75437713)(388.43456909,51.63438477)
\curveto(388.46455934,51.52437736)(388.49455931,51.41437747)(388.52456909,51.30438477)
\curveto(388.54455926,51.24437764)(388.55455925,51.1843777)(388.55456909,51.12438477)
\curveto(388.55455925,51.07437781)(388.56455924,51.01937787)(388.58456909,50.95938477)
\curveto(388.63455917,50.77937811)(388.65955915,50.57937831)(388.65956909,50.35938477)
\curveto(388.66955914,50.14937874)(388.67455913,49.93937895)(388.67456909,49.72938477)
\moveto(387.24956909,48.94938477)
\curveto(387.26956054,49.04937984)(387.27956053,49.15437973)(387.27956909,49.26438477)
\lineto(387.27956909,49.60938477)
\lineto(387.27956909,49.83438477)
\curveto(387.28956052,49.91437897)(387.28456052,49.9893789)(387.26456909,50.05938477)
\curveto(387.26456054,50.0893788)(387.25956055,50.11937877)(387.24956909,50.14938477)
\lineto(387.24956909,50.25438477)
\curveto(387.22956058,50.36437852)(387.21456059,50.47437841)(387.20456909,50.58438477)
\curveto(387.2045606,50.69437819)(387.18956062,50.80437808)(387.15956909,50.91438477)
\curveto(387.13956067,50.99437789)(387.11956069,51.06937782)(387.09956909,51.13938477)
\curveto(387.08956072,51.21937767)(387.07456073,51.29937759)(387.05456909,51.37938477)
\curveto(386.94456086,51.73937715)(386.804561,52.05437683)(386.63456909,52.32438477)
\curveto(386.35456145,52.77437611)(385.93956187,53.11437577)(385.38956909,53.34438477)
\curveto(385.29956251,53.39437549)(385.2045626,53.42937546)(385.10456909,53.44938477)
\curveto(385.0045628,53.47937541)(384.89956291,53.50937538)(384.78956909,53.53938477)
\curveto(384.67956313,53.56937532)(384.56456324,53.5843753)(384.44456909,53.58438477)
\curveto(384.33456347,53.59437529)(384.22456358,53.60937528)(384.11456909,53.62938477)
\lineto(383.79956909,53.62938477)
\curveto(383.76956404,53.63937525)(383.73456407,53.64437524)(383.69456909,53.64438477)
\lineto(383.57456909,53.64438477)
\lineto(381.74456909,53.64438477)
\curveto(381.72456608,53.63437525)(381.69956611,53.62937526)(381.66956909,53.62938477)
\curveto(381.63956617,53.63937525)(381.61456619,53.63937525)(381.59456909,53.62938477)
\lineto(381.44456909,53.56938477)
\curveto(381.4045664,53.54937534)(381.37456643,53.51937537)(381.35456909,53.47938477)
\curveto(381.33456647,53.43937545)(381.31456649,53.36937552)(381.29456909,53.26938477)
\lineto(381.29456909,53.14938477)
\curveto(381.28456652,53.10937578)(381.27956653,53.06437582)(381.27956909,53.01438477)
\lineto(381.27956909,52.87938477)
\lineto(381.27956909,46.06938477)
\lineto(381.27956909,45.91938477)
\curveto(381.27956653,45.87938301)(381.28456652,45.83938305)(381.29456909,45.79938477)
\lineto(381.29456909,45.67938477)
\curveto(381.31456649,45.57938331)(381.33456647,45.50938338)(381.35456909,45.46938477)
\curveto(381.43456637,45.34938354)(381.58456622,45.2893836)(381.80456909,45.28938477)
\curveto(382.02456578,45.29938359)(382.23456557,45.30438358)(382.43456909,45.30438477)
\lineto(383.30456909,45.30438477)
\curveto(383.37456443,45.30438358)(383.44956436,45.29938359)(383.52956909,45.28938477)
\curveto(383.6095642,45.2893836)(383.67956413,45.29938359)(383.73956909,45.31938477)
\lineto(383.90456909,45.31938477)
\curveto(383.95456385,45.32938356)(384.0095638,45.32938356)(384.06956909,45.31938477)
\curveto(384.12956368,45.31938357)(384.18956362,45.32438356)(384.24956909,45.33438477)
\curveto(384.3095635,45.35438353)(384.36956344,45.36438352)(384.42956909,45.36438477)
\curveto(384.48956332,45.37438351)(384.55456325,45.3893835)(384.62456909,45.40938477)
\curveto(384.73456307,45.43938345)(384.83956297,45.46938342)(384.93956909,45.49938477)
\curveto(385.04956276,45.52938336)(385.15956265,45.56938332)(385.26956909,45.61938477)
\curveto(385.63956217,45.77938311)(385.95456185,45.99438289)(386.21456909,46.26438477)
\curveto(386.48456132,46.54438234)(386.7045611,46.87438201)(386.87456909,47.25438477)
\curveto(386.92456088,47.36438152)(386.96456084,47.47938141)(386.99456909,47.59938477)
\lineto(387.11456909,47.98938477)
\curveto(387.14456066,48.09938079)(387.16456064,48.21438067)(387.17456909,48.33438477)
\curveto(387.19456061,48.46438042)(387.21456059,48.5893803)(387.23456909,48.70938477)
\curveto(387.24456056,48.75938013)(387.24956056,48.79938009)(387.24956909,48.82938477)
\lineto(387.24956909,48.94938477)
}
}
{
\newrgbcolor{curcolor}{0 0 0}
\pscustom[linestyle=none,fillstyle=solid,fillcolor=curcolor]
{
\newpath
\moveto(397.30144409,48.33438477)
\curveto(397.32143603,48.27438061)(397.33143602,48.17938071)(397.33144409,48.04938477)
\curveto(397.33143602,47.92938096)(397.32643603,47.84438104)(397.31644409,47.79438477)
\lineto(397.31644409,47.64438477)
\curveto(397.30643605,47.56438132)(397.29643606,47.4893814)(397.28644409,47.41938477)
\curveto(397.28643607,47.35938153)(397.28143607,47.2893816)(397.27144409,47.20938477)
\curveto(397.2514361,47.14938174)(397.23643612,47.0893818)(397.22644409,47.02938477)
\curveto(397.22643613,46.96938192)(397.21643614,46.90938198)(397.19644409,46.84938477)
\curveto(397.1564362,46.71938217)(397.12143623,46.5893823)(397.09144409,46.45938477)
\curveto(397.06143629,46.32938256)(397.02143633,46.20938268)(396.97144409,46.09938477)
\curveto(396.76143659,45.61938327)(396.48143687,45.21438367)(396.13144409,44.88438477)
\curveto(395.78143757,44.56438432)(395.351438,44.31938457)(394.84144409,44.14938477)
\curveto(394.73143862,44.10938478)(394.61143874,44.07938481)(394.48144409,44.05938477)
\curveto(394.36143899,44.03938485)(394.23643912,44.01938487)(394.10644409,43.99938477)
\curveto(394.04643931,43.9893849)(393.98143937,43.9843849)(393.91144409,43.98438477)
\curveto(393.8514395,43.97438491)(393.79143956,43.96938492)(393.73144409,43.96938477)
\curveto(393.69143966,43.95938493)(393.63143972,43.95438493)(393.55144409,43.95438477)
\curveto(393.48143987,43.95438493)(393.43143992,43.95938493)(393.40144409,43.96938477)
\curveto(393.36143999,43.97938491)(393.32144003,43.9843849)(393.28144409,43.98438477)
\curveto(393.24144011,43.97438491)(393.20644015,43.97438491)(393.17644409,43.98438477)
\lineto(393.08644409,43.98438477)
\lineto(392.72644409,44.02938477)
\curveto(392.58644077,44.06938482)(392.4514409,44.10938478)(392.32144409,44.14938477)
\curveto(392.19144116,44.1893847)(392.06644129,44.23438465)(391.94644409,44.28438477)
\curveto(391.49644186,44.4843844)(391.12644223,44.74438414)(390.83644409,45.06438477)
\curveto(390.54644281,45.3843835)(390.30644305,45.77438311)(390.11644409,46.23438477)
\curveto(390.06644329,46.33438255)(390.02644333,46.43438245)(389.99644409,46.53438477)
\curveto(389.97644338,46.63438225)(389.9564434,46.73938215)(389.93644409,46.84938477)
\curveto(389.91644344,46.889382)(389.90644345,46.91938197)(389.90644409,46.93938477)
\curveto(389.91644344,46.96938192)(389.91644344,47.00438188)(389.90644409,47.04438477)
\curveto(389.88644347,47.12438176)(389.87144348,47.20438168)(389.86144409,47.28438477)
\curveto(389.86144349,47.37438151)(389.8514435,47.45938143)(389.83144409,47.53938477)
\lineto(389.83144409,47.65938477)
\curveto(389.83144352,47.69938119)(389.82644353,47.74438114)(389.81644409,47.79438477)
\curveto(389.80644355,47.84438104)(389.80144355,47.92938096)(389.80144409,48.04938477)
\curveto(389.80144355,48.17938071)(389.81144354,48.27438061)(389.83144409,48.33438477)
\curveto(389.8514435,48.40438048)(389.8564435,48.47438041)(389.84644409,48.54438477)
\curveto(389.83644352,48.61438027)(389.84144351,48.6843802)(389.86144409,48.75438477)
\curveto(389.87144348,48.80438008)(389.87644348,48.84438004)(389.87644409,48.87438477)
\curveto(389.88644347,48.91437997)(389.89644346,48.95937993)(389.90644409,49.00938477)
\curveto(389.93644342,49.12937976)(389.96144339,49.24937964)(389.98144409,49.36938477)
\curveto(390.01144334,49.4893794)(390.0514433,49.60437928)(390.10144409,49.71438477)
\curveto(390.2514431,50.0843788)(390.43144292,50.41437847)(390.64144409,50.70438477)
\curveto(390.86144249,51.00437788)(391.12644223,51.25437763)(391.43644409,51.45438477)
\curveto(391.5564418,51.53437735)(391.68144167,51.59937729)(391.81144409,51.64938477)
\curveto(391.94144141,51.70937718)(392.07644128,51.76937712)(392.21644409,51.82938477)
\curveto(392.33644102,51.87937701)(392.46644089,51.90937698)(392.60644409,51.91938477)
\curveto(392.74644061,51.93937695)(392.88644047,51.96937692)(393.02644409,52.00938477)
\lineto(393.22144409,52.00938477)
\curveto(393.29144006,52.01937687)(393.35644,52.02937686)(393.41644409,52.03938477)
\curveto(394.30643905,52.04937684)(395.04643831,51.86437702)(395.63644409,51.48438477)
\curveto(396.22643713,51.10437778)(396.6514367,50.60937828)(396.91144409,49.99938477)
\curveto(396.96143639,49.89937899)(397.00143635,49.79937909)(397.03144409,49.69938477)
\curveto(397.06143629,49.59937929)(397.09643626,49.49437939)(397.13644409,49.38438477)
\curveto(397.16643619,49.27437961)(397.19143616,49.15437973)(397.21144409,49.02438477)
\curveto(397.23143612,48.90437998)(397.2564361,48.77938011)(397.28644409,48.64938477)
\curveto(397.29643606,48.59938029)(397.29643606,48.54438034)(397.28644409,48.48438477)
\curveto(397.28643607,48.43438045)(397.29143606,48.3843805)(397.30144409,48.33438477)
\moveto(395.96644409,47.47938477)
\curveto(395.98643737,47.54938134)(395.99143736,47.62938126)(395.98144409,47.71938477)
\lineto(395.98144409,47.97438477)
\curveto(395.98143737,48.36438052)(395.94643741,48.69438019)(395.87644409,48.96438477)
\curveto(395.84643751,49.04437984)(395.82143753,49.12437976)(395.80144409,49.20438477)
\curveto(395.78143757,49.2843796)(395.7564376,49.35937953)(395.72644409,49.42938477)
\curveto(395.44643791,50.07937881)(395.00143835,50.52937836)(394.39144409,50.77938477)
\curveto(394.32143903,50.80937808)(394.24643911,50.82937806)(394.16644409,50.83938477)
\lineto(393.92644409,50.89938477)
\curveto(393.84643951,50.91937797)(393.76143959,50.92937796)(393.67144409,50.92938477)
\lineto(393.40144409,50.92938477)
\lineto(393.13144409,50.88438477)
\curveto(393.03144032,50.86437802)(392.93644042,50.83937805)(392.84644409,50.80938477)
\curveto(392.76644059,50.7893781)(392.68644067,50.75937813)(392.60644409,50.71938477)
\curveto(392.53644082,50.69937819)(392.47144088,50.66937822)(392.41144409,50.62938477)
\curveto(392.351441,50.5893783)(392.29644106,50.54937834)(392.24644409,50.50938477)
\curveto(392.00644135,50.33937855)(391.81144154,50.13437875)(391.66144409,49.89438477)
\curveto(391.51144184,49.65437923)(391.38144197,49.37437951)(391.27144409,49.05438477)
\curveto(391.24144211,48.95437993)(391.22144213,48.84938004)(391.21144409,48.73938477)
\curveto(391.20144215,48.63938025)(391.18644217,48.53438035)(391.16644409,48.42438477)
\curveto(391.1564422,48.3843805)(391.1514422,48.31938057)(391.15144409,48.22938477)
\curveto(391.14144221,48.19938069)(391.13644222,48.16438072)(391.13644409,48.12438477)
\curveto(391.14644221,48.0843808)(391.1514422,48.03938085)(391.15144409,47.98938477)
\lineto(391.15144409,47.68938477)
\curveto(391.1514422,47.5893813)(391.16144219,47.49938139)(391.18144409,47.41938477)
\lineto(391.21144409,47.23938477)
\curveto(391.23144212,47.13938175)(391.24644211,47.03938185)(391.25644409,46.93938477)
\curveto(391.27644208,46.84938204)(391.30644205,46.76438212)(391.34644409,46.68438477)
\curveto(391.44644191,46.44438244)(391.56144179,46.21938267)(391.69144409,46.00938477)
\curveto(391.83144152,45.79938309)(392.00144135,45.62438326)(392.20144409,45.48438477)
\curveto(392.2514411,45.45438343)(392.29644106,45.42938346)(392.33644409,45.40938477)
\curveto(392.37644098,45.3893835)(392.42144093,45.36438352)(392.47144409,45.33438477)
\curveto(392.5514408,45.2843836)(392.63644072,45.23938365)(392.72644409,45.19938477)
\curveto(392.82644053,45.16938372)(392.93144042,45.13938375)(393.04144409,45.10938477)
\curveto(393.09144026,45.0893838)(393.13644022,45.07938381)(393.17644409,45.07938477)
\curveto(393.22644013,45.0893838)(393.27644008,45.0893838)(393.32644409,45.07938477)
\curveto(393.35644,45.06938382)(393.41643994,45.05938383)(393.50644409,45.04938477)
\curveto(393.60643975,45.03938385)(393.68143967,45.04438384)(393.73144409,45.06438477)
\curveto(393.77143958,45.07438381)(393.81143954,45.07438381)(393.85144409,45.06438477)
\curveto(393.89143946,45.06438382)(393.93143942,45.07438381)(393.97144409,45.09438477)
\curveto(394.0514393,45.11438377)(394.13143922,45.12938376)(394.21144409,45.13938477)
\curveto(394.29143906,45.15938373)(394.36643899,45.1843837)(394.43644409,45.21438477)
\curveto(394.77643858,45.35438353)(395.0514383,45.54938334)(395.26144409,45.79938477)
\curveto(395.47143788,46.04938284)(395.64643771,46.34438254)(395.78644409,46.68438477)
\curveto(395.83643752,46.80438208)(395.86643749,46.92938196)(395.87644409,47.05938477)
\curveto(395.89643746,47.19938169)(395.92643743,47.33938155)(395.96644409,47.47938477)
}
}
{
\newrgbcolor{curcolor}{0 0 0}
\pscustom[linestyle=none,fillstyle=solid,fillcolor=curcolor]
{
\newpath
\moveto(401.92472534,52.03938477)
\curveto(402.66472055,52.04937684)(403.27971994,51.93937695)(403.76972534,51.70938477)
\curveto(404.26971895,51.4893774)(404.66471855,51.15437773)(404.95472534,50.70438477)
\curveto(405.08471813,50.50437838)(405.19471802,50.25937863)(405.28472534,49.96938477)
\curveto(405.30471791,49.91937897)(405.3197179,49.85437903)(405.32972534,49.77438477)
\curveto(405.33971788,49.69437919)(405.33471788,49.62437926)(405.31472534,49.56438477)
\curveto(405.28471793,49.51437937)(405.23471798,49.46937942)(405.16472534,49.42938477)
\curveto(405.13471808,49.40937948)(405.10471811,49.39937949)(405.07472534,49.39938477)
\curveto(405.04471817,49.40937948)(405.00971821,49.40937948)(404.96972534,49.39938477)
\curveto(404.92971829,49.3893795)(404.88971833,49.3843795)(404.84972534,49.38438477)
\curveto(404.80971841,49.39437949)(404.76971845,49.39937949)(404.72972534,49.39938477)
\lineto(404.41472534,49.39938477)
\curveto(404.3147189,49.40937948)(404.22971899,49.43937945)(404.15972534,49.48938477)
\curveto(404.07971914,49.54937934)(404.02471919,49.63437925)(403.99472534,49.74438477)
\curveto(403.96471925,49.85437903)(403.92471929,49.94937894)(403.87472534,50.02938477)
\curveto(403.72471949,50.2893786)(403.52971969,50.49437839)(403.28972534,50.64438477)
\curveto(403.20972001,50.69437819)(403.12472009,50.73437815)(403.03472534,50.76438477)
\curveto(402.94472027,50.80437808)(402.84972037,50.83937805)(402.74972534,50.86938477)
\curveto(402.60972061,50.90937798)(402.42472079,50.92937796)(402.19472534,50.92938477)
\curveto(401.96472125,50.93937795)(401.77472144,50.91937797)(401.62472534,50.86938477)
\curveto(401.55472166,50.84937804)(401.48972173,50.83437805)(401.42972534,50.82438477)
\curveto(401.36972185,50.81437807)(401.30472191,50.79937809)(401.23472534,50.77938477)
\curveto(400.97472224,50.66937822)(400.74472247,50.51937837)(400.54472534,50.32938477)
\curveto(400.34472287,50.13937875)(400.18972303,49.91437897)(400.07972534,49.65438477)
\curveto(400.03972318,49.56437932)(400.00472321,49.46937942)(399.97472534,49.36938477)
\curveto(399.94472327,49.27937961)(399.9147233,49.17937971)(399.88472534,49.06938477)
\lineto(399.79472534,48.66438477)
\curveto(399.78472343,48.61438027)(399.77972344,48.55938033)(399.77972534,48.49938477)
\curveto(399.78972343,48.43938045)(399.78472343,48.3843805)(399.76472534,48.33438477)
\lineto(399.76472534,48.21438477)
\curveto(399.75472346,48.17438071)(399.74472347,48.10938078)(399.73472534,48.01938477)
\curveto(399.73472348,47.92938096)(399.74472347,47.86438102)(399.76472534,47.82438477)
\curveto(399.77472344,47.77438111)(399.77472344,47.72438116)(399.76472534,47.67438477)
\curveto(399.75472346,47.62438126)(399.75472346,47.57438131)(399.76472534,47.52438477)
\curveto(399.77472344,47.4843814)(399.77972344,47.41438147)(399.77972534,47.31438477)
\curveto(399.79972342,47.23438165)(399.8147234,47.14938174)(399.82472534,47.05938477)
\curveto(399.84472337,46.96938192)(399.86472335,46.884382)(399.88472534,46.80438477)
\curveto(399.99472322,46.4843824)(400.1197231,46.20438268)(400.25972534,45.96438477)
\curveto(400.40972281,45.73438315)(400.6147226,45.53438335)(400.87472534,45.36438477)
\curveto(400.96472225,45.31438357)(401.05472216,45.26938362)(401.14472534,45.22938477)
\curveto(401.24472197,45.1893837)(401.34972187,45.14938374)(401.45972534,45.10938477)
\curveto(401.50972171,45.09938379)(401.54972167,45.09438379)(401.57972534,45.09438477)
\curveto(401.60972161,45.09438379)(401.64972157,45.0893838)(401.69972534,45.07938477)
\curveto(401.72972149,45.06938382)(401.77972144,45.06438382)(401.84972534,45.06438477)
\lineto(402.01472534,45.06438477)
\curveto(402.0147212,45.05438383)(402.03472118,45.04938384)(402.07472534,45.04938477)
\curveto(402.09472112,45.05938383)(402.1197211,45.05938383)(402.14972534,45.04938477)
\curveto(402.17972104,45.04938384)(402.20972101,45.05438383)(402.23972534,45.06438477)
\curveto(402.30972091,45.0843838)(402.37472084,45.0893838)(402.43472534,45.07938477)
\curveto(402.50472071,45.07938381)(402.57472064,45.0893838)(402.64472534,45.10938477)
\curveto(402.90472031,45.1893837)(403.12972009,45.2893836)(403.31972534,45.40938477)
\curveto(403.50971971,45.53938335)(403.66971955,45.70438318)(403.79972534,45.90438477)
\curveto(403.84971937,45.9843829)(403.89471932,46.06938282)(403.93472534,46.15938477)
\lineto(404.05472534,46.42938477)
\curveto(404.07471914,46.50938238)(404.09471912,46.5843823)(404.11472534,46.65438477)
\curveto(404.14471907,46.73438215)(404.19471902,46.79938209)(404.26472534,46.84938477)
\curveto(404.29471892,46.87938201)(404.35471886,46.89938199)(404.44472534,46.90938477)
\curveto(404.53471868,46.92938196)(404.62971859,46.93938195)(404.72972534,46.93938477)
\curveto(404.83971838,46.94938194)(404.93971828,46.94938194)(405.02972534,46.93938477)
\curveto(405.12971809,46.92938196)(405.19971802,46.90938198)(405.23972534,46.87938477)
\curveto(405.29971792,46.83938205)(405.33471788,46.77938211)(405.34472534,46.69938477)
\curveto(405.36471785,46.61938227)(405.36471785,46.53438235)(405.34472534,46.44438477)
\curveto(405.29471792,46.29438259)(405.24471797,46.14938274)(405.19472534,46.00938477)
\curveto(405.15471806,45.87938301)(405.09971812,45.74938314)(405.02972534,45.61938477)
\curveto(404.87971834,45.31938357)(404.68971853,45.05438383)(404.45972534,44.82438477)
\curveto(404.23971898,44.59438429)(403.96971925,44.40938448)(403.64972534,44.26938477)
\curveto(403.56971965,44.22938466)(403.48471973,44.19438469)(403.39472534,44.16438477)
\curveto(403.30471991,44.14438474)(403.20972001,44.11938477)(403.10972534,44.08938477)
\curveto(402.99972022,44.04938484)(402.88972033,44.02938486)(402.77972534,44.02938477)
\curveto(402.66972055,44.01938487)(402.55972066,44.00438488)(402.44972534,43.98438477)
\curveto(402.40972081,43.96438492)(402.36972085,43.95938493)(402.32972534,43.96938477)
\curveto(402.28972093,43.97938491)(402.24972097,43.97938491)(402.20972534,43.96938477)
\lineto(402.07472534,43.96938477)
\lineto(401.83472534,43.96938477)
\curveto(401.76472145,43.95938493)(401.69972152,43.96438492)(401.63972534,43.98438477)
\lineto(401.56472534,43.98438477)
\lineto(401.20472534,44.02938477)
\curveto(401.07472214,44.06938482)(400.94972227,44.10438478)(400.82972534,44.13438477)
\curveto(400.70972251,44.16438472)(400.59472262,44.20438468)(400.48472534,44.25438477)
\curveto(400.12472309,44.41438447)(399.82472339,44.60438428)(399.58472534,44.82438477)
\curveto(399.35472386,45.04438384)(399.13972408,45.31438357)(398.93972534,45.63438477)
\curveto(398.88972433,45.71438317)(398.84472437,45.80438308)(398.80472534,45.90438477)
\lineto(398.68472534,46.20438477)
\curveto(398.63472458,46.31438257)(398.59972462,46.42938246)(398.57972534,46.54938477)
\curveto(398.55972466,46.66938222)(398.53472468,46.7893821)(398.50472534,46.90938477)
\curveto(398.49472472,46.94938194)(398.48972473,46.9893819)(398.48972534,47.02938477)
\curveto(398.48972473,47.06938182)(398.48472473,47.10938178)(398.47472534,47.14938477)
\curveto(398.45472476,47.20938168)(398.44472477,47.27438161)(398.44472534,47.34438477)
\curveto(398.45472476,47.41438147)(398.44972477,47.47938141)(398.42972534,47.53938477)
\lineto(398.42972534,47.68938477)
\curveto(398.4197248,47.73938115)(398.4147248,47.80938108)(398.41472534,47.89938477)
\curveto(398.4147248,47.9893809)(398.4197248,48.05938083)(398.42972534,48.10938477)
\curveto(398.43972478,48.15938073)(398.43972478,48.20438068)(398.42972534,48.24438477)
\curveto(398.42972479,48.2843806)(398.43472478,48.32438056)(398.44472534,48.36438477)
\curveto(398.46472475,48.43438045)(398.46972475,48.50438038)(398.45972534,48.57438477)
\curveto(398.45972476,48.64438024)(398.46972475,48.70938018)(398.48972534,48.76938477)
\curveto(398.52972469,48.93937995)(398.56472465,49.10937978)(398.59472534,49.27938477)
\curveto(398.62472459,49.44937944)(398.66972455,49.60937928)(398.72972534,49.75938477)
\curveto(398.93972428,50.27937861)(399.19472402,50.69937819)(399.49472534,51.01938477)
\curveto(399.79472342,51.33937755)(400.20472301,51.60437728)(400.72472534,51.81438477)
\curveto(400.83472238,51.86437702)(400.95472226,51.89937699)(401.08472534,51.91938477)
\curveto(401.214722,51.93937695)(401.34972187,51.96437692)(401.48972534,51.99438477)
\curveto(401.55972166,52.00437688)(401.62972159,52.00937688)(401.69972534,52.00938477)
\curveto(401.76972145,52.01937687)(401.84472137,52.02937686)(401.92472534,52.03938477)
}
}
{
\newrgbcolor{curcolor}{0 0 0}
\pscustom[linestyle=none,fillstyle=solid,fillcolor=curcolor]
{
\newpath
\moveto(413.59636597,48.30438477)
\curveto(413.61635828,48.20438068)(413.61635828,48.0893808)(413.59636597,47.95938477)
\curveto(413.58635831,47.83938105)(413.55635834,47.75438113)(413.50636597,47.70438477)
\curveto(413.45635844,47.66438122)(413.38135852,47.63438125)(413.28136597,47.61438477)
\curveto(413.19135871,47.60438128)(413.08635881,47.59938129)(412.96636597,47.59938477)
\lineto(412.60636597,47.59938477)
\curveto(412.48635941,47.60938128)(412.38135952,47.61438127)(412.29136597,47.61438477)
\lineto(408.45136597,47.61438477)
\curveto(408.37136353,47.61438127)(408.29136361,47.60938128)(408.21136597,47.59938477)
\curveto(408.13136377,47.59938129)(408.06636383,47.5843813)(408.01636597,47.55438477)
\curveto(407.97636392,47.53438135)(407.93636396,47.49438139)(407.89636597,47.43438477)
\curveto(407.87636402,47.40438148)(407.85636404,47.35938153)(407.83636597,47.29938477)
\curveto(407.81636408,47.24938164)(407.81636408,47.19938169)(407.83636597,47.14938477)
\curveto(407.84636405,47.09938179)(407.85136405,47.05438183)(407.85136597,47.01438477)
\curveto(407.85136405,46.97438191)(407.85636404,46.93438195)(407.86636597,46.89438477)
\curveto(407.88636401,46.81438207)(407.90636399,46.72938216)(407.92636597,46.63938477)
\curveto(407.94636395,46.55938233)(407.97636392,46.47938241)(408.01636597,46.39938477)
\curveto(408.24636365,45.85938303)(408.62636327,45.47438341)(409.15636597,45.24438477)
\curveto(409.21636268,45.21438367)(409.28136262,45.1893837)(409.35136597,45.16938477)
\lineto(409.56136597,45.10938477)
\curveto(409.59136231,45.09938379)(409.64136226,45.09438379)(409.71136597,45.09438477)
\curveto(409.85136205,45.05438383)(410.03636186,45.03438385)(410.26636597,45.03438477)
\curveto(410.4963614,45.03438385)(410.68136122,45.05438383)(410.82136597,45.09438477)
\curveto(410.96136094,45.13438375)(411.08636081,45.17438371)(411.19636597,45.21438477)
\curveto(411.31636058,45.26438362)(411.42636047,45.32438356)(411.52636597,45.39438477)
\curveto(411.63636026,45.46438342)(411.73136017,45.54438334)(411.81136597,45.63438477)
\curveto(411.89136001,45.73438315)(411.96135994,45.83938305)(412.02136597,45.94938477)
\curveto(412.08135982,46.04938284)(412.13135977,46.15438273)(412.17136597,46.26438477)
\curveto(412.22135968,46.37438251)(412.3013596,46.45438243)(412.41136597,46.50438477)
\curveto(412.45135945,46.52438236)(412.51635938,46.53938235)(412.60636597,46.54938477)
\curveto(412.6963592,46.55938233)(412.78635911,46.55938233)(412.87636597,46.54938477)
\curveto(412.96635893,46.54938234)(413.05135885,46.54438234)(413.13136597,46.53438477)
\curveto(413.21135869,46.52438236)(413.26635863,46.50438238)(413.29636597,46.47438477)
\curveto(413.3963585,46.40438248)(413.42135848,46.2893826)(413.37136597,46.12938477)
\curveto(413.29135861,45.85938303)(413.18635871,45.61938327)(413.05636597,45.40938477)
\curveto(412.85635904,45.0893838)(412.62635927,44.82438406)(412.36636597,44.61438477)
\curveto(412.11635978,44.41438447)(411.7963601,44.24938464)(411.40636597,44.11938477)
\curveto(411.30636059,44.07938481)(411.20636069,44.05438483)(411.10636597,44.04438477)
\curveto(411.00636089,44.02438486)(410.901361,44.00438488)(410.79136597,43.98438477)
\curveto(410.74136116,43.97438491)(410.69136121,43.96938492)(410.64136597,43.96938477)
\curveto(410.6013613,43.96938492)(410.55636134,43.96438492)(410.50636597,43.95438477)
\lineto(410.35636597,43.95438477)
\curveto(410.30636159,43.94438494)(410.24636165,43.93938495)(410.17636597,43.93938477)
\curveto(410.11636178,43.93938495)(410.06636183,43.94438494)(410.02636597,43.95438477)
\lineto(409.89136597,43.95438477)
\curveto(409.84136206,43.96438492)(409.7963621,43.96938492)(409.75636597,43.96938477)
\curveto(409.71636218,43.96938492)(409.67636222,43.97438491)(409.63636597,43.98438477)
\curveto(409.58636231,43.99438489)(409.53136237,44.00438488)(409.47136597,44.01438477)
\curveto(409.41136249,44.01438487)(409.35636254,44.01938487)(409.30636597,44.02938477)
\curveto(409.21636268,44.04938484)(409.12636277,44.07438481)(409.03636597,44.10438477)
\curveto(408.94636295,44.12438476)(408.86136304,44.14938474)(408.78136597,44.17938477)
\curveto(408.74136316,44.19938469)(408.70636319,44.20938468)(408.67636597,44.20938477)
\curveto(408.64636325,44.21938467)(408.61136329,44.23438465)(408.57136597,44.25438477)
\curveto(408.42136348,44.32438456)(408.26136364,44.40938448)(408.09136597,44.50938477)
\curveto(407.8013641,44.69938419)(407.55136435,44.92938396)(407.34136597,45.19938477)
\curveto(407.14136476,45.47938341)(406.97136493,45.7893831)(406.83136597,46.12938477)
\curveto(406.78136512,46.23938265)(406.74136516,46.35438253)(406.71136597,46.47438477)
\curveto(406.69136521,46.59438229)(406.66136524,46.71438217)(406.62136597,46.83438477)
\curveto(406.61136529,46.87438201)(406.60636529,46.90938198)(406.60636597,46.93938477)
\curveto(406.60636529,46.96938192)(406.6013653,47.00938188)(406.59136597,47.05938477)
\curveto(406.57136533,47.13938175)(406.55636534,47.22438166)(406.54636597,47.31438477)
\curveto(406.53636536,47.40438148)(406.52136538,47.49438139)(406.50136597,47.58438477)
\lineto(406.50136597,47.79438477)
\curveto(406.49136541,47.83438105)(406.48136542,47.889381)(406.47136597,47.95938477)
\curveto(406.47136543,48.03938085)(406.47636542,48.10438078)(406.48636597,48.15438477)
\lineto(406.48636597,48.31938477)
\curveto(406.50636539,48.36938052)(406.51136539,48.41938047)(406.50136597,48.46938477)
\curveto(406.5013654,48.52938036)(406.50636539,48.5843803)(406.51636597,48.63438477)
\curveto(406.55636534,48.79438009)(406.58636531,48.95437993)(406.60636597,49.11438477)
\curveto(406.63636526,49.27437961)(406.68136522,49.42437946)(406.74136597,49.56438477)
\curveto(406.79136511,49.67437921)(406.83636506,49.7843791)(406.87636597,49.89438477)
\curveto(406.92636497,50.01437887)(406.98136492,50.12937876)(407.04136597,50.23938477)
\curveto(407.26136464,50.5893783)(407.51136439,50.889378)(407.79136597,51.13938477)
\curveto(408.07136383,51.39937749)(408.41636348,51.61437727)(408.82636597,51.78438477)
\curveto(408.94636295,51.83437705)(409.06636283,51.86937702)(409.18636597,51.88938477)
\curveto(409.31636258,51.91937697)(409.45136245,51.94937694)(409.59136597,51.97938477)
\curveto(409.64136226,51.9893769)(409.68636221,51.99437689)(409.72636597,51.99438477)
\curveto(409.76636213,52.00437688)(409.81136209,52.00937688)(409.86136597,52.00938477)
\curveto(409.88136202,52.01937687)(409.90636199,52.01937687)(409.93636597,52.00938477)
\curveto(409.96636193,51.99937689)(409.99136191,52.00437688)(410.01136597,52.02438477)
\curveto(410.43136147,52.03437685)(410.7963611,51.9893769)(411.10636597,51.88938477)
\curveto(411.41636048,51.79937709)(411.6963602,51.67437721)(411.94636597,51.51438477)
\curveto(411.9963599,51.49437739)(412.03635986,51.46437742)(412.06636597,51.42438477)
\curveto(412.0963598,51.39437749)(412.13135977,51.36937752)(412.17136597,51.34938477)
\curveto(412.25135965,51.2893776)(412.33135957,51.21937767)(412.41136597,51.13938477)
\curveto(412.5013594,51.05937783)(412.57635932,50.97937791)(412.63636597,50.89938477)
\curveto(412.7963591,50.6893782)(412.93135897,50.4893784)(413.04136597,50.29938477)
\curveto(413.11135879,50.1893787)(413.16635873,50.06937882)(413.20636597,49.93938477)
\curveto(413.24635865,49.80937908)(413.29135861,49.67937921)(413.34136597,49.54938477)
\curveto(413.39135851,49.41937947)(413.42635847,49.2843796)(413.44636597,49.14438477)
\curveto(413.47635842,49.00437988)(413.51135839,48.86438002)(413.55136597,48.72438477)
\curveto(413.56135834,48.65438023)(413.56635833,48.5843803)(413.56636597,48.51438477)
\lineto(413.59636597,48.30438477)
\moveto(412.14136597,48.81438477)
\curveto(412.17135973,48.85438003)(412.1963597,48.90437998)(412.21636597,48.96438477)
\curveto(412.23635966,49.03437985)(412.23635966,49.10437978)(412.21636597,49.17438477)
\curveto(412.15635974,49.39437949)(412.07135983,49.59937929)(411.96136597,49.78938477)
\curveto(411.82136008,50.01937887)(411.66636023,50.21437867)(411.49636597,50.37438477)
\curveto(411.32636057,50.53437835)(411.10636079,50.66937822)(410.83636597,50.77938477)
\curveto(410.76636113,50.79937809)(410.6963612,50.81437807)(410.62636597,50.82438477)
\curveto(410.55636134,50.84437804)(410.48136142,50.86437802)(410.40136597,50.88438477)
\curveto(410.32136158,50.90437798)(410.23636166,50.91437797)(410.14636597,50.91438477)
\lineto(409.89136597,50.91438477)
\curveto(409.86136204,50.89437799)(409.82636207,50.884378)(409.78636597,50.88438477)
\curveto(409.74636215,50.89437799)(409.71136219,50.89437799)(409.68136597,50.88438477)
\lineto(409.44136597,50.82438477)
\curveto(409.37136253,50.81437807)(409.3013626,50.79937809)(409.23136597,50.77938477)
\curveto(408.94136296,50.65937823)(408.70636319,50.50937838)(408.52636597,50.32938477)
\curveto(408.35636354,50.14937874)(408.2013637,49.92437896)(408.06136597,49.65438477)
\curveto(408.03136387,49.60437928)(408.0013639,49.53937935)(407.97136597,49.45938477)
\curveto(407.94136396,49.3893795)(407.91636398,49.30937958)(407.89636597,49.21938477)
\curveto(407.87636402,49.12937976)(407.87136403,49.04437984)(407.88136597,48.96438477)
\curveto(407.89136401,48.88438)(407.92636397,48.82438006)(407.98636597,48.78438477)
\curveto(408.06636383,48.72438016)(408.2013637,48.69438019)(408.39136597,48.69438477)
\curveto(408.59136331,48.70438018)(408.76136314,48.70938018)(408.90136597,48.70938477)
\lineto(411.18136597,48.70938477)
\curveto(411.33136057,48.70938018)(411.51136039,48.70438018)(411.72136597,48.69438477)
\curveto(411.93135997,48.69438019)(412.07135983,48.73438015)(412.14136597,48.81438477)
}
}
{
\newrgbcolor{curcolor}{0 0 0}
\pscustom[linestyle=none,fillstyle=solid,fillcolor=curcolor]
{
\newpath
\moveto(418.59300659,52.00938477)
\curveto(419.22300136,52.02937686)(419.72800085,51.94437694)(420.10800659,51.75438477)
\curveto(420.48800009,51.56437732)(420.79299979,51.27937761)(421.02300659,50.89938477)
\curveto(421.0829995,50.79937809)(421.12799945,50.6893782)(421.15800659,50.56938477)
\curveto(421.19799938,50.45937843)(421.23299935,50.34437854)(421.26300659,50.22438477)
\curveto(421.31299927,50.03437885)(421.34299924,49.82937906)(421.35300659,49.60938477)
\curveto(421.36299922,49.3893795)(421.36799921,49.16437972)(421.36800659,48.93438477)
\lineto(421.36800659,47.32938477)
\lineto(421.36800659,44.98938477)
\curveto(421.36799921,44.81938407)(421.36299922,44.64938424)(421.35300659,44.47938477)
\curveto(421.35299923,44.30938458)(421.28799929,44.19938469)(421.15800659,44.14938477)
\curveto(421.10799947,44.12938476)(421.05299953,44.11938477)(420.99300659,44.11938477)
\curveto(420.94299964,44.10938478)(420.88799969,44.10438478)(420.82800659,44.10438477)
\curveto(420.69799988,44.10438478)(420.57300001,44.10938478)(420.45300659,44.11938477)
\curveto(420.33300025,44.11938477)(420.24800033,44.15938473)(420.19800659,44.23938477)
\curveto(420.14800043,44.30938458)(420.12300046,44.39938449)(420.12300659,44.50938477)
\lineto(420.12300659,44.83938477)
\lineto(420.12300659,46.12938477)
\lineto(420.12300659,48.57438477)
\curveto(420.12300046,48.84438004)(420.11800046,49.10937978)(420.10800659,49.36938477)
\curveto(420.09800048,49.63937925)(420.05300053,49.86937902)(419.97300659,50.05938477)
\curveto(419.89300069,50.25937863)(419.77300081,50.41937847)(419.61300659,50.53938477)
\curveto(419.45300113,50.66937822)(419.26800131,50.76937812)(419.05800659,50.83938477)
\curveto(418.99800158,50.85937803)(418.93300165,50.86937802)(418.86300659,50.86938477)
\curveto(418.80300178,50.87937801)(418.74300184,50.89437799)(418.68300659,50.91438477)
\curveto(418.63300195,50.92437796)(418.55300203,50.92437796)(418.44300659,50.91438477)
\curveto(418.34300224,50.91437797)(418.27300231,50.90937798)(418.23300659,50.89938477)
\curveto(418.19300239,50.87937801)(418.15800242,50.86937802)(418.12800659,50.86938477)
\curveto(418.09800248,50.87937801)(418.06300252,50.87937801)(418.02300659,50.86938477)
\curveto(417.89300269,50.83937805)(417.76800281,50.80437808)(417.64800659,50.76438477)
\curveto(417.53800304,50.73437815)(417.43300315,50.6893782)(417.33300659,50.62938477)
\curveto(417.29300329,50.60937828)(417.25800332,50.5893783)(417.22800659,50.56938477)
\curveto(417.19800338,50.54937834)(417.16300342,50.52937836)(417.12300659,50.50938477)
\curveto(416.77300381,50.25937863)(416.51800406,49.884379)(416.35800659,49.38438477)
\curveto(416.32800425,49.30437958)(416.30800427,49.21937967)(416.29800659,49.12938477)
\curveto(416.28800429,49.04937984)(416.27300431,48.96937992)(416.25300659,48.88938477)
\curveto(416.23300435,48.83938005)(416.22800435,48.7893801)(416.23800659,48.73938477)
\curveto(416.24800433,48.69938019)(416.24300434,48.65938023)(416.22300659,48.61938477)
\lineto(416.22300659,48.30438477)
\curveto(416.21300437,48.27438061)(416.20800437,48.23938065)(416.20800659,48.19938477)
\curveto(416.21800436,48.15938073)(416.22300436,48.11438077)(416.22300659,48.06438477)
\lineto(416.22300659,47.61438477)
\lineto(416.22300659,46.17438477)
\lineto(416.22300659,44.85438477)
\lineto(416.22300659,44.50938477)
\curveto(416.22300436,44.39938449)(416.19800438,44.30938458)(416.14800659,44.23938477)
\curveto(416.09800448,44.15938473)(416.00800457,44.11938477)(415.87800659,44.11938477)
\curveto(415.75800482,44.10938478)(415.63300495,44.10438478)(415.50300659,44.10438477)
\curveto(415.42300516,44.10438478)(415.34800523,44.10938478)(415.27800659,44.11938477)
\curveto(415.20800537,44.12938476)(415.14800543,44.15438473)(415.09800659,44.19438477)
\curveto(415.01800556,44.24438464)(414.9780056,44.33938455)(414.97800659,44.47938477)
\lineto(414.97800659,44.88438477)
\lineto(414.97800659,46.65438477)
\lineto(414.97800659,50.28438477)
\lineto(414.97800659,51.19938477)
\lineto(414.97800659,51.46938477)
\curveto(414.9780056,51.55937733)(414.99800558,51.62937726)(415.03800659,51.67938477)
\curveto(415.06800551,51.73937715)(415.11800546,51.77937711)(415.18800659,51.79938477)
\curveto(415.22800535,51.80937708)(415.2830053,51.81937707)(415.35300659,51.82938477)
\curveto(415.43300515,51.83937705)(415.51300507,51.84437704)(415.59300659,51.84438477)
\curveto(415.67300491,51.84437704)(415.74800483,51.83937705)(415.81800659,51.82938477)
\curveto(415.89800468,51.81937707)(415.95300463,51.80437708)(415.98300659,51.78438477)
\curveto(416.09300449,51.71437717)(416.14300444,51.62437726)(416.13300659,51.51438477)
\curveto(416.12300446,51.41437747)(416.13800444,51.29937759)(416.17800659,51.16938477)
\curveto(416.19800438,51.10937778)(416.23800434,51.05937783)(416.29800659,51.01938477)
\curveto(416.41800416,51.00937788)(416.51300407,51.05437783)(416.58300659,51.15438477)
\curveto(416.66300392,51.25437763)(416.74300384,51.33437755)(416.82300659,51.39438477)
\curveto(416.96300362,51.49437739)(417.10300348,51.5843773)(417.24300659,51.66438477)
\curveto(417.39300319,51.75437713)(417.56300302,51.82937706)(417.75300659,51.88938477)
\curveto(417.83300275,51.91937697)(417.91800266,51.93937695)(418.00800659,51.94938477)
\curveto(418.10800247,51.95937693)(418.20300238,51.97437691)(418.29300659,51.99438477)
\curveto(418.34300224,52.00437688)(418.39300219,52.00937688)(418.44300659,52.00938477)
\lineto(418.59300659,52.00938477)
}
}
{
\newrgbcolor{curcolor}{0 0 0}
\pscustom[linestyle=none,fillstyle=solid,fillcolor=curcolor]
{
\newpath
\moveto(424.19761597,54.19938477)
\curveto(424.34761396,54.19937469)(424.49761381,54.19437469)(424.64761597,54.18438477)
\curveto(424.79761351,54.1843747)(424.9026134,54.14437474)(424.96261597,54.06438477)
\curveto(425.01261329,54.00437488)(425.03761327,53.91937497)(425.03761597,53.80938477)
\curveto(425.04761326,53.70937518)(425.05261325,53.60437528)(425.05261597,53.49438477)
\lineto(425.05261597,52.62438477)
\curveto(425.05261325,52.54437634)(425.04761326,52.45937643)(425.03761597,52.36938477)
\curveto(425.03761327,52.2893766)(425.04761326,52.21937667)(425.06761597,52.15938477)
\curveto(425.1076132,52.01937687)(425.19761311,51.92937696)(425.33761597,51.88938477)
\curveto(425.38761292,51.87937701)(425.43261287,51.87437701)(425.47261597,51.87438477)
\lineto(425.62261597,51.87438477)
\lineto(426.02761597,51.87438477)
\curveto(426.18761212,51.884377)(426.302612,51.87437701)(426.37261597,51.84438477)
\curveto(426.46261184,51.7843771)(426.52261178,51.72437716)(426.55261597,51.66438477)
\curveto(426.57261173,51.62437726)(426.58261172,51.57937731)(426.58261597,51.52938477)
\lineto(426.58261597,51.37938477)
\curveto(426.58261172,51.26937762)(426.57761173,51.16437772)(426.56761597,51.06438477)
\curveto(426.55761175,50.97437791)(426.52261178,50.90437798)(426.46261597,50.85438477)
\curveto(426.4026119,50.80437808)(426.31761199,50.77437811)(426.20761597,50.76438477)
\lineto(425.87761597,50.76438477)
\curveto(425.76761254,50.77437811)(425.65761265,50.77937811)(425.54761597,50.77938477)
\curveto(425.43761287,50.77937811)(425.34261296,50.76437812)(425.26261597,50.73438477)
\curveto(425.19261311,50.70437818)(425.14261316,50.65437823)(425.11261597,50.58438477)
\curveto(425.08261322,50.51437837)(425.06261324,50.42937846)(425.05261597,50.32938477)
\curveto(425.04261326,50.23937865)(425.03761327,50.13937875)(425.03761597,50.02938477)
\curveto(425.04761326,49.92937896)(425.05261325,49.82937906)(425.05261597,49.72938477)
\lineto(425.05261597,46.75938477)
\curveto(425.05261325,46.53938235)(425.04761326,46.30438258)(425.03761597,46.05438477)
\curveto(425.03761327,45.81438307)(425.08261322,45.62938326)(425.17261597,45.49938477)
\curveto(425.22261308,45.41938347)(425.28761302,45.36438352)(425.36761597,45.33438477)
\curveto(425.44761286,45.30438358)(425.54261276,45.27938361)(425.65261597,45.25938477)
\curveto(425.68261262,45.24938364)(425.71261259,45.24438364)(425.74261597,45.24438477)
\curveto(425.78261252,45.25438363)(425.81761249,45.25438363)(425.84761597,45.24438477)
\lineto(426.04261597,45.24438477)
\curveto(426.14261216,45.24438364)(426.23261207,45.23438365)(426.31261597,45.21438477)
\curveto(426.4026119,45.20438368)(426.46761184,45.16938372)(426.50761597,45.10938477)
\curveto(426.52761178,45.07938381)(426.54261176,45.02438386)(426.55261597,44.94438477)
\curveto(426.57261173,44.87438401)(426.58261172,44.79938409)(426.58261597,44.71938477)
\curveto(426.59261171,44.63938425)(426.59261171,44.55938433)(426.58261597,44.47938477)
\curveto(426.57261173,44.40938448)(426.55261175,44.35438453)(426.52261597,44.31438477)
\curveto(426.48261182,44.24438464)(426.4076119,44.19438469)(426.29761597,44.16438477)
\curveto(426.21761209,44.14438474)(426.12761218,44.13438475)(426.02761597,44.13438477)
\curveto(425.92761238,44.14438474)(425.83761247,44.14938474)(425.75761597,44.14938477)
\curveto(425.69761261,44.14938474)(425.63761267,44.14438474)(425.57761597,44.13438477)
\curveto(425.51761279,44.13438475)(425.46261284,44.13938475)(425.41261597,44.14938477)
\lineto(425.23261597,44.14938477)
\curveto(425.18261312,44.15938473)(425.13261317,44.16438472)(425.08261597,44.16438477)
\curveto(425.04261326,44.17438471)(424.99761331,44.17938471)(424.94761597,44.17938477)
\curveto(424.74761356,44.22938466)(424.57261373,44.2843846)(424.42261597,44.34438477)
\curveto(424.28261402,44.40438448)(424.16261414,44.50938438)(424.06261597,44.65938477)
\curveto(423.92261438,44.85938403)(423.84261446,45.10938378)(423.82261597,45.40938477)
\curveto(423.8026145,45.71938317)(423.79261451,46.04938284)(423.79261597,46.39938477)
\lineto(423.79261597,50.32938477)
\curveto(423.76261454,50.45937843)(423.73261457,50.55437833)(423.70261597,50.61438477)
\curveto(423.68261462,50.67437821)(423.61261469,50.72437816)(423.49261597,50.76438477)
\curveto(423.45261485,50.77437811)(423.41261489,50.77437811)(423.37261597,50.76438477)
\curveto(423.33261497,50.75437813)(423.29261501,50.75937813)(423.25261597,50.77938477)
\lineto(423.01261597,50.77938477)
\curveto(422.88261542,50.77937811)(422.77261553,50.7893781)(422.68261597,50.80938477)
\curveto(422.6026157,50.83937805)(422.54761576,50.89937799)(422.51761597,50.98938477)
\curveto(422.49761581,51.02937786)(422.48261582,51.07437781)(422.47261597,51.12438477)
\lineto(422.47261597,51.27438477)
\curveto(422.47261583,51.41437747)(422.48261582,51.52937736)(422.50261597,51.61938477)
\curveto(422.52261578,51.71937717)(422.58261572,51.79437709)(422.68261597,51.84438477)
\curveto(422.79261551,51.884377)(422.93261537,51.89437699)(423.10261597,51.87438477)
\curveto(423.28261502,51.85437703)(423.43261487,51.86437702)(423.55261597,51.90438477)
\curveto(423.64261466,51.95437693)(423.71261459,52.02437686)(423.76261597,52.11438477)
\curveto(423.78261452,52.17437671)(423.79261451,52.24937664)(423.79261597,52.33938477)
\lineto(423.79261597,52.59438477)
\lineto(423.79261597,53.52438477)
\lineto(423.79261597,53.76438477)
\curveto(423.79261451,53.85437503)(423.8026145,53.92937496)(423.82261597,53.98938477)
\curveto(423.86261444,54.06937482)(423.93761437,54.13437475)(424.04761597,54.18438477)
\curveto(424.07761423,54.1843747)(424.1026142,54.1843747)(424.12261597,54.18438477)
\curveto(424.15261415,54.19437469)(424.17761413,54.19937469)(424.19761597,54.19938477)
}
}
{
\newrgbcolor{curcolor}{0 0 0}
\pscustom[linestyle=none,fillstyle=solid,fillcolor=curcolor]
{
\newpath
\moveto(434.71941284,48.30438477)
\curveto(434.73940516,48.20438068)(434.73940516,48.0893808)(434.71941284,47.95938477)
\curveto(434.70940519,47.83938105)(434.67940522,47.75438113)(434.62941284,47.70438477)
\curveto(434.57940532,47.66438122)(434.50440539,47.63438125)(434.40441284,47.61438477)
\curveto(434.31440558,47.60438128)(434.20940569,47.59938129)(434.08941284,47.59938477)
\lineto(433.72941284,47.59938477)
\curveto(433.60940629,47.60938128)(433.50440639,47.61438127)(433.41441284,47.61438477)
\lineto(429.57441284,47.61438477)
\curveto(429.4944104,47.61438127)(429.41441048,47.60938128)(429.33441284,47.59938477)
\curveto(429.25441064,47.59938129)(429.18941071,47.5843813)(429.13941284,47.55438477)
\curveto(429.0994108,47.53438135)(429.05941084,47.49438139)(429.01941284,47.43438477)
\curveto(428.9994109,47.40438148)(428.97941092,47.35938153)(428.95941284,47.29938477)
\curveto(428.93941096,47.24938164)(428.93941096,47.19938169)(428.95941284,47.14938477)
\curveto(428.96941093,47.09938179)(428.97441092,47.05438183)(428.97441284,47.01438477)
\curveto(428.97441092,46.97438191)(428.97941092,46.93438195)(428.98941284,46.89438477)
\curveto(429.00941089,46.81438207)(429.02941087,46.72938216)(429.04941284,46.63938477)
\curveto(429.06941083,46.55938233)(429.0994108,46.47938241)(429.13941284,46.39938477)
\curveto(429.36941053,45.85938303)(429.74941015,45.47438341)(430.27941284,45.24438477)
\curveto(430.33940956,45.21438367)(430.40440949,45.1893837)(430.47441284,45.16938477)
\lineto(430.68441284,45.10938477)
\curveto(430.71440918,45.09938379)(430.76440913,45.09438379)(430.83441284,45.09438477)
\curveto(430.97440892,45.05438383)(431.15940874,45.03438385)(431.38941284,45.03438477)
\curveto(431.61940828,45.03438385)(431.80440809,45.05438383)(431.94441284,45.09438477)
\curveto(432.08440781,45.13438375)(432.20940769,45.17438371)(432.31941284,45.21438477)
\curveto(432.43940746,45.26438362)(432.54940735,45.32438356)(432.64941284,45.39438477)
\curveto(432.75940714,45.46438342)(432.85440704,45.54438334)(432.93441284,45.63438477)
\curveto(433.01440688,45.73438315)(433.08440681,45.83938305)(433.14441284,45.94938477)
\curveto(433.20440669,46.04938284)(433.25440664,46.15438273)(433.29441284,46.26438477)
\curveto(433.34440655,46.37438251)(433.42440647,46.45438243)(433.53441284,46.50438477)
\curveto(433.57440632,46.52438236)(433.63940626,46.53938235)(433.72941284,46.54938477)
\curveto(433.81940608,46.55938233)(433.90940599,46.55938233)(433.99941284,46.54938477)
\curveto(434.08940581,46.54938234)(434.17440572,46.54438234)(434.25441284,46.53438477)
\curveto(434.33440556,46.52438236)(434.38940551,46.50438238)(434.41941284,46.47438477)
\curveto(434.51940538,46.40438248)(434.54440535,46.2893826)(434.49441284,46.12938477)
\curveto(434.41440548,45.85938303)(434.30940559,45.61938327)(434.17941284,45.40938477)
\curveto(433.97940592,45.0893838)(433.74940615,44.82438406)(433.48941284,44.61438477)
\curveto(433.23940666,44.41438447)(432.91940698,44.24938464)(432.52941284,44.11938477)
\curveto(432.42940747,44.07938481)(432.32940757,44.05438483)(432.22941284,44.04438477)
\curveto(432.12940777,44.02438486)(432.02440787,44.00438488)(431.91441284,43.98438477)
\curveto(431.86440803,43.97438491)(431.81440808,43.96938492)(431.76441284,43.96938477)
\curveto(431.72440817,43.96938492)(431.67940822,43.96438492)(431.62941284,43.95438477)
\lineto(431.47941284,43.95438477)
\curveto(431.42940847,43.94438494)(431.36940853,43.93938495)(431.29941284,43.93938477)
\curveto(431.23940866,43.93938495)(431.18940871,43.94438494)(431.14941284,43.95438477)
\lineto(431.01441284,43.95438477)
\curveto(430.96440893,43.96438492)(430.91940898,43.96938492)(430.87941284,43.96938477)
\curveto(430.83940906,43.96938492)(430.7994091,43.97438491)(430.75941284,43.98438477)
\curveto(430.70940919,43.99438489)(430.65440924,44.00438488)(430.59441284,44.01438477)
\curveto(430.53440936,44.01438487)(430.47940942,44.01938487)(430.42941284,44.02938477)
\curveto(430.33940956,44.04938484)(430.24940965,44.07438481)(430.15941284,44.10438477)
\curveto(430.06940983,44.12438476)(429.98440991,44.14938474)(429.90441284,44.17938477)
\curveto(429.86441003,44.19938469)(429.82941007,44.20938468)(429.79941284,44.20938477)
\curveto(429.76941013,44.21938467)(429.73441016,44.23438465)(429.69441284,44.25438477)
\curveto(429.54441035,44.32438456)(429.38441051,44.40938448)(429.21441284,44.50938477)
\curveto(428.92441097,44.69938419)(428.67441122,44.92938396)(428.46441284,45.19938477)
\curveto(428.26441163,45.47938341)(428.0944118,45.7893831)(427.95441284,46.12938477)
\curveto(427.90441199,46.23938265)(427.86441203,46.35438253)(427.83441284,46.47438477)
\curveto(427.81441208,46.59438229)(427.78441211,46.71438217)(427.74441284,46.83438477)
\curveto(427.73441216,46.87438201)(427.72941217,46.90938198)(427.72941284,46.93938477)
\curveto(427.72941217,46.96938192)(427.72441217,47.00938188)(427.71441284,47.05938477)
\curveto(427.6944122,47.13938175)(427.67941222,47.22438166)(427.66941284,47.31438477)
\curveto(427.65941224,47.40438148)(427.64441225,47.49438139)(427.62441284,47.58438477)
\lineto(427.62441284,47.79438477)
\curveto(427.61441228,47.83438105)(427.60441229,47.889381)(427.59441284,47.95938477)
\curveto(427.5944123,48.03938085)(427.5994123,48.10438078)(427.60941284,48.15438477)
\lineto(427.60941284,48.31938477)
\curveto(427.62941227,48.36938052)(427.63441226,48.41938047)(427.62441284,48.46938477)
\curveto(427.62441227,48.52938036)(427.62941227,48.5843803)(427.63941284,48.63438477)
\curveto(427.67941222,48.79438009)(427.70941219,48.95437993)(427.72941284,49.11438477)
\curveto(427.75941214,49.27437961)(427.80441209,49.42437946)(427.86441284,49.56438477)
\curveto(427.91441198,49.67437921)(427.95941194,49.7843791)(427.99941284,49.89438477)
\curveto(428.04941185,50.01437887)(428.10441179,50.12937876)(428.16441284,50.23938477)
\curveto(428.38441151,50.5893783)(428.63441126,50.889378)(428.91441284,51.13938477)
\curveto(429.1944107,51.39937749)(429.53941036,51.61437727)(429.94941284,51.78438477)
\curveto(430.06940983,51.83437705)(430.18940971,51.86937702)(430.30941284,51.88938477)
\curveto(430.43940946,51.91937697)(430.57440932,51.94937694)(430.71441284,51.97938477)
\curveto(430.76440913,51.9893769)(430.80940909,51.99437689)(430.84941284,51.99438477)
\curveto(430.88940901,52.00437688)(430.93440896,52.00937688)(430.98441284,52.00938477)
\curveto(431.00440889,52.01937687)(431.02940887,52.01937687)(431.05941284,52.00938477)
\curveto(431.08940881,51.99937689)(431.11440878,52.00437688)(431.13441284,52.02438477)
\curveto(431.55440834,52.03437685)(431.91940798,51.9893769)(432.22941284,51.88938477)
\curveto(432.53940736,51.79937709)(432.81940708,51.67437721)(433.06941284,51.51438477)
\curveto(433.11940678,51.49437739)(433.15940674,51.46437742)(433.18941284,51.42438477)
\curveto(433.21940668,51.39437749)(433.25440664,51.36937752)(433.29441284,51.34938477)
\curveto(433.37440652,51.2893776)(433.45440644,51.21937767)(433.53441284,51.13938477)
\curveto(433.62440627,51.05937783)(433.6994062,50.97937791)(433.75941284,50.89938477)
\curveto(433.91940598,50.6893782)(434.05440584,50.4893784)(434.16441284,50.29938477)
\curveto(434.23440566,50.1893787)(434.28940561,50.06937882)(434.32941284,49.93938477)
\curveto(434.36940553,49.80937908)(434.41440548,49.67937921)(434.46441284,49.54938477)
\curveto(434.51440538,49.41937947)(434.54940535,49.2843796)(434.56941284,49.14438477)
\curveto(434.5994053,49.00437988)(434.63440526,48.86438002)(434.67441284,48.72438477)
\curveto(434.68440521,48.65438023)(434.68940521,48.5843803)(434.68941284,48.51438477)
\lineto(434.71941284,48.30438477)
\moveto(433.26441284,48.81438477)
\curveto(433.2944066,48.85438003)(433.31940658,48.90437998)(433.33941284,48.96438477)
\curveto(433.35940654,49.03437985)(433.35940654,49.10437978)(433.33941284,49.17438477)
\curveto(433.27940662,49.39437949)(433.1944067,49.59937929)(433.08441284,49.78938477)
\curveto(432.94440695,50.01937887)(432.78940711,50.21437867)(432.61941284,50.37438477)
\curveto(432.44940745,50.53437835)(432.22940767,50.66937822)(431.95941284,50.77938477)
\curveto(431.88940801,50.79937809)(431.81940808,50.81437807)(431.74941284,50.82438477)
\curveto(431.67940822,50.84437804)(431.60440829,50.86437802)(431.52441284,50.88438477)
\curveto(431.44440845,50.90437798)(431.35940854,50.91437797)(431.26941284,50.91438477)
\lineto(431.01441284,50.91438477)
\curveto(430.98440891,50.89437799)(430.94940895,50.884378)(430.90941284,50.88438477)
\curveto(430.86940903,50.89437799)(430.83440906,50.89437799)(430.80441284,50.88438477)
\lineto(430.56441284,50.82438477)
\curveto(430.4944094,50.81437807)(430.42440947,50.79937809)(430.35441284,50.77938477)
\curveto(430.06440983,50.65937823)(429.82941007,50.50937838)(429.64941284,50.32938477)
\curveto(429.47941042,50.14937874)(429.32441057,49.92437896)(429.18441284,49.65438477)
\curveto(429.15441074,49.60437928)(429.12441077,49.53937935)(429.09441284,49.45938477)
\curveto(429.06441083,49.3893795)(429.03941086,49.30937958)(429.01941284,49.21938477)
\curveto(428.9994109,49.12937976)(428.9944109,49.04437984)(429.00441284,48.96438477)
\curveto(429.01441088,48.88438)(429.04941085,48.82438006)(429.10941284,48.78438477)
\curveto(429.18941071,48.72438016)(429.32441057,48.69438019)(429.51441284,48.69438477)
\curveto(429.71441018,48.70438018)(429.88441001,48.70938018)(430.02441284,48.70938477)
\lineto(432.30441284,48.70938477)
\curveto(432.45440744,48.70938018)(432.63440726,48.70438018)(432.84441284,48.69438477)
\curveto(433.05440684,48.69438019)(433.1944067,48.73438015)(433.26441284,48.81438477)
}
}
{
\newrgbcolor{curcolor}{0 0 0}
\pscustom[linestyle=none,fillstyle=solid,fillcolor=curcolor]
{
\newpath
\moveto(555.93253418,49.74439941)
\lineto(555.93253418,49.47439941)
\curveto(555.94252421,49.38439416)(555.93752421,49.30439424)(555.91753418,49.23439941)
\lineto(555.91753418,49.08439941)
\curveto(555.90752424,49.05439449)(555.90252425,49.01939453)(555.90253418,48.97939941)
\curveto(555.91252424,48.93939461)(555.91252424,48.90939464)(555.90253418,48.88939941)
\curveto(555.89252426,48.83939471)(555.88752426,48.78439476)(555.88753418,48.72439941)
\curveto(555.88752426,48.67439487)(555.88252427,48.62439492)(555.87253418,48.57439941)
\curveto(555.84252431,48.43439511)(555.82252433,48.28439526)(555.81253418,48.12439941)
\curveto(555.80252435,47.97439557)(555.77252438,47.82939572)(555.72253418,47.68939941)
\curveto(555.69252446,47.56939598)(555.65752449,47.4443961)(555.61753418,47.31439941)
\curveto(555.58752456,47.19439635)(555.5475246,47.07439647)(555.49753418,46.95439941)
\curveto(555.32752482,46.52439702)(555.11252504,46.13439741)(554.85253418,45.78439941)
\curveto(554.60252555,45.4443981)(554.28752586,45.15439839)(553.90753418,44.91439941)
\curveto(553.71752643,44.79439875)(553.51252664,44.68939886)(553.29253418,44.59939941)
\curveto(553.08252707,44.51939903)(552.8525273,44.43939911)(552.60253418,44.35939941)
\curveto(552.49252766,44.31939923)(552.37252778,44.28939926)(552.24253418,44.26939941)
\curveto(552.12252803,44.25939929)(552.00252815,44.23939931)(551.88253418,44.20939941)
\curveto(551.77252838,44.18939936)(551.66252849,44.17939937)(551.55253418,44.17939941)
\curveto(551.4525287,44.17939937)(551.3525288,44.16939938)(551.25253418,44.14939941)
\lineto(551.04253418,44.14939941)
\curveto(551.01252914,44.13939941)(550.97752917,44.13439941)(550.93753418,44.13439941)
\curveto(550.89752925,44.1443994)(550.85752929,44.1493994)(550.81753418,44.14939941)
\lineto(547.81753418,44.14939941)
\curveto(547.66753248,44.1493994)(547.53253262,44.15439939)(547.41253418,44.16439941)
\curveto(547.30253285,44.18439936)(547.22753292,44.2493993)(547.18753418,44.35939941)
\curveto(547.147533,44.43939911)(547.12753302,44.55439899)(547.12753418,44.70439941)
\curveto(547.13753301,44.85439869)(547.14253301,44.98939856)(547.14253418,45.10939941)
\lineto(547.14253418,53.97439941)
\curveto(547.14253301,54.09438945)(547.13753301,54.21938933)(547.12753418,54.34939941)
\curveto(547.12753302,54.48938906)(547.152533,54.59938895)(547.20253418,54.67939941)
\curveto(547.24253291,54.7493888)(547.31753283,54.79438875)(547.42753418,54.81439941)
\curveto(547.4475327,54.82438872)(547.46753268,54.82438872)(547.48753418,54.81439941)
\curveto(547.50753264,54.81438873)(547.52753262,54.81938873)(547.54753418,54.82939941)
\lineto(550.80253418,54.82939941)
\curveto(550.8525293,54.82938872)(550.89752925,54.82938872)(550.93753418,54.82939941)
\curveto(550.98752916,54.83938871)(551.03252912,54.83938871)(551.07253418,54.82939941)
\curveto(551.12252903,54.80938874)(551.17252898,54.80438874)(551.22253418,54.81439941)
\curveto(551.28252887,54.82438872)(551.33752881,54.82438872)(551.38753418,54.81439941)
\curveto(551.43752871,54.80438874)(551.49252866,54.79938875)(551.55253418,54.79939941)
\curveto(551.61252854,54.79938875)(551.66752848,54.79438875)(551.71753418,54.78439941)
\curveto(551.76752838,54.77438877)(551.81252834,54.76938878)(551.85253418,54.76939941)
\curveto(551.90252825,54.76938878)(551.9525282,54.76438878)(552.00253418,54.75439941)
\curveto(552.11252804,54.73438881)(552.21752793,54.71438883)(552.31753418,54.69439941)
\curveto(552.41752773,54.68438886)(552.51752763,54.66438888)(552.61753418,54.63439941)
\curveto(552.83752731,54.56438898)(553.0475271,54.49438905)(553.24753418,54.42439941)
\curveto(553.4475267,54.36438918)(553.63252652,54.27938927)(553.80253418,54.16939941)
\curveto(553.94252621,54.08938946)(554.06752608,54.00938954)(554.17753418,53.92939941)
\curveto(554.20752594,53.90938964)(554.23752591,53.88438966)(554.26753418,53.85439941)
\curveto(554.29752585,53.83438971)(554.32752582,53.81438973)(554.35753418,53.79439941)
\curveto(554.41752573,53.7443898)(554.47252568,53.69438985)(554.52253418,53.64439941)
\curveto(554.57252558,53.59438995)(554.62252553,53.54439)(554.67253418,53.49439941)
\curveto(554.72252543,53.4443901)(554.76252539,53.40939014)(554.79253418,53.38939941)
\curveto(554.83252532,53.32939022)(554.87252528,53.27439027)(554.91253418,53.22439941)
\curveto(554.96252519,53.17439037)(555.00752514,53.11939043)(555.04753418,53.05939941)
\curveto(555.09752505,52.99939055)(555.13752501,52.93439061)(555.16753418,52.86439941)
\curveto(555.20752494,52.80439074)(555.2525249,52.73939081)(555.30253418,52.66939941)
\curveto(555.32252483,52.62939092)(555.33752481,52.59439095)(555.34753418,52.56439941)
\curveto(555.35752479,52.53439101)(555.37252478,52.49939105)(555.39253418,52.45939941)
\curveto(555.43252472,52.37939117)(555.46752468,52.29939125)(555.49753418,52.21939941)
\curveto(555.52752462,52.1493914)(555.56252459,52.07439147)(555.60253418,51.99439941)
\curveto(555.64252451,51.88439166)(555.67252448,51.76939178)(555.69253418,51.64939941)
\curveto(555.72252443,51.53939201)(555.7525244,51.42939212)(555.78253418,51.31939941)
\curveto(555.80252435,51.25939229)(555.81252434,51.19939235)(555.81253418,51.13939941)
\curveto(555.81252434,51.08939246)(555.82252433,51.03439251)(555.84253418,50.97439941)
\curveto(555.89252426,50.79439275)(555.91752423,50.59439295)(555.91753418,50.37439941)
\curveto(555.92752422,50.16439338)(555.93252422,49.95439359)(555.93253418,49.74439941)
\moveto(554.50753418,48.96439941)
\curveto(554.52752562,49.06439448)(554.53752561,49.16939438)(554.53753418,49.27939941)
\lineto(554.53753418,49.62439941)
\lineto(554.53753418,49.84939941)
\curveto(554.5475256,49.92939362)(554.54252561,50.00439354)(554.52253418,50.07439941)
\curveto(554.52252563,50.10439344)(554.51752563,50.13439341)(554.50753418,50.16439941)
\lineto(554.50753418,50.26939941)
\curveto(554.48752566,50.37939317)(554.47252568,50.48939306)(554.46253418,50.59939941)
\curveto(554.46252569,50.70939284)(554.4475257,50.81939273)(554.41753418,50.92939941)
\curveto(554.39752575,51.00939254)(554.37752577,51.08439246)(554.35753418,51.15439941)
\curveto(554.3475258,51.23439231)(554.33252582,51.31439223)(554.31253418,51.39439941)
\curveto(554.20252595,51.75439179)(554.06252609,52.06939148)(553.89253418,52.33939941)
\curveto(553.61252654,52.78939076)(553.19752695,53.12939042)(552.64753418,53.35939941)
\curveto(552.55752759,53.40939014)(552.46252769,53.4443901)(552.36253418,53.46439941)
\curveto(552.26252789,53.49439005)(552.15752799,53.52439002)(552.04753418,53.55439941)
\curveto(551.93752821,53.58438996)(551.82252833,53.59938995)(551.70253418,53.59939941)
\curveto(551.59252856,53.60938994)(551.48252867,53.62438992)(551.37253418,53.64439941)
\lineto(551.05753418,53.64439941)
\curveto(551.02752912,53.65438989)(550.99252916,53.65938989)(550.95253418,53.65939941)
\lineto(550.83253418,53.65939941)
\lineto(549.00253418,53.65939941)
\curveto(548.98253117,53.6493899)(548.95753119,53.6443899)(548.92753418,53.64439941)
\curveto(548.89753125,53.65438989)(548.87253128,53.65438989)(548.85253418,53.64439941)
\lineto(548.70253418,53.58439941)
\curveto(548.66253149,53.56438998)(548.63253152,53.53439001)(548.61253418,53.49439941)
\curveto(548.59253156,53.45439009)(548.57253158,53.38439016)(548.55253418,53.28439941)
\lineto(548.55253418,53.16439941)
\curveto(548.54253161,53.12439042)(548.53753161,53.07939047)(548.53753418,53.02939941)
\lineto(548.53753418,52.89439941)
\lineto(548.53753418,46.08439941)
\lineto(548.53753418,45.93439941)
\curveto(548.53753161,45.89439765)(548.54253161,45.85439769)(548.55253418,45.81439941)
\lineto(548.55253418,45.69439941)
\curveto(548.57253158,45.59439795)(548.59253156,45.52439802)(548.61253418,45.48439941)
\curveto(548.69253146,45.36439818)(548.84253131,45.30439824)(549.06253418,45.30439941)
\curveto(549.28253087,45.31439823)(549.49253066,45.31939823)(549.69253418,45.31939941)
\lineto(550.56253418,45.31939941)
\curveto(550.63252952,45.31939823)(550.70752944,45.31439823)(550.78753418,45.30439941)
\curveto(550.86752928,45.30439824)(550.93752921,45.31439823)(550.99753418,45.33439941)
\lineto(551.16253418,45.33439941)
\curveto(551.21252894,45.3443982)(551.26752888,45.3443982)(551.32753418,45.33439941)
\curveto(551.38752876,45.33439821)(551.4475287,45.33939821)(551.50753418,45.34939941)
\curveto(551.56752858,45.36939818)(551.62752852,45.37939817)(551.68753418,45.37939941)
\curveto(551.7475284,45.38939816)(551.81252834,45.40439814)(551.88253418,45.42439941)
\curveto(551.99252816,45.45439809)(552.09752805,45.48439806)(552.19753418,45.51439941)
\curveto(552.30752784,45.544398)(552.41752773,45.58439796)(552.52753418,45.63439941)
\curveto(552.89752725,45.79439775)(553.21252694,46.00939754)(553.47253418,46.27939941)
\curveto(553.74252641,46.55939699)(553.96252619,46.88939666)(554.13253418,47.26939941)
\curveto(554.18252597,47.37939617)(554.22252593,47.49439605)(554.25253418,47.61439941)
\lineto(554.37253418,48.00439941)
\curveto(554.40252575,48.11439543)(554.42252573,48.22939532)(554.43253418,48.34939941)
\curveto(554.4525257,48.47939507)(554.47252568,48.60439494)(554.49253418,48.72439941)
\curveto(554.50252565,48.77439477)(554.50752564,48.81439473)(554.50753418,48.84439941)
\lineto(554.50753418,48.96439941)
}
}
{
\newrgbcolor{curcolor}{0 0 0}
\pscustom[linestyle=none,fillstyle=solid,fillcolor=curcolor]
{
\newpath
\moveto(564.18440918,48.31939941)
\curveto(564.20440149,48.21939533)(564.20440149,48.10439544)(564.18440918,47.97439941)
\curveto(564.17440152,47.85439569)(564.14440155,47.76939578)(564.09440918,47.71939941)
\curveto(564.04440165,47.67939587)(563.96940173,47.6493959)(563.86940918,47.62939941)
\curveto(563.77940192,47.61939593)(563.67440202,47.61439593)(563.55440918,47.61439941)
\lineto(563.19440918,47.61439941)
\curveto(563.07440262,47.62439592)(562.96940273,47.62939592)(562.87940918,47.62939941)
\lineto(559.03940918,47.62939941)
\curveto(558.95940674,47.62939592)(558.87940682,47.62439592)(558.79940918,47.61439941)
\curveto(558.71940698,47.61439593)(558.65440704,47.59939595)(558.60440918,47.56939941)
\curveto(558.56440713,47.549396)(558.52440717,47.50939604)(558.48440918,47.44939941)
\curveto(558.46440723,47.41939613)(558.44440725,47.37439617)(558.42440918,47.31439941)
\curveto(558.40440729,47.26439628)(558.40440729,47.21439633)(558.42440918,47.16439941)
\curveto(558.43440726,47.11439643)(558.43940726,47.06939648)(558.43940918,47.02939941)
\curveto(558.43940726,46.98939656)(558.44440725,46.9493966)(558.45440918,46.90939941)
\curveto(558.47440722,46.82939672)(558.4944072,46.7443968)(558.51440918,46.65439941)
\curveto(558.53440716,46.57439697)(558.56440713,46.49439705)(558.60440918,46.41439941)
\curveto(558.83440686,45.87439767)(559.21440648,45.48939806)(559.74440918,45.25939941)
\curveto(559.80440589,45.22939832)(559.86940583,45.20439834)(559.93940918,45.18439941)
\lineto(560.14940918,45.12439941)
\curveto(560.17940552,45.11439843)(560.22940547,45.10939844)(560.29940918,45.10939941)
\curveto(560.43940526,45.06939848)(560.62440507,45.0493985)(560.85440918,45.04939941)
\curveto(561.08440461,45.0493985)(561.26940443,45.06939848)(561.40940918,45.10939941)
\curveto(561.54940415,45.1493984)(561.67440402,45.18939836)(561.78440918,45.22939941)
\curveto(561.90440379,45.27939827)(562.01440368,45.33939821)(562.11440918,45.40939941)
\curveto(562.22440347,45.47939807)(562.31940338,45.55939799)(562.39940918,45.64939941)
\curveto(562.47940322,45.7493978)(562.54940315,45.85439769)(562.60940918,45.96439941)
\curveto(562.66940303,46.06439748)(562.71940298,46.16939738)(562.75940918,46.27939941)
\curveto(562.80940289,46.38939716)(562.88940281,46.46939708)(562.99940918,46.51939941)
\curveto(563.03940266,46.53939701)(563.10440259,46.55439699)(563.19440918,46.56439941)
\curveto(563.28440241,46.57439697)(563.37440232,46.57439697)(563.46440918,46.56439941)
\curveto(563.55440214,46.56439698)(563.63940206,46.55939699)(563.71940918,46.54939941)
\curveto(563.7994019,46.53939701)(563.85440184,46.51939703)(563.88440918,46.48939941)
\curveto(563.98440171,46.41939713)(564.00940169,46.30439724)(563.95940918,46.14439941)
\curveto(563.87940182,45.87439767)(563.77440192,45.63439791)(563.64440918,45.42439941)
\curveto(563.44440225,45.10439844)(563.21440248,44.83939871)(562.95440918,44.62939941)
\curveto(562.70440299,44.42939912)(562.38440331,44.26439928)(561.99440918,44.13439941)
\curveto(561.8944038,44.09439945)(561.7944039,44.06939948)(561.69440918,44.05939941)
\curveto(561.5944041,44.03939951)(561.48940421,44.01939953)(561.37940918,43.99939941)
\curveto(561.32940437,43.98939956)(561.27940442,43.98439956)(561.22940918,43.98439941)
\curveto(561.18940451,43.98439956)(561.14440455,43.97939957)(561.09440918,43.96939941)
\lineto(560.94440918,43.96939941)
\curveto(560.8944048,43.95939959)(560.83440486,43.95439959)(560.76440918,43.95439941)
\curveto(560.70440499,43.95439959)(560.65440504,43.95939959)(560.61440918,43.96939941)
\lineto(560.47940918,43.96939941)
\curveto(560.42940527,43.97939957)(560.38440531,43.98439956)(560.34440918,43.98439941)
\curveto(560.30440539,43.98439956)(560.26440543,43.98939956)(560.22440918,43.99939941)
\curveto(560.17440552,44.00939954)(560.11940558,44.01939953)(560.05940918,44.02939941)
\curveto(559.9994057,44.02939952)(559.94440575,44.03439951)(559.89440918,44.04439941)
\curveto(559.80440589,44.06439948)(559.71440598,44.08939946)(559.62440918,44.11939941)
\curveto(559.53440616,44.13939941)(559.44940625,44.16439938)(559.36940918,44.19439941)
\curveto(559.32940637,44.21439933)(559.2944064,44.22439932)(559.26440918,44.22439941)
\curveto(559.23440646,44.23439931)(559.1994065,44.2493993)(559.15940918,44.26939941)
\curveto(559.00940669,44.33939921)(558.84940685,44.42439912)(558.67940918,44.52439941)
\curveto(558.38940731,44.71439883)(558.13940756,44.9443986)(557.92940918,45.21439941)
\curveto(557.72940797,45.49439805)(557.55940814,45.80439774)(557.41940918,46.14439941)
\curveto(557.36940833,46.25439729)(557.32940837,46.36939718)(557.29940918,46.48939941)
\curveto(557.27940842,46.60939694)(557.24940845,46.72939682)(557.20940918,46.84939941)
\curveto(557.1994085,46.88939666)(557.1944085,46.92439662)(557.19440918,46.95439941)
\curveto(557.1944085,46.98439656)(557.18940851,47.02439652)(557.17940918,47.07439941)
\curveto(557.15940854,47.15439639)(557.14440855,47.23939631)(557.13440918,47.32939941)
\curveto(557.12440857,47.41939613)(557.10940859,47.50939604)(557.08940918,47.59939941)
\lineto(557.08940918,47.80939941)
\curveto(557.07940862,47.8493957)(557.06940863,47.90439564)(557.05940918,47.97439941)
\curveto(557.05940864,48.05439549)(557.06440863,48.11939543)(557.07440918,48.16939941)
\lineto(557.07440918,48.33439941)
\curveto(557.0944086,48.38439516)(557.0994086,48.43439511)(557.08940918,48.48439941)
\curveto(557.08940861,48.544395)(557.0944086,48.59939495)(557.10440918,48.64939941)
\curveto(557.14440855,48.80939474)(557.17440852,48.96939458)(557.19440918,49.12939941)
\curveto(557.22440847,49.28939426)(557.26940843,49.43939411)(557.32940918,49.57939941)
\curveto(557.37940832,49.68939386)(557.42440827,49.79939375)(557.46440918,49.90939941)
\curveto(557.51440818,50.02939352)(557.56940813,50.1443934)(557.62940918,50.25439941)
\curveto(557.84940785,50.60439294)(558.0994076,50.90439264)(558.37940918,51.15439941)
\curveto(558.65940704,51.41439213)(559.00440669,51.62939192)(559.41440918,51.79939941)
\curveto(559.53440616,51.8493917)(559.65440604,51.88439166)(559.77440918,51.90439941)
\curveto(559.90440579,51.93439161)(560.03940566,51.96439158)(560.17940918,51.99439941)
\curveto(560.22940547,52.00439154)(560.27440542,52.00939154)(560.31440918,52.00939941)
\curveto(560.35440534,52.01939153)(560.3994053,52.02439152)(560.44940918,52.02439941)
\curveto(560.46940523,52.03439151)(560.4944052,52.03439151)(560.52440918,52.02439941)
\curveto(560.55440514,52.01439153)(560.57940512,52.01939153)(560.59940918,52.03939941)
\curveto(561.01940468,52.0493915)(561.38440431,52.00439154)(561.69440918,51.90439941)
\curveto(562.00440369,51.81439173)(562.28440341,51.68939186)(562.53440918,51.52939941)
\curveto(562.58440311,51.50939204)(562.62440307,51.47939207)(562.65440918,51.43939941)
\curveto(562.68440301,51.40939214)(562.71940298,51.38439216)(562.75940918,51.36439941)
\curveto(562.83940286,51.30439224)(562.91940278,51.23439231)(562.99940918,51.15439941)
\curveto(563.08940261,51.07439247)(563.16440253,50.99439255)(563.22440918,50.91439941)
\curveto(563.38440231,50.70439284)(563.51940218,50.50439304)(563.62940918,50.31439941)
\curveto(563.699402,50.20439334)(563.75440194,50.08439346)(563.79440918,49.95439941)
\curveto(563.83440186,49.82439372)(563.87940182,49.69439385)(563.92940918,49.56439941)
\curveto(563.97940172,49.43439411)(564.01440168,49.29939425)(564.03440918,49.15939941)
\curveto(564.06440163,49.01939453)(564.0994016,48.87939467)(564.13940918,48.73939941)
\curveto(564.14940155,48.66939488)(564.15440154,48.59939495)(564.15440918,48.52939941)
\lineto(564.18440918,48.31939941)
\moveto(562.72940918,48.82939941)
\curveto(562.75940294,48.86939468)(562.78440291,48.91939463)(562.80440918,48.97939941)
\curveto(562.82440287,49.0493945)(562.82440287,49.11939443)(562.80440918,49.18939941)
\curveto(562.74440295,49.40939414)(562.65940304,49.61439393)(562.54940918,49.80439941)
\curveto(562.40940329,50.03439351)(562.25440344,50.22939332)(562.08440918,50.38939941)
\curveto(561.91440378,50.549393)(561.694404,50.68439286)(561.42440918,50.79439941)
\curveto(561.35440434,50.81439273)(561.28440441,50.82939272)(561.21440918,50.83939941)
\curveto(561.14440455,50.85939269)(561.06940463,50.87939267)(560.98940918,50.89939941)
\curveto(560.90940479,50.91939263)(560.82440487,50.92939262)(560.73440918,50.92939941)
\lineto(560.47940918,50.92939941)
\curveto(560.44940525,50.90939264)(560.41440528,50.89939265)(560.37440918,50.89939941)
\curveto(560.33440536,50.90939264)(560.2994054,50.90939264)(560.26940918,50.89939941)
\lineto(560.02940918,50.83939941)
\curveto(559.95940574,50.82939272)(559.88940581,50.81439273)(559.81940918,50.79439941)
\curveto(559.52940617,50.67439287)(559.2944064,50.52439302)(559.11440918,50.34439941)
\curveto(558.94440675,50.16439338)(558.78940691,49.93939361)(558.64940918,49.66939941)
\curveto(558.61940708,49.61939393)(558.58940711,49.55439399)(558.55940918,49.47439941)
\curveto(558.52940717,49.40439414)(558.50440719,49.32439422)(558.48440918,49.23439941)
\curveto(558.46440723,49.1443944)(558.45940724,49.05939449)(558.46940918,48.97939941)
\curveto(558.47940722,48.89939465)(558.51440718,48.83939471)(558.57440918,48.79939941)
\curveto(558.65440704,48.73939481)(558.78940691,48.70939484)(558.97940918,48.70939941)
\curveto(559.17940652,48.71939483)(559.34940635,48.72439482)(559.48940918,48.72439941)
\lineto(561.76940918,48.72439941)
\curveto(561.91940378,48.72439482)(562.0994036,48.71939483)(562.30940918,48.70939941)
\curveto(562.51940318,48.70939484)(562.65940304,48.7493948)(562.72940918,48.82939941)
}
}
{
\newrgbcolor{curcolor}{0 0 0}
\pscustom[linestyle=none,fillstyle=solid,fillcolor=curcolor]
{
\newpath
\moveto(567.9210498,52.05439941)
\curveto(568.64104574,52.06439148)(569.24604513,51.97939157)(569.7360498,51.79939941)
\curveto(570.22604415,51.62939192)(570.60604377,51.32439222)(570.8760498,50.88439941)
\curveto(570.94604343,50.77439277)(571.00104338,50.65939289)(571.0410498,50.53939941)
\curveto(571.0810433,50.42939312)(571.12104326,50.30439324)(571.1610498,50.16439941)
\curveto(571.1810432,50.09439345)(571.18604319,50.01939353)(571.1760498,49.93939941)
\curveto(571.16604321,49.86939368)(571.15104323,49.81439373)(571.1310498,49.77439941)
\curveto(571.11104327,49.75439379)(571.08604329,49.73439381)(571.0560498,49.71439941)
\curveto(571.02604335,49.70439384)(571.00104338,49.68939386)(570.9810498,49.66939941)
\curveto(570.93104345,49.6493939)(570.8810435,49.6443939)(570.8310498,49.65439941)
\curveto(570.7810436,49.66439388)(570.73104365,49.66439388)(570.6810498,49.65439941)
\curveto(570.60104378,49.63439391)(570.49604388,49.62939392)(570.3660498,49.63939941)
\curveto(570.23604414,49.65939389)(570.14604423,49.68439386)(570.0960498,49.71439941)
\curveto(570.01604436,49.76439378)(569.96104442,49.82939372)(569.9310498,49.90939941)
\curveto(569.91104447,49.99939355)(569.8760445,50.08439346)(569.8260498,50.16439941)
\curveto(569.73604464,50.32439322)(569.61104477,50.46939308)(569.4510498,50.59939941)
\curveto(569.34104504,50.67939287)(569.22104516,50.73939281)(569.0910498,50.77939941)
\curveto(568.96104542,50.81939273)(568.82104556,50.85939269)(568.6710498,50.89939941)
\curveto(568.62104576,50.91939263)(568.57104581,50.92439262)(568.5210498,50.91439941)
\curveto(568.47104591,50.91439263)(568.42104596,50.91939263)(568.3710498,50.92939941)
\curveto(568.31104607,50.9493926)(568.23604614,50.95939259)(568.1460498,50.95939941)
\curveto(568.05604632,50.95939259)(567.9810464,50.9493926)(567.9210498,50.92939941)
\lineto(567.8310498,50.92939941)
\lineto(567.6810498,50.89939941)
\curveto(567.63104675,50.89939265)(567.5810468,50.89439265)(567.5310498,50.88439941)
\curveto(567.27104711,50.82439272)(567.05604732,50.73939281)(566.8860498,50.62939941)
\curveto(566.71604766,50.51939303)(566.60104778,50.33439321)(566.5410498,50.07439941)
\curveto(566.52104786,50.00439354)(566.51604786,49.93439361)(566.5260498,49.86439941)
\curveto(566.54604783,49.79439375)(566.56604781,49.73439381)(566.5860498,49.68439941)
\curveto(566.64604773,49.53439401)(566.71604766,49.42439412)(566.7960498,49.35439941)
\curveto(566.88604749,49.29439425)(566.99604738,49.22439432)(567.1260498,49.14439941)
\curveto(567.28604709,49.0443945)(567.46604691,48.96939458)(567.6660498,48.91939941)
\curveto(567.86604651,48.87939467)(568.06604631,48.82939472)(568.2660498,48.76939941)
\curveto(568.39604598,48.72939482)(568.52604585,48.69939485)(568.6560498,48.67939941)
\curveto(568.78604559,48.65939489)(568.91604546,48.62939492)(569.0460498,48.58939941)
\curveto(569.25604512,48.52939502)(569.46104492,48.46939508)(569.6610498,48.40939941)
\curveto(569.86104452,48.35939519)(570.06104432,48.29439525)(570.2610498,48.21439941)
\lineto(570.4110498,48.15439941)
\curveto(570.46104392,48.13439541)(570.51104387,48.10939544)(570.5610498,48.07939941)
\curveto(570.76104362,47.95939559)(570.93604344,47.82439572)(571.0860498,47.67439941)
\curveto(571.23604314,47.52439602)(571.36104302,47.33439621)(571.4610498,47.10439941)
\curveto(571.4810429,47.03439651)(571.50104288,46.93939661)(571.5210498,46.81939941)
\curveto(571.54104284,46.7493968)(571.55104283,46.67439687)(571.5510498,46.59439941)
\curveto(571.56104282,46.52439702)(571.56604281,46.4443971)(571.5660498,46.35439941)
\lineto(571.5660498,46.20439941)
\curveto(571.54604283,46.13439741)(571.53604284,46.06439748)(571.5360498,45.99439941)
\curveto(571.53604284,45.92439762)(571.52604285,45.85439769)(571.5060498,45.78439941)
\curveto(571.4760429,45.67439787)(571.44104294,45.56939798)(571.4010498,45.46939941)
\curveto(571.36104302,45.36939818)(571.31604306,45.27939827)(571.2660498,45.19939941)
\curveto(571.10604327,44.93939861)(570.90104348,44.72939882)(570.6510498,44.56939941)
\curveto(570.40104398,44.41939913)(570.12104426,44.28939926)(569.8110498,44.17939941)
\curveto(569.72104466,44.1493994)(569.62604475,44.12939942)(569.5260498,44.11939941)
\curveto(569.43604494,44.09939945)(569.34604503,44.07439947)(569.2560498,44.04439941)
\curveto(569.15604522,44.02439952)(569.05604532,44.01439953)(568.9560498,44.01439941)
\curveto(568.85604552,44.01439953)(568.75604562,44.00439954)(568.6560498,43.98439941)
\lineto(568.5060498,43.98439941)
\curveto(568.45604592,43.97439957)(568.38604599,43.96939958)(568.2960498,43.96939941)
\curveto(568.20604617,43.96939958)(568.13604624,43.97439957)(568.0860498,43.98439941)
\lineto(567.9210498,43.98439941)
\curveto(567.86104652,44.00439954)(567.79604658,44.01439953)(567.7260498,44.01439941)
\curveto(567.65604672,44.00439954)(567.59604678,44.00939954)(567.5460498,44.02939941)
\curveto(567.49604688,44.03939951)(567.43104695,44.0443995)(567.3510498,44.04439941)
\lineto(567.1110498,44.10439941)
\curveto(567.04104734,44.11439943)(566.96604741,44.13439941)(566.8860498,44.16439941)
\curveto(566.5760478,44.26439928)(566.30604807,44.38939916)(566.0760498,44.53939941)
\curveto(565.84604853,44.68939886)(565.64604873,44.88439866)(565.4760498,45.12439941)
\curveto(565.38604899,45.25439829)(565.31104907,45.38939816)(565.2510498,45.52939941)
\curveto(565.19104919,45.66939788)(565.13604924,45.82439772)(565.0860498,45.99439941)
\curveto(565.06604931,46.05439749)(565.05604932,46.12439742)(565.0560498,46.20439941)
\curveto(565.06604931,46.29439725)(565.0810493,46.36439718)(565.1010498,46.41439941)
\curveto(565.13104925,46.45439709)(565.1810492,46.49439705)(565.2510498,46.53439941)
\curveto(565.30104908,46.55439699)(565.37104901,46.56439698)(565.4610498,46.56439941)
\curveto(565.55104883,46.57439697)(565.64104874,46.57439697)(565.7310498,46.56439941)
\curveto(565.82104856,46.55439699)(565.90604847,46.53939701)(565.9860498,46.51939941)
\curveto(566.0760483,46.50939704)(566.13604824,46.49439705)(566.1660498,46.47439941)
\curveto(566.23604814,46.42439712)(566.2810481,46.3493972)(566.3010498,46.24939941)
\curveto(566.33104805,46.15939739)(566.36604801,46.07439747)(566.4060498,45.99439941)
\curveto(566.50604787,45.77439777)(566.64104774,45.60439794)(566.8110498,45.48439941)
\curveto(566.93104745,45.39439815)(567.06604731,45.32439822)(567.2160498,45.27439941)
\curveto(567.36604701,45.22439832)(567.52604685,45.17439837)(567.6960498,45.12439941)
\lineto(568.0110498,45.07939941)
\lineto(568.1010498,45.07939941)
\curveto(568.17104621,45.05939849)(568.26104612,45.0493985)(568.3710498,45.04939941)
\curveto(568.49104589,45.0493985)(568.59104579,45.05939849)(568.6710498,45.07939941)
\curveto(568.74104564,45.07939847)(568.79604558,45.08439846)(568.8360498,45.09439941)
\curveto(568.89604548,45.10439844)(568.95604542,45.10939844)(569.0160498,45.10939941)
\curveto(569.0760453,45.11939843)(569.13104525,45.12939842)(569.1810498,45.13939941)
\curveto(569.47104491,45.21939833)(569.70104468,45.32439822)(569.8710498,45.45439941)
\curveto(570.04104434,45.58439796)(570.16104422,45.80439774)(570.2310498,46.11439941)
\curveto(570.25104413,46.16439738)(570.25604412,46.21939733)(570.2460498,46.27939941)
\curveto(570.23604414,46.33939721)(570.22604415,46.38439716)(570.2160498,46.41439941)
\curveto(570.16604421,46.60439694)(570.09604428,46.7443968)(570.0060498,46.83439941)
\curveto(569.91604446,46.93439661)(569.80104458,47.02439652)(569.6610498,47.10439941)
\curveto(569.57104481,47.16439638)(569.47104491,47.21439633)(569.3610498,47.25439941)
\lineto(569.0310498,47.37439941)
\curveto(569.00104538,47.38439616)(568.97104541,47.38939616)(568.9410498,47.38939941)
\curveto(568.92104546,47.38939616)(568.89604548,47.39939615)(568.8660498,47.41939941)
\curveto(568.52604585,47.52939602)(568.17104621,47.60939594)(567.8010498,47.65939941)
\curveto(567.44104694,47.71939583)(567.10104728,47.81439573)(566.7810498,47.94439941)
\curveto(566.6810477,47.98439556)(566.58604779,48.01939553)(566.4960498,48.04939941)
\curveto(566.40604797,48.07939547)(566.32104806,48.11939543)(566.2410498,48.16939941)
\curveto(566.05104833,48.27939527)(565.8760485,48.40439514)(565.7160498,48.54439941)
\curveto(565.55604882,48.68439486)(565.43104895,48.85939469)(565.3410498,49.06939941)
\curveto(565.31104907,49.13939441)(565.28604909,49.20939434)(565.2660498,49.27939941)
\curveto(565.25604912,49.3493942)(565.24104914,49.42439412)(565.2210498,49.50439941)
\curveto(565.19104919,49.62439392)(565.1810492,49.75939379)(565.1910498,49.90939941)
\curveto(565.20104918,50.06939348)(565.21604916,50.20439334)(565.2360498,50.31439941)
\curveto(565.25604912,50.36439318)(565.26604911,50.40439314)(565.2660498,50.43439941)
\curveto(565.2760491,50.47439307)(565.29104909,50.51439303)(565.3110498,50.55439941)
\curveto(565.40104898,50.78439276)(565.52104886,50.98439256)(565.6710498,51.15439941)
\curveto(565.83104855,51.32439222)(566.01104837,51.47439207)(566.2110498,51.60439941)
\curveto(566.36104802,51.69439185)(566.52604785,51.76439178)(566.7060498,51.81439941)
\curveto(566.88604749,51.87439167)(567.0760473,51.92939162)(567.2760498,51.97939941)
\curveto(567.34604703,51.98939156)(567.41104697,51.99939155)(567.4710498,52.00939941)
\curveto(567.54104684,52.01939153)(567.61604676,52.02939152)(567.6960498,52.03939941)
\curveto(567.72604665,52.0493915)(567.76604661,52.0493915)(567.8160498,52.03939941)
\curveto(567.86604651,52.02939152)(567.90104648,52.03439151)(567.9210498,52.05439941)
}
}
{
\newrgbcolor{curcolor}{0 0 0}
\pscustom[linestyle=none,fillstyle=solid,fillcolor=curcolor]
{
\newpath
\moveto(579.8760498,44.70439941)
\curveto(579.90604197,44.544399)(579.89104199,44.40939914)(579.8310498,44.29939941)
\curveto(579.77104211,44.19939935)(579.69104219,44.12439942)(579.5910498,44.07439941)
\curveto(579.54104234,44.05439949)(579.48604239,44.0443995)(579.4260498,44.04439941)
\curveto(579.3760425,44.0443995)(579.32104256,44.03439951)(579.2610498,44.01439941)
\curveto(579.04104284,43.96439958)(578.82104306,43.97939957)(578.6010498,44.05939941)
\curveto(578.39104349,44.12939942)(578.24604363,44.21939933)(578.1660498,44.32939941)
\curveto(578.11604376,44.39939915)(578.07104381,44.47939907)(578.0310498,44.56939941)
\curveto(577.99104389,44.66939888)(577.94104394,44.7493988)(577.8810498,44.80939941)
\curveto(577.86104402,44.82939872)(577.83604404,44.8493987)(577.8060498,44.86939941)
\curveto(577.78604409,44.88939866)(577.75604412,44.89439865)(577.7160498,44.88439941)
\curveto(577.60604427,44.85439869)(577.50104438,44.79939875)(577.4010498,44.71939941)
\curveto(577.31104457,44.63939891)(577.22104466,44.56939898)(577.1310498,44.50939941)
\curveto(577.00104488,44.42939912)(576.86104502,44.35439919)(576.7110498,44.28439941)
\curveto(576.56104532,44.22439932)(576.40104548,44.16939938)(576.2310498,44.11939941)
\curveto(576.13104575,44.08939946)(576.02104586,44.06939948)(575.9010498,44.05939941)
\curveto(575.79104609,44.0493995)(575.6810462,44.03439951)(575.5710498,44.01439941)
\curveto(575.52104636,44.00439954)(575.4760464,43.99939955)(575.4360498,43.99939941)
\lineto(575.3310498,43.99939941)
\curveto(575.22104666,43.97939957)(575.11604676,43.97939957)(575.0160498,43.99939941)
\lineto(574.8810498,43.99939941)
\curveto(574.83104705,44.00939954)(574.7810471,44.01439953)(574.7310498,44.01439941)
\curveto(574.6810472,44.01439953)(574.63604724,44.02439952)(574.5960498,44.04439941)
\curveto(574.55604732,44.05439949)(574.52104736,44.05939949)(574.4910498,44.05939941)
\curveto(574.47104741,44.0493995)(574.44604743,44.0493995)(574.4160498,44.05939941)
\lineto(574.1760498,44.11939941)
\curveto(574.09604778,44.12939942)(574.02104786,44.1493994)(573.9510498,44.17939941)
\curveto(573.65104823,44.30939924)(573.40604847,44.45439909)(573.2160498,44.61439941)
\curveto(573.03604884,44.78439876)(572.88604899,45.01939853)(572.7660498,45.31939941)
\curveto(572.6760492,45.53939801)(572.63104925,45.80439774)(572.6310498,46.11439941)
\lineto(572.6310498,46.42939941)
\curveto(572.64104924,46.47939707)(572.64604923,46.52939702)(572.6460498,46.57939941)
\lineto(572.6760498,46.75939941)
\lineto(572.7960498,47.08939941)
\curveto(572.83604904,47.19939635)(572.88604899,47.29939625)(572.9460498,47.38939941)
\curveto(573.12604875,47.67939587)(573.37104851,47.89439565)(573.6810498,48.03439941)
\curveto(573.99104789,48.17439537)(574.33104755,48.29939525)(574.7010498,48.40939941)
\curveto(574.84104704,48.4493951)(574.98604689,48.47939507)(575.1360498,48.49939941)
\curveto(575.28604659,48.51939503)(575.43604644,48.544395)(575.5860498,48.57439941)
\curveto(575.65604622,48.59439495)(575.72104616,48.60439494)(575.7810498,48.60439941)
\curveto(575.85104603,48.60439494)(575.92604595,48.61439493)(576.0060498,48.63439941)
\curveto(576.0760458,48.65439489)(576.14604573,48.66439488)(576.2160498,48.66439941)
\curveto(576.28604559,48.67439487)(576.36104552,48.68939486)(576.4410498,48.70939941)
\curveto(576.69104519,48.76939478)(576.92604495,48.81939473)(577.1460498,48.85939941)
\curveto(577.36604451,48.90939464)(577.54104434,49.02439452)(577.6710498,49.20439941)
\curveto(577.73104415,49.28439426)(577.7810441,49.38439416)(577.8210498,49.50439941)
\curveto(577.86104402,49.63439391)(577.86104402,49.77439377)(577.8210498,49.92439941)
\curveto(577.76104412,50.16439338)(577.67104421,50.35439319)(577.5510498,50.49439941)
\curveto(577.44104444,50.63439291)(577.2810446,50.7443928)(577.0710498,50.82439941)
\curveto(576.95104493,50.87439267)(576.80604507,50.90939264)(576.6360498,50.92939941)
\curveto(576.4760454,50.9493926)(576.30604557,50.95939259)(576.1260498,50.95939941)
\curveto(575.94604593,50.95939259)(575.77104611,50.9493926)(575.6010498,50.92939941)
\curveto(575.43104645,50.90939264)(575.28604659,50.87939267)(575.1660498,50.83939941)
\curveto(574.99604688,50.77939277)(574.83104705,50.69439285)(574.6710498,50.58439941)
\curveto(574.59104729,50.52439302)(574.51604736,50.4443931)(574.4460498,50.34439941)
\curveto(574.38604749,50.25439329)(574.33104755,50.15439339)(574.2810498,50.04439941)
\curveto(574.25104763,49.96439358)(574.22104766,49.87939367)(574.1910498,49.78939941)
\curveto(574.17104771,49.69939385)(574.12604775,49.62939392)(574.0560498,49.57939941)
\curveto(574.01604786,49.549394)(573.94604793,49.52439402)(573.8460498,49.50439941)
\curveto(573.75604812,49.49439405)(573.66104822,49.48939406)(573.5610498,49.48939941)
\curveto(573.46104842,49.48939406)(573.36104852,49.49439405)(573.2610498,49.50439941)
\curveto(573.17104871,49.52439402)(573.10604877,49.549394)(573.0660498,49.57939941)
\curveto(573.02604885,49.60939394)(572.99604888,49.65939389)(572.9760498,49.72939941)
\curveto(572.95604892,49.79939375)(572.95604892,49.87439367)(572.9760498,49.95439941)
\curveto(573.00604887,50.08439346)(573.03604884,50.20439334)(573.0660498,50.31439941)
\curveto(573.10604877,50.43439311)(573.15104873,50.549393)(573.2010498,50.65939941)
\curveto(573.39104849,51.00939254)(573.63104825,51.27939227)(573.9210498,51.46939941)
\curveto(574.21104767,51.66939188)(574.57104731,51.82939172)(575.0010498,51.94939941)
\curveto(575.10104678,51.96939158)(575.20104668,51.98439156)(575.3010498,51.99439941)
\curveto(575.41104647,52.00439154)(575.52104636,52.01939153)(575.6310498,52.03939941)
\curveto(575.67104621,52.0493915)(575.73604614,52.0493915)(575.8260498,52.03939941)
\curveto(575.91604596,52.03939151)(575.97104591,52.0493915)(575.9910498,52.06939941)
\curveto(576.69104519,52.07939147)(577.30104458,51.99939155)(577.8210498,51.82939941)
\curveto(578.34104354,51.65939189)(578.70604317,51.33439221)(578.9160498,50.85439941)
\curveto(579.00604287,50.65439289)(579.05604282,50.41939313)(579.0660498,50.14939941)
\curveto(579.08604279,49.88939366)(579.09604278,49.61439393)(579.0960498,49.32439941)
\lineto(579.0960498,46.00939941)
\curveto(579.09604278,45.86939768)(579.10104278,45.73439781)(579.1110498,45.60439941)
\curveto(579.12104276,45.47439807)(579.15104273,45.36939818)(579.2010498,45.28939941)
\curveto(579.25104263,45.21939833)(579.31604256,45.16939838)(579.3960498,45.13939941)
\curveto(579.48604239,45.09939845)(579.57104231,45.06939848)(579.6510498,45.04939941)
\curveto(579.73104215,45.03939851)(579.79104209,44.99439855)(579.8310498,44.91439941)
\curveto(579.85104203,44.88439866)(579.86104202,44.85439869)(579.8610498,44.82439941)
\curveto(579.86104202,44.79439875)(579.86604201,44.75439879)(579.8760498,44.70439941)
\moveto(577.7310498,46.36939941)
\curveto(577.79104409,46.50939704)(577.82104406,46.66939688)(577.8210498,46.84939941)
\curveto(577.83104405,47.03939651)(577.83604404,47.23439631)(577.8360498,47.43439941)
\curveto(577.83604404,47.544396)(577.83104405,47.6443959)(577.8210498,47.73439941)
\curveto(577.81104407,47.82439572)(577.77104411,47.89439565)(577.7010498,47.94439941)
\curveto(577.67104421,47.96439558)(577.60104428,47.97439557)(577.4910498,47.97439941)
\curveto(577.47104441,47.95439559)(577.43604444,47.9443956)(577.3860498,47.94439941)
\curveto(577.33604454,47.9443956)(577.29104459,47.93439561)(577.2510498,47.91439941)
\curveto(577.17104471,47.89439565)(577.0810448,47.87439567)(576.9810498,47.85439941)
\lineto(576.6810498,47.79439941)
\curveto(576.65104523,47.79439575)(576.61604526,47.78939576)(576.5760498,47.77939941)
\lineto(576.4710498,47.77939941)
\curveto(576.32104556,47.73939581)(576.15604572,47.71439583)(575.9760498,47.70439941)
\curveto(575.80604607,47.70439584)(575.64604623,47.68439586)(575.4960498,47.64439941)
\curveto(575.41604646,47.62439592)(575.34104654,47.60439594)(575.2710498,47.58439941)
\curveto(575.21104667,47.57439597)(575.14104674,47.55939599)(575.0610498,47.53939941)
\curveto(574.90104698,47.48939606)(574.75104713,47.42439612)(574.6110498,47.34439941)
\curveto(574.47104741,47.27439627)(574.35104753,47.18439636)(574.2510498,47.07439941)
\curveto(574.15104773,46.96439658)(574.0760478,46.82939672)(574.0260498,46.66939941)
\curveto(573.9760479,46.51939703)(573.95604792,46.33439721)(573.9660498,46.11439941)
\curveto(573.96604791,46.01439753)(573.9810479,45.91939763)(574.0110498,45.82939941)
\curveto(574.05104783,45.7493978)(574.09604778,45.67439787)(574.1460498,45.60439941)
\curveto(574.22604765,45.49439805)(574.33104755,45.39939815)(574.4610498,45.31939941)
\curveto(574.59104729,45.2493983)(574.73104715,45.18939836)(574.8810498,45.13939941)
\curveto(574.93104695,45.12939842)(574.9810469,45.12439842)(575.0310498,45.12439941)
\curveto(575.0810468,45.12439842)(575.13104675,45.11939843)(575.1810498,45.10939941)
\curveto(575.25104663,45.08939846)(575.33604654,45.07439847)(575.4360498,45.06439941)
\curveto(575.54604633,45.06439848)(575.63604624,45.07439847)(575.7060498,45.09439941)
\curveto(575.76604611,45.11439843)(575.82604605,45.11939843)(575.8860498,45.10939941)
\curveto(575.94604593,45.10939844)(576.00604587,45.11939843)(576.0660498,45.13939941)
\curveto(576.14604573,45.15939839)(576.22104566,45.17439837)(576.2910498,45.18439941)
\curveto(576.37104551,45.19439835)(576.44604543,45.21439833)(576.5160498,45.24439941)
\curveto(576.80604507,45.36439818)(577.05104483,45.50939804)(577.2510498,45.67939941)
\curveto(577.46104442,45.8493977)(577.62104426,46.07939747)(577.7310498,46.36939941)
}
}
{
\newrgbcolor{curcolor}{0 0 0}
\pscustom[linestyle=none,fillstyle=solid,fillcolor=curcolor]
{
\newpath
\moveto(584.69269043,52.05439941)
\curveto(584.92268564,52.05439149)(585.05268551,51.99439155)(585.08269043,51.87439941)
\curveto(585.11268545,51.76439178)(585.12768543,51.59939195)(585.12769043,51.37939941)
\lineto(585.12769043,51.09439941)
\curveto(585.12768543,51.00439254)(585.10268546,50.92939262)(585.05269043,50.86939941)
\curveto(584.99268557,50.78939276)(584.90768565,50.7443928)(584.79769043,50.73439941)
\curveto(584.68768587,50.73439281)(584.57768598,50.71939283)(584.46769043,50.68939941)
\curveto(584.32768623,50.65939289)(584.19268637,50.62939292)(584.06269043,50.59939941)
\curveto(583.94268662,50.56939298)(583.82768673,50.52939302)(583.71769043,50.47939941)
\curveto(583.42768713,50.3493932)(583.19268737,50.16939338)(583.01269043,49.93939941)
\curveto(582.83268773,49.71939383)(582.67768788,49.46439408)(582.54769043,49.17439941)
\curveto(582.50768805,49.06439448)(582.47768808,48.9493946)(582.45769043,48.82939941)
\curveto(582.43768812,48.71939483)(582.41268815,48.60439494)(582.38269043,48.48439941)
\curveto(582.37268819,48.43439511)(582.36768819,48.38439516)(582.36769043,48.33439941)
\curveto(582.37768818,48.28439526)(582.37768818,48.23439531)(582.36769043,48.18439941)
\curveto(582.33768822,48.06439548)(582.32268824,47.92439562)(582.32269043,47.76439941)
\curveto(582.33268823,47.61439593)(582.33768822,47.46939608)(582.33769043,47.32939941)
\lineto(582.33769043,45.48439941)
\lineto(582.33769043,45.13939941)
\curveto(582.33768822,45.01939853)(582.33268823,44.90439864)(582.32269043,44.79439941)
\curveto(582.31268825,44.68439886)(582.30768825,44.58939896)(582.30769043,44.50939941)
\curveto(582.31768824,44.42939912)(582.29768826,44.35939919)(582.24769043,44.29939941)
\curveto(582.19768836,44.22939932)(582.11768844,44.18939936)(582.00769043,44.17939941)
\curveto(581.90768865,44.16939938)(581.79768876,44.16439938)(581.67769043,44.16439941)
\lineto(581.40769043,44.16439941)
\curveto(581.3576892,44.18439936)(581.30768925,44.19939935)(581.25769043,44.20939941)
\curveto(581.21768934,44.22939932)(581.18768937,44.25439929)(581.16769043,44.28439941)
\curveto(581.11768944,44.35439919)(581.08768947,44.43939911)(581.07769043,44.53939941)
\lineto(581.07769043,44.86939941)
\lineto(581.07769043,46.02439941)
\lineto(581.07769043,50.17939941)
\lineto(581.07769043,51.21439941)
\lineto(581.07769043,51.51439941)
\curveto(581.08768947,51.61439193)(581.11768944,51.69939185)(581.16769043,51.76939941)
\curveto(581.19768936,51.80939174)(581.24768931,51.83939171)(581.31769043,51.85939941)
\curveto(581.39768916,51.87939167)(581.48268908,51.88939166)(581.57269043,51.88939941)
\curveto(581.6626889,51.89939165)(581.75268881,51.89939165)(581.84269043,51.88939941)
\curveto(581.93268863,51.87939167)(582.00268856,51.86439168)(582.05269043,51.84439941)
\curveto(582.13268843,51.81439173)(582.18268838,51.75439179)(582.20269043,51.66439941)
\curveto(582.23268833,51.58439196)(582.24768831,51.49439205)(582.24769043,51.39439941)
\lineto(582.24769043,51.09439941)
\curveto(582.24768831,50.99439255)(582.26768829,50.90439264)(582.30769043,50.82439941)
\curveto(582.31768824,50.80439274)(582.32768823,50.78939276)(582.33769043,50.77939941)
\lineto(582.38269043,50.73439941)
\curveto(582.49268807,50.73439281)(582.58268798,50.77939277)(582.65269043,50.86939941)
\curveto(582.72268784,50.96939258)(582.78268778,51.0493925)(582.83269043,51.10939941)
\lineto(582.92269043,51.19939941)
\curveto(583.01268755,51.30939224)(583.13768742,51.42439212)(583.29769043,51.54439941)
\curveto(583.4576871,51.66439188)(583.60768695,51.75439179)(583.74769043,51.81439941)
\curveto(583.83768672,51.86439168)(583.93268663,51.89939165)(584.03269043,51.91939941)
\curveto(584.13268643,51.9493916)(584.23768632,51.97939157)(584.34769043,52.00939941)
\curveto(584.40768615,52.01939153)(584.46768609,52.02439152)(584.52769043,52.02439941)
\curveto(584.58768597,52.03439151)(584.64268592,52.0443915)(584.69269043,52.05439941)
}
}
{
\newrgbcolor{curcolor}{0 0 0}
\pscustom[linestyle=none,fillstyle=solid,fillcolor=curcolor]
{
\newpath
\moveto(589.70245605,52.05439941)
\curveto(589.93245126,52.05439149)(590.06245113,51.99439155)(590.09245605,51.87439941)
\curveto(590.12245107,51.76439178)(590.13745106,51.59939195)(590.13745605,51.37939941)
\lineto(590.13745605,51.09439941)
\curveto(590.13745106,51.00439254)(590.11245108,50.92939262)(590.06245605,50.86939941)
\curveto(590.00245119,50.78939276)(589.91745128,50.7443928)(589.80745605,50.73439941)
\curveto(589.6974515,50.73439281)(589.58745161,50.71939283)(589.47745605,50.68939941)
\curveto(589.33745186,50.65939289)(589.20245199,50.62939292)(589.07245605,50.59939941)
\curveto(588.95245224,50.56939298)(588.83745236,50.52939302)(588.72745605,50.47939941)
\curveto(588.43745276,50.3493932)(588.20245299,50.16939338)(588.02245605,49.93939941)
\curveto(587.84245335,49.71939383)(587.68745351,49.46439408)(587.55745605,49.17439941)
\curveto(587.51745368,49.06439448)(587.48745371,48.9493946)(587.46745605,48.82939941)
\curveto(587.44745375,48.71939483)(587.42245377,48.60439494)(587.39245605,48.48439941)
\curveto(587.38245381,48.43439511)(587.37745382,48.38439516)(587.37745605,48.33439941)
\curveto(587.38745381,48.28439526)(587.38745381,48.23439531)(587.37745605,48.18439941)
\curveto(587.34745385,48.06439548)(587.33245386,47.92439562)(587.33245605,47.76439941)
\curveto(587.34245385,47.61439593)(587.34745385,47.46939608)(587.34745605,47.32939941)
\lineto(587.34745605,45.48439941)
\lineto(587.34745605,45.13939941)
\curveto(587.34745385,45.01939853)(587.34245385,44.90439864)(587.33245605,44.79439941)
\curveto(587.32245387,44.68439886)(587.31745388,44.58939896)(587.31745605,44.50939941)
\curveto(587.32745387,44.42939912)(587.30745389,44.35939919)(587.25745605,44.29939941)
\curveto(587.20745399,44.22939932)(587.12745407,44.18939936)(587.01745605,44.17939941)
\curveto(586.91745428,44.16939938)(586.80745439,44.16439938)(586.68745605,44.16439941)
\lineto(586.41745605,44.16439941)
\curveto(586.36745483,44.18439936)(586.31745488,44.19939935)(586.26745605,44.20939941)
\curveto(586.22745497,44.22939932)(586.197455,44.25439929)(586.17745605,44.28439941)
\curveto(586.12745507,44.35439919)(586.0974551,44.43939911)(586.08745605,44.53939941)
\lineto(586.08745605,44.86939941)
\lineto(586.08745605,46.02439941)
\lineto(586.08745605,50.17939941)
\lineto(586.08745605,51.21439941)
\lineto(586.08745605,51.51439941)
\curveto(586.0974551,51.61439193)(586.12745507,51.69939185)(586.17745605,51.76939941)
\curveto(586.20745499,51.80939174)(586.25745494,51.83939171)(586.32745605,51.85939941)
\curveto(586.40745479,51.87939167)(586.4924547,51.88939166)(586.58245605,51.88939941)
\curveto(586.67245452,51.89939165)(586.76245443,51.89939165)(586.85245605,51.88939941)
\curveto(586.94245425,51.87939167)(587.01245418,51.86439168)(587.06245605,51.84439941)
\curveto(587.14245405,51.81439173)(587.192454,51.75439179)(587.21245605,51.66439941)
\curveto(587.24245395,51.58439196)(587.25745394,51.49439205)(587.25745605,51.39439941)
\lineto(587.25745605,51.09439941)
\curveto(587.25745394,50.99439255)(587.27745392,50.90439264)(587.31745605,50.82439941)
\curveto(587.32745387,50.80439274)(587.33745386,50.78939276)(587.34745605,50.77939941)
\lineto(587.39245605,50.73439941)
\curveto(587.50245369,50.73439281)(587.5924536,50.77939277)(587.66245605,50.86939941)
\curveto(587.73245346,50.96939258)(587.7924534,51.0493925)(587.84245605,51.10939941)
\lineto(587.93245605,51.19939941)
\curveto(588.02245317,51.30939224)(588.14745305,51.42439212)(588.30745605,51.54439941)
\curveto(588.46745273,51.66439188)(588.61745258,51.75439179)(588.75745605,51.81439941)
\curveto(588.84745235,51.86439168)(588.94245225,51.89939165)(589.04245605,51.91939941)
\curveto(589.14245205,51.9493916)(589.24745195,51.97939157)(589.35745605,52.00939941)
\curveto(589.41745178,52.01939153)(589.47745172,52.02439152)(589.53745605,52.02439941)
\curveto(589.5974516,52.03439151)(589.65245154,52.0443915)(589.70245605,52.05439941)
}
}
{
\newrgbcolor{curcolor}{0 0 0}
\pscustom[linestyle=none,fillstyle=solid,fillcolor=curcolor]
{
\newpath
\moveto(598.19222168,48.34939941)
\curveto(598.21221362,48.28939526)(598.22221361,48.19439535)(598.22222168,48.06439941)
\curveto(598.22221361,47.9443956)(598.21721361,47.85939569)(598.20722168,47.80939941)
\lineto(598.20722168,47.65939941)
\curveto(598.19721363,47.57939597)(598.18721364,47.50439604)(598.17722168,47.43439941)
\curveto(598.17721365,47.37439617)(598.17221366,47.30439624)(598.16222168,47.22439941)
\curveto(598.14221369,47.16439638)(598.1272137,47.10439644)(598.11722168,47.04439941)
\curveto(598.11721371,46.98439656)(598.10721372,46.92439662)(598.08722168,46.86439941)
\curveto(598.04721378,46.73439681)(598.01221382,46.60439694)(597.98222168,46.47439941)
\curveto(597.95221388,46.3443972)(597.91221392,46.22439732)(597.86222168,46.11439941)
\curveto(597.65221418,45.63439791)(597.37221446,45.22939832)(597.02222168,44.89939941)
\curveto(596.67221516,44.57939897)(596.24221559,44.33439921)(595.73222168,44.16439941)
\curveto(595.62221621,44.12439942)(595.50221633,44.09439945)(595.37222168,44.07439941)
\curveto(595.25221658,44.05439949)(595.1272167,44.03439951)(594.99722168,44.01439941)
\curveto(594.93721689,44.00439954)(594.87221696,43.99939955)(594.80222168,43.99939941)
\curveto(594.74221709,43.98939956)(594.68221715,43.98439956)(594.62222168,43.98439941)
\curveto(594.58221725,43.97439957)(594.52221731,43.96939958)(594.44222168,43.96939941)
\curveto(594.37221746,43.96939958)(594.32221751,43.97439957)(594.29222168,43.98439941)
\curveto(594.25221758,43.99439955)(594.21221762,43.99939955)(594.17222168,43.99939941)
\curveto(594.1322177,43.98939956)(594.09721773,43.98939956)(594.06722168,43.99939941)
\lineto(593.97722168,43.99939941)
\lineto(593.61722168,44.04439941)
\curveto(593.47721835,44.08439946)(593.34221849,44.12439942)(593.21222168,44.16439941)
\curveto(593.08221875,44.20439934)(592.95721887,44.2493993)(592.83722168,44.29939941)
\curveto(592.38721944,44.49939905)(592.01721981,44.75939879)(591.72722168,45.07939941)
\curveto(591.43722039,45.39939815)(591.19722063,45.78939776)(591.00722168,46.24939941)
\curveto(590.95722087,46.3493972)(590.91722091,46.4493971)(590.88722168,46.54939941)
\curveto(590.86722096,46.6493969)(590.84722098,46.75439679)(590.82722168,46.86439941)
\curveto(590.80722102,46.90439664)(590.79722103,46.93439661)(590.79722168,46.95439941)
\curveto(590.80722102,46.98439656)(590.80722102,47.01939653)(590.79722168,47.05939941)
\curveto(590.77722105,47.13939641)(590.76222107,47.21939633)(590.75222168,47.29939941)
\curveto(590.75222108,47.38939616)(590.74222109,47.47439607)(590.72222168,47.55439941)
\lineto(590.72222168,47.67439941)
\curveto(590.72222111,47.71439583)(590.71722111,47.75939579)(590.70722168,47.80939941)
\curveto(590.69722113,47.85939569)(590.69222114,47.9443956)(590.69222168,48.06439941)
\curveto(590.69222114,48.19439535)(590.70222113,48.28939526)(590.72222168,48.34939941)
\curveto(590.74222109,48.41939513)(590.74722108,48.48939506)(590.73722168,48.55939941)
\curveto(590.7272211,48.62939492)(590.7322211,48.69939485)(590.75222168,48.76939941)
\curveto(590.76222107,48.81939473)(590.76722106,48.85939469)(590.76722168,48.88939941)
\curveto(590.77722105,48.92939462)(590.78722104,48.97439457)(590.79722168,49.02439941)
\curveto(590.827221,49.1443944)(590.85222098,49.26439428)(590.87222168,49.38439941)
\curveto(590.90222093,49.50439404)(590.94222089,49.61939393)(590.99222168,49.72939941)
\curveto(591.14222069,50.09939345)(591.32222051,50.42939312)(591.53222168,50.71939941)
\curveto(591.75222008,51.01939253)(592.01721981,51.26939228)(592.32722168,51.46939941)
\curveto(592.44721938,51.549392)(592.57221926,51.61439193)(592.70222168,51.66439941)
\curveto(592.832219,51.72439182)(592.96721886,51.78439176)(593.10722168,51.84439941)
\curveto(593.2272186,51.89439165)(593.35721847,51.92439162)(593.49722168,51.93439941)
\curveto(593.63721819,51.95439159)(593.77721805,51.98439156)(593.91722168,52.02439941)
\lineto(594.11222168,52.02439941)
\curveto(594.18221765,52.03439151)(594.24721758,52.0443915)(594.30722168,52.05439941)
\curveto(595.19721663,52.06439148)(595.93721589,51.87939167)(596.52722168,51.49939941)
\curveto(597.11721471,51.11939243)(597.54221429,50.62439292)(597.80222168,50.01439941)
\curveto(597.85221398,49.91439363)(597.89221394,49.81439373)(597.92222168,49.71439941)
\curveto(597.95221388,49.61439393)(597.98721384,49.50939404)(598.02722168,49.39939941)
\curveto(598.05721377,49.28939426)(598.08221375,49.16939438)(598.10222168,49.03939941)
\curveto(598.12221371,48.91939463)(598.14721368,48.79439475)(598.17722168,48.66439941)
\curveto(598.18721364,48.61439493)(598.18721364,48.55939499)(598.17722168,48.49939941)
\curveto(598.17721365,48.4493951)(598.18221365,48.39939515)(598.19222168,48.34939941)
\moveto(596.85722168,47.49439941)
\curveto(596.87721495,47.56439598)(596.88221495,47.6443959)(596.87222168,47.73439941)
\lineto(596.87222168,47.98939941)
\curveto(596.87221496,48.37939517)(596.83721499,48.70939484)(596.76722168,48.97939941)
\curveto(596.73721509,49.05939449)(596.71221512,49.13939441)(596.69222168,49.21939941)
\curveto(596.67221516,49.29939425)(596.64721518,49.37439417)(596.61722168,49.44439941)
\curveto(596.33721549,50.09439345)(595.89221594,50.544393)(595.28222168,50.79439941)
\curveto(595.21221662,50.82439272)(595.13721669,50.8443927)(595.05722168,50.85439941)
\lineto(594.81722168,50.91439941)
\curveto(594.73721709,50.93439261)(594.65221718,50.9443926)(594.56222168,50.94439941)
\lineto(594.29222168,50.94439941)
\lineto(594.02222168,50.89939941)
\curveto(593.92221791,50.87939267)(593.827218,50.85439269)(593.73722168,50.82439941)
\curveto(593.65721817,50.80439274)(593.57721825,50.77439277)(593.49722168,50.73439941)
\curveto(593.4272184,50.71439283)(593.36221847,50.68439286)(593.30222168,50.64439941)
\curveto(593.24221859,50.60439294)(593.18721864,50.56439298)(593.13722168,50.52439941)
\curveto(592.89721893,50.35439319)(592.70221913,50.1493934)(592.55222168,49.90939941)
\curveto(592.40221943,49.66939388)(592.27221956,49.38939416)(592.16222168,49.06939941)
\curveto(592.1322197,48.96939458)(592.11221972,48.86439468)(592.10222168,48.75439941)
\curveto(592.09221974,48.65439489)(592.07721975,48.549395)(592.05722168,48.43939941)
\curveto(592.04721978,48.39939515)(592.04221979,48.33439521)(592.04222168,48.24439941)
\curveto(592.0322198,48.21439533)(592.0272198,48.17939537)(592.02722168,48.13939941)
\curveto(592.03721979,48.09939545)(592.04221979,48.05439549)(592.04222168,48.00439941)
\lineto(592.04222168,47.70439941)
\curveto(592.04221979,47.60439594)(592.05221978,47.51439603)(592.07222168,47.43439941)
\lineto(592.10222168,47.25439941)
\curveto(592.12221971,47.15439639)(592.13721969,47.05439649)(592.14722168,46.95439941)
\curveto(592.16721966,46.86439668)(592.19721963,46.77939677)(592.23722168,46.69939941)
\curveto(592.33721949,46.45939709)(592.45221938,46.23439731)(592.58222168,46.02439941)
\curveto(592.72221911,45.81439773)(592.89221894,45.63939791)(593.09222168,45.49939941)
\curveto(593.14221869,45.46939808)(593.18721864,45.4443981)(593.22722168,45.42439941)
\curveto(593.26721856,45.40439814)(593.31221852,45.37939817)(593.36222168,45.34939941)
\curveto(593.44221839,45.29939825)(593.5272183,45.25439829)(593.61722168,45.21439941)
\curveto(593.71721811,45.18439836)(593.82221801,45.15439839)(593.93222168,45.12439941)
\curveto(593.98221785,45.10439844)(594.0272178,45.09439845)(594.06722168,45.09439941)
\curveto(594.11721771,45.10439844)(594.16721766,45.10439844)(594.21722168,45.09439941)
\curveto(594.24721758,45.08439846)(594.30721752,45.07439847)(594.39722168,45.06439941)
\curveto(594.49721733,45.05439849)(594.57221726,45.05939849)(594.62222168,45.07939941)
\curveto(594.66221717,45.08939846)(594.70221713,45.08939846)(594.74222168,45.07939941)
\curveto(594.78221705,45.07939847)(594.82221701,45.08939846)(594.86222168,45.10939941)
\curveto(594.94221689,45.12939842)(595.02221681,45.1443984)(595.10222168,45.15439941)
\curveto(595.18221665,45.17439837)(595.25721657,45.19939835)(595.32722168,45.22939941)
\curveto(595.66721616,45.36939818)(595.94221589,45.56439798)(596.15222168,45.81439941)
\curveto(596.36221547,46.06439748)(596.53721529,46.35939719)(596.67722168,46.69939941)
\curveto(596.7272151,46.81939673)(596.75721507,46.9443966)(596.76722168,47.07439941)
\curveto(596.78721504,47.21439633)(596.81721501,47.35439619)(596.85722168,47.49439941)
}
}
{
\newrgbcolor{curcolor}{0 0 0}
\pscustom[linestyle=none,fillstyle=solid,fillcolor=curcolor]
{
\newpath
\moveto(600.25050293,54.82939941)
\curveto(600.38050131,54.82938872)(600.51550118,54.82938872)(600.65550293,54.82939941)
\curveto(600.80550089,54.82938872)(600.91550078,54.79438875)(600.98550293,54.72439941)
\curveto(601.03550066,54.65438889)(601.06050063,54.55938899)(601.06050293,54.43939941)
\curveto(601.07050062,54.32938922)(601.07550062,54.21438933)(601.07550293,54.09439941)
\lineto(601.07550293,52.75939941)
\lineto(601.07550293,46.68439941)
\lineto(601.07550293,45.00439941)
\lineto(601.07550293,44.61439941)
\curveto(601.07550062,44.47439907)(601.05050064,44.36439918)(601.00050293,44.28439941)
\curveto(600.97050072,44.23439931)(600.92550077,44.20439934)(600.86550293,44.19439941)
\curveto(600.81550088,44.18439936)(600.75050094,44.16939938)(600.67050293,44.14939941)
\lineto(600.46050293,44.14939941)
\lineto(600.14550293,44.14939941)
\curveto(600.04550165,44.15939939)(599.97050172,44.19439935)(599.92050293,44.25439941)
\curveto(599.87050182,44.33439921)(599.84050185,44.43439911)(599.83050293,44.55439941)
\lineto(599.83050293,44.92939941)
\lineto(599.83050293,46.30939941)
\lineto(599.83050293,52.54939941)
\lineto(599.83050293,54.01939941)
\curveto(599.83050186,54.12938942)(599.82550187,54.2443893)(599.81550293,54.36439941)
\curveto(599.81550188,54.49438905)(599.84050185,54.59438895)(599.89050293,54.66439941)
\curveto(599.93050176,54.72438882)(600.00550169,54.77438877)(600.11550293,54.81439941)
\curveto(600.13550156,54.82438872)(600.15550154,54.82438872)(600.17550293,54.81439941)
\curveto(600.20550149,54.81438873)(600.23050146,54.81938873)(600.25050293,54.82939941)
}
}
{
\newrgbcolor{curcolor}{0 0 0}
\pscustom[linestyle=none,fillstyle=solid,fillcolor=curcolor]
{
\newpath
\moveto(603.59034668,54.82939941)
\curveto(603.72034506,54.82938872)(603.85534493,54.82938872)(603.99534668,54.82939941)
\curveto(604.14534464,54.82938872)(604.25534453,54.79438875)(604.32534668,54.72439941)
\curveto(604.37534441,54.65438889)(604.40034438,54.55938899)(604.40034668,54.43939941)
\curveto(604.41034437,54.32938922)(604.41534437,54.21438933)(604.41534668,54.09439941)
\lineto(604.41534668,52.75939941)
\lineto(604.41534668,46.68439941)
\lineto(604.41534668,45.00439941)
\lineto(604.41534668,44.61439941)
\curveto(604.41534437,44.47439907)(604.39034439,44.36439918)(604.34034668,44.28439941)
\curveto(604.31034447,44.23439931)(604.26534452,44.20439934)(604.20534668,44.19439941)
\curveto(604.15534463,44.18439936)(604.09034469,44.16939938)(604.01034668,44.14939941)
\lineto(603.80034668,44.14939941)
\lineto(603.48534668,44.14939941)
\curveto(603.3853454,44.15939939)(603.31034547,44.19439935)(603.26034668,44.25439941)
\curveto(603.21034557,44.33439921)(603.1803456,44.43439911)(603.17034668,44.55439941)
\lineto(603.17034668,44.92939941)
\lineto(603.17034668,46.30939941)
\lineto(603.17034668,52.54939941)
\lineto(603.17034668,54.01939941)
\curveto(603.17034561,54.12938942)(603.16534562,54.2443893)(603.15534668,54.36439941)
\curveto(603.15534563,54.49438905)(603.1803456,54.59438895)(603.23034668,54.66439941)
\curveto(603.27034551,54.72438882)(603.34534544,54.77438877)(603.45534668,54.81439941)
\curveto(603.47534531,54.82438872)(603.49534529,54.82438872)(603.51534668,54.81439941)
\curveto(603.54534524,54.81438873)(603.57034521,54.81938873)(603.59034668,54.82939941)
}
}
{
\newrgbcolor{curcolor}{0 0 0}
\pscustom[linestyle=none,fillstyle=solid,fillcolor=curcolor]
{
\newpath
\moveto(613.24519043,44.70439941)
\curveto(613.2751826,44.544399)(613.26018261,44.40939914)(613.20019043,44.29939941)
\curveto(613.14018273,44.19939935)(613.06018281,44.12439942)(612.96019043,44.07439941)
\curveto(612.91018296,44.05439949)(612.85518302,44.0443995)(612.79519043,44.04439941)
\curveto(612.74518313,44.0443995)(612.69018318,44.03439951)(612.63019043,44.01439941)
\curveto(612.41018346,43.96439958)(612.19018368,43.97939957)(611.97019043,44.05939941)
\curveto(611.76018411,44.12939942)(611.61518426,44.21939933)(611.53519043,44.32939941)
\curveto(611.48518439,44.39939915)(611.44018443,44.47939907)(611.40019043,44.56939941)
\curveto(611.36018451,44.66939888)(611.31018456,44.7493988)(611.25019043,44.80939941)
\curveto(611.23018464,44.82939872)(611.20518467,44.8493987)(611.17519043,44.86939941)
\curveto(611.15518472,44.88939866)(611.12518475,44.89439865)(611.08519043,44.88439941)
\curveto(610.9751849,44.85439869)(610.870185,44.79939875)(610.77019043,44.71939941)
\curveto(610.68018519,44.63939891)(610.59018528,44.56939898)(610.50019043,44.50939941)
\curveto(610.3701855,44.42939912)(610.23018564,44.35439919)(610.08019043,44.28439941)
\curveto(609.93018594,44.22439932)(609.7701861,44.16939938)(609.60019043,44.11939941)
\curveto(609.50018637,44.08939946)(609.39018648,44.06939948)(609.27019043,44.05939941)
\curveto(609.16018671,44.0493995)(609.05018682,44.03439951)(608.94019043,44.01439941)
\curveto(608.89018698,44.00439954)(608.84518703,43.99939955)(608.80519043,43.99939941)
\lineto(608.70019043,43.99939941)
\curveto(608.59018728,43.97939957)(608.48518739,43.97939957)(608.38519043,43.99939941)
\lineto(608.25019043,43.99939941)
\curveto(608.20018767,44.00939954)(608.15018772,44.01439953)(608.10019043,44.01439941)
\curveto(608.05018782,44.01439953)(608.00518787,44.02439952)(607.96519043,44.04439941)
\curveto(607.92518795,44.05439949)(607.89018798,44.05939949)(607.86019043,44.05939941)
\curveto(607.84018803,44.0493995)(607.81518806,44.0493995)(607.78519043,44.05939941)
\lineto(607.54519043,44.11939941)
\curveto(607.46518841,44.12939942)(607.39018848,44.1493994)(607.32019043,44.17939941)
\curveto(607.02018885,44.30939924)(606.7751891,44.45439909)(606.58519043,44.61439941)
\curveto(606.40518947,44.78439876)(606.25518962,45.01939853)(606.13519043,45.31939941)
\curveto(606.04518983,45.53939801)(606.00018987,45.80439774)(606.00019043,46.11439941)
\lineto(606.00019043,46.42939941)
\curveto(606.01018986,46.47939707)(606.01518986,46.52939702)(606.01519043,46.57939941)
\lineto(606.04519043,46.75939941)
\lineto(606.16519043,47.08939941)
\curveto(606.20518967,47.19939635)(606.25518962,47.29939625)(606.31519043,47.38939941)
\curveto(606.49518938,47.67939587)(606.74018913,47.89439565)(607.05019043,48.03439941)
\curveto(607.36018851,48.17439537)(607.70018817,48.29939525)(608.07019043,48.40939941)
\curveto(608.21018766,48.4493951)(608.35518752,48.47939507)(608.50519043,48.49939941)
\curveto(608.65518722,48.51939503)(608.80518707,48.544395)(608.95519043,48.57439941)
\curveto(609.02518685,48.59439495)(609.09018678,48.60439494)(609.15019043,48.60439941)
\curveto(609.22018665,48.60439494)(609.29518658,48.61439493)(609.37519043,48.63439941)
\curveto(609.44518643,48.65439489)(609.51518636,48.66439488)(609.58519043,48.66439941)
\curveto(609.65518622,48.67439487)(609.73018614,48.68939486)(609.81019043,48.70939941)
\curveto(610.06018581,48.76939478)(610.29518558,48.81939473)(610.51519043,48.85939941)
\curveto(610.73518514,48.90939464)(610.91018496,49.02439452)(611.04019043,49.20439941)
\curveto(611.10018477,49.28439426)(611.15018472,49.38439416)(611.19019043,49.50439941)
\curveto(611.23018464,49.63439391)(611.23018464,49.77439377)(611.19019043,49.92439941)
\curveto(611.13018474,50.16439338)(611.04018483,50.35439319)(610.92019043,50.49439941)
\curveto(610.81018506,50.63439291)(610.65018522,50.7443928)(610.44019043,50.82439941)
\curveto(610.32018555,50.87439267)(610.1751857,50.90939264)(610.00519043,50.92939941)
\curveto(609.84518603,50.9493926)(609.6751862,50.95939259)(609.49519043,50.95939941)
\curveto(609.31518656,50.95939259)(609.14018673,50.9493926)(608.97019043,50.92939941)
\curveto(608.80018707,50.90939264)(608.65518722,50.87939267)(608.53519043,50.83939941)
\curveto(608.36518751,50.77939277)(608.20018767,50.69439285)(608.04019043,50.58439941)
\curveto(607.96018791,50.52439302)(607.88518799,50.4443931)(607.81519043,50.34439941)
\curveto(607.75518812,50.25439329)(607.70018817,50.15439339)(607.65019043,50.04439941)
\curveto(607.62018825,49.96439358)(607.59018828,49.87939367)(607.56019043,49.78939941)
\curveto(607.54018833,49.69939385)(607.49518838,49.62939392)(607.42519043,49.57939941)
\curveto(607.38518849,49.549394)(607.31518856,49.52439402)(607.21519043,49.50439941)
\curveto(607.12518875,49.49439405)(607.03018884,49.48939406)(606.93019043,49.48939941)
\curveto(606.83018904,49.48939406)(606.73018914,49.49439405)(606.63019043,49.50439941)
\curveto(606.54018933,49.52439402)(606.4751894,49.549394)(606.43519043,49.57939941)
\curveto(606.39518948,49.60939394)(606.36518951,49.65939389)(606.34519043,49.72939941)
\curveto(606.32518955,49.79939375)(606.32518955,49.87439367)(606.34519043,49.95439941)
\curveto(606.3751895,50.08439346)(606.40518947,50.20439334)(606.43519043,50.31439941)
\curveto(606.4751894,50.43439311)(606.52018935,50.549393)(606.57019043,50.65939941)
\curveto(606.76018911,51.00939254)(607.00018887,51.27939227)(607.29019043,51.46939941)
\curveto(607.58018829,51.66939188)(607.94018793,51.82939172)(608.37019043,51.94939941)
\curveto(608.4701874,51.96939158)(608.5701873,51.98439156)(608.67019043,51.99439941)
\curveto(608.78018709,52.00439154)(608.89018698,52.01939153)(609.00019043,52.03939941)
\curveto(609.04018683,52.0493915)(609.10518677,52.0493915)(609.19519043,52.03939941)
\curveto(609.28518659,52.03939151)(609.34018653,52.0493915)(609.36019043,52.06939941)
\curveto(610.06018581,52.07939147)(610.6701852,51.99939155)(611.19019043,51.82939941)
\curveto(611.71018416,51.65939189)(612.0751838,51.33439221)(612.28519043,50.85439941)
\curveto(612.3751835,50.65439289)(612.42518345,50.41939313)(612.43519043,50.14939941)
\curveto(612.45518342,49.88939366)(612.46518341,49.61439393)(612.46519043,49.32439941)
\lineto(612.46519043,46.00939941)
\curveto(612.46518341,45.86939768)(612.4701834,45.73439781)(612.48019043,45.60439941)
\curveto(612.49018338,45.47439807)(612.52018335,45.36939818)(612.57019043,45.28939941)
\curveto(612.62018325,45.21939833)(612.68518319,45.16939838)(612.76519043,45.13939941)
\curveto(612.85518302,45.09939845)(612.94018293,45.06939848)(613.02019043,45.04939941)
\curveto(613.10018277,45.03939851)(613.16018271,44.99439855)(613.20019043,44.91439941)
\curveto(613.22018265,44.88439866)(613.23018264,44.85439869)(613.23019043,44.82439941)
\curveto(613.23018264,44.79439875)(613.23518264,44.75439879)(613.24519043,44.70439941)
\moveto(611.10019043,46.36939941)
\curveto(611.16018471,46.50939704)(611.19018468,46.66939688)(611.19019043,46.84939941)
\curveto(611.20018467,47.03939651)(611.20518467,47.23439631)(611.20519043,47.43439941)
\curveto(611.20518467,47.544396)(611.20018467,47.6443959)(611.19019043,47.73439941)
\curveto(611.18018469,47.82439572)(611.14018473,47.89439565)(611.07019043,47.94439941)
\curveto(611.04018483,47.96439558)(610.9701849,47.97439557)(610.86019043,47.97439941)
\curveto(610.84018503,47.95439559)(610.80518507,47.9443956)(610.75519043,47.94439941)
\curveto(610.70518517,47.9443956)(610.66018521,47.93439561)(610.62019043,47.91439941)
\curveto(610.54018533,47.89439565)(610.45018542,47.87439567)(610.35019043,47.85439941)
\lineto(610.05019043,47.79439941)
\curveto(610.02018585,47.79439575)(609.98518589,47.78939576)(609.94519043,47.77939941)
\lineto(609.84019043,47.77939941)
\curveto(609.69018618,47.73939581)(609.52518635,47.71439583)(609.34519043,47.70439941)
\curveto(609.1751867,47.70439584)(609.01518686,47.68439586)(608.86519043,47.64439941)
\curveto(608.78518709,47.62439592)(608.71018716,47.60439594)(608.64019043,47.58439941)
\curveto(608.58018729,47.57439597)(608.51018736,47.55939599)(608.43019043,47.53939941)
\curveto(608.2701876,47.48939606)(608.12018775,47.42439612)(607.98019043,47.34439941)
\curveto(607.84018803,47.27439627)(607.72018815,47.18439636)(607.62019043,47.07439941)
\curveto(607.52018835,46.96439658)(607.44518843,46.82939672)(607.39519043,46.66939941)
\curveto(607.34518853,46.51939703)(607.32518855,46.33439721)(607.33519043,46.11439941)
\curveto(607.33518854,46.01439753)(607.35018852,45.91939763)(607.38019043,45.82939941)
\curveto(607.42018845,45.7493978)(607.46518841,45.67439787)(607.51519043,45.60439941)
\curveto(607.59518828,45.49439805)(607.70018817,45.39939815)(607.83019043,45.31939941)
\curveto(607.96018791,45.2493983)(608.10018777,45.18939836)(608.25019043,45.13939941)
\curveto(608.30018757,45.12939842)(608.35018752,45.12439842)(608.40019043,45.12439941)
\curveto(608.45018742,45.12439842)(608.50018737,45.11939843)(608.55019043,45.10939941)
\curveto(608.62018725,45.08939846)(608.70518717,45.07439847)(608.80519043,45.06439941)
\curveto(608.91518696,45.06439848)(609.00518687,45.07439847)(609.07519043,45.09439941)
\curveto(609.13518674,45.11439843)(609.19518668,45.11939843)(609.25519043,45.10939941)
\curveto(609.31518656,45.10939844)(609.3751865,45.11939843)(609.43519043,45.13939941)
\curveto(609.51518636,45.15939839)(609.59018628,45.17439837)(609.66019043,45.18439941)
\curveto(609.74018613,45.19439835)(609.81518606,45.21439833)(609.88519043,45.24439941)
\curveto(610.1751857,45.36439818)(610.42018545,45.50939804)(610.62019043,45.67939941)
\curveto(610.83018504,45.8493977)(610.99018488,46.07939747)(611.10019043,46.36939941)
}
}
{
\newrgbcolor{curcolor}{0 0 0}
\pscustom[linestyle=none,fillstyle=solid,fillcolor=curcolor]
{
\newpath
\moveto(621.37683105,44.95939941)
\lineto(621.37683105,44.56939941)
\curveto(621.37682318,44.4493991)(621.3518232,44.3493992)(621.30183105,44.26939941)
\curveto(621.2518233,44.19939935)(621.16682339,44.15939939)(621.04683105,44.14939941)
\lineto(620.70183105,44.14939941)
\curveto(620.64182391,44.1493994)(620.58182397,44.1443994)(620.52183105,44.13439941)
\curveto(620.47182408,44.13439941)(620.42682413,44.1443994)(620.38683105,44.16439941)
\curveto(620.29682426,44.18439936)(620.23682432,44.22439932)(620.20683105,44.28439941)
\curveto(620.16682439,44.33439921)(620.14182441,44.39439915)(620.13183105,44.46439941)
\curveto(620.13182442,44.53439901)(620.11682444,44.60439894)(620.08683105,44.67439941)
\curveto(620.07682448,44.69439885)(620.06182449,44.70939884)(620.04183105,44.71939941)
\curveto(620.03182452,44.73939881)(620.01682454,44.75939879)(619.99683105,44.77939941)
\curveto(619.89682466,44.78939876)(619.81682474,44.76939878)(619.75683105,44.71939941)
\curveto(619.70682485,44.66939888)(619.6518249,44.61939893)(619.59183105,44.56939941)
\curveto(619.39182516,44.41939913)(619.19182536,44.30439924)(618.99183105,44.22439941)
\curveto(618.81182574,44.1443994)(618.60182595,44.08439946)(618.36183105,44.04439941)
\curveto(618.13182642,44.00439954)(617.89182666,43.98439956)(617.64183105,43.98439941)
\curveto(617.40182715,43.97439957)(617.16182739,43.98939956)(616.92183105,44.02939941)
\curveto(616.68182787,44.05939949)(616.47182808,44.11439943)(616.29183105,44.19439941)
\curveto(615.77182878,44.41439913)(615.3518292,44.70939884)(615.03183105,45.07939941)
\curveto(614.71182984,45.45939809)(614.46183009,45.92939762)(614.28183105,46.48939941)
\curveto(614.24183031,46.57939697)(614.21183034,46.66939688)(614.19183105,46.75939941)
\curveto(614.18183037,46.85939669)(614.16183039,46.95939659)(614.13183105,47.05939941)
\curveto(614.12183043,47.10939644)(614.11683044,47.15939639)(614.11683105,47.20939941)
\curveto(614.11683044,47.25939629)(614.11183044,47.30939624)(614.10183105,47.35939941)
\curveto(614.08183047,47.40939614)(614.07183048,47.45939609)(614.07183105,47.50939941)
\curveto(614.08183047,47.56939598)(614.08183047,47.62439592)(614.07183105,47.67439941)
\lineto(614.07183105,47.82439941)
\curveto(614.0518305,47.87439567)(614.04183051,47.93939561)(614.04183105,48.01939941)
\curveto(614.04183051,48.09939545)(614.0518305,48.16439538)(614.07183105,48.21439941)
\lineto(614.07183105,48.37939941)
\curveto(614.09183046,48.4493951)(614.09683046,48.51939503)(614.08683105,48.58939941)
\curveto(614.08683047,48.66939488)(614.09683046,48.7443948)(614.11683105,48.81439941)
\curveto(614.12683043,48.86439468)(614.13183042,48.90939464)(614.13183105,48.94939941)
\curveto(614.13183042,48.98939456)(614.13683042,49.03439451)(614.14683105,49.08439941)
\curveto(614.17683038,49.18439436)(614.20183035,49.27939427)(614.22183105,49.36939941)
\curveto(614.24183031,49.46939408)(614.26683029,49.56439398)(614.29683105,49.65439941)
\curveto(614.42683013,50.03439351)(614.59182996,50.37439317)(614.79183105,50.67439941)
\curveto(615.00182955,50.98439256)(615.2518293,51.23939231)(615.54183105,51.43939941)
\curveto(615.71182884,51.55939199)(615.88682867,51.65939189)(616.06683105,51.73939941)
\curveto(616.2568283,51.81939173)(616.46182809,51.88939166)(616.68183105,51.94939941)
\curveto(616.7518278,51.95939159)(616.81682774,51.96939158)(616.87683105,51.97939941)
\curveto(616.94682761,51.98939156)(617.01682754,52.00439154)(617.08683105,52.02439941)
\lineto(617.23683105,52.02439941)
\curveto(617.31682724,52.0443915)(617.43182712,52.05439149)(617.58183105,52.05439941)
\curveto(617.74182681,52.05439149)(617.86182669,52.0443915)(617.94183105,52.02439941)
\curveto(617.98182657,52.01439153)(618.03682652,52.00939154)(618.10683105,52.00939941)
\curveto(618.21682634,51.97939157)(618.32682623,51.95439159)(618.43683105,51.93439941)
\curveto(618.54682601,51.92439162)(618.6518259,51.89439165)(618.75183105,51.84439941)
\curveto(618.90182565,51.78439176)(619.04182551,51.71939183)(619.17183105,51.64939941)
\curveto(619.31182524,51.57939197)(619.44182511,51.49939205)(619.56183105,51.40939941)
\curveto(619.62182493,51.35939219)(619.68182487,51.30439224)(619.74183105,51.24439941)
\curveto(619.81182474,51.19439235)(619.90182465,51.17939237)(620.01183105,51.19939941)
\curveto(620.03182452,51.22939232)(620.04682451,51.25439229)(620.05683105,51.27439941)
\curveto(620.07682448,51.29439225)(620.09182446,51.32439222)(620.10183105,51.36439941)
\curveto(620.13182442,51.45439209)(620.14182441,51.56939198)(620.13183105,51.70939941)
\lineto(620.13183105,52.08439941)
\lineto(620.13183105,53.80939941)
\lineto(620.13183105,54.27439941)
\curveto(620.13182442,54.45438909)(620.1568244,54.58438896)(620.20683105,54.66439941)
\curveto(620.24682431,54.73438881)(620.30682425,54.77938877)(620.38683105,54.79939941)
\curveto(620.40682415,54.79938875)(620.43182412,54.79938875)(620.46183105,54.79939941)
\curveto(620.49182406,54.80938874)(620.51682404,54.81438873)(620.53683105,54.81439941)
\curveto(620.67682388,54.82438872)(620.82182373,54.82438872)(620.97183105,54.81439941)
\curveto(621.13182342,54.81438873)(621.24182331,54.77438877)(621.30183105,54.69439941)
\curveto(621.3518232,54.61438893)(621.37682318,54.51438903)(621.37683105,54.39439941)
\lineto(621.37683105,54.01939941)
\lineto(621.37683105,44.95939941)
\moveto(620.16183105,47.79439941)
\curveto(620.18182437,47.8443957)(620.19182436,47.90939564)(620.19183105,47.98939941)
\curveto(620.19182436,48.07939547)(620.18182437,48.1493954)(620.16183105,48.19939941)
\lineto(620.16183105,48.42439941)
\curveto(620.14182441,48.51439503)(620.12682443,48.60439494)(620.11683105,48.69439941)
\curveto(620.10682445,48.79439475)(620.08682447,48.88439466)(620.05683105,48.96439941)
\curveto(620.03682452,49.0443945)(620.01682454,49.11939443)(619.99683105,49.18939941)
\curveto(619.98682457,49.25939429)(619.96682459,49.32939422)(619.93683105,49.39939941)
\curveto(619.81682474,49.69939385)(619.66182489,49.96439358)(619.47183105,50.19439941)
\curveto(619.28182527,50.42439312)(619.04182551,50.60439294)(618.75183105,50.73439941)
\curveto(618.6518259,50.78439276)(618.54682601,50.81939273)(618.43683105,50.83939941)
\curveto(618.33682622,50.86939268)(618.22682633,50.89439265)(618.10683105,50.91439941)
\curveto(618.02682653,50.93439261)(617.93682662,50.9443926)(617.83683105,50.94439941)
\lineto(617.56683105,50.94439941)
\curveto(617.51682704,50.93439261)(617.47182708,50.92439262)(617.43183105,50.91439941)
\lineto(617.29683105,50.91439941)
\curveto(617.21682734,50.89439265)(617.13182742,50.87439267)(617.04183105,50.85439941)
\curveto(616.96182759,50.83439271)(616.88182767,50.80939274)(616.80183105,50.77939941)
\curveto(616.48182807,50.63939291)(616.22182833,50.43439311)(616.02183105,50.16439941)
\curveto(615.83182872,49.90439364)(615.67682888,49.59939395)(615.55683105,49.24939941)
\curveto(615.51682904,49.13939441)(615.48682907,49.02439452)(615.46683105,48.90439941)
\curveto(615.4568291,48.79439475)(615.44182911,48.68439486)(615.42183105,48.57439941)
\curveto(615.42182913,48.53439501)(615.41682914,48.49439505)(615.40683105,48.45439941)
\lineto(615.40683105,48.34939941)
\curveto(615.38682917,48.29939525)(615.37682918,48.2443953)(615.37683105,48.18439941)
\curveto(615.38682917,48.12439542)(615.39182916,48.06939548)(615.39183105,48.01939941)
\lineto(615.39183105,47.68939941)
\curveto(615.39182916,47.58939596)(615.40182915,47.49439605)(615.42183105,47.40439941)
\curveto(615.43182912,47.37439617)(615.43682912,47.32439622)(615.43683105,47.25439941)
\curveto(615.4568291,47.18439636)(615.47182908,47.11439643)(615.48183105,47.04439941)
\lineto(615.54183105,46.83439941)
\curveto(615.6518289,46.48439706)(615.80182875,46.18439736)(615.99183105,45.93439941)
\curveto(616.18182837,45.68439786)(616.42182813,45.47939807)(616.71183105,45.31939941)
\curveto(616.80182775,45.26939828)(616.89182766,45.22939832)(616.98183105,45.19939941)
\curveto(617.07182748,45.16939838)(617.17182738,45.13939841)(617.28183105,45.10939941)
\curveto(617.33182722,45.08939846)(617.38182717,45.08439846)(617.43183105,45.09439941)
\curveto(617.49182706,45.10439844)(617.54682701,45.09939845)(617.59683105,45.07939941)
\curveto(617.63682692,45.06939848)(617.67682688,45.06439848)(617.71683105,45.06439941)
\lineto(617.85183105,45.06439941)
\lineto(617.98683105,45.06439941)
\curveto(618.01682654,45.07439847)(618.06682649,45.07939847)(618.13683105,45.07939941)
\curveto(618.21682634,45.09939845)(618.29682626,45.11439843)(618.37683105,45.12439941)
\curveto(618.4568261,45.1443984)(618.53182602,45.16939838)(618.60183105,45.19939941)
\curveto(618.93182562,45.33939821)(619.19682536,45.51439803)(619.39683105,45.72439941)
\curveto(619.60682495,45.9443976)(619.78182477,46.21939733)(619.92183105,46.54939941)
\curveto(619.97182458,46.65939689)(620.00682455,46.76939678)(620.02683105,46.87939941)
\curveto(620.04682451,46.98939656)(620.07182448,47.09939645)(620.10183105,47.20939941)
\curveto(620.12182443,47.2493963)(620.13182442,47.28439626)(620.13183105,47.31439941)
\curveto(620.13182442,47.35439619)(620.13682442,47.39439615)(620.14683105,47.43439941)
\curveto(620.1568244,47.49439605)(620.1568244,47.55439599)(620.14683105,47.61439941)
\curveto(620.14682441,47.67439587)(620.1518244,47.73439581)(620.16183105,47.79439941)
}
}
{
\newrgbcolor{curcolor}{0 0 0}
\pscustom[linestyle=none,fillstyle=solid,fillcolor=curcolor]
{
\newpath
\moveto(630.44808105,48.34939941)
\curveto(630.46807299,48.28939526)(630.47807298,48.19439535)(630.47808105,48.06439941)
\curveto(630.47807298,47.9443956)(630.47307299,47.85939569)(630.46308105,47.80939941)
\lineto(630.46308105,47.65939941)
\curveto(630.45307301,47.57939597)(630.44307302,47.50439604)(630.43308105,47.43439941)
\curveto(630.43307303,47.37439617)(630.42807303,47.30439624)(630.41808105,47.22439941)
\curveto(630.39807306,47.16439638)(630.38307308,47.10439644)(630.37308105,47.04439941)
\curveto(630.37307309,46.98439656)(630.3630731,46.92439662)(630.34308105,46.86439941)
\curveto(630.30307316,46.73439681)(630.26807319,46.60439694)(630.23808105,46.47439941)
\curveto(630.20807325,46.3443972)(630.16807329,46.22439732)(630.11808105,46.11439941)
\curveto(629.90807355,45.63439791)(629.62807383,45.22939832)(629.27808105,44.89939941)
\curveto(628.92807453,44.57939897)(628.49807496,44.33439921)(627.98808105,44.16439941)
\curveto(627.87807558,44.12439942)(627.7580757,44.09439945)(627.62808105,44.07439941)
\curveto(627.50807595,44.05439949)(627.38307608,44.03439951)(627.25308105,44.01439941)
\curveto(627.19307627,44.00439954)(627.12807633,43.99939955)(627.05808105,43.99939941)
\curveto(626.99807646,43.98939956)(626.93807652,43.98439956)(626.87808105,43.98439941)
\curveto(626.83807662,43.97439957)(626.77807668,43.96939958)(626.69808105,43.96939941)
\curveto(626.62807683,43.96939958)(626.57807688,43.97439957)(626.54808105,43.98439941)
\curveto(626.50807695,43.99439955)(626.46807699,43.99939955)(626.42808105,43.99939941)
\curveto(626.38807707,43.98939956)(626.35307711,43.98939956)(626.32308105,43.99939941)
\lineto(626.23308105,43.99939941)
\lineto(625.87308105,44.04439941)
\curveto(625.73307773,44.08439946)(625.59807786,44.12439942)(625.46808105,44.16439941)
\curveto(625.33807812,44.20439934)(625.21307825,44.2493993)(625.09308105,44.29939941)
\curveto(624.64307882,44.49939905)(624.27307919,44.75939879)(623.98308105,45.07939941)
\curveto(623.69307977,45.39939815)(623.45308001,45.78939776)(623.26308105,46.24939941)
\curveto(623.21308025,46.3493972)(623.17308029,46.4493971)(623.14308105,46.54939941)
\curveto(623.12308034,46.6493969)(623.10308036,46.75439679)(623.08308105,46.86439941)
\curveto(623.0630804,46.90439664)(623.05308041,46.93439661)(623.05308105,46.95439941)
\curveto(623.0630804,46.98439656)(623.0630804,47.01939653)(623.05308105,47.05939941)
\curveto(623.03308043,47.13939641)(623.01808044,47.21939633)(623.00808105,47.29939941)
\curveto(623.00808045,47.38939616)(622.99808046,47.47439607)(622.97808105,47.55439941)
\lineto(622.97808105,47.67439941)
\curveto(622.97808048,47.71439583)(622.97308049,47.75939579)(622.96308105,47.80939941)
\curveto(622.95308051,47.85939569)(622.94808051,47.9443956)(622.94808105,48.06439941)
\curveto(622.94808051,48.19439535)(622.9580805,48.28939526)(622.97808105,48.34939941)
\curveto(622.99808046,48.41939513)(623.00308046,48.48939506)(622.99308105,48.55939941)
\curveto(622.98308048,48.62939492)(622.98808047,48.69939485)(623.00808105,48.76939941)
\curveto(623.01808044,48.81939473)(623.02308044,48.85939469)(623.02308105,48.88939941)
\curveto(623.03308043,48.92939462)(623.04308042,48.97439457)(623.05308105,49.02439941)
\curveto(623.08308038,49.1443944)(623.10808035,49.26439428)(623.12808105,49.38439941)
\curveto(623.1580803,49.50439404)(623.19808026,49.61939393)(623.24808105,49.72939941)
\curveto(623.39808006,50.09939345)(623.57807988,50.42939312)(623.78808105,50.71939941)
\curveto(624.00807945,51.01939253)(624.27307919,51.26939228)(624.58308105,51.46939941)
\curveto(624.70307876,51.549392)(624.82807863,51.61439193)(624.95808105,51.66439941)
\curveto(625.08807837,51.72439182)(625.22307824,51.78439176)(625.36308105,51.84439941)
\curveto(625.48307798,51.89439165)(625.61307785,51.92439162)(625.75308105,51.93439941)
\curveto(625.89307757,51.95439159)(626.03307743,51.98439156)(626.17308105,52.02439941)
\lineto(626.36808105,52.02439941)
\curveto(626.43807702,52.03439151)(626.50307696,52.0443915)(626.56308105,52.05439941)
\curveto(627.45307601,52.06439148)(628.19307527,51.87939167)(628.78308105,51.49939941)
\curveto(629.37307409,51.11939243)(629.79807366,50.62439292)(630.05808105,50.01439941)
\curveto(630.10807335,49.91439363)(630.14807331,49.81439373)(630.17808105,49.71439941)
\curveto(630.20807325,49.61439393)(630.24307322,49.50939404)(630.28308105,49.39939941)
\curveto(630.31307315,49.28939426)(630.33807312,49.16939438)(630.35808105,49.03939941)
\curveto(630.37807308,48.91939463)(630.40307306,48.79439475)(630.43308105,48.66439941)
\curveto(630.44307302,48.61439493)(630.44307302,48.55939499)(630.43308105,48.49939941)
\curveto(630.43307303,48.4493951)(630.43807302,48.39939515)(630.44808105,48.34939941)
\moveto(629.11308105,47.49439941)
\curveto(629.13307433,47.56439598)(629.13807432,47.6443959)(629.12808105,47.73439941)
\lineto(629.12808105,47.98939941)
\curveto(629.12807433,48.37939517)(629.09307437,48.70939484)(629.02308105,48.97939941)
\curveto(628.99307447,49.05939449)(628.96807449,49.13939441)(628.94808105,49.21939941)
\curveto(628.92807453,49.29939425)(628.90307456,49.37439417)(628.87308105,49.44439941)
\curveto(628.59307487,50.09439345)(628.14807531,50.544393)(627.53808105,50.79439941)
\curveto(627.46807599,50.82439272)(627.39307607,50.8443927)(627.31308105,50.85439941)
\lineto(627.07308105,50.91439941)
\curveto(626.99307647,50.93439261)(626.90807655,50.9443926)(626.81808105,50.94439941)
\lineto(626.54808105,50.94439941)
\lineto(626.27808105,50.89939941)
\curveto(626.17807728,50.87939267)(626.08307738,50.85439269)(625.99308105,50.82439941)
\curveto(625.91307755,50.80439274)(625.83307763,50.77439277)(625.75308105,50.73439941)
\curveto(625.68307778,50.71439283)(625.61807784,50.68439286)(625.55808105,50.64439941)
\curveto(625.49807796,50.60439294)(625.44307802,50.56439298)(625.39308105,50.52439941)
\curveto(625.15307831,50.35439319)(624.9580785,50.1493934)(624.80808105,49.90939941)
\curveto(624.6580788,49.66939388)(624.52807893,49.38939416)(624.41808105,49.06939941)
\curveto(624.38807907,48.96939458)(624.36807909,48.86439468)(624.35808105,48.75439941)
\curveto(624.34807911,48.65439489)(624.33307913,48.549395)(624.31308105,48.43939941)
\curveto(624.30307916,48.39939515)(624.29807916,48.33439521)(624.29808105,48.24439941)
\curveto(624.28807917,48.21439533)(624.28307918,48.17939537)(624.28308105,48.13939941)
\curveto(624.29307917,48.09939545)(624.29807916,48.05439549)(624.29808105,48.00439941)
\lineto(624.29808105,47.70439941)
\curveto(624.29807916,47.60439594)(624.30807915,47.51439603)(624.32808105,47.43439941)
\lineto(624.35808105,47.25439941)
\curveto(624.37807908,47.15439639)(624.39307907,47.05439649)(624.40308105,46.95439941)
\curveto(624.42307904,46.86439668)(624.45307901,46.77939677)(624.49308105,46.69939941)
\curveto(624.59307887,46.45939709)(624.70807875,46.23439731)(624.83808105,46.02439941)
\curveto(624.97807848,45.81439773)(625.14807831,45.63939791)(625.34808105,45.49939941)
\curveto(625.39807806,45.46939808)(625.44307802,45.4443981)(625.48308105,45.42439941)
\curveto(625.52307794,45.40439814)(625.56807789,45.37939817)(625.61808105,45.34939941)
\curveto(625.69807776,45.29939825)(625.78307768,45.25439829)(625.87308105,45.21439941)
\curveto(625.97307749,45.18439836)(626.07807738,45.15439839)(626.18808105,45.12439941)
\curveto(626.23807722,45.10439844)(626.28307718,45.09439845)(626.32308105,45.09439941)
\curveto(626.37307709,45.10439844)(626.42307704,45.10439844)(626.47308105,45.09439941)
\curveto(626.50307696,45.08439846)(626.5630769,45.07439847)(626.65308105,45.06439941)
\curveto(626.75307671,45.05439849)(626.82807663,45.05939849)(626.87808105,45.07939941)
\curveto(626.91807654,45.08939846)(626.9580765,45.08939846)(626.99808105,45.07939941)
\curveto(627.03807642,45.07939847)(627.07807638,45.08939846)(627.11808105,45.10939941)
\curveto(627.19807626,45.12939842)(627.27807618,45.1443984)(627.35808105,45.15439941)
\curveto(627.43807602,45.17439837)(627.51307595,45.19939835)(627.58308105,45.22939941)
\curveto(627.92307554,45.36939818)(628.19807526,45.56439798)(628.40808105,45.81439941)
\curveto(628.61807484,46.06439748)(628.79307467,46.35939719)(628.93308105,46.69939941)
\curveto(628.98307448,46.81939673)(629.01307445,46.9443966)(629.02308105,47.07439941)
\curveto(629.04307442,47.21439633)(629.07307439,47.35439619)(629.11308105,47.49439941)
}
}
{
\newrgbcolor{curcolor}{0 0 0}
\pscustom[linestyle=none,fillstyle=solid,fillcolor=curcolor]
{
\newpath
\moveto(635.5813623,52.05439941)
\curveto(635.81135751,52.05439149)(635.94135738,51.99439155)(635.9713623,51.87439941)
\curveto(636.00135732,51.76439178)(636.01635731,51.59939195)(636.0163623,51.37939941)
\lineto(636.0163623,51.09439941)
\curveto(636.01635731,51.00439254)(635.99135733,50.92939262)(635.9413623,50.86939941)
\curveto(635.88135744,50.78939276)(635.79635753,50.7443928)(635.6863623,50.73439941)
\curveto(635.57635775,50.73439281)(635.46635786,50.71939283)(635.3563623,50.68939941)
\curveto(635.21635811,50.65939289)(635.08135824,50.62939292)(634.9513623,50.59939941)
\curveto(634.83135849,50.56939298)(634.71635861,50.52939302)(634.6063623,50.47939941)
\curveto(634.31635901,50.3493932)(634.08135924,50.16939338)(633.9013623,49.93939941)
\curveto(633.7213596,49.71939383)(633.56635976,49.46439408)(633.4363623,49.17439941)
\curveto(633.39635993,49.06439448)(633.36635996,48.9493946)(633.3463623,48.82939941)
\curveto(633.32636,48.71939483)(633.30136002,48.60439494)(633.2713623,48.48439941)
\curveto(633.26136006,48.43439511)(633.25636007,48.38439516)(633.2563623,48.33439941)
\curveto(633.26636006,48.28439526)(633.26636006,48.23439531)(633.2563623,48.18439941)
\curveto(633.2263601,48.06439548)(633.21136011,47.92439562)(633.2113623,47.76439941)
\curveto(633.2213601,47.61439593)(633.2263601,47.46939608)(633.2263623,47.32939941)
\lineto(633.2263623,45.48439941)
\lineto(633.2263623,45.13939941)
\curveto(633.2263601,45.01939853)(633.2213601,44.90439864)(633.2113623,44.79439941)
\curveto(633.20136012,44.68439886)(633.19636013,44.58939896)(633.1963623,44.50939941)
\curveto(633.20636012,44.42939912)(633.18636014,44.35939919)(633.1363623,44.29939941)
\curveto(633.08636024,44.22939932)(633.00636032,44.18939936)(632.8963623,44.17939941)
\curveto(632.79636053,44.16939938)(632.68636064,44.16439938)(632.5663623,44.16439941)
\lineto(632.2963623,44.16439941)
\curveto(632.24636108,44.18439936)(632.19636113,44.19939935)(632.1463623,44.20939941)
\curveto(632.10636122,44.22939932)(632.07636125,44.25439929)(632.0563623,44.28439941)
\curveto(632.00636132,44.35439919)(631.97636135,44.43939911)(631.9663623,44.53939941)
\lineto(631.9663623,44.86939941)
\lineto(631.9663623,46.02439941)
\lineto(631.9663623,50.17939941)
\lineto(631.9663623,51.21439941)
\lineto(631.9663623,51.51439941)
\curveto(631.97636135,51.61439193)(632.00636132,51.69939185)(632.0563623,51.76939941)
\curveto(632.08636124,51.80939174)(632.13636119,51.83939171)(632.2063623,51.85939941)
\curveto(632.28636104,51.87939167)(632.37136095,51.88939166)(632.4613623,51.88939941)
\curveto(632.55136077,51.89939165)(632.64136068,51.89939165)(632.7313623,51.88939941)
\curveto(632.8213605,51.87939167)(632.89136043,51.86439168)(632.9413623,51.84439941)
\curveto(633.0213603,51.81439173)(633.07136025,51.75439179)(633.0913623,51.66439941)
\curveto(633.1213602,51.58439196)(633.13636019,51.49439205)(633.1363623,51.39439941)
\lineto(633.1363623,51.09439941)
\curveto(633.13636019,50.99439255)(633.15636017,50.90439264)(633.1963623,50.82439941)
\curveto(633.20636012,50.80439274)(633.21636011,50.78939276)(633.2263623,50.77939941)
\lineto(633.2713623,50.73439941)
\curveto(633.38135994,50.73439281)(633.47135985,50.77939277)(633.5413623,50.86939941)
\curveto(633.61135971,50.96939258)(633.67135965,51.0493925)(633.7213623,51.10939941)
\lineto(633.8113623,51.19939941)
\curveto(633.90135942,51.30939224)(634.0263593,51.42439212)(634.1863623,51.54439941)
\curveto(634.34635898,51.66439188)(634.49635883,51.75439179)(634.6363623,51.81439941)
\curveto(634.7263586,51.86439168)(634.8213585,51.89939165)(634.9213623,51.91939941)
\curveto(635.0213583,51.9493916)(635.1263582,51.97939157)(635.2363623,52.00939941)
\curveto(635.29635803,52.01939153)(635.35635797,52.02439152)(635.4163623,52.02439941)
\curveto(635.47635785,52.03439151)(635.53135779,52.0443915)(635.5813623,52.05439941)
}
}
{
\newrgbcolor{curcolor}{0 0 0}
\pscustom[linestyle=none,fillstyle=solid,fillcolor=curcolor]
{
\newpath
\moveto(70.59249146,83.16295776)
\lineto(70.59249146,82.90795776)
\curveto(70.60248375,82.827953)(70.59748376,82.75295307)(70.57749146,82.68295776)
\lineto(70.57749146,82.44295776)
\lineto(70.57749146,82.27795776)
\curveto(70.5574838,82.17795365)(70.54748381,82.07295375)(70.54749146,81.96295776)
\curveto(70.54748381,81.86295396)(70.53748382,81.76295406)(70.51749146,81.66295776)
\lineto(70.51749146,81.51295776)
\curveto(70.48748387,81.37295445)(70.46748389,81.23295459)(70.45749146,81.09295776)
\curveto(70.44748391,80.96295486)(70.42248393,80.83295499)(70.38249146,80.70295776)
\curveto(70.36248399,80.6229552)(70.34248401,80.53795529)(70.32249146,80.44795776)
\lineto(70.26249146,80.20795776)
\lineto(70.14249146,79.90795776)
\curveto(70.11248424,79.81795601)(70.07748428,79.7279561)(70.03749146,79.63795776)
\curveto(69.93748442,79.41795641)(69.80248455,79.20295662)(69.63249146,78.99295776)
\curveto(69.47248488,78.78295704)(69.29748506,78.61295721)(69.10749146,78.48295776)
\curveto(69.0574853,78.44295738)(68.99748536,78.40295742)(68.92749146,78.36295776)
\curveto(68.86748549,78.33295749)(68.80748555,78.29795753)(68.74749146,78.25795776)
\curveto(68.66748569,78.20795762)(68.57248578,78.16795766)(68.46249146,78.13795776)
\curveto(68.352486,78.10795772)(68.24748611,78.07795775)(68.14749146,78.04795776)
\curveto(68.03748632,78.00795782)(67.92748643,77.98295784)(67.81749146,77.97295776)
\curveto(67.70748665,77.96295786)(67.59248676,77.94795788)(67.47249146,77.92795776)
\curveto(67.43248692,77.91795791)(67.38748697,77.91795791)(67.33749146,77.92795776)
\curveto(67.29748706,77.9279579)(67.2574871,77.9229579)(67.21749146,77.91295776)
\curveto(67.17748718,77.90295792)(67.12248723,77.89795793)(67.05249146,77.89795776)
\curveto(66.98248737,77.89795793)(66.93248742,77.90295792)(66.90249146,77.91295776)
\curveto(66.8524875,77.93295789)(66.80748755,77.93795789)(66.76749146,77.92795776)
\curveto(66.72748763,77.91795791)(66.69248766,77.91795791)(66.66249146,77.92795776)
\lineto(66.57249146,77.92795776)
\curveto(66.51248784,77.94795788)(66.44748791,77.96295786)(66.37749146,77.97295776)
\curveto(66.31748804,77.97295785)(66.2524881,77.97795785)(66.18249146,77.98795776)
\curveto(66.01248834,78.03795779)(65.8524885,78.08795774)(65.70249146,78.13795776)
\curveto(65.5524888,78.18795764)(65.40748895,78.25295757)(65.26749146,78.33295776)
\curveto(65.21748914,78.37295745)(65.16248919,78.40295742)(65.10249146,78.42295776)
\curveto(65.0524893,78.45295737)(65.00248935,78.48795734)(64.95249146,78.52795776)
\curveto(64.71248964,78.70795712)(64.51248984,78.9279569)(64.35249146,79.18795776)
\curveto(64.19249016,79.44795638)(64.0524903,79.73295609)(63.93249146,80.04295776)
\curveto(63.87249048,80.18295564)(63.82749053,80.3229555)(63.79749146,80.46295776)
\curveto(63.76749059,80.61295521)(63.73249062,80.76795506)(63.69249146,80.92795776)
\curveto(63.67249068,81.03795479)(63.6574907,81.14795468)(63.64749146,81.25795776)
\curveto(63.63749072,81.36795446)(63.62249073,81.47795435)(63.60249146,81.58795776)
\curveto(63.59249076,81.6279542)(63.58749077,81.66795416)(63.58749146,81.70795776)
\curveto(63.59749076,81.74795408)(63.59749076,81.78795404)(63.58749146,81.82795776)
\curveto(63.57749078,81.87795395)(63.57249078,81.9279539)(63.57249146,81.97795776)
\lineto(63.57249146,82.14295776)
\curveto(63.5524908,82.19295363)(63.54749081,82.24295358)(63.55749146,82.29295776)
\curveto(63.56749079,82.35295347)(63.56749079,82.40795342)(63.55749146,82.45795776)
\curveto(63.54749081,82.49795333)(63.54749081,82.54295328)(63.55749146,82.59295776)
\curveto(63.56749079,82.64295318)(63.56249079,82.69295313)(63.54249146,82.74295776)
\curveto(63.52249083,82.81295301)(63.51749084,82.88795294)(63.52749146,82.96795776)
\curveto(63.53749082,83.05795277)(63.54249081,83.14295268)(63.54249146,83.22295776)
\curveto(63.54249081,83.31295251)(63.53749082,83.41295241)(63.52749146,83.52295776)
\curveto(63.51749084,83.64295218)(63.52249083,83.74295208)(63.54249146,83.82295776)
\lineto(63.54249146,84.10795776)
\lineto(63.58749146,84.73795776)
\curveto(63.59749076,84.83795099)(63.60749075,84.93295089)(63.61749146,85.02295776)
\lineto(63.64749146,85.32295776)
\curveto(63.66749069,85.37295045)(63.67249068,85.4229504)(63.66249146,85.47295776)
\curveto(63.66249069,85.53295029)(63.67249068,85.58795024)(63.69249146,85.63795776)
\curveto(63.74249061,85.80795002)(63.78249057,85.97294985)(63.81249146,86.13295776)
\curveto(63.84249051,86.30294952)(63.89249046,86.46294936)(63.96249146,86.61295776)
\curveto(64.1524902,87.07294875)(64.37248998,87.44794838)(64.62249146,87.73795776)
\curveto(64.88248947,88.0279478)(65.24248911,88.27294755)(65.70249146,88.47295776)
\curveto(65.83248852,88.5229473)(65.96248839,88.55794727)(66.09249146,88.57795776)
\curveto(66.23248812,88.59794723)(66.37248798,88.6229472)(66.51249146,88.65295776)
\curveto(66.58248777,88.66294716)(66.64748771,88.66794716)(66.70749146,88.66795776)
\curveto(66.76748759,88.66794716)(66.83248752,88.67294715)(66.90249146,88.68295776)
\curveto(67.73248662,88.70294712)(68.40248595,88.55294727)(68.91249146,88.23295776)
\curveto(69.42248493,87.9229479)(69.80248455,87.48294834)(70.05249146,86.91295776)
\curveto(70.10248425,86.79294903)(70.14748421,86.66794916)(70.18749146,86.53795776)
\curveto(70.22748413,86.40794942)(70.27248408,86.27294955)(70.32249146,86.13295776)
\curveto(70.34248401,86.05294977)(70.357484,85.96794986)(70.36749146,85.87795776)
\lineto(70.42749146,85.63795776)
\curveto(70.4574839,85.5279503)(70.47248388,85.41795041)(70.47249146,85.30795776)
\curveto(70.48248387,85.19795063)(70.49748386,85.08795074)(70.51749146,84.97795776)
\curveto(70.53748382,84.9279509)(70.54248381,84.88295094)(70.53249146,84.84295776)
\curveto(70.53248382,84.80295102)(70.53748382,84.76295106)(70.54749146,84.72295776)
\curveto(70.5574838,84.67295115)(70.5574838,84.61795121)(70.54749146,84.55795776)
\curveto(70.54748381,84.50795132)(70.5524838,84.45795137)(70.56249146,84.40795776)
\lineto(70.56249146,84.27295776)
\curveto(70.58248377,84.21295161)(70.58248377,84.14295168)(70.56249146,84.06295776)
\curveto(70.5524838,83.99295183)(70.5574838,83.9279519)(70.57749146,83.86795776)
\curveto(70.58748377,83.83795199)(70.59248376,83.79795203)(70.59249146,83.74795776)
\lineto(70.59249146,83.62795776)
\lineto(70.59249146,83.16295776)
\moveto(69.04749146,80.83795776)
\curveto(69.14748521,81.15795467)(69.20748515,81.5229543)(69.22749146,81.93295776)
\curveto(69.24748511,82.34295348)(69.2574851,82.75295307)(69.25749146,83.16295776)
\curveto(69.2574851,83.59295223)(69.24748511,84.01295181)(69.22749146,84.42295776)
\curveto(69.20748515,84.83295099)(69.16248519,85.21795061)(69.09249146,85.57795776)
\curveto(69.02248533,85.93794989)(68.91248544,86.25794957)(68.76249146,86.53795776)
\curveto(68.62248573,86.827949)(68.42748593,87.06294876)(68.17749146,87.24295776)
\curveto(68.01748634,87.35294847)(67.83748652,87.43294839)(67.63749146,87.48295776)
\curveto(67.43748692,87.54294828)(67.19248716,87.57294825)(66.90249146,87.57295776)
\curveto(66.88248747,87.55294827)(66.84748751,87.54294828)(66.79749146,87.54295776)
\curveto(66.74748761,87.55294827)(66.70748765,87.55294827)(66.67749146,87.54295776)
\curveto(66.59748776,87.5229483)(66.52248783,87.50294832)(66.45249146,87.48295776)
\curveto(66.39248796,87.47294835)(66.32748803,87.45294837)(66.25749146,87.42295776)
\curveto(65.98748837,87.30294852)(65.76748859,87.13294869)(65.59749146,86.91295776)
\curveto(65.43748892,86.70294912)(65.30248905,86.45794937)(65.19249146,86.17795776)
\curveto(65.14248921,86.06794976)(65.10248925,85.94794988)(65.07249146,85.81795776)
\curveto(65.0524893,85.69795013)(65.02748933,85.57295025)(64.99749146,85.44295776)
\curveto(64.97748938,85.39295043)(64.96748939,85.33795049)(64.96749146,85.27795776)
\curveto(64.96748939,85.2279506)(64.96248939,85.17795065)(64.95249146,85.12795776)
\curveto(64.94248941,85.03795079)(64.93248942,84.94295088)(64.92249146,84.84295776)
\curveto(64.91248944,84.75295107)(64.90248945,84.65795117)(64.89249146,84.55795776)
\curveto(64.89248946,84.47795135)(64.88748947,84.39295143)(64.87749146,84.30295776)
\lineto(64.87749146,84.06295776)
\lineto(64.87749146,83.88295776)
\curveto(64.86748949,83.85295197)(64.86248949,83.81795201)(64.86249146,83.77795776)
\lineto(64.86249146,83.64295776)
\lineto(64.86249146,83.19295776)
\curveto(64.86248949,83.11295271)(64.8574895,83.0279528)(64.84749146,82.93795776)
\curveto(64.84748951,82.85795297)(64.8574895,82.78295304)(64.87749146,82.71295776)
\lineto(64.87749146,82.44295776)
\curveto(64.87748948,82.4229534)(64.87248948,82.39295343)(64.86249146,82.35295776)
\curveto(64.86248949,82.3229535)(64.86748949,82.29795353)(64.87749146,82.27795776)
\curveto(64.88748947,82.17795365)(64.89248946,82.07795375)(64.89249146,81.97795776)
\curveto(64.90248945,81.88795394)(64.91248944,81.78795404)(64.92249146,81.67795776)
\curveto(64.9524894,81.55795427)(64.96748939,81.43295439)(64.96749146,81.30295776)
\curveto(64.97748938,81.18295464)(65.00248935,81.06795476)(65.04249146,80.95795776)
\curveto(65.12248923,80.65795517)(65.20748915,80.39295543)(65.29749146,80.16295776)
\curveto(65.39748896,79.93295589)(65.54248881,79.71795611)(65.73249146,79.51795776)
\curveto(65.94248841,79.31795651)(66.20748815,79.16795666)(66.52749146,79.06795776)
\curveto(66.56748779,79.04795678)(66.60248775,79.03795679)(66.63249146,79.03795776)
\curveto(66.67248768,79.04795678)(66.71748764,79.04295678)(66.76749146,79.02295776)
\curveto(66.80748755,79.01295681)(66.87748748,79.00295682)(66.97749146,78.99295776)
\curveto(67.08748727,78.98295684)(67.17248718,78.98795684)(67.23249146,79.00795776)
\curveto(67.30248705,79.0279568)(67.37248698,79.03795679)(67.44249146,79.03795776)
\curveto(67.51248684,79.04795678)(67.57748678,79.06295676)(67.63749146,79.08295776)
\curveto(67.83748652,79.14295668)(68.01748634,79.2279566)(68.17749146,79.33795776)
\curveto(68.20748615,79.35795647)(68.23248612,79.37795645)(68.25249146,79.39795776)
\lineto(68.31249146,79.45795776)
\curveto(68.352486,79.47795635)(68.40248595,79.51795631)(68.46249146,79.57795776)
\curveto(68.56248579,79.71795611)(68.64748571,79.84795598)(68.71749146,79.96795776)
\curveto(68.78748557,80.08795574)(68.8574855,80.23295559)(68.92749146,80.40295776)
\curveto(68.9574854,80.47295535)(68.97748538,80.54295528)(68.98749146,80.61295776)
\curveto(69.00748535,80.68295514)(69.02748533,80.75795507)(69.04749146,80.83795776)
}
}
{
\newrgbcolor{curcolor}{0 0 0}
\pscustom[linestyle=none,fillstyle=solid,fillcolor=curcolor]
{
\newpath
\moveto(58.55287964,162.96866333)
\curveto(59.242875,162.9786527)(59.8428744,162.85865282)(60.35287964,162.60866333)
\curveto(60.87287337,162.35865332)(61.26787298,162.02365366)(61.53787964,161.60366333)
\curveto(61.58787266,161.52365416)(61.63287261,161.43365425)(61.67287964,161.33366333)
\curveto(61.71287253,161.24365444)(61.75787249,161.14865453)(61.80787964,161.04866333)
\curveto(61.8478724,160.94865473)(61.87787237,160.84865483)(61.89787964,160.74866333)
\curveto(61.91787233,160.64865503)(61.93787231,160.54365514)(61.95787964,160.43366333)
\curveto(61.97787227,160.3836553)(61.98287226,160.33865534)(61.97287964,160.29866333)
\curveto(61.96287228,160.25865542)(61.96787228,160.21365547)(61.98787964,160.16366333)
\curveto(61.99787225,160.11365557)(62.00287224,160.02865565)(62.00287964,159.90866333)
\curveto(62.00287224,159.79865588)(61.99787225,159.71365597)(61.98787964,159.65366333)
\curveto(61.96787228,159.59365609)(61.95787229,159.53365615)(61.95787964,159.47366333)
\curveto(61.96787228,159.41365627)(61.96287228,159.35365633)(61.94287964,159.29366333)
\curveto(61.90287234,159.15365653)(61.86787238,159.01865666)(61.83787964,158.88866333)
\curveto(61.80787244,158.75865692)(61.76787248,158.63365705)(61.71787964,158.51366333)
\curveto(61.65787259,158.37365731)(61.58787266,158.24865743)(61.50787964,158.13866333)
\curveto(61.43787281,158.02865765)(61.36287288,157.91865776)(61.28287964,157.80866333)
\lineto(61.22287964,157.74866333)
\curveto(61.21287303,157.72865795)(61.19787305,157.70865797)(61.17787964,157.68866333)
\curveto(61.05787319,157.52865815)(60.92287332,157.3836583)(60.77287964,157.25366333)
\curveto(60.62287362,157.12365856)(60.46287378,156.99865868)(60.29287964,156.87866333)
\curveto(59.98287426,156.65865902)(59.68787456,156.45365923)(59.40787964,156.26366333)
\curveto(59.17787507,156.12365956)(58.9478753,155.98865969)(58.71787964,155.85866333)
\curveto(58.49787575,155.72865995)(58.27787597,155.59366009)(58.05787964,155.45366333)
\curveto(57.80787644,155.2836604)(57.56787668,155.10366058)(57.33787964,154.91366333)
\curveto(57.11787713,154.72366096)(56.92787732,154.49866118)(56.76787964,154.23866333)
\curveto(56.72787752,154.1786615)(56.69287755,154.11866156)(56.66287964,154.05866333)
\curveto(56.63287761,154.00866167)(56.60287764,153.94366174)(56.57287964,153.86366333)
\curveto(56.55287769,153.79366189)(56.5478777,153.73366195)(56.55787964,153.68366333)
\curveto(56.57787767,153.61366207)(56.61287763,153.55866212)(56.66287964,153.51866333)
\curveto(56.71287753,153.48866219)(56.77287747,153.46866221)(56.84287964,153.45866333)
\lineto(57.08287964,153.45866333)
\lineto(57.83287964,153.45866333)
\lineto(60.63787964,153.45866333)
\lineto(61.29787964,153.45866333)
\curveto(61.38787286,153.45866222)(61.47287277,153.45366223)(61.55287964,153.44366333)
\curveto(61.63287261,153.44366224)(61.69787255,153.42366226)(61.74787964,153.38366333)
\curveto(61.79787245,153.34366234)(61.83787241,153.26866241)(61.86787964,153.15866333)
\curveto(61.90787234,153.05866262)(61.91787233,152.95866272)(61.89787964,152.85866333)
\lineto(61.89787964,152.72366333)
\curveto(61.87787237,152.65366303)(61.85787239,152.59366309)(61.83787964,152.54366333)
\curveto(61.81787243,152.49366319)(61.78287246,152.45366323)(61.73287964,152.42366333)
\curveto(61.68287256,152.3836633)(61.61287263,152.36366332)(61.52287964,152.36366333)
\lineto(61.25287964,152.36366333)
\lineto(60.35287964,152.36366333)
\lineto(56.84287964,152.36366333)
\lineto(55.77787964,152.36366333)
\curveto(55.69787855,152.36366332)(55.60787864,152.35866332)(55.50787964,152.34866333)
\curveto(55.40787884,152.34866333)(55.32287892,152.35866332)(55.25287964,152.37866333)
\curveto(55.0428792,152.44866323)(54.97787927,152.62866305)(55.05787964,152.91866333)
\curveto(55.06787918,152.95866272)(55.06787918,152.99366269)(55.05787964,153.02366333)
\curveto(55.05787919,153.06366262)(55.06787918,153.10866257)(55.08787964,153.15866333)
\curveto(55.10787914,153.23866244)(55.12787912,153.32366236)(55.14787964,153.41366333)
\curveto(55.16787908,153.50366218)(55.19287905,153.58866209)(55.22287964,153.66866333)
\curveto(55.38287886,154.15866152)(55.58287866,154.57366111)(55.82287964,154.91366333)
\curveto(56.00287824,155.16366052)(56.20787804,155.38866029)(56.43787964,155.58866333)
\curveto(56.66787758,155.79865988)(56.90787734,155.99365969)(57.15787964,156.17366333)
\curveto(57.41787683,156.35365933)(57.68287656,156.52365916)(57.95287964,156.68366333)
\curveto(58.23287601,156.85365883)(58.50287574,157.02865865)(58.76287964,157.20866333)
\curveto(58.87287537,157.28865839)(58.97787527,157.36365832)(59.07787964,157.43366333)
\curveto(59.18787506,157.50365818)(59.29787495,157.5786581)(59.40787964,157.65866333)
\curveto(59.4478748,157.68865799)(59.48287476,157.71865796)(59.51287964,157.74866333)
\curveto(59.55287469,157.78865789)(59.59287465,157.81865786)(59.63287964,157.83866333)
\curveto(59.77287447,157.94865773)(59.89787435,158.07365761)(60.00787964,158.21366333)
\curveto(60.02787422,158.24365744)(60.05287419,158.26865741)(60.08287964,158.28866333)
\curveto(60.11287413,158.31865736)(60.13787411,158.34865733)(60.15787964,158.37866333)
\curveto(60.23787401,158.4786572)(60.30287394,158.5786571)(60.35287964,158.67866333)
\curveto(60.41287383,158.7786569)(60.46787378,158.88865679)(60.51787964,159.00866333)
\curveto(60.5478737,159.0786566)(60.56787368,159.15365653)(60.57787964,159.23366333)
\lineto(60.63787964,159.47366333)
\lineto(60.63787964,159.56366333)
\curveto(60.6478736,159.59365609)(60.65287359,159.62365606)(60.65287964,159.65366333)
\curveto(60.67287357,159.72365596)(60.67787357,159.81865586)(60.66787964,159.93866333)
\curveto(60.66787358,160.06865561)(60.65787359,160.16865551)(60.63787964,160.23866333)
\curveto(60.61787363,160.31865536)(60.59787365,160.39365529)(60.57787964,160.46366333)
\curveto(60.56787368,160.54365514)(60.5478737,160.62365506)(60.51787964,160.70366333)
\curveto(60.40787384,160.94365474)(60.25787399,161.14365454)(60.06787964,161.30366333)
\curveto(59.88787436,161.47365421)(59.66787458,161.61365407)(59.40787964,161.72366333)
\curveto(59.33787491,161.74365394)(59.26787498,161.75865392)(59.19787964,161.76866333)
\curveto(59.12787512,161.78865389)(59.05287519,161.80865387)(58.97287964,161.82866333)
\curveto(58.89287535,161.84865383)(58.78287546,161.85865382)(58.64287964,161.85866333)
\curveto(58.51287573,161.85865382)(58.40787584,161.84865383)(58.32787964,161.82866333)
\curveto(58.26787598,161.81865386)(58.21287603,161.81365387)(58.16287964,161.81366333)
\curveto(58.11287613,161.81365387)(58.06287618,161.80365388)(58.01287964,161.78366333)
\curveto(57.91287633,161.74365394)(57.81787643,161.70365398)(57.72787964,161.66366333)
\curveto(57.6478766,161.62365406)(57.56787668,161.5786541)(57.48787964,161.52866333)
\curveto(57.45787679,161.50865417)(57.42787682,161.4836542)(57.39787964,161.45366333)
\curveto(57.37787687,161.42365426)(57.35287689,161.39865428)(57.32287964,161.37866333)
\lineto(57.24787964,161.30366333)
\curveto(57.21787703,161.2836544)(57.19287705,161.26365442)(57.17287964,161.24366333)
\lineto(57.02287964,161.03366333)
\curveto(56.98287726,160.97365471)(56.93787731,160.90865477)(56.88787964,160.83866333)
\curveto(56.82787742,160.74865493)(56.77787747,160.64365504)(56.73787964,160.52366333)
\curveto(56.70787754,160.41365527)(56.67287757,160.30365538)(56.63287964,160.19366333)
\curveto(56.59287765,160.0836556)(56.56787768,159.93865574)(56.55787964,159.75866333)
\curveto(56.5478777,159.58865609)(56.51787773,159.46365622)(56.46787964,159.38366333)
\curveto(56.41787783,159.30365638)(56.3428779,159.25865642)(56.24287964,159.24866333)
\curveto(56.1428781,159.23865644)(56.03287821,159.23365645)(55.91287964,159.23366333)
\curveto(55.87287837,159.23365645)(55.83287841,159.22865645)(55.79287964,159.21866333)
\curveto(55.75287849,159.21865646)(55.71787853,159.22365646)(55.68787964,159.23366333)
\curveto(55.63787861,159.25365643)(55.58787866,159.26365642)(55.53787964,159.26366333)
\curveto(55.49787875,159.26365642)(55.45787879,159.27365641)(55.41787964,159.29366333)
\curveto(55.32787892,159.35365633)(55.28287896,159.48865619)(55.28287964,159.69866333)
\lineto(55.28287964,159.81866333)
\curveto(55.29287895,159.8786558)(55.29787895,159.93865574)(55.29787964,159.99866333)
\curveto(55.30787894,160.06865561)(55.31787893,160.13365555)(55.32787964,160.19366333)
\curveto(55.3478789,160.30365538)(55.36787888,160.40365528)(55.38787964,160.49366333)
\curveto(55.40787884,160.59365509)(55.43787881,160.68865499)(55.47787964,160.77866333)
\curveto(55.49787875,160.84865483)(55.51787873,160.90865477)(55.53787964,160.95866333)
\lineto(55.59787964,161.13866333)
\curveto(55.71787853,161.39865428)(55.87287837,161.64365404)(56.06287964,161.87366333)
\curveto(56.26287798,162.10365358)(56.47787777,162.28865339)(56.70787964,162.42866333)
\curveto(56.81787743,162.50865317)(56.93287731,162.57365311)(57.05287964,162.62366333)
\lineto(57.44287964,162.77366333)
\curveto(57.55287669,162.82365286)(57.66787658,162.85365283)(57.78787964,162.86366333)
\curveto(57.90787634,162.8836528)(58.03287621,162.90865277)(58.16287964,162.93866333)
\curveto(58.23287601,162.93865274)(58.29787595,162.93865274)(58.35787964,162.93866333)
\curveto(58.41787583,162.94865273)(58.48287576,162.95865272)(58.55287964,162.96866333)
}
}
{
\newrgbcolor{curcolor}{0 0 0}
\pscustom[linestyle=none,fillstyle=solid,fillcolor=curcolor]
{
\newpath
\moveto(65.16248901,162.77366333)
\lineto(68.76248901,162.77366333)
\lineto(69.40748901,162.77366333)
\curveto(69.48748248,162.77365291)(69.56248241,162.76865291)(69.63248901,162.75866333)
\curveto(69.70248227,162.75865292)(69.76248221,162.74865293)(69.81248901,162.72866333)
\curveto(69.88248209,162.69865298)(69.93748203,162.63865304)(69.97748901,162.54866333)
\curveto(69.99748197,162.51865316)(70.00748196,162.4786532)(70.00748901,162.42866333)
\lineto(70.00748901,162.29366333)
\curveto(70.01748195,162.1836535)(70.01248196,162.0786536)(69.99248901,161.97866333)
\curveto(69.98248199,161.8786538)(69.94748202,161.80865387)(69.88748901,161.76866333)
\curveto(69.79748217,161.69865398)(69.66248231,161.66365402)(69.48248901,161.66366333)
\curveto(69.30248267,161.67365401)(69.13748283,161.678654)(68.98748901,161.67866333)
\lineto(66.99248901,161.67866333)
\lineto(66.49748901,161.67866333)
\lineto(66.36248901,161.67866333)
\curveto(66.32248565,161.678654)(66.28248569,161.67365401)(66.24248901,161.66366333)
\lineto(66.03248901,161.66366333)
\curveto(65.92248605,161.63365405)(65.84248613,161.59365409)(65.79248901,161.54366333)
\curveto(65.74248623,161.50365418)(65.70748626,161.44865423)(65.68748901,161.37866333)
\curveto(65.6674863,161.31865436)(65.65248632,161.24865443)(65.64248901,161.16866333)
\curveto(65.63248634,161.08865459)(65.61248636,160.99865468)(65.58248901,160.89866333)
\curveto(65.53248644,160.69865498)(65.49248648,160.49365519)(65.46248901,160.28366333)
\curveto(65.43248654,160.07365561)(65.39248658,159.86865581)(65.34248901,159.66866333)
\curveto(65.32248665,159.59865608)(65.31248666,159.52865615)(65.31248901,159.45866333)
\curveto(65.31248666,159.39865628)(65.30248667,159.33365635)(65.28248901,159.26366333)
\curveto(65.2724867,159.23365645)(65.26248671,159.19365649)(65.25248901,159.14366333)
\curveto(65.25248672,159.10365658)(65.25748671,159.06365662)(65.26748901,159.02366333)
\curveto(65.28748668,158.97365671)(65.31248666,158.92865675)(65.34248901,158.88866333)
\curveto(65.38248659,158.85865682)(65.44248653,158.85365683)(65.52248901,158.87366333)
\curveto(65.58248639,158.89365679)(65.64248633,158.91865676)(65.70248901,158.94866333)
\curveto(65.76248621,158.98865669)(65.82248615,159.02365666)(65.88248901,159.05366333)
\curveto(65.94248603,159.07365661)(65.99248598,159.08865659)(66.03248901,159.09866333)
\curveto(66.22248575,159.1786565)(66.42748554,159.23365645)(66.64748901,159.26366333)
\curveto(66.87748509,159.29365639)(67.10748486,159.30365638)(67.33748901,159.29366333)
\curveto(67.57748439,159.29365639)(67.80748416,159.26865641)(68.02748901,159.21866333)
\curveto(68.24748372,159.1786565)(68.44748352,159.11865656)(68.62748901,159.03866333)
\curveto(68.67748329,159.01865666)(68.72248325,158.99865668)(68.76248901,158.97866333)
\curveto(68.81248316,158.95865672)(68.86248311,158.93365675)(68.91248901,158.90366333)
\curveto(69.26248271,158.69365699)(69.54248243,158.46365722)(69.75248901,158.21366333)
\curveto(69.972482,157.96365772)(70.1674818,157.63865804)(70.33748901,157.23866333)
\curveto(70.38748158,157.12865855)(70.42248155,157.01865866)(70.44248901,156.90866333)
\curveto(70.46248151,156.79865888)(70.48748148,156.683659)(70.51748901,156.56366333)
\curveto(70.52748144,156.53365915)(70.53248144,156.48865919)(70.53248901,156.42866333)
\curveto(70.55248142,156.36865931)(70.56248141,156.29865938)(70.56248901,156.21866333)
\curveto(70.56248141,156.14865953)(70.5724814,156.0836596)(70.59248901,156.02366333)
\lineto(70.59248901,155.85866333)
\curveto(70.60248137,155.80865987)(70.60748136,155.73865994)(70.60748901,155.64866333)
\curveto(70.60748136,155.55866012)(70.59748137,155.48866019)(70.57748901,155.43866333)
\curveto(70.55748141,155.3786603)(70.55248142,155.31866036)(70.56248901,155.25866333)
\curveto(70.5724814,155.20866047)(70.5674814,155.15866052)(70.54748901,155.10866333)
\curveto(70.50748146,154.94866073)(70.4724815,154.79866088)(70.44248901,154.65866333)
\curveto(70.41248156,154.51866116)(70.3674816,154.3836613)(70.30748901,154.25366333)
\curveto(70.14748182,153.8836618)(69.92748204,153.54866213)(69.64748901,153.24866333)
\curveto(69.3674826,152.94866273)(69.04748292,152.71866296)(68.68748901,152.55866333)
\curveto(68.51748345,152.4786632)(68.31748365,152.40366328)(68.08748901,152.33366333)
\curveto(67.97748399,152.29366339)(67.86248411,152.26866341)(67.74248901,152.25866333)
\curveto(67.62248435,152.24866343)(67.50248447,152.22866345)(67.38248901,152.19866333)
\curveto(67.33248464,152.1786635)(67.27748469,152.1786635)(67.21748901,152.19866333)
\curveto(67.15748481,152.20866347)(67.09748487,152.20366348)(67.03748901,152.18366333)
\curveto(66.93748503,152.16366352)(66.83748513,152.16366352)(66.73748901,152.18366333)
\lineto(66.60248901,152.18366333)
\curveto(66.55248542,152.20366348)(66.49248548,152.21366347)(66.42248901,152.21366333)
\curveto(66.36248561,152.20366348)(66.30748566,152.20866347)(66.25748901,152.22866333)
\curveto(66.21748575,152.23866344)(66.18248579,152.24366344)(66.15248901,152.24366333)
\curveto(66.12248585,152.24366344)(66.08748588,152.24866343)(66.04748901,152.25866333)
\lineto(65.77748901,152.31866333)
\curveto(65.68748628,152.33866334)(65.60248637,152.36866331)(65.52248901,152.40866333)
\curveto(65.18248679,152.54866313)(64.89248708,152.70366298)(64.65248901,152.87366333)
\curveto(64.41248756,153.05366263)(64.19248778,153.2836624)(63.99248901,153.56366333)
\curveto(63.84248813,153.79366189)(63.72748824,154.03366165)(63.64748901,154.28366333)
\curveto(63.62748834,154.33366135)(63.61748835,154.3786613)(63.61748901,154.41866333)
\curveto(63.61748835,154.46866121)(63.60748836,154.51866116)(63.58748901,154.56866333)
\curveto(63.5674884,154.62866105)(63.55248842,154.70866097)(63.54248901,154.80866333)
\curveto(63.54248843,154.90866077)(63.56248841,154.9836607)(63.60248901,155.03366333)
\curveto(63.65248832,155.11366057)(63.73248824,155.15866052)(63.84248901,155.16866333)
\curveto(63.95248802,155.1786605)(64.0674879,155.1836605)(64.18748901,155.18366333)
\lineto(64.35248901,155.18366333)
\curveto(64.41248756,155.1836605)(64.4674875,155.17366051)(64.51748901,155.15366333)
\curveto(64.60748736,155.13366055)(64.67748729,155.09366059)(64.72748901,155.03366333)
\curveto(64.79748717,154.94366074)(64.84248713,154.83366085)(64.86248901,154.70366333)
\curveto(64.89248708,154.5836611)(64.93748703,154.4786612)(64.99748901,154.38866333)
\curveto(65.18748678,154.04866163)(65.44748652,153.7786619)(65.77748901,153.57866333)
\curveto(65.87748609,153.51866216)(65.98248599,153.46866221)(66.09248901,153.42866333)
\curveto(66.21248576,153.39866228)(66.33248564,153.36366232)(66.45248901,153.32366333)
\curveto(66.62248535,153.27366241)(66.82748514,153.25366243)(67.06748901,153.26366333)
\curveto(67.31748465,153.2836624)(67.51748445,153.31866236)(67.66748901,153.36866333)
\curveto(68.03748393,153.48866219)(68.32748364,153.64866203)(68.53748901,153.84866333)
\curveto(68.75748321,154.05866162)(68.93748303,154.33866134)(69.07748901,154.68866333)
\curveto(69.12748284,154.78866089)(69.15748281,154.89366079)(69.16748901,155.00366333)
\curveto(69.18748278,155.11366057)(69.21248276,155.22866045)(69.24248901,155.34866333)
\lineto(69.24248901,155.45366333)
\curveto(69.25248272,155.49366019)(69.25748271,155.53366015)(69.25748901,155.57366333)
\curveto(69.2674827,155.60366008)(69.2674827,155.63866004)(69.25748901,155.67866333)
\lineto(69.25748901,155.79866333)
\curveto(69.25748271,156.05865962)(69.22748274,156.30365938)(69.16748901,156.53366333)
\curveto(69.05748291,156.8836588)(68.90248307,157.1786585)(68.70248901,157.41866333)
\curveto(68.50248347,157.66865801)(68.24248373,157.86365782)(67.92248901,158.00366333)
\lineto(67.74248901,158.06366333)
\curveto(67.69248428,158.0836576)(67.63248434,158.10365758)(67.56248901,158.12366333)
\curveto(67.51248446,158.14365754)(67.45248452,158.15365753)(67.38248901,158.15366333)
\curveto(67.32248465,158.16365752)(67.25748471,158.1786575)(67.18748901,158.19866333)
\lineto(67.03748901,158.19866333)
\curveto(66.99748497,158.21865746)(66.94248503,158.22865745)(66.87248901,158.22866333)
\curveto(66.81248516,158.22865745)(66.75748521,158.21865746)(66.70748901,158.19866333)
\lineto(66.60248901,158.19866333)
\curveto(66.5724854,158.19865748)(66.53748543,158.19365749)(66.49748901,158.18366333)
\lineto(66.25748901,158.12366333)
\curveto(66.17748579,158.11365757)(66.09748587,158.09365759)(66.01748901,158.06366333)
\curveto(65.77748619,157.96365772)(65.54748642,157.82865785)(65.32748901,157.65866333)
\curveto(65.23748673,157.58865809)(65.15248682,157.51365817)(65.07248901,157.43366333)
\curveto(64.99248698,157.36365832)(64.89248708,157.30865837)(64.77248901,157.26866333)
\curveto(64.68248729,157.23865844)(64.54248743,157.22865845)(64.35248901,157.23866333)
\curveto(64.1724878,157.24865843)(64.05248792,157.27365841)(63.99248901,157.31366333)
\curveto(63.94248803,157.35365833)(63.90248807,157.41365827)(63.87248901,157.49366333)
\curveto(63.85248812,157.57365811)(63.85248812,157.65865802)(63.87248901,157.74866333)
\curveto(63.90248807,157.86865781)(63.92248805,157.98865769)(63.93248901,158.10866333)
\curveto(63.95248802,158.23865744)(63.97748799,158.36365732)(64.00748901,158.48366333)
\curveto(64.02748794,158.52365716)(64.03248794,158.55865712)(64.02248901,158.58866333)
\curveto(64.02248795,158.62865705)(64.03248794,158.67365701)(64.05248901,158.72366333)
\curveto(64.0724879,158.81365687)(64.08748788,158.90365678)(64.09748901,158.99366333)
\curveto(64.10748786,159.09365659)(64.12748784,159.18865649)(64.15748901,159.27866333)
\curveto(64.1674878,159.33865634)(64.1724878,159.39865628)(64.17248901,159.45866333)
\curveto(64.18248779,159.51865616)(64.19748777,159.5786561)(64.21748901,159.63866333)
\curveto(64.2674877,159.83865584)(64.30248767,160.04365564)(64.32248901,160.25366333)
\curveto(64.35248762,160.47365521)(64.39248758,160.683655)(64.44248901,160.88366333)
\curveto(64.4724875,160.9836547)(64.49248748,161.0836546)(64.50248901,161.18366333)
\curveto(64.51248746,161.2836544)(64.52748744,161.3836543)(64.54748901,161.48366333)
\curveto(64.55748741,161.51365417)(64.56248741,161.55365413)(64.56248901,161.60366333)
\curveto(64.59248738,161.71365397)(64.61248736,161.81865386)(64.62248901,161.91866333)
\curveto(64.64248733,162.02865365)(64.6674873,162.13865354)(64.69748901,162.24866333)
\curveto(64.71748725,162.32865335)(64.73248724,162.39865328)(64.74248901,162.45866333)
\curveto(64.75248722,162.52865315)(64.77748719,162.58865309)(64.81748901,162.63866333)
\curveto(64.83748713,162.66865301)(64.8674871,162.68865299)(64.90748901,162.69866333)
\curveto(64.94748702,162.71865296)(64.99248698,162.73865294)(65.04248901,162.75866333)
\curveto(65.10248687,162.75865292)(65.14248683,162.76365292)(65.16248901,162.77366333)
}
}
{
\newrgbcolor{curcolor}{0 0 0}
\pscustom[linestyle=none,fillstyle=solid,fillcolor=curcolor]
{
\newpath
\moveto(56.75288208,236.70223694)
\lineto(60.35288208,236.70223694)
\lineto(60.99788208,236.70223694)
\curveto(61.07787555,236.70222651)(61.15287548,236.69722652)(61.22288208,236.68723694)
\curveto(61.29287534,236.68722653)(61.35287528,236.67722654)(61.40288208,236.65723694)
\curveto(61.47287516,236.62722659)(61.5278751,236.56722665)(61.56788208,236.47723694)
\curveto(61.58787504,236.44722677)(61.59787503,236.40722681)(61.59788208,236.35723694)
\lineto(61.59788208,236.22223694)
\curveto(61.60787502,236.1122271)(61.60287503,236.00722721)(61.58288208,235.90723694)
\curveto(61.57287506,235.80722741)(61.53787509,235.73722748)(61.47788208,235.69723694)
\curveto(61.38787524,235.62722759)(61.25287538,235.59222762)(61.07288208,235.59223694)
\curveto(60.89287574,235.60222761)(60.7278759,235.60722761)(60.57788208,235.60723694)
\lineto(58.58288208,235.60723694)
\lineto(58.08788208,235.60723694)
\lineto(57.95288208,235.60723694)
\curveto(57.91287872,235.60722761)(57.87287876,235.60222761)(57.83288208,235.59223694)
\lineto(57.62288208,235.59223694)
\curveto(57.51287912,235.56222765)(57.4328792,235.52222769)(57.38288208,235.47223694)
\curveto(57.3328793,235.43222778)(57.29787933,235.37722784)(57.27788208,235.30723694)
\curveto(57.25787937,235.24722797)(57.24287939,235.17722804)(57.23288208,235.09723694)
\curveto(57.22287941,235.0172282)(57.20287943,234.92722829)(57.17288208,234.82723694)
\curveto(57.12287951,234.62722859)(57.08287955,234.42222879)(57.05288208,234.21223694)
\curveto(57.02287961,234.00222921)(56.98287965,233.79722942)(56.93288208,233.59723694)
\curveto(56.91287972,233.52722969)(56.90287973,233.45722976)(56.90288208,233.38723694)
\curveto(56.90287973,233.32722989)(56.89287974,233.26222995)(56.87288208,233.19223694)
\curveto(56.86287977,233.16223005)(56.85287978,233.12223009)(56.84288208,233.07223694)
\curveto(56.84287979,233.03223018)(56.84787978,232.99223022)(56.85788208,232.95223694)
\curveto(56.87787975,232.90223031)(56.90287973,232.85723036)(56.93288208,232.81723694)
\curveto(56.97287966,232.78723043)(57.0328796,232.78223043)(57.11288208,232.80223694)
\curveto(57.17287946,232.82223039)(57.2328794,232.84723037)(57.29288208,232.87723694)
\curveto(57.35287928,232.9172303)(57.41287922,232.95223026)(57.47288208,232.98223694)
\curveto(57.5328791,233.00223021)(57.58287905,233.0172302)(57.62288208,233.02723694)
\curveto(57.81287882,233.10723011)(58.01787861,233.16223005)(58.23788208,233.19223694)
\curveto(58.46787816,233.22222999)(58.69787793,233.23222998)(58.92788208,233.22223694)
\curveto(59.16787746,233.22222999)(59.39787723,233.19723002)(59.61788208,233.14723694)
\curveto(59.83787679,233.10723011)(60.03787659,233.04723017)(60.21788208,232.96723694)
\curveto(60.26787636,232.94723027)(60.31287632,232.92723029)(60.35288208,232.90723694)
\curveto(60.40287623,232.88723033)(60.45287618,232.86223035)(60.50288208,232.83223694)
\curveto(60.85287578,232.62223059)(61.1328755,232.39223082)(61.34288208,232.14223694)
\curveto(61.56287507,231.89223132)(61.75787487,231.56723165)(61.92788208,231.16723694)
\curveto(61.97787465,231.05723216)(62.01287462,230.94723227)(62.03288208,230.83723694)
\curveto(62.05287458,230.72723249)(62.07787455,230.6122326)(62.10788208,230.49223694)
\curveto(62.11787451,230.46223275)(62.12287451,230.4172328)(62.12288208,230.35723694)
\curveto(62.14287449,230.29723292)(62.15287448,230.22723299)(62.15288208,230.14723694)
\curveto(62.15287448,230.07723314)(62.16287447,230.0122332)(62.18288208,229.95223694)
\lineto(62.18288208,229.78723694)
\curveto(62.19287444,229.73723348)(62.19787443,229.66723355)(62.19788208,229.57723694)
\curveto(62.19787443,229.48723373)(62.18787444,229.4172338)(62.16788208,229.36723694)
\curveto(62.14787448,229.30723391)(62.14287449,229.24723397)(62.15288208,229.18723694)
\curveto(62.16287447,229.13723408)(62.15787447,229.08723413)(62.13788208,229.03723694)
\curveto(62.09787453,228.87723434)(62.06287457,228.72723449)(62.03288208,228.58723694)
\curveto(62.00287463,228.44723477)(61.95787467,228.3122349)(61.89788208,228.18223694)
\curveto(61.73787489,227.8122354)(61.51787511,227.47723574)(61.23788208,227.17723694)
\curveto(60.95787567,226.87723634)(60.63787599,226.64723657)(60.27788208,226.48723694)
\curveto(60.10787652,226.40723681)(59.90787672,226.33223688)(59.67788208,226.26223694)
\curveto(59.56787706,226.22223699)(59.45287718,226.19723702)(59.33288208,226.18723694)
\curveto(59.21287742,226.17723704)(59.09287754,226.15723706)(58.97288208,226.12723694)
\curveto(58.92287771,226.10723711)(58.86787776,226.10723711)(58.80788208,226.12723694)
\curveto(58.74787788,226.13723708)(58.68787794,226.13223708)(58.62788208,226.11223694)
\curveto(58.5278781,226.09223712)(58.4278782,226.09223712)(58.32788208,226.11223694)
\lineto(58.19288208,226.11223694)
\curveto(58.14287849,226.13223708)(58.08287855,226.14223707)(58.01288208,226.14223694)
\curveto(57.95287868,226.13223708)(57.89787873,226.13723708)(57.84788208,226.15723694)
\curveto(57.80787882,226.16723705)(57.77287886,226.17223704)(57.74288208,226.17223694)
\curveto(57.71287892,226.17223704)(57.67787895,226.17723704)(57.63788208,226.18723694)
\lineto(57.36788208,226.24723694)
\curveto(57.27787935,226.26723695)(57.19287944,226.29723692)(57.11288208,226.33723694)
\curveto(56.77287986,226.47723674)(56.48288015,226.63223658)(56.24288208,226.80223694)
\curveto(56.00288063,226.98223623)(55.78288085,227.212236)(55.58288208,227.49223694)
\curveto(55.4328812,227.72223549)(55.31788131,227.96223525)(55.23788208,228.21223694)
\curveto(55.21788141,228.26223495)(55.20788142,228.30723491)(55.20788208,228.34723694)
\curveto(55.20788142,228.39723482)(55.19788143,228.44723477)(55.17788208,228.49723694)
\curveto(55.15788147,228.55723466)(55.14288149,228.63723458)(55.13288208,228.73723694)
\curveto(55.1328815,228.83723438)(55.15288148,228.9122343)(55.19288208,228.96223694)
\curveto(55.24288139,229.04223417)(55.32288131,229.08723413)(55.43288208,229.09723694)
\curveto(55.54288109,229.10723411)(55.65788097,229.1122341)(55.77788208,229.11223694)
\lineto(55.94288208,229.11223694)
\curveto(56.00288063,229.1122341)(56.05788057,229.10223411)(56.10788208,229.08223694)
\curveto(56.19788043,229.06223415)(56.26788036,229.02223419)(56.31788208,228.96223694)
\curveto(56.38788024,228.87223434)(56.4328802,228.76223445)(56.45288208,228.63223694)
\curveto(56.48288015,228.5122347)(56.5278801,228.40723481)(56.58788208,228.31723694)
\curveto(56.77787985,227.97723524)(57.03787959,227.70723551)(57.36788208,227.50723694)
\curveto(57.46787916,227.44723577)(57.57287906,227.39723582)(57.68288208,227.35723694)
\curveto(57.80287883,227.32723589)(57.92287871,227.29223592)(58.04288208,227.25223694)
\curveto(58.21287842,227.20223601)(58.41787821,227.18223603)(58.65788208,227.19223694)
\curveto(58.90787772,227.212236)(59.10787752,227.24723597)(59.25788208,227.29723694)
\curveto(59.627877,227.4172358)(59.91787671,227.57723564)(60.12788208,227.77723694)
\curveto(60.34787628,227.98723523)(60.5278761,228.26723495)(60.66788208,228.61723694)
\curveto(60.71787591,228.7172345)(60.74787588,228.82223439)(60.75788208,228.93223694)
\curveto(60.77787585,229.04223417)(60.80287583,229.15723406)(60.83288208,229.27723694)
\lineto(60.83288208,229.38223694)
\curveto(60.84287579,229.42223379)(60.84787578,229.46223375)(60.84788208,229.50223694)
\curveto(60.85787577,229.53223368)(60.85787577,229.56723365)(60.84788208,229.60723694)
\lineto(60.84788208,229.72723694)
\curveto(60.84787578,229.98723323)(60.81787581,230.23223298)(60.75788208,230.46223694)
\curveto(60.64787598,230.8122324)(60.49287614,231.10723211)(60.29288208,231.34723694)
\curveto(60.09287654,231.59723162)(59.8328768,231.79223142)(59.51288208,231.93223694)
\lineto(59.33288208,231.99223694)
\curveto(59.28287735,232.0122312)(59.22287741,232.03223118)(59.15288208,232.05223694)
\curveto(59.10287753,232.07223114)(59.04287759,232.08223113)(58.97288208,232.08223694)
\curveto(58.91287772,232.09223112)(58.84787778,232.10723111)(58.77788208,232.12723694)
\lineto(58.62788208,232.12723694)
\curveto(58.58787804,232.14723107)(58.5328781,232.15723106)(58.46288208,232.15723694)
\curveto(58.40287823,232.15723106)(58.34787828,232.14723107)(58.29788208,232.12723694)
\lineto(58.19288208,232.12723694)
\curveto(58.16287847,232.12723109)(58.1278785,232.12223109)(58.08788208,232.11223694)
\lineto(57.84788208,232.05223694)
\curveto(57.76787886,232.04223117)(57.68787894,232.02223119)(57.60788208,231.99223694)
\curveto(57.36787926,231.89223132)(57.13787949,231.75723146)(56.91788208,231.58723694)
\curveto(56.8278798,231.5172317)(56.74287989,231.44223177)(56.66288208,231.36223694)
\curveto(56.58288005,231.29223192)(56.48288015,231.23723198)(56.36288208,231.19723694)
\curveto(56.27288036,231.16723205)(56.1328805,231.15723206)(55.94288208,231.16723694)
\curveto(55.76288087,231.17723204)(55.64288099,231.20223201)(55.58288208,231.24223694)
\curveto(55.5328811,231.28223193)(55.49288114,231.34223187)(55.46288208,231.42223694)
\curveto(55.44288119,231.50223171)(55.44288119,231.58723163)(55.46288208,231.67723694)
\curveto(55.49288114,231.79723142)(55.51288112,231.9172313)(55.52288208,232.03723694)
\curveto(55.54288109,232.16723105)(55.56788106,232.29223092)(55.59788208,232.41223694)
\curveto(55.61788101,232.45223076)(55.62288101,232.48723073)(55.61288208,232.51723694)
\curveto(55.61288102,232.55723066)(55.62288101,232.60223061)(55.64288208,232.65223694)
\curveto(55.66288097,232.74223047)(55.67788095,232.83223038)(55.68788208,232.92223694)
\curveto(55.69788093,233.02223019)(55.71788091,233.1172301)(55.74788208,233.20723694)
\curveto(55.75788087,233.26722995)(55.76288087,233.32722989)(55.76288208,233.38723694)
\curveto(55.77288086,233.44722977)(55.78788084,233.50722971)(55.80788208,233.56723694)
\curveto(55.85788077,233.76722945)(55.89288074,233.97222924)(55.91288208,234.18223694)
\curveto(55.94288069,234.40222881)(55.98288065,234.6122286)(56.03288208,234.81223694)
\curveto(56.06288057,234.9122283)(56.08288055,235.0122282)(56.09288208,235.11223694)
\curveto(56.10288053,235.212228)(56.11788051,235.3122279)(56.13788208,235.41223694)
\curveto(56.14788048,235.44222777)(56.15288048,235.48222773)(56.15288208,235.53223694)
\curveto(56.18288045,235.64222757)(56.20288043,235.74722747)(56.21288208,235.84723694)
\curveto(56.2328804,235.95722726)(56.25788037,236.06722715)(56.28788208,236.17723694)
\curveto(56.30788032,236.25722696)(56.32288031,236.32722689)(56.33288208,236.38723694)
\curveto(56.34288029,236.45722676)(56.36788026,236.5172267)(56.40788208,236.56723694)
\curveto(56.4278802,236.59722662)(56.45788017,236.6172266)(56.49788208,236.62723694)
\curveto(56.53788009,236.64722657)(56.58288005,236.66722655)(56.63288208,236.68723694)
\curveto(56.69287994,236.68722653)(56.7328799,236.69222652)(56.75288208,236.70223694)
}
}
{
\newrgbcolor{curcolor}{0 0 0}
\pscustom[linestyle=none,fillstyle=solid,fillcolor=curcolor]
{
\newpath
\moveto(70.59249146,231.37723694)
\lineto(70.59249146,231.12223694)
\curveto(70.60248375,231.04223217)(70.59748376,230.96723225)(70.57749146,230.89723694)
\lineto(70.57749146,230.65723694)
\lineto(70.57749146,230.49223694)
\curveto(70.5574838,230.39223282)(70.54748381,230.28723293)(70.54749146,230.17723694)
\curveto(70.54748381,230.07723314)(70.53748382,229.97723324)(70.51749146,229.87723694)
\lineto(70.51749146,229.72723694)
\curveto(70.48748387,229.58723363)(70.46748389,229.44723377)(70.45749146,229.30723694)
\curveto(70.44748391,229.17723404)(70.42248393,229.04723417)(70.38249146,228.91723694)
\curveto(70.36248399,228.83723438)(70.34248401,228.75223446)(70.32249146,228.66223694)
\lineto(70.26249146,228.42223694)
\lineto(70.14249146,228.12223694)
\curveto(70.11248424,228.03223518)(70.07748428,227.94223527)(70.03749146,227.85223694)
\curveto(69.93748442,227.63223558)(69.80248455,227.4172358)(69.63249146,227.20723694)
\curveto(69.47248488,226.99723622)(69.29748506,226.82723639)(69.10749146,226.69723694)
\curveto(69.0574853,226.65723656)(68.99748536,226.6172366)(68.92749146,226.57723694)
\curveto(68.86748549,226.54723667)(68.80748555,226.5122367)(68.74749146,226.47223694)
\curveto(68.66748569,226.42223679)(68.57248578,226.38223683)(68.46249146,226.35223694)
\curveto(68.352486,226.32223689)(68.24748611,226.29223692)(68.14749146,226.26223694)
\curveto(68.03748632,226.22223699)(67.92748643,226.19723702)(67.81749146,226.18723694)
\curveto(67.70748665,226.17723704)(67.59248676,226.16223705)(67.47249146,226.14223694)
\curveto(67.43248692,226.13223708)(67.38748697,226.13223708)(67.33749146,226.14223694)
\curveto(67.29748706,226.14223707)(67.2574871,226.13723708)(67.21749146,226.12723694)
\curveto(67.17748718,226.1172371)(67.12248723,226.1122371)(67.05249146,226.11223694)
\curveto(66.98248737,226.1122371)(66.93248742,226.1172371)(66.90249146,226.12723694)
\curveto(66.8524875,226.14723707)(66.80748755,226.15223706)(66.76749146,226.14223694)
\curveto(66.72748763,226.13223708)(66.69248766,226.13223708)(66.66249146,226.14223694)
\lineto(66.57249146,226.14223694)
\curveto(66.51248784,226.16223705)(66.44748791,226.17723704)(66.37749146,226.18723694)
\curveto(66.31748804,226.18723703)(66.2524881,226.19223702)(66.18249146,226.20223694)
\curveto(66.01248834,226.25223696)(65.8524885,226.30223691)(65.70249146,226.35223694)
\curveto(65.5524888,226.40223681)(65.40748895,226.46723675)(65.26749146,226.54723694)
\curveto(65.21748914,226.58723663)(65.16248919,226.6172366)(65.10249146,226.63723694)
\curveto(65.0524893,226.66723655)(65.00248935,226.70223651)(64.95249146,226.74223694)
\curveto(64.71248964,226.92223629)(64.51248984,227.14223607)(64.35249146,227.40223694)
\curveto(64.19249016,227.66223555)(64.0524903,227.94723527)(63.93249146,228.25723694)
\curveto(63.87249048,228.39723482)(63.82749053,228.53723468)(63.79749146,228.67723694)
\curveto(63.76749059,228.82723439)(63.73249062,228.98223423)(63.69249146,229.14223694)
\curveto(63.67249068,229.25223396)(63.6574907,229.36223385)(63.64749146,229.47223694)
\curveto(63.63749072,229.58223363)(63.62249073,229.69223352)(63.60249146,229.80223694)
\curveto(63.59249076,229.84223337)(63.58749077,229.88223333)(63.58749146,229.92223694)
\curveto(63.59749076,229.96223325)(63.59749076,230.00223321)(63.58749146,230.04223694)
\curveto(63.57749078,230.09223312)(63.57249078,230.14223307)(63.57249146,230.19223694)
\lineto(63.57249146,230.35723694)
\curveto(63.5524908,230.40723281)(63.54749081,230.45723276)(63.55749146,230.50723694)
\curveto(63.56749079,230.56723265)(63.56749079,230.62223259)(63.55749146,230.67223694)
\curveto(63.54749081,230.7122325)(63.54749081,230.75723246)(63.55749146,230.80723694)
\curveto(63.56749079,230.85723236)(63.56249079,230.90723231)(63.54249146,230.95723694)
\curveto(63.52249083,231.02723219)(63.51749084,231.10223211)(63.52749146,231.18223694)
\curveto(63.53749082,231.27223194)(63.54249081,231.35723186)(63.54249146,231.43723694)
\curveto(63.54249081,231.52723169)(63.53749082,231.62723159)(63.52749146,231.73723694)
\curveto(63.51749084,231.85723136)(63.52249083,231.95723126)(63.54249146,232.03723694)
\lineto(63.54249146,232.32223694)
\lineto(63.58749146,232.95223694)
\curveto(63.59749076,233.05223016)(63.60749075,233.14723007)(63.61749146,233.23723694)
\lineto(63.64749146,233.53723694)
\curveto(63.66749069,233.58722963)(63.67249068,233.63722958)(63.66249146,233.68723694)
\curveto(63.66249069,233.74722947)(63.67249068,233.80222941)(63.69249146,233.85223694)
\curveto(63.74249061,234.02222919)(63.78249057,234.18722903)(63.81249146,234.34723694)
\curveto(63.84249051,234.5172287)(63.89249046,234.67722854)(63.96249146,234.82723694)
\curveto(64.1524902,235.28722793)(64.37248998,235.66222755)(64.62249146,235.95223694)
\curveto(64.88248947,236.24222697)(65.24248911,236.48722673)(65.70249146,236.68723694)
\curveto(65.83248852,236.73722648)(65.96248839,236.77222644)(66.09249146,236.79223694)
\curveto(66.23248812,236.8122264)(66.37248798,236.83722638)(66.51249146,236.86723694)
\curveto(66.58248777,236.87722634)(66.64748771,236.88222633)(66.70749146,236.88223694)
\curveto(66.76748759,236.88222633)(66.83248752,236.88722633)(66.90249146,236.89723694)
\curveto(67.73248662,236.9172263)(68.40248595,236.76722645)(68.91249146,236.44723694)
\curveto(69.42248493,236.13722708)(69.80248455,235.69722752)(70.05249146,235.12723694)
\curveto(70.10248425,235.00722821)(70.14748421,234.88222833)(70.18749146,234.75223694)
\curveto(70.22748413,234.62222859)(70.27248408,234.48722873)(70.32249146,234.34723694)
\curveto(70.34248401,234.26722895)(70.357484,234.18222903)(70.36749146,234.09223694)
\lineto(70.42749146,233.85223694)
\curveto(70.4574839,233.74222947)(70.47248388,233.63222958)(70.47249146,233.52223694)
\curveto(70.48248387,233.4122298)(70.49748386,233.30222991)(70.51749146,233.19223694)
\curveto(70.53748382,233.14223007)(70.54248381,233.09723012)(70.53249146,233.05723694)
\curveto(70.53248382,233.0172302)(70.53748382,232.97723024)(70.54749146,232.93723694)
\curveto(70.5574838,232.88723033)(70.5574838,232.83223038)(70.54749146,232.77223694)
\curveto(70.54748381,232.72223049)(70.5524838,232.67223054)(70.56249146,232.62223694)
\lineto(70.56249146,232.48723694)
\curveto(70.58248377,232.42723079)(70.58248377,232.35723086)(70.56249146,232.27723694)
\curveto(70.5524838,232.20723101)(70.5574838,232.14223107)(70.57749146,232.08223694)
\curveto(70.58748377,232.05223116)(70.59248376,232.0122312)(70.59249146,231.96223694)
\lineto(70.59249146,231.84223694)
\lineto(70.59249146,231.37723694)
\moveto(69.04749146,229.05223694)
\curveto(69.14748521,229.37223384)(69.20748515,229.73723348)(69.22749146,230.14723694)
\curveto(69.24748511,230.55723266)(69.2574851,230.96723225)(69.25749146,231.37723694)
\curveto(69.2574851,231.80723141)(69.24748511,232.22723099)(69.22749146,232.63723694)
\curveto(69.20748515,233.04723017)(69.16248519,233.43222978)(69.09249146,233.79223694)
\curveto(69.02248533,234.15222906)(68.91248544,234.47222874)(68.76249146,234.75223694)
\curveto(68.62248573,235.04222817)(68.42748593,235.27722794)(68.17749146,235.45723694)
\curveto(68.01748634,235.56722765)(67.83748652,235.64722757)(67.63749146,235.69723694)
\curveto(67.43748692,235.75722746)(67.19248716,235.78722743)(66.90249146,235.78723694)
\curveto(66.88248747,235.76722745)(66.84748751,235.75722746)(66.79749146,235.75723694)
\curveto(66.74748761,235.76722745)(66.70748765,235.76722745)(66.67749146,235.75723694)
\curveto(66.59748776,235.73722748)(66.52248783,235.7172275)(66.45249146,235.69723694)
\curveto(66.39248796,235.68722753)(66.32748803,235.66722755)(66.25749146,235.63723694)
\curveto(65.98748837,235.5172277)(65.76748859,235.34722787)(65.59749146,235.12723694)
\curveto(65.43748892,234.9172283)(65.30248905,234.67222854)(65.19249146,234.39223694)
\curveto(65.14248921,234.28222893)(65.10248925,234.16222905)(65.07249146,234.03223694)
\curveto(65.0524893,233.9122293)(65.02748933,233.78722943)(64.99749146,233.65723694)
\curveto(64.97748938,233.60722961)(64.96748939,233.55222966)(64.96749146,233.49223694)
\curveto(64.96748939,233.44222977)(64.96248939,233.39222982)(64.95249146,233.34223694)
\curveto(64.94248941,233.25222996)(64.93248942,233.15723006)(64.92249146,233.05723694)
\curveto(64.91248944,232.96723025)(64.90248945,232.87223034)(64.89249146,232.77223694)
\curveto(64.89248946,232.69223052)(64.88748947,232.60723061)(64.87749146,232.51723694)
\lineto(64.87749146,232.27723694)
\lineto(64.87749146,232.09723694)
\curveto(64.86748949,232.06723115)(64.86248949,232.03223118)(64.86249146,231.99223694)
\lineto(64.86249146,231.85723694)
\lineto(64.86249146,231.40723694)
\curveto(64.86248949,231.32723189)(64.8574895,231.24223197)(64.84749146,231.15223694)
\curveto(64.84748951,231.07223214)(64.8574895,230.99723222)(64.87749146,230.92723694)
\lineto(64.87749146,230.65723694)
\curveto(64.87748948,230.63723258)(64.87248948,230.60723261)(64.86249146,230.56723694)
\curveto(64.86248949,230.53723268)(64.86748949,230.5122327)(64.87749146,230.49223694)
\curveto(64.88748947,230.39223282)(64.89248946,230.29223292)(64.89249146,230.19223694)
\curveto(64.90248945,230.10223311)(64.91248944,230.00223321)(64.92249146,229.89223694)
\curveto(64.9524894,229.77223344)(64.96748939,229.64723357)(64.96749146,229.51723694)
\curveto(64.97748938,229.39723382)(65.00248935,229.28223393)(65.04249146,229.17223694)
\curveto(65.12248923,228.87223434)(65.20748915,228.60723461)(65.29749146,228.37723694)
\curveto(65.39748896,228.14723507)(65.54248881,227.93223528)(65.73249146,227.73223694)
\curveto(65.94248841,227.53223568)(66.20748815,227.38223583)(66.52749146,227.28223694)
\curveto(66.56748779,227.26223595)(66.60248775,227.25223596)(66.63249146,227.25223694)
\curveto(66.67248768,227.26223595)(66.71748764,227.25723596)(66.76749146,227.23723694)
\curveto(66.80748755,227.22723599)(66.87748748,227.217236)(66.97749146,227.20723694)
\curveto(67.08748727,227.19723602)(67.17248718,227.20223601)(67.23249146,227.22223694)
\curveto(67.30248705,227.24223597)(67.37248698,227.25223596)(67.44249146,227.25223694)
\curveto(67.51248684,227.26223595)(67.57748678,227.27723594)(67.63749146,227.29723694)
\curveto(67.83748652,227.35723586)(68.01748634,227.44223577)(68.17749146,227.55223694)
\curveto(68.20748615,227.57223564)(68.23248612,227.59223562)(68.25249146,227.61223694)
\lineto(68.31249146,227.67223694)
\curveto(68.352486,227.69223552)(68.40248595,227.73223548)(68.46249146,227.79223694)
\curveto(68.56248579,227.93223528)(68.64748571,228.06223515)(68.71749146,228.18223694)
\curveto(68.78748557,228.30223491)(68.8574855,228.44723477)(68.92749146,228.61723694)
\curveto(68.9574854,228.68723453)(68.97748538,228.75723446)(68.98749146,228.82723694)
\curveto(69.00748535,228.89723432)(69.02748533,228.97223424)(69.04749146,229.05223694)
}
}
{
\newrgbcolor{curcolor}{0 0 0}
\pscustom[linestyle=none,fillstyle=solid,fillcolor=curcolor]
{
\newpath
\moveto(55.76287964,311.70223694)
\lineto(60.56287964,311.70223694)
\lineto(61.56787964,311.70223694)
\curveto(61.70787254,311.70222651)(61.82787242,311.69222652)(61.92787964,311.67223694)
\curveto(62.03787221,311.66222655)(62.11787213,311.6172266)(62.16787964,311.53723694)
\curveto(62.18787206,311.49722672)(62.19787205,311.44722677)(62.19787964,311.38723694)
\curveto(62.20787204,311.32722689)(62.21287203,311.26222695)(62.21287964,311.19223694)
\lineto(62.21287964,310.92223694)
\curveto(62.21287203,310.83222738)(62.20287204,310.75222746)(62.18287964,310.68223694)
\curveto(62.1428721,310.60222761)(62.09787215,310.53222768)(62.04787964,310.47223694)
\lineto(61.89787964,310.29223694)
\curveto(61.86787238,310.24222797)(61.83287241,310.20222801)(61.79287964,310.17223694)
\curveto(61.75287249,310.14222807)(61.71287253,310.10222811)(61.67287964,310.05223694)
\curveto(61.59287265,309.94222827)(61.50787274,309.83222838)(61.41787964,309.72223694)
\curveto(61.32787292,309.62222859)(61.242873,309.5172287)(61.16287964,309.40723694)
\curveto(61.02287322,309.20722901)(60.88287336,308.99722922)(60.74287964,308.77723694)
\curveto(60.60287364,308.56722965)(60.46287378,308.35222986)(60.32287964,308.13223694)
\curveto(60.27287397,308.04223017)(60.22287402,307.94723027)(60.17287964,307.84723694)
\curveto(60.12287412,307.74723047)(60.06787418,307.65223056)(60.00787964,307.56223694)
\curveto(59.98787426,307.54223067)(59.97787427,307.5172307)(59.97787964,307.48723694)
\curveto(59.97787427,307.45723076)(59.96787428,307.43223078)(59.94787964,307.41223694)
\curveto(59.87787437,307.3122309)(59.81287443,307.19723102)(59.75287964,307.06723694)
\curveto(59.69287455,306.94723127)(59.63787461,306.83223138)(59.58787964,306.72223694)
\curveto(59.48787476,306.49223172)(59.39287485,306.25723196)(59.30287964,306.01723694)
\curveto(59.21287503,305.77723244)(59.11287513,305.53723268)(59.00287964,305.29723694)
\curveto(58.98287526,305.24723297)(58.96787528,305.20223301)(58.95787964,305.16223694)
\curveto(58.95787529,305.12223309)(58.9478753,305.07723314)(58.92787964,305.02723694)
\curveto(58.87787537,304.90723331)(58.83287541,304.78223343)(58.79287964,304.65223694)
\curveto(58.76287548,304.53223368)(58.72787552,304.4122338)(58.68787964,304.29223694)
\curveto(58.60787564,304.06223415)(58.5428757,303.82223439)(58.49287964,303.57223694)
\curveto(58.45287579,303.33223488)(58.40287584,303.09223512)(58.34287964,302.85223694)
\curveto(58.30287594,302.70223551)(58.27787597,302.55223566)(58.26787964,302.40223694)
\curveto(58.25787599,302.25223596)(58.23787601,302.10223611)(58.20787964,301.95223694)
\curveto(58.19787605,301.9122363)(58.19287605,301.85223636)(58.19287964,301.77223694)
\curveto(58.16287608,301.65223656)(58.13287611,301.55223666)(58.10287964,301.47223694)
\curveto(58.07287617,301.39223682)(58.00287624,301.33723688)(57.89287964,301.30723694)
\curveto(57.8428764,301.28723693)(57.78787646,301.27723694)(57.72787964,301.27723694)
\lineto(57.53287964,301.27723694)
\curveto(57.39287685,301.27723694)(57.25287699,301.28223693)(57.11287964,301.29223694)
\curveto(56.98287726,301.30223691)(56.88787736,301.34723687)(56.82787964,301.42723694)
\curveto(56.78787746,301.48723673)(56.76787748,301.57223664)(56.76787964,301.68223694)
\curveto(56.77787747,301.79223642)(56.79287745,301.88723633)(56.81287964,301.96723694)
\lineto(56.81287964,302.04223694)
\curveto(56.82287742,302.07223614)(56.82787742,302.10223611)(56.82787964,302.13223694)
\curveto(56.8478774,302.212236)(56.85787739,302.28723593)(56.85787964,302.35723694)
\curveto(56.85787739,302.42723579)(56.86787738,302.49723572)(56.88787964,302.56723694)
\curveto(56.93787731,302.75723546)(56.97787727,302.94223527)(57.00787964,303.12223694)
\curveto(57.03787721,303.3122349)(57.07787717,303.49223472)(57.12787964,303.66223694)
\curveto(57.1478771,303.7122345)(57.15787709,303.75223446)(57.15787964,303.78223694)
\curveto(57.15787709,303.8122344)(57.16287708,303.84723437)(57.17287964,303.88723694)
\curveto(57.27287697,304.18723403)(57.36287688,304.48223373)(57.44287964,304.77223694)
\curveto(57.53287671,305.06223315)(57.63787661,305.34223287)(57.75787964,305.61223694)
\curveto(58.01787623,306.19223202)(58.28787596,306.74223147)(58.56787964,307.26223694)
\curveto(58.8478754,307.79223042)(59.15787509,308.29722992)(59.49787964,308.77723694)
\curveto(59.63787461,308.97722924)(59.78787446,309.16722905)(59.94787964,309.34723694)
\curveto(60.10787414,309.53722868)(60.25787399,309.72722849)(60.39787964,309.91723694)
\curveto(60.43787381,309.96722825)(60.47287377,310.0122282)(60.50287964,310.05223694)
\curveto(60.5428737,310.10222811)(60.57787367,310.15222806)(60.60787964,310.20223694)
\curveto(60.61787363,310.22222799)(60.62787362,310.24722797)(60.63787964,310.27723694)
\curveto(60.65787359,310.30722791)(60.65787359,310.33722788)(60.63787964,310.36723694)
\curveto(60.61787363,310.42722779)(60.58287366,310.46222775)(60.53287964,310.47223694)
\curveto(60.48287376,310.49222772)(60.43287381,310.5122277)(60.38287964,310.53223694)
\lineto(60.27787964,310.53223694)
\curveto(60.23787401,310.54222767)(60.18787406,310.54222767)(60.12787964,310.53223694)
\lineto(59.97787964,310.53223694)
\lineto(59.37787964,310.53223694)
\lineto(56.73787964,310.53223694)
\lineto(56.00287964,310.53223694)
\lineto(55.76287964,310.53223694)
\curveto(55.69287855,310.54222767)(55.63287861,310.55722766)(55.58287964,310.57723694)
\curveto(55.49287875,310.6172276)(55.43287881,310.67722754)(55.40287964,310.75723694)
\curveto(55.35287889,310.85722736)(55.33787891,311.00222721)(55.35787964,311.19223694)
\curveto(55.37787887,311.39222682)(55.41287883,311.52722669)(55.46287964,311.59723694)
\curveto(55.48287876,311.6172266)(55.50787874,311.63222658)(55.53787964,311.64223694)
\lineto(55.65787964,311.70223694)
\curveto(55.67787857,311.70222651)(55.69287855,311.69722652)(55.70287964,311.68723694)
\curveto(55.72287852,311.68722653)(55.7428785,311.69222652)(55.76287964,311.70223694)
}
}
{
\newrgbcolor{curcolor}{0 0 0}
\pscustom[linestyle=none,fillstyle=solid,fillcolor=curcolor]
{
\newpath
\moveto(65.16248901,311.70223694)
\lineto(68.76248901,311.70223694)
\lineto(69.40748901,311.70223694)
\curveto(69.48748248,311.70222651)(69.56248241,311.69722652)(69.63248901,311.68723694)
\curveto(69.70248227,311.68722653)(69.76248221,311.67722654)(69.81248901,311.65723694)
\curveto(69.88248209,311.62722659)(69.93748203,311.56722665)(69.97748901,311.47723694)
\curveto(69.99748197,311.44722677)(70.00748196,311.40722681)(70.00748901,311.35723694)
\lineto(70.00748901,311.22223694)
\curveto(70.01748195,311.1122271)(70.01248196,311.00722721)(69.99248901,310.90723694)
\curveto(69.98248199,310.80722741)(69.94748202,310.73722748)(69.88748901,310.69723694)
\curveto(69.79748217,310.62722759)(69.66248231,310.59222762)(69.48248901,310.59223694)
\curveto(69.30248267,310.60222761)(69.13748283,310.60722761)(68.98748901,310.60723694)
\lineto(66.99248901,310.60723694)
\lineto(66.49748901,310.60723694)
\lineto(66.36248901,310.60723694)
\curveto(66.32248565,310.60722761)(66.28248569,310.60222761)(66.24248901,310.59223694)
\lineto(66.03248901,310.59223694)
\curveto(65.92248605,310.56222765)(65.84248613,310.52222769)(65.79248901,310.47223694)
\curveto(65.74248623,310.43222778)(65.70748626,310.37722784)(65.68748901,310.30723694)
\curveto(65.6674863,310.24722797)(65.65248632,310.17722804)(65.64248901,310.09723694)
\curveto(65.63248634,310.0172282)(65.61248636,309.92722829)(65.58248901,309.82723694)
\curveto(65.53248644,309.62722859)(65.49248648,309.42222879)(65.46248901,309.21223694)
\curveto(65.43248654,309.00222921)(65.39248658,308.79722942)(65.34248901,308.59723694)
\curveto(65.32248665,308.52722969)(65.31248666,308.45722976)(65.31248901,308.38723694)
\curveto(65.31248666,308.32722989)(65.30248667,308.26222995)(65.28248901,308.19223694)
\curveto(65.2724867,308.16223005)(65.26248671,308.12223009)(65.25248901,308.07223694)
\curveto(65.25248672,308.03223018)(65.25748671,307.99223022)(65.26748901,307.95223694)
\curveto(65.28748668,307.90223031)(65.31248666,307.85723036)(65.34248901,307.81723694)
\curveto(65.38248659,307.78723043)(65.44248653,307.78223043)(65.52248901,307.80223694)
\curveto(65.58248639,307.82223039)(65.64248633,307.84723037)(65.70248901,307.87723694)
\curveto(65.76248621,307.9172303)(65.82248615,307.95223026)(65.88248901,307.98223694)
\curveto(65.94248603,308.00223021)(65.99248598,308.0172302)(66.03248901,308.02723694)
\curveto(66.22248575,308.10723011)(66.42748554,308.16223005)(66.64748901,308.19223694)
\curveto(66.87748509,308.22222999)(67.10748486,308.23222998)(67.33748901,308.22223694)
\curveto(67.57748439,308.22222999)(67.80748416,308.19723002)(68.02748901,308.14723694)
\curveto(68.24748372,308.10723011)(68.44748352,308.04723017)(68.62748901,307.96723694)
\curveto(68.67748329,307.94723027)(68.72248325,307.92723029)(68.76248901,307.90723694)
\curveto(68.81248316,307.88723033)(68.86248311,307.86223035)(68.91248901,307.83223694)
\curveto(69.26248271,307.62223059)(69.54248243,307.39223082)(69.75248901,307.14223694)
\curveto(69.972482,306.89223132)(70.1674818,306.56723165)(70.33748901,306.16723694)
\curveto(70.38748158,306.05723216)(70.42248155,305.94723227)(70.44248901,305.83723694)
\curveto(70.46248151,305.72723249)(70.48748148,305.6122326)(70.51748901,305.49223694)
\curveto(70.52748144,305.46223275)(70.53248144,305.4172328)(70.53248901,305.35723694)
\curveto(70.55248142,305.29723292)(70.56248141,305.22723299)(70.56248901,305.14723694)
\curveto(70.56248141,305.07723314)(70.5724814,305.0122332)(70.59248901,304.95223694)
\lineto(70.59248901,304.78723694)
\curveto(70.60248137,304.73723348)(70.60748136,304.66723355)(70.60748901,304.57723694)
\curveto(70.60748136,304.48723373)(70.59748137,304.4172338)(70.57748901,304.36723694)
\curveto(70.55748141,304.30723391)(70.55248142,304.24723397)(70.56248901,304.18723694)
\curveto(70.5724814,304.13723408)(70.5674814,304.08723413)(70.54748901,304.03723694)
\curveto(70.50748146,303.87723434)(70.4724815,303.72723449)(70.44248901,303.58723694)
\curveto(70.41248156,303.44723477)(70.3674816,303.3122349)(70.30748901,303.18223694)
\curveto(70.14748182,302.8122354)(69.92748204,302.47723574)(69.64748901,302.17723694)
\curveto(69.3674826,301.87723634)(69.04748292,301.64723657)(68.68748901,301.48723694)
\curveto(68.51748345,301.40723681)(68.31748365,301.33223688)(68.08748901,301.26223694)
\curveto(67.97748399,301.22223699)(67.86248411,301.19723702)(67.74248901,301.18723694)
\curveto(67.62248435,301.17723704)(67.50248447,301.15723706)(67.38248901,301.12723694)
\curveto(67.33248464,301.10723711)(67.27748469,301.10723711)(67.21748901,301.12723694)
\curveto(67.15748481,301.13723708)(67.09748487,301.13223708)(67.03748901,301.11223694)
\curveto(66.93748503,301.09223712)(66.83748513,301.09223712)(66.73748901,301.11223694)
\lineto(66.60248901,301.11223694)
\curveto(66.55248542,301.13223708)(66.49248548,301.14223707)(66.42248901,301.14223694)
\curveto(66.36248561,301.13223708)(66.30748566,301.13723708)(66.25748901,301.15723694)
\curveto(66.21748575,301.16723705)(66.18248579,301.17223704)(66.15248901,301.17223694)
\curveto(66.12248585,301.17223704)(66.08748588,301.17723704)(66.04748901,301.18723694)
\lineto(65.77748901,301.24723694)
\curveto(65.68748628,301.26723695)(65.60248637,301.29723692)(65.52248901,301.33723694)
\curveto(65.18248679,301.47723674)(64.89248708,301.63223658)(64.65248901,301.80223694)
\curveto(64.41248756,301.98223623)(64.19248778,302.212236)(63.99248901,302.49223694)
\curveto(63.84248813,302.72223549)(63.72748824,302.96223525)(63.64748901,303.21223694)
\curveto(63.62748834,303.26223495)(63.61748835,303.30723491)(63.61748901,303.34723694)
\curveto(63.61748835,303.39723482)(63.60748836,303.44723477)(63.58748901,303.49723694)
\curveto(63.5674884,303.55723466)(63.55248842,303.63723458)(63.54248901,303.73723694)
\curveto(63.54248843,303.83723438)(63.56248841,303.9122343)(63.60248901,303.96223694)
\curveto(63.65248832,304.04223417)(63.73248824,304.08723413)(63.84248901,304.09723694)
\curveto(63.95248802,304.10723411)(64.0674879,304.1122341)(64.18748901,304.11223694)
\lineto(64.35248901,304.11223694)
\curveto(64.41248756,304.1122341)(64.4674875,304.10223411)(64.51748901,304.08223694)
\curveto(64.60748736,304.06223415)(64.67748729,304.02223419)(64.72748901,303.96223694)
\curveto(64.79748717,303.87223434)(64.84248713,303.76223445)(64.86248901,303.63223694)
\curveto(64.89248708,303.5122347)(64.93748703,303.40723481)(64.99748901,303.31723694)
\curveto(65.18748678,302.97723524)(65.44748652,302.70723551)(65.77748901,302.50723694)
\curveto(65.87748609,302.44723577)(65.98248599,302.39723582)(66.09248901,302.35723694)
\curveto(66.21248576,302.32723589)(66.33248564,302.29223592)(66.45248901,302.25223694)
\curveto(66.62248535,302.20223601)(66.82748514,302.18223603)(67.06748901,302.19223694)
\curveto(67.31748465,302.212236)(67.51748445,302.24723597)(67.66748901,302.29723694)
\curveto(68.03748393,302.4172358)(68.32748364,302.57723564)(68.53748901,302.77723694)
\curveto(68.75748321,302.98723523)(68.93748303,303.26723495)(69.07748901,303.61723694)
\curveto(69.12748284,303.7172345)(69.15748281,303.82223439)(69.16748901,303.93223694)
\curveto(69.18748278,304.04223417)(69.21248276,304.15723406)(69.24248901,304.27723694)
\lineto(69.24248901,304.38223694)
\curveto(69.25248272,304.42223379)(69.25748271,304.46223375)(69.25748901,304.50223694)
\curveto(69.2674827,304.53223368)(69.2674827,304.56723365)(69.25748901,304.60723694)
\lineto(69.25748901,304.72723694)
\curveto(69.25748271,304.98723323)(69.22748274,305.23223298)(69.16748901,305.46223694)
\curveto(69.05748291,305.8122324)(68.90248307,306.10723211)(68.70248901,306.34723694)
\curveto(68.50248347,306.59723162)(68.24248373,306.79223142)(67.92248901,306.93223694)
\lineto(67.74248901,306.99223694)
\curveto(67.69248428,307.0122312)(67.63248434,307.03223118)(67.56248901,307.05223694)
\curveto(67.51248446,307.07223114)(67.45248452,307.08223113)(67.38248901,307.08223694)
\curveto(67.32248465,307.09223112)(67.25748471,307.10723111)(67.18748901,307.12723694)
\lineto(67.03748901,307.12723694)
\curveto(66.99748497,307.14723107)(66.94248503,307.15723106)(66.87248901,307.15723694)
\curveto(66.81248516,307.15723106)(66.75748521,307.14723107)(66.70748901,307.12723694)
\lineto(66.60248901,307.12723694)
\curveto(66.5724854,307.12723109)(66.53748543,307.12223109)(66.49748901,307.11223694)
\lineto(66.25748901,307.05223694)
\curveto(66.17748579,307.04223117)(66.09748587,307.02223119)(66.01748901,306.99223694)
\curveto(65.77748619,306.89223132)(65.54748642,306.75723146)(65.32748901,306.58723694)
\curveto(65.23748673,306.5172317)(65.15248682,306.44223177)(65.07248901,306.36223694)
\curveto(64.99248698,306.29223192)(64.89248708,306.23723198)(64.77248901,306.19723694)
\curveto(64.68248729,306.16723205)(64.54248743,306.15723206)(64.35248901,306.16723694)
\curveto(64.1724878,306.17723204)(64.05248792,306.20223201)(63.99248901,306.24223694)
\curveto(63.94248803,306.28223193)(63.90248807,306.34223187)(63.87248901,306.42223694)
\curveto(63.85248812,306.50223171)(63.85248812,306.58723163)(63.87248901,306.67723694)
\curveto(63.90248807,306.79723142)(63.92248805,306.9172313)(63.93248901,307.03723694)
\curveto(63.95248802,307.16723105)(63.97748799,307.29223092)(64.00748901,307.41223694)
\curveto(64.02748794,307.45223076)(64.03248794,307.48723073)(64.02248901,307.51723694)
\curveto(64.02248795,307.55723066)(64.03248794,307.60223061)(64.05248901,307.65223694)
\curveto(64.0724879,307.74223047)(64.08748788,307.83223038)(64.09748901,307.92223694)
\curveto(64.10748786,308.02223019)(64.12748784,308.1172301)(64.15748901,308.20723694)
\curveto(64.1674878,308.26722995)(64.1724878,308.32722989)(64.17248901,308.38723694)
\curveto(64.18248779,308.44722977)(64.19748777,308.50722971)(64.21748901,308.56723694)
\curveto(64.2674877,308.76722945)(64.30248767,308.97222924)(64.32248901,309.18223694)
\curveto(64.35248762,309.40222881)(64.39248758,309.6122286)(64.44248901,309.81223694)
\curveto(64.4724875,309.9122283)(64.49248748,310.0122282)(64.50248901,310.11223694)
\curveto(64.51248746,310.212228)(64.52748744,310.3122279)(64.54748901,310.41223694)
\curveto(64.55748741,310.44222777)(64.56248741,310.48222773)(64.56248901,310.53223694)
\curveto(64.59248738,310.64222757)(64.61248736,310.74722747)(64.62248901,310.84723694)
\curveto(64.64248733,310.95722726)(64.6674873,311.06722715)(64.69748901,311.17723694)
\curveto(64.71748725,311.25722696)(64.73248724,311.32722689)(64.74248901,311.38723694)
\curveto(64.75248722,311.45722676)(64.77748719,311.5172267)(64.81748901,311.56723694)
\curveto(64.83748713,311.59722662)(64.8674871,311.6172266)(64.90748901,311.62723694)
\curveto(64.94748702,311.64722657)(64.99248698,311.66722655)(65.04248901,311.68723694)
\curveto(65.10248687,311.68722653)(65.14248683,311.69222652)(65.16248901,311.70223694)
}
}
{
\newrgbcolor{curcolor}{0 0 0}
\pscustom[linestyle=none,fillstyle=solid,fillcolor=curcolor]
{
\newpath
\moveto(50.95327271,386.18295013)
\curveto(51.05326785,386.18293951)(51.14826776,386.17293952)(51.23827271,386.15295013)
\curveto(51.32826758,386.14293955)(51.39326751,386.11293958)(51.43327271,386.06295013)
\curveto(51.49326741,385.98293971)(51.52326738,385.87793982)(51.52327271,385.74795013)
\lineto(51.52327271,385.35795013)
\lineto(51.52327271,383.85795013)
\lineto(51.52327271,377.46795013)
\lineto(51.52327271,376.29795013)
\lineto(51.52327271,375.98295013)
\curveto(51.53326737,375.88294981)(51.51826739,375.80294989)(51.47827271,375.74295013)
\curveto(51.42826748,375.66295003)(51.35326755,375.61295008)(51.25327271,375.59295013)
\curveto(51.16326774,375.58295011)(51.05326785,375.57795012)(50.92327271,375.57795013)
\lineto(50.69827271,375.57795013)
\curveto(50.61826829,375.5979501)(50.54826836,375.61295008)(50.48827271,375.62295013)
\curveto(50.42826848,375.64295005)(50.37826853,375.68295001)(50.33827271,375.74295013)
\curveto(50.29826861,375.80294989)(50.27826863,375.87794982)(50.27827271,375.96795013)
\lineto(50.27827271,376.26795013)
\lineto(50.27827271,377.36295013)
\lineto(50.27827271,382.70295013)
\curveto(50.25826865,382.7929429)(50.24326866,382.86794283)(50.23327271,382.92795013)
\curveto(50.23326867,382.9979427)(50.2032687,383.05794264)(50.14327271,383.10795013)
\curveto(50.07326883,383.15794254)(49.98326892,383.18294251)(49.87327271,383.18295013)
\curveto(49.77326913,383.1929425)(49.66326924,383.1979425)(49.54327271,383.19795013)
\lineto(48.40327271,383.19795013)
\lineto(47.90827271,383.19795013)
\curveto(47.74827116,383.20794249)(47.63827127,383.26794243)(47.57827271,383.37795013)
\curveto(47.55827135,383.40794229)(47.54827136,383.43794226)(47.54827271,383.46795013)
\curveto(47.54827136,383.50794219)(47.54327136,383.55294214)(47.53327271,383.60295013)
\curveto(47.51327139,383.72294197)(47.51827139,383.83294186)(47.54827271,383.93295013)
\curveto(47.58827132,384.03294166)(47.64327126,384.10294159)(47.71327271,384.14295013)
\curveto(47.79327111,384.1929415)(47.91327099,384.21794148)(48.07327271,384.21795013)
\curveto(48.23327067,384.21794148)(48.36827054,384.23294146)(48.47827271,384.26295013)
\curveto(48.52827038,384.27294142)(48.58327032,384.27794142)(48.64327271,384.27795013)
\curveto(48.7032702,384.28794141)(48.76327014,384.30294139)(48.82327271,384.32295013)
\curveto(48.97326993,384.37294132)(49.11826979,384.42294127)(49.25827271,384.47295013)
\curveto(49.39826951,384.53294116)(49.53326937,384.60294109)(49.66327271,384.68295013)
\curveto(49.8032691,384.77294092)(49.92326898,384.87794082)(50.02327271,384.99795013)
\curveto(50.12326878,385.11794058)(50.21826869,385.24794045)(50.30827271,385.38795013)
\curveto(50.36826854,385.48794021)(50.41326849,385.5979401)(50.44327271,385.71795013)
\curveto(50.48326842,385.83793986)(50.53326837,385.94293975)(50.59327271,386.03295013)
\curveto(50.64326826,386.0929396)(50.71326819,386.13293956)(50.80327271,386.15295013)
\curveto(50.82326808,386.16293953)(50.84826806,386.16793953)(50.87827271,386.16795013)
\curveto(50.908268,386.16793953)(50.93326797,386.17293952)(50.95327271,386.18295013)
}
}
{
\newrgbcolor{curcolor}{0 0 0}
\pscustom[linestyle=none,fillstyle=solid,fillcolor=curcolor]
{
\newpath
\moveto(62.24288208,380.66295013)
\lineto(62.24288208,380.40795013)
\curveto(62.25287438,380.32794537)(62.24787438,380.25294544)(62.22788208,380.18295013)
\lineto(62.22788208,379.94295013)
\lineto(62.22788208,379.77795013)
\curveto(62.20787442,379.67794602)(62.19787443,379.57294612)(62.19788208,379.46295013)
\curveto(62.19787443,379.36294633)(62.18787444,379.26294643)(62.16788208,379.16295013)
\lineto(62.16788208,379.01295013)
\curveto(62.13787449,378.87294682)(62.11787451,378.73294696)(62.10788208,378.59295013)
\curveto(62.09787453,378.46294723)(62.07287456,378.33294736)(62.03288208,378.20295013)
\curveto(62.01287462,378.12294757)(61.99287464,378.03794766)(61.97288208,377.94795013)
\lineto(61.91288208,377.70795013)
\lineto(61.79288208,377.40795013)
\curveto(61.76287487,377.31794838)(61.7278749,377.22794847)(61.68788208,377.13795013)
\curveto(61.58787504,376.91794878)(61.45287518,376.70294899)(61.28288208,376.49295013)
\curveto(61.12287551,376.28294941)(60.94787568,376.11294958)(60.75788208,375.98295013)
\curveto(60.70787592,375.94294975)(60.64787598,375.90294979)(60.57788208,375.86295013)
\curveto(60.51787611,375.83294986)(60.45787617,375.7979499)(60.39788208,375.75795013)
\curveto(60.31787631,375.70794999)(60.22287641,375.66795003)(60.11288208,375.63795013)
\curveto(60.00287663,375.60795009)(59.89787673,375.57795012)(59.79788208,375.54795013)
\curveto(59.68787694,375.50795019)(59.57787705,375.48295021)(59.46788208,375.47295013)
\curveto(59.35787727,375.46295023)(59.24287739,375.44795025)(59.12288208,375.42795013)
\curveto(59.08287755,375.41795028)(59.03787759,375.41795028)(58.98788208,375.42795013)
\curveto(58.94787768,375.42795027)(58.90787772,375.42295027)(58.86788208,375.41295013)
\curveto(58.8278778,375.40295029)(58.77287786,375.3979503)(58.70288208,375.39795013)
\curveto(58.632878,375.3979503)(58.58287805,375.40295029)(58.55288208,375.41295013)
\curveto(58.50287813,375.43295026)(58.45787817,375.43795026)(58.41788208,375.42795013)
\curveto(58.37787825,375.41795028)(58.34287829,375.41795028)(58.31288208,375.42795013)
\lineto(58.22288208,375.42795013)
\curveto(58.16287847,375.44795025)(58.09787853,375.46295023)(58.02788208,375.47295013)
\curveto(57.96787866,375.47295022)(57.90287873,375.47795022)(57.83288208,375.48795013)
\curveto(57.66287897,375.53795016)(57.50287913,375.58795011)(57.35288208,375.63795013)
\curveto(57.20287943,375.68795001)(57.05787957,375.75294994)(56.91788208,375.83295013)
\curveto(56.86787976,375.87294982)(56.81287982,375.90294979)(56.75288208,375.92295013)
\curveto(56.70287993,375.95294974)(56.65287998,375.98794971)(56.60288208,376.02795013)
\curveto(56.36288027,376.20794949)(56.16288047,376.42794927)(56.00288208,376.68795013)
\curveto(55.84288079,376.94794875)(55.70288093,377.23294846)(55.58288208,377.54295013)
\curveto(55.52288111,377.68294801)(55.47788115,377.82294787)(55.44788208,377.96295013)
\curveto(55.41788121,378.11294758)(55.38288125,378.26794743)(55.34288208,378.42795013)
\curveto(55.32288131,378.53794716)(55.30788132,378.64794705)(55.29788208,378.75795013)
\curveto(55.28788134,378.86794683)(55.27288136,378.97794672)(55.25288208,379.08795013)
\curveto(55.24288139,379.12794657)(55.23788139,379.16794653)(55.23788208,379.20795013)
\curveto(55.24788138,379.24794645)(55.24788138,379.28794641)(55.23788208,379.32795013)
\curveto(55.2278814,379.37794632)(55.22288141,379.42794627)(55.22288208,379.47795013)
\lineto(55.22288208,379.64295013)
\curveto(55.20288143,379.692946)(55.19788143,379.74294595)(55.20788208,379.79295013)
\curveto(55.21788141,379.85294584)(55.21788141,379.90794579)(55.20788208,379.95795013)
\curveto(55.19788143,379.9979457)(55.19788143,380.04294565)(55.20788208,380.09295013)
\curveto(55.21788141,380.14294555)(55.21288142,380.1929455)(55.19288208,380.24295013)
\curveto(55.17288146,380.31294538)(55.16788146,380.38794531)(55.17788208,380.46795013)
\curveto(55.18788144,380.55794514)(55.19288144,380.64294505)(55.19288208,380.72295013)
\curveto(55.19288144,380.81294488)(55.18788144,380.91294478)(55.17788208,381.02295013)
\curveto(55.16788146,381.14294455)(55.17288146,381.24294445)(55.19288208,381.32295013)
\lineto(55.19288208,381.60795013)
\lineto(55.23788208,382.23795013)
\curveto(55.24788138,382.33794336)(55.25788137,382.43294326)(55.26788208,382.52295013)
\lineto(55.29788208,382.82295013)
\curveto(55.31788131,382.87294282)(55.32288131,382.92294277)(55.31288208,382.97295013)
\curveto(55.31288132,383.03294266)(55.32288131,383.08794261)(55.34288208,383.13795013)
\curveto(55.39288124,383.30794239)(55.4328812,383.47294222)(55.46288208,383.63295013)
\curveto(55.49288114,383.80294189)(55.54288109,383.96294173)(55.61288208,384.11295013)
\curveto(55.80288083,384.57294112)(56.02288061,384.94794075)(56.27288208,385.23795013)
\curveto(56.5328801,385.52794017)(56.89287974,385.77293992)(57.35288208,385.97295013)
\curveto(57.48287915,386.02293967)(57.61287902,386.05793964)(57.74288208,386.07795013)
\curveto(57.88287875,386.0979396)(58.02287861,386.12293957)(58.16288208,386.15295013)
\curveto(58.2328784,386.16293953)(58.29787833,386.16793953)(58.35788208,386.16795013)
\curveto(58.41787821,386.16793953)(58.48287815,386.17293952)(58.55288208,386.18295013)
\curveto(59.38287725,386.20293949)(60.05287658,386.05293964)(60.56288208,385.73295013)
\curveto(61.07287556,385.42294027)(61.45287518,384.98294071)(61.70288208,384.41295013)
\curveto(61.75287488,384.2929414)(61.79787483,384.16794153)(61.83788208,384.03795013)
\curveto(61.87787475,383.90794179)(61.92287471,383.77294192)(61.97288208,383.63295013)
\curveto(61.99287464,383.55294214)(62.00787462,383.46794223)(62.01788208,383.37795013)
\lineto(62.07788208,383.13795013)
\curveto(62.10787452,383.02794267)(62.12287451,382.91794278)(62.12288208,382.80795013)
\curveto(62.1328745,382.697943)(62.14787448,382.58794311)(62.16788208,382.47795013)
\curveto(62.18787444,382.42794327)(62.19287444,382.38294331)(62.18288208,382.34295013)
\curveto(62.18287445,382.30294339)(62.18787444,382.26294343)(62.19788208,382.22295013)
\curveto(62.20787442,382.17294352)(62.20787442,382.11794358)(62.19788208,382.05795013)
\curveto(62.19787443,382.00794369)(62.20287443,381.95794374)(62.21288208,381.90795013)
\lineto(62.21288208,381.77295013)
\curveto(62.2328744,381.71294398)(62.2328744,381.64294405)(62.21288208,381.56295013)
\curveto(62.20287443,381.4929442)(62.20787442,381.42794427)(62.22788208,381.36795013)
\curveto(62.23787439,381.33794436)(62.24287439,381.2979444)(62.24288208,381.24795013)
\lineto(62.24288208,381.12795013)
\lineto(62.24288208,380.66295013)
\moveto(60.69788208,378.33795013)
\curveto(60.79787583,378.65794704)(60.85787577,379.02294667)(60.87788208,379.43295013)
\curveto(60.89787573,379.84294585)(60.90787572,380.25294544)(60.90788208,380.66295013)
\curveto(60.90787572,381.0929446)(60.89787573,381.51294418)(60.87788208,381.92295013)
\curveto(60.85787577,382.33294336)(60.81287582,382.71794298)(60.74288208,383.07795013)
\curveto(60.67287596,383.43794226)(60.56287607,383.75794194)(60.41288208,384.03795013)
\curveto(60.27287636,384.32794137)(60.07787655,384.56294113)(59.82788208,384.74295013)
\curveto(59.66787696,384.85294084)(59.48787714,384.93294076)(59.28788208,384.98295013)
\curveto(59.08787754,385.04294065)(58.84287779,385.07294062)(58.55288208,385.07295013)
\curveto(58.5328781,385.05294064)(58.49787813,385.04294065)(58.44788208,385.04295013)
\curveto(58.39787823,385.05294064)(58.35787827,385.05294064)(58.32788208,385.04295013)
\curveto(58.24787838,385.02294067)(58.17287846,385.00294069)(58.10288208,384.98295013)
\curveto(58.04287859,384.97294072)(57.97787865,384.95294074)(57.90788208,384.92295013)
\curveto(57.63787899,384.80294089)(57.41787921,384.63294106)(57.24788208,384.41295013)
\curveto(57.08787954,384.20294149)(56.95287968,383.95794174)(56.84288208,383.67795013)
\curveto(56.79287984,383.56794213)(56.75287988,383.44794225)(56.72288208,383.31795013)
\curveto(56.70287993,383.1979425)(56.67787995,383.07294262)(56.64788208,382.94295013)
\curveto(56.62788,382.8929428)(56.61788001,382.83794286)(56.61788208,382.77795013)
\curveto(56.61788001,382.72794297)(56.61288002,382.67794302)(56.60288208,382.62795013)
\curveto(56.59288004,382.53794316)(56.58288005,382.44294325)(56.57288208,382.34295013)
\curveto(56.56288007,382.25294344)(56.55288008,382.15794354)(56.54288208,382.05795013)
\curveto(56.54288009,381.97794372)(56.53788009,381.8929438)(56.52788208,381.80295013)
\lineto(56.52788208,381.56295013)
\lineto(56.52788208,381.38295013)
\curveto(56.51788011,381.35294434)(56.51288012,381.31794438)(56.51288208,381.27795013)
\lineto(56.51288208,381.14295013)
\lineto(56.51288208,380.69295013)
\curveto(56.51288012,380.61294508)(56.50788012,380.52794517)(56.49788208,380.43795013)
\curveto(56.49788013,380.35794534)(56.50788012,380.28294541)(56.52788208,380.21295013)
\lineto(56.52788208,379.94295013)
\curveto(56.5278801,379.92294577)(56.52288011,379.8929458)(56.51288208,379.85295013)
\curveto(56.51288012,379.82294587)(56.51788011,379.7979459)(56.52788208,379.77795013)
\curveto(56.53788009,379.67794602)(56.54288009,379.57794612)(56.54288208,379.47795013)
\curveto(56.55288008,379.38794631)(56.56288007,379.28794641)(56.57288208,379.17795013)
\curveto(56.60288003,379.05794664)(56.61788001,378.93294676)(56.61788208,378.80295013)
\curveto(56.62788,378.68294701)(56.65287998,378.56794713)(56.69288208,378.45795013)
\curveto(56.77287986,378.15794754)(56.85787977,377.8929478)(56.94788208,377.66295013)
\curveto(57.04787958,377.43294826)(57.19287944,377.21794848)(57.38288208,377.01795013)
\curveto(57.59287904,376.81794888)(57.85787877,376.66794903)(58.17788208,376.56795013)
\curveto(58.21787841,376.54794915)(58.25287838,376.53794916)(58.28288208,376.53795013)
\curveto(58.32287831,376.54794915)(58.36787826,376.54294915)(58.41788208,376.52295013)
\curveto(58.45787817,376.51294918)(58.5278781,376.50294919)(58.62788208,376.49295013)
\curveto(58.73787789,376.48294921)(58.82287781,376.48794921)(58.88288208,376.50795013)
\curveto(58.95287768,376.52794917)(59.02287761,376.53794916)(59.09288208,376.53795013)
\curveto(59.16287747,376.54794915)(59.2278774,376.56294913)(59.28788208,376.58295013)
\curveto(59.48787714,376.64294905)(59.66787696,376.72794897)(59.82788208,376.83795013)
\curveto(59.85787677,376.85794884)(59.88287675,376.87794882)(59.90288208,376.89795013)
\lineto(59.96288208,376.95795013)
\curveto(60.00287663,376.97794872)(60.05287658,377.01794868)(60.11288208,377.07795013)
\curveto(60.21287642,377.21794848)(60.29787633,377.34794835)(60.36788208,377.46795013)
\curveto(60.43787619,377.58794811)(60.50787612,377.73294796)(60.57788208,377.90295013)
\curveto(60.60787602,377.97294772)(60.627876,378.04294765)(60.63788208,378.11295013)
\curveto(60.65787597,378.18294751)(60.67787595,378.25794744)(60.69788208,378.33795013)
}
}
{
\newrgbcolor{curcolor}{0 0 0}
\pscustom[linestyle=none,fillstyle=solid,fillcolor=curcolor]
{
\newpath
\moveto(70.59249146,380.66295013)
\lineto(70.59249146,380.40795013)
\curveto(70.60248375,380.32794537)(70.59748376,380.25294544)(70.57749146,380.18295013)
\lineto(70.57749146,379.94295013)
\lineto(70.57749146,379.77795013)
\curveto(70.5574838,379.67794602)(70.54748381,379.57294612)(70.54749146,379.46295013)
\curveto(70.54748381,379.36294633)(70.53748382,379.26294643)(70.51749146,379.16295013)
\lineto(70.51749146,379.01295013)
\curveto(70.48748387,378.87294682)(70.46748389,378.73294696)(70.45749146,378.59295013)
\curveto(70.44748391,378.46294723)(70.42248393,378.33294736)(70.38249146,378.20295013)
\curveto(70.36248399,378.12294757)(70.34248401,378.03794766)(70.32249146,377.94795013)
\lineto(70.26249146,377.70795013)
\lineto(70.14249146,377.40795013)
\curveto(70.11248424,377.31794838)(70.07748428,377.22794847)(70.03749146,377.13795013)
\curveto(69.93748442,376.91794878)(69.80248455,376.70294899)(69.63249146,376.49295013)
\curveto(69.47248488,376.28294941)(69.29748506,376.11294958)(69.10749146,375.98295013)
\curveto(69.0574853,375.94294975)(68.99748536,375.90294979)(68.92749146,375.86295013)
\curveto(68.86748549,375.83294986)(68.80748555,375.7979499)(68.74749146,375.75795013)
\curveto(68.66748569,375.70794999)(68.57248578,375.66795003)(68.46249146,375.63795013)
\curveto(68.352486,375.60795009)(68.24748611,375.57795012)(68.14749146,375.54795013)
\curveto(68.03748632,375.50795019)(67.92748643,375.48295021)(67.81749146,375.47295013)
\curveto(67.70748665,375.46295023)(67.59248676,375.44795025)(67.47249146,375.42795013)
\curveto(67.43248692,375.41795028)(67.38748697,375.41795028)(67.33749146,375.42795013)
\curveto(67.29748706,375.42795027)(67.2574871,375.42295027)(67.21749146,375.41295013)
\curveto(67.17748718,375.40295029)(67.12248723,375.3979503)(67.05249146,375.39795013)
\curveto(66.98248737,375.3979503)(66.93248742,375.40295029)(66.90249146,375.41295013)
\curveto(66.8524875,375.43295026)(66.80748755,375.43795026)(66.76749146,375.42795013)
\curveto(66.72748763,375.41795028)(66.69248766,375.41795028)(66.66249146,375.42795013)
\lineto(66.57249146,375.42795013)
\curveto(66.51248784,375.44795025)(66.44748791,375.46295023)(66.37749146,375.47295013)
\curveto(66.31748804,375.47295022)(66.2524881,375.47795022)(66.18249146,375.48795013)
\curveto(66.01248834,375.53795016)(65.8524885,375.58795011)(65.70249146,375.63795013)
\curveto(65.5524888,375.68795001)(65.40748895,375.75294994)(65.26749146,375.83295013)
\curveto(65.21748914,375.87294982)(65.16248919,375.90294979)(65.10249146,375.92295013)
\curveto(65.0524893,375.95294974)(65.00248935,375.98794971)(64.95249146,376.02795013)
\curveto(64.71248964,376.20794949)(64.51248984,376.42794927)(64.35249146,376.68795013)
\curveto(64.19249016,376.94794875)(64.0524903,377.23294846)(63.93249146,377.54295013)
\curveto(63.87249048,377.68294801)(63.82749053,377.82294787)(63.79749146,377.96295013)
\curveto(63.76749059,378.11294758)(63.73249062,378.26794743)(63.69249146,378.42795013)
\curveto(63.67249068,378.53794716)(63.6574907,378.64794705)(63.64749146,378.75795013)
\curveto(63.63749072,378.86794683)(63.62249073,378.97794672)(63.60249146,379.08795013)
\curveto(63.59249076,379.12794657)(63.58749077,379.16794653)(63.58749146,379.20795013)
\curveto(63.59749076,379.24794645)(63.59749076,379.28794641)(63.58749146,379.32795013)
\curveto(63.57749078,379.37794632)(63.57249078,379.42794627)(63.57249146,379.47795013)
\lineto(63.57249146,379.64295013)
\curveto(63.5524908,379.692946)(63.54749081,379.74294595)(63.55749146,379.79295013)
\curveto(63.56749079,379.85294584)(63.56749079,379.90794579)(63.55749146,379.95795013)
\curveto(63.54749081,379.9979457)(63.54749081,380.04294565)(63.55749146,380.09295013)
\curveto(63.56749079,380.14294555)(63.56249079,380.1929455)(63.54249146,380.24295013)
\curveto(63.52249083,380.31294538)(63.51749084,380.38794531)(63.52749146,380.46795013)
\curveto(63.53749082,380.55794514)(63.54249081,380.64294505)(63.54249146,380.72295013)
\curveto(63.54249081,380.81294488)(63.53749082,380.91294478)(63.52749146,381.02295013)
\curveto(63.51749084,381.14294455)(63.52249083,381.24294445)(63.54249146,381.32295013)
\lineto(63.54249146,381.60795013)
\lineto(63.58749146,382.23795013)
\curveto(63.59749076,382.33794336)(63.60749075,382.43294326)(63.61749146,382.52295013)
\lineto(63.64749146,382.82295013)
\curveto(63.66749069,382.87294282)(63.67249068,382.92294277)(63.66249146,382.97295013)
\curveto(63.66249069,383.03294266)(63.67249068,383.08794261)(63.69249146,383.13795013)
\curveto(63.74249061,383.30794239)(63.78249057,383.47294222)(63.81249146,383.63295013)
\curveto(63.84249051,383.80294189)(63.89249046,383.96294173)(63.96249146,384.11295013)
\curveto(64.1524902,384.57294112)(64.37248998,384.94794075)(64.62249146,385.23795013)
\curveto(64.88248947,385.52794017)(65.24248911,385.77293992)(65.70249146,385.97295013)
\curveto(65.83248852,386.02293967)(65.96248839,386.05793964)(66.09249146,386.07795013)
\curveto(66.23248812,386.0979396)(66.37248798,386.12293957)(66.51249146,386.15295013)
\curveto(66.58248777,386.16293953)(66.64748771,386.16793953)(66.70749146,386.16795013)
\curveto(66.76748759,386.16793953)(66.83248752,386.17293952)(66.90249146,386.18295013)
\curveto(67.73248662,386.20293949)(68.40248595,386.05293964)(68.91249146,385.73295013)
\curveto(69.42248493,385.42294027)(69.80248455,384.98294071)(70.05249146,384.41295013)
\curveto(70.10248425,384.2929414)(70.14748421,384.16794153)(70.18749146,384.03795013)
\curveto(70.22748413,383.90794179)(70.27248408,383.77294192)(70.32249146,383.63295013)
\curveto(70.34248401,383.55294214)(70.357484,383.46794223)(70.36749146,383.37795013)
\lineto(70.42749146,383.13795013)
\curveto(70.4574839,383.02794267)(70.47248388,382.91794278)(70.47249146,382.80795013)
\curveto(70.48248387,382.697943)(70.49748386,382.58794311)(70.51749146,382.47795013)
\curveto(70.53748382,382.42794327)(70.54248381,382.38294331)(70.53249146,382.34295013)
\curveto(70.53248382,382.30294339)(70.53748382,382.26294343)(70.54749146,382.22295013)
\curveto(70.5574838,382.17294352)(70.5574838,382.11794358)(70.54749146,382.05795013)
\curveto(70.54748381,382.00794369)(70.5524838,381.95794374)(70.56249146,381.90795013)
\lineto(70.56249146,381.77295013)
\curveto(70.58248377,381.71294398)(70.58248377,381.64294405)(70.56249146,381.56295013)
\curveto(70.5524838,381.4929442)(70.5574838,381.42794427)(70.57749146,381.36795013)
\curveto(70.58748377,381.33794436)(70.59248376,381.2979444)(70.59249146,381.24795013)
\lineto(70.59249146,381.12795013)
\lineto(70.59249146,380.66295013)
\moveto(69.04749146,378.33795013)
\curveto(69.14748521,378.65794704)(69.20748515,379.02294667)(69.22749146,379.43295013)
\curveto(69.24748511,379.84294585)(69.2574851,380.25294544)(69.25749146,380.66295013)
\curveto(69.2574851,381.0929446)(69.24748511,381.51294418)(69.22749146,381.92295013)
\curveto(69.20748515,382.33294336)(69.16248519,382.71794298)(69.09249146,383.07795013)
\curveto(69.02248533,383.43794226)(68.91248544,383.75794194)(68.76249146,384.03795013)
\curveto(68.62248573,384.32794137)(68.42748593,384.56294113)(68.17749146,384.74295013)
\curveto(68.01748634,384.85294084)(67.83748652,384.93294076)(67.63749146,384.98295013)
\curveto(67.43748692,385.04294065)(67.19248716,385.07294062)(66.90249146,385.07295013)
\curveto(66.88248747,385.05294064)(66.84748751,385.04294065)(66.79749146,385.04295013)
\curveto(66.74748761,385.05294064)(66.70748765,385.05294064)(66.67749146,385.04295013)
\curveto(66.59748776,385.02294067)(66.52248783,385.00294069)(66.45249146,384.98295013)
\curveto(66.39248796,384.97294072)(66.32748803,384.95294074)(66.25749146,384.92295013)
\curveto(65.98748837,384.80294089)(65.76748859,384.63294106)(65.59749146,384.41295013)
\curveto(65.43748892,384.20294149)(65.30248905,383.95794174)(65.19249146,383.67795013)
\curveto(65.14248921,383.56794213)(65.10248925,383.44794225)(65.07249146,383.31795013)
\curveto(65.0524893,383.1979425)(65.02748933,383.07294262)(64.99749146,382.94295013)
\curveto(64.97748938,382.8929428)(64.96748939,382.83794286)(64.96749146,382.77795013)
\curveto(64.96748939,382.72794297)(64.96248939,382.67794302)(64.95249146,382.62795013)
\curveto(64.94248941,382.53794316)(64.93248942,382.44294325)(64.92249146,382.34295013)
\curveto(64.91248944,382.25294344)(64.90248945,382.15794354)(64.89249146,382.05795013)
\curveto(64.89248946,381.97794372)(64.88748947,381.8929438)(64.87749146,381.80295013)
\lineto(64.87749146,381.56295013)
\lineto(64.87749146,381.38295013)
\curveto(64.86748949,381.35294434)(64.86248949,381.31794438)(64.86249146,381.27795013)
\lineto(64.86249146,381.14295013)
\lineto(64.86249146,380.69295013)
\curveto(64.86248949,380.61294508)(64.8574895,380.52794517)(64.84749146,380.43795013)
\curveto(64.84748951,380.35794534)(64.8574895,380.28294541)(64.87749146,380.21295013)
\lineto(64.87749146,379.94295013)
\curveto(64.87748948,379.92294577)(64.87248948,379.8929458)(64.86249146,379.85295013)
\curveto(64.86248949,379.82294587)(64.86748949,379.7979459)(64.87749146,379.77795013)
\curveto(64.88748947,379.67794602)(64.89248946,379.57794612)(64.89249146,379.47795013)
\curveto(64.90248945,379.38794631)(64.91248944,379.28794641)(64.92249146,379.17795013)
\curveto(64.9524894,379.05794664)(64.96748939,378.93294676)(64.96749146,378.80295013)
\curveto(64.97748938,378.68294701)(65.00248935,378.56794713)(65.04249146,378.45795013)
\curveto(65.12248923,378.15794754)(65.20748915,377.8929478)(65.29749146,377.66295013)
\curveto(65.39748896,377.43294826)(65.54248881,377.21794848)(65.73249146,377.01795013)
\curveto(65.94248841,376.81794888)(66.20748815,376.66794903)(66.52749146,376.56795013)
\curveto(66.56748779,376.54794915)(66.60248775,376.53794916)(66.63249146,376.53795013)
\curveto(66.67248768,376.54794915)(66.71748764,376.54294915)(66.76749146,376.52295013)
\curveto(66.80748755,376.51294918)(66.87748748,376.50294919)(66.97749146,376.49295013)
\curveto(67.08748727,376.48294921)(67.17248718,376.48794921)(67.23249146,376.50795013)
\curveto(67.30248705,376.52794917)(67.37248698,376.53794916)(67.44249146,376.53795013)
\curveto(67.51248684,376.54794915)(67.57748678,376.56294913)(67.63749146,376.58295013)
\curveto(67.83748652,376.64294905)(68.01748634,376.72794897)(68.17749146,376.83795013)
\curveto(68.20748615,376.85794884)(68.23248612,376.87794882)(68.25249146,376.89795013)
\lineto(68.31249146,376.95795013)
\curveto(68.352486,376.97794872)(68.40248595,377.01794868)(68.46249146,377.07795013)
\curveto(68.56248579,377.21794848)(68.64748571,377.34794835)(68.71749146,377.46795013)
\curveto(68.78748557,377.58794811)(68.8574855,377.73294796)(68.92749146,377.90295013)
\curveto(68.9574854,377.97294772)(68.97748538,378.04294765)(68.98749146,378.11295013)
\curveto(69.00748535,378.18294751)(69.02748533,378.25794744)(69.04749146,378.33795013)
}
}
{
\newrgbcolor{curcolor}{0 0 0}
\pscustom[linestyle=none,fillstyle=solid,fillcolor=curcolor]
{
\newpath
\moveto(766.91147461,369.09397308)
\curveto(766.93146506,369.04397234)(766.95646504,368.9839724)(766.98647461,368.91397308)
\curveto(767.01646498,368.84397254)(767.03646496,368.76897261)(767.04647461,368.68897308)
\curveto(767.06646493,368.61897276)(767.06646493,368.54897283)(767.04647461,368.47897308)
\curveto(767.03646496,368.41897296)(766.996465,368.37397301)(766.92647461,368.34397308)
\curveto(766.87646512,368.32397306)(766.81646518,368.31397307)(766.74647461,368.31397308)
\lineto(766.53647461,368.31397308)
\lineto(766.08647461,368.31397308)
\curveto(765.93646606,368.31397307)(765.81646618,368.33897304)(765.72647461,368.38897308)
\curveto(765.62646637,368.44897293)(765.55146644,368.55397283)(765.50147461,368.70397308)
\curveto(765.46146653,368.85397253)(765.41646658,368.98897239)(765.36647461,369.10897308)
\curveto(765.25646674,369.36897201)(765.15646684,369.63897174)(765.06647461,369.91897308)
\curveto(764.97646702,370.19897118)(764.87646712,370.47397091)(764.76647461,370.74397308)
\curveto(764.73646726,370.83397055)(764.70646729,370.91897046)(764.67647461,370.99897308)
\curveto(764.65646734,371.0789703)(764.62646737,371.15397023)(764.58647461,371.22397308)
\curveto(764.55646744,371.29397009)(764.51146748,371.35397003)(764.45147461,371.40397308)
\curveto(764.3914676,371.45396993)(764.31146768,371.49396989)(764.21147461,371.52397308)
\curveto(764.16146783,371.54396984)(764.10146789,371.54896983)(764.03147461,371.53897308)
\lineto(763.83647461,371.53897308)
\lineto(761.00147461,371.53897308)
\lineto(760.70147461,371.53897308)
\curveto(760.5914714,371.54896983)(760.48647151,371.54896983)(760.38647461,371.53897308)
\curveto(760.28647171,371.52896985)(760.1914718,371.51396987)(760.10147461,371.49397308)
\curveto(760.02147197,371.47396991)(759.96147203,371.43396995)(759.92147461,371.37397308)
\curveto(759.84147215,371.27397011)(759.78147221,371.15897022)(759.74147461,371.02897308)
\curveto(759.71147228,370.90897047)(759.67147232,370.7839706)(759.62147461,370.65397308)
\curveto(759.52147247,370.42397096)(759.42647257,370.1839712)(759.33647461,369.93397308)
\curveto(759.25647274,369.6839717)(759.16647283,369.44397194)(759.06647461,369.21397308)
\curveto(759.04647295,369.15397223)(759.02147297,369.0839723)(758.99147461,369.00397308)
\curveto(758.97147302,368.93397245)(758.94647305,368.85897252)(758.91647461,368.77897308)
\curveto(758.88647311,368.69897268)(758.85147314,368.62397276)(758.81147461,368.55397308)
\curveto(758.78147321,368.49397289)(758.74647325,368.44897293)(758.70647461,368.41897308)
\curveto(758.62647337,368.35897302)(758.51647348,368.32397306)(758.37647461,368.31397308)
\lineto(757.95647461,368.31397308)
\lineto(757.71647461,368.31397308)
\curveto(757.64647435,368.32397306)(757.58647441,368.34897303)(757.53647461,368.38897308)
\curveto(757.48647451,368.41897296)(757.45647454,368.46397292)(757.44647461,368.52397308)
\curveto(757.44647455,368.5839728)(757.45147454,368.64397274)(757.46147461,368.70397308)
\curveto(757.48147451,368.77397261)(757.50147449,368.83897254)(757.52147461,368.89897308)
\curveto(757.55147444,368.96897241)(757.57647442,369.01897236)(757.59647461,369.04897308)
\curveto(757.73647426,369.36897201)(757.86147413,369.6839717)(757.97147461,369.99397308)
\curveto(758.08147391,370.31397107)(758.20147379,370.63397075)(758.33147461,370.95397308)
\curveto(758.42147357,371.17397021)(758.50647349,371.38896999)(758.58647461,371.59897308)
\curveto(758.66647333,371.81896956)(758.75147324,372.03896934)(758.84147461,372.25897308)
\curveto(759.14147285,372.9789684)(759.42647257,373.70396768)(759.69647461,374.43397308)
\curveto(759.96647203,375.17396621)(760.25147174,375.90896547)(760.55147461,376.63897308)
\curveto(760.66147133,376.89896448)(760.76147123,377.16396422)(760.85147461,377.43397308)
\curveto(760.95147104,377.70396368)(761.05647094,377.96896341)(761.16647461,378.22897308)
\curveto(761.21647078,378.33896304)(761.26147073,378.45896292)(761.30147461,378.58897308)
\curveto(761.35147064,378.72896265)(761.42147057,378.82896255)(761.51147461,378.88897308)
\curveto(761.55147044,378.92896245)(761.61647038,378.95896242)(761.70647461,378.97897308)
\curveto(761.72647027,378.98896239)(761.74647025,378.98896239)(761.76647461,378.97897308)
\curveto(761.7964702,378.9789624)(761.82147017,378.9839624)(761.84147461,378.99397308)
\curveto(762.02146997,378.99396239)(762.23146976,378.99396239)(762.47147461,378.99397308)
\curveto(762.71146928,379.00396238)(762.88646911,378.96896241)(762.99647461,378.88897308)
\curveto(763.07646892,378.82896255)(763.13646886,378.72896265)(763.17647461,378.58897308)
\curveto(763.22646877,378.45896292)(763.27646872,378.33896304)(763.32647461,378.22897308)
\curveto(763.42646857,377.99896338)(763.51646848,377.76896361)(763.59647461,377.53897308)
\curveto(763.67646832,377.30896407)(763.76646823,377.0789643)(763.86647461,376.84897308)
\curveto(763.94646805,376.64896473)(764.02146797,376.44396494)(764.09147461,376.23397308)
\curveto(764.17146782,376.02396536)(764.25646774,375.81896556)(764.34647461,375.61897308)
\curveto(764.64646735,374.88896649)(764.93146706,374.14896723)(765.20147461,373.39897308)
\curveto(765.48146651,372.65896872)(765.77646622,371.92396946)(766.08647461,371.19397308)
\curveto(766.12646587,371.10397028)(766.15646584,371.01897036)(766.17647461,370.93897308)
\curveto(766.20646579,370.85897052)(766.23646576,370.77397061)(766.26647461,370.68397308)
\curveto(766.37646562,370.42397096)(766.48146551,370.15897122)(766.58147461,369.88897308)
\curveto(766.6914653,369.61897176)(766.80146519,369.35397203)(766.91147461,369.09397308)
\moveto(763.70147461,372.73897308)
\curveto(763.7914682,372.76896861)(763.84646815,372.81896856)(763.86647461,372.88897308)
\curveto(763.8964681,372.95896842)(763.90146809,373.03396835)(763.88147461,373.11397308)
\curveto(763.87146812,373.20396818)(763.84646815,373.28896809)(763.80647461,373.36897308)
\curveto(763.77646822,373.45896792)(763.74646825,373.53396785)(763.71647461,373.59397308)
\curveto(763.6964683,373.63396775)(763.68646831,373.66896771)(763.68647461,373.69897308)
\curveto(763.68646831,373.72896765)(763.67646832,373.76396762)(763.65647461,373.80397308)
\lineto(763.56647461,374.04397308)
\curveto(763.54646845,374.13396725)(763.51646848,374.22396716)(763.47647461,374.31397308)
\curveto(763.32646867,374.67396671)(763.1914688,375.03896634)(763.07147461,375.40897308)
\curveto(762.96146903,375.78896559)(762.83146916,376.15896522)(762.68147461,376.51897308)
\curveto(762.63146936,376.62896475)(762.58646941,376.73896464)(762.54647461,376.84897308)
\curveto(762.51646948,376.95896442)(762.47646952,377.06396432)(762.42647461,377.16397308)
\curveto(762.40646959,377.21396417)(762.38146961,377.25896412)(762.35147461,377.29897308)
\curveto(762.33146966,377.34896403)(762.28146971,377.37396401)(762.20147461,377.37397308)
\curveto(762.18146981,377.35396403)(762.16146983,377.33896404)(762.14147461,377.32897308)
\curveto(762.12146987,377.31896406)(762.10146989,377.30396408)(762.08147461,377.28397308)
\curveto(762.04146995,377.23396415)(762.01146998,377.1789642)(761.99147461,377.11897308)
\curveto(761.97147002,377.06896431)(761.95147004,377.01396437)(761.93147461,376.95397308)
\curveto(761.88147011,376.84396454)(761.84147015,376.73396465)(761.81147461,376.62397308)
\curveto(761.78147021,376.51396487)(761.74147025,376.40396498)(761.69147461,376.29397308)
\curveto(761.52147047,375.90396548)(761.37147062,375.50896587)(761.24147461,375.10897308)
\curveto(761.12147087,374.70896667)(760.98147101,374.31896706)(760.82147461,373.93897308)
\lineto(760.76147461,373.78897308)
\curveto(760.75147124,373.73896764)(760.73647126,373.68896769)(760.71647461,373.63897308)
\lineto(760.62647461,373.39897308)
\curveto(760.5964714,373.31896806)(760.57147142,373.23896814)(760.55147461,373.15897308)
\curveto(760.53147146,373.10896827)(760.52147147,373.05396833)(760.52147461,372.99397308)
\curveto(760.53147146,372.93396845)(760.54647145,372.8839685)(760.56647461,372.84397308)
\curveto(760.61647138,372.76396862)(760.72147127,372.71896866)(760.88147461,372.70897308)
\lineto(761.33147461,372.70897308)
\lineto(762.93647461,372.70897308)
\curveto(763.04646895,372.70896867)(763.18146881,372.70396868)(763.34147461,372.69397308)
\curveto(763.50146849,372.69396869)(763.62146837,372.70896867)(763.70147461,372.73897308)
}
}
{
\newrgbcolor{curcolor}{0 0 0}
\pscustom[linestyle=none,fillstyle=solid,fillcolor=curcolor]
{
\newpath
\moveto(771.16303711,376.21897308)
\curveto(771.90303232,376.22896515)(772.5180317,376.11896526)(773.00803711,375.88897308)
\curveto(773.50803071,375.66896571)(773.90303032,375.33396605)(774.19303711,374.88397308)
\curveto(774.3230299,374.6839667)(774.43302979,374.43896694)(774.52303711,374.14897308)
\curveto(774.54302968,374.09896728)(774.55802966,374.03396735)(774.56803711,373.95397308)
\curveto(774.57802964,373.87396751)(774.57302965,373.80396758)(774.55303711,373.74397308)
\curveto(774.5230297,373.69396769)(774.47302975,373.64896773)(774.40303711,373.60897308)
\curveto(774.37302985,373.58896779)(774.34302988,373.5789678)(774.31303711,373.57897308)
\curveto(774.28302994,373.58896779)(774.24802997,373.58896779)(774.20803711,373.57897308)
\curveto(774.16803005,373.56896781)(774.12803009,373.56396782)(774.08803711,373.56397308)
\curveto(774.04803017,373.57396781)(774.00803021,373.5789678)(773.96803711,373.57897308)
\lineto(773.65303711,373.57897308)
\curveto(773.55303067,373.58896779)(773.46803075,373.61896776)(773.39803711,373.66897308)
\curveto(773.3180309,373.72896765)(773.26303096,373.81396757)(773.23303711,373.92397308)
\curveto(773.20303102,374.03396735)(773.16303106,374.12896725)(773.11303711,374.20897308)
\curveto(772.96303126,374.46896691)(772.76803145,374.67396671)(772.52803711,374.82397308)
\curveto(772.44803177,374.87396651)(772.36303186,374.91396647)(772.27303711,374.94397308)
\curveto(772.18303204,374.9839664)(772.08803213,375.01896636)(771.98803711,375.04897308)
\curveto(771.84803237,375.08896629)(771.66303256,375.10896627)(771.43303711,375.10897308)
\curveto(771.20303302,375.11896626)(771.01303321,375.09896628)(770.86303711,375.04897308)
\curveto(770.79303343,375.02896635)(770.72803349,375.01396637)(770.66803711,375.00397308)
\curveto(770.60803361,374.99396639)(770.54303368,374.9789664)(770.47303711,374.95897308)
\curveto(770.21303401,374.84896653)(769.98303424,374.69896668)(769.78303711,374.50897308)
\curveto(769.58303464,374.31896706)(769.42803479,374.09396729)(769.31803711,373.83397308)
\curveto(769.27803494,373.74396764)(769.24303498,373.64896773)(769.21303711,373.54897308)
\curveto(769.18303504,373.45896792)(769.15303507,373.35896802)(769.12303711,373.24897308)
\lineto(769.03303711,372.84397308)
\curveto(769.0230352,372.79396859)(769.0180352,372.73896864)(769.01803711,372.67897308)
\curveto(769.02803519,372.61896876)(769.0230352,372.56396882)(769.00303711,372.51397308)
\lineto(769.00303711,372.39397308)
\curveto(768.99303523,372.35396903)(768.98303524,372.28896909)(768.97303711,372.19897308)
\curveto(768.97303525,372.10896927)(768.98303524,372.04396934)(769.00303711,372.00397308)
\curveto(769.01303521,371.95396943)(769.01303521,371.90396948)(769.00303711,371.85397308)
\curveto(768.99303523,371.80396958)(768.99303523,371.75396963)(769.00303711,371.70397308)
\curveto(769.01303521,371.66396972)(769.0180352,371.59396979)(769.01803711,371.49397308)
\curveto(769.03803518,371.41396997)(769.05303517,371.32897005)(769.06303711,371.23897308)
\curveto(769.08303514,371.14897023)(769.10303512,371.06397032)(769.12303711,370.98397308)
\curveto(769.23303499,370.66397072)(769.35803486,370.383971)(769.49803711,370.14397308)
\curveto(769.64803457,369.91397147)(769.85303437,369.71397167)(770.11303711,369.54397308)
\curveto(770.20303402,369.49397189)(770.29303393,369.44897193)(770.38303711,369.40897308)
\curveto(770.48303374,369.36897201)(770.58803363,369.32897205)(770.69803711,369.28897308)
\curveto(770.74803347,369.2789721)(770.78803343,369.27397211)(770.81803711,369.27397308)
\curveto(770.84803337,369.27397211)(770.88803333,369.26897211)(770.93803711,369.25897308)
\curveto(770.96803325,369.24897213)(771.0180332,369.24397214)(771.08803711,369.24397308)
\lineto(771.25303711,369.24397308)
\curveto(771.25303297,369.23397215)(771.27303295,369.22897215)(771.31303711,369.22897308)
\curveto(771.33303289,369.23897214)(771.35803286,369.23897214)(771.38803711,369.22897308)
\curveto(771.4180328,369.22897215)(771.44803277,369.23397215)(771.47803711,369.24397308)
\curveto(771.54803267,369.26397212)(771.61303261,369.26897211)(771.67303711,369.25897308)
\curveto(771.74303248,369.25897212)(771.81303241,369.26897211)(771.88303711,369.28897308)
\curveto(772.14303208,369.36897201)(772.36803185,369.46897191)(772.55803711,369.58897308)
\curveto(772.74803147,369.71897166)(772.90803131,369.8839715)(773.03803711,370.08397308)
\curveto(773.08803113,370.16397122)(773.13303109,370.24897113)(773.17303711,370.33897308)
\lineto(773.29303711,370.60897308)
\curveto(773.31303091,370.68897069)(773.33303089,370.76397062)(773.35303711,370.83397308)
\curveto(773.38303084,370.91397047)(773.43303079,370.9789704)(773.50303711,371.02897308)
\curveto(773.53303069,371.05897032)(773.59303063,371.0789703)(773.68303711,371.08897308)
\curveto(773.77303045,371.10897027)(773.86803035,371.11897026)(773.96803711,371.11897308)
\curveto(774.07803014,371.12897025)(774.17803004,371.12897025)(774.26803711,371.11897308)
\curveto(774.36802985,371.10897027)(774.43802978,371.08897029)(774.47803711,371.05897308)
\curveto(774.53802968,371.01897036)(774.57302965,370.95897042)(774.58303711,370.87897308)
\curveto(774.60302962,370.79897058)(774.60302962,370.71397067)(774.58303711,370.62397308)
\curveto(774.53302969,370.47397091)(774.48302974,370.32897105)(774.43303711,370.18897308)
\curveto(774.39302983,370.05897132)(774.33802988,369.92897145)(774.26803711,369.79897308)
\curveto(774.1180301,369.49897188)(773.92803029,369.23397215)(773.69803711,369.00397308)
\curveto(773.47803074,368.77397261)(773.20803101,368.58897279)(772.88803711,368.44897308)
\curveto(772.80803141,368.40897297)(772.7230315,368.37397301)(772.63303711,368.34397308)
\curveto(772.54303168,368.32397306)(772.44803177,368.29897308)(772.34803711,368.26897308)
\curveto(772.23803198,368.22897315)(772.12803209,368.20897317)(772.01803711,368.20897308)
\curveto(771.90803231,368.19897318)(771.79803242,368.1839732)(771.68803711,368.16397308)
\curveto(771.64803257,368.14397324)(771.60803261,368.13897324)(771.56803711,368.14897308)
\curveto(771.52803269,368.15897322)(771.48803273,368.15897322)(771.44803711,368.14897308)
\lineto(771.31303711,368.14897308)
\lineto(771.07303711,368.14897308)
\curveto(771.00303322,368.13897324)(770.93803328,368.14397324)(770.87803711,368.16397308)
\lineto(770.80303711,368.16397308)
\lineto(770.44303711,368.20897308)
\curveto(770.31303391,368.24897313)(770.18803403,368.2839731)(770.06803711,368.31397308)
\curveto(769.94803427,368.34397304)(769.83303439,368.383973)(769.72303711,368.43397308)
\curveto(769.36303486,368.59397279)(769.06303516,368.7839726)(768.82303711,369.00397308)
\curveto(768.59303563,369.22397216)(768.37803584,369.49397189)(768.17803711,369.81397308)
\curveto(768.12803609,369.89397149)(768.08303614,369.9839714)(768.04303711,370.08397308)
\lineto(767.92303711,370.38397308)
\curveto(767.87303635,370.49397089)(767.83803638,370.60897077)(767.81803711,370.72897308)
\curveto(767.79803642,370.84897053)(767.77303645,370.96897041)(767.74303711,371.08897308)
\curveto(767.73303649,371.12897025)(767.72803649,371.16897021)(767.72803711,371.20897308)
\curveto(767.72803649,371.24897013)(767.7230365,371.28897009)(767.71303711,371.32897308)
\curveto(767.69303653,371.38896999)(767.68303654,371.45396993)(767.68303711,371.52397308)
\curveto(767.69303653,371.59396979)(767.68803653,371.65896972)(767.66803711,371.71897308)
\lineto(767.66803711,371.86897308)
\curveto(767.65803656,371.91896946)(767.65303657,371.98896939)(767.65303711,372.07897308)
\curveto(767.65303657,372.16896921)(767.65803656,372.23896914)(767.66803711,372.28897308)
\curveto(767.67803654,372.33896904)(767.67803654,372.383969)(767.66803711,372.42397308)
\curveto(767.66803655,372.46396892)(767.67303655,372.50396888)(767.68303711,372.54397308)
\curveto(767.70303652,372.61396877)(767.70803651,372.6839687)(767.69803711,372.75397308)
\curveto(767.69803652,372.82396856)(767.70803651,372.88896849)(767.72803711,372.94897308)
\curveto(767.76803645,373.11896826)(767.80303642,373.28896809)(767.83303711,373.45897308)
\curveto(767.86303636,373.62896775)(767.90803631,373.78896759)(767.96803711,373.93897308)
\curveto(768.17803604,374.45896692)(768.43303579,374.8789665)(768.73303711,375.19897308)
\curveto(769.03303519,375.51896586)(769.44303478,375.7839656)(769.96303711,375.99397308)
\curveto(770.07303415,376.04396534)(770.19303403,376.0789653)(770.32303711,376.09897308)
\curveto(770.45303377,376.11896526)(770.58803363,376.14396524)(770.72803711,376.17397308)
\curveto(770.79803342,376.1839652)(770.86803335,376.18896519)(770.93803711,376.18897308)
\curveto(771.00803321,376.19896518)(771.08303314,376.20896517)(771.16303711,376.21897308)
}
}
{
\newrgbcolor{curcolor}{0 0 0}
\pscustom[linestyle=none,fillstyle=solid,fillcolor=curcolor]
{
\newpath
\moveto(777.02967773,378.37897308)
\curveto(777.17967572,378.378963)(777.32967557,378.37396301)(777.47967773,378.36397308)
\curveto(777.62967527,378.36396302)(777.73467517,378.32396306)(777.79467773,378.24397308)
\curveto(777.84467506,378.1839632)(777.86967503,378.09896328)(777.86967773,377.98897308)
\curveto(777.87967502,377.88896349)(777.88467502,377.7839636)(777.88467773,377.67397308)
\lineto(777.88467773,376.80397308)
\curveto(777.88467502,376.72396466)(777.87967502,376.63896474)(777.86967773,376.54897308)
\curveto(777.86967503,376.46896491)(777.87967502,376.39896498)(777.89967773,376.33897308)
\curveto(777.93967496,376.19896518)(778.02967487,376.10896527)(778.16967773,376.06897308)
\curveto(778.21967468,376.05896532)(778.26467464,376.05396533)(778.30467773,376.05397308)
\lineto(778.45467773,376.05397308)
\lineto(778.85967773,376.05397308)
\curveto(779.01967388,376.06396532)(779.13467377,376.05396533)(779.20467773,376.02397308)
\curveto(779.29467361,375.96396542)(779.35467355,375.90396548)(779.38467773,375.84397308)
\curveto(779.4046735,375.80396558)(779.41467349,375.75896562)(779.41467773,375.70897308)
\lineto(779.41467773,375.55897308)
\curveto(779.41467349,375.44896593)(779.40967349,375.34396604)(779.39967773,375.24397308)
\curveto(779.38967351,375.15396623)(779.35467355,375.0839663)(779.29467773,375.03397308)
\curveto(779.23467367,374.9839664)(779.14967375,374.95396643)(779.03967773,374.94397308)
\lineto(778.70967773,374.94397308)
\curveto(778.5996743,374.95396643)(778.48967441,374.95896642)(778.37967773,374.95897308)
\curveto(778.26967463,374.95896642)(778.17467473,374.94396644)(778.09467773,374.91397308)
\curveto(778.02467488,374.8839665)(777.97467493,374.83396655)(777.94467773,374.76397308)
\curveto(777.91467499,374.69396669)(777.89467501,374.60896677)(777.88467773,374.50897308)
\curveto(777.87467503,374.41896696)(777.86967503,374.31896706)(777.86967773,374.20897308)
\curveto(777.87967502,374.10896727)(777.88467502,374.00896737)(777.88467773,373.90897308)
\lineto(777.88467773,370.93897308)
\curveto(777.88467502,370.71897066)(777.87967502,370.4839709)(777.86967773,370.23397308)
\curveto(777.86967503,369.99397139)(777.91467499,369.80897157)(778.00467773,369.67897308)
\curveto(778.05467485,369.59897178)(778.11967478,369.54397184)(778.19967773,369.51397308)
\curveto(778.27967462,369.4839719)(778.37467453,369.45897192)(778.48467773,369.43897308)
\curveto(778.51467439,369.42897195)(778.54467436,369.42397196)(778.57467773,369.42397308)
\curveto(778.61467429,369.43397195)(778.64967425,369.43397195)(778.67967773,369.42397308)
\lineto(778.87467773,369.42397308)
\curveto(778.97467393,369.42397196)(779.06467384,369.41397197)(779.14467773,369.39397308)
\curveto(779.23467367,369.383972)(779.2996736,369.34897203)(779.33967773,369.28897308)
\curveto(779.35967354,369.25897212)(779.37467353,369.20397218)(779.38467773,369.12397308)
\curveto(779.4046735,369.05397233)(779.41467349,368.9789724)(779.41467773,368.89897308)
\curveto(779.42467348,368.81897256)(779.42467348,368.73897264)(779.41467773,368.65897308)
\curveto(779.4046735,368.58897279)(779.38467352,368.53397285)(779.35467773,368.49397308)
\curveto(779.31467359,368.42397296)(779.23967366,368.37397301)(779.12967773,368.34397308)
\curveto(779.04967385,368.32397306)(778.95967394,368.31397307)(778.85967773,368.31397308)
\curveto(778.75967414,368.32397306)(778.66967423,368.32897305)(778.58967773,368.32897308)
\curveto(778.52967437,368.32897305)(778.46967443,368.32397306)(778.40967773,368.31397308)
\curveto(778.34967455,368.31397307)(778.29467461,368.31897306)(778.24467773,368.32897308)
\lineto(778.06467773,368.32897308)
\curveto(778.01467489,368.33897304)(777.96467494,368.34397304)(777.91467773,368.34397308)
\curveto(777.87467503,368.35397303)(777.82967507,368.35897302)(777.77967773,368.35897308)
\curveto(777.57967532,368.40897297)(777.4046755,368.46397292)(777.25467773,368.52397308)
\curveto(777.11467579,368.5839728)(776.99467591,368.68897269)(776.89467773,368.83897308)
\curveto(776.75467615,369.03897234)(776.67467623,369.28897209)(776.65467773,369.58897308)
\curveto(776.63467627,369.89897148)(776.62467628,370.22897115)(776.62467773,370.57897308)
\lineto(776.62467773,374.50897308)
\curveto(776.59467631,374.63896674)(776.56467634,374.73396665)(776.53467773,374.79397308)
\curveto(776.51467639,374.85396653)(776.44467646,374.90396648)(776.32467773,374.94397308)
\curveto(776.28467662,374.95396643)(776.24467666,374.95396643)(776.20467773,374.94397308)
\curveto(776.16467674,374.93396645)(776.12467678,374.93896644)(776.08467773,374.95897308)
\lineto(775.84467773,374.95897308)
\curveto(775.71467719,374.95896642)(775.6046773,374.96896641)(775.51467773,374.98897308)
\curveto(775.43467747,375.01896636)(775.37967752,375.0789663)(775.34967773,375.16897308)
\curveto(775.32967757,375.20896617)(775.31467759,375.25396613)(775.30467773,375.30397308)
\lineto(775.30467773,375.45397308)
\curveto(775.3046776,375.59396579)(775.31467759,375.70896567)(775.33467773,375.79897308)
\curveto(775.35467755,375.89896548)(775.41467749,375.97396541)(775.51467773,376.02397308)
\curveto(775.62467728,376.06396532)(775.76467714,376.07396531)(775.93467773,376.05397308)
\curveto(776.11467679,376.03396535)(776.26467664,376.04396534)(776.38467773,376.08397308)
\curveto(776.47467643,376.13396525)(776.54467636,376.20396518)(776.59467773,376.29397308)
\curveto(776.61467629,376.35396503)(776.62467628,376.42896495)(776.62467773,376.51897308)
\lineto(776.62467773,376.77397308)
\lineto(776.62467773,377.70397308)
\lineto(776.62467773,377.94397308)
\curveto(776.62467628,378.03396335)(776.63467627,378.10896327)(776.65467773,378.16897308)
\curveto(776.69467621,378.24896313)(776.76967613,378.31396307)(776.87967773,378.36397308)
\curveto(776.90967599,378.36396302)(776.93467597,378.36396302)(776.95467773,378.36397308)
\curveto(776.98467592,378.37396301)(777.00967589,378.378963)(777.02967773,378.37897308)
}
}
{
\newrgbcolor{curcolor}{0 0 0}
\pscustom[linestyle=none,fillstyle=solid,fillcolor=curcolor]
{
\newpath
\moveto(781.08647461,377.53897308)
\curveto(781.00647349,377.59896378)(780.96147353,377.70396368)(780.95147461,377.85397308)
\lineto(780.95147461,378.31897308)
\lineto(780.95147461,378.57397308)
\curveto(780.95147354,378.66396272)(780.96647353,378.73896264)(780.99647461,378.79897308)
\curveto(781.03647346,378.8789625)(781.11647338,378.93896244)(781.23647461,378.97897308)
\curveto(781.25647324,378.98896239)(781.27647322,378.98896239)(781.29647461,378.97897308)
\curveto(781.32647317,378.9789624)(781.35147314,378.9839624)(781.37147461,378.99397308)
\curveto(781.54147295,378.99396239)(781.70147279,378.98896239)(781.85147461,378.97897308)
\curveto(782.00147249,378.96896241)(782.10147239,378.90896247)(782.15147461,378.79897308)
\curveto(782.18147231,378.73896264)(782.1964723,378.66396272)(782.19647461,378.57397308)
\lineto(782.19647461,378.31897308)
\curveto(782.1964723,378.13896324)(782.1914723,377.96896341)(782.18147461,377.80897308)
\curveto(782.18147231,377.64896373)(782.11647238,377.54396384)(781.98647461,377.49397308)
\curveto(781.93647256,377.47396391)(781.88147261,377.46396392)(781.82147461,377.46397308)
\lineto(781.65647461,377.46397308)
\lineto(781.34147461,377.46397308)
\curveto(781.24147325,377.46396392)(781.15647334,377.48896389)(781.08647461,377.53897308)
\moveto(782.19647461,369.03397308)
\lineto(782.19647461,368.71897308)
\curveto(782.20647229,368.61897276)(782.18647231,368.53897284)(782.13647461,368.47897308)
\curveto(782.10647239,368.41897296)(782.06147243,368.378973)(782.00147461,368.35897308)
\curveto(781.94147255,368.34897303)(781.87147262,368.33397305)(781.79147461,368.31397308)
\lineto(781.56647461,368.31397308)
\curveto(781.43647306,368.31397307)(781.32147317,368.31897306)(781.22147461,368.32897308)
\curveto(781.13147336,368.34897303)(781.06147343,368.39897298)(781.01147461,368.47897308)
\curveto(780.97147352,368.53897284)(780.95147354,368.61397277)(780.95147461,368.70397308)
\lineto(780.95147461,368.98897308)
\lineto(780.95147461,375.33397308)
\lineto(780.95147461,375.64897308)
\curveto(780.95147354,375.75896562)(780.97647352,375.84396554)(781.02647461,375.90397308)
\curveto(781.05647344,375.95396543)(781.0964734,375.9839654)(781.14647461,375.99397308)
\curveto(781.1964733,376.00396538)(781.25147324,376.01896536)(781.31147461,376.03897308)
\curveto(781.33147316,376.03896534)(781.35147314,376.03396535)(781.37147461,376.02397308)
\curveto(781.40147309,376.02396536)(781.42647307,376.02896535)(781.44647461,376.03897308)
\curveto(781.57647292,376.03896534)(781.70647279,376.03396535)(781.83647461,376.02397308)
\curveto(781.97647252,376.02396536)(782.07147242,375.9839654)(782.12147461,375.90397308)
\curveto(782.17147232,375.84396554)(782.1964723,375.76396562)(782.19647461,375.66397308)
\lineto(782.19647461,375.37897308)
\lineto(782.19647461,369.03397308)
}
}
{
\newrgbcolor{curcolor}{0 0 0}
\pscustom[linestyle=none,fillstyle=solid,fillcolor=curcolor]
{
\newpath
\moveto(783.91631836,376.03897308)
\lineto(784.39631836,376.03897308)
\curveto(784.56631702,376.03896534)(784.69631689,376.00896537)(784.78631836,375.94897308)
\curveto(784.85631673,375.89896548)(784.90131668,375.83396555)(784.92131836,375.75397308)
\curveto(784.95131663,375.6839657)(784.9813166,375.60896577)(785.01131836,375.52897308)
\curveto(785.07131651,375.38896599)(785.12131646,375.24896613)(785.16131836,375.10897308)
\curveto(785.20131638,374.96896641)(785.24631634,374.82896655)(785.29631836,374.68897308)
\curveto(785.49631609,374.14896723)(785.6813159,373.60396778)(785.85131836,373.05397308)
\curveto(786.02131556,372.51396887)(786.20631538,371.97396941)(786.40631836,371.43397308)
\curveto(786.47631511,371.25397013)(786.53631505,371.06897031)(786.58631836,370.87897308)
\curveto(786.63631495,370.69897068)(786.70131488,370.51897086)(786.78131836,370.33897308)
\curveto(786.80131478,370.26897111)(786.82631476,370.19397119)(786.85631836,370.11397308)
\curveto(786.8863147,370.03397135)(786.93631465,369.9839714)(787.00631836,369.96397308)
\curveto(787.0863145,369.94397144)(787.14631444,369.9789714)(787.18631836,370.06897308)
\curveto(787.23631435,370.15897122)(787.27131431,370.22897115)(787.29131836,370.27897308)
\curveto(787.37131421,370.46897091)(787.43631415,370.65897072)(787.48631836,370.84897308)
\curveto(787.54631404,371.04897033)(787.61131397,371.24897013)(787.68131836,371.44897308)
\curveto(787.81131377,371.82896955)(787.93631365,372.20396918)(788.05631836,372.57397308)
\curveto(788.17631341,372.95396843)(788.30131328,373.33396805)(788.43131836,373.71397308)
\curveto(788.4813131,373.8839675)(788.53131305,374.04896733)(788.58131836,374.20897308)
\curveto(788.63131295,374.378967)(788.69131289,374.54396684)(788.76131836,374.70397308)
\curveto(788.81131277,374.84396654)(788.85631273,374.9839664)(788.89631836,375.12397308)
\curveto(788.93631265,375.26396612)(788.9813126,375.40396598)(789.03131836,375.54397308)
\curveto(789.05131253,375.61396577)(789.07631251,375.6839657)(789.10631836,375.75397308)
\curveto(789.13631245,375.82396556)(789.17631241,375.8839655)(789.22631836,375.93397308)
\curveto(789.30631228,375.9839654)(789.39631219,376.01396537)(789.49631836,376.02397308)
\curveto(789.59631199,376.03396535)(789.71631187,376.03896534)(789.85631836,376.03897308)
\curveto(789.92631166,376.03896534)(789.99131159,376.03396535)(790.05131836,376.02397308)
\curveto(790.11131147,376.02396536)(790.16631142,376.01396537)(790.21631836,375.99397308)
\curveto(790.30631128,375.95396543)(790.35131123,375.88896549)(790.35131836,375.79897308)
\curveto(790.36131122,375.70896567)(790.34631124,375.61896576)(790.30631836,375.52897308)
\curveto(790.24631134,375.35896602)(790.1863114,375.1839662)(790.12631836,375.00397308)
\curveto(790.06631152,374.82396656)(789.99631159,374.64896673)(789.91631836,374.47897308)
\curveto(789.89631169,374.42896695)(789.8813117,374.378967)(789.87131836,374.32897308)
\curveto(789.86131172,374.28896709)(789.84631174,374.24396714)(789.82631836,374.19397308)
\curveto(789.74631184,374.02396736)(789.6813119,373.84896753)(789.63131836,373.66897308)
\curveto(789.581312,373.48896789)(789.51631207,373.30896807)(789.43631836,373.12897308)
\curveto(789.3863122,372.99896838)(789.33631225,372.86396852)(789.28631836,372.72397308)
\curveto(789.24631234,372.59396879)(789.19631239,372.46396892)(789.13631836,372.33397308)
\curveto(788.96631262,371.92396946)(788.81131277,371.50896987)(788.67131836,371.08897308)
\curveto(788.54131304,370.66897071)(788.39131319,370.25397113)(788.22131836,369.84397308)
\curveto(788.16131342,369.6839717)(788.10631348,369.52397186)(788.05631836,369.36397308)
\curveto(788.00631358,369.20397218)(787.94631364,369.04397234)(787.87631836,368.88397308)
\curveto(787.82631376,368.77397261)(787.7813138,368.66897271)(787.74131836,368.56897308)
\curveto(787.71131387,368.4789729)(787.64131394,368.40897297)(787.53131836,368.35897308)
\curveto(787.47131411,368.32897305)(787.40131418,368.31397307)(787.32131836,368.31397308)
\lineto(787.09631836,368.31397308)
\lineto(786.63131836,368.31397308)
\curveto(786.4813151,368.32397306)(786.37131521,368.37397301)(786.30131836,368.46397308)
\curveto(786.23131535,368.54397284)(786.1813154,368.63897274)(786.15131836,368.74897308)
\curveto(786.12131546,368.86897251)(786.0813155,368.9839724)(786.03131836,369.09397308)
\curveto(785.97131561,369.23397215)(785.91131567,369.378972)(785.85131836,369.52897308)
\curveto(785.80131578,369.68897169)(785.75131583,369.83897154)(785.70131836,369.97897308)
\curveto(785.6813159,370.02897135)(785.66631592,370.06897131)(785.65631836,370.09897308)
\curveto(785.64631594,370.13897124)(785.63131595,370.1839712)(785.61131836,370.23397308)
\curveto(785.41131617,370.71397067)(785.22631636,371.19897018)(785.05631836,371.68897308)
\curveto(784.89631669,372.1789692)(784.71631687,372.66396872)(784.51631836,373.14397308)
\curveto(784.45631713,373.30396808)(784.39631719,373.45896792)(784.33631836,373.60897308)
\curveto(784.2863173,373.76896761)(784.23131735,373.92896745)(784.17131836,374.08897308)
\lineto(784.11131836,374.23897308)
\curveto(784.10131748,374.29896708)(784.0863175,374.35396703)(784.06631836,374.40397308)
\curveto(783.9863176,374.57396681)(783.91631767,374.74396664)(783.85631836,374.91397308)
\curveto(783.80631778,375.0839663)(783.74631784,375.25396613)(783.67631836,375.42397308)
\curveto(783.65631793,375.4839659)(783.63131795,375.56396582)(783.60131836,375.66397308)
\curveto(783.57131801,375.76396562)(783.57631801,375.84896553)(783.61631836,375.91897308)
\curveto(783.66631792,375.96896541)(783.72631786,376.00396538)(783.79631836,376.02397308)
\curveto(783.86631772,376.02396536)(783.90631768,376.02896535)(783.91631836,376.03897308)
}
}
{
\newrgbcolor{curcolor}{0 0 0}
\pscustom[linestyle=none,fillstyle=solid,fillcolor=curcolor]
{
\newpath
\moveto(791.92631836,377.53897308)
\curveto(791.84631724,377.59896378)(791.80131728,377.70396368)(791.79131836,377.85397308)
\lineto(791.79131836,378.31897308)
\lineto(791.79131836,378.57397308)
\curveto(791.79131729,378.66396272)(791.80631728,378.73896264)(791.83631836,378.79897308)
\curveto(791.87631721,378.8789625)(791.95631713,378.93896244)(792.07631836,378.97897308)
\curveto(792.09631699,378.98896239)(792.11631697,378.98896239)(792.13631836,378.97897308)
\curveto(792.16631692,378.9789624)(792.19131689,378.9839624)(792.21131836,378.99397308)
\curveto(792.3813167,378.99396239)(792.54131654,378.98896239)(792.69131836,378.97897308)
\curveto(792.84131624,378.96896241)(792.94131614,378.90896247)(792.99131836,378.79897308)
\curveto(793.02131606,378.73896264)(793.03631605,378.66396272)(793.03631836,378.57397308)
\lineto(793.03631836,378.31897308)
\curveto(793.03631605,378.13896324)(793.03131605,377.96896341)(793.02131836,377.80897308)
\curveto(793.02131606,377.64896373)(792.95631613,377.54396384)(792.82631836,377.49397308)
\curveto(792.77631631,377.47396391)(792.72131636,377.46396392)(792.66131836,377.46397308)
\lineto(792.49631836,377.46397308)
\lineto(792.18131836,377.46397308)
\curveto(792.081317,377.46396392)(791.99631709,377.48896389)(791.92631836,377.53897308)
\moveto(793.03631836,369.03397308)
\lineto(793.03631836,368.71897308)
\curveto(793.04631604,368.61897276)(793.02631606,368.53897284)(792.97631836,368.47897308)
\curveto(792.94631614,368.41897296)(792.90131618,368.378973)(792.84131836,368.35897308)
\curveto(792.7813163,368.34897303)(792.71131637,368.33397305)(792.63131836,368.31397308)
\lineto(792.40631836,368.31397308)
\curveto(792.27631681,368.31397307)(792.16131692,368.31897306)(792.06131836,368.32897308)
\curveto(791.97131711,368.34897303)(791.90131718,368.39897298)(791.85131836,368.47897308)
\curveto(791.81131727,368.53897284)(791.79131729,368.61397277)(791.79131836,368.70397308)
\lineto(791.79131836,368.98897308)
\lineto(791.79131836,375.33397308)
\lineto(791.79131836,375.64897308)
\curveto(791.79131729,375.75896562)(791.81631727,375.84396554)(791.86631836,375.90397308)
\curveto(791.89631719,375.95396543)(791.93631715,375.9839654)(791.98631836,375.99397308)
\curveto(792.03631705,376.00396538)(792.09131699,376.01896536)(792.15131836,376.03897308)
\curveto(792.17131691,376.03896534)(792.19131689,376.03396535)(792.21131836,376.02397308)
\curveto(792.24131684,376.02396536)(792.26631682,376.02896535)(792.28631836,376.03897308)
\curveto(792.41631667,376.03896534)(792.54631654,376.03396535)(792.67631836,376.02397308)
\curveto(792.81631627,376.02396536)(792.91131617,375.9839654)(792.96131836,375.90397308)
\curveto(793.01131607,375.84396554)(793.03631605,375.76396562)(793.03631836,375.66397308)
\lineto(793.03631836,375.37897308)
\lineto(793.03631836,369.03397308)
}
}
{
\newrgbcolor{curcolor}{0 0 0}
\pscustom[linestyle=none,fillstyle=solid,fillcolor=curcolor]
{
\newpath
\moveto(801.94116211,369.12397308)
\lineto(801.94116211,368.73397308)
\curveto(801.94115423,368.61397277)(801.91615426,368.51397287)(801.86616211,368.43397308)
\curveto(801.81615436,368.36397302)(801.73115444,368.32397306)(801.61116211,368.31397308)
\lineto(801.26616211,368.31397308)
\curveto(801.20615497,368.31397307)(801.14615503,368.30897307)(801.08616211,368.29897308)
\curveto(801.03615514,368.29897308)(800.99115518,368.30897307)(800.95116211,368.32897308)
\curveto(800.86115531,368.34897303)(800.80115537,368.38897299)(800.77116211,368.44897308)
\curveto(800.73115544,368.49897288)(800.70615547,368.55897282)(800.69616211,368.62897308)
\curveto(800.69615548,368.69897268)(800.68115549,368.76897261)(800.65116211,368.83897308)
\curveto(800.64115553,368.85897252)(800.62615555,368.87397251)(800.60616211,368.88397308)
\curveto(800.59615558,368.90397248)(800.58115559,368.92397246)(800.56116211,368.94397308)
\curveto(800.46115571,368.95397243)(800.38115579,368.93397245)(800.32116211,368.88397308)
\curveto(800.2711559,368.83397255)(800.21615596,368.7839726)(800.15616211,368.73397308)
\curveto(799.95615622,368.5839728)(799.75615642,368.46897291)(799.55616211,368.38897308)
\curveto(799.3761568,368.30897307)(799.16615701,368.24897313)(798.92616211,368.20897308)
\curveto(798.69615748,368.16897321)(798.45615772,368.14897323)(798.20616211,368.14897308)
\curveto(797.96615821,368.13897324)(797.72615845,368.15397323)(797.48616211,368.19397308)
\curveto(797.24615893,368.22397316)(797.03615914,368.2789731)(796.85616211,368.35897308)
\curveto(796.33615984,368.5789728)(795.91616026,368.87397251)(795.59616211,369.24397308)
\curveto(795.2761609,369.62397176)(795.02616115,370.09397129)(794.84616211,370.65397308)
\curveto(794.80616137,370.74397064)(794.7761614,370.83397055)(794.75616211,370.92397308)
\curveto(794.74616143,371.02397036)(794.72616145,371.12397026)(794.69616211,371.22397308)
\curveto(794.68616149,371.27397011)(794.68116149,371.32397006)(794.68116211,371.37397308)
\curveto(794.68116149,371.42396996)(794.6761615,371.47396991)(794.66616211,371.52397308)
\curveto(794.64616153,371.57396981)(794.63616154,371.62396976)(794.63616211,371.67397308)
\curveto(794.64616153,371.73396965)(794.64616153,371.78896959)(794.63616211,371.83897308)
\lineto(794.63616211,371.98897308)
\curveto(794.61616156,372.03896934)(794.60616157,372.10396928)(794.60616211,372.18397308)
\curveto(794.60616157,372.26396912)(794.61616156,372.32896905)(794.63616211,372.37897308)
\lineto(794.63616211,372.54397308)
\curveto(794.65616152,372.61396877)(794.66116151,372.6839687)(794.65116211,372.75397308)
\curveto(794.65116152,372.83396855)(794.66116151,372.90896847)(794.68116211,372.97897308)
\curveto(794.69116148,373.02896835)(794.69616148,373.07396831)(794.69616211,373.11397308)
\curveto(794.69616148,373.15396823)(794.70116147,373.19896818)(794.71116211,373.24897308)
\curveto(794.74116143,373.34896803)(794.76616141,373.44396794)(794.78616211,373.53397308)
\curveto(794.80616137,373.63396775)(794.83116134,373.72896765)(794.86116211,373.81897308)
\curveto(794.99116118,374.19896718)(795.15616102,374.53896684)(795.35616211,374.83897308)
\curveto(795.56616061,375.14896623)(795.81616036,375.40396598)(796.10616211,375.60397308)
\curveto(796.2761599,375.72396566)(796.45115972,375.82396556)(796.63116211,375.90397308)
\curveto(796.82115935,375.9839654)(797.02615915,376.05396533)(797.24616211,376.11397308)
\curveto(797.31615886,376.12396526)(797.38115879,376.13396525)(797.44116211,376.14397308)
\curveto(797.51115866,376.15396523)(797.58115859,376.16896521)(797.65116211,376.18897308)
\lineto(797.80116211,376.18897308)
\curveto(797.88115829,376.20896517)(797.99615818,376.21896516)(798.14616211,376.21897308)
\curveto(798.30615787,376.21896516)(798.42615775,376.20896517)(798.50616211,376.18897308)
\curveto(798.54615763,376.1789652)(798.60115757,376.17396521)(798.67116211,376.17397308)
\curveto(798.78115739,376.14396524)(798.89115728,376.11896526)(799.00116211,376.09897308)
\curveto(799.11115706,376.08896529)(799.21615696,376.05896532)(799.31616211,376.00897308)
\curveto(799.46615671,375.94896543)(799.60615657,375.8839655)(799.73616211,375.81397308)
\curveto(799.8761563,375.74396564)(800.00615617,375.66396572)(800.12616211,375.57397308)
\curveto(800.18615599,375.52396586)(800.24615593,375.46896591)(800.30616211,375.40897308)
\curveto(800.3761558,375.35896602)(800.46615571,375.34396604)(800.57616211,375.36397308)
\curveto(800.59615558,375.39396599)(800.61115556,375.41896596)(800.62116211,375.43897308)
\curveto(800.64115553,375.45896592)(800.65615552,375.48896589)(800.66616211,375.52897308)
\curveto(800.69615548,375.61896576)(800.70615547,375.73396565)(800.69616211,375.87397308)
\lineto(800.69616211,376.24897308)
\lineto(800.69616211,377.97397308)
\lineto(800.69616211,378.43897308)
\curveto(800.69615548,378.61896276)(800.72115545,378.74896263)(800.77116211,378.82897308)
\curveto(800.81115536,378.89896248)(800.8711553,378.94396244)(800.95116211,378.96397308)
\curveto(800.9711552,378.96396242)(800.99615518,378.96396242)(801.02616211,378.96397308)
\curveto(801.05615512,378.97396241)(801.08115509,378.9789624)(801.10116211,378.97897308)
\curveto(801.24115493,378.98896239)(801.38615479,378.98896239)(801.53616211,378.97897308)
\curveto(801.69615448,378.9789624)(801.80615437,378.93896244)(801.86616211,378.85897308)
\curveto(801.91615426,378.7789626)(801.94115423,378.6789627)(801.94116211,378.55897308)
\lineto(801.94116211,378.18397308)
\lineto(801.94116211,369.12397308)
\moveto(800.72616211,371.95897308)
\curveto(800.74615543,372.00896937)(800.75615542,372.07396931)(800.75616211,372.15397308)
\curveto(800.75615542,372.24396914)(800.74615543,372.31396907)(800.72616211,372.36397308)
\lineto(800.72616211,372.58897308)
\curveto(800.70615547,372.6789687)(800.69115548,372.76896861)(800.68116211,372.85897308)
\curveto(800.6711555,372.95896842)(800.65115552,373.04896833)(800.62116211,373.12897308)
\curveto(800.60115557,373.20896817)(800.58115559,373.2839681)(800.56116211,373.35397308)
\curveto(800.55115562,373.42396796)(800.53115564,373.49396789)(800.50116211,373.56397308)
\curveto(800.38115579,373.86396752)(800.22615595,374.12896725)(800.03616211,374.35897308)
\curveto(799.84615633,374.58896679)(799.60615657,374.76896661)(799.31616211,374.89897308)
\curveto(799.21615696,374.94896643)(799.11115706,374.9839664)(799.00116211,375.00397308)
\curveto(798.90115727,375.03396635)(798.79115738,375.05896632)(798.67116211,375.07897308)
\curveto(798.59115758,375.09896628)(798.50115767,375.10896627)(798.40116211,375.10897308)
\lineto(798.13116211,375.10897308)
\curveto(798.08115809,375.09896628)(798.03615814,375.08896629)(797.99616211,375.07897308)
\lineto(797.86116211,375.07897308)
\curveto(797.78115839,375.05896632)(797.69615848,375.03896634)(797.60616211,375.01897308)
\curveto(797.52615865,374.99896638)(797.44615873,374.97396641)(797.36616211,374.94397308)
\curveto(797.04615913,374.80396658)(796.78615939,374.59896678)(796.58616211,374.32897308)
\curveto(796.39615978,374.06896731)(796.24115993,373.76396762)(796.12116211,373.41397308)
\curveto(796.08116009,373.30396808)(796.05116012,373.18896819)(796.03116211,373.06897308)
\curveto(796.02116015,372.95896842)(796.00616017,372.84896853)(795.98616211,372.73897308)
\curveto(795.98616019,372.69896868)(795.98116019,372.65896872)(795.97116211,372.61897308)
\lineto(795.97116211,372.51397308)
\curveto(795.95116022,372.46396892)(795.94116023,372.40896897)(795.94116211,372.34897308)
\curveto(795.95116022,372.28896909)(795.95616022,372.23396915)(795.95616211,372.18397308)
\lineto(795.95616211,371.85397308)
\curveto(795.95616022,371.75396963)(795.96616021,371.65896972)(795.98616211,371.56897308)
\curveto(795.99616018,371.53896984)(796.00116017,371.48896989)(796.00116211,371.41897308)
\curveto(796.02116015,371.34897003)(796.03616014,371.2789701)(796.04616211,371.20897308)
\lineto(796.10616211,370.99897308)
\curveto(796.21615996,370.64897073)(796.36615981,370.34897103)(796.55616211,370.09897308)
\curveto(796.74615943,369.84897153)(796.98615919,369.64397174)(797.27616211,369.48397308)
\curveto(797.36615881,369.43397195)(797.45615872,369.39397199)(797.54616211,369.36397308)
\curveto(797.63615854,369.33397205)(797.73615844,369.30397208)(797.84616211,369.27397308)
\curveto(797.89615828,369.25397213)(797.94615823,369.24897213)(797.99616211,369.25897308)
\curveto(798.05615812,369.26897211)(798.11115806,369.26397212)(798.16116211,369.24397308)
\curveto(798.20115797,369.23397215)(798.24115793,369.22897215)(798.28116211,369.22897308)
\lineto(798.41616211,369.22897308)
\lineto(798.55116211,369.22897308)
\curveto(798.58115759,369.23897214)(798.63115754,369.24397214)(798.70116211,369.24397308)
\curveto(798.78115739,369.26397212)(798.86115731,369.2789721)(798.94116211,369.28897308)
\curveto(799.02115715,369.30897207)(799.09615708,369.33397205)(799.16616211,369.36397308)
\curveto(799.49615668,369.50397188)(799.76115641,369.6789717)(799.96116211,369.88897308)
\curveto(800.171156,370.10897127)(800.34615583,370.383971)(800.48616211,370.71397308)
\curveto(800.53615564,370.82397056)(800.5711556,370.93397045)(800.59116211,371.04397308)
\curveto(800.61115556,371.15397023)(800.63615554,371.26397012)(800.66616211,371.37397308)
\curveto(800.68615549,371.41396997)(800.69615548,371.44896993)(800.69616211,371.47897308)
\curveto(800.69615548,371.51896986)(800.70115547,371.55896982)(800.71116211,371.59897308)
\curveto(800.72115545,371.65896972)(800.72115545,371.71896966)(800.71116211,371.77897308)
\curveto(800.71115546,371.83896954)(800.71615546,371.89896948)(800.72616211,371.95897308)
}
}
{
\newrgbcolor{curcolor}{0 0 0}
\pscustom[linestyle=none,fillstyle=solid,fillcolor=curcolor]
{
\newpath
\moveto(810.77241211,368.86897308)
\curveto(810.80240428,368.70897267)(810.78740429,368.57397281)(810.72741211,368.46397308)
\curveto(810.66740441,368.36397302)(810.58740449,368.28897309)(810.48741211,368.23897308)
\curveto(810.43740464,368.21897316)(810.3824047,368.20897317)(810.32241211,368.20897308)
\curveto(810.27240481,368.20897317)(810.21740486,368.19897318)(810.15741211,368.17897308)
\curveto(809.93740514,368.12897325)(809.71740536,368.14397324)(809.49741211,368.22397308)
\curveto(809.28740579,368.29397309)(809.14240594,368.383973)(809.06241211,368.49397308)
\curveto(809.01240607,368.56397282)(808.96740611,368.64397274)(808.92741211,368.73397308)
\curveto(808.88740619,368.83397255)(808.83740624,368.91397247)(808.77741211,368.97397308)
\curveto(808.75740632,368.99397239)(808.73240635,369.01397237)(808.70241211,369.03397308)
\curveto(808.6824064,369.05397233)(808.65240643,369.05897232)(808.61241211,369.04897308)
\curveto(808.50240658,369.01897236)(808.39740668,368.96397242)(808.29741211,368.88397308)
\curveto(808.20740687,368.80397258)(808.11740696,368.73397265)(808.02741211,368.67397308)
\curveto(807.89740718,368.59397279)(807.75740732,368.51897286)(807.60741211,368.44897308)
\curveto(807.45740762,368.38897299)(807.29740778,368.33397305)(807.12741211,368.28397308)
\curveto(807.02740805,368.25397313)(806.91740816,368.23397315)(806.79741211,368.22397308)
\curveto(806.68740839,368.21397317)(806.5774085,368.19897318)(806.46741211,368.17897308)
\curveto(806.41740866,368.16897321)(806.37240871,368.16397322)(806.33241211,368.16397308)
\lineto(806.22741211,368.16397308)
\curveto(806.11740896,368.14397324)(806.01240907,368.14397324)(805.91241211,368.16397308)
\lineto(805.77741211,368.16397308)
\curveto(805.72740935,368.17397321)(805.6774094,368.1789732)(805.62741211,368.17897308)
\curveto(805.5774095,368.1789732)(805.53240955,368.18897319)(805.49241211,368.20897308)
\curveto(805.45240963,368.21897316)(805.41740966,368.22397316)(805.38741211,368.22397308)
\curveto(805.36740971,368.21397317)(805.34240974,368.21397317)(805.31241211,368.22397308)
\lineto(805.07241211,368.28397308)
\curveto(804.99241009,368.29397309)(804.91741016,368.31397307)(804.84741211,368.34397308)
\curveto(804.54741053,368.47397291)(804.30241078,368.61897276)(804.11241211,368.77897308)
\curveto(803.93241115,368.94897243)(803.7824113,369.1839722)(803.66241211,369.48397308)
\curveto(803.57241151,369.70397168)(803.52741155,369.96897141)(803.52741211,370.27897308)
\lineto(803.52741211,370.59397308)
\curveto(803.53741154,370.64397074)(803.54241154,370.69397069)(803.54241211,370.74397308)
\lineto(803.57241211,370.92397308)
\lineto(803.69241211,371.25397308)
\curveto(803.73241135,371.36397002)(803.7824113,371.46396992)(803.84241211,371.55397308)
\curveto(804.02241106,371.84396954)(804.26741081,372.05896932)(804.57741211,372.19897308)
\curveto(804.88741019,372.33896904)(805.22740985,372.46396892)(805.59741211,372.57397308)
\curveto(805.73740934,372.61396877)(805.8824092,372.64396874)(806.03241211,372.66397308)
\curveto(806.1824089,372.6839687)(806.33240875,372.70896867)(806.48241211,372.73897308)
\curveto(806.55240853,372.75896862)(806.61740846,372.76896861)(806.67741211,372.76897308)
\curveto(806.74740833,372.76896861)(806.82240826,372.7789686)(806.90241211,372.79897308)
\curveto(806.97240811,372.81896856)(807.04240804,372.82896855)(807.11241211,372.82897308)
\curveto(807.1824079,372.83896854)(807.25740782,372.85396853)(807.33741211,372.87397308)
\curveto(807.58740749,372.93396845)(807.82240726,372.9839684)(808.04241211,373.02397308)
\curveto(808.26240682,373.07396831)(808.43740664,373.18896819)(808.56741211,373.36897308)
\curveto(808.62740645,373.44896793)(808.6774064,373.54896783)(808.71741211,373.66897308)
\curveto(808.75740632,373.79896758)(808.75740632,373.93896744)(808.71741211,374.08897308)
\curveto(808.65740642,374.32896705)(808.56740651,374.51896686)(808.44741211,374.65897308)
\curveto(808.33740674,374.79896658)(808.1774069,374.90896647)(807.96741211,374.98897308)
\curveto(807.84740723,375.03896634)(807.70240738,375.07396631)(807.53241211,375.09397308)
\curveto(807.37240771,375.11396627)(807.20240788,375.12396626)(807.02241211,375.12397308)
\curveto(806.84240824,375.12396626)(806.66740841,375.11396627)(806.49741211,375.09397308)
\curveto(806.32740875,375.07396631)(806.1824089,375.04396634)(806.06241211,375.00397308)
\curveto(805.89240919,374.94396644)(805.72740935,374.85896652)(805.56741211,374.74897308)
\curveto(805.48740959,374.68896669)(805.41240967,374.60896677)(805.34241211,374.50897308)
\curveto(805.2824098,374.41896696)(805.22740985,374.31896706)(805.17741211,374.20897308)
\curveto(805.14740993,374.12896725)(805.11740996,374.04396734)(805.08741211,373.95397308)
\curveto(805.06741001,373.86396752)(805.02241006,373.79396759)(804.95241211,373.74397308)
\curveto(804.91241017,373.71396767)(804.84241024,373.68896769)(804.74241211,373.66897308)
\curveto(804.65241043,373.65896772)(804.55741052,373.65396773)(804.45741211,373.65397308)
\curveto(804.35741072,373.65396773)(804.25741082,373.65896772)(804.15741211,373.66897308)
\curveto(804.06741101,373.68896769)(804.00241108,373.71396767)(803.96241211,373.74397308)
\curveto(803.92241116,373.77396761)(803.89241119,373.82396756)(803.87241211,373.89397308)
\curveto(803.85241123,373.96396742)(803.85241123,374.03896734)(803.87241211,374.11897308)
\curveto(803.90241118,374.24896713)(803.93241115,374.36896701)(803.96241211,374.47897308)
\curveto(804.00241108,374.59896678)(804.04741103,374.71396667)(804.09741211,374.82397308)
\curveto(804.28741079,375.17396621)(804.52741055,375.44396594)(804.81741211,375.63397308)
\curveto(805.10740997,375.83396555)(805.46740961,375.99396539)(805.89741211,376.11397308)
\curveto(805.99740908,376.13396525)(806.09740898,376.14896523)(806.19741211,376.15897308)
\curveto(806.30740877,376.16896521)(806.41740866,376.1839652)(806.52741211,376.20397308)
\curveto(806.56740851,376.21396517)(806.63240845,376.21396517)(806.72241211,376.20397308)
\curveto(806.81240827,376.20396518)(806.86740821,376.21396517)(806.88741211,376.23397308)
\curveto(807.58740749,376.24396514)(808.19740688,376.16396522)(808.71741211,375.99397308)
\curveto(809.23740584,375.82396556)(809.60240548,375.49896588)(809.81241211,375.01897308)
\curveto(809.90240518,374.81896656)(809.95240513,374.5839668)(809.96241211,374.31397308)
\curveto(809.9824051,374.05396733)(809.99240509,373.7789676)(809.99241211,373.48897308)
\lineto(809.99241211,370.17397308)
\curveto(809.99240509,370.03397135)(809.99740508,369.89897148)(810.00741211,369.76897308)
\curveto(810.01740506,369.63897174)(810.04740503,369.53397185)(810.09741211,369.45397308)
\curveto(810.14740493,369.383972)(810.21240487,369.33397205)(810.29241211,369.30397308)
\curveto(810.3824047,369.26397212)(810.46740461,369.23397215)(810.54741211,369.21397308)
\curveto(810.62740445,369.20397218)(810.68740439,369.15897222)(810.72741211,369.07897308)
\curveto(810.74740433,369.04897233)(810.75740432,369.01897236)(810.75741211,368.98897308)
\curveto(810.75740432,368.95897242)(810.76240432,368.91897246)(810.77241211,368.86897308)
\moveto(808.62741211,370.53397308)
\curveto(808.68740639,370.67397071)(808.71740636,370.83397055)(808.71741211,371.01397308)
\curveto(808.72740635,371.20397018)(808.73240635,371.39896998)(808.73241211,371.59897308)
\curveto(808.73240635,371.70896967)(808.72740635,371.80896957)(808.71741211,371.89897308)
\curveto(808.70740637,371.98896939)(808.66740641,372.05896932)(808.59741211,372.10897308)
\curveto(808.56740651,372.12896925)(808.49740658,372.13896924)(808.38741211,372.13897308)
\curveto(808.36740671,372.11896926)(808.33240675,372.10896927)(808.28241211,372.10897308)
\curveto(808.23240685,372.10896927)(808.18740689,372.09896928)(808.14741211,372.07897308)
\curveto(808.06740701,372.05896932)(807.9774071,372.03896934)(807.87741211,372.01897308)
\lineto(807.57741211,371.95897308)
\curveto(807.54740753,371.95896942)(807.51240757,371.95396943)(807.47241211,371.94397308)
\lineto(807.36741211,371.94397308)
\curveto(807.21740786,371.90396948)(807.05240803,371.8789695)(806.87241211,371.86897308)
\curveto(806.70240838,371.86896951)(806.54240854,371.84896953)(806.39241211,371.80897308)
\curveto(806.31240877,371.78896959)(806.23740884,371.76896961)(806.16741211,371.74897308)
\curveto(806.10740897,371.73896964)(806.03740904,371.72396966)(805.95741211,371.70397308)
\curveto(805.79740928,371.65396973)(805.64740943,371.58896979)(805.50741211,371.50897308)
\curveto(805.36740971,371.43896994)(805.24740983,371.34897003)(805.14741211,371.23897308)
\curveto(805.04741003,371.12897025)(804.97241011,370.99397039)(804.92241211,370.83397308)
\curveto(804.87241021,370.6839707)(804.85241023,370.49897088)(804.86241211,370.27897308)
\curveto(804.86241022,370.1789712)(804.8774102,370.0839713)(804.90741211,369.99397308)
\curveto(804.94741013,369.91397147)(804.99241009,369.83897154)(805.04241211,369.76897308)
\curveto(805.12240996,369.65897172)(805.22740985,369.56397182)(805.35741211,369.48397308)
\curveto(805.48740959,369.41397197)(805.62740945,369.35397203)(805.77741211,369.30397308)
\curveto(805.82740925,369.29397209)(805.8774092,369.28897209)(805.92741211,369.28897308)
\curveto(805.9774091,369.28897209)(806.02740905,369.2839721)(806.07741211,369.27397308)
\curveto(806.14740893,369.25397213)(806.23240885,369.23897214)(806.33241211,369.22897308)
\curveto(806.44240864,369.22897215)(806.53240855,369.23897214)(806.60241211,369.25897308)
\curveto(806.66240842,369.2789721)(806.72240836,369.2839721)(806.78241211,369.27397308)
\curveto(806.84240824,369.27397211)(806.90240818,369.2839721)(806.96241211,369.30397308)
\curveto(807.04240804,369.32397206)(807.11740796,369.33897204)(807.18741211,369.34897308)
\curveto(807.26740781,369.35897202)(807.34240774,369.378972)(807.41241211,369.40897308)
\curveto(807.70240738,369.52897185)(807.94740713,369.67397171)(808.14741211,369.84397308)
\curveto(808.35740672,370.01397137)(808.51740656,370.24397114)(808.62741211,370.53397308)
}
}
{
\newrgbcolor{curcolor}{0 0 0}
\pscustom[linestyle=none,fillstyle=solid,fillcolor=curcolor]
{
\newpath
\moveto(818.90405273,369.12397308)
\lineto(818.90405273,368.73397308)
\curveto(818.90404486,368.61397277)(818.87904488,368.51397287)(818.82905273,368.43397308)
\curveto(818.77904498,368.36397302)(818.69404507,368.32397306)(818.57405273,368.31397308)
\lineto(818.22905273,368.31397308)
\curveto(818.16904559,368.31397307)(818.10904565,368.30897307)(818.04905273,368.29897308)
\curveto(817.99904576,368.29897308)(817.95404581,368.30897307)(817.91405273,368.32897308)
\curveto(817.82404594,368.34897303)(817.764046,368.38897299)(817.73405273,368.44897308)
\curveto(817.69404607,368.49897288)(817.66904609,368.55897282)(817.65905273,368.62897308)
\curveto(817.6590461,368.69897268)(817.64404612,368.76897261)(817.61405273,368.83897308)
\curveto(817.60404616,368.85897252)(817.58904617,368.87397251)(817.56905273,368.88397308)
\curveto(817.5590462,368.90397248)(817.54404622,368.92397246)(817.52405273,368.94397308)
\curveto(817.42404634,368.95397243)(817.34404642,368.93397245)(817.28405273,368.88397308)
\curveto(817.23404653,368.83397255)(817.17904658,368.7839726)(817.11905273,368.73397308)
\curveto(816.91904684,368.5839728)(816.71904704,368.46897291)(816.51905273,368.38897308)
\curveto(816.33904742,368.30897307)(816.12904763,368.24897313)(815.88905273,368.20897308)
\curveto(815.6590481,368.16897321)(815.41904834,368.14897323)(815.16905273,368.14897308)
\curveto(814.92904883,368.13897324)(814.68904907,368.15397323)(814.44905273,368.19397308)
\curveto(814.20904955,368.22397316)(813.99904976,368.2789731)(813.81905273,368.35897308)
\curveto(813.29905046,368.5789728)(812.87905088,368.87397251)(812.55905273,369.24397308)
\curveto(812.23905152,369.62397176)(811.98905177,370.09397129)(811.80905273,370.65397308)
\curveto(811.76905199,370.74397064)(811.73905202,370.83397055)(811.71905273,370.92397308)
\curveto(811.70905205,371.02397036)(811.68905207,371.12397026)(811.65905273,371.22397308)
\curveto(811.64905211,371.27397011)(811.64405212,371.32397006)(811.64405273,371.37397308)
\curveto(811.64405212,371.42396996)(811.63905212,371.47396991)(811.62905273,371.52397308)
\curveto(811.60905215,371.57396981)(811.59905216,371.62396976)(811.59905273,371.67397308)
\curveto(811.60905215,371.73396965)(811.60905215,371.78896959)(811.59905273,371.83897308)
\lineto(811.59905273,371.98897308)
\curveto(811.57905218,372.03896934)(811.56905219,372.10396928)(811.56905273,372.18397308)
\curveto(811.56905219,372.26396912)(811.57905218,372.32896905)(811.59905273,372.37897308)
\lineto(811.59905273,372.54397308)
\curveto(811.61905214,372.61396877)(811.62405214,372.6839687)(811.61405273,372.75397308)
\curveto(811.61405215,372.83396855)(811.62405214,372.90896847)(811.64405273,372.97897308)
\curveto(811.65405211,373.02896835)(811.6590521,373.07396831)(811.65905273,373.11397308)
\curveto(811.6590521,373.15396823)(811.6640521,373.19896818)(811.67405273,373.24897308)
\curveto(811.70405206,373.34896803)(811.72905203,373.44396794)(811.74905273,373.53397308)
\curveto(811.76905199,373.63396775)(811.79405197,373.72896765)(811.82405273,373.81897308)
\curveto(811.95405181,374.19896718)(812.11905164,374.53896684)(812.31905273,374.83897308)
\curveto(812.52905123,375.14896623)(812.77905098,375.40396598)(813.06905273,375.60397308)
\curveto(813.23905052,375.72396566)(813.41405035,375.82396556)(813.59405273,375.90397308)
\curveto(813.78404998,375.9839654)(813.98904977,376.05396533)(814.20905273,376.11397308)
\curveto(814.27904948,376.12396526)(814.34404942,376.13396525)(814.40405273,376.14397308)
\curveto(814.47404929,376.15396523)(814.54404922,376.16896521)(814.61405273,376.18897308)
\lineto(814.76405273,376.18897308)
\curveto(814.84404892,376.20896517)(814.9590488,376.21896516)(815.10905273,376.21897308)
\curveto(815.26904849,376.21896516)(815.38904837,376.20896517)(815.46905273,376.18897308)
\curveto(815.50904825,376.1789652)(815.5640482,376.17396521)(815.63405273,376.17397308)
\curveto(815.74404802,376.14396524)(815.85404791,376.11896526)(815.96405273,376.09897308)
\curveto(816.07404769,376.08896529)(816.17904758,376.05896532)(816.27905273,376.00897308)
\curveto(816.42904733,375.94896543)(816.56904719,375.8839655)(816.69905273,375.81397308)
\curveto(816.83904692,375.74396564)(816.96904679,375.66396572)(817.08905273,375.57397308)
\curveto(817.14904661,375.52396586)(817.20904655,375.46896591)(817.26905273,375.40897308)
\curveto(817.33904642,375.35896602)(817.42904633,375.34396604)(817.53905273,375.36397308)
\curveto(817.5590462,375.39396599)(817.57404619,375.41896596)(817.58405273,375.43897308)
\curveto(817.60404616,375.45896592)(817.61904614,375.48896589)(817.62905273,375.52897308)
\curveto(817.6590461,375.61896576)(817.66904609,375.73396565)(817.65905273,375.87397308)
\lineto(817.65905273,376.24897308)
\lineto(817.65905273,377.97397308)
\lineto(817.65905273,378.43897308)
\curveto(817.6590461,378.61896276)(817.68404608,378.74896263)(817.73405273,378.82897308)
\curveto(817.77404599,378.89896248)(817.83404593,378.94396244)(817.91405273,378.96397308)
\curveto(817.93404583,378.96396242)(817.9590458,378.96396242)(817.98905273,378.96397308)
\curveto(818.01904574,378.97396241)(818.04404572,378.9789624)(818.06405273,378.97897308)
\curveto(818.20404556,378.98896239)(818.34904541,378.98896239)(818.49905273,378.97897308)
\curveto(818.6590451,378.9789624)(818.76904499,378.93896244)(818.82905273,378.85897308)
\curveto(818.87904488,378.7789626)(818.90404486,378.6789627)(818.90405273,378.55897308)
\lineto(818.90405273,378.18397308)
\lineto(818.90405273,369.12397308)
\moveto(817.68905273,371.95897308)
\curveto(817.70904605,372.00896937)(817.71904604,372.07396931)(817.71905273,372.15397308)
\curveto(817.71904604,372.24396914)(817.70904605,372.31396907)(817.68905273,372.36397308)
\lineto(817.68905273,372.58897308)
\curveto(817.66904609,372.6789687)(817.65404611,372.76896861)(817.64405273,372.85897308)
\curveto(817.63404613,372.95896842)(817.61404615,373.04896833)(817.58405273,373.12897308)
\curveto(817.5640462,373.20896817)(817.54404622,373.2839681)(817.52405273,373.35397308)
\curveto(817.51404625,373.42396796)(817.49404627,373.49396789)(817.46405273,373.56397308)
\curveto(817.34404642,373.86396752)(817.18904657,374.12896725)(816.99905273,374.35897308)
\curveto(816.80904695,374.58896679)(816.56904719,374.76896661)(816.27905273,374.89897308)
\curveto(816.17904758,374.94896643)(816.07404769,374.9839664)(815.96405273,375.00397308)
\curveto(815.8640479,375.03396635)(815.75404801,375.05896632)(815.63405273,375.07897308)
\curveto(815.55404821,375.09896628)(815.4640483,375.10896627)(815.36405273,375.10897308)
\lineto(815.09405273,375.10897308)
\curveto(815.04404872,375.09896628)(814.99904876,375.08896629)(814.95905273,375.07897308)
\lineto(814.82405273,375.07897308)
\curveto(814.74404902,375.05896632)(814.6590491,375.03896634)(814.56905273,375.01897308)
\curveto(814.48904927,374.99896638)(814.40904935,374.97396641)(814.32905273,374.94397308)
\curveto(814.00904975,374.80396658)(813.74905001,374.59896678)(813.54905273,374.32897308)
\curveto(813.3590504,374.06896731)(813.20405056,373.76396762)(813.08405273,373.41397308)
\curveto(813.04405072,373.30396808)(813.01405075,373.18896819)(812.99405273,373.06897308)
\curveto(812.98405078,372.95896842)(812.96905079,372.84896853)(812.94905273,372.73897308)
\curveto(812.94905081,372.69896868)(812.94405082,372.65896872)(812.93405273,372.61897308)
\lineto(812.93405273,372.51397308)
\curveto(812.91405085,372.46396892)(812.90405086,372.40896897)(812.90405273,372.34897308)
\curveto(812.91405085,372.28896909)(812.91905084,372.23396915)(812.91905273,372.18397308)
\lineto(812.91905273,371.85397308)
\curveto(812.91905084,371.75396963)(812.92905083,371.65896972)(812.94905273,371.56897308)
\curveto(812.9590508,371.53896984)(812.9640508,371.48896989)(812.96405273,371.41897308)
\curveto(812.98405078,371.34897003)(812.99905076,371.2789701)(813.00905273,371.20897308)
\lineto(813.06905273,370.99897308)
\curveto(813.17905058,370.64897073)(813.32905043,370.34897103)(813.51905273,370.09897308)
\curveto(813.70905005,369.84897153)(813.94904981,369.64397174)(814.23905273,369.48397308)
\curveto(814.32904943,369.43397195)(814.41904934,369.39397199)(814.50905273,369.36397308)
\curveto(814.59904916,369.33397205)(814.69904906,369.30397208)(814.80905273,369.27397308)
\curveto(814.8590489,369.25397213)(814.90904885,369.24897213)(814.95905273,369.25897308)
\curveto(815.01904874,369.26897211)(815.07404869,369.26397212)(815.12405273,369.24397308)
\curveto(815.1640486,369.23397215)(815.20404856,369.22897215)(815.24405273,369.22897308)
\lineto(815.37905273,369.22897308)
\lineto(815.51405273,369.22897308)
\curveto(815.54404822,369.23897214)(815.59404817,369.24397214)(815.66405273,369.24397308)
\curveto(815.74404802,369.26397212)(815.82404794,369.2789721)(815.90405273,369.28897308)
\curveto(815.98404778,369.30897207)(816.0590477,369.33397205)(816.12905273,369.36397308)
\curveto(816.4590473,369.50397188)(816.72404704,369.6789717)(816.92405273,369.88897308)
\curveto(817.13404663,370.10897127)(817.30904645,370.383971)(817.44905273,370.71397308)
\curveto(817.49904626,370.82397056)(817.53404623,370.93397045)(817.55405273,371.04397308)
\curveto(817.57404619,371.15397023)(817.59904616,371.26397012)(817.62905273,371.37397308)
\curveto(817.64904611,371.41396997)(817.6590461,371.44896993)(817.65905273,371.47897308)
\curveto(817.6590461,371.51896986)(817.6640461,371.55896982)(817.67405273,371.59897308)
\curveto(817.68404608,371.65896972)(817.68404608,371.71896966)(817.67405273,371.77897308)
\curveto(817.67404609,371.83896954)(817.67904608,371.89896948)(817.68905273,371.95897308)
}
}
{
\newrgbcolor{curcolor}{0 0 0}
\pscustom[linestyle=none,fillstyle=solid,fillcolor=curcolor]
{
\newpath
\moveto(766.46144531,352.84265076)
\curveto(766.48143619,352.76264298)(766.49143618,352.65264309)(766.49144531,352.51265076)
\curveto(766.49143618,352.38264336)(766.48143619,352.28264346)(766.46144531,352.21265076)
\curveto(766.44143623,352.1426436)(766.43643623,352.07764366)(766.44644531,352.01765076)
\curveto(766.45643621,351.95764378)(766.45143622,351.89264385)(766.43144531,351.82265076)
\curveto(766.41143626,351.76264398)(766.39643627,351.69764404)(766.38644531,351.62765076)
\curveto(766.37643629,351.56764417)(766.36143631,351.50764423)(766.34144531,351.44765076)
\curveto(766.32143635,351.36764437)(766.29643637,351.29264445)(766.26644531,351.22265076)
\curveto(766.24643642,351.15264459)(766.22143645,351.08264466)(766.19144531,351.01265076)
\curveto(766.1714365,350.98264476)(766.15643651,350.95264479)(766.14644531,350.92265076)
\curveto(766.14643652,350.90264484)(766.13643653,350.88264486)(766.11644531,350.86265076)
\curveto(766.00643666,350.66264508)(765.88643678,350.48264526)(765.75644531,350.32265076)
\curveto(765.73643693,350.28264546)(765.70143697,350.2426455)(765.65144531,350.20265076)
\curveto(765.61143706,350.16264558)(765.57643709,350.13264561)(765.54644531,350.11265076)
\curveto(765.50643716,350.09264565)(765.4714372,350.06264568)(765.44144531,350.02265076)
\curveto(765.41143726,349.99264575)(765.38143729,349.96764577)(765.35144531,349.94765076)
\lineto(765.03644531,349.76765076)
\curveto(764.92643774,349.68764605)(764.79643787,349.62764611)(764.64644531,349.58765076)
\lineto(764.19644531,349.46765076)
\curveto(764.11643855,349.44764629)(764.03643863,349.43264631)(763.95644531,349.42265076)
\curveto(763.87643879,349.42264632)(763.79643887,349.41264633)(763.71644531,349.39265076)
\curveto(763.67643899,349.38264636)(763.63643903,349.37764636)(763.59644531,349.37765076)
\curveto(763.5664391,349.38764635)(763.53643913,349.38764635)(763.50644531,349.37765076)
\curveto(763.45643921,349.36764637)(763.40643926,349.36764637)(763.35644531,349.37765076)
\curveto(763.31643935,349.38764635)(763.2714394,349.38764635)(763.22144531,349.37765076)
\lineto(760.95644531,349.37765076)
\lineto(760.46144531,349.37765076)
\curveto(760.29144238,349.38764635)(760.16144251,349.35764638)(760.07144531,349.28765076)
\curveto(759.96144271,349.20764653)(759.90644276,349.06264668)(759.90644531,348.85265076)
\curveto(759.91644275,348.6426471)(759.92144275,348.44764729)(759.92144531,348.26765076)
\lineto(759.92144531,346.06265076)
\lineto(759.92144531,345.56765076)
\curveto(759.93144274,345.37765036)(759.91144276,345.2426505)(759.86144531,345.16265076)
\curveto(759.82144285,345.10265064)(759.7714429,345.06265068)(759.71144531,345.04265076)
\curveto(759.66144301,345.03265071)(759.59644307,345.01765072)(759.51644531,344.99765076)
\lineto(759.24644531,344.99765076)
\curveto(759.09644357,344.99765074)(758.96144371,345.00265074)(758.84144531,345.01265076)
\curveto(758.72144395,345.02265072)(758.63644403,345.07265067)(758.58644531,345.16265076)
\curveto(758.54644412,345.22265052)(758.52644414,345.30265044)(758.52644531,345.40265076)
\lineto(758.52644531,345.71765076)
\lineto(758.52644531,354.82265076)
\curveto(758.52644414,354.93264081)(758.52144415,355.05264069)(758.51144531,355.18265076)
\curveto(758.51144416,355.32264042)(758.53644413,355.43264031)(758.58644531,355.51265076)
\curveto(758.62644404,355.57264017)(758.70144397,355.62264012)(758.81144531,355.66265076)
\curveto(758.83144384,355.67264007)(758.85144382,355.67264007)(758.87144531,355.66265076)
\curveto(758.89144378,355.66264008)(758.91144376,355.66764007)(758.93144531,355.67765076)
\lineto(762.33644531,355.67765076)
\curveto(762.71643995,355.67764006)(763.08643958,355.67264007)(763.44644531,355.66265076)
\curveto(763.81643885,355.66264008)(764.14643852,355.61764012)(764.43644531,355.52765076)
\curveto(764.88643778,355.37764036)(765.25143742,355.18264056)(765.53144531,354.94265076)
\curveto(765.81143686,354.70264104)(766.04143663,354.37264137)(766.22144531,353.95265076)
\curveto(766.2714364,353.8426419)(766.30643636,353.72764201)(766.32644531,353.60765076)
\curveto(766.35643631,353.48764225)(766.39143628,353.36264238)(766.43144531,353.23265076)
\curveto(766.45143622,353.16264258)(766.45643621,353.09764264)(766.44644531,353.03765076)
\curveto(766.43643623,352.97764276)(766.44143623,352.91264283)(766.46144531,352.84265076)
\moveto(765.05144531,352.30265076)
\curveto(765.09143758,352.4426433)(765.09643757,352.60264314)(765.06644531,352.78265076)
\curveto(765.03643763,352.97264277)(765.00643766,353.12264262)(764.97644531,353.23265076)
\curveto(764.87643779,353.51264223)(764.74143793,353.73264201)(764.57144531,353.89265076)
\curveto(764.41143826,354.06264168)(764.20143847,354.20264154)(763.94144531,354.31265076)
\curveto(763.72143895,354.40264134)(763.4664392,354.45764128)(763.17644531,354.47765076)
\curveto(762.89643977,354.49764124)(762.60144007,354.50764123)(762.29144531,354.50765076)
\lineto(760.35644531,354.50765076)
\curveto(760.33644233,354.49764124)(760.31144236,354.49264125)(760.28144531,354.49265076)
\curveto(760.26144241,354.49264125)(760.23644243,354.48764125)(760.20644531,354.47765076)
\curveto(760.08644258,354.44764129)(760.00644266,354.38264136)(759.96644531,354.28265076)
\curveto(759.92644274,354.18264156)(759.90644276,354.04764169)(759.90644531,353.87765076)
\curveto(759.91644275,353.71764202)(759.92144275,353.56764217)(759.92144531,353.42765076)
\lineto(759.92144531,351.62765076)
\curveto(759.92144275,351.47764426)(759.91644275,351.31264443)(759.90644531,351.13265076)
\curveto(759.90644276,350.95264479)(759.93644273,350.81264493)(759.99644531,350.71265076)
\curveto(760.04644262,350.63264511)(760.12144255,350.58264516)(760.22144531,350.56265076)
\curveto(760.33144234,350.55264519)(760.45144222,350.54764519)(760.58144531,350.54765076)
\lineto(762.60644531,350.54765076)
\lineto(763.07144531,350.54765076)
\curveto(763.23143944,350.55764518)(763.3714393,350.57764516)(763.49144531,350.60765076)
\curveto(763.76143891,350.67764506)(763.99643867,350.75764498)(764.19644531,350.84765076)
\curveto(764.40643826,350.94764479)(764.58143809,351.09764464)(764.72144531,351.29765076)
\curveto(764.80143787,351.41764432)(764.86143781,351.5426442)(764.90144531,351.67265076)
\curveto(764.95143772,351.80264394)(764.99643767,351.94764379)(765.03644531,352.10765076)
\curveto(765.04643762,352.14764359)(765.05143762,352.21264353)(765.05144531,352.30265076)
}
}
{
\newrgbcolor{curcolor}{0 0 0}
\pscustom[linestyle=none,fillstyle=solid,fillcolor=curcolor]
{
\newpath
\moveto(774.88300781,345.55265076)
\curveto(774.91299998,345.39265035)(774.898,345.25765048)(774.83800781,345.14765076)
\curveto(774.77800012,345.04765069)(774.6980002,344.97265077)(774.59800781,344.92265076)
\curveto(774.54800035,344.90265084)(774.4930004,344.89265085)(774.43300781,344.89265076)
\curveto(774.38300051,344.89265085)(774.32800057,344.88265086)(774.26800781,344.86265076)
\curveto(774.04800085,344.81265093)(773.82800107,344.82765091)(773.60800781,344.90765076)
\curveto(773.3980015,344.97765076)(773.25300164,345.06765067)(773.17300781,345.17765076)
\curveto(773.12300177,345.24765049)(773.07800182,345.32765041)(773.03800781,345.41765076)
\curveto(772.9980019,345.51765022)(772.94800195,345.59765014)(772.88800781,345.65765076)
\curveto(772.86800203,345.67765006)(772.84300205,345.69765004)(772.81300781,345.71765076)
\curveto(772.7930021,345.73765)(772.76300213,345.74265)(772.72300781,345.73265076)
\curveto(772.61300228,345.70265004)(772.50800239,345.64765009)(772.40800781,345.56765076)
\curveto(772.31800258,345.48765025)(772.22800267,345.41765032)(772.13800781,345.35765076)
\curveto(772.00800289,345.27765046)(771.86800303,345.20265054)(771.71800781,345.13265076)
\curveto(771.56800333,345.07265067)(771.40800349,345.01765072)(771.23800781,344.96765076)
\curveto(771.13800376,344.9376508)(771.02800387,344.91765082)(770.90800781,344.90765076)
\curveto(770.7980041,344.89765084)(770.68800421,344.88265086)(770.57800781,344.86265076)
\curveto(770.52800437,344.85265089)(770.48300441,344.84765089)(770.44300781,344.84765076)
\lineto(770.33800781,344.84765076)
\curveto(770.22800467,344.82765091)(770.12300477,344.82765091)(770.02300781,344.84765076)
\lineto(769.88800781,344.84765076)
\curveto(769.83800506,344.85765088)(769.78800511,344.86265088)(769.73800781,344.86265076)
\curveto(769.68800521,344.86265088)(769.64300525,344.87265087)(769.60300781,344.89265076)
\curveto(769.56300533,344.90265084)(769.52800537,344.90765083)(769.49800781,344.90765076)
\curveto(769.47800542,344.89765084)(769.45300544,344.89765084)(769.42300781,344.90765076)
\lineto(769.18300781,344.96765076)
\curveto(769.10300579,344.97765076)(769.02800587,344.99765074)(768.95800781,345.02765076)
\curveto(768.65800624,345.15765058)(768.41300648,345.30265044)(768.22300781,345.46265076)
\curveto(768.04300685,345.63265011)(767.893007,345.86764987)(767.77300781,346.16765076)
\curveto(767.68300721,346.38764935)(767.63800726,346.65264909)(767.63800781,346.96265076)
\lineto(767.63800781,347.27765076)
\curveto(767.64800725,347.32764841)(767.65300724,347.37764836)(767.65300781,347.42765076)
\lineto(767.68300781,347.60765076)
\lineto(767.80300781,347.93765076)
\curveto(767.84300705,348.04764769)(767.893007,348.14764759)(767.95300781,348.23765076)
\curveto(768.13300676,348.52764721)(768.37800652,348.742647)(768.68800781,348.88265076)
\curveto(768.9980059,349.02264672)(769.33800556,349.14764659)(769.70800781,349.25765076)
\curveto(769.84800505,349.29764644)(769.9930049,349.32764641)(770.14300781,349.34765076)
\curveto(770.2930046,349.36764637)(770.44300445,349.39264635)(770.59300781,349.42265076)
\curveto(770.66300423,349.4426463)(770.72800417,349.45264629)(770.78800781,349.45265076)
\curveto(770.85800404,349.45264629)(770.93300396,349.46264628)(771.01300781,349.48265076)
\curveto(771.08300381,349.50264624)(771.15300374,349.51264623)(771.22300781,349.51265076)
\curveto(771.2930036,349.52264622)(771.36800353,349.5376462)(771.44800781,349.55765076)
\curveto(771.6980032,349.61764612)(771.93300296,349.66764607)(772.15300781,349.70765076)
\curveto(772.37300252,349.75764598)(772.54800235,349.87264587)(772.67800781,350.05265076)
\curveto(772.73800216,350.13264561)(772.78800211,350.23264551)(772.82800781,350.35265076)
\curveto(772.86800203,350.48264526)(772.86800203,350.62264512)(772.82800781,350.77265076)
\curveto(772.76800213,351.01264473)(772.67800222,351.20264454)(772.55800781,351.34265076)
\curveto(772.44800245,351.48264426)(772.28800261,351.59264415)(772.07800781,351.67265076)
\curveto(771.95800294,351.72264402)(771.81300308,351.75764398)(771.64300781,351.77765076)
\curveto(771.48300341,351.79764394)(771.31300358,351.80764393)(771.13300781,351.80765076)
\curveto(770.95300394,351.80764393)(770.77800412,351.79764394)(770.60800781,351.77765076)
\curveto(770.43800446,351.75764398)(770.2930046,351.72764401)(770.17300781,351.68765076)
\curveto(770.00300489,351.62764411)(769.83800506,351.5426442)(769.67800781,351.43265076)
\curveto(769.5980053,351.37264437)(769.52300537,351.29264445)(769.45300781,351.19265076)
\curveto(769.3930055,351.10264464)(769.33800556,351.00264474)(769.28800781,350.89265076)
\curveto(769.25800564,350.81264493)(769.22800567,350.72764501)(769.19800781,350.63765076)
\curveto(769.17800572,350.54764519)(769.13300576,350.47764526)(769.06300781,350.42765076)
\curveto(769.02300587,350.39764534)(768.95300594,350.37264537)(768.85300781,350.35265076)
\curveto(768.76300613,350.3426454)(768.66800623,350.3376454)(768.56800781,350.33765076)
\curveto(768.46800643,350.3376454)(768.36800653,350.3426454)(768.26800781,350.35265076)
\curveto(768.17800672,350.37264537)(768.11300678,350.39764534)(768.07300781,350.42765076)
\curveto(768.03300686,350.45764528)(768.00300689,350.50764523)(767.98300781,350.57765076)
\curveto(767.96300693,350.64764509)(767.96300693,350.72264502)(767.98300781,350.80265076)
\curveto(768.01300688,350.93264481)(768.04300685,351.05264469)(768.07300781,351.16265076)
\curveto(768.11300678,351.28264446)(768.15800674,351.39764434)(768.20800781,351.50765076)
\curveto(768.3980065,351.85764388)(768.63800626,352.12764361)(768.92800781,352.31765076)
\curveto(769.21800568,352.51764322)(769.57800532,352.67764306)(770.00800781,352.79765076)
\curveto(770.10800479,352.81764292)(770.20800469,352.83264291)(770.30800781,352.84265076)
\curveto(770.41800448,352.85264289)(770.52800437,352.86764287)(770.63800781,352.88765076)
\curveto(770.67800422,352.89764284)(770.74300415,352.89764284)(770.83300781,352.88765076)
\curveto(770.92300397,352.88764285)(770.97800392,352.89764284)(770.99800781,352.91765076)
\curveto(771.6980032,352.92764281)(772.30800259,352.84764289)(772.82800781,352.67765076)
\curveto(773.34800155,352.50764323)(773.71300118,352.18264356)(773.92300781,351.70265076)
\curveto(774.01300088,351.50264424)(774.06300083,351.26764447)(774.07300781,350.99765076)
\curveto(774.0930008,350.737645)(774.10300079,350.46264528)(774.10300781,350.17265076)
\lineto(774.10300781,346.85765076)
\curveto(774.10300079,346.71764902)(774.10800079,346.58264916)(774.11800781,346.45265076)
\curveto(774.12800077,346.32264942)(774.15800074,346.21764952)(774.20800781,346.13765076)
\curveto(774.25800064,346.06764967)(774.32300057,346.01764972)(774.40300781,345.98765076)
\curveto(774.4930004,345.94764979)(774.57800032,345.91764982)(774.65800781,345.89765076)
\curveto(774.73800016,345.88764985)(774.7980001,345.8426499)(774.83800781,345.76265076)
\curveto(774.85800004,345.73265001)(774.86800003,345.70265004)(774.86800781,345.67265076)
\curveto(774.86800003,345.6426501)(774.87300002,345.60265014)(774.88300781,345.55265076)
\moveto(772.73800781,347.21765076)
\curveto(772.7980021,347.35764838)(772.82800207,347.51764822)(772.82800781,347.69765076)
\curveto(772.83800206,347.88764785)(772.84300205,348.08264766)(772.84300781,348.28265076)
\curveto(772.84300205,348.39264735)(772.83800206,348.49264725)(772.82800781,348.58265076)
\curveto(772.81800208,348.67264707)(772.77800212,348.742647)(772.70800781,348.79265076)
\curveto(772.67800222,348.81264693)(772.60800229,348.82264692)(772.49800781,348.82265076)
\curveto(772.47800242,348.80264694)(772.44300245,348.79264695)(772.39300781,348.79265076)
\curveto(772.34300255,348.79264695)(772.2980026,348.78264696)(772.25800781,348.76265076)
\curveto(772.17800272,348.742647)(772.08800281,348.72264702)(771.98800781,348.70265076)
\lineto(771.68800781,348.64265076)
\curveto(771.65800324,348.6426471)(771.62300327,348.6376471)(771.58300781,348.62765076)
\lineto(771.47800781,348.62765076)
\curveto(771.32800357,348.58764715)(771.16300373,348.56264718)(770.98300781,348.55265076)
\curveto(770.81300408,348.55264719)(770.65300424,348.53264721)(770.50300781,348.49265076)
\curveto(770.42300447,348.47264727)(770.34800455,348.45264729)(770.27800781,348.43265076)
\curveto(770.21800468,348.42264732)(770.14800475,348.40764733)(770.06800781,348.38765076)
\curveto(769.90800499,348.3376474)(769.75800514,348.27264747)(769.61800781,348.19265076)
\curveto(769.47800542,348.12264762)(769.35800554,348.03264771)(769.25800781,347.92265076)
\curveto(769.15800574,347.81264793)(769.08300581,347.67764806)(769.03300781,347.51765076)
\curveto(768.98300591,347.36764837)(768.96300593,347.18264856)(768.97300781,346.96265076)
\curveto(768.97300592,346.86264888)(768.98800591,346.76764897)(769.01800781,346.67765076)
\curveto(769.05800584,346.59764914)(769.10300579,346.52264922)(769.15300781,346.45265076)
\curveto(769.23300566,346.3426494)(769.33800556,346.24764949)(769.46800781,346.16765076)
\curveto(769.5980053,346.09764964)(769.73800516,346.0376497)(769.88800781,345.98765076)
\curveto(769.93800496,345.97764976)(769.98800491,345.97264977)(770.03800781,345.97265076)
\curveto(770.08800481,345.97264977)(770.13800476,345.96764977)(770.18800781,345.95765076)
\curveto(770.25800464,345.9376498)(770.34300455,345.92264982)(770.44300781,345.91265076)
\curveto(770.55300434,345.91264983)(770.64300425,345.92264982)(770.71300781,345.94265076)
\curveto(770.77300412,345.96264978)(770.83300406,345.96764977)(770.89300781,345.95765076)
\curveto(770.95300394,345.95764978)(771.01300388,345.96764977)(771.07300781,345.98765076)
\curveto(771.15300374,346.00764973)(771.22800367,346.02264972)(771.29800781,346.03265076)
\curveto(771.37800352,346.0426497)(771.45300344,346.06264968)(771.52300781,346.09265076)
\curveto(771.81300308,346.21264953)(772.05800284,346.35764938)(772.25800781,346.52765076)
\curveto(772.46800243,346.69764904)(772.62800227,346.92764881)(772.73800781,347.21765076)
}
}
{
\newrgbcolor{curcolor}{0 0 0}
\pscustom[linestyle=none,fillstyle=solid,fillcolor=curcolor]
{
\newpath
\moveto(779.69964844,352.90265076)
\curveto(779.92964365,352.90264284)(780.05964352,352.8426429)(780.08964844,352.72265076)
\curveto(780.11964346,352.61264313)(780.13464344,352.44764329)(780.13464844,352.22765076)
\lineto(780.13464844,351.94265076)
\curveto(780.13464344,351.85264389)(780.10964347,351.77764396)(780.05964844,351.71765076)
\curveto(779.99964358,351.6376441)(779.91464366,351.59264415)(779.80464844,351.58265076)
\curveto(779.69464388,351.58264416)(779.58464399,351.56764417)(779.47464844,351.53765076)
\curveto(779.33464424,351.50764423)(779.19964438,351.47764426)(779.06964844,351.44765076)
\curveto(778.94964463,351.41764432)(778.83464474,351.37764436)(778.72464844,351.32765076)
\curveto(778.43464514,351.19764454)(778.19964538,351.01764472)(778.01964844,350.78765076)
\curveto(777.83964574,350.56764517)(777.68464589,350.31264543)(777.55464844,350.02265076)
\curveto(777.51464606,349.91264583)(777.48464609,349.79764594)(777.46464844,349.67765076)
\curveto(777.44464613,349.56764617)(777.41964616,349.45264629)(777.38964844,349.33265076)
\curveto(777.3796462,349.28264646)(777.3746462,349.23264651)(777.37464844,349.18265076)
\curveto(777.38464619,349.13264661)(777.38464619,349.08264666)(777.37464844,349.03265076)
\curveto(777.34464623,348.91264683)(777.32964625,348.77264697)(777.32964844,348.61265076)
\curveto(777.33964624,348.46264728)(777.34464623,348.31764742)(777.34464844,348.17765076)
\lineto(777.34464844,346.33265076)
\lineto(777.34464844,345.98765076)
\curveto(777.34464623,345.86764987)(777.33964624,345.75264999)(777.32964844,345.64265076)
\curveto(777.31964626,345.53265021)(777.31464626,345.4376503)(777.31464844,345.35765076)
\curveto(777.32464625,345.27765046)(777.30464627,345.20765053)(777.25464844,345.14765076)
\curveto(777.20464637,345.07765066)(777.12464645,345.0376507)(777.01464844,345.02765076)
\curveto(776.91464666,345.01765072)(776.80464677,345.01265073)(776.68464844,345.01265076)
\lineto(776.41464844,345.01265076)
\curveto(776.36464721,345.03265071)(776.31464726,345.04765069)(776.26464844,345.05765076)
\curveto(776.22464735,345.07765066)(776.19464738,345.10265064)(776.17464844,345.13265076)
\curveto(776.12464745,345.20265054)(776.09464748,345.28765045)(776.08464844,345.38765076)
\lineto(776.08464844,345.71765076)
\lineto(776.08464844,346.87265076)
\lineto(776.08464844,351.02765076)
\lineto(776.08464844,352.06265076)
\lineto(776.08464844,352.36265076)
\curveto(776.09464748,352.46264328)(776.12464745,352.54764319)(776.17464844,352.61765076)
\curveto(776.20464737,352.65764308)(776.25464732,352.68764305)(776.32464844,352.70765076)
\curveto(776.40464717,352.72764301)(776.48964709,352.737643)(776.57964844,352.73765076)
\curveto(776.66964691,352.74764299)(776.75964682,352.74764299)(776.84964844,352.73765076)
\curveto(776.93964664,352.72764301)(777.00964657,352.71264303)(777.05964844,352.69265076)
\curveto(777.13964644,352.66264308)(777.18964639,352.60264314)(777.20964844,352.51265076)
\curveto(777.23964634,352.43264331)(777.25464632,352.3426434)(777.25464844,352.24265076)
\lineto(777.25464844,351.94265076)
\curveto(777.25464632,351.8426439)(777.2746463,351.75264399)(777.31464844,351.67265076)
\curveto(777.32464625,351.65264409)(777.33464624,351.6376441)(777.34464844,351.62765076)
\lineto(777.38964844,351.58265076)
\curveto(777.49964608,351.58264416)(777.58964599,351.62764411)(777.65964844,351.71765076)
\curveto(777.72964585,351.81764392)(777.78964579,351.89764384)(777.83964844,351.95765076)
\lineto(777.92964844,352.04765076)
\curveto(778.01964556,352.15764358)(778.14464543,352.27264347)(778.30464844,352.39265076)
\curveto(778.46464511,352.51264323)(778.61464496,352.60264314)(778.75464844,352.66265076)
\curveto(778.84464473,352.71264303)(778.93964464,352.74764299)(779.03964844,352.76765076)
\curveto(779.13964444,352.79764294)(779.24464433,352.82764291)(779.35464844,352.85765076)
\curveto(779.41464416,352.86764287)(779.4746441,352.87264287)(779.53464844,352.87265076)
\curveto(779.59464398,352.88264286)(779.64964393,352.89264285)(779.69964844,352.90265076)
}
}
{
\newrgbcolor{curcolor}{0 0 0}
\pscustom[linestyle=none,fillstyle=solid,fillcolor=curcolor]
{
\newpath
\moveto(782.00941406,355.06265076)
\curveto(782.15941205,355.06264068)(782.3094119,355.05764068)(782.45941406,355.04765076)
\curveto(782.6094116,355.04764069)(782.7144115,355.00764073)(782.77441406,354.92765076)
\curveto(782.82441139,354.86764087)(782.84941136,354.78264096)(782.84941406,354.67265076)
\curveto(782.85941135,354.57264117)(782.86441135,354.46764127)(782.86441406,354.35765076)
\lineto(782.86441406,353.48765076)
\curveto(782.86441135,353.40764233)(782.85941135,353.32264242)(782.84941406,353.23265076)
\curveto(782.84941136,353.15264259)(782.85941135,353.08264266)(782.87941406,353.02265076)
\curveto(782.91941129,352.88264286)(783.0094112,352.79264295)(783.14941406,352.75265076)
\curveto(783.19941101,352.742643)(783.24441097,352.737643)(783.28441406,352.73765076)
\lineto(783.43441406,352.73765076)
\lineto(783.83941406,352.73765076)
\curveto(783.99941021,352.74764299)(784.1144101,352.737643)(784.18441406,352.70765076)
\curveto(784.27440994,352.64764309)(784.33440988,352.58764315)(784.36441406,352.52765076)
\curveto(784.38440983,352.48764325)(784.39440982,352.4426433)(784.39441406,352.39265076)
\lineto(784.39441406,352.24265076)
\curveto(784.39440982,352.13264361)(784.38940982,352.02764371)(784.37941406,351.92765076)
\curveto(784.36940984,351.8376439)(784.33440988,351.76764397)(784.27441406,351.71765076)
\curveto(784.21441,351.66764407)(784.12941008,351.6376441)(784.01941406,351.62765076)
\lineto(783.68941406,351.62765076)
\curveto(783.57941063,351.6376441)(783.46941074,351.6426441)(783.35941406,351.64265076)
\curveto(783.24941096,351.6426441)(783.15441106,351.62764411)(783.07441406,351.59765076)
\curveto(783.00441121,351.56764417)(782.95441126,351.51764422)(782.92441406,351.44765076)
\curveto(782.89441132,351.37764436)(782.87441134,351.29264445)(782.86441406,351.19265076)
\curveto(782.85441136,351.10264464)(782.84941136,351.00264474)(782.84941406,350.89265076)
\curveto(782.85941135,350.79264495)(782.86441135,350.69264505)(782.86441406,350.59265076)
\lineto(782.86441406,347.62265076)
\curveto(782.86441135,347.40264834)(782.85941135,347.16764857)(782.84941406,346.91765076)
\curveto(782.84941136,346.67764906)(782.89441132,346.49264925)(782.98441406,346.36265076)
\curveto(783.03441118,346.28264946)(783.09941111,346.22764951)(783.17941406,346.19765076)
\curveto(783.25941095,346.16764957)(783.35441086,346.1426496)(783.46441406,346.12265076)
\curveto(783.49441072,346.11264963)(783.52441069,346.10764963)(783.55441406,346.10765076)
\curveto(783.59441062,346.11764962)(783.62941058,346.11764962)(783.65941406,346.10765076)
\lineto(783.85441406,346.10765076)
\curveto(783.95441026,346.10764963)(784.04441017,346.09764964)(784.12441406,346.07765076)
\curveto(784.21441,346.06764967)(784.27940993,346.03264971)(784.31941406,345.97265076)
\curveto(784.33940987,345.9426498)(784.35440986,345.88764985)(784.36441406,345.80765076)
\curveto(784.38440983,345.73765)(784.39440982,345.66265008)(784.39441406,345.58265076)
\curveto(784.40440981,345.50265024)(784.40440981,345.42265032)(784.39441406,345.34265076)
\curveto(784.38440983,345.27265047)(784.36440985,345.21765052)(784.33441406,345.17765076)
\curveto(784.29440992,345.10765063)(784.21940999,345.05765068)(784.10941406,345.02765076)
\curveto(784.02941018,345.00765073)(783.93941027,344.99765074)(783.83941406,344.99765076)
\curveto(783.73941047,345.00765073)(783.64941056,345.01265073)(783.56941406,345.01265076)
\curveto(783.5094107,345.01265073)(783.44941076,345.00765073)(783.38941406,344.99765076)
\curveto(783.32941088,344.99765074)(783.27441094,345.00265074)(783.22441406,345.01265076)
\lineto(783.04441406,345.01265076)
\curveto(782.99441122,345.02265072)(782.94441127,345.02765071)(782.89441406,345.02765076)
\curveto(782.85441136,345.0376507)(782.8094114,345.0426507)(782.75941406,345.04265076)
\curveto(782.55941165,345.09265065)(782.38441183,345.14765059)(782.23441406,345.20765076)
\curveto(782.09441212,345.26765047)(781.97441224,345.37265037)(781.87441406,345.52265076)
\curveto(781.73441248,345.72265002)(781.65441256,345.97264977)(781.63441406,346.27265076)
\curveto(781.6144126,346.58264916)(781.60441261,346.91264883)(781.60441406,347.26265076)
\lineto(781.60441406,351.19265076)
\curveto(781.57441264,351.32264442)(781.54441267,351.41764432)(781.51441406,351.47765076)
\curveto(781.49441272,351.5376442)(781.42441279,351.58764415)(781.30441406,351.62765076)
\curveto(781.26441295,351.6376441)(781.22441299,351.6376441)(781.18441406,351.62765076)
\curveto(781.14441307,351.61764412)(781.10441311,351.62264412)(781.06441406,351.64265076)
\lineto(780.82441406,351.64265076)
\curveto(780.69441352,351.6426441)(780.58441363,351.65264409)(780.49441406,351.67265076)
\curveto(780.4144138,351.70264404)(780.35941385,351.76264398)(780.32941406,351.85265076)
\curveto(780.3094139,351.89264385)(780.29441392,351.9376438)(780.28441406,351.98765076)
\lineto(780.28441406,352.13765076)
\curveto(780.28441393,352.27764346)(780.29441392,352.39264335)(780.31441406,352.48265076)
\curveto(780.33441388,352.58264316)(780.39441382,352.65764308)(780.49441406,352.70765076)
\curveto(780.60441361,352.74764299)(780.74441347,352.75764298)(780.91441406,352.73765076)
\curveto(781.09441312,352.71764302)(781.24441297,352.72764301)(781.36441406,352.76765076)
\curveto(781.45441276,352.81764292)(781.52441269,352.88764285)(781.57441406,352.97765076)
\curveto(781.59441262,353.0376427)(781.60441261,353.11264263)(781.60441406,353.20265076)
\lineto(781.60441406,353.45765076)
\lineto(781.60441406,354.38765076)
\lineto(781.60441406,354.62765076)
\curveto(781.60441261,354.71764102)(781.6144126,354.79264095)(781.63441406,354.85265076)
\curveto(781.67441254,354.93264081)(781.74941246,354.99764074)(781.85941406,355.04765076)
\curveto(781.88941232,355.04764069)(781.9144123,355.04764069)(781.93441406,355.04765076)
\curveto(781.96441225,355.05764068)(781.98941222,355.06264068)(782.00941406,355.06265076)
}
}
{
\newrgbcolor{curcolor}{0 0 0}
\pscustom[linestyle=none,fillstyle=solid,fillcolor=curcolor]
{
\newpath
\moveto(786.06621094,354.22265076)
\curveto(785.98620982,354.28264146)(785.94120986,354.38764135)(785.93121094,354.53765076)
\lineto(785.93121094,355.00265076)
\lineto(785.93121094,355.25765076)
\curveto(785.93120987,355.34764039)(785.94620986,355.42264032)(785.97621094,355.48265076)
\curveto(786.01620979,355.56264018)(786.09620971,355.62264012)(786.21621094,355.66265076)
\curveto(786.23620957,355.67264007)(786.25620955,355.67264007)(786.27621094,355.66265076)
\curveto(786.3062095,355.66264008)(786.33120947,355.66764007)(786.35121094,355.67765076)
\curveto(786.52120928,355.67764006)(786.68120912,355.67264007)(786.83121094,355.66265076)
\curveto(786.98120882,355.65264009)(787.08120872,355.59264015)(787.13121094,355.48265076)
\curveto(787.16120864,355.42264032)(787.17620863,355.34764039)(787.17621094,355.25765076)
\lineto(787.17621094,355.00265076)
\curveto(787.17620863,354.82264092)(787.17120863,354.65264109)(787.16121094,354.49265076)
\curveto(787.16120864,354.33264141)(787.09620871,354.22764151)(786.96621094,354.17765076)
\curveto(786.91620889,354.15764158)(786.86120894,354.14764159)(786.80121094,354.14765076)
\lineto(786.63621094,354.14765076)
\lineto(786.32121094,354.14765076)
\curveto(786.22120958,354.14764159)(786.13620967,354.17264157)(786.06621094,354.22265076)
\moveto(787.17621094,345.71765076)
\lineto(787.17621094,345.40265076)
\curveto(787.18620862,345.30265044)(787.16620864,345.22265052)(787.11621094,345.16265076)
\curveto(787.08620872,345.10265064)(787.04120876,345.06265068)(786.98121094,345.04265076)
\curveto(786.92120888,345.03265071)(786.85120895,345.01765072)(786.77121094,344.99765076)
\lineto(786.54621094,344.99765076)
\curveto(786.41620939,344.99765074)(786.3012095,345.00265074)(786.20121094,345.01265076)
\curveto(786.11120969,345.03265071)(786.04120976,345.08265066)(785.99121094,345.16265076)
\curveto(785.95120985,345.22265052)(785.93120987,345.29765044)(785.93121094,345.38765076)
\lineto(785.93121094,345.67265076)
\lineto(785.93121094,352.01765076)
\lineto(785.93121094,352.33265076)
\curveto(785.93120987,352.4426433)(785.95620985,352.52764321)(786.00621094,352.58765076)
\curveto(786.03620977,352.6376431)(786.07620973,352.66764307)(786.12621094,352.67765076)
\curveto(786.17620963,352.68764305)(786.23120957,352.70264304)(786.29121094,352.72265076)
\curveto(786.31120949,352.72264302)(786.33120947,352.71764302)(786.35121094,352.70765076)
\curveto(786.38120942,352.70764303)(786.4062094,352.71264303)(786.42621094,352.72265076)
\curveto(786.55620925,352.72264302)(786.68620912,352.71764302)(786.81621094,352.70765076)
\curveto(786.95620885,352.70764303)(787.05120875,352.66764307)(787.10121094,352.58765076)
\curveto(787.15120865,352.52764321)(787.17620863,352.44764329)(787.17621094,352.34765076)
\lineto(787.17621094,352.06265076)
\lineto(787.17621094,345.71765076)
}
}
{
\newrgbcolor{curcolor}{0 0 0}
\pscustom[linestyle=none,fillstyle=solid,fillcolor=curcolor]
{
\newpath
\moveto(792.25605469,352.90265076)
\curveto(792.9960499,352.91264283)(793.61104928,352.80264294)(794.10105469,352.57265076)
\curveto(794.60104829,352.35264339)(794.9960479,352.01764372)(795.28605469,351.56765076)
\curveto(795.41604748,351.36764437)(795.52604737,351.12264462)(795.61605469,350.83265076)
\curveto(795.63604726,350.78264496)(795.65104724,350.71764502)(795.66105469,350.63765076)
\curveto(795.67104722,350.55764518)(795.66604723,350.48764525)(795.64605469,350.42765076)
\curveto(795.61604728,350.37764536)(795.56604733,350.33264541)(795.49605469,350.29265076)
\curveto(795.46604743,350.27264547)(795.43604746,350.26264548)(795.40605469,350.26265076)
\curveto(795.37604752,350.27264547)(795.34104755,350.27264547)(795.30105469,350.26265076)
\curveto(795.26104763,350.25264549)(795.22104767,350.24764549)(795.18105469,350.24765076)
\curveto(795.14104775,350.25764548)(795.10104779,350.26264548)(795.06105469,350.26265076)
\lineto(794.74605469,350.26265076)
\curveto(794.64604825,350.27264547)(794.56104833,350.30264544)(794.49105469,350.35265076)
\curveto(794.41104848,350.41264533)(794.35604854,350.49764524)(794.32605469,350.60765076)
\curveto(794.2960486,350.71764502)(794.25604864,350.81264493)(794.20605469,350.89265076)
\curveto(794.05604884,351.15264459)(793.86104903,351.35764438)(793.62105469,351.50765076)
\curveto(793.54104935,351.55764418)(793.45604944,351.59764414)(793.36605469,351.62765076)
\curveto(793.27604962,351.66764407)(793.18104971,351.70264404)(793.08105469,351.73265076)
\curveto(792.94104995,351.77264397)(792.75605014,351.79264395)(792.52605469,351.79265076)
\curveto(792.2960506,351.80264394)(792.10605079,351.78264396)(791.95605469,351.73265076)
\curveto(791.88605101,351.71264403)(791.82105107,351.69764404)(791.76105469,351.68765076)
\curveto(791.70105119,351.67764406)(791.63605126,351.66264408)(791.56605469,351.64265076)
\curveto(791.30605159,351.53264421)(791.07605182,351.38264436)(790.87605469,351.19265076)
\curveto(790.67605222,351.00264474)(790.52105237,350.77764496)(790.41105469,350.51765076)
\curveto(790.37105252,350.42764531)(790.33605256,350.33264541)(790.30605469,350.23265076)
\curveto(790.27605262,350.1426456)(790.24605265,350.0426457)(790.21605469,349.93265076)
\lineto(790.12605469,349.52765076)
\curveto(790.11605278,349.47764626)(790.11105278,349.42264632)(790.11105469,349.36265076)
\curveto(790.12105277,349.30264644)(790.11605278,349.24764649)(790.09605469,349.19765076)
\lineto(790.09605469,349.07765076)
\curveto(790.08605281,349.0376467)(790.07605282,348.97264677)(790.06605469,348.88265076)
\curveto(790.06605283,348.79264695)(790.07605282,348.72764701)(790.09605469,348.68765076)
\curveto(790.10605279,348.6376471)(790.10605279,348.58764715)(790.09605469,348.53765076)
\curveto(790.08605281,348.48764725)(790.08605281,348.4376473)(790.09605469,348.38765076)
\curveto(790.10605279,348.34764739)(790.11105278,348.27764746)(790.11105469,348.17765076)
\curveto(790.13105276,348.09764764)(790.14605275,348.01264773)(790.15605469,347.92265076)
\curveto(790.17605272,347.83264791)(790.1960527,347.74764799)(790.21605469,347.66765076)
\curveto(790.32605257,347.34764839)(790.45105244,347.06764867)(790.59105469,346.82765076)
\curveto(790.74105215,346.59764914)(790.94605195,346.39764934)(791.20605469,346.22765076)
\curveto(791.2960516,346.17764956)(791.38605151,346.13264961)(791.47605469,346.09265076)
\curveto(791.57605132,346.05264969)(791.68105121,346.01264973)(791.79105469,345.97265076)
\curveto(791.84105105,345.96264978)(791.88105101,345.95764978)(791.91105469,345.95765076)
\curveto(791.94105095,345.95764978)(791.98105091,345.95264979)(792.03105469,345.94265076)
\curveto(792.06105083,345.93264981)(792.11105078,345.92764981)(792.18105469,345.92765076)
\lineto(792.34605469,345.92765076)
\curveto(792.34605055,345.91764982)(792.36605053,345.91264983)(792.40605469,345.91265076)
\curveto(792.42605047,345.92264982)(792.45105044,345.92264982)(792.48105469,345.91265076)
\curveto(792.51105038,345.91264983)(792.54105035,345.91764982)(792.57105469,345.92765076)
\curveto(792.64105025,345.94764979)(792.70605019,345.95264979)(792.76605469,345.94265076)
\curveto(792.83605006,345.9426498)(792.90604999,345.95264979)(792.97605469,345.97265076)
\curveto(793.23604966,346.05264969)(793.46104943,346.15264959)(793.65105469,346.27265076)
\curveto(793.84104905,346.40264934)(794.00104889,346.56764917)(794.13105469,346.76765076)
\curveto(794.18104871,346.84764889)(794.22604867,346.93264881)(794.26605469,347.02265076)
\lineto(794.38605469,347.29265076)
\curveto(794.40604849,347.37264837)(794.42604847,347.44764829)(794.44605469,347.51765076)
\curveto(794.47604842,347.59764814)(794.52604837,347.66264808)(794.59605469,347.71265076)
\curveto(794.62604827,347.742648)(794.68604821,347.76264798)(794.77605469,347.77265076)
\curveto(794.86604803,347.79264795)(794.96104793,347.80264794)(795.06105469,347.80265076)
\curveto(795.17104772,347.81264793)(795.27104762,347.81264793)(795.36105469,347.80265076)
\curveto(795.46104743,347.79264795)(795.53104736,347.77264797)(795.57105469,347.74265076)
\curveto(795.63104726,347.70264804)(795.66604723,347.6426481)(795.67605469,347.56265076)
\curveto(795.6960472,347.48264826)(795.6960472,347.39764834)(795.67605469,347.30765076)
\curveto(795.62604727,347.15764858)(795.57604732,347.01264873)(795.52605469,346.87265076)
\curveto(795.48604741,346.742649)(795.43104746,346.61264913)(795.36105469,346.48265076)
\curveto(795.21104768,346.18264956)(795.02104787,345.91764982)(794.79105469,345.68765076)
\curveto(794.57104832,345.45765028)(794.30104859,345.27265047)(793.98105469,345.13265076)
\curveto(793.90104899,345.09265065)(793.81604908,345.05765068)(793.72605469,345.02765076)
\curveto(793.63604926,345.00765073)(793.54104935,344.98265076)(793.44105469,344.95265076)
\curveto(793.33104956,344.91265083)(793.22104967,344.89265085)(793.11105469,344.89265076)
\curveto(793.00104989,344.88265086)(792.89105,344.86765087)(792.78105469,344.84765076)
\curveto(792.74105015,344.82765091)(792.70105019,344.82265092)(792.66105469,344.83265076)
\curveto(792.62105027,344.8426509)(792.58105031,344.8426509)(792.54105469,344.83265076)
\lineto(792.40605469,344.83265076)
\lineto(792.16605469,344.83265076)
\curveto(792.0960508,344.82265092)(792.03105086,344.82765091)(791.97105469,344.84765076)
\lineto(791.89605469,344.84765076)
\lineto(791.53605469,344.89265076)
\curveto(791.40605149,344.93265081)(791.28105161,344.96765077)(791.16105469,344.99765076)
\curveto(791.04105185,345.02765071)(790.92605197,345.06765067)(790.81605469,345.11765076)
\curveto(790.45605244,345.27765046)(790.15605274,345.46765027)(789.91605469,345.68765076)
\curveto(789.68605321,345.90764983)(789.47105342,346.17764956)(789.27105469,346.49765076)
\curveto(789.22105367,346.57764916)(789.17605372,346.66764907)(789.13605469,346.76765076)
\lineto(789.01605469,347.06765076)
\curveto(788.96605393,347.17764856)(788.93105396,347.29264845)(788.91105469,347.41265076)
\curveto(788.891054,347.53264821)(788.86605403,347.65264809)(788.83605469,347.77265076)
\curveto(788.82605407,347.81264793)(788.82105407,347.85264789)(788.82105469,347.89265076)
\curveto(788.82105407,347.93264781)(788.81605408,347.97264777)(788.80605469,348.01265076)
\curveto(788.78605411,348.07264767)(788.77605412,348.1376476)(788.77605469,348.20765076)
\curveto(788.78605411,348.27764746)(788.78105411,348.3426474)(788.76105469,348.40265076)
\lineto(788.76105469,348.55265076)
\curveto(788.75105414,348.60264714)(788.74605415,348.67264707)(788.74605469,348.76265076)
\curveto(788.74605415,348.85264689)(788.75105414,348.92264682)(788.76105469,348.97265076)
\curveto(788.77105412,349.02264672)(788.77105412,349.06764667)(788.76105469,349.10765076)
\curveto(788.76105413,349.14764659)(788.76605413,349.18764655)(788.77605469,349.22765076)
\curveto(788.7960541,349.29764644)(788.80105409,349.36764637)(788.79105469,349.43765076)
\curveto(788.7910541,349.50764623)(788.80105409,349.57264617)(788.82105469,349.63265076)
\curveto(788.86105403,349.80264594)(788.896054,349.97264577)(788.92605469,350.14265076)
\curveto(788.95605394,350.31264543)(789.00105389,350.47264527)(789.06105469,350.62265076)
\curveto(789.27105362,351.1426446)(789.52605337,351.56264418)(789.82605469,351.88265076)
\curveto(790.12605277,352.20264354)(790.53605236,352.46764327)(791.05605469,352.67765076)
\curveto(791.16605173,352.72764301)(791.28605161,352.76264298)(791.41605469,352.78265076)
\curveto(791.54605135,352.80264294)(791.68105121,352.82764291)(791.82105469,352.85765076)
\curveto(791.891051,352.86764287)(791.96105093,352.87264287)(792.03105469,352.87265076)
\curveto(792.10105079,352.88264286)(792.17605072,352.89264285)(792.25605469,352.90265076)
}
}
{
\newrgbcolor{curcolor}{0 0 0}
\pscustom[linestyle=none,fillstyle=solid,fillcolor=curcolor]
{
\newpath
\moveto(797.46269531,354.22265076)
\curveto(797.38269419,354.28264146)(797.33769424,354.38764135)(797.32769531,354.53765076)
\lineto(797.32769531,355.00265076)
\lineto(797.32769531,355.25765076)
\curveto(797.32769425,355.34764039)(797.34269423,355.42264032)(797.37269531,355.48265076)
\curveto(797.41269416,355.56264018)(797.49269408,355.62264012)(797.61269531,355.66265076)
\curveto(797.63269394,355.67264007)(797.65269392,355.67264007)(797.67269531,355.66265076)
\curveto(797.70269387,355.66264008)(797.72769385,355.66764007)(797.74769531,355.67765076)
\curveto(797.91769366,355.67764006)(798.0776935,355.67264007)(798.22769531,355.66265076)
\curveto(798.3776932,355.65264009)(798.4776931,355.59264015)(798.52769531,355.48265076)
\curveto(798.55769302,355.42264032)(798.572693,355.34764039)(798.57269531,355.25765076)
\lineto(798.57269531,355.00265076)
\curveto(798.572693,354.82264092)(798.56769301,354.65264109)(798.55769531,354.49265076)
\curveto(798.55769302,354.33264141)(798.49269308,354.22764151)(798.36269531,354.17765076)
\curveto(798.31269326,354.15764158)(798.25769332,354.14764159)(798.19769531,354.14765076)
\lineto(798.03269531,354.14765076)
\lineto(797.71769531,354.14765076)
\curveto(797.61769396,354.14764159)(797.53269404,354.17264157)(797.46269531,354.22265076)
\moveto(798.57269531,345.71765076)
\lineto(798.57269531,345.40265076)
\curveto(798.58269299,345.30265044)(798.56269301,345.22265052)(798.51269531,345.16265076)
\curveto(798.48269309,345.10265064)(798.43769314,345.06265068)(798.37769531,345.04265076)
\curveto(798.31769326,345.03265071)(798.24769333,345.01765072)(798.16769531,344.99765076)
\lineto(797.94269531,344.99765076)
\curveto(797.81269376,344.99765074)(797.69769388,345.00265074)(797.59769531,345.01265076)
\curveto(797.50769407,345.03265071)(797.43769414,345.08265066)(797.38769531,345.16265076)
\curveto(797.34769423,345.22265052)(797.32769425,345.29765044)(797.32769531,345.38765076)
\lineto(797.32769531,345.67265076)
\lineto(797.32769531,352.01765076)
\lineto(797.32769531,352.33265076)
\curveto(797.32769425,352.4426433)(797.35269422,352.52764321)(797.40269531,352.58765076)
\curveto(797.43269414,352.6376431)(797.4726941,352.66764307)(797.52269531,352.67765076)
\curveto(797.572694,352.68764305)(797.62769395,352.70264304)(797.68769531,352.72265076)
\curveto(797.70769387,352.72264302)(797.72769385,352.71764302)(797.74769531,352.70765076)
\curveto(797.7776938,352.70764303)(797.80269377,352.71264303)(797.82269531,352.72265076)
\curveto(797.95269362,352.72264302)(798.08269349,352.71764302)(798.21269531,352.70765076)
\curveto(798.35269322,352.70764303)(798.44769313,352.66764307)(798.49769531,352.58765076)
\curveto(798.54769303,352.52764321)(798.572693,352.44764329)(798.57269531,352.34765076)
\lineto(798.57269531,352.06265076)
\lineto(798.57269531,345.71765076)
}
}
{
\newrgbcolor{curcolor}{0 0 0}
\pscustom[linestyle=none,fillstyle=solid,fillcolor=curcolor]
{
\newpath
\moveto(807.94253906,349.06265076)
\curveto(807.95253071,349.01264673)(807.95753071,348.94764679)(807.95753906,348.86765076)
\curveto(807.95753071,348.78764695)(807.95253071,348.72264702)(807.94253906,348.67265076)
\curveto(807.92253074,348.62264712)(807.91753075,348.57264717)(807.92753906,348.52265076)
\curveto(807.93753073,348.48264726)(807.93753073,348.4426473)(807.92753906,348.40265076)
\curveto(807.92753074,348.33264741)(807.92253074,348.27764746)(807.91253906,348.23765076)
\curveto(807.89253077,348.14764759)(807.87753079,348.05764768)(807.86753906,347.96765076)
\curveto(807.8675308,347.87764786)(807.85753081,347.78764795)(807.83753906,347.69765076)
\lineto(807.77753906,347.45765076)
\curveto(807.75753091,347.38764835)(807.73253093,347.31264843)(807.70253906,347.23265076)
\curveto(807.58253108,346.86264888)(807.41753125,346.52764921)(807.20753906,346.22765076)
\curveto(807.14753152,346.1376496)(807.08253158,346.04764969)(807.01253906,345.95765076)
\curveto(806.94253172,345.87764986)(806.8675318,345.80264994)(806.78753906,345.73265076)
\lineto(806.71253906,345.65765076)
\curveto(806.64253202,345.60765013)(806.57753209,345.55765018)(806.51753906,345.50765076)
\curveto(806.45753221,345.45765028)(806.38753228,345.40765033)(806.30753906,345.35765076)
\curveto(806.19753247,345.27765046)(806.07253259,345.20765053)(805.93253906,345.14765076)
\curveto(805.80253286,345.09765064)(805.667533,345.04765069)(805.52753906,344.99765076)
\curveto(805.44753322,344.97765076)(805.3675333,344.96265078)(805.28753906,344.95265076)
\curveto(805.21753345,344.9426508)(805.14253352,344.92765081)(805.06253906,344.90765076)
\lineto(805.00253906,344.90765076)
\curveto(804.99253367,344.89765084)(804.97753369,344.89265085)(804.95753906,344.89265076)
\curveto(804.8675338,344.87265087)(804.73253393,344.86265088)(804.55253906,344.86265076)
\curveto(804.38253428,344.85265089)(804.24753442,344.85765088)(804.14753906,344.87765076)
\lineto(804.07253906,344.87765076)
\curveto(804.00253466,344.88765085)(803.93753473,344.89765084)(803.87753906,344.90765076)
\curveto(803.81753485,344.90765083)(803.75753491,344.91765082)(803.69753906,344.93765076)
\curveto(803.52753514,344.98765075)(803.3675353,345.03265071)(803.21753906,345.07265076)
\curveto(803.0675356,345.11265063)(802.92753574,345.17265057)(802.79753906,345.25265076)
\curveto(802.63753603,345.3426504)(802.49753617,345.4376503)(802.37753906,345.53765076)
\curveto(802.33753633,345.56765017)(802.27753639,345.60765013)(802.19753906,345.65765076)
\curveto(802.11753655,345.71765002)(802.04253662,345.72265002)(801.97253906,345.67265076)
\curveto(801.93253673,345.6426501)(801.91253675,345.60265014)(801.91253906,345.55265076)
\curveto(801.91253675,345.50265024)(801.90253676,345.44765029)(801.88253906,345.38765076)
\curveto(801.87253679,345.35765038)(801.87253679,345.32265042)(801.88253906,345.28265076)
\curveto(801.89253677,345.25265049)(801.89253677,345.21765052)(801.88253906,345.17765076)
\curveto(801.8625368,345.11765062)(801.85253681,345.05265069)(801.85253906,344.98265076)
\curveto(801.8625368,344.90265084)(801.8675368,344.83265091)(801.86753906,344.77265076)
\lineto(801.86753906,342.97265076)
\lineto(801.86753906,342.53765076)
\curveto(801.8675368,342.38765335)(801.83753683,342.27265347)(801.77753906,342.19265076)
\curveto(801.72753694,342.12265362)(801.64753702,342.08765365)(801.53753906,342.08765076)
\curveto(801.42753724,342.07765366)(801.31753735,342.07265367)(801.20753906,342.07265076)
\lineto(800.96753906,342.07265076)
\curveto(800.89753777,342.09265365)(800.83753783,342.11265363)(800.78753906,342.13265076)
\curveto(800.74753792,342.15265359)(800.71253795,342.18765355)(800.68253906,342.23765076)
\curveto(800.63253803,342.30765343)(800.60753806,342.41765332)(800.60753906,342.56765076)
\curveto(800.61753805,342.71765302)(800.62253804,342.84765289)(800.62253906,342.95765076)
\lineto(800.62253906,351.95765076)
\lineto(800.62253906,352.31765076)
\curveto(800.63253803,352.44764329)(800.662538,352.55264319)(800.71253906,352.63265076)
\curveto(800.74253792,352.67264307)(800.80753786,352.70264304)(800.90753906,352.72265076)
\curveto(801.01753765,352.75264299)(801.13253753,352.76264298)(801.25253906,352.75265076)
\curveto(801.37253729,352.75264299)(801.48253718,352.737643)(801.58253906,352.70765076)
\curveto(801.69253697,352.68764305)(801.7625369,352.65764308)(801.79253906,352.61765076)
\curveto(801.83253683,352.56764317)(801.85253681,352.50764323)(801.85253906,352.43765076)
\curveto(801.8625368,352.36764337)(801.88253678,352.29764344)(801.91253906,352.22765076)
\curveto(801.93253673,352.19764354)(801.94753672,352.17264357)(801.95753906,352.15265076)
\curveto(801.97753669,352.1426436)(801.99753667,352.12764361)(802.01753906,352.10765076)
\curveto(802.12753654,352.09764364)(802.21753645,352.13264361)(802.28753906,352.21265076)
\curveto(802.3675363,352.29264345)(802.44253622,352.35764338)(802.51253906,352.40765076)
\curveto(802.77253589,352.58764315)(803.08253558,352.72764301)(803.44253906,352.82765076)
\curveto(803.53253513,352.84764289)(803.62253504,352.86264288)(803.71253906,352.87265076)
\curveto(803.81253485,352.88264286)(803.91253475,352.89764284)(804.01253906,352.91765076)
\curveto(804.05253461,352.92764281)(804.10253456,352.92764281)(804.16253906,352.91765076)
\curveto(804.22253444,352.90764283)(804.2625344,352.91264283)(804.28253906,352.93265076)
\curveto(804.71253395,352.9426428)(805.09253357,352.89764284)(805.42253906,352.79765076)
\curveto(805.75253291,352.70764303)(806.04753262,352.57764316)(806.30753906,352.40765076)
\lineto(806.45753906,352.28765076)
\curveto(806.50753216,352.25764348)(806.55753211,352.22264352)(806.60753906,352.18265076)
\curveto(806.62753204,352.16264358)(806.64253202,352.1426436)(806.65253906,352.12265076)
\curveto(806.67253199,352.11264363)(806.69253197,352.09764364)(806.71253906,352.07765076)
\curveto(806.7625319,352.02764371)(806.81753185,351.97264377)(806.87753906,351.91265076)
\curveto(806.93753173,351.85264389)(806.99253167,351.79264395)(807.04253906,351.73265076)
\curveto(807.1625315,351.56264418)(807.28753138,351.37764436)(807.41753906,351.17765076)
\curveto(807.49753117,351.04764469)(807.5625311,350.90264484)(807.61253906,350.74265076)
\curveto(807.67253099,350.58264516)(807.72753094,350.42264532)(807.77753906,350.26265076)
\curveto(807.79753087,350.18264556)(807.81253085,350.09764564)(807.82253906,350.00765076)
\curveto(807.84253082,349.91764582)(807.8625308,349.83264591)(807.88253906,349.75265076)
\lineto(807.88253906,349.63265076)
\curveto(807.89253077,349.60264614)(807.89753077,349.57264617)(807.89753906,349.54265076)
\curveto(807.91753075,349.49264625)(807.92253074,349.4376463)(807.91253906,349.37765076)
\curveto(807.91253075,349.31764642)(807.92253074,349.26264648)(807.94253906,349.21265076)
\lineto(807.94253906,349.06265076)
\moveto(806.60753906,348.65765076)
\curveto(806.62753204,348.70764703)(806.63253203,348.76764697)(806.62253906,348.83765076)
\curveto(806.61253205,348.91764682)(806.60753206,348.98764675)(806.60753906,349.04765076)
\curveto(806.60753206,349.21764652)(806.59753207,349.37764636)(806.57753906,349.52765076)
\curveto(806.5675321,349.67764606)(806.53753213,349.82264592)(806.48753906,349.96265076)
\lineto(806.42753906,350.14265076)
\curveto(806.41753225,350.21264553)(806.39753227,350.27764546)(806.36753906,350.33765076)
\curveto(806.25753241,350.60764513)(806.08253258,350.86764487)(805.84253906,351.11765076)
\curveto(805.61253305,351.36764437)(805.39253327,351.5376442)(805.18253906,351.62765076)
\curveto(805.10253356,351.66764407)(805.01753365,351.69764404)(804.92753906,351.71765076)
\curveto(804.84753382,351.737644)(804.7625339,351.76264398)(804.67253906,351.79265076)
\curveto(804.58253408,351.81264393)(804.47753419,351.82264392)(804.35753906,351.82265076)
\lineto(804.02753906,351.82265076)
\curveto(804.00753466,351.80264394)(803.9675347,351.79264395)(803.90753906,351.79265076)
\curveto(803.85753481,351.80264394)(803.81253485,351.80264394)(803.77253906,351.79265076)
\lineto(803.50253906,351.73265076)
\curveto(803.42253524,351.71264403)(803.34253532,351.68264406)(803.26253906,351.64265076)
\curveto(802.94253572,351.50264424)(802.67753599,351.29764444)(802.46753906,351.02765076)
\curveto(802.2675364,350.76764497)(802.11253655,350.46264528)(802.00253906,350.11265076)
\curveto(801.9625367,350.00264574)(801.93253673,349.89264585)(801.91253906,349.78265076)
\curveto(801.90253676,349.67264607)(801.88753678,349.56264618)(801.86753906,349.45265076)
\curveto(801.85753681,349.41264633)(801.85253681,349.37264637)(801.85253906,349.33265076)
\curveto(801.85253681,349.30264644)(801.84753682,349.26764647)(801.83753906,349.22765076)
\lineto(801.83753906,349.10765076)
\curveto(801.82753684,349.05764668)(801.82253684,348.98264676)(801.82253906,348.88265076)
\curveto(801.82253684,348.79264695)(801.82753684,348.72264702)(801.83753906,348.67265076)
\lineto(801.83753906,348.55265076)
\curveto(801.84753682,348.51264723)(801.85253681,348.47264727)(801.85253906,348.43265076)
\curveto(801.85253681,348.39264735)(801.85753681,348.35764738)(801.86753906,348.32765076)
\curveto(801.87753679,348.29764744)(801.88253678,348.26764747)(801.88253906,348.23765076)
\curveto(801.88253678,348.20764753)(801.88753678,348.17264757)(801.89753906,348.13265076)
\curveto(801.91753675,348.05264769)(801.93253673,347.97264777)(801.94253906,347.89265076)
\lineto(802.00253906,347.65265076)
\curveto(802.11253655,347.31264843)(802.2625364,347.01264873)(802.45253906,346.75265076)
\curveto(802.65253601,346.50264924)(802.91253575,346.30764943)(803.23253906,346.16765076)
\curveto(803.42253524,346.08764965)(803.61753505,346.02764971)(803.81753906,345.98765076)
\curveto(803.85753481,345.96764977)(803.89753477,345.95764978)(803.93753906,345.95765076)
\curveto(803.97753469,345.96764977)(804.01753465,345.96764977)(804.05753906,345.95765076)
\lineto(804.17753906,345.95765076)
\curveto(804.24753442,345.9376498)(804.31753435,345.9376498)(804.38753906,345.95765076)
\lineto(804.50753906,345.95765076)
\curveto(804.61753405,345.97764976)(804.72253394,345.99264975)(804.82253906,346.00265076)
\curveto(804.92253374,346.01264973)(805.02253364,346.0376497)(805.12253906,346.07765076)
\curveto(805.43253323,346.20764953)(805.68253298,346.37764936)(805.87253906,346.58765076)
\curveto(806.07253259,346.80764893)(806.23753243,347.07264867)(806.36753906,347.38265076)
\curveto(806.41753225,347.52264822)(806.45253221,347.66264808)(806.47253906,347.80265076)
\curveto(806.50253216,347.95264779)(806.53753213,348.10764763)(806.57753906,348.26765076)
\curveto(806.58753208,348.31764742)(806.59253207,348.36264738)(806.59253906,348.40265076)
\curveto(806.59253207,348.4426473)(806.59753207,348.48764725)(806.60753906,348.53765076)
\lineto(806.60753906,348.65765076)
}
}
{
\newrgbcolor{curcolor}{0 0 0}
\pscustom[linestyle=none,fillstyle=solid,fillcolor=curcolor]
{
\newpath
\moveto(816.30878906,345.55265076)
\curveto(816.33878123,345.39265035)(816.32378125,345.25765048)(816.26378906,345.14765076)
\curveto(816.20378137,345.04765069)(816.12378145,344.97265077)(816.02378906,344.92265076)
\curveto(815.9737816,344.90265084)(815.91878165,344.89265085)(815.85878906,344.89265076)
\curveto(815.80878176,344.89265085)(815.75378182,344.88265086)(815.69378906,344.86265076)
\curveto(815.4737821,344.81265093)(815.25378232,344.82765091)(815.03378906,344.90765076)
\curveto(814.82378275,344.97765076)(814.67878289,345.06765067)(814.59878906,345.17765076)
\curveto(814.54878302,345.24765049)(814.50378307,345.32765041)(814.46378906,345.41765076)
\curveto(814.42378315,345.51765022)(814.3737832,345.59765014)(814.31378906,345.65765076)
\curveto(814.29378328,345.67765006)(814.2687833,345.69765004)(814.23878906,345.71765076)
\curveto(814.21878335,345.73765)(814.18878338,345.74265)(814.14878906,345.73265076)
\curveto(814.03878353,345.70265004)(813.93378364,345.64765009)(813.83378906,345.56765076)
\curveto(813.74378383,345.48765025)(813.65378392,345.41765032)(813.56378906,345.35765076)
\curveto(813.43378414,345.27765046)(813.29378428,345.20265054)(813.14378906,345.13265076)
\curveto(812.99378458,345.07265067)(812.83378474,345.01765072)(812.66378906,344.96765076)
\curveto(812.56378501,344.9376508)(812.45378512,344.91765082)(812.33378906,344.90765076)
\curveto(812.22378535,344.89765084)(812.11378546,344.88265086)(812.00378906,344.86265076)
\curveto(811.95378562,344.85265089)(811.90878566,344.84765089)(811.86878906,344.84765076)
\lineto(811.76378906,344.84765076)
\curveto(811.65378592,344.82765091)(811.54878602,344.82765091)(811.44878906,344.84765076)
\lineto(811.31378906,344.84765076)
\curveto(811.26378631,344.85765088)(811.21378636,344.86265088)(811.16378906,344.86265076)
\curveto(811.11378646,344.86265088)(811.0687865,344.87265087)(811.02878906,344.89265076)
\curveto(810.98878658,344.90265084)(810.95378662,344.90765083)(810.92378906,344.90765076)
\curveto(810.90378667,344.89765084)(810.87878669,344.89765084)(810.84878906,344.90765076)
\lineto(810.60878906,344.96765076)
\curveto(810.52878704,344.97765076)(810.45378712,344.99765074)(810.38378906,345.02765076)
\curveto(810.08378749,345.15765058)(809.83878773,345.30265044)(809.64878906,345.46265076)
\curveto(809.4687881,345.63265011)(809.31878825,345.86764987)(809.19878906,346.16765076)
\curveto(809.10878846,346.38764935)(809.06378851,346.65264909)(809.06378906,346.96265076)
\lineto(809.06378906,347.27765076)
\curveto(809.0737885,347.32764841)(809.07878849,347.37764836)(809.07878906,347.42765076)
\lineto(809.10878906,347.60765076)
\lineto(809.22878906,347.93765076)
\curveto(809.2687883,348.04764769)(809.31878825,348.14764759)(809.37878906,348.23765076)
\curveto(809.55878801,348.52764721)(809.80378777,348.742647)(810.11378906,348.88265076)
\curveto(810.42378715,349.02264672)(810.76378681,349.14764659)(811.13378906,349.25765076)
\curveto(811.2737863,349.29764644)(811.41878615,349.32764641)(811.56878906,349.34765076)
\curveto(811.71878585,349.36764637)(811.8687857,349.39264635)(812.01878906,349.42265076)
\curveto(812.08878548,349.4426463)(812.15378542,349.45264629)(812.21378906,349.45265076)
\curveto(812.28378529,349.45264629)(812.35878521,349.46264628)(812.43878906,349.48265076)
\curveto(812.50878506,349.50264624)(812.57878499,349.51264623)(812.64878906,349.51265076)
\curveto(812.71878485,349.52264622)(812.79378478,349.5376462)(812.87378906,349.55765076)
\curveto(813.12378445,349.61764612)(813.35878421,349.66764607)(813.57878906,349.70765076)
\curveto(813.79878377,349.75764598)(813.9737836,349.87264587)(814.10378906,350.05265076)
\curveto(814.16378341,350.13264561)(814.21378336,350.23264551)(814.25378906,350.35265076)
\curveto(814.29378328,350.48264526)(814.29378328,350.62264512)(814.25378906,350.77265076)
\curveto(814.19378338,351.01264473)(814.10378347,351.20264454)(813.98378906,351.34265076)
\curveto(813.8737837,351.48264426)(813.71378386,351.59264415)(813.50378906,351.67265076)
\curveto(813.38378419,351.72264402)(813.23878433,351.75764398)(813.06878906,351.77765076)
\curveto(812.90878466,351.79764394)(812.73878483,351.80764393)(812.55878906,351.80765076)
\curveto(812.37878519,351.80764393)(812.20378537,351.79764394)(812.03378906,351.77765076)
\curveto(811.86378571,351.75764398)(811.71878585,351.72764401)(811.59878906,351.68765076)
\curveto(811.42878614,351.62764411)(811.26378631,351.5426442)(811.10378906,351.43265076)
\curveto(811.02378655,351.37264437)(810.94878662,351.29264445)(810.87878906,351.19265076)
\curveto(810.81878675,351.10264464)(810.76378681,351.00264474)(810.71378906,350.89265076)
\curveto(810.68378689,350.81264493)(810.65378692,350.72764501)(810.62378906,350.63765076)
\curveto(810.60378697,350.54764519)(810.55878701,350.47764526)(810.48878906,350.42765076)
\curveto(810.44878712,350.39764534)(810.37878719,350.37264537)(810.27878906,350.35265076)
\curveto(810.18878738,350.3426454)(810.09378748,350.3376454)(809.99378906,350.33765076)
\curveto(809.89378768,350.3376454)(809.79378778,350.3426454)(809.69378906,350.35265076)
\curveto(809.60378797,350.37264537)(809.53878803,350.39764534)(809.49878906,350.42765076)
\curveto(809.45878811,350.45764528)(809.42878814,350.50764523)(809.40878906,350.57765076)
\curveto(809.38878818,350.64764509)(809.38878818,350.72264502)(809.40878906,350.80265076)
\curveto(809.43878813,350.93264481)(809.4687881,351.05264469)(809.49878906,351.16265076)
\curveto(809.53878803,351.28264446)(809.58378799,351.39764434)(809.63378906,351.50765076)
\curveto(809.82378775,351.85764388)(810.06378751,352.12764361)(810.35378906,352.31765076)
\curveto(810.64378693,352.51764322)(811.00378657,352.67764306)(811.43378906,352.79765076)
\curveto(811.53378604,352.81764292)(811.63378594,352.83264291)(811.73378906,352.84265076)
\curveto(811.84378573,352.85264289)(811.95378562,352.86764287)(812.06378906,352.88765076)
\curveto(812.10378547,352.89764284)(812.1687854,352.89764284)(812.25878906,352.88765076)
\curveto(812.34878522,352.88764285)(812.40378517,352.89764284)(812.42378906,352.91765076)
\curveto(813.12378445,352.92764281)(813.73378384,352.84764289)(814.25378906,352.67765076)
\curveto(814.7737828,352.50764323)(815.13878243,352.18264356)(815.34878906,351.70265076)
\curveto(815.43878213,351.50264424)(815.48878208,351.26764447)(815.49878906,350.99765076)
\curveto(815.51878205,350.737645)(815.52878204,350.46264528)(815.52878906,350.17265076)
\lineto(815.52878906,346.85765076)
\curveto(815.52878204,346.71764902)(815.53378204,346.58264916)(815.54378906,346.45265076)
\curveto(815.55378202,346.32264942)(815.58378199,346.21764952)(815.63378906,346.13765076)
\curveto(815.68378189,346.06764967)(815.74878182,346.01764972)(815.82878906,345.98765076)
\curveto(815.91878165,345.94764979)(816.00378157,345.91764982)(816.08378906,345.89765076)
\curveto(816.16378141,345.88764985)(816.22378135,345.8426499)(816.26378906,345.76265076)
\curveto(816.28378129,345.73265001)(816.29378128,345.70265004)(816.29378906,345.67265076)
\curveto(816.29378128,345.6426501)(816.29878127,345.60265014)(816.30878906,345.55265076)
\moveto(814.16378906,347.21765076)
\curveto(814.22378335,347.35764838)(814.25378332,347.51764822)(814.25378906,347.69765076)
\curveto(814.26378331,347.88764785)(814.2687833,348.08264766)(814.26878906,348.28265076)
\curveto(814.2687833,348.39264735)(814.26378331,348.49264725)(814.25378906,348.58265076)
\curveto(814.24378333,348.67264707)(814.20378337,348.742647)(814.13378906,348.79265076)
\curveto(814.10378347,348.81264693)(814.03378354,348.82264692)(813.92378906,348.82265076)
\curveto(813.90378367,348.80264694)(813.8687837,348.79264695)(813.81878906,348.79265076)
\curveto(813.7687838,348.79264695)(813.72378385,348.78264696)(813.68378906,348.76265076)
\curveto(813.60378397,348.742647)(813.51378406,348.72264702)(813.41378906,348.70265076)
\lineto(813.11378906,348.64265076)
\curveto(813.08378449,348.6426471)(813.04878452,348.6376471)(813.00878906,348.62765076)
\lineto(812.90378906,348.62765076)
\curveto(812.75378482,348.58764715)(812.58878498,348.56264718)(812.40878906,348.55265076)
\curveto(812.23878533,348.55264719)(812.07878549,348.53264721)(811.92878906,348.49265076)
\curveto(811.84878572,348.47264727)(811.7737858,348.45264729)(811.70378906,348.43265076)
\curveto(811.64378593,348.42264732)(811.573786,348.40764733)(811.49378906,348.38765076)
\curveto(811.33378624,348.3376474)(811.18378639,348.27264747)(811.04378906,348.19265076)
\curveto(810.90378667,348.12264762)(810.78378679,348.03264771)(810.68378906,347.92265076)
\curveto(810.58378699,347.81264793)(810.50878706,347.67764806)(810.45878906,347.51765076)
\curveto(810.40878716,347.36764837)(810.38878718,347.18264856)(810.39878906,346.96265076)
\curveto(810.39878717,346.86264888)(810.41378716,346.76764897)(810.44378906,346.67765076)
\curveto(810.48378709,346.59764914)(810.52878704,346.52264922)(810.57878906,346.45265076)
\curveto(810.65878691,346.3426494)(810.76378681,346.24764949)(810.89378906,346.16765076)
\curveto(811.02378655,346.09764964)(811.16378641,346.0376497)(811.31378906,345.98765076)
\curveto(811.36378621,345.97764976)(811.41378616,345.97264977)(811.46378906,345.97265076)
\curveto(811.51378606,345.97264977)(811.56378601,345.96764977)(811.61378906,345.95765076)
\curveto(811.68378589,345.9376498)(811.7687858,345.92264982)(811.86878906,345.91265076)
\curveto(811.97878559,345.91264983)(812.0687855,345.92264982)(812.13878906,345.94265076)
\curveto(812.19878537,345.96264978)(812.25878531,345.96764977)(812.31878906,345.95765076)
\curveto(812.37878519,345.95764978)(812.43878513,345.96764977)(812.49878906,345.98765076)
\curveto(812.57878499,346.00764973)(812.65378492,346.02264972)(812.72378906,346.03265076)
\curveto(812.80378477,346.0426497)(812.87878469,346.06264968)(812.94878906,346.09265076)
\curveto(813.23878433,346.21264953)(813.48378409,346.35764938)(813.68378906,346.52765076)
\curveto(813.89378368,346.69764904)(814.05378352,346.92764881)(814.16378906,347.21765076)
}
}
{
\newrgbcolor{curcolor}{0 0 0}
\pscustom[linestyle=none,fillstyle=solid,fillcolor=curcolor]
{
\newpath
\moveto(820.61542969,352.90265076)
\curveto(821.3554249,352.91264283)(821.97042428,352.80264294)(822.46042969,352.57265076)
\curveto(822.96042329,352.35264339)(823.3554229,352.01764372)(823.64542969,351.56765076)
\curveto(823.77542248,351.36764437)(823.88542237,351.12264462)(823.97542969,350.83265076)
\curveto(823.99542226,350.78264496)(824.01042224,350.71764502)(824.02042969,350.63765076)
\curveto(824.03042222,350.55764518)(824.02542223,350.48764525)(824.00542969,350.42765076)
\curveto(823.97542228,350.37764536)(823.92542233,350.33264541)(823.85542969,350.29265076)
\curveto(823.82542243,350.27264547)(823.79542246,350.26264548)(823.76542969,350.26265076)
\curveto(823.73542252,350.27264547)(823.70042255,350.27264547)(823.66042969,350.26265076)
\curveto(823.62042263,350.25264549)(823.58042267,350.24764549)(823.54042969,350.24765076)
\curveto(823.50042275,350.25764548)(823.46042279,350.26264548)(823.42042969,350.26265076)
\lineto(823.10542969,350.26265076)
\curveto(823.00542325,350.27264547)(822.92042333,350.30264544)(822.85042969,350.35265076)
\curveto(822.77042348,350.41264533)(822.71542354,350.49764524)(822.68542969,350.60765076)
\curveto(822.6554236,350.71764502)(822.61542364,350.81264493)(822.56542969,350.89265076)
\curveto(822.41542384,351.15264459)(822.22042403,351.35764438)(821.98042969,351.50765076)
\curveto(821.90042435,351.55764418)(821.81542444,351.59764414)(821.72542969,351.62765076)
\curveto(821.63542462,351.66764407)(821.54042471,351.70264404)(821.44042969,351.73265076)
\curveto(821.30042495,351.77264397)(821.11542514,351.79264395)(820.88542969,351.79265076)
\curveto(820.6554256,351.80264394)(820.46542579,351.78264396)(820.31542969,351.73265076)
\curveto(820.24542601,351.71264403)(820.18042607,351.69764404)(820.12042969,351.68765076)
\curveto(820.06042619,351.67764406)(819.99542626,351.66264408)(819.92542969,351.64265076)
\curveto(819.66542659,351.53264421)(819.43542682,351.38264436)(819.23542969,351.19265076)
\curveto(819.03542722,351.00264474)(818.88042737,350.77764496)(818.77042969,350.51765076)
\curveto(818.73042752,350.42764531)(818.69542756,350.33264541)(818.66542969,350.23265076)
\curveto(818.63542762,350.1426456)(818.60542765,350.0426457)(818.57542969,349.93265076)
\lineto(818.48542969,349.52765076)
\curveto(818.47542778,349.47764626)(818.47042778,349.42264632)(818.47042969,349.36265076)
\curveto(818.48042777,349.30264644)(818.47542778,349.24764649)(818.45542969,349.19765076)
\lineto(818.45542969,349.07765076)
\curveto(818.44542781,349.0376467)(818.43542782,348.97264677)(818.42542969,348.88265076)
\curveto(818.42542783,348.79264695)(818.43542782,348.72764701)(818.45542969,348.68765076)
\curveto(818.46542779,348.6376471)(818.46542779,348.58764715)(818.45542969,348.53765076)
\curveto(818.44542781,348.48764725)(818.44542781,348.4376473)(818.45542969,348.38765076)
\curveto(818.46542779,348.34764739)(818.47042778,348.27764746)(818.47042969,348.17765076)
\curveto(818.49042776,348.09764764)(818.50542775,348.01264773)(818.51542969,347.92265076)
\curveto(818.53542772,347.83264791)(818.5554277,347.74764799)(818.57542969,347.66765076)
\curveto(818.68542757,347.34764839)(818.81042744,347.06764867)(818.95042969,346.82765076)
\curveto(819.10042715,346.59764914)(819.30542695,346.39764934)(819.56542969,346.22765076)
\curveto(819.6554266,346.17764956)(819.74542651,346.13264961)(819.83542969,346.09265076)
\curveto(819.93542632,346.05264969)(820.04042621,346.01264973)(820.15042969,345.97265076)
\curveto(820.20042605,345.96264978)(820.24042601,345.95764978)(820.27042969,345.95765076)
\curveto(820.30042595,345.95764978)(820.34042591,345.95264979)(820.39042969,345.94265076)
\curveto(820.42042583,345.93264981)(820.47042578,345.92764981)(820.54042969,345.92765076)
\lineto(820.70542969,345.92765076)
\curveto(820.70542555,345.91764982)(820.72542553,345.91264983)(820.76542969,345.91265076)
\curveto(820.78542547,345.92264982)(820.81042544,345.92264982)(820.84042969,345.91265076)
\curveto(820.87042538,345.91264983)(820.90042535,345.91764982)(820.93042969,345.92765076)
\curveto(821.00042525,345.94764979)(821.06542519,345.95264979)(821.12542969,345.94265076)
\curveto(821.19542506,345.9426498)(821.26542499,345.95264979)(821.33542969,345.97265076)
\curveto(821.59542466,346.05264969)(821.82042443,346.15264959)(822.01042969,346.27265076)
\curveto(822.20042405,346.40264934)(822.36042389,346.56764917)(822.49042969,346.76765076)
\curveto(822.54042371,346.84764889)(822.58542367,346.93264881)(822.62542969,347.02265076)
\lineto(822.74542969,347.29265076)
\curveto(822.76542349,347.37264837)(822.78542347,347.44764829)(822.80542969,347.51765076)
\curveto(822.83542342,347.59764814)(822.88542337,347.66264808)(822.95542969,347.71265076)
\curveto(822.98542327,347.742648)(823.04542321,347.76264798)(823.13542969,347.77265076)
\curveto(823.22542303,347.79264795)(823.32042293,347.80264794)(823.42042969,347.80265076)
\curveto(823.53042272,347.81264793)(823.63042262,347.81264793)(823.72042969,347.80265076)
\curveto(823.82042243,347.79264795)(823.89042236,347.77264797)(823.93042969,347.74265076)
\curveto(823.99042226,347.70264804)(824.02542223,347.6426481)(824.03542969,347.56265076)
\curveto(824.0554222,347.48264826)(824.0554222,347.39764834)(824.03542969,347.30765076)
\curveto(823.98542227,347.15764858)(823.93542232,347.01264873)(823.88542969,346.87265076)
\curveto(823.84542241,346.742649)(823.79042246,346.61264913)(823.72042969,346.48265076)
\curveto(823.57042268,346.18264956)(823.38042287,345.91764982)(823.15042969,345.68765076)
\curveto(822.93042332,345.45765028)(822.66042359,345.27265047)(822.34042969,345.13265076)
\curveto(822.26042399,345.09265065)(822.17542408,345.05765068)(822.08542969,345.02765076)
\curveto(821.99542426,345.00765073)(821.90042435,344.98265076)(821.80042969,344.95265076)
\curveto(821.69042456,344.91265083)(821.58042467,344.89265085)(821.47042969,344.89265076)
\curveto(821.36042489,344.88265086)(821.250425,344.86765087)(821.14042969,344.84765076)
\curveto(821.10042515,344.82765091)(821.06042519,344.82265092)(821.02042969,344.83265076)
\curveto(820.98042527,344.8426509)(820.94042531,344.8426509)(820.90042969,344.83265076)
\lineto(820.76542969,344.83265076)
\lineto(820.52542969,344.83265076)
\curveto(820.4554258,344.82265092)(820.39042586,344.82765091)(820.33042969,344.84765076)
\lineto(820.25542969,344.84765076)
\lineto(819.89542969,344.89265076)
\curveto(819.76542649,344.93265081)(819.64042661,344.96765077)(819.52042969,344.99765076)
\curveto(819.40042685,345.02765071)(819.28542697,345.06765067)(819.17542969,345.11765076)
\curveto(818.81542744,345.27765046)(818.51542774,345.46765027)(818.27542969,345.68765076)
\curveto(818.04542821,345.90764983)(817.83042842,346.17764956)(817.63042969,346.49765076)
\curveto(817.58042867,346.57764916)(817.53542872,346.66764907)(817.49542969,346.76765076)
\lineto(817.37542969,347.06765076)
\curveto(817.32542893,347.17764856)(817.29042896,347.29264845)(817.27042969,347.41265076)
\curveto(817.250429,347.53264821)(817.22542903,347.65264809)(817.19542969,347.77265076)
\curveto(817.18542907,347.81264793)(817.18042907,347.85264789)(817.18042969,347.89265076)
\curveto(817.18042907,347.93264781)(817.17542908,347.97264777)(817.16542969,348.01265076)
\curveto(817.14542911,348.07264767)(817.13542912,348.1376476)(817.13542969,348.20765076)
\curveto(817.14542911,348.27764746)(817.14042911,348.3426474)(817.12042969,348.40265076)
\lineto(817.12042969,348.55265076)
\curveto(817.11042914,348.60264714)(817.10542915,348.67264707)(817.10542969,348.76265076)
\curveto(817.10542915,348.85264689)(817.11042914,348.92264682)(817.12042969,348.97265076)
\curveto(817.13042912,349.02264672)(817.13042912,349.06764667)(817.12042969,349.10765076)
\curveto(817.12042913,349.14764659)(817.12542913,349.18764655)(817.13542969,349.22765076)
\curveto(817.1554291,349.29764644)(817.16042909,349.36764637)(817.15042969,349.43765076)
\curveto(817.1504291,349.50764623)(817.16042909,349.57264617)(817.18042969,349.63265076)
\curveto(817.22042903,349.80264594)(817.255429,349.97264577)(817.28542969,350.14265076)
\curveto(817.31542894,350.31264543)(817.36042889,350.47264527)(817.42042969,350.62265076)
\curveto(817.63042862,351.1426446)(817.88542837,351.56264418)(818.18542969,351.88265076)
\curveto(818.48542777,352.20264354)(818.89542736,352.46764327)(819.41542969,352.67765076)
\curveto(819.52542673,352.72764301)(819.64542661,352.76264298)(819.77542969,352.78265076)
\curveto(819.90542635,352.80264294)(820.04042621,352.82764291)(820.18042969,352.85765076)
\curveto(820.250426,352.86764287)(820.32042593,352.87264287)(820.39042969,352.87265076)
\curveto(820.46042579,352.88264286)(820.53542572,352.89264285)(820.61542969,352.90265076)
}
}
{
\newrgbcolor{curcolor}{0 0 0}
\pscustom[linestyle=none,fillstyle=solid,fillcolor=curcolor]
{
\newpath
\moveto(825.82207031,354.22265076)
\curveto(825.74206919,354.28264146)(825.69706924,354.38764135)(825.68707031,354.53765076)
\lineto(825.68707031,355.00265076)
\lineto(825.68707031,355.25765076)
\curveto(825.68706925,355.34764039)(825.70206923,355.42264032)(825.73207031,355.48265076)
\curveto(825.77206916,355.56264018)(825.85206908,355.62264012)(825.97207031,355.66265076)
\curveto(825.99206894,355.67264007)(826.01206892,355.67264007)(826.03207031,355.66265076)
\curveto(826.06206887,355.66264008)(826.08706885,355.66764007)(826.10707031,355.67765076)
\curveto(826.27706866,355.67764006)(826.4370685,355.67264007)(826.58707031,355.66265076)
\curveto(826.7370682,355.65264009)(826.8370681,355.59264015)(826.88707031,355.48265076)
\curveto(826.91706802,355.42264032)(826.932068,355.34764039)(826.93207031,355.25765076)
\lineto(826.93207031,355.00265076)
\curveto(826.932068,354.82264092)(826.92706801,354.65264109)(826.91707031,354.49265076)
\curveto(826.91706802,354.33264141)(826.85206808,354.22764151)(826.72207031,354.17765076)
\curveto(826.67206826,354.15764158)(826.61706832,354.14764159)(826.55707031,354.14765076)
\lineto(826.39207031,354.14765076)
\lineto(826.07707031,354.14765076)
\curveto(825.97706896,354.14764159)(825.89206904,354.17264157)(825.82207031,354.22265076)
\moveto(826.93207031,345.71765076)
\lineto(826.93207031,345.40265076)
\curveto(826.94206799,345.30265044)(826.92206801,345.22265052)(826.87207031,345.16265076)
\curveto(826.84206809,345.10265064)(826.79706814,345.06265068)(826.73707031,345.04265076)
\curveto(826.67706826,345.03265071)(826.60706833,345.01765072)(826.52707031,344.99765076)
\lineto(826.30207031,344.99765076)
\curveto(826.17206876,344.99765074)(826.05706888,345.00265074)(825.95707031,345.01265076)
\curveto(825.86706907,345.03265071)(825.79706914,345.08265066)(825.74707031,345.16265076)
\curveto(825.70706923,345.22265052)(825.68706925,345.29765044)(825.68707031,345.38765076)
\lineto(825.68707031,345.67265076)
\lineto(825.68707031,352.01765076)
\lineto(825.68707031,352.33265076)
\curveto(825.68706925,352.4426433)(825.71206922,352.52764321)(825.76207031,352.58765076)
\curveto(825.79206914,352.6376431)(825.8320691,352.66764307)(825.88207031,352.67765076)
\curveto(825.932069,352.68764305)(825.98706895,352.70264304)(826.04707031,352.72265076)
\curveto(826.06706887,352.72264302)(826.08706885,352.71764302)(826.10707031,352.70765076)
\curveto(826.1370688,352.70764303)(826.16206877,352.71264303)(826.18207031,352.72265076)
\curveto(826.31206862,352.72264302)(826.44206849,352.71764302)(826.57207031,352.70765076)
\curveto(826.71206822,352.70764303)(826.80706813,352.66764307)(826.85707031,352.58765076)
\curveto(826.90706803,352.52764321)(826.932068,352.44764329)(826.93207031,352.34765076)
\lineto(826.93207031,352.06265076)
\lineto(826.93207031,345.71765076)
}
}
{
\newrgbcolor{curcolor}{0 0 0}
\pscustom[linestyle=none,fillstyle=solid,fillcolor=curcolor]
{
\newpath
\moveto(832.95691406,355.91765076)
\curveto(833.026909,355.91763982)(833.11190891,355.91763982)(833.21191406,355.91765076)
\curveto(833.3219087,355.92763981)(833.4219086,355.92763981)(833.51191406,355.91765076)
\curveto(833.61190841,355.91763982)(833.70190832,355.90763983)(833.78191406,355.88765076)
\curveto(833.86190816,355.86763987)(833.91690811,355.8376399)(833.94691406,355.79765076)
\curveto(833.95690807,355.75763998)(833.95190807,355.70264004)(833.93191406,355.63265076)
\curveto(833.91190811,355.57264017)(833.87190815,355.51264023)(833.81191406,355.45265076)
\lineto(833.64691406,355.28765076)
\curveto(833.59690843,355.2376405)(833.54690848,355.18264056)(833.49691406,355.12265076)
\curveto(833.45690857,355.07264067)(833.41190861,355.01764072)(833.36191406,354.95765076)
\curveto(833.33190869,354.90764083)(833.29190873,354.86264088)(833.24191406,354.82265076)
\curveto(833.20190882,354.79264095)(833.16190886,354.75264099)(833.12191406,354.70265076)
\lineto(833.07691406,354.65765076)
\curveto(833.07690895,354.64764109)(833.06690896,354.6376411)(833.04691406,354.62765076)
\curveto(833.00690902,354.57764116)(832.96690906,354.53264121)(832.92691406,354.49265076)
\curveto(832.88690914,354.46264128)(832.84690918,354.42264132)(832.80691406,354.37265076)
\curveto(832.78690924,354.33264141)(832.75690927,354.29764144)(832.71691406,354.26765076)
\lineto(832.62691406,354.17765076)
\curveto(832.58690944,354.12764161)(832.54190948,354.07764166)(832.49191406,354.02765076)
\curveto(832.45190957,353.97764176)(832.40690962,353.9376418)(832.35691406,353.90765076)
\curveto(832.28690974,353.86764187)(832.17190985,353.83264191)(832.01191406,353.80265076)
\curveto(831.86191016,353.78264196)(831.74191028,353.79764194)(831.65191406,353.84765076)
\curveto(831.6219104,353.86764187)(831.59191043,353.89764184)(831.56191406,353.93765076)
\curveto(831.54191048,353.98764175)(831.54191048,354.0426417)(831.56191406,354.10265076)
\curveto(831.58191044,354.18264156)(831.61191041,354.25264149)(831.65191406,354.31265076)
\curveto(831.69191033,354.38264136)(831.73691029,354.44764129)(831.78691406,354.50765076)
\curveto(831.86691016,354.64764109)(831.95191007,354.79264095)(832.04191406,354.94265076)
\curveto(832.13190989,355.09264065)(832.2219098,355.2376405)(832.31191406,355.37765076)
\lineto(832.43191406,355.58765076)
\curveto(832.47190955,355.66764007)(832.5269095,355.73264001)(832.59691406,355.78265076)
\curveto(832.66690936,355.83263991)(832.73690929,355.87263987)(832.80691406,355.90265076)
\curveto(832.83690919,355.90263984)(832.86190916,355.90263984)(832.88191406,355.90265076)
\curveto(832.91190911,355.91263983)(832.93690909,355.91763982)(832.95691406,355.91765076)
\moveto(836.00191406,349.19765076)
\curveto(835.99190603,349.24764649)(835.98690604,349.29764644)(835.98691406,349.34765076)
\curveto(835.99690603,349.40764633)(835.99690603,349.46264628)(835.98691406,349.51265076)
\curveto(835.95690607,349.6426461)(835.93190609,349.76764597)(835.91191406,349.88765076)
\curveto(835.89190613,350.01764572)(835.86690616,350.1376456)(835.83691406,350.24765076)
\curveto(835.79690623,350.35764538)(835.76190626,350.46264528)(835.73191406,350.56265076)
\curveto(835.70190632,350.66264508)(835.66190636,350.76264498)(835.61191406,350.86265076)
\curveto(835.35190667,351.47264427)(834.9269071,351.96764377)(834.33691406,352.34765076)
\curveto(833.74690828,352.72764301)(833.00690902,352.91264283)(832.11691406,352.90265076)
\curveto(832.05690997,352.89264285)(831.99191003,352.88264286)(831.92191406,352.87265076)
\lineto(831.72691406,352.87265076)
\curveto(831.58691044,352.83264291)(831.44691058,352.80264294)(831.30691406,352.78265076)
\curveto(831.16691086,352.77264297)(831.03691099,352.742643)(830.91691406,352.69265076)
\curveto(830.77691125,352.63264311)(830.64191138,352.57264317)(830.51191406,352.51265076)
\curveto(830.38191164,352.46264328)(830.25691177,352.39764334)(830.13691406,352.31765076)
\curveto(829.8269122,352.11764362)(829.56191246,351.86764387)(829.34191406,351.56765076)
\curveto(829.13191289,351.27764446)(828.95191307,350.94764479)(828.80191406,350.57765076)
\curveto(828.75191327,350.46764527)(828.71191331,350.35264539)(828.68191406,350.23265076)
\curveto(828.66191336,350.11264563)(828.63691339,349.99264575)(828.60691406,349.87265076)
\curveto(828.59691343,349.82264592)(828.58691344,349.77764596)(828.57691406,349.73765076)
\curveto(828.57691345,349.70764603)(828.57191345,349.66764607)(828.56191406,349.61765076)
\curveto(828.54191348,349.54764619)(828.53691349,349.47764626)(828.54691406,349.40765076)
\curveto(828.55691347,349.3376464)(828.55191347,349.26764647)(828.53191406,349.19765076)
\curveto(828.51191351,349.1376466)(828.50191352,349.0426467)(828.50191406,348.91265076)
\curveto(828.50191352,348.79264695)(828.50691352,348.70764703)(828.51691406,348.65765076)
\curveto(828.5269135,348.60764713)(828.53191349,348.56264718)(828.53191406,348.52265076)
\lineto(828.53191406,348.40265076)
\curveto(828.55191347,348.32264742)(828.56191346,348.2376475)(828.56191406,348.14765076)
\curveto(828.57191345,348.06764767)(828.58691344,347.98764775)(828.60691406,347.90765076)
\curveto(828.61691341,347.86764787)(828.61691341,347.83264791)(828.60691406,347.80265076)
\curveto(828.60691342,347.78264796)(828.61691341,347.75264799)(828.63691406,347.71265076)
\curveto(828.65691337,347.60264814)(828.67691335,347.49764824)(828.69691406,347.39765076)
\curveto(828.7269133,347.29764844)(828.76691326,347.19764854)(828.81691406,347.09765076)
\curveto(829.00691302,346.6376491)(829.24691278,346.24764949)(829.53691406,345.92765076)
\curveto(829.8269122,345.60765013)(830.19691183,345.34765039)(830.64691406,345.14765076)
\curveto(830.76691126,345.09765064)(830.89191113,345.05265069)(831.02191406,345.01265076)
\curveto(831.15191087,344.97265077)(831.28691074,344.93265081)(831.42691406,344.89265076)
\lineto(831.78691406,344.84765076)
\lineto(831.87691406,344.84765076)
\curveto(831.90691012,344.8376509)(831.94191008,344.8376509)(831.98191406,344.84765076)
\curveto(832.02191,344.84765089)(832.06190996,344.8426509)(832.10191406,344.83265076)
\curveto(832.13190989,344.82265092)(832.18190984,344.81765092)(832.25191406,344.81765076)
\curveto(832.33190969,344.81765092)(832.39190963,344.82265092)(832.43191406,344.83265076)
\curveto(832.49190953,344.83265091)(832.55190947,344.8376509)(832.61191406,344.84765076)
\curveto(832.68190934,344.84765089)(832.74690928,344.85265089)(832.80691406,344.86265076)
\curveto(832.93690909,344.88265086)(833.06190896,344.90265084)(833.18191406,344.92265076)
\curveto(833.31190871,344.9426508)(833.43190859,344.97265077)(833.54191406,345.01265076)
\curveto(834.05190797,345.18265056)(834.48190754,345.42765031)(834.83191406,345.74765076)
\curveto(835.18190684,346.07764966)(835.46190656,346.48264926)(835.67191406,346.96265076)
\curveto(835.7219063,347.07264867)(835.76190626,347.19264855)(835.79191406,347.32265076)
\curveto(835.8219062,347.45264829)(835.85690617,347.58264816)(835.89691406,347.71265076)
\curveto(835.91690611,347.77264797)(835.9269061,347.83264791)(835.92691406,347.89265076)
\curveto(835.93690609,347.95264779)(835.95190607,348.01264773)(835.97191406,348.07265076)
\curveto(835.98190604,348.15264759)(835.98690604,348.22264752)(835.98691406,348.28265076)
\curveto(835.99690603,348.35264739)(836.00690602,348.42764731)(836.01691406,348.50765076)
\lineto(836.01691406,348.65765076)
\curveto(836.026906,348.70764703)(836.03190599,348.79264695)(836.03191406,348.91265076)
\curveto(836.03190599,349.0426467)(836.021906,349.1376466)(836.00191406,349.19765076)
\moveto(834.66691406,348.34265076)
\curveto(834.6269074,348.20264754)(834.59690743,348.06264768)(834.57691406,347.92265076)
\curveto(834.56690746,347.79264795)(834.53690749,347.66764807)(834.48691406,347.54765076)
\curveto(834.34690768,347.20764853)(834.17190785,346.91264883)(833.96191406,346.66265076)
\curveto(833.75190827,346.41264933)(833.47690855,346.21764952)(833.13691406,346.07765076)
\curveto(833.06690896,346.04764969)(832.99190903,346.02264972)(832.91191406,346.00265076)
\curveto(832.83190919,345.99264975)(832.75190927,345.97764976)(832.67191406,345.95765076)
\curveto(832.63190939,345.9376498)(832.59190943,345.92764981)(832.55191406,345.92765076)
\curveto(832.51190951,345.9376498)(832.47190955,345.9376498)(832.43191406,345.92765076)
\curveto(832.38190964,345.90764983)(832.30690972,345.90264984)(832.20691406,345.91265076)
\curveto(832.11690991,345.92264982)(832.05690997,345.93264981)(832.02691406,345.94265076)
\curveto(831.97691005,345.95264979)(831.9269101,345.95264979)(831.87691406,345.94265076)
\curveto(831.83691019,345.9426498)(831.79191023,345.95264979)(831.74191406,345.97265076)
\curveto(831.63191039,346.00264974)(831.5269105,346.03264971)(831.42691406,346.06265076)
\curveto(831.33691069,346.10264964)(831.25191077,346.14764959)(831.17191406,346.19765076)
\curveto(831.1219109,346.22764951)(831.07691095,346.25264949)(831.03691406,346.27265076)
\curveto(830.99691103,346.29264945)(830.95191107,346.31764942)(830.90191406,346.34765076)
\curveto(830.70191132,346.48764925)(830.53191149,346.66264908)(830.39191406,346.87265076)
\curveto(830.26191176,347.08264866)(830.14691188,347.30764843)(830.04691406,347.54765076)
\curveto(830.00691202,347.62764811)(829.97691205,347.71264803)(829.95691406,347.80265076)
\curveto(829.94691208,347.90264784)(829.93191209,348.00264774)(829.91191406,348.10265076)
\lineto(829.88191406,348.28265076)
\curveto(829.86191216,348.36264738)(829.85191217,348.45264729)(829.85191406,348.55265076)
\lineto(829.85191406,348.85265076)
\curveto(829.85191217,348.90264684)(829.84691218,348.94764679)(829.83691406,348.98765076)
\curveto(829.83691219,349.02764671)(829.84191218,349.06264668)(829.85191406,349.09265076)
\curveto(829.85191217,349.18264656)(829.85691217,349.24764649)(829.86691406,349.28765076)
\curveto(829.88691214,349.39764634)(829.90191212,349.50264624)(829.91191406,349.60265076)
\curveto(829.9219121,349.71264603)(829.94191208,349.81764592)(829.97191406,349.91765076)
\curveto(830.08191194,350.2376455)(830.21191181,350.51764522)(830.36191406,350.75765076)
\curveto(830.51191151,350.99764474)(830.70691132,351.20264454)(830.94691406,351.37265076)
\curveto(830.99691103,351.41264433)(831.05191097,351.45264429)(831.11191406,351.49265076)
\curveto(831.17191085,351.53264421)(831.23691079,351.56264418)(831.30691406,351.58265076)
\curveto(831.38691064,351.62264412)(831.46691056,351.65264409)(831.54691406,351.67265076)
\curveto(831.63691039,351.70264404)(831.73191029,351.72764401)(831.83191406,351.74765076)
\lineto(832.10191406,351.79265076)
\lineto(832.37191406,351.79265076)
\curveto(832.46190956,351.79264395)(832.54690948,351.78264396)(832.62691406,351.76265076)
\lineto(832.86691406,351.70265076)
\curveto(832.94690908,351.69264405)(833.021909,351.67264407)(833.09191406,351.64265076)
\curveto(833.70190832,351.39264435)(834.14690788,350.9426448)(834.42691406,350.29265076)
\curveto(834.45690757,350.22264552)(834.48190754,350.14764559)(834.50191406,350.06765076)
\curveto(834.5219075,349.98764575)(834.54690748,349.90764583)(834.57691406,349.82765076)
\curveto(834.64690738,349.55764618)(834.68190734,349.22764651)(834.68191406,348.83765076)
\lineto(834.68191406,348.58265076)
\curveto(834.69190733,348.49264725)(834.68690734,348.41264733)(834.66691406,348.34265076)
}
}
{
\newrgbcolor{curcolor}{0 0 0}
\pscustom[linestyle=none,fillstyle=solid,fillcolor=curcolor]
{
\newpath
\moveto(841.18019531,352.87265076)
\curveto(841.81019008,352.89264285)(842.31518957,352.80764293)(842.69519531,352.61765076)
\curveto(843.07518881,352.42764331)(843.38018851,352.1426436)(843.61019531,351.76265076)
\curveto(843.67018822,351.66264408)(843.71518817,351.55264419)(843.74519531,351.43265076)
\curveto(843.7851881,351.32264442)(843.82018807,351.20764453)(843.85019531,351.08765076)
\curveto(843.90018799,350.89764484)(843.93018796,350.69264505)(843.94019531,350.47265076)
\curveto(843.95018794,350.25264549)(843.95518793,350.02764571)(843.95519531,349.79765076)
\lineto(843.95519531,348.19265076)
\lineto(843.95519531,345.85265076)
\curveto(843.95518793,345.68265006)(843.95018794,345.51265023)(843.94019531,345.34265076)
\curveto(843.94018795,345.17265057)(843.87518801,345.06265068)(843.74519531,345.01265076)
\curveto(843.69518819,344.99265075)(843.64018825,344.98265076)(843.58019531,344.98265076)
\curveto(843.53018836,344.97265077)(843.47518841,344.96765077)(843.41519531,344.96765076)
\curveto(843.2851886,344.96765077)(843.16018873,344.97265077)(843.04019531,344.98265076)
\curveto(842.92018897,344.98265076)(842.83518905,345.02265072)(842.78519531,345.10265076)
\curveto(842.73518915,345.17265057)(842.71018918,345.26265048)(842.71019531,345.37265076)
\lineto(842.71019531,345.70265076)
\lineto(842.71019531,346.99265076)
\lineto(842.71019531,349.43765076)
\curveto(842.71018918,349.70764603)(842.70518918,349.97264577)(842.69519531,350.23265076)
\curveto(842.6851892,350.50264524)(842.64018925,350.73264501)(842.56019531,350.92265076)
\curveto(842.48018941,351.12264462)(842.36018953,351.28264446)(842.20019531,351.40265076)
\curveto(842.04018985,351.53264421)(841.85519003,351.63264411)(841.64519531,351.70265076)
\curveto(841.5851903,351.72264402)(841.52019037,351.73264401)(841.45019531,351.73265076)
\curveto(841.3901905,351.742644)(841.33019056,351.75764398)(841.27019531,351.77765076)
\curveto(841.22019067,351.78764395)(841.14019075,351.78764395)(841.03019531,351.77765076)
\curveto(840.93019096,351.77764396)(840.86019103,351.77264397)(840.82019531,351.76265076)
\curveto(840.78019111,351.742644)(840.74519114,351.73264401)(840.71519531,351.73265076)
\curveto(840.6851912,351.742644)(840.65019124,351.742644)(840.61019531,351.73265076)
\curveto(840.48019141,351.70264404)(840.35519153,351.66764407)(840.23519531,351.62765076)
\curveto(840.12519176,351.59764414)(840.02019187,351.55264419)(839.92019531,351.49265076)
\curveto(839.88019201,351.47264427)(839.84519204,351.45264429)(839.81519531,351.43265076)
\curveto(839.7851921,351.41264433)(839.75019214,351.39264435)(839.71019531,351.37265076)
\curveto(839.36019253,351.12264462)(839.10519278,350.74764499)(838.94519531,350.24765076)
\curveto(838.91519297,350.16764557)(838.89519299,350.08264566)(838.88519531,349.99265076)
\curveto(838.87519301,349.91264583)(838.86019303,349.83264591)(838.84019531,349.75265076)
\curveto(838.82019307,349.70264604)(838.81519307,349.65264609)(838.82519531,349.60265076)
\curveto(838.83519305,349.56264618)(838.83019306,349.52264622)(838.81019531,349.48265076)
\lineto(838.81019531,349.16765076)
\curveto(838.80019309,349.1376466)(838.79519309,349.10264664)(838.79519531,349.06265076)
\curveto(838.80519308,349.02264672)(838.81019308,348.97764676)(838.81019531,348.92765076)
\lineto(838.81019531,348.47765076)
\lineto(838.81019531,347.03765076)
\lineto(838.81019531,345.71765076)
\lineto(838.81019531,345.37265076)
\curveto(838.81019308,345.26265048)(838.7851931,345.17265057)(838.73519531,345.10265076)
\curveto(838.6851932,345.02265072)(838.59519329,344.98265076)(838.46519531,344.98265076)
\curveto(838.34519354,344.97265077)(838.22019367,344.96765077)(838.09019531,344.96765076)
\curveto(838.01019388,344.96765077)(837.93519395,344.97265077)(837.86519531,344.98265076)
\curveto(837.79519409,344.99265075)(837.73519415,345.01765072)(837.68519531,345.05765076)
\curveto(837.60519428,345.10765063)(837.56519432,345.20265054)(837.56519531,345.34265076)
\lineto(837.56519531,345.74765076)
\lineto(837.56519531,347.51765076)
\lineto(837.56519531,351.14765076)
\lineto(837.56519531,352.06265076)
\lineto(837.56519531,352.33265076)
\curveto(837.56519432,352.42264332)(837.5851943,352.49264325)(837.62519531,352.54265076)
\curveto(837.65519423,352.60264314)(837.70519418,352.6426431)(837.77519531,352.66265076)
\curveto(837.81519407,352.67264307)(837.87019402,352.68264306)(837.94019531,352.69265076)
\curveto(838.02019387,352.70264304)(838.10019379,352.70764303)(838.18019531,352.70765076)
\curveto(838.26019363,352.70764303)(838.33519355,352.70264304)(838.40519531,352.69265076)
\curveto(838.4851934,352.68264306)(838.54019335,352.66764307)(838.57019531,352.64765076)
\curveto(838.68019321,352.57764316)(838.73019316,352.48764325)(838.72019531,352.37765076)
\curveto(838.71019318,352.27764346)(838.72519316,352.16264358)(838.76519531,352.03265076)
\curveto(838.7851931,351.97264377)(838.82519306,351.92264382)(838.88519531,351.88265076)
\curveto(839.00519288,351.87264387)(839.10019279,351.91764382)(839.17019531,352.01765076)
\curveto(839.25019264,352.11764362)(839.33019256,352.19764354)(839.41019531,352.25765076)
\curveto(839.55019234,352.35764338)(839.6901922,352.44764329)(839.83019531,352.52765076)
\curveto(839.98019191,352.61764312)(840.15019174,352.69264305)(840.34019531,352.75265076)
\curveto(840.42019147,352.78264296)(840.50519138,352.80264294)(840.59519531,352.81265076)
\curveto(840.69519119,352.82264292)(840.7901911,352.8376429)(840.88019531,352.85765076)
\curveto(840.93019096,352.86764287)(840.98019091,352.87264287)(841.03019531,352.87265076)
\lineto(841.18019531,352.87265076)
}
}
{
\newrgbcolor{curcolor}{0 0 0}
\pscustom[linestyle=none,fillstyle=solid,fillcolor=curcolor]
{
\newpath
\moveto(762.48642334,332.33134644)
\curveto(763.05641818,332.3513355)(763.56141767,332.30633554)(764.00142334,332.19634644)
\curveto(764.44141679,332.08633576)(764.8314164,331.92133593)(765.17142334,331.70134644)
\curveto(765.231416,331.66133619)(765.28641595,331.62633622)(765.33642334,331.59634644)
\curveto(765.39641584,331.56633628)(765.45141578,331.52633632)(765.50142334,331.47634644)
\curveto(765.52141571,331.45633639)(765.54141569,331.43633641)(765.56142334,331.41634644)
\curveto(765.58141565,331.40633644)(765.60141563,331.39133646)(765.62142334,331.37134644)
\curveto(765.64141559,331.3513365)(765.66641557,331.32633652)(765.69642334,331.29634644)
\curveto(765.72641551,331.27633657)(765.75141548,331.2513366)(765.77142334,331.22134644)
\curveto(765.81141542,331.17133668)(765.85141538,331.12133673)(765.89142334,331.07134644)
\curveto(765.9314153,331.02133683)(765.97141526,330.97133688)(766.01142334,330.92134644)
\curveto(766.05141518,330.87133698)(766.08141515,330.81633703)(766.10142334,330.75634644)
\curveto(766.1314151,330.70633714)(766.16141507,330.65633719)(766.19142334,330.60634644)
\curveto(766.28141495,330.46633738)(766.35141488,330.30633754)(766.40142334,330.12634644)
\curveto(766.45141478,329.95633789)(766.50141473,329.77633807)(766.55142334,329.58634644)
\curveto(766.57141466,329.50633834)(766.57641466,329.41633843)(766.56642334,329.31634644)
\curveto(766.56641467,329.22633862)(766.55141468,329.15633869)(766.52142334,329.10634644)
\curveto(766.47141476,329.03633881)(766.39141484,328.99633885)(766.28142334,328.98634644)
\curveto(766.17141506,328.97633887)(766.05641518,328.97133888)(765.93642334,328.97134644)
\lineto(765.80142334,328.97134644)
\curveto(765.76141547,328.97133888)(765.72141551,328.97633887)(765.68142334,328.98634644)
\curveto(765.6314156,328.99633885)(765.58641565,329.00133885)(765.54642334,329.00134644)
\curveto(765.51641572,329.00133885)(765.48141575,329.01133884)(765.44142334,329.03134644)
\curveto(765.42141581,329.04133881)(765.40141583,329.0513388)(765.38142334,329.06134644)
\curveto(765.36141587,329.08133877)(765.34141589,329.10133875)(765.32142334,329.12134644)
\curveto(765.26141597,329.21133864)(765.21641602,329.32133853)(765.18642334,329.45134644)
\curveto(765.16641607,329.58133827)(765.1314161,329.69633815)(765.08142334,329.79634644)
\curveto(764.91141632,330.19633765)(764.67141656,330.49633735)(764.36142334,330.69634644)
\curveto(764.1314171,330.85633699)(763.85641738,330.97633687)(763.53642334,331.05634644)
\curveto(763.47641776,331.07633677)(763.41641782,331.08633676)(763.35642334,331.08634644)
\curveto(763.29641794,331.09633675)(763.236418,331.11133674)(763.17642334,331.13134644)
\lineto(763.05642334,331.13134644)
\curveto(762.97641826,331.1513367)(762.89641834,331.16133669)(762.81642334,331.16134644)
\lineto(762.57642334,331.16134644)
\curveto(762.05641918,331.16133669)(761.62141961,331.08133677)(761.27142334,330.92134644)
\curveto(761.07142016,330.84133701)(760.89142034,330.73133712)(760.73142334,330.59134644)
\curveto(760.57142066,330.46133739)(760.45142078,330.29633755)(760.37142334,330.09634644)
\curveto(760.3314209,330.01633783)(760.30142093,329.93133792)(760.28142334,329.84134644)
\curveto(760.27142096,329.76133809)(760.25642098,329.67133818)(760.23642334,329.57134644)
\curveto(760.21642102,329.49133836)(760.20642103,329.40133845)(760.20642334,329.30134644)
\curveto(760.21642102,329.20133865)(760.231421,329.11633873)(760.25142334,329.04634644)
\curveto(760.32142091,328.81633903)(760.40642083,328.64133921)(760.50642334,328.52134644)
\curveto(760.61642062,328.40133945)(760.76642047,328.28633956)(760.95642334,328.17634644)
\curveto(761.17642006,328.03633981)(761.42141981,327.93133992)(761.69142334,327.86134644)
\curveto(761.96141927,327.79134006)(762.24141899,327.71634013)(762.53142334,327.63634644)
\curveto(762.62141861,327.61634023)(762.71141852,327.60134025)(762.80142334,327.59134644)
\curveto(762.89141834,327.58134027)(762.98141825,327.56134029)(763.07142334,327.53134644)
\curveto(763.21141802,327.49134036)(763.35641788,327.45634039)(763.50642334,327.42634644)
\curveto(763.66641757,327.40634044)(763.81641742,327.37134048)(763.95642334,327.32134644)
\curveto(763.99641724,327.31134054)(764.0314172,327.30134055)(764.06142334,327.29134644)
\curveto(764.09141714,327.29134056)(764.12641711,327.28634056)(764.16642334,327.27634644)
\curveto(764.25641698,327.25634059)(764.34641689,327.23134062)(764.43642334,327.20134644)
\curveto(764.52641671,327.18134067)(764.61641662,327.15634069)(764.70642334,327.12634644)
\curveto(764.90641633,327.05634079)(765.10141613,326.98134087)(765.29142334,326.90134644)
\curveto(765.48141575,326.82134103)(765.66141557,326.72634112)(765.83142334,326.61634644)
\curveto(766.05141518,326.47634137)(766.24641499,326.31134154)(766.41642334,326.12134644)
\curveto(766.59641464,325.94134191)(766.74141449,325.72634212)(766.85142334,325.47634644)
\curveto(766.89141434,325.38634246)(766.92141431,325.29634255)(766.94142334,325.20634644)
\curveto(766.96141427,325.11634273)(766.98141425,325.02134283)(767.00142334,324.92134644)
\curveto(767.02141421,324.87134298)(767.0314142,324.81134304)(767.03142334,324.74134644)
\curveto(767.0314142,324.68134317)(767.04141419,324.62134323)(767.06142334,324.56134644)
\lineto(767.06142334,324.39634644)
\lineto(767.06142334,324.26134644)
\curveto(767.05141418,324.22134363)(767.04641419,324.18134367)(767.04642334,324.14134644)
\lineto(767.04642334,324.02134644)
\curveto(767.02641421,323.94134391)(767.01141422,323.85634399)(767.00142334,323.76634644)
\curveto(767.00141423,323.68634416)(766.98641425,323.60634424)(766.95642334,323.52634644)
\curveto(766.90641433,323.37634447)(766.85141438,323.23134462)(766.79142334,323.09134644)
\curveto(766.7314145,322.9513449)(766.65641458,322.82134503)(766.56642334,322.70134644)
\curveto(766.39641484,322.44134541)(766.18141505,322.22134563)(765.92142334,322.04134644)
\curveto(765.66141557,321.86134599)(765.38141585,321.70134615)(765.08142334,321.56134644)
\curveto(764.98141625,321.52134633)(764.88141635,321.48634636)(764.78142334,321.45634644)
\curveto(764.69141654,321.42634642)(764.59141664,321.39634645)(764.48142334,321.36634644)
\curveto(764.37141686,321.32634652)(764.25641698,321.30134655)(764.13642334,321.29134644)
\curveto(764.01641722,321.27134658)(763.89641734,321.2463466)(763.77642334,321.21634644)
\curveto(763.72641751,321.20634664)(763.67141756,321.20134665)(763.61142334,321.20134644)
\curveto(763.56141767,321.20134665)(763.51141772,321.19634665)(763.46142334,321.18634644)
\lineto(763.28142334,321.18634644)
\curveto(763.22141801,321.17634667)(763.16141807,321.17134668)(763.10142334,321.17134644)
\curveto(763.05141818,321.16134669)(762.97641826,321.15634669)(762.87642334,321.15634644)
\curveto(762.78641845,321.15634669)(762.71641852,321.16134669)(762.66642334,321.17134644)
\lineto(762.50142334,321.17134644)
\curveto(762.40141883,321.19134666)(762.30141893,321.20134665)(762.20142334,321.20134644)
\curveto(762.10141913,321.20134665)(762.00641923,321.21134664)(761.91642334,321.23134644)
\lineto(761.79642334,321.26134644)
\curveto(761.75641948,321.26134659)(761.71141952,321.26634658)(761.66142334,321.27634644)
\curveto(761.55141968,321.29634655)(761.44641979,321.31634653)(761.34642334,321.33634644)
\curveto(761.25641998,321.35634649)(761.16142007,321.38634646)(761.06142334,321.42634644)
\curveto(760.57142066,321.58634626)(760.15642108,321.79134606)(759.81642334,322.04134644)
\curveto(759.48642175,322.29134556)(759.20142203,322.61634523)(758.96142334,323.01634644)
\curveto(758.88142235,323.1463447)(758.81142242,323.28134457)(758.75142334,323.42134644)
\curveto(758.70142253,323.56134429)(758.65142258,323.71134414)(758.60142334,323.87134644)
\curveto(758.58142265,323.93134392)(758.56642267,323.99134386)(758.55642334,324.05134644)
\curveto(758.55642268,324.11134374)(758.54642269,324.16634368)(758.52642334,324.21634644)
\curveto(758.50642273,324.30634354)(758.49642274,324.40634344)(758.49642334,324.51634644)
\curveto(758.49642274,324.63634321)(758.52142271,324.72134313)(758.57142334,324.77134644)
\curveto(758.62142261,324.84134301)(758.70142253,324.88134297)(758.81142334,324.89134644)
\curveto(758.9314223,324.90134295)(759.04642219,324.90634294)(759.15642334,324.90634644)
\lineto(759.27642334,324.90634644)
\curveto(759.32642191,324.90634294)(759.36642187,324.90134295)(759.39642334,324.89134644)
\curveto(759.44642179,324.88134297)(759.48642175,324.87634297)(759.51642334,324.87634644)
\curveto(759.54642169,324.88634296)(759.58142165,324.88134297)(759.62142334,324.86134644)
\curveto(759.69142154,324.82134303)(759.7364215,324.78134307)(759.75642334,324.74134644)
\curveto(759.78642145,324.69134316)(759.80142143,324.63634321)(759.80142334,324.57634644)
\curveto(759.81142142,324.52634332)(759.82642141,324.47134338)(759.84642334,324.41134644)
\curveto(759.86642137,324.34134351)(759.88142135,324.27134358)(759.89142334,324.20134644)
\curveto(759.91142132,324.13134372)(759.9364213,324.06134379)(759.96642334,323.99134644)
\curveto(760.1364211,323.56134429)(760.39142084,323.22134463)(760.73142334,322.97134644)
\curveto(760.90142033,322.84134501)(761.08642015,322.73634511)(761.28642334,322.65634644)
\curveto(761.48641975,322.57634527)(761.70141953,322.50134535)(761.93142334,322.43134644)
\curveto(762.01141922,322.41134544)(762.09141914,322.39634545)(762.17142334,322.38634644)
\lineto(762.41142334,322.35634644)
\curveto(762.4314188,322.35634549)(762.45641878,322.3513455)(762.48642334,322.34134644)
\lineto(762.57642334,322.34134644)
\curveto(762.61641862,322.33134552)(762.66141857,322.33134552)(762.71142334,322.34134644)
\curveto(762.77141846,322.3513455)(762.82641841,322.3463455)(762.87642334,322.32634644)
\curveto(762.91641832,322.31634553)(762.97141826,322.31134554)(763.04142334,322.31134644)
\curveto(763.12141811,322.31134554)(763.18641805,322.32134553)(763.23642334,322.34134644)
\lineto(763.34142334,322.34134644)
\curveto(763.39141784,322.3513455)(763.44141779,322.35634549)(763.49142334,322.35634644)
\curveto(763.54141769,322.35634549)(763.59141764,322.36134549)(763.64142334,322.37134644)
\curveto(763.67141756,322.38134547)(763.69641754,322.38134547)(763.71642334,322.37134644)
\curveto(763.74641749,322.37134548)(763.78141745,322.38134547)(763.82142334,322.40134644)
\curveto(763.90141733,322.42134543)(763.98641725,322.43634541)(764.07642334,322.44634644)
\curveto(764.16641707,322.46634538)(764.25141698,322.49134536)(764.33142334,322.52134644)
\curveto(764.59141664,322.63134522)(764.82141641,322.75634509)(765.02142334,322.89634644)
\curveto(765.231416,323.0463448)(765.39141584,323.2463446)(765.50142334,323.49634644)
\curveto(765.5314157,323.57634427)(765.55641568,323.65634419)(765.57642334,323.73634644)
\lineto(765.63642334,323.97634644)
\curveto(765.65641558,324.05634379)(765.66641557,324.1513437)(765.66642334,324.26134644)
\curveto(765.66641557,324.37134348)(765.65641558,324.46134339)(765.63642334,324.53134644)
\curveto(765.61641562,324.58134327)(765.60641563,324.62134323)(765.60642334,324.65134644)
\curveto(765.60641563,324.69134316)(765.59641564,324.73134312)(765.57642334,324.77134644)
\curveto(765.51641572,324.94134291)(765.4364158,325.08134277)(765.33642334,325.19134644)
\curveto(765.236416,325.31134254)(765.11641612,325.42134243)(764.97642334,325.52134644)
\curveto(764.76641647,325.67134218)(764.5314167,325.79134206)(764.27142334,325.88134644)
\curveto(764.01141722,325.97134188)(763.7364175,326.0463418)(763.44642334,326.10634644)
\curveto(763.16641807,326.17634167)(762.87641836,326.24134161)(762.57642334,326.30134644)
\curveto(762.27641896,326.36134149)(761.98641925,326.43134142)(761.70642334,326.51134644)
\curveto(761.54641969,326.56134129)(761.38641985,326.60134125)(761.22642334,326.63134644)
\curveto(761.06642017,326.67134118)(760.91642032,326.72134113)(760.77642334,326.78134644)
\curveto(760.75642048,326.79134106)(760.74142049,326.79634105)(760.73142334,326.79634644)
\curveto(760.72142051,326.79634105)(760.70642053,326.80134105)(760.68642334,326.81134644)
\lineto(760.38642334,326.93134644)
\curveto(760.28642095,326.97134088)(760.19142104,327.02134083)(760.10142334,327.08134644)
\curveto(759.87142136,327.22134063)(759.66142157,327.37134048)(759.47142334,327.53134644)
\curveto(759.28142195,327.70134015)(759.1314221,327.91633993)(759.02142334,328.17634644)
\curveto(758.97142226,328.26633958)(758.9364223,328.36133949)(758.91642334,328.46134644)
\curveto(758.89642234,328.57133928)(758.87142236,328.68133917)(758.84142334,328.79134644)
\curveto(758.84142239,328.83133902)(758.8364224,328.86633898)(758.82642334,328.89634644)
\lineto(758.82642334,328.98634644)
\curveto(758.79642244,329.10633874)(758.79142244,329.25633859)(758.81142334,329.43634644)
\curveto(758.8314224,329.62633822)(758.85642238,329.77133808)(758.88642334,329.87134644)
\curveto(758.91642232,329.98133787)(758.94142229,330.08633776)(758.96142334,330.18634644)
\curveto(758.99142224,330.28633756)(759.02642221,330.38133747)(759.06642334,330.47134644)
\curveto(759.20642203,330.77133708)(759.38642185,331.02133683)(759.60642334,331.22134644)
\curveto(759.82642141,331.43133642)(760.07142116,331.62133623)(760.34142334,331.79134644)
\curveto(760.36142087,331.81133604)(760.38642085,331.82133603)(760.41642334,331.82134644)
\curveto(760.44642079,331.83133602)(760.47142076,331.846336)(760.49142334,331.86634644)
\curveto(760.71142052,331.97633587)(760.96642027,332.07133578)(761.25642334,332.15134644)
\curveto(761.3364199,332.17133568)(761.41141982,332.18633566)(761.48142334,332.19634644)
\curveto(761.55141968,332.21633563)(761.62641961,332.23633561)(761.70642334,332.25634644)
\curveto(761.78641945,332.27633557)(761.87141936,332.28633556)(761.96142334,332.28634644)
\curveto(762.06141917,332.28633556)(762.15141908,332.29633555)(762.23142334,332.31634644)
\curveto(762.26141897,332.32633552)(762.30641893,332.32633552)(762.36642334,332.31634644)
\curveto(762.42641881,332.30633554)(762.46641877,332.31133554)(762.48642334,332.33134644)
}
}
{
\newrgbcolor{curcolor}{0 0 0}
\pscustom[linestyle=none,fillstyle=solid,fillcolor=curcolor]
{
\newpath
\moveto(775.66298584,325.62634644)
\curveto(775.68297778,325.56634228)(775.69297777,325.47134238)(775.69298584,325.34134644)
\curveto(775.69297777,325.22134263)(775.68797777,325.13634271)(775.67798584,325.08634644)
\lineto(775.67798584,324.93634644)
\curveto(775.66797779,324.85634299)(775.6579778,324.78134307)(775.64798584,324.71134644)
\curveto(775.64797781,324.6513432)(775.64297782,324.58134327)(775.63298584,324.50134644)
\curveto(775.61297785,324.44134341)(775.59797786,324.38134347)(775.58798584,324.32134644)
\curveto(775.58797787,324.26134359)(775.57797788,324.20134365)(775.55798584,324.14134644)
\curveto(775.51797794,324.01134384)(775.48297798,323.88134397)(775.45298584,323.75134644)
\curveto(775.42297804,323.62134423)(775.38297808,323.50134435)(775.33298584,323.39134644)
\curveto(775.12297834,322.91134494)(774.84297862,322.50634534)(774.49298584,322.17634644)
\curveto(774.14297932,321.85634599)(773.71297975,321.61134624)(773.20298584,321.44134644)
\curveto(773.09298037,321.40134645)(772.97298049,321.37134648)(772.84298584,321.35134644)
\curveto(772.72298074,321.33134652)(772.59798086,321.31134654)(772.46798584,321.29134644)
\curveto(772.40798105,321.28134657)(772.34298112,321.27634657)(772.27298584,321.27634644)
\curveto(772.21298125,321.26634658)(772.15298131,321.26134659)(772.09298584,321.26134644)
\curveto(772.05298141,321.2513466)(771.99298147,321.2463466)(771.91298584,321.24634644)
\curveto(771.84298162,321.2463466)(771.79298167,321.2513466)(771.76298584,321.26134644)
\curveto(771.72298174,321.27134658)(771.68298178,321.27634657)(771.64298584,321.27634644)
\curveto(771.60298186,321.26634658)(771.56798189,321.26634658)(771.53798584,321.27634644)
\lineto(771.44798584,321.27634644)
\lineto(771.08798584,321.32134644)
\curveto(770.94798251,321.36134649)(770.81298265,321.40134645)(770.68298584,321.44134644)
\curveto(770.55298291,321.48134637)(770.42798303,321.52634632)(770.30798584,321.57634644)
\curveto(769.8579836,321.77634607)(769.48798397,322.03634581)(769.19798584,322.35634644)
\curveto(768.90798455,322.67634517)(768.66798479,323.06634478)(768.47798584,323.52634644)
\curveto(768.42798503,323.62634422)(768.38798507,323.72634412)(768.35798584,323.82634644)
\curveto(768.33798512,323.92634392)(768.31798514,324.03134382)(768.29798584,324.14134644)
\curveto(768.27798518,324.18134367)(768.26798519,324.21134364)(768.26798584,324.23134644)
\curveto(768.27798518,324.26134359)(768.27798518,324.29634355)(768.26798584,324.33634644)
\curveto(768.24798521,324.41634343)(768.23298523,324.49634335)(768.22298584,324.57634644)
\curveto(768.22298524,324.66634318)(768.21298525,324.7513431)(768.19298584,324.83134644)
\lineto(768.19298584,324.95134644)
\curveto(768.19298527,324.99134286)(768.18798527,325.03634281)(768.17798584,325.08634644)
\curveto(768.16798529,325.13634271)(768.1629853,325.22134263)(768.16298584,325.34134644)
\curveto(768.1629853,325.47134238)(768.17298529,325.56634228)(768.19298584,325.62634644)
\curveto(768.21298525,325.69634215)(768.21798524,325.76634208)(768.20798584,325.83634644)
\curveto(768.19798526,325.90634194)(768.20298526,325.97634187)(768.22298584,326.04634644)
\curveto(768.23298523,326.09634175)(768.23798522,326.13634171)(768.23798584,326.16634644)
\curveto(768.24798521,326.20634164)(768.2579852,326.2513416)(768.26798584,326.30134644)
\curveto(768.29798516,326.42134143)(768.32298514,326.54134131)(768.34298584,326.66134644)
\curveto(768.37298509,326.78134107)(768.41298505,326.89634095)(768.46298584,327.00634644)
\curveto(768.61298485,327.37634047)(768.79298467,327.70634014)(769.00298584,327.99634644)
\curveto(769.22298424,328.29633955)(769.48798397,328.5463393)(769.79798584,328.74634644)
\curveto(769.91798354,328.82633902)(770.04298342,328.89133896)(770.17298584,328.94134644)
\curveto(770.30298316,329.00133885)(770.43798302,329.06133879)(770.57798584,329.12134644)
\curveto(770.69798276,329.17133868)(770.82798263,329.20133865)(770.96798584,329.21134644)
\curveto(771.10798235,329.23133862)(771.24798221,329.26133859)(771.38798584,329.30134644)
\lineto(771.58298584,329.30134644)
\curveto(771.65298181,329.31133854)(771.71798174,329.32133853)(771.77798584,329.33134644)
\curveto(772.66798079,329.34133851)(773.40798005,329.15633869)(773.99798584,328.77634644)
\curveto(774.58797887,328.39633945)(775.01297845,327.90133995)(775.27298584,327.29134644)
\curveto(775.32297814,327.19134066)(775.3629781,327.09134076)(775.39298584,326.99134644)
\curveto(775.42297804,326.89134096)(775.457978,326.78634106)(775.49798584,326.67634644)
\curveto(775.52797793,326.56634128)(775.55297791,326.4463414)(775.57298584,326.31634644)
\curveto(775.59297787,326.19634165)(775.61797784,326.07134178)(775.64798584,325.94134644)
\curveto(775.6579778,325.89134196)(775.6579778,325.83634201)(775.64798584,325.77634644)
\curveto(775.64797781,325.72634212)(775.65297781,325.67634217)(775.66298584,325.62634644)
\moveto(774.32798584,324.77134644)
\curveto(774.34797911,324.84134301)(774.35297911,324.92134293)(774.34298584,325.01134644)
\lineto(774.34298584,325.26634644)
\curveto(774.34297912,325.65634219)(774.30797915,325.98634186)(774.23798584,326.25634644)
\curveto(774.20797925,326.33634151)(774.18297928,326.41634143)(774.16298584,326.49634644)
\curveto(774.14297932,326.57634127)(774.11797934,326.6513412)(774.08798584,326.72134644)
\curveto(773.80797965,327.37134048)(773.3629801,327.82134003)(772.75298584,328.07134644)
\curveto(772.68298078,328.10133975)(772.60798085,328.12133973)(772.52798584,328.13134644)
\lineto(772.28798584,328.19134644)
\curveto(772.20798125,328.21133964)(772.12298134,328.22133963)(772.03298584,328.22134644)
\lineto(771.76298584,328.22134644)
\lineto(771.49298584,328.17634644)
\curveto(771.39298207,328.15633969)(771.29798216,328.13133972)(771.20798584,328.10134644)
\curveto(771.12798233,328.08133977)(771.04798241,328.0513398)(770.96798584,328.01134644)
\curveto(770.89798256,327.99133986)(770.83298263,327.96133989)(770.77298584,327.92134644)
\curveto(770.71298275,327.88133997)(770.6579828,327.84134001)(770.60798584,327.80134644)
\curveto(770.36798309,327.63134022)(770.17298329,327.42634042)(770.02298584,327.18634644)
\curveto(769.87298359,326.9463409)(769.74298372,326.66634118)(769.63298584,326.34634644)
\curveto(769.60298386,326.2463416)(769.58298388,326.14134171)(769.57298584,326.03134644)
\curveto(769.5629839,325.93134192)(769.54798391,325.82634202)(769.52798584,325.71634644)
\curveto(769.51798394,325.67634217)(769.51298395,325.61134224)(769.51298584,325.52134644)
\curveto(769.50298396,325.49134236)(769.49798396,325.45634239)(769.49798584,325.41634644)
\curveto(769.50798395,325.37634247)(769.51298395,325.33134252)(769.51298584,325.28134644)
\lineto(769.51298584,324.98134644)
\curveto(769.51298395,324.88134297)(769.52298394,324.79134306)(769.54298584,324.71134644)
\lineto(769.57298584,324.53134644)
\curveto(769.59298387,324.43134342)(769.60798385,324.33134352)(769.61798584,324.23134644)
\curveto(769.63798382,324.14134371)(769.66798379,324.05634379)(769.70798584,323.97634644)
\curveto(769.80798365,323.73634411)(769.92298354,323.51134434)(770.05298584,323.30134644)
\curveto(770.19298327,323.09134476)(770.3629831,322.91634493)(770.56298584,322.77634644)
\curveto(770.61298285,322.7463451)(770.6579828,322.72134513)(770.69798584,322.70134644)
\curveto(770.73798272,322.68134517)(770.78298268,322.65634519)(770.83298584,322.62634644)
\curveto(770.91298255,322.57634527)(770.99798246,322.53134532)(771.08798584,322.49134644)
\curveto(771.18798227,322.46134539)(771.29298217,322.43134542)(771.40298584,322.40134644)
\curveto(771.45298201,322.38134547)(771.49798196,322.37134548)(771.53798584,322.37134644)
\curveto(771.58798187,322.38134547)(771.63798182,322.38134547)(771.68798584,322.37134644)
\curveto(771.71798174,322.36134549)(771.77798168,322.3513455)(771.86798584,322.34134644)
\curveto(771.96798149,322.33134552)(772.04298142,322.33634551)(772.09298584,322.35634644)
\curveto(772.13298133,322.36634548)(772.17298129,322.36634548)(772.21298584,322.35634644)
\curveto(772.25298121,322.35634549)(772.29298117,322.36634548)(772.33298584,322.38634644)
\curveto(772.41298105,322.40634544)(772.49298097,322.42134543)(772.57298584,322.43134644)
\curveto(772.65298081,322.4513454)(772.72798073,322.47634537)(772.79798584,322.50634644)
\curveto(773.13798032,322.6463452)(773.41298005,322.84134501)(773.62298584,323.09134644)
\curveto(773.83297963,323.34134451)(774.00797945,323.63634421)(774.14798584,323.97634644)
\curveto(774.19797926,324.09634375)(774.22797923,324.22134363)(774.23798584,324.35134644)
\curveto(774.2579792,324.49134336)(774.28797917,324.63134322)(774.32798584,324.77134644)
}
}
{
\newrgbcolor{curcolor}{0 0 0}
\pscustom[linestyle=none,fillstyle=solid,fillcolor=curcolor]
{
\newpath
\moveto(780.28626709,329.33134644)
\curveto(781.0262623,329.34133851)(781.64126168,329.23133862)(782.13126709,329.00134644)
\curveto(782.63126069,328.78133907)(783.0262603,328.4463394)(783.31626709,327.99634644)
\curveto(783.44625988,327.79634005)(783.55625977,327.5513403)(783.64626709,327.26134644)
\curveto(783.66625966,327.21134064)(783.68125964,327.1463407)(783.69126709,327.06634644)
\curveto(783.70125962,326.98634086)(783.69625963,326.91634093)(783.67626709,326.85634644)
\curveto(783.64625968,326.80634104)(783.59625973,326.76134109)(783.52626709,326.72134644)
\curveto(783.49625983,326.70134115)(783.46625986,326.69134116)(783.43626709,326.69134644)
\curveto(783.40625992,326.70134115)(783.37125995,326.70134115)(783.33126709,326.69134644)
\curveto(783.29126003,326.68134117)(783.25126007,326.67634117)(783.21126709,326.67634644)
\curveto(783.17126015,326.68634116)(783.13126019,326.69134116)(783.09126709,326.69134644)
\lineto(782.77626709,326.69134644)
\curveto(782.67626065,326.70134115)(782.59126073,326.73134112)(782.52126709,326.78134644)
\curveto(782.44126088,326.84134101)(782.38626094,326.92634092)(782.35626709,327.03634644)
\curveto(782.326261,327.1463407)(782.28626104,327.24134061)(782.23626709,327.32134644)
\curveto(782.08626124,327.58134027)(781.89126143,327.78634006)(781.65126709,327.93634644)
\curveto(781.57126175,327.98633986)(781.48626184,328.02633982)(781.39626709,328.05634644)
\curveto(781.30626202,328.09633975)(781.21126211,328.13133972)(781.11126709,328.16134644)
\curveto(780.97126235,328.20133965)(780.78626254,328.22133963)(780.55626709,328.22134644)
\curveto(780.326263,328.23133962)(780.13626319,328.21133964)(779.98626709,328.16134644)
\curveto(779.91626341,328.14133971)(779.85126347,328.12633972)(779.79126709,328.11634644)
\curveto(779.73126359,328.10633974)(779.66626366,328.09133976)(779.59626709,328.07134644)
\curveto(779.33626399,327.96133989)(779.10626422,327.81134004)(778.90626709,327.62134644)
\curveto(778.70626462,327.43134042)(778.55126477,327.20634064)(778.44126709,326.94634644)
\curveto(778.40126492,326.85634099)(778.36626496,326.76134109)(778.33626709,326.66134644)
\curveto(778.30626502,326.57134128)(778.27626505,326.47134138)(778.24626709,326.36134644)
\lineto(778.15626709,325.95634644)
\curveto(778.14626518,325.90634194)(778.14126518,325.851342)(778.14126709,325.79134644)
\curveto(778.15126517,325.73134212)(778.14626518,325.67634217)(778.12626709,325.62634644)
\lineto(778.12626709,325.50634644)
\curveto(778.11626521,325.46634238)(778.10626522,325.40134245)(778.09626709,325.31134644)
\curveto(778.09626523,325.22134263)(778.10626522,325.15634269)(778.12626709,325.11634644)
\curveto(778.13626519,325.06634278)(778.13626519,325.01634283)(778.12626709,324.96634644)
\curveto(778.11626521,324.91634293)(778.11626521,324.86634298)(778.12626709,324.81634644)
\curveto(778.13626519,324.77634307)(778.14126518,324.70634314)(778.14126709,324.60634644)
\curveto(778.16126516,324.52634332)(778.17626515,324.44134341)(778.18626709,324.35134644)
\curveto(778.20626512,324.26134359)(778.2262651,324.17634367)(778.24626709,324.09634644)
\curveto(778.35626497,323.77634407)(778.48126484,323.49634435)(778.62126709,323.25634644)
\curveto(778.77126455,323.02634482)(778.97626435,322.82634502)(779.23626709,322.65634644)
\curveto(779.326264,322.60634524)(779.41626391,322.56134529)(779.50626709,322.52134644)
\curveto(779.60626372,322.48134537)(779.71126361,322.44134541)(779.82126709,322.40134644)
\curveto(779.87126345,322.39134546)(779.91126341,322.38634546)(779.94126709,322.38634644)
\curveto(779.97126335,322.38634546)(780.01126331,322.38134547)(780.06126709,322.37134644)
\curveto(780.09126323,322.36134549)(780.14126318,322.35634549)(780.21126709,322.35634644)
\lineto(780.37626709,322.35634644)
\curveto(780.37626295,322.3463455)(780.39626293,322.34134551)(780.43626709,322.34134644)
\curveto(780.45626287,322.3513455)(780.48126284,322.3513455)(780.51126709,322.34134644)
\curveto(780.54126278,322.34134551)(780.57126275,322.3463455)(780.60126709,322.35634644)
\curveto(780.67126265,322.37634547)(780.73626259,322.38134547)(780.79626709,322.37134644)
\curveto(780.86626246,322.37134548)(780.93626239,322.38134547)(781.00626709,322.40134644)
\curveto(781.26626206,322.48134537)(781.49126183,322.58134527)(781.68126709,322.70134644)
\curveto(781.87126145,322.83134502)(782.03126129,322.99634485)(782.16126709,323.19634644)
\curveto(782.21126111,323.27634457)(782.25626107,323.36134449)(782.29626709,323.45134644)
\lineto(782.41626709,323.72134644)
\curveto(782.43626089,323.80134405)(782.45626087,323.87634397)(782.47626709,323.94634644)
\curveto(782.50626082,324.02634382)(782.55626077,324.09134376)(782.62626709,324.14134644)
\curveto(782.65626067,324.17134368)(782.71626061,324.19134366)(782.80626709,324.20134644)
\curveto(782.89626043,324.22134363)(782.99126033,324.23134362)(783.09126709,324.23134644)
\curveto(783.20126012,324.24134361)(783.30126002,324.24134361)(783.39126709,324.23134644)
\curveto(783.49125983,324.22134363)(783.56125976,324.20134365)(783.60126709,324.17134644)
\curveto(783.66125966,324.13134372)(783.69625963,324.07134378)(783.70626709,323.99134644)
\curveto(783.7262596,323.91134394)(783.7262596,323.82634402)(783.70626709,323.73634644)
\curveto(783.65625967,323.58634426)(783.60625972,323.44134441)(783.55626709,323.30134644)
\curveto(783.51625981,323.17134468)(783.46125986,323.04134481)(783.39126709,322.91134644)
\curveto(783.24126008,322.61134524)(783.05126027,322.3463455)(782.82126709,322.11634644)
\curveto(782.60126072,321.88634596)(782.33126099,321.70134615)(782.01126709,321.56134644)
\curveto(781.93126139,321.52134633)(781.84626148,321.48634636)(781.75626709,321.45634644)
\curveto(781.66626166,321.43634641)(781.57126175,321.41134644)(781.47126709,321.38134644)
\curveto(781.36126196,321.34134651)(781.25126207,321.32134653)(781.14126709,321.32134644)
\curveto(781.03126229,321.31134654)(780.9212624,321.29634655)(780.81126709,321.27634644)
\curveto(780.77126255,321.25634659)(780.73126259,321.2513466)(780.69126709,321.26134644)
\curveto(780.65126267,321.27134658)(780.61126271,321.27134658)(780.57126709,321.26134644)
\lineto(780.43626709,321.26134644)
\lineto(780.19626709,321.26134644)
\curveto(780.1262632,321.2513466)(780.06126326,321.25634659)(780.00126709,321.27634644)
\lineto(779.92626709,321.27634644)
\lineto(779.56626709,321.32134644)
\curveto(779.43626389,321.36134649)(779.31126401,321.39634645)(779.19126709,321.42634644)
\curveto(779.07126425,321.45634639)(778.95626437,321.49634635)(778.84626709,321.54634644)
\curveto(778.48626484,321.70634614)(778.18626514,321.89634595)(777.94626709,322.11634644)
\curveto(777.71626561,322.33634551)(777.50126582,322.60634524)(777.30126709,322.92634644)
\curveto(777.25126607,323.00634484)(777.20626612,323.09634475)(777.16626709,323.19634644)
\lineto(777.04626709,323.49634644)
\curveto(776.99626633,323.60634424)(776.96126636,323.72134413)(776.94126709,323.84134644)
\curveto(776.9212664,323.96134389)(776.89626643,324.08134377)(776.86626709,324.20134644)
\curveto(776.85626647,324.24134361)(776.85126647,324.28134357)(776.85126709,324.32134644)
\curveto(776.85126647,324.36134349)(776.84626648,324.40134345)(776.83626709,324.44134644)
\curveto(776.81626651,324.50134335)(776.80626652,324.56634328)(776.80626709,324.63634644)
\curveto(776.81626651,324.70634314)(776.81126651,324.77134308)(776.79126709,324.83134644)
\lineto(776.79126709,324.98134644)
\curveto(776.78126654,325.03134282)(776.77626655,325.10134275)(776.77626709,325.19134644)
\curveto(776.77626655,325.28134257)(776.78126654,325.3513425)(776.79126709,325.40134644)
\curveto(776.80126652,325.4513424)(776.80126652,325.49634235)(776.79126709,325.53634644)
\curveto(776.79126653,325.57634227)(776.79626653,325.61634223)(776.80626709,325.65634644)
\curveto(776.8262665,325.72634212)(776.83126649,325.79634205)(776.82126709,325.86634644)
\curveto(776.8212665,325.93634191)(776.83126649,326.00134185)(776.85126709,326.06134644)
\curveto(776.89126643,326.23134162)(776.9262664,326.40134145)(776.95626709,326.57134644)
\curveto(776.98626634,326.74134111)(777.03126629,326.90134095)(777.09126709,327.05134644)
\curveto(777.30126602,327.57134028)(777.55626577,327.99133986)(777.85626709,328.31134644)
\curveto(778.15626517,328.63133922)(778.56626476,328.89633895)(779.08626709,329.10634644)
\curveto(779.19626413,329.15633869)(779.31626401,329.19133866)(779.44626709,329.21134644)
\curveto(779.57626375,329.23133862)(779.71126361,329.25633859)(779.85126709,329.28634644)
\curveto(779.9212634,329.29633855)(779.99126333,329.30133855)(780.06126709,329.30134644)
\curveto(780.13126319,329.31133854)(780.20626312,329.32133853)(780.28626709,329.33134644)
}
}
{
\newrgbcolor{curcolor}{0 0 0}
\pscustom[linestyle=none,fillstyle=solid,fillcolor=curcolor]
{
\newpath
\moveto(785.49290771,330.65134644)
\curveto(785.41290659,330.71133714)(785.36790664,330.81633703)(785.35790771,330.96634644)
\lineto(785.35790771,331.43134644)
\lineto(785.35790771,331.68634644)
\curveto(785.35790665,331.77633607)(785.37290663,331.851336)(785.40290771,331.91134644)
\curveto(785.44290656,331.99133586)(785.52290648,332.0513358)(785.64290771,332.09134644)
\curveto(785.66290634,332.10133575)(785.68290632,332.10133575)(785.70290771,332.09134644)
\curveto(785.73290627,332.09133576)(785.75790625,332.09633575)(785.77790771,332.10634644)
\curveto(785.94790606,332.10633574)(786.1079059,332.10133575)(786.25790771,332.09134644)
\curveto(786.4079056,332.08133577)(786.5079055,332.02133583)(786.55790771,331.91134644)
\curveto(786.58790542,331.851336)(786.6029054,331.77633607)(786.60290771,331.68634644)
\lineto(786.60290771,331.43134644)
\curveto(786.6029054,331.2513366)(786.59790541,331.08133677)(786.58790771,330.92134644)
\curveto(786.58790542,330.76133709)(786.52290548,330.65633719)(786.39290771,330.60634644)
\curveto(786.34290566,330.58633726)(786.28790572,330.57633727)(786.22790771,330.57634644)
\lineto(786.06290771,330.57634644)
\lineto(785.74790771,330.57634644)
\curveto(785.64790636,330.57633727)(785.56290644,330.60133725)(785.49290771,330.65134644)
\moveto(786.60290771,322.14634644)
\lineto(786.60290771,321.83134644)
\curveto(786.61290539,321.73134612)(786.59290541,321.6513462)(786.54290771,321.59134644)
\curveto(786.51290549,321.53134632)(786.46790554,321.49134636)(786.40790771,321.47134644)
\curveto(786.34790566,321.46134639)(786.27790573,321.4463464)(786.19790771,321.42634644)
\lineto(785.97290771,321.42634644)
\curveto(785.84290616,321.42634642)(785.72790628,321.43134642)(785.62790771,321.44134644)
\curveto(785.53790647,321.46134639)(785.46790654,321.51134634)(785.41790771,321.59134644)
\curveto(785.37790663,321.6513462)(785.35790665,321.72634612)(785.35790771,321.81634644)
\lineto(785.35790771,322.10134644)
\lineto(785.35790771,328.44634644)
\lineto(785.35790771,328.76134644)
\curveto(785.35790665,328.87133898)(785.38290662,328.95633889)(785.43290771,329.01634644)
\curveto(785.46290654,329.06633878)(785.5029065,329.09633875)(785.55290771,329.10634644)
\curveto(785.6029064,329.11633873)(785.65790635,329.13133872)(785.71790771,329.15134644)
\curveto(785.73790627,329.1513387)(785.75790625,329.1463387)(785.77790771,329.13634644)
\curveto(785.8079062,329.13633871)(785.83290617,329.14133871)(785.85290771,329.15134644)
\curveto(785.98290602,329.1513387)(786.11290589,329.1463387)(786.24290771,329.13634644)
\curveto(786.38290562,329.13633871)(786.47790553,329.09633875)(786.52790771,329.01634644)
\curveto(786.57790543,328.95633889)(786.6029054,328.87633897)(786.60290771,328.77634644)
\lineto(786.60290771,328.49134644)
\lineto(786.60290771,322.14634644)
}
}
{
\newrgbcolor{curcolor}{0 0 0}
\pscustom[linestyle=none,fillstyle=solid,fillcolor=curcolor]
{
\newpath
\moveto(795.43275146,321.98134644)
\curveto(795.46274363,321.82134603)(795.44774365,321.68634616)(795.38775146,321.57634644)
\curveto(795.32774377,321.47634637)(795.24774385,321.40134645)(795.14775146,321.35134644)
\curveto(795.097744,321.33134652)(795.04274405,321.32134653)(794.98275146,321.32134644)
\curveto(794.93274416,321.32134653)(794.87774422,321.31134654)(794.81775146,321.29134644)
\curveto(794.5977445,321.24134661)(794.37774472,321.25634659)(794.15775146,321.33634644)
\curveto(793.94774515,321.40634644)(793.80274529,321.49634635)(793.72275146,321.60634644)
\curveto(793.67274542,321.67634617)(793.62774547,321.75634609)(793.58775146,321.84634644)
\curveto(793.54774555,321.9463459)(793.4977456,322.02634582)(793.43775146,322.08634644)
\curveto(793.41774568,322.10634574)(793.3927457,322.12634572)(793.36275146,322.14634644)
\curveto(793.34274575,322.16634568)(793.31274578,322.17134568)(793.27275146,322.16134644)
\curveto(793.16274593,322.13134572)(793.05774604,322.07634577)(792.95775146,321.99634644)
\curveto(792.86774623,321.91634593)(792.77774632,321.846346)(792.68775146,321.78634644)
\curveto(792.55774654,321.70634614)(792.41774668,321.63134622)(792.26775146,321.56134644)
\curveto(792.11774698,321.50134635)(791.95774714,321.4463464)(791.78775146,321.39634644)
\curveto(791.68774741,321.36634648)(791.57774752,321.3463465)(791.45775146,321.33634644)
\curveto(791.34774775,321.32634652)(791.23774786,321.31134654)(791.12775146,321.29134644)
\curveto(791.07774802,321.28134657)(791.03274806,321.27634657)(790.99275146,321.27634644)
\lineto(790.88775146,321.27634644)
\curveto(790.77774832,321.25634659)(790.67274842,321.25634659)(790.57275146,321.27634644)
\lineto(790.43775146,321.27634644)
\curveto(790.38774871,321.28634656)(790.33774876,321.29134656)(790.28775146,321.29134644)
\curveto(790.23774886,321.29134656)(790.1927489,321.30134655)(790.15275146,321.32134644)
\curveto(790.11274898,321.33134652)(790.07774902,321.33634651)(790.04775146,321.33634644)
\curveto(790.02774907,321.32634652)(790.00274909,321.32634652)(789.97275146,321.33634644)
\lineto(789.73275146,321.39634644)
\curveto(789.65274944,321.40634644)(789.57774952,321.42634642)(789.50775146,321.45634644)
\curveto(789.20774989,321.58634626)(788.96275013,321.73134612)(788.77275146,321.89134644)
\curveto(788.5927505,322.06134579)(788.44275065,322.29634555)(788.32275146,322.59634644)
\curveto(788.23275086,322.81634503)(788.18775091,323.08134477)(788.18775146,323.39134644)
\lineto(788.18775146,323.70634644)
\curveto(788.1977509,323.75634409)(788.20275089,323.80634404)(788.20275146,323.85634644)
\lineto(788.23275146,324.03634644)
\lineto(788.35275146,324.36634644)
\curveto(788.3927507,324.47634337)(788.44275065,324.57634327)(788.50275146,324.66634644)
\curveto(788.68275041,324.95634289)(788.92775017,325.17134268)(789.23775146,325.31134644)
\curveto(789.54774955,325.4513424)(789.88774921,325.57634227)(790.25775146,325.68634644)
\curveto(790.3977487,325.72634212)(790.54274855,325.75634209)(790.69275146,325.77634644)
\curveto(790.84274825,325.79634205)(790.9927481,325.82134203)(791.14275146,325.85134644)
\curveto(791.21274788,325.87134198)(791.27774782,325.88134197)(791.33775146,325.88134644)
\curveto(791.40774769,325.88134197)(791.48274761,325.89134196)(791.56275146,325.91134644)
\curveto(791.63274746,325.93134192)(791.70274739,325.94134191)(791.77275146,325.94134644)
\curveto(791.84274725,325.9513419)(791.91774718,325.96634188)(791.99775146,325.98634644)
\curveto(792.24774685,326.0463418)(792.48274661,326.09634175)(792.70275146,326.13634644)
\curveto(792.92274617,326.18634166)(793.097746,326.30134155)(793.22775146,326.48134644)
\curveto(793.28774581,326.56134129)(793.33774576,326.66134119)(793.37775146,326.78134644)
\curveto(793.41774568,326.91134094)(793.41774568,327.0513408)(793.37775146,327.20134644)
\curveto(793.31774578,327.44134041)(793.22774587,327.63134022)(793.10775146,327.77134644)
\curveto(792.9977461,327.91133994)(792.83774626,328.02133983)(792.62775146,328.10134644)
\curveto(792.50774659,328.1513397)(792.36274673,328.18633966)(792.19275146,328.20634644)
\curveto(792.03274706,328.22633962)(791.86274723,328.23633961)(791.68275146,328.23634644)
\curveto(791.50274759,328.23633961)(791.32774777,328.22633962)(791.15775146,328.20634644)
\curveto(790.98774811,328.18633966)(790.84274825,328.15633969)(790.72275146,328.11634644)
\curveto(790.55274854,328.05633979)(790.38774871,327.97133988)(790.22775146,327.86134644)
\curveto(790.14774895,327.80134005)(790.07274902,327.72134013)(790.00275146,327.62134644)
\curveto(789.94274915,327.53134032)(789.88774921,327.43134042)(789.83775146,327.32134644)
\curveto(789.80774929,327.24134061)(789.77774932,327.15634069)(789.74775146,327.06634644)
\curveto(789.72774937,326.97634087)(789.68274941,326.90634094)(789.61275146,326.85634644)
\curveto(789.57274952,326.82634102)(789.50274959,326.80134105)(789.40275146,326.78134644)
\curveto(789.31274978,326.77134108)(789.21774988,326.76634108)(789.11775146,326.76634644)
\curveto(789.01775008,326.76634108)(788.91775018,326.77134108)(788.81775146,326.78134644)
\curveto(788.72775037,326.80134105)(788.66275043,326.82634102)(788.62275146,326.85634644)
\curveto(788.58275051,326.88634096)(788.55275054,326.93634091)(788.53275146,327.00634644)
\curveto(788.51275058,327.07634077)(788.51275058,327.1513407)(788.53275146,327.23134644)
\curveto(788.56275053,327.36134049)(788.5927505,327.48134037)(788.62275146,327.59134644)
\curveto(788.66275043,327.71134014)(788.70775039,327.82634002)(788.75775146,327.93634644)
\curveto(788.94775015,328.28633956)(789.18774991,328.55633929)(789.47775146,328.74634644)
\curveto(789.76774933,328.9463389)(790.12774897,329.10633874)(790.55775146,329.22634644)
\curveto(790.65774844,329.2463386)(790.75774834,329.26133859)(790.85775146,329.27134644)
\curveto(790.96774813,329.28133857)(791.07774802,329.29633855)(791.18775146,329.31634644)
\curveto(791.22774787,329.32633852)(791.2927478,329.32633852)(791.38275146,329.31634644)
\curveto(791.47274762,329.31633853)(791.52774757,329.32633852)(791.54775146,329.34634644)
\curveto(792.24774685,329.35633849)(792.85774624,329.27633857)(793.37775146,329.10634644)
\curveto(793.8977452,328.93633891)(794.26274483,328.61133924)(794.47275146,328.13134644)
\curveto(794.56274453,327.93133992)(794.61274448,327.69634015)(794.62275146,327.42634644)
\curveto(794.64274445,327.16634068)(794.65274444,326.89134096)(794.65275146,326.60134644)
\lineto(794.65275146,323.28634644)
\curveto(794.65274444,323.1463447)(794.65774444,323.01134484)(794.66775146,322.88134644)
\curveto(794.67774442,322.7513451)(794.70774439,322.6463452)(794.75775146,322.56634644)
\curveto(794.80774429,322.49634535)(794.87274422,322.4463454)(794.95275146,322.41634644)
\curveto(795.04274405,322.37634547)(795.12774397,322.3463455)(795.20775146,322.32634644)
\curveto(795.28774381,322.31634553)(795.34774375,322.27134558)(795.38775146,322.19134644)
\curveto(795.40774369,322.16134569)(795.41774368,322.13134572)(795.41775146,322.10134644)
\curveto(795.41774368,322.07134578)(795.42274367,322.03134582)(795.43275146,321.98134644)
\moveto(793.28775146,323.64634644)
\curveto(793.34774575,323.78634406)(793.37774572,323.9463439)(793.37775146,324.12634644)
\curveto(793.38774571,324.31634353)(793.3927457,324.51134334)(793.39275146,324.71134644)
\curveto(793.3927457,324.82134303)(793.38774571,324.92134293)(793.37775146,325.01134644)
\curveto(793.36774573,325.10134275)(793.32774577,325.17134268)(793.25775146,325.22134644)
\curveto(793.22774587,325.24134261)(793.15774594,325.2513426)(793.04775146,325.25134644)
\curveto(793.02774607,325.23134262)(792.9927461,325.22134263)(792.94275146,325.22134644)
\curveto(792.8927462,325.22134263)(792.84774625,325.21134264)(792.80775146,325.19134644)
\curveto(792.72774637,325.17134268)(792.63774646,325.1513427)(792.53775146,325.13134644)
\lineto(792.23775146,325.07134644)
\curveto(792.20774689,325.07134278)(792.17274692,325.06634278)(792.13275146,325.05634644)
\lineto(792.02775146,325.05634644)
\curveto(791.87774722,325.01634283)(791.71274738,324.99134286)(791.53275146,324.98134644)
\curveto(791.36274773,324.98134287)(791.20274789,324.96134289)(791.05275146,324.92134644)
\curveto(790.97274812,324.90134295)(790.8977482,324.88134297)(790.82775146,324.86134644)
\curveto(790.76774833,324.851343)(790.6977484,324.83634301)(790.61775146,324.81634644)
\curveto(790.45774864,324.76634308)(790.30774879,324.70134315)(790.16775146,324.62134644)
\curveto(790.02774907,324.5513433)(789.90774919,324.46134339)(789.80775146,324.35134644)
\curveto(789.70774939,324.24134361)(789.63274946,324.10634374)(789.58275146,323.94634644)
\curveto(789.53274956,323.79634405)(789.51274958,323.61134424)(789.52275146,323.39134644)
\curveto(789.52274957,323.29134456)(789.53774956,323.19634465)(789.56775146,323.10634644)
\curveto(789.60774949,323.02634482)(789.65274944,322.9513449)(789.70275146,322.88134644)
\curveto(789.78274931,322.77134508)(789.88774921,322.67634517)(790.01775146,322.59634644)
\curveto(790.14774895,322.52634532)(790.28774881,322.46634538)(790.43775146,322.41634644)
\curveto(790.48774861,322.40634544)(790.53774856,322.40134545)(790.58775146,322.40134644)
\curveto(790.63774846,322.40134545)(790.68774841,322.39634545)(790.73775146,322.38634644)
\curveto(790.80774829,322.36634548)(790.8927482,322.3513455)(790.99275146,322.34134644)
\curveto(791.10274799,322.34134551)(791.1927479,322.3513455)(791.26275146,322.37134644)
\curveto(791.32274777,322.39134546)(791.38274771,322.39634545)(791.44275146,322.38634644)
\curveto(791.50274759,322.38634546)(791.56274753,322.39634545)(791.62275146,322.41634644)
\curveto(791.70274739,322.43634541)(791.77774732,322.4513454)(791.84775146,322.46134644)
\curveto(791.92774717,322.47134538)(792.00274709,322.49134536)(792.07275146,322.52134644)
\curveto(792.36274673,322.64134521)(792.60774649,322.78634506)(792.80775146,322.95634644)
\curveto(793.01774608,323.12634472)(793.17774592,323.35634449)(793.28775146,323.64634644)
}
}
{
\newrgbcolor{curcolor}{0 0 0}
\pscustom[linestyle=none,fillstyle=solid,fillcolor=curcolor]
{
\newpath
\moveto(804.02939209,325.46134644)
\curveto(804.03938374,325.41134244)(804.04438373,325.34134251)(804.04439209,325.25134644)
\curveto(804.04438373,325.17134268)(804.03938374,325.10634274)(804.02939209,325.05634644)
\curveto(804.02938375,325.01634283)(804.02438375,324.97634287)(804.01439209,324.93634644)
\lineto(804.01439209,324.81634644)
\curveto(803.99438378,324.73634311)(803.98438379,324.65634319)(803.98439209,324.57634644)
\curveto(803.98438379,324.49634335)(803.9743838,324.41634343)(803.95439209,324.33634644)
\curveto(803.94438383,324.29634355)(803.93938384,324.25634359)(803.93939209,324.21634644)
\curveto(803.93938384,324.18634366)(803.93438384,324.1513437)(803.92439209,324.11134644)
\curveto(803.89438388,324.00134385)(803.86438391,323.89634395)(803.83439209,323.79634644)
\curveto(803.81438396,323.69634415)(803.78438399,323.59634425)(803.74439209,323.49634644)
\curveto(803.60438417,323.1463447)(803.43438434,322.83134502)(803.23439209,322.55134644)
\curveto(803.03438474,322.27134558)(802.78438499,322.03134582)(802.48439209,321.83134644)
\curveto(802.33438544,321.73134612)(802.18938559,321.6463462)(802.04939209,321.57634644)
\curveto(801.93938584,321.52634632)(801.82938595,321.48634636)(801.71939209,321.45634644)
\curveto(801.61938616,321.42634642)(801.51438626,321.39634645)(801.40439209,321.36634644)
\curveto(801.33438644,321.3463465)(801.26938651,321.33634651)(801.20939209,321.33634644)
\curveto(801.14938663,321.32634652)(801.08938669,321.31134654)(801.02939209,321.29134644)
\lineto(800.87939209,321.29134644)
\curveto(800.82938695,321.27134658)(800.75438702,321.26134659)(800.65439209,321.26134644)
\curveto(800.55438722,321.2513466)(800.4743873,321.25634659)(800.41439209,321.27634644)
\lineto(800.26439209,321.27634644)
\curveto(800.22438755,321.28634656)(800.1793876,321.29134656)(800.12939209,321.29134644)
\curveto(800.08938769,321.29134656)(800.04438773,321.29634655)(799.99439209,321.30634644)
\curveto(799.84438793,321.3463465)(799.69438808,321.38134647)(799.54439209,321.41134644)
\curveto(799.40438837,321.44134641)(799.26438851,321.48634636)(799.12439209,321.54634644)
\curveto(798.92438885,321.62634622)(798.74438903,321.72634612)(798.58439209,321.84634644)
\lineto(798.40439209,321.99634644)
\curveto(798.34438943,322.05634579)(798.2743895,322.09634575)(798.19439209,322.11634644)
\curveto(798.13438964,322.12634572)(798.08438969,322.11134574)(798.04439209,322.07134644)
\curveto(798.01438976,322.04134581)(797.98938979,321.99634585)(797.96939209,321.93634644)
\curveto(797.95938982,321.87634597)(797.94938983,321.81134604)(797.93939209,321.74134644)
\curveto(797.93938984,321.68134617)(797.92938985,321.63634621)(797.90939209,321.60634644)
\curveto(797.86938991,321.55634629)(797.82438995,321.51134634)(797.77439209,321.47134644)
\curveto(797.72439005,321.4513464)(797.65439012,321.43634641)(797.56439209,321.42634644)
\lineto(797.29439209,321.42634644)
\curveto(797.20439057,321.42634642)(797.11939066,321.43134642)(797.03939209,321.44134644)
\curveto(796.95939082,321.46134639)(796.89939088,321.48134637)(796.85939209,321.50134644)
\curveto(796.83939094,321.52134633)(796.81939096,321.5463463)(796.79939209,321.57634644)
\lineto(796.73939209,321.66634644)
\curveto(796.70939107,321.7463461)(796.69439108,321.86634598)(796.69439209,322.02634644)
\curveto(796.70439107,322.18634566)(796.70939107,322.32134553)(796.70939209,322.43134644)
\lineto(796.70939209,331.23634644)
\curveto(796.70939107,331.35633649)(796.70439107,331.48133637)(796.69439209,331.61134644)
\curveto(796.69439108,331.7513361)(796.71939106,331.86133599)(796.76939209,331.94134644)
\curveto(796.80939097,332.00133585)(796.8743909,332.0513358)(796.96439209,332.09134644)
\curveto(796.98439079,332.09133576)(797.00939077,332.09133576)(797.03939209,332.09134644)
\curveto(797.06939071,332.10133575)(797.09439068,332.10633574)(797.11439209,332.10634644)
\curveto(797.25439052,332.11633573)(797.39939038,332.11633573)(797.54939209,332.10634644)
\curveto(797.70939007,332.10633574)(797.81938996,332.06633578)(797.87939209,331.98634644)
\curveto(797.92938985,331.90633594)(797.95438982,331.79133606)(797.95439209,331.64134644)
\lineto(797.95439209,331.23634644)
\lineto(797.95439209,329.48134644)
\lineto(797.95439209,329.22634644)
\lineto(797.95439209,328.94134644)
\curveto(797.96438981,328.851339)(797.9743898,328.76633908)(797.98439209,328.68634644)
\curveto(798.00438977,328.61633923)(798.03438974,328.56633928)(798.07439209,328.53634644)
\curveto(798.11438966,328.50633934)(798.15938962,328.50133935)(798.20939209,328.52134644)
\curveto(798.25938952,328.54133931)(798.29938948,328.56133929)(798.32939209,328.58134644)
\curveto(798.3793894,328.62133923)(798.42438935,328.66133919)(798.46439209,328.70134644)
\lineto(798.61439209,328.82134644)
\curveto(798.68438909,328.87133898)(798.75438902,328.91633893)(798.82439209,328.95634644)
\lineto(799.06439209,329.07634644)
\curveto(799.24438853,329.16633868)(799.45938832,329.23133862)(799.70939209,329.27134644)
\curveto(799.95938782,329.31133854)(800.21438756,329.33133852)(800.47439209,329.33134644)
\curveto(800.73438704,329.33133852)(800.98938679,329.30633854)(801.23939209,329.25634644)
\curveto(801.48938629,329.21633863)(801.70938607,329.15633869)(801.89939209,329.07634644)
\curveto(802.29938548,328.90633894)(802.64438513,328.67133918)(802.93439209,328.37134644)
\curveto(803.22438455,328.07133978)(803.45438432,327.72134013)(803.62439209,327.32134644)
\curveto(803.6743841,327.21134064)(803.71438406,327.10134075)(803.74439209,326.99134644)
\curveto(803.78438399,326.89134096)(803.82438395,326.78634106)(803.86439209,326.67634644)
\curveto(803.89438388,326.56634128)(803.91438386,326.4513414)(803.92439209,326.33134644)
\lineto(803.98439209,326.00134644)
\curveto(803.99438378,325.97134188)(803.99938378,325.93634191)(803.99939209,325.89634644)
\curveto(803.99938378,325.86634198)(804.00438377,325.83634201)(804.01439209,325.80634644)
\curveto(804.03438374,325.7463421)(804.03438374,325.68634216)(804.01439209,325.62634644)
\curveto(804.00438377,325.57634227)(804.00938377,325.52134233)(804.02939209,325.46134644)
\moveto(802.69439209,325.07134644)
\curveto(802.71438506,325.12134273)(802.71938506,325.18134267)(802.70939209,325.25134644)
\curveto(802.69938508,325.32134253)(802.69438508,325.38634246)(802.69439209,325.44634644)
\curveto(802.69438508,325.61634223)(802.68438509,325.77634207)(802.66439209,325.92634644)
\curveto(802.65438512,326.07634177)(802.62438515,326.21134164)(802.57439209,326.33134644)
\curveto(802.54438523,326.43134142)(802.51938526,326.52134133)(802.49939209,326.60134644)
\curveto(802.4793853,326.68134117)(802.44938533,326.76134109)(802.40939209,326.84134644)
\curveto(802.29938548,327.09134076)(802.14938563,327.32134053)(801.95939209,327.53134644)
\curveto(801.76938601,327.7513401)(801.54938623,327.91633993)(801.29939209,328.02634644)
\curveto(801.21938656,328.05633979)(801.13938664,328.08133977)(801.05939209,328.10134644)
\curveto(800.98938679,328.13133972)(800.91438686,328.15633969)(800.83439209,328.17634644)
\curveto(800.72438705,328.20633964)(800.61438716,328.22133963)(800.50439209,328.22134644)
\curveto(800.39438738,328.23133962)(800.2743875,328.23633961)(800.14439209,328.23634644)
\curveto(800.09438768,328.22633962)(800.04938773,328.21633963)(800.00939209,328.20634644)
\lineto(799.87439209,328.20634644)
\lineto(799.60439209,328.14634644)
\curveto(799.52438825,328.12633972)(799.44438833,328.09633975)(799.36439209,328.05634644)
\curveto(799.02438875,327.91633993)(798.75438902,327.70634014)(798.55439209,327.42634644)
\curveto(798.35438942,327.15634069)(798.19438958,326.83634101)(798.07439209,326.46634644)
\curveto(798.03438974,326.35634149)(798.00938977,326.2463416)(797.99939209,326.13634644)
\curveto(797.98938979,326.02634182)(797.96938981,325.91134194)(797.93939209,325.79134644)
\curveto(797.92938985,325.74134211)(797.92938985,325.69634215)(797.93939209,325.65634644)
\curveto(797.94938983,325.61634223)(797.94438983,325.57134228)(797.92439209,325.52134644)
\curveto(797.91438986,325.47134238)(797.90938987,325.39634245)(797.90939209,325.29634644)
\curveto(797.90938987,325.20634264)(797.91438986,325.13634271)(797.92439209,325.08634644)
\lineto(797.92439209,324.96634644)
\curveto(797.93438984,324.92634292)(797.93938984,324.88634296)(797.93939209,324.84634644)
\curveto(797.93938984,324.80634304)(797.94438983,324.77134308)(797.95439209,324.74134644)
\curveto(797.96438981,324.71134314)(797.96938981,324.67634317)(797.96939209,324.63634644)
\curveto(797.96938981,324.60634324)(797.9743898,324.57634327)(797.98439209,324.54634644)
\curveto(798.00438977,324.46634338)(798.01938976,324.38634346)(798.02939209,324.30634644)
\lineto(798.08939209,324.06634644)
\curveto(798.19938958,323.72634412)(798.34938943,323.43634441)(798.53939209,323.19634644)
\curveto(798.73938904,322.95634489)(798.98438879,322.75634509)(799.27439209,322.59634644)
\curveto(799.36438841,322.5463453)(799.45938832,322.50634534)(799.55939209,322.47634644)
\curveto(799.65938812,322.45634539)(799.76438801,322.43134542)(799.87439209,322.40134644)
\curveto(799.92438785,322.38134547)(799.96938781,322.37134548)(800.00939209,322.37134644)
\curveto(800.05938772,322.38134547)(800.10938767,322.38134547)(800.15939209,322.37134644)
\curveto(800.19938758,322.36134549)(800.24438753,322.35634549)(800.29439209,322.35634644)
\lineto(800.42939209,322.35634644)
\lineto(800.56439209,322.35634644)
\curveto(800.60438717,322.36634548)(800.63938714,322.37134548)(800.66939209,322.37134644)
\curveto(800.69938708,322.37134548)(800.73438704,322.37634547)(800.77439209,322.38634644)
\curveto(800.85438692,322.40634544)(800.92938685,322.42134543)(800.99939209,322.43134644)
\curveto(801.06938671,322.4513454)(801.14438663,322.47634537)(801.22439209,322.50634644)
\curveto(801.53438624,322.63634521)(801.78438599,322.80634504)(801.97439209,323.01634644)
\curveto(802.16438561,323.23634461)(802.32438545,323.50134435)(802.45439209,323.81134644)
\curveto(802.50438527,323.9513439)(802.53938524,324.09134376)(802.55939209,324.23134644)
\curveto(802.58938519,324.38134347)(802.62438515,324.53134332)(802.66439209,324.68134644)
\curveto(802.68438509,324.73134312)(802.68938509,324.77634307)(802.67939209,324.81634644)
\curveto(802.6793851,324.86634298)(802.68438509,324.91634293)(802.69439209,324.96634644)
\lineto(802.69439209,325.07134644)
}
}
{
\newrgbcolor{curcolor}{0 0 0}
\pscustom[linestyle=none,fillstyle=solid,fillcolor=curcolor]
{
\newpath
\moveto(805.79564209,330.65134644)
\curveto(805.71564097,330.71133714)(805.67064101,330.81633703)(805.66064209,330.96634644)
\lineto(805.66064209,331.43134644)
\lineto(805.66064209,331.68634644)
\curveto(805.66064102,331.77633607)(805.67564101,331.851336)(805.70564209,331.91134644)
\curveto(805.74564094,331.99133586)(805.82564086,332.0513358)(805.94564209,332.09134644)
\curveto(805.96564072,332.10133575)(805.9856407,332.10133575)(806.00564209,332.09134644)
\curveto(806.03564065,332.09133576)(806.06064062,332.09633575)(806.08064209,332.10634644)
\curveto(806.25064043,332.10633574)(806.41064027,332.10133575)(806.56064209,332.09134644)
\curveto(806.71063997,332.08133577)(806.81063987,332.02133583)(806.86064209,331.91134644)
\curveto(806.89063979,331.851336)(806.90563978,331.77633607)(806.90564209,331.68634644)
\lineto(806.90564209,331.43134644)
\curveto(806.90563978,331.2513366)(806.90063978,331.08133677)(806.89064209,330.92134644)
\curveto(806.89063979,330.76133709)(806.82563986,330.65633719)(806.69564209,330.60634644)
\curveto(806.64564004,330.58633726)(806.59064009,330.57633727)(806.53064209,330.57634644)
\lineto(806.36564209,330.57634644)
\lineto(806.05064209,330.57634644)
\curveto(805.95064073,330.57633727)(805.86564082,330.60133725)(805.79564209,330.65134644)
\moveto(806.90564209,322.14634644)
\lineto(806.90564209,321.83134644)
\curveto(806.91563977,321.73134612)(806.89563979,321.6513462)(806.84564209,321.59134644)
\curveto(806.81563987,321.53134632)(806.77063991,321.49134636)(806.71064209,321.47134644)
\curveto(806.65064003,321.46134639)(806.5806401,321.4463464)(806.50064209,321.42634644)
\lineto(806.27564209,321.42634644)
\curveto(806.14564054,321.42634642)(806.03064065,321.43134642)(805.93064209,321.44134644)
\curveto(805.84064084,321.46134639)(805.77064091,321.51134634)(805.72064209,321.59134644)
\curveto(805.680641,321.6513462)(805.66064102,321.72634612)(805.66064209,321.81634644)
\lineto(805.66064209,322.10134644)
\lineto(805.66064209,328.44634644)
\lineto(805.66064209,328.76134644)
\curveto(805.66064102,328.87133898)(805.685641,328.95633889)(805.73564209,329.01634644)
\curveto(805.76564092,329.06633878)(805.80564088,329.09633875)(805.85564209,329.10634644)
\curveto(805.90564078,329.11633873)(805.96064072,329.13133872)(806.02064209,329.15134644)
\curveto(806.04064064,329.1513387)(806.06064062,329.1463387)(806.08064209,329.13634644)
\curveto(806.11064057,329.13633871)(806.13564055,329.14133871)(806.15564209,329.15134644)
\curveto(806.2856404,329.1513387)(806.41564027,329.1463387)(806.54564209,329.13634644)
\curveto(806.68564,329.13633871)(806.7806399,329.09633875)(806.83064209,329.01634644)
\curveto(806.8806398,328.95633889)(806.90563978,328.87633897)(806.90564209,328.77634644)
\lineto(806.90564209,328.49134644)
\lineto(806.90564209,322.14634644)
}
}
{
\newrgbcolor{curcolor}{0 0 0}
\pscustom[linestyle=none,fillstyle=solid,fillcolor=curcolor]
{
\newpath
\moveto(809.42048584,332.10634644)
\curveto(809.55048422,332.10633574)(809.68548409,332.10633574)(809.82548584,332.10634644)
\curveto(809.9754838,332.10633574)(810.08548369,332.07133578)(810.15548584,332.00134644)
\curveto(810.20548357,331.93133592)(810.23048354,331.83633601)(810.23048584,331.71634644)
\curveto(810.24048353,331.60633624)(810.24548353,331.49133636)(810.24548584,331.37134644)
\lineto(810.24548584,330.03634644)
\lineto(810.24548584,323.96134644)
\lineto(810.24548584,322.28134644)
\lineto(810.24548584,321.89134644)
\curveto(810.24548353,321.7513461)(810.22048355,321.64134621)(810.17048584,321.56134644)
\curveto(810.14048363,321.51134634)(810.09548368,321.48134637)(810.03548584,321.47134644)
\curveto(809.98548379,321.46134639)(809.92048385,321.4463464)(809.84048584,321.42634644)
\lineto(809.63048584,321.42634644)
\lineto(809.31548584,321.42634644)
\curveto(809.21548456,321.43634641)(809.14048463,321.47134638)(809.09048584,321.53134644)
\curveto(809.04048473,321.61134624)(809.01048476,321.71134614)(809.00048584,321.83134644)
\lineto(809.00048584,322.20634644)
\lineto(809.00048584,323.58634644)
\lineto(809.00048584,329.82634644)
\lineto(809.00048584,331.29634644)
\curveto(809.00048477,331.40633644)(808.99548478,331.52133633)(808.98548584,331.64134644)
\curveto(808.98548479,331.77133608)(809.01048476,331.87133598)(809.06048584,331.94134644)
\curveto(809.10048467,332.00133585)(809.1754846,332.0513358)(809.28548584,332.09134644)
\curveto(809.30548447,332.10133575)(809.32548445,332.10133575)(809.34548584,332.09134644)
\curveto(809.3754844,332.09133576)(809.40048437,332.09633575)(809.42048584,332.10634644)
}
}
{
\newrgbcolor{curcolor}{0 0 0}
\pscustom[linestyle=none,fillstyle=solid,fillcolor=curcolor]
{
\newpath
\moveto(812.47532959,330.65134644)
\curveto(812.39532847,330.71133714)(812.35032851,330.81633703)(812.34032959,330.96634644)
\lineto(812.34032959,331.43134644)
\lineto(812.34032959,331.68634644)
\curveto(812.34032852,331.77633607)(812.35532851,331.851336)(812.38532959,331.91134644)
\curveto(812.42532844,331.99133586)(812.50532836,332.0513358)(812.62532959,332.09134644)
\curveto(812.64532822,332.10133575)(812.6653282,332.10133575)(812.68532959,332.09134644)
\curveto(812.71532815,332.09133576)(812.74032812,332.09633575)(812.76032959,332.10634644)
\curveto(812.93032793,332.10633574)(813.09032777,332.10133575)(813.24032959,332.09134644)
\curveto(813.39032747,332.08133577)(813.49032737,332.02133583)(813.54032959,331.91134644)
\curveto(813.57032729,331.851336)(813.58532728,331.77633607)(813.58532959,331.68634644)
\lineto(813.58532959,331.43134644)
\curveto(813.58532728,331.2513366)(813.58032728,331.08133677)(813.57032959,330.92134644)
\curveto(813.57032729,330.76133709)(813.50532736,330.65633719)(813.37532959,330.60634644)
\curveto(813.32532754,330.58633726)(813.27032759,330.57633727)(813.21032959,330.57634644)
\lineto(813.04532959,330.57634644)
\lineto(812.73032959,330.57634644)
\curveto(812.63032823,330.57633727)(812.54532832,330.60133725)(812.47532959,330.65134644)
\moveto(813.58532959,322.14634644)
\lineto(813.58532959,321.83134644)
\curveto(813.59532727,321.73134612)(813.57532729,321.6513462)(813.52532959,321.59134644)
\curveto(813.49532737,321.53134632)(813.45032741,321.49134636)(813.39032959,321.47134644)
\curveto(813.33032753,321.46134639)(813.2603276,321.4463464)(813.18032959,321.42634644)
\lineto(812.95532959,321.42634644)
\curveto(812.82532804,321.42634642)(812.71032815,321.43134642)(812.61032959,321.44134644)
\curveto(812.52032834,321.46134639)(812.45032841,321.51134634)(812.40032959,321.59134644)
\curveto(812.3603285,321.6513462)(812.34032852,321.72634612)(812.34032959,321.81634644)
\lineto(812.34032959,322.10134644)
\lineto(812.34032959,328.44634644)
\lineto(812.34032959,328.76134644)
\curveto(812.34032852,328.87133898)(812.3653285,328.95633889)(812.41532959,329.01634644)
\curveto(812.44532842,329.06633878)(812.48532838,329.09633875)(812.53532959,329.10634644)
\curveto(812.58532828,329.11633873)(812.64032822,329.13133872)(812.70032959,329.15134644)
\curveto(812.72032814,329.1513387)(812.74032812,329.1463387)(812.76032959,329.13634644)
\curveto(812.79032807,329.13633871)(812.81532805,329.14133871)(812.83532959,329.15134644)
\curveto(812.9653279,329.1513387)(813.09532777,329.1463387)(813.22532959,329.13634644)
\curveto(813.3653275,329.13633871)(813.4603274,329.09633875)(813.51032959,329.01634644)
\curveto(813.5603273,328.95633889)(813.58532728,328.87633897)(813.58532959,328.77634644)
\lineto(813.58532959,328.49134644)
\lineto(813.58532959,322.14634644)
}
}
{
\newrgbcolor{curcolor}{0 0 0}
\pscustom[linestyle=none,fillstyle=solid,fillcolor=curcolor]
{
\newpath
\moveto(822.49017334,322.23634644)
\lineto(822.49017334,321.84634644)
\curveto(822.49016546,321.72634612)(822.46516549,321.62634622)(822.41517334,321.54634644)
\curveto(822.36516559,321.47634637)(822.28016567,321.43634641)(822.16017334,321.42634644)
\lineto(821.81517334,321.42634644)
\curveto(821.7551662,321.42634642)(821.69516626,321.42134643)(821.63517334,321.41134644)
\curveto(821.58516637,321.41134644)(821.54016641,321.42134643)(821.50017334,321.44134644)
\curveto(821.41016654,321.46134639)(821.3501666,321.50134635)(821.32017334,321.56134644)
\curveto(821.28016667,321.61134624)(821.2551667,321.67134618)(821.24517334,321.74134644)
\curveto(821.24516671,321.81134604)(821.23016672,321.88134597)(821.20017334,321.95134644)
\curveto(821.19016676,321.97134588)(821.17516678,321.98634586)(821.15517334,321.99634644)
\curveto(821.14516681,322.01634583)(821.13016682,322.03634581)(821.11017334,322.05634644)
\curveto(821.01016694,322.06634578)(820.93016702,322.0463458)(820.87017334,321.99634644)
\curveto(820.82016713,321.9463459)(820.76516719,321.89634595)(820.70517334,321.84634644)
\curveto(820.50516745,321.69634615)(820.30516765,321.58134627)(820.10517334,321.50134644)
\curveto(819.92516803,321.42134643)(819.71516824,321.36134649)(819.47517334,321.32134644)
\curveto(819.24516871,321.28134657)(819.00516895,321.26134659)(818.75517334,321.26134644)
\curveto(818.51516944,321.2513466)(818.27516968,321.26634658)(818.03517334,321.30634644)
\curveto(817.79517016,321.33634651)(817.58517037,321.39134646)(817.40517334,321.47134644)
\curveto(816.88517107,321.69134616)(816.46517149,321.98634586)(816.14517334,322.35634644)
\curveto(815.82517213,322.73634511)(815.57517238,323.20634464)(815.39517334,323.76634644)
\curveto(815.3551726,323.85634399)(815.32517263,323.9463439)(815.30517334,324.03634644)
\curveto(815.29517266,324.13634371)(815.27517268,324.23634361)(815.24517334,324.33634644)
\curveto(815.23517272,324.38634346)(815.23017272,324.43634341)(815.23017334,324.48634644)
\curveto(815.23017272,324.53634331)(815.22517273,324.58634326)(815.21517334,324.63634644)
\curveto(815.19517276,324.68634316)(815.18517277,324.73634311)(815.18517334,324.78634644)
\curveto(815.19517276,324.846343)(815.19517276,324.90134295)(815.18517334,324.95134644)
\lineto(815.18517334,325.10134644)
\curveto(815.16517279,325.1513427)(815.1551728,325.21634263)(815.15517334,325.29634644)
\curveto(815.1551728,325.37634247)(815.16517279,325.44134241)(815.18517334,325.49134644)
\lineto(815.18517334,325.65634644)
\curveto(815.20517275,325.72634212)(815.21017274,325.79634205)(815.20017334,325.86634644)
\curveto(815.20017275,325.9463419)(815.21017274,326.02134183)(815.23017334,326.09134644)
\curveto(815.24017271,326.14134171)(815.24517271,326.18634166)(815.24517334,326.22634644)
\curveto(815.24517271,326.26634158)(815.2501727,326.31134154)(815.26017334,326.36134644)
\curveto(815.29017266,326.46134139)(815.31517264,326.55634129)(815.33517334,326.64634644)
\curveto(815.3551726,326.7463411)(815.38017257,326.84134101)(815.41017334,326.93134644)
\curveto(815.54017241,327.31134054)(815.70517225,327.6513402)(815.90517334,327.95134644)
\curveto(816.11517184,328.26133959)(816.36517159,328.51633933)(816.65517334,328.71634644)
\curveto(816.82517113,328.83633901)(817.00017095,328.93633891)(817.18017334,329.01634644)
\curveto(817.37017058,329.09633875)(817.57517038,329.16633868)(817.79517334,329.22634644)
\curveto(817.86517009,329.23633861)(817.93017002,329.2463386)(817.99017334,329.25634644)
\curveto(818.06016989,329.26633858)(818.13016982,329.28133857)(818.20017334,329.30134644)
\lineto(818.35017334,329.30134644)
\curveto(818.43016952,329.32133853)(818.54516941,329.33133852)(818.69517334,329.33134644)
\curveto(818.8551691,329.33133852)(818.97516898,329.32133853)(819.05517334,329.30134644)
\curveto(819.09516886,329.29133856)(819.1501688,329.28633856)(819.22017334,329.28634644)
\curveto(819.33016862,329.25633859)(819.44016851,329.23133862)(819.55017334,329.21134644)
\curveto(819.66016829,329.20133865)(819.76516819,329.17133868)(819.86517334,329.12134644)
\curveto(820.01516794,329.06133879)(820.1551678,328.99633885)(820.28517334,328.92634644)
\curveto(820.42516753,328.85633899)(820.5551674,328.77633907)(820.67517334,328.68634644)
\curveto(820.73516722,328.63633921)(820.79516716,328.58133927)(820.85517334,328.52134644)
\curveto(820.92516703,328.47133938)(821.01516694,328.45633939)(821.12517334,328.47634644)
\curveto(821.14516681,328.50633934)(821.16016679,328.53133932)(821.17017334,328.55134644)
\curveto(821.19016676,328.57133928)(821.20516675,328.60133925)(821.21517334,328.64134644)
\curveto(821.24516671,328.73133912)(821.2551667,328.846339)(821.24517334,328.98634644)
\lineto(821.24517334,329.36134644)
\lineto(821.24517334,331.08634644)
\lineto(821.24517334,331.55134644)
\curveto(821.24516671,331.73133612)(821.27016668,331.86133599)(821.32017334,331.94134644)
\curveto(821.36016659,332.01133584)(821.42016653,332.05633579)(821.50017334,332.07634644)
\curveto(821.52016643,332.07633577)(821.54516641,332.07633577)(821.57517334,332.07634644)
\curveto(821.60516635,332.08633576)(821.63016632,332.09133576)(821.65017334,332.09134644)
\curveto(821.79016616,332.10133575)(821.93516602,332.10133575)(822.08517334,332.09134644)
\curveto(822.24516571,332.09133576)(822.3551656,332.0513358)(822.41517334,331.97134644)
\curveto(822.46516549,331.89133596)(822.49016546,331.79133606)(822.49017334,331.67134644)
\lineto(822.49017334,331.29634644)
\lineto(822.49017334,322.23634644)
\moveto(821.27517334,325.07134644)
\curveto(821.29516666,325.12134273)(821.30516665,325.18634266)(821.30517334,325.26634644)
\curveto(821.30516665,325.35634249)(821.29516666,325.42634242)(821.27517334,325.47634644)
\lineto(821.27517334,325.70134644)
\curveto(821.2551667,325.79134206)(821.24016671,325.88134197)(821.23017334,325.97134644)
\curveto(821.22016673,326.07134178)(821.20016675,326.16134169)(821.17017334,326.24134644)
\curveto(821.1501668,326.32134153)(821.13016682,326.39634145)(821.11017334,326.46634644)
\curveto(821.10016685,326.53634131)(821.08016687,326.60634124)(821.05017334,326.67634644)
\curveto(820.93016702,326.97634087)(820.77516718,327.24134061)(820.58517334,327.47134644)
\curveto(820.39516756,327.70134015)(820.1551678,327.88133997)(819.86517334,328.01134644)
\curveto(819.76516819,328.06133979)(819.66016829,328.09633975)(819.55017334,328.11634644)
\curveto(819.4501685,328.1463397)(819.34016861,328.17133968)(819.22017334,328.19134644)
\curveto(819.14016881,328.21133964)(819.0501689,328.22133963)(818.95017334,328.22134644)
\lineto(818.68017334,328.22134644)
\curveto(818.63016932,328.21133964)(818.58516937,328.20133965)(818.54517334,328.19134644)
\lineto(818.41017334,328.19134644)
\curveto(818.33016962,328.17133968)(818.24516971,328.1513397)(818.15517334,328.13134644)
\curveto(818.07516988,328.11133974)(817.99516996,328.08633976)(817.91517334,328.05634644)
\curveto(817.59517036,327.91633993)(817.33517062,327.71134014)(817.13517334,327.44134644)
\curveto(816.94517101,327.18134067)(816.79017116,326.87634097)(816.67017334,326.52634644)
\curveto(816.63017132,326.41634143)(816.60017135,326.30134155)(816.58017334,326.18134644)
\curveto(816.57017138,326.07134178)(816.5551714,325.96134189)(816.53517334,325.85134644)
\curveto(816.53517142,325.81134204)(816.53017142,325.77134208)(816.52017334,325.73134644)
\lineto(816.52017334,325.62634644)
\curveto(816.50017145,325.57634227)(816.49017146,325.52134233)(816.49017334,325.46134644)
\curveto(816.50017145,325.40134245)(816.50517145,325.3463425)(816.50517334,325.29634644)
\lineto(816.50517334,324.96634644)
\curveto(816.50517145,324.86634298)(816.51517144,324.77134308)(816.53517334,324.68134644)
\curveto(816.54517141,324.6513432)(816.5501714,324.60134325)(816.55017334,324.53134644)
\curveto(816.57017138,324.46134339)(816.58517137,324.39134346)(816.59517334,324.32134644)
\lineto(816.65517334,324.11134644)
\curveto(816.76517119,323.76134409)(816.91517104,323.46134439)(817.10517334,323.21134644)
\curveto(817.29517066,322.96134489)(817.53517042,322.75634509)(817.82517334,322.59634644)
\curveto(817.91517004,322.5463453)(818.00516995,322.50634534)(818.09517334,322.47634644)
\curveto(818.18516977,322.4463454)(818.28516967,322.41634543)(818.39517334,322.38634644)
\curveto(818.44516951,322.36634548)(818.49516946,322.36134549)(818.54517334,322.37134644)
\curveto(818.60516935,322.38134547)(818.66016929,322.37634547)(818.71017334,322.35634644)
\curveto(818.7501692,322.3463455)(818.79016916,322.34134551)(818.83017334,322.34134644)
\lineto(818.96517334,322.34134644)
\lineto(819.10017334,322.34134644)
\curveto(819.13016882,322.3513455)(819.18016877,322.35634549)(819.25017334,322.35634644)
\curveto(819.33016862,322.37634547)(819.41016854,322.39134546)(819.49017334,322.40134644)
\curveto(819.57016838,322.42134543)(819.64516831,322.4463454)(819.71517334,322.47634644)
\curveto(820.04516791,322.61634523)(820.31016764,322.79134506)(820.51017334,323.00134644)
\curveto(820.72016723,323.22134463)(820.89516706,323.49634435)(821.03517334,323.82634644)
\curveto(821.08516687,323.93634391)(821.12016683,324.0463438)(821.14017334,324.15634644)
\curveto(821.16016679,324.26634358)(821.18516677,324.37634347)(821.21517334,324.48634644)
\curveto(821.23516672,324.52634332)(821.24516671,324.56134329)(821.24517334,324.59134644)
\curveto(821.24516671,324.63134322)(821.2501667,324.67134318)(821.26017334,324.71134644)
\curveto(821.27016668,324.77134308)(821.27016668,324.83134302)(821.26017334,324.89134644)
\curveto(821.26016669,324.9513429)(821.26516669,325.01134284)(821.27517334,325.07134644)
}
}
{
\newrgbcolor{curcolor}{0 0 0}
\pscustom[linestyle=none,fillstyle=solid,fillcolor=curcolor]
{
\newpath
\moveto(831.32142334,321.98134644)
\curveto(831.35141551,321.82134603)(831.33641552,321.68634616)(831.27642334,321.57634644)
\curveto(831.21641564,321.47634637)(831.13641572,321.40134645)(831.03642334,321.35134644)
\curveto(830.98641587,321.33134652)(830.93141593,321.32134653)(830.87142334,321.32134644)
\curveto(830.82141604,321.32134653)(830.76641609,321.31134654)(830.70642334,321.29134644)
\curveto(830.48641637,321.24134661)(830.26641659,321.25634659)(830.04642334,321.33634644)
\curveto(829.83641702,321.40634644)(829.69141717,321.49634635)(829.61142334,321.60634644)
\curveto(829.5614173,321.67634617)(829.51641734,321.75634609)(829.47642334,321.84634644)
\curveto(829.43641742,321.9463459)(829.38641747,322.02634582)(829.32642334,322.08634644)
\curveto(829.30641755,322.10634574)(829.28141758,322.12634572)(829.25142334,322.14634644)
\curveto(829.23141763,322.16634568)(829.20141766,322.17134568)(829.16142334,322.16134644)
\curveto(829.05141781,322.13134572)(828.94641791,322.07634577)(828.84642334,321.99634644)
\curveto(828.7564181,321.91634593)(828.66641819,321.846346)(828.57642334,321.78634644)
\curveto(828.44641841,321.70634614)(828.30641855,321.63134622)(828.15642334,321.56134644)
\curveto(828.00641885,321.50134635)(827.84641901,321.4463464)(827.67642334,321.39634644)
\curveto(827.57641928,321.36634648)(827.46641939,321.3463465)(827.34642334,321.33634644)
\curveto(827.23641962,321.32634652)(827.12641973,321.31134654)(827.01642334,321.29134644)
\curveto(826.96641989,321.28134657)(826.92141994,321.27634657)(826.88142334,321.27634644)
\lineto(826.77642334,321.27634644)
\curveto(826.66642019,321.25634659)(826.5614203,321.25634659)(826.46142334,321.27634644)
\lineto(826.32642334,321.27634644)
\curveto(826.27642058,321.28634656)(826.22642063,321.29134656)(826.17642334,321.29134644)
\curveto(826.12642073,321.29134656)(826.08142078,321.30134655)(826.04142334,321.32134644)
\curveto(826.00142086,321.33134652)(825.96642089,321.33634651)(825.93642334,321.33634644)
\curveto(825.91642094,321.32634652)(825.89142097,321.32634652)(825.86142334,321.33634644)
\lineto(825.62142334,321.39634644)
\curveto(825.54142132,321.40634644)(825.46642139,321.42634642)(825.39642334,321.45634644)
\curveto(825.09642176,321.58634626)(824.85142201,321.73134612)(824.66142334,321.89134644)
\curveto(824.48142238,322.06134579)(824.33142253,322.29634555)(824.21142334,322.59634644)
\curveto(824.12142274,322.81634503)(824.07642278,323.08134477)(824.07642334,323.39134644)
\lineto(824.07642334,323.70634644)
\curveto(824.08642277,323.75634409)(824.09142277,323.80634404)(824.09142334,323.85634644)
\lineto(824.12142334,324.03634644)
\lineto(824.24142334,324.36634644)
\curveto(824.28142258,324.47634337)(824.33142253,324.57634327)(824.39142334,324.66634644)
\curveto(824.57142229,324.95634289)(824.81642204,325.17134268)(825.12642334,325.31134644)
\curveto(825.43642142,325.4513424)(825.77642108,325.57634227)(826.14642334,325.68634644)
\curveto(826.28642057,325.72634212)(826.43142043,325.75634209)(826.58142334,325.77634644)
\curveto(826.73142013,325.79634205)(826.88141998,325.82134203)(827.03142334,325.85134644)
\curveto(827.10141976,325.87134198)(827.16641969,325.88134197)(827.22642334,325.88134644)
\curveto(827.29641956,325.88134197)(827.37141949,325.89134196)(827.45142334,325.91134644)
\curveto(827.52141934,325.93134192)(827.59141927,325.94134191)(827.66142334,325.94134644)
\curveto(827.73141913,325.9513419)(827.80641905,325.96634188)(827.88642334,325.98634644)
\curveto(828.13641872,326.0463418)(828.37141849,326.09634175)(828.59142334,326.13634644)
\curveto(828.81141805,326.18634166)(828.98641787,326.30134155)(829.11642334,326.48134644)
\curveto(829.17641768,326.56134129)(829.22641763,326.66134119)(829.26642334,326.78134644)
\curveto(829.30641755,326.91134094)(829.30641755,327.0513408)(829.26642334,327.20134644)
\curveto(829.20641765,327.44134041)(829.11641774,327.63134022)(828.99642334,327.77134644)
\curveto(828.88641797,327.91133994)(828.72641813,328.02133983)(828.51642334,328.10134644)
\curveto(828.39641846,328.1513397)(828.25141861,328.18633966)(828.08142334,328.20634644)
\curveto(827.92141894,328.22633962)(827.75141911,328.23633961)(827.57142334,328.23634644)
\curveto(827.39141947,328.23633961)(827.21641964,328.22633962)(827.04642334,328.20634644)
\curveto(826.87641998,328.18633966)(826.73142013,328.15633969)(826.61142334,328.11634644)
\curveto(826.44142042,328.05633979)(826.27642058,327.97133988)(826.11642334,327.86134644)
\curveto(826.03642082,327.80134005)(825.9614209,327.72134013)(825.89142334,327.62134644)
\curveto(825.83142103,327.53134032)(825.77642108,327.43134042)(825.72642334,327.32134644)
\curveto(825.69642116,327.24134061)(825.66642119,327.15634069)(825.63642334,327.06634644)
\curveto(825.61642124,326.97634087)(825.57142129,326.90634094)(825.50142334,326.85634644)
\curveto(825.4614214,326.82634102)(825.39142147,326.80134105)(825.29142334,326.78134644)
\curveto(825.20142166,326.77134108)(825.10642175,326.76634108)(825.00642334,326.76634644)
\curveto(824.90642195,326.76634108)(824.80642205,326.77134108)(824.70642334,326.78134644)
\curveto(824.61642224,326.80134105)(824.55142231,326.82634102)(824.51142334,326.85634644)
\curveto(824.47142239,326.88634096)(824.44142242,326.93634091)(824.42142334,327.00634644)
\curveto(824.40142246,327.07634077)(824.40142246,327.1513407)(824.42142334,327.23134644)
\curveto(824.45142241,327.36134049)(824.48142238,327.48134037)(824.51142334,327.59134644)
\curveto(824.55142231,327.71134014)(824.59642226,327.82634002)(824.64642334,327.93634644)
\curveto(824.83642202,328.28633956)(825.07642178,328.55633929)(825.36642334,328.74634644)
\curveto(825.6564212,328.9463389)(826.01642084,329.10633874)(826.44642334,329.22634644)
\curveto(826.54642031,329.2463386)(826.64642021,329.26133859)(826.74642334,329.27134644)
\curveto(826.85642,329.28133857)(826.96641989,329.29633855)(827.07642334,329.31634644)
\curveto(827.11641974,329.32633852)(827.18141968,329.32633852)(827.27142334,329.31634644)
\curveto(827.3614195,329.31633853)(827.41641944,329.32633852)(827.43642334,329.34634644)
\curveto(828.13641872,329.35633849)(828.74641811,329.27633857)(829.26642334,329.10634644)
\curveto(829.78641707,328.93633891)(830.15141671,328.61133924)(830.36142334,328.13134644)
\curveto(830.45141641,327.93133992)(830.50141636,327.69634015)(830.51142334,327.42634644)
\curveto(830.53141633,327.16634068)(830.54141632,326.89134096)(830.54142334,326.60134644)
\lineto(830.54142334,323.28634644)
\curveto(830.54141632,323.1463447)(830.54641631,323.01134484)(830.55642334,322.88134644)
\curveto(830.56641629,322.7513451)(830.59641626,322.6463452)(830.64642334,322.56634644)
\curveto(830.69641616,322.49634535)(830.7614161,322.4463454)(830.84142334,322.41634644)
\curveto(830.93141593,322.37634547)(831.01641584,322.3463455)(831.09642334,322.32634644)
\curveto(831.17641568,322.31634553)(831.23641562,322.27134558)(831.27642334,322.19134644)
\curveto(831.29641556,322.16134569)(831.30641555,322.13134572)(831.30642334,322.10134644)
\curveto(831.30641555,322.07134578)(831.31141555,322.03134582)(831.32142334,321.98134644)
\moveto(829.17642334,323.64634644)
\curveto(829.23641762,323.78634406)(829.26641759,323.9463439)(829.26642334,324.12634644)
\curveto(829.27641758,324.31634353)(829.28141758,324.51134334)(829.28142334,324.71134644)
\curveto(829.28141758,324.82134303)(829.27641758,324.92134293)(829.26642334,325.01134644)
\curveto(829.2564176,325.10134275)(829.21641764,325.17134268)(829.14642334,325.22134644)
\curveto(829.11641774,325.24134261)(829.04641781,325.2513426)(828.93642334,325.25134644)
\curveto(828.91641794,325.23134262)(828.88141798,325.22134263)(828.83142334,325.22134644)
\curveto(828.78141808,325.22134263)(828.73641812,325.21134264)(828.69642334,325.19134644)
\curveto(828.61641824,325.17134268)(828.52641833,325.1513427)(828.42642334,325.13134644)
\lineto(828.12642334,325.07134644)
\curveto(828.09641876,325.07134278)(828.0614188,325.06634278)(828.02142334,325.05634644)
\lineto(827.91642334,325.05634644)
\curveto(827.76641909,325.01634283)(827.60141926,324.99134286)(827.42142334,324.98134644)
\curveto(827.25141961,324.98134287)(827.09141977,324.96134289)(826.94142334,324.92134644)
\curveto(826.86142,324.90134295)(826.78642007,324.88134297)(826.71642334,324.86134644)
\curveto(826.6564202,324.851343)(826.58642027,324.83634301)(826.50642334,324.81634644)
\curveto(826.34642051,324.76634308)(826.19642066,324.70134315)(826.05642334,324.62134644)
\curveto(825.91642094,324.5513433)(825.79642106,324.46134339)(825.69642334,324.35134644)
\curveto(825.59642126,324.24134361)(825.52142134,324.10634374)(825.47142334,323.94634644)
\curveto(825.42142144,323.79634405)(825.40142146,323.61134424)(825.41142334,323.39134644)
\curveto(825.41142145,323.29134456)(825.42642143,323.19634465)(825.45642334,323.10634644)
\curveto(825.49642136,323.02634482)(825.54142132,322.9513449)(825.59142334,322.88134644)
\curveto(825.67142119,322.77134508)(825.77642108,322.67634517)(825.90642334,322.59634644)
\curveto(826.03642082,322.52634532)(826.17642068,322.46634538)(826.32642334,322.41634644)
\curveto(826.37642048,322.40634544)(826.42642043,322.40134545)(826.47642334,322.40134644)
\curveto(826.52642033,322.40134545)(826.57642028,322.39634545)(826.62642334,322.38634644)
\curveto(826.69642016,322.36634548)(826.78142008,322.3513455)(826.88142334,322.34134644)
\curveto(826.99141987,322.34134551)(827.08141978,322.3513455)(827.15142334,322.37134644)
\curveto(827.21141965,322.39134546)(827.27141959,322.39634545)(827.33142334,322.38634644)
\curveto(827.39141947,322.38634546)(827.45141941,322.39634545)(827.51142334,322.41634644)
\curveto(827.59141927,322.43634541)(827.66641919,322.4513454)(827.73642334,322.46134644)
\curveto(827.81641904,322.47134538)(827.89141897,322.49134536)(827.96142334,322.52134644)
\curveto(828.25141861,322.64134521)(828.49641836,322.78634506)(828.69642334,322.95634644)
\curveto(828.90641795,323.12634472)(829.06641779,323.35634449)(829.17642334,323.64634644)
}
}
{
\newrgbcolor{curcolor}{0 0 0}
\pscustom[linestyle=none,fillstyle=solid,fillcolor=curcolor]
{
\newpath
\moveto(839.45306396,322.23634644)
\lineto(839.45306396,321.84634644)
\curveto(839.45305609,321.72634612)(839.42805611,321.62634622)(839.37806396,321.54634644)
\curveto(839.32805621,321.47634637)(839.2430563,321.43634641)(839.12306396,321.42634644)
\lineto(838.77806396,321.42634644)
\curveto(838.71805682,321.42634642)(838.65805688,321.42134643)(838.59806396,321.41134644)
\curveto(838.54805699,321.41134644)(838.50305704,321.42134643)(838.46306396,321.44134644)
\curveto(838.37305717,321.46134639)(838.31305723,321.50134635)(838.28306396,321.56134644)
\curveto(838.2430573,321.61134624)(838.21805732,321.67134618)(838.20806396,321.74134644)
\curveto(838.20805733,321.81134604)(838.19305735,321.88134597)(838.16306396,321.95134644)
\curveto(838.15305739,321.97134588)(838.1380574,321.98634586)(838.11806396,321.99634644)
\curveto(838.10805743,322.01634583)(838.09305745,322.03634581)(838.07306396,322.05634644)
\curveto(837.97305757,322.06634578)(837.89305765,322.0463458)(837.83306396,321.99634644)
\curveto(837.78305776,321.9463459)(837.72805781,321.89634595)(837.66806396,321.84634644)
\curveto(837.46805807,321.69634615)(837.26805827,321.58134627)(837.06806396,321.50134644)
\curveto(836.88805865,321.42134643)(836.67805886,321.36134649)(836.43806396,321.32134644)
\curveto(836.20805933,321.28134657)(835.96805957,321.26134659)(835.71806396,321.26134644)
\curveto(835.47806006,321.2513466)(835.2380603,321.26634658)(834.99806396,321.30634644)
\curveto(834.75806078,321.33634651)(834.54806099,321.39134646)(834.36806396,321.47134644)
\curveto(833.84806169,321.69134616)(833.42806211,321.98634586)(833.10806396,322.35634644)
\curveto(832.78806275,322.73634511)(832.538063,323.20634464)(832.35806396,323.76634644)
\curveto(832.31806322,323.85634399)(832.28806325,323.9463439)(832.26806396,324.03634644)
\curveto(832.25806328,324.13634371)(832.2380633,324.23634361)(832.20806396,324.33634644)
\curveto(832.19806334,324.38634346)(832.19306335,324.43634341)(832.19306396,324.48634644)
\curveto(832.19306335,324.53634331)(832.18806335,324.58634326)(832.17806396,324.63634644)
\curveto(832.15806338,324.68634316)(832.14806339,324.73634311)(832.14806396,324.78634644)
\curveto(832.15806338,324.846343)(832.15806338,324.90134295)(832.14806396,324.95134644)
\lineto(832.14806396,325.10134644)
\curveto(832.12806341,325.1513427)(832.11806342,325.21634263)(832.11806396,325.29634644)
\curveto(832.11806342,325.37634247)(832.12806341,325.44134241)(832.14806396,325.49134644)
\lineto(832.14806396,325.65634644)
\curveto(832.16806337,325.72634212)(832.17306337,325.79634205)(832.16306396,325.86634644)
\curveto(832.16306338,325.9463419)(832.17306337,326.02134183)(832.19306396,326.09134644)
\curveto(832.20306334,326.14134171)(832.20806333,326.18634166)(832.20806396,326.22634644)
\curveto(832.20806333,326.26634158)(832.21306333,326.31134154)(832.22306396,326.36134644)
\curveto(832.25306329,326.46134139)(832.27806326,326.55634129)(832.29806396,326.64634644)
\curveto(832.31806322,326.7463411)(832.3430632,326.84134101)(832.37306396,326.93134644)
\curveto(832.50306304,327.31134054)(832.66806287,327.6513402)(832.86806396,327.95134644)
\curveto(833.07806246,328.26133959)(833.32806221,328.51633933)(833.61806396,328.71634644)
\curveto(833.78806175,328.83633901)(833.96306158,328.93633891)(834.14306396,329.01634644)
\curveto(834.33306121,329.09633875)(834.538061,329.16633868)(834.75806396,329.22634644)
\curveto(834.82806071,329.23633861)(834.89306065,329.2463386)(834.95306396,329.25634644)
\curveto(835.02306052,329.26633858)(835.09306045,329.28133857)(835.16306396,329.30134644)
\lineto(835.31306396,329.30134644)
\curveto(835.39306015,329.32133853)(835.50806003,329.33133852)(835.65806396,329.33134644)
\curveto(835.81805972,329.33133852)(835.9380596,329.32133853)(836.01806396,329.30134644)
\curveto(836.05805948,329.29133856)(836.11305943,329.28633856)(836.18306396,329.28634644)
\curveto(836.29305925,329.25633859)(836.40305914,329.23133862)(836.51306396,329.21134644)
\curveto(836.62305892,329.20133865)(836.72805881,329.17133868)(836.82806396,329.12134644)
\curveto(836.97805856,329.06133879)(837.11805842,328.99633885)(837.24806396,328.92634644)
\curveto(837.38805815,328.85633899)(837.51805802,328.77633907)(837.63806396,328.68634644)
\curveto(837.69805784,328.63633921)(837.75805778,328.58133927)(837.81806396,328.52134644)
\curveto(837.88805765,328.47133938)(837.97805756,328.45633939)(838.08806396,328.47634644)
\curveto(838.10805743,328.50633934)(838.12305742,328.53133932)(838.13306396,328.55134644)
\curveto(838.15305739,328.57133928)(838.16805737,328.60133925)(838.17806396,328.64134644)
\curveto(838.20805733,328.73133912)(838.21805732,328.846339)(838.20806396,328.98634644)
\lineto(838.20806396,329.36134644)
\lineto(838.20806396,331.08634644)
\lineto(838.20806396,331.55134644)
\curveto(838.20805733,331.73133612)(838.23305731,331.86133599)(838.28306396,331.94134644)
\curveto(838.32305722,332.01133584)(838.38305716,332.05633579)(838.46306396,332.07634644)
\curveto(838.48305706,332.07633577)(838.50805703,332.07633577)(838.53806396,332.07634644)
\curveto(838.56805697,332.08633576)(838.59305695,332.09133576)(838.61306396,332.09134644)
\curveto(838.75305679,332.10133575)(838.89805664,332.10133575)(839.04806396,332.09134644)
\curveto(839.20805633,332.09133576)(839.31805622,332.0513358)(839.37806396,331.97134644)
\curveto(839.42805611,331.89133596)(839.45305609,331.79133606)(839.45306396,331.67134644)
\lineto(839.45306396,331.29634644)
\lineto(839.45306396,322.23634644)
\moveto(838.23806396,325.07134644)
\curveto(838.25805728,325.12134273)(838.26805727,325.18634266)(838.26806396,325.26634644)
\curveto(838.26805727,325.35634249)(838.25805728,325.42634242)(838.23806396,325.47634644)
\lineto(838.23806396,325.70134644)
\curveto(838.21805732,325.79134206)(838.20305734,325.88134197)(838.19306396,325.97134644)
\curveto(838.18305736,326.07134178)(838.16305738,326.16134169)(838.13306396,326.24134644)
\curveto(838.11305743,326.32134153)(838.09305745,326.39634145)(838.07306396,326.46634644)
\curveto(838.06305748,326.53634131)(838.0430575,326.60634124)(838.01306396,326.67634644)
\curveto(837.89305765,326.97634087)(837.7380578,327.24134061)(837.54806396,327.47134644)
\curveto(837.35805818,327.70134015)(837.11805842,327.88133997)(836.82806396,328.01134644)
\curveto(836.72805881,328.06133979)(836.62305892,328.09633975)(836.51306396,328.11634644)
\curveto(836.41305913,328.1463397)(836.30305924,328.17133968)(836.18306396,328.19134644)
\curveto(836.10305944,328.21133964)(836.01305953,328.22133963)(835.91306396,328.22134644)
\lineto(835.64306396,328.22134644)
\curveto(835.59305995,328.21133964)(835.54805999,328.20133965)(835.50806396,328.19134644)
\lineto(835.37306396,328.19134644)
\curveto(835.29306025,328.17133968)(835.20806033,328.1513397)(835.11806396,328.13134644)
\curveto(835.0380605,328.11133974)(834.95806058,328.08633976)(834.87806396,328.05634644)
\curveto(834.55806098,327.91633993)(834.29806124,327.71134014)(834.09806396,327.44134644)
\curveto(833.90806163,327.18134067)(833.75306179,326.87634097)(833.63306396,326.52634644)
\curveto(833.59306195,326.41634143)(833.56306198,326.30134155)(833.54306396,326.18134644)
\curveto(833.53306201,326.07134178)(833.51806202,325.96134189)(833.49806396,325.85134644)
\curveto(833.49806204,325.81134204)(833.49306205,325.77134208)(833.48306396,325.73134644)
\lineto(833.48306396,325.62634644)
\curveto(833.46306208,325.57634227)(833.45306209,325.52134233)(833.45306396,325.46134644)
\curveto(833.46306208,325.40134245)(833.46806207,325.3463425)(833.46806396,325.29634644)
\lineto(833.46806396,324.96634644)
\curveto(833.46806207,324.86634298)(833.47806206,324.77134308)(833.49806396,324.68134644)
\curveto(833.50806203,324.6513432)(833.51306203,324.60134325)(833.51306396,324.53134644)
\curveto(833.53306201,324.46134339)(833.54806199,324.39134346)(833.55806396,324.32134644)
\lineto(833.61806396,324.11134644)
\curveto(833.72806181,323.76134409)(833.87806166,323.46134439)(834.06806396,323.21134644)
\curveto(834.25806128,322.96134489)(834.49806104,322.75634509)(834.78806396,322.59634644)
\curveto(834.87806066,322.5463453)(834.96806057,322.50634534)(835.05806396,322.47634644)
\curveto(835.14806039,322.4463454)(835.24806029,322.41634543)(835.35806396,322.38634644)
\curveto(835.40806013,322.36634548)(835.45806008,322.36134549)(835.50806396,322.37134644)
\curveto(835.56805997,322.38134547)(835.62305992,322.37634547)(835.67306396,322.35634644)
\curveto(835.71305983,322.3463455)(835.75305979,322.34134551)(835.79306396,322.34134644)
\lineto(835.92806396,322.34134644)
\lineto(836.06306396,322.34134644)
\curveto(836.09305945,322.3513455)(836.1430594,322.35634549)(836.21306396,322.35634644)
\curveto(836.29305925,322.37634547)(836.37305917,322.39134546)(836.45306396,322.40134644)
\curveto(836.53305901,322.42134543)(836.60805893,322.4463454)(836.67806396,322.47634644)
\curveto(837.00805853,322.61634523)(837.27305827,322.79134506)(837.47306396,323.00134644)
\curveto(837.68305786,323.22134463)(837.85805768,323.49634435)(837.99806396,323.82634644)
\curveto(838.04805749,323.93634391)(838.08305746,324.0463438)(838.10306396,324.15634644)
\curveto(838.12305742,324.26634358)(838.14805739,324.37634347)(838.17806396,324.48634644)
\curveto(838.19805734,324.52634332)(838.20805733,324.56134329)(838.20806396,324.59134644)
\curveto(838.20805733,324.63134322)(838.21305733,324.67134318)(838.22306396,324.71134644)
\curveto(838.23305731,324.77134308)(838.23305731,324.83134302)(838.22306396,324.89134644)
\curveto(838.22305732,324.9513429)(838.22805731,325.01134284)(838.23806396,325.07134644)
}
}
{
\newrgbcolor{curcolor}{0 0 0}
\pscustom[linestyle=none,fillstyle=solid,fillcolor=curcolor]
{
\newpath
\moveto(766.47645996,306.9330896)
\curveto(766.49645084,306.85308182)(766.50645083,306.74308193)(766.50645996,306.6030896)
\curveto(766.50645083,306.4730822)(766.49645084,306.3730823)(766.47645996,306.3030896)
\curveto(766.45645088,306.23308244)(766.45145088,306.1680825)(766.46145996,306.1080896)
\curveto(766.47145086,306.04808262)(766.46645087,305.98308269)(766.44645996,305.9130896)
\curveto(766.42645091,305.85308282)(766.41145092,305.78808288)(766.40145996,305.7180896)
\curveto(766.39145094,305.65808301)(766.37645096,305.59808307)(766.35645996,305.5380896)
\curveto(766.336451,305.45808321)(766.31145102,305.38308329)(766.28145996,305.3130896)
\curveto(766.26145107,305.24308343)(766.2364511,305.1730835)(766.20645996,305.1030896)
\curveto(766.18645115,305.0730836)(766.17145116,305.04308363)(766.16145996,305.0130896)
\curveto(766.16145117,304.99308368)(766.15145118,304.9730837)(766.13145996,304.9530896)
\curveto(766.02145131,304.75308392)(765.90145143,304.5730841)(765.77145996,304.4130896)
\curveto(765.75145158,304.3730843)(765.71645162,304.33308434)(765.66645996,304.2930896)
\curveto(765.62645171,304.25308442)(765.59145174,304.22308445)(765.56145996,304.2030896)
\curveto(765.52145181,304.18308449)(765.48645185,304.15308452)(765.45645996,304.1130896)
\curveto(765.42645191,304.08308459)(765.39645194,304.05808461)(765.36645996,304.0380896)
\lineto(765.05145996,303.8580896)
\curveto(764.94145239,303.77808489)(764.81145252,303.71808495)(764.66145996,303.6780896)
\lineto(764.21145996,303.5580896)
\curveto(764.1314532,303.53808513)(764.05145328,303.52308515)(763.97145996,303.5130896)
\curveto(763.89145344,303.51308516)(763.81145352,303.50308517)(763.73145996,303.4830896)
\curveto(763.69145364,303.4730852)(763.65145368,303.4680852)(763.61145996,303.4680896)
\curveto(763.58145375,303.47808519)(763.55145378,303.47808519)(763.52145996,303.4680896)
\curveto(763.47145386,303.45808521)(763.42145391,303.45808521)(763.37145996,303.4680896)
\curveto(763.331454,303.47808519)(763.28645405,303.47808519)(763.23645996,303.4680896)
\lineto(760.97145996,303.4680896)
\lineto(760.47645996,303.4680896)
\curveto(760.30645703,303.47808519)(760.17645716,303.44808522)(760.08645996,303.3780896)
\curveto(759.97645736,303.29808537)(759.92145741,303.15308552)(759.92145996,302.9430896)
\curveto(759.9314574,302.73308594)(759.9364574,302.53808613)(759.93645996,302.3580896)
\lineto(759.93645996,300.1530896)
\lineto(759.93645996,299.6580896)
\curveto(759.94645739,299.4680892)(759.92645741,299.33308934)(759.87645996,299.2530896)
\curveto(759.8364575,299.19308948)(759.78645755,299.15308952)(759.72645996,299.1330896)
\curveto(759.67645766,299.12308955)(759.61145772,299.10808956)(759.53145996,299.0880896)
\lineto(759.26145996,299.0880896)
\curveto(759.11145822,299.08808958)(758.97645836,299.09308958)(758.85645996,299.1030896)
\curveto(758.7364586,299.11308956)(758.65145868,299.16308951)(758.60145996,299.2530896)
\curveto(758.56145877,299.31308936)(758.54145879,299.39308928)(758.54145996,299.4930896)
\lineto(758.54145996,299.8080896)
\lineto(758.54145996,308.9130896)
\curveto(758.54145879,309.02307965)(758.5364588,309.14307953)(758.52645996,309.2730896)
\curveto(758.52645881,309.41307926)(758.55145878,309.52307915)(758.60145996,309.6030896)
\curveto(758.64145869,309.66307901)(758.71645862,309.71307896)(758.82645996,309.7530896)
\curveto(758.84645849,309.76307891)(758.86645847,309.76307891)(758.88645996,309.7530896)
\curveto(758.90645843,309.75307892)(758.92645841,309.75807891)(758.94645996,309.7680896)
\lineto(762.35145996,309.7680896)
\curveto(762.7314546,309.7680789)(763.10145423,309.76307891)(763.46145996,309.7530896)
\curveto(763.8314535,309.75307892)(764.16145317,309.70807896)(764.45145996,309.6180896)
\curveto(764.90145243,309.4680792)(765.26645207,309.2730794)(765.54645996,309.0330896)
\curveto(765.82645151,308.79307988)(766.05645128,308.46308021)(766.23645996,308.0430896)
\curveto(766.28645105,307.93308074)(766.32145101,307.81808085)(766.34145996,307.6980896)
\curveto(766.37145096,307.57808109)(766.40645093,307.45308122)(766.44645996,307.3230896)
\curveto(766.46645087,307.25308142)(766.47145086,307.18808148)(766.46145996,307.1280896)
\curveto(766.45145088,307.0680816)(766.45645088,307.00308167)(766.47645996,306.9330896)
\moveto(765.06645996,306.3930896)
\curveto(765.10645223,306.53308214)(765.11145222,306.69308198)(765.08145996,306.8730896)
\curveto(765.05145228,307.06308161)(765.02145231,307.21308146)(764.99145996,307.3230896)
\curveto(764.89145244,307.60308107)(764.75645258,307.82308085)(764.58645996,307.9830896)
\curveto(764.42645291,308.15308052)(764.21645312,308.29308038)(763.95645996,308.4030896)
\curveto(763.7364536,308.49308018)(763.48145385,308.54808012)(763.19145996,308.5680896)
\curveto(762.91145442,308.58808008)(762.61645472,308.59808007)(762.30645996,308.5980896)
\lineto(760.37145996,308.5980896)
\curveto(760.35145698,308.58808008)(760.32645701,308.58308009)(760.29645996,308.5830896)
\curveto(760.27645706,308.58308009)(760.25145708,308.57808009)(760.22145996,308.5680896)
\curveto(760.10145723,308.53808013)(760.02145731,308.4730802)(759.98145996,308.3730896)
\curveto(759.94145739,308.2730804)(759.92145741,308.13808053)(759.92145996,307.9680896)
\curveto(759.9314574,307.80808086)(759.9364574,307.65808101)(759.93645996,307.5180896)
\lineto(759.93645996,305.7180896)
\curveto(759.9364574,305.5680831)(759.9314574,305.40308327)(759.92145996,305.2230896)
\curveto(759.92145741,305.04308363)(759.95145738,304.90308377)(760.01145996,304.8030896)
\curveto(760.06145727,304.72308395)(760.1364572,304.673084)(760.23645996,304.6530896)
\curveto(760.34645699,304.64308403)(760.46645687,304.63808403)(760.59645996,304.6380896)
\lineto(762.62145996,304.6380896)
\lineto(763.08645996,304.6380896)
\curveto(763.24645409,304.64808402)(763.38645395,304.668084)(763.50645996,304.6980896)
\curveto(763.77645356,304.7680839)(764.01145332,304.84808382)(764.21145996,304.9380896)
\curveto(764.42145291,305.03808363)(764.59645274,305.18808348)(764.73645996,305.3880896)
\curveto(764.81645252,305.50808316)(764.87645246,305.63308304)(764.91645996,305.7630896)
\curveto(764.96645237,305.89308278)(765.01145232,306.03808263)(765.05145996,306.1980896)
\curveto(765.06145227,306.23808243)(765.06645227,306.30308237)(765.06645996,306.3930896)
}
}
{
\newrgbcolor{curcolor}{0 0 0}
\pscustom[linestyle=none,fillstyle=solid,fillcolor=curcolor]
{
\newpath
\moveto(775.13802246,303.2880896)
\curveto(775.1580144,303.22808544)(775.16801439,303.13308554)(775.16802246,303.0030896)
\curveto(775.16801439,302.88308579)(775.1630144,302.79808587)(775.15302246,302.7480896)
\lineto(775.15302246,302.5980896)
\curveto(775.14301442,302.51808615)(775.13301443,302.44308623)(775.12302246,302.3730896)
\curveto(775.12301444,302.31308636)(775.11801444,302.24308643)(775.10802246,302.1630896)
\curveto(775.08801447,302.10308657)(775.07301449,302.04308663)(775.06302246,301.9830896)
\curveto(775.0630145,301.92308675)(775.05301451,301.86308681)(775.03302246,301.8030896)
\curveto(774.99301457,301.673087)(774.9580146,301.54308713)(774.92802246,301.4130896)
\curveto(774.89801466,301.28308739)(774.8580147,301.16308751)(774.80802246,301.0530896)
\curveto(774.59801496,300.5730881)(774.31801524,300.1680885)(773.96802246,299.8380896)
\curveto(773.61801594,299.51808915)(773.18801637,299.2730894)(772.67802246,299.1030896)
\curveto(772.56801699,299.06308961)(772.44801711,299.03308964)(772.31802246,299.0130896)
\curveto(772.19801736,298.99308968)(772.07301749,298.9730897)(771.94302246,298.9530896)
\curveto(771.88301768,298.94308973)(771.81801774,298.93808973)(771.74802246,298.9380896)
\curveto(771.68801787,298.92808974)(771.62801793,298.92308975)(771.56802246,298.9230896)
\curveto(771.52801803,298.91308976)(771.46801809,298.90808976)(771.38802246,298.9080896)
\curveto(771.31801824,298.90808976)(771.26801829,298.91308976)(771.23802246,298.9230896)
\curveto(771.19801836,298.93308974)(771.1580184,298.93808973)(771.11802246,298.9380896)
\curveto(771.07801848,298.92808974)(771.04301852,298.92808974)(771.01302246,298.9380896)
\lineto(770.92302246,298.9380896)
\lineto(770.56302246,298.9830896)
\curveto(770.42301914,299.02308965)(770.28801927,299.06308961)(770.15802246,299.1030896)
\curveto(770.02801953,299.14308953)(769.90301966,299.18808948)(769.78302246,299.2380896)
\curveto(769.33302023,299.43808923)(768.9630206,299.69808897)(768.67302246,300.0180896)
\curveto(768.38302118,300.33808833)(768.14302142,300.72808794)(767.95302246,301.1880896)
\curveto(767.90302166,301.28808738)(767.8630217,301.38808728)(767.83302246,301.4880896)
\curveto(767.81302175,301.58808708)(767.79302177,301.69308698)(767.77302246,301.8030896)
\curveto(767.75302181,301.84308683)(767.74302182,301.8730868)(767.74302246,301.8930896)
\curveto(767.75302181,301.92308675)(767.75302181,301.95808671)(767.74302246,301.9980896)
\curveto(767.72302184,302.07808659)(767.70802185,302.15808651)(767.69802246,302.2380896)
\curveto(767.69802186,302.32808634)(767.68802187,302.41308626)(767.66802246,302.4930896)
\lineto(767.66802246,302.6130896)
\curveto(767.66802189,302.65308602)(767.6630219,302.69808597)(767.65302246,302.7480896)
\curveto(767.64302192,302.79808587)(767.63802192,302.88308579)(767.63802246,303.0030896)
\curveto(767.63802192,303.13308554)(767.64802191,303.22808544)(767.66802246,303.2880896)
\curveto(767.68802187,303.35808531)(767.69302187,303.42808524)(767.68302246,303.4980896)
\curveto(767.67302189,303.5680851)(767.67802188,303.63808503)(767.69802246,303.7080896)
\curveto(767.70802185,303.75808491)(767.71302185,303.79808487)(767.71302246,303.8280896)
\curveto(767.72302184,303.8680848)(767.73302183,303.91308476)(767.74302246,303.9630896)
\curveto(767.77302179,304.08308459)(767.79802176,304.20308447)(767.81802246,304.3230896)
\curveto(767.84802171,304.44308423)(767.88802167,304.55808411)(767.93802246,304.6680896)
\curveto(768.08802147,305.03808363)(768.26802129,305.3680833)(768.47802246,305.6580896)
\curveto(768.69802086,305.95808271)(768.9630206,306.20808246)(769.27302246,306.4080896)
\curveto(769.39302017,306.48808218)(769.51802004,306.55308212)(769.64802246,306.6030896)
\curveto(769.77801978,306.66308201)(769.91301965,306.72308195)(770.05302246,306.7830896)
\curveto(770.17301939,306.83308184)(770.30301926,306.86308181)(770.44302246,306.8730896)
\curveto(770.58301898,306.89308178)(770.72301884,306.92308175)(770.86302246,306.9630896)
\lineto(771.05802246,306.9630896)
\curveto(771.12801843,306.9730817)(771.19301837,306.98308169)(771.25302246,306.9930896)
\curveto(772.14301742,307.00308167)(772.88301668,306.81808185)(773.47302246,306.4380896)
\curveto(774.0630155,306.05808261)(774.48801507,305.56308311)(774.74802246,304.9530896)
\curveto(774.79801476,304.85308382)(774.83801472,304.75308392)(774.86802246,304.6530896)
\curveto(774.89801466,304.55308412)(774.93301463,304.44808422)(774.97302246,304.3380896)
\curveto(775.00301456,304.22808444)(775.02801453,304.10808456)(775.04802246,303.9780896)
\curveto(775.06801449,303.85808481)(775.09301447,303.73308494)(775.12302246,303.6030896)
\curveto(775.13301443,303.55308512)(775.13301443,303.49808517)(775.12302246,303.4380896)
\curveto(775.12301444,303.38808528)(775.12801443,303.33808533)(775.13802246,303.2880896)
\moveto(773.80302246,302.4330896)
\curveto(773.82301574,302.50308617)(773.82801573,302.58308609)(773.81802246,302.6730896)
\lineto(773.81802246,302.9280896)
\curveto(773.81801574,303.31808535)(773.78301578,303.64808502)(773.71302246,303.9180896)
\curveto(773.68301588,303.99808467)(773.6580159,304.07808459)(773.63802246,304.1580896)
\curveto(773.61801594,304.23808443)(773.59301597,304.31308436)(773.56302246,304.3830896)
\curveto(773.28301628,305.03308364)(772.83801672,305.48308319)(772.22802246,305.7330896)
\curveto(772.1580174,305.76308291)(772.08301748,305.78308289)(772.00302246,305.7930896)
\lineto(771.76302246,305.8530896)
\curveto(771.68301788,305.8730828)(771.59801796,305.88308279)(771.50802246,305.8830896)
\lineto(771.23802246,305.8830896)
\lineto(770.96802246,305.8380896)
\curveto(770.86801869,305.81808285)(770.77301879,305.79308288)(770.68302246,305.7630896)
\curveto(770.60301896,305.74308293)(770.52301904,305.71308296)(770.44302246,305.6730896)
\curveto(770.37301919,305.65308302)(770.30801925,305.62308305)(770.24802246,305.5830896)
\curveto(770.18801937,305.54308313)(770.13301943,305.50308317)(770.08302246,305.4630896)
\curveto(769.84301972,305.29308338)(769.64801991,305.08808358)(769.49802246,304.8480896)
\curveto(769.34802021,304.60808406)(769.21802034,304.32808434)(769.10802246,304.0080896)
\curveto(769.07802048,303.90808476)(769.0580205,303.80308487)(769.04802246,303.6930896)
\curveto(769.03802052,303.59308508)(769.02302054,303.48808518)(769.00302246,303.3780896)
\curveto(768.99302057,303.33808533)(768.98802057,303.2730854)(768.98802246,303.1830896)
\curveto(768.97802058,303.15308552)(768.97302059,303.11808555)(768.97302246,303.0780896)
\curveto(768.98302058,303.03808563)(768.98802057,302.99308568)(768.98802246,302.9430896)
\lineto(768.98802246,302.6430896)
\curveto(768.98802057,302.54308613)(768.99802056,302.45308622)(769.01802246,302.3730896)
\lineto(769.04802246,302.1930896)
\curveto(769.06802049,302.09308658)(769.08302048,301.99308668)(769.09302246,301.8930896)
\curveto(769.11302045,301.80308687)(769.14302042,301.71808695)(769.18302246,301.6380896)
\curveto(769.28302028,301.39808727)(769.39802016,301.1730875)(769.52802246,300.9630896)
\curveto(769.66801989,300.75308792)(769.83801972,300.57808809)(770.03802246,300.4380896)
\curveto(770.08801947,300.40808826)(770.13301943,300.38308829)(770.17302246,300.3630896)
\curveto(770.21301935,300.34308833)(770.2580193,300.31808835)(770.30802246,300.2880896)
\curveto(770.38801917,300.23808843)(770.47301909,300.19308848)(770.56302246,300.1530896)
\curveto(770.6630189,300.12308855)(770.76801879,300.09308858)(770.87802246,300.0630896)
\curveto(770.92801863,300.04308863)(770.97301859,300.03308864)(771.01302246,300.0330896)
\curveto(771.0630185,300.04308863)(771.11301845,300.04308863)(771.16302246,300.0330896)
\curveto(771.19301837,300.02308865)(771.25301831,300.01308866)(771.34302246,300.0030896)
\curveto(771.44301812,299.99308868)(771.51801804,299.99808867)(771.56802246,300.0180896)
\curveto(771.60801795,300.02808864)(771.64801791,300.02808864)(771.68802246,300.0180896)
\curveto(771.72801783,300.01808865)(771.76801779,300.02808864)(771.80802246,300.0480896)
\curveto(771.88801767,300.0680886)(771.96801759,300.08308859)(772.04802246,300.0930896)
\curveto(772.12801743,300.11308856)(772.20301736,300.13808853)(772.27302246,300.1680896)
\curveto(772.61301695,300.30808836)(772.88801667,300.50308817)(773.09802246,300.7530896)
\curveto(773.30801625,301.00308767)(773.48301608,301.29808737)(773.62302246,301.6380896)
\curveto(773.67301589,301.75808691)(773.70301586,301.88308679)(773.71302246,302.0130896)
\curveto(773.73301583,302.15308652)(773.7630158,302.29308638)(773.80302246,302.4330896)
}
}
{
\newrgbcolor{curcolor}{0 0 0}
\pscustom[linestyle=none,fillstyle=solid,fillcolor=curcolor]
{
\newpath
\moveto(784.05130371,303.1530896)
\curveto(784.06129536,303.10308557)(784.06629536,303.03808563)(784.06630371,302.9580896)
\curveto(784.06629536,302.87808579)(784.06129536,302.81308586)(784.05130371,302.7630896)
\curveto(784.03129539,302.71308596)(784.0262954,302.66308601)(784.03630371,302.6130896)
\curveto(784.04629538,302.5730861)(784.04629538,302.53308614)(784.03630371,302.4930896)
\curveto(784.03629539,302.42308625)(784.03129539,302.3680863)(784.02130371,302.3280896)
\curveto(784.00129542,302.23808643)(783.98629544,302.14808652)(783.97630371,302.0580896)
\curveto(783.97629545,301.9680867)(783.96629546,301.87808679)(783.94630371,301.7880896)
\lineto(783.88630371,301.5480896)
\curveto(783.86629556,301.47808719)(783.84129558,301.40308727)(783.81130371,301.3230896)
\curveto(783.69129573,300.95308772)(783.5262959,300.61808805)(783.31630371,300.3180896)
\curveto(783.25629617,300.22808844)(783.19129623,300.13808853)(783.12130371,300.0480896)
\curveto(783.05129637,299.9680887)(782.97629645,299.89308878)(782.89630371,299.8230896)
\lineto(782.82130371,299.7480896)
\curveto(782.75129667,299.69808897)(782.68629674,299.64808902)(782.62630371,299.5980896)
\curveto(782.56629686,299.54808912)(782.49629693,299.49808917)(782.41630371,299.4480896)
\curveto(782.30629712,299.3680893)(782.18129724,299.29808937)(782.04130371,299.2380896)
\curveto(781.91129751,299.18808948)(781.77629765,299.13808953)(781.63630371,299.0880896)
\curveto(781.55629787,299.0680896)(781.47629795,299.05308962)(781.39630371,299.0430896)
\curveto(781.3262981,299.03308964)(781.25129817,299.01808965)(781.17130371,298.9980896)
\lineto(781.11130371,298.9980896)
\curveto(781.10129832,298.98808968)(781.08629834,298.98308969)(781.06630371,298.9830896)
\curveto(780.97629845,298.96308971)(780.84129858,298.95308972)(780.66130371,298.9530896)
\curveto(780.49129893,298.94308973)(780.35629907,298.94808972)(780.25630371,298.9680896)
\lineto(780.18130371,298.9680896)
\curveto(780.11129931,298.97808969)(780.04629938,298.98808968)(779.98630371,298.9980896)
\curveto(779.9262995,298.99808967)(779.86629956,299.00808966)(779.80630371,299.0280896)
\curveto(779.63629979,299.07808959)(779.47629995,299.12308955)(779.32630371,299.1630896)
\curveto(779.17630025,299.20308947)(779.03630039,299.26308941)(778.90630371,299.3430896)
\curveto(778.74630068,299.43308924)(778.60630082,299.52808914)(778.48630371,299.6280896)
\curveto(778.44630098,299.65808901)(778.38630104,299.69808897)(778.30630371,299.7480896)
\curveto(778.2263012,299.80808886)(778.15130127,299.81308886)(778.08130371,299.7630896)
\curveto(778.04130138,299.73308894)(778.0213014,299.69308898)(778.02130371,299.6430896)
\curveto(778.0213014,299.59308908)(778.01130141,299.53808913)(777.99130371,299.4780896)
\curveto(777.98130144,299.44808922)(777.98130144,299.41308926)(777.99130371,299.3730896)
\curveto(778.00130142,299.34308933)(778.00130142,299.30808936)(777.99130371,299.2680896)
\curveto(777.97130145,299.20808946)(777.96130146,299.14308953)(777.96130371,299.0730896)
\curveto(777.97130145,298.99308968)(777.97630145,298.92308975)(777.97630371,298.8630896)
\lineto(777.97630371,297.0630896)
\lineto(777.97630371,296.6280896)
\curveto(777.97630145,296.47809219)(777.94630148,296.36309231)(777.88630371,296.2830896)
\curveto(777.83630159,296.21309246)(777.75630167,296.17809249)(777.64630371,296.1780896)
\curveto(777.53630189,296.1680925)(777.426302,296.16309251)(777.31630371,296.1630896)
\lineto(777.07630371,296.1630896)
\curveto(777.00630242,296.18309249)(776.94630248,296.20309247)(776.89630371,296.2230896)
\curveto(776.85630257,296.24309243)(776.8213026,296.27809239)(776.79130371,296.3280896)
\curveto(776.74130268,296.39809227)(776.71630271,296.50809216)(776.71630371,296.6580896)
\curveto(776.7263027,296.80809186)(776.73130269,296.93809173)(776.73130371,297.0480896)
\lineto(776.73130371,306.0480896)
\lineto(776.73130371,306.4080896)
\curveto(776.74130268,306.53808213)(776.77130265,306.64308203)(776.82130371,306.7230896)
\curveto(776.85130257,306.76308191)(776.91630251,306.79308188)(777.01630371,306.8130896)
\curveto(777.1263023,306.84308183)(777.24130218,306.85308182)(777.36130371,306.8430896)
\curveto(777.48130194,306.84308183)(777.59130183,306.82808184)(777.69130371,306.7980896)
\curveto(777.80130162,306.77808189)(777.87130155,306.74808192)(777.90130371,306.7080896)
\curveto(777.94130148,306.65808201)(777.96130146,306.59808207)(777.96130371,306.5280896)
\curveto(777.97130145,306.45808221)(777.99130143,306.38808228)(778.02130371,306.3180896)
\curveto(778.04130138,306.28808238)(778.05630137,306.26308241)(778.06630371,306.2430896)
\curveto(778.08630134,306.23308244)(778.10630132,306.21808245)(778.12630371,306.1980896)
\curveto(778.23630119,306.18808248)(778.3263011,306.22308245)(778.39630371,306.3030896)
\curveto(778.47630095,306.38308229)(778.55130087,306.44808222)(778.62130371,306.4980896)
\curveto(778.88130054,306.67808199)(779.19130023,306.81808185)(779.55130371,306.9180896)
\curveto(779.64129978,306.93808173)(779.73129969,306.95308172)(779.82130371,306.9630896)
\curveto(779.9212995,306.9730817)(780.0212994,306.98808168)(780.12130371,307.0080896)
\curveto(780.16129926,307.01808165)(780.21129921,307.01808165)(780.27130371,307.0080896)
\curveto(780.33129909,306.99808167)(780.37129905,307.00308167)(780.39130371,307.0230896)
\curveto(780.8212986,307.03308164)(781.20129822,306.98808168)(781.53130371,306.8880896)
\curveto(781.86129756,306.79808187)(782.15629727,306.668082)(782.41630371,306.4980896)
\lineto(782.56630371,306.3780896)
\curveto(782.61629681,306.34808232)(782.66629676,306.31308236)(782.71630371,306.2730896)
\curveto(782.73629669,306.25308242)(782.75129667,306.23308244)(782.76130371,306.2130896)
\curveto(782.78129664,306.20308247)(782.80129662,306.18808248)(782.82130371,306.1680896)
\curveto(782.87129655,306.11808255)(782.9262965,306.06308261)(782.98630371,306.0030896)
\curveto(783.04629638,305.94308273)(783.10129632,305.88308279)(783.15130371,305.8230896)
\curveto(783.27129615,305.65308302)(783.39629603,305.4680832)(783.52630371,305.2680896)
\curveto(783.60629582,305.13808353)(783.67129575,304.99308368)(783.72130371,304.8330896)
\curveto(783.78129564,304.673084)(783.83629559,304.51308416)(783.88630371,304.3530896)
\curveto(783.90629552,304.2730844)(783.9212955,304.18808448)(783.93130371,304.0980896)
\curveto(783.95129547,304.00808466)(783.97129545,303.92308475)(783.99130371,303.8430896)
\lineto(783.99130371,303.7230896)
\curveto(784.00129542,303.69308498)(784.00629542,303.66308501)(784.00630371,303.6330896)
\curveto(784.0262954,303.58308509)(784.03129539,303.52808514)(784.02130371,303.4680896)
\curveto(784.0212954,303.40808526)(784.03129539,303.35308532)(784.05130371,303.3030896)
\lineto(784.05130371,303.1530896)
\moveto(782.71630371,302.7480896)
\curveto(782.73629669,302.79808587)(782.74129668,302.85808581)(782.73130371,302.9280896)
\curveto(782.7212967,303.00808566)(782.71629671,303.07808559)(782.71630371,303.1380896)
\curveto(782.71629671,303.30808536)(782.70629672,303.4680852)(782.68630371,303.6180896)
\curveto(782.67629675,303.7680849)(782.64629678,303.91308476)(782.59630371,304.0530896)
\lineto(782.53630371,304.2330896)
\curveto(782.5262969,304.30308437)(782.50629692,304.3680843)(782.47630371,304.4280896)
\curveto(782.36629706,304.69808397)(782.19129723,304.95808371)(781.95130371,305.2080896)
\curveto(781.7212977,305.45808321)(781.50129792,305.62808304)(781.29130371,305.7180896)
\curveto(781.21129821,305.75808291)(781.1262983,305.78808288)(781.03630371,305.8080896)
\curveto(780.95629847,305.82808284)(780.87129855,305.85308282)(780.78130371,305.8830896)
\curveto(780.69129873,305.90308277)(780.58629884,305.91308276)(780.46630371,305.9130896)
\lineto(780.13630371,305.9130896)
\curveto(780.11629931,305.89308278)(780.07629935,305.88308279)(780.01630371,305.8830896)
\curveto(779.96629946,305.89308278)(779.9212995,305.89308278)(779.88130371,305.8830896)
\lineto(779.61130371,305.8230896)
\curveto(779.53129989,305.80308287)(779.45129997,305.7730829)(779.37130371,305.7330896)
\curveto(779.05130037,305.59308308)(778.78630064,305.38808328)(778.57630371,305.1180896)
\curveto(778.37630105,304.85808381)(778.2213012,304.55308412)(778.11130371,304.2030896)
\curveto(778.07130135,304.09308458)(778.04130138,303.98308469)(778.02130371,303.8730896)
\curveto(778.01130141,303.76308491)(777.99630143,303.65308502)(777.97630371,303.5430896)
\curveto(777.96630146,303.50308517)(777.96130146,303.46308521)(777.96130371,303.4230896)
\curveto(777.96130146,303.39308528)(777.95630147,303.35808531)(777.94630371,303.3180896)
\lineto(777.94630371,303.1980896)
\curveto(777.93630149,303.14808552)(777.93130149,303.0730856)(777.93130371,302.9730896)
\curveto(777.93130149,302.88308579)(777.93630149,302.81308586)(777.94630371,302.7630896)
\lineto(777.94630371,302.6430896)
\curveto(777.95630147,302.60308607)(777.96130146,302.56308611)(777.96130371,302.5230896)
\curveto(777.96130146,302.48308619)(777.96630146,302.44808622)(777.97630371,302.4180896)
\curveto(777.98630144,302.38808628)(777.99130143,302.35808631)(777.99130371,302.3280896)
\curveto(777.99130143,302.29808637)(777.99630143,302.26308641)(778.00630371,302.2230896)
\curveto(778.0263014,302.14308653)(778.04130138,302.06308661)(778.05130371,301.9830896)
\lineto(778.11130371,301.7430896)
\curveto(778.2213012,301.40308727)(778.37130105,301.10308757)(778.56130371,300.8430896)
\curveto(778.76130066,300.59308808)(779.0213004,300.39808827)(779.34130371,300.2580896)
\curveto(779.53129989,300.17808849)(779.7262997,300.11808855)(779.92630371,300.0780896)
\curveto(779.96629946,300.05808861)(780.00629942,300.04808862)(780.04630371,300.0480896)
\curveto(780.08629934,300.05808861)(780.1262993,300.05808861)(780.16630371,300.0480896)
\lineto(780.28630371,300.0480896)
\curveto(780.35629907,300.02808864)(780.426299,300.02808864)(780.49630371,300.0480896)
\lineto(780.61630371,300.0480896)
\curveto(780.7262987,300.0680886)(780.83129859,300.08308859)(780.93130371,300.0930896)
\curveto(781.03129839,300.10308857)(781.13129829,300.12808854)(781.23130371,300.1680896)
\curveto(781.54129788,300.29808837)(781.79129763,300.4680882)(781.98130371,300.6780896)
\curveto(782.18129724,300.89808777)(782.34629708,301.16308751)(782.47630371,301.4730896)
\curveto(782.5262969,301.61308706)(782.56129686,301.75308692)(782.58130371,301.8930896)
\curveto(782.61129681,302.04308663)(782.64629678,302.19808647)(782.68630371,302.3580896)
\curveto(782.69629673,302.40808626)(782.70129672,302.45308622)(782.70130371,302.4930896)
\curveto(782.70129672,302.53308614)(782.70629672,302.57808609)(782.71630371,302.6280896)
\lineto(782.71630371,302.7480896)
}
}
{
\newrgbcolor{curcolor}{0 0 0}
\pscustom[linestyle=none,fillstyle=solid,fillcolor=curcolor]
{
\newpath
\moveto(785.99755371,306.8130896)
\lineto(786.43255371,306.8130896)
\curveto(786.58255175,306.81308186)(786.68755164,306.7730819)(786.74755371,306.6930896)
\curveto(786.79755153,306.61308206)(786.82255151,306.51308216)(786.82255371,306.3930896)
\curveto(786.8325515,306.2730824)(786.83755149,306.15308252)(786.83755371,306.0330896)
\lineto(786.83755371,304.6080896)
\lineto(786.83755371,302.3430896)
\lineto(786.83755371,301.6530896)
\curveto(786.83755149,301.42308725)(786.86255147,301.22308745)(786.91255371,301.0530896)
\curveto(787.07255126,300.60308807)(787.37255096,300.28808838)(787.81255371,300.1080896)
\curveto(788.0325503,300.01808865)(788.29755003,299.98308869)(788.60755371,300.0030896)
\curveto(788.91754941,300.03308864)(789.16754916,300.08808858)(789.35755371,300.1680896)
\curveto(789.68754864,300.30808836)(789.94754838,300.48308819)(790.13755371,300.6930896)
\curveto(790.33754799,300.91308776)(790.49254784,301.19808747)(790.60255371,301.5480896)
\curveto(790.6325477,301.62808704)(790.65254768,301.70808696)(790.66255371,301.7880896)
\curveto(790.67254766,301.8680868)(790.68754764,301.95308672)(790.70755371,302.0430896)
\curveto(790.71754761,302.09308658)(790.71754761,302.13808653)(790.70755371,302.1780896)
\curveto(790.70754762,302.21808645)(790.71754761,302.26308641)(790.73755371,302.3130896)
\lineto(790.73755371,302.6280896)
\curveto(790.75754757,302.70808596)(790.76254757,302.79808587)(790.75255371,302.8980896)
\curveto(790.74254759,303.00808566)(790.73754759,303.10808556)(790.73755371,303.1980896)
\lineto(790.73755371,304.3680896)
\lineto(790.73755371,305.9580896)
\curveto(790.73754759,306.07808259)(790.7325476,306.20308247)(790.72255371,306.3330896)
\curveto(790.72254761,306.4730822)(790.74754758,306.58308209)(790.79755371,306.6630896)
\curveto(790.83754749,306.71308196)(790.88254745,306.74308193)(790.93255371,306.7530896)
\curveto(790.99254734,306.7730819)(791.06254727,306.79308188)(791.14255371,306.8130896)
\lineto(791.36755371,306.8130896)
\curveto(791.48754684,306.81308186)(791.59254674,306.80808186)(791.68255371,306.7980896)
\curveto(791.78254655,306.78808188)(791.85754647,306.74308193)(791.90755371,306.6630896)
\curveto(791.95754637,306.61308206)(791.98254635,306.53808213)(791.98255371,306.4380896)
\lineto(791.98255371,306.1530896)
\lineto(791.98255371,305.1330896)
\lineto(791.98255371,301.0980896)
\lineto(791.98255371,299.7480896)
\curveto(791.98254635,299.62808904)(791.97754635,299.51308916)(791.96755371,299.4030896)
\curveto(791.96754636,299.30308937)(791.9325464,299.22808944)(791.86255371,299.1780896)
\curveto(791.82254651,299.14808952)(791.76254657,299.12308955)(791.68255371,299.1030896)
\curveto(791.60254673,299.09308958)(791.51254682,299.08308959)(791.41255371,299.0730896)
\curveto(791.32254701,299.0730896)(791.2325471,299.07808959)(791.14255371,299.0880896)
\curveto(791.06254727,299.09808957)(791.00254733,299.11808955)(790.96255371,299.1480896)
\curveto(790.91254742,299.18808948)(790.86754746,299.25308942)(790.82755371,299.3430896)
\curveto(790.81754751,299.38308929)(790.80754752,299.43808923)(790.79755371,299.5080896)
\curveto(790.79754753,299.57808909)(790.79254754,299.64308903)(790.78255371,299.7030896)
\curveto(790.77254756,299.7730889)(790.75254758,299.82808884)(790.72255371,299.8680896)
\curveto(790.69254764,299.90808876)(790.64754768,299.92308875)(790.58755371,299.9130896)
\curveto(790.50754782,299.89308878)(790.4275479,299.83308884)(790.34755371,299.7330896)
\curveto(790.26754806,299.64308903)(790.19254814,299.5730891)(790.12255371,299.5230896)
\curveto(789.90254843,299.36308931)(789.65254868,299.22308945)(789.37255371,299.1030896)
\curveto(789.26254907,299.05308962)(789.14754918,299.02308965)(789.02755371,299.0130896)
\curveto(788.91754941,298.99308968)(788.80254953,298.9680897)(788.68255371,298.9380896)
\curveto(788.6325497,298.92808974)(788.57754975,298.92808974)(788.51755371,298.9380896)
\curveto(788.46754986,298.94808972)(788.41754991,298.94308973)(788.36755371,298.9230896)
\curveto(788.26755006,298.90308977)(788.17755015,298.90308977)(788.09755371,298.9230896)
\lineto(787.94755371,298.9230896)
\curveto(787.89755043,298.94308973)(787.83755049,298.95308972)(787.76755371,298.9530896)
\curveto(787.70755062,298.95308972)(787.65255068,298.95808971)(787.60255371,298.9680896)
\curveto(787.56255077,298.98808968)(787.52255081,298.99808967)(787.48255371,298.9980896)
\curveto(787.45255088,298.98808968)(787.41255092,298.99308968)(787.36255371,299.0130896)
\lineto(787.12255371,299.0730896)
\curveto(787.05255128,299.09308958)(786.97755135,299.12308955)(786.89755371,299.1630896)
\curveto(786.63755169,299.2730894)(786.41755191,299.41808925)(786.23755371,299.5980896)
\curveto(786.06755226,299.78808888)(785.9275524,300.01308866)(785.81755371,300.2730896)
\curveto(785.77755255,300.36308831)(785.74755258,300.45308822)(785.72755371,300.5430896)
\lineto(785.66755371,300.8430896)
\curveto(785.64755268,300.90308777)(785.63755269,300.95808771)(785.63755371,301.0080896)
\curveto(785.64755268,301.0680876)(785.64255269,301.13308754)(785.62255371,301.2030896)
\curveto(785.61255272,301.22308745)(785.60755272,301.24808742)(785.60755371,301.2780896)
\curveto(785.60755272,301.31808735)(785.60255273,301.35308732)(785.59255371,301.3830896)
\lineto(785.59255371,301.5330896)
\curveto(785.58255275,301.5730871)(785.57755275,301.61808705)(785.57755371,301.6680896)
\curveto(785.58755274,301.72808694)(785.59255274,301.78308689)(785.59255371,301.8330896)
\lineto(785.59255371,302.4330896)
\lineto(785.59255371,305.1930896)
\lineto(785.59255371,306.1530896)
\lineto(785.59255371,306.4230896)
\curveto(785.59255274,306.51308216)(785.61255272,306.58808208)(785.65255371,306.6480896)
\curveto(785.69255264,306.71808195)(785.76755256,306.7680819)(785.87755371,306.7980896)
\curveto(785.89755243,306.80808186)(785.91755241,306.80808186)(785.93755371,306.7980896)
\curveto(785.95755237,306.79808187)(785.97755235,306.80308187)(785.99755371,306.8130896)
}
}
{
\newrgbcolor{curcolor}{0 0 0}
\pscustom[linestyle=none,fillstyle=solid,fillcolor=curcolor]
{
\newpath
\moveto(794.45216309,309.7680896)
\curveto(794.58216147,309.7680789)(794.71716134,309.7680789)(794.85716309,309.7680896)
\curveto(795.00716105,309.7680789)(795.11716094,309.73307894)(795.18716309,309.6630896)
\curveto(795.23716082,309.59307908)(795.26216079,309.49807917)(795.26216309,309.3780896)
\curveto(795.27216078,309.2680794)(795.27716078,309.15307952)(795.27716309,309.0330896)
\lineto(795.27716309,307.6980896)
\lineto(795.27716309,301.6230896)
\lineto(795.27716309,299.9430896)
\lineto(795.27716309,299.5530896)
\curveto(795.27716078,299.41308926)(795.2521608,299.30308937)(795.20216309,299.2230896)
\curveto(795.17216088,299.1730895)(795.12716093,299.14308953)(795.06716309,299.1330896)
\curveto(795.01716104,299.12308955)(794.9521611,299.10808956)(794.87216309,299.0880896)
\lineto(794.66216309,299.0880896)
\lineto(794.34716309,299.0880896)
\curveto(794.24716181,299.09808957)(794.17216188,299.13308954)(794.12216309,299.1930896)
\curveto(794.07216198,299.2730894)(794.04216201,299.3730893)(794.03216309,299.4930896)
\lineto(794.03216309,299.8680896)
\lineto(794.03216309,301.2480896)
\lineto(794.03216309,307.4880896)
\lineto(794.03216309,308.9580896)
\curveto(794.03216202,309.0680796)(794.02716203,309.18307949)(794.01716309,309.3030896)
\curveto(794.01716204,309.43307924)(794.04216201,309.53307914)(794.09216309,309.6030896)
\curveto(794.13216192,309.66307901)(794.20716185,309.71307896)(794.31716309,309.7530896)
\curveto(794.33716172,309.76307891)(794.3571617,309.76307891)(794.37716309,309.7530896)
\curveto(794.40716165,309.75307892)(794.43216162,309.75807891)(794.45216309,309.7680896)
}
}
{
\newrgbcolor{curcolor}{0 0 0}
\pscustom[linestyle=none,fillstyle=solid,fillcolor=curcolor]
{
\newpath
\moveto(804.10700684,299.6430896)
\curveto(804.13699901,299.48308919)(804.12199902,299.34808932)(804.06200684,299.2380896)
\curveto(804.00199914,299.13808953)(803.92199922,299.06308961)(803.82200684,299.0130896)
\curveto(803.77199937,298.99308968)(803.71699943,298.98308969)(803.65700684,298.9830896)
\curveto(803.60699954,298.98308969)(803.55199959,298.9730897)(803.49200684,298.9530896)
\curveto(803.27199987,298.90308977)(803.05200009,298.91808975)(802.83200684,298.9980896)
\curveto(802.62200052,299.0680896)(802.47700067,299.15808951)(802.39700684,299.2680896)
\curveto(802.3470008,299.33808933)(802.30200084,299.41808925)(802.26200684,299.5080896)
\curveto(802.22200092,299.60808906)(802.17200097,299.68808898)(802.11200684,299.7480896)
\curveto(802.09200105,299.7680889)(802.06700108,299.78808888)(802.03700684,299.8080896)
\curveto(802.01700113,299.82808884)(801.98700116,299.83308884)(801.94700684,299.8230896)
\curveto(801.83700131,299.79308888)(801.73200141,299.73808893)(801.63200684,299.6580896)
\curveto(801.5420016,299.57808909)(801.45200169,299.50808916)(801.36200684,299.4480896)
\curveto(801.23200191,299.3680893)(801.09200205,299.29308938)(800.94200684,299.2230896)
\curveto(800.79200235,299.16308951)(800.63200251,299.10808956)(800.46200684,299.0580896)
\curveto(800.36200278,299.02808964)(800.25200289,299.00808966)(800.13200684,298.9980896)
\curveto(800.02200312,298.98808968)(799.91200323,298.9730897)(799.80200684,298.9530896)
\curveto(799.75200339,298.94308973)(799.70700344,298.93808973)(799.66700684,298.9380896)
\lineto(799.56200684,298.9380896)
\curveto(799.45200369,298.91808975)(799.3470038,298.91808975)(799.24700684,298.9380896)
\lineto(799.11200684,298.9380896)
\curveto(799.06200408,298.94808972)(799.01200413,298.95308972)(798.96200684,298.9530896)
\curveto(798.91200423,298.95308972)(798.86700428,298.96308971)(798.82700684,298.9830896)
\curveto(798.78700436,298.99308968)(798.75200439,298.99808967)(798.72200684,298.9980896)
\curveto(798.70200444,298.98808968)(798.67700447,298.98808968)(798.64700684,298.9980896)
\lineto(798.40700684,299.0580896)
\curveto(798.32700482,299.0680896)(798.25200489,299.08808958)(798.18200684,299.1180896)
\curveto(797.88200526,299.24808942)(797.63700551,299.39308928)(797.44700684,299.5530896)
\curveto(797.26700588,299.72308895)(797.11700603,299.95808871)(796.99700684,300.2580896)
\curveto(796.90700624,300.47808819)(796.86200628,300.74308793)(796.86200684,301.0530896)
\lineto(796.86200684,301.3680896)
\curveto(796.87200627,301.41808725)(796.87700627,301.4680872)(796.87700684,301.5180896)
\lineto(796.90700684,301.6980896)
\lineto(797.02700684,302.0280896)
\curveto(797.06700608,302.13808653)(797.11700603,302.23808643)(797.17700684,302.3280896)
\curveto(797.35700579,302.61808605)(797.60200554,302.83308584)(797.91200684,302.9730896)
\curveto(798.22200492,303.11308556)(798.56200458,303.23808543)(798.93200684,303.3480896)
\curveto(799.07200407,303.38808528)(799.21700393,303.41808525)(799.36700684,303.4380896)
\curveto(799.51700363,303.45808521)(799.66700348,303.48308519)(799.81700684,303.5130896)
\curveto(799.88700326,303.53308514)(799.95200319,303.54308513)(800.01200684,303.5430896)
\curveto(800.08200306,303.54308513)(800.15700299,303.55308512)(800.23700684,303.5730896)
\curveto(800.30700284,303.59308508)(800.37700277,303.60308507)(800.44700684,303.6030896)
\curveto(800.51700263,303.61308506)(800.59200255,303.62808504)(800.67200684,303.6480896)
\curveto(800.92200222,303.70808496)(801.15700199,303.75808491)(801.37700684,303.7980896)
\curveto(801.59700155,303.84808482)(801.77200137,303.96308471)(801.90200684,304.1430896)
\curveto(801.96200118,304.22308445)(802.01200113,304.32308435)(802.05200684,304.4430896)
\curveto(802.09200105,304.5730841)(802.09200105,304.71308396)(802.05200684,304.8630896)
\curveto(801.99200115,305.10308357)(801.90200124,305.29308338)(801.78200684,305.4330896)
\curveto(801.67200147,305.5730831)(801.51200163,305.68308299)(801.30200684,305.7630896)
\curveto(801.18200196,305.81308286)(801.03700211,305.84808282)(800.86700684,305.8680896)
\curveto(800.70700244,305.88808278)(800.53700261,305.89808277)(800.35700684,305.8980896)
\curveto(800.17700297,305.89808277)(800.00200314,305.88808278)(799.83200684,305.8680896)
\curveto(799.66200348,305.84808282)(799.51700363,305.81808285)(799.39700684,305.7780896)
\curveto(799.22700392,305.71808295)(799.06200408,305.63308304)(798.90200684,305.5230896)
\curveto(798.82200432,305.46308321)(798.7470044,305.38308329)(798.67700684,305.2830896)
\curveto(798.61700453,305.19308348)(798.56200458,305.09308358)(798.51200684,304.9830896)
\curveto(798.48200466,304.90308377)(798.45200469,304.81808385)(798.42200684,304.7280896)
\curveto(798.40200474,304.63808403)(798.35700479,304.5680841)(798.28700684,304.5180896)
\curveto(798.2470049,304.48808418)(798.17700497,304.46308421)(798.07700684,304.4430896)
\curveto(797.98700516,304.43308424)(797.89200525,304.42808424)(797.79200684,304.4280896)
\curveto(797.69200545,304.42808424)(797.59200555,304.43308424)(797.49200684,304.4430896)
\curveto(797.40200574,304.46308421)(797.33700581,304.48808418)(797.29700684,304.5180896)
\curveto(797.25700589,304.54808412)(797.22700592,304.59808407)(797.20700684,304.6680896)
\curveto(797.18700596,304.73808393)(797.18700596,304.81308386)(797.20700684,304.8930896)
\curveto(797.23700591,305.02308365)(797.26700588,305.14308353)(797.29700684,305.2530896)
\curveto(797.33700581,305.3730833)(797.38200576,305.48808318)(797.43200684,305.5980896)
\curveto(797.62200552,305.94808272)(797.86200528,306.21808245)(798.15200684,306.4080896)
\curveto(798.4420047,306.60808206)(798.80200434,306.7680819)(799.23200684,306.8880896)
\curveto(799.33200381,306.90808176)(799.43200371,306.92308175)(799.53200684,306.9330896)
\curveto(799.6420035,306.94308173)(799.75200339,306.95808171)(799.86200684,306.9780896)
\curveto(799.90200324,306.98808168)(799.96700318,306.98808168)(800.05700684,306.9780896)
\curveto(800.147003,306.97808169)(800.20200294,306.98808168)(800.22200684,307.0080896)
\curveto(800.92200222,307.01808165)(801.53200161,306.93808173)(802.05200684,306.7680896)
\curveto(802.57200057,306.59808207)(802.93700021,306.2730824)(803.14700684,305.7930896)
\curveto(803.23699991,305.59308308)(803.28699986,305.35808331)(803.29700684,305.0880896)
\curveto(803.31699983,304.82808384)(803.32699982,304.55308412)(803.32700684,304.2630896)
\lineto(803.32700684,300.9480896)
\curveto(803.32699982,300.80808786)(803.33199981,300.673088)(803.34200684,300.5430896)
\curveto(803.35199979,300.41308826)(803.38199976,300.30808836)(803.43200684,300.2280896)
\curveto(803.48199966,300.15808851)(803.5469996,300.10808856)(803.62700684,300.0780896)
\curveto(803.71699943,300.03808863)(803.80199934,300.00808866)(803.88200684,299.9880896)
\curveto(803.96199918,299.97808869)(804.02199912,299.93308874)(804.06200684,299.8530896)
\curveto(804.08199906,299.82308885)(804.09199905,299.79308888)(804.09200684,299.7630896)
\curveto(804.09199905,299.73308894)(804.09699905,299.69308898)(804.10700684,299.6430896)
\moveto(801.96200684,301.3080896)
\curveto(802.02200112,301.44808722)(802.05200109,301.60808706)(802.05200684,301.7880896)
\curveto(802.06200108,301.97808669)(802.06700108,302.1730865)(802.06700684,302.3730896)
\curveto(802.06700108,302.48308619)(802.06200108,302.58308609)(802.05200684,302.6730896)
\curveto(802.0420011,302.76308591)(802.00200114,302.83308584)(801.93200684,302.8830896)
\curveto(801.90200124,302.90308577)(801.83200131,302.91308576)(801.72200684,302.9130896)
\curveto(801.70200144,302.89308578)(801.66700148,302.88308579)(801.61700684,302.8830896)
\curveto(801.56700158,302.88308579)(801.52200162,302.8730858)(801.48200684,302.8530896)
\curveto(801.40200174,302.83308584)(801.31200183,302.81308586)(801.21200684,302.7930896)
\lineto(800.91200684,302.7330896)
\curveto(800.88200226,302.73308594)(800.8470023,302.72808594)(800.80700684,302.7180896)
\lineto(800.70200684,302.7180896)
\curveto(800.55200259,302.67808599)(800.38700276,302.65308602)(800.20700684,302.6430896)
\curveto(800.03700311,302.64308603)(799.87700327,302.62308605)(799.72700684,302.5830896)
\curveto(799.6470035,302.56308611)(799.57200357,302.54308613)(799.50200684,302.5230896)
\curveto(799.4420037,302.51308616)(799.37200377,302.49808617)(799.29200684,302.4780896)
\curveto(799.13200401,302.42808624)(798.98200416,302.36308631)(798.84200684,302.2830896)
\curveto(798.70200444,302.21308646)(798.58200456,302.12308655)(798.48200684,302.0130896)
\curveto(798.38200476,301.90308677)(798.30700484,301.7680869)(798.25700684,301.6080896)
\curveto(798.20700494,301.45808721)(798.18700496,301.2730874)(798.19700684,301.0530896)
\curveto(798.19700495,300.95308772)(798.21200493,300.85808781)(798.24200684,300.7680896)
\curveto(798.28200486,300.68808798)(798.32700482,300.61308806)(798.37700684,300.5430896)
\curveto(798.45700469,300.43308824)(798.56200458,300.33808833)(798.69200684,300.2580896)
\curveto(798.82200432,300.18808848)(798.96200418,300.12808854)(799.11200684,300.0780896)
\curveto(799.16200398,300.0680886)(799.21200393,300.06308861)(799.26200684,300.0630896)
\curveto(799.31200383,300.06308861)(799.36200378,300.05808861)(799.41200684,300.0480896)
\curveto(799.48200366,300.02808864)(799.56700358,300.01308866)(799.66700684,300.0030896)
\curveto(799.77700337,300.00308867)(799.86700328,300.01308866)(799.93700684,300.0330896)
\curveto(799.99700315,300.05308862)(800.05700309,300.05808861)(800.11700684,300.0480896)
\curveto(800.17700297,300.04808862)(800.23700291,300.05808861)(800.29700684,300.0780896)
\curveto(800.37700277,300.09808857)(800.45200269,300.11308856)(800.52200684,300.1230896)
\curveto(800.60200254,300.13308854)(800.67700247,300.15308852)(800.74700684,300.1830896)
\curveto(801.03700211,300.30308837)(801.28200186,300.44808822)(801.48200684,300.6180896)
\curveto(801.69200145,300.78808788)(801.85200129,301.01808765)(801.96200684,301.3080896)
}
}
{
\newrgbcolor{curcolor}{0 0 0}
\pscustom[linestyle=none,fillstyle=solid,fillcolor=curcolor]
{
\newpath
\moveto(808.92364746,306.9930896)
\curveto(809.15364267,306.99308168)(809.28364254,306.93308174)(809.31364746,306.8130896)
\curveto(809.34364248,306.70308197)(809.35864247,306.53808213)(809.35864746,306.3180896)
\lineto(809.35864746,306.0330896)
\curveto(809.35864247,305.94308273)(809.33364249,305.8680828)(809.28364746,305.8080896)
\curveto(809.2236426,305.72808294)(809.13864269,305.68308299)(809.02864746,305.6730896)
\curveto(808.91864291,305.673083)(808.80864302,305.65808301)(808.69864746,305.6280896)
\curveto(808.55864327,305.59808307)(808.4236434,305.5680831)(808.29364746,305.5380896)
\curveto(808.17364365,305.50808316)(808.05864377,305.4680832)(807.94864746,305.4180896)
\curveto(807.65864417,305.28808338)(807.4236444,305.10808356)(807.24364746,304.8780896)
\curveto(807.06364476,304.65808401)(806.90864492,304.40308427)(806.77864746,304.1130896)
\curveto(806.73864509,304.00308467)(806.70864512,303.88808478)(806.68864746,303.7680896)
\curveto(806.66864516,303.65808501)(806.64364518,303.54308513)(806.61364746,303.4230896)
\curveto(806.60364522,303.3730853)(806.59864523,303.32308535)(806.59864746,303.2730896)
\curveto(806.60864522,303.22308545)(806.60864522,303.1730855)(806.59864746,303.1230896)
\curveto(806.56864526,303.00308567)(806.55364527,302.86308581)(806.55364746,302.7030896)
\curveto(806.56364526,302.55308612)(806.56864526,302.40808626)(806.56864746,302.2680896)
\lineto(806.56864746,300.4230896)
\lineto(806.56864746,300.0780896)
\curveto(806.56864526,299.95808871)(806.56364526,299.84308883)(806.55364746,299.7330896)
\curveto(806.54364528,299.62308905)(806.53864529,299.52808914)(806.53864746,299.4480896)
\curveto(806.54864528,299.3680893)(806.5286453,299.29808937)(806.47864746,299.2380896)
\curveto(806.4286454,299.1680895)(806.34864548,299.12808954)(806.23864746,299.1180896)
\curveto(806.13864569,299.10808956)(806.0286458,299.10308957)(805.90864746,299.1030896)
\lineto(805.63864746,299.1030896)
\curveto(805.58864624,299.12308955)(805.53864629,299.13808953)(805.48864746,299.1480896)
\curveto(805.44864638,299.1680895)(805.41864641,299.19308948)(805.39864746,299.2230896)
\curveto(805.34864648,299.29308938)(805.31864651,299.37808929)(805.30864746,299.4780896)
\lineto(805.30864746,299.8080896)
\lineto(805.30864746,300.9630896)
\lineto(805.30864746,305.1180896)
\lineto(805.30864746,306.1530896)
\lineto(805.30864746,306.4530896)
\curveto(805.31864651,306.55308212)(805.34864648,306.63808203)(805.39864746,306.7080896)
\curveto(805.4286464,306.74808192)(805.47864635,306.77808189)(805.54864746,306.7980896)
\curveto(805.6286462,306.81808185)(805.71364611,306.82808184)(805.80364746,306.8280896)
\curveto(805.89364593,306.83808183)(805.98364584,306.83808183)(806.07364746,306.8280896)
\curveto(806.16364566,306.81808185)(806.23364559,306.80308187)(806.28364746,306.7830896)
\curveto(806.36364546,306.75308192)(806.41364541,306.69308198)(806.43364746,306.6030896)
\curveto(806.46364536,306.52308215)(806.47864535,306.43308224)(806.47864746,306.3330896)
\lineto(806.47864746,306.0330896)
\curveto(806.47864535,305.93308274)(806.49864533,305.84308283)(806.53864746,305.7630896)
\curveto(806.54864528,305.74308293)(806.55864527,305.72808294)(806.56864746,305.7180896)
\lineto(806.61364746,305.6730896)
\curveto(806.7236451,305.673083)(806.81364501,305.71808295)(806.88364746,305.8080896)
\curveto(806.95364487,305.90808276)(807.01364481,305.98808268)(807.06364746,306.0480896)
\lineto(807.15364746,306.1380896)
\curveto(807.24364458,306.24808242)(807.36864446,306.36308231)(807.52864746,306.4830896)
\curveto(807.68864414,306.60308207)(807.83864399,306.69308198)(807.97864746,306.7530896)
\curveto(808.06864376,306.80308187)(808.16364366,306.83808183)(808.26364746,306.8580896)
\curveto(808.36364346,306.88808178)(808.46864336,306.91808175)(808.57864746,306.9480896)
\curveto(808.63864319,306.95808171)(808.69864313,306.96308171)(808.75864746,306.9630896)
\curveto(808.81864301,306.9730817)(808.87364295,306.98308169)(808.92364746,306.9930896)
}
}
{
\newrgbcolor{curcolor}{0 0 0}
\pscustom[linestyle=none,fillstyle=solid,fillcolor=curcolor]
{
\newpath
\moveto(810.57341309,308.3130896)
\curveto(810.49341197,308.3730803)(810.44841201,308.47808019)(810.43841309,308.6280896)
\lineto(810.43841309,309.0930896)
\lineto(810.43841309,309.3480896)
\curveto(810.43841202,309.43807923)(810.45341201,309.51307916)(810.48341309,309.5730896)
\curveto(810.52341194,309.65307902)(810.60341186,309.71307896)(810.72341309,309.7530896)
\curveto(810.74341172,309.76307891)(810.7634117,309.76307891)(810.78341309,309.7530896)
\curveto(810.81341165,309.75307892)(810.83841162,309.75807891)(810.85841309,309.7680896)
\curveto(811.02841143,309.7680789)(811.18841127,309.76307891)(811.33841309,309.7530896)
\curveto(811.48841097,309.74307893)(811.58841087,309.68307899)(811.63841309,309.5730896)
\curveto(811.66841079,309.51307916)(811.68341078,309.43807923)(811.68341309,309.3480896)
\lineto(811.68341309,309.0930896)
\curveto(811.68341078,308.91307976)(811.67841078,308.74307993)(811.66841309,308.5830896)
\curveto(811.66841079,308.42308025)(811.60341086,308.31808035)(811.47341309,308.2680896)
\curveto(811.42341104,308.24808042)(811.36841109,308.23808043)(811.30841309,308.2380896)
\lineto(811.14341309,308.2380896)
\lineto(810.82841309,308.2380896)
\curveto(810.72841173,308.23808043)(810.64341182,308.26308041)(810.57341309,308.3130896)
\moveto(811.68341309,299.8080896)
\lineto(811.68341309,299.4930896)
\curveto(811.69341077,299.39308928)(811.67341079,299.31308936)(811.62341309,299.2530896)
\curveto(811.59341087,299.19308948)(811.54841091,299.15308952)(811.48841309,299.1330896)
\curveto(811.42841103,299.12308955)(811.3584111,299.10808956)(811.27841309,299.0880896)
\lineto(811.05341309,299.0880896)
\curveto(810.92341154,299.08808958)(810.80841165,299.09308958)(810.70841309,299.1030896)
\curveto(810.61841184,299.12308955)(810.54841191,299.1730895)(810.49841309,299.2530896)
\curveto(810.458412,299.31308936)(810.43841202,299.38808928)(810.43841309,299.4780896)
\lineto(810.43841309,299.7630896)
\lineto(810.43841309,306.1080896)
\lineto(810.43841309,306.4230896)
\curveto(810.43841202,306.53308214)(810.463412,306.61808205)(810.51341309,306.6780896)
\curveto(810.54341192,306.72808194)(810.58341188,306.75808191)(810.63341309,306.7680896)
\curveto(810.68341178,306.77808189)(810.73841172,306.79308188)(810.79841309,306.8130896)
\curveto(810.81841164,306.81308186)(810.83841162,306.80808186)(810.85841309,306.7980896)
\curveto(810.88841157,306.79808187)(810.91341155,306.80308187)(810.93341309,306.8130896)
\curveto(811.0634114,306.81308186)(811.19341127,306.80808186)(811.32341309,306.7980896)
\curveto(811.463411,306.79808187)(811.5584109,306.75808191)(811.60841309,306.6780896)
\curveto(811.6584108,306.61808205)(811.68341078,306.53808213)(811.68341309,306.4380896)
\lineto(811.68341309,306.1530896)
\lineto(811.68341309,299.8080896)
}
}
{
\newrgbcolor{curcolor}{0 0 0}
\pscustom[linestyle=none,fillstyle=solid,fillcolor=curcolor]
{
\newpath
\moveto(820.58825684,299.8980896)
\lineto(820.58825684,299.5080896)
\curveto(820.58824896,299.38808928)(820.56324899,299.28808938)(820.51325684,299.2080896)
\curveto(820.46324909,299.13808953)(820.37824917,299.09808957)(820.25825684,299.0880896)
\lineto(819.91325684,299.0880896)
\curveto(819.8532497,299.08808958)(819.79324976,299.08308959)(819.73325684,299.0730896)
\curveto(819.68324987,299.0730896)(819.63824991,299.08308959)(819.59825684,299.1030896)
\curveto(819.50825004,299.12308955)(819.4482501,299.16308951)(819.41825684,299.2230896)
\curveto(819.37825017,299.2730894)(819.3532502,299.33308934)(819.34325684,299.4030896)
\curveto(819.34325021,299.4730892)(819.32825022,299.54308913)(819.29825684,299.6130896)
\curveto(819.28825026,299.63308904)(819.27325028,299.64808902)(819.25325684,299.6580896)
\curveto(819.24325031,299.67808899)(819.22825032,299.69808897)(819.20825684,299.7180896)
\curveto(819.10825044,299.72808894)(819.02825052,299.70808896)(818.96825684,299.6580896)
\curveto(818.91825063,299.60808906)(818.86325069,299.55808911)(818.80325684,299.5080896)
\curveto(818.60325095,299.35808931)(818.40325115,299.24308943)(818.20325684,299.1630896)
\curveto(818.02325153,299.08308959)(817.81325174,299.02308965)(817.57325684,298.9830896)
\curveto(817.34325221,298.94308973)(817.10325245,298.92308975)(816.85325684,298.9230896)
\curveto(816.61325294,298.91308976)(816.37325318,298.92808974)(816.13325684,298.9680896)
\curveto(815.89325366,298.99808967)(815.68325387,299.05308962)(815.50325684,299.1330896)
\curveto(814.98325457,299.35308932)(814.56325499,299.64808902)(814.24325684,300.0180896)
\curveto(813.92325563,300.39808827)(813.67325588,300.8680878)(813.49325684,301.4280896)
\curveto(813.4532561,301.51808715)(813.42325613,301.60808706)(813.40325684,301.6980896)
\curveto(813.39325616,301.79808687)(813.37325618,301.89808677)(813.34325684,301.9980896)
\curveto(813.33325622,302.04808662)(813.32825622,302.09808657)(813.32825684,302.1480896)
\curveto(813.32825622,302.19808647)(813.32325623,302.24808642)(813.31325684,302.2980896)
\curveto(813.29325626,302.34808632)(813.28325627,302.39808627)(813.28325684,302.4480896)
\curveto(813.29325626,302.50808616)(813.29325626,302.56308611)(813.28325684,302.6130896)
\lineto(813.28325684,302.7630896)
\curveto(813.26325629,302.81308586)(813.2532563,302.87808579)(813.25325684,302.9580896)
\curveto(813.2532563,303.03808563)(813.26325629,303.10308557)(813.28325684,303.1530896)
\lineto(813.28325684,303.3180896)
\curveto(813.30325625,303.38808528)(813.30825624,303.45808521)(813.29825684,303.5280896)
\curveto(813.29825625,303.60808506)(813.30825624,303.68308499)(813.32825684,303.7530896)
\curveto(813.33825621,303.80308487)(813.34325621,303.84808482)(813.34325684,303.8880896)
\curveto(813.34325621,303.92808474)(813.3482562,303.9730847)(813.35825684,304.0230896)
\curveto(813.38825616,304.12308455)(813.41325614,304.21808445)(813.43325684,304.3080896)
\curveto(813.4532561,304.40808426)(813.47825607,304.50308417)(813.50825684,304.5930896)
\curveto(813.63825591,304.9730837)(813.80325575,305.31308336)(814.00325684,305.6130896)
\curveto(814.21325534,305.92308275)(814.46325509,306.17808249)(814.75325684,306.3780896)
\curveto(814.92325463,306.49808217)(815.09825445,306.59808207)(815.27825684,306.6780896)
\curveto(815.46825408,306.75808191)(815.67325388,306.82808184)(815.89325684,306.8880896)
\curveto(815.96325359,306.89808177)(816.02825352,306.90808176)(816.08825684,306.9180896)
\curveto(816.15825339,306.92808174)(816.22825332,306.94308173)(816.29825684,306.9630896)
\lineto(816.44825684,306.9630896)
\curveto(816.52825302,306.98308169)(816.64325291,306.99308168)(816.79325684,306.9930896)
\curveto(816.9532526,306.99308168)(817.07325248,306.98308169)(817.15325684,306.9630896)
\curveto(817.19325236,306.95308172)(817.2482523,306.94808172)(817.31825684,306.9480896)
\curveto(817.42825212,306.91808175)(817.53825201,306.89308178)(817.64825684,306.8730896)
\curveto(817.75825179,306.86308181)(817.86325169,306.83308184)(817.96325684,306.7830896)
\curveto(818.11325144,306.72308195)(818.2532513,306.65808201)(818.38325684,306.5880896)
\curveto(818.52325103,306.51808215)(818.6532509,306.43808223)(818.77325684,306.3480896)
\curveto(818.83325072,306.29808237)(818.89325066,306.24308243)(818.95325684,306.1830896)
\curveto(819.02325053,306.13308254)(819.11325044,306.11808255)(819.22325684,306.1380896)
\curveto(819.24325031,306.1680825)(819.25825029,306.19308248)(819.26825684,306.2130896)
\curveto(819.28825026,306.23308244)(819.30325025,306.26308241)(819.31325684,306.3030896)
\curveto(819.34325021,306.39308228)(819.3532502,306.50808216)(819.34325684,306.6480896)
\lineto(819.34325684,307.0230896)
\lineto(819.34325684,308.7480896)
\lineto(819.34325684,309.2130896)
\curveto(819.34325021,309.39307928)(819.36825018,309.52307915)(819.41825684,309.6030896)
\curveto(819.45825009,309.673079)(819.51825003,309.71807895)(819.59825684,309.7380896)
\curveto(819.61824993,309.73807893)(819.64324991,309.73807893)(819.67325684,309.7380896)
\curveto(819.70324985,309.74807892)(819.72824982,309.75307892)(819.74825684,309.7530896)
\curveto(819.88824966,309.76307891)(820.03324952,309.76307891)(820.18325684,309.7530896)
\curveto(820.34324921,309.75307892)(820.4532491,309.71307896)(820.51325684,309.6330896)
\curveto(820.56324899,309.55307912)(820.58824896,309.45307922)(820.58825684,309.3330896)
\lineto(820.58825684,308.9580896)
\lineto(820.58825684,299.8980896)
\moveto(819.37325684,302.7330896)
\curveto(819.39325016,302.78308589)(819.40325015,302.84808582)(819.40325684,302.9280896)
\curveto(819.40325015,303.01808565)(819.39325016,303.08808558)(819.37325684,303.1380896)
\lineto(819.37325684,303.3630896)
\curveto(819.3532502,303.45308522)(819.33825021,303.54308513)(819.32825684,303.6330896)
\curveto(819.31825023,303.73308494)(819.29825025,303.82308485)(819.26825684,303.9030896)
\curveto(819.2482503,303.98308469)(819.22825032,304.05808461)(819.20825684,304.1280896)
\curveto(819.19825035,304.19808447)(819.17825037,304.2680844)(819.14825684,304.3380896)
\curveto(819.02825052,304.63808403)(818.87325068,304.90308377)(818.68325684,305.1330896)
\curveto(818.49325106,305.36308331)(818.2532513,305.54308313)(817.96325684,305.6730896)
\curveto(817.86325169,305.72308295)(817.75825179,305.75808291)(817.64825684,305.7780896)
\curveto(817.548252,305.80808286)(817.43825211,305.83308284)(817.31825684,305.8530896)
\curveto(817.23825231,305.8730828)(817.1482524,305.88308279)(817.04825684,305.8830896)
\lineto(816.77825684,305.8830896)
\curveto(816.72825282,305.8730828)(816.68325287,305.86308281)(816.64325684,305.8530896)
\lineto(816.50825684,305.8530896)
\curveto(816.42825312,305.83308284)(816.34325321,305.81308286)(816.25325684,305.7930896)
\curveto(816.17325338,305.7730829)(816.09325346,305.74808292)(816.01325684,305.7180896)
\curveto(815.69325386,305.57808309)(815.43325412,305.3730833)(815.23325684,305.1030896)
\curveto(815.04325451,304.84308383)(814.88825466,304.53808413)(814.76825684,304.1880896)
\curveto(814.72825482,304.07808459)(814.69825485,303.96308471)(814.67825684,303.8430896)
\curveto(814.66825488,303.73308494)(814.6532549,303.62308505)(814.63325684,303.5130896)
\curveto(814.63325492,303.4730852)(814.62825492,303.43308524)(814.61825684,303.3930896)
\lineto(814.61825684,303.2880896)
\curveto(814.59825495,303.23808543)(814.58825496,303.18308549)(814.58825684,303.1230896)
\curveto(814.59825495,303.06308561)(814.60325495,303.00808566)(814.60325684,302.9580896)
\lineto(814.60325684,302.6280896)
\curveto(814.60325495,302.52808614)(814.61325494,302.43308624)(814.63325684,302.3430896)
\curveto(814.64325491,302.31308636)(814.6482549,302.26308641)(814.64825684,302.1930896)
\curveto(814.66825488,302.12308655)(814.68325487,302.05308662)(814.69325684,301.9830896)
\lineto(814.75325684,301.7730896)
\curveto(814.86325469,301.42308725)(815.01325454,301.12308755)(815.20325684,300.8730896)
\curveto(815.39325416,300.62308805)(815.63325392,300.41808825)(815.92325684,300.2580896)
\curveto(816.01325354,300.20808846)(816.10325345,300.1680885)(816.19325684,300.1380896)
\curveto(816.28325327,300.10808856)(816.38325317,300.07808859)(816.49325684,300.0480896)
\curveto(816.54325301,300.02808864)(816.59325296,300.02308865)(816.64325684,300.0330896)
\curveto(816.70325285,300.04308863)(816.75825279,300.03808863)(816.80825684,300.0180896)
\curveto(816.8482527,300.00808866)(816.88825266,300.00308867)(816.92825684,300.0030896)
\lineto(817.06325684,300.0030896)
\lineto(817.19825684,300.0030896)
\curveto(817.22825232,300.01308866)(817.27825227,300.01808865)(817.34825684,300.0180896)
\curveto(817.42825212,300.03808863)(817.50825204,300.05308862)(817.58825684,300.0630896)
\curveto(817.66825188,300.08308859)(817.74325181,300.10808856)(817.81325684,300.1380896)
\curveto(818.14325141,300.27808839)(818.40825114,300.45308822)(818.60825684,300.6630896)
\curveto(818.81825073,300.88308779)(818.99325056,301.15808751)(819.13325684,301.4880896)
\curveto(819.18325037,301.59808707)(819.21825033,301.70808696)(819.23825684,301.8180896)
\curveto(819.25825029,301.92808674)(819.28325027,302.03808663)(819.31325684,302.1480896)
\curveto(819.33325022,302.18808648)(819.34325021,302.22308645)(819.34325684,302.2530896)
\curveto(819.34325021,302.29308638)(819.3482502,302.33308634)(819.35825684,302.3730896)
\curveto(819.36825018,302.43308624)(819.36825018,302.49308618)(819.35825684,302.5530896)
\curveto(819.35825019,302.61308606)(819.36325019,302.673086)(819.37325684,302.7330896)
}
}
{
\newrgbcolor{curcolor}{0 0 0}
\pscustom[linestyle=none,fillstyle=solid,fillcolor=curcolor]
{
\newpath
\moveto(829.41950684,299.6430896)
\curveto(829.44949901,299.48308919)(829.43449902,299.34808932)(829.37450684,299.2380896)
\curveto(829.31449914,299.13808953)(829.23449922,299.06308961)(829.13450684,299.0130896)
\curveto(829.08449937,298.99308968)(829.02949943,298.98308969)(828.96950684,298.9830896)
\curveto(828.91949954,298.98308969)(828.86449959,298.9730897)(828.80450684,298.9530896)
\curveto(828.58449987,298.90308977)(828.36450009,298.91808975)(828.14450684,298.9980896)
\curveto(827.93450052,299.0680896)(827.78950067,299.15808951)(827.70950684,299.2680896)
\curveto(827.6595008,299.33808933)(827.61450084,299.41808925)(827.57450684,299.5080896)
\curveto(827.53450092,299.60808906)(827.48450097,299.68808898)(827.42450684,299.7480896)
\curveto(827.40450105,299.7680889)(827.37950108,299.78808888)(827.34950684,299.8080896)
\curveto(827.32950113,299.82808884)(827.29950116,299.83308884)(827.25950684,299.8230896)
\curveto(827.14950131,299.79308888)(827.04450141,299.73808893)(826.94450684,299.6580896)
\curveto(826.8545016,299.57808909)(826.76450169,299.50808916)(826.67450684,299.4480896)
\curveto(826.54450191,299.3680893)(826.40450205,299.29308938)(826.25450684,299.2230896)
\curveto(826.10450235,299.16308951)(825.94450251,299.10808956)(825.77450684,299.0580896)
\curveto(825.67450278,299.02808964)(825.56450289,299.00808966)(825.44450684,298.9980896)
\curveto(825.33450312,298.98808968)(825.22450323,298.9730897)(825.11450684,298.9530896)
\curveto(825.06450339,298.94308973)(825.01950344,298.93808973)(824.97950684,298.9380896)
\lineto(824.87450684,298.9380896)
\curveto(824.76450369,298.91808975)(824.6595038,298.91808975)(824.55950684,298.9380896)
\lineto(824.42450684,298.9380896)
\curveto(824.37450408,298.94808972)(824.32450413,298.95308972)(824.27450684,298.9530896)
\curveto(824.22450423,298.95308972)(824.17950428,298.96308971)(824.13950684,298.9830896)
\curveto(824.09950436,298.99308968)(824.06450439,298.99808967)(824.03450684,298.9980896)
\curveto(824.01450444,298.98808968)(823.98950447,298.98808968)(823.95950684,298.9980896)
\lineto(823.71950684,299.0580896)
\curveto(823.63950482,299.0680896)(823.56450489,299.08808958)(823.49450684,299.1180896)
\curveto(823.19450526,299.24808942)(822.94950551,299.39308928)(822.75950684,299.5530896)
\curveto(822.57950588,299.72308895)(822.42950603,299.95808871)(822.30950684,300.2580896)
\curveto(822.21950624,300.47808819)(822.17450628,300.74308793)(822.17450684,301.0530896)
\lineto(822.17450684,301.3680896)
\curveto(822.18450627,301.41808725)(822.18950627,301.4680872)(822.18950684,301.5180896)
\lineto(822.21950684,301.6980896)
\lineto(822.33950684,302.0280896)
\curveto(822.37950608,302.13808653)(822.42950603,302.23808643)(822.48950684,302.3280896)
\curveto(822.66950579,302.61808605)(822.91450554,302.83308584)(823.22450684,302.9730896)
\curveto(823.53450492,303.11308556)(823.87450458,303.23808543)(824.24450684,303.3480896)
\curveto(824.38450407,303.38808528)(824.52950393,303.41808525)(824.67950684,303.4380896)
\curveto(824.82950363,303.45808521)(824.97950348,303.48308519)(825.12950684,303.5130896)
\curveto(825.19950326,303.53308514)(825.26450319,303.54308513)(825.32450684,303.5430896)
\curveto(825.39450306,303.54308513)(825.46950299,303.55308512)(825.54950684,303.5730896)
\curveto(825.61950284,303.59308508)(825.68950277,303.60308507)(825.75950684,303.6030896)
\curveto(825.82950263,303.61308506)(825.90450255,303.62808504)(825.98450684,303.6480896)
\curveto(826.23450222,303.70808496)(826.46950199,303.75808491)(826.68950684,303.7980896)
\curveto(826.90950155,303.84808482)(827.08450137,303.96308471)(827.21450684,304.1430896)
\curveto(827.27450118,304.22308445)(827.32450113,304.32308435)(827.36450684,304.4430896)
\curveto(827.40450105,304.5730841)(827.40450105,304.71308396)(827.36450684,304.8630896)
\curveto(827.30450115,305.10308357)(827.21450124,305.29308338)(827.09450684,305.4330896)
\curveto(826.98450147,305.5730831)(826.82450163,305.68308299)(826.61450684,305.7630896)
\curveto(826.49450196,305.81308286)(826.34950211,305.84808282)(826.17950684,305.8680896)
\curveto(826.01950244,305.88808278)(825.84950261,305.89808277)(825.66950684,305.8980896)
\curveto(825.48950297,305.89808277)(825.31450314,305.88808278)(825.14450684,305.8680896)
\curveto(824.97450348,305.84808282)(824.82950363,305.81808285)(824.70950684,305.7780896)
\curveto(824.53950392,305.71808295)(824.37450408,305.63308304)(824.21450684,305.5230896)
\curveto(824.13450432,305.46308321)(824.0595044,305.38308329)(823.98950684,305.2830896)
\curveto(823.92950453,305.19308348)(823.87450458,305.09308358)(823.82450684,304.9830896)
\curveto(823.79450466,304.90308377)(823.76450469,304.81808385)(823.73450684,304.7280896)
\curveto(823.71450474,304.63808403)(823.66950479,304.5680841)(823.59950684,304.5180896)
\curveto(823.5595049,304.48808418)(823.48950497,304.46308421)(823.38950684,304.4430896)
\curveto(823.29950516,304.43308424)(823.20450525,304.42808424)(823.10450684,304.4280896)
\curveto(823.00450545,304.42808424)(822.90450555,304.43308424)(822.80450684,304.4430896)
\curveto(822.71450574,304.46308421)(822.64950581,304.48808418)(822.60950684,304.5180896)
\curveto(822.56950589,304.54808412)(822.53950592,304.59808407)(822.51950684,304.6680896)
\curveto(822.49950596,304.73808393)(822.49950596,304.81308386)(822.51950684,304.8930896)
\curveto(822.54950591,305.02308365)(822.57950588,305.14308353)(822.60950684,305.2530896)
\curveto(822.64950581,305.3730833)(822.69450576,305.48808318)(822.74450684,305.5980896)
\curveto(822.93450552,305.94808272)(823.17450528,306.21808245)(823.46450684,306.4080896)
\curveto(823.7545047,306.60808206)(824.11450434,306.7680819)(824.54450684,306.8880896)
\curveto(824.64450381,306.90808176)(824.74450371,306.92308175)(824.84450684,306.9330896)
\curveto(824.9545035,306.94308173)(825.06450339,306.95808171)(825.17450684,306.9780896)
\curveto(825.21450324,306.98808168)(825.27950318,306.98808168)(825.36950684,306.9780896)
\curveto(825.459503,306.97808169)(825.51450294,306.98808168)(825.53450684,307.0080896)
\curveto(826.23450222,307.01808165)(826.84450161,306.93808173)(827.36450684,306.7680896)
\curveto(827.88450057,306.59808207)(828.24950021,306.2730824)(828.45950684,305.7930896)
\curveto(828.54949991,305.59308308)(828.59949986,305.35808331)(828.60950684,305.0880896)
\curveto(828.62949983,304.82808384)(828.63949982,304.55308412)(828.63950684,304.2630896)
\lineto(828.63950684,300.9480896)
\curveto(828.63949982,300.80808786)(828.64449981,300.673088)(828.65450684,300.5430896)
\curveto(828.66449979,300.41308826)(828.69449976,300.30808836)(828.74450684,300.2280896)
\curveto(828.79449966,300.15808851)(828.8594996,300.10808856)(828.93950684,300.0780896)
\curveto(829.02949943,300.03808863)(829.11449934,300.00808866)(829.19450684,299.9880896)
\curveto(829.27449918,299.97808869)(829.33449912,299.93308874)(829.37450684,299.8530896)
\curveto(829.39449906,299.82308885)(829.40449905,299.79308888)(829.40450684,299.7630896)
\curveto(829.40449905,299.73308894)(829.40949905,299.69308898)(829.41950684,299.6430896)
\moveto(827.27450684,301.3080896)
\curveto(827.33450112,301.44808722)(827.36450109,301.60808706)(827.36450684,301.7880896)
\curveto(827.37450108,301.97808669)(827.37950108,302.1730865)(827.37950684,302.3730896)
\curveto(827.37950108,302.48308619)(827.37450108,302.58308609)(827.36450684,302.6730896)
\curveto(827.3545011,302.76308591)(827.31450114,302.83308584)(827.24450684,302.8830896)
\curveto(827.21450124,302.90308577)(827.14450131,302.91308576)(827.03450684,302.9130896)
\curveto(827.01450144,302.89308578)(826.97950148,302.88308579)(826.92950684,302.8830896)
\curveto(826.87950158,302.88308579)(826.83450162,302.8730858)(826.79450684,302.8530896)
\curveto(826.71450174,302.83308584)(826.62450183,302.81308586)(826.52450684,302.7930896)
\lineto(826.22450684,302.7330896)
\curveto(826.19450226,302.73308594)(826.1595023,302.72808594)(826.11950684,302.7180896)
\lineto(826.01450684,302.7180896)
\curveto(825.86450259,302.67808599)(825.69950276,302.65308602)(825.51950684,302.6430896)
\curveto(825.34950311,302.64308603)(825.18950327,302.62308605)(825.03950684,302.5830896)
\curveto(824.9595035,302.56308611)(824.88450357,302.54308613)(824.81450684,302.5230896)
\curveto(824.7545037,302.51308616)(824.68450377,302.49808617)(824.60450684,302.4780896)
\curveto(824.44450401,302.42808624)(824.29450416,302.36308631)(824.15450684,302.2830896)
\curveto(824.01450444,302.21308646)(823.89450456,302.12308655)(823.79450684,302.0130896)
\curveto(823.69450476,301.90308677)(823.61950484,301.7680869)(823.56950684,301.6080896)
\curveto(823.51950494,301.45808721)(823.49950496,301.2730874)(823.50950684,301.0530896)
\curveto(823.50950495,300.95308772)(823.52450493,300.85808781)(823.55450684,300.7680896)
\curveto(823.59450486,300.68808798)(823.63950482,300.61308806)(823.68950684,300.5430896)
\curveto(823.76950469,300.43308824)(823.87450458,300.33808833)(824.00450684,300.2580896)
\curveto(824.13450432,300.18808848)(824.27450418,300.12808854)(824.42450684,300.0780896)
\curveto(824.47450398,300.0680886)(824.52450393,300.06308861)(824.57450684,300.0630896)
\curveto(824.62450383,300.06308861)(824.67450378,300.05808861)(824.72450684,300.0480896)
\curveto(824.79450366,300.02808864)(824.87950358,300.01308866)(824.97950684,300.0030896)
\curveto(825.08950337,300.00308867)(825.17950328,300.01308866)(825.24950684,300.0330896)
\curveto(825.30950315,300.05308862)(825.36950309,300.05808861)(825.42950684,300.0480896)
\curveto(825.48950297,300.04808862)(825.54950291,300.05808861)(825.60950684,300.0780896)
\curveto(825.68950277,300.09808857)(825.76450269,300.11308856)(825.83450684,300.1230896)
\curveto(825.91450254,300.13308854)(825.98950247,300.15308852)(826.05950684,300.1830896)
\curveto(826.34950211,300.30308837)(826.59450186,300.44808822)(826.79450684,300.6180896)
\curveto(827.00450145,300.78808788)(827.16450129,301.01808765)(827.27450684,301.3080896)
}
}
{
\newrgbcolor{curcolor}{0 0 0}
\pscustom[linestyle=none,fillstyle=solid,fillcolor=curcolor]
{
\newpath
\moveto(837.55114746,299.8980896)
\lineto(837.55114746,299.5080896)
\curveto(837.55113959,299.38808928)(837.52613961,299.28808938)(837.47614746,299.2080896)
\curveto(837.42613971,299.13808953)(837.3411398,299.09808957)(837.22114746,299.0880896)
\lineto(836.87614746,299.0880896)
\curveto(836.81614032,299.08808958)(836.75614038,299.08308959)(836.69614746,299.0730896)
\curveto(836.64614049,299.0730896)(836.60114054,299.08308959)(836.56114746,299.1030896)
\curveto(836.47114067,299.12308955)(836.41114073,299.16308951)(836.38114746,299.2230896)
\curveto(836.3411408,299.2730894)(836.31614082,299.33308934)(836.30614746,299.4030896)
\curveto(836.30614083,299.4730892)(836.29114085,299.54308913)(836.26114746,299.6130896)
\curveto(836.25114089,299.63308904)(836.2361409,299.64808902)(836.21614746,299.6580896)
\curveto(836.20614093,299.67808899)(836.19114095,299.69808897)(836.17114746,299.7180896)
\curveto(836.07114107,299.72808894)(835.99114115,299.70808896)(835.93114746,299.6580896)
\curveto(835.88114126,299.60808906)(835.82614131,299.55808911)(835.76614746,299.5080896)
\curveto(835.56614157,299.35808931)(835.36614177,299.24308943)(835.16614746,299.1630896)
\curveto(834.98614215,299.08308959)(834.77614236,299.02308965)(834.53614746,298.9830896)
\curveto(834.30614283,298.94308973)(834.06614307,298.92308975)(833.81614746,298.9230896)
\curveto(833.57614356,298.91308976)(833.3361438,298.92808974)(833.09614746,298.9680896)
\curveto(832.85614428,298.99808967)(832.64614449,299.05308962)(832.46614746,299.1330896)
\curveto(831.94614519,299.35308932)(831.52614561,299.64808902)(831.20614746,300.0180896)
\curveto(830.88614625,300.39808827)(830.6361465,300.8680878)(830.45614746,301.4280896)
\curveto(830.41614672,301.51808715)(830.38614675,301.60808706)(830.36614746,301.6980896)
\curveto(830.35614678,301.79808687)(830.3361468,301.89808677)(830.30614746,301.9980896)
\curveto(830.29614684,302.04808662)(830.29114685,302.09808657)(830.29114746,302.1480896)
\curveto(830.29114685,302.19808647)(830.28614685,302.24808642)(830.27614746,302.2980896)
\curveto(830.25614688,302.34808632)(830.24614689,302.39808627)(830.24614746,302.4480896)
\curveto(830.25614688,302.50808616)(830.25614688,302.56308611)(830.24614746,302.6130896)
\lineto(830.24614746,302.7630896)
\curveto(830.22614691,302.81308586)(830.21614692,302.87808579)(830.21614746,302.9580896)
\curveto(830.21614692,303.03808563)(830.22614691,303.10308557)(830.24614746,303.1530896)
\lineto(830.24614746,303.3180896)
\curveto(830.26614687,303.38808528)(830.27114687,303.45808521)(830.26114746,303.5280896)
\curveto(830.26114688,303.60808506)(830.27114687,303.68308499)(830.29114746,303.7530896)
\curveto(830.30114684,303.80308487)(830.30614683,303.84808482)(830.30614746,303.8880896)
\curveto(830.30614683,303.92808474)(830.31114683,303.9730847)(830.32114746,304.0230896)
\curveto(830.35114679,304.12308455)(830.37614676,304.21808445)(830.39614746,304.3080896)
\curveto(830.41614672,304.40808426)(830.4411467,304.50308417)(830.47114746,304.5930896)
\curveto(830.60114654,304.9730837)(830.76614637,305.31308336)(830.96614746,305.6130896)
\curveto(831.17614596,305.92308275)(831.42614571,306.17808249)(831.71614746,306.3780896)
\curveto(831.88614525,306.49808217)(832.06114508,306.59808207)(832.24114746,306.6780896)
\curveto(832.43114471,306.75808191)(832.6361445,306.82808184)(832.85614746,306.8880896)
\curveto(832.92614421,306.89808177)(832.99114415,306.90808176)(833.05114746,306.9180896)
\curveto(833.12114402,306.92808174)(833.19114395,306.94308173)(833.26114746,306.9630896)
\lineto(833.41114746,306.9630896)
\curveto(833.49114365,306.98308169)(833.60614353,306.99308168)(833.75614746,306.9930896)
\curveto(833.91614322,306.99308168)(834.0361431,306.98308169)(834.11614746,306.9630896)
\curveto(834.15614298,306.95308172)(834.21114293,306.94808172)(834.28114746,306.9480896)
\curveto(834.39114275,306.91808175)(834.50114264,306.89308178)(834.61114746,306.8730896)
\curveto(834.72114242,306.86308181)(834.82614231,306.83308184)(834.92614746,306.7830896)
\curveto(835.07614206,306.72308195)(835.21614192,306.65808201)(835.34614746,306.5880896)
\curveto(835.48614165,306.51808215)(835.61614152,306.43808223)(835.73614746,306.3480896)
\curveto(835.79614134,306.29808237)(835.85614128,306.24308243)(835.91614746,306.1830896)
\curveto(835.98614115,306.13308254)(836.07614106,306.11808255)(836.18614746,306.1380896)
\curveto(836.20614093,306.1680825)(836.22114092,306.19308248)(836.23114746,306.2130896)
\curveto(836.25114089,306.23308244)(836.26614087,306.26308241)(836.27614746,306.3030896)
\curveto(836.30614083,306.39308228)(836.31614082,306.50808216)(836.30614746,306.6480896)
\lineto(836.30614746,307.0230896)
\lineto(836.30614746,308.7480896)
\lineto(836.30614746,309.2130896)
\curveto(836.30614083,309.39307928)(836.33114081,309.52307915)(836.38114746,309.6030896)
\curveto(836.42114072,309.673079)(836.48114066,309.71807895)(836.56114746,309.7380896)
\curveto(836.58114056,309.73807893)(836.60614053,309.73807893)(836.63614746,309.7380896)
\curveto(836.66614047,309.74807892)(836.69114045,309.75307892)(836.71114746,309.7530896)
\curveto(836.85114029,309.76307891)(836.99614014,309.76307891)(837.14614746,309.7530896)
\curveto(837.30613983,309.75307892)(837.41613972,309.71307896)(837.47614746,309.6330896)
\curveto(837.52613961,309.55307912)(837.55113959,309.45307922)(837.55114746,309.3330896)
\lineto(837.55114746,308.9580896)
\lineto(837.55114746,299.8980896)
\moveto(836.33614746,302.7330896)
\curveto(836.35614078,302.78308589)(836.36614077,302.84808582)(836.36614746,302.9280896)
\curveto(836.36614077,303.01808565)(836.35614078,303.08808558)(836.33614746,303.1380896)
\lineto(836.33614746,303.3630896)
\curveto(836.31614082,303.45308522)(836.30114084,303.54308513)(836.29114746,303.6330896)
\curveto(836.28114086,303.73308494)(836.26114088,303.82308485)(836.23114746,303.9030896)
\curveto(836.21114093,303.98308469)(836.19114095,304.05808461)(836.17114746,304.1280896)
\curveto(836.16114098,304.19808447)(836.141141,304.2680844)(836.11114746,304.3380896)
\curveto(835.99114115,304.63808403)(835.8361413,304.90308377)(835.64614746,305.1330896)
\curveto(835.45614168,305.36308331)(835.21614192,305.54308313)(834.92614746,305.6730896)
\curveto(834.82614231,305.72308295)(834.72114242,305.75808291)(834.61114746,305.7780896)
\curveto(834.51114263,305.80808286)(834.40114274,305.83308284)(834.28114746,305.8530896)
\curveto(834.20114294,305.8730828)(834.11114303,305.88308279)(834.01114746,305.8830896)
\lineto(833.74114746,305.8830896)
\curveto(833.69114345,305.8730828)(833.64614349,305.86308281)(833.60614746,305.8530896)
\lineto(833.47114746,305.8530896)
\curveto(833.39114375,305.83308284)(833.30614383,305.81308286)(833.21614746,305.7930896)
\curveto(833.136144,305.7730829)(833.05614408,305.74808292)(832.97614746,305.7180896)
\curveto(832.65614448,305.57808309)(832.39614474,305.3730833)(832.19614746,305.1030896)
\curveto(832.00614513,304.84308383)(831.85114529,304.53808413)(831.73114746,304.1880896)
\curveto(831.69114545,304.07808459)(831.66114548,303.96308471)(831.64114746,303.8430896)
\curveto(831.63114551,303.73308494)(831.61614552,303.62308505)(831.59614746,303.5130896)
\curveto(831.59614554,303.4730852)(831.59114555,303.43308524)(831.58114746,303.3930896)
\lineto(831.58114746,303.2880896)
\curveto(831.56114558,303.23808543)(831.55114559,303.18308549)(831.55114746,303.1230896)
\curveto(831.56114558,303.06308561)(831.56614557,303.00808566)(831.56614746,302.9580896)
\lineto(831.56614746,302.6280896)
\curveto(831.56614557,302.52808614)(831.57614556,302.43308624)(831.59614746,302.3430896)
\curveto(831.60614553,302.31308636)(831.61114553,302.26308641)(831.61114746,302.1930896)
\curveto(831.63114551,302.12308655)(831.64614549,302.05308662)(831.65614746,301.9830896)
\lineto(831.71614746,301.7730896)
\curveto(831.82614531,301.42308725)(831.97614516,301.12308755)(832.16614746,300.8730896)
\curveto(832.35614478,300.62308805)(832.59614454,300.41808825)(832.88614746,300.2580896)
\curveto(832.97614416,300.20808846)(833.06614407,300.1680885)(833.15614746,300.1380896)
\curveto(833.24614389,300.10808856)(833.34614379,300.07808859)(833.45614746,300.0480896)
\curveto(833.50614363,300.02808864)(833.55614358,300.02308865)(833.60614746,300.0330896)
\curveto(833.66614347,300.04308863)(833.72114342,300.03808863)(833.77114746,300.0180896)
\curveto(833.81114333,300.00808866)(833.85114329,300.00308867)(833.89114746,300.0030896)
\lineto(834.02614746,300.0030896)
\lineto(834.16114746,300.0030896)
\curveto(834.19114295,300.01308866)(834.2411429,300.01808865)(834.31114746,300.0180896)
\curveto(834.39114275,300.03808863)(834.47114267,300.05308862)(834.55114746,300.0630896)
\curveto(834.63114251,300.08308859)(834.70614243,300.10808856)(834.77614746,300.1380896)
\curveto(835.10614203,300.27808839)(835.37114177,300.45308822)(835.57114746,300.6630896)
\curveto(835.78114136,300.88308779)(835.95614118,301.15808751)(836.09614746,301.4880896)
\curveto(836.14614099,301.59808707)(836.18114096,301.70808696)(836.20114746,301.8180896)
\curveto(836.22114092,301.92808674)(836.24614089,302.03808663)(836.27614746,302.1480896)
\curveto(836.29614084,302.18808648)(836.30614083,302.22308645)(836.30614746,302.2530896)
\curveto(836.30614083,302.29308638)(836.31114083,302.33308634)(836.32114746,302.3730896)
\curveto(836.33114081,302.43308624)(836.33114081,302.49308618)(836.32114746,302.5530896)
\curveto(836.32114082,302.61308606)(836.32614081,302.673086)(836.33614746,302.7330896)
}
}
{
\newrgbcolor{curcolor}{0.80000001 0.80000001 0.80000001}
\pscustom[linestyle=none,fillstyle=solid,fillcolor=curcolor]
{
\newpath
\moveto(738.9732666,379.02397156)
\lineto(753.9732666,379.02397156)
\lineto(753.9732666,364.02397156)
\lineto(738.9732666,364.02397156)
\closepath
}
}
{
\newrgbcolor{curcolor}{0.7019608 0.7019608 0.7019608}
\pscustom[linestyle=none,fillstyle=solid,fillcolor=curcolor]
{
\newpath
\moveto(738.9732666,355.69264984)
\lineto(753.9732666,355.69264984)
\lineto(753.9732666,340.69264984)
\lineto(738.9732666,340.69264984)
\closepath
}
}
{
\newrgbcolor{curcolor}{0.60000002 0.60000002 0.60000002}
\pscustom[linestyle=none,fillstyle=solid,fillcolor=curcolor]
{
\newpath
\moveto(738.9732666,332.37634277)
\lineto(753.9732666,332.37634277)
\lineto(753.9732666,317.37634277)
\lineto(738.9732666,317.37634277)
\closepath
}
}
{
\newrgbcolor{curcolor}{0.50196081 0.50196081 0.50196081}
\pscustom[linestyle=none,fillstyle=solid,fillcolor=curcolor]
{
\newpath
\moveto(738.9732666,309.78309631)
\lineto(753.9732666,309.78309631)
\lineto(753.9732666,294.78309631)
\lineto(738.9732666,294.78309631)
\closepath
}
}
{
\newrgbcolor{curcolor}{0.80000001 0.80000001 0.80000001}
\pscustom[linestyle=none,fillstyle=solid,fillcolor=curcolor]
{
\newpath
\moveto(559.94793701,263.31411743)
\lineto(573.96645641,263.31411743)
\lineto(573.96645641,83.08279419)
\lineto(559.94793701,83.08279419)
\closepath
}
}
{
\newrgbcolor{curcolor}{0.7019608 0.7019608 0.7019608}
\pscustom[linestyle=none,fillstyle=solid,fillcolor=curcolor]
{
\newpath
\moveto(573.94274902,242.45448303)
\lineto(587.96126842,242.45448303)
\lineto(587.96126842,83.08280945)
\lineto(573.94274902,83.08280945)
\closepath
}
}
{
\newrgbcolor{curcolor}{0.60000002 0.60000002 0.60000002}
\pscustom[linestyle=none,fillstyle=solid,fillcolor=curcolor]
{
\newpath
\moveto(587.93756104,367.08204651)
\lineto(601.95608044,367.08204651)
\lineto(601.95608044,83.08280945)
\lineto(587.93756104,83.08280945)
\closepath
}
}
{
\newrgbcolor{curcolor}{0.50196081 0.50196081 0.50196081}
\pscustom[linestyle=none,fillstyle=solid,fillcolor=curcolor]
{
\newpath
\moveto(601.93237305,138.50976562)
\lineto(615.95089245,138.50976562)
\lineto(615.95089245,83.08277893)
\lineto(601.93237305,83.08277893)
\closepath
}
}
{
\newrgbcolor{curcolor}{0.7019608 0.7019608 0.7019608}
\pscustom[linestyle=none,fillstyle=solid,fillcolor=curcolor]
{
\newpath
\moveto(664.96801758,88.08184814)
\lineto(678.98653698,88.08184814)
\lineto(678.98653698,83.08279419)
\lineto(664.96801758,83.08279419)
\closepath
}
}
{
\newrgbcolor{curcolor}{0.60000002 0.60000002 0.60000002}
\pscustom[linestyle=none,fillstyle=solid,fillcolor=curcolor]
{
\newpath
\moveto(678.96282959,112.08184814)
\lineto(692.98134899,112.08184814)
\lineto(692.98134899,83.08279419)
\lineto(678.96282959,83.08279419)
\closepath
}
}
{
\newrgbcolor{curcolor}{0.80000001 0.80000001 0.80000001}
\pscustom[linestyle=none,fillstyle=solid,fillcolor=curcolor]
{
\newpath
\moveto(467.95080566,162.08184814)
\lineto(481.96932507,162.08184814)
\lineto(481.96932507,83.08279419)
\lineto(467.95080566,83.08279419)
\closepath
}
}
{
\newrgbcolor{curcolor}{0.60000002 0.60000002 0.60000002}
\pscustom[linestyle=none,fillstyle=solid,fillcolor=curcolor]
{
\newpath
\moveto(495.94042969,96.95684814)
\lineto(509.95894909,96.95684814)
\lineto(509.95894909,83.08279419)
\lineto(495.94042969,83.08279419)
\closepath
}
}
{
\newrgbcolor{curcolor}{0.50196081 0.50196081 0.50196081}
\pscustom[linestyle=none,fillstyle=solid,fillcolor=curcolor]
{
\newpath
\moveto(509.9352417,90.95684814)
\lineto(523.9537611,90.95684814)
\lineto(523.9537611,83.08279419)
\lineto(509.9352417,83.08279419)
\closepath
}
}
{
\newrgbcolor{curcolor}{0.80000001 0.80000001 0.80000001}
\pscustom[linestyle=none,fillstyle=solid,fillcolor=curcolor]
{
\newpath
\moveto(376.91079712,94.0062561)
\lineto(390.92931652,94.0062561)
\lineto(390.92931652,83.08280373)
\lineto(376.91079712,83.08280373)
\closepath
}
}
{
\newrgbcolor{curcolor}{0.60000002 0.60000002 0.60000002}
\pscustom[linestyle=none,fillstyle=solid,fillcolor=curcolor]
{
\newpath
\moveto(404.90042114,100.06085205)
\lineto(418.91894054,100.06085205)
\lineto(418.91894054,83.082798)
\lineto(404.90042114,83.082798)
\closepath
}
}
{
\newrgbcolor{curcolor}{0.80000001 0.80000001 0.80000001}
\pscustom[linestyle=none,fillstyle=solid,fillcolor=curcolor]
{
\newpath
\moveto(285.94403076,103.06604004)
\lineto(299.96255016,103.06604004)
\lineto(299.96255016,83.08278275)
\lineto(285.94403076,83.08278275)
\closepath
}
}
{
\newrgbcolor{curcolor}{0.60000002 0.60000002 0.60000002}
\pscustom[linestyle=none,fillstyle=solid,fillcolor=curcolor]
{
\newpath
\moveto(313.93365479,118.00369263)
\lineto(327.95217419,118.00369263)
\lineto(327.95217419,83.08280563)
\lineto(313.93365479,83.08280563)
\closepath
}
}
{
\newrgbcolor{curcolor}{0.80000001 0.80000001 0.80000001}
\pscustom[linestyle=none,fillstyle=solid,fillcolor=curcolor]
{
\newpath
\moveto(193.84335327,90.95684814)
\lineto(207.86187267,90.95684814)
\lineto(207.86187267,83.08279419)
\lineto(193.84335327,83.08279419)
\closepath
}
}
{
\newrgbcolor{curcolor}{0.7019608 0.7019608 0.7019608}
\pscustom[linestyle=none,fillstyle=solid,fillcolor=curcolor]
{
\newpath
\moveto(207.83816528,112.08166504)
\lineto(221.85668468,112.08166504)
\lineto(221.85668468,83.0827961)
\lineto(207.83816528,83.0827961)
\closepath
}
}
{
\newrgbcolor{curcolor}{0.60000002 0.60000002 0.60000002}
\pscustom[linestyle=none,fillstyle=solid,fillcolor=curcolor]
{
\newpath
\moveto(221.83297729,253.94497681)
\lineto(235.8514967,253.94497681)
\lineto(235.8514967,83.08280945)
\lineto(221.83297729,83.08280945)
\closepath
}
}
{
\newrgbcolor{curcolor}{0.80000001 0.80000001 0.80000001}
\pscustom[linestyle=none,fillstyle=solid,fillcolor=curcolor]
{
\newpath
\moveto(102.9801178,88.01934814)
\lineto(116.9986372,88.01934814)
\lineto(116.9986372,83.08279419)
\lineto(102.9801178,83.08279419)
\closepath
}
}
{
\newrgbcolor{curcolor}{0.60000002 0.60000002 0.60000002}
\pscustom[linestyle=none,fillstyle=solid,fillcolor=curcolor]
{
\newpath
\moveto(130.96974182,156.01934814)
\lineto(144.98826122,156.01934814)
\lineto(144.98826122,83.08279419)
\lineto(130.96974182,83.08279419)
\closepath
}
}
{
\newrgbcolor{curcolor}{0.50196081 0.50196081 0.50196081}
\pscustom[linestyle=none,fillstyle=solid,fillcolor=curcolor]
{
\newpath
\moveto(144.96455383,85.08184814)
\lineto(158.98307323,85.08184814)
\lineto(158.98307323,83.08279419)
\lineto(144.96455383,83.08279419)
\closepath
}
}
\end{pspicture}

\caption{Diagrama de barras de las valoraciones de los usuarios clasificados por
rol}
\label{usuarios_bars_2}
\end{figure}

En la figura \ref{usuarios_pie_2}, se muestran las gráficas circulares para cada
una de las valoraciones del sistema; puede notarse como el indicador de
sociabilidad es el mas homogéneo, lo que augura una conectividad mas que
deseable para los usuarios del sistema.

\begin{figure}
\centering
%LaTeX with PSTricks extensions
%%Creator: inkscape 0.48.5
%%Please note this file requires PSTricks extensions
\psset{xunit=.5pt,yunit=.5pt,runit=.5pt}
\begin{pspicture}(930,715)
{
\newrgbcolor{curcolor}{0 0 0}
\pscustom[linestyle=none,fillstyle=solid,fillcolor=curcolor]
{
\newpath
\moveto(13.54683094,699.52799194)
\lineto(14.83683094,699.52799194)
\curveto(14.94682812,699.52798126)(15.05182801,699.52298127)(15.15183094,699.51299194)
\curveto(15.25182781,699.51298128)(15.32682774,699.47798131)(15.37683094,699.40799194)
\curveto(15.42682764,699.33798145)(15.45182761,699.24798154)(15.45183094,699.13799194)
\curveto(15.4618276,699.02798176)(15.4668276,698.90798188)(15.46683094,698.77799194)
\lineto(15.46683094,697.47299194)
\lineto(15.46683094,692.26799194)
\lineto(15.46683094,689.80799194)
\lineto(15.46683094,689.37299194)
\curveto(15.47682759,689.21299158)(15.45682761,689.0929917)(15.40683094,689.01299194)
\curveto(15.3668277,688.94299185)(15.27682779,688.8879919)(15.13683094,688.84799194)
\curveto(15.066828,688.82799196)(14.99182807,688.82299197)(14.91183094,688.83299194)
\curveto(14.83182823,688.84299195)(14.75182831,688.84799194)(14.67183094,688.84799194)
\lineto(13.78683094,688.84799194)
\curveto(13.67682939,688.84799194)(13.57182949,688.85299194)(13.47183094,688.86299194)
\curveto(13.38182968,688.87299192)(13.30682976,688.90299189)(13.24683094,688.95299194)
\curveto(13.19682987,689.00299179)(13.1668299,689.07799171)(13.15683094,689.17799194)
\curveto(13.14682992,689.27799151)(13.14182992,689.38299141)(13.14183094,689.49299194)
\lineto(13.14183094,690.79799194)
\lineto(13.14183094,696.27299194)
\lineto(13.14183094,698.46299194)
\curveto(13.14182992,698.60298219)(13.13682993,698.76798202)(13.12683094,698.95799194)
\curveto(13.12682994,699.14798164)(13.15182991,699.28298151)(13.20183094,699.36299194)
\curveto(13.24182982,699.42298137)(13.30682976,699.47298132)(13.39683094,699.51299194)
\curveto(13.42682964,699.51298128)(13.45182961,699.51298128)(13.47183094,699.51299194)
\curveto(13.50182956,699.52298127)(13.52682954,699.52798126)(13.54683094,699.52799194)
}
}
{
\newrgbcolor{curcolor}{0 0 0}
\pscustom[linestyle=none,fillstyle=solid,fillcolor=curcolor]
{
\newpath
\moveto(21.75065907,696.78299194)
\curveto(22.35065326,696.80298399)(22.85065276,696.71798407)(23.25065907,696.52799194)
\curveto(23.65065196,696.33798445)(23.96565165,696.05798473)(24.19565907,695.68799194)
\curveto(24.26565135,695.57798521)(24.32065129,695.45798533)(24.36065907,695.32799194)
\curveto(24.40065121,695.20798558)(24.44065117,695.08298571)(24.48065907,694.95299194)
\curveto(24.50065111,694.87298592)(24.5106511,694.79798599)(24.51065907,694.72799194)
\curveto(24.52065109,694.65798613)(24.53565108,694.5879862)(24.55565907,694.51799194)
\curveto(24.55565106,694.45798633)(24.56065105,694.41798637)(24.57065907,694.39799194)
\curveto(24.59065102,694.25798653)(24.60065101,694.11298668)(24.60065907,693.96299194)
\lineto(24.60065907,693.52799194)
\lineto(24.60065907,692.19299194)
\lineto(24.60065907,689.76299194)
\curveto(24.60065101,689.57299122)(24.59565102,689.3879914)(24.58565907,689.20799194)
\curveto(24.58565103,689.03799175)(24.5156511,688.92799186)(24.37565907,688.87799194)
\curveto(24.3156513,688.85799193)(24.24565137,688.84799194)(24.16565907,688.84799194)
\lineto(23.92565907,688.84799194)
\lineto(23.11565907,688.84799194)
\curveto(22.99565262,688.84799194)(22.88565273,688.85299194)(22.78565907,688.86299194)
\curveto(22.69565292,688.88299191)(22.62565299,688.92799186)(22.57565907,688.99799194)
\curveto(22.53565308,689.05799173)(22.5106531,689.13299166)(22.50065907,689.22299194)
\lineto(22.50065907,689.53799194)
\lineto(22.50065907,690.58799194)
\lineto(22.50065907,692.82299194)
\curveto(22.50065311,693.1929876)(22.48565313,693.53298726)(22.45565907,693.84299194)
\curveto(22.42565319,694.16298663)(22.33565328,694.43298636)(22.18565907,694.65299194)
\curveto(22.04565357,694.85298594)(21.84065377,694.9929858)(21.57065907,695.07299194)
\curveto(21.52065409,695.0929857)(21.46565415,695.10298569)(21.40565907,695.10299194)
\curveto(21.35565426,695.10298569)(21.30065431,695.11298568)(21.24065907,695.13299194)
\curveto(21.19065442,695.14298565)(21.12565449,695.14298565)(21.04565907,695.13299194)
\curveto(20.97565464,695.13298566)(20.92065469,695.12798566)(20.88065907,695.11799194)
\curveto(20.84065477,695.10798568)(20.80565481,695.10298569)(20.77565907,695.10299194)
\curveto(20.74565487,695.10298569)(20.7156549,695.09798569)(20.68565907,695.08799194)
\curveto(20.45565516,695.02798576)(20.27065534,694.94798584)(20.13065907,694.84799194)
\curveto(19.8106558,694.61798617)(19.62065599,694.28298651)(19.56065907,693.84299194)
\curveto(19.50065611,693.40298739)(19.47065614,692.90798788)(19.47065907,692.35799194)
\lineto(19.47065907,690.48299194)
\lineto(19.47065907,689.56799194)
\lineto(19.47065907,689.29799194)
\curveto(19.47065614,689.20799158)(19.45565616,689.13299166)(19.42565907,689.07299194)
\curveto(19.37565624,688.96299183)(19.29565632,688.89799189)(19.18565907,688.87799194)
\curveto(19.07565654,688.85799193)(18.94065667,688.84799194)(18.78065907,688.84799194)
\lineto(18.03065907,688.84799194)
\curveto(17.92065769,688.84799194)(17.8106578,688.85299194)(17.70065907,688.86299194)
\curveto(17.59065802,688.87299192)(17.5106581,688.90799188)(17.46065907,688.96799194)
\curveto(17.39065822,689.05799173)(17.35565826,689.1879916)(17.35565907,689.35799194)
\curveto(17.36565825,689.52799126)(17.37065824,689.6879911)(17.37065907,689.83799194)
\lineto(17.37065907,691.87799194)
\lineto(17.37065907,695.17799194)
\lineto(17.37065907,695.94299194)
\lineto(17.37065907,696.24299194)
\curveto(17.38065823,696.33298446)(17.4106582,696.40798438)(17.46065907,696.46799194)
\curveto(17.48065813,696.49798429)(17.5106581,696.51798427)(17.55065907,696.52799194)
\curveto(17.60065801,696.54798424)(17.65065796,696.56298423)(17.70065907,696.57299194)
\lineto(17.77565907,696.57299194)
\curveto(17.82565779,696.58298421)(17.87565774,696.5879842)(17.92565907,696.58799194)
\lineto(18.09065907,696.58799194)
\lineto(18.72065907,696.58799194)
\curveto(18.80065681,696.5879842)(18.87565674,696.58298421)(18.94565907,696.57299194)
\curveto(19.02565659,696.57298422)(19.09565652,696.56298423)(19.15565907,696.54299194)
\curveto(19.22565639,696.51298428)(19.27065634,696.46798432)(19.29065907,696.40799194)
\curveto(19.32065629,696.34798444)(19.34565627,696.27798451)(19.36565907,696.19799194)
\curveto(19.37565624,696.15798463)(19.37565624,696.12298467)(19.36565907,696.09299194)
\curveto(19.36565625,696.06298473)(19.37565624,696.03298476)(19.39565907,696.00299194)
\curveto(19.4156562,695.95298484)(19.43065618,695.92298487)(19.44065907,695.91299194)
\curveto(19.46065615,695.90298489)(19.48565613,695.8879849)(19.51565907,695.86799194)
\curveto(19.62565599,695.85798493)(19.7156559,695.8929849)(19.78565907,695.97299194)
\curveto(19.85565576,696.06298473)(19.93065568,696.13298466)(20.01065907,696.18299194)
\curveto(20.28065533,696.38298441)(20.58065503,696.54298425)(20.91065907,696.66299194)
\curveto(21.00065461,696.6929841)(21.09065452,696.71298408)(21.18065907,696.72299194)
\curveto(21.28065433,696.73298406)(21.38565423,696.74798404)(21.49565907,696.76799194)
\curveto(21.52565409,696.77798401)(21.57065404,696.77798401)(21.63065907,696.76799194)
\curveto(21.69065392,696.76798402)(21.73065388,696.77298402)(21.75065907,696.78299194)
}
}
{
\newrgbcolor{curcolor}{0 0 0}
\pscustom[linestyle=none,fillstyle=solid,fillcolor=curcolor]
{
\newpath
\moveto(33.82190907,689.70299194)
\lineto(33.82190907,689.28299194)
\curveto(33.8219007,689.15299164)(33.79190073,689.04799174)(33.73190907,688.96799194)
\curveto(33.68190084,688.91799187)(33.6169009,688.88299191)(33.53690907,688.86299194)
\curveto(33.45690106,688.85299194)(33.36690115,688.84799194)(33.26690907,688.84799194)
\lineto(32.44190907,688.84799194)
\lineto(32.15690907,688.84799194)
\curveto(32.07690244,688.85799193)(32.01190251,688.88299191)(31.96190907,688.92299194)
\curveto(31.89190263,688.97299182)(31.85190267,689.03799175)(31.84190907,689.11799194)
\curveto(31.83190269,689.19799159)(31.81190271,689.27799151)(31.78190907,689.35799194)
\curveto(31.76190276,689.37799141)(31.74190278,689.3929914)(31.72190907,689.40299194)
\curveto(31.71190281,689.42299137)(31.69690282,689.44299135)(31.67690907,689.46299194)
\curveto(31.56690295,689.46299133)(31.48690303,689.43799135)(31.43690907,689.38799194)
\lineto(31.28690907,689.23799194)
\curveto(31.2169033,689.1879916)(31.15190337,689.14299165)(31.09190907,689.10299194)
\curveto(31.03190349,689.07299172)(30.96690355,689.03299176)(30.89690907,688.98299194)
\curveto(30.85690366,688.96299183)(30.81190371,688.94299185)(30.76190907,688.92299194)
\curveto(30.7219038,688.90299189)(30.67690384,688.88299191)(30.62690907,688.86299194)
\curveto(30.48690403,688.81299198)(30.33690418,688.76799202)(30.17690907,688.72799194)
\curveto(30.12690439,688.70799208)(30.08190444,688.69799209)(30.04190907,688.69799194)
\curveto(30.00190452,688.69799209)(29.96190456,688.6929921)(29.92190907,688.68299194)
\lineto(29.78690907,688.68299194)
\curveto(29.75690476,688.67299212)(29.7169048,688.66799212)(29.66690907,688.66799194)
\lineto(29.53190907,688.66799194)
\curveto(29.47190505,688.64799214)(29.38190514,688.64299215)(29.26190907,688.65299194)
\curveto(29.14190538,688.65299214)(29.05690546,688.66299213)(29.00690907,688.68299194)
\curveto(28.93690558,688.70299209)(28.87190565,688.71299208)(28.81190907,688.71299194)
\curveto(28.76190576,688.70299209)(28.70690581,688.70799208)(28.64690907,688.72799194)
\lineto(28.28690907,688.84799194)
\curveto(28.17690634,688.87799191)(28.06690645,688.91799187)(27.95690907,688.96799194)
\curveto(27.60690691,689.11799167)(27.29190723,689.34799144)(27.01190907,689.65799194)
\curveto(26.74190778,689.97799081)(26.52690799,690.31299048)(26.36690907,690.66299194)
\curveto(26.3169082,690.77299002)(26.27690824,690.87798991)(26.24690907,690.97799194)
\curveto(26.2169083,691.0879897)(26.18190834,691.19798959)(26.14190907,691.30799194)
\curveto(26.13190839,691.34798944)(26.12690839,691.38298941)(26.12690907,691.41299194)
\curveto(26.12690839,691.45298934)(26.1169084,691.49798929)(26.09690907,691.54799194)
\curveto(26.07690844,691.62798916)(26.05690846,691.71298908)(26.03690907,691.80299194)
\curveto(26.02690849,691.90298889)(26.01190851,692.00298879)(25.99190907,692.10299194)
\curveto(25.98190854,692.13298866)(25.97690854,692.16798862)(25.97690907,692.20799194)
\curveto(25.98690853,692.24798854)(25.98690853,692.28298851)(25.97690907,692.31299194)
\lineto(25.97690907,692.44799194)
\curveto(25.97690854,692.49798829)(25.97190855,692.54798824)(25.96190907,692.59799194)
\curveto(25.95190857,692.64798814)(25.94690857,692.70298809)(25.94690907,692.76299194)
\curveto(25.94690857,692.83298796)(25.95190857,692.8879879)(25.96190907,692.92799194)
\curveto(25.97190855,692.97798781)(25.97690854,693.02298777)(25.97690907,693.06299194)
\lineto(25.97690907,693.21299194)
\curveto(25.98690853,693.26298753)(25.98690853,693.30798748)(25.97690907,693.34799194)
\curveto(25.97690854,693.39798739)(25.98690853,693.44798734)(26.00690907,693.49799194)
\curveto(26.02690849,693.60798718)(26.04190848,693.71298708)(26.05190907,693.81299194)
\curveto(26.07190845,693.91298688)(26.09690842,694.01298678)(26.12690907,694.11299194)
\curveto(26.16690835,694.23298656)(26.20190832,694.34798644)(26.23190907,694.45799194)
\curveto(26.26190826,694.56798622)(26.30190822,694.67798611)(26.35190907,694.78799194)
\curveto(26.49190803,695.0879857)(26.66690785,695.37298542)(26.87690907,695.64299194)
\curveto(26.89690762,695.67298512)(26.9219076,695.69798509)(26.95190907,695.71799194)
\curveto(26.99190753,695.74798504)(27.0219075,695.77798501)(27.04190907,695.80799194)
\curveto(27.08190744,695.85798493)(27.1219074,695.90298489)(27.16190907,695.94299194)
\curveto(27.20190732,695.98298481)(27.24690727,696.02298477)(27.29690907,696.06299194)
\curveto(27.33690718,696.08298471)(27.37190715,696.10798468)(27.40190907,696.13799194)
\curveto(27.43190709,696.17798461)(27.46690705,696.20798458)(27.50690907,696.22799194)
\curveto(27.75690676,696.39798439)(28.04690647,696.53798425)(28.37690907,696.64799194)
\curveto(28.44690607,696.66798412)(28.516906,696.68298411)(28.58690907,696.69299194)
\curveto(28.66690585,696.70298409)(28.74690577,696.71798407)(28.82690907,696.73799194)
\curveto(28.89690562,696.75798403)(28.98690553,696.76798402)(29.09690907,696.76799194)
\curveto(29.20690531,696.77798401)(29.3169052,696.78298401)(29.42690907,696.78299194)
\curveto(29.53690498,696.78298401)(29.64190488,696.77798401)(29.74190907,696.76799194)
\curveto(29.85190467,696.75798403)(29.94190458,696.74298405)(30.01190907,696.72299194)
\curveto(30.16190436,696.67298412)(30.30690421,696.62798416)(30.44690907,696.58799194)
\curveto(30.58690393,696.54798424)(30.7169038,696.4929843)(30.83690907,696.42299194)
\curveto(30.90690361,696.37298442)(30.97190355,696.32298447)(31.03190907,696.27299194)
\curveto(31.09190343,696.23298456)(31.15690336,696.1879846)(31.22690907,696.13799194)
\curveto(31.26690325,696.10798468)(31.3219032,696.06798472)(31.39190907,696.01799194)
\curveto(31.47190305,695.96798482)(31.54690297,695.96798482)(31.61690907,696.01799194)
\curveto(31.65690286,696.03798475)(31.67690284,696.07298472)(31.67690907,696.12299194)
\curveto(31.67690284,696.17298462)(31.68690283,696.22298457)(31.70690907,696.27299194)
\lineto(31.70690907,696.42299194)
\curveto(31.7169028,696.45298434)(31.7219028,696.4879843)(31.72190907,696.52799194)
\lineto(31.72190907,696.64799194)
\lineto(31.72190907,698.68799194)
\curveto(31.7219028,698.79798199)(31.7169028,698.91798187)(31.70690907,699.04799194)
\curveto(31.70690281,699.1879816)(31.73190279,699.2929815)(31.78190907,699.36299194)
\curveto(31.8219027,699.44298135)(31.89690262,699.4929813)(32.00690907,699.51299194)
\curveto(32.02690249,699.52298127)(32.04690247,699.52298127)(32.06690907,699.51299194)
\curveto(32.08690243,699.51298128)(32.10690241,699.51798127)(32.12690907,699.52799194)
\lineto(33.19190907,699.52799194)
\curveto(33.31190121,699.52798126)(33.4219011,699.52298127)(33.52190907,699.51299194)
\curveto(33.6219009,699.50298129)(33.69690082,699.46298133)(33.74690907,699.39299194)
\curveto(33.79690072,699.31298148)(33.8219007,699.20798158)(33.82190907,699.07799194)
\lineto(33.82190907,698.71799194)
\lineto(33.82190907,689.70299194)
\moveto(31.78190907,692.64299194)
\curveto(31.79190273,692.68298811)(31.79190273,692.72298807)(31.78190907,692.76299194)
\lineto(31.78190907,692.89799194)
\curveto(31.78190274,692.99798779)(31.77690274,693.09798769)(31.76690907,693.19799194)
\curveto(31.75690276,693.29798749)(31.74190278,693.3879874)(31.72190907,693.46799194)
\curveto(31.70190282,693.57798721)(31.68190284,693.67798711)(31.66190907,693.76799194)
\curveto(31.65190287,693.85798693)(31.62690289,693.94298685)(31.58690907,694.02299194)
\curveto(31.44690307,694.38298641)(31.24190328,694.66798612)(30.97190907,694.87799194)
\curveto(30.71190381,695.0879857)(30.33190419,695.1929856)(29.83190907,695.19299194)
\curveto(29.77190475,695.1929856)(29.69190483,695.18298561)(29.59190907,695.16299194)
\curveto(29.51190501,695.14298565)(29.43690508,695.12298567)(29.36690907,695.10299194)
\curveto(29.30690521,695.0929857)(29.24690527,695.07298572)(29.18690907,695.04299194)
\curveto(28.9169056,694.93298586)(28.70690581,694.76298603)(28.55690907,694.53299194)
\curveto(28.40690611,694.30298649)(28.28690623,694.04298675)(28.19690907,693.75299194)
\curveto(28.16690635,693.65298714)(28.14690637,693.55298724)(28.13690907,693.45299194)
\curveto(28.12690639,693.35298744)(28.10690641,693.24798754)(28.07690907,693.13799194)
\lineto(28.07690907,692.92799194)
\curveto(28.05690646,692.83798795)(28.05190647,692.71298808)(28.06190907,692.55299194)
\curveto(28.07190645,692.40298839)(28.08690643,692.2929885)(28.10690907,692.22299194)
\lineto(28.10690907,692.13299194)
\curveto(28.1169064,692.11298868)(28.1219064,692.0929887)(28.12190907,692.07299194)
\curveto(28.14190638,691.9929888)(28.15690636,691.91798887)(28.16690907,691.84799194)
\curveto(28.18690633,691.77798901)(28.20690631,691.70298909)(28.22690907,691.62299194)
\curveto(28.39690612,691.10298969)(28.68690583,690.71799007)(29.09690907,690.46799194)
\curveto(29.22690529,690.37799041)(29.40690511,690.30799048)(29.63690907,690.25799194)
\curveto(29.67690484,690.24799054)(29.73690478,690.24299055)(29.81690907,690.24299194)
\curveto(29.84690467,690.23299056)(29.89190463,690.22299057)(29.95190907,690.21299194)
\curveto(30.0219045,690.21299058)(30.07690444,690.21799057)(30.11690907,690.22799194)
\curveto(30.19690432,690.24799054)(30.27690424,690.26299053)(30.35690907,690.27299194)
\curveto(30.43690408,690.28299051)(30.516904,690.30299049)(30.59690907,690.33299194)
\curveto(30.84690367,690.44299035)(31.04690347,690.58299021)(31.19690907,690.75299194)
\curveto(31.34690317,690.92298987)(31.47690304,691.13798965)(31.58690907,691.39799194)
\curveto(31.62690289,691.4879893)(31.65690286,691.57798921)(31.67690907,691.66799194)
\curveto(31.69690282,691.76798902)(31.7169028,691.87298892)(31.73690907,691.98299194)
\curveto(31.74690277,692.03298876)(31.74690277,692.07798871)(31.73690907,692.11799194)
\curveto(31.73690278,692.16798862)(31.74690277,692.21798857)(31.76690907,692.26799194)
\curveto(31.77690274,692.29798849)(31.78190274,692.33298846)(31.78190907,692.37299194)
\lineto(31.78190907,692.50799194)
\lineto(31.78190907,692.64299194)
}
}
{
\newrgbcolor{curcolor}{0 0 0}
\pscustom[linestyle=none,fillstyle=solid,fillcolor=curcolor]
{
\newpath
\moveto(37.50183094,699.43799194)
\curveto(37.57182799,699.35798143)(37.60682796,699.23798155)(37.60683094,699.07799194)
\lineto(37.60683094,698.61299194)
\lineto(37.60683094,698.20799194)
\curveto(37.60682796,698.06798272)(37.57182799,697.97298282)(37.50183094,697.92299194)
\curveto(37.44182812,697.87298292)(37.3618282,697.84298295)(37.26183094,697.83299194)
\curveto(37.17182839,697.82298297)(37.07182849,697.81798297)(36.96183094,697.81799194)
\lineto(36.12183094,697.81799194)
\curveto(36.01182955,697.81798297)(35.91182965,697.82298297)(35.82183094,697.83299194)
\curveto(35.74182982,697.84298295)(35.67182989,697.87298292)(35.61183094,697.92299194)
\curveto(35.57182999,697.95298284)(35.54183002,698.00798278)(35.52183094,698.08799194)
\curveto(35.51183005,698.17798261)(35.50183006,698.27298252)(35.49183094,698.37299194)
\lineto(35.49183094,698.70299194)
\curveto(35.50183006,698.81298198)(35.50683006,698.90798188)(35.50683094,698.98799194)
\lineto(35.50683094,699.19799194)
\curveto(35.51683005,699.26798152)(35.53683003,699.32798146)(35.56683094,699.37799194)
\curveto(35.58682998,699.41798137)(35.61182995,699.44798134)(35.64183094,699.46799194)
\lineto(35.76183094,699.52799194)
\curveto(35.78182978,699.52798126)(35.80682976,699.52798126)(35.83683094,699.52799194)
\curveto(35.8668297,699.53798125)(35.89182967,699.54298125)(35.91183094,699.54299194)
\lineto(37.00683094,699.54299194)
\curveto(37.10682846,699.54298125)(37.20182836,699.53798125)(37.29183094,699.52799194)
\curveto(37.38182818,699.51798127)(37.45182811,699.4879813)(37.50183094,699.43799194)
\moveto(37.60683094,689.67299194)
\curveto(37.60682796,689.47299132)(37.60182796,689.30299149)(37.59183094,689.16299194)
\curveto(37.58182798,689.02299177)(37.49182807,688.92799186)(37.32183094,688.87799194)
\curveto(37.2618283,688.85799193)(37.19682837,688.84799194)(37.12683094,688.84799194)
\curveto(37.05682851,688.85799193)(36.98182858,688.86299193)(36.90183094,688.86299194)
\lineto(36.06183094,688.86299194)
\curveto(35.97182959,688.86299193)(35.88182968,688.86799192)(35.79183094,688.87799194)
\curveto(35.71182985,688.8879919)(35.65182991,688.91799187)(35.61183094,688.96799194)
\curveto(35.55183001,689.03799175)(35.51683005,689.12299167)(35.50683094,689.22299194)
\lineto(35.50683094,689.56799194)
\lineto(35.50683094,695.89799194)
\lineto(35.50683094,696.19799194)
\curveto(35.50683006,696.29798449)(35.52683004,696.37798441)(35.56683094,696.43799194)
\curveto(35.62682994,696.50798428)(35.71182985,696.55298424)(35.82183094,696.57299194)
\curveto(35.84182972,696.58298421)(35.8668297,696.58298421)(35.89683094,696.57299194)
\curveto(35.93682963,696.57298422)(35.9668296,696.57798421)(35.98683094,696.58799194)
\lineto(36.73683094,696.58799194)
\lineto(36.93183094,696.58799194)
\curveto(37.01182855,696.59798419)(37.07682849,696.59798419)(37.12683094,696.58799194)
\lineto(37.24683094,696.58799194)
\curveto(37.30682826,696.56798422)(37.3618282,696.55298424)(37.41183094,696.54299194)
\curveto(37.4618281,696.53298426)(37.50182806,696.50298429)(37.53183094,696.45299194)
\curveto(37.57182799,696.40298439)(37.59182797,696.33298446)(37.59183094,696.24299194)
\curveto(37.60182796,696.15298464)(37.60682796,696.05798473)(37.60683094,695.95799194)
\lineto(37.60683094,689.67299194)
}
}
{
\newrgbcolor{curcolor}{0 0 0}
\pscustom[linestyle=none,fillstyle=solid,fillcolor=curcolor]
{
\newpath
\moveto(42.83901844,696.79799194)
\curveto(43.64901328,696.81798397)(44.32401261,696.69798409)(44.86401844,696.43799194)
\curveto(45.41401152,696.17798461)(45.84901108,695.80798498)(46.16901844,695.32799194)
\curveto(46.3290106,695.0879857)(46.44901048,694.81298598)(46.52901844,694.50299194)
\curveto(46.54901038,694.45298634)(46.56401037,694.3879864)(46.57401844,694.30799194)
\curveto(46.59401034,694.22798656)(46.59401034,694.15798663)(46.57401844,694.09799194)
\curveto(46.5340104,693.9879868)(46.46401047,693.92298687)(46.36401844,693.90299194)
\curveto(46.26401067,693.8929869)(46.14401079,693.8879869)(46.00401844,693.88799194)
\lineto(45.22401844,693.88799194)
\lineto(44.93901844,693.88799194)
\curveto(44.84901208,693.8879869)(44.77401216,693.90798688)(44.71401844,693.94799194)
\curveto(44.6340123,693.9879868)(44.57901235,694.04798674)(44.54901844,694.12799194)
\curveto(44.51901241,694.21798657)(44.47901245,694.30798648)(44.42901844,694.39799194)
\curveto(44.36901256,694.50798628)(44.30401263,694.60798618)(44.23401844,694.69799194)
\curveto(44.16401277,694.787986)(44.08401285,694.86798592)(43.99401844,694.93799194)
\curveto(43.85401308,695.02798576)(43.69901323,695.09798569)(43.52901844,695.14799194)
\curveto(43.46901346,695.16798562)(43.40901352,695.17798561)(43.34901844,695.17799194)
\curveto(43.28901364,695.17798561)(43.2340137,695.1879856)(43.18401844,695.20799194)
\lineto(43.03401844,695.20799194)
\curveto(42.8340141,695.20798558)(42.67401426,695.1879856)(42.55401844,695.14799194)
\curveto(42.26401467,695.05798573)(42.0290149,694.91798587)(41.84901844,694.72799194)
\curveto(41.66901526,694.54798624)(41.52401541,694.32798646)(41.41401844,694.06799194)
\curveto(41.36401557,693.95798683)(41.32401561,693.83798695)(41.29401844,693.70799194)
\curveto(41.27401566,693.5879872)(41.24901568,693.45798733)(41.21901844,693.31799194)
\curveto(41.20901572,693.27798751)(41.20401573,693.23798755)(41.20401844,693.19799194)
\curveto(41.20401573,693.15798763)(41.19901573,693.11798767)(41.18901844,693.07799194)
\curveto(41.16901576,692.97798781)(41.15901577,692.83798795)(41.15901844,692.65799194)
\curveto(41.16901576,692.47798831)(41.18401575,692.33798845)(41.20401844,692.23799194)
\curveto(41.20401573,692.15798863)(41.20901572,692.10298869)(41.21901844,692.07299194)
\curveto(41.23901569,692.00298879)(41.24901568,691.93298886)(41.24901844,691.86299194)
\curveto(41.25901567,691.792989)(41.27401566,691.72298907)(41.29401844,691.65299194)
\curveto(41.37401556,691.42298937)(41.46901546,691.21298958)(41.57901844,691.02299194)
\curveto(41.68901524,690.83298996)(41.8290151,690.67299012)(41.99901844,690.54299194)
\curveto(42.03901489,690.51299028)(42.09901483,690.47799031)(42.17901844,690.43799194)
\curveto(42.28901464,690.36799042)(42.39901453,690.32299047)(42.50901844,690.30299194)
\curveto(42.6290143,690.28299051)(42.77401416,690.26299053)(42.94401844,690.24299194)
\lineto(43.03401844,690.24299194)
\curveto(43.07401386,690.24299055)(43.10401383,690.24799054)(43.12401844,690.25799194)
\lineto(43.25901844,690.25799194)
\curveto(43.3290136,690.27799051)(43.39401354,690.2929905)(43.45401844,690.30299194)
\curveto(43.52401341,690.32299047)(43.58901334,690.34299045)(43.64901844,690.36299194)
\curveto(43.94901298,690.4929903)(44.17901275,690.68299011)(44.33901844,690.93299194)
\curveto(44.37901255,690.98298981)(44.41401252,691.03798975)(44.44401844,691.09799194)
\curveto(44.47401246,691.16798962)(44.49901243,691.22798956)(44.51901844,691.27799194)
\curveto(44.55901237,691.3879894)(44.59401234,691.48298931)(44.62401844,691.56299194)
\curveto(44.65401228,691.65298914)(44.72401221,691.72298907)(44.83401844,691.77299194)
\curveto(44.92401201,691.81298898)(45.06901186,691.82798896)(45.26901844,691.81799194)
\lineto(45.76401844,691.81799194)
\lineto(45.97401844,691.81799194)
\curveto(46.05401088,691.82798896)(46.11901081,691.82298897)(46.16901844,691.80299194)
\lineto(46.28901844,691.80299194)
\lineto(46.40901844,691.77299194)
\curveto(46.44901048,691.77298902)(46.47901045,691.76298903)(46.49901844,691.74299194)
\curveto(46.54901038,691.70298909)(46.57901035,691.64298915)(46.58901844,691.56299194)
\curveto(46.60901032,691.4929893)(46.60901032,691.41798937)(46.58901844,691.33799194)
\curveto(46.49901043,691.00798978)(46.38901054,690.71299008)(46.25901844,690.45299194)
\curveto(45.84901108,689.68299111)(45.19401174,689.14799164)(44.29401844,688.84799194)
\curveto(44.19401274,688.81799197)(44.08901284,688.79799199)(43.97901844,688.78799194)
\curveto(43.86901306,688.76799202)(43.75901317,688.74299205)(43.64901844,688.71299194)
\curveto(43.58901334,688.70299209)(43.5290134,688.69799209)(43.46901844,688.69799194)
\curveto(43.40901352,688.69799209)(43.34901358,688.6929921)(43.28901844,688.68299194)
\lineto(43.12401844,688.68299194)
\curveto(43.07401386,688.66299213)(42.99901393,688.65799213)(42.89901844,688.66799194)
\curveto(42.79901413,688.66799212)(42.72401421,688.67299212)(42.67401844,688.68299194)
\curveto(42.59401434,688.70299209)(42.51901441,688.71299208)(42.44901844,688.71299194)
\curveto(42.38901454,688.70299209)(42.32401461,688.70799208)(42.25401844,688.72799194)
\lineto(42.10401844,688.75799194)
\curveto(42.05401488,688.75799203)(42.00401493,688.76299203)(41.95401844,688.77299194)
\curveto(41.84401509,688.80299199)(41.73901519,688.83299196)(41.63901844,688.86299194)
\curveto(41.53901539,688.8929919)(41.44401549,688.92799186)(41.35401844,688.96799194)
\curveto(40.88401605,689.16799162)(40.48901644,689.42299137)(40.16901844,689.73299194)
\curveto(39.84901708,690.05299074)(39.58901734,690.44799034)(39.38901844,690.91799194)
\curveto(39.33901759,691.00798978)(39.29901763,691.10298969)(39.26901844,691.20299194)
\lineto(39.17901844,691.53299194)
\curveto(39.16901776,691.57298922)(39.16401777,691.60798918)(39.16401844,691.63799194)
\curveto(39.16401777,691.67798911)(39.15401778,691.72298907)(39.13401844,691.77299194)
\curveto(39.11401782,691.84298895)(39.10401783,691.91298888)(39.10401844,691.98299194)
\curveto(39.10401783,692.06298873)(39.09401784,692.13798865)(39.07401844,692.20799194)
\lineto(39.07401844,692.46299194)
\curveto(39.05401788,692.51298828)(39.04401789,692.56798822)(39.04401844,692.62799194)
\curveto(39.04401789,692.69798809)(39.05401788,692.75798803)(39.07401844,692.80799194)
\curveto(39.08401785,692.85798793)(39.08401785,692.90298789)(39.07401844,692.94299194)
\curveto(39.06401787,692.98298781)(39.06401787,693.02298777)(39.07401844,693.06299194)
\curveto(39.09401784,693.13298766)(39.09901783,693.19798759)(39.08901844,693.25799194)
\curveto(39.08901784,693.31798747)(39.09901783,693.37798741)(39.11901844,693.43799194)
\curveto(39.16901776,693.61798717)(39.20901772,693.787987)(39.23901844,693.94799194)
\curveto(39.26901766,694.11798667)(39.31401762,694.28298651)(39.37401844,694.44299194)
\curveto(39.59401734,694.95298584)(39.86901706,695.37798541)(40.19901844,695.71799194)
\curveto(40.53901639,696.05798473)(40.96901596,696.33298446)(41.48901844,696.54299194)
\curveto(41.6290153,696.60298419)(41.77401516,696.64298415)(41.92401844,696.66299194)
\curveto(42.07401486,696.6929841)(42.2290147,696.72798406)(42.38901844,696.76799194)
\curveto(42.46901446,696.77798401)(42.54401439,696.78298401)(42.61401844,696.78299194)
\curveto(42.68401425,696.78298401)(42.75901417,696.787984)(42.83901844,696.79799194)
}
}
{
\newrgbcolor{curcolor}{0 0 0}
\pscustom[linestyle=none,fillstyle=solid,fillcolor=curcolor]
{
\newpath
\moveto(54.93229969,689.44799194)
\curveto(54.95229184,689.33799145)(54.96229183,689.22799156)(54.96229969,689.11799194)
\curveto(54.97229182,689.00799178)(54.92229187,688.93299186)(54.81229969,688.89299194)
\curveto(54.75229204,688.86299193)(54.68229211,688.84799194)(54.60229969,688.84799194)
\lineto(54.36229969,688.84799194)
\lineto(53.55229969,688.84799194)
\lineto(53.28229969,688.84799194)
\curveto(53.20229359,688.85799193)(53.13729366,688.88299191)(53.08729969,688.92299194)
\curveto(53.01729378,688.96299183)(52.96229383,689.01799177)(52.92229969,689.08799194)
\curveto(52.8922939,689.16799162)(52.84729395,689.23299156)(52.78729969,689.28299194)
\curveto(52.76729403,689.30299149)(52.74229405,689.31799147)(52.71229969,689.32799194)
\curveto(52.68229411,689.34799144)(52.64229415,689.35299144)(52.59229969,689.34299194)
\curveto(52.54229425,689.32299147)(52.4922943,689.29799149)(52.44229969,689.26799194)
\curveto(52.40229439,689.23799155)(52.35729444,689.21299158)(52.30729969,689.19299194)
\curveto(52.25729454,689.15299164)(52.20229459,689.11799167)(52.14229969,689.08799194)
\lineto(51.96229969,688.99799194)
\curveto(51.83229496,688.93799185)(51.6972951,688.8879919)(51.55729969,688.84799194)
\curveto(51.41729538,688.81799197)(51.27229552,688.78299201)(51.12229969,688.74299194)
\curveto(51.05229574,688.72299207)(50.98229581,688.71299208)(50.91229969,688.71299194)
\curveto(50.85229594,688.70299209)(50.78729601,688.6929921)(50.71729969,688.68299194)
\lineto(50.62729969,688.68299194)
\curveto(50.5972962,688.67299212)(50.56729623,688.66799212)(50.53729969,688.66799194)
\lineto(50.37229969,688.66799194)
\curveto(50.27229652,688.64799214)(50.17229662,688.64799214)(50.07229969,688.66799194)
\lineto(49.93729969,688.66799194)
\curveto(49.86729693,688.6879921)(49.797297,688.69799209)(49.72729969,688.69799194)
\curveto(49.66729713,688.6879921)(49.60729719,688.6929921)(49.54729969,688.71299194)
\curveto(49.44729735,688.73299206)(49.35229744,688.75299204)(49.26229969,688.77299194)
\curveto(49.17229762,688.78299201)(49.08729771,688.80799198)(49.00729969,688.84799194)
\curveto(48.71729808,688.95799183)(48.46729833,689.09799169)(48.25729969,689.26799194)
\curveto(48.05729874,689.44799134)(47.8972989,689.68299111)(47.77729969,689.97299194)
\curveto(47.74729905,690.04299075)(47.71729908,690.11799067)(47.68729969,690.19799194)
\curveto(47.66729913,690.27799051)(47.64729915,690.36299043)(47.62729969,690.45299194)
\curveto(47.60729919,690.50299029)(47.5972992,690.55299024)(47.59729969,690.60299194)
\curveto(47.60729919,690.65299014)(47.60729919,690.70299009)(47.59729969,690.75299194)
\curveto(47.58729921,690.78299001)(47.57729922,690.84298995)(47.56729969,690.93299194)
\curveto(47.56729923,691.03298976)(47.57229922,691.10298969)(47.58229969,691.14299194)
\curveto(47.60229919,691.24298955)(47.61229918,691.32798946)(47.61229969,691.39799194)
\lineto(47.70229969,691.72799194)
\curveto(47.73229906,691.84798894)(47.77229902,691.95298884)(47.82229969,692.04299194)
\curveto(47.9922988,692.33298846)(48.18729861,692.55298824)(48.40729969,692.70299194)
\curveto(48.62729817,692.85298794)(48.90729789,692.98298781)(49.24729969,693.09299194)
\curveto(49.37729742,693.14298765)(49.51229728,693.17798761)(49.65229969,693.19799194)
\curveto(49.792297,693.21798757)(49.93229686,693.24298755)(50.07229969,693.27299194)
\curveto(50.15229664,693.2929875)(50.23729656,693.30298749)(50.32729969,693.30299194)
\curveto(50.41729638,693.31298748)(50.50729629,693.32798746)(50.59729969,693.34799194)
\curveto(50.66729613,693.36798742)(50.73729606,693.37298742)(50.80729969,693.36299194)
\curveto(50.87729592,693.36298743)(50.95229584,693.37298742)(51.03229969,693.39299194)
\curveto(51.10229569,693.41298738)(51.17229562,693.42298737)(51.24229969,693.42299194)
\curveto(51.31229548,693.42298737)(51.38729541,693.43298736)(51.46729969,693.45299194)
\curveto(51.67729512,693.50298729)(51.86729493,693.54298725)(52.03729969,693.57299194)
\curveto(52.21729458,693.61298718)(52.37729442,693.70298709)(52.51729969,693.84299194)
\curveto(52.60729419,693.93298686)(52.66729413,694.03298676)(52.69729969,694.14299194)
\curveto(52.70729409,694.17298662)(52.70729409,694.19798659)(52.69729969,694.21799194)
\curveto(52.6972941,694.23798655)(52.70229409,694.25798653)(52.71229969,694.27799194)
\curveto(52.72229407,694.29798649)(52.72729407,694.32798646)(52.72729969,694.36799194)
\lineto(52.72729969,694.45799194)
\lineto(52.69729969,694.57799194)
\curveto(52.6972941,694.61798617)(52.6922941,694.65298614)(52.68229969,694.68299194)
\curveto(52.58229421,694.98298581)(52.37229442,695.1879856)(52.05229969,695.29799194)
\curveto(51.96229483,695.32798546)(51.85229494,695.34798544)(51.72229969,695.35799194)
\curveto(51.60229519,695.37798541)(51.47729532,695.38298541)(51.34729969,695.37299194)
\curveto(51.21729558,695.37298542)(51.0922957,695.36298543)(50.97229969,695.34299194)
\curveto(50.85229594,695.32298547)(50.74729605,695.29798549)(50.65729969,695.26799194)
\curveto(50.5972962,695.24798554)(50.53729626,695.21798557)(50.47729969,695.17799194)
\curveto(50.42729637,695.14798564)(50.37729642,695.11298568)(50.32729969,695.07299194)
\curveto(50.27729652,695.03298576)(50.22229657,694.97798581)(50.16229969,694.90799194)
\curveto(50.11229668,694.83798595)(50.07729672,694.77298602)(50.05729969,694.71299194)
\curveto(50.00729679,694.61298618)(49.96229683,694.51798627)(49.92229969,694.42799194)
\curveto(49.8922969,694.33798645)(49.82229697,694.27798651)(49.71229969,694.24799194)
\curveto(49.63229716,694.22798656)(49.54729725,694.21798657)(49.45729969,694.21799194)
\lineto(49.18729969,694.21799194)
\lineto(48.61729969,694.21799194)
\curveto(48.56729823,694.21798657)(48.51729828,694.21298658)(48.46729969,694.20299194)
\curveto(48.41729838,694.20298659)(48.37229842,694.20798658)(48.33229969,694.21799194)
\lineto(48.19729969,694.21799194)
\curveto(48.17729862,694.22798656)(48.15229864,694.23298656)(48.12229969,694.23299194)
\curveto(48.0922987,694.23298656)(48.06729873,694.24298655)(48.04729969,694.26299194)
\curveto(47.96729883,694.28298651)(47.91229888,694.34798644)(47.88229969,694.45799194)
\curveto(47.87229892,694.50798628)(47.87229892,694.55798623)(47.88229969,694.60799194)
\curveto(47.8922989,694.65798613)(47.90229889,694.70298609)(47.91229969,694.74299194)
\curveto(47.94229885,694.85298594)(47.97229882,694.95298584)(48.00229969,695.04299194)
\curveto(48.04229875,695.14298565)(48.08729871,695.23298556)(48.13729969,695.31299194)
\lineto(48.22729969,695.46299194)
\lineto(48.31729969,695.61299194)
\curveto(48.3972984,695.72298507)(48.4972983,695.82798496)(48.61729969,695.92799194)
\curveto(48.63729816,695.93798485)(48.66729813,695.96298483)(48.70729969,696.00299194)
\curveto(48.75729804,696.04298475)(48.80229799,696.07798471)(48.84229969,696.10799194)
\curveto(48.88229791,696.13798465)(48.92729787,696.16798462)(48.97729969,696.19799194)
\curveto(49.14729765,696.30798448)(49.32729747,696.3929844)(49.51729969,696.45299194)
\curveto(49.70729709,696.52298427)(49.90229689,696.5879842)(50.10229969,696.64799194)
\curveto(50.22229657,696.67798411)(50.34729645,696.69798409)(50.47729969,696.70799194)
\curveto(50.60729619,696.71798407)(50.73729606,696.73798405)(50.86729969,696.76799194)
\curveto(50.90729589,696.77798401)(50.96729583,696.77798401)(51.04729969,696.76799194)
\curveto(51.13729566,696.75798403)(51.1922956,696.76298403)(51.21229969,696.78299194)
\curveto(51.62229517,696.792984)(52.01229478,696.77798401)(52.38229969,696.73799194)
\curveto(52.76229403,696.69798409)(53.10229369,696.62298417)(53.40229969,696.51299194)
\curveto(53.71229308,696.40298439)(53.97729282,696.25298454)(54.19729969,696.06299194)
\curveto(54.41729238,695.88298491)(54.58729221,695.64798514)(54.70729969,695.35799194)
\curveto(54.77729202,695.1879856)(54.81729198,694.9929858)(54.82729969,694.77299194)
\curveto(54.83729196,694.55298624)(54.84229195,694.32798646)(54.84229969,694.09799194)
\lineto(54.84229969,690.75299194)
\lineto(54.84229969,690.16799194)
\curveto(54.84229195,689.97799081)(54.86229193,689.80299099)(54.90229969,689.64299194)
\curveto(54.91229188,689.61299118)(54.91729188,689.57799121)(54.91729969,689.53799194)
\curveto(54.91729188,689.50799128)(54.92229187,689.47799131)(54.93229969,689.44799194)
\moveto(52.72729969,691.75799194)
\curveto(52.73729406,691.80798898)(52.74229405,691.86298893)(52.74229969,691.92299194)
\curveto(52.74229405,691.9929888)(52.73729406,692.05298874)(52.72729969,692.10299194)
\curveto(52.70729409,692.16298863)(52.6972941,692.21798857)(52.69729969,692.26799194)
\curveto(52.6972941,692.31798847)(52.67729412,692.35798843)(52.63729969,692.38799194)
\curveto(52.58729421,692.42798836)(52.51229428,692.44798834)(52.41229969,692.44799194)
\curveto(52.37229442,692.43798835)(52.33729446,692.42798836)(52.30729969,692.41799194)
\curveto(52.27729452,692.41798837)(52.24229455,692.41298838)(52.20229969,692.40299194)
\curveto(52.13229466,692.38298841)(52.05729474,692.36798842)(51.97729969,692.35799194)
\curveto(51.8972949,692.34798844)(51.81729498,692.33298846)(51.73729969,692.31299194)
\curveto(51.70729509,692.30298849)(51.66229513,692.29798849)(51.60229969,692.29799194)
\curveto(51.47229532,692.26798852)(51.34229545,692.24798854)(51.21229969,692.23799194)
\curveto(51.08229571,692.22798856)(50.95729584,692.20298859)(50.83729969,692.16299194)
\curveto(50.75729604,692.14298865)(50.68229611,692.12298867)(50.61229969,692.10299194)
\curveto(50.54229625,692.0929887)(50.47229632,692.07298872)(50.40229969,692.04299194)
\curveto(50.1922966,691.95298884)(50.01229678,691.81798897)(49.86229969,691.63799194)
\curveto(49.72229707,691.45798933)(49.67229712,691.20798958)(49.71229969,690.88799194)
\curveto(49.73229706,690.71799007)(49.78729701,690.57799021)(49.87729969,690.46799194)
\curveto(49.94729685,690.35799043)(50.05229674,690.26799052)(50.19229969,690.19799194)
\curveto(50.33229646,690.13799065)(50.48229631,690.0929907)(50.64229969,690.06299194)
\curveto(50.81229598,690.03299076)(50.98729581,690.02299077)(51.16729969,690.03299194)
\curveto(51.35729544,690.05299074)(51.53229526,690.0879907)(51.69229969,690.13799194)
\curveto(51.95229484,690.21799057)(52.15729464,690.34299045)(52.30729969,690.51299194)
\curveto(52.45729434,690.6929901)(52.57229422,690.91298988)(52.65229969,691.17299194)
\curveto(52.67229412,691.24298955)(52.68229411,691.31298948)(52.68229969,691.38299194)
\curveto(52.6922941,691.46298933)(52.70729409,691.54298925)(52.72729969,691.62299194)
\lineto(52.72729969,691.75799194)
}
}
{
\newrgbcolor{curcolor}{0 0 0}
\pscustom[linestyle=none,fillstyle=solid,fillcolor=curcolor]
{
\newpath
\moveto(64.08558094,689.70299194)
\lineto(64.08558094,689.28299194)
\curveto(64.08557257,689.15299164)(64.0555726,689.04799174)(63.99558094,688.96799194)
\curveto(63.94557271,688.91799187)(63.88057278,688.88299191)(63.80058094,688.86299194)
\curveto(63.72057294,688.85299194)(63.63057303,688.84799194)(63.53058094,688.84799194)
\lineto(62.70558094,688.84799194)
\lineto(62.42058094,688.84799194)
\curveto(62.34057432,688.85799193)(62.27557438,688.88299191)(62.22558094,688.92299194)
\curveto(62.1555745,688.97299182)(62.11557454,689.03799175)(62.10558094,689.11799194)
\curveto(62.09557456,689.19799159)(62.07557458,689.27799151)(62.04558094,689.35799194)
\curveto(62.02557463,689.37799141)(62.00557465,689.3929914)(61.98558094,689.40299194)
\curveto(61.97557468,689.42299137)(61.9605747,689.44299135)(61.94058094,689.46299194)
\curveto(61.83057483,689.46299133)(61.75057491,689.43799135)(61.70058094,689.38799194)
\lineto(61.55058094,689.23799194)
\curveto(61.48057518,689.1879916)(61.41557524,689.14299165)(61.35558094,689.10299194)
\curveto(61.29557536,689.07299172)(61.23057543,689.03299176)(61.16058094,688.98299194)
\curveto(61.12057554,688.96299183)(61.07557558,688.94299185)(61.02558094,688.92299194)
\curveto(60.98557567,688.90299189)(60.94057572,688.88299191)(60.89058094,688.86299194)
\curveto(60.75057591,688.81299198)(60.60057606,688.76799202)(60.44058094,688.72799194)
\curveto(60.39057627,688.70799208)(60.34557631,688.69799209)(60.30558094,688.69799194)
\curveto(60.26557639,688.69799209)(60.22557643,688.6929921)(60.18558094,688.68299194)
\lineto(60.05058094,688.68299194)
\curveto(60.02057664,688.67299212)(59.98057668,688.66799212)(59.93058094,688.66799194)
\lineto(59.79558094,688.66799194)
\curveto(59.73557692,688.64799214)(59.64557701,688.64299215)(59.52558094,688.65299194)
\curveto(59.40557725,688.65299214)(59.32057734,688.66299213)(59.27058094,688.68299194)
\curveto(59.20057746,688.70299209)(59.13557752,688.71299208)(59.07558094,688.71299194)
\curveto(59.02557763,688.70299209)(58.97057769,688.70799208)(58.91058094,688.72799194)
\lineto(58.55058094,688.84799194)
\curveto(58.44057822,688.87799191)(58.33057833,688.91799187)(58.22058094,688.96799194)
\curveto(57.87057879,689.11799167)(57.5555791,689.34799144)(57.27558094,689.65799194)
\curveto(57.00557965,689.97799081)(56.79057987,690.31299048)(56.63058094,690.66299194)
\curveto(56.58058008,690.77299002)(56.54058012,690.87798991)(56.51058094,690.97799194)
\curveto(56.48058018,691.0879897)(56.44558021,691.19798959)(56.40558094,691.30799194)
\curveto(56.39558026,691.34798944)(56.39058027,691.38298941)(56.39058094,691.41299194)
\curveto(56.39058027,691.45298934)(56.38058028,691.49798929)(56.36058094,691.54799194)
\curveto(56.34058032,691.62798916)(56.32058034,691.71298908)(56.30058094,691.80299194)
\curveto(56.29058037,691.90298889)(56.27558038,692.00298879)(56.25558094,692.10299194)
\curveto(56.24558041,692.13298866)(56.24058042,692.16798862)(56.24058094,692.20799194)
\curveto(56.25058041,692.24798854)(56.25058041,692.28298851)(56.24058094,692.31299194)
\lineto(56.24058094,692.44799194)
\curveto(56.24058042,692.49798829)(56.23558042,692.54798824)(56.22558094,692.59799194)
\curveto(56.21558044,692.64798814)(56.21058045,692.70298809)(56.21058094,692.76299194)
\curveto(56.21058045,692.83298796)(56.21558044,692.8879879)(56.22558094,692.92799194)
\curveto(56.23558042,692.97798781)(56.24058042,693.02298777)(56.24058094,693.06299194)
\lineto(56.24058094,693.21299194)
\curveto(56.25058041,693.26298753)(56.25058041,693.30798748)(56.24058094,693.34799194)
\curveto(56.24058042,693.39798739)(56.25058041,693.44798734)(56.27058094,693.49799194)
\curveto(56.29058037,693.60798718)(56.30558035,693.71298708)(56.31558094,693.81299194)
\curveto(56.33558032,693.91298688)(56.3605803,694.01298678)(56.39058094,694.11299194)
\curveto(56.43058023,694.23298656)(56.46558019,694.34798644)(56.49558094,694.45799194)
\curveto(56.52558013,694.56798622)(56.56558009,694.67798611)(56.61558094,694.78799194)
\curveto(56.7555799,695.0879857)(56.93057973,695.37298542)(57.14058094,695.64299194)
\curveto(57.1605795,695.67298512)(57.18557947,695.69798509)(57.21558094,695.71799194)
\curveto(57.2555794,695.74798504)(57.28557937,695.77798501)(57.30558094,695.80799194)
\curveto(57.34557931,695.85798493)(57.38557927,695.90298489)(57.42558094,695.94299194)
\curveto(57.46557919,695.98298481)(57.51057915,696.02298477)(57.56058094,696.06299194)
\curveto(57.60057906,696.08298471)(57.63557902,696.10798468)(57.66558094,696.13799194)
\curveto(57.69557896,696.17798461)(57.73057893,696.20798458)(57.77058094,696.22799194)
\curveto(58.02057864,696.39798439)(58.31057835,696.53798425)(58.64058094,696.64799194)
\curveto(58.71057795,696.66798412)(58.78057788,696.68298411)(58.85058094,696.69299194)
\curveto(58.93057773,696.70298409)(59.01057765,696.71798407)(59.09058094,696.73799194)
\curveto(59.1605775,696.75798403)(59.25057741,696.76798402)(59.36058094,696.76799194)
\curveto(59.47057719,696.77798401)(59.58057708,696.78298401)(59.69058094,696.78299194)
\curveto(59.80057686,696.78298401)(59.90557675,696.77798401)(60.00558094,696.76799194)
\curveto(60.11557654,696.75798403)(60.20557645,696.74298405)(60.27558094,696.72299194)
\curveto(60.42557623,696.67298412)(60.57057609,696.62798416)(60.71058094,696.58799194)
\curveto(60.85057581,696.54798424)(60.98057568,696.4929843)(61.10058094,696.42299194)
\curveto(61.17057549,696.37298442)(61.23557542,696.32298447)(61.29558094,696.27299194)
\curveto(61.3555753,696.23298456)(61.42057524,696.1879846)(61.49058094,696.13799194)
\curveto(61.53057513,696.10798468)(61.58557507,696.06798472)(61.65558094,696.01799194)
\curveto(61.73557492,695.96798482)(61.81057485,695.96798482)(61.88058094,696.01799194)
\curveto(61.92057474,696.03798475)(61.94057472,696.07298472)(61.94058094,696.12299194)
\curveto(61.94057472,696.17298462)(61.95057471,696.22298457)(61.97058094,696.27299194)
\lineto(61.97058094,696.42299194)
\curveto(61.98057468,696.45298434)(61.98557467,696.4879843)(61.98558094,696.52799194)
\lineto(61.98558094,696.64799194)
\lineto(61.98558094,698.68799194)
\curveto(61.98557467,698.79798199)(61.98057468,698.91798187)(61.97058094,699.04799194)
\curveto(61.97057469,699.1879816)(61.99557466,699.2929815)(62.04558094,699.36299194)
\curveto(62.08557457,699.44298135)(62.1605745,699.4929813)(62.27058094,699.51299194)
\curveto(62.29057437,699.52298127)(62.31057435,699.52298127)(62.33058094,699.51299194)
\curveto(62.35057431,699.51298128)(62.37057429,699.51798127)(62.39058094,699.52799194)
\lineto(63.45558094,699.52799194)
\curveto(63.57557308,699.52798126)(63.68557297,699.52298127)(63.78558094,699.51299194)
\curveto(63.88557277,699.50298129)(63.9605727,699.46298133)(64.01058094,699.39299194)
\curveto(64.0605726,699.31298148)(64.08557257,699.20798158)(64.08558094,699.07799194)
\lineto(64.08558094,698.71799194)
\lineto(64.08558094,689.70299194)
\moveto(62.04558094,692.64299194)
\curveto(62.0555746,692.68298811)(62.0555746,692.72298807)(62.04558094,692.76299194)
\lineto(62.04558094,692.89799194)
\curveto(62.04557461,692.99798779)(62.04057462,693.09798769)(62.03058094,693.19799194)
\curveto(62.02057464,693.29798749)(62.00557465,693.3879874)(61.98558094,693.46799194)
\curveto(61.96557469,693.57798721)(61.94557471,693.67798711)(61.92558094,693.76799194)
\curveto(61.91557474,693.85798693)(61.89057477,693.94298685)(61.85058094,694.02299194)
\curveto(61.71057495,694.38298641)(61.50557515,694.66798612)(61.23558094,694.87799194)
\curveto(60.97557568,695.0879857)(60.59557606,695.1929856)(60.09558094,695.19299194)
\curveto(60.03557662,695.1929856)(59.9555767,695.18298561)(59.85558094,695.16299194)
\curveto(59.77557688,695.14298565)(59.70057696,695.12298567)(59.63058094,695.10299194)
\curveto(59.57057709,695.0929857)(59.51057715,695.07298572)(59.45058094,695.04299194)
\curveto(59.18057748,694.93298586)(58.97057769,694.76298603)(58.82058094,694.53299194)
\curveto(58.67057799,694.30298649)(58.55057811,694.04298675)(58.46058094,693.75299194)
\curveto(58.43057823,693.65298714)(58.41057825,693.55298724)(58.40058094,693.45299194)
\curveto(58.39057827,693.35298744)(58.37057829,693.24798754)(58.34058094,693.13799194)
\lineto(58.34058094,692.92799194)
\curveto(58.32057834,692.83798795)(58.31557834,692.71298808)(58.32558094,692.55299194)
\curveto(58.33557832,692.40298839)(58.35057831,692.2929885)(58.37058094,692.22299194)
\lineto(58.37058094,692.13299194)
\curveto(58.38057828,692.11298868)(58.38557827,692.0929887)(58.38558094,692.07299194)
\curveto(58.40557825,691.9929888)(58.42057824,691.91798887)(58.43058094,691.84799194)
\curveto(58.45057821,691.77798901)(58.47057819,691.70298909)(58.49058094,691.62299194)
\curveto(58.660578,691.10298969)(58.95057771,690.71799007)(59.36058094,690.46799194)
\curveto(59.49057717,690.37799041)(59.67057699,690.30799048)(59.90058094,690.25799194)
\curveto(59.94057672,690.24799054)(60.00057666,690.24299055)(60.08058094,690.24299194)
\curveto(60.11057655,690.23299056)(60.1555765,690.22299057)(60.21558094,690.21299194)
\curveto(60.28557637,690.21299058)(60.34057632,690.21799057)(60.38058094,690.22799194)
\curveto(60.4605762,690.24799054)(60.54057612,690.26299053)(60.62058094,690.27299194)
\curveto(60.70057596,690.28299051)(60.78057588,690.30299049)(60.86058094,690.33299194)
\curveto(61.11057555,690.44299035)(61.31057535,690.58299021)(61.46058094,690.75299194)
\curveto(61.61057505,690.92298987)(61.74057492,691.13798965)(61.85058094,691.39799194)
\curveto(61.89057477,691.4879893)(61.92057474,691.57798921)(61.94058094,691.66799194)
\curveto(61.9605747,691.76798902)(61.98057468,691.87298892)(62.00058094,691.98299194)
\curveto(62.01057465,692.03298876)(62.01057465,692.07798871)(62.00058094,692.11799194)
\curveto(62.00057466,692.16798862)(62.01057465,692.21798857)(62.03058094,692.26799194)
\curveto(62.04057462,692.29798849)(62.04557461,692.33298846)(62.04558094,692.37299194)
\lineto(62.04558094,692.50799194)
\lineto(62.04558094,692.64299194)
}
}
{
\newrgbcolor{curcolor}{0 0 0}
\pscustom[linestyle=none,fillstyle=solid,fillcolor=curcolor]
{
\newpath
\moveto(73.43550282,693.03299194)
\curveto(73.45549425,692.97298782)(73.46549424,692.8879879)(73.46550282,692.77799194)
\curveto(73.46549424,692.66798812)(73.45549425,692.58298821)(73.43550282,692.52299194)
\lineto(73.43550282,692.37299194)
\curveto(73.41549429,692.2929885)(73.4054943,692.21298858)(73.40550282,692.13299194)
\curveto(73.41549429,692.05298874)(73.41049429,691.97298882)(73.39050282,691.89299194)
\curveto(73.37049433,691.82298897)(73.35549435,691.75798903)(73.34550282,691.69799194)
\curveto(73.33549437,691.63798915)(73.32549438,691.57298922)(73.31550282,691.50299194)
\curveto(73.27549443,691.3929894)(73.24049446,691.27798951)(73.21050282,691.15799194)
\curveto(73.18049452,691.04798974)(73.14049456,690.94298985)(73.09050282,690.84299194)
\curveto(72.88049482,690.36299043)(72.6054951,689.97299082)(72.26550282,689.67299194)
\curveto(71.92549578,689.37299142)(71.51549619,689.12299167)(71.03550282,688.92299194)
\curveto(70.91549679,688.87299192)(70.79049691,688.83799195)(70.66050282,688.81799194)
\curveto(70.54049716,688.787992)(70.41549729,688.75799203)(70.28550282,688.72799194)
\curveto(70.23549747,688.70799208)(70.18049752,688.69799209)(70.12050282,688.69799194)
\curveto(70.06049764,688.69799209)(70.0054977,688.6929921)(69.95550282,688.68299194)
\lineto(69.85050282,688.68299194)
\curveto(69.82049788,688.67299212)(69.79049791,688.66799212)(69.76050282,688.66799194)
\curveto(69.71049799,688.65799213)(69.63049807,688.65299214)(69.52050282,688.65299194)
\curveto(69.41049829,688.64299215)(69.32549838,688.64799214)(69.26550282,688.66799194)
\lineto(69.11550282,688.66799194)
\curveto(69.06549864,688.67799211)(69.01049869,688.68299211)(68.95050282,688.68299194)
\curveto(68.9004988,688.67299212)(68.85049885,688.67799211)(68.80050282,688.69799194)
\curveto(68.76049894,688.70799208)(68.72049898,688.71299208)(68.68050282,688.71299194)
\curveto(68.65049905,688.71299208)(68.61049909,688.71799207)(68.56050282,688.72799194)
\curveto(68.46049924,688.75799203)(68.36049934,688.78299201)(68.26050282,688.80299194)
\curveto(68.16049954,688.82299197)(68.06549964,688.85299194)(67.97550282,688.89299194)
\curveto(67.85549985,688.93299186)(67.74049996,688.97299182)(67.63050282,689.01299194)
\curveto(67.53050017,689.05299174)(67.42550028,689.10299169)(67.31550282,689.16299194)
\curveto(66.96550074,689.37299142)(66.66550104,689.61799117)(66.41550282,689.89799194)
\curveto(66.16550154,690.17799061)(65.95550175,690.51299028)(65.78550282,690.90299194)
\curveto(65.73550197,690.9929898)(65.69550201,691.0879897)(65.66550282,691.18799194)
\curveto(65.64550206,691.2879895)(65.62050208,691.3929894)(65.59050282,691.50299194)
\curveto(65.57050213,691.55298924)(65.56050214,691.59798919)(65.56050282,691.63799194)
\curveto(65.56050214,691.67798911)(65.55050215,691.72298907)(65.53050282,691.77299194)
\curveto(65.51050219,691.85298894)(65.5005022,691.93298886)(65.50050282,692.01299194)
\curveto(65.5005022,692.10298869)(65.49050221,692.1879886)(65.47050282,692.26799194)
\curveto(65.46050224,692.31798847)(65.45550225,692.36298843)(65.45550282,692.40299194)
\lineto(65.45550282,692.53799194)
\curveto(65.43550227,692.59798819)(65.42550228,692.68298811)(65.42550282,692.79299194)
\curveto(65.43550227,692.90298789)(65.45050225,692.9879878)(65.47050282,693.04799194)
\lineto(65.47050282,693.15299194)
\curveto(65.48050222,693.20298759)(65.48050222,693.25298754)(65.47050282,693.30299194)
\curveto(65.47050223,693.36298743)(65.48050222,693.41798737)(65.50050282,693.46799194)
\curveto(65.51050219,693.51798727)(65.51550219,693.56298723)(65.51550282,693.60299194)
\curveto(65.51550219,693.65298714)(65.52550218,693.70298709)(65.54550282,693.75299194)
\curveto(65.58550212,693.88298691)(65.62050208,694.00798678)(65.65050282,694.12799194)
\curveto(65.68050202,694.25798653)(65.72050198,694.38298641)(65.77050282,694.50299194)
\curveto(65.95050175,694.91298588)(66.16550154,695.25298554)(66.41550282,695.52299194)
\curveto(66.66550104,695.80298499)(66.97050073,696.05798473)(67.33050282,696.28799194)
\curveto(67.43050027,696.33798445)(67.53550017,696.38298441)(67.64550282,696.42299194)
\curveto(67.75549995,696.46298433)(67.86549984,696.50798428)(67.97550282,696.55799194)
\curveto(68.1054996,696.60798418)(68.24049946,696.64298415)(68.38050282,696.66299194)
\curveto(68.52049918,696.68298411)(68.66549904,696.71298408)(68.81550282,696.75299194)
\curveto(68.89549881,696.76298403)(68.97049873,696.76798402)(69.04050282,696.76799194)
\curveto(69.11049859,696.76798402)(69.18049852,696.77298402)(69.25050282,696.78299194)
\curveto(69.83049787,696.792984)(70.33049737,696.73298406)(70.75050282,696.60299194)
\curveto(71.18049652,696.47298432)(71.56049614,696.2929845)(71.89050282,696.06299194)
\curveto(72.0004957,695.98298481)(72.11049559,695.8929849)(72.22050282,695.79299194)
\curveto(72.34049536,695.70298509)(72.44049526,695.60298519)(72.52050282,695.49299194)
\curveto(72.6004951,695.3929854)(72.67049503,695.2929855)(72.73050282,695.19299194)
\curveto(72.8004949,695.0929857)(72.87049483,694.9879858)(72.94050282,694.87799194)
\curveto(73.01049469,694.76798602)(73.06549464,694.64798614)(73.10550282,694.51799194)
\curveto(73.14549456,694.39798639)(73.19049451,694.26798652)(73.24050282,694.12799194)
\curveto(73.27049443,694.04798674)(73.29549441,693.96298683)(73.31550282,693.87299194)
\lineto(73.37550282,693.60299194)
\curveto(73.38549432,693.56298723)(73.39049431,693.52298727)(73.39050282,693.48299194)
\curveto(73.39049431,693.44298735)(73.39549431,693.40298739)(73.40550282,693.36299194)
\curveto(73.42549428,693.31298748)(73.43049427,693.25798753)(73.42050282,693.19799194)
\curveto(73.41049429,693.13798765)(73.41549429,693.08298771)(73.43550282,693.03299194)
\moveto(71.33550282,692.49299194)
\curveto(71.34549636,692.54298825)(71.35049635,692.61298818)(71.35050282,692.70299194)
\curveto(71.35049635,692.80298799)(71.34549636,692.87798791)(71.33550282,692.92799194)
\lineto(71.33550282,693.04799194)
\curveto(71.31549639,693.09798769)(71.3054964,693.15298764)(71.30550282,693.21299194)
\curveto(71.3054964,693.27298752)(71.3004964,693.32798746)(71.29050282,693.37799194)
\curveto(71.29049641,693.41798737)(71.28549642,693.44798734)(71.27550282,693.46799194)
\lineto(71.21550282,693.70799194)
\curveto(71.2054965,693.79798699)(71.18549652,693.88298691)(71.15550282,693.96299194)
\curveto(71.04549666,694.22298657)(70.91549679,694.44298635)(70.76550282,694.62299194)
\curveto(70.61549709,694.81298598)(70.41549729,694.96298583)(70.16550282,695.07299194)
\curveto(70.1054976,695.0929857)(70.04549766,695.10798568)(69.98550282,695.11799194)
\curveto(69.92549778,695.13798565)(69.86049784,695.15798563)(69.79050282,695.17799194)
\curveto(69.71049799,695.19798559)(69.62549808,695.20298559)(69.53550282,695.19299194)
\lineto(69.26550282,695.19299194)
\curveto(69.23549847,695.17298562)(69.2004985,695.16298563)(69.16050282,695.16299194)
\curveto(69.12049858,695.17298562)(69.08549862,695.17298562)(69.05550282,695.16299194)
\lineto(68.84550282,695.10299194)
\curveto(68.78549892,695.0929857)(68.73049897,695.07298572)(68.68050282,695.04299194)
\curveto(68.43049927,694.93298586)(68.22549948,694.77298602)(68.06550282,694.56299194)
\curveto(67.91549979,694.36298643)(67.79549991,694.12798666)(67.70550282,693.85799194)
\curveto(67.67550003,693.75798703)(67.65050005,693.65298714)(67.63050282,693.54299194)
\curveto(67.62050008,693.43298736)(67.6055001,693.32298747)(67.58550282,693.21299194)
\curveto(67.57550013,693.16298763)(67.57050013,693.11298768)(67.57050282,693.06299194)
\lineto(67.57050282,692.91299194)
\curveto(67.55050015,692.84298795)(67.54050016,692.73798805)(67.54050282,692.59799194)
\curveto(67.55050015,692.45798833)(67.56550014,692.35298844)(67.58550282,692.28299194)
\lineto(67.58550282,692.14799194)
\curveto(67.6055001,692.06798872)(67.62050008,691.9879888)(67.63050282,691.90799194)
\curveto(67.64050006,691.83798895)(67.65550005,691.76298903)(67.67550282,691.68299194)
\curveto(67.77549993,691.38298941)(67.88049982,691.13798965)(67.99050282,690.94799194)
\curveto(68.11049959,690.76799002)(68.29549941,690.60299019)(68.54550282,690.45299194)
\curveto(68.61549909,690.40299039)(68.69049901,690.36299043)(68.77050282,690.33299194)
\curveto(68.86049884,690.30299049)(68.95049875,690.27799051)(69.04050282,690.25799194)
\curveto(69.08049862,690.24799054)(69.11549859,690.24299055)(69.14550282,690.24299194)
\curveto(69.17549853,690.25299054)(69.21049849,690.25299054)(69.25050282,690.24299194)
\lineto(69.37050282,690.21299194)
\curveto(69.42049828,690.21299058)(69.46549824,690.21799057)(69.50550282,690.22799194)
\lineto(69.62550282,690.22799194)
\curveto(69.705498,690.24799054)(69.78549792,690.26299053)(69.86550282,690.27299194)
\curveto(69.94549776,690.28299051)(70.02049768,690.30299049)(70.09050282,690.33299194)
\curveto(70.35049735,690.43299036)(70.56049714,690.56799022)(70.72050282,690.73799194)
\curveto(70.88049682,690.90798988)(71.01549669,691.11798967)(71.12550282,691.36799194)
\curveto(71.16549654,691.46798932)(71.19549651,691.56798922)(71.21550282,691.66799194)
\curveto(71.23549647,691.76798902)(71.26049644,691.87298892)(71.29050282,691.98299194)
\curveto(71.3004964,692.02298877)(71.3054964,692.05798873)(71.30550282,692.08799194)
\curveto(71.3054964,692.12798866)(71.31049639,692.16798862)(71.32050282,692.20799194)
\lineto(71.32050282,692.34299194)
\curveto(71.32049638,692.3929884)(71.32549638,692.44298835)(71.33550282,692.49299194)
}
}
{
\newrgbcolor{curcolor}{0 0 0}
\pscustom[linestyle=none,fillstyle=solid,fillcolor=curcolor]
{
\newpath
\moveto(79.26042469,696.78299194)
\curveto(79.37041938,696.78298401)(79.46541928,696.77298402)(79.54542469,696.75299194)
\curveto(79.63541911,696.73298406)(79.70541904,696.6879841)(79.75542469,696.61799194)
\curveto(79.81541893,696.53798425)(79.8454189,696.39798439)(79.84542469,696.19799194)
\lineto(79.84542469,695.68799194)
\lineto(79.84542469,695.31299194)
\curveto(79.85541889,695.17298562)(79.84041891,695.06298573)(79.80042469,694.98299194)
\curveto(79.76041899,694.91298588)(79.70041905,694.86798592)(79.62042469,694.84799194)
\curveto(79.5504192,694.82798596)(79.46541928,694.81798597)(79.36542469,694.81799194)
\curveto(79.27541947,694.81798597)(79.17541957,694.82298597)(79.06542469,694.83299194)
\curveto(78.96541978,694.84298595)(78.87041988,694.83798595)(78.78042469,694.81799194)
\curveto(78.71042004,694.79798599)(78.64042011,694.78298601)(78.57042469,694.77299194)
\curveto(78.50042025,694.77298602)(78.43542031,694.76298603)(78.37542469,694.74299194)
\curveto(78.21542053,694.6929861)(78.05542069,694.61798617)(77.89542469,694.51799194)
\curveto(77.73542101,694.42798636)(77.61042114,694.32298647)(77.52042469,694.20299194)
\curveto(77.47042128,694.12298667)(77.41542133,694.03798675)(77.35542469,693.94799194)
\curveto(77.30542144,693.86798692)(77.25542149,693.78298701)(77.20542469,693.69299194)
\curveto(77.17542157,693.61298718)(77.1454216,693.52798726)(77.11542469,693.43799194)
\lineto(77.05542469,693.19799194)
\curveto(77.03542171,693.12798766)(77.02542172,693.05298774)(77.02542469,692.97299194)
\curveto(77.02542172,692.90298789)(77.01542173,692.83298796)(76.99542469,692.76299194)
\curveto(76.98542176,692.72298807)(76.98042177,692.68298811)(76.98042469,692.64299194)
\curveto(76.99042176,692.61298818)(76.99042176,692.58298821)(76.98042469,692.55299194)
\lineto(76.98042469,692.31299194)
\curveto(76.96042179,692.24298855)(76.95542179,692.16298863)(76.96542469,692.07299194)
\curveto(76.97542177,691.9929888)(76.98042177,691.91298888)(76.98042469,691.83299194)
\lineto(76.98042469,690.87299194)
\lineto(76.98042469,689.59799194)
\curveto(76.98042177,689.46799132)(76.97542177,689.34799144)(76.96542469,689.23799194)
\curveto(76.95542179,689.12799166)(76.92542182,689.03799175)(76.87542469,688.96799194)
\curveto(76.85542189,688.93799185)(76.82042193,688.91299188)(76.77042469,688.89299194)
\curveto(76.73042202,688.88299191)(76.68542206,688.87299192)(76.63542469,688.86299194)
\lineto(76.56042469,688.86299194)
\curveto(76.51042224,688.85299194)(76.45542229,688.84799194)(76.39542469,688.84799194)
\lineto(76.23042469,688.84799194)
\lineto(75.58542469,688.84799194)
\curveto(75.52542322,688.85799193)(75.46042329,688.86299193)(75.39042469,688.86299194)
\lineto(75.19542469,688.86299194)
\curveto(75.1454236,688.88299191)(75.09542365,688.89799189)(75.04542469,688.90799194)
\curveto(74.99542375,688.92799186)(74.96042379,688.96299183)(74.94042469,689.01299194)
\curveto(74.90042385,689.06299173)(74.87542387,689.13299166)(74.86542469,689.22299194)
\lineto(74.86542469,689.52299194)
\lineto(74.86542469,690.54299194)
\lineto(74.86542469,694.77299194)
\lineto(74.86542469,695.88299194)
\lineto(74.86542469,696.16799194)
\curveto(74.86542388,696.26798452)(74.88542386,696.34798444)(74.92542469,696.40799194)
\curveto(74.97542377,696.4879843)(75.0504237,696.53798425)(75.15042469,696.55799194)
\curveto(75.2504235,696.57798421)(75.37042338,696.5879842)(75.51042469,696.58799194)
\lineto(76.27542469,696.58799194)
\curveto(76.39542235,696.5879842)(76.50042225,696.57798421)(76.59042469,696.55799194)
\curveto(76.68042207,696.54798424)(76.750422,696.50298429)(76.80042469,696.42299194)
\curveto(76.83042192,696.37298442)(76.8454219,696.30298449)(76.84542469,696.21299194)
\lineto(76.87542469,695.94299194)
\curveto(76.88542186,695.86298493)(76.90042185,695.787985)(76.92042469,695.71799194)
\curveto(76.9504218,695.64798514)(77.00042175,695.61298518)(77.07042469,695.61299194)
\curveto(77.09042166,695.63298516)(77.11042164,695.64298515)(77.13042469,695.64299194)
\curveto(77.1504216,695.64298515)(77.17042158,695.65298514)(77.19042469,695.67299194)
\curveto(77.2504215,695.72298507)(77.30042145,695.77798501)(77.34042469,695.83799194)
\curveto(77.39042136,695.90798488)(77.4504213,695.96798482)(77.52042469,696.01799194)
\curveto(77.56042119,696.04798474)(77.59542115,696.07798471)(77.62542469,696.10799194)
\curveto(77.65542109,696.14798464)(77.69042106,696.18298461)(77.73042469,696.21299194)
\lineto(78.00042469,696.39299194)
\curveto(78.10042065,696.45298434)(78.20042055,696.50798428)(78.30042469,696.55799194)
\curveto(78.40042035,696.59798419)(78.50042025,696.63298416)(78.60042469,696.66299194)
\lineto(78.93042469,696.75299194)
\curveto(78.96041979,696.76298403)(79.01541973,696.76298403)(79.09542469,696.75299194)
\curveto(79.18541956,696.75298404)(79.24041951,696.76298403)(79.26042469,696.78299194)
}
}
{
\newrgbcolor{curcolor}{0 0 0}
\pscustom[linestyle=none,fillstyle=solid,fillcolor=curcolor]
{
\newpath
\moveto(87.76683094,692.79299194)
\curveto(87.78682278,692.71298808)(87.78682278,692.62298817)(87.76683094,692.52299194)
\curveto(87.74682282,692.42298837)(87.71182285,692.35798843)(87.66183094,692.32799194)
\curveto(87.61182295,692.2879885)(87.53682303,692.25798853)(87.43683094,692.23799194)
\curveto(87.34682322,692.22798856)(87.24182332,692.21798857)(87.12183094,692.20799194)
\lineto(86.77683094,692.20799194)
\curveto(86.6668239,692.21798857)(86.566824,692.22298857)(86.47683094,692.22299194)
\lineto(82.81683094,692.22299194)
\lineto(82.60683094,692.22299194)
\curveto(82.54682802,692.22298857)(82.49182807,692.21298858)(82.44183094,692.19299194)
\curveto(82.3618282,692.15298864)(82.31182825,692.11298868)(82.29183094,692.07299194)
\curveto(82.27182829,692.05298874)(82.25182831,692.01298878)(82.23183094,691.95299194)
\curveto(82.21182835,691.90298889)(82.20682836,691.85298894)(82.21683094,691.80299194)
\curveto(82.23682833,691.74298905)(82.24682832,691.68298911)(82.24683094,691.62299194)
\curveto(82.25682831,691.57298922)(82.27182829,691.51798927)(82.29183094,691.45799194)
\curveto(82.37182819,691.21798957)(82.4668281,691.01798977)(82.57683094,690.85799194)
\curveto(82.69682787,690.70799008)(82.85682771,690.57299022)(83.05683094,690.45299194)
\curveto(83.13682743,690.40299039)(83.21682735,690.36799042)(83.29683094,690.34799194)
\curveto(83.38682718,690.33799045)(83.47682709,690.31799047)(83.56683094,690.28799194)
\curveto(83.64682692,690.26799052)(83.75682681,690.25299054)(83.89683094,690.24299194)
\curveto(84.03682653,690.23299056)(84.15682641,690.23799055)(84.25683094,690.25799194)
\lineto(84.39183094,690.25799194)
\curveto(84.49182607,690.27799051)(84.58182598,690.29799049)(84.66183094,690.31799194)
\curveto(84.75182581,690.34799044)(84.83682573,690.37799041)(84.91683094,690.40799194)
\curveto(85.01682555,690.45799033)(85.12682544,690.52299027)(85.24683094,690.60299194)
\curveto(85.37682519,690.68299011)(85.47182509,690.76299003)(85.53183094,690.84299194)
\curveto(85.58182498,690.91298988)(85.63182493,690.97798981)(85.68183094,691.03799194)
\curveto(85.74182482,691.10798968)(85.81182475,691.15798963)(85.89183094,691.18799194)
\curveto(85.99182457,691.23798955)(86.11682445,691.25798953)(86.26683094,691.24799194)
\lineto(86.70183094,691.24799194)
\lineto(86.88183094,691.24799194)
\curveto(86.95182361,691.25798953)(87.01182355,691.25298954)(87.06183094,691.23299194)
\lineto(87.21183094,691.23299194)
\curveto(87.31182325,691.21298958)(87.38182318,691.1879896)(87.42183094,691.15799194)
\curveto(87.4618231,691.13798965)(87.48182308,691.0929897)(87.48183094,691.02299194)
\curveto(87.49182307,690.95298984)(87.48682308,690.8929899)(87.46683094,690.84299194)
\curveto(87.41682315,690.70299009)(87.3618232,690.57799021)(87.30183094,690.46799194)
\curveto(87.24182332,690.35799043)(87.17182339,690.24799054)(87.09183094,690.13799194)
\curveto(86.87182369,689.80799098)(86.62182394,689.54299125)(86.34183094,689.34299194)
\curveto(86.0618245,689.14299165)(85.71182485,688.97299182)(85.29183094,688.83299194)
\curveto(85.18182538,688.792992)(85.07182549,688.76799202)(84.96183094,688.75799194)
\curveto(84.85182571,688.74799204)(84.73682583,688.72799206)(84.61683094,688.69799194)
\curveto(84.57682599,688.6879921)(84.53182603,688.6879921)(84.48183094,688.69799194)
\curveto(84.44182612,688.69799209)(84.40182616,688.6929921)(84.36183094,688.68299194)
\lineto(84.19683094,688.68299194)
\curveto(84.14682642,688.66299213)(84.08682648,688.65799213)(84.01683094,688.66799194)
\curveto(83.95682661,688.66799212)(83.90182666,688.67299212)(83.85183094,688.68299194)
\curveto(83.77182679,688.6929921)(83.70182686,688.6929921)(83.64183094,688.68299194)
\curveto(83.58182698,688.67299212)(83.51682705,688.67799211)(83.44683094,688.69799194)
\curveto(83.39682717,688.71799207)(83.34182722,688.72799206)(83.28183094,688.72799194)
\curveto(83.22182734,688.72799206)(83.1668274,688.73799205)(83.11683094,688.75799194)
\curveto(83.00682756,688.77799201)(82.89682767,688.80299199)(82.78683094,688.83299194)
\curveto(82.67682789,688.85299194)(82.57682799,688.8879919)(82.48683094,688.93799194)
\curveto(82.37682819,688.97799181)(82.27182829,689.01299178)(82.17183094,689.04299194)
\curveto(82.08182848,689.08299171)(81.99682857,689.12799166)(81.91683094,689.17799194)
\curveto(81.59682897,689.37799141)(81.31182925,689.60799118)(81.06183094,689.86799194)
\curveto(80.81182975,690.13799065)(80.60682996,690.44799034)(80.44683094,690.79799194)
\curveto(80.39683017,690.90798988)(80.35683021,691.01798977)(80.32683094,691.12799194)
\curveto(80.29683027,691.24798954)(80.25683031,691.36798942)(80.20683094,691.48799194)
\curveto(80.19683037,691.52798926)(80.19183037,691.56298923)(80.19183094,691.59299194)
\curveto(80.19183037,691.63298916)(80.18683038,691.67298912)(80.17683094,691.71299194)
\curveto(80.13683043,691.83298896)(80.11183045,691.96298883)(80.10183094,692.10299194)
\lineto(80.07183094,692.52299194)
\curveto(80.07183049,692.57298822)(80.0668305,692.62798816)(80.05683094,692.68799194)
\curveto(80.05683051,692.74798804)(80.0618305,692.80298799)(80.07183094,692.85299194)
\lineto(80.07183094,693.03299194)
\lineto(80.11683094,693.39299194)
\curveto(80.15683041,693.56298723)(80.19183037,693.72798706)(80.22183094,693.88799194)
\curveto(80.25183031,694.04798674)(80.29683027,694.19798659)(80.35683094,694.33799194)
\curveto(80.78682978,695.37798541)(81.51682905,696.11298468)(82.54683094,696.54299194)
\curveto(82.68682788,696.60298419)(82.82682774,696.64298415)(82.96683094,696.66299194)
\curveto(83.11682745,696.6929841)(83.27182729,696.72798406)(83.43183094,696.76799194)
\curveto(83.51182705,696.77798401)(83.58682698,696.78298401)(83.65683094,696.78299194)
\curveto(83.72682684,696.78298401)(83.80182676,696.787984)(83.88183094,696.79799194)
\curveto(84.39182617,696.80798398)(84.82682574,696.74798404)(85.18683094,696.61799194)
\curveto(85.55682501,696.49798429)(85.88682468,696.33798445)(86.17683094,696.13799194)
\curveto(86.2668243,696.07798471)(86.35682421,696.00798478)(86.44683094,695.92799194)
\curveto(86.53682403,695.85798493)(86.61682395,695.78298501)(86.68683094,695.70299194)
\curveto(86.71682385,695.65298514)(86.75682381,695.61298518)(86.80683094,695.58299194)
\curveto(86.88682368,695.47298532)(86.9618236,695.35798543)(87.03183094,695.23799194)
\curveto(87.10182346,695.12798566)(87.17682339,695.01298578)(87.25683094,694.89299194)
\curveto(87.30682326,694.80298599)(87.34682322,694.70798608)(87.37683094,694.60799194)
\curveto(87.41682315,694.51798627)(87.45682311,694.41798637)(87.49683094,694.30799194)
\curveto(87.54682302,694.17798661)(87.58682298,694.04298675)(87.61683094,693.90299194)
\curveto(87.64682292,693.76298703)(87.68182288,693.62298717)(87.72183094,693.48299194)
\curveto(87.74182282,693.40298739)(87.74682282,693.31298748)(87.73683094,693.21299194)
\curveto(87.73682283,693.12298767)(87.74682282,693.03798775)(87.76683094,692.95799194)
\lineto(87.76683094,692.79299194)
\moveto(85.51683094,693.67799194)
\curveto(85.58682498,693.77798701)(85.59182497,693.89798689)(85.53183094,694.03799194)
\curveto(85.48182508,694.1879866)(85.44182512,694.29798649)(85.41183094,694.36799194)
\curveto(85.27182529,694.63798615)(85.08682548,694.84298595)(84.85683094,694.98299194)
\curveto(84.62682594,695.13298566)(84.30682626,695.21298558)(83.89683094,695.22299194)
\curveto(83.8668267,695.20298559)(83.83182673,695.19798559)(83.79183094,695.20799194)
\curveto(83.75182681,695.21798557)(83.71682685,695.21798557)(83.68683094,695.20799194)
\curveto(83.63682693,695.1879856)(83.58182698,695.17298562)(83.52183094,695.16299194)
\curveto(83.4618271,695.16298563)(83.40682716,695.15298564)(83.35683094,695.13299194)
\curveto(82.91682765,694.9929858)(82.59182797,694.71798607)(82.38183094,694.30799194)
\curveto(82.3618282,694.26798652)(82.33682823,694.21298658)(82.30683094,694.14299194)
\curveto(82.28682828,694.08298671)(82.27182829,694.01798677)(82.26183094,693.94799194)
\curveto(82.25182831,693.8879869)(82.25182831,693.82798696)(82.26183094,693.76799194)
\curveto(82.28182828,693.70798708)(82.31682825,693.65798713)(82.36683094,693.61799194)
\curveto(82.44682812,693.56798722)(82.55682801,693.54298725)(82.69683094,693.54299194)
\lineto(83.10183094,693.54299194)
\lineto(84.76683094,693.54299194)
\lineto(85.20183094,693.54299194)
\curveto(85.3618252,693.55298724)(85.4668251,693.59798719)(85.51683094,693.67799194)
}
}
{
\newrgbcolor{curcolor}{0 0 0}
\pscustom[linestyle=none,fillstyle=solid,fillcolor=curcolor]
{
\newpath
\moveto(91.98511219,696.79799194)
\curveto(92.73510769,696.81798397)(93.38510704,696.73298406)(93.93511219,696.54299194)
\curveto(94.49510593,696.36298443)(94.92010551,696.04798474)(95.21011219,695.59799194)
\curveto(95.28010515,695.4879853)(95.34010509,695.37298542)(95.39011219,695.25299194)
\curveto(95.45010498,695.14298565)(95.50010493,695.01798577)(95.54011219,694.87799194)
\curveto(95.56010487,694.81798597)(95.57010486,694.75298604)(95.57011219,694.68299194)
\curveto(95.57010486,694.61298618)(95.56010487,694.55298624)(95.54011219,694.50299194)
\curveto(95.50010493,694.44298635)(95.44510498,694.40298639)(95.37511219,694.38299194)
\curveto(95.3251051,694.36298643)(95.26510516,694.35298644)(95.19511219,694.35299194)
\lineto(94.98511219,694.35299194)
\lineto(94.32511219,694.35299194)
\curveto(94.25510617,694.35298644)(94.18510624,694.34798644)(94.11511219,694.33799194)
\curveto(94.04510638,694.33798645)(93.98010645,694.34798644)(93.92011219,694.36799194)
\curveto(93.82010661,694.3879864)(93.74510668,694.42798636)(93.69511219,694.48799194)
\curveto(93.64510678,694.54798624)(93.60010683,694.60798618)(93.56011219,694.66799194)
\lineto(93.44011219,694.87799194)
\curveto(93.41010702,694.95798583)(93.36010707,695.02298577)(93.29011219,695.07299194)
\curveto(93.19010724,695.15298564)(93.09010734,695.21298558)(92.99011219,695.25299194)
\curveto(92.90010753,695.2929855)(92.78510764,695.32798546)(92.64511219,695.35799194)
\curveto(92.57510785,695.37798541)(92.47010796,695.3929854)(92.33011219,695.40299194)
\curveto(92.20010823,695.41298538)(92.10010833,695.40798538)(92.03011219,695.38799194)
\lineto(91.92511219,695.38799194)
\lineto(91.77511219,695.35799194)
\curveto(91.73510869,695.35798543)(91.69010874,695.35298544)(91.64011219,695.34299194)
\curveto(91.47010896,695.2929855)(91.3301091,695.22298557)(91.22011219,695.13299194)
\curveto(91.12010931,695.05298574)(91.05010938,694.92798586)(91.01011219,694.75799194)
\curveto(90.99010944,694.6879861)(90.99010944,694.62298617)(91.01011219,694.56299194)
\curveto(91.0301094,694.50298629)(91.05010938,694.45298634)(91.07011219,694.41299194)
\curveto(91.14010929,694.2929865)(91.22010921,694.19798659)(91.31011219,694.12799194)
\curveto(91.41010902,694.05798673)(91.5251089,693.99798679)(91.65511219,693.94799194)
\curveto(91.84510858,693.86798692)(92.05010838,693.79798699)(92.27011219,693.73799194)
\lineto(92.96011219,693.58799194)
\curveto(93.20010723,693.54798724)(93.430107,693.49798729)(93.65011219,693.43799194)
\curveto(93.88010655,693.3879874)(94.09510633,693.32298747)(94.29511219,693.24299194)
\curveto(94.38510604,693.20298759)(94.47010596,693.16798762)(94.55011219,693.13799194)
\curveto(94.64010579,693.11798767)(94.7251057,693.08298771)(94.80511219,693.03299194)
\curveto(94.99510543,692.91298788)(95.16510526,692.78298801)(95.31511219,692.64299194)
\curveto(95.47510495,692.50298829)(95.60010483,692.32798846)(95.69011219,692.11799194)
\curveto(95.72010471,692.04798874)(95.74510468,691.97798881)(95.76511219,691.90799194)
\curveto(95.78510464,691.83798895)(95.80510462,691.76298903)(95.82511219,691.68299194)
\curveto(95.83510459,691.62298917)(95.84010459,691.52798926)(95.84011219,691.39799194)
\curveto(95.85010458,691.27798951)(95.85010458,691.18298961)(95.84011219,691.11299194)
\lineto(95.84011219,691.03799194)
\curveto(95.82010461,690.97798981)(95.80510462,690.91798987)(95.79511219,690.85799194)
\curveto(95.79510463,690.80798998)(95.79010464,690.75799003)(95.78011219,690.70799194)
\curveto(95.71010472,690.40799038)(95.60010483,690.14299065)(95.45011219,689.91299194)
\curveto(95.29010514,689.67299112)(95.09510533,689.47799131)(94.86511219,689.32799194)
\curveto(94.63510579,689.17799161)(94.37510605,689.04799174)(94.08511219,688.93799194)
\curveto(93.97510645,688.8879919)(93.85510657,688.85299194)(93.72511219,688.83299194)
\curveto(93.60510682,688.81299198)(93.48510694,688.787992)(93.36511219,688.75799194)
\curveto(93.27510715,688.73799205)(93.18010725,688.72799206)(93.08011219,688.72799194)
\curveto(92.99010744,688.71799207)(92.90010753,688.70299209)(92.81011219,688.68299194)
\lineto(92.54011219,688.68299194)
\curveto(92.48010795,688.66299213)(92.37510805,688.65299214)(92.22511219,688.65299194)
\curveto(92.08510834,688.65299214)(91.98510844,688.66299213)(91.92511219,688.68299194)
\curveto(91.89510853,688.68299211)(91.86010857,688.6879921)(91.82011219,688.69799194)
\lineto(91.71511219,688.69799194)
\curveto(91.59510883,688.71799207)(91.47510895,688.73299206)(91.35511219,688.74299194)
\curveto(91.23510919,688.75299204)(91.12010931,688.77299202)(91.01011219,688.80299194)
\curveto(90.62010981,688.91299188)(90.27511015,689.03799175)(89.97511219,689.17799194)
\curveto(89.67511075,689.32799146)(89.42011101,689.54799124)(89.21011219,689.83799194)
\curveto(89.07011136,690.02799076)(88.95011148,690.24799054)(88.85011219,690.49799194)
\curveto(88.8301116,690.55799023)(88.81011162,690.63799015)(88.79011219,690.73799194)
\curveto(88.77011166,690.78799)(88.75511167,690.85798993)(88.74511219,690.94799194)
\curveto(88.73511169,691.03798975)(88.74011169,691.11298968)(88.76011219,691.17299194)
\curveto(88.79011164,691.24298955)(88.84011159,691.2929895)(88.91011219,691.32299194)
\curveto(88.96011147,691.34298945)(89.02011141,691.35298944)(89.09011219,691.35299194)
\lineto(89.31511219,691.35299194)
\lineto(90.02011219,691.35299194)
\lineto(90.26011219,691.35299194)
\curveto(90.34011009,691.35298944)(90.41011002,691.34298945)(90.47011219,691.32299194)
\curveto(90.58010985,691.28298951)(90.65010978,691.21798957)(90.68011219,691.12799194)
\curveto(90.72010971,691.03798975)(90.76510966,690.94298985)(90.81511219,690.84299194)
\curveto(90.83510959,690.79299)(90.87010956,690.72799006)(90.92011219,690.64799194)
\curveto(90.98010945,690.56799022)(91.0301094,690.51799027)(91.07011219,690.49799194)
\curveto(91.19010924,690.39799039)(91.30510912,690.31799047)(91.41511219,690.25799194)
\curveto(91.5251089,690.20799058)(91.66510876,690.15799063)(91.83511219,690.10799194)
\curveto(91.88510854,690.0879907)(91.93510849,690.07799071)(91.98511219,690.07799194)
\curveto(92.03510839,690.0879907)(92.08510834,690.0879907)(92.13511219,690.07799194)
\curveto(92.21510821,690.05799073)(92.30010813,690.04799074)(92.39011219,690.04799194)
\curveto(92.49010794,690.05799073)(92.57510785,690.07299072)(92.64511219,690.09299194)
\curveto(92.69510773,690.10299069)(92.74010769,690.10799068)(92.78011219,690.10799194)
\curveto(92.8301076,690.10799068)(92.88010755,690.11799067)(92.93011219,690.13799194)
\curveto(93.07010736,690.1879906)(93.19510723,690.24799054)(93.30511219,690.31799194)
\curveto(93.425107,690.3879904)(93.52010691,690.47799031)(93.59011219,690.58799194)
\curveto(93.64010679,690.66799012)(93.68010675,690.79299)(93.71011219,690.96299194)
\curveto(93.7301067,691.03298976)(93.7301067,691.09798969)(93.71011219,691.15799194)
\curveto(93.69010674,691.21798957)(93.67010676,691.26798952)(93.65011219,691.30799194)
\curveto(93.58010685,691.44798934)(93.49010694,691.55298924)(93.38011219,691.62299194)
\curveto(93.28010715,691.6929891)(93.16010727,691.75798903)(93.02011219,691.81799194)
\curveto(92.8301076,691.89798889)(92.6301078,691.96298883)(92.42011219,692.01299194)
\curveto(92.21010822,692.06298873)(92.00010843,692.11798867)(91.79011219,692.17799194)
\curveto(91.71010872,692.19798859)(91.6251088,692.21298858)(91.53511219,692.22299194)
\curveto(91.45510897,692.23298856)(91.37510905,692.24798854)(91.29511219,692.26799194)
\curveto(90.97510945,692.35798843)(90.67010976,692.44298835)(90.38011219,692.52299194)
\curveto(90.09011034,692.61298818)(89.8251106,692.74298805)(89.58511219,692.91299194)
\curveto(89.30511112,693.11298768)(89.10011133,693.38298741)(88.97011219,693.72299194)
\curveto(88.95011148,693.792987)(88.9301115,693.8879869)(88.91011219,694.00799194)
\curveto(88.89011154,694.07798671)(88.87511155,694.16298663)(88.86511219,694.26299194)
\curveto(88.85511157,694.36298643)(88.86011157,694.45298634)(88.88011219,694.53299194)
\curveto(88.90011153,694.58298621)(88.90511152,694.62298617)(88.89511219,694.65299194)
\curveto(88.88511154,694.6929861)(88.89011154,694.73798605)(88.91011219,694.78799194)
\curveto(88.9301115,694.89798589)(88.95011148,694.99798579)(88.97011219,695.08799194)
\curveto(89.00011143,695.1879856)(89.03511139,695.28298551)(89.07511219,695.37299194)
\curveto(89.20511122,695.66298513)(89.38511104,695.89798489)(89.61511219,696.07799194)
\curveto(89.84511058,696.25798453)(90.10511032,696.40298439)(90.39511219,696.51299194)
\curveto(90.50510992,696.56298423)(90.62010981,696.59798419)(90.74011219,696.61799194)
\curveto(90.86010957,696.64798414)(90.98510944,696.67798411)(91.11511219,696.70799194)
\curveto(91.17510925,696.72798406)(91.23510919,696.73798405)(91.29511219,696.73799194)
\lineto(91.47511219,696.76799194)
\curveto(91.55510887,696.77798401)(91.64010879,696.78298401)(91.73011219,696.78299194)
\curveto(91.82010861,696.78298401)(91.90510852,696.787984)(91.98511219,696.79799194)
}
}
{
\newrgbcolor{curcolor}{0 0 0}
\pscustom[linestyle=none,fillstyle=solid,fillcolor=curcolor]
{
}
}
{
\newrgbcolor{curcolor}{0 0 0}
\pscustom[linestyle=none,fillstyle=solid,fillcolor=curcolor]
{
\newpath
\moveto(108.82190907,689.70299194)
\lineto(108.82190907,689.28299194)
\curveto(108.8219007,689.15299164)(108.79190073,689.04799174)(108.73190907,688.96799194)
\curveto(108.68190084,688.91799187)(108.6169009,688.88299191)(108.53690907,688.86299194)
\curveto(108.45690106,688.85299194)(108.36690115,688.84799194)(108.26690907,688.84799194)
\lineto(107.44190907,688.84799194)
\lineto(107.15690907,688.84799194)
\curveto(107.07690244,688.85799193)(107.01190251,688.88299191)(106.96190907,688.92299194)
\curveto(106.89190263,688.97299182)(106.85190267,689.03799175)(106.84190907,689.11799194)
\curveto(106.83190269,689.19799159)(106.81190271,689.27799151)(106.78190907,689.35799194)
\curveto(106.76190276,689.37799141)(106.74190278,689.3929914)(106.72190907,689.40299194)
\curveto(106.71190281,689.42299137)(106.69690282,689.44299135)(106.67690907,689.46299194)
\curveto(106.56690295,689.46299133)(106.48690303,689.43799135)(106.43690907,689.38799194)
\lineto(106.28690907,689.23799194)
\curveto(106.2169033,689.1879916)(106.15190337,689.14299165)(106.09190907,689.10299194)
\curveto(106.03190349,689.07299172)(105.96690355,689.03299176)(105.89690907,688.98299194)
\curveto(105.85690366,688.96299183)(105.81190371,688.94299185)(105.76190907,688.92299194)
\curveto(105.7219038,688.90299189)(105.67690384,688.88299191)(105.62690907,688.86299194)
\curveto(105.48690403,688.81299198)(105.33690418,688.76799202)(105.17690907,688.72799194)
\curveto(105.12690439,688.70799208)(105.08190444,688.69799209)(105.04190907,688.69799194)
\curveto(105.00190452,688.69799209)(104.96190456,688.6929921)(104.92190907,688.68299194)
\lineto(104.78690907,688.68299194)
\curveto(104.75690476,688.67299212)(104.7169048,688.66799212)(104.66690907,688.66799194)
\lineto(104.53190907,688.66799194)
\curveto(104.47190505,688.64799214)(104.38190514,688.64299215)(104.26190907,688.65299194)
\curveto(104.14190538,688.65299214)(104.05690546,688.66299213)(104.00690907,688.68299194)
\curveto(103.93690558,688.70299209)(103.87190565,688.71299208)(103.81190907,688.71299194)
\curveto(103.76190576,688.70299209)(103.70690581,688.70799208)(103.64690907,688.72799194)
\lineto(103.28690907,688.84799194)
\curveto(103.17690634,688.87799191)(103.06690645,688.91799187)(102.95690907,688.96799194)
\curveto(102.60690691,689.11799167)(102.29190723,689.34799144)(102.01190907,689.65799194)
\curveto(101.74190778,689.97799081)(101.52690799,690.31299048)(101.36690907,690.66299194)
\curveto(101.3169082,690.77299002)(101.27690824,690.87798991)(101.24690907,690.97799194)
\curveto(101.2169083,691.0879897)(101.18190834,691.19798959)(101.14190907,691.30799194)
\curveto(101.13190839,691.34798944)(101.12690839,691.38298941)(101.12690907,691.41299194)
\curveto(101.12690839,691.45298934)(101.1169084,691.49798929)(101.09690907,691.54799194)
\curveto(101.07690844,691.62798916)(101.05690846,691.71298908)(101.03690907,691.80299194)
\curveto(101.02690849,691.90298889)(101.01190851,692.00298879)(100.99190907,692.10299194)
\curveto(100.98190854,692.13298866)(100.97690854,692.16798862)(100.97690907,692.20799194)
\curveto(100.98690853,692.24798854)(100.98690853,692.28298851)(100.97690907,692.31299194)
\lineto(100.97690907,692.44799194)
\curveto(100.97690854,692.49798829)(100.97190855,692.54798824)(100.96190907,692.59799194)
\curveto(100.95190857,692.64798814)(100.94690857,692.70298809)(100.94690907,692.76299194)
\curveto(100.94690857,692.83298796)(100.95190857,692.8879879)(100.96190907,692.92799194)
\curveto(100.97190855,692.97798781)(100.97690854,693.02298777)(100.97690907,693.06299194)
\lineto(100.97690907,693.21299194)
\curveto(100.98690853,693.26298753)(100.98690853,693.30798748)(100.97690907,693.34799194)
\curveto(100.97690854,693.39798739)(100.98690853,693.44798734)(101.00690907,693.49799194)
\curveto(101.02690849,693.60798718)(101.04190848,693.71298708)(101.05190907,693.81299194)
\curveto(101.07190845,693.91298688)(101.09690842,694.01298678)(101.12690907,694.11299194)
\curveto(101.16690835,694.23298656)(101.20190832,694.34798644)(101.23190907,694.45799194)
\curveto(101.26190826,694.56798622)(101.30190822,694.67798611)(101.35190907,694.78799194)
\curveto(101.49190803,695.0879857)(101.66690785,695.37298542)(101.87690907,695.64299194)
\curveto(101.89690762,695.67298512)(101.9219076,695.69798509)(101.95190907,695.71799194)
\curveto(101.99190753,695.74798504)(102.0219075,695.77798501)(102.04190907,695.80799194)
\curveto(102.08190744,695.85798493)(102.1219074,695.90298489)(102.16190907,695.94299194)
\curveto(102.20190732,695.98298481)(102.24690727,696.02298477)(102.29690907,696.06299194)
\curveto(102.33690718,696.08298471)(102.37190715,696.10798468)(102.40190907,696.13799194)
\curveto(102.43190709,696.17798461)(102.46690705,696.20798458)(102.50690907,696.22799194)
\curveto(102.75690676,696.39798439)(103.04690647,696.53798425)(103.37690907,696.64799194)
\curveto(103.44690607,696.66798412)(103.516906,696.68298411)(103.58690907,696.69299194)
\curveto(103.66690585,696.70298409)(103.74690577,696.71798407)(103.82690907,696.73799194)
\curveto(103.89690562,696.75798403)(103.98690553,696.76798402)(104.09690907,696.76799194)
\curveto(104.20690531,696.77798401)(104.3169052,696.78298401)(104.42690907,696.78299194)
\curveto(104.53690498,696.78298401)(104.64190488,696.77798401)(104.74190907,696.76799194)
\curveto(104.85190467,696.75798403)(104.94190458,696.74298405)(105.01190907,696.72299194)
\curveto(105.16190436,696.67298412)(105.30690421,696.62798416)(105.44690907,696.58799194)
\curveto(105.58690393,696.54798424)(105.7169038,696.4929843)(105.83690907,696.42299194)
\curveto(105.90690361,696.37298442)(105.97190355,696.32298447)(106.03190907,696.27299194)
\curveto(106.09190343,696.23298456)(106.15690336,696.1879846)(106.22690907,696.13799194)
\curveto(106.26690325,696.10798468)(106.3219032,696.06798472)(106.39190907,696.01799194)
\curveto(106.47190305,695.96798482)(106.54690297,695.96798482)(106.61690907,696.01799194)
\curveto(106.65690286,696.03798475)(106.67690284,696.07298472)(106.67690907,696.12299194)
\curveto(106.67690284,696.17298462)(106.68690283,696.22298457)(106.70690907,696.27299194)
\lineto(106.70690907,696.42299194)
\curveto(106.7169028,696.45298434)(106.7219028,696.4879843)(106.72190907,696.52799194)
\lineto(106.72190907,696.64799194)
\lineto(106.72190907,698.68799194)
\curveto(106.7219028,698.79798199)(106.7169028,698.91798187)(106.70690907,699.04799194)
\curveto(106.70690281,699.1879816)(106.73190279,699.2929815)(106.78190907,699.36299194)
\curveto(106.8219027,699.44298135)(106.89690262,699.4929813)(107.00690907,699.51299194)
\curveto(107.02690249,699.52298127)(107.04690247,699.52298127)(107.06690907,699.51299194)
\curveto(107.08690243,699.51298128)(107.10690241,699.51798127)(107.12690907,699.52799194)
\lineto(108.19190907,699.52799194)
\curveto(108.31190121,699.52798126)(108.4219011,699.52298127)(108.52190907,699.51299194)
\curveto(108.6219009,699.50298129)(108.69690082,699.46298133)(108.74690907,699.39299194)
\curveto(108.79690072,699.31298148)(108.8219007,699.20798158)(108.82190907,699.07799194)
\lineto(108.82190907,698.71799194)
\lineto(108.82190907,689.70299194)
\moveto(106.78190907,692.64299194)
\curveto(106.79190273,692.68298811)(106.79190273,692.72298807)(106.78190907,692.76299194)
\lineto(106.78190907,692.89799194)
\curveto(106.78190274,692.99798779)(106.77690274,693.09798769)(106.76690907,693.19799194)
\curveto(106.75690276,693.29798749)(106.74190278,693.3879874)(106.72190907,693.46799194)
\curveto(106.70190282,693.57798721)(106.68190284,693.67798711)(106.66190907,693.76799194)
\curveto(106.65190287,693.85798693)(106.62690289,693.94298685)(106.58690907,694.02299194)
\curveto(106.44690307,694.38298641)(106.24190328,694.66798612)(105.97190907,694.87799194)
\curveto(105.71190381,695.0879857)(105.33190419,695.1929856)(104.83190907,695.19299194)
\curveto(104.77190475,695.1929856)(104.69190483,695.18298561)(104.59190907,695.16299194)
\curveto(104.51190501,695.14298565)(104.43690508,695.12298567)(104.36690907,695.10299194)
\curveto(104.30690521,695.0929857)(104.24690527,695.07298572)(104.18690907,695.04299194)
\curveto(103.9169056,694.93298586)(103.70690581,694.76298603)(103.55690907,694.53299194)
\curveto(103.40690611,694.30298649)(103.28690623,694.04298675)(103.19690907,693.75299194)
\curveto(103.16690635,693.65298714)(103.14690637,693.55298724)(103.13690907,693.45299194)
\curveto(103.12690639,693.35298744)(103.10690641,693.24798754)(103.07690907,693.13799194)
\lineto(103.07690907,692.92799194)
\curveto(103.05690646,692.83798795)(103.05190647,692.71298808)(103.06190907,692.55299194)
\curveto(103.07190645,692.40298839)(103.08690643,692.2929885)(103.10690907,692.22299194)
\lineto(103.10690907,692.13299194)
\curveto(103.1169064,692.11298868)(103.1219064,692.0929887)(103.12190907,692.07299194)
\curveto(103.14190638,691.9929888)(103.15690636,691.91798887)(103.16690907,691.84799194)
\curveto(103.18690633,691.77798901)(103.20690631,691.70298909)(103.22690907,691.62299194)
\curveto(103.39690612,691.10298969)(103.68690583,690.71799007)(104.09690907,690.46799194)
\curveto(104.22690529,690.37799041)(104.40690511,690.30799048)(104.63690907,690.25799194)
\curveto(104.67690484,690.24799054)(104.73690478,690.24299055)(104.81690907,690.24299194)
\curveto(104.84690467,690.23299056)(104.89190463,690.22299057)(104.95190907,690.21299194)
\curveto(105.0219045,690.21299058)(105.07690444,690.21799057)(105.11690907,690.22799194)
\curveto(105.19690432,690.24799054)(105.27690424,690.26299053)(105.35690907,690.27299194)
\curveto(105.43690408,690.28299051)(105.516904,690.30299049)(105.59690907,690.33299194)
\curveto(105.84690367,690.44299035)(106.04690347,690.58299021)(106.19690907,690.75299194)
\curveto(106.34690317,690.92298987)(106.47690304,691.13798965)(106.58690907,691.39799194)
\curveto(106.62690289,691.4879893)(106.65690286,691.57798921)(106.67690907,691.66799194)
\curveto(106.69690282,691.76798902)(106.7169028,691.87298892)(106.73690907,691.98299194)
\curveto(106.74690277,692.03298876)(106.74690277,692.07798871)(106.73690907,692.11799194)
\curveto(106.73690278,692.16798862)(106.74690277,692.21798857)(106.76690907,692.26799194)
\curveto(106.77690274,692.29798849)(106.78190274,692.33298846)(106.78190907,692.37299194)
\lineto(106.78190907,692.50799194)
\lineto(106.78190907,692.64299194)
}
}
{
\newrgbcolor{curcolor}{0 0 0}
\pscustom[linestyle=none,fillstyle=solid,fillcolor=curcolor]
{
\newpath
\moveto(117.76683094,692.79299194)
\curveto(117.78682278,692.71298808)(117.78682278,692.62298817)(117.76683094,692.52299194)
\curveto(117.74682282,692.42298837)(117.71182285,692.35798843)(117.66183094,692.32799194)
\curveto(117.61182295,692.2879885)(117.53682303,692.25798853)(117.43683094,692.23799194)
\curveto(117.34682322,692.22798856)(117.24182332,692.21798857)(117.12183094,692.20799194)
\lineto(116.77683094,692.20799194)
\curveto(116.6668239,692.21798857)(116.566824,692.22298857)(116.47683094,692.22299194)
\lineto(112.81683094,692.22299194)
\lineto(112.60683094,692.22299194)
\curveto(112.54682802,692.22298857)(112.49182807,692.21298858)(112.44183094,692.19299194)
\curveto(112.3618282,692.15298864)(112.31182825,692.11298868)(112.29183094,692.07299194)
\curveto(112.27182829,692.05298874)(112.25182831,692.01298878)(112.23183094,691.95299194)
\curveto(112.21182835,691.90298889)(112.20682836,691.85298894)(112.21683094,691.80299194)
\curveto(112.23682833,691.74298905)(112.24682832,691.68298911)(112.24683094,691.62299194)
\curveto(112.25682831,691.57298922)(112.27182829,691.51798927)(112.29183094,691.45799194)
\curveto(112.37182819,691.21798957)(112.4668281,691.01798977)(112.57683094,690.85799194)
\curveto(112.69682787,690.70799008)(112.85682771,690.57299022)(113.05683094,690.45299194)
\curveto(113.13682743,690.40299039)(113.21682735,690.36799042)(113.29683094,690.34799194)
\curveto(113.38682718,690.33799045)(113.47682709,690.31799047)(113.56683094,690.28799194)
\curveto(113.64682692,690.26799052)(113.75682681,690.25299054)(113.89683094,690.24299194)
\curveto(114.03682653,690.23299056)(114.15682641,690.23799055)(114.25683094,690.25799194)
\lineto(114.39183094,690.25799194)
\curveto(114.49182607,690.27799051)(114.58182598,690.29799049)(114.66183094,690.31799194)
\curveto(114.75182581,690.34799044)(114.83682573,690.37799041)(114.91683094,690.40799194)
\curveto(115.01682555,690.45799033)(115.12682544,690.52299027)(115.24683094,690.60299194)
\curveto(115.37682519,690.68299011)(115.47182509,690.76299003)(115.53183094,690.84299194)
\curveto(115.58182498,690.91298988)(115.63182493,690.97798981)(115.68183094,691.03799194)
\curveto(115.74182482,691.10798968)(115.81182475,691.15798963)(115.89183094,691.18799194)
\curveto(115.99182457,691.23798955)(116.11682445,691.25798953)(116.26683094,691.24799194)
\lineto(116.70183094,691.24799194)
\lineto(116.88183094,691.24799194)
\curveto(116.95182361,691.25798953)(117.01182355,691.25298954)(117.06183094,691.23299194)
\lineto(117.21183094,691.23299194)
\curveto(117.31182325,691.21298958)(117.38182318,691.1879896)(117.42183094,691.15799194)
\curveto(117.4618231,691.13798965)(117.48182308,691.0929897)(117.48183094,691.02299194)
\curveto(117.49182307,690.95298984)(117.48682308,690.8929899)(117.46683094,690.84299194)
\curveto(117.41682315,690.70299009)(117.3618232,690.57799021)(117.30183094,690.46799194)
\curveto(117.24182332,690.35799043)(117.17182339,690.24799054)(117.09183094,690.13799194)
\curveto(116.87182369,689.80799098)(116.62182394,689.54299125)(116.34183094,689.34299194)
\curveto(116.0618245,689.14299165)(115.71182485,688.97299182)(115.29183094,688.83299194)
\curveto(115.18182538,688.792992)(115.07182549,688.76799202)(114.96183094,688.75799194)
\curveto(114.85182571,688.74799204)(114.73682583,688.72799206)(114.61683094,688.69799194)
\curveto(114.57682599,688.6879921)(114.53182603,688.6879921)(114.48183094,688.69799194)
\curveto(114.44182612,688.69799209)(114.40182616,688.6929921)(114.36183094,688.68299194)
\lineto(114.19683094,688.68299194)
\curveto(114.14682642,688.66299213)(114.08682648,688.65799213)(114.01683094,688.66799194)
\curveto(113.95682661,688.66799212)(113.90182666,688.67299212)(113.85183094,688.68299194)
\curveto(113.77182679,688.6929921)(113.70182686,688.6929921)(113.64183094,688.68299194)
\curveto(113.58182698,688.67299212)(113.51682705,688.67799211)(113.44683094,688.69799194)
\curveto(113.39682717,688.71799207)(113.34182722,688.72799206)(113.28183094,688.72799194)
\curveto(113.22182734,688.72799206)(113.1668274,688.73799205)(113.11683094,688.75799194)
\curveto(113.00682756,688.77799201)(112.89682767,688.80299199)(112.78683094,688.83299194)
\curveto(112.67682789,688.85299194)(112.57682799,688.8879919)(112.48683094,688.93799194)
\curveto(112.37682819,688.97799181)(112.27182829,689.01299178)(112.17183094,689.04299194)
\curveto(112.08182848,689.08299171)(111.99682857,689.12799166)(111.91683094,689.17799194)
\curveto(111.59682897,689.37799141)(111.31182925,689.60799118)(111.06183094,689.86799194)
\curveto(110.81182975,690.13799065)(110.60682996,690.44799034)(110.44683094,690.79799194)
\curveto(110.39683017,690.90798988)(110.35683021,691.01798977)(110.32683094,691.12799194)
\curveto(110.29683027,691.24798954)(110.25683031,691.36798942)(110.20683094,691.48799194)
\curveto(110.19683037,691.52798926)(110.19183037,691.56298923)(110.19183094,691.59299194)
\curveto(110.19183037,691.63298916)(110.18683038,691.67298912)(110.17683094,691.71299194)
\curveto(110.13683043,691.83298896)(110.11183045,691.96298883)(110.10183094,692.10299194)
\lineto(110.07183094,692.52299194)
\curveto(110.07183049,692.57298822)(110.0668305,692.62798816)(110.05683094,692.68799194)
\curveto(110.05683051,692.74798804)(110.0618305,692.80298799)(110.07183094,692.85299194)
\lineto(110.07183094,693.03299194)
\lineto(110.11683094,693.39299194)
\curveto(110.15683041,693.56298723)(110.19183037,693.72798706)(110.22183094,693.88799194)
\curveto(110.25183031,694.04798674)(110.29683027,694.19798659)(110.35683094,694.33799194)
\curveto(110.78682978,695.37798541)(111.51682905,696.11298468)(112.54683094,696.54299194)
\curveto(112.68682788,696.60298419)(112.82682774,696.64298415)(112.96683094,696.66299194)
\curveto(113.11682745,696.6929841)(113.27182729,696.72798406)(113.43183094,696.76799194)
\curveto(113.51182705,696.77798401)(113.58682698,696.78298401)(113.65683094,696.78299194)
\curveto(113.72682684,696.78298401)(113.80182676,696.787984)(113.88183094,696.79799194)
\curveto(114.39182617,696.80798398)(114.82682574,696.74798404)(115.18683094,696.61799194)
\curveto(115.55682501,696.49798429)(115.88682468,696.33798445)(116.17683094,696.13799194)
\curveto(116.2668243,696.07798471)(116.35682421,696.00798478)(116.44683094,695.92799194)
\curveto(116.53682403,695.85798493)(116.61682395,695.78298501)(116.68683094,695.70299194)
\curveto(116.71682385,695.65298514)(116.75682381,695.61298518)(116.80683094,695.58299194)
\curveto(116.88682368,695.47298532)(116.9618236,695.35798543)(117.03183094,695.23799194)
\curveto(117.10182346,695.12798566)(117.17682339,695.01298578)(117.25683094,694.89299194)
\curveto(117.30682326,694.80298599)(117.34682322,694.70798608)(117.37683094,694.60799194)
\curveto(117.41682315,694.51798627)(117.45682311,694.41798637)(117.49683094,694.30799194)
\curveto(117.54682302,694.17798661)(117.58682298,694.04298675)(117.61683094,693.90299194)
\curveto(117.64682292,693.76298703)(117.68182288,693.62298717)(117.72183094,693.48299194)
\curveto(117.74182282,693.40298739)(117.74682282,693.31298748)(117.73683094,693.21299194)
\curveto(117.73682283,693.12298767)(117.74682282,693.03798775)(117.76683094,692.95799194)
\lineto(117.76683094,692.79299194)
\moveto(115.51683094,693.67799194)
\curveto(115.58682498,693.77798701)(115.59182497,693.89798689)(115.53183094,694.03799194)
\curveto(115.48182508,694.1879866)(115.44182512,694.29798649)(115.41183094,694.36799194)
\curveto(115.27182529,694.63798615)(115.08682548,694.84298595)(114.85683094,694.98299194)
\curveto(114.62682594,695.13298566)(114.30682626,695.21298558)(113.89683094,695.22299194)
\curveto(113.8668267,695.20298559)(113.83182673,695.19798559)(113.79183094,695.20799194)
\curveto(113.75182681,695.21798557)(113.71682685,695.21798557)(113.68683094,695.20799194)
\curveto(113.63682693,695.1879856)(113.58182698,695.17298562)(113.52183094,695.16299194)
\curveto(113.4618271,695.16298563)(113.40682716,695.15298564)(113.35683094,695.13299194)
\curveto(112.91682765,694.9929858)(112.59182797,694.71798607)(112.38183094,694.30799194)
\curveto(112.3618282,694.26798652)(112.33682823,694.21298658)(112.30683094,694.14299194)
\curveto(112.28682828,694.08298671)(112.27182829,694.01798677)(112.26183094,693.94799194)
\curveto(112.25182831,693.8879869)(112.25182831,693.82798696)(112.26183094,693.76799194)
\curveto(112.28182828,693.70798708)(112.31682825,693.65798713)(112.36683094,693.61799194)
\curveto(112.44682812,693.56798722)(112.55682801,693.54298725)(112.69683094,693.54299194)
\lineto(113.10183094,693.54299194)
\lineto(114.76683094,693.54299194)
\lineto(115.20183094,693.54299194)
\curveto(115.3618252,693.55298724)(115.4668251,693.59798719)(115.51683094,693.67799194)
}
}
{
\newrgbcolor{curcolor}{0 0 0}
\pscustom[linestyle=none,fillstyle=solid,fillcolor=curcolor]
{
}
}
{
\newrgbcolor{curcolor}{0 0 0}
\pscustom[linestyle=none,fillstyle=solid,fillcolor=curcolor]
{
\newpath
\moveto(130.22526844,689.44799194)
\curveto(130.24526059,689.33799145)(130.25526058,689.22799156)(130.25526844,689.11799194)
\curveto(130.26526057,689.00799178)(130.21526062,688.93299186)(130.10526844,688.89299194)
\curveto(130.04526079,688.86299193)(129.97526086,688.84799194)(129.89526844,688.84799194)
\lineto(129.65526844,688.84799194)
\lineto(128.84526844,688.84799194)
\lineto(128.57526844,688.84799194)
\curveto(128.49526234,688.85799193)(128.43026241,688.88299191)(128.38026844,688.92299194)
\curveto(128.31026253,688.96299183)(128.25526258,689.01799177)(128.21526844,689.08799194)
\curveto(128.18526265,689.16799162)(128.1402627,689.23299156)(128.08026844,689.28299194)
\curveto(128.06026278,689.30299149)(128.0352628,689.31799147)(128.00526844,689.32799194)
\curveto(127.97526286,689.34799144)(127.9352629,689.35299144)(127.88526844,689.34299194)
\curveto(127.835263,689.32299147)(127.78526305,689.29799149)(127.73526844,689.26799194)
\curveto(127.69526314,689.23799155)(127.65026319,689.21299158)(127.60026844,689.19299194)
\curveto(127.55026329,689.15299164)(127.49526334,689.11799167)(127.43526844,689.08799194)
\lineto(127.25526844,688.99799194)
\curveto(127.12526371,688.93799185)(126.99026385,688.8879919)(126.85026844,688.84799194)
\curveto(126.71026413,688.81799197)(126.56526427,688.78299201)(126.41526844,688.74299194)
\curveto(126.34526449,688.72299207)(126.27526456,688.71299208)(126.20526844,688.71299194)
\curveto(126.14526469,688.70299209)(126.08026476,688.6929921)(126.01026844,688.68299194)
\lineto(125.92026844,688.68299194)
\curveto(125.89026495,688.67299212)(125.86026498,688.66799212)(125.83026844,688.66799194)
\lineto(125.66526844,688.66799194)
\curveto(125.56526527,688.64799214)(125.46526537,688.64799214)(125.36526844,688.66799194)
\lineto(125.23026844,688.66799194)
\curveto(125.16026568,688.6879921)(125.09026575,688.69799209)(125.02026844,688.69799194)
\curveto(124.96026588,688.6879921)(124.90026594,688.6929921)(124.84026844,688.71299194)
\curveto(124.7402661,688.73299206)(124.64526619,688.75299204)(124.55526844,688.77299194)
\curveto(124.46526637,688.78299201)(124.38026646,688.80799198)(124.30026844,688.84799194)
\curveto(124.01026683,688.95799183)(123.76026708,689.09799169)(123.55026844,689.26799194)
\curveto(123.35026749,689.44799134)(123.19026765,689.68299111)(123.07026844,689.97299194)
\curveto(123.0402678,690.04299075)(123.01026783,690.11799067)(122.98026844,690.19799194)
\curveto(122.96026788,690.27799051)(122.9402679,690.36299043)(122.92026844,690.45299194)
\curveto(122.90026794,690.50299029)(122.89026795,690.55299024)(122.89026844,690.60299194)
\curveto(122.90026794,690.65299014)(122.90026794,690.70299009)(122.89026844,690.75299194)
\curveto(122.88026796,690.78299001)(122.87026797,690.84298995)(122.86026844,690.93299194)
\curveto(122.86026798,691.03298976)(122.86526797,691.10298969)(122.87526844,691.14299194)
\curveto(122.89526794,691.24298955)(122.90526793,691.32798946)(122.90526844,691.39799194)
\lineto(122.99526844,691.72799194)
\curveto(123.02526781,691.84798894)(123.06526777,691.95298884)(123.11526844,692.04299194)
\curveto(123.28526755,692.33298846)(123.48026736,692.55298824)(123.70026844,692.70299194)
\curveto(123.92026692,692.85298794)(124.20026664,692.98298781)(124.54026844,693.09299194)
\curveto(124.67026617,693.14298765)(124.80526603,693.17798761)(124.94526844,693.19799194)
\curveto(125.08526575,693.21798757)(125.22526561,693.24298755)(125.36526844,693.27299194)
\curveto(125.44526539,693.2929875)(125.53026531,693.30298749)(125.62026844,693.30299194)
\curveto(125.71026513,693.31298748)(125.80026504,693.32798746)(125.89026844,693.34799194)
\curveto(125.96026488,693.36798742)(126.03026481,693.37298742)(126.10026844,693.36299194)
\curveto(126.17026467,693.36298743)(126.24526459,693.37298742)(126.32526844,693.39299194)
\curveto(126.39526444,693.41298738)(126.46526437,693.42298737)(126.53526844,693.42299194)
\curveto(126.60526423,693.42298737)(126.68026416,693.43298736)(126.76026844,693.45299194)
\curveto(126.97026387,693.50298729)(127.16026368,693.54298725)(127.33026844,693.57299194)
\curveto(127.51026333,693.61298718)(127.67026317,693.70298709)(127.81026844,693.84299194)
\curveto(127.90026294,693.93298686)(127.96026288,694.03298676)(127.99026844,694.14299194)
\curveto(128.00026284,694.17298662)(128.00026284,694.19798659)(127.99026844,694.21799194)
\curveto(127.99026285,694.23798655)(127.99526284,694.25798653)(128.00526844,694.27799194)
\curveto(128.01526282,694.29798649)(128.02026282,694.32798646)(128.02026844,694.36799194)
\lineto(128.02026844,694.45799194)
\lineto(127.99026844,694.57799194)
\curveto(127.99026285,694.61798617)(127.98526285,694.65298614)(127.97526844,694.68299194)
\curveto(127.87526296,694.98298581)(127.66526317,695.1879856)(127.34526844,695.29799194)
\curveto(127.25526358,695.32798546)(127.14526369,695.34798544)(127.01526844,695.35799194)
\curveto(126.89526394,695.37798541)(126.77026407,695.38298541)(126.64026844,695.37299194)
\curveto(126.51026433,695.37298542)(126.38526445,695.36298543)(126.26526844,695.34299194)
\curveto(126.14526469,695.32298547)(126.0402648,695.29798549)(125.95026844,695.26799194)
\curveto(125.89026495,695.24798554)(125.83026501,695.21798557)(125.77026844,695.17799194)
\curveto(125.72026512,695.14798564)(125.67026517,695.11298568)(125.62026844,695.07299194)
\curveto(125.57026527,695.03298576)(125.51526532,694.97798581)(125.45526844,694.90799194)
\curveto(125.40526543,694.83798595)(125.37026547,694.77298602)(125.35026844,694.71299194)
\curveto(125.30026554,694.61298618)(125.25526558,694.51798627)(125.21526844,694.42799194)
\curveto(125.18526565,694.33798645)(125.11526572,694.27798651)(125.00526844,694.24799194)
\curveto(124.92526591,694.22798656)(124.840266,694.21798657)(124.75026844,694.21799194)
\lineto(124.48026844,694.21799194)
\lineto(123.91026844,694.21799194)
\curveto(123.86026698,694.21798657)(123.81026703,694.21298658)(123.76026844,694.20299194)
\curveto(123.71026713,694.20298659)(123.66526717,694.20798658)(123.62526844,694.21799194)
\lineto(123.49026844,694.21799194)
\curveto(123.47026737,694.22798656)(123.44526739,694.23298656)(123.41526844,694.23299194)
\curveto(123.38526745,694.23298656)(123.36026748,694.24298655)(123.34026844,694.26299194)
\curveto(123.26026758,694.28298651)(123.20526763,694.34798644)(123.17526844,694.45799194)
\curveto(123.16526767,694.50798628)(123.16526767,694.55798623)(123.17526844,694.60799194)
\curveto(123.18526765,694.65798613)(123.19526764,694.70298609)(123.20526844,694.74299194)
\curveto(123.2352676,694.85298594)(123.26526757,694.95298584)(123.29526844,695.04299194)
\curveto(123.3352675,695.14298565)(123.38026746,695.23298556)(123.43026844,695.31299194)
\lineto(123.52026844,695.46299194)
\lineto(123.61026844,695.61299194)
\curveto(123.69026715,695.72298507)(123.79026705,695.82798496)(123.91026844,695.92799194)
\curveto(123.93026691,695.93798485)(123.96026688,695.96298483)(124.00026844,696.00299194)
\curveto(124.05026679,696.04298475)(124.09526674,696.07798471)(124.13526844,696.10799194)
\curveto(124.17526666,696.13798465)(124.22026662,696.16798462)(124.27026844,696.19799194)
\curveto(124.4402664,696.30798448)(124.62026622,696.3929844)(124.81026844,696.45299194)
\curveto(125.00026584,696.52298427)(125.19526564,696.5879842)(125.39526844,696.64799194)
\curveto(125.51526532,696.67798411)(125.6402652,696.69798409)(125.77026844,696.70799194)
\curveto(125.90026494,696.71798407)(126.03026481,696.73798405)(126.16026844,696.76799194)
\curveto(126.20026464,696.77798401)(126.26026458,696.77798401)(126.34026844,696.76799194)
\curveto(126.43026441,696.75798403)(126.48526435,696.76298403)(126.50526844,696.78299194)
\curveto(126.91526392,696.792984)(127.30526353,696.77798401)(127.67526844,696.73799194)
\curveto(128.05526278,696.69798409)(128.39526244,696.62298417)(128.69526844,696.51299194)
\curveto(129.00526183,696.40298439)(129.27026157,696.25298454)(129.49026844,696.06299194)
\curveto(129.71026113,695.88298491)(129.88026096,695.64798514)(130.00026844,695.35799194)
\curveto(130.07026077,695.1879856)(130.11026073,694.9929858)(130.12026844,694.77299194)
\curveto(130.13026071,694.55298624)(130.1352607,694.32798646)(130.13526844,694.09799194)
\lineto(130.13526844,690.75299194)
\lineto(130.13526844,690.16799194)
\curveto(130.1352607,689.97799081)(130.15526068,689.80299099)(130.19526844,689.64299194)
\curveto(130.20526063,689.61299118)(130.21026063,689.57799121)(130.21026844,689.53799194)
\curveto(130.21026063,689.50799128)(130.21526062,689.47799131)(130.22526844,689.44799194)
\moveto(128.02026844,691.75799194)
\curveto(128.03026281,691.80798898)(128.0352628,691.86298893)(128.03526844,691.92299194)
\curveto(128.0352628,691.9929888)(128.03026281,692.05298874)(128.02026844,692.10299194)
\curveto(128.00026284,692.16298863)(127.99026285,692.21798857)(127.99026844,692.26799194)
\curveto(127.99026285,692.31798847)(127.97026287,692.35798843)(127.93026844,692.38799194)
\curveto(127.88026296,692.42798836)(127.80526303,692.44798834)(127.70526844,692.44799194)
\curveto(127.66526317,692.43798835)(127.63026321,692.42798836)(127.60026844,692.41799194)
\curveto(127.57026327,692.41798837)(127.5352633,692.41298838)(127.49526844,692.40299194)
\curveto(127.42526341,692.38298841)(127.35026349,692.36798842)(127.27026844,692.35799194)
\curveto(127.19026365,692.34798844)(127.11026373,692.33298846)(127.03026844,692.31299194)
\curveto(127.00026384,692.30298849)(126.95526388,692.29798849)(126.89526844,692.29799194)
\curveto(126.76526407,692.26798852)(126.6352642,692.24798854)(126.50526844,692.23799194)
\curveto(126.37526446,692.22798856)(126.25026459,692.20298859)(126.13026844,692.16299194)
\curveto(126.05026479,692.14298865)(125.97526486,692.12298867)(125.90526844,692.10299194)
\curveto(125.835265,692.0929887)(125.76526507,692.07298872)(125.69526844,692.04299194)
\curveto(125.48526535,691.95298884)(125.30526553,691.81798897)(125.15526844,691.63799194)
\curveto(125.01526582,691.45798933)(124.96526587,691.20798958)(125.00526844,690.88799194)
\curveto(125.02526581,690.71799007)(125.08026576,690.57799021)(125.17026844,690.46799194)
\curveto(125.2402656,690.35799043)(125.34526549,690.26799052)(125.48526844,690.19799194)
\curveto(125.62526521,690.13799065)(125.77526506,690.0929907)(125.93526844,690.06299194)
\curveto(126.10526473,690.03299076)(126.28026456,690.02299077)(126.46026844,690.03299194)
\curveto(126.65026419,690.05299074)(126.82526401,690.0879907)(126.98526844,690.13799194)
\curveto(127.24526359,690.21799057)(127.45026339,690.34299045)(127.60026844,690.51299194)
\curveto(127.75026309,690.6929901)(127.86526297,690.91298988)(127.94526844,691.17299194)
\curveto(127.96526287,691.24298955)(127.97526286,691.31298948)(127.97526844,691.38299194)
\curveto(127.98526285,691.46298933)(128.00026284,691.54298925)(128.02026844,691.62299194)
\lineto(128.02026844,691.75799194)
}
}
{
\newrgbcolor{curcolor}{0 0 0}
\pscustom[linestyle=none,fillstyle=solid,fillcolor=curcolor]
{
\newpath
\moveto(135.35854969,696.79799194)
\curveto(136.16854453,696.81798397)(136.84354386,696.69798409)(137.38354969,696.43799194)
\curveto(137.93354277,696.17798461)(138.36854233,695.80798498)(138.68854969,695.32799194)
\curveto(138.84854185,695.0879857)(138.96854173,694.81298598)(139.04854969,694.50299194)
\curveto(139.06854163,694.45298634)(139.08354162,694.3879864)(139.09354969,694.30799194)
\curveto(139.11354159,694.22798656)(139.11354159,694.15798663)(139.09354969,694.09799194)
\curveto(139.05354165,693.9879868)(138.98354172,693.92298687)(138.88354969,693.90299194)
\curveto(138.78354192,693.8929869)(138.66354204,693.8879869)(138.52354969,693.88799194)
\lineto(137.74354969,693.88799194)
\lineto(137.45854969,693.88799194)
\curveto(137.36854333,693.8879869)(137.29354341,693.90798688)(137.23354969,693.94799194)
\curveto(137.15354355,693.9879868)(137.0985436,694.04798674)(137.06854969,694.12799194)
\curveto(137.03854366,694.21798657)(136.9985437,694.30798648)(136.94854969,694.39799194)
\curveto(136.88854381,694.50798628)(136.82354388,694.60798618)(136.75354969,694.69799194)
\curveto(136.68354402,694.787986)(136.6035441,694.86798592)(136.51354969,694.93799194)
\curveto(136.37354433,695.02798576)(136.21854448,695.09798569)(136.04854969,695.14799194)
\curveto(135.98854471,695.16798562)(135.92854477,695.17798561)(135.86854969,695.17799194)
\curveto(135.80854489,695.17798561)(135.75354495,695.1879856)(135.70354969,695.20799194)
\lineto(135.55354969,695.20799194)
\curveto(135.35354535,695.20798558)(135.19354551,695.1879856)(135.07354969,695.14799194)
\curveto(134.78354592,695.05798573)(134.54854615,694.91798587)(134.36854969,694.72799194)
\curveto(134.18854651,694.54798624)(134.04354666,694.32798646)(133.93354969,694.06799194)
\curveto(133.88354682,693.95798683)(133.84354686,693.83798695)(133.81354969,693.70799194)
\curveto(133.79354691,693.5879872)(133.76854693,693.45798733)(133.73854969,693.31799194)
\curveto(133.72854697,693.27798751)(133.72354698,693.23798755)(133.72354969,693.19799194)
\curveto(133.72354698,693.15798763)(133.71854698,693.11798767)(133.70854969,693.07799194)
\curveto(133.68854701,692.97798781)(133.67854702,692.83798795)(133.67854969,692.65799194)
\curveto(133.68854701,692.47798831)(133.703547,692.33798845)(133.72354969,692.23799194)
\curveto(133.72354698,692.15798863)(133.72854697,692.10298869)(133.73854969,692.07299194)
\curveto(133.75854694,692.00298879)(133.76854693,691.93298886)(133.76854969,691.86299194)
\curveto(133.77854692,691.792989)(133.79354691,691.72298907)(133.81354969,691.65299194)
\curveto(133.89354681,691.42298937)(133.98854671,691.21298958)(134.09854969,691.02299194)
\curveto(134.20854649,690.83298996)(134.34854635,690.67299012)(134.51854969,690.54299194)
\curveto(134.55854614,690.51299028)(134.61854608,690.47799031)(134.69854969,690.43799194)
\curveto(134.80854589,690.36799042)(134.91854578,690.32299047)(135.02854969,690.30299194)
\curveto(135.14854555,690.28299051)(135.29354541,690.26299053)(135.46354969,690.24299194)
\lineto(135.55354969,690.24299194)
\curveto(135.59354511,690.24299055)(135.62354508,690.24799054)(135.64354969,690.25799194)
\lineto(135.77854969,690.25799194)
\curveto(135.84854485,690.27799051)(135.91354479,690.2929905)(135.97354969,690.30299194)
\curveto(136.04354466,690.32299047)(136.10854459,690.34299045)(136.16854969,690.36299194)
\curveto(136.46854423,690.4929903)(136.698544,690.68299011)(136.85854969,690.93299194)
\curveto(136.8985438,690.98298981)(136.93354377,691.03798975)(136.96354969,691.09799194)
\curveto(136.99354371,691.16798962)(137.01854368,691.22798956)(137.03854969,691.27799194)
\curveto(137.07854362,691.3879894)(137.11354359,691.48298931)(137.14354969,691.56299194)
\curveto(137.17354353,691.65298914)(137.24354346,691.72298907)(137.35354969,691.77299194)
\curveto(137.44354326,691.81298898)(137.58854311,691.82798896)(137.78854969,691.81799194)
\lineto(138.28354969,691.81799194)
\lineto(138.49354969,691.81799194)
\curveto(138.57354213,691.82798896)(138.63854206,691.82298897)(138.68854969,691.80299194)
\lineto(138.80854969,691.80299194)
\lineto(138.92854969,691.77299194)
\curveto(138.96854173,691.77298902)(138.9985417,691.76298903)(139.01854969,691.74299194)
\curveto(139.06854163,691.70298909)(139.0985416,691.64298915)(139.10854969,691.56299194)
\curveto(139.12854157,691.4929893)(139.12854157,691.41798937)(139.10854969,691.33799194)
\curveto(139.01854168,691.00798978)(138.90854179,690.71299008)(138.77854969,690.45299194)
\curveto(138.36854233,689.68299111)(137.71354299,689.14799164)(136.81354969,688.84799194)
\curveto(136.71354399,688.81799197)(136.60854409,688.79799199)(136.49854969,688.78799194)
\curveto(136.38854431,688.76799202)(136.27854442,688.74299205)(136.16854969,688.71299194)
\curveto(136.10854459,688.70299209)(136.04854465,688.69799209)(135.98854969,688.69799194)
\curveto(135.92854477,688.69799209)(135.86854483,688.6929921)(135.80854969,688.68299194)
\lineto(135.64354969,688.68299194)
\curveto(135.59354511,688.66299213)(135.51854518,688.65799213)(135.41854969,688.66799194)
\curveto(135.31854538,688.66799212)(135.24354546,688.67299212)(135.19354969,688.68299194)
\curveto(135.11354559,688.70299209)(135.03854566,688.71299208)(134.96854969,688.71299194)
\curveto(134.90854579,688.70299209)(134.84354586,688.70799208)(134.77354969,688.72799194)
\lineto(134.62354969,688.75799194)
\curveto(134.57354613,688.75799203)(134.52354618,688.76299203)(134.47354969,688.77299194)
\curveto(134.36354634,688.80299199)(134.25854644,688.83299196)(134.15854969,688.86299194)
\curveto(134.05854664,688.8929919)(133.96354674,688.92799186)(133.87354969,688.96799194)
\curveto(133.4035473,689.16799162)(133.00854769,689.42299137)(132.68854969,689.73299194)
\curveto(132.36854833,690.05299074)(132.10854859,690.44799034)(131.90854969,690.91799194)
\curveto(131.85854884,691.00798978)(131.81854888,691.10298969)(131.78854969,691.20299194)
\lineto(131.69854969,691.53299194)
\curveto(131.68854901,691.57298922)(131.68354902,691.60798918)(131.68354969,691.63799194)
\curveto(131.68354902,691.67798911)(131.67354903,691.72298907)(131.65354969,691.77299194)
\curveto(131.63354907,691.84298895)(131.62354908,691.91298888)(131.62354969,691.98299194)
\curveto(131.62354908,692.06298873)(131.61354909,692.13798865)(131.59354969,692.20799194)
\lineto(131.59354969,692.46299194)
\curveto(131.57354913,692.51298828)(131.56354914,692.56798822)(131.56354969,692.62799194)
\curveto(131.56354914,692.69798809)(131.57354913,692.75798803)(131.59354969,692.80799194)
\curveto(131.6035491,692.85798793)(131.6035491,692.90298789)(131.59354969,692.94299194)
\curveto(131.58354912,692.98298781)(131.58354912,693.02298777)(131.59354969,693.06299194)
\curveto(131.61354909,693.13298766)(131.61854908,693.19798759)(131.60854969,693.25799194)
\curveto(131.60854909,693.31798747)(131.61854908,693.37798741)(131.63854969,693.43799194)
\curveto(131.68854901,693.61798717)(131.72854897,693.787987)(131.75854969,693.94799194)
\curveto(131.78854891,694.11798667)(131.83354887,694.28298651)(131.89354969,694.44299194)
\curveto(132.11354859,694.95298584)(132.38854831,695.37798541)(132.71854969,695.71799194)
\curveto(133.05854764,696.05798473)(133.48854721,696.33298446)(134.00854969,696.54299194)
\curveto(134.14854655,696.60298419)(134.29354641,696.64298415)(134.44354969,696.66299194)
\curveto(134.59354611,696.6929841)(134.74854595,696.72798406)(134.90854969,696.76799194)
\curveto(134.98854571,696.77798401)(135.06354564,696.78298401)(135.13354969,696.78299194)
\curveto(135.2035455,696.78298401)(135.27854542,696.787984)(135.35854969,696.79799194)
}
}
{
\newrgbcolor{curcolor}{0 0 0}
\pscustom[linestyle=none,fillstyle=solid,fillcolor=curcolor]
{
\newpath
\moveto(141.45183094,698.89799194)
\lineto(142.45683094,698.89799194)
\curveto(142.60682796,698.89798189)(142.73682783,698.8879819)(142.84683094,698.86799194)
\curveto(142.9668276,698.85798193)(143.05182751,698.79798199)(143.10183094,698.68799194)
\curveto(143.12182744,698.63798215)(143.13182743,698.57798221)(143.13183094,698.50799194)
\lineto(143.13183094,698.29799194)
\lineto(143.13183094,697.62299194)
\curveto(143.13182743,697.57298322)(143.12682744,697.51298328)(143.11683094,697.44299194)
\curveto(143.11682745,697.38298341)(143.12182744,697.32798346)(143.13183094,697.27799194)
\lineto(143.13183094,697.11299194)
\curveto(143.13182743,697.03298376)(143.13682743,696.95798383)(143.14683094,696.88799194)
\curveto(143.15682741,696.82798396)(143.18182738,696.77298402)(143.22183094,696.72299194)
\curveto(143.29182727,696.63298416)(143.41682715,696.58298421)(143.59683094,696.57299194)
\lineto(144.13683094,696.57299194)
\lineto(144.31683094,696.57299194)
\curveto(144.37682619,696.57298422)(144.43182613,696.56298423)(144.48183094,696.54299194)
\curveto(144.59182597,696.4929843)(144.65182591,696.40298439)(144.66183094,696.27299194)
\curveto(144.68182588,696.14298465)(144.69182587,695.99798479)(144.69183094,695.83799194)
\lineto(144.69183094,695.62799194)
\curveto(144.70182586,695.55798523)(144.69682587,695.49798529)(144.67683094,695.44799194)
\curveto(144.62682594,695.2879855)(144.52182604,695.20298559)(144.36183094,695.19299194)
\curveto(144.20182636,695.18298561)(144.02182654,695.17798561)(143.82183094,695.17799194)
\lineto(143.68683094,695.17799194)
\curveto(143.64682692,695.1879856)(143.61182695,695.1879856)(143.58183094,695.17799194)
\curveto(143.54182702,695.16798562)(143.50682706,695.16298563)(143.47683094,695.16299194)
\curveto(143.44682712,695.17298562)(143.41682715,695.16798562)(143.38683094,695.14799194)
\curveto(143.30682726,695.12798566)(143.24682732,695.08298571)(143.20683094,695.01299194)
\curveto(143.17682739,694.95298584)(143.15182741,694.87798591)(143.13183094,694.78799194)
\curveto(143.12182744,694.73798605)(143.12182744,694.68298611)(143.13183094,694.62299194)
\curveto(143.14182742,694.56298623)(143.14182742,694.50798628)(143.13183094,694.45799194)
\lineto(143.13183094,693.52799194)
\lineto(143.13183094,691.77299194)
\curveto(143.13182743,691.52298927)(143.13682743,691.30298949)(143.14683094,691.11299194)
\curveto(143.1668274,690.93298986)(143.23182733,690.77299002)(143.34183094,690.63299194)
\curveto(143.39182717,690.57299022)(143.45682711,690.52799026)(143.53683094,690.49799194)
\lineto(143.80683094,690.43799194)
\curveto(143.83682673,690.42799036)(143.8668267,690.42299037)(143.89683094,690.42299194)
\curveto(143.93682663,690.43299036)(143.9668266,690.43299036)(143.98683094,690.42299194)
\lineto(144.15183094,690.42299194)
\curveto(144.2618263,690.42299037)(144.35682621,690.41799037)(144.43683094,690.40799194)
\curveto(144.51682605,690.39799039)(144.58182598,690.35799043)(144.63183094,690.28799194)
\curveto(144.67182589,690.22799056)(144.69182587,690.14799064)(144.69183094,690.04799194)
\lineto(144.69183094,689.76299194)
\curveto(144.69182587,689.55299124)(144.68682588,689.35799143)(144.67683094,689.17799194)
\curveto(144.67682589,689.00799178)(144.59682597,688.8929919)(144.43683094,688.83299194)
\curveto(144.38682618,688.81299198)(144.34182622,688.80799198)(144.30183094,688.81799194)
\curveto(144.2618263,688.81799197)(144.21682635,688.80799198)(144.16683094,688.78799194)
\lineto(144.01683094,688.78799194)
\curveto(143.99682657,688.787992)(143.9668266,688.792992)(143.92683094,688.80299194)
\curveto(143.88682668,688.80299199)(143.85182671,688.79799199)(143.82183094,688.78799194)
\curveto(143.77182679,688.77799201)(143.71682685,688.77799201)(143.65683094,688.78799194)
\lineto(143.50683094,688.78799194)
\lineto(143.35683094,688.78799194)
\curveto(143.30682726,688.77799201)(143.2618273,688.77799201)(143.22183094,688.78799194)
\lineto(143.05683094,688.78799194)
\curveto(143.00682756,688.79799199)(142.95182761,688.80299199)(142.89183094,688.80299194)
\curveto(142.83182773,688.80299199)(142.77682779,688.80799198)(142.72683094,688.81799194)
\curveto(142.65682791,688.82799196)(142.59182797,688.83799195)(142.53183094,688.84799194)
\lineto(142.35183094,688.87799194)
\curveto(142.24182832,688.90799188)(142.13682843,688.94299185)(142.03683094,688.98299194)
\curveto(141.93682863,689.02299177)(141.84182872,689.06799172)(141.75183094,689.11799194)
\lineto(141.66183094,689.17799194)
\curveto(141.63182893,689.20799158)(141.59682897,689.23799155)(141.55683094,689.26799194)
\curveto(141.53682903,689.2879915)(141.51182905,689.30799148)(141.48183094,689.32799194)
\lineto(141.40683094,689.40299194)
\curveto(141.2668293,689.5929912)(141.1618294,689.80299099)(141.09183094,690.03299194)
\curveto(141.07182949,690.07299072)(141.0618295,690.10799068)(141.06183094,690.13799194)
\curveto(141.07182949,690.17799061)(141.07182949,690.22299057)(141.06183094,690.27299194)
\curveto(141.05182951,690.2929905)(141.04682952,690.31799047)(141.04683094,690.34799194)
\curveto(141.04682952,690.37799041)(141.04182952,690.40299039)(141.03183094,690.42299194)
\lineto(141.03183094,690.57299194)
\curveto(141.02182954,690.61299018)(141.01682955,690.65799013)(141.01683094,690.70799194)
\curveto(141.02682954,690.75799003)(141.03182953,690.80798998)(141.03183094,690.85799194)
\lineto(141.03183094,691.42799194)
\lineto(141.03183094,693.66299194)
\lineto(141.03183094,694.45799194)
\lineto(141.03183094,694.66799194)
\curveto(141.04182952,694.73798605)(141.03682953,694.80298599)(141.01683094,694.86299194)
\curveto(140.97682959,695.00298579)(140.90682966,695.0929857)(140.80683094,695.13299194)
\curveto(140.69682987,695.18298561)(140.55683001,695.19798559)(140.38683094,695.17799194)
\curveto(140.21683035,695.15798563)(140.07183049,695.17298562)(139.95183094,695.22299194)
\curveto(139.87183069,695.25298554)(139.82183074,695.29798549)(139.80183094,695.35799194)
\curveto(139.78183078,695.41798537)(139.7618308,695.4929853)(139.74183094,695.58299194)
\lineto(139.74183094,695.89799194)
\curveto(139.74183082,696.07798471)(139.75183081,696.22298457)(139.77183094,696.33299194)
\curveto(139.79183077,696.44298435)(139.87683069,696.51798427)(140.02683094,696.55799194)
\curveto(140.0668305,696.57798421)(140.10683046,696.58298421)(140.14683094,696.57299194)
\lineto(140.28183094,696.57299194)
\curveto(140.43183013,696.57298422)(140.57182999,696.57798421)(140.70183094,696.58799194)
\curveto(140.83182973,696.60798418)(140.92182964,696.66798412)(140.97183094,696.76799194)
\curveto(141.00182956,696.83798395)(141.01682955,696.91798387)(141.01683094,697.00799194)
\curveto(141.02682954,697.09798369)(141.03182953,697.1879836)(141.03183094,697.27799194)
\lineto(141.03183094,698.20799194)
\lineto(141.03183094,698.46299194)
\curveto(141.03182953,698.55298224)(141.04182952,698.62798216)(141.06183094,698.68799194)
\curveto(141.11182945,698.787982)(141.18682938,698.85298194)(141.28683094,698.88299194)
\curveto(141.30682926,698.8929819)(141.33182923,698.8929819)(141.36183094,698.88299194)
\curveto(141.40182916,698.88298191)(141.43182913,698.8879819)(141.45183094,698.89799194)
}
}
{
\newrgbcolor{curcolor}{0 0 0}
\pscustom[linestyle=none,fillstyle=solid,fillcolor=curcolor]
{
\newpath
\moveto(147.77526844,699.43799194)
\curveto(147.84526549,699.35798143)(147.88026546,699.23798155)(147.88026844,699.07799194)
\lineto(147.88026844,698.61299194)
\lineto(147.88026844,698.20799194)
\curveto(147.88026546,698.06798272)(147.84526549,697.97298282)(147.77526844,697.92299194)
\curveto(147.71526562,697.87298292)(147.6352657,697.84298295)(147.53526844,697.83299194)
\curveto(147.44526589,697.82298297)(147.34526599,697.81798297)(147.23526844,697.81799194)
\lineto(146.39526844,697.81799194)
\curveto(146.28526705,697.81798297)(146.18526715,697.82298297)(146.09526844,697.83299194)
\curveto(146.01526732,697.84298295)(145.94526739,697.87298292)(145.88526844,697.92299194)
\curveto(145.84526749,697.95298284)(145.81526752,698.00798278)(145.79526844,698.08799194)
\curveto(145.78526755,698.17798261)(145.77526756,698.27298252)(145.76526844,698.37299194)
\lineto(145.76526844,698.70299194)
\curveto(145.77526756,698.81298198)(145.78026756,698.90798188)(145.78026844,698.98799194)
\lineto(145.78026844,699.19799194)
\curveto(145.79026755,699.26798152)(145.81026753,699.32798146)(145.84026844,699.37799194)
\curveto(145.86026748,699.41798137)(145.88526745,699.44798134)(145.91526844,699.46799194)
\lineto(146.03526844,699.52799194)
\curveto(146.05526728,699.52798126)(146.08026726,699.52798126)(146.11026844,699.52799194)
\curveto(146.1402672,699.53798125)(146.16526717,699.54298125)(146.18526844,699.54299194)
\lineto(147.28026844,699.54299194)
\curveto(147.38026596,699.54298125)(147.47526586,699.53798125)(147.56526844,699.52799194)
\curveto(147.65526568,699.51798127)(147.72526561,699.4879813)(147.77526844,699.43799194)
\moveto(147.88026844,689.67299194)
\curveto(147.88026546,689.47299132)(147.87526546,689.30299149)(147.86526844,689.16299194)
\curveto(147.85526548,689.02299177)(147.76526557,688.92799186)(147.59526844,688.87799194)
\curveto(147.5352658,688.85799193)(147.47026587,688.84799194)(147.40026844,688.84799194)
\curveto(147.33026601,688.85799193)(147.25526608,688.86299193)(147.17526844,688.86299194)
\lineto(146.33526844,688.86299194)
\curveto(146.24526709,688.86299193)(146.15526718,688.86799192)(146.06526844,688.87799194)
\curveto(145.98526735,688.8879919)(145.92526741,688.91799187)(145.88526844,688.96799194)
\curveto(145.82526751,689.03799175)(145.79026755,689.12299167)(145.78026844,689.22299194)
\lineto(145.78026844,689.56799194)
\lineto(145.78026844,695.89799194)
\lineto(145.78026844,696.19799194)
\curveto(145.78026756,696.29798449)(145.80026754,696.37798441)(145.84026844,696.43799194)
\curveto(145.90026744,696.50798428)(145.98526735,696.55298424)(146.09526844,696.57299194)
\curveto(146.11526722,696.58298421)(146.1402672,696.58298421)(146.17026844,696.57299194)
\curveto(146.21026713,696.57298422)(146.2402671,696.57798421)(146.26026844,696.58799194)
\lineto(147.01026844,696.58799194)
\lineto(147.20526844,696.58799194)
\curveto(147.28526605,696.59798419)(147.35026599,696.59798419)(147.40026844,696.58799194)
\lineto(147.52026844,696.58799194)
\curveto(147.58026576,696.56798422)(147.6352657,696.55298424)(147.68526844,696.54299194)
\curveto(147.7352656,696.53298426)(147.77526556,696.50298429)(147.80526844,696.45299194)
\curveto(147.84526549,696.40298439)(147.86526547,696.33298446)(147.86526844,696.24299194)
\curveto(147.87526546,696.15298464)(147.88026546,696.05798473)(147.88026844,695.95799194)
\lineto(147.88026844,689.67299194)
}
}
{
\newrgbcolor{curcolor}{0 0 0}
\pscustom[linestyle=none,fillstyle=solid,fillcolor=curcolor]
{
\newpath
\moveto(149.28745594,696.58799194)
\lineto(150.38245594,696.58799194)
\curveto(150.49245421,696.5879842)(150.59745411,696.58298421)(150.69745594,696.57299194)
\curveto(150.79745391,696.57298422)(150.88245382,696.55298424)(150.95245594,696.51299194)
\curveto(151.06245364,696.44298435)(151.13745357,696.31298448)(151.17745594,696.12299194)
\curveto(151.22745348,695.94298485)(151.27745343,695.78298501)(151.32745594,695.64299194)
\curveto(151.43745327,695.31298548)(151.54245316,694.97298582)(151.64245594,694.62299194)
\curveto(151.74245296,694.28298651)(151.84745286,693.94298685)(151.95745594,693.60299194)
\curveto(152.03745267,693.36298743)(152.11245259,693.11798767)(152.18245594,692.86799194)
\curveto(152.25245245,692.61798817)(152.33245237,692.37798841)(152.42245594,692.14799194)
\curveto(152.45245225,692.05798873)(152.48245222,691.96298883)(152.51245594,691.86299194)
\curveto(152.55245215,691.76298903)(152.62745208,691.71298908)(152.73745594,691.71299194)
\curveto(152.75745195,691.73298906)(152.77245193,691.74298905)(152.78245594,691.74299194)
\lineto(152.82745594,691.78799194)
\curveto(152.85745185,691.83798895)(152.88245182,691.8879889)(152.90245594,691.93799194)
\curveto(152.92245178,691.9879888)(152.94245176,692.04298875)(152.96245594,692.10299194)
\curveto(153.01245169,692.21298858)(153.04745166,692.32798846)(153.06745594,692.44799194)
\curveto(153.09745161,692.56798822)(153.13245157,692.68298811)(153.17245594,692.79299194)
\curveto(153.31245139,693.21298758)(153.44245126,693.63298716)(153.56245594,694.05299194)
\curveto(153.69245101,694.48298631)(153.82745088,694.90798588)(153.96745594,695.32799194)
\curveto(154.01745069,695.44798534)(154.05745065,695.56798522)(154.08745594,695.68799194)
\curveto(154.11745059,695.81798497)(154.15245055,695.93798485)(154.19245594,696.04799194)
\curveto(154.21245049,696.12798466)(154.23745047,696.20798458)(154.26745594,696.28799194)
\curveto(154.29745041,696.36798442)(154.34245036,696.43298436)(154.40245594,696.48299194)
\curveto(154.43245027,696.50298429)(154.49745021,696.53298426)(154.59745594,696.57299194)
\curveto(154.65745005,696.5929842)(154.72244998,696.59798419)(154.79245594,696.58799194)
\lineto(154.98745594,696.58799194)
\lineto(155.72245594,696.58799194)
\curveto(155.83244887,696.5879842)(155.93244877,696.58298421)(156.02245594,696.57299194)
\curveto(156.12244858,696.57298422)(156.19744851,696.54798424)(156.24745594,696.49799194)
\curveto(156.3074484,696.44798434)(156.32744838,696.37298442)(156.30745594,696.27299194)
\curveto(156.29744841,696.18298461)(156.28244842,696.10798468)(156.26245594,696.04799194)
\curveto(156.21244849,695.90798488)(156.16244854,695.75798503)(156.11245594,695.59799194)
\curveto(156.06244864,695.44798534)(156.01244869,695.30298549)(155.96245594,695.16299194)
\curveto(155.93244877,695.0929857)(155.9074488,695.02298577)(155.88745594,694.95299194)
\curveto(155.87744883,694.8929859)(155.86244884,694.83298596)(155.84245594,694.77299194)
\curveto(155.79244891,694.66298613)(155.74744896,694.55298624)(155.70745594,694.44299194)
\curveto(155.67744903,694.33298646)(155.64244906,694.22298657)(155.60245594,694.11299194)
\curveto(155.59244911,694.08298671)(155.58244912,694.04798674)(155.57245594,694.00799194)
\curveto(155.57244913,693.97798681)(155.56244914,693.94798684)(155.54245594,693.91799194)
\curveto(155.48244922,693.76798702)(155.42744928,693.61298718)(155.37745594,693.45299194)
\curveto(155.33744937,693.30298749)(155.29244941,693.15298764)(155.24245594,693.00299194)
\curveto(155.13244957,692.70298809)(155.02744968,692.3929884)(154.92745594,692.07299194)
\curveto(154.82744988,691.76298903)(154.71744999,691.45798933)(154.59745594,691.15799194)
\curveto(154.54745016,691.01798977)(154.5024502,690.87798991)(154.46245594,690.73799194)
\curveto(154.42245028,690.60799018)(154.37745033,690.47299032)(154.32745594,690.33299194)
\curveto(154.3074504,690.28299051)(154.29245041,690.23799055)(154.28245594,690.19799194)
\curveto(154.27245043,690.16799062)(154.25745045,690.12799066)(154.23745594,690.07799194)
\curveto(154.19745051,689.96799082)(154.15745055,689.85299094)(154.11745594,689.73299194)
\curveto(154.08745062,689.62299117)(154.05245065,689.51299128)(154.01245594,689.40299194)
\curveto(153.96245074,689.2929915)(153.91745079,689.1879916)(153.87745594,689.08799194)
\curveto(153.83745087,688.99799179)(153.76245094,688.93299186)(153.65245594,688.89299194)
\curveto(153.57245113,688.86299193)(153.48745122,688.84799194)(153.39745594,688.84799194)
\curveto(153.3074514,688.85799193)(153.21745149,688.86299193)(153.12745594,688.86299194)
\lineto(152.15245594,688.86299194)
\lineto(151.82245594,688.86299194)
\curveto(151.72245298,688.86299193)(151.63245307,688.88299191)(151.55245594,688.92299194)
\curveto(151.48245322,688.97299182)(151.43245327,689.03299176)(151.40245594,689.10299194)
\curveto(151.37245333,689.18299161)(151.34245336,689.26799152)(151.31245594,689.35799194)
\curveto(151.26245344,689.47799131)(151.21745349,689.60299119)(151.17745594,689.73299194)
\curveto(151.13745357,689.86299093)(151.09245361,689.9879908)(151.04245594,690.10799194)
\curveto(151.02245368,690.14799064)(151.0074537,690.18299061)(150.99745594,690.21299194)
\curveto(150.99745371,690.25299054)(150.98745372,690.29799049)(150.96745594,690.34799194)
\curveto(150.91745379,690.46799032)(150.87245383,690.5929902)(150.83245594,690.72299194)
\curveto(150.79245391,690.85298994)(150.74745396,690.98298981)(150.69745594,691.11299194)
\curveto(150.67745403,691.16298963)(150.66245404,691.20798958)(150.65245594,691.24799194)
\curveto(150.64245406,691.29798949)(150.62745408,691.34798944)(150.60745594,691.39799194)
\curveto(150.56745414,691.4879893)(150.53245417,691.57798921)(150.50245594,691.66799194)
\curveto(150.47245423,691.76798902)(150.44245426,691.86298893)(150.41245594,691.95299194)
\curveto(150.39245431,691.98298881)(150.38245432,692.01298878)(150.38245594,692.04299194)
\curveto(150.38245432,692.08298871)(150.37245433,692.11798867)(150.35245594,692.14799194)
\curveto(150.31245439,692.23798855)(150.27745443,692.32798846)(150.24745594,692.41799194)
\curveto(150.22745448,692.50798828)(150.19745451,692.60298819)(150.15745594,692.70299194)
\curveto(150.06745464,692.93298786)(149.98245472,693.16798762)(149.90245594,693.40799194)
\curveto(149.82245488,693.65798713)(149.74245496,693.90298689)(149.66245594,694.14299194)
\curveto(149.55245515,694.41298638)(149.45745525,694.68298611)(149.37745594,694.95299194)
\curveto(149.29745541,695.23298556)(149.2074555,695.50798528)(149.10745594,695.77799194)
\lineto(148.97245594,696.18299194)
\curveto(148.96245574,696.21298458)(148.95245575,696.24798454)(148.94245594,696.28799194)
\curveto(148.94245576,696.33798445)(148.95245575,696.38298441)(148.97245594,696.42299194)
\curveto(149.0024557,696.4929843)(149.07245563,696.54298425)(149.18245594,696.57299194)
\curveto(149.23245547,696.57298422)(149.26745544,696.57798421)(149.28745594,696.58799194)
}
}
{
\newrgbcolor{curcolor}{0 0 0}
\pscustom[linestyle=none,fillstyle=solid,fillcolor=curcolor]
{
\newpath
\moveto(159.43542469,699.43799194)
\curveto(159.50542174,699.35798143)(159.54042171,699.23798155)(159.54042469,699.07799194)
\lineto(159.54042469,698.61299194)
\lineto(159.54042469,698.20799194)
\curveto(159.54042171,698.06798272)(159.50542174,697.97298282)(159.43542469,697.92299194)
\curveto(159.37542187,697.87298292)(159.29542195,697.84298295)(159.19542469,697.83299194)
\curveto(159.10542214,697.82298297)(159.00542224,697.81798297)(158.89542469,697.81799194)
\lineto(158.05542469,697.81799194)
\curveto(157.9454233,697.81798297)(157.8454234,697.82298297)(157.75542469,697.83299194)
\curveto(157.67542357,697.84298295)(157.60542364,697.87298292)(157.54542469,697.92299194)
\curveto(157.50542374,697.95298284)(157.47542377,698.00798278)(157.45542469,698.08799194)
\curveto(157.4454238,698.17798261)(157.43542381,698.27298252)(157.42542469,698.37299194)
\lineto(157.42542469,698.70299194)
\curveto(157.43542381,698.81298198)(157.44042381,698.90798188)(157.44042469,698.98799194)
\lineto(157.44042469,699.19799194)
\curveto(157.4504238,699.26798152)(157.47042378,699.32798146)(157.50042469,699.37799194)
\curveto(157.52042373,699.41798137)(157.5454237,699.44798134)(157.57542469,699.46799194)
\lineto(157.69542469,699.52799194)
\curveto(157.71542353,699.52798126)(157.74042351,699.52798126)(157.77042469,699.52799194)
\curveto(157.80042345,699.53798125)(157.82542342,699.54298125)(157.84542469,699.54299194)
\lineto(158.94042469,699.54299194)
\curveto(159.04042221,699.54298125)(159.13542211,699.53798125)(159.22542469,699.52799194)
\curveto(159.31542193,699.51798127)(159.38542186,699.4879813)(159.43542469,699.43799194)
\moveto(159.54042469,689.67299194)
\curveto(159.54042171,689.47299132)(159.53542171,689.30299149)(159.52542469,689.16299194)
\curveto(159.51542173,689.02299177)(159.42542182,688.92799186)(159.25542469,688.87799194)
\curveto(159.19542205,688.85799193)(159.13042212,688.84799194)(159.06042469,688.84799194)
\curveto(158.99042226,688.85799193)(158.91542233,688.86299193)(158.83542469,688.86299194)
\lineto(157.99542469,688.86299194)
\curveto(157.90542334,688.86299193)(157.81542343,688.86799192)(157.72542469,688.87799194)
\curveto(157.6454236,688.8879919)(157.58542366,688.91799187)(157.54542469,688.96799194)
\curveto(157.48542376,689.03799175)(157.4504238,689.12299167)(157.44042469,689.22299194)
\lineto(157.44042469,689.56799194)
\lineto(157.44042469,695.89799194)
\lineto(157.44042469,696.19799194)
\curveto(157.44042381,696.29798449)(157.46042379,696.37798441)(157.50042469,696.43799194)
\curveto(157.56042369,696.50798428)(157.6454236,696.55298424)(157.75542469,696.57299194)
\curveto(157.77542347,696.58298421)(157.80042345,696.58298421)(157.83042469,696.57299194)
\curveto(157.87042338,696.57298422)(157.90042335,696.57798421)(157.92042469,696.58799194)
\lineto(158.67042469,696.58799194)
\lineto(158.86542469,696.58799194)
\curveto(158.9454223,696.59798419)(159.01042224,696.59798419)(159.06042469,696.58799194)
\lineto(159.18042469,696.58799194)
\curveto(159.24042201,696.56798422)(159.29542195,696.55298424)(159.34542469,696.54299194)
\curveto(159.39542185,696.53298426)(159.43542181,696.50298429)(159.46542469,696.45299194)
\curveto(159.50542174,696.40298439)(159.52542172,696.33298446)(159.52542469,696.24299194)
\curveto(159.53542171,696.15298464)(159.54042171,696.05798473)(159.54042469,695.95799194)
\lineto(159.54042469,689.67299194)
}
}
{
\newrgbcolor{curcolor}{0 0 0}
\pscustom[linestyle=none,fillstyle=solid,fillcolor=curcolor]
{
\newpath
\moveto(168.79261219,689.70299194)
\lineto(168.79261219,689.28299194)
\curveto(168.79260382,689.15299164)(168.76260385,689.04799174)(168.70261219,688.96799194)
\curveto(168.65260396,688.91799187)(168.58760403,688.88299191)(168.50761219,688.86299194)
\curveto(168.42760419,688.85299194)(168.33760428,688.84799194)(168.23761219,688.84799194)
\lineto(167.41261219,688.84799194)
\lineto(167.12761219,688.84799194)
\curveto(167.04760557,688.85799193)(166.98260563,688.88299191)(166.93261219,688.92299194)
\curveto(166.86260575,688.97299182)(166.82260579,689.03799175)(166.81261219,689.11799194)
\curveto(166.80260581,689.19799159)(166.78260583,689.27799151)(166.75261219,689.35799194)
\curveto(166.73260588,689.37799141)(166.7126059,689.3929914)(166.69261219,689.40299194)
\curveto(166.68260593,689.42299137)(166.66760595,689.44299135)(166.64761219,689.46299194)
\curveto(166.53760608,689.46299133)(166.45760616,689.43799135)(166.40761219,689.38799194)
\lineto(166.25761219,689.23799194)
\curveto(166.18760643,689.1879916)(166.12260649,689.14299165)(166.06261219,689.10299194)
\curveto(166.00260661,689.07299172)(165.93760668,689.03299176)(165.86761219,688.98299194)
\curveto(165.82760679,688.96299183)(165.78260683,688.94299185)(165.73261219,688.92299194)
\curveto(165.69260692,688.90299189)(165.64760697,688.88299191)(165.59761219,688.86299194)
\curveto(165.45760716,688.81299198)(165.30760731,688.76799202)(165.14761219,688.72799194)
\curveto(165.09760752,688.70799208)(165.05260756,688.69799209)(165.01261219,688.69799194)
\curveto(164.97260764,688.69799209)(164.93260768,688.6929921)(164.89261219,688.68299194)
\lineto(164.75761219,688.68299194)
\curveto(164.72760789,688.67299212)(164.68760793,688.66799212)(164.63761219,688.66799194)
\lineto(164.50261219,688.66799194)
\curveto(164.44260817,688.64799214)(164.35260826,688.64299215)(164.23261219,688.65299194)
\curveto(164.1126085,688.65299214)(164.02760859,688.66299213)(163.97761219,688.68299194)
\curveto(163.90760871,688.70299209)(163.84260877,688.71299208)(163.78261219,688.71299194)
\curveto(163.73260888,688.70299209)(163.67760894,688.70799208)(163.61761219,688.72799194)
\lineto(163.25761219,688.84799194)
\curveto(163.14760947,688.87799191)(163.03760958,688.91799187)(162.92761219,688.96799194)
\curveto(162.57761004,689.11799167)(162.26261035,689.34799144)(161.98261219,689.65799194)
\curveto(161.7126109,689.97799081)(161.49761112,690.31299048)(161.33761219,690.66299194)
\curveto(161.28761133,690.77299002)(161.24761137,690.87798991)(161.21761219,690.97799194)
\curveto(161.18761143,691.0879897)(161.15261146,691.19798959)(161.11261219,691.30799194)
\curveto(161.10261151,691.34798944)(161.09761152,691.38298941)(161.09761219,691.41299194)
\curveto(161.09761152,691.45298934)(161.08761153,691.49798929)(161.06761219,691.54799194)
\curveto(161.04761157,691.62798916)(161.02761159,691.71298908)(161.00761219,691.80299194)
\curveto(160.99761162,691.90298889)(160.98261163,692.00298879)(160.96261219,692.10299194)
\curveto(160.95261166,692.13298866)(160.94761167,692.16798862)(160.94761219,692.20799194)
\curveto(160.95761166,692.24798854)(160.95761166,692.28298851)(160.94761219,692.31299194)
\lineto(160.94761219,692.44799194)
\curveto(160.94761167,692.49798829)(160.94261167,692.54798824)(160.93261219,692.59799194)
\curveto(160.92261169,692.64798814)(160.9176117,692.70298809)(160.91761219,692.76299194)
\curveto(160.9176117,692.83298796)(160.92261169,692.8879879)(160.93261219,692.92799194)
\curveto(160.94261167,692.97798781)(160.94761167,693.02298777)(160.94761219,693.06299194)
\lineto(160.94761219,693.21299194)
\curveto(160.95761166,693.26298753)(160.95761166,693.30798748)(160.94761219,693.34799194)
\curveto(160.94761167,693.39798739)(160.95761166,693.44798734)(160.97761219,693.49799194)
\curveto(160.99761162,693.60798718)(161.0126116,693.71298708)(161.02261219,693.81299194)
\curveto(161.04261157,693.91298688)(161.06761155,694.01298678)(161.09761219,694.11299194)
\curveto(161.13761148,694.23298656)(161.17261144,694.34798644)(161.20261219,694.45799194)
\curveto(161.23261138,694.56798622)(161.27261134,694.67798611)(161.32261219,694.78799194)
\curveto(161.46261115,695.0879857)(161.63761098,695.37298542)(161.84761219,695.64299194)
\curveto(161.86761075,695.67298512)(161.89261072,695.69798509)(161.92261219,695.71799194)
\curveto(161.96261065,695.74798504)(161.99261062,695.77798501)(162.01261219,695.80799194)
\curveto(162.05261056,695.85798493)(162.09261052,695.90298489)(162.13261219,695.94299194)
\curveto(162.17261044,695.98298481)(162.2176104,696.02298477)(162.26761219,696.06299194)
\curveto(162.30761031,696.08298471)(162.34261027,696.10798468)(162.37261219,696.13799194)
\curveto(162.40261021,696.17798461)(162.43761018,696.20798458)(162.47761219,696.22799194)
\curveto(162.72760989,696.39798439)(163.0176096,696.53798425)(163.34761219,696.64799194)
\curveto(163.4176092,696.66798412)(163.48760913,696.68298411)(163.55761219,696.69299194)
\curveto(163.63760898,696.70298409)(163.7176089,696.71798407)(163.79761219,696.73799194)
\curveto(163.86760875,696.75798403)(163.95760866,696.76798402)(164.06761219,696.76799194)
\curveto(164.17760844,696.77798401)(164.28760833,696.78298401)(164.39761219,696.78299194)
\curveto(164.50760811,696.78298401)(164.612608,696.77798401)(164.71261219,696.76799194)
\curveto(164.82260779,696.75798403)(164.9126077,696.74298405)(164.98261219,696.72299194)
\curveto(165.13260748,696.67298412)(165.27760734,696.62798416)(165.41761219,696.58799194)
\curveto(165.55760706,696.54798424)(165.68760693,696.4929843)(165.80761219,696.42299194)
\curveto(165.87760674,696.37298442)(165.94260667,696.32298447)(166.00261219,696.27299194)
\curveto(166.06260655,696.23298456)(166.12760649,696.1879846)(166.19761219,696.13799194)
\curveto(166.23760638,696.10798468)(166.29260632,696.06798472)(166.36261219,696.01799194)
\curveto(166.44260617,695.96798482)(166.5176061,695.96798482)(166.58761219,696.01799194)
\curveto(166.62760599,696.03798475)(166.64760597,696.07298472)(166.64761219,696.12299194)
\curveto(166.64760597,696.17298462)(166.65760596,696.22298457)(166.67761219,696.27299194)
\lineto(166.67761219,696.42299194)
\curveto(166.68760593,696.45298434)(166.69260592,696.4879843)(166.69261219,696.52799194)
\lineto(166.69261219,696.64799194)
\lineto(166.69261219,698.68799194)
\curveto(166.69260592,698.79798199)(166.68760593,698.91798187)(166.67761219,699.04799194)
\curveto(166.67760594,699.1879816)(166.70260591,699.2929815)(166.75261219,699.36299194)
\curveto(166.79260582,699.44298135)(166.86760575,699.4929813)(166.97761219,699.51299194)
\curveto(166.99760562,699.52298127)(167.0176056,699.52298127)(167.03761219,699.51299194)
\curveto(167.05760556,699.51298128)(167.07760554,699.51798127)(167.09761219,699.52799194)
\lineto(168.16261219,699.52799194)
\curveto(168.28260433,699.52798126)(168.39260422,699.52298127)(168.49261219,699.51299194)
\curveto(168.59260402,699.50298129)(168.66760395,699.46298133)(168.71761219,699.39299194)
\curveto(168.76760385,699.31298148)(168.79260382,699.20798158)(168.79261219,699.07799194)
\lineto(168.79261219,698.71799194)
\lineto(168.79261219,689.70299194)
\moveto(166.75261219,692.64299194)
\curveto(166.76260585,692.68298811)(166.76260585,692.72298807)(166.75261219,692.76299194)
\lineto(166.75261219,692.89799194)
\curveto(166.75260586,692.99798779)(166.74760587,693.09798769)(166.73761219,693.19799194)
\curveto(166.72760589,693.29798749)(166.7126059,693.3879874)(166.69261219,693.46799194)
\curveto(166.67260594,693.57798721)(166.65260596,693.67798711)(166.63261219,693.76799194)
\curveto(166.62260599,693.85798693)(166.59760602,693.94298685)(166.55761219,694.02299194)
\curveto(166.4176062,694.38298641)(166.2126064,694.66798612)(165.94261219,694.87799194)
\curveto(165.68260693,695.0879857)(165.30260731,695.1929856)(164.80261219,695.19299194)
\curveto(164.74260787,695.1929856)(164.66260795,695.18298561)(164.56261219,695.16299194)
\curveto(164.48260813,695.14298565)(164.40760821,695.12298567)(164.33761219,695.10299194)
\curveto(164.27760834,695.0929857)(164.2176084,695.07298572)(164.15761219,695.04299194)
\curveto(163.88760873,694.93298586)(163.67760894,694.76298603)(163.52761219,694.53299194)
\curveto(163.37760924,694.30298649)(163.25760936,694.04298675)(163.16761219,693.75299194)
\curveto(163.13760948,693.65298714)(163.1176095,693.55298724)(163.10761219,693.45299194)
\curveto(163.09760952,693.35298744)(163.07760954,693.24798754)(163.04761219,693.13799194)
\lineto(163.04761219,692.92799194)
\curveto(163.02760959,692.83798795)(163.02260959,692.71298808)(163.03261219,692.55299194)
\curveto(163.04260957,692.40298839)(163.05760956,692.2929885)(163.07761219,692.22299194)
\lineto(163.07761219,692.13299194)
\curveto(163.08760953,692.11298868)(163.09260952,692.0929887)(163.09261219,692.07299194)
\curveto(163.1126095,691.9929888)(163.12760949,691.91798887)(163.13761219,691.84799194)
\curveto(163.15760946,691.77798901)(163.17760944,691.70298909)(163.19761219,691.62299194)
\curveto(163.36760925,691.10298969)(163.65760896,690.71799007)(164.06761219,690.46799194)
\curveto(164.19760842,690.37799041)(164.37760824,690.30799048)(164.60761219,690.25799194)
\curveto(164.64760797,690.24799054)(164.70760791,690.24299055)(164.78761219,690.24299194)
\curveto(164.8176078,690.23299056)(164.86260775,690.22299057)(164.92261219,690.21299194)
\curveto(164.99260762,690.21299058)(165.04760757,690.21799057)(165.08761219,690.22799194)
\curveto(165.16760745,690.24799054)(165.24760737,690.26299053)(165.32761219,690.27299194)
\curveto(165.40760721,690.28299051)(165.48760713,690.30299049)(165.56761219,690.33299194)
\curveto(165.8176068,690.44299035)(166.0176066,690.58299021)(166.16761219,690.75299194)
\curveto(166.3176063,690.92298987)(166.44760617,691.13798965)(166.55761219,691.39799194)
\curveto(166.59760602,691.4879893)(166.62760599,691.57798921)(166.64761219,691.66799194)
\curveto(166.66760595,691.76798902)(166.68760593,691.87298892)(166.70761219,691.98299194)
\curveto(166.7176059,692.03298876)(166.7176059,692.07798871)(166.70761219,692.11799194)
\curveto(166.70760591,692.16798862)(166.7176059,692.21798857)(166.73761219,692.26799194)
\curveto(166.74760587,692.29798849)(166.75260586,692.33298846)(166.75261219,692.37299194)
\lineto(166.75261219,692.50799194)
\lineto(166.75261219,692.64299194)
}
}
{
\newrgbcolor{curcolor}{0 0 0}
\pscustom[linestyle=none,fillstyle=solid,fillcolor=curcolor]
{
\newpath
\moveto(177.42253407,689.44799194)
\curveto(177.44252622,689.33799145)(177.45252621,689.22799156)(177.45253407,689.11799194)
\curveto(177.4625262,689.00799178)(177.41252625,688.93299186)(177.30253407,688.89299194)
\curveto(177.24252642,688.86299193)(177.17252649,688.84799194)(177.09253407,688.84799194)
\lineto(176.85253407,688.84799194)
\lineto(176.04253407,688.84799194)
\lineto(175.77253407,688.84799194)
\curveto(175.69252797,688.85799193)(175.62752803,688.88299191)(175.57753407,688.92299194)
\curveto(175.50752815,688.96299183)(175.45252821,689.01799177)(175.41253407,689.08799194)
\curveto(175.38252828,689.16799162)(175.33752832,689.23299156)(175.27753407,689.28299194)
\curveto(175.2575284,689.30299149)(175.23252843,689.31799147)(175.20253407,689.32799194)
\curveto(175.17252849,689.34799144)(175.13252853,689.35299144)(175.08253407,689.34299194)
\curveto(175.03252863,689.32299147)(174.98252868,689.29799149)(174.93253407,689.26799194)
\curveto(174.89252877,689.23799155)(174.84752881,689.21299158)(174.79753407,689.19299194)
\curveto(174.74752891,689.15299164)(174.69252897,689.11799167)(174.63253407,689.08799194)
\lineto(174.45253407,688.99799194)
\curveto(174.32252934,688.93799185)(174.18752947,688.8879919)(174.04753407,688.84799194)
\curveto(173.90752975,688.81799197)(173.7625299,688.78299201)(173.61253407,688.74299194)
\curveto(173.54253012,688.72299207)(173.47253019,688.71299208)(173.40253407,688.71299194)
\curveto(173.34253032,688.70299209)(173.27753038,688.6929921)(173.20753407,688.68299194)
\lineto(173.11753407,688.68299194)
\curveto(173.08753057,688.67299212)(173.0575306,688.66799212)(173.02753407,688.66799194)
\lineto(172.86253407,688.66799194)
\curveto(172.7625309,688.64799214)(172.662531,688.64799214)(172.56253407,688.66799194)
\lineto(172.42753407,688.66799194)
\curveto(172.3575313,688.6879921)(172.28753137,688.69799209)(172.21753407,688.69799194)
\curveto(172.1575315,688.6879921)(172.09753156,688.6929921)(172.03753407,688.71299194)
\curveto(171.93753172,688.73299206)(171.84253182,688.75299204)(171.75253407,688.77299194)
\curveto(171.662532,688.78299201)(171.57753208,688.80799198)(171.49753407,688.84799194)
\curveto(171.20753245,688.95799183)(170.9575327,689.09799169)(170.74753407,689.26799194)
\curveto(170.54753311,689.44799134)(170.38753327,689.68299111)(170.26753407,689.97299194)
\curveto(170.23753342,690.04299075)(170.20753345,690.11799067)(170.17753407,690.19799194)
\curveto(170.1575335,690.27799051)(170.13753352,690.36299043)(170.11753407,690.45299194)
\curveto(170.09753356,690.50299029)(170.08753357,690.55299024)(170.08753407,690.60299194)
\curveto(170.09753356,690.65299014)(170.09753356,690.70299009)(170.08753407,690.75299194)
\curveto(170.07753358,690.78299001)(170.06753359,690.84298995)(170.05753407,690.93299194)
\curveto(170.0575336,691.03298976)(170.0625336,691.10298969)(170.07253407,691.14299194)
\curveto(170.09253357,691.24298955)(170.10253356,691.32798946)(170.10253407,691.39799194)
\lineto(170.19253407,691.72799194)
\curveto(170.22253344,691.84798894)(170.2625334,691.95298884)(170.31253407,692.04299194)
\curveto(170.48253318,692.33298846)(170.67753298,692.55298824)(170.89753407,692.70299194)
\curveto(171.11753254,692.85298794)(171.39753226,692.98298781)(171.73753407,693.09299194)
\curveto(171.86753179,693.14298765)(172.00253166,693.17798761)(172.14253407,693.19799194)
\curveto(172.28253138,693.21798757)(172.42253124,693.24298755)(172.56253407,693.27299194)
\curveto(172.64253102,693.2929875)(172.72753093,693.30298749)(172.81753407,693.30299194)
\curveto(172.90753075,693.31298748)(172.99753066,693.32798746)(173.08753407,693.34799194)
\curveto(173.1575305,693.36798742)(173.22753043,693.37298742)(173.29753407,693.36299194)
\curveto(173.36753029,693.36298743)(173.44253022,693.37298742)(173.52253407,693.39299194)
\curveto(173.59253007,693.41298738)(173.66253,693.42298737)(173.73253407,693.42299194)
\curveto(173.80252986,693.42298737)(173.87752978,693.43298736)(173.95753407,693.45299194)
\curveto(174.16752949,693.50298729)(174.3575293,693.54298725)(174.52753407,693.57299194)
\curveto(174.70752895,693.61298718)(174.86752879,693.70298709)(175.00753407,693.84299194)
\curveto(175.09752856,693.93298686)(175.1575285,694.03298676)(175.18753407,694.14299194)
\curveto(175.19752846,694.17298662)(175.19752846,694.19798659)(175.18753407,694.21799194)
\curveto(175.18752847,694.23798655)(175.19252847,694.25798653)(175.20253407,694.27799194)
\curveto(175.21252845,694.29798649)(175.21752844,694.32798646)(175.21753407,694.36799194)
\lineto(175.21753407,694.45799194)
\lineto(175.18753407,694.57799194)
\curveto(175.18752847,694.61798617)(175.18252848,694.65298614)(175.17253407,694.68299194)
\curveto(175.07252859,694.98298581)(174.8625288,695.1879856)(174.54253407,695.29799194)
\curveto(174.45252921,695.32798546)(174.34252932,695.34798544)(174.21253407,695.35799194)
\curveto(174.09252957,695.37798541)(173.96752969,695.38298541)(173.83753407,695.37299194)
\curveto(173.70752995,695.37298542)(173.58253008,695.36298543)(173.46253407,695.34299194)
\curveto(173.34253032,695.32298547)(173.23753042,695.29798549)(173.14753407,695.26799194)
\curveto(173.08753057,695.24798554)(173.02753063,695.21798557)(172.96753407,695.17799194)
\curveto(172.91753074,695.14798564)(172.86753079,695.11298568)(172.81753407,695.07299194)
\curveto(172.76753089,695.03298576)(172.71253095,694.97798581)(172.65253407,694.90799194)
\curveto(172.60253106,694.83798595)(172.56753109,694.77298602)(172.54753407,694.71299194)
\curveto(172.49753116,694.61298618)(172.45253121,694.51798627)(172.41253407,694.42799194)
\curveto(172.38253128,694.33798645)(172.31253135,694.27798651)(172.20253407,694.24799194)
\curveto(172.12253154,694.22798656)(172.03753162,694.21798657)(171.94753407,694.21799194)
\lineto(171.67753407,694.21799194)
\lineto(171.10753407,694.21799194)
\curveto(171.0575326,694.21798657)(171.00753265,694.21298658)(170.95753407,694.20299194)
\curveto(170.90753275,694.20298659)(170.8625328,694.20798658)(170.82253407,694.21799194)
\lineto(170.68753407,694.21799194)
\curveto(170.66753299,694.22798656)(170.64253302,694.23298656)(170.61253407,694.23299194)
\curveto(170.58253308,694.23298656)(170.5575331,694.24298655)(170.53753407,694.26299194)
\curveto(170.4575332,694.28298651)(170.40253326,694.34798644)(170.37253407,694.45799194)
\curveto(170.3625333,694.50798628)(170.3625333,694.55798623)(170.37253407,694.60799194)
\curveto(170.38253328,694.65798613)(170.39253327,694.70298609)(170.40253407,694.74299194)
\curveto(170.43253323,694.85298594)(170.4625332,694.95298584)(170.49253407,695.04299194)
\curveto(170.53253313,695.14298565)(170.57753308,695.23298556)(170.62753407,695.31299194)
\lineto(170.71753407,695.46299194)
\lineto(170.80753407,695.61299194)
\curveto(170.88753277,695.72298507)(170.98753267,695.82798496)(171.10753407,695.92799194)
\curveto(171.12753253,695.93798485)(171.1575325,695.96298483)(171.19753407,696.00299194)
\curveto(171.24753241,696.04298475)(171.29253237,696.07798471)(171.33253407,696.10799194)
\curveto(171.37253229,696.13798465)(171.41753224,696.16798462)(171.46753407,696.19799194)
\curveto(171.63753202,696.30798448)(171.81753184,696.3929844)(172.00753407,696.45299194)
\curveto(172.19753146,696.52298427)(172.39253127,696.5879842)(172.59253407,696.64799194)
\curveto(172.71253095,696.67798411)(172.83753082,696.69798409)(172.96753407,696.70799194)
\curveto(173.09753056,696.71798407)(173.22753043,696.73798405)(173.35753407,696.76799194)
\curveto(173.39753026,696.77798401)(173.4575302,696.77798401)(173.53753407,696.76799194)
\curveto(173.62753003,696.75798403)(173.68252998,696.76298403)(173.70253407,696.78299194)
\curveto(174.11252955,696.792984)(174.50252916,696.77798401)(174.87253407,696.73799194)
\curveto(175.25252841,696.69798409)(175.59252807,696.62298417)(175.89253407,696.51299194)
\curveto(176.20252746,696.40298439)(176.46752719,696.25298454)(176.68753407,696.06299194)
\curveto(176.90752675,695.88298491)(177.07752658,695.64798514)(177.19753407,695.35799194)
\curveto(177.26752639,695.1879856)(177.30752635,694.9929858)(177.31753407,694.77299194)
\curveto(177.32752633,694.55298624)(177.33252633,694.32798646)(177.33253407,694.09799194)
\lineto(177.33253407,690.75299194)
\lineto(177.33253407,690.16799194)
\curveto(177.33252633,689.97799081)(177.35252631,689.80299099)(177.39253407,689.64299194)
\curveto(177.40252626,689.61299118)(177.40752625,689.57799121)(177.40753407,689.53799194)
\curveto(177.40752625,689.50799128)(177.41252625,689.47799131)(177.42253407,689.44799194)
\moveto(175.21753407,691.75799194)
\curveto(175.22752843,691.80798898)(175.23252843,691.86298893)(175.23253407,691.92299194)
\curveto(175.23252843,691.9929888)(175.22752843,692.05298874)(175.21753407,692.10299194)
\curveto(175.19752846,692.16298863)(175.18752847,692.21798857)(175.18753407,692.26799194)
\curveto(175.18752847,692.31798847)(175.16752849,692.35798843)(175.12753407,692.38799194)
\curveto(175.07752858,692.42798836)(175.00252866,692.44798834)(174.90253407,692.44799194)
\curveto(174.8625288,692.43798835)(174.82752883,692.42798836)(174.79753407,692.41799194)
\curveto(174.76752889,692.41798837)(174.73252893,692.41298838)(174.69253407,692.40299194)
\curveto(174.62252904,692.38298841)(174.54752911,692.36798842)(174.46753407,692.35799194)
\curveto(174.38752927,692.34798844)(174.30752935,692.33298846)(174.22753407,692.31299194)
\curveto(174.19752946,692.30298849)(174.15252951,692.29798849)(174.09253407,692.29799194)
\curveto(173.9625297,692.26798852)(173.83252983,692.24798854)(173.70253407,692.23799194)
\curveto(173.57253009,692.22798856)(173.44753021,692.20298859)(173.32753407,692.16299194)
\curveto(173.24753041,692.14298865)(173.17253049,692.12298867)(173.10253407,692.10299194)
\curveto(173.03253063,692.0929887)(172.9625307,692.07298872)(172.89253407,692.04299194)
\curveto(172.68253098,691.95298884)(172.50253116,691.81798897)(172.35253407,691.63799194)
\curveto(172.21253145,691.45798933)(172.1625315,691.20798958)(172.20253407,690.88799194)
\curveto(172.22253144,690.71799007)(172.27753138,690.57799021)(172.36753407,690.46799194)
\curveto(172.43753122,690.35799043)(172.54253112,690.26799052)(172.68253407,690.19799194)
\curveto(172.82253084,690.13799065)(172.97253069,690.0929907)(173.13253407,690.06299194)
\curveto(173.30253036,690.03299076)(173.47753018,690.02299077)(173.65753407,690.03299194)
\curveto(173.84752981,690.05299074)(174.02252964,690.0879907)(174.18253407,690.13799194)
\curveto(174.44252922,690.21799057)(174.64752901,690.34299045)(174.79753407,690.51299194)
\curveto(174.94752871,690.6929901)(175.0625286,690.91298988)(175.14253407,691.17299194)
\curveto(175.1625285,691.24298955)(175.17252849,691.31298948)(175.17253407,691.38299194)
\curveto(175.18252848,691.46298933)(175.19752846,691.54298925)(175.21753407,691.62299194)
\lineto(175.21753407,691.75799194)
}
}
{
\newrgbcolor{curcolor}{0 0 0}
\pscustom[linestyle=none,fillstyle=solid,fillcolor=curcolor]
{
\newpath
\moveto(186.57581532,689.70299194)
\lineto(186.57581532,689.28299194)
\curveto(186.57580695,689.15299164)(186.54580698,689.04799174)(186.48581532,688.96799194)
\curveto(186.43580709,688.91799187)(186.37080715,688.88299191)(186.29081532,688.86299194)
\curveto(186.21080731,688.85299194)(186.1208074,688.84799194)(186.02081532,688.84799194)
\lineto(185.19581532,688.84799194)
\lineto(184.91081532,688.84799194)
\curveto(184.83080869,688.85799193)(184.76580876,688.88299191)(184.71581532,688.92299194)
\curveto(184.64580888,688.97299182)(184.60580892,689.03799175)(184.59581532,689.11799194)
\curveto(184.58580894,689.19799159)(184.56580896,689.27799151)(184.53581532,689.35799194)
\curveto(184.51580901,689.37799141)(184.49580903,689.3929914)(184.47581532,689.40299194)
\curveto(184.46580906,689.42299137)(184.45080907,689.44299135)(184.43081532,689.46299194)
\curveto(184.3208092,689.46299133)(184.24080928,689.43799135)(184.19081532,689.38799194)
\lineto(184.04081532,689.23799194)
\curveto(183.97080955,689.1879916)(183.90580962,689.14299165)(183.84581532,689.10299194)
\curveto(183.78580974,689.07299172)(183.7208098,689.03299176)(183.65081532,688.98299194)
\curveto(183.61080991,688.96299183)(183.56580996,688.94299185)(183.51581532,688.92299194)
\curveto(183.47581005,688.90299189)(183.43081009,688.88299191)(183.38081532,688.86299194)
\curveto(183.24081028,688.81299198)(183.09081043,688.76799202)(182.93081532,688.72799194)
\curveto(182.88081064,688.70799208)(182.83581069,688.69799209)(182.79581532,688.69799194)
\curveto(182.75581077,688.69799209)(182.71581081,688.6929921)(182.67581532,688.68299194)
\lineto(182.54081532,688.68299194)
\curveto(182.51081101,688.67299212)(182.47081105,688.66799212)(182.42081532,688.66799194)
\lineto(182.28581532,688.66799194)
\curveto(182.2258113,688.64799214)(182.13581139,688.64299215)(182.01581532,688.65299194)
\curveto(181.89581163,688.65299214)(181.81081171,688.66299213)(181.76081532,688.68299194)
\curveto(181.69081183,688.70299209)(181.6258119,688.71299208)(181.56581532,688.71299194)
\curveto(181.51581201,688.70299209)(181.46081206,688.70799208)(181.40081532,688.72799194)
\lineto(181.04081532,688.84799194)
\curveto(180.93081259,688.87799191)(180.8208127,688.91799187)(180.71081532,688.96799194)
\curveto(180.36081316,689.11799167)(180.04581348,689.34799144)(179.76581532,689.65799194)
\curveto(179.49581403,689.97799081)(179.28081424,690.31299048)(179.12081532,690.66299194)
\curveto(179.07081445,690.77299002)(179.03081449,690.87798991)(179.00081532,690.97799194)
\curveto(178.97081455,691.0879897)(178.93581459,691.19798959)(178.89581532,691.30799194)
\curveto(178.88581464,691.34798944)(178.88081464,691.38298941)(178.88081532,691.41299194)
\curveto(178.88081464,691.45298934)(178.87081465,691.49798929)(178.85081532,691.54799194)
\curveto(178.83081469,691.62798916)(178.81081471,691.71298908)(178.79081532,691.80299194)
\curveto(178.78081474,691.90298889)(178.76581476,692.00298879)(178.74581532,692.10299194)
\curveto(178.73581479,692.13298866)(178.73081479,692.16798862)(178.73081532,692.20799194)
\curveto(178.74081478,692.24798854)(178.74081478,692.28298851)(178.73081532,692.31299194)
\lineto(178.73081532,692.44799194)
\curveto(178.73081479,692.49798829)(178.7258148,692.54798824)(178.71581532,692.59799194)
\curveto(178.70581482,692.64798814)(178.70081482,692.70298809)(178.70081532,692.76299194)
\curveto(178.70081482,692.83298796)(178.70581482,692.8879879)(178.71581532,692.92799194)
\curveto(178.7258148,692.97798781)(178.73081479,693.02298777)(178.73081532,693.06299194)
\lineto(178.73081532,693.21299194)
\curveto(178.74081478,693.26298753)(178.74081478,693.30798748)(178.73081532,693.34799194)
\curveto(178.73081479,693.39798739)(178.74081478,693.44798734)(178.76081532,693.49799194)
\curveto(178.78081474,693.60798718)(178.79581473,693.71298708)(178.80581532,693.81299194)
\curveto(178.8258147,693.91298688)(178.85081467,694.01298678)(178.88081532,694.11299194)
\curveto(178.9208146,694.23298656)(178.95581457,694.34798644)(178.98581532,694.45799194)
\curveto(179.01581451,694.56798622)(179.05581447,694.67798611)(179.10581532,694.78799194)
\curveto(179.24581428,695.0879857)(179.4208141,695.37298542)(179.63081532,695.64299194)
\curveto(179.65081387,695.67298512)(179.67581385,695.69798509)(179.70581532,695.71799194)
\curveto(179.74581378,695.74798504)(179.77581375,695.77798501)(179.79581532,695.80799194)
\curveto(179.83581369,695.85798493)(179.87581365,695.90298489)(179.91581532,695.94299194)
\curveto(179.95581357,695.98298481)(180.00081352,696.02298477)(180.05081532,696.06299194)
\curveto(180.09081343,696.08298471)(180.1258134,696.10798468)(180.15581532,696.13799194)
\curveto(180.18581334,696.17798461)(180.2208133,696.20798458)(180.26081532,696.22799194)
\curveto(180.51081301,696.39798439)(180.80081272,696.53798425)(181.13081532,696.64799194)
\curveto(181.20081232,696.66798412)(181.27081225,696.68298411)(181.34081532,696.69299194)
\curveto(181.4208121,696.70298409)(181.50081202,696.71798407)(181.58081532,696.73799194)
\curveto(181.65081187,696.75798403)(181.74081178,696.76798402)(181.85081532,696.76799194)
\curveto(181.96081156,696.77798401)(182.07081145,696.78298401)(182.18081532,696.78299194)
\curveto(182.29081123,696.78298401)(182.39581113,696.77798401)(182.49581532,696.76799194)
\curveto(182.60581092,696.75798403)(182.69581083,696.74298405)(182.76581532,696.72299194)
\curveto(182.91581061,696.67298412)(183.06081046,696.62798416)(183.20081532,696.58799194)
\curveto(183.34081018,696.54798424)(183.47081005,696.4929843)(183.59081532,696.42299194)
\curveto(183.66080986,696.37298442)(183.7258098,696.32298447)(183.78581532,696.27299194)
\curveto(183.84580968,696.23298456)(183.91080961,696.1879846)(183.98081532,696.13799194)
\curveto(184.0208095,696.10798468)(184.07580945,696.06798472)(184.14581532,696.01799194)
\curveto(184.2258093,695.96798482)(184.30080922,695.96798482)(184.37081532,696.01799194)
\curveto(184.41080911,696.03798475)(184.43080909,696.07298472)(184.43081532,696.12299194)
\curveto(184.43080909,696.17298462)(184.44080908,696.22298457)(184.46081532,696.27299194)
\lineto(184.46081532,696.42299194)
\curveto(184.47080905,696.45298434)(184.47580905,696.4879843)(184.47581532,696.52799194)
\lineto(184.47581532,696.64799194)
\lineto(184.47581532,698.68799194)
\curveto(184.47580905,698.79798199)(184.47080905,698.91798187)(184.46081532,699.04799194)
\curveto(184.46080906,699.1879816)(184.48580904,699.2929815)(184.53581532,699.36299194)
\curveto(184.57580895,699.44298135)(184.65080887,699.4929813)(184.76081532,699.51299194)
\curveto(184.78080874,699.52298127)(184.80080872,699.52298127)(184.82081532,699.51299194)
\curveto(184.84080868,699.51298128)(184.86080866,699.51798127)(184.88081532,699.52799194)
\lineto(185.94581532,699.52799194)
\curveto(186.06580746,699.52798126)(186.17580735,699.52298127)(186.27581532,699.51299194)
\curveto(186.37580715,699.50298129)(186.45080707,699.46298133)(186.50081532,699.39299194)
\curveto(186.55080697,699.31298148)(186.57580695,699.20798158)(186.57581532,699.07799194)
\lineto(186.57581532,698.71799194)
\lineto(186.57581532,689.70299194)
\moveto(184.53581532,692.64299194)
\curveto(184.54580898,692.68298811)(184.54580898,692.72298807)(184.53581532,692.76299194)
\lineto(184.53581532,692.89799194)
\curveto(184.53580899,692.99798779)(184.53080899,693.09798769)(184.52081532,693.19799194)
\curveto(184.51080901,693.29798749)(184.49580903,693.3879874)(184.47581532,693.46799194)
\curveto(184.45580907,693.57798721)(184.43580909,693.67798711)(184.41581532,693.76799194)
\curveto(184.40580912,693.85798693)(184.38080914,693.94298685)(184.34081532,694.02299194)
\curveto(184.20080932,694.38298641)(183.99580953,694.66798612)(183.72581532,694.87799194)
\curveto(183.46581006,695.0879857)(183.08581044,695.1929856)(182.58581532,695.19299194)
\curveto(182.525811,695.1929856)(182.44581108,695.18298561)(182.34581532,695.16299194)
\curveto(182.26581126,695.14298565)(182.19081133,695.12298567)(182.12081532,695.10299194)
\curveto(182.06081146,695.0929857)(182.00081152,695.07298572)(181.94081532,695.04299194)
\curveto(181.67081185,694.93298586)(181.46081206,694.76298603)(181.31081532,694.53299194)
\curveto(181.16081236,694.30298649)(181.04081248,694.04298675)(180.95081532,693.75299194)
\curveto(180.9208126,693.65298714)(180.90081262,693.55298724)(180.89081532,693.45299194)
\curveto(180.88081264,693.35298744)(180.86081266,693.24798754)(180.83081532,693.13799194)
\lineto(180.83081532,692.92799194)
\curveto(180.81081271,692.83798795)(180.80581272,692.71298808)(180.81581532,692.55299194)
\curveto(180.8258127,692.40298839)(180.84081268,692.2929885)(180.86081532,692.22299194)
\lineto(180.86081532,692.13299194)
\curveto(180.87081265,692.11298868)(180.87581265,692.0929887)(180.87581532,692.07299194)
\curveto(180.89581263,691.9929888)(180.91081261,691.91798887)(180.92081532,691.84799194)
\curveto(180.94081258,691.77798901)(180.96081256,691.70298909)(180.98081532,691.62299194)
\curveto(181.15081237,691.10298969)(181.44081208,690.71799007)(181.85081532,690.46799194)
\curveto(181.98081154,690.37799041)(182.16081136,690.30799048)(182.39081532,690.25799194)
\curveto(182.43081109,690.24799054)(182.49081103,690.24299055)(182.57081532,690.24299194)
\curveto(182.60081092,690.23299056)(182.64581088,690.22299057)(182.70581532,690.21299194)
\curveto(182.77581075,690.21299058)(182.83081069,690.21799057)(182.87081532,690.22799194)
\curveto(182.95081057,690.24799054)(183.03081049,690.26299053)(183.11081532,690.27299194)
\curveto(183.19081033,690.28299051)(183.27081025,690.30299049)(183.35081532,690.33299194)
\curveto(183.60080992,690.44299035)(183.80080972,690.58299021)(183.95081532,690.75299194)
\curveto(184.10080942,690.92298987)(184.23080929,691.13798965)(184.34081532,691.39799194)
\curveto(184.38080914,691.4879893)(184.41080911,691.57798921)(184.43081532,691.66799194)
\curveto(184.45080907,691.76798902)(184.47080905,691.87298892)(184.49081532,691.98299194)
\curveto(184.50080902,692.03298876)(184.50080902,692.07798871)(184.49081532,692.11799194)
\curveto(184.49080903,692.16798862)(184.50080902,692.21798857)(184.52081532,692.26799194)
\curveto(184.53080899,692.29798849)(184.53580899,692.33298846)(184.53581532,692.37299194)
\lineto(184.53581532,692.50799194)
\lineto(184.53581532,692.64299194)
}
}
{
\newrgbcolor{curcolor}{0 0 0}
\pscustom[linestyle=none,fillstyle=solid,fillcolor=curcolor]
{
\newpath
\moveto(827.1682959,684.76149658)
\curveto(827.23829416,684.76148592)(827.31829408,684.76148592)(827.4082959,684.76149658)
\curveto(827.4982939,684.77148591)(827.58329382,684.77148591)(827.6632959,684.76149658)
\curveto(827.75329365,684.76148592)(827.83329357,684.75148593)(827.9032959,684.73149658)
\curveto(827.97329343,684.71148597)(828.02329338,684.681486)(828.0532959,684.64149658)
\curveto(828.11329329,684.57148611)(828.14329326,684.47148621)(828.1432959,684.34149658)
\curveto(828.15329325,684.22148646)(828.15829324,684.09648658)(828.1582959,683.96649658)
\lineto(828.1582959,682.51149658)
\lineto(828.1582959,676.72149658)
\lineto(828.1582959,674.96649658)
\lineto(828.1582959,674.54649658)
\curveto(828.15829324,674.40649627)(828.13329327,674.29649638)(828.0832959,674.21649658)
\curveto(828.04329336,674.16649651)(827.99329341,674.13649654)(827.9332959,674.12649658)
\curveto(827.88329352,674.11649656)(827.81829358,674.10149658)(827.7382959,674.08149658)
\lineto(827.4532959,674.08149658)
\curveto(827.31329409,674.0814966)(827.18329422,674.08649659)(827.0632959,674.09649658)
\curveto(826.94329446,674.10649657)(826.85829454,674.15649652)(826.8082959,674.24649658)
\curveto(826.76829463,674.30649637)(826.74829465,674.38649629)(826.7482959,674.48649658)
\lineto(826.7482959,674.81649658)
\lineto(826.7482959,676.01649658)
\lineto(826.7482959,682.28649658)
\lineto(826.7482959,683.90649658)
\curveto(826.74829465,684.01648666)(826.74329466,684.13648654)(826.7332959,684.26649658)
\curveto(826.73329467,684.40648627)(826.75829464,684.51648616)(826.8082959,684.59649658)
\curveto(826.84829455,684.66648601)(826.92829447,684.71648596)(827.0482959,684.74649658)
\curveto(827.06829433,684.75648592)(827.08829431,684.75648592)(827.1082959,684.74649658)
\curveto(827.12829427,684.74648593)(827.14829425,684.75148593)(827.1682959,684.76149658)
}
}
{
\newrgbcolor{curcolor}{0 0 0}
\pscustom[linestyle=none,fillstyle=solid,fillcolor=curcolor]
{
\newpath
\moveto(834.00478027,681.98649658)
\curveto(834.63477504,682.00648867)(835.13977453,681.92148876)(835.51978027,681.73149658)
\curveto(835.89977377,681.54148914)(836.20477347,681.25648942)(836.43478027,680.87649658)
\curveto(836.49477318,680.7764899)(836.53977313,680.66649001)(836.56978027,680.54649658)
\curveto(836.60977306,680.43649024)(836.64477303,680.32149036)(836.67478027,680.20149658)
\curveto(836.72477295,680.01149067)(836.75477292,679.80649087)(836.76478027,679.58649658)
\curveto(836.7747729,679.36649131)(836.77977289,679.14149154)(836.77978027,678.91149658)
\lineto(836.77978027,677.30649658)
\lineto(836.77978027,674.96649658)
\curveto(836.77977289,674.79649588)(836.7747729,674.62649605)(836.76478027,674.45649658)
\curveto(836.76477291,674.28649639)(836.69977297,674.1764965)(836.56978027,674.12649658)
\curveto(836.51977315,674.10649657)(836.46477321,674.09649658)(836.40478027,674.09649658)
\curveto(836.35477332,674.08649659)(836.29977337,674.0814966)(836.23978027,674.08149658)
\curveto(836.10977356,674.0814966)(835.98477369,674.08649659)(835.86478027,674.09649658)
\curveto(835.74477393,674.09649658)(835.65977401,674.13649654)(835.60978027,674.21649658)
\curveto(835.55977411,674.28649639)(835.53477414,674.3764963)(835.53478027,674.48649658)
\lineto(835.53478027,674.81649658)
\lineto(835.53478027,676.10649658)
\lineto(835.53478027,678.55149658)
\curveto(835.53477414,678.82149186)(835.52977414,679.08649159)(835.51978027,679.34649658)
\curveto(835.50977416,679.61649106)(835.46477421,679.84649083)(835.38478027,680.03649658)
\curveto(835.30477437,680.23649044)(835.18477449,680.39649028)(835.02478027,680.51649658)
\curveto(834.86477481,680.64649003)(834.67977499,680.74648993)(834.46978027,680.81649658)
\curveto(834.40977526,680.83648984)(834.34477533,680.84648983)(834.27478027,680.84649658)
\curveto(834.21477546,680.85648982)(834.15477552,680.87148981)(834.09478027,680.89149658)
\curveto(834.04477563,680.90148978)(833.96477571,680.90148978)(833.85478027,680.89149658)
\curveto(833.75477592,680.89148979)(833.68477599,680.88648979)(833.64478027,680.87649658)
\curveto(833.60477607,680.85648982)(833.5697761,680.84648983)(833.53978027,680.84649658)
\curveto(833.50977616,680.85648982)(833.4747762,680.85648982)(833.43478027,680.84649658)
\curveto(833.30477637,680.81648986)(833.17977649,680.7814899)(833.05978027,680.74149658)
\curveto(832.94977672,680.71148997)(832.84477683,680.66649001)(832.74478027,680.60649658)
\curveto(832.70477697,680.58649009)(832.669777,680.56649011)(832.63978027,680.54649658)
\curveto(832.60977706,680.52649015)(832.5747771,680.50649017)(832.53478027,680.48649658)
\curveto(832.18477749,680.23649044)(831.92977774,679.86149082)(831.76978027,679.36149658)
\curveto(831.73977793,679.2814914)(831.71977795,679.19649148)(831.70978027,679.10649658)
\curveto(831.69977797,679.02649165)(831.68477799,678.94649173)(831.66478027,678.86649658)
\curveto(831.64477803,678.81649186)(831.63977803,678.76649191)(831.64978027,678.71649658)
\curveto(831.65977801,678.676492)(831.65477802,678.63649204)(831.63478027,678.59649658)
\lineto(831.63478027,678.28149658)
\curveto(831.62477805,678.25149243)(831.61977805,678.21649246)(831.61978027,678.17649658)
\curveto(831.62977804,678.13649254)(831.63477804,678.09149259)(831.63478027,678.04149658)
\lineto(831.63478027,677.59149658)
\lineto(831.63478027,676.15149658)
\lineto(831.63478027,674.83149658)
\lineto(831.63478027,674.48649658)
\curveto(831.63477804,674.3764963)(831.60977806,674.28649639)(831.55978027,674.21649658)
\curveto(831.50977816,674.13649654)(831.41977825,674.09649658)(831.28978027,674.09649658)
\curveto(831.1697785,674.08649659)(831.04477863,674.0814966)(830.91478027,674.08149658)
\curveto(830.83477884,674.0814966)(830.75977891,674.08649659)(830.68978027,674.09649658)
\curveto(830.61977905,674.10649657)(830.55977911,674.13149655)(830.50978027,674.17149658)
\curveto(830.42977924,674.22149646)(830.38977928,674.31649636)(830.38978027,674.45649658)
\lineto(830.38978027,674.86149658)
\lineto(830.38978027,676.63149658)
\lineto(830.38978027,680.26149658)
\lineto(830.38978027,681.17649658)
\lineto(830.38978027,681.44649658)
\curveto(830.38977928,681.53648914)(830.40977926,681.60648907)(830.44978027,681.65649658)
\curveto(830.47977919,681.71648896)(830.52977914,681.75648892)(830.59978027,681.77649658)
\curveto(830.63977903,681.78648889)(830.69477898,681.79648888)(830.76478027,681.80649658)
\curveto(830.84477883,681.81648886)(830.92477875,681.82148886)(831.00478027,681.82149658)
\curveto(831.08477859,681.82148886)(831.15977851,681.81648886)(831.22978027,681.80649658)
\curveto(831.30977836,681.79648888)(831.36477831,681.7814889)(831.39478027,681.76149658)
\curveto(831.50477817,681.69148899)(831.55477812,681.60148908)(831.54478027,681.49149658)
\curveto(831.53477814,681.39148929)(831.54977812,681.2764894)(831.58978027,681.14649658)
\curveto(831.60977806,681.08648959)(831.64977802,681.03648964)(831.70978027,680.99649658)
\curveto(831.82977784,680.98648969)(831.92477775,681.03148965)(831.99478027,681.13149658)
\curveto(832.0747776,681.23148945)(832.15477752,681.31148937)(832.23478027,681.37149658)
\curveto(832.3747773,681.47148921)(832.51477716,681.56148912)(832.65478027,681.64149658)
\curveto(832.80477687,681.73148895)(832.9747767,681.80648887)(833.16478027,681.86649658)
\curveto(833.24477643,681.89648878)(833.32977634,681.91648876)(833.41978027,681.92649658)
\curveto(833.51977615,681.93648874)(833.61477606,681.95148873)(833.70478027,681.97149658)
\curveto(833.75477592,681.9814887)(833.80477587,681.98648869)(833.85478027,681.98649658)
\lineto(834.00478027,681.98649658)
}
}
{
\newrgbcolor{curcolor}{0 0 0}
\pscustom[linestyle=none,fillstyle=solid,fillcolor=curcolor]
{
\newpath
\moveto(838.43938965,681.83649658)
\lineto(838.91938965,681.83649658)
\curveto(839.08938831,681.83648884)(839.21938818,681.80648887)(839.30938965,681.74649658)
\curveto(839.37938802,681.69648898)(839.42438797,681.63148905)(839.44438965,681.55149658)
\curveto(839.47438792,681.4814892)(839.50438789,681.40648927)(839.53438965,681.32649658)
\curveto(839.5943878,681.18648949)(839.64438775,681.04648963)(839.68438965,680.90649658)
\curveto(839.72438767,680.76648991)(839.76938763,680.62649005)(839.81938965,680.48649658)
\curveto(840.01938738,679.94649073)(840.20438719,679.40149128)(840.37438965,678.85149658)
\curveto(840.54438685,678.31149237)(840.72938667,677.77149291)(840.92938965,677.23149658)
\curveto(840.9993864,677.05149363)(841.05938634,676.86649381)(841.10938965,676.67649658)
\curveto(841.15938624,676.49649418)(841.22438617,676.31649436)(841.30438965,676.13649658)
\curveto(841.32438607,676.06649461)(841.34938605,675.99149469)(841.37938965,675.91149658)
\curveto(841.40938599,675.83149485)(841.45938594,675.7814949)(841.52938965,675.76149658)
\curveto(841.60938579,675.74149494)(841.66938573,675.7764949)(841.70938965,675.86649658)
\curveto(841.75938564,675.95649472)(841.7943856,676.02649465)(841.81438965,676.07649658)
\curveto(841.8943855,676.26649441)(841.95938544,676.45649422)(842.00938965,676.64649658)
\curveto(842.06938533,676.84649383)(842.13438526,677.04649363)(842.20438965,677.24649658)
\curveto(842.33438506,677.62649305)(842.45938494,678.00149268)(842.57938965,678.37149658)
\curveto(842.6993847,678.75149193)(842.82438457,679.13149155)(842.95438965,679.51149658)
\curveto(843.00438439,679.681491)(843.05438434,679.84649083)(843.10438965,680.00649658)
\curveto(843.15438424,680.1764905)(843.21438418,680.34149034)(843.28438965,680.50149658)
\curveto(843.33438406,680.64149004)(843.37938402,680.7814899)(843.41938965,680.92149658)
\curveto(843.45938394,681.06148962)(843.50438389,681.20148948)(843.55438965,681.34149658)
\curveto(843.57438382,681.41148927)(843.5993838,681.4814892)(843.62938965,681.55149658)
\curveto(843.65938374,681.62148906)(843.6993837,681.681489)(843.74938965,681.73149658)
\curveto(843.82938357,681.7814889)(843.91938348,681.81148887)(844.01938965,681.82149658)
\curveto(844.11938328,681.83148885)(844.23938316,681.83648884)(844.37938965,681.83649658)
\curveto(844.44938295,681.83648884)(844.51438288,681.83148885)(844.57438965,681.82149658)
\curveto(844.63438276,681.82148886)(844.68938271,681.81148887)(844.73938965,681.79149658)
\curveto(844.82938257,681.75148893)(844.87438252,681.68648899)(844.87438965,681.59649658)
\curveto(844.88438251,681.50648917)(844.86938253,681.41648926)(844.82938965,681.32649658)
\curveto(844.76938263,681.15648952)(844.70938269,680.9814897)(844.64938965,680.80149658)
\curveto(844.58938281,680.62149006)(844.51938288,680.44649023)(844.43938965,680.27649658)
\curveto(844.41938298,680.22649045)(844.40438299,680.1764905)(844.39438965,680.12649658)
\curveto(844.38438301,680.08649059)(844.36938303,680.04149064)(844.34938965,679.99149658)
\curveto(844.26938313,679.82149086)(844.20438319,679.64649103)(844.15438965,679.46649658)
\curveto(844.10438329,679.28649139)(844.03938336,679.10649157)(843.95938965,678.92649658)
\curveto(843.90938349,678.79649188)(843.85938354,678.66149202)(843.80938965,678.52149658)
\curveto(843.76938363,678.39149229)(843.71938368,678.26149242)(843.65938965,678.13149658)
\curveto(843.48938391,677.72149296)(843.33438406,677.30649337)(843.19438965,676.88649658)
\curveto(843.06438433,676.46649421)(842.91438448,676.05149463)(842.74438965,675.64149658)
\curveto(842.68438471,675.4814952)(842.62938477,675.32149536)(842.57938965,675.16149658)
\curveto(842.52938487,675.00149568)(842.46938493,674.84149584)(842.39938965,674.68149658)
\curveto(842.34938505,674.57149611)(842.30438509,674.46649621)(842.26438965,674.36649658)
\curveto(842.23438516,674.2764964)(842.16438523,674.20649647)(842.05438965,674.15649658)
\curveto(841.9943854,674.12649655)(841.92438547,674.11149657)(841.84438965,674.11149658)
\lineto(841.61938965,674.11149658)
\lineto(841.15438965,674.11149658)
\curveto(841.00438639,674.12149656)(840.8943865,674.17149651)(840.82438965,674.26149658)
\curveto(840.75438664,674.34149634)(840.70438669,674.43649624)(840.67438965,674.54649658)
\curveto(840.64438675,674.66649601)(840.60438679,674.7814959)(840.55438965,674.89149658)
\curveto(840.4943869,675.03149565)(840.43438696,675.1764955)(840.37438965,675.32649658)
\curveto(840.32438707,675.48649519)(840.27438712,675.63649504)(840.22438965,675.77649658)
\curveto(840.20438719,675.82649485)(840.18938721,675.86649481)(840.17938965,675.89649658)
\curveto(840.16938723,675.93649474)(840.15438724,675.9814947)(840.13438965,676.03149658)
\curveto(839.93438746,676.51149417)(839.74938765,676.99649368)(839.57938965,677.48649658)
\curveto(839.41938798,677.9764927)(839.23938816,678.46149222)(839.03938965,678.94149658)
\curveto(838.97938842,679.10149158)(838.91938848,679.25649142)(838.85938965,679.40649658)
\curveto(838.80938859,679.56649111)(838.75438864,679.72649095)(838.69438965,679.88649658)
\lineto(838.63438965,680.03649658)
\curveto(838.62438877,680.09649058)(838.60938879,680.15149053)(838.58938965,680.20149658)
\curveto(838.50938889,680.37149031)(838.43938896,680.54149014)(838.37938965,680.71149658)
\curveto(838.32938907,680.8814898)(838.26938913,681.05148963)(838.19938965,681.22149658)
\curveto(838.17938922,681.2814894)(838.15438924,681.36148932)(838.12438965,681.46149658)
\curveto(838.0943893,681.56148912)(838.0993893,681.64648903)(838.13938965,681.71649658)
\curveto(838.18938921,681.76648891)(838.24938915,681.80148888)(838.31938965,681.82149658)
\curveto(838.38938901,681.82148886)(838.42938897,681.82648885)(838.43938965,681.83649658)
}
}
{
\newrgbcolor{curcolor}{0 0 0}
\pscustom[linestyle=none,fillstyle=solid,fillcolor=curcolor]
{
\newpath
\moveto(846.44938965,683.33649658)
\curveto(846.36938853,683.39648728)(846.32438857,683.50148718)(846.31438965,683.65149658)
\lineto(846.31438965,684.11649658)
\lineto(846.31438965,684.37149658)
\curveto(846.31438858,684.46148622)(846.32938857,684.53648614)(846.35938965,684.59649658)
\curveto(846.3993885,684.676486)(846.47938842,684.73648594)(846.59938965,684.77649658)
\curveto(846.61938828,684.78648589)(846.63938826,684.78648589)(846.65938965,684.77649658)
\curveto(846.68938821,684.7764859)(846.71438818,684.7814859)(846.73438965,684.79149658)
\curveto(846.90438799,684.79148589)(847.06438783,684.78648589)(847.21438965,684.77649658)
\curveto(847.36438753,684.76648591)(847.46438743,684.70648597)(847.51438965,684.59649658)
\curveto(847.54438735,684.53648614)(847.55938734,684.46148622)(847.55938965,684.37149658)
\lineto(847.55938965,684.11649658)
\curveto(847.55938734,683.93648674)(847.55438734,683.76648691)(847.54438965,683.60649658)
\curveto(847.54438735,683.44648723)(847.47938742,683.34148734)(847.34938965,683.29149658)
\curveto(847.2993876,683.27148741)(847.24438765,683.26148742)(847.18438965,683.26149658)
\lineto(847.01938965,683.26149658)
\lineto(846.70438965,683.26149658)
\curveto(846.60438829,683.26148742)(846.51938838,683.28648739)(846.44938965,683.33649658)
\moveto(847.55938965,674.83149658)
\lineto(847.55938965,674.51649658)
\curveto(847.56938733,674.41649626)(847.54938735,674.33649634)(847.49938965,674.27649658)
\curveto(847.46938743,674.21649646)(847.42438747,674.1764965)(847.36438965,674.15649658)
\curveto(847.30438759,674.14649653)(847.23438766,674.13149655)(847.15438965,674.11149658)
\lineto(846.92938965,674.11149658)
\curveto(846.7993881,674.11149657)(846.68438821,674.11649656)(846.58438965,674.12649658)
\curveto(846.4943884,674.14649653)(846.42438847,674.19649648)(846.37438965,674.27649658)
\curveto(846.33438856,674.33649634)(846.31438858,674.41149627)(846.31438965,674.50149658)
\lineto(846.31438965,674.78649658)
\lineto(846.31438965,681.13149658)
\lineto(846.31438965,681.44649658)
\curveto(846.31438858,681.55648912)(846.33938856,681.64148904)(846.38938965,681.70149658)
\curveto(846.41938848,681.75148893)(846.45938844,681.7814889)(846.50938965,681.79149658)
\curveto(846.55938834,681.80148888)(846.61438828,681.81648886)(846.67438965,681.83649658)
\curveto(846.6943882,681.83648884)(846.71438818,681.83148885)(846.73438965,681.82149658)
\curveto(846.76438813,681.82148886)(846.78938811,681.82648885)(846.80938965,681.83649658)
\curveto(846.93938796,681.83648884)(847.06938783,681.83148885)(847.19938965,681.82149658)
\curveto(847.33938756,681.82148886)(847.43438746,681.7814889)(847.48438965,681.70149658)
\curveto(847.53438736,681.64148904)(847.55938734,681.56148912)(847.55938965,681.46149658)
\lineto(847.55938965,681.17649658)
\lineto(847.55938965,674.83149658)
}
}
{
\newrgbcolor{curcolor}{0 0 0}
\pscustom[linestyle=none,fillstyle=solid,fillcolor=curcolor]
{
\newpath
\moveto(850.4492334,684.17649658)
\curveto(850.59923139,684.1764865)(850.74923124,684.17148651)(850.8992334,684.16149658)
\curveto(851.04923094,684.16148652)(851.15423083,684.12148656)(851.2142334,684.04149658)
\curveto(851.26423072,683.9814867)(851.2892307,683.89648678)(851.2892334,683.78649658)
\curveto(851.29923069,683.68648699)(851.30423068,683.5814871)(851.3042334,683.47149658)
\lineto(851.3042334,682.60149658)
\curveto(851.30423068,682.52148816)(851.29923069,682.43648824)(851.2892334,682.34649658)
\curveto(851.2892307,682.26648841)(851.29923069,682.19648848)(851.3192334,682.13649658)
\curveto(851.35923063,681.99648868)(851.44923054,681.90648877)(851.5892334,681.86649658)
\curveto(851.63923035,681.85648882)(851.6842303,681.85148883)(851.7242334,681.85149658)
\lineto(851.8742334,681.85149658)
\lineto(852.2792334,681.85149658)
\curveto(852.43922955,681.86148882)(852.55422943,681.85148883)(852.6242334,681.82149658)
\curveto(852.71422927,681.76148892)(852.77422921,681.70148898)(852.8042334,681.64149658)
\curveto(852.82422916,681.60148908)(852.83422915,681.55648912)(852.8342334,681.50649658)
\lineto(852.8342334,681.35649658)
\curveto(852.83422915,681.24648943)(852.82922916,681.14148954)(852.8192334,681.04149658)
\curveto(852.80922918,680.95148973)(852.77422921,680.8814898)(852.7142334,680.83149658)
\curveto(852.65422933,680.7814899)(852.56922942,680.75148993)(852.4592334,680.74149658)
\lineto(852.1292334,680.74149658)
\curveto(852.01922997,680.75148993)(851.90923008,680.75648992)(851.7992334,680.75649658)
\curveto(851.6892303,680.75648992)(851.59423039,680.74148994)(851.5142334,680.71149658)
\curveto(851.44423054,680.68149)(851.39423059,680.63149005)(851.3642334,680.56149658)
\curveto(851.33423065,680.49149019)(851.31423067,680.40649027)(851.3042334,680.30649658)
\curveto(851.29423069,680.21649046)(851.2892307,680.11649056)(851.2892334,680.00649658)
\curveto(851.29923069,679.90649077)(851.30423068,679.80649087)(851.3042334,679.70649658)
\lineto(851.3042334,676.73649658)
\curveto(851.30423068,676.51649416)(851.29923069,676.2814944)(851.2892334,676.03149658)
\curveto(851.2892307,675.79149489)(851.33423065,675.60649507)(851.4242334,675.47649658)
\curveto(851.47423051,675.39649528)(851.53923045,675.34149534)(851.6192334,675.31149658)
\curveto(851.69923029,675.2814954)(851.79423019,675.25649542)(851.9042334,675.23649658)
\curveto(851.93423005,675.22649545)(851.96423002,675.22149546)(851.9942334,675.22149658)
\curveto(852.03422995,675.23149545)(852.06922992,675.23149545)(852.0992334,675.22149658)
\lineto(852.2942334,675.22149658)
\curveto(852.39422959,675.22149546)(852.4842295,675.21149547)(852.5642334,675.19149658)
\curveto(852.65422933,675.1814955)(852.71922927,675.14649553)(852.7592334,675.08649658)
\curveto(852.77922921,675.05649562)(852.79422919,675.00149568)(852.8042334,674.92149658)
\curveto(852.82422916,674.85149583)(852.83422915,674.7764959)(852.8342334,674.69649658)
\curveto(852.84422914,674.61649606)(852.84422914,674.53649614)(852.8342334,674.45649658)
\curveto(852.82422916,674.38649629)(852.80422918,674.33149635)(852.7742334,674.29149658)
\curveto(852.73422925,674.22149646)(852.65922933,674.17149651)(852.5492334,674.14149658)
\curveto(852.46922952,674.12149656)(852.37922961,674.11149657)(852.2792334,674.11149658)
\curveto(852.17922981,674.12149656)(852.0892299,674.12649655)(852.0092334,674.12649658)
\curveto(851.94923004,674.12649655)(851.8892301,674.12149656)(851.8292334,674.11149658)
\curveto(851.76923022,674.11149657)(851.71423027,674.11649656)(851.6642334,674.12649658)
\lineto(851.4842334,674.12649658)
\curveto(851.43423055,674.13649654)(851.3842306,674.14149654)(851.3342334,674.14149658)
\curveto(851.29423069,674.15149653)(851.24923074,674.15649652)(851.1992334,674.15649658)
\curveto(850.99923099,674.20649647)(850.82423116,674.26149642)(850.6742334,674.32149658)
\curveto(850.53423145,674.3814963)(850.41423157,674.48649619)(850.3142334,674.63649658)
\curveto(850.17423181,674.83649584)(850.09423189,675.08649559)(850.0742334,675.38649658)
\curveto(850.05423193,675.69649498)(850.04423194,676.02649465)(850.0442334,676.37649658)
\lineto(850.0442334,680.30649658)
\curveto(850.01423197,680.43649024)(849.984232,680.53149015)(849.9542334,680.59149658)
\curveto(849.93423205,680.65149003)(849.86423212,680.70148998)(849.7442334,680.74149658)
\curveto(849.70423228,680.75148993)(849.66423232,680.75148993)(849.6242334,680.74149658)
\curveto(849.5842324,680.73148995)(849.54423244,680.73648994)(849.5042334,680.75649658)
\lineto(849.2642334,680.75649658)
\curveto(849.13423285,680.75648992)(849.02423296,680.76648991)(848.9342334,680.78649658)
\curveto(848.85423313,680.81648986)(848.79923319,680.8764898)(848.7692334,680.96649658)
\curveto(848.74923324,681.00648967)(848.73423325,681.05148963)(848.7242334,681.10149658)
\lineto(848.7242334,681.25149658)
\curveto(848.72423326,681.39148929)(848.73423325,681.50648917)(848.7542334,681.59649658)
\curveto(848.77423321,681.69648898)(848.83423315,681.77148891)(848.9342334,681.82149658)
\curveto(849.04423294,681.86148882)(849.1842328,681.87148881)(849.3542334,681.85149658)
\curveto(849.53423245,681.83148885)(849.6842323,681.84148884)(849.8042334,681.88149658)
\curveto(849.89423209,681.93148875)(849.96423202,682.00148868)(850.0142334,682.09149658)
\curveto(850.03423195,682.15148853)(850.04423194,682.22648845)(850.0442334,682.31649658)
\lineto(850.0442334,682.57149658)
\lineto(850.0442334,683.50149658)
\lineto(850.0442334,683.74149658)
\curveto(850.04423194,683.83148685)(850.05423193,683.90648677)(850.0742334,683.96649658)
\curveto(850.11423187,684.04648663)(850.1892318,684.11148657)(850.2992334,684.16149658)
\curveto(850.32923166,684.16148652)(850.35423163,684.16148652)(850.3742334,684.16149658)
\curveto(850.40423158,684.17148651)(850.42923156,684.1764865)(850.4492334,684.17649658)
}
}
{
\newrgbcolor{curcolor}{0 0 0}
\pscustom[linestyle=none,fillstyle=solid,fillcolor=curcolor]
{
\newpath
\moveto(861.10603027,674.66649658)
\curveto(861.13602244,674.50649617)(861.12102246,674.37149631)(861.06103027,674.26149658)
\curveto(861.00102258,674.16149652)(860.92102266,674.08649659)(860.82103027,674.03649658)
\curveto(860.77102281,674.01649666)(860.71602286,674.00649667)(860.65603027,674.00649658)
\curveto(860.60602297,674.00649667)(860.55102303,673.99649668)(860.49103027,673.97649658)
\curveto(860.27102331,673.92649675)(860.05102353,673.94149674)(859.83103027,674.02149658)
\curveto(859.62102396,674.09149659)(859.4760241,674.1814965)(859.39603027,674.29149658)
\curveto(859.34602423,674.36149632)(859.30102428,674.44149624)(859.26103027,674.53149658)
\curveto(859.22102436,674.63149605)(859.17102441,674.71149597)(859.11103027,674.77149658)
\curveto(859.09102449,674.79149589)(859.06602451,674.81149587)(859.03603027,674.83149658)
\curveto(859.01602456,674.85149583)(858.98602459,674.85649582)(858.94603027,674.84649658)
\curveto(858.83602474,674.81649586)(858.73102485,674.76149592)(858.63103027,674.68149658)
\curveto(858.54102504,674.60149608)(858.45102513,674.53149615)(858.36103027,674.47149658)
\curveto(858.23102535,674.39149629)(858.09102549,674.31649636)(857.94103027,674.24649658)
\curveto(857.79102579,674.18649649)(857.63102595,674.13149655)(857.46103027,674.08149658)
\curveto(857.36102622,674.05149663)(857.25102633,674.03149665)(857.13103027,674.02149658)
\curveto(857.02102656,674.01149667)(856.91102667,673.99649668)(856.80103027,673.97649658)
\curveto(856.75102683,673.96649671)(856.70602687,673.96149672)(856.66603027,673.96149658)
\lineto(856.56103027,673.96149658)
\curveto(856.45102713,673.94149674)(856.34602723,673.94149674)(856.24603027,673.96149658)
\lineto(856.11103027,673.96149658)
\curveto(856.06102752,673.97149671)(856.01102757,673.9764967)(855.96103027,673.97649658)
\curveto(855.91102767,673.9764967)(855.86602771,673.98649669)(855.82603027,674.00649658)
\curveto(855.78602779,674.01649666)(855.75102783,674.02149666)(855.72103027,674.02149658)
\curveto(855.70102788,674.01149667)(855.6760279,674.01149667)(855.64603027,674.02149658)
\lineto(855.40603027,674.08149658)
\curveto(855.32602825,674.09149659)(855.25102833,674.11149657)(855.18103027,674.14149658)
\curveto(854.8810287,674.27149641)(854.63602894,674.41649626)(854.44603027,674.57649658)
\curveto(854.26602931,674.74649593)(854.11602946,674.9814957)(853.99603027,675.28149658)
\curveto(853.90602967,675.50149518)(853.86102972,675.76649491)(853.86103027,676.07649658)
\lineto(853.86103027,676.39149658)
\curveto(853.87102971,676.44149424)(853.8760297,676.49149419)(853.87603027,676.54149658)
\lineto(853.90603027,676.72149658)
\lineto(854.02603027,677.05149658)
\curveto(854.06602951,677.16149352)(854.11602946,677.26149342)(854.17603027,677.35149658)
\curveto(854.35602922,677.64149304)(854.60102898,677.85649282)(854.91103027,677.99649658)
\curveto(855.22102836,678.13649254)(855.56102802,678.26149242)(855.93103027,678.37149658)
\curveto(856.07102751,678.41149227)(856.21602736,678.44149224)(856.36603027,678.46149658)
\curveto(856.51602706,678.4814922)(856.66602691,678.50649217)(856.81603027,678.53649658)
\curveto(856.88602669,678.55649212)(856.95102663,678.56649211)(857.01103027,678.56649658)
\curveto(857.0810265,678.56649211)(857.15602642,678.5764921)(857.23603027,678.59649658)
\curveto(857.30602627,678.61649206)(857.3760262,678.62649205)(857.44603027,678.62649658)
\curveto(857.51602606,678.63649204)(857.59102599,678.65149203)(857.67103027,678.67149658)
\curveto(857.92102566,678.73149195)(858.15602542,678.7814919)(858.37603027,678.82149658)
\curveto(858.59602498,678.87149181)(858.77102481,678.98649169)(858.90103027,679.16649658)
\curveto(858.96102462,679.24649143)(859.01102457,679.34649133)(859.05103027,679.46649658)
\curveto(859.09102449,679.59649108)(859.09102449,679.73649094)(859.05103027,679.88649658)
\curveto(858.99102459,680.12649055)(858.90102468,680.31649036)(858.78103027,680.45649658)
\curveto(858.67102491,680.59649008)(858.51102507,680.70648997)(858.30103027,680.78649658)
\curveto(858.1810254,680.83648984)(858.03602554,680.87148981)(857.86603027,680.89149658)
\curveto(857.70602587,680.91148977)(857.53602604,680.92148976)(857.35603027,680.92149658)
\curveto(857.1760264,680.92148976)(857.00102658,680.91148977)(856.83103027,680.89149658)
\curveto(856.66102692,680.87148981)(856.51602706,680.84148984)(856.39603027,680.80149658)
\curveto(856.22602735,680.74148994)(856.06102752,680.65649002)(855.90103027,680.54649658)
\curveto(855.82102776,680.48649019)(855.74602783,680.40649027)(855.67603027,680.30649658)
\curveto(855.61602796,680.21649046)(855.56102802,680.11649056)(855.51103027,680.00649658)
\curveto(855.4810281,679.92649075)(855.45102813,679.84149084)(855.42103027,679.75149658)
\curveto(855.40102818,679.66149102)(855.35602822,679.59149109)(855.28603027,679.54149658)
\curveto(855.24602833,679.51149117)(855.1760284,679.48649119)(855.07603027,679.46649658)
\curveto(854.98602859,679.45649122)(854.89102869,679.45149123)(854.79103027,679.45149658)
\curveto(854.69102889,679.45149123)(854.59102899,679.45649122)(854.49103027,679.46649658)
\curveto(854.40102918,679.48649119)(854.33602924,679.51149117)(854.29603027,679.54149658)
\curveto(854.25602932,679.57149111)(854.22602935,679.62149106)(854.20603027,679.69149658)
\curveto(854.18602939,679.76149092)(854.18602939,679.83649084)(854.20603027,679.91649658)
\curveto(854.23602934,680.04649063)(854.26602931,680.16649051)(854.29603027,680.27649658)
\curveto(854.33602924,680.39649028)(854.3810292,680.51149017)(854.43103027,680.62149658)
\curveto(854.62102896,680.97148971)(854.86102872,681.24148944)(855.15103027,681.43149658)
\curveto(855.44102814,681.63148905)(855.80102778,681.79148889)(856.23103027,681.91149658)
\curveto(856.33102725,681.93148875)(856.43102715,681.94648873)(856.53103027,681.95649658)
\curveto(856.64102694,681.96648871)(856.75102683,681.9814887)(856.86103027,682.00149658)
\curveto(856.90102668,682.01148867)(856.96602661,682.01148867)(857.05603027,682.00149658)
\curveto(857.14602643,682.00148868)(857.20102638,682.01148867)(857.22103027,682.03149658)
\curveto(857.92102566,682.04148864)(858.53102505,681.96148872)(859.05103027,681.79149658)
\curveto(859.57102401,681.62148906)(859.93602364,681.29648938)(860.14603027,680.81649658)
\curveto(860.23602334,680.61649006)(860.28602329,680.3814903)(860.29603027,680.11149658)
\curveto(860.31602326,679.85149083)(860.32602325,679.5764911)(860.32603027,679.28649658)
\lineto(860.32603027,675.97149658)
\curveto(860.32602325,675.83149485)(860.33102325,675.69649498)(860.34103027,675.56649658)
\curveto(860.35102323,675.43649524)(860.3810232,675.33149535)(860.43103027,675.25149658)
\curveto(860.4810231,675.1814955)(860.54602303,675.13149555)(860.62603027,675.10149658)
\curveto(860.71602286,675.06149562)(860.80102278,675.03149565)(860.88103027,675.01149658)
\curveto(860.96102262,675.00149568)(861.02102256,674.95649572)(861.06103027,674.87649658)
\curveto(861.0810225,674.84649583)(861.09102249,674.81649586)(861.09103027,674.78649658)
\curveto(861.09102249,674.75649592)(861.09602248,674.71649596)(861.10603027,674.66649658)
\moveto(858.96103027,676.33149658)
\curveto(859.02102456,676.47149421)(859.05102453,676.63149405)(859.05103027,676.81149658)
\curveto(859.06102452,677.00149368)(859.06602451,677.19649348)(859.06603027,677.39649658)
\curveto(859.06602451,677.50649317)(859.06102452,677.60649307)(859.05103027,677.69649658)
\curveto(859.04102454,677.78649289)(859.00102458,677.85649282)(858.93103027,677.90649658)
\curveto(858.90102468,677.92649275)(858.83102475,677.93649274)(858.72103027,677.93649658)
\curveto(858.70102488,677.91649276)(858.66602491,677.90649277)(858.61603027,677.90649658)
\curveto(858.56602501,677.90649277)(858.52102506,677.89649278)(858.48103027,677.87649658)
\curveto(858.40102518,677.85649282)(858.31102527,677.83649284)(858.21103027,677.81649658)
\lineto(857.91103027,677.75649658)
\curveto(857.8810257,677.75649292)(857.84602573,677.75149293)(857.80603027,677.74149658)
\lineto(857.70103027,677.74149658)
\curveto(857.55102603,677.70149298)(857.38602619,677.676493)(857.20603027,677.66649658)
\curveto(857.03602654,677.66649301)(856.8760267,677.64649303)(856.72603027,677.60649658)
\curveto(856.64602693,677.58649309)(856.57102701,677.56649311)(856.50103027,677.54649658)
\curveto(856.44102714,677.53649314)(856.37102721,677.52149316)(856.29103027,677.50149658)
\curveto(856.13102745,677.45149323)(855.9810276,677.38649329)(855.84103027,677.30649658)
\curveto(855.70102788,677.23649344)(855.581028,677.14649353)(855.48103027,677.03649658)
\curveto(855.3810282,676.92649375)(855.30602827,676.79149389)(855.25603027,676.63149658)
\curveto(855.20602837,676.4814942)(855.18602839,676.29649438)(855.19603027,676.07649658)
\curveto(855.19602838,675.9764947)(855.21102837,675.8814948)(855.24103027,675.79149658)
\curveto(855.2810283,675.71149497)(855.32602825,675.63649504)(855.37603027,675.56649658)
\curveto(855.45602812,675.45649522)(855.56102802,675.36149532)(855.69103027,675.28149658)
\curveto(855.82102776,675.21149547)(855.96102762,675.15149553)(856.11103027,675.10149658)
\curveto(856.16102742,675.09149559)(856.21102737,675.08649559)(856.26103027,675.08649658)
\curveto(856.31102727,675.08649559)(856.36102722,675.0814956)(856.41103027,675.07149658)
\curveto(856.4810271,675.05149563)(856.56602701,675.03649564)(856.66603027,675.02649658)
\curveto(856.7760268,675.02649565)(856.86602671,675.03649564)(856.93603027,675.05649658)
\curveto(856.99602658,675.0764956)(857.05602652,675.0814956)(857.11603027,675.07149658)
\curveto(857.1760264,675.07149561)(857.23602634,675.0814956)(857.29603027,675.10149658)
\curveto(857.3760262,675.12149556)(857.45102613,675.13649554)(857.52103027,675.14649658)
\curveto(857.60102598,675.15649552)(857.6760259,675.1764955)(857.74603027,675.20649658)
\curveto(858.03602554,675.32649535)(858.2810253,675.47149521)(858.48103027,675.64149658)
\curveto(858.69102489,675.81149487)(858.85102473,676.04149464)(858.96103027,676.33149658)
}
}
{
\newrgbcolor{curcolor}{0 0 0}
\pscustom[linestyle=none,fillstyle=solid,fillcolor=curcolor]
{
\newpath
\moveto(869.2376709,674.92149658)
\lineto(869.2376709,674.53149658)
\curveto(869.23766302,674.41149627)(869.21266305,674.31149637)(869.1626709,674.23149658)
\curveto(869.11266315,674.16149652)(869.02766323,674.12149656)(868.9076709,674.11149658)
\lineto(868.5626709,674.11149658)
\curveto(868.50266376,674.11149657)(868.44266382,674.10649657)(868.3826709,674.09649658)
\curveto(868.33266393,674.09649658)(868.28766397,674.10649657)(868.2476709,674.12649658)
\curveto(868.1576641,674.14649653)(868.09766416,674.18649649)(868.0676709,674.24649658)
\curveto(868.02766423,674.29649638)(868.00266426,674.35649632)(867.9926709,674.42649658)
\curveto(867.99266427,674.49649618)(867.97766428,674.56649611)(867.9476709,674.63649658)
\curveto(867.93766432,674.65649602)(867.92266434,674.67149601)(867.9026709,674.68149658)
\curveto(867.89266437,674.70149598)(867.87766438,674.72149596)(867.8576709,674.74149658)
\curveto(867.7576645,674.75149593)(867.67766458,674.73149595)(867.6176709,674.68149658)
\curveto(867.56766469,674.63149605)(867.51266475,674.5814961)(867.4526709,674.53149658)
\curveto(867.25266501,674.3814963)(867.05266521,674.26649641)(866.8526709,674.18649658)
\curveto(866.67266559,674.10649657)(866.4626658,674.04649663)(866.2226709,674.00649658)
\curveto(865.99266627,673.96649671)(865.75266651,673.94649673)(865.5026709,673.94649658)
\curveto(865.262667,673.93649674)(865.02266724,673.95149673)(864.7826709,673.99149658)
\curveto(864.54266772,674.02149666)(864.33266793,674.0764966)(864.1526709,674.15649658)
\curveto(863.63266863,674.3764963)(863.21266905,674.67149601)(862.8926709,675.04149658)
\curveto(862.57266969,675.42149526)(862.32266994,675.89149479)(862.1426709,676.45149658)
\curveto(862.10267016,676.54149414)(862.07267019,676.63149405)(862.0526709,676.72149658)
\curveto(862.04267022,676.82149386)(862.02267024,676.92149376)(861.9926709,677.02149658)
\curveto(861.98267028,677.07149361)(861.97767028,677.12149356)(861.9776709,677.17149658)
\curveto(861.97767028,677.22149346)(861.97267029,677.27149341)(861.9626709,677.32149658)
\curveto(861.94267032,677.37149331)(861.93267033,677.42149326)(861.9326709,677.47149658)
\curveto(861.94267032,677.53149315)(861.94267032,677.58649309)(861.9326709,677.63649658)
\lineto(861.9326709,677.78649658)
\curveto(861.91267035,677.83649284)(861.90267036,677.90149278)(861.9026709,677.98149658)
\curveto(861.90267036,678.06149262)(861.91267035,678.12649255)(861.9326709,678.17649658)
\lineto(861.9326709,678.34149658)
\curveto(861.95267031,678.41149227)(861.9576703,678.4814922)(861.9476709,678.55149658)
\curveto(861.94767031,678.63149205)(861.9576703,678.70649197)(861.9776709,678.77649658)
\curveto(861.98767027,678.82649185)(861.99267027,678.87149181)(861.9926709,678.91149658)
\curveto(861.99267027,678.95149173)(861.99767026,678.99649168)(862.0076709,679.04649658)
\curveto(862.03767022,679.14649153)(862.0626702,679.24149144)(862.0826709,679.33149658)
\curveto(862.10267016,679.43149125)(862.12767013,679.52649115)(862.1576709,679.61649658)
\curveto(862.28766997,679.99649068)(862.45266981,680.33649034)(862.6526709,680.63649658)
\curveto(862.8626694,680.94648973)(863.11266915,681.20148948)(863.4026709,681.40149658)
\curveto(863.57266869,681.52148916)(863.74766851,681.62148906)(863.9276709,681.70149658)
\curveto(864.11766814,681.7814889)(864.32266794,681.85148883)(864.5426709,681.91149658)
\curveto(864.61266765,681.92148876)(864.67766758,681.93148875)(864.7376709,681.94149658)
\curveto(864.80766745,681.95148873)(864.87766738,681.96648871)(864.9476709,681.98649658)
\lineto(865.0976709,681.98649658)
\curveto(865.17766708,682.00648867)(865.29266697,682.01648866)(865.4426709,682.01649658)
\curveto(865.60266666,682.01648866)(865.72266654,682.00648867)(865.8026709,681.98649658)
\curveto(865.84266642,681.9764887)(865.89766636,681.97148871)(865.9676709,681.97149658)
\curveto(866.07766618,681.94148874)(866.18766607,681.91648876)(866.2976709,681.89649658)
\curveto(866.40766585,681.88648879)(866.51266575,681.85648882)(866.6126709,681.80649658)
\curveto(866.7626655,681.74648893)(866.90266536,681.681489)(867.0326709,681.61149658)
\curveto(867.17266509,681.54148914)(867.30266496,681.46148922)(867.4226709,681.37149658)
\curveto(867.48266478,681.32148936)(867.54266472,681.26648941)(867.6026709,681.20649658)
\curveto(867.67266459,681.15648952)(867.7626645,681.14148954)(867.8726709,681.16149658)
\curveto(867.89266437,681.19148949)(867.90766435,681.21648946)(867.9176709,681.23649658)
\curveto(867.93766432,681.25648942)(867.95266431,681.28648939)(867.9626709,681.32649658)
\curveto(867.99266427,681.41648926)(868.00266426,681.53148915)(867.9926709,681.67149658)
\lineto(867.9926709,682.04649658)
\lineto(867.9926709,683.77149658)
\lineto(867.9926709,684.23649658)
\curveto(867.99266427,684.41648626)(868.01766424,684.54648613)(868.0676709,684.62649658)
\curveto(868.10766415,684.69648598)(868.16766409,684.74148594)(868.2476709,684.76149658)
\curveto(868.26766399,684.76148592)(868.29266397,684.76148592)(868.3226709,684.76149658)
\curveto(868.35266391,684.77148591)(868.37766388,684.7764859)(868.3976709,684.77649658)
\curveto(868.53766372,684.78648589)(868.68266358,684.78648589)(868.8326709,684.77649658)
\curveto(868.99266327,684.7764859)(869.10266316,684.73648594)(869.1626709,684.65649658)
\curveto(869.21266305,684.5764861)(869.23766302,684.4764862)(869.2376709,684.35649658)
\lineto(869.2376709,683.98149658)
\lineto(869.2376709,674.92149658)
\moveto(868.0226709,677.75649658)
\curveto(868.04266422,677.80649287)(868.05266421,677.87149281)(868.0526709,677.95149658)
\curveto(868.05266421,678.04149264)(868.04266422,678.11149257)(868.0226709,678.16149658)
\lineto(868.0226709,678.38649658)
\curveto(868.00266426,678.4764922)(867.98766427,678.56649211)(867.9776709,678.65649658)
\curveto(867.96766429,678.75649192)(867.94766431,678.84649183)(867.9176709,678.92649658)
\curveto(867.89766436,679.00649167)(867.87766438,679.0814916)(867.8576709,679.15149658)
\curveto(867.84766441,679.22149146)(867.82766443,679.29149139)(867.7976709,679.36149658)
\curveto(867.67766458,679.66149102)(867.52266474,679.92649075)(867.3326709,680.15649658)
\curveto(867.14266512,680.38649029)(866.90266536,680.56649011)(866.6126709,680.69649658)
\curveto(866.51266575,680.74648993)(866.40766585,680.7814899)(866.2976709,680.80149658)
\curveto(866.19766606,680.83148985)(866.08766617,680.85648982)(865.9676709,680.87649658)
\curveto(865.88766637,680.89648978)(865.79766646,680.90648977)(865.6976709,680.90649658)
\lineto(865.4276709,680.90649658)
\curveto(865.37766688,680.89648978)(865.33266693,680.88648979)(865.2926709,680.87649658)
\lineto(865.1576709,680.87649658)
\curveto(865.07766718,680.85648982)(864.99266727,680.83648984)(864.9026709,680.81649658)
\curveto(864.82266744,680.79648988)(864.74266752,680.77148991)(864.6626709,680.74149658)
\curveto(864.34266792,680.60149008)(864.08266818,680.39649028)(863.8826709,680.12649658)
\curveto(863.69266857,679.86649081)(863.53766872,679.56149112)(863.4176709,679.21149658)
\curveto(863.37766888,679.10149158)(863.34766891,678.98649169)(863.3276709,678.86649658)
\curveto(863.31766894,678.75649192)(863.30266896,678.64649203)(863.2826709,678.53649658)
\curveto(863.28266898,678.49649218)(863.27766898,678.45649222)(863.2676709,678.41649658)
\lineto(863.2676709,678.31149658)
\curveto(863.24766901,678.26149242)(863.23766902,678.20649247)(863.2376709,678.14649658)
\curveto(863.24766901,678.08649259)(863.25266901,678.03149265)(863.2526709,677.98149658)
\lineto(863.2526709,677.65149658)
\curveto(863.25266901,677.55149313)(863.262669,677.45649322)(863.2826709,677.36649658)
\curveto(863.29266897,677.33649334)(863.29766896,677.28649339)(863.2976709,677.21649658)
\curveto(863.31766894,677.14649353)(863.33266893,677.0764936)(863.3426709,677.00649658)
\lineto(863.4026709,676.79649658)
\curveto(863.51266875,676.44649423)(863.6626686,676.14649453)(863.8526709,675.89649658)
\curveto(864.04266822,675.64649503)(864.28266798,675.44149524)(864.5726709,675.28149658)
\curveto(864.6626676,675.23149545)(864.75266751,675.19149549)(864.8426709,675.16149658)
\curveto(864.93266733,675.13149555)(865.03266723,675.10149558)(865.1426709,675.07149658)
\curveto(865.19266707,675.05149563)(865.24266702,675.04649563)(865.2926709,675.05649658)
\curveto(865.35266691,675.06649561)(865.40766685,675.06149562)(865.4576709,675.04149658)
\curveto(865.49766676,675.03149565)(865.53766672,675.02649565)(865.5776709,675.02649658)
\lineto(865.7126709,675.02649658)
\lineto(865.8476709,675.02649658)
\curveto(865.87766638,675.03649564)(865.92766633,675.04149564)(865.9976709,675.04149658)
\curveto(866.07766618,675.06149562)(866.1576661,675.0764956)(866.2376709,675.08649658)
\curveto(866.31766594,675.10649557)(866.39266587,675.13149555)(866.4626709,675.16149658)
\curveto(866.79266547,675.30149538)(867.0576652,675.4764952)(867.2576709,675.68649658)
\curveto(867.46766479,675.90649477)(867.64266462,676.1814945)(867.7826709,676.51149658)
\curveto(867.83266443,676.62149406)(867.86766439,676.73149395)(867.8876709,676.84149658)
\curveto(867.90766435,676.95149373)(867.93266433,677.06149362)(867.9626709,677.17149658)
\curveto(867.98266428,677.21149347)(867.99266427,677.24649343)(867.9926709,677.27649658)
\curveto(867.99266427,677.31649336)(867.99766426,677.35649332)(868.0076709,677.39649658)
\curveto(868.01766424,677.45649322)(868.01766424,677.51649316)(868.0076709,677.57649658)
\curveto(868.00766425,677.63649304)(868.01266425,677.69649298)(868.0226709,677.75649658)
}
}
{
\newrgbcolor{curcolor}{0 0 0}
\pscustom[linestyle=none,fillstyle=solid,fillcolor=curcolor]
{
\newpath
\moveto(878.3089209,678.31149658)
\curveto(878.32891284,678.25149243)(878.33891283,678.15649252)(878.3389209,678.02649658)
\curveto(878.33891283,677.90649277)(878.33391283,677.82149286)(878.3239209,677.77149658)
\lineto(878.3239209,677.62149658)
\curveto(878.31391285,677.54149314)(878.30391286,677.46649321)(878.2939209,677.39649658)
\curveto(878.29391287,677.33649334)(878.28891288,677.26649341)(878.2789209,677.18649658)
\curveto(878.25891291,677.12649355)(878.24391292,677.06649361)(878.2339209,677.00649658)
\curveto(878.23391293,676.94649373)(878.22391294,676.88649379)(878.2039209,676.82649658)
\curveto(878.163913,676.69649398)(878.12891304,676.56649411)(878.0989209,676.43649658)
\curveto(878.0689131,676.30649437)(878.02891314,676.18649449)(877.9789209,676.07649658)
\curveto(877.7689134,675.59649508)(877.48891368,675.19149549)(877.1389209,674.86149658)
\curveto(876.78891438,674.54149614)(876.35891481,674.29649638)(875.8489209,674.12649658)
\curveto(875.73891543,674.08649659)(875.61891555,674.05649662)(875.4889209,674.03649658)
\curveto(875.3689158,674.01649666)(875.24391592,673.99649668)(875.1139209,673.97649658)
\curveto(875.05391611,673.96649671)(874.98891618,673.96149672)(874.9189209,673.96149658)
\curveto(874.85891631,673.95149673)(874.79891637,673.94649673)(874.7389209,673.94649658)
\curveto(874.69891647,673.93649674)(874.63891653,673.93149675)(874.5589209,673.93149658)
\curveto(874.48891668,673.93149675)(874.43891673,673.93649674)(874.4089209,673.94649658)
\curveto(874.3689168,673.95649672)(874.32891684,673.96149672)(874.2889209,673.96149658)
\curveto(874.24891692,673.95149673)(874.21391695,673.95149673)(874.1839209,673.96149658)
\lineto(874.0939209,673.96149658)
\lineto(873.7339209,674.00649658)
\curveto(873.59391757,674.04649663)(873.45891771,674.08649659)(873.3289209,674.12649658)
\curveto(873.19891797,674.16649651)(873.07391809,674.21149647)(872.9539209,674.26149658)
\curveto(872.50391866,674.46149622)(872.13391903,674.72149596)(871.8439209,675.04149658)
\curveto(871.55391961,675.36149532)(871.31391985,675.75149493)(871.1239209,676.21149658)
\curveto(871.07392009,676.31149437)(871.03392013,676.41149427)(871.0039209,676.51149658)
\curveto(870.98392018,676.61149407)(870.9639202,676.71649396)(870.9439209,676.82649658)
\curveto(870.92392024,676.86649381)(870.91392025,676.89649378)(870.9139209,676.91649658)
\curveto(870.92392024,676.94649373)(870.92392024,676.9814937)(870.9139209,677.02149658)
\curveto(870.89392027,677.10149358)(870.87892029,677.1814935)(870.8689209,677.26149658)
\curveto(870.8689203,677.35149333)(870.85892031,677.43649324)(870.8389209,677.51649658)
\lineto(870.8389209,677.63649658)
\curveto(870.83892033,677.676493)(870.83392033,677.72149296)(870.8239209,677.77149658)
\curveto(870.81392035,677.82149286)(870.80892036,677.90649277)(870.8089209,678.02649658)
\curveto(870.80892036,678.15649252)(870.81892035,678.25149243)(870.8389209,678.31149658)
\curveto(870.85892031,678.3814923)(870.8639203,678.45149223)(870.8539209,678.52149658)
\curveto(870.84392032,678.59149209)(870.84892032,678.66149202)(870.8689209,678.73149658)
\curveto(870.87892029,678.7814919)(870.88392028,678.82149186)(870.8839209,678.85149658)
\curveto(870.89392027,678.89149179)(870.90392026,678.93649174)(870.9139209,678.98649658)
\curveto(870.94392022,679.10649157)(870.9689202,679.22649145)(870.9889209,679.34649658)
\curveto(871.01892015,679.46649121)(871.05892011,679.5814911)(871.1089209,679.69149658)
\curveto(871.25891991,680.06149062)(871.43891973,680.39149029)(871.6489209,680.68149658)
\curveto(871.8689193,680.9814897)(872.13391903,681.23148945)(872.4439209,681.43149658)
\curveto(872.5639186,681.51148917)(872.68891848,681.5764891)(872.8189209,681.62649658)
\curveto(872.94891822,681.68648899)(873.08391808,681.74648893)(873.2239209,681.80649658)
\curveto(873.34391782,681.85648882)(873.47391769,681.88648879)(873.6139209,681.89649658)
\curveto(873.75391741,681.91648876)(873.89391727,681.94648873)(874.0339209,681.98649658)
\lineto(874.2289209,681.98649658)
\curveto(874.29891687,681.99648868)(874.3639168,682.00648867)(874.4239209,682.01649658)
\curveto(875.31391585,682.02648865)(876.05391511,681.84148884)(876.6439209,681.46149658)
\curveto(877.23391393,681.0814896)(877.65891351,680.58649009)(877.9189209,679.97649658)
\curveto(877.9689132,679.8764908)(878.00891316,679.7764909)(878.0389209,679.67649658)
\curveto(878.0689131,679.5764911)(878.10391306,679.47149121)(878.1439209,679.36149658)
\curveto(878.17391299,679.25149143)(878.19891297,679.13149155)(878.2189209,679.00149658)
\curveto(878.23891293,678.8814918)(878.2639129,678.75649192)(878.2939209,678.62649658)
\curveto(878.30391286,678.5764921)(878.30391286,678.52149216)(878.2939209,678.46149658)
\curveto(878.29391287,678.41149227)(878.29891287,678.36149232)(878.3089209,678.31149658)
\moveto(876.9739209,677.45649658)
\curveto(876.99391417,677.52649315)(876.99891417,677.60649307)(876.9889209,677.69649658)
\lineto(876.9889209,677.95149658)
\curveto(876.98891418,678.34149234)(876.95391421,678.67149201)(876.8839209,678.94149658)
\curveto(876.85391431,679.02149166)(876.82891434,679.10149158)(876.8089209,679.18149658)
\curveto(876.78891438,679.26149142)(876.7639144,679.33649134)(876.7339209,679.40649658)
\curveto(876.45391471,680.05649062)(876.00891516,680.50649017)(875.3989209,680.75649658)
\curveto(875.32891584,680.78648989)(875.25391591,680.80648987)(875.1739209,680.81649658)
\lineto(874.9339209,680.87649658)
\curveto(874.85391631,680.89648978)(874.7689164,680.90648977)(874.6789209,680.90649658)
\lineto(874.4089209,680.90649658)
\lineto(874.1389209,680.86149658)
\curveto(874.03891713,680.84148984)(873.94391722,680.81648986)(873.8539209,680.78649658)
\curveto(873.77391739,680.76648991)(873.69391747,680.73648994)(873.6139209,680.69649658)
\curveto(873.54391762,680.67649)(873.47891769,680.64649003)(873.4189209,680.60649658)
\curveto(873.35891781,680.56649011)(873.30391786,680.52649015)(873.2539209,680.48649658)
\curveto(873.01391815,680.31649036)(872.81891835,680.11149057)(872.6689209,679.87149658)
\curveto(872.51891865,679.63149105)(872.38891878,679.35149133)(872.2789209,679.03149658)
\curveto(872.24891892,678.93149175)(872.22891894,678.82649185)(872.2189209,678.71649658)
\curveto(872.20891896,678.61649206)(872.19391897,678.51149217)(872.1739209,678.40149658)
\curveto(872.163919,678.36149232)(872.15891901,678.29649238)(872.1589209,678.20649658)
\curveto(872.14891902,678.1764925)(872.14391902,678.14149254)(872.1439209,678.10149658)
\curveto(872.15391901,678.06149262)(872.15891901,678.01649266)(872.1589209,677.96649658)
\lineto(872.1589209,677.66649658)
\curveto(872.15891901,677.56649311)(872.168919,677.4764932)(872.1889209,677.39649658)
\lineto(872.2189209,677.21649658)
\curveto(872.23891893,677.11649356)(872.25391891,677.01649366)(872.2639209,676.91649658)
\curveto(872.28391888,676.82649385)(872.31391885,676.74149394)(872.3539209,676.66149658)
\curveto(872.45391871,676.42149426)(872.5689186,676.19649448)(872.6989209,675.98649658)
\curveto(872.83891833,675.7764949)(873.00891816,675.60149508)(873.2089209,675.46149658)
\curveto(873.25891791,675.43149525)(873.30391786,675.40649527)(873.3439209,675.38649658)
\curveto(873.38391778,675.36649531)(873.42891774,675.34149534)(873.4789209,675.31149658)
\curveto(873.55891761,675.26149542)(873.64391752,675.21649546)(873.7339209,675.17649658)
\curveto(873.83391733,675.14649553)(873.93891723,675.11649556)(874.0489209,675.08649658)
\curveto(874.09891707,675.06649561)(874.14391702,675.05649562)(874.1839209,675.05649658)
\curveto(874.23391693,675.06649561)(874.28391688,675.06649561)(874.3339209,675.05649658)
\curveto(874.3639168,675.04649563)(874.42391674,675.03649564)(874.5139209,675.02649658)
\curveto(874.61391655,675.01649566)(874.68891648,675.02149566)(874.7389209,675.04149658)
\curveto(874.77891639,675.05149563)(874.81891635,675.05149563)(874.8589209,675.04149658)
\curveto(874.89891627,675.04149564)(874.93891623,675.05149563)(874.9789209,675.07149658)
\curveto(875.05891611,675.09149559)(875.13891603,675.10649557)(875.2189209,675.11649658)
\curveto(875.29891587,675.13649554)(875.37391579,675.16149552)(875.4439209,675.19149658)
\curveto(875.78391538,675.33149535)(876.05891511,675.52649515)(876.2689209,675.77649658)
\curveto(876.47891469,676.02649465)(876.65391451,676.32149436)(876.7939209,676.66149658)
\curveto(876.84391432,676.7814939)(876.87391429,676.90649377)(876.8839209,677.03649658)
\curveto(876.90391426,677.1764935)(876.93391423,677.31649336)(876.9739209,677.45649658)
}
}
{
\newrgbcolor{curcolor}{0.90196079 0.90196079 0.90196079}
\pscustom[linestyle=none,fillstyle=solid,fillcolor=curcolor]
{
\newpath
\moveto(807.09008789,684.8215332)
\lineto(822.09008789,684.8215332)
\lineto(822.09008789,669.8215332)
\lineto(807.09008789,669.8215332)
\closepath
}
}
{
\newrgbcolor{curcolor}{0 0 0}
\pscustom[linestyle=none,fillstyle=solid,fillcolor=curcolor]
{
\newpath
\moveto(827.0782959,661.75579102)
\lineto(831.9832959,661.75579102)
\lineto(833.2732959,661.75579102)
\curveto(833.38328802,661.75578032)(833.49328791,661.75578032)(833.6032959,661.75579102)
\curveto(833.71328769,661.76578031)(833.8032876,661.74578033)(833.8732959,661.69579102)
\curveto(833.9032875,661.6757804)(833.92828747,661.65078043)(833.9482959,661.62079102)
\curveto(833.96828743,661.59078049)(833.98828741,661.56078052)(834.0082959,661.53079102)
\curveto(834.02828737,661.46078062)(834.03828736,661.34578073)(834.0382959,661.18579102)
\curveto(834.03828736,661.03578104)(834.02828737,660.92078116)(834.0082959,660.84079102)
\curveto(833.96828743,660.70078138)(833.88328752,660.62078146)(833.7532959,660.60079102)
\curveto(833.62328778,660.59078149)(833.46828793,660.58578149)(833.2882959,660.58579102)
\lineto(831.7882959,660.58579102)
\lineto(829.2682959,660.58579102)
\lineto(828.6982959,660.58579102)
\curveto(828.48829291,660.59578148)(828.33329307,660.57078151)(828.2332959,660.51079102)
\curveto(828.13329327,660.45078163)(828.07829332,660.34578173)(828.0682959,660.19579102)
\lineto(828.0682959,659.73079102)
\lineto(828.0682959,658.20079102)
\curveto(828.06829333,658.09078399)(828.06329334,657.96078412)(828.0532959,657.81079102)
\curveto(828.05329335,657.66078442)(828.06329334,657.54078454)(828.0832959,657.45079102)
\curveto(828.11329329,657.33078475)(828.17329323,657.25078483)(828.2632959,657.21079102)
\curveto(828.3032931,657.19078489)(828.37329303,657.17078491)(828.4732959,657.15079102)
\lineto(828.6232959,657.15079102)
\curveto(828.66329274,657.14078494)(828.7032927,657.13578494)(828.7432959,657.13579102)
\curveto(828.79329261,657.14578493)(828.84329256,657.15078493)(828.8932959,657.15079102)
\lineto(829.4032959,657.15079102)
\lineto(832.3432959,657.15079102)
\lineto(832.6432959,657.15079102)
\curveto(832.75328865,657.16078492)(832.86328854,657.16078492)(832.9732959,657.15079102)
\curveto(833.09328831,657.15078493)(833.1982882,657.14078494)(833.2882959,657.12079102)
\curveto(833.38828801,657.11078497)(833.46328794,657.09078499)(833.5132959,657.06079102)
\curveto(833.54328786,657.04078504)(833.56828783,656.99578508)(833.5882959,656.92579102)
\curveto(833.60828779,656.85578522)(833.62328778,656.7807853)(833.6332959,656.70079102)
\curveto(833.64328776,656.62078546)(833.64328776,656.53578554)(833.6332959,656.44579102)
\curveto(833.63328777,656.36578571)(833.62328778,656.29578578)(833.6032959,656.23579102)
\curveto(833.58328782,656.14578593)(833.53828786,656.080786)(833.4682959,656.04079102)
\curveto(833.44828795,656.02078606)(833.41828798,656.00578607)(833.3782959,655.99579102)
\curveto(833.34828805,655.99578608)(833.31828808,655.99078609)(833.2882959,655.98079102)
\lineto(833.1982959,655.98079102)
\curveto(833.14828825,655.97078611)(833.0982883,655.96578611)(833.0482959,655.96579102)
\curveto(832.9982884,655.9757861)(832.94828845,655.9807861)(832.8982959,655.98079102)
\lineto(832.3432959,655.98079102)
\lineto(829.1782959,655.98079102)
\lineto(828.8182959,655.98079102)
\curveto(828.70829269,655.99078609)(828.6032928,655.98578609)(828.5032959,655.96579102)
\curveto(828.403293,655.95578612)(828.31329309,655.93078615)(828.2332959,655.89079102)
\curveto(828.16329324,655.85078623)(828.11329329,655.7807863)(828.0832959,655.68079102)
\curveto(828.06329334,655.62078646)(828.05329335,655.55078653)(828.0532959,655.47079102)
\curveto(828.06329334,655.39078669)(828.06829333,655.31078677)(828.0682959,655.23079102)
\lineto(828.0682959,654.39079102)
\lineto(828.0682959,652.96579102)
\curveto(828.06829333,652.82578925)(828.07329333,652.69578938)(828.0832959,652.57579102)
\curveto(828.09329331,652.46578961)(828.13329327,652.38578969)(828.2032959,652.33579102)
\curveto(828.27329313,652.28578979)(828.35329305,652.25578982)(828.4432959,652.24579102)
\lineto(828.7432959,652.24579102)
\lineto(829.7032959,652.24579102)
\lineto(832.4782959,652.24579102)
\lineto(833.3332959,652.24579102)
\lineto(833.5732959,652.24579102)
\curveto(833.65328775,652.25578982)(833.72328768,652.25078983)(833.7832959,652.23079102)
\curveto(833.9032875,652.19078989)(833.98328742,652.13578994)(834.0232959,652.06579102)
\curveto(834.04328736,652.03579004)(834.05828734,651.98579009)(834.0682959,651.91579102)
\curveto(834.07828732,651.84579023)(834.08328732,651.77079031)(834.0832959,651.69079102)
\curveto(834.09328731,651.62079046)(834.09328731,651.54579053)(834.0832959,651.46579102)
\curveto(834.07328733,651.39579068)(834.06328734,651.34079074)(834.0532959,651.30079102)
\curveto(834.01328739,651.22079086)(833.96828743,651.16579091)(833.9182959,651.13579102)
\curveto(833.85828754,651.09579098)(833.77828762,651.075791)(833.6782959,651.07579102)
\lineto(833.4082959,651.07579102)
\lineto(832.3582959,651.07579102)
\lineto(828.3682959,651.07579102)
\lineto(827.3182959,651.07579102)
\curveto(827.17829422,651.075791)(827.05829434,651.080791)(826.9582959,651.09079102)
\curveto(826.85829454,651.11079097)(826.78329462,651.16079092)(826.7332959,651.24079102)
\curveto(826.69329471,651.30079078)(826.67329473,651.3757907)(826.6732959,651.46579102)
\lineto(826.6732959,651.75079102)
\lineto(826.6732959,652.80079102)
\lineto(826.6732959,656.82079102)
\lineto(826.6732959,660.18079102)
\lineto(826.6732959,661.11079102)
\lineto(826.6732959,661.38079102)
\curveto(826.67329473,661.47078061)(826.69329471,661.54078054)(826.7332959,661.59079102)
\curveto(826.77329463,661.66078042)(826.84829455,661.71078037)(826.9582959,661.74079102)
\curveto(826.97829442,661.75078033)(826.9982944,661.75078033)(827.0182959,661.74079102)
\curveto(827.03829436,661.74078034)(827.05829434,661.74578033)(827.0782959,661.75579102)
}
}
{
\newrgbcolor{curcolor}{0 0 0}
\pscustom[linestyle=none,fillstyle=solid,fillcolor=curcolor]
{
\newpath
\moveto(838.01821777,658.98079102)
\curveto(838.73821371,658.99078309)(839.3432131,658.90578317)(839.83321777,658.72579102)
\curveto(840.32321212,658.55578352)(840.70321174,658.25078383)(840.97321777,657.81079102)
\curveto(841.0432114,657.70078438)(841.09821135,657.58578449)(841.13821777,657.46579102)
\curveto(841.17821127,657.35578472)(841.21821123,657.23078485)(841.25821777,657.09079102)
\curveto(841.27821117,657.02078506)(841.28321116,656.94578513)(841.27321777,656.86579102)
\curveto(841.26321118,656.79578528)(841.2482112,656.74078534)(841.22821777,656.70079102)
\curveto(841.20821124,656.6807854)(841.18321126,656.66078542)(841.15321777,656.64079102)
\curveto(841.12321132,656.63078545)(841.09821135,656.61578546)(841.07821777,656.59579102)
\curveto(841.02821142,656.5757855)(840.97821147,656.57078551)(840.92821777,656.58079102)
\curveto(840.87821157,656.59078549)(840.82821162,656.59078549)(840.77821777,656.58079102)
\curveto(840.69821175,656.56078552)(840.59321185,656.55578552)(840.46321777,656.56579102)
\curveto(840.33321211,656.58578549)(840.2432122,656.61078547)(840.19321777,656.64079102)
\curveto(840.11321233,656.69078539)(840.05821239,656.75578532)(840.02821777,656.83579102)
\curveto(840.00821244,656.92578515)(839.97321247,657.01078507)(839.92321777,657.09079102)
\curveto(839.83321261,657.25078483)(839.70821274,657.39578468)(839.54821777,657.52579102)
\curveto(839.43821301,657.60578447)(839.31821313,657.66578441)(839.18821777,657.70579102)
\curveto(839.05821339,657.74578433)(838.91821353,657.78578429)(838.76821777,657.82579102)
\curveto(838.71821373,657.84578423)(838.66821378,657.85078423)(838.61821777,657.84079102)
\curveto(838.56821388,657.84078424)(838.51821393,657.84578423)(838.46821777,657.85579102)
\curveto(838.40821404,657.8757842)(838.33321411,657.88578419)(838.24321777,657.88579102)
\curveto(838.15321429,657.88578419)(838.07821437,657.8757842)(838.01821777,657.85579102)
\lineto(837.92821777,657.85579102)
\lineto(837.77821777,657.82579102)
\curveto(837.72821472,657.82578425)(837.67821477,657.82078426)(837.62821777,657.81079102)
\curveto(837.36821508,657.75078433)(837.15321529,657.66578441)(836.98321777,657.55579102)
\curveto(836.81321563,657.44578463)(836.69821575,657.26078482)(836.63821777,657.00079102)
\curveto(836.61821583,656.93078515)(836.61321583,656.86078522)(836.62321777,656.79079102)
\curveto(836.6432158,656.72078536)(836.66321578,656.66078542)(836.68321777,656.61079102)
\curveto(836.7432157,656.46078562)(836.81321563,656.35078573)(836.89321777,656.28079102)
\curveto(836.98321546,656.22078586)(837.09321535,656.15078593)(837.22321777,656.07079102)
\curveto(837.38321506,655.97078611)(837.56321488,655.89578618)(837.76321777,655.84579102)
\curveto(837.96321448,655.80578627)(838.16321428,655.75578632)(838.36321777,655.69579102)
\curveto(838.49321395,655.65578642)(838.62321382,655.62578645)(838.75321777,655.60579102)
\curveto(838.88321356,655.58578649)(839.01321343,655.55578652)(839.14321777,655.51579102)
\curveto(839.35321309,655.45578662)(839.55821289,655.39578668)(839.75821777,655.33579102)
\curveto(839.95821249,655.28578679)(840.15821229,655.22078686)(840.35821777,655.14079102)
\lineto(840.50821777,655.08079102)
\curveto(840.55821189,655.06078702)(840.60821184,655.03578704)(840.65821777,655.00579102)
\curveto(840.85821159,654.88578719)(841.03321141,654.75078733)(841.18321777,654.60079102)
\curveto(841.33321111,654.45078763)(841.45821099,654.26078782)(841.55821777,654.03079102)
\curveto(841.57821087,653.96078812)(841.59821085,653.86578821)(841.61821777,653.74579102)
\curveto(841.63821081,653.6757884)(841.6482108,653.60078848)(841.64821777,653.52079102)
\curveto(841.65821079,653.45078863)(841.66321078,653.37078871)(841.66321777,653.28079102)
\lineto(841.66321777,653.13079102)
\curveto(841.6432108,653.06078902)(841.63321081,652.99078909)(841.63321777,652.92079102)
\curveto(841.63321081,652.85078923)(841.62321082,652.7807893)(841.60321777,652.71079102)
\curveto(841.57321087,652.60078948)(841.53821091,652.49578958)(841.49821777,652.39579102)
\curveto(841.45821099,652.29578978)(841.41321103,652.20578987)(841.36321777,652.12579102)
\curveto(841.20321124,651.86579021)(840.99821145,651.65579042)(840.74821777,651.49579102)
\curveto(840.49821195,651.34579073)(840.21821223,651.21579086)(839.90821777,651.10579102)
\curveto(839.81821263,651.075791)(839.72321272,651.05579102)(839.62321777,651.04579102)
\curveto(839.53321291,651.02579105)(839.443213,651.00079108)(839.35321777,650.97079102)
\curveto(839.25321319,650.95079113)(839.15321329,650.94079114)(839.05321777,650.94079102)
\curveto(838.95321349,650.94079114)(838.85321359,650.93079115)(838.75321777,650.91079102)
\lineto(838.60321777,650.91079102)
\curveto(838.55321389,650.90079118)(838.48321396,650.89579118)(838.39321777,650.89579102)
\curveto(838.30321414,650.89579118)(838.23321421,650.90079118)(838.18321777,650.91079102)
\lineto(838.01821777,650.91079102)
\curveto(837.95821449,650.93079115)(837.89321455,650.94079114)(837.82321777,650.94079102)
\curveto(837.75321469,650.93079115)(837.69321475,650.93579114)(837.64321777,650.95579102)
\curveto(837.59321485,650.96579111)(837.52821492,650.97079111)(837.44821777,650.97079102)
\lineto(837.20821777,651.03079102)
\curveto(837.13821531,651.04079104)(837.06321538,651.06079102)(836.98321777,651.09079102)
\curveto(836.67321577,651.19079089)(836.40321604,651.31579076)(836.17321777,651.46579102)
\curveto(835.9432165,651.61579046)(835.7432167,651.81079027)(835.57321777,652.05079102)
\curveto(835.48321696,652.1807899)(835.40821704,652.31578976)(835.34821777,652.45579102)
\curveto(835.28821716,652.59578948)(835.23321721,652.75078933)(835.18321777,652.92079102)
\curveto(835.16321728,652.9807891)(835.15321729,653.05078903)(835.15321777,653.13079102)
\curveto(835.16321728,653.22078886)(835.17821727,653.29078879)(835.19821777,653.34079102)
\curveto(835.22821722,653.3807887)(835.27821717,653.42078866)(835.34821777,653.46079102)
\curveto(835.39821705,653.4807886)(835.46821698,653.49078859)(835.55821777,653.49079102)
\curveto(835.6482168,653.50078858)(835.73821671,653.50078858)(835.82821777,653.49079102)
\curveto(835.91821653,653.4807886)(836.00321644,653.46578861)(836.08321777,653.44579102)
\curveto(836.17321627,653.43578864)(836.23321621,653.42078866)(836.26321777,653.40079102)
\curveto(836.33321611,653.35078873)(836.37821607,653.2757888)(836.39821777,653.17579102)
\curveto(836.42821602,653.08578899)(836.46321598,653.00078908)(836.50321777,652.92079102)
\curveto(836.60321584,652.70078938)(836.73821571,652.53078955)(836.90821777,652.41079102)
\curveto(837.02821542,652.32078976)(837.16321528,652.25078983)(837.31321777,652.20079102)
\curveto(837.46321498,652.15078993)(837.62321482,652.10078998)(837.79321777,652.05079102)
\lineto(838.10821777,652.00579102)
\lineto(838.19821777,652.00579102)
\curveto(838.26821418,651.98579009)(838.35821409,651.9757901)(838.46821777,651.97579102)
\curveto(838.58821386,651.9757901)(838.68821376,651.98579009)(838.76821777,652.00579102)
\curveto(838.83821361,652.00579007)(838.89321355,652.01079007)(838.93321777,652.02079102)
\curveto(838.99321345,652.03079005)(839.05321339,652.03579004)(839.11321777,652.03579102)
\curveto(839.17321327,652.04579003)(839.22821322,652.05579002)(839.27821777,652.06579102)
\curveto(839.56821288,652.14578993)(839.79821265,652.25078983)(839.96821777,652.38079102)
\curveto(840.13821231,652.51078957)(840.25821219,652.73078935)(840.32821777,653.04079102)
\curveto(840.3482121,653.09078899)(840.35321209,653.14578893)(840.34321777,653.20579102)
\curveto(840.33321211,653.26578881)(840.32321212,653.31078877)(840.31321777,653.34079102)
\curveto(840.26321218,653.53078855)(840.19321225,653.67078841)(840.10321777,653.76079102)
\curveto(840.01321243,653.86078822)(839.89821255,653.95078813)(839.75821777,654.03079102)
\curveto(839.66821278,654.09078799)(839.56821288,654.14078794)(839.45821777,654.18079102)
\lineto(839.12821777,654.30079102)
\curveto(839.09821335,654.31078777)(839.06821338,654.31578776)(839.03821777,654.31579102)
\curveto(839.01821343,654.31578776)(838.99321345,654.32578775)(838.96321777,654.34579102)
\curveto(838.62321382,654.45578762)(838.26821418,654.53578754)(837.89821777,654.58579102)
\curveto(837.53821491,654.64578743)(837.19821525,654.74078734)(836.87821777,654.87079102)
\curveto(836.77821567,654.91078717)(836.68321576,654.94578713)(836.59321777,654.97579102)
\curveto(836.50321594,655.00578707)(836.41821603,655.04578703)(836.33821777,655.09579102)
\curveto(836.1482163,655.20578687)(835.97321647,655.33078675)(835.81321777,655.47079102)
\curveto(835.65321679,655.61078647)(835.52821692,655.78578629)(835.43821777,655.99579102)
\curveto(835.40821704,656.06578601)(835.38321706,656.13578594)(835.36321777,656.20579102)
\curveto(835.35321709,656.2757858)(835.33821711,656.35078573)(835.31821777,656.43079102)
\curveto(835.28821716,656.55078553)(835.27821717,656.68578539)(835.28821777,656.83579102)
\curveto(835.29821715,656.99578508)(835.31321713,657.13078495)(835.33321777,657.24079102)
\curveto(835.35321709,657.29078479)(835.36321708,657.33078475)(835.36321777,657.36079102)
\curveto(835.37321707,657.40078468)(835.38821706,657.44078464)(835.40821777,657.48079102)
\curveto(835.49821695,657.71078437)(835.61821683,657.91078417)(835.76821777,658.08079102)
\curveto(835.92821652,658.25078383)(836.10821634,658.40078368)(836.30821777,658.53079102)
\curveto(836.45821599,658.62078346)(836.62321582,658.69078339)(836.80321777,658.74079102)
\curveto(836.98321546,658.80078328)(837.17321527,658.85578322)(837.37321777,658.90579102)
\curveto(837.443215,658.91578316)(837.50821494,658.92578315)(837.56821777,658.93579102)
\curveto(837.63821481,658.94578313)(837.71321473,658.95578312)(837.79321777,658.96579102)
\curveto(837.82321462,658.9757831)(837.86321458,658.9757831)(837.91321777,658.96579102)
\curveto(837.96321448,658.95578312)(837.99821445,658.96078312)(838.01821777,658.98079102)
}
}
{
\newrgbcolor{curcolor}{0 0 0}
\pscustom[linestyle=none,fillstyle=solid,fillcolor=curcolor]
{
\newpath
\moveto(844.03321777,661.14079102)
\curveto(844.18321576,661.14078094)(844.33321561,661.13578094)(844.48321777,661.12579102)
\curveto(844.63321531,661.12578095)(844.73821521,661.08578099)(844.79821777,661.00579102)
\curveto(844.8482151,660.94578113)(844.87321507,660.86078122)(844.87321777,660.75079102)
\curveto(844.88321506,660.65078143)(844.88821506,660.54578153)(844.88821777,660.43579102)
\lineto(844.88821777,659.56579102)
\curveto(844.88821506,659.48578259)(844.88321506,659.40078268)(844.87321777,659.31079102)
\curveto(844.87321507,659.23078285)(844.88321506,659.16078292)(844.90321777,659.10079102)
\curveto(844.943215,658.96078312)(845.03321491,658.87078321)(845.17321777,658.83079102)
\curveto(845.22321472,658.82078326)(845.26821468,658.81578326)(845.30821777,658.81579102)
\lineto(845.45821777,658.81579102)
\lineto(845.86321777,658.81579102)
\curveto(846.02321392,658.82578325)(846.13821381,658.81578326)(846.20821777,658.78579102)
\curveto(846.29821365,658.72578335)(846.35821359,658.66578341)(846.38821777,658.60579102)
\curveto(846.40821354,658.56578351)(846.41821353,658.52078356)(846.41821777,658.47079102)
\lineto(846.41821777,658.32079102)
\curveto(846.41821353,658.21078387)(846.41321353,658.10578397)(846.40321777,658.00579102)
\curveto(846.39321355,657.91578416)(846.35821359,657.84578423)(846.29821777,657.79579102)
\curveto(846.23821371,657.74578433)(846.15321379,657.71578436)(846.04321777,657.70579102)
\lineto(845.71321777,657.70579102)
\curveto(845.60321434,657.71578436)(845.49321445,657.72078436)(845.38321777,657.72079102)
\curveto(845.27321467,657.72078436)(845.17821477,657.70578437)(845.09821777,657.67579102)
\curveto(845.02821492,657.64578443)(844.97821497,657.59578448)(844.94821777,657.52579102)
\curveto(844.91821503,657.45578462)(844.89821505,657.37078471)(844.88821777,657.27079102)
\curveto(844.87821507,657.1807849)(844.87321507,657.080785)(844.87321777,656.97079102)
\curveto(844.88321506,656.87078521)(844.88821506,656.77078531)(844.88821777,656.67079102)
\lineto(844.88821777,653.70079102)
\curveto(844.88821506,653.4807886)(844.88321506,653.24578883)(844.87321777,652.99579102)
\curveto(844.87321507,652.75578932)(844.91821503,652.57078951)(845.00821777,652.44079102)
\curveto(845.05821489,652.36078972)(845.12321482,652.30578977)(845.20321777,652.27579102)
\curveto(845.28321466,652.24578983)(845.37821457,652.22078986)(845.48821777,652.20079102)
\curveto(845.51821443,652.19078989)(845.5482144,652.18578989)(845.57821777,652.18579102)
\curveto(845.61821433,652.19578988)(845.65321429,652.19578988)(845.68321777,652.18579102)
\lineto(845.87821777,652.18579102)
\curveto(845.97821397,652.18578989)(846.06821388,652.1757899)(846.14821777,652.15579102)
\curveto(846.23821371,652.14578993)(846.30321364,652.11078997)(846.34321777,652.05079102)
\curveto(846.36321358,652.02079006)(846.37821357,651.96579011)(846.38821777,651.88579102)
\curveto(846.40821354,651.81579026)(846.41821353,651.74079034)(846.41821777,651.66079102)
\curveto(846.42821352,651.5807905)(846.42821352,651.50079058)(846.41821777,651.42079102)
\curveto(846.40821354,651.35079073)(846.38821356,651.29579078)(846.35821777,651.25579102)
\curveto(846.31821363,651.18579089)(846.2432137,651.13579094)(846.13321777,651.10579102)
\curveto(846.05321389,651.08579099)(845.96321398,651.075791)(845.86321777,651.07579102)
\curveto(845.76321418,651.08579099)(845.67321427,651.09079099)(845.59321777,651.09079102)
\curveto(845.53321441,651.09079099)(845.47321447,651.08579099)(845.41321777,651.07579102)
\curveto(845.35321459,651.075791)(845.29821465,651.080791)(845.24821777,651.09079102)
\lineto(845.06821777,651.09079102)
\curveto(845.01821493,651.10079098)(844.96821498,651.10579097)(844.91821777,651.10579102)
\curveto(844.87821507,651.11579096)(844.83321511,651.12079096)(844.78321777,651.12079102)
\curveto(844.58321536,651.17079091)(844.40821554,651.22579085)(844.25821777,651.28579102)
\curveto(844.11821583,651.34579073)(843.99821595,651.45079063)(843.89821777,651.60079102)
\curveto(843.75821619,651.80079028)(843.67821627,652.05079003)(843.65821777,652.35079102)
\curveto(843.63821631,652.66078942)(843.62821632,652.99078909)(843.62821777,653.34079102)
\lineto(843.62821777,657.27079102)
\curveto(843.59821635,657.40078468)(843.56821638,657.49578458)(843.53821777,657.55579102)
\curveto(843.51821643,657.61578446)(843.4482165,657.66578441)(843.32821777,657.70579102)
\curveto(843.28821666,657.71578436)(843.2482167,657.71578436)(843.20821777,657.70579102)
\curveto(843.16821678,657.69578438)(843.12821682,657.70078438)(843.08821777,657.72079102)
\lineto(842.84821777,657.72079102)
\curveto(842.71821723,657.72078436)(842.60821734,657.73078435)(842.51821777,657.75079102)
\curveto(842.43821751,657.7807843)(842.38321756,657.84078424)(842.35321777,657.93079102)
\curveto(842.33321761,657.97078411)(842.31821763,658.01578406)(842.30821777,658.06579102)
\lineto(842.30821777,658.21579102)
\curveto(842.30821764,658.35578372)(842.31821763,658.47078361)(842.33821777,658.56079102)
\curveto(842.35821759,658.66078342)(842.41821753,658.73578334)(842.51821777,658.78579102)
\curveto(842.62821732,658.82578325)(842.76821718,658.83578324)(842.93821777,658.81579102)
\curveto(843.11821683,658.79578328)(843.26821668,658.80578327)(843.38821777,658.84579102)
\curveto(843.47821647,658.89578318)(843.5482164,658.96578311)(843.59821777,659.05579102)
\curveto(843.61821633,659.11578296)(843.62821632,659.19078289)(843.62821777,659.28079102)
\lineto(843.62821777,659.53579102)
\lineto(843.62821777,660.46579102)
\lineto(843.62821777,660.70579102)
\curveto(843.62821632,660.79578128)(843.63821631,660.87078121)(843.65821777,660.93079102)
\curveto(843.69821625,661.01078107)(843.77321617,661.075781)(843.88321777,661.12579102)
\curveto(843.91321603,661.12578095)(843.93821601,661.12578095)(843.95821777,661.12579102)
\curveto(843.98821596,661.13578094)(844.01321593,661.14078094)(844.03321777,661.14079102)
}
}
{
\newrgbcolor{curcolor}{0 0 0}
\pscustom[linestyle=none,fillstyle=solid,fillcolor=curcolor]
{
\newpath
\moveto(848.27001465,658.80079102)
\lineto(848.70501465,658.80079102)
\curveto(848.85501268,658.80078328)(848.96001258,658.76078332)(849.02001465,658.68079102)
\curveto(849.07001247,658.60078348)(849.09501244,658.50078358)(849.09501465,658.38079102)
\curveto(849.10501243,658.26078382)(849.11001243,658.14078394)(849.11001465,658.02079102)
\lineto(849.11001465,656.59579102)
\lineto(849.11001465,654.33079102)
\lineto(849.11001465,653.64079102)
\curveto(849.11001243,653.41078867)(849.1350124,653.21078887)(849.18501465,653.04079102)
\curveto(849.34501219,652.59078949)(849.64501189,652.2757898)(850.08501465,652.09579102)
\curveto(850.30501123,652.00579007)(850.57001097,651.97079011)(850.88001465,651.99079102)
\curveto(851.19001035,652.02079006)(851.4400101,652.07579)(851.63001465,652.15579102)
\curveto(851.96000958,652.29578978)(852.22000932,652.47078961)(852.41001465,652.68079102)
\curveto(852.61000893,652.90078918)(852.76500877,653.18578889)(852.87501465,653.53579102)
\curveto(852.90500863,653.61578846)(852.92500861,653.69578838)(852.93501465,653.77579102)
\curveto(852.94500859,653.85578822)(852.96000858,653.94078814)(852.98001465,654.03079102)
\curveto(852.99000855,654.080788)(852.99000855,654.12578795)(852.98001465,654.16579102)
\curveto(852.98000856,654.20578787)(852.99000855,654.25078783)(853.01001465,654.30079102)
\lineto(853.01001465,654.61579102)
\curveto(853.03000851,654.69578738)(853.0350085,654.78578729)(853.02501465,654.88579102)
\curveto(853.01500852,654.99578708)(853.01000853,655.09578698)(853.01001465,655.18579102)
\lineto(853.01001465,656.35579102)
\lineto(853.01001465,657.94579102)
\curveto(853.01000853,658.06578401)(853.00500853,658.19078389)(852.99501465,658.32079102)
\curveto(852.99500854,658.46078362)(853.02000852,658.57078351)(853.07001465,658.65079102)
\curveto(853.11000843,658.70078338)(853.15500838,658.73078335)(853.20501465,658.74079102)
\curveto(853.26500827,658.76078332)(853.3350082,658.7807833)(853.41501465,658.80079102)
\lineto(853.64001465,658.80079102)
\curveto(853.76000778,658.80078328)(853.86500767,658.79578328)(853.95501465,658.78579102)
\curveto(854.05500748,658.7757833)(854.13000741,658.73078335)(854.18001465,658.65079102)
\curveto(854.23000731,658.60078348)(854.25500728,658.52578355)(854.25501465,658.42579102)
\lineto(854.25501465,658.14079102)
\lineto(854.25501465,657.12079102)
\lineto(854.25501465,653.08579102)
\lineto(854.25501465,651.73579102)
\curveto(854.25500728,651.61579046)(854.25000729,651.50079058)(854.24001465,651.39079102)
\curveto(854.2400073,651.29079079)(854.20500733,651.21579086)(854.13501465,651.16579102)
\curveto(854.09500744,651.13579094)(854.0350075,651.11079097)(853.95501465,651.09079102)
\curveto(853.87500766,651.080791)(853.78500775,651.07079101)(853.68501465,651.06079102)
\curveto(853.59500794,651.06079102)(853.50500803,651.06579101)(853.41501465,651.07579102)
\curveto(853.3350082,651.08579099)(853.27500826,651.10579097)(853.23501465,651.13579102)
\curveto(853.18500835,651.1757909)(853.1400084,651.24079084)(853.10001465,651.33079102)
\curveto(853.09000845,651.37079071)(853.08000846,651.42579065)(853.07001465,651.49579102)
\curveto(853.07000847,651.56579051)(853.06500847,651.63079045)(853.05501465,651.69079102)
\curveto(853.04500849,651.76079032)(853.02500851,651.81579026)(852.99501465,651.85579102)
\curveto(852.96500857,651.89579018)(852.92000862,651.91079017)(852.86001465,651.90079102)
\curveto(852.78000876,651.8807902)(852.70000884,651.82079026)(852.62001465,651.72079102)
\curveto(852.540009,651.63079045)(852.46500907,651.56079052)(852.39501465,651.51079102)
\curveto(852.17500936,651.35079073)(851.92500961,651.21079087)(851.64501465,651.09079102)
\curveto(851.53501,651.04079104)(851.42001012,651.01079107)(851.30001465,651.00079102)
\curveto(851.19001035,650.9807911)(851.07501046,650.95579112)(850.95501465,650.92579102)
\curveto(850.90501063,650.91579116)(850.85001069,650.91579116)(850.79001465,650.92579102)
\curveto(850.7400108,650.93579114)(850.69001085,650.93079115)(850.64001465,650.91079102)
\curveto(850.540011,650.89079119)(850.45001109,650.89079119)(850.37001465,650.91079102)
\lineto(850.22001465,650.91079102)
\curveto(850.17001137,650.93079115)(850.11001143,650.94079114)(850.04001465,650.94079102)
\curveto(849.98001156,650.94079114)(849.92501161,650.94579113)(849.87501465,650.95579102)
\curveto(849.8350117,650.9757911)(849.79501174,650.98579109)(849.75501465,650.98579102)
\curveto(849.72501181,650.9757911)(849.68501185,650.9807911)(849.63501465,651.00079102)
\lineto(849.39501465,651.06079102)
\curveto(849.32501221,651.080791)(849.25001229,651.11079097)(849.17001465,651.15079102)
\curveto(848.91001263,651.26079082)(848.69001285,651.40579067)(848.51001465,651.58579102)
\curveto(848.3400132,651.7757903)(848.20001334,652.00079008)(848.09001465,652.26079102)
\curveto(848.05001349,652.35078973)(848.02001352,652.44078964)(848.00001465,652.53079102)
\lineto(847.94001465,652.83079102)
\curveto(847.92001362,652.89078919)(847.91001363,652.94578913)(847.91001465,652.99579102)
\curveto(847.92001362,653.05578902)(847.91501362,653.12078896)(847.89501465,653.19079102)
\curveto(847.88501365,653.21078887)(847.88001366,653.23578884)(847.88001465,653.26579102)
\curveto(847.88001366,653.30578877)(847.87501366,653.34078874)(847.86501465,653.37079102)
\lineto(847.86501465,653.52079102)
\curveto(847.85501368,653.56078852)(847.85001369,653.60578847)(847.85001465,653.65579102)
\curveto(847.86001368,653.71578836)(847.86501367,653.77078831)(847.86501465,653.82079102)
\lineto(847.86501465,654.42079102)
\lineto(847.86501465,657.18079102)
\lineto(847.86501465,658.14079102)
\lineto(847.86501465,658.41079102)
\curveto(847.86501367,658.50078358)(847.88501365,658.5757835)(847.92501465,658.63579102)
\curveto(847.96501357,658.70578337)(848.0400135,658.75578332)(848.15001465,658.78579102)
\curveto(848.17001337,658.79578328)(848.19001335,658.79578328)(848.21001465,658.78579102)
\curveto(848.23001331,658.78578329)(848.25001329,658.79078329)(848.27001465,658.80079102)
}
}
{
\newrgbcolor{curcolor}{0 0 0}
\pscustom[linestyle=none,fillstyle=solid,fillcolor=curcolor]
{
\newpath
\moveto(863.11462402,651.88579102)
\lineto(863.11462402,651.49579102)
\curveto(863.11461615,651.3757907)(863.08961617,651.2757908)(863.03962402,651.19579102)
\curveto(862.98961627,651.12579095)(862.90461636,651.08579099)(862.78462402,651.07579102)
\lineto(862.43962402,651.07579102)
\curveto(862.37961688,651.075791)(862.31961694,651.07079101)(862.25962402,651.06079102)
\curveto(862.20961705,651.06079102)(862.1646171,651.07079101)(862.12462402,651.09079102)
\curveto(862.03461723,651.11079097)(861.97461729,651.15079093)(861.94462402,651.21079102)
\curveto(861.90461736,651.26079082)(861.87961738,651.32079076)(861.86962402,651.39079102)
\curveto(861.86961739,651.46079062)(861.85461741,651.53079055)(861.82462402,651.60079102)
\curveto(861.81461745,651.62079046)(861.79961746,651.63579044)(861.77962402,651.64579102)
\curveto(861.76961749,651.66579041)(861.75461751,651.68579039)(861.73462402,651.70579102)
\curveto(861.63461763,651.71579036)(861.55461771,651.69579038)(861.49462402,651.64579102)
\curveto(861.44461782,651.59579048)(861.38961787,651.54579053)(861.32962402,651.49579102)
\curveto(861.12961813,651.34579073)(860.92961833,651.23079085)(860.72962402,651.15079102)
\curveto(860.54961871,651.07079101)(860.33961892,651.01079107)(860.09962402,650.97079102)
\curveto(859.86961939,650.93079115)(859.62961963,650.91079117)(859.37962402,650.91079102)
\curveto(859.13962012,650.90079118)(858.89962036,650.91579116)(858.65962402,650.95579102)
\curveto(858.41962084,650.98579109)(858.20962105,651.04079104)(858.02962402,651.12079102)
\curveto(857.50962175,651.34079074)(857.08962217,651.63579044)(856.76962402,652.00579102)
\curveto(856.44962281,652.38578969)(856.19962306,652.85578922)(856.01962402,653.41579102)
\curveto(855.97962328,653.50578857)(855.94962331,653.59578848)(855.92962402,653.68579102)
\curveto(855.91962334,653.78578829)(855.89962336,653.88578819)(855.86962402,653.98579102)
\curveto(855.8596234,654.03578804)(855.85462341,654.08578799)(855.85462402,654.13579102)
\curveto(855.85462341,654.18578789)(855.84962341,654.23578784)(855.83962402,654.28579102)
\curveto(855.81962344,654.33578774)(855.80962345,654.38578769)(855.80962402,654.43579102)
\curveto(855.81962344,654.49578758)(855.81962344,654.55078753)(855.80962402,654.60079102)
\lineto(855.80962402,654.75079102)
\curveto(855.78962347,654.80078728)(855.77962348,654.86578721)(855.77962402,654.94579102)
\curveto(855.77962348,655.02578705)(855.78962347,655.09078699)(855.80962402,655.14079102)
\lineto(855.80962402,655.30579102)
\curveto(855.82962343,655.3757867)(855.83462343,655.44578663)(855.82462402,655.51579102)
\curveto(855.82462344,655.59578648)(855.83462343,655.67078641)(855.85462402,655.74079102)
\curveto(855.8646234,655.79078629)(855.86962339,655.83578624)(855.86962402,655.87579102)
\curveto(855.86962339,655.91578616)(855.87462339,655.96078612)(855.88462402,656.01079102)
\curveto(855.91462335,656.11078597)(855.93962332,656.20578587)(855.95962402,656.29579102)
\curveto(855.97962328,656.39578568)(856.00462326,656.49078559)(856.03462402,656.58079102)
\curveto(856.1646231,656.96078512)(856.32962293,657.30078478)(856.52962402,657.60079102)
\curveto(856.73962252,657.91078417)(856.98962227,658.16578391)(857.27962402,658.36579102)
\curveto(857.44962181,658.48578359)(857.62462164,658.58578349)(857.80462402,658.66579102)
\curveto(857.99462127,658.74578333)(858.19962106,658.81578326)(858.41962402,658.87579102)
\curveto(858.48962077,658.88578319)(858.55462071,658.89578318)(858.61462402,658.90579102)
\curveto(858.68462058,658.91578316)(858.75462051,658.93078315)(858.82462402,658.95079102)
\lineto(858.97462402,658.95079102)
\curveto(859.05462021,658.97078311)(859.16962009,658.9807831)(859.31962402,658.98079102)
\curveto(859.47961978,658.9807831)(859.59961966,658.97078311)(859.67962402,658.95079102)
\curveto(859.71961954,658.94078314)(859.77461949,658.93578314)(859.84462402,658.93579102)
\curveto(859.95461931,658.90578317)(860.0646192,658.8807832)(860.17462402,658.86079102)
\curveto(860.28461898,658.85078323)(860.38961887,658.82078326)(860.48962402,658.77079102)
\curveto(860.63961862,658.71078337)(860.77961848,658.64578343)(860.90962402,658.57579102)
\curveto(861.04961821,658.50578357)(861.17961808,658.42578365)(861.29962402,658.33579102)
\curveto(861.3596179,658.28578379)(861.41961784,658.23078385)(861.47962402,658.17079102)
\curveto(861.54961771,658.12078396)(861.63961762,658.10578397)(861.74962402,658.12579102)
\curveto(861.76961749,658.15578392)(861.78461748,658.1807839)(861.79462402,658.20079102)
\curveto(861.81461745,658.22078386)(861.82961743,658.25078383)(861.83962402,658.29079102)
\curveto(861.86961739,658.3807837)(861.87961738,658.49578358)(861.86962402,658.63579102)
\lineto(861.86962402,659.01079102)
\lineto(861.86962402,660.73579102)
\lineto(861.86962402,661.20079102)
\curveto(861.86961739,661.3807807)(861.89461737,661.51078057)(861.94462402,661.59079102)
\curveto(861.98461728,661.66078042)(862.04461722,661.70578037)(862.12462402,661.72579102)
\curveto(862.14461712,661.72578035)(862.16961709,661.72578035)(862.19962402,661.72579102)
\curveto(862.22961703,661.73578034)(862.25461701,661.74078034)(862.27462402,661.74079102)
\curveto(862.41461685,661.75078033)(862.5596167,661.75078033)(862.70962402,661.74079102)
\curveto(862.86961639,661.74078034)(862.97961628,661.70078038)(863.03962402,661.62079102)
\curveto(863.08961617,661.54078054)(863.11461615,661.44078064)(863.11462402,661.32079102)
\lineto(863.11462402,660.94579102)
\lineto(863.11462402,651.88579102)
\moveto(861.89962402,654.72079102)
\curveto(861.91961734,654.77078731)(861.92961733,654.83578724)(861.92962402,654.91579102)
\curveto(861.92961733,655.00578707)(861.91961734,655.075787)(861.89962402,655.12579102)
\lineto(861.89962402,655.35079102)
\curveto(861.87961738,655.44078664)(861.8646174,655.53078655)(861.85462402,655.62079102)
\curveto(861.84461742,655.72078636)(861.82461744,655.81078627)(861.79462402,655.89079102)
\curveto(861.77461749,655.97078611)(861.75461751,656.04578603)(861.73462402,656.11579102)
\curveto(861.72461754,656.18578589)(861.70461756,656.25578582)(861.67462402,656.32579102)
\curveto(861.55461771,656.62578545)(861.39961786,656.89078519)(861.20962402,657.12079102)
\curveto(861.01961824,657.35078473)(860.77961848,657.53078455)(860.48962402,657.66079102)
\curveto(860.38961887,657.71078437)(860.28461898,657.74578433)(860.17462402,657.76579102)
\curveto(860.07461919,657.79578428)(859.9646193,657.82078426)(859.84462402,657.84079102)
\curveto(859.7646195,657.86078422)(859.67461959,657.87078421)(859.57462402,657.87079102)
\lineto(859.30462402,657.87079102)
\curveto(859.25462001,657.86078422)(859.20962005,657.85078423)(859.16962402,657.84079102)
\lineto(859.03462402,657.84079102)
\curveto(858.95462031,657.82078426)(858.86962039,657.80078428)(858.77962402,657.78079102)
\curveto(858.69962056,657.76078432)(858.61962064,657.73578434)(858.53962402,657.70579102)
\curveto(858.21962104,657.56578451)(857.9596213,657.36078472)(857.75962402,657.09079102)
\curveto(857.56962169,656.83078525)(857.41462185,656.52578555)(857.29462402,656.17579102)
\curveto(857.25462201,656.06578601)(857.22462204,655.95078613)(857.20462402,655.83079102)
\curveto(857.19462207,655.72078636)(857.17962208,655.61078647)(857.15962402,655.50079102)
\curveto(857.1596221,655.46078662)(857.15462211,655.42078666)(857.14462402,655.38079102)
\lineto(857.14462402,655.27579102)
\curveto(857.12462214,655.22578685)(857.11462215,655.17078691)(857.11462402,655.11079102)
\curveto(857.12462214,655.05078703)(857.12962213,654.99578708)(857.12962402,654.94579102)
\lineto(857.12962402,654.61579102)
\curveto(857.12962213,654.51578756)(857.13962212,654.42078766)(857.15962402,654.33079102)
\curveto(857.16962209,654.30078778)(857.17462209,654.25078783)(857.17462402,654.18079102)
\curveto(857.19462207,654.11078797)(857.20962205,654.04078804)(857.21962402,653.97079102)
\lineto(857.27962402,653.76079102)
\curveto(857.38962187,653.41078867)(857.53962172,653.11078897)(857.72962402,652.86079102)
\curveto(857.91962134,652.61078947)(858.1596211,652.40578967)(858.44962402,652.24579102)
\curveto(858.53962072,652.19578988)(858.62962063,652.15578992)(858.71962402,652.12579102)
\curveto(858.80962045,652.09578998)(858.90962035,652.06579001)(859.01962402,652.03579102)
\curveto(859.06962019,652.01579006)(859.11962014,652.01079007)(859.16962402,652.02079102)
\curveto(859.22962003,652.03079005)(859.28461998,652.02579005)(859.33462402,652.00579102)
\curveto(859.37461989,651.99579008)(859.41461985,651.99079009)(859.45462402,651.99079102)
\lineto(859.58962402,651.99079102)
\lineto(859.72462402,651.99079102)
\curveto(859.75461951,652.00079008)(859.80461946,652.00579007)(859.87462402,652.00579102)
\curveto(859.95461931,652.02579005)(860.03461923,652.04079004)(860.11462402,652.05079102)
\curveto(860.19461907,652.07079001)(860.26961899,652.09578998)(860.33962402,652.12579102)
\curveto(860.66961859,652.26578981)(860.93461833,652.44078964)(861.13462402,652.65079102)
\curveto(861.34461792,652.87078921)(861.51961774,653.14578893)(861.65962402,653.47579102)
\curveto(861.70961755,653.58578849)(861.74461752,653.69578838)(861.76462402,653.80579102)
\curveto(861.78461748,653.91578816)(861.80961745,654.02578805)(861.83962402,654.13579102)
\curveto(861.8596174,654.1757879)(861.86961739,654.21078787)(861.86962402,654.24079102)
\curveto(861.86961739,654.2807878)(861.87461739,654.32078776)(861.88462402,654.36079102)
\curveto(861.89461737,654.42078766)(861.89461737,654.4807876)(861.88462402,654.54079102)
\curveto(861.88461738,654.60078748)(861.88961737,654.66078742)(861.89962402,654.72079102)
}
}
{
\newrgbcolor{curcolor}{0 0 0}
\pscustom[linestyle=none,fillstyle=solid,fillcolor=curcolor]
{
\newpath
\moveto(865.34587402,660.30079102)
\curveto(865.2658729,660.36078172)(865.22087295,660.46578161)(865.21087402,660.61579102)
\lineto(865.21087402,661.08079102)
\lineto(865.21087402,661.33579102)
\curveto(865.21087296,661.42578065)(865.22587294,661.50078058)(865.25587402,661.56079102)
\curveto(865.29587287,661.64078044)(865.37587279,661.70078038)(865.49587402,661.74079102)
\curveto(865.51587265,661.75078033)(865.53587263,661.75078033)(865.55587402,661.74079102)
\curveto(865.58587258,661.74078034)(865.61087256,661.74578033)(865.63087402,661.75579102)
\curveto(865.80087237,661.75578032)(865.96087221,661.75078033)(866.11087402,661.74079102)
\curveto(866.26087191,661.73078035)(866.36087181,661.67078041)(866.41087402,661.56079102)
\curveto(866.44087173,661.50078058)(866.45587171,661.42578065)(866.45587402,661.33579102)
\lineto(866.45587402,661.08079102)
\curveto(866.45587171,660.90078118)(866.45087172,660.73078135)(866.44087402,660.57079102)
\curveto(866.44087173,660.41078167)(866.37587179,660.30578177)(866.24587402,660.25579102)
\curveto(866.19587197,660.23578184)(866.14087203,660.22578185)(866.08087402,660.22579102)
\lineto(865.91587402,660.22579102)
\lineto(865.60087402,660.22579102)
\curveto(865.50087267,660.22578185)(865.41587275,660.25078183)(865.34587402,660.30079102)
\moveto(866.45587402,651.79579102)
\lineto(866.45587402,651.48079102)
\curveto(866.4658717,651.3807907)(866.44587172,651.30079078)(866.39587402,651.24079102)
\curveto(866.3658718,651.1807909)(866.32087185,651.14079094)(866.26087402,651.12079102)
\curveto(866.20087197,651.11079097)(866.13087204,651.09579098)(866.05087402,651.07579102)
\lineto(865.82587402,651.07579102)
\curveto(865.69587247,651.075791)(865.58087259,651.080791)(865.48087402,651.09079102)
\curveto(865.39087278,651.11079097)(865.32087285,651.16079092)(865.27087402,651.24079102)
\curveto(865.23087294,651.30079078)(865.21087296,651.3757907)(865.21087402,651.46579102)
\lineto(865.21087402,651.75079102)
\lineto(865.21087402,658.09579102)
\lineto(865.21087402,658.41079102)
\curveto(865.21087296,658.52078356)(865.23587293,658.60578347)(865.28587402,658.66579102)
\curveto(865.31587285,658.71578336)(865.35587281,658.74578333)(865.40587402,658.75579102)
\curveto(865.45587271,658.76578331)(865.51087266,658.7807833)(865.57087402,658.80079102)
\curveto(865.59087258,658.80078328)(865.61087256,658.79578328)(865.63087402,658.78579102)
\curveto(865.66087251,658.78578329)(865.68587248,658.79078329)(865.70587402,658.80079102)
\curveto(865.83587233,658.80078328)(865.9658722,658.79578328)(866.09587402,658.78579102)
\curveto(866.23587193,658.78578329)(866.33087184,658.74578333)(866.38087402,658.66579102)
\curveto(866.43087174,658.60578347)(866.45587171,658.52578355)(866.45587402,658.42579102)
\lineto(866.45587402,658.14079102)
\lineto(866.45587402,651.79579102)
}
}
{
\newrgbcolor{curcolor}{0 0 0}
\pscustom[linestyle=none,fillstyle=solid,fillcolor=curcolor]
{
\newpath
\moveto(875.28571777,651.63079102)
\curveto(875.31570994,651.47079061)(875.30070996,651.33579074)(875.24071777,651.22579102)
\curveto(875.18071008,651.12579095)(875.10071016,651.05079103)(875.00071777,651.00079102)
\curveto(874.95071031,650.9807911)(874.89571036,650.97079111)(874.83571777,650.97079102)
\curveto(874.78571047,650.97079111)(874.73071053,650.96079112)(874.67071777,650.94079102)
\curveto(874.45071081,650.89079119)(874.23071103,650.90579117)(874.01071777,650.98579102)
\curveto(873.80071146,651.05579102)(873.6557116,651.14579093)(873.57571777,651.25579102)
\curveto(873.52571173,651.32579075)(873.48071178,651.40579067)(873.44071777,651.49579102)
\curveto(873.40071186,651.59579048)(873.35071191,651.6757904)(873.29071777,651.73579102)
\curveto(873.27071199,651.75579032)(873.24571201,651.7757903)(873.21571777,651.79579102)
\curveto(873.19571206,651.81579026)(873.16571209,651.82079026)(873.12571777,651.81079102)
\curveto(873.01571224,651.7807903)(872.91071235,651.72579035)(872.81071777,651.64579102)
\curveto(872.72071254,651.56579051)(872.63071263,651.49579058)(872.54071777,651.43579102)
\curveto(872.41071285,651.35579072)(872.27071299,651.2807908)(872.12071777,651.21079102)
\curveto(871.97071329,651.15079093)(871.81071345,651.09579098)(871.64071777,651.04579102)
\curveto(871.54071372,651.01579106)(871.43071383,650.99579108)(871.31071777,650.98579102)
\curveto(871.20071406,650.9757911)(871.09071417,650.96079112)(870.98071777,650.94079102)
\curveto(870.93071433,650.93079115)(870.88571437,650.92579115)(870.84571777,650.92579102)
\lineto(870.74071777,650.92579102)
\curveto(870.63071463,650.90579117)(870.52571473,650.90579117)(870.42571777,650.92579102)
\lineto(870.29071777,650.92579102)
\curveto(870.24071502,650.93579114)(870.19071507,650.94079114)(870.14071777,650.94079102)
\curveto(870.09071517,650.94079114)(870.04571521,650.95079113)(870.00571777,650.97079102)
\curveto(869.96571529,650.9807911)(869.93071533,650.98579109)(869.90071777,650.98579102)
\curveto(869.88071538,650.9757911)(869.8557154,650.9757911)(869.82571777,650.98579102)
\lineto(869.58571777,651.04579102)
\curveto(869.50571575,651.05579102)(869.43071583,651.075791)(869.36071777,651.10579102)
\curveto(869.0607162,651.23579084)(868.81571644,651.3807907)(868.62571777,651.54079102)
\curveto(868.44571681,651.71079037)(868.29571696,651.94579013)(868.17571777,652.24579102)
\curveto(868.08571717,652.46578961)(868.04071722,652.73078935)(868.04071777,653.04079102)
\lineto(868.04071777,653.35579102)
\curveto(868.05071721,653.40578867)(868.0557172,653.45578862)(868.05571777,653.50579102)
\lineto(868.08571777,653.68579102)
\lineto(868.20571777,654.01579102)
\curveto(868.24571701,654.12578795)(868.29571696,654.22578785)(868.35571777,654.31579102)
\curveto(868.53571672,654.60578747)(868.78071648,654.82078726)(869.09071777,654.96079102)
\curveto(869.40071586,655.10078698)(869.74071552,655.22578685)(870.11071777,655.33579102)
\curveto(870.25071501,655.3757867)(870.39571486,655.40578667)(870.54571777,655.42579102)
\curveto(870.69571456,655.44578663)(870.84571441,655.47078661)(870.99571777,655.50079102)
\curveto(871.06571419,655.52078656)(871.13071413,655.53078655)(871.19071777,655.53079102)
\curveto(871.260714,655.53078655)(871.33571392,655.54078654)(871.41571777,655.56079102)
\curveto(871.48571377,655.5807865)(871.5557137,655.59078649)(871.62571777,655.59079102)
\curveto(871.69571356,655.60078648)(871.77071349,655.61578646)(871.85071777,655.63579102)
\curveto(872.10071316,655.69578638)(872.33571292,655.74578633)(872.55571777,655.78579102)
\curveto(872.77571248,655.83578624)(872.95071231,655.95078613)(873.08071777,656.13079102)
\curveto(873.14071212,656.21078587)(873.19071207,656.31078577)(873.23071777,656.43079102)
\curveto(873.27071199,656.56078552)(873.27071199,656.70078538)(873.23071777,656.85079102)
\curveto(873.17071209,657.09078499)(873.08071218,657.2807848)(872.96071777,657.42079102)
\curveto(872.85071241,657.56078452)(872.69071257,657.67078441)(872.48071777,657.75079102)
\curveto(872.3607129,657.80078428)(872.21571304,657.83578424)(872.04571777,657.85579102)
\curveto(871.88571337,657.8757842)(871.71571354,657.88578419)(871.53571777,657.88579102)
\curveto(871.3557139,657.88578419)(871.18071408,657.8757842)(871.01071777,657.85579102)
\curveto(870.84071442,657.83578424)(870.69571456,657.80578427)(870.57571777,657.76579102)
\curveto(870.40571485,657.70578437)(870.24071502,657.62078446)(870.08071777,657.51079102)
\curveto(870.00071526,657.45078463)(869.92571533,657.37078471)(869.85571777,657.27079102)
\curveto(869.79571546,657.1807849)(869.74071552,657.080785)(869.69071777,656.97079102)
\curveto(869.6607156,656.89078519)(869.63071563,656.80578527)(869.60071777,656.71579102)
\curveto(869.58071568,656.62578545)(869.53571572,656.55578552)(869.46571777,656.50579102)
\curveto(869.42571583,656.4757856)(869.3557159,656.45078563)(869.25571777,656.43079102)
\curveto(869.16571609,656.42078566)(869.07071619,656.41578566)(868.97071777,656.41579102)
\curveto(868.87071639,656.41578566)(868.77071649,656.42078566)(868.67071777,656.43079102)
\curveto(868.58071668,656.45078563)(868.51571674,656.4757856)(868.47571777,656.50579102)
\curveto(868.43571682,656.53578554)(868.40571685,656.58578549)(868.38571777,656.65579102)
\curveto(868.36571689,656.72578535)(868.36571689,656.80078528)(868.38571777,656.88079102)
\curveto(868.41571684,657.01078507)(868.44571681,657.13078495)(868.47571777,657.24079102)
\curveto(868.51571674,657.36078472)(868.5607167,657.4757846)(868.61071777,657.58579102)
\curveto(868.80071646,657.93578414)(869.04071622,658.20578387)(869.33071777,658.39579102)
\curveto(869.62071564,658.59578348)(869.98071528,658.75578332)(870.41071777,658.87579102)
\curveto(870.51071475,658.89578318)(870.61071465,658.91078317)(870.71071777,658.92079102)
\curveto(870.82071444,658.93078315)(870.93071433,658.94578313)(871.04071777,658.96579102)
\curveto(871.08071418,658.9757831)(871.14571411,658.9757831)(871.23571777,658.96579102)
\curveto(871.32571393,658.96578311)(871.38071388,658.9757831)(871.40071777,658.99579102)
\curveto(872.10071316,659.00578307)(872.71071255,658.92578315)(873.23071777,658.75579102)
\curveto(873.75071151,658.58578349)(874.11571114,658.26078382)(874.32571777,657.78079102)
\curveto(874.41571084,657.5807845)(874.46571079,657.34578473)(874.47571777,657.07579102)
\curveto(874.49571076,656.81578526)(874.50571075,656.54078554)(874.50571777,656.25079102)
\lineto(874.50571777,652.93579102)
\curveto(874.50571075,652.79578928)(874.51071075,652.66078942)(874.52071777,652.53079102)
\curveto(874.53071073,652.40078968)(874.5607107,652.29578978)(874.61071777,652.21579102)
\curveto(874.6607106,652.14578993)(874.72571053,652.09578998)(874.80571777,652.06579102)
\curveto(874.89571036,652.02579005)(874.98071028,651.99579008)(875.06071777,651.97579102)
\curveto(875.14071012,651.96579011)(875.20071006,651.92079016)(875.24071777,651.84079102)
\curveto(875.26071,651.81079027)(875.27070999,651.7807903)(875.27071777,651.75079102)
\curveto(875.27070999,651.72079036)(875.27570998,651.6807904)(875.28571777,651.63079102)
\moveto(873.14071777,653.29579102)
\curveto(873.20071206,653.43578864)(873.23071203,653.59578848)(873.23071777,653.77579102)
\curveto(873.24071202,653.96578811)(873.24571201,654.16078792)(873.24571777,654.36079102)
\curveto(873.24571201,654.47078761)(873.24071202,654.57078751)(873.23071777,654.66079102)
\curveto(873.22071204,654.75078733)(873.18071208,654.82078726)(873.11071777,654.87079102)
\curveto(873.08071218,654.89078719)(873.01071225,654.90078718)(872.90071777,654.90079102)
\curveto(872.88071238,654.8807872)(872.84571241,654.87078721)(872.79571777,654.87079102)
\curveto(872.74571251,654.87078721)(872.70071256,654.86078722)(872.66071777,654.84079102)
\curveto(872.58071268,654.82078726)(872.49071277,654.80078728)(872.39071777,654.78079102)
\lineto(872.09071777,654.72079102)
\curveto(872.0607132,654.72078736)(872.02571323,654.71578736)(871.98571777,654.70579102)
\lineto(871.88071777,654.70579102)
\curveto(871.73071353,654.66578741)(871.56571369,654.64078744)(871.38571777,654.63079102)
\curveto(871.21571404,654.63078745)(871.0557142,654.61078747)(870.90571777,654.57079102)
\curveto(870.82571443,654.55078753)(870.75071451,654.53078755)(870.68071777,654.51079102)
\curveto(870.62071464,654.50078758)(870.55071471,654.48578759)(870.47071777,654.46579102)
\curveto(870.31071495,654.41578766)(870.1607151,654.35078773)(870.02071777,654.27079102)
\curveto(869.88071538,654.20078788)(869.7607155,654.11078797)(869.66071777,654.00079102)
\curveto(869.5607157,653.89078819)(869.48571577,653.75578832)(869.43571777,653.59579102)
\curveto(869.38571587,653.44578863)(869.36571589,653.26078882)(869.37571777,653.04079102)
\curveto(869.37571588,652.94078914)(869.39071587,652.84578923)(869.42071777,652.75579102)
\curveto(869.4607158,652.6757894)(869.50571575,652.60078948)(869.55571777,652.53079102)
\curveto(869.63571562,652.42078966)(869.74071552,652.32578975)(869.87071777,652.24579102)
\curveto(870.00071526,652.1757899)(870.14071512,652.11578996)(870.29071777,652.06579102)
\curveto(870.34071492,652.05579002)(870.39071487,652.05079003)(870.44071777,652.05079102)
\curveto(870.49071477,652.05079003)(870.54071472,652.04579003)(870.59071777,652.03579102)
\curveto(870.6607146,652.01579006)(870.74571451,652.00079008)(870.84571777,651.99079102)
\curveto(870.9557143,651.99079009)(871.04571421,652.00079008)(871.11571777,652.02079102)
\curveto(871.17571408,652.04079004)(871.23571402,652.04579003)(871.29571777,652.03579102)
\curveto(871.3557139,652.03579004)(871.41571384,652.04579003)(871.47571777,652.06579102)
\curveto(871.5557137,652.08578999)(871.63071363,652.10078998)(871.70071777,652.11079102)
\curveto(871.78071348,652.12078996)(871.8557134,652.14078994)(871.92571777,652.17079102)
\curveto(872.21571304,652.29078979)(872.4607128,652.43578964)(872.66071777,652.60579102)
\curveto(872.87071239,652.7757893)(873.03071223,653.00578907)(873.14071777,653.29579102)
}
}
{
\newrgbcolor{curcolor}{0 0 0}
\pscustom[linestyle=none,fillstyle=solid,fillcolor=curcolor]
{
\newpath
\moveto(880.1473584,658.95079102)
\curveto(880.77735316,658.97078311)(881.28235266,658.88578319)(881.6623584,658.69579102)
\curveto(882.0423519,658.50578357)(882.34735159,658.22078386)(882.5773584,657.84079102)
\curveto(882.6373513,657.74078434)(882.68235126,657.63078445)(882.7123584,657.51079102)
\curveto(882.75235119,657.40078468)(882.78735115,657.28578479)(882.8173584,657.16579102)
\curveto(882.86735107,656.9757851)(882.89735104,656.77078531)(882.9073584,656.55079102)
\curveto(882.91735102,656.33078575)(882.92235102,656.10578597)(882.9223584,655.87579102)
\lineto(882.9223584,654.27079102)
\lineto(882.9223584,651.93079102)
\curveto(882.92235102,651.76079032)(882.91735102,651.59079049)(882.9073584,651.42079102)
\curveto(882.90735103,651.25079083)(882.8423511,651.14079094)(882.7123584,651.09079102)
\curveto(882.66235128,651.07079101)(882.60735133,651.06079102)(882.5473584,651.06079102)
\curveto(882.49735144,651.05079103)(882.4423515,651.04579103)(882.3823584,651.04579102)
\curveto(882.25235169,651.04579103)(882.12735181,651.05079103)(882.0073584,651.06079102)
\curveto(881.88735205,651.06079102)(881.80235214,651.10079098)(881.7523584,651.18079102)
\curveto(881.70235224,651.25079083)(881.67735226,651.34079074)(881.6773584,651.45079102)
\lineto(881.6773584,651.78079102)
\lineto(881.6773584,653.07079102)
\lineto(881.6773584,655.51579102)
\curveto(881.67735226,655.78578629)(881.67235227,656.05078603)(881.6623584,656.31079102)
\curveto(881.65235229,656.5807855)(881.60735233,656.81078527)(881.5273584,657.00079102)
\curveto(881.44735249,657.20078488)(881.32735261,657.36078472)(881.1673584,657.48079102)
\curveto(881.00735293,657.61078447)(880.82235312,657.71078437)(880.6123584,657.78079102)
\curveto(880.55235339,657.80078428)(880.48735345,657.81078427)(880.4173584,657.81079102)
\curveto(880.35735358,657.82078426)(880.29735364,657.83578424)(880.2373584,657.85579102)
\curveto(880.18735375,657.86578421)(880.10735383,657.86578421)(879.9973584,657.85579102)
\curveto(879.89735404,657.85578422)(879.82735411,657.85078423)(879.7873584,657.84079102)
\curveto(879.74735419,657.82078426)(879.71235423,657.81078427)(879.6823584,657.81079102)
\curveto(879.65235429,657.82078426)(879.61735432,657.82078426)(879.5773584,657.81079102)
\curveto(879.44735449,657.7807843)(879.32235462,657.74578433)(879.2023584,657.70579102)
\curveto(879.09235485,657.6757844)(878.98735495,657.63078445)(878.8873584,657.57079102)
\curveto(878.84735509,657.55078453)(878.81235513,657.53078455)(878.7823584,657.51079102)
\curveto(878.75235519,657.49078459)(878.71735522,657.47078461)(878.6773584,657.45079102)
\curveto(878.32735561,657.20078488)(878.07235587,656.82578525)(877.9123584,656.32579102)
\curveto(877.88235606,656.24578583)(877.86235608,656.16078592)(877.8523584,656.07079102)
\curveto(877.8423561,655.99078609)(877.82735611,655.91078617)(877.8073584,655.83079102)
\curveto(877.78735615,655.7807863)(877.78235616,655.73078635)(877.7923584,655.68079102)
\curveto(877.80235614,655.64078644)(877.79735614,655.60078648)(877.7773584,655.56079102)
\lineto(877.7773584,655.24579102)
\curveto(877.76735617,655.21578686)(877.76235618,655.1807869)(877.7623584,655.14079102)
\curveto(877.77235617,655.10078698)(877.77735616,655.05578702)(877.7773584,655.00579102)
\lineto(877.7773584,654.55579102)
\lineto(877.7773584,653.11579102)
\lineto(877.7773584,651.79579102)
\lineto(877.7773584,651.45079102)
\curveto(877.77735616,651.34079074)(877.75235619,651.25079083)(877.7023584,651.18079102)
\curveto(877.65235629,651.10079098)(877.56235638,651.06079102)(877.4323584,651.06079102)
\curveto(877.31235663,651.05079103)(877.18735675,651.04579103)(877.0573584,651.04579102)
\curveto(876.97735696,651.04579103)(876.90235704,651.05079103)(876.8323584,651.06079102)
\curveto(876.76235718,651.07079101)(876.70235724,651.09579098)(876.6523584,651.13579102)
\curveto(876.57235737,651.18579089)(876.53235741,651.2807908)(876.5323584,651.42079102)
\lineto(876.5323584,651.82579102)
\lineto(876.5323584,653.59579102)
\lineto(876.5323584,657.22579102)
\lineto(876.5323584,658.14079102)
\lineto(876.5323584,658.41079102)
\curveto(876.53235741,658.50078358)(876.55235739,658.57078351)(876.5923584,658.62079102)
\curveto(876.62235732,658.6807834)(876.67235727,658.72078336)(876.7423584,658.74079102)
\curveto(876.78235716,658.75078333)(876.8373571,658.76078332)(876.9073584,658.77079102)
\curveto(876.98735695,658.7807833)(877.06735687,658.78578329)(877.1473584,658.78579102)
\curveto(877.22735671,658.78578329)(877.30235664,658.7807833)(877.3723584,658.77079102)
\curveto(877.45235649,658.76078332)(877.50735643,658.74578333)(877.5373584,658.72579102)
\curveto(877.64735629,658.65578342)(877.69735624,658.56578351)(877.6873584,658.45579102)
\curveto(877.67735626,658.35578372)(877.69235625,658.24078384)(877.7323584,658.11079102)
\curveto(877.75235619,658.05078403)(877.79235615,658.00078408)(877.8523584,657.96079102)
\curveto(877.97235597,657.95078413)(878.06735587,657.99578408)(878.1373584,658.09579102)
\curveto(878.21735572,658.19578388)(878.29735564,658.2757838)(878.3773584,658.33579102)
\curveto(878.51735542,658.43578364)(878.65735528,658.52578355)(878.7973584,658.60579102)
\curveto(878.94735499,658.69578338)(879.11735482,658.77078331)(879.3073584,658.83079102)
\curveto(879.38735455,658.86078322)(879.47235447,658.8807832)(879.5623584,658.89079102)
\curveto(879.66235428,658.90078318)(879.75735418,658.91578316)(879.8473584,658.93579102)
\curveto(879.89735404,658.94578313)(879.94735399,658.95078313)(879.9973584,658.95079102)
\lineto(880.1473584,658.95079102)
}
}
{
\newrgbcolor{curcolor}{0 0 0}
\pscustom[linestyle=none,fillstyle=solid,fillcolor=curcolor]
{
\newpath
\moveto(885.75196777,661.14079102)
\curveto(885.90196576,661.14078094)(886.05196561,661.13578094)(886.20196777,661.12579102)
\curveto(886.35196531,661.12578095)(886.45696521,661.08578099)(886.51696777,661.00579102)
\curveto(886.5669651,660.94578113)(886.59196507,660.86078122)(886.59196777,660.75079102)
\curveto(886.60196506,660.65078143)(886.60696506,660.54578153)(886.60696777,660.43579102)
\lineto(886.60696777,659.56579102)
\curveto(886.60696506,659.48578259)(886.60196506,659.40078268)(886.59196777,659.31079102)
\curveto(886.59196507,659.23078285)(886.60196506,659.16078292)(886.62196777,659.10079102)
\curveto(886.661965,658.96078312)(886.75196491,658.87078321)(886.89196777,658.83079102)
\curveto(886.94196472,658.82078326)(886.98696468,658.81578326)(887.02696777,658.81579102)
\lineto(887.17696777,658.81579102)
\lineto(887.58196777,658.81579102)
\curveto(887.74196392,658.82578325)(887.85696381,658.81578326)(887.92696777,658.78579102)
\curveto(888.01696365,658.72578335)(888.07696359,658.66578341)(888.10696777,658.60579102)
\curveto(888.12696354,658.56578351)(888.13696353,658.52078356)(888.13696777,658.47079102)
\lineto(888.13696777,658.32079102)
\curveto(888.13696353,658.21078387)(888.13196353,658.10578397)(888.12196777,658.00579102)
\curveto(888.11196355,657.91578416)(888.07696359,657.84578423)(888.01696777,657.79579102)
\curveto(887.95696371,657.74578433)(887.87196379,657.71578436)(887.76196777,657.70579102)
\lineto(887.43196777,657.70579102)
\curveto(887.32196434,657.71578436)(887.21196445,657.72078436)(887.10196777,657.72079102)
\curveto(886.99196467,657.72078436)(886.89696477,657.70578437)(886.81696777,657.67579102)
\curveto(886.74696492,657.64578443)(886.69696497,657.59578448)(886.66696777,657.52579102)
\curveto(886.63696503,657.45578462)(886.61696505,657.37078471)(886.60696777,657.27079102)
\curveto(886.59696507,657.1807849)(886.59196507,657.080785)(886.59196777,656.97079102)
\curveto(886.60196506,656.87078521)(886.60696506,656.77078531)(886.60696777,656.67079102)
\lineto(886.60696777,653.70079102)
\curveto(886.60696506,653.4807886)(886.60196506,653.24578883)(886.59196777,652.99579102)
\curveto(886.59196507,652.75578932)(886.63696503,652.57078951)(886.72696777,652.44079102)
\curveto(886.77696489,652.36078972)(886.84196482,652.30578977)(886.92196777,652.27579102)
\curveto(887.00196466,652.24578983)(887.09696457,652.22078986)(887.20696777,652.20079102)
\curveto(887.23696443,652.19078989)(887.2669644,652.18578989)(887.29696777,652.18579102)
\curveto(887.33696433,652.19578988)(887.37196429,652.19578988)(887.40196777,652.18579102)
\lineto(887.59696777,652.18579102)
\curveto(887.69696397,652.18578989)(887.78696388,652.1757899)(887.86696777,652.15579102)
\curveto(887.95696371,652.14578993)(888.02196364,652.11078997)(888.06196777,652.05079102)
\curveto(888.08196358,652.02079006)(888.09696357,651.96579011)(888.10696777,651.88579102)
\curveto(888.12696354,651.81579026)(888.13696353,651.74079034)(888.13696777,651.66079102)
\curveto(888.14696352,651.5807905)(888.14696352,651.50079058)(888.13696777,651.42079102)
\curveto(888.12696354,651.35079073)(888.10696356,651.29579078)(888.07696777,651.25579102)
\curveto(888.03696363,651.18579089)(887.9619637,651.13579094)(887.85196777,651.10579102)
\curveto(887.77196389,651.08579099)(887.68196398,651.075791)(887.58196777,651.07579102)
\curveto(887.48196418,651.08579099)(887.39196427,651.09079099)(887.31196777,651.09079102)
\curveto(887.25196441,651.09079099)(887.19196447,651.08579099)(887.13196777,651.07579102)
\curveto(887.07196459,651.075791)(887.01696465,651.080791)(886.96696777,651.09079102)
\lineto(886.78696777,651.09079102)
\curveto(886.73696493,651.10079098)(886.68696498,651.10579097)(886.63696777,651.10579102)
\curveto(886.59696507,651.11579096)(886.55196511,651.12079096)(886.50196777,651.12079102)
\curveto(886.30196536,651.17079091)(886.12696554,651.22579085)(885.97696777,651.28579102)
\curveto(885.83696583,651.34579073)(885.71696595,651.45079063)(885.61696777,651.60079102)
\curveto(885.47696619,651.80079028)(885.39696627,652.05079003)(885.37696777,652.35079102)
\curveto(885.35696631,652.66078942)(885.34696632,652.99078909)(885.34696777,653.34079102)
\lineto(885.34696777,657.27079102)
\curveto(885.31696635,657.40078468)(885.28696638,657.49578458)(885.25696777,657.55579102)
\curveto(885.23696643,657.61578446)(885.1669665,657.66578441)(885.04696777,657.70579102)
\curveto(885.00696666,657.71578436)(884.9669667,657.71578436)(884.92696777,657.70579102)
\curveto(884.88696678,657.69578438)(884.84696682,657.70078438)(884.80696777,657.72079102)
\lineto(884.56696777,657.72079102)
\curveto(884.43696723,657.72078436)(884.32696734,657.73078435)(884.23696777,657.75079102)
\curveto(884.15696751,657.7807843)(884.10196756,657.84078424)(884.07196777,657.93079102)
\curveto(884.05196761,657.97078411)(884.03696763,658.01578406)(884.02696777,658.06579102)
\lineto(884.02696777,658.21579102)
\curveto(884.02696764,658.35578372)(884.03696763,658.47078361)(884.05696777,658.56079102)
\curveto(884.07696759,658.66078342)(884.13696753,658.73578334)(884.23696777,658.78579102)
\curveto(884.34696732,658.82578325)(884.48696718,658.83578324)(884.65696777,658.81579102)
\curveto(884.83696683,658.79578328)(884.98696668,658.80578327)(885.10696777,658.84579102)
\curveto(885.19696647,658.89578318)(885.2669664,658.96578311)(885.31696777,659.05579102)
\curveto(885.33696633,659.11578296)(885.34696632,659.19078289)(885.34696777,659.28079102)
\lineto(885.34696777,659.53579102)
\lineto(885.34696777,660.46579102)
\lineto(885.34696777,660.70579102)
\curveto(885.34696632,660.79578128)(885.35696631,660.87078121)(885.37696777,660.93079102)
\curveto(885.41696625,661.01078107)(885.49196617,661.075781)(885.60196777,661.12579102)
\curveto(885.63196603,661.12578095)(885.65696601,661.12578095)(885.67696777,661.12579102)
\curveto(885.70696596,661.13578094)(885.73196593,661.14078094)(885.75196777,661.14079102)
}
}
{
\newrgbcolor{curcolor}{0 0 0}
\pscustom[linestyle=none,fillstyle=solid,fillcolor=curcolor]
{
\newpath
\moveto(896.27376465,655.24579102)
\curveto(896.29375696,655.14578693)(896.29375696,655.03078705)(896.27376465,654.90079102)
\curveto(896.26375699,654.7807873)(896.23375702,654.69578738)(896.18376465,654.64579102)
\curveto(896.13375712,654.60578747)(896.0587572,654.5757875)(895.95876465,654.55579102)
\curveto(895.86875739,654.54578753)(895.76375749,654.54078754)(895.64376465,654.54079102)
\lineto(895.28376465,654.54079102)
\curveto(895.16375809,654.55078753)(895.0587582,654.55578752)(894.96876465,654.55579102)
\lineto(891.12876465,654.55579102)
\curveto(891.04876221,654.55578752)(890.96876229,654.55078753)(890.88876465,654.54079102)
\curveto(890.80876245,654.54078754)(890.74376251,654.52578755)(890.69376465,654.49579102)
\curveto(890.6537626,654.4757876)(890.61376264,654.43578764)(890.57376465,654.37579102)
\curveto(890.5537627,654.34578773)(890.53376272,654.30078778)(890.51376465,654.24079102)
\curveto(890.49376276,654.19078789)(890.49376276,654.14078794)(890.51376465,654.09079102)
\curveto(890.52376273,654.04078804)(890.52876273,653.99578808)(890.52876465,653.95579102)
\curveto(890.52876273,653.91578816)(890.53376272,653.8757882)(890.54376465,653.83579102)
\curveto(890.56376269,653.75578832)(890.58376267,653.67078841)(890.60376465,653.58079102)
\curveto(890.62376263,653.50078858)(890.6537626,653.42078866)(890.69376465,653.34079102)
\curveto(890.92376233,652.80078928)(891.30376195,652.41578966)(891.83376465,652.18579102)
\curveto(891.89376136,652.15578992)(891.9587613,652.13078995)(892.02876465,652.11079102)
\lineto(892.23876465,652.05079102)
\curveto(892.26876099,652.04079004)(892.31876094,652.03579004)(892.38876465,652.03579102)
\curveto(892.52876073,651.99579008)(892.71376054,651.9757901)(892.94376465,651.97579102)
\curveto(893.17376008,651.9757901)(893.3587599,651.99579008)(893.49876465,652.03579102)
\curveto(893.63875962,652.07579)(893.76375949,652.11578996)(893.87376465,652.15579102)
\curveto(893.99375926,652.20578987)(894.10375915,652.26578981)(894.20376465,652.33579102)
\curveto(894.31375894,652.40578967)(894.40875885,652.48578959)(894.48876465,652.57579102)
\curveto(894.56875869,652.6757894)(894.63875862,652.7807893)(894.69876465,652.89079102)
\curveto(894.7587585,652.99078909)(894.80875845,653.09578898)(894.84876465,653.20579102)
\curveto(894.89875836,653.31578876)(894.97875828,653.39578868)(895.08876465,653.44579102)
\curveto(895.12875813,653.46578861)(895.19375806,653.4807886)(895.28376465,653.49079102)
\curveto(895.37375788,653.50078858)(895.46375779,653.50078858)(895.55376465,653.49079102)
\curveto(895.64375761,653.49078859)(895.72875753,653.48578859)(895.80876465,653.47579102)
\curveto(895.88875737,653.46578861)(895.94375731,653.44578863)(895.97376465,653.41579102)
\curveto(896.07375718,653.34578873)(896.09875716,653.23078885)(896.04876465,653.07079102)
\curveto(895.96875729,652.80078928)(895.86375739,652.56078952)(895.73376465,652.35079102)
\curveto(895.53375772,652.03079005)(895.30375795,651.76579031)(895.04376465,651.55579102)
\curveto(894.79375846,651.35579072)(894.47375878,651.19079089)(894.08376465,651.06079102)
\curveto(893.98375927,651.02079106)(893.88375937,650.99579108)(893.78376465,650.98579102)
\curveto(893.68375957,650.96579111)(893.57875968,650.94579113)(893.46876465,650.92579102)
\curveto(893.41875984,650.91579116)(893.36875989,650.91079117)(893.31876465,650.91079102)
\curveto(893.27875998,650.91079117)(893.23376002,650.90579117)(893.18376465,650.89579102)
\lineto(893.03376465,650.89579102)
\curveto(892.98376027,650.88579119)(892.92376033,650.8807912)(892.85376465,650.88079102)
\curveto(892.79376046,650.8807912)(892.74376051,650.88579119)(892.70376465,650.89579102)
\lineto(892.56876465,650.89579102)
\curveto(892.51876074,650.90579117)(892.47376078,650.91079117)(892.43376465,650.91079102)
\curveto(892.39376086,650.91079117)(892.3537609,650.91579116)(892.31376465,650.92579102)
\curveto(892.26376099,650.93579114)(892.20876105,650.94579113)(892.14876465,650.95579102)
\curveto(892.08876117,650.95579112)(892.03376122,650.96079112)(891.98376465,650.97079102)
\curveto(891.89376136,650.99079109)(891.80376145,651.01579106)(891.71376465,651.04579102)
\curveto(891.62376163,651.06579101)(891.53876172,651.09079099)(891.45876465,651.12079102)
\curveto(891.41876184,651.14079094)(891.38376187,651.15079093)(891.35376465,651.15079102)
\curveto(891.32376193,651.16079092)(891.28876197,651.1757909)(891.24876465,651.19579102)
\curveto(891.09876216,651.26579081)(890.93876232,651.35079073)(890.76876465,651.45079102)
\curveto(890.47876278,651.64079044)(890.22876303,651.87079021)(890.01876465,652.14079102)
\curveto(889.81876344,652.42078966)(889.64876361,652.73078935)(889.50876465,653.07079102)
\curveto(889.4587638,653.1807889)(889.41876384,653.29578878)(889.38876465,653.41579102)
\curveto(889.36876389,653.53578854)(889.33876392,653.65578842)(889.29876465,653.77579102)
\curveto(889.28876397,653.81578826)(889.28376397,653.85078823)(889.28376465,653.88079102)
\curveto(889.28376397,653.91078817)(889.27876398,653.95078813)(889.26876465,654.00079102)
\curveto(889.24876401,654.080788)(889.23376402,654.16578791)(889.22376465,654.25579102)
\curveto(889.21376404,654.34578773)(889.19876406,654.43578764)(889.17876465,654.52579102)
\lineto(889.17876465,654.73579102)
\curveto(889.16876409,654.7757873)(889.1587641,654.83078725)(889.14876465,654.90079102)
\curveto(889.14876411,654.9807871)(889.1537641,655.04578703)(889.16376465,655.09579102)
\lineto(889.16376465,655.26079102)
\curveto(889.18376407,655.31078677)(889.18876407,655.36078672)(889.17876465,655.41079102)
\curveto(889.17876408,655.47078661)(889.18376407,655.52578655)(889.19376465,655.57579102)
\curveto(889.23376402,655.73578634)(889.26376399,655.89578618)(889.28376465,656.05579102)
\curveto(889.31376394,656.21578586)(889.3587639,656.36578571)(889.41876465,656.50579102)
\curveto(889.46876379,656.61578546)(889.51376374,656.72578535)(889.55376465,656.83579102)
\curveto(889.60376365,656.95578512)(889.6587636,657.07078501)(889.71876465,657.18079102)
\curveto(889.93876332,657.53078455)(890.18876307,657.83078425)(890.46876465,658.08079102)
\curveto(890.74876251,658.34078374)(891.09376216,658.55578352)(891.50376465,658.72579102)
\curveto(891.62376163,658.7757833)(891.74376151,658.81078327)(891.86376465,658.83079102)
\curveto(891.99376126,658.86078322)(892.12876113,658.89078319)(892.26876465,658.92079102)
\curveto(892.31876094,658.93078315)(892.36376089,658.93578314)(892.40376465,658.93579102)
\curveto(892.44376081,658.94578313)(892.48876077,658.95078313)(892.53876465,658.95079102)
\curveto(892.5587607,658.96078312)(892.58376067,658.96078312)(892.61376465,658.95079102)
\curveto(892.64376061,658.94078314)(892.66876059,658.94578313)(892.68876465,658.96579102)
\curveto(893.10876015,658.9757831)(893.47375978,658.93078315)(893.78376465,658.83079102)
\curveto(894.09375916,658.74078334)(894.37375888,658.61578346)(894.62376465,658.45579102)
\curveto(894.67375858,658.43578364)(894.71375854,658.40578367)(894.74376465,658.36579102)
\curveto(894.77375848,658.33578374)(894.80875845,658.31078377)(894.84876465,658.29079102)
\curveto(894.92875833,658.23078385)(895.00875825,658.16078392)(895.08876465,658.08079102)
\curveto(895.17875808,658.00078408)(895.253758,657.92078416)(895.31376465,657.84079102)
\curveto(895.47375778,657.63078445)(895.60875765,657.43078465)(895.71876465,657.24079102)
\curveto(895.78875747,657.13078495)(895.84375741,657.01078507)(895.88376465,656.88079102)
\curveto(895.92375733,656.75078533)(895.96875729,656.62078546)(896.01876465,656.49079102)
\curveto(896.06875719,656.36078572)(896.10375715,656.22578585)(896.12376465,656.08579102)
\curveto(896.1537571,655.94578613)(896.18875707,655.80578627)(896.22876465,655.66579102)
\curveto(896.23875702,655.59578648)(896.24375701,655.52578655)(896.24376465,655.45579102)
\lineto(896.27376465,655.24579102)
\moveto(894.81876465,655.75579102)
\curveto(894.84875841,655.79578628)(894.87375838,655.84578623)(894.89376465,655.90579102)
\curveto(894.91375834,655.9757861)(894.91375834,656.04578603)(894.89376465,656.11579102)
\curveto(894.83375842,656.33578574)(894.74875851,656.54078554)(894.63876465,656.73079102)
\curveto(894.49875876,656.96078512)(894.34375891,657.15578492)(894.17376465,657.31579102)
\curveto(894.00375925,657.4757846)(893.78375947,657.61078447)(893.51376465,657.72079102)
\curveto(893.44375981,657.74078434)(893.37375988,657.75578432)(893.30376465,657.76579102)
\curveto(893.23376002,657.78578429)(893.1587601,657.80578427)(893.07876465,657.82579102)
\curveto(892.99876026,657.84578423)(892.91376034,657.85578422)(892.82376465,657.85579102)
\lineto(892.56876465,657.85579102)
\curveto(892.53876072,657.83578424)(892.50376075,657.82578425)(892.46376465,657.82579102)
\curveto(892.42376083,657.83578424)(892.38876087,657.83578424)(892.35876465,657.82579102)
\lineto(892.11876465,657.76579102)
\curveto(892.04876121,657.75578432)(891.97876128,657.74078434)(891.90876465,657.72079102)
\curveto(891.61876164,657.60078448)(891.38376187,657.45078463)(891.20376465,657.27079102)
\curveto(891.03376222,657.09078499)(890.87876238,656.86578521)(890.73876465,656.59579102)
\curveto(890.70876255,656.54578553)(890.67876258,656.4807856)(890.64876465,656.40079102)
\curveto(890.61876264,656.33078575)(890.59376266,656.25078583)(890.57376465,656.16079102)
\curveto(890.5537627,656.07078601)(890.54876271,655.98578609)(890.55876465,655.90579102)
\curveto(890.56876269,655.82578625)(890.60376265,655.76578631)(890.66376465,655.72579102)
\curveto(890.74376251,655.66578641)(890.87876238,655.63578644)(891.06876465,655.63579102)
\curveto(891.26876199,655.64578643)(891.43876182,655.65078643)(891.57876465,655.65079102)
\lineto(893.85876465,655.65079102)
\curveto(894.00875925,655.65078643)(894.18875907,655.64578643)(894.39876465,655.63579102)
\curveto(894.60875865,655.63578644)(894.74875851,655.6757864)(894.81876465,655.75579102)
}
}
{
\newrgbcolor{curcolor}{0.80000001 0.80000001 0.80000001}
\pscustom[linestyle=none,fillstyle=solid,fillcolor=curcolor]
{
\newpath
\moveto(807.09008789,661.78582764)
\lineto(822.09008789,661.78582764)
\lineto(822.09008789,646.78582764)
\lineto(807.09008789,646.78582764)
\closepath
}
}
{
\newrgbcolor{curcolor}{0 0 0}
\pscustom[linestyle=none,fillstyle=solid,fillcolor=curcolor]
{
\newpath
\moveto(835.0282959,628.99861328)
\curveto(835.04828635,628.94861254)(835.07328633,628.8886126)(835.1032959,628.81861328)
\curveto(835.13328627,628.74861274)(835.15328625,628.67361281)(835.1632959,628.59361328)
\curveto(835.18328622,628.52361296)(835.18328622,628.45361303)(835.1632959,628.38361328)
\curveto(835.15328625,628.32361316)(835.11328629,628.27861321)(835.0432959,628.24861328)
\curveto(834.99328641,628.22861326)(834.93328647,628.21861327)(834.8632959,628.21861328)
\lineto(834.6532959,628.21861328)
\lineto(834.2032959,628.21861328)
\curveto(834.05328735,628.21861327)(833.93328747,628.24361324)(833.8432959,628.29361328)
\curveto(833.74328766,628.35361313)(833.66828773,628.45861303)(833.6182959,628.60861328)
\curveto(833.57828782,628.75861273)(833.53328787,628.89361259)(833.4832959,629.01361328)
\curveto(833.37328803,629.27361221)(833.27328813,629.54361194)(833.1832959,629.82361328)
\curveto(833.09328831,630.10361138)(832.99328841,630.37861111)(832.8832959,630.64861328)
\curveto(832.85328855,630.73861075)(832.82328858,630.82361066)(832.7932959,630.90361328)
\curveto(832.77328863,630.9836105)(832.74328866,631.05861043)(832.7032959,631.12861328)
\curveto(832.67328873,631.19861029)(832.62828877,631.25861023)(832.5682959,631.30861328)
\curveto(832.50828889,631.35861013)(832.42828897,631.39861009)(832.3282959,631.42861328)
\curveto(832.27828912,631.44861004)(832.21828918,631.45361003)(832.1482959,631.44361328)
\lineto(831.9532959,631.44361328)
\lineto(829.1182959,631.44361328)
\lineto(828.8182959,631.44361328)
\curveto(828.70829269,631.45361003)(828.6032928,631.45361003)(828.5032959,631.44361328)
\curveto(828.403293,631.43361005)(828.30829309,631.41861007)(828.2182959,631.39861328)
\curveto(828.13829326,631.37861011)(828.07829332,631.33861015)(828.0382959,631.27861328)
\curveto(827.95829344,631.17861031)(827.8982935,631.06361042)(827.8582959,630.93361328)
\curveto(827.82829357,630.81361067)(827.78829361,630.6886108)(827.7382959,630.55861328)
\curveto(827.63829376,630.32861116)(827.54329386,630.0886114)(827.4532959,629.83861328)
\curveto(827.37329403,629.5886119)(827.28329412,629.34861214)(827.1832959,629.11861328)
\curveto(827.16329424,629.05861243)(827.13829426,628.9886125)(827.1082959,628.90861328)
\curveto(827.08829431,628.83861265)(827.06329434,628.76361272)(827.0332959,628.68361328)
\curveto(827.0032944,628.60361288)(826.96829443,628.52861296)(826.9282959,628.45861328)
\curveto(826.8982945,628.39861309)(826.86329454,628.35361313)(826.8232959,628.32361328)
\curveto(826.74329466,628.26361322)(826.63329477,628.22861326)(826.4932959,628.21861328)
\lineto(826.0732959,628.21861328)
\lineto(825.8332959,628.21861328)
\curveto(825.76329564,628.22861326)(825.7032957,628.25361323)(825.6532959,628.29361328)
\curveto(825.6032958,628.32361316)(825.57329583,628.36861312)(825.5632959,628.42861328)
\curveto(825.56329584,628.488613)(825.56829583,628.54861294)(825.5782959,628.60861328)
\curveto(825.5982958,628.67861281)(825.61829578,628.74361274)(825.6382959,628.80361328)
\curveto(825.66829573,628.87361261)(825.69329571,628.92361256)(825.7132959,628.95361328)
\curveto(825.85329555,629.27361221)(825.97829542,629.5886119)(826.0882959,629.89861328)
\curveto(826.1982952,630.21861127)(826.31829508,630.53861095)(826.4482959,630.85861328)
\curveto(826.53829486,631.07861041)(826.62329478,631.29361019)(826.7032959,631.50361328)
\curveto(826.78329462,631.72360976)(826.86829453,631.94360954)(826.9582959,632.16361328)
\curveto(827.25829414,632.8836086)(827.54329386,633.60860788)(827.8132959,634.33861328)
\curveto(828.08329332,635.07860641)(828.36829303,635.81360567)(828.6682959,636.54361328)
\curveto(828.77829262,636.80360468)(828.87829252,637.06860442)(828.9682959,637.33861328)
\curveto(829.06829233,637.60860388)(829.17329223,637.87360361)(829.2832959,638.13361328)
\curveto(829.33329207,638.24360324)(829.37829202,638.36360312)(829.4182959,638.49361328)
\curveto(829.46829193,638.63360285)(829.53829186,638.73360275)(829.6282959,638.79361328)
\curveto(829.66829173,638.83360265)(829.73329167,638.86360262)(829.8232959,638.88361328)
\curveto(829.84329156,638.89360259)(829.86329154,638.89360259)(829.8832959,638.88361328)
\curveto(829.91329149,638.8836026)(829.93829146,638.8886026)(829.9582959,638.89861328)
\curveto(830.13829126,638.89860259)(830.34829105,638.89860259)(830.5882959,638.89861328)
\curveto(830.82829057,638.90860258)(831.0032904,638.87360261)(831.1132959,638.79361328)
\curveto(831.19329021,638.73360275)(831.25329015,638.63360285)(831.2932959,638.49361328)
\curveto(831.34329006,638.36360312)(831.39329001,638.24360324)(831.4432959,638.13361328)
\curveto(831.54328986,637.90360358)(831.63328977,637.67360381)(831.7132959,637.44361328)
\curveto(831.79328961,637.21360427)(831.88328952,636.9836045)(831.9832959,636.75361328)
\curveto(832.06328934,636.55360493)(832.13828926,636.34860514)(832.2082959,636.13861328)
\curveto(832.28828911,635.92860556)(832.37328903,635.72360576)(832.4632959,635.52361328)
\curveto(832.76328864,634.79360669)(833.04828835,634.05360743)(833.3182959,633.30361328)
\curveto(833.5982878,632.56360892)(833.89328751,631.82860966)(834.2032959,631.09861328)
\curveto(834.24328716,631.00861048)(834.27328713,630.92361056)(834.2932959,630.84361328)
\curveto(834.32328708,630.76361072)(834.35328705,630.67861081)(834.3832959,630.58861328)
\curveto(834.49328691,630.32861116)(834.5982868,630.06361142)(834.6982959,629.79361328)
\curveto(834.80828659,629.52361196)(834.91828648,629.25861223)(835.0282959,628.99861328)
\moveto(831.8182959,632.64361328)
\curveto(831.90828949,632.67360881)(831.96328944,632.72360876)(831.9832959,632.79361328)
\curveto(832.01328939,632.86360862)(832.01828938,632.93860855)(831.9982959,633.01861328)
\curveto(831.98828941,633.10860838)(831.96328944,633.19360829)(831.9232959,633.27361328)
\curveto(831.89328951,633.36360812)(831.86328954,633.43860805)(831.8332959,633.49861328)
\curveto(831.81328959,633.53860795)(831.8032896,633.57360791)(831.8032959,633.60361328)
\curveto(831.8032896,633.63360785)(831.79328961,633.66860782)(831.7732959,633.70861328)
\lineto(831.6832959,633.94861328)
\curveto(831.66328974,634.03860745)(831.63328977,634.12860736)(831.5932959,634.21861328)
\curveto(831.44328996,634.57860691)(831.30829009,634.94360654)(831.1882959,635.31361328)
\curveto(831.07829032,635.69360579)(830.94829045,636.06360542)(830.7982959,636.42361328)
\curveto(830.74829065,636.53360495)(830.7032907,636.64360484)(830.6632959,636.75361328)
\curveto(830.63329077,636.86360462)(830.59329081,636.96860452)(830.5432959,637.06861328)
\curveto(830.52329088,637.11860437)(830.4982909,637.16360432)(830.4682959,637.20361328)
\curveto(830.44829095,637.25360423)(830.398291,637.27860421)(830.3182959,637.27861328)
\curveto(830.2982911,637.25860423)(830.27829112,637.24360424)(830.2582959,637.23361328)
\curveto(830.23829116,637.22360426)(830.21829118,637.20860428)(830.1982959,637.18861328)
\curveto(830.15829124,637.13860435)(830.12829127,637.0836044)(830.1082959,637.02361328)
\curveto(830.08829131,636.97360451)(830.06829133,636.91860457)(830.0482959,636.85861328)
\curveto(829.9982914,636.74860474)(829.95829144,636.63860485)(829.9282959,636.52861328)
\curveto(829.8982915,636.41860507)(829.85829154,636.30860518)(829.8082959,636.19861328)
\curveto(829.63829176,635.80860568)(829.48829191,635.41360607)(829.3582959,635.01361328)
\curveto(829.23829216,634.61360687)(829.0982923,634.22360726)(828.9382959,633.84361328)
\lineto(828.8782959,633.69361328)
\curveto(828.86829253,633.64360784)(828.85329255,633.59360789)(828.8332959,633.54361328)
\lineto(828.7432959,633.30361328)
\curveto(828.71329269,633.22360826)(828.68829271,633.14360834)(828.6682959,633.06361328)
\curveto(828.64829275,633.01360847)(828.63829276,632.95860853)(828.6382959,632.89861328)
\curveto(828.64829275,632.83860865)(828.66329274,632.7886087)(828.6832959,632.74861328)
\curveto(828.73329267,632.66860882)(828.83829256,632.62360886)(828.9982959,632.61361328)
\lineto(829.4482959,632.61361328)
\lineto(831.0532959,632.61361328)
\curveto(831.16329024,632.61360887)(831.2982901,632.60860888)(831.4582959,632.59861328)
\curveto(831.61828978,632.59860889)(831.73828966,632.61360887)(831.8182959,632.64361328)
}
}
{
\newrgbcolor{curcolor}{0 0 0}
\pscustom[linestyle=none,fillstyle=solid,fillcolor=curcolor]
{
\newpath
\moveto(836.6098584,635.94361328)
\lineto(837.0448584,635.94361328)
\curveto(837.19485643,635.94360554)(837.29985633,635.90360558)(837.3598584,635.82361328)
\curveto(837.40985622,635.74360574)(837.43485619,635.64360584)(837.4348584,635.52361328)
\curveto(837.44485618,635.40360608)(837.44985618,635.2836062)(837.4498584,635.16361328)
\lineto(837.4498584,633.73861328)
\lineto(837.4498584,631.47361328)
\lineto(837.4498584,630.78361328)
\curveto(837.44985618,630.55361093)(837.47485615,630.35361113)(837.5248584,630.18361328)
\curveto(837.68485594,629.73361175)(837.98485564,629.41861207)(838.4248584,629.23861328)
\curveto(838.64485498,629.14861234)(838.90985472,629.11361237)(839.2198584,629.13361328)
\curveto(839.5298541,629.16361232)(839.77985385,629.21861227)(839.9698584,629.29861328)
\curveto(840.29985333,629.43861205)(840.55985307,629.61361187)(840.7498584,629.82361328)
\curveto(840.94985268,630.04361144)(841.10485252,630.32861116)(841.2148584,630.67861328)
\curveto(841.24485238,630.75861073)(841.26485236,630.83861065)(841.2748584,630.91861328)
\curveto(841.28485234,630.99861049)(841.29985233,631.0836104)(841.3198584,631.17361328)
\curveto(841.3298523,631.22361026)(841.3298523,631.26861022)(841.3198584,631.30861328)
\curveto(841.31985231,631.34861014)(841.3298523,631.39361009)(841.3498584,631.44361328)
\lineto(841.3498584,631.75861328)
\curveto(841.36985226,631.83860965)(841.37485225,631.92860956)(841.3648584,632.02861328)
\curveto(841.35485227,632.13860935)(841.34985228,632.23860925)(841.3498584,632.32861328)
\lineto(841.3498584,633.49861328)
\lineto(841.3498584,635.08861328)
\curveto(841.34985228,635.20860628)(841.34485228,635.33360615)(841.3348584,635.46361328)
\curveto(841.33485229,635.60360588)(841.35985227,635.71360577)(841.4098584,635.79361328)
\curveto(841.44985218,635.84360564)(841.49485213,635.87360561)(841.5448584,635.88361328)
\curveto(841.60485202,635.90360558)(841.67485195,635.92360556)(841.7548584,635.94361328)
\lineto(841.9798584,635.94361328)
\curveto(842.09985153,635.94360554)(842.20485142,635.93860555)(842.2948584,635.92861328)
\curveto(842.39485123,635.91860557)(842.46985116,635.87360561)(842.5198584,635.79361328)
\curveto(842.56985106,635.74360574)(842.59485103,635.66860582)(842.5948584,635.56861328)
\lineto(842.5948584,635.28361328)
\lineto(842.5948584,634.26361328)
\lineto(842.5948584,630.22861328)
\lineto(842.5948584,628.87861328)
\curveto(842.59485103,628.75861273)(842.58985104,628.64361284)(842.5798584,628.53361328)
\curveto(842.57985105,628.43361305)(842.54485108,628.35861313)(842.4748584,628.30861328)
\curveto(842.43485119,628.27861321)(842.37485125,628.25361323)(842.2948584,628.23361328)
\curveto(842.21485141,628.22361326)(842.1248515,628.21361327)(842.0248584,628.20361328)
\curveto(841.93485169,628.20361328)(841.84485178,628.20861328)(841.7548584,628.21861328)
\curveto(841.67485195,628.22861326)(841.61485201,628.24861324)(841.5748584,628.27861328)
\curveto(841.5248521,628.31861317)(841.47985215,628.3836131)(841.4398584,628.47361328)
\curveto(841.4298522,628.51361297)(841.41985221,628.56861292)(841.4098584,628.63861328)
\curveto(841.40985222,628.70861278)(841.40485222,628.77361271)(841.3948584,628.83361328)
\curveto(841.38485224,628.90361258)(841.36485226,628.95861253)(841.3348584,628.99861328)
\curveto(841.30485232,629.03861245)(841.25985237,629.05361243)(841.1998584,629.04361328)
\curveto(841.11985251,629.02361246)(841.03985259,628.96361252)(840.9598584,628.86361328)
\curveto(840.87985275,628.77361271)(840.80485282,628.70361278)(840.7348584,628.65361328)
\curveto(840.51485311,628.49361299)(840.26485336,628.35361313)(839.9848584,628.23361328)
\curveto(839.87485375,628.1836133)(839.75985387,628.15361333)(839.6398584,628.14361328)
\curveto(839.5298541,628.12361336)(839.41485421,628.09861339)(839.2948584,628.06861328)
\curveto(839.24485438,628.05861343)(839.18985444,628.05861343)(839.1298584,628.06861328)
\curveto(839.07985455,628.07861341)(839.0298546,628.07361341)(838.9798584,628.05361328)
\curveto(838.87985475,628.03361345)(838.78985484,628.03361345)(838.7098584,628.05361328)
\lineto(838.5598584,628.05361328)
\curveto(838.50985512,628.07361341)(838.44985518,628.0836134)(838.3798584,628.08361328)
\curveto(838.31985531,628.0836134)(838.26485536,628.0886134)(838.2148584,628.09861328)
\curveto(838.17485545,628.11861337)(838.13485549,628.12861336)(838.0948584,628.12861328)
\curveto(838.06485556,628.11861337)(838.0248556,628.12361336)(837.9748584,628.14361328)
\lineto(837.7348584,628.20361328)
\curveto(837.66485596,628.22361326)(837.58985604,628.25361323)(837.5098584,628.29361328)
\curveto(837.24985638,628.40361308)(837.0298566,628.54861294)(836.8498584,628.72861328)
\curveto(836.67985695,628.91861257)(836.53985709,629.14361234)(836.4298584,629.40361328)
\curveto(836.38985724,629.49361199)(836.35985727,629.5836119)(836.3398584,629.67361328)
\lineto(836.2798584,629.97361328)
\curveto(836.25985737,630.03361145)(836.24985738,630.0886114)(836.2498584,630.13861328)
\curveto(836.25985737,630.19861129)(836.25485737,630.26361122)(836.2348584,630.33361328)
\curveto(836.2248574,630.35361113)(836.21985741,630.37861111)(836.2198584,630.40861328)
\curveto(836.21985741,630.44861104)(836.21485741,630.483611)(836.2048584,630.51361328)
\lineto(836.2048584,630.66361328)
\curveto(836.19485743,630.70361078)(836.18985744,630.74861074)(836.1898584,630.79861328)
\curveto(836.19985743,630.85861063)(836.20485742,630.91361057)(836.2048584,630.96361328)
\lineto(836.2048584,631.56361328)
\lineto(836.2048584,634.32361328)
\lineto(836.2048584,635.28361328)
\lineto(836.2048584,635.55361328)
\curveto(836.20485742,635.64360584)(836.2248574,635.71860577)(836.2648584,635.77861328)
\curveto(836.30485732,635.84860564)(836.37985725,635.89860559)(836.4898584,635.92861328)
\curveto(836.50985712,635.93860555)(836.5298571,635.93860555)(836.5498584,635.92861328)
\curveto(836.56985706,635.92860556)(836.58985704,635.93360555)(836.6098584,635.94361328)
}
}
{
\newrgbcolor{curcolor}{0 0 0}
\pscustom[linestyle=none,fillstyle=solid,fillcolor=curcolor]
{
\newpath
\moveto(844.50946777,635.94361328)
\lineto(845.03446777,635.94361328)
\curveto(845.23446612,635.95360553)(845.38446597,635.93360555)(845.48446777,635.88361328)
\curveto(845.60446575,635.83360565)(845.69946565,635.75360573)(845.76946777,635.64361328)
\curveto(845.8494655,635.53360595)(845.92446543,635.42360606)(845.99446777,635.31361328)
\curveto(846.12446523,635.11360637)(846.2544651,634.91860657)(846.38446777,634.72861328)
\curveto(846.51446484,634.54860694)(846.6494647,634.35860713)(846.78946777,634.15861328)
\curveto(846.83946451,634.07860741)(846.88946446,634.00360748)(846.93946777,633.93361328)
\curveto(846.99946435,633.86360762)(847.0544643,633.79360769)(847.10446777,633.72361328)
\curveto(847.14446421,633.66360782)(847.18446417,633.60860788)(847.22446777,633.55861328)
\curveto(847.26446409,633.50860798)(847.32446403,633.47360801)(847.40446777,633.45361328)
\curveto(847.4544639,633.43360805)(847.49446386,633.43360805)(847.52446777,633.45361328)
\curveto(847.56446379,633.483608)(847.59446376,633.50860798)(847.61446777,633.52861328)
\curveto(847.69446366,633.57860791)(847.75946359,633.64860784)(847.80946777,633.73861328)
\curveto(847.86946348,633.82860766)(847.92446343,633.91360757)(847.97446777,633.99361328)
\curveto(848.12446323,634.19360729)(848.27446308,634.39860709)(848.42446777,634.60861328)
\lineto(848.87446777,635.23861328)
\curveto(848.9544624,635.34860614)(849.03446232,635.46360602)(849.11446777,635.58361328)
\curveto(849.19446216,635.70360578)(849.28946206,635.79860569)(849.39946777,635.86861328)
\curveto(849.47946187,635.91860557)(849.57446178,635.94360554)(849.68446777,635.94361328)
\lineto(850.02946777,635.94361328)
\lineto(850.16446777,635.94361328)
\curveto(850.21446114,635.94360554)(850.26446109,635.93860555)(850.31446777,635.92861328)
\lineto(850.38946777,635.92861328)
\curveto(850.50946084,635.90860558)(850.58946076,635.86860562)(850.62946777,635.80861328)
\curveto(850.6494607,635.75860573)(850.64446071,635.70360578)(850.61446777,635.64361328)
\curveto(850.59446076,635.59360589)(850.57446078,635.55360593)(850.55446777,635.52361328)
\lineto(850.34446777,635.22361328)
\curveto(850.27446108,635.13360635)(850.19946115,635.03860645)(850.11946777,634.93861328)
\curveto(849.88946146,634.61860687)(849.6544617,634.30360718)(849.41446777,633.99361328)
\curveto(849.18446217,633.69360779)(848.9544624,633.3836081)(848.72446777,633.06361328)
\curveto(848.67446268,632.9836085)(848.61946273,632.90360858)(848.55946777,632.82361328)
\curveto(848.49946285,632.75360873)(848.44446291,632.67360881)(848.39446777,632.58361328)
\curveto(848.37446298,632.55360893)(848.354463,632.51360897)(848.33446777,632.46361328)
\curveto(848.31446304,632.42360906)(848.31446304,632.37360911)(848.33446777,632.31361328)
\curveto(848.354463,632.22360926)(848.38446297,632.14860934)(848.42446777,632.08861328)
\curveto(848.47446288,632.02860946)(848.52446283,631.96360952)(848.57446777,631.89361328)
\lineto(848.75446777,631.62361328)
\curveto(848.82446253,631.53360995)(848.88946246,631.44361004)(848.94946777,631.35361328)
\lineto(849.63946777,630.39361328)
\lineto(850.32946777,629.43361328)
\curveto(850.40946094,629.32361216)(850.48946086,629.20861228)(850.56946777,629.08861328)
\lineto(850.80946777,628.75861328)
\curveto(850.85946049,628.6886128)(850.89946045,628.62361286)(850.92946777,628.56361328)
\curveto(850.96946038,628.51361297)(850.97946037,628.43361305)(850.95946777,628.32361328)
\curveto(850.93946041,628.31361317)(850.91946043,628.29861319)(850.89946777,628.27861328)
\curveto(850.88946046,628.26861322)(850.87446048,628.25861323)(850.85446777,628.24861328)
\curveto(850.80446055,628.22861326)(850.73946061,628.21861327)(850.65946777,628.21861328)
\lineto(850.41946777,628.21861328)
\lineto(849.90946777,628.21861328)
\curveto(849.76946158,628.22861326)(849.64446171,628.27361321)(849.53446777,628.35361328)
\curveto(849.48446187,628.3836131)(849.44446191,628.41861307)(849.41446777,628.45861328)
\curveto(849.39446196,628.50861298)(849.36946198,628.55861293)(849.33946777,628.60861328)
\lineto(849.18946777,628.81861328)
\curveto(849.13946221,628.8886126)(849.08946226,628.96361252)(849.03946777,629.04361328)
\lineto(848.09446777,630.43861328)
\curveto(848.04446331,630.51861097)(847.99446336,630.59361089)(847.94446777,630.66361328)
\curveto(847.89446346,630.73361075)(847.84446351,630.80861068)(847.79446777,630.88861328)
\curveto(847.74446361,630.95861053)(847.69446366,631.01861047)(847.64446777,631.06861328)
\curveto(847.60446375,631.12861036)(847.54446381,631.16861032)(847.46446777,631.18861328)
\curveto(847.41446394,631.20861028)(847.36446399,631.19861029)(847.31446777,631.15861328)
\curveto(847.27446408,631.12861036)(847.24446411,631.10361038)(847.22446777,631.08361328)
\curveto(847.14446421,631.00361048)(847.07446428,630.91361057)(847.01446777,630.81361328)
\curveto(846.9544644,630.71361077)(846.89446446,630.61861087)(846.83446777,630.52861328)
\curveto(846.66446469,630.26861122)(846.48946486,630.00861148)(846.30946777,629.74861328)
\curveto(846.13946521,629.49861199)(845.96446539,629.24861224)(845.78446777,628.99861328)
\curveto(845.73446562,628.91861257)(845.67946567,628.83861265)(845.61946777,628.75861328)
\lineto(845.46946777,628.51861328)
\curveto(845.4494659,628.488613)(845.42446593,628.45361303)(845.39446777,628.41361328)
\curveto(845.37446598,628.3836131)(845.349466,628.35861313)(845.31946777,628.33861328)
\curveto(845.21946613,628.26861322)(845.09946625,628.22861326)(844.95946777,628.21861328)
\lineto(844.50946777,628.21861328)
\lineto(844.28446777,628.21861328)
\curveto(844.21446714,628.21861327)(844.1544672,628.22861326)(844.10446777,628.24861328)
\curveto(844.07446728,628.26861322)(844.0494673,628.2836132)(844.02946777,628.29361328)
\curveto(844.01946733,628.31361317)(844.00446735,628.33361315)(843.98446777,628.35361328)
\curveto(843.97446738,628.46361302)(843.98946736,628.54861294)(844.02946777,628.60861328)
\curveto(844.07946727,628.66861282)(844.12946722,628.73361275)(844.17946777,628.80361328)
\curveto(844.25946709,628.91361257)(844.33446702,629.01361247)(844.40446777,629.10361328)
\curveto(844.47446688,629.20361228)(844.54446681,629.30861218)(844.61446777,629.41861328)
\curveto(844.83446652,629.71861177)(845.0494663,630.01861147)(845.25946777,630.31861328)
\lineto(845.88946777,631.21861328)
\curveto(845.95946539,631.30861018)(846.02446533,631.39861009)(846.08446777,631.48861328)
\curveto(846.1544652,631.57860991)(846.21946513,631.67360981)(846.27946777,631.77361328)
\curveto(846.32946502,631.84360964)(846.37946497,631.90860958)(846.42946777,631.96861328)
\curveto(846.47946487,632.03860945)(846.51446484,632.12860936)(846.53446777,632.23861328)
\curveto(846.5544648,632.2886092)(846.5494648,632.33860915)(846.51946777,632.38861328)
\curveto(846.49946485,632.43860905)(846.47946487,632.47860901)(846.45946777,632.50861328)
\curveto(846.40946494,632.59860889)(846.354465,632.6836088)(846.29446777,632.76361328)
\lineto(846.11446777,633.00361328)
\curveto(845.88446547,633.32360816)(845.6494657,633.64360784)(845.40946777,633.96361328)
\lineto(844.71946777,634.92361328)
\curveto(844.63946671,635.03360645)(844.55946679,635.13360635)(844.47946777,635.22361328)
\curveto(844.40946694,635.31360617)(844.33946701,635.41360607)(844.26946777,635.52361328)
\curveto(844.2494671,635.55360593)(844.22946712,635.59360589)(844.20946777,635.64361328)
\curveto(844.18946716,635.70360578)(844.18946716,635.75360573)(844.20946777,635.79361328)
\curveto(844.22946712,635.84360564)(844.25946709,635.87360561)(844.29946777,635.88361328)
\curveto(844.33946701,635.90360558)(844.38446697,635.91860557)(844.43446777,635.92861328)
\curveto(844.4544669,635.93860555)(844.46946688,635.93860555)(844.47946777,635.92861328)
\curveto(844.48946686,635.92860556)(844.49946685,635.93360555)(844.50946777,635.94361328)
}
}
{
\newrgbcolor{curcolor}{0 0 0}
\pscustom[linestyle=none,fillstyle=solid,fillcolor=curcolor]
{
\newpath
\moveto(852.54313965,637.44361328)
\curveto(852.46313853,637.50360398)(852.41813857,637.60860388)(852.40813965,637.75861328)
\lineto(852.40813965,638.22361328)
\lineto(852.40813965,638.47861328)
\curveto(852.40813858,638.56860292)(852.42313857,638.64360284)(852.45313965,638.70361328)
\curveto(852.4931385,638.7836027)(852.57313842,638.84360264)(852.69313965,638.88361328)
\curveto(852.71313828,638.89360259)(852.73313826,638.89360259)(852.75313965,638.88361328)
\curveto(852.78313821,638.8836026)(852.80813818,638.8886026)(852.82813965,638.89861328)
\curveto(852.99813799,638.89860259)(853.15813783,638.89360259)(853.30813965,638.88361328)
\curveto(853.45813753,638.87360261)(853.55813743,638.81360267)(853.60813965,638.70361328)
\curveto(853.63813735,638.64360284)(853.65313734,638.56860292)(853.65313965,638.47861328)
\lineto(853.65313965,638.22361328)
\curveto(853.65313734,638.04360344)(853.64813734,637.87360361)(853.63813965,637.71361328)
\curveto(853.63813735,637.55360393)(853.57313742,637.44860404)(853.44313965,637.39861328)
\curveto(853.3931376,637.37860411)(853.33813765,637.36860412)(853.27813965,637.36861328)
\lineto(853.11313965,637.36861328)
\lineto(852.79813965,637.36861328)
\curveto(852.69813829,637.36860412)(852.61313838,637.39360409)(852.54313965,637.44361328)
\moveto(853.65313965,628.93861328)
\lineto(853.65313965,628.62361328)
\curveto(853.66313733,628.52361296)(853.64313735,628.44361304)(853.59313965,628.38361328)
\curveto(853.56313743,628.32361316)(853.51813747,628.2836132)(853.45813965,628.26361328)
\curveto(853.39813759,628.25361323)(853.32813766,628.23861325)(853.24813965,628.21861328)
\lineto(853.02313965,628.21861328)
\curveto(852.8931381,628.21861327)(852.77813821,628.22361326)(852.67813965,628.23361328)
\curveto(852.5881384,628.25361323)(852.51813847,628.30361318)(852.46813965,628.38361328)
\curveto(852.42813856,628.44361304)(852.40813858,628.51861297)(852.40813965,628.60861328)
\lineto(852.40813965,628.89361328)
\lineto(852.40813965,635.23861328)
\lineto(852.40813965,635.55361328)
\curveto(852.40813858,635.66360582)(852.43313856,635.74860574)(852.48313965,635.80861328)
\curveto(852.51313848,635.85860563)(852.55313844,635.8886056)(852.60313965,635.89861328)
\curveto(852.65313834,635.90860558)(852.70813828,635.92360556)(852.76813965,635.94361328)
\curveto(852.7881382,635.94360554)(852.80813818,635.93860555)(852.82813965,635.92861328)
\curveto(852.85813813,635.92860556)(852.88313811,635.93360555)(852.90313965,635.94361328)
\curveto(853.03313796,635.94360554)(853.16313783,635.93860555)(853.29313965,635.92861328)
\curveto(853.43313756,635.92860556)(853.52813746,635.8886056)(853.57813965,635.80861328)
\curveto(853.62813736,635.74860574)(853.65313734,635.66860582)(853.65313965,635.56861328)
\lineto(853.65313965,635.28361328)
\lineto(853.65313965,628.93861328)
}
}
{
\newrgbcolor{curcolor}{0 0 0}
\pscustom[linestyle=none,fillstyle=solid,fillcolor=curcolor]
{
\newpath
\moveto(856.1679834,638.89861328)
\curveto(856.29798178,638.89860259)(856.43298165,638.89860259)(856.5729834,638.89861328)
\curveto(856.72298136,638.89860259)(856.83298125,638.86360262)(856.9029834,638.79361328)
\curveto(856.95298113,638.72360276)(856.9779811,638.62860286)(856.9779834,638.50861328)
\curveto(856.98798109,638.39860309)(856.99298109,638.2836032)(856.9929834,638.16361328)
\lineto(856.9929834,636.82861328)
\lineto(856.9929834,630.75361328)
\lineto(856.9929834,629.07361328)
\lineto(856.9929834,628.68361328)
\curveto(856.99298109,628.54361294)(856.96798111,628.43361305)(856.9179834,628.35361328)
\curveto(856.88798119,628.30361318)(856.84298124,628.27361321)(856.7829834,628.26361328)
\curveto(856.73298135,628.25361323)(856.66798141,628.23861325)(856.5879834,628.21861328)
\lineto(856.3779834,628.21861328)
\lineto(856.0629834,628.21861328)
\curveto(855.96298212,628.22861326)(855.88798219,628.26361322)(855.8379834,628.32361328)
\curveto(855.78798229,628.40361308)(855.75798232,628.50361298)(855.7479834,628.62361328)
\lineto(855.7479834,628.99861328)
\lineto(855.7479834,630.37861328)
\lineto(855.7479834,636.61861328)
\lineto(855.7479834,638.08861328)
\curveto(855.74798233,638.19860329)(855.74298234,638.31360317)(855.7329834,638.43361328)
\curveto(855.73298235,638.56360292)(855.75798232,638.66360282)(855.8079834,638.73361328)
\curveto(855.84798223,638.79360269)(855.92298216,638.84360264)(856.0329834,638.88361328)
\curveto(856.05298203,638.89360259)(856.07298201,638.89360259)(856.0929834,638.88361328)
\curveto(856.12298196,638.8836026)(856.14798193,638.8886026)(856.1679834,638.89861328)
}
}
{
\newrgbcolor{curcolor}{0 0 0}
\pscustom[linestyle=none,fillstyle=solid,fillcolor=curcolor]
{
\newpath
\moveto(859.22282715,637.44361328)
\curveto(859.14282603,637.50360398)(859.09782607,637.60860388)(859.08782715,637.75861328)
\lineto(859.08782715,638.22361328)
\lineto(859.08782715,638.47861328)
\curveto(859.08782608,638.56860292)(859.10282607,638.64360284)(859.13282715,638.70361328)
\curveto(859.172826,638.7836027)(859.25282592,638.84360264)(859.37282715,638.88361328)
\curveto(859.39282578,638.89360259)(859.41282576,638.89360259)(859.43282715,638.88361328)
\curveto(859.46282571,638.8836026)(859.48782568,638.8886026)(859.50782715,638.89861328)
\curveto(859.67782549,638.89860259)(859.83782533,638.89360259)(859.98782715,638.88361328)
\curveto(860.13782503,638.87360261)(860.23782493,638.81360267)(860.28782715,638.70361328)
\curveto(860.31782485,638.64360284)(860.33282484,638.56860292)(860.33282715,638.47861328)
\lineto(860.33282715,638.22361328)
\curveto(860.33282484,638.04360344)(860.32782484,637.87360361)(860.31782715,637.71361328)
\curveto(860.31782485,637.55360393)(860.25282492,637.44860404)(860.12282715,637.39861328)
\curveto(860.0728251,637.37860411)(860.01782515,637.36860412)(859.95782715,637.36861328)
\lineto(859.79282715,637.36861328)
\lineto(859.47782715,637.36861328)
\curveto(859.37782579,637.36860412)(859.29282588,637.39360409)(859.22282715,637.44361328)
\moveto(860.33282715,628.93861328)
\lineto(860.33282715,628.62361328)
\curveto(860.34282483,628.52361296)(860.32282485,628.44361304)(860.27282715,628.38361328)
\curveto(860.24282493,628.32361316)(860.19782497,628.2836132)(860.13782715,628.26361328)
\curveto(860.07782509,628.25361323)(860.00782516,628.23861325)(859.92782715,628.21861328)
\lineto(859.70282715,628.21861328)
\curveto(859.5728256,628.21861327)(859.45782571,628.22361326)(859.35782715,628.23361328)
\curveto(859.2678259,628.25361323)(859.19782597,628.30361318)(859.14782715,628.38361328)
\curveto(859.10782606,628.44361304)(859.08782608,628.51861297)(859.08782715,628.60861328)
\lineto(859.08782715,628.89361328)
\lineto(859.08782715,635.23861328)
\lineto(859.08782715,635.55361328)
\curveto(859.08782608,635.66360582)(859.11282606,635.74860574)(859.16282715,635.80861328)
\curveto(859.19282598,635.85860563)(859.23282594,635.8886056)(859.28282715,635.89861328)
\curveto(859.33282584,635.90860558)(859.38782578,635.92360556)(859.44782715,635.94361328)
\curveto(859.4678257,635.94360554)(859.48782568,635.93860555)(859.50782715,635.92861328)
\curveto(859.53782563,635.92860556)(859.56282561,635.93360555)(859.58282715,635.94361328)
\curveto(859.71282546,635.94360554)(859.84282533,635.93860555)(859.97282715,635.92861328)
\curveto(860.11282506,635.92860556)(860.20782496,635.8886056)(860.25782715,635.80861328)
\curveto(860.30782486,635.74860574)(860.33282484,635.66860582)(860.33282715,635.56861328)
\lineto(860.33282715,635.28361328)
\lineto(860.33282715,628.93861328)
}
}
{
\newrgbcolor{curcolor}{0 0 0}
\pscustom[linestyle=none,fillstyle=solid,fillcolor=curcolor]
{
\newpath
\moveto(869.1626709,628.77361328)
\curveto(869.19266307,628.61361287)(869.17766308,628.47861301)(869.1176709,628.36861328)
\curveto(869.0576632,628.26861322)(868.97766328,628.19361329)(868.8776709,628.14361328)
\curveto(868.82766343,628.12361336)(868.77266349,628.11361337)(868.7126709,628.11361328)
\curveto(868.6626636,628.11361337)(868.60766365,628.10361338)(868.5476709,628.08361328)
\curveto(868.32766393,628.03361345)(868.10766415,628.04861344)(867.8876709,628.12861328)
\curveto(867.67766458,628.19861329)(867.53266473,628.2886132)(867.4526709,628.39861328)
\curveto(867.40266486,628.46861302)(867.3576649,628.54861294)(867.3176709,628.63861328)
\curveto(867.27766498,628.73861275)(867.22766503,628.81861267)(867.1676709,628.87861328)
\curveto(867.14766511,628.89861259)(867.12266514,628.91861257)(867.0926709,628.93861328)
\curveto(867.07266519,628.95861253)(867.04266522,628.96361252)(867.0026709,628.95361328)
\curveto(866.89266537,628.92361256)(866.78766547,628.86861262)(866.6876709,628.78861328)
\curveto(866.59766566,628.70861278)(866.50766575,628.63861285)(866.4176709,628.57861328)
\curveto(866.28766597,628.49861299)(866.14766611,628.42361306)(865.9976709,628.35361328)
\curveto(865.84766641,628.29361319)(865.68766657,628.23861325)(865.5176709,628.18861328)
\curveto(865.41766684,628.15861333)(865.30766695,628.13861335)(865.1876709,628.12861328)
\curveto(865.07766718,628.11861337)(864.96766729,628.10361338)(864.8576709,628.08361328)
\curveto(864.80766745,628.07361341)(864.7626675,628.06861342)(864.7226709,628.06861328)
\lineto(864.6176709,628.06861328)
\curveto(864.50766775,628.04861344)(864.40266786,628.04861344)(864.3026709,628.06861328)
\lineto(864.1676709,628.06861328)
\curveto(864.11766814,628.07861341)(864.06766819,628.0836134)(864.0176709,628.08361328)
\curveto(863.96766829,628.0836134)(863.92266834,628.09361339)(863.8826709,628.11361328)
\curveto(863.84266842,628.12361336)(863.80766845,628.12861336)(863.7776709,628.12861328)
\curveto(863.7576685,628.11861337)(863.73266853,628.11861337)(863.7026709,628.12861328)
\lineto(863.4626709,628.18861328)
\curveto(863.38266888,628.19861329)(863.30766895,628.21861327)(863.2376709,628.24861328)
\curveto(862.93766932,628.37861311)(862.69266957,628.52361296)(862.5026709,628.68361328)
\curveto(862.32266994,628.85361263)(862.17267009,629.0886124)(862.0526709,629.38861328)
\curveto(861.9626703,629.60861188)(861.91767034,629.87361161)(861.9176709,630.18361328)
\lineto(861.9176709,630.49861328)
\curveto(861.92767033,630.54861094)(861.93267033,630.59861089)(861.9326709,630.64861328)
\lineto(861.9626709,630.82861328)
\lineto(862.0826709,631.15861328)
\curveto(862.12267014,631.26861022)(862.17267009,631.36861012)(862.2326709,631.45861328)
\curveto(862.41266985,631.74860974)(862.6576696,631.96360952)(862.9676709,632.10361328)
\curveto(863.27766898,632.24360924)(863.61766864,632.36860912)(863.9876709,632.47861328)
\curveto(864.12766813,632.51860897)(864.27266799,632.54860894)(864.4226709,632.56861328)
\curveto(864.57266769,632.5886089)(864.72266754,632.61360887)(864.8726709,632.64361328)
\curveto(864.94266732,632.66360882)(865.00766725,632.67360881)(865.0676709,632.67361328)
\curveto(865.13766712,632.67360881)(865.21266705,632.6836088)(865.2926709,632.70361328)
\curveto(865.3626669,632.72360876)(865.43266683,632.73360875)(865.5026709,632.73361328)
\curveto(865.57266669,632.74360874)(865.64766661,632.75860873)(865.7276709,632.77861328)
\curveto(865.97766628,632.83860865)(866.21266605,632.8886086)(866.4326709,632.92861328)
\curveto(866.65266561,632.97860851)(866.82766543,633.09360839)(866.9576709,633.27361328)
\curveto(867.01766524,633.35360813)(867.06766519,633.45360803)(867.1076709,633.57361328)
\curveto(867.14766511,633.70360778)(867.14766511,633.84360764)(867.1076709,633.99361328)
\curveto(867.04766521,634.23360725)(866.9576653,634.42360706)(866.8376709,634.56361328)
\curveto(866.72766553,634.70360678)(866.56766569,634.81360667)(866.3576709,634.89361328)
\curveto(866.23766602,634.94360654)(866.09266617,634.97860651)(865.9226709,634.99861328)
\curveto(865.7626665,635.01860647)(865.59266667,635.02860646)(865.4126709,635.02861328)
\curveto(865.23266703,635.02860646)(865.0576672,635.01860647)(864.8876709,634.99861328)
\curveto(864.71766754,634.97860651)(864.57266769,634.94860654)(864.4526709,634.90861328)
\curveto(864.28266798,634.84860664)(864.11766814,634.76360672)(863.9576709,634.65361328)
\curveto(863.87766838,634.59360689)(863.80266846,634.51360697)(863.7326709,634.41361328)
\curveto(863.67266859,634.32360716)(863.61766864,634.22360726)(863.5676709,634.11361328)
\curveto(863.53766872,634.03360745)(863.50766875,633.94860754)(863.4776709,633.85861328)
\curveto(863.4576688,633.76860772)(863.41266885,633.69860779)(863.3426709,633.64861328)
\curveto(863.30266896,633.61860787)(863.23266903,633.59360789)(863.1326709,633.57361328)
\curveto(863.04266922,633.56360792)(862.94766931,633.55860793)(862.8476709,633.55861328)
\curveto(862.74766951,633.55860793)(862.64766961,633.56360792)(862.5476709,633.57361328)
\curveto(862.4576698,633.59360789)(862.39266987,633.61860787)(862.3526709,633.64861328)
\curveto(862.31266995,633.67860781)(862.28266998,633.72860776)(862.2626709,633.79861328)
\curveto(862.24267002,633.86860762)(862.24267002,633.94360754)(862.2626709,634.02361328)
\curveto(862.29266997,634.15360733)(862.32266994,634.27360721)(862.3526709,634.38361328)
\curveto(862.39266987,634.50360698)(862.43766982,634.61860687)(862.4876709,634.72861328)
\curveto(862.67766958,635.07860641)(862.91766934,635.34860614)(863.2076709,635.53861328)
\curveto(863.49766876,635.73860575)(863.8576684,635.89860559)(864.2876709,636.01861328)
\curveto(864.38766787,636.03860545)(864.48766777,636.05360543)(864.5876709,636.06361328)
\curveto(864.69766756,636.07360541)(864.80766745,636.0886054)(864.9176709,636.10861328)
\curveto(864.9576673,636.11860537)(865.02266724,636.11860537)(865.1126709,636.10861328)
\curveto(865.20266706,636.10860538)(865.257667,636.11860537)(865.2776709,636.13861328)
\curveto(865.97766628,636.14860534)(866.58766567,636.06860542)(867.1076709,635.89861328)
\curveto(867.62766463,635.72860576)(867.99266427,635.40360608)(868.2026709,634.92361328)
\curveto(868.29266397,634.72360676)(868.34266392,634.488607)(868.3526709,634.21861328)
\curveto(868.37266389,633.95860753)(868.38266388,633.6836078)(868.3826709,633.39361328)
\lineto(868.3826709,630.07861328)
\curveto(868.38266388,629.93861155)(868.38766387,629.80361168)(868.3976709,629.67361328)
\curveto(868.40766385,629.54361194)(868.43766382,629.43861205)(868.4876709,629.35861328)
\curveto(868.53766372,629.2886122)(868.60266366,629.23861225)(868.6826709,629.20861328)
\curveto(868.77266349,629.16861232)(868.8576634,629.13861235)(868.9376709,629.11861328)
\curveto(869.01766324,629.10861238)(869.07766318,629.06361242)(869.1176709,628.98361328)
\curveto(869.13766312,628.95361253)(869.14766311,628.92361256)(869.1476709,628.89361328)
\curveto(869.14766311,628.86361262)(869.15266311,628.82361266)(869.1626709,628.77361328)
\moveto(867.0176709,630.43861328)
\curveto(867.07766518,630.57861091)(867.10766515,630.73861075)(867.1076709,630.91861328)
\curveto(867.11766514,631.10861038)(867.12266514,631.30361018)(867.1226709,631.50361328)
\curveto(867.12266514,631.61360987)(867.11766514,631.71360977)(867.1076709,631.80361328)
\curveto(867.09766516,631.89360959)(867.0576652,631.96360952)(866.9876709,632.01361328)
\curveto(866.9576653,632.03360945)(866.88766537,632.04360944)(866.7776709,632.04361328)
\curveto(866.7576655,632.02360946)(866.72266554,632.01360947)(866.6726709,632.01361328)
\curveto(866.62266564,632.01360947)(866.57766568,632.00360948)(866.5376709,631.98361328)
\curveto(866.4576658,631.96360952)(866.36766589,631.94360954)(866.2676709,631.92361328)
\lineto(865.9676709,631.86361328)
\curveto(865.93766632,631.86360962)(865.90266636,631.85860963)(865.8626709,631.84861328)
\lineto(865.7576709,631.84861328)
\curveto(865.60766665,631.80860968)(865.44266682,631.7836097)(865.2626709,631.77361328)
\curveto(865.09266717,631.77360971)(864.93266733,631.75360973)(864.7826709,631.71361328)
\curveto(864.70266756,631.69360979)(864.62766763,631.67360981)(864.5576709,631.65361328)
\curveto(864.49766776,631.64360984)(864.42766783,631.62860986)(864.3476709,631.60861328)
\curveto(864.18766807,631.55860993)(864.03766822,631.49360999)(863.8976709,631.41361328)
\curveto(863.7576685,631.34361014)(863.63766862,631.25361023)(863.5376709,631.14361328)
\curveto(863.43766882,631.03361045)(863.3626689,630.89861059)(863.3126709,630.73861328)
\curveto(863.262669,630.5886109)(863.24266902,630.40361108)(863.2526709,630.18361328)
\curveto(863.25266901,630.0836114)(863.26766899,629.9886115)(863.2976709,629.89861328)
\curveto(863.33766892,629.81861167)(863.38266888,629.74361174)(863.4326709,629.67361328)
\curveto(863.51266875,629.56361192)(863.61766864,629.46861202)(863.7476709,629.38861328)
\curveto(863.87766838,629.31861217)(864.01766824,629.25861223)(864.1676709,629.20861328)
\curveto(864.21766804,629.19861229)(864.26766799,629.19361229)(864.3176709,629.19361328)
\curveto(864.36766789,629.19361229)(864.41766784,629.1886123)(864.4676709,629.17861328)
\curveto(864.53766772,629.15861233)(864.62266764,629.14361234)(864.7226709,629.13361328)
\curveto(864.83266743,629.13361235)(864.92266734,629.14361234)(864.9926709,629.16361328)
\curveto(865.05266721,629.1836123)(865.11266715,629.1886123)(865.1726709,629.17861328)
\curveto(865.23266703,629.17861231)(865.29266697,629.1886123)(865.3526709,629.20861328)
\curveto(865.43266683,629.22861226)(865.50766675,629.24361224)(865.5776709,629.25361328)
\curveto(865.6576666,629.26361222)(865.73266653,629.2836122)(865.8026709,629.31361328)
\curveto(866.09266617,629.43361205)(866.33766592,629.57861191)(866.5376709,629.74861328)
\curveto(866.74766551,629.91861157)(866.90766535,630.14861134)(867.0176709,630.43861328)
}
}
{
\newrgbcolor{curcolor}{0 0 0}
\pscustom[linestyle=none,fillstyle=solid,fillcolor=curcolor]
{
\newpath
\moveto(873.97931152,636.12361328)
\curveto(874.20930673,636.12360536)(874.3393066,636.06360542)(874.36931152,635.94361328)
\curveto(874.39930654,635.83360565)(874.41430653,635.66860582)(874.41431152,635.44861328)
\lineto(874.41431152,635.16361328)
\curveto(874.41430653,635.07360641)(874.38930655,634.99860649)(874.33931152,634.93861328)
\curveto(874.27930666,634.85860663)(874.19430675,634.81360667)(874.08431152,634.80361328)
\curveto(873.97430697,634.80360668)(873.86430708,634.7886067)(873.75431152,634.75861328)
\curveto(873.61430733,634.72860676)(873.47930746,634.69860679)(873.34931152,634.66861328)
\curveto(873.22930771,634.63860685)(873.11430783,634.59860689)(873.00431152,634.54861328)
\curveto(872.71430823,634.41860707)(872.47930846,634.23860725)(872.29931152,634.00861328)
\curveto(872.11930882,633.7886077)(871.96430898,633.53360795)(871.83431152,633.24361328)
\curveto(871.79430915,633.13360835)(871.76430918,633.01860847)(871.74431152,632.89861328)
\curveto(871.72430922,632.7886087)(871.69930924,632.67360881)(871.66931152,632.55361328)
\curveto(871.65930928,632.50360898)(871.65430929,632.45360903)(871.65431152,632.40361328)
\curveto(871.66430928,632.35360913)(871.66430928,632.30360918)(871.65431152,632.25361328)
\curveto(871.62430932,632.13360935)(871.60930933,631.99360949)(871.60931152,631.83361328)
\curveto(871.61930932,631.6836098)(871.62430932,631.53860995)(871.62431152,631.39861328)
\lineto(871.62431152,629.55361328)
\lineto(871.62431152,629.20861328)
\curveto(871.62430932,629.0886124)(871.61930932,628.97361251)(871.60931152,628.86361328)
\curveto(871.59930934,628.75361273)(871.59430935,628.65861283)(871.59431152,628.57861328)
\curveto(871.60430934,628.49861299)(871.58430936,628.42861306)(871.53431152,628.36861328)
\curveto(871.48430946,628.29861319)(871.40430954,628.25861323)(871.29431152,628.24861328)
\curveto(871.19430975,628.23861325)(871.08430986,628.23361325)(870.96431152,628.23361328)
\lineto(870.69431152,628.23361328)
\curveto(870.6443103,628.25361323)(870.59431035,628.26861322)(870.54431152,628.27861328)
\curveto(870.50431044,628.29861319)(870.47431047,628.32361316)(870.45431152,628.35361328)
\curveto(870.40431054,628.42361306)(870.37431057,628.50861298)(870.36431152,628.60861328)
\lineto(870.36431152,628.93861328)
\lineto(870.36431152,630.09361328)
\lineto(870.36431152,634.24861328)
\lineto(870.36431152,635.28361328)
\lineto(870.36431152,635.58361328)
\curveto(870.37431057,635.6836058)(870.40431054,635.76860572)(870.45431152,635.83861328)
\curveto(870.48431046,635.87860561)(870.53431041,635.90860558)(870.60431152,635.92861328)
\curveto(870.68431026,635.94860554)(870.76931017,635.95860553)(870.85931152,635.95861328)
\curveto(870.94930999,635.96860552)(871.0393099,635.96860552)(871.12931152,635.95861328)
\curveto(871.21930972,635.94860554)(871.28930965,635.93360555)(871.33931152,635.91361328)
\curveto(871.41930952,635.8836056)(871.46930947,635.82360566)(871.48931152,635.73361328)
\curveto(871.51930942,635.65360583)(871.53430941,635.56360592)(871.53431152,635.46361328)
\lineto(871.53431152,635.16361328)
\curveto(871.53430941,635.06360642)(871.55430939,634.97360651)(871.59431152,634.89361328)
\curveto(871.60430934,634.87360661)(871.61430933,634.85860663)(871.62431152,634.84861328)
\lineto(871.66931152,634.80361328)
\curveto(871.77930916,634.80360668)(871.86930907,634.84860664)(871.93931152,634.93861328)
\curveto(872.00930893,635.03860645)(872.06930887,635.11860637)(872.11931152,635.17861328)
\lineto(872.20931152,635.26861328)
\curveto(872.29930864,635.37860611)(872.42430852,635.49360599)(872.58431152,635.61361328)
\curveto(872.7443082,635.73360575)(872.89430805,635.82360566)(873.03431152,635.88361328)
\curveto(873.12430782,635.93360555)(873.21930772,635.96860552)(873.31931152,635.98861328)
\curveto(873.41930752,636.01860547)(873.52430742,636.04860544)(873.63431152,636.07861328)
\curveto(873.69430725,636.0886054)(873.75430719,636.09360539)(873.81431152,636.09361328)
\curveto(873.87430707,636.10360538)(873.92930701,636.11360537)(873.97931152,636.12361328)
}
}
{
\newrgbcolor{curcolor}{0.7019608 0.7019608 0.7019608}
\pscustom[linestyle=none,fillstyle=solid,fillcolor=curcolor]
{
\newpath
\moveto(807.09008789,638.9286499)
\lineto(822.09008789,638.9286499)
\lineto(822.09008789,623.9286499)
\lineto(807.09008789,623.9286499)
\closepath
}
}
{
\newrgbcolor{curcolor}{0 0 0}
\pscustom[linestyle=none,fillstyle=solid,fillcolor=curcolor]
{
\newpath
\moveto(835.4632959,610.77784668)
\lineto(835.4632959,610.50784668)
\curveto(835.47328593,610.41784143)(835.46828593,610.33784151)(835.4482959,610.26784668)
\lineto(835.4482959,610.11784668)
\curveto(835.43828596,610.08784176)(835.43328597,610.05284179)(835.4332959,610.01284668)
\curveto(835.44328596,609.97284187)(835.44328596,609.9428419)(835.4332959,609.92284668)
\curveto(835.42328598,609.87284197)(835.41828598,609.81784203)(835.4182959,609.75784668)
\curveto(835.41828598,609.70784214)(835.41328599,609.65784219)(835.4032959,609.60784668)
\curveto(835.37328603,609.46784238)(835.35328605,609.31784253)(835.3432959,609.15784668)
\curveto(835.33328607,609.00784284)(835.3032861,608.86284298)(835.2532959,608.72284668)
\curveto(835.22328618,608.60284324)(835.18828621,608.47784337)(835.1482959,608.34784668)
\curveto(835.11828628,608.22784362)(835.07828632,608.10784374)(835.0282959,607.98784668)
\curveto(834.85828654,607.55784429)(834.64328676,607.16784468)(834.3832959,606.81784668)
\curveto(834.13328727,606.47784537)(833.81828758,606.18784566)(833.4382959,605.94784668)
\curveto(833.24828815,605.82784602)(833.04328836,605.72284612)(832.8232959,605.63284668)
\curveto(832.61328879,605.55284629)(832.38328902,605.47284637)(832.1332959,605.39284668)
\curveto(832.02328938,605.35284649)(831.9032895,605.32284652)(831.7732959,605.30284668)
\curveto(831.65328975,605.29284655)(831.53328987,605.27284657)(831.4132959,605.24284668)
\curveto(831.3032901,605.22284662)(831.19329021,605.21284663)(831.0832959,605.21284668)
\curveto(830.98329042,605.21284663)(830.88329052,605.20284664)(830.7832959,605.18284668)
\lineto(830.5732959,605.18284668)
\curveto(830.54329086,605.17284667)(830.50829089,605.16784668)(830.4682959,605.16784668)
\curveto(830.42829097,605.17784667)(830.38829101,605.18284666)(830.3482959,605.18284668)
\lineto(827.3482959,605.18284668)
\curveto(827.1982942,605.18284666)(827.06329434,605.18784666)(826.9432959,605.19784668)
\curveto(826.83329457,605.21784663)(826.75829464,605.28284656)(826.7182959,605.39284668)
\curveto(826.67829472,605.47284637)(826.65829474,605.58784626)(826.6582959,605.73784668)
\curveto(826.66829473,605.88784596)(826.67329473,606.02284582)(826.6732959,606.14284668)
\lineto(826.6732959,615.00784668)
\curveto(826.67329473,615.12783672)(826.66829473,615.25283659)(826.6582959,615.38284668)
\curveto(826.65829474,615.52283632)(826.68329472,615.63283621)(826.7332959,615.71284668)
\curveto(826.77329463,615.78283606)(826.84829455,615.82783602)(826.9582959,615.84784668)
\curveto(826.97829442,615.85783599)(826.9982944,615.85783599)(827.0182959,615.84784668)
\curveto(827.03829436,615.847836)(827.05829434,615.85283599)(827.0782959,615.86284668)
\lineto(830.3332959,615.86284668)
\curveto(830.38329102,615.86283598)(830.42829097,615.86283598)(830.4682959,615.86284668)
\curveto(830.51829088,615.87283597)(830.56329084,615.87283597)(830.6032959,615.86284668)
\curveto(830.65329075,615.842836)(830.7032907,615.83783601)(830.7532959,615.84784668)
\curveto(830.81329059,615.85783599)(830.86829053,615.85783599)(830.9182959,615.84784668)
\curveto(830.96829043,615.83783601)(831.02329038,615.83283601)(831.0832959,615.83284668)
\curveto(831.14329026,615.83283601)(831.1982902,615.82783602)(831.2482959,615.81784668)
\curveto(831.2982901,615.80783604)(831.34329006,615.80283604)(831.3832959,615.80284668)
\curveto(831.43328997,615.80283604)(831.48328992,615.79783605)(831.5332959,615.78784668)
\curveto(831.64328976,615.76783608)(831.74828965,615.7478361)(831.8482959,615.72784668)
\curveto(831.94828945,615.71783613)(832.04828935,615.69783615)(832.1482959,615.66784668)
\curveto(832.36828903,615.59783625)(832.57828882,615.52783632)(832.7782959,615.45784668)
\curveto(832.97828842,615.39783645)(833.16328824,615.31283653)(833.3332959,615.20284668)
\curveto(833.47328793,615.12283672)(833.5982878,615.0428368)(833.7082959,614.96284668)
\curveto(833.73828766,614.9428369)(833.76828763,614.91783693)(833.7982959,614.88784668)
\curveto(833.82828757,614.86783698)(833.85828754,614.847837)(833.8882959,614.82784668)
\curveto(833.94828745,614.77783707)(834.0032874,614.72783712)(834.0532959,614.67784668)
\curveto(834.1032873,614.62783722)(834.15328725,614.57783727)(834.2032959,614.52784668)
\curveto(834.25328715,614.47783737)(834.29328711,614.4428374)(834.3232959,614.42284668)
\curveto(834.36328704,614.36283748)(834.403287,614.30783754)(834.4432959,614.25784668)
\curveto(834.49328691,614.20783764)(834.53828686,614.15283769)(834.5782959,614.09284668)
\curveto(834.62828677,614.03283781)(834.66828673,613.96783788)(834.6982959,613.89784668)
\curveto(834.73828666,613.83783801)(834.78328662,613.77283807)(834.8332959,613.70284668)
\curveto(834.85328655,613.66283818)(834.86828653,613.62783822)(834.8782959,613.59784668)
\curveto(834.88828651,613.56783828)(834.9032865,613.53283831)(834.9232959,613.49284668)
\curveto(834.96328644,613.41283843)(834.9982864,613.33283851)(835.0282959,613.25284668)
\curveto(835.05828634,613.18283866)(835.09328631,613.10783874)(835.1332959,613.02784668)
\curveto(835.17328623,612.91783893)(835.2032862,612.80283904)(835.2232959,612.68284668)
\curveto(835.25328615,612.57283927)(835.28328612,612.46283938)(835.3132959,612.35284668)
\curveto(835.33328607,612.29283955)(835.34328606,612.23283961)(835.3432959,612.17284668)
\curveto(835.34328606,612.12283972)(835.35328605,612.06783978)(835.3732959,612.00784668)
\curveto(835.42328598,611.82784002)(835.44828595,611.62784022)(835.4482959,611.40784668)
\curveto(835.45828594,611.19784065)(835.46328594,610.98784086)(835.4632959,610.77784668)
\moveto(834.0382959,609.99784668)
\curveto(834.05828734,610.09784175)(834.06828733,610.20284164)(834.0682959,610.31284668)
\lineto(834.0682959,610.65784668)
\lineto(834.0682959,610.88284668)
\curveto(834.07828732,610.96284088)(834.07328733,611.03784081)(834.0532959,611.10784668)
\curveto(834.05328735,611.13784071)(834.04828735,611.16784068)(834.0382959,611.19784668)
\lineto(834.0382959,611.30284668)
\curveto(834.01828738,611.41284043)(834.0032874,611.52284032)(833.9932959,611.63284668)
\curveto(833.99328741,611.7428401)(833.97828742,611.85283999)(833.9482959,611.96284668)
\curveto(833.92828747,612.0428398)(833.90828749,612.11783973)(833.8882959,612.18784668)
\curveto(833.87828752,612.26783958)(833.86328754,612.3478395)(833.8432959,612.42784668)
\curveto(833.73328767,612.78783906)(833.59328781,613.10283874)(833.4232959,613.37284668)
\curveto(833.14328826,613.82283802)(832.72828867,614.16283768)(832.1782959,614.39284668)
\curveto(832.08828931,614.4428374)(831.99328941,614.47783737)(831.8932959,614.49784668)
\curveto(831.79328961,614.52783732)(831.68828971,614.55783729)(831.5782959,614.58784668)
\curveto(831.46828993,614.61783723)(831.35329005,614.63283721)(831.2332959,614.63284668)
\curveto(831.12329028,614.6428372)(831.01329039,614.65783719)(830.9032959,614.67784668)
\lineto(830.5882959,614.67784668)
\curveto(830.55829084,614.68783716)(830.52329088,614.69283715)(830.4832959,614.69284668)
\lineto(830.3632959,614.69284668)
\lineto(828.5332959,614.69284668)
\curveto(828.51329289,614.68283716)(828.48829291,614.67783717)(828.4582959,614.67784668)
\curveto(828.42829297,614.68783716)(828.403293,614.68783716)(828.3832959,614.67784668)
\lineto(828.2332959,614.61784668)
\curveto(828.19329321,614.59783725)(828.16329324,614.56783728)(828.1432959,614.52784668)
\curveto(828.12329328,614.48783736)(828.1032933,614.41783743)(828.0832959,614.31784668)
\lineto(828.0832959,614.19784668)
\curveto(828.07329333,614.15783769)(828.06829333,614.11283773)(828.0682959,614.06284668)
\lineto(828.0682959,613.92784668)
\lineto(828.0682959,607.11784668)
\lineto(828.0682959,606.96784668)
\curveto(828.06829333,606.92784492)(828.07329333,606.88784496)(828.0832959,606.84784668)
\lineto(828.0832959,606.72784668)
\curveto(828.1032933,606.62784522)(828.12329328,606.55784529)(828.1432959,606.51784668)
\curveto(828.22329318,606.39784545)(828.37329303,606.33784551)(828.5932959,606.33784668)
\curveto(828.81329259,606.3478455)(829.02329238,606.35284549)(829.2232959,606.35284668)
\lineto(830.0932959,606.35284668)
\curveto(830.16329124,606.35284549)(830.23829116,606.3478455)(830.3182959,606.33784668)
\curveto(830.398291,606.33784551)(830.46829093,606.3478455)(830.5282959,606.36784668)
\lineto(830.6932959,606.36784668)
\curveto(830.74329066,606.37784547)(830.7982906,606.37784547)(830.8582959,606.36784668)
\curveto(830.91829048,606.36784548)(830.97829042,606.37284547)(831.0382959,606.38284668)
\curveto(831.0982903,606.40284544)(831.15829024,606.41284543)(831.2182959,606.41284668)
\curveto(831.27829012,606.42284542)(831.34329006,606.43784541)(831.4132959,606.45784668)
\curveto(831.52328988,606.48784536)(831.62828977,606.51784533)(831.7282959,606.54784668)
\curveto(831.83828956,606.57784527)(831.94828945,606.61784523)(832.0582959,606.66784668)
\curveto(832.42828897,606.82784502)(832.74328866,607.0428448)(833.0032959,607.31284668)
\curveto(833.27328813,607.59284425)(833.49328791,607.92284392)(833.6632959,608.30284668)
\curveto(833.71328769,608.41284343)(833.75328765,608.52784332)(833.7832959,608.64784668)
\lineto(833.9032959,609.03784668)
\curveto(833.93328747,609.1478427)(833.95328745,609.26284258)(833.9632959,609.38284668)
\curveto(833.98328742,609.51284233)(834.0032874,609.63784221)(834.0232959,609.75784668)
\curveto(834.03328737,609.80784204)(834.03828736,609.847842)(834.0382959,609.87784668)
\lineto(834.0382959,609.99784668)
}
}
{
\newrgbcolor{curcolor}{0 0 0}
\pscustom[linestyle=none,fillstyle=solid,fillcolor=curcolor]
{
\newpath
\moveto(844.0901709,609.38284668)
\curveto(844.11016284,609.32284252)(844.12016283,609.22784262)(844.1201709,609.09784668)
\curveto(844.12016283,608.97784287)(844.11516283,608.89284295)(844.1051709,608.84284668)
\lineto(844.1051709,608.69284668)
\curveto(844.09516285,608.61284323)(844.08516286,608.53784331)(844.0751709,608.46784668)
\curveto(844.07516287,608.40784344)(844.07016288,608.33784351)(844.0601709,608.25784668)
\curveto(844.04016291,608.19784365)(844.02516292,608.13784371)(844.0151709,608.07784668)
\curveto(844.01516293,608.01784383)(844.00516294,607.95784389)(843.9851709,607.89784668)
\curveto(843.945163,607.76784408)(843.91016304,607.63784421)(843.8801709,607.50784668)
\curveto(843.8501631,607.37784447)(843.81016314,607.25784459)(843.7601709,607.14784668)
\curveto(843.5501634,606.66784518)(843.27016368,606.26284558)(842.9201709,605.93284668)
\curveto(842.57016438,605.61284623)(842.14016481,605.36784648)(841.6301709,605.19784668)
\curveto(841.52016543,605.15784669)(841.40016555,605.12784672)(841.2701709,605.10784668)
\curveto(841.1501658,605.08784676)(841.02516592,605.06784678)(840.8951709,605.04784668)
\curveto(840.83516611,605.03784681)(840.77016618,605.03284681)(840.7001709,605.03284668)
\curveto(840.64016631,605.02284682)(840.58016637,605.01784683)(840.5201709,605.01784668)
\curveto(840.48016647,605.00784684)(840.42016653,605.00284684)(840.3401709,605.00284668)
\curveto(840.27016668,605.00284684)(840.22016673,605.00784684)(840.1901709,605.01784668)
\curveto(840.1501668,605.02784682)(840.11016684,605.03284681)(840.0701709,605.03284668)
\curveto(840.03016692,605.02284682)(839.99516695,605.02284682)(839.9651709,605.03284668)
\lineto(839.8751709,605.03284668)
\lineto(839.5151709,605.07784668)
\curveto(839.37516757,605.11784673)(839.24016771,605.15784669)(839.1101709,605.19784668)
\curveto(838.98016797,605.23784661)(838.85516809,605.28284656)(838.7351709,605.33284668)
\curveto(838.28516866,605.53284631)(837.91516903,605.79284605)(837.6251709,606.11284668)
\curveto(837.33516961,606.43284541)(837.09516985,606.82284502)(836.9051709,607.28284668)
\curveto(836.85517009,607.38284446)(836.81517013,607.48284436)(836.7851709,607.58284668)
\curveto(836.76517018,607.68284416)(836.7451702,607.78784406)(836.7251709,607.89784668)
\curveto(836.70517024,607.93784391)(836.69517025,607.96784388)(836.6951709,607.98784668)
\curveto(836.70517024,608.01784383)(836.70517024,608.05284379)(836.6951709,608.09284668)
\curveto(836.67517027,608.17284367)(836.66017029,608.25284359)(836.6501709,608.33284668)
\curveto(836.6501703,608.42284342)(836.64017031,608.50784334)(836.6201709,608.58784668)
\lineto(836.6201709,608.70784668)
\curveto(836.62017033,608.7478431)(836.61517033,608.79284305)(836.6051709,608.84284668)
\curveto(836.59517035,608.89284295)(836.59017036,608.97784287)(836.5901709,609.09784668)
\curveto(836.59017036,609.22784262)(836.60017035,609.32284252)(836.6201709,609.38284668)
\curveto(836.64017031,609.45284239)(836.6451703,609.52284232)(836.6351709,609.59284668)
\curveto(836.62517032,609.66284218)(836.63017032,609.73284211)(836.6501709,609.80284668)
\curveto(836.66017029,609.85284199)(836.66517028,609.89284195)(836.6651709,609.92284668)
\curveto(836.67517027,609.96284188)(836.68517026,610.00784184)(836.6951709,610.05784668)
\curveto(836.72517022,610.17784167)(836.7501702,610.29784155)(836.7701709,610.41784668)
\curveto(836.80017015,610.53784131)(836.84017011,610.65284119)(836.8901709,610.76284668)
\curveto(837.04016991,611.13284071)(837.22016973,611.46284038)(837.4301709,611.75284668)
\curveto(837.6501693,612.05283979)(837.91516903,612.30283954)(838.2251709,612.50284668)
\curveto(838.3451686,612.58283926)(838.47016848,612.6478392)(838.6001709,612.69784668)
\curveto(838.73016822,612.75783909)(838.86516808,612.81783903)(839.0051709,612.87784668)
\curveto(839.12516782,612.92783892)(839.25516769,612.95783889)(839.3951709,612.96784668)
\curveto(839.53516741,612.98783886)(839.67516727,613.01783883)(839.8151709,613.05784668)
\lineto(840.0101709,613.05784668)
\curveto(840.08016687,613.06783878)(840.1451668,613.07783877)(840.2051709,613.08784668)
\curveto(841.09516585,613.09783875)(841.83516511,612.91283893)(842.4251709,612.53284668)
\curveto(843.01516393,612.15283969)(843.44016351,611.65784019)(843.7001709,611.04784668)
\curveto(843.7501632,610.9478409)(843.79016316,610.847841)(843.8201709,610.74784668)
\curveto(843.8501631,610.6478412)(843.88516306,610.5428413)(843.9251709,610.43284668)
\curveto(843.95516299,610.32284152)(843.98016297,610.20284164)(844.0001709,610.07284668)
\curveto(844.02016293,609.95284189)(844.0451629,609.82784202)(844.0751709,609.69784668)
\curveto(844.08516286,609.6478422)(844.08516286,609.59284225)(844.0751709,609.53284668)
\curveto(844.07516287,609.48284236)(844.08016287,609.43284241)(844.0901709,609.38284668)
\moveto(842.7551709,608.52784668)
\curveto(842.77516417,608.59784325)(842.78016417,608.67784317)(842.7701709,608.76784668)
\lineto(842.7701709,609.02284668)
\curveto(842.77016418,609.41284243)(842.73516421,609.7428421)(842.6651709,610.01284668)
\curveto(842.63516431,610.09284175)(842.61016434,610.17284167)(842.5901709,610.25284668)
\curveto(842.57016438,610.33284151)(842.5451644,610.40784144)(842.5151709,610.47784668)
\curveto(842.23516471,611.12784072)(841.79016516,611.57784027)(841.1801709,611.82784668)
\curveto(841.11016584,611.85783999)(841.03516591,611.87783997)(840.9551709,611.88784668)
\lineto(840.7151709,611.94784668)
\curveto(840.63516631,611.96783988)(840.5501664,611.97783987)(840.4601709,611.97784668)
\lineto(840.1901709,611.97784668)
\lineto(839.9201709,611.93284668)
\curveto(839.82016713,611.91283993)(839.72516722,611.88783996)(839.6351709,611.85784668)
\curveto(839.55516739,611.83784001)(839.47516747,611.80784004)(839.3951709,611.76784668)
\curveto(839.32516762,611.7478401)(839.26016769,611.71784013)(839.2001709,611.67784668)
\curveto(839.14016781,611.63784021)(839.08516786,611.59784025)(839.0351709,611.55784668)
\curveto(838.79516815,611.38784046)(838.60016835,611.18284066)(838.4501709,610.94284668)
\curveto(838.30016865,610.70284114)(838.17016878,610.42284142)(838.0601709,610.10284668)
\curveto(838.03016892,610.00284184)(838.01016894,609.89784195)(838.0001709,609.78784668)
\curveto(837.99016896,609.68784216)(837.97516897,609.58284226)(837.9551709,609.47284668)
\curveto(837.945169,609.43284241)(837.94016901,609.36784248)(837.9401709,609.27784668)
\curveto(837.93016902,609.2478426)(837.92516902,609.21284263)(837.9251709,609.17284668)
\curveto(837.93516901,609.13284271)(837.94016901,609.08784276)(837.9401709,609.03784668)
\lineto(837.9401709,608.73784668)
\curveto(837.94016901,608.63784321)(837.950169,608.5478433)(837.9701709,608.46784668)
\lineto(838.0001709,608.28784668)
\curveto(838.02016893,608.18784366)(838.03516891,608.08784376)(838.0451709,607.98784668)
\curveto(838.06516888,607.89784395)(838.09516885,607.81284403)(838.1351709,607.73284668)
\curveto(838.23516871,607.49284435)(838.3501686,607.26784458)(838.4801709,607.05784668)
\curveto(838.62016833,606.847845)(838.79016816,606.67284517)(838.9901709,606.53284668)
\curveto(839.04016791,606.50284534)(839.08516786,606.47784537)(839.1251709,606.45784668)
\curveto(839.16516778,606.43784541)(839.21016774,606.41284543)(839.2601709,606.38284668)
\curveto(839.34016761,606.33284551)(839.42516752,606.28784556)(839.5151709,606.24784668)
\curveto(839.61516733,606.21784563)(839.72016723,606.18784566)(839.8301709,606.15784668)
\curveto(839.88016707,606.13784571)(839.92516702,606.12784572)(839.9651709,606.12784668)
\curveto(840.01516693,606.13784571)(840.06516688,606.13784571)(840.1151709,606.12784668)
\curveto(840.1451668,606.11784573)(840.20516674,606.10784574)(840.2951709,606.09784668)
\curveto(840.39516655,606.08784576)(840.47016648,606.09284575)(840.5201709,606.11284668)
\curveto(840.56016639,606.12284572)(840.60016635,606.12284572)(840.6401709,606.11284668)
\curveto(840.68016627,606.11284573)(840.72016623,606.12284572)(840.7601709,606.14284668)
\curveto(840.84016611,606.16284568)(840.92016603,606.17784567)(841.0001709,606.18784668)
\curveto(841.08016587,606.20784564)(841.15516579,606.23284561)(841.2251709,606.26284668)
\curveto(841.56516538,606.40284544)(841.84016511,606.59784525)(842.0501709,606.84784668)
\curveto(842.26016469,607.09784475)(842.43516451,607.39284445)(842.5751709,607.73284668)
\curveto(842.62516432,607.85284399)(842.65516429,607.97784387)(842.6651709,608.10784668)
\curveto(842.68516426,608.2478436)(842.71516423,608.38784346)(842.7551709,608.52784668)
}
}
{
\newrgbcolor{curcolor}{0 0 0}
\pscustom[linestyle=none,fillstyle=solid,fillcolor=curcolor]
{
\newpath
\moveto(848.71345215,613.08784668)
\curveto(849.45344736,613.09783875)(850.06844674,612.98783886)(850.55845215,612.75784668)
\curveto(851.05844575,612.53783931)(851.45344536,612.20283964)(851.74345215,611.75284668)
\curveto(851.87344494,611.55284029)(851.98344483,611.30784054)(852.07345215,611.01784668)
\curveto(852.09344472,610.96784088)(852.1084447,610.90284094)(852.11845215,610.82284668)
\curveto(852.12844468,610.7428411)(852.12344469,610.67284117)(852.10345215,610.61284668)
\curveto(852.07344474,610.56284128)(852.02344479,610.51784133)(851.95345215,610.47784668)
\curveto(851.92344489,610.45784139)(851.89344492,610.4478414)(851.86345215,610.44784668)
\curveto(851.83344498,610.45784139)(851.79844501,610.45784139)(851.75845215,610.44784668)
\curveto(851.71844509,610.43784141)(851.67844513,610.43284141)(851.63845215,610.43284668)
\curveto(851.59844521,610.4428414)(851.55844525,610.4478414)(851.51845215,610.44784668)
\lineto(851.20345215,610.44784668)
\curveto(851.10344571,610.45784139)(851.01844579,610.48784136)(850.94845215,610.53784668)
\curveto(850.86844594,610.59784125)(850.813446,610.68284116)(850.78345215,610.79284668)
\curveto(850.75344606,610.90284094)(850.7134461,610.99784085)(850.66345215,611.07784668)
\curveto(850.5134463,611.33784051)(850.31844649,611.5428403)(850.07845215,611.69284668)
\curveto(849.99844681,611.7428401)(849.9134469,611.78284006)(849.82345215,611.81284668)
\curveto(849.73344708,611.85283999)(849.63844717,611.88783996)(849.53845215,611.91784668)
\curveto(849.39844741,611.95783989)(849.2134476,611.97783987)(848.98345215,611.97784668)
\curveto(848.75344806,611.98783986)(848.56344825,611.96783988)(848.41345215,611.91784668)
\curveto(848.34344847,611.89783995)(848.27844853,611.88283996)(848.21845215,611.87284668)
\curveto(848.15844865,611.86283998)(848.09344872,611.84784)(848.02345215,611.82784668)
\curveto(847.76344905,611.71784013)(847.53344928,611.56784028)(847.33345215,611.37784668)
\curveto(847.13344968,611.18784066)(846.97844983,610.96284088)(846.86845215,610.70284668)
\curveto(846.82844998,610.61284123)(846.79345002,610.51784133)(846.76345215,610.41784668)
\curveto(846.73345008,610.32784152)(846.70345011,610.22784162)(846.67345215,610.11784668)
\lineto(846.58345215,609.71284668)
\curveto(846.57345024,609.66284218)(846.56845024,609.60784224)(846.56845215,609.54784668)
\curveto(846.57845023,609.48784236)(846.57345024,609.43284241)(846.55345215,609.38284668)
\lineto(846.55345215,609.26284668)
\curveto(846.54345027,609.22284262)(846.53345028,609.15784269)(846.52345215,609.06784668)
\curveto(846.52345029,608.97784287)(846.53345028,608.91284293)(846.55345215,608.87284668)
\curveto(846.56345025,608.82284302)(846.56345025,608.77284307)(846.55345215,608.72284668)
\curveto(846.54345027,608.67284317)(846.54345027,608.62284322)(846.55345215,608.57284668)
\curveto(846.56345025,608.53284331)(846.56845024,608.46284338)(846.56845215,608.36284668)
\curveto(846.58845022,608.28284356)(846.60345021,608.19784365)(846.61345215,608.10784668)
\curveto(846.63345018,608.01784383)(846.65345016,607.93284391)(846.67345215,607.85284668)
\curveto(846.78345003,607.53284431)(846.9084499,607.25284459)(847.04845215,607.01284668)
\curveto(847.19844961,606.78284506)(847.40344941,606.58284526)(847.66345215,606.41284668)
\curveto(847.75344906,606.36284548)(847.84344897,606.31784553)(847.93345215,606.27784668)
\curveto(848.03344878,606.23784561)(848.13844867,606.19784565)(848.24845215,606.15784668)
\curveto(848.29844851,606.1478457)(848.33844847,606.1428457)(848.36845215,606.14284668)
\curveto(848.39844841,606.1428457)(848.43844837,606.13784571)(848.48845215,606.12784668)
\curveto(848.51844829,606.11784573)(848.56844824,606.11284573)(848.63845215,606.11284668)
\lineto(848.80345215,606.11284668)
\curveto(848.80344801,606.10284574)(848.82344799,606.09784575)(848.86345215,606.09784668)
\curveto(848.88344793,606.10784574)(848.9084479,606.10784574)(848.93845215,606.09784668)
\curveto(848.96844784,606.09784575)(848.99844781,606.10284574)(849.02845215,606.11284668)
\curveto(849.09844771,606.13284571)(849.16344765,606.13784571)(849.22345215,606.12784668)
\curveto(849.29344752,606.12784572)(849.36344745,606.13784571)(849.43345215,606.15784668)
\curveto(849.69344712,606.23784561)(849.91844689,606.33784551)(850.10845215,606.45784668)
\curveto(850.29844651,606.58784526)(850.45844635,606.75284509)(850.58845215,606.95284668)
\curveto(850.63844617,607.03284481)(850.68344613,607.11784473)(850.72345215,607.20784668)
\lineto(850.84345215,607.47784668)
\curveto(850.86344595,607.55784429)(850.88344593,607.63284421)(850.90345215,607.70284668)
\curveto(850.93344588,607.78284406)(850.98344583,607.847844)(851.05345215,607.89784668)
\curveto(851.08344573,607.92784392)(851.14344567,607.9478439)(851.23345215,607.95784668)
\curveto(851.32344549,607.97784387)(851.41844539,607.98784386)(851.51845215,607.98784668)
\curveto(851.62844518,607.99784385)(851.72844508,607.99784385)(851.81845215,607.98784668)
\curveto(851.91844489,607.97784387)(851.98844482,607.95784389)(852.02845215,607.92784668)
\curveto(852.08844472,607.88784396)(852.12344469,607.82784402)(852.13345215,607.74784668)
\curveto(852.15344466,607.66784418)(852.15344466,607.58284426)(852.13345215,607.49284668)
\curveto(852.08344473,607.3428445)(852.03344478,607.19784465)(851.98345215,607.05784668)
\curveto(851.94344487,606.92784492)(851.88844492,606.79784505)(851.81845215,606.66784668)
\curveto(851.66844514,606.36784548)(851.47844533,606.10284574)(851.24845215,605.87284668)
\curveto(851.02844578,605.6428462)(850.75844605,605.45784639)(850.43845215,605.31784668)
\curveto(850.35844645,605.27784657)(850.27344654,605.2428466)(850.18345215,605.21284668)
\curveto(850.09344672,605.19284665)(849.99844681,605.16784668)(849.89845215,605.13784668)
\curveto(849.78844702,605.09784675)(849.67844713,605.07784677)(849.56845215,605.07784668)
\curveto(849.45844735,605.06784678)(849.34844746,605.05284679)(849.23845215,605.03284668)
\curveto(849.19844761,605.01284683)(849.15844765,605.00784684)(849.11845215,605.01784668)
\curveto(849.07844773,605.02784682)(849.03844777,605.02784682)(848.99845215,605.01784668)
\lineto(848.86345215,605.01784668)
\lineto(848.62345215,605.01784668)
\curveto(848.55344826,605.00784684)(848.48844832,605.01284683)(848.42845215,605.03284668)
\lineto(848.35345215,605.03284668)
\lineto(847.99345215,605.07784668)
\curveto(847.86344895,605.11784673)(847.73844907,605.15284669)(847.61845215,605.18284668)
\curveto(847.49844931,605.21284663)(847.38344943,605.25284659)(847.27345215,605.30284668)
\curveto(846.9134499,605.46284638)(846.6134502,605.65284619)(846.37345215,605.87284668)
\curveto(846.14345067,606.09284575)(845.92845088,606.36284548)(845.72845215,606.68284668)
\curveto(845.67845113,606.76284508)(845.63345118,606.85284499)(845.59345215,606.95284668)
\lineto(845.47345215,607.25284668)
\curveto(845.42345139,607.36284448)(845.38845142,607.47784437)(845.36845215,607.59784668)
\curveto(845.34845146,607.71784413)(845.32345149,607.83784401)(845.29345215,607.95784668)
\curveto(845.28345153,607.99784385)(845.27845153,608.03784381)(845.27845215,608.07784668)
\curveto(845.27845153,608.11784373)(845.27345154,608.15784369)(845.26345215,608.19784668)
\curveto(845.24345157,608.25784359)(845.23345158,608.32284352)(845.23345215,608.39284668)
\curveto(845.24345157,608.46284338)(845.23845157,608.52784332)(845.21845215,608.58784668)
\lineto(845.21845215,608.73784668)
\curveto(845.2084516,608.78784306)(845.20345161,608.85784299)(845.20345215,608.94784668)
\curveto(845.20345161,609.03784281)(845.2084516,609.10784274)(845.21845215,609.15784668)
\curveto(845.22845158,609.20784264)(845.22845158,609.25284259)(845.21845215,609.29284668)
\curveto(845.21845159,609.33284251)(845.22345159,609.37284247)(845.23345215,609.41284668)
\curveto(845.25345156,609.48284236)(845.25845155,609.55284229)(845.24845215,609.62284668)
\curveto(845.24845156,609.69284215)(845.25845155,609.75784209)(845.27845215,609.81784668)
\curveto(845.31845149,609.98784186)(845.35345146,610.15784169)(845.38345215,610.32784668)
\curveto(845.4134514,610.49784135)(845.45845135,610.65784119)(845.51845215,610.80784668)
\curveto(845.72845108,611.32784052)(845.98345083,611.7478401)(846.28345215,612.06784668)
\curveto(846.58345023,612.38783946)(846.99344982,612.65283919)(847.51345215,612.86284668)
\curveto(847.62344919,612.91283893)(847.74344907,612.9478389)(847.87345215,612.96784668)
\curveto(848.00344881,612.98783886)(848.13844867,613.01283883)(848.27845215,613.04284668)
\curveto(848.34844846,613.05283879)(848.41844839,613.05783879)(848.48845215,613.05784668)
\curveto(848.55844825,613.06783878)(848.63344818,613.07783877)(848.71345215,613.08784668)
}
}
{
\newrgbcolor{curcolor}{0 0 0}
\pscustom[linestyle=none,fillstyle=solid,fillcolor=curcolor]
{
\newpath
\moveto(860.38509277,609.35284668)
\curveto(860.40508509,609.25284259)(860.40508509,609.13784271)(860.38509277,609.00784668)
\curveto(860.37508512,608.88784296)(860.34508515,608.80284304)(860.29509277,608.75284668)
\curveto(860.24508525,608.71284313)(860.17008532,608.68284316)(860.07009277,608.66284668)
\curveto(859.98008551,608.65284319)(859.87508562,608.6478432)(859.75509277,608.64784668)
\lineto(859.39509277,608.64784668)
\curveto(859.27508622,608.65784319)(859.17008632,608.66284318)(859.08009277,608.66284668)
\lineto(855.24009277,608.66284668)
\curveto(855.16009033,608.66284318)(855.08009041,608.65784319)(855.00009277,608.64784668)
\curveto(854.92009057,608.6478432)(854.85509064,608.63284321)(854.80509277,608.60284668)
\curveto(854.76509073,608.58284326)(854.72509077,608.5428433)(854.68509277,608.48284668)
\curveto(854.66509083,608.45284339)(854.64509085,608.40784344)(854.62509277,608.34784668)
\curveto(854.60509089,608.29784355)(854.60509089,608.2478436)(854.62509277,608.19784668)
\curveto(854.63509086,608.1478437)(854.64009085,608.10284374)(854.64009277,608.06284668)
\curveto(854.64009085,608.02284382)(854.64509085,607.98284386)(854.65509277,607.94284668)
\curveto(854.67509082,607.86284398)(854.6950908,607.77784407)(854.71509277,607.68784668)
\curveto(854.73509076,607.60784424)(854.76509073,607.52784432)(854.80509277,607.44784668)
\curveto(855.03509046,606.90784494)(855.41509008,606.52284532)(855.94509277,606.29284668)
\curveto(856.00508949,606.26284558)(856.07008942,606.23784561)(856.14009277,606.21784668)
\lineto(856.35009277,606.15784668)
\curveto(856.38008911,606.1478457)(856.43008906,606.1428457)(856.50009277,606.14284668)
\curveto(856.64008885,606.10284574)(856.82508867,606.08284576)(857.05509277,606.08284668)
\curveto(857.28508821,606.08284576)(857.47008802,606.10284574)(857.61009277,606.14284668)
\curveto(857.75008774,606.18284566)(857.87508762,606.22284562)(857.98509277,606.26284668)
\curveto(858.10508739,606.31284553)(858.21508728,606.37284547)(858.31509277,606.44284668)
\curveto(858.42508707,606.51284533)(858.52008697,606.59284525)(858.60009277,606.68284668)
\curveto(858.68008681,606.78284506)(858.75008674,606.88784496)(858.81009277,606.99784668)
\curveto(858.87008662,607.09784475)(858.92008657,607.20284464)(858.96009277,607.31284668)
\curveto(859.01008648,607.42284442)(859.0900864,607.50284434)(859.20009277,607.55284668)
\curveto(859.24008625,607.57284427)(859.30508619,607.58784426)(859.39509277,607.59784668)
\curveto(859.48508601,607.60784424)(859.57508592,607.60784424)(859.66509277,607.59784668)
\curveto(859.75508574,607.59784425)(859.84008565,607.59284425)(859.92009277,607.58284668)
\curveto(860.00008549,607.57284427)(860.05508544,607.55284429)(860.08509277,607.52284668)
\curveto(860.18508531,607.45284439)(860.21008528,607.33784451)(860.16009277,607.17784668)
\curveto(860.08008541,606.90784494)(859.97508552,606.66784518)(859.84509277,606.45784668)
\curveto(859.64508585,606.13784571)(859.41508608,605.87284597)(859.15509277,605.66284668)
\curveto(858.90508659,605.46284638)(858.58508691,605.29784655)(858.19509277,605.16784668)
\curveto(858.0950874,605.12784672)(857.9950875,605.10284674)(857.89509277,605.09284668)
\curveto(857.7950877,605.07284677)(857.6900878,605.05284679)(857.58009277,605.03284668)
\curveto(857.53008796,605.02284682)(857.48008801,605.01784683)(857.43009277,605.01784668)
\curveto(857.3900881,605.01784683)(857.34508815,605.01284683)(857.29509277,605.00284668)
\lineto(857.14509277,605.00284668)
\curveto(857.0950884,604.99284685)(857.03508846,604.98784686)(856.96509277,604.98784668)
\curveto(856.90508859,604.98784686)(856.85508864,604.99284685)(856.81509277,605.00284668)
\lineto(856.68009277,605.00284668)
\curveto(856.63008886,605.01284683)(856.58508891,605.01784683)(856.54509277,605.01784668)
\curveto(856.50508899,605.01784683)(856.46508903,605.02284682)(856.42509277,605.03284668)
\curveto(856.37508912,605.0428468)(856.32008917,605.05284679)(856.26009277,605.06284668)
\curveto(856.20008929,605.06284678)(856.14508935,605.06784678)(856.09509277,605.07784668)
\curveto(856.00508949,605.09784675)(855.91508958,605.12284672)(855.82509277,605.15284668)
\curveto(855.73508976,605.17284667)(855.65008984,605.19784665)(855.57009277,605.22784668)
\curveto(855.53008996,605.2478466)(855.49509,605.25784659)(855.46509277,605.25784668)
\curveto(855.43509006,605.26784658)(855.40009009,605.28284656)(855.36009277,605.30284668)
\curveto(855.21009028,605.37284647)(855.05009044,605.45784639)(854.88009277,605.55784668)
\curveto(854.5900909,605.7478461)(854.34009115,605.97784587)(854.13009277,606.24784668)
\curveto(853.93009156,606.52784532)(853.76009173,606.83784501)(853.62009277,607.17784668)
\curveto(853.57009192,607.28784456)(853.53009196,607.40284444)(853.50009277,607.52284668)
\curveto(853.48009201,607.6428442)(853.45009204,607.76284408)(853.41009277,607.88284668)
\curveto(853.40009209,607.92284392)(853.3950921,607.95784389)(853.39509277,607.98784668)
\curveto(853.3950921,608.01784383)(853.3900921,608.05784379)(853.38009277,608.10784668)
\curveto(853.36009213,608.18784366)(853.34509215,608.27284357)(853.33509277,608.36284668)
\curveto(853.32509217,608.45284339)(853.31009218,608.5428433)(853.29009277,608.63284668)
\lineto(853.29009277,608.84284668)
\curveto(853.28009221,608.88284296)(853.27009222,608.93784291)(853.26009277,609.00784668)
\curveto(853.26009223,609.08784276)(853.26509223,609.15284269)(853.27509277,609.20284668)
\lineto(853.27509277,609.36784668)
\curveto(853.2950922,609.41784243)(853.30009219,609.46784238)(853.29009277,609.51784668)
\curveto(853.2900922,609.57784227)(853.2950922,609.63284221)(853.30509277,609.68284668)
\curveto(853.34509215,609.842842)(853.37509212,610.00284184)(853.39509277,610.16284668)
\curveto(853.42509207,610.32284152)(853.47009202,610.47284137)(853.53009277,610.61284668)
\curveto(853.58009191,610.72284112)(853.62509187,610.83284101)(853.66509277,610.94284668)
\curveto(853.71509178,611.06284078)(853.77009172,611.17784067)(853.83009277,611.28784668)
\curveto(854.05009144,611.63784021)(854.30009119,611.93783991)(854.58009277,612.18784668)
\curveto(854.86009063,612.4478394)(855.20509029,612.66283918)(855.61509277,612.83284668)
\curveto(855.73508976,612.88283896)(855.85508964,612.91783893)(855.97509277,612.93784668)
\curveto(856.10508939,612.96783888)(856.24008925,612.99783885)(856.38009277,613.02784668)
\curveto(856.43008906,613.03783881)(856.47508902,613.0428388)(856.51509277,613.04284668)
\curveto(856.55508894,613.05283879)(856.60008889,613.05783879)(856.65009277,613.05784668)
\curveto(856.67008882,613.06783878)(856.6950888,613.06783878)(856.72509277,613.05784668)
\curveto(856.75508874,613.0478388)(856.78008871,613.05283879)(856.80009277,613.07284668)
\curveto(857.22008827,613.08283876)(857.58508791,613.03783881)(857.89509277,612.93784668)
\curveto(858.20508729,612.847839)(858.48508701,612.72283912)(858.73509277,612.56284668)
\curveto(858.78508671,612.5428393)(858.82508667,612.51283933)(858.85509277,612.47284668)
\curveto(858.88508661,612.4428394)(858.92008657,612.41783943)(858.96009277,612.39784668)
\curveto(859.04008645,612.33783951)(859.12008637,612.26783958)(859.20009277,612.18784668)
\curveto(859.2900862,612.10783974)(859.36508613,612.02783982)(859.42509277,611.94784668)
\curveto(859.58508591,611.73784011)(859.72008577,611.53784031)(859.83009277,611.34784668)
\curveto(859.90008559,611.23784061)(859.95508554,611.11784073)(859.99509277,610.98784668)
\curveto(860.03508546,610.85784099)(860.08008541,610.72784112)(860.13009277,610.59784668)
\curveto(860.18008531,610.46784138)(860.21508528,610.33284151)(860.23509277,610.19284668)
\curveto(860.26508523,610.05284179)(860.30008519,609.91284193)(860.34009277,609.77284668)
\curveto(860.35008514,609.70284214)(860.35508514,609.63284221)(860.35509277,609.56284668)
\lineto(860.38509277,609.35284668)
\moveto(858.93009277,609.86284668)
\curveto(858.96008653,609.90284194)(858.98508651,609.95284189)(859.00509277,610.01284668)
\curveto(859.02508647,610.08284176)(859.02508647,610.15284169)(859.00509277,610.22284668)
\curveto(858.94508655,610.4428414)(858.86008663,610.6478412)(858.75009277,610.83784668)
\curveto(858.61008688,611.06784078)(858.45508704,611.26284058)(858.28509277,611.42284668)
\curveto(858.11508738,611.58284026)(857.8950876,611.71784013)(857.62509277,611.82784668)
\curveto(857.55508794,611.84784)(857.48508801,611.86283998)(857.41509277,611.87284668)
\curveto(857.34508815,611.89283995)(857.27008822,611.91283993)(857.19009277,611.93284668)
\curveto(857.11008838,611.95283989)(857.02508847,611.96283988)(856.93509277,611.96284668)
\lineto(856.68009277,611.96284668)
\curveto(856.65008884,611.9428399)(856.61508888,611.93283991)(856.57509277,611.93284668)
\curveto(856.53508896,611.9428399)(856.50008899,611.9428399)(856.47009277,611.93284668)
\lineto(856.23009277,611.87284668)
\curveto(856.16008933,611.86283998)(856.0900894,611.84784)(856.02009277,611.82784668)
\curveto(855.73008976,611.70784014)(855.49509,611.55784029)(855.31509277,611.37784668)
\curveto(855.14509035,611.19784065)(854.9900905,610.97284087)(854.85009277,610.70284668)
\curveto(854.82009067,610.65284119)(854.7900907,610.58784126)(854.76009277,610.50784668)
\curveto(854.73009076,610.43784141)(854.70509079,610.35784149)(854.68509277,610.26784668)
\curveto(854.66509083,610.17784167)(854.66009083,610.09284175)(854.67009277,610.01284668)
\curveto(854.68009081,609.93284191)(854.71509078,609.87284197)(854.77509277,609.83284668)
\curveto(854.85509064,609.77284207)(854.9900905,609.7428421)(855.18009277,609.74284668)
\curveto(855.38009011,609.75284209)(855.55008994,609.75784209)(855.69009277,609.75784668)
\lineto(857.97009277,609.75784668)
\curveto(858.12008737,609.75784209)(858.30008719,609.75284209)(858.51009277,609.74284668)
\curveto(858.72008677,609.7428421)(858.86008663,609.78284206)(858.93009277,609.86284668)
}
}
{
\newrgbcolor{curcolor}{0 0 0}
\pscustom[linestyle=none,fillstyle=solid,fillcolor=curcolor]
{
\newpath
\moveto(865.3817334,613.05784668)
\curveto(866.01172816,613.07783877)(866.51672766,612.99283885)(866.8967334,612.80284668)
\curveto(867.2767269,612.61283923)(867.58172659,612.32783952)(867.8117334,611.94784668)
\curveto(867.8717263,611.84784)(867.91672626,611.73784011)(867.9467334,611.61784668)
\curveto(867.98672619,611.50784034)(868.02172615,611.39284045)(868.0517334,611.27284668)
\curveto(868.10172607,611.08284076)(868.13172604,610.87784097)(868.1417334,610.65784668)
\curveto(868.15172602,610.43784141)(868.15672602,610.21284163)(868.1567334,609.98284668)
\lineto(868.1567334,608.37784668)
\lineto(868.1567334,606.03784668)
\curveto(868.15672602,605.86784598)(868.15172602,605.69784615)(868.1417334,605.52784668)
\curveto(868.14172603,605.35784649)(868.0767261,605.2478466)(867.9467334,605.19784668)
\curveto(867.89672628,605.17784667)(867.84172633,605.16784668)(867.7817334,605.16784668)
\curveto(867.73172644,605.15784669)(867.6767265,605.15284669)(867.6167334,605.15284668)
\curveto(867.48672669,605.15284669)(867.36172681,605.15784669)(867.2417334,605.16784668)
\curveto(867.12172705,605.16784668)(867.03672714,605.20784664)(866.9867334,605.28784668)
\curveto(866.93672724,605.35784649)(866.91172726,605.4478464)(866.9117334,605.55784668)
\lineto(866.9117334,605.88784668)
\lineto(866.9117334,607.17784668)
\lineto(866.9117334,609.62284668)
\curveto(866.91172726,609.89284195)(866.90672727,610.15784169)(866.8967334,610.41784668)
\curveto(866.88672729,610.68784116)(866.84172733,610.91784093)(866.7617334,611.10784668)
\curveto(866.68172749,611.30784054)(866.56172761,611.46784038)(866.4017334,611.58784668)
\curveto(866.24172793,611.71784013)(866.05672812,611.81784003)(865.8467334,611.88784668)
\curveto(865.78672839,611.90783994)(865.72172845,611.91783993)(865.6517334,611.91784668)
\curveto(865.59172858,611.92783992)(865.53172864,611.9428399)(865.4717334,611.96284668)
\curveto(865.42172875,611.97283987)(865.34172883,611.97283987)(865.2317334,611.96284668)
\curveto(865.13172904,611.96283988)(865.06172911,611.95783989)(865.0217334,611.94784668)
\curveto(864.98172919,611.92783992)(864.94672923,611.91783993)(864.9167334,611.91784668)
\curveto(864.88672929,611.92783992)(864.85172932,611.92783992)(864.8117334,611.91784668)
\curveto(864.68172949,611.88783996)(864.55672962,611.85283999)(864.4367334,611.81284668)
\curveto(864.32672985,611.78284006)(864.22172995,611.73784011)(864.1217334,611.67784668)
\curveto(864.08173009,611.65784019)(864.04673013,611.63784021)(864.0167334,611.61784668)
\curveto(863.98673019,611.59784025)(863.95173022,611.57784027)(863.9117334,611.55784668)
\curveto(863.56173061,611.30784054)(863.30673087,610.93284091)(863.1467334,610.43284668)
\curveto(863.11673106,610.35284149)(863.09673108,610.26784158)(863.0867334,610.17784668)
\curveto(863.0767311,610.09784175)(863.06173111,610.01784183)(863.0417334,609.93784668)
\curveto(863.02173115,609.88784196)(863.01673116,609.83784201)(863.0267334,609.78784668)
\curveto(863.03673114,609.7478421)(863.03173114,609.70784214)(863.0117334,609.66784668)
\lineto(863.0117334,609.35284668)
\curveto(863.00173117,609.32284252)(862.99673118,609.28784256)(862.9967334,609.24784668)
\curveto(863.00673117,609.20784264)(863.01173116,609.16284268)(863.0117334,609.11284668)
\lineto(863.0117334,608.66284668)
\lineto(863.0117334,607.22284668)
\lineto(863.0117334,605.90284668)
\lineto(863.0117334,605.55784668)
\curveto(863.01173116,605.4478464)(862.98673119,605.35784649)(862.9367334,605.28784668)
\curveto(862.88673129,605.20784664)(862.79673138,605.16784668)(862.6667334,605.16784668)
\curveto(862.54673163,605.15784669)(862.42173175,605.15284669)(862.2917334,605.15284668)
\curveto(862.21173196,605.15284669)(862.13673204,605.15784669)(862.0667334,605.16784668)
\curveto(861.99673218,605.17784667)(861.93673224,605.20284664)(861.8867334,605.24284668)
\curveto(861.80673237,605.29284655)(861.76673241,605.38784646)(861.7667334,605.52784668)
\lineto(861.7667334,605.93284668)
\lineto(861.7667334,607.70284668)
\lineto(861.7667334,611.33284668)
\lineto(861.7667334,612.24784668)
\lineto(861.7667334,612.51784668)
\curveto(861.76673241,612.60783924)(861.78673239,612.67783917)(861.8267334,612.72784668)
\curveto(861.85673232,612.78783906)(861.90673227,612.82783902)(861.9767334,612.84784668)
\curveto(862.01673216,612.85783899)(862.0717321,612.86783898)(862.1417334,612.87784668)
\curveto(862.22173195,612.88783896)(862.30173187,612.89283895)(862.3817334,612.89284668)
\curveto(862.46173171,612.89283895)(862.53673164,612.88783896)(862.6067334,612.87784668)
\curveto(862.68673149,612.86783898)(862.74173143,612.85283899)(862.7717334,612.83284668)
\curveto(862.88173129,612.76283908)(862.93173124,612.67283917)(862.9217334,612.56284668)
\curveto(862.91173126,612.46283938)(862.92673125,612.3478395)(862.9667334,612.21784668)
\curveto(862.98673119,612.15783969)(863.02673115,612.10783974)(863.0867334,612.06784668)
\curveto(863.20673097,612.05783979)(863.30173087,612.10283974)(863.3717334,612.20284668)
\curveto(863.45173072,612.30283954)(863.53173064,612.38283946)(863.6117334,612.44284668)
\curveto(863.75173042,612.5428393)(863.89173028,612.63283921)(864.0317334,612.71284668)
\curveto(864.18172999,612.80283904)(864.35172982,612.87783897)(864.5417334,612.93784668)
\curveto(864.62172955,612.96783888)(864.70672947,612.98783886)(864.7967334,612.99784668)
\curveto(864.89672928,613.00783884)(864.99172918,613.02283882)(865.0817334,613.04284668)
\curveto(865.13172904,613.05283879)(865.18172899,613.05783879)(865.2317334,613.05784668)
\lineto(865.3817334,613.05784668)
}
}
{
\newrgbcolor{curcolor}{0 0 0}
\pscustom[linestyle=none,fillstyle=solid,fillcolor=curcolor]
{
\newpath
\moveto(870.98634277,615.24784668)
\curveto(871.13634076,615.2478366)(871.28634061,615.2428366)(871.43634277,615.23284668)
\curveto(871.58634031,615.23283661)(871.69134021,615.19283665)(871.75134277,615.11284668)
\curveto(871.8013401,615.05283679)(871.82634007,614.96783688)(871.82634277,614.85784668)
\curveto(871.83634006,614.75783709)(871.84134006,614.65283719)(871.84134277,614.54284668)
\lineto(871.84134277,613.67284668)
\curveto(871.84134006,613.59283825)(871.83634006,613.50783834)(871.82634277,613.41784668)
\curveto(871.82634007,613.33783851)(871.83634006,613.26783858)(871.85634277,613.20784668)
\curveto(871.89634,613.06783878)(871.98633991,612.97783887)(872.12634277,612.93784668)
\curveto(872.17633972,612.92783892)(872.22133968,612.92283892)(872.26134277,612.92284668)
\lineto(872.41134277,612.92284668)
\lineto(872.81634277,612.92284668)
\curveto(872.97633892,612.93283891)(873.09133881,612.92283892)(873.16134277,612.89284668)
\curveto(873.25133865,612.83283901)(873.31133859,612.77283907)(873.34134277,612.71284668)
\curveto(873.36133854,612.67283917)(873.37133853,612.62783922)(873.37134277,612.57784668)
\lineto(873.37134277,612.42784668)
\curveto(873.37133853,612.31783953)(873.36633853,612.21283963)(873.35634277,612.11284668)
\curveto(873.34633855,612.02283982)(873.31133859,611.95283989)(873.25134277,611.90284668)
\curveto(873.19133871,611.85283999)(873.10633879,611.82284002)(872.99634277,611.81284668)
\lineto(872.66634277,611.81284668)
\curveto(872.55633934,611.82284002)(872.44633945,611.82784002)(872.33634277,611.82784668)
\curveto(872.22633967,611.82784002)(872.13133977,611.81284003)(872.05134277,611.78284668)
\curveto(871.98133992,611.75284009)(871.93133997,611.70284014)(871.90134277,611.63284668)
\curveto(871.87134003,611.56284028)(871.85134005,611.47784037)(871.84134277,611.37784668)
\curveto(871.83134007,611.28784056)(871.82634007,611.18784066)(871.82634277,611.07784668)
\curveto(871.83634006,610.97784087)(871.84134006,610.87784097)(871.84134277,610.77784668)
\lineto(871.84134277,607.80784668)
\curveto(871.84134006,607.58784426)(871.83634006,607.35284449)(871.82634277,607.10284668)
\curveto(871.82634007,606.86284498)(871.87134003,606.67784517)(871.96134277,606.54784668)
\curveto(872.01133989,606.46784538)(872.07633982,606.41284543)(872.15634277,606.38284668)
\curveto(872.23633966,606.35284549)(872.33133957,606.32784552)(872.44134277,606.30784668)
\curveto(872.47133943,606.29784555)(872.5013394,606.29284555)(872.53134277,606.29284668)
\curveto(872.57133933,606.30284554)(872.60633929,606.30284554)(872.63634277,606.29284668)
\lineto(872.83134277,606.29284668)
\curveto(872.93133897,606.29284555)(873.02133888,606.28284556)(873.10134277,606.26284668)
\curveto(873.19133871,606.25284559)(873.25633864,606.21784563)(873.29634277,606.15784668)
\curveto(873.31633858,606.12784572)(873.33133857,606.07284577)(873.34134277,605.99284668)
\curveto(873.36133854,605.92284592)(873.37133853,605.847846)(873.37134277,605.76784668)
\curveto(873.38133852,605.68784616)(873.38133852,605.60784624)(873.37134277,605.52784668)
\curveto(873.36133854,605.45784639)(873.34133856,605.40284644)(873.31134277,605.36284668)
\curveto(873.27133863,605.29284655)(873.1963387,605.2428466)(873.08634277,605.21284668)
\curveto(873.00633889,605.19284665)(872.91633898,605.18284666)(872.81634277,605.18284668)
\curveto(872.71633918,605.19284665)(872.62633927,605.19784665)(872.54634277,605.19784668)
\curveto(872.48633941,605.19784665)(872.42633947,605.19284665)(872.36634277,605.18284668)
\curveto(872.30633959,605.18284666)(872.25133965,605.18784666)(872.20134277,605.19784668)
\lineto(872.02134277,605.19784668)
\curveto(871.97133993,605.20784664)(871.92133998,605.21284663)(871.87134277,605.21284668)
\curveto(871.83134007,605.22284662)(871.78634011,605.22784662)(871.73634277,605.22784668)
\curveto(871.53634036,605.27784657)(871.36134054,605.33284651)(871.21134277,605.39284668)
\curveto(871.07134083,605.45284639)(870.95134095,605.55784629)(870.85134277,605.70784668)
\curveto(870.71134119,605.90784594)(870.63134127,606.15784569)(870.61134277,606.45784668)
\curveto(870.59134131,606.76784508)(870.58134132,607.09784475)(870.58134277,607.44784668)
\lineto(870.58134277,611.37784668)
\curveto(870.55134135,611.50784034)(870.52134138,611.60284024)(870.49134277,611.66284668)
\curveto(870.47134143,611.72284012)(870.4013415,611.77284007)(870.28134277,611.81284668)
\curveto(870.24134166,611.82284002)(870.2013417,611.82284002)(870.16134277,611.81284668)
\curveto(870.12134178,611.80284004)(870.08134182,611.80784004)(870.04134277,611.82784668)
\lineto(869.80134277,611.82784668)
\curveto(869.67134223,611.82784002)(869.56134234,611.83784001)(869.47134277,611.85784668)
\curveto(869.39134251,611.88783996)(869.33634256,611.9478399)(869.30634277,612.03784668)
\curveto(869.28634261,612.07783977)(869.27134263,612.12283972)(869.26134277,612.17284668)
\lineto(869.26134277,612.32284668)
\curveto(869.26134264,612.46283938)(869.27134263,612.57783927)(869.29134277,612.66784668)
\curveto(869.31134259,612.76783908)(869.37134253,612.842839)(869.47134277,612.89284668)
\curveto(869.58134232,612.93283891)(869.72134218,612.9428389)(869.89134277,612.92284668)
\curveto(870.07134183,612.90283894)(870.22134168,612.91283893)(870.34134277,612.95284668)
\curveto(870.43134147,613.00283884)(870.5013414,613.07283877)(870.55134277,613.16284668)
\curveto(870.57134133,613.22283862)(870.58134132,613.29783855)(870.58134277,613.38784668)
\lineto(870.58134277,613.64284668)
\lineto(870.58134277,614.57284668)
\lineto(870.58134277,614.81284668)
\curveto(870.58134132,614.90283694)(870.59134131,614.97783687)(870.61134277,615.03784668)
\curveto(870.65134125,615.11783673)(870.72634117,615.18283666)(870.83634277,615.23284668)
\curveto(870.86634103,615.23283661)(870.89134101,615.23283661)(870.91134277,615.23284668)
\curveto(870.94134096,615.2428366)(870.96634093,615.2478366)(870.98634277,615.24784668)
}
}
{
\newrgbcolor{curcolor}{0 0 0}
\pscustom[linestyle=none,fillstyle=solid,fillcolor=curcolor]
{
\newpath
\moveto(881.50813965,609.35284668)
\curveto(881.52813196,609.25284259)(881.52813196,609.13784271)(881.50813965,609.00784668)
\curveto(881.49813199,608.88784296)(881.46813202,608.80284304)(881.41813965,608.75284668)
\curveto(881.36813212,608.71284313)(881.2931322,608.68284316)(881.19313965,608.66284668)
\curveto(881.10313239,608.65284319)(880.99813249,608.6478432)(880.87813965,608.64784668)
\lineto(880.51813965,608.64784668)
\curveto(880.39813309,608.65784319)(880.2931332,608.66284318)(880.20313965,608.66284668)
\lineto(876.36313965,608.66284668)
\curveto(876.28313721,608.66284318)(876.20313729,608.65784319)(876.12313965,608.64784668)
\curveto(876.04313745,608.6478432)(875.97813751,608.63284321)(875.92813965,608.60284668)
\curveto(875.8881376,608.58284326)(875.84813764,608.5428433)(875.80813965,608.48284668)
\curveto(875.7881377,608.45284339)(875.76813772,608.40784344)(875.74813965,608.34784668)
\curveto(875.72813776,608.29784355)(875.72813776,608.2478436)(875.74813965,608.19784668)
\curveto(875.75813773,608.1478437)(875.76313773,608.10284374)(875.76313965,608.06284668)
\curveto(875.76313773,608.02284382)(875.76813772,607.98284386)(875.77813965,607.94284668)
\curveto(875.79813769,607.86284398)(875.81813767,607.77784407)(875.83813965,607.68784668)
\curveto(875.85813763,607.60784424)(875.8881376,607.52784432)(875.92813965,607.44784668)
\curveto(876.15813733,606.90784494)(876.53813695,606.52284532)(877.06813965,606.29284668)
\curveto(877.12813636,606.26284558)(877.1931363,606.23784561)(877.26313965,606.21784668)
\lineto(877.47313965,606.15784668)
\curveto(877.50313599,606.1478457)(877.55313594,606.1428457)(877.62313965,606.14284668)
\curveto(877.76313573,606.10284574)(877.94813554,606.08284576)(878.17813965,606.08284668)
\curveto(878.40813508,606.08284576)(878.5931349,606.10284574)(878.73313965,606.14284668)
\curveto(878.87313462,606.18284566)(878.99813449,606.22284562)(879.10813965,606.26284668)
\curveto(879.22813426,606.31284553)(879.33813415,606.37284547)(879.43813965,606.44284668)
\curveto(879.54813394,606.51284533)(879.64313385,606.59284525)(879.72313965,606.68284668)
\curveto(879.80313369,606.78284506)(879.87313362,606.88784496)(879.93313965,606.99784668)
\curveto(879.9931335,607.09784475)(880.04313345,607.20284464)(880.08313965,607.31284668)
\curveto(880.13313336,607.42284442)(880.21313328,607.50284434)(880.32313965,607.55284668)
\curveto(880.36313313,607.57284427)(880.42813306,607.58784426)(880.51813965,607.59784668)
\curveto(880.60813288,607.60784424)(880.69813279,607.60784424)(880.78813965,607.59784668)
\curveto(880.87813261,607.59784425)(880.96313253,607.59284425)(881.04313965,607.58284668)
\curveto(881.12313237,607.57284427)(881.17813231,607.55284429)(881.20813965,607.52284668)
\curveto(881.30813218,607.45284439)(881.33313216,607.33784451)(881.28313965,607.17784668)
\curveto(881.20313229,606.90784494)(881.09813239,606.66784518)(880.96813965,606.45784668)
\curveto(880.76813272,606.13784571)(880.53813295,605.87284597)(880.27813965,605.66284668)
\curveto(880.02813346,605.46284638)(879.70813378,605.29784655)(879.31813965,605.16784668)
\curveto(879.21813427,605.12784672)(879.11813437,605.10284674)(879.01813965,605.09284668)
\curveto(878.91813457,605.07284677)(878.81313468,605.05284679)(878.70313965,605.03284668)
\curveto(878.65313484,605.02284682)(878.60313489,605.01784683)(878.55313965,605.01784668)
\curveto(878.51313498,605.01784683)(878.46813502,605.01284683)(878.41813965,605.00284668)
\lineto(878.26813965,605.00284668)
\curveto(878.21813527,604.99284685)(878.15813533,604.98784686)(878.08813965,604.98784668)
\curveto(878.02813546,604.98784686)(877.97813551,604.99284685)(877.93813965,605.00284668)
\lineto(877.80313965,605.00284668)
\curveto(877.75313574,605.01284683)(877.70813578,605.01784683)(877.66813965,605.01784668)
\curveto(877.62813586,605.01784683)(877.5881359,605.02284682)(877.54813965,605.03284668)
\curveto(877.49813599,605.0428468)(877.44313605,605.05284679)(877.38313965,605.06284668)
\curveto(877.32313617,605.06284678)(877.26813622,605.06784678)(877.21813965,605.07784668)
\curveto(877.12813636,605.09784675)(877.03813645,605.12284672)(876.94813965,605.15284668)
\curveto(876.85813663,605.17284667)(876.77313672,605.19784665)(876.69313965,605.22784668)
\curveto(876.65313684,605.2478466)(876.61813687,605.25784659)(876.58813965,605.25784668)
\curveto(876.55813693,605.26784658)(876.52313697,605.28284656)(876.48313965,605.30284668)
\curveto(876.33313716,605.37284647)(876.17313732,605.45784639)(876.00313965,605.55784668)
\curveto(875.71313778,605.7478461)(875.46313803,605.97784587)(875.25313965,606.24784668)
\curveto(875.05313844,606.52784532)(874.88313861,606.83784501)(874.74313965,607.17784668)
\curveto(874.6931388,607.28784456)(874.65313884,607.40284444)(874.62313965,607.52284668)
\curveto(874.60313889,607.6428442)(874.57313892,607.76284408)(874.53313965,607.88284668)
\curveto(874.52313897,607.92284392)(874.51813897,607.95784389)(874.51813965,607.98784668)
\curveto(874.51813897,608.01784383)(874.51313898,608.05784379)(874.50313965,608.10784668)
\curveto(874.48313901,608.18784366)(874.46813902,608.27284357)(874.45813965,608.36284668)
\curveto(874.44813904,608.45284339)(874.43313906,608.5428433)(874.41313965,608.63284668)
\lineto(874.41313965,608.84284668)
\curveto(874.40313909,608.88284296)(874.3931391,608.93784291)(874.38313965,609.00784668)
\curveto(874.38313911,609.08784276)(874.3881391,609.15284269)(874.39813965,609.20284668)
\lineto(874.39813965,609.36784668)
\curveto(874.41813907,609.41784243)(874.42313907,609.46784238)(874.41313965,609.51784668)
\curveto(874.41313908,609.57784227)(874.41813907,609.63284221)(874.42813965,609.68284668)
\curveto(874.46813902,609.842842)(874.49813899,610.00284184)(874.51813965,610.16284668)
\curveto(874.54813894,610.32284152)(874.5931389,610.47284137)(874.65313965,610.61284668)
\curveto(874.70313879,610.72284112)(874.74813874,610.83284101)(874.78813965,610.94284668)
\curveto(874.83813865,611.06284078)(874.8931386,611.17784067)(874.95313965,611.28784668)
\curveto(875.17313832,611.63784021)(875.42313807,611.93783991)(875.70313965,612.18784668)
\curveto(875.98313751,612.4478394)(876.32813716,612.66283918)(876.73813965,612.83284668)
\curveto(876.85813663,612.88283896)(876.97813651,612.91783893)(877.09813965,612.93784668)
\curveto(877.22813626,612.96783888)(877.36313613,612.99783885)(877.50313965,613.02784668)
\curveto(877.55313594,613.03783881)(877.59813589,613.0428388)(877.63813965,613.04284668)
\curveto(877.67813581,613.05283879)(877.72313577,613.05783879)(877.77313965,613.05784668)
\curveto(877.7931357,613.06783878)(877.81813567,613.06783878)(877.84813965,613.05784668)
\curveto(877.87813561,613.0478388)(877.90313559,613.05283879)(877.92313965,613.07284668)
\curveto(878.34313515,613.08283876)(878.70813478,613.03783881)(879.01813965,612.93784668)
\curveto(879.32813416,612.847839)(879.60813388,612.72283912)(879.85813965,612.56284668)
\curveto(879.90813358,612.5428393)(879.94813354,612.51283933)(879.97813965,612.47284668)
\curveto(880.00813348,612.4428394)(880.04313345,612.41783943)(880.08313965,612.39784668)
\curveto(880.16313333,612.33783951)(880.24313325,612.26783958)(880.32313965,612.18784668)
\curveto(880.41313308,612.10783974)(880.488133,612.02783982)(880.54813965,611.94784668)
\curveto(880.70813278,611.73784011)(880.84313265,611.53784031)(880.95313965,611.34784668)
\curveto(881.02313247,611.23784061)(881.07813241,611.11784073)(881.11813965,610.98784668)
\curveto(881.15813233,610.85784099)(881.20313229,610.72784112)(881.25313965,610.59784668)
\curveto(881.30313219,610.46784138)(881.33813215,610.33284151)(881.35813965,610.19284668)
\curveto(881.3881321,610.05284179)(881.42313207,609.91284193)(881.46313965,609.77284668)
\curveto(881.47313202,609.70284214)(881.47813201,609.63284221)(881.47813965,609.56284668)
\lineto(881.50813965,609.35284668)
\moveto(880.05313965,609.86284668)
\curveto(880.08313341,609.90284194)(880.10813338,609.95284189)(880.12813965,610.01284668)
\curveto(880.14813334,610.08284176)(880.14813334,610.15284169)(880.12813965,610.22284668)
\curveto(880.06813342,610.4428414)(879.98313351,610.6478412)(879.87313965,610.83784668)
\curveto(879.73313376,611.06784078)(879.57813391,611.26284058)(879.40813965,611.42284668)
\curveto(879.23813425,611.58284026)(879.01813447,611.71784013)(878.74813965,611.82784668)
\curveto(878.67813481,611.84784)(878.60813488,611.86283998)(878.53813965,611.87284668)
\curveto(878.46813502,611.89283995)(878.3931351,611.91283993)(878.31313965,611.93284668)
\curveto(878.23313526,611.95283989)(878.14813534,611.96283988)(878.05813965,611.96284668)
\lineto(877.80313965,611.96284668)
\curveto(877.77313572,611.9428399)(877.73813575,611.93283991)(877.69813965,611.93284668)
\curveto(877.65813583,611.9428399)(877.62313587,611.9428399)(877.59313965,611.93284668)
\lineto(877.35313965,611.87284668)
\curveto(877.28313621,611.86283998)(877.21313628,611.84784)(877.14313965,611.82784668)
\curveto(876.85313664,611.70784014)(876.61813687,611.55784029)(876.43813965,611.37784668)
\curveto(876.26813722,611.19784065)(876.11313738,610.97284087)(875.97313965,610.70284668)
\curveto(875.94313755,610.65284119)(875.91313758,610.58784126)(875.88313965,610.50784668)
\curveto(875.85313764,610.43784141)(875.82813766,610.35784149)(875.80813965,610.26784668)
\curveto(875.7881377,610.17784167)(875.78313771,610.09284175)(875.79313965,610.01284668)
\curveto(875.80313769,609.93284191)(875.83813765,609.87284197)(875.89813965,609.83284668)
\curveto(875.97813751,609.77284207)(876.11313738,609.7428421)(876.30313965,609.74284668)
\curveto(876.50313699,609.75284209)(876.67313682,609.75784209)(876.81313965,609.75784668)
\lineto(879.09313965,609.75784668)
\curveto(879.24313425,609.75784209)(879.42313407,609.75284209)(879.63313965,609.74284668)
\curveto(879.84313365,609.7428421)(879.98313351,609.78284206)(880.05313965,609.86284668)
}
}
{
\newrgbcolor{curcolor}{0.60000002 0.60000002 0.60000002}
\pscustom[linestyle=none,fillstyle=solid,fillcolor=curcolor]
{
\newpath
\moveto(807.09008789,615.8928833)
\lineto(822.09008789,615.8928833)
\lineto(822.09008789,600.8928833)
\lineto(807.09008789,600.8928833)
\closepath
}
}
{
\newrgbcolor{curcolor}{0 0 0}
\pscustom[linestyle=none,fillstyle=solid,fillcolor=curcolor]
{
\newpath
\moveto(827.1382959,592.82714111)
\lineto(828.0532959,592.82714111)
\curveto(828.15329325,592.82713042)(828.24829315,592.82713042)(828.3382959,592.82714111)
\curveto(828.42829297,592.82713042)(828.5032929,592.80713044)(828.5632959,592.76714111)
\curveto(828.65329275,592.70713054)(828.71329269,592.62713062)(828.7432959,592.52714111)
\curveto(828.78329262,592.42713082)(828.82829257,592.32213092)(828.8782959,592.21214111)
\curveto(828.95829244,592.02213122)(829.02829237,591.83213141)(829.0882959,591.64214111)
\curveto(829.15829224,591.45213179)(829.23329217,591.26213198)(829.3132959,591.07214111)
\curveto(829.38329202,590.89213235)(829.44829195,590.70713254)(829.5082959,590.51714111)
\curveto(829.56829183,590.33713291)(829.63829176,590.15713309)(829.7182959,589.97714111)
\curveto(829.77829162,589.83713341)(829.83329157,589.69213355)(829.8832959,589.54214111)
\curveto(829.93329147,589.39213385)(829.98829141,589.247134)(830.0482959,589.10714111)
\curveto(830.22829117,588.65713459)(830.398291,588.20213504)(830.5582959,587.74214111)
\curveto(830.71829068,587.29213595)(830.88829051,586.8421364)(831.0682959,586.39214111)
\curveto(831.08829031,586.3421369)(831.1032903,586.29213695)(831.1132959,586.24214111)
\lineto(831.1732959,586.09214111)
\curveto(831.26329014,585.87213737)(831.34829005,585.6471376)(831.4282959,585.41714111)
\curveto(831.50828989,585.19713805)(831.59328981,584.97713827)(831.6832959,584.75714111)
\curveto(831.72328968,584.66713858)(831.76328964,584.55713869)(831.8032959,584.42714111)
\curveto(831.84328956,584.30713894)(831.90828949,584.23713901)(831.9982959,584.21714111)
\curveto(832.03828936,584.20713904)(832.06828933,584.20713904)(832.0882959,584.21714111)
\lineto(832.1482959,584.27714111)
\curveto(832.1982892,584.32713892)(832.23328917,584.38213886)(832.2532959,584.44214111)
\curveto(832.28328912,584.50213874)(832.31328909,584.56713868)(832.3432959,584.63714111)
\lineto(832.5832959,585.26714111)
\curveto(832.66328874,585.48713776)(832.74328866,585.70213754)(832.8232959,585.91214111)
\lineto(832.8832959,586.06214111)
\lineto(832.9432959,586.24214111)
\curveto(833.02328838,586.43213681)(833.09328831,586.62213662)(833.1532959,586.81214111)
\curveto(833.22328818,587.01213623)(833.2982881,587.21213603)(833.3782959,587.41214111)
\curveto(833.61828778,587.99213525)(833.83828756,588.57713467)(834.0382959,589.16714111)
\curveto(834.24828715,589.75713349)(834.47328693,590.3421329)(834.7132959,590.92214111)
\curveto(834.79328661,591.12213212)(834.86828653,591.32713192)(834.9382959,591.53714111)
\curveto(835.01828638,591.7471315)(835.0982863,591.95213129)(835.1782959,592.15214111)
\curveto(835.21828618,592.23213101)(835.25328615,592.33213091)(835.2832959,592.45214111)
\curveto(835.32328608,592.57213067)(835.37828602,592.65713059)(835.4482959,592.70714111)
\curveto(835.50828589,592.7471305)(835.58328582,592.77713047)(835.6732959,592.79714111)
\curveto(835.77328563,592.81713043)(835.88328552,592.82713042)(836.0032959,592.82714111)
\curveto(836.12328528,592.83713041)(836.24328516,592.83713041)(836.3632959,592.82714111)
\curveto(836.48328492,592.82713042)(836.59328481,592.82713042)(836.6932959,592.82714111)
\curveto(836.78328462,592.82713042)(836.87328453,592.82713042)(836.9632959,592.82714111)
\curveto(837.06328434,592.82713042)(837.13828426,592.80713044)(837.1882959,592.76714111)
\curveto(837.27828412,592.71713053)(837.32828407,592.62713062)(837.3382959,592.49714111)
\curveto(837.34828405,592.36713088)(837.35328405,592.22713102)(837.3532959,592.07714111)
\lineto(837.3532959,590.42714111)
\lineto(837.3532959,584.15714111)
\lineto(837.3532959,582.89714111)
\curveto(837.35328405,582.78714046)(837.35328405,582.67714057)(837.3532959,582.56714111)
\curveto(837.36328404,582.45714079)(837.34328406,582.37214087)(837.2932959,582.31214111)
\curveto(837.26328414,582.25214099)(837.21828418,582.21214103)(837.1582959,582.19214111)
\curveto(837.0982843,582.18214106)(837.02828437,582.16714108)(836.9482959,582.14714111)
\lineto(836.7082959,582.14714111)
\lineto(836.3482959,582.14714111)
\curveto(836.23828516,582.15714109)(836.15828524,582.20214104)(836.1082959,582.28214111)
\curveto(836.08828531,582.31214093)(836.07328533,582.3421409)(836.0632959,582.37214111)
\curveto(836.06328534,582.41214083)(836.05328535,582.45714079)(836.0332959,582.50714111)
\lineto(836.0332959,582.67214111)
\curveto(836.02328538,582.73214051)(836.01828538,582.80214044)(836.0182959,582.88214111)
\curveto(836.02828537,582.96214028)(836.03328537,583.03714021)(836.0332959,583.10714111)
\lineto(836.0332959,583.94714111)
\lineto(836.0332959,588.37214111)
\curveto(836.03328537,588.62213462)(836.03328537,588.87213437)(836.0332959,589.12214111)
\curveto(836.03328537,589.38213386)(836.02828537,589.63213361)(836.0182959,589.87214111)
\curveto(836.01828538,589.97213327)(836.01328539,590.08213316)(836.0032959,590.20214111)
\curveto(835.99328541,590.32213292)(835.93828546,590.38213286)(835.8382959,590.38214111)
\lineto(835.8382959,590.36714111)
\curveto(835.76828563,590.3471329)(835.70828569,590.28213296)(835.6582959,590.17214111)
\curveto(835.61828578,590.06213318)(835.58328582,589.96713328)(835.5532959,589.88714111)
\curveto(835.48328592,589.71713353)(835.41828598,589.5421337)(835.3582959,589.36214111)
\curveto(835.2982861,589.19213405)(835.22828617,589.02213422)(835.1482959,588.85214111)
\curveto(835.12828627,588.80213444)(835.11328629,588.75713449)(835.1032959,588.71714111)
\curveto(835.09328631,588.67713457)(835.07828632,588.63213461)(835.0582959,588.58214111)
\curveto(834.97828642,588.40213484)(834.90828649,588.21713503)(834.8482959,588.02714111)
\curveto(834.7982866,587.8471354)(834.73328667,587.66713558)(834.6532959,587.48714111)
\curveto(834.58328682,587.33713591)(834.52328688,587.18213606)(834.4732959,587.02214111)
\curveto(834.42328698,586.87213637)(834.36828703,586.72213652)(834.3082959,586.57214111)
\curveto(834.10828729,586.10213714)(833.92828747,585.62713762)(833.7682959,585.14714111)
\curveto(833.60828779,584.67713857)(833.43328797,584.21213903)(833.2432959,583.75214111)
\curveto(833.16328824,583.57213967)(833.09328831,583.39213985)(833.0332959,583.21214111)
\curveto(832.97328843,583.03214021)(832.90828849,582.85214039)(832.8382959,582.67214111)
\curveto(832.78828861,582.56214068)(832.73828866,582.45714079)(832.6882959,582.35714111)
\curveto(832.64828875,582.26714098)(832.56328884,582.20214104)(832.4332959,582.16214111)
\curveto(832.41328899,582.15214109)(832.38828901,582.1471411)(832.3582959,582.14714111)
\curveto(832.33828906,582.15714109)(832.31328909,582.15714109)(832.2832959,582.14714111)
\curveto(832.25328915,582.13714111)(832.21828918,582.13214111)(832.1782959,582.13214111)
\curveto(832.13828926,582.1421411)(832.0982893,582.1471411)(832.0582959,582.14714111)
\lineto(831.7582959,582.14714111)
\curveto(831.65828974,582.1471411)(831.57828982,582.17214107)(831.5182959,582.22214111)
\curveto(831.43828996,582.27214097)(831.37829002,582.3421409)(831.3382959,582.43214111)
\curveto(831.30829009,582.53214071)(831.26829013,582.63214061)(831.2182959,582.73214111)
\curveto(831.13829026,582.93214031)(831.05829034,583.13714011)(830.9782959,583.34714111)
\curveto(830.90829049,583.56713968)(830.83329057,583.77713947)(830.7532959,583.97714111)
\curveto(830.67329073,584.15713909)(830.6032908,584.33713891)(830.5432959,584.51714111)
\curveto(830.49329091,584.70713854)(830.42829097,584.89213835)(830.3482959,585.07214111)
\curveto(830.11829128,585.63213761)(829.9032915,586.19713705)(829.7032959,586.76714111)
\curveto(829.5032919,587.33713591)(829.28829211,587.90213534)(829.0582959,588.46214111)
\lineto(828.8182959,589.09214111)
\curveto(828.74829265,589.31213393)(828.67329273,589.52213372)(828.5932959,589.72214111)
\curveto(828.54329286,589.83213341)(828.4982929,589.93713331)(828.4582959,590.03714111)
\curveto(828.42829297,590.1471331)(828.37829302,590.242133)(828.3082959,590.32214111)
\curveto(828.2982931,590.3421329)(828.28829311,590.35213289)(828.2782959,590.35214111)
\lineto(828.2482959,590.38214111)
\lineto(828.1732959,590.38214111)
\lineto(828.1432959,590.35214111)
\curveto(828.13329327,590.35213289)(828.12329328,590.3471329)(828.1132959,590.33714111)
\curveto(828.09329331,590.28713296)(828.08329332,590.23213301)(828.0832959,590.17214111)
\curveto(828.08329332,590.11213313)(828.07329333,590.05213319)(828.0532959,589.99214111)
\lineto(828.0532959,589.82714111)
\curveto(828.03329337,589.76713348)(828.02829337,589.70213354)(828.0382959,589.63214111)
\curveto(828.04829335,589.56213368)(828.05329335,589.49213375)(828.0532959,589.42214111)
\lineto(828.0532959,588.61214111)
\lineto(828.0532959,584.05214111)
\lineto(828.0532959,582.86714111)
\curveto(828.05329335,582.75714049)(828.04829335,582.6471406)(828.0382959,582.53714111)
\curveto(828.03829336,582.42714082)(828.01329339,582.3421409)(827.9632959,582.28214111)
\curveto(827.91329349,582.20214104)(827.82329358,582.15714109)(827.6932959,582.14714111)
\lineto(827.3032959,582.14714111)
\lineto(827.1082959,582.14714111)
\curveto(827.05829434,582.1471411)(827.00829439,582.15714109)(826.9582959,582.17714111)
\curveto(826.82829457,582.21714103)(826.75329465,582.30214094)(826.7332959,582.43214111)
\curveto(826.72329468,582.56214068)(826.71829468,582.71214053)(826.7182959,582.88214111)
\lineto(826.7182959,584.62214111)
\lineto(826.7182959,590.62214111)
\lineto(826.7182959,592.03214111)
\curveto(826.71829468,592.1421311)(826.71329469,592.25713099)(826.7032959,592.37714111)
\curveto(826.7032947,592.49713075)(826.72829467,592.59213065)(826.7782959,592.66214111)
\curveto(826.81829458,592.72213052)(826.89329451,592.77213047)(827.0032959,592.81214111)
\curveto(827.02329438,592.82213042)(827.04329436,592.82213042)(827.0632959,592.81214111)
\curveto(827.09329431,592.81213043)(827.11829428,592.81713043)(827.1382959,592.82714111)
}
}
{
\newrgbcolor{curcolor}{0 0 0}
\pscustom[linestyle=none,fillstyle=solid,fillcolor=curcolor]
{
\newpath
\moveto(846.58040527,586.34714111)
\curveto(846.60039721,586.28713696)(846.6103972,586.19213705)(846.61040527,586.06214111)
\curveto(846.6103972,585.9421373)(846.60539721,585.85713739)(846.59540527,585.80714111)
\lineto(846.59540527,585.65714111)
\curveto(846.58539723,585.57713767)(846.57539724,585.50213774)(846.56540527,585.43214111)
\curveto(846.56539725,585.37213787)(846.56039725,585.30213794)(846.55040527,585.22214111)
\curveto(846.53039728,585.16213808)(846.5153973,585.10213814)(846.50540527,585.04214111)
\curveto(846.50539731,584.98213826)(846.49539732,584.92213832)(846.47540527,584.86214111)
\curveto(846.43539738,584.73213851)(846.40039741,584.60213864)(846.37040527,584.47214111)
\curveto(846.34039747,584.3421389)(846.30039751,584.22213902)(846.25040527,584.11214111)
\curveto(846.04039777,583.63213961)(845.76039805,583.22714002)(845.41040527,582.89714111)
\curveto(845.06039875,582.57714067)(844.63039918,582.33214091)(844.12040527,582.16214111)
\curveto(844.0103998,582.12214112)(843.89039992,582.09214115)(843.76040527,582.07214111)
\curveto(843.64040017,582.05214119)(843.5154003,582.03214121)(843.38540527,582.01214111)
\curveto(843.32540049,582.00214124)(843.26040055,581.99714125)(843.19040527,581.99714111)
\curveto(843.13040068,581.98714126)(843.07040074,581.98214126)(843.01040527,581.98214111)
\curveto(842.97040084,581.97214127)(842.9104009,581.96714128)(842.83040527,581.96714111)
\curveto(842.76040105,581.96714128)(842.7104011,581.97214127)(842.68040527,581.98214111)
\curveto(842.64040117,581.99214125)(842.60040121,581.99714125)(842.56040527,581.99714111)
\curveto(842.52040129,581.98714126)(842.48540133,581.98714126)(842.45540527,581.99714111)
\lineto(842.36540527,581.99714111)
\lineto(842.00540527,582.04214111)
\curveto(841.86540195,582.08214116)(841.73040208,582.12214112)(841.60040527,582.16214111)
\curveto(841.47040234,582.20214104)(841.34540247,582.247141)(841.22540527,582.29714111)
\curveto(840.77540304,582.49714075)(840.40540341,582.75714049)(840.11540527,583.07714111)
\curveto(839.82540399,583.39713985)(839.58540423,583.78713946)(839.39540527,584.24714111)
\curveto(839.34540447,584.3471389)(839.30540451,584.4471388)(839.27540527,584.54714111)
\curveto(839.25540456,584.6471386)(839.23540458,584.75213849)(839.21540527,584.86214111)
\curveto(839.19540462,584.90213834)(839.18540463,584.93213831)(839.18540527,584.95214111)
\curveto(839.19540462,584.98213826)(839.19540462,585.01713823)(839.18540527,585.05714111)
\curveto(839.16540465,585.13713811)(839.15040466,585.21713803)(839.14040527,585.29714111)
\curveto(839.14040467,585.38713786)(839.13040468,585.47213777)(839.11040527,585.55214111)
\lineto(839.11040527,585.67214111)
\curveto(839.1104047,585.71213753)(839.10540471,585.75713749)(839.09540527,585.80714111)
\curveto(839.08540473,585.85713739)(839.08040473,585.9421373)(839.08040527,586.06214111)
\curveto(839.08040473,586.19213705)(839.09040472,586.28713696)(839.11040527,586.34714111)
\curveto(839.13040468,586.41713683)(839.13540468,586.48713676)(839.12540527,586.55714111)
\curveto(839.1154047,586.62713662)(839.12040469,586.69713655)(839.14040527,586.76714111)
\curveto(839.15040466,586.81713643)(839.15540466,586.85713639)(839.15540527,586.88714111)
\curveto(839.16540465,586.92713632)(839.17540464,586.97213627)(839.18540527,587.02214111)
\curveto(839.2154046,587.1421361)(839.24040457,587.26213598)(839.26040527,587.38214111)
\curveto(839.29040452,587.50213574)(839.33040448,587.61713563)(839.38040527,587.72714111)
\curveto(839.53040428,588.09713515)(839.7104041,588.42713482)(839.92040527,588.71714111)
\curveto(840.14040367,589.01713423)(840.40540341,589.26713398)(840.71540527,589.46714111)
\curveto(840.83540298,589.5471337)(840.96040285,589.61213363)(841.09040527,589.66214111)
\curveto(841.22040259,589.72213352)(841.35540246,589.78213346)(841.49540527,589.84214111)
\curveto(841.6154022,589.89213335)(841.74540207,589.92213332)(841.88540527,589.93214111)
\curveto(842.02540179,589.95213329)(842.16540165,589.98213326)(842.30540527,590.02214111)
\lineto(842.50040527,590.02214111)
\curveto(842.57040124,590.03213321)(842.63540118,590.0421332)(842.69540527,590.05214111)
\curveto(843.58540023,590.06213318)(844.32539949,589.87713337)(844.91540527,589.49714111)
\curveto(845.50539831,589.11713413)(845.93039788,588.62213462)(846.19040527,588.01214111)
\curveto(846.24039757,587.91213533)(846.28039753,587.81213543)(846.31040527,587.71214111)
\curveto(846.34039747,587.61213563)(846.37539744,587.50713574)(846.41540527,587.39714111)
\curveto(846.44539737,587.28713596)(846.47039734,587.16713608)(846.49040527,587.03714111)
\curveto(846.5103973,586.91713633)(846.53539728,586.79213645)(846.56540527,586.66214111)
\curveto(846.57539724,586.61213663)(846.57539724,586.55713669)(846.56540527,586.49714111)
\curveto(846.56539725,586.4471368)(846.57039724,586.39713685)(846.58040527,586.34714111)
\moveto(845.24540527,585.49214111)
\curveto(845.26539855,585.56213768)(845.27039854,585.6421376)(845.26040527,585.73214111)
\lineto(845.26040527,585.98714111)
\curveto(845.26039855,586.37713687)(845.22539859,586.70713654)(845.15540527,586.97714111)
\curveto(845.12539869,587.05713619)(845.10039871,587.13713611)(845.08040527,587.21714111)
\curveto(845.06039875,587.29713595)(845.03539878,587.37213587)(845.00540527,587.44214111)
\curveto(844.72539909,588.09213515)(844.28039953,588.5421347)(843.67040527,588.79214111)
\curveto(843.60040021,588.82213442)(843.52540029,588.8421344)(843.44540527,588.85214111)
\lineto(843.20540527,588.91214111)
\curveto(843.12540069,588.93213431)(843.04040077,588.9421343)(842.95040527,588.94214111)
\lineto(842.68040527,588.94214111)
\lineto(842.41040527,588.89714111)
\curveto(842.3104015,588.87713437)(842.2154016,588.85213439)(842.12540527,588.82214111)
\curveto(842.04540177,588.80213444)(841.96540185,588.77213447)(841.88540527,588.73214111)
\curveto(841.815402,588.71213453)(841.75040206,588.68213456)(841.69040527,588.64214111)
\curveto(841.63040218,588.60213464)(841.57540224,588.56213468)(841.52540527,588.52214111)
\curveto(841.28540253,588.35213489)(841.09040272,588.1471351)(840.94040527,587.90714111)
\curveto(840.79040302,587.66713558)(840.66040315,587.38713586)(840.55040527,587.06714111)
\curveto(840.52040329,586.96713628)(840.50040331,586.86213638)(840.49040527,586.75214111)
\curveto(840.48040333,586.65213659)(840.46540335,586.5471367)(840.44540527,586.43714111)
\curveto(840.43540338,586.39713685)(840.43040338,586.33213691)(840.43040527,586.24214111)
\curveto(840.42040339,586.21213703)(840.4154034,586.17713707)(840.41540527,586.13714111)
\curveto(840.42540339,586.09713715)(840.43040338,586.05213719)(840.43040527,586.00214111)
\lineto(840.43040527,585.70214111)
\curveto(840.43040338,585.60213764)(840.44040337,585.51213773)(840.46040527,585.43214111)
\lineto(840.49040527,585.25214111)
\curveto(840.5104033,585.15213809)(840.52540329,585.05213819)(840.53540527,584.95214111)
\curveto(840.55540326,584.86213838)(840.58540323,584.77713847)(840.62540527,584.69714111)
\curveto(840.72540309,584.45713879)(840.84040297,584.23213901)(840.97040527,584.02214111)
\curveto(841.1104027,583.81213943)(841.28040253,583.63713961)(841.48040527,583.49714111)
\curveto(841.53040228,583.46713978)(841.57540224,583.4421398)(841.61540527,583.42214111)
\curveto(841.65540216,583.40213984)(841.70040211,583.37713987)(841.75040527,583.34714111)
\curveto(841.83040198,583.29713995)(841.9154019,583.25213999)(842.00540527,583.21214111)
\curveto(842.10540171,583.18214006)(842.2104016,583.15214009)(842.32040527,583.12214111)
\curveto(842.37040144,583.10214014)(842.4154014,583.09214015)(842.45540527,583.09214111)
\curveto(842.50540131,583.10214014)(842.55540126,583.10214014)(842.60540527,583.09214111)
\curveto(842.63540118,583.08214016)(842.69540112,583.07214017)(842.78540527,583.06214111)
\curveto(842.88540093,583.05214019)(842.96040085,583.05714019)(843.01040527,583.07714111)
\curveto(843.05040076,583.08714016)(843.09040072,583.08714016)(843.13040527,583.07714111)
\curveto(843.17040064,583.07714017)(843.2104006,583.08714016)(843.25040527,583.10714111)
\curveto(843.33040048,583.12714012)(843.4104004,583.1421401)(843.49040527,583.15214111)
\curveto(843.57040024,583.17214007)(843.64540017,583.19714005)(843.71540527,583.22714111)
\curveto(844.05539976,583.36713988)(844.33039948,583.56213968)(844.54040527,583.81214111)
\curveto(844.75039906,584.06213918)(844.92539889,584.35713889)(845.06540527,584.69714111)
\curveto(845.1153987,584.81713843)(845.14539867,584.9421383)(845.15540527,585.07214111)
\curveto(845.17539864,585.21213803)(845.20539861,585.35213789)(845.24540527,585.49214111)
}
}
{
\newrgbcolor{curcolor}{0 0 0}
\pscustom[linestyle=none,fillstyle=solid,fillcolor=curcolor]
{
\newpath
\moveto(855.02868652,582.95714111)
\lineto(855.02868652,582.56714111)
\curveto(855.02867865,582.4471408)(855.00367867,582.3471409)(854.95368652,582.26714111)
\curveto(854.90367877,582.19714105)(854.81867886,582.15714109)(854.69868652,582.14714111)
\lineto(854.35368652,582.14714111)
\curveto(854.29367938,582.1471411)(854.23367944,582.1421411)(854.17368652,582.13214111)
\curveto(854.12367955,582.13214111)(854.0786796,582.1421411)(854.03868652,582.16214111)
\curveto(853.94867973,582.18214106)(853.88867979,582.22214102)(853.85868652,582.28214111)
\curveto(853.81867986,582.33214091)(853.79367988,582.39214085)(853.78368652,582.46214111)
\curveto(853.78367989,582.53214071)(853.76867991,582.60214064)(853.73868652,582.67214111)
\curveto(853.72867995,582.69214055)(853.71367996,582.70714054)(853.69368652,582.71714111)
\curveto(853.68367999,582.73714051)(853.66868001,582.75714049)(853.64868652,582.77714111)
\curveto(853.54868013,582.78714046)(853.46868021,582.76714048)(853.40868652,582.71714111)
\curveto(853.35868032,582.66714058)(853.30368037,582.61714063)(853.24368652,582.56714111)
\curveto(853.04368063,582.41714083)(852.84368083,582.30214094)(852.64368652,582.22214111)
\curveto(852.46368121,582.1421411)(852.25368142,582.08214116)(852.01368652,582.04214111)
\curveto(851.78368189,582.00214124)(851.54368213,581.98214126)(851.29368652,581.98214111)
\curveto(851.05368262,581.97214127)(850.81368286,581.98714126)(850.57368652,582.02714111)
\curveto(850.33368334,582.05714119)(850.12368355,582.11214113)(849.94368652,582.19214111)
\curveto(849.42368425,582.41214083)(849.00368467,582.70714054)(848.68368652,583.07714111)
\curveto(848.36368531,583.45713979)(848.11368556,583.92713932)(847.93368652,584.48714111)
\curveto(847.89368578,584.57713867)(847.86368581,584.66713858)(847.84368652,584.75714111)
\curveto(847.83368584,584.85713839)(847.81368586,584.95713829)(847.78368652,585.05714111)
\curveto(847.7736859,585.10713814)(847.76868591,585.15713809)(847.76868652,585.20714111)
\curveto(847.76868591,585.25713799)(847.76368591,585.30713794)(847.75368652,585.35714111)
\curveto(847.73368594,585.40713784)(847.72368595,585.45713779)(847.72368652,585.50714111)
\curveto(847.73368594,585.56713768)(847.73368594,585.62213762)(847.72368652,585.67214111)
\lineto(847.72368652,585.82214111)
\curveto(847.70368597,585.87213737)(847.69368598,585.93713731)(847.69368652,586.01714111)
\curveto(847.69368598,586.09713715)(847.70368597,586.16213708)(847.72368652,586.21214111)
\lineto(847.72368652,586.37714111)
\curveto(847.74368593,586.4471368)(847.74868593,586.51713673)(847.73868652,586.58714111)
\curveto(847.73868594,586.66713658)(847.74868593,586.7421365)(847.76868652,586.81214111)
\curveto(847.7786859,586.86213638)(847.78368589,586.90713634)(847.78368652,586.94714111)
\curveto(847.78368589,586.98713626)(847.78868589,587.03213621)(847.79868652,587.08214111)
\curveto(847.82868585,587.18213606)(847.85368582,587.27713597)(847.87368652,587.36714111)
\curveto(847.89368578,587.46713578)(847.91868576,587.56213568)(847.94868652,587.65214111)
\curveto(848.0786856,588.03213521)(848.24368543,588.37213487)(848.44368652,588.67214111)
\curveto(848.65368502,588.98213426)(848.90368477,589.23713401)(849.19368652,589.43714111)
\curveto(849.36368431,589.55713369)(849.53868414,589.65713359)(849.71868652,589.73714111)
\curveto(849.90868377,589.81713343)(850.11368356,589.88713336)(850.33368652,589.94714111)
\curveto(850.40368327,589.95713329)(850.46868321,589.96713328)(850.52868652,589.97714111)
\curveto(850.59868308,589.98713326)(850.66868301,590.00213324)(850.73868652,590.02214111)
\lineto(850.88868652,590.02214111)
\curveto(850.96868271,590.0421332)(851.08368259,590.05213319)(851.23368652,590.05214111)
\curveto(851.39368228,590.05213319)(851.51368216,590.0421332)(851.59368652,590.02214111)
\curveto(851.63368204,590.01213323)(851.68868199,590.00713324)(851.75868652,590.00714111)
\curveto(851.86868181,589.97713327)(851.9786817,589.95213329)(852.08868652,589.93214111)
\curveto(852.19868148,589.92213332)(852.30368137,589.89213335)(852.40368652,589.84214111)
\curveto(852.55368112,589.78213346)(852.69368098,589.71713353)(852.82368652,589.64714111)
\curveto(852.96368071,589.57713367)(853.09368058,589.49713375)(853.21368652,589.40714111)
\curveto(853.2736804,589.35713389)(853.33368034,589.30213394)(853.39368652,589.24214111)
\curveto(853.46368021,589.19213405)(853.55368012,589.17713407)(853.66368652,589.19714111)
\curveto(853.68367999,589.22713402)(853.69867998,589.25213399)(853.70868652,589.27214111)
\curveto(853.72867995,589.29213395)(853.74367993,589.32213392)(853.75368652,589.36214111)
\curveto(853.78367989,589.45213379)(853.79367988,589.56713368)(853.78368652,589.70714111)
\lineto(853.78368652,590.08214111)
\lineto(853.78368652,591.80714111)
\lineto(853.78368652,592.27214111)
\curveto(853.78367989,592.45213079)(853.80867987,592.58213066)(853.85868652,592.66214111)
\curveto(853.89867978,592.73213051)(853.95867972,592.77713047)(854.03868652,592.79714111)
\curveto(854.05867962,592.79713045)(854.08367959,592.79713045)(854.11368652,592.79714111)
\curveto(854.14367953,592.80713044)(854.16867951,592.81213043)(854.18868652,592.81214111)
\curveto(854.32867935,592.82213042)(854.4736792,592.82213042)(854.62368652,592.81214111)
\curveto(854.78367889,592.81213043)(854.89367878,592.77213047)(854.95368652,592.69214111)
\curveto(855.00367867,592.61213063)(855.02867865,592.51213073)(855.02868652,592.39214111)
\lineto(855.02868652,592.01714111)
\lineto(855.02868652,582.95714111)
\moveto(853.81368652,585.79214111)
\curveto(853.83367984,585.8421374)(853.84367983,585.90713734)(853.84368652,585.98714111)
\curveto(853.84367983,586.07713717)(853.83367984,586.1471371)(853.81368652,586.19714111)
\lineto(853.81368652,586.42214111)
\curveto(853.79367988,586.51213673)(853.7786799,586.60213664)(853.76868652,586.69214111)
\curveto(853.75867992,586.79213645)(853.73867994,586.88213636)(853.70868652,586.96214111)
\curveto(853.68867999,587.0421362)(853.66868001,587.11713613)(853.64868652,587.18714111)
\curveto(853.63868004,587.25713599)(853.61868006,587.32713592)(853.58868652,587.39714111)
\curveto(853.46868021,587.69713555)(853.31368036,587.96213528)(853.12368652,588.19214111)
\curveto(852.93368074,588.42213482)(852.69368098,588.60213464)(852.40368652,588.73214111)
\curveto(852.30368137,588.78213446)(852.19868148,588.81713443)(852.08868652,588.83714111)
\curveto(851.98868169,588.86713438)(851.8786818,588.89213435)(851.75868652,588.91214111)
\curveto(851.678682,588.93213431)(851.58868209,588.9421343)(851.48868652,588.94214111)
\lineto(851.21868652,588.94214111)
\curveto(851.16868251,588.93213431)(851.12368255,588.92213432)(851.08368652,588.91214111)
\lineto(850.94868652,588.91214111)
\curveto(850.86868281,588.89213435)(850.78368289,588.87213437)(850.69368652,588.85214111)
\curveto(850.61368306,588.83213441)(850.53368314,588.80713444)(850.45368652,588.77714111)
\curveto(850.13368354,588.63713461)(849.8736838,588.43213481)(849.67368652,588.16214111)
\curveto(849.48368419,587.90213534)(849.32868435,587.59713565)(849.20868652,587.24714111)
\curveto(849.16868451,587.13713611)(849.13868454,587.02213622)(849.11868652,586.90214111)
\curveto(849.10868457,586.79213645)(849.09368458,586.68213656)(849.07368652,586.57214111)
\curveto(849.0736846,586.53213671)(849.06868461,586.49213675)(849.05868652,586.45214111)
\lineto(849.05868652,586.34714111)
\curveto(849.03868464,586.29713695)(849.02868465,586.242137)(849.02868652,586.18214111)
\curveto(849.03868464,586.12213712)(849.04368463,586.06713718)(849.04368652,586.01714111)
\lineto(849.04368652,585.68714111)
\curveto(849.04368463,585.58713766)(849.05368462,585.49213775)(849.07368652,585.40214111)
\curveto(849.08368459,585.37213787)(849.08868459,585.32213792)(849.08868652,585.25214111)
\curveto(849.10868457,585.18213806)(849.12368455,585.11213813)(849.13368652,585.04214111)
\lineto(849.19368652,584.83214111)
\curveto(849.30368437,584.48213876)(849.45368422,584.18213906)(849.64368652,583.93214111)
\curveto(849.83368384,583.68213956)(850.0736836,583.47713977)(850.36368652,583.31714111)
\curveto(850.45368322,583.26713998)(850.54368313,583.22714002)(850.63368652,583.19714111)
\curveto(850.72368295,583.16714008)(850.82368285,583.13714011)(850.93368652,583.10714111)
\curveto(850.98368269,583.08714016)(851.03368264,583.08214016)(851.08368652,583.09214111)
\curveto(851.14368253,583.10214014)(851.19868248,583.09714015)(851.24868652,583.07714111)
\curveto(851.28868239,583.06714018)(851.32868235,583.06214018)(851.36868652,583.06214111)
\lineto(851.50368652,583.06214111)
\lineto(851.63868652,583.06214111)
\curveto(851.66868201,583.07214017)(851.71868196,583.07714017)(851.78868652,583.07714111)
\curveto(851.86868181,583.09714015)(851.94868173,583.11214013)(852.02868652,583.12214111)
\curveto(852.10868157,583.1421401)(852.18368149,583.16714008)(852.25368652,583.19714111)
\curveto(852.58368109,583.33713991)(852.84868083,583.51213973)(853.04868652,583.72214111)
\curveto(853.25868042,583.9421393)(853.43368024,584.21713903)(853.57368652,584.54714111)
\curveto(853.62368005,584.65713859)(853.65868002,584.76713848)(853.67868652,584.87714111)
\curveto(853.69867998,584.98713826)(853.72367995,585.09713815)(853.75368652,585.20714111)
\curveto(853.7736799,585.247138)(853.78367989,585.28213796)(853.78368652,585.31214111)
\curveto(853.78367989,585.35213789)(853.78867989,585.39213785)(853.79868652,585.43214111)
\curveto(853.80867987,585.49213775)(853.80867987,585.55213769)(853.79868652,585.61214111)
\curveto(853.79867988,585.67213757)(853.80367987,585.73213751)(853.81368652,585.79214111)
}
}
{
\newrgbcolor{curcolor}{0 0 0}
\pscustom[linestyle=none,fillstyle=solid,fillcolor=curcolor]
{
\newpath
\moveto(863.72493652,586.31714111)
\curveto(863.74492884,586.21713703)(863.74492884,586.10213714)(863.72493652,585.97214111)
\curveto(863.71492887,585.85213739)(863.6849289,585.76713748)(863.63493652,585.71714111)
\curveto(863.584929,585.67713757)(863.50992907,585.6471376)(863.40993652,585.62714111)
\curveto(863.31992926,585.61713763)(863.21492937,585.61213763)(863.09493652,585.61214111)
\lineto(862.73493652,585.61214111)
\curveto(862.61492997,585.62213762)(862.50993007,585.62713762)(862.41993652,585.62714111)
\lineto(858.57993652,585.62714111)
\curveto(858.49993408,585.62713762)(858.41993416,585.62213762)(858.33993652,585.61214111)
\curveto(858.25993432,585.61213763)(858.19493439,585.59713765)(858.14493652,585.56714111)
\curveto(858.10493448,585.5471377)(858.06493452,585.50713774)(858.02493652,585.44714111)
\curveto(858.00493458,585.41713783)(857.9849346,585.37213787)(857.96493652,585.31214111)
\curveto(857.94493464,585.26213798)(857.94493464,585.21213803)(857.96493652,585.16214111)
\curveto(857.97493461,585.11213813)(857.9799346,585.06713818)(857.97993652,585.02714111)
\curveto(857.9799346,584.98713826)(857.9849346,584.9471383)(857.99493652,584.90714111)
\curveto(858.01493457,584.82713842)(858.03493455,584.7421385)(858.05493652,584.65214111)
\curveto(858.07493451,584.57213867)(858.10493448,584.49213875)(858.14493652,584.41214111)
\curveto(858.37493421,583.87213937)(858.75493383,583.48713976)(859.28493652,583.25714111)
\curveto(859.34493324,583.22714002)(859.40993317,583.20214004)(859.47993652,583.18214111)
\lineto(859.68993652,583.12214111)
\curveto(859.71993286,583.11214013)(859.76993281,583.10714014)(859.83993652,583.10714111)
\curveto(859.9799326,583.06714018)(860.16493242,583.0471402)(860.39493652,583.04714111)
\curveto(860.62493196,583.0471402)(860.80993177,583.06714018)(860.94993652,583.10714111)
\curveto(861.08993149,583.1471401)(861.21493137,583.18714006)(861.32493652,583.22714111)
\curveto(861.44493114,583.27713997)(861.55493103,583.33713991)(861.65493652,583.40714111)
\curveto(861.76493082,583.47713977)(861.85993072,583.55713969)(861.93993652,583.64714111)
\curveto(862.01993056,583.7471395)(862.08993049,583.85213939)(862.14993652,583.96214111)
\curveto(862.20993037,584.06213918)(862.25993032,584.16713908)(862.29993652,584.27714111)
\curveto(862.34993023,584.38713886)(862.42993015,584.46713878)(862.53993652,584.51714111)
\curveto(862.57993,584.53713871)(862.64492994,584.55213869)(862.73493652,584.56214111)
\curveto(862.82492976,584.57213867)(862.91492967,584.57213867)(863.00493652,584.56214111)
\curveto(863.09492949,584.56213868)(863.1799294,584.55713869)(863.25993652,584.54714111)
\curveto(863.33992924,584.53713871)(863.39492919,584.51713873)(863.42493652,584.48714111)
\curveto(863.52492906,584.41713883)(863.54992903,584.30213894)(863.49993652,584.14214111)
\curveto(863.41992916,583.87213937)(863.31492927,583.63213961)(863.18493652,583.42214111)
\curveto(862.9849296,583.10214014)(862.75492983,582.83714041)(862.49493652,582.62714111)
\curveto(862.24493034,582.42714082)(861.92493066,582.26214098)(861.53493652,582.13214111)
\curveto(861.43493115,582.09214115)(861.33493125,582.06714118)(861.23493652,582.05714111)
\curveto(861.13493145,582.03714121)(861.02993155,582.01714123)(860.91993652,581.99714111)
\curveto(860.86993171,581.98714126)(860.81993176,581.98214126)(860.76993652,581.98214111)
\curveto(860.72993185,581.98214126)(860.6849319,581.97714127)(860.63493652,581.96714111)
\lineto(860.48493652,581.96714111)
\curveto(860.43493215,581.95714129)(860.37493221,581.95214129)(860.30493652,581.95214111)
\curveto(860.24493234,581.95214129)(860.19493239,581.95714129)(860.15493652,581.96714111)
\lineto(860.01993652,581.96714111)
\curveto(859.96993261,581.97714127)(859.92493266,581.98214126)(859.88493652,581.98214111)
\curveto(859.84493274,581.98214126)(859.80493278,581.98714126)(859.76493652,581.99714111)
\curveto(859.71493287,582.00714124)(859.65993292,582.01714123)(859.59993652,582.02714111)
\curveto(859.53993304,582.02714122)(859.4849331,582.03214121)(859.43493652,582.04214111)
\curveto(859.34493324,582.06214118)(859.25493333,582.08714116)(859.16493652,582.11714111)
\curveto(859.07493351,582.13714111)(858.98993359,582.16214108)(858.90993652,582.19214111)
\curveto(858.86993371,582.21214103)(858.83493375,582.22214102)(858.80493652,582.22214111)
\curveto(858.77493381,582.23214101)(858.73993384,582.247141)(858.69993652,582.26714111)
\curveto(858.54993403,582.33714091)(858.38993419,582.42214082)(858.21993652,582.52214111)
\curveto(857.92993465,582.71214053)(857.6799349,582.9421403)(857.46993652,583.21214111)
\curveto(857.26993531,583.49213975)(857.09993548,583.80213944)(856.95993652,584.14214111)
\curveto(856.90993567,584.25213899)(856.86993571,584.36713888)(856.83993652,584.48714111)
\curveto(856.81993576,584.60713864)(856.78993579,584.72713852)(856.74993652,584.84714111)
\curveto(856.73993584,584.88713836)(856.73493585,584.92213832)(856.73493652,584.95214111)
\curveto(856.73493585,584.98213826)(856.72993585,585.02213822)(856.71993652,585.07214111)
\curveto(856.69993588,585.15213809)(856.6849359,585.23713801)(856.67493652,585.32714111)
\curveto(856.66493592,585.41713783)(856.64993593,585.50713774)(856.62993652,585.59714111)
\lineto(856.62993652,585.80714111)
\curveto(856.61993596,585.8471374)(856.60993597,585.90213734)(856.59993652,585.97214111)
\curveto(856.59993598,586.05213719)(856.60493598,586.11713713)(856.61493652,586.16714111)
\lineto(856.61493652,586.33214111)
\curveto(856.63493595,586.38213686)(856.63993594,586.43213681)(856.62993652,586.48214111)
\curveto(856.62993595,586.5421367)(856.63493595,586.59713665)(856.64493652,586.64714111)
\curveto(856.6849359,586.80713644)(856.71493587,586.96713628)(856.73493652,587.12714111)
\curveto(856.76493582,587.28713596)(856.80993577,587.43713581)(856.86993652,587.57714111)
\curveto(856.91993566,587.68713556)(856.96493562,587.79713545)(857.00493652,587.90714111)
\curveto(857.05493553,588.02713522)(857.10993547,588.1421351)(857.16993652,588.25214111)
\curveto(857.38993519,588.60213464)(857.63993494,588.90213434)(857.91993652,589.15214111)
\curveto(858.19993438,589.41213383)(858.54493404,589.62713362)(858.95493652,589.79714111)
\curveto(859.07493351,589.8471334)(859.19493339,589.88213336)(859.31493652,589.90214111)
\curveto(859.44493314,589.93213331)(859.579933,589.96213328)(859.71993652,589.99214111)
\curveto(859.76993281,590.00213324)(859.81493277,590.00713324)(859.85493652,590.00714111)
\curveto(859.89493269,590.01713323)(859.93993264,590.02213322)(859.98993652,590.02214111)
\curveto(860.00993257,590.03213321)(860.03493255,590.03213321)(860.06493652,590.02214111)
\curveto(860.09493249,590.01213323)(860.11993246,590.01713323)(860.13993652,590.03714111)
\curveto(860.55993202,590.0471332)(860.92493166,590.00213324)(861.23493652,589.90214111)
\curveto(861.54493104,589.81213343)(861.82493076,589.68713356)(862.07493652,589.52714111)
\curveto(862.12493046,589.50713374)(862.16493042,589.47713377)(862.19493652,589.43714111)
\curveto(862.22493036,589.40713384)(862.25993032,589.38213386)(862.29993652,589.36214111)
\curveto(862.3799302,589.30213394)(862.45993012,589.23213401)(862.53993652,589.15214111)
\curveto(862.62992995,589.07213417)(862.70492988,588.99213425)(862.76493652,588.91214111)
\curveto(862.92492966,588.70213454)(863.05992952,588.50213474)(863.16993652,588.31214111)
\curveto(863.23992934,588.20213504)(863.29492929,588.08213516)(863.33493652,587.95214111)
\curveto(863.37492921,587.82213542)(863.41992916,587.69213555)(863.46993652,587.56214111)
\curveto(863.51992906,587.43213581)(863.55492903,587.29713595)(863.57493652,587.15714111)
\curveto(863.60492898,587.01713623)(863.63992894,586.87713637)(863.67993652,586.73714111)
\curveto(863.68992889,586.66713658)(863.69492889,586.59713665)(863.69493652,586.52714111)
\lineto(863.72493652,586.31714111)
\moveto(862.26993652,586.82714111)
\curveto(862.29993028,586.86713638)(862.32493026,586.91713633)(862.34493652,586.97714111)
\curveto(862.36493022,587.0471362)(862.36493022,587.11713613)(862.34493652,587.18714111)
\curveto(862.2849303,587.40713584)(862.19993038,587.61213563)(862.08993652,587.80214111)
\curveto(861.94993063,588.03213521)(861.79493079,588.22713502)(861.62493652,588.38714111)
\curveto(861.45493113,588.5471347)(861.23493135,588.68213456)(860.96493652,588.79214111)
\curveto(860.89493169,588.81213443)(860.82493176,588.82713442)(860.75493652,588.83714111)
\curveto(860.6849319,588.85713439)(860.60993197,588.87713437)(860.52993652,588.89714111)
\curveto(860.44993213,588.91713433)(860.36493222,588.92713432)(860.27493652,588.92714111)
\lineto(860.01993652,588.92714111)
\curveto(859.98993259,588.90713434)(859.95493263,588.89713435)(859.91493652,588.89714111)
\curveto(859.87493271,588.90713434)(859.83993274,588.90713434)(859.80993652,588.89714111)
\lineto(859.56993652,588.83714111)
\curveto(859.49993308,588.82713442)(859.42993315,588.81213443)(859.35993652,588.79214111)
\curveto(859.06993351,588.67213457)(858.83493375,588.52213472)(858.65493652,588.34214111)
\curveto(858.4849341,588.16213508)(858.32993425,587.93713531)(858.18993652,587.66714111)
\curveto(858.15993442,587.61713563)(858.12993445,587.55213569)(858.09993652,587.47214111)
\curveto(858.06993451,587.40213584)(858.04493454,587.32213592)(858.02493652,587.23214111)
\curveto(858.00493458,587.1421361)(857.99993458,587.05713619)(858.00993652,586.97714111)
\curveto(858.01993456,586.89713635)(858.05493453,586.83713641)(858.11493652,586.79714111)
\curveto(858.19493439,586.73713651)(858.32993425,586.70713654)(858.51993652,586.70714111)
\curveto(858.71993386,586.71713653)(858.88993369,586.72213652)(859.02993652,586.72214111)
\lineto(861.30993652,586.72214111)
\curveto(861.45993112,586.72213652)(861.63993094,586.71713653)(861.84993652,586.70714111)
\curveto(862.05993052,586.70713654)(862.19993038,586.7471365)(862.26993652,586.82714111)
}
}
{
\newrgbcolor{curcolor}{0 0 0}
\pscustom[linestyle=none,fillstyle=solid,fillcolor=curcolor]
{
\newpath
\moveto(868.67657715,590.05214111)
\curveto(868.90657236,590.05213319)(869.03657223,589.99213325)(869.06657715,589.87214111)
\curveto(869.09657217,589.76213348)(869.11157215,589.59713365)(869.11157715,589.37714111)
\lineto(869.11157715,589.09214111)
\curveto(869.11157215,589.00213424)(869.08657218,588.92713432)(869.03657715,588.86714111)
\curveto(868.97657229,588.78713446)(868.89157237,588.7421345)(868.78157715,588.73214111)
\curveto(868.67157259,588.73213451)(868.5615727,588.71713453)(868.45157715,588.68714111)
\curveto(868.31157295,588.65713459)(868.17657309,588.62713462)(868.04657715,588.59714111)
\curveto(867.92657334,588.56713468)(867.81157345,588.52713472)(867.70157715,588.47714111)
\curveto(867.41157385,588.3471349)(867.17657409,588.16713508)(866.99657715,587.93714111)
\curveto(866.81657445,587.71713553)(866.6615746,587.46213578)(866.53157715,587.17214111)
\curveto(866.49157477,587.06213618)(866.4615748,586.9471363)(866.44157715,586.82714111)
\curveto(866.42157484,586.71713653)(866.39657487,586.60213664)(866.36657715,586.48214111)
\curveto(866.35657491,586.43213681)(866.35157491,586.38213686)(866.35157715,586.33214111)
\curveto(866.3615749,586.28213696)(866.3615749,586.23213701)(866.35157715,586.18214111)
\curveto(866.32157494,586.06213718)(866.30657496,585.92213732)(866.30657715,585.76214111)
\curveto(866.31657495,585.61213763)(866.32157494,585.46713778)(866.32157715,585.32714111)
\lineto(866.32157715,583.48214111)
\lineto(866.32157715,583.13714111)
\curveto(866.32157494,583.01714023)(866.31657495,582.90214034)(866.30657715,582.79214111)
\curveto(866.29657497,582.68214056)(866.29157497,582.58714066)(866.29157715,582.50714111)
\curveto(866.30157496,582.42714082)(866.28157498,582.35714089)(866.23157715,582.29714111)
\curveto(866.18157508,582.22714102)(866.10157516,582.18714106)(865.99157715,582.17714111)
\curveto(865.89157537,582.16714108)(865.78157548,582.16214108)(865.66157715,582.16214111)
\lineto(865.39157715,582.16214111)
\curveto(865.34157592,582.18214106)(865.29157597,582.19714105)(865.24157715,582.20714111)
\curveto(865.20157606,582.22714102)(865.17157609,582.25214099)(865.15157715,582.28214111)
\curveto(865.10157616,582.35214089)(865.07157619,582.43714081)(865.06157715,582.53714111)
\lineto(865.06157715,582.86714111)
\lineto(865.06157715,584.02214111)
\lineto(865.06157715,588.17714111)
\lineto(865.06157715,589.21214111)
\lineto(865.06157715,589.51214111)
\curveto(865.07157619,589.61213363)(865.10157616,589.69713355)(865.15157715,589.76714111)
\curveto(865.18157608,589.80713344)(865.23157603,589.83713341)(865.30157715,589.85714111)
\curveto(865.38157588,589.87713337)(865.4665758,589.88713336)(865.55657715,589.88714111)
\curveto(865.64657562,589.89713335)(865.73657553,589.89713335)(865.82657715,589.88714111)
\curveto(865.91657535,589.87713337)(865.98657528,589.86213338)(866.03657715,589.84214111)
\curveto(866.11657515,589.81213343)(866.1665751,589.75213349)(866.18657715,589.66214111)
\curveto(866.21657505,589.58213366)(866.23157503,589.49213375)(866.23157715,589.39214111)
\lineto(866.23157715,589.09214111)
\curveto(866.23157503,588.99213425)(866.25157501,588.90213434)(866.29157715,588.82214111)
\curveto(866.30157496,588.80213444)(866.31157495,588.78713446)(866.32157715,588.77714111)
\lineto(866.36657715,588.73214111)
\curveto(866.47657479,588.73213451)(866.5665747,588.77713447)(866.63657715,588.86714111)
\curveto(866.70657456,588.96713428)(866.7665745,589.0471342)(866.81657715,589.10714111)
\lineto(866.90657715,589.19714111)
\curveto(866.99657427,589.30713394)(867.12157414,589.42213382)(867.28157715,589.54214111)
\curveto(867.44157382,589.66213358)(867.59157367,589.75213349)(867.73157715,589.81214111)
\curveto(867.82157344,589.86213338)(867.91657335,589.89713335)(868.01657715,589.91714111)
\curveto(868.11657315,589.9471333)(868.22157304,589.97713327)(868.33157715,590.00714111)
\curveto(868.39157287,590.01713323)(868.45157281,590.02213322)(868.51157715,590.02214111)
\curveto(868.57157269,590.03213321)(868.62657264,590.0421332)(868.67657715,590.05214111)
}
}
{
\newrgbcolor{curcolor}{0 0 0}
\pscustom[linestyle=none,fillstyle=solid,fillcolor=curcolor]
{
\newpath
\moveto(876.92634277,582.70214111)
\curveto(876.95633494,582.5421407)(876.94133496,582.40714084)(876.88134277,582.29714111)
\curveto(876.82133508,582.19714105)(876.74133516,582.12214112)(876.64134277,582.07214111)
\curveto(876.59133531,582.05214119)(876.53633536,582.0421412)(876.47634277,582.04214111)
\curveto(876.42633547,582.0421412)(876.37133553,582.03214121)(876.31134277,582.01214111)
\curveto(876.09133581,581.96214128)(875.87133603,581.97714127)(875.65134277,582.05714111)
\curveto(875.44133646,582.12714112)(875.2963366,582.21714103)(875.21634277,582.32714111)
\curveto(875.16633673,582.39714085)(875.12133678,582.47714077)(875.08134277,582.56714111)
\curveto(875.04133686,582.66714058)(874.99133691,582.7471405)(874.93134277,582.80714111)
\curveto(874.91133699,582.82714042)(874.88633701,582.8471404)(874.85634277,582.86714111)
\curveto(874.83633706,582.88714036)(874.80633709,582.89214035)(874.76634277,582.88214111)
\curveto(874.65633724,582.85214039)(874.55133735,582.79714045)(874.45134277,582.71714111)
\curveto(874.36133754,582.63714061)(874.27133763,582.56714068)(874.18134277,582.50714111)
\curveto(874.05133785,582.42714082)(873.91133799,582.35214089)(873.76134277,582.28214111)
\curveto(873.61133829,582.22214102)(873.45133845,582.16714108)(873.28134277,582.11714111)
\curveto(873.18133872,582.08714116)(873.07133883,582.06714118)(872.95134277,582.05714111)
\curveto(872.84133906,582.0471412)(872.73133917,582.03214121)(872.62134277,582.01214111)
\curveto(872.57133933,582.00214124)(872.52633937,581.99714125)(872.48634277,581.99714111)
\lineto(872.38134277,581.99714111)
\curveto(872.27133963,581.97714127)(872.16633973,581.97714127)(872.06634277,581.99714111)
\lineto(871.93134277,581.99714111)
\curveto(871.88134002,582.00714124)(871.83134007,582.01214123)(871.78134277,582.01214111)
\curveto(871.73134017,582.01214123)(871.68634021,582.02214122)(871.64634277,582.04214111)
\curveto(871.60634029,582.05214119)(871.57134033,582.05714119)(871.54134277,582.05714111)
\curveto(871.52134038,582.0471412)(871.4963404,582.0471412)(871.46634277,582.05714111)
\lineto(871.22634277,582.11714111)
\curveto(871.14634075,582.12714112)(871.07134083,582.1471411)(871.00134277,582.17714111)
\curveto(870.7013412,582.30714094)(870.45634144,582.45214079)(870.26634277,582.61214111)
\curveto(870.08634181,582.78214046)(869.93634196,583.01714023)(869.81634277,583.31714111)
\curveto(869.72634217,583.53713971)(869.68134222,583.80213944)(869.68134277,584.11214111)
\lineto(869.68134277,584.42714111)
\curveto(869.69134221,584.47713877)(869.6963422,584.52713872)(869.69634277,584.57714111)
\lineto(869.72634277,584.75714111)
\lineto(869.84634277,585.08714111)
\curveto(869.88634201,585.19713805)(869.93634196,585.29713795)(869.99634277,585.38714111)
\curveto(870.17634172,585.67713757)(870.42134148,585.89213735)(870.73134277,586.03214111)
\curveto(871.04134086,586.17213707)(871.38134052,586.29713695)(871.75134277,586.40714111)
\curveto(871.89134001,586.4471368)(872.03633986,586.47713677)(872.18634277,586.49714111)
\curveto(872.33633956,586.51713673)(872.48633941,586.5421367)(872.63634277,586.57214111)
\curveto(872.70633919,586.59213665)(872.77133913,586.60213664)(872.83134277,586.60214111)
\curveto(872.901339,586.60213664)(872.97633892,586.61213663)(873.05634277,586.63214111)
\curveto(873.12633877,586.65213659)(873.1963387,586.66213658)(873.26634277,586.66214111)
\curveto(873.33633856,586.67213657)(873.41133849,586.68713656)(873.49134277,586.70714111)
\curveto(873.74133816,586.76713648)(873.97633792,586.81713643)(874.19634277,586.85714111)
\curveto(874.41633748,586.90713634)(874.59133731,587.02213622)(874.72134277,587.20214111)
\curveto(874.78133712,587.28213596)(874.83133707,587.38213586)(874.87134277,587.50214111)
\curveto(874.91133699,587.63213561)(874.91133699,587.77213547)(874.87134277,587.92214111)
\curveto(874.81133709,588.16213508)(874.72133718,588.35213489)(874.60134277,588.49214111)
\curveto(874.49133741,588.63213461)(874.33133757,588.7421345)(874.12134277,588.82214111)
\curveto(874.0013379,588.87213437)(873.85633804,588.90713434)(873.68634277,588.92714111)
\curveto(873.52633837,588.9471343)(873.35633854,588.95713429)(873.17634277,588.95714111)
\curveto(872.9963389,588.95713429)(872.82133908,588.9471343)(872.65134277,588.92714111)
\curveto(872.48133942,588.90713434)(872.33633956,588.87713437)(872.21634277,588.83714111)
\curveto(872.04633985,588.77713447)(871.88134002,588.69213455)(871.72134277,588.58214111)
\curveto(871.64134026,588.52213472)(871.56634033,588.4421348)(871.49634277,588.34214111)
\curveto(871.43634046,588.25213499)(871.38134052,588.15213509)(871.33134277,588.04214111)
\curveto(871.3013406,587.96213528)(871.27134063,587.87713537)(871.24134277,587.78714111)
\curveto(871.22134068,587.69713555)(871.17634072,587.62713562)(871.10634277,587.57714111)
\curveto(871.06634083,587.5471357)(870.9963409,587.52213572)(870.89634277,587.50214111)
\curveto(870.80634109,587.49213575)(870.71134119,587.48713576)(870.61134277,587.48714111)
\curveto(870.51134139,587.48713576)(870.41134149,587.49213575)(870.31134277,587.50214111)
\curveto(870.22134168,587.52213572)(870.15634174,587.5471357)(870.11634277,587.57714111)
\curveto(870.07634182,587.60713564)(870.04634185,587.65713559)(870.02634277,587.72714111)
\curveto(870.00634189,587.79713545)(870.00634189,587.87213537)(870.02634277,587.95214111)
\curveto(870.05634184,588.08213516)(870.08634181,588.20213504)(870.11634277,588.31214111)
\curveto(870.15634174,588.43213481)(870.2013417,588.5471347)(870.25134277,588.65714111)
\curveto(870.44134146,589.00713424)(870.68134122,589.27713397)(870.97134277,589.46714111)
\curveto(871.26134064,589.66713358)(871.62134028,589.82713342)(872.05134277,589.94714111)
\curveto(872.15133975,589.96713328)(872.25133965,589.98213326)(872.35134277,589.99214111)
\curveto(872.46133944,590.00213324)(872.57133933,590.01713323)(872.68134277,590.03714111)
\curveto(872.72133918,590.0471332)(872.78633911,590.0471332)(872.87634277,590.03714111)
\curveto(872.96633893,590.03713321)(873.02133888,590.0471332)(873.04134277,590.06714111)
\curveto(873.74133816,590.07713317)(874.35133755,589.99713325)(874.87134277,589.82714111)
\curveto(875.39133651,589.65713359)(875.75633614,589.33213391)(875.96634277,588.85214111)
\curveto(876.05633584,588.65213459)(876.10633579,588.41713483)(876.11634277,588.14714111)
\curveto(876.13633576,587.88713536)(876.14633575,587.61213563)(876.14634277,587.32214111)
\lineto(876.14634277,584.00714111)
\curveto(876.14633575,583.86713938)(876.15133575,583.73213951)(876.16134277,583.60214111)
\curveto(876.17133573,583.47213977)(876.2013357,583.36713988)(876.25134277,583.28714111)
\curveto(876.3013356,583.21714003)(876.36633553,583.16714008)(876.44634277,583.13714111)
\curveto(876.53633536,583.09714015)(876.62133528,583.06714018)(876.70134277,583.04714111)
\curveto(876.78133512,583.03714021)(876.84133506,582.99214025)(876.88134277,582.91214111)
\curveto(876.901335,582.88214036)(876.91133499,582.85214039)(876.91134277,582.82214111)
\curveto(876.91133499,582.79214045)(876.91633498,582.75214049)(876.92634277,582.70214111)
\moveto(874.78134277,584.36714111)
\curveto(874.84133706,584.50713874)(874.87133703,584.66713858)(874.87134277,584.84714111)
\curveto(874.88133702,585.03713821)(874.88633701,585.23213801)(874.88634277,585.43214111)
\curveto(874.88633701,585.5421377)(874.88133702,585.6421376)(874.87134277,585.73214111)
\curveto(874.86133704,585.82213742)(874.82133708,585.89213735)(874.75134277,585.94214111)
\curveto(874.72133718,585.96213728)(874.65133725,585.97213727)(874.54134277,585.97214111)
\curveto(874.52133738,585.95213729)(874.48633741,585.9421373)(874.43634277,585.94214111)
\curveto(874.38633751,585.9421373)(874.34133756,585.93213731)(874.30134277,585.91214111)
\curveto(874.22133768,585.89213735)(874.13133777,585.87213737)(874.03134277,585.85214111)
\lineto(873.73134277,585.79214111)
\curveto(873.7013382,585.79213745)(873.66633823,585.78713746)(873.62634277,585.77714111)
\lineto(873.52134277,585.77714111)
\curveto(873.37133853,585.73713751)(873.20633869,585.71213753)(873.02634277,585.70214111)
\curveto(872.85633904,585.70213754)(872.6963392,585.68213756)(872.54634277,585.64214111)
\curveto(872.46633943,585.62213762)(872.39133951,585.60213764)(872.32134277,585.58214111)
\curveto(872.26133964,585.57213767)(872.19133971,585.55713769)(872.11134277,585.53714111)
\curveto(871.95133995,585.48713776)(871.8013401,585.42213782)(871.66134277,585.34214111)
\curveto(871.52134038,585.27213797)(871.4013405,585.18213806)(871.30134277,585.07214111)
\curveto(871.2013407,584.96213828)(871.12634077,584.82713842)(871.07634277,584.66714111)
\curveto(871.02634087,584.51713873)(871.00634089,584.33213891)(871.01634277,584.11214111)
\curveto(871.01634088,584.01213923)(871.03134087,583.91713933)(871.06134277,583.82714111)
\curveto(871.1013408,583.7471395)(871.14634075,583.67213957)(871.19634277,583.60214111)
\curveto(871.27634062,583.49213975)(871.38134052,583.39713985)(871.51134277,583.31714111)
\curveto(871.64134026,583.24714)(871.78134012,583.18714006)(871.93134277,583.13714111)
\curveto(871.98133992,583.12714012)(872.03133987,583.12214012)(872.08134277,583.12214111)
\curveto(872.13133977,583.12214012)(872.18133972,583.11714013)(872.23134277,583.10714111)
\curveto(872.3013396,583.08714016)(872.38633951,583.07214017)(872.48634277,583.06214111)
\curveto(872.5963393,583.06214018)(872.68633921,583.07214017)(872.75634277,583.09214111)
\curveto(872.81633908,583.11214013)(872.87633902,583.11714013)(872.93634277,583.10714111)
\curveto(872.9963389,583.10714014)(873.05633884,583.11714013)(873.11634277,583.13714111)
\curveto(873.1963387,583.15714009)(873.27133863,583.17214007)(873.34134277,583.18214111)
\curveto(873.42133848,583.19214005)(873.4963384,583.21214003)(873.56634277,583.24214111)
\curveto(873.85633804,583.36213988)(874.1013378,583.50713974)(874.30134277,583.67714111)
\curveto(874.51133739,583.8471394)(874.67133723,584.07713917)(874.78134277,584.36714111)
}
}
{
\newrgbcolor{curcolor}{0 0 0}
\pscustom[linestyle=none,fillstyle=solid,fillcolor=curcolor]
{
\newpath
\moveto(885.0579834,582.95714111)
\lineto(885.0579834,582.56714111)
\curveto(885.05797552,582.4471408)(885.03297555,582.3471409)(884.9829834,582.26714111)
\curveto(884.93297565,582.19714105)(884.84797573,582.15714109)(884.7279834,582.14714111)
\lineto(884.3829834,582.14714111)
\curveto(884.32297626,582.1471411)(884.26297632,582.1421411)(884.2029834,582.13214111)
\curveto(884.15297643,582.13214111)(884.10797647,582.1421411)(884.0679834,582.16214111)
\curveto(883.9779766,582.18214106)(883.91797666,582.22214102)(883.8879834,582.28214111)
\curveto(883.84797673,582.33214091)(883.82297676,582.39214085)(883.8129834,582.46214111)
\curveto(883.81297677,582.53214071)(883.79797678,582.60214064)(883.7679834,582.67214111)
\curveto(883.75797682,582.69214055)(883.74297684,582.70714054)(883.7229834,582.71714111)
\curveto(883.71297687,582.73714051)(883.69797688,582.75714049)(883.6779834,582.77714111)
\curveto(883.577977,582.78714046)(883.49797708,582.76714048)(883.4379834,582.71714111)
\curveto(883.38797719,582.66714058)(883.33297725,582.61714063)(883.2729834,582.56714111)
\curveto(883.07297751,582.41714083)(882.87297771,582.30214094)(882.6729834,582.22214111)
\curveto(882.49297809,582.1421411)(882.2829783,582.08214116)(882.0429834,582.04214111)
\curveto(881.81297877,582.00214124)(881.57297901,581.98214126)(881.3229834,581.98214111)
\curveto(881.0829795,581.97214127)(880.84297974,581.98714126)(880.6029834,582.02714111)
\curveto(880.36298022,582.05714119)(880.15298043,582.11214113)(879.9729834,582.19214111)
\curveto(879.45298113,582.41214083)(879.03298155,582.70714054)(878.7129834,583.07714111)
\curveto(878.39298219,583.45713979)(878.14298244,583.92713932)(877.9629834,584.48714111)
\curveto(877.92298266,584.57713867)(877.89298269,584.66713858)(877.8729834,584.75714111)
\curveto(877.86298272,584.85713839)(877.84298274,584.95713829)(877.8129834,585.05714111)
\curveto(877.80298278,585.10713814)(877.79798278,585.15713809)(877.7979834,585.20714111)
\curveto(877.79798278,585.25713799)(877.79298279,585.30713794)(877.7829834,585.35714111)
\curveto(877.76298282,585.40713784)(877.75298283,585.45713779)(877.7529834,585.50714111)
\curveto(877.76298282,585.56713768)(877.76298282,585.62213762)(877.7529834,585.67214111)
\lineto(877.7529834,585.82214111)
\curveto(877.73298285,585.87213737)(877.72298286,585.93713731)(877.7229834,586.01714111)
\curveto(877.72298286,586.09713715)(877.73298285,586.16213708)(877.7529834,586.21214111)
\lineto(877.7529834,586.37714111)
\curveto(877.77298281,586.4471368)(877.7779828,586.51713673)(877.7679834,586.58714111)
\curveto(877.76798281,586.66713658)(877.7779828,586.7421365)(877.7979834,586.81214111)
\curveto(877.80798277,586.86213638)(877.81298277,586.90713634)(877.8129834,586.94714111)
\curveto(877.81298277,586.98713626)(877.81798276,587.03213621)(877.8279834,587.08214111)
\curveto(877.85798272,587.18213606)(877.8829827,587.27713597)(877.9029834,587.36714111)
\curveto(877.92298266,587.46713578)(877.94798263,587.56213568)(877.9779834,587.65214111)
\curveto(878.10798247,588.03213521)(878.27298231,588.37213487)(878.4729834,588.67214111)
\curveto(878.6829819,588.98213426)(878.93298165,589.23713401)(879.2229834,589.43714111)
\curveto(879.39298119,589.55713369)(879.56798101,589.65713359)(879.7479834,589.73714111)
\curveto(879.93798064,589.81713343)(880.14298044,589.88713336)(880.3629834,589.94714111)
\curveto(880.43298015,589.95713329)(880.49798008,589.96713328)(880.5579834,589.97714111)
\curveto(880.62797995,589.98713326)(880.69797988,590.00213324)(880.7679834,590.02214111)
\lineto(880.9179834,590.02214111)
\curveto(880.99797958,590.0421332)(881.11297947,590.05213319)(881.2629834,590.05214111)
\curveto(881.42297916,590.05213319)(881.54297904,590.0421332)(881.6229834,590.02214111)
\curveto(881.66297892,590.01213323)(881.71797886,590.00713324)(881.7879834,590.00714111)
\curveto(881.89797868,589.97713327)(882.00797857,589.95213329)(882.1179834,589.93214111)
\curveto(882.22797835,589.92213332)(882.33297825,589.89213335)(882.4329834,589.84214111)
\curveto(882.582978,589.78213346)(882.72297786,589.71713353)(882.8529834,589.64714111)
\curveto(882.99297759,589.57713367)(883.12297746,589.49713375)(883.2429834,589.40714111)
\curveto(883.30297728,589.35713389)(883.36297722,589.30213394)(883.4229834,589.24214111)
\curveto(883.49297709,589.19213405)(883.582977,589.17713407)(883.6929834,589.19714111)
\curveto(883.71297687,589.22713402)(883.72797685,589.25213399)(883.7379834,589.27214111)
\curveto(883.75797682,589.29213395)(883.77297681,589.32213392)(883.7829834,589.36214111)
\curveto(883.81297677,589.45213379)(883.82297676,589.56713368)(883.8129834,589.70714111)
\lineto(883.8129834,590.08214111)
\lineto(883.8129834,591.80714111)
\lineto(883.8129834,592.27214111)
\curveto(883.81297677,592.45213079)(883.83797674,592.58213066)(883.8879834,592.66214111)
\curveto(883.92797665,592.73213051)(883.98797659,592.77713047)(884.0679834,592.79714111)
\curveto(884.08797649,592.79713045)(884.11297647,592.79713045)(884.1429834,592.79714111)
\curveto(884.17297641,592.80713044)(884.19797638,592.81213043)(884.2179834,592.81214111)
\curveto(884.35797622,592.82213042)(884.50297608,592.82213042)(884.6529834,592.81214111)
\curveto(884.81297577,592.81213043)(884.92297566,592.77213047)(884.9829834,592.69214111)
\curveto(885.03297555,592.61213063)(885.05797552,592.51213073)(885.0579834,592.39214111)
\lineto(885.0579834,592.01714111)
\lineto(885.0579834,582.95714111)
\moveto(883.8429834,585.79214111)
\curveto(883.86297672,585.8421374)(883.87297671,585.90713734)(883.8729834,585.98714111)
\curveto(883.87297671,586.07713717)(883.86297672,586.1471371)(883.8429834,586.19714111)
\lineto(883.8429834,586.42214111)
\curveto(883.82297676,586.51213673)(883.80797677,586.60213664)(883.7979834,586.69214111)
\curveto(883.78797679,586.79213645)(883.76797681,586.88213636)(883.7379834,586.96214111)
\curveto(883.71797686,587.0421362)(883.69797688,587.11713613)(883.6779834,587.18714111)
\curveto(883.66797691,587.25713599)(883.64797693,587.32713592)(883.6179834,587.39714111)
\curveto(883.49797708,587.69713555)(883.34297724,587.96213528)(883.1529834,588.19214111)
\curveto(882.96297762,588.42213482)(882.72297786,588.60213464)(882.4329834,588.73214111)
\curveto(882.33297825,588.78213446)(882.22797835,588.81713443)(882.1179834,588.83714111)
\curveto(882.01797856,588.86713438)(881.90797867,588.89213435)(881.7879834,588.91214111)
\curveto(881.70797887,588.93213431)(881.61797896,588.9421343)(881.5179834,588.94214111)
\lineto(881.2479834,588.94214111)
\curveto(881.19797938,588.93213431)(881.15297943,588.92213432)(881.1129834,588.91214111)
\lineto(880.9779834,588.91214111)
\curveto(880.89797968,588.89213435)(880.81297977,588.87213437)(880.7229834,588.85214111)
\curveto(880.64297994,588.83213441)(880.56298002,588.80713444)(880.4829834,588.77714111)
\curveto(880.16298042,588.63713461)(879.90298068,588.43213481)(879.7029834,588.16214111)
\curveto(879.51298107,587.90213534)(879.35798122,587.59713565)(879.2379834,587.24714111)
\curveto(879.19798138,587.13713611)(879.16798141,587.02213622)(879.1479834,586.90214111)
\curveto(879.13798144,586.79213645)(879.12298146,586.68213656)(879.1029834,586.57214111)
\curveto(879.10298148,586.53213671)(879.09798148,586.49213675)(879.0879834,586.45214111)
\lineto(879.0879834,586.34714111)
\curveto(879.06798151,586.29713695)(879.05798152,586.242137)(879.0579834,586.18214111)
\curveto(879.06798151,586.12213712)(879.07298151,586.06713718)(879.0729834,586.01714111)
\lineto(879.0729834,585.68714111)
\curveto(879.07298151,585.58713766)(879.0829815,585.49213775)(879.1029834,585.40214111)
\curveto(879.11298147,585.37213787)(879.11798146,585.32213792)(879.1179834,585.25214111)
\curveto(879.13798144,585.18213806)(879.15298143,585.11213813)(879.1629834,585.04214111)
\lineto(879.2229834,584.83214111)
\curveto(879.33298125,584.48213876)(879.4829811,584.18213906)(879.6729834,583.93214111)
\curveto(879.86298072,583.68213956)(880.10298048,583.47713977)(880.3929834,583.31714111)
\curveto(880.4829801,583.26713998)(880.57298001,583.22714002)(880.6629834,583.19714111)
\curveto(880.75297983,583.16714008)(880.85297973,583.13714011)(880.9629834,583.10714111)
\curveto(881.01297957,583.08714016)(881.06297952,583.08214016)(881.1129834,583.09214111)
\curveto(881.17297941,583.10214014)(881.22797935,583.09714015)(881.2779834,583.07714111)
\curveto(881.31797926,583.06714018)(881.35797922,583.06214018)(881.3979834,583.06214111)
\lineto(881.5329834,583.06214111)
\lineto(881.6679834,583.06214111)
\curveto(881.69797888,583.07214017)(881.74797883,583.07714017)(881.8179834,583.07714111)
\curveto(881.89797868,583.09714015)(881.9779786,583.11214013)(882.0579834,583.12214111)
\curveto(882.13797844,583.1421401)(882.21297837,583.16714008)(882.2829834,583.19714111)
\curveto(882.61297797,583.33713991)(882.8779777,583.51213973)(883.0779834,583.72214111)
\curveto(883.28797729,583.9421393)(883.46297712,584.21713903)(883.6029834,584.54714111)
\curveto(883.65297693,584.65713859)(883.68797689,584.76713848)(883.7079834,584.87714111)
\curveto(883.72797685,584.98713826)(883.75297683,585.09713815)(883.7829834,585.20714111)
\curveto(883.80297678,585.247138)(883.81297677,585.28213796)(883.8129834,585.31214111)
\curveto(883.81297677,585.35213789)(883.81797676,585.39213785)(883.8279834,585.43214111)
\curveto(883.83797674,585.49213775)(883.83797674,585.55213769)(883.8279834,585.61214111)
\curveto(883.82797675,585.67213757)(883.83297675,585.73213751)(883.8429834,585.79214111)
}
}
{
\newrgbcolor{curcolor}{0 0 0}
\pscustom[linestyle=none,fillstyle=solid,fillcolor=curcolor]
{
\newpath
\moveto(894.1292334,586.34714111)
\curveto(894.14922534,586.28713696)(894.15922533,586.19213705)(894.1592334,586.06214111)
\curveto(894.15922533,585.9421373)(894.15422533,585.85713739)(894.1442334,585.80714111)
\lineto(894.1442334,585.65714111)
\curveto(894.13422535,585.57713767)(894.12422536,585.50213774)(894.1142334,585.43214111)
\curveto(894.11422537,585.37213787)(894.10922538,585.30213794)(894.0992334,585.22214111)
\curveto(894.07922541,585.16213808)(894.06422542,585.10213814)(894.0542334,585.04214111)
\curveto(894.05422543,584.98213826)(894.04422544,584.92213832)(894.0242334,584.86214111)
\curveto(893.9842255,584.73213851)(893.94922554,584.60213864)(893.9192334,584.47214111)
\curveto(893.8892256,584.3421389)(893.84922564,584.22213902)(893.7992334,584.11214111)
\curveto(893.5892259,583.63213961)(893.30922618,583.22714002)(892.9592334,582.89714111)
\curveto(892.60922688,582.57714067)(892.17922731,582.33214091)(891.6692334,582.16214111)
\curveto(891.55922793,582.12214112)(891.43922805,582.09214115)(891.3092334,582.07214111)
\curveto(891.1892283,582.05214119)(891.06422842,582.03214121)(890.9342334,582.01214111)
\curveto(890.87422861,582.00214124)(890.80922868,581.99714125)(890.7392334,581.99714111)
\curveto(890.67922881,581.98714126)(890.61922887,581.98214126)(890.5592334,581.98214111)
\curveto(890.51922897,581.97214127)(890.45922903,581.96714128)(890.3792334,581.96714111)
\curveto(890.30922918,581.96714128)(890.25922923,581.97214127)(890.2292334,581.98214111)
\curveto(890.1892293,581.99214125)(890.14922934,581.99714125)(890.1092334,581.99714111)
\curveto(890.06922942,581.98714126)(890.03422945,581.98714126)(890.0042334,581.99714111)
\lineto(889.9142334,581.99714111)
\lineto(889.5542334,582.04214111)
\curveto(889.41423007,582.08214116)(889.27923021,582.12214112)(889.1492334,582.16214111)
\curveto(889.01923047,582.20214104)(888.89423059,582.247141)(888.7742334,582.29714111)
\curveto(888.32423116,582.49714075)(887.95423153,582.75714049)(887.6642334,583.07714111)
\curveto(887.37423211,583.39713985)(887.13423235,583.78713946)(886.9442334,584.24714111)
\curveto(886.89423259,584.3471389)(886.85423263,584.4471388)(886.8242334,584.54714111)
\curveto(886.80423268,584.6471386)(886.7842327,584.75213849)(886.7642334,584.86214111)
\curveto(886.74423274,584.90213834)(886.73423275,584.93213831)(886.7342334,584.95214111)
\curveto(886.74423274,584.98213826)(886.74423274,585.01713823)(886.7342334,585.05714111)
\curveto(886.71423277,585.13713811)(886.69923279,585.21713803)(886.6892334,585.29714111)
\curveto(886.6892328,585.38713786)(886.67923281,585.47213777)(886.6592334,585.55214111)
\lineto(886.6592334,585.67214111)
\curveto(886.65923283,585.71213753)(886.65423283,585.75713749)(886.6442334,585.80714111)
\curveto(886.63423285,585.85713739)(886.62923286,585.9421373)(886.6292334,586.06214111)
\curveto(886.62923286,586.19213705)(886.63923285,586.28713696)(886.6592334,586.34714111)
\curveto(886.67923281,586.41713683)(886.6842328,586.48713676)(886.6742334,586.55714111)
\curveto(886.66423282,586.62713662)(886.66923282,586.69713655)(886.6892334,586.76714111)
\curveto(886.69923279,586.81713643)(886.70423278,586.85713639)(886.7042334,586.88714111)
\curveto(886.71423277,586.92713632)(886.72423276,586.97213627)(886.7342334,587.02214111)
\curveto(886.76423272,587.1421361)(886.7892327,587.26213598)(886.8092334,587.38214111)
\curveto(886.83923265,587.50213574)(886.87923261,587.61713563)(886.9292334,587.72714111)
\curveto(887.07923241,588.09713515)(887.25923223,588.42713482)(887.4692334,588.71714111)
\curveto(887.6892318,589.01713423)(887.95423153,589.26713398)(888.2642334,589.46714111)
\curveto(888.3842311,589.5471337)(888.50923098,589.61213363)(888.6392334,589.66214111)
\curveto(888.76923072,589.72213352)(888.90423058,589.78213346)(889.0442334,589.84214111)
\curveto(889.16423032,589.89213335)(889.29423019,589.92213332)(889.4342334,589.93214111)
\curveto(889.57422991,589.95213329)(889.71422977,589.98213326)(889.8542334,590.02214111)
\lineto(890.0492334,590.02214111)
\curveto(890.11922937,590.03213321)(890.1842293,590.0421332)(890.2442334,590.05214111)
\curveto(891.13422835,590.06213318)(891.87422761,589.87713337)(892.4642334,589.49714111)
\curveto(893.05422643,589.11713413)(893.47922601,588.62213462)(893.7392334,588.01214111)
\curveto(893.7892257,587.91213533)(893.82922566,587.81213543)(893.8592334,587.71214111)
\curveto(893.8892256,587.61213563)(893.92422556,587.50713574)(893.9642334,587.39714111)
\curveto(893.99422549,587.28713596)(894.01922547,587.16713608)(894.0392334,587.03714111)
\curveto(894.05922543,586.91713633)(894.0842254,586.79213645)(894.1142334,586.66214111)
\curveto(894.12422536,586.61213663)(894.12422536,586.55713669)(894.1142334,586.49714111)
\curveto(894.11422537,586.4471368)(894.11922537,586.39713685)(894.1292334,586.34714111)
\moveto(892.7942334,585.49214111)
\curveto(892.81422667,585.56213768)(892.81922667,585.6421376)(892.8092334,585.73214111)
\lineto(892.8092334,585.98714111)
\curveto(892.80922668,586.37713687)(892.77422671,586.70713654)(892.7042334,586.97714111)
\curveto(892.67422681,587.05713619)(892.64922684,587.13713611)(892.6292334,587.21714111)
\curveto(892.60922688,587.29713595)(892.5842269,587.37213587)(892.5542334,587.44214111)
\curveto(892.27422721,588.09213515)(891.82922766,588.5421347)(891.2192334,588.79214111)
\curveto(891.14922834,588.82213442)(891.07422841,588.8421344)(890.9942334,588.85214111)
\lineto(890.7542334,588.91214111)
\curveto(890.67422881,588.93213431)(890.5892289,588.9421343)(890.4992334,588.94214111)
\lineto(890.2292334,588.94214111)
\lineto(889.9592334,588.89714111)
\curveto(889.85922963,588.87713437)(889.76422972,588.85213439)(889.6742334,588.82214111)
\curveto(889.59422989,588.80213444)(889.51422997,588.77213447)(889.4342334,588.73214111)
\curveto(889.36423012,588.71213453)(889.29923019,588.68213456)(889.2392334,588.64214111)
\curveto(889.17923031,588.60213464)(889.12423036,588.56213468)(889.0742334,588.52214111)
\curveto(888.83423065,588.35213489)(888.63923085,588.1471351)(888.4892334,587.90714111)
\curveto(888.33923115,587.66713558)(888.20923128,587.38713586)(888.0992334,587.06714111)
\curveto(888.06923142,586.96713628)(888.04923144,586.86213638)(888.0392334,586.75214111)
\curveto(888.02923146,586.65213659)(888.01423147,586.5471367)(887.9942334,586.43714111)
\curveto(887.9842315,586.39713685)(887.97923151,586.33213691)(887.9792334,586.24214111)
\curveto(887.96923152,586.21213703)(887.96423152,586.17713707)(887.9642334,586.13714111)
\curveto(887.97423151,586.09713715)(887.97923151,586.05213719)(887.9792334,586.00214111)
\lineto(887.9792334,585.70214111)
\curveto(887.97923151,585.60213764)(887.9892315,585.51213773)(888.0092334,585.43214111)
\lineto(888.0392334,585.25214111)
\curveto(888.05923143,585.15213809)(888.07423141,585.05213819)(888.0842334,584.95214111)
\curveto(888.10423138,584.86213838)(888.13423135,584.77713847)(888.1742334,584.69714111)
\curveto(888.27423121,584.45713879)(888.3892311,584.23213901)(888.5192334,584.02214111)
\curveto(888.65923083,583.81213943)(888.82923066,583.63713961)(889.0292334,583.49714111)
\curveto(889.07923041,583.46713978)(889.12423036,583.4421398)(889.1642334,583.42214111)
\curveto(889.20423028,583.40213984)(889.24923024,583.37713987)(889.2992334,583.34714111)
\curveto(889.37923011,583.29713995)(889.46423002,583.25213999)(889.5542334,583.21214111)
\curveto(889.65422983,583.18214006)(889.75922973,583.15214009)(889.8692334,583.12214111)
\curveto(889.91922957,583.10214014)(889.96422952,583.09214015)(890.0042334,583.09214111)
\curveto(890.05422943,583.10214014)(890.10422938,583.10214014)(890.1542334,583.09214111)
\curveto(890.1842293,583.08214016)(890.24422924,583.07214017)(890.3342334,583.06214111)
\curveto(890.43422905,583.05214019)(890.50922898,583.05714019)(890.5592334,583.07714111)
\curveto(890.59922889,583.08714016)(890.63922885,583.08714016)(890.6792334,583.07714111)
\curveto(890.71922877,583.07714017)(890.75922873,583.08714016)(890.7992334,583.10714111)
\curveto(890.87922861,583.12714012)(890.95922853,583.1421401)(891.0392334,583.15214111)
\curveto(891.11922837,583.17214007)(891.19422829,583.19714005)(891.2642334,583.22714111)
\curveto(891.60422788,583.36713988)(891.87922761,583.56213968)(892.0892334,583.81214111)
\curveto(892.29922719,584.06213918)(892.47422701,584.35713889)(892.6142334,584.69714111)
\curveto(892.66422682,584.81713843)(892.69422679,584.9421383)(892.7042334,585.07214111)
\curveto(892.72422676,585.21213803)(892.75422673,585.35213789)(892.7942334,585.49214111)
}
}
{
\newrgbcolor{curcolor}{0 0 0}
\pscustom[linestyle=none,fillstyle=solid,fillcolor=curcolor]
{
\newpath
\moveto(899.26251465,590.05214111)
\curveto(899.49250986,590.05213319)(899.62250973,589.99213325)(899.65251465,589.87214111)
\curveto(899.68250967,589.76213348)(899.69750965,589.59713365)(899.69751465,589.37714111)
\lineto(899.69751465,589.09214111)
\curveto(899.69750965,589.00213424)(899.67250968,588.92713432)(899.62251465,588.86714111)
\curveto(899.56250979,588.78713446)(899.47750987,588.7421345)(899.36751465,588.73214111)
\curveto(899.25751009,588.73213451)(899.1475102,588.71713453)(899.03751465,588.68714111)
\curveto(898.89751045,588.65713459)(898.76251059,588.62713462)(898.63251465,588.59714111)
\curveto(898.51251084,588.56713468)(898.39751095,588.52713472)(898.28751465,588.47714111)
\curveto(897.99751135,588.3471349)(897.76251159,588.16713508)(897.58251465,587.93714111)
\curveto(897.40251195,587.71713553)(897.2475121,587.46213578)(897.11751465,587.17214111)
\curveto(897.07751227,587.06213618)(897.0475123,586.9471363)(897.02751465,586.82714111)
\curveto(897.00751234,586.71713653)(896.98251237,586.60213664)(896.95251465,586.48214111)
\curveto(896.94251241,586.43213681)(896.93751241,586.38213686)(896.93751465,586.33214111)
\curveto(896.9475124,586.28213696)(896.9475124,586.23213701)(896.93751465,586.18214111)
\curveto(896.90751244,586.06213718)(896.89251246,585.92213732)(896.89251465,585.76214111)
\curveto(896.90251245,585.61213763)(896.90751244,585.46713778)(896.90751465,585.32714111)
\lineto(896.90751465,583.48214111)
\lineto(896.90751465,583.13714111)
\curveto(896.90751244,583.01714023)(896.90251245,582.90214034)(896.89251465,582.79214111)
\curveto(896.88251247,582.68214056)(896.87751247,582.58714066)(896.87751465,582.50714111)
\curveto(896.88751246,582.42714082)(896.86751248,582.35714089)(896.81751465,582.29714111)
\curveto(896.76751258,582.22714102)(896.68751266,582.18714106)(896.57751465,582.17714111)
\curveto(896.47751287,582.16714108)(896.36751298,582.16214108)(896.24751465,582.16214111)
\lineto(895.97751465,582.16214111)
\curveto(895.92751342,582.18214106)(895.87751347,582.19714105)(895.82751465,582.20714111)
\curveto(895.78751356,582.22714102)(895.75751359,582.25214099)(895.73751465,582.28214111)
\curveto(895.68751366,582.35214089)(895.65751369,582.43714081)(895.64751465,582.53714111)
\lineto(895.64751465,582.86714111)
\lineto(895.64751465,584.02214111)
\lineto(895.64751465,588.17714111)
\lineto(895.64751465,589.21214111)
\lineto(895.64751465,589.51214111)
\curveto(895.65751369,589.61213363)(895.68751366,589.69713355)(895.73751465,589.76714111)
\curveto(895.76751358,589.80713344)(895.81751353,589.83713341)(895.88751465,589.85714111)
\curveto(895.96751338,589.87713337)(896.0525133,589.88713336)(896.14251465,589.88714111)
\curveto(896.23251312,589.89713335)(896.32251303,589.89713335)(896.41251465,589.88714111)
\curveto(896.50251285,589.87713337)(896.57251278,589.86213338)(896.62251465,589.84214111)
\curveto(896.70251265,589.81213343)(896.7525126,589.75213349)(896.77251465,589.66214111)
\curveto(896.80251255,589.58213366)(896.81751253,589.49213375)(896.81751465,589.39214111)
\lineto(896.81751465,589.09214111)
\curveto(896.81751253,588.99213425)(896.83751251,588.90213434)(896.87751465,588.82214111)
\curveto(896.88751246,588.80213444)(896.89751245,588.78713446)(896.90751465,588.77714111)
\lineto(896.95251465,588.73214111)
\curveto(897.06251229,588.73213451)(897.1525122,588.77713447)(897.22251465,588.86714111)
\curveto(897.29251206,588.96713428)(897.352512,589.0471342)(897.40251465,589.10714111)
\lineto(897.49251465,589.19714111)
\curveto(897.58251177,589.30713394)(897.70751164,589.42213382)(897.86751465,589.54214111)
\curveto(898.02751132,589.66213358)(898.17751117,589.75213349)(898.31751465,589.81214111)
\curveto(898.40751094,589.86213338)(898.50251085,589.89713335)(898.60251465,589.91714111)
\curveto(898.70251065,589.9471333)(898.80751054,589.97713327)(898.91751465,590.00714111)
\curveto(898.97751037,590.01713323)(899.03751031,590.02213322)(899.09751465,590.02214111)
\curveto(899.15751019,590.03213321)(899.21251014,590.0421332)(899.26251465,590.05214111)
}
}
{
\newrgbcolor{curcolor}{0.50196081 0.50196081 0.50196081}
\pscustom[linestyle=none,fillstyle=solid,fillcolor=curcolor]
{
\newpath
\moveto(807.09008789,592.85717773)
\lineto(822.09008789,592.85717773)
\lineto(822.09008789,577.85717773)
\lineto(807.09008789,577.85717773)
\closepath
}
}
{
\newrgbcolor{curcolor}{0 0 0}
\pscustom[linestyle=none,fillstyle=solid,fillcolor=curcolor]
{
\newpath
\moveto(835.4632959,564.88496338)
\lineto(835.4632959,564.61496338)
\curveto(835.47328593,564.52495813)(835.46828593,564.44495821)(835.4482959,564.37496338)
\lineto(835.4482959,564.22496338)
\curveto(835.43828596,564.19495846)(835.43328597,564.15995849)(835.4332959,564.11996338)
\curveto(835.44328596,564.07995857)(835.44328596,564.0499586)(835.4332959,564.02996338)
\curveto(835.42328598,563.97995867)(835.41828598,563.92495873)(835.4182959,563.86496338)
\curveto(835.41828598,563.81495884)(835.41328599,563.76495889)(835.4032959,563.71496338)
\curveto(835.37328603,563.57495908)(835.35328605,563.42495923)(835.3432959,563.26496338)
\curveto(835.33328607,563.11495954)(835.3032861,562.96995968)(835.2532959,562.82996338)
\curveto(835.22328618,562.70995994)(835.18828621,562.58496007)(835.1482959,562.45496338)
\curveto(835.11828628,562.33496032)(835.07828632,562.21496044)(835.0282959,562.09496338)
\curveto(834.85828654,561.66496099)(834.64328676,561.27496138)(834.3832959,560.92496338)
\curveto(834.13328727,560.58496207)(833.81828758,560.29496236)(833.4382959,560.05496338)
\curveto(833.24828815,559.93496272)(833.04328836,559.82996282)(832.8232959,559.73996338)
\curveto(832.61328879,559.65996299)(832.38328902,559.57996307)(832.1332959,559.49996338)
\curveto(832.02328938,559.45996319)(831.9032895,559.42996322)(831.7732959,559.40996338)
\curveto(831.65328975,559.39996325)(831.53328987,559.37996327)(831.4132959,559.34996338)
\curveto(831.3032901,559.32996332)(831.19329021,559.31996333)(831.0832959,559.31996338)
\curveto(830.98329042,559.31996333)(830.88329052,559.30996334)(830.7832959,559.28996338)
\lineto(830.5732959,559.28996338)
\curveto(830.54329086,559.27996337)(830.50829089,559.27496338)(830.4682959,559.27496338)
\curveto(830.42829097,559.28496337)(830.38829101,559.28996336)(830.3482959,559.28996338)
\lineto(827.3482959,559.28996338)
\curveto(827.1982942,559.28996336)(827.06329434,559.29496336)(826.9432959,559.30496338)
\curveto(826.83329457,559.32496333)(826.75829464,559.38996326)(826.7182959,559.49996338)
\curveto(826.67829472,559.57996307)(826.65829474,559.69496296)(826.6582959,559.84496338)
\curveto(826.66829473,559.99496266)(826.67329473,560.12996252)(826.6732959,560.24996338)
\lineto(826.6732959,569.11496338)
\curveto(826.67329473,569.23495342)(826.66829473,569.35995329)(826.6582959,569.48996338)
\curveto(826.65829474,569.62995302)(826.68329472,569.73995291)(826.7332959,569.81996338)
\curveto(826.77329463,569.88995276)(826.84829455,569.93495272)(826.9582959,569.95496338)
\curveto(826.97829442,569.96495269)(826.9982944,569.96495269)(827.0182959,569.95496338)
\curveto(827.03829436,569.9549527)(827.05829434,569.95995269)(827.0782959,569.96996338)
\lineto(830.3332959,569.96996338)
\curveto(830.38329102,569.96995268)(830.42829097,569.96995268)(830.4682959,569.96996338)
\curveto(830.51829088,569.97995267)(830.56329084,569.97995267)(830.6032959,569.96996338)
\curveto(830.65329075,569.9499527)(830.7032907,569.94495271)(830.7532959,569.95496338)
\curveto(830.81329059,569.96495269)(830.86829053,569.96495269)(830.9182959,569.95496338)
\curveto(830.96829043,569.94495271)(831.02329038,569.93995271)(831.0832959,569.93996338)
\curveto(831.14329026,569.93995271)(831.1982902,569.93495272)(831.2482959,569.92496338)
\curveto(831.2982901,569.91495274)(831.34329006,569.90995274)(831.3832959,569.90996338)
\curveto(831.43328997,569.90995274)(831.48328992,569.90495275)(831.5332959,569.89496338)
\curveto(831.64328976,569.87495278)(831.74828965,569.8549528)(831.8482959,569.83496338)
\curveto(831.94828945,569.82495283)(832.04828935,569.80495285)(832.1482959,569.77496338)
\curveto(832.36828903,569.70495295)(832.57828882,569.63495302)(832.7782959,569.56496338)
\curveto(832.97828842,569.50495315)(833.16328824,569.41995323)(833.3332959,569.30996338)
\curveto(833.47328793,569.22995342)(833.5982878,569.1499535)(833.7082959,569.06996338)
\curveto(833.73828766,569.0499536)(833.76828763,569.02495363)(833.7982959,568.99496338)
\curveto(833.82828757,568.97495368)(833.85828754,568.9549537)(833.8882959,568.93496338)
\curveto(833.94828745,568.88495377)(834.0032874,568.83495382)(834.0532959,568.78496338)
\curveto(834.1032873,568.73495392)(834.15328725,568.68495397)(834.2032959,568.63496338)
\curveto(834.25328715,568.58495407)(834.29328711,568.5499541)(834.3232959,568.52996338)
\curveto(834.36328704,568.46995418)(834.403287,568.41495424)(834.4432959,568.36496338)
\curveto(834.49328691,568.31495434)(834.53828686,568.25995439)(834.5782959,568.19996338)
\curveto(834.62828677,568.13995451)(834.66828673,568.07495458)(834.6982959,568.00496338)
\curveto(834.73828666,567.94495471)(834.78328662,567.87995477)(834.8332959,567.80996338)
\curveto(834.85328655,567.76995488)(834.86828653,567.73495492)(834.8782959,567.70496338)
\curveto(834.88828651,567.67495498)(834.9032865,567.63995501)(834.9232959,567.59996338)
\curveto(834.96328644,567.51995513)(834.9982864,567.43995521)(835.0282959,567.35996338)
\curveto(835.05828634,567.28995536)(835.09328631,567.21495544)(835.1332959,567.13496338)
\curveto(835.17328623,567.02495563)(835.2032862,566.90995574)(835.2232959,566.78996338)
\curveto(835.25328615,566.67995597)(835.28328612,566.56995608)(835.3132959,566.45996338)
\curveto(835.33328607,566.39995625)(835.34328606,566.33995631)(835.3432959,566.27996338)
\curveto(835.34328606,566.22995642)(835.35328605,566.17495648)(835.3732959,566.11496338)
\curveto(835.42328598,565.93495672)(835.44828595,565.73495692)(835.4482959,565.51496338)
\curveto(835.45828594,565.30495735)(835.46328594,565.09495756)(835.4632959,564.88496338)
\moveto(834.0382959,564.10496338)
\curveto(834.05828734,564.20495845)(834.06828733,564.30995834)(834.0682959,564.41996338)
\lineto(834.0682959,564.76496338)
\lineto(834.0682959,564.98996338)
\curveto(834.07828732,565.06995758)(834.07328733,565.14495751)(834.0532959,565.21496338)
\curveto(834.05328735,565.24495741)(834.04828735,565.27495738)(834.0382959,565.30496338)
\lineto(834.0382959,565.40996338)
\curveto(834.01828738,565.51995713)(834.0032874,565.62995702)(833.9932959,565.73996338)
\curveto(833.99328741,565.8499568)(833.97828742,565.95995669)(833.9482959,566.06996338)
\curveto(833.92828747,566.1499565)(833.90828749,566.22495643)(833.8882959,566.29496338)
\curveto(833.87828752,566.37495628)(833.86328754,566.4549562)(833.8432959,566.53496338)
\curveto(833.73328767,566.89495576)(833.59328781,567.20995544)(833.4232959,567.47996338)
\curveto(833.14328826,567.92995472)(832.72828867,568.26995438)(832.1782959,568.49996338)
\curveto(832.08828931,568.5499541)(831.99328941,568.58495407)(831.8932959,568.60496338)
\curveto(831.79328961,568.63495402)(831.68828971,568.66495399)(831.5782959,568.69496338)
\curveto(831.46828993,568.72495393)(831.35329005,568.73995391)(831.2332959,568.73996338)
\curveto(831.12329028,568.7499539)(831.01329039,568.76495389)(830.9032959,568.78496338)
\lineto(830.5882959,568.78496338)
\curveto(830.55829084,568.79495386)(830.52329088,568.79995385)(830.4832959,568.79996338)
\lineto(830.3632959,568.79996338)
\lineto(828.5332959,568.79996338)
\curveto(828.51329289,568.78995386)(828.48829291,568.78495387)(828.4582959,568.78496338)
\curveto(828.42829297,568.79495386)(828.403293,568.79495386)(828.3832959,568.78496338)
\lineto(828.2332959,568.72496338)
\curveto(828.19329321,568.70495395)(828.16329324,568.67495398)(828.1432959,568.63496338)
\curveto(828.12329328,568.59495406)(828.1032933,568.52495413)(828.0832959,568.42496338)
\lineto(828.0832959,568.30496338)
\curveto(828.07329333,568.26495439)(828.06829333,568.21995443)(828.0682959,568.16996338)
\lineto(828.0682959,568.03496338)
\lineto(828.0682959,561.22496338)
\lineto(828.0682959,561.07496338)
\curveto(828.06829333,561.03496162)(828.07329333,560.99496166)(828.0832959,560.95496338)
\lineto(828.0832959,560.83496338)
\curveto(828.1032933,560.73496192)(828.12329328,560.66496199)(828.1432959,560.62496338)
\curveto(828.22329318,560.50496215)(828.37329303,560.44496221)(828.5932959,560.44496338)
\curveto(828.81329259,560.4549622)(829.02329238,560.45996219)(829.2232959,560.45996338)
\lineto(830.0932959,560.45996338)
\curveto(830.16329124,560.45996219)(830.23829116,560.4549622)(830.3182959,560.44496338)
\curveto(830.398291,560.44496221)(830.46829093,560.4549622)(830.5282959,560.47496338)
\lineto(830.6932959,560.47496338)
\curveto(830.74329066,560.48496217)(830.7982906,560.48496217)(830.8582959,560.47496338)
\curveto(830.91829048,560.47496218)(830.97829042,560.47996217)(831.0382959,560.48996338)
\curveto(831.0982903,560.50996214)(831.15829024,560.51996213)(831.2182959,560.51996338)
\curveto(831.27829012,560.52996212)(831.34329006,560.54496211)(831.4132959,560.56496338)
\curveto(831.52328988,560.59496206)(831.62828977,560.62496203)(831.7282959,560.65496338)
\curveto(831.83828956,560.68496197)(831.94828945,560.72496193)(832.0582959,560.77496338)
\curveto(832.42828897,560.93496172)(832.74328866,561.1499615)(833.0032959,561.41996338)
\curveto(833.27328813,561.69996095)(833.49328791,562.02996062)(833.6632959,562.40996338)
\curveto(833.71328769,562.51996013)(833.75328765,562.63496002)(833.7832959,562.75496338)
\lineto(833.9032959,563.14496338)
\curveto(833.93328747,563.2549594)(833.95328745,563.36995928)(833.9632959,563.48996338)
\curveto(833.98328742,563.61995903)(834.0032874,563.74495891)(834.0232959,563.86496338)
\curveto(834.03328737,563.91495874)(834.03828736,563.9549587)(834.0382959,563.98496338)
\lineto(834.0382959,564.10496338)
}
}
{
\newrgbcolor{curcolor}{0 0 0}
\pscustom[linestyle=none,fillstyle=solid,fillcolor=curcolor]
{
\newpath
\moveto(843.7151709,563.45996338)
\curveto(843.73516321,563.35995929)(843.73516321,563.24495941)(843.7151709,563.11496338)
\curveto(843.70516324,562.99495966)(843.67516327,562.90995974)(843.6251709,562.85996338)
\curveto(843.57516337,562.81995983)(843.50016345,562.78995986)(843.4001709,562.76996338)
\curveto(843.31016364,562.75995989)(843.20516374,562.7549599)(843.0851709,562.75496338)
\lineto(842.7251709,562.75496338)
\curveto(842.60516434,562.76495989)(842.50016445,562.76995988)(842.4101709,562.76996338)
\lineto(838.5701709,562.76996338)
\curveto(838.49016846,562.76995988)(838.41016854,562.76495989)(838.3301709,562.75496338)
\curveto(838.2501687,562.7549599)(838.18516876,562.73995991)(838.1351709,562.70996338)
\curveto(838.09516885,562.68995996)(838.05516889,562.64996)(838.0151709,562.58996338)
\curveto(837.99516895,562.55996009)(837.97516897,562.51496014)(837.9551709,562.45496338)
\curveto(837.93516901,562.40496025)(837.93516901,562.3549603)(837.9551709,562.30496338)
\curveto(837.96516898,562.2549604)(837.97016898,562.20996044)(837.9701709,562.16996338)
\curveto(837.97016898,562.12996052)(837.97516897,562.08996056)(837.9851709,562.04996338)
\curveto(838.00516894,561.96996068)(838.02516892,561.88496077)(838.0451709,561.79496338)
\curveto(838.06516888,561.71496094)(838.09516885,561.63496102)(838.1351709,561.55496338)
\curveto(838.36516858,561.01496164)(838.7451682,560.62996202)(839.2751709,560.39996338)
\curveto(839.33516761,560.36996228)(839.40016755,560.34496231)(839.4701709,560.32496338)
\lineto(839.6801709,560.26496338)
\curveto(839.71016724,560.2549624)(839.76016719,560.2499624)(839.8301709,560.24996338)
\curveto(839.97016698,560.20996244)(840.15516679,560.18996246)(840.3851709,560.18996338)
\curveto(840.61516633,560.18996246)(840.80016615,560.20996244)(840.9401709,560.24996338)
\curveto(841.08016587,560.28996236)(841.20516574,560.32996232)(841.3151709,560.36996338)
\curveto(841.43516551,560.41996223)(841.5451654,560.47996217)(841.6451709,560.54996338)
\curveto(841.75516519,560.61996203)(841.8501651,560.69996195)(841.9301709,560.78996338)
\curveto(842.01016494,560.88996176)(842.08016487,560.99496166)(842.1401709,561.10496338)
\curveto(842.20016475,561.20496145)(842.2501647,561.30996134)(842.2901709,561.41996338)
\curveto(842.34016461,561.52996112)(842.42016453,561.60996104)(842.5301709,561.65996338)
\curveto(842.57016438,561.67996097)(842.63516431,561.69496096)(842.7251709,561.70496338)
\curveto(842.81516413,561.71496094)(842.90516404,561.71496094)(842.9951709,561.70496338)
\curveto(843.08516386,561.70496095)(843.17016378,561.69996095)(843.2501709,561.68996338)
\curveto(843.33016362,561.67996097)(843.38516356,561.65996099)(843.4151709,561.62996338)
\curveto(843.51516343,561.55996109)(843.54016341,561.44496121)(843.4901709,561.28496338)
\curveto(843.41016354,561.01496164)(843.30516364,560.77496188)(843.1751709,560.56496338)
\curveto(842.97516397,560.24496241)(842.7451642,559.97996267)(842.4851709,559.76996338)
\curveto(842.23516471,559.56996308)(841.91516503,559.40496325)(841.5251709,559.27496338)
\curveto(841.42516552,559.23496342)(841.32516562,559.20996344)(841.2251709,559.19996338)
\curveto(841.12516582,559.17996347)(841.02016593,559.15996349)(840.9101709,559.13996338)
\curveto(840.86016609,559.12996352)(840.81016614,559.12496353)(840.7601709,559.12496338)
\curveto(840.72016623,559.12496353)(840.67516627,559.11996353)(840.6251709,559.10996338)
\lineto(840.4751709,559.10996338)
\curveto(840.42516652,559.09996355)(840.36516658,559.09496356)(840.2951709,559.09496338)
\curveto(840.23516671,559.09496356)(840.18516676,559.09996355)(840.1451709,559.10996338)
\lineto(840.0101709,559.10996338)
\curveto(839.96016699,559.11996353)(839.91516703,559.12496353)(839.8751709,559.12496338)
\curveto(839.83516711,559.12496353)(839.79516715,559.12996352)(839.7551709,559.13996338)
\curveto(839.70516724,559.1499635)(839.6501673,559.15996349)(839.5901709,559.16996338)
\curveto(839.53016742,559.16996348)(839.47516747,559.17496348)(839.4251709,559.18496338)
\curveto(839.33516761,559.20496345)(839.2451677,559.22996342)(839.1551709,559.25996338)
\curveto(839.06516788,559.27996337)(838.98016797,559.30496335)(838.9001709,559.33496338)
\curveto(838.86016809,559.3549633)(838.82516812,559.36496329)(838.7951709,559.36496338)
\curveto(838.76516818,559.37496328)(838.73016822,559.38996326)(838.6901709,559.40996338)
\curveto(838.54016841,559.47996317)(838.38016857,559.56496309)(838.2101709,559.66496338)
\curveto(837.92016903,559.8549628)(837.67016928,560.08496257)(837.4601709,560.35496338)
\curveto(837.26016969,560.63496202)(837.09016986,560.94496171)(836.9501709,561.28496338)
\curveto(836.90017005,561.39496126)(836.86017009,561.50996114)(836.8301709,561.62996338)
\curveto(836.81017014,561.7499609)(836.78017017,561.86996078)(836.7401709,561.98996338)
\curveto(836.73017022,562.02996062)(836.72517022,562.06496059)(836.7251709,562.09496338)
\curveto(836.72517022,562.12496053)(836.72017023,562.16496049)(836.7101709,562.21496338)
\curveto(836.69017026,562.29496036)(836.67517027,562.37996027)(836.6651709,562.46996338)
\curveto(836.65517029,562.55996009)(836.64017031,562.64996)(836.6201709,562.73996338)
\lineto(836.6201709,562.94996338)
\curveto(836.61017034,562.98995966)(836.60017035,563.04495961)(836.5901709,563.11496338)
\curveto(836.59017036,563.19495946)(836.59517035,563.25995939)(836.6051709,563.30996338)
\lineto(836.6051709,563.47496338)
\curveto(836.62517032,563.52495913)(836.63017032,563.57495908)(836.6201709,563.62496338)
\curveto(836.62017033,563.68495897)(836.62517032,563.73995891)(836.6351709,563.78996338)
\curveto(836.67517027,563.9499587)(836.70517024,564.10995854)(836.7251709,564.26996338)
\curveto(836.75517019,564.42995822)(836.80017015,564.57995807)(836.8601709,564.71996338)
\curveto(836.91017004,564.82995782)(836.95516999,564.93995771)(836.9951709,565.04996338)
\curveto(837.0451699,565.16995748)(837.10016985,565.28495737)(837.1601709,565.39496338)
\curveto(837.38016957,565.74495691)(837.63016932,566.04495661)(837.9101709,566.29496338)
\curveto(838.19016876,566.5549561)(838.53516841,566.76995588)(838.9451709,566.93996338)
\curveto(839.06516788,566.98995566)(839.18516776,567.02495563)(839.3051709,567.04496338)
\curveto(839.43516751,567.07495558)(839.57016738,567.10495555)(839.7101709,567.13496338)
\curveto(839.76016719,567.14495551)(839.80516714,567.1499555)(839.8451709,567.14996338)
\curveto(839.88516706,567.15995549)(839.93016702,567.16495549)(839.9801709,567.16496338)
\curveto(840.00016695,567.17495548)(840.02516692,567.17495548)(840.0551709,567.16496338)
\curveto(840.08516686,567.1549555)(840.11016684,567.15995549)(840.1301709,567.17996338)
\curveto(840.5501664,567.18995546)(840.91516603,567.14495551)(841.2251709,567.04496338)
\curveto(841.53516541,566.9549557)(841.81516513,566.82995582)(842.0651709,566.66996338)
\curveto(842.11516483,566.649956)(842.15516479,566.61995603)(842.1851709,566.57996338)
\curveto(842.21516473,566.5499561)(842.2501647,566.52495613)(842.2901709,566.50496338)
\curveto(842.37016458,566.44495621)(842.4501645,566.37495628)(842.5301709,566.29496338)
\curveto(842.62016433,566.21495644)(842.69516425,566.13495652)(842.7551709,566.05496338)
\curveto(842.91516403,565.84495681)(843.0501639,565.64495701)(843.1601709,565.45496338)
\curveto(843.23016372,565.34495731)(843.28516366,565.22495743)(843.3251709,565.09496338)
\curveto(843.36516358,564.96495769)(843.41016354,564.83495782)(843.4601709,564.70496338)
\curveto(843.51016344,564.57495808)(843.5451634,564.43995821)(843.5651709,564.29996338)
\curveto(843.59516335,564.15995849)(843.63016332,564.01995863)(843.6701709,563.87996338)
\curveto(843.68016327,563.80995884)(843.68516326,563.73995891)(843.6851709,563.66996338)
\lineto(843.7151709,563.45996338)
\moveto(842.2601709,563.96996338)
\curveto(842.29016466,564.00995864)(842.31516463,564.05995859)(842.3351709,564.11996338)
\curveto(842.35516459,564.18995846)(842.35516459,564.25995839)(842.3351709,564.32996338)
\curveto(842.27516467,564.5499581)(842.19016476,564.7549579)(842.0801709,564.94496338)
\curveto(841.94016501,565.17495748)(841.78516516,565.36995728)(841.6151709,565.52996338)
\curveto(841.4451655,565.68995696)(841.22516572,565.82495683)(840.9551709,565.93496338)
\curveto(840.88516606,565.9549567)(840.81516613,565.96995668)(840.7451709,565.97996338)
\curveto(840.67516627,565.99995665)(840.60016635,566.01995663)(840.5201709,566.03996338)
\curveto(840.44016651,566.05995659)(840.35516659,566.06995658)(840.2651709,566.06996338)
\lineto(840.0101709,566.06996338)
\curveto(839.98016697,566.0499566)(839.945167,566.03995661)(839.9051709,566.03996338)
\curveto(839.86516708,566.0499566)(839.83016712,566.0499566)(839.8001709,566.03996338)
\lineto(839.5601709,565.97996338)
\curveto(839.49016746,565.96995668)(839.42016753,565.9549567)(839.3501709,565.93496338)
\curveto(839.06016789,565.81495684)(838.82516812,565.66495699)(838.6451709,565.48496338)
\curveto(838.47516847,565.30495735)(838.32016863,565.07995757)(838.1801709,564.80996338)
\curveto(838.1501688,564.75995789)(838.12016883,564.69495796)(838.0901709,564.61496338)
\curveto(838.06016889,564.54495811)(838.03516891,564.46495819)(838.0151709,564.37496338)
\curveto(837.99516895,564.28495837)(837.99016896,564.19995845)(838.0001709,564.11996338)
\curveto(838.01016894,564.03995861)(838.0451689,563.97995867)(838.1051709,563.93996338)
\curveto(838.18516876,563.87995877)(838.32016863,563.8499588)(838.5101709,563.84996338)
\curveto(838.71016824,563.85995879)(838.88016807,563.86495879)(839.0201709,563.86496338)
\lineto(841.3001709,563.86496338)
\curveto(841.4501655,563.86495879)(841.63016532,563.85995879)(841.8401709,563.84996338)
\curveto(842.0501649,563.8499588)(842.19016476,563.88995876)(842.2601709,563.96996338)
}
}
{
\newrgbcolor{curcolor}{0 0 0}
\pscustom[linestyle=none,fillstyle=solid,fillcolor=curcolor]
{
\newpath
\moveto(847.45181152,567.19496338)
\curveto(848.17180746,567.20495545)(848.77680685,567.11995553)(849.26681152,566.93996338)
\curveto(849.75680587,566.76995588)(850.13680549,566.46495619)(850.40681152,566.02496338)
\curveto(850.47680515,565.91495674)(850.5318051,565.79995685)(850.57181152,565.67996338)
\curveto(850.61180502,565.56995708)(850.65180498,565.44495721)(850.69181152,565.30496338)
\curveto(850.71180492,565.23495742)(850.71680491,565.15995749)(850.70681152,565.07996338)
\curveto(850.69680493,565.00995764)(850.68180495,564.9549577)(850.66181152,564.91496338)
\curveto(850.64180499,564.89495776)(850.61680501,564.87495778)(850.58681152,564.85496338)
\curveto(850.55680507,564.84495781)(850.5318051,564.82995782)(850.51181152,564.80996338)
\curveto(850.46180517,564.78995786)(850.41180522,564.78495787)(850.36181152,564.79496338)
\curveto(850.31180532,564.80495785)(850.26180537,564.80495785)(850.21181152,564.79496338)
\curveto(850.1318055,564.77495788)(850.0268056,564.76995788)(849.89681152,564.77996338)
\curveto(849.76680586,564.79995785)(849.67680595,564.82495783)(849.62681152,564.85496338)
\curveto(849.54680608,564.90495775)(849.49180614,564.96995768)(849.46181152,565.04996338)
\curveto(849.44180619,565.13995751)(849.40680622,565.22495743)(849.35681152,565.30496338)
\curveto(849.26680636,565.46495719)(849.14180649,565.60995704)(848.98181152,565.73996338)
\curveto(848.87180676,565.81995683)(848.75180688,565.87995677)(848.62181152,565.91996338)
\curveto(848.49180714,565.95995669)(848.35180728,565.99995665)(848.20181152,566.03996338)
\curveto(848.15180748,566.05995659)(848.10180753,566.06495659)(848.05181152,566.05496338)
\curveto(848.00180763,566.0549566)(847.95180768,566.05995659)(847.90181152,566.06996338)
\curveto(847.84180779,566.08995656)(847.76680786,566.09995655)(847.67681152,566.09996338)
\curveto(847.58680804,566.09995655)(847.51180812,566.08995656)(847.45181152,566.06996338)
\lineto(847.36181152,566.06996338)
\lineto(847.21181152,566.03996338)
\curveto(847.16180847,566.03995661)(847.11180852,566.03495662)(847.06181152,566.02496338)
\curveto(846.80180883,565.96495669)(846.58680904,565.87995677)(846.41681152,565.76996338)
\curveto(846.24680938,565.65995699)(846.1318095,565.47495718)(846.07181152,565.21496338)
\curveto(846.05180958,565.14495751)(846.04680958,565.07495758)(846.05681152,565.00496338)
\curveto(846.07680955,564.93495772)(846.09680953,564.87495778)(846.11681152,564.82496338)
\curveto(846.17680945,564.67495798)(846.24680938,564.56495809)(846.32681152,564.49496338)
\curveto(846.41680921,564.43495822)(846.5268091,564.36495829)(846.65681152,564.28496338)
\curveto(846.81680881,564.18495847)(846.99680863,564.10995854)(847.19681152,564.05996338)
\curveto(847.39680823,564.01995863)(847.59680803,563.96995868)(847.79681152,563.90996338)
\curveto(847.9268077,563.86995878)(848.05680757,563.83995881)(848.18681152,563.81996338)
\curveto(848.31680731,563.79995885)(848.44680718,563.76995888)(848.57681152,563.72996338)
\curveto(848.78680684,563.66995898)(848.99180664,563.60995904)(849.19181152,563.54996338)
\curveto(849.39180624,563.49995915)(849.59180604,563.43495922)(849.79181152,563.35496338)
\lineto(849.94181152,563.29496338)
\curveto(849.99180564,563.27495938)(850.04180559,563.2499594)(850.09181152,563.21996338)
\curveto(850.29180534,563.09995955)(850.46680516,562.96495969)(850.61681152,562.81496338)
\curveto(850.76680486,562.66495999)(850.89180474,562.47496018)(850.99181152,562.24496338)
\curveto(851.01180462,562.17496048)(851.0318046,562.07996057)(851.05181152,561.95996338)
\curveto(851.07180456,561.88996076)(851.08180455,561.81496084)(851.08181152,561.73496338)
\curveto(851.09180454,561.66496099)(851.09680453,561.58496107)(851.09681152,561.49496338)
\lineto(851.09681152,561.34496338)
\curveto(851.07680455,561.27496138)(851.06680456,561.20496145)(851.06681152,561.13496338)
\curveto(851.06680456,561.06496159)(851.05680457,560.99496166)(851.03681152,560.92496338)
\curveto(851.00680462,560.81496184)(850.97180466,560.70996194)(850.93181152,560.60996338)
\curveto(850.89180474,560.50996214)(850.84680478,560.41996223)(850.79681152,560.33996338)
\curveto(850.63680499,560.07996257)(850.4318052,559.86996278)(850.18181152,559.70996338)
\curveto(849.9318057,559.55996309)(849.65180598,559.42996322)(849.34181152,559.31996338)
\curveto(849.25180638,559.28996336)(849.15680647,559.26996338)(849.05681152,559.25996338)
\curveto(848.96680666,559.23996341)(848.87680675,559.21496344)(848.78681152,559.18496338)
\curveto(848.68680694,559.16496349)(848.58680704,559.1549635)(848.48681152,559.15496338)
\curveto(848.38680724,559.1549635)(848.28680734,559.14496351)(848.18681152,559.12496338)
\lineto(848.03681152,559.12496338)
\curveto(847.98680764,559.11496354)(847.91680771,559.10996354)(847.82681152,559.10996338)
\curveto(847.73680789,559.10996354)(847.66680796,559.11496354)(847.61681152,559.12496338)
\lineto(847.45181152,559.12496338)
\curveto(847.39180824,559.14496351)(847.3268083,559.1549635)(847.25681152,559.15496338)
\curveto(847.18680844,559.14496351)(847.1268085,559.1499635)(847.07681152,559.16996338)
\curveto(847.0268086,559.17996347)(846.96180867,559.18496347)(846.88181152,559.18496338)
\lineto(846.64181152,559.24496338)
\curveto(846.57180906,559.2549634)(846.49680913,559.27496338)(846.41681152,559.30496338)
\curveto(846.10680952,559.40496325)(845.83680979,559.52996312)(845.60681152,559.67996338)
\curveto(845.37681025,559.82996282)(845.17681045,560.02496263)(845.00681152,560.26496338)
\curveto(844.91681071,560.39496226)(844.84181079,560.52996212)(844.78181152,560.66996338)
\curveto(844.72181091,560.80996184)(844.66681096,560.96496169)(844.61681152,561.13496338)
\curveto(844.59681103,561.19496146)(844.58681104,561.26496139)(844.58681152,561.34496338)
\curveto(844.59681103,561.43496122)(844.61181102,561.50496115)(844.63181152,561.55496338)
\curveto(844.66181097,561.59496106)(844.71181092,561.63496102)(844.78181152,561.67496338)
\curveto(844.8318108,561.69496096)(844.90181073,561.70496095)(844.99181152,561.70496338)
\curveto(845.08181055,561.71496094)(845.17181046,561.71496094)(845.26181152,561.70496338)
\curveto(845.35181028,561.69496096)(845.43681019,561.67996097)(845.51681152,561.65996338)
\curveto(845.60681002,561.649961)(845.66680996,561.63496102)(845.69681152,561.61496338)
\curveto(845.76680986,561.56496109)(845.81180982,561.48996116)(845.83181152,561.38996338)
\curveto(845.86180977,561.29996135)(845.89680973,561.21496144)(845.93681152,561.13496338)
\curveto(846.03680959,560.91496174)(846.17180946,560.74496191)(846.34181152,560.62496338)
\curveto(846.46180917,560.53496212)(846.59680903,560.46496219)(846.74681152,560.41496338)
\curveto(846.89680873,560.36496229)(847.05680857,560.31496234)(847.22681152,560.26496338)
\lineto(847.54181152,560.21996338)
\lineto(847.63181152,560.21996338)
\curveto(847.70180793,560.19996245)(847.79180784,560.18996246)(847.90181152,560.18996338)
\curveto(848.02180761,560.18996246)(848.12180751,560.19996245)(848.20181152,560.21996338)
\curveto(848.27180736,560.21996243)(848.3268073,560.22496243)(848.36681152,560.23496338)
\curveto(848.4268072,560.24496241)(848.48680714,560.2499624)(848.54681152,560.24996338)
\curveto(848.60680702,560.25996239)(848.66180697,560.26996238)(848.71181152,560.27996338)
\curveto(849.00180663,560.35996229)(849.2318064,560.46496219)(849.40181152,560.59496338)
\curveto(849.57180606,560.72496193)(849.69180594,560.94496171)(849.76181152,561.25496338)
\curveto(849.78180585,561.30496135)(849.78680584,561.35996129)(849.77681152,561.41996338)
\curveto(849.76680586,561.47996117)(849.75680587,561.52496113)(849.74681152,561.55496338)
\curveto(849.69680593,561.74496091)(849.626806,561.88496077)(849.53681152,561.97496338)
\curveto(849.44680618,562.07496058)(849.3318063,562.16496049)(849.19181152,562.24496338)
\curveto(849.10180653,562.30496035)(849.00180663,562.3549603)(848.89181152,562.39496338)
\lineto(848.56181152,562.51496338)
\curveto(848.5318071,562.52496013)(848.50180713,562.52996012)(848.47181152,562.52996338)
\curveto(848.45180718,562.52996012)(848.4268072,562.53996011)(848.39681152,562.55996338)
\curveto(848.05680757,562.66995998)(847.70180793,562.7499599)(847.33181152,562.79996338)
\curveto(846.97180866,562.85995979)(846.631809,562.9549597)(846.31181152,563.08496338)
\curveto(846.21180942,563.12495953)(846.11680951,563.15995949)(846.02681152,563.18996338)
\curveto(845.93680969,563.21995943)(845.85180978,563.25995939)(845.77181152,563.30996338)
\curveto(845.58181005,563.41995923)(845.40681022,563.54495911)(845.24681152,563.68496338)
\curveto(845.08681054,563.82495883)(844.96181067,563.99995865)(844.87181152,564.20996338)
\curveto(844.84181079,564.27995837)(844.81681081,564.3499583)(844.79681152,564.41996338)
\curveto(844.78681084,564.48995816)(844.77181086,564.56495809)(844.75181152,564.64496338)
\curveto(844.72181091,564.76495789)(844.71181092,564.89995775)(844.72181152,565.04996338)
\curveto(844.7318109,565.20995744)(844.74681088,565.34495731)(844.76681152,565.45496338)
\curveto(844.78681084,565.50495715)(844.79681083,565.54495711)(844.79681152,565.57496338)
\curveto(844.80681082,565.61495704)(844.82181081,565.654957)(844.84181152,565.69496338)
\curveto(844.9318107,565.92495673)(845.05181058,566.12495653)(845.20181152,566.29496338)
\curveto(845.36181027,566.46495619)(845.54181009,566.61495604)(845.74181152,566.74496338)
\curveto(845.89180974,566.83495582)(846.05680957,566.90495575)(846.23681152,566.95496338)
\curveto(846.41680921,567.01495564)(846.60680902,567.06995558)(846.80681152,567.11996338)
\curveto(846.87680875,567.12995552)(846.94180869,567.13995551)(847.00181152,567.14996338)
\curveto(847.07180856,567.15995549)(847.14680848,567.16995548)(847.22681152,567.17996338)
\curveto(847.25680837,567.18995546)(847.29680833,567.18995546)(847.34681152,567.17996338)
\curveto(847.39680823,567.16995548)(847.4318082,567.17495548)(847.45181152,567.19496338)
}
}
{
\newrgbcolor{curcolor}{0 0 0}
\pscustom[linestyle=none,fillstyle=solid,fillcolor=curcolor]
{
\newpath
\moveto(859.40681152,559.84496338)
\curveto(859.43680369,559.68496297)(859.42180371,559.5499631)(859.36181152,559.43996338)
\curveto(859.30180383,559.33996331)(859.22180391,559.26496339)(859.12181152,559.21496338)
\curveto(859.07180406,559.19496346)(859.01680411,559.18496347)(858.95681152,559.18496338)
\curveto(858.90680422,559.18496347)(858.85180428,559.17496348)(858.79181152,559.15496338)
\curveto(858.57180456,559.10496355)(858.35180478,559.11996353)(858.13181152,559.19996338)
\curveto(857.92180521,559.26996338)(857.77680535,559.35996329)(857.69681152,559.46996338)
\curveto(857.64680548,559.53996311)(857.60180553,559.61996303)(857.56181152,559.70996338)
\curveto(857.52180561,559.80996284)(857.47180566,559.88996276)(857.41181152,559.94996338)
\curveto(857.39180574,559.96996268)(857.36680576,559.98996266)(857.33681152,560.00996338)
\curveto(857.31680581,560.02996262)(857.28680584,560.03496262)(857.24681152,560.02496338)
\curveto(857.13680599,559.99496266)(857.0318061,559.93996271)(856.93181152,559.85996338)
\curveto(856.84180629,559.77996287)(856.75180638,559.70996294)(856.66181152,559.64996338)
\curveto(856.5318066,559.56996308)(856.39180674,559.49496316)(856.24181152,559.42496338)
\curveto(856.09180704,559.36496329)(855.9318072,559.30996334)(855.76181152,559.25996338)
\curveto(855.66180747,559.22996342)(855.55180758,559.20996344)(855.43181152,559.19996338)
\curveto(855.32180781,559.18996346)(855.21180792,559.17496348)(855.10181152,559.15496338)
\curveto(855.05180808,559.14496351)(855.00680812,559.13996351)(854.96681152,559.13996338)
\lineto(854.86181152,559.13996338)
\curveto(854.75180838,559.11996353)(854.64680848,559.11996353)(854.54681152,559.13996338)
\lineto(854.41181152,559.13996338)
\curveto(854.36180877,559.1499635)(854.31180882,559.1549635)(854.26181152,559.15496338)
\curveto(854.21180892,559.1549635)(854.16680896,559.16496349)(854.12681152,559.18496338)
\curveto(854.08680904,559.19496346)(854.05180908,559.19996345)(854.02181152,559.19996338)
\curveto(854.00180913,559.18996346)(853.97680915,559.18996346)(853.94681152,559.19996338)
\lineto(853.70681152,559.25996338)
\curveto(853.6268095,559.26996338)(853.55180958,559.28996336)(853.48181152,559.31996338)
\curveto(853.18180995,559.4499632)(852.93681019,559.59496306)(852.74681152,559.75496338)
\curveto(852.56681056,559.92496273)(852.41681071,560.15996249)(852.29681152,560.45996338)
\curveto(852.20681092,560.67996197)(852.16181097,560.94496171)(852.16181152,561.25496338)
\lineto(852.16181152,561.56996338)
\curveto(852.17181096,561.61996103)(852.17681095,561.66996098)(852.17681152,561.71996338)
\lineto(852.20681152,561.89996338)
\lineto(852.32681152,562.22996338)
\curveto(852.36681076,562.33996031)(852.41681071,562.43996021)(852.47681152,562.52996338)
\curveto(852.65681047,562.81995983)(852.90181023,563.03495962)(853.21181152,563.17496338)
\curveto(853.52180961,563.31495934)(853.86180927,563.43995921)(854.23181152,563.54996338)
\curveto(854.37180876,563.58995906)(854.51680861,563.61995903)(854.66681152,563.63996338)
\curveto(854.81680831,563.65995899)(854.96680816,563.68495897)(855.11681152,563.71496338)
\curveto(855.18680794,563.73495892)(855.25180788,563.74495891)(855.31181152,563.74496338)
\curveto(855.38180775,563.74495891)(855.45680767,563.7549589)(855.53681152,563.77496338)
\curveto(855.60680752,563.79495886)(855.67680745,563.80495885)(855.74681152,563.80496338)
\curveto(855.81680731,563.81495884)(855.89180724,563.82995882)(855.97181152,563.84996338)
\curveto(856.22180691,563.90995874)(856.45680667,563.95995869)(856.67681152,563.99996338)
\curveto(856.89680623,564.0499586)(857.07180606,564.16495849)(857.20181152,564.34496338)
\curveto(857.26180587,564.42495823)(857.31180582,564.52495813)(857.35181152,564.64496338)
\curveto(857.39180574,564.77495788)(857.39180574,564.91495774)(857.35181152,565.06496338)
\curveto(857.29180584,565.30495735)(857.20180593,565.49495716)(857.08181152,565.63496338)
\curveto(856.97180616,565.77495688)(856.81180632,565.88495677)(856.60181152,565.96496338)
\curveto(856.48180665,566.01495664)(856.33680679,566.0499566)(856.16681152,566.06996338)
\curveto(856.00680712,566.08995656)(855.83680729,566.09995655)(855.65681152,566.09996338)
\curveto(855.47680765,566.09995655)(855.30180783,566.08995656)(855.13181152,566.06996338)
\curveto(854.96180817,566.0499566)(854.81680831,566.01995663)(854.69681152,565.97996338)
\curveto(854.5268086,565.91995673)(854.36180877,565.83495682)(854.20181152,565.72496338)
\curveto(854.12180901,565.66495699)(854.04680908,565.58495707)(853.97681152,565.48496338)
\curveto(853.91680921,565.39495726)(853.86180927,565.29495736)(853.81181152,565.18496338)
\curveto(853.78180935,565.10495755)(853.75180938,565.01995763)(853.72181152,564.92996338)
\curveto(853.70180943,564.83995781)(853.65680947,564.76995788)(853.58681152,564.71996338)
\curveto(853.54680958,564.68995796)(853.47680965,564.66495799)(853.37681152,564.64496338)
\curveto(853.28680984,564.63495802)(853.19180994,564.62995802)(853.09181152,564.62996338)
\curveto(852.99181014,564.62995802)(852.89181024,564.63495802)(852.79181152,564.64496338)
\curveto(852.70181043,564.66495799)(852.63681049,564.68995796)(852.59681152,564.71996338)
\curveto(852.55681057,564.7499579)(852.5268106,564.79995785)(852.50681152,564.86996338)
\curveto(852.48681064,564.93995771)(852.48681064,565.01495764)(852.50681152,565.09496338)
\curveto(852.53681059,565.22495743)(852.56681056,565.34495731)(852.59681152,565.45496338)
\curveto(852.63681049,565.57495708)(852.68181045,565.68995696)(852.73181152,565.79996338)
\curveto(852.92181021,566.1499565)(853.16180997,566.41995623)(853.45181152,566.60996338)
\curveto(853.74180939,566.80995584)(854.10180903,566.96995568)(854.53181152,567.08996338)
\curveto(854.6318085,567.10995554)(854.7318084,567.12495553)(854.83181152,567.13496338)
\curveto(854.94180819,567.14495551)(855.05180808,567.15995549)(855.16181152,567.17996338)
\curveto(855.20180793,567.18995546)(855.26680786,567.18995546)(855.35681152,567.17996338)
\curveto(855.44680768,567.17995547)(855.50180763,567.18995546)(855.52181152,567.20996338)
\curveto(856.22180691,567.21995543)(856.8318063,567.13995551)(857.35181152,566.96996338)
\curveto(857.87180526,566.79995585)(858.23680489,566.47495618)(858.44681152,565.99496338)
\curveto(858.53680459,565.79495686)(858.58680454,565.55995709)(858.59681152,565.28996338)
\curveto(858.61680451,565.02995762)(858.6268045,564.7549579)(858.62681152,564.46496338)
\lineto(858.62681152,561.14996338)
\curveto(858.6268045,561.00996164)(858.6318045,560.87496178)(858.64181152,560.74496338)
\curveto(858.65180448,560.61496204)(858.68180445,560.50996214)(858.73181152,560.42996338)
\curveto(858.78180435,560.35996229)(858.84680428,560.30996234)(858.92681152,560.27996338)
\curveto(859.01680411,560.23996241)(859.10180403,560.20996244)(859.18181152,560.18996338)
\curveto(859.26180387,560.17996247)(859.32180381,560.13496252)(859.36181152,560.05496338)
\curveto(859.38180375,560.02496263)(859.39180374,559.99496266)(859.39181152,559.96496338)
\curveto(859.39180374,559.93496272)(859.39680373,559.89496276)(859.40681152,559.84496338)
\moveto(857.26181152,561.50996338)
\curveto(857.32180581,561.649961)(857.35180578,561.80996084)(857.35181152,561.98996338)
\curveto(857.36180577,562.17996047)(857.36680576,562.37496028)(857.36681152,562.57496338)
\curveto(857.36680576,562.68495997)(857.36180577,562.78495987)(857.35181152,562.87496338)
\curveto(857.34180579,562.96495969)(857.30180583,563.03495962)(857.23181152,563.08496338)
\curveto(857.20180593,563.10495955)(857.131806,563.11495954)(857.02181152,563.11496338)
\curveto(857.00180613,563.09495956)(856.96680616,563.08495957)(856.91681152,563.08496338)
\curveto(856.86680626,563.08495957)(856.82180631,563.07495958)(856.78181152,563.05496338)
\curveto(856.70180643,563.03495962)(856.61180652,563.01495964)(856.51181152,562.99496338)
\lineto(856.21181152,562.93496338)
\curveto(856.18180695,562.93495972)(856.14680698,562.92995972)(856.10681152,562.91996338)
\lineto(856.00181152,562.91996338)
\curveto(855.85180728,562.87995977)(855.68680744,562.8549598)(855.50681152,562.84496338)
\curveto(855.33680779,562.84495981)(855.17680795,562.82495983)(855.02681152,562.78496338)
\curveto(854.94680818,562.76495989)(854.87180826,562.74495991)(854.80181152,562.72496338)
\curveto(854.74180839,562.71495994)(854.67180846,562.69995995)(854.59181152,562.67996338)
\curveto(854.4318087,562.62996002)(854.28180885,562.56496009)(854.14181152,562.48496338)
\curveto(854.00180913,562.41496024)(853.88180925,562.32496033)(853.78181152,562.21496338)
\curveto(853.68180945,562.10496055)(853.60680952,561.96996068)(853.55681152,561.80996338)
\curveto(853.50680962,561.65996099)(853.48680964,561.47496118)(853.49681152,561.25496338)
\curveto(853.49680963,561.1549615)(853.51180962,561.05996159)(853.54181152,560.96996338)
\curveto(853.58180955,560.88996176)(853.6268095,560.81496184)(853.67681152,560.74496338)
\curveto(853.75680937,560.63496202)(853.86180927,560.53996211)(853.99181152,560.45996338)
\curveto(854.12180901,560.38996226)(854.26180887,560.32996232)(854.41181152,560.27996338)
\curveto(854.46180867,560.26996238)(854.51180862,560.26496239)(854.56181152,560.26496338)
\curveto(854.61180852,560.26496239)(854.66180847,560.25996239)(854.71181152,560.24996338)
\curveto(854.78180835,560.22996242)(854.86680826,560.21496244)(854.96681152,560.20496338)
\curveto(855.07680805,560.20496245)(855.16680796,560.21496244)(855.23681152,560.23496338)
\curveto(855.29680783,560.2549624)(855.35680777,560.25996239)(855.41681152,560.24996338)
\curveto(855.47680765,560.2499624)(855.53680759,560.25996239)(855.59681152,560.27996338)
\curveto(855.67680745,560.29996235)(855.75180738,560.31496234)(855.82181152,560.32496338)
\curveto(855.90180723,560.33496232)(855.97680715,560.3549623)(856.04681152,560.38496338)
\curveto(856.33680679,560.50496215)(856.58180655,560.649962)(856.78181152,560.81996338)
\curveto(856.99180614,560.98996166)(857.15180598,561.21996143)(857.26181152,561.50996338)
}
}
{
\newrgbcolor{curcolor}{0 0 0}
\pscustom[linestyle=none,fillstyle=solid,fillcolor=curcolor]
{
\newpath
\moveto(864.22345215,567.19496338)
\curveto(864.45344736,567.19495546)(864.58344723,567.13495552)(864.61345215,567.01496338)
\curveto(864.64344717,566.90495575)(864.65844715,566.73995591)(864.65845215,566.51996338)
\lineto(864.65845215,566.23496338)
\curveto(864.65844715,566.14495651)(864.63344718,566.06995658)(864.58345215,566.00996338)
\curveto(864.52344729,565.92995672)(864.43844737,565.88495677)(864.32845215,565.87496338)
\curveto(864.21844759,565.87495678)(864.1084477,565.85995679)(863.99845215,565.82996338)
\curveto(863.85844795,565.79995685)(863.72344809,565.76995688)(863.59345215,565.73996338)
\curveto(863.47344834,565.70995694)(863.35844845,565.66995698)(863.24845215,565.61996338)
\curveto(862.95844885,565.48995716)(862.72344909,565.30995734)(862.54345215,565.07996338)
\curveto(862.36344945,564.85995779)(862.2084496,564.60495805)(862.07845215,564.31496338)
\curveto(862.03844977,564.20495845)(862.0084498,564.08995856)(861.98845215,563.96996338)
\curveto(861.96844984,563.85995879)(861.94344987,563.74495891)(861.91345215,563.62496338)
\curveto(861.90344991,563.57495908)(861.89844991,563.52495913)(861.89845215,563.47496338)
\curveto(861.9084499,563.42495923)(861.9084499,563.37495928)(861.89845215,563.32496338)
\curveto(861.86844994,563.20495945)(861.85344996,563.06495959)(861.85345215,562.90496338)
\curveto(861.86344995,562.7549599)(861.86844994,562.60996004)(861.86845215,562.46996338)
\lineto(861.86845215,560.62496338)
\lineto(861.86845215,560.27996338)
\curveto(861.86844994,560.15996249)(861.86344995,560.04496261)(861.85345215,559.93496338)
\curveto(861.84344997,559.82496283)(861.83844997,559.72996292)(861.83845215,559.64996338)
\curveto(861.84844996,559.56996308)(861.82844998,559.49996315)(861.77845215,559.43996338)
\curveto(861.72845008,559.36996328)(861.64845016,559.32996332)(861.53845215,559.31996338)
\curveto(861.43845037,559.30996334)(861.32845048,559.30496335)(861.20845215,559.30496338)
\lineto(860.93845215,559.30496338)
\curveto(860.88845092,559.32496333)(860.83845097,559.33996331)(860.78845215,559.34996338)
\curveto(860.74845106,559.36996328)(860.71845109,559.39496326)(860.69845215,559.42496338)
\curveto(860.64845116,559.49496316)(860.61845119,559.57996307)(860.60845215,559.67996338)
\lineto(860.60845215,560.00996338)
\lineto(860.60845215,561.16496338)
\lineto(860.60845215,565.31996338)
\lineto(860.60845215,566.35496338)
\lineto(860.60845215,566.65496338)
\curveto(860.61845119,566.7549559)(860.64845116,566.83995581)(860.69845215,566.90996338)
\curveto(860.72845108,566.9499557)(860.77845103,566.97995567)(860.84845215,566.99996338)
\curveto(860.92845088,567.01995563)(861.0134508,567.02995562)(861.10345215,567.02996338)
\curveto(861.19345062,567.03995561)(861.28345053,567.03995561)(861.37345215,567.02996338)
\curveto(861.46345035,567.01995563)(861.53345028,567.00495565)(861.58345215,566.98496338)
\curveto(861.66345015,566.9549557)(861.7134501,566.89495576)(861.73345215,566.80496338)
\curveto(861.76345005,566.72495593)(861.77845003,566.63495602)(861.77845215,566.53496338)
\lineto(861.77845215,566.23496338)
\curveto(861.77845003,566.13495652)(861.79845001,566.04495661)(861.83845215,565.96496338)
\curveto(861.84844996,565.94495671)(861.85844995,565.92995672)(861.86845215,565.91996338)
\lineto(861.91345215,565.87496338)
\curveto(862.02344979,565.87495678)(862.1134497,565.91995673)(862.18345215,566.00996338)
\curveto(862.25344956,566.10995654)(862.3134495,566.18995646)(862.36345215,566.24996338)
\lineto(862.45345215,566.33996338)
\curveto(862.54344927,566.4499562)(862.66844914,566.56495609)(862.82845215,566.68496338)
\curveto(862.98844882,566.80495585)(863.13844867,566.89495576)(863.27845215,566.95496338)
\curveto(863.36844844,567.00495565)(863.46344835,567.03995561)(863.56345215,567.05996338)
\curveto(863.66344815,567.08995556)(863.76844804,567.11995553)(863.87845215,567.14996338)
\curveto(863.93844787,567.15995549)(863.99844781,567.16495549)(864.05845215,567.16496338)
\curveto(864.11844769,567.17495548)(864.17344764,567.18495547)(864.22345215,567.19496338)
}
}
{
\newrgbcolor{curcolor}{0 0 0}
\pscustom[linestyle=none,fillstyle=solid,fillcolor=curcolor]
{
\newpath
\moveto(869.23321777,567.19496338)
\curveto(869.46321298,567.19495546)(869.59321285,567.13495552)(869.62321777,567.01496338)
\curveto(869.65321279,566.90495575)(869.66821278,566.73995591)(869.66821777,566.51996338)
\lineto(869.66821777,566.23496338)
\curveto(869.66821278,566.14495651)(869.6432128,566.06995658)(869.59321777,566.00996338)
\curveto(869.53321291,565.92995672)(869.448213,565.88495677)(869.33821777,565.87496338)
\curveto(869.22821322,565.87495678)(869.11821333,565.85995679)(869.00821777,565.82996338)
\curveto(868.86821358,565.79995685)(868.73321371,565.76995688)(868.60321777,565.73996338)
\curveto(868.48321396,565.70995694)(868.36821408,565.66995698)(868.25821777,565.61996338)
\curveto(867.96821448,565.48995716)(867.73321471,565.30995734)(867.55321777,565.07996338)
\curveto(867.37321507,564.85995779)(867.21821523,564.60495805)(867.08821777,564.31496338)
\curveto(867.0482154,564.20495845)(867.01821543,564.08995856)(866.99821777,563.96996338)
\curveto(866.97821547,563.85995879)(866.95321549,563.74495891)(866.92321777,563.62496338)
\curveto(866.91321553,563.57495908)(866.90821554,563.52495913)(866.90821777,563.47496338)
\curveto(866.91821553,563.42495923)(866.91821553,563.37495928)(866.90821777,563.32496338)
\curveto(866.87821557,563.20495945)(866.86321558,563.06495959)(866.86321777,562.90496338)
\curveto(866.87321557,562.7549599)(866.87821557,562.60996004)(866.87821777,562.46996338)
\lineto(866.87821777,560.62496338)
\lineto(866.87821777,560.27996338)
\curveto(866.87821557,560.15996249)(866.87321557,560.04496261)(866.86321777,559.93496338)
\curveto(866.85321559,559.82496283)(866.8482156,559.72996292)(866.84821777,559.64996338)
\curveto(866.85821559,559.56996308)(866.83821561,559.49996315)(866.78821777,559.43996338)
\curveto(866.73821571,559.36996328)(866.65821579,559.32996332)(866.54821777,559.31996338)
\curveto(866.448216,559.30996334)(866.33821611,559.30496335)(866.21821777,559.30496338)
\lineto(865.94821777,559.30496338)
\curveto(865.89821655,559.32496333)(865.8482166,559.33996331)(865.79821777,559.34996338)
\curveto(865.75821669,559.36996328)(865.72821672,559.39496326)(865.70821777,559.42496338)
\curveto(865.65821679,559.49496316)(865.62821682,559.57996307)(865.61821777,559.67996338)
\lineto(865.61821777,560.00996338)
\lineto(865.61821777,561.16496338)
\lineto(865.61821777,565.31996338)
\lineto(865.61821777,566.35496338)
\lineto(865.61821777,566.65496338)
\curveto(865.62821682,566.7549559)(865.65821679,566.83995581)(865.70821777,566.90996338)
\curveto(865.73821671,566.9499557)(865.78821666,566.97995567)(865.85821777,566.99996338)
\curveto(865.93821651,567.01995563)(866.02321642,567.02995562)(866.11321777,567.02996338)
\curveto(866.20321624,567.03995561)(866.29321615,567.03995561)(866.38321777,567.02996338)
\curveto(866.47321597,567.01995563)(866.5432159,567.00495565)(866.59321777,566.98496338)
\curveto(866.67321577,566.9549557)(866.72321572,566.89495576)(866.74321777,566.80496338)
\curveto(866.77321567,566.72495593)(866.78821566,566.63495602)(866.78821777,566.53496338)
\lineto(866.78821777,566.23496338)
\curveto(866.78821566,566.13495652)(866.80821564,566.04495661)(866.84821777,565.96496338)
\curveto(866.85821559,565.94495671)(866.86821558,565.92995672)(866.87821777,565.91996338)
\lineto(866.92321777,565.87496338)
\curveto(867.03321541,565.87495678)(867.12321532,565.91995673)(867.19321777,566.00996338)
\curveto(867.26321518,566.10995654)(867.32321512,566.18995646)(867.37321777,566.24996338)
\lineto(867.46321777,566.33996338)
\curveto(867.55321489,566.4499562)(867.67821477,566.56495609)(867.83821777,566.68496338)
\curveto(867.99821445,566.80495585)(868.1482143,566.89495576)(868.28821777,566.95496338)
\curveto(868.37821407,567.00495565)(868.47321397,567.03995561)(868.57321777,567.05996338)
\curveto(868.67321377,567.08995556)(868.77821367,567.11995553)(868.88821777,567.14996338)
\curveto(868.9482135,567.15995549)(869.00821344,567.16495549)(869.06821777,567.16496338)
\curveto(869.12821332,567.17495548)(869.18321326,567.18495547)(869.23321777,567.19496338)
}
}
{
\newrgbcolor{curcolor}{0 0 0}
\pscustom[linestyle=none,fillstyle=solid,fillcolor=curcolor]
{
\newpath
\moveto(877.7229834,563.48996338)
\curveto(877.74297534,563.42995922)(877.75297533,563.33495932)(877.7529834,563.20496338)
\curveto(877.75297533,563.08495957)(877.74797533,562.99995965)(877.7379834,562.94996338)
\lineto(877.7379834,562.79996338)
\curveto(877.72797535,562.71995993)(877.71797536,562.64496001)(877.7079834,562.57496338)
\curveto(877.70797537,562.51496014)(877.70297538,562.44496021)(877.6929834,562.36496338)
\curveto(877.67297541,562.30496035)(877.65797542,562.24496041)(877.6479834,562.18496338)
\curveto(877.64797543,562.12496053)(877.63797544,562.06496059)(877.6179834,562.00496338)
\curveto(877.5779755,561.87496078)(877.54297554,561.74496091)(877.5129834,561.61496338)
\curveto(877.4829756,561.48496117)(877.44297564,561.36496129)(877.3929834,561.25496338)
\curveto(877.1829759,560.77496188)(876.90297618,560.36996228)(876.5529834,560.03996338)
\curveto(876.20297688,559.71996293)(875.77297731,559.47496318)(875.2629834,559.30496338)
\curveto(875.15297793,559.26496339)(875.03297805,559.23496342)(874.9029834,559.21496338)
\curveto(874.7829783,559.19496346)(874.65797842,559.17496348)(874.5279834,559.15496338)
\curveto(874.46797861,559.14496351)(874.40297868,559.13996351)(874.3329834,559.13996338)
\curveto(874.27297881,559.12996352)(874.21297887,559.12496353)(874.1529834,559.12496338)
\curveto(874.11297897,559.11496354)(874.05297903,559.10996354)(873.9729834,559.10996338)
\curveto(873.90297918,559.10996354)(873.85297923,559.11496354)(873.8229834,559.12496338)
\curveto(873.7829793,559.13496352)(873.74297934,559.13996351)(873.7029834,559.13996338)
\curveto(873.66297942,559.12996352)(873.62797945,559.12996352)(873.5979834,559.13996338)
\lineto(873.5079834,559.13996338)
\lineto(873.1479834,559.18496338)
\curveto(873.00798007,559.22496343)(872.87298021,559.26496339)(872.7429834,559.30496338)
\curveto(872.61298047,559.34496331)(872.48798059,559.38996326)(872.3679834,559.43996338)
\curveto(871.91798116,559.63996301)(871.54798153,559.89996275)(871.2579834,560.21996338)
\curveto(870.96798211,560.53996211)(870.72798235,560.92996172)(870.5379834,561.38996338)
\curveto(870.48798259,561.48996116)(870.44798263,561.58996106)(870.4179834,561.68996338)
\curveto(870.39798268,561.78996086)(870.3779827,561.89496076)(870.3579834,562.00496338)
\curveto(870.33798274,562.04496061)(870.32798275,562.07496058)(870.3279834,562.09496338)
\curveto(870.33798274,562.12496053)(870.33798274,562.15996049)(870.3279834,562.19996338)
\curveto(870.30798277,562.27996037)(870.29298279,562.35996029)(870.2829834,562.43996338)
\curveto(870.2829828,562.52996012)(870.27298281,562.61496004)(870.2529834,562.69496338)
\lineto(870.2529834,562.81496338)
\curveto(870.25298283,562.8549598)(870.24798283,562.89995975)(870.2379834,562.94996338)
\curveto(870.22798285,562.99995965)(870.22298286,563.08495957)(870.2229834,563.20496338)
\curveto(870.22298286,563.33495932)(870.23298285,563.42995922)(870.2529834,563.48996338)
\curveto(870.27298281,563.55995909)(870.2779828,563.62995902)(870.2679834,563.69996338)
\curveto(870.25798282,563.76995888)(870.26298282,563.83995881)(870.2829834,563.90996338)
\curveto(870.29298279,563.95995869)(870.29798278,563.99995865)(870.2979834,564.02996338)
\curveto(870.30798277,564.06995858)(870.31798276,564.11495854)(870.3279834,564.16496338)
\curveto(870.35798272,564.28495837)(870.3829827,564.40495825)(870.4029834,564.52496338)
\curveto(870.43298265,564.64495801)(870.47298261,564.75995789)(870.5229834,564.86996338)
\curveto(870.67298241,565.23995741)(870.85298223,565.56995708)(871.0629834,565.85996338)
\curveto(871.2829818,566.15995649)(871.54798153,566.40995624)(871.8579834,566.60996338)
\curveto(871.9779811,566.68995596)(872.10298098,566.7549559)(872.2329834,566.80496338)
\curveto(872.36298072,566.86495579)(872.49798058,566.92495573)(872.6379834,566.98496338)
\curveto(872.75798032,567.03495562)(872.88798019,567.06495559)(873.0279834,567.07496338)
\curveto(873.16797991,567.09495556)(873.30797977,567.12495553)(873.4479834,567.16496338)
\lineto(873.6429834,567.16496338)
\curveto(873.71297937,567.17495548)(873.7779793,567.18495547)(873.8379834,567.19496338)
\curveto(874.72797835,567.20495545)(875.46797761,567.01995563)(876.0579834,566.63996338)
\curveto(876.64797643,566.25995639)(877.07297601,565.76495689)(877.3329834,565.15496338)
\curveto(877.3829757,565.0549576)(877.42297566,564.9549577)(877.4529834,564.85496338)
\curveto(877.4829756,564.7549579)(877.51797556,564.649958)(877.5579834,564.53996338)
\curveto(877.58797549,564.42995822)(877.61297547,564.30995834)(877.6329834,564.17996338)
\curveto(877.65297543,564.05995859)(877.6779754,563.93495872)(877.7079834,563.80496338)
\curveto(877.71797536,563.7549589)(877.71797536,563.69995895)(877.7079834,563.63996338)
\curveto(877.70797537,563.58995906)(877.71297537,563.53995911)(877.7229834,563.48996338)
\moveto(876.3879834,562.63496338)
\curveto(876.40797667,562.70495995)(876.41297667,562.78495987)(876.4029834,562.87496338)
\lineto(876.4029834,563.12996338)
\curveto(876.40297668,563.51995913)(876.36797671,563.8499588)(876.2979834,564.11996338)
\curveto(876.26797681,564.19995845)(876.24297684,564.27995837)(876.2229834,564.35996338)
\curveto(876.20297688,564.43995821)(876.1779769,564.51495814)(876.1479834,564.58496338)
\curveto(875.86797721,565.23495742)(875.42297766,565.68495697)(874.8129834,565.93496338)
\curveto(874.74297834,565.96495669)(874.66797841,565.98495667)(874.5879834,565.99496338)
\lineto(874.3479834,566.05496338)
\curveto(874.26797881,566.07495658)(874.1829789,566.08495657)(874.0929834,566.08496338)
\lineto(873.8229834,566.08496338)
\lineto(873.5529834,566.03996338)
\curveto(873.45297963,566.01995663)(873.35797972,565.99495666)(873.2679834,565.96496338)
\curveto(873.18797989,565.94495671)(873.10797997,565.91495674)(873.0279834,565.87496338)
\curveto(872.95798012,565.8549568)(872.89298019,565.82495683)(872.8329834,565.78496338)
\curveto(872.77298031,565.74495691)(872.71798036,565.70495695)(872.6679834,565.66496338)
\curveto(872.42798065,565.49495716)(872.23298085,565.28995736)(872.0829834,565.04996338)
\curveto(871.93298115,564.80995784)(871.80298128,564.52995812)(871.6929834,564.20996338)
\curveto(871.66298142,564.10995854)(871.64298144,564.00495865)(871.6329834,563.89496338)
\curveto(871.62298146,563.79495886)(871.60798147,563.68995896)(871.5879834,563.57996338)
\curveto(871.5779815,563.53995911)(871.57298151,563.47495918)(871.5729834,563.38496338)
\curveto(871.56298152,563.3549593)(871.55798152,563.31995933)(871.5579834,563.27996338)
\curveto(871.56798151,563.23995941)(871.57298151,563.19495946)(871.5729834,563.14496338)
\lineto(871.5729834,562.84496338)
\curveto(871.57298151,562.74495991)(871.5829815,562.65496)(871.6029834,562.57496338)
\lineto(871.6329834,562.39496338)
\curveto(871.65298143,562.29496036)(871.66798141,562.19496046)(871.6779834,562.09496338)
\curveto(871.69798138,562.00496065)(871.72798135,561.91996073)(871.7679834,561.83996338)
\curveto(871.86798121,561.59996105)(871.9829811,561.37496128)(872.1129834,561.16496338)
\curveto(872.25298083,560.9549617)(872.42298066,560.77996187)(872.6229834,560.63996338)
\curveto(872.67298041,560.60996204)(872.71798036,560.58496207)(872.7579834,560.56496338)
\curveto(872.79798028,560.54496211)(872.84298024,560.51996213)(872.8929834,560.48996338)
\curveto(872.97298011,560.43996221)(873.05798002,560.39496226)(873.1479834,560.35496338)
\curveto(873.24797983,560.32496233)(873.35297973,560.29496236)(873.4629834,560.26496338)
\curveto(873.51297957,560.24496241)(873.55797952,560.23496242)(873.5979834,560.23496338)
\curveto(873.64797943,560.24496241)(873.69797938,560.24496241)(873.7479834,560.23496338)
\curveto(873.7779793,560.22496243)(873.83797924,560.21496244)(873.9279834,560.20496338)
\curveto(874.02797905,560.19496246)(874.10297898,560.19996245)(874.1529834,560.21996338)
\curveto(874.19297889,560.22996242)(874.23297885,560.22996242)(874.2729834,560.21996338)
\curveto(874.31297877,560.21996243)(874.35297873,560.22996242)(874.3929834,560.24996338)
\curveto(874.47297861,560.26996238)(874.55297853,560.28496237)(874.6329834,560.29496338)
\curveto(874.71297837,560.31496234)(874.78797829,560.33996231)(874.8579834,560.36996338)
\curveto(875.19797788,560.50996214)(875.47297761,560.70496195)(875.6829834,560.95496338)
\curveto(875.89297719,561.20496145)(876.06797701,561.49996115)(876.2079834,561.83996338)
\curveto(876.25797682,561.95996069)(876.28797679,562.08496057)(876.2979834,562.21496338)
\curveto(876.31797676,562.3549603)(876.34797673,562.49496016)(876.3879834,562.63496338)
}
}
{
\newrgbcolor{curcolor}{0 0 0}
\pscustom[linestyle=none,fillstyle=solid,fillcolor=curcolor]
{
\newpath
\moveto(879.78126465,569.96996338)
\curveto(879.91126303,569.96995268)(880.0462629,569.96995268)(880.18626465,569.96996338)
\curveto(880.33626261,569.96995268)(880.4462625,569.93495272)(880.51626465,569.86496338)
\curveto(880.56626238,569.79495286)(880.59126235,569.69995295)(880.59126465,569.57996338)
\curveto(880.60126234,569.46995318)(880.60626234,569.3549533)(880.60626465,569.23496338)
\lineto(880.60626465,567.89996338)
\lineto(880.60626465,561.82496338)
\lineto(880.60626465,560.14496338)
\lineto(880.60626465,559.75496338)
\curveto(880.60626234,559.61496304)(880.58126236,559.50496315)(880.53126465,559.42496338)
\curveto(880.50126244,559.37496328)(880.45626249,559.34496331)(880.39626465,559.33496338)
\curveto(880.3462626,559.32496333)(880.28126266,559.30996334)(880.20126465,559.28996338)
\lineto(879.99126465,559.28996338)
\lineto(879.67626465,559.28996338)
\curveto(879.57626337,559.29996335)(879.50126344,559.33496332)(879.45126465,559.39496338)
\curveto(879.40126354,559.47496318)(879.37126357,559.57496308)(879.36126465,559.69496338)
\lineto(879.36126465,560.06996338)
\lineto(879.36126465,561.44996338)
\lineto(879.36126465,567.68996338)
\lineto(879.36126465,569.15996338)
\curveto(879.36126358,569.26995338)(879.35626359,569.38495327)(879.34626465,569.50496338)
\curveto(879.3462636,569.63495302)(879.37126357,569.73495292)(879.42126465,569.80496338)
\curveto(879.46126348,569.86495279)(879.53626341,569.91495274)(879.64626465,569.95496338)
\curveto(879.66626328,569.96495269)(879.68626326,569.96495269)(879.70626465,569.95496338)
\curveto(879.73626321,569.9549527)(879.76126318,569.95995269)(879.78126465,569.96996338)
}
}
{
\newrgbcolor{curcolor}{0 0 0}
\pscustom[linestyle=none,fillstyle=solid,fillcolor=curcolor]
{
\newpath
\moveto(883.1211084,569.96996338)
\curveto(883.25110678,569.96995268)(883.38610665,569.96995268)(883.5261084,569.96996338)
\curveto(883.67610636,569.96995268)(883.78610625,569.93495272)(883.8561084,569.86496338)
\curveto(883.90610613,569.79495286)(883.9311061,569.69995295)(883.9311084,569.57996338)
\curveto(883.94110609,569.46995318)(883.94610609,569.3549533)(883.9461084,569.23496338)
\lineto(883.9461084,567.89996338)
\lineto(883.9461084,561.82496338)
\lineto(883.9461084,560.14496338)
\lineto(883.9461084,559.75496338)
\curveto(883.94610609,559.61496304)(883.92110611,559.50496315)(883.8711084,559.42496338)
\curveto(883.84110619,559.37496328)(883.79610624,559.34496331)(883.7361084,559.33496338)
\curveto(883.68610635,559.32496333)(883.62110641,559.30996334)(883.5411084,559.28996338)
\lineto(883.3311084,559.28996338)
\lineto(883.0161084,559.28996338)
\curveto(882.91610712,559.29996335)(882.84110719,559.33496332)(882.7911084,559.39496338)
\curveto(882.74110729,559.47496318)(882.71110732,559.57496308)(882.7011084,559.69496338)
\lineto(882.7011084,560.06996338)
\lineto(882.7011084,561.44996338)
\lineto(882.7011084,567.68996338)
\lineto(882.7011084,569.15996338)
\curveto(882.70110733,569.26995338)(882.69610734,569.38495327)(882.6861084,569.50496338)
\curveto(882.68610735,569.63495302)(882.71110732,569.73495292)(882.7611084,569.80496338)
\curveto(882.80110723,569.86495279)(882.87610716,569.91495274)(882.9861084,569.95496338)
\curveto(883.00610703,569.96495269)(883.02610701,569.96495269)(883.0461084,569.95496338)
\curveto(883.07610696,569.9549527)(883.10110693,569.95995269)(883.1211084,569.96996338)
}
}
{
\newrgbcolor{curcolor}{0 0 0}
\pscustom[linestyle=none,fillstyle=solid,fillcolor=curcolor]
{
\newpath
\moveto(892.77595215,559.84496338)
\curveto(892.80594432,559.68496297)(892.79094433,559.5499631)(892.73095215,559.43996338)
\curveto(892.67094445,559.33996331)(892.59094453,559.26496339)(892.49095215,559.21496338)
\curveto(892.44094468,559.19496346)(892.38594474,559.18496347)(892.32595215,559.18496338)
\curveto(892.27594485,559.18496347)(892.2209449,559.17496348)(892.16095215,559.15496338)
\curveto(891.94094518,559.10496355)(891.7209454,559.11996353)(891.50095215,559.19996338)
\curveto(891.29094583,559.26996338)(891.14594598,559.35996329)(891.06595215,559.46996338)
\curveto(891.01594611,559.53996311)(890.97094615,559.61996303)(890.93095215,559.70996338)
\curveto(890.89094623,559.80996284)(890.84094628,559.88996276)(890.78095215,559.94996338)
\curveto(890.76094636,559.96996268)(890.73594639,559.98996266)(890.70595215,560.00996338)
\curveto(890.68594644,560.02996262)(890.65594647,560.03496262)(890.61595215,560.02496338)
\curveto(890.50594662,559.99496266)(890.40094672,559.93996271)(890.30095215,559.85996338)
\curveto(890.21094691,559.77996287)(890.120947,559.70996294)(890.03095215,559.64996338)
\curveto(889.90094722,559.56996308)(889.76094736,559.49496316)(889.61095215,559.42496338)
\curveto(889.46094766,559.36496329)(889.30094782,559.30996334)(889.13095215,559.25996338)
\curveto(889.03094809,559.22996342)(888.9209482,559.20996344)(888.80095215,559.19996338)
\curveto(888.69094843,559.18996346)(888.58094854,559.17496348)(888.47095215,559.15496338)
\curveto(888.4209487,559.14496351)(888.37594875,559.13996351)(888.33595215,559.13996338)
\lineto(888.23095215,559.13996338)
\curveto(888.120949,559.11996353)(888.01594911,559.11996353)(887.91595215,559.13996338)
\lineto(887.78095215,559.13996338)
\curveto(887.73094939,559.1499635)(887.68094944,559.1549635)(887.63095215,559.15496338)
\curveto(887.58094954,559.1549635)(887.53594959,559.16496349)(887.49595215,559.18496338)
\curveto(887.45594967,559.19496346)(887.4209497,559.19996345)(887.39095215,559.19996338)
\curveto(887.37094975,559.18996346)(887.34594978,559.18996346)(887.31595215,559.19996338)
\lineto(887.07595215,559.25996338)
\curveto(886.99595013,559.26996338)(886.9209502,559.28996336)(886.85095215,559.31996338)
\curveto(886.55095057,559.4499632)(886.30595082,559.59496306)(886.11595215,559.75496338)
\curveto(885.93595119,559.92496273)(885.78595134,560.15996249)(885.66595215,560.45996338)
\curveto(885.57595155,560.67996197)(885.53095159,560.94496171)(885.53095215,561.25496338)
\lineto(885.53095215,561.56996338)
\curveto(885.54095158,561.61996103)(885.54595158,561.66996098)(885.54595215,561.71996338)
\lineto(885.57595215,561.89996338)
\lineto(885.69595215,562.22996338)
\curveto(885.73595139,562.33996031)(885.78595134,562.43996021)(885.84595215,562.52996338)
\curveto(886.0259511,562.81995983)(886.27095085,563.03495962)(886.58095215,563.17496338)
\curveto(886.89095023,563.31495934)(887.23094989,563.43995921)(887.60095215,563.54996338)
\curveto(887.74094938,563.58995906)(887.88594924,563.61995903)(888.03595215,563.63996338)
\curveto(888.18594894,563.65995899)(888.33594879,563.68495897)(888.48595215,563.71496338)
\curveto(888.55594857,563.73495892)(888.6209485,563.74495891)(888.68095215,563.74496338)
\curveto(888.75094837,563.74495891)(888.8259483,563.7549589)(888.90595215,563.77496338)
\curveto(888.97594815,563.79495886)(889.04594808,563.80495885)(889.11595215,563.80496338)
\curveto(889.18594794,563.81495884)(889.26094786,563.82995882)(889.34095215,563.84996338)
\curveto(889.59094753,563.90995874)(889.8259473,563.95995869)(890.04595215,563.99996338)
\curveto(890.26594686,564.0499586)(890.44094668,564.16495849)(890.57095215,564.34496338)
\curveto(890.63094649,564.42495823)(890.68094644,564.52495813)(890.72095215,564.64496338)
\curveto(890.76094636,564.77495788)(890.76094636,564.91495774)(890.72095215,565.06496338)
\curveto(890.66094646,565.30495735)(890.57094655,565.49495716)(890.45095215,565.63496338)
\curveto(890.34094678,565.77495688)(890.18094694,565.88495677)(889.97095215,565.96496338)
\curveto(889.85094727,566.01495664)(889.70594742,566.0499566)(889.53595215,566.06996338)
\curveto(889.37594775,566.08995656)(889.20594792,566.09995655)(889.02595215,566.09996338)
\curveto(888.84594828,566.09995655)(888.67094845,566.08995656)(888.50095215,566.06996338)
\curveto(888.33094879,566.0499566)(888.18594894,566.01995663)(888.06595215,565.97996338)
\curveto(887.89594923,565.91995673)(887.73094939,565.83495682)(887.57095215,565.72496338)
\curveto(887.49094963,565.66495699)(887.41594971,565.58495707)(887.34595215,565.48496338)
\curveto(887.28594984,565.39495726)(887.23094989,565.29495736)(887.18095215,565.18496338)
\curveto(887.15094997,565.10495755)(887.12095,565.01995763)(887.09095215,564.92996338)
\curveto(887.07095005,564.83995781)(887.0259501,564.76995788)(886.95595215,564.71996338)
\curveto(886.91595021,564.68995796)(886.84595028,564.66495799)(886.74595215,564.64496338)
\curveto(886.65595047,564.63495802)(886.56095056,564.62995802)(886.46095215,564.62996338)
\curveto(886.36095076,564.62995802)(886.26095086,564.63495802)(886.16095215,564.64496338)
\curveto(886.07095105,564.66495799)(886.00595112,564.68995796)(885.96595215,564.71996338)
\curveto(885.9259512,564.7499579)(885.89595123,564.79995785)(885.87595215,564.86996338)
\curveto(885.85595127,564.93995771)(885.85595127,565.01495764)(885.87595215,565.09496338)
\curveto(885.90595122,565.22495743)(885.93595119,565.34495731)(885.96595215,565.45496338)
\curveto(886.00595112,565.57495708)(886.05095107,565.68995696)(886.10095215,565.79996338)
\curveto(886.29095083,566.1499565)(886.53095059,566.41995623)(886.82095215,566.60996338)
\curveto(887.11095001,566.80995584)(887.47094965,566.96995568)(887.90095215,567.08996338)
\curveto(888.00094912,567.10995554)(888.10094902,567.12495553)(888.20095215,567.13496338)
\curveto(888.31094881,567.14495551)(888.4209487,567.15995549)(888.53095215,567.17996338)
\curveto(888.57094855,567.18995546)(888.63594849,567.18995546)(888.72595215,567.17996338)
\curveto(888.81594831,567.17995547)(888.87094825,567.18995546)(888.89095215,567.20996338)
\curveto(889.59094753,567.21995543)(890.20094692,567.13995551)(890.72095215,566.96996338)
\curveto(891.24094588,566.79995585)(891.60594552,566.47495618)(891.81595215,565.99496338)
\curveto(891.90594522,565.79495686)(891.95594517,565.55995709)(891.96595215,565.28996338)
\curveto(891.98594514,565.02995762)(891.99594513,564.7549579)(891.99595215,564.46496338)
\lineto(891.99595215,561.14996338)
\curveto(891.99594513,561.00996164)(892.00094512,560.87496178)(892.01095215,560.74496338)
\curveto(892.0209451,560.61496204)(892.05094507,560.50996214)(892.10095215,560.42996338)
\curveto(892.15094497,560.35996229)(892.21594491,560.30996234)(892.29595215,560.27996338)
\curveto(892.38594474,560.23996241)(892.47094465,560.20996244)(892.55095215,560.18996338)
\curveto(892.63094449,560.17996247)(892.69094443,560.13496252)(892.73095215,560.05496338)
\curveto(892.75094437,560.02496263)(892.76094436,559.99496266)(892.76095215,559.96496338)
\curveto(892.76094436,559.93496272)(892.76594436,559.89496276)(892.77595215,559.84496338)
\moveto(890.63095215,561.50996338)
\curveto(890.69094643,561.649961)(890.7209464,561.80996084)(890.72095215,561.98996338)
\curveto(890.73094639,562.17996047)(890.73594639,562.37496028)(890.73595215,562.57496338)
\curveto(890.73594639,562.68495997)(890.73094639,562.78495987)(890.72095215,562.87496338)
\curveto(890.71094641,562.96495969)(890.67094645,563.03495962)(890.60095215,563.08496338)
\curveto(890.57094655,563.10495955)(890.50094662,563.11495954)(890.39095215,563.11496338)
\curveto(890.37094675,563.09495956)(890.33594679,563.08495957)(890.28595215,563.08496338)
\curveto(890.23594689,563.08495957)(890.19094693,563.07495958)(890.15095215,563.05496338)
\curveto(890.07094705,563.03495962)(889.98094714,563.01495964)(889.88095215,562.99496338)
\lineto(889.58095215,562.93496338)
\curveto(889.55094757,562.93495972)(889.51594761,562.92995972)(889.47595215,562.91996338)
\lineto(889.37095215,562.91996338)
\curveto(889.2209479,562.87995977)(889.05594807,562.8549598)(888.87595215,562.84496338)
\curveto(888.70594842,562.84495981)(888.54594858,562.82495983)(888.39595215,562.78496338)
\curveto(888.31594881,562.76495989)(888.24094888,562.74495991)(888.17095215,562.72496338)
\curveto(888.11094901,562.71495994)(888.04094908,562.69995995)(887.96095215,562.67996338)
\curveto(887.80094932,562.62996002)(887.65094947,562.56496009)(887.51095215,562.48496338)
\curveto(887.37094975,562.41496024)(887.25094987,562.32496033)(887.15095215,562.21496338)
\curveto(887.05095007,562.10496055)(886.97595015,561.96996068)(886.92595215,561.80996338)
\curveto(886.87595025,561.65996099)(886.85595027,561.47496118)(886.86595215,561.25496338)
\curveto(886.86595026,561.1549615)(886.88095024,561.05996159)(886.91095215,560.96996338)
\curveto(886.95095017,560.88996176)(886.99595013,560.81496184)(887.04595215,560.74496338)
\curveto(887.12595,560.63496202)(887.23094989,560.53996211)(887.36095215,560.45996338)
\curveto(887.49094963,560.38996226)(887.63094949,560.32996232)(887.78095215,560.27996338)
\curveto(887.83094929,560.26996238)(887.88094924,560.26496239)(887.93095215,560.26496338)
\curveto(887.98094914,560.26496239)(888.03094909,560.25996239)(888.08095215,560.24996338)
\curveto(888.15094897,560.22996242)(888.23594889,560.21496244)(888.33595215,560.20496338)
\curveto(888.44594868,560.20496245)(888.53594859,560.21496244)(888.60595215,560.23496338)
\curveto(888.66594846,560.2549624)(888.7259484,560.25996239)(888.78595215,560.24996338)
\curveto(888.84594828,560.2499624)(888.90594822,560.25996239)(888.96595215,560.27996338)
\curveto(889.04594808,560.29996235)(889.120948,560.31496234)(889.19095215,560.32496338)
\curveto(889.27094785,560.33496232)(889.34594778,560.3549623)(889.41595215,560.38496338)
\curveto(889.70594742,560.50496215)(889.95094717,560.649962)(890.15095215,560.81996338)
\curveto(890.36094676,560.98996166)(890.5209466,561.21996143)(890.63095215,561.50996338)
}
}
{
\newrgbcolor{curcolor}{0 0 0}
\pscustom[linestyle=none,fillstyle=solid,fillcolor=curcolor]
{
\newpath
\moveto(900.90759277,560.09996338)
\lineto(900.90759277,559.70996338)
\curveto(900.9075849,559.58996306)(900.88258492,559.48996316)(900.83259277,559.40996338)
\curveto(900.78258502,559.33996331)(900.69758511,559.29996335)(900.57759277,559.28996338)
\lineto(900.23259277,559.28996338)
\curveto(900.17258563,559.28996336)(900.11258569,559.28496337)(900.05259277,559.27496338)
\curveto(900.0025858,559.27496338)(899.95758585,559.28496337)(899.91759277,559.30496338)
\curveto(899.82758598,559.32496333)(899.76758604,559.36496329)(899.73759277,559.42496338)
\curveto(899.69758611,559.47496318)(899.67258613,559.53496312)(899.66259277,559.60496338)
\curveto(899.66258614,559.67496298)(899.64758616,559.74496291)(899.61759277,559.81496338)
\curveto(899.6075862,559.83496282)(899.59258621,559.8499628)(899.57259277,559.85996338)
\curveto(899.56258624,559.87996277)(899.54758626,559.89996275)(899.52759277,559.91996338)
\curveto(899.42758638,559.92996272)(899.34758646,559.90996274)(899.28759277,559.85996338)
\curveto(899.23758657,559.80996284)(899.18258662,559.75996289)(899.12259277,559.70996338)
\curveto(898.92258688,559.55996309)(898.72258708,559.44496321)(898.52259277,559.36496338)
\curveto(898.34258746,559.28496337)(898.13258767,559.22496343)(897.89259277,559.18496338)
\curveto(897.66258814,559.14496351)(897.42258838,559.12496353)(897.17259277,559.12496338)
\curveto(896.93258887,559.11496354)(896.69258911,559.12996352)(896.45259277,559.16996338)
\curveto(896.21258959,559.19996345)(896.0025898,559.2549634)(895.82259277,559.33496338)
\curveto(895.3025905,559.5549631)(894.88259092,559.8499628)(894.56259277,560.21996338)
\curveto(894.24259156,560.59996205)(893.99259181,561.06996158)(893.81259277,561.62996338)
\curveto(893.77259203,561.71996093)(893.74259206,561.80996084)(893.72259277,561.89996338)
\curveto(893.71259209,561.99996065)(893.69259211,562.09996055)(893.66259277,562.19996338)
\curveto(893.65259215,562.2499604)(893.64759216,562.29996035)(893.64759277,562.34996338)
\curveto(893.64759216,562.39996025)(893.64259216,562.4499602)(893.63259277,562.49996338)
\curveto(893.61259219,562.5499601)(893.6025922,562.59996005)(893.60259277,562.64996338)
\curveto(893.61259219,562.70995994)(893.61259219,562.76495989)(893.60259277,562.81496338)
\lineto(893.60259277,562.96496338)
\curveto(893.58259222,563.01495964)(893.57259223,563.07995957)(893.57259277,563.15996338)
\curveto(893.57259223,563.23995941)(893.58259222,563.30495935)(893.60259277,563.35496338)
\lineto(893.60259277,563.51996338)
\curveto(893.62259218,563.58995906)(893.62759218,563.65995899)(893.61759277,563.72996338)
\curveto(893.61759219,563.80995884)(893.62759218,563.88495877)(893.64759277,563.95496338)
\curveto(893.65759215,564.00495865)(893.66259214,564.0499586)(893.66259277,564.08996338)
\curveto(893.66259214,564.12995852)(893.66759214,564.17495848)(893.67759277,564.22496338)
\curveto(893.7075921,564.32495833)(893.73259207,564.41995823)(893.75259277,564.50996338)
\curveto(893.77259203,564.60995804)(893.79759201,564.70495795)(893.82759277,564.79496338)
\curveto(893.95759185,565.17495748)(894.12259168,565.51495714)(894.32259277,565.81496338)
\curveto(894.53259127,566.12495653)(894.78259102,566.37995627)(895.07259277,566.57996338)
\curveto(895.24259056,566.69995595)(895.41759039,566.79995585)(895.59759277,566.87996338)
\curveto(895.78759002,566.95995569)(895.99258981,567.02995562)(896.21259277,567.08996338)
\curveto(896.28258952,567.09995555)(896.34758946,567.10995554)(896.40759277,567.11996338)
\curveto(896.47758933,567.12995552)(896.54758926,567.14495551)(896.61759277,567.16496338)
\lineto(896.76759277,567.16496338)
\curveto(896.84758896,567.18495547)(896.96258884,567.19495546)(897.11259277,567.19496338)
\curveto(897.27258853,567.19495546)(897.39258841,567.18495547)(897.47259277,567.16496338)
\curveto(897.51258829,567.1549555)(897.56758824,567.1499555)(897.63759277,567.14996338)
\curveto(897.74758806,567.11995553)(897.85758795,567.09495556)(897.96759277,567.07496338)
\curveto(898.07758773,567.06495559)(898.18258762,567.03495562)(898.28259277,566.98496338)
\curveto(898.43258737,566.92495573)(898.57258723,566.85995579)(898.70259277,566.78996338)
\curveto(898.84258696,566.71995593)(898.97258683,566.63995601)(899.09259277,566.54996338)
\curveto(899.15258665,566.49995615)(899.21258659,566.44495621)(899.27259277,566.38496338)
\curveto(899.34258646,566.33495632)(899.43258637,566.31995633)(899.54259277,566.33996338)
\curveto(899.56258624,566.36995628)(899.57758623,566.39495626)(899.58759277,566.41496338)
\curveto(899.6075862,566.43495622)(899.62258618,566.46495619)(899.63259277,566.50496338)
\curveto(899.66258614,566.59495606)(899.67258613,566.70995594)(899.66259277,566.84996338)
\lineto(899.66259277,567.22496338)
\lineto(899.66259277,568.94996338)
\lineto(899.66259277,569.41496338)
\curveto(899.66258614,569.59495306)(899.68758612,569.72495293)(899.73759277,569.80496338)
\curveto(899.77758603,569.87495278)(899.83758597,569.91995273)(899.91759277,569.93996338)
\curveto(899.93758587,569.93995271)(899.96258584,569.93995271)(899.99259277,569.93996338)
\curveto(900.02258578,569.9499527)(900.04758576,569.9549527)(900.06759277,569.95496338)
\curveto(900.2075856,569.96495269)(900.35258545,569.96495269)(900.50259277,569.95496338)
\curveto(900.66258514,569.9549527)(900.77258503,569.91495274)(900.83259277,569.83496338)
\curveto(900.88258492,569.7549529)(900.9075849,569.654953)(900.90759277,569.53496338)
\lineto(900.90759277,569.15996338)
\lineto(900.90759277,560.09996338)
\moveto(899.69259277,562.93496338)
\curveto(899.71258609,562.98495967)(899.72258608,563.0499596)(899.72259277,563.12996338)
\curveto(899.72258608,563.21995943)(899.71258609,563.28995936)(899.69259277,563.33996338)
\lineto(899.69259277,563.56496338)
\curveto(899.67258613,563.654959)(899.65758615,563.74495891)(899.64759277,563.83496338)
\curveto(899.63758617,563.93495872)(899.61758619,564.02495863)(899.58759277,564.10496338)
\curveto(899.56758624,564.18495847)(899.54758626,564.25995839)(899.52759277,564.32996338)
\curveto(899.51758629,564.39995825)(899.49758631,564.46995818)(899.46759277,564.53996338)
\curveto(899.34758646,564.83995781)(899.19258661,565.10495755)(899.00259277,565.33496338)
\curveto(898.81258699,565.56495709)(898.57258723,565.74495691)(898.28259277,565.87496338)
\curveto(898.18258762,565.92495673)(898.07758773,565.95995669)(897.96759277,565.97996338)
\curveto(897.86758794,566.00995664)(897.75758805,566.03495662)(897.63759277,566.05496338)
\curveto(897.55758825,566.07495658)(897.46758834,566.08495657)(897.36759277,566.08496338)
\lineto(897.09759277,566.08496338)
\curveto(897.04758876,566.07495658)(897.0025888,566.06495659)(896.96259277,566.05496338)
\lineto(896.82759277,566.05496338)
\curveto(896.74758906,566.03495662)(896.66258914,566.01495664)(896.57259277,565.99496338)
\curveto(896.49258931,565.97495668)(896.41258939,565.9499567)(896.33259277,565.91996338)
\curveto(896.01258979,565.77995687)(895.75259005,565.57495708)(895.55259277,565.30496338)
\curveto(895.36259044,565.04495761)(895.2075906,564.73995791)(895.08759277,564.38996338)
\curveto(895.04759076,564.27995837)(895.01759079,564.16495849)(894.99759277,564.04496338)
\curveto(894.98759082,563.93495872)(894.97259083,563.82495883)(894.95259277,563.71496338)
\curveto(894.95259085,563.67495898)(894.94759086,563.63495902)(894.93759277,563.59496338)
\lineto(894.93759277,563.48996338)
\curveto(894.91759089,563.43995921)(894.9075909,563.38495927)(894.90759277,563.32496338)
\curveto(894.91759089,563.26495939)(894.92259088,563.20995944)(894.92259277,563.15996338)
\lineto(894.92259277,562.82996338)
\curveto(894.92259088,562.72995992)(894.93259087,562.63496002)(894.95259277,562.54496338)
\curveto(894.96259084,562.51496014)(894.96759084,562.46496019)(894.96759277,562.39496338)
\curveto(894.98759082,562.32496033)(895.0025908,562.2549604)(895.01259277,562.18496338)
\lineto(895.07259277,561.97496338)
\curveto(895.18259062,561.62496103)(895.33259047,561.32496133)(895.52259277,561.07496338)
\curveto(895.71259009,560.82496183)(895.95258985,560.61996203)(896.24259277,560.45996338)
\curveto(896.33258947,560.40996224)(896.42258938,560.36996228)(896.51259277,560.33996338)
\curveto(896.6025892,560.30996234)(896.7025891,560.27996237)(896.81259277,560.24996338)
\curveto(896.86258894,560.22996242)(896.91258889,560.22496243)(896.96259277,560.23496338)
\curveto(897.02258878,560.24496241)(897.07758873,560.23996241)(897.12759277,560.21996338)
\curveto(897.16758864,560.20996244)(897.2075886,560.20496245)(897.24759277,560.20496338)
\lineto(897.38259277,560.20496338)
\lineto(897.51759277,560.20496338)
\curveto(897.54758826,560.21496244)(897.59758821,560.21996243)(897.66759277,560.21996338)
\curveto(897.74758806,560.23996241)(897.82758798,560.2549624)(897.90759277,560.26496338)
\curveto(897.98758782,560.28496237)(898.06258774,560.30996234)(898.13259277,560.33996338)
\curveto(898.46258734,560.47996217)(898.72758708,560.654962)(898.92759277,560.86496338)
\curveto(899.13758667,561.08496157)(899.31258649,561.35996129)(899.45259277,561.68996338)
\curveto(899.5025863,561.79996085)(899.53758627,561.90996074)(899.55759277,562.01996338)
\curveto(899.57758623,562.12996052)(899.6025862,562.23996041)(899.63259277,562.34996338)
\curveto(899.65258615,562.38996026)(899.66258614,562.42496023)(899.66259277,562.45496338)
\curveto(899.66258614,562.49496016)(899.66758614,562.53496012)(899.67759277,562.57496338)
\curveto(899.68758612,562.63496002)(899.68758612,562.69495996)(899.67759277,562.75496338)
\curveto(899.67758613,562.81495984)(899.68258612,562.87495978)(899.69259277,562.93496338)
}
}
{
\newrgbcolor{curcolor}{0 0 0}
\pscustom[linestyle=none,fillstyle=solid,fillcolor=curcolor]
{
\newpath
\moveto(909.97884277,563.48996338)
\curveto(909.99883471,563.42995922)(910.0088347,563.33495932)(910.00884277,563.20496338)
\curveto(910.0088347,563.08495957)(910.00383471,562.99995965)(909.99384277,562.94996338)
\lineto(909.99384277,562.79996338)
\curveto(909.98383473,562.71995993)(909.97383474,562.64496001)(909.96384277,562.57496338)
\curveto(909.96383475,562.51496014)(909.95883475,562.44496021)(909.94884277,562.36496338)
\curveto(909.92883478,562.30496035)(909.9138348,562.24496041)(909.90384277,562.18496338)
\curveto(909.90383481,562.12496053)(909.89383482,562.06496059)(909.87384277,562.00496338)
\curveto(909.83383488,561.87496078)(909.79883491,561.74496091)(909.76884277,561.61496338)
\curveto(909.73883497,561.48496117)(909.69883501,561.36496129)(909.64884277,561.25496338)
\curveto(909.43883527,560.77496188)(909.15883555,560.36996228)(908.80884277,560.03996338)
\curveto(908.45883625,559.71996293)(908.02883668,559.47496318)(907.51884277,559.30496338)
\curveto(907.4088373,559.26496339)(907.28883742,559.23496342)(907.15884277,559.21496338)
\curveto(907.03883767,559.19496346)(906.9138378,559.17496348)(906.78384277,559.15496338)
\curveto(906.72383799,559.14496351)(906.65883805,559.13996351)(906.58884277,559.13996338)
\curveto(906.52883818,559.12996352)(906.46883824,559.12496353)(906.40884277,559.12496338)
\curveto(906.36883834,559.11496354)(906.3088384,559.10996354)(906.22884277,559.10996338)
\curveto(906.15883855,559.10996354)(906.1088386,559.11496354)(906.07884277,559.12496338)
\curveto(906.03883867,559.13496352)(905.99883871,559.13996351)(905.95884277,559.13996338)
\curveto(905.91883879,559.12996352)(905.88383883,559.12996352)(905.85384277,559.13996338)
\lineto(905.76384277,559.13996338)
\lineto(905.40384277,559.18496338)
\curveto(905.26383945,559.22496343)(905.12883958,559.26496339)(904.99884277,559.30496338)
\curveto(904.86883984,559.34496331)(904.74383997,559.38996326)(904.62384277,559.43996338)
\curveto(904.17384054,559.63996301)(903.80384091,559.89996275)(903.51384277,560.21996338)
\curveto(903.22384149,560.53996211)(902.98384173,560.92996172)(902.79384277,561.38996338)
\curveto(902.74384197,561.48996116)(902.70384201,561.58996106)(902.67384277,561.68996338)
\curveto(902.65384206,561.78996086)(902.63384208,561.89496076)(902.61384277,562.00496338)
\curveto(902.59384212,562.04496061)(902.58384213,562.07496058)(902.58384277,562.09496338)
\curveto(902.59384212,562.12496053)(902.59384212,562.15996049)(902.58384277,562.19996338)
\curveto(902.56384215,562.27996037)(902.54884216,562.35996029)(902.53884277,562.43996338)
\curveto(902.53884217,562.52996012)(902.52884218,562.61496004)(902.50884277,562.69496338)
\lineto(902.50884277,562.81496338)
\curveto(902.5088422,562.8549598)(902.50384221,562.89995975)(902.49384277,562.94996338)
\curveto(902.48384223,562.99995965)(902.47884223,563.08495957)(902.47884277,563.20496338)
\curveto(902.47884223,563.33495932)(902.48884222,563.42995922)(902.50884277,563.48996338)
\curveto(902.52884218,563.55995909)(902.53384218,563.62995902)(902.52384277,563.69996338)
\curveto(902.5138422,563.76995888)(902.51884219,563.83995881)(902.53884277,563.90996338)
\curveto(902.54884216,563.95995869)(902.55384216,563.99995865)(902.55384277,564.02996338)
\curveto(902.56384215,564.06995858)(902.57384214,564.11495854)(902.58384277,564.16496338)
\curveto(902.6138421,564.28495837)(902.63884207,564.40495825)(902.65884277,564.52496338)
\curveto(902.68884202,564.64495801)(902.72884198,564.75995789)(902.77884277,564.86996338)
\curveto(902.92884178,565.23995741)(903.1088416,565.56995708)(903.31884277,565.85996338)
\curveto(903.53884117,566.15995649)(903.80384091,566.40995624)(904.11384277,566.60996338)
\curveto(904.23384048,566.68995596)(904.35884035,566.7549559)(904.48884277,566.80496338)
\curveto(904.61884009,566.86495579)(904.75383996,566.92495573)(904.89384277,566.98496338)
\curveto(905.0138397,567.03495562)(905.14383957,567.06495559)(905.28384277,567.07496338)
\curveto(905.42383929,567.09495556)(905.56383915,567.12495553)(905.70384277,567.16496338)
\lineto(905.89884277,567.16496338)
\curveto(905.96883874,567.17495548)(906.03383868,567.18495547)(906.09384277,567.19496338)
\curveto(906.98383773,567.20495545)(907.72383699,567.01995563)(908.31384277,566.63996338)
\curveto(908.90383581,566.25995639)(909.32883538,565.76495689)(909.58884277,565.15496338)
\curveto(909.63883507,565.0549576)(909.67883503,564.9549577)(909.70884277,564.85496338)
\curveto(909.73883497,564.7549579)(909.77383494,564.649958)(909.81384277,564.53996338)
\curveto(909.84383487,564.42995822)(909.86883484,564.30995834)(909.88884277,564.17996338)
\curveto(909.9088348,564.05995859)(909.93383478,563.93495872)(909.96384277,563.80496338)
\curveto(909.97383474,563.7549589)(909.97383474,563.69995895)(909.96384277,563.63996338)
\curveto(909.96383475,563.58995906)(909.96883474,563.53995911)(909.97884277,563.48996338)
\moveto(908.64384277,562.63496338)
\curveto(908.66383605,562.70495995)(908.66883604,562.78495987)(908.65884277,562.87496338)
\lineto(908.65884277,563.12996338)
\curveto(908.65883605,563.51995913)(908.62383609,563.8499588)(908.55384277,564.11996338)
\curveto(908.52383619,564.19995845)(908.49883621,564.27995837)(908.47884277,564.35996338)
\curveto(908.45883625,564.43995821)(908.43383628,564.51495814)(908.40384277,564.58496338)
\curveto(908.12383659,565.23495742)(907.67883703,565.68495697)(907.06884277,565.93496338)
\curveto(906.99883771,565.96495669)(906.92383779,565.98495667)(906.84384277,565.99496338)
\lineto(906.60384277,566.05496338)
\curveto(906.52383819,566.07495658)(906.43883827,566.08495657)(906.34884277,566.08496338)
\lineto(906.07884277,566.08496338)
\lineto(905.80884277,566.03996338)
\curveto(905.708839,566.01995663)(905.6138391,565.99495666)(905.52384277,565.96496338)
\curveto(905.44383927,565.94495671)(905.36383935,565.91495674)(905.28384277,565.87496338)
\curveto(905.2138395,565.8549568)(905.14883956,565.82495683)(905.08884277,565.78496338)
\curveto(905.02883968,565.74495691)(904.97383974,565.70495695)(904.92384277,565.66496338)
\curveto(904.68384003,565.49495716)(904.48884022,565.28995736)(904.33884277,565.04996338)
\curveto(904.18884052,564.80995784)(904.05884065,564.52995812)(903.94884277,564.20996338)
\curveto(903.91884079,564.10995854)(903.89884081,564.00495865)(903.88884277,563.89496338)
\curveto(903.87884083,563.79495886)(903.86384085,563.68995896)(903.84384277,563.57996338)
\curveto(903.83384088,563.53995911)(903.82884088,563.47495918)(903.82884277,563.38496338)
\curveto(903.81884089,563.3549593)(903.8138409,563.31995933)(903.81384277,563.27996338)
\curveto(903.82384089,563.23995941)(903.82884088,563.19495946)(903.82884277,563.14496338)
\lineto(903.82884277,562.84496338)
\curveto(903.82884088,562.74495991)(903.83884087,562.65496)(903.85884277,562.57496338)
\lineto(903.88884277,562.39496338)
\curveto(903.9088408,562.29496036)(903.92384079,562.19496046)(903.93384277,562.09496338)
\curveto(903.95384076,562.00496065)(903.98384073,561.91996073)(904.02384277,561.83996338)
\curveto(904.12384059,561.59996105)(904.23884047,561.37496128)(904.36884277,561.16496338)
\curveto(904.5088402,560.9549617)(904.67884003,560.77996187)(904.87884277,560.63996338)
\curveto(904.92883978,560.60996204)(904.97383974,560.58496207)(905.01384277,560.56496338)
\curveto(905.05383966,560.54496211)(905.09883961,560.51996213)(905.14884277,560.48996338)
\curveto(905.22883948,560.43996221)(905.3138394,560.39496226)(905.40384277,560.35496338)
\curveto(905.50383921,560.32496233)(905.6088391,560.29496236)(905.71884277,560.26496338)
\curveto(905.76883894,560.24496241)(905.8138389,560.23496242)(905.85384277,560.23496338)
\curveto(905.90383881,560.24496241)(905.95383876,560.24496241)(906.00384277,560.23496338)
\curveto(906.03383868,560.22496243)(906.09383862,560.21496244)(906.18384277,560.20496338)
\curveto(906.28383843,560.19496246)(906.35883835,560.19996245)(906.40884277,560.21996338)
\curveto(906.44883826,560.22996242)(906.48883822,560.22996242)(906.52884277,560.21996338)
\curveto(906.56883814,560.21996243)(906.6088381,560.22996242)(906.64884277,560.24996338)
\curveto(906.72883798,560.26996238)(906.8088379,560.28496237)(906.88884277,560.29496338)
\curveto(906.96883774,560.31496234)(907.04383767,560.33996231)(907.11384277,560.36996338)
\curveto(907.45383726,560.50996214)(907.72883698,560.70496195)(907.93884277,560.95496338)
\curveto(908.14883656,561.20496145)(908.32383639,561.49996115)(908.46384277,561.83996338)
\curveto(908.5138362,561.95996069)(908.54383617,562.08496057)(908.55384277,562.21496338)
\curveto(908.57383614,562.3549603)(908.60383611,562.49496016)(908.64384277,562.63496338)
}
}
{
\newrgbcolor{curcolor}{0 0 0}
\pscustom[linestyle=none,fillstyle=solid,fillcolor=curcolor]
{
\newpath
\moveto(915.11212402,567.19496338)
\curveto(915.34211923,567.19495546)(915.4721191,567.13495552)(915.50212402,567.01496338)
\curveto(915.53211904,566.90495575)(915.54711903,566.73995591)(915.54712402,566.51996338)
\lineto(915.54712402,566.23496338)
\curveto(915.54711903,566.14495651)(915.52211905,566.06995658)(915.47212402,566.00996338)
\curveto(915.41211916,565.92995672)(915.32711925,565.88495677)(915.21712402,565.87496338)
\curveto(915.10711947,565.87495678)(914.99711958,565.85995679)(914.88712402,565.82996338)
\curveto(914.74711983,565.79995685)(914.61211996,565.76995688)(914.48212402,565.73996338)
\curveto(914.36212021,565.70995694)(914.24712033,565.66995698)(914.13712402,565.61996338)
\curveto(913.84712073,565.48995716)(913.61212096,565.30995734)(913.43212402,565.07996338)
\curveto(913.25212132,564.85995779)(913.09712148,564.60495805)(912.96712402,564.31496338)
\curveto(912.92712165,564.20495845)(912.89712168,564.08995856)(912.87712402,563.96996338)
\curveto(912.85712172,563.85995879)(912.83212174,563.74495891)(912.80212402,563.62496338)
\curveto(912.79212178,563.57495908)(912.78712179,563.52495913)(912.78712402,563.47496338)
\curveto(912.79712178,563.42495923)(912.79712178,563.37495928)(912.78712402,563.32496338)
\curveto(912.75712182,563.20495945)(912.74212183,563.06495959)(912.74212402,562.90496338)
\curveto(912.75212182,562.7549599)(912.75712182,562.60996004)(912.75712402,562.46996338)
\lineto(912.75712402,560.62496338)
\lineto(912.75712402,560.27996338)
\curveto(912.75712182,560.15996249)(912.75212182,560.04496261)(912.74212402,559.93496338)
\curveto(912.73212184,559.82496283)(912.72712185,559.72996292)(912.72712402,559.64996338)
\curveto(912.73712184,559.56996308)(912.71712186,559.49996315)(912.66712402,559.43996338)
\curveto(912.61712196,559.36996328)(912.53712204,559.32996332)(912.42712402,559.31996338)
\curveto(912.32712225,559.30996334)(912.21712236,559.30496335)(912.09712402,559.30496338)
\lineto(911.82712402,559.30496338)
\curveto(911.7771228,559.32496333)(911.72712285,559.33996331)(911.67712402,559.34996338)
\curveto(911.63712294,559.36996328)(911.60712297,559.39496326)(911.58712402,559.42496338)
\curveto(911.53712304,559.49496316)(911.50712307,559.57996307)(911.49712402,559.67996338)
\lineto(911.49712402,560.00996338)
\lineto(911.49712402,561.16496338)
\lineto(911.49712402,565.31996338)
\lineto(911.49712402,566.35496338)
\lineto(911.49712402,566.65496338)
\curveto(911.50712307,566.7549559)(911.53712304,566.83995581)(911.58712402,566.90996338)
\curveto(911.61712296,566.9499557)(911.66712291,566.97995567)(911.73712402,566.99996338)
\curveto(911.81712276,567.01995563)(911.90212267,567.02995562)(911.99212402,567.02996338)
\curveto(912.08212249,567.03995561)(912.1721224,567.03995561)(912.26212402,567.02996338)
\curveto(912.35212222,567.01995563)(912.42212215,567.00495565)(912.47212402,566.98496338)
\curveto(912.55212202,566.9549557)(912.60212197,566.89495576)(912.62212402,566.80496338)
\curveto(912.65212192,566.72495593)(912.66712191,566.63495602)(912.66712402,566.53496338)
\lineto(912.66712402,566.23496338)
\curveto(912.66712191,566.13495652)(912.68712189,566.04495661)(912.72712402,565.96496338)
\curveto(912.73712184,565.94495671)(912.74712183,565.92995672)(912.75712402,565.91996338)
\lineto(912.80212402,565.87496338)
\curveto(912.91212166,565.87495678)(913.00212157,565.91995673)(913.07212402,566.00996338)
\curveto(913.14212143,566.10995654)(913.20212137,566.18995646)(913.25212402,566.24996338)
\lineto(913.34212402,566.33996338)
\curveto(913.43212114,566.4499562)(913.55712102,566.56495609)(913.71712402,566.68496338)
\curveto(913.8771207,566.80495585)(914.02712055,566.89495576)(914.16712402,566.95496338)
\curveto(914.25712032,567.00495565)(914.35212022,567.03995561)(914.45212402,567.05996338)
\curveto(914.55212002,567.08995556)(914.65711992,567.11995553)(914.76712402,567.14996338)
\curveto(914.82711975,567.15995549)(914.88711969,567.16495549)(914.94712402,567.16496338)
\curveto(915.00711957,567.17495548)(915.06211951,567.18495547)(915.11212402,567.19496338)
}
}
{
\newrgbcolor{curcolor}{0.40000001 0.40000001 0.40000001}
\pscustom[linestyle=none,fillstyle=solid,fillcolor=curcolor]
{
\newpath
\moveto(807.09008789,570)
\lineto(822.09008789,570)
\lineto(822.09008789,555)
\lineto(807.09008789,555)
\closepath
}
}
{
\newrgbcolor{curcolor}{0 0 0}
\pscustom[linestyle=none,fillstyle=solid,fillcolor=curcolor]
{
\newpath
\moveto(835.0282959,536.85548584)
\curveto(835.04828635,536.80548509)(835.07328633,536.74548515)(835.1032959,536.67548584)
\curveto(835.13328627,536.60548529)(835.15328625,536.53048537)(835.1632959,536.45048584)
\curveto(835.18328622,536.38048552)(835.18328622,536.31048559)(835.1632959,536.24048584)
\curveto(835.15328625,536.18048572)(835.11328629,536.13548576)(835.0432959,536.10548584)
\curveto(834.99328641,536.08548581)(834.93328647,536.07548582)(834.8632959,536.07548584)
\lineto(834.6532959,536.07548584)
\lineto(834.2032959,536.07548584)
\curveto(834.05328735,536.07548582)(833.93328747,536.1004858)(833.8432959,536.15048584)
\curveto(833.74328766,536.21048569)(833.66828773,536.31548558)(833.6182959,536.46548584)
\curveto(833.57828782,536.61548528)(833.53328787,536.75048515)(833.4832959,536.87048584)
\curveto(833.37328803,537.13048477)(833.27328813,537.4004845)(833.1832959,537.68048584)
\curveto(833.09328831,537.96048394)(832.99328841,538.23548366)(832.8832959,538.50548584)
\curveto(832.85328855,538.5954833)(832.82328858,538.68048322)(832.7932959,538.76048584)
\curveto(832.77328863,538.84048306)(832.74328866,538.91548298)(832.7032959,538.98548584)
\curveto(832.67328873,539.05548284)(832.62828877,539.11548278)(832.5682959,539.16548584)
\curveto(832.50828889,539.21548268)(832.42828897,539.25548264)(832.3282959,539.28548584)
\curveto(832.27828912,539.30548259)(832.21828918,539.31048259)(832.1482959,539.30048584)
\lineto(831.9532959,539.30048584)
\lineto(829.1182959,539.30048584)
\lineto(828.8182959,539.30048584)
\curveto(828.70829269,539.31048259)(828.6032928,539.31048259)(828.5032959,539.30048584)
\curveto(828.403293,539.29048261)(828.30829309,539.27548262)(828.2182959,539.25548584)
\curveto(828.13829326,539.23548266)(828.07829332,539.1954827)(828.0382959,539.13548584)
\curveto(827.95829344,539.03548286)(827.8982935,538.92048298)(827.8582959,538.79048584)
\curveto(827.82829357,538.67048323)(827.78829361,538.54548335)(827.7382959,538.41548584)
\curveto(827.63829376,538.18548371)(827.54329386,537.94548395)(827.4532959,537.69548584)
\curveto(827.37329403,537.44548445)(827.28329412,537.20548469)(827.1832959,536.97548584)
\curveto(827.16329424,536.91548498)(827.13829426,536.84548505)(827.1082959,536.76548584)
\curveto(827.08829431,536.6954852)(827.06329434,536.62048528)(827.0332959,536.54048584)
\curveto(827.0032944,536.46048544)(826.96829443,536.38548551)(826.9282959,536.31548584)
\curveto(826.8982945,536.25548564)(826.86329454,536.21048569)(826.8232959,536.18048584)
\curveto(826.74329466,536.12048578)(826.63329477,536.08548581)(826.4932959,536.07548584)
\lineto(826.0732959,536.07548584)
\lineto(825.8332959,536.07548584)
\curveto(825.76329564,536.08548581)(825.7032957,536.11048579)(825.6532959,536.15048584)
\curveto(825.6032958,536.18048572)(825.57329583,536.22548567)(825.5632959,536.28548584)
\curveto(825.56329584,536.34548555)(825.56829583,536.40548549)(825.5782959,536.46548584)
\curveto(825.5982958,536.53548536)(825.61829578,536.6004853)(825.6382959,536.66048584)
\curveto(825.66829573,536.73048517)(825.69329571,536.78048512)(825.7132959,536.81048584)
\curveto(825.85329555,537.13048477)(825.97829542,537.44548445)(826.0882959,537.75548584)
\curveto(826.1982952,538.07548382)(826.31829508,538.3954835)(826.4482959,538.71548584)
\curveto(826.53829486,538.93548296)(826.62329478,539.15048275)(826.7032959,539.36048584)
\curveto(826.78329462,539.58048232)(826.86829453,539.8004821)(826.9582959,540.02048584)
\curveto(827.25829414,540.74048116)(827.54329386,541.46548043)(827.8132959,542.19548584)
\curveto(828.08329332,542.93547896)(828.36829303,543.67047823)(828.6682959,544.40048584)
\curveto(828.77829262,544.66047724)(828.87829252,544.92547697)(828.9682959,545.19548584)
\curveto(829.06829233,545.46547643)(829.17329223,545.73047617)(829.2832959,545.99048584)
\curveto(829.33329207,546.1004758)(829.37829202,546.22047568)(829.4182959,546.35048584)
\curveto(829.46829193,546.49047541)(829.53829186,546.59047531)(829.6282959,546.65048584)
\curveto(829.66829173,546.69047521)(829.73329167,546.72047518)(829.8232959,546.74048584)
\curveto(829.84329156,546.75047515)(829.86329154,546.75047515)(829.8832959,546.74048584)
\curveto(829.91329149,546.74047516)(829.93829146,546.74547515)(829.9582959,546.75548584)
\curveto(830.13829126,546.75547514)(830.34829105,546.75547514)(830.5882959,546.75548584)
\curveto(830.82829057,546.76547513)(831.0032904,546.73047517)(831.1132959,546.65048584)
\curveto(831.19329021,546.59047531)(831.25329015,546.49047541)(831.2932959,546.35048584)
\curveto(831.34329006,546.22047568)(831.39329001,546.1004758)(831.4432959,545.99048584)
\curveto(831.54328986,545.76047614)(831.63328977,545.53047637)(831.7132959,545.30048584)
\curveto(831.79328961,545.07047683)(831.88328952,544.84047706)(831.9832959,544.61048584)
\curveto(832.06328934,544.41047749)(832.13828926,544.20547769)(832.2082959,543.99548584)
\curveto(832.28828911,543.78547811)(832.37328903,543.58047832)(832.4632959,543.38048584)
\curveto(832.76328864,542.65047925)(833.04828835,541.91047999)(833.3182959,541.16048584)
\curveto(833.5982878,540.42048148)(833.89328751,539.68548221)(834.2032959,538.95548584)
\curveto(834.24328716,538.86548303)(834.27328713,538.78048312)(834.2932959,538.70048584)
\curveto(834.32328708,538.62048328)(834.35328705,538.53548336)(834.3832959,538.44548584)
\curveto(834.49328691,538.18548371)(834.5982868,537.92048398)(834.6982959,537.65048584)
\curveto(834.80828659,537.38048452)(834.91828648,537.11548478)(835.0282959,536.85548584)
\moveto(831.8182959,540.50048584)
\curveto(831.90828949,540.53048137)(831.96328944,540.58048132)(831.9832959,540.65048584)
\curveto(832.01328939,540.72048118)(832.01828938,540.7954811)(831.9982959,540.87548584)
\curveto(831.98828941,540.96548093)(831.96328944,541.05048085)(831.9232959,541.13048584)
\curveto(831.89328951,541.22048068)(831.86328954,541.2954806)(831.8332959,541.35548584)
\curveto(831.81328959,541.3954805)(831.8032896,541.43048047)(831.8032959,541.46048584)
\curveto(831.8032896,541.49048041)(831.79328961,541.52548037)(831.7732959,541.56548584)
\lineto(831.6832959,541.80548584)
\curveto(831.66328974,541.89548)(831.63328977,541.98547991)(831.5932959,542.07548584)
\curveto(831.44328996,542.43547946)(831.30829009,542.8004791)(831.1882959,543.17048584)
\curveto(831.07829032,543.55047835)(830.94829045,543.92047798)(830.7982959,544.28048584)
\curveto(830.74829065,544.39047751)(830.7032907,544.5004774)(830.6632959,544.61048584)
\curveto(830.63329077,544.72047718)(830.59329081,544.82547707)(830.5432959,544.92548584)
\curveto(830.52329088,544.97547692)(830.4982909,545.02047688)(830.4682959,545.06048584)
\curveto(830.44829095,545.11047679)(830.398291,545.13547676)(830.3182959,545.13548584)
\curveto(830.2982911,545.11547678)(830.27829112,545.1004768)(830.2582959,545.09048584)
\curveto(830.23829116,545.08047682)(830.21829118,545.06547683)(830.1982959,545.04548584)
\curveto(830.15829124,544.9954769)(830.12829127,544.94047696)(830.1082959,544.88048584)
\curveto(830.08829131,544.83047707)(830.06829133,544.77547712)(830.0482959,544.71548584)
\curveto(829.9982914,544.60547729)(829.95829144,544.4954774)(829.9282959,544.38548584)
\curveto(829.8982915,544.27547762)(829.85829154,544.16547773)(829.8082959,544.05548584)
\curveto(829.63829176,543.66547823)(829.48829191,543.27047863)(829.3582959,542.87048584)
\curveto(829.23829216,542.47047943)(829.0982923,542.08047982)(828.9382959,541.70048584)
\lineto(828.8782959,541.55048584)
\curveto(828.86829253,541.5004804)(828.85329255,541.45048045)(828.8332959,541.40048584)
\lineto(828.7432959,541.16048584)
\curveto(828.71329269,541.08048082)(828.68829271,541.0004809)(828.6682959,540.92048584)
\curveto(828.64829275,540.87048103)(828.63829276,540.81548108)(828.6382959,540.75548584)
\curveto(828.64829275,540.6954812)(828.66329274,540.64548125)(828.6832959,540.60548584)
\curveto(828.73329267,540.52548137)(828.83829256,540.48048142)(828.9982959,540.47048584)
\lineto(829.4482959,540.47048584)
\lineto(831.0532959,540.47048584)
\curveto(831.16329024,540.47048143)(831.2982901,540.46548143)(831.4582959,540.45548584)
\curveto(831.61828978,540.45548144)(831.73828966,540.47048143)(831.8182959,540.50048584)
}
}
{
\newrgbcolor{curcolor}{0 0 0}
\pscustom[linestyle=none,fillstyle=solid,fillcolor=curcolor]
{
\newpath
\moveto(843.1048584,536.88548584)
\lineto(843.1048584,536.49548584)
\curveto(843.10485052,536.37548552)(843.07985055,536.27548562)(843.0298584,536.19548584)
\curveto(842.97985065,536.12548577)(842.89485073,536.08548581)(842.7748584,536.07548584)
\lineto(842.4298584,536.07548584)
\curveto(842.36985126,536.07548582)(842.30985132,536.07048583)(842.2498584,536.06048584)
\curveto(842.19985143,536.06048584)(842.15485147,536.07048583)(842.1148584,536.09048584)
\curveto(842.0248516,536.11048579)(841.96485166,536.15048575)(841.9348584,536.21048584)
\curveto(841.89485173,536.26048564)(841.86985176,536.32048558)(841.8598584,536.39048584)
\curveto(841.85985177,536.46048544)(841.84485178,536.53048537)(841.8148584,536.60048584)
\curveto(841.80485182,536.62048528)(841.78985184,536.63548526)(841.7698584,536.64548584)
\curveto(841.75985187,536.66548523)(841.74485188,536.68548521)(841.7248584,536.70548584)
\curveto(841.624852,536.71548518)(841.54485208,536.6954852)(841.4848584,536.64548584)
\curveto(841.43485219,536.5954853)(841.37985225,536.54548535)(841.3198584,536.49548584)
\curveto(841.11985251,536.34548555)(840.91985271,536.23048567)(840.7198584,536.15048584)
\curveto(840.53985309,536.07048583)(840.3298533,536.01048589)(840.0898584,535.97048584)
\curveto(839.85985377,535.93048597)(839.61985401,535.91048599)(839.3698584,535.91048584)
\curveto(839.1298545,535.900486)(838.88985474,535.91548598)(838.6498584,535.95548584)
\curveto(838.40985522,535.98548591)(838.19985543,536.04048586)(838.0198584,536.12048584)
\curveto(837.49985613,536.34048556)(837.07985655,536.63548526)(836.7598584,537.00548584)
\curveto(836.43985719,537.38548451)(836.18985744,537.85548404)(836.0098584,538.41548584)
\curveto(835.96985766,538.50548339)(835.93985769,538.5954833)(835.9198584,538.68548584)
\curveto(835.90985772,538.78548311)(835.88985774,538.88548301)(835.8598584,538.98548584)
\curveto(835.84985778,539.03548286)(835.84485778,539.08548281)(835.8448584,539.13548584)
\curveto(835.84485778,539.18548271)(835.83985779,539.23548266)(835.8298584,539.28548584)
\curveto(835.80985782,539.33548256)(835.79985783,539.38548251)(835.7998584,539.43548584)
\curveto(835.80985782,539.4954824)(835.80985782,539.55048235)(835.7998584,539.60048584)
\lineto(835.7998584,539.75048584)
\curveto(835.77985785,539.8004821)(835.76985786,539.86548203)(835.7698584,539.94548584)
\curveto(835.76985786,540.02548187)(835.77985785,540.09048181)(835.7998584,540.14048584)
\lineto(835.7998584,540.30548584)
\curveto(835.81985781,540.37548152)(835.8248578,540.44548145)(835.8148584,540.51548584)
\curveto(835.81485781,540.5954813)(835.8248578,540.67048123)(835.8448584,540.74048584)
\curveto(835.85485777,540.79048111)(835.85985777,540.83548106)(835.8598584,540.87548584)
\curveto(835.85985777,540.91548098)(835.86485776,540.96048094)(835.8748584,541.01048584)
\curveto(835.90485772,541.11048079)(835.9298577,541.20548069)(835.9498584,541.29548584)
\curveto(835.96985766,541.3954805)(835.99485763,541.49048041)(836.0248584,541.58048584)
\curveto(836.15485747,541.96047994)(836.31985731,542.3004796)(836.5198584,542.60048584)
\curveto(836.7298569,542.91047899)(836.97985665,543.16547873)(837.2698584,543.36548584)
\curveto(837.43985619,543.48547841)(837.61485601,543.58547831)(837.7948584,543.66548584)
\curveto(837.98485564,543.74547815)(838.18985544,543.81547808)(838.4098584,543.87548584)
\curveto(838.47985515,543.88547801)(838.54485508,543.895478)(838.6048584,543.90548584)
\curveto(838.67485495,543.91547798)(838.74485488,543.93047797)(838.8148584,543.95048584)
\lineto(838.9648584,543.95048584)
\curveto(839.04485458,543.97047793)(839.15985447,543.98047792)(839.3098584,543.98048584)
\curveto(839.46985416,543.98047792)(839.58985404,543.97047793)(839.6698584,543.95048584)
\curveto(839.70985392,543.94047796)(839.76485386,543.93547796)(839.8348584,543.93548584)
\curveto(839.94485368,543.90547799)(840.05485357,543.88047802)(840.1648584,543.86048584)
\curveto(840.27485335,543.85047805)(840.37985325,543.82047808)(840.4798584,543.77048584)
\curveto(840.629853,543.71047819)(840.76985286,543.64547825)(840.8998584,543.57548584)
\curveto(841.03985259,543.50547839)(841.16985246,543.42547847)(841.2898584,543.33548584)
\curveto(841.34985228,543.28547861)(841.40985222,543.23047867)(841.4698584,543.17048584)
\curveto(841.53985209,543.12047878)(841.629852,543.10547879)(841.7398584,543.12548584)
\curveto(841.75985187,543.15547874)(841.77485185,543.18047872)(841.7848584,543.20048584)
\curveto(841.80485182,543.22047868)(841.81985181,543.25047865)(841.8298584,543.29048584)
\curveto(841.85985177,543.38047852)(841.86985176,543.4954784)(841.8598584,543.63548584)
\lineto(841.8598584,544.01048584)
\lineto(841.8598584,545.73548584)
\lineto(841.8598584,546.20048584)
\curveto(841.85985177,546.38047552)(841.88485174,546.51047539)(841.9348584,546.59048584)
\curveto(841.97485165,546.66047524)(842.03485159,546.70547519)(842.1148584,546.72548584)
\curveto(842.13485149,546.72547517)(842.15985147,546.72547517)(842.1898584,546.72548584)
\curveto(842.21985141,546.73547516)(842.24485138,546.74047516)(842.2648584,546.74048584)
\curveto(842.40485122,546.75047515)(842.54985108,546.75047515)(842.6998584,546.74048584)
\curveto(842.85985077,546.74047516)(842.96985066,546.7004752)(843.0298584,546.62048584)
\curveto(843.07985055,546.54047536)(843.10485052,546.44047546)(843.1048584,546.32048584)
\lineto(843.1048584,545.94548584)
\lineto(843.1048584,536.88548584)
\moveto(841.8898584,539.72048584)
\curveto(841.90985172,539.77048213)(841.91985171,539.83548206)(841.9198584,539.91548584)
\curveto(841.91985171,540.00548189)(841.90985172,540.07548182)(841.8898584,540.12548584)
\lineto(841.8898584,540.35048584)
\curveto(841.86985176,540.44048146)(841.85485177,540.53048137)(841.8448584,540.62048584)
\curveto(841.83485179,540.72048118)(841.81485181,540.81048109)(841.7848584,540.89048584)
\curveto(841.76485186,540.97048093)(841.74485188,541.04548085)(841.7248584,541.11548584)
\curveto(841.71485191,541.18548071)(841.69485193,541.25548064)(841.6648584,541.32548584)
\curveto(841.54485208,541.62548027)(841.38985224,541.89048001)(841.1998584,542.12048584)
\curveto(841.00985262,542.35047955)(840.76985286,542.53047937)(840.4798584,542.66048584)
\curveto(840.37985325,542.71047919)(840.27485335,542.74547915)(840.1648584,542.76548584)
\curveto(840.06485356,542.7954791)(839.95485367,542.82047908)(839.8348584,542.84048584)
\curveto(839.75485387,542.86047904)(839.66485396,542.87047903)(839.5648584,542.87048584)
\lineto(839.2948584,542.87048584)
\curveto(839.24485438,542.86047904)(839.19985443,542.85047905)(839.1598584,542.84048584)
\lineto(839.0248584,542.84048584)
\curveto(838.94485468,542.82047908)(838.85985477,542.8004791)(838.7698584,542.78048584)
\curveto(838.68985494,542.76047914)(838.60985502,542.73547916)(838.5298584,542.70548584)
\curveto(838.20985542,542.56547933)(837.94985568,542.36047954)(837.7498584,542.09048584)
\curveto(837.55985607,541.83048007)(837.40485622,541.52548037)(837.2848584,541.17548584)
\curveto(837.24485638,541.06548083)(837.21485641,540.95048095)(837.1948584,540.83048584)
\curveto(837.18485644,540.72048118)(837.16985646,540.61048129)(837.1498584,540.50048584)
\curveto(837.14985648,540.46048144)(837.14485648,540.42048148)(837.1348584,540.38048584)
\lineto(837.1348584,540.27548584)
\curveto(837.11485651,540.22548167)(837.10485652,540.17048173)(837.1048584,540.11048584)
\curveto(837.11485651,540.05048185)(837.11985651,539.9954819)(837.1198584,539.94548584)
\lineto(837.1198584,539.61548584)
\curveto(837.11985651,539.51548238)(837.1298565,539.42048248)(837.1498584,539.33048584)
\curveto(837.15985647,539.3004826)(837.16485646,539.25048265)(837.1648584,539.18048584)
\curveto(837.18485644,539.11048279)(837.19985643,539.04048286)(837.2098584,538.97048584)
\lineto(837.2698584,538.76048584)
\curveto(837.37985625,538.41048349)(837.5298561,538.11048379)(837.7198584,537.86048584)
\curveto(837.90985572,537.61048429)(838.14985548,537.40548449)(838.4398584,537.24548584)
\curveto(838.5298551,537.1954847)(838.61985501,537.15548474)(838.7098584,537.12548584)
\curveto(838.79985483,537.0954848)(838.89985473,537.06548483)(839.0098584,537.03548584)
\curveto(839.05985457,537.01548488)(839.10985452,537.01048489)(839.1598584,537.02048584)
\curveto(839.21985441,537.03048487)(839.27485435,537.02548487)(839.3248584,537.00548584)
\curveto(839.36485426,536.9954849)(839.40485422,536.99048491)(839.4448584,536.99048584)
\lineto(839.5798584,536.99048584)
\lineto(839.7148584,536.99048584)
\curveto(839.74485388,537.0004849)(839.79485383,537.00548489)(839.8648584,537.00548584)
\curveto(839.94485368,537.02548487)(840.0248536,537.04048486)(840.1048584,537.05048584)
\curveto(840.18485344,537.07048483)(840.25985337,537.0954848)(840.3298584,537.12548584)
\curveto(840.65985297,537.26548463)(840.9248527,537.44048446)(841.1248584,537.65048584)
\curveto(841.33485229,537.87048403)(841.50985212,538.14548375)(841.6498584,538.47548584)
\curveto(841.69985193,538.58548331)(841.73485189,538.6954832)(841.7548584,538.80548584)
\curveto(841.77485185,538.91548298)(841.79985183,539.02548287)(841.8298584,539.13548584)
\curveto(841.84985178,539.17548272)(841.85985177,539.21048269)(841.8598584,539.24048584)
\curveto(841.85985177,539.28048262)(841.86485176,539.32048258)(841.8748584,539.36048584)
\curveto(841.88485174,539.42048248)(841.88485174,539.48048242)(841.8748584,539.54048584)
\curveto(841.87485175,539.6004823)(841.87985175,539.66048224)(841.8898584,539.72048584)
}
}
{
\newrgbcolor{curcolor}{0 0 0}
\pscustom[linestyle=none,fillstyle=solid,fillcolor=curcolor]
{
\newpath
\moveto(848.7411084,543.98048584)
\curveto(849.12110341,543.99047791)(849.44110309,543.95047795)(849.7011084,543.86048584)
\curveto(849.97110256,543.77047813)(850.21610232,543.64047826)(850.4361084,543.47048584)
\curveto(850.51610202,543.42047848)(850.58110195,543.35047855)(850.6311084,543.26048584)
\curveto(850.69110184,543.18047872)(850.75610178,543.10547879)(850.8261084,543.03548584)
\curveto(850.84610169,543.01547888)(850.87610166,542.99047891)(850.9161084,542.96048584)
\curveto(850.95610158,542.93047897)(851.00610153,542.92047898)(851.0661084,542.93048584)
\curveto(851.16610137,542.96047894)(851.25110128,543.02047888)(851.3211084,543.11048584)
\curveto(851.40110113,543.21047869)(851.48110105,543.28547861)(851.5611084,543.33548584)
\curveto(851.70110083,543.44547845)(851.84610069,543.54047836)(851.9961084,543.62048584)
\curveto(852.14610039,543.71047819)(852.31110022,543.78547811)(852.4911084,543.84548584)
\curveto(852.57109996,543.87547802)(852.65609988,543.895478)(852.7461084,543.90548584)
\curveto(852.84609969,543.92547797)(852.94109959,543.94547795)(853.0311084,543.96548584)
\curveto(853.08109945,543.97547792)(853.12609941,543.98047792)(853.1661084,543.98048584)
\lineto(853.3161084,543.98048584)
\curveto(853.36609917,544.0004779)(853.4360991,544.00547789)(853.5261084,543.99548584)
\curveto(853.61609892,543.9954779)(853.68109885,543.99047791)(853.7211084,543.98048584)
\curveto(853.77109876,543.97047793)(853.84609869,543.96547793)(853.9461084,543.96548584)
\curveto(854.0360985,543.94547795)(854.12109841,543.92547797)(854.2011084,543.90548584)
\curveto(854.29109824,543.895478)(854.37609816,543.87547802)(854.4561084,543.84548584)
\curveto(854.50609803,543.82547807)(854.55109798,543.81047809)(854.5911084,543.80048584)
\curveto(854.64109789,543.8004781)(854.69109784,543.79047811)(854.7411084,543.77048584)
\curveto(855.24109729,543.55047835)(855.58609695,543.21047869)(855.7761084,542.75048584)
\curveto(855.81609672,542.67047923)(855.84609669,542.58047932)(855.8661084,542.48048584)
\curveto(855.88609665,542.39047951)(855.90609663,542.29047961)(855.9261084,542.18048584)
\curveto(855.94609659,542.15047975)(855.95109658,542.11547978)(855.9411084,542.07548584)
\curveto(855.94109659,542.04547985)(855.94609659,542.01547988)(855.9561084,541.98548584)
\lineto(855.9561084,541.85048584)
\curveto(855.96609657,541.81048009)(855.96609657,541.76548013)(855.9561084,541.71548584)
\curveto(855.95609658,541.66548023)(855.95609658,541.61548028)(855.9561084,541.56548584)
\lineto(855.9561084,540.98048584)
\lineto(855.9561084,540.02048584)
\lineto(855.9561084,537.17048584)
\curveto(855.95609658,537.01048489)(855.95609658,536.82048508)(855.9561084,536.60048584)
\curveto(855.96609657,536.38048552)(855.92609661,536.23548566)(855.8361084,536.16548584)
\curveto(855.79609674,536.13548576)(855.7310968,536.11048579)(855.6411084,536.09048584)
\curveto(855.55109698,536.08048582)(855.45609708,536.07548582)(855.3561084,536.07548584)
\curveto(855.25609728,536.07548582)(855.15609738,536.08048582)(855.0561084,536.09048584)
\curveto(854.96609757,536.1004858)(854.90109763,536.12048578)(854.8611084,536.15048584)
\curveto(854.80109773,536.18048572)(854.76109777,536.24048566)(854.7411084,536.33048584)
\curveto(854.72109781,536.39048551)(854.71609782,536.45048545)(854.7261084,536.51048584)
\curveto(854.7360978,536.58048532)(854.7310978,536.64548525)(854.7111084,536.70548584)
\curveto(854.70109783,536.75548514)(854.69609784,536.81048509)(854.6961084,536.87048584)
\curveto(854.70609783,536.94048496)(854.71109782,537.00548489)(854.7111084,537.06548584)
\lineto(854.7111084,537.74048584)
\lineto(854.7111084,540.60548584)
\curveto(854.71109782,540.93548096)(854.70109783,541.24548065)(854.6811084,541.53548584)
\curveto(854.67109786,541.83548006)(854.60109793,542.08547981)(854.4711084,542.28548584)
\curveto(854.32109821,542.52547937)(854.09109844,542.7004792)(853.7811084,542.81048584)
\curveto(853.72109881,542.83047907)(853.65609888,542.84047906)(853.5861084,542.84048584)
\curveto(853.52609901,542.85047905)(853.46109907,542.86547903)(853.3911084,542.88548584)
\curveto(853.35109918,542.895479)(853.28609925,542.895479)(853.1961084,542.88548584)
\curveto(853.10609943,542.88547901)(853.04609949,542.88047902)(853.0161084,542.87048584)
\curveto(852.96609957,542.86047904)(852.91609962,542.85547904)(852.8661084,542.85548584)
\curveto(852.81609972,542.86547903)(852.76609977,542.86047904)(852.7161084,542.84048584)
\curveto(852.57609996,542.81047909)(852.44110009,542.77047913)(852.3111084,542.72048584)
\curveto(851.79110074,542.5004794)(851.44110109,542.11547978)(851.2611084,541.56548584)
\curveto(851.21110132,541.3954805)(851.18110135,541.2004807)(851.1711084,540.98048584)
\lineto(851.1711084,540.30548584)
\lineto(851.1711084,538.34048584)
\lineto(851.1711084,536.88548584)
\lineto(851.1711084,536.51048584)
\curveto(851.17110136,536.39048551)(851.14610139,536.2954856)(851.0961084,536.22548584)
\curveto(851.04610149,536.14548575)(850.96110157,536.1004858)(850.8411084,536.09048584)
\curveto(850.72110181,536.08048582)(850.59610194,536.07548582)(850.4661084,536.07548584)
\curveto(850.29610224,536.07548582)(850.17110236,536.0954858)(850.0911084,536.13548584)
\curveto(850.00110253,536.18548571)(849.94610259,536.26548563)(849.9261084,536.37548584)
\curveto(849.91610262,536.4954854)(849.91110262,536.62548527)(849.9111084,536.76548584)
\lineto(849.9111084,538.19048584)
\lineto(849.9111084,540.66548584)
\curveto(849.91110262,540.98548091)(849.90110263,541.28048062)(849.8811084,541.55048584)
\curveto(849.86110267,541.83048007)(849.79110274,542.07047983)(849.6711084,542.27048584)
\curveto(849.56110297,542.45047945)(849.4361031,542.58047932)(849.2961084,542.66048584)
\curveto(849.15610338,542.75047915)(848.96610357,542.82047908)(848.7261084,542.87048584)
\curveto(848.68610385,542.88047902)(848.64110389,542.88547901)(848.5911084,542.88548584)
\lineto(848.4561084,542.88548584)
\curveto(848.2361043,542.88547901)(848.04110449,542.86047904)(847.8711084,542.81048584)
\curveto(847.71110482,542.76047914)(847.56610497,542.6954792)(847.4361084,542.61548584)
\curveto(846.92610561,542.30547959)(846.58610595,541.84048006)(846.4161084,541.22048584)
\curveto(846.37610616,541.09048081)(846.35610618,540.94048096)(846.3561084,540.77048584)
\curveto(846.36610617,540.61048129)(846.37110616,540.45048145)(846.3711084,540.29048584)
\lineto(846.3711084,538.59548584)
\lineto(846.3711084,536.94548584)
\lineto(846.3711084,536.54048584)
\curveto(846.37110616,536.4004855)(846.34110619,536.29048561)(846.2811084,536.21048584)
\curveto(846.2311063,536.14048576)(846.15610638,536.1004858)(846.0561084,536.09048584)
\curveto(845.95610658,536.08048582)(845.85110668,536.07548582)(845.7411084,536.07548584)
\lineto(845.5161084,536.07548584)
\curveto(845.45610708,536.0954858)(845.39610714,536.11048579)(845.3361084,536.12048584)
\curveto(845.28610725,536.13048577)(845.24110729,536.16048574)(845.2011084,536.21048584)
\curveto(845.15110738,536.27048563)(845.12610741,536.34548555)(845.1261084,536.43548584)
\lineto(845.1261084,536.75048584)
\lineto(845.1261084,537.72548584)
\lineto(845.1261084,542.01548584)
\lineto(845.1261084,543.12548584)
\lineto(845.1261084,543.41048584)
\curveto(845.12610741,543.51047839)(845.14610739,543.59047831)(845.1861084,543.65048584)
\curveto(845.21610732,543.71047819)(845.26110727,543.75047815)(845.3211084,543.77048584)
\curveto(845.40110713,543.8004781)(845.52610701,543.81547808)(845.6961084,543.81548584)
\curveto(845.87610666,543.81547808)(846.00610653,543.8004781)(846.0861084,543.77048584)
\curveto(846.16610637,543.73047817)(846.22110631,543.68047822)(846.2511084,543.62048584)
\curveto(846.27110626,543.57047833)(846.28110625,543.51047839)(846.2811084,543.44048584)
\curveto(846.29110624,543.37047853)(846.30110623,543.30547859)(846.3111084,543.24548584)
\curveto(846.32110621,543.18547871)(846.34110619,543.13547876)(846.3711084,543.09548584)
\curveto(846.40110613,543.05547884)(846.45110608,543.03547886)(846.5211084,543.03548584)
\curveto(846.54110599,543.05547884)(846.56110597,543.06547883)(846.5811084,543.06548584)
\curveto(846.61110592,543.06547883)(846.6361059,543.07547882)(846.6561084,543.09548584)
\curveto(846.71610582,543.14547875)(846.77110576,543.1954787)(846.8211084,543.24548584)
\lineto(847.0011084,543.39548584)
\curveto(847.22110531,543.55547834)(847.47110506,543.6954782)(847.7511084,543.81548584)
\curveto(847.85110468,543.85547804)(847.95110458,543.88047802)(848.0511084,543.89048584)
\curveto(848.15110438,543.91047799)(848.25610428,543.93547796)(848.3661084,543.96548584)
\lineto(848.5461084,543.96548584)
\curveto(848.61610392,543.97547792)(848.68110385,543.98047792)(848.7411084,543.98048584)
}
}
{
\newrgbcolor{curcolor}{0 0 0}
\pscustom[linestyle=none,fillstyle=solid,fillcolor=curcolor]
{
\newpath
\moveto(858.13884277,545.30048584)
\curveto(858.05884165,545.36047654)(858.0138417,545.46547643)(858.00384277,545.61548584)
\lineto(858.00384277,546.08048584)
\lineto(858.00384277,546.33548584)
\curveto(858.00384171,546.42547547)(858.01884169,546.5004754)(858.04884277,546.56048584)
\curveto(858.08884162,546.64047526)(858.16884154,546.7004752)(858.28884277,546.74048584)
\curveto(858.3088414,546.75047515)(858.32884138,546.75047515)(858.34884277,546.74048584)
\curveto(858.37884133,546.74047516)(858.40384131,546.74547515)(858.42384277,546.75548584)
\curveto(858.59384112,546.75547514)(858.75384096,546.75047515)(858.90384277,546.74048584)
\curveto(859.05384066,546.73047517)(859.15384056,546.67047523)(859.20384277,546.56048584)
\curveto(859.23384048,546.5004754)(859.24884046,546.42547547)(859.24884277,546.33548584)
\lineto(859.24884277,546.08048584)
\curveto(859.24884046,545.900476)(859.24384047,545.73047617)(859.23384277,545.57048584)
\curveto(859.23384048,545.41047649)(859.16884054,545.30547659)(859.03884277,545.25548584)
\curveto(858.98884072,545.23547666)(858.93384078,545.22547667)(858.87384277,545.22548584)
\lineto(858.70884277,545.22548584)
\lineto(858.39384277,545.22548584)
\curveto(858.29384142,545.22547667)(858.2088415,545.25047665)(858.13884277,545.30048584)
\moveto(859.24884277,536.79548584)
\lineto(859.24884277,536.48048584)
\curveto(859.25884045,536.38048552)(859.23884047,536.3004856)(859.18884277,536.24048584)
\curveto(859.15884055,536.18048572)(859.1138406,536.14048576)(859.05384277,536.12048584)
\curveto(858.99384072,536.11048579)(858.92384079,536.0954858)(858.84384277,536.07548584)
\lineto(858.61884277,536.07548584)
\curveto(858.48884122,536.07548582)(858.37384134,536.08048582)(858.27384277,536.09048584)
\curveto(858.18384153,536.11048579)(858.1138416,536.16048574)(858.06384277,536.24048584)
\curveto(858.02384169,536.3004856)(858.00384171,536.37548552)(858.00384277,536.46548584)
\lineto(858.00384277,536.75048584)
\lineto(858.00384277,543.09548584)
\lineto(858.00384277,543.41048584)
\curveto(858.00384171,543.52047838)(858.02884168,543.60547829)(858.07884277,543.66548584)
\curveto(858.1088416,543.71547818)(858.14884156,543.74547815)(858.19884277,543.75548584)
\curveto(858.24884146,543.76547813)(858.30384141,543.78047812)(858.36384277,543.80048584)
\curveto(858.38384133,543.8004781)(858.40384131,543.7954781)(858.42384277,543.78548584)
\curveto(858.45384126,543.78547811)(858.47884123,543.79047811)(858.49884277,543.80048584)
\curveto(858.62884108,543.8004781)(858.75884095,543.7954781)(858.88884277,543.78548584)
\curveto(859.02884068,543.78547811)(859.12384059,543.74547815)(859.17384277,543.66548584)
\curveto(859.22384049,543.60547829)(859.24884046,543.52547837)(859.24884277,543.42548584)
\lineto(859.24884277,543.14048584)
\lineto(859.24884277,536.79548584)
}
}
{
\newrgbcolor{curcolor}{0 0 0}
\pscustom[linestyle=none,fillstyle=solid,fillcolor=curcolor]
{
\newpath
\moveto(864.88368652,543.95048584)
\curveto(865.51368129,543.97047793)(866.01868078,543.88547801)(866.39868652,543.69548584)
\curveto(866.77868002,543.50547839)(867.08367972,543.22047868)(867.31368652,542.84048584)
\curveto(867.37367943,542.74047916)(867.41867938,542.63047927)(867.44868652,542.51048584)
\curveto(867.48867931,542.4004795)(867.52367928,542.28547961)(867.55368652,542.16548584)
\curveto(867.6036792,541.97547992)(867.63367917,541.77048013)(867.64368652,541.55048584)
\curveto(867.65367915,541.33048057)(867.65867914,541.10548079)(867.65868652,540.87548584)
\lineto(867.65868652,539.27048584)
\lineto(867.65868652,536.93048584)
\curveto(867.65867914,536.76048514)(867.65367915,536.59048531)(867.64368652,536.42048584)
\curveto(867.64367916,536.25048565)(867.57867922,536.14048576)(867.44868652,536.09048584)
\curveto(867.3986794,536.07048583)(867.34367946,536.06048584)(867.28368652,536.06048584)
\curveto(867.23367957,536.05048585)(867.17867962,536.04548585)(867.11868652,536.04548584)
\curveto(866.98867981,536.04548585)(866.86367994,536.05048585)(866.74368652,536.06048584)
\curveto(866.62368018,536.06048584)(866.53868026,536.1004858)(866.48868652,536.18048584)
\curveto(866.43868036,536.25048565)(866.41368039,536.34048556)(866.41368652,536.45048584)
\lineto(866.41368652,536.78048584)
\lineto(866.41368652,538.07048584)
\lineto(866.41368652,540.51548584)
\curveto(866.41368039,540.78548111)(866.40868039,541.05048085)(866.39868652,541.31048584)
\curveto(866.38868041,541.58048032)(866.34368046,541.81048009)(866.26368652,542.00048584)
\curveto(866.18368062,542.2004797)(866.06368074,542.36047954)(865.90368652,542.48048584)
\curveto(865.74368106,542.61047929)(865.55868124,542.71047919)(865.34868652,542.78048584)
\curveto(865.28868151,542.8004791)(865.22368158,542.81047909)(865.15368652,542.81048584)
\curveto(865.09368171,542.82047908)(865.03368177,542.83547906)(864.97368652,542.85548584)
\curveto(864.92368188,542.86547903)(864.84368196,542.86547903)(864.73368652,542.85548584)
\curveto(864.63368217,542.85547904)(864.56368224,542.85047905)(864.52368652,542.84048584)
\curveto(864.48368232,542.82047908)(864.44868235,542.81047909)(864.41868652,542.81048584)
\curveto(864.38868241,542.82047908)(864.35368245,542.82047908)(864.31368652,542.81048584)
\curveto(864.18368262,542.78047912)(864.05868274,542.74547915)(863.93868652,542.70548584)
\curveto(863.82868297,542.67547922)(863.72368308,542.63047927)(863.62368652,542.57048584)
\curveto(863.58368322,542.55047935)(863.54868325,542.53047937)(863.51868652,542.51048584)
\curveto(863.48868331,542.49047941)(863.45368335,542.47047943)(863.41368652,542.45048584)
\curveto(863.06368374,542.2004797)(862.80868399,541.82548007)(862.64868652,541.32548584)
\curveto(862.61868418,541.24548065)(862.5986842,541.16048074)(862.58868652,541.07048584)
\curveto(862.57868422,540.99048091)(862.56368424,540.91048099)(862.54368652,540.83048584)
\curveto(862.52368428,540.78048112)(862.51868428,540.73048117)(862.52868652,540.68048584)
\curveto(862.53868426,540.64048126)(862.53368427,540.6004813)(862.51368652,540.56048584)
\lineto(862.51368652,540.24548584)
\curveto(862.5036843,540.21548168)(862.4986843,540.18048172)(862.49868652,540.14048584)
\curveto(862.50868429,540.1004818)(862.51368429,540.05548184)(862.51368652,540.00548584)
\lineto(862.51368652,539.55548584)
\lineto(862.51368652,538.11548584)
\lineto(862.51368652,536.79548584)
\lineto(862.51368652,536.45048584)
\curveto(862.51368429,536.34048556)(862.48868431,536.25048565)(862.43868652,536.18048584)
\curveto(862.38868441,536.1004858)(862.2986845,536.06048584)(862.16868652,536.06048584)
\curveto(862.04868475,536.05048585)(861.92368488,536.04548585)(861.79368652,536.04548584)
\curveto(861.71368509,536.04548585)(861.63868516,536.05048585)(861.56868652,536.06048584)
\curveto(861.4986853,536.07048583)(861.43868536,536.0954858)(861.38868652,536.13548584)
\curveto(861.30868549,536.18548571)(861.26868553,536.28048562)(861.26868652,536.42048584)
\lineto(861.26868652,536.82548584)
\lineto(861.26868652,538.59548584)
\lineto(861.26868652,542.22548584)
\lineto(861.26868652,543.14048584)
\lineto(861.26868652,543.41048584)
\curveto(861.26868553,543.5004784)(861.28868551,543.57047833)(861.32868652,543.62048584)
\curveto(861.35868544,543.68047822)(861.40868539,543.72047818)(861.47868652,543.74048584)
\curveto(861.51868528,543.75047815)(861.57368523,543.76047814)(861.64368652,543.77048584)
\curveto(861.72368508,543.78047812)(861.803685,543.78547811)(861.88368652,543.78548584)
\curveto(861.96368484,543.78547811)(862.03868476,543.78047812)(862.10868652,543.77048584)
\curveto(862.18868461,543.76047814)(862.24368456,543.74547815)(862.27368652,543.72548584)
\curveto(862.38368442,543.65547824)(862.43368437,543.56547833)(862.42368652,543.45548584)
\curveto(862.41368439,543.35547854)(862.42868437,543.24047866)(862.46868652,543.11048584)
\curveto(862.48868431,543.05047885)(862.52868427,543.0004789)(862.58868652,542.96048584)
\curveto(862.70868409,542.95047895)(862.803684,542.9954789)(862.87368652,543.09548584)
\curveto(862.95368385,543.1954787)(863.03368377,543.27547862)(863.11368652,543.33548584)
\curveto(863.25368355,543.43547846)(863.39368341,543.52547837)(863.53368652,543.60548584)
\curveto(863.68368312,543.6954782)(863.85368295,543.77047813)(864.04368652,543.83048584)
\curveto(864.12368268,543.86047804)(864.20868259,543.88047802)(864.29868652,543.89048584)
\curveto(864.3986824,543.900478)(864.49368231,543.91547798)(864.58368652,543.93548584)
\curveto(864.63368217,543.94547795)(864.68368212,543.95047795)(864.73368652,543.95048584)
\lineto(864.88368652,543.95048584)
}
}
{
\newrgbcolor{curcolor}{0 0 0}
\pscustom[linestyle=none,fillstyle=solid,fillcolor=curcolor]
{
\newpath
\moveto(869.8282959,545.30048584)
\curveto(869.74829478,545.36047654)(869.70329482,545.46547643)(869.6932959,545.61548584)
\lineto(869.6932959,546.08048584)
\lineto(869.6932959,546.33548584)
\curveto(869.69329483,546.42547547)(869.70829482,546.5004754)(869.7382959,546.56048584)
\curveto(869.77829475,546.64047526)(869.85829467,546.7004752)(869.9782959,546.74048584)
\curveto(869.99829453,546.75047515)(870.01829451,546.75047515)(870.0382959,546.74048584)
\curveto(870.06829446,546.74047516)(870.09329443,546.74547515)(870.1132959,546.75548584)
\curveto(870.28329424,546.75547514)(870.44329408,546.75047515)(870.5932959,546.74048584)
\curveto(870.74329378,546.73047517)(870.84329368,546.67047523)(870.8932959,546.56048584)
\curveto(870.9232936,546.5004754)(870.93829359,546.42547547)(870.9382959,546.33548584)
\lineto(870.9382959,546.08048584)
\curveto(870.93829359,545.900476)(870.93329359,545.73047617)(870.9232959,545.57048584)
\curveto(870.9232936,545.41047649)(870.85829367,545.30547659)(870.7282959,545.25548584)
\curveto(870.67829385,545.23547666)(870.6232939,545.22547667)(870.5632959,545.22548584)
\lineto(870.3982959,545.22548584)
\lineto(870.0832959,545.22548584)
\curveto(869.98329454,545.22547667)(869.89829463,545.25047665)(869.8282959,545.30048584)
\moveto(870.9382959,536.79548584)
\lineto(870.9382959,536.48048584)
\curveto(870.94829358,536.38048552)(870.9282936,536.3004856)(870.8782959,536.24048584)
\curveto(870.84829368,536.18048572)(870.80329372,536.14048576)(870.7432959,536.12048584)
\curveto(870.68329384,536.11048579)(870.61329391,536.0954858)(870.5332959,536.07548584)
\lineto(870.3082959,536.07548584)
\curveto(870.17829435,536.07548582)(870.06329446,536.08048582)(869.9632959,536.09048584)
\curveto(869.87329465,536.11048579)(869.80329472,536.16048574)(869.7532959,536.24048584)
\curveto(869.71329481,536.3004856)(869.69329483,536.37548552)(869.6932959,536.46548584)
\lineto(869.6932959,536.75048584)
\lineto(869.6932959,543.09548584)
\lineto(869.6932959,543.41048584)
\curveto(869.69329483,543.52047838)(869.71829481,543.60547829)(869.7682959,543.66548584)
\curveto(869.79829473,543.71547818)(869.83829469,543.74547815)(869.8882959,543.75548584)
\curveto(869.93829459,543.76547813)(869.99329453,543.78047812)(870.0532959,543.80048584)
\curveto(870.07329445,543.8004781)(870.09329443,543.7954781)(870.1132959,543.78548584)
\curveto(870.14329438,543.78547811)(870.16829436,543.79047811)(870.1882959,543.80048584)
\curveto(870.31829421,543.8004781)(870.44829408,543.7954781)(870.5782959,543.78548584)
\curveto(870.71829381,543.78547811)(870.81329371,543.74547815)(870.8632959,543.66548584)
\curveto(870.91329361,543.60547829)(870.93829359,543.52547837)(870.9382959,543.42548584)
\lineto(870.9382959,543.14048584)
\lineto(870.9382959,536.79548584)
}
}
{
\newrgbcolor{curcolor}{0 0 0}
\pscustom[linestyle=none,fillstyle=solid,fillcolor=curcolor]
{
\newpath
\moveto(875.31313965,543.98048584)
\curveto(876.03313558,543.99047791)(876.63813498,543.90547799)(877.12813965,543.72548584)
\curveto(877.618134,543.55547834)(877.99813362,543.25047865)(878.26813965,542.81048584)
\curveto(878.33813328,542.7004792)(878.39313322,542.58547931)(878.43313965,542.46548584)
\curveto(878.47313314,542.35547954)(878.5131331,542.23047967)(878.55313965,542.09048584)
\curveto(878.57313304,542.02047988)(878.57813304,541.94547995)(878.56813965,541.86548584)
\curveto(878.55813306,541.7954801)(878.54313307,541.74048016)(878.52313965,541.70048584)
\curveto(878.50313311,541.68048022)(878.47813314,541.66048024)(878.44813965,541.64048584)
\curveto(878.4181332,541.63048027)(878.39313322,541.61548028)(878.37313965,541.59548584)
\curveto(878.32313329,541.57548032)(878.27313334,541.57048033)(878.22313965,541.58048584)
\curveto(878.17313344,541.59048031)(878.12313349,541.59048031)(878.07313965,541.58048584)
\curveto(877.99313362,541.56048034)(877.88813373,541.55548034)(877.75813965,541.56548584)
\curveto(877.62813399,541.58548031)(877.53813408,541.61048029)(877.48813965,541.64048584)
\curveto(877.40813421,541.69048021)(877.35313426,541.75548014)(877.32313965,541.83548584)
\curveto(877.30313431,541.92547997)(877.26813435,542.01047989)(877.21813965,542.09048584)
\curveto(877.12813449,542.25047965)(877.00313461,542.3954795)(876.84313965,542.52548584)
\curveto(876.73313488,542.60547929)(876.613135,542.66547923)(876.48313965,542.70548584)
\curveto(876.35313526,542.74547915)(876.2131354,542.78547911)(876.06313965,542.82548584)
\curveto(876.0131356,542.84547905)(875.96313565,542.85047905)(875.91313965,542.84048584)
\curveto(875.86313575,542.84047906)(875.8131358,542.84547905)(875.76313965,542.85548584)
\curveto(875.70313591,542.87547902)(875.62813599,542.88547901)(875.53813965,542.88548584)
\curveto(875.44813617,542.88547901)(875.37313624,542.87547902)(875.31313965,542.85548584)
\lineto(875.22313965,542.85548584)
\lineto(875.07313965,542.82548584)
\curveto(875.02313659,542.82547907)(874.97313664,542.82047908)(874.92313965,542.81048584)
\curveto(874.66313695,542.75047915)(874.44813717,542.66547923)(874.27813965,542.55548584)
\curveto(874.10813751,542.44547945)(873.99313762,542.26047964)(873.93313965,542.00048584)
\curveto(873.9131377,541.93047997)(873.90813771,541.86048004)(873.91813965,541.79048584)
\curveto(873.93813768,541.72048018)(873.95813766,541.66048024)(873.97813965,541.61048584)
\curveto(874.03813758,541.46048044)(874.10813751,541.35048055)(874.18813965,541.28048584)
\curveto(874.27813734,541.22048068)(874.38813723,541.15048075)(874.51813965,541.07048584)
\curveto(874.67813694,540.97048093)(874.85813676,540.895481)(875.05813965,540.84548584)
\curveto(875.25813636,540.80548109)(875.45813616,540.75548114)(875.65813965,540.69548584)
\curveto(875.78813583,540.65548124)(875.9181357,540.62548127)(876.04813965,540.60548584)
\curveto(876.17813544,540.58548131)(876.30813531,540.55548134)(876.43813965,540.51548584)
\curveto(876.64813497,540.45548144)(876.85313476,540.3954815)(877.05313965,540.33548584)
\curveto(877.25313436,540.28548161)(877.45313416,540.22048168)(877.65313965,540.14048584)
\lineto(877.80313965,540.08048584)
\curveto(877.85313376,540.06048184)(877.90313371,540.03548186)(877.95313965,540.00548584)
\curveto(878.15313346,539.88548201)(878.32813329,539.75048215)(878.47813965,539.60048584)
\curveto(878.62813299,539.45048245)(878.75313286,539.26048264)(878.85313965,539.03048584)
\curveto(878.87313274,538.96048294)(878.89313272,538.86548303)(878.91313965,538.74548584)
\curveto(878.93313268,538.67548322)(878.94313267,538.6004833)(878.94313965,538.52048584)
\curveto(878.95313266,538.45048345)(878.95813266,538.37048353)(878.95813965,538.28048584)
\lineto(878.95813965,538.13048584)
\curveto(878.93813268,538.06048384)(878.92813269,537.99048391)(878.92813965,537.92048584)
\curveto(878.92813269,537.85048405)(878.9181327,537.78048412)(878.89813965,537.71048584)
\curveto(878.86813275,537.6004843)(878.83313278,537.4954844)(878.79313965,537.39548584)
\curveto(878.75313286,537.2954846)(878.70813291,537.20548469)(878.65813965,537.12548584)
\curveto(878.49813312,536.86548503)(878.29313332,536.65548524)(878.04313965,536.49548584)
\curveto(877.79313382,536.34548555)(877.5131341,536.21548568)(877.20313965,536.10548584)
\curveto(877.1131345,536.07548582)(877.0181346,536.05548584)(876.91813965,536.04548584)
\curveto(876.82813479,536.02548587)(876.73813488,536.0004859)(876.64813965,535.97048584)
\curveto(876.54813507,535.95048595)(876.44813517,535.94048596)(876.34813965,535.94048584)
\curveto(876.24813537,535.94048596)(876.14813547,535.93048597)(876.04813965,535.91048584)
\lineto(875.89813965,535.91048584)
\curveto(875.84813577,535.900486)(875.77813584,535.895486)(875.68813965,535.89548584)
\curveto(875.59813602,535.895486)(875.52813609,535.900486)(875.47813965,535.91048584)
\lineto(875.31313965,535.91048584)
\curveto(875.25313636,535.93048597)(875.18813643,535.94048596)(875.11813965,535.94048584)
\curveto(875.04813657,535.93048597)(874.98813663,535.93548596)(874.93813965,535.95548584)
\curveto(874.88813673,535.96548593)(874.82313679,535.97048593)(874.74313965,535.97048584)
\lineto(874.50313965,536.03048584)
\curveto(874.43313718,536.04048586)(874.35813726,536.06048584)(874.27813965,536.09048584)
\curveto(873.96813765,536.19048571)(873.69813792,536.31548558)(873.46813965,536.46548584)
\curveto(873.23813838,536.61548528)(873.03813858,536.81048509)(872.86813965,537.05048584)
\curveto(872.77813884,537.18048472)(872.70313891,537.31548458)(872.64313965,537.45548584)
\curveto(872.58313903,537.5954843)(872.52813909,537.75048415)(872.47813965,537.92048584)
\curveto(872.45813916,537.98048392)(872.44813917,538.05048385)(872.44813965,538.13048584)
\curveto(872.45813916,538.22048368)(872.47313914,538.29048361)(872.49313965,538.34048584)
\curveto(872.52313909,538.38048352)(872.57313904,538.42048348)(872.64313965,538.46048584)
\curveto(872.69313892,538.48048342)(872.76313885,538.49048341)(872.85313965,538.49048584)
\curveto(872.94313867,538.5004834)(873.03313858,538.5004834)(873.12313965,538.49048584)
\curveto(873.2131384,538.48048342)(873.29813832,538.46548343)(873.37813965,538.44548584)
\curveto(873.46813815,538.43548346)(873.52813809,538.42048348)(873.55813965,538.40048584)
\curveto(873.62813799,538.35048355)(873.67313794,538.27548362)(873.69313965,538.17548584)
\curveto(873.72313789,538.08548381)(873.75813786,538.0004839)(873.79813965,537.92048584)
\curveto(873.89813772,537.7004842)(874.03313758,537.53048437)(874.20313965,537.41048584)
\curveto(874.32313729,537.32048458)(874.45813716,537.25048465)(874.60813965,537.20048584)
\curveto(874.75813686,537.15048475)(874.9181367,537.1004848)(875.08813965,537.05048584)
\lineto(875.40313965,537.00548584)
\lineto(875.49313965,537.00548584)
\curveto(875.56313605,536.98548491)(875.65313596,536.97548492)(875.76313965,536.97548584)
\curveto(875.88313573,536.97548492)(875.98313563,536.98548491)(876.06313965,537.00548584)
\curveto(876.13313548,537.00548489)(876.18813543,537.01048489)(876.22813965,537.02048584)
\curveto(876.28813533,537.03048487)(876.34813527,537.03548486)(876.40813965,537.03548584)
\curveto(876.46813515,537.04548485)(876.52313509,537.05548484)(876.57313965,537.06548584)
\curveto(876.86313475,537.14548475)(877.09313452,537.25048465)(877.26313965,537.38048584)
\curveto(877.43313418,537.51048439)(877.55313406,537.73048417)(877.62313965,538.04048584)
\curveto(877.64313397,538.09048381)(877.64813397,538.14548375)(877.63813965,538.20548584)
\curveto(877.62813399,538.26548363)(877.618134,538.31048359)(877.60813965,538.34048584)
\curveto(877.55813406,538.53048337)(877.48813413,538.67048323)(877.39813965,538.76048584)
\curveto(877.30813431,538.86048304)(877.19313442,538.95048295)(877.05313965,539.03048584)
\curveto(876.96313465,539.09048281)(876.86313475,539.14048276)(876.75313965,539.18048584)
\lineto(876.42313965,539.30048584)
\curveto(876.39313522,539.31048259)(876.36313525,539.31548258)(876.33313965,539.31548584)
\curveto(876.3131353,539.31548258)(876.28813533,539.32548257)(876.25813965,539.34548584)
\curveto(875.9181357,539.45548244)(875.56313605,539.53548236)(875.19313965,539.58548584)
\curveto(874.83313678,539.64548225)(874.49313712,539.74048216)(874.17313965,539.87048584)
\curveto(874.07313754,539.91048199)(873.97813764,539.94548195)(873.88813965,539.97548584)
\curveto(873.79813782,540.00548189)(873.7131379,540.04548185)(873.63313965,540.09548584)
\curveto(873.44313817,540.20548169)(873.26813835,540.33048157)(873.10813965,540.47048584)
\curveto(872.94813867,540.61048129)(872.82313879,540.78548111)(872.73313965,540.99548584)
\curveto(872.70313891,541.06548083)(872.67813894,541.13548076)(872.65813965,541.20548584)
\curveto(872.64813897,541.27548062)(872.63313898,541.35048055)(872.61313965,541.43048584)
\curveto(872.58313903,541.55048035)(872.57313904,541.68548021)(872.58313965,541.83548584)
\curveto(872.59313902,541.9954799)(872.60813901,542.13047977)(872.62813965,542.24048584)
\curveto(872.64813897,542.29047961)(872.65813896,542.33047957)(872.65813965,542.36048584)
\curveto(872.66813895,542.4004795)(872.68313893,542.44047946)(872.70313965,542.48048584)
\curveto(872.79313882,542.71047919)(872.9131387,542.91047899)(873.06313965,543.08048584)
\curveto(873.22313839,543.25047865)(873.40313821,543.4004785)(873.60313965,543.53048584)
\curveto(873.75313786,543.62047828)(873.9181377,543.69047821)(874.09813965,543.74048584)
\curveto(874.27813734,543.8004781)(874.46813715,543.85547804)(874.66813965,543.90548584)
\curveto(874.73813688,543.91547798)(874.80313681,543.92547797)(874.86313965,543.93548584)
\curveto(874.93313668,543.94547795)(875.00813661,543.95547794)(875.08813965,543.96548584)
\curveto(875.1181365,543.97547792)(875.15813646,543.97547792)(875.20813965,543.96548584)
\curveto(875.25813636,543.95547794)(875.29313632,543.96047794)(875.31313965,543.98048584)
}
}
{
\newrgbcolor{curcolor}{0 0 0}
\pscustom[linestyle=none,fillstyle=solid,fillcolor=curcolor]
{
\newpath
\moveto(881.32813965,546.14048584)
\curveto(881.47813764,546.14047576)(881.62813749,546.13547576)(881.77813965,546.12548584)
\curveto(881.92813719,546.12547577)(882.03313708,546.08547581)(882.09313965,546.00548584)
\curveto(882.14313697,545.94547595)(882.16813695,545.86047604)(882.16813965,545.75048584)
\curveto(882.17813694,545.65047625)(882.18313693,545.54547635)(882.18313965,545.43548584)
\lineto(882.18313965,544.56548584)
\curveto(882.18313693,544.48547741)(882.17813694,544.4004775)(882.16813965,544.31048584)
\curveto(882.16813695,544.23047767)(882.17813694,544.16047774)(882.19813965,544.10048584)
\curveto(882.23813688,543.96047794)(882.32813679,543.87047803)(882.46813965,543.83048584)
\curveto(882.5181366,543.82047808)(882.56313655,543.81547808)(882.60313965,543.81548584)
\lineto(882.75313965,543.81548584)
\lineto(883.15813965,543.81548584)
\curveto(883.3181358,543.82547807)(883.43313568,543.81547808)(883.50313965,543.78548584)
\curveto(883.59313552,543.72547817)(883.65313546,543.66547823)(883.68313965,543.60548584)
\curveto(883.70313541,543.56547833)(883.7131354,543.52047838)(883.71313965,543.47048584)
\lineto(883.71313965,543.32048584)
\curveto(883.7131354,543.21047869)(883.70813541,543.10547879)(883.69813965,543.00548584)
\curveto(883.68813543,542.91547898)(883.65313546,542.84547905)(883.59313965,542.79548584)
\curveto(883.53313558,542.74547915)(883.44813567,542.71547918)(883.33813965,542.70548584)
\lineto(883.00813965,542.70548584)
\curveto(882.89813622,542.71547918)(882.78813633,542.72047918)(882.67813965,542.72048584)
\curveto(882.56813655,542.72047918)(882.47313664,542.70547919)(882.39313965,542.67548584)
\curveto(882.32313679,542.64547925)(882.27313684,542.5954793)(882.24313965,542.52548584)
\curveto(882.2131369,542.45547944)(882.19313692,542.37047953)(882.18313965,542.27048584)
\curveto(882.17313694,542.18047972)(882.16813695,542.08047982)(882.16813965,541.97048584)
\curveto(882.17813694,541.87048003)(882.18313693,541.77048013)(882.18313965,541.67048584)
\lineto(882.18313965,538.70048584)
\curveto(882.18313693,538.48048342)(882.17813694,538.24548365)(882.16813965,537.99548584)
\curveto(882.16813695,537.75548414)(882.2131369,537.57048433)(882.30313965,537.44048584)
\curveto(882.35313676,537.36048454)(882.4181367,537.30548459)(882.49813965,537.27548584)
\curveto(882.57813654,537.24548465)(882.67313644,537.22048468)(882.78313965,537.20048584)
\curveto(882.8131363,537.19048471)(882.84313627,537.18548471)(882.87313965,537.18548584)
\curveto(882.9131362,537.1954847)(882.94813617,537.1954847)(882.97813965,537.18548584)
\lineto(883.17313965,537.18548584)
\curveto(883.27313584,537.18548471)(883.36313575,537.17548472)(883.44313965,537.15548584)
\curveto(883.53313558,537.14548475)(883.59813552,537.11048479)(883.63813965,537.05048584)
\curveto(883.65813546,537.02048488)(883.67313544,536.96548493)(883.68313965,536.88548584)
\curveto(883.70313541,536.81548508)(883.7131354,536.74048516)(883.71313965,536.66048584)
\curveto(883.72313539,536.58048532)(883.72313539,536.5004854)(883.71313965,536.42048584)
\curveto(883.70313541,536.35048555)(883.68313543,536.2954856)(883.65313965,536.25548584)
\curveto(883.6131355,536.18548571)(883.53813558,536.13548576)(883.42813965,536.10548584)
\curveto(883.34813577,536.08548581)(883.25813586,536.07548582)(883.15813965,536.07548584)
\curveto(883.05813606,536.08548581)(882.96813615,536.09048581)(882.88813965,536.09048584)
\curveto(882.82813629,536.09048581)(882.76813635,536.08548581)(882.70813965,536.07548584)
\curveto(882.64813647,536.07548582)(882.59313652,536.08048582)(882.54313965,536.09048584)
\lineto(882.36313965,536.09048584)
\curveto(882.3131368,536.1004858)(882.26313685,536.10548579)(882.21313965,536.10548584)
\curveto(882.17313694,536.11548578)(882.12813699,536.12048578)(882.07813965,536.12048584)
\curveto(881.87813724,536.17048573)(881.70313741,536.22548567)(881.55313965,536.28548584)
\curveto(881.4131377,536.34548555)(881.29313782,536.45048545)(881.19313965,536.60048584)
\curveto(881.05313806,536.8004851)(880.97313814,537.05048485)(880.95313965,537.35048584)
\curveto(880.93313818,537.66048424)(880.92313819,537.99048391)(880.92313965,538.34048584)
\lineto(880.92313965,542.27048584)
\curveto(880.89313822,542.4004795)(880.86313825,542.4954794)(880.83313965,542.55548584)
\curveto(880.8131383,542.61547928)(880.74313837,542.66547923)(880.62313965,542.70548584)
\curveto(880.58313853,542.71547918)(880.54313857,542.71547918)(880.50313965,542.70548584)
\curveto(880.46313865,542.6954792)(880.42313869,542.7004792)(880.38313965,542.72048584)
\lineto(880.14313965,542.72048584)
\curveto(880.0131391,542.72047918)(879.90313921,542.73047917)(879.81313965,542.75048584)
\curveto(879.73313938,542.78047912)(879.67813944,542.84047906)(879.64813965,542.93048584)
\curveto(879.62813949,542.97047893)(879.6131395,543.01547888)(879.60313965,543.06548584)
\lineto(879.60313965,543.21548584)
\curveto(879.60313951,543.35547854)(879.6131395,543.47047843)(879.63313965,543.56048584)
\curveto(879.65313946,543.66047824)(879.7131394,543.73547816)(879.81313965,543.78548584)
\curveto(879.92313919,543.82547807)(880.06313905,543.83547806)(880.23313965,543.81548584)
\curveto(880.4131387,543.7954781)(880.56313855,543.80547809)(880.68313965,543.84548584)
\curveto(880.77313834,543.895478)(880.84313827,543.96547793)(880.89313965,544.05548584)
\curveto(880.9131382,544.11547778)(880.92313819,544.19047771)(880.92313965,544.28048584)
\lineto(880.92313965,544.53548584)
\lineto(880.92313965,545.46548584)
\lineto(880.92313965,545.70548584)
\curveto(880.92313819,545.7954761)(880.93313818,545.87047603)(880.95313965,545.93048584)
\curveto(880.99313812,546.01047589)(881.06813805,546.07547582)(881.17813965,546.12548584)
\curveto(881.20813791,546.12547577)(881.23313788,546.12547577)(881.25313965,546.12548584)
\curveto(881.28313783,546.13547576)(881.30813781,546.14047576)(881.32813965,546.14048584)
}
}
{
\newrgbcolor{curcolor}{0 0 0}
\pscustom[linestyle=none,fillstyle=solid,fillcolor=curcolor]
{
\newpath
\moveto(888.74493652,543.98048584)
\curveto(888.97493173,543.98047792)(889.1049316,543.92047798)(889.13493652,543.80048584)
\curveto(889.16493154,543.69047821)(889.17993153,543.52547837)(889.17993652,543.30548584)
\lineto(889.17993652,543.02048584)
\curveto(889.17993153,542.93047897)(889.15493155,542.85547904)(889.10493652,542.79548584)
\curveto(889.04493166,542.71547918)(888.95993175,542.67047923)(888.84993652,542.66048584)
\curveto(888.73993197,542.66047924)(888.62993208,542.64547925)(888.51993652,542.61548584)
\curveto(888.37993233,542.58547931)(888.24493246,542.55547934)(888.11493652,542.52548584)
\curveto(887.99493271,542.4954794)(887.87993283,542.45547944)(887.76993652,542.40548584)
\curveto(887.47993323,542.27547962)(887.24493346,542.0954798)(887.06493652,541.86548584)
\curveto(886.88493382,541.64548025)(886.72993398,541.39048051)(886.59993652,541.10048584)
\curveto(886.55993415,540.99048091)(886.52993418,540.87548102)(886.50993652,540.75548584)
\curveto(886.48993422,540.64548125)(886.46493424,540.53048137)(886.43493652,540.41048584)
\curveto(886.42493428,540.36048154)(886.41993429,540.31048159)(886.41993652,540.26048584)
\curveto(886.42993428,540.21048169)(886.42993428,540.16048174)(886.41993652,540.11048584)
\curveto(886.38993432,539.99048191)(886.37493433,539.85048205)(886.37493652,539.69048584)
\curveto(886.38493432,539.54048236)(886.38993432,539.3954825)(886.38993652,539.25548584)
\lineto(886.38993652,537.41048584)
\lineto(886.38993652,537.06548584)
\curveto(886.38993432,536.94548495)(886.38493432,536.83048507)(886.37493652,536.72048584)
\curveto(886.36493434,536.61048529)(886.35993435,536.51548538)(886.35993652,536.43548584)
\curveto(886.36993434,536.35548554)(886.34993436,536.28548561)(886.29993652,536.22548584)
\curveto(886.24993446,536.15548574)(886.16993454,536.11548578)(886.05993652,536.10548584)
\curveto(885.95993475,536.0954858)(885.84993486,536.09048581)(885.72993652,536.09048584)
\lineto(885.45993652,536.09048584)
\curveto(885.4099353,536.11048579)(885.35993535,536.12548577)(885.30993652,536.13548584)
\curveto(885.26993544,536.15548574)(885.23993547,536.18048572)(885.21993652,536.21048584)
\curveto(885.16993554,536.28048562)(885.13993557,536.36548553)(885.12993652,536.46548584)
\lineto(885.12993652,536.79548584)
\lineto(885.12993652,537.95048584)
\lineto(885.12993652,542.10548584)
\lineto(885.12993652,543.14048584)
\lineto(885.12993652,543.44048584)
\curveto(885.13993557,543.54047836)(885.16993554,543.62547827)(885.21993652,543.69548584)
\curveto(885.24993546,543.73547816)(885.29993541,543.76547813)(885.36993652,543.78548584)
\curveto(885.44993526,543.80547809)(885.53493517,543.81547808)(885.62493652,543.81548584)
\curveto(885.71493499,543.82547807)(885.8049349,543.82547807)(885.89493652,543.81548584)
\curveto(885.98493472,543.80547809)(886.05493465,543.79047811)(886.10493652,543.77048584)
\curveto(886.18493452,543.74047816)(886.23493447,543.68047822)(886.25493652,543.59048584)
\curveto(886.28493442,543.51047839)(886.29993441,543.42047848)(886.29993652,543.32048584)
\lineto(886.29993652,543.02048584)
\curveto(886.29993441,542.92047898)(886.31993439,542.83047907)(886.35993652,542.75048584)
\curveto(886.36993434,542.73047917)(886.37993433,542.71547918)(886.38993652,542.70548584)
\lineto(886.43493652,542.66048584)
\curveto(886.54493416,542.66047924)(886.63493407,542.70547919)(886.70493652,542.79548584)
\curveto(886.77493393,542.895479)(886.83493387,542.97547892)(886.88493652,543.03548584)
\lineto(886.97493652,543.12548584)
\curveto(887.06493364,543.23547866)(887.18993352,543.35047855)(887.34993652,543.47048584)
\curveto(887.5099332,543.59047831)(887.65993305,543.68047822)(887.79993652,543.74048584)
\curveto(887.88993282,543.79047811)(887.98493272,543.82547807)(888.08493652,543.84548584)
\curveto(888.18493252,543.87547802)(888.28993242,543.90547799)(888.39993652,543.93548584)
\curveto(888.45993225,543.94547795)(888.51993219,543.95047795)(888.57993652,543.95048584)
\curveto(888.63993207,543.96047794)(888.69493201,543.97047793)(888.74493652,543.98048584)
}
}
{
\newrgbcolor{curcolor}{0 0 0}
\pscustom[linestyle=none,fillstyle=solid,fillcolor=curcolor]
{
\newpath
\moveto(896.99470215,536.63048584)
\curveto(897.02469432,536.47048543)(897.00969433,536.33548556)(896.94970215,536.22548584)
\curveto(896.88969445,536.12548577)(896.80969453,536.05048585)(896.70970215,536.00048584)
\curveto(896.65969468,535.98048592)(896.60469474,535.97048593)(896.54470215,535.97048584)
\curveto(896.49469485,535.97048593)(896.4396949,535.96048594)(896.37970215,535.94048584)
\curveto(896.15969518,535.89048601)(895.9396954,535.90548599)(895.71970215,535.98548584)
\curveto(895.50969583,536.05548584)(895.36469598,536.14548575)(895.28470215,536.25548584)
\curveto(895.23469611,536.32548557)(895.18969615,536.40548549)(895.14970215,536.49548584)
\curveto(895.10969623,536.5954853)(895.05969628,536.67548522)(894.99970215,536.73548584)
\curveto(894.97969636,536.75548514)(894.95469639,536.77548512)(894.92470215,536.79548584)
\curveto(894.90469644,536.81548508)(894.87469647,536.82048508)(894.83470215,536.81048584)
\curveto(894.72469662,536.78048512)(894.61969672,536.72548517)(894.51970215,536.64548584)
\curveto(894.42969691,536.56548533)(894.339697,536.4954854)(894.24970215,536.43548584)
\curveto(894.11969722,536.35548554)(893.97969736,536.28048562)(893.82970215,536.21048584)
\curveto(893.67969766,536.15048575)(893.51969782,536.0954858)(893.34970215,536.04548584)
\curveto(893.24969809,536.01548588)(893.1396982,535.9954859)(893.01970215,535.98548584)
\curveto(892.90969843,535.97548592)(892.79969854,535.96048594)(892.68970215,535.94048584)
\curveto(892.6396987,535.93048597)(892.59469875,535.92548597)(892.55470215,535.92548584)
\lineto(892.44970215,535.92548584)
\curveto(892.339699,535.90548599)(892.23469911,535.90548599)(892.13470215,535.92548584)
\lineto(891.99970215,535.92548584)
\curveto(891.94969939,535.93548596)(891.89969944,535.94048596)(891.84970215,535.94048584)
\curveto(891.79969954,535.94048596)(891.75469959,535.95048595)(891.71470215,535.97048584)
\curveto(891.67469967,535.98048592)(891.6396997,535.98548591)(891.60970215,535.98548584)
\curveto(891.58969975,535.97548592)(891.56469978,535.97548592)(891.53470215,535.98548584)
\lineto(891.29470215,536.04548584)
\curveto(891.21470013,536.05548584)(891.1397002,536.07548582)(891.06970215,536.10548584)
\curveto(890.76970057,536.23548566)(890.52470082,536.38048552)(890.33470215,536.54048584)
\curveto(890.15470119,536.71048519)(890.00470134,536.94548495)(889.88470215,537.24548584)
\curveto(889.79470155,537.46548443)(889.74970159,537.73048417)(889.74970215,538.04048584)
\lineto(889.74970215,538.35548584)
\curveto(889.75970158,538.40548349)(889.76470158,538.45548344)(889.76470215,538.50548584)
\lineto(889.79470215,538.68548584)
\lineto(889.91470215,539.01548584)
\curveto(889.95470139,539.12548277)(890.00470134,539.22548267)(890.06470215,539.31548584)
\curveto(890.2447011,539.60548229)(890.48970085,539.82048208)(890.79970215,539.96048584)
\curveto(891.10970023,540.1004818)(891.44969989,540.22548167)(891.81970215,540.33548584)
\curveto(891.95969938,540.37548152)(892.10469924,540.40548149)(892.25470215,540.42548584)
\curveto(892.40469894,540.44548145)(892.55469879,540.47048143)(892.70470215,540.50048584)
\curveto(892.77469857,540.52048138)(892.8396985,540.53048137)(892.89970215,540.53048584)
\curveto(892.96969837,540.53048137)(893.0446983,540.54048136)(893.12470215,540.56048584)
\curveto(893.19469815,540.58048132)(893.26469808,540.59048131)(893.33470215,540.59048584)
\curveto(893.40469794,540.6004813)(893.47969786,540.61548128)(893.55970215,540.63548584)
\curveto(893.80969753,540.6954812)(894.0446973,540.74548115)(894.26470215,540.78548584)
\curveto(894.48469686,540.83548106)(894.65969668,540.95048095)(894.78970215,541.13048584)
\curveto(894.84969649,541.21048069)(894.89969644,541.31048059)(894.93970215,541.43048584)
\curveto(894.97969636,541.56048034)(894.97969636,541.7004802)(894.93970215,541.85048584)
\curveto(894.87969646,542.09047981)(894.78969655,542.28047962)(894.66970215,542.42048584)
\curveto(894.55969678,542.56047934)(894.39969694,542.67047923)(894.18970215,542.75048584)
\curveto(894.06969727,542.8004791)(893.92469742,542.83547906)(893.75470215,542.85548584)
\curveto(893.59469775,542.87547902)(893.42469792,542.88547901)(893.24470215,542.88548584)
\curveto(893.06469828,542.88547901)(892.88969845,542.87547902)(892.71970215,542.85548584)
\curveto(892.54969879,542.83547906)(892.40469894,542.80547909)(892.28470215,542.76548584)
\curveto(892.11469923,542.70547919)(891.94969939,542.62047928)(891.78970215,542.51048584)
\curveto(891.70969963,542.45047945)(891.63469971,542.37047953)(891.56470215,542.27048584)
\curveto(891.50469984,542.18047972)(891.44969989,542.08047982)(891.39970215,541.97048584)
\curveto(891.36969997,541.89048001)(891.3397,541.80548009)(891.30970215,541.71548584)
\curveto(891.28970005,541.62548027)(891.2447001,541.55548034)(891.17470215,541.50548584)
\curveto(891.13470021,541.47548042)(891.06470028,541.45048045)(890.96470215,541.43048584)
\curveto(890.87470047,541.42048048)(890.77970056,541.41548048)(890.67970215,541.41548584)
\curveto(890.57970076,541.41548048)(890.47970086,541.42048048)(890.37970215,541.43048584)
\curveto(890.28970105,541.45048045)(890.22470112,541.47548042)(890.18470215,541.50548584)
\curveto(890.1447012,541.53548036)(890.11470123,541.58548031)(890.09470215,541.65548584)
\curveto(890.07470127,541.72548017)(890.07470127,541.8004801)(890.09470215,541.88048584)
\curveto(890.12470122,542.01047989)(890.15470119,542.13047977)(890.18470215,542.24048584)
\curveto(890.22470112,542.36047954)(890.26970107,542.47547942)(890.31970215,542.58548584)
\curveto(890.50970083,542.93547896)(890.74970059,543.20547869)(891.03970215,543.39548584)
\curveto(891.32970001,543.5954783)(891.68969965,543.75547814)(892.11970215,543.87548584)
\curveto(892.21969912,543.895478)(892.31969902,543.91047799)(892.41970215,543.92048584)
\curveto(892.52969881,543.93047797)(892.6396987,543.94547795)(892.74970215,543.96548584)
\curveto(892.78969855,543.97547792)(892.85469849,543.97547792)(892.94470215,543.96548584)
\curveto(893.03469831,543.96547793)(893.08969825,543.97547792)(893.10970215,543.99548584)
\curveto(893.80969753,544.00547789)(894.41969692,543.92547797)(894.93970215,543.75548584)
\curveto(895.45969588,543.58547831)(895.82469552,543.26047864)(896.03470215,542.78048584)
\curveto(896.12469522,542.58047932)(896.17469517,542.34547955)(896.18470215,542.07548584)
\curveto(896.20469514,541.81548008)(896.21469513,541.54048036)(896.21470215,541.25048584)
\lineto(896.21470215,537.93548584)
\curveto(896.21469513,537.7954841)(896.21969512,537.66048424)(896.22970215,537.53048584)
\curveto(896.2396951,537.4004845)(896.26969507,537.2954846)(896.31970215,537.21548584)
\curveto(896.36969497,537.14548475)(896.43469491,537.0954848)(896.51470215,537.06548584)
\curveto(896.60469474,537.02548487)(896.68969465,536.9954849)(896.76970215,536.97548584)
\curveto(896.84969449,536.96548493)(896.90969443,536.92048498)(896.94970215,536.84048584)
\curveto(896.96969437,536.81048509)(896.97969436,536.78048512)(896.97970215,536.75048584)
\curveto(896.97969436,536.72048518)(896.98469436,536.68048522)(896.99470215,536.63048584)
\moveto(894.84970215,538.29548584)
\curveto(894.90969643,538.43548346)(894.9396964,538.5954833)(894.93970215,538.77548584)
\curveto(894.94969639,538.96548293)(894.95469639,539.16048274)(894.95470215,539.36048584)
\curveto(894.95469639,539.47048243)(894.94969639,539.57048233)(894.93970215,539.66048584)
\curveto(894.92969641,539.75048215)(894.88969645,539.82048208)(894.81970215,539.87048584)
\curveto(894.78969655,539.89048201)(894.71969662,539.900482)(894.60970215,539.90048584)
\curveto(894.58969675,539.88048202)(894.55469679,539.87048203)(894.50470215,539.87048584)
\curveto(894.45469689,539.87048203)(894.40969693,539.86048204)(894.36970215,539.84048584)
\curveto(894.28969705,539.82048208)(894.19969714,539.8004821)(894.09970215,539.78048584)
\lineto(893.79970215,539.72048584)
\curveto(893.76969757,539.72048218)(893.73469761,539.71548218)(893.69470215,539.70548584)
\lineto(893.58970215,539.70548584)
\curveto(893.4396979,539.66548223)(893.27469807,539.64048226)(893.09470215,539.63048584)
\curveto(892.92469842,539.63048227)(892.76469858,539.61048229)(892.61470215,539.57048584)
\curveto(892.53469881,539.55048235)(892.45969888,539.53048237)(892.38970215,539.51048584)
\curveto(892.32969901,539.5004824)(892.25969908,539.48548241)(892.17970215,539.46548584)
\curveto(892.01969932,539.41548248)(891.86969947,539.35048255)(891.72970215,539.27048584)
\curveto(891.58969975,539.2004827)(891.46969987,539.11048279)(891.36970215,539.00048584)
\curveto(891.26970007,538.89048301)(891.19470015,538.75548314)(891.14470215,538.59548584)
\curveto(891.09470025,538.44548345)(891.07470027,538.26048364)(891.08470215,538.04048584)
\curveto(891.08470026,537.94048396)(891.09970024,537.84548405)(891.12970215,537.75548584)
\curveto(891.16970017,537.67548422)(891.21470013,537.6004843)(891.26470215,537.53048584)
\curveto(891.3447,537.42048448)(891.44969989,537.32548457)(891.57970215,537.24548584)
\curveto(891.70969963,537.17548472)(891.84969949,537.11548478)(891.99970215,537.06548584)
\curveto(892.04969929,537.05548484)(892.09969924,537.05048485)(892.14970215,537.05048584)
\curveto(892.19969914,537.05048485)(892.24969909,537.04548485)(892.29970215,537.03548584)
\curveto(892.36969897,537.01548488)(892.45469889,537.0004849)(892.55470215,536.99048584)
\curveto(892.66469868,536.99048491)(892.75469859,537.0004849)(892.82470215,537.02048584)
\curveto(892.88469846,537.04048486)(892.9446984,537.04548485)(893.00470215,537.03548584)
\curveto(893.06469828,537.03548486)(893.12469822,537.04548485)(893.18470215,537.06548584)
\curveto(893.26469808,537.08548481)(893.339698,537.1004848)(893.40970215,537.11048584)
\curveto(893.48969785,537.12048478)(893.56469778,537.14048476)(893.63470215,537.17048584)
\curveto(893.92469742,537.29048461)(894.16969717,537.43548446)(894.36970215,537.60548584)
\curveto(894.57969676,537.77548412)(894.7396966,538.00548389)(894.84970215,538.29548584)
}
}
{
\newrgbcolor{curcolor}{0 0 0}
\pscustom[linestyle=none,fillstyle=solid,fillcolor=curcolor]
{
\newpath
\moveto(905.12634277,536.88548584)
\lineto(905.12634277,536.49548584)
\curveto(905.1263349,536.37548552)(905.10133492,536.27548562)(905.05134277,536.19548584)
\curveto(905.00133502,536.12548577)(904.91633511,536.08548581)(904.79634277,536.07548584)
\lineto(904.45134277,536.07548584)
\curveto(904.39133563,536.07548582)(904.33133569,536.07048583)(904.27134277,536.06048584)
\curveto(904.2213358,536.06048584)(904.17633585,536.07048583)(904.13634277,536.09048584)
\curveto(904.04633598,536.11048579)(903.98633604,536.15048575)(903.95634277,536.21048584)
\curveto(903.91633611,536.26048564)(903.89133613,536.32048558)(903.88134277,536.39048584)
\curveto(903.88133614,536.46048544)(903.86633616,536.53048537)(903.83634277,536.60048584)
\curveto(903.8263362,536.62048528)(903.81133621,536.63548526)(903.79134277,536.64548584)
\curveto(903.78133624,536.66548523)(903.76633626,536.68548521)(903.74634277,536.70548584)
\curveto(903.64633638,536.71548518)(903.56633646,536.6954852)(903.50634277,536.64548584)
\curveto(903.45633657,536.5954853)(903.40133662,536.54548535)(903.34134277,536.49548584)
\curveto(903.14133688,536.34548555)(902.94133708,536.23048567)(902.74134277,536.15048584)
\curveto(902.56133746,536.07048583)(902.35133767,536.01048589)(902.11134277,535.97048584)
\curveto(901.88133814,535.93048597)(901.64133838,535.91048599)(901.39134277,535.91048584)
\curveto(901.15133887,535.900486)(900.91133911,535.91548598)(900.67134277,535.95548584)
\curveto(900.43133959,535.98548591)(900.2213398,536.04048586)(900.04134277,536.12048584)
\curveto(899.5213405,536.34048556)(899.10134092,536.63548526)(898.78134277,537.00548584)
\curveto(898.46134156,537.38548451)(898.21134181,537.85548404)(898.03134277,538.41548584)
\curveto(897.99134203,538.50548339)(897.96134206,538.5954833)(897.94134277,538.68548584)
\curveto(897.93134209,538.78548311)(897.91134211,538.88548301)(897.88134277,538.98548584)
\curveto(897.87134215,539.03548286)(897.86634216,539.08548281)(897.86634277,539.13548584)
\curveto(897.86634216,539.18548271)(897.86134216,539.23548266)(897.85134277,539.28548584)
\curveto(897.83134219,539.33548256)(897.8213422,539.38548251)(897.82134277,539.43548584)
\curveto(897.83134219,539.4954824)(897.83134219,539.55048235)(897.82134277,539.60048584)
\lineto(897.82134277,539.75048584)
\curveto(897.80134222,539.8004821)(897.79134223,539.86548203)(897.79134277,539.94548584)
\curveto(897.79134223,540.02548187)(897.80134222,540.09048181)(897.82134277,540.14048584)
\lineto(897.82134277,540.30548584)
\curveto(897.84134218,540.37548152)(897.84634218,540.44548145)(897.83634277,540.51548584)
\curveto(897.83634219,540.5954813)(897.84634218,540.67048123)(897.86634277,540.74048584)
\curveto(897.87634215,540.79048111)(897.88134214,540.83548106)(897.88134277,540.87548584)
\curveto(897.88134214,540.91548098)(897.88634214,540.96048094)(897.89634277,541.01048584)
\curveto(897.9263421,541.11048079)(897.95134207,541.20548069)(897.97134277,541.29548584)
\curveto(897.99134203,541.3954805)(898.01634201,541.49048041)(898.04634277,541.58048584)
\curveto(898.17634185,541.96047994)(898.34134168,542.3004796)(898.54134277,542.60048584)
\curveto(898.75134127,542.91047899)(899.00134102,543.16547873)(899.29134277,543.36548584)
\curveto(899.46134056,543.48547841)(899.63634039,543.58547831)(899.81634277,543.66548584)
\curveto(900.00634002,543.74547815)(900.21133981,543.81547808)(900.43134277,543.87548584)
\curveto(900.50133952,543.88547801)(900.56633946,543.895478)(900.62634277,543.90548584)
\curveto(900.69633933,543.91547798)(900.76633926,543.93047797)(900.83634277,543.95048584)
\lineto(900.98634277,543.95048584)
\curveto(901.06633896,543.97047793)(901.18133884,543.98047792)(901.33134277,543.98048584)
\curveto(901.49133853,543.98047792)(901.61133841,543.97047793)(901.69134277,543.95048584)
\curveto(901.73133829,543.94047796)(901.78633824,543.93547796)(901.85634277,543.93548584)
\curveto(901.96633806,543.90547799)(902.07633795,543.88047802)(902.18634277,543.86048584)
\curveto(902.29633773,543.85047805)(902.40133762,543.82047808)(902.50134277,543.77048584)
\curveto(902.65133737,543.71047819)(902.79133723,543.64547825)(902.92134277,543.57548584)
\curveto(903.06133696,543.50547839)(903.19133683,543.42547847)(903.31134277,543.33548584)
\curveto(903.37133665,543.28547861)(903.43133659,543.23047867)(903.49134277,543.17048584)
\curveto(903.56133646,543.12047878)(903.65133637,543.10547879)(903.76134277,543.12548584)
\curveto(903.78133624,543.15547874)(903.79633623,543.18047872)(903.80634277,543.20048584)
\curveto(903.8263362,543.22047868)(903.84133618,543.25047865)(903.85134277,543.29048584)
\curveto(903.88133614,543.38047852)(903.89133613,543.4954784)(903.88134277,543.63548584)
\lineto(903.88134277,544.01048584)
\lineto(903.88134277,545.73548584)
\lineto(903.88134277,546.20048584)
\curveto(903.88133614,546.38047552)(903.90633612,546.51047539)(903.95634277,546.59048584)
\curveto(903.99633603,546.66047524)(904.05633597,546.70547519)(904.13634277,546.72548584)
\curveto(904.15633587,546.72547517)(904.18133584,546.72547517)(904.21134277,546.72548584)
\curveto(904.24133578,546.73547516)(904.26633576,546.74047516)(904.28634277,546.74048584)
\curveto(904.4263356,546.75047515)(904.57133545,546.75047515)(904.72134277,546.74048584)
\curveto(904.88133514,546.74047516)(904.99133503,546.7004752)(905.05134277,546.62048584)
\curveto(905.10133492,546.54047536)(905.1263349,546.44047546)(905.12634277,546.32048584)
\lineto(905.12634277,545.94548584)
\lineto(905.12634277,536.88548584)
\moveto(903.91134277,539.72048584)
\curveto(903.93133609,539.77048213)(903.94133608,539.83548206)(903.94134277,539.91548584)
\curveto(903.94133608,540.00548189)(903.93133609,540.07548182)(903.91134277,540.12548584)
\lineto(903.91134277,540.35048584)
\curveto(903.89133613,540.44048146)(903.87633615,540.53048137)(903.86634277,540.62048584)
\curveto(903.85633617,540.72048118)(903.83633619,540.81048109)(903.80634277,540.89048584)
\curveto(903.78633624,540.97048093)(903.76633626,541.04548085)(903.74634277,541.11548584)
\curveto(903.73633629,541.18548071)(903.71633631,541.25548064)(903.68634277,541.32548584)
\curveto(903.56633646,541.62548027)(903.41133661,541.89048001)(903.22134277,542.12048584)
\curveto(903.03133699,542.35047955)(902.79133723,542.53047937)(902.50134277,542.66048584)
\curveto(902.40133762,542.71047919)(902.29633773,542.74547915)(902.18634277,542.76548584)
\curveto(902.08633794,542.7954791)(901.97633805,542.82047908)(901.85634277,542.84048584)
\curveto(901.77633825,542.86047904)(901.68633834,542.87047903)(901.58634277,542.87048584)
\lineto(901.31634277,542.87048584)
\curveto(901.26633876,542.86047904)(901.2213388,542.85047905)(901.18134277,542.84048584)
\lineto(901.04634277,542.84048584)
\curveto(900.96633906,542.82047908)(900.88133914,542.8004791)(900.79134277,542.78048584)
\curveto(900.71133931,542.76047914)(900.63133939,542.73547916)(900.55134277,542.70548584)
\curveto(900.23133979,542.56547933)(899.97134005,542.36047954)(899.77134277,542.09048584)
\curveto(899.58134044,541.83048007)(899.4263406,541.52548037)(899.30634277,541.17548584)
\curveto(899.26634076,541.06548083)(899.23634079,540.95048095)(899.21634277,540.83048584)
\curveto(899.20634082,540.72048118)(899.19134083,540.61048129)(899.17134277,540.50048584)
\curveto(899.17134085,540.46048144)(899.16634086,540.42048148)(899.15634277,540.38048584)
\lineto(899.15634277,540.27548584)
\curveto(899.13634089,540.22548167)(899.1263409,540.17048173)(899.12634277,540.11048584)
\curveto(899.13634089,540.05048185)(899.14134088,539.9954819)(899.14134277,539.94548584)
\lineto(899.14134277,539.61548584)
\curveto(899.14134088,539.51548238)(899.15134087,539.42048248)(899.17134277,539.33048584)
\curveto(899.18134084,539.3004826)(899.18634084,539.25048265)(899.18634277,539.18048584)
\curveto(899.20634082,539.11048279)(899.2213408,539.04048286)(899.23134277,538.97048584)
\lineto(899.29134277,538.76048584)
\curveto(899.40134062,538.41048349)(899.55134047,538.11048379)(899.74134277,537.86048584)
\curveto(899.93134009,537.61048429)(900.17133985,537.40548449)(900.46134277,537.24548584)
\curveto(900.55133947,537.1954847)(900.64133938,537.15548474)(900.73134277,537.12548584)
\curveto(900.8213392,537.0954848)(900.9213391,537.06548483)(901.03134277,537.03548584)
\curveto(901.08133894,537.01548488)(901.13133889,537.01048489)(901.18134277,537.02048584)
\curveto(901.24133878,537.03048487)(901.29633873,537.02548487)(901.34634277,537.00548584)
\curveto(901.38633864,536.9954849)(901.4263386,536.99048491)(901.46634277,536.99048584)
\lineto(901.60134277,536.99048584)
\lineto(901.73634277,536.99048584)
\curveto(901.76633826,537.0004849)(901.81633821,537.00548489)(901.88634277,537.00548584)
\curveto(901.96633806,537.02548487)(902.04633798,537.04048486)(902.12634277,537.05048584)
\curveto(902.20633782,537.07048483)(902.28133774,537.0954848)(902.35134277,537.12548584)
\curveto(902.68133734,537.26548463)(902.94633708,537.44048446)(903.14634277,537.65048584)
\curveto(903.35633667,537.87048403)(903.53133649,538.14548375)(903.67134277,538.47548584)
\curveto(903.7213363,538.58548331)(903.75633627,538.6954832)(903.77634277,538.80548584)
\curveto(903.79633623,538.91548298)(903.8213362,539.02548287)(903.85134277,539.13548584)
\curveto(903.87133615,539.17548272)(903.88133614,539.21048269)(903.88134277,539.24048584)
\curveto(903.88133614,539.28048262)(903.88633614,539.32048258)(903.89634277,539.36048584)
\curveto(903.90633612,539.42048248)(903.90633612,539.48048242)(903.89634277,539.54048584)
\curveto(903.89633613,539.6004823)(903.90133612,539.66048224)(903.91134277,539.72048584)
}
}
{
\newrgbcolor{curcolor}{0 0 0}
\pscustom[linestyle=none,fillstyle=solid,fillcolor=curcolor]
{
\newpath
\moveto(914.19759277,540.27548584)
\curveto(914.21758471,540.21548168)(914.2275847,540.12048178)(914.22759277,539.99048584)
\curveto(914.2275847,539.87048203)(914.22258471,539.78548211)(914.21259277,539.73548584)
\lineto(914.21259277,539.58548584)
\curveto(914.20258473,539.50548239)(914.19258474,539.43048247)(914.18259277,539.36048584)
\curveto(914.18258475,539.3004826)(914.17758475,539.23048267)(914.16759277,539.15048584)
\curveto(914.14758478,539.09048281)(914.1325848,539.03048287)(914.12259277,538.97048584)
\curveto(914.12258481,538.91048299)(914.11258482,538.85048305)(914.09259277,538.79048584)
\curveto(914.05258488,538.66048324)(914.01758491,538.53048337)(913.98759277,538.40048584)
\curveto(913.95758497,538.27048363)(913.91758501,538.15048375)(913.86759277,538.04048584)
\curveto(913.65758527,537.56048434)(913.37758555,537.15548474)(913.02759277,536.82548584)
\curveto(912.67758625,536.50548539)(912.24758668,536.26048564)(911.73759277,536.09048584)
\curveto(911.6275873,536.05048585)(911.50758742,536.02048588)(911.37759277,536.00048584)
\curveto(911.25758767,535.98048592)(911.1325878,535.96048594)(911.00259277,535.94048584)
\curveto(910.94258799,535.93048597)(910.87758805,535.92548597)(910.80759277,535.92548584)
\curveto(910.74758818,535.91548598)(910.68758824,535.91048599)(910.62759277,535.91048584)
\curveto(910.58758834,535.900486)(910.5275884,535.895486)(910.44759277,535.89548584)
\curveto(910.37758855,535.895486)(910.3275886,535.900486)(910.29759277,535.91048584)
\curveto(910.25758867,535.92048598)(910.21758871,535.92548597)(910.17759277,535.92548584)
\curveto(910.13758879,535.91548598)(910.10258883,535.91548598)(910.07259277,535.92548584)
\lineto(909.98259277,535.92548584)
\lineto(909.62259277,535.97048584)
\curveto(909.48258945,536.01048589)(909.34758958,536.05048585)(909.21759277,536.09048584)
\curveto(909.08758984,536.13048577)(908.96258997,536.17548572)(908.84259277,536.22548584)
\curveto(908.39259054,536.42548547)(908.02259091,536.68548521)(907.73259277,537.00548584)
\curveto(907.44259149,537.32548457)(907.20259173,537.71548418)(907.01259277,538.17548584)
\curveto(906.96259197,538.27548362)(906.92259201,538.37548352)(906.89259277,538.47548584)
\curveto(906.87259206,538.57548332)(906.85259208,538.68048322)(906.83259277,538.79048584)
\curveto(906.81259212,538.83048307)(906.80259213,538.86048304)(906.80259277,538.88048584)
\curveto(906.81259212,538.91048299)(906.81259212,538.94548295)(906.80259277,538.98548584)
\curveto(906.78259215,539.06548283)(906.76759216,539.14548275)(906.75759277,539.22548584)
\curveto(906.75759217,539.31548258)(906.74759218,539.4004825)(906.72759277,539.48048584)
\lineto(906.72759277,539.60048584)
\curveto(906.7275922,539.64048226)(906.72259221,539.68548221)(906.71259277,539.73548584)
\curveto(906.70259223,539.78548211)(906.69759223,539.87048203)(906.69759277,539.99048584)
\curveto(906.69759223,540.12048178)(906.70759222,540.21548168)(906.72759277,540.27548584)
\curveto(906.74759218,540.34548155)(906.75259218,540.41548148)(906.74259277,540.48548584)
\curveto(906.7325922,540.55548134)(906.73759219,540.62548127)(906.75759277,540.69548584)
\curveto(906.76759216,540.74548115)(906.77259216,540.78548111)(906.77259277,540.81548584)
\curveto(906.78259215,540.85548104)(906.79259214,540.900481)(906.80259277,540.95048584)
\curveto(906.8325921,541.07048083)(906.85759207,541.19048071)(906.87759277,541.31048584)
\curveto(906.90759202,541.43048047)(906.94759198,541.54548035)(906.99759277,541.65548584)
\curveto(907.14759178,542.02547987)(907.3275916,542.35547954)(907.53759277,542.64548584)
\curveto(907.75759117,542.94547895)(908.02259091,543.1954787)(908.33259277,543.39548584)
\curveto(908.45259048,543.47547842)(908.57759035,543.54047836)(908.70759277,543.59048584)
\curveto(908.83759009,543.65047825)(908.97258996,543.71047819)(909.11259277,543.77048584)
\curveto(909.2325897,543.82047808)(909.36258957,543.85047805)(909.50259277,543.86048584)
\curveto(909.64258929,543.88047802)(909.78258915,543.91047799)(909.92259277,543.95048584)
\lineto(910.11759277,543.95048584)
\curveto(910.18758874,543.96047794)(910.25258868,543.97047793)(910.31259277,543.98048584)
\curveto(911.20258773,543.99047791)(911.94258699,543.80547809)(912.53259277,543.42548584)
\curveto(913.12258581,543.04547885)(913.54758538,542.55047935)(913.80759277,541.94048584)
\curveto(913.85758507,541.84048006)(913.89758503,541.74048016)(913.92759277,541.64048584)
\curveto(913.95758497,541.54048036)(913.99258494,541.43548046)(914.03259277,541.32548584)
\curveto(914.06258487,541.21548068)(914.08758484,541.0954808)(914.10759277,540.96548584)
\curveto(914.1275848,540.84548105)(914.15258478,540.72048118)(914.18259277,540.59048584)
\curveto(914.19258474,540.54048136)(914.19258474,540.48548141)(914.18259277,540.42548584)
\curveto(914.18258475,540.37548152)(914.18758474,540.32548157)(914.19759277,540.27548584)
\moveto(912.86259277,539.42048584)
\curveto(912.88258605,539.49048241)(912.88758604,539.57048233)(912.87759277,539.66048584)
\lineto(912.87759277,539.91548584)
\curveto(912.87758605,540.30548159)(912.84258609,540.63548126)(912.77259277,540.90548584)
\curveto(912.74258619,540.98548091)(912.71758621,541.06548083)(912.69759277,541.14548584)
\curveto(912.67758625,541.22548067)(912.65258628,541.3004806)(912.62259277,541.37048584)
\curveto(912.34258659,542.02047988)(911.89758703,542.47047943)(911.28759277,542.72048584)
\curveto(911.21758771,542.75047915)(911.14258779,542.77047913)(911.06259277,542.78048584)
\lineto(910.82259277,542.84048584)
\curveto(910.74258819,542.86047904)(910.65758827,542.87047903)(910.56759277,542.87048584)
\lineto(910.29759277,542.87048584)
\lineto(910.02759277,542.82548584)
\curveto(909.927589,542.80547909)(909.8325891,542.78047912)(909.74259277,542.75048584)
\curveto(909.66258927,542.73047917)(909.58258935,542.7004792)(909.50259277,542.66048584)
\curveto(909.4325895,542.64047926)(909.36758956,542.61047929)(909.30759277,542.57048584)
\curveto(909.24758968,542.53047937)(909.19258974,542.49047941)(909.14259277,542.45048584)
\curveto(908.90259003,542.28047962)(908.70759022,542.07547982)(908.55759277,541.83548584)
\curveto(908.40759052,541.5954803)(908.27759065,541.31548058)(908.16759277,540.99548584)
\curveto(908.13759079,540.895481)(908.11759081,540.79048111)(908.10759277,540.68048584)
\curveto(908.09759083,540.58048132)(908.08259085,540.47548142)(908.06259277,540.36548584)
\curveto(908.05259088,540.32548157)(908.04759088,540.26048164)(908.04759277,540.17048584)
\curveto(908.03759089,540.14048176)(908.0325909,540.10548179)(908.03259277,540.06548584)
\curveto(908.04259089,540.02548187)(908.04759088,539.98048192)(908.04759277,539.93048584)
\lineto(908.04759277,539.63048584)
\curveto(908.04759088,539.53048237)(908.05759087,539.44048246)(908.07759277,539.36048584)
\lineto(908.10759277,539.18048584)
\curveto(908.1275908,539.08048282)(908.14259079,538.98048292)(908.15259277,538.88048584)
\curveto(908.17259076,538.79048311)(908.20259073,538.70548319)(908.24259277,538.62548584)
\curveto(908.34259059,538.38548351)(908.45759047,538.16048374)(908.58759277,537.95048584)
\curveto(908.7275902,537.74048416)(908.89759003,537.56548433)(909.09759277,537.42548584)
\curveto(909.14758978,537.3954845)(909.19258974,537.37048453)(909.23259277,537.35048584)
\curveto(909.27258966,537.33048457)(909.31758961,537.30548459)(909.36759277,537.27548584)
\curveto(909.44758948,537.22548467)(909.5325894,537.18048472)(909.62259277,537.14048584)
\curveto(909.72258921,537.11048479)(909.8275891,537.08048482)(909.93759277,537.05048584)
\curveto(909.98758894,537.03048487)(910.0325889,537.02048488)(910.07259277,537.02048584)
\curveto(910.12258881,537.03048487)(910.17258876,537.03048487)(910.22259277,537.02048584)
\curveto(910.25258868,537.01048489)(910.31258862,537.0004849)(910.40259277,536.99048584)
\curveto(910.50258843,536.98048492)(910.57758835,536.98548491)(910.62759277,537.00548584)
\curveto(910.66758826,537.01548488)(910.70758822,537.01548488)(910.74759277,537.00548584)
\curveto(910.78758814,537.00548489)(910.8275881,537.01548488)(910.86759277,537.03548584)
\curveto(910.94758798,537.05548484)(911.0275879,537.07048483)(911.10759277,537.08048584)
\curveto(911.18758774,537.1004848)(911.26258767,537.12548477)(911.33259277,537.15548584)
\curveto(911.67258726,537.2954846)(911.94758698,537.49048441)(912.15759277,537.74048584)
\curveto(912.36758656,537.99048391)(912.54258639,538.28548361)(912.68259277,538.62548584)
\curveto(912.7325862,538.74548315)(912.76258617,538.87048303)(912.77259277,539.00048584)
\curveto(912.79258614,539.14048276)(912.82258611,539.28048262)(912.86259277,539.42048584)
}
}
{
\newrgbcolor{curcolor}{0 0 0}
\pscustom[linestyle=none,fillstyle=solid,fillcolor=curcolor]
{
\newpath
\moveto(919.33087402,543.98048584)
\curveto(919.56086923,543.98047792)(919.6908691,543.92047798)(919.72087402,543.80048584)
\curveto(919.75086904,543.69047821)(919.76586903,543.52547837)(919.76587402,543.30548584)
\lineto(919.76587402,543.02048584)
\curveto(919.76586903,542.93047897)(919.74086905,542.85547904)(919.69087402,542.79548584)
\curveto(919.63086916,542.71547918)(919.54586925,542.67047923)(919.43587402,542.66048584)
\curveto(919.32586947,542.66047924)(919.21586958,542.64547925)(919.10587402,542.61548584)
\curveto(918.96586983,542.58547931)(918.83086996,542.55547934)(918.70087402,542.52548584)
\curveto(918.58087021,542.4954794)(918.46587033,542.45547944)(918.35587402,542.40548584)
\curveto(918.06587073,542.27547962)(917.83087096,542.0954798)(917.65087402,541.86548584)
\curveto(917.47087132,541.64548025)(917.31587148,541.39048051)(917.18587402,541.10048584)
\curveto(917.14587165,540.99048091)(917.11587168,540.87548102)(917.09587402,540.75548584)
\curveto(917.07587172,540.64548125)(917.05087174,540.53048137)(917.02087402,540.41048584)
\curveto(917.01087178,540.36048154)(917.00587179,540.31048159)(917.00587402,540.26048584)
\curveto(917.01587178,540.21048169)(917.01587178,540.16048174)(917.00587402,540.11048584)
\curveto(916.97587182,539.99048191)(916.96087183,539.85048205)(916.96087402,539.69048584)
\curveto(916.97087182,539.54048236)(916.97587182,539.3954825)(916.97587402,539.25548584)
\lineto(916.97587402,537.41048584)
\lineto(916.97587402,537.06548584)
\curveto(916.97587182,536.94548495)(916.97087182,536.83048507)(916.96087402,536.72048584)
\curveto(916.95087184,536.61048529)(916.94587185,536.51548538)(916.94587402,536.43548584)
\curveto(916.95587184,536.35548554)(916.93587186,536.28548561)(916.88587402,536.22548584)
\curveto(916.83587196,536.15548574)(916.75587204,536.11548578)(916.64587402,536.10548584)
\curveto(916.54587225,536.0954858)(916.43587236,536.09048581)(916.31587402,536.09048584)
\lineto(916.04587402,536.09048584)
\curveto(915.9958728,536.11048579)(915.94587285,536.12548577)(915.89587402,536.13548584)
\curveto(915.85587294,536.15548574)(915.82587297,536.18048572)(915.80587402,536.21048584)
\curveto(915.75587304,536.28048562)(915.72587307,536.36548553)(915.71587402,536.46548584)
\lineto(915.71587402,536.79548584)
\lineto(915.71587402,537.95048584)
\lineto(915.71587402,542.10548584)
\lineto(915.71587402,543.14048584)
\lineto(915.71587402,543.44048584)
\curveto(915.72587307,543.54047836)(915.75587304,543.62547827)(915.80587402,543.69548584)
\curveto(915.83587296,543.73547816)(915.88587291,543.76547813)(915.95587402,543.78548584)
\curveto(916.03587276,543.80547809)(916.12087267,543.81547808)(916.21087402,543.81548584)
\curveto(916.30087249,543.82547807)(916.3908724,543.82547807)(916.48087402,543.81548584)
\curveto(916.57087222,543.80547809)(916.64087215,543.79047811)(916.69087402,543.77048584)
\curveto(916.77087202,543.74047816)(916.82087197,543.68047822)(916.84087402,543.59048584)
\curveto(916.87087192,543.51047839)(916.88587191,543.42047848)(916.88587402,543.32048584)
\lineto(916.88587402,543.02048584)
\curveto(916.88587191,542.92047898)(916.90587189,542.83047907)(916.94587402,542.75048584)
\curveto(916.95587184,542.73047917)(916.96587183,542.71547918)(916.97587402,542.70548584)
\lineto(917.02087402,542.66048584)
\curveto(917.13087166,542.66047924)(917.22087157,542.70547919)(917.29087402,542.79548584)
\curveto(917.36087143,542.895479)(917.42087137,542.97547892)(917.47087402,543.03548584)
\lineto(917.56087402,543.12548584)
\curveto(917.65087114,543.23547866)(917.77587102,543.35047855)(917.93587402,543.47048584)
\curveto(918.0958707,543.59047831)(918.24587055,543.68047822)(918.38587402,543.74048584)
\curveto(918.47587032,543.79047811)(918.57087022,543.82547807)(918.67087402,543.84548584)
\curveto(918.77087002,543.87547802)(918.87586992,543.90547799)(918.98587402,543.93548584)
\curveto(919.04586975,543.94547795)(919.10586969,543.95047795)(919.16587402,543.95048584)
\curveto(919.22586957,543.96047794)(919.28086951,543.97047793)(919.33087402,543.98048584)
}
}
{
\newrgbcolor{curcolor}{0.3019608 0.3019608 0.3019608}
\pscustom[linestyle=none,fillstyle=solid,fillcolor=curcolor]
{
\newpath
\moveto(807.09008789,546.7857666)
\lineto(822.09008789,546.7857666)
\lineto(822.09008789,531.7857666)
\lineto(807.09008789,531.7857666)
\closepath
}
}
{
\newrgbcolor{curcolor}{0.80000001 0.80000001 0.80000001}
\pscustom[linestyle=none,fillstyle=solid,fillcolor=curcolor]
{
\newpath
\moveto(311.58458959,262.12944353)
\curveto(371.50420255,207.36870896)(375.68629978,114.40191872)(320.92556521,54.48230576)
\curveto(316.04576714,49.14279354)(310.77845274,44.17080259)(305.16656305,39.60686635)
\lineto(212.43150851,153.63538684)
\closepath
}
}
{
\newrgbcolor{curcolor}{0.90196079 0.90196079 0.90196079}
\pscustom[linestyle=none,fillstyle=solid,fillcolor=curcolor]
{
\newpath
\moveto(212.43151139,300.61257495)
\curveto(248.87119448,300.61257423)(284.01165456,287.07611158)(311.03435357,262.6297572)
\lineto(212.43150851,153.63538684)
\closepath
}
}
{
\newrgbcolor{curcolor}{0.7019608 0.7019608 0.7019608}
\pscustom[linestyle=none,fillstyle=solid,fillcolor=curcolor]
{
\newpath
\moveto(305.13917561,39.58459865)
\curveto(291.6212243,28.59635605)(276.25745495,20.09880433)(259.76497899,14.48857086)
\lineto(212.43150851,153.63538684)
\closepath
}
}
{
\newrgbcolor{curcolor}{0.60000002 0.60000002 0.60000002}
\pscustom[linestyle=none,fillstyle=solid,fillcolor=curcolor]
{
\newpath
\moveto(259.77296594,14.49128803)
\curveto(251.46271509,11.66386387)(242.91587446,9.58660394)(234.23492397,8.28441539)
\lineto(212.43150851,153.63538684)
\closepath
}
}
{
\newrgbcolor{curcolor}{0.50196081 0.50196081 0.50196081}
\pscustom[linestyle=none,fillstyle=solid,fillcolor=curcolor]
{
\newpath
\moveto(234.15025333,8.27173956)
\curveto(226.96053809,7.19752608)(219.70103374,6.65819892)(212.43151245,6.65819873)
\lineto(212.43150851,153.63538684)
\closepath
}
}
{
\newrgbcolor{curcolor}{0.40000001 0.40000001 0.40000001}
\pscustom[linestyle=none,fillstyle=solid,fillcolor=curcolor]
{
\newpath
\moveto(212.43151245,6.65819873)
\curveto(131.25825288,6.65819655)(65.45432258,72.46212333)(65.4543204,153.6353829)
\curveto(65.45431867,218.30317971)(107.72207088,275.37343118)(169.58042957,294.22727443)
\lineto(212.43150851,153.63538684)
\closepath
}
}
{
\newrgbcolor{curcolor}{0.3019608 0.3019608 0.3019608}
\pscustom[linestyle=none,fillstyle=solid,fillcolor=curcolor]
{
\newpath
\moveto(169.55326802,294.21899298)
\curveto(183.45140551,298.4579487)(197.90130247,300.61257523)(212.43151139,300.61257495)
\lineto(212.43150851,153.63538684)
\closepath
}
}
{
\newrgbcolor{curcolor}{0.80000001 0.80000001 0.80000001}
\pscustom[linestyle=none,fillstyle=solid,fillcolor=curcolor]
{
\newpath
\moveto(598.10370909,661.49475495)
\curveto(645.92244062,661.49475401)(690.7519626,638.23185662)(718.28121089,599.13234802)
\lineto(598.10370621,514.51756684)
\closepath
}
}
{
\newrgbcolor{curcolor}{0.40000001 0.40000001 0.40000001}
\pscustom[linestyle=none,fillstyle=solid,fillcolor=curcolor]
{
\newpath
\moveto(718.26806876,599.15101056)
\curveto(765.00982906,532.78606565)(749.10209485,441.09496458)(682.73714994,394.35320428)
\curveto(616.37220503,347.61144399)(524.68110395,363.5191782)(477.93934366,429.88412311)
\curveto(431.19758336,496.24906802)(447.10531757,587.94016909)(513.47026248,634.68192939)
\curveto(530.39891489,646.6050164)(549.65580131,654.81665312)(569.97920149,658.77880924)
\lineto(598.10370621,514.51756684)
\closepath
}
}
{
\newrgbcolor{curcolor}{0.3019608 0.3019608 0.3019608}
\pscustom[linestyle=none,fillstyle=solid,fillcolor=curcolor]
{
\newpath
\moveto(569.93618507,658.7704163)
\curveto(579.21587546,660.5824143)(588.64876091,661.49475495)(598.10370621,661.49475495)
\lineto(598.10370621,514.51756684)
\closepath
}
}
{
\newrgbcolor{curcolor}{0.80000001 0.80000001 0.80000001}
\pscustom[linestyle=none,fillstyle=solid,fillcolor=curcolor]
{
\newpath
\moveto(231.59594117,660.80385429)
\curveto(240.3171494,659.77926945)(248.92899116,657.97517306)(257.3281342,655.41321222)
\lineto(214.44671101,514.83057684)
\closepath
}
}
{
\newrgbcolor{curcolor}{0.90196079 0.90196079 0.90196079}
\pscustom[linestyle=none,fillstyle=solid,fillcolor=curcolor]
{
\newpath
\moveto(214.44671389,661.80776495)
\curveto(220.19516985,661.80776483)(225.9386753,661.47052135)(231.64762923,660.79777259)
\lineto(214.44671101,514.83057684)
\closepath
}
}
{
\newrgbcolor{curcolor}{0.7019608 0.7019608 0.7019608}
\pscustom[linestyle=none,fillstyle=solid,fillcolor=curcolor]
{
\newpath
\moveto(257.17935974,655.45850634)
\curveto(277.37273665,649.32233919)(296.0077747,638.90922824)(311.81705653,624.92739622)
\lineto(214.44671101,514.83057684)
\closepath
}
}
{
\newrgbcolor{curcolor}{0.60000002 0.60000002 0.60000002}
\pscustom[linestyle=none,fillstyle=solid,fillcolor=curcolor]
{
\newpath
\moveto(311.6560461,625.06958535)
\curveto(320.54058307,617.23515411)(328.44544948,608.35541184)(335.19861485,598.62360485)
\lineto(214.44671101,514.83057684)
\closepath
}
}
{
\newrgbcolor{curcolor}{0.50196081 0.50196081 0.50196081}
\pscustom[linestyle=none,fillstyle=solid,fillcolor=curcolor]
{
\newpath
\moveto(334.98459232,598.93118917)
\curveto(381.43207796,532.35995555)(365.11855697,440.74018117)(298.54732335,394.29269553)
\curveto(295.47808907,392.15125685)(292.32818397,390.12782645)(289.10449563,388.22682878)
\lineto(214.44671101,514.83057684)
\closepath
}
}
{
\newrgbcolor{curcolor}{0.40000001 0.40000001 0.40000001}
\pscustom[linestyle=none,fillstyle=solid,fillcolor=curcolor]
{
\newpath
\moveto(289.21923033,388.29455754)
\curveto(219.33531656,346.99883541)(129.20641384,370.17414376)(87.91069171,440.05805752)
\curveto(46.61496959,509.94197128)(69.79027793,600.07087401)(139.6741917,641.36659613)
\curveto(162.32015146,654.74852069)(188.1424291,661.80776546)(214.44671389,661.80776495)
\lineto(214.44671101,514.83057684)
\closepath
}
}
{
\newrgbcolor{curcolor}{0 0 0}
\pscustom[linestyle=none,fillstyle=solid,fillcolor=curcolor]
{
\newpath
\moveto(86.77290161,490.63789062)
\lineto(90.37290161,490.63789062)
\lineto(91.01790161,490.63789062)
\curveto(91.09789508,490.6378802)(91.17289501,490.6328802)(91.24290161,490.62289062)
\curveto(91.31289487,490.62288022)(91.37289481,490.61288023)(91.42290161,490.59289062)
\curveto(91.49289469,490.56288027)(91.54789463,490.50288033)(91.58790161,490.41289063)
\curveto(91.60789457,490.38288045)(91.61789456,490.3428805)(91.61790161,490.29289063)
\lineto(91.61790161,490.15789063)
\curveto(91.62789455,490.04788079)(91.62289456,489.9428809)(91.60290161,489.84289062)
\curveto(91.59289459,489.7428811)(91.55789462,489.67288117)(91.49790161,489.63289062)
\curveto(91.40789477,489.56288127)(91.27289491,489.52788131)(91.09290161,489.52789063)
\curveto(90.91289527,489.5378813)(90.74789543,489.5428813)(90.59790161,489.54289063)
\lineto(88.60290161,489.54289063)
\lineto(88.10790161,489.54289063)
\lineto(87.97290161,489.54289063)
\curveto(87.93289825,489.5428813)(87.89289829,489.5378813)(87.85290161,489.52789063)
\lineto(87.64290161,489.52789063)
\curveto(87.53289865,489.49788134)(87.45289873,489.45788138)(87.40290161,489.40789063)
\curveto(87.35289883,489.36788147)(87.31789886,489.31288152)(87.29790161,489.24289062)
\curveto(87.2778989,489.18288165)(87.26289892,489.11288172)(87.25290161,489.03289063)
\curveto(87.24289894,488.95288189)(87.22289896,488.86288198)(87.19290161,488.76289062)
\curveto(87.14289904,488.56288227)(87.10289908,488.35788248)(87.07290161,488.14789062)
\curveto(87.04289914,487.9378829)(87.00289918,487.73288311)(86.95290161,487.53289063)
\curveto(86.93289925,487.46288338)(86.92289926,487.39288345)(86.92290161,487.32289063)
\curveto(86.92289926,487.26288358)(86.91289927,487.19788364)(86.89290161,487.12789062)
\curveto(86.8828993,487.09788374)(86.87289931,487.05788378)(86.86290161,487.00789062)
\curveto(86.86289932,486.96788387)(86.86789931,486.92788391)(86.87790161,486.88789062)
\curveto(86.89789928,486.837884)(86.92289926,486.79288405)(86.95290161,486.75289062)
\curveto(86.99289919,486.72288412)(87.05289913,486.71788412)(87.13290161,486.73789062)
\curveto(87.19289899,486.75788408)(87.25289893,486.78288405)(87.31290161,486.81289063)
\curveto(87.37289881,486.85288398)(87.43289875,486.88788395)(87.49290161,486.91789063)
\curveto(87.55289863,486.9378839)(87.60289858,486.95288389)(87.64290161,486.96289062)
\curveto(87.83289835,487.04288379)(88.03789814,487.09788374)(88.25790161,487.12789062)
\curveto(88.48789769,487.15788368)(88.71789746,487.16788367)(88.94790161,487.15789063)
\curveto(89.18789699,487.15788368)(89.41789676,487.13288371)(89.63790161,487.08289063)
\curveto(89.85789632,487.04288379)(90.05789612,486.98288385)(90.23790161,486.90289063)
\curveto(90.28789589,486.88288396)(90.33289585,486.86288398)(90.37290161,486.84289062)
\curveto(90.42289576,486.82288401)(90.47289571,486.79788404)(90.52290161,486.76789062)
\curveto(90.87289531,486.55788428)(91.15289503,486.32788451)(91.36290161,486.07789063)
\curveto(91.5828946,485.82788501)(91.7778944,485.50288533)(91.94790161,485.10289062)
\curveto(91.99789418,484.99288585)(92.03289415,484.88288596)(92.05290161,484.77289062)
\curveto(92.07289411,484.66288618)(92.09789408,484.54788629)(92.12790161,484.42789063)
\curveto(92.13789404,484.39788644)(92.14289404,484.35288648)(92.14290161,484.29289063)
\curveto(92.16289402,484.2328866)(92.17289401,484.16288667)(92.17290161,484.08289063)
\curveto(92.17289401,484.01288683)(92.182894,483.94788689)(92.20290161,483.88789062)
\lineto(92.20290161,483.72289062)
\curveto(92.21289397,483.67288717)(92.21789396,483.60288724)(92.21790161,483.51289062)
\curveto(92.21789396,483.42288741)(92.20789397,483.35288748)(92.18790161,483.30289063)
\curveto(92.16789401,483.24288759)(92.16289402,483.18288765)(92.17290161,483.12289062)
\curveto(92.182894,483.07288777)(92.177894,483.02288781)(92.15790161,482.97289062)
\curveto(92.11789406,482.81288803)(92.0828941,482.66288818)(92.05290161,482.52289062)
\curveto(92.02289416,482.38288846)(91.9778942,482.24788859)(91.91790161,482.11789062)
\curveto(91.75789442,481.74788909)(91.53789464,481.41288943)(91.25790161,481.11289062)
\curveto(90.9778952,480.81289003)(90.65789552,480.58289026)(90.29790161,480.42289063)
\curveto(90.12789605,480.3428905)(89.92789625,480.26789057)(89.69790161,480.19789063)
\curveto(89.58789659,480.15789068)(89.47289671,480.13289071)(89.35290161,480.12289062)
\curveto(89.23289695,480.11289073)(89.11289707,480.09289074)(88.99290161,480.06289063)
\curveto(88.94289724,480.04289079)(88.88789729,480.04289079)(88.82790161,480.06289063)
\curveto(88.76789741,480.07289077)(88.70789747,480.06789077)(88.64790161,480.04789063)
\curveto(88.54789763,480.02789081)(88.44789773,480.02789081)(88.34790161,480.04789063)
\lineto(88.21290161,480.04789063)
\curveto(88.16289802,480.06789077)(88.10289808,480.07789076)(88.03290161,480.07789063)
\curveto(87.97289821,480.06789077)(87.91789826,480.07289077)(87.86790161,480.09289062)
\curveto(87.82789835,480.10289073)(87.79289839,480.10789073)(87.76290161,480.10789062)
\curveto(87.73289845,480.10789073)(87.69789848,480.11289073)(87.65790161,480.12289062)
\lineto(87.38790161,480.18289063)
\curveto(87.29789888,480.20289064)(87.21289897,480.2328906)(87.13290161,480.27289062)
\curveto(86.79289939,480.41289043)(86.50289968,480.56789027)(86.26290161,480.73789062)
\curveto(86.02290016,480.91788992)(85.80290038,481.14788969)(85.60290161,481.42789063)
\curveto(85.45290073,481.65788918)(85.33790084,481.89788894)(85.25790161,482.14789062)
\curveto(85.23790094,482.19788864)(85.22790095,482.24288859)(85.22790161,482.28289063)
\curveto(85.22790095,482.33288851)(85.21790096,482.38288846)(85.19790161,482.43289063)
\curveto(85.177901,482.49288834)(85.16290102,482.57288826)(85.15290161,482.67289063)
\curveto(85.15290103,482.77288806)(85.17290101,482.84788799)(85.21290161,482.89789062)
\curveto(85.26290092,482.97788786)(85.34290084,483.02288781)(85.45290161,483.03289063)
\curveto(85.56290062,483.04288779)(85.6779005,483.04788779)(85.79790161,483.04789063)
\lineto(85.96290161,483.04789063)
\curveto(86.02290016,483.04788779)(86.0779001,483.0378878)(86.12790161,483.01789062)
\curveto(86.21789996,482.99788784)(86.28789989,482.95788788)(86.33790161,482.89789062)
\curveto(86.40789977,482.80788803)(86.45289973,482.69788814)(86.47290161,482.56789063)
\curveto(86.50289968,482.44788839)(86.54789963,482.3428885)(86.60790161,482.25289062)
\curveto(86.79789938,481.91288892)(87.05789912,481.64288919)(87.38790161,481.44289063)
\curveto(87.48789869,481.38288946)(87.59289859,481.33288951)(87.70290161,481.29289063)
\curveto(87.82289836,481.26288958)(87.94289824,481.22788961)(88.06290161,481.18789063)
\curveto(88.23289795,481.1378897)(88.43789774,481.11788972)(88.67790161,481.12789062)
\curveto(88.92789725,481.14788969)(89.12789705,481.18288966)(89.27790161,481.23289062)
\curveto(89.64789653,481.35288948)(89.93789624,481.51288932)(90.14790161,481.71289062)
\curveto(90.36789581,481.92288892)(90.54789563,482.20288864)(90.68790161,482.55289063)
\curveto(90.73789544,482.65288819)(90.76789541,482.75788808)(90.77790161,482.86789062)
\curveto(90.79789538,482.97788786)(90.82289536,483.09288774)(90.85290161,483.21289062)
\lineto(90.85290161,483.31789063)
\curveto(90.86289532,483.35788748)(90.86789531,483.39788744)(90.86790161,483.43789063)
\curveto(90.8778953,483.46788737)(90.8778953,483.50288733)(90.86790161,483.54289063)
\lineto(90.86790161,483.66289063)
\curveto(90.86789531,483.92288692)(90.83789534,484.16788667)(90.77790161,484.39789062)
\curveto(90.66789551,484.74788609)(90.51289567,485.04288579)(90.31290161,485.28289063)
\curveto(90.11289607,485.53288531)(89.85289633,485.72788511)(89.53290161,485.86789062)
\lineto(89.35290161,485.92789063)
\curveto(89.30289688,485.94788489)(89.24289694,485.96788487)(89.17290161,485.98789062)
\curveto(89.12289706,486.00788483)(89.06289712,486.01788482)(88.99290161,486.01789062)
\curveto(88.93289725,486.02788481)(88.86789731,486.04288479)(88.79790161,486.06289063)
\lineto(88.64790161,486.06289063)
\curveto(88.60789757,486.08288476)(88.55289763,486.09288474)(88.48290161,486.09289062)
\curveto(88.42289776,486.09288474)(88.36789781,486.08288476)(88.31790161,486.06289063)
\lineto(88.21290161,486.06289063)
\curveto(88.182898,486.06288478)(88.14789803,486.05788478)(88.10790161,486.04789063)
\lineto(87.86790161,485.98789062)
\curveto(87.78789839,485.97788486)(87.70789847,485.95788488)(87.62790161,485.92789063)
\curveto(87.38789879,485.82788501)(87.15789902,485.69288514)(86.93790161,485.52289062)
\curveto(86.84789933,485.45288538)(86.76289942,485.37788546)(86.68290161,485.29789063)
\curveto(86.60289958,485.22788561)(86.50289968,485.17288566)(86.38290161,485.13289062)
\curveto(86.29289989,485.10288573)(86.15290003,485.09288574)(85.96290161,485.10289062)
\curveto(85.7829004,485.11288572)(85.66290052,485.1378857)(85.60290161,485.17789063)
\curveto(85.55290063,485.21788562)(85.51290067,485.27788556)(85.48290161,485.35789062)
\curveto(85.46290072,485.4378854)(85.46290072,485.52288532)(85.48290161,485.61289062)
\curveto(85.51290067,485.73288511)(85.53290065,485.85288498)(85.54290161,485.97289062)
\curveto(85.56290062,486.10288473)(85.58790059,486.22788461)(85.61790161,486.34789062)
\curveto(85.63790054,486.38788445)(85.64290054,486.42288441)(85.63290161,486.45289063)
\curveto(85.63290055,486.49288434)(85.64290054,486.5378843)(85.66290161,486.58789062)
\curveto(85.6829005,486.67788416)(85.69790048,486.76788407)(85.70790161,486.85789062)
\curveto(85.71790046,486.95788388)(85.73790044,487.05288378)(85.76790161,487.14289062)
\curveto(85.7779004,487.20288364)(85.7829004,487.26288358)(85.78290161,487.32289063)
\curveto(85.79290039,487.38288345)(85.80790037,487.44288339)(85.82790161,487.50289062)
\curveto(85.8779003,487.70288313)(85.91290027,487.90788293)(85.93290161,488.11789062)
\curveto(85.96290022,488.3378825)(86.00290018,488.54788229)(86.05290161,488.74789062)
\curveto(86.0829001,488.84788199)(86.10290008,488.94788189)(86.11290161,489.04789063)
\curveto(86.12290006,489.14788169)(86.13790004,489.24788159)(86.15790161,489.34789062)
\curveto(86.16790001,489.37788146)(86.17290001,489.41788142)(86.17290161,489.46789062)
\curveto(86.20289998,489.57788126)(86.22289996,489.68288116)(86.23290161,489.78289063)
\curveto(86.25289993,489.89288095)(86.2778999,490.00288084)(86.30790161,490.11289062)
\curveto(86.32789985,490.19288064)(86.34289984,490.26288058)(86.35290161,490.32289063)
\curveto(86.36289982,490.39288044)(86.38789979,490.45288038)(86.42790161,490.50289062)
\curveto(86.44789973,490.53288031)(86.4778997,490.55288029)(86.51790161,490.56289063)
\curveto(86.55789962,490.58288025)(86.60289958,490.60288024)(86.65290161,490.62289062)
\curveto(86.71289947,490.62288022)(86.75289943,490.62788021)(86.77290161,490.63789062)
}
}
{
\newrgbcolor{curcolor}{0 0 0}
\pscustom[linestyle=none,fillstyle=solid,fillcolor=curcolor]
{
\newpath
\moveto(100.62751099,483.30289063)
\curveto(100.63750327,483.26288758)(100.63750327,483.21288763)(100.62751099,483.15289063)
\curveto(100.62750328,483.09288774)(100.62250328,483.04288779)(100.61251099,483.00289062)
\curveto(100.61250329,482.96288787)(100.6075033,482.92288792)(100.59751099,482.88289062)
\lineto(100.59751099,482.77789063)
\curveto(100.57750333,482.69788814)(100.56250334,482.61788822)(100.55251099,482.53789063)
\curveto(100.54250336,482.45788838)(100.52250338,482.38288846)(100.49251099,482.31289063)
\curveto(100.47250343,482.2328886)(100.45250345,482.15788868)(100.43251099,482.08789062)
\curveto(100.41250349,482.01788882)(100.38250352,481.9428889)(100.34251099,481.86289062)
\curveto(100.16250374,481.44288939)(99.907504,481.10288973)(99.57751099,480.84289062)
\curveto(99.24750466,480.58289026)(98.85750505,480.37789046)(98.40751099,480.22789062)
\curveto(98.28750562,480.18789065)(98.16250574,480.16289067)(98.03251099,480.15289063)
\curveto(97.91250599,480.13289071)(97.78750612,480.10789073)(97.65751099,480.07789063)
\curveto(97.59750631,480.06789077)(97.53250637,480.06289078)(97.46251099,480.06289063)
\curveto(97.4025065,480.06289078)(97.33750657,480.05789078)(97.26751099,480.04789063)
\lineto(97.14751099,480.04789063)
\lineto(96.95251099,480.04789063)
\curveto(96.89250701,480.0378908)(96.83750707,480.04289079)(96.78751099,480.06289063)
\curveto(96.71750719,480.08289075)(96.65250725,480.08789075)(96.59251099,480.07789063)
\curveto(96.53250737,480.06789077)(96.47250743,480.07289077)(96.41251099,480.09289062)
\curveto(96.36250754,480.10289073)(96.31750759,480.10789073)(96.27751099,480.10789062)
\curveto(96.23750767,480.10789073)(96.19250771,480.11789072)(96.14251099,480.13789062)
\curveto(96.06250784,480.15789068)(95.98750792,480.17789066)(95.91751099,480.19789063)
\curveto(95.84750806,480.20789063)(95.77750813,480.22289061)(95.70751099,480.24289062)
\curveto(95.22750868,480.41289043)(94.82750908,480.62289021)(94.50751099,480.87289062)
\curveto(94.19750971,481.13288971)(93.94750996,481.48788935)(93.75751099,481.93789063)
\curveto(93.72751018,481.99788884)(93.7025102,482.05788878)(93.68251099,482.11789062)
\curveto(93.67251023,482.18788865)(93.65751025,482.26288858)(93.63751099,482.34289062)
\curveto(93.61751029,482.40288844)(93.6025103,482.46788837)(93.59251099,482.53789063)
\curveto(93.58251032,482.60788823)(93.56751034,482.67788816)(93.54751099,482.74789062)
\curveto(93.53751037,482.79788804)(93.53251037,482.837888)(93.53251099,482.86789062)
\lineto(93.53251099,482.98789062)
\curveto(93.52251038,483.02788781)(93.51251039,483.07788776)(93.50251099,483.13789062)
\curveto(93.5025104,483.19788764)(93.5075104,483.24788759)(93.51751099,483.28789063)
\lineto(93.51751099,483.42289063)
\curveto(93.52751038,483.47288737)(93.53251037,483.52288732)(93.53251099,483.57289063)
\curveto(93.55251035,483.67288717)(93.56751034,483.76788707)(93.57751099,483.85789062)
\curveto(93.58751032,483.95788688)(93.6075103,484.05288679)(93.63751099,484.14289062)
\curveto(93.68751022,484.29288654)(93.74251016,484.4328864)(93.80251099,484.56289063)
\curveto(93.86251004,484.69288614)(93.93250997,484.81288603)(94.01251099,484.92289063)
\curveto(94.04250986,484.97288586)(94.07250983,485.01288583)(94.10251099,485.04289063)
\curveto(94.14250976,485.07288577)(94.17750973,485.10788573)(94.20751099,485.14789062)
\curveto(94.26750964,485.22788561)(94.33750957,485.29788554)(94.41751099,485.35789062)
\curveto(94.47750943,485.40788543)(94.53750937,485.45288538)(94.59751099,485.49289062)
\lineto(94.80751099,485.64289062)
\curveto(94.85750905,485.68288516)(94.907509,485.71788512)(94.95751099,485.74789062)
\curveto(95.0075089,485.78788505)(95.04250886,485.84288499)(95.06251099,485.91289063)
\curveto(95.06250884,485.9428849)(95.05250885,485.96788487)(95.03251099,485.98789062)
\curveto(95.02250888,486.01788482)(95.01250889,486.04288479)(95.00251099,486.06289063)
\curveto(94.96250894,486.11288472)(94.91250899,486.15788468)(94.85251099,486.19789063)
\curveto(94.8025091,486.24788459)(94.75250915,486.29288454)(94.70251099,486.33289063)
\curveto(94.66250924,486.36288447)(94.61250929,486.41788442)(94.55251099,486.49789062)
\curveto(94.53250937,486.52788431)(94.5025094,486.55288429)(94.46251099,486.57289063)
\curveto(94.43250947,486.60288424)(94.4075095,486.6378842)(94.38751099,486.67789063)
\curveto(94.21750969,486.88788395)(94.08750982,487.13288371)(93.99751099,487.41289063)
\curveto(93.97750993,487.49288334)(93.96250994,487.57288326)(93.95251099,487.65289063)
\curveto(93.94250996,487.73288311)(93.92750998,487.81288303)(93.90751099,487.89289062)
\curveto(93.88751002,487.9428829)(93.87751003,488.00788283)(93.87751099,488.08789062)
\curveto(93.87751003,488.17788266)(93.88751002,488.24788259)(93.90751099,488.29789063)
\curveto(93.90751,488.39788244)(93.91250999,488.46788237)(93.92251099,488.50789062)
\curveto(93.94250996,488.58788225)(93.95750995,488.65788218)(93.96751099,488.71789062)
\curveto(93.97750993,488.78788205)(93.99250991,488.85788198)(94.01251099,488.92789063)
\curveto(94.16250974,489.35788148)(94.37750953,489.70288113)(94.65751099,489.96289062)
\curveto(94.94750896,490.22288062)(95.29750861,490.4378804)(95.70751099,490.60789062)
\curveto(95.81750809,490.65788018)(95.93250797,490.68788015)(96.05251099,490.69789063)
\curveto(96.18250772,490.71788012)(96.31250759,490.74788009)(96.44251099,490.78789063)
\curveto(96.52250738,490.78788005)(96.59250731,490.78788005)(96.65251099,490.78789063)
\curveto(96.72250718,490.79788004)(96.79750711,490.80788003)(96.87751099,490.81789063)
\curveto(97.66750624,490.83788)(98.32250558,490.70788013)(98.84251099,490.42789063)
\curveto(99.37250453,490.14788069)(99.75250415,489.7378811)(99.98251099,489.19789063)
\curveto(100.09250381,488.96788187)(100.16250374,488.68288216)(100.19251099,488.34289062)
\curveto(100.23250367,488.01288283)(100.2025037,487.70788313)(100.10251099,487.42789063)
\curveto(100.06250384,487.29788354)(100.01250389,487.17788366)(99.95251099,487.06789063)
\curveto(99.902504,486.95788388)(99.84250406,486.85288398)(99.77251099,486.75289062)
\curveto(99.75250415,486.71288412)(99.72250418,486.67788416)(99.68251099,486.64789062)
\lineto(99.59251099,486.55789063)
\curveto(99.54250436,486.46788437)(99.48250442,486.40288444)(99.41251099,486.36289062)
\curveto(99.36250454,486.31288452)(99.3075046,486.26288458)(99.24751099,486.21289062)
\curveto(99.19750471,486.17288466)(99.15250475,486.12788471)(99.11251099,486.07789063)
\curveto(99.09250481,486.05788478)(99.07250483,486.0328848)(99.05251099,486.00289062)
\curveto(99.04250486,485.98288485)(99.04250486,485.95788488)(99.05251099,485.92789063)
\curveto(99.06250484,485.87788496)(99.09250481,485.82788501)(99.14251099,485.77789063)
\curveto(99.19250471,485.7378851)(99.24750466,485.69788514)(99.30751099,485.65789063)
\lineto(99.48751099,485.53789063)
\curveto(99.54750436,485.50788533)(99.59750431,485.47788536)(99.63751099,485.44789063)
\curveto(99.96750394,485.20788563)(100.21750369,484.89788594)(100.38751099,484.51789062)
\curveto(100.42750348,484.4378864)(100.45750345,484.35288648)(100.47751099,484.26289062)
\curveto(100.5075034,484.17288666)(100.53250337,484.08288676)(100.55251099,483.99289062)
\curveto(100.56250334,483.9428869)(100.57250333,483.88788695)(100.58251099,483.82789063)
\lineto(100.61251099,483.67789063)
\curveto(100.62250328,483.61788722)(100.62250328,483.55288728)(100.61251099,483.48289062)
\curveto(100.6025033,483.42288741)(100.6075033,483.36288747)(100.62751099,483.30289063)
\moveto(95.24251099,488.34289062)
\curveto(95.21250869,488.2328826)(95.2075087,488.09288274)(95.22751099,487.92289063)
\curveto(95.24750866,487.76288307)(95.27250863,487.6378832)(95.30251099,487.54789063)
\curveto(95.41250849,487.22788361)(95.56250834,486.98288385)(95.75251099,486.81289063)
\curveto(95.94250796,486.65288418)(96.2075077,486.52288432)(96.54751099,486.42289063)
\curveto(96.67750723,486.39288445)(96.84250706,486.36788447)(97.04251099,486.34789062)
\curveto(97.24250666,486.3378845)(97.41250649,486.35288449)(97.55251099,486.39289062)
\curveto(97.84250606,486.47288437)(98.08250582,486.58288425)(98.27251099,486.72289062)
\curveto(98.47250543,486.87288397)(98.62750528,487.07288377)(98.73751099,487.32289063)
\curveto(98.75750515,487.37288346)(98.76750514,487.41788342)(98.76751099,487.45789063)
\curveto(98.77750513,487.49788334)(98.79250511,487.5428833)(98.81251099,487.59289062)
\curveto(98.84250506,487.70288313)(98.86250504,487.84288299)(98.87251099,488.01289062)
\curveto(98.88250502,488.18288265)(98.87250503,488.32788251)(98.84251099,488.44789063)
\curveto(98.82250508,488.5378823)(98.79750511,488.62288222)(98.76751099,488.70289063)
\curveto(98.74750516,488.78288205)(98.71250519,488.86288198)(98.66251099,488.94289063)
\curveto(98.49250541,489.21288163)(98.26750564,489.40788143)(97.98751099,489.52789063)
\curveto(97.71750619,489.64788119)(97.35750655,489.70788113)(96.90751099,489.70789063)
\curveto(96.88750702,489.68788115)(96.85750705,489.68288116)(96.81751099,489.69289063)
\curveto(96.77750713,489.70288113)(96.74250716,489.70288113)(96.71251099,489.69289063)
\curveto(96.66250724,489.67288117)(96.6075073,489.65788118)(96.54751099,489.64789062)
\curveto(96.49750741,489.64788119)(96.44750746,489.6378812)(96.39751099,489.61789062)
\curveto(96.15750775,489.52788131)(95.94750796,489.41288143)(95.76751099,489.27289062)
\curveto(95.58750832,489.1428817)(95.44750846,488.96288187)(95.34751099,488.73289062)
\curveto(95.32750858,488.67288217)(95.3075086,488.60788223)(95.28751099,488.53789063)
\curveto(95.27750863,488.47788236)(95.26250864,488.41288243)(95.24251099,488.34289062)
\moveto(99.26251099,482.80789063)
\curveto(99.31250459,482.99788784)(99.31750459,483.20288764)(99.27751099,483.42289063)
\curveto(99.24750466,483.64288719)(99.2025047,483.82288701)(99.14251099,483.96289062)
\curveto(98.97250493,484.33288651)(98.71250519,484.6378862)(98.36251099,484.87789062)
\curveto(98.02250588,485.11788572)(97.58750632,485.2378856)(97.05751099,485.23789062)
\curveto(97.02750688,485.21788562)(96.98750692,485.21288563)(96.93751099,485.22289062)
\curveto(96.88750702,485.24288559)(96.84750706,485.24788559)(96.81751099,485.23789062)
\lineto(96.54751099,485.17789063)
\curveto(96.46750744,485.16788567)(96.38750752,485.15288569)(96.30751099,485.13289062)
\curveto(96.0075079,485.02288581)(95.74250816,484.87788596)(95.51251099,484.69789063)
\curveto(95.29250861,484.51788632)(95.12250878,484.28788655)(95.00251099,484.00789062)
\curveto(94.97250893,483.92788691)(94.94750896,483.84788699)(94.92751099,483.76789062)
\curveto(94.907509,483.68788715)(94.88750902,483.60288724)(94.86751099,483.51289062)
\curveto(94.83750907,483.39288745)(94.82750908,483.24288759)(94.83751099,483.06289063)
\curveto(94.85750905,482.88288796)(94.88250902,482.7428881)(94.91251099,482.64289062)
\curveto(94.93250897,482.59288825)(94.94250896,482.54788829)(94.94251099,482.50789062)
\curveto(94.95250895,482.47788836)(94.96750894,482.4378884)(94.98751099,482.38789062)
\curveto(95.08750882,482.16788867)(95.21750869,481.96788887)(95.37751099,481.78789063)
\curveto(95.54750836,481.60788923)(95.74250816,481.47288937)(95.96251099,481.38289062)
\curveto(96.03250787,481.3428895)(96.12750778,481.30788953)(96.24751099,481.27789063)
\curveto(96.46750744,481.18788965)(96.72250718,481.1428897)(97.01251099,481.14289062)
\lineto(97.29751099,481.14289062)
\curveto(97.39750651,481.16288967)(97.49250641,481.17788966)(97.58251099,481.18789063)
\curveto(97.67250623,481.19788964)(97.76250614,481.21788962)(97.85251099,481.24789062)
\curveto(98.11250579,481.32788951)(98.35250555,481.45788938)(98.57251099,481.63789062)
\curveto(98.8025051,481.82788901)(98.97250493,482.04288879)(99.08251099,482.28289063)
\curveto(99.12250478,482.36288847)(99.15250475,482.44288839)(99.17251099,482.52289062)
\curveto(99.2025047,482.61288823)(99.23250467,482.70788813)(99.26251099,482.80789063)
}
}
{
\newrgbcolor{curcolor}{0 0 0}
\pscustom[linestyle=none,fillstyle=solid,fillcolor=curcolor]
{
\newpath
\moveto(102.91712036,481.86289062)
\lineto(103.21712036,481.86289062)
\curveto(103.3271183,481.87288897)(103.4321182,481.87288897)(103.53212036,481.86289062)
\curveto(103.64211799,481.86288898)(103.74211789,481.85288899)(103.83212036,481.83289063)
\curveto(103.92211771,481.82288901)(103.99211764,481.79788904)(104.04212036,481.75789062)
\curveto(104.06211757,481.7378891)(104.07711755,481.70788913)(104.08712036,481.66789063)
\curveto(104.10711752,481.62788921)(104.1271175,481.58288926)(104.14712036,481.53289063)
\lineto(104.14712036,481.45789063)
\curveto(104.15711747,481.40788943)(104.15711747,481.35288948)(104.14712036,481.29289063)
\lineto(104.14712036,481.14289062)
\lineto(104.14712036,480.66289063)
\curveto(104.14711748,480.49289034)(104.10711752,480.37289046)(104.02712036,480.30289063)
\curveto(103.95711767,480.25289059)(103.86711776,480.22789061)(103.75712036,480.22789062)
\lineto(103.42712036,480.22789062)
\lineto(102.97712036,480.22789062)
\curveto(102.8271188,480.22789061)(102.71211892,480.25789058)(102.63212036,480.31789063)
\curveto(102.59211904,480.34789049)(102.56211907,480.39789044)(102.54212036,480.46789062)
\curveto(102.52211911,480.54789029)(102.50711912,480.6328902)(102.49712036,480.72289062)
\lineto(102.49712036,481.00789062)
\curveto(102.50711912,481.10788973)(102.51211912,481.19288965)(102.51212036,481.26289062)
\lineto(102.51212036,481.45789063)
\curveto(102.51211912,481.51788932)(102.52211911,481.57288926)(102.54212036,481.62289062)
\curveto(102.58211905,481.73288911)(102.65211898,481.80288904)(102.75212036,481.83289063)
\curveto(102.78211885,481.832889)(102.83711879,481.84288899)(102.91712036,481.86289062)
}
}
{
\newrgbcolor{curcolor}{0 0 0}
\pscustom[linestyle=none,fillstyle=solid,fillcolor=curcolor]
{
\newpath
\moveto(106.58227661,490.63789062)
\lineto(111.38227661,490.63789062)
\lineto(112.38727661,490.63789062)
\curveto(112.52726951,490.6378802)(112.64726939,490.62788021)(112.74727661,490.60789062)
\curveto(112.85726918,490.59788024)(112.9372691,490.55288029)(112.98727661,490.47289062)
\curveto(113.00726903,490.4328804)(113.01726902,490.38288045)(113.01727661,490.32289063)
\curveto(113.02726901,490.26288058)(113.03226901,490.19788064)(113.03227661,490.12789062)
\lineto(113.03227661,489.85789062)
\curveto(113.03226901,489.76788107)(113.02226902,489.68788115)(113.00227661,489.61789062)
\curveto(112.96226908,489.5378813)(112.91726912,489.46788137)(112.86727661,489.40789063)
\lineto(112.71727661,489.22789062)
\curveto(112.68726935,489.17788166)(112.65226939,489.1378817)(112.61227661,489.10789062)
\curveto(112.57226947,489.07788176)(112.53226951,489.0378818)(112.49227661,488.98789062)
\curveto(112.41226963,488.87788196)(112.32726971,488.76788207)(112.23727661,488.65789063)
\curveto(112.14726989,488.55788228)(112.06226998,488.45288238)(111.98227661,488.34289062)
\curveto(111.8422702,488.1428827)(111.70227034,487.93288291)(111.56227661,487.71289062)
\curveto(111.42227062,487.50288333)(111.28227076,487.28788355)(111.14227661,487.06789063)
\curveto(111.09227095,486.97788386)(111.042271,486.88288396)(110.99227661,486.78289063)
\curveto(110.9422711,486.68288416)(110.88727115,486.58788425)(110.82727661,486.49789062)
\curveto(110.80727123,486.47788436)(110.79727124,486.45288438)(110.79727661,486.42289063)
\curveto(110.79727124,486.39288445)(110.78727125,486.36788447)(110.76727661,486.34789062)
\curveto(110.69727134,486.24788459)(110.63227141,486.13288471)(110.57227661,486.00289062)
\curveto(110.51227153,485.88288496)(110.45727158,485.76788507)(110.40727661,485.65789063)
\curveto(110.30727173,485.42788541)(110.21227183,485.19288565)(110.12227661,484.95289063)
\curveto(110.03227201,484.71288612)(109.93227211,484.47288637)(109.82227661,484.23289062)
\curveto(109.80227224,484.18288665)(109.78727225,484.1378867)(109.77727661,484.09789062)
\curveto(109.77727226,484.05788678)(109.76727227,484.01288683)(109.74727661,483.96289062)
\curveto(109.69727234,483.84288699)(109.65227239,483.71788712)(109.61227661,483.58789062)
\curveto(109.58227246,483.46788737)(109.54727249,483.34788749)(109.50727661,483.22789062)
\curveto(109.42727261,482.99788784)(109.36227268,482.75788808)(109.31227661,482.50789062)
\curveto(109.27227277,482.26788857)(109.22227282,482.02788881)(109.16227661,481.78789063)
\curveto(109.12227292,481.6378892)(109.09727294,481.48788935)(109.08727661,481.33789062)
\curveto(109.07727296,481.18788965)(109.05727298,481.0378898)(109.02727661,480.88789062)
\curveto(109.01727302,480.84788999)(109.01227303,480.78789005)(109.01227661,480.70789063)
\curveto(108.98227306,480.58789025)(108.95227309,480.48789035)(108.92227661,480.40789063)
\curveto(108.89227315,480.32789051)(108.82227322,480.27289057)(108.71227661,480.24289062)
\curveto(108.66227338,480.22289061)(108.60727343,480.21289062)(108.54727661,480.21289062)
\lineto(108.35227661,480.21289062)
\curveto(108.21227383,480.21289062)(108.07227397,480.21789062)(107.93227661,480.22789062)
\curveto(107.80227424,480.2378906)(107.70727433,480.28289055)(107.64727661,480.36289062)
\curveto(107.60727443,480.42289041)(107.58727445,480.50789033)(107.58727661,480.61789062)
\curveto(107.59727444,480.72789011)(107.61227443,480.82289001)(107.63227661,480.90289063)
\lineto(107.63227661,480.97789062)
\curveto(107.6422744,481.00788983)(107.64727439,481.0378898)(107.64727661,481.06789063)
\curveto(107.66727437,481.14788969)(107.67727436,481.22288961)(107.67727661,481.29289063)
\curveto(107.67727436,481.36288947)(107.68727435,481.4328894)(107.70727661,481.50289062)
\curveto(107.75727428,481.69288914)(107.79727424,481.87788896)(107.82727661,482.05789063)
\curveto(107.85727418,482.24788859)(107.89727414,482.42788841)(107.94727661,482.59789062)
\curveto(107.96727407,482.64788819)(107.97727406,482.68788815)(107.97727661,482.71789062)
\curveto(107.97727406,482.74788809)(107.98227406,482.78288805)(107.99227661,482.82289063)
\curveto(108.09227395,483.12288772)(108.18227386,483.41788742)(108.26227661,483.70789063)
\curveto(108.35227369,483.99788684)(108.45727358,484.27788656)(108.57727661,484.54789063)
\curveto(108.8372732,485.12788571)(109.10727293,485.67788516)(109.38727661,486.19789063)
\curveto(109.66727237,486.72788411)(109.97727206,487.2328836)(110.31727661,487.71289062)
\curveto(110.45727158,487.91288292)(110.60727143,488.10288273)(110.76727661,488.28289063)
\curveto(110.92727111,488.47288237)(111.07727096,488.66288218)(111.21727661,488.85289062)
\curveto(111.25727078,488.90288193)(111.29227075,488.94788189)(111.32227661,488.98789062)
\curveto(111.36227068,489.0378818)(111.39727064,489.08788175)(111.42727661,489.13789062)
\curveto(111.4372706,489.15788168)(111.44727059,489.18288165)(111.45727661,489.21289062)
\curveto(111.47727056,489.24288159)(111.47727056,489.27288157)(111.45727661,489.30289063)
\curveto(111.4372706,489.36288147)(111.40227064,489.39788144)(111.35227661,489.40789063)
\curveto(111.30227074,489.42788141)(111.25227079,489.44788139)(111.20227661,489.46789062)
\lineto(111.09727661,489.46789062)
\curveto(111.05727098,489.47788136)(111.00727103,489.47788136)(110.94727661,489.46789062)
\lineto(110.79727661,489.46789062)
\lineto(110.19727661,489.46789062)
\lineto(107.55727661,489.46789062)
\lineto(106.82227661,489.46789062)
\lineto(106.58227661,489.46789062)
\curveto(106.51227553,489.47788136)(106.45227559,489.49288135)(106.40227661,489.51289062)
\curveto(106.31227573,489.55288129)(106.25227579,489.61288123)(106.22227661,489.69289063)
\curveto(106.17227587,489.79288104)(106.15727588,489.9378809)(106.17727661,490.12789062)
\curveto(106.19727584,490.32788051)(106.23227581,490.46288038)(106.28227661,490.53289063)
\curveto(106.30227574,490.55288029)(106.32727571,490.56788027)(106.35727661,490.57789063)
\lineto(106.47727661,490.63789062)
\curveto(106.49727554,490.6378802)(106.51227553,490.6328802)(106.52227661,490.62289062)
\curveto(106.5422755,490.62288022)(106.56227548,490.62788021)(106.58227661,490.63789062)
}
}
{
\newrgbcolor{curcolor}{0 0 0}
\pscustom[linestyle=none,fillstyle=solid,fillcolor=curcolor]
{
\newpath
\moveto(124.27688599,488.74789062)
\curveto(124.07687569,488.45788238)(123.8668759,488.17288266)(123.64688599,487.89289062)
\curveto(123.43687633,487.61288323)(123.23187653,487.32788351)(123.03188599,487.03789063)
\curveto(122.43187733,486.18788465)(121.82687794,485.34788549)(121.21688599,484.51789062)
\curveto(120.60687916,483.69788714)(120.00187976,482.86288798)(119.40188599,482.01289062)
\lineto(118.89188599,481.29289063)
\lineto(118.38188599,480.60289062)
\curveto(118.30188146,480.49289034)(118.22188154,480.37789046)(118.14188599,480.25789062)
\curveto(118.0618817,480.1378907)(117.9668818,480.04289079)(117.85688599,479.97289062)
\curveto(117.81688195,479.95289088)(117.75188201,479.9378909)(117.66188599,479.92789063)
\curveto(117.58188218,479.90789093)(117.49188227,479.89789094)(117.39188599,479.89789062)
\curveto(117.29188247,479.89789094)(117.19688257,479.90289093)(117.10688599,479.91289063)
\curveto(117.02688274,479.92289092)(116.9668828,479.9428909)(116.92688599,479.97289062)
\curveto(116.89688287,479.99289085)(116.87188289,480.02789081)(116.85188599,480.07789063)
\curveto(116.84188292,480.11789072)(116.84688292,480.16289067)(116.86688599,480.21289062)
\curveto(116.90688286,480.29289054)(116.95188281,480.36789047)(117.00188599,480.43789063)
\curveto(117.0618827,480.51789032)(117.11688265,480.59789024)(117.16688599,480.67789063)
\curveto(117.40688236,481.01788982)(117.65188211,481.35288948)(117.90188599,481.68289063)
\curveto(118.15188161,482.01288883)(118.39188137,482.34788849)(118.62188599,482.68789063)
\curveto(118.78188098,482.90788793)(118.94188082,483.12288772)(119.10188599,483.33289063)
\curveto(119.2618805,483.5428873)(119.42188034,483.75788708)(119.58188599,483.97789062)
\curveto(119.94187982,484.49788634)(120.30687946,485.00788583)(120.67688599,485.50789062)
\curveto(121.04687872,486.00788483)(121.41687835,486.51788432)(121.78688599,487.03789063)
\curveto(121.92687784,487.2378836)(122.0668777,487.4328834)(122.20688599,487.62289062)
\curveto(122.35687741,487.81288303)(122.50187726,488.00788283)(122.64188599,488.20789063)
\curveto(122.85187691,488.50788233)(123.0668767,488.80788203)(123.28688599,489.10789062)
\lineto(123.94688599,490.00789062)
\lineto(124.12688599,490.27789063)
\lineto(124.33688599,490.54789063)
\lineto(124.45688599,490.72789062)
\curveto(124.50687526,490.78788005)(124.55687521,490.84287999)(124.60688599,490.89289062)
\curveto(124.67687509,490.9428799)(124.75187501,490.97787986)(124.83188599,490.99789062)
\curveto(124.85187491,491.00787983)(124.87687489,491.00787983)(124.90688599,490.99789062)
\curveto(124.94687482,490.99787984)(124.97687479,491.00787983)(124.99688599,491.02789063)
\curveto(125.11687465,491.02787981)(125.25187451,491.02287982)(125.40188599,491.01289062)
\curveto(125.55187421,491.01287983)(125.64187412,490.96787987)(125.67188599,490.87789062)
\curveto(125.69187407,490.84787999)(125.69687407,490.81288003)(125.68688599,490.77289062)
\curveto(125.67687409,490.73288011)(125.6618741,490.70288013)(125.64188599,490.68289063)
\curveto(125.60187416,490.60288024)(125.5618742,490.53288031)(125.52188599,490.47289062)
\curveto(125.48187428,490.41288043)(125.43687433,490.35288049)(125.38688599,490.29289063)
\lineto(124.81688599,489.51289062)
\curveto(124.63687513,489.26288158)(124.45687531,489.00788183)(124.27688599,488.74789062)
\moveto(117.42188599,484.84789062)
\curveto(117.37188239,484.86788597)(117.32188244,484.87288597)(117.27188599,484.86289062)
\curveto(117.22188254,484.85288598)(117.17188259,484.85788598)(117.12188599,484.87789062)
\curveto(117.01188275,484.89788594)(116.90688286,484.91788592)(116.80688599,484.93789063)
\curveto(116.71688305,484.96788587)(116.62188314,485.00788583)(116.52188599,485.05789063)
\curveto(116.19188357,485.19788564)(115.93688383,485.39288545)(115.75688599,485.64289062)
\curveto(115.57688419,485.90288493)(115.43188433,486.21288463)(115.32188599,486.57289063)
\curveto(115.29188447,486.65288418)(115.27188449,486.73288411)(115.26188599,486.81289063)
\curveto(115.25188451,486.90288393)(115.23688453,486.98788385)(115.21688599,487.06789063)
\curveto(115.20688456,487.11788372)(115.20188456,487.18288365)(115.20188599,487.26289062)
\curveto(115.19188457,487.29288354)(115.18688458,487.32288352)(115.18688599,487.35289062)
\curveto(115.18688458,487.39288345)(115.18188458,487.42788341)(115.17188599,487.45789063)
\lineto(115.17188599,487.60789062)
\curveto(115.1618846,487.65788318)(115.15688461,487.71788312)(115.15688599,487.78789063)
\curveto(115.15688461,487.86788297)(115.1618846,487.93288291)(115.17188599,487.98289062)
\lineto(115.17188599,488.14789062)
\curveto(115.19188457,488.19788264)(115.19688457,488.24288259)(115.18688599,488.28289063)
\curveto(115.18688458,488.33288251)(115.19188457,488.37788246)(115.20188599,488.41789063)
\curveto(115.21188455,488.45788238)(115.21688455,488.49288234)(115.21688599,488.52289062)
\curveto(115.21688455,488.56288227)(115.22188454,488.60288224)(115.23188599,488.64289062)
\curveto(115.2618845,488.75288209)(115.28188448,488.86288198)(115.29188599,488.97289062)
\curveto(115.31188445,489.09288174)(115.34688442,489.20788163)(115.39688599,489.31789063)
\curveto(115.53688423,489.65788118)(115.69688407,489.93288091)(115.87688599,490.14289062)
\curveto(116.0668837,490.36288047)(116.33688343,490.5428803)(116.68688599,490.68289063)
\curveto(116.766883,490.71288012)(116.85188291,490.73288011)(116.94188599,490.74289062)
\curveto(117.03188273,490.76288007)(117.12688264,490.78288005)(117.22688599,490.80289063)
\curveto(117.25688251,490.81288003)(117.31188245,490.81288003)(117.39188599,490.80289063)
\curveto(117.47188229,490.80288004)(117.52188224,490.81288003)(117.54188599,490.83289063)
\curveto(118.10188166,490.84287999)(118.55188121,490.73288011)(118.89188599,490.50289062)
\curveto(119.24188052,490.27288057)(119.50188026,489.96788087)(119.67188599,489.58789062)
\curveto(119.71188005,489.49788134)(119.74688002,489.40288144)(119.77688599,489.30289063)
\curveto(119.80687996,489.20288164)(119.83187993,489.10288173)(119.85188599,489.00289062)
\curveto(119.87187989,488.97288186)(119.87687989,488.9428819)(119.86688599,488.91289063)
\curveto(119.8668799,488.88288196)(119.87187989,488.85288198)(119.88188599,488.82289063)
\curveto(119.91187985,488.71288212)(119.93187983,488.58788225)(119.94188599,488.44789063)
\curveto(119.95187981,488.31788252)(119.9618798,488.18288265)(119.97188599,488.04289063)
\lineto(119.97188599,487.87789062)
\curveto(119.98187978,487.81788302)(119.98187978,487.76288307)(119.97188599,487.71289062)
\curveto(119.9618798,487.66288318)(119.95687981,487.61288323)(119.95688599,487.56289063)
\lineto(119.95688599,487.42789063)
\curveto(119.94687982,487.38788345)(119.94187982,487.34788349)(119.94188599,487.30789063)
\curveto(119.95187981,487.26788357)(119.94687982,487.22288362)(119.92688599,487.17289063)
\curveto(119.90687986,487.06288378)(119.88687988,486.95788388)(119.86688599,486.85789062)
\curveto(119.85687991,486.75788408)(119.83687993,486.65788418)(119.80688599,486.55789063)
\curveto(119.67688009,486.19788464)(119.51188025,485.88288496)(119.31188599,485.61289062)
\curveto(119.11188065,485.3428855)(118.83688093,485.1378857)(118.48688599,484.99789062)
\curveto(118.40688136,484.96788587)(118.32188144,484.9428859)(118.23188599,484.92289063)
\lineto(117.96188599,484.86289062)
\curveto(117.91188185,484.85288598)(117.8668819,484.84788599)(117.82688599,484.84789062)
\curveto(117.78688198,484.85788598)(117.74688202,484.85788598)(117.70688599,484.84789062)
\curveto(117.60688216,484.82788601)(117.51188225,484.82788601)(117.42188599,484.84789062)
\moveto(116.58188599,486.24289062)
\curveto(116.62188314,486.17288466)(116.6618831,486.10788473)(116.70188599,486.04789063)
\curveto(116.74188302,485.99788484)(116.79188297,485.94788489)(116.85188599,485.89789062)
\lineto(117.00188599,485.77789063)
\curveto(117.0618827,485.74788509)(117.12688264,485.72288512)(117.19688599,485.70289063)
\curveto(117.23688253,485.68288516)(117.27188249,485.67288517)(117.30188599,485.67289063)
\curveto(117.34188242,485.68288516)(117.38188238,485.67788516)(117.42188599,485.65789063)
\curveto(117.45188231,485.65788518)(117.49188227,485.65288518)(117.54188599,485.64289062)
\curveto(117.59188217,485.64288519)(117.63188213,485.64788519)(117.66188599,485.65789063)
\lineto(117.88688599,485.70289063)
\curveto(118.13688163,485.78288505)(118.32188144,485.90788493)(118.44188599,486.07789063)
\curveto(118.52188124,486.17788466)(118.59188117,486.30788453)(118.65188599,486.46789062)
\curveto(118.73188103,486.64788419)(118.79188097,486.87288397)(118.83188599,487.14289062)
\curveto(118.87188089,487.42288342)(118.88688088,487.70288313)(118.87688599,487.98289062)
\curveto(118.8668809,488.27288257)(118.83688093,488.54788229)(118.78688599,488.80789063)
\curveto(118.73688103,489.06788177)(118.6618811,489.27788156)(118.56188599,489.43789063)
\curveto(118.44188132,489.6378812)(118.29188147,489.78788105)(118.11188599,489.88789062)
\curveto(118.03188173,489.9378809)(117.94188182,489.96788087)(117.84188599,489.97789062)
\curveto(117.74188202,489.99788084)(117.63688213,490.00788083)(117.52688599,490.00789062)
\curveto(117.50688226,489.99788084)(117.48188228,489.99288084)(117.45188599,489.99289062)
\curveto(117.43188233,490.00288084)(117.41188235,490.00288084)(117.39188599,489.99289062)
\curveto(117.34188242,489.98288085)(117.29688247,489.97288086)(117.25688599,489.96289062)
\curveto(117.21688255,489.96288087)(117.17688259,489.95288089)(117.13688599,489.93289063)
\curveto(116.95688281,489.85288098)(116.80688296,489.73288111)(116.68688599,489.57289063)
\curveto(116.57688319,489.41288143)(116.48688328,489.2328816)(116.41688599,489.03289063)
\curveto(116.35688341,488.84288199)(116.31188345,488.61788222)(116.28188599,488.35789062)
\curveto(116.2618835,488.09788274)(116.25688351,487.832883)(116.26688599,487.56289063)
\curveto(116.27688349,487.30288353)(116.30688346,487.05288378)(116.35688599,486.81289063)
\curveto(116.41688335,486.58288425)(116.49188327,486.39288445)(116.58188599,486.24289062)
\moveto(127.38188599,483.25789062)
\curveto(127.39187237,483.20788763)(127.39687237,483.11788772)(127.39688599,482.98789062)
\curveto(127.39687237,482.85788798)(127.38687238,482.76788807)(127.36688599,482.71789062)
\curveto(127.34687242,482.66788817)(127.34187242,482.61288823)(127.35188599,482.55289063)
\curveto(127.3618724,482.50288833)(127.3618724,482.45288839)(127.35188599,482.40289063)
\curveto(127.31187245,482.26288858)(127.28187248,482.12788871)(127.26188599,481.99789062)
\curveto(127.25187251,481.86788897)(127.22187254,481.74788909)(127.17188599,481.63789062)
\curveto(127.03187273,481.28788955)(126.8668729,480.99288985)(126.67688599,480.75289062)
\curveto(126.48687328,480.52289032)(126.21687355,480.3378905)(125.86688599,480.19789063)
\curveto(125.78687398,480.16789067)(125.70187406,480.14789069)(125.61188599,480.13789062)
\curveto(125.52187424,480.11789072)(125.43687433,480.09789074)(125.35688599,480.07789063)
\curveto(125.30687446,480.06789077)(125.25687451,480.06289078)(125.20688599,480.06289063)
\curveto(125.15687461,480.06289078)(125.10687466,480.05789078)(125.05688599,480.04789063)
\curveto(125.02687474,480.0378908)(124.97687479,480.0378908)(124.90688599,480.04789063)
\curveto(124.83687493,480.04789079)(124.78687498,480.05289079)(124.75688599,480.06289063)
\curveto(124.69687507,480.08289075)(124.63687513,480.09289074)(124.57688599,480.09289062)
\curveto(124.52687524,480.08289075)(124.47687529,480.08789075)(124.42688599,480.10789062)
\curveto(124.33687543,480.12789071)(124.24687552,480.15289068)(124.15688599,480.18289063)
\curveto(124.07687569,480.20289064)(123.99687577,480.2328906)(123.91688599,480.27289062)
\curveto(123.59687617,480.41289043)(123.34687642,480.60789023)(123.16688599,480.85789062)
\curveto(122.98687678,481.11788972)(122.83687693,481.42288941)(122.71688599,481.77289062)
\curveto(122.69687707,481.85288899)(122.68187708,481.9378889)(122.67188599,482.02789063)
\curveto(122.6618771,482.11788872)(122.64687712,482.20288864)(122.62688599,482.28289063)
\curveto(122.61687715,482.31288852)(122.61187715,482.3428885)(122.61188599,482.37289062)
\lineto(122.61188599,482.47789062)
\curveto(122.59187717,482.55788828)(122.58187718,482.6378882)(122.58188599,482.71789062)
\lineto(122.58188599,482.85289062)
\curveto(122.5618772,482.95288788)(122.5618772,483.05288779)(122.58188599,483.15289063)
\lineto(122.58188599,483.33289063)
\curveto(122.59187717,483.38288745)(122.59687717,483.42788741)(122.59688599,483.46789062)
\curveto(122.59687717,483.51788732)(122.60187716,483.56288727)(122.61188599,483.60289062)
\curveto(122.62187714,483.64288719)(122.62687714,483.67788716)(122.62688599,483.70789063)
\curveto(122.62687714,483.74788709)(122.63187713,483.78788705)(122.64188599,483.82789063)
\lineto(122.70188599,484.15789063)
\curveto(122.72187704,484.27788656)(122.75187701,484.38788645)(122.79188599,484.48789062)
\curveto(122.93187683,484.81788602)(123.09187667,485.09288574)(123.27188599,485.31289063)
\curveto(123.4618763,485.5428853)(123.72187604,485.72788511)(124.05188599,485.86789062)
\curveto(124.13187563,485.90788493)(124.21687555,485.93288491)(124.30688599,485.94289063)
\lineto(124.60688599,486.00289062)
\lineto(124.74188599,486.00289062)
\curveto(124.79187497,486.01288483)(124.84187492,486.01788482)(124.89188599,486.01789062)
\curveto(125.4618743,486.0378848)(125.92187384,485.93288491)(126.27188599,485.70289063)
\curveto(126.63187313,485.48288536)(126.89687287,485.18288565)(127.06688599,484.80289063)
\curveto(127.11687265,484.70288613)(127.15687261,484.60288624)(127.18688599,484.50289062)
\curveto(127.21687255,484.40288644)(127.24687252,484.29788654)(127.27688599,484.18789063)
\curveto(127.28687248,484.14788669)(127.29187247,484.11288672)(127.29188599,484.08289063)
\curveto(127.29187247,484.06288678)(127.29687247,484.0328868)(127.30688599,483.99289062)
\curveto(127.32687244,483.92288692)(127.33687243,483.84788699)(127.33688599,483.76789062)
\curveto(127.33687243,483.68788715)(127.34687242,483.60788723)(127.36688599,483.52789063)
\curveto(127.3668724,483.47788736)(127.3668724,483.4328874)(127.36688599,483.39289062)
\curveto(127.3668724,483.35288748)(127.37187239,483.30788753)(127.38188599,483.25789062)
\moveto(126.27188599,482.82289063)
\curveto(126.28187348,482.87288797)(126.28687348,482.94788789)(126.28688599,483.04789063)
\curveto(126.29687347,483.14788769)(126.29187347,483.22288761)(126.27188599,483.27289062)
\curveto(126.25187351,483.33288751)(126.24687352,483.38788745)(126.25688599,483.43789063)
\curveto(126.27687349,483.49788734)(126.27687349,483.55788728)(126.25688599,483.61789062)
\curveto(126.24687352,483.64788719)(126.24187352,483.68288716)(126.24188599,483.72289062)
\curveto(126.24187352,483.76288707)(126.23687353,483.80288704)(126.22688599,483.84289062)
\curveto(126.20687356,483.92288692)(126.18687358,483.99788684)(126.16688599,484.06789063)
\curveto(126.15687361,484.14788669)(126.14187362,484.22788661)(126.12188599,484.30789063)
\curveto(126.09187367,484.36788647)(126.0668737,484.42788641)(126.04688599,484.48789062)
\curveto(126.02687374,484.54788629)(125.99687377,484.60788623)(125.95688599,484.66789063)
\curveto(125.85687391,484.837886)(125.72687404,484.97288586)(125.56688599,485.07289063)
\curveto(125.48687428,485.12288572)(125.39187437,485.15788568)(125.28188599,485.17789063)
\curveto(125.17187459,485.19788564)(125.04687472,485.20788563)(124.90688599,485.20789063)
\curveto(124.88687488,485.19788564)(124.8618749,485.19288565)(124.83188599,485.19289063)
\curveto(124.80187496,485.20288564)(124.77187499,485.20288564)(124.74188599,485.19289063)
\lineto(124.59188599,485.13289062)
\curveto(124.54187522,485.12288572)(124.49687527,485.10788573)(124.45688599,485.08789062)
\curveto(124.2668755,484.97788586)(124.12187564,484.832886)(124.02188599,484.65289063)
\curveto(123.93187583,484.47288637)(123.85187591,484.26788657)(123.78188599,484.03789063)
\curveto(123.74187602,483.90788693)(123.72187604,483.77288706)(123.72188599,483.63289062)
\curveto(123.72187604,483.50288733)(123.71187605,483.35788748)(123.69188599,483.19789063)
\curveto(123.68187608,483.14788769)(123.67187609,483.08788775)(123.66188599,483.01789062)
\curveto(123.6618761,482.94788789)(123.67187609,482.88788795)(123.69188599,482.83789062)
\lineto(123.69188599,482.67289063)
\lineto(123.69188599,482.49289062)
\curveto(123.70187606,482.44288839)(123.71187605,482.38788845)(123.72188599,482.32789063)
\curveto(123.73187603,482.27788856)(123.73687603,482.22288861)(123.73688599,482.16289063)
\curveto(123.74687602,482.10288873)(123.761876,482.04788879)(123.78188599,481.99789062)
\curveto(123.83187593,481.80788903)(123.89187587,481.6328892)(123.96188599,481.47289062)
\curveto(124.03187573,481.31288952)(124.13687563,481.18288966)(124.27688599,481.08289063)
\curveto(124.40687536,480.98288986)(124.54687522,480.91288992)(124.69688599,480.87289062)
\curveto(124.72687504,480.86288998)(124.75187501,480.85788998)(124.77188599,480.85789062)
\curveto(124.80187496,480.86788997)(124.83187493,480.86788997)(124.86188599,480.85789062)
\curveto(124.88187488,480.85788998)(124.91187485,480.85288999)(124.95188599,480.84289062)
\curveto(124.99187477,480.84288999)(125.02687474,480.84788999)(125.05688599,480.85789062)
\curveto(125.09687467,480.86788997)(125.13687463,480.87288997)(125.17688599,480.87289062)
\curveto(125.21687455,480.87288997)(125.25687451,480.88288995)(125.29688599,480.90289063)
\curveto(125.53687423,480.98288986)(125.73187403,481.11788972)(125.88188599,481.30789063)
\curveto(126.00187376,481.48788935)(126.09187367,481.69288914)(126.15188599,481.92289063)
\curveto(126.17187359,481.99288885)(126.18687358,482.06288878)(126.19688599,482.13289062)
\curveto(126.20687356,482.21288863)(126.22187354,482.29288854)(126.24188599,482.37289062)
\curveto(126.24187352,482.4328884)(126.24687352,482.47788836)(126.25688599,482.50789062)
\curveto(126.25687351,482.52788831)(126.25687351,482.55288828)(126.25688599,482.58289063)
\curveto(126.25687351,482.62288821)(126.2618735,482.65288819)(126.27188599,482.67289063)
\lineto(126.27188599,482.82289063)
}
}
{
\newrgbcolor{curcolor}{0 0 0}
\pscustom[linestyle=none,fillstyle=solid,fillcolor=curcolor]
{
\newpath
\moveto(400.43525391,699.52799194)
\lineto(401.72525391,699.52799194)
\curveto(401.83525108,699.52798126)(401.94025098,699.52298127)(402.04025391,699.51299194)
\curveto(402.14025078,699.51298128)(402.2152507,699.47798131)(402.26525391,699.40799194)
\curveto(402.3152506,699.33798145)(402.34025058,699.24798154)(402.34025391,699.13799194)
\curveto(402.35025057,699.02798176)(402.35525056,698.90798188)(402.35525391,698.77799194)
\lineto(402.35525391,697.47299194)
\lineto(402.35525391,692.26799194)
\lineto(402.35525391,689.80799194)
\lineto(402.35525391,689.37299194)
\curveto(402.36525055,689.21299158)(402.34525057,689.0929917)(402.29525391,689.01299194)
\curveto(402.25525066,688.94299185)(402.16525075,688.8879919)(402.02525391,688.84799194)
\curveto(401.95525096,688.82799196)(401.88025104,688.82299197)(401.80025391,688.83299194)
\curveto(401.7202512,688.84299195)(401.64025128,688.84799194)(401.56025391,688.84799194)
\lineto(400.67525391,688.84799194)
\curveto(400.56525235,688.84799194)(400.46025246,688.85299194)(400.36025391,688.86299194)
\curveto(400.27025265,688.87299192)(400.19525272,688.90299189)(400.13525391,688.95299194)
\curveto(400.08525283,689.00299179)(400.05525286,689.07799171)(400.04525391,689.17799194)
\curveto(400.03525288,689.27799151)(400.03025289,689.38299141)(400.03025391,689.49299194)
\lineto(400.03025391,690.79799194)
\lineto(400.03025391,696.27299194)
\lineto(400.03025391,698.46299194)
\curveto(400.03025289,698.60298219)(400.02525289,698.76798202)(400.01525391,698.95799194)
\curveto(400.0152529,699.14798164)(400.04025288,699.28298151)(400.09025391,699.36299194)
\curveto(400.13025279,699.42298137)(400.19525272,699.47298132)(400.28525391,699.51299194)
\curveto(400.3152526,699.51298128)(400.34025258,699.51298128)(400.36025391,699.51299194)
\curveto(400.39025253,699.52298127)(400.4152525,699.52798126)(400.43525391,699.52799194)
}
}
{
\newrgbcolor{curcolor}{0 0 0}
\pscustom[linestyle=none,fillstyle=solid,fillcolor=curcolor]
{
\newpath
\moveto(408.63908203,696.78299194)
\curveto(409.23907623,696.80298399)(409.73907573,696.71798407)(410.13908203,696.52799194)
\curveto(410.53907493,696.33798445)(410.85407461,696.05798473)(411.08408203,695.68799194)
\curveto(411.15407431,695.57798521)(411.20907426,695.45798533)(411.24908203,695.32799194)
\curveto(411.28907418,695.20798558)(411.32907414,695.08298571)(411.36908203,694.95299194)
\curveto(411.38907408,694.87298592)(411.39907407,694.79798599)(411.39908203,694.72799194)
\curveto(411.40907406,694.65798613)(411.42407404,694.5879862)(411.44408203,694.51799194)
\curveto(411.44407402,694.45798633)(411.44907402,694.41798637)(411.45908203,694.39799194)
\curveto(411.47907399,694.25798653)(411.48907398,694.11298668)(411.48908203,693.96299194)
\lineto(411.48908203,693.52799194)
\lineto(411.48908203,692.19299194)
\lineto(411.48908203,689.76299194)
\curveto(411.48907398,689.57299122)(411.48407398,689.3879914)(411.47408203,689.20799194)
\curveto(411.47407399,689.03799175)(411.40407406,688.92799186)(411.26408203,688.87799194)
\curveto(411.20407426,688.85799193)(411.13407433,688.84799194)(411.05408203,688.84799194)
\lineto(410.81408203,688.84799194)
\lineto(410.00408203,688.84799194)
\curveto(409.88407558,688.84799194)(409.77407569,688.85299194)(409.67408203,688.86299194)
\curveto(409.58407588,688.88299191)(409.51407595,688.92799186)(409.46408203,688.99799194)
\curveto(409.42407604,689.05799173)(409.39907607,689.13299166)(409.38908203,689.22299194)
\lineto(409.38908203,689.53799194)
\lineto(409.38908203,690.58799194)
\lineto(409.38908203,692.82299194)
\curveto(409.38907608,693.1929876)(409.37407609,693.53298726)(409.34408203,693.84299194)
\curveto(409.31407615,694.16298663)(409.22407624,694.43298636)(409.07408203,694.65299194)
\curveto(408.93407653,694.85298594)(408.72907674,694.9929858)(408.45908203,695.07299194)
\curveto(408.40907706,695.0929857)(408.35407711,695.10298569)(408.29408203,695.10299194)
\curveto(408.24407722,695.10298569)(408.18907728,695.11298568)(408.12908203,695.13299194)
\curveto(408.07907739,695.14298565)(408.01407745,695.14298565)(407.93408203,695.13299194)
\curveto(407.8640776,695.13298566)(407.80907766,695.12798566)(407.76908203,695.11799194)
\curveto(407.72907774,695.10798568)(407.69407777,695.10298569)(407.66408203,695.10299194)
\curveto(407.63407783,695.10298569)(407.60407786,695.09798569)(407.57408203,695.08799194)
\curveto(407.34407812,695.02798576)(407.15907831,694.94798584)(407.01908203,694.84799194)
\curveto(406.69907877,694.61798617)(406.50907896,694.28298651)(406.44908203,693.84299194)
\curveto(406.38907908,693.40298739)(406.35907911,692.90798788)(406.35908203,692.35799194)
\lineto(406.35908203,690.48299194)
\lineto(406.35908203,689.56799194)
\lineto(406.35908203,689.29799194)
\curveto(406.35907911,689.20799158)(406.34407912,689.13299166)(406.31408203,689.07299194)
\curveto(406.2640792,688.96299183)(406.18407928,688.89799189)(406.07408203,688.87799194)
\curveto(405.9640795,688.85799193)(405.82907964,688.84799194)(405.66908203,688.84799194)
\lineto(404.91908203,688.84799194)
\curveto(404.80908066,688.84799194)(404.69908077,688.85299194)(404.58908203,688.86299194)
\curveto(404.47908099,688.87299192)(404.39908107,688.90799188)(404.34908203,688.96799194)
\curveto(404.27908119,689.05799173)(404.24408122,689.1879916)(404.24408203,689.35799194)
\curveto(404.25408121,689.52799126)(404.25908121,689.6879911)(404.25908203,689.83799194)
\lineto(404.25908203,691.87799194)
\lineto(404.25908203,695.17799194)
\lineto(404.25908203,695.94299194)
\lineto(404.25908203,696.24299194)
\curveto(404.2690812,696.33298446)(404.29908117,696.40798438)(404.34908203,696.46799194)
\curveto(404.3690811,696.49798429)(404.39908107,696.51798427)(404.43908203,696.52799194)
\curveto(404.48908098,696.54798424)(404.53908093,696.56298423)(404.58908203,696.57299194)
\lineto(404.66408203,696.57299194)
\curveto(404.71408075,696.58298421)(404.7640807,696.5879842)(404.81408203,696.58799194)
\lineto(404.97908203,696.58799194)
\lineto(405.60908203,696.58799194)
\curveto(405.68907978,696.5879842)(405.7640797,696.58298421)(405.83408203,696.57299194)
\curveto(405.91407955,696.57298422)(405.98407948,696.56298423)(406.04408203,696.54299194)
\curveto(406.11407935,696.51298428)(406.15907931,696.46798432)(406.17908203,696.40799194)
\curveto(406.20907926,696.34798444)(406.23407923,696.27798451)(406.25408203,696.19799194)
\curveto(406.2640792,696.15798463)(406.2640792,696.12298467)(406.25408203,696.09299194)
\curveto(406.25407921,696.06298473)(406.2640792,696.03298476)(406.28408203,696.00299194)
\curveto(406.30407916,695.95298484)(406.31907915,695.92298487)(406.32908203,695.91299194)
\curveto(406.34907912,695.90298489)(406.37407909,695.8879849)(406.40408203,695.86799194)
\curveto(406.51407895,695.85798493)(406.60407886,695.8929849)(406.67408203,695.97299194)
\curveto(406.74407872,696.06298473)(406.81907865,696.13298466)(406.89908203,696.18299194)
\curveto(407.1690783,696.38298441)(407.469078,696.54298425)(407.79908203,696.66299194)
\curveto(407.88907758,696.6929841)(407.97907749,696.71298408)(408.06908203,696.72299194)
\curveto(408.1690773,696.73298406)(408.27407719,696.74798404)(408.38408203,696.76799194)
\curveto(408.41407705,696.77798401)(408.45907701,696.77798401)(408.51908203,696.76799194)
\curveto(408.57907689,696.76798402)(408.61907685,696.77298402)(408.63908203,696.78299194)
}
}
{
\newrgbcolor{curcolor}{0 0 0}
\pscustom[linestyle=none,fillstyle=solid,fillcolor=curcolor]
{
\newpath
\moveto(420.71033203,689.70299194)
\lineto(420.71033203,689.28299194)
\curveto(420.71032366,689.15299164)(420.68032369,689.04799174)(420.62033203,688.96799194)
\curveto(420.5703238,688.91799187)(420.50532387,688.88299191)(420.42533203,688.86299194)
\curveto(420.34532403,688.85299194)(420.25532412,688.84799194)(420.15533203,688.84799194)
\lineto(419.33033203,688.84799194)
\lineto(419.04533203,688.84799194)
\curveto(418.96532541,688.85799193)(418.90032547,688.88299191)(418.85033203,688.92299194)
\curveto(418.78032559,688.97299182)(418.74032563,689.03799175)(418.73033203,689.11799194)
\curveto(418.72032565,689.19799159)(418.70032567,689.27799151)(418.67033203,689.35799194)
\curveto(418.65032572,689.37799141)(418.63032574,689.3929914)(418.61033203,689.40299194)
\curveto(418.60032577,689.42299137)(418.58532579,689.44299135)(418.56533203,689.46299194)
\curveto(418.45532592,689.46299133)(418.375326,689.43799135)(418.32533203,689.38799194)
\lineto(418.17533203,689.23799194)
\curveto(418.10532627,689.1879916)(418.04032633,689.14299165)(417.98033203,689.10299194)
\curveto(417.92032645,689.07299172)(417.85532652,689.03299176)(417.78533203,688.98299194)
\curveto(417.74532663,688.96299183)(417.70032667,688.94299185)(417.65033203,688.92299194)
\curveto(417.61032676,688.90299189)(417.56532681,688.88299191)(417.51533203,688.86299194)
\curveto(417.375327,688.81299198)(417.22532715,688.76799202)(417.06533203,688.72799194)
\curveto(417.01532736,688.70799208)(416.9703274,688.69799209)(416.93033203,688.69799194)
\curveto(416.89032748,688.69799209)(416.85032752,688.6929921)(416.81033203,688.68299194)
\lineto(416.67533203,688.68299194)
\curveto(416.64532773,688.67299212)(416.60532777,688.66799212)(416.55533203,688.66799194)
\lineto(416.42033203,688.66799194)
\curveto(416.36032801,688.64799214)(416.2703281,688.64299215)(416.15033203,688.65299194)
\curveto(416.03032834,688.65299214)(415.94532843,688.66299213)(415.89533203,688.68299194)
\curveto(415.82532855,688.70299209)(415.76032861,688.71299208)(415.70033203,688.71299194)
\curveto(415.65032872,688.70299209)(415.59532878,688.70799208)(415.53533203,688.72799194)
\lineto(415.17533203,688.84799194)
\curveto(415.06532931,688.87799191)(414.95532942,688.91799187)(414.84533203,688.96799194)
\curveto(414.49532988,689.11799167)(414.18033019,689.34799144)(413.90033203,689.65799194)
\curveto(413.63033074,689.97799081)(413.41533096,690.31299048)(413.25533203,690.66299194)
\curveto(413.20533117,690.77299002)(413.16533121,690.87798991)(413.13533203,690.97799194)
\curveto(413.10533127,691.0879897)(413.0703313,691.19798959)(413.03033203,691.30799194)
\curveto(413.02033135,691.34798944)(413.01533136,691.38298941)(413.01533203,691.41299194)
\curveto(413.01533136,691.45298934)(413.00533137,691.49798929)(412.98533203,691.54799194)
\curveto(412.96533141,691.62798916)(412.94533143,691.71298908)(412.92533203,691.80299194)
\curveto(412.91533146,691.90298889)(412.90033147,692.00298879)(412.88033203,692.10299194)
\curveto(412.8703315,692.13298866)(412.86533151,692.16798862)(412.86533203,692.20799194)
\curveto(412.8753315,692.24798854)(412.8753315,692.28298851)(412.86533203,692.31299194)
\lineto(412.86533203,692.44799194)
\curveto(412.86533151,692.49798829)(412.86033151,692.54798824)(412.85033203,692.59799194)
\curveto(412.84033153,692.64798814)(412.83533154,692.70298809)(412.83533203,692.76299194)
\curveto(412.83533154,692.83298796)(412.84033153,692.8879879)(412.85033203,692.92799194)
\curveto(412.86033151,692.97798781)(412.86533151,693.02298777)(412.86533203,693.06299194)
\lineto(412.86533203,693.21299194)
\curveto(412.8753315,693.26298753)(412.8753315,693.30798748)(412.86533203,693.34799194)
\curveto(412.86533151,693.39798739)(412.8753315,693.44798734)(412.89533203,693.49799194)
\curveto(412.91533146,693.60798718)(412.93033144,693.71298708)(412.94033203,693.81299194)
\curveto(412.96033141,693.91298688)(412.98533139,694.01298678)(413.01533203,694.11299194)
\curveto(413.05533132,694.23298656)(413.09033128,694.34798644)(413.12033203,694.45799194)
\curveto(413.15033122,694.56798622)(413.19033118,694.67798611)(413.24033203,694.78799194)
\curveto(413.38033099,695.0879857)(413.55533082,695.37298542)(413.76533203,695.64299194)
\curveto(413.78533059,695.67298512)(413.81033056,695.69798509)(413.84033203,695.71799194)
\curveto(413.88033049,695.74798504)(413.91033046,695.77798501)(413.93033203,695.80799194)
\curveto(413.9703304,695.85798493)(414.01033036,695.90298489)(414.05033203,695.94299194)
\curveto(414.09033028,695.98298481)(414.13533024,696.02298477)(414.18533203,696.06299194)
\curveto(414.22533015,696.08298471)(414.26033011,696.10798468)(414.29033203,696.13799194)
\curveto(414.32033005,696.17798461)(414.35533002,696.20798458)(414.39533203,696.22799194)
\curveto(414.64532973,696.39798439)(414.93532944,696.53798425)(415.26533203,696.64799194)
\curveto(415.33532904,696.66798412)(415.40532897,696.68298411)(415.47533203,696.69299194)
\curveto(415.55532882,696.70298409)(415.63532874,696.71798407)(415.71533203,696.73799194)
\curveto(415.78532859,696.75798403)(415.8753285,696.76798402)(415.98533203,696.76799194)
\curveto(416.09532828,696.77798401)(416.20532817,696.78298401)(416.31533203,696.78299194)
\curveto(416.42532795,696.78298401)(416.53032784,696.77798401)(416.63033203,696.76799194)
\curveto(416.74032763,696.75798403)(416.83032754,696.74298405)(416.90033203,696.72299194)
\curveto(417.05032732,696.67298412)(417.19532718,696.62798416)(417.33533203,696.58799194)
\curveto(417.4753269,696.54798424)(417.60532677,696.4929843)(417.72533203,696.42299194)
\curveto(417.79532658,696.37298442)(417.86032651,696.32298447)(417.92033203,696.27299194)
\curveto(417.98032639,696.23298456)(418.04532633,696.1879846)(418.11533203,696.13799194)
\curveto(418.15532622,696.10798468)(418.21032616,696.06798472)(418.28033203,696.01799194)
\curveto(418.36032601,695.96798482)(418.43532594,695.96798482)(418.50533203,696.01799194)
\curveto(418.54532583,696.03798475)(418.56532581,696.07298472)(418.56533203,696.12299194)
\curveto(418.56532581,696.17298462)(418.5753258,696.22298457)(418.59533203,696.27299194)
\lineto(418.59533203,696.42299194)
\curveto(418.60532577,696.45298434)(418.61032576,696.4879843)(418.61033203,696.52799194)
\lineto(418.61033203,696.64799194)
\lineto(418.61033203,698.68799194)
\curveto(418.61032576,698.79798199)(418.60532577,698.91798187)(418.59533203,699.04799194)
\curveto(418.59532578,699.1879816)(418.62032575,699.2929815)(418.67033203,699.36299194)
\curveto(418.71032566,699.44298135)(418.78532559,699.4929813)(418.89533203,699.51299194)
\curveto(418.91532546,699.52298127)(418.93532544,699.52298127)(418.95533203,699.51299194)
\curveto(418.9753254,699.51298128)(418.99532538,699.51798127)(419.01533203,699.52799194)
\lineto(420.08033203,699.52799194)
\curveto(420.20032417,699.52798126)(420.31032406,699.52298127)(420.41033203,699.51299194)
\curveto(420.51032386,699.50298129)(420.58532379,699.46298133)(420.63533203,699.39299194)
\curveto(420.68532369,699.31298148)(420.71032366,699.20798158)(420.71033203,699.07799194)
\lineto(420.71033203,698.71799194)
\lineto(420.71033203,689.70299194)
\moveto(418.67033203,692.64299194)
\curveto(418.68032569,692.68298811)(418.68032569,692.72298807)(418.67033203,692.76299194)
\lineto(418.67033203,692.89799194)
\curveto(418.6703257,692.99798779)(418.66532571,693.09798769)(418.65533203,693.19799194)
\curveto(418.64532573,693.29798749)(418.63032574,693.3879874)(418.61033203,693.46799194)
\curveto(418.59032578,693.57798721)(418.5703258,693.67798711)(418.55033203,693.76799194)
\curveto(418.54032583,693.85798693)(418.51532586,693.94298685)(418.47533203,694.02299194)
\curveto(418.33532604,694.38298641)(418.13032624,694.66798612)(417.86033203,694.87799194)
\curveto(417.60032677,695.0879857)(417.22032715,695.1929856)(416.72033203,695.19299194)
\curveto(416.66032771,695.1929856)(416.58032779,695.18298561)(416.48033203,695.16299194)
\curveto(416.40032797,695.14298565)(416.32532805,695.12298567)(416.25533203,695.10299194)
\curveto(416.19532818,695.0929857)(416.13532824,695.07298572)(416.07533203,695.04299194)
\curveto(415.80532857,694.93298586)(415.59532878,694.76298603)(415.44533203,694.53299194)
\curveto(415.29532908,694.30298649)(415.1753292,694.04298675)(415.08533203,693.75299194)
\curveto(415.05532932,693.65298714)(415.03532934,693.55298724)(415.02533203,693.45299194)
\curveto(415.01532936,693.35298744)(414.99532938,693.24798754)(414.96533203,693.13799194)
\lineto(414.96533203,692.92799194)
\curveto(414.94532943,692.83798795)(414.94032943,692.71298808)(414.95033203,692.55299194)
\curveto(414.96032941,692.40298839)(414.9753294,692.2929885)(414.99533203,692.22299194)
\lineto(414.99533203,692.13299194)
\curveto(415.00532937,692.11298868)(415.01032936,692.0929887)(415.01033203,692.07299194)
\curveto(415.03032934,691.9929888)(415.04532933,691.91798887)(415.05533203,691.84799194)
\curveto(415.0753293,691.77798901)(415.09532928,691.70298909)(415.11533203,691.62299194)
\curveto(415.28532909,691.10298969)(415.5753288,690.71799007)(415.98533203,690.46799194)
\curveto(416.11532826,690.37799041)(416.29532808,690.30799048)(416.52533203,690.25799194)
\curveto(416.56532781,690.24799054)(416.62532775,690.24299055)(416.70533203,690.24299194)
\curveto(416.73532764,690.23299056)(416.78032759,690.22299057)(416.84033203,690.21299194)
\curveto(416.91032746,690.21299058)(416.96532741,690.21799057)(417.00533203,690.22799194)
\curveto(417.08532729,690.24799054)(417.16532721,690.26299053)(417.24533203,690.27299194)
\curveto(417.32532705,690.28299051)(417.40532697,690.30299049)(417.48533203,690.33299194)
\curveto(417.73532664,690.44299035)(417.93532644,690.58299021)(418.08533203,690.75299194)
\curveto(418.23532614,690.92298987)(418.36532601,691.13798965)(418.47533203,691.39799194)
\curveto(418.51532586,691.4879893)(418.54532583,691.57798921)(418.56533203,691.66799194)
\curveto(418.58532579,691.76798902)(418.60532577,691.87298892)(418.62533203,691.98299194)
\curveto(418.63532574,692.03298876)(418.63532574,692.07798871)(418.62533203,692.11799194)
\curveto(418.62532575,692.16798862)(418.63532574,692.21798857)(418.65533203,692.26799194)
\curveto(418.66532571,692.29798849)(418.6703257,692.33298846)(418.67033203,692.37299194)
\lineto(418.67033203,692.50799194)
\lineto(418.67033203,692.64299194)
}
}
{
\newrgbcolor{curcolor}{0 0 0}
\pscustom[linestyle=none,fillstyle=solid,fillcolor=curcolor]
{
\newpath
\moveto(424.39025391,699.43799194)
\curveto(424.46025096,699.35798143)(424.49525092,699.23798155)(424.49525391,699.07799194)
\lineto(424.49525391,698.61299194)
\lineto(424.49525391,698.20799194)
\curveto(424.49525092,698.06798272)(424.46025096,697.97298282)(424.39025391,697.92299194)
\curveto(424.33025109,697.87298292)(424.25025117,697.84298295)(424.15025391,697.83299194)
\curveto(424.06025136,697.82298297)(423.96025146,697.81798297)(423.85025391,697.81799194)
\lineto(423.01025391,697.81799194)
\curveto(422.90025252,697.81798297)(422.80025262,697.82298297)(422.71025391,697.83299194)
\curveto(422.63025279,697.84298295)(422.56025286,697.87298292)(422.50025391,697.92299194)
\curveto(422.46025296,697.95298284)(422.43025299,698.00798278)(422.41025391,698.08799194)
\curveto(422.40025302,698.17798261)(422.39025303,698.27298252)(422.38025391,698.37299194)
\lineto(422.38025391,698.70299194)
\curveto(422.39025303,698.81298198)(422.39525302,698.90798188)(422.39525391,698.98799194)
\lineto(422.39525391,699.19799194)
\curveto(422.40525301,699.26798152)(422.42525299,699.32798146)(422.45525391,699.37799194)
\curveto(422.47525294,699.41798137)(422.50025292,699.44798134)(422.53025391,699.46799194)
\lineto(422.65025391,699.52799194)
\curveto(422.67025275,699.52798126)(422.69525272,699.52798126)(422.72525391,699.52799194)
\curveto(422.75525266,699.53798125)(422.78025264,699.54298125)(422.80025391,699.54299194)
\lineto(423.89525391,699.54299194)
\curveto(423.99525142,699.54298125)(424.09025133,699.53798125)(424.18025391,699.52799194)
\curveto(424.27025115,699.51798127)(424.34025108,699.4879813)(424.39025391,699.43799194)
\moveto(424.49525391,689.67299194)
\curveto(424.49525092,689.47299132)(424.49025093,689.30299149)(424.48025391,689.16299194)
\curveto(424.47025095,689.02299177)(424.38025104,688.92799186)(424.21025391,688.87799194)
\curveto(424.15025127,688.85799193)(424.08525133,688.84799194)(424.01525391,688.84799194)
\curveto(423.94525147,688.85799193)(423.87025155,688.86299193)(423.79025391,688.86299194)
\lineto(422.95025391,688.86299194)
\curveto(422.86025256,688.86299193)(422.77025265,688.86799192)(422.68025391,688.87799194)
\curveto(422.60025282,688.8879919)(422.54025288,688.91799187)(422.50025391,688.96799194)
\curveto(422.44025298,689.03799175)(422.40525301,689.12299167)(422.39525391,689.22299194)
\lineto(422.39525391,689.56799194)
\lineto(422.39525391,695.89799194)
\lineto(422.39525391,696.19799194)
\curveto(422.39525302,696.29798449)(422.415253,696.37798441)(422.45525391,696.43799194)
\curveto(422.5152529,696.50798428)(422.60025282,696.55298424)(422.71025391,696.57299194)
\curveto(422.73025269,696.58298421)(422.75525266,696.58298421)(422.78525391,696.57299194)
\curveto(422.82525259,696.57298422)(422.85525256,696.57798421)(422.87525391,696.58799194)
\lineto(423.62525391,696.58799194)
\lineto(423.82025391,696.58799194)
\curveto(423.90025152,696.59798419)(423.96525145,696.59798419)(424.01525391,696.58799194)
\lineto(424.13525391,696.58799194)
\curveto(424.19525122,696.56798422)(424.25025117,696.55298424)(424.30025391,696.54299194)
\curveto(424.35025107,696.53298426)(424.39025103,696.50298429)(424.42025391,696.45299194)
\curveto(424.46025096,696.40298439)(424.48025094,696.33298446)(424.48025391,696.24299194)
\curveto(424.49025093,696.15298464)(424.49525092,696.05798473)(424.49525391,695.95799194)
\lineto(424.49525391,689.67299194)
}
}
{
\newrgbcolor{curcolor}{0 0 0}
\pscustom[linestyle=none,fillstyle=solid,fillcolor=curcolor]
{
\newpath
\moveto(429.72744141,696.79799194)
\curveto(430.53743625,696.81798397)(431.21243557,696.69798409)(431.75244141,696.43799194)
\curveto(432.30243448,696.17798461)(432.73743405,695.80798498)(433.05744141,695.32799194)
\curveto(433.21743357,695.0879857)(433.33743345,694.81298598)(433.41744141,694.50299194)
\curveto(433.43743335,694.45298634)(433.45243333,694.3879864)(433.46244141,694.30799194)
\curveto(433.4824333,694.22798656)(433.4824333,694.15798663)(433.46244141,694.09799194)
\curveto(433.42243336,693.9879868)(433.35243343,693.92298687)(433.25244141,693.90299194)
\curveto(433.15243363,693.8929869)(433.03243375,693.8879869)(432.89244141,693.88799194)
\lineto(432.11244141,693.88799194)
\lineto(431.82744141,693.88799194)
\curveto(431.73743505,693.8879869)(431.66243512,693.90798688)(431.60244141,693.94799194)
\curveto(431.52243526,693.9879868)(431.46743532,694.04798674)(431.43744141,694.12799194)
\curveto(431.40743538,694.21798657)(431.36743542,694.30798648)(431.31744141,694.39799194)
\curveto(431.25743553,694.50798628)(431.19243559,694.60798618)(431.12244141,694.69799194)
\curveto(431.05243573,694.787986)(430.97243581,694.86798592)(430.88244141,694.93799194)
\curveto(430.74243604,695.02798576)(430.5874362,695.09798569)(430.41744141,695.14799194)
\curveto(430.35743643,695.16798562)(430.29743649,695.17798561)(430.23744141,695.17799194)
\curveto(430.17743661,695.17798561)(430.12243666,695.1879856)(430.07244141,695.20799194)
\lineto(429.92244141,695.20799194)
\curveto(429.72243706,695.20798558)(429.56243722,695.1879856)(429.44244141,695.14799194)
\curveto(429.15243763,695.05798573)(428.91743787,694.91798587)(428.73744141,694.72799194)
\curveto(428.55743823,694.54798624)(428.41243837,694.32798646)(428.30244141,694.06799194)
\curveto(428.25243853,693.95798683)(428.21243857,693.83798695)(428.18244141,693.70799194)
\curveto(428.16243862,693.5879872)(428.13743865,693.45798733)(428.10744141,693.31799194)
\curveto(428.09743869,693.27798751)(428.09243869,693.23798755)(428.09244141,693.19799194)
\curveto(428.09243869,693.15798763)(428.0874387,693.11798767)(428.07744141,693.07799194)
\curveto(428.05743873,692.97798781)(428.04743874,692.83798795)(428.04744141,692.65799194)
\curveto(428.05743873,692.47798831)(428.07243871,692.33798845)(428.09244141,692.23799194)
\curveto(428.09243869,692.15798863)(428.09743869,692.10298869)(428.10744141,692.07299194)
\curveto(428.12743866,692.00298879)(428.13743865,691.93298886)(428.13744141,691.86299194)
\curveto(428.14743864,691.792989)(428.16243862,691.72298907)(428.18244141,691.65299194)
\curveto(428.26243852,691.42298937)(428.35743843,691.21298958)(428.46744141,691.02299194)
\curveto(428.57743821,690.83298996)(428.71743807,690.67299012)(428.88744141,690.54299194)
\curveto(428.92743786,690.51299028)(428.9874378,690.47799031)(429.06744141,690.43799194)
\curveto(429.17743761,690.36799042)(429.2874375,690.32299047)(429.39744141,690.30299194)
\curveto(429.51743727,690.28299051)(429.66243712,690.26299053)(429.83244141,690.24299194)
\lineto(429.92244141,690.24299194)
\curveto(429.96243682,690.24299055)(429.99243679,690.24799054)(430.01244141,690.25799194)
\lineto(430.14744141,690.25799194)
\curveto(430.21743657,690.27799051)(430.2824365,690.2929905)(430.34244141,690.30299194)
\curveto(430.41243637,690.32299047)(430.47743631,690.34299045)(430.53744141,690.36299194)
\curveto(430.83743595,690.4929903)(431.06743572,690.68299011)(431.22744141,690.93299194)
\curveto(431.26743552,690.98298981)(431.30243548,691.03798975)(431.33244141,691.09799194)
\curveto(431.36243542,691.16798962)(431.3874354,691.22798956)(431.40744141,691.27799194)
\curveto(431.44743534,691.3879894)(431.4824353,691.48298931)(431.51244141,691.56299194)
\curveto(431.54243524,691.65298914)(431.61243517,691.72298907)(431.72244141,691.77299194)
\curveto(431.81243497,691.81298898)(431.95743483,691.82798896)(432.15744141,691.81799194)
\lineto(432.65244141,691.81799194)
\lineto(432.86244141,691.81799194)
\curveto(432.94243384,691.82798896)(433.00743378,691.82298897)(433.05744141,691.80299194)
\lineto(433.17744141,691.80299194)
\lineto(433.29744141,691.77299194)
\curveto(433.33743345,691.77298902)(433.36743342,691.76298903)(433.38744141,691.74299194)
\curveto(433.43743335,691.70298909)(433.46743332,691.64298915)(433.47744141,691.56299194)
\curveto(433.49743329,691.4929893)(433.49743329,691.41798937)(433.47744141,691.33799194)
\curveto(433.3874334,691.00798978)(433.27743351,690.71299008)(433.14744141,690.45299194)
\curveto(432.73743405,689.68299111)(432.0824347,689.14799164)(431.18244141,688.84799194)
\curveto(431.0824357,688.81799197)(430.97743581,688.79799199)(430.86744141,688.78799194)
\curveto(430.75743603,688.76799202)(430.64743614,688.74299205)(430.53744141,688.71299194)
\curveto(430.47743631,688.70299209)(430.41743637,688.69799209)(430.35744141,688.69799194)
\curveto(430.29743649,688.69799209)(430.23743655,688.6929921)(430.17744141,688.68299194)
\lineto(430.01244141,688.68299194)
\curveto(429.96243682,688.66299213)(429.8874369,688.65799213)(429.78744141,688.66799194)
\curveto(429.6874371,688.66799212)(429.61243717,688.67299212)(429.56244141,688.68299194)
\curveto(429.4824373,688.70299209)(429.40743738,688.71299208)(429.33744141,688.71299194)
\curveto(429.27743751,688.70299209)(429.21243757,688.70799208)(429.14244141,688.72799194)
\lineto(428.99244141,688.75799194)
\curveto(428.94243784,688.75799203)(428.89243789,688.76299203)(428.84244141,688.77299194)
\curveto(428.73243805,688.80299199)(428.62743816,688.83299196)(428.52744141,688.86299194)
\curveto(428.42743836,688.8929919)(428.33243845,688.92799186)(428.24244141,688.96799194)
\curveto(427.77243901,689.16799162)(427.37743941,689.42299137)(427.05744141,689.73299194)
\curveto(426.73744005,690.05299074)(426.47744031,690.44799034)(426.27744141,690.91799194)
\curveto(426.22744056,691.00798978)(426.1874406,691.10298969)(426.15744141,691.20299194)
\lineto(426.06744141,691.53299194)
\curveto(426.05744073,691.57298922)(426.05244073,691.60798918)(426.05244141,691.63799194)
\curveto(426.05244073,691.67798911)(426.04244074,691.72298907)(426.02244141,691.77299194)
\curveto(426.00244078,691.84298895)(425.99244079,691.91298888)(425.99244141,691.98299194)
\curveto(425.99244079,692.06298873)(425.9824408,692.13798865)(425.96244141,692.20799194)
\lineto(425.96244141,692.46299194)
\curveto(425.94244084,692.51298828)(425.93244085,692.56798822)(425.93244141,692.62799194)
\curveto(425.93244085,692.69798809)(425.94244084,692.75798803)(425.96244141,692.80799194)
\curveto(425.97244081,692.85798793)(425.97244081,692.90298789)(425.96244141,692.94299194)
\curveto(425.95244083,692.98298781)(425.95244083,693.02298777)(425.96244141,693.06299194)
\curveto(425.9824408,693.13298766)(425.9874408,693.19798759)(425.97744141,693.25799194)
\curveto(425.97744081,693.31798747)(425.9874408,693.37798741)(426.00744141,693.43799194)
\curveto(426.05744073,693.61798717)(426.09744069,693.787987)(426.12744141,693.94799194)
\curveto(426.15744063,694.11798667)(426.20244058,694.28298651)(426.26244141,694.44299194)
\curveto(426.4824403,694.95298584)(426.75744003,695.37798541)(427.08744141,695.71799194)
\curveto(427.42743936,696.05798473)(427.85743893,696.33298446)(428.37744141,696.54299194)
\curveto(428.51743827,696.60298419)(428.66243812,696.64298415)(428.81244141,696.66299194)
\curveto(428.96243782,696.6929841)(429.11743767,696.72798406)(429.27744141,696.76799194)
\curveto(429.35743743,696.77798401)(429.43243735,696.78298401)(429.50244141,696.78299194)
\curveto(429.57243721,696.78298401)(429.64743714,696.787984)(429.72744141,696.79799194)
}
}
{
\newrgbcolor{curcolor}{0 0 0}
\pscustom[linestyle=none,fillstyle=solid,fillcolor=curcolor]
{
\newpath
\moveto(441.82072266,689.44799194)
\curveto(441.84071481,689.33799145)(441.8507148,689.22799156)(441.85072266,689.11799194)
\curveto(441.86071479,689.00799178)(441.81071484,688.93299186)(441.70072266,688.89299194)
\curveto(441.64071501,688.86299193)(441.57071508,688.84799194)(441.49072266,688.84799194)
\lineto(441.25072266,688.84799194)
\lineto(440.44072266,688.84799194)
\lineto(440.17072266,688.84799194)
\curveto(440.09071656,688.85799193)(440.02571662,688.88299191)(439.97572266,688.92299194)
\curveto(439.90571674,688.96299183)(439.8507168,689.01799177)(439.81072266,689.08799194)
\curveto(439.78071687,689.16799162)(439.73571691,689.23299156)(439.67572266,689.28299194)
\curveto(439.65571699,689.30299149)(439.63071702,689.31799147)(439.60072266,689.32799194)
\curveto(439.57071708,689.34799144)(439.53071712,689.35299144)(439.48072266,689.34299194)
\curveto(439.43071722,689.32299147)(439.38071727,689.29799149)(439.33072266,689.26799194)
\curveto(439.29071736,689.23799155)(439.2457174,689.21299158)(439.19572266,689.19299194)
\curveto(439.1457175,689.15299164)(439.09071756,689.11799167)(439.03072266,689.08799194)
\lineto(438.85072266,688.99799194)
\curveto(438.72071793,688.93799185)(438.58571806,688.8879919)(438.44572266,688.84799194)
\curveto(438.30571834,688.81799197)(438.16071849,688.78299201)(438.01072266,688.74299194)
\curveto(437.94071871,688.72299207)(437.87071878,688.71299208)(437.80072266,688.71299194)
\curveto(437.74071891,688.70299209)(437.67571897,688.6929921)(437.60572266,688.68299194)
\lineto(437.51572266,688.68299194)
\curveto(437.48571916,688.67299212)(437.45571919,688.66799212)(437.42572266,688.66799194)
\lineto(437.26072266,688.66799194)
\curveto(437.16071949,688.64799214)(437.06071959,688.64799214)(436.96072266,688.66799194)
\lineto(436.82572266,688.66799194)
\curveto(436.75571989,688.6879921)(436.68571996,688.69799209)(436.61572266,688.69799194)
\curveto(436.55572009,688.6879921)(436.49572015,688.6929921)(436.43572266,688.71299194)
\curveto(436.33572031,688.73299206)(436.24072041,688.75299204)(436.15072266,688.77299194)
\curveto(436.06072059,688.78299201)(435.97572067,688.80799198)(435.89572266,688.84799194)
\curveto(435.60572104,688.95799183)(435.35572129,689.09799169)(435.14572266,689.26799194)
\curveto(434.9457217,689.44799134)(434.78572186,689.68299111)(434.66572266,689.97299194)
\curveto(434.63572201,690.04299075)(434.60572204,690.11799067)(434.57572266,690.19799194)
\curveto(434.55572209,690.27799051)(434.53572211,690.36299043)(434.51572266,690.45299194)
\curveto(434.49572215,690.50299029)(434.48572216,690.55299024)(434.48572266,690.60299194)
\curveto(434.49572215,690.65299014)(434.49572215,690.70299009)(434.48572266,690.75299194)
\curveto(434.47572217,690.78299001)(434.46572218,690.84298995)(434.45572266,690.93299194)
\curveto(434.45572219,691.03298976)(434.46072219,691.10298969)(434.47072266,691.14299194)
\curveto(434.49072216,691.24298955)(434.50072215,691.32798946)(434.50072266,691.39799194)
\lineto(434.59072266,691.72799194)
\curveto(434.62072203,691.84798894)(434.66072199,691.95298884)(434.71072266,692.04299194)
\curveto(434.88072177,692.33298846)(435.07572157,692.55298824)(435.29572266,692.70299194)
\curveto(435.51572113,692.85298794)(435.79572085,692.98298781)(436.13572266,693.09299194)
\curveto(436.26572038,693.14298765)(436.40072025,693.17798761)(436.54072266,693.19799194)
\curveto(436.68071997,693.21798757)(436.82071983,693.24298755)(436.96072266,693.27299194)
\curveto(437.04071961,693.2929875)(437.12571952,693.30298749)(437.21572266,693.30299194)
\curveto(437.30571934,693.31298748)(437.39571925,693.32798746)(437.48572266,693.34799194)
\curveto(437.55571909,693.36798742)(437.62571902,693.37298742)(437.69572266,693.36299194)
\curveto(437.76571888,693.36298743)(437.84071881,693.37298742)(437.92072266,693.39299194)
\curveto(437.99071866,693.41298738)(438.06071859,693.42298737)(438.13072266,693.42299194)
\curveto(438.20071845,693.42298737)(438.27571837,693.43298736)(438.35572266,693.45299194)
\curveto(438.56571808,693.50298729)(438.75571789,693.54298725)(438.92572266,693.57299194)
\curveto(439.10571754,693.61298718)(439.26571738,693.70298709)(439.40572266,693.84299194)
\curveto(439.49571715,693.93298686)(439.55571709,694.03298676)(439.58572266,694.14299194)
\curveto(439.59571705,694.17298662)(439.59571705,694.19798659)(439.58572266,694.21799194)
\curveto(439.58571706,694.23798655)(439.59071706,694.25798653)(439.60072266,694.27799194)
\curveto(439.61071704,694.29798649)(439.61571703,694.32798646)(439.61572266,694.36799194)
\lineto(439.61572266,694.45799194)
\lineto(439.58572266,694.57799194)
\curveto(439.58571706,694.61798617)(439.58071707,694.65298614)(439.57072266,694.68299194)
\curveto(439.47071718,694.98298581)(439.26071739,695.1879856)(438.94072266,695.29799194)
\curveto(438.8507178,695.32798546)(438.74071791,695.34798544)(438.61072266,695.35799194)
\curveto(438.49071816,695.37798541)(438.36571828,695.38298541)(438.23572266,695.37299194)
\curveto(438.10571854,695.37298542)(437.98071867,695.36298543)(437.86072266,695.34299194)
\curveto(437.74071891,695.32298547)(437.63571901,695.29798549)(437.54572266,695.26799194)
\curveto(437.48571916,695.24798554)(437.42571922,695.21798557)(437.36572266,695.17799194)
\curveto(437.31571933,695.14798564)(437.26571938,695.11298568)(437.21572266,695.07299194)
\curveto(437.16571948,695.03298576)(437.11071954,694.97798581)(437.05072266,694.90799194)
\curveto(437.00071965,694.83798595)(436.96571968,694.77298602)(436.94572266,694.71299194)
\curveto(436.89571975,694.61298618)(436.8507198,694.51798627)(436.81072266,694.42799194)
\curveto(436.78071987,694.33798645)(436.71071994,694.27798651)(436.60072266,694.24799194)
\curveto(436.52072013,694.22798656)(436.43572021,694.21798657)(436.34572266,694.21799194)
\lineto(436.07572266,694.21799194)
\lineto(435.50572266,694.21799194)
\curveto(435.45572119,694.21798657)(435.40572124,694.21298658)(435.35572266,694.20299194)
\curveto(435.30572134,694.20298659)(435.26072139,694.20798658)(435.22072266,694.21799194)
\lineto(435.08572266,694.21799194)
\curveto(435.06572158,694.22798656)(435.04072161,694.23298656)(435.01072266,694.23299194)
\curveto(434.98072167,694.23298656)(434.95572169,694.24298655)(434.93572266,694.26299194)
\curveto(434.85572179,694.28298651)(434.80072185,694.34798644)(434.77072266,694.45799194)
\curveto(434.76072189,694.50798628)(434.76072189,694.55798623)(434.77072266,694.60799194)
\curveto(434.78072187,694.65798613)(434.79072186,694.70298609)(434.80072266,694.74299194)
\curveto(434.83072182,694.85298594)(434.86072179,694.95298584)(434.89072266,695.04299194)
\curveto(434.93072172,695.14298565)(434.97572167,695.23298556)(435.02572266,695.31299194)
\lineto(435.11572266,695.46299194)
\lineto(435.20572266,695.61299194)
\curveto(435.28572136,695.72298507)(435.38572126,695.82798496)(435.50572266,695.92799194)
\curveto(435.52572112,695.93798485)(435.55572109,695.96298483)(435.59572266,696.00299194)
\curveto(435.645721,696.04298475)(435.69072096,696.07798471)(435.73072266,696.10799194)
\curveto(435.77072088,696.13798465)(435.81572083,696.16798462)(435.86572266,696.19799194)
\curveto(436.03572061,696.30798448)(436.21572043,696.3929844)(436.40572266,696.45299194)
\curveto(436.59572005,696.52298427)(436.79071986,696.5879842)(436.99072266,696.64799194)
\curveto(437.11071954,696.67798411)(437.23571941,696.69798409)(437.36572266,696.70799194)
\curveto(437.49571915,696.71798407)(437.62571902,696.73798405)(437.75572266,696.76799194)
\curveto(437.79571885,696.77798401)(437.85571879,696.77798401)(437.93572266,696.76799194)
\curveto(438.02571862,696.75798403)(438.08071857,696.76298403)(438.10072266,696.78299194)
\curveto(438.51071814,696.792984)(438.90071775,696.77798401)(439.27072266,696.73799194)
\curveto(439.650717,696.69798409)(439.99071666,696.62298417)(440.29072266,696.51299194)
\curveto(440.60071605,696.40298439)(440.86571578,696.25298454)(441.08572266,696.06299194)
\curveto(441.30571534,695.88298491)(441.47571517,695.64798514)(441.59572266,695.35799194)
\curveto(441.66571498,695.1879856)(441.70571494,694.9929858)(441.71572266,694.77299194)
\curveto(441.72571492,694.55298624)(441.73071492,694.32798646)(441.73072266,694.09799194)
\lineto(441.73072266,690.75299194)
\lineto(441.73072266,690.16799194)
\curveto(441.73071492,689.97799081)(441.7507149,689.80299099)(441.79072266,689.64299194)
\curveto(441.80071485,689.61299118)(441.80571484,689.57799121)(441.80572266,689.53799194)
\curveto(441.80571484,689.50799128)(441.81071484,689.47799131)(441.82072266,689.44799194)
\moveto(439.61572266,691.75799194)
\curveto(439.62571702,691.80798898)(439.63071702,691.86298893)(439.63072266,691.92299194)
\curveto(439.63071702,691.9929888)(439.62571702,692.05298874)(439.61572266,692.10299194)
\curveto(439.59571705,692.16298863)(439.58571706,692.21798857)(439.58572266,692.26799194)
\curveto(439.58571706,692.31798847)(439.56571708,692.35798843)(439.52572266,692.38799194)
\curveto(439.47571717,692.42798836)(439.40071725,692.44798834)(439.30072266,692.44799194)
\curveto(439.26071739,692.43798835)(439.22571742,692.42798836)(439.19572266,692.41799194)
\curveto(439.16571748,692.41798837)(439.13071752,692.41298838)(439.09072266,692.40299194)
\curveto(439.02071763,692.38298841)(438.9457177,692.36798842)(438.86572266,692.35799194)
\curveto(438.78571786,692.34798844)(438.70571794,692.33298846)(438.62572266,692.31299194)
\curveto(438.59571805,692.30298849)(438.5507181,692.29798849)(438.49072266,692.29799194)
\curveto(438.36071829,692.26798852)(438.23071842,692.24798854)(438.10072266,692.23799194)
\curveto(437.97071868,692.22798856)(437.8457188,692.20298859)(437.72572266,692.16299194)
\curveto(437.645719,692.14298865)(437.57071908,692.12298867)(437.50072266,692.10299194)
\curveto(437.43071922,692.0929887)(437.36071929,692.07298872)(437.29072266,692.04299194)
\curveto(437.08071957,691.95298884)(436.90071975,691.81798897)(436.75072266,691.63799194)
\curveto(436.61072004,691.45798933)(436.56072009,691.20798958)(436.60072266,690.88799194)
\curveto(436.62072003,690.71799007)(436.67571997,690.57799021)(436.76572266,690.46799194)
\curveto(436.83571981,690.35799043)(436.94071971,690.26799052)(437.08072266,690.19799194)
\curveto(437.22071943,690.13799065)(437.37071928,690.0929907)(437.53072266,690.06299194)
\curveto(437.70071895,690.03299076)(437.87571877,690.02299077)(438.05572266,690.03299194)
\curveto(438.2457184,690.05299074)(438.42071823,690.0879907)(438.58072266,690.13799194)
\curveto(438.84071781,690.21799057)(439.0457176,690.34299045)(439.19572266,690.51299194)
\curveto(439.3457173,690.6929901)(439.46071719,690.91298988)(439.54072266,691.17299194)
\curveto(439.56071709,691.24298955)(439.57071708,691.31298948)(439.57072266,691.38299194)
\curveto(439.58071707,691.46298933)(439.59571705,691.54298925)(439.61572266,691.62299194)
\lineto(439.61572266,691.75799194)
}
}
{
\newrgbcolor{curcolor}{0 0 0}
\pscustom[linestyle=none,fillstyle=solid,fillcolor=curcolor]
{
\newpath
\moveto(450.97400391,689.70299194)
\lineto(450.97400391,689.28299194)
\curveto(450.97399554,689.15299164)(450.94399557,689.04799174)(450.88400391,688.96799194)
\curveto(450.83399568,688.91799187)(450.76899574,688.88299191)(450.68900391,688.86299194)
\curveto(450.6089959,688.85299194)(450.51899599,688.84799194)(450.41900391,688.84799194)
\lineto(449.59400391,688.84799194)
\lineto(449.30900391,688.84799194)
\curveto(449.22899728,688.85799193)(449.16399735,688.88299191)(449.11400391,688.92299194)
\curveto(449.04399747,688.97299182)(449.00399751,689.03799175)(448.99400391,689.11799194)
\curveto(448.98399753,689.19799159)(448.96399755,689.27799151)(448.93400391,689.35799194)
\curveto(448.9139976,689.37799141)(448.89399762,689.3929914)(448.87400391,689.40299194)
\curveto(448.86399765,689.42299137)(448.84899766,689.44299135)(448.82900391,689.46299194)
\curveto(448.71899779,689.46299133)(448.63899787,689.43799135)(448.58900391,689.38799194)
\lineto(448.43900391,689.23799194)
\curveto(448.36899814,689.1879916)(448.30399821,689.14299165)(448.24400391,689.10299194)
\curveto(448.18399833,689.07299172)(448.11899839,689.03299176)(448.04900391,688.98299194)
\curveto(448.0089985,688.96299183)(447.96399855,688.94299185)(447.91400391,688.92299194)
\curveto(447.87399864,688.90299189)(447.82899868,688.88299191)(447.77900391,688.86299194)
\curveto(447.63899887,688.81299198)(447.48899902,688.76799202)(447.32900391,688.72799194)
\curveto(447.27899923,688.70799208)(447.23399928,688.69799209)(447.19400391,688.69799194)
\curveto(447.15399936,688.69799209)(447.1139994,688.6929921)(447.07400391,688.68299194)
\lineto(446.93900391,688.68299194)
\curveto(446.9089996,688.67299212)(446.86899964,688.66799212)(446.81900391,688.66799194)
\lineto(446.68400391,688.66799194)
\curveto(446.62399989,688.64799214)(446.53399998,688.64299215)(446.41400391,688.65299194)
\curveto(446.29400022,688.65299214)(446.2090003,688.66299213)(446.15900391,688.68299194)
\curveto(446.08900042,688.70299209)(446.02400049,688.71299208)(445.96400391,688.71299194)
\curveto(445.9140006,688.70299209)(445.85900065,688.70799208)(445.79900391,688.72799194)
\lineto(445.43900391,688.84799194)
\curveto(445.32900118,688.87799191)(445.21900129,688.91799187)(445.10900391,688.96799194)
\curveto(444.75900175,689.11799167)(444.44400207,689.34799144)(444.16400391,689.65799194)
\curveto(443.89400262,689.97799081)(443.67900283,690.31299048)(443.51900391,690.66299194)
\curveto(443.46900304,690.77299002)(443.42900308,690.87798991)(443.39900391,690.97799194)
\curveto(443.36900314,691.0879897)(443.33400318,691.19798959)(443.29400391,691.30799194)
\curveto(443.28400323,691.34798944)(443.27900323,691.38298941)(443.27900391,691.41299194)
\curveto(443.27900323,691.45298934)(443.26900324,691.49798929)(443.24900391,691.54799194)
\curveto(443.22900328,691.62798916)(443.2090033,691.71298908)(443.18900391,691.80299194)
\curveto(443.17900333,691.90298889)(443.16400335,692.00298879)(443.14400391,692.10299194)
\curveto(443.13400338,692.13298866)(443.12900338,692.16798862)(443.12900391,692.20799194)
\curveto(443.13900337,692.24798854)(443.13900337,692.28298851)(443.12900391,692.31299194)
\lineto(443.12900391,692.44799194)
\curveto(443.12900338,692.49798829)(443.12400339,692.54798824)(443.11400391,692.59799194)
\curveto(443.10400341,692.64798814)(443.09900341,692.70298809)(443.09900391,692.76299194)
\curveto(443.09900341,692.83298796)(443.10400341,692.8879879)(443.11400391,692.92799194)
\curveto(443.12400339,692.97798781)(443.12900338,693.02298777)(443.12900391,693.06299194)
\lineto(443.12900391,693.21299194)
\curveto(443.13900337,693.26298753)(443.13900337,693.30798748)(443.12900391,693.34799194)
\curveto(443.12900338,693.39798739)(443.13900337,693.44798734)(443.15900391,693.49799194)
\curveto(443.17900333,693.60798718)(443.19400332,693.71298708)(443.20400391,693.81299194)
\curveto(443.22400329,693.91298688)(443.24900326,694.01298678)(443.27900391,694.11299194)
\curveto(443.31900319,694.23298656)(443.35400316,694.34798644)(443.38400391,694.45799194)
\curveto(443.4140031,694.56798622)(443.45400306,694.67798611)(443.50400391,694.78799194)
\curveto(443.64400287,695.0879857)(443.81900269,695.37298542)(444.02900391,695.64299194)
\curveto(444.04900246,695.67298512)(444.07400244,695.69798509)(444.10400391,695.71799194)
\curveto(444.14400237,695.74798504)(444.17400234,695.77798501)(444.19400391,695.80799194)
\curveto(444.23400228,695.85798493)(444.27400224,695.90298489)(444.31400391,695.94299194)
\curveto(444.35400216,695.98298481)(444.39900211,696.02298477)(444.44900391,696.06299194)
\curveto(444.48900202,696.08298471)(444.52400199,696.10798468)(444.55400391,696.13799194)
\curveto(444.58400193,696.17798461)(444.61900189,696.20798458)(444.65900391,696.22799194)
\curveto(444.9090016,696.39798439)(445.19900131,696.53798425)(445.52900391,696.64799194)
\curveto(445.59900091,696.66798412)(445.66900084,696.68298411)(445.73900391,696.69299194)
\curveto(445.81900069,696.70298409)(445.89900061,696.71798407)(445.97900391,696.73799194)
\curveto(446.04900046,696.75798403)(446.13900037,696.76798402)(446.24900391,696.76799194)
\curveto(446.35900015,696.77798401)(446.46900004,696.78298401)(446.57900391,696.78299194)
\curveto(446.68899982,696.78298401)(446.79399972,696.77798401)(446.89400391,696.76799194)
\curveto(447.00399951,696.75798403)(447.09399942,696.74298405)(447.16400391,696.72299194)
\curveto(447.3139992,696.67298412)(447.45899905,696.62798416)(447.59900391,696.58799194)
\curveto(447.73899877,696.54798424)(447.86899864,696.4929843)(447.98900391,696.42299194)
\curveto(448.05899845,696.37298442)(448.12399839,696.32298447)(448.18400391,696.27299194)
\curveto(448.24399827,696.23298456)(448.3089982,696.1879846)(448.37900391,696.13799194)
\curveto(448.41899809,696.10798468)(448.47399804,696.06798472)(448.54400391,696.01799194)
\curveto(448.62399789,695.96798482)(448.69899781,695.96798482)(448.76900391,696.01799194)
\curveto(448.8089977,696.03798475)(448.82899768,696.07298472)(448.82900391,696.12299194)
\curveto(448.82899768,696.17298462)(448.83899767,696.22298457)(448.85900391,696.27299194)
\lineto(448.85900391,696.42299194)
\curveto(448.86899764,696.45298434)(448.87399764,696.4879843)(448.87400391,696.52799194)
\lineto(448.87400391,696.64799194)
\lineto(448.87400391,698.68799194)
\curveto(448.87399764,698.79798199)(448.86899764,698.91798187)(448.85900391,699.04799194)
\curveto(448.85899765,699.1879816)(448.88399763,699.2929815)(448.93400391,699.36299194)
\curveto(448.97399754,699.44298135)(449.04899746,699.4929813)(449.15900391,699.51299194)
\curveto(449.17899733,699.52298127)(449.19899731,699.52298127)(449.21900391,699.51299194)
\curveto(449.23899727,699.51298128)(449.25899725,699.51798127)(449.27900391,699.52799194)
\lineto(450.34400391,699.52799194)
\curveto(450.46399605,699.52798126)(450.57399594,699.52298127)(450.67400391,699.51299194)
\curveto(450.77399574,699.50298129)(450.84899566,699.46298133)(450.89900391,699.39299194)
\curveto(450.94899556,699.31298148)(450.97399554,699.20798158)(450.97400391,699.07799194)
\lineto(450.97400391,698.71799194)
\lineto(450.97400391,689.70299194)
\moveto(448.93400391,692.64299194)
\curveto(448.94399757,692.68298811)(448.94399757,692.72298807)(448.93400391,692.76299194)
\lineto(448.93400391,692.89799194)
\curveto(448.93399758,692.99798779)(448.92899758,693.09798769)(448.91900391,693.19799194)
\curveto(448.9089976,693.29798749)(448.89399762,693.3879874)(448.87400391,693.46799194)
\curveto(448.85399766,693.57798721)(448.83399768,693.67798711)(448.81400391,693.76799194)
\curveto(448.80399771,693.85798693)(448.77899773,693.94298685)(448.73900391,694.02299194)
\curveto(448.59899791,694.38298641)(448.39399812,694.66798612)(448.12400391,694.87799194)
\curveto(447.86399865,695.0879857)(447.48399903,695.1929856)(446.98400391,695.19299194)
\curveto(446.92399959,695.1929856)(446.84399967,695.18298561)(446.74400391,695.16299194)
\curveto(446.66399985,695.14298565)(446.58899992,695.12298567)(446.51900391,695.10299194)
\curveto(446.45900005,695.0929857)(446.39900011,695.07298572)(446.33900391,695.04299194)
\curveto(446.06900044,694.93298586)(445.85900065,694.76298603)(445.70900391,694.53299194)
\curveto(445.55900095,694.30298649)(445.43900107,694.04298675)(445.34900391,693.75299194)
\curveto(445.31900119,693.65298714)(445.29900121,693.55298724)(445.28900391,693.45299194)
\curveto(445.27900123,693.35298744)(445.25900125,693.24798754)(445.22900391,693.13799194)
\lineto(445.22900391,692.92799194)
\curveto(445.2090013,692.83798795)(445.20400131,692.71298808)(445.21400391,692.55299194)
\curveto(445.22400129,692.40298839)(445.23900127,692.2929885)(445.25900391,692.22299194)
\lineto(445.25900391,692.13299194)
\curveto(445.26900124,692.11298868)(445.27400124,692.0929887)(445.27400391,692.07299194)
\curveto(445.29400122,691.9929888)(445.3090012,691.91798887)(445.31900391,691.84799194)
\curveto(445.33900117,691.77798901)(445.35900115,691.70298909)(445.37900391,691.62299194)
\curveto(445.54900096,691.10298969)(445.83900067,690.71799007)(446.24900391,690.46799194)
\curveto(446.37900013,690.37799041)(446.55899995,690.30799048)(446.78900391,690.25799194)
\curveto(446.82899968,690.24799054)(446.88899962,690.24299055)(446.96900391,690.24299194)
\curveto(446.99899951,690.23299056)(447.04399947,690.22299057)(447.10400391,690.21299194)
\curveto(447.17399934,690.21299058)(447.22899928,690.21799057)(447.26900391,690.22799194)
\curveto(447.34899916,690.24799054)(447.42899908,690.26299053)(447.50900391,690.27299194)
\curveto(447.58899892,690.28299051)(447.66899884,690.30299049)(447.74900391,690.33299194)
\curveto(447.99899851,690.44299035)(448.19899831,690.58299021)(448.34900391,690.75299194)
\curveto(448.49899801,690.92298987)(448.62899788,691.13798965)(448.73900391,691.39799194)
\curveto(448.77899773,691.4879893)(448.8089977,691.57798921)(448.82900391,691.66799194)
\curveto(448.84899766,691.76798902)(448.86899764,691.87298892)(448.88900391,691.98299194)
\curveto(448.89899761,692.03298876)(448.89899761,692.07798871)(448.88900391,692.11799194)
\curveto(448.88899762,692.16798862)(448.89899761,692.21798857)(448.91900391,692.26799194)
\curveto(448.92899758,692.29798849)(448.93399758,692.33298846)(448.93400391,692.37299194)
\lineto(448.93400391,692.50799194)
\lineto(448.93400391,692.64299194)
}
}
{
\newrgbcolor{curcolor}{0 0 0}
\pscustom[linestyle=none,fillstyle=solid,fillcolor=curcolor]
{
\newpath
\moveto(460.32392578,693.03299194)
\curveto(460.34391721,692.97298782)(460.3539172,692.8879879)(460.35392578,692.77799194)
\curveto(460.3539172,692.66798812)(460.34391721,692.58298821)(460.32392578,692.52299194)
\lineto(460.32392578,692.37299194)
\curveto(460.30391725,692.2929885)(460.29391726,692.21298858)(460.29392578,692.13299194)
\curveto(460.30391725,692.05298874)(460.29891726,691.97298882)(460.27892578,691.89299194)
\curveto(460.2589173,691.82298897)(460.24391731,691.75798903)(460.23392578,691.69799194)
\curveto(460.22391733,691.63798915)(460.21391734,691.57298922)(460.20392578,691.50299194)
\curveto(460.16391739,691.3929894)(460.12891743,691.27798951)(460.09892578,691.15799194)
\curveto(460.06891749,691.04798974)(460.02891753,690.94298985)(459.97892578,690.84299194)
\curveto(459.76891779,690.36299043)(459.49391806,689.97299082)(459.15392578,689.67299194)
\curveto(458.81391874,689.37299142)(458.40391915,689.12299167)(457.92392578,688.92299194)
\curveto(457.80391975,688.87299192)(457.67891988,688.83799195)(457.54892578,688.81799194)
\curveto(457.42892013,688.787992)(457.30392025,688.75799203)(457.17392578,688.72799194)
\curveto(457.12392043,688.70799208)(457.06892049,688.69799209)(457.00892578,688.69799194)
\curveto(456.94892061,688.69799209)(456.89392066,688.6929921)(456.84392578,688.68299194)
\lineto(456.73892578,688.68299194)
\curveto(456.70892085,688.67299212)(456.67892088,688.66799212)(456.64892578,688.66799194)
\curveto(456.59892096,688.65799213)(456.51892104,688.65299214)(456.40892578,688.65299194)
\curveto(456.29892126,688.64299215)(456.21392134,688.64799214)(456.15392578,688.66799194)
\lineto(456.00392578,688.66799194)
\curveto(455.9539216,688.67799211)(455.89892166,688.68299211)(455.83892578,688.68299194)
\curveto(455.78892177,688.67299212)(455.73892182,688.67799211)(455.68892578,688.69799194)
\curveto(455.64892191,688.70799208)(455.60892195,688.71299208)(455.56892578,688.71299194)
\curveto(455.53892202,688.71299208)(455.49892206,688.71799207)(455.44892578,688.72799194)
\curveto(455.34892221,688.75799203)(455.24892231,688.78299201)(455.14892578,688.80299194)
\curveto(455.04892251,688.82299197)(454.9539226,688.85299194)(454.86392578,688.89299194)
\curveto(454.74392281,688.93299186)(454.62892293,688.97299182)(454.51892578,689.01299194)
\curveto(454.41892314,689.05299174)(454.31392324,689.10299169)(454.20392578,689.16299194)
\curveto(453.8539237,689.37299142)(453.553924,689.61799117)(453.30392578,689.89799194)
\curveto(453.0539245,690.17799061)(452.84392471,690.51299028)(452.67392578,690.90299194)
\curveto(452.62392493,690.9929898)(452.58392497,691.0879897)(452.55392578,691.18799194)
\curveto(452.53392502,691.2879895)(452.50892505,691.3929894)(452.47892578,691.50299194)
\curveto(452.4589251,691.55298924)(452.44892511,691.59798919)(452.44892578,691.63799194)
\curveto(452.44892511,691.67798911)(452.43892512,691.72298907)(452.41892578,691.77299194)
\curveto(452.39892516,691.85298894)(452.38892517,691.93298886)(452.38892578,692.01299194)
\curveto(452.38892517,692.10298869)(452.37892518,692.1879886)(452.35892578,692.26799194)
\curveto(452.34892521,692.31798847)(452.34392521,692.36298843)(452.34392578,692.40299194)
\lineto(452.34392578,692.53799194)
\curveto(452.32392523,692.59798819)(452.31392524,692.68298811)(452.31392578,692.79299194)
\curveto(452.32392523,692.90298789)(452.33892522,692.9879878)(452.35892578,693.04799194)
\lineto(452.35892578,693.15299194)
\curveto(452.36892519,693.20298759)(452.36892519,693.25298754)(452.35892578,693.30299194)
\curveto(452.3589252,693.36298743)(452.36892519,693.41798737)(452.38892578,693.46799194)
\curveto(452.39892516,693.51798727)(452.40392515,693.56298723)(452.40392578,693.60299194)
\curveto(452.40392515,693.65298714)(452.41392514,693.70298709)(452.43392578,693.75299194)
\curveto(452.47392508,693.88298691)(452.50892505,694.00798678)(452.53892578,694.12799194)
\curveto(452.56892499,694.25798653)(452.60892495,694.38298641)(452.65892578,694.50299194)
\curveto(452.83892472,694.91298588)(453.0539245,695.25298554)(453.30392578,695.52299194)
\curveto(453.553924,695.80298499)(453.8589237,696.05798473)(454.21892578,696.28799194)
\curveto(454.31892324,696.33798445)(454.42392313,696.38298441)(454.53392578,696.42299194)
\curveto(454.64392291,696.46298433)(454.7539228,696.50798428)(454.86392578,696.55799194)
\curveto(454.99392256,696.60798418)(455.12892243,696.64298415)(455.26892578,696.66299194)
\curveto(455.40892215,696.68298411)(455.553922,696.71298408)(455.70392578,696.75299194)
\curveto(455.78392177,696.76298403)(455.8589217,696.76798402)(455.92892578,696.76799194)
\curveto(455.99892156,696.76798402)(456.06892149,696.77298402)(456.13892578,696.78299194)
\curveto(456.71892084,696.792984)(457.21892034,696.73298406)(457.63892578,696.60299194)
\curveto(458.06891949,696.47298432)(458.44891911,696.2929845)(458.77892578,696.06299194)
\curveto(458.88891867,695.98298481)(458.99891856,695.8929849)(459.10892578,695.79299194)
\curveto(459.22891833,695.70298509)(459.32891823,695.60298519)(459.40892578,695.49299194)
\curveto(459.48891807,695.3929854)(459.558918,695.2929855)(459.61892578,695.19299194)
\curveto(459.68891787,695.0929857)(459.7589178,694.9879858)(459.82892578,694.87799194)
\curveto(459.89891766,694.76798602)(459.9539176,694.64798614)(459.99392578,694.51799194)
\curveto(460.03391752,694.39798639)(460.07891748,694.26798652)(460.12892578,694.12799194)
\curveto(460.1589174,694.04798674)(460.18391737,693.96298683)(460.20392578,693.87299194)
\lineto(460.26392578,693.60299194)
\curveto(460.27391728,693.56298723)(460.27891728,693.52298727)(460.27892578,693.48299194)
\curveto(460.27891728,693.44298735)(460.28391727,693.40298739)(460.29392578,693.36299194)
\curveto(460.31391724,693.31298748)(460.31891724,693.25798753)(460.30892578,693.19799194)
\curveto(460.29891726,693.13798765)(460.30391725,693.08298771)(460.32392578,693.03299194)
\moveto(458.22392578,692.49299194)
\curveto(458.23391932,692.54298825)(458.23891932,692.61298818)(458.23892578,692.70299194)
\curveto(458.23891932,692.80298799)(458.23391932,692.87798791)(458.22392578,692.92799194)
\lineto(458.22392578,693.04799194)
\curveto(458.20391935,693.09798769)(458.19391936,693.15298764)(458.19392578,693.21299194)
\curveto(458.19391936,693.27298752)(458.18891937,693.32798746)(458.17892578,693.37799194)
\curveto(458.17891938,693.41798737)(458.17391938,693.44798734)(458.16392578,693.46799194)
\lineto(458.10392578,693.70799194)
\curveto(458.09391946,693.79798699)(458.07391948,693.88298691)(458.04392578,693.96299194)
\curveto(457.93391962,694.22298657)(457.80391975,694.44298635)(457.65392578,694.62299194)
\curveto(457.50392005,694.81298598)(457.30392025,694.96298583)(457.05392578,695.07299194)
\curveto(456.99392056,695.0929857)(456.93392062,695.10798568)(456.87392578,695.11799194)
\curveto(456.81392074,695.13798565)(456.74892081,695.15798563)(456.67892578,695.17799194)
\curveto(456.59892096,695.19798559)(456.51392104,695.20298559)(456.42392578,695.19299194)
\lineto(456.15392578,695.19299194)
\curveto(456.12392143,695.17298562)(456.08892147,695.16298563)(456.04892578,695.16299194)
\curveto(456.00892155,695.17298562)(455.97392158,695.17298562)(455.94392578,695.16299194)
\lineto(455.73392578,695.10299194)
\curveto(455.67392188,695.0929857)(455.61892194,695.07298572)(455.56892578,695.04299194)
\curveto(455.31892224,694.93298586)(455.11392244,694.77298602)(454.95392578,694.56299194)
\curveto(454.80392275,694.36298643)(454.68392287,694.12798666)(454.59392578,693.85799194)
\curveto(454.56392299,693.75798703)(454.53892302,693.65298714)(454.51892578,693.54299194)
\curveto(454.50892305,693.43298736)(454.49392306,693.32298747)(454.47392578,693.21299194)
\curveto(454.46392309,693.16298763)(454.4589231,693.11298768)(454.45892578,693.06299194)
\lineto(454.45892578,692.91299194)
\curveto(454.43892312,692.84298795)(454.42892313,692.73798805)(454.42892578,692.59799194)
\curveto(454.43892312,692.45798833)(454.4539231,692.35298844)(454.47392578,692.28299194)
\lineto(454.47392578,692.14799194)
\curveto(454.49392306,692.06798872)(454.50892305,691.9879888)(454.51892578,691.90799194)
\curveto(454.52892303,691.83798895)(454.54392301,691.76298903)(454.56392578,691.68299194)
\curveto(454.66392289,691.38298941)(454.76892279,691.13798965)(454.87892578,690.94799194)
\curveto(454.99892256,690.76799002)(455.18392237,690.60299019)(455.43392578,690.45299194)
\curveto(455.50392205,690.40299039)(455.57892198,690.36299043)(455.65892578,690.33299194)
\curveto(455.74892181,690.30299049)(455.83892172,690.27799051)(455.92892578,690.25799194)
\curveto(455.96892159,690.24799054)(456.00392155,690.24299055)(456.03392578,690.24299194)
\curveto(456.06392149,690.25299054)(456.09892146,690.25299054)(456.13892578,690.24299194)
\lineto(456.25892578,690.21299194)
\curveto(456.30892125,690.21299058)(456.3539212,690.21799057)(456.39392578,690.22799194)
\lineto(456.51392578,690.22799194)
\curveto(456.59392096,690.24799054)(456.67392088,690.26299053)(456.75392578,690.27299194)
\curveto(456.83392072,690.28299051)(456.90892065,690.30299049)(456.97892578,690.33299194)
\curveto(457.23892032,690.43299036)(457.44892011,690.56799022)(457.60892578,690.73799194)
\curveto(457.76891979,690.90798988)(457.90391965,691.11798967)(458.01392578,691.36799194)
\curveto(458.0539195,691.46798932)(458.08391947,691.56798922)(458.10392578,691.66799194)
\curveto(458.12391943,691.76798902)(458.14891941,691.87298892)(458.17892578,691.98299194)
\curveto(458.18891937,692.02298877)(458.19391936,692.05798873)(458.19392578,692.08799194)
\curveto(458.19391936,692.12798866)(458.19891936,692.16798862)(458.20892578,692.20799194)
\lineto(458.20892578,692.34299194)
\curveto(458.20891935,692.3929884)(458.21391934,692.44298835)(458.22392578,692.49299194)
}
}
{
\newrgbcolor{curcolor}{0 0 0}
\pscustom[linestyle=none,fillstyle=solid,fillcolor=curcolor]
{
\newpath
\moveto(466.14884766,696.78299194)
\curveto(466.25884234,696.78298401)(466.35384225,696.77298402)(466.43384766,696.75299194)
\curveto(466.52384208,696.73298406)(466.59384201,696.6879841)(466.64384766,696.61799194)
\curveto(466.7038419,696.53798425)(466.73384187,696.39798439)(466.73384766,696.19799194)
\lineto(466.73384766,695.68799194)
\lineto(466.73384766,695.31299194)
\curveto(466.74384186,695.17298562)(466.72884187,695.06298573)(466.68884766,694.98299194)
\curveto(466.64884195,694.91298588)(466.58884201,694.86798592)(466.50884766,694.84799194)
\curveto(466.43884216,694.82798596)(466.35384225,694.81798597)(466.25384766,694.81799194)
\curveto(466.16384244,694.81798597)(466.06384254,694.82298597)(465.95384766,694.83299194)
\curveto(465.85384275,694.84298595)(465.75884284,694.83798595)(465.66884766,694.81799194)
\curveto(465.598843,694.79798599)(465.52884307,694.78298601)(465.45884766,694.77299194)
\curveto(465.38884321,694.77298602)(465.32384328,694.76298603)(465.26384766,694.74299194)
\curveto(465.1038435,694.6929861)(464.94384366,694.61798617)(464.78384766,694.51799194)
\curveto(464.62384398,694.42798636)(464.4988441,694.32298647)(464.40884766,694.20299194)
\curveto(464.35884424,694.12298667)(464.3038443,694.03798675)(464.24384766,693.94799194)
\curveto(464.19384441,693.86798692)(464.14384446,693.78298701)(464.09384766,693.69299194)
\curveto(464.06384454,693.61298718)(464.03384457,693.52798726)(464.00384766,693.43799194)
\lineto(463.94384766,693.19799194)
\curveto(463.92384468,693.12798766)(463.91384469,693.05298774)(463.91384766,692.97299194)
\curveto(463.91384469,692.90298789)(463.9038447,692.83298796)(463.88384766,692.76299194)
\curveto(463.87384473,692.72298807)(463.86884473,692.68298811)(463.86884766,692.64299194)
\curveto(463.87884472,692.61298818)(463.87884472,692.58298821)(463.86884766,692.55299194)
\lineto(463.86884766,692.31299194)
\curveto(463.84884475,692.24298855)(463.84384476,692.16298863)(463.85384766,692.07299194)
\curveto(463.86384474,691.9929888)(463.86884473,691.91298888)(463.86884766,691.83299194)
\lineto(463.86884766,690.87299194)
\lineto(463.86884766,689.59799194)
\curveto(463.86884473,689.46799132)(463.86384474,689.34799144)(463.85384766,689.23799194)
\curveto(463.84384476,689.12799166)(463.81384479,689.03799175)(463.76384766,688.96799194)
\curveto(463.74384486,688.93799185)(463.70884489,688.91299188)(463.65884766,688.89299194)
\curveto(463.61884498,688.88299191)(463.57384503,688.87299192)(463.52384766,688.86299194)
\lineto(463.44884766,688.86299194)
\curveto(463.3988452,688.85299194)(463.34384526,688.84799194)(463.28384766,688.84799194)
\lineto(463.11884766,688.84799194)
\lineto(462.47384766,688.84799194)
\curveto(462.41384619,688.85799193)(462.34884625,688.86299193)(462.27884766,688.86299194)
\lineto(462.08384766,688.86299194)
\curveto(462.03384657,688.88299191)(461.98384662,688.89799189)(461.93384766,688.90799194)
\curveto(461.88384672,688.92799186)(461.84884675,688.96299183)(461.82884766,689.01299194)
\curveto(461.78884681,689.06299173)(461.76384684,689.13299166)(461.75384766,689.22299194)
\lineto(461.75384766,689.52299194)
\lineto(461.75384766,690.54299194)
\lineto(461.75384766,694.77299194)
\lineto(461.75384766,695.88299194)
\lineto(461.75384766,696.16799194)
\curveto(461.75384685,696.26798452)(461.77384683,696.34798444)(461.81384766,696.40799194)
\curveto(461.86384674,696.4879843)(461.93884666,696.53798425)(462.03884766,696.55799194)
\curveto(462.13884646,696.57798421)(462.25884634,696.5879842)(462.39884766,696.58799194)
\lineto(463.16384766,696.58799194)
\curveto(463.28384532,696.5879842)(463.38884521,696.57798421)(463.47884766,696.55799194)
\curveto(463.56884503,696.54798424)(463.63884496,696.50298429)(463.68884766,696.42299194)
\curveto(463.71884488,696.37298442)(463.73384487,696.30298449)(463.73384766,696.21299194)
\lineto(463.76384766,695.94299194)
\curveto(463.77384483,695.86298493)(463.78884481,695.787985)(463.80884766,695.71799194)
\curveto(463.83884476,695.64798514)(463.88884471,695.61298518)(463.95884766,695.61299194)
\curveto(463.97884462,695.63298516)(463.9988446,695.64298515)(464.01884766,695.64299194)
\curveto(464.03884456,695.64298515)(464.05884454,695.65298514)(464.07884766,695.67299194)
\curveto(464.13884446,695.72298507)(464.18884441,695.77798501)(464.22884766,695.83799194)
\curveto(464.27884432,695.90798488)(464.33884426,695.96798482)(464.40884766,696.01799194)
\curveto(464.44884415,696.04798474)(464.48384412,696.07798471)(464.51384766,696.10799194)
\curveto(464.54384406,696.14798464)(464.57884402,696.18298461)(464.61884766,696.21299194)
\lineto(464.88884766,696.39299194)
\curveto(464.98884361,696.45298434)(465.08884351,696.50798428)(465.18884766,696.55799194)
\curveto(465.28884331,696.59798419)(465.38884321,696.63298416)(465.48884766,696.66299194)
\lineto(465.81884766,696.75299194)
\curveto(465.84884275,696.76298403)(465.9038427,696.76298403)(465.98384766,696.75299194)
\curveto(466.07384253,696.75298404)(466.12884247,696.76298403)(466.14884766,696.78299194)
}
}
{
\newrgbcolor{curcolor}{0 0 0}
\pscustom[linestyle=none,fillstyle=solid,fillcolor=curcolor]
{
\newpath
\moveto(474.65525391,692.79299194)
\curveto(474.67524574,692.71298808)(474.67524574,692.62298817)(474.65525391,692.52299194)
\curveto(474.63524578,692.42298837)(474.60024582,692.35798843)(474.55025391,692.32799194)
\curveto(474.50024592,692.2879885)(474.42524599,692.25798853)(474.32525391,692.23799194)
\curveto(474.23524618,692.22798856)(474.13024629,692.21798857)(474.01025391,692.20799194)
\lineto(473.66525391,692.20799194)
\curveto(473.55524686,692.21798857)(473.45524696,692.22298857)(473.36525391,692.22299194)
\lineto(469.70525391,692.22299194)
\lineto(469.49525391,692.22299194)
\curveto(469.43525098,692.22298857)(469.38025104,692.21298858)(469.33025391,692.19299194)
\curveto(469.25025117,692.15298864)(469.20025122,692.11298868)(469.18025391,692.07299194)
\curveto(469.16025126,692.05298874)(469.14025128,692.01298878)(469.12025391,691.95299194)
\curveto(469.10025132,691.90298889)(469.09525132,691.85298894)(469.10525391,691.80299194)
\curveto(469.12525129,691.74298905)(469.13525128,691.68298911)(469.13525391,691.62299194)
\curveto(469.14525127,691.57298922)(469.16025126,691.51798927)(469.18025391,691.45799194)
\curveto(469.26025116,691.21798957)(469.35525106,691.01798977)(469.46525391,690.85799194)
\curveto(469.58525083,690.70799008)(469.74525067,690.57299022)(469.94525391,690.45299194)
\curveto(470.02525039,690.40299039)(470.10525031,690.36799042)(470.18525391,690.34799194)
\curveto(470.27525014,690.33799045)(470.36525005,690.31799047)(470.45525391,690.28799194)
\curveto(470.53524988,690.26799052)(470.64524977,690.25299054)(470.78525391,690.24299194)
\curveto(470.92524949,690.23299056)(471.04524937,690.23799055)(471.14525391,690.25799194)
\lineto(471.28025391,690.25799194)
\curveto(471.38024904,690.27799051)(471.47024895,690.29799049)(471.55025391,690.31799194)
\curveto(471.64024878,690.34799044)(471.72524869,690.37799041)(471.80525391,690.40799194)
\curveto(471.90524851,690.45799033)(472.0152484,690.52299027)(472.13525391,690.60299194)
\curveto(472.26524815,690.68299011)(472.36024806,690.76299003)(472.42025391,690.84299194)
\curveto(472.47024795,690.91298988)(472.5202479,690.97798981)(472.57025391,691.03799194)
\curveto(472.63024779,691.10798968)(472.70024772,691.15798963)(472.78025391,691.18799194)
\curveto(472.88024754,691.23798955)(473.00524741,691.25798953)(473.15525391,691.24799194)
\lineto(473.59025391,691.24799194)
\lineto(473.77025391,691.24799194)
\curveto(473.84024658,691.25798953)(473.90024652,691.25298954)(473.95025391,691.23299194)
\lineto(474.10025391,691.23299194)
\curveto(474.20024622,691.21298958)(474.27024615,691.1879896)(474.31025391,691.15799194)
\curveto(474.35024607,691.13798965)(474.37024605,691.0929897)(474.37025391,691.02299194)
\curveto(474.38024604,690.95298984)(474.37524604,690.8929899)(474.35525391,690.84299194)
\curveto(474.30524611,690.70299009)(474.25024617,690.57799021)(474.19025391,690.46799194)
\curveto(474.13024629,690.35799043)(474.06024636,690.24799054)(473.98025391,690.13799194)
\curveto(473.76024666,689.80799098)(473.51024691,689.54299125)(473.23025391,689.34299194)
\curveto(472.95024747,689.14299165)(472.60024782,688.97299182)(472.18025391,688.83299194)
\curveto(472.07024835,688.792992)(471.96024846,688.76799202)(471.85025391,688.75799194)
\curveto(471.74024868,688.74799204)(471.62524879,688.72799206)(471.50525391,688.69799194)
\curveto(471.46524895,688.6879921)(471.420249,688.6879921)(471.37025391,688.69799194)
\curveto(471.33024909,688.69799209)(471.29024913,688.6929921)(471.25025391,688.68299194)
\lineto(471.08525391,688.68299194)
\curveto(471.03524938,688.66299213)(470.97524944,688.65799213)(470.90525391,688.66799194)
\curveto(470.84524957,688.66799212)(470.79024963,688.67299212)(470.74025391,688.68299194)
\curveto(470.66024976,688.6929921)(470.59024983,688.6929921)(470.53025391,688.68299194)
\curveto(470.47024995,688.67299212)(470.40525001,688.67799211)(470.33525391,688.69799194)
\curveto(470.28525013,688.71799207)(470.23025019,688.72799206)(470.17025391,688.72799194)
\curveto(470.11025031,688.72799206)(470.05525036,688.73799205)(470.00525391,688.75799194)
\curveto(469.89525052,688.77799201)(469.78525063,688.80299199)(469.67525391,688.83299194)
\curveto(469.56525085,688.85299194)(469.46525095,688.8879919)(469.37525391,688.93799194)
\curveto(469.26525115,688.97799181)(469.16025126,689.01299178)(469.06025391,689.04299194)
\curveto(468.97025145,689.08299171)(468.88525153,689.12799166)(468.80525391,689.17799194)
\curveto(468.48525193,689.37799141)(468.20025222,689.60799118)(467.95025391,689.86799194)
\curveto(467.70025272,690.13799065)(467.49525292,690.44799034)(467.33525391,690.79799194)
\curveto(467.28525313,690.90798988)(467.24525317,691.01798977)(467.21525391,691.12799194)
\curveto(467.18525323,691.24798954)(467.14525327,691.36798942)(467.09525391,691.48799194)
\curveto(467.08525333,691.52798926)(467.08025334,691.56298923)(467.08025391,691.59299194)
\curveto(467.08025334,691.63298916)(467.07525334,691.67298912)(467.06525391,691.71299194)
\curveto(467.02525339,691.83298896)(467.00025342,691.96298883)(466.99025391,692.10299194)
\lineto(466.96025391,692.52299194)
\curveto(466.96025346,692.57298822)(466.95525346,692.62798816)(466.94525391,692.68799194)
\curveto(466.94525347,692.74798804)(466.95025347,692.80298799)(466.96025391,692.85299194)
\lineto(466.96025391,693.03299194)
\lineto(467.00525391,693.39299194)
\curveto(467.04525337,693.56298723)(467.08025334,693.72798706)(467.11025391,693.88799194)
\curveto(467.14025328,694.04798674)(467.18525323,694.19798659)(467.24525391,694.33799194)
\curveto(467.67525274,695.37798541)(468.40525201,696.11298468)(469.43525391,696.54299194)
\curveto(469.57525084,696.60298419)(469.7152507,696.64298415)(469.85525391,696.66299194)
\curveto(470.00525041,696.6929841)(470.16025026,696.72798406)(470.32025391,696.76799194)
\curveto(470.40025002,696.77798401)(470.47524994,696.78298401)(470.54525391,696.78299194)
\curveto(470.6152498,696.78298401)(470.69024973,696.787984)(470.77025391,696.79799194)
\curveto(471.28024914,696.80798398)(471.7152487,696.74798404)(472.07525391,696.61799194)
\curveto(472.44524797,696.49798429)(472.77524764,696.33798445)(473.06525391,696.13799194)
\curveto(473.15524726,696.07798471)(473.24524717,696.00798478)(473.33525391,695.92799194)
\curveto(473.42524699,695.85798493)(473.50524691,695.78298501)(473.57525391,695.70299194)
\curveto(473.60524681,695.65298514)(473.64524677,695.61298518)(473.69525391,695.58299194)
\curveto(473.77524664,695.47298532)(473.85024657,695.35798543)(473.92025391,695.23799194)
\curveto(473.99024643,695.12798566)(474.06524635,695.01298578)(474.14525391,694.89299194)
\curveto(474.19524622,694.80298599)(474.23524618,694.70798608)(474.26525391,694.60799194)
\curveto(474.30524611,694.51798627)(474.34524607,694.41798637)(474.38525391,694.30799194)
\curveto(474.43524598,694.17798661)(474.47524594,694.04298675)(474.50525391,693.90299194)
\curveto(474.53524588,693.76298703)(474.57024585,693.62298717)(474.61025391,693.48299194)
\curveto(474.63024579,693.40298739)(474.63524578,693.31298748)(474.62525391,693.21299194)
\curveto(474.62524579,693.12298767)(474.63524578,693.03798775)(474.65525391,692.95799194)
\lineto(474.65525391,692.79299194)
\moveto(472.40525391,693.67799194)
\curveto(472.47524794,693.77798701)(472.48024794,693.89798689)(472.42025391,694.03799194)
\curveto(472.37024805,694.1879866)(472.33024809,694.29798649)(472.30025391,694.36799194)
\curveto(472.16024826,694.63798615)(471.97524844,694.84298595)(471.74525391,694.98299194)
\curveto(471.5152489,695.13298566)(471.19524922,695.21298558)(470.78525391,695.22299194)
\curveto(470.75524966,695.20298559)(470.7202497,695.19798559)(470.68025391,695.20799194)
\curveto(470.64024978,695.21798557)(470.60524981,695.21798557)(470.57525391,695.20799194)
\curveto(470.52524989,695.1879856)(470.47024995,695.17298562)(470.41025391,695.16299194)
\curveto(470.35025007,695.16298563)(470.29525012,695.15298564)(470.24525391,695.13299194)
\curveto(469.80525061,694.9929858)(469.48025094,694.71798607)(469.27025391,694.30799194)
\curveto(469.25025117,694.26798652)(469.22525119,694.21298658)(469.19525391,694.14299194)
\curveto(469.17525124,694.08298671)(469.16025126,694.01798677)(469.15025391,693.94799194)
\curveto(469.14025128,693.8879869)(469.14025128,693.82798696)(469.15025391,693.76799194)
\curveto(469.17025125,693.70798708)(469.20525121,693.65798713)(469.25525391,693.61799194)
\curveto(469.33525108,693.56798722)(469.44525097,693.54298725)(469.58525391,693.54299194)
\lineto(469.99025391,693.54299194)
\lineto(471.65525391,693.54299194)
\lineto(472.09025391,693.54299194)
\curveto(472.25024817,693.55298724)(472.35524806,693.59798719)(472.40525391,693.67799194)
}
}
{
\newrgbcolor{curcolor}{0 0 0}
\pscustom[linestyle=none,fillstyle=solid,fillcolor=curcolor]
{
\newpath
\moveto(478.87353516,696.79799194)
\curveto(479.62353066,696.81798397)(480.27353001,696.73298406)(480.82353516,696.54299194)
\curveto(481.3835289,696.36298443)(481.80852847,696.04798474)(482.09853516,695.59799194)
\curveto(482.16852811,695.4879853)(482.22852805,695.37298542)(482.27853516,695.25299194)
\curveto(482.33852794,695.14298565)(482.38852789,695.01798577)(482.42853516,694.87799194)
\curveto(482.44852783,694.81798597)(482.45852782,694.75298604)(482.45853516,694.68299194)
\curveto(482.45852782,694.61298618)(482.44852783,694.55298624)(482.42853516,694.50299194)
\curveto(482.38852789,694.44298635)(482.33352795,694.40298639)(482.26353516,694.38299194)
\curveto(482.21352807,694.36298643)(482.15352813,694.35298644)(482.08353516,694.35299194)
\lineto(481.87353516,694.35299194)
\lineto(481.21353516,694.35299194)
\curveto(481.14352914,694.35298644)(481.07352921,694.34798644)(481.00353516,694.33799194)
\curveto(480.93352935,694.33798645)(480.86852941,694.34798644)(480.80853516,694.36799194)
\curveto(480.70852957,694.3879864)(480.63352965,694.42798636)(480.58353516,694.48799194)
\curveto(480.53352975,694.54798624)(480.48852979,694.60798618)(480.44853516,694.66799194)
\lineto(480.32853516,694.87799194)
\curveto(480.29852998,694.95798583)(480.24853003,695.02298577)(480.17853516,695.07299194)
\curveto(480.0785302,695.15298564)(479.9785303,695.21298558)(479.87853516,695.25299194)
\curveto(479.78853049,695.2929855)(479.67353061,695.32798546)(479.53353516,695.35799194)
\curveto(479.46353082,695.37798541)(479.35853092,695.3929854)(479.21853516,695.40299194)
\curveto(479.08853119,695.41298538)(478.98853129,695.40798538)(478.91853516,695.38799194)
\lineto(478.81353516,695.38799194)
\lineto(478.66353516,695.35799194)
\curveto(478.62353166,695.35798543)(478.5785317,695.35298544)(478.52853516,695.34299194)
\curveto(478.35853192,695.2929855)(478.21853206,695.22298557)(478.10853516,695.13299194)
\curveto(478.00853227,695.05298574)(477.93853234,694.92798586)(477.89853516,694.75799194)
\curveto(477.8785324,694.6879861)(477.8785324,694.62298617)(477.89853516,694.56299194)
\curveto(477.91853236,694.50298629)(477.93853234,694.45298634)(477.95853516,694.41299194)
\curveto(478.02853225,694.2929865)(478.10853217,694.19798659)(478.19853516,694.12799194)
\curveto(478.29853198,694.05798673)(478.41353187,693.99798679)(478.54353516,693.94799194)
\curveto(478.73353155,693.86798692)(478.93853134,693.79798699)(479.15853516,693.73799194)
\lineto(479.84853516,693.58799194)
\curveto(480.08853019,693.54798724)(480.31852996,693.49798729)(480.53853516,693.43799194)
\curveto(480.76852951,693.3879874)(480.9835293,693.32298747)(481.18353516,693.24299194)
\curveto(481.27352901,693.20298759)(481.35852892,693.16798762)(481.43853516,693.13799194)
\curveto(481.52852875,693.11798767)(481.61352867,693.08298771)(481.69353516,693.03299194)
\curveto(481.8835284,692.91298788)(482.05352823,692.78298801)(482.20353516,692.64299194)
\curveto(482.36352792,692.50298829)(482.48852779,692.32798846)(482.57853516,692.11799194)
\curveto(482.60852767,692.04798874)(482.63352765,691.97798881)(482.65353516,691.90799194)
\curveto(482.67352761,691.83798895)(482.69352759,691.76298903)(482.71353516,691.68299194)
\curveto(482.72352756,691.62298917)(482.72852755,691.52798926)(482.72853516,691.39799194)
\curveto(482.73852754,691.27798951)(482.73852754,691.18298961)(482.72853516,691.11299194)
\lineto(482.72853516,691.03799194)
\curveto(482.70852757,690.97798981)(482.69352759,690.91798987)(482.68353516,690.85799194)
\curveto(482.6835276,690.80798998)(482.6785276,690.75799003)(482.66853516,690.70799194)
\curveto(482.59852768,690.40799038)(482.48852779,690.14299065)(482.33853516,689.91299194)
\curveto(482.1785281,689.67299112)(481.9835283,689.47799131)(481.75353516,689.32799194)
\curveto(481.52352876,689.17799161)(481.26352902,689.04799174)(480.97353516,688.93799194)
\curveto(480.86352942,688.8879919)(480.74352954,688.85299194)(480.61353516,688.83299194)
\curveto(480.49352979,688.81299198)(480.37352991,688.787992)(480.25353516,688.75799194)
\curveto(480.16353012,688.73799205)(480.06853021,688.72799206)(479.96853516,688.72799194)
\curveto(479.8785304,688.71799207)(479.78853049,688.70299209)(479.69853516,688.68299194)
\lineto(479.42853516,688.68299194)
\curveto(479.36853091,688.66299213)(479.26353102,688.65299214)(479.11353516,688.65299194)
\curveto(478.97353131,688.65299214)(478.87353141,688.66299213)(478.81353516,688.68299194)
\curveto(478.7835315,688.68299211)(478.74853153,688.6879921)(478.70853516,688.69799194)
\lineto(478.60353516,688.69799194)
\curveto(478.4835318,688.71799207)(478.36353192,688.73299206)(478.24353516,688.74299194)
\curveto(478.12353216,688.75299204)(478.00853227,688.77299202)(477.89853516,688.80299194)
\curveto(477.50853277,688.91299188)(477.16353312,689.03799175)(476.86353516,689.17799194)
\curveto(476.56353372,689.32799146)(476.30853397,689.54799124)(476.09853516,689.83799194)
\curveto(475.95853432,690.02799076)(475.83853444,690.24799054)(475.73853516,690.49799194)
\curveto(475.71853456,690.55799023)(475.69853458,690.63799015)(475.67853516,690.73799194)
\curveto(475.65853462,690.78799)(475.64353464,690.85798993)(475.63353516,690.94799194)
\curveto(475.62353466,691.03798975)(475.62853465,691.11298968)(475.64853516,691.17299194)
\curveto(475.6785346,691.24298955)(475.72853455,691.2929895)(475.79853516,691.32299194)
\curveto(475.84853443,691.34298945)(475.90853437,691.35298944)(475.97853516,691.35299194)
\lineto(476.20353516,691.35299194)
\lineto(476.90853516,691.35299194)
\lineto(477.14853516,691.35299194)
\curveto(477.22853305,691.35298944)(477.29853298,691.34298945)(477.35853516,691.32299194)
\curveto(477.46853281,691.28298951)(477.53853274,691.21798957)(477.56853516,691.12799194)
\curveto(477.60853267,691.03798975)(477.65353263,690.94298985)(477.70353516,690.84299194)
\curveto(477.72353256,690.79299)(477.75853252,690.72799006)(477.80853516,690.64799194)
\curveto(477.86853241,690.56799022)(477.91853236,690.51799027)(477.95853516,690.49799194)
\curveto(478.0785322,690.39799039)(478.19353209,690.31799047)(478.30353516,690.25799194)
\curveto(478.41353187,690.20799058)(478.55353173,690.15799063)(478.72353516,690.10799194)
\curveto(478.77353151,690.0879907)(478.82353146,690.07799071)(478.87353516,690.07799194)
\curveto(478.92353136,690.0879907)(478.97353131,690.0879907)(479.02353516,690.07799194)
\curveto(479.10353118,690.05799073)(479.18853109,690.04799074)(479.27853516,690.04799194)
\curveto(479.3785309,690.05799073)(479.46353082,690.07299072)(479.53353516,690.09299194)
\curveto(479.5835307,690.10299069)(479.62853065,690.10799068)(479.66853516,690.10799194)
\curveto(479.71853056,690.10799068)(479.76853051,690.11799067)(479.81853516,690.13799194)
\curveto(479.95853032,690.1879906)(480.0835302,690.24799054)(480.19353516,690.31799194)
\curveto(480.31352997,690.3879904)(480.40852987,690.47799031)(480.47853516,690.58799194)
\curveto(480.52852975,690.66799012)(480.56852971,690.79299)(480.59853516,690.96299194)
\curveto(480.61852966,691.03298976)(480.61852966,691.09798969)(480.59853516,691.15799194)
\curveto(480.5785297,691.21798957)(480.55852972,691.26798952)(480.53853516,691.30799194)
\curveto(480.46852981,691.44798934)(480.3785299,691.55298924)(480.26853516,691.62299194)
\curveto(480.16853011,691.6929891)(480.04853023,691.75798903)(479.90853516,691.81799194)
\curveto(479.71853056,691.89798889)(479.51853076,691.96298883)(479.30853516,692.01299194)
\curveto(479.09853118,692.06298873)(478.88853139,692.11798867)(478.67853516,692.17799194)
\curveto(478.59853168,692.19798859)(478.51353177,692.21298858)(478.42353516,692.22299194)
\curveto(478.34353194,692.23298856)(478.26353202,692.24798854)(478.18353516,692.26799194)
\curveto(477.86353242,692.35798843)(477.55853272,692.44298835)(477.26853516,692.52299194)
\curveto(476.9785333,692.61298818)(476.71353357,692.74298805)(476.47353516,692.91299194)
\curveto(476.19353409,693.11298768)(475.98853429,693.38298741)(475.85853516,693.72299194)
\curveto(475.83853444,693.792987)(475.81853446,693.8879869)(475.79853516,694.00799194)
\curveto(475.7785345,694.07798671)(475.76353452,694.16298663)(475.75353516,694.26299194)
\curveto(475.74353454,694.36298643)(475.74853453,694.45298634)(475.76853516,694.53299194)
\curveto(475.78853449,694.58298621)(475.79353449,694.62298617)(475.78353516,694.65299194)
\curveto(475.77353451,694.6929861)(475.7785345,694.73798605)(475.79853516,694.78799194)
\curveto(475.81853446,694.89798589)(475.83853444,694.99798579)(475.85853516,695.08799194)
\curveto(475.88853439,695.1879856)(475.92353436,695.28298551)(475.96353516,695.37299194)
\curveto(476.09353419,695.66298513)(476.27353401,695.89798489)(476.50353516,696.07799194)
\curveto(476.73353355,696.25798453)(476.99353329,696.40298439)(477.28353516,696.51299194)
\curveto(477.39353289,696.56298423)(477.50853277,696.59798419)(477.62853516,696.61799194)
\curveto(477.74853253,696.64798414)(477.87353241,696.67798411)(478.00353516,696.70799194)
\curveto(478.06353222,696.72798406)(478.12353216,696.73798405)(478.18353516,696.73799194)
\lineto(478.36353516,696.76799194)
\curveto(478.44353184,696.77798401)(478.52853175,696.78298401)(478.61853516,696.78299194)
\curveto(478.70853157,696.78298401)(478.79353149,696.787984)(478.87353516,696.79799194)
}
}
{
\newrgbcolor{curcolor}{0 0 0}
\pscustom[linestyle=none,fillstyle=solid,fillcolor=curcolor]
{
}
}
{
\newrgbcolor{curcolor}{0 0 0}
\pscustom[linestyle=none,fillstyle=solid,fillcolor=curcolor]
{
\newpath
\moveto(495.71033203,689.70299194)
\lineto(495.71033203,689.28299194)
\curveto(495.71032366,689.15299164)(495.68032369,689.04799174)(495.62033203,688.96799194)
\curveto(495.5703238,688.91799187)(495.50532387,688.88299191)(495.42533203,688.86299194)
\curveto(495.34532403,688.85299194)(495.25532412,688.84799194)(495.15533203,688.84799194)
\lineto(494.33033203,688.84799194)
\lineto(494.04533203,688.84799194)
\curveto(493.96532541,688.85799193)(493.90032547,688.88299191)(493.85033203,688.92299194)
\curveto(493.78032559,688.97299182)(493.74032563,689.03799175)(493.73033203,689.11799194)
\curveto(493.72032565,689.19799159)(493.70032567,689.27799151)(493.67033203,689.35799194)
\curveto(493.65032572,689.37799141)(493.63032574,689.3929914)(493.61033203,689.40299194)
\curveto(493.60032577,689.42299137)(493.58532579,689.44299135)(493.56533203,689.46299194)
\curveto(493.45532592,689.46299133)(493.375326,689.43799135)(493.32533203,689.38799194)
\lineto(493.17533203,689.23799194)
\curveto(493.10532627,689.1879916)(493.04032633,689.14299165)(492.98033203,689.10299194)
\curveto(492.92032645,689.07299172)(492.85532652,689.03299176)(492.78533203,688.98299194)
\curveto(492.74532663,688.96299183)(492.70032667,688.94299185)(492.65033203,688.92299194)
\curveto(492.61032676,688.90299189)(492.56532681,688.88299191)(492.51533203,688.86299194)
\curveto(492.375327,688.81299198)(492.22532715,688.76799202)(492.06533203,688.72799194)
\curveto(492.01532736,688.70799208)(491.9703274,688.69799209)(491.93033203,688.69799194)
\curveto(491.89032748,688.69799209)(491.85032752,688.6929921)(491.81033203,688.68299194)
\lineto(491.67533203,688.68299194)
\curveto(491.64532773,688.67299212)(491.60532777,688.66799212)(491.55533203,688.66799194)
\lineto(491.42033203,688.66799194)
\curveto(491.36032801,688.64799214)(491.2703281,688.64299215)(491.15033203,688.65299194)
\curveto(491.03032834,688.65299214)(490.94532843,688.66299213)(490.89533203,688.68299194)
\curveto(490.82532855,688.70299209)(490.76032861,688.71299208)(490.70033203,688.71299194)
\curveto(490.65032872,688.70299209)(490.59532878,688.70799208)(490.53533203,688.72799194)
\lineto(490.17533203,688.84799194)
\curveto(490.06532931,688.87799191)(489.95532942,688.91799187)(489.84533203,688.96799194)
\curveto(489.49532988,689.11799167)(489.18033019,689.34799144)(488.90033203,689.65799194)
\curveto(488.63033074,689.97799081)(488.41533096,690.31299048)(488.25533203,690.66299194)
\curveto(488.20533117,690.77299002)(488.16533121,690.87798991)(488.13533203,690.97799194)
\curveto(488.10533127,691.0879897)(488.0703313,691.19798959)(488.03033203,691.30799194)
\curveto(488.02033135,691.34798944)(488.01533136,691.38298941)(488.01533203,691.41299194)
\curveto(488.01533136,691.45298934)(488.00533137,691.49798929)(487.98533203,691.54799194)
\curveto(487.96533141,691.62798916)(487.94533143,691.71298908)(487.92533203,691.80299194)
\curveto(487.91533146,691.90298889)(487.90033147,692.00298879)(487.88033203,692.10299194)
\curveto(487.8703315,692.13298866)(487.86533151,692.16798862)(487.86533203,692.20799194)
\curveto(487.8753315,692.24798854)(487.8753315,692.28298851)(487.86533203,692.31299194)
\lineto(487.86533203,692.44799194)
\curveto(487.86533151,692.49798829)(487.86033151,692.54798824)(487.85033203,692.59799194)
\curveto(487.84033153,692.64798814)(487.83533154,692.70298809)(487.83533203,692.76299194)
\curveto(487.83533154,692.83298796)(487.84033153,692.8879879)(487.85033203,692.92799194)
\curveto(487.86033151,692.97798781)(487.86533151,693.02298777)(487.86533203,693.06299194)
\lineto(487.86533203,693.21299194)
\curveto(487.8753315,693.26298753)(487.8753315,693.30798748)(487.86533203,693.34799194)
\curveto(487.86533151,693.39798739)(487.8753315,693.44798734)(487.89533203,693.49799194)
\curveto(487.91533146,693.60798718)(487.93033144,693.71298708)(487.94033203,693.81299194)
\curveto(487.96033141,693.91298688)(487.98533139,694.01298678)(488.01533203,694.11299194)
\curveto(488.05533132,694.23298656)(488.09033128,694.34798644)(488.12033203,694.45799194)
\curveto(488.15033122,694.56798622)(488.19033118,694.67798611)(488.24033203,694.78799194)
\curveto(488.38033099,695.0879857)(488.55533082,695.37298542)(488.76533203,695.64299194)
\curveto(488.78533059,695.67298512)(488.81033056,695.69798509)(488.84033203,695.71799194)
\curveto(488.88033049,695.74798504)(488.91033046,695.77798501)(488.93033203,695.80799194)
\curveto(488.9703304,695.85798493)(489.01033036,695.90298489)(489.05033203,695.94299194)
\curveto(489.09033028,695.98298481)(489.13533024,696.02298477)(489.18533203,696.06299194)
\curveto(489.22533015,696.08298471)(489.26033011,696.10798468)(489.29033203,696.13799194)
\curveto(489.32033005,696.17798461)(489.35533002,696.20798458)(489.39533203,696.22799194)
\curveto(489.64532973,696.39798439)(489.93532944,696.53798425)(490.26533203,696.64799194)
\curveto(490.33532904,696.66798412)(490.40532897,696.68298411)(490.47533203,696.69299194)
\curveto(490.55532882,696.70298409)(490.63532874,696.71798407)(490.71533203,696.73799194)
\curveto(490.78532859,696.75798403)(490.8753285,696.76798402)(490.98533203,696.76799194)
\curveto(491.09532828,696.77798401)(491.20532817,696.78298401)(491.31533203,696.78299194)
\curveto(491.42532795,696.78298401)(491.53032784,696.77798401)(491.63033203,696.76799194)
\curveto(491.74032763,696.75798403)(491.83032754,696.74298405)(491.90033203,696.72299194)
\curveto(492.05032732,696.67298412)(492.19532718,696.62798416)(492.33533203,696.58799194)
\curveto(492.4753269,696.54798424)(492.60532677,696.4929843)(492.72533203,696.42299194)
\curveto(492.79532658,696.37298442)(492.86032651,696.32298447)(492.92033203,696.27299194)
\curveto(492.98032639,696.23298456)(493.04532633,696.1879846)(493.11533203,696.13799194)
\curveto(493.15532622,696.10798468)(493.21032616,696.06798472)(493.28033203,696.01799194)
\curveto(493.36032601,695.96798482)(493.43532594,695.96798482)(493.50533203,696.01799194)
\curveto(493.54532583,696.03798475)(493.56532581,696.07298472)(493.56533203,696.12299194)
\curveto(493.56532581,696.17298462)(493.5753258,696.22298457)(493.59533203,696.27299194)
\lineto(493.59533203,696.42299194)
\curveto(493.60532577,696.45298434)(493.61032576,696.4879843)(493.61033203,696.52799194)
\lineto(493.61033203,696.64799194)
\lineto(493.61033203,698.68799194)
\curveto(493.61032576,698.79798199)(493.60532577,698.91798187)(493.59533203,699.04799194)
\curveto(493.59532578,699.1879816)(493.62032575,699.2929815)(493.67033203,699.36299194)
\curveto(493.71032566,699.44298135)(493.78532559,699.4929813)(493.89533203,699.51299194)
\curveto(493.91532546,699.52298127)(493.93532544,699.52298127)(493.95533203,699.51299194)
\curveto(493.9753254,699.51298128)(493.99532538,699.51798127)(494.01533203,699.52799194)
\lineto(495.08033203,699.52799194)
\curveto(495.20032417,699.52798126)(495.31032406,699.52298127)(495.41033203,699.51299194)
\curveto(495.51032386,699.50298129)(495.58532379,699.46298133)(495.63533203,699.39299194)
\curveto(495.68532369,699.31298148)(495.71032366,699.20798158)(495.71033203,699.07799194)
\lineto(495.71033203,698.71799194)
\lineto(495.71033203,689.70299194)
\moveto(493.67033203,692.64299194)
\curveto(493.68032569,692.68298811)(493.68032569,692.72298807)(493.67033203,692.76299194)
\lineto(493.67033203,692.89799194)
\curveto(493.6703257,692.99798779)(493.66532571,693.09798769)(493.65533203,693.19799194)
\curveto(493.64532573,693.29798749)(493.63032574,693.3879874)(493.61033203,693.46799194)
\curveto(493.59032578,693.57798721)(493.5703258,693.67798711)(493.55033203,693.76799194)
\curveto(493.54032583,693.85798693)(493.51532586,693.94298685)(493.47533203,694.02299194)
\curveto(493.33532604,694.38298641)(493.13032624,694.66798612)(492.86033203,694.87799194)
\curveto(492.60032677,695.0879857)(492.22032715,695.1929856)(491.72033203,695.19299194)
\curveto(491.66032771,695.1929856)(491.58032779,695.18298561)(491.48033203,695.16299194)
\curveto(491.40032797,695.14298565)(491.32532805,695.12298567)(491.25533203,695.10299194)
\curveto(491.19532818,695.0929857)(491.13532824,695.07298572)(491.07533203,695.04299194)
\curveto(490.80532857,694.93298586)(490.59532878,694.76298603)(490.44533203,694.53299194)
\curveto(490.29532908,694.30298649)(490.1753292,694.04298675)(490.08533203,693.75299194)
\curveto(490.05532932,693.65298714)(490.03532934,693.55298724)(490.02533203,693.45299194)
\curveto(490.01532936,693.35298744)(489.99532938,693.24798754)(489.96533203,693.13799194)
\lineto(489.96533203,692.92799194)
\curveto(489.94532943,692.83798795)(489.94032943,692.71298808)(489.95033203,692.55299194)
\curveto(489.96032941,692.40298839)(489.9753294,692.2929885)(489.99533203,692.22299194)
\lineto(489.99533203,692.13299194)
\curveto(490.00532937,692.11298868)(490.01032936,692.0929887)(490.01033203,692.07299194)
\curveto(490.03032934,691.9929888)(490.04532933,691.91798887)(490.05533203,691.84799194)
\curveto(490.0753293,691.77798901)(490.09532928,691.70298909)(490.11533203,691.62299194)
\curveto(490.28532909,691.10298969)(490.5753288,690.71799007)(490.98533203,690.46799194)
\curveto(491.11532826,690.37799041)(491.29532808,690.30799048)(491.52533203,690.25799194)
\curveto(491.56532781,690.24799054)(491.62532775,690.24299055)(491.70533203,690.24299194)
\curveto(491.73532764,690.23299056)(491.78032759,690.22299057)(491.84033203,690.21299194)
\curveto(491.91032746,690.21299058)(491.96532741,690.21799057)(492.00533203,690.22799194)
\curveto(492.08532729,690.24799054)(492.16532721,690.26299053)(492.24533203,690.27299194)
\curveto(492.32532705,690.28299051)(492.40532697,690.30299049)(492.48533203,690.33299194)
\curveto(492.73532664,690.44299035)(492.93532644,690.58299021)(493.08533203,690.75299194)
\curveto(493.23532614,690.92298987)(493.36532601,691.13798965)(493.47533203,691.39799194)
\curveto(493.51532586,691.4879893)(493.54532583,691.57798921)(493.56533203,691.66799194)
\curveto(493.58532579,691.76798902)(493.60532577,691.87298892)(493.62533203,691.98299194)
\curveto(493.63532574,692.03298876)(493.63532574,692.07798871)(493.62533203,692.11799194)
\curveto(493.62532575,692.16798862)(493.63532574,692.21798857)(493.65533203,692.26799194)
\curveto(493.66532571,692.29798849)(493.6703257,692.33298846)(493.67033203,692.37299194)
\lineto(493.67033203,692.50799194)
\lineto(493.67033203,692.64299194)
}
}
{
\newrgbcolor{curcolor}{0 0 0}
\pscustom[linestyle=none,fillstyle=solid,fillcolor=curcolor]
{
\newpath
\moveto(504.65525391,692.79299194)
\curveto(504.67524574,692.71298808)(504.67524574,692.62298817)(504.65525391,692.52299194)
\curveto(504.63524578,692.42298837)(504.60024582,692.35798843)(504.55025391,692.32799194)
\curveto(504.50024592,692.2879885)(504.42524599,692.25798853)(504.32525391,692.23799194)
\curveto(504.23524618,692.22798856)(504.13024629,692.21798857)(504.01025391,692.20799194)
\lineto(503.66525391,692.20799194)
\curveto(503.55524686,692.21798857)(503.45524696,692.22298857)(503.36525391,692.22299194)
\lineto(499.70525391,692.22299194)
\lineto(499.49525391,692.22299194)
\curveto(499.43525098,692.22298857)(499.38025104,692.21298858)(499.33025391,692.19299194)
\curveto(499.25025117,692.15298864)(499.20025122,692.11298868)(499.18025391,692.07299194)
\curveto(499.16025126,692.05298874)(499.14025128,692.01298878)(499.12025391,691.95299194)
\curveto(499.10025132,691.90298889)(499.09525132,691.85298894)(499.10525391,691.80299194)
\curveto(499.12525129,691.74298905)(499.13525128,691.68298911)(499.13525391,691.62299194)
\curveto(499.14525127,691.57298922)(499.16025126,691.51798927)(499.18025391,691.45799194)
\curveto(499.26025116,691.21798957)(499.35525106,691.01798977)(499.46525391,690.85799194)
\curveto(499.58525083,690.70799008)(499.74525067,690.57299022)(499.94525391,690.45299194)
\curveto(500.02525039,690.40299039)(500.10525031,690.36799042)(500.18525391,690.34799194)
\curveto(500.27525014,690.33799045)(500.36525005,690.31799047)(500.45525391,690.28799194)
\curveto(500.53524988,690.26799052)(500.64524977,690.25299054)(500.78525391,690.24299194)
\curveto(500.92524949,690.23299056)(501.04524937,690.23799055)(501.14525391,690.25799194)
\lineto(501.28025391,690.25799194)
\curveto(501.38024904,690.27799051)(501.47024895,690.29799049)(501.55025391,690.31799194)
\curveto(501.64024878,690.34799044)(501.72524869,690.37799041)(501.80525391,690.40799194)
\curveto(501.90524851,690.45799033)(502.0152484,690.52299027)(502.13525391,690.60299194)
\curveto(502.26524815,690.68299011)(502.36024806,690.76299003)(502.42025391,690.84299194)
\curveto(502.47024795,690.91298988)(502.5202479,690.97798981)(502.57025391,691.03799194)
\curveto(502.63024779,691.10798968)(502.70024772,691.15798963)(502.78025391,691.18799194)
\curveto(502.88024754,691.23798955)(503.00524741,691.25798953)(503.15525391,691.24799194)
\lineto(503.59025391,691.24799194)
\lineto(503.77025391,691.24799194)
\curveto(503.84024658,691.25798953)(503.90024652,691.25298954)(503.95025391,691.23299194)
\lineto(504.10025391,691.23299194)
\curveto(504.20024622,691.21298958)(504.27024615,691.1879896)(504.31025391,691.15799194)
\curveto(504.35024607,691.13798965)(504.37024605,691.0929897)(504.37025391,691.02299194)
\curveto(504.38024604,690.95298984)(504.37524604,690.8929899)(504.35525391,690.84299194)
\curveto(504.30524611,690.70299009)(504.25024617,690.57799021)(504.19025391,690.46799194)
\curveto(504.13024629,690.35799043)(504.06024636,690.24799054)(503.98025391,690.13799194)
\curveto(503.76024666,689.80799098)(503.51024691,689.54299125)(503.23025391,689.34299194)
\curveto(502.95024747,689.14299165)(502.60024782,688.97299182)(502.18025391,688.83299194)
\curveto(502.07024835,688.792992)(501.96024846,688.76799202)(501.85025391,688.75799194)
\curveto(501.74024868,688.74799204)(501.62524879,688.72799206)(501.50525391,688.69799194)
\curveto(501.46524895,688.6879921)(501.420249,688.6879921)(501.37025391,688.69799194)
\curveto(501.33024909,688.69799209)(501.29024913,688.6929921)(501.25025391,688.68299194)
\lineto(501.08525391,688.68299194)
\curveto(501.03524938,688.66299213)(500.97524944,688.65799213)(500.90525391,688.66799194)
\curveto(500.84524957,688.66799212)(500.79024963,688.67299212)(500.74025391,688.68299194)
\curveto(500.66024976,688.6929921)(500.59024983,688.6929921)(500.53025391,688.68299194)
\curveto(500.47024995,688.67299212)(500.40525001,688.67799211)(500.33525391,688.69799194)
\curveto(500.28525013,688.71799207)(500.23025019,688.72799206)(500.17025391,688.72799194)
\curveto(500.11025031,688.72799206)(500.05525036,688.73799205)(500.00525391,688.75799194)
\curveto(499.89525052,688.77799201)(499.78525063,688.80299199)(499.67525391,688.83299194)
\curveto(499.56525085,688.85299194)(499.46525095,688.8879919)(499.37525391,688.93799194)
\curveto(499.26525115,688.97799181)(499.16025126,689.01299178)(499.06025391,689.04299194)
\curveto(498.97025145,689.08299171)(498.88525153,689.12799166)(498.80525391,689.17799194)
\curveto(498.48525193,689.37799141)(498.20025222,689.60799118)(497.95025391,689.86799194)
\curveto(497.70025272,690.13799065)(497.49525292,690.44799034)(497.33525391,690.79799194)
\curveto(497.28525313,690.90798988)(497.24525317,691.01798977)(497.21525391,691.12799194)
\curveto(497.18525323,691.24798954)(497.14525327,691.36798942)(497.09525391,691.48799194)
\curveto(497.08525333,691.52798926)(497.08025334,691.56298923)(497.08025391,691.59299194)
\curveto(497.08025334,691.63298916)(497.07525334,691.67298912)(497.06525391,691.71299194)
\curveto(497.02525339,691.83298896)(497.00025342,691.96298883)(496.99025391,692.10299194)
\lineto(496.96025391,692.52299194)
\curveto(496.96025346,692.57298822)(496.95525346,692.62798816)(496.94525391,692.68799194)
\curveto(496.94525347,692.74798804)(496.95025347,692.80298799)(496.96025391,692.85299194)
\lineto(496.96025391,693.03299194)
\lineto(497.00525391,693.39299194)
\curveto(497.04525337,693.56298723)(497.08025334,693.72798706)(497.11025391,693.88799194)
\curveto(497.14025328,694.04798674)(497.18525323,694.19798659)(497.24525391,694.33799194)
\curveto(497.67525274,695.37798541)(498.40525201,696.11298468)(499.43525391,696.54299194)
\curveto(499.57525084,696.60298419)(499.7152507,696.64298415)(499.85525391,696.66299194)
\curveto(500.00525041,696.6929841)(500.16025026,696.72798406)(500.32025391,696.76799194)
\curveto(500.40025002,696.77798401)(500.47524994,696.78298401)(500.54525391,696.78299194)
\curveto(500.6152498,696.78298401)(500.69024973,696.787984)(500.77025391,696.79799194)
\curveto(501.28024914,696.80798398)(501.7152487,696.74798404)(502.07525391,696.61799194)
\curveto(502.44524797,696.49798429)(502.77524764,696.33798445)(503.06525391,696.13799194)
\curveto(503.15524726,696.07798471)(503.24524717,696.00798478)(503.33525391,695.92799194)
\curveto(503.42524699,695.85798493)(503.50524691,695.78298501)(503.57525391,695.70299194)
\curveto(503.60524681,695.65298514)(503.64524677,695.61298518)(503.69525391,695.58299194)
\curveto(503.77524664,695.47298532)(503.85024657,695.35798543)(503.92025391,695.23799194)
\curveto(503.99024643,695.12798566)(504.06524635,695.01298578)(504.14525391,694.89299194)
\curveto(504.19524622,694.80298599)(504.23524618,694.70798608)(504.26525391,694.60799194)
\curveto(504.30524611,694.51798627)(504.34524607,694.41798637)(504.38525391,694.30799194)
\curveto(504.43524598,694.17798661)(504.47524594,694.04298675)(504.50525391,693.90299194)
\curveto(504.53524588,693.76298703)(504.57024585,693.62298717)(504.61025391,693.48299194)
\curveto(504.63024579,693.40298739)(504.63524578,693.31298748)(504.62525391,693.21299194)
\curveto(504.62524579,693.12298767)(504.63524578,693.03798775)(504.65525391,692.95799194)
\lineto(504.65525391,692.79299194)
\moveto(502.40525391,693.67799194)
\curveto(502.47524794,693.77798701)(502.48024794,693.89798689)(502.42025391,694.03799194)
\curveto(502.37024805,694.1879866)(502.33024809,694.29798649)(502.30025391,694.36799194)
\curveto(502.16024826,694.63798615)(501.97524844,694.84298595)(501.74525391,694.98299194)
\curveto(501.5152489,695.13298566)(501.19524922,695.21298558)(500.78525391,695.22299194)
\curveto(500.75524966,695.20298559)(500.7202497,695.19798559)(500.68025391,695.20799194)
\curveto(500.64024978,695.21798557)(500.60524981,695.21798557)(500.57525391,695.20799194)
\curveto(500.52524989,695.1879856)(500.47024995,695.17298562)(500.41025391,695.16299194)
\curveto(500.35025007,695.16298563)(500.29525012,695.15298564)(500.24525391,695.13299194)
\curveto(499.80525061,694.9929858)(499.48025094,694.71798607)(499.27025391,694.30799194)
\curveto(499.25025117,694.26798652)(499.22525119,694.21298658)(499.19525391,694.14299194)
\curveto(499.17525124,694.08298671)(499.16025126,694.01798677)(499.15025391,693.94799194)
\curveto(499.14025128,693.8879869)(499.14025128,693.82798696)(499.15025391,693.76799194)
\curveto(499.17025125,693.70798708)(499.20525121,693.65798713)(499.25525391,693.61799194)
\curveto(499.33525108,693.56798722)(499.44525097,693.54298725)(499.58525391,693.54299194)
\lineto(499.99025391,693.54299194)
\lineto(501.65525391,693.54299194)
\lineto(502.09025391,693.54299194)
\curveto(502.25024817,693.55298724)(502.35524806,693.59798719)(502.40525391,693.67799194)
}
}
{
\newrgbcolor{curcolor}{0 0 0}
\pscustom[linestyle=none,fillstyle=solid,fillcolor=curcolor]
{
}
}
{
\newrgbcolor{curcolor}{0 0 0}
\pscustom[linestyle=none,fillstyle=solid,fillcolor=curcolor]
{
\newpath
\moveto(517.95369141,692.80799194)
\curveto(517.96368273,692.74798804)(517.96868272,692.65798813)(517.96869141,692.53799194)
\curveto(517.96868272,692.41798837)(517.95868273,692.33298846)(517.93869141,692.28299194)
\lineto(517.93869141,692.08799194)
\curveto(517.90868278,691.97798881)(517.8886828,691.87298892)(517.87869141,691.77299194)
\curveto(517.87868281,691.67298912)(517.86368283,691.57298922)(517.83369141,691.47299194)
\curveto(517.81368288,691.38298941)(517.7936829,691.2879895)(517.77369141,691.18799194)
\curveto(517.75368294,691.09798969)(517.72368297,691.00798978)(517.68369141,690.91799194)
\curveto(517.61368308,690.74799004)(517.54368315,690.5879902)(517.47369141,690.43799194)
\curveto(517.40368329,690.29799049)(517.32368337,690.15799063)(517.23369141,690.01799194)
\curveto(517.17368352,689.92799086)(517.10868358,689.84299095)(517.03869141,689.76299194)
\curveto(516.97868371,689.6929911)(516.90868378,689.61799117)(516.82869141,689.53799194)
\lineto(516.72369141,689.43299194)
\curveto(516.67368402,689.38299141)(516.61868407,689.33799145)(516.55869141,689.29799194)
\lineto(516.40869141,689.17799194)
\curveto(516.32868436,689.11799167)(516.23868445,689.06299173)(516.13869141,689.01299194)
\curveto(516.04868464,688.97299182)(515.95368474,688.92799186)(515.85369141,688.87799194)
\curveto(515.75368494,688.82799196)(515.64868504,688.792992)(515.53869141,688.77299194)
\curveto(515.43868525,688.75299204)(515.33368536,688.73299206)(515.22369141,688.71299194)
\curveto(515.16368553,688.6929921)(515.09868559,688.68299211)(515.02869141,688.68299194)
\curveto(514.96868572,688.68299211)(514.90368579,688.67299212)(514.83369141,688.65299194)
\lineto(514.69869141,688.65299194)
\curveto(514.61868607,688.63299216)(514.54368615,688.63299216)(514.47369141,688.65299194)
\lineto(514.32369141,688.65299194)
\curveto(514.26368643,688.67299212)(514.19868649,688.68299211)(514.12869141,688.68299194)
\curveto(514.06868662,688.67299212)(514.00868668,688.67799211)(513.94869141,688.69799194)
\curveto(513.7886869,688.74799204)(513.63368706,688.792992)(513.48369141,688.83299194)
\curveto(513.34368735,688.87299192)(513.21368748,688.93299186)(513.09369141,689.01299194)
\curveto(513.02368767,689.05299174)(512.95868773,689.0929917)(512.89869141,689.13299194)
\curveto(512.83868785,689.18299161)(512.77368792,689.23299156)(512.70369141,689.28299194)
\lineto(512.52369141,689.41799194)
\curveto(512.44368825,689.47799131)(512.37368832,689.48299131)(512.31369141,689.43299194)
\curveto(512.26368843,689.40299139)(512.23868845,689.36299143)(512.23869141,689.31299194)
\curveto(512.23868845,689.27299152)(512.22868846,689.22299157)(512.20869141,689.16299194)
\curveto(512.1886885,689.06299173)(512.17868851,688.94799184)(512.17869141,688.81799194)
\curveto(512.1886885,688.6879921)(512.1936885,688.56799222)(512.19369141,688.45799194)
\lineto(512.19369141,686.92799194)
\curveto(512.1936885,686.79799399)(512.1886885,686.67299412)(512.17869141,686.55299194)
\curveto(512.17868851,686.42299437)(512.15368854,686.31799447)(512.10369141,686.23799194)
\curveto(512.07368862,686.19799459)(512.01868867,686.16799462)(511.93869141,686.14799194)
\curveto(511.85868883,686.12799466)(511.76868892,686.11799467)(511.66869141,686.11799194)
\curveto(511.56868912,686.10799468)(511.46868922,686.10799468)(511.36869141,686.11799194)
\lineto(511.11369141,686.11799194)
\lineto(510.70869141,686.11799194)
\lineto(510.60369141,686.11799194)
\curveto(510.56369013,686.11799467)(510.52869016,686.12299467)(510.49869141,686.13299194)
\lineto(510.37869141,686.13299194)
\curveto(510.20869048,686.18299461)(510.11869057,686.28299451)(510.10869141,686.43299194)
\curveto(510.09869059,686.57299422)(510.0936906,686.74299405)(510.09369141,686.94299194)
\lineto(510.09369141,695.74799194)
\curveto(510.0936906,695.85798493)(510.0886906,695.97298482)(510.07869141,696.09299194)
\curveto(510.07869061,696.22298457)(510.10369059,696.32298447)(510.15369141,696.39299194)
\curveto(510.1936905,696.46298433)(510.24869044,696.50798428)(510.31869141,696.52799194)
\curveto(510.36869032,696.54798424)(510.42869026,696.55798423)(510.49869141,696.55799194)
\lineto(510.72369141,696.55799194)
\lineto(511.44369141,696.55799194)
\lineto(511.72869141,696.55799194)
\curveto(511.81868887,696.55798423)(511.8936888,696.53298426)(511.95369141,696.48299194)
\curveto(512.02368867,696.43298436)(512.05868863,696.36798442)(512.05869141,696.28799194)
\curveto(512.06868862,696.21798457)(512.0936886,696.14298465)(512.13369141,696.06299194)
\curveto(512.14368855,696.03298476)(512.15368854,696.00798478)(512.16369141,695.98799194)
\curveto(512.18368851,695.97798481)(512.20368849,695.96298483)(512.22369141,695.94299194)
\curveto(512.33368836,695.93298486)(512.42368827,695.96298483)(512.49369141,696.03299194)
\curveto(512.56368813,696.10298469)(512.63368806,696.16298463)(512.70369141,696.21299194)
\curveto(512.83368786,696.30298449)(512.96868772,696.38298441)(513.10869141,696.45299194)
\curveto(513.24868744,696.53298426)(513.40368729,696.59798419)(513.57369141,696.64799194)
\curveto(513.65368704,696.67798411)(513.73868695,696.69798409)(513.82869141,696.70799194)
\curveto(513.92868676,696.71798407)(514.02368667,696.73298406)(514.11369141,696.75299194)
\curveto(514.15368654,696.76298403)(514.1936865,696.76298403)(514.23369141,696.75299194)
\curveto(514.28368641,696.74298405)(514.32368637,696.74798404)(514.35369141,696.76799194)
\curveto(514.92368577,696.787984)(515.40368529,696.70798408)(515.79369141,696.52799194)
\curveto(516.1936845,696.35798443)(516.53368416,696.13298466)(516.81369141,695.85299194)
\curveto(516.86368383,695.80298499)(516.90868378,695.75298504)(516.94869141,695.70299194)
\curveto(516.9886837,695.66298513)(517.02868366,695.61798517)(517.06869141,695.56799194)
\curveto(517.13868355,695.47798531)(517.19868349,695.3879854)(517.24869141,695.29799194)
\curveto(517.30868338,695.20798558)(517.36368333,695.11798567)(517.41369141,695.02799194)
\curveto(517.43368326,695.00798578)(517.44368325,694.98298581)(517.44369141,694.95299194)
\curveto(517.45368324,694.92298587)(517.46868322,694.8879859)(517.48869141,694.84799194)
\curveto(517.54868314,694.74798604)(517.60368309,694.62798616)(517.65369141,694.48799194)
\curveto(517.67368302,694.42798636)(517.693683,694.36298643)(517.71369141,694.29299194)
\curveto(517.73368296,694.23298656)(517.75368294,694.16798662)(517.77369141,694.09799194)
\curveto(517.81368288,693.97798681)(517.83868285,693.85298694)(517.84869141,693.72299194)
\curveto(517.86868282,693.5929872)(517.8936828,693.45798733)(517.92369141,693.31799194)
\lineto(517.92369141,693.15299194)
\lineto(517.95369141,692.97299194)
\lineto(517.95369141,692.80799194)
\moveto(515.83869141,692.46299194)
\curveto(515.84868484,692.51298828)(515.85368484,692.57798821)(515.85369141,692.65799194)
\curveto(515.85368484,692.74798804)(515.84868484,692.81798797)(515.83869141,692.86799194)
\lineto(515.83869141,693.00299194)
\curveto(515.81868487,693.06298773)(515.80868488,693.12798766)(515.80869141,693.19799194)
\curveto(515.80868488,693.26798752)(515.79868489,693.33798745)(515.77869141,693.40799194)
\curveto(515.75868493,693.50798728)(515.73868495,693.60298719)(515.71869141,693.69299194)
\curveto(515.69868499,693.792987)(515.66868502,693.88298691)(515.62869141,693.96299194)
\curveto(515.50868518,694.28298651)(515.35368534,694.53798625)(515.16369141,694.72799194)
\curveto(514.97368572,694.91798587)(514.70368599,695.05798573)(514.35369141,695.14799194)
\curveto(514.27368642,695.16798562)(514.18368651,695.17798561)(514.08369141,695.17799194)
\lineto(513.81369141,695.17799194)
\curveto(513.77368692,695.16798562)(513.73868695,695.16298563)(513.70869141,695.16299194)
\curveto(513.67868701,695.16298563)(513.64368705,695.15798563)(513.60369141,695.14799194)
\lineto(513.39369141,695.08799194)
\curveto(513.33368736,695.07798571)(513.27368742,695.05798573)(513.21369141,695.02799194)
\curveto(512.95368774,694.91798587)(512.74868794,694.74798604)(512.59869141,694.51799194)
\curveto(512.45868823,694.2879865)(512.34368835,694.03298676)(512.25369141,693.75299194)
\curveto(512.23368846,693.67298712)(512.21868847,693.5879872)(512.20869141,693.49799194)
\curveto(512.19868849,693.41798737)(512.18368851,693.33798745)(512.16369141,693.25799194)
\curveto(512.15368854,693.21798757)(512.14868854,693.15298764)(512.14869141,693.06299194)
\curveto(512.12868856,693.02298777)(512.12368857,692.97298782)(512.13369141,692.91299194)
\curveto(512.14368855,692.86298793)(512.14368855,692.81298798)(512.13369141,692.76299194)
\curveto(512.11368858,692.70298809)(512.11368858,692.64798814)(512.13369141,692.59799194)
\lineto(512.13369141,692.41799194)
\lineto(512.13369141,692.28299194)
\curveto(512.13368856,692.24298855)(512.14368855,692.20298859)(512.16369141,692.16299194)
\curveto(512.16368853,692.0929887)(512.16868852,692.03798875)(512.17869141,691.99799194)
\lineto(512.20869141,691.81799194)
\curveto(512.21868847,691.75798903)(512.23368846,691.69798909)(512.25369141,691.63799194)
\curveto(512.34368835,691.34798944)(512.44868824,691.10798968)(512.56869141,690.91799194)
\curveto(512.69868799,690.73799005)(512.87868781,690.57799021)(513.10869141,690.43799194)
\curveto(513.24868744,690.35799043)(513.41368728,690.2929905)(513.60369141,690.24299194)
\curveto(513.64368705,690.23299056)(513.67868701,690.22799056)(513.70869141,690.22799194)
\curveto(513.73868695,690.23799055)(513.77368692,690.23799055)(513.81369141,690.22799194)
\curveto(513.85368684,690.21799057)(513.91368678,690.20799058)(513.99369141,690.19799194)
\curveto(514.07368662,690.19799059)(514.13868655,690.20299059)(514.18869141,690.21299194)
\curveto(514.26868642,690.23299056)(514.34868634,690.24799054)(514.42869141,690.25799194)
\curveto(514.51868617,690.27799051)(514.60368609,690.30299049)(514.68369141,690.33299194)
\curveto(514.92368577,690.43299036)(515.11868557,690.57299022)(515.26869141,690.75299194)
\curveto(515.41868527,690.93298986)(515.54368515,691.14298965)(515.64369141,691.38299194)
\curveto(515.693685,691.50298929)(515.72868496,691.62798916)(515.74869141,691.75799194)
\curveto(515.76868492,691.8879889)(515.7936849,692.02298877)(515.82369141,692.16299194)
\lineto(515.82369141,692.31299194)
\curveto(515.83368486,692.36298843)(515.83868485,692.41298838)(515.83869141,692.46299194)
}
}
{
\newrgbcolor{curcolor}{0 0 0}
\pscustom[linestyle=none,fillstyle=solid,fillcolor=curcolor]
{
\newpath
\moveto(526.28361328,689.44799194)
\curveto(526.30360543,689.33799145)(526.31360542,689.22799156)(526.31361328,689.11799194)
\curveto(526.32360541,689.00799178)(526.27360546,688.93299186)(526.16361328,688.89299194)
\curveto(526.10360563,688.86299193)(526.0336057,688.84799194)(525.95361328,688.84799194)
\lineto(525.71361328,688.84799194)
\lineto(524.90361328,688.84799194)
\lineto(524.63361328,688.84799194)
\curveto(524.55360718,688.85799193)(524.48860725,688.88299191)(524.43861328,688.92299194)
\curveto(524.36860737,688.96299183)(524.31360742,689.01799177)(524.27361328,689.08799194)
\curveto(524.24360749,689.16799162)(524.19860754,689.23299156)(524.13861328,689.28299194)
\curveto(524.11860762,689.30299149)(524.09360764,689.31799147)(524.06361328,689.32799194)
\curveto(524.0336077,689.34799144)(523.99360774,689.35299144)(523.94361328,689.34299194)
\curveto(523.89360784,689.32299147)(523.84360789,689.29799149)(523.79361328,689.26799194)
\curveto(523.75360798,689.23799155)(523.70860803,689.21299158)(523.65861328,689.19299194)
\curveto(523.60860813,689.15299164)(523.55360818,689.11799167)(523.49361328,689.08799194)
\lineto(523.31361328,688.99799194)
\curveto(523.18360855,688.93799185)(523.04860869,688.8879919)(522.90861328,688.84799194)
\curveto(522.76860897,688.81799197)(522.62360911,688.78299201)(522.47361328,688.74299194)
\curveto(522.40360933,688.72299207)(522.3336094,688.71299208)(522.26361328,688.71299194)
\curveto(522.20360953,688.70299209)(522.1386096,688.6929921)(522.06861328,688.68299194)
\lineto(521.97861328,688.68299194)
\curveto(521.94860979,688.67299212)(521.91860982,688.66799212)(521.88861328,688.66799194)
\lineto(521.72361328,688.66799194)
\curveto(521.62361011,688.64799214)(521.52361021,688.64799214)(521.42361328,688.66799194)
\lineto(521.28861328,688.66799194)
\curveto(521.21861052,688.6879921)(521.14861059,688.69799209)(521.07861328,688.69799194)
\curveto(521.01861072,688.6879921)(520.95861078,688.6929921)(520.89861328,688.71299194)
\curveto(520.79861094,688.73299206)(520.70361103,688.75299204)(520.61361328,688.77299194)
\curveto(520.52361121,688.78299201)(520.4386113,688.80799198)(520.35861328,688.84799194)
\curveto(520.06861167,688.95799183)(519.81861192,689.09799169)(519.60861328,689.26799194)
\curveto(519.40861233,689.44799134)(519.24861249,689.68299111)(519.12861328,689.97299194)
\curveto(519.09861264,690.04299075)(519.06861267,690.11799067)(519.03861328,690.19799194)
\curveto(519.01861272,690.27799051)(518.99861274,690.36299043)(518.97861328,690.45299194)
\curveto(518.95861278,690.50299029)(518.94861279,690.55299024)(518.94861328,690.60299194)
\curveto(518.95861278,690.65299014)(518.95861278,690.70299009)(518.94861328,690.75299194)
\curveto(518.9386128,690.78299001)(518.92861281,690.84298995)(518.91861328,690.93299194)
\curveto(518.91861282,691.03298976)(518.92361281,691.10298969)(518.93361328,691.14299194)
\curveto(518.95361278,691.24298955)(518.96361277,691.32798946)(518.96361328,691.39799194)
\lineto(519.05361328,691.72799194)
\curveto(519.08361265,691.84798894)(519.12361261,691.95298884)(519.17361328,692.04299194)
\curveto(519.34361239,692.33298846)(519.5386122,692.55298824)(519.75861328,692.70299194)
\curveto(519.97861176,692.85298794)(520.25861148,692.98298781)(520.59861328,693.09299194)
\curveto(520.72861101,693.14298765)(520.86361087,693.17798761)(521.00361328,693.19799194)
\curveto(521.14361059,693.21798757)(521.28361045,693.24298755)(521.42361328,693.27299194)
\curveto(521.50361023,693.2929875)(521.58861015,693.30298749)(521.67861328,693.30299194)
\curveto(521.76860997,693.31298748)(521.85860988,693.32798746)(521.94861328,693.34799194)
\curveto(522.01860972,693.36798742)(522.08860965,693.37298742)(522.15861328,693.36299194)
\curveto(522.22860951,693.36298743)(522.30360943,693.37298742)(522.38361328,693.39299194)
\curveto(522.45360928,693.41298738)(522.52360921,693.42298737)(522.59361328,693.42299194)
\curveto(522.66360907,693.42298737)(522.738609,693.43298736)(522.81861328,693.45299194)
\curveto(523.02860871,693.50298729)(523.21860852,693.54298725)(523.38861328,693.57299194)
\curveto(523.56860817,693.61298718)(523.72860801,693.70298709)(523.86861328,693.84299194)
\curveto(523.95860778,693.93298686)(524.01860772,694.03298676)(524.04861328,694.14299194)
\curveto(524.05860768,694.17298662)(524.05860768,694.19798659)(524.04861328,694.21799194)
\curveto(524.04860769,694.23798655)(524.05360768,694.25798653)(524.06361328,694.27799194)
\curveto(524.07360766,694.29798649)(524.07860766,694.32798646)(524.07861328,694.36799194)
\lineto(524.07861328,694.45799194)
\lineto(524.04861328,694.57799194)
\curveto(524.04860769,694.61798617)(524.04360769,694.65298614)(524.03361328,694.68299194)
\curveto(523.9336078,694.98298581)(523.72360801,695.1879856)(523.40361328,695.29799194)
\curveto(523.31360842,695.32798546)(523.20360853,695.34798544)(523.07361328,695.35799194)
\curveto(522.95360878,695.37798541)(522.82860891,695.38298541)(522.69861328,695.37299194)
\curveto(522.56860917,695.37298542)(522.44360929,695.36298543)(522.32361328,695.34299194)
\curveto(522.20360953,695.32298547)(522.09860964,695.29798549)(522.00861328,695.26799194)
\curveto(521.94860979,695.24798554)(521.88860985,695.21798557)(521.82861328,695.17799194)
\curveto(521.77860996,695.14798564)(521.72861001,695.11298568)(521.67861328,695.07299194)
\curveto(521.62861011,695.03298576)(521.57361016,694.97798581)(521.51361328,694.90799194)
\curveto(521.46361027,694.83798595)(521.42861031,694.77298602)(521.40861328,694.71299194)
\curveto(521.35861038,694.61298618)(521.31361042,694.51798627)(521.27361328,694.42799194)
\curveto(521.24361049,694.33798645)(521.17361056,694.27798651)(521.06361328,694.24799194)
\curveto(520.98361075,694.22798656)(520.89861084,694.21798657)(520.80861328,694.21799194)
\lineto(520.53861328,694.21799194)
\lineto(519.96861328,694.21799194)
\curveto(519.91861182,694.21798657)(519.86861187,694.21298658)(519.81861328,694.20299194)
\curveto(519.76861197,694.20298659)(519.72361201,694.20798658)(519.68361328,694.21799194)
\lineto(519.54861328,694.21799194)
\curveto(519.52861221,694.22798656)(519.50361223,694.23298656)(519.47361328,694.23299194)
\curveto(519.44361229,694.23298656)(519.41861232,694.24298655)(519.39861328,694.26299194)
\curveto(519.31861242,694.28298651)(519.26361247,694.34798644)(519.23361328,694.45799194)
\curveto(519.22361251,694.50798628)(519.22361251,694.55798623)(519.23361328,694.60799194)
\curveto(519.24361249,694.65798613)(519.25361248,694.70298609)(519.26361328,694.74299194)
\curveto(519.29361244,694.85298594)(519.32361241,694.95298584)(519.35361328,695.04299194)
\curveto(519.39361234,695.14298565)(519.4386123,695.23298556)(519.48861328,695.31299194)
\lineto(519.57861328,695.46299194)
\lineto(519.66861328,695.61299194)
\curveto(519.74861199,695.72298507)(519.84861189,695.82798496)(519.96861328,695.92799194)
\curveto(519.98861175,695.93798485)(520.01861172,695.96298483)(520.05861328,696.00299194)
\curveto(520.10861163,696.04298475)(520.15361158,696.07798471)(520.19361328,696.10799194)
\curveto(520.2336115,696.13798465)(520.27861146,696.16798462)(520.32861328,696.19799194)
\curveto(520.49861124,696.30798448)(520.67861106,696.3929844)(520.86861328,696.45299194)
\curveto(521.05861068,696.52298427)(521.25361048,696.5879842)(521.45361328,696.64799194)
\curveto(521.57361016,696.67798411)(521.69861004,696.69798409)(521.82861328,696.70799194)
\curveto(521.95860978,696.71798407)(522.08860965,696.73798405)(522.21861328,696.76799194)
\curveto(522.25860948,696.77798401)(522.31860942,696.77798401)(522.39861328,696.76799194)
\curveto(522.48860925,696.75798403)(522.54360919,696.76298403)(522.56361328,696.78299194)
\curveto(522.97360876,696.792984)(523.36360837,696.77798401)(523.73361328,696.73799194)
\curveto(524.11360762,696.69798409)(524.45360728,696.62298417)(524.75361328,696.51299194)
\curveto(525.06360667,696.40298439)(525.32860641,696.25298454)(525.54861328,696.06299194)
\curveto(525.76860597,695.88298491)(525.9386058,695.64798514)(526.05861328,695.35799194)
\curveto(526.12860561,695.1879856)(526.16860557,694.9929858)(526.17861328,694.77299194)
\curveto(526.18860555,694.55298624)(526.19360554,694.32798646)(526.19361328,694.09799194)
\lineto(526.19361328,690.75299194)
\lineto(526.19361328,690.16799194)
\curveto(526.19360554,689.97799081)(526.21360552,689.80299099)(526.25361328,689.64299194)
\curveto(526.26360547,689.61299118)(526.26860547,689.57799121)(526.26861328,689.53799194)
\curveto(526.26860547,689.50799128)(526.27360546,689.47799131)(526.28361328,689.44799194)
\moveto(524.07861328,691.75799194)
\curveto(524.08860765,691.80798898)(524.09360764,691.86298893)(524.09361328,691.92299194)
\curveto(524.09360764,691.9929888)(524.08860765,692.05298874)(524.07861328,692.10299194)
\curveto(524.05860768,692.16298863)(524.04860769,692.21798857)(524.04861328,692.26799194)
\curveto(524.04860769,692.31798847)(524.02860771,692.35798843)(523.98861328,692.38799194)
\curveto(523.9386078,692.42798836)(523.86360787,692.44798834)(523.76361328,692.44799194)
\curveto(523.72360801,692.43798835)(523.68860805,692.42798836)(523.65861328,692.41799194)
\curveto(523.62860811,692.41798837)(523.59360814,692.41298838)(523.55361328,692.40299194)
\curveto(523.48360825,692.38298841)(523.40860833,692.36798842)(523.32861328,692.35799194)
\curveto(523.24860849,692.34798844)(523.16860857,692.33298846)(523.08861328,692.31299194)
\curveto(523.05860868,692.30298849)(523.01360872,692.29798849)(522.95361328,692.29799194)
\curveto(522.82360891,692.26798852)(522.69360904,692.24798854)(522.56361328,692.23799194)
\curveto(522.4336093,692.22798856)(522.30860943,692.20298859)(522.18861328,692.16299194)
\curveto(522.10860963,692.14298865)(522.0336097,692.12298867)(521.96361328,692.10299194)
\curveto(521.89360984,692.0929887)(521.82360991,692.07298872)(521.75361328,692.04299194)
\curveto(521.54361019,691.95298884)(521.36361037,691.81798897)(521.21361328,691.63799194)
\curveto(521.07361066,691.45798933)(521.02361071,691.20798958)(521.06361328,690.88799194)
\curveto(521.08361065,690.71799007)(521.1386106,690.57799021)(521.22861328,690.46799194)
\curveto(521.29861044,690.35799043)(521.40361033,690.26799052)(521.54361328,690.19799194)
\curveto(521.68361005,690.13799065)(521.8336099,690.0929907)(521.99361328,690.06299194)
\curveto(522.16360957,690.03299076)(522.3386094,690.02299077)(522.51861328,690.03299194)
\curveto(522.70860903,690.05299074)(522.88360885,690.0879907)(523.04361328,690.13799194)
\curveto(523.30360843,690.21799057)(523.50860823,690.34299045)(523.65861328,690.51299194)
\curveto(523.80860793,690.6929901)(523.92360781,690.91298988)(524.00361328,691.17299194)
\curveto(524.02360771,691.24298955)(524.0336077,691.31298948)(524.03361328,691.38299194)
\curveto(524.04360769,691.46298933)(524.05860768,691.54298925)(524.07861328,691.62299194)
\lineto(524.07861328,691.75799194)
}
}
{
\newrgbcolor{curcolor}{0 0 0}
\pscustom[linestyle=none,fillstyle=solid,fillcolor=curcolor]
{
\newpath
\moveto(532.27189453,696.78299194)
\curveto(532.38188922,696.78298401)(532.47688912,696.77298402)(532.55689453,696.75299194)
\curveto(532.64688895,696.73298406)(532.71688888,696.6879841)(532.76689453,696.61799194)
\curveto(532.82688877,696.53798425)(532.85688874,696.39798439)(532.85689453,696.19799194)
\lineto(532.85689453,695.68799194)
\lineto(532.85689453,695.31299194)
\curveto(532.86688873,695.17298562)(532.85188875,695.06298573)(532.81189453,694.98299194)
\curveto(532.77188883,694.91298588)(532.71188889,694.86798592)(532.63189453,694.84799194)
\curveto(532.56188904,694.82798596)(532.47688912,694.81798597)(532.37689453,694.81799194)
\curveto(532.28688931,694.81798597)(532.18688941,694.82298597)(532.07689453,694.83299194)
\curveto(531.97688962,694.84298595)(531.88188972,694.83798595)(531.79189453,694.81799194)
\curveto(531.72188988,694.79798599)(531.65188995,694.78298601)(531.58189453,694.77299194)
\curveto(531.51189009,694.77298602)(531.44689015,694.76298603)(531.38689453,694.74299194)
\curveto(531.22689037,694.6929861)(531.06689053,694.61798617)(530.90689453,694.51799194)
\curveto(530.74689085,694.42798636)(530.62189098,694.32298647)(530.53189453,694.20299194)
\curveto(530.48189112,694.12298667)(530.42689117,694.03798675)(530.36689453,693.94799194)
\curveto(530.31689128,693.86798692)(530.26689133,693.78298701)(530.21689453,693.69299194)
\curveto(530.18689141,693.61298718)(530.15689144,693.52798726)(530.12689453,693.43799194)
\lineto(530.06689453,693.19799194)
\curveto(530.04689155,693.12798766)(530.03689156,693.05298774)(530.03689453,692.97299194)
\curveto(530.03689156,692.90298789)(530.02689157,692.83298796)(530.00689453,692.76299194)
\curveto(529.9968916,692.72298807)(529.99189161,692.68298811)(529.99189453,692.64299194)
\curveto(530.0018916,692.61298818)(530.0018916,692.58298821)(529.99189453,692.55299194)
\lineto(529.99189453,692.31299194)
\curveto(529.97189163,692.24298855)(529.96689163,692.16298863)(529.97689453,692.07299194)
\curveto(529.98689161,691.9929888)(529.99189161,691.91298888)(529.99189453,691.83299194)
\lineto(529.99189453,690.87299194)
\lineto(529.99189453,689.59799194)
\curveto(529.99189161,689.46799132)(529.98689161,689.34799144)(529.97689453,689.23799194)
\curveto(529.96689163,689.12799166)(529.93689166,689.03799175)(529.88689453,688.96799194)
\curveto(529.86689173,688.93799185)(529.83189177,688.91299188)(529.78189453,688.89299194)
\curveto(529.74189186,688.88299191)(529.6968919,688.87299192)(529.64689453,688.86299194)
\lineto(529.57189453,688.86299194)
\curveto(529.52189208,688.85299194)(529.46689213,688.84799194)(529.40689453,688.84799194)
\lineto(529.24189453,688.84799194)
\lineto(528.59689453,688.84799194)
\curveto(528.53689306,688.85799193)(528.47189313,688.86299193)(528.40189453,688.86299194)
\lineto(528.20689453,688.86299194)
\curveto(528.15689344,688.88299191)(528.10689349,688.89799189)(528.05689453,688.90799194)
\curveto(528.00689359,688.92799186)(527.97189363,688.96299183)(527.95189453,689.01299194)
\curveto(527.91189369,689.06299173)(527.88689371,689.13299166)(527.87689453,689.22299194)
\lineto(527.87689453,689.52299194)
\lineto(527.87689453,690.54299194)
\lineto(527.87689453,694.77299194)
\lineto(527.87689453,695.88299194)
\lineto(527.87689453,696.16799194)
\curveto(527.87689372,696.26798452)(527.8968937,696.34798444)(527.93689453,696.40799194)
\curveto(527.98689361,696.4879843)(528.06189354,696.53798425)(528.16189453,696.55799194)
\curveto(528.26189334,696.57798421)(528.38189322,696.5879842)(528.52189453,696.58799194)
\lineto(529.28689453,696.58799194)
\curveto(529.40689219,696.5879842)(529.51189209,696.57798421)(529.60189453,696.55799194)
\curveto(529.69189191,696.54798424)(529.76189184,696.50298429)(529.81189453,696.42299194)
\curveto(529.84189176,696.37298442)(529.85689174,696.30298449)(529.85689453,696.21299194)
\lineto(529.88689453,695.94299194)
\curveto(529.8968917,695.86298493)(529.91189169,695.787985)(529.93189453,695.71799194)
\curveto(529.96189164,695.64798514)(530.01189159,695.61298518)(530.08189453,695.61299194)
\curveto(530.1018915,695.63298516)(530.12189148,695.64298515)(530.14189453,695.64299194)
\curveto(530.16189144,695.64298515)(530.18189142,695.65298514)(530.20189453,695.67299194)
\curveto(530.26189134,695.72298507)(530.31189129,695.77798501)(530.35189453,695.83799194)
\curveto(530.4018912,695.90798488)(530.46189114,695.96798482)(530.53189453,696.01799194)
\curveto(530.57189103,696.04798474)(530.60689099,696.07798471)(530.63689453,696.10799194)
\curveto(530.66689093,696.14798464)(530.7018909,696.18298461)(530.74189453,696.21299194)
\lineto(531.01189453,696.39299194)
\curveto(531.11189049,696.45298434)(531.21189039,696.50798428)(531.31189453,696.55799194)
\curveto(531.41189019,696.59798419)(531.51189009,696.63298416)(531.61189453,696.66299194)
\lineto(531.94189453,696.75299194)
\curveto(531.97188963,696.76298403)(532.02688957,696.76298403)(532.10689453,696.75299194)
\curveto(532.1968894,696.75298404)(532.25188935,696.76298403)(532.27189453,696.78299194)
}
}
{
\newrgbcolor{curcolor}{0 0 0}
\pscustom[linestyle=none,fillstyle=solid,fillcolor=curcolor]
{
\newpath
\moveto(534.72697266,698.89799194)
\lineto(535.73197266,698.89799194)
\curveto(535.88196967,698.89798189)(536.01196954,698.8879819)(536.12197266,698.86799194)
\curveto(536.24196931,698.85798193)(536.32696923,698.79798199)(536.37697266,698.68799194)
\curveto(536.39696916,698.63798215)(536.40696915,698.57798221)(536.40697266,698.50799194)
\lineto(536.40697266,698.29799194)
\lineto(536.40697266,697.62299194)
\curveto(536.40696915,697.57298322)(536.40196915,697.51298328)(536.39197266,697.44299194)
\curveto(536.39196916,697.38298341)(536.39696916,697.32798346)(536.40697266,697.27799194)
\lineto(536.40697266,697.11299194)
\curveto(536.40696915,697.03298376)(536.41196914,696.95798383)(536.42197266,696.88799194)
\curveto(536.43196912,696.82798396)(536.4569691,696.77298402)(536.49697266,696.72299194)
\curveto(536.56696899,696.63298416)(536.69196886,696.58298421)(536.87197266,696.57299194)
\lineto(537.41197266,696.57299194)
\lineto(537.59197266,696.57299194)
\curveto(537.6519679,696.57298422)(537.70696785,696.56298423)(537.75697266,696.54299194)
\curveto(537.86696769,696.4929843)(537.92696763,696.40298439)(537.93697266,696.27299194)
\curveto(537.9569676,696.14298465)(537.96696759,695.99798479)(537.96697266,695.83799194)
\lineto(537.96697266,695.62799194)
\curveto(537.97696758,695.55798523)(537.97196758,695.49798529)(537.95197266,695.44799194)
\curveto(537.90196765,695.2879855)(537.79696776,695.20298559)(537.63697266,695.19299194)
\curveto(537.47696808,695.18298561)(537.29696826,695.17798561)(537.09697266,695.17799194)
\lineto(536.96197266,695.17799194)
\curveto(536.92196863,695.1879856)(536.88696867,695.1879856)(536.85697266,695.17799194)
\curveto(536.81696874,695.16798562)(536.78196877,695.16298563)(536.75197266,695.16299194)
\curveto(536.72196883,695.17298562)(536.69196886,695.16798562)(536.66197266,695.14799194)
\curveto(536.58196897,695.12798566)(536.52196903,695.08298571)(536.48197266,695.01299194)
\curveto(536.4519691,694.95298584)(536.42696913,694.87798591)(536.40697266,694.78799194)
\curveto(536.39696916,694.73798605)(536.39696916,694.68298611)(536.40697266,694.62299194)
\curveto(536.41696914,694.56298623)(536.41696914,694.50798628)(536.40697266,694.45799194)
\lineto(536.40697266,693.52799194)
\lineto(536.40697266,691.77299194)
\curveto(536.40696915,691.52298927)(536.41196914,691.30298949)(536.42197266,691.11299194)
\curveto(536.44196911,690.93298986)(536.50696905,690.77299002)(536.61697266,690.63299194)
\curveto(536.66696889,690.57299022)(536.73196882,690.52799026)(536.81197266,690.49799194)
\lineto(537.08197266,690.43799194)
\curveto(537.11196844,690.42799036)(537.14196841,690.42299037)(537.17197266,690.42299194)
\curveto(537.21196834,690.43299036)(537.24196831,690.43299036)(537.26197266,690.42299194)
\lineto(537.42697266,690.42299194)
\curveto(537.53696802,690.42299037)(537.63196792,690.41799037)(537.71197266,690.40799194)
\curveto(537.79196776,690.39799039)(537.8569677,690.35799043)(537.90697266,690.28799194)
\curveto(537.94696761,690.22799056)(537.96696759,690.14799064)(537.96697266,690.04799194)
\lineto(537.96697266,689.76299194)
\curveto(537.96696759,689.55299124)(537.96196759,689.35799143)(537.95197266,689.17799194)
\curveto(537.9519676,689.00799178)(537.87196768,688.8929919)(537.71197266,688.83299194)
\curveto(537.66196789,688.81299198)(537.61696794,688.80799198)(537.57697266,688.81799194)
\curveto(537.53696802,688.81799197)(537.49196806,688.80799198)(537.44197266,688.78799194)
\lineto(537.29197266,688.78799194)
\curveto(537.27196828,688.787992)(537.24196831,688.792992)(537.20197266,688.80299194)
\curveto(537.16196839,688.80299199)(537.12696843,688.79799199)(537.09697266,688.78799194)
\curveto(537.04696851,688.77799201)(536.99196856,688.77799201)(536.93197266,688.78799194)
\lineto(536.78197266,688.78799194)
\lineto(536.63197266,688.78799194)
\curveto(536.58196897,688.77799201)(536.53696902,688.77799201)(536.49697266,688.78799194)
\lineto(536.33197266,688.78799194)
\curveto(536.28196927,688.79799199)(536.22696933,688.80299199)(536.16697266,688.80299194)
\curveto(536.10696945,688.80299199)(536.0519695,688.80799198)(536.00197266,688.81799194)
\curveto(535.93196962,688.82799196)(535.86696969,688.83799195)(535.80697266,688.84799194)
\lineto(535.62697266,688.87799194)
\curveto(535.51697004,688.90799188)(535.41197014,688.94299185)(535.31197266,688.98299194)
\curveto(535.21197034,689.02299177)(535.11697044,689.06799172)(535.02697266,689.11799194)
\lineto(534.93697266,689.17799194)
\curveto(534.90697065,689.20799158)(534.87197068,689.23799155)(534.83197266,689.26799194)
\curveto(534.81197074,689.2879915)(534.78697077,689.30799148)(534.75697266,689.32799194)
\lineto(534.68197266,689.40299194)
\curveto(534.54197101,689.5929912)(534.43697112,689.80299099)(534.36697266,690.03299194)
\curveto(534.34697121,690.07299072)(534.33697122,690.10799068)(534.33697266,690.13799194)
\curveto(534.34697121,690.17799061)(534.34697121,690.22299057)(534.33697266,690.27299194)
\curveto(534.32697123,690.2929905)(534.32197123,690.31799047)(534.32197266,690.34799194)
\curveto(534.32197123,690.37799041)(534.31697124,690.40299039)(534.30697266,690.42299194)
\lineto(534.30697266,690.57299194)
\curveto(534.29697126,690.61299018)(534.29197126,690.65799013)(534.29197266,690.70799194)
\curveto(534.30197125,690.75799003)(534.30697125,690.80798998)(534.30697266,690.85799194)
\lineto(534.30697266,691.42799194)
\lineto(534.30697266,693.66299194)
\lineto(534.30697266,694.45799194)
\lineto(534.30697266,694.66799194)
\curveto(534.31697124,694.73798605)(534.31197124,694.80298599)(534.29197266,694.86299194)
\curveto(534.2519713,695.00298579)(534.18197137,695.0929857)(534.08197266,695.13299194)
\curveto(533.97197158,695.18298561)(533.83197172,695.19798559)(533.66197266,695.17799194)
\curveto(533.49197206,695.15798563)(533.34697221,695.17298562)(533.22697266,695.22299194)
\curveto(533.14697241,695.25298554)(533.09697246,695.29798549)(533.07697266,695.35799194)
\curveto(533.0569725,695.41798537)(533.03697252,695.4929853)(533.01697266,695.58299194)
\lineto(533.01697266,695.89799194)
\curveto(533.01697254,696.07798471)(533.02697253,696.22298457)(533.04697266,696.33299194)
\curveto(533.06697249,696.44298435)(533.1519724,696.51798427)(533.30197266,696.55799194)
\curveto(533.34197221,696.57798421)(533.38197217,696.58298421)(533.42197266,696.57299194)
\lineto(533.55697266,696.57299194)
\curveto(533.70697185,696.57298422)(533.84697171,696.57798421)(533.97697266,696.58799194)
\curveto(534.10697145,696.60798418)(534.19697136,696.66798412)(534.24697266,696.76799194)
\curveto(534.27697128,696.83798395)(534.29197126,696.91798387)(534.29197266,697.00799194)
\curveto(534.30197125,697.09798369)(534.30697125,697.1879836)(534.30697266,697.27799194)
\lineto(534.30697266,698.20799194)
\lineto(534.30697266,698.46299194)
\curveto(534.30697125,698.55298224)(534.31697124,698.62798216)(534.33697266,698.68799194)
\curveto(534.38697117,698.787982)(534.46197109,698.85298194)(534.56197266,698.88299194)
\curveto(534.58197097,698.8929819)(534.60697095,698.8929819)(534.63697266,698.88299194)
\curveto(534.67697088,698.88298191)(534.70697085,698.8879819)(534.72697266,698.89799194)
}
}
{
\newrgbcolor{curcolor}{0 0 0}
\pscustom[linestyle=none,fillstyle=solid,fillcolor=curcolor]
{
\newpath
\moveto(541.05041016,699.43799194)
\curveto(541.12040721,699.35798143)(541.15540717,699.23798155)(541.15541016,699.07799194)
\lineto(541.15541016,698.61299194)
\lineto(541.15541016,698.20799194)
\curveto(541.15540717,698.06798272)(541.12040721,697.97298282)(541.05041016,697.92299194)
\curveto(540.99040734,697.87298292)(540.91040742,697.84298295)(540.81041016,697.83299194)
\curveto(540.72040761,697.82298297)(540.62040771,697.81798297)(540.51041016,697.81799194)
\lineto(539.67041016,697.81799194)
\curveto(539.56040877,697.81798297)(539.46040887,697.82298297)(539.37041016,697.83299194)
\curveto(539.29040904,697.84298295)(539.22040911,697.87298292)(539.16041016,697.92299194)
\curveto(539.12040921,697.95298284)(539.09040924,698.00798278)(539.07041016,698.08799194)
\curveto(539.06040927,698.17798261)(539.05040928,698.27298252)(539.04041016,698.37299194)
\lineto(539.04041016,698.70299194)
\curveto(539.05040928,698.81298198)(539.05540927,698.90798188)(539.05541016,698.98799194)
\lineto(539.05541016,699.19799194)
\curveto(539.06540926,699.26798152)(539.08540924,699.32798146)(539.11541016,699.37799194)
\curveto(539.13540919,699.41798137)(539.16040917,699.44798134)(539.19041016,699.46799194)
\lineto(539.31041016,699.52799194)
\curveto(539.330409,699.52798126)(539.35540897,699.52798126)(539.38541016,699.52799194)
\curveto(539.41540891,699.53798125)(539.44040889,699.54298125)(539.46041016,699.54299194)
\lineto(540.55541016,699.54299194)
\curveto(540.65540767,699.54298125)(540.75040758,699.53798125)(540.84041016,699.52799194)
\curveto(540.9304074,699.51798127)(541.00040733,699.4879813)(541.05041016,699.43799194)
\moveto(541.15541016,689.67299194)
\curveto(541.15540717,689.47299132)(541.15040718,689.30299149)(541.14041016,689.16299194)
\curveto(541.1304072,689.02299177)(541.04040729,688.92799186)(540.87041016,688.87799194)
\curveto(540.81040752,688.85799193)(540.74540758,688.84799194)(540.67541016,688.84799194)
\curveto(540.60540772,688.85799193)(540.5304078,688.86299193)(540.45041016,688.86299194)
\lineto(539.61041016,688.86299194)
\curveto(539.52040881,688.86299193)(539.4304089,688.86799192)(539.34041016,688.87799194)
\curveto(539.26040907,688.8879919)(539.20040913,688.91799187)(539.16041016,688.96799194)
\curveto(539.10040923,689.03799175)(539.06540926,689.12299167)(539.05541016,689.22299194)
\lineto(539.05541016,689.56799194)
\lineto(539.05541016,695.89799194)
\lineto(539.05541016,696.19799194)
\curveto(539.05540927,696.29798449)(539.07540925,696.37798441)(539.11541016,696.43799194)
\curveto(539.17540915,696.50798428)(539.26040907,696.55298424)(539.37041016,696.57299194)
\curveto(539.39040894,696.58298421)(539.41540891,696.58298421)(539.44541016,696.57299194)
\curveto(539.48540884,696.57298422)(539.51540881,696.57798421)(539.53541016,696.58799194)
\lineto(540.28541016,696.58799194)
\lineto(540.48041016,696.58799194)
\curveto(540.56040777,696.59798419)(540.6254077,696.59798419)(540.67541016,696.58799194)
\lineto(540.79541016,696.58799194)
\curveto(540.85540747,696.56798422)(540.91040742,696.55298424)(540.96041016,696.54299194)
\curveto(541.01040732,696.53298426)(541.05040728,696.50298429)(541.08041016,696.45299194)
\curveto(541.12040721,696.40298439)(541.14040719,696.33298446)(541.14041016,696.24299194)
\curveto(541.15040718,696.15298464)(541.15540717,696.05798473)(541.15541016,695.95799194)
\lineto(541.15541016,689.67299194)
}
}
{
\newrgbcolor{curcolor}{0 0 0}
\pscustom[linestyle=none,fillstyle=solid,fillcolor=curcolor]
{
\newpath
\moveto(546.38759766,696.79799194)
\curveto(547.1975925,696.81798397)(547.87259182,696.69798409)(548.41259766,696.43799194)
\curveto(548.96259073,696.17798461)(549.3975903,695.80798498)(549.71759766,695.32799194)
\curveto(549.87758982,695.0879857)(549.9975897,694.81298598)(550.07759766,694.50299194)
\curveto(550.0975896,694.45298634)(550.11258958,694.3879864)(550.12259766,694.30799194)
\curveto(550.14258955,694.22798656)(550.14258955,694.15798663)(550.12259766,694.09799194)
\curveto(550.08258961,693.9879868)(550.01258968,693.92298687)(549.91259766,693.90299194)
\curveto(549.81258988,693.8929869)(549.69259,693.8879869)(549.55259766,693.88799194)
\lineto(548.77259766,693.88799194)
\lineto(548.48759766,693.88799194)
\curveto(548.3975913,693.8879869)(548.32259137,693.90798688)(548.26259766,693.94799194)
\curveto(548.18259151,693.9879868)(548.12759157,694.04798674)(548.09759766,694.12799194)
\curveto(548.06759163,694.21798657)(548.02759167,694.30798648)(547.97759766,694.39799194)
\curveto(547.91759178,694.50798628)(547.85259184,694.60798618)(547.78259766,694.69799194)
\curveto(547.71259198,694.787986)(547.63259206,694.86798592)(547.54259766,694.93799194)
\curveto(547.40259229,695.02798576)(547.24759245,695.09798569)(547.07759766,695.14799194)
\curveto(547.01759268,695.16798562)(546.95759274,695.17798561)(546.89759766,695.17799194)
\curveto(546.83759286,695.17798561)(546.78259291,695.1879856)(546.73259766,695.20799194)
\lineto(546.58259766,695.20799194)
\curveto(546.38259331,695.20798558)(546.22259347,695.1879856)(546.10259766,695.14799194)
\curveto(545.81259388,695.05798573)(545.57759412,694.91798587)(545.39759766,694.72799194)
\curveto(545.21759448,694.54798624)(545.07259462,694.32798646)(544.96259766,694.06799194)
\curveto(544.91259478,693.95798683)(544.87259482,693.83798695)(544.84259766,693.70799194)
\curveto(544.82259487,693.5879872)(544.7975949,693.45798733)(544.76759766,693.31799194)
\curveto(544.75759494,693.27798751)(544.75259494,693.23798755)(544.75259766,693.19799194)
\curveto(544.75259494,693.15798763)(544.74759495,693.11798767)(544.73759766,693.07799194)
\curveto(544.71759498,692.97798781)(544.70759499,692.83798795)(544.70759766,692.65799194)
\curveto(544.71759498,692.47798831)(544.73259496,692.33798845)(544.75259766,692.23799194)
\curveto(544.75259494,692.15798863)(544.75759494,692.10298869)(544.76759766,692.07299194)
\curveto(544.78759491,692.00298879)(544.7975949,691.93298886)(544.79759766,691.86299194)
\curveto(544.80759489,691.792989)(544.82259487,691.72298907)(544.84259766,691.65299194)
\curveto(544.92259477,691.42298937)(545.01759468,691.21298958)(545.12759766,691.02299194)
\curveto(545.23759446,690.83298996)(545.37759432,690.67299012)(545.54759766,690.54299194)
\curveto(545.58759411,690.51299028)(545.64759405,690.47799031)(545.72759766,690.43799194)
\curveto(545.83759386,690.36799042)(545.94759375,690.32299047)(546.05759766,690.30299194)
\curveto(546.17759352,690.28299051)(546.32259337,690.26299053)(546.49259766,690.24299194)
\lineto(546.58259766,690.24299194)
\curveto(546.62259307,690.24299055)(546.65259304,690.24799054)(546.67259766,690.25799194)
\lineto(546.80759766,690.25799194)
\curveto(546.87759282,690.27799051)(546.94259275,690.2929905)(547.00259766,690.30299194)
\curveto(547.07259262,690.32299047)(547.13759256,690.34299045)(547.19759766,690.36299194)
\curveto(547.4975922,690.4929903)(547.72759197,690.68299011)(547.88759766,690.93299194)
\curveto(547.92759177,690.98298981)(547.96259173,691.03798975)(547.99259766,691.09799194)
\curveto(548.02259167,691.16798962)(548.04759165,691.22798956)(548.06759766,691.27799194)
\curveto(548.10759159,691.3879894)(548.14259155,691.48298931)(548.17259766,691.56299194)
\curveto(548.20259149,691.65298914)(548.27259142,691.72298907)(548.38259766,691.77299194)
\curveto(548.47259122,691.81298898)(548.61759108,691.82798896)(548.81759766,691.81799194)
\lineto(549.31259766,691.81799194)
\lineto(549.52259766,691.81799194)
\curveto(549.60259009,691.82798896)(549.66759003,691.82298897)(549.71759766,691.80299194)
\lineto(549.83759766,691.80299194)
\lineto(549.95759766,691.77299194)
\curveto(549.9975897,691.77298902)(550.02758967,691.76298903)(550.04759766,691.74299194)
\curveto(550.0975896,691.70298909)(550.12758957,691.64298915)(550.13759766,691.56299194)
\curveto(550.15758954,691.4929893)(550.15758954,691.41798937)(550.13759766,691.33799194)
\curveto(550.04758965,691.00798978)(549.93758976,690.71299008)(549.80759766,690.45299194)
\curveto(549.3975903,689.68299111)(548.74259095,689.14799164)(547.84259766,688.84799194)
\curveto(547.74259195,688.81799197)(547.63759206,688.79799199)(547.52759766,688.78799194)
\curveto(547.41759228,688.76799202)(547.30759239,688.74299205)(547.19759766,688.71299194)
\curveto(547.13759256,688.70299209)(547.07759262,688.69799209)(547.01759766,688.69799194)
\curveto(546.95759274,688.69799209)(546.8975928,688.6929921)(546.83759766,688.68299194)
\lineto(546.67259766,688.68299194)
\curveto(546.62259307,688.66299213)(546.54759315,688.65799213)(546.44759766,688.66799194)
\curveto(546.34759335,688.66799212)(546.27259342,688.67299212)(546.22259766,688.68299194)
\curveto(546.14259355,688.70299209)(546.06759363,688.71299208)(545.99759766,688.71299194)
\curveto(545.93759376,688.70299209)(545.87259382,688.70799208)(545.80259766,688.72799194)
\lineto(545.65259766,688.75799194)
\curveto(545.60259409,688.75799203)(545.55259414,688.76299203)(545.50259766,688.77299194)
\curveto(545.3925943,688.80299199)(545.28759441,688.83299196)(545.18759766,688.86299194)
\curveto(545.08759461,688.8929919)(544.9925947,688.92799186)(544.90259766,688.96799194)
\curveto(544.43259526,689.16799162)(544.03759566,689.42299137)(543.71759766,689.73299194)
\curveto(543.3975963,690.05299074)(543.13759656,690.44799034)(542.93759766,690.91799194)
\curveto(542.88759681,691.00798978)(542.84759685,691.10298969)(542.81759766,691.20299194)
\lineto(542.72759766,691.53299194)
\curveto(542.71759698,691.57298922)(542.71259698,691.60798918)(542.71259766,691.63799194)
\curveto(542.71259698,691.67798911)(542.70259699,691.72298907)(542.68259766,691.77299194)
\curveto(542.66259703,691.84298895)(542.65259704,691.91298888)(542.65259766,691.98299194)
\curveto(542.65259704,692.06298873)(542.64259705,692.13798865)(542.62259766,692.20799194)
\lineto(542.62259766,692.46299194)
\curveto(542.60259709,692.51298828)(542.5925971,692.56798822)(542.59259766,692.62799194)
\curveto(542.5925971,692.69798809)(542.60259709,692.75798803)(542.62259766,692.80799194)
\curveto(542.63259706,692.85798793)(542.63259706,692.90298789)(542.62259766,692.94299194)
\curveto(542.61259708,692.98298781)(542.61259708,693.02298777)(542.62259766,693.06299194)
\curveto(542.64259705,693.13298766)(542.64759705,693.19798759)(542.63759766,693.25799194)
\curveto(542.63759706,693.31798747)(542.64759705,693.37798741)(542.66759766,693.43799194)
\curveto(542.71759698,693.61798717)(542.75759694,693.787987)(542.78759766,693.94799194)
\curveto(542.81759688,694.11798667)(542.86259683,694.28298651)(542.92259766,694.44299194)
\curveto(543.14259655,694.95298584)(543.41759628,695.37798541)(543.74759766,695.71799194)
\curveto(544.08759561,696.05798473)(544.51759518,696.33298446)(545.03759766,696.54299194)
\curveto(545.17759452,696.60298419)(545.32259437,696.64298415)(545.47259766,696.66299194)
\curveto(545.62259407,696.6929841)(545.77759392,696.72798406)(545.93759766,696.76799194)
\curveto(546.01759368,696.77798401)(546.0925936,696.78298401)(546.16259766,696.78299194)
\curveto(546.23259346,696.78298401)(546.30759339,696.787984)(546.38759766,696.79799194)
}
}
{
\newrgbcolor{curcolor}{0 0 0}
\pscustom[linestyle=none,fillstyle=solid,fillcolor=curcolor]
{
\newpath
\moveto(553.53087891,699.43799194)
\curveto(553.60087596,699.35798143)(553.63587592,699.23798155)(553.63587891,699.07799194)
\lineto(553.63587891,698.61299194)
\lineto(553.63587891,698.20799194)
\curveto(553.63587592,698.06798272)(553.60087596,697.97298282)(553.53087891,697.92299194)
\curveto(553.47087609,697.87298292)(553.39087617,697.84298295)(553.29087891,697.83299194)
\curveto(553.20087636,697.82298297)(553.10087646,697.81798297)(552.99087891,697.81799194)
\lineto(552.15087891,697.81799194)
\curveto(552.04087752,697.81798297)(551.94087762,697.82298297)(551.85087891,697.83299194)
\curveto(551.77087779,697.84298295)(551.70087786,697.87298292)(551.64087891,697.92299194)
\curveto(551.60087796,697.95298284)(551.57087799,698.00798278)(551.55087891,698.08799194)
\curveto(551.54087802,698.17798261)(551.53087803,698.27298252)(551.52087891,698.37299194)
\lineto(551.52087891,698.70299194)
\curveto(551.53087803,698.81298198)(551.53587802,698.90798188)(551.53587891,698.98799194)
\lineto(551.53587891,699.19799194)
\curveto(551.54587801,699.26798152)(551.56587799,699.32798146)(551.59587891,699.37799194)
\curveto(551.61587794,699.41798137)(551.64087792,699.44798134)(551.67087891,699.46799194)
\lineto(551.79087891,699.52799194)
\curveto(551.81087775,699.52798126)(551.83587772,699.52798126)(551.86587891,699.52799194)
\curveto(551.89587766,699.53798125)(551.92087764,699.54298125)(551.94087891,699.54299194)
\lineto(553.03587891,699.54299194)
\curveto(553.13587642,699.54298125)(553.23087633,699.53798125)(553.32087891,699.52799194)
\curveto(553.41087615,699.51798127)(553.48087608,699.4879813)(553.53087891,699.43799194)
\moveto(553.63587891,689.67299194)
\curveto(553.63587592,689.47299132)(553.63087593,689.30299149)(553.62087891,689.16299194)
\curveto(553.61087595,689.02299177)(553.52087604,688.92799186)(553.35087891,688.87799194)
\curveto(553.29087627,688.85799193)(553.22587633,688.84799194)(553.15587891,688.84799194)
\curveto(553.08587647,688.85799193)(553.01087655,688.86299193)(552.93087891,688.86299194)
\lineto(552.09087891,688.86299194)
\curveto(552.00087756,688.86299193)(551.91087765,688.86799192)(551.82087891,688.87799194)
\curveto(551.74087782,688.8879919)(551.68087788,688.91799187)(551.64087891,688.96799194)
\curveto(551.58087798,689.03799175)(551.54587801,689.12299167)(551.53587891,689.22299194)
\lineto(551.53587891,689.56799194)
\lineto(551.53587891,695.89799194)
\lineto(551.53587891,696.19799194)
\curveto(551.53587802,696.29798449)(551.555878,696.37798441)(551.59587891,696.43799194)
\curveto(551.6558779,696.50798428)(551.74087782,696.55298424)(551.85087891,696.57299194)
\curveto(551.87087769,696.58298421)(551.89587766,696.58298421)(551.92587891,696.57299194)
\curveto(551.96587759,696.57298422)(551.99587756,696.57798421)(552.01587891,696.58799194)
\lineto(552.76587891,696.58799194)
\lineto(552.96087891,696.58799194)
\curveto(553.04087652,696.59798419)(553.10587645,696.59798419)(553.15587891,696.58799194)
\lineto(553.27587891,696.58799194)
\curveto(553.33587622,696.56798422)(553.39087617,696.55298424)(553.44087891,696.54299194)
\curveto(553.49087607,696.53298426)(553.53087603,696.50298429)(553.56087891,696.45299194)
\curveto(553.60087596,696.40298439)(553.62087594,696.33298446)(553.62087891,696.24299194)
\curveto(553.63087593,696.15298464)(553.63587592,696.05798473)(553.63587891,695.95799194)
\lineto(553.63587891,689.67299194)
}
}
{
\newrgbcolor{curcolor}{0 0 0}
\pscustom[linestyle=none,fillstyle=solid,fillcolor=curcolor]
{
\newpath
\moveto(563.18806641,692.80799194)
\curveto(563.19805773,692.74798804)(563.20305772,692.65798813)(563.20306641,692.53799194)
\curveto(563.20305772,692.41798837)(563.19305773,692.33298846)(563.17306641,692.28299194)
\lineto(563.17306641,692.08799194)
\curveto(563.14305778,691.97798881)(563.1230578,691.87298892)(563.11306641,691.77299194)
\curveto(563.11305781,691.67298912)(563.09805783,691.57298922)(563.06806641,691.47299194)
\curveto(563.04805788,691.38298941)(563.0280579,691.2879895)(563.00806641,691.18799194)
\curveto(562.98805794,691.09798969)(562.95805797,691.00798978)(562.91806641,690.91799194)
\curveto(562.84805808,690.74799004)(562.77805815,690.5879902)(562.70806641,690.43799194)
\curveto(562.63805829,690.29799049)(562.55805837,690.15799063)(562.46806641,690.01799194)
\curveto(562.40805852,689.92799086)(562.34305858,689.84299095)(562.27306641,689.76299194)
\curveto(562.21305871,689.6929911)(562.14305878,689.61799117)(562.06306641,689.53799194)
\lineto(561.95806641,689.43299194)
\curveto(561.90805902,689.38299141)(561.85305907,689.33799145)(561.79306641,689.29799194)
\lineto(561.64306641,689.17799194)
\curveto(561.56305936,689.11799167)(561.47305945,689.06299173)(561.37306641,689.01299194)
\curveto(561.28305964,688.97299182)(561.18805974,688.92799186)(561.08806641,688.87799194)
\curveto(560.98805994,688.82799196)(560.88306004,688.792992)(560.77306641,688.77299194)
\curveto(560.67306025,688.75299204)(560.56806036,688.73299206)(560.45806641,688.71299194)
\curveto(560.39806053,688.6929921)(560.33306059,688.68299211)(560.26306641,688.68299194)
\curveto(560.20306072,688.68299211)(560.13806079,688.67299212)(560.06806641,688.65299194)
\lineto(559.93306641,688.65299194)
\curveto(559.85306107,688.63299216)(559.77806115,688.63299216)(559.70806641,688.65299194)
\lineto(559.55806641,688.65299194)
\curveto(559.49806143,688.67299212)(559.43306149,688.68299211)(559.36306641,688.68299194)
\curveto(559.30306162,688.67299212)(559.24306168,688.67799211)(559.18306641,688.69799194)
\curveto(559.0230619,688.74799204)(558.86806206,688.792992)(558.71806641,688.83299194)
\curveto(558.57806235,688.87299192)(558.44806248,688.93299186)(558.32806641,689.01299194)
\curveto(558.25806267,689.05299174)(558.19306273,689.0929917)(558.13306641,689.13299194)
\curveto(558.07306285,689.18299161)(558.00806292,689.23299156)(557.93806641,689.28299194)
\lineto(557.75806641,689.41799194)
\curveto(557.67806325,689.47799131)(557.60806332,689.48299131)(557.54806641,689.43299194)
\curveto(557.49806343,689.40299139)(557.47306345,689.36299143)(557.47306641,689.31299194)
\curveto(557.47306345,689.27299152)(557.46306346,689.22299157)(557.44306641,689.16299194)
\curveto(557.4230635,689.06299173)(557.41306351,688.94799184)(557.41306641,688.81799194)
\curveto(557.4230635,688.6879921)(557.4280635,688.56799222)(557.42806641,688.45799194)
\lineto(557.42806641,686.92799194)
\curveto(557.4280635,686.79799399)(557.4230635,686.67299412)(557.41306641,686.55299194)
\curveto(557.41306351,686.42299437)(557.38806354,686.31799447)(557.33806641,686.23799194)
\curveto(557.30806362,686.19799459)(557.25306367,686.16799462)(557.17306641,686.14799194)
\curveto(557.09306383,686.12799466)(557.00306392,686.11799467)(556.90306641,686.11799194)
\curveto(556.80306412,686.10799468)(556.70306422,686.10799468)(556.60306641,686.11799194)
\lineto(556.34806641,686.11799194)
\lineto(555.94306641,686.11799194)
\lineto(555.83806641,686.11799194)
\curveto(555.79806513,686.11799467)(555.76306516,686.12299467)(555.73306641,686.13299194)
\lineto(555.61306641,686.13299194)
\curveto(555.44306548,686.18299461)(555.35306557,686.28299451)(555.34306641,686.43299194)
\curveto(555.33306559,686.57299422)(555.3280656,686.74299405)(555.32806641,686.94299194)
\lineto(555.32806641,695.74799194)
\curveto(555.3280656,695.85798493)(555.3230656,695.97298482)(555.31306641,696.09299194)
\curveto(555.31306561,696.22298457)(555.33806559,696.32298447)(555.38806641,696.39299194)
\curveto(555.4280655,696.46298433)(555.48306544,696.50798428)(555.55306641,696.52799194)
\curveto(555.60306532,696.54798424)(555.66306526,696.55798423)(555.73306641,696.55799194)
\lineto(555.95806641,696.55799194)
\lineto(556.67806641,696.55799194)
\lineto(556.96306641,696.55799194)
\curveto(557.05306387,696.55798423)(557.1280638,696.53298426)(557.18806641,696.48299194)
\curveto(557.25806367,696.43298436)(557.29306363,696.36798442)(557.29306641,696.28799194)
\curveto(557.30306362,696.21798457)(557.3280636,696.14298465)(557.36806641,696.06299194)
\curveto(557.37806355,696.03298476)(557.38806354,696.00798478)(557.39806641,695.98799194)
\curveto(557.41806351,695.97798481)(557.43806349,695.96298483)(557.45806641,695.94299194)
\curveto(557.56806336,695.93298486)(557.65806327,695.96298483)(557.72806641,696.03299194)
\curveto(557.79806313,696.10298469)(557.86806306,696.16298463)(557.93806641,696.21299194)
\curveto(558.06806286,696.30298449)(558.20306272,696.38298441)(558.34306641,696.45299194)
\curveto(558.48306244,696.53298426)(558.63806229,696.59798419)(558.80806641,696.64799194)
\curveto(558.88806204,696.67798411)(558.97306195,696.69798409)(559.06306641,696.70799194)
\curveto(559.16306176,696.71798407)(559.25806167,696.73298406)(559.34806641,696.75299194)
\curveto(559.38806154,696.76298403)(559.4280615,696.76298403)(559.46806641,696.75299194)
\curveto(559.51806141,696.74298405)(559.55806137,696.74798404)(559.58806641,696.76799194)
\curveto(560.15806077,696.787984)(560.63806029,696.70798408)(561.02806641,696.52799194)
\curveto(561.4280595,696.35798443)(561.76805916,696.13298466)(562.04806641,695.85299194)
\curveto(562.09805883,695.80298499)(562.14305878,695.75298504)(562.18306641,695.70299194)
\curveto(562.2230587,695.66298513)(562.26305866,695.61798517)(562.30306641,695.56799194)
\curveto(562.37305855,695.47798531)(562.43305849,695.3879854)(562.48306641,695.29799194)
\curveto(562.54305838,695.20798558)(562.59805833,695.11798567)(562.64806641,695.02799194)
\curveto(562.66805826,695.00798578)(562.67805825,694.98298581)(562.67806641,694.95299194)
\curveto(562.68805824,694.92298587)(562.70305822,694.8879859)(562.72306641,694.84799194)
\curveto(562.78305814,694.74798604)(562.83805809,694.62798616)(562.88806641,694.48799194)
\curveto(562.90805802,694.42798636)(562.928058,694.36298643)(562.94806641,694.29299194)
\curveto(562.96805796,694.23298656)(562.98805794,694.16798662)(563.00806641,694.09799194)
\curveto(563.04805788,693.97798681)(563.07305785,693.85298694)(563.08306641,693.72299194)
\curveto(563.10305782,693.5929872)(563.1280578,693.45798733)(563.15806641,693.31799194)
\lineto(563.15806641,693.15299194)
\lineto(563.18806641,692.97299194)
\lineto(563.18806641,692.80799194)
\moveto(561.07306641,692.46299194)
\curveto(561.08305984,692.51298828)(561.08805984,692.57798821)(561.08806641,692.65799194)
\curveto(561.08805984,692.74798804)(561.08305984,692.81798797)(561.07306641,692.86799194)
\lineto(561.07306641,693.00299194)
\curveto(561.05305987,693.06298773)(561.04305988,693.12798766)(561.04306641,693.19799194)
\curveto(561.04305988,693.26798752)(561.03305989,693.33798745)(561.01306641,693.40799194)
\curveto(560.99305993,693.50798728)(560.97305995,693.60298719)(560.95306641,693.69299194)
\curveto(560.93305999,693.792987)(560.90306002,693.88298691)(560.86306641,693.96299194)
\curveto(560.74306018,694.28298651)(560.58806034,694.53798625)(560.39806641,694.72799194)
\curveto(560.20806072,694.91798587)(559.93806099,695.05798573)(559.58806641,695.14799194)
\curveto(559.50806142,695.16798562)(559.41806151,695.17798561)(559.31806641,695.17799194)
\lineto(559.04806641,695.17799194)
\curveto(559.00806192,695.16798562)(558.97306195,695.16298563)(558.94306641,695.16299194)
\curveto(558.91306201,695.16298563)(558.87806205,695.15798563)(558.83806641,695.14799194)
\lineto(558.62806641,695.08799194)
\curveto(558.56806236,695.07798571)(558.50806242,695.05798573)(558.44806641,695.02799194)
\curveto(558.18806274,694.91798587)(557.98306294,694.74798604)(557.83306641,694.51799194)
\curveto(557.69306323,694.2879865)(557.57806335,694.03298676)(557.48806641,693.75299194)
\curveto(557.46806346,693.67298712)(557.45306347,693.5879872)(557.44306641,693.49799194)
\curveto(557.43306349,693.41798737)(557.41806351,693.33798745)(557.39806641,693.25799194)
\curveto(557.38806354,693.21798757)(557.38306354,693.15298764)(557.38306641,693.06299194)
\curveto(557.36306356,693.02298777)(557.35806357,692.97298782)(557.36806641,692.91299194)
\curveto(557.37806355,692.86298793)(557.37806355,692.81298798)(557.36806641,692.76299194)
\curveto(557.34806358,692.70298809)(557.34806358,692.64798814)(557.36806641,692.59799194)
\lineto(557.36806641,692.41799194)
\lineto(557.36806641,692.28299194)
\curveto(557.36806356,692.24298855)(557.37806355,692.20298859)(557.39806641,692.16299194)
\curveto(557.39806353,692.0929887)(557.40306352,692.03798875)(557.41306641,691.99799194)
\lineto(557.44306641,691.81799194)
\curveto(557.45306347,691.75798903)(557.46806346,691.69798909)(557.48806641,691.63799194)
\curveto(557.57806335,691.34798944)(557.68306324,691.10798968)(557.80306641,690.91799194)
\curveto(557.93306299,690.73799005)(558.11306281,690.57799021)(558.34306641,690.43799194)
\curveto(558.48306244,690.35799043)(558.64806228,690.2929905)(558.83806641,690.24299194)
\curveto(558.87806205,690.23299056)(558.91306201,690.22799056)(558.94306641,690.22799194)
\curveto(558.97306195,690.23799055)(559.00806192,690.23799055)(559.04806641,690.22799194)
\curveto(559.08806184,690.21799057)(559.14806178,690.20799058)(559.22806641,690.19799194)
\curveto(559.30806162,690.19799059)(559.37306155,690.20299059)(559.42306641,690.21299194)
\curveto(559.50306142,690.23299056)(559.58306134,690.24799054)(559.66306641,690.25799194)
\curveto(559.75306117,690.27799051)(559.83806109,690.30299049)(559.91806641,690.33299194)
\curveto(560.15806077,690.43299036)(560.35306057,690.57299022)(560.50306641,690.75299194)
\curveto(560.65306027,690.93298986)(560.77806015,691.14298965)(560.87806641,691.38299194)
\curveto(560.92806,691.50298929)(560.96305996,691.62798916)(560.98306641,691.75799194)
\curveto(561.00305992,691.8879889)(561.0280599,692.02298877)(561.05806641,692.16299194)
\lineto(561.05806641,692.31299194)
\curveto(561.06805986,692.36298843)(561.07305985,692.41298838)(561.07306641,692.46299194)
}
}
{
\newrgbcolor{curcolor}{0 0 0}
\pscustom[linestyle=none,fillstyle=solid,fillcolor=curcolor]
{
\newpath
\moveto(571.51798828,689.44799194)
\curveto(571.53798043,689.33799145)(571.54798042,689.22799156)(571.54798828,689.11799194)
\curveto(571.55798041,689.00799178)(571.50798046,688.93299186)(571.39798828,688.89299194)
\curveto(571.33798063,688.86299193)(571.2679807,688.84799194)(571.18798828,688.84799194)
\lineto(570.94798828,688.84799194)
\lineto(570.13798828,688.84799194)
\lineto(569.86798828,688.84799194)
\curveto(569.78798218,688.85799193)(569.72298225,688.88299191)(569.67298828,688.92299194)
\curveto(569.60298237,688.96299183)(569.54798242,689.01799177)(569.50798828,689.08799194)
\curveto(569.47798249,689.16799162)(569.43298254,689.23299156)(569.37298828,689.28299194)
\curveto(569.35298262,689.30299149)(569.32798264,689.31799147)(569.29798828,689.32799194)
\curveto(569.2679827,689.34799144)(569.22798274,689.35299144)(569.17798828,689.34299194)
\curveto(569.12798284,689.32299147)(569.07798289,689.29799149)(569.02798828,689.26799194)
\curveto(568.98798298,689.23799155)(568.94298303,689.21299158)(568.89298828,689.19299194)
\curveto(568.84298313,689.15299164)(568.78798318,689.11799167)(568.72798828,689.08799194)
\lineto(568.54798828,688.99799194)
\curveto(568.41798355,688.93799185)(568.28298369,688.8879919)(568.14298828,688.84799194)
\curveto(568.00298397,688.81799197)(567.85798411,688.78299201)(567.70798828,688.74299194)
\curveto(567.63798433,688.72299207)(567.5679844,688.71299208)(567.49798828,688.71299194)
\curveto(567.43798453,688.70299209)(567.3729846,688.6929921)(567.30298828,688.68299194)
\lineto(567.21298828,688.68299194)
\curveto(567.18298479,688.67299212)(567.15298482,688.66799212)(567.12298828,688.66799194)
\lineto(566.95798828,688.66799194)
\curveto(566.85798511,688.64799214)(566.75798521,688.64799214)(566.65798828,688.66799194)
\lineto(566.52298828,688.66799194)
\curveto(566.45298552,688.6879921)(566.38298559,688.69799209)(566.31298828,688.69799194)
\curveto(566.25298572,688.6879921)(566.19298578,688.6929921)(566.13298828,688.71299194)
\curveto(566.03298594,688.73299206)(565.93798603,688.75299204)(565.84798828,688.77299194)
\curveto(565.75798621,688.78299201)(565.6729863,688.80799198)(565.59298828,688.84799194)
\curveto(565.30298667,688.95799183)(565.05298692,689.09799169)(564.84298828,689.26799194)
\curveto(564.64298733,689.44799134)(564.48298749,689.68299111)(564.36298828,689.97299194)
\curveto(564.33298764,690.04299075)(564.30298767,690.11799067)(564.27298828,690.19799194)
\curveto(564.25298772,690.27799051)(564.23298774,690.36299043)(564.21298828,690.45299194)
\curveto(564.19298778,690.50299029)(564.18298779,690.55299024)(564.18298828,690.60299194)
\curveto(564.19298778,690.65299014)(564.19298778,690.70299009)(564.18298828,690.75299194)
\curveto(564.1729878,690.78299001)(564.16298781,690.84298995)(564.15298828,690.93299194)
\curveto(564.15298782,691.03298976)(564.15798781,691.10298969)(564.16798828,691.14299194)
\curveto(564.18798778,691.24298955)(564.19798777,691.32798946)(564.19798828,691.39799194)
\lineto(564.28798828,691.72799194)
\curveto(564.31798765,691.84798894)(564.35798761,691.95298884)(564.40798828,692.04299194)
\curveto(564.57798739,692.33298846)(564.7729872,692.55298824)(564.99298828,692.70299194)
\curveto(565.21298676,692.85298794)(565.49298648,692.98298781)(565.83298828,693.09299194)
\curveto(565.96298601,693.14298765)(566.09798587,693.17798761)(566.23798828,693.19799194)
\curveto(566.37798559,693.21798757)(566.51798545,693.24298755)(566.65798828,693.27299194)
\curveto(566.73798523,693.2929875)(566.82298515,693.30298749)(566.91298828,693.30299194)
\curveto(567.00298497,693.31298748)(567.09298488,693.32798746)(567.18298828,693.34799194)
\curveto(567.25298472,693.36798742)(567.32298465,693.37298742)(567.39298828,693.36299194)
\curveto(567.46298451,693.36298743)(567.53798443,693.37298742)(567.61798828,693.39299194)
\curveto(567.68798428,693.41298738)(567.75798421,693.42298737)(567.82798828,693.42299194)
\curveto(567.89798407,693.42298737)(567.972984,693.43298736)(568.05298828,693.45299194)
\curveto(568.26298371,693.50298729)(568.45298352,693.54298725)(568.62298828,693.57299194)
\curveto(568.80298317,693.61298718)(568.96298301,693.70298709)(569.10298828,693.84299194)
\curveto(569.19298278,693.93298686)(569.25298272,694.03298676)(569.28298828,694.14299194)
\curveto(569.29298268,694.17298662)(569.29298268,694.19798659)(569.28298828,694.21799194)
\curveto(569.28298269,694.23798655)(569.28798268,694.25798653)(569.29798828,694.27799194)
\curveto(569.30798266,694.29798649)(569.31298266,694.32798646)(569.31298828,694.36799194)
\lineto(569.31298828,694.45799194)
\lineto(569.28298828,694.57799194)
\curveto(569.28298269,694.61798617)(569.27798269,694.65298614)(569.26798828,694.68299194)
\curveto(569.1679828,694.98298581)(568.95798301,695.1879856)(568.63798828,695.29799194)
\curveto(568.54798342,695.32798546)(568.43798353,695.34798544)(568.30798828,695.35799194)
\curveto(568.18798378,695.37798541)(568.06298391,695.38298541)(567.93298828,695.37299194)
\curveto(567.80298417,695.37298542)(567.67798429,695.36298543)(567.55798828,695.34299194)
\curveto(567.43798453,695.32298547)(567.33298464,695.29798549)(567.24298828,695.26799194)
\curveto(567.18298479,695.24798554)(567.12298485,695.21798557)(567.06298828,695.17799194)
\curveto(567.01298496,695.14798564)(566.96298501,695.11298568)(566.91298828,695.07299194)
\curveto(566.86298511,695.03298576)(566.80798516,694.97798581)(566.74798828,694.90799194)
\curveto(566.69798527,694.83798595)(566.66298531,694.77298602)(566.64298828,694.71299194)
\curveto(566.59298538,694.61298618)(566.54798542,694.51798627)(566.50798828,694.42799194)
\curveto(566.47798549,694.33798645)(566.40798556,694.27798651)(566.29798828,694.24799194)
\curveto(566.21798575,694.22798656)(566.13298584,694.21798657)(566.04298828,694.21799194)
\lineto(565.77298828,694.21799194)
\lineto(565.20298828,694.21799194)
\curveto(565.15298682,694.21798657)(565.10298687,694.21298658)(565.05298828,694.20299194)
\curveto(565.00298697,694.20298659)(564.95798701,694.20798658)(564.91798828,694.21799194)
\lineto(564.78298828,694.21799194)
\curveto(564.76298721,694.22798656)(564.73798723,694.23298656)(564.70798828,694.23299194)
\curveto(564.67798729,694.23298656)(564.65298732,694.24298655)(564.63298828,694.26299194)
\curveto(564.55298742,694.28298651)(564.49798747,694.34798644)(564.46798828,694.45799194)
\curveto(564.45798751,694.50798628)(564.45798751,694.55798623)(564.46798828,694.60799194)
\curveto(564.47798749,694.65798613)(564.48798748,694.70298609)(564.49798828,694.74299194)
\curveto(564.52798744,694.85298594)(564.55798741,694.95298584)(564.58798828,695.04299194)
\curveto(564.62798734,695.14298565)(564.6729873,695.23298556)(564.72298828,695.31299194)
\lineto(564.81298828,695.46299194)
\lineto(564.90298828,695.61299194)
\curveto(564.98298699,695.72298507)(565.08298689,695.82798496)(565.20298828,695.92799194)
\curveto(565.22298675,695.93798485)(565.25298672,695.96298483)(565.29298828,696.00299194)
\curveto(565.34298663,696.04298475)(565.38798658,696.07798471)(565.42798828,696.10799194)
\curveto(565.4679865,696.13798465)(565.51298646,696.16798462)(565.56298828,696.19799194)
\curveto(565.73298624,696.30798448)(565.91298606,696.3929844)(566.10298828,696.45299194)
\curveto(566.29298568,696.52298427)(566.48798548,696.5879842)(566.68798828,696.64799194)
\curveto(566.80798516,696.67798411)(566.93298504,696.69798409)(567.06298828,696.70799194)
\curveto(567.19298478,696.71798407)(567.32298465,696.73798405)(567.45298828,696.76799194)
\curveto(567.49298448,696.77798401)(567.55298442,696.77798401)(567.63298828,696.76799194)
\curveto(567.72298425,696.75798403)(567.77798419,696.76298403)(567.79798828,696.78299194)
\curveto(568.20798376,696.792984)(568.59798337,696.77798401)(568.96798828,696.73799194)
\curveto(569.34798262,696.69798409)(569.68798228,696.62298417)(569.98798828,696.51299194)
\curveto(570.29798167,696.40298439)(570.56298141,696.25298454)(570.78298828,696.06299194)
\curveto(571.00298097,695.88298491)(571.1729808,695.64798514)(571.29298828,695.35799194)
\curveto(571.36298061,695.1879856)(571.40298057,694.9929858)(571.41298828,694.77299194)
\curveto(571.42298055,694.55298624)(571.42798054,694.32798646)(571.42798828,694.09799194)
\lineto(571.42798828,690.75299194)
\lineto(571.42798828,690.16799194)
\curveto(571.42798054,689.97799081)(571.44798052,689.80299099)(571.48798828,689.64299194)
\curveto(571.49798047,689.61299118)(571.50298047,689.57799121)(571.50298828,689.53799194)
\curveto(571.50298047,689.50799128)(571.50798046,689.47799131)(571.51798828,689.44799194)
\moveto(569.31298828,691.75799194)
\curveto(569.32298265,691.80798898)(569.32798264,691.86298893)(569.32798828,691.92299194)
\curveto(569.32798264,691.9929888)(569.32298265,692.05298874)(569.31298828,692.10299194)
\curveto(569.29298268,692.16298863)(569.28298269,692.21798857)(569.28298828,692.26799194)
\curveto(569.28298269,692.31798847)(569.26298271,692.35798843)(569.22298828,692.38799194)
\curveto(569.1729828,692.42798836)(569.09798287,692.44798834)(568.99798828,692.44799194)
\curveto(568.95798301,692.43798835)(568.92298305,692.42798836)(568.89298828,692.41799194)
\curveto(568.86298311,692.41798837)(568.82798314,692.41298838)(568.78798828,692.40299194)
\curveto(568.71798325,692.38298841)(568.64298333,692.36798842)(568.56298828,692.35799194)
\curveto(568.48298349,692.34798844)(568.40298357,692.33298846)(568.32298828,692.31299194)
\curveto(568.29298368,692.30298849)(568.24798372,692.29798849)(568.18798828,692.29799194)
\curveto(568.05798391,692.26798852)(567.92798404,692.24798854)(567.79798828,692.23799194)
\curveto(567.6679843,692.22798856)(567.54298443,692.20298859)(567.42298828,692.16299194)
\curveto(567.34298463,692.14298865)(567.2679847,692.12298867)(567.19798828,692.10299194)
\curveto(567.12798484,692.0929887)(567.05798491,692.07298872)(566.98798828,692.04299194)
\curveto(566.77798519,691.95298884)(566.59798537,691.81798897)(566.44798828,691.63799194)
\curveto(566.30798566,691.45798933)(566.25798571,691.20798958)(566.29798828,690.88799194)
\curveto(566.31798565,690.71799007)(566.3729856,690.57799021)(566.46298828,690.46799194)
\curveto(566.53298544,690.35799043)(566.63798533,690.26799052)(566.77798828,690.19799194)
\curveto(566.91798505,690.13799065)(567.0679849,690.0929907)(567.22798828,690.06299194)
\curveto(567.39798457,690.03299076)(567.5729844,690.02299077)(567.75298828,690.03299194)
\curveto(567.94298403,690.05299074)(568.11798385,690.0879907)(568.27798828,690.13799194)
\curveto(568.53798343,690.21799057)(568.74298323,690.34299045)(568.89298828,690.51299194)
\curveto(569.04298293,690.6929901)(569.15798281,690.91298988)(569.23798828,691.17299194)
\curveto(569.25798271,691.24298955)(569.2679827,691.31298948)(569.26798828,691.38299194)
\curveto(569.27798269,691.46298933)(569.29298268,691.54298925)(569.31298828,691.62299194)
\lineto(569.31298828,691.75799194)
}
}
{
\newrgbcolor{curcolor}{0 0 0}
\pscustom[linestyle=none,fillstyle=solid,fillcolor=curcolor]
{
\newpath
\moveto(576.65126953,696.79799194)
\curveto(577.46126437,696.81798397)(578.1362637,696.69798409)(578.67626953,696.43799194)
\curveto(579.22626261,696.17798461)(579.66126217,695.80798498)(579.98126953,695.32799194)
\curveto(580.14126169,695.0879857)(580.26126157,694.81298598)(580.34126953,694.50299194)
\curveto(580.36126147,694.45298634)(580.37626146,694.3879864)(580.38626953,694.30799194)
\curveto(580.40626143,694.22798656)(580.40626143,694.15798663)(580.38626953,694.09799194)
\curveto(580.34626149,693.9879868)(580.27626156,693.92298687)(580.17626953,693.90299194)
\curveto(580.07626176,693.8929869)(579.95626188,693.8879869)(579.81626953,693.88799194)
\lineto(579.03626953,693.88799194)
\lineto(578.75126953,693.88799194)
\curveto(578.66126317,693.8879869)(578.58626325,693.90798688)(578.52626953,693.94799194)
\curveto(578.44626339,693.9879868)(578.39126344,694.04798674)(578.36126953,694.12799194)
\curveto(578.3312635,694.21798657)(578.29126354,694.30798648)(578.24126953,694.39799194)
\curveto(578.18126365,694.50798628)(578.11626372,694.60798618)(578.04626953,694.69799194)
\curveto(577.97626386,694.787986)(577.89626394,694.86798592)(577.80626953,694.93799194)
\curveto(577.66626417,695.02798576)(577.51126432,695.09798569)(577.34126953,695.14799194)
\curveto(577.28126455,695.16798562)(577.22126461,695.17798561)(577.16126953,695.17799194)
\curveto(577.10126473,695.17798561)(577.04626479,695.1879856)(576.99626953,695.20799194)
\lineto(576.84626953,695.20799194)
\curveto(576.64626519,695.20798558)(576.48626535,695.1879856)(576.36626953,695.14799194)
\curveto(576.07626576,695.05798573)(575.84126599,694.91798587)(575.66126953,694.72799194)
\curveto(575.48126635,694.54798624)(575.3362665,694.32798646)(575.22626953,694.06799194)
\curveto(575.17626666,693.95798683)(575.1362667,693.83798695)(575.10626953,693.70799194)
\curveto(575.08626675,693.5879872)(575.06126677,693.45798733)(575.03126953,693.31799194)
\curveto(575.02126681,693.27798751)(575.01626682,693.23798755)(575.01626953,693.19799194)
\curveto(575.01626682,693.15798763)(575.01126682,693.11798767)(575.00126953,693.07799194)
\curveto(574.98126685,692.97798781)(574.97126686,692.83798795)(574.97126953,692.65799194)
\curveto(574.98126685,692.47798831)(574.99626684,692.33798845)(575.01626953,692.23799194)
\curveto(575.01626682,692.15798863)(575.02126681,692.10298869)(575.03126953,692.07299194)
\curveto(575.05126678,692.00298879)(575.06126677,691.93298886)(575.06126953,691.86299194)
\curveto(575.07126676,691.792989)(575.08626675,691.72298907)(575.10626953,691.65299194)
\curveto(575.18626665,691.42298937)(575.28126655,691.21298958)(575.39126953,691.02299194)
\curveto(575.50126633,690.83298996)(575.64126619,690.67299012)(575.81126953,690.54299194)
\curveto(575.85126598,690.51299028)(575.91126592,690.47799031)(575.99126953,690.43799194)
\curveto(576.10126573,690.36799042)(576.21126562,690.32299047)(576.32126953,690.30299194)
\curveto(576.44126539,690.28299051)(576.58626525,690.26299053)(576.75626953,690.24299194)
\lineto(576.84626953,690.24299194)
\curveto(576.88626495,690.24299055)(576.91626492,690.24799054)(576.93626953,690.25799194)
\lineto(577.07126953,690.25799194)
\curveto(577.14126469,690.27799051)(577.20626463,690.2929905)(577.26626953,690.30299194)
\curveto(577.3362645,690.32299047)(577.40126443,690.34299045)(577.46126953,690.36299194)
\curveto(577.76126407,690.4929903)(577.99126384,690.68299011)(578.15126953,690.93299194)
\curveto(578.19126364,690.98298981)(578.22626361,691.03798975)(578.25626953,691.09799194)
\curveto(578.28626355,691.16798962)(578.31126352,691.22798956)(578.33126953,691.27799194)
\curveto(578.37126346,691.3879894)(578.40626343,691.48298931)(578.43626953,691.56299194)
\curveto(578.46626337,691.65298914)(578.5362633,691.72298907)(578.64626953,691.77299194)
\curveto(578.7362631,691.81298898)(578.88126295,691.82798896)(579.08126953,691.81799194)
\lineto(579.57626953,691.81799194)
\lineto(579.78626953,691.81799194)
\curveto(579.86626197,691.82798896)(579.9312619,691.82298897)(579.98126953,691.80299194)
\lineto(580.10126953,691.80299194)
\lineto(580.22126953,691.77299194)
\curveto(580.26126157,691.77298902)(580.29126154,691.76298903)(580.31126953,691.74299194)
\curveto(580.36126147,691.70298909)(580.39126144,691.64298915)(580.40126953,691.56299194)
\curveto(580.42126141,691.4929893)(580.42126141,691.41798937)(580.40126953,691.33799194)
\curveto(580.31126152,691.00798978)(580.20126163,690.71299008)(580.07126953,690.45299194)
\curveto(579.66126217,689.68299111)(579.00626283,689.14799164)(578.10626953,688.84799194)
\curveto(578.00626383,688.81799197)(577.90126393,688.79799199)(577.79126953,688.78799194)
\curveto(577.68126415,688.76799202)(577.57126426,688.74299205)(577.46126953,688.71299194)
\curveto(577.40126443,688.70299209)(577.34126449,688.69799209)(577.28126953,688.69799194)
\curveto(577.22126461,688.69799209)(577.16126467,688.6929921)(577.10126953,688.68299194)
\lineto(576.93626953,688.68299194)
\curveto(576.88626495,688.66299213)(576.81126502,688.65799213)(576.71126953,688.66799194)
\curveto(576.61126522,688.66799212)(576.5362653,688.67299212)(576.48626953,688.68299194)
\curveto(576.40626543,688.70299209)(576.3312655,688.71299208)(576.26126953,688.71299194)
\curveto(576.20126563,688.70299209)(576.1362657,688.70799208)(576.06626953,688.72799194)
\lineto(575.91626953,688.75799194)
\curveto(575.86626597,688.75799203)(575.81626602,688.76299203)(575.76626953,688.77299194)
\curveto(575.65626618,688.80299199)(575.55126628,688.83299196)(575.45126953,688.86299194)
\curveto(575.35126648,688.8929919)(575.25626658,688.92799186)(575.16626953,688.96799194)
\curveto(574.69626714,689.16799162)(574.30126753,689.42299137)(573.98126953,689.73299194)
\curveto(573.66126817,690.05299074)(573.40126843,690.44799034)(573.20126953,690.91799194)
\curveto(573.15126868,691.00798978)(573.11126872,691.10298969)(573.08126953,691.20299194)
\lineto(572.99126953,691.53299194)
\curveto(572.98126885,691.57298922)(572.97626886,691.60798918)(572.97626953,691.63799194)
\curveto(572.97626886,691.67798911)(572.96626887,691.72298907)(572.94626953,691.77299194)
\curveto(572.92626891,691.84298895)(572.91626892,691.91298888)(572.91626953,691.98299194)
\curveto(572.91626892,692.06298873)(572.90626893,692.13798865)(572.88626953,692.20799194)
\lineto(572.88626953,692.46299194)
\curveto(572.86626897,692.51298828)(572.85626898,692.56798822)(572.85626953,692.62799194)
\curveto(572.85626898,692.69798809)(572.86626897,692.75798803)(572.88626953,692.80799194)
\curveto(572.89626894,692.85798793)(572.89626894,692.90298789)(572.88626953,692.94299194)
\curveto(572.87626896,692.98298781)(572.87626896,693.02298777)(572.88626953,693.06299194)
\curveto(572.90626893,693.13298766)(572.91126892,693.19798759)(572.90126953,693.25799194)
\curveto(572.90126893,693.31798747)(572.91126892,693.37798741)(572.93126953,693.43799194)
\curveto(572.98126885,693.61798717)(573.02126881,693.787987)(573.05126953,693.94799194)
\curveto(573.08126875,694.11798667)(573.12626871,694.28298651)(573.18626953,694.44299194)
\curveto(573.40626843,694.95298584)(573.68126815,695.37798541)(574.01126953,695.71799194)
\curveto(574.35126748,696.05798473)(574.78126705,696.33298446)(575.30126953,696.54299194)
\curveto(575.44126639,696.60298419)(575.58626625,696.64298415)(575.73626953,696.66299194)
\curveto(575.88626595,696.6929841)(576.04126579,696.72798406)(576.20126953,696.76799194)
\curveto(576.28126555,696.77798401)(576.35626548,696.78298401)(576.42626953,696.78299194)
\curveto(576.49626534,696.78298401)(576.57126526,696.787984)(576.65126953,696.79799194)
}
}
{
\newrgbcolor{curcolor}{0 0 0}
\pscustom[linestyle=none,fillstyle=solid,fillcolor=curcolor]
{
\newpath
\moveto(583.79455078,699.43799194)
\curveto(583.86454783,699.35798143)(583.8995478,699.23798155)(583.89955078,699.07799194)
\lineto(583.89955078,698.61299194)
\lineto(583.89955078,698.20799194)
\curveto(583.8995478,698.06798272)(583.86454783,697.97298282)(583.79455078,697.92299194)
\curveto(583.73454796,697.87298292)(583.65454804,697.84298295)(583.55455078,697.83299194)
\curveto(583.46454823,697.82298297)(583.36454833,697.81798297)(583.25455078,697.81799194)
\lineto(582.41455078,697.81799194)
\curveto(582.30454939,697.81798297)(582.20454949,697.82298297)(582.11455078,697.83299194)
\curveto(582.03454966,697.84298295)(581.96454973,697.87298292)(581.90455078,697.92299194)
\curveto(581.86454983,697.95298284)(581.83454986,698.00798278)(581.81455078,698.08799194)
\curveto(581.80454989,698.17798261)(581.7945499,698.27298252)(581.78455078,698.37299194)
\lineto(581.78455078,698.70299194)
\curveto(581.7945499,698.81298198)(581.7995499,698.90798188)(581.79955078,698.98799194)
\lineto(581.79955078,699.19799194)
\curveto(581.80954989,699.26798152)(581.82954987,699.32798146)(581.85955078,699.37799194)
\curveto(581.87954982,699.41798137)(581.90454979,699.44798134)(581.93455078,699.46799194)
\lineto(582.05455078,699.52799194)
\curveto(582.07454962,699.52798126)(582.0995496,699.52798126)(582.12955078,699.52799194)
\curveto(582.15954954,699.53798125)(582.18454951,699.54298125)(582.20455078,699.54299194)
\lineto(583.29955078,699.54299194)
\curveto(583.3995483,699.54298125)(583.4945482,699.53798125)(583.58455078,699.52799194)
\curveto(583.67454802,699.51798127)(583.74454795,699.4879813)(583.79455078,699.43799194)
\moveto(583.89955078,689.67299194)
\curveto(583.8995478,689.47299132)(583.8945478,689.30299149)(583.88455078,689.16299194)
\curveto(583.87454782,689.02299177)(583.78454791,688.92799186)(583.61455078,688.87799194)
\curveto(583.55454814,688.85799193)(583.48954821,688.84799194)(583.41955078,688.84799194)
\curveto(583.34954835,688.85799193)(583.27454842,688.86299193)(583.19455078,688.86299194)
\lineto(582.35455078,688.86299194)
\curveto(582.26454943,688.86299193)(582.17454952,688.86799192)(582.08455078,688.87799194)
\curveto(582.00454969,688.8879919)(581.94454975,688.91799187)(581.90455078,688.96799194)
\curveto(581.84454985,689.03799175)(581.80954989,689.12299167)(581.79955078,689.22299194)
\lineto(581.79955078,689.56799194)
\lineto(581.79955078,695.89799194)
\lineto(581.79955078,696.19799194)
\curveto(581.7995499,696.29798449)(581.81954988,696.37798441)(581.85955078,696.43799194)
\curveto(581.91954978,696.50798428)(582.00454969,696.55298424)(582.11455078,696.57299194)
\curveto(582.13454956,696.58298421)(582.15954954,696.58298421)(582.18955078,696.57299194)
\curveto(582.22954947,696.57298422)(582.25954944,696.57798421)(582.27955078,696.58799194)
\lineto(583.02955078,696.58799194)
\lineto(583.22455078,696.58799194)
\curveto(583.30454839,696.59798419)(583.36954833,696.59798419)(583.41955078,696.58799194)
\lineto(583.53955078,696.58799194)
\curveto(583.5995481,696.56798422)(583.65454804,696.55298424)(583.70455078,696.54299194)
\curveto(583.75454794,696.53298426)(583.7945479,696.50298429)(583.82455078,696.45299194)
\curveto(583.86454783,696.40298439)(583.88454781,696.33298446)(583.88455078,696.24299194)
\curveto(583.8945478,696.15298464)(583.8995478,696.05798473)(583.89955078,695.95799194)
\lineto(583.89955078,689.67299194)
}
}
{
\newrgbcolor{curcolor}{0 0 0}
\pscustom[linestyle=none,fillstyle=solid,fillcolor=curcolor]
{
\newpath
\moveto(593.33173828,693.03299194)
\curveto(593.31172975,693.08298771)(593.30672976,693.13798765)(593.31673828,693.19799194)
\curveto(593.32672974,693.25798753)(593.32172974,693.31298748)(593.30173828,693.36299194)
\curveto(593.29172977,693.40298739)(593.28672978,693.44298735)(593.28673828,693.48299194)
\curveto(593.28672978,693.52298727)(593.28172978,693.56298723)(593.27173828,693.60299194)
\lineto(593.21173828,693.87299194)
\curveto(593.19172987,693.96298683)(593.1667299,694.04798674)(593.13673828,694.12799194)
\curveto(593.08672998,694.26798652)(593.04173002,694.39798639)(593.00173828,694.51799194)
\curveto(592.9617301,694.64798614)(592.90673016,694.76798602)(592.83673828,694.87799194)
\curveto(592.7667303,694.9879858)(592.69673037,695.0929857)(592.62673828,695.19299194)
\curveto(592.5667305,695.2929855)(592.49673057,695.3929854)(592.41673828,695.49299194)
\curveto(592.33673073,695.60298519)(592.23673083,695.70298509)(592.11673828,695.79299194)
\curveto(592.00673106,695.8929849)(591.89673117,695.98298481)(591.78673828,696.06299194)
\curveto(591.45673161,696.2929845)(591.07673199,696.47298432)(590.64673828,696.60299194)
\curveto(590.22673284,696.73298406)(589.72673334,696.792984)(589.14673828,696.78299194)
\curveto(589.07673399,696.77298402)(589.00673406,696.76798402)(588.93673828,696.76799194)
\curveto(588.8667342,696.76798402)(588.79173427,696.76298403)(588.71173828,696.75299194)
\curveto(588.5617345,696.71298408)(588.41673465,696.68298411)(588.27673828,696.66299194)
\curveto(588.13673493,696.64298415)(588.00173506,696.60798418)(587.87173828,696.55799194)
\curveto(587.7617353,696.50798428)(587.65173541,696.46298433)(587.54173828,696.42299194)
\curveto(587.43173563,696.38298441)(587.32673574,696.33798445)(587.22673828,696.28799194)
\curveto(586.8667362,696.05798473)(586.5617365,695.80298499)(586.31173828,695.52299194)
\curveto(586.061737,695.25298554)(585.84673722,694.91298588)(585.66673828,694.50299194)
\curveto(585.61673745,694.38298641)(585.57673749,694.25798653)(585.54673828,694.12799194)
\curveto(585.51673755,694.00798678)(585.48173758,693.88298691)(585.44173828,693.75299194)
\curveto(585.42173764,693.70298709)(585.41173765,693.65298714)(585.41173828,693.60299194)
\curveto(585.41173765,693.56298723)(585.40673766,693.51798727)(585.39673828,693.46799194)
\curveto(585.37673769,693.41798737)(585.3667377,693.36298743)(585.36673828,693.30299194)
\curveto(585.37673769,693.25298754)(585.37673769,693.20298759)(585.36673828,693.15299194)
\lineto(585.36673828,693.04799194)
\curveto(585.34673772,692.9879878)(585.33173773,692.90298789)(585.32173828,692.79299194)
\curveto(585.32173774,692.68298811)(585.33173773,692.59798819)(585.35173828,692.53799194)
\lineto(585.35173828,692.40299194)
\curveto(585.35173771,692.36298843)(585.35673771,692.31798847)(585.36673828,692.26799194)
\curveto(585.38673768,692.1879886)(585.39673767,692.10298869)(585.39673828,692.01299194)
\curveto(585.39673767,691.93298886)(585.40673766,691.85298894)(585.42673828,691.77299194)
\curveto(585.44673762,691.72298907)(585.45673761,691.67798911)(585.45673828,691.63799194)
\curveto(585.45673761,691.59798919)(585.4667376,691.55298924)(585.48673828,691.50299194)
\curveto(585.51673755,691.3929894)(585.54173752,691.2879895)(585.56173828,691.18799194)
\curveto(585.59173747,691.0879897)(585.63173743,690.9929898)(585.68173828,690.90299194)
\curveto(585.85173721,690.51299028)(586.061737,690.17799061)(586.31173828,689.89799194)
\curveto(586.5617365,689.61799117)(586.8617362,689.37299142)(587.21173828,689.16299194)
\curveto(587.32173574,689.10299169)(587.42673564,689.05299174)(587.52673828,689.01299194)
\curveto(587.63673543,688.97299182)(587.75173531,688.93299186)(587.87173828,688.89299194)
\curveto(587.9617351,688.85299194)(588.05673501,688.82299197)(588.15673828,688.80299194)
\curveto(588.25673481,688.78299201)(588.35673471,688.75799203)(588.45673828,688.72799194)
\curveto(588.50673456,688.71799207)(588.54673452,688.71299208)(588.57673828,688.71299194)
\curveto(588.61673445,688.71299208)(588.65673441,688.70799208)(588.69673828,688.69799194)
\curveto(588.74673432,688.67799211)(588.79673427,688.67299212)(588.84673828,688.68299194)
\curveto(588.90673416,688.68299211)(588.9617341,688.67799211)(589.01173828,688.66799194)
\lineto(589.16173828,688.66799194)
\curveto(589.22173384,688.64799214)(589.30673376,688.64299215)(589.41673828,688.65299194)
\curveto(589.52673354,688.65299214)(589.60673346,688.65799213)(589.65673828,688.66799194)
\curveto(589.68673338,688.66799212)(589.71673335,688.67299212)(589.74673828,688.68299194)
\lineto(589.85173828,688.68299194)
\curveto(589.90173316,688.6929921)(589.95673311,688.69799209)(590.01673828,688.69799194)
\curveto(590.07673299,688.69799209)(590.13173293,688.70799208)(590.18173828,688.72799194)
\curveto(590.31173275,688.75799203)(590.43673263,688.787992)(590.55673828,688.81799194)
\curveto(590.68673238,688.83799195)(590.81173225,688.87299192)(590.93173828,688.92299194)
\curveto(591.41173165,689.12299167)(591.82173124,689.37299142)(592.16173828,689.67299194)
\curveto(592.50173056,689.97299082)(592.77673029,690.36299043)(592.98673828,690.84299194)
\curveto(593.03673003,690.94298985)(593.07672999,691.04798974)(593.10673828,691.15799194)
\curveto(593.13672993,691.27798951)(593.17172989,691.3929894)(593.21173828,691.50299194)
\curveto(593.22172984,691.57298922)(593.23172983,691.63798915)(593.24173828,691.69799194)
\curveto(593.25172981,691.75798903)(593.2667298,691.82298897)(593.28673828,691.89299194)
\curveto(593.30672976,691.97298882)(593.31172975,692.05298874)(593.30173828,692.13299194)
\curveto(593.30172976,692.21298858)(593.31172975,692.2929885)(593.33173828,692.37299194)
\lineto(593.33173828,692.52299194)
\curveto(593.35172971,692.58298821)(593.3617297,692.66798812)(593.36173828,692.77799194)
\curveto(593.3617297,692.8879879)(593.35172971,692.97298782)(593.33173828,693.03299194)
\moveto(591.23173828,692.49299194)
\curveto(591.22173184,692.44298835)(591.21673185,692.3929884)(591.21673828,692.34299194)
\lineto(591.21673828,692.20799194)
\curveto(591.20673186,692.16798862)(591.20173186,692.12798866)(591.20173828,692.08799194)
\curveto(591.20173186,692.05798873)(591.19673187,692.02298877)(591.18673828,691.98299194)
\curveto(591.15673191,691.87298892)(591.13173193,691.76798902)(591.11173828,691.66799194)
\curveto(591.09173197,691.56798922)(591.061732,691.46798932)(591.02173828,691.36799194)
\curveto(590.91173215,691.11798967)(590.77673229,690.90798988)(590.61673828,690.73799194)
\curveto(590.45673261,690.56799022)(590.24673282,690.43299036)(589.98673828,690.33299194)
\curveto(589.91673315,690.30299049)(589.84173322,690.28299051)(589.76173828,690.27299194)
\curveto(589.68173338,690.26299053)(589.60173346,690.24799054)(589.52173828,690.22799194)
\lineto(589.40173828,690.22799194)
\curveto(589.3617337,690.21799057)(589.31673375,690.21299058)(589.26673828,690.21299194)
\lineto(589.14673828,690.24299194)
\curveto(589.10673396,690.25299054)(589.07173399,690.25299054)(589.04173828,690.24299194)
\curveto(589.01173405,690.24299055)(588.97673409,690.24799054)(588.93673828,690.25799194)
\curveto(588.84673422,690.27799051)(588.75673431,690.30299049)(588.66673828,690.33299194)
\curveto(588.58673448,690.36299043)(588.51173455,690.40299039)(588.44173828,690.45299194)
\curveto(588.19173487,690.60299019)(588.00673506,690.76799002)(587.88673828,690.94799194)
\curveto(587.77673529,691.13798965)(587.67173539,691.38298941)(587.57173828,691.68299194)
\curveto(587.55173551,691.76298903)(587.53673553,691.83798895)(587.52673828,691.90799194)
\curveto(587.51673555,691.9879888)(587.50173556,692.06798872)(587.48173828,692.14799194)
\lineto(587.48173828,692.28299194)
\curveto(587.4617356,692.35298844)(587.44673562,692.45798833)(587.43673828,692.59799194)
\curveto(587.43673563,692.73798805)(587.44673562,692.84298795)(587.46673828,692.91299194)
\lineto(587.46673828,693.06299194)
\curveto(587.4667356,693.11298768)(587.47173559,693.16298763)(587.48173828,693.21299194)
\curveto(587.50173556,693.32298747)(587.51673555,693.43298736)(587.52673828,693.54299194)
\curveto(587.54673552,693.65298714)(587.57173549,693.75798703)(587.60173828,693.85799194)
\curveto(587.69173537,694.12798666)(587.81173525,694.36298643)(587.96173828,694.56299194)
\curveto(588.12173494,694.77298602)(588.32673474,694.93298586)(588.57673828,695.04299194)
\curveto(588.62673444,695.07298572)(588.68173438,695.0929857)(588.74173828,695.10299194)
\lineto(588.95173828,695.16299194)
\curveto(588.98173408,695.17298562)(589.01673405,695.17298562)(589.05673828,695.16299194)
\curveto(589.09673397,695.16298563)(589.13173393,695.17298562)(589.16173828,695.19299194)
\lineto(589.43173828,695.19299194)
\curveto(589.52173354,695.20298559)(589.60673346,695.19798559)(589.68673828,695.17799194)
\curveto(589.75673331,695.15798563)(589.82173324,695.13798565)(589.88173828,695.11799194)
\curveto(589.94173312,695.10798568)(590.00173306,695.0929857)(590.06173828,695.07299194)
\curveto(590.31173275,694.96298583)(590.51173255,694.81298598)(590.66173828,694.62299194)
\curveto(590.81173225,694.44298635)(590.94173212,694.22298657)(591.05173828,693.96299194)
\curveto(591.08173198,693.88298691)(591.10173196,693.79798699)(591.11173828,693.70799194)
\lineto(591.17173828,693.46799194)
\curveto(591.18173188,693.44798734)(591.18673188,693.41798737)(591.18673828,693.37799194)
\curveto(591.19673187,693.32798746)(591.20173186,693.27298752)(591.20173828,693.21299194)
\curveto(591.20173186,693.15298764)(591.21173185,693.09798769)(591.23173828,693.04799194)
\lineto(591.23173828,692.92799194)
\curveto(591.24173182,692.87798791)(591.24673182,692.80298799)(591.24673828,692.70299194)
\curveto(591.24673182,692.61298818)(591.24173182,692.54298825)(591.23173828,692.49299194)
\moveto(590.00173828,699.66299194)
\lineto(591.06673828,699.66299194)
\curveto(591.14673192,699.66298113)(591.24173182,699.66298113)(591.35173828,699.66299194)
\curveto(591.4617316,699.66298113)(591.54173152,699.64798114)(591.59173828,699.61799194)
\curveto(591.61173145,699.60798118)(591.62173144,699.5929812)(591.62173828,699.57299194)
\curveto(591.63173143,699.56298123)(591.64673142,699.55298124)(591.66673828,699.54299194)
\curveto(591.67673139,699.42298137)(591.62673144,699.31798147)(591.51673828,699.22799194)
\curveto(591.41673165,699.13798165)(591.33173173,699.05798173)(591.26173828,698.98799194)
\curveto(591.18173188,698.91798187)(591.10173196,698.84298195)(591.02173828,698.76299194)
\curveto(590.95173211,698.6929821)(590.87673219,698.62798216)(590.79673828,698.56799194)
\curveto(590.75673231,698.53798225)(590.72173234,698.50298229)(590.69173828,698.46299194)
\curveto(590.67173239,698.43298236)(590.64173242,698.40798238)(590.60173828,698.38799194)
\curveto(590.58173248,698.35798243)(590.55673251,698.33298246)(590.52673828,698.31299194)
\lineto(590.37673828,698.16299194)
\lineto(590.22673828,698.04299194)
\lineto(590.18173828,697.99799194)
\curveto(590.18173288,697.9879828)(590.17173289,697.97298282)(590.15173828,697.95299194)
\curveto(590.07173299,697.8929829)(589.99173307,697.82798296)(589.91173828,697.75799194)
\curveto(589.84173322,697.6879831)(589.75173331,697.63298316)(589.64173828,697.59299194)
\curveto(589.60173346,697.58298321)(589.5617335,697.57798321)(589.52173828,697.57799194)
\curveto(589.49173357,697.57798321)(589.45173361,697.57298322)(589.40173828,697.56299194)
\curveto(589.37173369,697.55298324)(589.33173373,697.54798324)(589.28173828,697.54799194)
\curveto(589.23173383,697.55798323)(589.18673388,697.56298323)(589.14673828,697.56299194)
\lineto(588.80173828,697.56299194)
\curveto(588.68173438,697.56298323)(588.59173447,697.5879832)(588.53173828,697.63799194)
\curveto(588.47173459,697.67798311)(588.45673461,697.74798304)(588.48673828,697.84799194)
\curveto(588.50673456,697.92798286)(588.54173452,697.99798279)(588.59173828,698.05799194)
\curveto(588.64173442,698.12798266)(588.68673438,698.19798259)(588.72673828,698.26799194)
\curveto(588.82673424,698.40798238)(588.92173414,698.54298225)(589.01173828,698.67299194)
\curveto(589.10173396,698.80298199)(589.19173387,698.93798185)(589.28173828,699.07799194)
\curveto(589.33173373,699.15798163)(589.38173368,699.24298155)(589.43173828,699.33299194)
\curveto(589.49173357,699.42298137)(589.55673351,699.4929813)(589.62673828,699.54299194)
\curveto(589.6667334,699.57298122)(589.73673333,699.60798118)(589.83673828,699.64799194)
\curveto(589.85673321,699.65798113)(589.88173318,699.65798113)(589.91173828,699.64799194)
\curveto(589.95173311,699.64798114)(589.98173308,699.65298114)(590.00173828,699.66299194)
}
}
{
\newrgbcolor{curcolor}{0 0 0}
\pscustom[linestyle=none,fillstyle=solid,fillcolor=curcolor]
{
\newpath
\moveto(599.15666016,696.78299194)
\curveto(599.75665435,696.80298399)(600.25665385,696.71798407)(600.65666016,696.52799194)
\curveto(601.05665305,696.33798445)(601.37165274,696.05798473)(601.60166016,695.68799194)
\curveto(601.67165244,695.57798521)(601.72665238,695.45798533)(601.76666016,695.32799194)
\curveto(601.8066523,695.20798558)(601.84665226,695.08298571)(601.88666016,694.95299194)
\curveto(601.9066522,694.87298592)(601.91665219,694.79798599)(601.91666016,694.72799194)
\curveto(601.92665218,694.65798613)(601.94165217,694.5879862)(601.96166016,694.51799194)
\curveto(601.96165215,694.45798633)(601.96665214,694.41798637)(601.97666016,694.39799194)
\curveto(601.99665211,694.25798653)(602.0066521,694.11298668)(602.00666016,693.96299194)
\lineto(602.00666016,693.52799194)
\lineto(602.00666016,692.19299194)
\lineto(602.00666016,689.76299194)
\curveto(602.0066521,689.57299122)(602.00165211,689.3879914)(601.99166016,689.20799194)
\curveto(601.99165212,689.03799175)(601.92165219,688.92799186)(601.78166016,688.87799194)
\curveto(601.72165239,688.85799193)(601.65165246,688.84799194)(601.57166016,688.84799194)
\lineto(601.33166016,688.84799194)
\lineto(600.52166016,688.84799194)
\curveto(600.40165371,688.84799194)(600.29165382,688.85299194)(600.19166016,688.86299194)
\curveto(600.10165401,688.88299191)(600.03165408,688.92799186)(599.98166016,688.99799194)
\curveto(599.94165417,689.05799173)(599.91665419,689.13299166)(599.90666016,689.22299194)
\lineto(599.90666016,689.53799194)
\lineto(599.90666016,690.58799194)
\lineto(599.90666016,692.82299194)
\curveto(599.9066542,693.1929876)(599.89165422,693.53298726)(599.86166016,693.84299194)
\curveto(599.83165428,694.16298663)(599.74165437,694.43298636)(599.59166016,694.65299194)
\curveto(599.45165466,694.85298594)(599.24665486,694.9929858)(598.97666016,695.07299194)
\curveto(598.92665518,695.0929857)(598.87165524,695.10298569)(598.81166016,695.10299194)
\curveto(598.76165535,695.10298569)(598.7066554,695.11298568)(598.64666016,695.13299194)
\curveto(598.59665551,695.14298565)(598.53165558,695.14298565)(598.45166016,695.13299194)
\curveto(598.38165573,695.13298566)(598.32665578,695.12798566)(598.28666016,695.11799194)
\curveto(598.24665586,695.10798568)(598.2116559,695.10298569)(598.18166016,695.10299194)
\curveto(598.15165596,695.10298569)(598.12165599,695.09798569)(598.09166016,695.08799194)
\curveto(597.86165625,695.02798576)(597.67665643,694.94798584)(597.53666016,694.84799194)
\curveto(597.21665689,694.61798617)(597.02665708,694.28298651)(596.96666016,693.84299194)
\curveto(596.9066572,693.40298739)(596.87665723,692.90798788)(596.87666016,692.35799194)
\lineto(596.87666016,690.48299194)
\lineto(596.87666016,689.56799194)
\lineto(596.87666016,689.29799194)
\curveto(596.87665723,689.20799158)(596.86165725,689.13299166)(596.83166016,689.07299194)
\curveto(596.78165733,688.96299183)(596.70165741,688.89799189)(596.59166016,688.87799194)
\curveto(596.48165763,688.85799193)(596.34665776,688.84799194)(596.18666016,688.84799194)
\lineto(595.43666016,688.84799194)
\curveto(595.32665878,688.84799194)(595.21665889,688.85299194)(595.10666016,688.86299194)
\curveto(594.99665911,688.87299192)(594.91665919,688.90799188)(594.86666016,688.96799194)
\curveto(594.79665931,689.05799173)(594.76165935,689.1879916)(594.76166016,689.35799194)
\curveto(594.77165934,689.52799126)(594.77665933,689.6879911)(594.77666016,689.83799194)
\lineto(594.77666016,691.87799194)
\lineto(594.77666016,695.17799194)
\lineto(594.77666016,695.94299194)
\lineto(594.77666016,696.24299194)
\curveto(594.78665932,696.33298446)(594.81665929,696.40798438)(594.86666016,696.46799194)
\curveto(594.88665922,696.49798429)(594.91665919,696.51798427)(594.95666016,696.52799194)
\curveto(595.0066591,696.54798424)(595.05665905,696.56298423)(595.10666016,696.57299194)
\lineto(595.18166016,696.57299194)
\curveto(595.23165888,696.58298421)(595.28165883,696.5879842)(595.33166016,696.58799194)
\lineto(595.49666016,696.58799194)
\lineto(596.12666016,696.58799194)
\curveto(596.2066579,696.5879842)(596.28165783,696.58298421)(596.35166016,696.57299194)
\curveto(596.43165768,696.57298422)(596.50165761,696.56298423)(596.56166016,696.54299194)
\curveto(596.63165748,696.51298428)(596.67665743,696.46798432)(596.69666016,696.40799194)
\curveto(596.72665738,696.34798444)(596.75165736,696.27798451)(596.77166016,696.19799194)
\curveto(596.78165733,696.15798463)(596.78165733,696.12298467)(596.77166016,696.09299194)
\curveto(596.77165734,696.06298473)(596.78165733,696.03298476)(596.80166016,696.00299194)
\curveto(596.82165729,695.95298484)(596.83665727,695.92298487)(596.84666016,695.91299194)
\curveto(596.86665724,695.90298489)(596.89165722,695.8879849)(596.92166016,695.86799194)
\curveto(597.03165708,695.85798493)(597.12165699,695.8929849)(597.19166016,695.97299194)
\curveto(597.26165685,696.06298473)(597.33665677,696.13298466)(597.41666016,696.18299194)
\curveto(597.68665642,696.38298441)(597.98665612,696.54298425)(598.31666016,696.66299194)
\curveto(598.4066557,696.6929841)(598.49665561,696.71298408)(598.58666016,696.72299194)
\curveto(598.68665542,696.73298406)(598.79165532,696.74798404)(598.90166016,696.76799194)
\curveto(598.93165518,696.77798401)(598.97665513,696.77798401)(599.03666016,696.76799194)
\curveto(599.09665501,696.76798402)(599.13665497,696.77298402)(599.15666016,696.78299194)
}
}
{
\newrgbcolor{curcolor}{0 0 0}
\pscustom[linestyle=none,fillstyle=solid,fillcolor=curcolor]
{
\newpath
\moveto(13.54683094,339.66083496)
\lineto(14.83683094,339.66083496)
\curveto(14.94682812,339.66082428)(15.05182801,339.65582429)(15.15183094,339.64583496)
\curveto(15.25182781,339.6458243)(15.32682774,339.61082433)(15.37683094,339.54083496)
\curveto(15.42682764,339.47082447)(15.45182761,339.38082456)(15.45183094,339.27083496)
\curveto(15.4618276,339.16082478)(15.4668276,339.0408249)(15.46683094,338.91083496)
\lineto(15.46683094,337.60583496)
\lineto(15.46683094,332.40083496)
\lineto(15.46683094,329.94083496)
\lineto(15.46683094,329.50583496)
\curveto(15.47682759,329.3458346)(15.45682761,329.22583472)(15.40683094,329.14583496)
\curveto(15.3668277,329.07583487)(15.27682779,329.02083492)(15.13683094,328.98083496)
\curveto(15.066828,328.96083498)(14.99182807,328.95583499)(14.91183094,328.96583496)
\curveto(14.83182823,328.97583497)(14.75182831,328.98083496)(14.67183094,328.98083496)
\lineto(13.78683094,328.98083496)
\curveto(13.67682939,328.98083496)(13.57182949,328.98583496)(13.47183094,328.99583496)
\curveto(13.38182968,329.00583494)(13.30682976,329.03583491)(13.24683094,329.08583496)
\curveto(13.19682987,329.13583481)(13.1668299,329.21083473)(13.15683094,329.31083496)
\curveto(13.14682992,329.41083453)(13.14182992,329.51583443)(13.14183094,329.62583496)
\lineto(13.14183094,330.93083496)
\lineto(13.14183094,336.40583496)
\lineto(13.14183094,338.59583496)
\curveto(13.14182992,338.73582521)(13.13682993,338.90082504)(13.12683094,339.09083496)
\curveto(13.12682994,339.28082466)(13.15182991,339.41582453)(13.20183094,339.49583496)
\curveto(13.24182982,339.55582439)(13.30682976,339.60582434)(13.39683094,339.64583496)
\curveto(13.42682964,339.6458243)(13.45182961,339.6458243)(13.47183094,339.64583496)
\curveto(13.50182956,339.65582429)(13.52682954,339.66082428)(13.54683094,339.66083496)
}
}
{
\newrgbcolor{curcolor}{0 0 0}
\pscustom[linestyle=none,fillstyle=solid,fillcolor=curcolor]
{
\newpath
\moveto(21.75065907,336.91583496)
\curveto(22.35065326,336.93582701)(22.85065276,336.85082709)(23.25065907,336.66083496)
\curveto(23.65065196,336.47082747)(23.96565165,336.19082775)(24.19565907,335.82083496)
\curveto(24.26565135,335.71082823)(24.32065129,335.59082835)(24.36065907,335.46083496)
\curveto(24.40065121,335.3408286)(24.44065117,335.21582873)(24.48065907,335.08583496)
\curveto(24.50065111,335.00582894)(24.5106511,334.93082901)(24.51065907,334.86083496)
\curveto(24.52065109,334.79082915)(24.53565108,334.72082922)(24.55565907,334.65083496)
\curveto(24.55565106,334.59082935)(24.56065105,334.55082939)(24.57065907,334.53083496)
\curveto(24.59065102,334.39082955)(24.60065101,334.2458297)(24.60065907,334.09583496)
\lineto(24.60065907,333.66083496)
\lineto(24.60065907,332.32583496)
\lineto(24.60065907,329.89583496)
\curveto(24.60065101,329.70583424)(24.59565102,329.52083442)(24.58565907,329.34083496)
\curveto(24.58565103,329.17083477)(24.5156511,329.06083488)(24.37565907,329.01083496)
\curveto(24.3156513,328.99083495)(24.24565137,328.98083496)(24.16565907,328.98083496)
\lineto(23.92565907,328.98083496)
\lineto(23.11565907,328.98083496)
\curveto(22.99565262,328.98083496)(22.88565273,328.98583496)(22.78565907,328.99583496)
\curveto(22.69565292,329.01583493)(22.62565299,329.06083488)(22.57565907,329.13083496)
\curveto(22.53565308,329.19083475)(22.5106531,329.26583468)(22.50065907,329.35583496)
\lineto(22.50065907,329.67083496)
\lineto(22.50065907,330.72083496)
\lineto(22.50065907,332.95583496)
\curveto(22.50065311,333.32583062)(22.48565313,333.66583028)(22.45565907,333.97583496)
\curveto(22.42565319,334.29582965)(22.33565328,334.56582938)(22.18565907,334.78583496)
\curveto(22.04565357,334.98582896)(21.84065377,335.12582882)(21.57065907,335.20583496)
\curveto(21.52065409,335.22582872)(21.46565415,335.23582871)(21.40565907,335.23583496)
\curveto(21.35565426,335.23582871)(21.30065431,335.2458287)(21.24065907,335.26583496)
\curveto(21.19065442,335.27582867)(21.12565449,335.27582867)(21.04565907,335.26583496)
\curveto(20.97565464,335.26582868)(20.92065469,335.26082868)(20.88065907,335.25083496)
\curveto(20.84065477,335.2408287)(20.80565481,335.23582871)(20.77565907,335.23583496)
\curveto(20.74565487,335.23582871)(20.7156549,335.23082871)(20.68565907,335.22083496)
\curveto(20.45565516,335.16082878)(20.27065534,335.08082886)(20.13065907,334.98083496)
\curveto(19.8106558,334.75082919)(19.62065599,334.41582953)(19.56065907,333.97583496)
\curveto(19.50065611,333.53583041)(19.47065614,333.0408309)(19.47065907,332.49083496)
\lineto(19.47065907,330.61583496)
\lineto(19.47065907,329.70083496)
\lineto(19.47065907,329.43083496)
\curveto(19.47065614,329.3408346)(19.45565616,329.26583468)(19.42565907,329.20583496)
\curveto(19.37565624,329.09583485)(19.29565632,329.03083491)(19.18565907,329.01083496)
\curveto(19.07565654,328.99083495)(18.94065667,328.98083496)(18.78065907,328.98083496)
\lineto(18.03065907,328.98083496)
\curveto(17.92065769,328.98083496)(17.8106578,328.98583496)(17.70065907,328.99583496)
\curveto(17.59065802,329.00583494)(17.5106581,329.0408349)(17.46065907,329.10083496)
\curveto(17.39065822,329.19083475)(17.35565826,329.32083462)(17.35565907,329.49083496)
\curveto(17.36565825,329.66083428)(17.37065824,329.82083412)(17.37065907,329.97083496)
\lineto(17.37065907,332.01083496)
\lineto(17.37065907,335.31083496)
\lineto(17.37065907,336.07583496)
\lineto(17.37065907,336.37583496)
\curveto(17.38065823,336.46582748)(17.4106582,336.5408274)(17.46065907,336.60083496)
\curveto(17.48065813,336.63082731)(17.5106581,336.65082729)(17.55065907,336.66083496)
\curveto(17.60065801,336.68082726)(17.65065796,336.69582725)(17.70065907,336.70583496)
\lineto(17.77565907,336.70583496)
\curveto(17.82565779,336.71582723)(17.87565774,336.72082722)(17.92565907,336.72083496)
\lineto(18.09065907,336.72083496)
\lineto(18.72065907,336.72083496)
\curveto(18.80065681,336.72082722)(18.87565674,336.71582723)(18.94565907,336.70583496)
\curveto(19.02565659,336.70582724)(19.09565652,336.69582725)(19.15565907,336.67583496)
\curveto(19.22565639,336.6458273)(19.27065634,336.60082734)(19.29065907,336.54083496)
\curveto(19.32065629,336.48082746)(19.34565627,336.41082753)(19.36565907,336.33083496)
\curveto(19.37565624,336.29082765)(19.37565624,336.25582769)(19.36565907,336.22583496)
\curveto(19.36565625,336.19582775)(19.37565624,336.16582778)(19.39565907,336.13583496)
\curveto(19.4156562,336.08582786)(19.43065618,336.05582789)(19.44065907,336.04583496)
\curveto(19.46065615,336.03582791)(19.48565613,336.02082792)(19.51565907,336.00083496)
\curveto(19.62565599,335.99082795)(19.7156559,336.02582792)(19.78565907,336.10583496)
\curveto(19.85565576,336.19582775)(19.93065568,336.26582768)(20.01065907,336.31583496)
\curveto(20.28065533,336.51582743)(20.58065503,336.67582727)(20.91065907,336.79583496)
\curveto(21.00065461,336.82582712)(21.09065452,336.8458271)(21.18065907,336.85583496)
\curveto(21.28065433,336.86582708)(21.38565423,336.88082706)(21.49565907,336.90083496)
\curveto(21.52565409,336.91082703)(21.57065404,336.91082703)(21.63065907,336.90083496)
\curveto(21.69065392,336.90082704)(21.73065388,336.90582704)(21.75065907,336.91583496)
}
}
{
\newrgbcolor{curcolor}{0 0 0}
\pscustom[linestyle=none,fillstyle=solid,fillcolor=curcolor]
{
\newpath
\moveto(33.82190907,329.83583496)
\lineto(33.82190907,329.41583496)
\curveto(33.8219007,329.28583466)(33.79190073,329.18083476)(33.73190907,329.10083496)
\curveto(33.68190084,329.05083489)(33.6169009,329.01583493)(33.53690907,328.99583496)
\curveto(33.45690106,328.98583496)(33.36690115,328.98083496)(33.26690907,328.98083496)
\lineto(32.44190907,328.98083496)
\lineto(32.15690907,328.98083496)
\curveto(32.07690244,328.99083495)(32.01190251,329.01583493)(31.96190907,329.05583496)
\curveto(31.89190263,329.10583484)(31.85190267,329.17083477)(31.84190907,329.25083496)
\curveto(31.83190269,329.33083461)(31.81190271,329.41083453)(31.78190907,329.49083496)
\curveto(31.76190276,329.51083443)(31.74190278,329.52583442)(31.72190907,329.53583496)
\curveto(31.71190281,329.55583439)(31.69690282,329.57583437)(31.67690907,329.59583496)
\curveto(31.56690295,329.59583435)(31.48690303,329.57083437)(31.43690907,329.52083496)
\lineto(31.28690907,329.37083496)
\curveto(31.2169033,329.32083462)(31.15190337,329.27583467)(31.09190907,329.23583496)
\curveto(31.03190349,329.20583474)(30.96690355,329.16583478)(30.89690907,329.11583496)
\curveto(30.85690366,329.09583485)(30.81190371,329.07583487)(30.76190907,329.05583496)
\curveto(30.7219038,329.03583491)(30.67690384,329.01583493)(30.62690907,328.99583496)
\curveto(30.48690403,328.945835)(30.33690418,328.90083504)(30.17690907,328.86083496)
\curveto(30.12690439,328.8408351)(30.08190444,328.83083511)(30.04190907,328.83083496)
\curveto(30.00190452,328.83083511)(29.96190456,328.82583512)(29.92190907,328.81583496)
\lineto(29.78690907,328.81583496)
\curveto(29.75690476,328.80583514)(29.7169048,328.80083514)(29.66690907,328.80083496)
\lineto(29.53190907,328.80083496)
\curveto(29.47190505,328.78083516)(29.38190514,328.77583517)(29.26190907,328.78583496)
\curveto(29.14190538,328.78583516)(29.05690546,328.79583515)(29.00690907,328.81583496)
\curveto(28.93690558,328.83583511)(28.87190565,328.8458351)(28.81190907,328.84583496)
\curveto(28.76190576,328.83583511)(28.70690581,328.8408351)(28.64690907,328.86083496)
\lineto(28.28690907,328.98083496)
\curveto(28.17690634,329.01083493)(28.06690645,329.05083489)(27.95690907,329.10083496)
\curveto(27.60690691,329.25083469)(27.29190723,329.48083446)(27.01190907,329.79083496)
\curveto(26.74190778,330.11083383)(26.52690799,330.4458335)(26.36690907,330.79583496)
\curveto(26.3169082,330.90583304)(26.27690824,331.01083293)(26.24690907,331.11083496)
\curveto(26.2169083,331.22083272)(26.18190834,331.33083261)(26.14190907,331.44083496)
\curveto(26.13190839,331.48083246)(26.12690839,331.51583243)(26.12690907,331.54583496)
\curveto(26.12690839,331.58583236)(26.1169084,331.63083231)(26.09690907,331.68083496)
\curveto(26.07690844,331.76083218)(26.05690846,331.8458321)(26.03690907,331.93583496)
\curveto(26.02690849,332.03583191)(26.01190851,332.13583181)(25.99190907,332.23583496)
\curveto(25.98190854,332.26583168)(25.97690854,332.30083164)(25.97690907,332.34083496)
\curveto(25.98690853,332.38083156)(25.98690853,332.41583153)(25.97690907,332.44583496)
\lineto(25.97690907,332.58083496)
\curveto(25.97690854,332.63083131)(25.97190855,332.68083126)(25.96190907,332.73083496)
\curveto(25.95190857,332.78083116)(25.94690857,332.83583111)(25.94690907,332.89583496)
\curveto(25.94690857,332.96583098)(25.95190857,333.02083092)(25.96190907,333.06083496)
\curveto(25.97190855,333.11083083)(25.97690854,333.15583079)(25.97690907,333.19583496)
\lineto(25.97690907,333.34583496)
\curveto(25.98690853,333.39583055)(25.98690853,333.4408305)(25.97690907,333.48083496)
\curveto(25.97690854,333.53083041)(25.98690853,333.58083036)(26.00690907,333.63083496)
\curveto(26.02690849,333.7408302)(26.04190848,333.8458301)(26.05190907,333.94583496)
\curveto(26.07190845,334.0458299)(26.09690842,334.1458298)(26.12690907,334.24583496)
\curveto(26.16690835,334.36582958)(26.20190832,334.48082946)(26.23190907,334.59083496)
\curveto(26.26190826,334.70082924)(26.30190822,334.81082913)(26.35190907,334.92083496)
\curveto(26.49190803,335.22082872)(26.66690785,335.50582844)(26.87690907,335.77583496)
\curveto(26.89690762,335.80582814)(26.9219076,335.83082811)(26.95190907,335.85083496)
\curveto(26.99190753,335.88082806)(27.0219075,335.91082803)(27.04190907,335.94083496)
\curveto(27.08190744,335.99082795)(27.1219074,336.03582791)(27.16190907,336.07583496)
\curveto(27.20190732,336.11582783)(27.24690727,336.15582779)(27.29690907,336.19583496)
\curveto(27.33690718,336.21582773)(27.37190715,336.2408277)(27.40190907,336.27083496)
\curveto(27.43190709,336.31082763)(27.46690705,336.3408276)(27.50690907,336.36083496)
\curveto(27.75690676,336.53082741)(28.04690647,336.67082727)(28.37690907,336.78083496)
\curveto(28.44690607,336.80082714)(28.516906,336.81582713)(28.58690907,336.82583496)
\curveto(28.66690585,336.83582711)(28.74690577,336.85082709)(28.82690907,336.87083496)
\curveto(28.89690562,336.89082705)(28.98690553,336.90082704)(29.09690907,336.90083496)
\curveto(29.20690531,336.91082703)(29.3169052,336.91582703)(29.42690907,336.91583496)
\curveto(29.53690498,336.91582703)(29.64190488,336.91082703)(29.74190907,336.90083496)
\curveto(29.85190467,336.89082705)(29.94190458,336.87582707)(30.01190907,336.85583496)
\curveto(30.16190436,336.80582714)(30.30690421,336.76082718)(30.44690907,336.72083496)
\curveto(30.58690393,336.68082726)(30.7169038,336.62582732)(30.83690907,336.55583496)
\curveto(30.90690361,336.50582744)(30.97190355,336.45582749)(31.03190907,336.40583496)
\curveto(31.09190343,336.36582758)(31.15690336,336.32082762)(31.22690907,336.27083496)
\curveto(31.26690325,336.2408277)(31.3219032,336.20082774)(31.39190907,336.15083496)
\curveto(31.47190305,336.10082784)(31.54690297,336.10082784)(31.61690907,336.15083496)
\curveto(31.65690286,336.17082777)(31.67690284,336.20582774)(31.67690907,336.25583496)
\curveto(31.67690284,336.30582764)(31.68690283,336.35582759)(31.70690907,336.40583496)
\lineto(31.70690907,336.55583496)
\curveto(31.7169028,336.58582736)(31.7219028,336.62082732)(31.72190907,336.66083496)
\lineto(31.72190907,336.78083496)
\lineto(31.72190907,338.82083496)
\curveto(31.7219028,338.93082501)(31.7169028,339.05082489)(31.70690907,339.18083496)
\curveto(31.70690281,339.32082462)(31.73190279,339.42582452)(31.78190907,339.49583496)
\curveto(31.8219027,339.57582437)(31.89690262,339.62582432)(32.00690907,339.64583496)
\curveto(32.02690249,339.65582429)(32.04690247,339.65582429)(32.06690907,339.64583496)
\curveto(32.08690243,339.6458243)(32.10690241,339.65082429)(32.12690907,339.66083496)
\lineto(33.19190907,339.66083496)
\curveto(33.31190121,339.66082428)(33.4219011,339.65582429)(33.52190907,339.64583496)
\curveto(33.6219009,339.63582431)(33.69690082,339.59582435)(33.74690907,339.52583496)
\curveto(33.79690072,339.4458245)(33.8219007,339.3408246)(33.82190907,339.21083496)
\lineto(33.82190907,338.85083496)
\lineto(33.82190907,329.83583496)
\moveto(31.78190907,332.77583496)
\curveto(31.79190273,332.81583113)(31.79190273,332.85583109)(31.78190907,332.89583496)
\lineto(31.78190907,333.03083496)
\curveto(31.78190274,333.13083081)(31.77690274,333.23083071)(31.76690907,333.33083496)
\curveto(31.75690276,333.43083051)(31.74190278,333.52083042)(31.72190907,333.60083496)
\curveto(31.70190282,333.71083023)(31.68190284,333.81083013)(31.66190907,333.90083496)
\curveto(31.65190287,333.99082995)(31.62690289,334.07582987)(31.58690907,334.15583496)
\curveto(31.44690307,334.51582943)(31.24190328,334.80082914)(30.97190907,335.01083496)
\curveto(30.71190381,335.22082872)(30.33190419,335.32582862)(29.83190907,335.32583496)
\curveto(29.77190475,335.32582862)(29.69190483,335.31582863)(29.59190907,335.29583496)
\curveto(29.51190501,335.27582867)(29.43690508,335.25582869)(29.36690907,335.23583496)
\curveto(29.30690521,335.22582872)(29.24690527,335.20582874)(29.18690907,335.17583496)
\curveto(28.9169056,335.06582888)(28.70690581,334.89582905)(28.55690907,334.66583496)
\curveto(28.40690611,334.43582951)(28.28690623,334.17582977)(28.19690907,333.88583496)
\curveto(28.16690635,333.78583016)(28.14690637,333.68583026)(28.13690907,333.58583496)
\curveto(28.12690639,333.48583046)(28.10690641,333.38083056)(28.07690907,333.27083496)
\lineto(28.07690907,333.06083496)
\curveto(28.05690646,332.97083097)(28.05190647,332.8458311)(28.06190907,332.68583496)
\curveto(28.07190645,332.53583141)(28.08690643,332.42583152)(28.10690907,332.35583496)
\lineto(28.10690907,332.26583496)
\curveto(28.1169064,332.2458317)(28.1219064,332.22583172)(28.12190907,332.20583496)
\curveto(28.14190638,332.12583182)(28.15690636,332.05083189)(28.16690907,331.98083496)
\curveto(28.18690633,331.91083203)(28.20690631,331.83583211)(28.22690907,331.75583496)
\curveto(28.39690612,331.23583271)(28.68690583,330.85083309)(29.09690907,330.60083496)
\curveto(29.22690529,330.51083343)(29.40690511,330.4408335)(29.63690907,330.39083496)
\curveto(29.67690484,330.38083356)(29.73690478,330.37583357)(29.81690907,330.37583496)
\curveto(29.84690467,330.36583358)(29.89190463,330.35583359)(29.95190907,330.34583496)
\curveto(30.0219045,330.3458336)(30.07690444,330.35083359)(30.11690907,330.36083496)
\curveto(30.19690432,330.38083356)(30.27690424,330.39583355)(30.35690907,330.40583496)
\curveto(30.43690408,330.41583353)(30.516904,330.43583351)(30.59690907,330.46583496)
\curveto(30.84690367,330.57583337)(31.04690347,330.71583323)(31.19690907,330.88583496)
\curveto(31.34690317,331.05583289)(31.47690304,331.27083267)(31.58690907,331.53083496)
\curveto(31.62690289,331.62083232)(31.65690286,331.71083223)(31.67690907,331.80083496)
\curveto(31.69690282,331.90083204)(31.7169028,332.00583194)(31.73690907,332.11583496)
\curveto(31.74690277,332.16583178)(31.74690277,332.21083173)(31.73690907,332.25083496)
\curveto(31.73690278,332.30083164)(31.74690277,332.35083159)(31.76690907,332.40083496)
\curveto(31.77690274,332.43083151)(31.78190274,332.46583148)(31.78190907,332.50583496)
\lineto(31.78190907,332.64083496)
\lineto(31.78190907,332.77583496)
}
}
{
\newrgbcolor{curcolor}{0 0 0}
\pscustom[linestyle=none,fillstyle=solid,fillcolor=curcolor]
{
\newpath
\moveto(37.50183094,339.57083496)
\curveto(37.57182799,339.49082445)(37.60682796,339.37082457)(37.60683094,339.21083496)
\lineto(37.60683094,338.74583496)
\lineto(37.60683094,338.34083496)
\curveto(37.60682796,338.20082574)(37.57182799,338.10582584)(37.50183094,338.05583496)
\curveto(37.44182812,338.00582594)(37.3618282,337.97582597)(37.26183094,337.96583496)
\curveto(37.17182839,337.95582599)(37.07182849,337.95082599)(36.96183094,337.95083496)
\lineto(36.12183094,337.95083496)
\curveto(36.01182955,337.95082599)(35.91182965,337.95582599)(35.82183094,337.96583496)
\curveto(35.74182982,337.97582597)(35.67182989,338.00582594)(35.61183094,338.05583496)
\curveto(35.57182999,338.08582586)(35.54183002,338.1408258)(35.52183094,338.22083496)
\curveto(35.51183005,338.31082563)(35.50183006,338.40582554)(35.49183094,338.50583496)
\lineto(35.49183094,338.83583496)
\curveto(35.50183006,338.945825)(35.50683006,339.0408249)(35.50683094,339.12083496)
\lineto(35.50683094,339.33083496)
\curveto(35.51683005,339.40082454)(35.53683003,339.46082448)(35.56683094,339.51083496)
\curveto(35.58682998,339.55082439)(35.61182995,339.58082436)(35.64183094,339.60083496)
\lineto(35.76183094,339.66083496)
\curveto(35.78182978,339.66082428)(35.80682976,339.66082428)(35.83683094,339.66083496)
\curveto(35.8668297,339.67082427)(35.89182967,339.67582427)(35.91183094,339.67583496)
\lineto(37.00683094,339.67583496)
\curveto(37.10682846,339.67582427)(37.20182836,339.67082427)(37.29183094,339.66083496)
\curveto(37.38182818,339.65082429)(37.45182811,339.62082432)(37.50183094,339.57083496)
\moveto(37.60683094,329.80583496)
\curveto(37.60682796,329.60583434)(37.60182796,329.43583451)(37.59183094,329.29583496)
\curveto(37.58182798,329.15583479)(37.49182807,329.06083488)(37.32183094,329.01083496)
\curveto(37.2618283,328.99083495)(37.19682837,328.98083496)(37.12683094,328.98083496)
\curveto(37.05682851,328.99083495)(36.98182858,328.99583495)(36.90183094,328.99583496)
\lineto(36.06183094,328.99583496)
\curveto(35.97182959,328.99583495)(35.88182968,329.00083494)(35.79183094,329.01083496)
\curveto(35.71182985,329.02083492)(35.65182991,329.05083489)(35.61183094,329.10083496)
\curveto(35.55183001,329.17083477)(35.51683005,329.25583469)(35.50683094,329.35583496)
\lineto(35.50683094,329.70083496)
\lineto(35.50683094,336.03083496)
\lineto(35.50683094,336.33083496)
\curveto(35.50683006,336.43082751)(35.52683004,336.51082743)(35.56683094,336.57083496)
\curveto(35.62682994,336.6408273)(35.71182985,336.68582726)(35.82183094,336.70583496)
\curveto(35.84182972,336.71582723)(35.8668297,336.71582723)(35.89683094,336.70583496)
\curveto(35.93682963,336.70582724)(35.9668296,336.71082723)(35.98683094,336.72083496)
\lineto(36.73683094,336.72083496)
\lineto(36.93183094,336.72083496)
\curveto(37.01182855,336.73082721)(37.07682849,336.73082721)(37.12683094,336.72083496)
\lineto(37.24683094,336.72083496)
\curveto(37.30682826,336.70082724)(37.3618282,336.68582726)(37.41183094,336.67583496)
\curveto(37.4618281,336.66582728)(37.50182806,336.63582731)(37.53183094,336.58583496)
\curveto(37.57182799,336.53582741)(37.59182797,336.46582748)(37.59183094,336.37583496)
\curveto(37.60182796,336.28582766)(37.60682796,336.19082775)(37.60683094,336.09083496)
\lineto(37.60683094,329.80583496)
}
}
{
\newrgbcolor{curcolor}{0 0 0}
\pscustom[linestyle=none,fillstyle=solid,fillcolor=curcolor]
{
\newpath
\moveto(42.83901844,336.93083496)
\curveto(43.64901328,336.95082699)(44.32401261,336.83082711)(44.86401844,336.57083496)
\curveto(45.41401152,336.31082763)(45.84901108,335.940828)(46.16901844,335.46083496)
\curveto(46.3290106,335.22082872)(46.44901048,334.945829)(46.52901844,334.63583496)
\curveto(46.54901038,334.58582936)(46.56401037,334.52082942)(46.57401844,334.44083496)
\curveto(46.59401034,334.36082958)(46.59401034,334.29082965)(46.57401844,334.23083496)
\curveto(46.5340104,334.12082982)(46.46401047,334.05582989)(46.36401844,334.03583496)
\curveto(46.26401067,334.02582992)(46.14401079,334.02082992)(46.00401844,334.02083496)
\lineto(45.22401844,334.02083496)
\lineto(44.93901844,334.02083496)
\curveto(44.84901208,334.02082992)(44.77401216,334.0408299)(44.71401844,334.08083496)
\curveto(44.6340123,334.12082982)(44.57901235,334.18082976)(44.54901844,334.26083496)
\curveto(44.51901241,334.35082959)(44.47901245,334.4408295)(44.42901844,334.53083496)
\curveto(44.36901256,334.6408293)(44.30401263,334.7408292)(44.23401844,334.83083496)
\curveto(44.16401277,334.92082902)(44.08401285,335.00082894)(43.99401844,335.07083496)
\curveto(43.85401308,335.16082878)(43.69901323,335.23082871)(43.52901844,335.28083496)
\curveto(43.46901346,335.30082864)(43.40901352,335.31082863)(43.34901844,335.31083496)
\curveto(43.28901364,335.31082863)(43.2340137,335.32082862)(43.18401844,335.34083496)
\lineto(43.03401844,335.34083496)
\curveto(42.8340141,335.3408286)(42.67401426,335.32082862)(42.55401844,335.28083496)
\curveto(42.26401467,335.19082875)(42.0290149,335.05082889)(41.84901844,334.86083496)
\curveto(41.66901526,334.68082926)(41.52401541,334.46082948)(41.41401844,334.20083496)
\curveto(41.36401557,334.09082985)(41.32401561,333.97082997)(41.29401844,333.84083496)
\curveto(41.27401566,333.72083022)(41.24901568,333.59083035)(41.21901844,333.45083496)
\curveto(41.20901572,333.41083053)(41.20401573,333.37083057)(41.20401844,333.33083496)
\curveto(41.20401573,333.29083065)(41.19901573,333.25083069)(41.18901844,333.21083496)
\curveto(41.16901576,333.11083083)(41.15901577,332.97083097)(41.15901844,332.79083496)
\curveto(41.16901576,332.61083133)(41.18401575,332.47083147)(41.20401844,332.37083496)
\curveto(41.20401573,332.29083165)(41.20901572,332.23583171)(41.21901844,332.20583496)
\curveto(41.23901569,332.13583181)(41.24901568,332.06583188)(41.24901844,331.99583496)
\curveto(41.25901567,331.92583202)(41.27401566,331.85583209)(41.29401844,331.78583496)
\curveto(41.37401556,331.55583239)(41.46901546,331.3458326)(41.57901844,331.15583496)
\curveto(41.68901524,330.96583298)(41.8290151,330.80583314)(41.99901844,330.67583496)
\curveto(42.03901489,330.6458333)(42.09901483,330.61083333)(42.17901844,330.57083496)
\curveto(42.28901464,330.50083344)(42.39901453,330.45583349)(42.50901844,330.43583496)
\curveto(42.6290143,330.41583353)(42.77401416,330.39583355)(42.94401844,330.37583496)
\lineto(43.03401844,330.37583496)
\curveto(43.07401386,330.37583357)(43.10401383,330.38083356)(43.12401844,330.39083496)
\lineto(43.25901844,330.39083496)
\curveto(43.3290136,330.41083353)(43.39401354,330.42583352)(43.45401844,330.43583496)
\curveto(43.52401341,330.45583349)(43.58901334,330.47583347)(43.64901844,330.49583496)
\curveto(43.94901298,330.62583332)(44.17901275,330.81583313)(44.33901844,331.06583496)
\curveto(44.37901255,331.11583283)(44.41401252,331.17083277)(44.44401844,331.23083496)
\curveto(44.47401246,331.30083264)(44.49901243,331.36083258)(44.51901844,331.41083496)
\curveto(44.55901237,331.52083242)(44.59401234,331.61583233)(44.62401844,331.69583496)
\curveto(44.65401228,331.78583216)(44.72401221,331.85583209)(44.83401844,331.90583496)
\curveto(44.92401201,331.945832)(45.06901186,331.96083198)(45.26901844,331.95083496)
\lineto(45.76401844,331.95083496)
\lineto(45.97401844,331.95083496)
\curveto(46.05401088,331.96083198)(46.11901081,331.95583199)(46.16901844,331.93583496)
\lineto(46.28901844,331.93583496)
\lineto(46.40901844,331.90583496)
\curveto(46.44901048,331.90583204)(46.47901045,331.89583205)(46.49901844,331.87583496)
\curveto(46.54901038,331.83583211)(46.57901035,331.77583217)(46.58901844,331.69583496)
\curveto(46.60901032,331.62583232)(46.60901032,331.55083239)(46.58901844,331.47083496)
\curveto(46.49901043,331.1408328)(46.38901054,330.8458331)(46.25901844,330.58583496)
\curveto(45.84901108,329.81583413)(45.19401174,329.28083466)(44.29401844,328.98083496)
\curveto(44.19401274,328.95083499)(44.08901284,328.93083501)(43.97901844,328.92083496)
\curveto(43.86901306,328.90083504)(43.75901317,328.87583507)(43.64901844,328.84583496)
\curveto(43.58901334,328.83583511)(43.5290134,328.83083511)(43.46901844,328.83083496)
\curveto(43.40901352,328.83083511)(43.34901358,328.82583512)(43.28901844,328.81583496)
\lineto(43.12401844,328.81583496)
\curveto(43.07401386,328.79583515)(42.99901393,328.79083515)(42.89901844,328.80083496)
\curveto(42.79901413,328.80083514)(42.72401421,328.80583514)(42.67401844,328.81583496)
\curveto(42.59401434,328.83583511)(42.51901441,328.8458351)(42.44901844,328.84583496)
\curveto(42.38901454,328.83583511)(42.32401461,328.8408351)(42.25401844,328.86083496)
\lineto(42.10401844,328.89083496)
\curveto(42.05401488,328.89083505)(42.00401493,328.89583505)(41.95401844,328.90583496)
\curveto(41.84401509,328.93583501)(41.73901519,328.96583498)(41.63901844,328.99583496)
\curveto(41.53901539,329.02583492)(41.44401549,329.06083488)(41.35401844,329.10083496)
\curveto(40.88401605,329.30083464)(40.48901644,329.55583439)(40.16901844,329.86583496)
\curveto(39.84901708,330.18583376)(39.58901734,330.58083336)(39.38901844,331.05083496)
\curveto(39.33901759,331.1408328)(39.29901763,331.23583271)(39.26901844,331.33583496)
\lineto(39.17901844,331.66583496)
\curveto(39.16901776,331.70583224)(39.16401777,331.7408322)(39.16401844,331.77083496)
\curveto(39.16401777,331.81083213)(39.15401778,331.85583209)(39.13401844,331.90583496)
\curveto(39.11401782,331.97583197)(39.10401783,332.0458319)(39.10401844,332.11583496)
\curveto(39.10401783,332.19583175)(39.09401784,332.27083167)(39.07401844,332.34083496)
\lineto(39.07401844,332.59583496)
\curveto(39.05401788,332.6458313)(39.04401789,332.70083124)(39.04401844,332.76083496)
\curveto(39.04401789,332.83083111)(39.05401788,332.89083105)(39.07401844,332.94083496)
\curveto(39.08401785,332.99083095)(39.08401785,333.03583091)(39.07401844,333.07583496)
\curveto(39.06401787,333.11583083)(39.06401787,333.15583079)(39.07401844,333.19583496)
\curveto(39.09401784,333.26583068)(39.09901783,333.33083061)(39.08901844,333.39083496)
\curveto(39.08901784,333.45083049)(39.09901783,333.51083043)(39.11901844,333.57083496)
\curveto(39.16901776,333.75083019)(39.20901772,333.92083002)(39.23901844,334.08083496)
\curveto(39.26901766,334.25082969)(39.31401762,334.41582953)(39.37401844,334.57583496)
\curveto(39.59401734,335.08582886)(39.86901706,335.51082843)(40.19901844,335.85083496)
\curveto(40.53901639,336.19082775)(40.96901596,336.46582748)(41.48901844,336.67583496)
\curveto(41.6290153,336.73582721)(41.77401516,336.77582717)(41.92401844,336.79583496)
\curveto(42.07401486,336.82582712)(42.2290147,336.86082708)(42.38901844,336.90083496)
\curveto(42.46901446,336.91082703)(42.54401439,336.91582703)(42.61401844,336.91583496)
\curveto(42.68401425,336.91582703)(42.75901417,336.92082702)(42.83901844,336.93083496)
}
}
{
\newrgbcolor{curcolor}{0 0 0}
\pscustom[linestyle=none,fillstyle=solid,fillcolor=curcolor]
{
\newpath
\moveto(54.93229969,329.58083496)
\curveto(54.95229184,329.47083447)(54.96229183,329.36083458)(54.96229969,329.25083496)
\curveto(54.97229182,329.1408348)(54.92229187,329.06583488)(54.81229969,329.02583496)
\curveto(54.75229204,328.99583495)(54.68229211,328.98083496)(54.60229969,328.98083496)
\lineto(54.36229969,328.98083496)
\lineto(53.55229969,328.98083496)
\lineto(53.28229969,328.98083496)
\curveto(53.20229359,328.99083495)(53.13729366,329.01583493)(53.08729969,329.05583496)
\curveto(53.01729378,329.09583485)(52.96229383,329.15083479)(52.92229969,329.22083496)
\curveto(52.8922939,329.30083464)(52.84729395,329.36583458)(52.78729969,329.41583496)
\curveto(52.76729403,329.43583451)(52.74229405,329.45083449)(52.71229969,329.46083496)
\curveto(52.68229411,329.48083446)(52.64229415,329.48583446)(52.59229969,329.47583496)
\curveto(52.54229425,329.45583449)(52.4922943,329.43083451)(52.44229969,329.40083496)
\curveto(52.40229439,329.37083457)(52.35729444,329.3458346)(52.30729969,329.32583496)
\curveto(52.25729454,329.28583466)(52.20229459,329.25083469)(52.14229969,329.22083496)
\lineto(51.96229969,329.13083496)
\curveto(51.83229496,329.07083487)(51.6972951,329.02083492)(51.55729969,328.98083496)
\curveto(51.41729538,328.95083499)(51.27229552,328.91583503)(51.12229969,328.87583496)
\curveto(51.05229574,328.85583509)(50.98229581,328.8458351)(50.91229969,328.84583496)
\curveto(50.85229594,328.83583511)(50.78729601,328.82583512)(50.71729969,328.81583496)
\lineto(50.62729969,328.81583496)
\curveto(50.5972962,328.80583514)(50.56729623,328.80083514)(50.53729969,328.80083496)
\lineto(50.37229969,328.80083496)
\curveto(50.27229652,328.78083516)(50.17229662,328.78083516)(50.07229969,328.80083496)
\lineto(49.93729969,328.80083496)
\curveto(49.86729693,328.82083512)(49.797297,328.83083511)(49.72729969,328.83083496)
\curveto(49.66729713,328.82083512)(49.60729719,328.82583512)(49.54729969,328.84583496)
\curveto(49.44729735,328.86583508)(49.35229744,328.88583506)(49.26229969,328.90583496)
\curveto(49.17229762,328.91583503)(49.08729771,328.940835)(49.00729969,328.98083496)
\curveto(48.71729808,329.09083485)(48.46729833,329.23083471)(48.25729969,329.40083496)
\curveto(48.05729874,329.58083436)(47.8972989,329.81583413)(47.77729969,330.10583496)
\curveto(47.74729905,330.17583377)(47.71729908,330.25083369)(47.68729969,330.33083496)
\curveto(47.66729913,330.41083353)(47.64729915,330.49583345)(47.62729969,330.58583496)
\curveto(47.60729919,330.63583331)(47.5972992,330.68583326)(47.59729969,330.73583496)
\curveto(47.60729919,330.78583316)(47.60729919,330.83583311)(47.59729969,330.88583496)
\curveto(47.58729921,330.91583303)(47.57729922,330.97583297)(47.56729969,331.06583496)
\curveto(47.56729923,331.16583278)(47.57229922,331.23583271)(47.58229969,331.27583496)
\curveto(47.60229919,331.37583257)(47.61229918,331.46083248)(47.61229969,331.53083496)
\lineto(47.70229969,331.86083496)
\curveto(47.73229906,331.98083196)(47.77229902,332.08583186)(47.82229969,332.17583496)
\curveto(47.9922988,332.46583148)(48.18729861,332.68583126)(48.40729969,332.83583496)
\curveto(48.62729817,332.98583096)(48.90729789,333.11583083)(49.24729969,333.22583496)
\curveto(49.37729742,333.27583067)(49.51229728,333.31083063)(49.65229969,333.33083496)
\curveto(49.792297,333.35083059)(49.93229686,333.37583057)(50.07229969,333.40583496)
\curveto(50.15229664,333.42583052)(50.23729656,333.43583051)(50.32729969,333.43583496)
\curveto(50.41729638,333.4458305)(50.50729629,333.46083048)(50.59729969,333.48083496)
\curveto(50.66729613,333.50083044)(50.73729606,333.50583044)(50.80729969,333.49583496)
\curveto(50.87729592,333.49583045)(50.95229584,333.50583044)(51.03229969,333.52583496)
\curveto(51.10229569,333.5458304)(51.17229562,333.55583039)(51.24229969,333.55583496)
\curveto(51.31229548,333.55583039)(51.38729541,333.56583038)(51.46729969,333.58583496)
\curveto(51.67729512,333.63583031)(51.86729493,333.67583027)(52.03729969,333.70583496)
\curveto(52.21729458,333.7458302)(52.37729442,333.83583011)(52.51729969,333.97583496)
\curveto(52.60729419,334.06582988)(52.66729413,334.16582978)(52.69729969,334.27583496)
\curveto(52.70729409,334.30582964)(52.70729409,334.33082961)(52.69729969,334.35083496)
\curveto(52.6972941,334.37082957)(52.70229409,334.39082955)(52.71229969,334.41083496)
\curveto(52.72229407,334.43082951)(52.72729407,334.46082948)(52.72729969,334.50083496)
\lineto(52.72729969,334.59083496)
\lineto(52.69729969,334.71083496)
\curveto(52.6972941,334.75082919)(52.6922941,334.78582916)(52.68229969,334.81583496)
\curveto(52.58229421,335.11582883)(52.37229442,335.32082862)(52.05229969,335.43083496)
\curveto(51.96229483,335.46082848)(51.85229494,335.48082846)(51.72229969,335.49083496)
\curveto(51.60229519,335.51082843)(51.47729532,335.51582843)(51.34729969,335.50583496)
\curveto(51.21729558,335.50582844)(51.0922957,335.49582845)(50.97229969,335.47583496)
\curveto(50.85229594,335.45582849)(50.74729605,335.43082851)(50.65729969,335.40083496)
\curveto(50.5972962,335.38082856)(50.53729626,335.35082859)(50.47729969,335.31083496)
\curveto(50.42729637,335.28082866)(50.37729642,335.2458287)(50.32729969,335.20583496)
\curveto(50.27729652,335.16582878)(50.22229657,335.11082883)(50.16229969,335.04083496)
\curveto(50.11229668,334.97082897)(50.07729672,334.90582904)(50.05729969,334.84583496)
\curveto(50.00729679,334.7458292)(49.96229683,334.65082929)(49.92229969,334.56083496)
\curveto(49.8922969,334.47082947)(49.82229697,334.41082953)(49.71229969,334.38083496)
\curveto(49.63229716,334.36082958)(49.54729725,334.35082959)(49.45729969,334.35083496)
\lineto(49.18729969,334.35083496)
\lineto(48.61729969,334.35083496)
\curveto(48.56729823,334.35082959)(48.51729828,334.3458296)(48.46729969,334.33583496)
\curveto(48.41729838,334.33582961)(48.37229842,334.3408296)(48.33229969,334.35083496)
\lineto(48.19729969,334.35083496)
\curveto(48.17729862,334.36082958)(48.15229864,334.36582958)(48.12229969,334.36583496)
\curveto(48.0922987,334.36582958)(48.06729873,334.37582957)(48.04729969,334.39583496)
\curveto(47.96729883,334.41582953)(47.91229888,334.48082946)(47.88229969,334.59083496)
\curveto(47.87229892,334.6408293)(47.87229892,334.69082925)(47.88229969,334.74083496)
\curveto(47.8922989,334.79082915)(47.90229889,334.83582911)(47.91229969,334.87583496)
\curveto(47.94229885,334.98582896)(47.97229882,335.08582886)(48.00229969,335.17583496)
\curveto(48.04229875,335.27582867)(48.08729871,335.36582858)(48.13729969,335.44583496)
\lineto(48.22729969,335.59583496)
\lineto(48.31729969,335.74583496)
\curveto(48.3972984,335.85582809)(48.4972983,335.96082798)(48.61729969,336.06083496)
\curveto(48.63729816,336.07082787)(48.66729813,336.09582785)(48.70729969,336.13583496)
\curveto(48.75729804,336.17582777)(48.80229799,336.21082773)(48.84229969,336.24083496)
\curveto(48.88229791,336.27082767)(48.92729787,336.30082764)(48.97729969,336.33083496)
\curveto(49.14729765,336.4408275)(49.32729747,336.52582742)(49.51729969,336.58583496)
\curveto(49.70729709,336.65582729)(49.90229689,336.72082722)(50.10229969,336.78083496)
\curveto(50.22229657,336.81082713)(50.34729645,336.83082711)(50.47729969,336.84083496)
\curveto(50.60729619,336.85082709)(50.73729606,336.87082707)(50.86729969,336.90083496)
\curveto(50.90729589,336.91082703)(50.96729583,336.91082703)(51.04729969,336.90083496)
\curveto(51.13729566,336.89082705)(51.1922956,336.89582705)(51.21229969,336.91583496)
\curveto(51.62229517,336.92582702)(52.01229478,336.91082703)(52.38229969,336.87083496)
\curveto(52.76229403,336.83082711)(53.10229369,336.75582719)(53.40229969,336.64583496)
\curveto(53.71229308,336.53582741)(53.97729282,336.38582756)(54.19729969,336.19583496)
\curveto(54.41729238,336.01582793)(54.58729221,335.78082816)(54.70729969,335.49083496)
\curveto(54.77729202,335.32082862)(54.81729198,335.12582882)(54.82729969,334.90583496)
\curveto(54.83729196,334.68582926)(54.84229195,334.46082948)(54.84229969,334.23083496)
\lineto(54.84229969,330.88583496)
\lineto(54.84229969,330.30083496)
\curveto(54.84229195,330.11083383)(54.86229193,329.93583401)(54.90229969,329.77583496)
\curveto(54.91229188,329.7458342)(54.91729188,329.71083423)(54.91729969,329.67083496)
\curveto(54.91729188,329.6408343)(54.92229187,329.61083433)(54.93229969,329.58083496)
\moveto(52.72729969,331.89083496)
\curveto(52.73729406,331.940832)(52.74229405,331.99583195)(52.74229969,332.05583496)
\curveto(52.74229405,332.12583182)(52.73729406,332.18583176)(52.72729969,332.23583496)
\curveto(52.70729409,332.29583165)(52.6972941,332.35083159)(52.69729969,332.40083496)
\curveto(52.6972941,332.45083149)(52.67729412,332.49083145)(52.63729969,332.52083496)
\curveto(52.58729421,332.56083138)(52.51229428,332.58083136)(52.41229969,332.58083496)
\curveto(52.37229442,332.57083137)(52.33729446,332.56083138)(52.30729969,332.55083496)
\curveto(52.27729452,332.55083139)(52.24229455,332.5458314)(52.20229969,332.53583496)
\curveto(52.13229466,332.51583143)(52.05729474,332.50083144)(51.97729969,332.49083496)
\curveto(51.8972949,332.48083146)(51.81729498,332.46583148)(51.73729969,332.44583496)
\curveto(51.70729509,332.43583151)(51.66229513,332.43083151)(51.60229969,332.43083496)
\curveto(51.47229532,332.40083154)(51.34229545,332.38083156)(51.21229969,332.37083496)
\curveto(51.08229571,332.36083158)(50.95729584,332.33583161)(50.83729969,332.29583496)
\curveto(50.75729604,332.27583167)(50.68229611,332.25583169)(50.61229969,332.23583496)
\curveto(50.54229625,332.22583172)(50.47229632,332.20583174)(50.40229969,332.17583496)
\curveto(50.1922966,332.08583186)(50.01229678,331.95083199)(49.86229969,331.77083496)
\curveto(49.72229707,331.59083235)(49.67229712,331.3408326)(49.71229969,331.02083496)
\curveto(49.73229706,330.85083309)(49.78729701,330.71083323)(49.87729969,330.60083496)
\curveto(49.94729685,330.49083345)(50.05229674,330.40083354)(50.19229969,330.33083496)
\curveto(50.33229646,330.27083367)(50.48229631,330.22583372)(50.64229969,330.19583496)
\curveto(50.81229598,330.16583378)(50.98729581,330.15583379)(51.16729969,330.16583496)
\curveto(51.35729544,330.18583376)(51.53229526,330.22083372)(51.69229969,330.27083496)
\curveto(51.95229484,330.35083359)(52.15729464,330.47583347)(52.30729969,330.64583496)
\curveto(52.45729434,330.82583312)(52.57229422,331.0458329)(52.65229969,331.30583496)
\curveto(52.67229412,331.37583257)(52.68229411,331.4458325)(52.68229969,331.51583496)
\curveto(52.6922941,331.59583235)(52.70729409,331.67583227)(52.72729969,331.75583496)
\lineto(52.72729969,331.89083496)
}
}
{
\newrgbcolor{curcolor}{0 0 0}
\pscustom[linestyle=none,fillstyle=solid,fillcolor=curcolor]
{
\newpath
\moveto(64.08558094,329.83583496)
\lineto(64.08558094,329.41583496)
\curveto(64.08557257,329.28583466)(64.0555726,329.18083476)(63.99558094,329.10083496)
\curveto(63.94557271,329.05083489)(63.88057278,329.01583493)(63.80058094,328.99583496)
\curveto(63.72057294,328.98583496)(63.63057303,328.98083496)(63.53058094,328.98083496)
\lineto(62.70558094,328.98083496)
\lineto(62.42058094,328.98083496)
\curveto(62.34057432,328.99083495)(62.27557438,329.01583493)(62.22558094,329.05583496)
\curveto(62.1555745,329.10583484)(62.11557454,329.17083477)(62.10558094,329.25083496)
\curveto(62.09557456,329.33083461)(62.07557458,329.41083453)(62.04558094,329.49083496)
\curveto(62.02557463,329.51083443)(62.00557465,329.52583442)(61.98558094,329.53583496)
\curveto(61.97557468,329.55583439)(61.9605747,329.57583437)(61.94058094,329.59583496)
\curveto(61.83057483,329.59583435)(61.75057491,329.57083437)(61.70058094,329.52083496)
\lineto(61.55058094,329.37083496)
\curveto(61.48057518,329.32083462)(61.41557524,329.27583467)(61.35558094,329.23583496)
\curveto(61.29557536,329.20583474)(61.23057543,329.16583478)(61.16058094,329.11583496)
\curveto(61.12057554,329.09583485)(61.07557558,329.07583487)(61.02558094,329.05583496)
\curveto(60.98557567,329.03583491)(60.94057572,329.01583493)(60.89058094,328.99583496)
\curveto(60.75057591,328.945835)(60.60057606,328.90083504)(60.44058094,328.86083496)
\curveto(60.39057627,328.8408351)(60.34557631,328.83083511)(60.30558094,328.83083496)
\curveto(60.26557639,328.83083511)(60.22557643,328.82583512)(60.18558094,328.81583496)
\lineto(60.05058094,328.81583496)
\curveto(60.02057664,328.80583514)(59.98057668,328.80083514)(59.93058094,328.80083496)
\lineto(59.79558094,328.80083496)
\curveto(59.73557692,328.78083516)(59.64557701,328.77583517)(59.52558094,328.78583496)
\curveto(59.40557725,328.78583516)(59.32057734,328.79583515)(59.27058094,328.81583496)
\curveto(59.20057746,328.83583511)(59.13557752,328.8458351)(59.07558094,328.84583496)
\curveto(59.02557763,328.83583511)(58.97057769,328.8408351)(58.91058094,328.86083496)
\lineto(58.55058094,328.98083496)
\curveto(58.44057822,329.01083493)(58.33057833,329.05083489)(58.22058094,329.10083496)
\curveto(57.87057879,329.25083469)(57.5555791,329.48083446)(57.27558094,329.79083496)
\curveto(57.00557965,330.11083383)(56.79057987,330.4458335)(56.63058094,330.79583496)
\curveto(56.58058008,330.90583304)(56.54058012,331.01083293)(56.51058094,331.11083496)
\curveto(56.48058018,331.22083272)(56.44558021,331.33083261)(56.40558094,331.44083496)
\curveto(56.39558026,331.48083246)(56.39058027,331.51583243)(56.39058094,331.54583496)
\curveto(56.39058027,331.58583236)(56.38058028,331.63083231)(56.36058094,331.68083496)
\curveto(56.34058032,331.76083218)(56.32058034,331.8458321)(56.30058094,331.93583496)
\curveto(56.29058037,332.03583191)(56.27558038,332.13583181)(56.25558094,332.23583496)
\curveto(56.24558041,332.26583168)(56.24058042,332.30083164)(56.24058094,332.34083496)
\curveto(56.25058041,332.38083156)(56.25058041,332.41583153)(56.24058094,332.44583496)
\lineto(56.24058094,332.58083496)
\curveto(56.24058042,332.63083131)(56.23558042,332.68083126)(56.22558094,332.73083496)
\curveto(56.21558044,332.78083116)(56.21058045,332.83583111)(56.21058094,332.89583496)
\curveto(56.21058045,332.96583098)(56.21558044,333.02083092)(56.22558094,333.06083496)
\curveto(56.23558042,333.11083083)(56.24058042,333.15583079)(56.24058094,333.19583496)
\lineto(56.24058094,333.34583496)
\curveto(56.25058041,333.39583055)(56.25058041,333.4408305)(56.24058094,333.48083496)
\curveto(56.24058042,333.53083041)(56.25058041,333.58083036)(56.27058094,333.63083496)
\curveto(56.29058037,333.7408302)(56.30558035,333.8458301)(56.31558094,333.94583496)
\curveto(56.33558032,334.0458299)(56.3605803,334.1458298)(56.39058094,334.24583496)
\curveto(56.43058023,334.36582958)(56.46558019,334.48082946)(56.49558094,334.59083496)
\curveto(56.52558013,334.70082924)(56.56558009,334.81082913)(56.61558094,334.92083496)
\curveto(56.7555799,335.22082872)(56.93057973,335.50582844)(57.14058094,335.77583496)
\curveto(57.1605795,335.80582814)(57.18557947,335.83082811)(57.21558094,335.85083496)
\curveto(57.2555794,335.88082806)(57.28557937,335.91082803)(57.30558094,335.94083496)
\curveto(57.34557931,335.99082795)(57.38557927,336.03582791)(57.42558094,336.07583496)
\curveto(57.46557919,336.11582783)(57.51057915,336.15582779)(57.56058094,336.19583496)
\curveto(57.60057906,336.21582773)(57.63557902,336.2408277)(57.66558094,336.27083496)
\curveto(57.69557896,336.31082763)(57.73057893,336.3408276)(57.77058094,336.36083496)
\curveto(58.02057864,336.53082741)(58.31057835,336.67082727)(58.64058094,336.78083496)
\curveto(58.71057795,336.80082714)(58.78057788,336.81582713)(58.85058094,336.82583496)
\curveto(58.93057773,336.83582711)(59.01057765,336.85082709)(59.09058094,336.87083496)
\curveto(59.1605775,336.89082705)(59.25057741,336.90082704)(59.36058094,336.90083496)
\curveto(59.47057719,336.91082703)(59.58057708,336.91582703)(59.69058094,336.91583496)
\curveto(59.80057686,336.91582703)(59.90557675,336.91082703)(60.00558094,336.90083496)
\curveto(60.11557654,336.89082705)(60.20557645,336.87582707)(60.27558094,336.85583496)
\curveto(60.42557623,336.80582714)(60.57057609,336.76082718)(60.71058094,336.72083496)
\curveto(60.85057581,336.68082726)(60.98057568,336.62582732)(61.10058094,336.55583496)
\curveto(61.17057549,336.50582744)(61.23557542,336.45582749)(61.29558094,336.40583496)
\curveto(61.3555753,336.36582758)(61.42057524,336.32082762)(61.49058094,336.27083496)
\curveto(61.53057513,336.2408277)(61.58557507,336.20082774)(61.65558094,336.15083496)
\curveto(61.73557492,336.10082784)(61.81057485,336.10082784)(61.88058094,336.15083496)
\curveto(61.92057474,336.17082777)(61.94057472,336.20582774)(61.94058094,336.25583496)
\curveto(61.94057472,336.30582764)(61.95057471,336.35582759)(61.97058094,336.40583496)
\lineto(61.97058094,336.55583496)
\curveto(61.98057468,336.58582736)(61.98557467,336.62082732)(61.98558094,336.66083496)
\lineto(61.98558094,336.78083496)
\lineto(61.98558094,338.82083496)
\curveto(61.98557467,338.93082501)(61.98057468,339.05082489)(61.97058094,339.18083496)
\curveto(61.97057469,339.32082462)(61.99557466,339.42582452)(62.04558094,339.49583496)
\curveto(62.08557457,339.57582437)(62.1605745,339.62582432)(62.27058094,339.64583496)
\curveto(62.29057437,339.65582429)(62.31057435,339.65582429)(62.33058094,339.64583496)
\curveto(62.35057431,339.6458243)(62.37057429,339.65082429)(62.39058094,339.66083496)
\lineto(63.45558094,339.66083496)
\curveto(63.57557308,339.66082428)(63.68557297,339.65582429)(63.78558094,339.64583496)
\curveto(63.88557277,339.63582431)(63.9605727,339.59582435)(64.01058094,339.52583496)
\curveto(64.0605726,339.4458245)(64.08557257,339.3408246)(64.08558094,339.21083496)
\lineto(64.08558094,338.85083496)
\lineto(64.08558094,329.83583496)
\moveto(62.04558094,332.77583496)
\curveto(62.0555746,332.81583113)(62.0555746,332.85583109)(62.04558094,332.89583496)
\lineto(62.04558094,333.03083496)
\curveto(62.04557461,333.13083081)(62.04057462,333.23083071)(62.03058094,333.33083496)
\curveto(62.02057464,333.43083051)(62.00557465,333.52083042)(61.98558094,333.60083496)
\curveto(61.96557469,333.71083023)(61.94557471,333.81083013)(61.92558094,333.90083496)
\curveto(61.91557474,333.99082995)(61.89057477,334.07582987)(61.85058094,334.15583496)
\curveto(61.71057495,334.51582943)(61.50557515,334.80082914)(61.23558094,335.01083496)
\curveto(60.97557568,335.22082872)(60.59557606,335.32582862)(60.09558094,335.32583496)
\curveto(60.03557662,335.32582862)(59.9555767,335.31582863)(59.85558094,335.29583496)
\curveto(59.77557688,335.27582867)(59.70057696,335.25582869)(59.63058094,335.23583496)
\curveto(59.57057709,335.22582872)(59.51057715,335.20582874)(59.45058094,335.17583496)
\curveto(59.18057748,335.06582888)(58.97057769,334.89582905)(58.82058094,334.66583496)
\curveto(58.67057799,334.43582951)(58.55057811,334.17582977)(58.46058094,333.88583496)
\curveto(58.43057823,333.78583016)(58.41057825,333.68583026)(58.40058094,333.58583496)
\curveto(58.39057827,333.48583046)(58.37057829,333.38083056)(58.34058094,333.27083496)
\lineto(58.34058094,333.06083496)
\curveto(58.32057834,332.97083097)(58.31557834,332.8458311)(58.32558094,332.68583496)
\curveto(58.33557832,332.53583141)(58.35057831,332.42583152)(58.37058094,332.35583496)
\lineto(58.37058094,332.26583496)
\curveto(58.38057828,332.2458317)(58.38557827,332.22583172)(58.38558094,332.20583496)
\curveto(58.40557825,332.12583182)(58.42057824,332.05083189)(58.43058094,331.98083496)
\curveto(58.45057821,331.91083203)(58.47057819,331.83583211)(58.49058094,331.75583496)
\curveto(58.660578,331.23583271)(58.95057771,330.85083309)(59.36058094,330.60083496)
\curveto(59.49057717,330.51083343)(59.67057699,330.4408335)(59.90058094,330.39083496)
\curveto(59.94057672,330.38083356)(60.00057666,330.37583357)(60.08058094,330.37583496)
\curveto(60.11057655,330.36583358)(60.1555765,330.35583359)(60.21558094,330.34583496)
\curveto(60.28557637,330.3458336)(60.34057632,330.35083359)(60.38058094,330.36083496)
\curveto(60.4605762,330.38083356)(60.54057612,330.39583355)(60.62058094,330.40583496)
\curveto(60.70057596,330.41583353)(60.78057588,330.43583351)(60.86058094,330.46583496)
\curveto(61.11057555,330.57583337)(61.31057535,330.71583323)(61.46058094,330.88583496)
\curveto(61.61057505,331.05583289)(61.74057492,331.27083267)(61.85058094,331.53083496)
\curveto(61.89057477,331.62083232)(61.92057474,331.71083223)(61.94058094,331.80083496)
\curveto(61.9605747,331.90083204)(61.98057468,332.00583194)(62.00058094,332.11583496)
\curveto(62.01057465,332.16583178)(62.01057465,332.21083173)(62.00058094,332.25083496)
\curveto(62.00057466,332.30083164)(62.01057465,332.35083159)(62.03058094,332.40083496)
\curveto(62.04057462,332.43083151)(62.04557461,332.46583148)(62.04558094,332.50583496)
\lineto(62.04558094,332.64083496)
\lineto(62.04558094,332.77583496)
}
}
{
\newrgbcolor{curcolor}{0 0 0}
\pscustom[linestyle=none,fillstyle=solid,fillcolor=curcolor]
{
\newpath
\moveto(73.43550282,333.16583496)
\curveto(73.45549425,333.10583084)(73.46549424,333.02083092)(73.46550282,332.91083496)
\curveto(73.46549424,332.80083114)(73.45549425,332.71583123)(73.43550282,332.65583496)
\lineto(73.43550282,332.50583496)
\curveto(73.41549429,332.42583152)(73.4054943,332.3458316)(73.40550282,332.26583496)
\curveto(73.41549429,332.18583176)(73.41049429,332.10583184)(73.39050282,332.02583496)
\curveto(73.37049433,331.95583199)(73.35549435,331.89083205)(73.34550282,331.83083496)
\curveto(73.33549437,331.77083217)(73.32549438,331.70583224)(73.31550282,331.63583496)
\curveto(73.27549443,331.52583242)(73.24049446,331.41083253)(73.21050282,331.29083496)
\curveto(73.18049452,331.18083276)(73.14049456,331.07583287)(73.09050282,330.97583496)
\curveto(72.88049482,330.49583345)(72.6054951,330.10583384)(72.26550282,329.80583496)
\curveto(71.92549578,329.50583444)(71.51549619,329.25583469)(71.03550282,329.05583496)
\curveto(70.91549679,329.00583494)(70.79049691,328.97083497)(70.66050282,328.95083496)
\curveto(70.54049716,328.92083502)(70.41549729,328.89083505)(70.28550282,328.86083496)
\curveto(70.23549747,328.8408351)(70.18049752,328.83083511)(70.12050282,328.83083496)
\curveto(70.06049764,328.83083511)(70.0054977,328.82583512)(69.95550282,328.81583496)
\lineto(69.85050282,328.81583496)
\curveto(69.82049788,328.80583514)(69.79049791,328.80083514)(69.76050282,328.80083496)
\curveto(69.71049799,328.79083515)(69.63049807,328.78583516)(69.52050282,328.78583496)
\curveto(69.41049829,328.77583517)(69.32549838,328.78083516)(69.26550282,328.80083496)
\lineto(69.11550282,328.80083496)
\curveto(69.06549864,328.81083513)(69.01049869,328.81583513)(68.95050282,328.81583496)
\curveto(68.9004988,328.80583514)(68.85049885,328.81083513)(68.80050282,328.83083496)
\curveto(68.76049894,328.8408351)(68.72049898,328.8458351)(68.68050282,328.84583496)
\curveto(68.65049905,328.8458351)(68.61049909,328.85083509)(68.56050282,328.86083496)
\curveto(68.46049924,328.89083505)(68.36049934,328.91583503)(68.26050282,328.93583496)
\curveto(68.16049954,328.95583499)(68.06549964,328.98583496)(67.97550282,329.02583496)
\curveto(67.85549985,329.06583488)(67.74049996,329.10583484)(67.63050282,329.14583496)
\curveto(67.53050017,329.18583476)(67.42550028,329.23583471)(67.31550282,329.29583496)
\curveto(66.96550074,329.50583444)(66.66550104,329.75083419)(66.41550282,330.03083496)
\curveto(66.16550154,330.31083363)(65.95550175,330.6458333)(65.78550282,331.03583496)
\curveto(65.73550197,331.12583282)(65.69550201,331.22083272)(65.66550282,331.32083496)
\curveto(65.64550206,331.42083252)(65.62050208,331.52583242)(65.59050282,331.63583496)
\curveto(65.57050213,331.68583226)(65.56050214,331.73083221)(65.56050282,331.77083496)
\curveto(65.56050214,331.81083213)(65.55050215,331.85583209)(65.53050282,331.90583496)
\curveto(65.51050219,331.98583196)(65.5005022,332.06583188)(65.50050282,332.14583496)
\curveto(65.5005022,332.23583171)(65.49050221,332.32083162)(65.47050282,332.40083496)
\curveto(65.46050224,332.45083149)(65.45550225,332.49583145)(65.45550282,332.53583496)
\lineto(65.45550282,332.67083496)
\curveto(65.43550227,332.73083121)(65.42550228,332.81583113)(65.42550282,332.92583496)
\curveto(65.43550227,333.03583091)(65.45050225,333.12083082)(65.47050282,333.18083496)
\lineto(65.47050282,333.28583496)
\curveto(65.48050222,333.33583061)(65.48050222,333.38583056)(65.47050282,333.43583496)
\curveto(65.47050223,333.49583045)(65.48050222,333.55083039)(65.50050282,333.60083496)
\curveto(65.51050219,333.65083029)(65.51550219,333.69583025)(65.51550282,333.73583496)
\curveto(65.51550219,333.78583016)(65.52550218,333.83583011)(65.54550282,333.88583496)
\curveto(65.58550212,334.01582993)(65.62050208,334.1408298)(65.65050282,334.26083496)
\curveto(65.68050202,334.39082955)(65.72050198,334.51582943)(65.77050282,334.63583496)
\curveto(65.95050175,335.0458289)(66.16550154,335.38582856)(66.41550282,335.65583496)
\curveto(66.66550104,335.93582801)(66.97050073,336.19082775)(67.33050282,336.42083496)
\curveto(67.43050027,336.47082747)(67.53550017,336.51582743)(67.64550282,336.55583496)
\curveto(67.75549995,336.59582735)(67.86549984,336.6408273)(67.97550282,336.69083496)
\curveto(68.1054996,336.7408272)(68.24049946,336.77582717)(68.38050282,336.79583496)
\curveto(68.52049918,336.81582713)(68.66549904,336.8458271)(68.81550282,336.88583496)
\curveto(68.89549881,336.89582705)(68.97049873,336.90082704)(69.04050282,336.90083496)
\curveto(69.11049859,336.90082704)(69.18049852,336.90582704)(69.25050282,336.91583496)
\curveto(69.83049787,336.92582702)(70.33049737,336.86582708)(70.75050282,336.73583496)
\curveto(71.18049652,336.60582734)(71.56049614,336.42582752)(71.89050282,336.19583496)
\curveto(72.0004957,336.11582783)(72.11049559,336.02582792)(72.22050282,335.92583496)
\curveto(72.34049536,335.83582811)(72.44049526,335.73582821)(72.52050282,335.62583496)
\curveto(72.6004951,335.52582842)(72.67049503,335.42582852)(72.73050282,335.32583496)
\curveto(72.8004949,335.22582872)(72.87049483,335.12082882)(72.94050282,335.01083496)
\curveto(73.01049469,334.90082904)(73.06549464,334.78082916)(73.10550282,334.65083496)
\curveto(73.14549456,334.53082941)(73.19049451,334.40082954)(73.24050282,334.26083496)
\curveto(73.27049443,334.18082976)(73.29549441,334.09582985)(73.31550282,334.00583496)
\lineto(73.37550282,333.73583496)
\curveto(73.38549432,333.69583025)(73.39049431,333.65583029)(73.39050282,333.61583496)
\curveto(73.39049431,333.57583037)(73.39549431,333.53583041)(73.40550282,333.49583496)
\curveto(73.42549428,333.4458305)(73.43049427,333.39083055)(73.42050282,333.33083496)
\curveto(73.41049429,333.27083067)(73.41549429,333.21583073)(73.43550282,333.16583496)
\moveto(71.33550282,332.62583496)
\curveto(71.34549636,332.67583127)(71.35049635,332.7458312)(71.35050282,332.83583496)
\curveto(71.35049635,332.93583101)(71.34549636,333.01083093)(71.33550282,333.06083496)
\lineto(71.33550282,333.18083496)
\curveto(71.31549639,333.23083071)(71.3054964,333.28583066)(71.30550282,333.34583496)
\curveto(71.3054964,333.40583054)(71.3004964,333.46083048)(71.29050282,333.51083496)
\curveto(71.29049641,333.55083039)(71.28549642,333.58083036)(71.27550282,333.60083496)
\lineto(71.21550282,333.84083496)
\curveto(71.2054965,333.93083001)(71.18549652,334.01582993)(71.15550282,334.09583496)
\curveto(71.04549666,334.35582959)(70.91549679,334.57582937)(70.76550282,334.75583496)
\curveto(70.61549709,334.945829)(70.41549729,335.09582885)(70.16550282,335.20583496)
\curveto(70.1054976,335.22582872)(70.04549766,335.2408287)(69.98550282,335.25083496)
\curveto(69.92549778,335.27082867)(69.86049784,335.29082865)(69.79050282,335.31083496)
\curveto(69.71049799,335.33082861)(69.62549808,335.33582861)(69.53550282,335.32583496)
\lineto(69.26550282,335.32583496)
\curveto(69.23549847,335.30582864)(69.2004985,335.29582865)(69.16050282,335.29583496)
\curveto(69.12049858,335.30582864)(69.08549862,335.30582864)(69.05550282,335.29583496)
\lineto(68.84550282,335.23583496)
\curveto(68.78549892,335.22582872)(68.73049897,335.20582874)(68.68050282,335.17583496)
\curveto(68.43049927,335.06582888)(68.22549948,334.90582904)(68.06550282,334.69583496)
\curveto(67.91549979,334.49582945)(67.79549991,334.26082968)(67.70550282,333.99083496)
\curveto(67.67550003,333.89083005)(67.65050005,333.78583016)(67.63050282,333.67583496)
\curveto(67.62050008,333.56583038)(67.6055001,333.45583049)(67.58550282,333.34583496)
\curveto(67.57550013,333.29583065)(67.57050013,333.2458307)(67.57050282,333.19583496)
\lineto(67.57050282,333.04583496)
\curveto(67.55050015,332.97583097)(67.54050016,332.87083107)(67.54050282,332.73083496)
\curveto(67.55050015,332.59083135)(67.56550014,332.48583146)(67.58550282,332.41583496)
\lineto(67.58550282,332.28083496)
\curveto(67.6055001,332.20083174)(67.62050008,332.12083182)(67.63050282,332.04083496)
\curveto(67.64050006,331.97083197)(67.65550005,331.89583205)(67.67550282,331.81583496)
\curveto(67.77549993,331.51583243)(67.88049982,331.27083267)(67.99050282,331.08083496)
\curveto(68.11049959,330.90083304)(68.29549941,330.73583321)(68.54550282,330.58583496)
\curveto(68.61549909,330.53583341)(68.69049901,330.49583345)(68.77050282,330.46583496)
\curveto(68.86049884,330.43583351)(68.95049875,330.41083353)(69.04050282,330.39083496)
\curveto(69.08049862,330.38083356)(69.11549859,330.37583357)(69.14550282,330.37583496)
\curveto(69.17549853,330.38583356)(69.21049849,330.38583356)(69.25050282,330.37583496)
\lineto(69.37050282,330.34583496)
\curveto(69.42049828,330.3458336)(69.46549824,330.35083359)(69.50550282,330.36083496)
\lineto(69.62550282,330.36083496)
\curveto(69.705498,330.38083356)(69.78549792,330.39583355)(69.86550282,330.40583496)
\curveto(69.94549776,330.41583353)(70.02049768,330.43583351)(70.09050282,330.46583496)
\curveto(70.35049735,330.56583338)(70.56049714,330.70083324)(70.72050282,330.87083496)
\curveto(70.88049682,331.0408329)(71.01549669,331.25083269)(71.12550282,331.50083496)
\curveto(71.16549654,331.60083234)(71.19549651,331.70083224)(71.21550282,331.80083496)
\curveto(71.23549647,331.90083204)(71.26049644,332.00583194)(71.29050282,332.11583496)
\curveto(71.3004964,332.15583179)(71.3054964,332.19083175)(71.30550282,332.22083496)
\curveto(71.3054964,332.26083168)(71.31049639,332.30083164)(71.32050282,332.34083496)
\lineto(71.32050282,332.47583496)
\curveto(71.32049638,332.52583142)(71.32549638,332.57583137)(71.33550282,332.62583496)
}
}
{
\newrgbcolor{curcolor}{0 0 0}
\pscustom[linestyle=none,fillstyle=solid,fillcolor=curcolor]
{
\newpath
\moveto(79.26042469,336.91583496)
\curveto(79.37041938,336.91582703)(79.46541928,336.90582704)(79.54542469,336.88583496)
\curveto(79.63541911,336.86582708)(79.70541904,336.82082712)(79.75542469,336.75083496)
\curveto(79.81541893,336.67082727)(79.8454189,336.53082741)(79.84542469,336.33083496)
\lineto(79.84542469,335.82083496)
\lineto(79.84542469,335.44583496)
\curveto(79.85541889,335.30582864)(79.84041891,335.19582875)(79.80042469,335.11583496)
\curveto(79.76041899,335.0458289)(79.70041905,335.00082894)(79.62042469,334.98083496)
\curveto(79.5504192,334.96082898)(79.46541928,334.95082899)(79.36542469,334.95083496)
\curveto(79.27541947,334.95082899)(79.17541957,334.95582899)(79.06542469,334.96583496)
\curveto(78.96541978,334.97582897)(78.87041988,334.97082897)(78.78042469,334.95083496)
\curveto(78.71042004,334.93082901)(78.64042011,334.91582903)(78.57042469,334.90583496)
\curveto(78.50042025,334.90582904)(78.43542031,334.89582905)(78.37542469,334.87583496)
\curveto(78.21542053,334.82582912)(78.05542069,334.75082919)(77.89542469,334.65083496)
\curveto(77.73542101,334.56082938)(77.61042114,334.45582949)(77.52042469,334.33583496)
\curveto(77.47042128,334.25582969)(77.41542133,334.17082977)(77.35542469,334.08083496)
\curveto(77.30542144,334.00082994)(77.25542149,333.91583003)(77.20542469,333.82583496)
\curveto(77.17542157,333.7458302)(77.1454216,333.66083028)(77.11542469,333.57083496)
\lineto(77.05542469,333.33083496)
\curveto(77.03542171,333.26083068)(77.02542172,333.18583076)(77.02542469,333.10583496)
\curveto(77.02542172,333.03583091)(77.01542173,332.96583098)(76.99542469,332.89583496)
\curveto(76.98542176,332.85583109)(76.98042177,332.81583113)(76.98042469,332.77583496)
\curveto(76.99042176,332.7458312)(76.99042176,332.71583123)(76.98042469,332.68583496)
\lineto(76.98042469,332.44583496)
\curveto(76.96042179,332.37583157)(76.95542179,332.29583165)(76.96542469,332.20583496)
\curveto(76.97542177,332.12583182)(76.98042177,332.0458319)(76.98042469,331.96583496)
\lineto(76.98042469,331.00583496)
\lineto(76.98042469,329.73083496)
\curveto(76.98042177,329.60083434)(76.97542177,329.48083446)(76.96542469,329.37083496)
\curveto(76.95542179,329.26083468)(76.92542182,329.17083477)(76.87542469,329.10083496)
\curveto(76.85542189,329.07083487)(76.82042193,329.0458349)(76.77042469,329.02583496)
\curveto(76.73042202,329.01583493)(76.68542206,329.00583494)(76.63542469,328.99583496)
\lineto(76.56042469,328.99583496)
\curveto(76.51042224,328.98583496)(76.45542229,328.98083496)(76.39542469,328.98083496)
\lineto(76.23042469,328.98083496)
\lineto(75.58542469,328.98083496)
\curveto(75.52542322,328.99083495)(75.46042329,328.99583495)(75.39042469,328.99583496)
\lineto(75.19542469,328.99583496)
\curveto(75.1454236,329.01583493)(75.09542365,329.03083491)(75.04542469,329.04083496)
\curveto(74.99542375,329.06083488)(74.96042379,329.09583485)(74.94042469,329.14583496)
\curveto(74.90042385,329.19583475)(74.87542387,329.26583468)(74.86542469,329.35583496)
\lineto(74.86542469,329.65583496)
\lineto(74.86542469,330.67583496)
\lineto(74.86542469,334.90583496)
\lineto(74.86542469,336.01583496)
\lineto(74.86542469,336.30083496)
\curveto(74.86542388,336.40082754)(74.88542386,336.48082746)(74.92542469,336.54083496)
\curveto(74.97542377,336.62082732)(75.0504237,336.67082727)(75.15042469,336.69083496)
\curveto(75.2504235,336.71082723)(75.37042338,336.72082722)(75.51042469,336.72083496)
\lineto(76.27542469,336.72083496)
\curveto(76.39542235,336.72082722)(76.50042225,336.71082723)(76.59042469,336.69083496)
\curveto(76.68042207,336.68082726)(76.750422,336.63582731)(76.80042469,336.55583496)
\curveto(76.83042192,336.50582744)(76.8454219,336.43582751)(76.84542469,336.34583496)
\lineto(76.87542469,336.07583496)
\curveto(76.88542186,335.99582795)(76.90042185,335.92082802)(76.92042469,335.85083496)
\curveto(76.9504218,335.78082816)(77.00042175,335.7458282)(77.07042469,335.74583496)
\curveto(77.09042166,335.76582818)(77.11042164,335.77582817)(77.13042469,335.77583496)
\curveto(77.1504216,335.77582817)(77.17042158,335.78582816)(77.19042469,335.80583496)
\curveto(77.2504215,335.85582809)(77.30042145,335.91082803)(77.34042469,335.97083496)
\curveto(77.39042136,336.0408279)(77.4504213,336.10082784)(77.52042469,336.15083496)
\curveto(77.56042119,336.18082776)(77.59542115,336.21082773)(77.62542469,336.24083496)
\curveto(77.65542109,336.28082766)(77.69042106,336.31582763)(77.73042469,336.34583496)
\lineto(78.00042469,336.52583496)
\curveto(78.10042065,336.58582736)(78.20042055,336.6408273)(78.30042469,336.69083496)
\curveto(78.40042035,336.73082721)(78.50042025,336.76582718)(78.60042469,336.79583496)
\lineto(78.93042469,336.88583496)
\curveto(78.96041979,336.89582705)(79.01541973,336.89582705)(79.09542469,336.88583496)
\curveto(79.18541956,336.88582706)(79.24041951,336.89582705)(79.26042469,336.91583496)
}
}
{
\newrgbcolor{curcolor}{0 0 0}
\pscustom[linestyle=none,fillstyle=solid,fillcolor=curcolor]
{
\newpath
\moveto(87.76683094,332.92583496)
\curveto(87.78682278,332.8458311)(87.78682278,332.75583119)(87.76683094,332.65583496)
\curveto(87.74682282,332.55583139)(87.71182285,332.49083145)(87.66183094,332.46083496)
\curveto(87.61182295,332.42083152)(87.53682303,332.39083155)(87.43683094,332.37083496)
\curveto(87.34682322,332.36083158)(87.24182332,332.35083159)(87.12183094,332.34083496)
\lineto(86.77683094,332.34083496)
\curveto(86.6668239,332.35083159)(86.566824,332.35583159)(86.47683094,332.35583496)
\lineto(82.81683094,332.35583496)
\lineto(82.60683094,332.35583496)
\curveto(82.54682802,332.35583159)(82.49182807,332.3458316)(82.44183094,332.32583496)
\curveto(82.3618282,332.28583166)(82.31182825,332.2458317)(82.29183094,332.20583496)
\curveto(82.27182829,332.18583176)(82.25182831,332.1458318)(82.23183094,332.08583496)
\curveto(82.21182835,332.03583191)(82.20682836,331.98583196)(82.21683094,331.93583496)
\curveto(82.23682833,331.87583207)(82.24682832,331.81583213)(82.24683094,331.75583496)
\curveto(82.25682831,331.70583224)(82.27182829,331.65083229)(82.29183094,331.59083496)
\curveto(82.37182819,331.35083259)(82.4668281,331.15083279)(82.57683094,330.99083496)
\curveto(82.69682787,330.8408331)(82.85682771,330.70583324)(83.05683094,330.58583496)
\curveto(83.13682743,330.53583341)(83.21682735,330.50083344)(83.29683094,330.48083496)
\curveto(83.38682718,330.47083347)(83.47682709,330.45083349)(83.56683094,330.42083496)
\curveto(83.64682692,330.40083354)(83.75682681,330.38583356)(83.89683094,330.37583496)
\curveto(84.03682653,330.36583358)(84.15682641,330.37083357)(84.25683094,330.39083496)
\lineto(84.39183094,330.39083496)
\curveto(84.49182607,330.41083353)(84.58182598,330.43083351)(84.66183094,330.45083496)
\curveto(84.75182581,330.48083346)(84.83682573,330.51083343)(84.91683094,330.54083496)
\curveto(85.01682555,330.59083335)(85.12682544,330.65583329)(85.24683094,330.73583496)
\curveto(85.37682519,330.81583313)(85.47182509,330.89583305)(85.53183094,330.97583496)
\curveto(85.58182498,331.0458329)(85.63182493,331.11083283)(85.68183094,331.17083496)
\curveto(85.74182482,331.2408327)(85.81182475,331.29083265)(85.89183094,331.32083496)
\curveto(85.99182457,331.37083257)(86.11682445,331.39083255)(86.26683094,331.38083496)
\lineto(86.70183094,331.38083496)
\lineto(86.88183094,331.38083496)
\curveto(86.95182361,331.39083255)(87.01182355,331.38583256)(87.06183094,331.36583496)
\lineto(87.21183094,331.36583496)
\curveto(87.31182325,331.3458326)(87.38182318,331.32083262)(87.42183094,331.29083496)
\curveto(87.4618231,331.27083267)(87.48182308,331.22583272)(87.48183094,331.15583496)
\curveto(87.49182307,331.08583286)(87.48682308,331.02583292)(87.46683094,330.97583496)
\curveto(87.41682315,330.83583311)(87.3618232,330.71083323)(87.30183094,330.60083496)
\curveto(87.24182332,330.49083345)(87.17182339,330.38083356)(87.09183094,330.27083496)
\curveto(86.87182369,329.940834)(86.62182394,329.67583427)(86.34183094,329.47583496)
\curveto(86.0618245,329.27583467)(85.71182485,329.10583484)(85.29183094,328.96583496)
\curveto(85.18182538,328.92583502)(85.07182549,328.90083504)(84.96183094,328.89083496)
\curveto(84.85182571,328.88083506)(84.73682583,328.86083508)(84.61683094,328.83083496)
\curveto(84.57682599,328.82083512)(84.53182603,328.82083512)(84.48183094,328.83083496)
\curveto(84.44182612,328.83083511)(84.40182616,328.82583512)(84.36183094,328.81583496)
\lineto(84.19683094,328.81583496)
\curveto(84.14682642,328.79583515)(84.08682648,328.79083515)(84.01683094,328.80083496)
\curveto(83.95682661,328.80083514)(83.90182666,328.80583514)(83.85183094,328.81583496)
\curveto(83.77182679,328.82583512)(83.70182686,328.82583512)(83.64183094,328.81583496)
\curveto(83.58182698,328.80583514)(83.51682705,328.81083513)(83.44683094,328.83083496)
\curveto(83.39682717,328.85083509)(83.34182722,328.86083508)(83.28183094,328.86083496)
\curveto(83.22182734,328.86083508)(83.1668274,328.87083507)(83.11683094,328.89083496)
\curveto(83.00682756,328.91083503)(82.89682767,328.93583501)(82.78683094,328.96583496)
\curveto(82.67682789,328.98583496)(82.57682799,329.02083492)(82.48683094,329.07083496)
\curveto(82.37682819,329.11083483)(82.27182829,329.1458348)(82.17183094,329.17583496)
\curveto(82.08182848,329.21583473)(81.99682857,329.26083468)(81.91683094,329.31083496)
\curveto(81.59682897,329.51083443)(81.31182925,329.7408342)(81.06183094,330.00083496)
\curveto(80.81182975,330.27083367)(80.60682996,330.58083336)(80.44683094,330.93083496)
\curveto(80.39683017,331.0408329)(80.35683021,331.15083279)(80.32683094,331.26083496)
\curveto(80.29683027,331.38083256)(80.25683031,331.50083244)(80.20683094,331.62083496)
\curveto(80.19683037,331.66083228)(80.19183037,331.69583225)(80.19183094,331.72583496)
\curveto(80.19183037,331.76583218)(80.18683038,331.80583214)(80.17683094,331.84583496)
\curveto(80.13683043,331.96583198)(80.11183045,332.09583185)(80.10183094,332.23583496)
\lineto(80.07183094,332.65583496)
\curveto(80.07183049,332.70583124)(80.0668305,332.76083118)(80.05683094,332.82083496)
\curveto(80.05683051,332.88083106)(80.0618305,332.93583101)(80.07183094,332.98583496)
\lineto(80.07183094,333.16583496)
\lineto(80.11683094,333.52583496)
\curveto(80.15683041,333.69583025)(80.19183037,333.86083008)(80.22183094,334.02083496)
\curveto(80.25183031,334.18082976)(80.29683027,334.33082961)(80.35683094,334.47083496)
\curveto(80.78682978,335.51082843)(81.51682905,336.2458277)(82.54683094,336.67583496)
\curveto(82.68682788,336.73582721)(82.82682774,336.77582717)(82.96683094,336.79583496)
\curveto(83.11682745,336.82582712)(83.27182729,336.86082708)(83.43183094,336.90083496)
\curveto(83.51182705,336.91082703)(83.58682698,336.91582703)(83.65683094,336.91583496)
\curveto(83.72682684,336.91582703)(83.80182676,336.92082702)(83.88183094,336.93083496)
\curveto(84.39182617,336.940827)(84.82682574,336.88082706)(85.18683094,336.75083496)
\curveto(85.55682501,336.63082731)(85.88682468,336.47082747)(86.17683094,336.27083496)
\curveto(86.2668243,336.21082773)(86.35682421,336.1408278)(86.44683094,336.06083496)
\curveto(86.53682403,335.99082795)(86.61682395,335.91582803)(86.68683094,335.83583496)
\curveto(86.71682385,335.78582816)(86.75682381,335.7458282)(86.80683094,335.71583496)
\curveto(86.88682368,335.60582834)(86.9618236,335.49082845)(87.03183094,335.37083496)
\curveto(87.10182346,335.26082868)(87.17682339,335.1458288)(87.25683094,335.02583496)
\curveto(87.30682326,334.93582901)(87.34682322,334.8408291)(87.37683094,334.74083496)
\curveto(87.41682315,334.65082929)(87.45682311,334.55082939)(87.49683094,334.44083496)
\curveto(87.54682302,334.31082963)(87.58682298,334.17582977)(87.61683094,334.03583496)
\curveto(87.64682292,333.89583005)(87.68182288,333.75583019)(87.72183094,333.61583496)
\curveto(87.74182282,333.53583041)(87.74682282,333.4458305)(87.73683094,333.34583496)
\curveto(87.73682283,333.25583069)(87.74682282,333.17083077)(87.76683094,333.09083496)
\lineto(87.76683094,332.92583496)
\moveto(85.51683094,333.81083496)
\curveto(85.58682498,333.91083003)(85.59182497,334.03082991)(85.53183094,334.17083496)
\curveto(85.48182508,334.32082962)(85.44182512,334.43082951)(85.41183094,334.50083496)
\curveto(85.27182529,334.77082917)(85.08682548,334.97582897)(84.85683094,335.11583496)
\curveto(84.62682594,335.26582868)(84.30682626,335.3458286)(83.89683094,335.35583496)
\curveto(83.8668267,335.33582861)(83.83182673,335.33082861)(83.79183094,335.34083496)
\curveto(83.75182681,335.35082859)(83.71682685,335.35082859)(83.68683094,335.34083496)
\curveto(83.63682693,335.32082862)(83.58182698,335.30582864)(83.52183094,335.29583496)
\curveto(83.4618271,335.29582865)(83.40682716,335.28582866)(83.35683094,335.26583496)
\curveto(82.91682765,335.12582882)(82.59182797,334.85082909)(82.38183094,334.44083496)
\curveto(82.3618282,334.40082954)(82.33682823,334.3458296)(82.30683094,334.27583496)
\curveto(82.28682828,334.21582973)(82.27182829,334.15082979)(82.26183094,334.08083496)
\curveto(82.25182831,334.02082992)(82.25182831,333.96082998)(82.26183094,333.90083496)
\curveto(82.28182828,333.8408301)(82.31682825,333.79083015)(82.36683094,333.75083496)
\curveto(82.44682812,333.70083024)(82.55682801,333.67583027)(82.69683094,333.67583496)
\lineto(83.10183094,333.67583496)
\lineto(84.76683094,333.67583496)
\lineto(85.20183094,333.67583496)
\curveto(85.3618252,333.68583026)(85.4668251,333.73083021)(85.51683094,333.81083496)
}
}
{
\newrgbcolor{curcolor}{0 0 0}
\pscustom[linestyle=none,fillstyle=solid,fillcolor=curcolor]
{
\newpath
\moveto(91.98511219,336.93083496)
\curveto(92.73510769,336.95082699)(93.38510704,336.86582708)(93.93511219,336.67583496)
\curveto(94.49510593,336.49582745)(94.92010551,336.18082776)(95.21011219,335.73083496)
\curveto(95.28010515,335.62082832)(95.34010509,335.50582844)(95.39011219,335.38583496)
\curveto(95.45010498,335.27582867)(95.50010493,335.15082879)(95.54011219,335.01083496)
\curveto(95.56010487,334.95082899)(95.57010486,334.88582906)(95.57011219,334.81583496)
\curveto(95.57010486,334.7458292)(95.56010487,334.68582926)(95.54011219,334.63583496)
\curveto(95.50010493,334.57582937)(95.44510498,334.53582941)(95.37511219,334.51583496)
\curveto(95.3251051,334.49582945)(95.26510516,334.48582946)(95.19511219,334.48583496)
\lineto(94.98511219,334.48583496)
\lineto(94.32511219,334.48583496)
\curveto(94.25510617,334.48582946)(94.18510624,334.48082946)(94.11511219,334.47083496)
\curveto(94.04510638,334.47082947)(93.98010645,334.48082946)(93.92011219,334.50083496)
\curveto(93.82010661,334.52082942)(93.74510668,334.56082938)(93.69511219,334.62083496)
\curveto(93.64510678,334.68082926)(93.60010683,334.7408292)(93.56011219,334.80083496)
\lineto(93.44011219,335.01083496)
\curveto(93.41010702,335.09082885)(93.36010707,335.15582879)(93.29011219,335.20583496)
\curveto(93.19010724,335.28582866)(93.09010734,335.3458286)(92.99011219,335.38583496)
\curveto(92.90010753,335.42582852)(92.78510764,335.46082848)(92.64511219,335.49083496)
\curveto(92.57510785,335.51082843)(92.47010796,335.52582842)(92.33011219,335.53583496)
\curveto(92.20010823,335.5458284)(92.10010833,335.5408284)(92.03011219,335.52083496)
\lineto(91.92511219,335.52083496)
\lineto(91.77511219,335.49083496)
\curveto(91.73510869,335.49082845)(91.69010874,335.48582846)(91.64011219,335.47583496)
\curveto(91.47010896,335.42582852)(91.3301091,335.35582859)(91.22011219,335.26583496)
\curveto(91.12010931,335.18582876)(91.05010938,335.06082888)(91.01011219,334.89083496)
\curveto(90.99010944,334.82082912)(90.99010944,334.75582919)(91.01011219,334.69583496)
\curveto(91.0301094,334.63582931)(91.05010938,334.58582936)(91.07011219,334.54583496)
\curveto(91.14010929,334.42582952)(91.22010921,334.33082961)(91.31011219,334.26083496)
\curveto(91.41010902,334.19082975)(91.5251089,334.13082981)(91.65511219,334.08083496)
\curveto(91.84510858,334.00082994)(92.05010838,333.93083001)(92.27011219,333.87083496)
\lineto(92.96011219,333.72083496)
\curveto(93.20010723,333.68083026)(93.430107,333.63083031)(93.65011219,333.57083496)
\curveto(93.88010655,333.52083042)(94.09510633,333.45583049)(94.29511219,333.37583496)
\curveto(94.38510604,333.33583061)(94.47010596,333.30083064)(94.55011219,333.27083496)
\curveto(94.64010579,333.25083069)(94.7251057,333.21583073)(94.80511219,333.16583496)
\curveto(94.99510543,333.0458309)(95.16510526,332.91583103)(95.31511219,332.77583496)
\curveto(95.47510495,332.63583131)(95.60010483,332.46083148)(95.69011219,332.25083496)
\curveto(95.72010471,332.18083176)(95.74510468,332.11083183)(95.76511219,332.04083496)
\curveto(95.78510464,331.97083197)(95.80510462,331.89583205)(95.82511219,331.81583496)
\curveto(95.83510459,331.75583219)(95.84010459,331.66083228)(95.84011219,331.53083496)
\curveto(95.85010458,331.41083253)(95.85010458,331.31583263)(95.84011219,331.24583496)
\lineto(95.84011219,331.17083496)
\curveto(95.82010461,331.11083283)(95.80510462,331.05083289)(95.79511219,330.99083496)
\curveto(95.79510463,330.940833)(95.79010464,330.89083305)(95.78011219,330.84083496)
\curveto(95.71010472,330.5408334)(95.60010483,330.27583367)(95.45011219,330.04583496)
\curveto(95.29010514,329.80583414)(95.09510533,329.61083433)(94.86511219,329.46083496)
\curveto(94.63510579,329.31083463)(94.37510605,329.18083476)(94.08511219,329.07083496)
\curveto(93.97510645,329.02083492)(93.85510657,328.98583496)(93.72511219,328.96583496)
\curveto(93.60510682,328.945835)(93.48510694,328.92083502)(93.36511219,328.89083496)
\curveto(93.27510715,328.87083507)(93.18010725,328.86083508)(93.08011219,328.86083496)
\curveto(92.99010744,328.85083509)(92.90010753,328.83583511)(92.81011219,328.81583496)
\lineto(92.54011219,328.81583496)
\curveto(92.48010795,328.79583515)(92.37510805,328.78583516)(92.22511219,328.78583496)
\curveto(92.08510834,328.78583516)(91.98510844,328.79583515)(91.92511219,328.81583496)
\curveto(91.89510853,328.81583513)(91.86010857,328.82083512)(91.82011219,328.83083496)
\lineto(91.71511219,328.83083496)
\curveto(91.59510883,328.85083509)(91.47510895,328.86583508)(91.35511219,328.87583496)
\curveto(91.23510919,328.88583506)(91.12010931,328.90583504)(91.01011219,328.93583496)
\curveto(90.62010981,329.0458349)(90.27511015,329.17083477)(89.97511219,329.31083496)
\curveto(89.67511075,329.46083448)(89.42011101,329.68083426)(89.21011219,329.97083496)
\curveto(89.07011136,330.16083378)(88.95011148,330.38083356)(88.85011219,330.63083496)
\curveto(88.8301116,330.69083325)(88.81011162,330.77083317)(88.79011219,330.87083496)
\curveto(88.77011166,330.92083302)(88.75511167,330.99083295)(88.74511219,331.08083496)
\curveto(88.73511169,331.17083277)(88.74011169,331.2458327)(88.76011219,331.30583496)
\curveto(88.79011164,331.37583257)(88.84011159,331.42583252)(88.91011219,331.45583496)
\curveto(88.96011147,331.47583247)(89.02011141,331.48583246)(89.09011219,331.48583496)
\lineto(89.31511219,331.48583496)
\lineto(90.02011219,331.48583496)
\lineto(90.26011219,331.48583496)
\curveto(90.34011009,331.48583246)(90.41011002,331.47583247)(90.47011219,331.45583496)
\curveto(90.58010985,331.41583253)(90.65010978,331.35083259)(90.68011219,331.26083496)
\curveto(90.72010971,331.17083277)(90.76510966,331.07583287)(90.81511219,330.97583496)
\curveto(90.83510959,330.92583302)(90.87010956,330.86083308)(90.92011219,330.78083496)
\curveto(90.98010945,330.70083324)(91.0301094,330.65083329)(91.07011219,330.63083496)
\curveto(91.19010924,330.53083341)(91.30510912,330.45083349)(91.41511219,330.39083496)
\curveto(91.5251089,330.3408336)(91.66510876,330.29083365)(91.83511219,330.24083496)
\curveto(91.88510854,330.22083372)(91.93510849,330.21083373)(91.98511219,330.21083496)
\curveto(92.03510839,330.22083372)(92.08510834,330.22083372)(92.13511219,330.21083496)
\curveto(92.21510821,330.19083375)(92.30010813,330.18083376)(92.39011219,330.18083496)
\curveto(92.49010794,330.19083375)(92.57510785,330.20583374)(92.64511219,330.22583496)
\curveto(92.69510773,330.23583371)(92.74010769,330.2408337)(92.78011219,330.24083496)
\curveto(92.8301076,330.2408337)(92.88010755,330.25083369)(92.93011219,330.27083496)
\curveto(93.07010736,330.32083362)(93.19510723,330.38083356)(93.30511219,330.45083496)
\curveto(93.425107,330.52083342)(93.52010691,330.61083333)(93.59011219,330.72083496)
\curveto(93.64010679,330.80083314)(93.68010675,330.92583302)(93.71011219,331.09583496)
\curveto(93.7301067,331.16583278)(93.7301067,331.23083271)(93.71011219,331.29083496)
\curveto(93.69010674,331.35083259)(93.67010676,331.40083254)(93.65011219,331.44083496)
\curveto(93.58010685,331.58083236)(93.49010694,331.68583226)(93.38011219,331.75583496)
\curveto(93.28010715,331.82583212)(93.16010727,331.89083205)(93.02011219,331.95083496)
\curveto(92.8301076,332.03083191)(92.6301078,332.09583185)(92.42011219,332.14583496)
\curveto(92.21010822,332.19583175)(92.00010843,332.25083169)(91.79011219,332.31083496)
\curveto(91.71010872,332.33083161)(91.6251088,332.3458316)(91.53511219,332.35583496)
\curveto(91.45510897,332.36583158)(91.37510905,332.38083156)(91.29511219,332.40083496)
\curveto(90.97510945,332.49083145)(90.67010976,332.57583137)(90.38011219,332.65583496)
\curveto(90.09011034,332.7458312)(89.8251106,332.87583107)(89.58511219,333.04583496)
\curveto(89.30511112,333.2458307)(89.10011133,333.51583043)(88.97011219,333.85583496)
\curveto(88.95011148,333.92583002)(88.9301115,334.02082992)(88.91011219,334.14083496)
\curveto(88.89011154,334.21082973)(88.87511155,334.29582965)(88.86511219,334.39583496)
\curveto(88.85511157,334.49582945)(88.86011157,334.58582936)(88.88011219,334.66583496)
\curveto(88.90011153,334.71582923)(88.90511152,334.75582919)(88.89511219,334.78583496)
\curveto(88.88511154,334.82582912)(88.89011154,334.87082907)(88.91011219,334.92083496)
\curveto(88.9301115,335.03082891)(88.95011148,335.13082881)(88.97011219,335.22083496)
\curveto(89.00011143,335.32082862)(89.03511139,335.41582853)(89.07511219,335.50583496)
\curveto(89.20511122,335.79582815)(89.38511104,336.03082791)(89.61511219,336.21083496)
\curveto(89.84511058,336.39082755)(90.10511032,336.53582741)(90.39511219,336.64583496)
\curveto(90.50510992,336.69582725)(90.62010981,336.73082721)(90.74011219,336.75083496)
\curveto(90.86010957,336.78082716)(90.98510944,336.81082713)(91.11511219,336.84083496)
\curveto(91.17510925,336.86082708)(91.23510919,336.87082707)(91.29511219,336.87083496)
\lineto(91.47511219,336.90083496)
\curveto(91.55510887,336.91082703)(91.64010879,336.91582703)(91.73011219,336.91583496)
\curveto(91.82010861,336.91582703)(91.90510852,336.92082702)(91.98511219,336.93083496)
}
}
{
\newrgbcolor{curcolor}{0 0 0}
\pscustom[linestyle=none,fillstyle=solid,fillcolor=curcolor]
{
}
}
{
\newrgbcolor{curcolor}{0 0 0}
\pscustom[linestyle=none,fillstyle=solid,fillcolor=curcolor]
{
\newpath
\moveto(108.82190907,329.83583496)
\lineto(108.82190907,329.41583496)
\curveto(108.8219007,329.28583466)(108.79190073,329.18083476)(108.73190907,329.10083496)
\curveto(108.68190084,329.05083489)(108.6169009,329.01583493)(108.53690907,328.99583496)
\curveto(108.45690106,328.98583496)(108.36690115,328.98083496)(108.26690907,328.98083496)
\lineto(107.44190907,328.98083496)
\lineto(107.15690907,328.98083496)
\curveto(107.07690244,328.99083495)(107.01190251,329.01583493)(106.96190907,329.05583496)
\curveto(106.89190263,329.10583484)(106.85190267,329.17083477)(106.84190907,329.25083496)
\curveto(106.83190269,329.33083461)(106.81190271,329.41083453)(106.78190907,329.49083496)
\curveto(106.76190276,329.51083443)(106.74190278,329.52583442)(106.72190907,329.53583496)
\curveto(106.71190281,329.55583439)(106.69690282,329.57583437)(106.67690907,329.59583496)
\curveto(106.56690295,329.59583435)(106.48690303,329.57083437)(106.43690907,329.52083496)
\lineto(106.28690907,329.37083496)
\curveto(106.2169033,329.32083462)(106.15190337,329.27583467)(106.09190907,329.23583496)
\curveto(106.03190349,329.20583474)(105.96690355,329.16583478)(105.89690907,329.11583496)
\curveto(105.85690366,329.09583485)(105.81190371,329.07583487)(105.76190907,329.05583496)
\curveto(105.7219038,329.03583491)(105.67690384,329.01583493)(105.62690907,328.99583496)
\curveto(105.48690403,328.945835)(105.33690418,328.90083504)(105.17690907,328.86083496)
\curveto(105.12690439,328.8408351)(105.08190444,328.83083511)(105.04190907,328.83083496)
\curveto(105.00190452,328.83083511)(104.96190456,328.82583512)(104.92190907,328.81583496)
\lineto(104.78690907,328.81583496)
\curveto(104.75690476,328.80583514)(104.7169048,328.80083514)(104.66690907,328.80083496)
\lineto(104.53190907,328.80083496)
\curveto(104.47190505,328.78083516)(104.38190514,328.77583517)(104.26190907,328.78583496)
\curveto(104.14190538,328.78583516)(104.05690546,328.79583515)(104.00690907,328.81583496)
\curveto(103.93690558,328.83583511)(103.87190565,328.8458351)(103.81190907,328.84583496)
\curveto(103.76190576,328.83583511)(103.70690581,328.8408351)(103.64690907,328.86083496)
\lineto(103.28690907,328.98083496)
\curveto(103.17690634,329.01083493)(103.06690645,329.05083489)(102.95690907,329.10083496)
\curveto(102.60690691,329.25083469)(102.29190723,329.48083446)(102.01190907,329.79083496)
\curveto(101.74190778,330.11083383)(101.52690799,330.4458335)(101.36690907,330.79583496)
\curveto(101.3169082,330.90583304)(101.27690824,331.01083293)(101.24690907,331.11083496)
\curveto(101.2169083,331.22083272)(101.18190834,331.33083261)(101.14190907,331.44083496)
\curveto(101.13190839,331.48083246)(101.12690839,331.51583243)(101.12690907,331.54583496)
\curveto(101.12690839,331.58583236)(101.1169084,331.63083231)(101.09690907,331.68083496)
\curveto(101.07690844,331.76083218)(101.05690846,331.8458321)(101.03690907,331.93583496)
\curveto(101.02690849,332.03583191)(101.01190851,332.13583181)(100.99190907,332.23583496)
\curveto(100.98190854,332.26583168)(100.97690854,332.30083164)(100.97690907,332.34083496)
\curveto(100.98690853,332.38083156)(100.98690853,332.41583153)(100.97690907,332.44583496)
\lineto(100.97690907,332.58083496)
\curveto(100.97690854,332.63083131)(100.97190855,332.68083126)(100.96190907,332.73083496)
\curveto(100.95190857,332.78083116)(100.94690857,332.83583111)(100.94690907,332.89583496)
\curveto(100.94690857,332.96583098)(100.95190857,333.02083092)(100.96190907,333.06083496)
\curveto(100.97190855,333.11083083)(100.97690854,333.15583079)(100.97690907,333.19583496)
\lineto(100.97690907,333.34583496)
\curveto(100.98690853,333.39583055)(100.98690853,333.4408305)(100.97690907,333.48083496)
\curveto(100.97690854,333.53083041)(100.98690853,333.58083036)(101.00690907,333.63083496)
\curveto(101.02690849,333.7408302)(101.04190848,333.8458301)(101.05190907,333.94583496)
\curveto(101.07190845,334.0458299)(101.09690842,334.1458298)(101.12690907,334.24583496)
\curveto(101.16690835,334.36582958)(101.20190832,334.48082946)(101.23190907,334.59083496)
\curveto(101.26190826,334.70082924)(101.30190822,334.81082913)(101.35190907,334.92083496)
\curveto(101.49190803,335.22082872)(101.66690785,335.50582844)(101.87690907,335.77583496)
\curveto(101.89690762,335.80582814)(101.9219076,335.83082811)(101.95190907,335.85083496)
\curveto(101.99190753,335.88082806)(102.0219075,335.91082803)(102.04190907,335.94083496)
\curveto(102.08190744,335.99082795)(102.1219074,336.03582791)(102.16190907,336.07583496)
\curveto(102.20190732,336.11582783)(102.24690727,336.15582779)(102.29690907,336.19583496)
\curveto(102.33690718,336.21582773)(102.37190715,336.2408277)(102.40190907,336.27083496)
\curveto(102.43190709,336.31082763)(102.46690705,336.3408276)(102.50690907,336.36083496)
\curveto(102.75690676,336.53082741)(103.04690647,336.67082727)(103.37690907,336.78083496)
\curveto(103.44690607,336.80082714)(103.516906,336.81582713)(103.58690907,336.82583496)
\curveto(103.66690585,336.83582711)(103.74690577,336.85082709)(103.82690907,336.87083496)
\curveto(103.89690562,336.89082705)(103.98690553,336.90082704)(104.09690907,336.90083496)
\curveto(104.20690531,336.91082703)(104.3169052,336.91582703)(104.42690907,336.91583496)
\curveto(104.53690498,336.91582703)(104.64190488,336.91082703)(104.74190907,336.90083496)
\curveto(104.85190467,336.89082705)(104.94190458,336.87582707)(105.01190907,336.85583496)
\curveto(105.16190436,336.80582714)(105.30690421,336.76082718)(105.44690907,336.72083496)
\curveto(105.58690393,336.68082726)(105.7169038,336.62582732)(105.83690907,336.55583496)
\curveto(105.90690361,336.50582744)(105.97190355,336.45582749)(106.03190907,336.40583496)
\curveto(106.09190343,336.36582758)(106.15690336,336.32082762)(106.22690907,336.27083496)
\curveto(106.26690325,336.2408277)(106.3219032,336.20082774)(106.39190907,336.15083496)
\curveto(106.47190305,336.10082784)(106.54690297,336.10082784)(106.61690907,336.15083496)
\curveto(106.65690286,336.17082777)(106.67690284,336.20582774)(106.67690907,336.25583496)
\curveto(106.67690284,336.30582764)(106.68690283,336.35582759)(106.70690907,336.40583496)
\lineto(106.70690907,336.55583496)
\curveto(106.7169028,336.58582736)(106.7219028,336.62082732)(106.72190907,336.66083496)
\lineto(106.72190907,336.78083496)
\lineto(106.72190907,338.82083496)
\curveto(106.7219028,338.93082501)(106.7169028,339.05082489)(106.70690907,339.18083496)
\curveto(106.70690281,339.32082462)(106.73190279,339.42582452)(106.78190907,339.49583496)
\curveto(106.8219027,339.57582437)(106.89690262,339.62582432)(107.00690907,339.64583496)
\curveto(107.02690249,339.65582429)(107.04690247,339.65582429)(107.06690907,339.64583496)
\curveto(107.08690243,339.6458243)(107.10690241,339.65082429)(107.12690907,339.66083496)
\lineto(108.19190907,339.66083496)
\curveto(108.31190121,339.66082428)(108.4219011,339.65582429)(108.52190907,339.64583496)
\curveto(108.6219009,339.63582431)(108.69690082,339.59582435)(108.74690907,339.52583496)
\curveto(108.79690072,339.4458245)(108.8219007,339.3408246)(108.82190907,339.21083496)
\lineto(108.82190907,338.85083496)
\lineto(108.82190907,329.83583496)
\moveto(106.78190907,332.77583496)
\curveto(106.79190273,332.81583113)(106.79190273,332.85583109)(106.78190907,332.89583496)
\lineto(106.78190907,333.03083496)
\curveto(106.78190274,333.13083081)(106.77690274,333.23083071)(106.76690907,333.33083496)
\curveto(106.75690276,333.43083051)(106.74190278,333.52083042)(106.72190907,333.60083496)
\curveto(106.70190282,333.71083023)(106.68190284,333.81083013)(106.66190907,333.90083496)
\curveto(106.65190287,333.99082995)(106.62690289,334.07582987)(106.58690907,334.15583496)
\curveto(106.44690307,334.51582943)(106.24190328,334.80082914)(105.97190907,335.01083496)
\curveto(105.71190381,335.22082872)(105.33190419,335.32582862)(104.83190907,335.32583496)
\curveto(104.77190475,335.32582862)(104.69190483,335.31582863)(104.59190907,335.29583496)
\curveto(104.51190501,335.27582867)(104.43690508,335.25582869)(104.36690907,335.23583496)
\curveto(104.30690521,335.22582872)(104.24690527,335.20582874)(104.18690907,335.17583496)
\curveto(103.9169056,335.06582888)(103.70690581,334.89582905)(103.55690907,334.66583496)
\curveto(103.40690611,334.43582951)(103.28690623,334.17582977)(103.19690907,333.88583496)
\curveto(103.16690635,333.78583016)(103.14690637,333.68583026)(103.13690907,333.58583496)
\curveto(103.12690639,333.48583046)(103.10690641,333.38083056)(103.07690907,333.27083496)
\lineto(103.07690907,333.06083496)
\curveto(103.05690646,332.97083097)(103.05190647,332.8458311)(103.06190907,332.68583496)
\curveto(103.07190645,332.53583141)(103.08690643,332.42583152)(103.10690907,332.35583496)
\lineto(103.10690907,332.26583496)
\curveto(103.1169064,332.2458317)(103.1219064,332.22583172)(103.12190907,332.20583496)
\curveto(103.14190638,332.12583182)(103.15690636,332.05083189)(103.16690907,331.98083496)
\curveto(103.18690633,331.91083203)(103.20690631,331.83583211)(103.22690907,331.75583496)
\curveto(103.39690612,331.23583271)(103.68690583,330.85083309)(104.09690907,330.60083496)
\curveto(104.22690529,330.51083343)(104.40690511,330.4408335)(104.63690907,330.39083496)
\curveto(104.67690484,330.38083356)(104.73690478,330.37583357)(104.81690907,330.37583496)
\curveto(104.84690467,330.36583358)(104.89190463,330.35583359)(104.95190907,330.34583496)
\curveto(105.0219045,330.3458336)(105.07690444,330.35083359)(105.11690907,330.36083496)
\curveto(105.19690432,330.38083356)(105.27690424,330.39583355)(105.35690907,330.40583496)
\curveto(105.43690408,330.41583353)(105.516904,330.43583351)(105.59690907,330.46583496)
\curveto(105.84690367,330.57583337)(106.04690347,330.71583323)(106.19690907,330.88583496)
\curveto(106.34690317,331.05583289)(106.47690304,331.27083267)(106.58690907,331.53083496)
\curveto(106.62690289,331.62083232)(106.65690286,331.71083223)(106.67690907,331.80083496)
\curveto(106.69690282,331.90083204)(106.7169028,332.00583194)(106.73690907,332.11583496)
\curveto(106.74690277,332.16583178)(106.74690277,332.21083173)(106.73690907,332.25083496)
\curveto(106.73690278,332.30083164)(106.74690277,332.35083159)(106.76690907,332.40083496)
\curveto(106.77690274,332.43083151)(106.78190274,332.46583148)(106.78190907,332.50583496)
\lineto(106.78190907,332.64083496)
\lineto(106.78190907,332.77583496)
}
}
{
\newrgbcolor{curcolor}{0 0 0}
\pscustom[linestyle=none,fillstyle=solid,fillcolor=curcolor]
{
\newpath
\moveto(117.76683094,332.92583496)
\curveto(117.78682278,332.8458311)(117.78682278,332.75583119)(117.76683094,332.65583496)
\curveto(117.74682282,332.55583139)(117.71182285,332.49083145)(117.66183094,332.46083496)
\curveto(117.61182295,332.42083152)(117.53682303,332.39083155)(117.43683094,332.37083496)
\curveto(117.34682322,332.36083158)(117.24182332,332.35083159)(117.12183094,332.34083496)
\lineto(116.77683094,332.34083496)
\curveto(116.6668239,332.35083159)(116.566824,332.35583159)(116.47683094,332.35583496)
\lineto(112.81683094,332.35583496)
\lineto(112.60683094,332.35583496)
\curveto(112.54682802,332.35583159)(112.49182807,332.3458316)(112.44183094,332.32583496)
\curveto(112.3618282,332.28583166)(112.31182825,332.2458317)(112.29183094,332.20583496)
\curveto(112.27182829,332.18583176)(112.25182831,332.1458318)(112.23183094,332.08583496)
\curveto(112.21182835,332.03583191)(112.20682836,331.98583196)(112.21683094,331.93583496)
\curveto(112.23682833,331.87583207)(112.24682832,331.81583213)(112.24683094,331.75583496)
\curveto(112.25682831,331.70583224)(112.27182829,331.65083229)(112.29183094,331.59083496)
\curveto(112.37182819,331.35083259)(112.4668281,331.15083279)(112.57683094,330.99083496)
\curveto(112.69682787,330.8408331)(112.85682771,330.70583324)(113.05683094,330.58583496)
\curveto(113.13682743,330.53583341)(113.21682735,330.50083344)(113.29683094,330.48083496)
\curveto(113.38682718,330.47083347)(113.47682709,330.45083349)(113.56683094,330.42083496)
\curveto(113.64682692,330.40083354)(113.75682681,330.38583356)(113.89683094,330.37583496)
\curveto(114.03682653,330.36583358)(114.15682641,330.37083357)(114.25683094,330.39083496)
\lineto(114.39183094,330.39083496)
\curveto(114.49182607,330.41083353)(114.58182598,330.43083351)(114.66183094,330.45083496)
\curveto(114.75182581,330.48083346)(114.83682573,330.51083343)(114.91683094,330.54083496)
\curveto(115.01682555,330.59083335)(115.12682544,330.65583329)(115.24683094,330.73583496)
\curveto(115.37682519,330.81583313)(115.47182509,330.89583305)(115.53183094,330.97583496)
\curveto(115.58182498,331.0458329)(115.63182493,331.11083283)(115.68183094,331.17083496)
\curveto(115.74182482,331.2408327)(115.81182475,331.29083265)(115.89183094,331.32083496)
\curveto(115.99182457,331.37083257)(116.11682445,331.39083255)(116.26683094,331.38083496)
\lineto(116.70183094,331.38083496)
\lineto(116.88183094,331.38083496)
\curveto(116.95182361,331.39083255)(117.01182355,331.38583256)(117.06183094,331.36583496)
\lineto(117.21183094,331.36583496)
\curveto(117.31182325,331.3458326)(117.38182318,331.32083262)(117.42183094,331.29083496)
\curveto(117.4618231,331.27083267)(117.48182308,331.22583272)(117.48183094,331.15583496)
\curveto(117.49182307,331.08583286)(117.48682308,331.02583292)(117.46683094,330.97583496)
\curveto(117.41682315,330.83583311)(117.3618232,330.71083323)(117.30183094,330.60083496)
\curveto(117.24182332,330.49083345)(117.17182339,330.38083356)(117.09183094,330.27083496)
\curveto(116.87182369,329.940834)(116.62182394,329.67583427)(116.34183094,329.47583496)
\curveto(116.0618245,329.27583467)(115.71182485,329.10583484)(115.29183094,328.96583496)
\curveto(115.18182538,328.92583502)(115.07182549,328.90083504)(114.96183094,328.89083496)
\curveto(114.85182571,328.88083506)(114.73682583,328.86083508)(114.61683094,328.83083496)
\curveto(114.57682599,328.82083512)(114.53182603,328.82083512)(114.48183094,328.83083496)
\curveto(114.44182612,328.83083511)(114.40182616,328.82583512)(114.36183094,328.81583496)
\lineto(114.19683094,328.81583496)
\curveto(114.14682642,328.79583515)(114.08682648,328.79083515)(114.01683094,328.80083496)
\curveto(113.95682661,328.80083514)(113.90182666,328.80583514)(113.85183094,328.81583496)
\curveto(113.77182679,328.82583512)(113.70182686,328.82583512)(113.64183094,328.81583496)
\curveto(113.58182698,328.80583514)(113.51682705,328.81083513)(113.44683094,328.83083496)
\curveto(113.39682717,328.85083509)(113.34182722,328.86083508)(113.28183094,328.86083496)
\curveto(113.22182734,328.86083508)(113.1668274,328.87083507)(113.11683094,328.89083496)
\curveto(113.00682756,328.91083503)(112.89682767,328.93583501)(112.78683094,328.96583496)
\curveto(112.67682789,328.98583496)(112.57682799,329.02083492)(112.48683094,329.07083496)
\curveto(112.37682819,329.11083483)(112.27182829,329.1458348)(112.17183094,329.17583496)
\curveto(112.08182848,329.21583473)(111.99682857,329.26083468)(111.91683094,329.31083496)
\curveto(111.59682897,329.51083443)(111.31182925,329.7408342)(111.06183094,330.00083496)
\curveto(110.81182975,330.27083367)(110.60682996,330.58083336)(110.44683094,330.93083496)
\curveto(110.39683017,331.0408329)(110.35683021,331.15083279)(110.32683094,331.26083496)
\curveto(110.29683027,331.38083256)(110.25683031,331.50083244)(110.20683094,331.62083496)
\curveto(110.19683037,331.66083228)(110.19183037,331.69583225)(110.19183094,331.72583496)
\curveto(110.19183037,331.76583218)(110.18683038,331.80583214)(110.17683094,331.84583496)
\curveto(110.13683043,331.96583198)(110.11183045,332.09583185)(110.10183094,332.23583496)
\lineto(110.07183094,332.65583496)
\curveto(110.07183049,332.70583124)(110.0668305,332.76083118)(110.05683094,332.82083496)
\curveto(110.05683051,332.88083106)(110.0618305,332.93583101)(110.07183094,332.98583496)
\lineto(110.07183094,333.16583496)
\lineto(110.11683094,333.52583496)
\curveto(110.15683041,333.69583025)(110.19183037,333.86083008)(110.22183094,334.02083496)
\curveto(110.25183031,334.18082976)(110.29683027,334.33082961)(110.35683094,334.47083496)
\curveto(110.78682978,335.51082843)(111.51682905,336.2458277)(112.54683094,336.67583496)
\curveto(112.68682788,336.73582721)(112.82682774,336.77582717)(112.96683094,336.79583496)
\curveto(113.11682745,336.82582712)(113.27182729,336.86082708)(113.43183094,336.90083496)
\curveto(113.51182705,336.91082703)(113.58682698,336.91582703)(113.65683094,336.91583496)
\curveto(113.72682684,336.91582703)(113.80182676,336.92082702)(113.88183094,336.93083496)
\curveto(114.39182617,336.940827)(114.82682574,336.88082706)(115.18683094,336.75083496)
\curveto(115.55682501,336.63082731)(115.88682468,336.47082747)(116.17683094,336.27083496)
\curveto(116.2668243,336.21082773)(116.35682421,336.1408278)(116.44683094,336.06083496)
\curveto(116.53682403,335.99082795)(116.61682395,335.91582803)(116.68683094,335.83583496)
\curveto(116.71682385,335.78582816)(116.75682381,335.7458282)(116.80683094,335.71583496)
\curveto(116.88682368,335.60582834)(116.9618236,335.49082845)(117.03183094,335.37083496)
\curveto(117.10182346,335.26082868)(117.17682339,335.1458288)(117.25683094,335.02583496)
\curveto(117.30682326,334.93582901)(117.34682322,334.8408291)(117.37683094,334.74083496)
\curveto(117.41682315,334.65082929)(117.45682311,334.55082939)(117.49683094,334.44083496)
\curveto(117.54682302,334.31082963)(117.58682298,334.17582977)(117.61683094,334.03583496)
\curveto(117.64682292,333.89583005)(117.68182288,333.75583019)(117.72183094,333.61583496)
\curveto(117.74182282,333.53583041)(117.74682282,333.4458305)(117.73683094,333.34583496)
\curveto(117.73682283,333.25583069)(117.74682282,333.17083077)(117.76683094,333.09083496)
\lineto(117.76683094,332.92583496)
\moveto(115.51683094,333.81083496)
\curveto(115.58682498,333.91083003)(115.59182497,334.03082991)(115.53183094,334.17083496)
\curveto(115.48182508,334.32082962)(115.44182512,334.43082951)(115.41183094,334.50083496)
\curveto(115.27182529,334.77082917)(115.08682548,334.97582897)(114.85683094,335.11583496)
\curveto(114.62682594,335.26582868)(114.30682626,335.3458286)(113.89683094,335.35583496)
\curveto(113.8668267,335.33582861)(113.83182673,335.33082861)(113.79183094,335.34083496)
\curveto(113.75182681,335.35082859)(113.71682685,335.35082859)(113.68683094,335.34083496)
\curveto(113.63682693,335.32082862)(113.58182698,335.30582864)(113.52183094,335.29583496)
\curveto(113.4618271,335.29582865)(113.40682716,335.28582866)(113.35683094,335.26583496)
\curveto(112.91682765,335.12582882)(112.59182797,334.85082909)(112.38183094,334.44083496)
\curveto(112.3618282,334.40082954)(112.33682823,334.3458296)(112.30683094,334.27583496)
\curveto(112.28682828,334.21582973)(112.27182829,334.15082979)(112.26183094,334.08083496)
\curveto(112.25182831,334.02082992)(112.25182831,333.96082998)(112.26183094,333.90083496)
\curveto(112.28182828,333.8408301)(112.31682825,333.79083015)(112.36683094,333.75083496)
\curveto(112.44682812,333.70083024)(112.55682801,333.67583027)(112.69683094,333.67583496)
\lineto(113.10183094,333.67583496)
\lineto(114.76683094,333.67583496)
\lineto(115.20183094,333.67583496)
\curveto(115.3618252,333.68583026)(115.4668251,333.73083021)(115.51683094,333.81083496)
}
}
{
\newrgbcolor{curcolor}{0 0 0}
\pscustom[linestyle=none,fillstyle=solid,fillcolor=curcolor]
{
}
}
{
\newrgbcolor{curcolor}{0 0 0}
\pscustom[linestyle=none,fillstyle=solid,fillcolor=curcolor]
{
\newpath
\moveto(126.14526844,336.93083496)
\curveto(126.89526394,336.95082699)(127.54526329,336.86582708)(128.09526844,336.67583496)
\curveto(128.65526218,336.49582745)(129.08026176,336.18082776)(129.37026844,335.73083496)
\curveto(129.4402614,335.62082832)(129.50026134,335.50582844)(129.55026844,335.38583496)
\curveto(129.61026123,335.27582867)(129.66026118,335.15082879)(129.70026844,335.01083496)
\curveto(129.72026112,334.95082899)(129.73026111,334.88582906)(129.73026844,334.81583496)
\curveto(129.73026111,334.7458292)(129.72026112,334.68582926)(129.70026844,334.63583496)
\curveto(129.66026118,334.57582937)(129.60526123,334.53582941)(129.53526844,334.51583496)
\curveto(129.48526135,334.49582945)(129.42526141,334.48582946)(129.35526844,334.48583496)
\lineto(129.14526844,334.48583496)
\lineto(128.48526844,334.48583496)
\curveto(128.41526242,334.48582946)(128.34526249,334.48082946)(128.27526844,334.47083496)
\curveto(128.20526263,334.47082947)(128.1402627,334.48082946)(128.08026844,334.50083496)
\curveto(127.98026286,334.52082942)(127.90526293,334.56082938)(127.85526844,334.62083496)
\curveto(127.80526303,334.68082926)(127.76026308,334.7408292)(127.72026844,334.80083496)
\lineto(127.60026844,335.01083496)
\curveto(127.57026327,335.09082885)(127.52026332,335.15582879)(127.45026844,335.20583496)
\curveto(127.35026349,335.28582866)(127.25026359,335.3458286)(127.15026844,335.38583496)
\curveto(127.06026378,335.42582852)(126.94526389,335.46082848)(126.80526844,335.49083496)
\curveto(126.7352641,335.51082843)(126.63026421,335.52582842)(126.49026844,335.53583496)
\curveto(126.36026448,335.5458284)(126.26026458,335.5408284)(126.19026844,335.52083496)
\lineto(126.08526844,335.52083496)
\lineto(125.93526844,335.49083496)
\curveto(125.89526494,335.49082845)(125.85026499,335.48582846)(125.80026844,335.47583496)
\curveto(125.63026521,335.42582852)(125.49026535,335.35582859)(125.38026844,335.26583496)
\curveto(125.28026556,335.18582876)(125.21026563,335.06082888)(125.17026844,334.89083496)
\curveto(125.15026569,334.82082912)(125.15026569,334.75582919)(125.17026844,334.69583496)
\curveto(125.19026565,334.63582931)(125.21026563,334.58582936)(125.23026844,334.54583496)
\curveto(125.30026554,334.42582952)(125.38026546,334.33082961)(125.47026844,334.26083496)
\curveto(125.57026527,334.19082975)(125.68526515,334.13082981)(125.81526844,334.08083496)
\curveto(126.00526483,334.00082994)(126.21026463,333.93083001)(126.43026844,333.87083496)
\lineto(127.12026844,333.72083496)
\curveto(127.36026348,333.68083026)(127.59026325,333.63083031)(127.81026844,333.57083496)
\curveto(128.0402628,333.52083042)(128.25526258,333.45583049)(128.45526844,333.37583496)
\curveto(128.54526229,333.33583061)(128.63026221,333.30083064)(128.71026844,333.27083496)
\curveto(128.80026204,333.25083069)(128.88526195,333.21583073)(128.96526844,333.16583496)
\curveto(129.15526168,333.0458309)(129.32526151,332.91583103)(129.47526844,332.77583496)
\curveto(129.6352612,332.63583131)(129.76026108,332.46083148)(129.85026844,332.25083496)
\curveto(129.88026096,332.18083176)(129.90526093,332.11083183)(129.92526844,332.04083496)
\curveto(129.94526089,331.97083197)(129.96526087,331.89583205)(129.98526844,331.81583496)
\curveto(129.99526084,331.75583219)(130.00026084,331.66083228)(130.00026844,331.53083496)
\curveto(130.01026083,331.41083253)(130.01026083,331.31583263)(130.00026844,331.24583496)
\lineto(130.00026844,331.17083496)
\curveto(129.98026086,331.11083283)(129.96526087,331.05083289)(129.95526844,330.99083496)
\curveto(129.95526088,330.940833)(129.95026089,330.89083305)(129.94026844,330.84083496)
\curveto(129.87026097,330.5408334)(129.76026108,330.27583367)(129.61026844,330.04583496)
\curveto(129.45026139,329.80583414)(129.25526158,329.61083433)(129.02526844,329.46083496)
\curveto(128.79526204,329.31083463)(128.5352623,329.18083476)(128.24526844,329.07083496)
\curveto(128.1352627,329.02083492)(128.01526282,328.98583496)(127.88526844,328.96583496)
\curveto(127.76526307,328.945835)(127.64526319,328.92083502)(127.52526844,328.89083496)
\curveto(127.4352634,328.87083507)(127.3402635,328.86083508)(127.24026844,328.86083496)
\curveto(127.15026369,328.85083509)(127.06026378,328.83583511)(126.97026844,328.81583496)
\lineto(126.70026844,328.81583496)
\curveto(126.6402642,328.79583515)(126.5352643,328.78583516)(126.38526844,328.78583496)
\curveto(126.24526459,328.78583516)(126.14526469,328.79583515)(126.08526844,328.81583496)
\curveto(126.05526478,328.81583513)(126.02026482,328.82083512)(125.98026844,328.83083496)
\lineto(125.87526844,328.83083496)
\curveto(125.75526508,328.85083509)(125.6352652,328.86583508)(125.51526844,328.87583496)
\curveto(125.39526544,328.88583506)(125.28026556,328.90583504)(125.17026844,328.93583496)
\curveto(124.78026606,329.0458349)(124.4352664,329.17083477)(124.13526844,329.31083496)
\curveto(123.835267,329.46083448)(123.58026726,329.68083426)(123.37026844,329.97083496)
\curveto(123.23026761,330.16083378)(123.11026773,330.38083356)(123.01026844,330.63083496)
\curveto(122.99026785,330.69083325)(122.97026787,330.77083317)(122.95026844,330.87083496)
\curveto(122.93026791,330.92083302)(122.91526792,330.99083295)(122.90526844,331.08083496)
\curveto(122.89526794,331.17083277)(122.90026794,331.2458327)(122.92026844,331.30583496)
\curveto(122.95026789,331.37583257)(123.00026784,331.42583252)(123.07026844,331.45583496)
\curveto(123.12026772,331.47583247)(123.18026766,331.48583246)(123.25026844,331.48583496)
\lineto(123.47526844,331.48583496)
\lineto(124.18026844,331.48583496)
\lineto(124.42026844,331.48583496)
\curveto(124.50026634,331.48583246)(124.57026627,331.47583247)(124.63026844,331.45583496)
\curveto(124.7402661,331.41583253)(124.81026603,331.35083259)(124.84026844,331.26083496)
\curveto(124.88026596,331.17083277)(124.92526591,331.07583287)(124.97526844,330.97583496)
\curveto(124.99526584,330.92583302)(125.03026581,330.86083308)(125.08026844,330.78083496)
\curveto(125.1402657,330.70083324)(125.19026565,330.65083329)(125.23026844,330.63083496)
\curveto(125.35026549,330.53083341)(125.46526537,330.45083349)(125.57526844,330.39083496)
\curveto(125.68526515,330.3408336)(125.82526501,330.29083365)(125.99526844,330.24083496)
\curveto(126.04526479,330.22083372)(126.09526474,330.21083373)(126.14526844,330.21083496)
\curveto(126.19526464,330.22083372)(126.24526459,330.22083372)(126.29526844,330.21083496)
\curveto(126.37526446,330.19083375)(126.46026438,330.18083376)(126.55026844,330.18083496)
\curveto(126.65026419,330.19083375)(126.7352641,330.20583374)(126.80526844,330.22583496)
\curveto(126.85526398,330.23583371)(126.90026394,330.2408337)(126.94026844,330.24083496)
\curveto(126.99026385,330.2408337)(127.0402638,330.25083369)(127.09026844,330.27083496)
\curveto(127.23026361,330.32083362)(127.35526348,330.38083356)(127.46526844,330.45083496)
\curveto(127.58526325,330.52083342)(127.68026316,330.61083333)(127.75026844,330.72083496)
\curveto(127.80026304,330.80083314)(127.840263,330.92583302)(127.87026844,331.09583496)
\curveto(127.89026295,331.16583278)(127.89026295,331.23083271)(127.87026844,331.29083496)
\curveto(127.85026299,331.35083259)(127.83026301,331.40083254)(127.81026844,331.44083496)
\curveto(127.7402631,331.58083236)(127.65026319,331.68583226)(127.54026844,331.75583496)
\curveto(127.4402634,331.82583212)(127.32026352,331.89083205)(127.18026844,331.95083496)
\curveto(126.99026385,332.03083191)(126.79026405,332.09583185)(126.58026844,332.14583496)
\curveto(126.37026447,332.19583175)(126.16026468,332.25083169)(125.95026844,332.31083496)
\curveto(125.87026497,332.33083161)(125.78526505,332.3458316)(125.69526844,332.35583496)
\curveto(125.61526522,332.36583158)(125.5352653,332.38083156)(125.45526844,332.40083496)
\curveto(125.1352657,332.49083145)(124.83026601,332.57583137)(124.54026844,332.65583496)
\curveto(124.25026659,332.7458312)(123.98526685,332.87583107)(123.74526844,333.04583496)
\curveto(123.46526737,333.2458307)(123.26026758,333.51583043)(123.13026844,333.85583496)
\curveto(123.11026773,333.92583002)(123.09026775,334.02082992)(123.07026844,334.14083496)
\curveto(123.05026779,334.21082973)(123.0352678,334.29582965)(123.02526844,334.39583496)
\curveto(123.01526782,334.49582945)(123.02026782,334.58582936)(123.04026844,334.66583496)
\curveto(123.06026778,334.71582923)(123.06526777,334.75582919)(123.05526844,334.78583496)
\curveto(123.04526779,334.82582912)(123.05026779,334.87082907)(123.07026844,334.92083496)
\curveto(123.09026775,335.03082891)(123.11026773,335.13082881)(123.13026844,335.22083496)
\curveto(123.16026768,335.32082862)(123.19526764,335.41582853)(123.23526844,335.50583496)
\curveto(123.36526747,335.79582815)(123.54526729,336.03082791)(123.77526844,336.21083496)
\curveto(124.00526683,336.39082755)(124.26526657,336.53582741)(124.55526844,336.64583496)
\curveto(124.66526617,336.69582725)(124.78026606,336.73082721)(124.90026844,336.75083496)
\curveto(125.02026582,336.78082716)(125.14526569,336.81082713)(125.27526844,336.84083496)
\curveto(125.3352655,336.86082708)(125.39526544,336.87082707)(125.45526844,336.87083496)
\lineto(125.63526844,336.90083496)
\curveto(125.71526512,336.91082703)(125.80026504,336.91582703)(125.89026844,336.91583496)
\curveto(125.98026486,336.91582703)(126.06526477,336.92082702)(126.14526844,336.93083496)
}
}
{
\newrgbcolor{curcolor}{0 0 0}
\pscustom[linestyle=none,fillstyle=solid,fillcolor=curcolor]
{
\newpath
\moveto(139.00190907,333.16583496)
\curveto(139.0219005,333.10583084)(139.03190049,333.02083092)(139.03190907,332.91083496)
\curveto(139.03190049,332.80083114)(139.0219005,332.71583123)(139.00190907,332.65583496)
\lineto(139.00190907,332.50583496)
\curveto(138.98190054,332.42583152)(138.97190055,332.3458316)(138.97190907,332.26583496)
\curveto(138.98190054,332.18583176)(138.97690054,332.10583184)(138.95690907,332.02583496)
\curveto(138.93690058,331.95583199)(138.9219006,331.89083205)(138.91190907,331.83083496)
\curveto(138.90190062,331.77083217)(138.89190063,331.70583224)(138.88190907,331.63583496)
\curveto(138.84190068,331.52583242)(138.80690071,331.41083253)(138.77690907,331.29083496)
\curveto(138.74690077,331.18083276)(138.70690081,331.07583287)(138.65690907,330.97583496)
\curveto(138.44690107,330.49583345)(138.17190135,330.10583384)(137.83190907,329.80583496)
\curveto(137.49190203,329.50583444)(137.08190244,329.25583469)(136.60190907,329.05583496)
\curveto(136.48190304,329.00583494)(136.35690316,328.97083497)(136.22690907,328.95083496)
\curveto(136.10690341,328.92083502)(135.98190354,328.89083505)(135.85190907,328.86083496)
\curveto(135.80190372,328.8408351)(135.74690377,328.83083511)(135.68690907,328.83083496)
\curveto(135.62690389,328.83083511)(135.57190395,328.82583512)(135.52190907,328.81583496)
\lineto(135.41690907,328.81583496)
\curveto(135.38690413,328.80583514)(135.35690416,328.80083514)(135.32690907,328.80083496)
\curveto(135.27690424,328.79083515)(135.19690432,328.78583516)(135.08690907,328.78583496)
\curveto(134.97690454,328.77583517)(134.89190463,328.78083516)(134.83190907,328.80083496)
\lineto(134.68190907,328.80083496)
\curveto(134.63190489,328.81083513)(134.57690494,328.81583513)(134.51690907,328.81583496)
\curveto(134.46690505,328.80583514)(134.4169051,328.81083513)(134.36690907,328.83083496)
\curveto(134.32690519,328.8408351)(134.28690523,328.8458351)(134.24690907,328.84583496)
\curveto(134.2169053,328.8458351)(134.17690534,328.85083509)(134.12690907,328.86083496)
\curveto(134.02690549,328.89083505)(133.92690559,328.91583503)(133.82690907,328.93583496)
\curveto(133.72690579,328.95583499)(133.63190589,328.98583496)(133.54190907,329.02583496)
\curveto(133.4219061,329.06583488)(133.30690621,329.10583484)(133.19690907,329.14583496)
\curveto(133.09690642,329.18583476)(132.99190653,329.23583471)(132.88190907,329.29583496)
\curveto(132.53190699,329.50583444)(132.23190729,329.75083419)(131.98190907,330.03083496)
\curveto(131.73190779,330.31083363)(131.521908,330.6458333)(131.35190907,331.03583496)
\curveto(131.30190822,331.12583282)(131.26190826,331.22083272)(131.23190907,331.32083496)
\curveto(131.21190831,331.42083252)(131.18690833,331.52583242)(131.15690907,331.63583496)
\curveto(131.13690838,331.68583226)(131.12690839,331.73083221)(131.12690907,331.77083496)
\curveto(131.12690839,331.81083213)(131.1169084,331.85583209)(131.09690907,331.90583496)
\curveto(131.07690844,331.98583196)(131.06690845,332.06583188)(131.06690907,332.14583496)
\curveto(131.06690845,332.23583171)(131.05690846,332.32083162)(131.03690907,332.40083496)
\curveto(131.02690849,332.45083149)(131.0219085,332.49583145)(131.02190907,332.53583496)
\lineto(131.02190907,332.67083496)
\curveto(131.00190852,332.73083121)(130.99190853,332.81583113)(130.99190907,332.92583496)
\curveto(131.00190852,333.03583091)(131.0169085,333.12083082)(131.03690907,333.18083496)
\lineto(131.03690907,333.28583496)
\curveto(131.04690847,333.33583061)(131.04690847,333.38583056)(131.03690907,333.43583496)
\curveto(131.03690848,333.49583045)(131.04690847,333.55083039)(131.06690907,333.60083496)
\curveto(131.07690844,333.65083029)(131.08190844,333.69583025)(131.08190907,333.73583496)
\curveto(131.08190844,333.78583016)(131.09190843,333.83583011)(131.11190907,333.88583496)
\curveto(131.15190837,334.01582993)(131.18690833,334.1408298)(131.21690907,334.26083496)
\curveto(131.24690827,334.39082955)(131.28690823,334.51582943)(131.33690907,334.63583496)
\curveto(131.516908,335.0458289)(131.73190779,335.38582856)(131.98190907,335.65583496)
\curveto(132.23190729,335.93582801)(132.53690698,336.19082775)(132.89690907,336.42083496)
\curveto(132.99690652,336.47082747)(133.10190642,336.51582743)(133.21190907,336.55583496)
\curveto(133.3219062,336.59582735)(133.43190609,336.6408273)(133.54190907,336.69083496)
\curveto(133.67190585,336.7408272)(133.80690571,336.77582717)(133.94690907,336.79583496)
\curveto(134.08690543,336.81582713)(134.23190529,336.8458271)(134.38190907,336.88583496)
\curveto(134.46190506,336.89582705)(134.53690498,336.90082704)(134.60690907,336.90083496)
\curveto(134.67690484,336.90082704)(134.74690477,336.90582704)(134.81690907,336.91583496)
\curveto(135.39690412,336.92582702)(135.89690362,336.86582708)(136.31690907,336.73583496)
\curveto(136.74690277,336.60582734)(137.12690239,336.42582752)(137.45690907,336.19583496)
\curveto(137.56690195,336.11582783)(137.67690184,336.02582792)(137.78690907,335.92583496)
\curveto(137.90690161,335.83582811)(138.00690151,335.73582821)(138.08690907,335.62583496)
\curveto(138.16690135,335.52582842)(138.23690128,335.42582852)(138.29690907,335.32583496)
\curveto(138.36690115,335.22582872)(138.43690108,335.12082882)(138.50690907,335.01083496)
\curveto(138.57690094,334.90082904)(138.63190089,334.78082916)(138.67190907,334.65083496)
\curveto(138.71190081,334.53082941)(138.75690076,334.40082954)(138.80690907,334.26083496)
\curveto(138.83690068,334.18082976)(138.86190066,334.09582985)(138.88190907,334.00583496)
\lineto(138.94190907,333.73583496)
\curveto(138.95190057,333.69583025)(138.95690056,333.65583029)(138.95690907,333.61583496)
\curveto(138.95690056,333.57583037)(138.96190056,333.53583041)(138.97190907,333.49583496)
\curveto(138.99190053,333.4458305)(138.99690052,333.39083055)(138.98690907,333.33083496)
\curveto(138.97690054,333.27083067)(138.98190054,333.21583073)(139.00190907,333.16583496)
\moveto(136.90190907,332.62583496)
\curveto(136.91190261,332.67583127)(136.9169026,332.7458312)(136.91690907,332.83583496)
\curveto(136.9169026,332.93583101)(136.91190261,333.01083093)(136.90190907,333.06083496)
\lineto(136.90190907,333.18083496)
\curveto(136.88190264,333.23083071)(136.87190265,333.28583066)(136.87190907,333.34583496)
\curveto(136.87190265,333.40583054)(136.86690265,333.46083048)(136.85690907,333.51083496)
\curveto(136.85690266,333.55083039)(136.85190267,333.58083036)(136.84190907,333.60083496)
\lineto(136.78190907,333.84083496)
\curveto(136.77190275,333.93083001)(136.75190277,334.01582993)(136.72190907,334.09583496)
\curveto(136.61190291,334.35582959)(136.48190304,334.57582937)(136.33190907,334.75583496)
\curveto(136.18190334,334.945829)(135.98190354,335.09582885)(135.73190907,335.20583496)
\curveto(135.67190385,335.22582872)(135.61190391,335.2408287)(135.55190907,335.25083496)
\curveto(135.49190403,335.27082867)(135.42690409,335.29082865)(135.35690907,335.31083496)
\curveto(135.27690424,335.33082861)(135.19190433,335.33582861)(135.10190907,335.32583496)
\lineto(134.83190907,335.32583496)
\curveto(134.80190472,335.30582864)(134.76690475,335.29582865)(134.72690907,335.29583496)
\curveto(134.68690483,335.30582864)(134.65190487,335.30582864)(134.62190907,335.29583496)
\lineto(134.41190907,335.23583496)
\curveto(134.35190517,335.22582872)(134.29690522,335.20582874)(134.24690907,335.17583496)
\curveto(133.99690552,335.06582888)(133.79190573,334.90582904)(133.63190907,334.69583496)
\curveto(133.48190604,334.49582945)(133.36190616,334.26082968)(133.27190907,333.99083496)
\curveto(133.24190628,333.89083005)(133.2169063,333.78583016)(133.19690907,333.67583496)
\curveto(133.18690633,333.56583038)(133.17190635,333.45583049)(133.15190907,333.34583496)
\curveto(133.14190638,333.29583065)(133.13690638,333.2458307)(133.13690907,333.19583496)
\lineto(133.13690907,333.04583496)
\curveto(133.1169064,332.97583097)(133.10690641,332.87083107)(133.10690907,332.73083496)
\curveto(133.1169064,332.59083135)(133.13190639,332.48583146)(133.15190907,332.41583496)
\lineto(133.15190907,332.28083496)
\curveto(133.17190635,332.20083174)(133.18690633,332.12083182)(133.19690907,332.04083496)
\curveto(133.20690631,331.97083197)(133.2219063,331.89583205)(133.24190907,331.81583496)
\curveto(133.34190618,331.51583243)(133.44690607,331.27083267)(133.55690907,331.08083496)
\curveto(133.67690584,330.90083304)(133.86190566,330.73583321)(134.11190907,330.58583496)
\curveto(134.18190534,330.53583341)(134.25690526,330.49583345)(134.33690907,330.46583496)
\curveto(134.42690509,330.43583351)(134.516905,330.41083353)(134.60690907,330.39083496)
\curveto(134.64690487,330.38083356)(134.68190484,330.37583357)(134.71190907,330.37583496)
\curveto(134.74190478,330.38583356)(134.77690474,330.38583356)(134.81690907,330.37583496)
\lineto(134.93690907,330.34583496)
\curveto(134.98690453,330.3458336)(135.03190449,330.35083359)(135.07190907,330.36083496)
\lineto(135.19190907,330.36083496)
\curveto(135.27190425,330.38083356)(135.35190417,330.39583355)(135.43190907,330.40583496)
\curveto(135.51190401,330.41583353)(135.58690393,330.43583351)(135.65690907,330.46583496)
\curveto(135.9169036,330.56583338)(136.12690339,330.70083324)(136.28690907,330.87083496)
\curveto(136.44690307,331.0408329)(136.58190294,331.25083269)(136.69190907,331.50083496)
\curveto(136.73190279,331.60083234)(136.76190276,331.70083224)(136.78190907,331.80083496)
\curveto(136.80190272,331.90083204)(136.82690269,332.00583194)(136.85690907,332.11583496)
\curveto(136.86690265,332.15583179)(136.87190265,332.19083175)(136.87190907,332.22083496)
\curveto(136.87190265,332.26083168)(136.87690264,332.30083164)(136.88690907,332.34083496)
\lineto(136.88690907,332.47583496)
\curveto(136.88690263,332.52583142)(136.89190263,332.57583137)(136.90190907,332.62583496)
}
}
{
\newrgbcolor{curcolor}{0 0 0}
\pscustom[linestyle=none,fillstyle=solid,fillcolor=curcolor]
{
\newpath
\moveto(143.97183094,336.93083496)
\curveto(144.78182578,336.95082699)(145.45682511,336.83082711)(145.99683094,336.57083496)
\curveto(146.54682402,336.31082763)(146.98182358,335.940828)(147.30183094,335.46083496)
\curveto(147.4618231,335.22082872)(147.58182298,334.945829)(147.66183094,334.63583496)
\curveto(147.68182288,334.58582936)(147.69682287,334.52082942)(147.70683094,334.44083496)
\curveto(147.72682284,334.36082958)(147.72682284,334.29082965)(147.70683094,334.23083496)
\curveto(147.6668229,334.12082982)(147.59682297,334.05582989)(147.49683094,334.03583496)
\curveto(147.39682317,334.02582992)(147.27682329,334.02082992)(147.13683094,334.02083496)
\lineto(146.35683094,334.02083496)
\lineto(146.07183094,334.02083496)
\curveto(145.98182458,334.02082992)(145.90682466,334.0408299)(145.84683094,334.08083496)
\curveto(145.7668248,334.12082982)(145.71182485,334.18082976)(145.68183094,334.26083496)
\curveto(145.65182491,334.35082959)(145.61182495,334.4408295)(145.56183094,334.53083496)
\curveto(145.50182506,334.6408293)(145.43682513,334.7408292)(145.36683094,334.83083496)
\curveto(145.29682527,334.92082902)(145.21682535,335.00082894)(145.12683094,335.07083496)
\curveto(144.98682558,335.16082878)(144.83182573,335.23082871)(144.66183094,335.28083496)
\curveto(144.60182596,335.30082864)(144.54182602,335.31082863)(144.48183094,335.31083496)
\curveto(144.42182614,335.31082863)(144.3668262,335.32082862)(144.31683094,335.34083496)
\lineto(144.16683094,335.34083496)
\curveto(143.9668266,335.3408286)(143.80682676,335.32082862)(143.68683094,335.28083496)
\curveto(143.39682717,335.19082875)(143.1618274,335.05082889)(142.98183094,334.86083496)
\curveto(142.80182776,334.68082926)(142.65682791,334.46082948)(142.54683094,334.20083496)
\curveto(142.49682807,334.09082985)(142.45682811,333.97082997)(142.42683094,333.84083496)
\curveto(142.40682816,333.72083022)(142.38182818,333.59083035)(142.35183094,333.45083496)
\curveto(142.34182822,333.41083053)(142.33682823,333.37083057)(142.33683094,333.33083496)
\curveto(142.33682823,333.29083065)(142.33182823,333.25083069)(142.32183094,333.21083496)
\curveto(142.30182826,333.11083083)(142.29182827,332.97083097)(142.29183094,332.79083496)
\curveto(142.30182826,332.61083133)(142.31682825,332.47083147)(142.33683094,332.37083496)
\curveto(142.33682823,332.29083165)(142.34182822,332.23583171)(142.35183094,332.20583496)
\curveto(142.37182819,332.13583181)(142.38182818,332.06583188)(142.38183094,331.99583496)
\curveto(142.39182817,331.92583202)(142.40682816,331.85583209)(142.42683094,331.78583496)
\curveto(142.50682806,331.55583239)(142.60182796,331.3458326)(142.71183094,331.15583496)
\curveto(142.82182774,330.96583298)(142.9618276,330.80583314)(143.13183094,330.67583496)
\curveto(143.17182739,330.6458333)(143.23182733,330.61083333)(143.31183094,330.57083496)
\curveto(143.42182714,330.50083344)(143.53182703,330.45583349)(143.64183094,330.43583496)
\curveto(143.7618268,330.41583353)(143.90682666,330.39583355)(144.07683094,330.37583496)
\lineto(144.16683094,330.37583496)
\curveto(144.20682636,330.37583357)(144.23682633,330.38083356)(144.25683094,330.39083496)
\lineto(144.39183094,330.39083496)
\curveto(144.4618261,330.41083353)(144.52682604,330.42583352)(144.58683094,330.43583496)
\curveto(144.65682591,330.45583349)(144.72182584,330.47583347)(144.78183094,330.49583496)
\curveto(145.08182548,330.62583332)(145.31182525,330.81583313)(145.47183094,331.06583496)
\curveto(145.51182505,331.11583283)(145.54682502,331.17083277)(145.57683094,331.23083496)
\curveto(145.60682496,331.30083264)(145.63182493,331.36083258)(145.65183094,331.41083496)
\curveto(145.69182487,331.52083242)(145.72682484,331.61583233)(145.75683094,331.69583496)
\curveto(145.78682478,331.78583216)(145.85682471,331.85583209)(145.96683094,331.90583496)
\curveto(146.05682451,331.945832)(146.20182436,331.96083198)(146.40183094,331.95083496)
\lineto(146.89683094,331.95083496)
\lineto(147.10683094,331.95083496)
\curveto(147.18682338,331.96083198)(147.25182331,331.95583199)(147.30183094,331.93583496)
\lineto(147.42183094,331.93583496)
\lineto(147.54183094,331.90583496)
\curveto(147.58182298,331.90583204)(147.61182295,331.89583205)(147.63183094,331.87583496)
\curveto(147.68182288,331.83583211)(147.71182285,331.77583217)(147.72183094,331.69583496)
\curveto(147.74182282,331.62583232)(147.74182282,331.55083239)(147.72183094,331.47083496)
\curveto(147.63182293,331.1408328)(147.52182304,330.8458331)(147.39183094,330.58583496)
\curveto(146.98182358,329.81583413)(146.32682424,329.28083466)(145.42683094,328.98083496)
\curveto(145.32682524,328.95083499)(145.22182534,328.93083501)(145.11183094,328.92083496)
\curveto(145.00182556,328.90083504)(144.89182567,328.87583507)(144.78183094,328.84583496)
\curveto(144.72182584,328.83583511)(144.6618259,328.83083511)(144.60183094,328.83083496)
\curveto(144.54182602,328.83083511)(144.48182608,328.82583512)(144.42183094,328.81583496)
\lineto(144.25683094,328.81583496)
\curveto(144.20682636,328.79583515)(144.13182643,328.79083515)(144.03183094,328.80083496)
\curveto(143.93182663,328.80083514)(143.85682671,328.80583514)(143.80683094,328.81583496)
\curveto(143.72682684,328.83583511)(143.65182691,328.8458351)(143.58183094,328.84583496)
\curveto(143.52182704,328.83583511)(143.45682711,328.8408351)(143.38683094,328.86083496)
\lineto(143.23683094,328.89083496)
\curveto(143.18682738,328.89083505)(143.13682743,328.89583505)(143.08683094,328.90583496)
\curveto(142.97682759,328.93583501)(142.87182769,328.96583498)(142.77183094,328.99583496)
\curveto(142.67182789,329.02583492)(142.57682799,329.06083488)(142.48683094,329.10083496)
\curveto(142.01682855,329.30083464)(141.62182894,329.55583439)(141.30183094,329.86583496)
\curveto(140.98182958,330.18583376)(140.72182984,330.58083336)(140.52183094,331.05083496)
\curveto(140.47183009,331.1408328)(140.43183013,331.23583271)(140.40183094,331.33583496)
\lineto(140.31183094,331.66583496)
\curveto(140.30183026,331.70583224)(140.29683027,331.7408322)(140.29683094,331.77083496)
\curveto(140.29683027,331.81083213)(140.28683028,331.85583209)(140.26683094,331.90583496)
\curveto(140.24683032,331.97583197)(140.23683033,332.0458319)(140.23683094,332.11583496)
\curveto(140.23683033,332.19583175)(140.22683034,332.27083167)(140.20683094,332.34083496)
\lineto(140.20683094,332.59583496)
\curveto(140.18683038,332.6458313)(140.17683039,332.70083124)(140.17683094,332.76083496)
\curveto(140.17683039,332.83083111)(140.18683038,332.89083105)(140.20683094,332.94083496)
\curveto(140.21683035,332.99083095)(140.21683035,333.03583091)(140.20683094,333.07583496)
\curveto(140.19683037,333.11583083)(140.19683037,333.15583079)(140.20683094,333.19583496)
\curveto(140.22683034,333.26583068)(140.23183033,333.33083061)(140.22183094,333.39083496)
\curveto(140.22183034,333.45083049)(140.23183033,333.51083043)(140.25183094,333.57083496)
\curveto(140.30183026,333.75083019)(140.34183022,333.92083002)(140.37183094,334.08083496)
\curveto(140.40183016,334.25082969)(140.44683012,334.41582953)(140.50683094,334.57583496)
\curveto(140.72682984,335.08582886)(141.00182956,335.51082843)(141.33183094,335.85083496)
\curveto(141.67182889,336.19082775)(142.10182846,336.46582748)(142.62183094,336.67583496)
\curveto(142.7618278,336.73582721)(142.90682766,336.77582717)(143.05683094,336.79583496)
\curveto(143.20682736,336.82582712)(143.3618272,336.86082708)(143.52183094,336.90083496)
\curveto(143.60182696,336.91082703)(143.67682689,336.91582703)(143.74683094,336.91583496)
\curveto(143.81682675,336.91582703)(143.89182667,336.92082702)(143.97183094,336.93083496)
}
}
{
\newrgbcolor{curcolor}{0 0 0}
\pscustom[linestyle=none,fillstyle=solid,fillcolor=curcolor]
{
\newpath
\moveto(151.11511219,339.57083496)
\curveto(151.18510924,339.49082445)(151.22010921,339.37082457)(151.22011219,339.21083496)
\lineto(151.22011219,338.74583496)
\lineto(151.22011219,338.34083496)
\curveto(151.22010921,338.20082574)(151.18510924,338.10582584)(151.11511219,338.05583496)
\curveto(151.05510937,338.00582594)(150.97510945,337.97582597)(150.87511219,337.96583496)
\curveto(150.78510964,337.95582599)(150.68510974,337.95082599)(150.57511219,337.95083496)
\lineto(149.73511219,337.95083496)
\curveto(149.6251108,337.95082599)(149.5251109,337.95582599)(149.43511219,337.96583496)
\curveto(149.35511107,337.97582597)(149.28511114,338.00582594)(149.22511219,338.05583496)
\curveto(149.18511124,338.08582586)(149.15511127,338.1408258)(149.13511219,338.22083496)
\curveto(149.1251113,338.31082563)(149.11511131,338.40582554)(149.10511219,338.50583496)
\lineto(149.10511219,338.83583496)
\curveto(149.11511131,338.945825)(149.12011131,339.0408249)(149.12011219,339.12083496)
\lineto(149.12011219,339.33083496)
\curveto(149.1301113,339.40082454)(149.15011128,339.46082448)(149.18011219,339.51083496)
\curveto(149.20011123,339.55082439)(149.2251112,339.58082436)(149.25511219,339.60083496)
\lineto(149.37511219,339.66083496)
\curveto(149.39511103,339.66082428)(149.42011101,339.66082428)(149.45011219,339.66083496)
\curveto(149.48011095,339.67082427)(149.50511092,339.67582427)(149.52511219,339.67583496)
\lineto(150.62011219,339.67583496)
\curveto(150.72010971,339.67582427)(150.81510961,339.67082427)(150.90511219,339.66083496)
\curveto(150.99510943,339.65082429)(151.06510936,339.62082432)(151.11511219,339.57083496)
\moveto(151.22011219,329.80583496)
\curveto(151.22010921,329.60583434)(151.21510921,329.43583451)(151.20511219,329.29583496)
\curveto(151.19510923,329.15583479)(151.10510932,329.06083488)(150.93511219,329.01083496)
\curveto(150.87510955,328.99083495)(150.81010962,328.98083496)(150.74011219,328.98083496)
\curveto(150.67010976,328.99083495)(150.59510983,328.99583495)(150.51511219,328.99583496)
\lineto(149.67511219,328.99583496)
\curveto(149.58511084,328.99583495)(149.49511093,329.00083494)(149.40511219,329.01083496)
\curveto(149.3251111,329.02083492)(149.26511116,329.05083489)(149.22511219,329.10083496)
\curveto(149.16511126,329.17083477)(149.1301113,329.25583469)(149.12011219,329.35583496)
\lineto(149.12011219,329.70083496)
\lineto(149.12011219,336.03083496)
\lineto(149.12011219,336.33083496)
\curveto(149.12011131,336.43082751)(149.14011129,336.51082743)(149.18011219,336.57083496)
\curveto(149.24011119,336.6408273)(149.3251111,336.68582726)(149.43511219,336.70583496)
\curveto(149.45511097,336.71582723)(149.48011095,336.71582723)(149.51011219,336.70583496)
\curveto(149.55011088,336.70582724)(149.58011085,336.71082723)(149.60011219,336.72083496)
\lineto(150.35011219,336.72083496)
\lineto(150.54511219,336.72083496)
\curveto(150.6251098,336.73082721)(150.69010974,336.73082721)(150.74011219,336.72083496)
\lineto(150.86011219,336.72083496)
\curveto(150.92010951,336.70082724)(150.97510945,336.68582726)(151.02511219,336.67583496)
\curveto(151.07510935,336.66582728)(151.11510931,336.63582731)(151.14511219,336.58583496)
\curveto(151.18510924,336.53582741)(151.20510922,336.46582748)(151.20511219,336.37583496)
\curveto(151.21510921,336.28582766)(151.22010921,336.19082775)(151.22011219,336.09083496)
\lineto(151.22011219,329.80583496)
}
}
{
\newrgbcolor{curcolor}{0 0 0}
\pscustom[linestyle=none,fillstyle=solid,fillcolor=curcolor]
{
\newpath
\moveto(159.93229969,329.58083496)
\curveto(159.95229184,329.47083447)(159.96229183,329.36083458)(159.96229969,329.25083496)
\curveto(159.97229182,329.1408348)(159.92229187,329.06583488)(159.81229969,329.02583496)
\curveto(159.75229204,328.99583495)(159.68229211,328.98083496)(159.60229969,328.98083496)
\lineto(159.36229969,328.98083496)
\lineto(158.55229969,328.98083496)
\lineto(158.28229969,328.98083496)
\curveto(158.20229359,328.99083495)(158.13729366,329.01583493)(158.08729969,329.05583496)
\curveto(158.01729378,329.09583485)(157.96229383,329.15083479)(157.92229969,329.22083496)
\curveto(157.8922939,329.30083464)(157.84729395,329.36583458)(157.78729969,329.41583496)
\curveto(157.76729403,329.43583451)(157.74229405,329.45083449)(157.71229969,329.46083496)
\curveto(157.68229411,329.48083446)(157.64229415,329.48583446)(157.59229969,329.47583496)
\curveto(157.54229425,329.45583449)(157.4922943,329.43083451)(157.44229969,329.40083496)
\curveto(157.40229439,329.37083457)(157.35729444,329.3458346)(157.30729969,329.32583496)
\curveto(157.25729454,329.28583466)(157.20229459,329.25083469)(157.14229969,329.22083496)
\lineto(156.96229969,329.13083496)
\curveto(156.83229496,329.07083487)(156.6972951,329.02083492)(156.55729969,328.98083496)
\curveto(156.41729538,328.95083499)(156.27229552,328.91583503)(156.12229969,328.87583496)
\curveto(156.05229574,328.85583509)(155.98229581,328.8458351)(155.91229969,328.84583496)
\curveto(155.85229594,328.83583511)(155.78729601,328.82583512)(155.71729969,328.81583496)
\lineto(155.62729969,328.81583496)
\curveto(155.5972962,328.80583514)(155.56729623,328.80083514)(155.53729969,328.80083496)
\lineto(155.37229969,328.80083496)
\curveto(155.27229652,328.78083516)(155.17229662,328.78083516)(155.07229969,328.80083496)
\lineto(154.93729969,328.80083496)
\curveto(154.86729693,328.82083512)(154.797297,328.83083511)(154.72729969,328.83083496)
\curveto(154.66729713,328.82083512)(154.60729719,328.82583512)(154.54729969,328.84583496)
\curveto(154.44729735,328.86583508)(154.35229744,328.88583506)(154.26229969,328.90583496)
\curveto(154.17229762,328.91583503)(154.08729771,328.940835)(154.00729969,328.98083496)
\curveto(153.71729808,329.09083485)(153.46729833,329.23083471)(153.25729969,329.40083496)
\curveto(153.05729874,329.58083436)(152.8972989,329.81583413)(152.77729969,330.10583496)
\curveto(152.74729905,330.17583377)(152.71729908,330.25083369)(152.68729969,330.33083496)
\curveto(152.66729913,330.41083353)(152.64729915,330.49583345)(152.62729969,330.58583496)
\curveto(152.60729919,330.63583331)(152.5972992,330.68583326)(152.59729969,330.73583496)
\curveto(152.60729919,330.78583316)(152.60729919,330.83583311)(152.59729969,330.88583496)
\curveto(152.58729921,330.91583303)(152.57729922,330.97583297)(152.56729969,331.06583496)
\curveto(152.56729923,331.16583278)(152.57229922,331.23583271)(152.58229969,331.27583496)
\curveto(152.60229919,331.37583257)(152.61229918,331.46083248)(152.61229969,331.53083496)
\lineto(152.70229969,331.86083496)
\curveto(152.73229906,331.98083196)(152.77229902,332.08583186)(152.82229969,332.17583496)
\curveto(152.9922988,332.46583148)(153.18729861,332.68583126)(153.40729969,332.83583496)
\curveto(153.62729817,332.98583096)(153.90729789,333.11583083)(154.24729969,333.22583496)
\curveto(154.37729742,333.27583067)(154.51229728,333.31083063)(154.65229969,333.33083496)
\curveto(154.792297,333.35083059)(154.93229686,333.37583057)(155.07229969,333.40583496)
\curveto(155.15229664,333.42583052)(155.23729656,333.43583051)(155.32729969,333.43583496)
\curveto(155.41729638,333.4458305)(155.50729629,333.46083048)(155.59729969,333.48083496)
\curveto(155.66729613,333.50083044)(155.73729606,333.50583044)(155.80729969,333.49583496)
\curveto(155.87729592,333.49583045)(155.95229584,333.50583044)(156.03229969,333.52583496)
\curveto(156.10229569,333.5458304)(156.17229562,333.55583039)(156.24229969,333.55583496)
\curveto(156.31229548,333.55583039)(156.38729541,333.56583038)(156.46729969,333.58583496)
\curveto(156.67729512,333.63583031)(156.86729493,333.67583027)(157.03729969,333.70583496)
\curveto(157.21729458,333.7458302)(157.37729442,333.83583011)(157.51729969,333.97583496)
\curveto(157.60729419,334.06582988)(157.66729413,334.16582978)(157.69729969,334.27583496)
\curveto(157.70729409,334.30582964)(157.70729409,334.33082961)(157.69729969,334.35083496)
\curveto(157.6972941,334.37082957)(157.70229409,334.39082955)(157.71229969,334.41083496)
\curveto(157.72229407,334.43082951)(157.72729407,334.46082948)(157.72729969,334.50083496)
\lineto(157.72729969,334.59083496)
\lineto(157.69729969,334.71083496)
\curveto(157.6972941,334.75082919)(157.6922941,334.78582916)(157.68229969,334.81583496)
\curveto(157.58229421,335.11582883)(157.37229442,335.32082862)(157.05229969,335.43083496)
\curveto(156.96229483,335.46082848)(156.85229494,335.48082846)(156.72229969,335.49083496)
\curveto(156.60229519,335.51082843)(156.47729532,335.51582843)(156.34729969,335.50583496)
\curveto(156.21729558,335.50582844)(156.0922957,335.49582845)(155.97229969,335.47583496)
\curveto(155.85229594,335.45582849)(155.74729605,335.43082851)(155.65729969,335.40083496)
\curveto(155.5972962,335.38082856)(155.53729626,335.35082859)(155.47729969,335.31083496)
\curveto(155.42729637,335.28082866)(155.37729642,335.2458287)(155.32729969,335.20583496)
\curveto(155.27729652,335.16582878)(155.22229657,335.11082883)(155.16229969,335.04083496)
\curveto(155.11229668,334.97082897)(155.07729672,334.90582904)(155.05729969,334.84583496)
\curveto(155.00729679,334.7458292)(154.96229683,334.65082929)(154.92229969,334.56083496)
\curveto(154.8922969,334.47082947)(154.82229697,334.41082953)(154.71229969,334.38083496)
\curveto(154.63229716,334.36082958)(154.54729725,334.35082959)(154.45729969,334.35083496)
\lineto(154.18729969,334.35083496)
\lineto(153.61729969,334.35083496)
\curveto(153.56729823,334.35082959)(153.51729828,334.3458296)(153.46729969,334.33583496)
\curveto(153.41729838,334.33582961)(153.37229842,334.3408296)(153.33229969,334.35083496)
\lineto(153.19729969,334.35083496)
\curveto(153.17729862,334.36082958)(153.15229864,334.36582958)(153.12229969,334.36583496)
\curveto(153.0922987,334.36582958)(153.06729873,334.37582957)(153.04729969,334.39583496)
\curveto(152.96729883,334.41582953)(152.91229888,334.48082946)(152.88229969,334.59083496)
\curveto(152.87229892,334.6408293)(152.87229892,334.69082925)(152.88229969,334.74083496)
\curveto(152.8922989,334.79082915)(152.90229889,334.83582911)(152.91229969,334.87583496)
\curveto(152.94229885,334.98582896)(152.97229882,335.08582886)(153.00229969,335.17583496)
\curveto(153.04229875,335.27582867)(153.08729871,335.36582858)(153.13729969,335.44583496)
\lineto(153.22729969,335.59583496)
\lineto(153.31729969,335.74583496)
\curveto(153.3972984,335.85582809)(153.4972983,335.96082798)(153.61729969,336.06083496)
\curveto(153.63729816,336.07082787)(153.66729813,336.09582785)(153.70729969,336.13583496)
\curveto(153.75729804,336.17582777)(153.80229799,336.21082773)(153.84229969,336.24083496)
\curveto(153.88229791,336.27082767)(153.92729787,336.30082764)(153.97729969,336.33083496)
\curveto(154.14729765,336.4408275)(154.32729747,336.52582742)(154.51729969,336.58583496)
\curveto(154.70729709,336.65582729)(154.90229689,336.72082722)(155.10229969,336.78083496)
\curveto(155.22229657,336.81082713)(155.34729645,336.83082711)(155.47729969,336.84083496)
\curveto(155.60729619,336.85082709)(155.73729606,336.87082707)(155.86729969,336.90083496)
\curveto(155.90729589,336.91082703)(155.96729583,336.91082703)(156.04729969,336.90083496)
\curveto(156.13729566,336.89082705)(156.1922956,336.89582705)(156.21229969,336.91583496)
\curveto(156.62229517,336.92582702)(157.01229478,336.91082703)(157.38229969,336.87083496)
\curveto(157.76229403,336.83082711)(158.10229369,336.75582719)(158.40229969,336.64583496)
\curveto(158.71229308,336.53582741)(158.97729282,336.38582756)(159.19729969,336.19583496)
\curveto(159.41729238,336.01582793)(159.58729221,335.78082816)(159.70729969,335.49083496)
\curveto(159.77729202,335.32082862)(159.81729198,335.12582882)(159.82729969,334.90583496)
\curveto(159.83729196,334.68582926)(159.84229195,334.46082948)(159.84229969,334.23083496)
\lineto(159.84229969,330.88583496)
\lineto(159.84229969,330.30083496)
\curveto(159.84229195,330.11083383)(159.86229193,329.93583401)(159.90229969,329.77583496)
\curveto(159.91229188,329.7458342)(159.91729188,329.71083423)(159.91729969,329.67083496)
\curveto(159.91729188,329.6408343)(159.92229187,329.61083433)(159.93229969,329.58083496)
\moveto(157.72729969,331.89083496)
\curveto(157.73729406,331.940832)(157.74229405,331.99583195)(157.74229969,332.05583496)
\curveto(157.74229405,332.12583182)(157.73729406,332.18583176)(157.72729969,332.23583496)
\curveto(157.70729409,332.29583165)(157.6972941,332.35083159)(157.69729969,332.40083496)
\curveto(157.6972941,332.45083149)(157.67729412,332.49083145)(157.63729969,332.52083496)
\curveto(157.58729421,332.56083138)(157.51229428,332.58083136)(157.41229969,332.58083496)
\curveto(157.37229442,332.57083137)(157.33729446,332.56083138)(157.30729969,332.55083496)
\curveto(157.27729452,332.55083139)(157.24229455,332.5458314)(157.20229969,332.53583496)
\curveto(157.13229466,332.51583143)(157.05729474,332.50083144)(156.97729969,332.49083496)
\curveto(156.8972949,332.48083146)(156.81729498,332.46583148)(156.73729969,332.44583496)
\curveto(156.70729509,332.43583151)(156.66229513,332.43083151)(156.60229969,332.43083496)
\curveto(156.47229532,332.40083154)(156.34229545,332.38083156)(156.21229969,332.37083496)
\curveto(156.08229571,332.36083158)(155.95729584,332.33583161)(155.83729969,332.29583496)
\curveto(155.75729604,332.27583167)(155.68229611,332.25583169)(155.61229969,332.23583496)
\curveto(155.54229625,332.22583172)(155.47229632,332.20583174)(155.40229969,332.17583496)
\curveto(155.1922966,332.08583186)(155.01229678,331.95083199)(154.86229969,331.77083496)
\curveto(154.72229707,331.59083235)(154.67229712,331.3408326)(154.71229969,331.02083496)
\curveto(154.73229706,330.85083309)(154.78729701,330.71083323)(154.87729969,330.60083496)
\curveto(154.94729685,330.49083345)(155.05229674,330.40083354)(155.19229969,330.33083496)
\curveto(155.33229646,330.27083367)(155.48229631,330.22583372)(155.64229969,330.19583496)
\curveto(155.81229598,330.16583378)(155.98729581,330.15583379)(156.16729969,330.16583496)
\curveto(156.35729544,330.18583376)(156.53229526,330.22083372)(156.69229969,330.27083496)
\curveto(156.95229484,330.35083359)(157.15729464,330.47583347)(157.30729969,330.64583496)
\curveto(157.45729434,330.82583312)(157.57229422,331.0458329)(157.65229969,331.30583496)
\curveto(157.67229412,331.37583257)(157.68229411,331.4458325)(157.68229969,331.51583496)
\curveto(157.6922941,331.59583235)(157.70729409,331.67583227)(157.72729969,331.75583496)
\lineto(157.72729969,331.89083496)
}
}
{
\newrgbcolor{curcolor}{0 0 0}
\pscustom[linestyle=none,fillstyle=solid,fillcolor=curcolor]
{
\newpath
\moveto(169.29558094,333.24083496)
\curveto(169.31557234,333.18083076)(169.32557233,333.07583087)(169.32558094,332.92583496)
\curveto(169.32557233,332.78583116)(169.32057234,332.68583126)(169.31058094,332.62583496)
\curveto(169.31057235,332.57583137)(169.30557235,332.53083141)(169.29558094,332.49083496)
\lineto(169.29558094,332.37083496)
\curveto(169.27557238,332.29083165)(169.26557239,332.21083173)(169.26558094,332.13083496)
\curveto(169.26557239,332.06083188)(169.2555724,331.98583196)(169.23558094,331.90583496)
\curveto(169.23557242,331.86583208)(169.22557243,331.79583215)(169.20558094,331.69583496)
\curveto(169.17557248,331.57583237)(169.14557251,331.45083249)(169.11558094,331.32083496)
\curveto(169.09557256,331.20083274)(169.0605726,331.08583286)(169.01058094,330.97583496)
\curveto(168.83057283,330.52583342)(168.60557305,330.13583381)(168.33558094,329.80583496)
\curveto(168.06557359,329.47583447)(167.71057395,329.21583473)(167.27058094,329.02583496)
\curveto(167.18057448,328.98583496)(167.08557457,328.95583499)(166.98558094,328.93583496)
\curveto(166.89557476,328.90583504)(166.79557486,328.87583507)(166.68558094,328.84583496)
\curveto(166.62557503,328.82583512)(166.5605751,328.81583513)(166.49058094,328.81583496)
\curveto(166.43057523,328.81583513)(166.37057529,328.81083513)(166.31058094,328.80083496)
\lineto(166.17558094,328.80083496)
\curveto(166.11557554,328.78083516)(166.03557562,328.77583517)(165.93558094,328.78583496)
\curveto(165.83557582,328.78583516)(165.7555759,328.79583515)(165.69558094,328.81583496)
\lineto(165.60558094,328.81583496)
\curveto(165.5555761,328.82583512)(165.50057616,328.83583511)(165.44058094,328.84583496)
\curveto(165.38057628,328.8458351)(165.32057634,328.85083509)(165.26058094,328.86083496)
\curveto(165.07057659,328.91083503)(164.89557676,328.96083498)(164.73558094,329.01083496)
\curveto(164.57557708,329.06083488)(164.42557723,329.13083481)(164.28558094,329.22083496)
\lineto(164.10558094,329.34083496)
\curveto(164.0555776,329.38083456)(164.00557765,329.42583452)(163.95558094,329.47583496)
\lineto(163.86558094,329.53583496)
\curveto(163.83557782,329.55583439)(163.80557785,329.57083437)(163.77558094,329.58083496)
\curveto(163.68557797,329.61083433)(163.63057803,329.59083435)(163.61058094,329.52083496)
\curveto(163.5605781,329.45083449)(163.52557813,329.36583458)(163.50558094,329.26583496)
\curveto(163.49557816,329.17583477)(163.4605782,329.10583484)(163.40058094,329.05583496)
\curveto(163.34057832,329.01583493)(163.27057839,328.99083495)(163.19058094,328.98083496)
\lineto(162.92058094,328.98083496)
\lineto(162.20058094,328.98083496)
\lineto(161.97558094,328.98083496)
\curveto(161.90557975,328.97083497)(161.84057982,328.97583497)(161.78058094,328.99583496)
\curveto(161.64058002,329.0458349)(161.5605801,329.13583481)(161.54058094,329.26583496)
\curveto(161.53058013,329.40583454)(161.52558013,329.56083438)(161.52558094,329.73083496)
\lineto(161.52558094,338.88083496)
\lineto(161.52558094,339.22583496)
\curveto(161.52558013,339.3458246)(161.55058011,339.4408245)(161.60058094,339.51083496)
\curveto(161.64058002,339.58082436)(161.71057995,339.62582432)(161.81058094,339.64583496)
\curveto(161.83057983,339.65582429)(161.85057981,339.65582429)(161.87058094,339.64583496)
\curveto(161.90057976,339.6458243)(161.92557973,339.65082429)(161.94558094,339.66083496)
\lineto(162.89058094,339.66083496)
\curveto(163.07057859,339.66082428)(163.22557843,339.65082429)(163.35558094,339.63083496)
\curveto(163.48557817,339.62082432)(163.57057809,339.5458244)(163.61058094,339.40583496)
\curveto(163.64057802,339.30582464)(163.65057801,339.17082477)(163.64058094,339.00083496)
\curveto(163.63057803,338.8408251)(163.62557803,338.70082524)(163.62558094,338.58083496)
\lineto(163.62558094,336.94583496)
\lineto(163.62558094,336.61583496)
\curveto(163.62557803,336.50582744)(163.63557802,336.41082753)(163.65558094,336.33083496)
\curveto(163.66557799,336.28082766)(163.67557798,336.23582771)(163.68558094,336.19583496)
\curveto(163.69557796,336.16582778)(163.72057794,336.1458278)(163.76058094,336.13583496)
\curveto(163.78057788,336.11582783)(163.80557785,336.10582784)(163.83558094,336.10583496)
\curveto(163.87557778,336.10582784)(163.90557775,336.11082783)(163.92558094,336.12083496)
\curveto(163.99557766,336.16082778)(164.0605776,336.20082774)(164.12058094,336.24083496)
\curveto(164.18057748,336.29082765)(164.24557741,336.3408276)(164.31558094,336.39083496)
\curveto(164.44557721,336.48082746)(164.58057708,336.55582739)(164.72058094,336.61583496)
\curveto(164.8605768,336.68582726)(165.01557664,336.7458272)(165.18558094,336.79583496)
\curveto(165.26557639,336.82582712)(165.34557631,336.8408271)(165.42558094,336.84083496)
\curveto(165.50557615,336.85082709)(165.58557607,336.86582708)(165.66558094,336.88583496)
\curveto(165.73557592,336.90582704)(165.81057585,336.91582703)(165.89058094,336.91583496)
\lineto(166.13058094,336.91583496)
\lineto(166.28058094,336.91583496)
\curveto(166.31057535,336.90582704)(166.34557531,336.90082704)(166.38558094,336.90083496)
\curveto(166.42557523,336.91082703)(166.46557519,336.91082703)(166.50558094,336.90083496)
\curveto(166.61557504,336.87082707)(166.71557494,336.8458271)(166.80558094,336.82583496)
\curveto(166.90557475,336.81582713)(167.00057466,336.79082715)(167.09058094,336.75083496)
\curveto(167.55057411,336.56082738)(167.92557373,336.31582763)(168.21558094,336.01583496)
\curveto(168.50557315,335.71582823)(168.75057291,335.3408286)(168.95058094,334.89083496)
\curveto(169.00057266,334.77082917)(169.04057262,334.6458293)(169.07058094,334.51583496)
\curveto(169.11057255,334.38582956)(169.15057251,334.25082969)(169.19058094,334.11083496)
\curveto(169.21057245,334.0408299)(169.22057244,333.97082997)(169.22058094,333.90083496)
\curveto(169.23057243,333.8408301)(169.24557241,333.77083017)(169.26558094,333.69083496)
\curveto(169.28557237,333.6408303)(169.29057237,333.58583036)(169.28058094,333.52583496)
\curveto(169.28057238,333.46583048)(169.28557237,333.40583054)(169.29558094,333.34583496)
\lineto(169.29558094,333.24083496)
\moveto(167.07558094,331.83083496)
\curveto(167.10557455,331.93083201)(167.13057453,332.05583189)(167.15058094,332.20583496)
\curveto(167.18057448,332.35583159)(167.19557446,332.50583144)(167.19558094,332.65583496)
\curveto(167.20557445,332.81583113)(167.20557445,332.97083097)(167.19558094,333.12083496)
\curveto(167.19557446,333.28083066)(167.18057448,333.41583053)(167.15058094,333.52583496)
\curveto(167.12057454,333.62583032)(167.10057456,333.72083022)(167.09058094,333.81083496)
\curveto(167.08057458,333.90083004)(167.0555746,333.98582996)(167.01558094,334.06583496)
\curveto(166.87557478,334.41582953)(166.67557498,334.71082923)(166.41558094,334.95083496)
\curveto(166.16557549,335.20082874)(165.79557586,335.32582862)(165.30558094,335.32583496)
\curveto(165.26557639,335.32582862)(165.23057643,335.32082862)(165.20058094,335.31083496)
\lineto(165.09558094,335.31083496)
\curveto(165.02557663,335.29082865)(164.9605767,335.27082867)(164.90058094,335.25083496)
\curveto(164.84057682,335.2408287)(164.78057688,335.22582872)(164.72058094,335.20583496)
\curveto(164.43057723,335.07582887)(164.21057745,334.89082905)(164.06058094,334.65083496)
\curveto(163.91057775,334.42082952)(163.78557787,334.15582979)(163.68558094,333.85583496)
\curveto(163.655578,333.77583017)(163.63557802,333.69083025)(163.62558094,333.60083496)
\curveto(163.62557803,333.52083042)(163.61557804,333.4408305)(163.59558094,333.36083496)
\curveto(163.58557807,333.33083061)(163.58057808,333.28083066)(163.58058094,333.21083496)
\curveto(163.57057809,333.17083077)(163.56557809,333.13083081)(163.56558094,333.09083496)
\curveto(163.57557808,333.05083089)(163.57557808,333.01083093)(163.56558094,332.97083496)
\curveto(163.54557811,332.89083105)(163.54057812,332.78083116)(163.55058094,332.64083496)
\curveto(163.5605781,332.50083144)(163.57557808,332.40083154)(163.59558094,332.34083496)
\curveto(163.61557804,332.25083169)(163.62557803,332.16583178)(163.62558094,332.08583496)
\curveto(163.63557802,332.00583194)(163.655578,331.92583202)(163.68558094,331.84583496)
\curveto(163.77557788,331.56583238)(163.88057778,331.32083262)(164.00058094,331.11083496)
\curveto(164.13057753,330.91083303)(164.31057735,330.7408332)(164.54058094,330.60083496)
\curveto(164.70057696,330.50083344)(164.86557679,330.43083351)(165.03558094,330.39083496)
\curveto(165.0555766,330.39083355)(165.07557658,330.38583356)(165.09558094,330.37583496)
\lineto(165.18558094,330.37583496)
\curveto(165.21557644,330.36583358)(165.26557639,330.35583359)(165.33558094,330.34583496)
\curveto(165.40557625,330.3458336)(165.46557619,330.35083359)(165.51558094,330.36083496)
\curveto(165.61557604,330.38083356)(165.70557595,330.39583355)(165.78558094,330.40583496)
\curveto(165.87557578,330.42583352)(165.9605757,330.45083349)(166.04058094,330.48083496)
\curveto(166.32057534,330.61083333)(166.53557512,330.79083315)(166.68558094,331.02083496)
\curveto(166.84557481,331.25083269)(166.97557468,331.52083242)(167.07558094,331.83083496)
}
}
{
\newrgbcolor{curcolor}{0 0 0}
\pscustom[linestyle=none,fillstyle=solid,fillcolor=curcolor]
{
\newpath
\moveto(172.76550282,339.57083496)
\curveto(172.83549987,339.49082445)(172.87049983,339.37082457)(172.87050282,339.21083496)
\lineto(172.87050282,338.74583496)
\lineto(172.87050282,338.34083496)
\curveto(172.87049983,338.20082574)(172.83549987,338.10582584)(172.76550282,338.05583496)
\curveto(172.7055,338.00582594)(172.62550008,337.97582597)(172.52550282,337.96583496)
\curveto(172.43550027,337.95582599)(172.33550037,337.95082599)(172.22550282,337.95083496)
\lineto(171.38550282,337.95083496)
\curveto(171.27550143,337.95082599)(171.17550153,337.95582599)(171.08550282,337.96583496)
\curveto(171.0055017,337.97582597)(170.93550177,338.00582594)(170.87550282,338.05583496)
\curveto(170.83550187,338.08582586)(170.8055019,338.1408258)(170.78550282,338.22083496)
\curveto(170.77550193,338.31082563)(170.76550194,338.40582554)(170.75550282,338.50583496)
\lineto(170.75550282,338.83583496)
\curveto(170.76550194,338.945825)(170.77050193,339.0408249)(170.77050282,339.12083496)
\lineto(170.77050282,339.33083496)
\curveto(170.78050192,339.40082454)(170.8005019,339.46082448)(170.83050282,339.51083496)
\curveto(170.85050185,339.55082439)(170.87550183,339.58082436)(170.90550282,339.60083496)
\lineto(171.02550282,339.66083496)
\curveto(171.04550166,339.66082428)(171.07050163,339.66082428)(171.10050282,339.66083496)
\curveto(171.13050157,339.67082427)(171.15550155,339.67582427)(171.17550282,339.67583496)
\lineto(172.27050282,339.67583496)
\curveto(172.37050033,339.67582427)(172.46550024,339.67082427)(172.55550282,339.66083496)
\curveto(172.64550006,339.65082429)(172.71549999,339.62082432)(172.76550282,339.57083496)
\moveto(172.87050282,329.80583496)
\curveto(172.87049983,329.60583434)(172.86549984,329.43583451)(172.85550282,329.29583496)
\curveto(172.84549986,329.15583479)(172.75549995,329.06083488)(172.58550282,329.01083496)
\curveto(172.52550018,328.99083495)(172.46050024,328.98083496)(172.39050282,328.98083496)
\curveto(172.32050038,328.99083495)(172.24550046,328.99583495)(172.16550282,328.99583496)
\lineto(171.32550282,328.99583496)
\curveto(171.23550147,328.99583495)(171.14550156,329.00083494)(171.05550282,329.01083496)
\curveto(170.97550173,329.02083492)(170.91550179,329.05083489)(170.87550282,329.10083496)
\curveto(170.81550189,329.17083477)(170.78050192,329.25583469)(170.77050282,329.35583496)
\lineto(170.77050282,329.70083496)
\lineto(170.77050282,336.03083496)
\lineto(170.77050282,336.33083496)
\curveto(170.77050193,336.43082751)(170.79050191,336.51082743)(170.83050282,336.57083496)
\curveto(170.89050181,336.6408273)(170.97550173,336.68582726)(171.08550282,336.70583496)
\curveto(171.1055016,336.71582723)(171.13050157,336.71582723)(171.16050282,336.70583496)
\curveto(171.2005015,336.70582724)(171.23050147,336.71082723)(171.25050282,336.72083496)
\lineto(172.00050282,336.72083496)
\lineto(172.19550282,336.72083496)
\curveto(172.27550043,336.73082721)(172.34050036,336.73082721)(172.39050282,336.72083496)
\lineto(172.51050282,336.72083496)
\curveto(172.57050013,336.70082724)(172.62550008,336.68582726)(172.67550282,336.67583496)
\curveto(172.72549998,336.66582728)(172.76549994,336.63582731)(172.79550282,336.58583496)
\curveto(172.83549987,336.53582741)(172.85549985,336.46582748)(172.85550282,336.37583496)
\curveto(172.86549984,336.28582766)(172.87049983,336.19082775)(172.87050282,336.09083496)
\lineto(172.87050282,329.80583496)
}
}
{
\newrgbcolor{curcolor}{0 0 0}
\pscustom[linestyle=none,fillstyle=solid,fillcolor=curcolor]
{
\newpath
\moveto(175.04269032,339.67583496)
\lineto(176.13769032,339.67583496)
\curveto(176.23768783,339.67582427)(176.33268774,339.67082427)(176.42269032,339.66083496)
\curveto(176.51268756,339.65082429)(176.58268749,339.62082432)(176.63269032,339.57083496)
\curveto(176.69268738,339.50082444)(176.72268735,339.40582454)(176.72269032,339.28583496)
\curveto(176.73268734,339.17582477)(176.73768733,339.06082488)(176.73769032,338.94083496)
\lineto(176.73769032,337.60583496)
\lineto(176.73769032,332.22083496)
\lineto(176.73769032,329.92583496)
\lineto(176.73769032,329.50583496)
\curveto(176.74768732,329.35583459)(176.72768734,329.2408347)(176.67769032,329.16083496)
\curveto(176.62768744,329.08083486)(176.53768753,329.02583492)(176.40769032,328.99583496)
\curveto(176.34768772,328.97583497)(176.27768779,328.97083497)(176.19769032,328.98083496)
\curveto(176.12768794,328.99083495)(176.05768801,328.99583495)(175.98769032,328.99583496)
\lineto(175.26769032,328.99583496)
\curveto(175.15768891,328.99583495)(175.05768901,329.00083494)(174.96769032,329.01083496)
\curveto(174.87768919,329.02083492)(174.80268927,329.05083489)(174.74269032,329.10083496)
\curveto(174.68268939,329.15083479)(174.64768942,329.22583472)(174.63769032,329.32583496)
\lineto(174.63769032,329.65583496)
\lineto(174.63769032,330.99083496)
\lineto(174.63769032,336.61583496)
\lineto(174.63769032,338.65583496)
\curveto(174.63768943,338.78582516)(174.63268944,338.940825)(174.62269032,339.12083496)
\curveto(174.62268945,339.30082464)(174.64768942,339.43082451)(174.69769032,339.51083496)
\curveto(174.71768935,339.55082439)(174.74268933,339.58082436)(174.77269032,339.60083496)
\lineto(174.89269032,339.66083496)
\curveto(174.91268916,339.66082428)(174.93768913,339.66082428)(174.96769032,339.66083496)
\curveto(174.99768907,339.67082427)(175.02268905,339.67582427)(175.04269032,339.67583496)
}
}
{
\newrgbcolor{curcolor}{0 0 0}
\pscustom[linestyle=none,fillstyle=solid,fillcolor=curcolor]
{
\newpath
\moveto(180.49987782,339.57083496)
\curveto(180.56987487,339.49082445)(180.60487483,339.37082457)(180.60487782,339.21083496)
\lineto(180.60487782,338.74583496)
\lineto(180.60487782,338.34083496)
\curveto(180.60487483,338.20082574)(180.56987487,338.10582584)(180.49987782,338.05583496)
\curveto(180.439875,338.00582594)(180.35987508,337.97582597)(180.25987782,337.96583496)
\curveto(180.16987527,337.95582599)(180.06987537,337.95082599)(179.95987782,337.95083496)
\lineto(179.11987782,337.95083496)
\curveto(179.00987643,337.95082599)(178.90987653,337.95582599)(178.81987782,337.96583496)
\curveto(178.7398767,337.97582597)(178.66987677,338.00582594)(178.60987782,338.05583496)
\curveto(178.56987687,338.08582586)(178.5398769,338.1408258)(178.51987782,338.22083496)
\curveto(178.50987693,338.31082563)(178.49987694,338.40582554)(178.48987782,338.50583496)
\lineto(178.48987782,338.83583496)
\curveto(178.49987694,338.945825)(178.50487693,339.0408249)(178.50487782,339.12083496)
\lineto(178.50487782,339.33083496)
\curveto(178.51487692,339.40082454)(178.5348769,339.46082448)(178.56487782,339.51083496)
\curveto(178.58487685,339.55082439)(178.60987683,339.58082436)(178.63987782,339.60083496)
\lineto(178.75987782,339.66083496)
\curveto(178.77987666,339.66082428)(178.80487663,339.66082428)(178.83487782,339.66083496)
\curveto(178.86487657,339.67082427)(178.88987655,339.67582427)(178.90987782,339.67583496)
\lineto(180.00487782,339.67583496)
\curveto(180.10487533,339.67582427)(180.19987524,339.67082427)(180.28987782,339.66083496)
\curveto(180.37987506,339.65082429)(180.44987499,339.62082432)(180.49987782,339.57083496)
\moveto(180.60487782,329.80583496)
\curveto(180.60487483,329.60583434)(180.59987484,329.43583451)(180.58987782,329.29583496)
\curveto(180.57987486,329.15583479)(180.48987495,329.06083488)(180.31987782,329.01083496)
\curveto(180.25987518,328.99083495)(180.19487524,328.98083496)(180.12487782,328.98083496)
\curveto(180.05487538,328.99083495)(179.97987546,328.99583495)(179.89987782,328.99583496)
\lineto(179.05987782,328.99583496)
\curveto(178.96987647,328.99583495)(178.87987656,329.00083494)(178.78987782,329.01083496)
\curveto(178.70987673,329.02083492)(178.64987679,329.05083489)(178.60987782,329.10083496)
\curveto(178.54987689,329.17083477)(178.51487692,329.25583469)(178.50487782,329.35583496)
\lineto(178.50487782,329.70083496)
\lineto(178.50487782,336.03083496)
\lineto(178.50487782,336.33083496)
\curveto(178.50487693,336.43082751)(178.52487691,336.51082743)(178.56487782,336.57083496)
\curveto(178.62487681,336.6408273)(178.70987673,336.68582726)(178.81987782,336.70583496)
\curveto(178.8398766,336.71582723)(178.86487657,336.71582723)(178.89487782,336.70583496)
\curveto(178.9348765,336.70582724)(178.96487647,336.71082723)(178.98487782,336.72083496)
\lineto(179.73487782,336.72083496)
\lineto(179.92987782,336.72083496)
\curveto(180.00987543,336.73082721)(180.07487536,336.73082721)(180.12487782,336.72083496)
\lineto(180.24487782,336.72083496)
\curveto(180.30487513,336.70082724)(180.35987508,336.68582726)(180.40987782,336.67583496)
\curveto(180.45987498,336.66582728)(180.49987494,336.63582731)(180.52987782,336.58583496)
\curveto(180.56987487,336.53582741)(180.58987485,336.46582748)(180.58987782,336.37583496)
\curveto(180.59987484,336.28582766)(180.60487483,336.19082775)(180.60487782,336.09083496)
\lineto(180.60487782,329.80583496)
}
}
{
\newrgbcolor{curcolor}{0 0 0}
\pscustom[linestyle=none,fillstyle=solid,fillcolor=curcolor]
{
\newpath
\moveto(189.85706532,329.83583496)
\lineto(189.85706532,329.41583496)
\curveto(189.85705695,329.28583466)(189.82705698,329.18083476)(189.76706532,329.10083496)
\curveto(189.71705709,329.05083489)(189.65205715,329.01583493)(189.57206532,328.99583496)
\curveto(189.49205731,328.98583496)(189.4020574,328.98083496)(189.30206532,328.98083496)
\lineto(188.47706532,328.98083496)
\lineto(188.19206532,328.98083496)
\curveto(188.11205869,328.99083495)(188.04705876,329.01583493)(187.99706532,329.05583496)
\curveto(187.92705888,329.10583484)(187.88705892,329.17083477)(187.87706532,329.25083496)
\curveto(187.86705894,329.33083461)(187.84705896,329.41083453)(187.81706532,329.49083496)
\curveto(187.79705901,329.51083443)(187.77705903,329.52583442)(187.75706532,329.53583496)
\curveto(187.74705906,329.55583439)(187.73205907,329.57583437)(187.71206532,329.59583496)
\curveto(187.6020592,329.59583435)(187.52205928,329.57083437)(187.47206532,329.52083496)
\lineto(187.32206532,329.37083496)
\curveto(187.25205955,329.32083462)(187.18705962,329.27583467)(187.12706532,329.23583496)
\curveto(187.06705974,329.20583474)(187.0020598,329.16583478)(186.93206532,329.11583496)
\curveto(186.89205991,329.09583485)(186.84705996,329.07583487)(186.79706532,329.05583496)
\curveto(186.75706005,329.03583491)(186.71206009,329.01583493)(186.66206532,328.99583496)
\curveto(186.52206028,328.945835)(186.37206043,328.90083504)(186.21206532,328.86083496)
\curveto(186.16206064,328.8408351)(186.11706069,328.83083511)(186.07706532,328.83083496)
\curveto(186.03706077,328.83083511)(185.99706081,328.82583512)(185.95706532,328.81583496)
\lineto(185.82206532,328.81583496)
\curveto(185.79206101,328.80583514)(185.75206105,328.80083514)(185.70206532,328.80083496)
\lineto(185.56706532,328.80083496)
\curveto(185.5070613,328.78083516)(185.41706139,328.77583517)(185.29706532,328.78583496)
\curveto(185.17706163,328.78583516)(185.09206171,328.79583515)(185.04206532,328.81583496)
\curveto(184.97206183,328.83583511)(184.9070619,328.8458351)(184.84706532,328.84583496)
\curveto(184.79706201,328.83583511)(184.74206206,328.8408351)(184.68206532,328.86083496)
\lineto(184.32206532,328.98083496)
\curveto(184.21206259,329.01083493)(184.1020627,329.05083489)(183.99206532,329.10083496)
\curveto(183.64206316,329.25083469)(183.32706348,329.48083446)(183.04706532,329.79083496)
\curveto(182.77706403,330.11083383)(182.56206424,330.4458335)(182.40206532,330.79583496)
\curveto(182.35206445,330.90583304)(182.31206449,331.01083293)(182.28206532,331.11083496)
\curveto(182.25206455,331.22083272)(182.21706459,331.33083261)(182.17706532,331.44083496)
\curveto(182.16706464,331.48083246)(182.16206464,331.51583243)(182.16206532,331.54583496)
\curveto(182.16206464,331.58583236)(182.15206465,331.63083231)(182.13206532,331.68083496)
\curveto(182.11206469,331.76083218)(182.09206471,331.8458321)(182.07206532,331.93583496)
\curveto(182.06206474,332.03583191)(182.04706476,332.13583181)(182.02706532,332.23583496)
\curveto(182.01706479,332.26583168)(182.01206479,332.30083164)(182.01206532,332.34083496)
\curveto(182.02206478,332.38083156)(182.02206478,332.41583153)(182.01206532,332.44583496)
\lineto(182.01206532,332.58083496)
\curveto(182.01206479,332.63083131)(182.0070648,332.68083126)(181.99706532,332.73083496)
\curveto(181.98706482,332.78083116)(181.98206482,332.83583111)(181.98206532,332.89583496)
\curveto(181.98206482,332.96583098)(181.98706482,333.02083092)(181.99706532,333.06083496)
\curveto(182.0070648,333.11083083)(182.01206479,333.15583079)(182.01206532,333.19583496)
\lineto(182.01206532,333.34583496)
\curveto(182.02206478,333.39583055)(182.02206478,333.4408305)(182.01206532,333.48083496)
\curveto(182.01206479,333.53083041)(182.02206478,333.58083036)(182.04206532,333.63083496)
\curveto(182.06206474,333.7408302)(182.07706473,333.8458301)(182.08706532,333.94583496)
\curveto(182.1070647,334.0458299)(182.13206467,334.1458298)(182.16206532,334.24583496)
\curveto(182.2020646,334.36582958)(182.23706457,334.48082946)(182.26706532,334.59083496)
\curveto(182.29706451,334.70082924)(182.33706447,334.81082913)(182.38706532,334.92083496)
\curveto(182.52706428,335.22082872)(182.7020641,335.50582844)(182.91206532,335.77583496)
\curveto(182.93206387,335.80582814)(182.95706385,335.83082811)(182.98706532,335.85083496)
\curveto(183.02706378,335.88082806)(183.05706375,335.91082803)(183.07706532,335.94083496)
\curveto(183.11706369,335.99082795)(183.15706365,336.03582791)(183.19706532,336.07583496)
\curveto(183.23706357,336.11582783)(183.28206352,336.15582779)(183.33206532,336.19583496)
\curveto(183.37206343,336.21582773)(183.4070634,336.2408277)(183.43706532,336.27083496)
\curveto(183.46706334,336.31082763)(183.5020633,336.3408276)(183.54206532,336.36083496)
\curveto(183.79206301,336.53082741)(184.08206272,336.67082727)(184.41206532,336.78083496)
\curveto(184.48206232,336.80082714)(184.55206225,336.81582713)(184.62206532,336.82583496)
\curveto(184.7020621,336.83582711)(184.78206202,336.85082709)(184.86206532,336.87083496)
\curveto(184.93206187,336.89082705)(185.02206178,336.90082704)(185.13206532,336.90083496)
\curveto(185.24206156,336.91082703)(185.35206145,336.91582703)(185.46206532,336.91583496)
\curveto(185.57206123,336.91582703)(185.67706113,336.91082703)(185.77706532,336.90083496)
\curveto(185.88706092,336.89082705)(185.97706083,336.87582707)(186.04706532,336.85583496)
\curveto(186.19706061,336.80582714)(186.34206046,336.76082718)(186.48206532,336.72083496)
\curveto(186.62206018,336.68082726)(186.75206005,336.62582732)(186.87206532,336.55583496)
\curveto(186.94205986,336.50582744)(187.0070598,336.45582749)(187.06706532,336.40583496)
\curveto(187.12705968,336.36582758)(187.19205961,336.32082762)(187.26206532,336.27083496)
\curveto(187.3020595,336.2408277)(187.35705945,336.20082774)(187.42706532,336.15083496)
\curveto(187.5070593,336.10082784)(187.58205922,336.10082784)(187.65206532,336.15083496)
\curveto(187.69205911,336.17082777)(187.71205909,336.20582774)(187.71206532,336.25583496)
\curveto(187.71205909,336.30582764)(187.72205908,336.35582759)(187.74206532,336.40583496)
\lineto(187.74206532,336.55583496)
\curveto(187.75205905,336.58582736)(187.75705905,336.62082732)(187.75706532,336.66083496)
\lineto(187.75706532,336.78083496)
\lineto(187.75706532,338.82083496)
\curveto(187.75705905,338.93082501)(187.75205905,339.05082489)(187.74206532,339.18083496)
\curveto(187.74205906,339.32082462)(187.76705904,339.42582452)(187.81706532,339.49583496)
\curveto(187.85705895,339.57582437)(187.93205887,339.62582432)(188.04206532,339.64583496)
\curveto(188.06205874,339.65582429)(188.08205872,339.65582429)(188.10206532,339.64583496)
\curveto(188.12205868,339.6458243)(188.14205866,339.65082429)(188.16206532,339.66083496)
\lineto(189.22706532,339.66083496)
\curveto(189.34705746,339.66082428)(189.45705735,339.65582429)(189.55706532,339.64583496)
\curveto(189.65705715,339.63582431)(189.73205707,339.59582435)(189.78206532,339.52583496)
\curveto(189.83205697,339.4458245)(189.85705695,339.3408246)(189.85706532,339.21083496)
\lineto(189.85706532,338.85083496)
\lineto(189.85706532,329.83583496)
\moveto(187.81706532,332.77583496)
\curveto(187.82705898,332.81583113)(187.82705898,332.85583109)(187.81706532,332.89583496)
\lineto(187.81706532,333.03083496)
\curveto(187.81705899,333.13083081)(187.81205899,333.23083071)(187.80206532,333.33083496)
\curveto(187.79205901,333.43083051)(187.77705903,333.52083042)(187.75706532,333.60083496)
\curveto(187.73705907,333.71083023)(187.71705909,333.81083013)(187.69706532,333.90083496)
\curveto(187.68705912,333.99082995)(187.66205914,334.07582987)(187.62206532,334.15583496)
\curveto(187.48205932,334.51582943)(187.27705953,334.80082914)(187.00706532,335.01083496)
\curveto(186.74706006,335.22082872)(186.36706044,335.32582862)(185.86706532,335.32583496)
\curveto(185.807061,335.32582862)(185.72706108,335.31582863)(185.62706532,335.29583496)
\curveto(185.54706126,335.27582867)(185.47206133,335.25582869)(185.40206532,335.23583496)
\curveto(185.34206146,335.22582872)(185.28206152,335.20582874)(185.22206532,335.17583496)
\curveto(184.95206185,335.06582888)(184.74206206,334.89582905)(184.59206532,334.66583496)
\curveto(184.44206236,334.43582951)(184.32206248,334.17582977)(184.23206532,333.88583496)
\curveto(184.2020626,333.78583016)(184.18206262,333.68583026)(184.17206532,333.58583496)
\curveto(184.16206264,333.48583046)(184.14206266,333.38083056)(184.11206532,333.27083496)
\lineto(184.11206532,333.06083496)
\curveto(184.09206271,332.97083097)(184.08706272,332.8458311)(184.09706532,332.68583496)
\curveto(184.1070627,332.53583141)(184.12206268,332.42583152)(184.14206532,332.35583496)
\lineto(184.14206532,332.26583496)
\curveto(184.15206265,332.2458317)(184.15706265,332.22583172)(184.15706532,332.20583496)
\curveto(184.17706263,332.12583182)(184.19206261,332.05083189)(184.20206532,331.98083496)
\curveto(184.22206258,331.91083203)(184.24206256,331.83583211)(184.26206532,331.75583496)
\curveto(184.43206237,331.23583271)(184.72206208,330.85083309)(185.13206532,330.60083496)
\curveto(185.26206154,330.51083343)(185.44206136,330.4408335)(185.67206532,330.39083496)
\curveto(185.71206109,330.38083356)(185.77206103,330.37583357)(185.85206532,330.37583496)
\curveto(185.88206092,330.36583358)(185.92706088,330.35583359)(185.98706532,330.34583496)
\curveto(186.05706075,330.3458336)(186.11206069,330.35083359)(186.15206532,330.36083496)
\curveto(186.23206057,330.38083356)(186.31206049,330.39583355)(186.39206532,330.40583496)
\curveto(186.47206033,330.41583353)(186.55206025,330.43583351)(186.63206532,330.46583496)
\curveto(186.88205992,330.57583337)(187.08205972,330.71583323)(187.23206532,330.88583496)
\curveto(187.38205942,331.05583289)(187.51205929,331.27083267)(187.62206532,331.53083496)
\curveto(187.66205914,331.62083232)(187.69205911,331.71083223)(187.71206532,331.80083496)
\curveto(187.73205907,331.90083204)(187.75205905,332.00583194)(187.77206532,332.11583496)
\curveto(187.78205902,332.16583178)(187.78205902,332.21083173)(187.77206532,332.25083496)
\curveto(187.77205903,332.30083164)(187.78205902,332.35083159)(187.80206532,332.40083496)
\curveto(187.81205899,332.43083151)(187.81705899,332.46583148)(187.81706532,332.50583496)
\lineto(187.81706532,332.64083496)
\lineto(187.81706532,332.77583496)
}
}
{
\newrgbcolor{curcolor}{0 0 0}
\pscustom[linestyle=none,fillstyle=solid,fillcolor=curcolor]
{
\newpath
\moveto(198.48698719,329.58083496)
\curveto(198.50697934,329.47083447)(198.51697933,329.36083458)(198.51698719,329.25083496)
\curveto(198.52697932,329.1408348)(198.47697937,329.06583488)(198.36698719,329.02583496)
\curveto(198.30697954,328.99583495)(198.23697961,328.98083496)(198.15698719,328.98083496)
\lineto(197.91698719,328.98083496)
\lineto(197.10698719,328.98083496)
\lineto(196.83698719,328.98083496)
\curveto(196.75698109,328.99083495)(196.69198116,329.01583493)(196.64198719,329.05583496)
\curveto(196.57198128,329.09583485)(196.51698133,329.15083479)(196.47698719,329.22083496)
\curveto(196.4469814,329.30083464)(196.40198145,329.36583458)(196.34198719,329.41583496)
\curveto(196.32198153,329.43583451)(196.29698155,329.45083449)(196.26698719,329.46083496)
\curveto(196.23698161,329.48083446)(196.19698165,329.48583446)(196.14698719,329.47583496)
\curveto(196.09698175,329.45583449)(196.0469818,329.43083451)(195.99698719,329.40083496)
\curveto(195.95698189,329.37083457)(195.91198194,329.3458346)(195.86198719,329.32583496)
\curveto(195.81198204,329.28583466)(195.75698209,329.25083469)(195.69698719,329.22083496)
\lineto(195.51698719,329.13083496)
\curveto(195.38698246,329.07083487)(195.2519826,329.02083492)(195.11198719,328.98083496)
\curveto(194.97198288,328.95083499)(194.82698302,328.91583503)(194.67698719,328.87583496)
\curveto(194.60698324,328.85583509)(194.53698331,328.8458351)(194.46698719,328.84583496)
\curveto(194.40698344,328.83583511)(194.34198351,328.82583512)(194.27198719,328.81583496)
\lineto(194.18198719,328.81583496)
\curveto(194.1519837,328.80583514)(194.12198373,328.80083514)(194.09198719,328.80083496)
\lineto(193.92698719,328.80083496)
\curveto(193.82698402,328.78083516)(193.72698412,328.78083516)(193.62698719,328.80083496)
\lineto(193.49198719,328.80083496)
\curveto(193.42198443,328.82083512)(193.3519845,328.83083511)(193.28198719,328.83083496)
\curveto(193.22198463,328.82083512)(193.16198469,328.82583512)(193.10198719,328.84583496)
\curveto(193.00198485,328.86583508)(192.90698494,328.88583506)(192.81698719,328.90583496)
\curveto(192.72698512,328.91583503)(192.64198521,328.940835)(192.56198719,328.98083496)
\curveto(192.27198558,329.09083485)(192.02198583,329.23083471)(191.81198719,329.40083496)
\curveto(191.61198624,329.58083436)(191.4519864,329.81583413)(191.33198719,330.10583496)
\curveto(191.30198655,330.17583377)(191.27198658,330.25083369)(191.24198719,330.33083496)
\curveto(191.22198663,330.41083353)(191.20198665,330.49583345)(191.18198719,330.58583496)
\curveto(191.16198669,330.63583331)(191.1519867,330.68583326)(191.15198719,330.73583496)
\curveto(191.16198669,330.78583316)(191.16198669,330.83583311)(191.15198719,330.88583496)
\curveto(191.14198671,330.91583303)(191.13198672,330.97583297)(191.12198719,331.06583496)
\curveto(191.12198673,331.16583278)(191.12698672,331.23583271)(191.13698719,331.27583496)
\curveto(191.15698669,331.37583257)(191.16698668,331.46083248)(191.16698719,331.53083496)
\lineto(191.25698719,331.86083496)
\curveto(191.28698656,331.98083196)(191.32698652,332.08583186)(191.37698719,332.17583496)
\curveto(191.5469863,332.46583148)(191.74198611,332.68583126)(191.96198719,332.83583496)
\curveto(192.18198567,332.98583096)(192.46198539,333.11583083)(192.80198719,333.22583496)
\curveto(192.93198492,333.27583067)(193.06698478,333.31083063)(193.20698719,333.33083496)
\curveto(193.3469845,333.35083059)(193.48698436,333.37583057)(193.62698719,333.40583496)
\curveto(193.70698414,333.42583052)(193.79198406,333.43583051)(193.88198719,333.43583496)
\curveto(193.97198388,333.4458305)(194.06198379,333.46083048)(194.15198719,333.48083496)
\curveto(194.22198363,333.50083044)(194.29198356,333.50583044)(194.36198719,333.49583496)
\curveto(194.43198342,333.49583045)(194.50698334,333.50583044)(194.58698719,333.52583496)
\curveto(194.65698319,333.5458304)(194.72698312,333.55583039)(194.79698719,333.55583496)
\curveto(194.86698298,333.55583039)(194.94198291,333.56583038)(195.02198719,333.58583496)
\curveto(195.23198262,333.63583031)(195.42198243,333.67583027)(195.59198719,333.70583496)
\curveto(195.77198208,333.7458302)(195.93198192,333.83583011)(196.07198719,333.97583496)
\curveto(196.16198169,334.06582988)(196.22198163,334.16582978)(196.25198719,334.27583496)
\curveto(196.26198159,334.30582964)(196.26198159,334.33082961)(196.25198719,334.35083496)
\curveto(196.2519816,334.37082957)(196.25698159,334.39082955)(196.26698719,334.41083496)
\curveto(196.27698157,334.43082951)(196.28198157,334.46082948)(196.28198719,334.50083496)
\lineto(196.28198719,334.59083496)
\lineto(196.25198719,334.71083496)
\curveto(196.2519816,334.75082919)(196.2469816,334.78582916)(196.23698719,334.81583496)
\curveto(196.13698171,335.11582883)(195.92698192,335.32082862)(195.60698719,335.43083496)
\curveto(195.51698233,335.46082848)(195.40698244,335.48082846)(195.27698719,335.49083496)
\curveto(195.15698269,335.51082843)(195.03198282,335.51582843)(194.90198719,335.50583496)
\curveto(194.77198308,335.50582844)(194.6469832,335.49582845)(194.52698719,335.47583496)
\curveto(194.40698344,335.45582849)(194.30198355,335.43082851)(194.21198719,335.40083496)
\curveto(194.1519837,335.38082856)(194.09198376,335.35082859)(194.03198719,335.31083496)
\curveto(193.98198387,335.28082866)(193.93198392,335.2458287)(193.88198719,335.20583496)
\curveto(193.83198402,335.16582878)(193.77698407,335.11082883)(193.71698719,335.04083496)
\curveto(193.66698418,334.97082897)(193.63198422,334.90582904)(193.61198719,334.84583496)
\curveto(193.56198429,334.7458292)(193.51698433,334.65082929)(193.47698719,334.56083496)
\curveto(193.4469844,334.47082947)(193.37698447,334.41082953)(193.26698719,334.38083496)
\curveto(193.18698466,334.36082958)(193.10198475,334.35082959)(193.01198719,334.35083496)
\lineto(192.74198719,334.35083496)
\lineto(192.17198719,334.35083496)
\curveto(192.12198573,334.35082959)(192.07198578,334.3458296)(192.02198719,334.33583496)
\curveto(191.97198588,334.33582961)(191.92698592,334.3408296)(191.88698719,334.35083496)
\lineto(191.75198719,334.35083496)
\curveto(191.73198612,334.36082958)(191.70698614,334.36582958)(191.67698719,334.36583496)
\curveto(191.6469862,334.36582958)(191.62198623,334.37582957)(191.60198719,334.39583496)
\curveto(191.52198633,334.41582953)(191.46698638,334.48082946)(191.43698719,334.59083496)
\curveto(191.42698642,334.6408293)(191.42698642,334.69082925)(191.43698719,334.74083496)
\curveto(191.4469864,334.79082915)(191.45698639,334.83582911)(191.46698719,334.87583496)
\curveto(191.49698635,334.98582896)(191.52698632,335.08582886)(191.55698719,335.17583496)
\curveto(191.59698625,335.27582867)(191.64198621,335.36582858)(191.69198719,335.44583496)
\lineto(191.78198719,335.59583496)
\lineto(191.87198719,335.74583496)
\curveto(191.9519859,335.85582809)(192.0519858,335.96082798)(192.17198719,336.06083496)
\curveto(192.19198566,336.07082787)(192.22198563,336.09582785)(192.26198719,336.13583496)
\curveto(192.31198554,336.17582777)(192.35698549,336.21082773)(192.39698719,336.24083496)
\curveto(192.43698541,336.27082767)(192.48198537,336.30082764)(192.53198719,336.33083496)
\curveto(192.70198515,336.4408275)(192.88198497,336.52582742)(193.07198719,336.58583496)
\curveto(193.26198459,336.65582729)(193.45698439,336.72082722)(193.65698719,336.78083496)
\curveto(193.77698407,336.81082713)(193.90198395,336.83082711)(194.03198719,336.84083496)
\curveto(194.16198369,336.85082709)(194.29198356,336.87082707)(194.42198719,336.90083496)
\curveto(194.46198339,336.91082703)(194.52198333,336.91082703)(194.60198719,336.90083496)
\curveto(194.69198316,336.89082705)(194.7469831,336.89582705)(194.76698719,336.91583496)
\curveto(195.17698267,336.92582702)(195.56698228,336.91082703)(195.93698719,336.87083496)
\curveto(196.31698153,336.83082711)(196.65698119,336.75582719)(196.95698719,336.64583496)
\curveto(197.26698058,336.53582741)(197.53198032,336.38582756)(197.75198719,336.19583496)
\curveto(197.97197988,336.01582793)(198.14197971,335.78082816)(198.26198719,335.49083496)
\curveto(198.33197952,335.32082862)(198.37197948,335.12582882)(198.38198719,334.90583496)
\curveto(198.39197946,334.68582926)(198.39697945,334.46082948)(198.39698719,334.23083496)
\lineto(198.39698719,330.88583496)
\lineto(198.39698719,330.30083496)
\curveto(198.39697945,330.11083383)(198.41697943,329.93583401)(198.45698719,329.77583496)
\curveto(198.46697938,329.7458342)(198.47197938,329.71083423)(198.47198719,329.67083496)
\curveto(198.47197938,329.6408343)(198.47697937,329.61083433)(198.48698719,329.58083496)
\moveto(196.28198719,331.89083496)
\curveto(196.29198156,331.940832)(196.29698155,331.99583195)(196.29698719,332.05583496)
\curveto(196.29698155,332.12583182)(196.29198156,332.18583176)(196.28198719,332.23583496)
\curveto(196.26198159,332.29583165)(196.2519816,332.35083159)(196.25198719,332.40083496)
\curveto(196.2519816,332.45083149)(196.23198162,332.49083145)(196.19198719,332.52083496)
\curveto(196.14198171,332.56083138)(196.06698178,332.58083136)(195.96698719,332.58083496)
\curveto(195.92698192,332.57083137)(195.89198196,332.56083138)(195.86198719,332.55083496)
\curveto(195.83198202,332.55083139)(195.79698205,332.5458314)(195.75698719,332.53583496)
\curveto(195.68698216,332.51583143)(195.61198224,332.50083144)(195.53198719,332.49083496)
\curveto(195.4519824,332.48083146)(195.37198248,332.46583148)(195.29198719,332.44583496)
\curveto(195.26198259,332.43583151)(195.21698263,332.43083151)(195.15698719,332.43083496)
\curveto(195.02698282,332.40083154)(194.89698295,332.38083156)(194.76698719,332.37083496)
\curveto(194.63698321,332.36083158)(194.51198334,332.33583161)(194.39198719,332.29583496)
\curveto(194.31198354,332.27583167)(194.23698361,332.25583169)(194.16698719,332.23583496)
\curveto(194.09698375,332.22583172)(194.02698382,332.20583174)(193.95698719,332.17583496)
\curveto(193.7469841,332.08583186)(193.56698428,331.95083199)(193.41698719,331.77083496)
\curveto(193.27698457,331.59083235)(193.22698462,331.3408326)(193.26698719,331.02083496)
\curveto(193.28698456,330.85083309)(193.34198451,330.71083323)(193.43198719,330.60083496)
\curveto(193.50198435,330.49083345)(193.60698424,330.40083354)(193.74698719,330.33083496)
\curveto(193.88698396,330.27083367)(194.03698381,330.22583372)(194.19698719,330.19583496)
\curveto(194.36698348,330.16583378)(194.54198331,330.15583379)(194.72198719,330.16583496)
\curveto(194.91198294,330.18583376)(195.08698276,330.22083372)(195.24698719,330.27083496)
\curveto(195.50698234,330.35083359)(195.71198214,330.47583347)(195.86198719,330.64583496)
\curveto(196.01198184,330.82583312)(196.12698172,331.0458329)(196.20698719,331.30583496)
\curveto(196.22698162,331.37583257)(196.23698161,331.4458325)(196.23698719,331.51583496)
\curveto(196.2469816,331.59583235)(196.26198159,331.67583227)(196.28198719,331.75583496)
\lineto(196.28198719,331.89083496)
}
}
{
\newrgbcolor{curcolor}{0 0 0}
\pscustom[linestyle=none,fillstyle=solid,fillcolor=curcolor]
{
\newpath
\moveto(207.64026844,329.83583496)
\lineto(207.64026844,329.41583496)
\curveto(207.64026007,329.28583466)(207.6102601,329.18083476)(207.55026844,329.10083496)
\curveto(207.50026021,329.05083489)(207.43526028,329.01583493)(207.35526844,328.99583496)
\curveto(207.27526044,328.98583496)(207.18526053,328.98083496)(207.08526844,328.98083496)
\lineto(206.26026844,328.98083496)
\lineto(205.97526844,328.98083496)
\curveto(205.89526182,328.99083495)(205.83026188,329.01583493)(205.78026844,329.05583496)
\curveto(205.710262,329.10583484)(205.67026204,329.17083477)(205.66026844,329.25083496)
\curveto(205.65026206,329.33083461)(205.63026208,329.41083453)(205.60026844,329.49083496)
\curveto(205.58026213,329.51083443)(205.56026215,329.52583442)(205.54026844,329.53583496)
\curveto(205.53026218,329.55583439)(205.5152622,329.57583437)(205.49526844,329.59583496)
\curveto(205.38526233,329.59583435)(205.30526241,329.57083437)(205.25526844,329.52083496)
\lineto(205.10526844,329.37083496)
\curveto(205.03526268,329.32083462)(204.97026274,329.27583467)(204.91026844,329.23583496)
\curveto(204.85026286,329.20583474)(204.78526293,329.16583478)(204.71526844,329.11583496)
\curveto(204.67526304,329.09583485)(204.63026308,329.07583487)(204.58026844,329.05583496)
\curveto(204.54026317,329.03583491)(204.49526322,329.01583493)(204.44526844,328.99583496)
\curveto(204.30526341,328.945835)(204.15526356,328.90083504)(203.99526844,328.86083496)
\curveto(203.94526377,328.8408351)(203.90026381,328.83083511)(203.86026844,328.83083496)
\curveto(203.82026389,328.83083511)(203.78026393,328.82583512)(203.74026844,328.81583496)
\lineto(203.60526844,328.81583496)
\curveto(203.57526414,328.80583514)(203.53526418,328.80083514)(203.48526844,328.80083496)
\lineto(203.35026844,328.80083496)
\curveto(203.29026442,328.78083516)(203.20026451,328.77583517)(203.08026844,328.78583496)
\curveto(202.96026475,328.78583516)(202.87526484,328.79583515)(202.82526844,328.81583496)
\curveto(202.75526496,328.83583511)(202.69026502,328.8458351)(202.63026844,328.84583496)
\curveto(202.58026513,328.83583511)(202.52526519,328.8408351)(202.46526844,328.86083496)
\lineto(202.10526844,328.98083496)
\curveto(201.99526572,329.01083493)(201.88526583,329.05083489)(201.77526844,329.10083496)
\curveto(201.42526629,329.25083469)(201.1102666,329.48083446)(200.83026844,329.79083496)
\curveto(200.56026715,330.11083383)(200.34526737,330.4458335)(200.18526844,330.79583496)
\curveto(200.13526758,330.90583304)(200.09526762,331.01083293)(200.06526844,331.11083496)
\curveto(200.03526768,331.22083272)(200.00026771,331.33083261)(199.96026844,331.44083496)
\curveto(199.95026776,331.48083246)(199.94526777,331.51583243)(199.94526844,331.54583496)
\curveto(199.94526777,331.58583236)(199.93526778,331.63083231)(199.91526844,331.68083496)
\curveto(199.89526782,331.76083218)(199.87526784,331.8458321)(199.85526844,331.93583496)
\curveto(199.84526787,332.03583191)(199.83026788,332.13583181)(199.81026844,332.23583496)
\curveto(199.80026791,332.26583168)(199.79526792,332.30083164)(199.79526844,332.34083496)
\curveto(199.80526791,332.38083156)(199.80526791,332.41583153)(199.79526844,332.44583496)
\lineto(199.79526844,332.58083496)
\curveto(199.79526792,332.63083131)(199.79026792,332.68083126)(199.78026844,332.73083496)
\curveto(199.77026794,332.78083116)(199.76526795,332.83583111)(199.76526844,332.89583496)
\curveto(199.76526795,332.96583098)(199.77026794,333.02083092)(199.78026844,333.06083496)
\curveto(199.79026792,333.11083083)(199.79526792,333.15583079)(199.79526844,333.19583496)
\lineto(199.79526844,333.34583496)
\curveto(199.80526791,333.39583055)(199.80526791,333.4408305)(199.79526844,333.48083496)
\curveto(199.79526792,333.53083041)(199.80526791,333.58083036)(199.82526844,333.63083496)
\curveto(199.84526787,333.7408302)(199.86026785,333.8458301)(199.87026844,333.94583496)
\curveto(199.89026782,334.0458299)(199.9152678,334.1458298)(199.94526844,334.24583496)
\curveto(199.98526773,334.36582958)(200.02026769,334.48082946)(200.05026844,334.59083496)
\curveto(200.08026763,334.70082924)(200.12026759,334.81082913)(200.17026844,334.92083496)
\curveto(200.3102674,335.22082872)(200.48526723,335.50582844)(200.69526844,335.77583496)
\curveto(200.715267,335.80582814)(200.74026697,335.83082811)(200.77026844,335.85083496)
\curveto(200.8102669,335.88082806)(200.84026687,335.91082803)(200.86026844,335.94083496)
\curveto(200.90026681,335.99082795)(200.94026677,336.03582791)(200.98026844,336.07583496)
\curveto(201.02026669,336.11582783)(201.06526665,336.15582779)(201.11526844,336.19583496)
\curveto(201.15526656,336.21582773)(201.19026652,336.2408277)(201.22026844,336.27083496)
\curveto(201.25026646,336.31082763)(201.28526643,336.3408276)(201.32526844,336.36083496)
\curveto(201.57526614,336.53082741)(201.86526585,336.67082727)(202.19526844,336.78083496)
\curveto(202.26526545,336.80082714)(202.33526538,336.81582713)(202.40526844,336.82583496)
\curveto(202.48526523,336.83582711)(202.56526515,336.85082709)(202.64526844,336.87083496)
\curveto(202.715265,336.89082705)(202.80526491,336.90082704)(202.91526844,336.90083496)
\curveto(203.02526469,336.91082703)(203.13526458,336.91582703)(203.24526844,336.91583496)
\curveto(203.35526436,336.91582703)(203.46026425,336.91082703)(203.56026844,336.90083496)
\curveto(203.67026404,336.89082705)(203.76026395,336.87582707)(203.83026844,336.85583496)
\curveto(203.98026373,336.80582714)(204.12526359,336.76082718)(204.26526844,336.72083496)
\curveto(204.40526331,336.68082726)(204.53526318,336.62582732)(204.65526844,336.55583496)
\curveto(204.72526299,336.50582744)(204.79026292,336.45582749)(204.85026844,336.40583496)
\curveto(204.9102628,336.36582758)(204.97526274,336.32082762)(205.04526844,336.27083496)
\curveto(205.08526263,336.2408277)(205.14026257,336.20082774)(205.21026844,336.15083496)
\curveto(205.29026242,336.10082784)(205.36526235,336.10082784)(205.43526844,336.15083496)
\curveto(205.47526224,336.17082777)(205.49526222,336.20582774)(205.49526844,336.25583496)
\curveto(205.49526222,336.30582764)(205.50526221,336.35582759)(205.52526844,336.40583496)
\lineto(205.52526844,336.55583496)
\curveto(205.53526218,336.58582736)(205.54026217,336.62082732)(205.54026844,336.66083496)
\lineto(205.54026844,336.78083496)
\lineto(205.54026844,338.82083496)
\curveto(205.54026217,338.93082501)(205.53526218,339.05082489)(205.52526844,339.18083496)
\curveto(205.52526219,339.32082462)(205.55026216,339.42582452)(205.60026844,339.49583496)
\curveto(205.64026207,339.57582437)(205.715262,339.62582432)(205.82526844,339.64583496)
\curveto(205.84526187,339.65582429)(205.86526185,339.65582429)(205.88526844,339.64583496)
\curveto(205.90526181,339.6458243)(205.92526179,339.65082429)(205.94526844,339.66083496)
\lineto(207.01026844,339.66083496)
\curveto(207.13026058,339.66082428)(207.24026047,339.65582429)(207.34026844,339.64583496)
\curveto(207.44026027,339.63582431)(207.5152602,339.59582435)(207.56526844,339.52583496)
\curveto(207.6152601,339.4458245)(207.64026007,339.3408246)(207.64026844,339.21083496)
\lineto(207.64026844,338.85083496)
\lineto(207.64026844,329.83583496)
\moveto(205.60026844,332.77583496)
\curveto(205.6102621,332.81583113)(205.6102621,332.85583109)(205.60026844,332.89583496)
\lineto(205.60026844,333.03083496)
\curveto(205.60026211,333.13083081)(205.59526212,333.23083071)(205.58526844,333.33083496)
\curveto(205.57526214,333.43083051)(205.56026215,333.52083042)(205.54026844,333.60083496)
\curveto(205.52026219,333.71083023)(205.50026221,333.81083013)(205.48026844,333.90083496)
\curveto(205.47026224,333.99082995)(205.44526227,334.07582987)(205.40526844,334.15583496)
\curveto(205.26526245,334.51582943)(205.06026265,334.80082914)(204.79026844,335.01083496)
\curveto(204.53026318,335.22082872)(204.15026356,335.32582862)(203.65026844,335.32583496)
\curveto(203.59026412,335.32582862)(203.5102642,335.31582863)(203.41026844,335.29583496)
\curveto(203.33026438,335.27582867)(203.25526446,335.25582869)(203.18526844,335.23583496)
\curveto(203.12526459,335.22582872)(203.06526465,335.20582874)(203.00526844,335.17583496)
\curveto(202.73526498,335.06582888)(202.52526519,334.89582905)(202.37526844,334.66583496)
\curveto(202.22526549,334.43582951)(202.10526561,334.17582977)(202.01526844,333.88583496)
\curveto(201.98526573,333.78583016)(201.96526575,333.68583026)(201.95526844,333.58583496)
\curveto(201.94526577,333.48583046)(201.92526579,333.38083056)(201.89526844,333.27083496)
\lineto(201.89526844,333.06083496)
\curveto(201.87526584,332.97083097)(201.87026584,332.8458311)(201.88026844,332.68583496)
\curveto(201.89026582,332.53583141)(201.90526581,332.42583152)(201.92526844,332.35583496)
\lineto(201.92526844,332.26583496)
\curveto(201.93526578,332.2458317)(201.94026577,332.22583172)(201.94026844,332.20583496)
\curveto(201.96026575,332.12583182)(201.97526574,332.05083189)(201.98526844,331.98083496)
\curveto(202.00526571,331.91083203)(202.02526569,331.83583211)(202.04526844,331.75583496)
\curveto(202.2152655,331.23583271)(202.50526521,330.85083309)(202.91526844,330.60083496)
\curveto(203.04526467,330.51083343)(203.22526449,330.4408335)(203.45526844,330.39083496)
\curveto(203.49526422,330.38083356)(203.55526416,330.37583357)(203.63526844,330.37583496)
\curveto(203.66526405,330.36583358)(203.710264,330.35583359)(203.77026844,330.34583496)
\curveto(203.84026387,330.3458336)(203.89526382,330.35083359)(203.93526844,330.36083496)
\curveto(204.0152637,330.38083356)(204.09526362,330.39583355)(204.17526844,330.40583496)
\curveto(204.25526346,330.41583353)(204.33526338,330.43583351)(204.41526844,330.46583496)
\curveto(204.66526305,330.57583337)(204.86526285,330.71583323)(205.01526844,330.88583496)
\curveto(205.16526255,331.05583289)(205.29526242,331.27083267)(205.40526844,331.53083496)
\curveto(205.44526227,331.62083232)(205.47526224,331.71083223)(205.49526844,331.80083496)
\curveto(205.5152622,331.90083204)(205.53526218,332.00583194)(205.55526844,332.11583496)
\curveto(205.56526215,332.16583178)(205.56526215,332.21083173)(205.55526844,332.25083496)
\curveto(205.55526216,332.30083164)(205.56526215,332.35083159)(205.58526844,332.40083496)
\curveto(205.59526212,332.43083151)(205.60026211,332.46583148)(205.60026844,332.50583496)
\lineto(205.60026844,332.64083496)
\lineto(205.60026844,332.77583496)
}
}
{
\newrgbcolor{curcolor}{0 0 0}
\pscustom[linestyle=none,fillstyle=solid,fillcolor=curcolor]
{
\newpath
\moveto(400.43525391,339.66083496)
\lineto(401.72525391,339.66083496)
\curveto(401.83525108,339.66082428)(401.94025098,339.65582429)(402.04025391,339.64583496)
\curveto(402.14025078,339.6458243)(402.2152507,339.61082433)(402.26525391,339.54083496)
\curveto(402.3152506,339.47082447)(402.34025058,339.38082456)(402.34025391,339.27083496)
\curveto(402.35025057,339.16082478)(402.35525056,339.0408249)(402.35525391,338.91083496)
\lineto(402.35525391,337.60583496)
\lineto(402.35525391,332.40083496)
\lineto(402.35525391,329.94083496)
\lineto(402.35525391,329.50583496)
\curveto(402.36525055,329.3458346)(402.34525057,329.22583472)(402.29525391,329.14583496)
\curveto(402.25525066,329.07583487)(402.16525075,329.02083492)(402.02525391,328.98083496)
\curveto(401.95525096,328.96083498)(401.88025104,328.95583499)(401.80025391,328.96583496)
\curveto(401.7202512,328.97583497)(401.64025128,328.98083496)(401.56025391,328.98083496)
\lineto(400.67525391,328.98083496)
\curveto(400.56525235,328.98083496)(400.46025246,328.98583496)(400.36025391,328.99583496)
\curveto(400.27025265,329.00583494)(400.19525272,329.03583491)(400.13525391,329.08583496)
\curveto(400.08525283,329.13583481)(400.05525286,329.21083473)(400.04525391,329.31083496)
\curveto(400.03525288,329.41083453)(400.03025289,329.51583443)(400.03025391,329.62583496)
\lineto(400.03025391,330.93083496)
\lineto(400.03025391,336.40583496)
\lineto(400.03025391,338.59583496)
\curveto(400.03025289,338.73582521)(400.02525289,338.90082504)(400.01525391,339.09083496)
\curveto(400.0152529,339.28082466)(400.04025288,339.41582453)(400.09025391,339.49583496)
\curveto(400.13025279,339.55582439)(400.19525272,339.60582434)(400.28525391,339.64583496)
\curveto(400.3152526,339.6458243)(400.34025258,339.6458243)(400.36025391,339.64583496)
\curveto(400.39025253,339.65582429)(400.4152525,339.66082428)(400.43525391,339.66083496)
}
}
{
\newrgbcolor{curcolor}{0 0 0}
\pscustom[linestyle=none,fillstyle=solid,fillcolor=curcolor]
{
\newpath
\moveto(408.63908203,336.91583496)
\curveto(409.23907623,336.93582701)(409.73907573,336.85082709)(410.13908203,336.66083496)
\curveto(410.53907493,336.47082747)(410.85407461,336.19082775)(411.08408203,335.82083496)
\curveto(411.15407431,335.71082823)(411.20907426,335.59082835)(411.24908203,335.46083496)
\curveto(411.28907418,335.3408286)(411.32907414,335.21582873)(411.36908203,335.08583496)
\curveto(411.38907408,335.00582894)(411.39907407,334.93082901)(411.39908203,334.86083496)
\curveto(411.40907406,334.79082915)(411.42407404,334.72082922)(411.44408203,334.65083496)
\curveto(411.44407402,334.59082935)(411.44907402,334.55082939)(411.45908203,334.53083496)
\curveto(411.47907399,334.39082955)(411.48907398,334.2458297)(411.48908203,334.09583496)
\lineto(411.48908203,333.66083496)
\lineto(411.48908203,332.32583496)
\lineto(411.48908203,329.89583496)
\curveto(411.48907398,329.70583424)(411.48407398,329.52083442)(411.47408203,329.34083496)
\curveto(411.47407399,329.17083477)(411.40407406,329.06083488)(411.26408203,329.01083496)
\curveto(411.20407426,328.99083495)(411.13407433,328.98083496)(411.05408203,328.98083496)
\lineto(410.81408203,328.98083496)
\lineto(410.00408203,328.98083496)
\curveto(409.88407558,328.98083496)(409.77407569,328.98583496)(409.67408203,328.99583496)
\curveto(409.58407588,329.01583493)(409.51407595,329.06083488)(409.46408203,329.13083496)
\curveto(409.42407604,329.19083475)(409.39907607,329.26583468)(409.38908203,329.35583496)
\lineto(409.38908203,329.67083496)
\lineto(409.38908203,330.72083496)
\lineto(409.38908203,332.95583496)
\curveto(409.38907608,333.32583062)(409.37407609,333.66583028)(409.34408203,333.97583496)
\curveto(409.31407615,334.29582965)(409.22407624,334.56582938)(409.07408203,334.78583496)
\curveto(408.93407653,334.98582896)(408.72907674,335.12582882)(408.45908203,335.20583496)
\curveto(408.40907706,335.22582872)(408.35407711,335.23582871)(408.29408203,335.23583496)
\curveto(408.24407722,335.23582871)(408.18907728,335.2458287)(408.12908203,335.26583496)
\curveto(408.07907739,335.27582867)(408.01407745,335.27582867)(407.93408203,335.26583496)
\curveto(407.8640776,335.26582868)(407.80907766,335.26082868)(407.76908203,335.25083496)
\curveto(407.72907774,335.2408287)(407.69407777,335.23582871)(407.66408203,335.23583496)
\curveto(407.63407783,335.23582871)(407.60407786,335.23082871)(407.57408203,335.22083496)
\curveto(407.34407812,335.16082878)(407.15907831,335.08082886)(407.01908203,334.98083496)
\curveto(406.69907877,334.75082919)(406.50907896,334.41582953)(406.44908203,333.97583496)
\curveto(406.38907908,333.53583041)(406.35907911,333.0408309)(406.35908203,332.49083496)
\lineto(406.35908203,330.61583496)
\lineto(406.35908203,329.70083496)
\lineto(406.35908203,329.43083496)
\curveto(406.35907911,329.3408346)(406.34407912,329.26583468)(406.31408203,329.20583496)
\curveto(406.2640792,329.09583485)(406.18407928,329.03083491)(406.07408203,329.01083496)
\curveto(405.9640795,328.99083495)(405.82907964,328.98083496)(405.66908203,328.98083496)
\lineto(404.91908203,328.98083496)
\curveto(404.80908066,328.98083496)(404.69908077,328.98583496)(404.58908203,328.99583496)
\curveto(404.47908099,329.00583494)(404.39908107,329.0408349)(404.34908203,329.10083496)
\curveto(404.27908119,329.19083475)(404.24408122,329.32083462)(404.24408203,329.49083496)
\curveto(404.25408121,329.66083428)(404.25908121,329.82083412)(404.25908203,329.97083496)
\lineto(404.25908203,332.01083496)
\lineto(404.25908203,335.31083496)
\lineto(404.25908203,336.07583496)
\lineto(404.25908203,336.37583496)
\curveto(404.2690812,336.46582748)(404.29908117,336.5408274)(404.34908203,336.60083496)
\curveto(404.3690811,336.63082731)(404.39908107,336.65082729)(404.43908203,336.66083496)
\curveto(404.48908098,336.68082726)(404.53908093,336.69582725)(404.58908203,336.70583496)
\lineto(404.66408203,336.70583496)
\curveto(404.71408075,336.71582723)(404.7640807,336.72082722)(404.81408203,336.72083496)
\lineto(404.97908203,336.72083496)
\lineto(405.60908203,336.72083496)
\curveto(405.68907978,336.72082722)(405.7640797,336.71582723)(405.83408203,336.70583496)
\curveto(405.91407955,336.70582724)(405.98407948,336.69582725)(406.04408203,336.67583496)
\curveto(406.11407935,336.6458273)(406.15907931,336.60082734)(406.17908203,336.54083496)
\curveto(406.20907926,336.48082746)(406.23407923,336.41082753)(406.25408203,336.33083496)
\curveto(406.2640792,336.29082765)(406.2640792,336.25582769)(406.25408203,336.22583496)
\curveto(406.25407921,336.19582775)(406.2640792,336.16582778)(406.28408203,336.13583496)
\curveto(406.30407916,336.08582786)(406.31907915,336.05582789)(406.32908203,336.04583496)
\curveto(406.34907912,336.03582791)(406.37407909,336.02082792)(406.40408203,336.00083496)
\curveto(406.51407895,335.99082795)(406.60407886,336.02582792)(406.67408203,336.10583496)
\curveto(406.74407872,336.19582775)(406.81907865,336.26582768)(406.89908203,336.31583496)
\curveto(407.1690783,336.51582743)(407.469078,336.67582727)(407.79908203,336.79583496)
\curveto(407.88907758,336.82582712)(407.97907749,336.8458271)(408.06908203,336.85583496)
\curveto(408.1690773,336.86582708)(408.27407719,336.88082706)(408.38408203,336.90083496)
\curveto(408.41407705,336.91082703)(408.45907701,336.91082703)(408.51908203,336.90083496)
\curveto(408.57907689,336.90082704)(408.61907685,336.90582704)(408.63908203,336.91583496)
}
}
{
\newrgbcolor{curcolor}{0 0 0}
\pscustom[linestyle=none,fillstyle=solid,fillcolor=curcolor]
{
\newpath
\moveto(420.71033203,329.83583496)
\lineto(420.71033203,329.41583496)
\curveto(420.71032366,329.28583466)(420.68032369,329.18083476)(420.62033203,329.10083496)
\curveto(420.5703238,329.05083489)(420.50532387,329.01583493)(420.42533203,328.99583496)
\curveto(420.34532403,328.98583496)(420.25532412,328.98083496)(420.15533203,328.98083496)
\lineto(419.33033203,328.98083496)
\lineto(419.04533203,328.98083496)
\curveto(418.96532541,328.99083495)(418.90032547,329.01583493)(418.85033203,329.05583496)
\curveto(418.78032559,329.10583484)(418.74032563,329.17083477)(418.73033203,329.25083496)
\curveto(418.72032565,329.33083461)(418.70032567,329.41083453)(418.67033203,329.49083496)
\curveto(418.65032572,329.51083443)(418.63032574,329.52583442)(418.61033203,329.53583496)
\curveto(418.60032577,329.55583439)(418.58532579,329.57583437)(418.56533203,329.59583496)
\curveto(418.45532592,329.59583435)(418.375326,329.57083437)(418.32533203,329.52083496)
\lineto(418.17533203,329.37083496)
\curveto(418.10532627,329.32083462)(418.04032633,329.27583467)(417.98033203,329.23583496)
\curveto(417.92032645,329.20583474)(417.85532652,329.16583478)(417.78533203,329.11583496)
\curveto(417.74532663,329.09583485)(417.70032667,329.07583487)(417.65033203,329.05583496)
\curveto(417.61032676,329.03583491)(417.56532681,329.01583493)(417.51533203,328.99583496)
\curveto(417.375327,328.945835)(417.22532715,328.90083504)(417.06533203,328.86083496)
\curveto(417.01532736,328.8408351)(416.9703274,328.83083511)(416.93033203,328.83083496)
\curveto(416.89032748,328.83083511)(416.85032752,328.82583512)(416.81033203,328.81583496)
\lineto(416.67533203,328.81583496)
\curveto(416.64532773,328.80583514)(416.60532777,328.80083514)(416.55533203,328.80083496)
\lineto(416.42033203,328.80083496)
\curveto(416.36032801,328.78083516)(416.2703281,328.77583517)(416.15033203,328.78583496)
\curveto(416.03032834,328.78583516)(415.94532843,328.79583515)(415.89533203,328.81583496)
\curveto(415.82532855,328.83583511)(415.76032861,328.8458351)(415.70033203,328.84583496)
\curveto(415.65032872,328.83583511)(415.59532878,328.8408351)(415.53533203,328.86083496)
\lineto(415.17533203,328.98083496)
\curveto(415.06532931,329.01083493)(414.95532942,329.05083489)(414.84533203,329.10083496)
\curveto(414.49532988,329.25083469)(414.18033019,329.48083446)(413.90033203,329.79083496)
\curveto(413.63033074,330.11083383)(413.41533096,330.4458335)(413.25533203,330.79583496)
\curveto(413.20533117,330.90583304)(413.16533121,331.01083293)(413.13533203,331.11083496)
\curveto(413.10533127,331.22083272)(413.0703313,331.33083261)(413.03033203,331.44083496)
\curveto(413.02033135,331.48083246)(413.01533136,331.51583243)(413.01533203,331.54583496)
\curveto(413.01533136,331.58583236)(413.00533137,331.63083231)(412.98533203,331.68083496)
\curveto(412.96533141,331.76083218)(412.94533143,331.8458321)(412.92533203,331.93583496)
\curveto(412.91533146,332.03583191)(412.90033147,332.13583181)(412.88033203,332.23583496)
\curveto(412.8703315,332.26583168)(412.86533151,332.30083164)(412.86533203,332.34083496)
\curveto(412.8753315,332.38083156)(412.8753315,332.41583153)(412.86533203,332.44583496)
\lineto(412.86533203,332.58083496)
\curveto(412.86533151,332.63083131)(412.86033151,332.68083126)(412.85033203,332.73083496)
\curveto(412.84033153,332.78083116)(412.83533154,332.83583111)(412.83533203,332.89583496)
\curveto(412.83533154,332.96583098)(412.84033153,333.02083092)(412.85033203,333.06083496)
\curveto(412.86033151,333.11083083)(412.86533151,333.15583079)(412.86533203,333.19583496)
\lineto(412.86533203,333.34583496)
\curveto(412.8753315,333.39583055)(412.8753315,333.4408305)(412.86533203,333.48083496)
\curveto(412.86533151,333.53083041)(412.8753315,333.58083036)(412.89533203,333.63083496)
\curveto(412.91533146,333.7408302)(412.93033144,333.8458301)(412.94033203,333.94583496)
\curveto(412.96033141,334.0458299)(412.98533139,334.1458298)(413.01533203,334.24583496)
\curveto(413.05533132,334.36582958)(413.09033128,334.48082946)(413.12033203,334.59083496)
\curveto(413.15033122,334.70082924)(413.19033118,334.81082913)(413.24033203,334.92083496)
\curveto(413.38033099,335.22082872)(413.55533082,335.50582844)(413.76533203,335.77583496)
\curveto(413.78533059,335.80582814)(413.81033056,335.83082811)(413.84033203,335.85083496)
\curveto(413.88033049,335.88082806)(413.91033046,335.91082803)(413.93033203,335.94083496)
\curveto(413.9703304,335.99082795)(414.01033036,336.03582791)(414.05033203,336.07583496)
\curveto(414.09033028,336.11582783)(414.13533024,336.15582779)(414.18533203,336.19583496)
\curveto(414.22533015,336.21582773)(414.26033011,336.2408277)(414.29033203,336.27083496)
\curveto(414.32033005,336.31082763)(414.35533002,336.3408276)(414.39533203,336.36083496)
\curveto(414.64532973,336.53082741)(414.93532944,336.67082727)(415.26533203,336.78083496)
\curveto(415.33532904,336.80082714)(415.40532897,336.81582713)(415.47533203,336.82583496)
\curveto(415.55532882,336.83582711)(415.63532874,336.85082709)(415.71533203,336.87083496)
\curveto(415.78532859,336.89082705)(415.8753285,336.90082704)(415.98533203,336.90083496)
\curveto(416.09532828,336.91082703)(416.20532817,336.91582703)(416.31533203,336.91583496)
\curveto(416.42532795,336.91582703)(416.53032784,336.91082703)(416.63033203,336.90083496)
\curveto(416.74032763,336.89082705)(416.83032754,336.87582707)(416.90033203,336.85583496)
\curveto(417.05032732,336.80582714)(417.19532718,336.76082718)(417.33533203,336.72083496)
\curveto(417.4753269,336.68082726)(417.60532677,336.62582732)(417.72533203,336.55583496)
\curveto(417.79532658,336.50582744)(417.86032651,336.45582749)(417.92033203,336.40583496)
\curveto(417.98032639,336.36582758)(418.04532633,336.32082762)(418.11533203,336.27083496)
\curveto(418.15532622,336.2408277)(418.21032616,336.20082774)(418.28033203,336.15083496)
\curveto(418.36032601,336.10082784)(418.43532594,336.10082784)(418.50533203,336.15083496)
\curveto(418.54532583,336.17082777)(418.56532581,336.20582774)(418.56533203,336.25583496)
\curveto(418.56532581,336.30582764)(418.5753258,336.35582759)(418.59533203,336.40583496)
\lineto(418.59533203,336.55583496)
\curveto(418.60532577,336.58582736)(418.61032576,336.62082732)(418.61033203,336.66083496)
\lineto(418.61033203,336.78083496)
\lineto(418.61033203,338.82083496)
\curveto(418.61032576,338.93082501)(418.60532577,339.05082489)(418.59533203,339.18083496)
\curveto(418.59532578,339.32082462)(418.62032575,339.42582452)(418.67033203,339.49583496)
\curveto(418.71032566,339.57582437)(418.78532559,339.62582432)(418.89533203,339.64583496)
\curveto(418.91532546,339.65582429)(418.93532544,339.65582429)(418.95533203,339.64583496)
\curveto(418.9753254,339.6458243)(418.99532538,339.65082429)(419.01533203,339.66083496)
\lineto(420.08033203,339.66083496)
\curveto(420.20032417,339.66082428)(420.31032406,339.65582429)(420.41033203,339.64583496)
\curveto(420.51032386,339.63582431)(420.58532379,339.59582435)(420.63533203,339.52583496)
\curveto(420.68532369,339.4458245)(420.71032366,339.3408246)(420.71033203,339.21083496)
\lineto(420.71033203,338.85083496)
\lineto(420.71033203,329.83583496)
\moveto(418.67033203,332.77583496)
\curveto(418.68032569,332.81583113)(418.68032569,332.85583109)(418.67033203,332.89583496)
\lineto(418.67033203,333.03083496)
\curveto(418.6703257,333.13083081)(418.66532571,333.23083071)(418.65533203,333.33083496)
\curveto(418.64532573,333.43083051)(418.63032574,333.52083042)(418.61033203,333.60083496)
\curveto(418.59032578,333.71083023)(418.5703258,333.81083013)(418.55033203,333.90083496)
\curveto(418.54032583,333.99082995)(418.51532586,334.07582987)(418.47533203,334.15583496)
\curveto(418.33532604,334.51582943)(418.13032624,334.80082914)(417.86033203,335.01083496)
\curveto(417.60032677,335.22082872)(417.22032715,335.32582862)(416.72033203,335.32583496)
\curveto(416.66032771,335.32582862)(416.58032779,335.31582863)(416.48033203,335.29583496)
\curveto(416.40032797,335.27582867)(416.32532805,335.25582869)(416.25533203,335.23583496)
\curveto(416.19532818,335.22582872)(416.13532824,335.20582874)(416.07533203,335.17583496)
\curveto(415.80532857,335.06582888)(415.59532878,334.89582905)(415.44533203,334.66583496)
\curveto(415.29532908,334.43582951)(415.1753292,334.17582977)(415.08533203,333.88583496)
\curveto(415.05532932,333.78583016)(415.03532934,333.68583026)(415.02533203,333.58583496)
\curveto(415.01532936,333.48583046)(414.99532938,333.38083056)(414.96533203,333.27083496)
\lineto(414.96533203,333.06083496)
\curveto(414.94532943,332.97083097)(414.94032943,332.8458311)(414.95033203,332.68583496)
\curveto(414.96032941,332.53583141)(414.9753294,332.42583152)(414.99533203,332.35583496)
\lineto(414.99533203,332.26583496)
\curveto(415.00532937,332.2458317)(415.01032936,332.22583172)(415.01033203,332.20583496)
\curveto(415.03032934,332.12583182)(415.04532933,332.05083189)(415.05533203,331.98083496)
\curveto(415.0753293,331.91083203)(415.09532928,331.83583211)(415.11533203,331.75583496)
\curveto(415.28532909,331.23583271)(415.5753288,330.85083309)(415.98533203,330.60083496)
\curveto(416.11532826,330.51083343)(416.29532808,330.4408335)(416.52533203,330.39083496)
\curveto(416.56532781,330.38083356)(416.62532775,330.37583357)(416.70533203,330.37583496)
\curveto(416.73532764,330.36583358)(416.78032759,330.35583359)(416.84033203,330.34583496)
\curveto(416.91032746,330.3458336)(416.96532741,330.35083359)(417.00533203,330.36083496)
\curveto(417.08532729,330.38083356)(417.16532721,330.39583355)(417.24533203,330.40583496)
\curveto(417.32532705,330.41583353)(417.40532697,330.43583351)(417.48533203,330.46583496)
\curveto(417.73532664,330.57583337)(417.93532644,330.71583323)(418.08533203,330.88583496)
\curveto(418.23532614,331.05583289)(418.36532601,331.27083267)(418.47533203,331.53083496)
\curveto(418.51532586,331.62083232)(418.54532583,331.71083223)(418.56533203,331.80083496)
\curveto(418.58532579,331.90083204)(418.60532577,332.00583194)(418.62533203,332.11583496)
\curveto(418.63532574,332.16583178)(418.63532574,332.21083173)(418.62533203,332.25083496)
\curveto(418.62532575,332.30083164)(418.63532574,332.35083159)(418.65533203,332.40083496)
\curveto(418.66532571,332.43083151)(418.6703257,332.46583148)(418.67033203,332.50583496)
\lineto(418.67033203,332.64083496)
\lineto(418.67033203,332.77583496)
}
}
{
\newrgbcolor{curcolor}{0 0 0}
\pscustom[linestyle=none,fillstyle=solid,fillcolor=curcolor]
{
\newpath
\moveto(424.39025391,339.57083496)
\curveto(424.46025096,339.49082445)(424.49525092,339.37082457)(424.49525391,339.21083496)
\lineto(424.49525391,338.74583496)
\lineto(424.49525391,338.34083496)
\curveto(424.49525092,338.20082574)(424.46025096,338.10582584)(424.39025391,338.05583496)
\curveto(424.33025109,338.00582594)(424.25025117,337.97582597)(424.15025391,337.96583496)
\curveto(424.06025136,337.95582599)(423.96025146,337.95082599)(423.85025391,337.95083496)
\lineto(423.01025391,337.95083496)
\curveto(422.90025252,337.95082599)(422.80025262,337.95582599)(422.71025391,337.96583496)
\curveto(422.63025279,337.97582597)(422.56025286,338.00582594)(422.50025391,338.05583496)
\curveto(422.46025296,338.08582586)(422.43025299,338.1408258)(422.41025391,338.22083496)
\curveto(422.40025302,338.31082563)(422.39025303,338.40582554)(422.38025391,338.50583496)
\lineto(422.38025391,338.83583496)
\curveto(422.39025303,338.945825)(422.39525302,339.0408249)(422.39525391,339.12083496)
\lineto(422.39525391,339.33083496)
\curveto(422.40525301,339.40082454)(422.42525299,339.46082448)(422.45525391,339.51083496)
\curveto(422.47525294,339.55082439)(422.50025292,339.58082436)(422.53025391,339.60083496)
\lineto(422.65025391,339.66083496)
\curveto(422.67025275,339.66082428)(422.69525272,339.66082428)(422.72525391,339.66083496)
\curveto(422.75525266,339.67082427)(422.78025264,339.67582427)(422.80025391,339.67583496)
\lineto(423.89525391,339.67583496)
\curveto(423.99525142,339.67582427)(424.09025133,339.67082427)(424.18025391,339.66083496)
\curveto(424.27025115,339.65082429)(424.34025108,339.62082432)(424.39025391,339.57083496)
\moveto(424.49525391,329.80583496)
\curveto(424.49525092,329.60583434)(424.49025093,329.43583451)(424.48025391,329.29583496)
\curveto(424.47025095,329.15583479)(424.38025104,329.06083488)(424.21025391,329.01083496)
\curveto(424.15025127,328.99083495)(424.08525133,328.98083496)(424.01525391,328.98083496)
\curveto(423.94525147,328.99083495)(423.87025155,328.99583495)(423.79025391,328.99583496)
\lineto(422.95025391,328.99583496)
\curveto(422.86025256,328.99583495)(422.77025265,329.00083494)(422.68025391,329.01083496)
\curveto(422.60025282,329.02083492)(422.54025288,329.05083489)(422.50025391,329.10083496)
\curveto(422.44025298,329.17083477)(422.40525301,329.25583469)(422.39525391,329.35583496)
\lineto(422.39525391,329.70083496)
\lineto(422.39525391,336.03083496)
\lineto(422.39525391,336.33083496)
\curveto(422.39525302,336.43082751)(422.415253,336.51082743)(422.45525391,336.57083496)
\curveto(422.5152529,336.6408273)(422.60025282,336.68582726)(422.71025391,336.70583496)
\curveto(422.73025269,336.71582723)(422.75525266,336.71582723)(422.78525391,336.70583496)
\curveto(422.82525259,336.70582724)(422.85525256,336.71082723)(422.87525391,336.72083496)
\lineto(423.62525391,336.72083496)
\lineto(423.82025391,336.72083496)
\curveto(423.90025152,336.73082721)(423.96525145,336.73082721)(424.01525391,336.72083496)
\lineto(424.13525391,336.72083496)
\curveto(424.19525122,336.70082724)(424.25025117,336.68582726)(424.30025391,336.67583496)
\curveto(424.35025107,336.66582728)(424.39025103,336.63582731)(424.42025391,336.58583496)
\curveto(424.46025096,336.53582741)(424.48025094,336.46582748)(424.48025391,336.37583496)
\curveto(424.49025093,336.28582766)(424.49525092,336.19082775)(424.49525391,336.09083496)
\lineto(424.49525391,329.80583496)
}
}
{
\newrgbcolor{curcolor}{0 0 0}
\pscustom[linestyle=none,fillstyle=solid,fillcolor=curcolor]
{
\newpath
\moveto(429.72744141,336.93083496)
\curveto(430.53743625,336.95082699)(431.21243557,336.83082711)(431.75244141,336.57083496)
\curveto(432.30243448,336.31082763)(432.73743405,335.940828)(433.05744141,335.46083496)
\curveto(433.21743357,335.22082872)(433.33743345,334.945829)(433.41744141,334.63583496)
\curveto(433.43743335,334.58582936)(433.45243333,334.52082942)(433.46244141,334.44083496)
\curveto(433.4824333,334.36082958)(433.4824333,334.29082965)(433.46244141,334.23083496)
\curveto(433.42243336,334.12082982)(433.35243343,334.05582989)(433.25244141,334.03583496)
\curveto(433.15243363,334.02582992)(433.03243375,334.02082992)(432.89244141,334.02083496)
\lineto(432.11244141,334.02083496)
\lineto(431.82744141,334.02083496)
\curveto(431.73743505,334.02082992)(431.66243512,334.0408299)(431.60244141,334.08083496)
\curveto(431.52243526,334.12082982)(431.46743532,334.18082976)(431.43744141,334.26083496)
\curveto(431.40743538,334.35082959)(431.36743542,334.4408295)(431.31744141,334.53083496)
\curveto(431.25743553,334.6408293)(431.19243559,334.7408292)(431.12244141,334.83083496)
\curveto(431.05243573,334.92082902)(430.97243581,335.00082894)(430.88244141,335.07083496)
\curveto(430.74243604,335.16082878)(430.5874362,335.23082871)(430.41744141,335.28083496)
\curveto(430.35743643,335.30082864)(430.29743649,335.31082863)(430.23744141,335.31083496)
\curveto(430.17743661,335.31082863)(430.12243666,335.32082862)(430.07244141,335.34083496)
\lineto(429.92244141,335.34083496)
\curveto(429.72243706,335.3408286)(429.56243722,335.32082862)(429.44244141,335.28083496)
\curveto(429.15243763,335.19082875)(428.91743787,335.05082889)(428.73744141,334.86083496)
\curveto(428.55743823,334.68082926)(428.41243837,334.46082948)(428.30244141,334.20083496)
\curveto(428.25243853,334.09082985)(428.21243857,333.97082997)(428.18244141,333.84083496)
\curveto(428.16243862,333.72083022)(428.13743865,333.59083035)(428.10744141,333.45083496)
\curveto(428.09743869,333.41083053)(428.09243869,333.37083057)(428.09244141,333.33083496)
\curveto(428.09243869,333.29083065)(428.0874387,333.25083069)(428.07744141,333.21083496)
\curveto(428.05743873,333.11083083)(428.04743874,332.97083097)(428.04744141,332.79083496)
\curveto(428.05743873,332.61083133)(428.07243871,332.47083147)(428.09244141,332.37083496)
\curveto(428.09243869,332.29083165)(428.09743869,332.23583171)(428.10744141,332.20583496)
\curveto(428.12743866,332.13583181)(428.13743865,332.06583188)(428.13744141,331.99583496)
\curveto(428.14743864,331.92583202)(428.16243862,331.85583209)(428.18244141,331.78583496)
\curveto(428.26243852,331.55583239)(428.35743843,331.3458326)(428.46744141,331.15583496)
\curveto(428.57743821,330.96583298)(428.71743807,330.80583314)(428.88744141,330.67583496)
\curveto(428.92743786,330.6458333)(428.9874378,330.61083333)(429.06744141,330.57083496)
\curveto(429.17743761,330.50083344)(429.2874375,330.45583349)(429.39744141,330.43583496)
\curveto(429.51743727,330.41583353)(429.66243712,330.39583355)(429.83244141,330.37583496)
\lineto(429.92244141,330.37583496)
\curveto(429.96243682,330.37583357)(429.99243679,330.38083356)(430.01244141,330.39083496)
\lineto(430.14744141,330.39083496)
\curveto(430.21743657,330.41083353)(430.2824365,330.42583352)(430.34244141,330.43583496)
\curveto(430.41243637,330.45583349)(430.47743631,330.47583347)(430.53744141,330.49583496)
\curveto(430.83743595,330.62583332)(431.06743572,330.81583313)(431.22744141,331.06583496)
\curveto(431.26743552,331.11583283)(431.30243548,331.17083277)(431.33244141,331.23083496)
\curveto(431.36243542,331.30083264)(431.3874354,331.36083258)(431.40744141,331.41083496)
\curveto(431.44743534,331.52083242)(431.4824353,331.61583233)(431.51244141,331.69583496)
\curveto(431.54243524,331.78583216)(431.61243517,331.85583209)(431.72244141,331.90583496)
\curveto(431.81243497,331.945832)(431.95743483,331.96083198)(432.15744141,331.95083496)
\lineto(432.65244141,331.95083496)
\lineto(432.86244141,331.95083496)
\curveto(432.94243384,331.96083198)(433.00743378,331.95583199)(433.05744141,331.93583496)
\lineto(433.17744141,331.93583496)
\lineto(433.29744141,331.90583496)
\curveto(433.33743345,331.90583204)(433.36743342,331.89583205)(433.38744141,331.87583496)
\curveto(433.43743335,331.83583211)(433.46743332,331.77583217)(433.47744141,331.69583496)
\curveto(433.49743329,331.62583232)(433.49743329,331.55083239)(433.47744141,331.47083496)
\curveto(433.3874334,331.1408328)(433.27743351,330.8458331)(433.14744141,330.58583496)
\curveto(432.73743405,329.81583413)(432.0824347,329.28083466)(431.18244141,328.98083496)
\curveto(431.0824357,328.95083499)(430.97743581,328.93083501)(430.86744141,328.92083496)
\curveto(430.75743603,328.90083504)(430.64743614,328.87583507)(430.53744141,328.84583496)
\curveto(430.47743631,328.83583511)(430.41743637,328.83083511)(430.35744141,328.83083496)
\curveto(430.29743649,328.83083511)(430.23743655,328.82583512)(430.17744141,328.81583496)
\lineto(430.01244141,328.81583496)
\curveto(429.96243682,328.79583515)(429.8874369,328.79083515)(429.78744141,328.80083496)
\curveto(429.6874371,328.80083514)(429.61243717,328.80583514)(429.56244141,328.81583496)
\curveto(429.4824373,328.83583511)(429.40743738,328.8458351)(429.33744141,328.84583496)
\curveto(429.27743751,328.83583511)(429.21243757,328.8408351)(429.14244141,328.86083496)
\lineto(428.99244141,328.89083496)
\curveto(428.94243784,328.89083505)(428.89243789,328.89583505)(428.84244141,328.90583496)
\curveto(428.73243805,328.93583501)(428.62743816,328.96583498)(428.52744141,328.99583496)
\curveto(428.42743836,329.02583492)(428.33243845,329.06083488)(428.24244141,329.10083496)
\curveto(427.77243901,329.30083464)(427.37743941,329.55583439)(427.05744141,329.86583496)
\curveto(426.73744005,330.18583376)(426.47744031,330.58083336)(426.27744141,331.05083496)
\curveto(426.22744056,331.1408328)(426.1874406,331.23583271)(426.15744141,331.33583496)
\lineto(426.06744141,331.66583496)
\curveto(426.05744073,331.70583224)(426.05244073,331.7408322)(426.05244141,331.77083496)
\curveto(426.05244073,331.81083213)(426.04244074,331.85583209)(426.02244141,331.90583496)
\curveto(426.00244078,331.97583197)(425.99244079,332.0458319)(425.99244141,332.11583496)
\curveto(425.99244079,332.19583175)(425.9824408,332.27083167)(425.96244141,332.34083496)
\lineto(425.96244141,332.59583496)
\curveto(425.94244084,332.6458313)(425.93244085,332.70083124)(425.93244141,332.76083496)
\curveto(425.93244085,332.83083111)(425.94244084,332.89083105)(425.96244141,332.94083496)
\curveto(425.97244081,332.99083095)(425.97244081,333.03583091)(425.96244141,333.07583496)
\curveto(425.95244083,333.11583083)(425.95244083,333.15583079)(425.96244141,333.19583496)
\curveto(425.9824408,333.26583068)(425.9874408,333.33083061)(425.97744141,333.39083496)
\curveto(425.97744081,333.45083049)(425.9874408,333.51083043)(426.00744141,333.57083496)
\curveto(426.05744073,333.75083019)(426.09744069,333.92083002)(426.12744141,334.08083496)
\curveto(426.15744063,334.25082969)(426.20244058,334.41582953)(426.26244141,334.57583496)
\curveto(426.4824403,335.08582886)(426.75744003,335.51082843)(427.08744141,335.85083496)
\curveto(427.42743936,336.19082775)(427.85743893,336.46582748)(428.37744141,336.67583496)
\curveto(428.51743827,336.73582721)(428.66243812,336.77582717)(428.81244141,336.79583496)
\curveto(428.96243782,336.82582712)(429.11743767,336.86082708)(429.27744141,336.90083496)
\curveto(429.35743743,336.91082703)(429.43243735,336.91582703)(429.50244141,336.91583496)
\curveto(429.57243721,336.91582703)(429.64743714,336.92082702)(429.72744141,336.93083496)
}
}
{
\newrgbcolor{curcolor}{0 0 0}
\pscustom[linestyle=none,fillstyle=solid,fillcolor=curcolor]
{
\newpath
\moveto(441.82072266,329.58083496)
\curveto(441.84071481,329.47083447)(441.8507148,329.36083458)(441.85072266,329.25083496)
\curveto(441.86071479,329.1408348)(441.81071484,329.06583488)(441.70072266,329.02583496)
\curveto(441.64071501,328.99583495)(441.57071508,328.98083496)(441.49072266,328.98083496)
\lineto(441.25072266,328.98083496)
\lineto(440.44072266,328.98083496)
\lineto(440.17072266,328.98083496)
\curveto(440.09071656,328.99083495)(440.02571662,329.01583493)(439.97572266,329.05583496)
\curveto(439.90571674,329.09583485)(439.8507168,329.15083479)(439.81072266,329.22083496)
\curveto(439.78071687,329.30083464)(439.73571691,329.36583458)(439.67572266,329.41583496)
\curveto(439.65571699,329.43583451)(439.63071702,329.45083449)(439.60072266,329.46083496)
\curveto(439.57071708,329.48083446)(439.53071712,329.48583446)(439.48072266,329.47583496)
\curveto(439.43071722,329.45583449)(439.38071727,329.43083451)(439.33072266,329.40083496)
\curveto(439.29071736,329.37083457)(439.2457174,329.3458346)(439.19572266,329.32583496)
\curveto(439.1457175,329.28583466)(439.09071756,329.25083469)(439.03072266,329.22083496)
\lineto(438.85072266,329.13083496)
\curveto(438.72071793,329.07083487)(438.58571806,329.02083492)(438.44572266,328.98083496)
\curveto(438.30571834,328.95083499)(438.16071849,328.91583503)(438.01072266,328.87583496)
\curveto(437.94071871,328.85583509)(437.87071878,328.8458351)(437.80072266,328.84583496)
\curveto(437.74071891,328.83583511)(437.67571897,328.82583512)(437.60572266,328.81583496)
\lineto(437.51572266,328.81583496)
\curveto(437.48571916,328.80583514)(437.45571919,328.80083514)(437.42572266,328.80083496)
\lineto(437.26072266,328.80083496)
\curveto(437.16071949,328.78083516)(437.06071959,328.78083516)(436.96072266,328.80083496)
\lineto(436.82572266,328.80083496)
\curveto(436.75571989,328.82083512)(436.68571996,328.83083511)(436.61572266,328.83083496)
\curveto(436.55572009,328.82083512)(436.49572015,328.82583512)(436.43572266,328.84583496)
\curveto(436.33572031,328.86583508)(436.24072041,328.88583506)(436.15072266,328.90583496)
\curveto(436.06072059,328.91583503)(435.97572067,328.940835)(435.89572266,328.98083496)
\curveto(435.60572104,329.09083485)(435.35572129,329.23083471)(435.14572266,329.40083496)
\curveto(434.9457217,329.58083436)(434.78572186,329.81583413)(434.66572266,330.10583496)
\curveto(434.63572201,330.17583377)(434.60572204,330.25083369)(434.57572266,330.33083496)
\curveto(434.55572209,330.41083353)(434.53572211,330.49583345)(434.51572266,330.58583496)
\curveto(434.49572215,330.63583331)(434.48572216,330.68583326)(434.48572266,330.73583496)
\curveto(434.49572215,330.78583316)(434.49572215,330.83583311)(434.48572266,330.88583496)
\curveto(434.47572217,330.91583303)(434.46572218,330.97583297)(434.45572266,331.06583496)
\curveto(434.45572219,331.16583278)(434.46072219,331.23583271)(434.47072266,331.27583496)
\curveto(434.49072216,331.37583257)(434.50072215,331.46083248)(434.50072266,331.53083496)
\lineto(434.59072266,331.86083496)
\curveto(434.62072203,331.98083196)(434.66072199,332.08583186)(434.71072266,332.17583496)
\curveto(434.88072177,332.46583148)(435.07572157,332.68583126)(435.29572266,332.83583496)
\curveto(435.51572113,332.98583096)(435.79572085,333.11583083)(436.13572266,333.22583496)
\curveto(436.26572038,333.27583067)(436.40072025,333.31083063)(436.54072266,333.33083496)
\curveto(436.68071997,333.35083059)(436.82071983,333.37583057)(436.96072266,333.40583496)
\curveto(437.04071961,333.42583052)(437.12571952,333.43583051)(437.21572266,333.43583496)
\curveto(437.30571934,333.4458305)(437.39571925,333.46083048)(437.48572266,333.48083496)
\curveto(437.55571909,333.50083044)(437.62571902,333.50583044)(437.69572266,333.49583496)
\curveto(437.76571888,333.49583045)(437.84071881,333.50583044)(437.92072266,333.52583496)
\curveto(437.99071866,333.5458304)(438.06071859,333.55583039)(438.13072266,333.55583496)
\curveto(438.20071845,333.55583039)(438.27571837,333.56583038)(438.35572266,333.58583496)
\curveto(438.56571808,333.63583031)(438.75571789,333.67583027)(438.92572266,333.70583496)
\curveto(439.10571754,333.7458302)(439.26571738,333.83583011)(439.40572266,333.97583496)
\curveto(439.49571715,334.06582988)(439.55571709,334.16582978)(439.58572266,334.27583496)
\curveto(439.59571705,334.30582964)(439.59571705,334.33082961)(439.58572266,334.35083496)
\curveto(439.58571706,334.37082957)(439.59071706,334.39082955)(439.60072266,334.41083496)
\curveto(439.61071704,334.43082951)(439.61571703,334.46082948)(439.61572266,334.50083496)
\lineto(439.61572266,334.59083496)
\lineto(439.58572266,334.71083496)
\curveto(439.58571706,334.75082919)(439.58071707,334.78582916)(439.57072266,334.81583496)
\curveto(439.47071718,335.11582883)(439.26071739,335.32082862)(438.94072266,335.43083496)
\curveto(438.8507178,335.46082848)(438.74071791,335.48082846)(438.61072266,335.49083496)
\curveto(438.49071816,335.51082843)(438.36571828,335.51582843)(438.23572266,335.50583496)
\curveto(438.10571854,335.50582844)(437.98071867,335.49582845)(437.86072266,335.47583496)
\curveto(437.74071891,335.45582849)(437.63571901,335.43082851)(437.54572266,335.40083496)
\curveto(437.48571916,335.38082856)(437.42571922,335.35082859)(437.36572266,335.31083496)
\curveto(437.31571933,335.28082866)(437.26571938,335.2458287)(437.21572266,335.20583496)
\curveto(437.16571948,335.16582878)(437.11071954,335.11082883)(437.05072266,335.04083496)
\curveto(437.00071965,334.97082897)(436.96571968,334.90582904)(436.94572266,334.84583496)
\curveto(436.89571975,334.7458292)(436.8507198,334.65082929)(436.81072266,334.56083496)
\curveto(436.78071987,334.47082947)(436.71071994,334.41082953)(436.60072266,334.38083496)
\curveto(436.52072013,334.36082958)(436.43572021,334.35082959)(436.34572266,334.35083496)
\lineto(436.07572266,334.35083496)
\lineto(435.50572266,334.35083496)
\curveto(435.45572119,334.35082959)(435.40572124,334.3458296)(435.35572266,334.33583496)
\curveto(435.30572134,334.33582961)(435.26072139,334.3408296)(435.22072266,334.35083496)
\lineto(435.08572266,334.35083496)
\curveto(435.06572158,334.36082958)(435.04072161,334.36582958)(435.01072266,334.36583496)
\curveto(434.98072167,334.36582958)(434.95572169,334.37582957)(434.93572266,334.39583496)
\curveto(434.85572179,334.41582953)(434.80072185,334.48082946)(434.77072266,334.59083496)
\curveto(434.76072189,334.6408293)(434.76072189,334.69082925)(434.77072266,334.74083496)
\curveto(434.78072187,334.79082915)(434.79072186,334.83582911)(434.80072266,334.87583496)
\curveto(434.83072182,334.98582896)(434.86072179,335.08582886)(434.89072266,335.17583496)
\curveto(434.93072172,335.27582867)(434.97572167,335.36582858)(435.02572266,335.44583496)
\lineto(435.11572266,335.59583496)
\lineto(435.20572266,335.74583496)
\curveto(435.28572136,335.85582809)(435.38572126,335.96082798)(435.50572266,336.06083496)
\curveto(435.52572112,336.07082787)(435.55572109,336.09582785)(435.59572266,336.13583496)
\curveto(435.645721,336.17582777)(435.69072096,336.21082773)(435.73072266,336.24083496)
\curveto(435.77072088,336.27082767)(435.81572083,336.30082764)(435.86572266,336.33083496)
\curveto(436.03572061,336.4408275)(436.21572043,336.52582742)(436.40572266,336.58583496)
\curveto(436.59572005,336.65582729)(436.79071986,336.72082722)(436.99072266,336.78083496)
\curveto(437.11071954,336.81082713)(437.23571941,336.83082711)(437.36572266,336.84083496)
\curveto(437.49571915,336.85082709)(437.62571902,336.87082707)(437.75572266,336.90083496)
\curveto(437.79571885,336.91082703)(437.85571879,336.91082703)(437.93572266,336.90083496)
\curveto(438.02571862,336.89082705)(438.08071857,336.89582705)(438.10072266,336.91583496)
\curveto(438.51071814,336.92582702)(438.90071775,336.91082703)(439.27072266,336.87083496)
\curveto(439.650717,336.83082711)(439.99071666,336.75582719)(440.29072266,336.64583496)
\curveto(440.60071605,336.53582741)(440.86571578,336.38582756)(441.08572266,336.19583496)
\curveto(441.30571534,336.01582793)(441.47571517,335.78082816)(441.59572266,335.49083496)
\curveto(441.66571498,335.32082862)(441.70571494,335.12582882)(441.71572266,334.90583496)
\curveto(441.72571492,334.68582926)(441.73071492,334.46082948)(441.73072266,334.23083496)
\lineto(441.73072266,330.88583496)
\lineto(441.73072266,330.30083496)
\curveto(441.73071492,330.11083383)(441.7507149,329.93583401)(441.79072266,329.77583496)
\curveto(441.80071485,329.7458342)(441.80571484,329.71083423)(441.80572266,329.67083496)
\curveto(441.80571484,329.6408343)(441.81071484,329.61083433)(441.82072266,329.58083496)
\moveto(439.61572266,331.89083496)
\curveto(439.62571702,331.940832)(439.63071702,331.99583195)(439.63072266,332.05583496)
\curveto(439.63071702,332.12583182)(439.62571702,332.18583176)(439.61572266,332.23583496)
\curveto(439.59571705,332.29583165)(439.58571706,332.35083159)(439.58572266,332.40083496)
\curveto(439.58571706,332.45083149)(439.56571708,332.49083145)(439.52572266,332.52083496)
\curveto(439.47571717,332.56083138)(439.40071725,332.58083136)(439.30072266,332.58083496)
\curveto(439.26071739,332.57083137)(439.22571742,332.56083138)(439.19572266,332.55083496)
\curveto(439.16571748,332.55083139)(439.13071752,332.5458314)(439.09072266,332.53583496)
\curveto(439.02071763,332.51583143)(438.9457177,332.50083144)(438.86572266,332.49083496)
\curveto(438.78571786,332.48083146)(438.70571794,332.46583148)(438.62572266,332.44583496)
\curveto(438.59571805,332.43583151)(438.5507181,332.43083151)(438.49072266,332.43083496)
\curveto(438.36071829,332.40083154)(438.23071842,332.38083156)(438.10072266,332.37083496)
\curveto(437.97071868,332.36083158)(437.8457188,332.33583161)(437.72572266,332.29583496)
\curveto(437.645719,332.27583167)(437.57071908,332.25583169)(437.50072266,332.23583496)
\curveto(437.43071922,332.22583172)(437.36071929,332.20583174)(437.29072266,332.17583496)
\curveto(437.08071957,332.08583186)(436.90071975,331.95083199)(436.75072266,331.77083496)
\curveto(436.61072004,331.59083235)(436.56072009,331.3408326)(436.60072266,331.02083496)
\curveto(436.62072003,330.85083309)(436.67571997,330.71083323)(436.76572266,330.60083496)
\curveto(436.83571981,330.49083345)(436.94071971,330.40083354)(437.08072266,330.33083496)
\curveto(437.22071943,330.27083367)(437.37071928,330.22583372)(437.53072266,330.19583496)
\curveto(437.70071895,330.16583378)(437.87571877,330.15583379)(438.05572266,330.16583496)
\curveto(438.2457184,330.18583376)(438.42071823,330.22083372)(438.58072266,330.27083496)
\curveto(438.84071781,330.35083359)(439.0457176,330.47583347)(439.19572266,330.64583496)
\curveto(439.3457173,330.82583312)(439.46071719,331.0458329)(439.54072266,331.30583496)
\curveto(439.56071709,331.37583257)(439.57071708,331.4458325)(439.57072266,331.51583496)
\curveto(439.58071707,331.59583235)(439.59571705,331.67583227)(439.61572266,331.75583496)
\lineto(439.61572266,331.89083496)
}
}
{
\newrgbcolor{curcolor}{0 0 0}
\pscustom[linestyle=none,fillstyle=solid,fillcolor=curcolor]
{
\newpath
\moveto(450.97400391,329.83583496)
\lineto(450.97400391,329.41583496)
\curveto(450.97399554,329.28583466)(450.94399557,329.18083476)(450.88400391,329.10083496)
\curveto(450.83399568,329.05083489)(450.76899574,329.01583493)(450.68900391,328.99583496)
\curveto(450.6089959,328.98583496)(450.51899599,328.98083496)(450.41900391,328.98083496)
\lineto(449.59400391,328.98083496)
\lineto(449.30900391,328.98083496)
\curveto(449.22899728,328.99083495)(449.16399735,329.01583493)(449.11400391,329.05583496)
\curveto(449.04399747,329.10583484)(449.00399751,329.17083477)(448.99400391,329.25083496)
\curveto(448.98399753,329.33083461)(448.96399755,329.41083453)(448.93400391,329.49083496)
\curveto(448.9139976,329.51083443)(448.89399762,329.52583442)(448.87400391,329.53583496)
\curveto(448.86399765,329.55583439)(448.84899766,329.57583437)(448.82900391,329.59583496)
\curveto(448.71899779,329.59583435)(448.63899787,329.57083437)(448.58900391,329.52083496)
\lineto(448.43900391,329.37083496)
\curveto(448.36899814,329.32083462)(448.30399821,329.27583467)(448.24400391,329.23583496)
\curveto(448.18399833,329.20583474)(448.11899839,329.16583478)(448.04900391,329.11583496)
\curveto(448.0089985,329.09583485)(447.96399855,329.07583487)(447.91400391,329.05583496)
\curveto(447.87399864,329.03583491)(447.82899868,329.01583493)(447.77900391,328.99583496)
\curveto(447.63899887,328.945835)(447.48899902,328.90083504)(447.32900391,328.86083496)
\curveto(447.27899923,328.8408351)(447.23399928,328.83083511)(447.19400391,328.83083496)
\curveto(447.15399936,328.83083511)(447.1139994,328.82583512)(447.07400391,328.81583496)
\lineto(446.93900391,328.81583496)
\curveto(446.9089996,328.80583514)(446.86899964,328.80083514)(446.81900391,328.80083496)
\lineto(446.68400391,328.80083496)
\curveto(446.62399989,328.78083516)(446.53399998,328.77583517)(446.41400391,328.78583496)
\curveto(446.29400022,328.78583516)(446.2090003,328.79583515)(446.15900391,328.81583496)
\curveto(446.08900042,328.83583511)(446.02400049,328.8458351)(445.96400391,328.84583496)
\curveto(445.9140006,328.83583511)(445.85900065,328.8408351)(445.79900391,328.86083496)
\lineto(445.43900391,328.98083496)
\curveto(445.32900118,329.01083493)(445.21900129,329.05083489)(445.10900391,329.10083496)
\curveto(444.75900175,329.25083469)(444.44400207,329.48083446)(444.16400391,329.79083496)
\curveto(443.89400262,330.11083383)(443.67900283,330.4458335)(443.51900391,330.79583496)
\curveto(443.46900304,330.90583304)(443.42900308,331.01083293)(443.39900391,331.11083496)
\curveto(443.36900314,331.22083272)(443.33400318,331.33083261)(443.29400391,331.44083496)
\curveto(443.28400323,331.48083246)(443.27900323,331.51583243)(443.27900391,331.54583496)
\curveto(443.27900323,331.58583236)(443.26900324,331.63083231)(443.24900391,331.68083496)
\curveto(443.22900328,331.76083218)(443.2090033,331.8458321)(443.18900391,331.93583496)
\curveto(443.17900333,332.03583191)(443.16400335,332.13583181)(443.14400391,332.23583496)
\curveto(443.13400338,332.26583168)(443.12900338,332.30083164)(443.12900391,332.34083496)
\curveto(443.13900337,332.38083156)(443.13900337,332.41583153)(443.12900391,332.44583496)
\lineto(443.12900391,332.58083496)
\curveto(443.12900338,332.63083131)(443.12400339,332.68083126)(443.11400391,332.73083496)
\curveto(443.10400341,332.78083116)(443.09900341,332.83583111)(443.09900391,332.89583496)
\curveto(443.09900341,332.96583098)(443.10400341,333.02083092)(443.11400391,333.06083496)
\curveto(443.12400339,333.11083083)(443.12900338,333.15583079)(443.12900391,333.19583496)
\lineto(443.12900391,333.34583496)
\curveto(443.13900337,333.39583055)(443.13900337,333.4408305)(443.12900391,333.48083496)
\curveto(443.12900338,333.53083041)(443.13900337,333.58083036)(443.15900391,333.63083496)
\curveto(443.17900333,333.7408302)(443.19400332,333.8458301)(443.20400391,333.94583496)
\curveto(443.22400329,334.0458299)(443.24900326,334.1458298)(443.27900391,334.24583496)
\curveto(443.31900319,334.36582958)(443.35400316,334.48082946)(443.38400391,334.59083496)
\curveto(443.4140031,334.70082924)(443.45400306,334.81082913)(443.50400391,334.92083496)
\curveto(443.64400287,335.22082872)(443.81900269,335.50582844)(444.02900391,335.77583496)
\curveto(444.04900246,335.80582814)(444.07400244,335.83082811)(444.10400391,335.85083496)
\curveto(444.14400237,335.88082806)(444.17400234,335.91082803)(444.19400391,335.94083496)
\curveto(444.23400228,335.99082795)(444.27400224,336.03582791)(444.31400391,336.07583496)
\curveto(444.35400216,336.11582783)(444.39900211,336.15582779)(444.44900391,336.19583496)
\curveto(444.48900202,336.21582773)(444.52400199,336.2408277)(444.55400391,336.27083496)
\curveto(444.58400193,336.31082763)(444.61900189,336.3408276)(444.65900391,336.36083496)
\curveto(444.9090016,336.53082741)(445.19900131,336.67082727)(445.52900391,336.78083496)
\curveto(445.59900091,336.80082714)(445.66900084,336.81582713)(445.73900391,336.82583496)
\curveto(445.81900069,336.83582711)(445.89900061,336.85082709)(445.97900391,336.87083496)
\curveto(446.04900046,336.89082705)(446.13900037,336.90082704)(446.24900391,336.90083496)
\curveto(446.35900015,336.91082703)(446.46900004,336.91582703)(446.57900391,336.91583496)
\curveto(446.68899982,336.91582703)(446.79399972,336.91082703)(446.89400391,336.90083496)
\curveto(447.00399951,336.89082705)(447.09399942,336.87582707)(447.16400391,336.85583496)
\curveto(447.3139992,336.80582714)(447.45899905,336.76082718)(447.59900391,336.72083496)
\curveto(447.73899877,336.68082726)(447.86899864,336.62582732)(447.98900391,336.55583496)
\curveto(448.05899845,336.50582744)(448.12399839,336.45582749)(448.18400391,336.40583496)
\curveto(448.24399827,336.36582758)(448.3089982,336.32082762)(448.37900391,336.27083496)
\curveto(448.41899809,336.2408277)(448.47399804,336.20082774)(448.54400391,336.15083496)
\curveto(448.62399789,336.10082784)(448.69899781,336.10082784)(448.76900391,336.15083496)
\curveto(448.8089977,336.17082777)(448.82899768,336.20582774)(448.82900391,336.25583496)
\curveto(448.82899768,336.30582764)(448.83899767,336.35582759)(448.85900391,336.40583496)
\lineto(448.85900391,336.55583496)
\curveto(448.86899764,336.58582736)(448.87399764,336.62082732)(448.87400391,336.66083496)
\lineto(448.87400391,336.78083496)
\lineto(448.87400391,338.82083496)
\curveto(448.87399764,338.93082501)(448.86899764,339.05082489)(448.85900391,339.18083496)
\curveto(448.85899765,339.32082462)(448.88399763,339.42582452)(448.93400391,339.49583496)
\curveto(448.97399754,339.57582437)(449.04899746,339.62582432)(449.15900391,339.64583496)
\curveto(449.17899733,339.65582429)(449.19899731,339.65582429)(449.21900391,339.64583496)
\curveto(449.23899727,339.6458243)(449.25899725,339.65082429)(449.27900391,339.66083496)
\lineto(450.34400391,339.66083496)
\curveto(450.46399605,339.66082428)(450.57399594,339.65582429)(450.67400391,339.64583496)
\curveto(450.77399574,339.63582431)(450.84899566,339.59582435)(450.89900391,339.52583496)
\curveto(450.94899556,339.4458245)(450.97399554,339.3408246)(450.97400391,339.21083496)
\lineto(450.97400391,338.85083496)
\lineto(450.97400391,329.83583496)
\moveto(448.93400391,332.77583496)
\curveto(448.94399757,332.81583113)(448.94399757,332.85583109)(448.93400391,332.89583496)
\lineto(448.93400391,333.03083496)
\curveto(448.93399758,333.13083081)(448.92899758,333.23083071)(448.91900391,333.33083496)
\curveto(448.9089976,333.43083051)(448.89399762,333.52083042)(448.87400391,333.60083496)
\curveto(448.85399766,333.71083023)(448.83399768,333.81083013)(448.81400391,333.90083496)
\curveto(448.80399771,333.99082995)(448.77899773,334.07582987)(448.73900391,334.15583496)
\curveto(448.59899791,334.51582943)(448.39399812,334.80082914)(448.12400391,335.01083496)
\curveto(447.86399865,335.22082872)(447.48399903,335.32582862)(446.98400391,335.32583496)
\curveto(446.92399959,335.32582862)(446.84399967,335.31582863)(446.74400391,335.29583496)
\curveto(446.66399985,335.27582867)(446.58899992,335.25582869)(446.51900391,335.23583496)
\curveto(446.45900005,335.22582872)(446.39900011,335.20582874)(446.33900391,335.17583496)
\curveto(446.06900044,335.06582888)(445.85900065,334.89582905)(445.70900391,334.66583496)
\curveto(445.55900095,334.43582951)(445.43900107,334.17582977)(445.34900391,333.88583496)
\curveto(445.31900119,333.78583016)(445.29900121,333.68583026)(445.28900391,333.58583496)
\curveto(445.27900123,333.48583046)(445.25900125,333.38083056)(445.22900391,333.27083496)
\lineto(445.22900391,333.06083496)
\curveto(445.2090013,332.97083097)(445.20400131,332.8458311)(445.21400391,332.68583496)
\curveto(445.22400129,332.53583141)(445.23900127,332.42583152)(445.25900391,332.35583496)
\lineto(445.25900391,332.26583496)
\curveto(445.26900124,332.2458317)(445.27400124,332.22583172)(445.27400391,332.20583496)
\curveto(445.29400122,332.12583182)(445.3090012,332.05083189)(445.31900391,331.98083496)
\curveto(445.33900117,331.91083203)(445.35900115,331.83583211)(445.37900391,331.75583496)
\curveto(445.54900096,331.23583271)(445.83900067,330.85083309)(446.24900391,330.60083496)
\curveto(446.37900013,330.51083343)(446.55899995,330.4408335)(446.78900391,330.39083496)
\curveto(446.82899968,330.38083356)(446.88899962,330.37583357)(446.96900391,330.37583496)
\curveto(446.99899951,330.36583358)(447.04399947,330.35583359)(447.10400391,330.34583496)
\curveto(447.17399934,330.3458336)(447.22899928,330.35083359)(447.26900391,330.36083496)
\curveto(447.34899916,330.38083356)(447.42899908,330.39583355)(447.50900391,330.40583496)
\curveto(447.58899892,330.41583353)(447.66899884,330.43583351)(447.74900391,330.46583496)
\curveto(447.99899851,330.57583337)(448.19899831,330.71583323)(448.34900391,330.88583496)
\curveto(448.49899801,331.05583289)(448.62899788,331.27083267)(448.73900391,331.53083496)
\curveto(448.77899773,331.62083232)(448.8089977,331.71083223)(448.82900391,331.80083496)
\curveto(448.84899766,331.90083204)(448.86899764,332.00583194)(448.88900391,332.11583496)
\curveto(448.89899761,332.16583178)(448.89899761,332.21083173)(448.88900391,332.25083496)
\curveto(448.88899762,332.30083164)(448.89899761,332.35083159)(448.91900391,332.40083496)
\curveto(448.92899758,332.43083151)(448.93399758,332.46583148)(448.93400391,332.50583496)
\lineto(448.93400391,332.64083496)
\lineto(448.93400391,332.77583496)
}
}
{
\newrgbcolor{curcolor}{0 0 0}
\pscustom[linestyle=none,fillstyle=solid,fillcolor=curcolor]
{
\newpath
\moveto(460.32392578,333.16583496)
\curveto(460.34391721,333.10583084)(460.3539172,333.02083092)(460.35392578,332.91083496)
\curveto(460.3539172,332.80083114)(460.34391721,332.71583123)(460.32392578,332.65583496)
\lineto(460.32392578,332.50583496)
\curveto(460.30391725,332.42583152)(460.29391726,332.3458316)(460.29392578,332.26583496)
\curveto(460.30391725,332.18583176)(460.29891726,332.10583184)(460.27892578,332.02583496)
\curveto(460.2589173,331.95583199)(460.24391731,331.89083205)(460.23392578,331.83083496)
\curveto(460.22391733,331.77083217)(460.21391734,331.70583224)(460.20392578,331.63583496)
\curveto(460.16391739,331.52583242)(460.12891743,331.41083253)(460.09892578,331.29083496)
\curveto(460.06891749,331.18083276)(460.02891753,331.07583287)(459.97892578,330.97583496)
\curveto(459.76891779,330.49583345)(459.49391806,330.10583384)(459.15392578,329.80583496)
\curveto(458.81391874,329.50583444)(458.40391915,329.25583469)(457.92392578,329.05583496)
\curveto(457.80391975,329.00583494)(457.67891988,328.97083497)(457.54892578,328.95083496)
\curveto(457.42892013,328.92083502)(457.30392025,328.89083505)(457.17392578,328.86083496)
\curveto(457.12392043,328.8408351)(457.06892049,328.83083511)(457.00892578,328.83083496)
\curveto(456.94892061,328.83083511)(456.89392066,328.82583512)(456.84392578,328.81583496)
\lineto(456.73892578,328.81583496)
\curveto(456.70892085,328.80583514)(456.67892088,328.80083514)(456.64892578,328.80083496)
\curveto(456.59892096,328.79083515)(456.51892104,328.78583516)(456.40892578,328.78583496)
\curveto(456.29892126,328.77583517)(456.21392134,328.78083516)(456.15392578,328.80083496)
\lineto(456.00392578,328.80083496)
\curveto(455.9539216,328.81083513)(455.89892166,328.81583513)(455.83892578,328.81583496)
\curveto(455.78892177,328.80583514)(455.73892182,328.81083513)(455.68892578,328.83083496)
\curveto(455.64892191,328.8408351)(455.60892195,328.8458351)(455.56892578,328.84583496)
\curveto(455.53892202,328.8458351)(455.49892206,328.85083509)(455.44892578,328.86083496)
\curveto(455.34892221,328.89083505)(455.24892231,328.91583503)(455.14892578,328.93583496)
\curveto(455.04892251,328.95583499)(454.9539226,328.98583496)(454.86392578,329.02583496)
\curveto(454.74392281,329.06583488)(454.62892293,329.10583484)(454.51892578,329.14583496)
\curveto(454.41892314,329.18583476)(454.31392324,329.23583471)(454.20392578,329.29583496)
\curveto(453.8539237,329.50583444)(453.553924,329.75083419)(453.30392578,330.03083496)
\curveto(453.0539245,330.31083363)(452.84392471,330.6458333)(452.67392578,331.03583496)
\curveto(452.62392493,331.12583282)(452.58392497,331.22083272)(452.55392578,331.32083496)
\curveto(452.53392502,331.42083252)(452.50892505,331.52583242)(452.47892578,331.63583496)
\curveto(452.4589251,331.68583226)(452.44892511,331.73083221)(452.44892578,331.77083496)
\curveto(452.44892511,331.81083213)(452.43892512,331.85583209)(452.41892578,331.90583496)
\curveto(452.39892516,331.98583196)(452.38892517,332.06583188)(452.38892578,332.14583496)
\curveto(452.38892517,332.23583171)(452.37892518,332.32083162)(452.35892578,332.40083496)
\curveto(452.34892521,332.45083149)(452.34392521,332.49583145)(452.34392578,332.53583496)
\lineto(452.34392578,332.67083496)
\curveto(452.32392523,332.73083121)(452.31392524,332.81583113)(452.31392578,332.92583496)
\curveto(452.32392523,333.03583091)(452.33892522,333.12083082)(452.35892578,333.18083496)
\lineto(452.35892578,333.28583496)
\curveto(452.36892519,333.33583061)(452.36892519,333.38583056)(452.35892578,333.43583496)
\curveto(452.3589252,333.49583045)(452.36892519,333.55083039)(452.38892578,333.60083496)
\curveto(452.39892516,333.65083029)(452.40392515,333.69583025)(452.40392578,333.73583496)
\curveto(452.40392515,333.78583016)(452.41392514,333.83583011)(452.43392578,333.88583496)
\curveto(452.47392508,334.01582993)(452.50892505,334.1408298)(452.53892578,334.26083496)
\curveto(452.56892499,334.39082955)(452.60892495,334.51582943)(452.65892578,334.63583496)
\curveto(452.83892472,335.0458289)(453.0539245,335.38582856)(453.30392578,335.65583496)
\curveto(453.553924,335.93582801)(453.8589237,336.19082775)(454.21892578,336.42083496)
\curveto(454.31892324,336.47082747)(454.42392313,336.51582743)(454.53392578,336.55583496)
\curveto(454.64392291,336.59582735)(454.7539228,336.6408273)(454.86392578,336.69083496)
\curveto(454.99392256,336.7408272)(455.12892243,336.77582717)(455.26892578,336.79583496)
\curveto(455.40892215,336.81582713)(455.553922,336.8458271)(455.70392578,336.88583496)
\curveto(455.78392177,336.89582705)(455.8589217,336.90082704)(455.92892578,336.90083496)
\curveto(455.99892156,336.90082704)(456.06892149,336.90582704)(456.13892578,336.91583496)
\curveto(456.71892084,336.92582702)(457.21892034,336.86582708)(457.63892578,336.73583496)
\curveto(458.06891949,336.60582734)(458.44891911,336.42582752)(458.77892578,336.19583496)
\curveto(458.88891867,336.11582783)(458.99891856,336.02582792)(459.10892578,335.92583496)
\curveto(459.22891833,335.83582811)(459.32891823,335.73582821)(459.40892578,335.62583496)
\curveto(459.48891807,335.52582842)(459.558918,335.42582852)(459.61892578,335.32583496)
\curveto(459.68891787,335.22582872)(459.7589178,335.12082882)(459.82892578,335.01083496)
\curveto(459.89891766,334.90082904)(459.9539176,334.78082916)(459.99392578,334.65083496)
\curveto(460.03391752,334.53082941)(460.07891748,334.40082954)(460.12892578,334.26083496)
\curveto(460.1589174,334.18082976)(460.18391737,334.09582985)(460.20392578,334.00583496)
\lineto(460.26392578,333.73583496)
\curveto(460.27391728,333.69583025)(460.27891728,333.65583029)(460.27892578,333.61583496)
\curveto(460.27891728,333.57583037)(460.28391727,333.53583041)(460.29392578,333.49583496)
\curveto(460.31391724,333.4458305)(460.31891724,333.39083055)(460.30892578,333.33083496)
\curveto(460.29891726,333.27083067)(460.30391725,333.21583073)(460.32392578,333.16583496)
\moveto(458.22392578,332.62583496)
\curveto(458.23391932,332.67583127)(458.23891932,332.7458312)(458.23892578,332.83583496)
\curveto(458.23891932,332.93583101)(458.23391932,333.01083093)(458.22392578,333.06083496)
\lineto(458.22392578,333.18083496)
\curveto(458.20391935,333.23083071)(458.19391936,333.28583066)(458.19392578,333.34583496)
\curveto(458.19391936,333.40583054)(458.18891937,333.46083048)(458.17892578,333.51083496)
\curveto(458.17891938,333.55083039)(458.17391938,333.58083036)(458.16392578,333.60083496)
\lineto(458.10392578,333.84083496)
\curveto(458.09391946,333.93083001)(458.07391948,334.01582993)(458.04392578,334.09583496)
\curveto(457.93391962,334.35582959)(457.80391975,334.57582937)(457.65392578,334.75583496)
\curveto(457.50392005,334.945829)(457.30392025,335.09582885)(457.05392578,335.20583496)
\curveto(456.99392056,335.22582872)(456.93392062,335.2408287)(456.87392578,335.25083496)
\curveto(456.81392074,335.27082867)(456.74892081,335.29082865)(456.67892578,335.31083496)
\curveto(456.59892096,335.33082861)(456.51392104,335.33582861)(456.42392578,335.32583496)
\lineto(456.15392578,335.32583496)
\curveto(456.12392143,335.30582864)(456.08892147,335.29582865)(456.04892578,335.29583496)
\curveto(456.00892155,335.30582864)(455.97392158,335.30582864)(455.94392578,335.29583496)
\lineto(455.73392578,335.23583496)
\curveto(455.67392188,335.22582872)(455.61892194,335.20582874)(455.56892578,335.17583496)
\curveto(455.31892224,335.06582888)(455.11392244,334.90582904)(454.95392578,334.69583496)
\curveto(454.80392275,334.49582945)(454.68392287,334.26082968)(454.59392578,333.99083496)
\curveto(454.56392299,333.89083005)(454.53892302,333.78583016)(454.51892578,333.67583496)
\curveto(454.50892305,333.56583038)(454.49392306,333.45583049)(454.47392578,333.34583496)
\curveto(454.46392309,333.29583065)(454.4589231,333.2458307)(454.45892578,333.19583496)
\lineto(454.45892578,333.04583496)
\curveto(454.43892312,332.97583097)(454.42892313,332.87083107)(454.42892578,332.73083496)
\curveto(454.43892312,332.59083135)(454.4539231,332.48583146)(454.47392578,332.41583496)
\lineto(454.47392578,332.28083496)
\curveto(454.49392306,332.20083174)(454.50892305,332.12083182)(454.51892578,332.04083496)
\curveto(454.52892303,331.97083197)(454.54392301,331.89583205)(454.56392578,331.81583496)
\curveto(454.66392289,331.51583243)(454.76892279,331.27083267)(454.87892578,331.08083496)
\curveto(454.99892256,330.90083304)(455.18392237,330.73583321)(455.43392578,330.58583496)
\curveto(455.50392205,330.53583341)(455.57892198,330.49583345)(455.65892578,330.46583496)
\curveto(455.74892181,330.43583351)(455.83892172,330.41083353)(455.92892578,330.39083496)
\curveto(455.96892159,330.38083356)(456.00392155,330.37583357)(456.03392578,330.37583496)
\curveto(456.06392149,330.38583356)(456.09892146,330.38583356)(456.13892578,330.37583496)
\lineto(456.25892578,330.34583496)
\curveto(456.30892125,330.3458336)(456.3539212,330.35083359)(456.39392578,330.36083496)
\lineto(456.51392578,330.36083496)
\curveto(456.59392096,330.38083356)(456.67392088,330.39583355)(456.75392578,330.40583496)
\curveto(456.83392072,330.41583353)(456.90892065,330.43583351)(456.97892578,330.46583496)
\curveto(457.23892032,330.56583338)(457.44892011,330.70083324)(457.60892578,330.87083496)
\curveto(457.76891979,331.0408329)(457.90391965,331.25083269)(458.01392578,331.50083496)
\curveto(458.0539195,331.60083234)(458.08391947,331.70083224)(458.10392578,331.80083496)
\curveto(458.12391943,331.90083204)(458.14891941,332.00583194)(458.17892578,332.11583496)
\curveto(458.18891937,332.15583179)(458.19391936,332.19083175)(458.19392578,332.22083496)
\curveto(458.19391936,332.26083168)(458.19891936,332.30083164)(458.20892578,332.34083496)
\lineto(458.20892578,332.47583496)
\curveto(458.20891935,332.52583142)(458.21391934,332.57583137)(458.22392578,332.62583496)
}
}
{
\newrgbcolor{curcolor}{0 0 0}
\pscustom[linestyle=none,fillstyle=solid,fillcolor=curcolor]
{
\newpath
\moveto(466.14884766,336.91583496)
\curveto(466.25884234,336.91582703)(466.35384225,336.90582704)(466.43384766,336.88583496)
\curveto(466.52384208,336.86582708)(466.59384201,336.82082712)(466.64384766,336.75083496)
\curveto(466.7038419,336.67082727)(466.73384187,336.53082741)(466.73384766,336.33083496)
\lineto(466.73384766,335.82083496)
\lineto(466.73384766,335.44583496)
\curveto(466.74384186,335.30582864)(466.72884187,335.19582875)(466.68884766,335.11583496)
\curveto(466.64884195,335.0458289)(466.58884201,335.00082894)(466.50884766,334.98083496)
\curveto(466.43884216,334.96082898)(466.35384225,334.95082899)(466.25384766,334.95083496)
\curveto(466.16384244,334.95082899)(466.06384254,334.95582899)(465.95384766,334.96583496)
\curveto(465.85384275,334.97582897)(465.75884284,334.97082897)(465.66884766,334.95083496)
\curveto(465.598843,334.93082901)(465.52884307,334.91582903)(465.45884766,334.90583496)
\curveto(465.38884321,334.90582904)(465.32384328,334.89582905)(465.26384766,334.87583496)
\curveto(465.1038435,334.82582912)(464.94384366,334.75082919)(464.78384766,334.65083496)
\curveto(464.62384398,334.56082938)(464.4988441,334.45582949)(464.40884766,334.33583496)
\curveto(464.35884424,334.25582969)(464.3038443,334.17082977)(464.24384766,334.08083496)
\curveto(464.19384441,334.00082994)(464.14384446,333.91583003)(464.09384766,333.82583496)
\curveto(464.06384454,333.7458302)(464.03384457,333.66083028)(464.00384766,333.57083496)
\lineto(463.94384766,333.33083496)
\curveto(463.92384468,333.26083068)(463.91384469,333.18583076)(463.91384766,333.10583496)
\curveto(463.91384469,333.03583091)(463.9038447,332.96583098)(463.88384766,332.89583496)
\curveto(463.87384473,332.85583109)(463.86884473,332.81583113)(463.86884766,332.77583496)
\curveto(463.87884472,332.7458312)(463.87884472,332.71583123)(463.86884766,332.68583496)
\lineto(463.86884766,332.44583496)
\curveto(463.84884475,332.37583157)(463.84384476,332.29583165)(463.85384766,332.20583496)
\curveto(463.86384474,332.12583182)(463.86884473,332.0458319)(463.86884766,331.96583496)
\lineto(463.86884766,331.00583496)
\lineto(463.86884766,329.73083496)
\curveto(463.86884473,329.60083434)(463.86384474,329.48083446)(463.85384766,329.37083496)
\curveto(463.84384476,329.26083468)(463.81384479,329.17083477)(463.76384766,329.10083496)
\curveto(463.74384486,329.07083487)(463.70884489,329.0458349)(463.65884766,329.02583496)
\curveto(463.61884498,329.01583493)(463.57384503,329.00583494)(463.52384766,328.99583496)
\lineto(463.44884766,328.99583496)
\curveto(463.3988452,328.98583496)(463.34384526,328.98083496)(463.28384766,328.98083496)
\lineto(463.11884766,328.98083496)
\lineto(462.47384766,328.98083496)
\curveto(462.41384619,328.99083495)(462.34884625,328.99583495)(462.27884766,328.99583496)
\lineto(462.08384766,328.99583496)
\curveto(462.03384657,329.01583493)(461.98384662,329.03083491)(461.93384766,329.04083496)
\curveto(461.88384672,329.06083488)(461.84884675,329.09583485)(461.82884766,329.14583496)
\curveto(461.78884681,329.19583475)(461.76384684,329.26583468)(461.75384766,329.35583496)
\lineto(461.75384766,329.65583496)
\lineto(461.75384766,330.67583496)
\lineto(461.75384766,334.90583496)
\lineto(461.75384766,336.01583496)
\lineto(461.75384766,336.30083496)
\curveto(461.75384685,336.40082754)(461.77384683,336.48082746)(461.81384766,336.54083496)
\curveto(461.86384674,336.62082732)(461.93884666,336.67082727)(462.03884766,336.69083496)
\curveto(462.13884646,336.71082723)(462.25884634,336.72082722)(462.39884766,336.72083496)
\lineto(463.16384766,336.72083496)
\curveto(463.28384532,336.72082722)(463.38884521,336.71082723)(463.47884766,336.69083496)
\curveto(463.56884503,336.68082726)(463.63884496,336.63582731)(463.68884766,336.55583496)
\curveto(463.71884488,336.50582744)(463.73384487,336.43582751)(463.73384766,336.34583496)
\lineto(463.76384766,336.07583496)
\curveto(463.77384483,335.99582795)(463.78884481,335.92082802)(463.80884766,335.85083496)
\curveto(463.83884476,335.78082816)(463.88884471,335.7458282)(463.95884766,335.74583496)
\curveto(463.97884462,335.76582818)(463.9988446,335.77582817)(464.01884766,335.77583496)
\curveto(464.03884456,335.77582817)(464.05884454,335.78582816)(464.07884766,335.80583496)
\curveto(464.13884446,335.85582809)(464.18884441,335.91082803)(464.22884766,335.97083496)
\curveto(464.27884432,336.0408279)(464.33884426,336.10082784)(464.40884766,336.15083496)
\curveto(464.44884415,336.18082776)(464.48384412,336.21082773)(464.51384766,336.24083496)
\curveto(464.54384406,336.28082766)(464.57884402,336.31582763)(464.61884766,336.34583496)
\lineto(464.88884766,336.52583496)
\curveto(464.98884361,336.58582736)(465.08884351,336.6408273)(465.18884766,336.69083496)
\curveto(465.28884331,336.73082721)(465.38884321,336.76582718)(465.48884766,336.79583496)
\lineto(465.81884766,336.88583496)
\curveto(465.84884275,336.89582705)(465.9038427,336.89582705)(465.98384766,336.88583496)
\curveto(466.07384253,336.88582706)(466.12884247,336.89582705)(466.14884766,336.91583496)
}
}
{
\newrgbcolor{curcolor}{0 0 0}
\pscustom[linestyle=none,fillstyle=solid,fillcolor=curcolor]
{
\newpath
\moveto(474.65525391,332.92583496)
\curveto(474.67524574,332.8458311)(474.67524574,332.75583119)(474.65525391,332.65583496)
\curveto(474.63524578,332.55583139)(474.60024582,332.49083145)(474.55025391,332.46083496)
\curveto(474.50024592,332.42083152)(474.42524599,332.39083155)(474.32525391,332.37083496)
\curveto(474.23524618,332.36083158)(474.13024629,332.35083159)(474.01025391,332.34083496)
\lineto(473.66525391,332.34083496)
\curveto(473.55524686,332.35083159)(473.45524696,332.35583159)(473.36525391,332.35583496)
\lineto(469.70525391,332.35583496)
\lineto(469.49525391,332.35583496)
\curveto(469.43525098,332.35583159)(469.38025104,332.3458316)(469.33025391,332.32583496)
\curveto(469.25025117,332.28583166)(469.20025122,332.2458317)(469.18025391,332.20583496)
\curveto(469.16025126,332.18583176)(469.14025128,332.1458318)(469.12025391,332.08583496)
\curveto(469.10025132,332.03583191)(469.09525132,331.98583196)(469.10525391,331.93583496)
\curveto(469.12525129,331.87583207)(469.13525128,331.81583213)(469.13525391,331.75583496)
\curveto(469.14525127,331.70583224)(469.16025126,331.65083229)(469.18025391,331.59083496)
\curveto(469.26025116,331.35083259)(469.35525106,331.15083279)(469.46525391,330.99083496)
\curveto(469.58525083,330.8408331)(469.74525067,330.70583324)(469.94525391,330.58583496)
\curveto(470.02525039,330.53583341)(470.10525031,330.50083344)(470.18525391,330.48083496)
\curveto(470.27525014,330.47083347)(470.36525005,330.45083349)(470.45525391,330.42083496)
\curveto(470.53524988,330.40083354)(470.64524977,330.38583356)(470.78525391,330.37583496)
\curveto(470.92524949,330.36583358)(471.04524937,330.37083357)(471.14525391,330.39083496)
\lineto(471.28025391,330.39083496)
\curveto(471.38024904,330.41083353)(471.47024895,330.43083351)(471.55025391,330.45083496)
\curveto(471.64024878,330.48083346)(471.72524869,330.51083343)(471.80525391,330.54083496)
\curveto(471.90524851,330.59083335)(472.0152484,330.65583329)(472.13525391,330.73583496)
\curveto(472.26524815,330.81583313)(472.36024806,330.89583305)(472.42025391,330.97583496)
\curveto(472.47024795,331.0458329)(472.5202479,331.11083283)(472.57025391,331.17083496)
\curveto(472.63024779,331.2408327)(472.70024772,331.29083265)(472.78025391,331.32083496)
\curveto(472.88024754,331.37083257)(473.00524741,331.39083255)(473.15525391,331.38083496)
\lineto(473.59025391,331.38083496)
\lineto(473.77025391,331.38083496)
\curveto(473.84024658,331.39083255)(473.90024652,331.38583256)(473.95025391,331.36583496)
\lineto(474.10025391,331.36583496)
\curveto(474.20024622,331.3458326)(474.27024615,331.32083262)(474.31025391,331.29083496)
\curveto(474.35024607,331.27083267)(474.37024605,331.22583272)(474.37025391,331.15583496)
\curveto(474.38024604,331.08583286)(474.37524604,331.02583292)(474.35525391,330.97583496)
\curveto(474.30524611,330.83583311)(474.25024617,330.71083323)(474.19025391,330.60083496)
\curveto(474.13024629,330.49083345)(474.06024636,330.38083356)(473.98025391,330.27083496)
\curveto(473.76024666,329.940834)(473.51024691,329.67583427)(473.23025391,329.47583496)
\curveto(472.95024747,329.27583467)(472.60024782,329.10583484)(472.18025391,328.96583496)
\curveto(472.07024835,328.92583502)(471.96024846,328.90083504)(471.85025391,328.89083496)
\curveto(471.74024868,328.88083506)(471.62524879,328.86083508)(471.50525391,328.83083496)
\curveto(471.46524895,328.82083512)(471.420249,328.82083512)(471.37025391,328.83083496)
\curveto(471.33024909,328.83083511)(471.29024913,328.82583512)(471.25025391,328.81583496)
\lineto(471.08525391,328.81583496)
\curveto(471.03524938,328.79583515)(470.97524944,328.79083515)(470.90525391,328.80083496)
\curveto(470.84524957,328.80083514)(470.79024963,328.80583514)(470.74025391,328.81583496)
\curveto(470.66024976,328.82583512)(470.59024983,328.82583512)(470.53025391,328.81583496)
\curveto(470.47024995,328.80583514)(470.40525001,328.81083513)(470.33525391,328.83083496)
\curveto(470.28525013,328.85083509)(470.23025019,328.86083508)(470.17025391,328.86083496)
\curveto(470.11025031,328.86083508)(470.05525036,328.87083507)(470.00525391,328.89083496)
\curveto(469.89525052,328.91083503)(469.78525063,328.93583501)(469.67525391,328.96583496)
\curveto(469.56525085,328.98583496)(469.46525095,329.02083492)(469.37525391,329.07083496)
\curveto(469.26525115,329.11083483)(469.16025126,329.1458348)(469.06025391,329.17583496)
\curveto(468.97025145,329.21583473)(468.88525153,329.26083468)(468.80525391,329.31083496)
\curveto(468.48525193,329.51083443)(468.20025222,329.7408342)(467.95025391,330.00083496)
\curveto(467.70025272,330.27083367)(467.49525292,330.58083336)(467.33525391,330.93083496)
\curveto(467.28525313,331.0408329)(467.24525317,331.15083279)(467.21525391,331.26083496)
\curveto(467.18525323,331.38083256)(467.14525327,331.50083244)(467.09525391,331.62083496)
\curveto(467.08525333,331.66083228)(467.08025334,331.69583225)(467.08025391,331.72583496)
\curveto(467.08025334,331.76583218)(467.07525334,331.80583214)(467.06525391,331.84583496)
\curveto(467.02525339,331.96583198)(467.00025342,332.09583185)(466.99025391,332.23583496)
\lineto(466.96025391,332.65583496)
\curveto(466.96025346,332.70583124)(466.95525346,332.76083118)(466.94525391,332.82083496)
\curveto(466.94525347,332.88083106)(466.95025347,332.93583101)(466.96025391,332.98583496)
\lineto(466.96025391,333.16583496)
\lineto(467.00525391,333.52583496)
\curveto(467.04525337,333.69583025)(467.08025334,333.86083008)(467.11025391,334.02083496)
\curveto(467.14025328,334.18082976)(467.18525323,334.33082961)(467.24525391,334.47083496)
\curveto(467.67525274,335.51082843)(468.40525201,336.2458277)(469.43525391,336.67583496)
\curveto(469.57525084,336.73582721)(469.7152507,336.77582717)(469.85525391,336.79583496)
\curveto(470.00525041,336.82582712)(470.16025026,336.86082708)(470.32025391,336.90083496)
\curveto(470.40025002,336.91082703)(470.47524994,336.91582703)(470.54525391,336.91583496)
\curveto(470.6152498,336.91582703)(470.69024973,336.92082702)(470.77025391,336.93083496)
\curveto(471.28024914,336.940827)(471.7152487,336.88082706)(472.07525391,336.75083496)
\curveto(472.44524797,336.63082731)(472.77524764,336.47082747)(473.06525391,336.27083496)
\curveto(473.15524726,336.21082773)(473.24524717,336.1408278)(473.33525391,336.06083496)
\curveto(473.42524699,335.99082795)(473.50524691,335.91582803)(473.57525391,335.83583496)
\curveto(473.60524681,335.78582816)(473.64524677,335.7458282)(473.69525391,335.71583496)
\curveto(473.77524664,335.60582834)(473.85024657,335.49082845)(473.92025391,335.37083496)
\curveto(473.99024643,335.26082868)(474.06524635,335.1458288)(474.14525391,335.02583496)
\curveto(474.19524622,334.93582901)(474.23524618,334.8408291)(474.26525391,334.74083496)
\curveto(474.30524611,334.65082929)(474.34524607,334.55082939)(474.38525391,334.44083496)
\curveto(474.43524598,334.31082963)(474.47524594,334.17582977)(474.50525391,334.03583496)
\curveto(474.53524588,333.89583005)(474.57024585,333.75583019)(474.61025391,333.61583496)
\curveto(474.63024579,333.53583041)(474.63524578,333.4458305)(474.62525391,333.34583496)
\curveto(474.62524579,333.25583069)(474.63524578,333.17083077)(474.65525391,333.09083496)
\lineto(474.65525391,332.92583496)
\moveto(472.40525391,333.81083496)
\curveto(472.47524794,333.91083003)(472.48024794,334.03082991)(472.42025391,334.17083496)
\curveto(472.37024805,334.32082962)(472.33024809,334.43082951)(472.30025391,334.50083496)
\curveto(472.16024826,334.77082917)(471.97524844,334.97582897)(471.74525391,335.11583496)
\curveto(471.5152489,335.26582868)(471.19524922,335.3458286)(470.78525391,335.35583496)
\curveto(470.75524966,335.33582861)(470.7202497,335.33082861)(470.68025391,335.34083496)
\curveto(470.64024978,335.35082859)(470.60524981,335.35082859)(470.57525391,335.34083496)
\curveto(470.52524989,335.32082862)(470.47024995,335.30582864)(470.41025391,335.29583496)
\curveto(470.35025007,335.29582865)(470.29525012,335.28582866)(470.24525391,335.26583496)
\curveto(469.80525061,335.12582882)(469.48025094,334.85082909)(469.27025391,334.44083496)
\curveto(469.25025117,334.40082954)(469.22525119,334.3458296)(469.19525391,334.27583496)
\curveto(469.17525124,334.21582973)(469.16025126,334.15082979)(469.15025391,334.08083496)
\curveto(469.14025128,334.02082992)(469.14025128,333.96082998)(469.15025391,333.90083496)
\curveto(469.17025125,333.8408301)(469.20525121,333.79083015)(469.25525391,333.75083496)
\curveto(469.33525108,333.70083024)(469.44525097,333.67583027)(469.58525391,333.67583496)
\lineto(469.99025391,333.67583496)
\lineto(471.65525391,333.67583496)
\lineto(472.09025391,333.67583496)
\curveto(472.25024817,333.68583026)(472.35524806,333.73083021)(472.40525391,333.81083496)
}
}
{
\newrgbcolor{curcolor}{0 0 0}
\pscustom[linestyle=none,fillstyle=solid,fillcolor=curcolor]
{
\newpath
\moveto(478.87353516,336.93083496)
\curveto(479.62353066,336.95082699)(480.27353001,336.86582708)(480.82353516,336.67583496)
\curveto(481.3835289,336.49582745)(481.80852847,336.18082776)(482.09853516,335.73083496)
\curveto(482.16852811,335.62082832)(482.22852805,335.50582844)(482.27853516,335.38583496)
\curveto(482.33852794,335.27582867)(482.38852789,335.15082879)(482.42853516,335.01083496)
\curveto(482.44852783,334.95082899)(482.45852782,334.88582906)(482.45853516,334.81583496)
\curveto(482.45852782,334.7458292)(482.44852783,334.68582926)(482.42853516,334.63583496)
\curveto(482.38852789,334.57582937)(482.33352795,334.53582941)(482.26353516,334.51583496)
\curveto(482.21352807,334.49582945)(482.15352813,334.48582946)(482.08353516,334.48583496)
\lineto(481.87353516,334.48583496)
\lineto(481.21353516,334.48583496)
\curveto(481.14352914,334.48582946)(481.07352921,334.48082946)(481.00353516,334.47083496)
\curveto(480.93352935,334.47082947)(480.86852941,334.48082946)(480.80853516,334.50083496)
\curveto(480.70852957,334.52082942)(480.63352965,334.56082938)(480.58353516,334.62083496)
\curveto(480.53352975,334.68082926)(480.48852979,334.7408292)(480.44853516,334.80083496)
\lineto(480.32853516,335.01083496)
\curveto(480.29852998,335.09082885)(480.24853003,335.15582879)(480.17853516,335.20583496)
\curveto(480.0785302,335.28582866)(479.9785303,335.3458286)(479.87853516,335.38583496)
\curveto(479.78853049,335.42582852)(479.67353061,335.46082848)(479.53353516,335.49083496)
\curveto(479.46353082,335.51082843)(479.35853092,335.52582842)(479.21853516,335.53583496)
\curveto(479.08853119,335.5458284)(478.98853129,335.5408284)(478.91853516,335.52083496)
\lineto(478.81353516,335.52083496)
\lineto(478.66353516,335.49083496)
\curveto(478.62353166,335.49082845)(478.5785317,335.48582846)(478.52853516,335.47583496)
\curveto(478.35853192,335.42582852)(478.21853206,335.35582859)(478.10853516,335.26583496)
\curveto(478.00853227,335.18582876)(477.93853234,335.06082888)(477.89853516,334.89083496)
\curveto(477.8785324,334.82082912)(477.8785324,334.75582919)(477.89853516,334.69583496)
\curveto(477.91853236,334.63582931)(477.93853234,334.58582936)(477.95853516,334.54583496)
\curveto(478.02853225,334.42582952)(478.10853217,334.33082961)(478.19853516,334.26083496)
\curveto(478.29853198,334.19082975)(478.41353187,334.13082981)(478.54353516,334.08083496)
\curveto(478.73353155,334.00082994)(478.93853134,333.93083001)(479.15853516,333.87083496)
\lineto(479.84853516,333.72083496)
\curveto(480.08853019,333.68083026)(480.31852996,333.63083031)(480.53853516,333.57083496)
\curveto(480.76852951,333.52083042)(480.9835293,333.45583049)(481.18353516,333.37583496)
\curveto(481.27352901,333.33583061)(481.35852892,333.30083064)(481.43853516,333.27083496)
\curveto(481.52852875,333.25083069)(481.61352867,333.21583073)(481.69353516,333.16583496)
\curveto(481.8835284,333.0458309)(482.05352823,332.91583103)(482.20353516,332.77583496)
\curveto(482.36352792,332.63583131)(482.48852779,332.46083148)(482.57853516,332.25083496)
\curveto(482.60852767,332.18083176)(482.63352765,332.11083183)(482.65353516,332.04083496)
\curveto(482.67352761,331.97083197)(482.69352759,331.89583205)(482.71353516,331.81583496)
\curveto(482.72352756,331.75583219)(482.72852755,331.66083228)(482.72853516,331.53083496)
\curveto(482.73852754,331.41083253)(482.73852754,331.31583263)(482.72853516,331.24583496)
\lineto(482.72853516,331.17083496)
\curveto(482.70852757,331.11083283)(482.69352759,331.05083289)(482.68353516,330.99083496)
\curveto(482.6835276,330.940833)(482.6785276,330.89083305)(482.66853516,330.84083496)
\curveto(482.59852768,330.5408334)(482.48852779,330.27583367)(482.33853516,330.04583496)
\curveto(482.1785281,329.80583414)(481.9835283,329.61083433)(481.75353516,329.46083496)
\curveto(481.52352876,329.31083463)(481.26352902,329.18083476)(480.97353516,329.07083496)
\curveto(480.86352942,329.02083492)(480.74352954,328.98583496)(480.61353516,328.96583496)
\curveto(480.49352979,328.945835)(480.37352991,328.92083502)(480.25353516,328.89083496)
\curveto(480.16353012,328.87083507)(480.06853021,328.86083508)(479.96853516,328.86083496)
\curveto(479.8785304,328.85083509)(479.78853049,328.83583511)(479.69853516,328.81583496)
\lineto(479.42853516,328.81583496)
\curveto(479.36853091,328.79583515)(479.26353102,328.78583516)(479.11353516,328.78583496)
\curveto(478.97353131,328.78583516)(478.87353141,328.79583515)(478.81353516,328.81583496)
\curveto(478.7835315,328.81583513)(478.74853153,328.82083512)(478.70853516,328.83083496)
\lineto(478.60353516,328.83083496)
\curveto(478.4835318,328.85083509)(478.36353192,328.86583508)(478.24353516,328.87583496)
\curveto(478.12353216,328.88583506)(478.00853227,328.90583504)(477.89853516,328.93583496)
\curveto(477.50853277,329.0458349)(477.16353312,329.17083477)(476.86353516,329.31083496)
\curveto(476.56353372,329.46083448)(476.30853397,329.68083426)(476.09853516,329.97083496)
\curveto(475.95853432,330.16083378)(475.83853444,330.38083356)(475.73853516,330.63083496)
\curveto(475.71853456,330.69083325)(475.69853458,330.77083317)(475.67853516,330.87083496)
\curveto(475.65853462,330.92083302)(475.64353464,330.99083295)(475.63353516,331.08083496)
\curveto(475.62353466,331.17083277)(475.62853465,331.2458327)(475.64853516,331.30583496)
\curveto(475.6785346,331.37583257)(475.72853455,331.42583252)(475.79853516,331.45583496)
\curveto(475.84853443,331.47583247)(475.90853437,331.48583246)(475.97853516,331.48583496)
\lineto(476.20353516,331.48583496)
\lineto(476.90853516,331.48583496)
\lineto(477.14853516,331.48583496)
\curveto(477.22853305,331.48583246)(477.29853298,331.47583247)(477.35853516,331.45583496)
\curveto(477.46853281,331.41583253)(477.53853274,331.35083259)(477.56853516,331.26083496)
\curveto(477.60853267,331.17083277)(477.65353263,331.07583287)(477.70353516,330.97583496)
\curveto(477.72353256,330.92583302)(477.75853252,330.86083308)(477.80853516,330.78083496)
\curveto(477.86853241,330.70083324)(477.91853236,330.65083329)(477.95853516,330.63083496)
\curveto(478.0785322,330.53083341)(478.19353209,330.45083349)(478.30353516,330.39083496)
\curveto(478.41353187,330.3408336)(478.55353173,330.29083365)(478.72353516,330.24083496)
\curveto(478.77353151,330.22083372)(478.82353146,330.21083373)(478.87353516,330.21083496)
\curveto(478.92353136,330.22083372)(478.97353131,330.22083372)(479.02353516,330.21083496)
\curveto(479.10353118,330.19083375)(479.18853109,330.18083376)(479.27853516,330.18083496)
\curveto(479.3785309,330.19083375)(479.46353082,330.20583374)(479.53353516,330.22583496)
\curveto(479.5835307,330.23583371)(479.62853065,330.2408337)(479.66853516,330.24083496)
\curveto(479.71853056,330.2408337)(479.76853051,330.25083369)(479.81853516,330.27083496)
\curveto(479.95853032,330.32083362)(480.0835302,330.38083356)(480.19353516,330.45083496)
\curveto(480.31352997,330.52083342)(480.40852987,330.61083333)(480.47853516,330.72083496)
\curveto(480.52852975,330.80083314)(480.56852971,330.92583302)(480.59853516,331.09583496)
\curveto(480.61852966,331.16583278)(480.61852966,331.23083271)(480.59853516,331.29083496)
\curveto(480.5785297,331.35083259)(480.55852972,331.40083254)(480.53853516,331.44083496)
\curveto(480.46852981,331.58083236)(480.3785299,331.68583226)(480.26853516,331.75583496)
\curveto(480.16853011,331.82583212)(480.04853023,331.89083205)(479.90853516,331.95083496)
\curveto(479.71853056,332.03083191)(479.51853076,332.09583185)(479.30853516,332.14583496)
\curveto(479.09853118,332.19583175)(478.88853139,332.25083169)(478.67853516,332.31083496)
\curveto(478.59853168,332.33083161)(478.51353177,332.3458316)(478.42353516,332.35583496)
\curveto(478.34353194,332.36583158)(478.26353202,332.38083156)(478.18353516,332.40083496)
\curveto(477.86353242,332.49083145)(477.55853272,332.57583137)(477.26853516,332.65583496)
\curveto(476.9785333,332.7458312)(476.71353357,332.87583107)(476.47353516,333.04583496)
\curveto(476.19353409,333.2458307)(475.98853429,333.51583043)(475.85853516,333.85583496)
\curveto(475.83853444,333.92583002)(475.81853446,334.02082992)(475.79853516,334.14083496)
\curveto(475.7785345,334.21082973)(475.76353452,334.29582965)(475.75353516,334.39583496)
\curveto(475.74353454,334.49582945)(475.74853453,334.58582936)(475.76853516,334.66583496)
\curveto(475.78853449,334.71582923)(475.79353449,334.75582919)(475.78353516,334.78583496)
\curveto(475.77353451,334.82582912)(475.7785345,334.87082907)(475.79853516,334.92083496)
\curveto(475.81853446,335.03082891)(475.83853444,335.13082881)(475.85853516,335.22083496)
\curveto(475.88853439,335.32082862)(475.92353436,335.41582853)(475.96353516,335.50583496)
\curveto(476.09353419,335.79582815)(476.27353401,336.03082791)(476.50353516,336.21083496)
\curveto(476.73353355,336.39082755)(476.99353329,336.53582741)(477.28353516,336.64583496)
\curveto(477.39353289,336.69582725)(477.50853277,336.73082721)(477.62853516,336.75083496)
\curveto(477.74853253,336.78082716)(477.87353241,336.81082713)(478.00353516,336.84083496)
\curveto(478.06353222,336.86082708)(478.12353216,336.87082707)(478.18353516,336.87083496)
\lineto(478.36353516,336.90083496)
\curveto(478.44353184,336.91082703)(478.52853175,336.91582703)(478.61853516,336.91583496)
\curveto(478.70853157,336.91582703)(478.79353149,336.92082702)(478.87353516,336.93083496)
}
}
{
\newrgbcolor{curcolor}{0 0 0}
\pscustom[linestyle=none,fillstyle=solid,fillcolor=curcolor]
{
}
}
{
\newrgbcolor{curcolor}{0 0 0}
\pscustom[linestyle=none,fillstyle=solid,fillcolor=curcolor]
{
\newpath
\moveto(495.71033203,329.83583496)
\lineto(495.71033203,329.41583496)
\curveto(495.71032366,329.28583466)(495.68032369,329.18083476)(495.62033203,329.10083496)
\curveto(495.5703238,329.05083489)(495.50532387,329.01583493)(495.42533203,328.99583496)
\curveto(495.34532403,328.98583496)(495.25532412,328.98083496)(495.15533203,328.98083496)
\lineto(494.33033203,328.98083496)
\lineto(494.04533203,328.98083496)
\curveto(493.96532541,328.99083495)(493.90032547,329.01583493)(493.85033203,329.05583496)
\curveto(493.78032559,329.10583484)(493.74032563,329.17083477)(493.73033203,329.25083496)
\curveto(493.72032565,329.33083461)(493.70032567,329.41083453)(493.67033203,329.49083496)
\curveto(493.65032572,329.51083443)(493.63032574,329.52583442)(493.61033203,329.53583496)
\curveto(493.60032577,329.55583439)(493.58532579,329.57583437)(493.56533203,329.59583496)
\curveto(493.45532592,329.59583435)(493.375326,329.57083437)(493.32533203,329.52083496)
\lineto(493.17533203,329.37083496)
\curveto(493.10532627,329.32083462)(493.04032633,329.27583467)(492.98033203,329.23583496)
\curveto(492.92032645,329.20583474)(492.85532652,329.16583478)(492.78533203,329.11583496)
\curveto(492.74532663,329.09583485)(492.70032667,329.07583487)(492.65033203,329.05583496)
\curveto(492.61032676,329.03583491)(492.56532681,329.01583493)(492.51533203,328.99583496)
\curveto(492.375327,328.945835)(492.22532715,328.90083504)(492.06533203,328.86083496)
\curveto(492.01532736,328.8408351)(491.9703274,328.83083511)(491.93033203,328.83083496)
\curveto(491.89032748,328.83083511)(491.85032752,328.82583512)(491.81033203,328.81583496)
\lineto(491.67533203,328.81583496)
\curveto(491.64532773,328.80583514)(491.60532777,328.80083514)(491.55533203,328.80083496)
\lineto(491.42033203,328.80083496)
\curveto(491.36032801,328.78083516)(491.2703281,328.77583517)(491.15033203,328.78583496)
\curveto(491.03032834,328.78583516)(490.94532843,328.79583515)(490.89533203,328.81583496)
\curveto(490.82532855,328.83583511)(490.76032861,328.8458351)(490.70033203,328.84583496)
\curveto(490.65032872,328.83583511)(490.59532878,328.8408351)(490.53533203,328.86083496)
\lineto(490.17533203,328.98083496)
\curveto(490.06532931,329.01083493)(489.95532942,329.05083489)(489.84533203,329.10083496)
\curveto(489.49532988,329.25083469)(489.18033019,329.48083446)(488.90033203,329.79083496)
\curveto(488.63033074,330.11083383)(488.41533096,330.4458335)(488.25533203,330.79583496)
\curveto(488.20533117,330.90583304)(488.16533121,331.01083293)(488.13533203,331.11083496)
\curveto(488.10533127,331.22083272)(488.0703313,331.33083261)(488.03033203,331.44083496)
\curveto(488.02033135,331.48083246)(488.01533136,331.51583243)(488.01533203,331.54583496)
\curveto(488.01533136,331.58583236)(488.00533137,331.63083231)(487.98533203,331.68083496)
\curveto(487.96533141,331.76083218)(487.94533143,331.8458321)(487.92533203,331.93583496)
\curveto(487.91533146,332.03583191)(487.90033147,332.13583181)(487.88033203,332.23583496)
\curveto(487.8703315,332.26583168)(487.86533151,332.30083164)(487.86533203,332.34083496)
\curveto(487.8753315,332.38083156)(487.8753315,332.41583153)(487.86533203,332.44583496)
\lineto(487.86533203,332.58083496)
\curveto(487.86533151,332.63083131)(487.86033151,332.68083126)(487.85033203,332.73083496)
\curveto(487.84033153,332.78083116)(487.83533154,332.83583111)(487.83533203,332.89583496)
\curveto(487.83533154,332.96583098)(487.84033153,333.02083092)(487.85033203,333.06083496)
\curveto(487.86033151,333.11083083)(487.86533151,333.15583079)(487.86533203,333.19583496)
\lineto(487.86533203,333.34583496)
\curveto(487.8753315,333.39583055)(487.8753315,333.4408305)(487.86533203,333.48083496)
\curveto(487.86533151,333.53083041)(487.8753315,333.58083036)(487.89533203,333.63083496)
\curveto(487.91533146,333.7408302)(487.93033144,333.8458301)(487.94033203,333.94583496)
\curveto(487.96033141,334.0458299)(487.98533139,334.1458298)(488.01533203,334.24583496)
\curveto(488.05533132,334.36582958)(488.09033128,334.48082946)(488.12033203,334.59083496)
\curveto(488.15033122,334.70082924)(488.19033118,334.81082913)(488.24033203,334.92083496)
\curveto(488.38033099,335.22082872)(488.55533082,335.50582844)(488.76533203,335.77583496)
\curveto(488.78533059,335.80582814)(488.81033056,335.83082811)(488.84033203,335.85083496)
\curveto(488.88033049,335.88082806)(488.91033046,335.91082803)(488.93033203,335.94083496)
\curveto(488.9703304,335.99082795)(489.01033036,336.03582791)(489.05033203,336.07583496)
\curveto(489.09033028,336.11582783)(489.13533024,336.15582779)(489.18533203,336.19583496)
\curveto(489.22533015,336.21582773)(489.26033011,336.2408277)(489.29033203,336.27083496)
\curveto(489.32033005,336.31082763)(489.35533002,336.3408276)(489.39533203,336.36083496)
\curveto(489.64532973,336.53082741)(489.93532944,336.67082727)(490.26533203,336.78083496)
\curveto(490.33532904,336.80082714)(490.40532897,336.81582713)(490.47533203,336.82583496)
\curveto(490.55532882,336.83582711)(490.63532874,336.85082709)(490.71533203,336.87083496)
\curveto(490.78532859,336.89082705)(490.8753285,336.90082704)(490.98533203,336.90083496)
\curveto(491.09532828,336.91082703)(491.20532817,336.91582703)(491.31533203,336.91583496)
\curveto(491.42532795,336.91582703)(491.53032784,336.91082703)(491.63033203,336.90083496)
\curveto(491.74032763,336.89082705)(491.83032754,336.87582707)(491.90033203,336.85583496)
\curveto(492.05032732,336.80582714)(492.19532718,336.76082718)(492.33533203,336.72083496)
\curveto(492.4753269,336.68082726)(492.60532677,336.62582732)(492.72533203,336.55583496)
\curveto(492.79532658,336.50582744)(492.86032651,336.45582749)(492.92033203,336.40583496)
\curveto(492.98032639,336.36582758)(493.04532633,336.32082762)(493.11533203,336.27083496)
\curveto(493.15532622,336.2408277)(493.21032616,336.20082774)(493.28033203,336.15083496)
\curveto(493.36032601,336.10082784)(493.43532594,336.10082784)(493.50533203,336.15083496)
\curveto(493.54532583,336.17082777)(493.56532581,336.20582774)(493.56533203,336.25583496)
\curveto(493.56532581,336.30582764)(493.5753258,336.35582759)(493.59533203,336.40583496)
\lineto(493.59533203,336.55583496)
\curveto(493.60532577,336.58582736)(493.61032576,336.62082732)(493.61033203,336.66083496)
\lineto(493.61033203,336.78083496)
\lineto(493.61033203,338.82083496)
\curveto(493.61032576,338.93082501)(493.60532577,339.05082489)(493.59533203,339.18083496)
\curveto(493.59532578,339.32082462)(493.62032575,339.42582452)(493.67033203,339.49583496)
\curveto(493.71032566,339.57582437)(493.78532559,339.62582432)(493.89533203,339.64583496)
\curveto(493.91532546,339.65582429)(493.93532544,339.65582429)(493.95533203,339.64583496)
\curveto(493.9753254,339.6458243)(493.99532538,339.65082429)(494.01533203,339.66083496)
\lineto(495.08033203,339.66083496)
\curveto(495.20032417,339.66082428)(495.31032406,339.65582429)(495.41033203,339.64583496)
\curveto(495.51032386,339.63582431)(495.58532379,339.59582435)(495.63533203,339.52583496)
\curveto(495.68532369,339.4458245)(495.71032366,339.3408246)(495.71033203,339.21083496)
\lineto(495.71033203,338.85083496)
\lineto(495.71033203,329.83583496)
\moveto(493.67033203,332.77583496)
\curveto(493.68032569,332.81583113)(493.68032569,332.85583109)(493.67033203,332.89583496)
\lineto(493.67033203,333.03083496)
\curveto(493.6703257,333.13083081)(493.66532571,333.23083071)(493.65533203,333.33083496)
\curveto(493.64532573,333.43083051)(493.63032574,333.52083042)(493.61033203,333.60083496)
\curveto(493.59032578,333.71083023)(493.5703258,333.81083013)(493.55033203,333.90083496)
\curveto(493.54032583,333.99082995)(493.51532586,334.07582987)(493.47533203,334.15583496)
\curveto(493.33532604,334.51582943)(493.13032624,334.80082914)(492.86033203,335.01083496)
\curveto(492.60032677,335.22082872)(492.22032715,335.32582862)(491.72033203,335.32583496)
\curveto(491.66032771,335.32582862)(491.58032779,335.31582863)(491.48033203,335.29583496)
\curveto(491.40032797,335.27582867)(491.32532805,335.25582869)(491.25533203,335.23583496)
\curveto(491.19532818,335.22582872)(491.13532824,335.20582874)(491.07533203,335.17583496)
\curveto(490.80532857,335.06582888)(490.59532878,334.89582905)(490.44533203,334.66583496)
\curveto(490.29532908,334.43582951)(490.1753292,334.17582977)(490.08533203,333.88583496)
\curveto(490.05532932,333.78583016)(490.03532934,333.68583026)(490.02533203,333.58583496)
\curveto(490.01532936,333.48583046)(489.99532938,333.38083056)(489.96533203,333.27083496)
\lineto(489.96533203,333.06083496)
\curveto(489.94532943,332.97083097)(489.94032943,332.8458311)(489.95033203,332.68583496)
\curveto(489.96032941,332.53583141)(489.9753294,332.42583152)(489.99533203,332.35583496)
\lineto(489.99533203,332.26583496)
\curveto(490.00532937,332.2458317)(490.01032936,332.22583172)(490.01033203,332.20583496)
\curveto(490.03032934,332.12583182)(490.04532933,332.05083189)(490.05533203,331.98083496)
\curveto(490.0753293,331.91083203)(490.09532928,331.83583211)(490.11533203,331.75583496)
\curveto(490.28532909,331.23583271)(490.5753288,330.85083309)(490.98533203,330.60083496)
\curveto(491.11532826,330.51083343)(491.29532808,330.4408335)(491.52533203,330.39083496)
\curveto(491.56532781,330.38083356)(491.62532775,330.37583357)(491.70533203,330.37583496)
\curveto(491.73532764,330.36583358)(491.78032759,330.35583359)(491.84033203,330.34583496)
\curveto(491.91032746,330.3458336)(491.96532741,330.35083359)(492.00533203,330.36083496)
\curveto(492.08532729,330.38083356)(492.16532721,330.39583355)(492.24533203,330.40583496)
\curveto(492.32532705,330.41583353)(492.40532697,330.43583351)(492.48533203,330.46583496)
\curveto(492.73532664,330.57583337)(492.93532644,330.71583323)(493.08533203,330.88583496)
\curveto(493.23532614,331.05583289)(493.36532601,331.27083267)(493.47533203,331.53083496)
\curveto(493.51532586,331.62083232)(493.54532583,331.71083223)(493.56533203,331.80083496)
\curveto(493.58532579,331.90083204)(493.60532577,332.00583194)(493.62533203,332.11583496)
\curveto(493.63532574,332.16583178)(493.63532574,332.21083173)(493.62533203,332.25083496)
\curveto(493.62532575,332.30083164)(493.63532574,332.35083159)(493.65533203,332.40083496)
\curveto(493.66532571,332.43083151)(493.6703257,332.46583148)(493.67033203,332.50583496)
\lineto(493.67033203,332.64083496)
\lineto(493.67033203,332.77583496)
}
}
{
\newrgbcolor{curcolor}{0 0 0}
\pscustom[linestyle=none,fillstyle=solid,fillcolor=curcolor]
{
\newpath
\moveto(504.65525391,332.92583496)
\curveto(504.67524574,332.8458311)(504.67524574,332.75583119)(504.65525391,332.65583496)
\curveto(504.63524578,332.55583139)(504.60024582,332.49083145)(504.55025391,332.46083496)
\curveto(504.50024592,332.42083152)(504.42524599,332.39083155)(504.32525391,332.37083496)
\curveto(504.23524618,332.36083158)(504.13024629,332.35083159)(504.01025391,332.34083496)
\lineto(503.66525391,332.34083496)
\curveto(503.55524686,332.35083159)(503.45524696,332.35583159)(503.36525391,332.35583496)
\lineto(499.70525391,332.35583496)
\lineto(499.49525391,332.35583496)
\curveto(499.43525098,332.35583159)(499.38025104,332.3458316)(499.33025391,332.32583496)
\curveto(499.25025117,332.28583166)(499.20025122,332.2458317)(499.18025391,332.20583496)
\curveto(499.16025126,332.18583176)(499.14025128,332.1458318)(499.12025391,332.08583496)
\curveto(499.10025132,332.03583191)(499.09525132,331.98583196)(499.10525391,331.93583496)
\curveto(499.12525129,331.87583207)(499.13525128,331.81583213)(499.13525391,331.75583496)
\curveto(499.14525127,331.70583224)(499.16025126,331.65083229)(499.18025391,331.59083496)
\curveto(499.26025116,331.35083259)(499.35525106,331.15083279)(499.46525391,330.99083496)
\curveto(499.58525083,330.8408331)(499.74525067,330.70583324)(499.94525391,330.58583496)
\curveto(500.02525039,330.53583341)(500.10525031,330.50083344)(500.18525391,330.48083496)
\curveto(500.27525014,330.47083347)(500.36525005,330.45083349)(500.45525391,330.42083496)
\curveto(500.53524988,330.40083354)(500.64524977,330.38583356)(500.78525391,330.37583496)
\curveto(500.92524949,330.36583358)(501.04524937,330.37083357)(501.14525391,330.39083496)
\lineto(501.28025391,330.39083496)
\curveto(501.38024904,330.41083353)(501.47024895,330.43083351)(501.55025391,330.45083496)
\curveto(501.64024878,330.48083346)(501.72524869,330.51083343)(501.80525391,330.54083496)
\curveto(501.90524851,330.59083335)(502.0152484,330.65583329)(502.13525391,330.73583496)
\curveto(502.26524815,330.81583313)(502.36024806,330.89583305)(502.42025391,330.97583496)
\curveto(502.47024795,331.0458329)(502.5202479,331.11083283)(502.57025391,331.17083496)
\curveto(502.63024779,331.2408327)(502.70024772,331.29083265)(502.78025391,331.32083496)
\curveto(502.88024754,331.37083257)(503.00524741,331.39083255)(503.15525391,331.38083496)
\lineto(503.59025391,331.38083496)
\lineto(503.77025391,331.38083496)
\curveto(503.84024658,331.39083255)(503.90024652,331.38583256)(503.95025391,331.36583496)
\lineto(504.10025391,331.36583496)
\curveto(504.20024622,331.3458326)(504.27024615,331.32083262)(504.31025391,331.29083496)
\curveto(504.35024607,331.27083267)(504.37024605,331.22583272)(504.37025391,331.15583496)
\curveto(504.38024604,331.08583286)(504.37524604,331.02583292)(504.35525391,330.97583496)
\curveto(504.30524611,330.83583311)(504.25024617,330.71083323)(504.19025391,330.60083496)
\curveto(504.13024629,330.49083345)(504.06024636,330.38083356)(503.98025391,330.27083496)
\curveto(503.76024666,329.940834)(503.51024691,329.67583427)(503.23025391,329.47583496)
\curveto(502.95024747,329.27583467)(502.60024782,329.10583484)(502.18025391,328.96583496)
\curveto(502.07024835,328.92583502)(501.96024846,328.90083504)(501.85025391,328.89083496)
\curveto(501.74024868,328.88083506)(501.62524879,328.86083508)(501.50525391,328.83083496)
\curveto(501.46524895,328.82083512)(501.420249,328.82083512)(501.37025391,328.83083496)
\curveto(501.33024909,328.83083511)(501.29024913,328.82583512)(501.25025391,328.81583496)
\lineto(501.08525391,328.81583496)
\curveto(501.03524938,328.79583515)(500.97524944,328.79083515)(500.90525391,328.80083496)
\curveto(500.84524957,328.80083514)(500.79024963,328.80583514)(500.74025391,328.81583496)
\curveto(500.66024976,328.82583512)(500.59024983,328.82583512)(500.53025391,328.81583496)
\curveto(500.47024995,328.80583514)(500.40525001,328.81083513)(500.33525391,328.83083496)
\curveto(500.28525013,328.85083509)(500.23025019,328.86083508)(500.17025391,328.86083496)
\curveto(500.11025031,328.86083508)(500.05525036,328.87083507)(500.00525391,328.89083496)
\curveto(499.89525052,328.91083503)(499.78525063,328.93583501)(499.67525391,328.96583496)
\curveto(499.56525085,328.98583496)(499.46525095,329.02083492)(499.37525391,329.07083496)
\curveto(499.26525115,329.11083483)(499.16025126,329.1458348)(499.06025391,329.17583496)
\curveto(498.97025145,329.21583473)(498.88525153,329.26083468)(498.80525391,329.31083496)
\curveto(498.48525193,329.51083443)(498.20025222,329.7408342)(497.95025391,330.00083496)
\curveto(497.70025272,330.27083367)(497.49525292,330.58083336)(497.33525391,330.93083496)
\curveto(497.28525313,331.0408329)(497.24525317,331.15083279)(497.21525391,331.26083496)
\curveto(497.18525323,331.38083256)(497.14525327,331.50083244)(497.09525391,331.62083496)
\curveto(497.08525333,331.66083228)(497.08025334,331.69583225)(497.08025391,331.72583496)
\curveto(497.08025334,331.76583218)(497.07525334,331.80583214)(497.06525391,331.84583496)
\curveto(497.02525339,331.96583198)(497.00025342,332.09583185)(496.99025391,332.23583496)
\lineto(496.96025391,332.65583496)
\curveto(496.96025346,332.70583124)(496.95525346,332.76083118)(496.94525391,332.82083496)
\curveto(496.94525347,332.88083106)(496.95025347,332.93583101)(496.96025391,332.98583496)
\lineto(496.96025391,333.16583496)
\lineto(497.00525391,333.52583496)
\curveto(497.04525337,333.69583025)(497.08025334,333.86083008)(497.11025391,334.02083496)
\curveto(497.14025328,334.18082976)(497.18525323,334.33082961)(497.24525391,334.47083496)
\curveto(497.67525274,335.51082843)(498.40525201,336.2458277)(499.43525391,336.67583496)
\curveto(499.57525084,336.73582721)(499.7152507,336.77582717)(499.85525391,336.79583496)
\curveto(500.00525041,336.82582712)(500.16025026,336.86082708)(500.32025391,336.90083496)
\curveto(500.40025002,336.91082703)(500.47524994,336.91582703)(500.54525391,336.91583496)
\curveto(500.6152498,336.91582703)(500.69024973,336.92082702)(500.77025391,336.93083496)
\curveto(501.28024914,336.940827)(501.7152487,336.88082706)(502.07525391,336.75083496)
\curveto(502.44524797,336.63082731)(502.77524764,336.47082747)(503.06525391,336.27083496)
\curveto(503.15524726,336.21082773)(503.24524717,336.1408278)(503.33525391,336.06083496)
\curveto(503.42524699,335.99082795)(503.50524691,335.91582803)(503.57525391,335.83583496)
\curveto(503.60524681,335.78582816)(503.64524677,335.7458282)(503.69525391,335.71583496)
\curveto(503.77524664,335.60582834)(503.85024657,335.49082845)(503.92025391,335.37083496)
\curveto(503.99024643,335.26082868)(504.06524635,335.1458288)(504.14525391,335.02583496)
\curveto(504.19524622,334.93582901)(504.23524618,334.8408291)(504.26525391,334.74083496)
\curveto(504.30524611,334.65082929)(504.34524607,334.55082939)(504.38525391,334.44083496)
\curveto(504.43524598,334.31082963)(504.47524594,334.17582977)(504.50525391,334.03583496)
\curveto(504.53524588,333.89583005)(504.57024585,333.75583019)(504.61025391,333.61583496)
\curveto(504.63024579,333.53583041)(504.63524578,333.4458305)(504.62525391,333.34583496)
\curveto(504.62524579,333.25583069)(504.63524578,333.17083077)(504.65525391,333.09083496)
\lineto(504.65525391,332.92583496)
\moveto(502.40525391,333.81083496)
\curveto(502.47524794,333.91083003)(502.48024794,334.03082991)(502.42025391,334.17083496)
\curveto(502.37024805,334.32082962)(502.33024809,334.43082951)(502.30025391,334.50083496)
\curveto(502.16024826,334.77082917)(501.97524844,334.97582897)(501.74525391,335.11583496)
\curveto(501.5152489,335.26582868)(501.19524922,335.3458286)(500.78525391,335.35583496)
\curveto(500.75524966,335.33582861)(500.7202497,335.33082861)(500.68025391,335.34083496)
\curveto(500.64024978,335.35082859)(500.60524981,335.35082859)(500.57525391,335.34083496)
\curveto(500.52524989,335.32082862)(500.47024995,335.30582864)(500.41025391,335.29583496)
\curveto(500.35025007,335.29582865)(500.29525012,335.28582866)(500.24525391,335.26583496)
\curveto(499.80525061,335.12582882)(499.48025094,334.85082909)(499.27025391,334.44083496)
\curveto(499.25025117,334.40082954)(499.22525119,334.3458296)(499.19525391,334.27583496)
\curveto(499.17525124,334.21582973)(499.16025126,334.15082979)(499.15025391,334.08083496)
\curveto(499.14025128,334.02082992)(499.14025128,333.96082998)(499.15025391,333.90083496)
\curveto(499.17025125,333.8408301)(499.20525121,333.79083015)(499.25525391,333.75083496)
\curveto(499.33525108,333.70083024)(499.44525097,333.67583027)(499.58525391,333.67583496)
\lineto(499.99025391,333.67583496)
\lineto(501.65525391,333.67583496)
\lineto(502.09025391,333.67583496)
\curveto(502.25024817,333.68583026)(502.35524806,333.73083021)(502.40525391,333.81083496)
}
}
{
\newrgbcolor{curcolor}{0 0 0}
\pscustom[linestyle=none,fillstyle=solid,fillcolor=curcolor]
{
}
}
{
\newrgbcolor{curcolor}{0 0 0}
\pscustom[linestyle=none,fillstyle=solid,fillcolor=curcolor]
{
\newpath
\moveto(517.95369141,332.94083496)
\curveto(517.96368273,332.88083106)(517.96868272,332.79083115)(517.96869141,332.67083496)
\curveto(517.96868272,332.55083139)(517.95868273,332.46583148)(517.93869141,332.41583496)
\lineto(517.93869141,332.22083496)
\curveto(517.90868278,332.11083183)(517.8886828,332.00583194)(517.87869141,331.90583496)
\curveto(517.87868281,331.80583214)(517.86368283,331.70583224)(517.83369141,331.60583496)
\curveto(517.81368288,331.51583243)(517.7936829,331.42083252)(517.77369141,331.32083496)
\curveto(517.75368294,331.23083271)(517.72368297,331.1408328)(517.68369141,331.05083496)
\curveto(517.61368308,330.88083306)(517.54368315,330.72083322)(517.47369141,330.57083496)
\curveto(517.40368329,330.43083351)(517.32368337,330.29083365)(517.23369141,330.15083496)
\curveto(517.17368352,330.06083388)(517.10868358,329.97583397)(517.03869141,329.89583496)
\curveto(516.97868371,329.82583412)(516.90868378,329.75083419)(516.82869141,329.67083496)
\lineto(516.72369141,329.56583496)
\curveto(516.67368402,329.51583443)(516.61868407,329.47083447)(516.55869141,329.43083496)
\lineto(516.40869141,329.31083496)
\curveto(516.32868436,329.25083469)(516.23868445,329.19583475)(516.13869141,329.14583496)
\curveto(516.04868464,329.10583484)(515.95368474,329.06083488)(515.85369141,329.01083496)
\curveto(515.75368494,328.96083498)(515.64868504,328.92583502)(515.53869141,328.90583496)
\curveto(515.43868525,328.88583506)(515.33368536,328.86583508)(515.22369141,328.84583496)
\curveto(515.16368553,328.82583512)(515.09868559,328.81583513)(515.02869141,328.81583496)
\curveto(514.96868572,328.81583513)(514.90368579,328.80583514)(514.83369141,328.78583496)
\lineto(514.69869141,328.78583496)
\curveto(514.61868607,328.76583518)(514.54368615,328.76583518)(514.47369141,328.78583496)
\lineto(514.32369141,328.78583496)
\curveto(514.26368643,328.80583514)(514.19868649,328.81583513)(514.12869141,328.81583496)
\curveto(514.06868662,328.80583514)(514.00868668,328.81083513)(513.94869141,328.83083496)
\curveto(513.7886869,328.88083506)(513.63368706,328.92583502)(513.48369141,328.96583496)
\curveto(513.34368735,329.00583494)(513.21368748,329.06583488)(513.09369141,329.14583496)
\curveto(513.02368767,329.18583476)(512.95868773,329.22583472)(512.89869141,329.26583496)
\curveto(512.83868785,329.31583463)(512.77368792,329.36583458)(512.70369141,329.41583496)
\lineto(512.52369141,329.55083496)
\curveto(512.44368825,329.61083433)(512.37368832,329.61583433)(512.31369141,329.56583496)
\curveto(512.26368843,329.53583441)(512.23868845,329.49583445)(512.23869141,329.44583496)
\curveto(512.23868845,329.40583454)(512.22868846,329.35583459)(512.20869141,329.29583496)
\curveto(512.1886885,329.19583475)(512.17868851,329.08083486)(512.17869141,328.95083496)
\curveto(512.1886885,328.82083512)(512.1936885,328.70083524)(512.19369141,328.59083496)
\lineto(512.19369141,327.06083496)
\curveto(512.1936885,326.93083701)(512.1886885,326.80583714)(512.17869141,326.68583496)
\curveto(512.17868851,326.55583739)(512.15368854,326.45083749)(512.10369141,326.37083496)
\curveto(512.07368862,326.33083761)(512.01868867,326.30083764)(511.93869141,326.28083496)
\curveto(511.85868883,326.26083768)(511.76868892,326.25083769)(511.66869141,326.25083496)
\curveto(511.56868912,326.2408377)(511.46868922,326.2408377)(511.36869141,326.25083496)
\lineto(511.11369141,326.25083496)
\lineto(510.70869141,326.25083496)
\lineto(510.60369141,326.25083496)
\curveto(510.56369013,326.25083769)(510.52869016,326.25583769)(510.49869141,326.26583496)
\lineto(510.37869141,326.26583496)
\curveto(510.20869048,326.31583763)(510.11869057,326.41583753)(510.10869141,326.56583496)
\curveto(510.09869059,326.70583724)(510.0936906,326.87583707)(510.09369141,327.07583496)
\lineto(510.09369141,335.88083496)
\curveto(510.0936906,335.99082795)(510.0886906,336.10582784)(510.07869141,336.22583496)
\curveto(510.07869061,336.35582759)(510.10369059,336.45582749)(510.15369141,336.52583496)
\curveto(510.1936905,336.59582735)(510.24869044,336.6408273)(510.31869141,336.66083496)
\curveto(510.36869032,336.68082726)(510.42869026,336.69082725)(510.49869141,336.69083496)
\lineto(510.72369141,336.69083496)
\lineto(511.44369141,336.69083496)
\lineto(511.72869141,336.69083496)
\curveto(511.81868887,336.69082725)(511.8936888,336.66582728)(511.95369141,336.61583496)
\curveto(512.02368867,336.56582738)(512.05868863,336.50082744)(512.05869141,336.42083496)
\curveto(512.06868862,336.35082759)(512.0936886,336.27582767)(512.13369141,336.19583496)
\curveto(512.14368855,336.16582778)(512.15368854,336.1408278)(512.16369141,336.12083496)
\curveto(512.18368851,336.11082783)(512.20368849,336.09582785)(512.22369141,336.07583496)
\curveto(512.33368836,336.06582788)(512.42368827,336.09582785)(512.49369141,336.16583496)
\curveto(512.56368813,336.23582771)(512.63368806,336.29582765)(512.70369141,336.34583496)
\curveto(512.83368786,336.43582751)(512.96868772,336.51582743)(513.10869141,336.58583496)
\curveto(513.24868744,336.66582728)(513.40368729,336.73082721)(513.57369141,336.78083496)
\curveto(513.65368704,336.81082713)(513.73868695,336.83082711)(513.82869141,336.84083496)
\curveto(513.92868676,336.85082709)(514.02368667,336.86582708)(514.11369141,336.88583496)
\curveto(514.15368654,336.89582705)(514.1936865,336.89582705)(514.23369141,336.88583496)
\curveto(514.28368641,336.87582707)(514.32368637,336.88082706)(514.35369141,336.90083496)
\curveto(514.92368577,336.92082702)(515.40368529,336.8408271)(515.79369141,336.66083496)
\curveto(516.1936845,336.49082745)(516.53368416,336.26582768)(516.81369141,335.98583496)
\curveto(516.86368383,335.93582801)(516.90868378,335.88582806)(516.94869141,335.83583496)
\curveto(516.9886837,335.79582815)(517.02868366,335.75082819)(517.06869141,335.70083496)
\curveto(517.13868355,335.61082833)(517.19868349,335.52082842)(517.24869141,335.43083496)
\curveto(517.30868338,335.3408286)(517.36368333,335.25082869)(517.41369141,335.16083496)
\curveto(517.43368326,335.1408288)(517.44368325,335.11582883)(517.44369141,335.08583496)
\curveto(517.45368324,335.05582889)(517.46868322,335.02082892)(517.48869141,334.98083496)
\curveto(517.54868314,334.88082906)(517.60368309,334.76082918)(517.65369141,334.62083496)
\curveto(517.67368302,334.56082938)(517.693683,334.49582945)(517.71369141,334.42583496)
\curveto(517.73368296,334.36582958)(517.75368294,334.30082964)(517.77369141,334.23083496)
\curveto(517.81368288,334.11082983)(517.83868285,333.98582996)(517.84869141,333.85583496)
\curveto(517.86868282,333.72583022)(517.8936828,333.59083035)(517.92369141,333.45083496)
\lineto(517.92369141,333.28583496)
\lineto(517.95369141,333.10583496)
\lineto(517.95369141,332.94083496)
\moveto(515.83869141,332.59583496)
\curveto(515.84868484,332.6458313)(515.85368484,332.71083123)(515.85369141,332.79083496)
\curveto(515.85368484,332.88083106)(515.84868484,332.95083099)(515.83869141,333.00083496)
\lineto(515.83869141,333.13583496)
\curveto(515.81868487,333.19583075)(515.80868488,333.26083068)(515.80869141,333.33083496)
\curveto(515.80868488,333.40083054)(515.79868489,333.47083047)(515.77869141,333.54083496)
\curveto(515.75868493,333.6408303)(515.73868495,333.73583021)(515.71869141,333.82583496)
\curveto(515.69868499,333.92583002)(515.66868502,334.01582993)(515.62869141,334.09583496)
\curveto(515.50868518,334.41582953)(515.35368534,334.67082927)(515.16369141,334.86083496)
\curveto(514.97368572,335.05082889)(514.70368599,335.19082875)(514.35369141,335.28083496)
\curveto(514.27368642,335.30082864)(514.18368651,335.31082863)(514.08369141,335.31083496)
\lineto(513.81369141,335.31083496)
\curveto(513.77368692,335.30082864)(513.73868695,335.29582865)(513.70869141,335.29583496)
\curveto(513.67868701,335.29582865)(513.64368705,335.29082865)(513.60369141,335.28083496)
\lineto(513.39369141,335.22083496)
\curveto(513.33368736,335.21082873)(513.27368742,335.19082875)(513.21369141,335.16083496)
\curveto(512.95368774,335.05082889)(512.74868794,334.88082906)(512.59869141,334.65083496)
\curveto(512.45868823,334.42082952)(512.34368835,334.16582978)(512.25369141,333.88583496)
\curveto(512.23368846,333.80583014)(512.21868847,333.72083022)(512.20869141,333.63083496)
\curveto(512.19868849,333.55083039)(512.18368851,333.47083047)(512.16369141,333.39083496)
\curveto(512.15368854,333.35083059)(512.14868854,333.28583066)(512.14869141,333.19583496)
\curveto(512.12868856,333.15583079)(512.12368857,333.10583084)(512.13369141,333.04583496)
\curveto(512.14368855,332.99583095)(512.14368855,332.945831)(512.13369141,332.89583496)
\curveto(512.11368858,332.83583111)(512.11368858,332.78083116)(512.13369141,332.73083496)
\lineto(512.13369141,332.55083496)
\lineto(512.13369141,332.41583496)
\curveto(512.13368856,332.37583157)(512.14368855,332.33583161)(512.16369141,332.29583496)
\curveto(512.16368853,332.22583172)(512.16868852,332.17083177)(512.17869141,332.13083496)
\lineto(512.20869141,331.95083496)
\curveto(512.21868847,331.89083205)(512.23368846,331.83083211)(512.25369141,331.77083496)
\curveto(512.34368835,331.48083246)(512.44868824,331.2408327)(512.56869141,331.05083496)
\curveto(512.69868799,330.87083307)(512.87868781,330.71083323)(513.10869141,330.57083496)
\curveto(513.24868744,330.49083345)(513.41368728,330.42583352)(513.60369141,330.37583496)
\curveto(513.64368705,330.36583358)(513.67868701,330.36083358)(513.70869141,330.36083496)
\curveto(513.73868695,330.37083357)(513.77368692,330.37083357)(513.81369141,330.36083496)
\curveto(513.85368684,330.35083359)(513.91368678,330.3408336)(513.99369141,330.33083496)
\curveto(514.07368662,330.33083361)(514.13868655,330.33583361)(514.18869141,330.34583496)
\curveto(514.26868642,330.36583358)(514.34868634,330.38083356)(514.42869141,330.39083496)
\curveto(514.51868617,330.41083353)(514.60368609,330.43583351)(514.68369141,330.46583496)
\curveto(514.92368577,330.56583338)(515.11868557,330.70583324)(515.26869141,330.88583496)
\curveto(515.41868527,331.06583288)(515.54368515,331.27583267)(515.64369141,331.51583496)
\curveto(515.693685,331.63583231)(515.72868496,331.76083218)(515.74869141,331.89083496)
\curveto(515.76868492,332.02083192)(515.7936849,332.15583179)(515.82369141,332.29583496)
\lineto(515.82369141,332.44583496)
\curveto(515.83368486,332.49583145)(515.83868485,332.5458314)(515.83869141,332.59583496)
}
}
{
\newrgbcolor{curcolor}{0 0 0}
\pscustom[linestyle=none,fillstyle=solid,fillcolor=curcolor]
{
\newpath
\moveto(527.00361328,333.16583496)
\curveto(527.02360471,333.10583084)(527.0336047,333.02083092)(527.03361328,332.91083496)
\curveto(527.0336047,332.80083114)(527.02360471,332.71583123)(527.00361328,332.65583496)
\lineto(527.00361328,332.50583496)
\curveto(526.98360475,332.42583152)(526.97360476,332.3458316)(526.97361328,332.26583496)
\curveto(526.98360475,332.18583176)(526.97860476,332.10583184)(526.95861328,332.02583496)
\curveto(526.9386048,331.95583199)(526.92360481,331.89083205)(526.91361328,331.83083496)
\curveto(526.90360483,331.77083217)(526.89360484,331.70583224)(526.88361328,331.63583496)
\curveto(526.84360489,331.52583242)(526.80860493,331.41083253)(526.77861328,331.29083496)
\curveto(526.74860499,331.18083276)(526.70860503,331.07583287)(526.65861328,330.97583496)
\curveto(526.44860529,330.49583345)(526.17360556,330.10583384)(525.83361328,329.80583496)
\curveto(525.49360624,329.50583444)(525.08360665,329.25583469)(524.60361328,329.05583496)
\curveto(524.48360725,329.00583494)(524.35860738,328.97083497)(524.22861328,328.95083496)
\curveto(524.10860763,328.92083502)(523.98360775,328.89083505)(523.85361328,328.86083496)
\curveto(523.80360793,328.8408351)(523.74860799,328.83083511)(523.68861328,328.83083496)
\curveto(523.62860811,328.83083511)(523.57360816,328.82583512)(523.52361328,328.81583496)
\lineto(523.41861328,328.81583496)
\curveto(523.38860835,328.80583514)(523.35860838,328.80083514)(523.32861328,328.80083496)
\curveto(523.27860846,328.79083515)(523.19860854,328.78583516)(523.08861328,328.78583496)
\curveto(522.97860876,328.77583517)(522.89360884,328.78083516)(522.83361328,328.80083496)
\lineto(522.68361328,328.80083496)
\curveto(522.6336091,328.81083513)(522.57860916,328.81583513)(522.51861328,328.81583496)
\curveto(522.46860927,328.80583514)(522.41860932,328.81083513)(522.36861328,328.83083496)
\curveto(522.32860941,328.8408351)(522.28860945,328.8458351)(522.24861328,328.84583496)
\curveto(522.21860952,328.8458351)(522.17860956,328.85083509)(522.12861328,328.86083496)
\curveto(522.02860971,328.89083505)(521.92860981,328.91583503)(521.82861328,328.93583496)
\curveto(521.72861001,328.95583499)(521.6336101,328.98583496)(521.54361328,329.02583496)
\curveto(521.42361031,329.06583488)(521.30861043,329.10583484)(521.19861328,329.14583496)
\curveto(521.09861064,329.18583476)(520.99361074,329.23583471)(520.88361328,329.29583496)
\curveto(520.5336112,329.50583444)(520.2336115,329.75083419)(519.98361328,330.03083496)
\curveto(519.733612,330.31083363)(519.52361221,330.6458333)(519.35361328,331.03583496)
\curveto(519.30361243,331.12583282)(519.26361247,331.22083272)(519.23361328,331.32083496)
\curveto(519.21361252,331.42083252)(519.18861255,331.52583242)(519.15861328,331.63583496)
\curveto(519.1386126,331.68583226)(519.12861261,331.73083221)(519.12861328,331.77083496)
\curveto(519.12861261,331.81083213)(519.11861262,331.85583209)(519.09861328,331.90583496)
\curveto(519.07861266,331.98583196)(519.06861267,332.06583188)(519.06861328,332.14583496)
\curveto(519.06861267,332.23583171)(519.05861268,332.32083162)(519.03861328,332.40083496)
\curveto(519.02861271,332.45083149)(519.02361271,332.49583145)(519.02361328,332.53583496)
\lineto(519.02361328,332.67083496)
\curveto(519.00361273,332.73083121)(518.99361274,332.81583113)(518.99361328,332.92583496)
\curveto(519.00361273,333.03583091)(519.01861272,333.12083082)(519.03861328,333.18083496)
\lineto(519.03861328,333.28583496)
\curveto(519.04861269,333.33583061)(519.04861269,333.38583056)(519.03861328,333.43583496)
\curveto(519.0386127,333.49583045)(519.04861269,333.55083039)(519.06861328,333.60083496)
\curveto(519.07861266,333.65083029)(519.08361265,333.69583025)(519.08361328,333.73583496)
\curveto(519.08361265,333.78583016)(519.09361264,333.83583011)(519.11361328,333.88583496)
\curveto(519.15361258,334.01582993)(519.18861255,334.1408298)(519.21861328,334.26083496)
\curveto(519.24861249,334.39082955)(519.28861245,334.51582943)(519.33861328,334.63583496)
\curveto(519.51861222,335.0458289)(519.733612,335.38582856)(519.98361328,335.65583496)
\curveto(520.2336115,335.93582801)(520.5386112,336.19082775)(520.89861328,336.42083496)
\curveto(520.99861074,336.47082747)(521.10361063,336.51582743)(521.21361328,336.55583496)
\curveto(521.32361041,336.59582735)(521.4336103,336.6408273)(521.54361328,336.69083496)
\curveto(521.67361006,336.7408272)(521.80860993,336.77582717)(521.94861328,336.79583496)
\curveto(522.08860965,336.81582713)(522.2336095,336.8458271)(522.38361328,336.88583496)
\curveto(522.46360927,336.89582705)(522.5386092,336.90082704)(522.60861328,336.90083496)
\curveto(522.67860906,336.90082704)(522.74860899,336.90582704)(522.81861328,336.91583496)
\curveto(523.39860834,336.92582702)(523.89860784,336.86582708)(524.31861328,336.73583496)
\curveto(524.74860699,336.60582734)(525.12860661,336.42582752)(525.45861328,336.19583496)
\curveto(525.56860617,336.11582783)(525.67860606,336.02582792)(525.78861328,335.92583496)
\curveto(525.90860583,335.83582811)(526.00860573,335.73582821)(526.08861328,335.62583496)
\curveto(526.16860557,335.52582842)(526.2386055,335.42582852)(526.29861328,335.32583496)
\curveto(526.36860537,335.22582872)(526.4386053,335.12082882)(526.50861328,335.01083496)
\curveto(526.57860516,334.90082904)(526.6336051,334.78082916)(526.67361328,334.65083496)
\curveto(526.71360502,334.53082941)(526.75860498,334.40082954)(526.80861328,334.26083496)
\curveto(526.8386049,334.18082976)(526.86360487,334.09582985)(526.88361328,334.00583496)
\lineto(526.94361328,333.73583496)
\curveto(526.95360478,333.69583025)(526.95860478,333.65583029)(526.95861328,333.61583496)
\curveto(526.95860478,333.57583037)(526.96360477,333.53583041)(526.97361328,333.49583496)
\curveto(526.99360474,333.4458305)(526.99860474,333.39083055)(526.98861328,333.33083496)
\curveto(526.97860476,333.27083067)(526.98360475,333.21583073)(527.00361328,333.16583496)
\moveto(524.90361328,332.62583496)
\curveto(524.91360682,332.67583127)(524.91860682,332.7458312)(524.91861328,332.83583496)
\curveto(524.91860682,332.93583101)(524.91360682,333.01083093)(524.90361328,333.06083496)
\lineto(524.90361328,333.18083496)
\curveto(524.88360685,333.23083071)(524.87360686,333.28583066)(524.87361328,333.34583496)
\curveto(524.87360686,333.40583054)(524.86860687,333.46083048)(524.85861328,333.51083496)
\curveto(524.85860688,333.55083039)(524.85360688,333.58083036)(524.84361328,333.60083496)
\lineto(524.78361328,333.84083496)
\curveto(524.77360696,333.93083001)(524.75360698,334.01582993)(524.72361328,334.09583496)
\curveto(524.61360712,334.35582959)(524.48360725,334.57582937)(524.33361328,334.75583496)
\curveto(524.18360755,334.945829)(523.98360775,335.09582885)(523.73361328,335.20583496)
\curveto(523.67360806,335.22582872)(523.61360812,335.2408287)(523.55361328,335.25083496)
\curveto(523.49360824,335.27082867)(523.42860831,335.29082865)(523.35861328,335.31083496)
\curveto(523.27860846,335.33082861)(523.19360854,335.33582861)(523.10361328,335.32583496)
\lineto(522.83361328,335.32583496)
\curveto(522.80360893,335.30582864)(522.76860897,335.29582865)(522.72861328,335.29583496)
\curveto(522.68860905,335.30582864)(522.65360908,335.30582864)(522.62361328,335.29583496)
\lineto(522.41361328,335.23583496)
\curveto(522.35360938,335.22582872)(522.29860944,335.20582874)(522.24861328,335.17583496)
\curveto(521.99860974,335.06582888)(521.79360994,334.90582904)(521.63361328,334.69583496)
\curveto(521.48361025,334.49582945)(521.36361037,334.26082968)(521.27361328,333.99083496)
\curveto(521.24361049,333.89083005)(521.21861052,333.78583016)(521.19861328,333.67583496)
\curveto(521.18861055,333.56583038)(521.17361056,333.45583049)(521.15361328,333.34583496)
\curveto(521.14361059,333.29583065)(521.1386106,333.2458307)(521.13861328,333.19583496)
\lineto(521.13861328,333.04583496)
\curveto(521.11861062,332.97583097)(521.10861063,332.87083107)(521.10861328,332.73083496)
\curveto(521.11861062,332.59083135)(521.1336106,332.48583146)(521.15361328,332.41583496)
\lineto(521.15361328,332.28083496)
\curveto(521.17361056,332.20083174)(521.18861055,332.12083182)(521.19861328,332.04083496)
\curveto(521.20861053,331.97083197)(521.22361051,331.89583205)(521.24361328,331.81583496)
\curveto(521.34361039,331.51583243)(521.44861029,331.27083267)(521.55861328,331.08083496)
\curveto(521.67861006,330.90083304)(521.86360987,330.73583321)(522.11361328,330.58583496)
\curveto(522.18360955,330.53583341)(522.25860948,330.49583345)(522.33861328,330.46583496)
\curveto(522.42860931,330.43583351)(522.51860922,330.41083353)(522.60861328,330.39083496)
\curveto(522.64860909,330.38083356)(522.68360905,330.37583357)(522.71361328,330.37583496)
\curveto(522.74360899,330.38583356)(522.77860896,330.38583356)(522.81861328,330.37583496)
\lineto(522.93861328,330.34583496)
\curveto(522.98860875,330.3458336)(523.0336087,330.35083359)(523.07361328,330.36083496)
\lineto(523.19361328,330.36083496)
\curveto(523.27360846,330.38083356)(523.35360838,330.39583355)(523.43361328,330.40583496)
\curveto(523.51360822,330.41583353)(523.58860815,330.43583351)(523.65861328,330.46583496)
\curveto(523.91860782,330.56583338)(524.12860761,330.70083324)(524.28861328,330.87083496)
\curveto(524.44860729,331.0408329)(524.58360715,331.25083269)(524.69361328,331.50083496)
\curveto(524.733607,331.60083234)(524.76360697,331.70083224)(524.78361328,331.80083496)
\curveto(524.80360693,331.90083204)(524.82860691,332.00583194)(524.85861328,332.11583496)
\curveto(524.86860687,332.15583179)(524.87360686,332.19083175)(524.87361328,332.22083496)
\curveto(524.87360686,332.26083168)(524.87860686,332.30083164)(524.88861328,332.34083496)
\lineto(524.88861328,332.47583496)
\curveto(524.88860685,332.52583142)(524.89360684,332.57583137)(524.90361328,332.62583496)
}
}
{
\newrgbcolor{curcolor}{0 0 0}
\pscustom[linestyle=none,fillstyle=solid,fillcolor=curcolor]
{
\newpath
\moveto(536.29353516,332.94083496)
\curveto(536.30352648,332.88083106)(536.30852647,332.79083115)(536.30853516,332.67083496)
\curveto(536.30852647,332.55083139)(536.29852648,332.46583148)(536.27853516,332.41583496)
\lineto(536.27853516,332.22083496)
\curveto(536.24852653,332.11083183)(536.22852655,332.00583194)(536.21853516,331.90583496)
\curveto(536.21852656,331.80583214)(536.20352658,331.70583224)(536.17353516,331.60583496)
\curveto(536.15352663,331.51583243)(536.13352665,331.42083252)(536.11353516,331.32083496)
\curveto(536.09352669,331.23083271)(536.06352672,331.1408328)(536.02353516,331.05083496)
\curveto(535.95352683,330.88083306)(535.8835269,330.72083322)(535.81353516,330.57083496)
\curveto(535.74352704,330.43083351)(535.66352712,330.29083365)(535.57353516,330.15083496)
\curveto(535.51352727,330.06083388)(535.44852733,329.97583397)(535.37853516,329.89583496)
\curveto(535.31852746,329.82583412)(535.24852753,329.75083419)(535.16853516,329.67083496)
\lineto(535.06353516,329.56583496)
\curveto(535.01352777,329.51583443)(534.95852782,329.47083447)(534.89853516,329.43083496)
\lineto(534.74853516,329.31083496)
\curveto(534.66852811,329.25083469)(534.5785282,329.19583475)(534.47853516,329.14583496)
\curveto(534.38852839,329.10583484)(534.29352849,329.06083488)(534.19353516,329.01083496)
\curveto(534.09352869,328.96083498)(533.98852879,328.92583502)(533.87853516,328.90583496)
\curveto(533.778529,328.88583506)(533.67352911,328.86583508)(533.56353516,328.84583496)
\curveto(533.50352928,328.82583512)(533.43852934,328.81583513)(533.36853516,328.81583496)
\curveto(533.30852947,328.81583513)(533.24352954,328.80583514)(533.17353516,328.78583496)
\lineto(533.03853516,328.78583496)
\curveto(532.95852982,328.76583518)(532.8835299,328.76583518)(532.81353516,328.78583496)
\lineto(532.66353516,328.78583496)
\curveto(532.60353018,328.80583514)(532.53853024,328.81583513)(532.46853516,328.81583496)
\curveto(532.40853037,328.80583514)(532.34853043,328.81083513)(532.28853516,328.83083496)
\curveto(532.12853065,328.88083506)(531.97353081,328.92583502)(531.82353516,328.96583496)
\curveto(531.6835311,329.00583494)(531.55353123,329.06583488)(531.43353516,329.14583496)
\curveto(531.36353142,329.18583476)(531.29853148,329.22583472)(531.23853516,329.26583496)
\curveto(531.1785316,329.31583463)(531.11353167,329.36583458)(531.04353516,329.41583496)
\lineto(530.86353516,329.55083496)
\curveto(530.783532,329.61083433)(530.71353207,329.61583433)(530.65353516,329.56583496)
\curveto(530.60353218,329.53583441)(530.5785322,329.49583445)(530.57853516,329.44583496)
\curveto(530.5785322,329.40583454)(530.56853221,329.35583459)(530.54853516,329.29583496)
\curveto(530.52853225,329.19583475)(530.51853226,329.08083486)(530.51853516,328.95083496)
\curveto(530.52853225,328.82083512)(530.53353225,328.70083524)(530.53353516,328.59083496)
\lineto(530.53353516,327.06083496)
\curveto(530.53353225,326.93083701)(530.52853225,326.80583714)(530.51853516,326.68583496)
\curveto(530.51853226,326.55583739)(530.49353229,326.45083749)(530.44353516,326.37083496)
\curveto(530.41353237,326.33083761)(530.35853242,326.30083764)(530.27853516,326.28083496)
\curveto(530.19853258,326.26083768)(530.10853267,326.25083769)(530.00853516,326.25083496)
\curveto(529.90853287,326.2408377)(529.80853297,326.2408377)(529.70853516,326.25083496)
\lineto(529.45353516,326.25083496)
\lineto(529.04853516,326.25083496)
\lineto(528.94353516,326.25083496)
\curveto(528.90353388,326.25083769)(528.86853391,326.25583769)(528.83853516,326.26583496)
\lineto(528.71853516,326.26583496)
\curveto(528.54853423,326.31583763)(528.45853432,326.41583753)(528.44853516,326.56583496)
\curveto(528.43853434,326.70583724)(528.43353435,326.87583707)(528.43353516,327.07583496)
\lineto(528.43353516,335.88083496)
\curveto(528.43353435,335.99082795)(528.42853435,336.10582784)(528.41853516,336.22583496)
\curveto(528.41853436,336.35582759)(528.44353434,336.45582749)(528.49353516,336.52583496)
\curveto(528.53353425,336.59582735)(528.58853419,336.6408273)(528.65853516,336.66083496)
\curveto(528.70853407,336.68082726)(528.76853401,336.69082725)(528.83853516,336.69083496)
\lineto(529.06353516,336.69083496)
\lineto(529.78353516,336.69083496)
\lineto(530.06853516,336.69083496)
\curveto(530.15853262,336.69082725)(530.23353255,336.66582728)(530.29353516,336.61583496)
\curveto(530.36353242,336.56582738)(530.39853238,336.50082744)(530.39853516,336.42083496)
\curveto(530.40853237,336.35082759)(530.43353235,336.27582767)(530.47353516,336.19583496)
\curveto(530.4835323,336.16582778)(530.49353229,336.1408278)(530.50353516,336.12083496)
\curveto(530.52353226,336.11082783)(530.54353224,336.09582785)(530.56353516,336.07583496)
\curveto(530.67353211,336.06582788)(530.76353202,336.09582785)(530.83353516,336.16583496)
\curveto(530.90353188,336.23582771)(530.97353181,336.29582765)(531.04353516,336.34583496)
\curveto(531.17353161,336.43582751)(531.30853147,336.51582743)(531.44853516,336.58583496)
\curveto(531.58853119,336.66582728)(531.74353104,336.73082721)(531.91353516,336.78083496)
\curveto(531.99353079,336.81082713)(532.0785307,336.83082711)(532.16853516,336.84083496)
\curveto(532.26853051,336.85082709)(532.36353042,336.86582708)(532.45353516,336.88583496)
\curveto(532.49353029,336.89582705)(532.53353025,336.89582705)(532.57353516,336.88583496)
\curveto(532.62353016,336.87582707)(532.66353012,336.88082706)(532.69353516,336.90083496)
\curveto(533.26352952,336.92082702)(533.74352904,336.8408271)(534.13353516,336.66083496)
\curveto(534.53352825,336.49082745)(534.87352791,336.26582768)(535.15353516,335.98583496)
\curveto(535.20352758,335.93582801)(535.24852753,335.88582806)(535.28853516,335.83583496)
\curveto(535.32852745,335.79582815)(535.36852741,335.75082819)(535.40853516,335.70083496)
\curveto(535.4785273,335.61082833)(535.53852724,335.52082842)(535.58853516,335.43083496)
\curveto(535.64852713,335.3408286)(535.70352708,335.25082869)(535.75353516,335.16083496)
\curveto(535.77352701,335.1408288)(535.783527,335.11582883)(535.78353516,335.08583496)
\curveto(535.79352699,335.05582889)(535.80852697,335.02082892)(535.82853516,334.98083496)
\curveto(535.88852689,334.88082906)(535.94352684,334.76082918)(535.99353516,334.62083496)
\curveto(536.01352677,334.56082938)(536.03352675,334.49582945)(536.05353516,334.42583496)
\curveto(536.07352671,334.36582958)(536.09352669,334.30082964)(536.11353516,334.23083496)
\curveto(536.15352663,334.11082983)(536.1785266,333.98582996)(536.18853516,333.85583496)
\curveto(536.20852657,333.72583022)(536.23352655,333.59083035)(536.26353516,333.45083496)
\lineto(536.26353516,333.28583496)
\lineto(536.29353516,333.10583496)
\lineto(536.29353516,332.94083496)
\moveto(534.17853516,332.59583496)
\curveto(534.18852859,332.6458313)(534.19352859,332.71083123)(534.19353516,332.79083496)
\curveto(534.19352859,332.88083106)(534.18852859,332.95083099)(534.17853516,333.00083496)
\lineto(534.17853516,333.13583496)
\curveto(534.15852862,333.19583075)(534.14852863,333.26083068)(534.14853516,333.33083496)
\curveto(534.14852863,333.40083054)(534.13852864,333.47083047)(534.11853516,333.54083496)
\curveto(534.09852868,333.6408303)(534.0785287,333.73583021)(534.05853516,333.82583496)
\curveto(534.03852874,333.92583002)(534.00852877,334.01582993)(533.96853516,334.09583496)
\curveto(533.84852893,334.41582953)(533.69352909,334.67082927)(533.50353516,334.86083496)
\curveto(533.31352947,335.05082889)(533.04352974,335.19082875)(532.69353516,335.28083496)
\curveto(532.61353017,335.30082864)(532.52353026,335.31082863)(532.42353516,335.31083496)
\lineto(532.15353516,335.31083496)
\curveto(532.11353067,335.30082864)(532.0785307,335.29582865)(532.04853516,335.29583496)
\curveto(532.01853076,335.29582865)(531.9835308,335.29082865)(531.94353516,335.28083496)
\lineto(531.73353516,335.22083496)
\curveto(531.67353111,335.21082873)(531.61353117,335.19082875)(531.55353516,335.16083496)
\curveto(531.29353149,335.05082889)(531.08853169,334.88082906)(530.93853516,334.65083496)
\curveto(530.79853198,334.42082952)(530.6835321,334.16582978)(530.59353516,333.88583496)
\curveto(530.57353221,333.80583014)(530.55853222,333.72083022)(530.54853516,333.63083496)
\curveto(530.53853224,333.55083039)(530.52353226,333.47083047)(530.50353516,333.39083496)
\curveto(530.49353229,333.35083059)(530.48853229,333.28583066)(530.48853516,333.19583496)
\curveto(530.46853231,333.15583079)(530.46353232,333.10583084)(530.47353516,333.04583496)
\curveto(530.4835323,332.99583095)(530.4835323,332.945831)(530.47353516,332.89583496)
\curveto(530.45353233,332.83583111)(530.45353233,332.78083116)(530.47353516,332.73083496)
\lineto(530.47353516,332.55083496)
\lineto(530.47353516,332.41583496)
\curveto(530.47353231,332.37583157)(530.4835323,332.33583161)(530.50353516,332.29583496)
\curveto(530.50353228,332.22583172)(530.50853227,332.17083177)(530.51853516,332.13083496)
\lineto(530.54853516,331.95083496)
\curveto(530.55853222,331.89083205)(530.57353221,331.83083211)(530.59353516,331.77083496)
\curveto(530.6835321,331.48083246)(530.78853199,331.2408327)(530.90853516,331.05083496)
\curveto(531.03853174,330.87083307)(531.21853156,330.71083323)(531.44853516,330.57083496)
\curveto(531.58853119,330.49083345)(531.75353103,330.42583352)(531.94353516,330.37583496)
\curveto(531.9835308,330.36583358)(532.01853076,330.36083358)(532.04853516,330.36083496)
\curveto(532.0785307,330.37083357)(532.11353067,330.37083357)(532.15353516,330.36083496)
\curveto(532.19353059,330.35083359)(532.25353053,330.3408336)(532.33353516,330.33083496)
\curveto(532.41353037,330.33083361)(532.4785303,330.33583361)(532.52853516,330.34583496)
\curveto(532.60853017,330.36583358)(532.68853009,330.38083356)(532.76853516,330.39083496)
\curveto(532.85852992,330.41083353)(532.94352984,330.43583351)(533.02353516,330.46583496)
\curveto(533.26352952,330.56583338)(533.45852932,330.70583324)(533.60853516,330.88583496)
\curveto(533.75852902,331.06583288)(533.8835289,331.27583267)(533.98353516,331.51583496)
\curveto(534.03352875,331.63583231)(534.06852871,331.76083218)(534.08853516,331.89083496)
\curveto(534.10852867,332.02083192)(534.13352865,332.15583179)(534.16353516,332.29583496)
\lineto(534.16353516,332.44583496)
\curveto(534.17352861,332.49583145)(534.1785286,332.5458314)(534.17853516,332.59583496)
}
}
{
\newrgbcolor{curcolor}{0 0 0}
\pscustom[linestyle=none,fillstyle=solid,fillcolor=curcolor]
{
\newpath
\moveto(537.99345703,336.70583496)
\lineto(539.11845703,336.70583496)
\curveto(539.2284546,336.70582724)(539.3284545,336.70082724)(539.41845703,336.69083496)
\curveto(539.50845432,336.68082726)(539.57345425,336.6458273)(539.61345703,336.58583496)
\curveto(539.66345416,336.52582742)(539.69345413,336.4408275)(539.70345703,336.33083496)
\curveto(539.71345411,336.23082771)(539.71845411,336.12582782)(539.71845703,336.01583496)
\lineto(539.71845703,334.96583496)
\lineto(539.71845703,332.73083496)
\curveto(539.71845411,332.37083157)(539.73345409,332.03083191)(539.76345703,331.71083496)
\curveto(539.79345403,331.39083255)(539.88345394,331.12583282)(540.03345703,330.91583496)
\curveto(540.17345365,330.70583324)(540.39845343,330.55583339)(540.70845703,330.46583496)
\curveto(540.75845307,330.45583349)(540.79845303,330.45083349)(540.82845703,330.45083496)
\curveto(540.86845296,330.45083349)(540.91345291,330.4458335)(540.96345703,330.43583496)
\curveto(541.01345281,330.42583352)(541.06845276,330.42083352)(541.12845703,330.42083496)
\curveto(541.18845264,330.42083352)(541.23345259,330.42583352)(541.26345703,330.43583496)
\curveto(541.31345251,330.45583349)(541.35345247,330.46083348)(541.38345703,330.45083496)
\curveto(541.4234524,330.4408335)(541.46345236,330.4458335)(541.50345703,330.46583496)
\curveto(541.71345211,330.51583343)(541.87845195,330.58083336)(541.99845703,330.66083496)
\curveto(542.17845165,330.77083317)(542.31845151,330.91083303)(542.41845703,331.08083496)
\curveto(542.5284513,331.26083268)(542.60345122,331.45583249)(542.64345703,331.66583496)
\curveto(542.69345113,331.88583206)(542.7234511,332.12583182)(542.73345703,332.38583496)
\curveto(542.74345108,332.65583129)(542.74845108,332.93583101)(542.74845703,333.22583496)
\lineto(542.74845703,335.04083496)
\lineto(542.74845703,336.01583496)
\lineto(542.74845703,336.28583496)
\curveto(542.74845108,336.38582756)(542.76845106,336.46582748)(542.80845703,336.52583496)
\curveto(542.85845097,336.61582733)(542.93345089,336.66582728)(543.03345703,336.67583496)
\curveto(543.13345069,336.69582725)(543.25345057,336.70582724)(543.39345703,336.70583496)
\lineto(544.18845703,336.70583496)
\lineto(544.47345703,336.70583496)
\curveto(544.56344926,336.70582724)(544.63844919,336.68582726)(544.69845703,336.64583496)
\curveto(544.77844905,336.59582735)(544.823449,336.52082742)(544.83345703,336.42083496)
\curveto(544.84344898,336.32082762)(544.84844898,336.20582774)(544.84845703,336.07583496)
\lineto(544.84845703,334.93583496)
\lineto(544.84845703,330.72083496)
\lineto(544.84845703,329.65583496)
\lineto(544.84845703,329.35583496)
\curveto(544.84844898,329.25583469)(544.828449,329.18083476)(544.78845703,329.13083496)
\curveto(544.73844909,329.05083489)(544.66344916,329.00583494)(544.56345703,328.99583496)
\curveto(544.46344936,328.98583496)(544.35844947,328.98083496)(544.24845703,328.98083496)
\lineto(543.43845703,328.98083496)
\curveto(543.3284505,328.98083496)(543.2284506,328.98583496)(543.13845703,328.99583496)
\curveto(543.05845077,329.00583494)(542.99345083,329.0458349)(542.94345703,329.11583496)
\curveto(542.9234509,329.1458348)(542.90345092,329.19083475)(542.88345703,329.25083496)
\curveto(542.87345095,329.31083463)(542.85845097,329.37083457)(542.83845703,329.43083496)
\curveto(542.828451,329.49083445)(542.81345101,329.5458344)(542.79345703,329.59583496)
\curveto(542.77345105,329.6458343)(542.74345108,329.67583427)(542.70345703,329.68583496)
\curveto(542.68345114,329.70583424)(542.65845117,329.71083423)(542.62845703,329.70083496)
\curveto(542.59845123,329.69083425)(542.57345125,329.68083426)(542.55345703,329.67083496)
\curveto(542.48345134,329.63083431)(542.4234514,329.58583436)(542.37345703,329.53583496)
\curveto(542.3234515,329.48583446)(542.26845156,329.4408345)(542.20845703,329.40083496)
\curveto(542.16845166,329.37083457)(542.1284517,329.33583461)(542.08845703,329.29583496)
\curveto(542.05845177,329.26583468)(542.01845181,329.23583471)(541.96845703,329.20583496)
\curveto(541.73845209,329.06583488)(541.46845236,328.95583499)(541.15845703,328.87583496)
\curveto(541.08845274,328.85583509)(541.01845281,328.8458351)(540.94845703,328.84583496)
\curveto(540.87845295,328.83583511)(540.80345302,328.82083512)(540.72345703,328.80083496)
\curveto(540.68345314,328.79083515)(540.63845319,328.79083515)(540.58845703,328.80083496)
\curveto(540.54845328,328.80083514)(540.50845332,328.79583515)(540.46845703,328.78583496)
\curveto(540.43845339,328.77583517)(540.37345345,328.77583517)(540.27345703,328.78583496)
\curveto(540.18345364,328.78583516)(540.1234537,328.79083515)(540.09345703,328.80083496)
\curveto(540.04345378,328.80083514)(539.99345383,328.80583514)(539.94345703,328.81583496)
\lineto(539.79345703,328.81583496)
\curveto(539.67345415,328.8458351)(539.55845427,328.87083507)(539.44845703,328.89083496)
\curveto(539.33845449,328.91083503)(539.2284546,328.940835)(539.11845703,328.98083496)
\curveto(539.06845476,329.00083494)(539.0234548,329.01583493)(538.98345703,329.02583496)
\curveto(538.95345487,329.0458349)(538.91345491,329.06583488)(538.86345703,329.08583496)
\curveto(538.51345531,329.27583467)(538.23345559,329.5408344)(538.02345703,329.88083496)
\curveto(537.89345593,330.09083385)(537.79845603,330.3408336)(537.73845703,330.63083496)
\curveto(537.67845615,330.93083301)(537.63845619,331.2458327)(537.61845703,331.57583496)
\curveto(537.60845622,331.91583203)(537.60345622,332.26083168)(537.60345703,332.61083496)
\curveto(537.61345621,332.97083097)(537.61845621,333.32583062)(537.61845703,333.67583496)
\lineto(537.61845703,335.71583496)
\curveto(537.61845621,335.8458281)(537.61345621,335.99582795)(537.60345703,336.16583496)
\curveto(537.60345622,336.3458276)(537.6284562,336.47582747)(537.67845703,336.55583496)
\curveto(537.70845612,336.60582734)(537.76845606,336.65082729)(537.85845703,336.69083496)
\curveto(537.91845591,336.69082725)(537.96345586,336.69582725)(537.99345703,336.70583496)
}
}
{
\newrgbcolor{curcolor}{0 0 0}
\pscustom[linestyle=none,fillstyle=solid,fillcolor=curcolor]
{
\newpath
\moveto(546.98970703,339.67583496)
\lineto(548.08470703,339.67583496)
\curveto(548.18470455,339.67582427)(548.27970445,339.67082427)(548.36970703,339.66083496)
\curveto(548.45970427,339.65082429)(548.5297042,339.62082432)(548.57970703,339.57083496)
\curveto(548.63970409,339.50082444)(548.66970406,339.40582454)(548.66970703,339.28583496)
\curveto(548.67970405,339.17582477)(548.68470405,339.06082488)(548.68470703,338.94083496)
\lineto(548.68470703,337.60583496)
\lineto(548.68470703,332.22083496)
\lineto(548.68470703,329.92583496)
\lineto(548.68470703,329.50583496)
\curveto(548.69470404,329.35583459)(548.67470406,329.2408347)(548.62470703,329.16083496)
\curveto(548.57470416,329.08083486)(548.48470425,329.02583492)(548.35470703,328.99583496)
\curveto(548.29470444,328.97583497)(548.22470451,328.97083497)(548.14470703,328.98083496)
\curveto(548.07470466,328.99083495)(548.00470473,328.99583495)(547.93470703,328.99583496)
\lineto(547.21470703,328.99583496)
\curveto(547.10470563,328.99583495)(547.00470573,329.00083494)(546.91470703,329.01083496)
\curveto(546.82470591,329.02083492)(546.74970598,329.05083489)(546.68970703,329.10083496)
\curveto(546.6297061,329.15083479)(546.59470614,329.22583472)(546.58470703,329.32583496)
\lineto(546.58470703,329.65583496)
\lineto(546.58470703,330.99083496)
\lineto(546.58470703,336.61583496)
\lineto(546.58470703,338.65583496)
\curveto(546.58470615,338.78582516)(546.57970615,338.940825)(546.56970703,339.12083496)
\curveto(546.56970616,339.30082464)(546.59470614,339.43082451)(546.64470703,339.51083496)
\curveto(546.66470607,339.55082439)(546.68970604,339.58082436)(546.71970703,339.60083496)
\lineto(546.83970703,339.66083496)
\curveto(546.85970587,339.66082428)(546.88470585,339.66082428)(546.91470703,339.66083496)
\curveto(546.94470579,339.67082427)(546.96970576,339.67582427)(546.98970703,339.67583496)
}
}
{
\newrgbcolor{curcolor}{0 0 0}
\pscustom[linestyle=none,fillstyle=solid,fillcolor=curcolor]
{
\newpath
\moveto(557.39689453,329.58083496)
\curveto(557.41688668,329.47083447)(557.42688667,329.36083458)(557.42689453,329.25083496)
\curveto(557.43688666,329.1408348)(557.38688671,329.06583488)(557.27689453,329.02583496)
\curveto(557.21688688,328.99583495)(557.14688695,328.98083496)(557.06689453,328.98083496)
\lineto(556.82689453,328.98083496)
\lineto(556.01689453,328.98083496)
\lineto(555.74689453,328.98083496)
\curveto(555.66688843,328.99083495)(555.6018885,329.01583493)(555.55189453,329.05583496)
\curveto(555.48188862,329.09583485)(555.42688867,329.15083479)(555.38689453,329.22083496)
\curveto(555.35688874,329.30083464)(555.31188879,329.36583458)(555.25189453,329.41583496)
\curveto(555.23188887,329.43583451)(555.20688889,329.45083449)(555.17689453,329.46083496)
\curveto(555.14688895,329.48083446)(555.10688899,329.48583446)(555.05689453,329.47583496)
\curveto(555.00688909,329.45583449)(554.95688914,329.43083451)(554.90689453,329.40083496)
\curveto(554.86688923,329.37083457)(554.82188928,329.3458346)(554.77189453,329.32583496)
\curveto(554.72188938,329.28583466)(554.66688943,329.25083469)(554.60689453,329.22083496)
\lineto(554.42689453,329.13083496)
\curveto(554.2968898,329.07083487)(554.16188994,329.02083492)(554.02189453,328.98083496)
\curveto(553.88189022,328.95083499)(553.73689036,328.91583503)(553.58689453,328.87583496)
\curveto(553.51689058,328.85583509)(553.44689065,328.8458351)(553.37689453,328.84583496)
\curveto(553.31689078,328.83583511)(553.25189085,328.82583512)(553.18189453,328.81583496)
\lineto(553.09189453,328.81583496)
\curveto(553.06189104,328.80583514)(553.03189107,328.80083514)(553.00189453,328.80083496)
\lineto(552.83689453,328.80083496)
\curveto(552.73689136,328.78083516)(552.63689146,328.78083516)(552.53689453,328.80083496)
\lineto(552.40189453,328.80083496)
\curveto(552.33189177,328.82083512)(552.26189184,328.83083511)(552.19189453,328.83083496)
\curveto(552.13189197,328.82083512)(552.07189203,328.82583512)(552.01189453,328.84583496)
\curveto(551.91189219,328.86583508)(551.81689228,328.88583506)(551.72689453,328.90583496)
\curveto(551.63689246,328.91583503)(551.55189255,328.940835)(551.47189453,328.98083496)
\curveto(551.18189292,329.09083485)(550.93189317,329.23083471)(550.72189453,329.40083496)
\curveto(550.52189358,329.58083436)(550.36189374,329.81583413)(550.24189453,330.10583496)
\curveto(550.21189389,330.17583377)(550.18189392,330.25083369)(550.15189453,330.33083496)
\curveto(550.13189397,330.41083353)(550.11189399,330.49583345)(550.09189453,330.58583496)
\curveto(550.07189403,330.63583331)(550.06189404,330.68583326)(550.06189453,330.73583496)
\curveto(550.07189403,330.78583316)(550.07189403,330.83583311)(550.06189453,330.88583496)
\curveto(550.05189405,330.91583303)(550.04189406,330.97583297)(550.03189453,331.06583496)
\curveto(550.03189407,331.16583278)(550.03689406,331.23583271)(550.04689453,331.27583496)
\curveto(550.06689403,331.37583257)(550.07689402,331.46083248)(550.07689453,331.53083496)
\lineto(550.16689453,331.86083496)
\curveto(550.1968939,331.98083196)(550.23689386,332.08583186)(550.28689453,332.17583496)
\curveto(550.45689364,332.46583148)(550.65189345,332.68583126)(550.87189453,332.83583496)
\curveto(551.09189301,332.98583096)(551.37189273,333.11583083)(551.71189453,333.22583496)
\curveto(551.84189226,333.27583067)(551.97689212,333.31083063)(552.11689453,333.33083496)
\curveto(552.25689184,333.35083059)(552.3968917,333.37583057)(552.53689453,333.40583496)
\curveto(552.61689148,333.42583052)(552.7018914,333.43583051)(552.79189453,333.43583496)
\curveto(552.88189122,333.4458305)(552.97189113,333.46083048)(553.06189453,333.48083496)
\curveto(553.13189097,333.50083044)(553.2018909,333.50583044)(553.27189453,333.49583496)
\curveto(553.34189076,333.49583045)(553.41689068,333.50583044)(553.49689453,333.52583496)
\curveto(553.56689053,333.5458304)(553.63689046,333.55583039)(553.70689453,333.55583496)
\curveto(553.77689032,333.55583039)(553.85189025,333.56583038)(553.93189453,333.58583496)
\curveto(554.14188996,333.63583031)(554.33188977,333.67583027)(554.50189453,333.70583496)
\curveto(554.68188942,333.7458302)(554.84188926,333.83583011)(554.98189453,333.97583496)
\curveto(555.07188903,334.06582988)(555.13188897,334.16582978)(555.16189453,334.27583496)
\curveto(555.17188893,334.30582964)(555.17188893,334.33082961)(555.16189453,334.35083496)
\curveto(555.16188894,334.37082957)(555.16688893,334.39082955)(555.17689453,334.41083496)
\curveto(555.18688891,334.43082951)(555.19188891,334.46082948)(555.19189453,334.50083496)
\lineto(555.19189453,334.59083496)
\lineto(555.16189453,334.71083496)
\curveto(555.16188894,334.75082919)(555.15688894,334.78582916)(555.14689453,334.81583496)
\curveto(555.04688905,335.11582883)(554.83688926,335.32082862)(554.51689453,335.43083496)
\curveto(554.42688967,335.46082848)(554.31688978,335.48082846)(554.18689453,335.49083496)
\curveto(554.06689003,335.51082843)(553.94189016,335.51582843)(553.81189453,335.50583496)
\curveto(553.68189042,335.50582844)(553.55689054,335.49582845)(553.43689453,335.47583496)
\curveto(553.31689078,335.45582849)(553.21189089,335.43082851)(553.12189453,335.40083496)
\curveto(553.06189104,335.38082856)(553.0018911,335.35082859)(552.94189453,335.31083496)
\curveto(552.89189121,335.28082866)(552.84189126,335.2458287)(552.79189453,335.20583496)
\curveto(552.74189136,335.16582878)(552.68689141,335.11082883)(552.62689453,335.04083496)
\curveto(552.57689152,334.97082897)(552.54189156,334.90582904)(552.52189453,334.84583496)
\curveto(552.47189163,334.7458292)(552.42689167,334.65082929)(552.38689453,334.56083496)
\curveto(552.35689174,334.47082947)(552.28689181,334.41082953)(552.17689453,334.38083496)
\curveto(552.096892,334.36082958)(552.01189209,334.35082959)(551.92189453,334.35083496)
\lineto(551.65189453,334.35083496)
\lineto(551.08189453,334.35083496)
\curveto(551.03189307,334.35082959)(550.98189312,334.3458296)(550.93189453,334.33583496)
\curveto(550.88189322,334.33582961)(550.83689326,334.3408296)(550.79689453,334.35083496)
\lineto(550.66189453,334.35083496)
\curveto(550.64189346,334.36082958)(550.61689348,334.36582958)(550.58689453,334.36583496)
\curveto(550.55689354,334.36582958)(550.53189357,334.37582957)(550.51189453,334.39583496)
\curveto(550.43189367,334.41582953)(550.37689372,334.48082946)(550.34689453,334.59083496)
\curveto(550.33689376,334.6408293)(550.33689376,334.69082925)(550.34689453,334.74083496)
\curveto(550.35689374,334.79082915)(550.36689373,334.83582911)(550.37689453,334.87583496)
\curveto(550.40689369,334.98582896)(550.43689366,335.08582886)(550.46689453,335.17583496)
\curveto(550.50689359,335.27582867)(550.55189355,335.36582858)(550.60189453,335.44583496)
\lineto(550.69189453,335.59583496)
\lineto(550.78189453,335.74583496)
\curveto(550.86189324,335.85582809)(550.96189314,335.96082798)(551.08189453,336.06083496)
\curveto(551.101893,336.07082787)(551.13189297,336.09582785)(551.17189453,336.13583496)
\curveto(551.22189288,336.17582777)(551.26689283,336.21082773)(551.30689453,336.24083496)
\curveto(551.34689275,336.27082767)(551.39189271,336.30082764)(551.44189453,336.33083496)
\curveto(551.61189249,336.4408275)(551.79189231,336.52582742)(551.98189453,336.58583496)
\curveto(552.17189193,336.65582729)(552.36689173,336.72082722)(552.56689453,336.78083496)
\curveto(552.68689141,336.81082713)(552.81189129,336.83082711)(552.94189453,336.84083496)
\curveto(553.07189103,336.85082709)(553.2018909,336.87082707)(553.33189453,336.90083496)
\curveto(553.37189073,336.91082703)(553.43189067,336.91082703)(553.51189453,336.90083496)
\curveto(553.6018905,336.89082705)(553.65689044,336.89582705)(553.67689453,336.91583496)
\curveto(554.08689001,336.92582702)(554.47688962,336.91082703)(554.84689453,336.87083496)
\curveto(555.22688887,336.83082711)(555.56688853,336.75582719)(555.86689453,336.64583496)
\curveto(556.17688792,336.53582741)(556.44188766,336.38582756)(556.66189453,336.19583496)
\curveto(556.88188722,336.01582793)(557.05188705,335.78082816)(557.17189453,335.49083496)
\curveto(557.24188686,335.32082862)(557.28188682,335.12582882)(557.29189453,334.90583496)
\curveto(557.3018868,334.68582926)(557.30688679,334.46082948)(557.30689453,334.23083496)
\lineto(557.30689453,330.88583496)
\lineto(557.30689453,330.30083496)
\curveto(557.30688679,330.11083383)(557.32688677,329.93583401)(557.36689453,329.77583496)
\curveto(557.37688672,329.7458342)(557.38188672,329.71083423)(557.38189453,329.67083496)
\curveto(557.38188672,329.6408343)(557.38688671,329.61083433)(557.39689453,329.58083496)
\moveto(555.19189453,331.89083496)
\curveto(555.2018889,331.940832)(555.20688889,331.99583195)(555.20689453,332.05583496)
\curveto(555.20688889,332.12583182)(555.2018889,332.18583176)(555.19189453,332.23583496)
\curveto(555.17188893,332.29583165)(555.16188894,332.35083159)(555.16189453,332.40083496)
\curveto(555.16188894,332.45083149)(555.14188896,332.49083145)(555.10189453,332.52083496)
\curveto(555.05188905,332.56083138)(554.97688912,332.58083136)(554.87689453,332.58083496)
\curveto(554.83688926,332.57083137)(554.8018893,332.56083138)(554.77189453,332.55083496)
\curveto(554.74188936,332.55083139)(554.70688939,332.5458314)(554.66689453,332.53583496)
\curveto(554.5968895,332.51583143)(554.52188958,332.50083144)(554.44189453,332.49083496)
\curveto(554.36188974,332.48083146)(554.28188982,332.46583148)(554.20189453,332.44583496)
\curveto(554.17188993,332.43583151)(554.12688997,332.43083151)(554.06689453,332.43083496)
\curveto(553.93689016,332.40083154)(553.80689029,332.38083156)(553.67689453,332.37083496)
\curveto(553.54689055,332.36083158)(553.42189068,332.33583161)(553.30189453,332.29583496)
\curveto(553.22189088,332.27583167)(553.14689095,332.25583169)(553.07689453,332.23583496)
\curveto(553.00689109,332.22583172)(552.93689116,332.20583174)(552.86689453,332.17583496)
\curveto(552.65689144,332.08583186)(552.47689162,331.95083199)(552.32689453,331.77083496)
\curveto(552.18689191,331.59083235)(552.13689196,331.3408326)(552.17689453,331.02083496)
\curveto(552.1968919,330.85083309)(552.25189185,330.71083323)(552.34189453,330.60083496)
\curveto(552.41189169,330.49083345)(552.51689158,330.40083354)(552.65689453,330.33083496)
\curveto(552.7968913,330.27083367)(552.94689115,330.22583372)(553.10689453,330.19583496)
\curveto(553.27689082,330.16583378)(553.45189065,330.15583379)(553.63189453,330.16583496)
\curveto(553.82189028,330.18583376)(553.9968901,330.22083372)(554.15689453,330.27083496)
\curveto(554.41688968,330.35083359)(554.62188948,330.47583347)(554.77189453,330.64583496)
\curveto(554.92188918,330.82583312)(555.03688906,331.0458329)(555.11689453,331.30583496)
\curveto(555.13688896,331.37583257)(555.14688895,331.4458325)(555.14689453,331.51583496)
\curveto(555.15688894,331.59583235)(555.17188893,331.67583227)(555.19189453,331.75583496)
\lineto(555.19189453,331.89083496)
}
}
{
\newrgbcolor{curcolor}{0 0 0}
\pscustom[linestyle=none,fillstyle=solid,fillcolor=curcolor]
{
\newpath
\moveto(563.38517578,336.91583496)
\curveto(563.49517047,336.91582703)(563.59017037,336.90582704)(563.67017578,336.88583496)
\curveto(563.7601702,336.86582708)(563.83017013,336.82082712)(563.88017578,336.75083496)
\curveto(563.94017002,336.67082727)(563.97016999,336.53082741)(563.97017578,336.33083496)
\lineto(563.97017578,335.82083496)
\lineto(563.97017578,335.44583496)
\curveto(563.98016998,335.30582864)(563.96517,335.19582875)(563.92517578,335.11583496)
\curveto(563.88517008,335.0458289)(563.82517014,335.00082894)(563.74517578,334.98083496)
\curveto(563.67517029,334.96082898)(563.59017037,334.95082899)(563.49017578,334.95083496)
\curveto(563.40017056,334.95082899)(563.30017066,334.95582899)(563.19017578,334.96583496)
\curveto(563.09017087,334.97582897)(562.99517097,334.97082897)(562.90517578,334.95083496)
\curveto(562.83517113,334.93082901)(562.7651712,334.91582903)(562.69517578,334.90583496)
\curveto(562.62517134,334.90582904)(562.5601714,334.89582905)(562.50017578,334.87583496)
\curveto(562.34017162,334.82582912)(562.18017178,334.75082919)(562.02017578,334.65083496)
\curveto(561.8601721,334.56082938)(561.73517223,334.45582949)(561.64517578,334.33583496)
\curveto(561.59517237,334.25582969)(561.54017242,334.17082977)(561.48017578,334.08083496)
\curveto(561.43017253,334.00082994)(561.38017258,333.91583003)(561.33017578,333.82583496)
\curveto(561.30017266,333.7458302)(561.27017269,333.66083028)(561.24017578,333.57083496)
\lineto(561.18017578,333.33083496)
\curveto(561.1601728,333.26083068)(561.15017281,333.18583076)(561.15017578,333.10583496)
\curveto(561.15017281,333.03583091)(561.14017282,332.96583098)(561.12017578,332.89583496)
\curveto(561.11017285,332.85583109)(561.10517286,332.81583113)(561.10517578,332.77583496)
\curveto(561.11517285,332.7458312)(561.11517285,332.71583123)(561.10517578,332.68583496)
\lineto(561.10517578,332.44583496)
\curveto(561.08517288,332.37583157)(561.08017288,332.29583165)(561.09017578,332.20583496)
\curveto(561.10017286,332.12583182)(561.10517286,332.0458319)(561.10517578,331.96583496)
\lineto(561.10517578,331.00583496)
\lineto(561.10517578,329.73083496)
\curveto(561.10517286,329.60083434)(561.10017286,329.48083446)(561.09017578,329.37083496)
\curveto(561.08017288,329.26083468)(561.05017291,329.17083477)(561.00017578,329.10083496)
\curveto(560.98017298,329.07083487)(560.94517302,329.0458349)(560.89517578,329.02583496)
\curveto(560.85517311,329.01583493)(560.81017315,329.00583494)(560.76017578,328.99583496)
\lineto(560.68517578,328.99583496)
\curveto(560.63517333,328.98583496)(560.58017338,328.98083496)(560.52017578,328.98083496)
\lineto(560.35517578,328.98083496)
\lineto(559.71017578,328.98083496)
\curveto(559.65017431,328.99083495)(559.58517438,328.99583495)(559.51517578,328.99583496)
\lineto(559.32017578,328.99583496)
\curveto(559.27017469,329.01583493)(559.22017474,329.03083491)(559.17017578,329.04083496)
\curveto(559.12017484,329.06083488)(559.08517488,329.09583485)(559.06517578,329.14583496)
\curveto(559.02517494,329.19583475)(559.00017496,329.26583468)(558.99017578,329.35583496)
\lineto(558.99017578,329.65583496)
\lineto(558.99017578,330.67583496)
\lineto(558.99017578,334.90583496)
\lineto(558.99017578,336.01583496)
\lineto(558.99017578,336.30083496)
\curveto(558.99017497,336.40082754)(559.01017495,336.48082746)(559.05017578,336.54083496)
\curveto(559.10017486,336.62082732)(559.17517479,336.67082727)(559.27517578,336.69083496)
\curveto(559.37517459,336.71082723)(559.49517447,336.72082722)(559.63517578,336.72083496)
\lineto(560.40017578,336.72083496)
\curveto(560.52017344,336.72082722)(560.62517334,336.71082723)(560.71517578,336.69083496)
\curveto(560.80517316,336.68082726)(560.87517309,336.63582731)(560.92517578,336.55583496)
\curveto(560.95517301,336.50582744)(560.97017299,336.43582751)(560.97017578,336.34583496)
\lineto(561.00017578,336.07583496)
\curveto(561.01017295,335.99582795)(561.02517294,335.92082802)(561.04517578,335.85083496)
\curveto(561.07517289,335.78082816)(561.12517284,335.7458282)(561.19517578,335.74583496)
\curveto(561.21517275,335.76582818)(561.23517273,335.77582817)(561.25517578,335.77583496)
\curveto(561.27517269,335.77582817)(561.29517267,335.78582816)(561.31517578,335.80583496)
\curveto(561.37517259,335.85582809)(561.42517254,335.91082803)(561.46517578,335.97083496)
\curveto(561.51517245,336.0408279)(561.57517239,336.10082784)(561.64517578,336.15083496)
\curveto(561.68517228,336.18082776)(561.72017224,336.21082773)(561.75017578,336.24083496)
\curveto(561.78017218,336.28082766)(561.81517215,336.31582763)(561.85517578,336.34583496)
\lineto(562.12517578,336.52583496)
\curveto(562.22517174,336.58582736)(562.32517164,336.6408273)(562.42517578,336.69083496)
\curveto(562.52517144,336.73082721)(562.62517134,336.76582718)(562.72517578,336.79583496)
\lineto(563.05517578,336.88583496)
\curveto(563.08517088,336.89582705)(563.14017082,336.89582705)(563.22017578,336.88583496)
\curveto(563.31017065,336.88582706)(563.3651706,336.89582705)(563.38517578,336.91583496)
}
}
{
\newrgbcolor{curcolor}{0 0 0}
\pscustom[linestyle=none,fillstyle=solid,fillcolor=curcolor]
{
\newpath
\moveto(566.89025391,339.57083496)
\curveto(566.96025096,339.49082445)(566.99525092,339.37082457)(566.99525391,339.21083496)
\lineto(566.99525391,338.74583496)
\lineto(566.99525391,338.34083496)
\curveto(566.99525092,338.20082574)(566.96025096,338.10582584)(566.89025391,338.05583496)
\curveto(566.83025109,338.00582594)(566.75025117,337.97582597)(566.65025391,337.96583496)
\curveto(566.56025136,337.95582599)(566.46025146,337.95082599)(566.35025391,337.95083496)
\lineto(565.51025391,337.95083496)
\curveto(565.40025252,337.95082599)(565.30025262,337.95582599)(565.21025391,337.96583496)
\curveto(565.13025279,337.97582597)(565.06025286,338.00582594)(565.00025391,338.05583496)
\curveto(564.96025296,338.08582586)(564.93025299,338.1408258)(564.91025391,338.22083496)
\curveto(564.90025302,338.31082563)(564.89025303,338.40582554)(564.88025391,338.50583496)
\lineto(564.88025391,338.83583496)
\curveto(564.89025303,338.945825)(564.89525302,339.0408249)(564.89525391,339.12083496)
\lineto(564.89525391,339.33083496)
\curveto(564.90525301,339.40082454)(564.92525299,339.46082448)(564.95525391,339.51083496)
\curveto(564.97525294,339.55082439)(565.00025292,339.58082436)(565.03025391,339.60083496)
\lineto(565.15025391,339.66083496)
\curveto(565.17025275,339.66082428)(565.19525272,339.66082428)(565.22525391,339.66083496)
\curveto(565.25525266,339.67082427)(565.28025264,339.67582427)(565.30025391,339.67583496)
\lineto(566.39525391,339.67583496)
\curveto(566.49525142,339.67582427)(566.59025133,339.67082427)(566.68025391,339.66083496)
\curveto(566.77025115,339.65082429)(566.84025108,339.62082432)(566.89025391,339.57083496)
\moveto(566.99525391,329.80583496)
\curveto(566.99525092,329.60583434)(566.99025093,329.43583451)(566.98025391,329.29583496)
\curveto(566.97025095,329.15583479)(566.88025104,329.06083488)(566.71025391,329.01083496)
\curveto(566.65025127,328.99083495)(566.58525133,328.98083496)(566.51525391,328.98083496)
\curveto(566.44525147,328.99083495)(566.37025155,328.99583495)(566.29025391,328.99583496)
\lineto(565.45025391,328.99583496)
\curveto(565.36025256,328.99583495)(565.27025265,329.00083494)(565.18025391,329.01083496)
\curveto(565.10025282,329.02083492)(565.04025288,329.05083489)(565.00025391,329.10083496)
\curveto(564.94025298,329.17083477)(564.90525301,329.25583469)(564.89525391,329.35583496)
\lineto(564.89525391,329.70083496)
\lineto(564.89525391,336.03083496)
\lineto(564.89525391,336.33083496)
\curveto(564.89525302,336.43082751)(564.915253,336.51082743)(564.95525391,336.57083496)
\curveto(565.0152529,336.6408273)(565.10025282,336.68582726)(565.21025391,336.70583496)
\curveto(565.23025269,336.71582723)(565.25525266,336.71582723)(565.28525391,336.70583496)
\curveto(565.32525259,336.70582724)(565.35525256,336.71082723)(565.37525391,336.72083496)
\lineto(566.12525391,336.72083496)
\lineto(566.32025391,336.72083496)
\curveto(566.40025152,336.73082721)(566.46525145,336.73082721)(566.51525391,336.72083496)
\lineto(566.63525391,336.72083496)
\curveto(566.69525122,336.70082724)(566.75025117,336.68582726)(566.80025391,336.67583496)
\curveto(566.85025107,336.66582728)(566.89025103,336.63582731)(566.92025391,336.58583496)
\curveto(566.96025096,336.53582741)(566.98025094,336.46582748)(566.98025391,336.37583496)
\curveto(566.99025093,336.28582766)(566.99525092,336.19082775)(566.99525391,336.09083496)
\lineto(566.99525391,329.80583496)
}
}
{
\newrgbcolor{curcolor}{0 0 0}
\pscustom[linestyle=none,fillstyle=solid,fillcolor=curcolor]
{
\newpath
\moveto(576.24744141,329.83583496)
\lineto(576.24744141,329.41583496)
\curveto(576.24743304,329.28583466)(576.21743307,329.18083476)(576.15744141,329.10083496)
\curveto(576.10743318,329.05083489)(576.04243324,329.01583493)(575.96244141,328.99583496)
\curveto(575.8824334,328.98583496)(575.79243349,328.98083496)(575.69244141,328.98083496)
\lineto(574.86744141,328.98083496)
\lineto(574.58244141,328.98083496)
\curveto(574.50243478,328.99083495)(574.43743485,329.01583493)(574.38744141,329.05583496)
\curveto(574.31743497,329.10583484)(574.27743501,329.17083477)(574.26744141,329.25083496)
\curveto(574.25743503,329.33083461)(574.23743505,329.41083453)(574.20744141,329.49083496)
\curveto(574.1874351,329.51083443)(574.16743512,329.52583442)(574.14744141,329.53583496)
\curveto(574.13743515,329.55583439)(574.12243516,329.57583437)(574.10244141,329.59583496)
\curveto(573.99243529,329.59583435)(573.91243537,329.57083437)(573.86244141,329.52083496)
\lineto(573.71244141,329.37083496)
\curveto(573.64243564,329.32083462)(573.57743571,329.27583467)(573.51744141,329.23583496)
\curveto(573.45743583,329.20583474)(573.39243589,329.16583478)(573.32244141,329.11583496)
\curveto(573.282436,329.09583485)(573.23743605,329.07583487)(573.18744141,329.05583496)
\curveto(573.14743614,329.03583491)(573.10243618,329.01583493)(573.05244141,328.99583496)
\curveto(572.91243637,328.945835)(572.76243652,328.90083504)(572.60244141,328.86083496)
\curveto(572.55243673,328.8408351)(572.50743678,328.83083511)(572.46744141,328.83083496)
\curveto(572.42743686,328.83083511)(572.3874369,328.82583512)(572.34744141,328.81583496)
\lineto(572.21244141,328.81583496)
\curveto(572.1824371,328.80583514)(572.14243714,328.80083514)(572.09244141,328.80083496)
\lineto(571.95744141,328.80083496)
\curveto(571.89743739,328.78083516)(571.80743748,328.77583517)(571.68744141,328.78583496)
\curveto(571.56743772,328.78583516)(571.4824378,328.79583515)(571.43244141,328.81583496)
\curveto(571.36243792,328.83583511)(571.29743799,328.8458351)(571.23744141,328.84583496)
\curveto(571.1874381,328.83583511)(571.13243815,328.8408351)(571.07244141,328.86083496)
\lineto(570.71244141,328.98083496)
\curveto(570.60243868,329.01083493)(570.49243879,329.05083489)(570.38244141,329.10083496)
\curveto(570.03243925,329.25083469)(569.71743957,329.48083446)(569.43744141,329.79083496)
\curveto(569.16744012,330.11083383)(568.95244033,330.4458335)(568.79244141,330.79583496)
\curveto(568.74244054,330.90583304)(568.70244058,331.01083293)(568.67244141,331.11083496)
\curveto(568.64244064,331.22083272)(568.60744068,331.33083261)(568.56744141,331.44083496)
\curveto(568.55744073,331.48083246)(568.55244073,331.51583243)(568.55244141,331.54583496)
\curveto(568.55244073,331.58583236)(568.54244074,331.63083231)(568.52244141,331.68083496)
\curveto(568.50244078,331.76083218)(568.4824408,331.8458321)(568.46244141,331.93583496)
\curveto(568.45244083,332.03583191)(568.43744085,332.13583181)(568.41744141,332.23583496)
\curveto(568.40744088,332.26583168)(568.40244088,332.30083164)(568.40244141,332.34083496)
\curveto(568.41244087,332.38083156)(568.41244087,332.41583153)(568.40244141,332.44583496)
\lineto(568.40244141,332.58083496)
\curveto(568.40244088,332.63083131)(568.39744089,332.68083126)(568.38744141,332.73083496)
\curveto(568.37744091,332.78083116)(568.37244091,332.83583111)(568.37244141,332.89583496)
\curveto(568.37244091,332.96583098)(568.37744091,333.02083092)(568.38744141,333.06083496)
\curveto(568.39744089,333.11083083)(568.40244088,333.15583079)(568.40244141,333.19583496)
\lineto(568.40244141,333.34583496)
\curveto(568.41244087,333.39583055)(568.41244087,333.4408305)(568.40244141,333.48083496)
\curveto(568.40244088,333.53083041)(568.41244087,333.58083036)(568.43244141,333.63083496)
\curveto(568.45244083,333.7408302)(568.46744082,333.8458301)(568.47744141,333.94583496)
\curveto(568.49744079,334.0458299)(568.52244076,334.1458298)(568.55244141,334.24583496)
\curveto(568.59244069,334.36582958)(568.62744066,334.48082946)(568.65744141,334.59083496)
\curveto(568.6874406,334.70082924)(568.72744056,334.81082913)(568.77744141,334.92083496)
\curveto(568.91744037,335.22082872)(569.09244019,335.50582844)(569.30244141,335.77583496)
\curveto(569.32243996,335.80582814)(569.34743994,335.83082811)(569.37744141,335.85083496)
\curveto(569.41743987,335.88082806)(569.44743984,335.91082803)(569.46744141,335.94083496)
\curveto(569.50743978,335.99082795)(569.54743974,336.03582791)(569.58744141,336.07583496)
\curveto(569.62743966,336.11582783)(569.67243961,336.15582779)(569.72244141,336.19583496)
\curveto(569.76243952,336.21582773)(569.79743949,336.2408277)(569.82744141,336.27083496)
\curveto(569.85743943,336.31082763)(569.89243939,336.3408276)(569.93244141,336.36083496)
\curveto(570.1824391,336.53082741)(570.47243881,336.67082727)(570.80244141,336.78083496)
\curveto(570.87243841,336.80082714)(570.94243834,336.81582713)(571.01244141,336.82583496)
\curveto(571.09243819,336.83582711)(571.17243811,336.85082709)(571.25244141,336.87083496)
\curveto(571.32243796,336.89082705)(571.41243787,336.90082704)(571.52244141,336.90083496)
\curveto(571.63243765,336.91082703)(571.74243754,336.91582703)(571.85244141,336.91583496)
\curveto(571.96243732,336.91582703)(572.06743722,336.91082703)(572.16744141,336.90083496)
\curveto(572.27743701,336.89082705)(572.36743692,336.87582707)(572.43744141,336.85583496)
\curveto(572.5874367,336.80582714)(572.73243655,336.76082718)(572.87244141,336.72083496)
\curveto(573.01243627,336.68082726)(573.14243614,336.62582732)(573.26244141,336.55583496)
\curveto(573.33243595,336.50582744)(573.39743589,336.45582749)(573.45744141,336.40583496)
\curveto(573.51743577,336.36582758)(573.5824357,336.32082762)(573.65244141,336.27083496)
\curveto(573.69243559,336.2408277)(573.74743554,336.20082774)(573.81744141,336.15083496)
\curveto(573.89743539,336.10082784)(573.97243531,336.10082784)(574.04244141,336.15083496)
\curveto(574.0824352,336.17082777)(574.10243518,336.20582774)(574.10244141,336.25583496)
\curveto(574.10243518,336.30582764)(574.11243517,336.35582759)(574.13244141,336.40583496)
\lineto(574.13244141,336.55583496)
\curveto(574.14243514,336.58582736)(574.14743514,336.62082732)(574.14744141,336.66083496)
\lineto(574.14744141,336.78083496)
\lineto(574.14744141,338.82083496)
\curveto(574.14743514,338.93082501)(574.14243514,339.05082489)(574.13244141,339.18083496)
\curveto(574.13243515,339.32082462)(574.15743513,339.42582452)(574.20744141,339.49583496)
\curveto(574.24743504,339.57582437)(574.32243496,339.62582432)(574.43244141,339.64583496)
\curveto(574.45243483,339.65582429)(574.47243481,339.65582429)(574.49244141,339.64583496)
\curveto(574.51243477,339.6458243)(574.53243475,339.65082429)(574.55244141,339.66083496)
\lineto(575.61744141,339.66083496)
\curveto(575.73743355,339.66082428)(575.84743344,339.65582429)(575.94744141,339.64583496)
\curveto(576.04743324,339.63582431)(576.12243316,339.59582435)(576.17244141,339.52583496)
\curveto(576.22243306,339.4458245)(576.24743304,339.3408246)(576.24744141,339.21083496)
\lineto(576.24744141,338.85083496)
\lineto(576.24744141,329.83583496)
\moveto(574.20744141,332.77583496)
\curveto(574.21743507,332.81583113)(574.21743507,332.85583109)(574.20744141,332.89583496)
\lineto(574.20744141,333.03083496)
\curveto(574.20743508,333.13083081)(574.20243508,333.23083071)(574.19244141,333.33083496)
\curveto(574.1824351,333.43083051)(574.16743512,333.52083042)(574.14744141,333.60083496)
\curveto(574.12743516,333.71083023)(574.10743518,333.81083013)(574.08744141,333.90083496)
\curveto(574.07743521,333.99082995)(574.05243523,334.07582987)(574.01244141,334.15583496)
\curveto(573.87243541,334.51582943)(573.66743562,334.80082914)(573.39744141,335.01083496)
\curveto(573.13743615,335.22082872)(572.75743653,335.32582862)(572.25744141,335.32583496)
\curveto(572.19743709,335.32582862)(572.11743717,335.31582863)(572.01744141,335.29583496)
\curveto(571.93743735,335.27582867)(571.86243742,335.25582869)(571.79244141,335.23583496)
\curveto(571.73243755,335.22582872)(571.67243761,335.20582874)(571.61244141,335.17583496)
\curveto(571.34243794,335.06582888)(571.13243815,334.89582905)(570.98244141,334.66583496)
\curveto(570.83243845,334.43582951)(570.71243857,334.17582977)(570.62244141,333.88583496)
\curveto(570.59243869,333.78583016)(570.57243871,333.68583026)(570.56244141,333.58583496)
\curveto(570.55243873,333.48583046)(570.53243875,333.38083056)(570.50244141,333.27083496)
\lineto(570.50244141,333.06083496)
\curveto(570.4824388,332.97083097)(570.47743881,332.8458311)(570.48744141,332.68583496)
\curveto(570.49743879,332.53583141)(570.51243877,332.42583152)(570.53244141,332.35583496)
\lineto(570.53244141,332.26583496)
\curveto(570.54243874,332.2458317)(570.54743874,332.22583172)(570.54744141,332.20583496)
\curveto(570.56743872,332.12583182)(570.5824387,332.05083189)(570.59244141,331.98083496)
\curveto(570.61243867,331.91083203)(570.63243865,331.83583211)(570.65244141,331.75583496)
\curveto(570.82243846,331.23583271)(571.11243817,330.85083309)(571.52244141,330.60083496)
\curveto(571.65243763,330.51083343)(571.83243745,330.4408335)(572.06244141,330.39083496)
\curveto(572.10243718,330.38083356)(572.16243712,330.37583357)(572.24244141,330.37583496)
\curveto(572.27243701,330.36583358)(572.31743697,330.35583359)(572.37744141,330.34583496)
\curveto(572.44743684,330.3458336)(572.50243678,330.35083359)(572.54244141,330.36083496)
\curveto(572.62243666,330.38083356)(572.70243658,330.39583355)(572.78244141,330.40583496)
\curveto(572.86243642,330.41583353)(572.94243634,330.43583351)(573.02244141,330.46583496)
\curveto(573.27243601,330.57583337)(573.47243581,330.71583323)(573.62244141,330.88583496)
\curveto(573.77243551,331.05583289)(573.90243538,331.27083267)(574.01244141,331.53083496)
\curveto(574.05243523,331.62083232)(574.0824352,331.71083223)(574.10244141,331.80083496)
\curveto(574.12243516,331.90083204)(574.14243514,332.00583194)(574.16244141,332.11583496)
\curveto(574.17243511,332.16583178)(574.17243511,332.21083173)(574.16244141,332.25083496)
\curveto(574.16243512,332.30083164)(574.17243511,332.35083159)(574.19244141,332.40083496)
\curveto(574.20243508,332.43083151)(574.20743508,332.46583148)(574.20744141,332.50583496)
\lineto(574.20744141,332.64083496)
\lineto(574.20744141,332.77583496)
}
}
{
\newrgbcolor{curcolor}{0 0 0}
\pscustom[linestyle=none,fillstyle=solid,fillcolor=curcolor]
{
\newpath
\moveto(584.87736328,329.58083496)
\curveto(584.89735543,329.47083447)(584.90735542,329.36083458)(584.90736328,329.25083496)
\curveto(584.91735541,329.1408348)(584.86735546,329.06583488)(584.75736328,329.02583496)
\curveto(584.69735563,328.99583495)(584.6273557,328.98083496)(584.54736328,328.98083496)
\lineto(584.30736328,328.98083496)
\lineto(583.49736328,328.98083496)
\lineto(583.22736328,328.98083496)
\curveto(583.14735718,328.99083495)(583.08235725,329.01583493)(583.03236328,329.05583496)
\curveto(582.96235737,329.09583485)(582.90735742,329.15083479)(582.86736328,329.22083496)
\curveto(582.83735749,329.30083464)(582.79235754,329.36583458)(582.73236328,329.41583496)
\curveto(582.71235762,329.43583451)(582.68735764,329.45083449)(582.65736328,329.46083496)
\curveto(582.6273577,329.48083446)(582.58735774,329.48583446)(582.53736328,329.47583496)
\curveto(582.48735784,329.45583449)(582.43735789,329.43083451)(582.38736328,329.40083496)
\curveto(582.34735798,329.37083457)(582.30235803,329.3458346)(582.25236328,329.32583496)
\curveto(582.20235813,329.28583466)(582.14735818,329.25083469)(582.08736328,329.22083496)
\lineto(581.90736328,329.13083496)
\curveto(581.77735855,329.07083487)(581.64235869,329.02083492)(581.50236328,328.98083496)
\curveto(581.36235897,328.95083499)(581.21735911,328.91583503)(581.06736328,328.87583496)
\curveto(580.99735933,328.85583509)(580.9273594,328.8458351)(580.85736328,328.84583496)
\curveto(580.79735953,328.83583511)(580.7323596,328.82583512)(580.66236328,328.81583496)
\lineto(580.57236328,328.81583496)
\curveto(580.54235979,328.80583514)(580.51235982,328.80083514)(580.48236328,328.80083496)
\lineto(580.31736328,328.80083496)
\curveto(580.21736011,328.78083516)(580.11736021,328.78083516)(580.01736328,328.80083496)
\lineto(579.88236328,328.80083496)
\curveto(579.81236052,328.82083512)(579.74236059,328.83083511)(579.67236328,328.83083496)
\curveto(579.61236072,328.82083512)(579.55236078,328.82583512)(579.49236328,328.84583496)
\curveto(579.39236094,328.86583508)(579.29736103,328.88583506)(579.20736328,328.90583496)
\curveto(579.11736121,328.91583503)(579.0323613,328.940835)(578.95236328,328.98083496)
\curveto(578.66236167,329.09083485)(578.41236192,329.23083471)(578.20236328,329.40083496)
\curveto(578.00236233,329.58083436)(577.84236249,329.81583413)(577.72236328,330.10583496)
\curveto(577.69236264,330.17583377)(577.66236267,330.25083369)(577.63236328,330.33083496)
\curveto(577.61236272,330.41083353)(577.59236274,330.49583345)(577.57236328,330.58583496)
\curveto(577.55236278,330.63583331)(577.54236279,330.68583326)(577.54236328,330.73583496)
\curveto(577.55236278,330.78583316)(577.55236278,330.83583311)(577.54236328,330.88583496)
\curveto(577.5323628,330.91583303)(577.52236281,330.97583297)(577.51236328,331.06583496)
\curveto(577.51236282,331.16583278)(577.51736281,331.23583271)(577.52736328,331.27583496)
\curveto(577.54736278,331.37583257)(577.55736277,331.46083248)(577.55736328,331.53083496)
\lineto(577.64736328,331.86083496)
\curveto(577.67736265,331.98083196)(577.71736261,332.08583186)(577.76736328,332.17583496)
\curveto(577.93736239,332.46583148)(578.1323622,332.68583126)(578.35236328,332.83583496)
\curveto(578.57236176,332.98583096)(578.85236148,333.11583083)(579.19236328,333.22583496)
\curveto(579.32236101,333.27583067)(579.45736087,333.31083063)(579.59736328,333.33083496)
\curveto(579.73736059,333.35083059)(579.87736045,333.37583057)(580.01736328,333.40583496)
\curveto(580.09736023,333.42583052)(580.18236015,333.43583051)(580.27236328,333.43583496)
\curveto(580.36235997,333.4458305)(580.45235988,333.46083048)(580.54236328,333.48083496)
\curveto(580.61235972,333.50083044)(580.68235965,333.50583044)(580.75236328,333.49583496)
\curveto(580.82235951,333.49583045)(580.89735943,333.50583044)(580.97736328,333.52583496)
\curveto(581.04735928,333.5458304)(581.11735921,333.55583039)(581.18736328,333.55583496)
\curveto(581.25735907,333.55583039)(581.332359,333.56583038)(581.41236328,333.58583496)
\curveto(581.62235871,333.63583031)(581.81235852,333.67583027)(581.98236328,333.70583496)
\curveto(582.16235817,333.7458302)(582.32235801,333.83583011)(582.46236328,333.97583496)
\curveto(582.55235778,334.06582988)(582.61235772,334.16582978)(582.64236328,334.27583496)
\curveto(582.65235768,334.30582964)(582.65235768,334.33082961)(582.64236328,334.35083496)
\curveto(582.64235769,334.37082957)(582.64735768,334.39082955)(582.65736328,334.41083496)
\curveto(582.66735766,334.43082951)(582.67235766,334.46082948)(582.67236328,334.50083496)
\lineto(582.67236328,334.59083496)
\lineto(582.64236328,334.71083496)
\curveto(582.64235769,334.75082919)(582.63735769,334.78582916)(582.62736328,334.81583496)
\curveto(582.5273578,335.11582883)(582.31735801,335.32082862)(581.99736328,335.43083496)
\curveto(581.90735842,335.46082848)(581.79735853,335.48082846)(581.66736328,335.49083496)
\curveto(581.54735878,335.51082843)(581.42235891,335.51582843)(581.29236328,335.50583496)
\curveto(581.16235917,335.50582844)(581.03735929,335.49582845)(580.91736328,335.47583496)
\curveto(580.79735953,335.45582849)(580.69235964,335.43082851)(580.60236328,335.40083496)
\curveto(580.54235979,335.38082856)(580.48235985,335.35082859)(580.42236328,335.31083496)
\curveto(580.37235996,335.28082866)(580.32236001,335.2458287)(580.27236328,335.20583496)
\curveto(580.22236011,335.16582878)(580.16736016,335.11082883)(580.10736328,335.04083496)
\curveto(580.05736027,334.97082897)(580.02236031,334.90582904)(580.00236328,334.84583496)
\curveto(579.95236038,334.7458292)(579.90736042,334.65082929)(579.86736328,334.56083496)
\curveto(579.83736049,334.47082947)(579.76736056,334.41082953)(579.65736328,334.38083496)
\curveto(579.57736075,334.36082958)(579.49236084,334.35082959)(579.40236328,334.35083496)
\lineto(579.13236328,334.35083496)
\lineto(578.56236328,334.35083496)
\curveto(578.51236182,334.35082959)(578.46236187,334.3458296)(578.41236328,334.33583496)
\curveto(578.36236197,334.33582961)(578.31736201,334.3408296)(578.27736328,334.35083496)
\lineto(578.14236328,334.35083496)
\curveto(578.12236221,334.36082958)(578.09736223,334.36582958)(578.06736328,334.36583496)
\curveto(578.03736229,334.36582958)(578.01236232,334.37582957)(577.99236328,334.39583496)
\curveto(577.91236242,334.41582953)(577.85736247,334.48082946)(577.82736328,334.59083496)
\curveto(577.81736251,334.6408293)(577.81736251,334.69082925)(577.82736328,334.74083496)
\curveto(577.83736249,334.79082915)(577.84736248,334.83582911)(577.85736328,334.87583496)
\curveto(577.88736244,334.98582896)(577.91736241,335.08582886)(577.94736328,335.17583496)
\curveto(577.98736234,335.27582867)(578.0323623,335.36582858)(578.08236328,335.44583496)
\lineto(578.17236328,335.59583496)
\lineto(578.26236328,335.74583496)
\curveto(578.34236199,335.85582809)(578.44236189,335.96082798)(578.56236328,336.06083496)
\curveto(578.58236175,336.07082787)(578.61236172,336.09582785)(578.65236328,336.13583496)
\curveto(578.70236163,336.17582777)(578.74736158,336.21082773)(578.78736328,336.24083496)
\curveto(578.8273615,336.27082767)(578.87236146,336.30082764)(578.92236328,336.33083496)
\curveto(579.09236124,336.4408275)(579.27236106,336.52582742)(579.46236328,336.58583496)
\curveto(579.65236068,336.65582729)(579.84736048,336.72082722)(580.04736328,336.78083496)
\curveto(580.16736016,336.81082713)(580.29236004,336.83082711)(580.42236328,336.84083496)
\curveto(580.55235978,336.85082709)(580.68235965,336.87082707)(580.81236328,336.90083496)
\curveto(580.85235948,336.91082703)(580.91235942,336.91082703)(580.99236328,336.90083496)
\curveto(581.08235925,336.89082705)(581.13735919,336.89582705)(581.15736328,336.91583496)
\curveto(581.56735876,336.92582702)(581.95735837,336.91082703)(582.32736328,336.87083496)
\curveto(582.70735762,336.83082711)(583.04735728,336.75582719)(583.34736328,336.64583496)
\curveto(583.65735667,336.53582741)(583.92235641,336.38582756)(584.14236328,336.19583496)
\curveto(584.36235597,336.01582793)(584.5323558,335.78082816)(584.65236328,335.49083496)
\curveto(584.72235561,335.32082862)(584.76235557,335.12582882)(584.77236328,334.90583496)
\curveto(584.78235555,334.68582926)(584.78735554,334.46082948)(584.78736328,334.23083496)
\lineto(584.78736328,330.88583496)
\lineto(584.78736328,330.30083496)
\curveto(584.78735554,330.11083383)(584.80735552,329.93583401)(584.84736328,329.77583496)
\curveto(584.85735547,329.7458342)(584.86235547,329.71083423)(584.86236328,329.67083496)
\curveto(584.86235547,329.6408343)(584.86735546,329.61083433)(584.87736328,329.58083496)
\moveto(582.67236328,331.89083496)
\curveto(582.68235765,331.940832)(582.68735764,331.99583195)(582.68736328,332.05583496)
\curveto(582.68735764,332.12583182)(582.68235765,332.18583176)(582.67236328,332.23583496)
\curveto(582.65235768,332.29583165)(582.64235769,332.35083159)(582.64236328,332.40083496)
\curveto(582.64235769,332.45083149)(582.62235771,332.49083145)(582.58236328,332.52083496)
\curveto(582.5323578,332.56083138)(582.45735787,332.58083136)(582.35736328,332.58083496)
\curveto(582.31735801,332.57083137)(582.28235805,332.56083138)(582.25236328,332.55083496)
\curveto(582.22235811,332.55083139)(582.18735814,332.5458314)(582.14736328,332.53583496)
\curveto(582.07735825,332.51583143)(582.00235833,332.50083144)(581.92236328,332.49083496)
\curveto(581.84235849,332.48083146)(581.76235857,332.46583148)(581.68236328,332.44583496)
\curveto(581.65235868,332.43583151)(581.60735872,332.43083151)(581.54736328,332.43083496)
\curveto(581.41735891,332.40083154)(581.28735904,332.38083156)(581.15736328,332.37083496)
\curveto(581.0273593,332.36083158)(580.90235943,332.33583161)(580.78236328,332.29583496)
\curveto(580.70235963,332.27583167)(580.6273597,332.25583169)(580.55736328,332.23583496)
\curveto(580.48735984,332.22583172)(580.41735991,332.20583174)(580.34736328,332.17583496)
\curveto(580.13736019,332.08583186)(579.95736037,331.95083199)(579.80736328,331.77083496)
\curveto(579.66736066,331.59083235)(579.61736071,331.3408326)(579.65736328,331.02083496)
\curveto(579.67736065,330.85083309)(579.7323606,330.71083323)(579.82236328,330.60083496)
\curveto(579.89236044,330.49083345)(579.99736033,330.40083354)(580.13736328,330.33083496)
\curveto(580.27736005,330.27083367)(580.4273599,330.22583372)(580.58736328,330.19583496)
\curveto(580.75735957,330.16583378)(580.9323594,330.15583379)(581.11236328,330.16583496)
\curveto(581.30235903,330.18583376)(581.47735885,330.22083372)(581.63736328,330.27083496)
\curveto(581.89735843,330.35083359)(582.10235823,330.47583347)(582.25236328,330.64583496)
\curveto(582.40235793,330.82583312)(582.51735781,331.0458329)(582.59736328,331.30583496)
\curveto(582.61735771,331.37583257)(582.6273577,331.4458325)(582.62736328,331.51583496)
\curveto(582.63735769,331.59583235)(582.65235768,331.67583227)(582.67236328,331.75583496)
\lineto(582.67236328,331.89083496)
}
}
{
\newrgbcolor{curcolor}{0 0 0}
\pscustom[linestyle=none,fillstyle=solid,fillcolor=curcolor]
{
\newpath
\moveto(594.03064453,329.83583496)
\lineto(594.03064453,329.41583496)
\curveto(594.03063616,329.28583466)(594.00063619,329.18083476)(593.94064453,329.10083496)
\curveto(593.8906363,329.05083489)(593.82563637,329.01583493)(593.74564453,328.99583496)
\curveto(593.66563653,328.98583496)(593.57563662,328.98083496)(593.47564453,328.98083496)
\lineto(592.65064453,328.98083496)
\lineto(592.36564453,328.98083496)
\curveto(592.28563791,328.99083495)(592.22063797,329.01583493)(592.17064453,329.05583496)
\curveto(592.10063809,329.10583484)(592.06063813,329.17083477)(592.05064453,329.25083496)
\curveto(592.04063815,329.33083461)(592.02063817,329.41083453)(591.99064453,329.49083496)
\curveto(591.97063822,329.51083443)(591.95063824,329.52583442)(591.93064453,329.53583496)
\curveto(591.92063827,329.55583439)(591.90563829,329.57583437)(591.88564453,329.59583496)
\curveto(591.77563842,329.59583435)(591.6956385,329.57083437)(591.64564453,329.52083496)
\lineto(591.49564453,329.37083496)
\curveto(591.42563877,329.32083462)(591.36063883,329.27583467)(591.30064453,329.23583496)
\curveto(591.24063895,329.20583474)(591.17563902,329.16583478)(591.10564453,329.11583496)
\curveto(591.06563913,329.09583485)(591.02063917,329.07583487)(590.97064453,329.05583496)
\curveto(590.93063926,329.03583491)(590.88563931,329.01583493)(590.83564453,328.99583496)
\curveto(590.6956395,328.945835)(590.54563965,328.90083504)(590.38564453,328.86083496)
\curveto(590.33563986,328.8408351)(590.2906399,328.83083511)(590.25064453,328.83083496)
\curveto(590.21063998,328.83083511)(590.17064002,328.82583512)(590.13064453,328.81583496)
\lineto(589.99564453,328.81583496)
\curveto(589.96564023,328.80583514)(589.92564027,328.80083514)(589.87564453,328.80083496)
\lineto(589.74064453,328.80083496)
\curveto(589.68064051,328.78083516)(589.5906406,328.77583517)(589.47064453,328.78583496)
\curveto(589.35064084,328.78583516)(589.26564093,328.79583515)(589.21564453,328.81583496)
\curveto(589.14564105,328.83583511)(589.08064111,328.8458351)(589.02064453,328.84583496)
\curveto(588.97064122,328.83583511)(588.91564128,328.8408351)(588.85564453,328.86083496)
\lineto(588.49564453,328.98083496)
\curveto(588.38564181,329.01083493)(588.27564192,329.05083489)(588.16564453,329.10083496)
\curveto(587.81564238,329.25083469)(587.50064269,329.48083446)(587.22064453,329.79083496)
\curveto(586.95064324,330.11083383)(586.73564346,330.4458335)(586.57564453,330.79583496)
\curveto(586.52564367,330.90583304)(586.48564371,331.01083293)(586.45564453,331.11083496)
\curveto(586.42564377,331.22083272)(586.3906438,331.33083261)(586.35064453,331.44083496)
\curveto(586.34064385,331.48083246)(586.33564386,331.51583243)(586.33564453,331.54583496)
\curveto(586.33564386,331.58583236)(586.32564387,331.63083231)(586.30564453,331.68083496)
\curveto(586.28564391,331.76083218)(586.26564393,331.8458321)(586.24564453,331.93583496)
\curveto(586.23564396,332.03583191)(586.22064397,332.13583181)(586.20064453,332.23583496)
\curveto(586.190644,332.26583168)(586.18564401,332.30083164)(586.18564453,332.34083496)
\curveto(586.195644,332.38083156)(586.195644,332.41583153)(586.18564453,332.44583496)
\lineto(586.18564453,332.58083496)
\curveto(586.18564401,332.63083131)(586.18064401,332.68083126)(586.17064453,332.73083496)
\curveto(586.16064403,332.78083116)(586.15564404,332.83583111)(586.15564453,332.89583496)
\curveto(586.15564404,332.96583098)(586.16064403,333.02083092)(586.17064453,333.06083496)
\curveto(586.18064401,333.11083083)(586.18564401,333.15583079)(586.18564453,333.19583496)
\lineto(586.18564453,333.34583496)
\curveto(586.195644,333.39583055)(586.195644,333.4408305)(586.18564453,333.48083496)
\curveto(586.18564401,333.53083041)(586.195644,333.58083036)(586.21564453,333.63083496)
\curveto(586.23564396,333.7408302)(586.25064394,333.8458301)(586.26064453,333.94583496)
\curveto(586.28064391,334.0458299)(586.30564389,334.1458298)(586.33564453,334.24583496)
\curveto(586.37564382,334.36582958)(586.41064378,334.48082946)(586.44064453,334.59083496)
\curveto(586.47064372,334.70082924)(586.51064368,334.81082913)(586.56064453,334.92083496)
\curveto(586.70064349,335.22082872)(586.87564332,335.50582844)(587.08564453,335.77583496)
\curveto(587.10564309,335.80582814)(587.13064306,335.83082811)(587.16064453,335.85083496)
\curveto(587.20064299,335.88082806)(587.23064296,335.91082803)(587.25064453,335.94083496)
\curveto(587.2906429,335.99082795)(587.33064286,336.03582791)(587.37064453,336.07583496)
\curveto(587.41064278,336.11582783)(587.45564274,336.15582779)(587.50564453,336.19583496)
\curveto(587.54564265,336.21582773)(587.58064261,336.2408277)(587.61064453,336.27083496)
\curveto(587.64064255,336.31082763)(587.67564252,336.3408276)(587.71564453,336.36083496)
\curveto(587.96564223,336.53082741)(588.25564194,336.67082727)(588.58564453,336.78083496)
\curveto(588.65564154,336.80082714)(588.72564147,336.81582713)(588.79564453,336.82583496)
\curveto(588.87564132,336.83582711)(588.95564124,336.85082709)(589.03564453,336.87083496)
\curveto(589.10564109,336.89082705)(589.195641,336.90082704)(589.30564453,336.90083496)
\curveto(589.41564078,336.91082703)(589.52564067,336.91582703)(589.63564453,336.91583496)
\curveto(589.74564045,336.91582703)(589.85064034,336.91082703)(589.95064453,336.90083496)
\curveto(590.06064013,336.89082705)(590.15064004,336.87582707)(590.22064453,336.85583496)
\curveto(590.37063982,336.80582714)(590.51563968,336.76082718)(590.65564453,336.72083496)
\curveto(590.7956394,336.68082726)(590.92563927,336.62582732)(591.04564453,336.55583496)
\curveto(591.11563908,336.50582744)(591.18063901,336.45582749)(591.24064453,336.40583496)
\curveto(591.30063889,336.36582758)(591.36563883,336.32082762)(591.43564453,336.27083496)
\curveto(591.47563872,336.2408277)(591.53063866,336.20082774)(591.60064453,336.15083496)
\curveto(591.68063851,336.10082784)(591.75563844,336.10082784)(591.82564453,336.15083496)
\curveto(591.86563833,336.17082777)(591.88563831,336.20582774)(591.88564453,336.25583496)
\curveto(591.88563831,336.30582764)(591.8956383,336.35582759)(591.91564453,336.40583496)
\lineto(591.91564453,336.55583496)
\curveto(591.92563827,336.58582736)(591.93063826,336.62082732)(591.93064453,336.66083496)
\lineto(591.93064453,336.78083496)
\lineto(591.93064453,338.82083496)
\curveto(591.93063826,338.93082501)(591.92563827,339.05082489)(591.91564453,339.18083496)
\curveto(591.91563828,339.32082462)(591.94063825,339.42582452)(591.99064453,339.49583496)
\curveto(592.03063816,339.57582437)(592.10563809,339.62582432)(592.21564453,339.64583496)
\curveto(592.23563796,339.65582429)(592.25563794,339.65582429)(592.27564453,339.64583496)
\curveto(592.2956379,339.6458243)(592.31563788,339.65082429)(592.33564453,339.66083496)
\lineto(593.40064453,339.66083496)
\curveto(593.52063667,339.66082428)(593.63063656,339.65582429)(593.73064453,339.64583496)
\curveto(593.83063636,339.63582431)(593.90563629,339.59582435)(593.95564453,339.52583496)
\curveto(594.00563619,339.4458245)(594.03063616,339.3408246)(594.03064453,339.21083496)
\lineto(594.03064453,338.85083496)
\lineto(594.03064453,329.83583496)
\moveto(591.99064453,332.77583496)
\curveto(592.00063819,332.81583113)(592.00063819,332.85583109)(591.99064453,332.89583496)
\lineto(591.99064453,333.03083496)
\curveto(591.9906382,333.13083081)(591.98563821,333.23083071)(591.97564453,333.33083496)
\curveto(591.96563823,333.43083051)(591.95063824,333.52083042)(591.93064453,333.60083496)
\curveto(591.91063828,333.71083023)(591.8906383,333.81083013)(591.87064453,333.90083496)
\curveto(591.86063833,333.99082995)(591.83563836,334.07582987)(591.79564453,334.15583496)
\curveto(591.65563854,334.51582943)(591.45063874,334.80082914)(591.18064453,335.01083496)
\curveto(590.92063927,335.22082872)(590.54063965,335.32582862)(590.04064453,335.32583496)
\curveto(589.98064021,335.32582862)(589.90064029,335.31582863)(589.80064453,335.29583496)
\curveto(589.72064047,335.27582867)(589.64564055,335.25582869)(589.57564453,335.23583496)
\curveto(589.51564068,335.22582872)(589.45564074,335.20582874)(589.39564453,335.17583496)
\curveto(589.12564107,335.06582888)(588.91564128,334.89582905)(588.76564453,334.66583496)
\curveto(588.61564158,334.43582951)(588.4956417,334.17582977)(588.40564453,333.88583496)
\curveto(588.37564182,333.78583016)(588.35564184,333.68583026)(588.34564453,333.58583496)
\curveto(588.33564186,333.48583046)(588.31564188,333.38083056)(588.28564453,333.27083496)
\lineto(588.28564453,333.06083496)
\curveto(588.26564193,332.97083097)(588.26064193,332.8458311)(588.27064453,332.68583496)
\curveto(588.28064191,332.53583141)(588.2956419,332.42583152)(588.31564453,332.35583496)
\lineto(588.31564453,332.26583496)
\curveto(588.32564187,332.2458317)(588.33064186,332.22583172)(588.33064453,332.20583496)
\curveto(588.35064184,332.12583182)(588.36564183,332.05083189)(588.37564453,331.98083496)
\curveto(588.3956418,331.91083203)(588.41564178,331.83583211)(588.43564453,331.75583496)
\curveto(588.60564159,331.23583271)(588.8956413,330.85083309)(589.30564453,330.60083496)
\curveto(589.43564076,330.51083343)(589.61564058,330.4408335)(589.84564453,330.39083496)
\curveto(589.88564031,330.38083356)(589.94564025,330.37583357)(590.02564453,330.37583496)
\curveto(590.05564014,330.36583358)(590.10064009,330.35583359)(590.16064453,330.34583496)
\curveto(590.23063996,330.3458336)(590.28563991,330.35083359)(590.32564453,330.36083496)
\curveto(590.40563979,330.38083356)(590.48563971,330.39583355)(590.56564453,330.40583496)
\curveto(590.64563955,330.41583353)(590.72563947,330.43583351)(590.80564453,330.46583496)
\curveto(591.05563914,330.57583337)(591.25563894,330.71583323)(591.40564453,330.88583496)
\curveto(591.55563864,331.05583289)(591.68563851,331.27083267)(591.79564453,331.53083496)
\curveto(591.83563836,331.62083232)(591.86563833,331.71083223)(591.88564453,331.80083496)
\curveto(591.90563829,331.90083204)(591.92563827,332.00583194)(591.94564453,332.11583496)
\curveto(591.95563824,332.16583178)(591.95563824,332.21083173)(591.94564453,332.25083496)
\curveto(591.94563825,332.30083164)(591.95563824,332.35083159)(591.97564453,332.40083496)
\curveto(591.98563821,332.43083151)(591.9906382,332.46583148)(591.99064453,332.50583496)
\lineto(591.99064453,332.64083496)
\lineto(591.99064453,332.77583496)
}
}
{
\newrgbcolor{curcolor}{0.90196079 0.90196079 0.90196079}
\pscustom[linestyle=none,fillstyle=solid,fillcolor=curcolor]
{
\newpath
\moveto(599.15717909,301.68518495)
\curveto(612.47015071,301.68518469)(625.72140314,299.87641932)(638.54736417,296.30851561)
\lineto(599.15717621,154.70799684)
\closepath
}
}
{
\newrgbcolor{curcolor}{0.50196081 0.50196081 0.50196081}
\pscustom[linestyle=none,fillstyle=solid,fillcolor=curcolor]
{
\newpath
\moveto(638.45663264,296.33372384)
\curveto(677.44036293,285.51620189)(710.24131158,259.10799063)(729.13058132,223.33166469)
\lineto(599.15717621,154.70799684)
\closepath
}
}
{
\newrgbcolor{curcolor}{0.40000001 0.40000001 0.40000001}
\pscustom[linestyle=none,fillstyle=solid,fillcolor=curcolor]
{
\newpath
\moveto(729.0168312,223.5466766)
\curveto(767.03538423,151.82716953)(739.71536304,62.86689488)(667.99585597,24.84834185)
\curveto(596.27634891,-13.17021118)(507.31607426,14.14981001)(469.29752122,85.86931707)
\curveto(431.27896819,157.58882414)(458.59898938,246.54909879)(530.31849645,284.56765182)
\curveto(551.52301961,295.80818207)(575.15752306,301.68517804)(599.15713395,301.68518495)
\lineto(599.15717621,154.70799684)
\closepath
}
}
{
\newrgbcolor{curcolor}{0 0 0}
\pscustom[linestyle=none,fillstyle=solid,fillcolor=curcolor]
{
\newpath
\moveto(262.26290161,620.26215454)
\curveto(262.29289389,620.14215033)(262.31789386,620.00215047)(262.33790161,619.84215454)
\curveto(262.35789382,619.68215079)(262.36789381,619.51715095)(262.36790161,619.34715454)
\curveto(262.36789381,619.17715129)(262.35789382,619.01215146)(262.33790161,618.85215454)
\curveto(262.31789386,618.69215178)(262.29289389,618.55215192)(262.26290161,618.43215454)
\curveto(262.22289396,618.29215218)(262.18789399,618.1671523)(262.15790161,618.05715454)
\curveto(262.12789405,617.94715252)(262.08789409,617.83715263)(262.03790161,617.72715454)
\curveto(261.76789441,617.08715338)(261.35289483,616.60215387)(260.79290161,616.27215454)
\curveto(260.71289547,616.21215426)(260.62789555,616.16215431)(260.53790161,616.12215454)
\curveto(260.44789573,616.09215438)(260.34789583,616.05715441)(260.23790161,616.01715454)
\curveto(260.12789605,615.9671545)(260.00789617,615.93215454)(259.87790161,615.91215454)
\curveto(259.75789642,615.88215459)(259.62789655,615.85215462)(259.48790161,615.82215454)
\curveto(259.42789675,615.80215467)(259.36789681,615.79715467)(259.30790161,615.80715454)
\curveto(259.25789692,615.81715465)(259.19789698,615.81215466)(259.12790161,615.79215454)
\curveto(259.10789707,615.78215469)(259.0828971,615.78215469)(259.05290161,615.79215454)
\curveto(259.02289716,615.79215468)(258.99789718,615.78715468)(258.97790161,615.77715454)
\lineto(258.82790161,615.77715454)
\curveto(258.75789742,615.7671547)(258.70789747,615.7671547)(258.67790161,615.77715454)
\curveto(258.63789754,615.78715468)(258.59289759,615.79215468)(258.54290161,615.79215454)
\curveto(258.50289768,615.78215469)(258.46289772,615.78215469)(258.42290161,615.79215454)
\curveto(258.33289785,615.81215466)(258.24289794,615.82715464)(258.15290161,615.83715454)
\curveto(258.06289812,615.83715463)(257.97289821,615.84715462)(257.88290161,615.86715454)
\curveto(257.79289839,615.89715457)(257.70289848,615.92215455)(257.61290161,615.94215454)
\curveto(257.52289866,615.96215451)(257.43789874,615.99215448)(257.35790161,616.03215454)
\curveto(257.11789906,616.14215433)(256.89289929,616.2721542)(256.68290161,616.42215454)
\curveto(256.47289971,616.58215389)(256.29289989,616.76215371)(256.14290161,616.96215454)
\curveto(256.02290016,617.13215334)(255.91790026,617.30715316)(255.82790161,617.48715454)
\curveto(255.73790044,617.6671528)(255.64790053,617.85715261)(255.55790161,618.05715454)
\curveto(255.51790066,618.15715231)(255.4829007,618.25715221)(255.45290161,618.35715454)
\curveto(255.43290075,618.467152)(255.40790077,618.57715189)(255.37790161,618.68715454)
\curveto(255.33790084,618.82715164)(255.31290087,618.9671515)(255.30290161,619.10715454)
\curveto(255.29290089,619.24715122)(255.27290091,619.38715108)(255.24290161,619.52715454)
\curveto(255.23290095,619.63715083)(255.22290096,619.73715073)(255.21290161,619.82715454)
\curveto(255.21290097,619.92715054)(255.20290098,620.02715044)(255.18290161,620.12715454)
\lineto(255.18290161,620.21715454)
\curveto(255.19290099,620.24715022)(255.19290099,620.2721502)(255.18290161,620.29215454)
\lineto(255.18290161,620.50215454)
\curveto(255.16290102,620.56214991)(255.15290103,620.62714984)(255.15290161,620.69715454)
\curveto(255.16290102,620.77714969)(255.16790101,620.85214962)(255.16790161,620.92215454)
\lineto(255.16790161,621.07215454)
\curveto(255.16790101,621.12214935)(255.17290101,621.1721493)(255.18290161,621.22215454)
\lineto(255.18290161,621.59715454)
\curveto(255.19290099,621.62714884)(255.19290099,621.66214881)(255.18290161,621.70215454)
\curveto(255.182901,621.74214873)(255.18790099,621.78214869)(255.19790161,621.82215454)
\curveto(255.21790096,621.93214854)(255.23290095,622.04214843)(255.24290161,622.15215454)
\curveto(255.25290093,622.2721482)(255.26290092,622.38714808)(255.27290161,622.49715454)
\curveto(255.31290087,622.64714782)(255.33790084,622.79214768)(255.34790161,622.93215454)
\curveto(255.36790081,623.08214739)(255.39790078,623.22714724)(255.43790161,623.36715454)
\curveto(255.52790065,623.6671468)(255.62290056,623.95214652)(255.72290161,624.22215454)
\curveto(255.82290036,624.49214598)(255.94790023,624.74214573)(256.09790161,624.97215454)
\curveto(256.29789988,625.29214518)(256.54289964,625.5721449)(256.83290161,625.81215454)
\curveto(257.12289906,626.05214442)(257.46289872,626.23714423)(257.85290161,626.36715454)
\curveto(257.96289822,626.40714406)(258.07289811,626.43214404)(258.18290161,626.44215454)
\curveto(258.30289788,626.46214401)(258.42289776,626.48714398)(258.54290161,626.51715454)
\curveto(258.61289757,626.52714394)(258.6778975,626.53214394)(258.73790161,626.53215454)
\curveto(258.79789738,626.53214394)(258.86289732,626.53714393)(258.93290161,626.54715454)
\curveto(259.63289655,626.5671439)(260.20789597,626.45214402)(260.65790161,626.20215454)
\curveto(261.10789507,625.95214452)(261.45289473,625.60214487)(261.69290161,625.15215454)
\curveto(261.80289438,624.92214555)(261.90289428,624.64714582)(261.99290161,624.32715454)
\curveto(262.01289417,624.25714621)(262.01289417,624.18214629)(261.99290161,624.10215454)
\curveto(261.9828942,624.03214644)(261.95789422,623.98214649)(261.91790161,623.95215454)
\curveto(261.88789429,623.92214655)(261.82789435,623.89714657)(261.73790161,623.87715454)
\curveto(261.64789453,623.8671466)(261.54789463,623.85714661)(261.43790161,623.84715454)
\curveto(261.33789484,623.84714662)(261.23789494,623.85214662)(261.13790161,623.86215454)
\curveto(261.04789513,623.8721466)(260.9828952,623.89214658)(260.94290161,623.92215454)
\curveto(260.83289535,623.99214648)(260.75289543,624.10214637)(260.70290161,624.25215454)
\curveto(260.66289552,624.40214607)(260.60789557,624.53214594)(260.53790161,624.64215454)
\curveto(260.34789583,624.95214552)(260.06789611,625.18214529)(259.69790161,625.33215454)
\curveto(259.62789655,625.36214511)(259.55289663,625.38214509)(259.47290161,625.39215454)
\curveto(259.40289678,625.40214507)(259.32789685,625.41714505)(259.24790161,625.43715454)
\curveto(259.19789698,625.44714502)(259.12789705,625.45214502)(259.03790161,625.45215454)
\curveto(258.95789722,625.45214502)(258.89289729,625.44714502)(258.84290161,625.43715454)
\curveto(258.80289738,625.41714505)(258.76789741,625.41214506)(258.73790161,625.42215454)
\curveto(258.70789747,625.43214504)(258.67289751,625.43214504)(258.63290161,625.42215454)
\lineto(258.39290161,625.36215454)
\curveto(258.32289786,625.34214513)(258.25289793,625.31714515)(258.18290161,625.28715454)
\curveto(257.80289838,625.12714534)(257.51289867,624.91714555)(257.31290161,624.65715454)
\curveto(257.12289906,624.39714607)(256.94789923,624.08214639)(256.78790161,623.71215454)
\curveto(256.75789942,623.63214684)(256.73289945,623.55214692)(256.71290161,623.47215454)
\curveto(256.70289948,623.39214708)(256.6828995,623.31214716)(256.65290161,623.23215454)
\curveto(256.62289956,623.12214735)(256.59789958,623.00714746)(256.57790161,622.88715454)
\curveto(256.56789961,622.7671477)(256.54789963,622.64714782)(256.51790161,622.52715454)
\curveto(256.49789968,622.47714799)(256.48789969,622.42714804)(256.48790161,622.37715454)
\curveto(256.49789968,622.32714814)(256.49289969,622.27714819)(256.47290161,622.22715454)
\curveto(256.46289972,622.1671483)(256.46289972,622.08714838)(256.47290161,621.98715454)
\curveto(256.4828997,621.89714857)(256.49789968,621.84214863)(256.51790161,621.82215454)
\curveto(256.53789964,621.78214869)(256.56789961,621.76214871)(256.60790161,621.76215454)
\curveto(256.65789952,621.76214871)(256.70289948,621.7721487)(256.74290161,621.79215454)
\curveto(256.81289937,621.83214864)(256.87289931,621.87714859)(256.92290161,621.92715454)
\curveto(256.97289921,621.97714849)(257.03289915,622.02714844)(257.10290161,622.07715454)
\lineto(257.16290161,622.13715454)
\curveto(257.19289899,622.1671483)(257.22289896,622.19214828)(257.25290161,622.21215454)
\curveto(257.4828987,622.3721481)(257.75789842,622.50714796)(258.07790161,622.61715454)
\curveto(258.14789803,622.63714783)(258.21789796,622.65214782)(258.28790161,622.66215454)
\curveto(258.35789782,622.6721478)(258.43289775,622.68714778)(258.51290161,622.70715454)
\curveto(258.55289763,622.70714776)(258.58789759,622.71214776)(258.61790161,622.72215454)
\curveto(258.64789753,622.73214774)(258.6828975,622.73214774)(258.72290161,622.72215454)
\curveto(258.77289741,622.72214775)(258.81289737,622.73214774)(258.84290161,622.75215454)
\lineto(259.00790161,622.75215454)
\lineto(259.09790161,622.75215454)
\curveto(259.14789703,622.76214771)(259.18789699,622.76214771)(259.21790161,622.75215454)
\curveto(259.26789691,622.74214773)(259.31789686,622.73714773)(259.36790161,622.73715454)
\curveto(259.42789675,622.74714772)(259.4828967,622.74714772)(259.53290161,622.73715454)
\curveto(259.64289654,622.70714776)(259.74789643,622.68714778)(259.84790161,622.67715454)
\curveto(259.95789622,622.6671478)(260.06289612,622.64214783)(260.16290161,622.60215454)
\curveto(260.5828956,622.46214801)(260.92789525,622.27714819)(261.19790161,622.04715454)
\curveto(261.46789471,621.82714864)(261.70789447,621.54214893)(261.91790161,621.19215454)
\curveto(261.99789418,621.05214942)(262.06289412,620.90214957)(262.11290161,620.74215454)
\curveto(262.16289402,620.59214988)(262.21289397,620.43215004)(262.26290161,620.26215454)
\moveto(261.01790161,618.95715454)
\curveto(261.02789515,619.00715146)(261.03289515,619.05215142)(261.03290161,619.09215454)
\lineto(261.03290161,619.24215454)
\curveto(261.03289515,619.55215092)(260.99289519,619.83715063)(260.91290161,620.09715454)
\curveto(260.89289529,620.15715031)(260.87289531,620.21215026)(260.85290161,620.26215454)
\curveto(260.84289534,620.32215015)(260.82789535,620.37715009)(260.80790161,620.42715454)
\curveto(260.58789559,620.91714955)(260.24289594,621.2671492)(259.77290161,621.47715454)
\curveto(259.69289649,621.50714896)(259.61289657,621.53214894)(259.53290161,621.55215454)
\lineto(259.29290161,621.61215454)
\curveto(259.21289697,621.63214884)(259.12289706,621.64214883)(259.02290161,621.64215454)
\lineto(258.70790161,621.64215454)
\curveto(258.68789749,621.62214885)(258.64789753,621.61214886)(258.58790161,621.61215454)
\curveto(258.53789764,621.62214885)(258.49289769,621.62214885)(258.45290161,621.61215454)
\lineto(258.21290161,621.55215454)
\curveto(258.14289804,621.54214893)(258.07289811,621.52214895)(258.00290161,621.49215454)
\curveto(257.40289878,621.23214924)(256.99789918,620.7671497)(256.78790161,620.09715454)
\curveto(256.75789942,620.01715045)(256.73789944,619.93715053)(256.72790161,619.85715454)
\curveto(256.71789946,619.77715069)(256.70289948,619.69215078)(256.68290161,619.60215454)
\lineto(256.68290161,619.45215454)
\curveto(256.67289951,619.41215106)(256.66789951,619.34215113)(256.66790161,619.24215454)
\curveto(256.66789951,619.01215146)(256.68789949,618.81715165)(256.72790161,618.65715454)
\curveto(256.74789943,618.58715188)(256.76289942,618.52215195)(256.77290161,618.46215454)
\curveto(256.7828994,618.40215207)(256.80289938,618.33715213)(256.83290161,618.26715454)
\curveto(256.94289924,617.98715248)(257.08789909,617.74215273)(257.26790161,617.53215454)
\curveto(257.44789873,617.33215314)(257.6828985,617.1721533)(257.97290161,617.05215454)
\lineto(258.21290161,616.96215454)
\lineto(258.45290161,616.90215454)
\curveto(258.50289768,616.88215359)(258.54289764,616.87715359)(258.57290161,616.88715454)
\curveto(258.61289757,616.89715357)(258.65789752,616.89215358)(258.70790161,616.87215454)
\curveto(258.73789744,616.86215361)(258.79289739,616.85715361)(258.87290161,616.85715454)
\curveto(258.95289723,616.85715361)(259.01289717,616.86215361)(259.05290161,616.87215454)
\curveto(259.16289702,616.89215358)(259.26789691,616.90715356)(259.36790161,616.91715454)
\curveto(259.46789671,616.92715354)(259.56289662,616.95715351)(259.65290161,617.00715454)
\curveto(260.182896,617.20715326)(260.57289561,617.58215289)(260.82290161,618.13215454)
\curveto(260.86289532,618.23215224)(260.89289529,618.33715213)(260.91290161,618.44715454)
\lineto(261.00290161,618.77715454)
\curveto(261.00289518,618.85715161)(261.00789517,618.91715155)(261.01790161,618.95715454)
}
}
{
\newrgbcolor{curcolor}{0 0 0}
\pscustom[linestyle=none,fillstyle=solid,fillcolor=curcolor]
{
\newpath
\moveto(264.56751099,617.57715454)
\lineto(264.86751099,617.57715454)
\curveto(264.97750893,617.58715288)(265.08250882,617.58715288)(265.18251099,617.57715454)
\curveto(265.29250861,617.57715289)(265.39250851,617.5671529)(265.48251099,617.54715454)
\curveto(265.57250833,617.53715293)(265.64250826,617.51215296)(265.69251099,617.47215454)
\curveto(265.71250819,617.45215302)(265.72750818,617.42215305)(265.73751099,617.38215454)
\curveto(265.75750815,617.34215313)(265.77750813,617.29715317)(265.79751099,617.24715454)
\lineto(265.79751099,617.17215454)
\curveto(265.8075081,617.12215335)(265.8075081,617.0671534)(265.79751099,617.00715454)
\lineto(265.79751099,616.85715454)
\lineto(265.79751099,616.37715454)
\curveto(265.79750811,616.20715426)(265.75750815,616.08715438)(265.67751099,616.01715454)
\curveto(265.6075083,615.9671545)(265.51750839,615.94215453)(265.40751099,615.94215454)
\lineto(265.07751099,615.94215454)
\lineto(264.62751099,615.94215454)
\curveto(264.47750943,615.94215453)(264.36250954,615.9721545)(264.28251099,616.03215454)
\curveto(264.24250966,616.06215441)(264.21250969,616.11215436)(264.19251099,616.18215454)
\curveto(264.17250973,616.26215421)(264.15750975,616.34715412)(264.14751099,616.43715454)
\lineto(264.14751099,616.72215454)
\curveto(264.15750975,616.82215365)(264.16250974,616.90715356)(264.16251099,616.97715454)
\lineto(264.16251099,617.17215454)
\curveto(264.16250974,617.23215324)(264.17250973,617.28715318)(264.19251099,617.33715454)
\curveto(264.23250967,617.44715302)(264.3025096,617.51715295)(264.40251099,617.54715454)
\curveto(264.43250947,617.54715292)(264.48750942,617.55715291)(264.56751099,617.57715454)
}
}
{
\newrgbcolor{curcolor}{0 0 0}
\pscustom[linestyle=none,fillstyle=solid,fillcolor=curcolor]
{
\newpath
\moveto(268.23266724,626.35215454)
\lineto(273.03266724,626.35215454)
\lineto(274.03766724,626.35215454)
\curveto(274.17766014,626.35214412)(274.29766002,626.34214413)(274.39766724,626.32215454)
\curveto(274.50765981,626.31214416)(274.58765973,626.2671442)(274.63766724,626.18715454)
\curveto(274.65765966,626.14714432)(274.66765965,626.09714437)(274.66766724,626.03715454)
\curveto(274.67765964,625.97714449)(274.68265963,625.91214456)(274.68266724,625.84215454)
\lineto(274.68266724,625.57215454)
\curveto(274.68265963,625.48214499)(274.67265964,625.40214507)(274.65266724,625.33215454)
\curveto(274.6126597,625.25214522)(274.56765975,625.18214529)(274.51766724,625.12215454)
\lineto(274.36766724,624.94215454)
\curveto(274.33765998,624.89214558)(274.30266001,624.85214562)(274.26266724,624.82215454)
\curveto(274.22266009,624.79214568)(274.18266013,624.75214572)(274.14266724,624.70215454)
\curveto(274.06266025,624.59214588)(273.97766034,624.48214599)(273.88766724,624.37215454)
\curveto(273.79766052,624.2721462)(273.7126606,624.1671463)(273.63266724,624.05715454)
\curveto(273.49266082,623.85714661)(273.35266096,623.64714682)(273.21266724,623.42715454)
\curveto(273.07266124,623.21714725)(272.93266138,623.00214747)(272.79266724,622.78215454)
\curveto(272.74266157,622.69214778)(272.69266162,622.59714787)(272.64266724,622.49715454)
\curveto(272.59266172,622.39714807)(272.53766178,622.30214817)(272.47766724,622.21215454)
\curveto(272.45766186,622.19214828)(272.44766187,622.1671483)(272.44766724,622.13715454)
\curveto(272.44766187,622.10714836)(272.43766188,622.08214839)(272.41766724,622.06215454)
\curveto(272.34766197,621.96214851)(272.28266203,621.84714862)(272.22266724,621.71715454)
\curveto(272.16266215,621.59714887)(272.10766221,621.48214899)(272.05766724,621.37215454)
\curveto(271.95766236,621.14214933)(271.86266245,620.90714956)(271.77266724,620.66715454)
\curveto(271.68266263,620.42715004)(271.58266273,620.18715028)(271.47266724,619.94715454)
\curveto(271.45266286,619.89715057)(271.43766288,619.85215062)(271.42766724,619.81215454)
\curveto(271.42766289,619.7721507)(271.4176629,619.72715074)(271.39766724,619.67715454)
\curveto(271.34766297,619.55715091)(271.30266301,619.43215104)(271.26266724,619.30215454)
\curveto(271.23266308,619.18215129)(271.19766312,619.06215141)(271.15766724,618.94215454)
\curveto(271.07766324,618.71215176)(271.0126633,618.472152)(270.96266724,618.22215454)
\curveto(270.92266339,617.98215249)(270.87266344,617.74215273)(270.81266724,617.50215454)
\curveto(270.77266354,617.35215312)(270.74766357,617.20215327)(270.73766724,617.05215454)
\curveto(270.72766359,616.90215357)(270.70766361,616.75215372)(270.67766724,616.60215454)
\curveto(270.66766365,616.56215391)(270.66266365,616.50215397)(270.66266724,616.42215454)
\curveto(270.63266368,616.30215417)(270.60266371,616.20215427)(270.57266724,616.12215454)
\curveto(270.54266377,616.04215443)(270.47266384,615.98715448)(270.36266724,615.95715454)
\curveto(270.312664,615.93715453)(270.25766406,615.92715454)(270.19766724,615.92715454)
\lineto(270.00266724,615.92715454)
\curveto(269.86266445,615.92715454)(269.72266459,615.93215454)(269.58266724,615.94215454)
\curveto(269.45266486,615.95215452)(269.35766496,615.99715447)(269.29766724,616.07715454)
\curveto(269.25766506,616.13715433)(269.23766508,616.22215425)(269.23766724,616.33215454)
\curveto(269.24766507,616.44215403)(269.26266505,616.53715393)(269.28266724,616.61715454)
\lineto(269.28266724,616.69215454)
\curveto(269.29266502,616.72215375)(269.29766502,616.75215372)(269.29766724,616.78215454)
\curveto(269.317665,616.86215361)(269.32766499,616.93715353)(269.32766724,617.00715454)
\curveto(269.32766499,617.07715339)(269.33766498,617.14715332)(269.35766724,617.21715454)
\curveto(269.40766491,617.40715306)(269.44766487,617.59215288)(269.47766724,617.77215454)
\curveto(269.50766481,617.96215251)(269.54766477,618.14215233)(269.59766724,618.31215454)
\curveto(269.6176647,618.36215211)(269.62766469,618.40215207)(269.62766724,618.43215454)
\curveto(269.62766469,618.46215201)(269.63266468,618.49715197)(269.64266724,618.53715454)
\curveto(269.74266457,618.83715163)(269.83266448,619.13215134)(269.91266724,619.42215454)
\curveto(270.00266431,619.71215076)(270.10766421,619.99215048)(270.22766724,620.26215454)
\curveto(270.48766383,620.84214963)(270.75766356,621.39214908)(271.03766724,621.91215454)
\curveto(271.317663,622.44214803)(271.62766269,622.94714752)(271.96766724,623.42715454)
\curveto(272.10766221,623.62714684)(272.25766206,623.81714665)(272.41766724,623.99715454)
\curveto(272.57766174,624.18714628)(272.72766159,624.37714609)(272.86766724,624.56715454)
\curveto(272.90766141,624.61714585)(272.94266137,624.66214581)(272.97266724,624.70215454)
\curveto(273.0126613,624.75214572)(273.04766127,624.80214567)(273.07766724,624.85215454)
\curveto(273.08766123,624.8721456)(273.09766122,624.89714557)(273.10766724,624.92715454)
\curveto(273.12766119,624.95714551)(273.12766119,624.98714548)(273.10766724,625.01715454)
\curveto(273.08766123,625.07714539)(273.05266126,625.11214536)(273.00266724,625.12215454)
\curveto(272.95266136,625.14214533)(272.90266141,625.16214531)(272.85266724,625.18215454)
\lineto(272.74766724,625.18215454)
\curveto(272.70766161,625.19214528)(272.65766166,625.19214528)(272.59766724,625.18215454)
\lineto(272.44766724,625.18215454)
\lineto(271.84766724,625.18215454)
\lineto(269.20766724,625.18215454)
\lineto(268.47266724,625.18215454)
\lineto(268.23266724,625.18215454)
\curveto(268.16266615,625.19214528)(268.10266621,625.20714526)(268.05266724,625.22715454)
\curveto(267.96266635,625.2671452)(267.90266641,625.32714514)(267.87266724,625.40715454)
\curveto(267.82266649,625.50714496)(267.80766651,625.65214482)(267.82766724,625.84215454)
\curveto(267.84766647,626.04214443)(267.88266643,626.17714429)(267.93266724,626.24715454)
\curveto(267.95266636,626.2671442)(267.97766634,626.28214419)(268.00766724,626.29215454)
\lineto(268.12766724,626.35215454)
\curveto(268.14766617,626.35214412)(268.16266615,626.34714412)(268.17266724,626.33715454)
\curveto(268.19266612,626.33714413)(268.2126661,626.34214413)(268.23266724,626.35215454)
}
}
{
\newrgbcolor{curcolor}{0 0 0}
\pscustom[linestyle=none,fillstyle=solid,fillcolor=curcolor]
{
\newpath
\moveto(285.92727661,624.46215454)
\curveto(285.72726631,624.1721463)(285.51726652,623.88714658)(285.29727661,623.60715454)
\curveto(285.08726695,623.32714714)(284.88226716,623.04214743)(284.68227661,622.75215454)
\curveto(284.08226796,621.90214857)(283.47726856,621.06214941)(282.86727661,620.23215454)
\curveto(282.25726978,619.41215106)(281.65227039,618.57715189)(281.05227661,617.72715454)
\lineto(280.54227661,617.00715454)
\lineto(280.03227661,616.31715454)
\curveto(279.95227209,616.20715426)(279.87227217,616.09215438)(279.79227661,615.97215454)
\curveto(279.71227233,615.85215462)(279.61727242,615.75715471)(279.50727661,615.68715454)
\curveto(279.46727257,615.6671548)(279.40227264,615.65215482)(279.31227661,615.64215454)
\curveto(279.23227281,615.62215485)(279.1422729,615.61215486)(279.04227661,615.61215454)
\curveto(278.9422731,615.61215486)(278.84727319,615.61715485)(278.75727661,615.62715454)
\curveto(278.67727336,615.63715483)(278.61727342,615.65715481)(278.57727661,615.68715454)
\curveto(278.54727349,615.70715476)(278.52227352,615.74215473)(278.50227661,615.79215454)
\curveto(278.49227355,615.83215464)(278.49727354,615.87715459)(278.51727661,615.92715454)
\curveto(278.55727348,616.00715446)(278.60227344,616.08215439)(278.65227661,616.15215454)
\curveto(278.71227333,616.23215424)(278.76727327,616.31215416)(278.81727661,616.39215454)
\curveto(279.05727298,616.73215374)(279.30227274,617.0671534)(279.55227661,617.39715454)
\curveto(279.80227224,617.72715274)(280.042272,618.06215241)(280.27227661,618.40215454)
\curveto(280.43227161,618.62215185)(280.59227145,618.83715163)(280.75227661,619.04715454)
\curveto(280.91227113,619.25715121)(281.07227097,619.472151)(281.23227661,619.69215454)
\curveto(281.59227045,620.21215026)(281.95727008,620.72214975)(282.32727661,621.22215454)
\curveto(282.69726934,621.72214875)(283.06726897,622.23214824)(283.43727661,622.75215454)
\curveto(283.57726846,622.95214752)(283.71726832,623.14714732)(283.85727661,623.33715454)
\curveto(284.00726803,623.52714694)(284.15226789,623.72214675)(284.29227661,623.92215454)
\curveto(284.50226754,624.22214625)(284.71726732,624.52214595)(284.93727661,624.82215454)
\lineto(285.59727661,625.72215454)
\lineto(285.77727661,625.99215454)
\lineto(285.98727661,626.26215454)
\lineto(286.10727661,626.44215454)
\curveto(286.15726588,626.50214397)(286.20726583,626.55714391)(286.25727661,626.60715454)
\curveto(286.32726571,626.65714381)(286.40226564,626.69214378)(286.48227661,626.71215454)
\curveto(286.50226554,626.72214375)(286.52726551,626.72214375)(286.55727661,626.71215454)
\curveto(286.59726544,626.71214376)(286.62726541,626.72214375)(286.64727661,626.74215454)
\curveto(286.76726527,626.74214373)(286.90226514,626.73714373)(287.05227661,626.72715454)
\curveto(287.20226484,626.72714374)(287.29226475,626.68214379)(287.32227661,626.59215454)
\curveto(287.3422647,626.56214391)(287.34726469,626.52714394)(287.33727661,626.48715454)
\curveto(287.32726471,626.44714402)(287.31226473,626.41714405)(287.29227661,626.39715454)
\curveto(287.25226479,626.31714415)(287.21226483,626.24714422)(287.17227661,626.18715454)
\curveto(287.13226491,626.12714434)(287.08726495,626.0671444)(287.03727661,626.00715454)
\lineto(286.46727661,625.22715454)
\curveto(286.28726575,624.97714549)(286.10726593,624.72214575)(285.92727661,624.46215454)
\moveto(279.07227661,620.56215454)
\curveto(279.02227302,620.58214989)(278.97227307,620.58714988)(278.92227661,620.57715454)
\curveto(278.87227317,620.5671499)(278.82227322,620.5721499)(278.77227661,620.59215454)
\curveto(278.66227338,620.61214986)(278.55727348,620.63214984)(278.45727661,620.65215454)
\curveto(278.36727367,620.68214979)(278.27227377,620.72214975)(278.17227661,620.77215454)
\curveto(277.8422742,620.91214956)(277.58727445,621.10714936)(277.40727661,621.35715454)
\curveto(277.22727481,621.61714885)(277.08227496,621.92714854)(276.97227661,622.28715454)
\curveto(276.9422751,622.3671481)(276.92227512,622.44714802)(276.91227661,622.52715454)
\curveto(276.90227514,622.61714785)(276.88727515,622.70214777)(276.86727661,622.78215454)
\curveto(276.85727518,622.83214764)(276.85227519,622.89714757)(276.85227661,622.97715454)
\curveto(276.8422752,623.00714746)(276.8372752,623.03714743)(276.83727661,623.06715454)
\curveto(276.8372752,623.10714736)(276.83227521,623.14214733)(276.82227661,623.17215454)
\lineto(276.82227661,623.32215454)
\curveto(276.81227523,623.3721471)(276.80727523,623.43214704)(276.80727661,623.50215454)
\curveto(276.80727523,623.58214689)(276.81227523,623.64714682)(276.82227661,623.69715454)
\lineto(276.82227661,623.86215454)
\curveto(276.8422752,623.91214656)(276.84727519,623.95714651)(276.83727661,623.99715454)
\curveto(276.8372752,624.04714642)(276.8422752,624.09214638)(276.85227661,624.13215454)
\curveto(276.86227518,624.1721463)(276.86727517,624.20714626)(276.86727661,624.23715454)
\curveto(276.86727517,624.27714619)(276.87227517,624.31714615)(276.88227661,624.35715454)
\curveto(276.91227513,624.467146)(276.93227511,624.57714589)(276.94227661,624.68715454)
\curveto(276.96227508,624.80714566)(276.99727504,624.92214555)(277.04727661,625.03215454)
\curveto(277.18727485,625.3721451)(277.34727469,625.64714482)(277.52727661,625.85715454)
\curveto(277.71727432,626.07714439)(277.98727405,626.25714421)(278.33727661,626.39715454)
\curveto(278.41727362,626.42714404)(278.50227354,626.44714402)(278.59227661,626.45715454)
\curveto(278.68227336,626.47714399)(278.77727326,626.49714397)(278.87727661,626.51715454)
\curveto(278.90727313,626.52714394)(278.96227308,626.52714394)(279.04227661,626.51715454)
\curveto(279.12227292,626.51714395)(279.17227287,626.52714394)(279.19227661,626.54715454)
\curveto(279.75227229,626.55714391)(280.20227184,626.44714402)(280.54227661,626.21715454)
\curveto(280.89227115,625.98714448)(281.15227089,625.68214479)(281.32227661,625.30215454)
\curveto(281.36227068,625.21214526)(281.39727064,625.11714535)(281.42727661,625.01715454)
\curveto(281.45727058,624.91714555)(281.48227056,624.81714565)(281.50227661,624.71715454)
\curveto(281.52227052,624.68714578)(281.52727051,624.65714581)(281.51727661,624.62715454)
\curveto(281.51727052,624.59714587)(281.52227052,624.5671459)(281.53227661,624.53715454)
\curveto(281.56227048,624.42714604)(281.58227046,624.30214617)(281.59227661,624.16215454)
\curveto(281.60227044,624.03214644)(281.61227043,623.89714657)(281.62227661,623.75715454)
\lineto(281.62227661,623.59215454)
\curveto(281.63227041,623.53214694)(281.63227041,623.47714699)(281.62227661,623.42715454)
\curveto(281.61227043,623.37714709)(281.60727043,623.32714714)(281.60727661,623.27715454)
\lineto(281.60727661,623.14215454)
\curveto(281.59727044,623.10214737)(281.59227045,623.06214741)(281.59227661,623.02215454)
\curveto(281.60227044,622.98214749)(281.59727044,622.93714753)(281.57727661,622.88715454)
\curveto(281.55727048,622.77714769)(281.5372705,622.6721478)(281.51727661,622.57215454)
\curveto(281.50727053,622.472148)(281.48727055,622.3721481)(281.45727661,622.27215454)
\curveto(281.32727071,621.91214856)(281.16227088,621.59714887)(280.96227661,621.32715454)
\curveto(280.76227128,621.05714941)(280.48727155,620.85214962)(280.13727661,620.71215454)
\curveto(280.05727198,620.68214979)(279.97227207,620.65714981)(279.88227661,620.63715454)
\lineto(279.61227661,620.57715454)
\curveto(279.56227248,620.5671499)(279.51727252,620.56214991)(279.47727661,620.56215454)
\curveto(279.4372726,620.5721499)(279.39727264,620.5721499)(279.35727661,620.56215454)
\curveto(279.25727278,620.54214993)(279.16227288,620.54214993)(279.07227661,620.56215454)
\moveto(278.23227661,621.95715454)
\curveto(278.27227377,621.88714858)(278.31227373,621.82214865)(278.35227661,621.76215454)
\curveto(278.39227365,621.71214876)(278.4422736,621.66214881)(278.50227661,621.61215454)
\lineto(278.65227661,621.49215454)
\curveto(278.71227333,621.46214901)(278.77727326,621.43714903)(278.84727661,621.41715454)
\curveto(278.88727315,621.39714907)(278.92227312,621.38714908)(278.95227661,621.38715454)
\curveto(278.99227305,621.39714907)(279.03227301,621.39214908)(279.07227661,621.37215454)
\curveto(279.10227294,621.3721491)(279.1422729,621.3671491)(279.19227661,621.35715454)
\curveto(279.2422728,621.35714911)(279.28227276,621.36214911)(279.31227661,621.37215454)
\lineto(279.53727661,621.41715454)
\curveto(279.78727225,621.49714897)(279.97227207,621.62214885)(280.09227661,621.79215454)
\curveto(280.17227187,621.89214858)(280.2422718,622.02214845)(280.30227661,622.18215454)
\curveto(280.38227166,622.36214811)(280.4422716,622.58714788)(280.48227661,622.85715454)
\curveto(280.52227152,623.13714733)(280.5372715,623.41714705)(280.52727661,623.69715454)
\curveto(280.51727152,623.98714648)(280.48727155,624.26214621)(280.43727661,624.52215454)
\curveto(280.38727165,624.78214569)(280.31227173,624.99214548)(280.21227661,625.15215454)
\curveto(280.09227195,625.35214512)(279.9422721,625.50214497)(279.76227661,625.60215454)
\curveto(279.68227236,625.65214482)(279.59227245,625.68214479)(279.49227661,625.69215454)
\curveto(279.39227265,625.71214476)(279.28727275,625.72214475)(279.17727661,625.72215454)
\curveto(279.15727288,625.71214476)(279.13227291,625.70714476)(279.10227661,625.70715454)
\curveto(279.08227296,625.71714475)(279.06227298,625.71714475)(279.04227661,625.70715454)
\curveto(278.99227305,625.69714477)(278.94727309,625.68714478)(278.90727661,625.67715454)
\curveto(278.86727317,625.67714479)(278.82727321,625.6671448)(278.78727661,625.64715454)
\curveto(278.60727343,625.5671449)(278.45727358,625.44714502)(278.33727661,625.28715454)
\curveto(278.22727381,625.12714534)(278.1372739,624.94714552)(278.06727661,624.74715454)
\curveto(278.00727403,624.55714591)(277.96227408,624.33214614)(277.93227661,624.07215454)
\curveto(277.91227413,623.81214666)(277.90727413,623.54714692)(277.91727661,623.27715454)
\curveto(277.92727411,623.01714745)(277.95727408,622.7671477)(278.00727661,622.52715454)
\curveto(278.06727397,622.29714817)(278.1422739,622.10714836)(278.23227661,621.95715454)
\moveto(289.03227661,618.97215454)
\curveto(289.042263,618.92215155)(289.04726299,618.83215164)(289.04727661,618.70215454)
\curveto(289.04726299,618.5721519)(289.037263,618.48215199)(289.01727661,618.43215454)
\curveto(288.99726304,618.38215209)(288.99226305,618.32715214)(289.00227661,618.26715454)
\curveto(289.01226303,618.21715225)(289.01226303,618.1671523)(289.00227661,618.11715454)
\curveto(288.96226308,617.97715249)(288.93226311,617.84215263)(288.91227661,617.71215454)
\curveto(288.90226314,617.58215289)(288.87226317,617.46215301)(288.82227661,617.35215454)
\curveto(288.68226336,617.00215347)(288.51726352,616.70715376)(288.32727661,616.46715454)
\curveto(288.1372639,616.23715423)(287.86726417,616.05215442)(287.51727661,615.91215454)
\curveto(287.4372646,615.88215459)(287.35226469,615.86215461)(287.26227661,615.85215454)
\curveto(287.17226487,615.83215464)(287.08726495,615.81215466)(287.00727661,615.79215454)
\curveto(286.95726508,615.78215469)(286.90726513,615.77715469)(286.85727661,615.77715454)
\curveto(286.80726523,615.77715469)(286.75726528,615.7721547)(286.70727661,615.76215454)
\curveto(286.67726536,615.75215472)(286.62726541,615.75215472)(286.55727661,615.76215454)
\curveto(286.48726555,615.76215471)(286.4372656,615.7671547)(286.40727661,615.77715454)
\curveto(286.34726569,615.79715467)(286.28726575,615.80715466)(286.22727661,615.80715454)
\curveto(286.17726586,615.79715467)(286.12726591,615.80215467)(286.07727661,615.82215454)
\curveto(285.98726605,615.84215463)(285.89726614,615.8671546)(285.80727661,615.89715454)
\curveto(285.72726631,615.91715455)(285.64726639,615.94715452)(285.56727661,615.98715454)
\curveto(285.24726679,616.12715434)(284.99726704,616.32215415)(284.81727661,616.57215454)
\curveto(284.6372674,616.83215364)(284.48726755,617.13715333)(284.36727661,617.48715454)
\curveto(284.34726769,617.5671529)(284.33226771,617.65215282)(284.32227661,617.74215454)
\curveto(284.31226773,617.83215264)(284.29726774,617.91715255)(284.27727661,617.99715454)
\curveto(284.26726777,618.02715244)(284.26226778,618.05715241)(284.26227661,618.08715454)
\lineto(284.26227661,618.19215454)
\curveto(284.2422678,618.2721522)(284.23226781,618.35215212)(284.23227661,618.43215454)
\lineto(284.23227661,618.56715454)
\curveto(284.21226783,618.6671518)(284.21226783,618.7671517)(284.23227661,618.86715454)
\lineto(284.23227661,619.04715454)
\curveto(284.2422678,619.09715137)(284.24726779,619.14215133)(284.24727661,619.18215454)
\curveto(284.24726779,619.23215124)(284.25226779,619.27715119)(284.26227661,619.31715454)
\curveto(284.27226777,619.35715111)(284.27726776,619.39215108)(284.27727661,619.42215454)
\curveto(284.27726776,619.46215101)(284.28226776,619.50215097)(284.29227661,619.54215454)
\lineto(284.35227661,619.87215454)
\curveto(284.37226767,619.99215048)(284.40226764,620.10215037)(284.44227661,620.20215454)
\curveto(284.58226746,620.53214994)(284.7422673,620.80714966)(284.92227661,621.02715454)
\curveto(285.11226693,621.25714921)(285.37226667,621.44214903)(285.70227661,621.58215454)
\curveto(285.78226626,621.62214885)(285.86726617,621.64714882)(285.95727661,621.65715454)
\lineto(286.25727661,621.71715454)
\lineto(286.39227661,621.71715454)
\curveto(286.4422656,621.72714874)(286.49226555,621.73214874)(286.54227661,621.73215454)
\curveto(287.11226493,621.75214872)(287.57226447,621.64714882)(287.92227661,621.41715454)
\curveto(288.28226376,621.19714927)(288.54726349,620.89714957)(288.71727661,620.51715454)
\curveto(288.76726327,620.41715005)(288.80726323,620.31715015)(288.83727661,620.21715454)
\curveto(288.86726317,620.11715035)(288.89726314,620.01215046)(288.92727661,619.90215454)
\curveto(288.9372631,619.86215061)(288.9422631,619.82715064)(288.94227661,619.79715454)
\curveto(288.9422631,619.77715069)(288.94726309,619.74715072)(288.95727661,619.70715454)
\curveto(288.97726306,619.63715083)(288.98726305,619.56215091)(288.98727661,619.48215454)
\curveto(288.98726305,619.40215107)(288.99726304,619.32215115)(289.01727661,619.24215454)
\curveto(289.01726302,619.19215128)(289.01726302,619.14715132)(289.01727661,619.10715454)
\curveto(289.01726302,619.0671514)(289.02226302,619.02215145)(289.03227661,618.97215454)
\moveto(287.92227661,618.53715454)
\curveto(287.93226411,618.58715188)(287.9372641,618.66215181)(287.93727661,618.76215454)
\curveto(287.94726409,618.86215161)(287.9422641,618.93715153)(287.92227661,618.98715454)
\curveto(287.90226414,619.04715142)(287.89726414,619.10215137)(287.90727661,619.15215454)
\curveto(287.92726411,619.21215126)(287.92726411,619.2721512)(287.90727661,619.33215454)
\curveto(287.89726414,619.36215111)(287.89226415,619.39715107)(287.89227661,619.43715454)
\curveto(287.89226415,619.47715099)(287.88726415,619.51715095)(287.87727661,619.55715454)
\curveto(287.85726418,619.63715083)(287.8372642,619.71215076)(287.81727661,619.78215454)
\curveto(287.80726423,619.86215061)(287.79226425,619.94215053)(287.77227661,620.02215454)
\curveto(287.7422643,620.08215039)(287.71726432,620.14215033)(287.69727661,620.20215454)
\curveto(287.67726436,620.26215021)(287.64726439,620.32215015)(287.60727661,620.38215454)
\curveto(287.50726453,620.55214992)(287.37726466,620.68714978)(287.21727661,620.78715454)
\curveto(287.1372649,620.83714963)(287.042265,620.8721496)(286.93227661,620.89215454)
\curveto(286.82226522,620.91214956)(286.69726534,620.92214955)(286.55727661,620.92215454)
\curveto(286.5372655,620.91214956)(286.51226553,620.90714956)(286.48227661,620.90715454)
\curveto(286.45226559,620.91714955)(286.42226562,620.91714955)(286.39227661,620.90715454)
\lineto(286.24227661,620.84715454)
\curveto(286.19226585,620.83714963)(286.14726589,620.82214965)(286.10727661,620.80215454)
\curveto(285.91726612,620.69214978)(285.77226627,620.54714992)(285.67227661,620.36715454)
\curveto(285.58226646,620.18715028)(285.50226654,619.98215049)(285.43227661,619.75215454)
\curveto(285.39226665,619.62215085)(285.37226667,619.48715098)(285.37227661,619.34715454)
\curveto(285.37226667,619.21715125)(285.36226668,619.0721514)(285.34227661,618.91215454)
\curveto(285.33226671,618.86215161)(285.32226672,618.80215167)(285.31227661,618.73215454)
\curveto(285.31226673,618.66215181)(285.32226672,618.60215187)(285.34227661,618.55215454)
\lineto(285.34227661,618.38715454)
\lineto(285.34227661,618.20715454)
\curveto(285.35226669,618.15715231)(285.36226668,618.10215237)(285.37227661,618.04215454)
\curveto(285.38226666,617.99215248)(285.38726665,617.93715253)(285.38727661,617.87715454)
\curveto(285.39726664,617.81715265)(285.41226663,617.76215271)(285.43227661,617.71215454)
\curveto(285.48226656,617.52215295)(285.5422665,617.34715312)(285.61227661,617.18715454)
\curveto(285.68226636,617.02715344)(285.78726625,616.89715357)(285.92727661,616.79715454)
\curveto(286.05726598,616.69715377)(286.19726584,616.62715384)(286.34727661,616.58715454)
\curveto(286.37726566,616.57715389)(286.40226564,616.5721539)(286.42227661,616.57215454)
\curveto(286.45226559,616.58215389)(286.48226556,616.58215389)(286.51227661,616.57215454)
\curveto(286.53226551,616.5721539)(286.56226548,616.5671539)(286.60227661,616.55715454)
\curveto(286.6422654,616.55715391)(286.67726536,616.56215391)(286.70727661,616.57215454)
\curveto(286.74726529,616.58215389)(286.78726525,616.58715388)(286.82727661,616.58715454)
\curveto(286.86726517,616.58715388)(286.90726513,616.59715387)(286.94727661,616.61715454)
\curveto(287.18726485,616.69715377)(287.38226466,616.83215364)(287.53227661,617.02215454)
\curveto(287.65226439,617.20215327)(287.7422643,617.40715306)(287.80227661,617.63715454)
\curveto(287.82226422,617.70715276)(287.8372642,617.77715269)(287.84727661,617.84715454)
\curveto(287.85726418,617.92715254)(287.87226417,618.00715246)(287.89227661,618.08715454)
\curveto(287.89226415,618.14715232)(287.89726414,618.19215228)(287.90727661,618.22215454)
\curveto(287.90726413,618.24215223)(287.90726413,618.2671522)(287.90727661,618.29715454)
\curveto(287.90726413,618.33715213)(287.91226413,618.3671521)(287.92227661,618.38715454)
\lineto(287.92227661,618.53715454)
}
}
{
\newrgbcolor{curcolor}{0 0 0}
\pscustom[linestyle=none,fillstyle=solid,fillcolor=curcolor]
{
\newpath
\moveto(314.94145996,496.54712402)
\curveto(315.63145533,496.55711339)(316.23145473,496.43711351)(316.74145996,496.18712402)
\curveto(317.2614537,495.93711401)(317.6564533,495.60211435)(317.92645996,495.18212402)
\curveto(317.97645298,495.10211485)(318.02145294,495.01211494)(318.06145996,494.91212402)
\curveto(318.10145286,494.82211513)(318.14645281,494.72711522)(318.19645996,494.62712402)
\curveto(318.23645272,494.52711542)(318.26645269,494.42711552)(318.28645996,494.32712402)
\curveto(318.30645265,494.22711572)(318.32645263,494.12211583)(318.34645996,494.01212402)
\curveto(318.36645259,493.96211599)(318.37145259,493.91711603)(318.36145996,493.87712402)
\curveto(318.35145261,493.83711611)(318.3564526,493.79211616)(318.37645996,493.74212402)
\curveto(318.38645257,493.69211626)(318.39145257,493.60711634)(318.39145996,493.48712402)
\curveto(318.39145257,493.37711657)(318.38645257,493.29211666)(318.37645996,493.23212402)
\curveto(318.3564526,493.17211678)(318.34645261,493.11211684)(318.34645996,493.05212402)
\curveto(318.3564526,492.99211696)(318.35145261,492.93211702)(318.33145996,492.87212402)
\curveto(318.29145267,492.73211722)(318.2564527,492.59711735)(318.22645996,492.46712402)
\curveto(318.19645276,492.33711761)(318.1564528,492.21211774)(318.10645996,492.09212402)
\curveto(318.04645291,491.952118)(317.97645298,491.82711812)(317.89645996,491.71712402)
\curveto(317.82645313,491.60711834)(317.75145321,491.49711845)(317.67145996,491.38712402)
\lineto(317.61145996,491.32712402)
\curveto(317.60145336,491.30711864)(317.58645337,491.28711866)(317.56645996,491.26712402)
\curveto(317.44645351,491.10711884)(317.31145365,490.96211899)(317.16145996,490.83212402)
\curveto(317.01145395,490.70211925)(316.85145411,490.57711937)(316.68145996,490.45712402)
\curveto(316.37145459,490.23711971)(316.07645488,490.03211992)(315.79645996,489.84212402)
\curveto(315.56645539,489.70212025)(315.33645562,489.56712038)(315.10645996,489.43712402)
\curveto(314.88645607,489.30712064)(314.66645629,489.17212078)(314.44645996,489.03212402)
\curveto(314.19645676,488.86212109)(313.956457,488.68212127)(313.72645996,488.49212402)
\curveto(313.50645745,488.30212165)(313.31645764,488.07712187)(313.15645996,487.81712402)
\curveto(313.11645784,487.75712219)(313.08145788,487.69712225)(313.05145996,487.63712402)
\curveto(313.02145794,487.58712236)(312.99145797,487.52212243)(312.96145996,487.44212402)
\curveto(312.94145802,487.37212258)(312.93645802,487.31212264)(312.94645996,487.26212402)
\curveto(312.96645799,487.19212276)(313.00145796,487.13712281)(313.05145996,487.09712402)
\curveto(313.10145786,487.06712288)(313.1614578,487.0471229)(313.23145996,487.03712402)
\lineto(313.47145996,487.03712402)
\lineto(314.22145996,487.03712402)
\lineto(317.02645996,487.03712402)
\lineto(317.68645996,487.03712402)
\curveto(317.77645318,487.03712291)(317.8614531,487.03212292)(317.94145996,487.02212402)
\curveto(318.02145294,487.02212293)(318.08645287,487.00212295)(318.13645996,486.96212402)
\curveto(318.18645277,486.92212303)(318.22645273,486.8471231)(318.25645996,486.73712402)
\curveto(318.29645266,486.63712331)(318.30645265,486.53712341)(318.28645996,486.43712402)
\lineto(318.28645996,486.30212402)
\curveto(318.26645269,486.23212372)(318.24645271,486.17212378)(318.22645996,486.12212402)
\curveto(318.20645275,486.07212388)(318.17145279,486.03212392)(318.12145996,486.00212402)
\curveto(318.07145289,485.96212399)(318.00145296,485.94212401)(317.91145996,485.94212402)
\lineto(317.64145996,485.94212402)
\lineto(316.74145996,485.94212402)
\lineto(313.23145996,485.94212402)
\lineto(312.16645996,485.94212402)
\curveto(312.08645887,485.94212401)(311.99645896,485.93712401)(311.89645996,485.92712402)
\curveto(311.79645916,485.92712402)(311.71145925,485.93712401)(311.64145996,485.95712402)
\curveto(311.43145953,486.02712392)(311.36645959,486.20712374)(311.44645996,486.49712402)
\curveto(311.4564595,486.53712341)(311.4564595,486.57212338)(311.44645996,486.60212402)
\curveto(311.44645951,486.64212331)(311.4564595,486.68712326)(311.47645996,486.73712402)
\curveto(311.49645946,486.81712313)(311.51645944,486.90212305)(311.53645996,486.99212402)
\curveto(311.5564594,487.08212287)(311.58145938,487.16712278)(311.61145996,487.24712402)
\curveto(311.77145919,487.73712221)(311.97145899,488.1521218)(312.21145996,488.49212402)
\curveto(312.39145857,488.74212121)(312.59645836,488.96712098)(312.82645996,489.16712402)
\curveto(313.0564579,489.37712057)(313.29645766,489.57212038)(313.54645996,489.75212402)
\curveto(313.80645715,489.93212002)(314.07145689,490.10211985)(314.34145996,490.26212402)
\curveto(314.62145634,490.43211952)(314.89145607,490.60711934)(315.15145996,490.78712402)
\curveto(315.2614557,490.86711908)(315.36645559,490.94211901)(315.46645996,491.01212402)
\curveto(315.57645538,491.08211887)(315.68645527,491.15711879)(315.79645996,491.23712402)
\curveto(315.83645512,491.26711868)(315.87145509,491.29711865)(315.90145996,491.32712402)
\curveto(315.94145502,491.36711858)(315.98145498,491.39711855)(316.02145996,491.41712402)
\curveto(316.1614548,491.52711842)(316.28645467,491.6521183)(316.39645996,491.79212402)
\curveto(316.41645454,491.82211813)(316.44145452,491.8471181)(316.47145996,491.86712402)
\curveto(316.50145446,491.89711805)(316.52645443,491.92711802)(316.54645996,491.95712402)
\curveto(316.62645433,492.05711789)(316.69145427,492.15711779)(316.74145996,492.25712402)
\curveto(316.80145416,492.35711759)(316.8564541,492.46711748)(316.90645996,492.58712402)
\curveto(316.93645402,492.65711729)(316.956454,492.73211722)(316.96645996,492.81212402)
\lineto(317.02645996,493.05212402)
\lineto(317.02645996,493.14212402)
\curveto(317.03645392,493.17211678)(317.04145392,493.20211675)(317.04145996,493.23212402)
\curveto(317.0614539,493.30211665)(317.06645389,493.39711655)(317.05645996,493.51712402)
\curveto(317.0564539,493.6471163)(317.04645391,493.7471162)(317.02645996,493.81712402)
\curveto(317.00645395,493.89711605)(316.98645397,493.97211598)(316.96645996,494.04212402)
\curveto(316.956454,494.12211583)(316.93645402,494.20211575)(316.90645996,494.28212402)
\curveto(316.79645416,494.52211543)(316.64645431,494.72211523)(316.45645996,494.88212402)
\curveto(316.27645468,495.0521149)(316.0564549,495.19211476)(315.79645996,495.30212402)
\curveto(315.72645523,495.32211463)(315.6564553,495.33711461)(315.58645996,495.34712402)
\curveto(315.51645544,495.36711458)(315.44145552,495.38711456)(315.36145996,495.40712402)
\curveto(315.28145568,495.42711452)(315.17145579,495.43711451)(315.03145996,495.43712402)
\curveto(314.90145606,495.43711451)(314.79645616,495.42711452)(314.71645996,495.40712402)
\curveto(314.6564563,495.39711455)(314.60145636,495.39211456)(314.55145996,495.39212402)
\curveto(314.50145646,495.39211456)(314.45145651,495.38211457)(314.40145996,495.36212402)
\curveto(314.30145666,495.32211463)(314.20645675,495.28211467)(314.11645996,495.24212402)
\curveto(314.03645692,495.20211475)(313.956457,495.15711479)(313.87645996,495.10712402)
\curveto(313.84645711,495.08711486)(313.81645714,495.06211489)(313.78645996,495.03212402)
\curveto(313.76645719,495.00211495)(313.74145722,494.97711497)(313.71145996,494.95712402)
\lineto(313.63645996,494.88212402)
\curveto(313.60645735,494.86211509)(313.58145738,494.84211511)(313.56145996,494.82212402)
\lineto(313.41145996,494.61212402)
\curveto(313.37145759,494.5521154)(313.32645763,494.48711546)(313.27645996,494.41712402)
\curveto(313.21645774,494.32711562)(313.16645779,494.22211573)(313.12645996,494.10212402)
\curveto(313.09645786,493.99211596)(313.0614579,493.88211607)(313.02145996,493.77212402)
\curveto(312.98145798,493.66211629)(312.956458,493.51711643)(312.94645996,493.33712402)
\curveto(312.93645802,493.16711678)(312.90645805,493.04211691)(312.85645996,492.96212402)
\curveto(312.80645815,492.88211707)(312.73145823,492.83711711)(312.63145996,492.82712402)
\curveto(312.53145843,492.81711713)(312.42145854,492.81211714)(312.30145996,492.81212402)
\curveto(312.2614587,492.81211714)(312.22145874,492.80711714)(312.18145996,492.79712402)
\curveto(312.14145882,492.79711715)(312.10645885,492.80211715)(312.07645996,492.81212402)
\curveto(312.02645893,492.83211712)(311.97645898,492.84211711)(311.92645996,492.84212402)
\curveto(311.88645907,492.84211711)(311.84645911,492.8521171)(311.80645996,492.87212402)
\curveto(311.71645924,492.93211702)(311.67145929,493.06711688)(311.67145996,493.27712402)
\lineto(311.67145996,493.39712402)
\curveto(311.68145928,493.45711649)(311.68645927,493.51711643)(311.68645996,493.57712402)
\curveto(311.69645926,493.6471163)(311.70645925,493.71211624)(311.71645996,493.77212402)
\curveto(311.73645922,493.88211607)(311.7564592,493.98211597)(311.77645996,494.07212402)
\curveto(311.79645916,494.17211578)(311.82645913,494.26711568)(311.86645996,494.35712402)
\curveto(311.88645907,494.42711552)(311.90645905,494.48711546)(311.92645996,494.53712402)
\lineto(311.98645996,494.71712402)
\curveto(312.10645885,494.97711497)(312.2614587,495.22211473)(312.45145996,495.45212402)
\curveto(312.65145831,495.68211427)(312.86645809,495.86711408)(313.09645996,496.00712402)
\curveto(313.20645775,496.08711386)(313.32145764,496.1521138)(313.44145996,496.20212402)
\lineto(313.83145996,496.35212402)
\curveto(313.94145702,496.40211355)(314.0564569,496.43211352)(314.17645996,496.44212402)
\curveto(314.29645666,496.46211349)(314.42145654,496.48711346)(314.55145996,496.51712402)
\curveto(314.62145634,496.51711343)(314.68645627,496.51711343)(314.74645996,496.51712402)
\curveto(314.80645615,496.52711342)(314.87145609,496.53711341)(314.94145996,496.54712402)
}
}
{
\newrgbcolor{curcolor}{0 0 0}
\pscustom[linestyle=none,fillstyle=solid,fillcolor=curcolor]
{
\newpath
\moveto(327.04106934,490.26212402)
\curveto(327.07106161,490.14211981)(327.09606159,490.00211995)(327.11606934,489.84212402)
\curveto(327.13606155,489.68212027)(327.14606154,489.51712043)(327.14606934,489.34712402)
\curveto(327.14606154,489.17712077)(327.13606155,489.01212094)(327.11606934,488.85212402)
\curveto(327.09606159,488.69212126)(327.07106161,488.5521214)(327.04106934,488.43212402)
\curveto(327.00106168,488.29212166)(326.96606172,488.16712178)(326.93606934,488.05712402)
\curveto(326.90606178,487.947122)(326.86606182,487.83712211)(326.81606934,487.72712402)
\curveto(326.54606214,487.08712286)(326.13106255,486.60212335)(325.57106934,486.27212402)
\curveto(325.49106319,486.21212374)(325.40606328,486.16212379)(325.31606934,486.12212402)
\curveto(325.22606346,486.09212386)(325.12606356,486.05712389)(325.01606934,486.01712402)
\curveto(324.90606378,485.96712398)(324.7860639,485.93212402)(324.65606934,485.91212402)
\curveto(324.53606415,485.88212407)(324.40606428,485.8521241)(324.26606934,485.82212402)
\curveto(324.20606448,485.80212415)(324.14606454,485.79712415)(324.08606934,485.80712402)
\curveto(324.03606465,485.81712413)(323.97606471,485.81212414)(323.90606934,485.79212402)
\curveto(323.8860648,485.78212417)(323.86106482,485.78212417)(323.83106934,485.79212402)
\curveto(323.80106488,485.79212416)(323.77606491,485.78712416)(323.75606934,485.77712402)
\lineto(323.60606934,485.77712402)
\curveto(323.53606515,485.76712418)(323.4860652,485.76712418)(323.45606934,485.77712402)
\curveto(323.41606527,485.78712416)(323.37106531,485.79212416)(323.32106934,485.79212402)
\curveto(323.2810654,485.78212417)(323.24106544,485.78212417)(323.20106934,485.79212402)
\curveto(323.11106557,485.81212414)(323.02106566,485.82712412)(322.93106934,485.83712402)
\curveto(322.84106584,485.83712411)(322.75106593,485.8471241)(322.66106934,485.86712402)
\curveto(322.57106611,485.89712405)(322.4810662,485.92212403)(322.39106934,485.94212402)
\curveto(322.30106638,485.96212399)(322.21606647,485.99212396)(322.13606934,486.03212402)
\curveto(321.89606679,486.14212381)(321.67106701,486.27212368)(321.46106934,486.42212402)
\curveto(321.25106743,486.58212337)(321.07106761,486.76212319)(320.92106934,486.96212402)
\curveto(320.80106788,487.13212282)(320.69606799,487.30712264)(320.60606934,487.48712402)
\curveto(320.51606817,487.66712228)(320.42606826,487.85712209)(320.33606934,488.05712402)
\curveto(320.29606839,488.15712179)(320.26106842,488.25712169)(320.23106934,488.35712402)
\curveto(320.21106847,488.46712148)(320.1860685,488.57712137)(320.15606934,488.68712402)
\curveto(320.11606857,488.82712112)(320.09106859,488.96712098)(320.08106934,489.10712402)
\curveto(320.07106861,489.2471207)(320.05106863,489.38712056)(320.02106934,489.52712402)
\curveto(320.01106867,489.63712031)(320.00106868,489.73712021)(319.99106934,489.82712402)
\curveto(319.99106869,489.92712002)(319.9810687,490.02711992)(319.96106934,490.12712402)
\lineto(319.96106934,490.21712402)
\curveto(319.97106871,490.2471197)(319.97106871,490.27211968)(319.96106934,490.29212402)
\lineto(319.96106934,490.50212402)
\curveto(319.94106874,490.56211939)(319.93106875,490.62711932)(319.93106934,490.69712402)
\curveto(319.94106874,490.77711917)(319.94606874,490.8521191)(319.94606934,490.92212402)
\lineto(319.94606934,491.07212402)
\curveto(319.94606874,491.12211883)(319.95106873,491.17211878)(319.96106934,491.22212402)
\lineto(319.96106934,491.59712402)
\curveto(319.97106871,491.62711832)(319.97106871,491.66211829)(319.96106934,491.70212402)
\curveto(319.96106872,491.74211821)(319.96606872,491.78211817)(319.97606934,491.82212402)
\curveto(319.99606869,491.93211802)(320.01106867,492.04211791)(320.02106934,492.15212402)
\curveto(320.03106865,492.27211768)(320.04106864,492.38711756)(320.05106934,492.49712402)
\curveto(320.09106859,492.6471173)(320.11606857,492.79211716)(320.12606934,492.93212402)
\curveto(320.14606854,493.08211687)(320.17606851,493.22711672)(320.21606934,493.36712402)
\curveto(320.30606838,493.66711628)(320.40106828,493.952116)(320.50106934,494.22212402)
\curveto(320.60106808,494.49211546)(320.72606796,494.74211521)(320.87606934,494.97212402)
\curveto(321.07606761,495.29211466)(321.32106736,495.57211438)(321.61106934,495.81212402)
\curveto(321.90106678,496.0521139)(322.24106644,496.23711371)(322.63106934,496.36712402)
\curveto(322.74106594,496.40711354)(322.85106583,496.43211352)(322.96106934,496.44212402)
\curveto(323.0810656,496.46211349)(323.20106548,496.48711346)(323.32106934,496.51712402)
\curveto(323.39106529,496.52711342)(323.45606523,496.53211342)(323.51606934,496.53212402)
\curveto(323.57606511,496.53211342)(323.64106504,496.53711341)(323.71106934,496.54712402)
\curveto(324.41106427,496.56711338)(324.9860637,496.4521135)(325.43606934,496.20212402)
\curveto(325.8860628,495.952114)(326.23106245,495.60211435)(326.47106934,495.15212402)
\curveto(326.5810621,494.92211503)(326.681062,494.6471153)(326.77106934,494.32712402)
\curveto(326.79106189,494.25711569)(326.79106189,494.18211577)(326.77106934,494.10212402)
\curveto(326.76106192,494.03211592)(326.73606195,493.98211597)(326.69606934,493.95212402)
\curveto(326.66606202,493.92211603)(326.60606208,493.89711605)(326.51606934,493.87712402)
\curveto(326.42606226,493.86711608)(326.32606236,493.85711609)(326.21606934,493.84712402)
\curveto(326.11606257,493.8471161)(326.01606267,493.8521161)(325.91606934,493.86212402)
\curveto(325.82606286,493.87211608)(325.76106292,493.89211606)(325.72106934,493.92212402)
\curveto(325.61106307,493.99211596)(325.53106315,494.10211585)(325.48106934,494.25212402)
\curveto(325.44106324,494.40211555)(325.3860633,494.53211542)(325.31606934,494.64212402)
\curveto(325.12606356,494.952115)(324.84606384,495.18211477)(324.47606934,495.33212402)
\curveto(324.40606428,495.36211459)(324.33106435,495.38211457)(324.25106934,495.39212402)
\curveto(324.1810645,495.40211455)(324.10606458,495.41711453)(324.02606934,495.43712402)
\curveto(323.97606471,495.4471145)(323.90606478,495.4521145)(323.81606934,495.45212402)
\curveto(323.73606495,495.4521145)(323.67106501,495.4471145)(323.62106934,495.43712402)
\curveto(323.5810651,495.41711453)(323.54606514,495.41211454)(323.51606934,495.42212402)
\curveto(323.4860652,495.43211452)(323.45106523,495.43211452)(323.41106934,495.42212402)
\lineto(323.17106934,495.36212402)
\curveto(323.10106558,495.34211461)(323.03106565,495.31711463)(322.96106934,495.28712402)
\curveto(322.5810661,495.12711482)(322.29106639,494.91711503)(322.09106934,494.65712402)
\curveto(321.90106678,494.39711555)(321.72606696,494.08211587)(321.56606934,493.71212402)
\curveto(321.53606715,493.63211632)(321.51106717,493.5521164)(321.49106934,493.47212402)
\curveto(321.4810672,493.39211656)(321.46106722,493.31211664)(321.43106934,493.23212402)
\curveto(321.40106728,493.12211683)(321.37606731,493.00711694)(321.35606934,492.88712402)
\curveto(321.34606734,492.76711718)(321.32606736,492.6471173)(321.29606934,492.52712402)
\curveto(321.27606741,492.47711747)(321.26606742,492.42711752)(321.26606934,492.37712402)
\curveto(321.27606741,492.32711762)(321.27106741,492.27711767)(321.25106934,492.22712402)
\curveto(321.24106744,492.16711778)(321.24106744,492.08711786)(321.25106934,491.98712402)
\curveto(321.26106742,491.89711805)(321.27606741,491.84211811)(321.29606934,491.82212402)
\curveto(321.31606737,491.78211817)(321.34606734,491.76211819)(321.38606934,491.76212402)
\curveto(321.43606725,491.76211819)(321.4810672,491.77211818)(321.52106934,491.79212402)
\curveto(321.59106709,491.83211812)(321.65106703,491.87711807)(321.70106934,491.92712402)
\curveto(321.75106693,491.97711797)(321.81106687,492.02711792)(321.88106934,492.07712402)
\lineto(321.94106934,492.13712402)
\curveto(321.97106671,492.16711778)(322.00106668,492.19211776)(322.03106934,492.21212402)
\curveto(322.26106642,492.37211758)(322.53606615,492.50711744)(322.85606934,492.61712402)
\curveto(322.92606576,492.63711731)(322.99606569,492.6521173)(323.06606934,492.66212402)
\curveto(323.13606555,492.67211728)(323.21106547,492.68711726)(323.29106934,492.70712402)
\curveto(323.33106535,492.70711724)(323.36606532,492.71211724)(323.39606934,492.72212402)
\curveto(323.42606526,492.73211722)(323.46106522,492.73211722)(323.50106934,492.72212402)
\curveto(323.55106513,492.72211723)(323.59106509,492.73211722)(323.62106934,492.75212402)
\lineto(323.78606934,492.75212402)
\lineto(323.87606934,492.75212402)
\curveto(323.92606476,492.76211719)(323.96606472,492.76211719)(323.99606934,492.75212402)
\curveto(324.04606464,492.74211721)(324.09606459,492.73711721)(324.14606934,492.73712402)
\curveto(324.20606448,492.7471172)(324.26106442,492.7471172)(324.31106934,492.73712402)
\curveto(324.42106426,492.70711724)(324.52606416,492.68711726)(324.62606934,492.67712402)
\curveto(324.73606395,492.66711728)(324.84106384,492.64211731)(324.94106934,492.60212402)
\curveto(325.36106332,492.46211749)(325.70606298,492.27711767)(325.97606934,492.04712402)
\curveto(326.24606244,491.82711812)(326.4860622,491.54211841)(326.69606934,491.19212402)
\curveto(326.77606191,491.0521189)(326.84106184,490.90211905)(326.89106934,490.74212402)
\curveto(326.94106174,490.59211936)(326.99106169,490.43211952)(327.04106934,490.26212402)
\moveto(325.79606934,488.95712402)
\curveto(325.80606288,489.00712094)(325.81106287,489.0521209)(325.81106934,489.09212402)
\lineto(325.81106934,489.24212402)
\curveto(325.81106287,489.5521204)(325.77106291,489.83712011)(325.69106934,490.09712402)
\curveto(325.67106301,490.15711979)(325.65106303,490.21211974)(325.63106934,490.26212402)
\curveto(325.62106306,490.32211963)(325.60606308,490.37711957)(325.58606934,490.42712402)
\curveto(325.36606332,490.91711903)(325.02106366,491.26711868)(324.55106934,491.47712402)
\curveto(324.47106421,491.50711844)(324.39106429,491.53211842)(324.31106934,491.55212402)
\lineto(324.07106934,491.61212402)
\curveto(323.99106469,491.63211832)(323.90106478,491.64211831)(323.80106934,491.64212402)
\lineto(323.48606934,491.64212402)
\curveto(323.46606522,491.62211833)(323.42606526,491.61211834)(323.36606934,491.61212402)
\curveto(323.31606537,491.62211833)(323.27106541,491.62211833)(323.23106934,491.61212402)
\lineto(322.99106934,491.55212402)
\curveto(322.92106576,491.54211841)(322.85106583,491.52211843)(322.78106934,491.49212402)
\curveto(322.1810665,491.23211872)(321.77606691,490.76711918)(321.56606934,490.09712402)
\curveto(321.53606715,490.01711993)(321.51606717,489.93712001)(321.50606934,489.85712402)
\curveto(321.49606719,489.77712017)(321.4810672,489.69212026)(321.46106934,489.60212402)
\lineto(321.46106934,489.45212402)
\curveto(321.45106723,489.41212054)(321.44606724,489.34212061)(321.44606934,489.24212402)
\curveto(321.44606724,489.01212094)(321.46606722,488.81712113)(321.50606934,488.65712402)
\curveto(321.52606716,488.58712136)(321.54106714,488.52212143)(321.55106934,488.46212402)
\curveto(321.56106712,488.40212155)(321.5810671,488.33712161)(321.61106934,488.26712402)
\curveto(321.72106696,487.98712196)(321.86606682,487.74212221)(322.04606934,487.53212402)
\curveto(322.22606646,487.33212262)(322.46106622,487.17212278)(322.75106934,487.05212402)
\lineto(322.99106934,486.96212402)
\lineto(323.23106934,486.90212402)
\curveto(323.2810654,486.88212307)(323.32106536,486.87712307)(323.35106934,486.88712402)
\curveto(323.39106529,486.89712305)(323.43606525,486.89212306)(323.48606934,486.87212402)
\curveto(323.51606517,486.86212309)(323.57106511,486.85712309)(323.65106934,486.85712402)
\curveto(323.73106495,486.85712309)(323.79106489,486.86212309)(323.83106934,486.87212402)
\curveto(323.94106474,486.89212306)(324.04606464,486.90712304)(324.14606934,486.91712402)
\curveto(324.24606444,486.92712302)(324.34106434,486.95712299)(324.43106934,487.00712402)
\curveto(324.96106372,487.20712274)(325.35106333,487.58212237)(325.60106934,488.13212402)
\curveto(325.64106304,488.23212172)(325.67106301,488.33712161)(325.69106934,488.44712402)
\lineto(325.78106934,488.77712402)
\curveto(325.7810629,488.85712109)(325.7860629,488.91712103)(325.79606934,488.95712402)
}
}
{
\newrgbcolor{curcolor}{0 0 0}
\pscustom[linestyle=none,fillstyle=solid,fillcolor=curcolor]
{
\newpath
\moveto(338.19567871,494.46212402)
\curveto(337.99566841,494.17211578)(337.78566862,493.88711606)(337.56567871,493.60712402)
\curveto(337.35566905,493.32711662)(337.15066926,493.04211691)(336.95067871,492.75212402)
\curveto(336.35067006,491.90211805)(335.74567066,491.06211889)(335.13567871,490.23212402)
\curveto(334.52567188,489.41212054)(333.92067249,488.57712137)(333.32067871,487.72712402)
\lineto(332.81067871,487.00712402)
\lineto(332.30067871,486.31712402)
\curveto(332.22067419,486.20712374)(332.14067427,486.09212386)(332.06067871,485.97212402)
\curveto(331.98067443,485.8521241)(331.88567452,485.75712419)(331.77567871,485.68712402)
\curveto(331.73567467,485.66712428)(331.67067474,485.6521243)(331.58067871,485.64212402)
\curveto(331.50067491,485.62212433)(331.410675,485.61212434)(331.31067871,485.61212402)
\curveto(331.2106752,485.61212434)(331.11567529,485.61712433)(331.02567871,485.62712402)
\curveto(330.94567546,485.63712431)(330.88567552,485.65712429)(330.84567871,485.68712402)
\curveto(330.81567559,485.70712424)(330.79067562,485.74212421)(330.77067871,485.79212402)
\curveto(330.76067565,485.83212412)(330.76567564,485.87712407)(330.78567871,485.92712402)
\curveto(330.82567558,486.00712394)(330.87067554,486.08212387)(330.92067871,486.15212402)
\curveto(330.98067543,486.23212372)(331.03567537,486.31212364)(331.08567871,486.39212402)
\curveto(331.32567508,486.73212322)(331.57067484,487.06712288)(331.82067871,487.39712402)
\curveto(332.07067434,487.72712222)(332.3106741,488.06212189)(332.54067871,488.40212402)
\curveto(332.70067371,488.62212133)(332.86067355,488.83712111)(333.02067871,489.04712402)
\curveto(333.18067323,489.25712069)(333.34067307,489.47212048)(333.50067871,489.69212402)
\curveto(333.86067255,490.21211974)(334.22567218,490.72211923)(334.59567871,491.22212402)
\curveto(334.96567144,491.72211823)(335.33567107,492.23211772)(335.70567871,492.75212402)
\curveto(335.84567056,492.952117)(335.98567042,493.1471168)(336.12567871,493.33712402)
\curveto(336.27567013,493.52711642)(336.42066999,493.72211623)(336.56067871,493.92212402)
\curveto(336.77066964,494.22211573)(336.98566942,494.52211543)(337.20567871,494.82212402)
\lineto(337.86567871,495.72212402)
\lineto(338.04567871,495.99212402)
\lineto(338.25567871,496.26212402)
\lineto(338.37567871,496.44212402)
\curveto(338.42566798,496.50211345)(338.47566793,496.55711339)(338.52567871,496.60712402)
\curveto(338.59566781,496.65711329)(338.67066774,496.69211326)(338.75067871,496.71212402)
\curveto(338.77066764,496.72211323)(338.79566761,496.72211323)(338.82567871,496.71212402)
\curveto(338.86566754,496.71211324)(338.89566751,496.72211323)(338.91567871,496.74212402)
\curveto(339.03566737,496.74211321)(339.17066724,496.73711321)(339.32067871,496.72712402)
\curveto(339.47066694,496.72711322)(339.56066685,496.68211327)(339.59067871,496.59212402)
\curveto(339.6106668,496.56211339)(339.61566679,496.52711342)(339.60567871,496.48712402)
\curveto(339.59566681,496.4471135)(339.58066683,496.41711353)(339.56067871,496.39712402)
\curveto(339.52066689,496.31711363)(339.48066693,496.2471137)(339.44067871,496.18712402)
\curveto(339.40066701,496.12711382)(339.35566705,496.06711388)(339.30567871,496.00712402)
\lineto(338.73567871,495.22712402)
\curveto(338.55566785,494.97711497)(338.37566803,494.72211523)(338.19567871,494.46212402)
\moveto(331.34067871,490.56212402)
\curveto(331.29067512,490.58211937)(331.24067517,490.58711936)(331.19067871,490.57712402)
\curveto(331.14067527,490.56711938)(331.09067532,490.57211938)(331.04067871,490.59212402)
\curveto(330.93067548,490.61211934)(330.82567558,490.63211932)(330.72567871,490.65212402)
\curveto(330.63567577,490.68211927)(330.54067587,490.72211923)(330.44067871,490.77212402)
\curveto(330.1106763,490.91211904)(329.85567655,491.10711884)(329.67567871,491.35712402)
\curveto(329.49567691,491.61711833)(329.35067706,491.92711802)(329.24067871,492.28712402)
\curveto(329.2106772,492.36711758)(329.19067722,492.4471175)(329.18067871,492.52712402)
\curveto(329.17067724,492.61711733)(329.15567725,492.70211725)(329.13567871,492.78212402)
\curveto(329.12567728,492.83211712)(329.12067729,492.89711705)(329.12067871,492.97712402)
\curveto(329.1106773,493.00711694)(329.1056773,493.03711691)(329.10567871,493.06712402)
\curveto(329.1056773,493.10711684)(329.10067731,493.14211681)(329.09067871,493.17212402)
\lineto(329.09067871,493.32212402)
\curveto(329.08067733,493.37211658)(329.07567733,493.43211652)(329.07567871,493.50212402)
\curveto(329.07567733,493.58211637)(329.08067733,493.6471163)(329.09067871,493.69712402)
\lineto(329.09067871,493.86212402)
\curveto(329.1106773,493.91211604)(329.11567729,493.95711599)(329.10567871,493.99712402)
\curveto(329.1056773,494.0471159)(329.1106773,494.09211586)(329.12067871,494.13212402)
\curveto(329.13067728,494.17211578)(329.13567727,494.20711574)(329.13567871,494.23712402)
\curveto(329.13567727,494.27711567)(329.14067727,494.31711563)(329.15067871,494.35712402)
\curveto(329.18067723,494.46711548)(329.20067721,494.57711537)(329.21067871,494.68712402)
\curveto(329.23067718,494.80711514)(329.26567714,494.92211503)(329.31567871,495.03212402)
\curveto(329.45567695,495.37211458)(329.61567679,495.6471143)(329.79567871,495.85712402)
\curveto(329.98567642,496.07711387)(330.25567615,496.25711369)(330.60567871,496.39712402)
\curveto(330.68567572,496.42711352)(330.77067564,496.4471135)(330.86067871,496.45712402)
\curveto(330.95067546,496.47711347)(331.04567536,496.49711345)(331.14567871,496.51712402)
\curveto(331.17567523,496.52711342)(331.23067518,496.52711342)(331.31067871,496.51712402)
\curveto(331.39067502,496.51711343)(331.44067497,496.52711342)(331.46067871,496.54712402)
\curveto(332.02067439,496.55711339)(332.47067394,496.4471135)(332.81067871,496.21712402)
\curveto(333.16067325,495.98711396)(333.42067299,495.68211427)(333.59067871,495.30212402)
\curveto(333.63067278,495.21211474)(333.66567274,495.11711483)(333.69567871,495.01712402)
\curveto(333.72567268,494.91711503)(333.75067266,494.81711513)(333.77067871,494.71712402)
\curveto(333.79067262,494.68711526)(333.79567261,494.65711529)(333.78567871,494.62712402)
\curveto(333.78567262,494.59711535)(333.79067262,494.56711538)(333.80067871,494.53712402)
\curveto(333.83067258,494.42711552)(333.85067256,494.30211565)(333.86067871,494.16212402)
\curveto(333.87067254,494.03211592)(333.88067253,493.89711605)(333.89067871,493.75712402)
\lineto(333.89067871,493.59212402)
\curveto(333.90067251,493.53211642)(333.90067251,493.47711647)(333.89067871,493.42712402)
\curveto(333.88067253,493.37711657)(333.87567253,493.32711662)(333.87567871,493.27712402)
\lineto(333.87567871,493.14212402)
\curveto(333.86567254,493.10211685)(333.86067255,493.06211689)(333.86067871,493.02212402)
\curveto(333.87067254,492.98211697)(333.86567254,492.93711701)(333.84567871,492.88712402)
\curveto(333.82567258,492.77711717)(333.8056726,492.67211728)(333.78567871,492.57212402)
\curveto(333.77567263,492.47211748)(333.75567265,492.37211758)(333.72567871,492.27212402)
\curveto(333.59567281,491.91211804)(333.43067298,491.59711835)(333.23067871,491.32712402)
\curveto(333.03067338,491.05711889)(332.75567365,490.8521191)(332.40567871,490.71212402)
\curveto(332.32567408,490.68211927)(332.24067417,490.65711929)(332.15067871,490.63712402)
\lineto(331.88067871,490.57712402)
\curveto(331.83067458,490.56711938)(331.78567462,490.56211939)(331.74567871,490.56212402)
\curveto(331.7056747,490.57211938)(331.66567474,490.57211938)(331.62567871,490.56212402)
\curveto(331.52567488,490.54211941)(331.43067498,490.54211941)(331.34067871,490.56212402)
\moveto(330.50067871,491.95712402)
\curveto(330.54067587,491.88711806)(330.58067583,491.82211813)(330.62067871,491.76212402)
\curveto(330.66067575,491.71211824)(330.7106757,491.66211829)(330.77067871,491.61212402)
\lineto(330.92067871,491.49212402)
\curveto(330.98067543,491.46211849)(331.04567536,491.43711851)(331.11567871,491.41712402)
\curveto(331.15567525,491.39711855)(331.19067522,491.38711856)(331.22067871,491.38712402)
\curveto(331.26067515,491.39711855)(331.30067511,491.39211856)(331.34067871,491.37212402)
\curveto(331.37067504,491.37211858)(331.410675,491.36711858)(331.46067871,491.35712402)
\curveto(331.5106749,491.35711859)(331.55067486,491.36211859)(331.58067871,491.37212402)
\lineto(331.80567871,491.41712402)
\curveto(332.05567435,491.49711845)(332.24067417,491.62211833)(332.36067871,491.79212402)
\curveto(332.44067397,491.89211806)(332.5106739,492.02211793)(332.57067871,492.18212402)
\curveto(332.65067376,492.36211759)(332.7106737,492.58711736)(332.75067871,492.85712402)
\curveto(332.79067362,493.13711681)(332.8056736,493.41711653)(332.79567871,493.69712402)
\curveto(332.78567362,493.98711596)(332.75567365,494.26211569)(332.70567871,494.52212402)
\curveto(332.65567375,494.78211517)(332.58067383,494.99211496)(332.48067871,495.15212402)
\curveto(332.36067405,495.3521146)(332.2106742,495.50211445)(332.03067871,495.60212402)
\curveto(331.95067446,495.6521143)(331.86067455,495.68211427)(331.76067871,495.69212402)
\curveto(331.66067475,495.71211424)(331.55567485,495.72211423)(331.44567871,495.72212402)
\curveto(331.42567498,495.71211424)(331.40067501,495.70711424)(331.37067871,495.70712402)
\curveto(331.35067506,495.71711423)(331.33067508,495.71711423)(331.31067871,495.70712402)
\curveto(331.26067515,495.69711425)(331.21567519,495.68711426)(331.17567871,495.67712402)
\curveto(331.13567527,495.67711427)(331.09567531,495.66711428)(331.05567871,495.64712402)
\curveto(330.87567553,495.56711438)(330.72567568,495.4471145)(330.60567871,495.28712402)
\curveto(330.49567591,495.12711482)(330.405676,494.947115)(330.33567871,494.74712402)
\curveto(330.27567613,494.55711539)(330.23067618,494.33211562)(330.20067871,494.07212402)
\curveto(330.18067623,493.81211614)(330.17567623,493.5471164)(330.18567871,493.27712402)
\curveto(330.19567621,493.01711693)(330.22567618,492.76711718)(330.27567871,492.52712402)
\curveto(330.33567607,492.29711765)(330.410676,492.10711784)(330.50067871,491.95712402)
\moveto(341.30067871,488.97212402)
\curveto(341.3106651,488.92212103)(341.31566509,488.83212112)(341.31567871,488.70212402)
\curveto(341.31566509,488.57212138)(341.3056651,488.48212147)(341.28567871,488.43212402)
\curveto(341.26566514,488.38212157)(341.26066515,488.32712162)(341.27067871,488.26712402)
\curveto(341.28066513,488.21712173)(341.28066513,488.16712178)(341.27067871,488.11712402)
\curveto(341.23066518,487.97712197)(341.20066521,487.84212211)(341.18067871,487.71212402)
\curveto(341.17066524,487.58212237)(341.14066527,487.46212249)(341.09067871,487.35212402)
\curveto(340.95066546,487.00212295)(340.78566562,486.70712324)(340.59567871,486.46712402)
\curveto(340.405666,486.23712371)(340.13566627,486.0521239)(339.78567871,485.91212402)
\curveto(339.7056667,485.88212407)(339.62066679,485.86212409)(339.53067871,485.85212402)
\curveto(339.44066697,485.83212412)(339.35566705,485.81212414)(339.27567871,485.79212402)
\curveto(339.22566718,485.78212417)(339.17566723,485.77712417)(339.12567871,485.77712402)
\curveto(339.07566733,485.77712417)(339.02566738,485.77212418)(338.97567871,485.76212402)
\curveto(338.94566746,485.7521242)(338.89566751,485.7521242)(338.82567871,485.76212402)
\curveto(338.75566765,485.76212419)(338.7056677,485.76712418)(338.67567871,485.77712402)
\curveto(338.61566779,485.79712415)(338.55566785,485.80712414)(338.49567871,485.80712402)
\curveto(338.44566796,485.79712415)(338.39566801,485.80212415)(338.34567871,485.82212402)
\curveto(338.25566815,485.84212411)(338.16566824,485.86712408)(338.07567871,485.89712402)
\curveto(337.99566841,485.91712403)(337.91566849,485.947124)(337.83567871,485.98712402)
\curveto(337.51566889,486.12712382)(337.26566914,486.32212363)(337.08567871,486.57212402)
\curveto(336.9056695,486.83212312)(336.75566965,487.13712281)(336.63567871,487.48712402)
\curveto(336.61566979,487.56712238)(336.60066981,487.6521223)(336.59067871,487.74212402)
\curveto(336.58066983,487.83212212)(336.56566984,487.91712203)(336.54567871,487.99712402)
\curveto(336.53566987,488.02712192)(336.53066988,488.05712189)(336.53067871,488.08712402)
\lineto(336.53067871,488.19212402)
\curveto(336.5106699,488.27212168)(336.50066991,488.3521216)(336.50067871,488.43212402)
\lineto(336.50067871,488.56712402)
\curveto(336.48066993,488.66712128)(336.48066993,488.76712118)(336.50067871,488.86712402)
\lineto(336.50067871,489.04712402)
\curveto(336.5106699,489.09712085)(336.51566989,489.14212081)(336.51567871,489.18212402)
\curveto(336.51566989,489.23212072)(336.52066989,489.27712067)(336.53067871,489.31712402)
\curveto(336.54066987,489.35712059)(336.54566986,489.39212056)(336.54567871,489.42212402)
\curveto(336.54566986,489.46212049)(336.55066986,489.50212045)(336.56067871,489.54212402)
\lineto(336.62067871,489.87212402)
\curveto(336.64066977,489.99211996)(336.67066974,490.10211985)(336.71067871,490.20212402)
\curveto(336.85066956,490.53211942)(337.0106694,490.80711914)(337.19067871,491.02712402)
\curveto(337.38066903,491.25711869)(337.64066877,491.44211851)(337.97067871,491.58212402)
\curveto(338.05066836,491.62211833)(338.13566827,491.6471183)(338.22567871,491.65712402)
\lineto(338.52567871,491.71712402)
\lineto(338.66067871,491.71712402)
\curveto(338.7106677,491.72711822)(338.76066765,491.73211822)(338.81067871,491.73212402)
\curveto(339.38066703,491.7521182)(339.84066657,491.6471183)(340.19067871,491.41712402)
\curveto(340.55066586,491.19711875)(340.81566559,490.89711905)(340.98567871,490.51712402)
\curveto(341.03566537,490.41711953)(341.07566533,490.31711963)(341.10567871,490.21712402)
\curveto(341.13566527,490.11711983)(341.16566524,490.01211994)(341.19567871,489.90212402)
\curveto(341.2056652,489.86212009)(341.2106652,489.82712012)(341.21067871,489.79712402)
\curveto(341.2106652,489.77712017)(341.21566519,489.7471202)(341.22567871,489.70712402)
\curveto(341.24566516,489.63712031)(341.25566515,489.56212039)(341.25567871,489.48212402)
\curveto(341.25566515,489.40212055)(341.26566514,489.32212063)(341.28567871,489.24212402)
\curveto(341.28566512,489.19212076)(341.28566512,489.1471208)(341.28567871,489.10712402)
\curveto(341.28566512,489.06712088)(341.29066512,489.02212093)(341.30067871,488.97212402)
\moveto(340.19067871,488.53712402)
\curveto(340.20066621,488.58712136)(340.2056662,488.66212129)(340.20567871,488.76212402)
\curveto(340.21566619,488.86212109)(340.2106662,488.93712101)(340.19067871,488.98712402)
\curveto(340.17066624,489.0471209)(340.16566624,489.10212085)(340.17567871,489.15212402)
\curveto(340.19566621,489.21212074)(340.19566621,489.27212068)(340.17567871,489.33212402)
\curveto(340.16566624,489.36212059)(340.16066625,489.39712055)(340.16067871,489.43712402)
\curveto(340.16066625,489.47712047)(340.15566625,489.51712043)(340.14567871,489.55712402)
\curveto(340.12566628,489.63712031)(340.1056663,489.71212024)(340.08567871,489.78212402)
\curveto(340.07566633,489.86212009)(340.06066635,489.94212001)(340.04067871,490.02212402)
\curveto(340.0106664,490.08211987)(339.98566642,490.14211981)(339.96567871,490.20212402)
\curveto(339.94566646,490.26211969)(339.91566649,490.32211963)(339.87567871,490.38212402)
\curveto(339.77566663,490.5521194)(339.64566676,490.68711926)(339.48567871,490.78712402)
\curveto(339.405667,490.83711911)(339.3106671,490.87211908)(339.20067871,490.89212402)
\curveto(339.09066732,490.91211904)(338.96566744,490.92211903)(338.82567871,490.92212402)
\curveto(338.8056676,490.91211904)(338.78066763,490.90711904)(338.75067871,490.90712402)
\curveto(338.72066769,490.91711903)(338.69066772,490.91711903)(338.66067871,490.90712402)
\lineto(338.51067871,490.84712402)
\curveto(338.46066795,490.83711911)(338.41566799,490.82211913)(338.37567871,490.80212402)
\curveto(338.18566822,490.69211926)(338.04066837,490.5471194)(337.94067871,490.36712402)
\curveto(337.85066856,490.18711976)(337.77066864,489.98211997)(337.70067871,489.75212402)
\curveto(337.66066875,489.62212033)(337.64066877,489.48712046)(337.64067871,489.34712402)
\curveto(337.64066877,489.21712073)(337.63066878,489.07212088)(337.61067871,488.91212402)
\curveto(337.60066881,488.86212109)(337.59066882,488.80212115)(337.58067871,488.73212402)
\curveto(337.58066883,488.66212129)(337.59066882,488.60212135)(337.61067871,488.55212402)
\lineto(337.61067871,488.38712402)
\lineto(337.61067871,488.20712402)
\curveto(337.62066879,488.15712179)(337.63066878,488.10212185)(337.64067871,488.04212402)
\curveto(337.65066876,487.99212196)(337.65566875,487.93712201)(337.65567871,487.87712402)
\curveto(337.66566874,487.81712213)(337.68066873,487.76212219)(337.70067871,487.71212402)
\curveto(337.75066866,487.52212243)(337.8106686,487.3471226)(337.88067871,487.18712402)
\curveto(337.95066846,487.02712292)(338.05566835,486.89712305)(338.19567871,486.79712402)
\curveto(338.32566808,486.69712325)(338.46566794,486.62712332)(338.61567871,486.58712402)
\curveto(338.64566776,486.57712337)(338.67066774,486.57212338)(338.69067871,486.57212402)
\curveto(338.72066769,486.58212337)(338.75066766,486.58212337)(338.78067871,486.57212402)
\curveto(338.80066761,486.57212338)(338.83066758,486.56712338)(338.87067871,486.55712402)
\curveto(338.9106675,486.55712339)(338.94566746,486.56212339)(338.97567871,486.57212402)
\curveto(339.01566739,486.58212337)(339.05566735,486.58712336)(339.09567871,486.58712402)
\curveto(339.13566727,486.58712336)(339.17566723,486.59712335)(339.21567871,486.61712402)
\curveto(339.45566695,486.69712325)(339.65066676,486.83212312)(339.80067871,487.02212402)
\curveto(339.92066649,487.20212275)(340.0106664,487.40712254)(340.07067871,487.63712402)
\curveto(340.09066632,487.70712224)(340.1056663,487.77712217)(340.11567871,487.84712402)
\curveto(340.12566628,487.92712202)(340.14066627,488.00712194)(340.16067871,488.08712402)
\curveto(340.16066625,488.1471218)(340.16566624,488.19212176)(340.17567871,488.22212402)
\curveto(340.17566623,488.24212171)(340.17566623,488.26712168)(340.17567871,488.29712402)
\curveto(340.17566623,488.33712161)(340.18066623,488.36712158)(340.19067871,488.38712402)
\lineto(340.19067871,488.53712402)
}
}
{
\newrgbcolor{curcolor}{0 0 0}
\pscustom[linestyle=none,fillstyle=solid,fillcolor=curcolor]
{
\newpath
\moveto(635.75145996,625.83282959)
\curveto(635.85145511,625.83281897)(635.94645501,625.82281898)(636.03645996,625.80282959)
\curveto(636.12645483,625.79281901)(636.19145477,625.76281904)(636.23145996,625.71282959)
\curveto(636.29145467,625.63281917)(636.32145464,625.52781927)(636.32145996,625.39782959)
\lineto(636.32145996,625.00782959)
\lineto(636.32145996,623.50782959)
\lineto(636.32145996,617.11782959)
\lineto(636.32145996,615.94782959)
\lineto(636.32145996,615.63282959)
\curveto(636.33145463,615.53282927)(636.31645464,615.45282935)(636.27645996,615.39282959)
\curveto(636.22645473,615.31282949)(636.15145481,615.26282954)(636.05145996,615.24282959)
\curveto(635.961455,615.23282957)(635.85145511,615.22782957)(635.72145996,615.22782959)
\lineto(635.49645996,615.22782959)
\curveto(635.41645554,615.24782955)(635.34645561,615.26282954)(635.28645996,615.27282959)
\curveto(635.22645573,615.29282951)(635.17645578,615.33282947)(635.13645996,615.39282959)
\curveto(635.09645586,615.45282935)(635.07645588,615.52782927)(635.07645996,615.61782959)
\lineto(635.07645996,615.91782959)
\lineto(635.07645996,617.01282959)
\lineto(635.07645996,622.35282959)
\curveto(635.0564559,622.44282236)(635.04145592,622.51782228)(635.03145996,622.57782959)
\curveto(635.03145593,622.64782215)(635.00145596,622.70782209)(634.94145996,622.75782959)
\curveto(634.87145609,622.80782199)(634.78145618,622.83282197)(634.67145996,622.83282959)
\curveto(634.57145639,622.84282196)(634.4614565,622.84782195)(634.34145996,622.84782959)
\lineto(633.20145996,622.84782959)
\lineto(632.70645996,622.84782959)
\curveto(632.54645841,622.85782194)(632.43645852,622.91782188)(632.37645996,623.02782959)
\curveto(632.3564586,623.05782174)(632.34645861,623.08782171)(632.34645996,623.11782959)
\curveto(632.34645861,623.15782164)(632.34145862,623.2028216)(632.33145996,623.25282959)
\curveto(632.31145865,623.37282143)(632.31645864,623.48282132)(632.34645996,623.58282959)
\curveto(632.38645857,623.68282112)(632.44145852,623.75282105)(632.51145996,623.79282959)
\curveto(632.59145837,623.84282096)(632.71145825,623.86782093)(632.87145996,623.86782959)
\curveto(633.03145793,623.86782093)(633.16645779,623.88282092)(633.27645996,623.91282959)
\curveto(633.32645763,623.92282088)(633.38145758,623.92782087)(633.44145996,623.92782959)
\curveto(633.50145746,623.93782086)(633.5614574,623.95282085)(633.62145996,623.97282959)
\curveto(633.77145719,624.02282078)(633.91645704,624.07282073)(634.05645996,624.12282959)
\curveto(634.19645676,624.18282062)(634.33145663,624.25282055)(634.46145996,624.33282959)
\curveto(634.60145636,624.42282038)(634.72145624,624.52782027)(634.82145996,624.64782959)
\curveto(634.92145604,624.76782003)(635.01645594,624.8978199)(635.10645996,625.03782959)
\curveto(635.16645579,625.13781966)(635.21145575,625.24781955)(635.24145996,625.36782959)
\curveto(635.28145568,625.48781931)(635.33145563,625.59281921)(635.39145996,625.68282959)
\curveto(635.44145552,625.74281906)(635.51145545,625.78281902)(635.60145996,625.80282959)
\curveto(635.62145534,625.81281899)(635.64645531,625.81781898)(635.67645996,625.81782959)
\curveto(635.70645525,625.81781898)(635.73145523,625.82281898)(635.75145996,625.83282959)
}
}
{
\newrgbcolor{curcolor}{0 0 0}
\pscustom[linestyle=none,fillstyle=solid,fillcolor=curcolor]
{
\newpath
\moveto(641.55106934,625.63782959)
\lineto(645.15106934,625.63782959)
\lineto(645.79606934,625.63782959)
\curveto(645.87606281,625.63781916)(645.95106273,625.63281917)(646.02106934,625.62282959)
\curveto(646.09106259,625.62281918)(646.15106253,625.61281919)(646.20106934,625.59282959)
\curveto(646.27106241,625.56281924)(646.32606236,625.5028193)(646.36606934,625.41282959)
\curveto(646.3860623,625.38281942)(646.39606229,625.34281946)(646.39606934,625.29282959)
\lineto(646.39606934,625.15782959)
\curveto(646.40606228,625.04781975)(646.40106228,624.94281986)(646.38106934,624.84282959)
\curveto(646.37106231,624.74282006)(646.33606235,624.67282013)(646.27606934,624.63282959)
\curveto(646.1860625,624.56282024)(646.05106263,624.52782027)(645.87106934,624.52782959)
\curveto(645.69106299,624.53782026)(645.52606316,624.54282026)(645.37606934,624.54282959)
\lineto(643.38106934,624.54282959)
\lineto(642.88606934,624.54282959)
\lineto(642.75106934,624.54282959)
\curveto(642.71106597,624.54282026)(642.67106601,624.53782026)(642.63106934,624.52782959)
\lineto(642.42106934,624.52782959)
\curveto(642.31106637,624.4978203)(642.23106645,624.45782034)(642.18106934,624.40782959)
\curveto(642.13106655,624.36782043)(642.09606659,624.31282049)(642.07606934,624.24282959)
\curveto(642.05606663,624.18282062)(642.04106664,624.11282069)(642.03106934,624.03282959)
\curveto(642.02106666,623.95282085)(642.00106668,623.86282094)(641.97106934,623.76282959)
\curveto(641.92106676,623.56282124)(641.8810668,623.35782144)(641.85106934,623.14782959)
\curveto(641.82106686,622.93782186)(641.7810669,622.73282207)(641.73106934,622.53282959)
\curveto(641.71106697,622.46282234)(641.70106698,622.39282241)(641.70106934,622.32282959)
\curveto(641.70106698,622.26282254)(641.69106699,622.1978226)(641.67106934,622.12782959)
\curveto(641.66106702,622.0978227)(641.65106703,622.05782274)(641.64106934,622.00782959)
\curveto(641.64106704,621.96782283)(641.64606704,621.92782287)(641.65606934,621.88782959)
\curveto(641.67606701,621.83782296)(641.70106698,621.79282301)(641.73106934,621.75282959)
\curveto(641.77106691,621.72282308)(641.83106685,621.71782308)(641.91106934,621.73782959)
\curveto(641.97106671,621.75782304)(642.03106665,621.78282302)(642.09106934,621.81282959)
\curveto(642.15106653,621.85282295)(642.21106647,621.88782291)(642.27106934,621.91782959)
\curveto(642.33106635,621.93782286)(642.3810663,621.95282285)(642.42106934,621.96282959)
\curveto(642.61106607,622.04282276)(642.81606587,622.0978227)(643.03606934,622.12782959)
\curveto(643.26606542,622.15782264)(643.49606519,622.16782263)(643.72606934,622.15782959)
\curveto(643.96606472,622.15782264)(644.19606449,622.13282267)(644.41606934,622.08282959)
\curveto(644.63606405,622.04282276)(644.83606385,621.98282282)(645.01606934,621.90282959)
\curveto(645.06606362,621.88282292)(645.11106357,621.86282294)(645.15106934,621.84282959)
\curveto(645.20106348,621.82282298)(645.25106343,621.797823)(645.30106934,621.76782959)
\curveto(645.65106303,621.55782324)(645.93106275,621.32782347)(646.14106934,621.07782959)
\curveto(646.36106232,620.82782397)(646.55606213,620.5028243)(646.72606934,620.10282959)
\curveto(646.77606191,619.99282481)(646.81106187,619.88282492)(646.83106934,619.77282959)
\curveto(646.85106183,619.66282514)(646.87606181,619.54782525)(646.90606934,619.42782959)
\curveto(646.91606177,619.3978254)(646.92106176,619.35282545)(646.92106934,619.29282959)
\curveto(646.94106174,619.23282557)(646.95106173,619.16282564)(646.95106934,619.08282959)
\curveto(646.95106173,619.01282579)(646.96106172,618.94782585)(646.98106934,618.88782959)
\lineto(646.98106934,618.72282959)
\curveto(646.99106169,618.67282613)(646.99606169,618.6028262)(646.99606934,618.51282959)
\curveto(646.99606169,618.42282638)(646.9860617,618.35282645)(646.96606934,618.30282959)
\curveto(646.94606174,618.24282656)(646.94106174,618.18282662)(646.95106934,618.12282959)
\curveto(646.96106172,618.07282673)(646.95606173,618.02282678)(646.93606934,617.97282959)
\curveto(646.89606179,617.81282699)(646.86106182,617.66282714)(646.83106934,617.52282959)
\curveto(646.80106188,617.38282742)(646.75606193,617.24782755)(646.69606934,617.11782959)
\curveto(646.53606215,616.74782805)(646.31606237,616.41282839)(646.03606934,616.11282959)
\curveto(645.75606293,615.81282899)(645.43606325,615.58282922)(645.07606934,615.42282959)
\curveto(644.90606378,615.34282946)(644.70606398,615.26782953)(644.47606934,615.19782959)
\curveto(644.36606432,615.15782964)(644.25106443,615.13282967)(644.13106934,615.12282959)
\curveto(644.01106467,615.11282969)(643.89106479,615.09282971)(643.77106934,615.06282959)
\curveto(643.72106496,615.04282976)(643.66606502,615.04282976)(643.60606934,615.06282959)
\curveto(643.54606514,615.07282973)(643.4860652,615.06782973)(643.42606934,615.04782959)
\curveto(643.32606536,615.02782977)(643.22606546,615.02782977)(643.12606934,615.04782959)
\lineto(642.99106934,615.04782959)
\curveto(642.94106574,615.06782973)(642.8810658,615.07782972)(642.81106934,615.07782959)
\curveto(642.75106593,615.06782973)(642.69606599,615.07282973)(642.64606934,615.09282959)
\curveto(642.60606608,615.1028297)(642.57106611,615.10782969)(642.54106934,615.10782959)
\curveto(642.51106617,615.10782969)(642.47606621,615.11282969)(642.43606934,615.12282959)
\lineto(642.16606934,615.18282959)
\curveto(642.07606661,615.2028296)(641.99106669,615.23282957)(641.91106934,615.27282959)
\curveto(641.57106711,615.41282939)(641.2810674,615.56782923)(641.04106934,615.73782959)
\curveto(640.80106788,615.91782888)(640.5810681,616.14782865)(640.38106934,616.42782959)
\curveto(640.23106845,616.65782814)(640.11606857,616.8978279)(640.03606934,617.14782959)
\curveto(640.01606867,617.1978276)(640.00606868,617.24282756)(640.00606934,617.28282959)
\curveto(640.00606868,617.33282747)(639.99606869,617.38282742)(639.97606934,617.43282959)
\curveto(639.95606873,617.49282731)(639.94106874,617.57282723)(639.93106934,617.67282959)
\curveto(639.93106875,617.77282703)(639.95106873,617.84782695)(639.99106934,617.89782959)
\curveto(640.04106864,617.97782682)(640.12106856,618.02282678)(640.23106934,618.03282959)
\curveto(640.34106834,618.04282676)(640.45606823,618.04782675)(640.57606934,618.04782959)
\lineto(640.74106934,618.04782959)
\curveto(640.80106788,618.04782675)(640.85606783,618.03782676)(640.90606934,618.01782959)
\curveto(640.99606769,617.9978268)(641.06606762,617.95782684)(641.11606934,617.89782959)
\curveto(641.1860675,617.80782699)(641.23106745,617.6978271)(641.25106934,617.56782959)
\curveto(641.2810674,617.44782735)(641.32606736,617.34282746)(641.38606934,617.25282959)
\curveto(641.57606711,616.91282789)(641.83606685,616.64282816)(642.16606934,616.44282959)
\curveto(642.26606642,616.38282842)(642.37106631,616.33282847)(642.48106934,616.29282959)
\curveto(642.60106608,616.26282854)(642.72106596,616.22782857)(642.84106934,616.18782959)
\curveto(643.01106567,616.13782866)(643.21606547,616.11782868)(643.45606934,616.12782959)
\curveto(643.70606498,616.14782865)(643.90606478,616.18282862)(644.05606934,616.23282959)
\curveto(644.42606426,616.35282845)(644.71606397,616.51282829)(644.92606934,616.71282959)
\curveto(645.14606354,616.92282788)(645.32606336,617.2028276)(645.46606934,617.55282959)
\curveto(645.51606317,617.65282715)(645.54606314,617.75782704)(645.55606934,617.86782959)
\curveto(645.57606311,617.97782682)(645.60106308,618.09282671)(645.63106934,618.21282959)
\lineto(645.63106934,618.31782959)
\curveto(645.64106304,618.35782644)(645.64606304,618.3978264)(645.64606934,618.43782959)
\curveto(645.65606303,618.46782633)(645.65606303,618.5028263)(645.64606934,618.54282959)
\lineto(645.64606934,618.66282959)
\curveto(645.64606304,618.92282588)(645.61606307,619.16782563)(645.55606934,619.39782959)
\curveto(645.44606324,619.74782505)(645.29106339,620.04282476)(645.09106934,620.28282959)
\curveto(644.89106379,620.53282427)(644.63106405,620.72782407)(644.31106934,620.86782959)
\lineto(644.13106934,620.92782959)
\curveto(644.0810646,620.94782385)(644.02106466,620.96782383)(643.95106934,620.98782959)
\curveto(643.90106478,621.00782379)(643.84106484,621.01782378)(643.77106934,621.01782959)
\curveto(643.71106497,621.02782377)(643.64606504,621.04282376)(643.57606934,621.06282959)
\lineto(643.42606934,621.06282959)
\curveto(643.3860653,621.08282372)(643.33106535,621.09282371)(643.26106934,621.09282959)
\curveto(643.20106548,621.09282371)(643.14606554,621.08282372)(643.09606934,621.06282959)
\lineto(642.99106934,621.06282959)
\curveto(642.96106572,621.06282374)(642.92606576,621.05782374)(642.88606934,621.04782959)
\lineto(642.64606934,620.98782959)
\curveto(642.56606612,620.97782382)(642.4860662,620.95782384)(642.40606934,620.92782959)
\curveto(642.16606652,620.82782397)(641.93606675,620.69282411)(641.71606934,620.52282959)
\curveto(641.62606706,620.45282435)(641.54106714,620.37782442)(641.46106934,620.29782959)
\curveto(641.3810673,620.22782457)(641.2810674,620.17282463)(641.16106934,620.13282959)
\curveto(641.07106761,620.1028247)(640.93106775,620.09282471)(640.74106934,620.10282959)
\curveto(640.56106812,620.11282469)(640.44106824,620.13782466)(640.38106934,620.17782959)
\curveto(640.33106835,620.21782458)(640.29106839,620.27782452)(640.26106934,620.35782959)
\curveto(640.24106844,620.43782436)(640.24106844,620.52282428)(640.26106934,620.61282959)
\curveto(640.29106839,620.73282407)(640.31106837,620.85282395)(640.32106934,620.97282959)
\curveto(640.34106834,621.1028237)(640.36606832,621.22782357)(640.39606934,621.34782959)
\curveto(640.41606827,621.38782341)(640.42106826,621.42282338)(640.41106934,621.45282959)
\curveto(640.41106827,621.49282331)(640.42106826,621.53782326)(640.44106934,621.58782959)
\curveto(640.46106822,621.67782312)(640.47606821,621.76782303)(640.48606934,621.85782959)
\curveto(640.49606819,621.95782284)(640.51606817,622.05282275)(640.54606934,622.14282959)
\curveto(640.55606813,622.2028226)(640.56106812,622.26282254)(640.56106934,622.32282959)
\curveto(640.57106811,622.38282242)(640.5860681,622.44282236)(640.60606934,622.50282959)
\curveto(640.65606803,622.7028221)(640.69106799,622.90782189)(640.71106934,623.11782959)
\curveto(640.74106794,623.33782146)(640.7810679,623.54782125)(640.83106934,623.74782959)
\curveto(640.86106782,623.84782095)(640.8810678,623.94782085)(640.89106934,624.04782959)
\curveto(640.90106778,624.14782065)(640.91606777,624.24782055)(640.93606934,624.34782959)
\curveto(640.94606774,624.37782042)(640.95106773,624.41782038)(640.95106934,624.46782959)
\curveto(640.9810677,624.57782022)(641.00106768,624.68282012)(641.01106934,624.78282959)
\curveto(641.03106765,624.89281991)(641.05606763,625.0028198)(641.08606934,625.11282959)
\curveto(641.10606758,625.19281961)(641.12106756,625.26281954)(641.13106934,625.32282959)
\curveto(641.14106754,625.39281941)(641.16606752,625.45281935)(641.20606934,625.50282959)
\curveto(641.22606746,625.53281927)(641.25606743,625.55281925)(641.29606934,625.56282959)
\curveto(641.33606735,625.58281922)(641.3810673,625.6028192)(641.43106934,625.62282959)
\curveto(641.49106719,625.62281918)(641.53106715,625.62781917)(641.55106934,625.63782959)
}
}
{
\newrgbcolor{curcolor}{0 0 0}
\pscustom[linestyle=none,fillstyle=solid,fillcolor=curcolor]
{
\newpath
\moveto(649.34567871,616.86282959)
\lineto(649.64567871,616.86282959)
\curveto(649.75567665,616.87282793)(649.86067655,616.87282793)(649.96067871,616.86282959)
\curveto(650.07067634,616.86282794)(650.17067624,616.85282795)(650.26067871,616.83282959)
\curveto(650.35067606,616.82282798)(650.42067599,616.797828)(650.47067871,616.75782959)
\curveto(650.49067592,616.73782806)(650.5056759,616.70782809)(650.51567871,616.66782959)
\curveto(650.53567587,616.62782817)(650.55567585,616.58282822)(650.57567871,616.53282959)
\lineto(650.57567871,616.45782959)
\curveto(650.58567582,616.40782839)(650.58567582,616.35282845)(650.57567871,616.29282959)
\lineto(650.57567871,616.14282959)
\lineto(650.57567871,615.66282959)
\curveto(650.57567583,615.49282931)(650.53567587,615.37282943)(650.45567871,615.30282959)
\curveto(650.38567602,615.25282955)(650.29567611,615.22782957)(650.18567871,615.22782959)
\lineto(649.85567871,615.22782959)
\lineto(649.40567871,615.22782959)
\curveto(649.25567715,615.22782957)(649.14067727,615.25782954)(649.06067871,615.31782959)
\curveto(649.02067739,615.34782945)(648.99067742,615.3978294)(648.97067871,615.46782959)
\curveto(648.95067746,615.54782925)(648.93567747,615.63282917)(648.92567871,615.72282959)
\lineto(648.92567871,616.00782959)
\curveto(648.93567747,616.10782869)(648.94067747,616.19282861)(648.94067871,616.26282959)
\lineto(648.94067871,616.45782959)
\curveto(648.94067747,616.51782828)(648.95067746,616.57282823)(648.97067871,616.62282959)
\curveto(649.0106774,616.73282807)(649.08067733,616.802828)(649.18067871,616.83282959)
\curveto(649.2106772,616.83282797)(649.26567714,616.84282796)(649.34567871,616.86282959)
}
}
{
\newrgbcolor{curcolor}{0 0 0}
\pscustom[linestyle=none,fillstyle=solid,fillcolor=curcolor]
{
\newpath
\moveto(655.80083496,625.83282959)
\curveto(656.49083033,625.84281896)(657.09082973,625.72281908)(657.60083496,625.47282959)
\curveto(658.1208287,625.22281958)(658.5158283,624.88781991)(658.78583496,624.46782959)
\curveto(658.83582798,624.38782041)(658.88082794,624.2978205)(658.92083496,624.19782959)
\curveto(658.96082786,624.10782069)(659.00582781,624.01282079)(659.05583496,623.91282959)
\curveto(659.09582772,623.81282099)(659.12582769,623.71282109)(659.14583496,623.61282959)
\curveto(659.16582765,623.51282129)(659.18582763,623.40782139)(659.20583496,623.29782959)
\curveto(659.22582759,623.24782155)(659.23082759,623.2028216)(659.22083496,623.16282959)
\curveto(659.21082761,623.12282168)(659.2158276,623.07782172)(659.23583496,623.02782959)
\curveto(659.24582757,622.97782182)(659.25082757,622.89282191)(659.25083496,622.77282959)
\curveto(659.25082757,622.66282214)(659.24582757,622.57782222)(659.23583496,622.51782959)
\curveto(659.2158276,622.45782234)(659.20582761,622.3978224)(659.20583496,622.33782959)
\curveto(659.2158276,622.27782252)(659.21082761,622.21782258)(659.19083496,622.15782959)
\curveto(659.15082767,622.01782278)(659.1158277,621.88282292)(659.08583496,621.75282959)
\curveto(659.05582776,621.62282318)(659.0158278,621.4978233)(658.96583496,621.37782959)
\curveto(658.90582791,621.23782356)(658.83582798,621.11282369)(658.75583496,621.00282959)
\curveto(658.68582813,620.89282391)(658.61082821,620.78282402)(658.53083496,620.67282959)
\lineto(658.47083496,620.61282959)
\curveto(658.46082836,620.59282421)(658.44582837,620.57282423)(658.42583496,620.55282959)
\curveto(658.30582851,620.39282441)(658.17082865,620.24782455)(658.02083496,620.11782959)
\curveto(657.87082895,619.98782481)(657.71082911,619.86282494)(657.54083496,619.74282959)
\curveto(657.23082959,619.52282528)(656.93582988,619.31782548)(656.65583496,619.12782959)
\curveto(656.42583039,618.98782581)(656.19583062,618.85282595)(655.96583496,618.72282959)
\curveto(655.74583107,618.59282621)(655.52583129,618.45782634)(655.30583496,618.31782959)
\curveto(655.05583176,618.14782665)(654.815832,617.96782683)(654.58583496,617.77782959)
\curveto(654.36583245,617.58782721)(654.17583264,617.36282744)(654.01583496,617.10282959)
\curveto(653.97583284,617.04282776)(653.94083288,616.98282782)(653.91083496,616.92282959)
\curveto(653.88083294,616.87282793)(653.85083297,616.80782799)(653.82083496,616.72782959)
\curveto(653.80083302,616.65782814)(653.79583302,616.5978282)(653.80583496,616.54782959)
\curveto(653.82583299,616.47782832)(653.86083296,616.42282838)(653.91083496,616.38282959)
\curveto(653.96083286,616.35282845)(654.0208328,616.33282847)(654.09083496,616.32282959)
\lineto(654.33083496,616.32282959)
\lineto(655.08083496,616.32282959)
\lineto(657.88583496,616.32282959)
\lineto(658.54583496,616.32282959)
\curveto(658.63582818,616.32282848)(658.7208281,616.31782848)(658.80083496,616.30782959)
\curveto(658.88082794,616.30782849)(658.94582787,616.28782851)(658.99583496,616.24782959)
\curveto(659.04582777,616.20782859)(659.08582773,616.13282867)(659.11583496,616.02282959)
\curveto(659.15582766,615.92282888)(659.16582765,615.82282898)(659.14583496,615.72282959)
\lineto(659.14583496,615.58782959)
\curveto(659.12582769,615.51782928)(659.10582771,615.45782934)(659.08583496,615.40782959)
\curveto(659.06582775,615.35782944)(659.03082779,615.31782948)(658.98083496,615.28782959)
\curveto(658.93082789,615.24782955)(658.86082796,615.22782957)(658.77083496,615.22782959)
\lineto(658.50083496,615.22782959)
\lineto(657.60083496,615.22782959)
\lineto(654.09083496,615.22782959)
\lineto(653.02583496,615.22782959)
\curveto(652.94583387,615.22782957)(652.85583396,615.22282958)(652.75583496,615.21282959)
\curveto(652.65583416,615.21282959)(652.57083425,615.22282958)(652.50083496,615.24282959)
\curveto(652.29083453,615.31282949)(652.22583459,615.49282931)(652.30583496,615.78282959)
\curveto(652.3158345,615.82282898)(652.3158345,615.85782894)(652.30583496,615.88782959)
\curveto(652.30583451,615.92782887)(652.3158345,615.97282883)(652.33583496,616.02282959)
\curveto(652.35583446,616.1028287)(652.37583444,616.18782861)(652.39583496,616.27782959)
\curveto(652.4158344,616.36782843)(652.44083438,616.45282835)(652.47083496,616.53282959)
\curveto(652.63083419,617.02282778)(652.83083399,617.43782736)(653.07083496,617.77782959)
\curveto(653.25083357,618.02782677)(653.45583336,618.25282655)(653.68583496,618.45282959)
\curveto(653.9158329,618.66282614)(654.15583266,618.85782594)(654.40583496,619.03782959)
\curveto(654.66583215,619.21782558)(654.93083189,619.38782541)(655.20083496,619.54782959)
\curveto(655.48083134,619.71782508)(655.75083107,619.89282491)(656.01083496,620.07282959)
\curveto(656.1208307,620.15282465)(656.22583059,620.22782457)(656.32583496,620.29782959)
\curveto(656.43583038,620.36782443)(656.54583027,620.44282436)(656.65583496,620.52282959)
\curveto(656.69583012,620.55282425)(656.73083009,620.58282422)(656.76083496,620.61282959)
\curveto(656.80083002,620.65282415)(656.84082998,620.68282412)(656.88083496,620.70282959)
\curveto(657.0208298,620.81282399)(657.14582967,620.93782386)(657.25583496,621.07782959)
\curveto(657.27582954,621.10782369)(657.30082952,621.13282367)(657.33083496,621.15282959)
\curveto(657.36082946,621.18282362)(657.38582943,621.21282359)(657.40583496,621.24282959)
\curveto(657.48582933,621.34282346)(657.55082927,621.44282336)(657.60083496,621.54282959)
\curveto(657.66082916,621.64282316)(657.7158291,621.75282305)(657.76583496,621.87282959)
\curveto(657.79582902,621.94282286)(657.815829,622.01782278)(657.82583496,622.09782959)
\lineto(657.88583496,622.33782959)
\lineto(657.88583496,622.42782959)
\curveto(657.89582892,622.45782234)(657.90082892,622.48782231)(657.90083496,622.51782959)
\curveto(657.9208289,622.58782221)(657.92582889,622.68282212)(657.91583496,622.80282959)
\curveto(657.9158289,622.93282187)(657.90582891,623.03282177)(657.88583496,623.10282959)
\curveto(657.86582895,623.18282162)(657.84582897,623.25782154)(657.82583496,623.32782959)
\curveto(657.815829,623.40782139)(657.79582902,623.48782131)(657.76583496,623.56782959)
\curveto(657.65582916,623.80782099)(657.50582931,624.00782079)(657.31583496,624.16782959)
\curveto(657.13582968,624.33782046)(656.9158299,624.47782032)(656.65583496,624.58782959)
\curveto(656.58583023,624.60782019)(656.5158303,624.62282018)(656.44583496,624.63282959)
\curveto(656.37583044,624.65282015)(656.30083052,624.67282013)(656.22083496,624.69282959)
\curveto(656.14083068,624.71282009)(656.03083079,624.72282008)(655.89083496,624.72282959)
\curveto(655.76083106,624.72282008)(655.65583116,624.71282009)(655.57583496,624.69282959)
\curveto(655.5158313,624.68282012)(655.46083136,624.67782012)(655.41083496,624.67782959)
\curveto(655.36083146,624.67782012)(655.31083151,624.66782013)(655.26083496,624.64782959)
\curveto(655.16083166,624.60782019)(655.06583175,624.56782023)(654.97583496,624.52782959)
\curveto(654.89583192,624.48782031)(654.815832,624.44282036)(654.73583496,624.39282959)
\curveto(654.70583211,624.37282043)(654.67583214,624.34782045)(654.64583496,624.31782959)
\curveto(654.62583219,624.28782051)(654.60083222,624.26282054)(654.57083496,624.24282959)
\lineto(654.49583496,624.16782959)
\curveto(654.46583235,624.14782065)(654.44083238,624.12782067)(654.42083496,624.10782959)
\lineto(654.27083496,623.89782959)
\curveto(654.23083259,623.83782096)(654.18583263,623.77282103)(654.13583496,623.70282959)
\curveto(654.07583274,623.61282119)(654.02583279,623.50782129)(653.98583496,623.38782959)
\curveto(653.95583286,623.27782152)(653.9208329,623.16782163)(653.88083496,623.05782959)
\curveto(653.84083298,622.94782185)(653.815833,622.802822)(653.80583496,622.62282959)
\curveto(653.79583302,622.45282235)(653.76583305,622.32782247)(653.71583496,622.24782959)
\curveto(653.66583315,622.16782263)(653.59083323,622.12282268)(653.49083496,622.11282959)
\curveto(653.39083343,622.1028227)(653.28083354,622.0978227)(653.16083496,622.09782959)
\curveto(653.1208337,622.0978227)(653.08083374,622.09282271)(653.04083496,622.08282959)
\curveto(653.00083382,622.08282272)(652.96583385,622.08782271)(652.93583496,622.09782959)
\curveto(652.88583393,622.11782268)(652.83583398,622.12782267)(652.78583496,622.12782959)
\curveto(652.74583407,622.12782267)(652.70583411,622.13782266)(652.66583496,622.15782959)
\curveto(652.57583424,622.21782258)(652.53083429,622.35282245)(652.53083496,622.56282959)
\lineto(652.53083496,622.68282959)
\curveto(652.54083428,622.74282206)(652.54583427,622.802822)(652.54583496,622.86282959)
\curveto(652.55583426,622.93282187)(652.56583425,622.9978218)(652.57583496,623.05782959)
\curveto(652.59583422,623.16782163)(652.6158342,623.26782153)(652.63583496,623.35782959)
\curveto(652.65583416,623.45782134)(652.68583413,623.55282125)(652.72583496,623.64282959)
\curveto(652.74583407,623.71282109)(652.76583405,623.77282103)(652.78583496,623.82282959)
\lineto(652.84583496,624.00282959)
\curveto(652.96583385,624.26282054)(653.1208337,624.50782029)(653.31083496,624.73782959)
\curveto(653.51083331,624.96781983)(653.72583309,625.15281965)(653.95583496,625.29282959)
\curveto(654.06583275,625.37281943)(654.18083264,625.43781936)(654.30083496,625.48782959)
\lineto(654.69083496,625.63782959)
\curveto(654.80083202,625.68781911)(654.9158319,625.71781908)(655.03583496,625.72782959)
\curveto(655.15583166,625.74781905)(655.28083154,625.77281903)(655.41083496,625.80282959)
\curveto(655.48083134,625.802819)(655.54583127,625.802819)(655.60583496,625.80282959)
\curveto(655.66583115,625.81281899)(655.73083109,625.82281898)(655.80083496,625.83282959)
}
}
{
\newrgbcolor{curcolor}{0 0 0}
\pscustom[linestyle=none,fillstyle=solid,fillcolor=curcolor]
{
\newpath
\moveto(670.70544434,623.74782959)
\curveto(670.50543404,623.45782134)(670.29543425,623.17282163)(670.07544434,622.89282959)
\curveto(669.86543468,622.61282219)(669.66043488,622.32782247)(669.46044434,622.03782959)
\curveto(668.86043568,621.18782361)(668.25543629,620.34782445)(667.64544434,619.51782959)
\curveto(667.03543751,618.6978261)(666.43043811,617.86282694)(665.83044434,617.01282959)
\lineto(665.32044434,616.29282959)
\lineto(664.81044434,615.60282959)
\curveto(664.73043981,615.49282931)(664.65043989,615.37782942)(664.57044434,615.25782959)
\curveto(664.49044005,615.13782966)(664.39544015,615.04282976)(664.28544434,614.97282959)
\curveto(664.2454403,614.95282985)(664.18044036,614.93782986)(664.09044434,614.92782959)
\curveto(664.01044053,614.90782989)(663.92044062,614.8978299)(663.82044434,614.89782959)
\curveto(663.72044082,614.8978299)(663.62544092,614.9028299)(663.53544434,614.91282959)
\curveto(663.45544109,614.92282988)(663.39544115,614.94282986)(663.35544434,614.97282959)
\curveto(663.32544122,614.99282981)(663.30044124,615.02782977)(663.28044434,615.07782959)
\curveto(663.27044127,615.11782968)(663.27544127,615.16282964)(663.29544434,615.21282959)
\curveto(663.33544121,615.29282951)(663.38044116,615.36782943)(663.43044434,615.43782959)
\curveto(663.49044105,615.51782928)(663.545441,615.5978292)(663.59544434,615.67782959)
\curveto(663.83544071,616.01782878)(664.08044046,616.35282845)(664.33044434,616.68282959)
\curveto(664.58043996,617.01282779)(664.82043972,617.34782745)(665.05044434,617.68782959)
\curveto(665.21043933,617.90782689)(665.37043917,618.12282668)(665.53044434,618.33282959)
\curveto(665.69043885,618.54282626)(665.85043869,618.75782604)(666.01044434,618.97782959)
\curveto(666.37043817,619.4978253)(666.73543781,620.00782479)(667.10544434,620.50782959)
\curveto(667.47543707,621.00782379)(667.8454367,621.51782328)(668.21544434,622.03782959)
\curveto(668.35543619,622.23782256)(668.49543605,622.43282237)(668.63544434,622.62282959)
\curveto(668.78543576,622.81282199)(668.93043561,623.00782179)(669.07044434,623.20782959)
\curveto(669.28043526,623.50782129)(669.49543505,623.80782099)(669.71544434,624.10782959)
\lineto(670.37544434,625.00782959)
\lineto(670.55544434,625.27782959)
\lineto(670.76544434,625.54782959)
\lineto(670.88544434,625.72782959)
\curveto(670.93543361,625.78781901)(670.98543356,625.84281896)(671.03544434,625.89282959)
\curveto(671.10543344,625.94281886)(671.18043336,625.97781882)(671.26044434,625.99782959)
\curveto(671.28043326,626.00781879)(671.30543324,626.00781879)(671.33544434,625.99782959)
\curveto(671.37543317,625.9978188)(671.40543314,626.00781879)(671.42544434,626.02782959)
\curveto(671.545433,626.02781877)(671.68043286,626.02281878)(671.83044434,626.01282959)
\curveto(671.98043256,626.01281879)(672.07043247,625.96781883)(672.10044434,625.87782959)
\curveto(672.12043242,625.84781895)(672.12543242,625.81281899)(672.11544434,625.77282959)
\curveto(672.10543244,625.73281907)(672.09043245,625.7028191)(672.07044434,625.68282959)
\curveto(672.03043251,625.6028192)(671.99043255,625.53281927)(671.95044434,625.47282959)
\curveto(671.91043263,625.41281939)(671.86543268,625.35281945)(671.81544434,625.29282959)
\lineto(671.24544434,624.51282959)
\curveto(671.06543348,624.26282054)(670.88543366,624.00782079)(670.70544434,623.74782959)
\moveto(663.85044434,619.84782959)
\curveto(663.80044074,619.86782493)(663.75044079,619.87282493)(663.70044434,619.86282959)
\curveto(663.65044089,619.85282495)(663.60044094,619.85782494)(663.55044434,619.87782959)
\curveto(663.4404411,619.8978249)(663.33544121,619.91782488)(663.23544434,619.93782959)
\curveto(663.1454414,619.96782483)(663.05044149,620.00782479)(662.95044434,620.05782959)
\curveto(662.62044192,620.1978246)(662.36544218,620.39282441)(662.18544434,620.64282959)
\curveto(662.00544254,620.9028239)(661.86044268,621.21282359)(661.75044434,621.57282959)
\curveto(661.72044282,621.65282315)(661.70044284,621.73282307)(661.69044434,621.81282959)
\curveto(661.68044286,621.9028229)(661.66544288,621.98782281)(661.64544434,622.06782959)
\curveto(661.63544291,622.11782268)(661.63044291,622.18282262)(661.63044434,622.26282959)
\curveto(661.62044292,622.29282251)(661.61544293,622.32282248)(661.61544434,622.35282959)
\curveto(661.61544293,622.39282241)(661.61044293,622.42782237)(661.60044434,622.45782959)
\lineto(661.60044434,622.60782959)
\curveto(661.59044295,622.65782214)(661.58544296,622.71782208)(661.58544434,622.78782959)
\curveto(661.58544296,622.86782193)(661.59044295,622.93282187)(661.60044434,622.98282959)
\lineto(661.60044434,623.14782959)
\curveto(661.62044292,623.1978216)(661.62544292,623.24282156)(661.61544434,623.28282959)
\curveto(661.61544293,623.33282147)(661.62044292,623.37782142)(661.63044434,623.41782959)
\curveto(661.6404429,623.45782134)(661.6454429,623.49282131)(661.64544434,623.52282959)
\curveto(661.6454429,623.56282124)(661.65044289,623.6028212)(661.66044434,623.64282959)
\curveto(661.69044285,623.75282105)(661.71044283,623.86282094)(661.72044434,623.97282959)
\curveto(661.7404428,624.09282071)(661.77544277,624.20782059)(661.82544434,624.31782959)
\curveto(661.96544258,624.65782014)(662.12544242,624.93281987)(662.30544434,625.14282959)
\curveto(662.49544205,625.36281944)(662.76544178,625.54281926)(663.11544434,625.68282959)
\curveto(663.19544135,625.71281909)(663.28044126,625.73281907)(663.37044434,625.74282959)
\curveto(663.46044108,625.76281904)(663.55544099,625.78281902)(663.65544434,625.80282959)
\curveto(663.68544086,625.81281899)(663.7404408,625.81281899)(663.82044434,625.80282959)
\curveto(663.90044064,625.802819)(663.95044059,625.81281899)(663.97044434,625.83282959)
\curveto(664.53044001,625.84281896)(664.98043956,625.73281907)(665.32044434,625.50282959)
\curveto(665.67043887,625.27281953)(665.93043861,624.96781983)(666.10044434,624.58782959)
\curveto(666.1404384,624.4978203)(666.17543837,624.4028204)(666.20544434,624.30282959)
\curveto(666.23543831,624.2028206)(666.26043828,624.1028207)(666.28044434,624.00282959)
\curveto(666.30043824,623.97282083)(666.30543824,623.94282086)(666.29544434,623.91282959)
\curveto(666.29543825,623.88282092)(666.30043824,623.85282095)(666.31044434,623.82282959)
\curveto(666.3404382,623.71282109)(666.36043818,623.58782121)(666.37044434,623.44782959)
\curveto(666.38043816,623.31782148)(666.39043815,623.18282162)(666.40044434,623.04282959)
\lineto(666.40044434,622.87782959)
\curveto(666.41043813,622.81782198)(666.41043813,622.76282204)(666.40044434,622.71282959)
\curveto(666.39043815,622.66282214)(666.38543816,622.61282219)(666.38544434,622.56282959)
\lineto(666.38544434,622.42782959)
\curveto(666.37543817,622.38782241)(666.37043817,622.34782245)(666.37044434,622.30782959)
\curveto(666.38043816,622.26782253)(666.37543817,622.22282258)(666.35544434,622.17282959)
\curveto(666.33543821,622.06282274)(666.31543823,621.95782284)(666.29544434,621.85782959)
\curveto(666.28543826,621.75782304)(666.26543828,621.65782314)(666.23544434,621.55782959)
\curveto(666.10543844,621.1978236)(665.9404386,620.88282392)(665.74044434,620.61282959)
\curveto(665.540439,620.34282446)(665.26543928,620.13782466)(664.91544434,619.99782959)
\curveto(664.83543971,619.96782483)(664.75043979,619.94282486)(664.66044434,619.92282959)
\lineto(664.39044434,619.86282959)
\curveto(664.3404402,619.85282495)(664.29544025,619.84782495)(664.25544434,619.84782959)
\curveto(664.21544033,619.85782494)(664.17544037,619.85782494)(664.13544434,619.84782959)
\curveto(664.03544051,619.82782497)(663.9404406,619.82782497)(663.85044434,619.84782959)
\moveto(663.01044434,621.24282959)
\curveto(663.05044149,621.17282363)(663.09044145,621.10782369)(663.13044434,621.04782959)
\curveto(663.17044137,620.9978238)(663.22044132,620.94782385)(663.28044434,620.89782959)
\lineto(663.43044434,620.77782959)
\curveto(663.49044105,620.74782405)(663.55544099,620.72282408)(663.62544434,620.70282959)
\curveto(663.66544088,620.68282412)(663.70044084,620.67282413)(663.73044434,620.67282959)
\curveto(663.77044077,620.68282412)(663.81044073,620.67782412)(663.85044434,620.65782959)
\curveto(663.88044066,620.65782414)(663.92044062,620.65282415)(663.97044434,620.64282959)
\curveto(664.02044052,620.64282416)(664.06044048,620.64782415)(664.09044434,620.65782959)
\lineto(664.31544434,620.70282959)
\curveto(664.56543998,620.78282402)(664.75043979,620.90782389)(664.87044434,621.07782959)
\curveto(664.95043959,621.17782362)(665.02043952,621.30782349)(665.08044434,621.46782959)
\curveto(665.16043938,621.64782315)(665.22043932,621.87282293)(665.26044434,622.14282959)
\curveto(665.30043924,622.42282238)(665.31543923,622.7028221)(665.30544434,622.98282959)
\curveto(665.29543925,623.27282153)(665.26543928,623.54782125)(665.21544434,623.80782959)
\curveto(665.16543938,624.06782073)(665.09043945,624.27782052)(664.99044434,624.43782959)
\curveto(664.87043967,624.63782016)(664.72043982,624.78782001)(664.54044434,624.88782959)
\curveto(664.46044008,624.93781986)(664.37044017,624.96781983)(664.27044434,624.97782959)
\curveto(664.17044037,624.9978198)(664.06544048,625.00781979)(663.95544434,625.00782959)
\curveto(663.93544061,624.9978198)(663.91044063,624.99281981)(663.88044434,624.99282959)
\curveto(663.86044068,625.0028198)(663.8404407,625.0028198)(663.82044434,624.99282959)
\curveto(663.77044077,624.98281982)(663.72544082,624.97281983)(663.68544434,624.96282959)
\curveto(663.6454409,624.96281984)(663.60544094,624.95281985)(663.56544434,624.93282959)
\curveto(663.38544116,624.85281995)(663.23544131,624.73282007)(663.11544434,624.57282959)
\curveto(663.00544154,624.41282039)(662.91544163,624.23282057)(662.84544434,624.03282959)
\curveto(662.78544176,623.84282096)(662.7404418,623.61782118)(662.71044434,623.35782959)
\curveto(662.69044185,623.0978217)(662.68544186,622.83282197)(662.69544434,622.56282959)
\curveto(662.70544184,622.3028225)(662.73544181,622.05282275)(662.78544434,621.81282959)
\curveto(662.8454417,621.58282322)(662.92044162,621.39282341)(663.01044434,621.24282959)
\moveto(673.81044434,618.25782959)
\curveto(673.82043072,618.20782659)(673.82543072,618.11782668)(673.82544434,617.98782959)
\curveto(673.82543072,617.85782694)(673.81543073,617.76782703)(673.79544434,617.71782959)
\curveto(673.77543077,617.66782713)(673.77043077,617.61282719)(673.78044434,617.55282959)
\curveto(673.79043075,617.5028273)(673.79043075,617.45282735)(673.78044434,617.40282959)
\curveto(673.7404308,617.26282754)(673.71043083,617.12782767)(673.69044434,616.99782959)
\curveto(673.68043086,616.86782793)(673.65043089,616.74782805)(673.60044434,616.63782959)
\curveto(673.46043108,616.28782851)(673.29543125,615.99282881)(673.10544434,615.75282959)
\curveto(672.91543163,615.52282928)(672.6454319,615.33782946)(672.29544434,615.19782959)
\curveto(672.21543233,615.16782963)(672.13043241,615.14782965)(672.04044434,615.13782959)
\curveto(671.95043259,615.11782968)(671.86543268,615.0978297)(671.78544434,615.07782959)
\curveto(671.73543281,615.06782973)(671.68543286,615.06282974)(671.63544434,615.06282959)
\curveto(671.58543296,615.06282974)(671.53543301,615.05782974)(671.48544434,615.04782959)
\curveto(671.45543309,615.03782976)(671.40543314,615.03782976)(671.33544434,615.04782959)
\curveto(671.26543328,615.04782975)(671.21543333,615.05282975)(671.18544434,615.06282959)
\curveto(671.12543342,615.08282972)(671.06543348,615.09282971)(671.00544434,615.09282959)
\curveto(670.95543359,615.08282972)(670.90543364,615.08782971)(670.85544434,615.10782959)
\curveto(670.76543378,615.12782967)(670.67543387,615.15282965)(670.58544434,615.18282959)
\curveto(670.50543404,615.2028296)(670.42543412,615.23282957)(670.34544434,615.27282959)
\curveto(670.02543452,615.41282939)(669.77543477,615.60782919)(669.59544434,615.85782959)
\curveto(669.41543513,616.11782868)(669.26543528,616.42282838)(669.14544434,616.77282959)
\curveto(669.12543542,616.85282795)(669.11043543,616.93782786)(669.10044434,617.02782959)
\curveto(669.09043545,617.11782768)(669.07543547,617.2028276)(669.05544434,617.28282959)
\curveto(669.0454355,617.31282749)(669.0404355,617.34282746)(669.04044434,617.37282959)
\lineto(669.04044434,617.47782959)
\curveto(669.02043552,617.55782724)(669.01043553,617.63782716)(669.01044434,617.71782959)
\lineto(669.01044434,617.85282959)
\curveto(668.99043555,617.95282685)(668.99043555,618.05282675)(669.01044434,618.15282959)
\lineto(669.01044434,618.33282959)
\curveto(669.02043552,618.38282642)(669.02543552,618.42782637)(669.02544434,618.46782959)
\curveto(669.02543552,618.51782628)(669.03043551,618.56282624)(669.04044434,618.60282959)
\curveto(669.05043549,618.64282616)(669.05543549,618.67782612)(669.05544434,618.70782959)
\curveto(669.05543549,618.74782605)(669.06043548,618.78782601)(669.07044434,618.82782959)
\lineto(669.13044434,619.15782959)
\curveto(669.15043539,619.27782552)(669.18043536,619.38782541)(669.22044434,619.48782959)
\curveto(669.36043518,619.81782498)(669.52043502,620.09282471)(669.70044434,620.31282959)
\curveto(669.89043465,620.54282426)(670.15043439,620.72782407)(670.48044434,620.86782959)
\curveto(670.56043398,620.90782389)(670.6454339,620.93282387)(670.73544434,620.94282959)
\lineto(671.03544434,621.00282959)
\lineto(671.17044434,621.00282959)
\curveto(671.22043332,621.01282379)(671.27043327,621.01782378)(671.32044434,621.01782959)
\curveto(671.89043265,621.03782376)(672.35043219,620.93282387)(672.70044434,620.70282959)
\curveto(673.06043148,620.48282432)(673.32543122,620.18282462)(673.49544434,619.80282959)
\curveto(673.545431,619.7028251)(673.58543096,619.6028252)(673.61544434,619.50282959)
\curveto(673.6454309,619.4028254)(673.67543087,619.2978255)(673.70544434,619.18782959)
\curveto(673.71543083,619.14782565)(673.72043082,619.11282569)(673.72044434,619.08282959)
\curveto(673.72043082,619.06282574)(673.72543082,619.03282577)(673.73544434,618.99282959)
\curveto(673.75543079,618.92282588)(673.76543078,618.84782595)(673.76544434,618.76782959)
\curveto(673.76543078,618.68782611)(673.77543077,618.60782619)(673.79544434,618.52782959)
\curveto(673.79543075,618.47782632)(673.79543075,618.43282637)(673.79544434,618.39282959)
\curveto(673.79543075,618.35282645)(673.80043074,618.30782649)(673.81044434,618.25782959)
\moveto(672.70044434,617.82282959)
\curveto(672.71043183,617.87282693)(672.71543183,617.94782685)(672.71544434,618.04782959)
\curveto(672.72543182,618.14782665)(672.72043182,618.22282658)(672.70044434,618.27282959)
\curveto(672.68043186,618.33282647)(672.67543187,618.38782641)(672.68544434,618.43782959)
\curveto(672.70543184,618.4978263)(672.70543184,618.55782624)(672.68544434,618.61782959)
\curveto(672.67543187,618.64782615)(672.67043187,618.68282612)(672.67044434,618.72282959)
\curveto(672.67043187,618.76282604)(672.66543188,618.802826)(672.65544434,618.84282959)
\curveto(672.63543191,618.92282588)(672.61543193,618.9978258)(672.59544434,619.06782959)
\curveto(672.58543196,619.14782565)(672.57043197,619.22782557)(672.55044434,619.30782959)
\curveto(672.52043202,619.36782543)(672.49543205,619.42782537)(672.47544434,619.48782959)
\curveto(672.45543209,619.54782525)(672.42543212,619.60782519)(672.38544434,619.66782959)
\curveto(672.28543226,619.83782496)(672.15543239,619.97282483)(671.99544434,620.07282959)
\curveto(671.91543263,620.12282468)(671.82043272,620.15782464)(671.71044434,620.17782959)
\curveto(671.60043294,620.1978246)(671.47543307,620.20782459)(671.33544434,620.20782959)
\curveto(671.31543323,620.1978246)(671.29043325,620.19282461)(671.26044434,620.19282959)
\curveto(671.23043331,620.2028246)(671.20043334,620.2028246)(671.17044434,620.19282959)
\lineto(671.02044434,620.13282959)
\curveto(670.97043357,620.12282468)(670.92543362,620.10782469)(670.88544434,620.08782959)
\curveto(670.69543385,619.97782482)(670.55043399,619.83282497)(670.45044434,619.65282959)
\curveto(670.36043418,619.47282533)(670.28043426,619.26782553)(670.21044434,619.03782959)
\curveto(670.17043437,618.90782589)(670.15043439,618.77282603)(670.15044434,618.63282959)
\curveto(670.15043439,618.5028263)(670.1404344,618.35782644)(670.12044434,618.19782959)
\curveto(670.11043443,618.14782665)(670.10043444,618.08782671)(670.09044434,618.01782959)
\curveto(670.09043445,617.94782685)(670.10043444,617.88782691)(670.12044434,617.83782959)
\lineto(670.12044434,617.67282959)
\lineto(670.12044434,617.49282959)
\curveto(670.13043441,617.44282736)(670.1404344,617.38782741)(670.15044434,617.32782959)
\curveto(670.16043438,617.27782752)(670.16543438,617.22282758)(670.16544434,617.16282959)
\curveto(670.17543437,617.1028277)(670.19043435,617.04782775)(670.21044434,616.99782959)
\curveto(670.26043428,616.80782799)(670.32043422,616.63282817)(670.39044434,616.47282959)
\curveto(670.46043408,616.31282849)(670.56543398,616.18282862)(670.70544434,616.08282959)
\curveto(670.83543371,615.98282882)(670.97543357,615.91282889)(671.12544434,615.87282959)
\curveto(671.15543339,615.86282894)(671.18043336,615.85782894)(671.20044434,615.85782959)
\curveto(671.23043331,615.86782893)(671.26043328,615.86782893)(671.29044434,615.85782959)
\curveto(671.31043323,615.85782894)(671.3404332,615.85282895)(671.38044434,615.84282959)
\curveto(671.42043312,615.84282896)(671.45543309,615.84782895)(671.48544434,615.85782959)
\curveto(671.52543302,615.86782893)(671.56543298,615.87282893)(671.60544434,615.87282959)
\curveto(671.6454329,615.87282893)(671.68543286,615.88282892)(671.72544434,615.90282959)
\curveto(671.96543258,615.98282882)(672.16043238,616.11782868)(672.31044434,616.30782959)
\curveto(672.43043211,616.48782831)(672.52043202,616.69282811)(672.58044434,616.92282959)
\curveto(672.60043194,616.99282781)(672.61543193,617.06282774)(672.62544434,617.13282959)
\curveto(672.63543191,617.21282759)(672.65043189,617.29282751)(672.67044434,617.37282959)
\curveto(672.67043187,617.43282737)(672.67543187,617.47782732)(672.68544434,617.50782959)
\curveto(672.68543186,617.52782727)(672.68543186,617.55282725)(672.68544434,617.58282959)
\curveto(672.68543186,617.62282718)(672.69043185,617.65282715)(672.70044434,617.67282959)
\lineto(672.70044434,617.82282959)
}
}
{
\newrgbcolor{curcolor}{0 0 0}
\pscustom[linestyle=none,fillstyle=solid,fillcolor=curcolor]
{
\newpath
\moveto(541.5636377,401.87430176)
\curveto(541.57362998,401.83429871)(541.57362998,401.78429876)(541.5636377,401.72430176)
\curveto(541.56362999,401.66429888)(541.55862999,401.61429893)(541.5486377,401.57430176)
\curveto(541.54863,401.53429901)(541.54363001,401.49429905)(541.5336377,401.45430176)
\lineto(541.5336377,401.34930176)
\curveto(541.51363004,401.26929927)(541.49863005,401.18929935)(541.4886377,401.10930176)
\curveto(541.47863007,401.02929951)(541.45863009,400.95429959)(541.4286377,400.88430176)
\curveto(541.40863014,400.80429974)(541.38863016,400.72929981)(541.3686377,400.65930176)
\curveto(541.3486302,400.58929995)(541.31863023,400.51430003)(541.2786377,400.43430176)
\curveto(541.09863045,400.01430053)(540.84363071,399.67430087)(540.5136377,399.41430176)
\curveto(540.18363137,399.15430139)(539.79363176,398.94930159)(539.3436377,398.79930176)
\curveto(539.22363233,398.75930178)(539.09863245,398.73430181)(538.9686377,398.72430176)
\curveto(538.8486327,398.70430184)(538.72363283,398.67930186)(538.5936377,398.64930176)
\curveto(538.53363302,398.6393019)(538.46863308,398.63430191)(538.3986377,398.63430176)
\curveto(538.33863321,398.63430191)(538.27363328,398.62930191)(538.2036377,398.61930176)
\lineto(538.0836377,398.61930176)
\lineto(537.8886377,398.61930176)
\curveto(537.82863372,398.60930193)(537.77363378,398.61430193)(537.7236377,398.63430176)
\curveto(537.6536339,398.65430189)(537.58863396,398.65930188)(537.5286377,398.64930176)
\curveto(537.46863408,398.6393019)(537.40863414,398.6443019)(537.3486377,398.66430176)
\curveto(537.29863425,398.67430187)(537.2536343,398.67930186)(537.2136377,398.67930176)
\curveto(537.17363438,398.67930186)(537.12863442,398.68930185)(537.0786377,398.70930176)
\curveto(536.99863455,398.72930181)(536.92363463,398.74930179)(536.8536377,398.76930176)
\curveto(536.78363477,398.77930176)(536.71363484,398.79430175)(536.6436377,398.81430176)
\curveto(536.16363539,398.98430156)(535.76363579,399.19430135)(535.4436377,399.44430176)
\curveto(535.13363642,399.70430084)(534.88363667,400.05930048)(534.6936377,400.50930176)
\curveto(534.66363689,400.56929997)(534.63863691,400.62929991)(534.6186377,400.68930176)
\curveto(534.60863694,400.75929978)(534.59363696,400.83429971)(534.5736377,400.91430176)
\curveto(534.553637,400.97429957)(534.53863701,401.0392995)(534.5286377,401.10930176)
\curveto(534.51863703,401.17929936)(534.50363705,401.24929929)(534.4836377,401.31930176)
\curveto(534.47363708,401.36929917)(534.46863708,401.40929913)(534.4686377,401.43930176)
\lineto(534.4686377,401.55930176)
\curveto(534.45863709,401.59929894)(534.4486371,401.64929889)(534.4386377,401.70930176)
\curveto(534.43863711,401.76929877)(534.44363711,401.81929872)(534.4536377,401.85930176)
\lineto(534.4536377,401.99430176)
\curveto(534.46363709,402.0442985)(534.46863708,402.09429845)(534.4686377,402.14430176)
\curveto(534.48863706,402.2442983)(534.50363705,402.3392982)(534.5136377,402.42930176)
\curveto(534.52363703,402.52929801)(534.54363701,402.62429792)(534.5736377,402.71430176)
\curveto(534.62363693,402.86429768)(534.67863687,403.00429754)(534.7386377,403.13430176)
\curveto(534.79863675,403.26429728)(534.86863668,403.38429716)(534.9486377,403.49430176)
\curveto(534.97863657,403.544297)(535.00863654,403.58429696)(535.0386377,403.61430176)
\curveto(535.07863647,403.6442969)(535.11363644,403.67929686)(535.1436377,403.71930176)
\curveto(535.20363635,403.79929674)(535.27363628,403.86929667)(535.3536377,403.92930176)
\curveto(535.41363614,403.97929656)(535.47363608,404.02429652)(535.5336377,404.06430176)
\lineto(535.7436377,404.21430176)
\curveto(535.79363576,404.25429629)(535.84363571,404.28929625)(535.8936377,404.31930176)
\curveto(535.94363561,404.35929618)(535.97863557,404.41429613)(535.9986377,404.48430176)
\curveto(535.99863555,404.51429603)(535.98863556,404.539296)(535.9686377,404.55930176)
\curveto(535.95863559,404.58929595)(535.9486356,404.61429593)(535.9386377,404.63430176)
\curveto(535.89863565,404.68429586)(535.8486357,404.72929581)(535.7886377,404.76930176)
\curveto(535.73863581,404.81929572)(535.68863586,404.86429568)(535.6386377,404.90430176)
\curveto(535.59863595,404.93429561)(535.548636,404.98929555)(535.4886377,405.06930176)
\curveto(535.46863608,405.09929544)(535.43863611,405.12429542)(535.3986377,405.14430176)
\curveto(535.36863618,405.17429537)(535.34363621,405.20929533)(535.3236377,405.24930176)
\curveto(535.1536364,405.45929508)(535.02363653,405.70429484)(534.9336377,405.98430176)
\curveto(534.91363664,406.06429448)(534.89863665,406.1442944)(534.8886377,406.22430176)
\curveto(534.87863667,406.30429424)(534.86363669,406.38429416)(534.8436377,406.46430176)
\curveto(534.82363673,406.51429403)(534.81363674,406.57929396)(534.8136377,406.65930176)
\curveto(534.81363674,406.74929379)(534.82363673,406.81929372)(534.8436377,406.86930176)
\curveto(534.84363671,406.96929357)(534.8486367,407.0392935)(534.8586377,407.07930176)
\curveto(534.87863667,407.15929338)(534.89363666,407.22929331)(534.9036377,407.28930176)
\curveto(534.91363664,407.35929318)(534.92863662,407.42929311)(534.9486377,407.49930176)
\curveto(535.09863645,407.92929261)(535.31363624,408.27429227)(535.5936377,408.53430176)
\curveto(535.88363567,408.79429175)(536.23363532,409.00929153)(536.6436377,409.17930176)
\curveto(536.7536348,409.22929131)(536.86863468,409.25929128)(536.9886377,409.26930176)
\curveto(537.11863443,409.28929125)(537.2486343,409.31929122)(537.3786377,409.35930176)
\curveto(537.45863409,409.35929118)(537.52863402,409.35929118)(537.5886377,409.35930176)
\curveto(537.65863389,409.36929117)(537.73363382,409.37929116)(537.8136377,409.38930176)
\curveto(538.60363295,409.40929113)(539.25863229,409.27929126)(539.7786377,408.99930176)
\curveto(540.30863124,408.71929182)(540.68863086,408.30929223)(540.9186377,407.76930176)
\curveto(541.02863052,407.539293)(541.09863045,407.25429329)(541.1286377,406.91430176)
\curveto(541.16863038,406.58429396)(541.13863041,406.27929426)(541.0386377,405.99930176)
\curveto(540.99863055,405.86929467)(540.9486306,405.74929479)(540.8886377,405.63930176)
\curveto(540.83863071,405.52929501)(540.77863077,405.42429512)(540.7086377,405.32430176)
\curveto(540.68863086,405.28429526)(540.65863089,405.24929529)(540.6186377,405.21930176)
\lineto(540.5286377,405.12930176)
\curveto(540.47863107,405.0392955)(540.41863113,404.97429557)(540.3486377,404.93430176)
\curveto(540.29863125,404.88429566)(540.24363131,404.83429571)(540.1836377,404.78430176)
\curveto(540.13363142,404.7442958)(540.08863146,404.69929584)(540.0486377,404.64930176)
\curveto(540.02863152,404.62929591)(540.00863154,404.60429594)(539.9886377,404.57430176)
\curveto(539.97863157,404.55429599)(539.97863157,404.52929601)(539.9886377,404.49930176)
\curveto(539.99863155,404.44929609)(540.02863152,404.39929614)(540.0786377,404.34930176)
\curveto(540.12863142,404.30929623)(540.18363137,404.26929627)(540.2436377,404.22930176)
\lineto(540.4236377,404.10930176)
\curveto(540.48363107,404.07929646)(540.53363102,404.04929649)(540.5736377,404.01930176)
\curveto(540.90363065,403.77929676)(541.1536304,403.46929707)(541.3236377,403.08930176)
\curveto(541.36363019,403.00929753)(541.39363016,402.92429762)(541.4136377,402.83430176)
\curveto(541.44363011,402.7442978)(541.46863008,402.65429789)(541.4886377,402.56430176)
\curveto(541.49863005,402.51429803)(541.50863004,402.45929808)(541.5186377,402.39930176)
\lineto(541.5486377,402.24930176)
\curveto(541.55862999,402.18929835)(541.55862999,402.12429842)(541.5486377,402.05430176)
\curveto(541.53863001,401.99429855)(541.54363001,401.93429861)(541.5636377,401.87430176)
\moveto(536.1786377,406.91430176)
\curveto(536.1486354,406.80429374)(536.14363541,406.66429388)(536.1636377,406.49430176)
\curveto(536.18363537,406.33429421)(536.20863534,406.20929433)(536.2386377,406.11930176)
\curveto(536.3486352,405.79929474)(536.49863505,405.55429499)(536.6886377,405.38430176)
\curveto(536.87863467,405.22429532)(537.14363441,405.09429545)(537.4836377,404.99430176)
\curveto(537.61363394,404.96429558)(537.77863377,404.9392956)(537.9786377,404.91930176)
\curveto(538.17863337,404.90929563)(538.3486332,404.92429562)(538.4886377,404.96430176)
\curveto(538.77863277,405.0442955)(539.01863253,405.15429539)(539.2086377,405.29430176)
\curveto(539.40863214,405.4442951)(539.56363199,405.6442949)(539.6736377,405.89430176)
\curveto(539.69363186,405.9442946)(539.70363185,405.98929455)(539.7036377,406.02930176)
\curveto(539.71363184,406.06929447)(539.72863182,406.11429443)(539.7486377,406.16430176)
\curveto(539.77863177,406.27429427)(539.79863175,406.41429413)(539.8086377,406.58430176)
\curveto(539.81863173,406.75429379)(539.80863174,406.89929364)(539.7786377,407.01930176)
\curveto(539.75863179,407.10929343)(539.73363182,407.19429335)(539.7036377,407.27430176)
\curveto(539.68363187,407.35429319)(539.6486319,407.43429311)(539.5986377,407.51430176)
\curveto(539.42863212,407.78429276)(539.20363235,407.97929256)(538.9236377,408.09930176)
\curveto(538.6536329,408.21929232)(538.29363326,408.27929226)(537.8436377,408.27930176)
\curveto(537.82363373,408.25929228)(537.79363376,408.25429229)(537.7536377,408.26430176)
\curveto(537.71363384,408.27429227)(537.67863387,408.27429227)(537.6486377,408.26430176)
\curveto(537.59863395,408.2442923)(537.54363401,408.22929231)(537.4836377,408.21930176)
\curveto(537.43363412,408.21929232)(537.38363417,408.20929233)(537.3336377,408.18930176)
\curveto(537.09363446,408.09929244)(536.88363467,407.98429256)(536.7036377,407.84430176)
\curveto(536.52363503,407.71429283)(536.38363517,407.53429301)(536.2836377,407.30430176)
\curveto(536.26363529,407.2442933)(536.24363531,407.17929336)(536.2236377,407.10930176)
\curveto(536.21363534,407.04929349)(536.19863535,406.98429356)(536.1786377,406.91430176)
\moveto(540.1986377,401.37930176)
\curveto(540.2486313,401.56929897)(540.2536313,401.77429877)(540.2136377,401.99430176)
\curveto(540.18363137,402.21429833)(540.13863141,402.39429815)(540.0786377,402.53430176)
\curveto(539.90863164,402.90429764)(539.6486319,403.20929733)(539.2986377,403.44930176)
\curveto(538.95863259,403.68929685)(538.52363303,403.80929673)(537.9936377,403.80930176)
\curveto(537.96363359,403.78929675)(537.92363363,403.78429676)(537.8736377,403.79430176)
\curveto(537.82363373,403.81429673)(537.78363377,403.81929672)(537.7536377,403.80930176)
\lineto(537.4836377,403.74930176)
\curveto(537.40363415,403.7392968)(537.32363423,403.72429682)(537.2436377,403.70430176)
\curveto(536.94363461,403.59429695)(536.67863487,403.44929709)(536.4486377,403.26930176)
\curveto(536.22863532,403.08929745)(536.05863549,402.85929768)(535.9386377,402.57930176)
\curveto(535.90863564,402.49929804)(535.88363567,402.41929812)(535.8636377,402.33930176)
\curveto(535.84363571,402.25929828)(535.82363573,402.17429837)(535.8036377,402.08430176)
\curveto(535.77363578,401.96429858)(535.76363579,401.81429873)(535.7736377,401.63430176)
\curveto(535.79363576,401.45429909)(535.81863573,401.31429923)(535.8486377,401.21430176)
\curveto(535.86863568,401.16429938)(535.87863567,401.11929942)(535.8786377,401.07930176)
\curveto(535.88863566,401.04929949)(535.90363565,401.00929953)(535.9236377,400.95930176)
\curveto(536.02363553,400.7392998)(536.1536354,400.5393)(536.3136377,400.35930176)
\curveto(536.48363507,400.17930036)(536.67863487,400.0443005)(536.8986377,399.95430176)
\curveto(536.96863458,399.91430063)(537.06363449,399.87930066)(537.1836377,399.84930176)
\curveto(537.40363415,399.75930078)(537.65863389,399.71430083)(537.9486377,399.71430176)
\lineto(538.2336377,399.71430176)
\curveto(538.33363322,399.73430081)(538.42863312,399.74930079)(538.5186377,399.75930176)
\curveto(538.60863294,399.76930077)(538.69863285,399.78930075)(538.7886377,399.81930176)
\curveto(539.0486325,399.89930064)(539.28863226,400.02930051)(539.5086377,400.20930176)
\curveto(539.73863181,400.39930014)(539.90863164,400.61429993)(540.0186377,400.85430176)
\curveto(540.05863149,400.93429961)(540.08863146,401.01429953)(540.1086377,401.09430176)
\curveto(540.13863141,401.18429936)(540.16863138,401.27929926)(540.1986377,401.37930176)
}
}
{
\newrgbcolor{curcolor}{0 0 0}
\pscustom[linestyle=none,fillstyle=solid,fillcolor=curcolor]
{
\newpath
\moveto(546.95824707,409.40430176)
\curveto(547.05824222,409.40429114)(547.15324212,409.39429115)(547.24324707,409.37430176)
\curveto(547.33324194,409.36429118)(547.39824188,409.33429121)(547.43824707,409.28430176)
\curveto(547.49824178,409.20429134)(547.52824175,409.09929144)(547.52824707,408.96930176)
\lineto(547.52824707,408.57930176)
\lineto(547.52824707,407.07930176)
\lineto(547.52824707,400.68930176)
\lineto(547.52824707,399.51930176)
\lineto(547.52824707,399.20430176)
\curveto(547.53824174,399.10430144)(547.52324175,399.02430152)(547.48324707,398.96430176)
\curveto(547.43324184,398.88430166)(547.35824192,398.83430171)(547.25824707,398.81430176)
\curveto(547.16824211,398.80430174)(547.05824222,398.79930174)(546.92824707,398.79930176)
\lineto(546.70324707,398.79930176)
\curveto(546.62324265,398.81930172)(546.55324272,398.83430171)(546.49324707,398.84430176)
\curveto(546.43324284,398.86430168)(546.38324289,398.90430164)(546.34324707,398.96430176)
\curveto(546.30324297,399.02430152)(546.28324299,399.09930144)(546.28324707,399.18930176)
\lineto(546.28324707,399.48930176)
\lineto(546.28324707,400.58430176)
\lineto(546.28324707,405.92430176)
\curveto(546.26324301,406.01429453)(546.24824303,406.08929445)(546.23824707,406.14930176)
\curveto(546.23824304,406.21929432)(546.20824307,406.27929426)(546.14824707,406.32930176)
\curveto(546.0782432,406.37929416)(545.98824329,406.40429414)(545.87824707,406.40430176)
\curveto(545.7782435,406.41429413)(545.66824361,406.41929412)(545.54824707,406.41930176)
\lineto(544.40824707,406.41930176)
\lineto(543.91324707,406.41930176)
\curveto(543.75324552,406.42929411)(543.64324563,406.48929405)(543.58324707,406.59930176)
\curveto(543.56324571,406.62929391)(543.55324572,406.65929388)(543.55324707,406.68930176)
\curveto(543.55324572,406.72929381)(543.54824573,406.77429377)(543.53824707,406.82430176)
\curveto(543.51824576,406.9442936)(543.52324575,407.05429349)(543.55324707,407.15430176)
\curveto(543.59324568,407.25429329)(543.64824563,407.32429322)(543.71824707,407.36430176)
\curveto(543.79824548,407.41429313)(543.91824536,407.4392931)(544.07824707,407.43930176)
\curveto(544.23824504,407.4392931)(544.3732449,407.45429309)(544.48324707,407.48430176)
\curveto(544.53324474,407.49429305)(544.58824469,407.49929304)(544.64824707,407.49930176)
\curveto(544.70824457,407.50929303)(544.76824451,407.52429302)(544.82824707,407.54430176)
\curveto(544.9782443,407.59429295)(545.12324415,407.6442929)(545.26324707,407.69430176)
\curveto(545.40324387,407.75429279)(545.53824374,407.82429272)(545.66824707,407.90430176)
\curveto(545.80824347,407.99429255)(545.92824335,408.09929244)(546.02824707,408.21930176)
\curveto(546.12824315,408.3392922)(546.22324305,408.46929207)(546.31324707,408.60930176)
\curveto(546.3732429,408.70929183)(546.41824286,408.81929172)(546.44824707,408.93930176)
\curveto(546.48824279,409.05929148)(546.53824274,409.16429138)(546.59824707,409.25430176)
\curveto(546.64824263,409.31429123)(546.71824256,409.35429119)(546.80824707,409.37430176)
\curveto(546.82824245,409.38429116)(546.85324242,409.38929115)(546.88324707,409.38930176)
\curveto(546.91324236,409.38929115)(546.93824234,409.39429115)(546.95824707,409.40430176)
}
}
{
\newrgbcolor{curcolor}{0 0 0}
\pscustom[linestyle=none,fillstyle=solid,fillcolor=curcolor]
{
\newpath
\moveto(552.20285645,400.43430176)
\lineto(552.50285645,400.43430176)
\curveto(552.61285439,400.4443001)(552.71785428,400.4443001)(552.81785645,400.43430176)
\curveto(552.92785407,400.43430011)(553.02785397,400.42430012)(553.11785645,400.40430176)
\curveto(553.20785379,400.39430015)(553.27785372,400.36930017)(553.32785645,400.32930176)
\curveto(553.34785365,400.30930023)(553.36285364,400.27930026)(553.37285645,400.23930176)
\curveto(553.39285361,400.19930034)(553.41285359,400.15430039)(553.43285645,400.10430176)
\lineto(553.43285645,400.02930176)
\curveto(553.44285356,399.97930056)(553.44285356,399.92430062)(553.43285645,399.86430176)
\lineto(553.43285645,399.71430176)
\lineto(553.43285645,399.23430176)
\curveto(553.43285357,399.06430148)(553.39285361,398.9443016)(553.31285645,398.87430176)
\curveto(553.24285376,398.82430172)(553.15285385,398.79930174)(553.04285645,398.79930176)
\lineto(552.71285645,398.79930176)
\lineto(552.26285645,398.79930176)
\curveto(552.11285489,398.79930174)(551.997855,398.82930171)(551.91785645,398.88930176)
\curveto(551.87785512,398.91930162)(551.84785515,398.96930157)(551.82785645,399.03930176)
\curveto(551.80785519,399.11930142)(551.79285521,399.20430134)(551.78285645,399.29430176)
\lineto(551.78285645,399.57930176)
\curveto(551.79285521,399.67930086)(551.7978552,399.76430078)(551.79785645,399.83430176)
\lineto(551.79785645,400.02930176)
\curveto(551.7978552,400.08930045)(551.80785519,400.1443004)(551.82785645,400.19430176)
\curveto(551.86785513,400.30430024)(551.93785506,400.37430017)(552.03785645,400.40430176)
\curveto(552.06785493,400.40430014)(552.12285488,400.41430013)(552.20285645,400.43430176)
}
}
{
\newrgbcolor{curcolor}{0 0 0}
\pscustom[linestyle=none,fillstyle=solid,fillcolor=curcolor]
{
\newpath
\moveto(562.4230127,401.87430176)
\curveto(562.43300498,401.83429871)(562.43300498,401.78429876)(562.4230127,401.72430176)
\curveto(562.42300499,401.66429888)(562.41800499,401.61429893)(562.4080127,401.57430176)
\curveto(562.408005,401.53429901)(562.40300501,401.49429905)(562.3930127,401.45430176)
\lineto(562.3930127,401.34930176)
\curveto(562.37300504,401.26929927)(562.35800505,401.18929935)(562.3480127,401.10930176)
\curveto(562.33800507,401.02929951)(562.31800509,400.95429959)(562.2880127,400.88430176)
\curveto(562.26800514,400.80429974)(562.24800516,400.72929981)(562.2280127,400.65930176)
\curveto(562.2080052,400.58929995)(562.17800523,400.51430003)(562.1380127,400.43430176)
\curveto(561.95800545,400.01430053)(561.70300571,399.67430087)(561.3730127,399.41430176)
\curveto(561.04300637,399.15430139)(560.65300676,398.94930159)(560.2030127,398.79930176)
\curveto(560.08300733,398.75930178)(559.95800745,398.73430181)(559.8280127,398.72430176)
\curveto(559.7080077,398.70430184)(559.58300783,398.67930186)(559.4530127,398.64930176)
\curveto(559.39300802,398.6393019)(559.32800808,398.63430191)(559.2580127,398.63430176)
\curveto(559.19800821,398.63430191)(559.13300828,398.62930191)(559.0630127,398.61930176)
\lineto(558.9430127,398.61930176)
\lineto(558.7480127,398.61930176)
\curveto(558.68800872,398.60930193)(558.63300878,398.61430193)(558.5830127,398.63430176)
\curveto(558.5130089,398.65430189)(558.44800896,398.65930188)(558.3880127,398.64930176)
\curveto(558.32800908,398.6393019)(558.26800914,398.6443019)(558.2080127,398.66430176)
\curveto(558.15800925,398.67430187)(558.1130093,398.67930186)(558.0730127,398.67930176)
\curveto(558.03300938,398.67930186)(557.98800942,398.68930185)(557.9380127,398.70930176)
\curveto(557.85800955,398.72930181)(557.78300963,398.74930179)(557.7130127,398.76930176)
\curveto(557.64300977,398.77930176)(557.57300984,398.79430175)(557.5030127,398.81430176)
\curveto(557.02301039,398.98430156)(556.62301079,399.19430135)(556.3030127,399.44430176)
\curveto(555.99301142,399.70430084)(555.74301167,400.05930048)(555.5530127,400.50930176)
\curveto(555.52301189,400.56929997)(555.49801191,400.62929991)(555.4780127,400.68930176)
\curveto(555.46801194,400.75929978)(555.45301196,400.83429971)(555.4330127,400.91430176)
\curveto(555.413012,400.97429957)(555.39801201,401.0392995)(555.3880127,401.10930176)
\curveto(555.37801203,401.17929936)(555.36301205,401.24929929)(555.3430127,401.31930176)
\curveto(555.33301208,401.36929917)(555.32801208,401.40929913)(555.3280127,401.43930176)
\lineto(555.3280127,401.55930176)
\curveto(555.31801209,401.59929894)(555.3080121,401.64929889)(555.2980127,401.70930176)
\curveto(555.29801211,401.76929877)(555.30301211,401.81929872)(555.3130127,401.85930176)
\lineto(555.3130127,401.99430176)
\curveto(555.32301209,402.0442985)(555.32801208,402.09429845)(555.3280127,402.14430176)
\curveto(555.34801206,402.2442983)(555.36301205,402.3392982)(555.3730127,402.42930176)
\curveto(555.38301203,402.52929801)(555.40301201,402.62429792)(555.4330127,402.71430176)
\curveto(555.48301193,402.86429768)(555.53801187,403.00429754)(555.5980127,403.13430176)
\curveto(555.65801175,403.26429728)(555.72801168,403.38429716)(555.8080127,403.49430176)
\curveto(555.83801157,403.544297)(555.86801154,403.58429696)(555.8980127,403.61430176)
\curveto(555.93801147,403.6442969)(555.97301144,403.67929686)(556.0030127,403.71930176)
\curveto(556.06301135,403.79929674)(556.13301128,403.86929667)(556.2130127,403.92930176)
\curveto(556.27301114,403.97929656)(556.33301108,404.02429652)(556.3930127,404.06430176)
\lineto(556.6030127,404.21430176)
\curveto(556.65301076,404.25429629)(556.70301071,404.28929625)(556.7530127,404.31930176)
\curveto(556.80301061,404.35929618)(556.83801057,404.41429613)(556.8580127,404.48430176)
\curveto(556.85801055,404.51429603)(556.84801056,404.539296)(556.8280127,404.55930176)
\curveto(556.81801059,404.58929595)(556.8080106,404.61429593)(556.7980127,404.63430176)
\curveto(556.75801065,404.68429586)(556.7080107,404.72929581)(556.6480127,404.76930176)
\curveto(556.59801081,404.81929572)(556.54801086,404.86429568)(556.4980127,404.90430176)
\curveto(556.45801095,404.93429561)(556.408011,404.98929555)(556.3480127,405.06930176)
\curveto(556.32801108,405.09929544)(556.29801111,405.12429542)(556.2580127,405.14430176)
\curveto(556.22801118,405.17429537)(556.20301121,405.20929533)(556.1830127,405.24930176)
\curveto(556.0130114,405.45929508)(555.88301153,405.70429484)(555.7930127,405.98430176)
\curveto(555.77301164,406.06429448)(555.75801165,406.1442944)(555.7480127,406.22430176)
\curveto(555.73801167,406.30429424)(555.72301169,406.38429416)(555.7030127,406.46430176)
\curveto(555.68301173,406.51429403)(555.67301174,406.57929396)(555.6730127,406.65930176)
\curveto(555.67301174,406.74929379)(555.68301173,406.81929372)(555.7030127,406.86930176)
\curveto(555.70301171,406.96929357)(555.7080117,407.0392935)(555.7180127,407.07930176)
\curveto(555.73801167,407.15929338)(555.75301166,407.22929331)(555.7630127,407.28930176)
\curveto(555.77301164,407.35929318)(555.78801162,407.42929311)(555.8080127,407.49930176)
\curveto(555.95801145,407.92929261)(556.17301124,408.27429227)(556.4530127,408.53430176)
\curveto(556.74301067,408.79429175)(557.09301032,409.00929153)(557.5030127,409.17930176)
\curveto(557.6130098,409.22929131)(557.72800968,409.25929128)(557.8480127,409.26930176)
\curveto(557.97800943,409.28929125)(558.1080093,409.31929122)(558.2380127,409.35930176)
\curveto(558.31800909,409.35929118)(558.38800902,409.35929118)(558.4480127,409.35930176)
\curveto(558.51800889,409.36929117)(558.59300882,409.37929116)(558.6730127,409.38930176)
\curveto(559.46300795,409.40929113)(560.11800729,409.27929126)(560.6380127,408.99930176)
\curveto(561.16800624,408.71929182)(561.54800586,408.30929223)(561.7780127,407.76930176)
\curveto(561.88800552,407.539293)(561.95800545,407.25429329)(561.9880127,406.91430176)
\curveto(562.02800538,406.58429396)(561.99800541,406.27929426)(561.8980127,405.99930176)
\curveto(561.85800555,405.86929467)(561.8080056,405.74929479)(561.7480127,405.63930176)
\curveto(561.69800571,405.52929501)(561.63800577,405.42429512)(561.5680127,405.32430176)
\curveto(561.54800586,405.28429526)(561.51800589,405.24929529)(561.4780127,405.21930176)
\lineto(561.3880127,405.12930176)
\curveto(561.33800607,405.0392955)(561.27800613,404.97429557)(561.2080127,404.93430176)
\curveto(561.15800625,404.88429566)(561.10300631,404.83429571)(561.0430127,404.78430176)
\curveto(560.99300642,404.7442958)(560.94800646,404.69929584)(560.9080127,404.64930176)
\curveto(560.88800652,404.62929591)(560.86800654,404.60429594)(560.8480127,404.57430176)
\curveto(560.83800657,404.55429599)(560.83800657,404.52929601)(560.8480127,404.49930176)
\curveto(560.85800655,404.44929609)(560.88800652,404.39929614)(560.9380127,404.34930176)
\curveto(560.98800642,404.30929623)(561.04300637,404.26929627)(561.1030127,404.22930176)
\lineto(561.2830127,404.10930176)
\curveto(561.34300607,404.07929646)(561.39300602,404.04929649)(561.4330127,404.01930176)
\curveto(561.76300565,403.77929676)(562.0130054,403.46929707)(562.1830127,403.08930176)
\curveto(562.22300519,403.00929753)(562.25300516,402.92429762)(562.2730127,402.83430176)
\curveto(562.30300511,402.7442978)(562.32800508,402.65429789)(562.3480127,402.56430176)
\curveto(562.35800505,402.51429803)(562.36800504,402.45929808)(562.3780127,402.39930176)
\lineto(562.4080127,402.24930176)
\curveto(562.41800499,402.18929835)(562.41800499,402.12429842)(562.4080127,402.05430176)
\curveto(562.39800501,401.99429855)(562.40300501,401.93429861)(562.4230127,401.87430176)
\moveto(557.0380127,406.91430176)
\curveto(557.0080104,406.80429374)(557.00301041,406.66429388)(557.0230127,406.49430176)
\curveto(557.04301037,406.33429421)(557.06801034,406.20929433)(557.0980127,406.11930176)
\curveto(557.2080102,405.79929474)(557.35801005,405.55429499)(557.5480127,405.38430176)
\curveto(557.73800967,405.22429532)(558.00300941,405.09429545)(558.3430127,404.99430176)
\curveto(558.47300894,404.96429558)(558.63800877,404.9392956)(558.8380127,404.91930176)
\curveto(559.03800837,404.90929563)(559.2080082,404.92429562)(559.3480127,404.96430176)
\curveto(559.63800777,405.0442955)(559.87800753,405.15429539)(560.0680127,405.29430176)
\curveto(560.26800714,405.4442951)(560.42300699,405.6442949)(560.5330127,405.89430176)
\curveto(560.55300686,405.9442946)(560.56300685,405.98929455)(560.5630127,406.02930176)
\curveto(560.57300684,406.06929447)(560.58800682,406.11429443)(560.6080127,406.16430176)
\curveto(560.63800677,406.27429427)(560.65800675,406.41429413)(560.6680127,406.58430176)
\curveto(560.67800673,406.75429379)(560.66800674,406.89929364)(560.6380127,407.01930176)
\curveto(560.61800679,407.10929343)(560.59300682,407.19429335)(560.5630127,407.27430176)
\curveto(560.54300687,407.35429319)(560.5080069,407.43429311)(560.4580127,407.51430176)
\curveto(560.28800712,407.78429276)(560.06300735,407.97929256)(559.7830127,408.09930176)
\curveto(559.5130079,408.21929232)(559.15300826,408.27929226)(558.7030127,408.27930176)
\curveto(558.68300873,408.25929228)(558.65300876,408.25429229)(558.6130127,408.26430176)
\curveto(558.57300884,408.27429227)(558.53800887,408.27429227)(558.5080127,408.26430176)
\curveto(558.45800895,408.2442923)(558.40300901,408.22929231)(558.3430127,408.21930176)
\curveto(558.29300912,408.21929232)(558.24300917,408.20929233)(558.1930127,408.18930176)
\curveto(557.95300946,408.09929244)(557.74300967,407.98429256)(557.5630127,407.84430176)
\curveto(557.38301003,407.71429283)(557.24301017,407.53429301)(557.1430127,407.30430176)
\curveto(557.12301029,407.2442933)(557.10301031,407.17929336)(557.0830127,407.10930176)
\curveto(557.07301034,407.04929349)(557.05801035,406.98429356)(557.0380127,406.91430176)
\moveto(561.0580127,401.37930176)
\curveto(561.1080063,401.56929897)(561.1130063,401.77429877)(561.0730127,401.99430176)
\curveto(561.04300637,402.21429833)(560.99800641,402.39429815)(560.9380127,402.53430176)
\curveto(560.76800664,402.90429764)(560.5080069,403.20929733)(560.1580127,403.44930176)
\curveto(559.81800759,403.68929685)(559.38300803,403.80929673)(558.8530127,403.80930176)
\curveto(558.82300859,403.78929675)(558.78300863,403.78429676)(558.7330127,403.79430176)
\curveto(558.68300873,403.81429673)(558.64300877,403.81929672)(558.6130127,403.80930176)
\lineto(558.3430127,403.74930176)
\curveto(558.26300915,403.7392968)(558.18300923,403.72429682)(558.1030127,403.70430176)
\curveto(557.80300961,403.59429695)(557.53800987,403.44929709)(557.3080127,403.26930176)
\curveto(557.08801032,403.08929745)(556.91801049,402.85929768)(556.7980127,402.57930176)
\curveto(556.76801064,402.49929804)(556.74301067,402.41929812)(556.7230127,402.33930176)
\curveto(556.70301071,402.25929828)(556.68301073,402.17429837)(556.6630127,402.08430176)
\curveto(556.63301078,401.96429858)(556.62301079,401.81429873)(556.6330127,401.63430176)
\curveto(556.65301076,401.45429909)(556.67801073,401.31429923)(556.7080127,401.21430176)
\curveto(556.72801068,401.16429938)(556.73801067,401.11929942)(556.7380127,401.07930176)
\curveto(556.74801066,401.04929949)(556.76301065,401.00929953)(556.7830127,400.95930176)
\curveto(556.88301053,400.7392998)(557.0130104,400.5393)(557.1730127,400.35930176)
\curveto(557.34301007,400.17930036)(557.53800987,400.0443005)(557.7580127,399.95430176)
\curveto(557.82800958,399.91430063)(557.92300949,399.87930066)(558.0430127,399.84930176)
\curveto(558.26300915,399.75930078)(558.51800889,399.71430083)(558.8080127,399.71430176)
\lineto(559.0930127,399.71430176)
\curveto(559.19300822,399.73430081)(559.28800812,399.74930079)(559.3780127,399.75930176)
\curveto(559.46800794,399.76930077)(559.55800785,399.78930075)(559.6480127,399.81930176)
\curveto(559.9080075,399.89930064)(560.14800726,400.02930051)(560.3680127,400.20930176)
\curveto(560.59800681,400.39930014)(560.76800664,400.61429993)(560.8780127,400.85430176)
\curveto(560.91800649,400.93429961)(560.94800646,401.01429953)(560.9680127,401.09430176)
\curveto(560.99800641,401.18429936)(561.02800638,401.27929926)(561.0580127,401.37930176)
}
}
{
\newrgbcolor{curcolor}{0 0 0}
\pscustom[linestyle=none,fillstyle=solid,fillcolor=curcolor]
{
\newpath
\moveto(573.56262207,407.31930176)
\curveto(573.36261177,407.02929351)(573.15261198,406.7442938)(572.93262207,406.46430176)
\curveto(572.72261241,406.18429436)(572.51761262,405.89929464)(572.31762207,405.60930176)
\curveto(571.71761342,404.75929578)(571.11261402,403.91929662)(570.50262207,403.08930176)
\curveto(569.89261524,402.26929827)(569.28761585,401.43429911)(568.68762207,400.58430176)
\lineto(568.17762207,399.86430176)
\lineto(567.66762207,399.17430176)
\curveto(567.58761755,399.06430148)(567.50761763,398.94930159)(567.42762207,398.82930176)
\curveto(567.34761779,398.70930183)(567.25261788,398.61430193)(567.14262207,398.54430176)
\curveto(567.10261803,398.52430202)(567.0376181,398.50930203)(566.94762207,398.49930176)
\curveto(566.86761827,398.47930206)(566.77761836,398.46930207)(566.67762207,398.46930176)
\curveto(566.57761856,398.46930207)(566.48261865,398.47430207)(566.39262207,398.48430176)
\curveto(566.31261882,398.49430205)(566.25261888,398.51430203)(566.21262207,398.54430176)
\curveto(566.18261895,398.56430198)(566.15761898,398.59930194)(566.13762207,398.64930176)
\curveto(566.12761901,398.68930185)(566.132619,398.73430181)(566.15262207,398.78430176)
\curveto(566.19261894,398.86430168)(566.2376189,398.9393016)(566.28762207,399.00930176)
\curveto(566.34761879,399.08930145)(566.40261873,399.16930137)(566.45262207,399.24930176)
\curveto(566.69261844,399.58930095)(566.9376182,399.92430062)(567.18762207,400.25430176)
\curveto(567.4376177,400.58429996)(567.67761746,400.91929962)(567.90762207,401.25930176)
\curveto(568.06761707,401.47929906)(568.22761691,401.69429885)(568.38762207,401.90430176)
\curveto(568.54761659,402.11429843)(568.70761643,402.32929821)(568.86762207,402.54930176)
\curveto(569.22761591,403.06929747)(569.59261554,403.57929696)(569.96262207,404.07930176)
\curveto(570.3326148,404.57929596)(570.70261443,405.08929545)(571.07262207,405.60930176)
\curveto(571.21261392,405.80929473)(571.35261378,406.00429454)(571.49262207,406.19430176)
\curveto(571.64261349,406.38429416)(571.78761335,406.57929396)(571.92762207,406.77930176)
\curveto(572.137613,407.07929346)(572.35261278,407.37929316)(572.57262207,407.67930176)
\lineto(573.23262207,408.57930176)
\lineto(573.41262207,408.84930176)
\lineto(573.62262207,409.11930176)
\lineto(573.74262207,409.29930176)
\curveto(573.79261134,409.35929118)(573.84261129,409.41429113)(573.89262207,409.46430176)
\curveto(573.96261117,409.51429103)(574.0376111,409.54929099)(574.11762207,409.56930176)
\curveto(574.137611,409.57929096)(574.16261097,409.57929096)(574.19262207,409.56930176)
\curveto(574.2326109,409.56929097)(574.26261087,409.57929096)(574.28262207,409.59930176)
\curveto(574.40261073,409.59929094)(574.5376106,409.59429095)(574.68762207,409.58430176)
\curveto(574.8376103,409.58429096)(574.92761021,409.539291)(574.95762207,409.44930176)
\curveto(574.97761016,409.41929112)(574.98261015,409.38429116)(574.97262207,409.34430176)
\curveto(574.96261017,409.30429124)(574.94761019,409.27429127)(574.92762207,409.25430176)
\curveto(574.88761025,409.17429137)(574.84761029,409.10429144)(574.80762207,409.04430176)
\curveto(574.76761037,408.98429156)(574.72261041,408.92429162)(574.67262207,408.86430176)
\lineto(574.10262207,408.08430176)
\curveto(573.92261121,407.83429271)(573.74261139,407.57929296)(573.56262207,407.31930176)
\moveto(566.70762207,403.41930176)
\curveto(566.65761848,403.4392971)(566.60761853,403.4442971)(566.55762207,403.43430176)
\curveto(566.50761863,403.42429712)(566.45761868,403.42929711)(566.40762207,403.44930176)
\curveto(566.29761884,403.46929707)(566.19261894,403.48929705)(566.09262207,403.50930176)
\curveto(566.00261913,403.539297)(565.90761923,403.57929696)(565.80762207,403.62930176)
\curveto(565.47761966,403.76929677)(565.22261991,403.96429658)(565.04262207,404.21430176)
\curveto(564.86262027,404.47429607)(564.71762042,404.78429576)(564.60762207,405.14430176)
\curveto(564.57762056,405.22429532)(564.55762058,405.30429524)(564.54762207,405.38430176)
\curveto(564.5376206,405.47429507)(564.52262061,405.55929498)(564.50262207,405.63930176)
\curveto(564.49262064,405.68929485)(564.48762065,405.75429479)(564.48762207,405.83430176)
\curveto(564.47762066,405.86429468)(564.47262066,405.89429465)(564.47262207,405.92430176)
\curveto(564.47262066,405.96429458)(564.46762067,405.99929454)(564.45762207,406.02930176)
\lineto(564.45762207,406.17930176)
\curveto(564.44762069,406.22929431)(564.44262069,406.28929425)(564.44262207,406.35930176)
\curveto(564.44262069,406.4392941)(564.44762069,406.50429404)(564.45762207,406.55430176)
\lineto(564.45762207,406.71930176)
\curveto(564.47762066,406.76929377)(564.48262065,406.81429373)(564.47262207,406.85430176)
\curveto(564.47262066,406.90429364)(564.47762066,406.94929359)(564.48762207,406.98930176)
\curveto(564.49762064,407.02929351)(564.50262063,407.06429348)(564.50262207,407.09430176)
\curveto(564.50262063,407.13429341)(564.50762063,407.17429337)(564.51762207,407.21430176)
\curveto(564.54762059,407.32429322)(564.56762057,407.43429311)(564.57762207,407.54430176)
\curveto(564.59762054,407.66429288)(564.6326205,407.77929276)(564.68262207,407.88930176)
\curveto(564.82262031,408.22929231)(564.98262015,408.50429204)(565.16262207,408.71430176)
\curveto(565.35261978,408.93429161)(565.62261951,409.11429143)(565.97262207,409.25430176)
\curveto(566.05261908,409.28429126)(566.137619,409.30429124)(566.22762207,409.31430176)
\curveto(566.31761882,409.33429121)(566.41261872,409.35429119)(566.51262207,409.37430176)
\curveto(566.54261859,409.38429116)(566.59761854,409.38429116)(566.67762207,409.37430176)
\curveto(566.75761838,409.37429117)(566.80761833,409.38429116)(566.82762207,409.40430176)
\curveto(567.38761775,409.41429113)(567.8376173,409.30429124)(568.17762207,409.07430176)
\curveto(568.52761661,408.8442917)(568.78761635,408.539292)(568.95762207,408.15930176)
\curveto(568.99761614,408.06929247)(569.0326161,407.97429257)(569.06262207,407.87430176)
\curveto(569.09261604,407.77429277)(569.11761602,407.67429287)(569.13762207,407.57430176)
\curveto(569.15761598,407.544293)(569.16261597,407.51429303)(569.15262207,407.48430176)
\curveto(569.15261598,407.45429309)(569.15761598,407.42429312)(569.16762207,407.39430176)
\curveto(569.19761594,407.28429326)(569.21761592,407.15929338)(569.22762207,407.01930176)
\curveto(569.2376159,406.88929365)(569.24761589,406.75429379)(569.25762207,406.61430176)
\lineto(569.25762207,406.44930176)
\curveto(569.26761587,406.38929415)(569.26761587,406.33429421)(569.25762207,406.28430176)
\curveto(569.24761589,406.23429431)(569.24261589,406.18429436)(569.24262207,406.13430176)
\lineto(569.24262207,405.99930176)
\curveto(569.2326159,405.95929458)(569.22761591,405.91929462)(569.22762207,405.87930176)
\curveto(569.2376159,405.8392947)(569.2326159,405.79429475)(569.21262207,405.74430176)
\curveto(569.19261594,405.63429491)(569.17261596,405.52929501)(569.15262207,405.42930176)
\curveto(569.14261599,405.32929521)(569.12261601,405.22929531)(569.09262207,405.12930176)
\curveto(568.96261617,404.76929577)(568.79761634,404.45429609)(568.59762207,404.18430176)
\curveto(568.39761674,403.91429663)(568.12261701,403.70929683)(567.77262207,403.56930176)
\curveto(567.69261744,403.539297)(567.60761753,403.51429703)(567.51762207,403.49430176)
\lineto(567.24762207,403.43430176)
\curveto(567.19761794,403.42429712)(567.15261798,403.41929712)(567.11262207,403.41930176)
\curveto(567.07261806,403.42929711)(567.0326181,403.42929711)(566.99262207,403.41930176)
\curveto(566.89261824,403.39929714)(566.79761834,403.39929714)(566.70762207,403.41930176)
\moveto(565.86762207,404.81430176)
\curveto(565.90761923,404.7442958)(565.94761919,404.67929586)(565.98762207,404.61930176)
\curveto(566.02761911,404.56929597)(566.07761906,404.51929602)(566.13762207,404.46930176)
\lineto(566.28762207,404.34930176)
\curveto(566.34761879,404.31929622)(566.41261872,404.29429625)(566.48262207,404.27430176)
\curveto(566.52261861,404.25429629)(566.55761858,404.2442963)(566.58762207,404.24430176)
\curveto(566.62761851,404.25429629)(566.66761847,404.24929629)(566.70762207,404.22930176)
\curveto(566.7376184,404.22929631)(566.77761836,404.22429632)(566.82762207,404.21430176)
\curveto(566.87761826,404.21429633)(566.91761822,404.21929632)(566.94762207,404.22930176)
\lineto(567.17262207,404.27430176)
\curveto(567.42261771,404.35429619)(567.60761753,404.47929606)(567.72762207,404.64930176)
\curveto(567.80761733,404.74929579)(567.87761726,404.87929566)(567.93762207,405.03930176)
\curveto(568.01761712,405.21929532)(568.07761706,405.4442951)(568.11762207,405.71430176)
\curveto(568.15761698,405.99429455)(568.17261696,406.27429427)(568.16262207,406.55430176)
\curveto(568.15261698,406.8442937)(568.12261701,407.11929342)(568.07262207,407.37930176)
\curveto(568.02261711,407.6392929)(567.94761719,407.84929269)(567.84762207,408.00930176)
\curveto(567.72761741,408.20929233)(567.57761756,408.35929218)(567.39762207,408.45930176)
\curveto(567.31761782,408.50929203)(567.22761791,408.539292)(567.12762207,408.54930176)
\curveto(567.02761811,408.56929197)(566.92261821,408.57929196)(566.81262207,408.57930176)
\curveto(566.79261834,408.56929197)(566.76761837,408.56429198)(566.73762207,408.56430176)
\curveto(566.71761842,408.57429197)(566.69761844,408.57429197)(566.67762207,408.56430176)
\curveto(566.62761851,408.55429199)(566.58261855,408.544292)(566.54262207,408.53430176)
\curveto(566.50261863,408.53429201)(566.46261867,408.52429202)(566.42262207,408.50430176)
\curveto(566.24261889,408.42429212)(566.09261904,408.30429224)(565.97262207,408.14430176)
\curveto(565.86261927,407.98429256)(565.77261936,407.80429274)(565.70262207,407.60430176)
\curveto(565.64261949,407.41429313)(565.59761954,407.18929335)(565.56762207,406.92930176)
\curveto(565.54761959,406.66929387)(565.54261959,406.40429414)(565.55262207,406.13430176)
\curveto(565.56261957,405.87429467)(565.59261954,405.62429492)(565.64262207,405.38430176)
\curveto(565.70261943,405.15429539)(565.77761936,404.96429558)(565.86762207,404.81430176)
\moveto(576.66762207,401.82930176)
\curveto(576.67760846,401.77929876)(576.68260845,401.68929885)(576.68262207,401.55930176)
\curveto(576.68260845,401.42929911)(576.67260846,401.3392992)(576.65262207,401.28930176)
\curveto(576.6326085,401.2392993)(576.62760851,401.18429936)(576.63762207,401.12430176)
\curveto(576.64760849,401.07429947)(576.64760849,401.02429952)(576.63762207,400.97430176)
\curveto(576.59760854,400.83429971)(576.56760857,400.69929984)(576.54762207,400.56930176)
\curveto(576.5376086,400.4393001)(576.50760863,400.31930022)(576.45762207,400.20930176)
\curveto(576.31760882,399.85930068)(576.15260898,399.56430098)(575.96262207,399.32430176)
\curveto(575.77260936,399.09430145)(575.50260963,398.90930163)(575.15262207,398.76930176)
\curveto(575.07261006,398.7393018)(574.98761015,398.71930182)(574.89762207,398.70930176)
\curveto(574.80761033,398.68930185)(574.72261041,398.66930187)(574.64262207,398.64930176)
\curveto(574.59261054,398.6393019)(574.54261059,398.63430191)(574.49262207,398.63430176)
\curveto(574.44261069,398.63430191)(574.39261074,398.62930191)(574.34262207,398.61930176)
\curveto(574.31261082,398.60930193)(574.26261087,398.60930193)(574.19262207,398.61930176)
\curveto(574.12261101,398.61930192)(574.07261106,398.62430192)(574.04262207,398.63430176)
\curveto(573.98261115,398.65430189)(573.92261121,398.66430188)(573.86262207,398.66430176)
\curveto(573.81261132,398.65430189)(573.76261137,398.65930188)(573.71262207,398.67930176)
\curveto(573.62261151,398.69930184)(573.5326116,398.72430182)(573.44262207,398.75430176)
\curveto(573.36261177,398.77430177)(573.28261185,398.80430174)(573.20262207,398.84430176)
\curveto(572.88261225,398.98430156)(572.6326125,399.17930136)(572.45262207,399.42930176)
\curveto(572.27261286,399.68930085)(572.12261301,399.99430055)(572.00262207,400.34430176)
\curveto(571.98261315,400.42430012)(571.96761317,400.50930003)(571.95762207,400.59930176)
\curveto(571.94761319,400.68929985)(571.9326132,400.77429977)(571.91262207,400.85430176)
\curveto(571.90261323,400.88429966)(571.89761324,400.91429963)(571.89762207,400.94430176)
\lineto(571.89762207,401.04930176)
\curveto(571.87761326,401.12929941)(571.86761327,401.20929933)(571.86762207,401.28930176)
\lineto(571.86762207,401.42430176)
\curveto(571.84761329,401.52429902)(571.84761329,401.62429892)(571.86762207,401.72430176)
\lineto(571.86762207,401.90430176)
\curveto(571.87761326,401.95429859)(571.88261325,401.99929854)(571.88262207,402.03930176)
\curveto(571.88261325,402.08929845)(571.88761325,402.13429841)(571.89762207,402.17430176)
\curveto(571.90761323,402.21429833)(571.91261322,402.24929829)(571.91262207,402.27930176)
\curveto(571.91261322,402.31929822)(571.91761322,402.35929818)(571.92762207,402.39930176)
\lineto(571.98762207,402.72930176)
\curveto(572.00761313,402.84929769)(572.0376131,402.95929758)(572.07762207,403.05930176)
\curveto(572.21761292,403.38929715)(572.37761276,403.66429688)(572.55762207,403.88430176)
\curveto(572.74761239,404.11429643)(573.00761213,404.29929624)(573.33762207,404.43930176)
\curveto(573.41761172,404.47929606)(573.50261163,404.50429604)(573.59262207,404.51430176)
\lineto(573.89262207,404.57430176)
\lineto(574.02762207,404.57430176)
\curveto(574.07761106,404.58429596)(574.12761101,404.58929595)(574.17762207,404.58930176)
\curveto(574.74761039,404.60929593)(575.20760993,404.50429604)(575.55762207,404.27430176)
\curveto(575.91760922,404.05429649)(576.18260895,403.75429679)(576.35262207,403.37430176)
\curveto(576.40260873,403.27429727)(576.44260869,403.17429737)(576.47262207,403.07430176)
\curveto(576.50260863,402.97429757)(576.5326086,402.86929767)(576.56262207,402.75930176)
\curveto(576.57260856,402.71929782)(576.57760856,402.68429786)(576.57762207,402.65430176)
\curveto(576.57760856,402.63429791)(576.58260855,402.60429794)(576.59262207,402.56430176)
\curveto(576.61260852,402.49429805)(576.62260851,402.41929812)(576.62262207,402.33930176)
\curveto(576.62260851,402.25929828)(576.6326085,402.17929836)(576.65262207,402.09930176)
\curveto(576.65260848,402.04929849)(576.65260848,402.00429854)(576.65262207,401.96430176)
\curveto(576.65260848,401.92429862)(576.65760848,401.87929866)(576.66762207,401.82930176)
\moveto(575.55762207,401.39430176)
\curveto(575.56760957,401.4442991)(575.57260956,401.51929902)(575.57262207,401.61930176)
\curveto(575.58260955,401.71929882)(575.57760956,401.79429875)(575.55762207,401.84430176)
\curveto(575.5376096,401.90429864)(575.5326096,401.95929858)(575.54262207,402.00930176)
\curveto(575.56260957,402.06929847)(575.56260957,402.12929841)(575.54262207,402.18930176)
\curveto(575.5326096,402.21929832)(575.52760961,402.25429829)(575.52762207,402.29430176)
\curveto(575.52760961,402.33429821)(575.52260961,402.37429817)(575.51262207,402.41430176)
\curveto(575.49260964,402.49429805)(575.47260966,402.56929797)(575.45262207,402.63930176)
\curveto(575.44260969,402.71929782)(575.42760971,402.79929774)(575.40762207,402.87930176)
\curveto(575.37760976,402.9392976)(575.35260978,402.99929754)(575.33262207,403.05930176)
\curveto(575.31260982,403.11929742)(575.28260985,403.17929736)(575.24262207,403.23930176)
\curveto(575.14260999,403.40929713)(575.01261012,403.544297)(574.85262207,403.64430176)
\curveto(574.77261036,403.69429685)(574.67761046,403.72929681)(574.56762207,403.74930176)
\curveto(574.45761068,403.76929677)(574.3326108,403.77929676)(574.19262207,403.77930176)
\curveto(574.17261096,403.76929677)(574.14761099,403.76429678)(574.11762207,403.76430176)
\curveto(574.08761105,403.77429677)(574.05761108,403.77429677)(574.02762207,403.76430176)
\lineto(573.87762207,403.70430176)
\curveto(573.82761131,403.69429685)(573.78261135,403.67929686)(573.74262207,403.65930176)
\curveto(573.55261158,403.54929699)(573.40761173,403.40429714)(573.30762207,403.22430176)
\curveto(573.21761192,403.0442975)(573.137612,402.8392977)(573.06762207,402.60930176)
\curveto(573.02761211,402.47929806)(573.00761213,402.3442982)(573.00762207,402.20430176)
\curveto(573.00761213,402.07429847)(572.99761214,401.92929861)(572.97762207,401.76930176)
\curveto(572.96761217,401.71929882)(572.95761218,401.65929888)(572.94762207,401.58930176)
\curveto(572.94761219,401.51929902)(572.95761218,401.45929908)(572.97762207,401.40930176)
\lineto(572.97762207,401.24430176)
\lineto(572.97762207,401.06430176)
\curveto(572.98761215,401.01429953)(572.99761214,400.95929958)(573.00762207,400.89930176)
\curveto(573.01761212,400.84929969)(573.02261211,400.79429975)(573.02262207,400.73430176)
\curveto(573.0326121,400.67429987)(573.04761209,400.61929992)(573.06762207,400.56930176)
\curveto(573.11761202,400.37930016)(573.17761196,400.20430034)(573.24762207,400.04430176)
\curveto(573.31761182,399.88430066)(573.42261171,399.75430079)(573.56262207,399.65430176)
\curveto(573.69261144,399.55430099)(573.8326113,399.48430106)(573.98262207,399.44430176)
\curveto(574.01261112,399.43430111)(574.0376111,399.42930111)(574.05762207,399.42930176)
\curveto(574.08761105,399.4393011)(574.11761102,399.4393011)(574.14762207,399.42930176)
\curveto(574.16761097,399.42930111)(574.19761094,399.42430112)(574.23762207,399.41430176)
\curveto(574.27761086,399.41430113)(574.31261082,399.41930112)(574.34262207,399.42930176)
\curveto(574.38261075,399.4393011)(574.42261071,399.4443011)(574.46262207,399.44430176)
\curveto(574.50261063,399.4443011)(574.54261059,399.45430109)(574.58262207,399.47430176)
\curveto(574.82261031,399.55430099)(575.01761012,399.68930085)(575.16762207,399.87930176)
\curveto(575.28760985,400.05930048)(575.37760976,400.26430028)(575.43762207,400.49430176)
\curveto(575.45760968,400.56429998)(575.47260966,400.63429991)(575.48262207,400.70430176)
\curveto(575.49260964,400.78429976)(575.50760963,400.86429968)(575.52762207,400.94430176)
\curveto(575.52760961,401.00429954)(575.5326096,401.04929949)(575.54262207,401.07930176)
\curveto(575.54260959,401.09929944)(575.54260959,401.12429942)(575.54262207,401.15430176)
\curveto(575.54260959,401.19429935)(575.54760959,401.22429932)(575.55762207,401.24430176)
\lineto(575.55762207,401.39430176)
}
}
{
\newrgbcolor{curcolor}{0 0 0}
\pscustom[linestyle=none,fillstyle=solid,fillcolor=curcolor]
{
\newpath
\moveto(239.32290161,270.83289063)
\curveto(239.42289676,270.83288)(239.51789666,270.82288002)(239.60790161,270.80289063)
\curveto(239.69789648,270.79288004)(239.76289642,270.76288007)(239.80290161,270.71289062)
\curveto(239.86289632,270.6328802)(239.89289629,270.52788031)(239.89290161,270.39789062)
\lineto(239.89290161,270.00789062)
\lineto(239.89290161,268.50789062)
\lineto(239.89290161,262.11789062)
\lineto(239.89290161,260.94789063)
\lineto(239.89290161,260.63289062)
\curveto(239.90289628,260.53289031)(239.88789629,260.45289039)(239.84790161,260.39289062)
\curveto(239.79789638,260.31289052)(239.72289646,260.26289058)(239.62290161,260.24289062)
\curveto(239.53289665,260.2328906)(239.42289676,260.22789061)(239.29290161,260.22789062)
\lineto(239.06790161,260.22789062)
\curveto(238.98789719,260.24789059)(238.91789726,260.26289058)(238.85790161,260.27289062)
\curveto(238.79789738,260.29289054)(238.74789743,260.33289051)(238.70790161,260.39289062)
\curveto(238.66789751,260.45289039)(238.64789753,260.52789031)(238.64790161,260.61789062)
\lineto(238.64790161,260.91789063)
\lineto(238.64790161,262.01289062)
\lineto(238.64790161,267.35289062)
\curveto(238.62789755,267.44288339)(238.61289757,267.51788332)(238.60290161,267.57789063)
\curveto(238.60289758,267.64788319)(238.57289761,267.70788313)(238.51290161,267.75789062)
\curveto(238.44289774,267.80788303)(238.35289783,267.832883)(238.24290161,267.83289063)
\curveto(238.14289804,267.84288299)(238.03289815,267.84788299)(237.91290161,267.84789062)
\lineto(236.77290161,267.84789062)
\lineto(236.27790161,267.84789062)
\curveto(236.11790006,267.85788298)(236.00790017,267.91788292)(235.94790161,268.02789063)
\curveto(235.92790025,268.05788278)(235.91790026,268.08788275)(235.91790161,268.11789062)
\curveto(235.91790026,268.15788268)(235.91290027,268.20288264)(235.90290161,268.25289062)
\curveto(235.8829003,268.37288246)(235.88790029,268.48288236)(235.91790161,268.58289063)
\curveto(235.95790022,268.68288216)(236.01290017,268.75288209)(236.08290161,268.79289063)
\curveto(236.16290002,268.84288199)(236.2828999,268.86788197)(236.44290161,268.86789062)
\curveto(236.60289958,268.86788197)(236.73789944,268.88288196)(236.84790161,268.91289063)
\curveto(236.89789928,268.92288191)(236.95289923,268.92788191)(237.01290161,268.92789063)
\curveto(237.07289911,268.9378819)(237.13289905,268.95288189)(237.19290161,268.97289062)
\curveto(237.34289884,269.02288182)(237.48789869,269.07288177)(237.62790161,269.12289062)
\curveto(237.76789841,269.18288165)(237.90289828,269.25288158)(238.03290161,269.33289063)
\curveto(238.17289801,269.42288142)(238.29289789,269.52788131)(238.39290161,269.64789062)
\curveto(238.49289769,269.76788107)(238.58789759,269.89788094)(238.67790161,270.03789063)
\curveto(238.73789744,270.1378807)(238.7828974,270.24788059)(238.81290161,270.36789062)
\curveto(238.85289733,270.48788035)(238.90289728,270.59288024)(238.96290161,270.68289063)
\curveto(239.01289717,270.7428801)(239.0828971,270.78288005)(239.17290161,270.80289063)
\curveto(239.19289699,270.81288003)(239.21789696,270.81788002)(239.24790161,270.81789063)
\curveto(239.2778969,270.81788002)(239.30289688,270.82288002)(239.32290161,270.83289063)
}
}
{
\newrgbcolor{curcolor}{0 0 0}
\pscustom[linestyle=none,fillstyle=solid,fillcolor=curcolor]
{
\newpath
\moveto(247.67251099,270.83289063)
\curveto(247.77250613,270.83288)(247.86750604,270.82288002)(247.95751099,270.80289063)
\curveto(248.04750586,270.79288004)(248.11250579,270.76288007)(248.15251099,270.71289062)
\curveto(248.21250569,270.6328802)(248.24250566,270.52788031)(248.24251099,270.39789062)
\lineto(248.24251099,270.00789062)
\lineto(248.24251099,268.50789062)
\lineto(248.24251099,262.11789062)
\lineto(248.24251099,260.94789063)
\lineto(248.24251099,260.63289062)
\curveto(248.25250565,260.53289031)(248.23750567,260.45289039)(248.19751099,260.39289062)
\curveto(248.14750576,260.31289052)(248.07250583,260.26289058)(247.97251099,260.24289062)
\curveto(247.88250602,260.2328906)(247.77250613,260.22789061)(247.64251099,260.22789062)
\lineto(247.41751099,260.22789062)
\curveto(247.33750657,260.24789059)(247.26750664,260.26289058)(247.20751099,260.27289062)
\curveto(247.14750676,260.29289054)(247.09750681,260.33289051)(247.05751099,260.39289062)
\curveto(247.01750689,260.45289039)(246.99750691,260.52789031)(246.99751099,260.61789062)
\lineto(246.99751099,260.91789063)
\lineto(246.99751099,262.01289062)
\lineto(246.99751099,267.35289062)
\curveto(246.97750693,267.44288339)(246.96250694,267.51788332)(246.95251099,267.57789063)
\curveto(246.95250695,267.64788319)(246.92250698,267.70788313)(246.86251099,267.75789062)
\curveto(246.79250711,267.80788303)(246.7025072,267.832883)(246.59251099,267.83289063)
\curveto(246.49250741,267.84288299)(246.38250752,267.84788299)(246.26251099,267.84789062)
\lineto(245.12251099,267.84789062)
\lineto(244.62751099,267.84789062)
\curveto(244.46750944,267.85788298)(244.35750955,267.91788292)(244.29751099,268.02789063)
\curveto(244.27750963,268.05788278)(244.26750964,268.08788275)(244.26751099,268.11789062)
\curveto(244.26750964,268.15788268)(244.26250964,268.20288264)(244.25251099,268.25289062)
\curveto(244.23250967,268.37288246)(244.23750967,268.48288236)(244.26751099,268.58289063)
\curveto(244.3075096,268.68288216)(244.36250954,268.75288209)(244.43251099,268.79289063)
\curveto(244.51250939,268.84288199)(244.63250927,268.86788197)(244.79251099,268.86789062)
\curveto(244.95250895,268.86788197)(245.08750882,268.88288196)(245.19751099,268.91289063)
\curveto(245.24750866,268.92288191)(245.3025086,268.92788191)(245.36251099,268.92789063)
\curveto(245.42250848,268.9378819)(245.48250842,268.95288189)(245.54251099,268.97289062)
\curveto(245.69250821,269.02288182)(245.83750807,269.07288177)(245.97751099,269.12289062)
\curveto(246.11750779,269.18288165)(246.25250765,269.25288158)(246.38251099,269.33289063)
\curveto(246.52250738,269.42288142)(246.64250726,269.52788131)(246.74251099,269.64789062)
\curveto(246.84250706,269.76788107)(246.93750697,269.89788094)(247.02751099,270.03789063)
\curveto(247.08750682,270.1378807)(247.13250677,270.24788059)(247.16251099,270.36789062)
\curveto(247.2025067,270.48788035)(247.25250665,270.59288024)(247.31251099,270.68289063)
\curveto(247.36250654,270.7428801)(247.43250647,270.78288005)(247.52251099,270.80289063)
\curveto(247.54250636,270.81288003)(247.56750634,270.81788002)(247.59751099,270.81789063)
\curveto(247.62750628,270.81788002)(247.65250625,270.82288002)(247.67251099,270.83289063)
}
}
{
\newrgbcolor{curcolor}{0 0 0}
\pscustom[linestyle=none,fillstyle=solid,fillcolor=curcolor]
{
\newpath
\moveto(252.91712036,261.86289062)
\lineto(253.21712036,261.86289062)
\curveto(253.3271183,261.87288897)(253.4321182,261.87288897)(253.53212036,261.86289062)
\curveto(253.64211799,261.86288898)(253.74211789,261.85288899)(253.83212036,261.83289063)
\curveto(253.92211771,261.82288901)(253.99211764,261.79788904)(254.04212036,261.75789062)
\curveto(254.06211757,261.7378891)(254.07711755,261.70788913)(254.08712036,261.66789063)
\curveto(254.10711752,261.62788921)(254.1271175,261.58288926)(254.14712036,261.53289063)
\lineto(254.14712036,261.45789063)
\curveto(254.15711747,261.40788943)(254.15711747,261.35288948)(254.14712036,261.29289063)
\lineto(254.14712036,261.14289062)
\lineto(254.14712036,260.66289063)
\curveto(254.14711748,260.49289034)(254.10711752,260.37289046)(254.02712036,260.30289063)
\curveto(253.95711767,260.25289059)(253.86711776,260.22789061)(253.75712036,260.22789062)
\lineto(253.42712036,260.22789062)
\lineto(252.97712036,260.22789062)
\curveto(252.8271188,260.22789061)(252.71211892,260.25789058)(252.63212036,260.31789063)
\curveto(252.59211904,260.34789049)(252.56211907,260.39789044)(252.54212036,260.46789062)
\curveto(252.52211911,260.54789029)(252.50711912,260.6328902)(252.49712036,260.72289062)
\lineto(252.49712036,261.00789062)
\curveto(252.50711912,261.10788973)(252.51211912,261.19288965)(252.51212036,261.26289062)
\lineto(252.51212036,261.45789063)
\curveto(252.51211912,261.51788932)(252.52211911,261.57288926)(252.54212036,261.62289062)
\curveto(252.58211905,261.73288911)(252.65211898,261.80288904)(252.75212036,261.83289063)
\curveto(252.78211885,261.832889)(252.83711879,261.84288899)(252.91712036,261.86289062)
}
}
{
\newrgbcolor{curcolor}{0 0 0}
\pscustom[linestyle=none,fillstyle=solid,fillcolor=curcolor]
{
\newpath
\moveto(263.13727661,263.30289063)
\curveto(263.14726889,263.26288758)(263.14726889,263.21288763)(263.13727661,263.15289063)
\curveto(263.1372689,263.09288774)(263.13226891,263.04288779)(263.12227661,263.00289062)
\curveto(263.12226892,262.96288787)(263.11726892,262.92288792)(263.10727661,262.88289062)
\lineto(263.10727661,262.77789063)
\curveto(263.08726895,262.69788814)(263.07226897,262.61788822)(263.06227661,262.53789063)
\curveto(263.05226899,262.45788838)(263.03226901,262.38288846)(263.00227661,262.31289063)
\curveto(262.98226906,262.2328886)(262.96226908,262.15788868)(262.94227661,262.08789062)
\curveto(262.92226912,262.01788882)(262.89226915,261.9428889)(262.85227661,261.86289062)
\curveto(262.67226937,261.44288939)(262.41726962,261.10288973)(262.08727661,260.84289062)
\curveto(261.75727028,260.58289026)(261.36727067,260.37789046)(260.91727661,260.22789062)
\curveto(260.79727124,260.18789065)(260.67227137,260.16289067)(260.54227661,260.15289063)
\curveto(260.42227162,260.13289071)(260.29727174,260.10789073)(260.16727661,260.07789063)
\curveto(260.10727193,260.06789077)(260.042272,260.06289078)(259.97227661,260.06289063)
\curveto(259.91227213,260.06289078)(259.84727219,260.05789078)(259.77727661,260.04789063)
\lineto(259.65727661,260.04789063)
\lineto(259.46227661,260.04789063)
\curveto(259.40227264,260.0378908)(259.34727269,260.04289079)(259.29727661,260.06289063)
\curveto(259.22727281,260.08289075)(259.16227288,260.08789075)(259.10227661,260.07789063)
\curveto(259.042273,260.06789077)(258.98227306,260.07289077)(258.92227661,260.09289062)
\curveto(258.87227317,260.10289073)(258.82727321,260.10789073)(258.78727661,260.10789062)
\curveto(258.74727329,260.10789073)(258.70227334,260.11789072)(258.65227661,260.13789062)
\curveto(258.57227347,260.15789068)(258.49727354,260.17789066)(258.42727661,260.19789063)
\curveto(258.35727368,260.20789063)(258.28727375,260.22289061)(258.21727661,260.24289062)
\curveto(257.7372743,260.41289043)(257.3372747,260.62289021)(257.01727661,260.87289062)
\curveto(256.70727533,261.13288971)(256.45727558,261.48788935)(256.26727661,261.93789063)
\curveto(256.2372758,261.99788884)(256.21227583,262.05788878)(256.19227661,262.11789062)
\curveto(256.18227586,262.18788865)(256.16727587,262.26288858)(256.14727661,262.34289062)
\curveto(256.12727591,262.40288844)(256.11227593,262.46788837)(256.10227661,262.53789063)
\curveto(256.09227595,262.60788823)(256.07727596,262.67788816)(256.05727661,262.74789062)
\curveto(256.04727599,262.79788804)(256.042276,262.837888)(256.04227661,262.86789062)
\lineto(256.04227661,262.98789062)
\curveto(256.03227601,263.02788781)(256.02227602,263.07788776)(256.01227661,263.13789062)
\curveto(256.01227603,263.19788764)(256.01727602,263.24788759)(256.02727661,263.28789063)
\lineto(256.02727661,263.42289063)
\curveto(256.037276,263.47288737)(256.042276,263.52288732)(256.04227661,263.57289063)
\curveto(256.06227598,263.67288717)(256.07727596,263.76788707)(256.08727661,263.85789062)
\curveto(256.09727594,263.95788688)(256.11727592,264.05288679)(256.14727661,264.14289062)
\curveto(256.19727584,264.29288654)(256.25227579,264.4328864)(256.31227661,264.56289063)
\curveto(256.37227567,264.69288614)(256.4422756,264.81288603)(256.52227661,264.92289063)
\curveto(256.55227549,264.97288586)(256.58227546,265.01288583)(256.61227661,265.04289063)
\curveto(256.65227539,265.07288577)(256.68727535,265.10788573)(256.71727661,265.14789062)
\curveto(256.77727526,265.22788561)(256.84727519,265.29788554)(256.92727661,265.35789062)
\curveto(256.98727505,265.40788543)(257.04727499,265.45288538)(257.10727661,265.49289062)
\lineto(257.31727661,265.64289062)
\curveto(257.36727467,265.68288516)(257.41727462,265.71788512)(257.46727661,265.74789062)
\curveto(257.51727452,265.78788505)(257.55227449,265.84288499)(257.57227661,265.91289063)
\curveto(257.57227447,265.9428849)(257.56227448,265.96788487)(257.54227661,265.98789062)
\curveto(257.53227451,266.01788482)(257.52227452,266.04288479)(257.51227661,266.06289063)
\curveto(257.47227457,266.11288472)(257.42227462,266.15788468)(257.36227661,266.19789063)
\curveto(257.31227473,266.24788459)(257.26227478,266.29288454)(257.21227661,266.33289063)
\curveto(257.17227487,266.36288447)(257.12227492,266.41788442)(257.06227661,266.49789062)
\curveto(257.042275,266.52788431)(257.01227503,266.55288429)(256.97227661,266.57289063)
\curveto(256.9422751,266.60288424)(256.91727512,266.6378842)(256.89727661,266.67789063)
\curveto(256.72727531,266.88788395)(256.59727544,267.13288371)(256.50727661,267.41289063)
\curveto(256.48727555,267.49288334)(256.47227557,267.57288326)(256.46227661,267.65289063)
\curveto(256.45227559,267.73288311)(256.4372756,267.81288303)(256.41727661,267.89289062)
\curveto(256.39727564,267.9428829)(256.38727565,268.00788283)(256.38727661,268.08789062)
\curveto(256.38727565,268.17788266)(256.39727564,268.24788259)(256.41727661,268.29789063)
\curveto(256.41727562,268.39788244)(256.42227562,268.46788237)(256.43227661,268.50789062)
\curveto(256.45227559,268.58788225)(256.46727557,268.65788218)(256.47727661,268.71789062)
\curveto(256.48727555,268.78788205)(256.50227554,268.85788198)(256.52227661,268.92789063)
\curveto(256.67227537,269.35788148)(256.88727515,269.70288113)(257.16727661,269.96289062)
\curveto(257.45727458,270.22288062)(257.80727423,270.4378804)(258.21727661,270.60789062)
\curveto(258.32727371,270.65788018)(258.4422736,270.68788015)(258.56227661,270.69789063)
\curveto(258.69227335,270.71788012)(258.82227322,270.74788009)(258.95227661,270.78789063)
\curveto(259.03227301,270.78788005)(259.10227294,270.78788005)(259.16227661,270.78789063)
\curveto(259.23227281,270.79788004)(259.30727273,270.80788003)(259.38727661,270.81789063)
\curveto(260.17727186,270.83788)(260.83227121,270.70788013)(261.35227661,270.42789063)
\curveto(261.88227016,270.14788069)(262.26226978,269.7378811)(262.49227661,269.19789063)
\curveto(262.60226944,268.96788187)(262.67226937,268.68288216)(262.70227661,268.34289062)
\curveto(262.7422693,268.01288283)(262.71226933,267.70788313)(262.61227661,267.42789063)
\curveto(262.57226947,267.29788354)(262.52226952,267.17788366)(262.46227661,267.06789063)
\curveto(262.41226963,266.95788388)(262.35226969,266.85288398)(262.28227661,266.75289062)
\curveto(262.26226978,266.71288412)(262.23226981,266.67788416)(262.19227661,266.64789062)
\lineto(262.10227661,266.55789063)
\curveto(262.05226999,266.46788437)(261.99227005,266.40288444)(261.92227661,266.36289062)
\curveto(261.87227017,266.31288452)(261.81727022,266.26288458)(261.75727661,266.21289062)
\curveto(261.70727033,266.17288466)(261.66227038,266.12788471)(261.62227661,266.07789063)
\curveto(261.60227044,266.05788478)(261.58227046,266.0328848)(261.56227661,266.00289062)
\curveto(261.55227049,265.98288485)(261.55227049,265.95788488)(261.56227661,265.92789063)
\curveto(261.57227047,265.87788496)(261.60227044,265.82788501)(261.65227661,265.77789063)
\curveto(261.70227034,265.7378851)(261.75727028,265.69788514)(261.81727661,265.65789063)
\lineto(261.99727661,265.53789063)
\curveto(262.05726998,265.50788533)(262.10726993,265.47788536)(262.14727661,265.44789063)
\curveto(262.47726956,265.20788563)(262.72726931,264.89788594)(262.89727661,264.51789062)
\curveto(262.9372691,264.4378864)(262.96726907,264.35288648)(262.98727661,264.26289062)
\curveto(263.01726902,264.17288666)(263.042269,264.08288676)(263.06227661,263.99289062)
\curveto(263.07226897,263.9428869)(263.08226896,263.88788695)(263.09227661,263.82789063)
\lineto(263.12227661,263.67789063)
\curveto(263.13226891,263.61788722)(263.13226891,263.55288728)(263.12227661,263.48289062)
\curveto(263.11226893,263.42288741)(263.11726892,263.36288747)(263.13727661,263.30289063)
\moveto(257.75227661,268.34289062)
\curveto(257.72227432,268.2328826)(257.71727432,268.09288274)(257.73727661,267.92289063)
\curveto(257.75727428,267.76288307)(257.78227426,267.6378832)(257.81227661,267.54789063)
\curveto(257.92227412,267.22788361)(258.07227397,266.98288385)(258.26227661,266.81289063)
\curveto(258.45227359,266.65288418)(258.71727332,266.52288432)(259.05727661,266.42289063)
\curveto(259.18727285,266.39288445)(259.35227269,266.36788447)(259.55227661,266.34789062)
\curveto(259.75227229,266.3378845)(259.92227212,266.35288449)(260.06227661,266.39289062)
\curveto(260.35227169,266.47288437)(260.59227145,266.58288425)(260.78227661,266.72289062)
\curveto(260.98227106,266.87288397)(261.1372709,267.07288377)(261.24727661,267.32289063)
\curveto(261.26727077,267.37288346)(261.27727076,267.41788342)(261.27727661,267.45789063)
\curveto(261.28727075,267.49788334)(261.30227074,267.5428833)(261.32227661,267.59289062)
\curveto(261.35227069,267.70288313)(261.37227067,267.84288299)(261.38227661,268.01289062)
\curveto(261.39227065,268.18288265)(261.38227066,268.32788251)(261.35227661,268.44789063)
\curveto(261.33227071,268.5378823)(261.30727073,268.62288222)(261.27727661,268.70289063)
\curveto(261.25727078,268.78288205)(261.22227082,268.86288198)(261.17227661,268.94289063)
\curveto(261.00227104,269.21288163)(260.77727126,269.40788143)(260.49727661,269.52789063)
\curveto(260.22727181,269.64788119)(259.86727217,269.70788113)(259.41727661,269.70789063)
\curveto(259.39727264,269.68788115)(259.36727267,269.68288116)(259.32727661,269.69289063)
\curveto(259.28727275,269.70288113)(259.25227279,269.70288113)(259.22227661,269.69289063)
\curveto(259.17227287,269.67288117)(259.11727292,269.65788118)(259.05727661,269.64789062)
\curveto(259.00727303,269.64788119)(258.95727308,269.6378812)(258.90727661,269.61789062)
\curveto(258.66727337,269.52788131)(258.45727358,269.41288143)(258.27727661,269.27289062)
\curveto(258.09727394,269.1428817)(257.95727408,268.96288187)(257.85727661,268.73289062)
\curveto(257.8372742,268.67288217)(257.81727422,268.60788223)(257.79727661,268.53789063)
\curveto(257.78727425,268.47788236)(257.77227427,268.41288243)(257.75227661,268.34289062)
\moveto(261.77227661,262.80789063)
\curveto(261.82227022,262.99788784)(261.82727021,263.20288764)(261.78727661,263.42289063)
\curveto(261.75727028,263.64288719)(261.71227033,263.82288701)(261.65227661,263.96289062)
\curveto(261.48227056,264.33288651)(261.22227082,264.6378862)(260.87227661,264.87789062)
\curveto(260.53227151,265.11788572)(260.09727194,265.2378856)(259.56727661,265.23789062)
\curveto(259.5372725,265.21788562)(259.49727254,265.21288563)(259.44727661,265.22289062)
\curveto(259.39727264,265.24288559)(259.35727268,265.24788559)(259.32727661,265.23789062)
\lineto(259.05727661,265.17789063)
\curveto(258.97727306,265.16788567)(258.89727314,265.15288569)(258.81727661,265.13289062)
\curveto(258.51727352,265.02288581)(258.25227379,264.87788596)(258.02227661,264.69789063)
\curveto(257.80227424,264.51788632)(257.63227441,264.28788655)(257.51227661,264.00789062)
\curveto(257.48227456,263.92788691)(257.45727458,263.84788699)(257.43727661,263.76789062)
\curveto(257.41727462,263.68788715)(257.39727464,263.60288724)(257.37727661,263.51289062)
\curveto(257.34727469,263.39288745)(257.3372747,263.24288759)(257.34727661,263.06289063)
\curveto(257.36727467,262.88288796)(257.39227465,262.7428881)(257.42227661,262.64289062)
\curveto(257.4422746,262.59288825)(257.45227459,262.54788829)(257.45227661,262.50789062)
\curveto(257.46227458,262.47788836)(257.47727456,262.4378884)(257.49727661,262.38789062)
\curveto(257.59727444,262.16788867)(257.72727431,261.96788887)(257.88727661,261.78789063)
\curveto(258.05727398,261.60788923)(258.25227379,261.47288937)(258.47227661,261.38289062)
\curveto(258.5422735,261.3428895)(258.6372734,261.30788953)(258.75727661,261.27789063)
\curveto(258.97727306,261.18788965)(259.23227281,261.1428897)(259.52227661,261.14289062)
\lineto(259.80727661,261.14289062)
\curveto(259.90727213,261.16288967)(260.00227204,261.17788966)(260.09227661,261.18789063)
\curveto(260.18227186,261.19788964)(260.27227177,261.21788962)(260.36227661,261.24789062)
\curveto(260.62227142,261.32788951)(260.86227118,261.45788938)(261.08227661,261.63789062)
\curveto(261.31227073,261.82788901)(261.48227056,262.04288879)(261.59227661,262.28289063)
\curveto(261.63227041,262.36288847)(261.66227038,262.44288839)(261.68227661,262.52289062)
\curveto(261.71227033,262.61288823)(261.7422703,262.70788813)(261.77227661,262.80789063)
}
}
{
\newrgbcolor{curcolor}{0 0 0}
\pscustom[linestyle=none,fillstyle=solid,fillcolor=curcolor]
{
\newpath
\moveto(274.27688599,268.74789062)
\curveto(274.07687569,268.45788238)(273.8668759,268.17288266)(273.64688599,267.89289062)
\curveto(273.43687633,267.61288323)(273.23187653,267.32788351)(273.03188599,267.03789063)
\curveto(272.43187733,266.18788465)(271.82687794,265.34788549)(271.21688599,264.51789062)
\curveto(270.60687916,263.69788714)(270.00187976,262.86288798)(269.40188599,262.01289062)
\lineto(268.89188599,261.29289063)
\lineto(268.38188599,260.60289062)
\curveto(268.30188146,260.49289034)(268.22188154,260.37789046)(268.14188599,260.25789062)
\curveto(268.0618817,260.1378907)(267.9668818,260.04289079)(267.85688599,259.97289062)
\curveto(267.81688195,259.95289088)(267.75188201,259.9378909)(267.66188599,259.92789063)
\curveto(267.58188218,259.90789093)(267.49188227,259.89789094)(267.39188599,259.89789062)
\curveto(267.29188247,259.89789094)(267.19688257,259.90289093)(267.10688599,259.91289063)
\curveto(267.02688274,259.92289092)(266.9668828,259.9428909)(266.92688599,259.97289062)
\curveto(266.89688287,259.99289085)(266.87188289,260.02789081)(266.85188599,260.07789063)
\curveto(266.84188292,260.11789072)(266.84688292,260.16289067)(266.86688599,260.21289062)
\curveto(266.90688286,260.29289054)(266.95188281,260.36789047)(267.00188599,260.43789063)
\curveto(267.0618827,260.51789032)(267.11688265,260.59789024)(267.16688599,260.67789063)
\curveto(267.40688236,261.01788982)(267.65188211,261.35288948)(267.90188599,261.68289063)
\curveto(268.15188161,262.01288883)(268.39188137,262.34788849)(268.62188599,262.68789063)
\curveto(268.78188098,262.90788793)(268.94188082,263.12288772)(269.10188599,263.33289063)
\curveto(269.2618805,263.5428873)(269.42188034,263.75788708)(269.58188599,263.97789062)
\curveto(269.94187982,264.49788634)(270.30687946,265.00788583)(270.67688599,265.50789062)
\curveto(271.04687872,266.00788483)(271.41687835,266.51788432)(271.78688599,267.03789063)
\curveto(271.92687784,267.2378836)(272.0668777,267.4328834)(272.20688599,267.62289062)
\curveto(272.35687741,267.81288303)(272.50187726,268.00788283)(272.64188599,268.20789063)
\curveto(272.85187691,268.50788233)(273.0668767,268.80788203)(273.28688599,269.10789062)
\lineto(273.94688599,270.00789062)
\lineto(274.12688599,270.27789063)
\lineto(274.33688599,270.54789063)
\lineto(274.45688599,270.72789062)
\curveto(274.50687526,270.78788005)(274.55687521,270.84287999)(274.60688599,270.89289062)
\curveto(274.67687509,270.9428799)(274.75187501,270.97787986)(274.83188599,270.99789062)
\curveto(274.85187491,271.00787983)(274.87687489,271.00787983)(274.90688599,270.99789062)
\curveto(274.94687482,270.99787984)(274.97687479,271.00787983)(274.99688599,271.02789063)
\curveto(275.11687465,271.02787981)(275.25187451,271.02287982)(275.40188599,271.01289062)
\curveto(275.55187421,271.01287983)(275.64187412,270.96787987)(275.67188599,270.87789062)
\curveto(275.69187407,270.84787999)(275.69687407,270.81288003)(275.68688599,270.77289062)
\curveto(275.67687409,270.73288011)(275.6618741,270.70288013)(275.64188599,270.68289063)
\curveto(275.60187416,270.60288024)(275.5618742,270.53288031)(275.52188599,270.47289062)
\curveto(275.48187428,270.41288043)(275.43687433,270.35288049)(275.38688599,270.29289063)
\lineto(274.81688599,269.51289062)
\curveto(274.63687513,269.26288158)(274.45687531,269.00788183)(274.27688599,268.74789062)
\moveto(267.42188599,264.84789062)
\curveto(267.37188239,264.86788597)(267.32188244,264.87288597)(267.27188599,264.86289062)
\curveto(267.22188254,264.85288598)(267.17188259,264.85788598)(267.12188599,264.87789062)
\curveto(267.01188275,264.89788594)(266.90688286,264.91788592)(266.80688599,264.93789063)
\curveto(266.71688305,264.96788587)(266.62188314,265.00788583)(266.52188599,265.05789063)
\curveto(266.19188357,265.19788564)(265.93688383,265.39288545)(265.75688599,265.64289062)
\curveto(265.57688419,265.90288493)(265.43188433,266.21288463)(265.32188599,266.57289063)
\curveto(265.29188447,266.65288418)(265.27188449,266.73288411)(265.26188599,266.81289063)
\curveto(265.25188451,266.90288393)(265.23688453,266.98788385)(265.21688599,267.06789063)
\curveto(265.20688456,267.11788372)(265.20188456,267.18288365)(265.20188599,267.26289062)
\curveto(265.19188457,267.29288354)(265.18688458,267.32288352)(265.18688599,267.35289062)
\curveto(265.18688458,267.39288345)(265.18188458,267.42788341)(265.17188599,267.45789063)
\lineto(265.17188599,267.60789062)
\curveto(265.1618846,267.65788318)(265.15688461,267.71788312)(265.15688599,267.78789063)
\curveto(265.15688461,267.86788297)(265.1618846,267.93288291)(265.17188599,267.98289062)
\lineto(265.17188599,268.14789062)
\curveto(265.19188457,268.19788264)(265.19688457,268.24288259)(265.18688599,268.28289063)
\curveto(265.18688458,268.33288251)(265.19188457,268.37788246)(265.20188599,268.41789063)
\curveto(265.21188455,268.45788238)(265.21688455,268.49288234)(265.21688599,268.52289062)
\curveto(265.21688455,268.56288227)(265.22188454,268.60288224)(265.23188599,268.64289062)
\curveto(265.2618845,268.75288209)(265.28188448,268.86288198)(265.29188599,268.97289062)
\curveto(265.31188445,269.09288174)(265.34688442,269.20788163)(265.39688599,269.31789063)
\curveto(265.53688423,269.65788118)(265.69688407,269.93288091)(265.87688599,270.14289062)
\curveto(266.0668837,270.36288047)(266.33688343,270.5428803)(266.68688599,270.68289063)
\curveto(266.766883,270.71288012)(266.85188291,270.73288011)(266.94188599,270.74289062)
\curveto(267.03188273,270.76288007)(267.12688264,270.78288005)(267.22688599,270.80289063)
\curveto(267.25688251,270.81288003)(267.31188245,270.81288003)(267.39188599,270.80289063)
\curveto(267.47188229,270.80288004)(267.52188224,270.81288003)(267.54188599,270.83289063)
\curveto(268.10188166,270.84287999)(268.55188121,270.73288011)(268.89188599,270.50289062)
\curveto(269.24188052,270.27288057)(269.50188026,269.96788087)(269.67188599,269.58789062)
\curveto(269.71188005,269.49788134)(269.74688002,269.40288144)(269.77688599,269.30289063)
\curveto(269.80687996,269.20288164)(269.83187993,269.10288173)(269.85188599,269.00289062)
\curveto(269.87187989,268.97288186)(269.87687989,268.9428819)(269.86688599,268.91289063)
\curveto(269.8668799,268.88288196)(269.87187989,268.85288198)(269.88188599,268.82289063)
\curveto(269.91187985,268.71288212)(269.93187983,268.58788225)(269.94188599,268.44789063)
\curveto(269.95187981,268.31788252)(269.9618798,268.18288265)(269.97188599,268.04289063)
\lineto(269.97188599,267.87789062)
\curveto(269.98187978,267.81788302)(269.98187978,267.76288307)(269.97188599,267.71289062)
\curveto(269.9618798,267.66288318)(269.95687981,267.61288323)(269.95688599,267.56289063)
\lineto(269.95688599,267.42789063)
\curveto(269.94687982,267.38788345)(269.94187982,267.34788349)(269.94188599,267.30789063)
\curveto(269.95187981,267.26788357)(269.94687982,267.22288362)(269.92688599,267.17289063)
\curveto(269.90687986,267.06288378)(269.88687988,266.95788388)(269.86688599,266.85789062)
\curveto(269.85687991,266.75788408)(269.83687993,266.65788418)(269.80688599,266.55789063)
\curveto(269.67688009,266.19788464)(269.51188025,265.88288496)(269.31188599,265.61289062)
\curveto(269.11188065,265.3428855)(268.83688093,265.1378857)(268.48688599,264.99789062)
\curveto(268.40688136,264.96788587)(268.32188144,264.9428859)(268.23188599,264.92289063)
\lineto(267.96188599,264.86289062)
\curveto(267.91188185,264.85288598)(267.8668819,264.84788599)(267.82688599,264.84789062)
\curveto(267.78688198,264.85788598)(267.74688202,264.85788598)(267.70688599,264.84789062)
\curveto(267.60688216,264.82788601)(267.51188225,264.82788601)(267.42188599,264.84789062)
\moveto(266.58188599,266.24289062)
\curveto(266.62188314,266.17288466)(266.6618831,266.10788473)(266.70188599,266.04789063)
\curveto(266.74188302,265.99788484)(266.79188297,265.94788489)(266.85188599,265.89789062)
\lineto(267.00188599,265.77789063)
\curveto(267.0618827,265.74788509)(267.12688264,265.72288512)(267.19688599,265.70289063)
\curveto(267.23688253,265.68288516)(267.27188249,265.67288517)(267.30188599,265.67289063)
\curveto(267.34188242,265.68288516)(267.38188238,265.67788516)(267.42188599,265.65789063)
\curveto(267.45188231,265.65788518)(267.49188227,265.65288518)(267.54188599,265.64289062)
\curveto(267.59188217,265.64288519)(267.63188213,265.64788519)(267.66188599,265.65789063)
\lineto(267.88688599,265.70289063)
\curveto(268.13688163,265.78288505)(268.32188144,265.90788493)(268.44188599,266.07789063)
\curveto(268.52188124,266.17788466)(268.59188117,266.30788453)(268.65188599,266.46789062)
\curveto(268.73188103,266.64788419)(268.79188097,266.87288397)(268.83188599,267.14289062)
\curveto(268.87188089,267.42288342)(268.88688088,267.70288313)(268.87688599,267.98289062)
\curveto(268.8668809,268.27288257)(268.83688093,268.54788229)(268.78688599,268.80789063)
\curveto(268.73688103,269.06788177)(268.6618811,269.27788156)(268.56188599,269.43789063)
\curveto(268.44188132,269.6378812)(268.29188147,269.78788105)(268.11188599,269.88789062)
\curveto(268.03188173,269.9378809)(267.94188182,269.96788087)(267.84188599,269.97789062)
\curveto(267.74188202,269.99788084)(267.63688213,270.00788083)(267.52688599,270.00789062)
\curveto(267.50688226,269.99788084)(267.48188228,269.99288084)(267.45188599,269.99289062)
\curveto(267.43188233,270.00288084)(267.41188235,270.00288084)(267.39188599,269.99289062)
\curveto(267.34188242,269.98288085)(267.29688247,269.97288086)(267.25688599,269.96289062)
\curveto(267.21688255,269.96288087)(267.17688259,269.95288089)(267.13688599,269.93289063)
\curveto(266.95688281,269.85288098)(266.80688296,269.73288111)(266.68688599,269.57289063)
\curveto(266.57688319,269.41288143)(266.48688328,269.2328816)(266.41688599,269.03289063)
\curveto(266.35688341,268.84288199)(266.31188345,268.61788222)(266.28188599,268.35789062)
\curveto(266.2618835,268.09788274)(266.25688351,267.832883)(266.26688599,267.56289063)
\curveto(266.27688349,267.30288353)(266.30688346,267.05288378)(266.35688599,266.81289063)
\curveto(266.41688335,266.58288425)(266.49188327,266.39288445)(266.58188599,266.24289062)
\moveto(277.38188599,263.25789062)
\curveto(277.39187237,263.20788763)(277.39687237,263.11788772)(277.39688599,262.98789062)
\curveto(277.39687237,262.85788798)(277.38687238,262.76788807)(277.36688599,262.71789062)
\curveto(277.34687242,262.66788817)(277.34187242,262.61288823)(277.35188599,262.55289063)
\curveto(277.3618724,262.50288833)(277.3618724,262.45288839)(277.35188599,262.40289063)
\curveto(277.31187245,262.26288858)(277.28187248,262.12788871)(277.26188599,261.99789062)
\curveto(277.25187251,261.86788897)(277.22187254,261.74788909)(277.17188599,261.63789062)
\curveto(277.03187273,261.28788955)(276.8668729,260.99288985)(276.67688599,260.75289062)
\curveto(276.48687328,260.52289032)(276.21687355,260.3378905)(275.86688599,260.19789063)
\curveto(275.78687398,260.16789067)(275.70187406,260.14789069)(275.61188599,260.13789062)
\curveto(275.52187424,260.11789072)(275.43687433,260.09789074)(275.35688599,260.07789063)
\curveto(275.30687446,260.06789077)(275.25687451,260.06289078)(275.20688599,260.06289063)
\curveto(275.15687461,260.06289078)(275.10687466,260.05789078)(275.05688599,260.04789063)
\curveto(275.02687474,260.0378908)(274.97687479,260.0378908)(274.90688599,260.04789063)
\curveto(274.83687493,260.04789079)(274.78687498,260.05289079)(274.75688599,260.06289063)
\curveto(274.69687507,260.08289075)(274.63687513,260.09289074)(274.57688599,260.09289062)
\curveto(274.52687524,260.08289075)(274.47687529,260.08789075)(274.42688599,260.10789062)
\curveto(274.33687543,260.12789071)(274.24687552,260.15289068)(274.15688599,260.18289063)
\curveto(274.07687569,260.20289064)(273.99687577,260.2328906)(273.91688599,260.27289062)
\curveto(273.59687617,260.41289043)(273.34687642,260.60789023)(273.16688599,260.85789062)
\curveto(272.98687678,261.11788972)(272.83687693,261.42288941)(272.71688599,261.77289062)
\curveto(272.69687707,261.85288899)(272.68187708,261.9378889)(272.67188599,262.02789063)
\curveto(272.6618771,262.11788872)(272.64687712,262.20288864)(272.62688599,262.28289063)
\curveto(272.61687715,262.31288852)(272.61187715,262.3428885)(272.61188599,262.37289062)
\lineto(272.61188599,262.47789062)
\curveto(272.59187717,262.55788828)(272.58187718,262.6378882)(272.58188599,262.71789062)
\lineto(272.58188599,262.85289062)
\curveto(272.5618772,262.95288788)(272.5618772,263.05288779)(272.58188599,263.15289063)
\lineto(272.58188599,263.33289063)
\curveto(272.59187717,263.38288745)(272.59687717,263.42788741)(272.59688599,263.46789062)
\curveto(272.59687717,263.51788732)(272.60187716,263.56288727)(272.61188599,263.60289062)
\curveto(272.62187714,263.64288719)(272.62687714,263.67788716)(272.62688599,263.70789063)
\curveto(272.62687714,263.74788709)(272.63187713,263.78788705)(272.64188599,263.82789063)
\lineto(272.70188599,264.15789063)
\curveto(272.72187704,264.27788656)(272.75187701,264.38788645)(272.79188599,264.48789062)
\curveto(272.93187683,264.81788602)(273.09187667,265.09288574)(273.27188599,265.31289063)
\curveto(273.4618763,265.5428853)(273.72187604,265.72788511)(274.05188599,265.86789062)
\curveto(274.13187563,265.90788493)(274.21687555,265.93288491)(274.30688599,265.94289063)
\lineto(274.60688599,266.00289062)
\lineto(274.74188599,266.00289062)
\curveto(274.79187497,266.01288483)(274.84187492,266.01788482)(274.89188599,266.01789062)
\curveto(275.4618743,266.0378848)(275.92187384,265.93288491)(276.27188599,265.70289063)
\curveto(276.63187313,265.48288536)(276.89687287,265.18288565)(277.06688599,264.80289063)
\curveto(277.11687265,264.70288613)(277.15687261,264.60288624)(277.18688599,264.50289062)
\curveto(277.21687255,264.40288644)(277.24687252,264.29788654)(277.27688599,264.18789063)
\curveto(277.28687248,264.14788669)(277.29187247,264.11288672)(277.29188599,264.08289063)
\curveto(277.29187247,264.06288678)(277.29687247,264.0328868)(277.30688599,263.99289062)
\curveto(277.32687244,263.92288692)(277.33687243,263.84788699)(277.33688599,263.76789062)
\curveto(277.33687243,263.68788715)(277.34687242,263.60788723)(277.36688599,263.52789063)
\curveto(277.3668724,263.47788736)(277.3668724,263.4328874)(277.36688599,263.39289062)
\curveto(277.3668724,263.35288748)(277.37187239,263.30788753)(277.38188599,263.25789062)
\moveto(276.27188599,262.82289063)
\curveto(276.28187348,262.87288797)(276.28687348,262.94788789)(276.28688599,263.04789063)
\curveto(276.29687347,263.14788769)(276.29187347,263.22288761)(276.27188599,263.27289062)
\curveto(276.25187351,263.33288751)(276.24687352,263.38788745)(276.25688599,263.43789063)
\curveto(276.27687349,263.49788734)(276.27687349,263.55788728)(276.25688599,263.61789062)
\curveto(276.24687352,263.64788719)(276.24187352,263.68288716)(276.24188599,263.72289062)
\curveto(276.24187352,263.76288707)(276.23687353,263.80288704)(276.22688599,263.84289062)
\curveto(276.20687356,263.92288692)(276.18687358,263.99788684)(276.16688599,264.06789063)
\curveto(276.15687361,264.14788669)(276.14187362,264.22788661)(276.12188599,264.30789063)
\curveto(276.09187367,264.36788647)(276.0668737,264.42788641)(276.04688599,264.48789062)
\curveto(276.02687374,264.54788629)(275.99687377,264.60788623)(275.95688599,264.66789063)
\curveto(275.85687391,264.837886)(275.72687404,264.97288586)(275.56688599,265.07289063)
\curveto(275.48687428,265.12288572)(275.39187437,265.15788568)(275.28188599,265.17789063)
\curveto(275.17187459,265.19788564)(275.04687472,265.20788563)(274.90688599,265.20789063)
\curveto(274.88687488,265.19788564)(274.8618749,265.19288565)(274.83188599,265.19289063)
\curveto(274.80187496,265.20288564)(274.77187499,265.20288564)(274.74188599,265.19289063)
\lineto(274.59188599,265.13289062)
\curveto(274.54187522,265.12288572)(274.49687527,265.10788573)(274.45688599,265.08789062)
\curveto(274.2668755,264.97788586)(274.12187564,264.832886)(274.02188599,264.65289063)
\curveto(273.93187583,264.47288637)(273.85187591,264.26788657)(273.78188599,264.03789063)
\curveto(273.74187602,263.90788693)(273.72187604,263.77288706)(273.72188599,263.63289062)
\curveto(273.72187604,263.50288733)(273.71187605,263.35788748)(273.69188599,263.19789063)
\curveto(273.68187608,263.14788769)(273.67187609,263.08788775)(273.66188599,263.01789062)
\curveto(273.6618761,262.94788789)(273.67187609,262.88788795)(273.69188599,262.83789062)
\lineto(273.69188599,262.67289063)
\lineto(273.69188599,262.49289062)
\curveto(273.70187606,262.44288839)(273.71187605,262.38788845)(273.72188599,262.32789063)
\curveto(273.73187603,262.27788856)(273.73687603,262.22288861)(273.73688599,262.16289063)
\curveto(273.74687602,262.10288873)(273.761876,262.04788879)(273.78188599,261.99789062)
\curveto(273.83187593,261.80788903)(273.89187587,261.6328892)(273.96188599,261.47289062)
\curveto(274.03187573,261.31288952)(274.13687563,261.18288966)(274.27688599,261.08289063)
\curveto(274.40687536,260.98288986)(274.54687522,260.91288992)(274.69688599,260.87289062)
\curveto(274.72687504,260.86288998)(274.75187501,260.85788998)(274.77188599,260.85789062)
\curveto(274.80187496,260.86788997)(274.83187493,260.86788997)(274.86188599,260.85789062)
\curveto(274.88187488,260.85788998)(274.91187485,260.85288999)(274.95188599,260.84289062)
\curveto(274.99187477,260.84288999)(275.02687474,260.84788999)(275.05688599,260.85789062)
\curveto(275.09687467,260.86788997)(275.13687463,260.87288997)(275.17688599,260.87289062)
\curveto(275.21687455,260.87288997)(275.25687451,260.88288995)(275.29688599,260.90289063)
\curveto(275.53687423,260.98288986)(275.73187403,261.11788972)(275.88188599,261.30789063)
\curveto(276.00187376,261.48788935)(276.09187367,261.69288914)(276.15188599,261.92289063)
\curveto(276.17187359,261.99288885)(276.18687358,262.06288878)(276.19688599,262.13289062)
\curveto(276.20687356,262.21288863)(276.22187354,262.29288854)(276.24188599,262.37289062)
\curveto(276.24187352,262.4328884)(276.24687352,262.47788836)(276.25688599,262.50789062)
\curveto(276.25687351,262.52788831)(276.25687351,262.55288828)(276.25688599,262.58289063)
\curveto(276.25687351,262.62288821)(276.2618735,262.65288819)(276.27188599,262.67289063)
\lineto(276.27188599,262.82289063)
}
}
{
\newrgbcolor{curcolor}{0 0 0}
\pscustom[linestyle=none,fillstyle=solid,fillcolor=curcolor]
{
\newpath
\moveto(307.79860718,157.26147949)
\curveto(308.48860254,157.27146886)(309.08860194,157.15146898)(309.59860718,156.90147949)
\curveto(310.11860091,156.65146948)(310.51360052,156.31646982)(310.78360718,155.89647949)
\curveto(310.8336002,155.81647032)(310.87860015,155.72647041)(310.91860718,155.62647949)
\curveto(310.95860007,155.5364706)(311.00360003,155.44147069)(311.05360718,155.34147949)
\curveto(311.09359994,155.24147089)(311.12359991,155.14147099)(311.14360718,155.04147949)
\curveto(311.16359987,154.94147119)(311.18359985,154.8364713)(311.20360718,154.72647949)
\curveto(311.22359981,154.67647146)(311.2285998,154.6314715)(311.21860718,154.59147949)
\curveto(311.20859982,154.55147158)(311.21359982,154.50647163)(311.23360718,154.45647949)
\curveto(311.24359979,154.40647173)(311.24859978,154.32147181)(311.24860718,154.20147949)
\curveto(311.24859978,154.09147204)(311.24359979,154.00647213)(311.23360718,153.94647949)
\curveto(311.21359982,153.88647225)(311.20359983,153.82647231)(311.20360718,153.76647949)
\curveto(311.21359982,153.70647243)(311.20859982,153.64647249)(311.18860718,153.58647949)
\curveto(311.14859988,153.44647269)(311.11359992,153.31147282)(311.08360718,153.18147949)
\curveto(311.05359998,153.05147308)(311.01360002,152.92647321)(310.96360718,152.80647949)
\curveto(310.90360013,152.66647347)(310.8336002,152.54147359)(310.75360718,152.43147949)
\curveto(310.68360035,152.32147381)(310.60860042,152.21147392)(310.52860718,152.10147949)
\lineto(310.46860718,152.04147949)
\curveto(310.45860057,152.02147411)(310.44360059,152.00147413)(310.42360718,151.98147949)
\curveto(310.30360073,151.82147431)(310.16860086,151.67647446)(310.01860718,151.54647949)
\curveto(309.86860116,151.41647472)(309.70860132,151.29147484)(309.53860718,151.17147949)
\curveto(309.2286018,150.95147518)(308.9336021,150.74647539)(308.65360718,150.55647949)
\curveto(308.42360261,150.41647572)(308.19360284,150.28147585)(307.96360718,150.15147949)
\curveto(307.74360329,150.02147611)(307.52360351,149.88647625)(307.30360718,149.74647949)
\curveto(307.05360398,149.57647656)(306.81360422,149.39647674)(306.58360718,149.20647949)
\curveto(306.36360467,149.01647712)(306.17360486,148.79147734)(306.01360718,148.53147949)
\curveto(305.97360506,148.47147766)(305.93860509,148.41147772)(305.90860718,148.35147949)
\curveto(305.87860515,148.30147783)(305.84860518,148.2364779)(305.81860718,148.15647949)
\curveto(305.79860523,148.08647805)(305.79360524,148.02647811)(305.80360718,147.97647949)
\curveto(305.82360521,147.90647823)(305.85860517,147.85147828)(305.90860718,147.81147949)
\curveto(305.95860507,147.78147835)(306.01860501,147.76147837)(306.08860718,147.75147949)
\lineto(306.32860718,147.75147949)
\lineto(307.07860718,147.75147949)
\lineto(309.88360718,147.75147949)
\lineto(310.54360718,147.75147949)
\curveto(310.6336004,147.75147838)(310.71860031,147.74647839)(310.79860718,147.73647949)
\curveto(310.87860015,147.7364784)(310.94360009,147.71647842)(310.99360718,147.67647949)
\curveto(311.04359999,147.6364785)(311.08359995,147.56147857)(311.11360718,147.45147949)
\curveto(311.15359988,147.35147878)(311.16359987,147.25147888)(311.14360718,147.15147949)
\lineto(311.14360718,147.01647949)
\curveto(311.12359991,146.94647919)(311.10359993,146.88647925)(311.08360718,146.83647949)
\curveto(311.06359997,146.78647935)(311.0286,146.74647939)(310.97860718,146.71647949)
\curveto(310.9286001,146.67647946)(310.85860017,146.65647948)(310.76860718,146.65647949)
\lineto(310.49860718,146.65647949)
\lineto(309.59860718,146.65647949)
\lineto(306.08860718,146.65647949)
\lineto(305.02360718,146.65647949)
\curveto(304.94360609,146.65647948)(304.85360618,146.65147948)(304.75360718,146.64147949)
\curveto(304.65360638,146.64147949)(304.56860646,146.65147948)(304.49860718,146.67147949)
\curveto(304.28860674,146.74147939)(304.22360681,146.92147921)(304.30360718,147.21147949)
\curveto(304.31360672,147.25147888)(304.31360672,147.28647885)(304.30360718,147.31647949)
\curveto(304.30360673,147.35647878)(304.31360672,147.40147873)(304.33360718,147.45147949)
\curveto(304.35360668,147.5314786)(304.37360666,147.61647852)(304.39360718,147.70647949)
\curveto(304.41360662,147.79647834)(304.43860659,147.88147825)(304.46860718,147.96147949)
\curveto(304.6286064,148.45147768)(304.8286062,148.86647727)(305.06860718,149.20647949)
\curveto(305.24860578,149.45647668)(305.45360558,149.68147645)(305.68360718,149.88147949)
\curveto(305.91360512,150.09147604)(306.15360488,150.28647585)(306.40360718,150.46647949)
\curveto(306.66360437,150.64647549)(306.9286041,150.81647532)(307.19860718,150.97647949)
\curveto(307.47860355,151.14647499)(307.74860328,151.32147481)(308.00860718,151.50147949)
\curveto(308.11860291,151.58147455)(308.22360281,151.65647448)(308.32360718,151.72647949)
\curveto(308.4336026,151.79647434)(308.54360249,151.87147426)(308.65360718,151.95147949)
\curveto(308.69360234,151.98147415)(308.7286023,152.01147412)(308.75860718,152.04147949)
\curveto(308.79860223,152.08147405)(308.83860219,152.11147402)(308.87860718,152.13147949)
\curveto(309.01860201,152.24147389)(309.14360189,152.36647377)(309.25360718,152.50647949)
\curveto(309.27360176,152.5364736)(309.29860173,152.56147357)(309.32860718,152.58147949)
\curveto(309.35860167,152.61147352)(309.38360165,152.64147349)(309.40360718,152.67147949)
\curveto(309.48360155,152.77147336)(309.54860148,152.87147326)(309.59860718,152.97147949)
\curveto(309.65860137,153.07147306)(309.71360132,153.18147295)(309.76360718,153.30147949)
\curveto(309.79360124,153.37147276)(309.81360122,153.44647269)(309.82360718,153.52647949)
\lineto(309.88360718,153.76647949)
\lineto(309.88360718,153.85647949)
\curveto(309.89360114,153.88647225)(309.89860113,153.91647222)(309.89860718,153.94647949)
\curveto(309.91860111,154.01647212)(309.92360111,154.11147202)(309.91360718,154.23147949)
\curveto(309.91360112,154.36147177)(309.90360113,154.46147167)(309.88360718,154.53147949)
\curveto(309.86360117,154.61147152)(309.84360119,154.68647145)(309.82360718,154.75647949)
\curveto(309.81360122,154.8364713)(309.79360124,154.91647122)(309.76360718,154.99647949)
\curveto(309.65360138,155.2364709)(309.50360153,155.4364707)(309.31360718,155.59647949)
\curveto(309.1336019,155.76647037)(308.91360212,155.90647023)(308.65360718,156.01647949)
\curveto(308.58360245,156.0364701)(308.51360252,156.05147008)(308.44360718,156.06147949)
\curveto(308.37360266,156.08147005)(308.29860273,156.10147003)(308.21860718,156.12147949)
\curveto(308.13860289,156.14146999)(308.028603,156.15146998)(307.88860718,156.15147949)
\curveto(307.75860327,156.15146998)(307.65360338,156.14146999)(307.57360718,156.12147949)
\curveto(307.51360352,156.11147002)(307.45860357,156.10647003)(307.40860718,156.10647949)
\curveto(307.35860367,156.10647003)(307.30860372,156.09647004)(307.25860718,156.07647949)
\curveto(307.15860387,156.0364701)(307.06360397,155.99647014)(306.97360718,155.95647949)
\curveto(306.89360414,155.91647022)(306.81360422,155.87147026)(306.73360718,155.82147949)
\curveto(306.70360433,155.80147033)(306.67360436,155.77647036)(306.64360718,155.74647949)
\curveto(306.62360441,155.71647042)(306.59860443,155.69147044)(306.56860718,155.67147949)
\lineto(306.49360718,155.59647949)
\curveto(306.46360457,155.57647056)(306.43860459,155.55647058)(306.41860718,155.53647949)
\lineto(306.26860718,155.32647949)
\curveto(306.2286048,155.26647087)(306.18360485,155.20147093)(306.13360718,155.13147949)
\curveto(306.07360496,155.04147109)(306.02360501,154.9364712)(305.98360718,154.81647949)
\curveto(305.95360508,154.70647143)(305.91860511,154.59647154)(305.87860718,154.48647949)
\curveto(305.83860519,154.37647176)(305.81360522,154.2314719)(305.80360718,154.05147949)
\curveto(305.79360524,153.88147225)(305.76360527,153.75647238)(305.71360718,153.67647949)
\curveto(305.66360537,153.59647254)(305.58860544,153.55147258)(305.48860718,153.54147949)
\curveto(305.38860564,153.5314726)(305.27860575,153.52647261)(305.15860718,153.52647949)
\curveto(305.11860591,153.52647261)(305.07860595,153.52147261)(305.03860718,153.51147949)
\curveto(304.99860603,153.51147262)(304.96360607,153.51647262)(304.93360718,153.52647949)
\curveto(304.88360615,153.54647259)(304.8336062,153.55647258)(304.78360718,153.55647949)
\curveto(304.74360629,153.55647258)(304.70360633,153.56647257)(304.66360718,153.58647949)
\curveto(304.57360646,153.64647249)(304.5286065,153.78147235)(304.52860718,153.99147949)
\lineto(304.52860718,154.11147949)
\curveto(304.53860649,154.17147196)(304.54360649,154.2314719)(304.54360718,154.29147949)
\curveto(304.55360648,154.36147177)(304.56360647,154.42647171)(304.57360718,154.48647949)
\curveto(304.59360644,154.59647154)(304.61360642,154.69647144)(304.63360718,154.78647949)
\curveto(304.65360638,154.88647125)(304.68360635,154.98147115)(304.72360718,155.07147949)
\curveto(304.74360629,155.14147099)(304.76360627,155.20147093)(304.78360718,155.25147949)
\lineto(304.84360718,155.43147949)
\curveto(304.96360607,155.69147044)(305.11860591,155.9364702)(305.30860718,156.16647949)
\curveto(305.50860552,156.39646974)(305.72360531,156.58146955)(305.95360718,156.72147949)
\curveto(306.06360497,156.80146933)(306.17860485,156.86646927)(306.29860718,156.91647949)
\lineto(306.68860718,157.06647949)
\curveto(306.79860423,157.11646902)(306.91360412,157.14646899)(307.03360718,157.15647949)
\curveto(307.15360388,157.17646896)(307.27860375,157.20146893)(307.40860718,157.23147949)
\curveto(307.47860355,157.2314689)(307.54360349,157.2314689)(307.60360718,157.23147949)
\curveto(307.66360337,157.24146889)(307.7286033,157.25146888)(307.79860718,157.26147949)
}
}
{
\newrgbcolor{curcolor}{0 0 0}
\pscustom[linestyle=none,fillstyle=solid,fillcolor=curcolor]
{
\newpath
\moveto(313.35821655,157.06647949)
\lineto(318.15821655,157.06647949)
\lineto(319.16321655,157.06647949)
\curveto(319.30320945,157.06646907)(319.42320933,157.05646908)(319.52321655,157.03647949)
\curveto(319.63320912,157.02646911)(319.71320904,156.98146915)(319.76321655,156.90147949)
\curveto(319.78320897,156.86146927)(319.79320896,156.81146932)(319.79321655,156.75147949)
\curveto(319.80320895,156.69146944)(319.80820895,156.62646951)(319.80821655,156.55647949)
\lineto(319.80821655,156.28647949)
\curveto(319.80820895,156.19646994)(319.79820896,156.11647002)(319.77821655,156.04647949)
\curveto(319.73820902,155.96647017)(319.69320906,155.89647024)(319.64321655,155.83647949)
\lineto(319.49321655,155.65647949)
\curveto(319.46320929,155.60647053)(319.42820933,155.56647057)(319.38821655,155.53647949)
\curveto(319.34820941,155.50647063)(319.30820945,155.46647067)(319.26821655,155.41647949)
\curveto(319.18820957,155.30647083)(319.10320965,155.19647094)(319.01321655,155.08647949)
\curveto(318.92320983,154.98647115)(318.83820992,154.88147125)(318.75821655,154.77147949)
\curveto(318.61821014,154.57147156)(318.47821028,154.36147177)(318.33821655,154.14147949)
\curveto(318.19821056,153.9314722)(318.0582107,153.71647242)(317.91821655,153.49647949)
\curveto(317.86821089,153.40647273)(317.81821094,153.31147282)(317.76821655,153.21147949)
\curveto(317.71821104,153.11147302)(317.66321109,153.01647312)(317.60321655,152.92647949)
\curveto(317.58321117,152.90647323)(317.57321118,152.88147325)(317.57321655,152.85147949)
\curveto(317.57321118,152.82147331)(317.56321119,152.79647334)(317.54321655,152.77647949)
\curveto(317.47321128,152.67647346)(317.40821135,152.56147357)(317.34821655,152.43147949)
\curveto(317.28821147,152.31147382)(317.23321152,152.19647394)(317.18321655,152.08647949)
\curveto(317.08321167,151.85647428)(316.98821177,151.62147451)(316.89821655,151.38147949)
\curveto(316.80821195,151.14147499)(316.70821205,150.90147523)(316.59821655,150.66147949)
\curveto(316.57821218,150.61147552)(316.56321219,150.56647557)(316.55321655,150.52647949)
\curveto(316.5532122,150.48647565)(316.54321221,150.44147569)(316.52321655,150.39147949)
\curveto(316.47321228,150.27147586)(316.42821233,150.14647599)(316.38821655,150.01647949)
\curveto(316.3582124,149.89647624)(316.32321243,149.77647636)(316.28321655,149.65647949)
\curveto(316.20321255,149.42647671)(316.13821262,149.18647695)(316.08821655,148.93647949)
\curveto(316.04821271,148.69647744)(315.99821276,148.45647768)(315.93821655,148.21647949)
\curveto(315.89821286,148.06647807)(315.87321288,147.91647822)(315.86321655,147.76647949)
\curveto(315.8532129,147.61647852)(315.83321292,147.46647867)(315.80321655,147.31647949)
\curveto(315.79321296,147.27647886)(315.78821297,147.21647892)(315.78821655,147.13647949)
\curveto(315.758213,147.01647912)(315.72821303,146.91647922)(315.69821655,146.83647949)
\curveto(315.66821309,146.75647938)(315.59821316,146.70147943)(315.48821655,146.67147949)
\curveto(315.43821332,146.65147948)(315.38321337,146.64147949)(315.32321655,146.64147949)
\lineto(315.12821655,146.64147949)
\curveto(314.98821377,146.64147949)(314.84821391,146.64647949)(314.70821655,146.65647949)
\curveto(314.57821418,146.66647947)(314.48321427,146.71147942)(314.42321655,146.79147949)
\curveto(314.38321437,146.85147928)(314.36321439,146.9364792)(314.36321655,147.04647949)
\curveto(314.37321438,147.15647898)(314.38821437,147.25147888)(314.40821655,147.33147949)
\lineto(314.40821655,147.40647949)
\curveto(314.41821434,147.4364787)(314.42321433,147.46647867)(314.42321655,147.49647949)
\curveto(314.44321431,147.57647856)(314.4532143,147.65147848)(314.45321655,147.72147949)
\curveto(314.4532143,147.79147834)(314.46321429,147.86147827)(314.48321655,147.93147949)
\curveto(314.53321422,148.12147801)(314.57321418,148.30647783)(314.60321655,148.48647949)
\curveto(314.63321412,148.67647746)(314.67321408,148.85647728)(314.72321655,149.02647949)
\curveto(314.74321401,149.07647706)(314.753214,149.11647702)(314.75321655,149.14647949)
\curveto(314.753214,149.17647696)(314.758214,149.21147692)(314.76821655,149.25147949)
\curveto(314.86821389,149.55147658)(314.9582138,149.84647629)(315.03821655,150.13647949)
\curveto(315.12821363,150.42647571)(315.23321352,150.70647543)(315.35321655,150.97647949)
\curveto(315.61321314,151.55647458)(315.88321287,152.10647403)(316.16321655,152.62647949)
\curveto(316.44321231,153.15647298)(316.753212,153.66147247)(317.09321655,154.14147949)
\curveto(317.23321152,154.34147179)(317.38321137,154.5314716)(317.54321655,154.71147949)
\curveto(317.70321105,154.90147123)(317.8532109,155.09147104)(317.99321655,155.28147949)
\curveto(318.03321072,155.3314708)(318.06821069,155.37647076)(318.09821655,155.41647949)
\curveto(318.13821062,155.46647067)(318.17321058,155.51647062)(318.20321655,155.56647949)
\curveto(318.21321054,155.58647055)(318.22321053,155.61147052)(318.23321655,155.64147949)
\curveto(318.2532105,155.67147046)(318.2532105,155.70147043)(318.23321655,155.73147949)
\curveto(318.21321054,155.79147034)(318.17821058,155.82647031)(318.12821655,155.83647949)
\curveto(318.07821068,155.85647028)(318.02821073,155.87647026)(317.97821655,155.89647949)
\lineto(317.87321655,155.89647949)
\curveto(317.83321092,155.90647023)(317.78321097,155.90647023)(317.72321655,155.89647949)
\lineto(317.57321655,155.89647949)
\lineto(316.97321655,155.89647949)
\lineto(314.33321655,155.89647949)
\lineto(313.59821655,155.89647949)
\lineto(313.35821655,155.89647949)
\curveto(313.28821547,155.90647023)(313.22821553,155.92147021)(313.17821655,155.94147949)
\curveto(313.08821567,155.98147015)(313.02821573,156.04147009)(312.99821655,156.12147949)
\curveto(312.94821581,156.22146991)(312.93321582,156.36646977)(312.95321655,156.55647949)
\curveto(312.97321578,156.75646938)(313.00821575,156.89146924)(313.05821655,156.96147949)
\curveto(313.07821568,156.98146915)(313.10321565,156.99646914)(313.13321655,157.00647949)
\lineto(313.25321655,157.06647949)
\curveto(313.27321548,157.06646907)(313.28821547,157.06146907)(313.29821655,157.05147949)
\curveto(313.31821544,157.05146908)(313.33821542,157.05646908)(313.35821655,157.06647949)
}
}
{
\newrgbcolor{curcolor}{0 0 0}
\pscustom[linestyle=none,fillstyle=solid,fillcolor=curcolor]
{
\newpath
\moveto(322.20282593,148.29147949)
\lineto(322.50282593,148.29147949)
\curveto(322.61282387,148.30147783)(322.71782376,148.30147783)(322.81782593,148.29147949)
\curveto(322.92782355,148.29147784)(323.02782345,148.28147785)(323.11782593,148.26147949)
\curveto(323.20782327,148.25147788)(323.2778232,148.22647791)(323.32782593,148.18647949)
\curveto(323.34782313,148.16647797)(323.36282312,148.136478)(323.37282593,148.09647949)
\curveto(323.39282309,148.05647808)(323.41282307,148.01147812)(323.43282593,147.96147949)
\lineto(323.43282593,147.88647949)
\curveto(323.44282304,147.8364783)(323.44282304,147.78147835)(323.43282593,147.72147949)
\lineto(323.43282593,147.57147949)
\lineto(323.43282593,147.09147949)
\curveto(323.43282305,146.92147921)(323.39282309,146.80147933)(323.31282593,146.73147949)
\curveto(323.24282324,146.68147945)(323.15282333,146.65647948)(323.04282593,146.65647949)
\lineto(322.71282593,146.65647949)
\lineto(322.26282593,146.65647949)
\curveto(322.11282437,146.65647948)(321.99782448,146.68647945)(321.91782593,146.74647949)
\curveto(321.8778246,146.77647936)(321.84782463,146.82647931)(321.82782593,146.89647949)
\curveto(321.80782467,146.97647916)(321.79282469,147.06147907)(321.78282593,147.15147949)
\lineto(321.78282593,147.43647949)
\curveto(321.79282469,147.5364786)(321.79782468,147.62147851)(321.79782593,147.69147949)
\lineto(321.79782593,147.88647949)
\curveto(321.79782468,147.94647819)(321.80782467,148.00147813)(321.82782593,148.05147949)
\curveto(321.86782461,148.16147797)(321.93782454,148.2314779)(322.03782593,148.26147949)
\curveto(322.06782441,148.26147787)(322.12282436,148.27147786)(322.20282593,148.29147949)
}
}
{
\newrgbcolor{curcolor}{0 0 0}
\pscustom[linestyle=none,fillstyle=solid,fillcolor=curcolor]
{
\newpath
\moveto(332.28798218,150.15147949)
\curveto(332.35797453,150.10147603)(332.39797449,150.0314761)(332.40798218,149.94147949)
\curveto(332.42797446,149.85147628)(332.43797445,149.74647639)(332.43798218,149.62647949)
\curveto(332.43797445,149.57647656)(332.43297446,149.52647661)(332.42298218,149.47647949)
\curveto(332.42297447,149.42647671)(332.41297448,149.38147675)(332.39298218,149.34147949)
\curveto(332.36297453,149.25147688)(332.30297459,149.19147694)(332.21298218,149.16147949)
\curveto(332.13297476,149.14147699)(332.03797485,149.131477)(331.92798218,149.13147949)
\lineto(331.61298218,149.13147949)
\curveto(331.50297539,149.14147699)(331.39797549,149.131477)(331.29798218,149.10147949)
\curveto(331.15797573,149.07147706)(331.06797582,148.99147714)(331.02798218,148.86147949)
\curveto(331.00797588,148.79147734)(330.99797589,148.70647743)(330.99798218,148.60647949)
\lineto(330.99798218,148.33647949)
\lineto(330.99798218,147.39147949)
\lineto(330.99798218,147.06147949)
\curveto(330.99797589,146.95147918)(330.97797591,146.86647927)(330.93798218,146.80647949)
\curveto(330.89797599,146.74647939)(330.84797604,146.70647943)(330.78798218,146.68647949)
\curveto(330.73797615,146.67647946)(330.67297622,146.66147947)(330.59298218,146.64147949)
\lineto(330.39798218,146.64147949)
\curveto(330.27797661,146.64147949)(330.17297672,146.64647949)(330.08298218,146.65647949)
\curveto(329.9929769,146.67647946)(329.92297697,146.72647941)(329.87298218,146.80647949)
\curveto(329.84297705,146.85647928)(329.82797706,146.92647921)(329.82798218,147.01647949)
\lineto(329.82798218,147.31647949)
\lineto(329.82798218,148.35147949)
\curveto(329.82797706,148.51147762)(329.81797707,148.65647748)(329.79798218,148.78647949)
\curveto(329.7879771,148.92647721)(329.73297716,149.02147711)(329.63298218,149.07147949)
\curveto(329.58297731,149.09147704)(329.51297738,149.10647703)(329.42298218,149.11647949)
\curveto(329.34297755,149.12647701)(329.25297764,149.131477)(329.15298218,149.13147949)
\lineto(328.86798218,149.13147949)
\lineto(328.62798218,149.13147949)
\lineto(326.36298218,149.13147949)
\curveto(326.27298062,149.131477)(326.16798072,149.12647701)(326.04798218,149.11647949)
\lineto(325.71798218,149.11647949)
\curveto(325.60798128,149.11647702)(325.50798138,149.12647701)(325.41798218,149.14647949)
\curveto(325.32798156,149.16647697)(325.26798162,149.20147693)(325.23798218,149.25147949)
\curveto(325.1879817,149.32147681)(325.16298173,149.41647672)(325.16298218,149.53647949)
\lineto(325.16298218,149.88147949)
\lineto(325.16298218,150.15147949)
\curveto(325.20298169,150.32147581)(325.25798163,150.46147567)(325.32798218,150.57147949)
\curveto(325.39798149,150.68147545)(325.47798141,150.79647534)(325.56798218,150.91647949)
\lineto(325.92798218,151.45647949)
\curveto(326.36798052,152.08647405)(326.80298009,152.70647343)(327.23298218,153.31647949)
\lineto(328.55298218,155.17647949)
\curveto(328.71297818,155.40647073)(328.86797802,155.62647051)(329.01798218,155.83647949)
\curveto(329.16797772,156.05647008)(329.32297757,156.28146985)(329.48298218,156.51147949)
\curveto(329.53297736,156.58146955)(329.58297731,156.64646949)(329.63298218,156.70647949)
\curveto(329.68297721,156.77646936)(329.73297716,156.85146928)(329.78298218,156.93147949)
\lineto(329.84298218,157.02147949)
\curveto(329.87297702,157.06146907)(329.90297699,157.09146904)(329.93298218,157.11147949)
\curveto(329.97297692,157.14146899)(330.01297688,157.16146897)(330.05298218,157.17147949)
\curveto(330.0929768,157.19146894)(330.13797675,157.21146892)(330.18798218,157.23147949)
\curveto(330.20797668,157.2314689)(330.22797666,157.22646891)(330.24798218,157.21647949)
\curveto(330.27797661,157.21646892)(330.30297659,157.22646891)(330.32298218,157.24647949)
\curveto(330.45297644,157.24646889)(330.57297632,157.24146889)(330.68298218,157.23147949)
\curveto(330.7929761,157.22146891)(330.87297602,157.17646896)(330.92298218,157.09647949)
\curveto(330.96297593,157.04646909)(330.98297591,156.97646916)(330.98298218,156.88647949)
\curveto(330.9929759,156.79646934)(330.99797589,156.70146943)(330.99798218,156.60147949)
\lineto(330.99798218,151.14147949)
\curveto(330.99797589,151.07147506)(330.9929759,150.99647514)(330.98298218,150.91647949)
\curveto(330.98297591,150.84647529)(330.9879759,150.77647536)(330.99798218,150.70647949)
\lineto(330.99798218,150.60147949)
\curveto(331.01797587,150.55147558)(331.03297586,150.49647564)(331.04298218,150.43647949)
\curveto(331.05297584,150.38647575)(331.07797581,150.34647579)(331.11798218,150.31647949)
\curveto(331.1879757,150.26647587)(331.27297562,150.2364759)(331.37298218,150.22647949)
\lineto(331.70298218,150.22647949)
\curveto(331.81297508,150.22647591)(331.91797497,150.22147591)(332.01798218,150.21147949)
\curveto(332.12797476,150.21147592)(332.21797467,150.19147594)(332.28798218,150.15147949)
\moveto(329.72298218,150.34647949)
\curveto(329.80297709,150.45647568)(329.83797705,150.62647551)(329.82798218,150.85647949)
\lineto(329.82798218,151.47147949)
\lineto(329.82798218,153.94647949)
\lineto(329.82798218,154.26147949)
\curveto(329.83797705,154.38147175)(329.83297706,154.48147165)(329.81298218,154.56147949)
\lineto(329.81298218,154.71147949)
\curveto(329.81297708,154.80147133)(329.79797709,154.88647125)(329.76798218,154.96647949)
\curveto(329.75797713,154.98647115)(329.74797714,154.99647114)(329.73798218,154.99647949)
\lineto(329.69298218,155.04147949)
\curveto(329.67297722,155.05147108)(329.64297725,155.05647108)(329.60298218,155.05647949)
\curveto(329.58297731,155.0364711)(329.56297733,155.02147111)(329.54298218,155.01147949)
\curveto(329.53297736,155.01147112)(329.51797737,155.00647113)(329.49798218,154.99647949)
\curveto(329.43797745,154.94647119)(329.37797751,154.87647126)(329.31798218,154.78647949)
\curveto(329.25797763,154.69647144)(329.20297769,154.61647152)(329.15298218,154.54647949)
\curveto(329.05297784,154.40647173)(328.95797793,154.26147187)(328.86798218,154.11147949)
\curveto(328.77797811,153.97147216)(328.68297821,153.8314723)(328.58298218,153.69147949)
\lineto(328.04298218,152.91147949)
\curveto(327.87297902,152.65147348)(327.69797919,152.39147374)(327.51798218,152.13147949)
\curveto(327.43797945,152.02147411)(327.36297953,151.91647422)(327.29298218,151.81647949)
\lineto(327.08298218,151.51647949)
\curveto(327.03297986,151.4364747)(326.98297991,151.36147477)(326.93298218,151.29147949)
\curveto(326.89298,151.22147491)(326.84798004,151.14647499)(326.79798218,151.06647949)
\curveto(326.74798014,151.00647513)(326.69798019,150.94147519)(326.64798218,150.87147949)
\curveto(326.60798028,150.81147532)(326.56798032,150.74147539)(326.52798218,150.66147949)
\curveto(326.4879804,150.60147553)(326.46298043,150.5314756)(326.45298218,150.45147949)
\curveto(326.44298045,150.38147575)(326.47798041,150.32647581)(326.55798218,150.28647949)
\curveto(326.62798026,150.2364759)(326.73798015,150.21147592)(326.88798218,150.21147949)
\curveto(327.04797984,150.22147591)(327.18297971,150.22647591)(327.29298218,150.22647949)
\lineto(328.97298218,150.22647949)
\lineto(329.40798218,150.22647949)
\curveto(329.55797733,150.22647591)(329.66297723,150.26647587)(329.72298218,150.34647949)
}
}
{
\newrgbcolor{curcolor}{0 0 0}
\pscustom[linestyle=none,fillstyle=solid,fillcolor=curcolor]
{
\newpath
\moveto(343.56259155,155.17647949)
\curveto(343.36258125,154.88647125)(343.15258146,154.60147153)(342.93259155,154.32147949)
\curveto(342.72258189,154.04147209)(342.5175821,153.75647238)(342.31759155,153.46647949)
\curveto(341.7175829,152.61647352)(341.1125835,151.77647436)(340.50259155,150.94647949)
\curveto(339.89258472,150.12647601)(339.28758533,149.29147684)(338.68759155,148.44147949)
\lineto(338.17759155,147.72147949)
\lineto(337.66759155,147.03147949)
\curveto(337.58758703,146.92147921)(337.50758711,146.80647933)(337.42759155,146.68647949)
\curveto(337.34758727,146.56647957)(337.25258736,146.47147966)(337.14259155,146.40147949)
\curveto(337.10258751,146.38147975)(337.03758758,146.36647977)(336.94759155,146.35647949)
\curveto(336.86758775,146.3364798)(336.77758784,146.32647981)(336.67759155,146.32647949)
\curveto(336.57758804,146.32647981)(336.48258813,146.3314798)(336.39259155,146.34147949)
\curveto(336.3125883,146.35147978)(336.25258836,146.37147976)(336.21259155,146.40147949)
\curveto(336.18258843,146.42147971)(336.15758846,146.45647968)(336.13759155,146.50647949)
\curveto(336.12758849,146.54647959)(336.13258848,146.59147954)(336.15259155,146.64147949)
\curveto(336.19258842,146.72147941)(336.23758838,146.79647934)(336.28759155,146.86647949)
\curveto(336.34758827,146.94647919)(336.40258821,147.02647911)(336.45259155,147.10647949)
\curveto(336.69258792,147.44647869)(336.93758768,147.78147835)(337.18759155,148.11147949)
\curveto(337.43758718,148.44147769)(337.67758694,148.77647736)(337.90759155,149.11647949)
\curveto(338.06758655,149.3364768)(338.22758639,149.55147658)(338.38759155,149.76147949)
\curveto(338.54758607,149.97147616)(338.70758591,150.18647595)(338.86759155,150.40647949)
\curveto(339.22758539,150.92647521)(339.59258502,151.4364747)(339.96259155,151.93647949)
\curveto(340.33258428,152.4364737)(340.70258391,152.94647319)(341.07259155,153.46647949)
\curveto(341.2125834,153.66647247)(341.35258326,153.86147227)(341.49259155,154.05147949)
\curveto(341.64258297,154.24147189)(341.78758283,154.4364717)(341.92759155,154.63647949)
\curveto(342.13758248,154.9364712)(342.35258226,155.2364709)(342.57259155,155.53647949)
\lineto(343.23259155,156.43647949)
\lineto(343.41259155,156.70647949)
\lineto(343.62259155,156.97647949)
\lineto(343.74259155,157.15647949)
\curveto(343.79258082,157.21646892)(343.84258077,157.27146886)(343.89259155,157.32147949)
\curveto(343.96258065,157.37146876)(344.03758058,157.40646873)(344.11759155,157.42647949)
\curveto(344.13758048,157.4364687)(344.16258045,157.4364687)(344.19259155,157.42647949)
\curveto(344.23258038,157.42646871)(344.26258035,157.4364687)(344.28259155,157.45647949)
\curveto(344.40258021,157.45646868)(344.53758008,157.45146868)(344.68759155,157.44147949)
\curveto(344.83757978,157.44146869)(344.92757969,157.39646874)(344.95759155,157.30647949)
\curveto(344.97757964,157.27646886)(344.98257963,157.24146889)(344.97259155,157.20147949)
\curveto(344.96257965,157.16146897)(344.94757967,157.131469)(344.92759155,157.11147949)
\curveto(344.88757973,157.0314691)(344.84757977,156.96146917)(344.80759155,156.90147949)
\curveto(344.76757985,156.84146929)(344.72257989,156.78146935)(344.67259155,156.72147949)
\lineto(344.10259155,155.94147949)
\curveto(343.92258069,155.69147044)(343.74258087,155.4364707)(343.56259155,155.17647949)
\moveto(336.70759155,151.27647949)
\curveto(336.65758796,151.29647484)(336.60758801,151.30147483)(336.55759155,151.29147949)
\curveto(336.50758811,151.28147485)(336.45758816,151.28647485)(336.40759155,151.30647949)
\curveto(336.29758832,151.32647481)(336.19258842,151.34647479)(336.09259155,151.36647949)
\curveto(336.00258861,151.39647474)(335.90758871,151.4364747)(335.80759155,151.48647949)
\curveto(335.47758914,151.62647451)(335.22258939,151.82147431)(335.04259155,152.07147949)
\curveto(334.86258975,152.3314738)(334.7175899,152.64147349)(334.60759155,153.00147949)
\curveto(334.57759004,153.08147305)(334.55759006,153.16147297)(334.54759155,153.24147949)
\curveto(334.53759008,153.3314728)(334.52259009,153.41647272)(334.50259155,153.49647949)
\curveto(334.49259012,153.54647259)(334.48759013,153.61147252)(334.48759155,153.69147949)
\curveto(334.47759014,153.72147241)(334.47259014,153.75147238)(334.47259155,153.78147949)
\curveto(334.47259014,153.82147231)(334.46759015,153.85647228)(334.45759155,153.88647949)
\lineto(334.45759155,154.03647949)
\curveto(334.44759017,154.08647205)(334.44259017,154.14647199)(334.44259155,154.21647949)
\curveto(334.44259017,154.29647184)(334.44759017,154.36147177)(334.45759155,154.41147949)
\lineto(334.45759155,154.57647949)
\curveto(334.47759014,154.62647151)(334.48259013,154.67147146)(334.47259155,154.71147949)
\curveto(334.47259014,154.76147137)(334.47759014,154.80647133)(334.48759155,154.84647949)
\curveto(334.49759012,154.88647125)(334.50259011,154.92147121)(334.50259155,154.95147949)
\curveto(334.50259011,154.99147114)(334.50759011,155.0314711)(334.51759155,155.07147949)
\curveto(334.54759007,155.18147095)(334.56759005,155.29147084)(334.57759155,155.40147949)
\curveto(334.59759002,155.52147061)(334.63258998,155.6364705)(334.68259155,155.74647949)
\curveto(334.82258979,156.08647005)(334.98258963,156.36146977)(335.16259155,156.57147949)
\curveto(335.35258926,156.79146934)(335.62258899,156.97146916)(335.97259155,157.11147949)
\curveto(336.05258856,157.14146899)(336.13758848,157.16146897)(336.22759155,157.17147949)
\curveto(336.3175883,157.19146894)(336.4125882,157.21146892)(336.51259155,157.23147949)
\curveto(336.54258807,157.24146889)(336.59758802,157.24146889)(336.67759155,157.23147949)
\curveto(336.75758786,157.2314689)(336.80758781,157.24146889)(336.82759155,157.26147949)
\curveto(337.38758723,157.27146886)(337.83758678,157.16146897)(338.17759155,156.93147949)
\curveto(338.52758609,156.70146943)(338.78758583,156.39646974)(338.95759155,156.01647949)
\curveto(338.99758562,155.92647021)(339.03258558,155.8314703)(339.06259155,155.73147949)
\curveto(339.09258552,155.6314705)(339.1175855,155.5314706)(339.13759155,155.43147949)
\curveto(339.15758546,155.40147073)(339.16258545,155.37147076)(339.15259155,155.34147949)
\curveto(339.15258546,155.31147082)(339.15758546,155.28147085)(339.16759155,155.25147949)
\curveto(339.19758542,155.14147099)(339.2175854,155.01647112)(339.22759155,154.87647949)
\curveto(339.23758538,154.74647139)(339.24758537,154.61147152)(339.25759155,154.47147949)
\lineto(339.25759155,154.30647949)
\curveto(339.26758535,154.24647189)(339.26758535,154.19147194)(339.25759155,154.14147949)
\curveto(339.24758537,154.09147204)(339.24258537,154.04147209)(339.24259155,153.99147949)
\lineto(339.24259155,153.85647949)
\curveto(339.23258538,153.81647232)(339.22758539,153.77647236)(339.22759155,153.73647949)
\curveto(339.23758538,153.69647244)(339.23258538,153.65147248)(339.21259155,153.60147949)
\curveto(339.19258542,153.49147264)(339.17258544,153.38647275)(339.15259155,153.28647949)
\curveto(339.14258547,153.18647295)(339.12258549,153.08647305)(339.09259155,152.98647949)
\curveto(338.96258565,152.62647351)(338.79758582,152.31147382)(338.59759155,152.04147949)
\curveto(338.39758622,151.77147436)(338.12258649,151.56647457)(337.77259155,151.42647949)
\curveto(337.69258692,151.39647474)(337.60758701,151.37147476)(337.51759155,151.35147949)
\lineto(337.24759155,151.29147949)
\curveto(337.19758742,151.28147485)(337.15258746,151.27647486)(337.11259155,151.27647949)
\curveto(337.07258754,151.28647485)(337.03258758,151.28647485)(336.99259155,151.27647949)
\curveto(336.89258772,151.25647488)(336.79758782,151.25647488)(336.70759155,151.27647949)
\moveto(335.86759155,152.67147949)
\curveto(335.90758871,152.60147353)(335.94758867,152.5364736)(335.98759155,152.47647949)
\curveto(336.02758859,152.42647371)(336.07758854,152.37647376)(336.13759155,152.32647949)
\lineto(336.28759155,152.20647949)
\curveto(336.34758827,152.17647396)(336.4125882,152.15147398)(336.48259155,152.13147949)
\curveto(336.52258809,152.11147402)(336.55758806,152.10147403)(336.58759155,152.10147949)
\curveto(336.62758799,152.11147402)(336.66758795,152.10647403)(336.70759155,152.08647949)
\curveto(336.73758788,152.08647405)(336.77758784,152.08147405)(336.82759155,152.07147949)
\curveto(336.87758774,152.07147406)(336.9175877,152.07647406)(336.94759155,152.08647949)
\lineto(337.17259155,152.13147949)
\curveto(337.42258719,152.21147392)(337.60758701,152.3364738)(337.72759155,152.50647949)
\curveto(337.80758681,152.60647353)(337.87758674,152.7364734)(337.93759155,152.89647949)
\curveto(338.0175866,153.07647306)(338.07758654,153.30147283)(338.11759155,153.57147949)
\curveto(338.15758646,153.85147228)(338.17258644,154.131472)(338.16259155,154.41147949)
\curveto(338.15258646,154.70147143)(338.12258649,154.97647116)(338.07259155,155.23647949)
\curveto(338.02258659,155.49647064)(337.94758667,155.70647043)(337.84759155,155.86647949)
\curveto(337.72758689,156.06647007)(337.57758704,156.21646992)(337.39759155,156.31647949)
\curveto(337.3175873,156.36646977)(337.22758739,156.39646974)(337.12759155,156.40647949)
\curveto(337.02758759,156.42646971)(336.92258769,156.4364697)(336.81259155,156.43647949)
\curveto(336.79258782,156.42646971)(336.76758785,156.42146971)(336.73759155,156.42147949)
\curveto(336.7175879,156.4314697)(336.69758792,156.4314697)(336.67759155,156.42147949)
\curveto(336.62758799,156.41146972)(336.58258803,156.40146973)(336.54259155,156.39147949)
\curveto(336.50258811,156.39146974)(336.46258815,156.38146975)(336.42259155,156.36147949)
\curveto(336.24258837,156.28146985)(336.09258852,156.16146997)(335.97259155,156.00147949)
\curveto(335.86258875,155.84147029)(335.77258884,155.66147047)(335.70259155,155.46147949)
\curveto(335.64258897,155.27147086)(335.59758902,155.04647109)(335.56759155,154.78647949)
\curveto(335.54758907,154.52647161)(335.54258907,154.26147187)(335.55259155,153.99147949)
\curveto(335.56258905,153.7314724)(335.59258902,153.48147265)(335.64259155,153.24147949)
\curveto(335.70258891,153.01147312)(335.77758884,152.82147331)(335.86759155,152.67147949)
\moveto(346.66759155,149.68647949)
\curveto(346.67757794,149.6364765)(346.68257793,149.54647659)(346.68259155,149.41647949)
\curveto(346.68257793,149.28647685)(346.67257794,149.19647694)(346.65259155,149.14647949)
\curveto(346.63257798,149.09647704)(346.62757799,149.04147709)(346.63759155,148.98147949)
\curveto(346.64757797,148.9314772)(346.64757797,148.88147725)(346.63759155,148.83147949)
\curveto(346.59757802,148.69147744)(346.56757805,148.55647758)(346.54759155,148.42647949)
\curveto(346.53757808,148.29647784)(346.50757811,148.17647796)(346.45759155,148.06647949)
\curveto(346.3175783,147.71647842)(346.15257846,147.42147871)(345.96259155,147.18147949)
\curveto(345.77257884,146.95147918)(345.50257911,146.76647937)(345.15259155,146.62647949)
\curveto(345.07257954,146.59647954)(344.98757963,146.57647956)(344.89759155,146.56647949)
\curveto(344.80757981,146.54647959)(344.72257989,146.52647961)(344.64259155,146.50647949)
\curveto(344.59258002,146.49647964)(344.54258007,146.49147964)(344.49259155,146.49147949)
\curveto(344.44258017,146.49147964)(344.39258022,146.48647965)(344.34259155,146.47647949)
\curveto(344.3125803,146.46647967)(344.26258035,146.46647967)(344.19259155,146.47647949)
\curveto(344.12258049,146.47647966)(344.07258054,146.48147965)(344.04259155,146.49147949)
\curveto(343.98258063,146.51147962)(343.92258069,146.52147961)(343.86259155,146.52147949)
\curveto(343.8125808,146.51147962)(343.76258085,146.51647962)(343.71259155,146.53647949)
\curveto(343.62258099,146.55647958)(343.53258108,146.58147955)(343.44259155,146.61147949)
\curveto(343.36258125,146.6314795)(343.28258133,146.66147947)(343.20259155,146.70147949)
\curveto(342.88258173,146.84147929)(342.63258198,147.0364791)(342.45259155,147.28647949)
\curveto(342.27258234,147.54647859)(342.12258249,147.85147828)(342.00259155,148.20147949)
\curveto(341.98258263,148.28147785)(341.96758265,148.36647777)(341.95759155,148.45647949)
\curveto(341.94758267,148.54647759)(341.93258268,148.6314775)(341.91259155,148.71147949)
\curveto(341.90258271,148.74147739)(341.89758272,148.77147736)(341.89759155,148.80147949)
\lineto(341.89759155,148.90647949)
\curveto(341.87758274,148.98647715)(341.86758275,149.06647707)(341.86759155,149.14647949)
\lineto(341.86759155,149.28147949)
\curveto(341.84758277,149.38147675)(341.84758277,149.48147665)(341.86759155,149.58147949)
\lineto(341.86759155,149.76147949)
\curveto(341.87758274,149.81147632)(341.88258273,149.85647628)(341.88259155,149.89647949)
\curveto(341.88258273,149.94647619)(341.88758273,149.99147614)(341.89759155,150.03147949)
\curveto(341.90758271,150.07147606)(341.9125827,150.10647603)(341.91259155,150.13647949)
\curveto(341.9125827,150.17647596)(341.9175827,150.21647592)(341.92759155,150.25647949)
\lineto(341.98759155,150.58647949)
\curveto(342.00758261,150.70647543)(342.03758258,150.81647532)(342.07759155,150.91647949)
\curveto(342.2175824,151.24647489)(342.37758224,151.52147461)(342.55759155,151.74147949)
\curveto(342.74758187,151.97147416)(343.00758161,152.15647398)(343.33759155,152.29647949)
\curveto(343.4175812,152.3364738)(343.50258111,152.36147377)(343.59259155,152.37147949)
\lineto(343.89259155,152.43147949)
\lineto(344.02759155,152.43147949)
\curveto(344.07758054,152.44147369)(344.12758049,152.44647369)(344.17759155,152.44647949)
\curveto(344.74757987,152.46647367)(345.20757941,152.36147377)(345.55759155,152.13147949)
\curveto(345.9175787,151.91147422)(346.18257843,151.61147452)(346.35259155,151.23147949)
\curveto(346.40257821,151.131475)(346.44257817,151.0314751)(346.47259155,150.93147949)
\curveto(346.50257811,150.8314753)(346.53257808,150.72647541)(346.56259155,150.61647949)
\curveto(346.57257804,150.57647556)(346.57757804,150.54147559)(346.57759155,150.51147949)
\curveto(346.57757804,150.49147564)(346.58257803,150.46147567)(346.59259155,150.42147949)
\curveto(346.612578,150.35147578)(346.62257799,150.27647586)(346.62259155,150.19647949)
\curveto(346.62257799,150.11647602)(346.63257798,150.0364761)(346.65259155,149.95647949)
\curveto(346.65257796,149.90647623)(346.65257796,149.86147627)(346.65259155,149.82147949)
\curveto(346.65257796,149.78147635)(346.65757796,149.7364764)(346.66759155,149.68647949)
\moveto(345.55759155,149.25147949)
\curveto(345.56757905,149.30147683)(345.57257904,149.37647676)(345.57259155,149.47647949)
\curveto(345.58257903,149.57647656)(345.57757904,149.65147648)(345.55759155,149.70147949)
\curveto(345.53757908,149.76147637)(345.53257908,149.81647632)(345.54259155,149.86647949)
\curveto(345.56257905,149.92647621)(345.56257905,149.98647615)(345.54259155,150.04647949)
\curveto(345.53257908,150.07647606)(345.52757909,150.11147602)(345.52759155,150.15147949)
\curveto(345.52757909,150.19147594)(345.52257909,150.2314759)(345.51259155,150.27147949)
\curveto(345.49257912,150.35147578)(345.47257914,150.42647571)(345.45259155,150.49647949)
\curveto(345.44257917,150.57647556)(345.42757919,150.65647548)(345.40759155,150.73647949)
\curveto(345.37757924,150.79647534)(345.35257926,150.85647528)(345.33259155,150.91647949)
\curveto(345.3125793,150.97647516)(345.28257933,151.0364751)(345.24259155,151.09647949)
\curveto(345.14257947,151.26647487)(345.0125796,151.40147473)(344.85259155,151.50147949)
\curveto(344.77257984,151.55147458)(344.67757994,151.58647455)(344.56759155,151.60647949)
\curveto(344.45758016,151.62647451)(344.33258028,151.6364745)(344.19259155,151.63647949)
\curveto(344.17258044,151.62647451)(344.14758047,151.62147451)(344.11759155,151.62147949)
\curveto(344.08758053,151.6314745)(344.05758056,151.6314745)(344.02759155,151.62147949)
\lineto(343.87759155,151.56147949)
\curveto(343.82758079,151.55147458)(343.78258083,151.5364746)(343.74259155,151.51647949)
\curveto(343.55258106,151.40647473)(343.40758121,151.26147487)(343.30759155,151.08147949)
\curveto(343.2175814,150.90147523)(343.13758148,150.69647544)(343.06759155,150.46647949)
\curveto(343.02758159,150.3364758)(343.00758161,150.20147593)(343.00759155,150.06147949)
\curveto(343.00758161,149.9314762)(342.99758162,149.78647635)(342.97759155,149.62647949)
\curveto(342.96758165,149.57647656)(342.95758166,149.51647662)(342.94759155,149.44647949)
\curveto(342.94758167,149.37647676)(342.95758166,149.31647682)(342.97759155,149.26647949)
\lineto(342.97759155,149.10147949)
\lineto(342.97759155,148.92147949)
\curveto(342.98758163,148.87147726)(342.99758162,148.81647732)(343.00759155,148.75647949)
\curveto(343.0175816,148.70647743)(343.02258159,148.65147748)(343.02259155,148.59147949)
\curveto(343.03258158,148.5314776)(343.04758157,148.47647766)(343.06759155,148.42647949)
\curveto(343.1175815,148.2364779)(343.17758144,148.06147807)(343.24759155,147.90147949)
\curveto(343.3175813,147.74147839)(343.42258119,147.61147852)(343.56259155,147.51147949)
\curveto(343.69258092,147.41147872)(343.83258078,147.34147879)(343.98259155,147.30147949)
\curveto(344.0125806,147.29147884)(344.03758058,147.28647885)(344.05759155,147.28647949)
\curveto(344.08758053,147.29647884)(344.1175805,147.29647884)(344.14759155,147.28647949)
\curveto(344.16758045,147.28647885)(344.19758042,147.28147885)(344.23759155,147.27147949)
\curveto(344.27758034,147.27147886)(344.3125803,147.27647886)(344.34259155,147.28647949)
\curveto(344.38258023,147.29647884)(344.42258019,147.30147883)(344.46259155,147.30147949)
\curveto(344.50258011,147.30147883)(344.54258007,147.31147882)(344.58259155,147.33147949)
\curveto(344.82257979,147.41147872)(345.0175796,147.54647859)(345.16759155,147.73647949)
\curveto(345.28757933,147.91647822)(345.37757924,148.12147801)(345.43759155,148.35147949)
\curveto(345.45757916,148.42147771)(345.47257914,148.49147764)(345.48259155,148.56147949)
\curveto(345.49257912,148.64147749)(345.50757911,148.72147741)(345.52759155,148.80147949)
\curveto(345.52757909,148.86147727)(345.53257908,148.90647723)(345.54259155,148.93647949)
\curveto(345.54257907,148.95647718)(345.54257907,148.98147715)(345.54259155,149.01147949)
\curveto(345.54257907,149.05147708)(345.54757907,149.08147705)(345.55759155,149.10147949)
\lineto(345.55759155,149.25147949)
}
}
{
\newrgbcolor{curcolor}{0 0 0}
\pscustom[linestyle=none,fillstyle=solid,fillcolor=curcolor]
{
\newpath
\moveto(255.344328,54.20930176)
\lineto(258.944328,54.20930176)
\lineto(259.589328,54.20930176)
\curveto(259.66932147,54.20929133)(259.7443214,54.20429134)(259.814328,54.19430176)
\curveto(259.88432126,54.19429135)(259.9443212,54.18429136)(259.994328,54.16430176)
\curveto(260.06432108,54.13429141)(260.11932102,54.07429147)(260.159328,53.98430176)
\curveto(260.17932096,53.95429159)(260.18932095,53.91429163)(260.189328,53.86430176)
\lineto(260.189328,53.72930176)
\curveto(260.19932094,53.61929192)(260.19432095,53.51429203)(260.174328,53.41430176)
\curveto(260.16432098,53.31429223)(260.12932101,53.2442923)(260.069328,53.20430176)
\curveto(259.97932116,53.13429241)(259.8443213,53.09929244)(259.664328,53.09930176)
\curveto(259.48432166,53.10929243)(259.31932182,53.11429243)(259.169328,53.11430176)
\lineto(257.174328,53.11430176)
\lineto(256.679328,53.11430176)
\lineto(256.544328,53.11430176)
\curveto(256.50432464,53.11429243)(256.46432468,53.10929243)(256.424328,53.09930176)
\lineto(256.214328,53.09930176)
\curveto(256.10432504,53.06929247)(256.02432512,53.02929251)(255.974328,52.97930176)
\curveto(255.92432522,52.9392926)(255.88932525,52.88429266)(255.869328,52.81430176)
\curveto(255.84932529,52.75429279)(255.83432531,52.68429286)(255.824328,52.60430176)
\curveto(255.81432533,52.52429302)(255.79432535,52.43429311)(255.764328,52.33430176)
\curveto(255.71432543,52.13429341)(255.67432547,51.92929361)(255.644328,51.71930176)
\curveto(255.61432553,51.50929403)(255.57432557,51.30429424)(255.524328,51.10430176)
\curveto(255.50432564,51.03429451)(255.49432565,50.96429458)(255.494328,50.89430176)
\curveto(255.49432565,50.83429471)(255.48432566,50.76929477)(255.464328,50.69930176)
\curveto(255.45432569,50.66929487)(255.4443257,50.62929491)(255.434328,50.57930176)
\curveto(255.43432571,50.539295)(255.4393257,50.49929504)(255.449328,50.45930176)
\curveto(255.46932567,50.40929513)(255.49432565,50.36429518)(255.524328,50.32430176)
\curveto(255.56432558,50.29429525)(255.62432552,50.28929525)(255.704328,50.30930176)
\curveto(255.76432538,50.32929521)(255.82432532,50.35429519)(255.884328,50.38430176)
\curveto(255.9443252,50.42429512)(256.00432514,50.45929508)(256.064328,50.48930176)
\curveto(256.12432502,50.50929503)(256.17432497,50.52429502)(256.214328,50.53430176)
\curveto(256.40432474,50.61429493)(256.60932453,50.66929487)(256.829328,50.69930176)
\curveto(257.05932408,50.72929481)(257.28932385,50.7392948)(257.519328,50.72930176)
\curveto(257.75932338,50.72929481)(257.98932315,50.70429484)(258.209328,50.65430176)
\curveto(258.42932271,50.61429493)(258.62932251,50.55429499)(258.809328,50.47430176)
\curveto(258.85932228,50.45429509)(258.90432224,50.43429511)(258.944328,50.41430176)
\curveto(258.99432215,50.39429515)(259.0443221,50.36929517)(259.094328,50.33930176)
\curveto(259.4443217,50.12929541)(259.72432142,49.89929564)(259.934328,49.64930176)
\curveto(260.15432099,49.39929614)(260.34932079,49.07429647)(260.519328,48.67430176)
\curveto(260.56932057,48.56429698)(260.60432054,48.45429709)(260.624328,48.34430176)
\curveto(260.6443205,48.23429731)(260.66932047,48.11929742)(260.699328,47.99930176)
\curveto(260.70932043,47.96929757)(260.71432043,47.92429762)(260.714328,47.86430176)
\curveto(260.73432041,47.80429774)(260.7443204,47.73429781)(260.744328,47.65430176)
\curveto(260.7443204,47.58429796)(260.75432039,47.51929802)(260.774328,47.45930176)
\lineto(260.774328,47.29430176)
\curveto(260.78432036,47.2442983)(260.78932035,47.17429837)(260.789328,47.08430176)
\curveto(260.78932035,46.99429855)(260.77932036,46.92429862)(260.759328,46.87430176)
\curveto(260.7393204,46.81429873)(260.73432041,46.75429879)(260.744328,46.69430176)
\curveto(260.75432039,46.6442989)(260.74932039,46.59429895)(260.729328,46.54430176)
\curveto(260.68932045,46.38429916)(260.65432049,46.23429931)(260.624328,46.09430176)
\curveto(260.59432055,45.95429959)(260.54932059,45.81929972)(260.489328,45.68930176)
\curveto(260.32932081,45.31930022)(260.10932103,44.98430056)(259.829328,44.68430176)
\curveto(259.54932159,44.38430116)(259.22932191,44.15430139)(258.869328,43.99430176)
\curveto(258.69932244,43.91430163)(258.49932264,43.8393017)(258.269328,43.76930176)
\curveto(258.15932298,43.72930181)(258.0443231,43.70430184)(257.924328,43.69430176)
\curveto(257.80432334,43.68430186)(257.68432346,43.66430188)(257.564328,43.63430176)
\curveto(257.51432363,43.61430193)(257.45932368,43.61430193)(257.399328,43.63430176)
\curveto(257.3393238,43.6443019)(257.27932386,43.6393019)(257.219328,43.61930176)
\curveto(257.11932402,43.59930194)(257.01932412,43.59930194)(256.919328,43.61930176)
\lineto(256.784328,43.61930176)
\curveto(256.73432441,43.6393019)(256.67432447,43.64930189)(256.604328,43.64930176)
\curveto(256.5443246,43.6393019)(256.48932465,43.6443019)(256.439328,43.66430176)
\curveto(256.39932474,43.67430187)(256.36432478,43.67930186)(256.334328,43.67930176)
\curveto(256.30432484,43.67930186)(256.26932487,43.68430186)(256.229328,43.69430176)
\lineto(255.959328,43.75430176)
\curveto(255.86932527,43.77430177)(255.78432536,43.80430174)(255.704328,43.84430176)
\curveto(255.36432578,43.98430156)(255.07432607,44.1393014)(254.834328,44.30930176)
\curveto(254.59432655,44.48930105)(254.37432677,44.71930082)(254.174328,44.99930176)
\curveto(254.02432712,45.22930031)(253.90932723,45.46930007)(253.829328,45.71930176)
\curveto(253.80932733,45.76929977)(253.79932734,45.81429973)(253.799328,45.85430176)
\curveto(253.79932734,45.90429964)(253.78932735,45.95429959)(253.769328,46.00430176)
\curveto(253.74932739,46.06429948)(253.73432741,46.1442994)(253.724328,46.24430176)
\curveto(253.72432742,46.3442992)(253.7443274,46.41929912)(253.784328,46.46930176)
\curveto(253.83432731,46.54929899)(253.91432723,46.59429895)(254.024328,46.60430176)
\curveto(254.13432701,46.61429893)(254.24932689,46.61929892)(254.369328,46.61930176)
\lineto(254.534328,46.61930176)
\curveto(254.59432655,46.61929892)(254.64932649,46.60929893)(254.699328,46.58930176)
\curveto(254.78932635,46.56929897)(254.85932628,46.52929901)(254.909328,46.46930176)
\curveto(254.97932616,46.37929916)(255.02432612,46.26929927)(255.044328,46.13930176)
\curveto(255.07432607,46.01929952)(255.11932602,45.91429963)(255.179328,45.82430176)
\curveto(255.36932577,45.48430006)(255.62932551,45.21430033)(255.959328,45.01430176)
\curveto(256.05932508,44.95430059)(256.16432498,44.90430064)(256.274328,44.86430176)
\curveto(256.39432475,44.83430071)(256.51432463,44.79930074)(256.634328,44.75930176)
\curveto(256.80432434,44.70930083)(257.00932413,44.68930085)(257.249328,44.69930176)
\curveto(257.49932364,44.71930082)(257.69932344,44.75430079)(257.849328,44.80430176)
\curveto(258.21932292,44.92430062)(258.50932263,45.08430046)(258.719328,45.28430176)
\curveto(258.9393222,45.49430005)(259.11932202,45.77429977)(259.259328,46.12430176)
\curveto(259.30932183,46.22429932)(259.3393218,46.32929921)(259.349328,46.43930176)
\curveto(259.36932177,46.54929899)(259.39432175,46.66429888)(259.424328,46.78430176)
\lineto(259.424328,46.88930176)
\curveto(259.43432171,46.92929861)(259.4393217,46.96929857)(259.439328,47.00930176)
\curveto(259.44932169,47.0392985)(259.44932169,47.07429847)(259.439328,47.11430176)
\lineto(259.439328,47.23430176)
\curveto(259.4393217,47.49429805)(259.40932173,47.7392978)(259.349328,47.96930176)
\curveto(259.2393219,48.31929722)(259.08432206,48.61429693)(258.884328,48.85430176)
\curveto(258.68432246,49.10429644)(258.42432272,49.29929624)(258.104328,49.43930176)
\lineto(257.924328,49.49930176)
\curveto(257.87432327,49.51929602)(257.81432333,49.539296)(257.744328,49.55930176)
\curveto(257.69432345,49.57929596)(257.63432351,49.58929595)(257.564328,49.58930176)
\curveto(257.50432364,49.59929594)(257.4393237,49.61429593)(257.369328,49.63430176)
\lineto(257.219328,49.63430176)
\curveto(257.17932396,49.65429589)(257.12432402,49.66429588)(257.054328,49.66430176)
\curveto(256.99432415,49.66429588)(256.9393242,49.65429589)(256.889328,49.63430176)
\lineto(256.784328,49.63430176)
\curveto(256.75432439,49.63429591)(256.71932442,49.62929591)(256.679328,49.61930176)
\lineto(256.439328,49.55930176)
\curveto(256.35932478,49.54929599)(256.27932486,49.52929601)(256.199328,49.49930176)
\curveto(255.95932518,49.39929614)(255.72932541,49.26429628)(255.509328,49.09430176)
\curveto(255.41932572,49.02429652)(255.33432581,48.94929659)(255.254328,48.86930176)
\curveto(255.17432597,48.79929674)(255.07432607,48.7442968)(254.954328,48.70430176)
\curveto(254.86432628,48.67429687)(254.72432642,48.66429688)(254.534328,48.67430176)
\curveto(254.35432679,48.68429686)(254.23432691,48.70929683)(254.174328,48.74930176)
\curveto(254.12432702,48.78929675)(254.08432706,48.84929669)(254.054328,48.92930176)
\curveto(254.03432711,49.00929653)(254.03432711,49.09429645)(254.054328,49.18430176)
\curveto(254.08432706,49.30429624)(254.10432704,49.42429612)(254.114328,49.54430176)
\curveto(254.13432701,49.67429587)(254.15932698,49.79929574)(254.189328,49.91930176)
\curveto(254.20932693,49.95929558)(254.21432693,49.99429555)(254.204328,50.02430176)
\curveto(254.20432694,50.06429548)(254.21432693,50.10929543)(254.234328,50.15930176)
\curveto(254.25432689,50.24929529)(254.26932687,50.3392952)(254.279328,50.42930176)
\curveto(254.28932685,50.52929501)(254.30932683,50.62429492)(254.339328,50.71430176)
\curveto(254.34932679,50.77429477)(254.35432679,50.83429471)(254.354328,50.89430176)
\curveto(254.36432678,50.95429459)(254.37932676,51.01429453)(254.399328,51.07430176)
\curveto(254.44932669,51.27429427)(254.48432666,51.47929406)(254.504328,51.68930176)
\curveto(254.53432661,51.90929363)(254.57432657,52.11929342)(254.624328,52.31930176)
\curveto(254.65432649,52.41929312)(254.67432647,52.51929302)(254.684328,52.61930176)
\curveto(254.69432645,52.71929282)(254.70932643,52.81929272)(254.729328,52.91930176)
\curveto(254.7393264,52.94929259)(254.7443264,52.98929255)(254.744328,53.03930176)
\curveto(254.77432637,53.14929239)(254.79432635,53.25429229)(254.804328,53.35430176)
\curveto(254.82432632,53.46429208)(254.84932629,53.57429197)(254.879328,53.68430176)
\curveto(254.89932624,53.76429178)(254.91432623,53.83429171)(254.924328,53.89430176)
\curveto(254.93432621,53.96429158)(254.95932618,54.02429152)(254.999328,54.07430176)
\curveto(255.01932612,54.10429144)(255.04932609,54.12429142)(255.089328,54.13430176)
\curveto(255.12932601,54.15429139)(255.17432597,54.17429137)(255.224328,54.19430176)
\curveto(255.28432586,54.19429135)(255.32432582,54.19929134)(255.344328,54.20930176)
}
}
{
\newrgbcolor{curcolor}{0 0 0}
\pscustom[linestyle=none,fillstyle=solid,fillcolor=curcolor]
{
\newpath
\moveto(263.13893738,45.43430176)
\lineto(263.43893738,45.43430176)
\curveto(263.54893532,45.4443001)(263.65393521,45.4443001)(263.75393738,45.43430176)
\curveto(263.863935,45.43430011)(263.9639349,45.42430012)(264.05393738,45.40430176)
\curveto(264.14393472,45.39430015)(264.21393465,45.36930017)(264.26393738,45.32930176)
\curveto(264.28393458,45.30930023)(264.29893457,45.27930026)(264.30893738,45.23930176)
\curveto(264.32893454,45.19930034)(264.34893452,45.15430039)(264.36893738,45.10430176)
\lineto(264.36893738,45.02930176)
\curveto(264.37893449,44.97930056)(264.37893449,44.92430062)(264.36893738,44.86430176)
\lineto(264.36893738,44.71430176)
\lineto(264.36893738,44.23430176)
\curveto(264.3689345,44.06430148)(264.32893454,43.9443016)(264.24893738,43.87430176)
\curveto(264.17893469,43.82430172)(264.08893478,43.79930174)(263.97893738,43.79930176)
\lineto(263.64893738,43.79930176)
\lineto(263.19893738,43.79930176)
\curveto(263.04893582,43.79930174)(262.93393593,43.82930171)(262.85393738,43.88930176)
\curveto(262.81393605,43.91930162)(262.78393608,43.96930157)(262.76393738,44.03930176)
\curveto(262.74393612,44.11930142)(262.72893614,44.20430134)(262.71893738,44.29430176)
\lineto(262.71893738,44.57930176)
\curveto(262.72893614,44.67930086)(262.73393613,44.76430078)(262.73393738,44.83430176)
\lineto(262.73393738,45.02930176)
\curveto(262.73393613,45.08930045)(262.74393612,45.1443004)(262.76393738,45.19430176)
\curveto(262.80393606,45.30430024)(262.87393599,45.37430017)(262.97393738,45.40430176)
\curveto(263.00393586,45.40430014)(263.05893581,45.41430013)(263.13893738,45.43430176)
}
}
{
\newrgbcolor{curcolor}{0 0 0}
\pscustom[linestyle=none,fillstyle=solid,fillcolor=curcolor]
{
\newpath
\moveto(266.80409363,54.20930176)
\lineto(271.60409363,54.20930176)
\lineto(272.60909363,54.20930176)
\curveto(272.74908653,54.20929133)(272.86908641,54.19929134)(272.96909363,54.17930176)
\curveto(273.0790862,54.16929137)(273.15908612,54.12429142)(273.20909363,54.04430176)
\curveto(273.22908605,54.00429154)(273.23908604,53.95429159)(273.23909363,53.89430176)
\curveto(273.24908603,53.83429171)(273.25408602,53.76929177)(273.25409363,53.69930176)
\lineto(273.25409363,53.42930176)
\curveto(273.25408602,53.3392922)(273.24408603,53.25929228)(273.22409363,53.18930176)
\curveto(273.18408609,53.10929243)(273.13908614,53.0392925)(273.08909363,52.97930176)
\lineto(272.93909363,52.79930176)
\curveto(272.90908637,52.74929279)(272.8740864,52.70929283)(272.83409363,52.67930176)
\curveto(272.79408648,52.64929289)(272.75408652,52.60929293)(272.71409363,52.55930176)
\curveto(272.63408664,52.44929309)(272.54908673,52.3392932)(272.45909363,52.22930176)
\curveto(272.36908691,52.12929341)(272.28408699,52.02429352)(272.20409363,51.91430176)
\curveto(272.06408721,51.71429383)(271.92408735,51.50429404)(271.78409363,51.28430176)
\curveto(271.64408763,51.07429447)(271.50408777,50.85929468)(271.36409363,50.63930176)
\curveto(271.31408796,50.54929499)(271.26408801,50.45429509)(271.21409363,50.35430176)
\curveto(271.16408811,50.25429529)(271.10908817,50.15929538)(271.04909363,50.06930176)
\curveto(271.02908825,50.04929549)(271.01908826,50.02429552)(271.01909363,49.99430176)
\curveto(271.01908826,49.96429558)(271.00908827,49.9392956)(270.98909363,49.91930176)
\curveto(270.91908836,49.81929572)(270.85408842,49.70429584)(270.79409363,49.57430176)
\curveto(270.73408854,49.45429609)(270.6790886,49.3392962)(270.62909363,49.22930176)
\curveto(270.52908875,48.99929654)(270.43408884,48.76429678)(270.34409363,48.52430176)
\curveto(270.25408902,48.28429726)(270.15408912,48.0442975)(270.04409363,47.80430176)
\curveto(270.02408925,47.75429779)(270.00908927,47.70929783)(269.99909363,47.66930176)
\curveto(269.99908928,47.62929791)(269.98908929,47.58429796)(269.96909363,47.53430176)
\curveto(269.91908936,47.41429813)(269.8740894,47.28929825)(269.83409363,47.15930176)
\curveto(269.80408947,47.0392985)(269.76908951,46.91929862)(269.72909363,46.79930176)
\curveto(269.64908963,46.56929897)(269.58408969,46.32929921)(269.53409363,46.07930176)
\curveto(269.49408978,45.8392997)(269.44408983,45.59929994)(269.38409363,45.35930176)
\curveto(269.34408993,45.20930033)(269.31908996,45.05930048)(269.30909363,44.90930176)
\curveto(269.29908998,44.75930078)(269.27909,44.60930093)(269.24909363,44.45930176)
\curveto(269.23909004,44.41930112)(269.23409004,44.35930118)(269.23409363,44.27930176)
\curveto(269.20409007,44.15930138)(269.1740901,44.05930148)(269.14409363,43.97930176)
\curveto(269.11409016,43.89930164)(269.04409023,43.8443017)(268.93409363,43.81430176)
\curveto(268.88409039,43.79430175)(268.82909045,43.78430176)(268.76909363,43.78430176)
\lineto(268.57409363,43.78430176)
\curveto(268.43409084,43.78430176)(268.29409098,43.78930175)(268.15409363,43.79930176)
\curveto(268.02409125,43.80930173)(267.92909135,43.85430169)(267.86909363,43.93430176)
\curveto(267.82909145,43.99430155)(267.80909147,44.07930146)(267.80909363,44.18930176)
\curveto(267.81909146,44.29930124)(267.83409144,44.39430115)(267.85409363,44.47430176)
\lineto(267.85409363,44.54930176)
\curveto(267.86409141,44.57930096)(267.86909141,44.60930093)(267.86909363,44.63930176)
\curveto(267.88909139,44.71930082)(267.89909138,44.79430075)(267.89909363,44.86430176)
\curveto(267.89909138,44.93430061)(267.90909137,45.00430054)(267.92909363,45.07430176)
\curveto(267.9790913,45.26430028)(268.01909126,45.44930009)(268.04909363,45.62930176)
\curveto(268.0790912,45.81929972)(268.11909116,45.99929954)(268.16909363,46.16930176)
\curveto(268.18909109,46.21929932)(268.19909108,46.25929928)(268.19909363,46.28930176)
\curveto(268.19909108,46.31929922)(268.20409107,46.35429919)(268.21409363,46.39430176)
\curveto(268.31409096,46.69429885)(268.40409087,46.98929855)(268.48409363,47.27930176)
\curveto(268.5740907,47.56929797)(268.6790906,47.84929769)(268.79909363,48.11930176)
\curveto(269.05909022,48.69929684)(269.32908995,49.24929629)(269.60909363,49.76930176)
\curveto(269.88908939,50.29929524)(270.19908908,50.80429474)(270.53909363,51.28430176)
\curveto(270.6790886,51.48429406)(270.82908845,51.67429387)(270.98909363,51.85430176)
\curveto(271.14908813,52.0442935)(271.29908798,52.23429331)(271.43909363,52.42430176)
\curveto(271.4790878,52.47429307)(271.51408776,52.51929302)(271.54409363,52.55930176)
\curveto(271.58408769,52.60929293)(271.61908766,52.65929288)(271.64909363,52.70930176)
\curveto(271.65908762,52.72929281)(271.66908761,52.75429279)(271.67909363,52.78430176)
\curveto(271.69908758,52.81429273)(271.69908758,52.8442927)(271.67909363,52.87430176)
\curveto(271.65908762,52.93429261)(271.62408765,52.96929257)(271.57409363,52.97930176)
\curveto(271.52408775,52.99929254)(271.4740878,53.01929252)(271.42409363,53.03930176)
\lineto(271.31909363,53.03930176)
\curveto(271.279088,53.04929249)(271.22908805,53.04929249)(271.16909363,53.03930176)
\lineto(271.01909363,53.03930176)
\lineto(270.41909363,53.03930176)
\lineto(267.77909363,53.03930176)
\lineto(267.04409363,53.03930176)
\lineto(266.80409363,53.03930176)
\curveto(266.73409254,53.04929249)(266.6740926,53.06429248)(266.62409363,53.08430176)
\curveto(266.53409274,53.12429242)(266.4740928,53.18429236)(266.44409363,53.26430176)
\curveto(266.39409288,53.36429218)(266.3790929,53.50929203)(266.39909363,53.69930176)
\curveto(266.41909286,53.89929164)(266.45409282,54.03429151)(266.50409363,54.10430176)
\curveto(266.52409275,54.12429142)(266.54909273,54.1392914)(266.57909363,54.14930176)
\lineto(266.69909363,54.20930176)
\curveto(266.71909256,54.20929133)(266.73409254,54.20429134)(266.74409363,54.19430176)
\curveto(266.76409251,54.19429135)(266.78409249,54.19929134)(266.80409363,54.20930176)
}
}
{
\newrgbcolor{curcolor}{0 0 0}
\pscustom[linestyle=none,fillstyle=solid,fillcolor=curcolor]
{
\newpath
\moveto(284.498703,52.31930176)
\curveto(284.2986927,52.02929351)(284.08869291,51.7442938)(283.868703,51.46430176)
\curveto(283.65869334,51.18429436)(283.45369355,50.89929464)(283.253703,50.60930176)
\curveto(282.65369435,49.75929578)(282.04869495,48.91929662)(281.438703,48.08930176)
\curveto(280.82869617,47.26929827)(280.22369678,46.43429911)(279.623703,45.58430176)
\lineto(279.113703,44.86430176)
\lineto(278.603703,44.17430176)
\curveto(278.52369848,44.06430148)(278.44369856,43.94930159)(278.363703,43.82930176)
\curveto(278.28369872,43.70930183)(278.18869881,43.61430193)(278.078703,43.54430176)
\curveto(278.03869896,43.52430202)(277.97369903,43.50930203)(277.883703,43.49930176)
\curveto(277.8036992,43.47930206)(277.71369929,43.46930207)(277.613703,43.46930176)
\curveto(277.51369949,43.46930207)(277.41869958,43.47430207)(277.328703,43.48430176)
\curveto(277.24869975,43.49430205)(277.18869981,43.51430203)(277.148703,43.54430176)
\curveto(277.11869988,43.56430198)(277.09369991,43.59930194)(277.073703,43.64930176)
\curveto(277.06369994,43.68930185)(277.06869993,43.73430181)(277.088703,43.78430176)
\curveto(277.12869987,43.86430168)(277.17369983,43.9393016)(277.223703,44.00930176)
\curveto(277.28369972,44.08930145)(277.33869966,44.16930137)(277.388703,44.24930176)
\curveto(277.62869937,44.58930095)(277.87369913,44.92430062)(278.123703,45.25430176)
\curveto(278.37369863,45.58429996)(278.61369839,45.91929962)(278.843703,46.25930176)
\curveto(279.003698,46.47929906)(279.16369784,46.69429885)(279.323703,46.90430176)
\curveto(279.48369752,47.11429843)(279.64369736,47.32929821)(279.803703,47.54930176)
\curveto(280.16369684,48.06929747)(280.52869647,48.57929696)(280.898703,49.07930176)
\curveto(281.26869573,49.57929596)(281.63869536,50.08929545)(282.008703,50.60930176)
\curveto(282.14869485,50.80929473)(282.28869471,51.00429454)(282.428703,51.19430176)
\curveto(282.57869442,51.38429416)(282.72369428,51.57929396)(282.863703,51.77930176)
\curveto(283.07369393,52.07929346)(283.28869371,52.37929316)(283.508703,52.67930176)
\lineto(284.168703,53.57930176)
\lineto(284.348703,53.84930176)
\lineto(284.558703,54.11930176)
\lineto(284.678703,54.29930176)
\curveto(284.72869227,54.35929118)(284.77869222,54.41429113)(284.828703,54.46430176)
\curveto(284.8986921,54.51429103)(284.97369203,54.54929099)(285.053703,54.56930176)
\curveto(285.07369193,54.57929096)(285.0986919,54.57929096)(285.128703,54.56930176)
\curveto(285.16869183,54.56929097)(285.1986918,54.57929096)(285.218703,54.59930176)
\curveto(285.33869166,54.59929094)(285.47369153,54.59429095)(285.623703,54.58430176)
\curveto(285.77369123,54.58429096)(285.86369114,54.539291)(285.893703,54.44930176)
\curveto(285.91369109,54.41929112)(285.91869108,54.38429116)(285.908703,54.34430176)
\curveto(285.8986911,54.30429124)(285.88369112,54.27429127)(285.863703,54.25430176)
\curveto(285.82369118,54.17429137)(285.78369122,54.10429144)(285.743703,54.04430176)
\curveto(285.7036913,53.98429156)(285.65869134,53.92429162)(285.608703,53.86430176)
\lineto(285.038703,53.08430176)
\curveto(284.85869214,52.83429271)(284.67869232,52.57929296)(284.498703,52.31930176)
\moveto(277.643703,48.41930176)
\curveto(277.59369941,48.4392971)(277.54369946,48.4442971)(277.493703,48.43430176)
\curveto(277.44369956,48.42429712)(277.39369961,48.42929711)(277.343703,48.44930176)
\curveto(277.23369977,48.46929707)(277.12869987,48.48929705)(277.028703,48.50930176)
\curveto(276.93870006,48.539297)(276.84370016,48.57929696)(276.743703,48.62930176)
\curveto(276.41370059,48.76929677)(276.15870084,48.96429658)(275.978703,49.21430176)
\curveto(275.7987012,49.47429607)(275.65370135,49.78429576)(275.543703,50.14430176)
\curveto(275.51370149,50.22429532)(275.49370151,50.30429524)(275.483703,50.38430176)
\curveto(275.47370153,50.47429507)(275.45870154,50.55929498)(275.438703,50.63930176)
\curveto(275.42870157,50.68929485)(275.42370158,50.75429479)(275.423703,50.83430176)
\curveto(275.41370159,50.86429468)(275.40870159,50.89429465)(275.408703,50.92430176)
\curveto(275.40870159,50.96429458)(275.4037016,50.99929454)(275.393703,51.02930176)
\lineto(275.393703,51.17930176)
\curveto(275.38370162,51.22929431)(275.37870162,51.28929425)(275.378703,51.35930176)
\curveto(275.37870162,51.4392941)(275.38370162,51.50429404)(275.393703,51.55430176)
\lineto(275.393703,51.71930176)
\curveto(275.41370159,51.76929377)(275.41870158,51.81429373)(275.408703,51.85430176)
\curveto(275.40870159,51.90429364)(275.41370159,51.94929359)(275.423703,51.98930176)
\curveto(275.43370157,52.02929351)(275.43870156,52.06429348)(275.438703,52.09430176)
\curveto(275.43870156,52.13429341)(275.44370156,52.17429337)(275.453703,52.21430176)
\curveto(275.48370152,52.32429322)(275.5037015,52.43429311)(275.513703,52.54430176)
\curveto(275.53370147,52.66429288)(275.56870143,52.77929276)(275.618703,52.88930176)
\curveto(275.75870124,53.22929231)(275.91870108,53.50429204)(276.098703,53.71430176)
\curveto(276.28870071,53.93429161)(276.55870044,54.11429143)(276.908703,54.25430176)
\curveto(276.98870001,54.28429126)(277.07369993,54.30429124)(277.163703,54.31430176)
\curveto(277.25369975,54.33429121)(277.34869965,54.35429119)(277.448703,54.37430176)
\curveto(277.47869952,54.38429116)(277.53369947,54.38429116)(277.613703,54.37430176)
\curveto(277.69369931,54.37429117)(277.74369926,54.38429116)(277.763703,54.40430176)
\curveto(278.32369868,54.41429113)(278.77369823,54.30429124)(279.113703,54.07430176)
\curveto(279.46369754,53.8442917)(279.72369728,53.539292)(279.893703,53.15930176)
\curveto(279.93369707,53.06929247)(279.96869703,52.97429257)(279.998703,52.87430176)
\curveto(280.02869697,52.77429277)(280.05369695,52.67429287)(280.073703,52.57430176)
\curveto(280.09369691,52.544293)(280.0986969,52.51429303)(280.088703,52.48430176)
\curveto(280.08869691,52.45429309)(280.09369691,52.42429312)(280.103703,52.39430176)
\curveto(280.13369687,52.28429326)(280.15369685,52.15929338)(280.163703,52.01930176)
\curveto(280.17369683,51.88929365)(280.18369682,51.75429379)(280.193703,51.61430176)
\lineto(280.193703,51.44930176)
\curveto(280.2036968,51.38929415)(280.2036968,51.33429421)(280.193703,51.28430176)
\curveto(280.18369682,51.23429431)(280.17869682,51.18429436)(280.178703,51.13430176)
\lineto(280.178703,50.99930176)
\curveto(280.16869683,50.95929458)(280.16369684,50.91929462)(280.163703,50.87930176)
\curveto(280.17369683,50.8392947)(280.16869683,50.79429475)(280.148703,50.74430176)
\curveto(280.12869687,50.63429491)(280.10869689,50.52929501)(280.088703,50.42930176)
\curveto(280.07869692,50.32929521)(280.05869694,50.22929531)(280.028703,50.12930176)
\curveto(279.8986971,49.76929577)(279.73369727,49.45429609)(279.533703,49.18430176)
\curveto(279.33369767,48.91429663)(279.05869794,48.70929683)(278.708703,48.56930176)
\curveto(278.62869837,48.539297)(278.54369846,48.51429703)(278.453703,48.49430176)
\lineto(278.183703,48.43430176)
\curveto(278.13369887,48.42429712)(278.08869891,48.41929712)(278.048703,48.41930176)
\curveto(278.00869899,48.42929711)(277.96869903,48.42929711)(277.928703,48.41930176)
\curveto(277.82869917,48.39929714)(277.73369927,48.39929714)(277.643703,48.41930176)
\moveto(276.803703,49.81430176)
\curveto(276.84370016,49.7442958)(276.88370012,49.67929586)(276.923703,49.61930176)
\curveto(276.96370004,49.56929597)(277.01369999,49.51929602)(277.073703,49.46930176)
\lineto(277.223703,49.34930176)
\curveto(277.28369972,49.31929622)(277.34869965,49.29429625)(277.418703,49.27430176)
\curveto(277.45869954,49.25429629)(277.49369951,49.2442963)(277.523703,49.24430176)
\curveto(277.56369944,49.25429629)(277.6036994,49.24929629)(277.643703,49.22930176)
\curveto(277.67369933,49.22929631)(277.71369929,49.22429632)(277.763703,49.21430176)
\curveto(277.81369919,49.21429633)(277.85369915,49.21929632)(277.883703,49.22930176)
\lineto(278.108703,49.27430176)
\curveto(278.35869864,49.35429619)(278.54369846,49.47929606)(278.663703,49.64930176)
\curveto(278.74369826,49.74929579)(278.81369819,49.87929566)(278.873703,50.03930176)
\curveto(278.95369805,50.21929532)(279.01369799,50.4442951)(279.053703,50.71430176)
\curveto(279.09369791,50.99429455)(279.10869789,51.27429427)(279.098703,51.55430176)
\curveto(279.08869791,51.8442937)(279.05869794,52.11929342)(279.008703,52.37930176)
\curveto(278.95869804,52.6392929)(278.88369812,52.84929269)(278.783703,53.00930176)
\curveto(278.66369834,53.20929233)(278.51369849,53.35929218)(278.333703,53.45930176)
\curveto(278.25369875,53.50929203)(278.16369884,53.539292)(278.063703,53.54930176)
\curveto(277.96369904,53.56929197)(277.85869914,53.57929196)(277.748703,53.57930176)
\curveto(277.72869927,53.56929197)(277.7036993,53.56429198)(277.673703,53.56430176)
\curveto(277.65369935,53.57429197)(277.63369937,53.57429197)(277.613703,53.56430176)
\curveto(277.56369944,53.55429199)(277.51869948,53.544292)(277.478703,53.53430176)
\curveto(277.43869956,53.53429201)(277.3986996,53.52429202)(277.358703,53.50430176)
\curveto(277.17869982,53.42429212)(277.02869997,53.30429224)(276.908703,53.14430176)
\curveto(276.7987002,52.98429256)(276.70870029,52.80429274)(276.638703,52.60430176)
\curveto(276.57870042,52.41429313)(276.53370047,52.18929335)(276.503703,51.92930176)
\curveto(276.48370052,51.66929387)(276.47870052,51.40429414)(276.488703,51.13430176)
\curveto(276.4987005,50.87429467)(276.52870047,50.62429492)(276.578703,50.38430176)
\curveto(276.63870036,50.15429539)(276.71370029,49.96429558)(276.803703,49.81430176)
\moveto(287.603703,46.82930176)
\curveto(287.61368939,46.77929876)(287.61868938,46.68929885)(287.618703,46.55930176)
\curveto(287.61868938,46.42929911)(287.60868939,46.3392992)(287.588703,46.28930176)
\curveto(287.56868943,46.2392993)(287.56368944,46.18429936)(287.573703,46.12430176)
\curveto(287.58368942,46.07429947)(287.58368942,46.02429952)(287.573703,45.97430176)
\curveto(287.53368947,45.83429971)(287.5036895,45.69929984)(287.483703,45.56930176)
\curveto(287.47368953,45.4393001)(287.44368956,45.31930022)(287.393703,45.20930176)
\curveto(287.25368975,44.85930068)(287.08868991,44.56430098)(286.898703,44.32430176)
\curveto(286.70869029,44.09430145)(286.43869056,43.90930163)(286.088703,43.76930176)
\curveto(286.00869099,43.7393018)(285.92369108,43.71930182)(285.833703,43.70930176)
\curveto(285.74369126,43.68930185)(285.65869134,43.66930187)(285.578703,43.64930176)
\curveto(285.52869147,43.6393019)(285.47869152,43.63430191)(285.428703,43.63430176)
\curveto(285.37869162,43.63430191)(285.32869167,43.62930191)(285.278703,43.61930176)
\curveto(285.24869175,43.60930193)(285.1986918,43.60930193)(285.128703,43.61930176)
\curveto(285.05869194,43.61930192)(285.00869199,43.62430192)(284.978703,43.63430176)
\curveto(284.91869208,43.65430189)(284.85869214,43.66430188)(284.798703,43.66430176)
\curveto(284.74869225,43.65430189)(284.6986923,43.65930188)(284.648703,43.67930176)
\curveto(284.55869244,43.69930184)(284.46869253,43.72430182)(284.378703,43.75430176)
\curveto(284.2986927,43.77430177)(284.21869278,43.80430174)(284.138703,43.84430176)
\curveto(283.81869318,43.98430156)(283.56869343,44.17930136)(283.388703,44.42930176)
\curveto(283.20869379,44.68930085)(283.05869394,44.99430055)(282.938703,45.34430176)
\curveto(282.91869408,45.42430012)(282.9036941,45.50930003)(282.893703,45.59930176)
\curveto(282.88369412,45.68929985)(282.86869413,45.77429977)(282.848703,45.85430176)
\curveto(282.83869416,45.88429966)(282.83369417,45.91429963)(282.833703,45.94430176)
\lineto(282.833703,46.04930176)
\curveto(282.81369419,46.12929941)(282.8036942,46.20929933)(282.803703,46.28930176)
\lineto(282.803703,46.42430176)
\curveto(282.78369422,46.52429902)(282.78369422,46.62429892)(282.803703,46.72430176)
\lineto(282.803703,46.90430176)
\curveto(282.81369419,46.95429859)(282.81869418,46.99929854)(282.818703,47.03930176)
\curveto(282.81869418,47.08929845)(282.82369418,47.13429841)(282.833703,47.17430176)
\curveto(282.84369416,47.21429833)(282.84869415,47.24929829)(282.848703,47.27930176)
\curveto(282.84869415,47.31929822)(282.85369415,47.35929818)(282.863703,47.39930176)
\lineto(282.923703,47.72930176)
\curveto(282.94369406,47.84929769)(282.97369403,47.95929758)(283.013703,48.05930176)
\curveto(283.15369385,48.38929715)(283.31369369,48.66429688)(283.493703,48.88430176)
\curveto(283.68369332,49.11429643)(283.94369306,49.29929624)(284.273703,49.43930176)
\curveto(284.35369265,49.47929606)(284.43869256,49.50429604)(284.528703,49.51430176)
\lineto(284.828703,49.57430176)
\lineto(284.963703,49.57430176)
\curveto(285.01369199,49.58429596)(285.06369194,49.58929595)(285.113703,49.58930176)
\curveto(285.68369132,49.60929593)(286.14369086,49.50429604)(286.493703,49.27430176)
\curveto(286.85369015,49.05429649)(287.11868988,48.75429679)(287.288703,48.37430176)
\curveto(287.33868966,48.27429727)(287.37868962,48.17429737)(287.408703,48.07430176)
\curveto(287.43868956,47.97429757)(287.46868953,47.86929767)(287.498703,47.75930176)
\curveto(287.50868949,47.71929782)(287.51368949,47.68429786)(287.513703,47.65430176)
\curveto(287.51368949,47.63429791)(287.51868948,47.60429794)(287.528703,47.56430176)
\curveto(287.54868945,47.49429805)(287.55868944,47.41929812)(287.558703,47.33930176)
\curveto(287.55868944,47.25929828)(287.56868943,47.17929836)(287.588703,47.09930176)
\curveto(287.58868941,47.04929849)(287.58868941,47.00429854)(287.588703,46.96430176)
\curveto(287.58868941,46.92429862)(287.59368941,46.87929866)(287.603703,46.82930176)
\moveto(286.493703,46.39430176)
\curveto(286.5036905,46.4442991)(286.50869049,46.51929902)(286.508703,46.61930176)
\curveto(286.51869048,46.71929882)(286.51369049,46.79429875)(286.493703,46.84430176)
\curveto(286.47369053,46.90429864)(286.46869053,46.95929858)(286.478703,47.00930176)
\curveto(286.4986905,47.06929847)(286.4986905,47.12929841)(286.478703,47.18930176)
\curveto(286.46869053,47.21929832)(286.46369054,47.25429829)(286.463703,47.29430176)
\curveto(286.46369054,47.33429821)(286.45869054,47.37429817)(286.448703,47.41430176)
\curveto(286.42869057,47.49429805)(286.40869059,47.56929797)(286.388703,47.63930176)
\curveto(286.37869062,47.71929782)(286.36369064,47.79929774)(286.343703,47.87930176)
\curveto(286.31369069,47.9392976)(286.28869071,47.99929754)(286.268703,48.05930176)
\curveto(286.24869075,48.11929742)(286.21869078,48.17929736)(286.178703,48.23930176)
\curveto(286.07869092,48.40929713)(285.94869105,48.544297)(285.788703,48.64430176)
\curveto(285.70869129,48.69429685)(285.61369139,48.72929681)(285.503703,48.74930176)
\curveto(285.39369161,48.76929677)(285.26869173,48.77929676)(285.128703,48.77930176)
\curveto(285.10869189,48.76929677)(285.08369192,48.76429678)(285.053703,48.76430176)
\curveto(285.02369198,48.77429677)(284.99369201,48.77429677)(284.963703,48.76430176)
\lineto(284.813703,48.70430176)
\curveto(284.76369224,48.69429685)(284.71869228,48.67929686)(284.678703,48.65930176)
\curveto(284.48869251,48.54929699)(284.34369266,48.40429714)(284.243703,48.22430176)
\curveto(284.15369285,48.0442975)(284.07369293,47.8392977)(284.003703,47.60930176)
\curveto(283.96369304,47.47929806)(283.94369306,47.3442982)(283.943703,47.20430176)
\curveto(283.94369306,47.07429847)(283.93369307,46.92929861)(283.913703,46.76930176)
\curveto(283.9036931,46.71929882)(283.89369311,46.65929888)(283.883703,46.58930176)
\curveto(283.88369312,46.51929902)(283.89369311,46.45929908)(283.913703,46.40930176)
\lineto(283.913703,46.24430176)
\lineto(283.913703,46.06430176)
\curveto(283.92369308,46.01429953)(283.93369307,45.95929958)(283.943703,45.89930176)
\curveto(283.95369305,45.84929969)(283.95869304,45.79429975)(283.958703,45.73430176)
\curveto(283.96869303,45.67429987)(283.98369302,45.61929992)(284.003703,45.56930176)
\curveto(284.05369295,45.37930016)(284.11369289,45.20430034)(284.183703,45.04430176)
\curveto(284.25369275,44.88430066)(284.35869264,44.75430079)(284.498703,44.65430176)
\curveto(284.62869237,44.55430099)(284.76869223,44.48430106)(284.918703,44.44430176)
\curveto(284.94869205,44.43430111)(284.97369203,44.42930111)(284.993703,44.42930176)
\curveto(285.02369198,44.4393011)(285.05369195,44.4393011)(285.083703,44.42930176)
\curveto(285.1036919,44.42930111)(285.13369187,44.42430112)(285.173703,44.41430176)
\curveto(285.21369179,44.41430113)(285.24869175,44.41930112)(285.278703,44.42930176)
\curveto(285.31869168,44.4393011)(285.35869164,44.4443011)(285.398703,44.44430176)
\curveto(285.43869156,44.4443011)(285.47869152,44.45430109)(285.518703,44.47430176)
\curveto(285.75869124,44.55430099)(285.95369105,44.68930085)(286.103703,44.87930176)
\curveto(286.22369078,45.05930048)(286.31369069,45.26430028)(286.373703,45.49430176)
\curveto(286.39369061,45.56429998)(286.40869059,45.63429991)(286.418703,45.70430176)
\curveto(286.42869057,45.78429976)(286.44369056,45.86429968)(286.463703,45.94430176)
\curveto(286.46369054,46.00429954)(286.46869053,46.04929949)(286.478703,46.07930176)
\curveto(286.47869052,46.09929944)(286.47869052,46.12429942)(286.478703,46.15430176)
\curveto(286.47869052,46.19429935)(286.48369052,46.22429932)(286.493703,46.24430176)
\lineto(286.493703,46.39430176)
}
}
{
\newrgbcolor{curcolor}{0 0 0}
\pscustom[linestyle=none,fillstyle=solid,fillcolor=curcolor]
{
\newpath
\moveto(86.42862244,135.86571289)
\curveto(86.49861479,135.81570943)(86.53861475,135.7457095)(86.54862244,135.65571289)
\curveto(86.56861472,135.56570968)(86.57861471,135.46070979)(86.57862244,135.34071289)
\curveto(86.57861471,135.29070996)(86.57361472,135.24071001)(86.56362244,135.19071289)
\curveto(86.56361473,135.14071011)(86.55361474,135.09571015)(86.53362244,135.05571289)
\curveto(86.50361479,134.96571028)(86.44361485,134.90571034)(86.35362244,134.87571289)
\curveto(86.27361502,134.85571039)(86.17861511,134.8457104)(86.06862244,134.84571289)
\lineto(85.75362244,134.84571289)
\curveto(85.64361565,134.85571039)(85.53861575,134.8457104)(85.43862244,134.81571289)
\curveto(85.29861599,134.78571046)(85.20861608,134.70571054)(85.16862244,134.57571289)
\curveto(85.14861614,134.50571074)(85.13861615,134.42071083)(85.13862244,134.32071289)
\lineto(85.13862244,134.05071289)
\lineto(85.13862244,133.10571289)
\lineto(85.13862244,132.77571289)
\curveto(85.13861615,132.66571258)(85.11861617,132.58071267)(85.07862244,132.52071289)
\curveto(85.03861625,132.46071279)(84.9886163,132.42071283)(84.92862244,132.40071289)
\curveto(84.87861641,132.39071286)(84.81361648,132.37571287)(84.73362244,132.35571289)
\lineto(84.53862244,132.35571289)
\curveto(84.41861687,132.35571289)(84.31361698,132.36071289)(84.22362244,132.37071289)
\curveto(84.13361716,132.39071286)(84.06361723,132.44071281)(84.01362244,132.52071289)
\curveto(83.98361731,132.57071268)(83.96861732,132.64071261)(83.96862244,132.73071289)
\lineto(83.96862244,133.03071289)
\lineto(83.96862244,134.06571289)
\curveto(83.96861732,134.22571102)(83.95861733,134.37071088)(83.93862244,134.50071289)
\curveto(83.92861736,134.64071061)(83.87361742,134.73571051)(83.77362244,134.78571289)
\curveto(83.72361757,134.80571044)(83.65361764,134.82071043)(83.56362244,134.83071289)
\curveto(83.48361781,134.84071041)(83.3936179,134.8457104)(83.29362244,134.84571289)
\lineto(83.00862244,134.84571289)
\lineto(82.76862244,134.84571289)
\lineto(80.50362244,134.84571289)
\curveto(80.41362088,134.8457104)(80.30862098,134.84071041)(80.18862244,134.83071289)
\lineto(79.85862244,134.83071289)
\curveto(79.74862154,134.83071042)(79.64862164,134.84071041)(79.55862244,134.86071289)
\curveto(79.46862182,134.88071037)(79.40862188,134.91571033)(79.37862244,134.96571289)
\curveto(79.32862196,135.03571021)(79.30362199,135.13071012)(79.30362244,135.25071289)
\lineto(79.30362244,135.59571289)
\lineto(79.30362244,135.86571289)
\curveto(79.34362195,136.03570921)(79.39862189,136.17570907)(79.46862244,136.28571289)
\curveto(79.53862175,136.39570885)(79.61862167,136.51070874)(79.70862244,136.63071289)
\lineto(80.06862244,137.17071289)
\curveto(80.50862078,137.80070745)(80.94362035,138.42070683)(81.37362244,139.03071289)
\lineto(82.69362244,140.89071289)
\curveto(82.85361844,141.12070413)(83.00861828,141.34070391)(83.15862244,141.55071289)
\curveto(83.30861798,141.77070348)(83.46361783,141.99570325)(83.62362244,142.22571289)
\curveto(83.67361762,142.29570295)(83.72361757,142.36070289)(83.77362244,142.42071289)
\curveto(83.82361747,142.49070276)(83.87361742,142.56570268)(83.92362244,142.64571289)
\lineto(83.98362244,142.73571289)
\curveto(84.01361728,142.77570247)(84.04361725,142.80570244)(84.07362244,142.82571289)
\curveto(84.11361718,142.85570239)(84.15361714,142.87570237)(84.19362244,142.88571289)
\curveto(84.23361706,142.90570234)(84.27861701,142.92570232)(84.32862244,142.94571289)
\curveto(84.34861694,142.9457023)(84.36861692,142.94070231)(84.38862244,142.93071289)
\curveto(84.41861687,142.93070232)(84.44361685,142.94070231)(84.46362244,142.96071289)
\curveto(84.5936167,142.96070229)(84.71361658,142.95570229)(84.82362244,142.94571289)
\curveto(84.93361636,142.93570231)(85.01361628,142.89070236)(85.06362244,142.81071289)
\curveto(85.10361619,142.76070249)(85.12361617,142.69070256)(85.12362244,142.60071289)
\curveto(85.13361616,142.51070274)(85.13861615,142.41570283)(85.13862244,142.31571289)
\lineto(85.13862244,136.85571289)
\curveto(85.13861615,136.78570846)(85.13361616,136.71070854)(85.12362244,136.63071289)
\curveto(85.12361617,136.56070869)(85.12861616,136.49070876)(85.13862244,136.42071289)
\lineto(85.13862244,136.31571289)
\curveto(85.15861613,136.26570898)(85.17361612,136.21070904)(85.18362244,136.15071289)
\curveto(85.1936161,136.10070915)(85.21861607,136.06070919)(85.25862244,136.03071289)
\curveto(85.32861596,135.98070927)(85.41361588,135.9507093)(85.51362244,135.94071289)
\lineto(85.84362244,135.94071289)
\curveto(85.95361534,135.94070931)(86.05861523,135.93570931)(86.15862244,135.92571289)
\curveto(86.26861502,135.92570932)(86.35861493,135.90570934)(86.42862244,135.86571289)
\moveto(83.86362244,136.06071289)
\curveto(83.94361735,136.17070908)(83.97861731,136.34070891)(83.96862244,136.57071289)
\lineto(83.96862244,137.18571289)
\lineto(83.96862244,139.66071289)
\lineto(83.96862244,139.97571289)
\curveto(83.97861731,140.09570515)(83.97361732,140.19570505)(83.95362244,140.27571289)
\lineto(83.95362244,140.42571289)
\curveto(83.95361734,140.51570473)(83.93861735,140.60070465)(83.90862244,140.68071289)
\curveto(83.89861739,140.70070455)(83.8886174,140.71070454)(83.87862244,140.71071289)
\lineto(83.83362244,140.75571289)
\curveto(83.81361748,140.76570448)(83.78361751,140.77070448)(83.74362244,140.77071289)
\curveto(83.72361757,140.7507045)(83.70361759,140.73570451)(83.68362244,140.72571289)
\curveto(83.67361762,140.72570452)(83.65861763,140.72070453)(83.63862244,140.71071289)
\curveto(83.57861771,140.66070459)(83.51861777,140.59070466)(83.45862244,140.50071289)
\curveto(83.39861789,140.41070484)(83.34361795,140.33070492)(83.29362244,140.26071289)
\curveto(83.1936181,140.12070513)(83.09861819,139.97570527)(83.00862244,139.82571289)
\curveto(82.91861837,139.68570556)(82.82361847,139.5457057)(82.72362244,139.40571289)
\lineto(82.18362244,138.62571289)
\curveto(82.01361928,138.36570688)(81.83861945,138.10570714)(81.65862244,137.84571289)
\curveto(81.57861971,137.73570751)(81.50361979,137.63070762)(81.43362244,137.53071289)
\lineto(81.22362244,137.23071289)
\curveto(81.17362012,137.1507081)(81.12362017,137.07570817)(81.07362244,137.00571289)
\curveto(81.03362026,136.93570831)(80.9886203,136.86070839)(80.93862244,136.78071289)
\curveto(80.8886204,136.72070853)(80.83862045,136.65570859)(80.78862244,136.58571289)
\curveto(80.74862054,136.52570872)(80.70862058,136.45570879)(80.66862244,136.37571289)
\curveto(80.62862066,136.31570893)(80.60362069,136.245709)(80.59362244,136.16571289)
\curveto(80.58362071,136.09570915)(80.61862067,136.04070921)(80.69862244,136.00071289)
\curveto(80.76862052,135.9507093)(80.87862041,135.92570932)(81.02862244,135.92571289)
\curveto(81.1886201,135.93570931)(81.32361997,135.94070931)(81.43362244,135.94071289)
\lineto(83.11362244,135.94071289)
\lineto(83.54862244,135.94071289)
\curveto(83.69861759,135.94070931)(83.80361749,135.98070927)(83.86362244,136.06071289)
}
}
{
\newrgbcolor{curcolor}{0 0 0}
\pscustom[linestyle=none,fillstyle=solid,fillcolor=curcolor]
{
\newpath
\moveto(89.40823181,142.78071289)
\lineto(93.00823181,142.78071289)
\lineto(93.65323181,142.78071289)
\curveto(93.73322528,142.78070247)(93.80822521,142.77570247)(93.87823181,142.76571289)
\curveto(93.94822507,142.76570248)(94.00822501,142.75570249)(94.05823181,142.73571289)
\curveto(94.12822489,142.70570254)(94.18322483,142.6457026)(94.22323181,142.55571289)
\curveto(94.24322477,142.52570272)(94.25322476,142.48570276)(94.25323181,142.43571289)
\lineto(94.25323181,142.30071289)
\curveto(94.26322475,142.19070306)(94.25822476,142.08570316)(94.23823181,141.98571289)
\curveto(94.22822479,141.88570336)(94.19322482,141.81570343)(94.13323181,141.77571289)
\curveto(94.04322497,141.70570354)(93.90822511,141.67070358)(93.72823181,141.67071289)
\curveto(93.54822547,141.68070357)(93.38322563,141.68570356)(93.23323181,141.68571289)
\lineto(91.23823181,141.68571289)
\lineto(90.74323181,141.68571289)
\lineto(90.60823181,141.68571289)
\curveto(90.56822845,141.68570356)(90.52822849,141.68070357)(90.48823181,141.67071289)
\lineto(90.27823181,141.67071289)
\curveto(90.16822885,141.64070361)(90.08822893,141.60070365)(90.03823181,141.55071289)
\curveto(89.98822903,141.51070374)(89.95322906,141.45570379)(89.93323181,141.38571289)
\curveto(89.9132291,141.32570392)(89.89822912,141.25570399)(89.88823181,141.17571289)
\curveto(89.87822914,141.09570415)(89.85822916,141.00570424)(89.82823181,140.90571289)
\curveto(89.77822924,140.70570454)(89.73822928,140.50070475)(89.70823181,140.29071289)
\curveto(89.67822934,140.08070517)(89.63822938,139.87570537)(89.58823181,139.67571289)
\curveto(89.56822945,139.60570564)(89.55822946,139.53570571)(89.55823181,139.46571289)
\curveto(89.55822946,139.40570584)(89.54822947,139.34070591)(89.52823181,139.27071289)
\curveto(89.5182295,139.24070601)(89.50822951,139.20070605)(89.49823181,139.15071289)
\curveto(89.49822952,139.11070614)(89.50322951,139.07070618)(89.51323181,139.03071289)
\curveto(89.53322948,138.98070627)(89.55822946,138.93570631)(89.58823181,138.89571289)
\curveto(89.62822939,138.86570638)(89.68822933,138.86070639)(89.76823181,138.88071289)
\curveto(89.82822919,138.90070635)(89.88822913,138.92570632)(89.94823181,138.95571289)
\curveto(90.00822901,138.99570625)(90.06822895,139.03070622)(90.12823181,139.06071289)
\curveto(90.18822883,139.08070617)(90.23822878,139.09570615)(90.27823181,139.10571289)
\curveto(90.46822855,139.18570606)(90.67322834,139.24070601)(90.89323181,139.27071289)
\curveto(91.12322789,139.30070595)(91.35322766,139.31070594)(91.58323181,139.30071289)
\curveto(91.82322719,139.30070595)(92.05322696,139.27570597)(92.27323181,139.22571289)
\curveto(92.49322652,139.18570606)(92.69322632,139.12570612)(92.87323181,139.04571289)
\curveto(92.92322609,139.02570622)(92.96822605,139.00570624)(93.00823181,138.98571289)
\curveto(93.05822596,138.96570628)(93.10822591,138.94070631)(93.15823181,138.91071289)
\curveto(93.50822551,138.70070655)(93.78822523,138.47070678)(93.99823181,138.22071289)
\curveto(94.2182248,137.97070728)(94.4132246,137.6457076)(94.58323181,137.24571289)
\curveto(94.63322438,137.13570811)(94.66822435,137.02570822)(94.68823181,136.91571289)
\curveto(94.70822431,136.80570844)(94.73322428,136.69070856)(94.76323181,136.57071289)
\curveto(94.77322424,136.54070871)(94.77822424,136.49570875)(94.77823181,136.43571289)
\curveto(94.79822422,136.37570887)(94.80822421,136.30570894)(94.80823181,136.22571289)
\curveto(94.80822421,136.15570909)(94.8182242,136.09070916)(94.83823181,136.03071289)
\lineto(94.83823181,135.86571289)
\curveto(94.84822417,135.81570943)(94.85322416,135.7457095)(94.85323181,135.65571289)
\curveto(94.85322416,135.56570968)(94.84322417,135.49570975)(94.82323181,135.44571289)
\curveto(94.80322421,135.38570986)(94.79822422,135.32570992)(94.80823181,135.26571289)
\curveto(94.8182242,135.21571003)(94.8132242,135.16571008)(94.79323181,135.11571289)
\curveto(94.75322426,134.95571029)(94.7182243,134.80571044)(94.68823181,134.66571289)
\curveto(94.65822436,134.52571072)(94.6132244,134.39071086)(94.55323181,134.26071289)
\curveto(94.39322462,133.89071136)(94.17322484,133.55571169)(93.89323181,133.25571289)
\curveto(93.6132254,132.95571229)(93.29322572,132.72571252)(92.93323181,132.56571289)
\curveto(92.76322625,132.48571276)(92.56322645,132.41071284)(92.33323181,132.34071289)
\curveto(92.22322679,132.30071295)(92.10822691,132.27571297)(91.98823181,132.26571289)
\curveto(91.86822715,132.25571299)(91.74822727,132.23571301)(91.62823181,132.20571289)
\curveto(91.57822744,132.18571306)(91.52322749,132.18571306)(91.46323181,132.20571289)
\curveto(91.40322761,132.21571303)(91.34322767,132.21071304)(91.28323181,132.19071289)
\curveto(91.18322783,132.17071308)(91.08322793,132.17071308)(90.98323181,132.19071289)
\lineto(90.84823181,132.19071289)
\curveto(90.79822822,132.21071304)(90.73822828,132.22071303)(90.66823181,132.22071289)
\curveto(90.60822841,132.21071304)(90.55322846,132.21571303)(90.50323181,132.23571289)
\curveto(90.46322855,132.245713)(90.42822859,132.250713)(90.39823181,132.25071289)
\curveto(90.36822865,132.250713)(90.33322868,132.25571299)(90.29323181,132.26571289)
\lineto(90.02323181,132.32571289)
\curveto(89.93322908,132.3457129)(89.84822917,132.37571287)(89.76823181,132.41571289)
\curveto(89.42822959,132.55571269)(89.13822988,132.71071254)(88.89823181,132.88071289)
\curveto(88.65823036,133.06071219)(88.43823058,133.29071196)(88.23823181,133.57071289)
\curveto(88.08823093,133.80071145)(87.97323104,134.04071121)(87.89323181,134.29071289)
\curveto(87.87323114,134.34071091)(87.86323115,134.38571086)(87.86323181,134.42571289)
\curveto(87.86323115,134.47571077)(87.85323116,134.52571072)(87.83323181,134.57571289)
\curveto(87.8132312,134.63571061)(87.79823122,134.71571053)(87.78823181,134.81571289)
\curveto(87.78823123,134.91571033)(87.80823121,134.99071026)(87.84823181,135.04071289)
\curveto(87.89823112,135.12071013)(87.97823104,135.16571008)(88.08823181,135.17571289)
\curveto(88.19823082,135.18571006)(88.3132307,135.19071006)(88.43323181,135.19071289)
\lineto(88.59823181,135.19071289)
\curveto(88.65823036,135.19071006)(88.7132303,135.18071007)(88.76323181,135.16071289)
\curveto(88.85323016,135.14071011)(88.92323009,135.10071015)(88.97323181,135.04071289)
\curveto(89.04322997,134.9507103)(89.08822993,134.84071041)(89.10823181,134.71071289)
\curveto(89.13822988,134.59071066)(89.18322983,134.48571076)(89.24323181,134.39571289)
\curveto(89.43322958,134.05571119)(89.69322932,133.78571146)(90.02323181,133.58571289)
\curveto(90.12322889,133.52571172)(90.22822879,133.47571177)(90.33823181,133.43571289)
\curveto(90.45822856,133.40571184)(90.57822844,133.37071188)(90.69823181,133.33071289)
\curveto(90.86822815,133.28071197)(91.07322794,133.26071199)(91.31323181,133.27071289)
\curveto(91.56322745,133.29071196)(91.76322725,133.32571192)(91.91323181,133.37571289)
\curveto(92.28322673,133.49571175)(92.57322644,133.65571159)(92.78323181,133.85571289)
\curveto(93.00322601,134.06571118)(93.18322583,134.3457109)(93.32323181,134.69571289)
\curveto(93.37322564,134.79571045)(93.40322561,134.90071035)(93.41323181,135.01071289)
\curveto(93.43322558,135.12071013)(93.45822556,135.23571001)(93.48823181,135.35571289)
\lineto(93.48823181,135.46071289)
\curveto(93.49822552,135.50070975)(93.50322551,135.54070971)(93.50323181,135.58071289)
\curveto(93.5132255,135.61070964)(93.5132255,135.6457096)(93.50323181,135.68571289)
\lineto(93.50323181,135.80571289)
\curveto(93.50322551,136.06570918)(93.47322554,136.31070894)(93.41323181,136.54071289)
\curveto(93.30322571,136.89070836)(93.14822587,137.18570806)(92.94823181,137.42571289)
\curveto(92.74822627,137.67570757)(92.48822653,137.87070738)(92.16823181,138.01071289)
\lineto(91.98823181,138.07071289)
\curveto(91.93822708,138.09070716)(91.87822714,138.11070714)(91.80823181,138.13071289)
\curveto(91.75822726,138.1507071)(91.69822732,138.16070709)(91.62823181,138.16071289)
\curveto(91.56822745,138.17070708)(91.50322751,138.18570706)(91.43323181,138.20571289)
\lineto(91.28323181,138.20571289)
\curveto(91.24322777,138.22570702)(91.18822783,138.23570701)(91.11823181,138.23571289)
\curveto(91.05822796,138.23570701)(91.00322801,138.22570702)(90.95323181,138.20571289)
\lineto(90.84823181,138.20571289)
\curveto(90.8182282,138.20570704)(90.78322823,138.20070705)(90.74323181,138.19071289)
\lineto(90.50323181,138.13071289)
\curveto(90.42322859,138.12070713)(90.34322867,138.10070715)(90.26323181,138.07071289)
\curveto(90.02322899,137.97070728)(89.79322922,137.83570741)(89.57323181,137.66571289)
\curveto(89.48322953,137.59570765)(89.39822962,137.52070773)(89.31823181,137.44071289)
\curveto(89.23822978,137.37070788)(89.13822988,137.31570793)(89.01823181,137.27571289)
\curveto(88.92823009,137.245708)(88.78823023,137.23570801)(88.59823181,137.24571289)
\curveto(88.4182306,137.25570799)(88.29823072,137.28070797)(88.23823181,137.32071289)
\curveto(88.18823083,137.36070789)(88.14823087,137.42070783)(88.11823181,137.50071289)
\curveto(88.09823092,137.58070767)(88.09823092,137.66570758)(88.11823181,137.75571289)
\curveto(88.14823087,137.87570737)(88.16823085,137.99570725)(88.17823181,138.11571289)
\curveto(88.19823082,138.245707)(88.22323079,138.37070688)(88.25323181,138.49071289)
\curveto(88.27323074,138.53070672)(88.27823074,138.56570668)(88.26823181,138.59571289)
\curveto(88.26823075,138.63570661)(88.27823074,138.68070657)(88.29823181,138.73071289)
\curveto(88.3182307,138.82070643)(88.33323068,138.91070634)(88.34323181,139.00071289)
\curveto(88.35323066,139.10070615)(88.37323064,139.19570605)(88.40323181,139.28571289)
\curveto(88.4132306,139.3457059)(88.4182306,139.40570584)(88.41823181,139.46571289)
\curveto(88.42823059,139.52570572)(88.44323057,139.58570566)(88.46323181,139.64571289)
\curveto(88.5132305,139.8457054)(88.54823047,140.0507052)(88.56823181,140.26071289)
\curveto(88.59823042,140.48070477)(88.63823038,140.69070456)(88.68823181,140.89071289)
\curveto(88.7182303,140.99070426)(88.73823028,141.09070416)(88.74823181,141.19071289)
\curveto(88.75823026,141.29070396)(88.77323024,141.39070386)(88.79323181,141.49071289)
\curveto(88.80323021,141.52070373)(88.80823021,141.56070369)(88.80823181,141.61071289)
\curveto(88.83823018,141.72070353)(88.85823016,141.82570342)(88.86823181,141.92571289)
\curveto(88.88823013,142.03570321)(88.9132301,142.1457031)(88.94323181,142.25571289)
\curveto(88.96323005,142.33570291)(88.97823004,142.40570284)(88.98823181,142.46571289)
\curveto(88.99823002,142.53570271)(89.02322999,142.59570265)(89.06323181,142.64571289)
\curveto(89.08322993,142.67570257)(89.1132299,142.69570255)(89.15323181,142.70571289)
\curveto(89.19322982,142.72570252)(89.23822978,142.7457025)(89.28823181,142.76571289)
\curveto(89.34822967,142.76570248)(89.38822963,142.77070248)(89.40823181,142.78071289)
}
}
{
\newrgbcolor{curcolor}{0 0 0}
\pscustom[linestyle=none,fillstyle=solid,fillcolor=curcolor]
{
\newpath
\moveto(97.20284119,134.00571289)
\lineto(97.50284119,134.00571289)
\curveto(97.61283913,134.01571123)(97.71783902,134.01571123)(97.81784119,134.00571289)
\curveto(97.92783881,134.00571124)(98.02783871,133.99571125)(98.11784119,133.97571289)
\curveto(98.20783853,133.96571128)(98.27783846,133.94071131)(98.32784119,133.90071289)
\curveto(98.34783839,133.88071137)(98.36283838,133.8507114)(98.37284119,133.81071289)
\curveto(98.39283835,133.77071148)(98.41283833,133.72571152)(98.43284119,133.67571289)
\lineto(98.43284119,133.60071289)
\curveto(98.4428383,133.5507117)(98.4428383,133.49571175)(98.43284119,133.43571289)
\lineto(98.43284119,133.28571289)
\lineto(98.43284119,132.80571289)
\curveto(98.43283831,132.63571261)(98.39283835,132.51571273)(98.31284119,132.44571289)
\curveto(98.2428385,132.39571285)(98.15283859,132.37071288)(98.04284119,132.37071289)
\lineto(97.71284119,132.37071289)
\lineto(97.26284119,132.37071289)
\curveto(97.11283963,132.37071288)(96.99783974,132.40071285)(96.91784119,132.46071289)
\curveto(96.87783986,132.49071276)(96.84783989,132.54071271)(96.82784119,132.61071289)
\curveto(96.80783993,132.69071256)(96.79283995,132.77571247)(96.78284119,132.86571289)
\lineto(96.78284119,133.15071289)
\curveto(96.79283995,133.250712)(96.79783994,133.33571191)(96.79784119,133.40571289)
\lineto(96.79784119,133.60071289)
\curveto(96.79783994,133.66071159)(96.80783993,133.71571153)(96.82784119,133.76571289)
\curveto(96.86783987,133.87571137)(96.9378398,133.9457113)(97.03784119,133.97571289)
\curveto(97.06783967,133.97571127)(97.12283962,133.98571126)(97.20284119,134.00571289)
}
}
{
\newrgbcolor{curcolor}{0 0 0}
\pscustom[linestyle=none,fillstyle=solid,fillcolor=curcolor]
{
\newpath
\moveto(103.52299744,142.97571289)
\curveto(105.152992,143.00570224)(106.20299095,142.4507028)(106.67299744,141.31071289)
\curveto(106.77299038,141.08070417)(106.83799031,140.79070446)(106.86799744,140.44071289)
\curveto(106.90799024,140.10070515)(106.88299027,139.79070546)(106.79299744,139.51071289)
\curveto(106.70299045,139.250706)(106.58299057,139.02570622)(106.43299744,138.83571289)
\curveto(106.41299074,138.79570645)(106.38799076,138.76070649)(106.35799744,138.73071289)
\curveto(106.32799082,138.71070654)(106.30299085,138.68570656)(106.28299744,138.65571289)
\lineto(106.19299744,138.53571289)
\curveto(106.16299099,138.50570674)(106.12799102,138.48070677)(106.08799744,138.46071289)
\curveto(106.03799111,138.41070684)(105.98299117,138.36570688)(105.92299744,138.32571289)
\curveto(105.87299128,138.28570696)(105.82799132,138.23570701)(105.78799744,138.17571289)
\curveto(105.7479914,138.13570711)(105.73299142,138.08570716)(105.74299744,138.02571289)
\curveto(105.7529914,137.97570727)(105.78299137,137.93070732)(105.83299744,137.89071289)
\curveto(105.88299127,137.8507074)(105.93799121,137.81070744)(105.99799744,137.77071289)
\curveto(106.06799108,137.74070751)(106.13299102,137.71070754)(106.19299744,137.68071289)
\curveto(106.2529909,137.6507076)(106.30299085,137.62070763)(106.34299744,137.59071289)
\curveto(106.66299049,137.37070788)(106.91799023,137.06070819)(107.10799744,136.66071289)
\curveto(107.14799,136.57070868)(107.17798997,136.47570877)(107.19799744,136.37571289)
\curveto(107.22798992,136.28570896)(107.2529899,136.19570905)(107.27299744,136.10571289)
\curveto(107.28298987,136.05570919)(107.28798986,136.00570924)(107.28799744,135.95571289)
\curveto(107.29798985,135.91570933)(107.30798984,135.87070938)(107.31799744,135.82071289)
\curveto(107.32798982,135.77070948)(107.32798982,135.72070953)(107.31799744,135.67071289)
\curveto(107.30798984,135.62070963)(107.31298984,135.57070968)(107.33299744,135.52071289)
\curveto(107.34298981,135.47070978)(107.3479898,135.41070984)(107.34799744,135.34071289)
\curveto(107.3479898,135.27070998)(107.33798981,135.21071004)(107.31799744,135.16071289)
\lineto(107.31799744,134.93571289)
\lineto(107.25799744,134.69571289)
\curveto(107.2479899,134.62571062)(107.23298992,134.55571069)(107.21299744,134.48571289)
\curveto(107.18298997,134.39571085)(107.15299,134.31071094)(107.12299744,134.23071289)
\curveto(107.10299005,134.1507111)(107.07299008,134.07071118)(107.03299744,133.99071289)
\curveto(107.01299014,133.93071132)(106.98299017,133.87071138)(106.94299744,133.81071289)
\curveto(106.91299024,133.76071149)(106.87799027,133.71071154)(106.83799744,133.66071289)
\curveto(106.63799051,133.3507119)(106.38799076,133.09071216)(106.08799744,132.88071289)
\curveto(105.78799136,132.68071257)(105.44299171,132.51571273)(105.05299744,132.38571289)
\curveto(104.93299222,132.3457129)(104.80299235,132.32071293)(104.66299744,132.31071289)
\curveto(104.53299262,132.29071296)(104.39799275,132.26571298)(104.25799744,132.23571289)
\curveto(104.18799296,132.22571302)(104.11799303,132.22071303)(104.04799744,132.22071289)
\curveto(103.98799316,132.22071303)(103.92299323,132.21571303)(103.85299744,132.20571289)
\curveto(103.81299334,132.19571305)(103.7529934,132.19071306)(103.67299744,132.19071289)
\curveto(103.60299355,132.19071306)(103.5529936,132.19571305)(103.52299744,132.20571289)
\curveto(103.47299368,132.21571303)(103.42799372,132.22071303)(103.38799744,132.22071289)
\lineto(103.26799744,132.22071289)
\curveto(103.16799398,132.24071301)(103.06799408,132.25571299)(102.96799744,132.26571289)
\curveto(102.86799428,132.27571297)(102.77299438,132.29071296)(102.68299744,132.31071289)
\curveto(102.57299458,132.34071291)(102.46299469,132.36571288)(102.35299744,132.38571289)
\curveto(102.2529949,132.41571283)(102.147995,132.45571279)(102.03799744,132.50571289)
\curveto(101.66799548,132.66571258)(101.3529958,132.86571238)(101.09299744,133.10571289)
\curveto(100.83299632,133.35571189)(100.62299653,133.66571158)(100.46299744,134.03571289)
\curveto(100.42299673,134.12571112)(100.38799676,134.22071103)(100.35799744,134.32071289)
\curveto(100.32799682,134.42071083)(100.29799685,134.52571072)(100.26799744,134.63571289)
\curveto(100.2479969,134.68571056)(100.23799691,134.73571051)(100.23799744,134.78571289)
\curveto(100.23799691,134.8457104)(100.22799692,134.90571034)(100.20799744,134.96571289)
\curveto(100.18799696,135.02571022)(100.17799697,135.10571014)(100.17799744,135.20571289)
\curveto(100.17799697,135.30570994)(100.19299696,135.38070987)(100.22299744,135.43071289)
\curveto(100.23299692,135.46070979)(100.2479969,135.48570976)(100.26799744,135.50571289)
\lineto(100.32799744,135.56571289)
\curveto(100.36799678,135.58570966)(100.42799672,135.60070965)(100.50799744,135.61071289)
\curveto(100.59799655,135.62070963)(100.68799646,135.62570962)(100.77799744,135.62571289)
\curveto(100.86799628,135.62570962)(100.9529962,135.62070963)(101.03299744,135.61071289)
\curveto(101.12299603,135.60070965)(101.18799596,135.59070966)(101.22799744,135.58071289)
\curveto(101.2479959,135.56070969)(101.26799588,135.5457097)(101.28799744,135.53571289)
\curveto(101.30799584,135.53570971)(101.32799582,135.52570972)(101.34799744,135.50571289)
\curveto(101.41799573,135.41570983)(101.45799569,135.30070995)(101.46799744,135.16071289)
\curveto(101.48799566,135.02071023)(101.51799563,134.89571035)(101.55799744,134.78571289)
\lineto(101.70799744,134.42571289)
\curveto(101.75799539,134.31571093)(101.82299533,134.21071104)(101.90299744,134.11071289)
\curveto(101.92299523,134.08071117)(101.94299521,134.05571119)(101.96299744,134.03571289)
\curveto(101.99299516,134.01571123)(102.01799513,133.99071126)(102.03799744,133.96071289)
\curveto(102.07799507,133.90071135)(102.11299504,133.85571139)(102.14299744,133.82571289)
\curveto(102.18299497,133.79571145)(102.21799493,133.76571148)(102.24799744,133.73571289)
\curveto(102.28799486,133.70571154)(102.33299482,133.67571157)(102.38299744,133.64571289)
\curveto(102.47299468,133.58571166)(102.56799458,133.53571171)(102.66799744,133.49571289)
\lineto(102.99799744,133.37571289)
\curveto(103.147994,133.32571192)(103.3479938,133.29571195)(103.59799744,133.28571289)
\curveto(103.8479933,133.27571197)(104.05799309,133.29571195)(104.22799744,133.34571289)
\curveto(104.30799284,133.36571188)(104.37799277,133.38071187)(104.43799744,133.39071289)
\lineto(104.64799744,133.45071289)
\curveto(104.92799222,133.57071168)(105.16799198,133.72071153)(105.36799744,133.90071289)
\curveto(105.57799157,134.08071117)(105.74299141,134.31071094)(105.86299744,134.59071289)
\curveto(105.89299126,134.66071059)(105.91299124,134.73071052)(105.92299744,134.80071289)
\lineto(105.98299744,135.04071289)
\curveto(106.02299113,135.18071007)(106.03299112,135.34070991)(106.01299744,135.52071289)
\curveto(105.99299116,135.71070954)(105.96299119,135.86070939)(105.92299744,135.97071289)
\curveto(105.79299136,136.3507089)(105.60799154,136.64070861)(105.36799744,136.84071289)
\curveto(105.13799201,137.04070821)(104.82799232,137.20070805)(104.43799744,137.32071289)
\curveto(104.32799282,137.3507079)(104.20799294,137.37070788)(104.07799744,137.38071289)
\curveto(103.95799319,137.39070786)(103.83299332,137.39570785)(103.70299744,137.39571289)
\curveto(103.54299361,137.39570785)(103.40299375,137.40070785)(103.28299744,137.41071289)
\curveto(103.16299399,137.42070783)(103.07799407,137.48070777)(103.02799744,137.59071289)
\curveto(103.00799414,137.62070763)(102.99799415,137.65570759)(102.99799744,137.69571289)
\lineto(102.99799744,137.83071289)
\curveto(102.98799416,137.93070732)(102.98799416,138.02570722)(102.99799744,138.11571289)
\curveto(103.01799413,138.20570704)(103.05799409,138.27070698)(103.11799744,138.31071289)
\curveto(103.15799399,138.34070691)(103.19799395,138.36070689)(103.23799744,138.37071289)
\curveto(103.28799386,138.38070687)(103.34299381,138.39070686)(103.40299744,138.40071289)
\curveto(103.42299373,138.41070684)(103.4479937,138.41070684)(103.47799744,138.40071289)
\curveto(103.50799364,138.40070685)(103.53299362,138.40570684)(103.55299744,138.41571289)
\lineto(103.68799744,138.41571289)
\curveto(103.79799335,138.43570681)(103.89799325,138.4457068)(103.98799744,138.44571289)
\curveto(104.08799306,138.45570679)(104.18299297,138.47570677)(104.27299744,138.50571289)
\curveto(104.59299256,138.61570663)(104.8479923,138.76070649)(105.03799744,138.94071289)
\curveto(105.22799192,139.12070613)(105.37799177,139.37070588)(105.48799744,139.69071289)
\curveto(105.51799163,139.79070546)(105.53799161,139.91570533)(105.54799744,140.06571289)
\curveto(105.56799158,140.22570502)(105.56299159,140.37070488)(105.53299744,140.50071289)
\curveto(105.51299164,140.57070468)(105.49299166,140.63570461)(105.47299744,140.69571289)
\curveto(105.46299169,140.76570448)(105.44299171,140.83070442)(105.41299744,140.89071289)
\curveto(105.31299184,141.13070412)(105.16799198,141.32070393)(104.97799744,141.46071289)
\curveto(104.78799236,141.60070365)(104.56299259,141.71070354)(104.30299744,141.79071289)
\curveto(104.24299291,141.81070344)(104.18299297,141.82070343)(104.12299744,141.82071289)
\curveto(104.06299309,141.82070343)(103.99799315,141.83070342)(103.92799744,141.85071289)
\curveto(103.8479933,141.87070338)(103.7529934,141.88070337)(103.64299744,141.88071289)
\curveto(103.53299362,141.88070337)(103.43799371,141.87070338)(103.35799744,141.85071289)
\curveto(103.30799384,141.83070342)(103.25799389,141.82070343)(103.20799744,141.82071289)
\curveto(103.16799398,141.82070343)(103.12299403,141.81070344)(103.07299744,141.79071289)
\curveto(102.89299426,141.74070351)(102.72299443,141.66570358)(102.56299744,141.56571289)
\curveto(102.41299474,141.47570377)(102.28299487,141.36070389)(102.17299744,141.22071289)
\curveto(102.08299507,141.10070415)(102.00299515,140.97070428)(101.93299744,140.83071289)
\curveto(101.86299529,140.69070456)(101.79799535,140.53570471)(101.73799744,140.36571289)
\curveto(101.70799544,140.25570499)(101.68799546,140.13570511)(101.67799744,140.00571289)
\curveto(101.66799548,139.88570536)(101.63299552,139.78570546)(101.57299744,139.70571289)
\curveto(101.5529956,139.66570558)(101.49299566,139.62570562)(101.39299744,139.58571289)
\curveto(101.3529958,139.57570567)(101.29299586,139.56570568)(101.21299744,139.55571289)
\lineto(100.95799744,139.55571289)
\curveto(100.86799628,139.56570568)(100.78299637,139.57570567)(100.70299744,139.58571289)
\curveto(100.63299652,139.59570565)(100.58299657,139.61070564)(100.55299744,139.63071289)
\curveto(100.51299664,139.66070559)(100.47799667,139.71570553)(100.44799744,139.79571289)
\curveto(100.41799673,139.87570537)(100.41299674,139.96070529)(100.43299744,140.05071289)
\curveto(100.44299671,140.10070515)(100.4479967,140.1507051)(100.44799744,140.20071289)
\lineto(100.47799744,140.38071289)
\curveto(100.50799664,140.48070477)(100.53299662,140.58070467)(100.55299744,140.68071289)
\curveto(100.58299657,140.78070447)(100.61799653,140.87070438)(100.65799744,140.95071289)
\curveto(100.70799644,141.06070419)(100.7529964,141.16570408)(100.79299744,141.26571289)
\curveto(100.83299632,141.37570387)(100.88299627,141.48070377)(100.94299744,141.58071289)
\curveto(101.27299588,142.12070313)(101.74299541,142.51570273)(102.35299744,142.76571289)
\curveto(102.47299468,142.81570243)(102.59799455,142.8507024)(102.72799744,142.87071289)
\curveto(102.86799428,142.89070236)(103.00799414,142.91570233)(103.14799744,142.94571289)
\curveto(103.20799394,142.95570229)(103.26799388,142.96070229)(103.32799744,142.96071289)
\curveto(103.39799375,142.96070229)(103.46299369,142.96570228)(103.52299744,142.97571289)
}
}
{
\newrgbcolor{curcolor}{0 0 0}
\pscustom[linestyle=none,fillstyle=solid,fillcolor=curcolor]
{
\newpath
\moveto(118.56260681,140.89071289)
\curveto(118.36259651,140.60070465)(118.15259672,140.31570493)(117.93260681,140.03571289)
\curveto(117.72259715,139.75570549)(117.51759736,139.47070578)(117.31760681,139.18071289)
\curveto(116.71759816,138.33070692)(116.11259876,137.49070776)(115.50260681,136.66071289)
\curveto(114.89259998,135.84070941)(114.28760059,135.00571024)(113.68760681,134.15571289)
\lineto(113.17760681,133.43571289)
\lineto(112.66760681,132.74571289)
\curveto(112.58760229,132.63571261)(112.50760237,132.52071273)(112.42760681,132.40071289)
\curveto(112.34760253,132.28071297)(112.25260262,132.18571306)(112.14260681,132.11571289)
\curveto(112.10260277,132.09571315)(112.03760284,132.08071317)(111.94760681,132.07071289)
\curveto(111.86760301,132.0507132)(111.7776031,132.04071321)(111.67760681,132.04071289)
\curveto(111.5776033,132.04071321)(111.48260339,132.0457132)(111.39260681,132.05571289)
\curveto(111.31260356,132.06571318)(111.25260362,132.08571316)(111.21260681,132.11571289)
\curveto(111.18260369,132.13571311)(111.15760372,132.17071308)(111.13760681,132.22071289)
\curveto(111.12760375,132.26071299)(111.13260374,132.30571294)(111.15260681,132.35571289)
\curveto(111.19260368,132.43571281)(111.23760364,132.51071274)(111.28760681,132.58071289)
\curveto(111.34760353,132.66071259)(111.40260347,132.74071251)(111.45260681,132.82071289)
\curveto(111.69260318,133.16071209)(111.93760294,133.49571175)(112.18760681,133.82571289)
\curveto(112.43760244,134.15571109)(112.6776022,134.49071076)(112.90760681,134.83071289)
\curveto(113.06760181,135.0507102)(113.22760165,135.26570998)(113.38760681,135.47571289)
\curveto(113.54760133,135.68570956)(113.70760117,135.90070935)(113.86760681,136.12071289)
\curveto(114.22760065,136.64070861)(114.59260028,137.1507081)(114.96260681,137.65071289)
\curveto(115.33259954,138.1507071)(115.70259917,138.66070659)(116.07260681,139.18071289)
\curveto(116.21259866,139.38070587)(116.35259852,139.57570567)(116.49260681,139.76571289)
\curveto(116.64259823,139.95570529)(116.78759809,140.1507051)(116.92760681,140.35071289)
\curveto(117.13759774,140.6507046)(117.35259752,140.9507043)(117.57260681,141.25071289)
\lineto(118.23260681,142.15071289)
\lineto(118.41260681,142.42071289)
\lineto(118.62260681,142.69071289)
\lineto(118.74260681,142.87071289)
\curveto(118.79259608,142.93070232)(118.84259603,142.98570226)(118.89260681,143.03571289)
\curveto(118.96259591,143.08570216)(119.03759584,143.12070213)(119.11760681,143.14071289)
\curveto(119.13759574,143.1507021)(119.16259571,143.1507021)(119.19260681,143.14071289)
\curveto(119.23259564,143.14070211)(119.26259561,143.1507021)(119.28260681,143.17071289)
\curveto(119.40259547,143.17070208)(119.53759534,143.16570208)(119.68760681,143.15571289)
\curveto(119.83759504,143.15570209)(119.92759495,143.11070214)(119.95760681,143.02071289)
\curveto(119.9775949,142.99070226)(119.98259489,142.95570229)(119.97260681,142.91571289)
\curveto(119.96259491,142.87570237)(119.94759493,142.8457024)(119.92760681,142.82571289)
\curveto(119.88759499,142.7457025)(119.84759503,142.67570257)(119.80760681,142.61571289)
\curveto(119.76759511,142.55570269)(119.72259515,142.49570275)(119.67260681,142.43571289)
\lineto(119.10260681,141.65571289)
\curveto(118.92259595,141.40570384)(118.74259613,141.1507041)(118.56260681,140.89071289)
\moveto(111.70760681,136.99071289)
\curveto(111.65760322,137.01070824)(111.60760327,137.01570823)(111.55760681,137.00571289)
\curveto(111.50760337,136.99570825)(111.45760342,137.00070825)(111.40760681,137.02071289)
\curveto(111.29760358,137.04070821)(111.19260368,137.06070819)(111.09260681,137.08071289)
\curveto(111.00260387,137.11070814)(110.90760397,137.1507081)(110.80760681,137.20071289)
\curveto(110.4776044,137.34070791)(110.22260465,137.53570771)(110.04260681,137.78571289)
\curveto(109.86260501,138.0457072)(109.71760516,138.35570689)(109.60760681,138.71571289)
\curveto(109.5776053,138.79570645)(109.55760532,138.87570637)(109.54760681,138.95571289)
\curveto(109.53760534,139.0457062)(109.52260535,139.13070612)(109.50260681,139.21071289)
\curveto(109.49260538,139.26070599)(109.48760539,139.32570592)(109.48760681,139.40571289)
\curveto(109.4776054,139.43570581)(109.4726054,139.46570578)(109.47260681,139.49571289)
\curveto(109.4726054,139.53570571)(109.46760541,139.57070568)(109.45760681,139.60071289)
\lineto(109.45760681,139.75071289)
\curveto(109.44760543,139.80070545)(109.44260543,139.86070539)(109.44260681,139.93071289)
\curveto(109.44260543,140.01070524)(109.44760543,140.07570517)(109.45760681,140.12571289)
\lineto(109.45760681,140.29071289)
\curveto(109.4776054,140.34070491)(109.48260539,140.38570486)(109.47260681,140.42571289)
\curveto(109.4726054,140.47570477)(109.4776054,140.52070473)(109.48760681,140.56071289)
\curveto(109.49760538,140.60070465)(109.50260537,140.63570461)(109.50260681,140.66571289)
\curveto(109.50260537,140.70570454)(109.50760537,140.7457045)(109.51760681,140.78571289)
\curveto(109.54760533,140.89570435)(109.56760531,141.00570424)(109.57760681,141.11571289)
\curveto(109.59760528,141.23570401)(109.63260524,141.3507039)(109.68260681,141.46071289)
\curveto(109.82260505,141.80070345)(109.98260489,142.07570317)(110.16260681,142.28571289)
\curveto(110.35260452,142.50570274)(110.62260425,142.68570256)(110.97260681,142.82571289)
\curveto(111.05260382,142.85570239)(111.13760374,142.87570237)(111.22760681,142.88571289)
\curveto(111.31760356,142.90570234)(111.41260346,142.92570232)(111.51260681,142.94571289)
\curveto(111.54260333,142.95570229)(111.59760328,142.95570229)(111.67760681,142.94571289)
\curveto(111.75760312,142.9457023)(111.80760307,142.95570229)(111.82760681,142.97571289)
\curveto(112.38760249,142.98570226)(112.83760204,142.87570237)(113.17760681,142.64571289)
\curveto(113.52760135,142.41570283)(113.78760109,142.11070314)(113.95760681,141.73071289)
\curveto(113.99760088,141.64070361)(114.03260084,141.5457037)(114.06260681,141.44571289)
\curveto(114.09260078,141.3457039)(114.11760076,141.245704)(114.13760681,141.14571289)
\curveto(114.15760072,141.11570413)(114.16260071,141.08570416)(114.15260681,141.05571289)
\curveto(114.15260072,141.02570422)(114.15760072,140.99570425)(114.16760681,140.96571289)
\curveto(114.19760068,140.85570439)(114.21760066,140.73070452)(114.22760681,140.59071289)
\curveto(114.23760064,140.46070479)(114.24760063,140.32570492)(114.25760681,140.18571289)
\lineto(114.25760681,140.02071289)
\curveto(114.26760061,139.96070529)(114.26760061,139.90570534)(114.25760681,139.85571289)
\curveto(114.24760063,139.80570544)(114.24260063,139.75570549)(114.24260681,139.70571289)
\lineto(114.24260681,139.57071289)
\curveto(114.23260064,139.53070572)(114.22760065,139.49070576)(114.22760681,139.45071289)
\curveto(114.23760064,139.41070584)(114.23260064,139.36570588)(114.21260681,139.31571289)
\curveto(114.19260068,139.20570604)(114.1726007,139.10070615)(114.15260681,139.00071289)
\curveto(114.14260073,138.90070635)(114.12260075,138.80070645)(114.09260681,138.70071289)
\curveto(113.96260091,138.34070691)(113.79760108,138.02570722)(113.59760681,137.75571289)
\curveto(113.39760148,137.48570776)(113.12260175,137.28070797)(112.77260681,137.14071289)
\curveto(112.69260218,137.11070814)(112.60760227,137.08570816)(112.51760681,137.06571289)
\lineto(112.24760681,137.00571289)
\curveto(112.19760268,136.99570825)(112.15260272,136.99070826)(112.11260681,136.99071289)
\curveto(112.0726028,137.00070825)(112.03260284,137.00070825)(111.99260681,136.99071289)
\curveto(111.89260298,136.97070828)(111.79760308,136.97070828)(111.70760681,136.99071289)
\moveto(110.86760681,138.38571289)
\curveto(110.90760397,138.31570693)(110.94760393,138.250707)(110.98760681,138.19071289)
\curveto(111.02760385,138.14070711)(111.0776038,138.09070716)(111.13760681,138.04071289)
\lineto(111.28760681,137.92071289)
\curveto(111.34760353,137.89070736)(111.41260346,137.86570738)(111.48260681,137.84571289)
\curveto(111.52260335,137.82570742)(111.55760332,137.81570743)(111.58760681,137.81571289)
\curveto(111.62760325,137.82570742)(111.66760321,137.82070743)(111.70760681,137.80071289)
\curveto(111.73760314,137.80070745)(111.7776031,137.79570745)(111.82760681,137.78571289)
\curveto(111.877603,137.78570746)(111.91760296,137.79070746)(111.94760681,137.80071289)
\lineto(112.17260681,137.84571289)
\curveto(112.42260245,137.92570732)(112.60760227,138.0507072)(112.72760681,138.22071289)
\curveto(112.80760207,138.32070693)(112.877602,138.4507068)(112.93760681,138.61071289)
\curveto(113.01760186,138.79070646)(113.0776018,139.01570623)(113.11760681,139.28571289)
\curveto(113.15760172,139.56570568)(113.1726017,139.8457054)(113.16260681,140.12571289)
\curveto(113.15260172,140.41570483)(113.12260175,140.69070456)(113.07260681,140.95071289)
\curveto(113.02260185,141.21070404)(112.94760193,141.42070383)(112.84760681,141.58071289)
\curveto(112.72760215,141.78070347)(112.5776023,141.93070332)(112.39760681,142.03071289)
\curveto(112.31760256,142.08070317)(112.22760265,142.11070314)(112.12760681,142.12071289)
\curveto(112.02760285,142.14070311)(111.92260295,142.1507031)(111.81260681,142.15071289)
\curveto(111.79260308,142.14070311)(111.76760311,142.13570311)(111.73760681,142.13571289)
\curveto(111.71760316,142.1457031)(111.69760318,142.1457031)(111.67760681,142.13571289)
\curveto(111.62760325,142.12570312)(111.58260329,142.11570313)(111.54260681,142.10571289)
\curveto(111.50260337,142.10570314)(111.46260341,142.09570315)(111.42260681,142.07571289)
\curveto(111.24260363,141.99570325)(111.09260378,141.87570337)(110.97260681,141.71571289)
\curveto(110.86260401,141.55570369)(110.7726041,141.37570387)(110.70260681,141.17571289)
\curveto(110.64260423,140.98570426)(110.59760428,140.76070449)(110.56760681,140.50071289)
\curveto(110.54760433,140.24070501)(110.54260433,139.97570527)(110.55260681,139.70571289)
\curveto(110.56260431,139.4457058)(110.59260428,139.19570605)(110.64260681,138.95571289)
\curveto(110.70260417,138.72570652)(110.7776041,138.53570671)(110.86760681,138.38571289)
\moveto(121.66760681,135.40071289)
\curveto(121.6775932,135.3507099)(121.68259319,135.26070999)(121.68260681,135.13071289)
\curveto(121.68259319,135.00071025)(121.6725932,134.91071034)(121.65260681,134.86071289)
\curveto(121.63259324,134.81071044)(121.62759325,134.75571049)(121.63760681,134.69571289)
\curveto(121.64759323,134.6457106)(121.64759323,134.59571065)(121.63760681,134.54571289)
\curveto(121.59759328,134.40571084)(121.56759331,134.27071098)(121.54760681,134.14071289)
\curveto(121.53759334,134.01071124)(121.50759337,133.89071136)(121.45760681,133.78071289)
\curveto(121.31759356,133.43071182)(121.15259372,133.13571211)(120.96260681,132.89571289)
\curveto(120.7725941,132.66571258)(120.50259437,132.48071277)(120.15260681,132.34071289)
\curveto(120.0725948,132.31071294)(119.98759489,132.29071296)(119.89760681,132.28071289)
\curveto(119.80759507,132.26071299)(119.72259515,132.24071301)(119.64260681,132.22071289)
\curveto(119.59259528,132.21071304)(119.54259533,132.20571304)(119.49260681,132.20571289)
\curveto(119.44259543,132.20571304)(119.39259548,132.20071305)(119.34260681,132.19071289)
\curveto(119.31259556,132.18071307)(119.26259561,132.18071307)(119.19260681,132.19071289)
\curveto(119.12259575,132.19071306)(119.0725958,132.19571305)(119.04260681,132.20571289)
\curveto(118.98259589,132.22571302)(118.92259595,132.23571301)(118.86260681,132.23571289)
\curveto(118.81259606,132.22571302)(118.76259611,132.23071302)(118.71260681,132.25071289)
\curveto(118.62259625,132.27071298)(118.53259634,132.29571295)(118.44260681,132.32571289)
\curveto(118.36259651,132.3457129)(118.28259659,132.37571287)(118.20260681,132.41571289)
\curveto(117.88259699,132.55571269)(117.63259724,132.7507125)(117.45260681,133.00071289)
\curveto(117.2725976,133.26071199)(117.12259775,133.56571168)(117.00260681,133.91571289)
\curveto(116.98259789,133.99571125)(116.96759791,134.08071117)(116.95760681,134.17071289)
\curveto(116.94759793,134.26071099)(116.93259794,134.3457109)(116.91260681,134.42571289)
\curveto(116.90259797,134.45571079)(116.89759798,134.48571076)(116.89760681,134.51571289)
\lineto(116.89760681,134.62071289)
\curveto(116.877598,134.70071055)(116.86759801,134.78071047)(116.86760681,134.86071289)
\lineto(116.86760681,134.99571289)
\curveto(116.84759803,135.09571015)(116.84759803,135.19571005)(116.86760681,135.29571289)
\lineto(116.86760681,135.47571289)
\curveto(116.877598,135.52570972)(116.88259799,135.57070968)(116.88260681,135.61071289)
\curveto(116.88259799,135.66070959)(116.88759799,135.70570954)(116.89760681,135.74571289)
\curveto(116.90759797,135.78570946)(116.91259796,135.82070943)(116.91260681,135.85071289)
\curveto(116.91259796,135.89070936)(116.91759796,135.93070932)(116.92760681,135.97071289)
\lineto(116.98760681,136.30071289)
\curveto(117.00759787,136.42070883)(117.03759784,136.53070872)(117.07760681,136.63071289)
\curveto(117.21759766,136.96070829)(117.3775975,137.23570801)(117.55760681,137.45571289)
\curveto(117.74759713,137.68570756)(118.00759687,137.87070738)(118.33760681,138.01071289)
\curveto(118.41759646,138.0507072)(118.50259637,138.07570717)(118.59260681,138.08571289)
\lineto(118.89260681,138.14571289)
\lineto(119.02760681,138.14571289)
\curveto(119.0775958,138.15570709)(119.12759575,138.16070709)(119.17760681,138.16071289)
\curveto(119.74759513,138.18070707)(120.20759467,138.07570717)(120.55760681,137.84571289)
\curveto(120.91759396,137.62570762)(121.18259369,137.32570792)(121.35260681,136.94571289)
\curveto(121.40259347,136.8457084)(121.44259343,136.7457085)(121.47260681,136.64571289)
\curveto(121.50259337,136.5457087)(121.53259334,136.44070881)(121.56260681,136.33071289)
\curveto(121.5725933,136.29070896)(121.5775933,136.25570899)(121.57760681,136.22571289)
\curveto(121.5775933,136.20570904)(121.58259329,136.17570907)(121.59260681,136.13571289)
\curveto(121.61259326,136.06570918)(121.62259325,135.99070926)(121.62260681,135.91071289)
\curveto(121.62259325,135.83070942)(121.63259324,135.7507095)(121.65260681,135.67071289)
\curveto(121.65259322,135.62070963)(121.65259322,135.57570967)(121.65260681,135.53571289)
\curveto(121.65259322,135.49570975)(121.65759322,135.4507098)(121.66760681,135.40071289)
\moveto(120.55760681,134.96571289)
\curveto(120.56759431,135.01571023)(120.5725943,135.09071016)(120.57260681,135.19071289)
\curveto(120.58259429,135.29070996)(120.5775943,135.36570988)(120.55760681,135.41571289)
\curveto(120.53759434,135.47570977)(120.53259434,135.53070972)(120.54260681,135.58071289)
\curveto(120.56259431,135.64070961)(120.56259431,135.70070955)(120.54260681,135.76071289)
\curveto(120.53259434,135.79070946)(120.52759435,135.82570942)(120.52760681,135.86571289)
\curveto(120.52759435,135.90570934)(120.52259435,135.9457093)(120.51260681,135.98571289)
\curveto(120.49259438,136.06570918)(120.4725944,136.14070911)(120.45260681,136.21071289)
\curveto(120.44259443,136.29070896)(120.42759445,136.37070888)(120.40760681,136.45071289)
\curveto(120.3775945,136.51070874)(120.35259452,136.57070868)(120.33260681,136.63071289)
\curveto(120.31259456,136.69070856)(120.28259459,136.7507085)(120.24260681,136.81071289)
\curveto(120.14259473,136.98070827)(120.01259486,137.11570813)(119.85260681,137.21571289)
\curveto(119.7725951,137.26570798)(119.6775952,137.30070795)(119.56760681,137.32071289)
\curveto(119.45759542,137.34070791)(119.33259554,137.3507079)(119.19260681,137.35071289)
\curveto(119.1725957,137.34070791)(119.14759573,137.33570791)(119.11760681,137.33571289)
\curveto(119.08759579,137.3457079)(119.05759582,137.3457079)(119.02760681,137.33571289)
\lineto(118.87760681,137.27571289)
\curveto(118.82759605,137.26570798)(118.78259609,137.250708)(118.74260681,137.23071289)
\curveto(118.55259632,137.12070813)(118.40759647,136.97570827)(118.30760681,136.79571289)
\curveto(118.21759666,136.61570863)(118.13759674,136.41070884)(118.06760681,136.18071289)
\curveto(118.02759685,136.0507092)(118.00759687,135.91570933)(118.00760681,135.77571289)
\curveto(118.00759687,135.6457096)(117.99759688,135.50070975)(117.97760681,135.34071289)
\curveto(117.96759691,135.29070996)(117.95759692,135.23071002)(117.94760681,135.16071289)
\curveto(117.94759693,135.09071016)(117.95759692,135.03071022)(117.97760681,134.98071289)
\lineto(117.97760681,134.81571289)
\lineto(117.97760681,134.63571289)
\curveto(117.98759689,134.58571066)(117.99759688,134.53071072)(118.00760681,134.47071289)
\curveto(118.01759686,134.42071083)(118.02259685,134.36571088)(118.02260681,134.30571289)
\curveto(118.03259684,134.245711)(118.04759683,134.19071106)(118.06760681,134.14071289)
\curveto(118.11759676,133.9507113)(118.1775967,133.77571147)(118.24760681,133.61571289)
\curveto(118.31759656,133.45571179)(118.42259645,133.32571192)(118.56260681,133.22571289)
\curveto(118.69259618,133.12571212)(118.83259604,133.05571219)(118.98260681,133.01571289)
\curveto(119.01259586,133.00571224)(119.03759584,133.00071225)(119.05760681,133.00071289)
\curveto(119.08759579,133.01071224)(119.11759576,133.01071224)(119.14760681,133.00071289)
\curveto(119.16759571,133.00071225)(119.19759568,132.99571225)(119.23760681,132.98571289)
\curveto(119.2775956,132.98571226)(119.31259556,132.99071226)(119.34260681,133.00071289)
\curveto(119.38259549,133.01071224)(119.42259545,133.01571223)(119.46260681,133.01571289)
\curveto(119.50259537,133.01571223)(119.54259533,133.02571222)(119.58260681,133.04571289)
\curveto(119.82259505,133.12571212)(120.01759486,133.26071199)(120.16760681,133.45071289)
\curveto(120.28759459,133.63071162)(120.3775945,133.83571141)(120.43760681,134.06571289)
\curveto(120.45759442,134.13571111)(120.4725944,134.20571104)(120.48260681,134.27571289)
\curveto(120.49259438,134.35571089)(120.50759437,134.43571081)(120.52760681,134.51571289)
\curveto(120.52759435,134.57571067)(120.53259434,134.62071063)(120.54260681,134.65071289)
\curveto(120.54259433,134.67071058)(120.54259433,134.69571055)(120.54260681,134.72571289)
\curveto(120.54259433,134.76571048)(120.54759433,134.79571045)(120.55760681,134.81571289)
\lineto(120.55760681,134.96571289)
}
}
{
\newrgbcolor{curcolor}{0 0 0}
\pscustom[linestyle=none,fillstyle=solid,fillcolor=curcolor]
{
\newpath
\moveto(664.32293213,253.69000732)
\curveto(664.42292727,253.6899967)(664.51792718,253.67999671)(664.60793213,253.66000732)
\curveto(664.697927,253.64999674)(664.76292693,253.61999677)(664.80293213,253.57000732)
\curveto(664.86292683,253.4899969)(664.8929268,253.38499701)(664.89293213,253.25500732)
\lineto(664.89293213,252.86500732)
\lineto(664.89293213,251.36500732)
\lineto(664.89293213,244.97500732)
\lineto(664.89293213,243.80500732)
\lineto(664.89293213,243.49000732)
\curveto(664.90292679,243.390007)(664.88792681,243.31000708)(664.84793213,243.25000732)
\curveto(664.7979269,243.17000722)(664.72292697,243.12000727)(664.62293213,243.10000732)
\curveto(664.53292716,243.0900073)(664.42292727,243.08500731)(664.29293213,243.08500732)
\lineto(664.06793213,243.08500732)
\curveto(663.98792771,243.10500729)(663.91792778,243.12000727)(663.85793213,243.13000732)
\curveto(663.7979279,243.15000724)(663.74792795,243.1900072)(663.70793213,243.25000732)
\curveto(663.66792803,243.31000708)(663.64792805,243.38500701)(663.64793213,243.47500732)
\lineto(663.64793213,243.77500732)
\lineto(663.64793213,244.87000732)
\lineto(663.64793213,250.21000732)
\curveto(663.62792807,250.30000009)(663.61292808,250.37500002)(663.60293213,250.43500732)
\curveto(663.60292809,250.50499989)(663.57292812,250.56499983)(663.51293213,250.61500732)
\curveto(663.44292825,250.66499973)(663.35292834,250.6899997)(663.24293213,250.69000732)
\curveto(663.14292855,250.69999969)(663.03292866,250.70499969)(662.91293213,250.70500732)
\lineto(661.77293213,250.70500732)
\lineto(661.27793213,250.70500732)
\curveto(661.11793058,250.71499968)(661.00793069,250.77499962)(660.94793213,250.88500732)
\curveto(660.92793077,250.91499948)(660.91793078,250.94499945)(660.91793213,250.97500732)
\curveto(660.91793078,251.01499938)(660.91293078,251.05999933)(660.90293213,251.11000732)
\curveto(660.88293081,251.22999916)(660.88793081,251.33999905)(660.91793213,251.44000732)
\curveto(660.95793074,251.53999885)(661.01293068,251.60999878)(661.08293213,251.65000732)
\curveto(661.16293053,251.69999869)(661.28293041,251.72499867)(661.44293213,251.72500732)
\curveto(661.60293009,251.72499867)(661.73792996,251.73999865)(661.84793213,251.77000732)
\curveto(661.8979298,251.77999861)(661.95292974,251.78499861)(662.01293213,251.78500732)
\curveto(662.07292962,251.7949986)(662.13292956,251.80999858)(662.19293213,251.83000732)
\curveto(662.34292935,251.87999851)(662.48792921,251.92999846)(662.62793213,251.98000732)
\curveto(662.76792893,252.03999835)(662.90292879,252.10999828)(663.03293213,252.19000732)
\curveto(663.17292852,252.27999811)(663.2929284,252.38499801)(663.39293213,252.50500732)
\curveto(663.4929282,252.62499777)(663.58792811,252.75499764)(663.67793213,252.89500732)
\curveto(663.73792796,252.9949974)(663.78292791,253.10499729)(663.81293213,253.22500732)
\curveto(663.85292784,253.34499705)(663.90292779,253.44999694)(663.96293213,253.54000732)
\curveto(664.01292768,253.59999679)(664.08292761,253.63999675)(664.17293213,253.66000732)
\curveto(664.1929275,253.66999672)(664.21792748,253.67499672)(664.24793213,253.67500732)
\curveto(664.27792742,253.67499672)(664.30292739,253.67999671)(664.32293213,253.69000732)
}
}
{
\newrgbcolor{curcolor}{0 0 0}
\pscustom[linestyle=none,fillstyle=solid,fillcolor=curcolor]
{
\newpath
\moveto(671.7275415,253.69000732)
\curveto(673.35753606,253.71999667)(674.40753501,253.16499723)(674.8775415,252.02500732)
\curveto(674.97753444,251.7949986)(675.04253438,251.50499889)(675.0725415,251.15500732)
\curveto(675.11253431,250.81499958)(675.08753433,250.50499989)(674.9975415,250.22500732)
\curveto(674.90753451,249.96500043)(674.78753463,249.74000065)(674.6375415,249.55000732)
\curveto(674.6175348,249.51000088)(674.59253483,249.47500092)(674.5625415,249.44500732)
\curveto(674.53253489,249.42500097)(674.50753491,249.40000099)(674.4875415,249.37000732)
\lineto(674.3975415,249.25000732)
\curveto(674.36753505,249.22000117)(674.33253509,249.1950012)(674.2925415,249.17500732)
\curveto(674.24253518,249.12500127)(674.18753523,249.08000131)(674.1275415,249.04000732)
\curveto(674.07753534,249.00000139)(674.03253539,248.95000144)(673.9925415,248.89000732)
\curveto(673.95253547,248.85000154)(673.93753548,248.80000159)(673.9475415,248.74000732)
\curveto(673.95753546,248.6900017)(673.98753543,248.64500175)(674.0375415,248.60500732)
\curveto(674.08753533,248.56500183)(674.14253528,248.52500187)(674.2025415,248.48500732)
\curveto(674.27253515,248.45500194)(674.33753508,248.42500197)(674.3975415,248.39500732)
\curveto(674.45753496,248.36500203)(674.50753491,248.33500206)(674.5475415,248.30500732)
\curveto(674.86753455,248.08500231)(675.1225343,247.77500262)(675.3125415,247.37500732)
\curveto(675.35253407,247.28500311)(675.38253404,247.1900032)(675.4025415,247.09000732)
\curveto(675.43253399,247.00000339)(675.45753396,246.91000348)(675.4775415,246.82000732)
\curveto(675.48753393,246.77000362)(675.49253393,246.72000367)(675.4925415,246.67000732)
\curveto(675.50253392,246.63000376)(675.51253391,246.58500381)(675.5225415,246.53500732)
\curveto(675.53253389,246.48500391)(675.53253389,246.43500396)(675.5225415,246.38500732)
\curveto(675.51253391,246.33500406)(675.5175339,246.28500411)(675.5375415,246.23500732)
\curveto(675.54753387,246.18500421)(675.55253387,246.12500427)(675.5525415,246.05500732)
\curveto(675.55253387,245.98500441)(675.54253388,245.92500447)(675.5225415,245.87500732)
\lineto(675.5225415,245.65000732)
\lineto(675.4625415,245.41000732)
\curveto(675.45253397,245.34000505)(675.43753398,245.27000512)(675.4175415,245.20000732)
\curveto(675.38753403,245.11000528)(675.35753406,245.02500537)(675.3275415,244.94500732)
\curveto(675.30753411,244.86500553)(675.27753414,244.78500561)(675.2375415,244.70500732)
\curveto(675.2175342,244.64500575)(675.18753423,244.58500581)(675.1475415,244.52500732)
\curveto(675.1175343,244.47500592)(675.08253434,244.42500597)(675.0425415,244.37500732)
\curveto(674.84253458,244.06500633)(674.59253483,243.80500659)(674.2925415,243.59500732)
\curveto(673.99253543,243.395007)(673.64753577,243.23000716)(673.2575415,243.10000732)
\curveto(673.13753628,243.06000733)(673.00753641,243.03500736)(672.8675415,243.02500732)
\curveto(672.73753668,243.00500739)(672.60253682,242.98000741)(672.4625415,242.95000732)
\curveto(672.39253703,242.94000745)(672.3225371,242.93500746)(672.2525415,242.93500732)
\curveto(672.19253723,242.93500746)(672.12753729,242.93000746)(672.0575415,242.92000732)
\curveto(672.0175374,242.91000748)(671.95753746,242.90500749)(671.8775415,242.90500732)
\curveto(671.80753761,242.90500749)(671.75753766,242.91000748)(671.7275415,242.92000732)
\curveto(671.67753774,242.93000746)(671.63253779,242.93500746)(671.5925415,242.93500732)
\lineto(671.4725415,242.93500732)
\curveto(671.37253805,242.95500744)(671.27253815,242.97000742)(671.1725415,242.98000732)
\curveto(671.07253835,242.9900074)(670.97753844,243.00500739)(670.8875415,243.02500732)
\curveto(670.77753864,243.05500734)(670.66753875,243.08000731)(670.5575415,243.10000732)
\curveto(670.45753896,243.13000726)(670.35253907,243.17000722)(670.2425415,243.22000732)
\curveto(669.87253955,243.38000701)(669.55753986,243.58000681)(669.2975415,243.82000732)
\curveto(669.03754038,244.07000632)(668.82754059,244.38000601)(668.6675415,244.75000732)
\curveto(668.62754079,244.84000555)(668.59254083,244.93500546)(668.5625415,245.03500732)
\curveto(668.53254089,245.13500526)(668.50254092,245.24000515)(668.4725415,245.35000732)
\curveto(668.45254097,245.40000499)(668.44254098,245.45000494)(668.4425415,245.50000732)
\curveto(668.44254098,245.56000483)(668.43254099,245.62000477)(668.4125415,245.68000732)
\curveto(668.39254103,245.74000465)(668.38254104,245.82000457)(668.3825415,245.92000732)
\curveto(668.38254104,246.02000437)(668.39754102,246.0950043)(668.4275415,246.14500732)
\curveto(668.43754098,246.17500422)(668.45254097,246.20000419)(668.4725415,246.22000732)
\lineto(668.5325415,246.28000732)
\curveto(668.57254085,246.30000409)(668.63254079,246.31500408)(668.7125415,246.32500732)
\curveto(668.80254062,246.33500406)(668.89254053,246.34000405)(668.9825415,246.34000732)
\curveto(669.07254035,246.34000405)(669.15754026,246.33500406)(669.2375415,246.32500732)
\curveto(669.32754009,246.31500408)(669.39254003,246.30500409)(669.4325415,246.29500732)
\curveto(669.45253997,246.27500412)(669.47253995,246.26000413)(669.4925415,246.25000732)
\curveto(669.51253991,246.25000414)(669.53253989,246.24000415)(669.5525415,246.22000732)
\curveto(669.6225398,246.13000426)(669.66253976,246.01500438)(669.6725415,245.87500732)
\curveto(669.69253973,245.73500466)(669.7225397,245.61000478)(669.7625415,245.50000732)
\lineto(669.9125415,245.14000732)
\curveto(669.96253946,245.03000536)(670.02753939,244.92500547)(670.1075415,244.82500732)
\curveto(670.12753929,244.7950056)(670.14753927,244.77000562)(670.1675415,244.75000732)
\curveto(670.19753922,244.73000566)(670.2225392,244.70500569)(670.2425415,244.67500732)
\curveto(670.28253914,244.61500578)(670.3175391,244.57000582)(670.3475415,244.54000732)
\curveto(670.38753903,244.51000588)(670.422539,244.48000591)(670.4525415,244.45000732)
\curveto(670.49253893,244.42000597)(670.53753888,244.390006)(670.5875415,244.36000732)
\curveto(670.67753874,244.30000609)(670.77253865,244.25000614)(670.8725415,244.21000732)
\lineto(671.2025415,244.09000732)
\curveto(671.35253807,244.04000635)(671.55253787,244.01000638)(671.8025415,244.00000732)
\curveto(672.05253737,243.9900064)(672.26253716,244.01000638)(672.4325415,244.06000732)
\curveto(672.51253691,244.08000631)(672.58253684,244.0950063)(672.6425415,244.10500732)
\lineto(672.8525415,244.16500732)
\curveto(673.13253629,244.28500611)(673.37253605,244.43500596)(673.5725415,244.61500732)
\curveto(673.78253564,244.7950056)(673.94753547,245.02500537)(674.0675415,245.30500732)
\curveto(674.09753532,245.37500502)(674.1175353,245.44500495)(674.1275415,245.51500732)
\lineto(674.1875415,245.75500732)
\curveto(674.22753519,245.8950045)(674.23753518,246.05500434)(674.2175415,246.23500732)
\curveto(674.19753522,246.42500397)(674.16753525,246.57500382)(674.1275415,246.68500732)
\curveto(673.99753542,247.06500333)(673.81253561,247.35500304)(673.5725415,247.55500732)
\curveto(673.34253608,247.75500264)(673.03253639,247.91500248)(672.6425415,248.03500732)
\curveto(672.53253689,248.06500233)(672.41253701,248.08500231)(672.2825415,248.09500732)
\curveto(672.16253726,248.10500229)(672.03753738,248.11000228)(671.9075415,248.11000732)
\curveto(671.74753767,248.11000228)(671.60753781,248.11500228)(671.4875415,248.12500732)
\curveto(671.36753805,248.13500226)(671.28253814,248.1950022)(671.2325415,248.30500732)
\curveto(671.21253821,248.33500206)(671.20253822,248.37000202)(671.2025415,248.41000732)
\lineto(671.2025415,248.54500732)
\curveto(671.19253823,248.64500175)(671.19253823,248.74000165)(671.2025415,248.83000732)
\curveto(671.2225382,248.92000147)(671.26253816,248.98500141)(671.3225415,249.02500732)
\curveto(671.36253806,249.05500134)(671.40253802,249.07500132)(671.4425415,249.08500732)
\curveto(671.49253793,249.0950013)(671.54753787,249.10500129)(671.6075415,249.11500732)
\curveto(671.62753779,249.12500127)(671.65253777,249.12500127)(671.6825415,249.11500732)
\curveto(671.71253771,249.11500128)(671.73753768,249.12000127)(671.7575415,249.13000732)
\lineto(671.8925415,249.13000732)
\curveto(672.00253742,249.15000124)(672.10253732,249.16000123)(672.1925415,249.16000732)
\curveto(672.29253713,249.17000122)(672.38753703,249.1900012)(672.4775415,249.22000732)
\curveto(672.79753662,249.33000106)(673.05253637,249.47500092)(673.2425415,249.65500732)
\curveto(673.43253599,249.83500056)(673.58253584,250.08500031)(673.6925415,250.40500732)
\curveto(673.7225357,250.50499989)(673.74253568,250.62999976)(673.7525415,250.78000732)
\curveto(673.77253565,250.93999945)(673.76753565,251.08499931)(673.7375415,251.21500732)
\curveto(673.7175357,251.28499911)(673.69753572,251.34999904)(673.6775415,251.41000732)
\curveto(673.66753575,251.47999891)(673.64753577,251.54499885)(673.6175415,251.60500732)
\curveto(673.5175359,251.84499855)(673.37253605,252.03499836)(673.1825415,252.17500732)
\curveto(672.99253643,252.31499808)(672.76753665,252.42499797)(672.5075415,252.50500732)
\curveto(672.44753697,252.52499787)(672.38753703,252.53499786)(672.3275415,252.53500732)
\curveto(672.26753715,252.53499786)(672.20253722,252.54499785)(672.1325415,252.56500732)
\curveto(672.05253737,252.58499781)(671.95753746,252.5949978)(671.8475415,252.59500732)
\curveto(671.73753768,252.5949978)(671.64253778,252.58499781)(671.5625415,252.56500732)
\curveto(671.51253791,252.54499785)(671.46253796,252.53499786)(671.4125415,252.53500732)
\curveto(671.37253805,252.53499786)(671.32753809,252.52499787)(671.2775415,252.50500732)
\curveto(671.09753832,252.45499794)(670.92753849,252.37999801)(670.7675415,252.28000732)
\curveto(670.6175388,252.1899982)(670.48753893,252.07499832)(670.3775415,251.93500732)
\curveto(670.28753913,251.81499858)(670.20753921,251.68499871)(670.1375415,251.54500732)
\curveto(670.06753935,251.40499899)(670.00253942,251.24999914)(669.9425415,251.08000732)
\curveto(669.91253951,250.96999942)(669.89253953,250.84999954)(669.8825415,250.72000732)
\curveto(669.87253955,250.59999979)(669.83753958,250.49999989)(669.7775415,250.42000732)
\curveto(669.75753966,250.38000001)(669.69753972,250.34000005)(669.5975415,250.30000732)
\curveto(669.55753986,250.2900001)(669.49753992,250.28000011)(669.4175415,250.27000732)
\lineto(669.1625415,250.27000732)
\curveto(669.07254035,250.28000011)(668.98754043,250.2900001)(668.9075415,250.30000732)
\curveto(668.83754058,250.31000008)(668.78754063,250.32500007)(668.7575415,250.34500732)
\curveto(668.7175407,250.37500002)(668.68254074,250.42999996)(668.6525415,250.51000732)
\curveto(668.6225408,250.5899998)(668.6175408,250.67499972)(668.6375415,250.76500732)
\curveto(668.64754077,250.81499958)(668.65254077,250.86499953)(668.6525415,250.91500732)
\lineto(668.6825415,251.09500732)
\curveto(668.71254071,251.1949992)(668.73754068,251.2949991)(668.7575415,251.39500732)
\curveto(668.78754063,251.4949989)(668.8225406,251.58499881)(668.8625415,251.66500732)
\curveto(668.91254051,251.77499862)(668.95754046,251.87999851)(668.9975415,251.98000732)
\curveto(669.03754038,252.0899983)(669.08754033,252.1949982)(669.1475415,252.29500732)
\curveto(669.47753994,252.83499756)(669.94753947,253.22999716)(670.5575415,253.48000732)
\curveto(670.67753874,253.52999686)(670.80253862,253.56499683)(670.9325415,253.58500732)
\curveto(671.07253835,253.60499679)(671.21253821,253.62999676)(671.3525415,253.66000732)
\curveto(671.41253801,253.66999672)(671.47253795,253.67499672)(671.5325415,253.67500732)
\curveto(671.60253782,253.67499672)(671.66753775,253.67999671)(671.7275415,253.69000732)
}
}
{
\newrgbcolor{curcolor}{0 0 0}
\pscustom[linestyle=none,fillstyle=solid,fillcolor=curcolor]
{
\newpath
\moveto(686.76715088,251.60500732)
\curveto(686.56714058,251.31499908)(686.35714079,251.02999936)(686.13715088,250.75000732)
\curveto(685.92714122,250.46999992)(685.72214142,250.18500021)(685.52215088,249.89500732)
\curveto(684.92214222,249.04500135)(684.31714283,248.20500219)(683.70715088,247.37500732)
\curveto(683.09714405,246.55500384)(682.49214465,245.72000467)(681.89215088,244.87000732)
\lineto(681.38215088,244.15000732)
\lineto(680.87215088,243.46000732)
\curveto(680.79214635,243.35000704)(680.71214643,243.23500716)(680.63215088,243.11500732)
\curveto(680.55214659,242.9950074)(680.45714669,242.90000749)(680.34715088,242.83000732)
\curveto(680.30714684,242.81000758)(680.2421469,242.7950076)(680.15215088,242.78500732)
\curveto(680.07214707,242.76500763)(679.98214716,242.75500764)(679.88215088,242.75500732)
\curveto(679.78214736,242.75500764)(679.68714746,242.76000763)(679.59715088,242.77000732)
\curveto(679.51714763,242.78000761)(679.45714769,242.80000759)(679.41715088,242.83000732)
\curveto(679.38714776,242.85000754)(679.36214778,242.88500751)(679.34215088,242.93500732)
\curveto(679.33214781,242.97500742)(679.33714781,243.02000737)(679.35715088,243.07000732)
\curveto(679.39714775,243.15000724)(679.4421477,243.22500717)(679.49215088,243.29500732)
\curveto(679.55214759,243.37500702)(679.60714754,243.45500694)(679.65715088,243.53500732)
\curveto(679.89714725,243.87500652)(680.142147,244.21000618)(680.39215088,244.54000732)
\curveto(680.6421465,244.87000552)(680.88214626,245.20500519)(681.11215088,245.54500732)
\curveto(681.27214587,245.76500463)(681.43214571,245.98000441)(681.59215088,246.19000732)
\curveto(681.75214539,246.40000399)(681.91214523,246.61500378)(682.07215088,246.83500732)
\curveto(682.43214471,247.35500304)(682.79714435,247.86500253)(683.16715088,248.36500732)
\curveto(683.53714361,248.86500153)(683.90714324,249.37500102)(684.27715088,249.89500732)
\curveto(684.41714273,250.0950003)(684.55714259,250.2900001)(684.69715088,250.48000732)
\curveto(684.8471423,250.66999972)(684.99214215,250.86499953)(685.13215088,251.06500732)
\curveto(685.3421418,251.36499903)(685.55714159,251.66499873)(685.77715088,251.96500732)
\lineto(686.43715088,252.86500732)
\lineto(686.61715088,253.13500732)
\lineto(686.82715088,253.40500732)
\lineto(686.94715088,253.58500732)
\curveto(686.99714015,253.64499675)(687.0471401,253.69999669)(687.09715088,253.75000732)
\curveto(687.16713998,253.79999659)(687.2421399,253.83499656)(687.32215088,253.85500732)
\curveto(687.3421398,253.86499653)(687.36713978,253.86499653)(687.39715088,253.85500732)
\curveto(687.43713971,253.85499654)(687.46713968,253.86499653)(687.48715088,253.88500732)
\curveto(687.60713954,253.88499651)(687.7421394,253.87999651)(687.89215088,253.87000732)
\curveto(688.0421391,253.86999652)(688.13213901,253.82499657)(688.16215088,253.73500732)
\curveto(688.18213896,253.70499669)(688.18713896,253.66999672)(688.17715088,253.63000732)
\curveto(688.16713898,253.5899968)(688.15213899,253.55999683)(688.13215088,253.54000732)
\curveto(688.09213905,253.45999693)(688.05213909,253.389997)(688.01215088,253.33000732)
\curveto(687.97213917,253.26999712)(687.92713922,253.20999718)(687.87715088,253.15000732)
\lineto(687.30715088,252.37000732)
\curveto(687.12714002,252.11999827)(686.9471402,251.86499853)(686.76715088,251.60500732)
\moveto(679.91215088,247.70500732)
\curveto(679.86214728,247.72500267)(679.81214733,247.73000266)(679.76215088,247.72000732)
\curveto(679.71214743,247.71000268)(679.66214748,247.71500268)(679.61215088,247.73500732)
\curveto(679.50214764,247.75500264)(679.39714775,247.77500262)(679.29715088,247.79500732)
\curveto(679.20714794,247.82500257)(679.11214803,247.86500253)(679.01215088,247.91500732)
\curveto(678.68214846,248.05500234)(678.42714872,248.25000214)(678.24715088,248.50000732)
\curveto(678.06714908,248.76000163)(677.92214922,249.07000132)(677.81215088,249.43000732)
\curveto(677.78214936,249.51000088)(677.76214938,249.5900008)(677.75215088,249.67000732)
\curveto(677.7421494,249.76000063)(677.72714942,249.84500055)(677.70715088,249.92500732)
\curveto(677.69714945,249.97500042)(677.69214945,250.04000035)(677.69215088,250.12000732)
\curveto(677.68214946,250.15000024)(677.67714947,250.18000021)(677.67715088,250.21000732)
\curveto(677.67714947,250.25000014)(677.67214947,250.28500011)(677.66215088,250.31500732)
\lineto(677.66215088,250.46500732)
\curveto(677.65214949,250.51499988)(677.6471495,250.57499982)(677.64715088,250.64500732)
\curveto(677.6471495,250.72499967)(677.65214949,250.7899996)(677.66215088,250.84000732)
\lineto(677.66215088,251.00500732)
\curveto(677.68214946,251.05499934)(677.68714946,251.09999929)(677.67715088,251.14000732)
\curveto(677.67714947,251.1899992)(677.68214946,251.23499916)(677.69215088,251.27500732)
\curveto(677.70214944,251.31499908)(677.70714944,251.34999904)(677.70715088,251.38000732)
\curveto(677.70714944,251.41999897)(677.71214943,251.45999893)(677.72215088,251.50000732)
\curveto(677.75214939,251.60999878)(677.77214937,251.71999867)(677.78215088,251.83000732)
\curveto(677.80214934,251.94999844)(677.83714931,252.06499833)(677.88715088,252.17500732)
\curveto(678.02714912,252.51499788)(678.18714896,252.7899976)(678.36715088,253.00000732)
\curveto(678.55714859,253.21999717)(678.82714832,253.39999699)(679.17715088,253.54000732)
\curveto(679.25714789,253.56999682)(679.3421478,253.5899968)(679.43215088,253.60000732)
\curveto(679.52214762,253.61999677)(679.61714753,253.63999675)(679.71715088,253.66000732)
\curveto(679.7471474,253.66999672)(679.80214734,253.66999672)(679.88215088,253.66000732)
\curveto(679.96214718,253.65999673)(680.01214713,253.66999672)(680.03215088,253.69000732)
\curveto(680.59214655,253.69999669)(681.0421461,253.5899968)(681.38215088,253.36000732)
\curveto(681.73214541,253.12999726)(681.99214515,252.82499757)(682.16215088,252.44500732)
\curveto(682.20214494,252.35499804)(682.23714491,252.25999813)(682.26715088,252.16000732)
\curveto(682.29714485,252.05999833)(682.32214482,251.95999843)(682.34215088,251.86000732)
\curveto(682.36214478,251.82999856)(682.36714478,251.79999859)(682.35715088,251.77000732)
\curveto(682.35714479,251.73999865)(682.36214478,251.70999868)(682.37215088,251.68000732)
\curveto(682.40214474,251.56999882)(682.42214472,251.44499895)(682.43215088,251.30500732)
\curveto(682.4421447,251.17499922)(682.45214469,251.03999935)(682.46215088,250.90000732)
\lineto(682.46215088,250.73500732)
\curveto(682.47214467,250.67499972)(682.47214467,250.61999977)(682.46215088,250.57000732)
\curveto(682.45214469,250.51999987)(682.4471447,250.46999992)(682.44715088,250.42000732)
\lineto(682.44715088,250.28500732)
\curveto(682.43714471,250.24500015)(682.43214471,250.20500019)(682.43215088,250.16500732)
\curveto(682.4421447,250.12500027)(682.43714471,250.08000031)(682.41715088,250.03000732)
\curveto(682.39714475,249.92000047)(682.37714477,249.81500058)(682.35715088,249.71500732)
\curveto(682.3471448,249.61500078)(682.32714482,249.51500088)(682.29715088,249.41500732)
\curveto(682.16714498,249.05500134)(682.00214514,248.74000165)(681.80215088,248.47000732)
\curveto(681.60214554,248.20000219)(681.32714582,247.9950024)(680.97715088,247.85500732)
\curveto(680.89714625,247.82500257)(680.81214633,247.80000259)(680.72215088,247.78000732)
\lineto(680.45215088,247.72000732)
\curveto(680.40214674,247.71000268)(680.35714679,247.70500269)(680.31715088,247.70500732)
\curveto(680.27714687,247.71500268)(680.23714691,247.71500268)(680.19715088,247.70500732)
\curveto(680.09714705,247.68500271)(680.00214714,247.68500271)(679.91215088,247.70500732)
\moveto(679.07215088,249.10000732)
\curveto(679.11214803,249.03000136)(679.15214799,248.96500143)(679.19215088,248.90500732)
\curveto(679.23214791,248.85500154)(679.28214786,248.80500159)(679.34215088,248.75500732)
\lineto(679.49215088,248.63500732)
\curveto(679.55214759,248.60500179)(679.61714753,248.58000181)(679.68715088,248.56000732)
\curveto(679.72714742,248.54000185)(679.76214738,248.53000186)(679.79215088,248.53000732)
\curveto(679.83214731,248.54000185)(679.87214727,248.53500186)(679.91215088,248.51500732)
\curveto(679.9421472,248.51500188)(679.98214716,248.51000188)(680.03215088,248.50000732)
\curveto(680.08214706,248.50000189)(680.12214702,248.50500189)(680.15215088,248.51500732)
\lineto(680.37715088,248.56000732)
\curveto(680.62714652,248.64000175)(680.81214633,248.76500163)(680.93215088,248.93500732)
\curveto(681.01214613,249.03500136)(681.08214606,249.16500123)(681.14215088,249.32500732)
\curveto(681.22214592,249.50500089)(681.28214586,249.73000066)(681.32215088,250.00000732)
\curveto(681.36214578,250.28000011)(681.37714577,250.55999983)(681.36715088,250.84000732)
\curveto(681.35714579,251.12999926)(681.32714582,251.40499899)(681.27715088,251.66500732)
\curveto(681.22714592,251.92499847)(681.15214599,252.13499826)(681.05215088,252.29500732)
\curveto(680.93214621,252.4949979)(680.78214636,252.64499775)(680.60215088,252.74500732)
\curveto(680.52214662,252.7949976)(680.43214671,252.82499757)(680.33215088,252.83500732)
\curveto(680.23214691,252.85499754)(680.12714702,252.86499753)(680.01715088,252.86500732)
\curveto(679.99714715,252.85499754)(679.97214717,252.84999754)(679.94215088,252.85000732)
\curveto(679.92214722,252.85999753)(679.90214724,252.85999753)(679.88215088,252.85000732)
\curveto(679.83214731,252.83999755)(679.78714736,252.82999756)(679.74715088,252.82000732)
\curveto(679.70714744,252.81999757)(679.66714748,252.80999758)(679.62715088,252.79000732)
\curveto(679.4471477,252.70999768)(679.29714785,252.5899978)(679.17715088,252.43000732)
\curveto(679.06714808,252.26999812)(678.97714817,252.0899983)(678.90715088,251.89000732)
\curveto(678.8471483,251.69999869)(678.80214834,251.47499892)(678.77215088,251.21500732)
\curveto(678.75214839,250.95499944)(678.7471484,250.6899997)(678.75715088,250.42000732)
\curveto(678.76714838,250.16000023)(678.79714835,249.91000048)(678.84715088,249.67000732)
\curveto(678.90714824,249.44000095)(678.98214816,249.25000114)(679.07215088,249.10000732)
\moveto(689.87215088,246.11500732)
\curveto(689.88213726,246.06500433)(689.88713726,245.97500442)(689.88715088,245.84500732)
\curveto(689.88713726,245.71500468)(689.87713727,245.62500477)(689.85715088,245.57500732)
\curveto(689.83713731,245.52500487)(689.83213731,245.47000492)(689.84215088,245.41000732)
\curveto(689.85213729,245.36000503)(689.85213729,245.31000508)(689.84215088,245.26000732)
\curveto(689.80213734,245.12000527)(689.77213737,244.98500541)(689.75215088,244.85500732)
\curveto(689.7421374,244.72500567)(689.71213743,244.60500579)(689.66215088,244.49500732)
\curveto(689.52213762,244.14500625)(689.35713779,243.85000654)(689.16715088,243.61000732)
\curveto(688.97713817,243.38000701)(688.70713844,243.1950072)(688.35715088,243.05500732)
\curveto(688.27713887,243.02500737)(688.19213895,243.00500739)(688.10215088,242.99500732)
\curveto(688.01213913,242.97500742)(687.92713922,242.95500744)(687.84715088,242.93500732)
\curveto(687.79713935,242.92500747)(687.7471394,242.92000747)(687.69715088,242.92000732)
\curveto(687.6471395,242.92000747)(687.59713955,242.91500748)(687.54715088,242.90500732)
\curveto(687.51713963,242.8950075)(687.46713968,242.8950075)(687.39715088,242.90500732)
\curveto(687.32713982,242.90500749)(687.27713987,242.91000748)(687.24715088,242.92000732)
\curveto(687.18713996,242.94000745)(687.12714002,242.95000744)(687.06715088,242.95000732)
\curveto(687.01714013,242.94000745)(686.96714018,242.94500745)(686.91715088,242.96500732)
\curveto(686.82714032,242.98500741)(686.73714041,243.01000738)(686.64715088,243.04000732)
\curveto(686.56714058,243.06000733)(686.48714066,243.0900073)(686.40715088,243.13000732)
\curveto(686.08714106,243.27000712)(685.83714131,243.46500693)(685.65715088,243.71500732)
\curveto(685.47714167,243.97500642)(685.32714182,244.28000611)(685.20715088,244.63000732)
\curveto(685.18714196,244.71000568)(685.17214197,244.7950056)(685.16215088,244.88500732)
\curveto(685.15214199,244.97500542)(685.13714201,245.06000533)(685.11715088,245.14000732)
\curveto(685.10714204,245.17000522)(685.10214204,245.20000519)(685.10215088,245.23000732)
\lineto(685.10215088,245.33500732)
\curveto(685.08214206,245.41500498)(685.07214207,245.4950049)(685.07215088,245.57500732)
\lineto(685.07215088,245.71000732)
\curveto(685.05214209,245.81000458)(685.05214209,245.91000448)(685.07215088,246.01000732)
\lineto(685.07215088,246.19000732)
\curveto(685.08214206,246.24000415)(685.08714206,246.28500411)(685.08715088,246.32500732)
\curveto(685.08714206,246.37500402)(685.09214205,246.42000397)(685.10215088,246.46000732)
\curveto(685.11214203,246.50000389)(685.11714203,246.53500386)(685.11715088,246.56500732)
\curveto(685.11714203,246.60500379)(685.12214202,246.64500375)(685.13215088,246.68500732)
\lineto(685.19215088,247.01500732)
\curveto(685.21214193,247.13500326)(685.2421419,247.24500315)(685.28215088,247.34500732)
\curveto(685.42214172,247.67500272)(685.58214156,247.95000244)(685.76215088,248.17000732)
\curveto(685.95214119,248.40000199)(686.21214093,248.58500181)(686.54215088,248.72500732)
\curveto(686.62214052,248.76500163)(686.70714044,248.7900016)(686.79715088,248.80000732)
\lineto(687.09715088,248.86000732)
\lineto(687.23215088,248.86000732)
\curveto(687.28213986,248.87000152)(687.33213981,248.87500152)(687.38215088,248.87500732)
\curveto(687.95213919,248.8950015)(688.41213873,248.7900016)(688.76215088,248.56000732)
\curveto(689.12213802,248.34000205)(689.38713776,248.04000235)(689.55715088,247.66000732)
\curveto(689.60713754,247.56000283)(689.6471375,247.46000293)(689.67715088,247.36000732)
\curveto(689.70713744,247.26000313)(689.73713741,247.15500324)(689.76715088,247.04500732)
\curveto(689.77713737,247.00500339)(689.78213736,246.97000342)(689.78215088,246.94000732)
\curveto(689.78213736,246.92000347)(689.78713736,246.8900035)(689.79715088,246.85000732)
\curveto(689.81713733,246.78000361)(689.82713732,246.70500369)(689.82715088,246.62500732)
\curveto(689.82713732,246.54500385)(689.83713731,246.46500393)(689.85715088,246.38500732)
\curveto(689.85713729,246.33500406)(689.85713729,246.2900041)(689.85715088,246.25000732)
\curveto(689.85713729,246.21000418)(689.86213728,246.16500423)(689.87215088,246.11500732)
\moveto(688.76215088,245.68000732)
\curveto(688.77213837,245.73000466)(688.77713837,245.80500459)(688.77715088,245.90500732)
\curveto(688.78713836,246.00500439)(688.78213836,246.08000431)(688.76215088,246.13000732)
\curveto(688.7421384,246.1900042)(688.73713841,246.24500415)(688.74715088,246.29500732)
\curveto(688.76713838,246.35500404)(688.76713838,246.41500398)(688.74715088,246.47500732)
\curveto(688.73713841,246.50500389)(688.73213841,246.54000385)(688.73215088,246.58000732)
\curveto(688.73213841,246.62000377)(688.72713842,246.66000373)(688.71715088,246.70000732)
\curveto(688.69713845,246.78000361)(688.67713847,246.85500354)(688.65715088,246.92500732)
\curveto(688.6471385,247.00500339)(688.63213851,247.08500331)(688.61215088,247.16500732)
\curveto(688.58213856,247.22500317)(688.55713859,247.28500311)(688.53715088,247.34500732)
\curveto(688.51713863,247.40500299)(688.48713866,247.46500293)(688.44715088,247.52500732)
\curveto(688.3471388,247.6950027)(688.21713893,247.83000256)(688.05715088,247.93000732)
\curveto(687.97713917,247.98000241)(687.88213926,248.01500238)(687.77215088,248.03500732)
\curveto(687.66213948,248.05500234)(687.53713961,248.06500233)(687.39715088,248.06500732)
\curveto(687.37713977,248.05500234)(687.35213979,248.05000234)(687.32215088,248.05000732)
\curveto(687.29213985,248.06000233)(687.26213988,248.06000233)(687.23215088,248.05000732)
\lineto(687.08215088,247.99000732)
\curveto(687.03214011,247.98000241)(686.98714016,247.96500243)(686.94715088,247.94500732)
\curveto(686.75714039,247.83500256)(686.61214053,247.6900027)(686.51215088,247.51000732)
\curveto(686.42214072,247.33000306)(686.3421408,247.12500327)(686.27215088,246.89500732)
\curveto(686.23214091,246.76500363)(686.21214093,246.63000376)(686.21215088,246.49000732)
\curveto(686.21214093,246.36000403)(686.20214094,246.21500418)(686.18215088,246.05500732)
\curveto(686.17214097,246.00500439)(686.16214098,245.94500445)(686.15215088,245.87500732)
\curveto(686.15214099,245.80500459)(686.16214098,245.74500465)(686.18215088,245.69500732)
\lineto(686.18215088,245.53000732)
\lineto(686.18215088,245.35000732)
\curveto(686.19214095,245.30000509)(686.20214094,245.24500515)(686.21215088,245.18500732)
\curveto(686.22214092,245.13500526)(686.22714092,245.08000531)(686.22715088,245.02000732)
\curveto(686.23714091,244.96000543)(686.25214089,244.90500549)(686.27215088,244.85500732)
\curveto(686.32214082,244.66500573)(686.38214076,244.4900059)(686.45215088,244.33000732)
\curveto(686.52214062,244.17000622)(686.62714052,244.04000635)(686.76715088,243.94000732)
\curveto(686.89714025,243.84000655)(687.03714011,243.77000662)(687.18715088,243.73000732)
\curveto(687.21713993,243.72000667)(687.2421399,243.71500668)(687.26215088,243.71500732)
\curveto(687.29213985,243.72500667)(687.32213982,243.72500667)(687.35215088,243.71500732)
\curveto(687.37213977,243.71500668)(687.40213974,243.71000668)(687.44215088,243.70000732)
\curveto(687.48213966,243.70000669)(687.51713963,243.70500669)(687.54715088,243.71500732)
\curveto(687.58713956,243.72500667)(687.62713952,243.73000666)(687.66715088,243.73000732)
\curveto(687.70713944,243.73000666)(687.7471394,243.74000665)(687.78715088,243.76000732)
\curveto(688.02713912,243.84000655)(688.22213892,243.97500642)(688.37215088,244.16500732)
\curveto(688.49213865,244.34500605)(688.58213856,244.55000584)(688.64215088,244.78000732)
\curveto(688.66213848,244.85000554)(688.67713847,244.92000547)(688.68715088,244.99000732)
\curveto(688.69713845,245.07000532)(688.71213843,245.15000524)(688.73215088,245.23000732)
\curveto(688.73213841,245.2900051)(688.73713841,245.33500506)(688.74715088,245.36500732)
\curveto(688.7471384,245.38500501)(688.7471384,245.41000498)(688.74715088,245.44000732)
\curveto(688.7471384,245.48000491)(688.75213839,245.51000488)(688.76215088,245.53000732)
\lineto(688.76215088,245.68000732)
}
}
{
\newrgbcolor{curcolor}{0 0 0}
\pscustom[linestyle=none,fillstyle=solid,fillcolor=curcolor]
{
\newpath
\moveto(523.706521,52.58859619)
\curveto(523.71651328,52.54859314)(523.71651328,52.49859319)(523.706521,52.43859619)
\curveto(523.70651329,52.37859331)(523.70151329,52.32859336)(523.691521,52.28859619)
\curveto(523.6915133,52.24859344)(523.68651331,52.20859348)(523.676521,52.16859619)
\lineto(523.676521,52.06359619)
\curveto(523.65651334,51.98359371)(523.64151335,51.90359379)(523.631521,51.82359619)
\curveto(523.62151337,51.74359395)(523.60151339,51.66859402)(523.571521,51.59859619)
\curveto(523.55151344,51.51859417)(523.53151346,51.44359425)(523.511521,51.37359619)
\curveto(523.4915135,51.30359439)(523.46151353,51.22859446)(523.421521,51.14859619)
\curveto(523.24151375,50.72859496)(522.98651401,50.3885953)(522.656521,50.12859619)
\curveto(522.32651467,49.86859582)(521.93651506,49.66359603)(521.486521,49.51359619)
\curveto(521.36651563,49.47359622)(521.24151575,49.44859624)(521.111521,49.43859619)
\curveto(520.991516,49.41859627)(520.86651613,49.3935963)(520.736521,49.36359619)
\curveto(520.67651632,49.35359634)(520.61151638,49.34859634)(520.541521,49.34859619)
\curveto(520.48151651,49.34859634)(520.41651658,49.34359635)(520.346521,49.33359619)
\lineto(520.226521,49.33359619)
\lineto(520.031521,49.33359619)
\curveto(519.97151702,49.32359637)(519.91651708,49.32859636)(519.866521,49.34859619)
\curveto(519.7965172,49.36859632)(519.73151726,49.37359632)(519.671521,49.36359619)
\curveto(519.61151738,49.35359634)(519.55151744,49.35859633)(519.491521,49.37859619)
\curveto(519.44151755,49.3885963)(519.3965176,49.3935963)(519.356521,49.39359619)
\curveto(519.31651768,49.3935963)(519.27151772,49.40359629)(519.221521,49.42359619)
\curveto(519.14151785,49.44359625)(519.06651793,49.46359623)(518.996521,49.48359619)
\curveto(518.92651807,49.4935962)(518.85651814,49.50859618)(518.786521,49.52859619)
\curveto(518.30651869,49.69859599)(517.90651909,49.90859578)(517.586521,50.15859619)
\curveto(517.27651972,50.41859527)(517.02651997,50.77359492)(516.836521,51.22359619)
\curveto(516.80652019,51.28359441)(516.78152021,51.34359435)(516.761521,51.40359619)
\curveto(516.75152024,51.47359422)(516.73652026,51.54859414)(516.716521,51.62859619)
\curveto(516.6965203,51.688594)(516.68152031,51.75359394)(516.671521,51.82359619)
\curveto(516.66152033,51.8935938)(516.64652035,51.96359373)(516.626521,52.03359619)
\curveto(516.61652038,52.08359361)(516.61152038,52.12359357)(516.611521,52.15359619)
\lineto(516.611521,52.27359619)
\curveto(516.60152039,52.31359338)(516.5915204,52.36359333)(516.581521,52.42359619)
\curveto(516.58152041,52.48359321)(516.58652041,52.53359316)(516.596521,52.57359619)
\lineto(516.596521,52.70859619)
\curveto(516.60652039,52.75859293)(516.61152038,52.80859288)(516.611521,52.85859619)
\curveto(516.63152036,52.95859273)(516.64652035,53.05359264)(516.656521,53.14359619)
\curveto(516.66652033,53.24359245)(516.68652031,53.33859235)(516.716521,53.42859619)
\curveto(516.76652023,53.57859211)(516.82152017,53.71859197)(516.881521,53.84859619)
\curveto(516.94152005,53.97859171)(517.01151998,54.09859159)(517.091521,54.20859619)
\curveto(517.12151987,54.25859143)(517.15151984,54.29859139)(517.181521,54.32859619)
\curveto(517.22151977,54.35859133)(517.25651974,54.3935913)(517.286521,54.43359619)
\curveto(517.34651965,54.51359118)(517.41651958,54.58359111)(517.496521,54.64359619)
\curveto(517.55651944,54.693591)(517.61651938,54.73859095)(517.676521,54.77859619)
\lineto(517.886521,54.92859619)
\curveto(517.93651906,54.96859072)(517.98651901,55.00359069)(518.036521,55.03359619)
\curveto(518.08651891,55.07359062)(518.12151887,55.12859056)(518.141521,55.19859619)
\curveto(518.14151885,55.22859046)(518.13151886,55.25359044)(518.111521,55.27359619)
\curveto(518.10151889,55.30359039)(518.0915189,55.32859036)(518.081521,55.34859619)
\curveto(518.04151895,55.39859029)(517.991519,55.44359025)(517.931521,55.48359619)
\curveto(517.88151911,55.53359016)(517.83151916,55.57859011)(517.781521,55.61859619)
\curveto(517.74151925,55.64859004)(517.6915193,55.70358999)(517.631521,55.78359619)
\curveto(517.61151938,55.81358988)(517.58151941,55.83858985)(517.541521,55.85859619)
\curveto(517.51151948,55.8885898)(517.48651951,55.92358977)(517.466521,55.96359619)
\curveto(517.2965197,56.17358952)(517.16651983,56.41858927)(517.076521,56.69859619)
\curveto(517.05651994,56.77858891)(517.04151995,56.85858883)(517.031521,56.93859619)
\curveto(517.02151997,57.01858867)(517.00651999,57.09858859)(516.986521,57.17859619)
\curveto(516.96652003,57.22858846)(516.95652004,57.2935884)(516.956521,57.37359619)
\curveto(516.95652004,57.46358823)(516.96652003,57.53358816)(516.986521,57.58359619)
\curveto(516.98652001,57.68358801)(516.99152,57.75358794)(517.001521,57.79359619)
\curveto(517.02151997,57.87358782)(517.03651996,57.94358775)(517.046521,58.00359619)
\curveto(517.05651994,58.07358762)(517.07151992,58.14358755)(517.091521,58.21359619)
\curveto(517.24151975,58.64358705)(517.45651954,58.9885867)(517.736521,59.24859619)
\curveto(518.02651897,59.50858618)(518.37651862,59.72358597)(518.786521,59.89359619)
\curveto(518.8965181,59.94358575)(519.01151798,59.97358572)(519.131521,59.98359619)
\curveto(519.26151773,60.00358569)(519.3915176,60.03358566)(519.521521,60.07359619)
\curveto(519.60151739,60.07358562)(519.67151732,60.07358562)(519.731521,60.07359619)
\curveto(519.80151719,60.08358561)(519.87651712,60.0935856)(519.956521,60.10359619)
\curveto(520.74651625,60.12358557)(521.40151559,59.9935857)(521.921521,59.71359619)
\curveto(522.45151454,59.43358626)(522.83151416,59.02358667)(523.061521,58.48359619)
\curveto(523.17151382,58.25358744)(523.24151375,57.96858772)(523.271521,57.62859619)
\curveto(523.31151368,57.29858839)(523.28151371,56.9935887)(523.181521,56.71359619)
\curveto(523.14151385,56.58358911)(523.0915139,56.46358923)(523.031521,56.35359619)
\curveto(522.98151401,56.24358945)(522.92151407,56.13858955)(522.851521,56.03859619)
\curveto(522.83151416,55.99858969)(522.80151419,55.96358973)(522.761521,55.93359619)
\lineto(522.671521,55.84359619)
\curveto(522.62151437,55.75358994)(522.56151443,55.68859)(522.491521,55.64859619)
\curveto(522.44151455,55.59859009)(522.38651461,55.54859014)(522.326521,55.49859619)
\curveto(522.27651472,55.45859023)(522.23151476,55.41359028)(522.191521,55.36359619)
\curveto(522.17151482,55.34359035)(522.15151484,55.31859037)(522.131521,55.28859619)
\curveto(522.12151487,55.26859042)(522.12151487,55.24359045)(522.131521,55.21359619)
\curveto(522.14151485,55.16359053)(522.17151482,55.11359058)(522.221521,55.06359619)
\curveto(522.27151472,55.02359067)(522.32651467,54.98359071)(522.386521,54.94359619)
\lineto(522.566521,54.82359619)
\curveto(522.62651437,54.7935909)(522.67651432,54.76359093)(522.716521,54.73359619)
\curveto(523.04651395,54.4935912)(523.2965137,54.18359151)(523.466521,53.80359619)
\curveto(523.50651349,53.72359197)(523.53651346,53.63859205)(523.556521,53.54859619)
\curveto(523.58651341,53.45859223)(523.61151338,53.36859232)(523.631521,53.27859619)
\curveto(523.64151335,53.22859246)(523.65151334,53.17359252)(523.661521,53.11359619)
\lineto(523.691521,52.96359619)
\curveto(523.70151329,52.90359279)(523.70151329,52.83859285)(523.691521,52.76859619)
\curveto(523.68151331,52.70859298)(523.68651331,52.64859304)(523.706521,52.58859619)
\moveto(518.321521,57.62859619)
\curveto(518.2915187,57.51858817)(518.28651871,57.37858831)(518.306521,57.20859619)
\curveto(518.32651867,57.04858864)(518.35151864,56.92358877)(518.381521,56.83359619)
\curveto(518.4915185,56.51358918)(518.64151835,56.26858942)(518.831521,56.09859619)
\curveto(519.02151797,55.93858975)(519.28651771,55.80858988)(519.626521,55.70859619)
\curveto(519.75651724,55.67859001)(519.92151707,55.65359004)(520.121521,55.63359619)
\curveto(520.32151667,55.62359007)(520.4915165,55.63859005)(520.631521,55.67859619)
\curveto(520.92151607,55.75858993)(521.16151583,55.86858982)(521.351521,56.00859619)
\curveto(521.55151544,56.15858953)(521.70651529,56.35858933)(521.816521,56.60859619)
\curveto(521.83651516,56.65858903)(521.84651515,56.70358899)(521.846521,56.74359619)
\curveto(521.85651514,56.78358891)(521.87151512,56.82858886)(521.891521,56.87859619)
\curveto(521.92151507,56.9885887)(521.94151505,57.12858856)(521.951521,57.29859619)
\curveto(521.96151503,57.46858822)(521.95151504,57.61358808)(521.921521,57.73359619)
\curveto(521.90151509,57.82358787)(521.87651512,57.90858778)(521.846521,57.98859619)
\curveto(521.82651517,58.06858762)(521.7915152,58.14858754)(521.741521,58.22859619)
\curveto(521.57151542,58.49858719)(521.34651565,58.693587)(521.066521,58.81359619)
\curveto(520.7965162,58.93358676)(520.43651656,58.9935867)(519.986521,58.99359619)
\curveto(519.96651703,58.97358672)(519.93651706,58.96858672)(519.896521,58.97859619)
\curveto(519.85651714,58.9885867)(519.82151717,58.9885867)(519.791521,58.97859619)
\curveto(519.74151725,58.95858673)(519.68651731,58.94358675)(519.626521,58.93359619)
\curveto(519.57651742,58.93358676)(519.52651747,58.92358677)(519.476521,58.90359619)
\curveto(519.23651776,58.81358688)(519.02651797,58.69858699)(518.846521,58.55859619)
\curveto(518.66651833,58.42858726)(518.52651847,58.24858744)(518.426521,58.01859619)
\curveto(518.40651859,57.95858773)(518.38651861,57.8935878)(518.366521,57.82359619)
\curveto(518.35651864,57.76358793)(518.34151865,57.69858799)(518.321521,57.62859619)
\moveto(522.341521,52.09359619)
\curveto(522.3915146,52.28359341)(522.3965146,52.4885932)(522.356521,52.70859619)
\curveto(522.32651467,52.92859276)(522.28151471,53.10859258)(522.221521,53.24859619)
\curveto(522.05151494,53.61859207)(521.7915152,53.92359177)(521.441521,54.16359619)
\curveto(521.10151589,54.40359129)(520.66651633,54.52359117)(520.136521,54.52359619)
\curveto(520.10651689,54.50359119)(520.06651693,54.49859119)(520.016521,54.50859619)
\curveto(519.96651703,54.52859116)(519.92651707,54.53359116)(519.896521,54.52359619)
\lineto(519.626521,54.46359619)
\curveto(519.54651745,54.45359124)(519.46651753,54.43859125)(519.386521,54.41859619)
\curveto(519.08651791,54.30859138)(518.82151817,54.16359153)(518.591521,53.98359619)
\curveto(518.37151862,53.80359189)(518.20151879,53.57359212)(518.081521,53.29359619)
\curveto(518.05151894,53.21359248)(518.02651897,53.13359256)(518.006521,53.05359619)
\curveto(517.98651901,52.97359272)(517.96651903,52.8885928)(517.946521,52.79859619)
\curveto(517.91651908,52.67859301)(517.90651909,52.52859316)(517.916521,52.34859619)
\curveto(517.93651906,52.16859352)(517.96151903,52.02859366)(517.991521,51.92859619)
\curveto(518.01151898,51.87859381)(518.02151897,51.83359386)(518.021521,51.79359619)
\curveto(518.03151896,51.76359393)(518.04651895,51.72359397)(518.066521,51.67359619)
\curveto(518.16651883,51.45359424)(518.2965187,51.25359444)(518.456521,51.07359619)
\curveto(518.62651837,50.8935948)(518.82151817,50.75859493)(519.041521,50.66859619)
\curveto(519.11151788,50.62859506)(519.20651779,50.5935951)(519.326521,50.56359619)
\curveto(519.54651745,50.47359522)(519.80151719,50.42859526)(520.091521,50.42859619)
\lineto(520.376521,50.42859619)
\curveto(520.47651652,50.44859524)(520.57151642,50.46359523)(520.661521,50.47359619)
\curveto(520.75151624,50.48359521)(520.84151615,50.50359519)(520.931521,50.53359619)
\curveto(521.1915158,50.61359508)(521.43151556,50.74359495)(521.651521,50.92359619)
\curveto(521.88151511,51.11359458)(522.05151494,51.32859436)(522.161521,51.56859619)
\curveto(522.20151479,51.64859404)(522.23151476,51.72859396)(522.251521,51.80859619)
\curveto(522.28151471,51.89859379)(522.31151468,51.9935937)(522.341521,52.09359619)
}
}
{
\newrgbcolor{curcolor}{0 0 0}
\pscustom[linestyle=none,fillstyle=solid,fillcolor=curcolor]
{
\newpath
\moveto(528.29113037,60.11859619)
\curveto(528.98112574,60.12858556)(529.58112514,60.00858568)(530.09113037,59.75859619)
\curveto(530.61112411,59.50858618)(531.00612371,59.17358652)(531.27613037,58.75359619)
\curveto(531.32612339,58.67358702)(531.37112335,58.58358711)(531.41113037,58.48359619)
\curveto(531.45112327,58.3935873)(531.49612322,58.29858739)(531.54613037,58.19859619)
\curveto(531.58612313,58.09858759)(531.6161231,57.99858769)(531.63613037,57.89859619)
\curveto(531.65612306,57.79858789)(531.67612304,57.693588)(531.69613037,57.58359619)
\curveto(531.716123,57.53358816)(531.721123,57.4885882)(531.71113037,57.44859619)
\curveto(531.70112302,57.40858828)(531.70612301,57.36358833)(531.72613037,57.31359619)
\curveto(531.73612298,57.26358843)(531.74112298,57.17858851)(531.74113037,57.05859619)
\curveto(531.74112298,56.94858874)(531.73612298,56.86358883)(531.72613037,56.80359619)
\curveto(531.70612301,56.74358895)(531.69612302,56.68358901)(531.69613037,56.62359619)
\curveto(531.70612301,56.56358913)(531.70112302,56.50358919)(531.68113037,56.44359619)
\curveto(531.64112308,56.30358939)(531.60612311,56.16858952)(531.57613037,56.03859619)
\curveto(531.54612317,55.90858978)(531.50612321,55.78358991)(531.45613037,55.66359619)
\curveto(531.39612332,55.52359017)(531.32612339,55.39859029)(531.24613037,55.28859619)
\curveto(531.17612354,55.17859051)(531.10112362,55.06859062)(531.02113037,54.95859619)
\lineto(530.96113037,54.89859619)
\curveto(530.95112377,54.87859081)(530.93612378,54.85859083)(530.91613037,54.83859619)
\curveto(530.79612392,54.67859101)(530.66112406,54.53359116)(530.51113037,54.40359619)
\curveto(530.36112436,54.27359142)(530.20112452,54.14859154)(530.03113037,54.02859619)
\curveto(529.721125,53.80859188)(529.42612529,53.60359209)(529.14613037,53.41359619)
\curveto(528.9161258,53.27359242)(528.68612603,53.13859255)(528.45613037,53.00859619)
\curveto(528.23612648,52.87859281)(528.0161267,52.74359295)(527.79613037,52.60359619)
\curveto(527.54612717,52.43359326)(527.30612741,52.25359344)(527.07613037,52.06359619)
\curveto(526.85612786,51.87359382)(526.66612805,51.64859404)(526.50613037,51.38859619)
\curveto(526.46612825,51.32859436)(526.43112829,51.26859442)(526.40113037,51.20859619)
\curveto(526.37112835,51.15859453)(526.34112838,51.0935946)(526.31113037,51.01359619)
\curveto(526.29112843,50.94359475)(526.28612843,50.88359481)(526.29613037,50.83359619)
\curveto(526.3161284,50.76359493)(526.35112837,50.70859498)(526.40113037,50.66859619)
\curveto(526.45112827,50.63859505)(526.51112821,50.61859507)(526.58113037,50.60859619)
\lineto(526.82113037,50.60859619)
\lineto(527.57113037,50.60859619)
\lineto(530.37613037,50.60859619)
\lineto(531.03613037,50.60859619)
\curveto(531.12612359,50.60859508)(531.21112351,50.60359509)(531.29113037,50.59359619)
\curveto(531.37112335,50.5935951)(531.43612328,50.57359512)(531.48613037,50.53359619)
\curveto(531.53612318,50.4935952)(531.57612314,50.41859527)(531.60613037,50.30859619)
\curveto(531.64612307,50.20859548)(531.65612306,50.10859558)(531.63613037,50.00859619)
\lineto(531.63613037,49.87359619)
\curveto(531.6161231,49.80359589)(531.59612312,49.74359595)(531.57613037,49.69359619)
\curveto(531.55612316,49.64359605)(531.5211232,49.60359609)(531.47113037,49.57359619)
\curveto(531.4211233,49.53359616)(531.35112337,49.51359618)(531.26113037,49.51359619)
\lineto(530.99113037,49.51359619)
\lineto(530.09113037,49.51359619)
\lineto(526.58113037,49.51359619)
\lineto(525.51613037,49.51359619)
\curveto(525.43612928,49.51359618)(525.34612937,49.50859618)(525.24613037,49.49859619)
\curveto(525.14612957,49.49859619)(525.06112966,49.50859618)(524.99113037,49.52859619)
\curveto(524.78112994,49.59859609)(524.71613,49.77859591)(524.79613037,50.06859619)
\curveto(524.80612991,50.10859558)(524.80612991,50.14359555)(524.79613037,50.17359619)
\curveto(524.79612992,50.21359548)(524.80612991,50.25859543)(524.82613037,50.30859619)
\curveto(524.84612987,50.3885953)(524.86612985,50.47359522)(524.88613037,50.56359619)
\curveto(524.90612981,50.65359504)(524.93112979,50.73859495)(524.96113037,50.81859619)
\curveto(525.1211296,51.30859438)(525.3211294,51.72359397)(525.56113037,52.06359619)
\curveto(525.74112898,52.31359338)(525.94612877,52.53859315)(526.17613037,52.73859619)
\curveto(526.40612831,52.94859274)(526.64612807,53.14359255)(526.89613037,53.32359619)
\curveto(527.15612756,53.50359219)(527.4211273,53.67359202)(527.69113037,53.83359619)
\curveto(527.97112675,54.00359169)(528.24112648,54.17859151)(528.50113037,54.35859619)
\curveto(528.61112611,54.43859125)(528.716126,54.51359118)(528.81613037,54.58359619)
\curveto(528.92612579,54.65359104)(529.03612568,54.72859096)(529.14613037,54.80859619)
\curveto(529.18612553,54.83859085)(529.2211255,54.86859082)(529.25113037,54.89859619)
\curveto(529.29112543,54.93859075)(529.33112539,54.96859072)(529.37113037,54.98859619)
\curveto(529.51112521,55.09859059)(529.63612508,55.22359047)(529.74613037,55.36359619)
\curveto(529.76612495,55.3935903)(529.79112493,55.41859027)(529.82113037,55.43859619)
\curveto(529.85112487,55.46859022)(529.87612484,55.49859019)(529.89613037,55.52859619)
\curveto(529.97612474,55.62859006)(530.04112468,55.72858996)(530.09113037,55.82859619)
\curveto(530.15112457,55.92858976)(530.20612451,56.03858965)(530.25613037,56.15859619)
\curveto(530.28612443,56.22858946)(530.30612441,56.30358939)(530.31613037,56.38359619)
\lineto(530.37613037,56.62359619)
\lineto(530.37613037,56.71359619)
\curveto(530.38612433,56.74358895)(530.39112433,56.77358892)(530.39113037,56.80359619)
\curveto(530.41112431,56.87358882)(530.4161243,56.96858872)(530.40613037,57.08859619)
\curveto(530.40612431,57.21858847)(530.39612432,57.31858837)(530.37613037,57.38859619)
\curveto(530.35612436,57.46858822)(530.33612438,57.54358815)(530.31613037,57.61359619)
\curveto(530.30612441,57.693588)(530.28612443,57.77358792)(530.25613037,57.85359619)
\curveto(530.14612457,58.0935876)(529.99612472,58.2935874)(529.80613037,58.45359619)
\curveto(529.62612509,58.62358707)(529.40612531,58.76358693)(529.14613037,58.87359619)
\curveto(529.07612564,58.8935868)(529.00612571,58.90858678)(528.93613037,58.91859619)
\curveto(528.86612585,58.93858675)(528.79112593,58.95858673)(528.71113037,58.97859619)
\curveto(528.63112609,58.99858669)(528.5211262,59.00858668)(528.38113037,59.00859619)
\curveto(528.25112647,59.00858668)(528.14612657,58.99858669)(528.06613037,58.97859619)
\curveto(528.00612671,58.96858672)(527.95112677,58.96358673)(527.90113037,58.96359619)
\curveto(527.85112687,58.96358673)(527.80112692,58.95358674)(527.75113037,58.93359619)
\curveto(527.65112707,58.8935868)(527.55612716,58.85358684)(527.46613037,58.81359619)
\curveto(527.38612733,58.77358692)(527.30612741,58.72858696)(527.22613037,58.67859619)
\curveto(527.19612752,58.65858703)(527.16612755,58.63358706)(527.13613037,58.60359619)
\curveto(527.1161276,58.57358712)(527.09112763,58.54858714)(527.06113037,58.52859619)
\lineto(526.98613037,58.45359619)
\curveto(526.95612776,58.43358726)(526.93112779,58.41358728)(526.91113037,58.39359619)
\lineto(526.76113037,58.18359619)
\curveto(526.721128,58.12358757)(526.67612804,58.05858763)(526.62613037,57.98859619)
\curveto(526.56612815,57.89858779)(526.5161282,57.7935879)(526.47613037,57.67359619)
\curveto(526.44612827,57.56358813)(526.41112831,57.45358824)(526.37113037,57.34359619)
\curveto(526.33112839,57.23358846)(526.30612841,57.0885886)(526.29613037,56.90859619)
\curveto(526.28612843,56.73858895)(526.25612846,56.61358908)(526.20613037,56.53359619)
\curveto(526.15612856,56.45358924)(526.08112864,56.40858928)(525.98113037,56.39859619)
\curveto(525.88112884,56.3885893)(525.77112895,56.38358931)(525.65113037,56.38359619)
\curveto(525.61112911,56.38358931)(525.57112915,56.37858931)(525.53113037,56.36859619)
\curveto(525.49112923,56.36858932)(525.45612926,56.37358932)(525.42613037,56.38359619)
\curveto(525.37612934,56.40358929)(525.32612939,56.41358928)(525.27613037,56.41359619)
\curveto(525.23612948,56.41358928)(525.19612952,56.42358927)(525.15613037,56.44359619)
\curveto(525.06612965,56.50358919)(525.0211297,56.63858905)(525.02113037,56.84859619)
\lineto(525.02113037,56.96859619)
\curveto(525.03112969,57.02858866)(525.03612968,57.0885886)(525.03613037,57.14859619)
\curveto(525.04612967,57.21858847)(525.05612966,57.28358841)(525.06613037,57.34359619)
\curveto(525.08612963,57.45358824)(525.10612961,57.55358814)(525.12613037,57.64359619)
\curveto(525.14612957,57.74358795)(525.17612954,57.83858785)(525.21613037,57.92859619)
\curveto(525.23612948,57.99858769)(525.25612946,58.05858763)(525.27613037,58.10859619)
\lineto(525.33613037,58.28859619)
\curveto(525.45612926,58.54858714)(525.61112911,58.7935869)(525.80113037,59.02359619)
\curveto(526.00112872,59.25358644)(526.2161285,59.43858625)(526.44613037,59.57859619)
\curveto(526.55612816,59.65858603)(526.67112805,59.72358597)(526.79113037,59.77359619)
\lineto(527.18113037,59.92359619)
\curveto(527.29112743,59.97358572)(527.40612731,60.00358569)(527.52613037,60.01359619)
\curveto(527.64612707,60.03358566)(527.77112695,60.05858563)(527.90113037,60.08859619)
\curveto(527.97112675,60.0885856)(528.03612668,60.0885856)(528.09613037,60.08859619)
\curveto(528.15612656,60.09858559)(528.2211265,60.10858558)(528.29113037,60.11859619)
}
}
{
\newrgbcolor{curcolor}{0 0 0}
\pscustom[linestyle=none,fillstyle=solid,fillcolor=curcolor]
{
\newpath
\moveto(534.34573975,51.14859619)
\lineto(534.64573975,51.14859619)
\curveto(534.75573769,51.15859453)(534.86073758,51.15859453)(534.96073975,51.14859619)
\curveto(535.07073737,51.14859454)(535.17073727,51.13859455)(535.26073975,51.11859619)
\curveto(535.35073709,51.10859458)(535.42073702,51.08359461)(535.47073975,51.04359619)
\curveto(535.49073695,51.02359467)(535.50573694,50.9935947)(535.51573975,50.95359619)
\curveto(535.53573691,50.91359478)(535.55573689,50.86859482)(535.57573975,50.81859619)
\lineto(535.57573975,50.74359619)
\curveto(535.58573686,50.693595)(535.58573686,50.63859505)(535.57573975,50.57859619)
\lineto(535.57573975,50.42859619)
\lineto(535.57573975,49.94859619)
\curveto(535.57573687,49.77859591)(535.53573691,49.65859603)(535.45573975,49.58859619)
\curveto(535.38573706,49.53859615)(535.29573715,49.51359618)(535.18573975,49.51359619)
\lineto(534.85573975,49.51359619)
\lineto(534.40573975,49.51359619)
\curveto(534.25573819,49.51359618)(534.1407383,49.54359615)(534.06073975,49.60359619)
\curveto(534.02073842,49.63359606)(533.99073845,49.68359601)(533.97073975,49.75359619)
\curveto(533.95073849,49.83359586)(533.93573851,49.91859577)(533.92573975,50.00859619)
\lineto(533.92573975,50.29359619)
\curveto(533.93573851,50.3935953)(533.9407385,50.47859521)(533.94073975,50.54859619)
\lineto(533.94073975,50.74359619)
\curveto(533.9407385,50.80359489)(533.95073849,50.85859483)(533.97073975,50.90859619)
\curveto(534.01073843,51.01859467)(534.08073836,51.0885946)(534.18073975,51.11859619)
\curveto(534.21073823,51.11859457)(534.26573818,51.12859456)(534.34573975,51.14859619)
}
}
{
\newrgbcolor{curcolor}{0 0 0}
\pscustom[linestyle=none,fillstyle=solid,fillcolor=curcolor]
{
\newpath
\moveto(544.550896,53.83359619)
\curveto(544.58088827,53.71359198)(544.60588825,53.57359212)(544.625896,53.41359619)
\curveto(544.64588821,53.25359244)(544.6558882,53.0885926)(544.655896,52.91859619)
\curveto(544.6558882,52.74859294)(544.64588821,52.58359311)(544.625896,52.42359619)
\curveto(544.60588825,52.26359343)(544.58088827,52.12359357)(544.550896,52.00359619)
\curveto(544.51088834,51.86359383)(544.47588838,51.73859395)(544.445896,51.62859619)
\curveto(544.41588844,51.51859417)(544.37588848,51.40859428)(544.325896,51.29859619)
\curveto(544.0558888,50.65859503)(543.64088921,50.17359552)(543.080896,49.84359619)
\curveto(543.00088985,49.78359591)(542.91588994,49.73359596)(542.825896,49.69359619)
\curveto(542.73589012,49.66359603)(542.63589022,49.62859606)(542.525896,49.58859619)
\curveto(542.41589044,49.53859615)(542.29589056,49.50359619)(542.165896,49.48359619)
\curveto(542.04589081,49.45359624)(541.91589094,49.42359627)(541.775896,49.39359619)
\curveto(541.71589114,49.37359632)(541.6558912,49.36859632)(541.595896,49.37859619)
\curveto(541.54589131,49.3885963)(541.48589137,49.38359631)(541.415896,49.36359619)
\curveto(541.39589146,49.35359634)(541.37089148,49.35359634)(541.340896,49.36359619)
\curveto(541.31089154,49.36359633)(541.28589157,49.35859633)(541.265896,49.34859619)
\lineto(541.115896,49.34859619)
\curveto(541.04589181,49.33859635)(540.99589186,49.33859635)(540.965896,49.34859619)
\curveto(540.92589193,49.35859633)(540.88089197,49.36359633)(540.830896,49.36359619)
\curveto(540.79089206,49.35359634)(540.7508921,49.35359634)(540.710896,49.36359619)
\curveto(540.62089223,49.38359631)(540.53089232,49.39859629)(540.440896,49.40859619)
\curveto(540.3508925,49.40859628)(540.26089259,49.41859627)(540.170896,49.43859619)
\curveto(540.08089277,49.46859622)(539.99089286,49.4935962)(539.900896,49.51359619)
\curveto(539.81089304,49.53359616)(539.72589313,49.56359613)(539.645896,49.60359619)
\curveto(539.40589345,49.71359598)(539.18089367,49.84359585)(538.970896,49.99359619)
\curveto(538.76089409,50.15359554)(538.58089427,50.33359536)(538.430896,50.53359619)
\curveto(538.31089454,50.70359499)(538.20589465,50.87859481)(538.115896,51.05859619)
\curveto(538.02589483,51.23859445)(537.93589492,51.42859426)(537.845896,51.62859619)
\curveto(537.80589505,51.72859396)(537.77089508,51.82859386)(537.740896,51.92859619)
\curveto(537.72089513,52.03859365)(537.69589516,52.14859354)(537.665896,52.25859619)
\curveto(537.62589523,52.39859329)(537.60089525,52.53859315)(537.590896,52.67859619)
\curveto(537.58089527,52.81859287)(537.56089529,52.95859273)(537.530896,53.09859619)
\curveto(537.52089533,53.20859248)(537.51089534,53.30859238)(537.500896,53.39859619)
\curveto(537.50089535,53.49859219)(537.49089536,53.59859209)(537.470896,53.69859619)
\lineto(537.470896,53.78859619)
\curveto(537.48089537,53.81859187)(537.48089537,53.84359185)(537.470896,53.86359619)
\lineto(537.470896,54.07359619)
\curveto(537.4508954,54.13359156)(537.44089541,54.19859149)(537.440896,54.26859619)
\curveto(537.4508954,54.34859134)(537.4558954,54.42359127)(537.455896,54.49359619)
\lineto(537.455896,54.64359619)
\curveto(537.4558954,54.693591)(537.46089539,54.74359095)(537.470896,54.79359619)
\lineto(537.470896,55.16859619)
\curveto(537.48089537,55.19859049)(537.48089537,55.23359046)(537.470896,55.27359619)
\curveto(537.47089538,55.31359038)(537.47589538,55.35359034)(537.485896,55.39359619)
\curveto(537.50589535,55.50359019)(537.52089533,55.61359008)(537.530896,55.72359619)
\curveto(537.54089531,55.84358985)(537.5508953,55.95858973)(537.560896,56.06859619)
\curveto(537.60089525,56.21858947)(537.62589523,56.36358933)(537.635896,56.50359619)
\curveto(537.6558952,56.65358904)(537.68589517,56.79858889)(537.725896,56.93859619)
\curveto(537.81589504,57.23858845)(537.91089494,57.52358817)(538.010896,57.79359619)
\curveto(538.11089474,58.06358763)(538.23589462,58.31358738)(538.385896,58.54359619)
\curveto(538.58589427,58.86358683)(538.83089402,59.14358655)(539.120896,59.38359619)
\curveto(539.41089344,59.62358607)(539.7508931,59.80858588)(540.140896,59.93859619)
\curveto(540.2508926,59.97858571)(540.36089249,60.00358569)(540.470896,60.01359619)
\curveto(540.59089226,60.03358566)(540.71089214,60.05858563)(540.830896,60.08859619)
\curveto(540.90089195,60.09858559)(540.96589189,60.10358559)(541.025896,60.10359619)
\curveto(541.08589177,60.10358559)(541.1508917,60.10858558)(541.220896,60.11859619)
\curveto(541.92089093,60.13858555)(542.49589036,60.02358567)(542.945896,59.77359619)
\curveto(543.39588946,59.52358617)(543.74088911,59.17358652)(543.980896,58.72359619)
\curveto(544.09088876,58.4935872)(544.19088866,58.21858747)(544.280896,57.89859619)
\curveto(544.30088855,57.82858786)(544.30088855,57.75358794)(544.280896,57.67359619)
\curveto(544.27088858,57.60358809)(544.24588861,57.55358814)(544.205896,57.52359619)
\curveto(544.17588868,57.4935882)(544.11588874,57.46858822)(544.025896,57.44859619)
\curveto(543.93588892,57.43858825)(543.83588902,57.42858826)(543.725896,57.41859619)
\curveto(543.62588923,57.41858827)(543.52588933,57.42358827)(543.425896,57.43359619)
\curveto(543.33588952,57.44358825)(543.27088958,57.46358823)(543.230896,57.49359619)
\curveto(543.12088973,57.56358813)(543.04088981,57.67358802)(542.990896,57.82359619)
\curveto(542.9508899,57.97358772)(542.89588996,58.10358759)(542.825896,58.21359619)
\curveto(542.63589022,58.52358717)(542.3558905,58.75358694)(541.985896,58.90359619)
\curveto(541.91589094,58.93358676)(541.84089101,58.95358674)(541.760896,58.96359619)
\curveto(541.69089116,58.97358672)(541.61589124,58.9885867)(541.535896,59.00859619)
\curveto(541.48589137,59.01858667)(541.41589144,59.02358667)(541.325896,59.02359619)
\curveto(541.24589161,59.02358667)(541.18089167,59.01858667)(541.130896,59.00859619)
\curveto(541.09089176,58.9885867)(541.0558918,58.98358671)(541.025896,58.99359619)
\curveto(540.99589186,59.00358669)(540.96089189,59.00358669)(540.920896,58.99359619)
\lineto(540.680896,58.93359619)
\curveto(540.61089224,58.91358678)(540.54089231,58.8885868)(540.470896,58.85859619)
\curveto(540.09089276,58.69858699)(539.80089305,58.4885872)(539.600896,58.22859619)
\curveto(539.41089344,57.96858772)(539.23589362,57.65358804)(539.075896,57.28359619)
\curveto(539.04589381,57.20358849)(539.02089383,57.12358857)(539.000896,57.04359619)
\curveto(538.99089386,56.96358873)(538.97089388,56.88358881)(538.940896,56.80359619)
\curveto(538.91089394,56.693589)(538.88589397,56.57858911)(538.865896,56.45859619)
\curveto(538.855894,56.33858935)(538.83589402,56.21858947)(538.805896,56.09859619)
\curveto(538.78589407,56.04858964)(538.77589408,55.99858969)(538.775896,55.94859619)
\curveto(538.78589407,55.89858979)(538.78089407,55.84858984)(538.760896,55.79859619)
\curveto(538.7508941,55.73858995)(538.7508941,55.65859003)(538.760896,55.55859619)
\curveto(538.77089408,55.46859022)(538.78589407,55.41359028)(538.805896,55.39359619)
\curveto(538.82589403,55.35359034)(538.855894,55.33359036)(538.895896,55.33359619)
\curveto(538.94589391,55.33359036)(538.99089386,55.34359035)(539.030896,55.36359619)
\curveto(539.10089375,55.40359029)(539.16089369,55.44859024)(539.210896,55.49859619)
\curveto(539.26089359,55.54859014)(539.32089353,55.59859009)(539.390896,55.64859619)
\lineto(539.450896,55.70859619)
\curveto(539.48089337,55.73858995)(539.51089334,55.76358993)(539.540896,55.78359619)
\curveto(539.77089308,55.94358975)(540.04589281,56.07858961)(540.365896,56.18859619)
\curveto(540.43589242,56.20858948)(540.50589235,56.22358947)(540.575896,56.23359619)
\curveto(540.64589221,56.24358945)(540.72089213,56.25858943)(540.800896,56.27859619)
\curveto(540.84089201,56.27858941)(540.87589198,56.28358941)(540.905896,56.29359619)
\curveto(540.93589192,56.30358939)(540.97089188,56.30358939)(541.010896,56.29359619)
\curveto(541.06089179,56.2935894)(541.10089175,56.30358939)(541.130896,56.32359619)
\lineto(541.295896,56.32359619)
\lineto(541.385896,56.32359619)
\curveto(541.43589142,56.33358936)(541.47589138,56.33358936)(541.505896,56.32359619)
\curveto(541.5558913,56.31358938)(541.60589125,56.30858938)(541.655896,56.30859619)
\curveto(541.71589114,56.31858937)(541.77089108,56.31858937)(541.820896,56.30859619)
\curveto(541.93089092,56.27858941)(542.03589082,56.25858943)(542.135896,56.24859619)
\curveto(542.24589061,56.23858945)(542.3508905,56.21358948)(542.450896,56.17359619)
\curveto(542.87088998,56.03358966)(543.21588964,55.84858984)(543.485896,55.61859619)
\curveto(543.7558891,55.39859029)(543.99588886,55.11359058)(544.205896,54.76359619)
\curveto(544.28588857,54.62359107)(544.3508885,54.47359122)(544.400896,54.31359619)
\curveto(544.4508884,54.16359153)(544.50088835,54.00359169)(544.550896,53.83359619)
\moveto(543.305896,52.52859619)
\curveto(543.31588954,52.57859311)(543.32088953,52.62359307)(543.320896,52.66359619)
\lineto(543.320896,52.81359619)
\curveto(543.32088953,53.12359257)(543.28088957,53.40859228)(543.200896,53.66859619)
\curveto(543.18088967,53.72859196)(543.16088969,53.78359191)(543.140896,53.83359619)
\curveto(543.13088972,53.8935918)(543.11588974,53.94859174)(543.095896,53.99859619)
\curveto(542.87588998,54.4885912)(542.53089032,54.83859085)(542.060896,55.04859619)
\curveto(541.98089087,55.07859061)(541.90089095,55.10359059)(541.820896,55.12359619)
\lineto(541.580896,55.18359619)
\curveto(541.50089135,55.20359049)(541.41089144,55.21359048)(541.310896,55.21359619)
\lineto(540.995896,55.21359619)
\curveto(540.97589188,55.1935905)(540.93589192,55.18359051)(540.875896,55.18359619)
\curveto(540.82589203,55.1935905)(540.78089207,55.1935905)(540.740896,55.18359619)
\lineto(540.500896,55.12359619)
\curveto(540.43089242,55.11359058)(540.36089249,55.0935906)(540.290896,55.06359619)
\curveto(539.69089316,54.80359089)(539.28589357,54.33859135)(539.075896,53.66859619)
\curveto(539.04589381,53.5885921)(539.02589383,53.50859218)(539.015896,53.42859619)
\curveto(539.00589385,53.34859234)(538.99089386,53.26359243)(538.970896,53.17359619)
\lineto(538.970896,53.02359619)
\curveto(538.96089389,52.98359271)(538.9558939,52.91359278)(538.955896,52.81359619)
\curveto(538.9558939,52.58359311)(538.97589388,52.3885933)(539.015896,52.22859619)
\curveto(539.03589382,52.15859353)(539.0508938,52.0935936)(539.060896,52.03359619)
\curveto(539.07089378,51.97359372)(539.09089376,51.90859378)(539.120896,51.83859619)
\curveto(539.23089362,51.55859413)(539.37589348,51.31359438)(539.555896,51.10359619)
\curveto(539.73589312,50.90359479)(539.97089288,50.74359495)(540.260896,50.62359619)
\lineto(540.500896,50.53359619)
\lineto(540.740896,50.47359619)
\curveto(540.79089206,50.45359524)(540.83089202,50.44859524)(540.860896,50.45859619)
\curveto(540.90089195,50.46859522)(540.94589191,50.46359523)(540.995896,50.44359619)
\curveto(541.02589183,50.43359526)(541.08089177,50.42859526)(541.160896,50.42859619)
\curveto(541.24089161,50.42859526)(541.30089155,50.43359526)(541.340896,50.44359619)
\curveto(541.4508914,50.46359523)(541.5558913,50.47859521)(541.655896,50.48859619)
\curveto(541.7558911,50.49859519)(541.850891,50.52859516)(541.940896,50.57859619)
\curveto(542.47089038,50.77859491)(542.86088999,51.15359454)(543.110896,51.70359619)
\curveto(543.1508897,51.80359389)(543.18088967,51.90859378)(543.200896,52.01859619)
\lineto(543.290896,52.34859619)
\curveto(543.29088956,52.42859326)(543.29588956,52.4885932)(543.305896,52.52859619)
}
}
{
\newrgbcolor{curcolor}{0 0 0}
\pscustom[linestyle=none,fillstyle=solid,fillcolor=curcolor]
{
\newpath
\moveto(555.70550537,58.03359619)
\curveto(555.50549507,57.74358795)(555.29549528,57.45858823)(555.07550537,57.17859619)
\curveto(554.86549571,56.89858879)(554.66049592,56.61358908)(554.46050537,56.32359619)
\curveto(553.86049672,55.47359022)(553.25549732,54.63359106)(552.64550537,53.80359619)
\curveto(552.03549854,52.98359271)(551.43049915,52.14859354)(550.83050537,51.29859619)
\lineto(550.32050537,50.57859619)
\lineto(549.81050537,49.88859619)
\curveto(549.73050085,49.77859591)(549.65050093,49.66359603)(549.57050537,49.54359619)
\curveto(549.49050109,49.42359627)(549.39550118,49.32859636)(549.28550537,49.25859619)
\curveto(549.24550133,49.23859645)(549.1805014,49.22359647)(549.09050537,49.21359619)
\curveto(549.01050157,49.1935965)(548.92050166,49.18359651)(548.82050537,49.18359619)
\curveto(548.72050186,49.18359651)(548.62550195,49.1885965)(548.53550537,49.19859619)
\curveto(548.45550212,49.20859648)(548.39550218,49.22859646)(548.35550537,49.25859619)
\curveto(548.32550225,49.27859641)(548.30050228,49.31359638)(548.28050537,49.36359619)
\curveto(548.27050231,49.40359629)(548.2755023,49.44859624)(548.29550537,49.49859619)
\curveto(548.33550224,49.57859611)(548.3805022,49.65359604)(548.43050537,49.72359619)
\curveto(548.49050209,49.80359589)(548.54550203,49.88359581)(548.59550537,49.96359619)
\curveto(548.83550174,50.30359539)(549.0805015,50.63859505)(549.33050537,50.96859619)
\curveto(549.580501,51.29859439)(549.82050076,51.63359406)(550.05050537,51.97359619)
\curveto(550.21050037,52.1935935)(550.37050021,52.40859328)(550.53050537,52.61859619)
\curveto(550.69049989,52.82859286)(550.85049973,53.04359265)(551.01050537,53.26359619)
\curveto(551.37049921,53.78359191)(551.73549884,54.2935914)(552.10550537,54.79359619)
\curveto(552.4754981,55.2935904)(552.84549773,55.80358989)(553.21550537,56.32359619)
\curveto(553.35549722,56.52358917)(553.49549708,56.71858897)(553.63550537,56.90859619)
\curveto(553.78549679,57.09858859)(553.93049665,57.2935884)(554.07050537,57.49359619)
\curveto(554.2804963,57.7935879)(554.49549608,58.0935876)(554.71550537,58.39359619)
\lineto(555.37550537,59.29359619)
\lineto(555.55550537,59.56359619)
\lineto(555.76550537,59.83359619)
\lineto(555.88550537,60.01359619)
\curveto(555.93549464,60.07358562)(555.98549459,60.12858556)(556.03550537,60.17859619)
\curveto(556.10549447,60.22858546)(556.1804944,60.26358543)(556.26050537,60.28359619)
\curveto(556.2804943,60.2935854)(556.30549427,60.2935854)(556.33550537,60.28359619)
\curveto(556.3754942,60.28358541)(556.40549417,60.2935854)(556.42550537,60.31359619)
\curveto(556.54549403,60.31358538)(556.6804939,60.30858538)(556.83050537,60.29859619)
\curveto(556.9804936,60.29858539)(557.07049351,60.25358544)(557.10050537,60.16359619)
\curveto(557.12049346,60.13358556)(557.12549345,60.09858559)(557.11550537,60.05859619)
\curveto(557.10549347,60.01858567)(557.09049349,59.9885857)(557.07050537,59.96859619)
\curveto(557.03049355,59.8885858)(556.99049359,59.81858587)(556.95050537,59.75859619)
\curveto(556.91049367,59.69858599)(556.86549371,59.63858605)(556.81550537,59.57859619)
\lineto(556.24550537,58.79859619)
\curveto(556.06549451,58.54858714)(555.88549469,58.2935874)(555.70550537,58.03359619)
\moveto(548.85050537,54.13359619)
\curveto(548.80050178,54.15359154)(548.75050183,54.15859153)(548.70050537,54.14859619)
\curveto(548.65050193,54.13859155)(548.60050198,54.14359155)(548.55050537,54.16359619)
\curveto(548.44050214,54.18359151)(548.33550224,54.20359149)(548.23550537,54.22359619)
\curveto(548.14550243,54.25359144)(548.05050253,54.2935914)(547.95050537,54.34359619)
\curveto(547.62050296,54.48359121)(547.36550321,54.67859101)(547.18550537,54.92859619)
\curveto(547.00550357,55.1885905)(546.86050372,55.49859019)(546.75050537,55.85859619)
\curveto(546.72050386,55.93858975)(546.70050388,56.01858967)(546.69050537,56.09859619)
\curveto(546.6805039,56.1885895)(546.66550391,56.27358942)(546.64550537,56.35359619)
\curveto(546.63550394,56.40358929)(546.63050395,56.46858922)(546.63050537,56.54859619)
\curveto(546.62050396,56.57858911)(546.61550396,56.60858908)(546.61550537,56.63859619)
\curveto(546.61550396,56.67858901)(546.61050397,56.71358898)(546.60050537,56.74359619)
\lineto(546.60050537,56.89359619)
\curveto(546.59050399,56.94358875)(546.58550399,57.00358869)(546.58550537,57.07359619)
\curveto(546.58550399,57.15358854)(546.59050399,57.21858847)(546.60050537,57.26859619)
\lineto(546.60050537,57.43359619)
\curveto(546.62050396,57.48358821)(546.62550395,57.52858816)(546.61550537,57.56859619)
\curveto(546.61550396,57.61858807)(546.62050396,57.66358803)(546.63050537,57.70359619)
\curveto(546.64050394,57.74358795)(546.64550393,57.77858791)(546.64550537,57.80859619)
\curveto(546.64550393,57.84858784)(546.65050393,57.8885878)(546.66050537,57.92859619)
\curveto(546.69050389,58.03858765)(546.71050387,58.14858754)(546.72050537,58.25859619)
\curveto(546.74050384,58.37858731)(546.7755038,58.4935872)(546.82550537,58.60359619)
\curveto(546.96550361,58.94358675)(547.12550345,59.21858647)(547.30550537,59.42859619)
\curveto(547.49550308,59.64858604)(547.76550281,59.82858586)(548.11550537,59.96859619)
\curveto(548.19550238,59.99858569)(548.2805023,60.01858567)(548.37050537,60.02859619)
\curveto(548.46050212,60.04858564)(548.55550202,60.06858562)(548.65550537,60.08859619)
\curveto(548.68550189,60.09858559)(548.74050184,60.09858559)(548.82050537,60.08859619)
\curveto(548.90050168,60.0885856)(548.95050163,60.09858559)(548.97050537,60.11859619)
\curveto(549.53050105,60.12858556)(549.9805006,60.01858567)(550.32050537,59.78859619)
\curveto(550.67049991,59.55858613)(550.93049965,59.25358644)(551.10050537,58.87359619)
\curveto(551.14049944,58.78358691)(551.1754994,58.688587)(551.20550537,58.58859619)
\curveto(551.23549934,58.4885872)(551.26049932,58.3885873)(551.28050537,58.28859619)
\curveto(551.30049928,58.25858743)(551.30549927,58.22858746)(551.29550537,58.19859619)
\curveto(551.29549928,58.16858752)(551.30049928,58.13858755)(551.31050537,58.10859619)
\curveto(551.34049924,57.99858769)(551.36049922,57.87358782)(551.37050537,57.73359619)
\curveto(551.3804992,57.60358809)(551.39049919,57.46858822)(551.40050537,57.32859619)
\lineto(551.40050537,57.16359619)
\curveto(551.41049917,57.10358859)(551.41049917,57.04858864)(551.40050537,56.99859619)
\curveto(551.39049919,56.94858874)(551.38549919,56.89858879)(551.38550537,56.84859619)
\lineto(551.38550537,56.71359619)
\curveto(551.3754992,56.67358902)(551.37049921,56.63358906)(551.37050537,56.59359619)
\curveto(551.3804992,56.55358914)(551.3754992,56.50858918)(551.35550537,56.45859619)
\curveto(551.33549924,56.34858934)(551.31549926,56.24358945)(551.29550537,56.14359619)
\curveto(551.28549929,56.04358965)(551.26549931,55.94358975)(551.23550537,55.84359619)
\curveto(551.10549947,55.48359021)(550.94049964,55.16859052)(550.74050537,54.89859619)
\curveto(550.54050004,54.62859106)(550.26550031,54.42359127)(549.91550537,54.28359619)
\curveto(549.83550074,54.25359144)(549.75050083,54.22859146)(549.66050537,54.20859619)
\lineto(549.39050537,54.14859619)
\curveto(549.34050124,54.13859155)(549.29550128,54.13359156)(549.25550537,54.13359619)
\curveto(549.21550136,54.14359155)(549.1755014,54.14359155)(549.13550537,54.13359619)
\curveto(549.03550154,54.11359158)(548.94050164,54.11359158)(548.85050537,54.13359619)
\moveto(548.01050537,55.52859619)
\curveto(548.05050253,55.45859023)(548.09050249,55.3935903)(548.13050537,55.33359619)
\curveto(548.17050241,55.28359041)(548.22050236,55.23359046)(548.28050537,55.18359619)
\lineto(548.43050537,55.06359619)
\curveto(548.49050209,55.03359066)(548.55550202,55.00859068)(548.62550537,54.98859619)
\curveto(548.66550191,54.96859072)(548.70050188,54.95859073)(548.73050537,54.95859619)
\curveto(548.77050181,54.96859072)(548.81050177,54.96359073)(548.85050537,54.94359619)
\curveto(548.8805017,54.94359075)(548.92050166,54.93859075)(548.97050537,54.92859619)
\curveto(549.02050156,54.92859076)(549.06050152,54.93359076)(549.09050537,54.94359619)
\lineto(549.31550537,54.98859619)
\curveto(549.56550101,55.06859062)(549.75050083,55.1935905)(549.87050537,55.36359619)
\curveto(549.95050063,55.46359023)(550.02050056,55.5935901)(550.08050537,55.75359619)
\curveto(550.16050042,55.93358976)(550.22050036,56.15858953)(550.26050537,56.42859619)
\curveto(550.30050028,56.70858898)(550.31550026,56.9885887)(550.30550537,57.26859619)
\curveto(550.29550028,57.55858813)(550.26550031,57.83358786)(550.21550537,58.09359619)
\curveto(550.16550041,58.35358734)(550.09050049,58.56358713)(549.99050537,58.72359619)
\curveto(549.87050071,58.92358677)(549.72050086,59.07358662)(549.54050537,59.17359619)
\curveto(549.46050112,59.22358647)(549.37050121,59.25358644)(549.27050537,59.26359619)
\curveto(549.17050141,59.28358641)(549.06550151,59.2935864)(548.95550537,59.29359619)
\curveto(548.93550164,59.28358641)(548.91050167,59.27858641)(548.88050537,59.27859619)
\curveto(548.86050172,59.2885864)(548.84050174,59.2885864)(548.82050537,59.27859619)
\curveto(548.77050181,59.26858642)(548.72550185,59.25858643)(548.68550537,59.24859619)
\curveto(548.64550193,59.24858644)(548.60550197,59.23858645)(548.56550537,59.21859619)
\curveto(548.38550219,59.13858655)(548.23550234,59.01858667)(548.11550537,58.85859619)
\curveto(548.00550257,58.69858699)(547.91550266,58.51858717)(547.84550537,58.31859619)
\curveto(547.78550279,58.12858756)(547.74050284,57.90358779)(547.71050537,57.64359619)
\curveto(547.69050289,57.38358831)(547.68550289,57.11858857)(547.69550537,56.84859619)
\curveto(547.70550287,56.5885891)(547.73550284,56.33858935)(547.78550537,56.09859619)
\curveto(547.84550273,55.86858982)(547.92050266,55.67859001)(548.01050537,55.52859619)
\moveto(558.81050537,52.54359619)
\curveto(558.82049176,52.4935932)(558.82549175,52.40359329)(558.82550537,52.27359619)
\curveto(558.82549175,52.14359355)(558.81549176,52.05359364)(558.79550537,52.00359619)
\curveto(558.7754918,51.95359374)(558.77049181,51.89859379)(558.78050537,51.83859619)
\curveto(558.79049179,51.7885939)(558.79049179,51.73859395)(558.78050537,51.68859619)
\curveto(558.74049184,51.54859414)(558.71049187,51.41359428)(558.69050537,51.28359619)
\curveto(558.6804919,51.15359454)(558.65049193,51.03359466)(558.60050537,50.92359619)
\curveto(558.46049212,50.57359512)(558.29549228,50.27859541)(558.10550537,50.03859619)
\curveto(557.91549266,49.80859588)(557.64549293,49.62359607)(557.29550537,49.48359619)
\curveto(557.21549336,49.45359624)(557.13049345,49.43359626)(557.04050537,49.42359619)
\curveto(556.95049363,49.40359629)(556.86549371,49.38359631)(556.78550537,49.36359619)
\curveto(556.73549384,49.35359634)(556.68549389,49.34859634)(556.63550537,49.34859619)
\curveto(556.58549399,49.34859634)(556.53549404,49.34359635)(556.48550537,49.33359619)
\curveto(556.45549412,49.32359637)(556.40549417,49.32359637)(556.33550537,49.33359619)
\curveto(556.26549431,49.33359636)(556.21549436,49.33859635)(556.18550537,49.34859619)
\curveto(556.12549445,49.36859632)(556.06549451,49.37859631)(556.00550537,49.37859619)
\curveto(555.95549462,49.36859632)(555.90549467,49.37359632)(555.85550537,49.39359619)
\curveto(555.76549481,49.41359628)(555.6754949,49.43859625)(555.58550537,49.46859619)
\curveto(555.50549507,49.4885962)(555.42549515,49.51859617)(555.34550537,49.55859619)
\curveto(555.02549555,49.69859599)(554.7754958,49.8935958)(554.59550537,50.14359619)
\curveto(554.41549616,50.40359529)(554.26549631,50.70859498)(554.14550537,51.05859619)
\curveto(554.12549645,51.13859455)(554.11049647,51.22359447)(554.10050537,51.31359619)
\curveto(554.09049649,51.40359429)(554.0754965,51.4885942)(554.05550537,51.56859619)
\curveto(554.04549653,51.59859409)(554.04049654,51.62859406)(554.04050537,51.65859619)
\lineto(554.04050537,51.76359619)
\curveto(554.02049656,51.84359385)(554.01049657,51.92359377)(554.01050537,52.00359619)
\lineto(554.01050537,52.13859619)
\curveto(553.99049659,52.23859345)(553.99049659,52.33859335)(554.01050537,52.43859619)
\lineto(554.01050537,52.61859619)
\curveto(554.02049656,52.66859302)(554.02549655,52.71359298)(554.02550537,52.75359619)
\curveto(554.02549655,52.80359289)(554.03049655,52.84859284)(554.04050537,52.88859619)
\curveto(554.05049653,52.92859276)(554.05549652,52.96359273)(554.05550537,52.99359619)
\curveto(554.05549652,53.03359266)(554.06049652,53.07359262)(554.07050537,53.11359619)
\lineto(554.13050537,53.44359619)
\curveto(554.15049643,53.56359213)(554.1804964,53.67359202)(554.22050537,53.77359619)
\curveto(554.36049622,54.10359159)(554.52049606,54.37859131)(554.70050537,54.59859619)
\curveto(554.89049569,54.82859086)(555.15049543,55.01359068)(555.48050537,55.15359619)
\curveto(555.56049502,55.1935905)(555.64549493,55.21859047)(555.73550537,55.22859619)
\lineto(556.03550537,55.28859619)
\lineto(556.17050537,55.28859619)
\curveto(556.22049436,55.29859039)(556.27049431,55.30359039)(556.32050537,55.30359619)
\curveto(556.89049369,55.32359037)(557.35049323,55.21859047)(557.70050537,54.98859619)
\curveto(558.06049252,54.76859092)(558.32549225,54.46859122)(558.49550537,54.08859619)
\curveto(558.54549203,53.9885917)(558.58549199,53.8885918)(558.61550537,53.78859619)
\curveto(558.64549193,53.688592)(558.6754919,53.58359211)(558.70550537,53.47359619)
\curveto(558.71549186,53.43359226)(558.72049186,53.39859229)(558.72050537,53.36859619)
\curveto(558.72049186,53.34859234)(558.72549185,53.31859237)(558.73550537,53.27859619)
\curveto(558.75549182,53.20859248)(558.76549181,53.13359256)(558.76550537,53.05359619)
\curveto(558.76549181,52.97359272)(558.7754918,52.8935928)(558.79550537,52.81359619)
\curveto(558.79549178,52.76359293)(558.79549178,52.71859297)(558.79550537,52.67859619)
\curveto(558.79549178,52.63859305)(558.80049178,52.5935931)(558.81050537,52.54359619)
\moveto(557.70050537,52.10859619)
\curveto(557.71049287,52.15859353)(557.71549286,52.23359346)(557.71550537,52.33359619)
\curveto(557.72549285,52.43359326)(557.72049286,52.50859318)(557.70050537,52.55859619)
\curveto(557.6804929,52.61859307)(557.6754929,52.67359302)(557.68550537,52.72359619)
\curveto(557.70549287,52.78359291)(557.70549287,52.84359285)(557.68550537,52.90359619)
\curveto(557.6754929,52.93359276)(557.67049291,52.96859272)(557.67050537,53.00859619)
\curveto(557.67049291,53.04859264)(557.66549291,53.0885926)(557.65550537,53.12859619)
\curveto(557.63549294,53.20859248)(557.61549296,53.28359241)(557.59550537,53.35359619)
\curveto(557.58549299,53.43359226)(557.57049301,53.51359218)(557.55050537,53.59359619)
\curveto(557.52049306,53.65359204)(557.49549308,53.71359198)(557.47550537,53.77359619)
\curveto(557.45549312,53.83359186)(557.42549315,53.8935918)(557.38550537,53.95359619)
\curveto(557.28549329,54.12359157)(557.15549342,54.25859143)(556.99550537,54.35859619)
\curveto(556.91549366,54.40859128)(556.82049376,54.44359125)(556.71050537,54.46359619)
\curveto(556.60049398,54.48359121)(556.4754941,54.4935912)(556.33550537,54.49359619)
\curveto(556.31549426,54.48359121)(556.29049429,54.47859121)(556.26050537,54.47859619)
\curveto(556.23049435,54.4885912)(556.20049438,54.4885912)(556.17050537,54.47859619)
\lineto(556.02050537,54.41859619)
\curveto(555.97049461,54.40859128)(555.92549465,54.3935913)(555.88550537,54.37359619)
\curveto(555.69549488,54.26359143)(555.55049503,54.11859157)(555.45050537,53.93859619)
\curveto(555.36049522,53.75859193)(555.2804953,53.55359214)(555.21050537,53.32359619)
\curveto(555.17049541,53.1935925)(555.15049543,53.05859263)(555.15050537,52.91859619)
\curveto(555.15049543,52.7885929)(555.14049544,52.64359305)(555.12050537,52.48359619)
\curveto(555.11049547,52.43359326)(555.10049548,52.37359332)(555.09050537,52.30359619)
\curveto(555.09049549,52.23359346)(555.10049548,52.17359352)(555.12050537,52.12359619)
\lineto(555.12050537,51.95859619)
\lineto(555.12050537,51.77859619)
\curveto(555.13049545,51.72859396)(555.14049544,51.67359402)(555.15050537,51.61359619)
\curveto(555.16049542,51.56359413)(555.16549541,51.50859418)(555.16550537,51.44859619)
\curveto(555.1754954,51.3885943)(555.19049539,51.33359436)(555.21050537,51.28359619)
\curveto(555.26049532,51.0935946)(555.32049526,50.91859477)(555.39050537,50.75859619)
\curveto(555.46049512,50.59859509)(555.56549501,50.46859522)(555.70550537,50.36859619)
\curveto(555.83549474,50.26859542)(555.9754946,50.19859549)(556.12550537,50.15859619)
\curveto(556.15549442,50.14859554)(556.1804944,50.14359555)(556.20050537,50.14359619)
\curveto(556.23049435,50.15359554)(556.26049432,50.15359554)(556.29050537,50.14359619)
\curveto(556.31049427,50.14359555)(556.34049424,50.13859555)(556.38050537,50.12859619)
\curveto(556.42049416,50.12859556)(556.45549412,50.13359556)(556.48550537,50.14359619)
\curveto(556.52549405,50.15359554)(556.56549401,50.15859553)(556.60550537,50.15859619)
\curveto(556.64549393,50.15859553)(556.68549389,50.16859552)(556.72550537,50.18859619)
\curveto(556.96549361,50.26859542)(557.16049342,50.40359529)(557.31050537,50.59359619)
\curveto(557.43049315,50.77359492)(557.52049306,50.97859471)(557.58050537,51.20859619)
\curveto(557.60049298,51.27859441)(557.61549296,51.34859434)(557.62550537,51.41859619)
\curveto(557.63549294,51.49859419)(557.65049293,51.57859411)(557.67050537,51.65859619)
\curveto(557.67049291,51.71859397)(557.6754929,51.76359393)(557.68550537,51.79359619)
\curveto(557.68549289,51.81359388)(557.68549289,51.83859385)(557.68550537,51.86859619)
\curveto(557.68549289,51.90859378)(557.69049289,51.93859375)(557.70050537,51.95859619)
\lineto(557.70050537,52.10859619)
}
}
\end{pspicture}

\caption{Gráficas circulares de las diferentes valoraciones clasificadas por
rol}
\label{usuarios_pie_2}
\end{figure}

\subsection{Contactos}
La tercera variable que se analizó, fue la relación de contactos de usuario,
en la figura \ref{contactos_matriz}, se presenta las matriz de adyacencias,
generada a partir de las relaciones (ya sean fuertes o débiles), entre usuarios
del sistema, en este únicamente puede apreciarse los usuarios que algún
contacto, y no así la totalidad de usuarios del sistema.

Considerando los indicadores de sociabilidad,  puede verse que los enlaces
fuertes son casi exclusividad propia de los desarrolladores, siendo entre los
otros roles predominantes los enlaces débiles. Puede verse también una sutil
relación entre los usuarios que establecieron el nombre de usuario en su perfil,
y los niveles de sociabilidad. Cuya interrelación, es motivo de seguimiento e
intención de demostración.

\begin{figure}
\centering
%LaTeX with PSTricks extensions
%%Creator: inkscape 0.48.5
%%Please note this file requires PSTricks extensions
\psset{xunit=.5pt,yunit=.5pt,runit=.5pt}
\begin{pspicture}(880,545)
{
\newrgbcolor{curcolor}{0.80000001 0.80000001 0.80000001}
\pscustom[linestyle=none,fillstyle=solid,fillcolor=curcolor]
{
\newpath
\moveto(102.59358215,527.46856428)
\lineto(135.89073563,527.46856428)
\lineto(135.89073563,510.44859434)
\lineto(102.59358215,510.44859434)
\closepath
}
}
{
\newrgbcolor{curcolor}{0.80000001 0.80000001 0.80000001}
\pscustom[linestyle=none,fillstyle=solid,fillcolor=curcolor]
{
\newpath
\moveto(102.59358215,510.47228742)
\lineto(135.89073563,510.47228742)
\lineto(135.89073563,493.45231748)
\lineto(102.59358215,493.45231748)
\closepath
}
}
{
\newrgbcolor{curcolor}{0.80000001 0.80000001 0.80000001}
\pscustom[linestyle=none,fillstyle=solid,fillcolor=curcolor]
{
\newpath
\moveto(102.59358215,493.47613264)
\lineto(135.89073563,493.47613264)
\lineto(135.89073563,476.4561627)
\lineto(102.59358215,476.4561627)
\closepath
}
}
{
\newrgbcolor{curcolor}{0.80000001 0.80000001 0.80000001}
\pscustom[linestyle=none,fillstyle=solid,fillcolor=curcolor]
{
\newpath
\moveto(169.12477112,527.46899152)
\lineto(202.42192459,527.46899152)
\lineto(202.42192459,510.44902158)
\lineto(169.12477112,510.44902158)
\closepath
}
}
{
\newrgbcolor{curcolor}{0.80000001 0.80000001 0.80000001}
\pscustom[linestyle=none,fillstyle=solid,fillcolor=curcolor]
{
\newpath
\moveto(169.12480164,510.44903303)
\lineto(202.42195511,510.44903303)
\lineto(202.42195511,493.42906309)
\lineto(169.12480164,493.42906309)
\closepath
}
}
{
\newrgbcolor{curcolor}{0.80000001 0.80000001 0.80000001}
\pscustom[linestyle=none,fillstyle=solid,fillcolor=curcolor]
{
\newpath
\moveto(102.59358215,476.47991682)
\lineto(135.89073563,476.47991682)
\lineto(135.89073563,459.45994688)
\lineto(102.59358215,459.45994688)
\closepath
}
}
{
\newrgbcolor{curcolor}{0.80000001 0.80000001 0.80000001}
\pscustom[linestyle=none,fillstyle=solid,fillcolor=curcolor]
{
\newpath
\moveto(102.59358215,459.483701)
\lineto(135.89073563,459.483701)
\lineto(135.89073563,442.46373106)
\lineto(102.59358215,442.46373106)
\closepath
}
}
{
\newrgbcolor{curcolor}{0.80000001 0.80000001 0.80000001}
\pscustom[linestyle=none,fillstyle=solid,fillcolor=curcolor]
{
\newpath
\moveto(102.59358215,442.48748518)
\lineto(135.89073563,442.48748518)
\lineto(135.89073563,425.46751524)
\lineto(102.59358215,425.46751524)
\closepath
}
}
{
\newrgbcolor{curcolor}{0.80000001 0.80000001 0.80000001}
\pscustom[linestyle=none,fillstyle=solid,fillcolor=curcolor]
{
\newpath
\moveto(102.59358215,425.49126936)
\lineto(135.89073563,425.49126936)
\lineto(135.89073563,408.47129942)
\lineto(102.59358215,408.47129942)
\closepath
}
}
{
\newrgbcolor{curcolor}{0.80000001 0.80000001 0.80000001}
\pscustom[linestyle=none,fillstyle=solid,fillcolor=curcolor]
{
\newpath
\moveto(102.59358215,408.49505354)
\lineto(135.89073563,408.49505354)
\lineto(135.89073563,391.47508359)
\lineto(102.59358215,391.47508359)
\closepath
}
}
{
\newrgbcolor{curcolor}{0.80000001 0.80000001 0.80000001}
\pscustom[linestyle=none,fillstyle=solid,fillcolor=curcolor]
{
\newpath
\moveto(102.59358215,391.49883771)
\lineto(135.89073563,391.49883771)
\lineto(135.89073563,374.47886777)
\lineto(102.59358215,374.47886777)
\closepath
}
}
{
\newrgbcolor{curcolor}{0.80000001 0.80000001 0.80000001}
\pscustom[linestyle=none,fillstyle=solid,fillcolor=curcolor]
{
\newpath
\moveto(102.59358215,374.50262189)
\lineto(135.89073563,374.50262189)
\lineto(135.89073563,357.48265195)
\lineto(102.59358215,357.48265195)
\closepath
}
}
{
\newrgbcolor{curcolor}{0.80000001 0.80000001 0.80000001}
\pscustom[linestyle=none,fillstyle=solid,fillcolor=curcolor]
{
\newpath
\moveto(102.59358215,357.50640607)
\lineto(135.89073563,357.50640607)
\lineto(135.89073563,340.48643613)
\lineto(102.59358215,340.48643613)
\closepath
}
}
{
\newrgbcolor{curcolor}{0.80000001 0.80000001 0.80000001}
\pscustom[linestyle=none,fillstyle=solid,fillcolor=curcolor]
{
\newpath
\moveto(102.59358215,340.51019025)
\lineto(135.89073563,340.51019025)
\lineto(135.89073563,323.49022031)
\lineto(102.59358215,323.49022031)
\closepath
}
}
{
\newrgbcolor{curcolor}{0.80000001 0.80000001 0.80000001}
\pscustom[linestyle=none,fillstyle=solid,fillcolor=curcolor]
{
\newpath
\moveto(102.59358215,323.51397443)
\lineto(135.89073563,323.51397443)
\lineto(135.89073563,306.49400449)
\lineto(102.59358215,306.49400449)
\closepath
}
}
{
\newrgbcolor{curcolor}{0.80000001 0.80000001 0.80000001}
\pscustom[linestyle=none,fillstyle=solid,fillcolor=curcolor]
{
\newpath
\moveto(102.59358215,306.51775861)
\lineto(135.89073563,306.51775861)
\lineto(135.89073563,289.49778867)
\lineto(102.59358215,289.49778867)
\closepath
}
}
{
\newrgbcolor{curcolor}{0.80000001 0.80000001 0.80000001}
\pscustom[linestyle=none,fillstyle=solid,fillcolor=curcolor]
{
\newpath
\moveto(135.89073181,527.46856428)
\lineto(169.18788528,527.46856428)
\lineto(169.18788528,510.44859434)
\lineto(135.89073181,510.44859434)
\closepath
}
}
{
\newrgbcolor{curcolor}{0.80000001 0.80000001 0.80000001}
\pscustom[linestyle=none,fillstyle=solid,fillcolor=curcolor]
{
\newpath
\moveto(135.89073181,510.47228742)
\lineto(169.18788528,510.47228742)
\lineto(169.18788528,493.45231748)
\lineto(135.89073181,493.45231748)
\closepath
}
}
{
\newrgbcolor{curcolor}{0.80000001 0.80000001 0.80000001}
\pscustom[linestyle=none,fillstyle=solid,fillcolor=curcolor]
{
\newpath
\moveto(135.89073181,493.47613264)
\lineto(169.18788528,493.47613264)
\lineto(169.18788528,476.4561627)
\lineto(135.89073181,476.4561627)
\closepath
}
}
{
\newrgbcolor{curcolor}{0.80000001 0.80000001 0.80000001}
\pscustom[linestyle=none,fillstyle=solid,fillcolor=curcolor]
{
\newpath
\moveto(135.89073181,476.47991682)
\lineto(169.18788528,476.47991682)
\lineto(169.18788528,459.45994688)
\lineto(135.89073181,459.45994688)
\closepath
}
}
{
\newrgbcolor{curcolor}{0.80000001 0.80000001 0.80000001}
\pscustom[linestyle=none,fillstyle=solid,fillcolor=curcolor]
{
\newpath
\moveto(135.89073181,459.483701)
\lineto(169.18788528,459.483701)
\lineto(169.18788528,442.46373106)
\lineto(135.89073181,442.46373106)
\closepath
}
}
{
\newrgbcolor{curcolor}{0.80000001 0.80000001 0.80000001}
\pscustom[linestyle=none,fillstyle=solid,fillcolor=curcolor]
{
\newpath
\moveto(135.89073181,442.48748518)
\lineto(169.18788528,442.48748518)
\lineto(169.18788528,425.46751524)
\lineto(135.89073181,425.46751524)
\closepath
}
}
{
\newrgbcolor{curcolor}{0.80000001 0.80000001 0.80000001}
\pscustom[linestyle=none,fillstyle=solid,fillcolor=curcolor]
{
\newpath
\moveto(135.89073181,425.49126936)
\lineto(169.18788528,425.49126936)
\lineto(169.18788528,408.47129942)
\lineto(135.89073181,408.47129942)
\closepath
}
}
{
\newrgbcolor{curcolor}{0.80000001 0.80000001 0.80000001}
\pscustom[linestyle=none,fillstyle=solid,fillcolor=curcolor]
{
\newpath
\moveto(135.89073181,408.49505354)
\lineto(169.18788528,408.49505354)
\lineto(169.18788528,391.47508359)
\lineto(135.89073181,391.47508359)
\closepath
}
}
{
\newrgbcolor{curcolor}{0.80000001 0.80000001 0.80000001}
\pscustom[linestyle=none,fillstyle=solid,fillcolor=curcolor]
{
\newpath
\moveto(135.89073181,391.49883771)
\lineto(169.18788528,391.49883771)
\lineto(169.18788528,374.47886777)
\lineto(135.89073181,374.47886777)
\closepath
}
}
{
\newrgbcolor{curcolor}{0.80000001 0.80000001 0.80000001}
\pscustom[linestyle=none,fillstyle=solid,fillcolor=curcolor]
{
\newpath
\moveto(135.89073181,374.50262189)
\lineto(169.18788528,374.50262189)
\lineto(169.18788528,357.48265195)
\lineto(135.89073181,357.48265195)
\closepath
}
}
{
\newrgbcolor{curcolor}{0.80000001 0.80000001 0.80000001}
\pscustom[linestyle=none,fillstyle=solid,fillcolor=curcolor]
{
\newpath
\moveto(135.89073181,357.50640607)
\lineto(169.18788528,357.50640607)
\lineto(169.18788528,340.48643613)
\lineto(135.89073181,340.48643613)
\closepath
}
}
{
\newrgbcolor{curcolor}{0.80000001 0.80000001 0.80000001}
\pscustom[linestyle=none,fillstyle=solid,fillcolor=curcolor]
{
\newpath
\moveto(135.89073181,340.51019025)
\lineto(169.18788528,340.51019025)
\lineto(169.18788528,323.49022031)
\lineto(135.89073181,323.49022031)
\closepath
}
}
{
\newrgbcolor{curcolor}{0.80000001 0.80000001 0.80000001}
\pscustom[linestyle=none,fillstyle=solid,fillcolor=curcolor]
{
\newpath
\moveto(135.89073181,323.51397443)
\lineto(169.18788528,323.51397443)
\lineto(169.18788528,306.49400449)
\lineto(135.89073181,306.49400449)
\closepath
}
}
{
\newrgbcolor{curcolor}{0.80000001 0.80000001 0.80000001}
\pscustom[linestyle=none,fillstyle=solid,fillcolor=curcolor]
{
\newpath
\moveto(135.89073181,306.51775861)
\lineto(169.18788528,306.51775861)
\lineto(169.18788528,289.49778867)
\lineto(135.89073181,289.49778867)
\closepath
}
}
{
\newrgbcolor{curcolor}{0.80000001 0.80000001 0.80000001}
\pscustom[linestyle=none,fillstyle=solid,fillcolor=curcolor]
{
\newpath
\moveto(135.89073181,289.46856428)
\lineto(169.18788528,289.46856428)
\lineto(169.18788528,272.44859434)
\lineto(135.89073181,272.44859434)
\closepath
}
}
{
\newrgbcolor{curcolor}{0.80000001 0.80000001 0.80000001}
\pscustom[linestyle=none,fillstyle=solid,fillcolor=curcolor]
{
\newpath
\moveto(135.89073181,272.47234846)
\lineto(169.18788528,272.47234846)
\lineto(169.18788528,255.45237852)
\lineto(135.89073181,255.45237852)
\closepath
}
}
{
\newrgbcolor{curcolor}{0.80000001 0.80000001 0.80000001}
\pscustom[linestyle=none,fillstyle=solid,fillcolor=curcolor]
{
\newpath
\moveto(135.89073181,255.47613264)
\lineto(169.18788528,255.47613264)
\lineto(169.18788528,238.4561627)
\lineto(135.89073181,238.4561627)
\closepath
}
}
{
\newrgbcolor{curcolor}{0.80000001 0.80000001 0.80000001}
\pscustom[linestyle=none,fillstyle=solid,fillcolor=curcolor]
{
\newpath
\moveto(135.89073181,238.48003889)
\lineto(169.18788528,238.48003889)
\lineto(169.18788528,221.46006895)
\lineto(135.89073181,221.46006895)
\closepath
}
}
{
\newrgbcolor{curcolor}{0.80000001 0.80000001 0.80000001}
\pscustom[linestyle=none,fillstyle=solid,fillcolor=curcolor]
{
\newpath
\moveto(135.89073181,221.48382307)
\lineto(169.18788528,221.48382307)
\lineto(169.18788528,204.46385313)
\lineto(135.89073181,204.46385313)
\closepath
}
}
{
\newrgbcolor{curcolor}{0.80000001 0.80000001 0.80000001}
\pscustom[linestyle=none,fillstyle=solid,fillcolor=curcolor]
{
\newpath
\moveto(135.89073181,204.48760725)
\lineto(169.18788528,204.48760725)
\lineto(169.18788528,187.46763731)
\lineto(135.89073181,187.46763731)
\closepath
}
}
{
\newrgbcolor{curcolor}{0.80000001 0.80000001 0.80000001}
\pscustom[linestyle=none,fillstyle=solid,fillcolor=curcolor]
{
\newpath
\moveto(135.89073181,187.49139143)
\lineto(169.18788528,187.49139143)
\lineto(169.18788528,170.47142149)
\lineto(135.89073181,170.47142149)
\closepath
}
}
{
\newrgbcolor{curcolor}{0.80000001 0.80000001 0.80000001}
\pscustom[linestyle=none,fillstyle=solid,fillcolor=curcolor]
{
\newpath
\moveto(135.89073181,170.49517561)
\lineto(169.18788528,170.49517561)
\lineto(169.18788528,153.47520567)
\lineto(135.89073181,153.47520567)
\closepath
}
}
{
\newrgbcolor{curcolor}{0.80000001 0.80000001 0.80000001}
\pscustom[linestyle=none,fillstyle=solid,fillcolor=curcolor]
{
\newpath
\moveto(135.89073181,153.49895979)
\lineto(169.18788528,153.49895979)
\lineto(169.18788528,136.47898984)
\lineto(135.89073181,136.47898984)
\closepath
}
}
{
\newrgbcolor{curcolor}{0.80000001 0.80000001 0.80000001}
\pscustom[linestyle=none,fillstyle=solid,fillcolor=curcolor]
{
\newpath
\moveto(135.89073181,136.50274396)
\lineto(169.18788528,136.50274396)
\lineto(169.18788528,119.48277402)
\lineto(135.89073181,119.48277402)
\closepath
}
}
{
\newrgbcolor{curcolor}{0.80000001 0.80000001 0.80000001}
\pscustom[linestyle=none,fillstyle=solid,fillcolor=curcolor]
{
\newpath
\moveto(135.89073181,119.50652814)
\lineto(169.18788528,119.50652814)
\lineto(169.18788528,102.4865582)
\lineto(135.89073181,102.4865582)
\closepath
}
}
{
\newrgbcolor{curcolor}{0.80000001 0.80000001 0.80000001}
\pscustom[linestyle=none,fillstyle=solid,fillcolor=curcolor]
{
\newpath
\moveto(135.89073181,102.51031232)
\lineto(169.18788528,102.51031232)
\lineto(169.18788528,85.49034238)
\lineto(135.89073181,85.49034238)
\closepath
}
}
{
\newrgbcolor{curcolor}{0.80000001 0.80000001 0.80000001}
\pscustom[linestyle=none,fillstyle=solid,fillcolor=curcolor]
{
\newpath
\moveto(135.89073181,68.51788068)
\lineto(169.18788528,68.51788068)
\lineto(169.18788528,51.49791074)
\lineto(135.89073181,51.49791074)
\closepath
}
}
{
\newrgbcolor{curcolor}{0.80000001 0.80000001 0.80000001}
\pscustom[linestyle=none,fillstyle=solid,fillcolor=curcolor]
{
\newpath
\moveto(169.12477112,425.46905256)
\lineto(202.42192459,425.46905256)
\lineto(202.42192459,408.44908262)
\lineto(169.12477112,408.44908262)
\closepath
}
}
{
\newrgbcolor{curcolor}{0.80000001 0.80000001 0.80000001}
\pscustom[linestyle=none,fillstyle=solid,fillcolor=curcolor]
{
\newpath
\moveto(169.12480164,408.44903303)
\lineto(202.42195511,408.44903303)
\lineto(202.42195511,391.42906309)
\lineto(169.12480164,391.42906309)
\closepath
}
}
{
\newrgbcolor{curcolor}{0.80000001 0.80000001 0.80000001}
\pscustom[linestyle=none,fillstyle=solid,fillcolor=curcolor]
{
\newpath
\moveto(169.12480164,374.50799299)
\lineto(202.42195511,374.50799299)
\lineto(202.42195511,357.48802305)
\lineto(169.12480164,357.48802305)
\closepath
}
}
{
\newrgbcolor{curcolor}{0.80000001 0.80000001 0.80000001}
\pscustom[linestyle=none,fillstyle=solid,fillcolor=curcolor]
{
\newpath
\moveto(169.39073181,255.00274396)
\lineto(202.68788528,255.00274396)
\lineto(202.68788528,237.98277402)
\lineto(169.39073181,237.98277402)
\closepath
}
}
{
\newrgbcolor{curcolor}{0.80000001 0.80000001 0.80000001}
\pscustom[linestyle=none,fillstyle=solid,fillcolor=curcolor]
{
\newpath
\moveto(169.39073181,238.00652814)
\lineto(202.68788528,238.00652814)
\lineto(202.68788528,220.9865582)
\lineto(169.39073181,220.9865582)
\closepath
}
}
{
\newrgbcolor{curcolor}{0.80000001 0.80000001 0.80000001}
\pscustom[linestyle=none,fillstyle=solid,fillcolor=curcolor]
{
\newpath
\moveto(169.39073181,221.01031232)
\lineto(202.68788528,221.01031232)
\lineto(202.68788528,203.99034238)
\lineto(169.39073181,203.99034238)
\closepath
}
}
{
\newrgbcolor{curcolor}{0.80000001 0.80000001 0.80000001}
\pscustom[linestyle=none,fillstyle=solid,fillcolor=curcolor]
{
\newpath
\moveto(169.39073181,170.75652814)
\lineto(202.68788528,170.75652814)
\lineto(202.68788528,153.7365582)
\lineto(169.39073181,153.7365582)
\closepath
}
}
{
\newrgbcolor{curcolor}{0.80000001 0.80000001 0.80000001}
\pscustom[linestyle=none,fillstyle=solid,fillcolor=curcolor]
{
\newpath
\moveto(169.39073181,153.76031232)
\lineto(202.68788528,153.76031232)
\lineto(202.68788528,136.74034238)
\lineto(169.39073181,136.74034238)
\closepath
}
}
{
\newrgbcolor{curcolor}{0.80000001 0.80000001 0.80000001}
\pscustom[linestyle=none,fillstyle=solid,fillcolor=curcolor]
{
\newpath
\moveto(169.39073181,85.76031232)
\lineto(202.68788528,85.76031232)
\lineto(202.68788528,68.74034238)
\lineto(169.39073181,68.74034238)
\closepath
}
}
{
\newrgbcolor{curcolor}{0.80000001 0.80000001 0.80000001}
\pscustom[linestyle=none,fillstyle=solid,fillcolor=curcolor]
{
\newpath
\moveto(169.39073181,34.26025129)
\lineto(202.68788528,34.26025129)
\lineto(202.68788528,17.24028135)
\lineto(169.39073181,17.24028135)
\closepath
}
}
{
\newrgbcolor{curcolor}{0.80000001 0.80000001 0.80000001}
\pscustom[linestyle=none,fillstyle=solid,fillcolor=curcolor]
{
\newpath
\moveto(135.89073181,51.52898908)
\lineto(169.18788528,51.52898908)
\lineto(169.18788528,34.50901914)
\lineto(135.89073181,34.50901914)
\closepath
}
}
{
\newrgbcolor{curcolor}{0.80000001 0.80000001 0.80000001}
\pscustom[linestyle=none,fillstyle=solid,fillcolor=curcolor]
{
\newpath
\moveto(135.89073181,34.53277326)
\lineto(169.18788528,34.53277326)
\lineto(169.18788528,17.51280332)
\lineto(135.89073181,17.51280332)
\closepath
}
}
{
\newrgbcolor{curcolor}{0.80000001 0.80000001 0.80000001}
\pscustom[linestyle=none,fillstyle=solid,fillcolor=curcolor]
{
\newpath
\moveto(135.89073181,17.53661848)
\lineto(169.18788528,17.53661848)
\lineto(169.18788528,0.51664854)
\lineto(135.89073181,0.51664854)
\closepath
}
}
{
\newrgbcolor{curcolor}{0.80000001 0.80000001 0.80000001}
\pscustom[linestyle=none,fillstyle=solid,fillcolor=curcolor]
{
\newpath
\moveto(102.59358215,34.53277326)
\lineto(135.89073563,34.53277326)
\lineto(135.89073563,17.51280332)
\lineto(102.59358215,17.51280332)
\closepath
}
}
{
\newrgbcolor{curcolor}{0.80000001 0.80000001 0.80000001}
\pscustom[linestyle=none,fillstyle=solid,fillcolor=curcolor]
{
\newpath
\moveto(102.59358215,17.53661848)
\lineto(135.89073563,17.53661848)
\lineto(135.89073563,0.51664854)
\lineto(102.59358215,0.51664854)
\closepath
}
}
{
\newrgbcolor{curcolor}{0.80000001 0.80000001 0.80000001}
\pscustom[linestyle=none,fillstyle=solid,fillcolor=curcolor]
{
\newpath
\moveto(102.59358215,85.51043439)
\lineto(135.89073563,85.51043439)
\lineto(135.89073563,68.49046445)
\lineto(102.59358215,68.49046445)
\closepath
}
}
{
\newrgbcolor{curcolor}{0.80000001 0.80000001 0.80000001}
\pscustom[linestyle=none,fillstyle=solid,fillcolor=curcolor]
{
\newpath
\moveto(102.59358215,68.51421857)
\lineto(135.89073563,68.51421857)
\lineto(135.89073563,51.49424863)
\lineto(102.59358215,51.49424863)
\closepath
}
}
{
\newrgbcolor{curcolor}{0.80000001 0.80000001 0.80000001}
\pscustom[linestyle=none,fillstyle=solid,fillcolor=curcolor]
{
\newpath
\moveto(102.59358215,119.48028303)
\lineto(135.89073563,119.48028303)
\lineto(135.89073563,102.46031309)
\lineto(102.59358215,102.46031309)
\closepath
}
}
{
\newrgbcolor{curcolor}{0.80000001 0.80000001 0.80000001}
\pscustom[linestyle=none,fillstyle=solid,fillcolor=curcolor]
{
\newpath
\moveto(102.59358215,102.48406721)
\lineto(135.89073563,102.48406721)
\lineto(135.89073563,85.46409727)
\lineto(102.59358215,85.46409727)
\closepath
}
}
{
\newrgbcolor{curcolor}{0.80000001 0.80000001 0.80000001}
\pscustom[linestyle=none,fillstyle=solid,fillcolor=curcolor]
{
\newpath
\moveto(102.59358215,153.44524885)
\lineto(135.89073563,153.44524885)
\lineto(135.89073563,136.42527891)
\lineto(102.59358215,136.42527891)
\closepath
}
}
{
\newrgbcolor{curcolor}{0.80000001 0.80000001 0.80000001}
\pscustom[linestyle=none,fillstyle=solid,fillcolor=curcolor]
{
\newpath
\moveto(102.59358215,136.44903303)
\lineto(135.89073563,136.44903303)
\lineto(135.89073563,119.42906309)
\lineto(102.59358215,119.42906309)
\closepath
}
}
{
\newrgbcolor{curcolor}{0.80000001 0.80000001 0.80000001}
\pscustom[linestyle=none,fillstyle=solid,fillcolor=curcolor]
{
\newpath
\moveto(102.59358215,187.41521955)
\lineto(135.89073563,187.41521955)
\lineto(135.89073563,170.39524961)
\lineto(102.59358215,170.39524961)
\closepath
}
}
{
\newrgbcolor{curcolor}{0.80000001 0.80000001 0.80000001}
\pscustom[linestyle=none,fillstyle=solid,fillcolor=curcolor]
{
\newpath
\moveto(102.59358215,170.41900373)
\lineto(135.89073563,170.41900373)
\lineto(135.89073563,153.39903379)
\lineto(102.59358215,153.39903379)
\closepath
}
}
{
\newrgbcolor{curcolor}{0.80000001 0.80000001 0.80000001}
\pscustom[linestyle=none,fillstyle=solid,fillcolor=curcolor]
{
\newpath
\moveto(102.59358215,272.48028303)
\lineto(135.89073563,272.48028303)
\lineto(135.89073563,255.46031309)
\lineto(102.59358215,255.46031309)
\closepath
}
}
{
\newrgbcolor{curcolor}{0.80000001 0.80000001 0.80000001}
\pscustom[linestyle=none,fillstyle=solid,fillcolor=curcolor]
{
\newpath
\moveto(102.59358215,221.50030256)
\lineto(135.89073563,221.50030256)
\lineto(135.89073563,204.48033262)
\lineto(102.59358215,204.48033262)
\closepath
}
}
{
\newrgbcolor{curcolor}{0.3019608 0.3019608 0.3019608}
\pscustom[linestyle=none,fillstyle=solid,fillcolor=curcolor]
{
\newpath
\moveto(335.53973389,527.50445295)
\lineto(353.08185196,527.50445295)
\lineto(353.08185196,510.48211408)
\lineto(335.53973389,510.48211408)
\closepath
}
}
{
\newrgbcolor{curcolor}{0.3019608 0.3019608 0.3019608}
\pscustom[linestyle=none,fillstyle=solid,fillcolor=curcolor]
{
\newpath
\moveto(353.0246582,510.53686262)
\lineto(370.56677628,510.53686262)
\lineto(370.56677628,493.51452375)
\lineto(353.0246582,493.51452375)
\closepath
}
}
{
\newrgbcolor{curcolor}{0.3019608 0.3019608 0.3019608}
\pscustom[linestyle=none,fillstyle=solid,fillcolor=curcolor]
{
\newpath
\moveto(370.67959595,493.50915266)
\lineto(388.22171402,493.50915266)
\lineto(388.22171402,476.48681379)
\lineto(370.67959595,476.48681379)
\closepath
}
}
{
\newrgbcolor{curcolor}{0.3019608 0.3019608 0.3019608}
\pscustom[linestyle=none,fillstyle=solid,fillcolor=curcolor]
{
\newpath
\moveto(388.16455078,476.54150129)
\lineto(405.70666885,476.54150129)
\lineto(405.70666885,459.51916242)
\lineto(388.16455078,459.51916242)
\closepath
}
}
{
\newrgbcolor{curcolor}{0.3019608 0.3019608 0.3019608}
\pscustom[linestyle=none,fillstyle=solid,fillcolor=curcolor]
{
\newpath
\moveto(405.78027344,459.52435041)
\lineto(423.32239151,459.52435041)
\lineto(423.32239151,442.50201154)
\lineto(405.78027344,442.50201154)
\closepath
}
}
{
\newrgbcolor{curcolor}{0.3019608 0.3019608 0.3019608}
\pscustom[linestyle=none,fillstyle=solid,fillcolor=curcolor]
{
\newpath
\moveto(423.26513672,442.55669904)
\lineto(440.80725479,442.55669904)
\lineto(440.80725479,425.53436018)
\lineto(423.26513672,425.53436018)
\closepath
}
}
{
\newrgbcolor{curcolor}{0.3019608 0.3019608 0.3019608}
\pscustom[linestyle=none,fillstyle=solid,fillcolor=curcolor]
{
\newpath
\moveto(440.9201355,425.52917219)
\lineto(458.46225357,425.52917219)
\lineto(458.46225357,408.50683332)
\lineto(440.9201355,408.50683332)
\closepath
}
}
{
\newrgbcolor{curcolor}{0.3019608 0.3019608 0.3019608}
\pscustom[linestyle=none,fillstyle=solid,fillcolor=curcolor]
{
\newpath
\moveto(458.4050293,408.56145979)
\lineto(475.94714737,408.56145979)
\lineto(475.94714737,391.53912092)
\lineto(458.4050293,391.53912092)
\closepath
}
}
{
\newrgbcolor{curcolor}{0.3019608 0.3019608 0.3019608}
\pscustom[linestyle=none,fillstyle=solid,fillcolor=curcolor]
{
\newpath
\moveto(475.95471191,391.48327375)
\lineto(493.49682999,391.48327375)
\lineto(493.49682999,374.46093488)
\lineto(475.95471191,374.46093488)
\closepath
}
}
{
\newrgbcolor{curcolor}{0.3019608 0.3019608 0.3019608}
\pscustom[linestyle=none,fillstyle=solid,fillcolor=curcolor]
{
\newpath
\moveto(493.4395752,374.51543928)
\lineto(510.98169327,374.51543928)
\lineto(510.98169327,357.49310041)
\lineto(493.4395752,357.49310041)
\closepath
}
}
{
\newrgbcolor{curcolor}{0.3019608 0.3019608 0.3019608}
\pscustom[linestyle=none,fillstyle=solid,fillcolor=curcolor]
{
\newpath
\moveto(511.09457397,357.48797346)
\lineto(528.63669205,357.48797346)
\lineto(528.63669205,340.46563459)
\lineto(511.09457397,340.46563459)
\closepath
}
}
{
\newrgbcolor{curcolor}{0.3019608 0.3019608 0.3019608}
\pscustom[linestyle=none,fillstyle=solid,fillcolor=curcolor]
{
\newpath
\moveto(528.57946777,340.52026105)
\lineto(546.12158585,340.52026105)
\lineto(546.12158585,323.49792219)
\lineto(528.57946777,323.49792219)
\closepath
}
}
{
\newrgbcolor{curcolor}{0.3019608 0.3019608 0.3019608}
\pscustom[linestyle=none,fillstyle=solid,fillcolor=curcolor]
{
\newpath
\moveto(546.19525146,323.50311018)
\lineto(563.73736954,323.50311018)
\lineto(563.73736954,306.48077131)
\lineto(546.19525146,306.48077131)
\closepath
}
}
{
\newrgbcolor{curcolor}{0.3019608 0.3019608 0.3019608}
\pscustom[linestyle=none,fillstyle=solid,fillcolor=curcolor]
{
\newpath
\moveto(563.68011475,306.53545881)
\lineto(581.22223282,306.53545881)
\lineto(581.22223282,289.51311994)
\lineto(563.68011475,289.51311994)
\closepath
}
}
{
\newrgbcolor{curcolor}{0.3019608 0.3019608 0.3019608}
\pscustom[linestyle=none,fillstyle=solid,fillcolor=curcolor]
{
\newpath
\moveto(581.33514404,289.50793195)
\lineto(598.87726212,289.50793195)
\lineto(598.87726212,272.48559309)
\lineto(581.33514404,272.48559309)
\closepath
}
}
{
\newrgbcolor{curcolor}{0.3019608 0.3019608 0.3019608}
\pscustom[linestyle=none,fillstyle=solid,fillcolor=curcolor]
{
\newpath
\moveto(598.82000732,272.54021955)
\lineto(616.3621254,272.54021955)
\lineto(616.3621254,255.51788068)
\lineto(598.82000732,255.51788068)
\closepath
}
}
{
\newrgbcolor{curcolor}{0.3019608 0.3019608 0.3019608}
\pscustom[linestyle=none,fillstyle=solid,fillcolor=curcolor]
{
\newpath
\moveto(616.3694458,255.42443586)
\lineto(633.91156387,255.42443586)
\lineto(633.91156387,238.40209699)
\lineto(616.3694458,238.40209699)
\closepath
}
}
{
\newrgbcolor{curcolor}{0.3019608 0.3019608 0.3019608}
\pscustom[linestyle=none,fillstyle=solid,fillcolor=curcolor]
{
\newpath
\moveto(633.85437012,238.45672346)
\lineto(651.39648819,238.45672346)
\lineto(651.39648819,221.43438459)
\lineto(633.85437012,221.43438459)
\closepath
}
}
{
\newrgbcolor{curcolor}{0.3019608 0.3019608 0.3019608}
\pscustom[linestyle=none,fillstyle=solid,fillcolor=curcolor]
{
\newpath
\moveto(651.50933838,221.42913557)
\lineto(669.05145645,221.42913557)
\lineto(669.05145645,204.4067967)
\lineto(651.50933838,204.4067967)
\closepath
}
}
{
\newrgbcolor{curcolor}{0.3019608 0.3019608 0.3019608}
\pscustom[linestyle=none,fillstyle=solid,fillcolor=curcolor]
{
\newpath
\moveto(668.99420166,204.46142316)
\lineto(686.53631973,204.46142316)
\lineto(686.53631973,187.4390843)
\lineto(668.99420166,187.4390843)
\closepath
}
}
{
\newrgbcolor{curcolor}{0.3019608 0.3019608 0.3019608}
\pscustom[linestyle=none,fillstyle=solid,fillcolor=curcolor]
{
\newpath
\moveto(686.61004639,187.44427229)
\lineto(704.15216446,187.44427229)
\lineto(704.15216446,170.42193342)
\lineto(686.61004639,170.42193342)
\closepath
}
}
{
\newrgbcolor{curcolor}{0.3019608 0.3019608 0.3019608}
\pscustom[linestyle=none,fillstyle=solid,fillcolor=curcolor]
{
\newpath
\moveto(704.09490967,170.47655988)
\lineto(721.63702774,170.47655988)
\lineto(721.63702774,153.45422102)
\lineto(704.09490967,153.45422102)
\closepath
}
}
{
\newrgbcolor{curcolor}{0.3019608 0.3019608 0.3019608}
\pscustom[linestyle=none,fillstyle=solid,fillcolor=curcolor]
{
\newpath
\moveto(721.74987793,153.44909406)
\lineto(739.291996,153.44909406)
\lineto(739.291996,136.4267552)
\lineto(721.74987793,136.4267552)
\closepath
}
}
{
\newrgbcolor{curcolor}{0.3019608 0.3019608 0.3019608}
\pscustom[linestyle=none,fillstyle=solid,fillcolor=curcolor]
{
\newpath
\moveto(739.23480225,136.48138166)
\lineto(756.77692032,136.48138166)
\lineto(756.77692032,119.45904279)
\lineto(739.23480225,119.45904279)
\closepath
}
}
{
\newrgbcolor{curcolor}{0.3019608 0.3019608 0.3019608}
\pscustom[linestyle=none,fillstyle=solid,fillcolor=curcolor]
{
\newpath
\moveto(756.83251953,119.45898176)
\lineto(774.3746376,119.45898176)
\lineto(774.3746376,102.43664289)
\lineto(756.83251953,102.43664289)
\closepath
}
}
{
\newrgbcolor{curcolor}{0.3019608 0.3019608 0.3019608}
\pscustom[linestyle=none,fillstyle=solid,fillcolor=curcolor]
{
\newpath
\moveto(774.31738281,102.49126936)
\lineto(791.85950089,102.49126936)
\lineto(791.85950089,85.46893049)
\lineto(774.31738281,85.46893049)
\closepath
}
}
{
\newrgbcolor{curcolor}{0.3019608 0.3019608 0.3019608}
\pscustom[linestyle=none,fillstyle=solid,fillcolor=curcolor]
{
\newpath
\moveto(791.97241211,85.46380354)
\lineto(809.51453018,85.46380354)
\lineto(809.51453018,68.44146467)
\lineto(791.97241211,68.44146467)
\closepath
}
}
{
\newrgbcolor{curcolor}{0.3019608 0.3019608 0.3019608}
\pscustom[linestyle=none,fillstyle=solid,fillcolor=curcolor]
{
\newpath
\moveto(809.45727539,68.49609113)
\lineto(826.99939346,68.49609113)
\lineto(826.99939346,51.47375227)
\lineto(809.45727539,51.47375227)
\closepath
}
}
{
\newrgbcolor{curcolor}{0.3019608 0.3019608 0.3019608}
\pscustom[linestyle=none,fillstyle=solid,fillcolor=curcolor]
{
\newpath
\moveto(826.92254639,51.4501927)
\lineto(844.46466446,51.4501927)
\lineto(844.46466446,34.42785383)
\lineto(826.92254639,34.42785383)
\closepath
}
}
{
\newrgbcolor{curcolor}{0.3019608 0.3019608 0.3019608}
\pscustom[linestyle=none,fillstyle=solid,fillcolor=curcolor]
{
\newpath
\moveto(844.4074707,34.48248029)
\lineto(861.94958878,34.48248029)
\lineto(861.94958878,17.46014143)
\lineto(844.4074707,17.46014143)
\closepath
}
}
{
\newrgbcolor{curcolor}{0.3019608 0.3019608 0.3019608}
\pscustom[linestyle=none,fillstyle=solid,fillcolor=curcolor]
{
\newpath
\moveto(862.06243896,17.45495344)
\lineto(879.60455704,17.45495344)
\lineto(879.60455704,0.43261457)
\lineto(862.06243896,0.43261457)
\closepath
}
}
{
\newrgbcolor{curcolor}{0.80000001 0.80000001 0.80000001}
\pscustom[linestyle=none,fillstyle=solid,fillcolor=curcolor]
{
\newpath
\moveto(353.0246582,527.50445295)
\lineto(370.56677628,527.50445295)
\lineto(370.56677628,510.48211408)
\lineto(353.0246582,510.48211408)
\closepath
}
}
{
\newrgbcolor{curcolor}{0.80000001 0.80000001 0.80000001}
\pscustom[linestyle=none,fillstyle=solid,fillcolor=curcolor]
{
\newpath
\moveto(370.62237549,527.50445295)
\lineto(388.16449356,527.50445295)
\lineto(388.16449356,510.48211408)
\lineto(370.62237549,510.48211408)
\closepath
}
}
{
\newrgbcolor{curcolor}{0.80000001 0.80000001 0.80000001}
\pscustom[linestyle=none,fillstyle=solid,fillcolor=curcolor]
{
\newpath
\moveto(405.74261475,510.53686262)
\lineto(423.28473282,510.53686262)
\lineto(423.28473282,493.51452375)
\lineto(405.74261475,493.51452375)
\closepath
}
}
{
\newrgbcolor{curcolor}{0.80000001 0.80000001 0.80000001}
\pscustom[linestyle=none,fillstyle=solid,fillcolor=curcolor]
{
\newpath
\moveto(335.53973389,510.53686262)
\lineto(353.08185196,510.53686262)
\lineto(353.08185196,493.51452375)
\lineto(335.53973389,493.51452375)
\closepath
}
}
{
\newrgbcolor{curcolor}{0.80000001 0.80000001 0.80000001}
\pscustom[linestyle=none,fillstyle=solid,fillcolor=curcolor]
{
\newpath
\moveto(335.53973389,493.56384016)
\lineto(353.08185196,493.56384016)
\lineto(353.08185196,476.54150129)
\lineto(335.53973389,476.54150129)
\closepath
}
}
{
\newrgbcolor{curcolor}{0.80000001 0.80000001 0.80000001}
\pscustom[linestyle=none,fillstyle=solid,fillcolor=curcolor]
{
\newpath
\moveto(440.82525635,510.53686262)
\lineto(458.36737442,510.53686262)
\lineto(458.36737442,493.51452375)
\lineto(440.82525635,493.51452375)
\closepath
}
}
{
\newrgbcolor{curcolor}{0.80000001 0.80000001 0.80000001}
\pscustom[linestyle=none,fillstyle=solid,fillcolor=curcolor]
{
\newpath
\moveto(458.38543701,510.53686262)
\lineto(475.92755508,510.53686262)
\lineto(475.92755508,493.51452375)
\lineto(458.38543701,493.51452375)
\closepath
}
}
{
\newrgbcolor{curcolor}{0.80000001 0.80000001 0.80000001}
\pscustom[linestyle=none,fillstyle=solid,fillcolor=curcolor]
{
\newpath
\moveto(458.38543701,527.50445295)
\lineto(475.92755508,527.50445295)
\lineto(475.92755508,510.48211408)
\lineto(458.38543701,510.48211408)
\closepath
}
}
{
\newrgbcolor{curcolor}{0.80000001 0.80000001 0.80000001}
\pscustom[linestyle=none,fillstyle=solid,fillcolor=curcolor]
{
\newpath
\moveto(598.80004883,527.50445295)
\lineto(616.3421669,527.50445295)
\lineto(616.3421669,510.48211408)
\lineto(598.80004883,510.48211408)
\closepath
}
}
{
\newrgbcolor{curcolor}{0.80000001 0.80000001 0.80000001}
\pscustom[linestyle=none,fillstyle=solid,fillcolor=curcolor]
{
\newpath
\moveto(651.47216797,527.50445295)
\lineto(669.01428604,527.50445295)
\lineto(669.01428604,510.48211408)
\lineto(651.47216797,510.48211408)
\closepath
}
}
{
\newrgbcolor{curcolor}{0.80000001 0.80000001 0.80000001}
\pscustom[linestyle=none,fillstyle=solid,fillcolor=curcolor]
{
\newpath
\moveto(669.01428223,527.50445295)
\lineto(686.5564003,527.50445295)
\lineto(686.5564003,510.48211408)
\lineto(669.01428223,510.48211408)
\closepath
}
}
{
\newrgbcolor{curcolor}{0.80000001 0.80000001 0.80000001}
\pscustom[linestyle=none,fillstyle=solid,fillcolor=curcolor]
{
\newpath
\moveto(651.46575928,510.53686262)
\lineto(669.00787735,510.53686262)
\lineto(669.00787735,493.51452375)
\lineto(651.46575928,493.51452375)
\closepath
}
}
{
\newrgbcolor{curcolor}{0.80000001 0.80000001 0.80000001}
\pscustom[linestyle=none,fillstyle=solid,fillcolor=curcolor]
{
\newpath
\moveto(809.43774414,527.50445295)
\lineto(826.97986221,527.50445295)
\lineto(826.97986221,510.48211408)
\lineto(809.43774414,510.48211408)
\closepath
}
}
{
\newrgbcolor{curcolor}{0.80000001 0.80000001 0.80000001}
\pscustom[linestyle=none,fillstyle=solid,fillcolor=curcolor]
{
\newpath
\moveto(862.0993042,510.53686262)
\lineto(879.64142227,510.53686262)
\lineto(879.64142227,493.51452375)
\lineto(862.0993042,493.51452375)
\closepath
}
}
{
\newrgbcolor{curcolor}{0.80000001 0.80000001 0.80000001}
\pscustom[linestyle=none,fillstyle=solid,fillcolor=curcolor]
{
\newpath
\moveto(353.0246582,459.50750471)
\lineto(370.56677628,459.50750471)
\lineto(370.56677628,442.48516584)
\lineto(353.0246582,442.48516584)
\closepath
}
}
{
\newrgbcolor{curcolor}{0.80000001 0.80000001 0.80000001}
\pscustom[linestyle=none,fillstyle=solid,fillcolor=curcolor]
{
\newpath
\moveto(353.08187866,425.51745344)
\lineto(370.62399673,425.51745344)
\lineto(370.62399673,408.49511457)
\lineto(353.08187866,408.49511457)
\closepath
}
}
{
\newrgbcolor{curcolor}{0.80000001 0.80000001 0.80000001}
\pscustom[linestyle=none,fillstyle=solid,fillcolor=curcolor]
{
\newpath
\moveto(353.08187866,408.47820783)
\lineto(370.62399673,408.47820783)
\lineto(370.62399673,391.45586896)
\lineto(353.08187866,391.45586896)
\closepath
}
}
{
\newrgbcolor{curcolor}{0.80000001 0.80000001 0.80000001}
\pscustom[linestyle=none,fillstyle=solid,fillcolor=curcolor]
{
\newpath
\moveto(335.53973389,408.47820783)
\lineto(353.08185196,408.47820783)
\lineto(353.08185196,391.45586896)
\lineto(335.53973389,391.45586896)
\closepath
}
}
{
\newrgbcolor{curcolor}{0.80000001 0.80000001 0.80000001}
\pscustom[linestyle=none,fillstyle=solid,fillcolor=curcolor]
{
\newpath
\moveto(335.53973389,272.48028303)
\lineto(353.08185196,272.48028303)
\lineto(353.08185196,255.45794416)
\lineto(335.53973389,255.45794416)
\closepath
}
}
{
\newrgbcolor{curcolor}{0.80000001 0.80000001 0.80000001}
\pscustom[linestyle=none,fillstyle=solid,fillcolor=curcolor]
{
\newpath
\moveto(335.53973389,221.47320295)
\lineto(353.08185196,221.47320295)
\lineto(353.08185196,204.45086408)
\lineto(335.53973389,204.45086408)
\closepath
}
}
{
\newrgbcolor{curcolor}{0.80000001 0.80000001 0.80000001}
\pscustom[linestyle=none,fillstyle=solid,fillcolor=curcolor]
{
\newpath
\moveto(353.0246582,221.47320295)
\lineto(370.56677628,221.47320295)
\lineto(370.56677628,204.45086408)
\lineto(353.0246582,204.45086408)
\closepath
}
}
{
\newrgbcolor{curcolor}{0.80000001 0.80000001 0.80000001}
\pscustom[linestyle=none,fillstyle=solid,fillcolor=curcolor]
{
\newpath
\moveto(335.53973389,204.49084211)
\lineto(353.08185196,204.49084211)
\lineto(353.08185196,187.46850324)
\lineto(335.53973389,187.46850324)
\closepath
}
}
{
\newrgbcolor{curcolor}{0.80000001 0.80000001 0.80000001}
\pscustom[linestyle=none,fillstyle=solid,fillcolor=curcolor]
{
\newpath
\moveto(335.53973389,68.44518781)
\lineto(353.08185196,68.44518781)
\lineto(353.08185196,51.42284895)
\lineto(335.53973389,51.42284895)
\closepath
}
}
{
\newrgbcolor{curcolor}{0.80000001 0.80000001 0.80000001}
\pscustom[linestyle=none,fillstyle=solid,fillcolor=curcolor]
{
\newpath
\moveto(353.0246582,17.45495344)
\lineto(370.56677628,17.45495344)
\lineto(370.56677628,0.43261457)
\lineto(353.0246582,0.43261457)
\closepath
}
}
{
\newrgbcolor{curcolor}{0.80000001 0.80000001 0.80000001}
\pscustom[linestyle=none,fillstyle=solid,fillcolor=curcolor]
{
\newpath
\moveto(616.34851074,238.47552229)
\lineto(633.89062881,238.47552229)
\lineto(633.89062881,221.45318342)
\lineto(616.34851074,221.45318342)
\closepath
}
}
{
\newrgbcolor{curcolor}{0.80000001 0.80000001 0.80000001}
\pscustom[linestyle=none,fillstyle=solid,fillcolor=curcolor]
{
\newpath
\moveto(633.92364502,255.50915266)
\lineto(651.46576309,255.50915266)
\lineto(651.46576309,238.48681379)
\lineto(633.92364502,238.48681379)
\closepath
}
}
{
\newrgbcolor{curcolor}{0.50196081 0.50196081 0.50196081}
\pscustom[linestyle=none,fillstyle=solid,fillcolor=curcolor]
{
\newpath
\moveto(370.62237549,510.53686262)
\lineto(388.16449356,510.53686262)
\lineto(388.16449356,493.51452375)
\lineto(370.62237549,493.51452375)
\closepath
}
}
{
\newrgbcolor{curcolor}{0.50196081 0.50196081 0.50196081}
\pscustom[linestyle=none,fillstyle=solid,fillcolor=curcolor]
{
\newpath
\moveto(388.14489746,510.53686262)
\lineto(405.68701553,510.53686262)
\lineto(405.68701553,493.51452375)
\lineto(388.14489746,493.51452375)
\closepath
}
}
{
\newrgbcolor{curcolor}{0.50196081 0.50196081 0.50196081}
\pscustom[linestyle=none,fillstyle=solid,fillcolor=curcolor]
{
\newpath
\moveto(388.14489746,527.50445295)
\lineto(405.68701553,527.50445295)
\lineto(405.68701553,510.48211408)
\lineto(388.14489746,510.48211408)
\closepath
}
}
{
\newrgbcolor{curcolor}{0.50196081 0.50196081 0.50196081}
\pscustom[linestyle=none,fillstyle=solid,fillcolor=curcolor]
{
\newpath
\moveto(423.26513672,510.53686262)
\lineto(440.80725479,510.53686262)
\lineto(440.80725479,493.51452375)
\lineto(423.26513672,493.51452375)
\closepath
}
}
{
\newrgbcolor{curcolor}{0.50196081 0.50196081 0.50196081}
\pscustom[linestyle=none,fillstyle=solid,fillcolor=curcolor]
{
\newpath
\moveto(475.92755127,510.53686262)
\lineto(493.46966934,510.53686262)
\lineto(493.46966934,493.51452375)
\lineto(475.92755127,493.51452375)
\closepath
}
}
{
\newrgbcolor{curcolor}{0.50196081 0.50196081 0.50196081}
\pscustom[linestyle=none,fillstyle=solid,fillcolor=curcolor]
{
\newpath
\moveto(493.47613525,510.53686262)
\lineto(511.01825333,510.53686262)
\lineto(511.01825333,493.51452375)
\lineto(493.47613525,493.51452375)
\closepath
}
}
{
\newrgbcolor{curcolor}{0.50196081 0.50196081 0.50196081}
\pscustom[linestyle=none,fillstyle=solid,fillcolor=curcolor]
{
\newpath
\moveto(493.48254395,527.50445295)
\lineto(511.02466202,527.50445295)
\lineto(511.02466202,510.48211408)
\lineto(493.48254395,510.48211408)
\closepath
}
}
{
\newrgbcolor{curcolor}{0.50196081 0.50196081 0.50196081}
\pscustom[linestyle=none,fillstyle=solid,fillcolor=curcolor]
{
\newpath
\moveto(511.03112793,527.50445295)
\lineto(528.573246,527.50445295)
\lineto(528.573246,510.48211408)
\lineto(511.03112793,510.48211408)
\closepath
}
}
{
\newrgbcolor{curcolor}{0.50196081 0.50196081 0.50196081}
\pscustom[linestyle=none,fillstyle=solid,fillcolor=curcolor]
{
\newpath
\moveto(528.57952881,527.50445295)
\lineto(546.12164688,527.50445295)
\lineto(546.12164688,510.48211408)
\lineto(528.57952881,510.48211408)
\closepath
}
}
{
\newrgbcolor{curcolor}{0.50196081 0.50196081 0.50196081}
\pscustom[linestyle=none,fillstyle=solid,fillcolor=curcolor]
{
\newpath
\moveto(546.12811279,527.50445295)
\lineto(563.67023087,527.50445295)
\lineto(563.67023087,510.48211408)
\lineto(546.12811279,510.48211408)
\closepath
}
}
{
\newrgbcolor{curcolor}{0.50196081 0.50196081 0.50196081}
\pscustom[linestyle=none,fillstyle=solid,fillcolor=curcolor]
{
\newpath
\moveto(511.02462769,510.53686262)
\lineto(528.56674576,510.53686262)
\lineto(528.56674576,493.51452375)
\lineto(511.02462769,493.51452375)
\closepath
}
}
{
\newrgbcolor{curcolor}{0.50196081 0.50196081 0.50196081}
\pscustom[linestyle=none,fillstyle=solid,fillcolor=curcolor]
{
\newpath
\moveto(563.70318604,527.44988752)
\lineto(581.24530411,527.44988752)
\lineto(581.24530411,510.42754865)
\lineto(563.70318604,510.42754865)
\closepath
}
}
{
\newrgbcolor{curcolor}{0.50196081 0.50196081 0.50196081}
\pscustom[linestyle=none,fillstyle=solid,fillcolor=curcolor]
{
\newpath
\moveto(563.69671631,510.48211408)
\lineto(581.23883438,510.48211408)
\lineto(581.23883438,493.45977521)
\lineto(563.69671631,493.45977521)
\closepath
}
}
{
\newrgbcolor{curcolor}{0.50196081 0.50196081 0.50196081}
\pscustom[linestyle=none,fillstyle=solid,fillcolor=curcolor]
{
\newpath
\moveto(598.80004883,510.48211408)
\lineto(616.3421669,510.48211408)
\lineto(616.3421669,493.45977521)
\lineto(598.80004883,493.45977521)
\closepath
}
}
{
\newrgbcolor{curcolor}{0.50196081 0.50196081 0.50196081}
\pscustom[linestyle=none,fillstyle=solid,fillcolor=curcolor]
{
\newpath
\moveto(686.56915283,510.53686262)
\lineto(704.1112709,510.53686262)
\lineto(704.1112709,493.51452375)
\lineto(686.56915283,493.51452375)
\closepath
}
}
{
\newrgbcolor{curcolor}{0.50196081 0.50196081 0.50196081}
\pscustom[linestyle=none,fillstyle=solid,fillcolor=curcolor]
{
\newpath
\moveto(704.11767578,527.50445295)
\lineto(721.65979385,527.50445295)
\lineto(721.65979385,510.48211408)
\lineto(704.11767578,510.48211408)
\closepath
}
}
{
\newrgbcolor{curcolor}{0.50196081 0.50196081 0.50196081}
\pscustom[linestyle=none,fillstyle=solid,fillcolor=curcolor]
{
\newpath
\moveto(739.23480225,527.50445295)
\lineto(756.77692032,527.50445295)
\lineto(756.77692032,510.48211408)
\lineto(739.23480225,510.48211408)
\closepath
}
}
{
\newrgbcolor{curcolor}{0.50196081 0.50196081 0.50196081}
\pscustom[linestyle=none,fillstyle=solid,fillcolor=curcolor]
{
\newpath
\moveto(756.83251953,527.50445295)
\lineto(774.3746376,527.50445295)
\lineto(774.3746376,510.48211408)
\lineto(756.83251953,510.48211408)
\closepath
}
}
{
\newrgbcolor{curcolor}{0.50196081 0.50196081 0.50196081}
\pscustom[linestyle=none,fillstyle=solid,fillcolor=curcolor]
{
\newpath
\moveto(774.31750488,527.50445295)
\lineto(791.85962296,527.50445295)
\lineto(791.85962296,510.48211408)
\lineto(774.31750488,510.48211408)
\closepath
}
}
{
\newrgbcolor{curcolor}{0.50196081 0.50196081 0.50196081}
\pscustom[linestyle=none,fillstyle=solid,fillcolor=curcolor]
{
\newpath
\moveto(475.93405151,408.47820783)
\lineto(493.47616959,408.47820783)
\lineto(493.47616959,391.45586896)
\lineto(475.93405151,391.45586896)
\closepath
}
}
{
\newrgbcolor{curcolor}{0.50196081 0.50196081 0.50196081}
\pscustom[linestyle=none,fillstyle=solid,fillcolor=curcolor]
{
\newpath
\moveto(563.703125,289.50268293)
\lineto(581.24524307,289.50268293)
\lineto(581.24524307,272.48034406)
\lineto(563.703125,272.48034406)
\closepath
}
}
{
\newrgbcolor{curcolor}{0.50196081 0.50196081 0.50196081}
\pscustom[linestyle=none,fillstyle=solid,fillcolor=curcolor]
{
\newpath
\moveto(563.69671631,272.53155256)
\lineto(581.23883438,272.53155256)
\lineto(581.23883438,255.50921369)
\lineto(563.69671631,255.50921369)
\closepath
}
}
{
\newrgbcolor{curcolor}{0.50196081 0.50196081 0.50196081}
\pscustom[linestyle=none,fillstyle=solid,fillcolor=curcolor]
{
\newpath
\moveto(581.2845459,255.44665266)
\lineto(598.82666397,255.44665266)
\lineto(598.82666397,238.42431379)
\lineto(581.2845459,238.42431379)
\closepath
}
}
{
\newrgbcolor{curcolor}{0.50196081 0.50196081 0.50196081}
\pscustom[linestyle=none,fillstyle=solid,fillcolor=curcolor]
{
\newpath
\moveto(581.27819824,238.47552229)
\lineto(598.82031631,238.47552229)
\lineto(598.82031631,221.45318342)
\lineto(581.27819824,221.45318342)
\closepath
}
}
{
\newrgbcolor{curcolor}{0.50196081 0.50196081 0.50196081}
\pscustom[linestyle=none,fillstyle=solid,fillcolor=curcolor]
{
\newpath
\moveto(704.11767578,238.49554182)
\lineto(721.65979385,238.49554182)
\lineto(721.65979385,221.47320295)
\lineto(704.11767578,221.47320295)
\closepath
}
}
{
\newrgbcolor{curcolor}{0.50196081 0.50196081 0.50196081}
\pscustom[linestyle=none,fillstyle=solid,fillcolor=curcolor]
{
\newpath
\moveto(704.11126709,204.49084211)
\lineto(721.65338516,204.49084211)
\lineto(721.65338516,187.46850324)
\lineto(704.11126709,187.46850324)
\closepath
}
}
{
\newrgbcolor{curcolor}{0.50196081 0.50196081 0.50196081}
\pscustom[linestyle=none,fillstyle=solid,fillcolor=curcolor]
{
\newpath
\moveto(721.67468262,238.49554182)
\lineto(739.21680069,238.49554182)
\lineto(739.21680069,221.47320295)
\lineto(721.67468262,221.47320295)
\closepath
}
}
{
\newrgbcolor{curcolor}{0.50196081 0.50196081 0.50196081}
\pscustom[linestyle=none,fillstyle=solid,fillcolor=curcolor]
{
\newpath
\moveto(826.97979736,68.44518781)
\lineto(844.52191544,68.44518781)
\lineto(844.52191544,51.42284895)
\lineto(826.97979736,51.42284895)
\closepath
}
}
{
\newrgbcolor{curcolor}{0.50196081 0.50196081 0.50196081}
\pscustom[linestyle=none,fillstyle=solid,fillcolor=curcolor]
{
\newpath
\moveto(844.5368042,68.44518781)
\lineto(862.07892227,68.44518781)
\lineto(862.07892227,51.42284895)
\lineto(844.5368042,51.42284895)
\closepath
}
}
{
\newrgbcolor{curcolor}{0.50196081 0.50196081 0.50196081}
\pscustom[linestyle=none,fillstyle=solid,fillcolor=curcolor]
{
\newpath
\moveto(704.11126709,68.46203352)
\lineto(721.65338516,68.46203352)
\lineto(721.65338516,51.43969465)
\lineto(704.11126709,51.43969465)
\closepath
}
}
{
\newrgbcolor{curcolor}{0.50196081 0.50196081 0.50196081}
\pscustom[linestyle=none,fillstyle=solid,fillcolor=curcolor]
{
\newpath
\moveto(721.66827393,68.46203352)
\lineto(739.210392,68.46203352)
\lineto(739.210392,51.43969465)
\lineto(721.66827393,51.43969465)
\closepath
}
}
{
\newrgbcolor{curcolor}{0.50196081 0.50196081 0.50196081}
\pscustom[linestyle=none,fillstyle=solid,fillcolor=curcolor]
{
\newpath
\moveto(704.11767578,85.48443342)
\lineto(721.65979385,85.48443342)
\lineto(721.65979385,68.46209455)
\lineto(704.11767578,68.46209455)
\closepath
}
}
{
\newrgbcolor{curcolor}{0.50196081 0.50196081 0.50196081}
\pscustom[linestyle=none,fillstyle=solid,fillcolor=curcolor]
{
\newpath
\moveto(563.69671631,68.46203352)
\lineto(581.23883438,68.46203352)
\lineto(581.23883438,51.43969465)
\lineto(563.69671631,51.43969465)
\closepath
}
}
{
\newrgbcolor{curcolor}{0.50196081 0.50196081 0.50196081}
\pscustom[linestyle=none,fillstyle=solid,fillcolor=curcolor]
{
\newpath
\moveto(493.47613525,68.44518781)
\lineto(511.01825333,68.44518781)
\lineto(511.01825333,51.42284895)
\lineto(493.47613525,51.42284895)
\closepath
}
}
{
\newrgbcolor{curcolor}{0.50196081 0.50196081 0.50196081}
\pscustom[linestyle=none,fillstyle=solid,fillcolor=curcolor]
{
\newpath
\moveto(353.08187866,68.44518781)
\lineto(370.62399673,68.44518781)
\lineto(370.62399673,51.42284895)
\lineto(353.08187866,51.42284895)
\closepath
}
}
{
\newrgbcolor{curcolor}{0 0 0}
\pscustom[linewidth=0.97635722,linecolor=curcolor]
{
\newpath
\moveto(102.62741,544.5349)
\lineto(102.62741,0.48818)
}
}
{
\newrgbcolor{curcolor}{0 0 0}
\pscustom[linewidth=0.96543622,linecolor=curcolor]
{
\newpath
\moveto(0.48271811,527.48601)
\lineto(879.60683,527.48601)
}
}
{
\newrgbcolor{curcolor}{0 0 0}
\pscustom[linewidth=0.96543622,linecolor=curcolor]
{
\newpath
\moveto(0.48271811,0.49928)
\lineto(879.60683,0.49928)
}
}
{
\newrgbcolor{curcolor}{0 0 0}
\pscustom[linewidth=0.90066206,linecolor=curcolor]
{
\newpath
\moveto(335.51919,544.57275)
\lineto(335.51919,0.45038)
}
}
{
\newrgbcolor{curcolor}{0 0 0}
\pscustom[linewidth=0.90066206,linecolor=curcolor]
{
\newpath
\moveto(879.64153,527.51372)
\lineto(879.64153,0.45038)
}
}
{
\newrgbcolor{curcolor}{0 0 0}
\pscustom[linewidth=0.97635722,linecolor=curcolor]
{
\newpath
\moveto(302.2814,544.5349)
\lineto(302.2814,0.48818)
}
}
{
\newrgbcolor{curcolor}{0 0 0}
\pscustom[linewidth=0.97635722,linecolor=curcolor]
{
\newpath
\moveto(269.00574,544.5349)
\lineto(269.00574,0.48818)
}
}
{
\newrgbcolor{curcolor}{0 0 0}
\pscustom[linewidth=0.97635722,linecolor=curcolor]
{
\newpath
\moveto(235.73006,544.5349)
\lineto(235.73006,0.48818)
}
}
{
\newrgbcolor{curcolor}{0 0 0}
\pscustom[linewidth=0.97635722,linecolor=curcolor]
{
\newpath
\moveto(202.45439,544.5349)
\lineto(202.45439,0.48818)
}
}
{
\newrgbcolor{curcolor}{0 0 0}
\pscustom[linewidth=0.97635722,linecolor=curcolor]
{
\newpath
\moveto(169.17873,544.5349)
\lineto(169.17873,0.48818)
}
}
{
\newrgbcolor{curcolor}{0 0 0}
\pscustom[linewidth=0.97635722,linecolor=curcolor]
{
\newpath
\moveto(135.90306,544.5349)
\lineto(135.90306,0.48818)
}
}
{
\newrgbcolor{curcolor}{0 0 0}
\pscustom[linewidth=0.90066206,linecolor=curcolor]
{
\newpath
\moveto(353.07153,527.50448)
\lineto(353.07153,0.45038)
}
}
{
\newrgbcolor{curcolor}{0 0 0}
\pscustom[linewidth=0.90066206,linecolor=curcolor]
{
\newpath
\moveto(370.62386,527.50448)
\lineto(370.62386,0.45038)
}
}
{
\newrgbcolor{curcolor}{0 0 0}
\pscustom[linewidth=0.90066206,linecolor=curcolor]
{
\newpath
\moveto(388.17619,527.50448)
\lineto(388.17619,0.45038)
}
}
{
\newrgbcolor{curcolor}{0 0 0}
\pscustom[linewidth=0.90066206,linecolor=curcolor]
{
\newpath
\moveto(405.72853,527.50448)
\lineto(405.72853,0.45038)
}
}
{
\newrgbcolor{curcolor}{0 0 0}
\pscustom[linewidth=0.90066206,linecolor=curcolor]
{
\newpath
\moveto(423.28086,527.50448)
\lineto(423.28086,0.45038)
}
}
{
\newrgbcolor{curcolor}{0 0 0}
\pscustom[linewidth=0.90066206,linecolor=curcolor]
{
\newpath
\moveto(440.83319,527.50448)
\lineto(440.83319,0.45038)
}
}
{
\newrgbcolor{curcolor}{0 0 0}
\pscustom[linewidth=0.90066206,linecolor=curcolor]
{
\newpath
\moveto(458.38553,527.50448)
\lineto(458.38553,0.45038)
}
}
{
\newrgbcolor{curcolor}{0 0 0}
\pscustom[linewidth=0.90066206,linecolor=curcolor]
{
\newpath
\moveto(475.93786,527.50448)
\lineto(475.93786,0.45038)
}
}
{
\newrgbcolor{curcolor}{0 0 0}
\pscustom[linewidth=0.90066206,linecolor=curcolor]
{
\newpath
\moveto(493.4902,527.50448)
\lineto(493.4902,0.45038)
}
}
{
\newrgbcolor{curcolor}{0 0 0}
\pscustom[linewidth=0.90066206,linecolor=curcolor]
{
\newpath
\moveto(511.04253,527.50448)
\lineto(511.04253,0.45038)
}
}
{
\newrgbcolor{curcolor}{0 0 0}
\pscustom[linewidth=0.90066206,linecolor=curcolor]
{
\newpath
\moveto(528.59486,527.50448)
\lineto(528.59486,0.45038)
}
}
{
\newrgbcolor{curcolor}{0 0 0}
\pscustom[linewidth=0.90066206,linecolor=curcolor]
{
\newpath
\moveto(546.1472,527.50448)
\lineto(546.1472,0.45038)
}
}
{
\newrgbcolor{curcolor}{0 0 0}
\pscustom[linewidth=0.90066206,linecolor=curcolor]
{
\newpath
\moveto(563.69953,527.50448)
\lineto(563.69953,0.45038)
}
}
{
\newrgbcolor{curcolor}{0 0 0}
\pscustom[linewidth=0.90066206,linecolor=curcolor]
{
\newpath
\moveto(581.25186,527.50448)
\lineto(581.25186,0.45038)
}
}
{
\newrgbcolor{curcolor}{0 0 0}
\pscustom[linewidth=0.90066206,linecolor=curcolor]
{
\newpath
\moveto(598.8042,527.50448)
\lineto(598.8042,0.45038)
}
}
{
\newrgbcolor{curcolor}{0 0 0}
\pscustom[linewidth=0.90066206,linecolor=curcolor]
{
\newpath
\moveto(616.35653,527.50448)
\lineto(616.35653,0.45038)
}
}
{
\newrgbcolor{curcolor}{0 0 0}
\pscustom[linewidth=0.90066206,linecolor=curcolor]
{
\newpath
\moveto(633.90886,527.50448)
\lineto(633.90886,0.45038)
}
}
{
\newrgbcolor{curcolor}{0 0 0}
\pscustom[linewidth=0.90066206,linecolor=curcolor]
{
\newpath
\moveto(651.4612,527.50448)
\lineto(651.4612,0.45038)
}
}
{
\newrgbcolor{curcolor}{0 0 0}
\pscustom[linewidth=0.90066206,linecolor=curcolor]
{
\newpath
\moveto(669.01353,527.50448)
\lineto(669.01353,0.45038)
}
}
{
\newrgbcolor{curcolor}{0 0 0}
\pscustom[linewidth=0.90066206,linecolor=curcolor]
{
\newpath
\moveto(686.56586,527.50448)
\lineto(686.56586,0.45038)
}
}
{
\newrgbcolor{curcolor}{0 0 0}
\pscustom[linewidth=0.90066206,linecolor=curcolor]
{
\newpath
\moveto(704.1182,527.50448)
\lineto(704.1182,0.45038)
}
}
{
\newrgbcolor{curcolor}{0 0 0}
\pscustom[linewidth=0.90066206,linecolor=curcolor]
{
\newpath
\moveto(721.67053,527.50448)
\lineto(721.67053,0.45038)
}
}
{
\newrgbcolor{curcolor}{0 0 0}
\pscustom[linewidth=0.90066206,linecolor=curcolor]
{
\newpath
\moveto(739.22286,527.50448)
\lineto(739.22286,0.45038)
}
}
{
\newrgbcolor{curcolor}{0 0 0}
\pscustom[linewidth=0.90066206,linecolor=curcolor]
{
\newpath
\moveto(756.77521,527.50448)
\lineto(756.77521,0.45038)
}
}
{
\newrgbcolor{curcolor}{0 0 0}
\pscustom[linewidth=0.90066206,linecolor=curcolor]
{
\newpath
\moveto(774.32753,527.50448)
\lineto(774.32753,0.45038)
}
}
{
\newrgbcolor{curcolor}{0 0 0}
\pscustom[linewidth=0.90066206,linecolor=curcolor]
{
\newpath
\moveto(791.87985,527.50448)
\lineto(791.87985,0.45038)
}
}
{
\newrgbcolor{curcolor}{0 0 0}
\pscustom[linewidth=0.90066206,linecolor=curcolor]
{
\newpath
\moveto(809.43217,527.50448)
\lineto(809.43217,0.45038)
}
}
{
\newrgbcolor{curcolor}{0 0 0}
\pscustom[linewidth=0.90066206,linecolor=curcolor]
{
\newpath
\moveto(826.98457,527.50448)
\lineto(826.98457,0.45038)
}
}
{
\newrgbcolor{curcolor}{0 0 0}
\pscustom[linewidth=0.90066206,linecolor=curcolor]
{
\newpath
\moveto(844.53689,527.50448)
\lineto(844.53689,0.45038)
}
}
{
\newrgbcolor{curcolor}{0 0 0}
\pscustom[linewidth=0.90066206,linecolor=curcolor]
{
\newpath
\moveto(862.08921,527.50448)
\lineto(862.08921,0.45038)
}
}
{
\newrgbcolor{curcolor}{0 0 0}
\pscustom[linestyle=none,fillstyle=solid,fillcolor=curcolor]
{
\newpath
\moveto(7.10569843,513.8486302)
\lineto(11.41069843,513.8486302)
\curveto(14.95069489,513.8486302)(16.00069843,516.96863261)(16.00069843,519.3836302)
\curveto(16.00069843,522.48862709)(14.27569563,524.6186302)(11.47069843,524.6186302)
\lineto(7.10569843,524.6186302)
\lineto(7.10569843,513.8486302)
\moveto(8.56069843,523.3736302)
\lineto(11.27569843,523.3736302)
\curveto(13.25569645,523.3736302)(14.50069843,522.00862748)(14.50069843,519.2936302)
\curveto(14.50069843,516.57863291)(13.27069654,515.0936302)(11.38069843,515.0936302)
\lineto(8.56069843,515.0936302)
\lineto(8.56069843,523.3736302)
}
}
{
\newrgbcolor{curcolor}{0 0 0}
\pscustom[linestyle=none,fillstyle=solid,fillcolor=curcolor]
{
\newpath
\moveto(23.04554218,516.3086302)
\curveto(23.00054223,515.72363078)(22.26554094,514.7636302)(21.02054218,514.7636302)
\curveto(19.5055437,514.7636302)(18.74054218,515.70863183)(18.74054218,517.3436302)
\lineto(24.47054218,517.3436302)
\curveto(24.47054218,520.11862742)(23.36053992,521.9186302)(21.09554218,521.9186302)
\curveto(18.50054478,521.9186302)(17.33054218,519.98362777)(17.33054218,517.5536302)
\curveto(17.33054218,515.28863246)(18.63554439,513.6236302)(20.84054218,513.6236302)
\curveto(22.10054092,513.6236302)(22.61054254,513.92363044)(22.97054218,514.1636302)
\curveto(23.96054119,514.82362954)(24.32054223,515.93363057)(24.36554218,516.3086302)
\lineto(23.04554218,516.3086302)
\moveto(18.74054218,518.3936302)
\curveto(18.74054218,519.60862898)(19.7005434,520.7336302)(20.91554218,520.7336302)
\curveto(22.52054058,520.7336302)(23.03054226,519.60862898)(23.10554218,518.3936302)
\lineto(18.74054218,518.3936302)
}
}
{
\newrgbcolor{curcolor}{0 0 0}
\pscustom[linestyle=none,fillstyle=solid,fillcolor=curcolor]
{
\newpath
\moveto(31.78515156,519.4586302)
\curveto(31.78515156,519.84862981)(31.59014875,521.9186302)(28.78515156,521.9186302)
\curveto(27.2401531,521.9186302)(25.81515156,521.13862847)(25.81515156,519.4136302)
\curveto(25.81515156,518.33363128)(26.53515265,517.77862993)(27.63015156,517.5086302)
\lineto(29.16015156,517.1336302)
\curveto(30.28515043,516.84863048)(30.72015156,516.63862957)(30.72015156,516.0086302)
\curveto(30.72015156,515.13863107)(29.86515061,514.7636302)(28.92015156,514.7636302)
\curveto(27.06015342,514.7636302)(26.88015151,515.75363081)(26.83515156,516.3686302)
\lineto(25.56015156,516.3686302)
\curveto(25.60515151,515.42363114)(25.83015466,513.6236302)(28.93515156,513.6236302)
\curveto(30.70514979,513.6236302)(32.04015156,514.59863182)(32.04015156,516.2186302)
\curveto(32.04015156,517.28362913)(31.47014992,517.8836306)(29.83515156,518.2886302)
\lineto(28.51515156,518.6186302)
\curveto(27.49515258,518.87362994)(27.09015156,519.02363084)(27.09015156,519.6686302)
\curveto(27.09015156,520.64362922)(28.24515196,520.7786302)(28.65015156,520.7786302)
\curveto(30.31514989,520.7786302)(30.49515157,519.9536297)(30.51015156,519.4586302)
\lineto(31.78515156,519.4586302)
}
}
{
\newrgbcolor{curcolor}{0 0 0}
\pscustom[linestyle=none,fillstyle=solid,fillcolor=curcolor]
{
\newpath
\moveto(34.72515156,519.3086302)
\curveto(34.81515147,519.9086296)(35.02515306,520.8236302)(36.52515156,520.8236302)
\curveto(37.77015031,520.8236302)(38.37015156,520.37362937)(38.37015156,519.5486302)
\curveto(38.37015156,518.76863098)(37.99515124,518.64863017)(37.68015156,518.6186302)
\lineto(35.50515156,518.3486302)
\curveto(33.31515375,518.07863047)(33.12015156,516.54862954)(33.12015156,515.8886302)
\curveto(33.12015156,514.53863155)(34.140153,513.6236302)(35.58015156,513.6236302)
\curveto(37.11015003,513.6236302)(37.90515207,514.34363075)(38.41515156,514.8986302)
\curveto(38.46015151,514.2986308)(38.64015273,513.6986302)(39.81015156,513.6986302)
\curveto(40.11015126,513.6986302)(40.30515178,513.78863026)(40.53015156,513.8486302)
\lineto(40.53015156,514.8086302)
\curveto(40.38015171,514.77863023)(40.21515144,514.7486302)(40.09515156,514.7486302)
\curveto(39.82515183,514.7486302)(39.66015156,514.88363053)(39.66015156,515.2136302)
\lineto(39.66015156,519.7286302)
\curveto(39.66015156,521.73862819)(37.38015093,521.9186302)(36.75015156,521.9186302)
\curveto(34.81515349,521.9186302)(33.5701515,521.18362832)(33.51015156,519.3086302)
\lineto(34.72515156,519.3086302)
\moveto(38.34015156,516.5636302)
\curveto(38.34015156,515.51363125)(37.14015033,514.7186302)(35.91015156,514.7186302)
\curveto(34.92015255,514.7186302)(34.48515156,515.22863105)(34.48515156,516.0836302)
\curveto(34.48515156,517.07362921)(35.5201522,517.26863029)(36.16515156,517.3586302)
\curveto(37.80014992,517.56862999)(38.13015177,517.68863036)(38.34015156,517.8536302)
\lineto(38.34015156,516.5636302)
}
}
{
\newrgbcolor{curcolor}{0 0 0}
\pscustom[linestyle=none,fillstyle=solid,fillcolor=curcolor]
{
\newpath
\moveto(43.40476093,518.4086302)
\curveto(43.40476093,519.54862906)(44.18476216,520.5086302)(45.41476093,520.5086302)
\lineto(45.90976093,520.5086302)
\lineto(45.90976093,521.8736302)
\curveto(45.80476104,521.90363017)(45.72976077,521.9186302)(45.56476093,521.9186302)
\curveto(44.57476192,521.9186302)(43.88476041,521.30362928)(43.35976093,520.3886302)
\lineto(43.32976093,520.3886302)
\lineto(43.32976093,521.6936302)
\lineto(42.08476093,521.6936302)
\lineto(42.08476093,513.8486302)
\lineto(43.40476093,513.8486302)
\lineto(43.40476093,518.4086302)
}
}
{
\newrgbcolor{curcolor}{0 0 0}
\pscustom[linestyle=none,fillstyle=solid,fillcolor=curcolor]
{
\newpath
\moveto(48.41452656,518.4086302)
\curveto(48.41452656,519.54862906)(49.19452779,520.5086302)(50.42452656,520.5086302)
\lineto(50.91952656,520.5086302)
\lineto(50.91952656,521.8736302)
\curveto(50.81452666,521.90363017)(50.73952639,521.9186302)(50.57452656,521.9186302)
\curveto(49.58452755,521.9186302)(48.89452603,521.30362928)(48.36952656,520.3886302)
\lineto(48.33952656,520.3886302)
\lineto(48.33952656,521.6936302)
\lineto(47.09452656,521.6936302)
\lineto(47.09452656,513.8486302)
\lineto(48.41452656,513.8486302)
\lineto(48.41452656,518.4086302)
}
}
{
\newrgbcolor{curcolor}{0 0 0}
\pscustom[linestyle=none,fillstyle=solid,fillcolor=curcolor]
{
\newpath
\moveto(51.18132343,517.7786302)
\curveto(51.18132343,515.75363222)(52.32132594,513.6386302)(54.82632343,513.6386302)
\curveto(57.33132093,513.6386302)(58.47132343,515.75363222)(58.47132343,517.7786302)
\curveto(58.47132343,519.80362817)(57.33132093,521.9186302)(54.82632343,521.9186302)
\curveto(52.32132594,521.9186302)(51.18132343,519.80362817)(51.18132343,517.7786302)
\moveto(52.54632343,517.7786302)
\curveto(52.54632343,518.82862915)(52.93632532,520.7786302)(54.82632343,520.7786302)
\curveto(56.71632154,520.7786302)(57.10632343,518.82862915)(57.10632343,517.7786302)
\curveto(57.10632343,516.72863125)(56.71632154,514.7786302)(54.82632343,514.7786302)
\curveto(52.93632532,514.7786302)(52.54632343,516.72863125)(52.54632343,517.7786302)
}
}
{
\newrgbcolor{curcolor}{0 0 0}
\pscustom[linestyle=none,fillstyle=solid,fillcolor=curcolor]
{
\newpath
\moveto(61.33093281,524.6186302)
\lineto(60.01093281,524.6186302)
\lineto(60.01093281,513.8486302)
\lineto(61.33093281,513.8486302)
\lineto(61.33093281,524.6186302)
}
}
{
\newrgbcolor{curcolor}{0 0 0}
\pscustom[linestyle=none,fillstyle=solid,fillcolor=curcolor]
{
\newpath
\moveto(64.67077656,524.6186302)
\lineto(63.35077656,524.6186302)
\lineto(63.35077656,513.8486302)
\lineto(64.67077656,513.8486302)
\lineto(64.67077656,524.6186302)
}
}
{
\newrgbcolor{curcolor}{0 0 0}
\pscustom[linestyle=none,fillstyle=solid,fillcolor=curcolor]
{
\newpath
\moveto(67.83062031,519.3086302)
\curveto(67.92062022,519.9086296)(68.13062181,520.8236302)(69.63062031,520.8236302)
\curveto(70.87561906,520.8236302)(71.47562031,520.37362937)(71.47562031,519.5486302)
\curveto(71.47562031,518.76863098)(71.10061999,518.64863017)(70.78562031,518.6186302)
\lineto(68.61062031,518.3486302)
\curveto(66.4206225,518.07863047)(66.22562031,516.54862954)(66.22562031,515.8886302)
\curveto(66.22562031,514.53863155)(67.24562175,513.6236302)(68.68562031,513.6236302)
\curveto(70.21561878,513.6236302)(71.01062082,514.34363075)(71.52062031,514.8986302)
\curveto(71.56562026,514.2986308)(71.74562148,513.6986302)(72.91562031,513.6986302)
\curveto(73.21562001,513.6986302)(73.41062053,513.78863026)(73.63562031,513.8486302)
\lineto(73.63562031,514.8086302)
\curveto(73.48562046,514.77863023)(73.32062019,514.7486302)(73.20062031,514.7486302)
\curveto(72.93062058,514.7486302)(72.76562031,514.88363053)(72.76562031,515.2136302)
\lineto(72.76562031,519.7286302)
\curveto(72.76562031,521.73862819)(70.48561968,521.9186302)(69.85562031,521.9186302)
\curveto(67.92062224,521.9186302)(66.67562025,521.18362832)(66.61562031,519.3086302)
\lineto(67.83062031,519.3086302)
\moveto(71.44562031,516.5636302)
\curveto(71.44562031,515.51363125)(70.24561908,514.7186302)(69.01562031,514.7186302)
\curveto(68.0256213,514.7186302)(67.59062031,515.22863105)(67.59062031,516.0836302)
\curveto(67.59062031,517.07362921)(68.62562095,517.26863029)(69.27062031,517.3586302)
\curveto(70.90561867,517.56862999)(71.23562052,517.68863036)(71.44562031,517.8536302)
\lineto(71.44562031,516.5636302)
}
}
{
\newrgbcolor{curcolor}{0 0 0}
\pscustom[linestyle=none,fillstyle=solid,fillcolor=curcolor]
{
\newpath
\moveto(81.52022968,524.6186302)
\lineto(80.20022968,524.6186302)
\lineto(80.20022968,520.6886302)
\lineto(80.17022968,520.5836302)
\curveto(79.85523,521.03362975)(79.25522826,521.9186302)(77.83022968,521.9186302)
\curveto(75.74523177,521.9186302)(74.56022968,520.20862799)(74.56022968,518.0036302)
\curveto(74.56022968,516.12863207)(75.34023235,513.6236302)(78.01022968,513.6236302)
\curveto(78.77522892,513.6236302)(79.67523025,513.86363126)(80.24522968,514.9286302)
\lineto(80.27522968,514.9286302)
\lineto(80.27522968,513.8486302)
\lineto(81.52022968,513.8486302)
\lineto(81.52022968,524.6186302)
\moveto(75.92522968,517.7936302)
\curveto(75.92522968,518.79862919)(76.03023172,520.7336302)(78.07022968,520.7336302)
\curveto(79.97522778,520.7336302)(80.18522968,518.67862892)(80.18522968,517.4036302)
\curveto(80.18522968,515.31863228)(78.88022884,514.7636302)(78.04022968,514.7636302)
\curveto(76.60023112,514.7636302)(75.92522968,516.06863192)(75.92522968,517.7936302)
}
}
{
\newrgbcolor{curcolor}{0 0 0}
\pscustom[linestyle=none,fillstyle=solid,fillcolor=curcolor]
{
\newpath
\moveto(82.90983906,517.7786302)
\curveto(82.90983906,515.75363222)(84.04984156,513.6386302)(86.55483906,513.6386302)
\curveto(89.05983655,513.6386302)(90.19983906,515.75363222)(90.19983906,517.7786302)
\curveto(90.19983906,519.80362817)(89.05983655,521.9186302)(86.55483906,521.9186302)
\curveto(84.04984156,521.9186302)(82.90983906,519.80362817)(82.90983906,517.7786302)
\moveto(84.27483906,517.7786302)
\curveto(84.27483906,518.82862915)(84.66484095,520.7786302)(86.55483906,520.7786302)
\curveto(88.44483717,520.7786302)(88.83483906,518.82862915)(88.83483906,517.7786302)
\curveto(88.83483906,516.72863125)(88.44483717,514.7786302)(86.55483906,514.7786302)
\curveto(84.66484095,514.7786302)(84.27483906,516.72863125)(84.27483906,517.7786302)
}
}
{
\newrgbcolor{curcolor}{0 0 0}
\pscustom[linestyle=none,fillstyle=solid,fillcolor=curcolor]
{
\newpath
\moveto(93.20944843,518.4086302)
\curveto(93.20944843,519.54862906)(93.98944966,520.5086302)(95.21944843,520.5086302)
\lineto(95.71444843,520.5086302)
\lineto(95.71444843,521.8736302)
\curveto(95.60944854,521.90363017)(95.53444827,521.9186302)(95.36944843,521.9186302)
\curveto(94.37944942,521.9186302)(93.68944791,521.30362928)(93.16444843,520.3886302)
\lineto(93.13444843,520.3886302)
\lineto(93.13444843,521.6936302)
\lineto(91.88944843,521.6936302)
\lineto(91.88944843,513.8486302)
\lineto(93.20944843,513.8486302)
\lineto(93.20944843,518.4086302)
}
}
{
\newrgbcolor{curcolor}{0 0 0}
\pscustom[linestyle=none,fillstyle=solid,fillcolor=curcolor]
{
\newpath
\moveto(7.10569843,496.85839582)
\lineto(11.41069843,496.85839582)
\curveto(14.95069489,496.85839582)(16.00069843,499.97839824)(16.00069843,502.39339582)
\curveto(16.00069843,505.49839272)(14.27569563,507.62839582)(11.47069843,507.62839582)
\lineto(7.10569843,507.62839582)
\lineto(7.10569843,496.85839582)
\moveto(8.56069843,506.38339582)
\lineto(11.27569843,506.38339582)
\curveto(13.25569645,506.38339582)(14.50069843,505.01839311)(14.50069843,502.30339582)
\curveto(14.50069843,499.58839854)(13.27069654,498.10339582)(11.38069843,498.10339582)
\lineto(8.56069843,498.10339582)
\lineto(8.56069843,506.38339582)
}
}
{
\newrgbcolor{curcolor}{0 0 0}
\pscustom[linestyle=none,fillstyle=solid,fillcolor=curcolor]
{
\newpath
\moveto(23.04554218,499.31839582)
\curveto(23.00054223,498.73339641)(22.26554094,497.77339582)(21.02054218,497.77339582)
\curveto(19.5055437,497.77339582)(18.74054218,498.71839746)(18.74054218,500.35339582)
\lineto(24.47054218,500.35339582)
\curveto(24.47054218,503.12839305)(23.36053992,504.92839582)(21.09554218,504.92839582)
\curveto(18.50054478,504.92839582)(17.33054218,502.99339339)(17.33054218,500.56339582)
\curveto(17.33054218,498.29839809)(18.63554439,496.63339582)(20.84054218,496.63339582)
\curveto(22.10054092,496.63339582)(22.61054254,496.93339606)(22.97054218,497.17339582)
\curveto(23.96054119,497.83339516)(24.32054223,498.9433962)(24.36554218,499.31839582)
\lineto(23.04554218,499.31839582)
\moveto(18.74054218,501.40339582)
\curveto(18.74054218,502.61839461)(19.7005434,503.74339582)(20.91554218,503.74339582)
\curveto(22.52054058,503.74339582)(23.03054226,502.61839461)(23.10554218,501.40339582)
\lineto(18.74054218,501.40339582)
}
}
{
\newrgbcolor{curcolor}{0 0 0}
\pscustom[linestyle=none,fillstyle=solid,fillcolor=curcolor]
{
\newpath
\moveto(31.78515156,502.46839582)
\curveto(31.78515156,502.85839543)(31.59014875,504.92839582)(28.78515156,504.92839582)
\curveto(27.2401531,504.92839582)(25.81515156,504.1483941)(25.81515156,502.42339582)
\curveto(25.81515156,501.3433969)(26.53515265,500.78839555)(27.63015156,500.51839582)
\lineto(29.16015156,500.14339582)
\curveto(30.28515043,499.85839611)(30.72015156,499.64839519)(30.72015156,499.01839582)
\curveto(30.72015156,498.14839669)(29.86515061,497.77339582)(28.92015156,497.77339582)
\curveto(27.06015342,497.77339582)(26.88015151,498.76339644)(26.83515156,499.37839582)
\lineto(25.56015156,499.37839582)
\curveto(25.60515151,498.43339677)(25.83015466,496.63339582)(28.93515156,496.63339582)
\curveto(30.70514979,496.63339582)(32.04015156,497.60839744)(32.04015156,499.22839582)
\curveto(32.04015156,500.29339476)(31.47014992,500.89339623)(29.83515156,501.29839582)
\lineto(28.51515156,501.62839582)
\curveto(27.49515258,501.88339557)(27.09015156,502.03339647)(27.09015156,502.67839582)
\curveto(27.09015156,503.65339485)(28.24515196,503.78839582)(28.65015156,503.78839582)
\curveto(30.31514989,503.78839582)(30.49515157,502.96339533)(30.51015156,502.46839582)
\lineto(31.78515156,502.46839582)
}
}
{
\newrgbcolor{curcolor}{0 0 0}
\pscustom[linestyle=none,fillstyle=solid,fillcolor=curcolor]
{
\newpath
\moveto(34.72515156,502.31839582)
\curveto(34.81515147,502.91839522)(35.02515306,503.83339582)(36.52515156,503.83339582)
\curveto(37.77015031,503.83339582)(38.37015156,503.383395)(38.37015156,502.55839582)
\curveto(38.37015156,501.7783966)(37.99515124,501.65839579)(37.68015156,501.62839582)
\lineto(35.50515156,501.35839582)
\curveto(33.31515375,501.08839609)(33.12015156,499.55839516)(33.12015156,498.89839582)
\curveto(33.12015156,497.54839717)(34.140153,496.63339582)(35.58015156,496.63339582)
\curveto(37.11015003,496.63339582)(37.90515207,497.35339638)(38.41515156,497.90839582)
\curveto(38.46015151,497.30839642)(38.64015273,496.70839582)(39.81015156,496.70839582)
\curveto(40.11015126,496.70839582)(40.30515178,496.79839588)(40.53015156,496.85839582)
\lineto(40.53015156,497.81839582)
\curveto(40.38015171,497.78839585)(40.21515144,497.75839582)(40.09515156,497.75839582)
\curveto(39.82515183,497.75839582)(39.66015156,497.89339615)(39.66015156,498.22339582)
\lineto(39.66015156,502.73839582)
\curveto(39.66015156,504.74839381)(37.38015093,504.92839582)(36.75015156,504.92839582)
\curveto(34.81515349,504.92839582)(33.5701515,504.19339395)(33.51015156,502.31839582)
\lineto(34.72515156,502.31839582)
\moveto(38.34015156,499.57339582)
\curveto(38.34015156,498.52339687)(37.14015033,497.72839582)(35.91015156,497.72839582)
\curveto(34.92015255,497.72839582)(34.48515156,498.23839668)(34.48515156,499.09339582)
\curveto(34.48515156,500.08339483)(35.5201522,500.27839591)(36.16515156,500.36839582)
\curveto(37.80014992,500.57839561)(38.13015177,500.69839599)(38.34015156,500.86339582)
\lineto(38.34015156,499.57339582)
}
}
{
\newrgbcolor{curcolor}{0 0 0}
\pscustom[linestyle=none,fillstyle=solid,fillcolor=curcolor]
{
\newpath
\moveto(43.40476093,501.41839582)
\curveto(43.40476093,502.55839468)(44.18476216,503.51839582)(45.41476093,503.51839582)
\lineto(45.90976093,503.51839582)
\lineto(45.90976093,504.88339582)
\curveto(45.80476104,504.91339579)(45.72976077,504.92839582)(45.56476093,504.92839582)
\curveto(44.57476192,504.92839582)(43.88476041,504.31339491)(43.35976093,503.39839582)
\lineto(43.32976093,503.39839582)
\lineto(43.32976093,504.70339582)
\lineto(42.08476093,504.70339582)
\lineto(42.08476093,496.85839582)
\lineto(43.40476093,496.85839582)
\lineto(43.40476093,501.41839582)
}
}
{
\newrgbcolor{curcolor}{0 0 0}
\pscustom[linestyle=none,fillstyle=solid,fillcolor=curcolor]
{
\newpath
\moveto(48.41452656,501.41839582)
\curveto(48.41452656,502.55839468)(49.19452779,503.51839582)(50.42452656,503.51839582)
\lineto(50.91952656,503.51839582)
\lineto(50.91952656,504.88339582)
\curveto(50.81452666,504.91339579)(50.73952639,504.92839582)(50.57452656,504.92839582)
\curveto(49.58452755,504.92839582)(48.89452603,504.31339491)(48.36952656,503.39839582)
\lineto(48.33952656,503.39839582)
\lineto(48.33952656,504.70339582)
\lineto(47.09452656,504.70339582)
\lineto(47.09452656,496.85839582)
\lineto(48.41452656,496.85839582)
\lineto(48.41452656,501.41839582)
}
}
{
\newrgbcolor{curcolor}{0 0 0}
\pscustom[linestyle=none,fillstyle=solid,fillcolor=curcolor]
{
\newpath
\moveto(51.18132343,500.78839582)
\curveto(51.18132343,498.76339785)(52.32132594,496.64839582)(54.82632343,496.64839582)
\curveto(57.33132093,496.64839582)(58.47132343,498.76339785)(58.47132343,500.78839582)
\curveto(58.47132343,502.8133938)(57.33132093,504.92839582)(54.82632343,504.92839582)
\curveto(52.32132594,504.92839582)(51.18132343,502.8133938)(51.18132343,500.78839582)
\moveto(52.54632343,500.78839582)
\curveto(52.54632343,501.83839477)(52.93632532,503.78839582)(54.82632343,503.78839582)
\curveto(56.71632154,503.78839582)(57.10632343,501.83839477)(57.10632343,500.78839582)
\curveto(57.10632343,499.73839687)(56.71632154,497.78839582)(54.82632343,497.78839582)
\curveto(52.93632532,497.78839582)(52.54632343,499.73839687)(52.54632343,500.78839582)
}
}
{
\newrgbcolor{curcolor}{0 0 0}
\pscustom[linestyle=none,fillstyle=solid,fillcolor=curcolor]
{
\newpath
\moveto(61.33093281,507.62839582)
\lineto(60.01093281,507.62839582)
\lineto(60.01093281,496.85839582)
\lineto(61.33093281,496.85839582)
\lineto(61.33093281,507.62839582)
}
}
{
\newrgbcolor{curcolor}{0 0 0}
\pscustom[linestyle=none,fillstyle=solid,fillcolor=curcolor]
{
\newpath
\moveto(64.67077656,507.62839582)
\lineto(63.35077656,507.62839582)
\lineto(63.35077656,496.85839582)
\lineto(64.67077656,496.85839582)
\lineto(64.67077656,507.62839582)
}
}
{
\newrgbcolor{curcolor}{0 0 0}
\pscustom[linestyle=none,fillstyle=solid,fillcolor=curcolor]
{
\newpath
\moveto(67.83062031,502.31839582)
\curveto(67.92062022,502.91839522)(68.13062181,503.83339582)(69.63062031,503.83339582)
\curveto(70.87561906,503.83339582)(71.47562031,503.383395)(71.47562031,502.55839582)
\curveto(71.47562031,501.7783966)(71.10061999,501.65839579)(70.78562031,501.62839582)
\lineto(68.61062031,501.35839582)
\curveto(66.4206225,501.08839609)(66.22562031,499.55839516)(66.22562031,498.89839582)
\curveto(66.22562031,497.54839717)(67.24562175,496.63339582)(68.68562031,496.63339582)
\curveto(70.21561878,496.63339582)(71.01062082,497.35339638)(71.52062031,497.90839582)
\curveto(71.56562026,497.30839642)(71.74562148,496.70839582)(72.91562031,496.70839582)
\curveto(73.21562001,496.70839582)(73.41062053,496.79839588)(73.63562031,496.85839582)
\lineto(73.63562031,497.81839582)
\curveto(73.48562046,497.78839585)(73.32062019,497.75839582)(73.20062031,497.75839582)
\curveto(72.93062058,497.75839582)(72.76562031,497.89339615)(72.76562031,498.22339582)
\lineto(72.76562031,502.73839582)
\curveto(72.76562031,504.74839381)(70.48561968,504.92839582)(69.85562031,504.92839582)
\curveto(67.92062224,504.92839582)(66.67562025,504.19339395)(66.61562031,502.31839582)
\lineto(67.83062031,502.31839582)
\moveto(71.44562031,499.57339582)
\curveto(71.44562031,498.52339687)(70.24561908,497.72839582)(69.01562031,497.72839582)
\curveto(68.0256213,497.72839582)(67.59062031,498.23839668)(67.59062031,499.09339582)
\curveto(67.59062031,500.08339483)(68.62562095,500.27839591)(69.27062031,500.36839582)
\curveto(70.90561867,500.57839561)(71.23562052,500.69839599)(71.44562031,500.86339582)
\lineto(71.44562031,499.57339582)
}
}
{
\newrgbcolor{curcolor}{0 0 0}
\pscustom[linestyle=none,fillstyle=solid,fillcolor=curcolor]
{
\newpath
\moveto(81.52022968,507.62839582)
\lineto(80.20022968,507.62839582)
\lineto(80.20022968,503.69839582)
\lineto(80.17022968,503.59339582)
\curveto(79.85523,504.04339537)(79.25522826,504.92839582)(77.83022968,504.92839582)
\curveto(75.74523177,504.92839582)(74.56022968,503.21839362)(74.56022968,501.01339582)
\curveto(74.56022968,499.1383977)(75.34023235,496.63339582)(78.01022968,496.63339582)
\curveto(78.77522892,496.63339582)(79.67523025,496.87339689)(80.24522968,497.93839582)
\lineto(80.27522968,497.93839582)
\lineto(80.27522968,496.85839582)
\lineto(81.52022968,496.85839582)
\lineto(81.52022968,507.62839582)
\moveto(75.92522968,500.80339582)
\curveto(75.92522968,501.80839482)(76.03023172,503.74339582)(78.07022968,503.74339582)
\curveto(79.97522778,503.74339582)(80.18522968,501.68839455)(80.18522968,500.41339582)
\curveto(80.18522968,498.32839791)(78.88022884,497.77339582)(78.04022968,497.77339582)
\curveto(76.60023112,497.77339582)(75.92522968,499.07839755)(75.92522968,500.80339582)
}
}
{
\newrgbcolor{curcolor}{0 0 0}
\pscustom[linestyle=none,fillstyle=solid,fillcolor=curcolor]
{
\newpath
\moveto(82.90983906,500.78839582)
\curveto(82.90983906,498.76339785)(84.04984156,496.64839582)(86.55483906,496.64839582)
\curveto(89.05983655,496.64839582)(90.19983906,498.76339785)(90.19983906,500.78839582)
\curveto(90.19983906,502.8133938)(89.05983655,504.92839582)(86.55483906,504.92839582)
\curveto(84.04984156,504.92839582)(82.90983906,502.8133938)(82.90983906,500.78839582)
\moveto(84.27483906,500.78839582)
\curveto(84.27483906,501.83839477)(84.66484095,503.78839582)(86.55483906,503.78839582)
\curveto(88.44483717,503.78839582)(88.83483906,501.83839477)(88.83483906,500.78839582)
\curveto(88.83483906,499.73839687)(88.44483717,497.78839582)(86.55483906,497.78839582)
\curveto(84.66484095,497.78839582)(84.27483906,499.73839687)(84.27483906,500.78839582)
}
}
{
\newrgbcolor{curcolor}{0 0 0}
\pscustom[linestyle=none,fillstyle=solid,fillcolor=curcolor]
{
\newpath
\moveto(93.20944843,501.41839582)
\curveto(93.20944843,502.55839468)(93.98944966,503.51839582)(95.21944843,503.51839582)
\lineto(95.71444843,503.51839582)
\lineto(95.71444843,504.88339582)
\curveto(95.60944854,504.91339579)(95.53444827,504.92839582)(95.36944843,504.92839582)
\curveto(94.37944942,504.92839582)(93.68944791,504.31339491)(93.16444843,503.39839582)
\lineto(93.13444843,503.39839582)
\lineto(93.13444843,504.70339582)
\lineto(91.88944843,504.70339582)
\lineto(91.88944843,496.85839582)
\lineto(93.20944843,496.85839582)
\lineto(93.20944843,501.41839582)
}
}
{
\newrgbcolor{curcolor}{0 0 0}
\pscustom[linestyle=none,fillstyle=solid,fillcolor=curcolor]
{
\newpath
\moveto(14.17069843,398.05205061)
\lineto(15.23569843,394.91705061)
\lineto(16.82569843,394.91705061)
\lineto(12.92569843,405.68705061)
\lineto(11.27569843,405.68705061)
\lineto(7.22569843,394.91705061)
\lineto(8.72569843,394.91705061)
\lineto(9.85069843,398.05205061)
\lineto(14.17069843,398.05205061)
\moveto(10.30069843,399.34205061)
\lineto(12.02569843,404.08205061)
\lineto(12.05569843,404.08205061)
\lineto(13.64569843,399.34205061)
\lineto(10.30069843,399.34205061)
}
}
{
\newrgbcolor{curcolor}{0 0 0}
\pscustom[linestyle=none,fillstyle=solid,fillcolor=curcolor]
{
\newpath
\moveto(24.22726093,405.68705061)
\lineto(22.90726093,405.68705061)
\lineto(22.90726093,401.75705061)
\lineto(22.87726093,401.65205061)
\curveto(22.56226125,402.10205016)(21.96225951,402.98705061)(20.53726093,402.98705061)
\curveto(18.45226302,402.98705061)(17.26726093,401.2770484)(17.26726093,399.07205061)
\curveto(17.26726093,397.19705248)(18.0472636,394.69205061)(20.71726093,394.69205061)
\curveto(21.48226017,394.69205061)(22.3822615,394.93205167)(22.95226093,395.99705061)
\lineto(22.98226093,395.99705061)
\lineto(22.98226093,394.91705061)
\lineto(24.22726093,394.91705061)
\lineto(24.22726093,405.68705061)
\moveto(18.63226093,398.86205061)
\curveto(18.63226093,399.8670496)(18.73726297,401.80205061)(20.77726093,401.80205061)
\curveto(22.68225903,401.80205061)(22.89226093,399.74704933)(22.89226093,398.47205061)
\curveto(22.89226093,396.38705269)(21.58726009,395.83205061)(20.74726093,395.83205061)
\curveto(19.30726237,395.83205061)(18.63226093,397.13705233)(18.63226093,398.86205061)
}
}
{
\newrgbcolor{curcolor}{0 0 0}
\pscustom[linestyle=none,fillstyle=solid,fillcolor=curcolor]
{
\newpath
\moveto(26.06687031,394.91705061)
\lineto(27.38687031,394.91705061)
\lineto(27.38687031,399.19205061)
\curveto(27.38687031,401.32204848)(28.70687106,401.80205061)(29.45687031,401.80205061)
\curveto(30.43186933,401.80205061)(30.68687031,401.00704995)(30.68687031,400.34705061)
\lineto(30.68687031,394.91705061)
\lineto(32.00687031,394.91705061)
\lineto(32.00687031,399.70205061)
\curveto(32.00687031,400.75204956)(32.74187142,401.80205061)(33.85187031,401.80205061)
\curveto(34.97686918,401.80205061)(35.30687031,401.06704953)(35.30687031,399.98705061)
\lineto(35.30687031,394.91705061)
\lineto(36.62687031,394.91705061)
\lineto(36.62687031,400.34705061)
\curveto(36.62687031,402.5520484)(35.03686947,402.98705061)(34.19687031,402.98705061)
\curveto(32.98187152,402.98705061)(32.45686965,402.44704987)(31.79687031,401.71205061)
\curveto(31.57187053,402.13205019)(31.1218689,402.98705061)(29.71187031,402.98705061)
\curveto(28.30187172,402.98705061)(27.62687002,402.07205019)(27.34187031,401.65205061)
\lineto(27.31187031,401.65205061)
\lineto(27.31187031,402.76205061)
\lineto(26.06687031,402.76205061)
\lineto(26.06687031,394.91705061)
}
}
{
\newrgbcolor{curcolor}{0 0 0}
\pscustom[linestyle=none,fillstyle=solid,fillcolor=curcolor]
{
\newpath
\moveto(39.92663593,402.76205061)
\lineto(38.60663593,402.76205061)
\lineto(38.60663593,394.91705061)
\lineto(39.92663593,394.91705061)
\lineto(39.92663593,402.76205061)
\moveto(39.92663593,404.18705061)
\lineto(39.92663593,405.68705061)
\lineto(38.60663593,405.68705061)
\lineto(38.60663593,404.18705061)
\lineto(39.92663593,404.18705061)
}
}
{
\newrgbcolor{curcolor}{0 0 0}
\pscustom[linestyle=none,fillstyle=solid,fillcolor=curcolor]
{
\newpath
\moveto(48.30647968,400.25705061)
\curveto(48.30647968,402.49204837)(46.77647847,402.98705061)(45.56147968,402.98705061)
\curveto(44.21148103,402.98705061)(43.4764794,402.07205019)(43.19147968,401.65205061)
\lineto(43.16147968,401.65205061)
\lineto(43.16147968,402.76205061)
\lineto(41.91647968,402.76205061)
\lineto(41.91647968,394.91705061)
\lineto(43.23647968,394.91705061)
\lineto(43.23647968,399.19205061)
\curveto(43.23647968,401.32204848)(44.55648043,401.80205061)(45.30647968,401.80205061)
\curveto(46.59647839,401.80205061)(46.98647968,401.11204924)(46.98647968,399.74705061)
\lineto(46.98647968,394.91705061)
\lineto(48.30647968,394.91705061)
\lineto(48.30647968,400.25705061)
}
}
{
\newrgbcolor{curcolor}{0 0 0}
\pscustom[linestyle=none,fillstyle=solid,fillcolor=curcolor]
{
\newpath
\moveto(51.61608906,402.76205061)
\lineto(50.29608906,402.76205061)
\lineto(50.29608906,394.91705061)
\lineto(51.61608906,394.91705061)
\lineto(51.61608906,402.76205061)
\moveto(51.61608906,404.18705061)
\lineto(51.61608906,405.68705061)
\lineto(50.29608906,405.68705061)
\lineto(50.29608906,404.18705061)
\lineto(51.61608906,404.18705061)
}
}
{
\newrgbcolor{curcolor}{0 0 0}
\pscustom[linestyle=none,fillstyle=solid,fillcolor=curcolor]
{
\newpath
\moveto(59.33593281,400.52705061)
\curveto(59.33593281,400.91705022)(59.14093,402.98705061)(56.33593281,402.98705061)
\curveto(54.79093435,402.98705061)(53.36593281,402.20704888)(53.36593281,400.48205061)
\curveto(53.36593281,399.40205169)(54.0859339,398.84705034)(55.18093281,398.57705061)
\lineto(56.71093281,398.20205061)
\curveto(57.83593168,397.91705089)(58.27093281,397.70704998)(58.27093281,397.07705061)
\curveto(58.27093281,396.20705148)(57.41593186,395.83205061)(56.47093281,395.83205061)
\curveto(54.61093467,395.83205061)(54.43093276,396.82205122)(54.38593281,397.43705061)
\lineto(53.11093281,397.43705061)
\curveto(53.15593276,396.49205155)(53.38093591,394.69205061)(56.48593281,394.69205061)
\curveto(58.25593104,394.69205061)(59.59093281,395.66705223)(59.59093281,397.28705061)
\curveto(59.59093281,398.35204954)(59.02093117,398.95205101)(57.38593281,399.35705061)
\lineto(56.06593281,399.68705061)
\curveto(55.04593383,399.94205035)(54.64093281,400.09205125)(54.64093281,400.73705061)
\curveto(54.64093281,401.71204963)(55.79593321,401.84705061)(56.20093281,401.84705061)
\curveto(57.86593114,401.84705061)(58.04593282,401.02205011)(58.06093281,400.52705061)
\lineto(59.33593281,400.52705061)
}
}
{
\newrgbcolor{curcolor}{0 0 0}
\pscustom[linestyle=none,fillstyle=solid,fillcolor=curcolor]
{
\newpath
\moveto(63.98593281,401.66705061)
\lineto(63.98593281,402.76205061)
\lineto(62.72593281,402.76205061)
\lineto(62.72593281,404.95205061)
\lineto(61.40593281,404.95205061)
\lineto(61.40593281,402.76205061)
\lineto(60.34093281,402.76205061)
\lineto(60.34093281,401.66705061)
\lineto(61.40593281,401.66705061)
\lineto(61.40593281,396.49205061)
\curveto(61.40593281,395.54705155)(61.69093411,394.81205061)(62.99593281,394.81205061)
\curveto(63.13093267,394.81205061)(63.50593329,394.87205065)(63.98593281,394.91705061)
\lineto(63.98593281,395.95205061)
\lineto(63.52093281,395.95205061)
\curveto(63.25093308,395.95205061)(62.72593281,395.95205122)(62.72593281,396.56705061)
\lineto(62.72593281,401.66705061)
\lineto(63.98593281,401.66705061)
}
}
{
\newrgbcolor{curcolor}{0 0 0}
\pscustom[linestyle=none,fillstyle=solid,fillcolor=curcolor]
{
\newpath
\moveto(66.76608906,399.47705061)
\curveto(66.76608906,400.61704947)(67.54609029,401.57705061)(68.77608906,401.57705061)
\lineto(69.27108906,401.57705061)
\lineto(69.27108906,402.94205061)
\curveto(69.16608916,402.97205058)(69.09108889,402.98705061)(68.92608906,402.98705061)
\curveto(67.93609005,402.98705061)(67.24608853,402.37204969)(66.72108906,401.45705061)
\lineto(66.69108906,401.45705061)
\lineto(66.69108906,402.76205061)
\lineto(65.44608906,402.76205061)
\lineto(65.44608906,394.91705061)
\lineto(66.76608906,394.91705061)
\lineto(66.76608906,399.47705061)
}
}
{
\newrgbcolor{curcolor}{0 0 0}
\pscustom[linestyle=none,fillstyle=solid,fillcolor=curcolor]
{
\newpath
\moveto(71.29937031,400.37705061)
\curveto(71.38937022,400.97705001)(71.59937181,401.89205061)(73.09937031,401.89205061)
\curveto(74.34436906,401.89205061)(74.94437031,401.44204978)(74.94437031,400.61705061)
\curveto(74.94437031,399.83705139)(74.56936999,399.71705058)(74.25437031,399.68705061)
\lineto(72.07937031,399.41705061)
\curveto(69.8893725,399.14705088)(69.69437031,397.61704995)(69.69437031,396.95705061)
\curveto(69.69437031,395.60705196)(70.71437175,394.69205061)(72.15437031,394.69205061)
\curveto(73.68436878,394.69205061)(74.47937082,395.41205116)(74.98937031,395.96705061)
\curveto(75.03437026,395.36705121)(75.21437148,394.76705061)(76.38437031,394.76705061)
\curveto(76.68437001,394.76705061)(76.87937053,394.85705067)(77.10437031,394.91705061)
\lineto(77.10437031,395.87705061)
\curveto(76.95437046,395.84705064)(76.78937019,395.81705061)(76.66937031,395.81705061)
\curveto(76.39937058,395.81705061)(76.23437031,395.95205094)(76.23437031,396.28205061)
\lineto(76.23437031,400.79705061)
\curveto(76.23437031,402.8070486)(73.95436968,402.98705061)(73.32437031,402.98705061)
\curveto(71.38937224,402.98705061)(70.14437025,402.25204873)(70.08437031,400.37705061)
\lineto(71.29937031,400.37705061)
\moveto(74.91437031,397.63205061)
\curveto(74.91437031,396.58205166)(73.71436908,395.78705061)(72.48437031,395.78705061)
\curveto(71.4943713,395.78705061)(71.05937031,396.29705146)(71.05937031,397.15205061)
\curveto(71.05937031,398.14204962)(72.09437095,398.3370507)(72.73937031,398.42705061)
\curveto(74.37436867,398.6370504)(74.70437052,398.75705077)(74.91437031,398.92205061)
\lineto(74.91437031,397.63205061)
}
}
{
\newrgbcolor{curcolor}{0 0 0}
\pscustom[linestyle=none,fillstyle=solid,fillcolor=curcolor]
{
\newpath
\moveto(84.98897968,405.68705061)
\lineto(83.66897968,405.68705061)
\lineto(83.66897968,401.75705061)
\lineto(83.63897968,401.65205061)
\curveto(83.32398,402.10205016)(82.72397826,402.98705061)(81.29897968,402.98705061)
\curveto(79.21398177,402.98705061)(78.02897968,401.2770484)(78.02897968,399.07205061)
\curveto(78.02897968,397.19705248)(78.80898235,394.69205061)(81.47897968,394.69205061)
\curveto(82.24397892,394.69205061)(83.14398025,394.93205167)(83.71397968,395.99705061)
\lineto(83.74397968,395.99705061)
\lineto(83.74397968,394.91705061)
\lineto(84.98897968,394.91705061)
\lineto(84.98897968,405.68705061)
\moveto(79.39397968,398.86205061)
\curveto(79.39397968,399.8670496)(79.49898172,401.80205061)(81.53897968,401.80205061)
\curveto(83.44397778,401.80205061)(83.65397968,399.74704933)(83.65397968,398.47205061)
\curveto(83.65397968,396.38705269)(82.34897884,395.83205061)(81.50897968,395.83205061)
\curveto(80.06898112,395.83205061)(79.39397968,397.13705233)(79.39397968,398.86205061)
}
}
{
\newrgbcolor{curcolor}{0 0 0}
\pscustom[linestyle=none,fillstyle=solid,fillcolor=curcolor]
{
\newpath
\moveto(86.37858906,398.84705061)
\curveto(86.37858906,396.82205263)(87.51859156,394.70705061)(90.02358906,394.70705061)
\curveto(92.52858655,394.70705061)(93.66858906,396.82205263)(93.66858906,398.84705061)
\curveto(93.66858906,400.87204858)(92.52858655,402.98705061)(90.02358906,402.98705061)
\curveto(87.51859156,402.98705061)(86.37858906,400.87204858)(86.37858906,398.84705061)
\moveto(87.74358906,398.84705061)
\curveto(87.74358906,399.89704956)(88.13359095,401.84705061)(90.02358906,401.84705061)
\curveto(91.91358717,401.84705061)(92.30358906,399.89704956)(92.30358906,398.84705061)
\curveto(92.30358906,397.79705166)(91.91358717,395.84705061)(90.02358906,395.84705061)
\curveto(88.13359095,395.84705061)(87.74358906,397.79705166)(87.74358906,398.84705061)
}
}
{
\newrgbcolor{curcolor}{0 0 0}
\pscustom[linestyle=none,fillstyle=solid,fillcolor=curcolor]
{
\newpath
\moveto(96.67819843,399.47705061)
\curveto(96.67819843,400.61704947)(97.45819966,401.57705061)(98.68819843,401.57705061)
\lineto(99.18319843,401.57705061)
\lineto(99.18319843,402.94205061)
\curveto(99.07819854,402.97205058)(99.00319827,402.98705061)(98.83819843,402.98705061)
\curveto(97.84819942,402.98705061)(97.15819791,402.37204969)(96.63319843,401.45705061)
\lineto(96.60319843,401.45705061)
\lineto(96.60319843,402.76205061)
\lineto(95.35819843,402.76205061)
\lineto(95.35819843,394.91705061)
\lineto(96.67819843,394.91705061)
\lineto(96.67819843,399.47705061)
}
}
{
\newrgbcolor{curcolor}{0 0 0}
\pscustom[linestyle=none,fillstyle=solid,fillcolor=curcolor]
{
\newpath
\moveto(7.10569843,377.92681623)
\lineto(11.41069843,377.92681623)
\curveto(14.95069489,377.92681623)(16.00069843,381.04681865)(16.00069843,383.46181623)
\curveto(16.00069843,386.56681313)(14.27569563,388.69681623)(11.47069843,388.69681623)
\lineto(7.10569843,388.69681623)
\lineto(7.10569843,377.92681623)
\moveto(8.56069843,387.45181623)
\lineto(11.27569843,387.45181623)
\curveto(13.25569645,387.45181623)(14.50069843,386.08681352)(14.50069843,383.37181623)
\curveto(14.50069843,380.65681895)(13.27069654,379.17181623)(11.38069843,379.17181623)
\lineto(8.56069843,379.17181623)
\lineto(8.56069843,387.45181623)
}
}
{
\newrgbcolor{curcolor}{0 0 0}
\pscustom[linestyle=none,fillstyle=solid,fillcolor=curcolor]
{
\newpath
\moveto(23.04554218,380.38681623)
\curveto(23.00054223,379.80181682)(22.26554094,378.84181623)(21.02054218,378.84181623)
\curveto(19.5055437,378.84181623)(18.74054218,379.78681787)(18.74054218,381.42181623)
\lineto(24.47054218,381.42181623)
\curveto(24.47054218,384.19681346)(23.36053992,385.99681623)(21.09554218,385.99681623)
\curveto(18.50054478,385.99681623)(17.33054218,384.0618138)(17.33054218,381.63181623)
\curveto(17.33054218,379.3668185)(18.63554439,377.70181623)(20.84054218,377.70181623)
\curveto(22.10054092,377.70181623)(22.61054254,378.00181647)(22.97054218,378.24181623)
\curveto(23.96054119,378.90181557)(24.32054223,380.01181661)(24.36554218,380.38681623)
\lineto(23.04554218,380.38681623)
\moveto(18.74054218,382.47181623)
\curveto(18.74054218,383.68681502)(19.7005434,384.81181623)(20.91554218,384.81181623)
\curveto(22.52054058,384.81181623)(23.03054226,383.68681502)(23.10554218,382.47181623)
\lineto(18.74054218,382.47181623)
}
}
{
\newrgbcolor{curcolor}{0 0 0}
\pscustom[linestyle=none,fillstyle=solid,fillcolor=curcolor]
{
\newpath
\moveto(31.78515156,383.53681623)
\curveto(31.78515156,383.92681584)(31.59014875,385.99681623)(28.78515156,385.99681623)
\curveto(27.2401531,385.99681623)(25.81515156,385.21681451)(25.81515156,383.49181623)
\curveto(25.81515156,382.41181731)(26.53515265,381.85681596)(27.63015156,381.58681623)
\lineto(29.16015156,381.21181623)
\curveto(30.28515043,380.92681652)(30.72015156,380.7168156)(30.72015156,380.08681623)
\curveto(30.72015156,379.2168171)(29.86515061,378.84181623)(28.92015156,378.84181623)
\curveto(27.06015342,378.84181623)(26.88015151,379.83181685)(26.83515156,380.44681623)
\lineto(25.56015156,380.44681623)
\curveto(25.60515151,379.50181718)(25.83015466,377.70181623)(28.93515156,377.70181623)
\curveto(30.70514979,377.70181623)(32.04015156,378.67681785)(32.04015156,380.29681623)
\curveto(32.04015156,381.36181517)(31.47014992,381.96181664)(29.83515156,382.36681623)
\lineto(28.51515156,382.69681623)
\curveto(27.49515258,382.95181598)(27.09015156,383.10181688)(27.09015156,383.74681623)
\curveto(27.09015156,384.72181526)(28.24515196,384.85681623)(28.65015156,384.85681623)
\curveto(30.31514989,384.85681623)(30.49515157,384.03181574)(30.51015156,383.53681623)
\lineto(31.78515156,383.53681623)
}
}
{
\newrgbcolor{curcolor}{0 0 0}
\pscustom[linestyle=none,fillstyle=solid,fillcolor=curcolor]
{
\newpath
\moveto(34.72515156,383.38681623)
\curveto(34.81515147,383.98681563)(35.02515306,384.90181623)(36.52515156,384.90181623)
\curveto(37.77015031,384.90181623)(38.37015156,384.45181541)(38.37015156,383.62681623)
\curveto(38.37015156,382.84681701)(37.99515124,382.7268162)(37.68015156,382.69681623)
\lineto(35.50515156,382.42681623)
\curveto(33.31515375,382.1568165)(33.12015156,380.62681557)(33.12015156,379.96681623)
\curveto(33.12015156,378.61681758)(34.140153,377.70181623)(35.58015156,377.70181623)
\curveto(37.11015003,377.70181623)(37.90515207,378.42181679)(38.41515156,378.97681623)
\curveto(38.46015151,378.37681683)(38.64015273,377.77681623)(39.81015156,377.77681623)
\curveto(40.11015126,377.77681623)(40.30515178,377.86681629)(40.53015156,377.92681623)
\lineto(40.53015156,378.88681623)
\curveto(40.38015171,378.85681626)(40.21515144,378.82681623)(40.09515156,378.82681623)
\curveto(39.82515183,378.82681623)(39.66015156,378.96181656)(39.66015156,379.29181623)
\lineto(39.66015156,383.80681623)
\curveto(39.66015156,385.81681422)(37.38015093,385.99681623)(36.75015156,385.99681623)
\curveto(34.81515349,385.99681623)(33.5701515,385.26181436)(33.51015156,383.38681623)
\lineto(34.72515156,383.38681623)
\moveto(38.34015156,380.64181623)
\curveto(38.34015156,379.59181728)(37.14015033,378.79681623)(35.91015156,378.79681623)
\curveto(34.92015255,378.79681623)(34.48515156,379.30681709)(34.48515156,380.16181623)
\curveto(34.48515156,381.15181524)(35.5201522,381.34681632)(36.16515156,381.43681623)
\curveto(37.80014992,381.64681602)(38.13015177,381.7668164)(38.34015156,381.93181623)
\lineto(38.34015156,380.64181623)
}
}
{
\newrgbcolor{curcolor}{0 0 0}
\pscustom[linestyle=none,fillstyle=solid,fillcolor=curcolor]
{
\newpath
\moveto(43.40476093,382.48681623)
\curveto(43.40476093,383.62681509)(44.18476216,384.58681623)(45.41476093,384.58681623)
\lineto(45.90976093,384.58681623)
\lineto(45.90976093,385.95181623)
\curveto(45.80476104,385.9818162)(45.72976077,385.99681623)(45.56476093,385.99681623)
\curveto(44.57476192,385.99681623)(43.88476041,385.38181532)(43.35976093,384.46681623)
\lineto(43.32976093,384.46681623)
\lineto(43.32976093,385.77181623)
\lineto(42.08476093,385.77181623)
\lineto(42.08476093,377.92681623)
\lineto(43.40476093,377.92681623)
\lineto(43.40476093,382.48681623)
}
}
{
\newrgbcolor{curcolor}{0 0 0}
\pscustom[linestyle=none,fillstyle=solid,fillcolor=curcolor]
{
\newpath
\moveto(48.41452656,382.48681623)
\curveto(48.41452656,383.62681509)(49.19452779,384.58681623)(50.42452656,384.58681623)
\lineto(50.91952656,384.58681623)
\lineto(50.91952656,385.95181623)
\curveto(50.81452666,385.9818162)(50.73952639,385.99681623)(50.57452656,385.99681623)
\curveto(49.58452755,385.99681623)(48.89452603,385.38181532)(48.36952656,384.46681623)
\lineto(48.33952656,384.46681623)
\lineto(48.33952656,385.77181623)
\lineto(47.09452656,385.77181623)
\lineto(47.09452656,377.92681623)
\lineto(48.41452656,377.92681623)
\lineto(48.41452656,382.48681623)
}
}
{
\newrgbcolor{curcolor}{0 0 0}
\pscustom[linestyle=none,fillstyle=solid,fillcolor=curcolor]
{
\newpath
\moveto(51.18132343,381.85681623)
\curveto(51.18132343,379.83181826)(52.32132594,377.71681623)(54.82632343,377.71681623)
\curveto(57.33132093,377.71681623)(58.47132343,379.83181826)(58.47132343,381.85681623)
\curveto(58.47132343,383.88181421)(57.33132093,385.99681623)(54.82632343,385.99681623)
\curveto(52.32132594,385.99681623)(51.18132343,383.88181421)(51.18132343,381.85681623)
\moveto(52.54632343,381.85681623)
\curveto(52.54632343,382.90681518)(52.93632532,384.85681623)(54.82632343,384.85681623)
\curveto(56.71632154,384.85681623)(57.10632343,382.90681518)(57.10632343,381.85681623)
\curveto(57.10632343,380.80681728)(56.71632154,378.85681623)(54.82632343,378.85681623)
\curveto(52.93632532,378.85681623)(52.54632343,380.80681728)(52.54632343,381.85681623)
}
}
{
\newrgbcolor{curcolor}{0 0 0}
\pscustom[linestyle=none,fillstyle=solid,fillcolor=curcolor]
{
\newpath
\moveto(61.33093281,388.69681623)
\lineto(60.01093281,388.69681623)
\lineto(60.01093281,377.92681623)
\lineto(61.33093281,377.92681623)
\lineto(61.33093281,388.69681623)
}
}
{
\newrgbcolor{curcolor}{0 0 0}
\pscustom[linestyle=none,fillstyle=solid,fillcolor=curcolor]
{
\newpath
\moveto(64.67077656,388.69681623)
\lineto(63.35077656,388.69681623)
\lineto(63.35077656,377.92681623)
\lineto(64.67077656,377.92681623)
\lineto(64.67077656,388.69681623)
}
}
{
\newrgbcolor{curcolor}{0 0 0}
\pscustom[linestyle=none,fillstyle=solid,fillcolor=curcolor]
{
\newpath
\moveto(67.83062031,383.38681623)
\curveto(67.92062022,383.98681563)(68.13062181,384.90181623)(69.63062031,384.90181623)
\curveto(70.87561906,384.90181623)(71.47562031,384.45181541)(71.47562031,383.62681623)
\curveto(71.47562031,382.84681701)(71.10061999,382.7268162)(70.78562031,382.69681623)
\lineto(68.61062031,382.42681623)
\curveto(66.4206225,382.1568165)(66.22562031,380.62681557)(66.22562031,379.96681623)
\curveto(66.22562031,378.61681758)(67.24562175,377.70181623)(68.68562031,377.70181623)
\curveto(70.21561878,377.70181623)(71.01062082,378.42181679)(71.52062031,378.97681623)
\curveto(71.56562026,378.37681683)(71.74562148,377.77681623)(72.91562031,377.77681623)
\curveto(73.21562001,377.77681623)(73.41062053,377.86681629)(73.63562031,377.92681623)
\lineto(73.63562031,378.88681623)
\curveto(73.48562046,378.85681626)(73.32062019,378.82681623)(73.20062031,378.82681623)
\curveto(72.93062058,378.82681623)(72.76562031,378.96181656)(72.76562031,379.29181623)
\lineto(72.76562031,383.80681623)
\curveto(72.76562031,385.81681422)(70.48561968,385.99681623)(69.85562031,385.99681623)
\curveto(67.92062224,385.99681623)(66.67562025,385.26181436)(66.61562031,383.38681623)
\lineto(67.83062031,383.38681623)
\moveto(71.44562031,380.64181623)
\curveto(71.44562031,379.59181728)(70.24561908,378.79681623)(69.01562031,378.79681623)
\curveto(68.0256213,378.79681623)(67.59062031,379.30681709)(67.59062031,380.16181623)
\curveto(67.59062031,381.15181524)(68.62562095,381.34681632)(69.27062031,381.43681623)
\curveto(70.90561867,381.64681602)(71.23562052,381.7668164)(71.44562031,381.93181623)
\lineto(71.44562031,380.64181623)
}
}
{
\newrgbcolor{curcolor}{0 0 0}
\pscustom[linestyle=none,fillstyle=solid,fillcolor=curcolor]
{
\newpath
\moveto(81.52022968,388.69681623)
\lineto(80.20022968,388.69681623)
\lineto(80.20022968,384.76681623)
\lineto(80.17022968,384.66181623)
\curveto(79.85523,385.11181578)(79.25522826,385.99681623)(77.83022968,385.99681623)
\curveto(75.74523177,385.99681623)(74.56022968,384.28681403)(74.56022968,382.08181623)
\curveto(74.56022968,380.20681811)(75.34023235,377.70181623)(78.01022968,377.70181623)
\curveto(78.77522892,377.70181623)(79.67523025,377.9418173)(80.24522968,379.00681623)
\lineto(80.27522968,379.00681623)
\lineto(80.27522968,377.92681623)
\lineto(81.52022968,377.92681623)
\lineto(81.52022968,388.69681623)
\moveto(75.92522968,381.87181623)
\curveto(75.92522968,382.87681523)(76.03023172,384.81181623)(78.07022968,384.81181623)
\curveto(79.97522778,384.81181623)(80.18522968,382.75681496)(80.18522968,381.48181623)
\curveto(80.18522968,379.39681832)(78.88022884,378.84181623)(78.04022968,378.84181623)
\curveto(76.60023112,378.84181623)(75.92522968,380.14681796)(75.92522968,381.87181623)
}
}
{
\newrgbcolor{curcolor}{0 0 0}
\pscustom[linestyle=none,fillstyle=solid,fillcolor=curcolor]
{
\newpath
\moveto(82.90983906,381.85681623)
\curveto(82.90983906,379.83181826)(84.04984156,377.71681623)(86.55483906,377.71681623)
\curveto(89.05983655,377.71681623)(90.19983906,379.83181826)(90.19983906,381.85681623)
\curveto(90.19983906,383.88181421)(89.05983655,385.99681623)(86.55483906,385.99681623)
\curveto(84.04984156,385.99681623)(82.90983906,383.88181421)(82.90983906,381.85681623)
\moveto(84.27483906,381.85681623)
\curveto(84.27483906,382.90681518)(84.66484095,384.85681623)(86.55483906,384.85681623)
\curveto(88.44483717,384.85681623)(88.83483906,382.90681518)(88.83483906,381.85681623)
\curveto(88.83483906,380.80681728)(88.44483717,378.85681623)(86.55483906,378.85681623)
\curveto(84.66484095,378.85681623)(84.27483906,380.80681728)(84.27483906,381.85681623)
}
}
{
\newrgbcolor{curcolor}{0 0 0}
\pscustom[linestyle=none,fillstyle=solid,fillcolor=curcolor]
{
\newpath
\moveto(93.20944843,382.48681623)
\curveto(93.20944843,383.62681509)(93.98944966,384.58681623)(95.21944843,384.58681623)
\lineto(95.71444843,384.58681623)
\lineto(95.71444843,385.95181623)
\curveto(95.60944854,385.9818162)(95.53444827,385.99681623)(95.36944843,385.99681623)
\curveto(94.37944942,385.99681623)(93.68944791,385.38181532)(93.16444843,384.46681623)
\lineto(93.13444843,384.46681623)
\lineto(93.13444843,385.77181623)
\lineto(91.88944843,385.77181623)
\lineto(91.88944843,377.92681623)
\lineto(93.20944843,377.92681623)
\lineto(93.20944843,382.48681623)
}
}
{
\newrgbcolor{curcolor}{0 0 0}
\pscustom[linestyle=none,fillstyle=solid,fillcolor=curcolor]
{
\newpath
\moveto(8.8457082,490.63810041)
\lineto(7.3907082,490.63810041)
\lineto(7.3907082,479.86810041)
\lineto(8.8457082,479.86810041)
\lineto(8.8457082,490.63810041)
}
}
{
\newrgbcolor{curcolor}{0 0 0}
\pscustom[linestyle=none,fillstyle=solid,fillcolor=curcolor]
{
\newpath
\moveto(17.55086445,485.20810041)
\curveto(17.55086445,487.44309818)(16.02086323,487.93810041)(14.80586445,487.93810041)
\curveto(13.4558658,487.93810041)(12.72086416,487.02309999)(12.43586445,486.60310041)
\lineto(12.40586445,486.60310041)
\lineto(12.40586445,487.71310041)
\lineto(11.16086445,487.71310041)
\lineto(11.16086445,479.86810041)
\lineto(12.48086445,479.86810041)
\lineto(12.48086445,484.14310041)
\curveto(12.48086445,486.27309828)(13.8008652,486.75310041)(14.55086445,486.75310041)
\curveto(15.84086316,486.75310041)(16.23086445,486.06309905)(16.23086445,484.69810041)
\lineto(16.23086445,479.86810041)
\lineto(17.55086445,479.86810041)
\lineto(17.55086445,485.20810041)
}
}
{
\newrgbcolor{curcolor}{0 0 0}
\pscustom[linestyle=none,fillstyle=solid,fillcolor=curcolor]
{
\newpath
\moveto(21.96250507,481.32310041)
\lineto(21.93250507,481.32310041)
\lineto(19.89250507,487.71310041)
\lineto(18.36250507,487.71310041)
\lineto(21.22750507,479.86810041)
\lineto(22.63750507,479.86810041)
\lineto(25.62250507,487.71310041)
\lineto(24.18250507,487.71310041)
\lineto(21.96250507,481.32310041)
}
}
{
\newrgbcolor{curcolor}{0 0 0}
\pscustom[linestyle=none,fillstyle=solid,fillcolor=curcolor]
{
\newpath
\moveto(28.06750507,487.71310041)
\lineto(26.74750507,487.71310041)
\lineto(26.74750507,479.86810041)
\lineto(28.06750507,479.86810041)
\lineto(28.06750507,487.71310041)
\moveto(28.06750507,489.13810041)
\lineto(28.06750507,490.63810041)
\lineto(26.74750507,490.63810041)
\lineto(26.74750507,489.13810041)
\lineto(28.06750507,489.13810041)
}
}
{
\newrgbcolor{curcolor}{0 0 0}
\pscustom[linestyle=none,fillstyle=solid,fillcolor=curcolor]
{
\newpath
\moveto(32.93734882,486.61810041)
\lineto(32.93734882,487.71310041)
\lineto(31.67734882,487.71310041)
\lineto(31.67734882,489.90310041)
\lineto(30.35734882,489.90310041)
\lineto(30.35734882,487.71310041)
\lineto(29.29234882,487.71310041)
\lineto(29.29234882,486.61810041)
\lineto(30.35734882,486.61810041)
\lineto(30.35734882,481.44310041)
\curveto(30.35734882,480.49810136)(30.64235013,479.76310041)(31.94734882,479.76310041)
\curveto(32.08234869,479.76310041)(32.4573493,479.82310046)(32.93734882,479.86810041)
\lineto(32.93734882,480.90310041)
\lineto(32.47234882,480.90310041)
\curveto(32.20234909,480.90310041)(31.67734882,480.90310103)(31.67734882,481.51810041)
\lineto(31.67734882,486.61810041)
\lineto(32.93734882,486.61810041)
}
}
{
\newrgbcolor{curcolor}{0 0 0}
\pscustom[linestyle=none,fillstyle=solid,fillcolor=curcolor]
{
\newpath
\moveto(35.2410207,485.32810041)
\curveto(35.33102061,485.92809981)(35.5410222,486.84310041)(37.0410207,486.84310041)
\curveto(38.28601945,486.84310041)(38.8860207,486.39309959)(38.8860207,485.56810041)
\curveto(38.8860207,484.78810119)(38.51102038,484.66810038)(38.1960207,484.63810041)
\lineto(36.0210207,484.36810041)
\curveto(33.83102289,484.09810068)(33.6360207,482.56809975)(33.6360207,481.90810041)
\curveto(33.6360207,480.55810176)(34.65602214,479.64310041)(36.0960207,479.64310041)
\curveto(37.62601917,479.64310041)(38.42102121,480.36310097)(38.9310207,480.91810041)
\curveto(38.97602065,480.31810101)(39.15602187,479.71810041)(40.3260207,479.71810041)
\curveto(40.6260204,479.71810041)(40.82102092,479.80810047)(41.0460207,479.86810041)
\lineto(41.0460207,480.82810041)
\curveto(40.89602085,480.79810044)(40.73102058,480.76810041)(40.6110207,480.76810041)
\curveto(40.34102097,480.76810041)(40.1760207,480.90310074)(40.1760207,481.23310041)
\lineto(40.1760207,485.74810041)
\curveto(40.1760207,487.7580984)(37.89602007,487.93810041)(37.2660207,487.93810041)
\curveto(35.33102263,487.93810041)(34.08602064,487.20309854)(34.0260207,485.32810041)
\lineto(35.2410207,485.32810041)
\moveto(38.8560207,482.58310041)
\curveto(38.8560207,481.53310146)(37.65601947,480.73810041)(36.4260207,480.73810041)
\curveto(35.43602169,480.73810041)(35.0010207,481.24810127)(35.0010207,482.10310041)
\curveto(35.0010207,483.09309942)(36.03602134,483.2881005)(36.6810207,483.37810041)
\curveto(38.31601906,483.5881002)(38.64602091,483.70810058)(38.8560207,483.87310041)
\lineto(38.8560207,482.58310041)
}
}
{
\newrgbcolor{curcolor}{0 0 0}
\pscustom[linestyle=none,fillstyle=solid,fillcolor=curcolor]
{
\newpath
\moveto(48.93063007,490.63810041)
\lineto(47.61063007,490.63810041)
\lineto(47.61063007,486.70810041)
\lineto(47.58063007,486.60310041)
\curveto(47.26563039,487.05309996)(46.66562865,487.93810041)(45.24063007,487.93810041)
\curveto(43.15563216,487.93810041)(41.97063007,486.22809821)(41.97063007,484.02310041)
\curveto(41.97063007,482.14810229)(42.75063274,479.64310041)(45.42063007,479.64310041)
\curveto(46.18562931,479.64310041)(47.08563064,479.88310148)(47.65563007,480.94810041)
\lineto(47.68563007,480.94810041)
\lineto(47.68563007,479.86810041)
\lineto(48.93063007,479.86810041)
\lineto(48.93063007,490.63810041)
\moveto(43.33563007,483.81310041)
\curveto(43.33563007,484.81809941)(43.44063211,486.75310041)(45.48063007,486.75310041)
\curveto(47.38562817,486.75310041)(47.59563007,484.69809914)(47.59563007,483.42310041)
\curveto(47.59563007,481.3381025)(46.29062923,480.78310041)(45.45063007,480.78310041)
\curveto(44.01063151,480.78310041)(43.33563007,482.08810214)(43.33563007,483.81310041)
}
}
{
\newrgbcolor{curcolor}{0 0 0}
\pscustom[linestyle=none,fillstyle=solid,fillcolor=curcolor]
{
\newpath
\moveto(50.32023945,483.79810041)
\curveto(50.32023945,481.77310244)(51.46024195,479.65810041)(53.96523945,479.65810041)
\curveto(56.47023694,479.65810041)(57.61023945,481.77310244)(57.61023945,483.79810041)
\curveto(57.61023945,485.82309839)(56.47023694,487.93810041)(53.96523945,487.93810041)
\curveto(51.46024195,487.93810041)(50.32023945,485.82309839)(50.32023945,483.79810041)
\moveto(51.68523945,483.79810041)
\curveto(51.68523945,484.84809936)(52.07524134,486.79810041)(53.96523945,486.79810041)
\curveto(55.85523756,486.79810041)(56.24523945,484.84809936)(56.24523945,483.79810041)
\curveto(56.24523945,482.74810146)(55.85523756,480.79810041)(53.96523945,480.79810041)
\curveto(52.07524134,480.79810041)(51.68523945,482.74810146)(51.68523945,483.79810041)
}
}
{
\newrgbcolor{curcolor}{0 0 0}
\pscustom[linestyle=none,fillstyle=solid,fillcolor=curcolor]
{
\newpath
\moveto(8.8457082,473.64786604)
\lineto(7.3907082,473.64786604)
\lineto(7.3907082,462.87786604)
\lineto(8.8457082,462.87786604)
\lineto(8.8457082,473.64786604)
}
}
{
\newrgbcolor{curcolor}{0 0 0}
\pscustom[linestyle=none,fillstyle=solid,fillcolor=curcolor]
{
\newpath
\moveto(17.55086445,468.21786604)
\curveto(17.55086445,470.4528638)(16.02086323,470.94786604)(14.80586445,470.94786604)
\curveto(13.4558658,470.94786604)(12.72086416,470.03286562)(12.43586445,469.61286604)
\lineto(12.40586445,469.61286604)
\lineto(12.40586445,470.72286604)
\lineto(11.16086445,470.72286604)
\lineto(11.16086445,462.87786604)
\lineto(12.48086445,462.87786604)
\lineto(12.48086445,467.15286604)
\curveto(12.48086445,469.28286391)(13.8008652,469.76286604)(14.55086445,469.76286604)
\curveto(15.84086316,469.76286604)(16.23086445,469.07286467)(16.23086445,467.70786604)
\lineto(16.23086445,462.87786604)
\lineto(17.55086445,462.87786604)
\lineto(17.55086445,468.21786604)
}
}
{
\newrgbcolor{curcolor}{0 0 0}
\pscustom[linestyle=none,fillstyle=solid,fillcolor=curcolor]
{
\newpath
\moveto(21.96250507,464.33286604)
\lineto(21.93250507,464.33286604)
\lineto(19.89250507,470.72286604)
\lineto(18.36250507,470.72286604)
\lineto(21.22750507,462.87786604)
\lineto(22.63750507,462.87786604)
\lineto(25.62250507,470.72286604)
\lineto(24.18250507,470.72286604)
\lineto(21.96250507,464.33286604)
}
}
{
\newrgbcolor{curcolor}{0 0 0}
\pscustom[linestyle=none,fillstyle=solid,fillcolor=curcolor]
{
\newpath
\moveto(28.06750507,470.72286604)
\lineto(26.74750507,470.72286604)
\lineto(26.74750507,462.87786604)
\lineto(28.06750507,462.87786604)
\lineto(28.06750507,470.72286604)
\moveto(28.06750507,472.14786604)
\lineto(28.06750507,473.64786604)
\lineto(26.74750507,473.64786604)
\lineto(26.74750507,472.14786604)
\lineto(28.06750507,472.14786604)
}
}
{
\newrgbcolor{curcolor}{0 0 0}
\pscustom[linestyle=none,fillstyle=solid,fillcolor=curcolor]
{
\newpath
\moveto(32.93734882,469.62786604)
\lineto(32.93734882,470.72286604)
\lineto(31.67734882,470.72286604)
\lineto(31.67734882,472.91286604)
\lineto(30.35734882,472.91286604)
\lineto(30.35734882,470.72286604)
\lineto(29.29234882,470.72286604)
\lineto(29.29234882,469.62786604)
\lineto(30.35734882,469.62786604)
\lineto(30.35734882,464.45286604)
\curveto(30.35734882,463.50786698)(30.64235013,462.77286604)(31.94734882,462.77286604)
\curveto(32.08234869,462.77286604)(32.4573493,462.83286608)(32.93734882,462.87786604)
\lineto(32.93734882,463.91286604)
\lineto(32.47234882,463.91286604)
\curveto(32.20234909,463.91286604)(31.67734882,463.91286665)(31.67734882,464.52786604)
\lineto(31.67734882,469.62786604)
\lineto(32.93734882,469.62786604)
}
}
{
\newrgbcolor{curcolor}{0 0 0}
\pscustom[linestyle=none,fillstyle=solid,fillcolor=curcolor]
{
\newpath
\moveto(35.2410207,468.33786604)
\curveto(35.33102061,468.93786544)(35.5410222,469.85286604)(37.0410207,469.85286604)
\curveto(38.28601945,469.85286604)(38.8860207,469.40286521)(38.8860207,468.57786604)
\curveto(38.8860207,467.79786682)(38.51102038,467.67786601)(38.1960207,467.64786604)
\lineto(36.0210207,467.37786604)
\curveto(33.83102289,467.10786631)(33.6360207,465.57786538)(33.6360207,464.91786604)
\curveto(33.6360207,463.56786739)(34.65602214,462.65286604)(36.0960207,462.65286604)
\curveto(37.62601917,462.65286604)(38.42102121,463.37286659)(38.9310207,463.92786604)
\curveto(38.97602065,463.32786664)(39.15602187,462.72786604)(40.3260207,462.72786604)
\curveto(40.6260204,462.72786604)(40.82102092,462.8178661)(41.0460207,462.87786604)
\lineto(41.0460207,463.83786604)
\curveto(40.89602085,463.80786607)(40.73102058,463.77786604)(40.6110207,463.77786604)
\curveto(40.34102097,463.77786604)(40.1760207,463.91286637)(40.1760207,464.24286604)
\lineto(40.1760207,468.75786604)
\curveto(40.1760207,470.76786403)(37.89602007,470.94786604)(37.2660207,470.94786604)
\curveto(35.33102263,470.94786604)(34.08602064,470.21286416)(34.0260207,468.33786604)
\lineto(35.2410207,468.33786604)
\moveto(38.8560207,465.59286604)
\curveto(38.8560207,464.54286709)(37.65601947,463.74786604)(36.4260207,463.74786604)
\curveto(35.43602169,463.74786604)(35.0010207,464.25786689)(35.0010207,465.11286604)
\curveto(35.0010207,466.10286505)(36.03602134,466.29786613)(36.6810207,466.38786604)
\curveto(38.31601906,466.59786583)(38.64602091,466.7178662)(38.8560207,466.88286604)
\lineto(38.8560207,465.59286604)
}
}
{
\newrgbcolor{curcolor}{0 0 0}
\pscustom[linestyle=none,fillstyle=solid,fillcolor=curcolor]
{
\newpath
\moveto(48.93063007,473.64786604)
\lineto(47.61063007,473.64786604)
\lineto(47.61063007,469.71786604)
\lineto(47.58063007,469.61286604)
\curveto(47.26563039,470.06286559)(46.66562865,470.94786604)(45.24063007,470.94786604)
\curveto(43.15563216,470.94786604)(41.97063007,469.23786383)(41.97063007,467.03286604)
\curveto(41.97063007,465.15786791)(42.75063274,462.65286604)(45.42063007,462.65286604)
\curveto(46.18562931,462.65286604)(47.08563064,462.8928671)(47.65563007,463.95786604)
\lineto(47.68563007,463.95786604)
\lineto(47.68563007,462.87786604)
\lineto(48.93063007,462.87786604)
\lineto(48.93063007,473.64786604)
\moveto(43.33563007,466.82286604)
\curveto(43.33563007,467.82786503)(43.44063211,469.76286604)(45.48063007,469.76286604)
\curveto(47.38562817,469.76286604)(47.59563007,467.70786476)(47.59563007,466.43286604)
\curveto(47.59563007,464.34786812)(46.29062923,463.79286604)(45.45063007,463.79286604)
\curveto(44.01063151,463.79286604)(43.33563007,465.09786776)(43.33563007,466.82286604)
}
}
{
\newrgbcolor{curcolor}{0 0 0}
\pscustom[linestyle=none,fillstyle=solid,fillcolor=curcolor]
{
\newpath
\moveto(50.32023945,466.80786604)
\curveto(50.32023945,464.78286806)(51.46024195,462.66786604)(53.96523945,462.66786604)
\curveto(56.47023694,462.66786604)(57.61023945,464.78286806)(57.61023945,466.80786604)
\curveto(57.61023945,468.83286401)(56.47023694,470.94786604)(53.96523945,470.94786604)
\curveto(51.46024195,470.94786604)(50.32023945,468.83286401)(50.32023945,466.80786604)
\moveto(51.68523945,466.80786604)
\curveto(51.68523945,467.85786499)(52.07524134,469.80786604)(53.96523945,469.80786604)
\curveto(55.85523756,469.80786604)(56.24523945,467.85786499)(56.24523945,466.80786604)
\curveto(56.24523945,465.75786709)(55.85523756,463.80786604)(53.96523945,463.80786604)
\curveto(52.07524134,463.80786604)(51.68523945,465.75786709)(51.68523945,466.80786604)
}
}
{
\newrgbcolor{curcolor}{0 0 0}
\pscustom[linestyle=none,fillstyle=solid,fillcolor=curcolor]
{
\newpath
\moveto(8.8457082,456.65763166)
\lineto(7.3907082,456.65763166)
\lineto(7.3907082,445.88763166)
\lineto(8.8457082,445.88763166)
\lineto(8.8457082,456.65763166)
}
}
{
\newrgbcolor{curcolor}{0 0 0}
\pscustom[linestyle=none,fillstyle=solid,fillcolor=curcolor]
{
\newpath
\moveto(17.55086445,451.22763166)
\curveto(17.55086445,453.46262943)(16.02086323,453.95763166)(14.80586445,453.95763166)
\curveto(13.4558658,453.95763166)(12.72086416,453.04263124)(12.43586445,452.62263166)
\lineto(12.40586445,452.62263166)
\lineto(12.40586445,453.73263166)
\lineto(11.16086445,453.73263166)
\lineto(11.16086445,445.88763166)
\lineto(12.48086445,445.88763166)
\lineto(12.48086445,450.16263166)
\curveto(12.48086445,452.29262953)(13.8008652,452.77263166)(14.55086445,452.77263166)
\curveto(15.84086316,452.77263166)(16.23086445,452.0826303)(16.23086445,450.71763166)
\lineto(16.23086445,445.88763166)
\lineto(17.55086445,445.88763166)
\lineto(17.55086445,451.22763166)
}
}
{
\newrgbcolor{curcolor}{0 0 0}
\pscustom[linestyle=none,fillstyle=solid,fillcolor=curcolor]
{
\newpath
\moveto(21.96250507,447.34263166)
\lineto(21.93250507,447.34263166)
\lineto(19.89250507,453.73263166)
\lineto(18.36250507,453.73263166)
\lineto(21.22750507,445.88763166)
\lineto(22.63750507,445.88763166)
\lineto(25.62250507,453.73263166)
\lineto(24.18250507,453.73263166)
\lineto(21.96250507,447.34263166)
}
}
{
\newrgbcolor{curcolor}{0 0 0}
\pscustom[linestyle=none,fillstyle=solid,fillcolor=curcolor]
{
\newpath
\moveto(28.06750507,453.73263166)
\lineto(26.74750507,453.73263166)
\lineto(26.74750507,445.88763166)
\lineto(28.06750507,445.88763166)
\lineto(28.06750507,453.73263166)
\moveto(28.06750507,455.15763166)
\lineto(28.06750507,456.65763166)
\lineto(26.74750507,456.65763166)
\lineto(26.74750507,455.15763166)
\lineto(28.06750507,455.15763166)
}
}
{
\newrgbcolor{curcolor}{0 0 0}
\pscustom[linestyle=none,fillstyle=solid,fillcolor=curcolor]
{
\newpath
\moveto(32.93734882,452.63763166)
\lineto(32.93734882,453.73263166)
\lineto(31.67734882,453.73263166)
\lineto(31.67734882,455.92263166)
\lineto(30.35734882,455.92263166)
\lineto(30.35734882,453.73263166)
\lineto(29.29234882,453.73263166)
\lineto(29.29234882,452.63763166)
\lineto(30.35734882,452.63763166)
\lineto(30.35734882,447.46263166)
\curveto(30.35734882,446.51763261)(30.64235013,445.78263166)(31.94734882,445.78263166)
\curveto(32.08234869,445.78263166)(32.4573493,445.84263171)(32.93734882,445.88763166)
\lineto(32.93734882,446.92263166)
\lineto(32.47234882,446.92263166)
\curveto(32.20234909,446.92263166)(31.67734882,446.92263228)(31.67734882,447.53763166)
\lineto(31.67734882,452.63763166)
\lineto(32.93734882,452.63763166)
}
}
{
\newrgbcolor{curcolor}{0 0 0}
\pscustom[linestyle=none,fillstyle=solid,fillcolor=curcolor]
{
\newpath
\moveto(35.2410207,451.34763166)
\curveto(35.33102061,451.94763106)(35.5410222,452.86263166)(37.0410207,452.86263166)
\curveto(38.28601945,452.86263166)(38.8860207,452.41263084)(38.8860207,451.58763166)
\curveto(38.8860207,450.80763244)(38.51102038,450.68763163)(38.1960207,450.65763166)
\lineto(36.0210207,450.38763166)
\curveto(33.83102289,450.11763193)(33.6360207,448.587631)(33.6360207,447.92763166)
\curveto(33.6360207,446.57763301)(34.65602214,445.66263166)(36.0960207,445.66263166)
\curveto(37.62601917,445.66263166)(38.42102121,446.38263222)(38.9310207,446.93763166)
\curveto(38.97602065,446.33763226)(39.15602187,445.73763166)(40.3260207,445.73763166)
\curveto(40.6260204,445.73763166)(40.82102092,445.82763172)(41.0460207,445.88763166)
\lineto(41.0460207,446.84763166)
\curveto(40.89602085,446.81763169)(40.73102058,446.78763166)(40.6110207,446.78763166)
\curveto(40.34102097,446.78763166)(40.1760207,446.92263199)(40.1760207,447.25263166)
\lineto(40.1760207,451.76763166)
\curveto(40.1760207,453.77762965)(37.89602007,453.95763166)(37.2660207,453.95763166)
\curveto(35.33102263,453.95763166)(34.08602064,453.22262979)(34.0260207,451.34763166)
\lineto(35.2410207,451.34763166)
\moveto(38.8560207,448.60263166)
\curveto(38.8560207,447.55263271)(37.65601947,446.75763166)(36.4260207,446.75763166)
\curveto(35.43602169,446.75763166)(35.0010207,447.26763252)(35.0010207,448.12263166)
\curveto(35.0010207,449.11263067)(36.03602134,449.30763175)(36.6810207,449.39763166)
\curveto(38.31601906,449.60763145)(38.64602091,449.72763183)(38.8560207,449.89263166)
\lineto(38.8560207,448.60263166)
}
}
{
\newrgbcolor{curcolor}{0 0 0}
\pscustom[linestyle=none,fillstyle=solid,fillcolor=curcolor]
{
\newpath
\moveto(48.93063007,456.65763166)
\lineto(47.61063007,456.65763166)
\lineto(47.61063007,452.72763166)
\lineto(47.58063007,452.62263166)
\curveto(47.26563039,453.07263121)(46.66562865,453.95763166)(45.24063007,453.95763166)
\curveto(43.15563216,453.95763166)(41.97063007,452.24762946)(41.97063007,450.04263166)
\curveto(41.97063007,448.16763354)(42.75063274,445.66263166)(45.42063007,445.66263166)
\curveto(46.18562931,445.66263166)(47.08563064,445.90263273)(47.65563007,446.96763166)
\lineto(47.68563007,446.96763166)
\lineto(47.68563007,445.88763166)
\lineto(48.93063007,445.88763166)
\lineto(48.93063007,456.65763166)
\moveto(43.33563007,449.83263166)
\curveto(43.33563007,450.83763066)(43.44063211,452.77263166)(45.48063007,452.77263166)
\curveto(47.38562817,452.77263166)(47.59563007,450.71763039)(47.59563007,449.44263166)
\curveto(47.59563007,447.35763375)(46.29062923,446.80263166)(45.45063007,446.80263166)
\curveto(44.01063151,446.80263166)(43.33563007,448.10763339)(43.33563007,449.83263166)
}
}
{
\newrgbcolor{curcolor}{0 0 0}
\pscustom[linestyle=none,fillstyle=solid,fillcolor=curcolor]
{
\newpath
\moveto(50.32023945,449.81763166)
\curveto(50.32023945,447.79263369)(51.46024195,445.67763166)(53.96523945,445.67763166)
\curveto(56.47023694,445.67763166)(57.61023945,447.79263369)(57.61023945,449.81763166)
\curveto(57.61023945,451.84262964)(56.47023694,453.95763166)(53.96523945,453.95763166)
\curveto(51.46024195,453.95763166)(50.32023945,451.84262964)(50.32023945,449.81763166)
\moveto(51.68523945,449.81763166)
\curveto(51.68523945,450.86763061)(52.07524134,452.81763166)(53.96523945,452.81763166)
\curveto(55.85523756,452.81763166)(56.24523945,450.86763061)(56.24523945,449.81763166)
\curveto(56.24523945,448.76763271)(55.85523756,446.81763166)(53.96523945,446.81763166)
\curveto(52.07524134,446.81763166)(51.68523945,448.76763271)(51.68523945,449.81763166)
}
}
{
\newrgbcolor{curcolor}{0 0 0}
\pscustom[linestyle=none,fillstyle=solid,fillcolor=curcolor]
{
\newpath
\moveto(8.8457082,439.66739729)
\lineto(7.3907082,439.66739729)
\lineto(7.3907082,428.89739729)
\lineto(8.8457082,428.89739729)
\lineto(8.8457082,439.66739729)
}
}
{
\newrgbcolor{curcolor}{0 0 0}
\pscustom[linestyle=none,fillstyle=solid,fillcolor=curcolor]
{
\newpath
\moveto(17.55086445,434.23739729)
\curveto(17.55086445,436.47239505)(16.02086323,436.96739729)(14.80586445,436.96739729)
\curveto(13.4558658,436.96739729)(12.72086416,436.05239687)(12.43586445,435.63239729)
\lineto(12.40586445,435.63239729)
\lineto(12.40586445,436.74239729)
\lineto(11.16086445,436.74239729)
\lineto(11.16086445,428.89739729)
\lineto(12.48086445,428.89739729)
\lineto(12.48086445,433.17239729)
\curveto(12.48086445,435.30239516)(13.8008652,435.78239729)(14.55086445,435.78239729)
\curveto(15.84086316,435.78239729)(16.23086445,435.09239592)(16.23086445,433.72739729)
\lineto(16.23086445,428.89739729)
\lineto(17.55086445,428.89739729)
\lineto(17.55086445,434.23739729)
}
}
{
\newrgbcolor{curcolor}{0 0 0}
\pscustom[linestyle=none,fillstyle=solid,fillcolor=curcolor]
{
\newpath
\moveto(21.96250507,430.35239729)
\lineto(21.93250507,430.35239729)
\lineto(19.89250507,436.74239729)
\lineto(18.36250507,436.74239729)
\lineto(21.22750507,428.89739729)
\lineto(22.63750507,428.89739729)
\lineto(25.62250507,436.74239729)
\lineto(24.18250507,436.74239729)
\lineto(21.96250507,430.35239729)
}
}
{
\newrgbcolor{curcolor}{0 0 0}
\pscustom[linestyle=none,fillstyle=solid,fillcolor=curcolor]
{
\newpath
\moveto(28.06750507,436.74239729)
\lineto(26.74750507,436.74239729)
\lineto(26.74750507,428.89739729)
\lineto(28.06750507,428.89739729)
\lineto(28.06750507,436.74239729)
\moveto(28.06750507,438.16739729)
\lineto(28.06750507,439.66739729)
\lineto(26.74750507,439.66739729)
\lineto(26.74750507,438.16739729)
\lineto(28.06750507,438.16739729)
}
}
{
\newrgbcolor{curcolor}{0 0 0}
\pscustom[linestyle=none,fillstyle=solid,fillcolor=curcolor]
{
\newpath
\moveto(32.93734882,435.64739729)
\lineto(32.93734882,436.74239729)
\lineto(31.67734882,436.74239729)
\lineto(31.67734882,438.93239729)
\lineto(30.35734882,438.93239729)
\lineto(30.35734882,436.74239729)
\lineto(29.29234882,436.74239729)
\lineto(29.29234882,435.64739729)
\lineto(30.35734882,435.64739729)
\lineto(30.35734882,430.47239729)
\curveto(30.35734882,429.52739823)(30.64235013,428.79239729)(31.94734882,428.79239729)
\curveto(32.08234869,428.79239729)(32.4573493,428.85239733)(32.93734882,428.89739729)
\lineto(32.93734882,429.93239729)
\lineto(32.47234882,429.93239729)
\curveto(32.20234909,429.93239729)(31.67734882,429.9323979)(31.67734882,430.54739729)
\lineto(31.67734882,435.64739729)
\lineto(32.93734882,435.64739729)
}
}
{
\newrgbcolor{curcolor}{0 0 0}
\pscustom[linestyle=none,fillstyle=solid,fillcolor=curcolor]
{
\newpath
\moveto(35.2410207,434.35739729)
\curveto(35.33102061,434.95739669)(35.5410222,435.87239729)(37.0410207,435.87239729)
\curveto(38.28601945,435.87239729)(38.8860207,435.42239646)(38.8860207,434.59739729)
\curveto(38.8860207,433.81739807)(38.51102038,433.69739726)(38.1960207,433.66739729)
\lineto(36.0210207,433.39739729)
\curveto(33.83102289,433.12739756)(33.6360207,431.59739663)(33.6360207,430.93739729)
\curveto(33.6360207,429.58739864)(34.65602214,428.67239729)(36.0960207,428.67239729)
\curveto(37.62601917,428.67239729)(38.42102121,429.39239784)(38.9310207,429.94739729)
\curveto(38.97602065,429.34739789)(39.15602187,428.74739729)(40.3260207,428.74739729)
\curveto(40.6260204,428.74739729)(40.82102092,428.83739735)(41.0460207,428.89739729)
\lineto(41.0460207,429.85739729)
\curveto(40.89602085,429.82739732)(40.73102058,429.79739729)(40.6110207,429.79739729)
\curveto(40.34102097,429.79739729)(40.1760207,429.93239762)(40.1760207,430.26239729)
\lineto(40.1760207,434.77739729)
\curveto(40.1760207,436.78739528)(37.89602007,436.96739729)(37.2660207,436.96739729)
\curveto(35.33102263,436.96739729)(34.08602064,436.23239541)(34.0260207,434.35739729)
\lineto(35.2410207,434.35739729)
\moveto(38.8560207,431.61239729)
\curveto(38.8560207,430.56239834)(37.65601947,429.76739729)(36.4260207,429.76739729)
\curveto(35.43602169,429.76739729)(35.0010207,430.27739814)(35.0010207,431.13239729)
\curveto(35.0010207,432.1223963)(36.03602134,432.31739738)(36.6810207,432.40739729)
\curveto(38.31601906,432.61739708)(38.64602091,432.73739745)(38.8560207,432.90239729)
\lineto(38.8560207,431.61239729)
}
}
{
\newrgbcolor{curcolor}{0 0 0}
\pscustom[linestyle=none,fillstyle=solid,fillcolor=curcolor]
{
\newpath
\moveto(48.93063007,439.66739729)
\lineto(47.61063007,439.66739729)
\lineto(47.61063007,435.73739729)
\lineto(47.58063007,435.63239729)
\curveto(47.26563039,436.08239684)(46.66562865,436.96739729)(45.24063007,436.96739729)
\curveto(43.15563216,436.96739729)(41.97063007,435.25739508)(41.97063007,433.05239729)
\curveto(41.97063007,431.17739916)(42.75063274,428.67239729)(45.42063007,428.67239729)
\curveto(46.18562931,428.67239729)(47.08563064,428.91239835)(47.65563007,429.97739729)
\lineto(47.68563007,429.97739729)
\lineto(47.68563007,428.89739729)
\lineto(48.93063007,428.89739729)
\lineto(48.93063007,439.66739729)
\moveto(43.33563007,432.84239729)
\curveto(43.33563007,433.84739628)(43.44063211,435.78239729)(45.48063007,435.78239729)
\curveto(47.38562817,435.78239729)(47.59563007,433.72739601)(47.59563007,432.45239729)
\curveto(47.59563007,430.36739937)(46.29062923,429.81239729)(45.45063007,429.81239729)
\curveto(44.01063151,429.81239729)(43.33563007,431.11739901)(43.33563007,432.84239729)
}
}
{
\newrgbcolor{curcolor}{0 0 0}
\pscustom[linestyle=none,fillstyle=solid,fillcolor=curcolor]
{
\newpath
\moveto(50.32023945,432.82739729)
\curveto(50.32023945,430.80239931)(51.46024195,428.68739729)(53.96523945,428.68739729)
\curveto(56.47023694,428.68739729)(57.61023945,430.80239931)(57.61023945,432.82739729)
\curveto(57.61023945,434.85239526)(56.47023694,436.96739729)(53.96523945,436.96739729)
\curveto(51.46024195,436.96739729)(50.32023945,434.85239526)(50.32023945,432.82739729)
\moveto(51.68523945,432.82739729)
\curveto(51.68523945,433.87739624)(52.07524134,435.82739729)(53.96523945,435.82739729)
\curveto(55.85523756,435.82739729)(56.24523945,433.87739624)(56.24523945,432.82739729)
\curveto(56.24523945,431.77739834)(55.85523756,429.82739729)(53.96523945,429.82739729)
\curveto(52.07524134,429.82739729)(51.68523945,431.77739834)(51.68523945,432.82739729)
}
}
{
\newrgbcolor{curcolor}{0 0 0}
\pscustom[linestyle=none,fillstyle=solid,fillcolor=curcolor]
{
\newpath
\moveto(8.8457082,337.72611311)
\lineto(7.3907082,337.72611311)
\lineto(7.3907082,326.95611311)
\lineto(8.8457082,326.95611311)
\lineto(8.8457082,337.72611311)
}
}
{
\newrgbcolor{curcolor}{0 0 0}
\pscustom[linestyle=none,fillstyle=solid,fillcolor=curcolor]
{
\newpath
\moveto(17.55086445,332.29611311)
\curveto(17.55086445,334.53111087)(16.02086323,335.02611311)(14.80586445,335.02611311)
\curveto(13.4558658,335.02611311)(12.72086416,334.11111269)(12.43586445,333.69111311)
\lineto(12.40586445,333.69111311)
\lineto(12.40586445,334.80111311)
\lineto(11.16086445,334.80111311)
\lineto(11.16086445,326.95611311)
\lineto(12.48086445,326.95611311)
\lineto(12.48086445,331.23111311)
\curveto(12.48086445,333.36111098)(13.8008652,333.84111311)(14.55086445,333.84111311)
\curveto(15.84086316,333.84111311)(16.23086445,333.15111174)(16.23086445,331.78611311)
\lineto(16.23086445,326.95611311)
\lineto(17.55086445,326.95611311)
\lineto(17.55086445,332.29611311)
}
}
{
\newrgbcolor{curcolor}{0 0 0}
\pscustom[linestyle=none,fillstyle=solid,fillcolor=curcolor]
{
\newpath
\moveto(21.96250507,328.41111311)
\lineto(21.93250507,328.41111311)
\lineto(19.89250507,334.80111311)
\lineto(18.36250507,334.80111311)
\lineto(21.22750507,326.95611311)
\lineto(22.63750507,326.95611311)
\lineto(25.62250507,334.80111311)
\lineto(24.18250507,334.80111311)
\lineto(21.96250507,328.41111311)
}
}
{
\newrgbcolor{curcolor}{0 0 0}
\pscustom[linestyle=none,fillstyle=solid,fillcolor=curcolor]
{
\newpath
\moveto(28.06750507,334.80111311)
\lineto(26.74750507,334.80111311)
\lineto(26.74750507,326.95611311)
\lineto(28.06750507,326.95611311)
\lineto(28.06750507,334.80111311)
\moveto(28.06750507,336.22611311)
\lineto(28.06750507,337.72611311)
\lineto(26.74750507,337.72611311)
\lineto(26.74750507,336.22611311)
\lineto(28.06750507,336.22611311)
}
}
{
\newrgbcolor{curcolor}{0 0 0}
\pscustom[linestyle=none,fillstyle=solid,fillcolor=curcolor]
{
\newpath
\moveto(32.93734882,333.70611311)
\lineto(32.93734882,334.80111311)
\lineto(31.67734882,334.80111311)
\lineto(31.67734882,336.99111311)
\lineto(30.35734882,336.99111311)
\lineto(30.35734882,334.80111311)
\lineto(29.29234882,334.80111311)
\lineto(29.29234882,333.70611311)
\lineto(30.35734882,333.70611311)
\lineto(30.35734882,328.53111311)
\curveto(30.35734882,327.58611405)(30.64235013,326.85111311)(31.94734882,326.85111311)
\curveto(32.08234869,326.85111311)(32.4573493,326.91111315)(32.93734882,326.95611311)
\lineto(32.93734882,327.99111311)
\lineto(32.47234882,327.99111311)
\curveto(32.20234909,327.99111311)(31.67734882,327.99111372)(31.67734882,328.60611311)
\lineto(31.67734882,333.70611311)
\lineto(32.93734882,333.70611311)
}
}
{
\newrgbcolor{curcolor}{0 0 0}
\pscustom[linestyle=none,fillstyle=solid,fillcolor=curcolor]
{
\newpath
\moveto(35.2410207,332.41611311)
\curveto(35.33102061,333.01611251)(35.5410222,333.93111311)(37.0410207,333.93111311)
\curveto(38.28601945,333.93111311)(38.8860207,333.48111228)(38.8860207,332.65611311)
\curveto(38.8860207,331.87611389)(38.51102038,331.75611308)(38.1960207,331.72611311)
\lineto(36.0210207,331.45611311)
\curveto(33.83102289,331.18611338)(33.6360207,329.65611245)(33.6360207,328.99611311)
\curveto(33.6360207,327.64611446)(34.65602214,326.73111311)(36.0960207,326.73111311)
\curveto(37.62601917,326.73111311)(38.42102121,327.45111366)(38.9310207,328.00611311)
\curveto(38.97602065,327.40611371)(39.15602187,326.80611311)(40.3260207,326.80611311)
\curveto(40.6260204,326.80611311)(40.82102092,326.89611317)(41.0460207,326.95611311)
\lineto(41.0460207,327.91611311)
\curveto(40.89602085,327.88611314)(40.73102058,327.85611311)(40.6110207,327.85611311)
\curveto(40.34102097,327.85611311)(40.1760207,327.99111344)(40.1760207,328.32111311)
\lineto(40.1760207,332.83611311)
\curveto(40.1760207,334.8461111)(37.89602007,335.02611311)(37.2660207,335.02611311)
\curveto(35.33102263,335.02611311)(34.08602064,334.29111123)(34.0260207,332.41611311)
\lineto(35.2410207,332.41611311)
\moveto(38.8560207,329.67111311)
\curveto(38.8560207,328.62111416)(37.65601947,327.82611311)(36.4260207,327.82611311)
\curveto(35.43602169,327.82611311)(35.0010207,328.33611396)(35.0010207,329.19111311)
\curveto(35.0010207,330.18111212)(36.03602134,330.3761132)(36.6810207,330.46611311)
\curveto(38.31601906,330.6761129)(38.64602091,330.79611327)(38.8560207,330.96111311)
\lineto(38.8560207,329.67111311)
}
}
{
\newrgbcolor{curcolor}{0 0 0}
\pscustom[linestyle=none,fillstyle=solid,fillcolor=curcolor]
{
\newpath
\moveto(48.93063007,337.72611311)
\lineto(47.61063007,337.72611311)
\lineto(47.61063007,333.79611311)
\lineto(47.58063007,333.69111311)
\curveto(47.26563039,334.14111266)(46.66562865,335.02611311)(45.24063007,335.02611311)
\curveto(43.15563216,335.02611311)(41.97063007,333.3161109)(41.97063007,331.11111311)
\curveto(41.97063007,329.23611498)(42.75063274,326.73111311)(45.42063007,326.73111311)
\curveto(46.18562931,326.73111311)(47.08563064,326.97111417)(47.65563007,328.03611311)
\lineto(47.68563007,328.03611311)
\lineto(47.68563007,326.95611311)
\lineto(48.93063007,326.95611311)
\lineto(48.93063007,337.72611311)
\moveto(43.33563007,330.90111311)
\curveto(43.33563007,331.9061121)(43.44063211,333.84111311)(45.48063007,333.84111311)
\curveto(47.38562817,333.84111311)(47.59563007,331.78611183)(47.59563007,330.51111311)
\curveto(47.59563007,328.42611519)(46.29062923,327.87111311)(45.45063007,327.87111311)
\curveto(44.01063151,327.87111311)(43.33563007,329.17611483)(43.33563007,330.90111311)
}
}
{
\newrgbcolor{curcolor}{0 0 0}
\pscustom[linestyle=none,fillstyle=solid,fillcolor=curcolor]
{
\newpath
\moveto(50.32023945,330.88611311)
\curveto(50.32023945,328.86111513)(51.46024195,326.74611311)(53.96523945,326.74611311)
\curveto(56.47023694,326.74611311)(57.61023945,328.86111513)(57.61023945,330.88611311)
\curveto(57.61023945,332.91111108)(56.47023694,335.02611311)(53.96523945,335.02611311)
\curveto(51.46024195,335.02611311)(50.32023945,332.91111108)(50.32023945,330.88611311)
\moveto(51.68523945,330.88611311)
\curveto(51.68523945,331.93611206)(52.07524134,333.88611311)(53.96523945,333.88611311)
\curveto(55.85523756,333.88611311)(56.24523945,331.93611206)(56.24523945,330.88611311)
\curveto(56.24523945,329.83611416)(55.85523756,327.88611311)(53.96523945,327.88611311)
\curveto(52.07524134,327.88611311)(51.68523945,329.83611416)(51.68523945,330.88611311)
}
}
{
\newrgbcolor{curcolor}{0 0 0}
\pscustom[linestyle=none,fillstyle=solid,fillcolor=curcolor]
{
\newpath
\moveto(8.8457082,320.73587873)
\lineto(7.3907082,320.73587873)
\lineto(7.3907082,309.96587873)
\lineto(8.8457082,309.96587873)
\lineto(8.8457082,320.73587873)
}
}
{
\newrgbcolor{curcolor}{0 0 0}
\pscustom[linestyle=none,fillstyle=solid,fillcolor=curcolor]
{
\newpath
\moveto(17.55086445,315.30587873)
\curveto(17.55086445,317.5408765)(16.02086323,318.03587873)(14.80586445,318.03587873)
\curveto(13.4558658,318.03587873)(12.72086416,317.12087831)(12.43586445,316.70087873)
\lineto(12.40586445,316.70087873)
\lineto(12.40586445,317.81087873)
\lineto(11.16086445,317.81087873)
\lineto(11.16086445,309.96587873)
\lineto(12.48086445,309.96587873)
\lineto(12.48086445,314.24087873)
\curveto(12.48086445,316.3708766)(13.8008652,316.85087873)(14.55086445,316.85087873)
\curveto(15.84086316,316.85087873)(16.23086445,316.16087737)(16.23086445,314.79587873)
\lineto(16.23086445,309.96587873)
\lineto(17.55086445,309.96587873)
\lineto(17.55086445,315.30587873)
}
}
{
\newrgbcolor{curcolor}{0 0 0}
\pscustom[linestyle=none,fillstyle=solid,fillcolor=curcolor]
{
\newpath
\moveto(21.96250507,311.42087873)
\lineto(21.93250507,311.42087873)
\lineto(19.89250507,317.81087873)
\lineto(18.36250507,317.81087873)
\lineto(21.22750507,309.96587873)
\lineto(22.63750507,309.96587873)
\lineto(25.62250507,317.81087873)
\lineto(24.18250507,317.81087873)
\lineto(21.96250507,311.42087873)
}
}
{
\newrgbcolor{curcolor}{0 0 0}
\pscustom[linestyle=none,fillstyle=solid,fillcolor=curcolor]
{
\newpath
\moveto(28.06750507,317.81087873)
\lineto(26.74750507,317.81087873)
\lineto(26.74750507,309.96587873)
\lineto(28.06750507,309.96587873)
\lineto(28.06750507,317.81087873)
\moveto(28.06750507,319.23587873)
\lineto(28.06750507,320.73587873)
\lineto(26.74750507,320.73587873)
\lineto(26.74750507,319.23587873)
\lineto(28.06750507,319.23587873)
}
}
{
\newrgbcolor{curcolor}{0 0 0}
\pscustom[linestyle=none,fillstyle=solid,fillcolor=curcolor]
{
\newpath
\moveto(32.93734882,316.71587873)
\lineto(32.93734882,317.81087873)
\lineto(31.67734882,317.81087873)
\lineto(31.67734882,320.00087873)
\lineto(30.35734882,320.00087873)
\lineto(30.35734882,317.81087873)
\lineto(29.29234882,317.81087873)
\lineto(29.29234882,316.71587873)
\lineto(30.35734882,316.71587873)
\lineto(30.35734882,311.54087873)
\curveto(30.35734882,310.59587968)(30.64235013,309.86087873)(31.94734882,309.86087873)
\curveto(32.08234869,309.86087873)(32.4573493,309.92087878)(32.93734882,309.96587873)
\lineto(32.93734882,311.00087873)
\lineto(32.47234882,311.00087873)
\curveto(32.20234909,311.00087873)(31.67734882,311.00087935)(31.67734882,311.61587873)
\lineto(31.67734882,316.71587873)
\lineto(32.93734882,316.71587873)
}
}
{
\newrgbcolor{curcolor}{0 0 0}
\pscustom[linestyle=none,fillstyle=solid,fillcolor=curcolor]
{
\newpath
\moveto(35.2410207,315.42587873)
\curveto(35.33102061,316.02587813)(35.5410222,316.94087873)(37.0410207,316.94087873)
\curveto(38.28601945,316.94087873)(38.8860207,316.49087791)(38.8860207,315.66587873)
\curveto(38.8860207,314.88587951)(38.51102038,314.7658787)(38.1960207,314.73587873)
\lineto(36.0210207,314.46587873)
\curveto(33.83102289,314.195879)(33.6360207,312.66587807)(33.6360207,312.00587873)
\curveto(33.6360207,310.65588008)(34.65602214,309.74087873)(36.0960207,309.74087873)
\curveto(37.62601917,309.74087873)(38.42102121,310.46087929)(38.9310207,311.01587873)
\curveto(38.97602065,310.41587933)(39.15602187,309.81587873)(40.3260207,309.81587873)
\curveto(40.6260204,309.81587873)(40.82102092,309.90587879)(41.0460207,309.96587873)
\lineto(41.0460207,310.92587873)
\curveto(40.89602085,310.89587876)(40.73102058,310.86587873)(40.6110207,310.86587873)
\curveto(40.34102097,310.86587873)(40.1760207,311.00087906)(40.1760207,311.33087873)
\lineto(40.1760207,315.84587873)
\curveto(40.1760207,317.85587672)(37.89602007,318.03587873)(37.2660207,318.03587873)
\curveto(35.33102263,318.03587873)(34.08602064,317.30087686)(34.0260207,315.42587873)
\lineto(35.2410207,315.42587873)
\moveto(38.8560207,312.68087873)
\curveto(38.8560207,311.63087978)(37.65601947,310.83587873)(36.4260207,310.83587873)
\curveto(35.43602169,310.83587873)(35.0010207,311.34587959)(35.0010207,312.20087873)
\curveto(35.0010207,313.19087774)(36.03602134,313.38587882)(36.6810207,313.47587873)
\curveto(38.31601906,313.68587852)(38.64602091,313.8058789)(38.8560207,313.97087873)
\lineto(38.8560207,312.68087873)
}
}
{
\newrgbcolor{curcolor}{0 0 0}
\pscustom[linestyle=none,fillstyle=solid,fillcolor=curcolor]
{
\newpath
\moveto(48.93063007,320.73587873)
\lineto(47.61063007,320.73587873)
\lineto(47.61063007,316.80587873)
\lineto(47.58063007,316.70087873)
\curveto(47.26563039,317.15087828)(46.66562865,318.03587873)(45.24063007,318.03587873)
\curveto(43.15563216,318.03587873)(41.97063007,316.32587653)(41.97063007,314.12087873)
\curveto(41.97063007,312.24588061)(42.75063274,309.74087873)(45.42063007,309.74087873)
\curveto(46.18562931,309.74087873)(47.08563064,309.9808798)(47.65563007,311.04587873)
\lineto(47.68563007,311.04587873)
\lineto(47.68563007,309.96587873)
\lineto(48.93063007,309.96587873)
\lineto(48.93063007,320.73587873)
\moveto(43.33563007,313.91087873)
\curveto(43.33563007,314.91587773)(43.44063211,316.85087873)(45.48063007,316.85087873)
\curveto(47.38562817,316.85087873)(47.59563007,314.79587746)(47.59563007,313.52087873)
\curveto(47.59563007,311.43588082)(46.29062923,310.88087873)(45.45063007,310.88087873)
\curveto(44.01063151,310.88087873)(43.33563007,312.18588046)(43.33563007,313.91087873)
}
}
{
\newrgbcolor{curcolor}{0 0 0}
\pscustom[linestyle=none,fillstyle=solid,fillcolor=curcolor]
{
\newpath
\moveto(50.32023945,313.89587873)
\curveto(50.32023945,311.87088076)(51.46024195,309.75587873)(53.96523945,309.75587873)
\curveto(56.47023694,309.75587873)(57.61023945,311.87088076)(57.61023945,313.89587873)
\curveto(57.61023945,315.92087671)(56.47023694,318.03587873)(53.96523945,318.03587873)
\curveto(51.46024195,318.03587873)(50.32023945,315.92087671)(50.32023945,313.89587873)
\moveto(51.68523945,313.89587873)
\curveto(51.68523945,314.94587768)(52.07524134,316.89587873)(53.96523945,316.89587873)
\curveto(55.85523756,316.89587873)(56.24523945,314.94587768)(56.24523945,313.89587873)
\curveto(56.24523945,312.84587978)(55.85523756,310.89587873)(53.96523945,310.89587873)
\curveto(52.07524134,310.89587873)(51.68523945,312.84587978)(51.68523945,313.89587873)
}
}
{
\newrgbcolor{curcolor}{0 0 0}
\pscustom[linestyle=none,fillstyle=solid,fillcolor=curcolor]
{
\newpath
\moveto(15.2657082,413.19728498)
\lineto(8.7707082,413.19728498)
\lineto(8.7707082,416.79728498)
\lineto(14.6657082,416.79728498)
\lineto(14.6657082,418.08728498)
\lineto(8.7707082,418.08728498)
\lineto(8.7707082,421.38728498)
\lineto(15.1607082,421.38728498)
\lineto(15.1607082,422.67728498)
\lineto(7.3157082,422.67728498)
\lineto(7.3157082,411.90728498)
\lineto(15.2657082,411.90728498)
\lineto(15.2657082,413.19728498)
}
}
{
\newrgbcolor{curcolor}{0 0 0}
\pscustom[linestyle=none,fillstyle=solid,fillcolor=curcolor]
{
\newpath
\moveto(22.36938007,417.51728498)
\curveto(22.36938007,417.90728459)(22.17437727,419.97728498)(19.36938007,419.97728498)
\curveto(17.82438162,419.97728498)(16.39938007,419.19728326)(16.39938007,417.47228498)
\curveto(16.39938007,416.39228606)(17.11938117,415.83728471)(18.21438007,415.56728498)
\lineto(19.74438007,415.19228498)
\curveto(20.86937895,414.90728527)(21.30438007,414.69728435)(21.30438007,414.06728498)
\curveto(21.30438007,413.19728585)(20.44937913,412.82228498)(19.50438007,412.82228498)
\curveto(17.64438193,412.82228498)(17.46438003,413.8122856)(17.41938007,414.42728498)
\lineto(16.14438007,414.42728498)
\curveto(16.18938003,413.48228593)(16.41438318,411.68228498)(19.51938007,411.68228498)
\curveto(21.2893783,411.68228498)(22.62438007,412.6572866)(22.62438007,414.27728498)
\curveto(22.62438007,415.34228392)(22.05437844,415.94228539)(20.41938007,416.34728498)
\lineto(19.09938007,416.67728498)
\curveto(18.07938109,416.93228473)(17.67438007,417.08228563)(17.67438007,417.72728498)
\curveto(17.67438007,418.70228401)(18.82938048,418.83728498)(19.23438007,418.83728498)
\curveto(20.89937841,418.83728498)(21.07938009,418.01228449)(21.09438007,417.51728498)
\lineto(22.36938007,417.51728498)
}
}
{
\newrgbcolor{curcolor}{0 0 0}
\pscustom[linestyle=none,fillstyle=solid,fillcolor=curcolor]
{
\newpath
\moveto(27.01938007,418.65728498)
\lineto(27.01938007,419.75228498)
\lineto(25.75938007,419.75228498)
\lineto(25.75938007,421.94228498)
\lineto(24.43938007,421.94228498)
\lineto(24.43938007,419.75228498)
\lineto(23.37438007,419.75228498)
\lineto(23.37438007,418.65728498)
\lineto(24.43938007,418.65728498)
\lineto(24.43938007,413.48228498)
\curveto(24.43938007,412.53728593)(24.72438138,411.80228498)(26.02938007,411.80228498)
\curveto(26.16437994,411.80228498)(26.53938055,411.86228503)(27.01938007,411.90728498)
\lineto(27.01938007,412.94228498)
\lineto(26.55438007,412.94228498)
\curveto(26.28438034,412.94228498)(25.75938007,412.9422856)(25.75938007,413.55728498)
\lineto(25.75938007,418.65728498)
\lineto(27.01938007,418.65728498)
}
}
{
\newrgbcolor{curcolor}{0 0 0}
\pscustom[linestyle=none,fillstyle=solid,fillcolor=curcolor]
{
\newpath
\moveto(34.65953632,411.90728498)
\lineto(34.65953632,419.75228498)
\lineto(33.33953632,419.75228498)
\lineto(33.33953632,415.43228498)
\curveto(33.33953632,414.29228612)(32.84453466,412.82228498)(31.17953632,412.82228498)
\curveto(30.32453718,412.82228498)(29.66453632,413.25728627)(29.66453632,414.54728498)
\lineto(29.66453632,419.75228498)
\lineto(28.34453632,419.75228498)
\lineto(28.34453632,414.11228498)
\curveto(28.34453632,412.23728686)(29.73953748,411.68228498)(30.89453632,411.68228498)
\curveto(32.15453506,411.68228498)(32.82953688,412.1622859)(33.38453632,413.07728498)
\lineto(33.41453632,413.04728498)
\lineto(33.41453632,411.90728498)
\lineto(34.65953632,411.90728498)
}
}
{
\newrgbcolor{curcolor}{0 0 0}
\pscustom[linestyle=none,fillstyle=solid,fillcolor=curcolor]
{
\newpath
\moveto(43.1591457,422.67728498)
\lineto(41.8391457,422.67728498)
\lineto(41.8391457,418.74728498)
\lineto(41.8091457,418.64228498)
\curveto(41.49414601,419.09228453)(40.89414427,419.97728498)(39.4691457,419.97728498)
\curveto(37.38414778,419.97728498)(36.1991457,418.26728278)(36.1991457,416.06228498)
\curveto(36.1991457,414.18728686)(36.97914837,411.68228498)(39.6491457,411.68228498)
\curveto(40.41414493,411.68228498)(41.31414627,411.92228605)(41.8841457,412.98728498)
\lineto(41.9141457,412.98728498)
\lineto(41.9141457,411.90728498)
\lineto(43.1591457,411.90728498)
\lineto(43.1591457,422.67728498)
\moveto(37.5641457,415.85228498)
\curveto(37.5641457,416.85728398)(37.66914774,418.79228498)(39.7091457,418.79228498)
\curveto(41.61414379,418.79228498)(41.8241457,416.73728371)(41.8241457,415.46228498)
\curveto(41.8241457,413.37728707)(40.51914486,412.82228498)(39.6791457,412.82228498)
\curveto(38.23914714,412.82228498)(37.5641457,414.12728671)(37.5641457,415.85228498)
}
}
{
\newrgbcolor{curcolor}{0 0 0}
\pscustom[linestyle=none,fillstyle=solid,fillcolor=curcolor]
{
\newpath
\moveto(46.34875507,419.75228498)
\lineto(45.02875507,419.75228498)
\lineto(45.02875507,411.90728498)
\lineto(46.34875507,411.90728498)
\lineto(46.34875507,419.75228498)
\moveto(46.34875507,421.17728498)
\lineto(46.34875507,422.67728498)
\lineto(45.02875507,422.67728498)
\lineto(45.02875507,421.17728498)
\lineto(46.34875507,421.17728498)
}
}
{
\newrgbcolor{curcolor}{0 0 0}
\pscustom[linestyle=none,fillstyle=solid,fillcolor=curcolor]
{
\newpath
\moveto(49.50859882,417.36728498)
\curveto(49.59859873,417.96728438)(49.80860032,418.88228498)(51.30859882,418.88228498)
\curveto(52.55359758,418.88228498)(53.15359882,418.43228416)(53.15359882,417.60728498)
\curveto(53.15359882,416.82728576)(52.77859851,416.70728495)(52.46359882,416.67728498)
\lineto(50.28859882,416.40728498)
\curveto(48.09860101,416.13728525)(47.90359882,414.60728432)(47.90359882,413.94728498)
\curveto(47.90359882,412.59728633)(48.92360026,411.68228498)(50.36359882,411.68228498)
\curveto(51.89359729,411.68228498)(52.68859933,412.40228554)(53.19859882,412.95728498)
\curveto(53.24359878,412.35728558)(53.42359999,411.75728498)(54.59359882,411.75728498)
\curveto(54.89359852,411.75728498)(55.08859905,411.84728504)(55.31359882,411.90728498)
\lineto(55.31359882,412.86728498)
\curveto(55.16359897,412.83728501)(54.9985987,412.80728498)(54.87859882,412.80728498)
\curveto(54.60859909,412.80728498)(54.44359882,412.94228531)(54.44359882,413.27228498)
\lineto(54.44359882,417.78728498)
\curveto(54.44359882,419.79728297)(52.16359819,419.97728498)(51.53359882,419.97728498)
\curveto(49.59860076,419.97728498)(48.35359876,419.24228311)(48.29359882,417.36728498)
\lineto(49.50859882,417.36728498)
\moveto(53.12359882,414.62228498)
\curveto(53.12359882,413.57228603)(51.92359759,412.77728498)(50.69359882,412.77728498)
\curveto(49.70359981,412.77728498)(49.26859882,413.28728584)(49.26859882,414.14228498)
\curveto(49.26859882,415.13228399)(50.30359947,415.32728507)(50.94859882,415.41728498)
\curveto(52.58359719,415.62728477)(52.91359903,415.74728515)(53.12359882,415.91228498)
\lineto(53.12359882,414.62228498)
}
}
{
\newrgbcolor{curcolor}{0 0 0}
\pscustom[linestyle=none,fillstyle=solid,fillcolor=curcolor]
{
\newpath
\moveto(63.0782082,417.24728498)
\curveto(63.0782082,419.48228275)(61.54820698,419.97728498)(60.3332082,419.97728498)
\curveto(58.98320955,419.97728498)(58.24820791,419.06228456)(57.9632082,418.64228498)
\lineto(57.9332082,418.64228498)
\lineto(57.9332082,419.75228498)
\lineto(56.6882082,419.75228498)
\lineto(56.6882082,411.90728498)
\lineto(58.0082082,411.90728498)
\lineto(58.0082082,416.18228498)
\curveto(58.0082082,418.31228285)(59.32820895,418.79228498)(60.0782082,418.79228498)
\curveto(61.36820691,418.79228498)(61.7582082,418.10228362)(61.7582082,416.73728498)
\lineto(61.7582082,411.90728498)
\lineto(63.0782082,411.90728498)
\lineto(63.0782082,417.24728498)
}
}
{
\newrgbcolor{curcolor}{0 0 0}
\pscustom[linestyle=none,fillstyle=solid,fillcolor=curcolor]
{
\newpath
\moveto(67.91781757,418.65728498)
\lineto(67.91781757,419.75228498)
\lineto(66.65781757,419.75228498)
\lineto(66.65781757,421.94228498)
\lineto(65.33781757,421.94228498)
\lineto(65.33781757,419.75228498)
\lineto(64.27281757,419.75228498)
\lineto(64.27281757,418.65728498)
\lineto(65.33781757,418.65728498)
\lineto(65.33781757,413.48228498)
\curveto(65.33781757,412.53728593)(65.62281888,411.80228498)(66.92781757,411.80228498)
\curveto(67.06281744,411.80228498)(67.43781805,411.86228503)(67.91781757,411.90728498)
\lineto(67.91781757,412.94228498)
\lineto(67.45281757,412.94228498)
\curveto(67.18281784,412.94228498)(66.65781757,412.9422856)(66.65781757,413.55728498)
\lineto(66.65781757,418.65728498)
\lineto(67.91781757,418.65728498)
}
}
{
\newrgbcolor{curcolor}{0 0 0}
\pscustom[linestyle=none,fillstyle=solid,fillcolor=curcolor]
{
\newpath
\moveto(74.24500507,414.36728498)
\curveto(74.20000512,413.78228557)(73.46500383,412.82228498)(72.22000507,412.82228498)
\curveto(70.70500659,412.82228498)(69.94000507,413.76728662)(69.94000507,415.40228498)
\lineto(75.67000507,415.40228498)
\curveto(75.67000507,418.17728221)(74.56000281,419.97728498)(72.29500507,419.97728498)
\curveto(69.70000767,419.97728498)(68.53000507,418.04228255)(68.53000507,415.61228498)
\curveto(68.53000507,413.34728725)(69.83500728,411.68228498)(72.04000507,411.68228498)
\curveto(73.30000381,411.68228498)(73.81000543,411.98228522)(74.17000507,412.22228498)
\curveto(75.16000408,412.88228432)(75.52000512,413.99228536)(75.56500507,414.36728498)
\lineto(74.24500507,414.36728498)
\moveto(69.94000507,416.45228498)
\curveto(69.94000507,417.66728377)(70.90000629,418.79228498)(72.11500507,418.79228498)
\curveto(73.72000347,418.79228498)(74.23000515,417.66728377)(74.30500507,416.45228498)
\lineto(69.94000507,416.45228498)
}
}
{
\newrgbcolor{curcolor}{0 0 0}
\pscustom[linestyle=none,fillstyle=solid,fillcolor=curcolor]
{
\newpath
\moveto(15.2657082,277.27540998)
\lineto(8.7707082,277.27540998)
\lineto(8.7707082,280.87540998)
\lineto(14.6657082,280.87540998)
\lineto(14.6657082,282.16540998)
\lineto(8.7707082,282.16540998)
\lineto(8.7707082,285.46540998)
\lineto(15.1607082,285.46540998)
\lineto(15.1607082,286.75540998)
\lineto(7.3157082,286.75540998)
\lineto(7.3157082,275.98540998)
\lineto(15.2657082,275.98540998)
\lineto(15.2657082,277.27540998)
}
}
{
\newrgbcolor{curcolor}{0 0 0}
\pscustom[linestyle=none,fillstyle=solid,fillcolor=curcolor]
{
\newpath
\moveto(22.36938007,281.59540998)
\curveto(22.36938007,281.98540959)(22.17437727,284.05540998)(19.36938007,284.05540998)
\curveto(17.82438162,284.05540998)(16.39938007,283.27540826)(16.39938007,281.55040998)
\curveto(16.39938007,280.47041106)(17.11938117,279.91540971)(18.21438007,279.64540998)
\lineto(19.74438007,279.27040998)
\curveto(20.86937895,278.98541027)(21.30438007,278.77540935)(21.30438007,278.14540998)
\curveto(21.30438007,277.27541085)(20.44937913,276.90040998)(19.50438007,276.90040998)
\curveto(17.64438193,276.90040998)(17.46438003,277.8904106)(17.41938007,278.50540998)
\lineto(16.14438007,278.50540998)
\curveto(16.18938003,277.56041093)(16.41438318,275.76040998)(19.51938007,275.76040998)
\curveto(21.2893783,275.76040998)(22.62438007,276.7354116)(22.62438007,278.35540998)
\curveto(22.62438007,279.42040892)(22.05437844,280.02041039)(20.41938007,280.42540998)
\lineto(19.09938007,280.75540998)
\curveto(18.07938109,281.01040973)(17.67438007,281.16041063)(17.67438007,281.80540998)
\curveto(17.67438007,282.78040901)(18.82938048,282.91540998)(19.23438007,282.91540998)
\curveto(20.89937841,282.91540998)(21.07938009,282.09040949)(21.09438007,281.59540998)
\lineto(22.36938007,281.59540998)
}
}
{
\newrgbcolor{curcolor}{0 0 0}
\pscustom[linestyle=none,fillstyle=solid,fillcolor=curcolor]
{
\newpath
\moveto(27.01938007,282.73540998)
\lineto(27.01938007,283.83040998)
\lineto(25.75938007,283.83040998)
\lineto(25.75938007,286.02040998)
\lineto(24.43938007,286.02040998)
\lineto(24.43938007,283.83040998)
\lineto(23.37438007,283.83040998)
\lineto(23.37438007,282.73540998)
\lineto(24.43938007,282.73540998)
\lineto(24.43938007,277.56040998)
\curveto(24.43938007,276.61541093)(24.72438138,275.88040998)(26.02938007,275.88040998)
\curveto(26.16437994,275.88040998)(26.53938055,275.94041003)(27.01938007,275.98540998)
\lineto(27.01938007,277.02040998)
\lineto(26.55438007,277.02040998)
\curveto(26.28438034,277.02040998)(25.75938007,277.0204106)(25.75938007,277.63540998)
\lineto(25.75938007,282.73540998)
\lineto(27.01938007,282.73540998)
}
}
{
\newrgbcolor{curcolor}{0 0 0}
\pscustom[linestyle=none,fillstyle=solid,fillcolor=curcolor]
{
\newpath
\moveto(34.65953632,275.98540998)
\lineto(34.65953632,283.83040998)
\lineto(33.33953632,283.83040998)
\lineto(33.33953632,279.51040998)
\curveto(33.33953632,278.37041112)(32.84453466,276.90040998)(31.17953632,276.90040998)
\curveto(30.32453718,276.90040998)(29.66453632,277.33541127)(29.66453632,278.62540998)
\lineto(29.66453632,283.83040998)
\lineto(28.34453632,283.83040998)
\lineto(28.34453632,278.19040998)
\curveto(28.34453632,276.31541186)(29.73953748,275.76040998)(30.89453632,275.76040998)
\curveto(32.15453506,275.76040998)(32.82953688,276.2404109)(33.38453632,277.15540998)
\lineto(33.41453632,277.12540998)
\lineto(33.41453632,275.98540998)
\lineto(34.65953632,275.98540998)
}
}
{
\newrgbcolor{curcolor}{0 0 0}
\pscustom[linestyle=none,fillstyle=solid,fillcolor=curcolor]
{
\newpath
\moveto(43.1591457,286.75540998)
\lineto(41.8391457,286.75540998)
\lineto(41.8391457,282.82540998)
\lineto(41.8091457,282.72040998)
\curveto(41.49414601,283.17040953)(40.89414427,284.05540998)(39.4691457,284.05540998)
\curveto(37.38414778,284.05540998)(36.1991457,282.34540778)(36.1991457,280.14040998)
\curveto(36.1991457,278.26541186)(36.97914837,275.76040998)(39.6491457,275.76040998)
\curveto(40.41414493,275.76040998)(41.31414627,276.00041105)(41.8841457,277.06540998)
\lineto(41.9141457,277.06540998)
\lineto(41.9141457,275.98540998)
\lineto(43.1591457,275.98540998)
\lineto(43.1591457,286.75540998)
\moveto(37.5641457,279.93040998)
\curveto(37.5641457,280.93540898)(37.66914774,282.87040998)(39.7091457,282.87040998)
\curveto(41.61414379,282.87040998)(41.8241457,280.81540871)(41.8241457,279.54040998)
\curveto(41.8241457,277.45541207)(40.51914486,276.90040998)(39.6791457,276.90040998)
\curveto(38.23914714,276.90040998)(37.5641457,278.20541171)(37.5641457,279.93040998)
}
}
{
\newrgbcolor{curcolor}{0 0 0}
\pscustom[linestyle=none,fillstyle=solid,fillcolor=curcolor]
{
\newpath
\moveto(46.34875507,283.83040998)
\lineto(45.02875507,283.83040998)
\lineto(45.02875507,275.98540998)
\lineto(46.34875507,275.98540998)
\lineto(46.34875507,283.83040998)
\moveto(46.34875507,285.25540998)
\lineto(46.34875507,286.75540998)
\lineto(45.02875507,286.75540998)
\lineto(45.02875507,285.25540998)
\lineto(46.34875507,285.25540998)
}
}
{
\newrgbcolor{curcolor}{0 0 0}
\pscustom[linestyle=none,fillstyle=solid,fillcolor=curcolor]
{
\newpath
\moveto(49.50859882,281.44540998)
\curveto(49.59859873,282.04540938)(49.80860032,282.96040998)(51.30859882,282.96040998)
\curveto(52.55359758,282.96040998)(53.15359882,282.51040916)(53.15359882,281.68540998)
\curveto(53.15359882,280.90541076)(52.77859851,280.78540995)(52.46359882,280.75540998)
\lineto(50.28859882,280.48540998)
\curveto(48.09860101,280.21541025)(47.90359882,278.68540932)(47.90359882,278.02540998)
\curveto(47.90359882,276.67541133)(48.92360026,275.76040998)(50.36359882,275.76040998)
\curveto(51.89359729,275.76040998)(52.68859933,276.48041054)(53.19859882,277.03540998)
\curveto(53.24359878,276.43541058)(53.42359999,275.83540998)(54.59359882,275.83540998)
\curveto(54.89359852,275.83540998)(55.08859905,275.92541004)(55.31359882,275.98540998)
\lineto(55.31359882,276.94540998)
\curveto(55.16359897,276.91541001)(54.9985987,276.88540998)(54.87859882,276.88540998)
\curveto(54.60859909,276.88540998)(54.44359882,277.02041031)(54.44359882,277.35040998)
\lineto(54.44359882,281.86540998)
\curveto(54.44359882,283.87540797)(52.16359819,284.05540998)(51.53359882,284.05540998)
\curveto(49.59860076,284.05540998)(48.35359876,283.32040811)(48.29359882,281.44540998)
\lineto(49.50859882,281.44540998)
\moveto(53.12359882,278.70040998)
\curveto(53.12359882,277.65041103)(51.92359759,276.85540998)(50.69359882,276.85540998)
\curveto(49.70359981,276.85540998)(49.26859882,277.36541084)(49.26859882,278.22040998)
\curveto(49.26859882,279.21040899)(50.30359947,279.40541007)(50.94859882,279.49540998)
\curveto(52.58359719,279.70540977)(52.91359903,279.82541015)(53.12359882,279.99040998)
\lineto(53.12359882,278.70040998)
}
}
{
\newrgbcolor{curcolor}{0 0 0}
\pscustom[linestyle=none,fillstyle=solid,fillcolor=curcolor]
{
\newpath
\moveto(63.0782082,281.32540998)
\curveto(63.0782082,283.56040775)(61.54820698,284.05540998)(60.3332082,284.05540998)
\curveto(58.98320955,284.05540998)(58.24820791,283.14040956)(57.9632082,282.72040998)
\lineto(57.9332082,282.72040998)
\lineto(57.9332082,283.83040998)
\lineto(56.6882082,283.83040998)
\lineto(56.6882082,275.98540998)
\lineto(58.0082082,275.98540998)
\lineto(58.0082082,280.26040998)
\curveto(58.0082082,282.39040785)(59.32820895,282.87040998)(60.0782082,282.87040998)
\curveto(61.36820691,282.87040998)(61.7582082,282.18040862)(61.7582082,280.81540998)
\lineto(61.7582082,275.98540998)
\lineto(63.0782082,275.98540998)
\lineto(63.0782082,281.32540998)
}
}
{
\newrgbcolor{curcolor}{0 0 0}
\pscustom[linestyle=none,fillstyle=solid,fillcolor=curcolor]
{
\newpath
\moveto(67.91781757,282.73540998)
\lineto(67.91781757,283.83040998)
\lineto(66.65781757,283.83040998)
\lineto(66.65781757,286.02040998)
\lineto(65.33781757,286.02040998)
\lineto(65.33781757,283.83040998)
\lineto(64.27281757,283.83040998)
\lineto(64.27281757,282.73540998)
\lineto(65.33781757,282.73540998)
\lineto(65.33781757,277.56040998)
\curveto(65.33781757,276.61541093)(65.62281888,275.88040998)(66.92781757,275.88040998)
\curveto(67.06281744,275.88040998)(67.43781805,275.94041003)(67.91781757,275.98540998)
\lineto(67.91781757,277.02040998)
\lineto(67.45281757,277.02040998)
\curveto(67.18281784,277.02040998)(66.65781757,277.0204106)(66.65781757,277.63540998)
\lineto(66.65781757,282.73540998)
\lineto(67.91781757,282.73540998)
}
}
{
\newrgbcolor{curcolor}{0 0 0}
\pscustom[linestyle=none,fillstyle=solid,fillcolor=curcolor]
{
\newpath
\moveto(74.24500507,278.44540998)
\curveto(74.20000512,277.86041057)(73.46500383,276.90040998)(72.22000507,276.90040998)
\curveto(70.70500659,276.90040998)(69.94000507,277.84541162)(69.94000507,279.48040998)
\lineto(75.67000507,279.48040998)
\curveto(75.67000507,282.25540721)(74.56000281,284.05540998)(72.29500507,284.05540998)
\curveto(69.70000767,284.05540998)(68.53000507,282.12040755)(68.53000507,279.69040998)
\curveto(68.53000507,277.42541225)(69.83500728,275.76040998)(72.04000507,275.76040998)
\curveto(73.30000381,275.76040998)(73.81000543,276.06041022)(74.17000507,276.30040998)
\curveto(75.16000408,276.96040932)(75.52000512,278.07041036)(75.56500507,278.44540998)
\lineto(74.24500507,278.44540998)
\moveto(69.94000507,280.53040998)
\curveto(69.94000507,281.74540877)(70.90000629,282.87040998)(72.11500507,282.87040998)
\curveto(73.72000347,282.87040998)(74.23000515,281.74540877)(74.30500507,280.53040998)
\lineto(69.94000507,280.53040998)
}
}
{
\newrgbcolor{curcolor}{0 0 0}
\pscustom[linestyle=none,fillstyle=solid,fillcolor=curcolor]
{
\newpath
\moveto(15.2657082,243.29494123)
\lineto(8.7707082,243.29494123)
\lineto(8.7707082,246.89494123)
\lineto(14.6657082,246.89494123)
\lineto(14.6657082,248.18494123)
\lineto(8.7707082,248.18494123)
\lineto(8.7707082,251.48494123)
\lineto(15.1607082,251.48494123)
\lineto(15.1607082,252.77494123)
\lineto(7.3157082,252.77494123)
\lineto(7.3157082,242.00494123)
\lineto(15.2657082,242.00494123)
\lineto(15.2657082,243.29494123)
}
}
{
\newrgbcolor{curcolor}{0 0 0}
\pscustom[linestyle=none,fillstyle=solid,fillcolor=curcolor]
{
\newpath
\moveto(22.36938007,247.61494123)
\curveto(22.36938007,248.00494084)(22.17437727,250.07494123)(19.36938007,250.07494123)
\curveto(17.82438162,250.07494123)(16.39938007,249.29493951)(16.39938007,247.56994123)
\curveto(16.39938007,246.48994231)(17.11938117,245.93494096)(18.21438007,245.66494123)
\lineto(19.74438007,245.28994123)
\curveto(20.86937895,245.00494152)(21.30438007,244.7949406)(21.30438007,244.16494123)
\curveto(21.30438007,243.2949421)(20.44937913,242.91994123)(19.50438007,242.91994123)
\curveto(17.64438193,242.91994123)(17.46438003,243.90994185)(17.41938007,244.52494123)
\lineto(16.14438007,244.52494123)
\curveto(16.18938003,243.57994218)(16.41438318,241.77994123)(19.51938007,241.77994123)
\curveto(21.2893783,241.77994123)(22.62438007,242.75494285)(22.62438007,244.37494123)
\curveto(22.62438007,245.43994017)(22.05437844,246.03994164)(20.41938007,246.44494123)
\lineto(19.09938007,246.77494123)
\curveto(18.07938109,247.02994098)(17.67438007,247.17994188)(17.67438007,247.82494123)
\curveto(17.67438007,248.79994026)(18.82938048,248.93494123)(19.23438007,248.93494123)
\curveto(20.89937841,248.93494123)(21.07938009,248.10994074)(21.09438007,247.61494123)
\lineto(22.36938007,247.61494123)
}
}
{
\newrgbcolor{curcolor}{0 0 0}
\pscustom[linestyle=none,fillstyle=solid,fillcolor=curcolor]
{
\newpath
\moveto(27.01938007,248.75494123)
\lineto(27.01938007,249.84994123)
\lineto(25.75938007,249.84994123)
\lineto(25.75938007,252.03994123)
\lineto(24.43938007,252.03994123)
\lineto(24.43938007,249.84994123)
\lineto(23.37438007,249.84994123)
\lineto(23.37438007,248.75494123)
\lineto(24.43938007,248.75494123)
\lineto(24.43938007,243.57994123)
\curveto(24.43938007,242.63494218)(24.72438138,241.89994123)(26.02938007,241.89994123)
\curveto(26.16437994,241.89994123)(26.53938055,241.95994128)(27.01938007,242.00494123)
\lineto(27.01938007,243.03994123)
\lineto(26.55438007,243.03994123)
\curveto(26.28438034,243.03994123)(25.75938007,243.03994185)(25.75938007,243.65494123)
\lineto(25.75938007,248.75494123)
\lineto(27.01938007,248.75494123)
}
}
{
\newrgbcolor{curcolor}{0 0 0}
\pscustom[linestyle=none,fillstyle=solid,fillcolor=curcolor]
{
\newpath
\moveto(34.65953632,242.00494123)
\lineto(34.65953632,249.84994123)
\lineto(33.33953632,249.84994123)
\lineto(33.33953632,245.52994123)
\curveto(33.33953632,244.38994237)(32.84453466,242.91994123)(31.17953632,242.91994123)
\curveto(30.32453718,242.91994123)(29.66453632,243.35494252)(29.66453632,244.64494123)
\lineto(29.66453632,249.84994123)
\lineto(28.34453632,249.84994123)
\lineto(28.34453632,244.20994123)
\curveto(28.34453632,242.33494311)(29.73953748,241.77994123)(30.89453632,241.77994123)
\curveto(32.15453506,241.77994123)(32.82953688,242.25994215)(33.38453632,243.17494123)
\lineto(33.41453632,243.14494123)
\lineto(33.41453632,242.00494123)
\lineto(34.65953632,242.00494123)
}
}
{
\newrgbcolor{curcolor}{0 0 0}
\pscustom[linestyle=none,fillstyle=solid,fillcolor=curcolor]
{
\newpath
\moveto(43.1591457,252.77494123)
\lineto(41.8391457,252.77494123)
\lineto(41.8391457,248.84494123)
\lineto(41.8091457,248.73994123)
\curveto(41.49414601,249.18994078)(40.89414427,250.07494123)(39.4691457,250.07494123)
\curveto(37.38414778,250.07494123)(36.1991457,248.36493903)(36.1991457,246.15994123)
\curveto(36.1991457,244.28494311)(36.97914837,241.77994123)(39.6491457,241.77994123)
\curveto(40.41414493,241.77994123)(41.31414627,242.0199423)(41.8841457,243.08494123)
\lineto(41.9141457,243.08494123)
\lineto(41.9141457,242.00494123)
\lineto(43.1591457,242.00494123)
\lineto(43.1591457,252.77494123)
\moveto(37.5641457,245.94994123)
\curveto(37.5641457,246.95494023)(37.66914774,248.88994123)(39.7091457,248.88994123)
\curveto(41.61414379,248.88994123)(41.8241457,246.83493996)(41.8241457,245.55994123)
\curveto(41.8241457,243.47494332)(40.51914486,242.91994123)(39.6791457,242.91994123)
\curveto(38.23914714,242.91994123)(37.5641457,244.22494296)(37.5641457,245.94994123)
}
}
{
\newrgbcolor{curcolor}{0 0 0}
\pscustom[linestyle=none,fillstyle=solid,fillcolor=curcolor]
{
\newpath
\moveto(46.34875507,249.84994123)
\lineto(45.02875507,249.84994123)
\lineto(45.02875507,242.00494123)
\lineto(46.34875507,242.00494123)
\lineto(46.34875507,249.84994123)
\moveto(46.34875507,251.27494123)
\lineto(46.34875507,252.77494123)
\lineto(45.02875507,252.77494123)
\lineto(45.02875507,251.27494123)
\lineto(46.34875507,251.27494123)
}
}
{
\newrgbcolor{curcolor}{0 0 0}
\pscustom[linestyle=none,fillstyle=solid,fillcolor=curcolor]
{
\newpath
\moveto(49.50859882,247.46494123)
\curveto(49.59859873,248.06494063)(49.80860032,248.97994123)(51.30859882,248.97994123)
\curveto(52.55359758,248.97994123)(53.15359882,248.52994041)(53.15359882,247.70494123)
\curveto(53.15359882,246.92494201)(52.77859851,246.8049412)(52.46359882,246.77494123)
\lineto(50.28859882,246.50494123)
\curveto(48.09860101,246.2349415)(47.90359882,244.70494057)(47.90359882,244.04494123)
\curveto(47.90359882,242.69494258)(48.92360026,241.77994123)(50.36359882,241.77994123)
\curveto(51.89359729,241.77994123)(52.68859933,242.49994179)(53.19859882,243.05494123)
\curveto(53.24359878,242.45494183)(53.42359999,241.85494123)(54.59359882,241.85494123)
\curveto(54.89359852,241.85494123)(55.08859905,241.94494129)(55.31359882,242.00494123)
\lineto(55.31359882,242.96494123)
\curveto(55.16359897,242.93494126)(54.9985987,242.90494123)(54.87859882,242.90494123)
\curveto(54.60859909,242.90494123)(54.44359882,243.03994156)(54.44359882,243.36994123)
\lineto(54.44359882,247.88494123)
\curveto(54.44359882,249.89493922)(52.16359819,250.07494123)(51.53359882,250.07494123)
\curveto(49.59860076,250.07494123)(48.35359876,249.33993936)(48.29359882,247.46494123)
\lineto(49.50859882,247.46494123)
\moveto(53.12359882,244.71994123)
\curveto(53.12359882,243.66994228)(51.92359759,242.87494123)(50.69359882,242.87494123)
\curveto(49.70359981,242.87494123)(49.26859882,243.38494209)(49.26859882,244.23994123)
\curveto(49.26859882,245.22994024)(50.30359947,245.42494132)(50.94859882,245.51494123)
\curveto(52.58359719,245.72494102)(52.91359903,245.8449414)(53.12359882,246.00994123)
\lineto(53.12359882,244.71994123)
}
}
{
\newrgbcolor{curcolor}{0 0 0}
\pscustom[linestyle=none,fillstyle=solid,fillcolor=curcolor]
{
\newpath
\moveto(63.0782082,247.34494123)
\curveto(63.0782082,249.579939)(61.54820698,250.07494123)(60.3332082,250.07494123)
\curveto(58.98320955,250.07494123)(58.24820791,249.15994081)(57.9632082,248.73994123)
\lineto(57.9332082,248.73994123)
\lineto(57.9332082,249.84994123)
\lineto(56.6882082,249.84994123)
\lineto(56.6882082,242.00494123)
\lineto(58.0082082,242.00494123)
\lineto(58.0082082,246.27994123)
\curveto(58.0082082,248.4099391)(59.32820895,248.88994123)(60.0782082,248.88994123)
\curveto(61.36820691,248.88994123)(61.7582082,248.19993987)(61.7582082,246.83494123)
\lineto(61.7582082,242.00494123)
\lineto(63.0782082,242.00494123)
\lineto(63.0782082,247.34494123)
}
}
{
\newrgbcolor{curcolor}{0 0 0}
\pscustom[linestyle=none,fillstyle=solid,fillcolor=curcolor]
{
\newpath
\moveto(67.91781757,248.75494123)
\lineto(67.91781757,249.84994123)
\lineto(66.65781757,249.84994123)
\lineto(66.65781757,252.03994123)
\lineto(65.33781757,252.03994123)
\lineto(65.33781757,249.84994123)
\lineto(64.27281757,249.84994123)
\lineto(64.27281757,248.75494123)
\lineto(65.33781757,248.75494123)
\lineto(65.33781757,243.57994123)
\curveto(65.33781757,242.63494218)(65.62281888,241.89994123)(66.92781757,241.89994123)
\curveto(67.06281744,241.89994123)(67.43781805,241.95994128)(67.91781757,242.00494123)
\lineto(67.91781757,243.03994123)
\lineto(67.45281757,243.03994123)
\curveto(67.18281784,243.03994123)(66.65781757,243.03994185)(66.65781757,243.65494123)
\lineto(66.65781757,248.75494123)
\lineto(67.91781757,248.75494123)
}
}
{
\newrgbcolor{curcolor}{0 0 0}
\pscustom[linestyle=none,fillstyle=solid,fillcolor=curcolor]
{
\newpath
\moveto(74.24500507,244.46494123)
\curveto(74.20000512,243.87994182)(73.46500383,242.91994123)(72.22000507,242.91994123)
\curveto(70.70500659,242.91994123)(69.94000507,243.86494287)(69.94000507,245.49994123)
\lineto(75.67000507,245.49994123)
\curveto(75.67000507,248.27493846)(74.56000281,250.07494123)(72.29500507,250.07494123)
\curveto(69.70000767,250.07494123)(68.53000507,248.1399388)(68.53000507,245.70994123)
\curveto(68.53000507,243.4449435)(69.83500728,241.77994123)(72.04000507,241.77994123)
\curveto(73.30000381,241.77994123)(73.81000543,242.07994147)(74.17000507,242.31994123)
\curveto(75.16000408,242.97994057)(75.52000512,244.08994161)(75.56500507,244.46494123)
\lineto(74.24500507,244.46494123)
\moveto(69.94000507,246.54994123)
\curveto(69.94000507,247.76494002)(70.90000629,248.88994123)(72.11500507,248.88994123)
\curveto(73.72000347,248.88994123)(74.23000515,247.76494002)(74.30500507,246.54994123)
\lineto(69.94000507,246.54994123)
}
}
{
\newrgbcolor{curcolor}{0 0 0}
\pscustom[linestyle=none,fillstyle=solid,fillcolor=curcolor]
{
\newpath
\moveto(15.2657082,226.30470686)
\lineto(8.7707082,226.30470686)
\lineto(8.7707082,229.90470686)
\lineto(14.6657082,229.90470686)
\lineto(14.6657082,231.19470686)
\lineto(8.7707082,231.19470686)
\lineto(8.7707082,234.49470686)
\lineto(15.1607082,234.49470686)
\lineto(15.1607082,235.78470686)
\lineto(7.3157082,235.78470686)
\lineto(7.3157082,225.01470686)
\lineto(15.2657082,225.01470686)
\lineto(15.2657082,226.30470686)
}
}
{
\newrgbcolor{curcolor}{0 0 0}
\pscustom[linestyle=none,fillstyle=solid,fillcolor=curcolor]
{
\newpath
\moveto(22.36938007,230.62470686)
\curveto(22.36938007,231.01470647)(22.17437727,233.08470686)(19.36938007,233.08470686)
\curveto(17.82438162,233.08470686)(16.39938007,232.30470513)(16.39938007,230.57970686)
\curveto(16.39938007,229.49970794)(17.11938117,228.94470659)(18.21438007,228.67470686)
\lineto(19.74438007,228.29970686)
\curveto(20.86937895,228.01470714)(21.30438007,227.80470623)(21.30438007,227.17470686)
\curveto(21.30438007,226.30470773)(20.44937913,225.92970686)(19.50438007,225.92970686)
\curveto(17.64438193,225.92970686)(17.46438003,226.91970747)(17.41938007,227.53470686)
\lineto(16.14438007,227.53470686)
\curveto(16.18938003,226.5897078)(16.41438318,224.78970686)(19.51938007,224.78970686)
\curveto(21.2893783,224.78970686)(22.62438007,225.76470848)(22.62438007,227.38470686)
\curveto(22.62438007,228.44970579)(22.05437844,229.04970726)(20.41938007,229.45470686)
\lineto(19.09938007,229.78470686)
\curveto(18.07938109,230.0397066)(17.67438007,230.1897075)(17.67438007,230.83470686)
\curveto(17.67438007,231.80970588)(18.82938048,231.94470686)(19.23438007,231.94470686)
\curveto(20.89937841,231.94470686)(21.07938009,231.11970636)(21.09438007,230.62470686)
\lineto(22.36938007,230.62470686)
}
}
{
\newrgbcolor{curcolor}{0 0 0}
\pscustom[linestyle=none,fillstyle=solid,fillcolor=curcolor]
{
\newpath
\moveto(27.01938007,231.76470686)
\lineto(27.01938007,232.85970686)
\lineto(25.75938007,232.85970686)
\lineto(25.75938007,235.04970686)
\lineto(24.43938007,235.04970686)
\lineto(24.43938007,232.85970686)
\lineto(23.37438007,232.85970686)
\lineto(23.37438007,231.76470686)
\lineto(24.43938007,231.76470686)
\lineto(24.43938007,226.58970686)
\curveto(24.43938007,225.6447078)(24.72438138,224.90970686)(26.02938007,224.90970686)
\curveto(26.16437994,224.90970686)(26.53938055,224.9697069)(27.01938007,225.01470686)
\lineto(27.01938007,226.04970686)
\lineto(26.55438007,226.04970686)
\curveto(26.28438034,226.04970686)(25.75938007,226.04970747)(25.75938007,226.66470686)
\lineto(25.75938007,231.76470686)
\lineto(27.01938007,231.76470686)
}
}
{
\newrgbcolor{curcolor}{0 0 0}
\pscustom[linestyle=none,fillstyle=solid,fillcolor=curcolor]
{
\newpath
\moveto(34.65953632,225.01470686)
\lineto(34.65953632,232.85970686)
\lineto(33.33953632,232.85970686)
\lineto(33.33953632,228.53970686)
\curveto(33.33953632,227.399708)(32.84453466,225.92970686)(31.17953632,225.92970686)
\curveto(30.32453718,225.92970686)(29.66453632,226.36470815)(29.66453632,227.65470686)
\lineto(29.66453632,232.85970686)
\lineto(28.34453632,232.85970686)
\lineto(28.34453632,227.21970686)
\curveto(28.34453632,225.34470873)(29.73953748,224.78970686)(30.89453632,224.78970686)
\curveto(32.15453506,224.78970686)(32.82953688,225.26970777)(33.38453632,226.18470686)
\lineto(33.41453632,226.15470686)
\lineto(33.41453632,225.01470686)
\lineto(34.65953632,225.01470686)
}
}
{
\newrgbcolor{curcolor}{0 0 0}
\pscustom[linestyle=none,fillstyle=solid,fillcolor=curcolor]
{
\newpath
\moveto(43.1591457,235.78470686)
\lineto(41.8391457,235.78470686)
\lineto(41.8391457,231.85470686)
\lineto(41.8091457,231.74970686)
\curveto(41.49414601,232.19970641)(40.89414427,233.08470686)(39.4691457,233.08470686)
\curveto(37.38414778,233.08470686)(36.1991457,231.37470465)(36.1991457,229.16970686)
\curveto(36.1991457,227.29470873)(36.97914837,224.78970686)(39.6491457,224.78970686)
\curveto(40.41414493,224.78970686)(41.31414627,225.02970792)(41.8841457,226.09470686)
\lineto(41.9141457,226.09470686)
\lineto(41.9141457,225.01470686)
\lineto(43.1591457,225.01470686)
\lineto(43.1591457,235.78470686)
\moveto(37.5641457,228.95970686)
\curveto(37.5641457,229.96470585)(37.66914774,231.89970686)(39.7091457,231.89970686)
\curveto(41.61414379,231.89970686)(41.8241457,229.84470558)(41.8241457,228.56970686)
\curveto(41.8241457,226.48470894)(40.51914486,225.92970686)(39.6791457,225.92970686)
\curveto(38.23914714,225.92970686)(37.5641457,227.23470858)(37.5641457,228.95970686)
}
}
{
\newrgbcolor{curcolor}{0 0 0}
\pscustom[linestyle=none,fillstyle=solid,fillcolor=curcolor]
{
\newpath
\moveto(46.34875507,232.85970686)
\lineto(45.02875507,232.85970686)
\lineto(45.02875507,225.01470686)
\lineto(46.34875507,225.01470686)
\lineto(46.34875507,232.85970686)
\moveto(46.34875507,234.28470686)
\lineto(46.34875507,235.78470686)
\lineto(45.02875507,235.78470686)
\lineto(45.02875507,234.28470686)
\lineto(46.34875507,234.28470686)
}
}
{
\newrgbcolor{curcolor}{0 0 0}
\pscustom[linestyle=none,fillstyle=solid,fillcolor=curcolor]
{
\newpath
\moveto(49.50859882,230.47470686)
\curveto(49.59859873,231.07470626)(49.80860032,231.98970686)(51.30859882,231.98970686)
\curveto(52.55359758,231.98970686)(53.15359882,231.53970603)(53.15359882,230.71470686)
\curveto(53.15359882,229.93470764)(52.77859851,229.81470683)(52.46359882,229.78470686)
\lineto(50.28859882,229.51470686)
\curveto(48.09860101,229.24470713)(47.90359882,227.7147062)(47.90359882,227.05470686)
\curveto(47.90359882,225.70470821)(48.92360026,224.78970686)(50.36359882,224.78970686)
\curveto(51.89359729,224.78970686)(52.68859933,225.50970741)(53.19859882,226.06470686)
\curveto(53.24359878,225.46470746)(53.42359999,224.86470686)(54.59359882,224.86470686)
\curveto(54.89359852,224.86470686)(55.08859905,224.95470692)(55.31359882,225.01470686)
\lineto(55.31359882,225.97470686)
\curveto(55.16359897,225.94470689)(54.9985987,225.91470686)(54.87859882,225.91470686)
\curveto(54.60859909,225.91470686)(54.44359882,226.04970719)(54.44359882,226.37970686)
\lineto(54.44359882,230.89470686)
\curveto(54.44359882,232.90470485)(52.16359819,233.08470686)(51.53359882,233.08470686)
\curveto(49.59860076,233.08470686)(48.35359876,232.34970498)(48.29359882,230.47470686)
\lineto(49.50859882,230.47470686)
\moveto(53.12359882,227.72970686)
\curveto(53.12359882,226.67970791)(51.92359759,225.88470686)(50.69359882,225.88470686)
\curveto(49.70359981,225.88470686)(49.26859882,226.39470771)(49.26859882,227.24970686)
\curveto(49.26859882,228.23970587)(50.30359947,228.43470695)(50.94859882,228.52470686)
\curveto(52.58359719,228.73470665)(52.91359903,228.85470702)(53.12359882,229.01970686)
\lineto(53.12359882,227.72970686)
}
}
{
\newrgbcolor{curcolor}{0 0 0}
\pscustom[linestyle=none,fillstyle=solid,fillcolor=curcolor]
{
\newpath
\moveto(63.0782082,230.35470686)
\curveto(63.0782082,232.58970462)(61.54820698,233.08470686)(60.3332082,233.08470686)
\curveto(58.98320955,233.08470686)(58.24820791,232.16970644)(57.9632082,231.74970686)
\lineto(57.9332082,231.74970686)
\lineto(57.9332082,232.85970686)
\lineto(56.6882082,232.85970686)
\lineto(56.6882082,225.01470686)
\lineto(58.0082082,225.01470686)
\lineto(58.0082082,229.28970686)
\curveto(58.0082082,231.41970473)(59.32820895,231.89970686)(60.0782082,231.89970686)
\curveto(61.36820691,231.89970686)(61.7582082,231.20970549)(61.7582082,229.84470686)
\lineto(61.7582082,225.01470686)
\lineto(63.0782082,225.01470686)
\lineto(63.0782082,230.35470686)
}
}
{
\newrgbcolor{curcolor}{0 0 0}
\pscustom[linestyle=none,fillstyle=solid,fillcolor=curcolor]
{
\newpath
\moveto(67.91781757,231.76470686)
\lineto(67.91781757,232.85970686)
\lineto(66.65781757,232.85970686)
\lineto(66.65781757,235.04970686)
\lineto(65.33781757,235.04970686)
\lineto(65.33781757,232.85970686)
\lineto(64.27281757,232.85970686)
\lineto(64.27281757,231.76470686)
\lineto(65.33781757,231.76470686)
\lineto(65.33781757,226.58970686)
\curveto(65.33781757,225.6447078)(65.62281888,224.90970686)(66.92781757,224.90970686)
\curveto(67.06281744,224.90970686)(67.43781805,224.9697069)(67.91781757,225.01470686)
\lineto(67.91781757,226.04970686)
\lineto(67.45281757,226.04970686)
\curveto(67.18281784,226.04970686)(66.65781757,226.04970747)(66.65781757,226.66470686)
\lineto(66.65781757,231.76470686)
\lineto(67.91781757,231.76470686)
}
}
{
\newrgbcolor{curcolor}{0 0 0}
\pscustom[linestyle=none,fillstyle=solid,fillcolor=curcolor]
{
\newpath
\moveto(74.24500507,227.47470686)
\curveto(74.20000512,226.88970744)(73.46500383,225.92970686)(72.22000507,225.92970686)
\curveto(70.70500659,225.92970686)(69.94000507,226.87470849)(69.94000507,228.50970686)
\lineto(75.67000507,228.50970686)
\curveto(75.67000507,231.28470408)(74.56000281,233.08470686)(72.29500507,233.08470686)
\curveto(69.70000767,233.08470686)(68.53000507,231.14970443)(68.53000507,228.71970686)
\curveto(68.53000507,226.45470912)(69.83500728,224.78970686)(72.04000507,224.78970686)
\curveto(73.30000381,224.78970686)(73.81000543,225.0897071)(74.17000507,225.32970686)
\curveto(75.16000408,225.9897062)(75.52000512,227.09970723)(75.56500507,227.47470686)
\lineto(74.24500507,227.47470686)
\moveto(69.94000507,229.55970686)
\curveto(69.94000507,230.77470564)(70.90000629,231.89970686)(72.11500507,231.89970686)
\curveto(73.72000347,231.89970686)(74.23000515,230.77470564)(74.30500507,229.55970686)
\lineto(69.94000507,229.55970686)
}
}
{
\newrgbcolor{curcolor}{0 0 0}
\pscustom[linestyle=none,fillstyle=solid,fillcolor=curcolor]
{
\newpath
\moveto(15.2657082,192.32423811)
\lineto(8.7707082,192.32423811)
\lineto(8.7707082,195.92423811)
\lineto(14.6657082,195.92423811)
\lineto(14.6657082,197.21423811)
\lineto(8.7707082,197.21423811)
\lineto(8.7707082,200.51423811)
\lineto(15.1607082,200.51423811)
\lineto(15.1607082,201.80423811)
\lineto(7.3157082,201.80423811)
\lineto(7.3157082,191.03423811)
\lineto(15.2657082,191.03423811)
\lineto(15.2657082,192.32423811)
}
}
{
\newrgbcolor{curcolor}{0 0 0}
\pscustom[linestyle=none,fillstyle=solid,fillcolor=curcolor]
{
\newpath
\moveto(22.36938007,196.64423811)
\curveto(22.36938007,197.03423772)(22.17437727,199.10423811)(19.36938007,199.10423811)
\curveto(17.82438162,199.10423811)(16.39938007,198.32423638)(16.39938007,196.59923811)
\curveto(16.39938007,195.51923919)(17.11938117,194.96423784)(18.21438007,194.69423811)
\lineto(19.74438007,194.31923811)
\curveto(20.86937895,194.03423839)(21.30438007,193.82423748)(21.30438007,193.19423811)
\curveto(21.30438007,192.32423898)(20.44937913,191.94923811)(19.50438007,191.94923811)
\curveto(17.64438193,191.94923811)(17.46438003,192.93923872)(17.41938007,193.55423811)
\lineto(16.14438007,193.55423811)
\curveto(16.18938003,192.60923905)(16.41438318,190.80923811)(19.51938007,190.80923811)
\curveto(21.2893783,190.80923811)(22.62438007,191.78423973)(22.62438007,193.40423811)
\curveto(22.62438007,194.46923704)(22.05437844,195.06923851)(20.41938007,195.47423811)
\lineto(19.09938007,195.80423811)
\curveto(18.07938109,196.05923785)(17.67438007,196.20923875)(17.67438007,196.85423811)
\curveto(17.67438007,197.82923713)(18.82938048,197.96423811)(19.23438007,197.96423811)
\curveto(20.89937841,197.96423811)(21.07938009,197.13923761)(21.09438007,196.64423811)
\lineto(22.36938007,196.64423811)
}
}
{
\newrgbcolor{curcolor}{0 0 0}
\pscustom[linestyle=none,fillstyle=solid,fillcolor=curcolor]
{
\newpath
\moveto(27.01938007,197.78423811)
\lineto(27.01938007,198.87923811)
\lineto(25.75938007,198.87923811)
\lineto(25.75938007,201.06923811)
\lineto(24.43938007,201.06923811)
\lineto(24.43938007,198.87923811)
\lineto(23.37438007,198.87923811)
\lineto(23.37438007,197.78423811)
\lineto(24.43938007,197.78423811)
\lineto(24.43938007,192.60923811)
\curveto(24.43938007,191.66423905)(24.72438138,190.92923811)(26.02938007,190.92923811)
\curveto(26.16437994,190.92923811)(26.53938055,190.98923815)(27.01938007,191.03423811)
\lineto(27.01938007,192.06923811)
\lineto(26.55438007,192.06923811)
\curveto(26.28438034,192.06923811)(25.75938007,192.06923872)(25.75938007,192.68423811)
\lineto(25.75938007,197.78423811)
\lineto(27.01938007,197.78423811)
}
}
{
\newrgbcolor{curcolor}{0 0 0}
\pscustom[linestyle=none,fillstyle=solid,fillcolor=curcolor]
{
\newpath
\moveto(34.65953632,191.03423811)
\lineto(34.65953632,198.87923811)
\lineto(33.33953632,198.87923811)
\lineto(33.33953632,194.55923811)
\curveto(33.33953632,193.41923925)(32.84453466,191.94923811)(31.17953632,191.94923811)
\curveto(30.32453718,191.94923811)(29.66453632,192.3842394)(29.66453632,193.67423811)
\lineto(29.66453632,198.87923811)
\lineto(28.34453632,198.87923811)
\lineto(28.34453632,193.23923811)
\curveto(28.34453632,191.36423998)(29.73953748,190.80923811)(30.89453632,190.80923811)
\curveto(32.15453506,190.80923811)(32.82953688,191.28923902)(33.38453632,192.20423811)
\lineto(33.41453632,192.17423811)
\lineto(33.41453632,191.03423811)
\lineto(34.65953632,191.03423811)
}
}
{
\newrgbcolor{curcolor}{0 0 0}
\pscustom[linestyle=none,fillstyle=solid,fillcolor=curcolor]
{
\newpath
\moveto(43.1591457,201.80423811)
\lineto(41.8391457,201.80423811)
\lineto(41.8391457,197.87423811)
\lineto(41.8091457,197.76923811)
\curveto(41.49414601,198.21923766)(40.89414427,199.10423811)(39.4691457,199.10423811)
\curveto(37.38414778,199.10423811)(36.1991457,197.3942359)(36.1991457,195.18923811)
\curveto(36.1991457,193.31423998)(36.97914837,190.80923811)(39.6491457,190.80923811)
\curveto(40.41414493,190.80923811)(41.31414627,191.04923917)(41.8841457,192.11423811)
\lineto(41.9141457,192.11423811)
\lineto(41.9141457,191.03423811)
\lineto(43.1591457,191.03423811)
\lineto(43.1591457,201.80423811)
\moveto(37.5641457,194.97923811)
\curveto(37.5641457,195.9842371)(37.66914774,197.91923811)(39.7091457,197.91923811)
\curveto(41.61414379,197.91923811)(41.8241457,195.86423683)(41.8241457,194.58923811)
\curveto(41.8241457,192.50424019)(40.51914486,191.94923811)(39.6791457,191.94923811)
\curveto(38.23914714,191.94923811)(37.5641457,193.25423983)(37.5641457,194.97923811)
}
}
{
\newrgbcolor{curcolor}{0 0 0}
\pscustom[linestyle=none,fillstyle=solid,fillcolor=curcolor]
{
\newpath
\moveto(46.34875507,198.87923811)
\lineto(45.02875507,198.87923811)
\lineto(45.02875507,191.03423811)
\lineto(46.34875507,191.03423811)
\lineto(46.34875507,198.87923811)
\moveto(46.34875507,200.30423811)
\lineto(46.34875507,201.80423811)
\lineto(45.02875507,201.80423811)
\lineto(45.02875507,200.30423811)
\lineto(46.34875507,200.30423811)
}
}
{
\newrgbcolor{curcolor}{0 0 0}
\pscustom[linestyle=none,fillstyle=solid,fillcolor=curcolor]
{
\newpath
\moveto(49.50859882,196.49423811)
\curveto(49.59859873,197.09423751)(49.80860032,198.00923811)(51.30859882,198.00923811)
\curveto(52.55359758,198.00923811)(53.15359882,197.55923728)(53.15359882,196.73423811)
\curveto(53.15359882,195.95423889)(52.77859851,195.83423808)(52.46359882,195.80423811)
\lineto(50.28859882,195.53423811)
\curveto(48.09860101,195.26423838)(47.90359882,193.73423745)(47.90359882,193.07423811)
\curveto(47.90359882,191.72423946)(48.92360026,190.80923811)(50.36359882,190.80923811)
\curveto(51.89359729,190.80923811)(52.68859933,191.52923866)(53.19859882,192.08423811)
\curveto(53.24359878,191.48423871)(53.42359999,190.88423811)(54.59359882,190.88423811)
\curveto(54.89359852,190.88423811)(55.08859905,190.97423817)(55.31359882,191.03423811)
\lineto(55.31359882,191.99423811)
\curveto(55.16359897,191.96423814)(54.9985987,191.93423811)(54.87859882,191.93423811)
\curveto(54.60859909,191.93423811)(54.44359882,192.06923844)(54.44359882,192.39923811)
\lineto(54.44359882,196.91423811)
\curveto(54.44359882,198.9242361)(52.16359819,199.10423811)(51.53359882,199.10423811)
\curveto(49.59860076,199.10423811)(48.35359876,198.36923623)(48.29359882,196.49423811)
\lineto(49.50859882,196.49423811)
\moveto(53.12359882,193.74923811)
\curveto(53.12359882,192.69923916)(51.92359759,191.90423811)(50.69359882,191.90423811)
\curveto(49.70359981,191.90423811)(49.26859882,192.41423896)(49.26859882,193.26923811)
\curveto(49.26859882,194.25923712)(50.30359947,194.4542382)(50.94859882,194.54423811)
\curveto(52.58359719,194.7542379)(52.91359903,194.87423827)(53.12359882,195.03923811)
\lineto(53.12359882,193.74923811)
}
}
{
\newrgbcolor{curcolor}{0 0 0}
\pscustom[linestyle=none,fillstyle=solid,fillcolor=curcolor]
{
\newpath
\moveto(63.0782082,196.37423811)
\curveto(63.0782082,198.60923587)(61.54820698,199.10423811)(60.3332082,199.10423811)
\curveto(58.98320955,199.10423811)(58.24820791,198.18923769)(57.9632082,197.76923811)
\lineto(57.9332082,197.76923811)
\lineto(57.9332082,198.87923811)
\lineto(56.6882082,198.87923811)
\lineto(56.6882082,191.03423811)
\lineto(58.0082082,191.03423811)
\lineto(58.0082082,195.30923811)
\curveto(58.0082082,197.43923598)(59.32820895,197.91923811)(60.0782082,197.91923811)
\curveto(61.36820691,197.91923811)(61.7582082,197.22923674)(61.7582082,195.86423811)
\lineto(61.7582082,191.03423811)
\lineto(63.0782082,191.03423811)
\lineto(63.0782082,196.37423811)
}
}
{
\newrgbcolor{curcolor}{0 0 0}
\pscustom[linestyle=none,fillstyle=solid,fillcolor=curcolor]
{
\newpath
\moveto(67.91781757,197.78423811)
\lineto(67.91781757,198.87923811)
\lineto(66.65781757,198.87923811)
\lineto(66.65781757,201.06923811)
\lineto(65.33781757,201.06923811)
\lineto(65.33781757,198.87923811)
\lineto(64.27281757,198.87923811)
\lineto(64.27281757,197.78423811)
\lineto(65.33781757,197.78423811)
\lineto(65.33781757,192.60923811)
\curveto(65.33781757,191.66423905)(65.62281888,190.92923811)(66.92781757,190.92923811)
\curveto(67.06281744,190.92923811)(67.43781805,190.98923815)(67.91781757,191.03423811)
\lineto(67.91781757,192.06923811)
\lineto(67.45281757,192.06923811)
\curveto(67.18281784,192.06923811)(66.65781757,192.06923872)(66.65781757,192.68423811)
\lineto(66.65781757,197.78423811)
\lineto(67.91781757,197.78423811)
}
}
{
\newrgbcolor{curcolor}{0 0 0}
\pscustom[linestyle=none,fillstyle=solid,fillcolor=curcolor]
{
\newpath
\moveto(74.24500507,193.49423811)
\curveto(74.20000512,192.90923869)(73.46500383,191.94923811)(72.22000507,191.94923811)
\curveto(70.70500659,191.94923811)(69.94000507,192.89423974)(69.94000507,194.52923811)
\lineto(75.67000507,194.52923811)
\curveto(75.67000507,197.30423533)(74.56000281,199.10423811)(72.29500507,199.10423811)
\curveto(69.70000767,199.10423811)(68.53000507,197.16923568)(68.53000507,194.73923811)
\curveto(68.53000507,192.47424037)(69.83500728,190.80923811)(72.04000507,190.80923811)
\curveto(73.30000381,190.80923811)(73.81000543,191.10923835)(74.17000507,191.34923811)
\curveto(75.16000408,192.00923745)(75.52000512,193.11923848)(75.56500507,193.49423811)
\lineto(74.24500507,193.49423811)
\moveto(69.94000507,195.57923811)
\curveto(69.94000507,196.79423689)(70.90000629,197.91923811)(72.11500507,197.91923811)
\curveto(73.72000347,197.91923811)(74.23000515,196.79423689)(74.30500507,195.57923811)
\lineto(69.94000507,195.57923811)
}
}
{
\newrgbcolor{curcolor}{0 0 0}
\pscustom[linestyle=none,fillstyle=solid,fillcolor=curcolor]
{
\newpath
\moveto(15.2657082,141.35353498)
\lineto(8.7707082,141.35353498)
\lineto(8.7707082,144.95353498)
\lineto(14.6657082,144.95353498)
\lineto(14.6657082,146.24353498)
\lineto(8.7707082,146.24353498)
\lineto(8.7707082,149.54353498)
\lineto(15.1607082,149.54353498)
\lineto(15.1607082,150.83353498)
\lineto(7.3157082,150.83353498)
\lineto(7.3157082,140.06353498)
\lineto(15.2657082,140.06353498)
\lineto(15.2657082,141.35353498)
}
}
{
\newrgbcolor{curcolor}{0 0 0}
\pscustom[linestyle=none,fillstyle=solid,fillcolor=curcolor]
{
\newpath
\moveto(22.36938007,145.67353498)
\curveto(22.36938007,146.06353459)(22.17437727,148.13353498)(19.36938007,148.13353498)
\curveto(17.82438162,148.13353498)(16.39938007,147.35353326)(16.39938007,145.62853498)
\curveto(16.39938007,144.54853606)(17.11938117,143.99353471)(18.21438007,143.72353498)
\lineto(19.74438007,143.34853498)
\curveto(20.86937895,143.06353527)(21.30438007,142.85353435)(21.30438007,142.22353498)
\curveto(21.30438007,141.35353585)(20.44937913,140.97853498)(19.50438007,140.97853498)
\curveto(17.64438193,140.97853498)(17.46438003,141.9685356)(17.41938007,142.58353498)
\lineto(16.14438007,142.58353498)
\curveto(16.18938003,141.63853593)(16.41438318,139.83853498)(19.51938007,139.83853498)
\curveto(21.2893783,139.83853498)(22.62438007,140.8135366)(22.62438007,142.43353498)
\curveto(22.62438007,143.49853392)(22.05437844,144.09853539)(20.41938007,144.50353498)
\lineto(19.09938007,144.83353498)
\curveto(18.07938109,145.08853473)(17.67438007,145.23853563)(17.67438007,145.88353498)
\curveto(17.67438007,146.85853401)(18.82938048,146.99353498)(19.23438007,146.99353498)
\curveto(20.89937841,146.99353498)(21.07938009,146.16853449)(21.09438007,145.67353498)
\lineto(22.36938007,145.67353498)
}
}
{
\newrgbcolor{curcolor}{0 0 0}
\pscustom[linestyle=none,fillstyle=solid,fillcolor=curcolor]
{
\newpath
\moveto(27.01938007,146.81353498)
\lineto(27.01938007,147.90853498)
\lineto(25.75938007,147.90853498)
\lineto(25.75938007,150.09853498)
\lineto(24.43938007,150.09853498)
\lineto(24.43938007,147.90853498)
\lineto(23.37438007,147.90853498)
\lineto(23.37438007,146.81353498)
\lineto(24.43938007,146.81353498)
\lineto(24.43938007,141.63853498)
\curveto(24.43938007,140.69353593)(24.72438138,139.95853498)(26.02938007,139.95853498)
\curveto(26.16437994,139.95853498)(26.53938055,140.01853503)(27.01938007,140.06353498)
\lineto(27.01938007,141.09853498)
\lineto(26.55438007,141.09853498)
\curveto(26.28438034,141.09853498)(25.75938007,141.0985356)(25.75938007,141.71353498)
\lineto(25.75938007,146.81353498)
\lineto(27.01938007,146.81353498)
}
}
{
\newrgbcolor{curcolor}{0 0 0}
\pscustom[linestyle=none,fillstyle=solid,fillcolor=curcolor]
{
\newpath
\moveto(34.65953632,140.06353498)
\lineto(34.65953632,147.90853498)
\lineto(33.33953632,147.90853498)
\lineto(33.33953632,143.58853498)
\curveto(33.33953632,142.44853612)(32.84453466,140.97853498)(31.17953632,140.97853498)
\curveto(30.32453718,140.97853498)(29.66453632,141.41353627)(29.66453632,142.70353498)
\lineto(29.66453632,147.90853498)
\lineto(28.34453632,147.90853498)
\lineto(28.34453632,142.26853498)
\curveto(28.34453632,140.39353686)(29.73953748,139.83853498)(30.89453632,139.83853498)
\curveto(32.15453506,139.83853498)(32.82953688,140.3185359)(33.38453632,141.23353498)
\lineto(33.41453632,141.20353498)
\lineto(33.41453632,140.06353498)
\lineto(34.65953632,140.06353498)
}
}
{
\newrgbcolor{curcolor}{0 0 0}
\pscustom[linestyle=none,fillstyle=solid,fillcolor=curcolor]
{
\newpath
\moveto(43.1591457,150.83353498)
\lineto(41.8391457,150.83353498)
\lineto(41.8391457,146.90353498)
\lineto(41.8091457,146.79853498)
\curveto(41.49414601,147.24853453)(40.89414427,148.13353498)(39.4691457,148.13353498)
\curveto(37.38414778,148.13353498)(36.1991457,146.42353278)(36.1991457,144.21853498)
\curveto(36.1991457,142.34353686)(36.97914837,139.83853498)(39.6491457,139.83853498)
\curveto(40.41414493,139.83853498)(41.31414627,140.07853605)(41.8841457,141.14353498)
\lineto(41.9141457,141.14353498)
\lineto(41.9141457,140.06353498)
\lineto(43.1591457,140.06353498)
\lineto(43.1591457,150.83353498)
\moveto(37.5641457,144.00853498)
\curveto(37.5641457,145.01353398)(37.66914774,146.94853498)(39.7091457,146.94853498)
\curveto(41.61414379,146.94853498)(41.8241457,144.89353371)(41.8241457,143.61853498)
\curveto(41.8241457,141.53353707)(40.51914486,140.97853498)(39.6791457,140.97853498)
\curveto(38.23914714,140.97853498)(37.5641457,142.28353671)(37.5641457,144.00853498)
}
}
{
\newrgbcolor{curcolor}{0 0 0}
\pscustom[linestyle=none,fillstyle=solid,fillcolor=curcolor]
{
\newpath
\moveto(46.34875507,147.90853498)
\lineto(45.02875507,147.90853498)
\lineto(45.02875507,140.06353498)
\lineto(46.34875507,140.06353498)
\lineto(46.34875507,147.90853498)
\moveto(46.34875507,149.33353498)
\lineto(46.34875507,150.83353498)
\lineto(45.02875507,150.83353498)
\lineto(45.02875507,149.33353498)
\lineto(46.34875507,149.33353498)
}
}
{
\newrgbcolor{curcolor}{0 0 0}
\pscustom[linestyle=none,fillstyle=solid,fillcolor=curcolor]
{
\newpath
\moveto(49.50859882,145.52353498)
\curveto(49.59859873,146.12353438)(49.80860032,147.03853498)(51.30859882,147.03853498)
\curveto(52.55359758,147.03853498)(53.15359882,146.58853416)(53.15359882,145.76353498)
\curveto(53.15359882,144.98353576)(52.77859851,144.86353495)(52.46359882,144.83353498)
\lineto(50.28859882,144.56353498)
\curveto(48.09860101,144.29353525)(47.90359882,142.76353432)(47.90359882,142.10353498)
\curveto(47.90359882,140.75353633)(48.92360026,139.83853498)(50.36359882,139.83853498)
\curveto(51.89359729,139.83853498)(52.68859933,140.55853554)(53.19859882,141.11353498)
\curveto(53.24359878,140.51353558)(53.42359999,139.91353498)(54.59359882,139.91353498)
\curveto(54.89359852,139.91353498)(55.08859905,140.00353504)(55.31359882,140.06353498)
\lineto(55.31359882,141.02353498)
\curveto(55.16359897,140.99353501)(54.9985987,140.96353498)(54.87859882,140.96353498)
\curveto(54.60859909,140.96353498)(54.44359882,141.09853531)(54.44359882,141.42853498)
\lineto(54.44359882,145.94353498)
\curveto(54.44359882,147.95353297)(52.16359819,148.13353498)(51.53359882,148.13353498)
\curveto(49.59860076,148.13353498)(48.35359876,147.39853311)(48.29359882,145.52353498)
\lineto(49.50859882,145.52353498)
\moveto(53.12359882,142.77853498)
\curveto(53.12359882,141.72853603)(51.92359759,140.93353498)(50.69359882,140.93353498)
\curveto(49.70359981,140.93353498)(49.26859882,141.44353584)(49.26859882,142.29853498)
\curveto(49.26859882,143.28853399)(50.30359947,143.48353507)(50.94859882,143.57353498)
\curveto(52.58359719,143.78353477)(52.91359903,143.90353515)(53.12359882,144.06853498)
\lineto(53.12359882,142.77853498)
}
}
{
\newrgbcolor{curcolor}{0 0 0}
\pscustom[linestyle=none,fillstyle=solid,fillcolor=curcolor]
{
\newpath
\moveto(63.0782082,145.40353498)
\curveto(63.0782082,147.63853275)(61.54820698,148.13353498)(60.3332082,148.13353498)
\curveto(58.98320955,148.13353498)(58.24820791,147.21853456)(57.9632082,146.79853498)
\lineto(57.9332082,146.79853498)
\lineto(57.9332082,147.90853498)
\lineto(56.6882082,147.90853498)
\lineto(56.6882082,140.06353498)
\lineto(58.0082082,140.06353498)
\lineto(58.0082082,144.33853498)
\curveto(58.0082082,146.46853285)(59.32820895,146.94853498)(60.0782082,146.94853498)
\curveto(61.36820691,146.94853498)(61.7582082,146.25853362)(61.7582082,144.89353498)
\lineto(61.7582082,140.06353498)
\lineto(63.0782082,140.06353498)
\lineto(63.0782082,145.40353498)
}
}
{
\newrgbcolor{curcolor}{0 0 0}
\pscustom[linestyle=none,fillstyle=solid,fillcolor=curcolor]
{
\newpath
\moveto(67.91781757,146.81353498)
\lineto(67.91781757,147.90853498)
\lineto(66.65781757,147.90853498)
\lineto(66.65781757,150.09853498)
\lineto(65.33781757,150.09853498)
\lineto(65.33781757,147.90853498)
\lineto(64.27281757,147.90853498)
\lineto(64.27281757,146.81353498)
\lineto(65.33781757,146.81353498)
\lineto(65.33781757,141.63853498)
\curveto(65.33781757,140.69353593)(65.62281888,139.95853498)(66.92781757,139.95853498)
\curveto(67.06281744,139.95853498)(67.43781805,140.01853503)(67.91781757,140.06353498)
\lineto(67.91781757,141.09853498)
\lineto(67.45281757,141.09853498)
\curveto(67.18281784,141.09853498)(66.65781757,141.0985356)(66.65781757,141.71353498)
\lineto(66.65781757,146.81353498)
\lineto(67.91781757,146.81353498)
}
}
{
\newrgbcolor{curcolor}{0 0 0}
\pscustom[linestyle=none,fillstyle=solid,fillcolor=curcolor]
{
\newpath
\moveto(74.24500507,142.52353498)
\curveto(74.20000512,141.93853557)(73.46500383,140.97853498)(72.22000507,140.97853498)
\curveto(70.70500659,140.97853498)(69.94000507,141.92353662)(69.94000507,143.55853498)
\lineto(75.67000507,143.55853498)
\curveto(75.67000507,146.33353221)(74.56000281,148.13353498)(72.29500507,148.13353498)
\curveto(69.70000767,148.13353498)(68.53000507,146.19853255)(68.53000507,143.76853498)
\curveto(68.53000507,141.50353725)(69.83500728,139.83853498)(72.04000507,139.83853498)
\curveto(73.30000381,139.83853498)(73.81000543,140.13853522)(74.17000507,140.37853498)
\curveto(75.16000408,141.03853432)(75.52000512,142.14853536)(75.56500507,142.52353498)
\lineto(74.24500507,142.52353498)
\moveto(69.94000507,144.60853498)
\curveto(69.94000507,145.82353377)(70.90000629,146.94853498)(72.11500507,146.94853498)
\curveto(73.72000347,146.94853498)(74.23000515,145.82353377)(74.30500507,144.60853498)
\lineto(69.94000507,144.60853498)
}
}
{
\newrgbcolor{curcolor}{0 0 0}
\pscustom[linestyle=none,fillstyle=solid,fillcolor=curcolor]
{
\newpath
\moveto(15.2657082,73.39259748)
\lineto(8.7707082,73.39259748)
\lineto(8.7707082,76.99259748)
\lineto(14.6657082,76.99259748)
\lineto(14.6657082,78.28259748)
\lineto(8.7707082,78.28259748)
\lineto(8.7707082,81.58259748)
\lineto(15.1607082,81.58259748)
\lineto(15.1607082,82.87259748)
\lineto(7.3157082,82.87259748)
\lineto(7.3157082,72.10259748)
\lineto(15.2657082,72.10259748)
\lineto(15.2657082,73.39259748)
}
}
{
\newrgbcolor{curcolor}{0 0 0}
\pscustom[linestyle=none,fillstyle=solid,fillcolor=curcolor]
{
\newpath
\moveto(22.36938007,77.71259748)
\curveto(22.36938007,78.10259709)(22.17437727,80.17259748)(19.36938007,80.17259748)
\curveto(17.82438162,80.17259748)(16.39938007,79.39259576)(16.39938007,77.66759748)
\curveto(16.39938007,76.58759856)(17.11938117,76.03259721)(18.21438007,75.76259748)
\lineto(19.74438007,75.38759748)
\curveto(20.86937895,75.10259777)(21.30438007,74.89259685)(21.30438007,74.26259748)
\curveto(21.30438007,73.39259835)(20.44937913,73.01759748)(19.50438007,73.01759748)
\curveto(17.64438193,73.01759748)(17.46438003,74.0075981)(17.41938007,74.62259748)
\lineto(16.14438007,74.62259748)
\curveto(16.18938003,73.67759843)(16.41438318,71.87759748)(19.51938007,71.87759748)
\curveto(21.2893783,71.87759748)(22.62438007,72.8525991)(22.62438007,74.47259748)
\curveto(22.62438007,75.53759642)(22.05437844,76.13759789)(20.41938007,76.54259748)
\lineto(19.09938007,76.87259748)
\curveto(18.07938109,77.12759723)(17.67438007,77.27759813)(17.67438007,77.92259748)
\curveto(17.67438007,78.89759651)(18.82938048,79.03259748)(19.23438007,79.03259748)
\curveto(20.89937841,79.03259748)(21.07938009,78.20759699)(21.09438007,77.71259748)
\lineto(22.36938007,77.71259748)
}
}
{
\newrgbcolor{curcolor}{0 0 0}
\pscustom[linestyle=none,fillstyle=solid,fillcolor=curcolor]
{
\newpath
\moveto(27.01938007,78.85259748)
\lineto(27.01938007,79.94759748)
\lineto(25.75938007,79.94759748)
\lineto(25.75938007,82.13759748)
\lineto(24.43938007,82.13759748)
\lineto(24.43938007,79.94759748)
\lineto(23.37438007,79.94759748)
\lineto(23.37438007,78.85259748)
\lineto(24.43938007,78.85259748)
\lineto(24.43938007,73.67759748)
\curveto(24.43938007,72.73259843)(24.72438138,71.99759748)(26.02938007,71.99759748)
\curveto(26.16437994,71.99759748)(26.53938055,72.05759753)(27.01938007,72.10259748)
\lineto(27.01938007,73.13759748)
\lineto(26.55438007,73.13759748)
\curveto(26.28438034,73.13759748)(25.75938007,73.1375981)(25.75938007,73.75259748)
\lineto(25.75938007,78.85259748)
\lineto(27.01938007,78.85259748)
}
}
{
\newrgbcolor{curcolor}{0 0 0}
\pscustom[linestyle=none,fillstyle=solid,fillcolor=curcolor]
{
\newpath
\moveto(34.65953632,72.10259748)
\lineto(34.65953632,79.94759748)
\lineto(33.33953632,79.94759748)
\lineto(33.33953632,75.62759748)
\curveto(33.33953632,74.48759862)(32.84453466,73.01759748)(31.17953632,73.01759748)
\curveto(30.32453718,73.01759748)(29.66453632,73.45259877)(29.66453632,74.74259748)
\lineto(29.66453632,79.94759748)
\lineto(28.34453632,79.94759748)
\lineto(28.34453632,74.30759748)
\curveto(28.34453632,72.43259936)(29.73953748,71.87759748)(30.89453632,71.87759748)
\curveto(32.15453506,71.87759748)(32.82953688,72.3575984)(33.38453632,73.27259748)
\lineto(33.41453632,73.24259748)
\lineto(33.41453632,72.10259748)
\lineto(34.65953632,72.10259748)
}
}
{
\newrgbcolor{curcolor}{0 0 0}
\pscustom[linestyle=none,fillstyle=solid,fillcolor=curcolor]
{
\newpath
\moveto(43.1591457,82.87259748)
\lineto(41.8391457,82.87259748)
\lineto(41.8391457,78.94259748)
\lineto(41.8091457,78.83759748)
\curveto(41.49414601,79.28759703)(40.89414427,80.17259748)(39.4691457,80.17259748)
\curveto(37.38414778,80.17259748)(36.1991457,78.46259528)(36.1991457,76.25759748)
\curveto(36.1991457,74.38259936)(36.97914837,71.87759748)(39.6491457,71.87759748)
\curveto(40.41414493,71.87759748)(41.31414627,72.11759855)(41.8841457,73.18259748)
\lineto(41.9141457,73.18259748)
\lineto(41.9141457,72.10259748)
\lineto(43.1591457,72.10259748)
\lineto(43.1591457,82.87259748)
\moveto(37.5641457,76.04759748)
\curveto(37.5641457,77.05259648)(37.66914774,78.98759748)(39.7091457,78.98759748)
\curveto(41.61414379,78.98759748)(41.8241457,76.93259621)(41.8241457,75.65759748)
\curveto(41.8241457,73.57259957)(40.51914486,73.01759748)(39.6791457,73.01759748)
\curveto(38.23914714,73.01759748)(37.5641457,74.32259921)(37.5641457,76.04759748)
}
}
{
\newrgbcolor{curcolor}{0 0 0}
\pscustom[linestyle=none,fillstyle=solid,fillcolor=curcolor]
{
\newpath
\moveto(46.34875507,79.94759748)
\lineto(45.02875507,79.94759748)
\lineto(45.02875507,72.10259748)
\lineto(46.34875507,72.10259748)
\lineto(46.34875507,79.94759748)
\moveto(46.34875507,81.37259748)
\lineto(46.34875507,82.87259748)
\lineto(45.02875507,82.87259748)
\lineto(45.02875507,81.37259748)
\lineto(46.34875507,81.37259748)
}
}
{
\newrgbcolor{curcolor}{0 0 0}
\pscustom[linestyle=none,fillstyle=solid,fillcolor=curcolor]
{
\newpath
\moveto(49.50859882,77.56259748)
\curveto(49.59859873,78.16259688)(49.80860032,79.07759748)(51.30859882,79.07759748)
\curveto(52.55359758,79.07759748)(53.15359882,78.62759666)(53.15359882,77.80259748)
\curveto(53.15359882,77.02259826)(52.77859851,76.90259745)(52.46359882,76.87259748)
\lineto(50.28859882,76.60259748)
\curveto(48.09860101,76.33259775)(47.90359882,74.80259682)(47.90359882,74.14259748)
\curveto(47.90359882,72.79259883)(48.92360026,71.87759748)(50.36359882,71.87759748)
\curveto(51.89359729,71.87759748)(52.68859933,72.59759804)(53.19859882,73.15259748)
\curveto(53.24359878,72.55259808)(53.42359999,71.95259748)(54.59359882,71.95259748)
\curveto(54.89359852,71.95259748)(55.08859905,72.04259754)(55.31359882,72.10259748)
\lineto(55.31359882,73.06259748)
\curveto(55.16359897,73.03259751)(54.9985987,73.00259748)(54.87859882,73.00259748)
\curveto(54.60859909,73.00259748)(54.44359882,73.13759781)(54.44359882,73.46759748)
\lineto(54.44359882,77.98259748)
\curveto(54.44359882,79.99259547)(52.16359819,80.17259748)(51.53359882,80.17259748)
\curveto(49.59860076,80.17259748)(48.35359876,79.43759561)(48.29359882,77.56259748)
\lineto(49.50859882,77.56259748)
\moveto(53.12359882,74.81759748)
\curveto(53.12359882,73.76759853)(51.92359759,72.97259748)(50.69359882,72.97259748)
\curveto(49.70359981,72.97259748)(49.26859882,73.48259834)(49.26859882,74.33759748)
\curveto(49.26859882,75.32759649)(50.30359947,75.52259757)(50.94859882,75.61259748)
\curveto(52.58359719,75.82259727)(52.91359903,75.94259765)(53.12359882,76.10759748)
\lineto(53.12359882,74.81759748)
}
}
{
\newrgbcolor{curcolor}{0 0 0}
\pscustom[linestyle=none,fillstyle=solid,fillcolor=curcolor]
{
\newpath
\moveto(63.0782082,77.44259748)
\curveto(63.0782082,79.67759525)(61.54820698,80.17259748)(60.3332082,80.17259748)
\curveto(58.98320955,80.17259748)(58.24820791,79.25759706)(57.9632082,78.83759748)
\lineto(57.9332082,78.83759748)
\lineto(57.9332082,79.94759748)
\lineto(56.6882082,79.94759748)
\lineto(56.6882082,72.10259748)
\lineto(58.0082082,72.10259748)
\lineto(58.0082082,76.37759748)
\curveto(58.0082082,78.50759535)(59.32820895,78.98759748)(60.0782082,78.98759748)
\curveto(61.36820691,78.98759748)(61.7582082,78.29759612)(61.7582082,76.93259748)
\lineto(61.7582082,72.10259748)
\lineto(63.0782082,72.10259748)
\lineto(63.0782082,77.44259748)
}
}
{
\newrgbcolor{curcolor}{0 0 0}
\pscustom[linestyle=none,fillstyle=solid,fillcolor=curcolor]
{
\newpath
\moveto(67.91781757,78.85259748)
\lineto(67.91781757,79.94759748)
\lineto(66.65781757,79.94759748)
\lineto(66.65781757,82.13759748)
\lineto(65.33781757,82.13759748)
\lineto(65.33781757,79.94759748)
\lineto(64.27281757,79.94759748)
\lineto(64.27281757,78.85259748)
\lineto(65.33781757,78.85259748)
\lineto(65.33781757,73.67759748)
\curveto(65.33781757,72.73259843)(65.62281888,71.99759748)(66.92781757,71.99759748)
\curveto(67.06281744,71.99759748)(67.43781805,72.05759753)(67.91781757,72.10259748)
\lineto(67.91781757,73.13759748)
\lineto(67.45281757,73.13759748)
\curveto(67.18281784,73.13759748)(66.65781757,73.1375981)(66.65781757,73.75259748)
\lineto(66.65781757,78.85259748)
\lineto(67.91781757,78.85259748)
}
}
{
\newrgbcolor{curcolor}{0 0 0}
\pscustom[linestyle=none,fillstyle=solid,fillcolor=curcolor]
{
\newpath
\moveto(74.24500507,74.56259748)
\curveto(74.20000512,73.97759807)(73.46500383,73.01759748)(72.22000507,73.01759748)
\curveto(70.70500659,73.01759748)(69.94000507,73.96259912)(69.94000507,75.59759748)
\lineto(75.67000507,75.59759748)
\curveto(75.67000507,78.37259471)(74.56000281,80.17259748)(72.29500507,80.17259748)
\curveto(69.70000767,80.17259748)(68.53000507,78.23759505)(68.53000507,75.80759748)
\curveto(68.53000507,73.54259975)(69.83500728,71.87759748)(72.04000507,71.87759748)
\curveto(73.30000381,71.87759748)(73.81000543,72.17759772)(74.17000507,72.41759748)
\curveto(75.16000408,73.07759682)(75.52000512,74.18759786)(75.56500507,74.56259748)
\lineto(74.24500507,74.56259748)
\moveto(69.94000507,76.64759748)
\curveto(69.94000507,77.86259627)(70.90000629,78.98759748)(72.11500507,78.98759748)
\curveto(73.72000347,78.98759748)(74.23000515,77.86259627)(74.30500507,76.64759748)
\lineto(69.94000507,76.64759748)
}
}
{
\newrgbcolor{curcolor}{0 0 0}
\pscustom[linestyle=none,fillstyle=solid,fillcolor=curcolor]
{
\newpath
\moveto(15.2657082,56.40236311)
\lineto(8.7707082,56.40236311)
\lineto(8.7707082,60.00236311)
\lineto(14.6657082,60.00236311)
\lineto(14.6657082,61.29236311)
\lineto(8.7707082,61.29236311)
\lineto(8.7707082,64.59236311)
\lineto(15.1607082,64.59236311)
\lineto(15.1607082,65.88236311)
\lineto(7.3157082,65.88236311)
\lineto(7.3157082,55.11236311)
\lineto(15.2657082,55.11236311)
\lineto(15.2657082,56.40236311)
}
}
{
\newrgbcolor{curcolor}{0 0 0}
\pscustom[linestyle=none,fillstyle=solid,fillcolor=curcolor]
{
\newpath
\moveto(22.36938007,60.72236311)
\curveto(22.36938007,61.11236272)(22.17437727,63.18236311)(19.36938007,63.18236311)
\curveto(17.82438162,63.18236311)(16.39938007,62.40236138)(16.39938007,60.67736311)
\curveto(16.39938007,59.59736419)(17.11938117,59.04236284)(18.21438007,58.77236311)
\lineto(19.74438007,58.39736311)
\curveto(20.86937895,58.11236339)(21.30438007,57.90236248)(21.30438007,57.27236311)
\curveto(21.30438007,56.40236398)(20.44937913,56.02736311)(19.50438007,56.02736311)
\curveto(17.64438193,56.02736311)(17.46438003,57.01736372)(17.41938007,57.63236311)
\lineto(16.14438007,57.63236311)
\curveto(16.18938003,56.68736405)(16.41438318,54.88736311)(19.51938007,54.88736311)
\curveto(21.2893783,54.88736311)(22.62438007,55.86236473)(22.62438007,57.48236311)
\curveto(22.62438007,58.54736204)(22.05437844,59.14736351)(20.41938007,59.55236311)
\lineto(19.09938007,59.88236311)
\curveto(18.07938109,60.13736285)(17.67438007,60.28736375)(17.67438007,60.93236311)
\curveto(17.67438007,61.90736213)(18.82938048,62.04236311)(19.23438007,62.04236311)
\curveto(20.89937841,62.04236311)(21.07938009,61.21736261)(21.09438007,60.72236311)
\lineto(22.36938007,60.72236311)
}
}
{
\newrgbcolor{curcolor}{0 0 0}
\pscustom[linestyle=none,fillstyle=solid,fillcolor=curcolor]
{
\newpath
\moveto(27.01938007,61.86236311)
\lineto(27.01938007,62.95736311)
\lineto(25.75938007,62.95736311)
\lineto(25.75938007,65.14736311)
\lineto(24.43938007,65.14736311)
\lineto(24.43938007,62.95736311)
\lineto(23.37438007,62.95736311)
\lineto(23.37438007,61.86236311)
\lineto(24.43938007,61.86236311)
\lineto(24.43938007,56.68736311)
\curveto(24.43938007,55.74236405)(24.72438138,55.00736311)(26.02938007,55.00736311)
\curveto(26.16437994,55.00736311)(26.53938055,55.06736315)(27.01938007,55.11236311)
\lineto(27.01938007,56.14736311)
\lineto(26.55438007,56.14736311)
\curveto(26.28438034,56.14736311)(25.75938007,56.14736372)(25.75938007,56.76236311)
\lineto(25.75938007,61.86236311)
\lineto(27.01938007,61.86236311)
}
}
{
\newrgbcolor{curcolor}{0 0 0}
\pscustom[linestyle=none,fillstyle=solid,fillcolor=curcolor]
{
\newpath
\moveto(34.65953632,55.11236311)
\lineto(34.65953632,62.95736311)
\lineto(33.33953632,62.95736311)
\lineto(33.33953632,58.63736311)
\curveto(33.33953632,57.49736425)(32.84453466,56.02736311)(31.17953632,56.02736311)
\curveto(30.32453718,56.02736311)(29.66453632,56.4623644)(29.66453632,57.75236311)
\lineto(29.66453632,62.95736311)
\lineto(28.34453632,62.95736311)
\lineto(28.34453632,57.31736311)
\curveto(28.34453632,55.44236498)(29.73953748,54.88736311)(30.89453632,54.88736311)
\curveto(32.15453506,54.88736311)(32.82953688,55.36736402)(33.38453632,56.28236311)
\lineto(33.41453632,56.25236311)
\lineto(33.41453632,55.11236311)
\lineto(34.65953632,55.11236311)
}
}
{
\newrgbcolor{curcolor}{0 0 0}
\pscustom[linestyle=none,fillstyle=solid,fillcolor=curcolor]
{
\newpath
\moveto(43.1591457,65.88236311)
\lineto(41.8391457,65.88236311)
\lineto(41.8391457,61.95236311)
\lineto(41.8091457,61.84736311)
\curveto(41.49414601,62.29736266)(40.89414427,63.18236311)(39.4691457,63.18236311)
\curveto(37.38414778,63.18236311)(36.1991457,61.4723609)(36.1991457,59.26736311)
\curveto(36.1991457,57.39236498)(36.97914837,54.88736311)(39.6491457,54.88736311)
\curveto(40.41414493,54.88736311)(41.31414627,55.12736417)(41.8841457,56.19236311)
\lineto(41.9141457,56.19236311)
\lineto(41.9141457,55.11236311)
\lineto(43.1591457,55.11236311)
\lineto(43.1591457,65.88236311)
\moveto(37.5641457,59.05736311)
\curveto(37.5641457,60.0623621)(37.66914774,61.99736311)(39.7091457,61.99736311)
\curveto(41.61414379,61.99736311)(41.8241457,59.94236183)(41.8241457,58.66736311)
\curveto(41.8241457,56.58236519)(40.51914486,56.02736311)(39.6791457,56.02736311)
\curveto(38.23914714,56.02736311)(37.5641457,57.33236483)(37.5641457,59.05736311)
}
}
{
\newrgbcolor{curcolor}{0 0 0}
\pscustom[linestyle=none,fillstyle=solid,fillcolor=curcolor]
{
\newpath
\moveto(46.34875507,62.95736311)
\lineto(45.02875507,62.95736311)
\lineto(45.02875507,55.11236311)
\lineto(46.34875507,55.11236311)
\lineto(46.34875507,62.95736311)
\moveto(46.34875507,64.38236311)
\lineto(46.34875507,65.88236311)
\lineto(45.02875507,65.88236311)
\lineto(45.02875507,64.38236311)
\lineto(46.34875507,64.38236311)
}
}
{
\newrgbcolor{curcolor}{0 0 0}
\pscustom[linestyle=none,fillstyle=solid,fillcolor=curcolor]
{
\newpath
\moveto(49.50859882,60.57236311)
\curveto(49.59859873,61.17236251)(49.80860032,62.08736311)(51.30859882,62.08736311)
\curveto(52.55359758,62.08736311)(53.15359882,61.63736228)(53.15359882,60.81236311)
\curveto(53.15359882,60.03236389)(52.77859851,59.91236308)(52.46359882,59.88236311)
\lineto(50.28859882,59.61236311)
\curveto(48.09860101,59.34236338)(47.90359882,57.81236245)(47.90359882,57.15236311)
\curveto(47.90359882,55.80236446)(48.92360026,54.88736311)(50.36359882,54.88736311)
\curveto(51.89359729,54.88736311)(52.68859933,55.60736366)(53.19859882,56.16236311)
\curveto(53.24359878,55.56236371)(53.42359999,54.96236311)(54.59359882,54.96236311)
\curveto(54.89359852,54.96236311)(55.08859905,55.05236317)(55.31359882,55.11236311)
\lineto(55.31359882,56.07236311)
\curveto(55.16359897,56.04236314)(54.9985987,56.01236311)(54.87859882,56.01236311)
\curveto(54.60859909,56.01236311)(54.44359882,56.14736344)(54.44359882,56.47736311)
\lineto(54.44359882,60.99236311)
\curveto(54.44359882,63.0023611)(52.16359819,63.18236311)(51.53359882,63.18236311)
\curveto(49.59860076,63.18236311)(48.35359876,62.44736123)(48.29359882,60.57236311)
\lineto(49.50859882,60.57236311)
\moveto(53.12359882,57.82736311)
\curveto(53.12359882,56.77736416)(51.92359759,55.98236311)(50.69359882,55.98236311)
\curveto(49.70359981,55.98236311)(49.26859882,56.49236396)(49.26859882,57.34736311)
\curveto(49.26859882,58.33736212)(50.30359947,58.5323632)(50.94859882,58.62236311)
\curveto(52.58359719,58.8323629)(52.91359903,58.95236327)(53.12359882,59.11736311)
\lineto(53.12359882,57.82736311)
}
}
{
\newrgbcolor{curcolor}{0 0 0}
\pscustom[linestyle=none,fillstyle=solid,fillcolor=curcolor]
{
\newpath
\moveto(63.0782082,60.45236311)
\curveto(63.0782082,62.68736087)(61.54820698,63.18236311)(60.3332082,63.18236311)
\curveto(58.98320955,63.18236311)(58.24820791,62.26736269)(57.9632082,61.84736311)
\lineto(57.9332082,61.84736311)
\lineto(57.9332082,62.95736311)
\lineto(56.6882082,62.95736311)
\lineto(56.6882082,55.11236311)
\lineto(58.0082082,55.11236311)
\lineto(58.0082082,59.38736311)
\curveto(58.0082082,61.51736098)(59.32820895,61.99736311)(60.0782082,61.99736311)
\curveto(61.36820691,61.99736311)(61.7582082,61.30736174)(61.7582082,59.94236311)
\lineto(61.7582082,55.11236311)
\lineto(63.0782082,55.11236311)
\lineto(63.0782082,60.45236311)
}
}
{
\newrgbcolor{curcolor}{0 0 0}
\pscustom[linestyle=none,fillstyle=solid,fillcolor=curcolor]
{
\newpath
\moveto(67.91781757,61.86236311)
\lineto(67.91781757,62.95736311)
\lineto(66.65781757,62.95736311)
\lineto(66.65781757,65.14736311)
\lineto(65.33781757,65.14736311)
\lineto(65.33781757,62.95736311)
\lineto(64.27281757,62.95736311)
\lineto(64.27281757,61.86236311)
\lineto(65.33781757,61.86236311)
\lineto(65.33781757,56.68736311)
\curveto(65.33781757,55.74236405)(65.62281888,55.00736311)(66.92781757,55.00736311)
\curveto(67.06281744,55.00736311)(67.43781805,55.06736315)(67.91781757,55.11236311)
\lineto(67.91781757,56.14736311)
\lineto(67.45281757,56.14736311)
\curveto(67.18281784,56.14736311)(66.65781757,56.14736372)(66.65781757,56.76236311)
\lineto(66.65781757,61.86236311)
\lineto(67.91781757,61.86236311)
}
}
{
\newrgbcolor{curcolor}{0 0 0}
\pscustom[linestyle=none,fillstyle=solid,fillcolor=curcolor]
{
\newpath
\moveto(74.24500507,57.57236311)
\curveto(74.20000512,56.98736369)(73.46500383,56.02736311)(72.22000507,56.02736311)
\curveto(70.70500659,56.02736311)(69.94000507,56.97236474)(69.94000507,58.60736311)
\lineto(75.67000507,58.60736311)
\curveto(75.67000507,61.38236033)(74.56000281,63.18236311)(72.29500507,63.18236311)
\curveto(69.70000767,63.18236311)(68.53000507,61.24736068)(68.53000507,58.81736311)
\curveto(68.53000507,56.55236537)(69.83500728,54.88736311)(72.04000507,54.88736311)
\curveto(73.30000381,54.88736311)(73.81000543,55.18736335)(74.17000507,55.42736311)
\curveto(75.16000408,56.08736245)(75.52000512,57.19736348)(75.56500507,57.57236311)
\lineto(74.24500507,57.57236311)
\moveto(69.94000507,59.65736311)
\curveto(69.94000507,60.87236189)(70.90000629,61.99736311)(72.11500507,61.99736311)
\curveto(73.72000347,61.99736311)(74.23000515,60.87236189)(74.30500507,59.65736311)
\lineto(69.94000507,59.65736311)
}
}
{
\newrgbcolor{curcolor}{0 0 0}
\pscustom[linestyle=none,fillstyle=solid,fillcolor=curcolor]
{
\newpath
\moveto(15.2657082,39.4120677)
\lineto(8.7707082,39.4120677)
\lineto(8.7707082,43.0120677)
\lineto(14.6657082,43.0120677)
\lineto(14.6657082,44.3020677)
\lineto(8.7707082,44.3020677)
\lineto(8.7707082,47.6020677)
\lineto(15.1607082,47.6020677)
\lineto(15.1607082,48.8920677)
\lineto(7.3157082,48.8920677)
\lineto(7.3157082,38.1220677)
\lineto(15.2657082,38.1220677)
\lineto(15.2657082,39.4120677)
}
}
{
\newrgbcolor{curcolor}{0 0 0}
\pscustom[linestyle=none,fillstyle=solid,fillcolor=curcolor]
{
\newpath
\moveto(22.36938007,43.7320677)
\curveto(22.36938007,44.12206731)(22.17437727,46.1920677)(19.36938007,46.1920677)
\curveto(17.82438162,46.1920677)(16.39938007,45.41206597)(16.39938007,43.6870677)
\curveto(16.39938007,42.60706878)(17.11938117,42.05206743)(18.21438007,41.7820677)
\lineto(19.74438007,41.4070677)
\curveto(20.86937895,41.12206798)(21.30438007,40.91206707)(21.30438007,40.2820677)
\curveto(21.30438007,39.41206857)(20.44937913,39.0370677)(19.50438007,39.0370677)
\curveto(17.64438193,39.0370677)(17.46438003,40.02706831)(17.41938007,40.6420677)
\lineto(16.14438007,40.6420677)
\curveto(16.18938003,39.69706864)(16.41438318,37.8970677)(19.51938007,37.8970677)
\curveto(21.2893783,37.8970677)(22.62438007,38.87206932)(22.62438007,40.4920677)
\curveto(22.62438007,41.55706663)(22.05437844,42.1570681)(20.41938007,42.5620677)
\lineto(19.09938007,42.8920677)
\curveto(18.07938109,43.14706744)(17.67438007,43.29706834)(17.67438007,43.9420677)
\curveto(17.67438007,44.91706672)(18.82938048,45.0520677)(19.23438007,45.0520677)
\curveto(20.89937841,45.0520677)(21.07938009,44.2270672)(21.09438007,43.7320677)
\lineto(22.36938007,43.7320677)
}
}
{
\newrgbcolor{curcolor}{0 0 0}
\pscustom[linestyle=none,fillstyle=solid,fillcolor=curcolor]
{
\newpath
\moveto(27.01938007,44.8720677)
\lineto(27.01938007,45.9670677)
\lineto(25.75938007,45.9670677)
\lineto(25.75938007,48.1570677)
\lineto(24.43938007,48.1570677)
\lineto(24.43938007,45.9670677)
\lineto(23.37438007,45.9670677)
\lineto(23.37438007,44.8720677)
\lineto(24.43938007,44.8720677)
\lineto(24.43938007,39.6970677)
\curveto(24.43938007,38.75206864)(24.72438138,38.0170677)(26.02938007,38.0170677)
\curveto(26.16437994,38.0170677)(26.53938055,38.07706774)(27.01938007,38.1220677)
\lineto(27.01938007,39.1570677)
\lineto(26.55438007,39.1570677)
\curveto(26.28438034,39.1570677)(25.75938007,39.15706831)(25.75938007,39.7720677)
\lineto(25.75938007,44.8720677)
\lineto(27.01938007,44.8720677)
}
}
{
\newrgbcolor{curcolor}{0 0 0}
\pscustom[linestyle=none,fillstyle=solid,fillcolor=curcolor]
{
\newpath
\moveto(34.65953632,38.1220677)
\lineto(34.65953632,45.9670677)
\lineto(33.33953632,45.9670677)
\lineto(33.33953632,41.6470677)
\curveto(33.33953632,40.50706884)(32.84453466,39.0370677)(31.17953632,39.0370677)
\curveto(30.32453718,39.0370677)(29.66453632,39.47206899)(29.66453632,40.7620677)
\lineto(29.66453632,45.9670677)
\lineto(28.34453632,45.9670677)
\lineto(28.34453632,40.3270677)
\curveto(28.34453632,38.45206957)(29.73953748,37.8970677)(30.89453632,37.8970677)
\curveto(32.15453506,37.8970677)(32.82953688,38.37706861)(33.38453632,39.2920677)
\lineto(33.41453632,39.2620677)
\lineto(33.41453632,38.1220677)
\lineto(34.65953632,38.1220677)
}
}
{
\newrgbcolor{curcolor}{0 0 0}
\pscustom[linestyle=none,fillstyle=solid,fillcolor=curcolor]
{
\newpath
\moveto(43.1591457,48.8920677)
\lineto(41.8391457,48.8920677)
\lineto(41.8391457,44.9620677)
\lineto(41.8091457,44.8570677)
\curveto(41.49414601,45.30706725)(40.89414427,46.1920677)(39.4691457,46.1920677)
\curveto(37.38414778,46.1920677)(36.1991457,44.48206549)(36.1991457,42.2770677)
\curveto(36.1991457,40.40206957)(36.97914837,37.8970677)(39.6491457,37.8970677)
\curveto(40.41414493,37.8970677)(41.31414627,38.13706876)(41.8841457,39.2020677)
\lineto(41.9141457,39.2020677)
\lineto(41.9141457,38.1220677)
\lineto(43.1591457,38.1220677)
\lineto(43.1591457,48.8920677)
\moveto(37.5641457,42.0670677)
\curveto(37.5641457,43.07206669)(37.66914774,45.0070677)(39.7091457,45.0070677)
\curveto(41.61414379,45.0070677)(41.8241457,42.95206642)(41.8241457,41.6770677)
\curveto(41.8241457,39.59206978)(40.51914486,39.0370677)(39.6791457,39.0370677)
\curveto(38.23914714,39.0370677)(37.5641457,40.34206942)(37.5641457,42.0670677)
}
}
{
\newrgbcolor{curcolor}{0 0 0}
\pscustom[linestyle=none,fillstyle=solid,fillcolor=curcolor]
{
\newpath
\moveto(46.34875507,45.9670677)
\lineto(45.02875507,45.9670677)
\lineto(45.02875507,38.1220677)
\lineto(46.34875507,38.1220677)
\lineto(46.34875507,45.9670677)
\moveto(46.34875507,47.3920677)
\lineto(46.34875507,48.8920677)
\lineto(45.02875507,48.8920677)
\lineto(45.02875507,47.3920677)
\lineto(46.34875507,47.3920677)
}
}
{
\newrgbcolor{curcolor}{0 0 0}
\pscustom[linestyle=none,fillstyle=solid,fillcolor=curcolor]
{
\newpath
\moveto(49.50859882,43.5820677)
\curveto(49.59859873,44.1820671)(49.80860032,45.0970677)(51.30859882,45.0970677)
\curveto(52.55359758,45.0970677)(53.15359882,44.64706687)(53.15359882,43.8220677)
\curveto(53.15359882,43.04206848)(52.77859851,42.92206767)(52.46359882,42.8920677)
\lineto(50.28859882,42.6220677)
\curveto(48.09860101,42.35206797)(47.90359882,40.82206704)(47.90359882,40.1620677)
\curveto(47.90359882,38.81206905)(48.92360026,37.8970677)(50.36359882,37.8970677)
\curveto(51.89359729,37.8970677)(52.68859933,38.61706825)(53.19859882,39.1720677)
\curveto(53.24359878,38.5720683)(53.42359999,37.9720677)(54.59359882,37.9720677)
\curveto(54.89359852,37.9720677)(55.08859905,38.06206776)(55.31359882,38.1220677)
\lineto(55.31359882,39.0820677)
\curveto(55.16359897,39.05206773)(54.9985987,39.0220677)(54.87859882,39.0220677)
\curveto(54.60859909,39.0220677)(54.44359882,39.15706803)(54.44359882,39.4870677)
\lineto(54.44359882,44.0020677)
\curveto(54.44359882,46.01206569)(52.16359819,46.1920677)(51.53359882,46.1920677)
\curveto(49.59860076,46.1920677)(48.35359876,45.45706582)(48.29359882,43.5820677)
\lineto(49.50859882,43.5820677)
\moveto(53.12359882,40.8370677)
\curveto(53.12359882,39.78706875)(51.92359759,38.9920677)(50.69359882,38.9920677)
\curveto(49.70359981,38.9920677)(49.26859882,39.50206855)(49.26859882,40.3570677)
\curveto(49.26859882,41.34706671)(50.30359947,41.54206779)(50.94859882,41.6320677)
\curveto(52.58359719,41.84206749)(52.91359903,41.96206786)(53.12359882,42.1270677)
\lineto(53.12359882,40.8370677)
}
}
{
\newrgbcolor{curcolor}{0 0 0}
\pscustom[linestyle=none,fillstyle=solid,fillcolor=curcolor]
{
\newpath
\moveto(63.0782082,43.4620677)
\curveto(63.0782082,45.69706546)(61.54820698,46.1920677)(60.3332082,46.1920677)
\curveto(58.98320955,46.1920677)(58.24820791,45.27706728)(57.9632082,44.8570677)
\lineto(57.9332082,44.8570677)
\lineto(57.9332082,45.9670677)
\lineto(56.6882082,45.9670677)
\lineto(56.6882082,38.1220677)
\lineto(58.0082082,38.1220677)
\lineto(58.0082082,42.3970677)
\curveto(58.0082082,44.52706557)(59.32820895,45.0070677)(60.0782082,45.0070677)
\curveto(61.36820691,45.0070677)(61.7582082,44.31706633)(61.7582082,42.9520677)
\lineto(61.7582082,38.1220677)
\lineto(63.0782082,38.1220677)
\lineto(63.0782082,43.4620677)
}
}
{
\newrgbcolor{curcolor}{0 0 0}
\pscustom[linestyle=none,fillstyle=solid,fillcolor=curcolor]
{
\newpath
\moveto(67.91781757,44.8720677)
\lineto(67.91781757,45.9670677)
\lineto(66.65781757,45.9670677)
\lineto(66.65781757,48.1570677)
\lineto(65.33781757,48.1570677)
\lineto(65.33781757,45.9670677)
\lineto(64.27281757,45.9670677)
\lineto(64.27281757,44.8720677)
\lineto(65.33781757,44.8720677)
\lineto(65.33781757,39.6970677)
\curveto(65.33781757,38.75206864)(65.62281888,38.0170677)(66.92781757,38.0170677)
\curveto(67.06281744,38.0170677)(67.43781805,38.07706774)(67.91781757,38.1220677)
\lineto(67.91781757,39.1570677)
\lineto(67.45281757,39.1570677)
\curveto(67.18281784,39.1570677)(66.65781757,39.15706831)(66.65781757,39.7720677)
\lineto(66.65781757,44.8720677)
\lineto(67.91781757,44.8720677)
}
}
{
\newrgbcolor{curcolor}{0 0 0}
\pscustom[linestyle=none,fillstyle=solid,fillcolor=curcolor]
{
\newpath
\moveto(74.24500507,40.5820677)
\curveto(74.20000512,39.99706828)(73.46500383,39.0370677)(72.22000507,39.0370677)
\curveto(70.70500659,39.0370677)(69.94000507,39.98206933)(69.94000507,41.6170677)
\lineto(75.67000507,41.6170677)
\curveto(75.67000507,44.39206492)(74.56000281,46.1920677)(72.29500507,46.1920677)
\curveto(69.70000767,46.1920677)(68.53000507,44.25706527)(68.53000507,41.8270677)
\curveto(68.53000507,39.56206996)(69.83500728,37.8970677)(72.04000507,37.8970677)
\curveto(73.30000381,37.8970677)(73.81000543,38.19706794)(74.17000507,38.4370677)
\curveto(75.16000408,39.09706704)(75.52000512,40.20706807)(75.56500507,40.5820677)
\lineto(74.24500507,40.5820677)
\moveto(69.94000507,42.6670677)
\curveto(69.94000507,43.88206648)(70.90000629,45.0070677)(72.11500507,45.0070677)
\curveto(73.72000347,45.0070677)(74.23000515,43.88206648)(74.30500507,42.6670677)
\lineto(69.94000507,42.6670677)
}
}
{
\newrgbcolor{curcolor}{0 0 0}
\pscustom[linestyle=none,fillstyle=solid,fillcolor=curcolor]
{
\newpath
\moveto(15.2657082,22.42183332)
\lineto(8.7707082,22.42183332)
\lineto(8.7707082,26.02183332)
\lineto(14.6657082,26.02183332)
\lineto(14.6657082,27.31183332)
\lineto(8.7707082,27.31183332)
\lineto(8.7707082,30.61183332)
\lineto(15.1607082,30.61183332)
\lineto(15.1607082,31.90183332)
\lineto(7.3157082,31.90183332)
\lineto(7.3157082,21.13183332)
\lineto(15.2657082,21.13183332)
\lineto(15.2657082,22.42183332)
}
}
{
\newrgbcolor{curcolor}{0 0 0}
\pscustom[linestyle=none,fillstyle=solid,fillcolor=curcolor]
{
\newpath
\moveto(22.36938007,26.74183332)
\curveto(22.36938007,27.13183293)(22.17437727,29.20183332)(19.36938007,29.20183332)
\curveto(17.82438162,29.20183332)(16.39938007,28.4218316)(16.39938007,26.69683332)
\curveto(16.39938007,25.6168344)(17.11938117,25.06183305)(18.21438007,24.79183332)
\lineto(19.74438007,24.41683332)
\curveto(20.86937895,24.13183361)(21.30438007,23.92183269)(21.30438007,23.29183332)
\curveto(21.30438007,22.42183419)(20.44937913,22.04683332)(19.50438007,22.04683332)
\curveto(17.64438193,22.04683332)(17.46438003,23.03683394)(17.41938007,23.65183332)
\lineto(16.14438007,23.65183332)
\curveto(16.18938003,22.70683427)(16.41438318,20.90683332)(19.51938007,20.90683332)
\curveto(21.2893783,20.90683332)(22.62438007,21.88183494)(22.62438007,23.50183332)
\curveto(22.62438007,24.56683226)(22.05437844,25.16683373)(20.41938007,25.57183332)
\lineto(19.09938007,25.90183332)
\curveto(18.07938109,26.15683307)(17.67438007,26.30683397)(17.67438007,26.95183332)
\curveto(17.67438007,27.92683235)(18.82938048,28.06183332)(19.23438007,28.06183332)
\curveto(20.89937841,28.06183332)(21.07938009,27.23683283)(21.09438007,26.74183332)
\lineto(22.36938007,26.74183332)
}
}
{
\newrgbcolor{curcolor}{0 0 0}
\pscustom[linestyle=none,fillstyle=solid,fillcolor=curcolor]
{
\newpath
\moveto(27.01938007,27.88183332)
\lineto(27.01938007,28.97683332)
\lineto(25.75938007,28.97683332)
\lineto(25.75938007,31.16683332)
\lineto(24.43938007,31.16683332)
\lineto(24.43938007,28.97683332)
\lineto(23.37438007,28.97683332)
\lineto(23.37438007,27.88183332)
\lineto(24.43938007,27.88183332)
\lineto(24.43938007,22.70683332)
\curveto(24.43938007,21.76183427)(24.72438138,21.02683332)(26.02938007,21.02683332)
\curveto(26.16437994,21.02683332)(26.53938055,21.08683337)(27.01938007,21.13183332)
\lineto(27.01938007,22.16683332)
\lineto(26.55438007,22.16683332)
\curveto(26.28438034,22.16683332)(25.75938007,22.16683394)(25.75938007,22.78183332)
\lineto(25.75938007,27.88183332)
\lineto(27.01938007,27.88183332)
}
}
{
\newrgbcolor{curcolor}{0 0 0}
\pscustom[linestyle=none,fillstyle=solid,fillcolor=curcolor]
{
\newpath
\moveto(34.65953632,21.13183332)
\lineto(34.65953632,28.97683332)
\lineto(33.33953632,28.97683332)
\lineto(33.33953632,24.65683332)
\curveto(33.33953632,23.51683446)(32.84453466,22.04683332)(31.17953632,22.04683332)
\curveto(30.32453718,22.04683332)(29.66453632,22.48183461)(29.66453632,23.77183332)
\lineto(29.66453632,28.97683332)
\lineto(28.34453632,28.97683332)
\lineto(28.34453632,23.33683332)
\curveto(28.34453632,21.4618352)(29.73953748,20.90683332)(30.89453632,20.90683332)
\curveto(32.15453506,20.90683332)(32.82953688,21.38683424)(33.38453632,22.30183332)
\lineto(33.41453632,22.27183332)
\lineto(33.41453632,21.13183332)
\lineto(34.65953632,21.13183332)
}
}
{
\newrgbcolor{curcolor}{0 0 0}
\pscustom[linestyle=none,fillstyle=solid,fillcolor=curcolor]
{
\newpath
\moveto(43.1591457,31.90183332)
\lineto(41.8391457,31.90183332)
\lineto(41.8391457,27.97183332)
\lineto(41.8091457,27.86683332)
\curveto(41.49414601,28.31683287)(40.89414427,29.20183332)(39.4691457,29.20183332)
\curveto(37.38414778,29.20183332)(36.1991457,27.49183112)(36.1991457,25.28683332)
\curveto(36.1991457,23.4118352)(36.97914837,20.90683332)(39.6491457,20.90683332)
\curveto(40.41414493,20.90683332)(41.31414627,21.14683439)(41.8841457,22.21183332)
\lineto(41.9141457,22.21183332)
\lineto(41.9141457,21.13183332)
\lineto(43.1591457,21.13183332)
\lineto(43.1591457,31.90183332)
\moveto(37.5641457,25.07683332)
\curveto(37.5641457,26.08183232)(37.66914774,28.01683332)(39.7091457,28.01683332)
\curveto(41.61414379,28.01683332)(41.8241457,25.96183205)(41.8241457,24.68683332)
\curveto(41.8241457,22.60183541)(40.51914486,22.04683332)(39.6791457,22.04683332)
\curveto(38.23914714,22.04683332)(37.5641457,23.35183505)(37.5641457,25.07683332)
}
}
{
\newrgbcolor{curcolor}{0 0 0}
\pscustom[linestyle=none,fillstyle=solid,fillcolor=curcolor]
{
\newpath
\moveto(46.34875507,28.97683332)
\lineto(45.02875507,28.97683332)
\lineto(45.02875507,21.13183332)
\lineto(46.34875507,21.13183332)
\lineto(46.34875507,28.97683332)
\moveto(46.34875507,30.40183332)
\lineto(46.34875507,31.90183332)
\lineto(45.02875507,31.90183332)
\lineto(45.02875507,30.40183332)
\lineto(46.34875507,30.40183332)
}
}
{
\newrgbcolor{curcolor}{0 0 0}
\pscustom[linestyle=none,fillstyle=solid,fillcolor=curcolor]
{
\newpath
\moveto(49.50859882,26.59183332)
\curveto(49.59859873,27.19183272)(49.80860032,28.10683332)(51.30859882,28.10683332)
\curveto(52.55359758,28.10683332)(53.15359882,27.6568325)(53.15359882,26.83183332)
\curveto(53.15359882,26.0518341)(52.77859851,25.93183329)(52.46359882,25.90183332)
\lineto(50.28859882,25.63183332)
\curveto(48.09860101,25.36183359)(47.90359882,23.83183266)(47.90359882,23.17183332)
\curveto(47.90359882,21.82183467)(48.92360026,20.90683332)(50.36359882,20.90683332)
\curveto(51.89359729,20.90683332)(52.68859933,21.62683388)(53.19859882,22.18183332)
\curveto(53.24359878,21.58183392)(53.42359999,20.98183332)(54.59359882,20.98183332)
\curveto(54.89359852,20.98183332)(55.08859905,21.07183338)(55.31359882,21.13183332)
\lineto(55.31359882,22.09183332)
\curveto(55.16359897,22.06183335)(54.9985987,22.03183332)(54.87859882,22.03183332)
\curveto(54.60859909,22.03183332)(54.44359882,22.16683365)(54.44359882,22.49683332)
\lineto(54.44359882,27.01183332)
\curveto(54.44359882,29.02183131)(52.16359819,29.20183332)(51.53359882,29.20183332)
\curveto(49.59860076,29.20183332)(48.35359876,28.46683145)(48.29359882,26.59183332)
\lineto(49.50859882,26.59183332)
\moveto(53.12359882,23.84683332)
\curveto(53.12359882,22.79683437)(51.92359759,22.00183332)(50.69359882,22.00183332)
\curveto(49.70359981,22.00183332)(49.26859882,22.51183418)(49.26859882,23.36683332)
\curveto(49.26859882,24.35683233)(50.30359947,24.55183341)(50.94859882,24.64183332)
\curveto(52.58359719,24.85183311)(52.91359903,24.97183349)(53.12359882,25.13683332)
\lineto(53.12359882,23.84683332)
}
}
{
\newrgbcolor{curcolor}{0 0 0}
\pscustom[linestyle=none,fillstyle=solid,fillcolor=curcolor]
{
\newpath
\moveto(63.0782082,26.47183332)
\curveto(63.0782082,28.70683109)(61.54820698,29.20183332)(60.3332082,29.20183332)
\curveto(58.98320955,29.20183332)(58.24820791,28.2868329)(57.9632082,27.86683332)
\lineto(57.9332082,27.86683332)
\lineto(57.9332082,28.97683332)
\lineto(56.6882082,28.97683332)
\lineto(56.6882082,21.13183332)
\lineto(58.0082082,21.13183332)
\lineto(58.0082082,25.40683332)
\curveto(58.0082082,27.53683119)(59.32820895,28.01683332)(60.0782082,28.01683332)
\curveto(61.36820691,28.01683332)(61.7582082,27.32683196)(61.7582082,25.96183332)
\lineto(61.7582082,21.13183332)
\lineto(63.0782082,21.13183332)
\lineto(63.0782082,26.47183332)
}
}
{
\newrgbcolor{curcolor}{0 0 0}
\pscustom[linestyle=none,fillstyle=solid,fillcolor=curcolor]
{
\newpath
\moveto(67.91781757,27.88183332)
\lineto(67.91781757,28.97683332)
\lineto(66.65781757,28.97683332)
\lineto(66.65781757,31.16683332)
\lineto(65.33781757,31.16683332)
\lineto(65.33781757,28.97683332)
\lineto(64.27281757,28.97683332)
\lineto(64.27281757,27.88183332)
\lineto(65.33781757,27.88183332)
\lineto(65.33781757,22.70683332)
\curveto(65.33781757,21.76183427)(65.62281888,21.02683332)(66.92781757,21.02683332)
\curveto(67.06281744,21.02683332)(67.43781805,21.08683337)(67.91781757,21.13183332)
\lineto(67.91781757,22.16683332)
\lineto(67.45281757,22.16683332)
\curveto(67.18281784,22.16683332)(66.65781757,22.16683394)(66.65781757,22.78183332)
\lineto(66.65781757,27.88183332)
\lineto(67.91781757,27.88183332)
}
}
{
\newrgbcolor{curcolor}{0 0 0}
\pscustom[linestyle=none,fillstyle=solid,fillcolor=curcolor]
{
\newpath
\moveto(74.24500507,23.59183332)
\curveto(74.20000512,23.00683391)(73.46500383,22.04683332)(72.22000507,22.04683332)
\curveto(70.70500659,22.04683332)(69.94000507,22.99183496)(69.94000507,24.62683332)
\lineto(75.67000507,24.62683332)
\curveto(75.67000507,27.40183055)(74.56000281,29.20183332)(72.29500507,29.20183332)
\curveto(69.70000767,29.20183332)(68.53000507,27.26683089)(68.53000507,24.83683332)
\curveto(68.53000507,22.57183559)(69.83500728,20.90683332)(72.04000507,20.90683332)
\curveto(73.30000381,20.90683332)(73.81000543,21.20683356)(74.17000507,21.44683332)
\curveto(75.16000408,22.10683266)(75.52000512,23.2168337)(75.56500507,23.59183332)
\lineto(74.24500507,23.59183332)
\moveto(69.94000507,25.67683332)
\curveto(69.94000507,26.89183211)(70.90000629,28.01683332)(72.11500507,28.01683332)
\curveto(73.72000347,28.01683332)(74.23000515,26.89183211)(74.30500507,25.67683332)
\lineto(69.94000507,25.67683332)
}
}
{
\newrgbcolor{curcolor}{0 0 0}
\pscustom[linestyle=none,fillstyle=solid,fillcolor=curcolor]
{
\newpath
\moveto(15.2657082,5.43172102)
\lineto(8.7707082,5.43172102)
\lineto(8.7707082,9.03172102)
\lineto(14.6657082,9.03172102)
\lineto(14.6657082,10.32172102)
\lineto(8.7707082,10.32172102)
\lineto(8.7707082,13.62172102)
\lineto(15.1607082,13.62172102)
\lineto(15.1607082,14.91172102)
\lineto(7.3157082,14.91172102)
\lineto(7.3157082,4.14172102)
\lineto(15.2657082,4.14172102)
\lineto(15.2657082,5.43172102)
}
}
{
\newrgbcolor{curcolor}{0 0 0}
\pscustom[linestyle=none,fillstyle=solid,fillcolor=curcolor]
{
\newpath
\moveto(22.36938007,9.75172102)
\curveto(22.36938007,10.14172063)(22.17437727,12.21172102)(19.36938007,12.21172102)
\curveto(17.82438162,12.21172102)(16.39938007,11.43171929)(16.39938007,9.70672102)
\curveto(16.39938007,8.6267221)(17.11938117,8.07172075)(18.21438007,7.80172102)
\lineto(19.74438007,7.42672102)
\curveto(20.86937895,7.1417213)(21.30438007,6.93172039)(21.30438007,6.30172102)
\curveto(21.30438007,5.43172189)(20.44937913,5.05672102)(19.50438007,5.05672102)
\curveto(17.64438193,5.05672102)(17.46438003,6.04672163)(17.41938007,6.66172102)
\lineto(16.14438007,6.66172102)
\curveto(16.18938003,5.71672196)(16.41438318,3.91672102)(19.51938007,3.91672102)
\curveto(21.2893783,3.91672102)(22.62438007,4.89172264)(22.62438007,6.51172102)
\curveto(22.62438007,7.57671995)(22.05437844,8.17672142)(20.41938007,8.58172102)
\lineto(19.09938007,8.91172102)
\curveto(18.07938109,9.16672076)(17.67438007,9.31672166)(17.67438007,9.96172102)
\curveto(17.67438007,10.93672004)(18.82938048,11.07172102)(19.23438007,11.07172102)
\curveto(20.89937841,11.07172102)(21.07938009,10.24672052)(21.09438007,9.75172102)
\lineto(22.36938007,9.75172102)
}
}
{
\newrgbcolor{curcolor}{0 0 0}
\pscustom[linestyle=none,fillstyle=solid,fillcolor=curcolor]
{
\newpath
\moveto(27.01938007,10.89172102)
\lineto(27.01938007,11.98672102)
\lineto(25.75938007,11.98672102)
\lineto(25.75938007,14.17672102)
\lineto(24.43938007,14.17672102)
\lineto(24.43938007,11.98672102)
\lineto(23.37438007,11.98672102)
\lineto(23.37438007,10.89172102)
\lineto(24.43938007,10.89172102)
\lineto(24.43938007,5.71672102)
\curveto(24.43938007,4.77172196)(24.72438138,4.03672102)(26.02938007,4.03672102)
\curveto(26.16437994,4.03672102)(26.53938055,4.09672106)(27.01938007,4.14172102)
\lineto(27.01938007,5.17672102)
\lineto(26.55438007,5.17672102)
\curveto(26.28438034,5.17672102)(25.75938007,5.17672163)(25.75938007,5.79172102)
\lineto(25.75938007,10.89172102)
\lineto(27.01938007,10.89172102)
}
}
{
\newrgbcolor{curcolor}{0 0 0}
\pscustom[linestyle=none,fillstyle=solid,fillcolor=curcolor]
{
\newpath
\moveto(34.65953632,4.14172102)
\lineto(34.65953632,11.98672102)
\lineto(33.33953632,11.98672102)
\lineto(33.33953632,7.66672102)
\curveto(33.33953632,6.52672216)(32.84453466,5.05672102)(31.17953632,5.05672102)
\curveto(30.32453718,5.05672102)(29.66453632,5.49172231)(29.66453632,6.78172102)
\lineto(29.66453632,11.98672102)
\lineto(28.34453632,11.98672102)
\lineto(28.34453632,6.34672102)
\curveto(28.34453632,4.47172289)(29.73953748,3.91672102)(30.89453632,3.91672102)
\curveto(32.15453506,3.91672102)(32.82953688,4.39672193)(33.38453632,5.31172102)
\lineto(33.41453632,5.28172102)
\lineto(33.41453632,4.14172102)
\lineto(34.65953632,4.14172102)
}
}
{
\newrgbcolor{curcolor}{0 0 0}
\pscustom[linestyle=none,fillstyle=solid,fillcolor=curcolor]
{
\newpath
\moveto(43.1591457,14.91172102)
\lineto(41.8391457,14.91172102)
\lineto(41.8391457,10.98172102)
\lineto(41.8091457,10.87672102)
\curveto(41.49414601,11.32672057)(40.89414427,12.21172102)(39.4691457,12.21172102)
\curveto(37.38414778,12.21172102)(36.1991457,10.50171881)(36.1991457,8.29672102)
\curveto(36.1991457,6.42172289)(36.97914837,3.91672102)(39.6491457,3.91672102)
\curveto(40.41414493,3.91672102)(41.31414627,4.15672208)(41.8841457,5.22172102)
\lineto(41.9141457,5.22172102)
\lineto(41.9141457,4.14172102)
\lineto(43.1591457,4.14172102)
\lineto(43.1591457,14.91172102)
\moveto(37.5641457,8.08672102)
\curveto(37.5641457,9.09172001)(37.66914774,11.02672102)(39.7091457,11.02672102)
\curveto(41.61414379,11.02672102)(41.8241457,8.97171974)(41.8241457,7.69672102)
\curveto(41.8241457,5.6117231)(40.51914486,5.05672102)(39.6791457,5.05672102)
\curveto(38.23914714,5.05672102)(37.5641457,6.36172274)(37.5641457,8.08672102)
}
}
{
\newrgbcolor{curcolor}{0 0 0}
\pscustom[linestyle=none,fillstyle=solid,fillcolor=curcolor]
{
\newpath
\moveto(46.34875507,11.98672102)
\lineto(45.02875507,11.98672102)
\lineto(45.02875507,4.14172102)
\lineto(46.34875507,4.14172102)
\lineto(46.34875507,11.98672102)
\moveto(46.34875507,13.41172102)
\lineto(46.34875507,14.91172102)
\lineto(45.02875507,14.91172102)
\lineto(45.02875507,13.41172102)
\lineto(46.34875507,13.41172102)
}
}
{
\newrgbcolor{curcolor}{0 0 0}
\pscustom[linestyle=none,fillstyle=solid,fillcolor=curcolor]
{
\newpath
\moveto(49.50859882,9.60172102)
\curveto(49.59859873,10.20172042)(49.80860032,11.11672102)(51.30859882,11.11672102)
\curveto(52.55359758,11.11672102)(53.15359882,10.66672019)(53.15359882,9.84172102)
\curveto(53.15359882,9.0617218)(52.77859851,8.94172099)(52.46359882,8.91172102)
\lineto(50.28859882,8.64172102)
\curveto(48.09860101,8.37172129)(47.90359882,6.84172036)(47.90359882,6.18172102)
\curveto(47.90359882,4.83172237)(48.92360026,3.91672102)(50.36359882,3.91672102)
\curveto(51.89359729,3.91672102)(52.68859933,4.63672157)(53.19859882,5.19172102)
\curveto(53.24359878,4.59172162)(53.42359999,3.99172102)(54.59359882,3.99172102)
\curveto(54.89359852,3.99172102)(55.08859905,4.08172108)(55.31359882,4.14172102)
\lineto(55.31359882,5.10172102)
\curveto(55.16359897,5.07172105)(54.9985987,5.04172102)(54.87859882,5.04172102)
\curveto(54.60859909,5.04172102)(54.44359882,5.17672135)(54.44359882,5.50672102)
\lineto(54.44359882,10.02172102)
\curveto(54.44359882,12.03171901)(52.16359819,12.21172102)(51.53359882,12.21172102)
\curveto(49.59860076,12.21172102)(48.35359876,11.47671914)(48.29359882,9.60172102)
\lineto(49.50859882,9.60172102)
\moveto(53.12359882,6.85672102)
\curveto(53.12359882,5.80672207)(51.92359759,5.01172102)(50.69359882,5.01172102)
\curveto(49.70359981,5.01172102)(49.26859882,5.52172187)(49.26859882,6.37672102)
\curveto(49.26859882,7.36672003)(50.30359947,7.56172111)(50.94859882,7.65172102)
\curveto(52.58359719,7.86172081)(52.91359903,7.98172118)(53.12359882,8.14672102)
\lineto(53.12359882,6.85672102)
}
}
{
\newrgbcolor{curcolor}{0 0 0}
\pscustom[linestyle=none,fillstyle=solid,fillcolor=curcolor]
{
\newpath
\moveto(63.0782082,9.48172102)
\curveto(63.0782082,11.71671878)(61.54820698,12.21172102)(60.3332082,12.21172102)
\curveto(58.98320955,12.21172102)(58.24820791,11.2967206)(57.9632082,10.87672102)
\lineto(57.9332082,10.87672102)
\lineto(57.9332082,11.98672102)
\lineto(56.6882082,11.98672102)
\lineto(56.6882082,4.14172102)
\lineto(58.0082082,4.14172102)
\lineto(58.0082082,8.41672102)
\curveto(58.0082082,10.54671889)(59.32820895,11.02672102)(60.0782082,11.02672102)
\curveto(61.36820691,11.02672102)(61.7582082,10.33671965)(61.7582082,8.97172102)
\lineto(61.7582082,4.14172102)
\lineto(63.0782082,4.14172102)
\lineto(63.0782082,9.48172102)
}
}
{
\newrgbcolor{curcolor}{0 0 0}
\pscustom[linestyle=none,fillstyle=solid,fillcolor=curcolor]
{
\newpath
\moveto(67.91781757,10.89172102)
\lineto(67.91781757,11.98672102)
\lineto(66.65781757,11.98672102)
\lineto(66.65781757,14.17672102)
\lineto(65.33781757,14.17672102)
\lineto(65.33781757,11.98672102)
\lineto(64.27281757,11.98672102)
\lineto(64.27281757,10.89172102)
\lineto(65.33781757,10.89172102)
\lineto(65.33781757,5.71672102)
\curveto(65.33781757,4.77172196)(65.62281888,4.03672102)(66.92781757,4.03672102)
\curveto(67.06281744,4.03672102)(67.43781805,4.09672106)(67.91781757,4.14172102)
\lineto(67.91781757,5.17672102)
\lineto(67.45281757,5.17672102)
\curveto(67.18281784,5.17672102)(66.65781757,5.17672163)(66.65781757,5.79172102)
\lineto(66.65781757,10.89172102)
\lineto(67.91781757,10.89172102)
}
}
{
\newrgbcolor{curcolor}{0 0 0}
\pscustom[linestyle=none,fillstyle=solid,fillcolor=curcolor]
{
\newpath
\moveto(74.24500507,6.60172102)
\curveto(74.20000512,6.0167216)(73.46500383,5.05672102)(72.22000507,5.05672102)
\curveto(70.70500659,5.05672102)(69.94000507,6.00172265)(69.94000507,7.63672102)
\lineto(75.67000507,7.63672102)
\curveto(75.67000507,10.41171824)(74.56000281,12.21172102)(72.29500507,12.21172102)
\curveto(69.70000767,12.21172102)(68.53000507,10.27671859)(68.53000507,7.84672102)
\curveto(68.53000507,5.58172328)(69.83500728,3.91672102)(72.04000507,3.91672102)
\curveto(73.30000381,3.91672102)(73.81000543,4.21672126)(74.17000507,4.45672102)
\curveto(75.16000408,5.11672036)(75.52000512,6.22672139)(75.56500507,6.60172102)
\lineto(74.24500507,6.60172102)
\moveto(69.94000507,8.68672102)
\curveto(69.94000507,9.9017198)(70.90000629,11.02672102)(72.11500507,11.02672102)
\curveto(73.72000347,11.02672102)(74.23000515,9.9017198)(74.30500507,8.68672102)
\lineto(69.94000507,8.68672102)
}
}
{
\newrgbcolor{curcolor}{0 0 0}
\pscustom[linestyle=none,fillstyle=solid,fillcolor=curcolor]
{
\newpath
\moveto(8.8457082,133.84330061)
\lineto(7.3907082,133.84330061)
\lineto(7.3907082,123.07330061)
\lineto(8.8457082,123.07330061)
\lineto(8.8457082,133.84330061)
}
}
{
\newrgbcolor{curcolor}{0 0 0}
\pscustom[linestyle=none,fillstyle=solid,fillcolor=curcolor]
{
\newpath
\moveto(17.55086445,128.41330061)
\curveto(17.55086445,130.64829837)(16.02086323,131.14330061)(14.80586445,131.14330061)
\curveto(13.4558658,131.14330061)(12.72086416,130.22830019)(12.43586445,129.80830061)
\lineto(12.40586445,129.80830061)
\lineto(12.40586445,130.91830061)
\lineto(11.16086445,130.91830061)
\lineto(11.16086445,123.07330061)
\lineto(12.48086445,123.07330061)
\lineto(12.48086445,127.34830061)
\curveto(12.48086445,129.47829848)(13.8008652,129.95830061)(14.55086445,129.95830061)
\curveto(15.84086316,129.95830061)(16.23086445,129.26829924)(16.23086445,127.90330061)
\lineto(16.23086445,123.07330061)
\lineto(17.55086445,123.07330061)
\lineto(17.55086445,128.41330061)
}
}
{
\newrgbcolor{curcolor}{0 0 0}
\pscustom[linestyle=none,fillstyle=solid,fillcolor=curcolor]
{
\newpath
\moveto(21.96250507,124.52830061)
\lineto(21.93250507,124.52830061)
\lineto(19.89250507,130.91830061)
\lineto(18.36250507,130.91830061)
\lineto(21.22750507,123.07330061)
\lineto(22.63750507,123.07330061)
\lineto(25.62250507,130.91830061)
\lineto(24.18250507,130.91830061)
\lineto(21.96250507,124.52830061)
}
}
{
\newrgbcolor{curcolor}{0 0 0}
\pscustom[linestyle=none,fillstyle=solid,fillcolor=curcolor]
{
\newpath
\moveto(28.06750507,130.91830061)
\lineto(26.74750507,130.91830061)
\lineto(26.74750507,123.07330061)
\lineto(28.06750507,123.07330061)
\lineto(28.06750507,130.91830061)
\moveto(28.06750507,132.34330061)
\lineto(28.06750507,133.84330061)
\lineto(26.74750507,133.84330061)
\lineto(26.74750507,132.34330061)
\lineto(28.06750507,132.34330061)
}
}
{
\newrgbcolor{curcolor}{0 0 0}
\pscustom[linestyle=none,fillstyle=solid,fillcolor=curcolor]
{
\newpath
\moveto(32.93734882,129.82330061)
\lineto(32.93734882,130.91830061)
\lineto(31.67734882,130.91830061)
\lineto(31.67734882,133.10830061)
\lineto(30.35734882,133.10830061)
\lineto(30.35734882,130.91830061)
\lineto(29.29234882,130.91830061)
\lineto(29.29234882,129.82330061)
\lineto(30.35734882,129.82330061)
\lineto(30.35734882,124.64830061)
\curveto(30.35734882,123.70330155)(30.64235013,122.96830061)(31.94734882,122.96830061)
\curveto(32.08234869,122.96830061)(32.4573493,123.02830065)(32.93734882,123.07330061)
\lineto(32.93734882,124.10830061)
\lineto(32.47234882,124.10830061)
\curveto(32.20234909,124.10830061)(31.67734882,124.10830122)(31.67734882,124.72330061)
\lineto(31.67734882,129.82330061)
\lineto(32.93734882,129.82330061)
}
}
{
\newrgbcolor{curcolor}{0 0 0}
\pscustom[linestyle=none,fillstyle=solid,fillcolor=curcolor]
{
\newpath
\moveto(35.2410207,128.53330061)
\curveto(35.33102061,129.13330001)(35.5410222,130.04830061)(37.0410207,130.04830061)
\curveto(38.28601945,130.04830061)(38.8860207,129.59829978)(38.8860207,128.77330061)
\curveto(38.8860207,127.99330139)(38.51102038,127.87330058)(38.1960207,127.84330061)
\lineto(36.0210207,127.57330061)
\curveto(33.83102289,127.30330088)(33.6360207,125.77329995)(33.6360207,125.11330061)
\curveto(33.6360207,123.76330196)(34.65602214,122.84830061)(36.0960207,122.84830061)
\curveto(37.62601917,122.84830061)(38.42102121,123.56830116)(38.9310207,124.12330061)
\curveto(38.97602065,123.52330121)(39.15602187,122.92330061)(40.3260207,122.92330061)
\curveto(40.6260204,122.92330061)(40.82102092,123.01330067)(41.0460207,123.07330061)
\lineto(41.0460207,124.03330061)
\curveto(40.89602085,124.00330064)(40.73102058,123.97330061)(40.6110207,123.97330061)
\curveto(40.34102097,123.97330061)(40.1760207,124.10830094)(40.1760207,124.43830061)
\lineto(40.1760207,128.95330061)
\curveto(40.1760207,130.9632986)(37.89602007,131.14330061)(37.2660207,131.14330061)
\curveto(35.33102263,131.14330061)(34.08602064,130.40829873)(34.0260207,128.53330061)
\lineto(35.2410207,128.53330061)
\moveto(38.8560207,125.78830061)
\curveto(38.8560207,124.73830166)(37.65601947,123.94330061)(36.4260207,123.94330061)
\curveto(35.43602169,123.94330061)(35.0010207,124.45330146)(35.0010207,125.30830061)
\curveto(35.0010207,126.29829962)(36.03602134,126.4933007)(36.6810207,126.58330061)
\curveto(38.31601906,126.7933004)(38.64602091,126.91330077)(38.8560207,127.07830061)
\lineto(38.8560207,125.78830061)
}
}
{
\newrgbcolor{curcolor}{0 0 0}
\pscustom[linestyle=none,fillstyle=solid,fillcolor=curcolor]
{
\newpath
\moveto(48.93063007,133.84330061)
\lineto(47.61063007,133.84330061)
\lineto(47.61063007,129.91330061)
\lineto(47.58063007,129.80830061)
\curveto(47.26563039,130.25830016)(46.66562865,131.14330061)(45.24063007,131.14330061)
\curveto(43.15563216,131.14330061)(41.97063007,129.4332984)(41.97063007,127.22830061)
\curveto(41.97063007,125.35330248)(42.75063274,122.84830061)(45.42063007,122.84830061)
\curveto(46.18562931,122.84830061)(47.08563064,123.08830167)(47.65563007,124.15330061)
\lineto(47.68563007,124.15330061)
\lineto(47.68563007,123.07330061)
\lineto(48.93063007,123.07330061)
\lineto(48.93063007,133.84330061)
\moveto(43.33563007,127.01830061)
\curveto(43.33563007,128.0232996)(43.44063211,129.95830061)(45.48063007,129.95830061)
\curveto(47.38562817,129.95830061)(47.59563007,127.90329933)(47.59563007,126.62830061)
\curveto(47.59563007,124.54330269)(46.29062923,123.98830061)(45.45063007,123.98830061)
\curveto(44.01063151,123.98830061)(43.33563007,125.29330233)(43.33563007,127.01830061)
}
}
{
\newrgbcolor{curcolor}{0 0 0}
\pscustom[linestyle=none,fillstyle=solid,fillcolor=curcolor]
{
\newpath
\moveto(50.32023945,127.00330061)
\curveto(50.32023945,124.97830263)(51.46024195,122.86330061)(53.96523945,122.86330061)
\curveto(56.47023694,122.86330061)(57.61023945,124.97830263)(57.61023945,127.00330061)
\curveto(57.61023945,129.02829858)(56.47023694,131.14330061)(53.96523945,131.14330061)
\curveto(51.46024195,131.14330061)(50.32023945,129.02829858)(50.32023945,127.00330061)
\moveto(51.68523945,127.00330061)
\curveto(51.68523945,128.05329956)(52.07524134,130.00330061)(53.96523945,130.00330061)
\curveto(55.85523756,130.00330061)(56.24523945,128.05329956)(56.24523945,127.00330061)
\curveto(56.24523945,125.95330166)(55.85523756,124.00330061)(53.96523945,124.00330061)
\curveto(52.07524134,124.00330061)(51.68523945,125.95330166)(51.68523945,127.00330061)
}
}
{
\newrgbcolor{curcolor}{0 0 0}
\pscustom[linestyle=none,fillstyle=solid,fillcolor=curcolor]
{
\newpath
\moveto(8.8457082,116.85306623)
\lineto(7.3907082,116.85306623)
\lineto(7.3907082,106.08306623)
\lineto(8.8457082,106.08306623)
\lineto(8.8457082,116.85306623)
}
}
{
\newrgbcolor{curcolor}{0 0 0}
\pscustom[linestyle=none,fillstyle=solid,fillcolor=curcolor]
{
\newpath
\moveto(17.55086445,111.42306623)
\curveto(17.55086445,113.658064)(16.02086323,114.15306623)(14.80586445,114.15306623)
\curveto(13.4558658,114.15306623)(12.72086416,113.23806581)(12.43586445,112.81806623)
\lineto(12.40586445,112.81806623)
\lineto(12.40586445,113.92806623)
\lineto(11.16086445,113.92806623)
\lineto(11.16086445,106.08306623)
\lineto(12.48086445,106.08306623)
\lineto(12.48086445,110.35806623)
\curveto(12.48086445,112.4880641)(13.8008652,112.96806623)(14.55086445,112.96806623)
\curveto(15.84086316,112.96806623)(16.23086445,112.27806487)(16.23086445,110.91306623)
\lineto(16.23086445,106.08306623)
\lineto(17.55086445,106.08306623)
\lineto(17.55086445,111.42306623)
}
}
{
\newrgbcolor{curcolor}{0 0 0}
\pscustom[linestyle=none,fillstyle=solid,fillcolor=curcolor]
{
\newpath
\moveto(21.96250507,107.53806623)
\lineto(21.93250507,107.53806623)
\lineto(19.89250507,113.92806623)
\lineto(18.36250507,113.92806623)
\lineto(21.22750507,106.08306623)
\lineto(22.63750507,106.08306623)
\lineto(25.62250507,113.92806623)
\lineto(24.18250507,113.92806623)
\lineto(21.96250507,107.53806623)
}
}
{
\newrgbcolor{curcolor}{0 0 0}
\pscustom[linestyle=none,fillstyle=solid,fillcolor=curcolor]
{
\newpath
\moveto(28.06750507,113.92806623)
\lineto(26.74750507,113.92806623)
\lineto(26.74750507,106.08306623)
\lineto(28.06750507,106.08306623)
\lineto(28.06750507,113.92806623)
\moveto(28.06750507,115.35306623)
\lineto(28.06750507,116.85306623)
\lineto(26.74750507,116.85306623)
\lineto(26.74750507,115.35306623)
\lineto(28.06750507,115.35306623)
}
}
{
\newrgbcolor{curcolor}{0 0 0}
\pscustom[linestyle=none,fillstyle=solid,fillcolor=curcolor]
{
\newpath
\moveto(32.93734882,112.83306623)
\lineto(32.93734882,113.92806623)
\lineto(31.67734882,113.92806623)
\lineto(31.67734882,116.11806623)
\lineto(30.35734882,116.11806623)
\lineto(30.35734882,113.92806623)
\lineto(29.29234882,113.92806623)
\lineto(29.29234882,112.83306623)
\lineto(30.35734882,112.83306623)
\lineto(30.35734882,107.65806623)
\curveto(30.35734882,106.71306718)(30.64235013,105.97806623)(31.94734882,105.97806623)
\curveto(32.08234869,105.97806623)(32.4573493,106.03806628)(32.93734882,106.08306623)
\lineto(32.93734882,107.11806623)
\lineto(32.47234882,107.11806623)
\curveto(32.20234909,107.11806623)(31.67734882,107.11806685)(31.67734882,107.73306623)
\lineto(31.67734882,112.83306623)
\lineto(32.93734882,112.83306623)
}
}
{
\newrgbcolor{curcolor}{0 0 0}
\pscustom[linestyle=none,fillstyle=solid,fillcolor=curcolor]
{
\newpath
\moveto(35.2410207,111.54306623)
\curveto(35.33102061,112.14306563)(35.5410222,113.05806623)(37.0410207,113.05806623)
\curveto(38.28601945,113.05806623)(38.8860207,112.60806541)(38.8860207,111.78306623)
\curveto(38.8860207,111.00306701)(38.51102038,110.8830662)(38.1960207,110.85306623)
\lineto(36.0210207,110.58306623)
\curveto(33.83102289,110.3130665)(33.6360207,108.78306557)(33.6360207,108.12306623)
\curveto(33.6360207,106.77306758)(34.65602214,105.85806623)(36.0960207,105.85806623)
\curveto(37.62601917,105.85806623)(38.42102121,106.57806679)(38.9310207,107.13306623)
\curveto(38.97602065,106.53306683)(39.15602187,105.93306623)(40.3260207,105.93306623)
\curveto(40.6260204,105.93306623)(40.82102092,106.02306629)(41.0460207,106.08306623)
\lineto(41.0460207,107.04306623)
\curveto(40.89602085,107.01306626)(40.73102058,106.98306623)(40.6110207,106.98306623)
\curveto(40.34102097,106.98306623)(40.1760207,107.11806656)(40.1760207,107.44806623)
\lineto(40.1760207,111.96306623)
\curveto(40.1760207,113.97306422)(37.89602007,114.15306623)(37.2660207,114.15306623)
\curveto(35.33102263,114.15306623)(34.08602064,113.41806436)(34.0260207,111.54306623)
\lineto(35.2410207,111.54306623)
\moveto(38.8560207,108.79806623)
\curveto(38.8560207,107.74806728)(37.65601947,106.95306623)(36.4260207,106.95306623)
\curveto(35.43602169,106.95306623)(35.0010207,107.46306709)(35.0010207,108.31806623)
\curveto(35.0010207,109.30806524)(36.03602134,109.50306632)(36.6810207,109.59306623)
\curveto(38.31601906,109.80306602)(38.64602091,109.9230664)(38.8560207,110.08806623)
\lineto(38.8560207,108.79806623)
}
}
{
\newrgbcolor{curcolor}{0 0 0}
\pscustom[linestyle=none,fillstyle=solid,fillcolor=curcolor]
{
\newpath
\moveto(48.93063007,116.85306623)
\lineto(47.61063007,116.85306623)
\lineto(47.61063007,112.92306623)
\lineto(47.58063007,112.81806623)
\curveto(47.26563039,113.26806578)(46.66562865,114.15306623)(45.24063007,114.15306623)
\curveto(43.15563216,114.15306623)(41.97063007,112.44306403)(41.97063007,110.23806623)
\curveto(41.97063007,108.36306811)(42.75063274,105.85806623)(45.42063007,105.85806623)
\curveto(46.18562931,105.85806623)(47.08563064,106.0980673)(47.65563007,107.16306623)
\lineto(47.68563007,107.16306623)
\lineto(47.68563007,106.08306623)
\lineto(48.93063007,106.08306623)
\lineto(48.93063007,116.85306623)
\moveto(43.33563007,110.02806623)
\curveto(43.33563007,111.03306523)(43.44063211,112.96806623)(45.48063007,112.96806623)
\curveto(47.38562817,112.96806623)(47.59563007,110.91306496)(47.59563007,109.63806623)
\curveto(47.59563007,107.55306832)(46.29062923,106.99806623)(45.45063007,106.99806623)
\curveto(44.01063151,106.99806623)(43.33563007,108.30306796)(43.33563007,110.02806623)
}
}
{
\newrgbcolor{curcolor}{0 0 0}
\pscustom[linestyle=none,fillstyle=solid,fillcolor=curcolor]
{
\newpath
\moveto(50.32023945,110.01306623)
\curveto(50.32023945,107.98806826)(51.46024195,105.87306623)(53.96523945,105.87306623)
\curveto(56.47023694,105.87306623)(57.61023945,107.98806826)(57.61023945,110.01306623)
\curveto(57.61023945,112.03806421)(56.47023694,114.15306623)(53.96523945,114.15306623)
\curveto(51.46024195,114.15306623)(50.32023945,112.03806421)(50.32023945,110.01306623)
\moveto(51.68523945,110.01306623)
\curveto(51.68523945,111.06306518)(52.07524134,113.01306623)(53.96523945,113.01306623)
\curveto(55.85523756,113.01306623)(56.24523945,111.06306518)(56.24523945,110.01306623)
\curveto(56.24523945,108.96306728)(55.85523756,107.01306623)(53.96523945,107.01306623)
\curveto(52.07524134,107.01306623)(51.68523945,108.96306728)(51.68523945,110.01306623)
}
}
{
\newrgbcolor{curcolor}{0 0 0}
\pscustom[linestyle=none,fillstyle=solid,fillcolor=curcolor]
{
\newpath
\moveto(8.8457082,184.81400373)
\lineto(7.3907082,184.81400373)
\lineto(7.3907082,174.04400373)
\lineto(8.8457082,174.04400373)
\lineto(8.8457082,184.81400373)
}
}
{
\newrgbcolor{curcolor}{0 0 0}
\pscustom[linestyle=none,fillstyle=solid,fillcolor=curcolor]
{
\newpath
\moveto(17.55086445,179.38400373)
\curveto(17.55086445,181.6190015)(16.02086323,182.11400373)(14.80586445,182.11400373)
\curveto(13.4558658,182.11400373)(12.72086416,181.19900331)(12.43586445,180.77900373)
\lineto(12.40586445,180.77900373)
\lineto(12.40586445,181.88900373)
\lineto(11.16086445,181.88900373)
\lineto(11.16086445,174.04400373)
\lineto(12.48086445,174.04400373)
\lineto(12.48086445,178.31900373)
\curveto(12.48086445,180.4490016)(13.8008652,180.92900373)(14.55086445,180.92900373)
\curveto(15.84086316,180.92900373)(16.23086445,180.23900237)(16.23086445,178.87400373)
\lineto(16.23086445,174.04400373)
\lineto(17.55086445,174.04400373)
\lineto(17.55086445,179.38400373)
}
}
{
\newrgbcolor{curcolor}{0 0 0}
\pscustom[linestyle=none,fillstyle=solid,fillcolor=curcolor]
{
\newpath
\moveto(21.96250507,175.49900373)
\lineto(21.93250507,175.49900373)
\lineto(19.89250507,181.88900373)
\lineto(18.36250507,181.88900373)
\lineto(21.22750507,174.04400373)
\lineto(22.63750507,174.04400373)
\lineto(25.62250507,181.88900373)
\lineto(24.18250507,181.88900373)
\lineto(21.96250507,175.49900373)
}
}
{
\newrgbcolor{curcolor}{0 0 0}
\pscustom[linestyle=none,fillstyle=solid,fillcolor=curcolor]
{
\newpath
\moveto(28.06750507,181.88900373)
\lineto(26.74750507,181.88900373)
\lineto(26.74750507,174.04400373)
\lineto(28.06750507,174.04400373)
\lineto(28.06750507,181.88900373)
\moveto(28.06750507,183.31400373)
\lineto(28.06750507,184.81400373)
\lineto(26.74750507,184.81400373)
\lineto(26.74750507,183.31400373)
\lineto(28.06750507,183.31400373)
}
}
{
\newrgbcolor{curcolor}{0 0 0}
\pscustom[linestyle=none,fillstyle=solid,fillcolor=curcolor]
{
\newpath
\moveto(32.93734882,180.79400373)
\lineto(32.93734882,181.88900373)
\lineto(31.67734882,181.88900373)
\lineto(31.67734882,184.07900373)
\lineto(30.35734882,184.07900373)
\lineto(30.35734882,181.88900373)
\lineto(29.29234882,181.88900373)
\lineto(29.29234882,180.79400373)
\lineto(30.35734882,180.79400373)
\lineto(30.35734882,175.61900373)
\curveto(30.35734882,174.67400468)(30.64235013,173.93900373)(31.94734882,173.93900373)
\curveto(32.08234869,173.93900373)(32.4573493,173.99900378)(32.93734882,174.04400373)
\lineto(32.93734882,175.07900373)
\lineto(32.47234882,175.07900373)
\curveto(32.20234909,175.07900373)(31.67734882,175.07900435)(31.67734882,175.69400373)
\lineto(31.67734882,180.79400373)
\lineto(32.93734882,180.79400373)
}
}
{
\newrgbcolor{curcolor}{0 0 0}
\pscustom[linestyle=none,fillstyle=solid,fillcolor=curcolor]
{
\newpath
\moveto(35.2410207,179.50400373)
\curveto(35.33102061,180.10400313)(35.5410222,181.01900373)(37.0410207,181.01900373)
\curveto(38.28601945,181.01900373)(38.8860207,180.56900291)(38.8860207,179.74400373)
\curveto(38.8860207,178.96400451)(38.51102038,178.8440037)(38.1960207,178.81400373)
\lineto(36.0210207,178.54400373)
\curveto(33.83102289,178.274004)(33.6360207,176.74400307)(33.6360207,176.08400373)
\curveto(33.6360207,174.73400508)(34.65602214,173.81900373)(36.0960207,173.81900373)
\curveto(37.62601917,173.81900373)(38.42102121,174.53900429)(38.9310207,175.09400373)
\curveto(38.97602065,174.49400433)(39.15602187,173.89400373)(40.3260207,173.89400373)
\curveto(40.6260204,173.89400373)(40.82102092,173.98400379)(41.0460207,174.04400373)
\lineto(41.0460207,175.00400373)
\curveto(40.89602085,174.97400376)(40.73102058,174.94400373)(40.6110207,174.94400373)
\curveto(40.34102097,174.94400373)(40.1760207,175.07900406)(40.1760207,175.40900373)
\lineto(40.1760207,179.92400373)
\curveto(40.1760207,181.93400172)(37.89602007,182.11400373)(37.2660207,182.11400373)
\curveto(35.33102263,182.11400373)(34.08602064,181.37900186)(34.0260207,179.50400373)
\lineto(35.2410207,179.50400373)
\moveto(38.8560207,176.75900373)
\curveto(38.8560207,175.70900478)(37.65601947,174.91400373)(36.4260207,174.91400373)
\curveto(35.43602169,174.91400373)(35.0010207,175.42400459)(35.0010207,176.27900373)
\curveto(35.0010207,177.26900274)(36.03602134,177.46400382)(36.6810207,177.55400373)
\curveto(38.31601906,177.76400352)(38.64602091,177.8840039)(38.8560207,178.04900373)
\lineto(38.8560207,176.75900373)
}
}
{
\newrgbcolor{curcolor}{0 0 0}
\pscustom[linestyle=none,fillstyle=solid,fillcolor=curcolor]
{
\newpath
\moveto(48.93063007,184.81400373)
\lineto(47.61063007,184.81400373)
\lineto(47.61063007,180.88400373)
\lineto(47.58063007,180.77900373)
\curveto(47.26563039,181.22900328)(46.66562865,182.11400373)(45.24063007,182.11400373)
\curveto(43.15563216,182.11400373)(41.97063007,180.40400153)(41.97063007,178.19900373)
\curveto(41.97063007,176.32400561)(42.75063274,173.81900373)(45.42063007,173.81900373)
\curveto(46.18562931,173.81900373)(47.08563064,174.0590048)(47.65563007,175.12400373)
\lineto(47.68563007,175.12400373)
\lineto(47.68563007,174.04400373)
\lineto(48.93063007,174.04400373)
\lineto(48.93063007,184.81400373)
\moveto(43.33563007,177.98900373)
\curveto(43.33563007,178.99400273)(43.44063211,180.92900373)(45.48063007,180.92900373)
\curveto(47.38562817,180.92900373)(47.59563007,178.87400246)(47.59563007,177.59900373)
\curveto(47.59563007,175.51400582)(46.29062923,174.95900373)(45.45063007,174.95900373)
\curveto(44.01063151,174.95900373)(43.33563007,176.26400546)(43.33563007,177.98900373)
}
}
{
\newrgbcolor{curcolor}{0 0 0}
\pscustom[linestyle=none,fillstyle=solid,fillcolor=curcolor]
{
\newpath
\moveto(50.32023945,177.97400373)
\curveto(50.32023945,175.94900576)(51.46024195,173.83400373)(53.96523945,173.83400373)
\curveto(56.47023694,173.83400373)(57.61023945,175.94900576)(57.61023945,177.97400373)
\curveto(57.61023945,179.99900171)(56.47023694,182.11400373)(53.96523945,182.11400373)
\curveto(51.46024195,182.11400373)(50.32023945,179.99900171)(50.32023945,177.97400373)
\moveto(51.68523945,177.97400373)
\curveto(51.68523945,179.02400268)(52.07524134,180.97400373)(53.96523945,180.97400373)
\curveto(55.85523756,180.97400373)(56.24523945,179.02400268)(56.24523945,177.97400373)
\curveto(56.24523945,176.92400478)(55.85523756,174.97400373)(53.96523945,174.97400373)
\curveto(52.07524134,174.97400373)(51.68523945,176.92400478)(51.68523945,177.97400373)
}
}
{
\newrgbcolor{curcolor}{0 0 0}
\pscustom[linestyle=none,fillstyle=solid,fillcolor=curcolor]
{
\newpath
\moveto(8.8457082,218.79447248)
\lineto(7.3907082,218.79447248)
\lineto(7.3907082,208.02447248)
\lineto(8.8457082,208.02447248)
\lineto(8.8457082,218.79447248)
}
}
{
\newrgbcolor{curcolor}{0 0 0}
\pscustom[linestyle=none,fillstyle=solid,fillcolor=curcolor]
{
\newpath
\moveto(17.55086445,213.36447248)
\curveto(17.55086445,215.59947025)(16.02086323,216.09447248)(14.80586445,216.09447248)
\curveto(13.4558658,216.09447248)(12.72086416,215.17947206)(12.43586445,214.75947248)
\lineto(12.40586445,214.75947248)
\lineto(12.40586445,215.86947248)
\lineto(11.16086445,215.86947248)
\lineto(11.16086445,208.02447248)
\lineto(12.48086445,208.02447248)
\lineto(12.48086445,212.29947248)
\curveto(12.48086445,214.42947035)(13.8008652,214.90947248)(14.55086445,214.90947248)
\curveto(15.84086316,214.90947248)(16.23086445,214.21947112)(16.23086445,212.85447248)
\lineto(16.23086445,208.02447248)
\lineto(17.55086445,208.02447248)
\lineto(17.55086445,213.36447248)
}
}
{
\newrgbcolor{curcolor}{0 0 0}
\pscustom[linestyle=none,fillstyle=solid,fillcolor=curcolor]
{
\newpath
\moveto(21.96250507,209.47947248)
\lineto(21.93250507,209.47947248)
\lineto(19.89250507,215.86947248)
\lineto(18.36250507,215.86947248)
\lineto(21.22750507,208.02447248)
\lineto(22.63750507,208.02447248)
\lineto(25.62250507,215.86947248)
\lineto(24.18250507,215.86947248)
\lineto(21.96250507,209.47947248)
}
}
{
\newrgbcolor{curcolor}{0 0 0}
\pscustom[linestyle=none,fillstyle=solid,fillcolor=curcolor]
{
\newpath
\moveto(28.06750507,215.86947248)
\lineto(26.74750507,215.86947248)
\lineto(26.74750507,208.02447248)
\lineto(28.06750507,208.02447248)
\lineto(28.06750507,215.86947248)
\moveto(28.06750507,217.29447248)
\lineto(28.06750507,218.79447248)
\lineto(26.74750507,218.79447248)
\lineto(26.74750507,217.29447248)
\lineto(28.06750507,217.29447248)
}
}
{
\newrgbcolor{curcolor}{0 0 0}
\pscustom[linestyle=none,fillstyle=solid,fillcolor=curcolor]
{
\newpath
\moveto(32.93734882,214.77447248)
\lineto(32.93734882,215.86947248)
\lineto(31.67734882,215.86947248)
\lineto(31.67734882,218.05947248)
\lineto(30.35734882,218.05947248)
\lineto(30.35734882,215.86947248)
\lineto(29.29234882,215.86947248)
\lineto(29.29234882,214.77447248)
\lineto(30.35734882,214.77447248)
\lineto(30.35734882,209.59947248)
\curveto(30.35734882,208.65447343)(30.64235013,207.91947248)(31.94734882,207.91947248)
\curveto(32.08234869,207.91947248)(32.4573493,207.97947253)(32.93734882,208.02447248)
\lineto(32.93734882,209.05947248)
\lineto(32.47234882,209.05947248)
\curveto(32.20234909,209.05947248)(31.67734882,209.0594731)(31.67734882,209.67447248)
\lineto(31.67734882,214.77447248)
\lineto(32.93734882,214.77447248)
}
}
{
\newrgbcolor{curcolor}{0 0 0}
\pscustom[linestyle=none,fillstyle=solid,fillcolor=curcolor]
{
\newpath
\moveto(35.2410207,213.48447248)
\curveto(35.33102061,214.08447188)(35.5410222,214.99947248)(37.0410207,214.99947248)
\curveto(38.28601945,214.99947248)(38.8860207,214.54947166)(38.8860207,213.72447248)
\curveto(38.8860207,212.94447326)(38.51102038,212.82447245)(38.1960207,212.79447248)
\lineto(36.0210207,212.52447248)
\curveto(33.83102289,212.25447275)(33.6360207,210.72447182)(33.6360207,210.06447248)
\curveto(33.6360207,208.71447383)(34.65602214,207.79947248)(36.0960207,207.79947248)
\curveto(37.62601917,207.79947248)(38.42102121,208.51947304)(38.9310207,209.07447248)
\curveto(38.97602065,208.47447308)(39.15602187,207.87447248)(40.3260207,207.87447248)
\curveto(40.6260204,207.87447248)(40.82102092,207.96447254)(41.0460207,208.02447248)
\lineto(41.0460207,208.98447248)
\curveto(40.89602085,208.95447251)(40.73102058,208.92447248)(40.6110207,208.92447248)
\curveto(40.34102097,208.92447248)(40.1760207,209.05947281)(40.1760207,209.38947248)
\lineto(40.1760207,213.90447248)
\curveto(40.1760207,215.91447047)(37.89602007,216.09447248)(37.2660207,216.09447248)
\curveto(35.33102263,216.09447248)(34.08602064,215.35947061)(34.0260207,213.48447248)
\lineto(35.2410207,213.48447248)
\moveto(38.8560207,210.73947248)
\curveto(38.8560207,209.68947353)(37.65601947,208.89447248)(36.4260207,208.89447248)
\curveto(35.43602169,208.89447248)(35.0010207,209.40447334)(35.0010207,210.25947248)
\curveto(35.0010207,211.24947149)(36.03602134,211.44447257)(36.6810207,211.53447248)
\curveto(38.31601906,211.74447227)(38.64602091,211.86447265)(38.8560207,212.02947248)
\lineto(38.8560207,210.73947248)
}
}
{
\newrgbcolor{curcolor}{0 0 0}
\pscustom[linestyle=none,fillstyle=solid,fillcolor=curcolor]
{
\newpath
\moveto(48.93063007,218.79447248)
\lineto(47.61063007,218.79447248)
\lineto(47.61063007,214.86447248)
\lineto(47.58063007,214.75947248)
\curveto(47.26563039,215.20947203)(46.66562865,216.09447248)(45.24063007,216.09447248)
\curveto(43.15563216,216.09447248)(41.97063007,214.38447028)(41.97063007,212.17947248)
\curveto(41.97063007,210.30447436)(42.75063274,207.79947248)(45.42063007,207.79947248)
\curveto(46.18562931,207.79947248)(47.08563064,208.03947355)(47.65563007,209.10447248)
\lineto(47.68563007,209.10447248)
\lineto(47.68563007,208.02447248)
\lineto(48.93063007,208.02447248)
\lineto(48.93063007,218.79447248)
\moveto(43.33563007,211.96947248)
\curveto(43.33563007,212.97447148)(43.44063211,214.90947248)(45.48063007,214.90947248)
\curveto(47.38562817,214.90947248)(47.59563007,212.85447121)(47.59563007,211.57947248)
\curveto(47.59563007,209.49447457)(46.29062923,208.93947248)(45.45063007,208.93947248)
\curveto(44.01063151,208.93947248)(43.33563007,210.24447421)(43.33563007,211.96947248)
}
}
{
\newrgbcolor{curcolor}{0 0 0}
\pscustom[linestyle=none,fillstyle=solid,fillcolor=curcolor]
{
\newpath
\moveto(50.32023945,211.95447248)
\curveto(50.32023945,209.92947451)(51.46024195,207.81447248)(53.96523945,207.81447248)
\curveto(56.47023694,207.81447248)(57.61023945,209.92947451)(57.61023945,211.95447248)
\curveto(57.61023945,213.97947046)(56.47023694,216.09447248)(53.96523945,216.09447248)
\curveto(51.46024195,216.09447248)(50.32023945,213.97947046)(50.32023945,211.95447248)
\moveto(51.68523945,211.95447248)
\curveto(51.68523945,213.00447143)(52.07524134,214.95447248)(53.96523945,214.95447248)
\curveto(55.85523756,214.95447248)(56.24523945,213.00447143)(56.24523945,211.95447248)
\curveto(56.24523945,210.90447353)(55.85523756,208.95447248)(53.96523945,208.95447248)
\curveto(52.07524134,208.95447248)(51.68523945,210.90447353)(51.68523945,211.95447248)
}
}
{
\newrgbcolor{curcolor}{0 0 0}
\pscustom[linestyle=none,fillstyle=solid,fillcolor=curcolor]
{
\newpath
\moveto(7.10569843,343.94634748)
\lineto(11.41069843,343.94634748)
\curveto(14.95069489,343.94634748)(16.00069843,347.0663499)(16.00069843,349.48134748)
\curveto(16.00069843,352.58634438)(14.27569563,354.71634748)(11.47069843,354.71634748)
\lineto(7.10569843,354.71634748)
\lineto(7.10569843,343.94634748)
\moveto(8.56069843,353.47134748)
\lineto(11.27569843,353.47134748)
\curveto(13.25569645,353.47134748)(14.50069843,352.10634477)(14.50069843,349.39134748)
\curveto(14.50069843,346.6763502)(13.27069654,345.19134748)(11.38069843,345.19134748)
\lineto(8.56069843,345.19134748)
\lineto(8.56069843,353.47134748)
}
}
{
\newrgbcolor{curcolor}{0 0 0}
\pscustom[linestyle=none,fillstyle=solid,fillcolor=curcolor]
{
\newpath
\moveto(17.25554218,347.87634748)
\curveto(17.25554218,345.85134951)(18.39554469,343.73634748)(20.90054218,343.73634748)
\curveto(23.40553968,343.73634748)(24.54554218,345.85134951)(24.54554218,347.87634748)
\curveto(24.54554218,349.90134546)(23.40553968,352.01634748)(20.90054218,352.01634748)
\curveto(18.39554469,352.01634748)(17.25554218,349.90134546)(17.25554218,347.87634748)
\moveto(18.62054218,347.87634748)
\curveto(18.62054218,348.92634643)(19.01054407,350.87634748)(20.90054218,350.87634748)
\curveto(22.79054029,350.87634748)(23.18054218,348.92634643)(23.18054218,347.87634748)
\curveto(23.18054218,346.82634853)(22.79054029,344.87634748)(20.90054218,344.87634748)
\curveto(19.01054407,344.87634748)(18.62054218,346.82634853)(18.62054218,347.87634748)
}
}
{
\newrgbcolor{curcolor}{0 0 0}
\pscustom[linestyle=none,fillstyle=solid,fillcolor=curcolor]
{
\newpath
\moveto(32.23515156,349.15134748)
\curveto(32.13015166,350.51634612)(31.3501495,352.01634748)(29.29515156,352.01634748)
\curveto(26.70015415,352.01634748)(25.53015156,350.08134505)(25.53015156,347.65134748)
\curveto(25.53015156,345.38634975)(26.83515376,343.72134748)(29.04015156,343.72134748)
\curveto(31.33514926,343.72134748)(32.10015169,345.47634873)(32.23515156,346.72134748)
\lineto(30.96015156,346.72134748)
\curveto(30.73515178,345.52134868)(29.97015067,344.86134748)(29.08515156,344.86134748)
\curveto(27.27015337,344.86134748)(26.94015156,346.52634883)(26.94015156,347.87634748)
\curveto(26.94015156,349.27134609)(27.46515319,350.83134748)(29.10015156,350.83134748)
\curveto(30.21015045,350.83134748)(30.79515172,350.20134643)(30.96015156,349.15134748)
\lineto(32.23515156,349.15134748)
}
}
{
\newrgbcolor{curcolor}{0 0 0}
\pscustom[linestyle=none,fillstyle=solid,fillcolor=curcolor]
{
\newpath
\moveto(38.89515156,346.40634748)
\curveto(38.8501516,345.82134807)(38.11515031,344.86134748)(36.87015156,344.86134748)
\curveto(35.35515307,344.86134748)(34.59015156,345.80634912)(34.59015156,347.44134748)
\lineto(40.32015156,347.44134748)
\curveto(40.32015156,350.21634471)(39.21014929,352.01634748)(36.94515156,352.01634748)
\curveto(34.35015415,352.01634748)(33.18015156,350.08134505)(33.18015156,347.65134748)
\curveto(33.18015156,345.38634975)(34.48515376,343.72134748)(36.69015156,343.72134748)
\curveto(37.9501503,343.72134748)(38.46015192,344.02134772)(38.82015156,344.26134748)
\curveto(39.81015057,344.92134682)(40.1701516,346.03134786)(40.21515156,346.40634748)
\lineto(38.89515156,346.40634748)
\moveto(34.59015156,348.49134748)
\curveto(34.59015156,349.70634627)(35.55015277,350.83134748)(36.76515156,350.83134748)
\curveto(38.37014995,350.83134748)(38.88015163,349.70634627)(38.95515156,348.49134748)
\lineto(34.59015156,348.49134748)
}
}
{
\newrgbcolor{curcolor}{0 0 0}
\pscustom[linestyle=none,fillstyle=solid,fillcolor=curcolor]
{
\newpath
\moveto(48.29476093,349.28634748)
\curveto(48.29476093,351.52134525)(46.76475972,352.01634748)(45.54976093,352.01634748)
\curveto(44.19976228,352.01634748)(43.46476065,351.10134706)(43.17976093,350.68134748)
\lineto(43.14976093,350.68134748)
\lineto(43.14976093,351.79134748)
\lineto(41.90476093,351.79134748)
\lineto(41.90476093,343.94634748)
\lineto(43.22476093,343.94634748)
\lineto(43.22476093,348.22134748)
\curveto(43.22476093,350.35134535)(44.54476168,350.83134748)(45.29476093,350.83134748)
\curveto(46.58475964,350.83134748)(46.97476093,350.14134612)(46.97476093,348.77634748)
\lineto(46.97476093,343.94634748)
\lineto(48.29476093,343.94634748)
\lineto(48.29476093,349.28634748)
}
}
{
\newrgbcolor{curcolor}{0 0 0}
\pscustom[linestyle=none,fillstyle=solid,fillcolor=curcolor]
{
\newpath
\moveto(53.13437031,350.69634748)
\lineto(53.13437031,351.79134748)
\lineto(51.87437031,351.79134748)
\lineto(51.87437031,353.98134748)
\lineto(50.55437031,353.98134748)
\lineto(50.55437031,351.79134748)
\lineto(49.48937031,351.79134748)
\lineto(49.48937031,350.69634748)
\lineto(50.55437031,350.69634748)
\lineto(50.55437031,345.52134748)
\curveto(50.55437031,344.57634843)(50.83937161,343.84134748)(52.14437031,343.84134748)
\curveto(52.27937017,343.84134748)(52.65437079,343.90134753)(53.13437031,343.94634748)
\lineto(53.13437031,344.98134748)
\lineto(52.66937031,344.98134748)
\curveto(52.39937058,344.98134748)(51.87437031,344.9813481)(51.87437031,345.59634748)
\lineto(51.87437031,350.69634748)
\lineto(53.13437031,350.69634748)
}
}
{
\newrgbcolor{curcolor}{0 0 0}
\pscustom[linestyle=none,fillstyle=solid,fillcolor=curcolor]
{
\newpath
\moveto(59.46155781,346.40634748)
\curveto(59.41655785,345.82134807)(58.68155656,344.86134748)(57.43655781,344.86134748)
\curveto(55.92155932,344.86134748)(55.15655781,345.80634912)(55.15655781,347.44134748)
\lineto(60.88655781,347.44134748)
\curveto(60.88655781,350.21634471)(59.77655554,352.01634748)(57.51155781,352.01634748)
\curveto(54.9165604,352.01634748)(53.74655781,350.08134505)(53.74655781,347.65134748)
\curveto(53.74655781,345.38634975)(55.05156001,343.72134748)(57.25655781,343.72134748)
\curveto(58.51655655,343.72134748)(59.02655817,344.02134772)(59.38655781,344.26134748)
\curveto(60.37655682,344.92134682)(60.73655785,346.03134786)(60.78155781,346.40634748)
\lineto(59.46155781,346.40634748)
\moveto(55.15655781,348.49134748)
\curveto(55.15655781,349.70634627)(56.11655902,350.83134748)(57.33155781,350.83134748)
\curveto(58.9365562,350.83134748)(59.44655788,349.70634627)(59.52155781,348.49134748)
\lineto(55.15655781,348.49134748)
}
}
{
\newrgbcolor{curcolor}{0 0 0}
\pscustom[linestyle=none,fillstyle=solid,fillcolor=curcolor]
{
\newpath
\moveto(7.10569843,360.93658186)
\lineto(11.41069843,360.93658186)
\curveto(14.95069489,360.93658186)(16.00069843,364.05658427)(16.00069843,366.47158186)
\curveto(16.00069843,369.57657875)(14.27569563,371.70658186)(11.47069843,371.70658186)
\lineto(7.10569843,371.70658186)
\lineto(7.10569843,360.93658186)
\moveto(8.56069843,370.46158186)
\lineto(11.27569843,370.46158186)
\curveto(13.25569645,370.46158186)(14.50069843,369.09657914)(14.50069843,366.38158186)
\curveto(14.50069843,363.66658457)(13.27069654,362.18158186)(11.38069843,362.18158186)
\lineto(8.56069843,362.18158186)
\lineto(8.56069843,370.46158186)
}
}
{
\newrgbcolor{curcolor}{0 0 0}
\pscustom[linestyle=none,fillstyle=solid,fillcolor=curcolor]
{
\newpath
\moveto(17.25554218,364.86658186)
\curveto(17.25554218,362.84158388)(18.39554469,360.72658186)(20.90054218,360.72658186)
\curveto(23.40553968,360.72658186)(24.54554218,362.84158388)(24.54554218,364.86658186)
\curveto(24.54554218,366.89157983)(23.40553968,369.00658186)(20.90054218,369.00658186)
\curveto(18.39554469,369.00658186)(17.25554218,366.89157983)(17.25554218,364.86658186)
\moveto(18.62054218,364.86658186)
\curveto(18.62054218,365.91658081)(19.01054407,367.86658186)(20.90054218,367.86658186)
\curveto(22.79054029,367.86658186)(23.18054218,365.91658081)(23.18054218,364.86658186)
\curveto(23.18054218,363.81658291)(22.79054029,361.86658186)(20.90054218,361.86658186)
\curveto(19.01054407,361.86658186)(18.62054218,363.81658291)(18.62054218,364.86658186)
}
}
{
\newrgbcolor{curcolor}{0 0 0}
\pscustom[linestyle=none,fillstyle=solid,fillcolor=curcolor]
{
\newpath
\moveto(32.23515156,366.14158186)
\curveto(32.13015166,367.50658049)(31.3501495,369.00658186)(29.29515156,369.00658186)
\curveto(26.70015415,369.00658186)(25.53015156,367.07157943)(25.53015156,364.64158186)
\curveto(25.53015156,362.37658412)(26.83515376,360.71158186)(29.04015156,360.71158186)
\curveto(31.33514926,360.71158186)(32.10015169,362.4665831)(32.23515156,363.71158186)
\lineto(30.96015156,363.71158186)
\curveto(30.73515178,362.51158306)(29.97015067,361.85158186)(29.08515156,361.85158186)
\curveto(27.27015337,361.85158186)(26.94015156,363.51658321)(26.94015156,364.86658186)
\curveto(26.94015156,366.26158046)(27.46515319,367.82158186)(29.10015156,367.82158186)
\curveto(30.21015045,367.82158186)(30.79515172,367.19158081)(30.96015156,366.14158186)
\lineto(32.23515156,366.14158186)
}
}
{
\newrgbcolor{curcolor}{0 0 0}
\pscustom[linestyle=none,fillstyle=solid,fillcolor=curcolor]
{
\newpath
\moveto(38.89515156,363.39658186)
\curveto(38.8501516,362.81158244)(38.11515031,361.85158186)(36.87015156,361.85158186)
\curveto(35.35515307,361.85158186)(34.59015156,362.79658349)(34.59015156,364.43158186)
\lineto(40.32015156,364.43158186)
\curveto(40.32015156,367.20657908)(39.21014929,369.00658186)(36.94515156,369.00658186)
\curveto(34.35015415,369.00658186)(33.18015156,367.07157943)(33.18015156,364.64158186)
\curveto(33.18015156,362.37658412)(34.48515376,360.71158186)(36.69015156,360.71158186)
\curveto(37.9501503,360.71158186)(38.46015192,361.0115821)(38.82015156,361.25158186)
\curveto(39.81015057,361.9115812)(40.1701516,363.02158223)(40.21515156,363.39658186)
\lineto(38.89515156,363.39658186)
\moveto(34.59015156,365.48158186)
\curveto(34.59015156,366.69658064)(35.55015277,367.82158186)(36.76515156,367.82158186)
\curveto(38.37014995,367.82158186)(38.88015163,366.69658064)(38.95515156,365.48158186)
\lineto(34.59015156,365.48158186)
}
}
{
\newrgbcolor{curcolor}{0 0 0}
\pscustom[linestyle=none,fillstyle=solid,fillcolor=curcolor]
{
\newpath
\moveto(48.29476093,366.27658186)
\curveto(48.29476093,368.51157962)(46.76475972,369.00658186)(45.54976093,369.00658186)
\curveto(44.19976228,369.00658186)(43.46476065,368.09158144)(43.17976093,367.67158186)
\lineto(43.14976093,367.67158186)
\lineto(43.14976093,368.78158186)
\lineto(41.90476093,368.78158186)
\lineto(41.90476093,360.93658186)
\lineto(43.22476093,360.93658186)
\lineto(43.22476093,365.21158186)
\curveto(43.22476093,367.34157973)(44.54476168,367.82158186)(45.29476093,367.82158186)
\curveto(46.58475964,367.82158186)(46.97476093,367.13158049)(46.97476093,365.76658186)
\lineto(46.97476093,360.93658186)
\lineto(48.29476093,360.93658186)
\lineto(48.29476093,366.27658186)
}
}
{
\newrgbcolor{curcolor}{0 0 0}
\pscustom[linestyle=none,fillstyle=solid,fillcolor=curcolor]
{
\newpath
\moveto(53.13437031,367.68658186)
\lineto(53.13437031,368.78158186)
\lineto(51.87437031,368.78158186)
\lineto(51.87437031,370.97158186)
\lineto(50.55437031,370.97158186)
\lineto(50.55437031,368.78158186)
\lineto(49.48937031,368.78158186)
\lineto(49.48937031,367.68658186)
\lineto(50.55437031,367.68658186)
\lineto(50.55437031,362.51158186)
\curveto(50.55437031,361.5665828)(50.83937161,360.83158186)(52.14437031,360.83158186)
\curveto(52.27937017,360.83158186)(52.65437079,360.8915819)(53.13437031,360.93658186)
\lineto(53.13437031,361.97158186)
\lineto(52.66937031,361.97158186)
\curveto(52.39937058,361.97158186)(51.87437031,361.97158247)(51.87437031,362.58658186)
\lineto(51.87437031,367.68658186)
\lineto(53.13437031,367.68658186)
}
}
{
\newrgbcolor{curcolor}{0 0 0}
\pscustom[linestyle=none,fillstyle=solid,fillcolor=curcolor]
{
\newpath
\moveto(59.46155781,363.39658186)
\curveto(59.41655785,362.81158244)(58.68155656,361.85158186)(57.43655781,361.85158186)
\curveto(55.92155932,361.85158186)(55.15655781,362.79658349)(55.15655781,364.43158186)
\lineto(60.88655781,364.43158186)
\curveto(60.88655781,367.20657908)(59.77655554,369.00658186)(57.51155781,369.00658186)
\curveto(54.9165604,369.00658186)(53.74655781,367.07157943)(53.74655781,364.64158186)
\curveto(53.74655781,362.37658412)(55.05156001,360.71158186)(57.25655781,360.71158186)
\curveto(58.51655655,360.71158186)(59.02655817,361.0115821)(59.38655781,361.25158186)
\curveto(60.37655682,361.9115812)(60.73655785,363.02158223)(60.78155781,363.39658186)
\lineto(59.46155781,363.39658186)
\moveto(55.15655781,365.48158186)
\curveto(55.15655781,366.69658064)(56.11655902,367.82158186)(57.33155781,367.82158186)
\curveto(58.9365562,367.82158186)(59.44655788,366.69658064)(59.52155781,365.48158186)
\lineto(55.15655781,365.48158186)
}
}
{
\newrgbcolor{curcolor}{0 0 0}
\pscustom[linestyle=none,fillstyle=solid,fillcolor=curcolor]
{
\newpath
\moveto(7.10569843,89.09283186)
\lineto(11.41069843,89.09283186)
\curveto(14.95069489,89.09283186)(16.00069843,92.21283427)(16.00069843,94.62783186)
\curveto(16.00069843,97.73282875)(14.27569563,99.86283186)(11.47069843,99.86283186)
\lineto(7.10569843,99.86283186)
\lineto(7.10569843,89.09283186)
\moveto(8.56069843,98.61783186)
\lineto(11.27569843,98.61783186)
\curveto(13.25569645,98.61783186)(14.50069843,97.25282914)(14.50069843,94.53783186)
\curveto(14.50069843,91.82283457)(13.27069654,90.33783186)(11.38069843,90.33783186)
\lineto(8.56069843,90.33783186)
\lineto(8.56069843,98.61783186)
}
}
{
\newrgbcolor{curcolor}{0 0 0}
\pscustom[linestyle=none,fillstyle=solid,fillcolor=curcolor]
{
\newpath
\moveto(17.25554218,93.02283186)
\curveto(17.25554218,90.99783388)(18.39554469,88.88283186)(20.90054218,88.88283186)
\curveto(23.40553968,88.88283186)(24.54554218,90.99783388)(24.54554218,93.02283186)
\curveto(24.54554218,95.04782983)(23.40553968,97.16283186)(20.90054218,97.16283186)
\curveto(18.39554469,97.16283186)(17.25554218,95.04782983)(17.25554218,93.02283186)
\moveto(18.62054218,93.02283186)
\curveto(18.62054218,94.07283081)(19.01054407,96.02283186)(20.90054218,96.02283186)
\curveto(22.79054029,96.02283186)(23.18054218,94.07283081)(23.18054218,93.02283186)
\curveto(23.18054218,91.97283291)(22.79054029,90.02283186)(20.90054218,90.02283186)
\curveto(19.01054407,90.02283186)(18.62054218,91.97283291)(18.62054218,93.02283186)
}
}
{
\newrgbcolor{curcolor}{0 0 0}
\pscustom[linestyle=none,fillstyle=solid,fillcolor=curcolor]
{
\newpath
\moveto(32.23515156,94.29783186)
\curveto(32.13015166,95.66283049)(31.3501495,97.16283186)(29.29515156,97.16283186)
\curveto(26.70015415,97.16283186)(25.53015156,95.22782943)(25.53015156,92.79783186)
\curveto(25.53015156,90.53283412)(26.83515376,88.86783186)(29.04015156,88.86783186)
\curveto(31.33514926,88.86783186)(32.10015169,90.6228331)(32.23515156,91.86783186)
\lineto(30.96015156,91.86783186)
\curveto(30.73515178,90.66783306)(29.97015067,90.00783186)(29.08515156,90.00783186)
\curveto(27.27015337,90.00783186)(26.94015156,91.67283321)(26.94015156,93.02283186)
\curveto(26.94015156,94.41783046)(27.46515319,95.97783186)(29.10015156,95.97783186)
\curveto(30.21015045,95.97783186)(30.79515172,95.34783081)(30.96015156,94.29783186)
\lineto(32.23515156,94.29783186)
}
}
{
\newrgbcolor{curcolor}{0 0 0}
\pscustom[linestyle=none,fillstyle=solid,fillcolor=curcolor]
{
\newpath
\moveto(38.89515156,91.55283186)
\curveto(38.8501516,90.96783244)(38.11515031,90.00783186)(36.87015156,90.00783186)
\curveto(35.35515307,90.00783186)(34.59015156,90.95283349)(34.59015156,92.58783186)
\lineto(40.32015156,92.58783186)
\curveto(40.32015156,95.36282908)(39.21014929,97.16283186)(36.94515156,97.16283186)
\curveto(34.35015415,97.16283186)(33.18015156,95.22782943)(33.18015156,92.79783186)
\curveto(33.18015156,90.53283412)(34.48515376,88.86783186)(36.69015156,88.86783186)
\curveto(37.9501503,88.86783186)(38.46015192,89.1678321)(38.82015156,89.40783186)
\curveto(39.81015057,90.0678312)(40.1701516,91.17783223)(40.21515156,91.55283186)
\lineto(38.89515156,91.55283186)
\moveto(34.59015156,93.63783186)
\curveto(34.59015156,94.85283064)(35.55015277,95.97783186)(36.76515156,95.97783186)
\curveto(38.37014995,95.97783186)(38.88015163,94.85283064)(38.95515156,93.63783186)
\lineto(34.59015156,93.63783186)
}
}
{
\newrgbcolor{curcolor}{0 0 0}
\pscustom[linestyle=none,fillstyle=solid,fillcolor=curcolor]
{
\newpath
\moveto(48.29476093,94.43283186)
\curveto(48.29476093,96.66782962)(46.76475972,97.16283186)(45.54976093,97.16283186)
\curveto(44.19976228,97.16283186)(43.46476065,96.24783144)(43.17976093,95.82783186)
\lineto(43.14976093,95.82783186)
\lineto(43.14976093,96.93783186)
\lineto(41.90476093,96.93783186)
\lineto(41.90476093,89.09283186)
\lineto(43.22476093,89.09283186)
\lineto(43.22476093,93.36783186)
\curveto(43.22476093,95.49782973)(44.54476168,95.97783186)(45.29476093,95.97783186)
\curveto(46.58475964,95.97783186)(46.97476093,95.28783049)(46.97476093,93.92283186)
\lineto(46.97476093,89.09283186)
\lineto(48.29476093,89.09283186)
\lineto(48.29476093,94.43283186)
}
}
{
\newrgbcolor{curcolor}{0 0 0}
\pscustom[linestyle=none,fillstyle=solid,fillcolor=curcolor]
{
\newpath
\moveto(53.13437031,95.84283186)
\lineto(53.13437031,96.93783186)
\lineto(51.87437031,96.93783186)
\lineto(51.87437031,99.12783186)
\lineto(50.55437031,99.12783186)
\lineto(50.55437031,96.93783186)
\lineto(49.48937031,96.93783186)
\lineto(49.48937031,95.84283186)
\lineto(50.55437031,95.84283186)
\lineto(50.55437031,90.66783186)
\curveto(50.55437031,89.7228328)(50.83937161,88.98783186)(52.14437031,88.98783186)
\curveto(52.27937017,88.98783186)(52.65437079,89.0478319)(53.13437031,89.09283186)
\lineto(53.13437031,90.12783186)
\lineto(52.66937031,90.12783186)
\curveto(52.39937058,90.12783186)(51.87437031,90.12783247)(51.87437031,90.74283186)
\lineto(51.87437031,95.84283186)
\lineto(53.13437031,95.84283186)
}
}
{
\newrgbcolor{curcolor}{0 0 0}
\pscustom[linestyle=none,fillstyle=solid,fillcolor=curcolor]
{
\newpath
\moveto(59.46155781,91.55283186)
\curveto(59.41655785,90.96783244)(58.68155656,90.00783186)(57.43655781,90.00783186)
\curveto(55.92155932,90.00783186)(55.15655781,90.95283349)(55.15655781,92.58783186)
\lineto(60.88655781,92.58783186)
\curveto(60.88655781,95.36282908)(59.77655554,97.16283186)(57.51155781,97.16283186)
\curveto(54.9165604,97.16283186)(53.74655781,95.22782943)(53.74655781,92.79783186)
\curveto(53.74655781,90.53283412)(55.05156001,88.86783186)(57.25655781,88.86783186)
\curveto(58.51655655,88.86783186)(59.02655817,89.1678321)(59.38655781,89.40783186)
\curveto(60.37655682,90.0678312)(60.73655785,91.17783223)(60.78155781,91.55283186)
\lineto(59.46155781,91.55283186)
\moveto(55.15655781,93.63783186)
\curveto(55.15655781,94.85283064)(56.11655902,95.97783186)(57.33155781,95.97783186)
\curveto(58.9365562,95.97783186)(59.44655788,94.85283064)(59.52155781,93.63783186)
\lineto(55.15655781,93.63783186)
}
}
{
\newrgbcolor{curcolor}{0 0 0}
\pscustom[linestyle=none,fillstyle=solid,fillcolor=curcolor]
{
\newpath
\moveto(14.17069843,160.18876936)
\lineto(15.23569843,157.05376936)
\lineto(16.82569843,157.05376936)
\lineto(12.92569843,167.82376936)
\lineto(11.27569843,167.82376936)
\lineto(7.22569843,157.05376936)
\lineto(8.72569843,157.05376936)
\lineto(9.85069843,160.18876936)
\lineto(14.17069843,160.18876936)
\moveto(10.30069843,161.47876936)
\lineto(12.02569843,166.21876936)
\lineto(12.05569843,166.21876936)
\lineto(13.64569843,161.47876936)
\lineto(10.30069843,161.47876936)
}
}
{
\newrgbcolor{curcolor}{0 0 0}
\pscustom[linestyle=none,fillstyle=solid,fillcolor=curcolor]
{
\newpath
\moveto(23.93077656,157.05376936)
\lineto(23.93077656,164.89876936)
\lineto(22.61077656,164.89876936)
\lineto(22.61077656,160.57876936)
\curveto(22.61077656,159.4387705)(22.11577489,157.96876936)(20.45077656,157.96876936)
\curveto(19.59577741,157.96876936)(18.93577656,158.40377065)(18.93577656,159.69376936)
\lineto(18.93577656,164.89876936)
\lineto(17.61577656,164.89876936)
\lineto(17.61577656,159.25876936)
\curveto(17.61577656,157.38377123)(19.01077771,156.82876936)(20.16577656,156.82876936)
\curveto(21.4257753,156.82876936)(22.10077711,157.30877027)(22.65577656,158.22376936)
\lineto(22.68577656,158.19376936)
\lineto(22.68577656,157.05376936)
\lineto(23.93077656,157.05376936)
}
}
{
\newrgbcolor{curcolor}{0 0 0}
\pscustom[linestyle=none,fillstyle=solid,fillcolor=curcolor]
{
\newpath
\moveto(29.53538593,161.08876936)
\lineto(32.19038593,164.89876936)
\lineto(30.57038593,164.89876936)
\lineto(28.75538593,162.13876936)
\lineto(26.94038593,164.89876936)
\lineto(25.24538593,164.89876936)
\lineto(27.87038593,161.08876936)
\lineto(25.11038593,157.05376936)
\lineto(26.77538593,157.05376936)
\lineto(28.66538593,160.00876936)
\lineto(30.61538593,157.05376936)
\lineto(32.29538593,157.05376936)
\lineto(29.53538593,161.08876936)
}
}
{
\newrgbcolor{curcolor}{0 0 0}
\pscustom[linestyle=none,fillstyle=solid,fillcolor=curcolor]
{
\newpath
\moveto(34.77038593,164.89876936)
\lineto(33.45038593,164.89876936)
\lineto(33.45038593,157.05376936)
\lineto(34.77038593,157.05376936)
\lineto(34.77038593,164.89876936)
\moveto(34.77038593,166.32376936)
\lineto(34.77038593,167.82376936)
\lineto(33.45038593,167.82376936)
\lineto(33.45038593,166.32376936)
\lineto(34.77038593,166.32376936)
}
}
{
\newrgbcolor{curcolor}{0 0 0}
\pscustom[linestyle=none,fillstyle=solid,fillcolor=curcolor]
{
\newpath
\moveto(38.11022968,167.82376936)
\lineto(36.79022968,167.82376936)
\lineto(36.79022968,157.05376936)
\lineto(38.11022968,157.05376936)
\lineto(38.11022968,167.82376936)
}
}
{
\newrgbcolor{curcolor}{0 0 0}
\pscustom[linestyle=none,fillstyle=solid,fillcolor=curcolor]
{
\newpath
\moveto(41.45007343,164.89876936)
\lineto(40.13007343,164.89876936)
\lineto(40.13007343,157.05376936)
\lineto(41.45007343,157.05376936)
\lineto(41.45007343,164.89876936)
\moveto(41.45007343,166.32376936)
\lineto(41.45007343,167.82376936)
\lineto(40.13007343,167.82376936)
\lineto(40.13007343,166.32376936)
\lineto(41.45007343,166.32376936)
}
}
{
\newrgbcolor{curcolor}{0 0 0}
\pscustom[linestyle=none,fillstyle=solid,fillcolor=curcolor]
{
\newpath
\moveto(44.60991718,162.51376936)
\curveto(44.69991709,163.11376876)(44.90991868,164.02876936)(46.40991718,164.02876936)
\curveto(47.65491594,164.02876936)(48.25491718,163.57876853)(48.25491718,162.75376936)
\curveto(48.25491718,161.97377014)(47.87991687,161.85376933)(47.56491718,161.82376936)
\lineto(45.38991718,161.55376936)
\curveto(43.19991937,161.28376963)(43.00491718,159.7537687)(43.00491718,159.09376936)
\curveto(43.00491718,157.74377071)(44.02491862,156.82876936)(45.46491718,156.82876936)
\curveto(46.99491565,156.82876936)(47.78991769,157.54876991)(48.29991718,158.10376936)
\curveto(48.34491714,157.50376996)(48.52491835,156.90376936)(49.69491718,156.90376936)
\curveto(49.99491688,156.90376936)(50.18991741,156.99376942)(50.41491718,157.05376936)
\lineto(50.41491718,158.01376936)
\curveto(50.26491733,157.98376939)(50.09991706,157.95376936)(49.97991718,157.95376936)
\curveto(49.70991745,157.95376936)(49.54491718,158.08876969)(49.54491718,158.41876936)
\lineto(49.54491718,162.93376936)
\curveto(49.54491718,164.94376735)(47.26491655,165.12376936)(46.63491718,165.12376936)
\curveto(44.69991912,165.12376936)(43.45491712,164.38876748)(43.39491718,162.51376936)
\lineto(44.60991718,162.51376936)
\moveto(48.22491718,159.76876936)
\curveto(48.22491718,158.71877041)(47.02491595,157.92376936)(45.79491718,157.92376936)
\curveto(44.80491817,157.92376936)(44.36991718,158.43377021)(44.36991718,159.28876936)
\curveto(44.36991718,160.27876837)(45.40491783,160.47376945)(46.04991718,160.56376936)
\curveto(47.68491555,160.77376915)(48.01491739,160.89376952)(48.22491718,161.05876936)
\lineto(48.22491718,159.76876936)
}
}
{
\newrgbcolor{curcolor}{0 0 0}
\pscustom[linestyle=none,fillstyle=solid,fillcolor=curcolor]
{
\newpath
\moveto(53.28952656,161.61376936)
\curveto(53.28952656,162.75376822)(54.06952779,163.71376936)(55.29952656,163.71376936)
\lineto(55.79452656,163.71376936)
\lineto(55.79452656,165.07876936)
\curveto(55.68952666,165.10876933)(55.61452639,165.12376936)(55.44952656,165.12376936)
\curveto(54.45952755,165.12376936)(53.76952603,164.50876844)(53.24452656,163.59376936)
\lineto(53.21452656,163.59376936)
\lineto(53.21452656,164.89876936)
\lineto(51.96952656,164.89876936)
\lineto(51.96952656,157.05376936)
\lineto(53.28952656,157.05376936)
\lineto(53.28952656,161.61376936)
}
}
{
\newrgbcolor{curcolor}{0 0 0}
\pscustom[linestyle=none,fillstyle=solid,fillcolor=curcolor]
{
\newpath
\moveto(14.17069843,262.13005354)
\lineto(15.23569843,258.99505354)
\lineto(16.82569843,258.99505354)
\lineto(12.92569843,269.76505354)
\lineto(11.27569843,269.76505354)
\lineto(7.22569843,258.99505354)
\lineto(8.72569843,258.99505354)
\lineto(9.85069843,262.13005354)
\lineto(14.17069843,262.13005354)
\moveto(10.30069843,263.42005354)
\lineto(12.02569843,268.16005354)
\lineto(12.05569843,268.16005354)
\lineto(13.64569843,263.42005354)
\lineto(10.30069843,263.42005354)
}
}
{
\newrgbcolor{curcolor}{0 0 0}
\pscustom[linestyle=none,fillstyle=solid,fillcolor=curcolor]
{
\newpath
\moveto(23.93077656,258.99505354)
\lineto(23.93077656,266.84005354)
\lineto(22.61077656,266.84005354)
\lineto(22.61077656,262.52005354)
\curveto(22.61077656,261.38005468)(22.11577489,259.91005354)(20.45077656,259.91005354)
\curveto(19.59577741,259.91005354)(18.93577656,260.34505483)(18.93577656,261.63505354)
\lineto(18.93577656,266.84005354)
\lineto(17.61577656,266.84005354)
\lineto(17.61577656,261.20005354)
\curveto(17.61577656,259.32505541)(19.01077771,258.77005354)(20.16577656,258.77005354)
\curveto(21.4257753,258.77005354)(22.10077711,259.25005445)(22.65577656,260.16505354)
\lineto(22.68577656,260.13505354)
\lineto(22.68577656,258.99505354)
\lineto(23.93077656,258.99505354)
}
}
{
\newrgbcolor{curcolor}{0 0 0}
\pscustom[linestyle=none,fillstyle=solid,fillcolor=curcolor]
{
\newpath
\moveto(29.53538593,263.03005354)
\lineto(32.19038593,266.84005354)
\lineto(30.57038593,266.84005354)
\lineto(28.75538593,264.08005354)
\lineto(26.94038593,266.84005354)
\lineto(25.24538593,266.84005354)
\lineto(27.87038593,263.03005354)
\lineto(25.11038593,258.99505354)
\lineto(26.77538593,258.99505354)
\lineto(28.66538593,261.95005354)
\lineto(30.61538593,258.99505354)
\lineto(32.29538593,258.99505354)
\lineto(29.53538593,263.03005354)
}
}
{
\newrgbcolor{curcolor}{0 0 0}
\pscustom[linestyle=none,fillstyle=solid,fillcolor=curcolor]
{
\newpath
\moveto(34.77038593,266.84005354)
\lineto(33.45038593,266.84005354)
\lineto(33.45038593,258.99505354)
\lineto(34.77038593,258.99505354)
\lineto(34.77038593,266.84005354)
\moveto(34.77038593,268.26505354)
\lineto(34.77038593,269.76505354)
\lineto(33.45038593,269.76505354)
\lineto(33.45038593,268.26505354)
\lineto(34.77038593,268.26505354)
}
}
{
\newrgbcolor{curcolor}{0 0 0}
\pscustom[linestyle=none,fillstyle=solid,fillcolor=curcolor]
{
\newpath
\moveto(38.11022968,269.76505354)
\lineto(36.79022968,269.76505354)
\lineto(36.79022968,258.99505354)
\lineto(38.11022968,258.99505354)
\lineto(38.11022968,269.76505354)
}
}
{
\newrgbcolor{curcolor}{0 0 0}
\pscustom[linestyle=none,fillstyle=solid,fillcolor=curcolor]
{
\newpath
\moveto(41.45007343,266.84005354)
\lineto(40.13007343,266.84005354)
\lineto(40.13007343,258.99505354)
\lineto(41.45007343,258.99505354)
\lineto(41.45007343,266.84005354)
\moveto(41.45007343,268.26505354)
\lineto(41.45007343,269.76505354)
\lineto(40.13007343,269.76505354)
\lineto(40.13007343,268.26505354)
\lineto(41.45007343,268.26505354)
}
}
{
\newrgbcolor{curcolor}{0 0 0}
\pscustom[linestyle=none,fillstyle=solid,fillcolor=curcolor]
{
\newpath
\moveto(44.60991718,264.45505354)
\curveto(44.69991709,265.05505294)(44.90991868,265.97005354)(46.40991718,265.97005354)
\curveto(47.65491594,265.97005354)(48.25491718,265.52005271)(48.25491718,264.69505354)
\curveto(48.25491718,263.91505432)(47.87991687,263.79505351)(47.56491718,263.76505354)
\lineto(45.38991718,263.49505354)
\curveto(43.19991937,263.22505381)(43.00491718,261.69505288)(43.00491718,261.03505354)
\curveto(43.00491718,259.68505489)(44.02491862,258.77005354)(45.46491718,258.77005354)
\curveto(46.99491565,258.77005354)(47.78991769,259.49005409)(48.29991718,260.04505354)
\curveto(48.34491714,259.44505414)(48.52491835,258.84505354)(49.69491718,258.84505354)
\curveto(49.99491688,258.84505354)(50.18991741,258.9350536)(50.41491718,258.99505354)
\lineto(50.41491718,259.95505354)
\curveto(50.26491733,259.92505357)(50.09991706,259.89505354)(49.97991718,259.89505354)
\curveto(49.70991745,259.89505354)(49.54491718,260.03005387)(49.54491718,260.36005354)
\lineto(49.54491718,264.87505354)
\curveto(49.54491718,266.88505153)(47.26491655,267.06505354)(46.63491718,267.06505354)
\curveto(44.69991912,267.06505354)(43.45491712,266.33005166)(43.39491718,264.45505354)
\lineto(44.60991718,264.45505354)
\moveto(48.22491718,261.71005354)
\curveto(48.22491718,260.66005459)(47.02491595,259.86505354)(45.79491718,259.86505354)
\curveto(44.80491817,259.86505354)(44.36991718,260.37505439)(44.36991718,261.23005354)
\curveto(44.36991718,262.22005255)(45.40491783,262.41505363)(46.04991718,262.50505354)
\curveto(47.68491555,262.71505333)(48.01491739,262.8350537)(48.22491718,263.00005354)
\lineto(48.22491718,261.71005354)
}
}
{
\newrgbcolor{curcolor}{0 0 0}
\pscustom[linestyle=none,fillstyle=solid,fillcolor=curcolor]
{
\newpath
\moveto(53.28952656,263.55505354)
\curveto(53.28952656,264.6950524)(54.06952779,265.65505354)(55.29952656,265.65505354)
\lineto(55.79452656,265.65505354)
\lineto(55.79452656,267.02005354)
\curveto(55.68952666,267.05005351)(55.61452639,267.06505354)(55.44952656,267.06505354)
\curveto(54.45952755,267.06505354)(53.76952603,266.45005262)(53.24452656,265.53505354)
\lineto(53.21452656,265.53505354)
\lineto(53.21452656,266.84005354)
\lineto(51.96952656,266.84005354)
\lineto(51.96952656,258.99505354)
\lineto(53.28952656,258.99505354)
\lineto(53.28952656,263.55505354)
}
}
{
\newrgbcolor{curcolor}{0 0 0}
\pscustom[linestyle=none,fillstyle=solid,fillcolor=curcolor]
{
\newpath
\moveto(17.41070942,292.97564436)
\lineto(17.41070942,303.74564436)
\lineto(15.34070942,303.74564436)
\lineto(12.28070942,294.64064436)
\lineto(12.25070942,294.64064436)
\lineto(9.17570942,303.74564436)
\lineto(7.09070942,303.74564436)
\lineto(7.09070942,292.97564436)
\lineto(8.50070942,292.97564436)
\lineto(8.50070942,299.33564436)
\curveto(8.50070942,299.65064404)(8.47070942,301.01564535)(8.47070942,302.00564436)
\lineto(8.50070942,302.00564436)
\lineto(11.53070942,292.97564436)
\lineto(12.97070942,292.97564436)
\lineto(16.00070942,302.02064436)
\lineto(16.03070942,302.02064436)
\curveto(16.03070942,301.01564536)(16.00070942,299.65064404)(16.00070942,299.33564436)
\lineto(16.00070942,292.97564436)
\lineto(17.41070942,292.97564436)
}
}
{
\newrgbcolor{curcolor}{0 0 0}
\pscustom[linestyle=none,fillstyle=solid,fillcolor=curcolor]
{
\newpath
\moveto(19.03047504,296.90564436)
\curveto(19.03047504,294.88064638)(20.17047755,292.76564436)(22.67547504,292.76564436)
\curveto(25.18047254,292.76564436)(26.32047504,294.88064638)(26.32047504,296.90564436)
\curveto(26.32047504,298.93064233)(25.18047254,301.04564436)(22.67547504,301.04564436)
\curveto(20.17047755,301.04564436)(19.03047504,298.93064233)(19.03047504,296.90564436)
\moveto(20.39547504,296.90564436)
\curveto(20.39547504,297.95564331)(20.78547693,299.90564436)(22.67547504,299.90564436)
\curveto(24.56547315,299.90564436)(24.95547504,297.95564331)(24.95547504,296.90564436)
\curveto(24.95547504,295.85564541)(24.56547315,293.90564436)(22.67547504,293.90564436)
\curveto(20.78547693,293.90564436)(20.39547504,295.85564541)(20.39547504,296.90564436)
}
}
{
\newrgbcolor{curcolor}{0 0 0}
\pscustom[linestyle=none,fillstyle=solid,fillcolor=curcolor]
{
\newpath
\moveto(34.34008442,303.74564436)
\lineto(33.02008442,303.74564436)
\lineto(33.02008442,299.81564436)
\lineto(32.99008442,299.71064436)
\curveto(32.67508473,300.16064391)(32.07508299,301.04564436)(30.65008442,301.04564436)
\curveto(28.5650865,301.04564436)(27.38008442,299.33564215)(27.38008442,297.13064436)
\curveto(27.38008442,295.25564623)(28.16008709,292.75064436)(30.83008442,292.75064436)
\curveto(31.59508365,292.75064436)(32.49508499,292.99064542)(33.06508442,294.05564436)
\lineto(33.09508442,294.05564436)
\lineto(33.09508442,292.97564436)
\lineto(34.34008442,292.97564436)
\lineto(34.34008442,303.74564436)
\moveto(28.74508442,296.92064436)
\curveto(28.74508442,297.92564335)(28.85008646,299.86064436)(30.89008442,299.86064436)
\curveto(32.79508251,299.86064436)(33.00508442,297.80564308)(33.00508442,296.53064436)
\curveto(33.00508442,294.44564644)(31.70008358,293.89064436)(30.86008442,293.89064436)
\curveto(29.42008586,293.89064436)(28.74508442,295.19564608)(28.74508442,296.92064436)
}
}
{
\newrgbcolor{curcolor}{0 0 0}
\pscustom[linestyle=none,fillstyle=solid,fillcolor=curcolor]
{
\newpath
\moveto(41.51969379,295.43564436)
\curveto(41.47469384,294.85064494)(40.73969255,293.89064436)(39.49469379,293.89064436)
\curveto(37.97969531,293.89064436)(37.21469379,294.83564599)(37.21469379,296.47064436)
\lineto(42.94469379,296.47064436)
\curveto(42.94469379,299.24564158)(41.83469153,301.04564436)(39.56969379,301.04564436)
\curveto(36.97469639,301.04564436)(35.80469379,299.11064193)(35.80469379,296.68064436)
\curveto(35.80469379,294.41564662)(37.109696,292.75064436)(39.31469379,292.75064436)
\curveto(40.57469253,292.75064436)(41.08469415,293.0506446)(41.44469379,293.29064436)
\curveto(42.4346928,293.9506437)(42.79469384,295.06064473)(42.83969379,295.43564436)
\lineto(41.51969379,295.43564436)
\moveto(37.21469379,297.52064436)
\curveto(37.21469379,298.73564314)(38.17469501,299.86064436)(39.38969379,299.86064436)
\curveto(40.99469219,299.86064436)(41.50469387,298.73564314)(41.57969379,297.52064436)
\lineto(37.21469379,297.52064436)
}
}
{
\newrgbcolor{curcolor}{0 0 0}
\pscustom[linestyle=none,fillstyle=solid,fillcolor=curcolor]
{
\newpath
\moveto(46.02930317,297.53564436)
\curveto(46.02930317,298.67564322)(46.8093044,299.63564436)(48.03930317,299.63564436)
\lineto(48.53430317,299.63564436)
\lineto(48.53430317,301.00064436)
\curveto(48.42930327,301.03064433)(48.354303,301.04564436)(48.18930317,301.04564436)
\curveto(47.19930416,301.04564436)(46.50930264,300.43064344)(45.98430317,299.51564436)
\lineto(45.95430317,299.51564436)
\lineto(45.95430317,300.82064436)
\lineto(44.70930317,300.82064436)
\lineto(44.70930317,292.97564436)
\lineto(46.02930317,292.97564436)
\lineto(46.02930317,297.53564436)
}
}
{
\newrgbcolor{curcolor}{0 0 0}
\pscustom[linestyle=none,fillstyle=solid,fillcolor=curcolor]
{
\newpath
\moveto(50.56258442,298.43564436)
\curveto(50.65258433,299.03564376)(50.86258592,299.95064436)(52.36258442,299.95064436)
\curveto(53.60758317,299.95064436)(54.20758442,299.50064353)(54.20758442,298.67564436)
\curveto(54.20758442,297.89564514)(53.8325841,297.77564433)(53.51758442,297.74564436)
\lineto(51.34258442,297.47564436)
\curveto(49.15258661,297.20564463)(48.95758442,295.6756437)(48.95758442,295.01564436)
\curveto(48.95758442,293.66564571)(49.97758586,292.75064436)(51.41758442,292.75064436)
\curveto(52.94758289,292.75064436)(53.74258493,293.47064491)(54.25258442,294.02564436)
\curveto(54.29758437,293.42564496)(54.47758559,292.82564436)(55.64758442,292.82564436)
\curveto(55.94758412,292.82564436)(56.14258464,292.91564442)(56.36758442,292.97564436)
\lineto(56.36758442,293.93564436)
\curveto(56.21758457,293.90564439)(56.0525843,293.87564436)(55.93258442,293.87564436)
\curveto(55.66258469,293.87564436)(55.49758442,294.01064469)(55.49758442,294.34064436)
\lineto(55.49758442,298.85564436)
\curveto(55.49758442,300.86564235)(53.21758379,301.04564436)(52.58758442,301.04564436)
\curveto(50.65258635,301.04564436)(49.40758436,300.31064248)(49.34758442,298.43564436)
\lineto(50.56258442,298.43564436)
\moveto(54.17758442,295.69064436)
\curveto(54.17758442,294.64064541)(52.97758319,293.84564436)(51.74758442,293.84564436)
\curveto(50.75758541,293.84564436)(50.32258442,294.35564521)(50.32258442,295.21064436)
\curveto(50.32258442,296.20064337)(51.35758506,296.39564445)(52.00258442,296.48564436)
\curveto(53.63758278,296.69564415)(53.96758463,296.81564452)(54.17758442,296.98064436)
\lineto(54.17758442,295.69064436)
}
}
{
\newrgbcolor{curcolor}{0 0 0}
\pscustom[linestyle=none,fillstyle=solid,fillcolor=curcolor]
{
\newpath
\moveto(64.25219379,303.74564436)
\lineto(62.93219379,303.74564436)
\lineto(62.93219379,299.81564436)
\lineto(62.90219379,299.71064436)
\curveto(62.58719411,300.16064391)(61.98719237,301.04564436)(60.56219379,301.04564436)
\curveto(58.47719588,301.04564436)(57.29219379,299.33564215)(57.29219379,297.13064436)
\curveto(57.29219379,295.25564623)(58.07219646,292.75064436)(60.74219379,292.75064436)
\curveto(61.50719303,292.75064436)(62.40719436,292.99064542)(62.97719379,294.05564436)
\lineto(63.00719379,294.05564436)
\lineto(63.00719379,292.97564436)
\lineto(64.25219379,292.97564436)
\lineto(64.25219379,303.74564436)
\moveto(58.65719379,296.92064436)
\curveto(58.65719379,297.92564335)(58.76219583,299.86064436)(60.80219379,299.86064436)
\curveto(62.70719189,299.86064436)(62.91719379,297.80564308)(62.91719379,296.53064436)
\curveto(62.91719379,294.44564644)(61.61219295,293.89064436)(60.77219379,293.89064436)
\curveto(59.33219523,293.89064436)(58.65719379,295.19564608)(58.65719379,296.92064436)
}
}
{
\newrgbcolor{curcolor}{0 0 0}
\pscustom[linestyle=none,fillstyle=solid,fillcolor=curcolor]
{
\newpath
\moveto(65.64180317,296.90564436)
\curveto(65.64180317,294.88064638)(66.78180567,292.76564436)(69.28680317,292.76564436)
\curveto(71.79180066,292.76564436)(72.93180317,294.88064638)(72.93180317,296.90564436)
\curveto(72.93180317,298.93064233)(71.79180066,301.04564436)(69.28680317,301.04564436)
\curveto(66.78180567,301.04564436)(65.64180317,298.93064233)(65.64180317,296.90564436)
\moveto(67.00680317,296.90564436)
\curveto(67.00680317,297.95564331)(67.39680506,299.90564436)(69.28680317,299.90564436)
\curveto(71.17680128,299.90564436)(71.56680317,297.95564331)(71.56680317,296.90564436)
\curveto(71.56680317,295.85564541)(71.17680128,293.90564436)(69.28680317,293.90564436)
\curveto(67.39680506,293.90564436)(67.00680317,295.85564541)(67.00680317,296.90564436)
}
}
{
\newrgbcolor{curcolor}{0 0 0}
\pscustom[linestyle=none,fillstyle=solid,fillcolor=curcolor]
{
\newpath
\moveto(75.94141254,297.53564436)
\curveto(75.94141254,298.67564322)(76.72141377,299.63564436)(77.95141254,299.63564436)
\lineto(78.44641254,299.63564436)
\lineto(78.44641254,301.00064436)
\curveto(78.34141265,301.03064433)(78.26641238,301.04564436)(78.10141254,301.04564436)
\curveto(77.11141353,301.04564436)(76.42141202,300.43064344)(75.89641254,299.51564436)
\lineto(75.86641254,299.51564436)
\lineto(75.86641254,300.82064436)
\lineto(74.62141254,300.82064436)
\lineto(74.62141254,292.97564436)
\lineto(75.94141254,292.97564436)
\lineto(75.94141254,297.53564436)
}
}
{
\newrgbcolor{curcolor}{0 0 0}
\pscustom[linestyle=none,fillstyle=solid,fillcolor=curcolor]
{
\newpath
\moveto(111.68000793,541.71457502)
\lineto(113.00000793,541.71457502)
\lineto(113.00000793,534.18457502)
\curveto(113.00000586,533.39457263)(113.13500573,532.80957321)(113.40500793,532.42957502)
\curveto(113.68500518,532.05957396)(114.15500471,531.87457415)(114.81500793,531.87457502)
\curveto(115.51500335,531.87457415)(115.99500287,532.06957395)(116.25500793,532.45957502)
\curveto(116.51500235,532.85957316)(116.64500222,533.43457259)(116.64500793,534.18457502)
\lineto(116.64500793,541.71457502)
\lineto(117.96500793,541.71457502)
\lineto(117.96500793,534.18457502)
\curveto(117.9650009,533.15457287)(117.70000116,532.32957369)(117.17000793,531.70957502)
\curveto(116.65000221,531.09957492)(115.865003,530.79457523)(114.81500793,530.79457502)
\curveto(113.72500514,530.79457523)(112.93000593,531.07957494)(112.43000793,531.64957502)
\curveto(111.93000693,532.2195738)(111.68000718,533.06457296)(111.68000793,534.18457502)
\lineto(111.68000793,541.71457502)
}
}
{
\newrgbcolor{curcolor}{0 0 0}
\pscustom[linestyle=none,fillstyle=solid,fillcolor=curcolor]
{
\newpath
\moveto(119.59297668,541.71457502)
\lineto(120.89797668,541.71457502)
\lineto(121.30297668,541.71457502)
\lineto(124.84297668,532.77457502)
\lineto(124.87297668,532.77457502)
\lineto(124.87297668,541.71457502)
\lineto(126.19297668,541.71457502)
\lineto(126.19297668,531.00457502)
\lineto(124.88797668,531.00457502)
\lineto(124.37797668,531.00457502)
\lineto(120.94297668,539.67457502)
\lineto(120.91297668,539.67457502)
\lineto(120.91297668,531.00457502)
\lineto(119.59297668,531.00457502)
\lineto(119.59297668,541.71457502)
}
}
{
\newrgbcolor{curcolor}{0 0 0}
\pscustom[linestyle=none,fillstyle=solid,fillcolor=curcolor]
{
\newpath
\moveto(144.96510803,541.50455305)
\lineto(150.50010803,541.50455305)
\lineto(150.50010803,540.33455305)
\lineto(146.28510803,540.33455305)
\lineto(146.28510803,536.97455305)
\lineto(150.26010803,536.97455305)
\lineto(150.26010803,535.80455305)
\lineto(146.28510803,535.80455305)
\lineto(146.28510803,531.96455305)
\lineto(150.68010803,531.96455305)
\lineto(150.68010803,530.79455305)
\lineto(144.96510803,530.79455305)
\lineto(144.96510803,541.50455305)
}
}
{
\newrgbcolor{curcolor}{0 0 0}
\pscustom[linestyle=none,fillstyle=solid,fillcolor=curcolor]
{
\newpath
\moveto(151.93846741,541.50455305)
\lineto(154.15846741,541.50455305)
\lineto(156.30346741,533.01455305)
\lineto(156.33346741,533.01455305)
\lineto(158.47846741,541.50455305)
\lineto(160.69846741,541.50455305)
\lineto(160.69846741,530.79455305)
\lineto(159.37846741,530.79455305)
\lineto(159.37846741,540.15455305)
\lineto(159.34846741,540.15455305)
\lineto(156.97846741,530.79455305)
\lineto(155.65846741,530.79455305)
\lineto(153.28846741,540.15455305)
\lineto(153.25846741,540.15455305)
\lineto(153.25846741,530.79455305)
\lineto(151.93846741,530.79455305)
\lineto(151.93846741,541.50455305)
}
}
{
\newrgbcolor{curcolor}{0 0 0}
\pscustom[linestyle=none,fillstyle=solid,fillcolor=curcolor]
{
\newpath
\moveto(181.9155304,534.81455305)
\lineto(184.7955304,534.81455305)
\lineto(183.4305304,539.98955305)
\lineto(183.4005304,539.98955305)
\lineto(181.9155304,534.81455305)
\moveto(182.5605304,541.50455305)
\lineto(184.3305304,541.50455305)
\lineto(187.2105304,530.79455305)
\lineto(185.8305304,530.79455305)
\lineto(185.0655304,533.73455305)
\lineto(181.6455304,533.73455305)
\lineto(180.8505304,530.79455305)
\lineto(179.4705304,530.79455305)
\lineto(182.5605304,541.50455305)
}
}
{
\newrgbcolor{curcolor}{0 0 0}
\pscustom[linestyle=none,fillstyle=solid,fillcolor=curcolor]
{
\newpath
\moveto(186.76756165,541.50455305)
\lineto(188.14756165,541.50455305)
\lineto(190.24756165,532.30955305)
\lineto(190.27756165,532.30955305)
\lineto(192.37756165,541.50455305)
\lineto(193.75756165,541.50455305)
\lineto(191.05756165,530.79455305)
\lineto(189.37756165,530.79455305)
\lineto(186.76756165,541.50455305)
}
}
{
\newrgbcolor{curcolor}{0 0 0}
\pscustom[linestyle=none,fillstyle=solid,fillcolor=curcolor]
{
\newpath
\moveto(214.97467346,535.02457502)
\lineto(217.85467346,535.02457502)
\lineto(216.48967346,540.19957502)
\lineto(216.45967346,540.19957502)
\lineto(214.97467346,535.02457502)
\moveto(215.61967346,541.71457502)
\lineto(217.38967346,541.71457502)
\lineto(220.26967346,531.00457502)
\lineto(218.88967346,531.00457502)
\lineto(218.12467346,533.94457502)
\lineto(214.70467346,533.94457502)
\lineto(213.90967346,531.00457502)
\lineto(212.52967346,531.00457502)
\lineto(215.61967346,541.71457502)
}
}
{
\newrgbcolor{curcolor}{0 0 0}
\pscustom[linestyle=none,fillstyle=solid,fillcolor=curcolor]
{
\newpath
\moveto(227.57467346,534.67957502)
\curveto(227.54466607,534.16957185)(227.46466615,533.67957234)(227.33467346,533.20957502)
\curveto(227.2146664,532.74957327)(227.02966658,532.33957368)(226.77967346,531.97957502)
\curveto(226.52966708,531.6195744)(226.19966741,531.32957469)(225.78967346,531.10957502)
\curveto(225.38966822,530.89957512)(224.89966871,530.79457523)(224.31967346,530.79457502)
\curveto(223.55967005,530.79457523)(222.94967066,530.95457507)(222.48967346,531.27457502)
\curveto(222.03967157,531.60457442)(221.69467192,532.03457399)(221.45467346,532.56457502)
\curveto(221.2146724,533.09457293)(221.05467256,533.68957233)(220.97467346,534.34957502)
\curveto(220.90467271,535.00957101)(220.86967274,535.67957034)(220.86967346,536.35957502)
\curveto(220.86967274,537.02956899)(220.9096727,537.69456833)(220.98967346,538.35457502)
\curveto(221.06967254,539.024567)(221.23467238,539.6245664)(221.48467346,540.15457502)
\curveto(221.73467188,540.68456534)(222.08467153,541.10956491)(222.53467346,541.42957502)
\curveto(222.98467063,541.75956426)(223.57967003,541.9245641)(224.31967346,541.92457502)
\curveto(225.4096682,541.9245641)(226.19966741,541.63456439)(226.68967346,541.05457502)
\curveto(227.18966642,540.47456555)(227.45466616,539.65456637)(227.48467346,538.59457502)
\lineto(226.10467346,538.59457502)
\curveto(226.09466752,538.89456713)(226.05966755,539.17956684)(225.99967346,539.44957502)
\curveto(225.93966767,539.72956629)(225.83966777,539.96956605)(225.69967346,540.16957502)
\curveto(225.56966804,540.37956564)(225.38966822,540.54456548)(225.15967346,540.66457502)
\curveto(224.93966867,540.78456524)(224.65966895,540.84456518)(224.31967346,540.84457502)
\curveto(223.85966975,540.84456518)(223.49467012,540.7245653)(223.22467346,540.48457502)
\curveto(222.95467066,540.25456577)(222.74467087,539.93456609)(222.59467346,539.52457502)
\curveto(222.45467116,539.1245669)(222.35967125,538.64956737)(222.30967346,538.09957502)
\curveto(222.26967134,537.55956846)(222.24967136,536.97956904)(222.24967346,536.35957502)
\curveto(222.24967136,535.73957028)(222.26967134,535.15457087)(222.30967346,534.60457502)
\curveto(222.35967125,534.06457196)(222.45467116,533.58957243)(222.59467346,533.17957502)
\curveto(222.74467087,532.77957324)(222.95467066,532.45957356)(223.22467346,532.21957502)
\curveto(223.49467012,531.98957403)(223.85966975,531.87457415)(224.31967346,531.87457502)
\curveto(224.71966889,531.87457415)(225.03966857,531.95957406)(225.27967346,532.12957502)
\curveto(225.51966809,532.29957372)(225.70466791,532.5195735)(225.83467346,532.78957502)
\curveto(225.97466764,533.05957296)(226.06466755,533.35957266)(226.10467346,533.68957502)
\curveto(226.15466746,534.019572)(226.18466743,534.34957167)(226.19467346,534.67957502)
\lineto(227.57467346,534.67957502)
}
}
{
\newrgbcolor{curcolor}{0 0 0}
\pscustom[linestyle=none,fillstyle=solid,fillcolor=curcolor]
{
\newpath
\moveto(247.66676086,536.40455305)
\lineto(249.24176086,536.40455305)
\curveto(249.72175662,536.40454744)(250.1367562,536.57454727)(250.48676086,536.91455305)
\curveto(250.8367555,537.25454659)(251.01175533,537.77954606)(251.01176086,538.48955305)
\curveto(251.01175533,539.08954475)(250.86675547,539.55954428)(250.57676086,539.89955305)
\curveto(250.28675605,540.24954359)(249.81175653,540.42454342)(249.15176086,540.42455305)
\lineto(247.66676086,540.42455305)
\lineto(247.66676086,536.40455305)
\moveto(246.34676086,541.50455305)
\lineto(249.07676086,541.50455305)
\curveto(249.22675711,541.50454234)(249.41175693,541.49954234)(249.63176086,541.48955305)
\curveto(249.86175648,541.47954236)(250.09675624,541.4445424)(250.33676086,541.38455305)
\curveto(250.58675575,541.33454251)(250.83175551,541.2445426)(251.07176086,541.11455305)
\curveto(251.32175502,540.99454285)(251.5417548,540.81954302)(251.73176086,540.58955305)
\curveto(251.93175441,540.35954348)(252.09175425,540.06954377)(252.21176086,539.71955305)
\curveto(252.33175401,539.36954447)(252.39175395,538.9395449)(252.39176086,538.42955305)
\curveto(252.39175395,537.92954591)(252.31675402,537.48454636)(252.16676086,537.09455305)
\curveto(252.01675432,536.71454713)(251.80175454,536.38954745)(251.52176086,536.11955305)
\curveto(251.25175509,535.85954798)(250.92675541,535.65954818)(250.54676086,535.51955305)
\curveto(250.16675617,535.38954845)(249.75175659,535.32454852)(249.30176086,535.32455305)
\lineto(247.66676086,535.32455305)
\lineto(247.66676086,530.79455305)
\lineto(246.34676086,530.79455305)
\lineto(246.34676086,541.50455305)
}
}
{
\newrgbcolor{curcolor}{0 0 0}
\pscustom[linestyle=none,fillstyle=solid,fillcolor=curcolor]
{
\newpath
\moveto(254.46679993,534.81455305)
\lineto(257.34679993,534.81455305)
\lineto(255.98179993,539.98955305)
\lineto(255.95179993,539.98955305)
\lineto(254.46679993,534.81455305)
\moveto(255.11179993,541.50455305)
\lineto(256.88179993,541.50455305)
\lineto(259.76179993,530.79455305)
\lineto(258.38179993,530.79455305)
\lineto(257.61679993,533.73455305)
\lineto(254.19679993,533.73455305)
\lineto(253.40179993,530.79455305)
\lineto(252.02179993,530.79455305)
\lineto(255.11179993,541.50455305)
}
}
{
\newrgbcolor{curcolor}{0 0 0}
\pscustom[linestyle=none,fillstyle=solid,fillcolor=curcolor]
{
\newpath
\moveto(279.85390991,536.61457502)
\lineto(281.42890991,536.61457502)
\curveto(281.90890567,536.61456941)(282.32390525,536.78456924)(282.67390991,537.12457502)
\curveto(283.02390455,537.46456856)(283.19890438,537.98956803)(283.19890991,538.69957502)
\curveto(283.19890438,539.29956672)(283.05390452,539.76956625)(282.76390991,540.10957502)
\curveto(282.4739051,540.45956556)(281.99890558,540.63456539)(281.33890991,540.63457502)
\lineto(279.85390991,540.63457502)
\lineto(279.85390991,536.61457502)
\moveto(278.53390991,541.71457502)
\lineto(281.26390991,541.71457502)
\curveto(281.41390616,541.71456431)(281.59890598,541.70956431)(281.81890991,541.69957502)
\curveto(282.04890553,541.68956433)(282.28390529,541.65456437)(282.52390991,541.59457502)
\curveto(282.7739048,541.54456448)(283.01890456,541.45456457)(283.25890991,541.32457502)
\curveto(283.50890407,541.20456482)(283.72890385,541.02956499)(283.91890991,540.79957502)
\curveto(284.11890346,540.56956545)(284.2789033,540.27956574)(284.39890991,539.92957502)
\curveto(284.51890306,539.57956644)(284.578903,539.14956687)(284.57890991,538.63957502)
\curveto(284.578903,538.13956788)(284.50390307,537.69456833)(284.35390991,537.30457502)
\curveto(284.20390337,536.9245691)(283.98890359,536.59956942)(283.70890991,536.32957502)
\curveto(283.43890414,536.06956995)(283.11390446,535.86957015)(282.73390991,535.72957502)
\curveto(282.35390522,535.59957042)(281.93890564,535.53457049)(281.48890991,535.53457502)
\lineto(279.85390991,535.53457502)
\lineto(279.85390991,531.00457502)
\lineto(278.53390991,531.00457502)
\lineto(278.53390991,541.71457502)
}
}
{
\newrgbcolor{curcolor}{0 0 0}
\pscustom[linestyle=none,fillstyle=solid,fillcolor=curcolor]
{
\newpath
\moveto(289.04094116,540.84457502)
\curveto(288.58093745,540.84456518)(288.21593782,540.7245653)(287.94594116,540.48457502)
\curveto(287.67593836,540.25456577)(287.46593857,539.93456609)(287.31594116,539.52457502)
\curveto(287.17593886,539.1245669)(287.08093895,538.64956737)(287.03094116,538.09957502)
\curveto(286.99093904,537.55956846)(286.97093906,536.97956904)(286.97094116,536.35957502)
\curveto(286.97093906,535.73957028)(286.99093904,535.15457087)(287.03094116,534.60457502)
\curveto(287.08093895,534.06457196)(287.17593886,533.58957243)(287.31594116,533.17957502)
\curveto(287.46593857,532.77957324)(287.67593836,532.45957356)(287.94594116,532.21957502)
\curveto(288.21593782,531.98957403)(288.58093745,531.87457415)(289.04094116,531.87457502)
\curveto(289.50093653,531.87457415)(289.86593617,531.98957403)(290.13594116,532.21957502)
\curveto(290.40593563,532.45957356)(290.61093542,532.77957324)(290.75094116,533.17957502)
\curveto(290.90093513,533.58957243)(290.99593504,534.06457196)(291.03594116,534.60457502)
\curveto(291.08593495,535.15457087)(291.11093492,535.73957028)(291.11094116,536.35957502)
\curveto(291.11093492,536.97956904)(291.08593495,537.55956846)(291.03594116,538.09957502)
\curveto(290.99593504,538.64956737)(290.90093513,539.1245669)(290.75094116,539.52457502)
\curveto(290.61093542,539.93456609)(290.40593563,540.25456577)(290.13594116,540.48457502)
\curveto(289.86593617,540.7245653)(289.50093653,540.84456518)(289.04094116,540.84457502)
\moveto(289.04094116,541.92457502)
\curveto(289.78093625,541.9245641)(290.37593566,541.75956426)(290.82594116,541.42957502)
\curveto(291.27593476,541.10956491)(291.62593441,540.68456534)(291.87594116,540.15457502)
\curveto(292.12593391,539.6245664)(292.29093374,539.024567)(292.37094116,538.35457502)
\curveto(292.45093358,537.69456833)(292.49093354,537.02956899)(292.49094116,536.35957502)
\curveto(292.49093354,535.67957034)(292.45093358,535.00957101)(292.37094116,534.34957502)
\curveto(292.29093374,533.68957233)(292.12593391,533.09457293)(291.87594116,532.56457502)
\curveto(291.62593441,532.03457399)(291.27593476,531.60457442)(290.82594116,531.27457502)
\curveto(290.37593566,530.95457507)(289.78093625,530.79457523)(289.04094116,530.79457502)
\curveto(288.30093773,530.79457523)(287.70593833,530.95457507)(287.25594116,531.27457502)
\curveto(286.80593923,531.60457442)(286.45593958,532.03457399)(286.20594116,532.56457502)
\curveto(285.95594008,533.09457293)(285.79094024,533.68957233)(285.71094116,534.34957502)
\curveto(285.6309404,535.00957101)(285.59094044,535.67957034)(285.59094116,536.35957502)
\curveto(285.59094044,537.02956899)(285.6309404,537.69456833)(285.71094116,538.35457502)
\curveto(285.79094024,539.024567)(285.95594008,539.6245664)(286.20594116,540.15457502)
\curveto(286.45593958,540.68456534)(286.80593923,541.10956491)(287.25594116,541.42957502)
\curveto(287.70593833,541.75956426)(288.30093773,541.9245641)(289.04094116,541.92457502)
}
}
{
\newrgbcolor{curcolor}{0 0 0}
\pscustom[linestyle=none,fillstyle=solid,fillcolor=curcolor]
{
\newpath
\moveto(316.21300415,538.93957502)
\curveto(316.21299874,539.2195668)(316.18299877,539.47456655)(316.12300415,539.70457502)
\curveto(316.07299888,539.94456608)(315.98299897,540.14456588)(315.85300415,540.30457502)
\curveto(315.72299923,540.47456555)(315.5479994,540.60456542)(315.32800415,540.69457502)
\curveto(315.11799983,540.79456523)(314.85800009,540.84456518)(314.54800415,540.84457502)
\curveto(313.97800097,540.84456518)(313.53800141,540.69456533)(313.22800415,540.39457502)
\curveto(312.92800202,540.10456592)(312.77800217,539.67456635)(312.77800415,539.10457502)
\curveto(312.77800217,538.60456742)(312.90300205,538.2195678)(313.15300415,537.94957502)
\curveto(313.40300155,537.67956834)(313.71300124,537.46456856)(314.08300415,537.30457502)
\curveto(314.46300049,537.14456888)(314.86800008,536.99956902)(315.29800415,536.86957502)
\curveto(315.73799921,536.74956927)(316.14299881,536.58456944)(316.51300415,536.37457502)
\curveto(316.89299806,536.16456986)(317.20799774,535.87457015)(317.45800415,535.50457502)
\curveto(317.70799724,535.13457089)(317.83299712,534.6245714)(317.83300415,533.97457502)
\curveto(317.83299712,533.35457267)(317.72799722,532.83957318)(317.51800415,532.42957502)
\curveto(317.31799763,532.019574)(317.05799789,531.69457433)(316.73800415,531.45457502)
\curveto(316.41799853,531.21457481)(316.05799889,531.04457498)(315.65800415,530.94457502)
\curveto(315.26799968,530.84457518)(314.88300007,530.79457523)(314.50300415,530.79457502)
\curveto(313.87300108,530.79457523)(313.3480016,530.87457515)(312.92800415,531.03457502)
\curveto(312.51800243,531.19457483)(312.18800276,531.4245746)(311.93800415,531.72457502)
\curveto(311.69800325,532.024574)(311.52300343,532.39457363)(311.41300415,532.83457502)
\curveto(311.31300364,533.28457274)(311.26300369,533.79457223)(311.26300415,534.36457502)
\lineto(312.58300415,534.36457502)
\curveto(312.58300237,534.06457196)(312.59800235,533.76457226)(312.62800415,533.46457502)
\curveto(312.65800229,533.17457285)(312.73800221,532.90957311)(312.86800415,532.66957502)
\curveto(312.99800195,532.42957359)(313.19800175,532.23457379)(313.46800415,532.08457502)
\curveto(313.73800121,531.94457408)(314.11300084,531.87457415)(314.59300415,531.87457502)
\curveto(314.8530001,531.87457415)(315.09799985,531.9195741)(315.32800415,532.00957502)
\curveto(315.55799939,532.09957392)(315.7529992,532.2245738)(315.91300415,532.38457502)
\curveto(316.08299887,532.55457347)(316.21299874,532.75457327)(316.30300415,532.98457502)
\curveto(316.40299855,533.21457281)(316.4529985,533.47457255)(316.45300415,533.76457502)
\curveto(316.4529985,534.14457188)(316.37799857,534.45457157)(316.22800415,534.69457502)
\curveto(316.08799886,534.93457109)(315.89799905,535.13457089)(315.65800415,535.29457502)
\curveto(315.42799952,535.46457056)(315.15799979,535.59957042)(314.84800415,535.69957502)
\curveto(314.5480004,535.80957021)(314.23800071,535.9195701)(313.91800415,536.02957502)
\curveto(313.60800134,536.13956988)(313.29800165,536.25956976)(312.98800415,536.38957502)
\curveto(312.68800226,536.52956949)(312.41800253,536.70456932)(312.17800415,536.91457502)
\curveto(311.948003,537.13456889)(311.75800319,537.40956861)(311.60800415,537.73957502)
\curveto(311.46800348,538.06956795)(311.39800355,538.47956754)(311.39800415,538.96957502)
\curveto(311.39800355,539.2195668)(311.43300352,539.50956651)(311.50300415,539.83957502)
\curveto(311.57300338,540.17956584)(311.71800323,540.50456552)(311.93800415,540.81457502)
\curveto(312.16800278,541.1245649)(312.48800246,541.38456464)(312.89800415,541.59457502)
\curveto(313.30800164,541.81456421)(313.8530011,541.9245641)(314.53300415,541.92457502)
\curveto(315.56299939,541.9245641)(316.31299864,541.67456435)(316.78300415,541.17457502)
\curveto(317.26299769,540.68456534)(317.51299744,539.93956608)(317.53300415,538.93957502)
\lineto(316.21300415,538.93957502)
}
}
{
\newrgbcolor{curcolor}{0 0 0}
\pscustom[linestyle=none,fillstyle=solid,fillcolor=curcolor]
{
\newpath
\moveto(322.46800415,540.84457502)
\curveto(322.00800044,540.84456518)(321.64300081,540.7245653)(321.37300415,540.48457502)
\curveto(321.10300135,540.25456577)(320.89300156,539.93456609)(320.74300415,539.52457502)
\curveto(320.60300185,539.1245669)(320.50800194,538.64956737)(320.45800415,538.09957502)
\curveto(320.41800203,537.55956846)(320.39800205,536.97956904)(320.39800415,536.35957502)
\curveto(320.39800205,535.73957028)(320.41800203,535.15457087)(320.45800415,534.60457502)
\curveto(320.50800194,534.06457196)(320.60300185,533.58957243)(320.74300415,533.17957502)
\curveto(320.89300156,532.77957324)(321.10300135,532.45957356)(321.37300415,532.21957502)
\curveto(321.64300081,531.98957403)(322.00800044,531.87457415)(322.46800415,531.87457502)
\curveto(322.92799952,531.87457415)(323.29299916,531.98957403)(323.56300415,532.21957502)
\curveto(323.83299862,532.45957356)(324.03799841,532.77957324)(324.17800415,533.17957502)
\curveto(324.32799812,533.58957243)(324.42299803,534.06457196)(324.46300415,534.60457502)
\curveto(324.51299794,535.15457087)(324.53799791,535.73957028)(324.53800415,536.35957502)
\curveto(324.53799791,536.97956904)(324.51299794,537.55956846)(324.46300415,538.09957502)
\curveto(324.42299803,538.64956737)(324.32799812,539.1245669)(324.17800415,539.52457502)
\curveto(324.03799841,539.93456609)(323.83299862,540.25456577)(323.56300415,540.48457502)
\curveto(323.29299916,540.7245653)(322.92799952,540.84456518)(322.46800415,540.84457502)
\moveto(322.46800415,541.92457502)
\curveto(323.20799924,541.9245641)(323.80299865,541.75956426)(324.25300415,541.42957502)
\curveto(324.70299775,541.10956491)(325.0529974,540.68456534)(325.30300415,540.15457502)
\curveto(325.5529969,539.6245664)(325.71799673,539.024567)(325.79800415,538.35457502)
\curveto(325.87799657,537.69456833)(325.91799653,537.02956899)(325.91800415,536.35957502)
\curveto(325.91799653,535.67957034)(325.87799657,535.00957101)(325.79800415,534.34957502)
\curveto(325.71799673,533.68957233)(325.5529969,533.09457293)(325.30300415,532.56457502)
\curveto(325.0529974,532.03457399)(324.70299775,531.60457442)(324.25300415,531.27457502)
\curveto(323.80299865,530.95457507)(323.20799924,530.79457523)(322.46800415,530.79457502)
\curveto(321.72800072,530.79457523)(321.13300132,530.95457507)(320.68300415,531.27457502)
\curveto(320.23300222,531.60457442)(319.88300257,532.03457399)(319.63300415,532.56457502)
\curveto(319.38300307,533.09457293)(319.21800323,533.68957233)(319.13800415,534.34957502)
\curveto(319.05800339,535.00957101)(319.01800343,535.67957034)(319.01800415,536.35957502)
\curveto(319.01800343,537.02956899)(319.05800339,537.69456833)(319.13800415,538.35457502)
\curveto(319.21800323,539.024567)(319.38300307,539.6245664)(319.63300415,540.15457502)
\curveto(319.88300257,540.68456534)(320.23300222,541.10956491)(320.68300415,541.42957502)
\curveto(321.13300132,541.75956426)(321.72800072,541.9245641)(322.46800415,541.92457502)
}
}
{
\newrgbcolor{curcolor}{0 0 0}
\pscustom[linestyle=none,fillstyle=solid,fillcolor=curcolor]
{
\newpath
\moveto(250.0216217,518.91546369)
\curveto(250.59583533,518.76311541)(251.03528802,518.49163131)(251.33998108,518.10101057)
\curveto(251.64466241,517.71428834)(251.79700601,517.22991382)(251.79701233,516.64788557)
\curveto(251.79700601,515.84319646)(251.5255219,515.21038459)(250.9825592,514.74944807)
\curveto(250.44349174,514.29241676)(249.69544561,514.06390136)(248.73841858,514.06390119)
\curveto(248.33607197,514.06390136)(247.92591613,514.1010107)(247.50794983,514.17522932)
\curveto(247.08997946,514.24944805)(246.67982362,514.35686982)(246.27748108,514.49749494)
\lineto(246.27748108,515.67522932)
\curveto(246.67591738,515.46819683)(247.06849511,515.31390011)(247.45521545,515.21233869)
\curveto(247.84193184,515.11077531)(248.22669708,515.05999412)(248.60951233,515.05999494)
\curveto(249.25794605,515.05999412)(249.75599242,515.20647834)(250.10365295,515.49944807)
\curveto(250.45130423,515.79241526)(250.62513218,516.21428984)(250.62513733,516.76507307)
\curveto(250.62513218,517.27288253)(250.45130423,517.67522588)(250.10365295,517.97210432)
\curveto(249.75599242,518.27288153)(249.28528977,518.423272)(248.69154358,518.42327619)
\lineto(247.78919983,518.42327619)
\lineto(247.78919983,519.39593244)
\lineto(248.69154358,519.39593244)
\curveto(249.23450857,519.39592728)(249.65833627,519.51506779)(249.96302795,519.75335432)
\curveto(250.26771066,519.99162981)(250.42005426,520.32366073)(250.4200592,520.74944807)
\curveto(250.42005426,521.19865985)(250.27747628,521.54240951)(249.99232483,521.78069807)
\curveto(249.71107059,522.02287778)(249.30872725,522.14397141)(248.78529358,522.14397932)
\curveto(248.43763437,522.14397141)(248.07825973,522.10490895)(247.70716858,522.02679182)
\curveto(247.33607297,521.9486591)(246.94740148,521.83147172)(246.54115295,521.67522932)
\lineto(246.54115295,522.76507307)
\curveto(247.01380767,522.89006441)(247.43372912,522.98381432)(247.80091858,523.04632307)
\curveto(248.17200963,523.10881419)(248.5001343,523.14006416)(248.78529358,523.14007307)
\curveto(249.63685192,523.14006416)(250.31653874,522.92522063)(250.82435608,522.49554182)
\curveto(251.33606897,522.06975273)(251.59192809,521.50334705)(251.5919342,520.79632307)
\curveto(251.59192809,520.31584823)(251.4571626,519.91545801)(251.18763733,519.59515119)
\curveto(250.92200688,519.27483365)(250.5333354,519.04827138)(250.0216217,518.91546369)
}
}
{
\newrgbcolor{curcolor}{0 0 0}
\pscustom[linestyle=none,fillstyle=solid,fillcolor=curcolor]
{
\newpath
\moveto(255.52357483,518.62835432)
\curveto(255.523572,518.84319346)(255.59779067,519.02874015)(255.74623108,519.18499494)
\curveto(255.89857162,519.34123983)(256.08021207,519.41936476)(256.29115295,519.41936994)
\curveto(256.50989914,519.41936476)(256.69739895,519.34123983)(256.85365295,519.18499494)
\curveto(257.00989864,519.02874015)(257.08802356,518.84319346)(257.08802795,518.62835432)
\curveto(257.08802356,518.40960014)(257.00989864,518.22405345)(256.85365295,518.07171369)
\curveto(256.7013052,517.91936626)(256.51380538,517.84319446)(256.29115295,517.84319807)
\curveto(256.07239958,517.84319446)(255.88880601,517.91741313)(255.7403717,518.06585432)
\curveto(255.59583755,518.21428784)(255.523572,518.40178765)(255.52357483,518.62835432)
\moveto(256.3028717,522.20257307)
\curveto(255.7520874,522.2025651)(255.33997843,521.90569039)(255.06654358,521.31194807)
\curveto(254.79701023,520.71819158)(254.66224474,519.81389561)(254.6622467,518.59905744)
\curveto(254.66224474,517.38811679)(254.79701023,516.48577394)(255.06654358,515.89202619)
\curveto(255.33997843,515.29827513)(255.7520874,515.00140042)(256.3028717,515.00140119)
\curveto(256.85755504,515.00140042)(257.269664,515.29827513)(257.53919983,515.89202619)
\curveto(257.81263221,516.48577394)(257.94935082,517.38811679)(257.94935608,518.59905744)
\curveto(257.94935082,519.81389561)(257.81263221,520.71819158)(257.53919983,521.31194807)
\curveto(257.269664,521.90569039)(256.85755504,522.2025651)(256.3028717,522.20257307)
\moveto(256.3028717,523.14007307)
\curveto(257.23646091,523.14006416)(257.94153833,522.75725204)(258.41810608,521.99163557)
\curveto(258.89856862,521.22600357)(259.13880276,520.09514533)(259.1388092,518.59905744)
\curveto(259.13880276,517.10686707)(258.89856862,515.97796195)(258.41810608,515.21233869)
\curveto(257.94153833,514.44671348)(257.23646091,514.06390136)(256.3028717,514.06390119)
\curveto(255.36927528,514.06390136)(254.66419786,514.44671348)(254.18763733,515.21233869)
\curveto(253.71107381,515.97796195)(253.4727928,517.10686707)(253.47279358,518.59905744)
\curveto(253.4727928,520.09514533)(253.71107381,521.22600357)(254.18763733,521.99163557)
\curveto(254.66419786,522.75725204)(255.36927528,523.14006416)(256.3028717,523.14007307)
}
}
{
\newrgbcolor{curcolor}{0 0 0}
\pscustom[linestyle=none,fillstyle=solid,fillcolor=curcolor]
{
\newpath
\moveto(217.62138367,521.73345686)
\lineto(214.86161804,517.11040998)
\lineto(217.62138367,517.11040998)
\lineto(217.62138367,521.73345686)
\moveto(217.42802429,522.81158186)
\lineto(218.79911804,522.81158186)
\lineto(218.79911804,517.11040998)
\lineto(219.96513367,517.11040998)
\lineto(219.96513367,516.14947248)
\lineto(218.79911804,516.14947248)
\lineto(218.79911804,514.06353498)
\lineto(217.62138367,514.06353498)
\lineto(217.62138367,516.14947248)
\lineto(213.91239929,516.14947248)
\lineto(213.91239929,517.26861311)
\lineto(217.42802429,522.81158186)
}
}
{
\newrgbcolor{curcolor}{0 0 0}
\pscustom[linestyle=none,fillstyle=solid,fillcolor=curcolor]
{
\newpath
\moveto(222.11552429,515.05962873)
\lineto(223.95536804,515.05962873)
\lineto(223.95536804,521.74517561)
\lineto(221.97489929,521.29986311)
\lineto(221.97489929,522.37798811)
\lineto(223.94364929,522.81158186)
\lineto(225.12724304,522.81158186)
\lineto(225.12724304,515.05962873)
\lineto(226.94364929,515.05962873)
\lineto(226.94364929,514.06353498)
\lineto(222.11552429,514.06353498)
\lineto(222.11552429,515.05962873)
}
}
{
\newrgbcolor{curcolor}{0 0 0}
\pscustom[linestyle=none,fillstyle=solid,fillcolor=curcolor]
{
\newpath
\moveto(215.46513367,498.17779279)
\lineto(219.48466492,498.17779279)
\lineto(219.48466492,497.18169904)
\lineto(214.17021179,497.18169904)
\lineto(214.17021179,498.17779279)
\curveto(214.90067892,498.94732228)(215.53935016,499.6270091)(216.08622742,500.21685529)
\curveto(216.63309906,500.80669542)(217.01005181,501.22271063)(217.21708679,501.46490217)
\curveto(217.60770746,501.94145991)(217.87137908,502.32622515)(218.00810242,502.61919904)
\curveto(218.1448163,502.91606831)(218.21317561,503.21880238)(218.21318054,503.52740217)
\curveto(218.21317561,504.01567658)(218.0686445,504.3984887)(217.77958679,504.67583967)
\curveto(217.49442633,504.95317565)(217.10184859,505.09184738)(216.60185242,505.09185529)
\curveto(216.2463807,505.09184738)(215.8733342,505.02739432)(215.48271179,504.89849592)
\curveto(215.09208498,504.76958208)(214.67802289,504.57426978)(214.24052429,504.31255842)
\lineto(214.24052429,505.50787092)
\curveto(214.64286668,505.69926865)(215.03739753,505.84379976)(215.42411804,505.94146467)
\curveto(215.81474051,506.03911206)(216.19950575,506.08794014)(216.57841492,506.08794904)
\curveto(217.43387951,506.08794014)(218.12137883,505.85942474)(218.64091492,505.40240217)
\curveto(219.16434653,504.9492694)(219.42606502,504.35356687)(219.42607117,503.61529279)
\curveto(219.42606502,503.24028673)(219.33817448,502.86528711)(219.16239929,502.49029279)
\curveto(218.99051858,502.11528786)(218.70926886,501.70122577)(218.31864929,501.24810529)
\curveto(218.09989447,500.99419523)(217.78153542,500.64263308)(217.36357117,500.19341779)
\curveto(216.949505,499.74419648)(216.31669313,499.07232215)(215.46513367,498.17779279)
}
}
{
\newrgbcolor{curcolor}{0 0 0}
\pscustom[linestyle=none,fillstyle=solid,fillcolor=curcolor]
{
\newpath
\moveto(223.32841492,501.57623029)
\curveto(223.32841209,501.79106943)(223.40263076,501.97661612)(223.55107117,502.13287092)
\curveto(223.70341171,502.28911581)(223.88505216,502.36724073)(224.09599304,502.36724592)
\curveto(224.31473923,502.36724073)(224.50223904,502.28911581)(224.65849304,502.13287092)
\curveto(224.81473873,501.97661612)(224.89286365,501.79106943)(224.89286804,501.57623029)
\curveto(224.89286365,501.35747612)(224.81473873,501.17192943)(224.65849304,501.01958967)
\curveto(224.50614528,500.86724223)(224.31864547,500.79107043)(224.09599304,500.79107404)
\curveto(223.87723966,500.79107043)(223.6936461,500.86528911)(223.54521179,501.01373029)
\curveto(223.40067764,501.16216381)(223.32841209,501.34966363)(223.32841492,501.57623029)
\moveto(224.10771179,505.15044904)
\curveto(223.55692748,505.15044107)(223.14481852,504.85356637)(222.87138367,504.25982404)
\curveto(222.60185031,503.66606756)(222.46708482,502.76177159)(222.46708679,501.54693342)
\curveto(222.46708482,500.33599276)(222.60185031,499.43364992)(222.87138367,498.83990217)
\curveto(223.14481852,498.2461511)(223.55692748,497.9492764)(224.10771179,497.94927717)
\curveto(224.66239513,497.9492764)(225.07450409,498.2461511)(225.34403992,498.83990217)
\curveto(225.6174723,499.43364992)(225.75419091,500.33599276)(225.75419617,501.54693342)
\curveto(225.75419091,502.76177159)(225.6174723,503.66606756)(225.34403992,504.25982404)
\curveto(225.07450409,504.85356637)(224.66239513,505.15044107)(224.10771179,505.15044904)
\moveto(224.10771179,506.08794904)
\curveto(225.041301,506.08794014)(225.74637842,505.70512802)(226.22294617,504.93951154)
\curveto(226.70340871,504.17387955)(226.94364285,503.04302131)(226.94364929,501.54693342)
\curveto(226.94364285,500.05474304)(226.70340871,498.92583792)(226.22294617,498.16021467)
\curveto(225.74637842,497.39458946)(225.041301,497.01177734)(224.10771179,497.01177717)
\curveto(223.17411537,497.01177734)(222.46903795,497.39458946)(221.99247742,498.16021467)
\curveto(221.5159139,498.92583792)(221.27763289,500.05474304)(221.27763367,501.54693342)
\curveto(221.27763289,503.04302131)(221.5159139,504.17387955)(221.99247742,504.93951154)
\curveto(222.46903795,505.70512802)(223.17411537,506.08794014)(224.10771179,506.08794904)
}
}
{
\newrgbcolor{curcolor}{0 0 0}
\pscustom[linestyle=none,fillstyle=solid,fillcolor=curcolor]
{
\newpath
\moveto(247.45521545,498.20562482)
\lineto(251.4747467,498.20562482)
\lineto(251.4747467,497.20953107)
\lineto(246.16029358,497.20953107)
\lineto(246.16029358,498.20562482)
\curveto(246.89076071,498.97515431)(247.52943194,499.65484113)(248.0763092,500.24468732)
\curveto(248.62318085,500.83452745)(249.0001336,501.25054266)(249.20716858,501.4927342)
\curveto(249.59778925,501.96929194)(249.86146086,502.35405718)(249.9981842,502.64703107)
\curveto(250.13489809,502.94390034)(250.2032574,503.24663441)(250.20326233,503.5552342)
\curveto(250.2032574,504.04350862)(250.05872629,504.42632073)(249.76966858,504.7036717)
\curveto(249.48450811,504.98100768)(249.09193038,505.11967941)(248.5919342,505.11968732)
\curveto(248.23646249,505.11967941)(247.86341599,505.05522635)(247.47279358,504.92632795)
\curveto(247.08216677,504.79741411)(246.66810468,504.60210181)(246.23060608,504.34039045)
\lineto(246.23060608,505.53570295)
\curveto(246.63294847,505.72710068)(247.02747932,505.87163179)(247.41419983,505.9692967)
\curveto(247.80482229,506.06694409)(248.18958753,506.11577217)(248.5684967,506.11578107)
\curveto(249.4239613,506.11577217)(250.11146061,505.88725677)(250.6309967,505.4302342)
\curveto(251.15442832,504.97710143)(251.41614681,504.3813989)(251.41615295,503.64312482)
\curveto(251.41614681,503.26811877)(251.32825627,502.89311914)(251.15248108,502.51812482)
\curveto(250.98060037,502.14311989)(250.69935065,501.7290578)(250.30873108,501.27593732)
\curveto(250.08997626,501.02202726)(249.7716172,500.67046511)(249.35365295,500.22124982)
\curveto(248.93958678,499.77202851)(248.30677492,499.10015418)(247.45521545,498.20562482)
}
}
{
\newrgbcolor{curcolor}{0 0 0}
\pscustom[linestyle=none,fillstyle=solid,fillcolor=curcolor]
{
\newpath
\moveto(256.7950592,504.87945295)
\lineto(254.03529358,500.25640607)
\lineto(256.7950592,500.25640607)
\lineto(256.7950592,504.87945295)
\moveto(256.60169983,505.95757795)
\lineto(257.97279358,505.95757795)
\lineto(257.97279358,500.25640607)
\lineto(259.1388092,500.25640607)
\lineto(259.1388092,499.29546857)
\lineto(257.97279358,499.29546857)
\lineto(257.97279358,497.20953107)
\lineto(256.7950592,497.20953107)
\lineto(256.7950592,499.29546857)
\lineto(253.08607483,499.29546857)
\lineto(253.08607483,500.4146092)
\lineto(256.60169983,505.95757795)
}
}
{
\newrgbcolor{curcolor}{0 0 0}
\pscustom[linestyle=none,fillstyle=solid,fillcolor=curcolor]
{
\newpath
\moveto(288.01174927,501.37011457)
\curveto(287.48440244,501.37011042)(287.07619972,501.22167306)(286.78713989,500.92480207)
\curveto(286.50198154,500.6318299)(286.35940356,500.2158147)(286.35940552,499.6767552)
\curveto(286.35940356,499.13769077)(286.50393467,498.71776932)(286.79299927,498.41698957)
\curveto(287.08596533,498.12011367)(287.49221493,497.97167631)(288.01174927,497.97167707)
\curveto(288.54299513,497.97167631)(288.95119784,498.11816054)(289.23635864,498.4111302)
\curveto(289.52541602,498.7080037)(289.66994713,499.12987828)(289.66995239,499.6767552)
\curveto(289.66994713,500.21190845)(289.5234629,500.62792366)(289.23049927,500.92480207)
\curveto(288.94143223,501.22167306)(288.53518263,501.37011042)(288.01174927,501.37011457)
\moveto(286.98049927,501.86230207)
\curveto(286.47659094,501.99120354)(286.08206009,502.23143768)(285.79690552,502.5830052)
\curveto(285.5156544,502.93456198)(285.37502954,503.35838968)(285.37503052,503.85448957)
\curveto(285.37502954,504.54979474)(285.61135743,505.10057544)(286.08401489,505.50683332)
\curveto(286.55666899,505.91698087)(287.19924647,506.12205879)(288.01174927,506.1220677)
\curveto(288.82815109,506.12205879)(289.4726817,505.91698087)(289.94534302,505.50683332)
\curveto(290.41799325,505.10057544)(290.65432114,504.54979474)(290.65432739,503.85448957)
\curveto(290.65432114,503.35838968)(290.51174316,502.93456198)(290.22659302,502.5830052)
\curveto(289.94533747,502.23143768)(289.55275974,501.99120354)(289.04885864,501.86230207)
\curveto(289.63479091,501.7333913)(290.08205609,501.47362594)(290.39065552,501.0830052)
\curveto(290.70314922,500.69237672)(290.85939906,500.18651785)(290.85940552,499.56542707)
\curveto(290.85939906,498.77636301)(290.60744619,498.15917613)(290.10354614,497.71386457)
\curveto(289.5996347,497.26855202)(288.90236977,497.04589599)(288.01174927,497.04589582)
\curveto(287.12112155,497.04589599)(286.42385662,497.26659889)(285.91995239,497.7080052)
\curveto(285.41995137,498.15331676)(285.16995163,498.76855052)(285.16995239,499.55370832)
\curveto(285.16995163,500.17870536)(285.32424835,500.68651735)(285.63284302,501.07714582)
\curveto(285.94534147,501.47167281)(286.39455978,501.7333913)(286.98049927,501.86230207)
\moveto(286.55276489,503.74316145)
\curveto(286.55276274,503.27440539)(286.67776262,502.91698387)(286.92776489,502.67089582)
\curveto(287.17776212,502.42479686)(287.53908988,502.30175011)(288.01174927,502.3017552)
\curveto(288.48830768,502.30175011)(288.85158857,502.42479686)(289.10159302,502.67089582)
\curveto(289.35158807,502.91698387)(289.47658794,503.27440539)(289.47659302,503.74316145)
\curveto(289.47658794,504.21971694)(289.35158807,504.58299783)(289.10159302,504.8330052)
\curveto(288.85549481,505.08299733)(288.49221393,505.2079972)(288.01174927,505.2080052)
\curveto(287.53908988,505.2079972)(287.17776212,505.08104421)(286.92776489,504.82714582)
\curveto(286.67776262,504.57713846)(286.55276274,504.2158107)(286.55276489,503.74316145)
}
}
{
\newrgbcolor{curcolor}{0 0 0}
\pscustom[linestyle=none,fillstyle=solid,fillcolor=curcolor]
{
\newpath
\moveto(278.81253052,515.06011701)
\lineto(280.65237427,515.06011701)
\lineto(280.65237427,521.74566389)
\lineto(278.67190552,521.30035139)
\lineto(278.67190552,522.37847639)
\lineto(280.64065552,522.81207014)
\lineto(281.82424927,522.81207014)
\lineto(281.82424927,515.06011701)
\lineto(283.64065552,515.06011701)
\lineto(283.64065552,514.06402326)
\lineto(278.81253052,514.06402326)
\lineto(278.81253052,515.06011701)
}
}
{
\newrgbcolor{curcolor}{0 0 0}
\pscustom[linestyle=none,fillstyle=solid,fillcolor=curcolor]
{
\newpath
\moveto(286.03128052,515.06011701)
\lineto(287.87112427,515.06011701)
\lineto(287.87112427,521.74566389)
\lineto(285.89065552,521.30035139)
\lineto(285.89065552,522.37847639)
\lineto(287.85940552,522.81207014)
\lineto(289.04299927,522.81207014)
\lineto(289.04299927,515.06011701)
\lineto(290.85940552,515.06011701)
\lineto(290.85940552,514.06402326)
\lineto(286.03128052,514.06402326)
\lineto(286.03128052,515.06011701)
}
}
{
\newrgbcolor{curcolor}{0 0 0}
\pscustom[linestyle=none,fillstyle=solid,fillcolor=curcolor]
{
\newpath
\moveto(312.26507568,522.98193098)
\lineto(316.69476318,522.98193098)
\lineto(316.69476318,521.98583723)
\lineto(313.34320068,521.98583723)
\lineto(313.34320068,519.8354466)
\curveto(313.51116697,519.89794094)(313.67913556,519.94286277)(313.84710693,519.97021223)
\curveto(314.01897897,520.00145646)(314.19085379,520.01708144)(314.36273193,520.01708723)
\curveto(315.26897772,520.01708144)(315.987727,519.74950359)(316.51898193,519.21435285)
\curveto(317.05022594,518.67919216)(317.31585067,517.95458351)(317.31585693,517.04052473)
\curveto(317.31585067,516.11864784)(317.03655407,515.39208607)(316.47796631,514.86083723)
\curveto(315.92327394,514.32958713)(315.16350907,514.0639624)(314.19866943,514.06396223)
\curveto(313.733823,514.0639624)(313.30804218,514.09521237)(312.92132568,514.15771223)
\curveto(312.5385117,514.22021224)(312.19476204,514.31396215)(311.89007568,514.43896223)
\lineto(311.89007568,515.6401341)
\curveto(312.24944949,515.44482039)(312.61077725,515.29833616)(312.97406006,515.20068098)
\curveto(313.33733902,515.1069301)(313.7084324,515.06005515)(314.08734131,515.06005598)
\curveto(314.73968137,515.06005515)(315.24163399,515.23192998)(315.59320068,515.57568098)
\curveto(315.94866454,515.91942929)(316.12639873,516.40771005)(316.12640381,517.04052473)
\curveto(316.12639873,517.66552129)(315.94280517,518.15184893)(315.57562256,518.4995091)
\curveto(315.21233715,518.84716074)(314.70452516,519.02098869)(314.05218506,519.02099348)
\curveto(313.73577612,519.02098869)(313.42718268,518.98387935)(313.12640381,518.90966535)
\curveto(312.82562079,518.83934825)(312.5385117,518.73192648)(312.26507568,518.58739973)
\lineto(312.26507568,522.98193098)
}
}
{
\newrgbcolor{curcolor}{0 0 0}
\pscustom[linestyle=none,fillstyle=solid,fillcolor=curcolor]
{
\newpath
\moveto(321.10101318,518.62841535)
\curveto(321.10101035,518.84325449)(321.17522903,519.02880118)(321.32366943,519.18505598)
\curveto(321.47600998,519.34130087)(321.65765042,519.41942579)(321.86859131,519.41943098)
\curveto(322.08733749,519.41942579)(322.2748373,519.34130087)(322.43109131,519.18505598)
\curveto(322.58733699,519.02880118)(322.66546191,518.84325449)(322.66546631,518.62841535)
\curveto(322.66546191,518.40966118)(322.58733699,518.22411449)(322.43109131,518.07177473)
\curveto(322.27874355,517.91942729)(322.09124374,517.84325549)(321.86859131,517.8432591)
\curveto(321.64983793,517.84325549)(321.46624436,517.91747417)(321.31781006,518.06591535)
\curveto(321.17327591,518.21434887)(321.10101035,518.40184868)(321.10101318,518.62841535)
\moveto(321.88031006,522.2026341)
\curveto(321.32952575,522.20262613)(320.91741679,521.90575143)(320.64398193,521.3120091)
\curveto(320.37444858,520.71825262)(320.23968309,519.81395665)(320.23968506,518.59911848)
\curveto(320.23968309,517.38817782)(320.37444858,516.48583497)(320.64398193,515.89208723)
\curveto(320.91741679,515.29833616)(321.32952575,515.00146146)(321.88031006,515.00146223)
\curveto(322.43499339,515.00146146)(322.84710236,515.29833616)(323.11663818,515.89208723)
\curveto(323.39007056,516.48583497)(323.52678918,517.38817782)(323.52679443,518.59911848)
\curveto(323.52678918,519.81395665)(323.39007056,520.71825262)(323.11663818,521.3120091)
\curveto(322.84710236,521.90575143)(322.43499339,522.20262613)(321.88031006,522.2026341)
\moveto(321.88031006,523.1401341)
\curveto(322.81389927,523.1401252)(323.51897669,522.75731308)(323.99554443,521.9916966)
\curveto(324.47600698,521.22606461)(324.71624111,520.09520637)(324.71624756,518.59911848)
\curveto(324.71624111,517.1069281)(324.47600698,515.97802298)(323.99554443,515.21239973)
\curveto(323.51897669,514.44677451)(322.81389927,514.0639624)(321.88031006,514.06396223)
\curveto(320.94671363,514.0639624)(320.24163621,514.44677451)(319.76507568,515.21239973)
\curveto(319.28851217,515.97802298)(319.05023115,517.1069281)(319.05023193,518.59911848)
\curveto(319.05023115,520.09520637)(319.28851217,521.22606461)(319.76507568,521.9916966)
\curveto(320.24163621,522.75731308)(320.94671363,523.1401252)(321.88031006,523.1401341)
}
}
{
\newrgbcolor{curcolor}{0 0 0}
\pscustom[linestyle=none,fillstyle=solid,fillcolor=curcolor]
{
\newpath
\moveto(315.15374756,504.84661604)
\lineto(312.39398193,500.22356916)
\lineto(315.15374756,500.22356916)
\lineto(315.15374756,504.84661604)
\moveto(314.96038818,505.92474104)
\lineto(316.33148193,505.92474104)
\lineto(316.33148193,500.22356916)
\lineto(317.49749756,500.22356916)
\lineto(317.49749756,499.26263166)
\lineto(316.33148193,499.26263166)
\lineto(316.33148193,497.17669416)
\lineto(315.15374756,497.17669416)
\lineto(315.15374756,499.26263166)
\lineto(311.44476318,499.26263166)
\lineto(311.44476318,500.38177229)
\lineto(314.96038818,505.92474104)
}
}
{
\newrgbcolor{curcolor}{0 0 0}
\pscustom[linestyle=none,fillstyle=solid,fillcolor=curcolor]
{
\newpath
\moveto(322.37249756,504.84661604)
\lineto(319.61273193,500.22356916)
\lineto(322.37249756,500.22356916)
\lineto(322.37249756,504.84661604)
\moveto(322.17913818,505.92474104)
\lineto(323.55023193,505.92474104)
\lineto(323.55023193,500.22356916)
\lineto(324.71624756,500.22356916)
\lineto(324.71624756,499.26263166)
\lineto(323.55023193,499.26263166)
\lineto(323.55023193,497.17669416)
\lineto(322.37249756,497.17669416)
\lineto(322.37249756,499.26263166)
\lineto(318.66351318,499.26263166)
\lineto(318.66351318,500.38177229)
\lineto(322.17913818,505.92474104)
}
}
{
\newrgbcolor{curcolor}{0 0 0}
\pscustom[linestyle=none,fillstyle=solid,fillcolor=curcolor]
{
\newpath
\moveto(223.32841492,484.52447248)
\curveto(223.32841209,484.73931162)(223.40263076,484.92485831)(223.55107117,485.08111311)
\curveto(223.70341171,485.237358)(223.88505216,485.31548292)(224.09599304,485.31548811)
\curveto(224.31473923,485.31548292)(224.50223904,485.237358)(224.65849304,485.08111311)
\curveto(224.81473873,484.92485831)(224.89286365,484.73931162)(224.89286804,484.52447248)
\curveto(224.89286365,484.3057183)(224.81473873,484.12017162)(224.65849304,483.96783186)
\curveto(224.50614528,483.81548442)(224.31864547,483.73931262)(224.09599304,483.73931623)
\curveto(223.87723966,483.73931262)(223.6936461,483.8135313)(223.54521179,483.96197248)
\curveto(223.40067764,484.110406)(223.32841209,484.29790581)(223.32841492,484.52447248)
\moveto(224.10771179,488.09869123)
\curveto(223.55692748,488.09868326)(223.14481852,487.80180856)(222.87138367,487.20806623)
\curveto(222.60185031,486.61430975)(222.46708482,485.71001378)(222.46708679,484.49517561)
\curveto(222.46708482,483.28423495)(222.60185031,482.3818921)(222.87138367,481.78814436)
\curveto(223.14481852,481.19439329)(223.55692748,480.89751859)(224.10771179,480.89751936)
\curveto(224.66239513,480.89751859)(225.07450409,481.19439329)(225.34403992,481.78814436)
\curveto(225.6174723,482.3818921)(225.75419091,483.28423495)(225.75419617,484.49517561)
\curveto(225.75419091,485.71001378)(225.6174723,486.61430975)(225.34403992,487.20806623)
\curveto(225.07450409,487.80180856)(224.66239513,488.09868326)(224.10771179,488.09869123)
\moveto(224.10771179,489.03619123)
\curveto(225.041301,489.03618232)(225.74637842,488.65337021)(226.22294617,487.88775373)
\curveto(226.70340871,487.12212174)(226.94364285,485.99126349)(226.94364929,484.49517561)
\curveto(226.94364285,483.00298523)(226.70340871,481.87408011)(226.22294617,481.10845686)
\curveto(225.74637842,480.34283164)(225.041301,479.96001953)(224.10771179,479.96001936)
\curveto(223.17411537,479.96001953)(222.46903795,480.34283164)(221.99247742,481.10845686)
\curveto(221.5159139,481.87408011)(221.27763289,483.00298523)(221.27763367,484.49517561)
\curveto(221.27763289,485.99126349)(221.5159139,487.12212174)(221.99247742,487.88775373)
\curveto(222.46903795,488.65337021)(223.17411537,489.03618232)(224.10771179,489.03619123)
}
}
{
\newrgbcolor{curcolor}{0 0 0}
\pscustom[linestyle=none,fillstyle=solid,fillcolor=curcolor]
{
\newpath
\moveto(255.52357483,484.74981428)
\curveto(255.523572,484.96465342)(255.59779067,485.15020011)(255.74623108,485.3064549)
\curveto(255.89857162,485.46269979)(256.08021207,485.54082472)(256.29115295,485.5408299)
\curveto(256.50989914,485.54082472)(256.69739895,485.46269979)(256.85365295,485.3064549)
\curveto(257.00989864,485.15020011)(257.08802356,484.96465342)(257.08802795,484.74981428)
\curveto(257.08802356,484.5310601)(257.00989864,484.34551341)(256.85365295,484.19317365)
\curveto(256.7013052,484.04082622)(256.51380538,483.96465442)(256.29115295,483.96465803)
\curveto(256.07239958,483.96465442)(255.88880601,484.03887309)(255.7403717,484.18731428)
\curveto(255.59583755,484.3357478)(255.523572,484.52324761)(255.52357483,484.74981428)
\moveto(256.3028717,488.32403303)
\curveto(255.7520874,488.32402506)(255.33997843,488.02715036)(255.06654358,487.43340803)
\curveto(254.79701023,486.83965154)(254.66224474,485.93535557)(254.6622467,484.7205174)
\curveto(254.66224474,483.50957675)(254.79701023,482.6072339)(255.06654358,482.01348615)
\curveto(255.33997843,481.41973509)(255.7520874,481.12286038)(256.3028717,481.12286115)
\curveto(256.85755504,481.12286038)(257.269664,481.41973509)(257.53919983,482.01348615)
\curveto(257.81263221,482.6072339)(257.94935082,483.50957675)(257.94935608,484.7205174)
\curveto(257.94935082,485.93535557)(257.81263221,486.83965154)(257.53919983,487.43340803)
\curveto(257.269664,488.02715036)(256.85755504,488.32402506)(256.3028717,488.32403303)
\moveto(256.3028717,489.26153303)
\curveto(257.23646091,489.26152412)(257.94153833,488.878712)(258.41810608,488.11309553)
\curveto(258.89856862,487.34746354)(259.13880276,486.21660529)(259.1388092,484.7205174)
\curveto(259.13880276,483.22832703)(258.89856862,482.09942191)(258.41810608,481.33379865)
\curveto(257.94153833,480.56817344)(257.23646091,480.18536132)(256.3028717,480.18536115)
\curveto(255.36927528,480.18536132)(254.66419786,480.56817344)(254.18763733,481.33379865)
\curveto(253.71107381,482.09942191)(253.4727928,483.22832703)(253.47279358,484.7205174)
\curveto(253.4727928,486.21660529)(253.71107381,487.34746354)(254.18763733,488.11309553)
\curveto(254.66419786,488.878712)(255.36927528,489.26152412)(256.3028717,489.26153303)
}
}
{
\newrgbcolor{curcolor}{0 0 0}
\pscustom[linestyle=none,fillstyle=solid,fillcolor=curcolor]
{
\newpath
\moveto(287.24414062,484.59209943)
\curveto(287.24413779,484.80693857)(287.31835647,484.99248526)(287.46679688,485.14874006)
\curveto(287.61913742,485.30498495)(287.80077786,485.38310987)(288.01171875,485.38311506)
\curveto(288.23046493,485.38310987)(288.41796475,485.30498495)(288.57421875,485.14874006)
\curveto(288.73046443,484.99248526)(288.80858936,484.80693857)(288.80859375,484.59209943)
\curveto(288.80858936,484.37334526)(288.73046443,484.18779857)(288.57421875,484.03545881)
\curveto(288.42187099,483.88311137)(288.23437118,483.80693957)(288.01171875,483.80694318)
\curveto(287.79296537,483.80693957)(287.6093718,483.88115825)(287.4609375,484.02959943)
\curveto(287.31640335,484.17803295)(287.24413779,484.36553277)(287.24414062,484.59209943)
\moveto(288.0234375,488.16631818)
\curveto(287.47265319,488.16631021)(287.06054423,487.86943551)(286.78710938,487.27569318)
\curveto(286.51757602,486.6819367)(286.38281053,485.77764073)(286.3828125,484.56280256)
\curveto(286.38281053,483.3518619)(286.51757602,482.44951906)(286.78710938,481.85577131)
\curveto(287.06054423,481.26202024)(287.47265319,480.96514554)(288.0234375,480.96514631)
\curveto(288.57812084,480.96514554)(288.9902298,481.26202024)(289.25976562,481.85577131)
\curveto(289.53319801,482.44951906)(289.66991662,483.3518619)(289.66992188,484.56280256)
\curveto(289.66991662,485.77764073)(289.53319801,486.6819367)(289.25976562,487.27569318)
\curveto(288.9902298,487.86943551)(288.57812084,488.16631021)(288.0234375,488.16631818)
\moveto(288.0234375,489.10381818)
\curveto(288.95702671,489.10380928)(289.66210413,488.72099716)(290.13867188,487.95538068)
\curveto(290.61913442,487.18974869)(290.85936855,486.05889045)(290.859375,484.56280256)
\curveto(290.85936855,483.07061219)(290.61913442,481.94170706)(290.13867188,481.17608381)
\curveto(289.66210413,480.4104586)(288.95702671,480.02764648)(288.0234375,480.02764631)
\curveto(287.08984107,480.02764648)(286.38476365,480.4104586)(285.90820312,481.17608381)
\curveto(285.43163961,481.94170706)(285.1933586,483.07061219)(285.19335938,484.56280256)
\curveto(285.1933586,486.05889045)(285.43163961,487.18974869)(285.90820312,487.95538068)
\curveto(286.38476365,488.72099716)(287.08984107,489.10380928)(288.0234375,489.10381818)
}
}
{
\newrgbcolor{curcolor}{0 0 0}
\pscustom[linestyle=none,fillstyle=solid,fillcolor=curcolor]
{
\newpath
\moveto(319.66546631,489.03741193)
\lineto(324.09515381,489.03741193)
\lineto(324.09515381,488.04131818)
\lineto(320.74359131,488.04131818)
\lineto(320.74359131,485.89092756)
\curveto(320.9115576,485.95342189)(321.07952618,485.99834372)(321.24749756,486.02569318)
\curveto(321.41936959,486.05693742)(321.59124442,486.0725624)(321.76312256,486.07256818)
\curveto(322.66936834,486.0725624)(323.38811762,485.80498454)(323.91937256,485.26983381)
\curveto(324.45061656,484.73467311)(324.71624129,484.01006446)(324.71624756,483.09600568)
\curveto(324.71624129,482.1741288)(324.4369447,481.44756703)(323.87835693,480.91631818)
\curveto(323.32366456,480.38506809)(322.5638997,480.11944335)(321.59906006,480.11944318)
\curveto(321.13421363,480.11944335)(320.7084328,480.15069332)(320.32171631,480.21319318)
\curveto(319.93890232,480.2756932)(319.59515267,480.3694431)(319.29046631,480.49444318)
\lineto(319.29046631,481.69561506)
\curveto(319.64984011,481.50030135)(320.01116788,481.35381712)(320.37445068,481.25616193)
\curveto(320.73772965,481.16241106)(321.10882303,481.11553611)(321.48773193,481.11553693)
\curveto(322.140072,481.11553611)(322.64202462,481.28741094)(322.99359131,481.63116193)
\curveto(323.34905516,481.97491025)(323.52678936,482.46319101)(323.52679443,483.09600568)
\curveto(323.52678936,483.72100225)(323.34319579,484.20732989)(322.97601318,484.55499006)
\curveto(322.61272777,484.9026417)(322.10491578,485.07646965)(321.45257568,485.07647443)
\curveto(321.13616675,485.07646965)(320.82757331,485.03936031)(320.52679443,484.96514631)
\curveto(320.22601141,484.8948292)(319.93890232,484.78740744)(319.66546631,484.64288068)
\lineto(319.66546631,489.03741193)
}
}
{
\newrgbcolor{curcolor}{0 0 0}
\pscustom[linestyle=none,fillstyle=solid,fillcolor=curcolor]
{
\newpath
\moveto(223.32841492,467.47271467)
\curveto(223.32841209,467.68755381)(223.40263076,467.8731005)(223.55107117,468.02935529)
\curveto(223.70341171,468.18560019)(223.88505216,468.26372511)(224.09599304,468.26373029)
\curveto(224.31473923,468.26372511)(224.50223904,468.18560019)(224.65849304,468.02935529)
\curveto(224.81473873,467.8731005)(224.89286365,467.68755381)(224.89286804,467.47271467)
\curveto(224.89286365,467.25396049)(224.81473873,467.0684138)(224.65849304,466.91607404)
\curveto(224.50614528,466.76372661)(224.31864547,466.68755481)(224.09599304,466.68755842)
\curveto(223.87723966,466.68755481)(223.6936461,466.76177348)(223.54521179,466.91021467)
\curveto(223.40067764,467.05864819)(223.32841209,467.246148)(223.32841492,467.47271467)
\moveto(224.10771179,471.04693342)
\curveto(223.55692748,471.04692545)(223.14481852,470.75005075)(222.87138367,470.15630842)
\curveto(222.60185031,469.56255193)(222.46708482,468.65825596)(222.46708679,467.44341779)
\curveto(222.46708482,466.23247714)(222.60185031,465.33013429)(222.87138367,464.73638654)
\curveto(223.14481852,464.14263548)(223.55692748,463.84576078)(224.10771179,463.84576154)
\curveto(224.66239513,463.84576078)(225.07450409,464.14263548)(225.34403992,464.73638654)
\curveto(225.6174723,465.33013429)(225.75419091,466.23247714)(225.75419617,467.44341779)
\curveto(225.75419091,468.65825596)(225.6174723,469.56255193)(225.34403992,470.15630842)
\curveto(225.07450409,470.75005075)(224.66239513,471.04692545)(224.10771179,471.04693342)
\moveto(224.10771179,471.98443342)
\curveto(225.041301,471.98442451)(225.74637842,471.60161239)(226.22294617,470.83599592)
\curveto(226.70340871,470.07036393)(226.94364285,468.93950568)(226.94364929,467.44341779)
\curveto(226.94364285,465.95122742)(226.70340871,464.8223223)(226.22294617,464.05669904)
\curveto(225.74637842,463.29107383)(225.041301,462.90826171)(224.10771179,462.90826154)
\curveto(223.17411537,462.90826171)(222.46903795,463.29107383)(221.99247742,464.05669904)
\curveto(221.5159139,464.8223223)(221.27763289,465.95122742)(221.27763367,467.44341779)
\curveto(221.27763289,468.93950568)(221.5159139,470.07036393)(221.99247742,470.83599592)
\curveto(222.46903795,471.60161239)(223.17411537,471.98442451)(224.10771179,471.98443342)
}
}
{
\newrgbcolor{curcolor}{0 0 0}
\pscustom[linestyle=none,fillstyle=solid,fillcolor=curcolor]
{
\newpath
\moveto(255.52357483,467.72552229)
\curveto(255.523572,467.94036143)(255.59779067,468.12590812)(255.74623108,468.28216291)
\curveto(255.89857162,468.4384078)(256.08021207,468.51653272)(256.29115295,468.51653791)
\curveto(256.50989914,468.51653272)(256.69739895,468.4384078)(256.85365295,468.28216291)
\curveto(257.00989864,468.12590812)(257.08802356,467.94036143)(257.08802795,467.72552229)
\curveto(257.08802356,467.50676811)(257.00989864,467.32122142)(256.85365295,467.16888166)
\curveto(256.7013052,467.01653422)(256.51380538,466.94036243)(256.29115295,466.94036604)
\curveto(256.07239958,466.94036243)(255.88880601,467.0145811)(255.7403717,467.16302229)
\curveto(255.59583755,467.3114558)(255.523572,467.49895562)(255.52357483,467.72552229)
\moveto(256.3028717,471.29974104)
\curveto(255.7520874,471.29973307)(255.33997843,471.00285836)(255.06654358,470.40911604)
\curveto(254.79701023,469.81535955)(254.66224474,468.91106358)(254.6622467,467.69622541)
\curveto(254.66224474,466.48528476)(254.79701023,465.58294191)(255.06654358,464.98919416)
\curveto(255.33997843,464.3954431)(255.7520874,464.09856839)(256.3028717,464.09856916)
\curveto(256.85755504,464.09856839)(257.269664,464.3954431)(257.53919983,464.98919416)
\curveto(257.81263221,465.58294191)(257.94935082,466.48528476)(257.94935608,467.69622541)
\curveto(257.94935082,468.91106358)(257.81263221,469.81535955)(257.53919983,470.40911604)
\curveto(257.269664,471.00285836)(256.85755504,471.29973307)(256.3028717,471.29974104)
\moveto(256.3028717,472.23724104)
\curveto(257.23646091,472.23723213)(257.94153833,471.85442001)(258.41810608,471.08880354)
\curveto(258.89856862,470.32317154)(259.13880276,469.1923133)(259.1388092,467.69622541)
\curveto(259.13880276,466.20403504)(258.89856862,465.07512992)(258.41810608,464.30950666)
\curveto(257.94153833,463.54388145)(257.23646091,463.16106933)(256.3028717,463.16106916)
\curveto(255.36927528,463.16106933)(254.66419786,463.54388145)(254.18763733,464.30950666)
\curveto(253.71107381,465.07512992)(253.4727928,466.20403504)(253.47279358,467.69622541)
\curveto(253.4727928,469.1923133)(253.71107381,470.32317154)(254.18763733,471.08880354)
\curveto(254.66419786,471.85442001)(255.36927528,472.23723213)(256.3028717,472.23724104)
}
}
{
\newrgbcolor{curcolor}{0 0 0}
\pscustom[linestyle=none,fillstyle=solid,fillcolor=curcolor]
{
\newpath
\moveto(287.24414062,467.57391096)
\curveto(287.24413779,467.7887501)(287.31835647,467.97429679)(287.46679688,468.13055158)
\curveto(287.61913742,468.28679647)(287.80077786,468.3649214)(288.01171875,468.36492658)
\curveto(288.23046493,468.3649214)(288.41796475,468.28679647)(288.57421875,468.13055158)
\curveto(288.73046443,467.97429679)(288.80858936,467.7887501)(288.80859375,467.57391096)
\curveto(288.80858936,467.35515678)(288.73046443,467.16961009)(288.57421875,467.01727033)
\curveto(288.42187099,466.8649229)(288.23437118,466.7887511)(288.01171875,466.78875471)
\curveto(287.79296537,466.7887511)(287.6093718,466.86296977)(287.4609375,467.01141096)
\curveto(287.31640335,467.15984448)(287.24413779,467.34734429)(287.24414062,467.57391096)
\moveto(288.0234375,471.14812971)
\curveto(287.47265319,471.14812174)(287.06054423,470.85124704)(286.78710938,470.25750471)
\curveto(286.51757602,469.66374822)(286.38281053,468.75945225)(286.3828125,467.54461408)
\curveto(286.38281053,466.33367343)(286.51757602,465.43133058)(286.78710938,464.83758283)
\curveto(287.06054423,464.24383177)(287.47265319,463.94695706)(288.0234375,463.94695783)
\curveto(288.57812084,463.94695706)(288.9902298,464.24383177)(289.25976562,464.83758283)
\curveto(289.53319801,465.43133058)(289.66991662,466.33367343)(289.66992188,467.54461408)
\curveto(289.66991662,468.75945225)(289.53319801,469.66374822)(289.25976562,470.25750471)
\curveto(288.9902298,470.85124704)(288.57812084,471.14812174)(288.0234375,471.14812971)
\moveto(288.0234375,472.08562971)
\curveto(288.95702671,472.0856208)(289.66210413,471.70280868)(290.13867188,470.93719221)
\curveto(290.61913442,470.17156021)(290.85936855,469.04070197)(290.859375,467.54461408)
\curveto(290.85936855,466.05242371)(290.61913442,464.92351859)(290.13867188,464.15789533)
\curveto(289.66210413,463.39227012)(288.95702671,463.009458)(288.0234375,463.00945783)
\curveto(287.08984107,463.009458)(286.38476365,463.39227012)(285.90820312,464.15789533)
\curveto(285.43163961,464.92351859)(285.1933586,466.05242371)(285.19335938,467.54461408)
\curveto(285.1933586,469.04070197)(285.43163961,470.17156021)(285.90820312,470.93719221)
\curveto(286.38476365,471.70280868)(287.08984107,472.0856208)(288.0234375,472.08562971)
}
}
{
\newrgbcolor{curcolor}{0 0 0}
\pscustom[linestyle=none,fillstyle=solid,fillcolor=curcolor]
{
\newpath
\moveto(320.69671631,464.07000471)
\lineto(324.71624756,464.07000471)
\lineto(324.71624756,463.07391096)
\lineto(319.40179443,463.07391096)
\lineto(319.40179443,464.07000471)
\curveto(320.13226156,464.83953419)(320.7709328,465.51922101)(321.31781006,466.10906721)
\curveto(321.86468171,466.69890733)(322.24163445,467.11492254)(322.44866943,467.35711408)
\curveto(322.83929011,467.83367182)(323.10296172,468.21843706)(323.23968506,468.51141096)
\curveto(323.37639894,468.80828022)(323.44475825,469.11101429)(323.44476318,469.41961408)
\curveto(323.44475825,469.9078885)(323.30022714,470.29070062)(323.01116943,470.56805158)
\curveto(322.72600897,470.84538756)(322.33343124,470.9840593)(321.83343506,470.98406721)
\curveto(321.47796334,470.9840593)(321.10491684,470.91960624)(320.71429443,470.79070783)
\curveto(320.32366762,470.66179399)(319.90960554,470.46648169)(319.47210693,470.20477033)
\lineto(319.47210693,471.40008283)
\curveto(319.87444932,471.59148056)(320.26898018,471.73601167)(320.65570068,471.83367658)
\curveto(321.04632315,471.93132397)(321.43108839,471.98015205)(321.80999756,471.98016096)
\curveto(322.66546215,471.98015205)(323.35296147,471.75163665)(323.87249756,471.29461408)
\curveto(324.39592917,470.84148131)(324.65764766,470.24577879)(324.65765381,469.50750471)
\curveto(324.65764766,469.13249865)(324.56975713,468.75749902)(324.39398193,468.38250471)
\curveto(324.22210122,468.00749977)(323.9408515,467.59343769)(323.55023193,467.14031721)
\curveto(323.33147711,466.88640714)(323.01311806,466.534845)(322.59515381,466.08562971)
\curveto(322.18108764,465.63640839)(321.54827577,464.96453407)(320.69671631,464.07000471)
}
}
{
\newrgbcolor{curcolor}{0 0 0}
\pscustom[linestyle=none,fillstyle=solid,fillcolor=curcolor]
{
\newpath
\moveto(223.32841492,450.42095686)
\curveto(223.32841209,450.635796)(223.40263076,450.82134269)(223.55107117,450.97759748)
\curveto(223.70341171,451.13384237)(223.88505216,451.21196729)(224.09599304,451.21197248)
\curveto(224.31473923,451.21196729)(224.50223904,451.13384237)(224.65849304,450.97759748)
\curveto(224.81473873,450.82134269)(224.89286365,450.635796)(224.89286804,450.42095686)
\curveto(224.89286365,450.20220268)(224.81473873,450.01665599)(224.65849304,449.86431623)
\curveto(224.50614528,449.71196879)(224.31864547,449.635797)(224.09599304,449.63580061)
\curveto(223.87723966,449.635797)(223.6936461,449.71001567)(223.54521179,449.85845686)
\curveto(223.40067764,450.00689038)(223.32841209,450.19439019)(223.32841492,450.42095686)
\moveto(224.10771179,453.99517561)
\curveto(223.55692748,453.99516764)(223.14481852,453.69829293)(222.87138367,453.10455061)
\curveto(222.60185031,452.51079412)(222.46708482,451.60649815)(222.46708679,450.39165998)
\curveto(222.46708482,449.18071933)(222.60185031,448.27837648)(222.87138367,447.68462873)
\curveto(223.14481852,447.09087767)(223.55692748,446.79400296)(224.10771179,446.79400373)
\curveto(224.66239513,446.79400296)(225.07450409,447.09087767)(225.34403992,447.68462873)
\curveto(225.6174723,448.27837648)(225.75419091,449.18071933)(225.75419617,450.39165998)
\curveto(225.75419091,451.60649815)(225.6174723,452.51079412)(225.34403992,453.10455061)
\curveto(225.07450409,453.69829293)(224.66239513,453.99516764)(224.10771179,453.99517561)
\moveto(224.10771179,454.93267561)
\curveto(225.041301,454.9326667)(225.74637842,454.54985458)(226.22294617,453.78423811)
\curveto(226.70340871,453.01860611)(226.94364285,451.88774787)(226.94364929,450.39165998)
\curveto(226.94364285,448.89946961)(226.70340871,447.77056449)(226.22294617,447.00494123)
\curveto(225.74637842,446.23931602)(225.041301,445.8565039)(224.10771179,445.85650373)
\curveto(223.17411537,445.8565039)(222.46903795,446.23931602)(221.99247742,447.00494123)
\curveto(221.5159139,447.77056449)(221.27763289,448.89946961)(221.27763367,450.39165998)
\curveto(221.27763289,451.88774787)(221.5159139,453.01860611)(221.99247742,453.78423811)
\curveto(222.46903795,454.54985458)(223.17411537,454.9326667)(224.10771179,454.93267561)
}
}
{
\newrgbcolor{curcolor}{0 0 0}
\pscustom[linestyle=none,fillstyle=solid,fillcolor=curcolor]
{
\newpath
\moveto(255.52357483,450.70135236)
\curveto(255.523572,450.9161915)(255.59779067,451.10173819)(255.74623108,451.25799299)
\curveto(255.89857162,451.41423788)(256.08021207,451.4923628)(256.29115295,451.49236799)
\curveto(256.50989914,451.4923628)(256.69739895,451.41423788)(256.85365295,451.25799299)
\curveto(257.00989864,451.10173819)(257.08802356,450.9161915)(257.08802795,450.70135236)
\curveto(257.08802356,450.48259819)(257.00989864,450.2970515)(256.85365295,450.14471174)
\curveto(256.7013052,449.9923643)(256.51380538,449.9161925)(256.29115295,449.91619611)
\curveto(256.07239958,449.9161925)(255.88880601,449.99041118)(255.7403717,450.13885236)
\curveto(255.59583755,450.28728588)(255.523572,450.4747857)(255.52357483,450.70135236)
\moveto(256.3028717,454.27557111)
\curveto(255.7520874,454.27556314)(255.33997843,453.97868844)(255.06654358,453.38494611)
\curveto(254.79701023,452.79118963)(254.66224474,451.88689366)(254.6622467,450.67205549)
\curveto(254.66224474,449.46111483)(254.79701023,448.55877199)(255.06654358,447.96502424)
\curveto(255.33997843,447.37127317)(255.7520874,447.07439847)(256.3028717,447.07439924)
\curveto(256.85755504,447.07439847)(257.269664,447.37127317)(257.53919983,447.96502424)
\curveto(257.81263221,448.55877199)(257.94935082,449.46111483)(257.94935608,450.67205549)
\curveto(257.94935082,451.88689366)(257.81263221,452.79118963)(257.53919983,453.38494611)
\curveto(257.269664,453.97868844)(256.85755504,454.27556314)(256.3028717,454.27557111)
\moveto(256.3028717,455.21307111)
\curveto(257.23646091,455.21306221)(257.94153833,454.83025009)(258.41810608,454.06463361)
\curveto(258.89856862,453.29900162)(259.13880276,452.16814338)(259.1388092,450.67205549)
\curveto(259.13880276,449.17986512)(258.89856862,448.05095999)(258.41810608,447.28533674)
\curveto(257.94153833,446.51971153)(257.23646091,446.13689941)(256.3028717,446.13689924)
\curveto(255.36927528,446.13689941)(254.66419786,446.51971153)(254.18763733,447.28533674)
\curveto(253.71107381,448.05095999)(253.4727928,449.17986512)(253.47279358,450.67205549)
\curveto(253.4727928,452.16814338)(253.71107381,453.29900162)(254.18763733,454.06463361)
\curveto(254.66419786,454.83025009)(255.36927528,455.21306221)(256.3028717,455.21307111)
}
}
{
\newrgbcolor{curcolor}{0 0 0}
\pscustom[linestyle=none,fillstyle=solid,fillcolor=curcolor]
{
\newpath
\moveto(287.24414062,450.55572248)
\curveto(287.24413779,450.77056162)(287.31835647,450.95610831)(287.46679688,451.11236311)
\curveto(287.61913742,451.268608)(287.80077786,451.34673292)(288.01171875,451.34673811)
\curveto(288.23046493,451.34673292)(288.41796475,451.268608)(288.57421875,451.11236311)
\curveto(288.73046443,450.95610831)(288.80858936,450.77056162)(288.80859375,450.55572248)
\curveto(288.80858936,450.3369683)(288.73046443,450.15142162)(288.57421875,449.99908186)
\curveto(288.42187099,449.84673442)(288.23437118,449.77056262)(288.01171875,449.77056623)
\curveto(287.79296537,449.77056262)(287.6093718,449.8447813)(287.4609375,449.99322248)
\curveto(287.31640335,450.141656)(287.24413779,450.32915581)(287.24414062,450.55572248)
\moveto(288.0234375,454.12994123)
\curveto(287.47265319,454.12993326)(287.06054423,453.83305856)(286.78710938,453.23931623)
\curveto(286.51757602,452.64555975)(286.38281053,451.74126378)(286.3828125,450.52642561)
\curveto(286.38281053,449.31548495)(286.51757602,448.4131421)(286.78710938,447.81939436)
\curveto(287.06054423,447.22564329)(287.47265319,446.92876859)(288.0234375,446.92876936)
\curveto(288.57812084,446.92876859)(288.9902298,447.22564329)(289.25976562,447.81939436)
\curveto(289.53319801,448.4131421)(289.66991662,449.31548495)(289.66992188,450.52642561)
\curveto(289.66991662,451.74126378)(289.53319801,452.64555975)(289.25976562,453.23931623)
\curveto(288.9902298,453.83305856)(288.57812084,454.12993326)(288.0234375,454.12994123)
\moveto(288.0234375,455.06744123)
\curveto(288.95702671,455.06743232)(289.66210413,454.68462021)(290.13867188,453.91900373)
\curveto(290.61913442,453.15337174)(290.85936855,452.02251349)(290.859375,450.52642561)
\curveto(290.85936855,449.03423523)(290.61913442,447.90533011)(290.13867188,447.13970686)
\curveto(289.66210413,446.37408164)(288.95702671,445.99126953)(288.0234375,445.99126936)
\curveto(287.08984107,445.99126953)(286.38476365,446.37408164)(285.90820312,447.13970686)
\curveto(285.43163961,447.90533011)(285.1933586,449.03423523)(285.19335938,450.52642561)
\curveto(285.1933586,452.02251349)(285.43163961,453.15337174)(285.90820312,453.91900373)
\curveto(286.38476365,454.68462021)(287.08984107,455.06743232)(288.0234375,455.06744123)
}
}
{
\newrgbcolor{curcolor}{0 0 0}
\pscustom[linestyle=none,fillstyle=solid,fillcolor=curcolor]
{
\newpath
\moveto(322.37249756,453.85650373)
\lineto(319.61273193,449.23345686)
\lineto(322.37249756,449.23345686)
\lineto(322.37249756,453.85650373)
\moveto(322.17913818,454.93462873)
\lineto(323.55023193,454.93462873)
\lineto(323.55023193,449.23345686)
\lineto(324.71624756,449.23345686)
\lineto(324.71624756,448.27251936)
\lineto(323.55023193,448.27251936)
\lineto(323.55023193,446.18658186)
\lineto(322.37249756,446.18658186)
\lineto(322.37249756,448.27251936)
\lineto(318.66351318,448.27251936)
\lineto(318.66351318,449.39165998)
\lineto(322.17913818,454.93462873)
}
}
{
\newrgbcolor{curcolor}{0 0 0}
\pscustom[linestyle=none,fillstyle=solid,fillcolor=curcolor]
{
\newpath
\moveto(223.32841492,433.36919904)
\curveto(223.32841209,433.58403818)(223.40263076,433.76958487)(223.55107117,433.92583967)
\curveto(223.70341171,434.08208456)(223.88505216,434.16020948)(224.09599304,434.16021467)
\curveto(224.31473923,434.16020948)(224.50223904,434.08208456)(224.65849304,433.92583967)
\curveto(224.81473873,433.76958487)(224.89286365,433.58403818)(224.89286804,433.36919904)
\curveto(224.89286365,433.15044487)(224.81473873,432.96489818)(224.65849304,432.81255842)
\curveto(224.50614528,432.66021098)(224.31864547,432.58403918)(224.09599304,432.58404279)
\curveto(223.87723966,432.58403918)(223.6936461,432.65825786)(223.54521179,432.80669904)
\curveto(223.40067764,432.95513256)(223.32841209,433.14263238)(223.32841492,433.36919904)
\moveto(224.10771179,436.94341779)
\curveto(223.55692748,436.94340982)(223.14481852,436.64653512)(222.87138367,436.05279279)
\curveto(222.60185031,435.45903631)(222.46708482,434.55474034)(222.46708679,433.33990217)
\curveto(222.46708482,432.12896151)(222.60185031,431.22661867)(222.87138367,430.63287092)
\curveto(223.14481852,430.03911985)(223.55692748,429.74224515)(224.10771179,429.74224592)
\curveto(224.66239513,429.74224515)(225.07450409,430.03911985)(225.34403992,430.63287092)
\curveto(225.6174723,431.22661867)(225.75419091,432.12896151)(225.75419617,433.33990217)
\curveto(225.75419091,434.55474034)(225.6174723,435.45903631)(225.34403992,436.05279279)
\curveto(225.07450409,436.64653512)(224.66239513,436.94340982)(224.10771179,436.94341779)
\moveto(224.10771179,437.88091779)
\curveto(225.041301,437.88090889)(225.74637842,437.49809677)(226.22294617,436.73248029)
\curveto(226.70340871,435.9668483)(226.94364285,434.83599006)(226.94364929,433.33990217)
\curveto(226.94364285,431.84771179)(226.70340871,430.71880667)(226.22294617,429.95318342)
\curveto(225.74637842,429.18755821)(225.041301,428.80474609)(224.10771179,428.80474592)
\curveto(223.17411537,428.80474609)(222.46903795,429.18755821)(221.99247742,429.95318342)
\curveto(221.5159139,430.71880667)(221.27763289,431.84771179)(221.27763367,433.33990217)
\curveto(221.27763289,434.83599006)(221.5159139,435.9668483)(221.99247742,436.73248029)
\curveto(222.46903795,437.49809677)(223.17411537,437.88090889)(224.10771179,437.88091779)
}
}
{
\newrgbcolor{curcolor}{0 0 0}
\pscustom[linestyle=none,fillstyle=solid,fillcolor=curcolor]
{
\newpath
\moveto(255.52357483,433.67706037)
\curveto(255.523572,433.89189951)(255.59779067,434.0774462)(255.74623108,434.233701)
\curveto(255.89857162,434.38994589)(256.08021207,434.46807081)(256.29115295,434.468076)
\curveto(256.50989914,434.46807081)(256.69739895,434.38994589)(256.85365295,434.233701)
\curveto(257.00989864,434.0774462)(257.08802356,433.89189951)(257.08802795,433.67706037)
\curveto(257.08802356,433.4583062)(257.00989864,433.27275951)(256.85365295,433.12041975)
\curveto(256.7013052,432.96807231)(256.51380538,432.89190051)(256.29115295,432.89190412)
\curveto(256.07239958,432.89190051)(255.88880601,432.96611919)(255.7403717,433.11456037)
\curveto(255.59583755,433.26299389)(255.523572,433.4504937)(255.52357483,433.67706037)
\moveto(256.3028717,437.25127912)
\curveto(255.7520874,437.25127115)(255.33997843,436.95439645)(255.06654358,436.36065412)
\curveto(254.79701023,435.76689764)(254.66224474,434.86260167)(254.6622467,433.6477635)
\curveto(254.66224474,432.43682284)(254.79701023,431.53447999)(255.06654358,430.94073225)
\curveto(255.33997843,430.34698118)(255.7520874,430.05010648)(256.3028717,430.05010725)
\curveto(256.85755504,430.05010648)(257.269664,430.34698118)(257.53919983,430.94073225)
\curveto(257.81263221,431.53447999)(257.94935082,432.43682284)(257.94935608,433.6477635)
\curveto(257.94935082,434.86260167)(257.81263221,435.76689764)(257.53919983,436.36065412)
\curveto(257.269664,436.95439645)(256.85755504,437.25127115)(256.3028717,437.25127912)
\moveto(256.3028717,438.18877912)
\curveto(257.23646091,438.18877021)(257.94153833,437.8059581)(258.41810608,437.04034162)
\curveto(258.89856862,436.27470963)(259.13880276,435.14385138)(259.1388092,433.6477635)
\curveto(259.13880276,432.15557312)(258.89856862,431.026668)(258.41810608,430.26104475)
\curveto(257.94153833,429.49541953)(257.23646091,429.11260742)(256.3028717,429.11260725)
\curveto(255.36927528,429.11260742)(254.66419786,429.49541953)(254.18763733,430.26104475)
\curveto(253.71107381,431.026668)(253.4727928,432.15557312)(253.47279358,433.6477635)
\curveto(253.4727928,435.14385138)(253.71107381,436.27470963)(254.18763733,437.04034162)
\curveto(254.66419786,437.8059581)(255.36927528,438.18877021)(256.3028717,438.18877912)
}
}
{
\newrgbcolor{curcolor}{0 0 0}
\pscustom[linestyle=none,fillstyle=solid,fillcolor=curcolor]
{
\newpath
\moveto(287.24414062,433.537534)
\curveto(287.24413779,433.75237314)(287.31835647,433.93791983)(287.46679688,434.09417463)
\curveto(287.61913742,434.25041952)(287.80077786,434.32854444)(288.01171875,434.32854963)
\curveto(288.23046493,434.32854444)(288.41796475,434.25041952)(288.57421875,434.09417463)
\curveto(288.73046443,433.93791983)(288.80858936,433.75237314)(288.80859375,433.537534)
\curveto(288.80858936,433.31877983)(288.73046443,433.13323314)(288.57421875,432.98089338)
\curveto(288.42187099,432.82854594)(288.23437118,432.75237414)(288.01171875,432.75237775)
\curveto(287.79296537,432.75237414)(287.6093718,432.82659282)(287.4609375,432.975034)
\curveto(287.31640335,433.12346752)(287.24413779,433.31096734)(287.24414062,433.537534)
\moveto(288.0234375,437.11175275)
\curveto(287.47265319,437.11174479)(287.06054423,436.81487008)(286.78710938,436.22112775)
\curveto(286.51757602,435.62737127)(286.38281053,434.7230753)(286.3828125,433.50823713)
\curveto(286.38281053,432.29729647)(286.51757602,431.39495363)(286.78710938,430.80120588)
\curveto(287.06054423,430.20745481)(287.47265319,429.91058011)(288.0234375,429.91058088)
\curveto(288.57812084,429.91058011)(288.9902298,430.20745481)(289.25976562,430.80120588)
\curveto(289.53319801,431.39495363)(289.66991662,432.29729647)(289.66992188,433.50823713)
\curveto(289.66991662,434.7230753)(289.53319801,435.62737127)(289.25976562,436.22112775)
\curveto(288.9902298,436.81487008)(288.57812084,437.11174479)(288.0234375,437.11175275)
\moveto(288.0234375,438.04925275)
\curveto(288.95702671,438.04924385)(289.66210413,437.66643173)(290.13867188,436.90081525)
\curveto(290.61913442,436.13518326)(290.85936855,435.00432502)(290.859375,433.50823713)
\curveto(290.85936855,432.01604676)(290.61913442,430.88714163)(290.13867188,430.12151838)
\curveto(289.66210413,429.35589317)(288.95702671,428.97308105)(288.0234375,428.97308088)
\curveto(287.08984107,428.97308105)(286.38476365,429.35589317)(285.90820312,430.12151838)
\curveto(285.43163961,430.88714163)(285.1933586,432.01604676)(285.19335938,433.50823713)
\curveto(285.1933586,435.00432502)(285.43163961,436.13518326)(285.90820312,436.90081525)
\curveto(286.38476365,437.66643173)(287.08984107,438.04924385)(288.0234375,438.04925275)
}
}
{
\newrgbcolor{curcolor}{0 0 0}
\pscustom[linestyle=none,fillstyle=solid,fillcolor=curcolor]
{
\newpath
\moveto(319.88812256,430.2953465)
\lineto(321.72796631,430.2953465)
\lineto(321.72796631,436.98089338)
\lineto(319.74749756,436.53558088)
\lineto(319.74749756,437.61370588)
\lineto(321.71624756,438.04729963)
\lineto(322.89984131,438.04729963)
\lineto(322.89984131,430.2953465)
\lineto(324.71624756,430.2953465)
\lineto(324.71624756,429.29925275)
\lineto(319.88812256,429.29925275)
\lineto(319.88812256,430.2953465)
}
}
{
\newrgbcolor{curcolor}{0 0 0}
\pscustom[linestyle=none,fillstyle=solid,fillcolor=curcolor]
{
\newpath
\moveto(222.92411804,412.91900373)
\lineto(226.94364929,412.91900373)
\lineto(226.94364929,411.92290998)
\lineto(221.62919617,411.92290998)
\lineto(221.62919617,412.91900373)
\curveto(222.3596633,413.68853321)(222.99833453,414.36822004)(223.54521179,414.95806623)
\curveto(224.09208344,415.54790636)(224.46903619,415.96392156)(224.67607117,416.20611311)
\curveto(225.06669184,416.68267085)(225.33036345,417.06743609)(225.46708679,417.36040998)
\curveto(225.60380068,417.65727925)(225.67215998,417.96001332)(225.67216492,418.26861311)
\curveto(225.67215998,418.75688752)(225.52762888,419.13969964)(225.23857117,419.41705061)
\curveto(224.9534107,419.69438658)(224.56083297,419.83305832)(224.06083679,419.83306623)
\curveto(223.70536508,419.83305832)(223.33231857,419.76860526)(222.94169617,419.63970686)
\curveto(222.55106935,419.51079302)(222.13700727,419.31548071)(221.69950867,419.05376936)
\lineto(221.69950867,420.24908186)
\curveto(222.10185105,420.44047959)(222.49638191,420.58501069)(222.88310242,420.68267561)
\curveto(223.27372488,420.780323)(223.65849012,420.82915107)(224.03739929,420.82915998)
\curveto(224.89286389,420.82915107)(225.5803632,420.60063568)(226.09989929,420.14361311)
\curveto(226.62333091,419.69048034)(226.8850494,419.09477781)(226.88505554,418.35650373)
\curveto(226.8850494,417.98149767)(226.79715886,417.60649805)(226.62138367,417.23150373)
\curveto(226.44950296,416.8564988)(226.16825324,416.44243671)(225.77763367,415.98931623)
\curveto(225.55887885,415.73540617)(225.24051979,415.38384402)(224.82255554,414.93462873)
\curveto(224.40848937,414.48540742)(223.7756775,413.81353309)(222.92411804,412.91900373)
}
}
{
\newrgbcolor{curcolor}{0 0 0}
\pscustom[linestyle=none,fillstyle=solid,fillcolor=curcolor]
{
\newpath
\moveto(255.52357483,416.65289045)
\curveto(255.523572,416.86772959)(255.59779067,417.05327628)(255.74623108,417.20953107)
\curveto(255.89857162,417.36577597)(256.08021207,417.44390089)(256.29115295,417.44390607)
\curveto(256.50989914,417.44390089)(256.69739895,417.36577597)(256.85365295,417.20953107)
\curveto(257.00989864,417.05327628)(257.08802356,416.86772959)(257.08802795,416.65289045)
\curveto(257.08802356,416.43413627)(257.00989864,416.24858958)(256.85365295,416.09624982)
\curveto(256.7013052,415.94390239)(256.51380538,415.86773059)(256.29115295,415.8677342)
\curveto(256.07239958,415.86773059)(255.88880601,415.94194927)(255.7403717,416.09039045)
\curveto(255.59583755,416.23882397)(255.523572,416.42632378)(255.52357483,416.65289045)
\moveto(256.3028717,420.2271092)
\curveto(255.7520874,420.22710123)(255.33997843,419.93022653)(255.06654358,419.3364842)
\curveto(254.79701023,418.74272771)(254.66224474,417.83843174)(254.6622467,416.62359357)
\curveto(254.66224474,415.41265292)(254.79701023,414.51031007)(255.06654358,413.91656232)
\curveto(255.33997843,413.32281126)(255.7520874,413.02593656)(256.3028717,413.02593732)
\curveto(256.85755504,413.02593656)(257.269664,413.32281126)(257.53919983,413.91656232)
\curveto(257.81263221,414.51031007)(257.94935082,415.41265292)(257.94935608,416.62359357)
\curveto(257.94935082,417.83843174)(257.81263221,418.74272771)(257.53919983,419.3364842)
\curveto(257.269664,419.93022653)(256.85755504,420.22710123)(256.3028717,420.2271092)
\moveto(256.3028717,421.1646092)
\curveto(257.23646091,421.16460029)(257.94153833,420.78178818)(258.41810608,420.0161717)
\curveto(258.89856862,419.25053971)(259.13880276,418.11968146)(259.1388092,416.62359357)
\curveto(259.13880276,415.1314032)(258.89856862,414.00249808)(258.41810608,413.23687482)
\curveto(257.94153833,412.47124961)(257.23646091,412.08843749)(256.3028717,412.08843732)
\curveto(255.36927528,412.08843749)(254.66419786,412.47124961)(254.18763733,413.23687482)
\curveto(253.71107381,414.00249808)(253.4727928,415.1314032)(253.47279358,416.62359357)
\curveto(253.4727928,418.11968146)(253.71107381,419.25053971)(254.18763733,420.0161717)
\curveto(254.66419786,420.78178818)(255.36927528,421.16460029)(256.3028717,421.1646092)
}
}
{
\newrgbcolor{curcolor}{0 0 0}
\pscustom[linestyle=none,fillstyle=solid,fillcolor=curcolor]
{
\newpath
\moveto(287.24414062,416.51934553)
\curveto(287.24413779,416.73418467)(287.31835647,416.91973136)(287.46679688,417.07598615)
\curveto(287.61913742,417.23223104)(287.80077786,417.31035597)(288.01171875,417.31036115)
\curveto(288.23046493,417.31035597)(288.41796475,417.23223104)(288.57421875,417.07598615)
\curveto(288.73046443,416.91973136)(288.80858936,416.73418467)(288.80859375,416.51934553)
\curveto(288.80858936,416.30059135)(288.73046443,416.11504466)(288.57421875,415.9627049)
\curveto(288.42187099,415.81035747)(288.23437118,415.73418567)(288.01171875,415.73418928)
\curveto(287.79296537,415.73418567)(287.6093718,415.80840434)(287.4609375,415.95684553)
\curveto(287.31640335,416.10527905)(287.24413779,416.29277886)(287.24414062,416.51934553)
\moveto(288.0234375,420.09356428)
\curveto(287.47265319,420.09355631)(287.06054423,419.79668161)(286.78710938,419.20293928)
\curveto(286.51757602,418.60918279)(286.38281053,417.70488682)(286.3828125,416.49004865)
\curveto(286.38281053,415.279108)(286.51757602,414.37676515)(286.78710938,413.7830174)
\curveto(287.06054423,413.18926634)(287.47265319,412.89239163)(288.0234375,412.8923924)
\curveto(288.57812084,412.89239163)(288.9902298,413.18926634)(289.25976562,413.7830174)
\curveto(289.53319801,414.37676515)(289.66991662,415.279108)(289.66992188,416.49004865)
\curveto(289.66991662,417.70488682)(289.53319801,418.60918279)(289.25976562,419.20293928)
\curveto(288.9902298,419.79668161)(288.57812084,420.09355631)(288.0234375,420.09356428)
\moveto(288.0234375,421.03106428)
\curveto(288.95702671,421.03105537)(289.66210413,420.64824325)(290.13867188,419.88262678)
\curveto(290.61913442,419.11699479)(290.85936855,417.98613654)(290.859375,416.49004865)
\curveto(290.85936855,414.99785828)(290.61913442,413.86895316)(290.13867188,413.1033299)
\curveto(289.66210413,412.33770469)(288.95702671,411.95489257)(288.0234375,411.9548924)
\curveto(287.08984107,411.95489257)(286.38476365,412.33770469)(285.90820312,413.1033299)
\curveto(285.43163961,413.86895316)(285.1933586,414.99785828)(285.19335938,416.49004865)
\curveto(285.1933586,417.98613654)(285.43163961,419.11699479)(285.90820312,419.88262678)
\curveto(286.38476365,420.64824325)(287.08984107,421.03105537)(288.0234375,421.03106428)
}
}
{
\newrgbcolor{curcolor}{0 0 0}
\pscustom[linestyle=none,fillstyle=solid,fillcolor=curcolor]
{
\newpath
\moveto(322.37249756,420.08184553)
\lineto(319.61273193,415.45879865)
\lineto(322.37249756,415.45879865)
\lineto(322.37249756,420.08184553)
\moveto(322.17913818,421.15997053)
\lineto(323.55023193,421.15997053)
\lineto(323.55023193,415.45879865)
\lineto(324.71624756,415.45879865)
\lineto(324.71624756,414.49786115)
\lineto(323.55023193,414.49786115)
\lineto(323.55023193,412.41192365)
\lineto(322.37249756,412.41192365)
\lineto(322.37249756,414.49786115)
\lineto(318.66351318,414.49786115)
\lineto(318.66351318,415.61700178)
\lineto(322.17913818,421.15997053)
}
}
{
\newrgbcolor{curcolor}{0 0 0}
\pscustom[linestyle=none,fillstyle=solid,fillcolor=curcolor]
{
\newpath
\moveto(223.32841492,399.43560529)
\curveto(223.32841209,399.65044443)(223.40263076,399.83599112)(223.55107117,399.99224592)
\curveto(223.70341171,400.14849081)(223.88505216,400.22661573)(224.09599304,400.22662092)
\curveto(224.31473923,400.22661573)(224.50223904,400.14849081)(224.65849304,399.99224592)
\curveto(224.81473873,399.83599112)(224.89286365,399.65044443)(224.89286804,399.43560529)
\curveto(224.89286365,399.21685112)(224.81473873,399.03130443)(224.65849304,398.87896467)
\curveto(224.50614528,398.72661723)(224.31864547,398.65044543)(224.09599304,398.65044904)
\curveto(223.87723966,398.65044543)(223.6936461,398.72466411)(223.54521179,398.87310529)
\curveto(223.40067764,399.02153881)(223.32841209,399.20903863)(223.32841492,399.43560529)
\moveto(224.10771179,403.00982404)
\curveto(223.55692748,403.00981607)(223.14481852,402.71294137)(222.87138367,402.11919904)
\curveto(222.60185031,401.52544256)(222.46708482,400.62114659)(222.46708679,399.40630842)
\curveto(222.46708482,398.19536776)(222.60185031,397.29302492)(222.87138367,396.69927717)
\curveto(223.14481852,396.1055261)(223.55692748,395.8086514)(224.10771179,395.80865217)
\curveto(224.66239513,395.8086514)(225.07450409,396.1055261)(225.34403992,396.69927717)
\curveto(225.6174723,397.29302492)(225.75419091,398.19536776)(225.75419617,399.40630842)
\curveto(225.75419091,400.62114659)(225.6174723,401.52544256)(225.34403992,402.11919904)
\curveto(225.07450409,402.71294137)(224.66239513,403.00981607)(224.10771179,403.00982404)
\moveto(224.10771179,403.94732404)
\curveto(225.041301,403.94731514)(225.74637842,403.56450302)(226.22294617,402.79888654)
\curveto(226.70340871,402.03325455)(226.94364285,400.90239631)(226.94364929,399.40630842)
\curveto(226.94364285,397.91411804)(226.70340871,396.78521292)(226.22294617,396.01958967)
\curveto(225.74637842,395.25396446)(225.041301,394.87115234)(224.10771179,394.87115217)
\curveto(223.17411537,394.87115234)(222.46903795,395.25396446)(221.99247742,396.01958967)
\curveto(221.5159139,396.78521292)(221.27763289,397.91411804)(221.27763367,399.40630842)
\curveto(221.27763289,400.90239631)(221.5159139,402.03325455)(221.99247742,402.79888654)
\curveto(222.46903795,403.56450302)(223.17411537,403.94731514)(224.10771179,403.94732404)
}
}
{
\newrgbcolor{curcolor}{0 0 0}
\pscustom[linestyle=none,fillstyle=solid,fillcolor=curcolor]
{
\newpath
\moveto(255.11927795,396.23016096)
\lineto(259.1388092,396.23016096)
\lineto(259.1388092,395.23406721)
\lineto(253.82435608,395.23406721)
\lineto(253.82435608,396.23016096)
\curveto(254.55482321,396.99969044)(255.19349444,397.67937726)(255.7403717,398.26922346)
\curveto(256.28724335,398.85906358)(256.6641961,399.27507879)(256.87123108,399.51727033)
\curveto(257.26185175,399.99382807)(257.52552336,400.37859331)(257.6622467,400.67156721)
\curveto(257.79896059,400.96843647)(257.8673199,401.27117054)(257.86732483,401.57977033)
\curveto(257.8673199,402.06804475)(257.72278879,402.45085687)(257.43373108,402.72820783)
\curveto(257.14857061,403.00554381)(256.75599288,403.14421555)(256.2559967,403.14422346)
\curveto(255.90052499,403.14421555)(255.52747849,403.07976249)(255.13685608,402.95086408)
\curveto(254.74622927,402.82195024)(254.33216718,402.62663794)(253.89466858,402.36492658)
\lineto(253.89466858,403.56023908)
\curveto(254.29701097,403.75163681)(254.69154182,403.89616792)(255.07826233,403.99383283)
\curveto(255.46888479,404.09148022)(255.85365003,404.1403083)(256.2325592,404.14031721)
\curveto(257.0880238,404.1403083)(257.77552311,403.9117929)(258.2950592,403.45477033)
\curveto(258.81849082,403.00163756)(259.08020931,402.40593504)(259.08021545,401.66766096)
\curveto(259.08020931,401.2926549)(258.99231877,400.91765527)(258.81654358,400.54266096)
\curveto(258.64466287,400.16765602)(258.36341315,399.75359394)(257.97279358,399.30047346)
\curveto(257.75403876,399.04656339)(257.4356797,398.69500125)(257.01771545,398.24578596)
\curveto(256.60364928,397.79656464)(255.97083742,397.12469032)(255.11927795,396.23016096)
}
}
{
\newrgbcolor{curcolor}{0 0 0}
\pscustom[linestyle=none,fillstyle=solid,fillcolor=curcolor]
{
\newpath
\moveto(287.24414062,399.50115705)
\curveto(287.24413779,399.71599619)(287.31835647,399.90154288)(287.46679688,400.05779768)
\curveto(287.61913742,400.21404257)(287.80077786,400.29216749)(288.01171875,400.29217268)
\curveto(288.23046493,400.29216749)(288.41796475,400.21404257)(288.57421875,400.05779768)
\curveto(288.73046443,399.90154288)(288.80858936,399.71599619)(288.80859375,399.50115705)
\curveto(288.80858936,399.28240288)(288.73046443,399.09685619)(288.57421875,398.94451643)
\curveto(288.42187099,398.79216899)(288.23437118,398.71599719)(288.01171875,398.7160008)
\curveto(287.79296537,398.71599719)(287.6093718,398.79021587)(287.4609375,398.93865705)
\curveto(287.31640335,399.08709057)(287.24413779,399.27459038)(287.24414062,399.50115705)
\moveto(288.0234375,403.0753758)
\curveto(287.47265319,403.07536783)(287.06054423,402.77849313)(286.78710938,402.1847508)
\curveto(286.51757602,401.59099432)(286.38281053,400.68669835)(286.3828125,399.47186018)
\curveto(286.38281053,398.26091952)(286.51757602,397.35857667)(286.78710938,396.76482893)
\curveto(287.06054423,396.17107786)(287.47265319,395.87420316)(288.0234375,395.87420393)
\curveto(288.57812084,395.87420316)(288.9902298,396.17107786)(289.25976562,396.76482893)
\curveto(289.53319801,397.35857667)(289.66991662,398.26091952)(289.66992188,399.47186018)
\curveto(289.66991662,400.68669835)(289.53319801,401.59099432)(289.25976562,402.1847508)
\curveto(288.9902298,402.77849313)(288.57812084,403.07536783)(288.0234375,403.0753758)
\moveto(288.0234375,404.0128758)
\curveto(288.95702671,404.01286689)(289.66210413,403.63005478)(290.13867188,402.8644383)
\curveto(290.61913442,402.09880631)(290.85936855,400.96794806)(290.859375,399.47186018)
\curveto(290.85936855,397.9796698)(290.61913442,396.85076468)(290.13867188,396.08514143)
\curveto(289.66210413,395.31951621)(288.95702671,394.9367041)(288.0234375,394.93670393)
\curveto(287.08984107,394.9367041)(286.38476365,395.31951621)(285.90820312,396.08514143)
\curveto(285.43163961,396.85076468)(285.1933586,397.9796698)(285.19335938,399.47186018)
\curveto(285.1933586,400.96794806)(285.43163961,402.09880631)(285.90820312,402.8644383)
\curveto(286.38476365,403.63005478)(287.08984107,404.01286689)(288.0234375,404.0128758)
}
}
{
\newrgbcolor{curcolor}{0 0 0}
\pscustom[linestyle=none,fillstyle=solid,fillcolor=curcolor]
{
\newpath
\moveto(312.63421631,396.36248518)
\lineto(314.47406006,396.36248518)
\lineto(314.47406006,403.04803205)
\lineto(312.49359131,402.60271955)
\lineto(312.49359131,403.68084455)
\lineto(314.46234131,404.1144383)
\lineto(315.64593506,404.1144383)
\lineto(315.64593506,396.36248518)
\lineto(317.46234131,396.36248518)
\lineto(317.46234131,395.36639143)
\lineto(312.63421631,395.36639143)
\lineto(312.63421631,396.36248518)
}
}
{
\newrgbcolor{curcolor}{0 0 0}
\pscustom[linestyle=none,fillstyle=solid,fillcolor=curcolor]
{
\newpath
\moveto(321.10101318,399.76092268)
\curveto(321.10101035,399.97576182)(321.17522903,400.16130851)(321.32366943,400.3175633)
\curveto(321.47600998,400.47380819)(321.65765042,400.55193312)(321.86859131,400.5519383)
\curveto(322.08733749,400.55193312)(322.2748373,400.47380819)(322.43109131,400.3175633)
\curveto(322.58733699,400.16130851)(322.66546191,399.97576182)(322.66546631,399.76092268)
\curveto(322.66546191,399.5421685)(322.58733699,399.35662181)(322.43109131,399.20428205)
\curveto(322.27874355,399.05193462)(322.09124374,398.97576282)(321.86859131,398.97576643)
\curveto(321.64983793,398.97576282)(321.46624436,399.04998149)(321.31781006,399.19842268)
\curveto(321.17327591,399.3468562)(321.10101035,399.53435601)(321.10101318,399.76092268)
\moveto(321.88031006,403.33514143)
\curveto(321.32952575,403.33513346)(320.91741679,403.03825875)(320.64398193,402.44451643)
\curveto(320.37444858,401.85075994)(320.23968309,400.94646397)(320.23968506,399.7316258)
\curveto(320.23968309,398.52068515)(320.37444858,397.6183423)(320.64398193,397.02459455)
\curveto(320.91741679,396.43084349)(321.32952575,396.13396878)(321.88031006,396.13396955)
\curveto(322.43499339,396.13396878)(322.84710236,396.43084349)(323.11663818,397.02459455)
\curveto(323.39007056,397.6183423)(323.52678918,398.52068515)(323.52679443,399.7316258)
\curveto(323.52678918,400.94646397)(323.39007056,401.85075994)(323.11663818,402.44451643)
\curveto(322.84710236,403.03825875)(322.43499339,403.33513346)(321.88031006,403.33514143)
\moveto(321.88031006,404.27264143)
\curveto(322.81389927,404.27263252)(323.51897669,403.8898204)(323.99554443,403.12420393)
\curveto(324.47600698,402.35857193)(324.71624111,401.22771369)(324.71624756,399.7316258)
\curveto(324.71624111,398.23943543)(324.47600698,397.11053031)(323.99554443,396.34490705)
\curveto(323.51897669,395.57928184)(322.81389927,395.19646972)(321.88031006,395.19646955)
\curveto(320.94671363,395.19646972)(320.24163621,395.57928184)(319.76507568,396.34490705)
\curveto(319.28851217,397.11053031)(319.05023115,398.23943543)(319.05023193,399.7316258)
\curveto(319.05023115,401.22771369)(319.28851217,402.35857193)(319.76507568,403.12420393)
\curveto(320.24163621,403.8898204)(320.94671363,404.27263252)(321.88031006,404.27264143)
}
}
{
\newrgbcolor{curcolor}{0 0 0}
\pscustom[linestyle=none,fillstyle=solid,fillcolor=curcolor]
{
\newpath
\moveto(223.32841492,382.38384748)
\curveto(223.32841209,382.59868662)(223.40263076,382.78423331)(223.55107117,382.94048811)
\curveto(223.70341171,383.096733)(223.88505216,383.17485792)(224.09599304,383.17486311)
\curveto(224.31473923,383.17485792)(224.50223904,383.096733)(224.65849304,382.94048811)
\curveto(224.81473873,382.78423331)(224.89286365,382.59868662)(224.89286804,382.38384748)
\curveto(224.89286365,382.1650933)(224.81473873,381.97954662)(224.65849304,381.82720686)
\curveto(224.50614528,381.67485942)(224.31864547,381.59868762)(224.09599304,381.59869123)
\curveto(223.87723966,381.59868762)(223.6936461,381.6729063)(223.54521179,381.82134748)
\curveto(223.40067764,381.969781)(223.32841209,382.15728081)(223.32841492,382.38384748)
\moveto(224.10771179,385.95806623)
\curveto(223.55692748,385.95805826)(223.14481852,385.66118356)(222.87138367,385.06744123)
\curveto(222.60185031,384.47368475)(222.46708482,383.56938878)(222.46708679,382.35455061)
\curveto(222.46708482,381.14360995)(222.60185031,380.2412671)(222.87138367,379.64751936)
\curveto(223.14481852,379.05376829)(223.55692748,378.75689359)(224.10771179,378.75689436)
\curveto(224.66239513,378.75689359)(225.07450409,379.05376829)(225.34403992,379.64751936)
\curveto(225.6174723,380.2412671)(225.75419091,381.14360995)(225.75419617,382.35455061)
\curveto(225.75419091,383.56938878)(225.6174723,384.47368475)(225.34403992,385.06744123)
\curveto(225.07450409,385.66118356)(224.66239513,385.95805826)(224.10771179,385.95806623)
\moveto(224.10771179,386.89556623)
\curveto(225.041301,386.89555732)(225.74637842,386.51274521)(226.22294617,385.74712873)
\curveto(226.70340871,384.98149674)(226.94364285,383.85063849)(226.94364929,382.35455061)
\curveto(226.94364285,380.86236023)(226.70340871,379.73345511)(226.22294617,378.96783186)
\curveto(225.74637842,378.20220664)(225.041301,377.81939453)(224.10771179,377.81939436)
\curveto(223.17411537,377.81939453)(222.46903795,378.20220664)(221.99247742,378.96783186)
\curveto(221.5159139,379.73345511)(221.27763289,380.86236023)(221.27763367,382.35455061)
\curveto(221.27763289,383.85063849)(221.5159139,384.98149674)(221.99247742,385.74712873)
\curveto(222.46903795,386.51274521)(223.17411537,386.89555732)(224.10771179,386.89556623)
}
}
{
\newrgbcolor{curcolor}{0 0 0}
\pscustom[linestyle=none,fillstyle=solid,fillcolor=curcolor]
{
\newpath
\moveto(255.52357483,382.77435041)
\curveto(255.523572,382.98918955)(255.59779067,383.17473624)(255.74623108,383.33099104)
\curveto(255.89857162,383.48723593)(256.08021207,383.56536085)(256.29115295,383.56536604)
\curveto(256.50989914,383.56536085)(256.69739895,383.48723593)(256.85365295,383.33099104)
\curveto(257.00989864,383.17473624)(257.08802356,382.98918955)(257.08802795,382.77435041)
\curveto(257.08802356,382.55559623)(257.00989864,382.37004954)(256.85365295,382.21770979)
\curveto(256.7013052,382.06536235)(256.51380538,381.98919055)(256.29115295,381.98919416)
\curveto(256.07239958,381.98919055)(255.88880601,382.06340923)(255.7403717,382.21185041)
\curveto(255.59583755,382.36028393)(255.523572,382.54778374)(255.52357483,382.77435041)
\moveto(256.3028717,386.34856916)
\curveto(255.7520874,386.34856119)(255.33997843,386.05168649)(255.06654358,385.45794416)
\curveto(254.79701023,384.86418768)(254.66224474,383.95989171)(254.6622467,382.74505354)
\curveto(254.66224474,381.53411288)(254.79701023,380.63177003)(255.06654358,380.03802229)
\curveto(255.33997843,379.44427122)(255.7520874,379.14739652)(256.3028717,379.14739729)
\curveto(256.85755504,379.14739652)(257.269664,379.44427122)(257.53919983,380.03802229)
\curveto(257.81263221,380.63177003)(257.94935082,381.53411288)(257.94935608,382.74505354)
\curveto(257.94935082,383.95989171)(257.81263221,384.86418768)(257.53919983,385.45794416)
\curveto(257.269664,386.05168649)(256.85755504,386.34856119)(256.3028717,386.34856916)
\moveto(256.3028717,387.28606916)
\curveto(257.23646091,387.28606025)(257.94153833,386.90324814)(258.41810608,386.13763166)
\curveto(258.89856862,385.37199967)(259.13880276,384.24114142)(259.1388092,382.74505354)
\curveto(259.13880276,381.25286316)(258.89856862,380.12395804)(258.41810608,379.35833479)
\curveto(257.94153833,378.59270957)(257.23646091,378.20989746)(256.3028717,378.20989729)
\curveto(255.36927528,378.20989746)(254.66419786,378.59270957)(254.18763733,379.35833479)
\curveto(253.71107381,380.12395804)(253.4727928,381.25286316)(253.47279358,382.74505354)
\curveto(253.4727928,384.24114142)(253.71107381,385.37199967)(254.18763733,386.13763166)
\curveto(254.66419786,386.90324814)(255.36927528,387.28606025)(256.3028717,387.28606916)
}
}
{
\newrgbcolor{curcolor}{0 0 0}
\pscustom[linestyle=none,fillstyle=solid,fillcolor=curcolor]
{
\newpath
\moveto(287.24414062,382.48296857)
\curveto(287.24413779,382.69780771)(287.31835647,382.8833544)(287.46679688,383.0396092)
\curveto(287.61913742,383.19585409)(287.80077786,383.27397901)(288.01171875,383.2739842)
\curveto(288.23046493,383.27397901)(288.41796475,383.19585409)(288.57421875,383.0396092)
\curveto(288.73046443,382.8833544)(288.80858936,382.69780771)(288.80859375,382.48296857)
\curveto(288.80858936,382.2642144)(288.73046443,382.07866771)(288.57421875,381.92632795)
\curveto(288.42187099,381.77398051)(288.23437118,381.69780871)(288.01171875,381.69781232)
\curveto(287.79296537,381.69780871)(287.6093718,381.77202739)(287.4609375,381.92046857)
\curveto(287.31640335,382.06890209)(287.24413779,382.25640191)(287.24414062,382.48296857)
\moveto(288.0234375,386.05718732)
\curveto(287.47265319,386.05717936)(287.06054423,385.76030465)(286.78710938,385.16656232)
\curveto(286.51757602,384.57280584)(286.38281053,383.66850987)(286.3828125,382.4536717)
\curveto(286.38281053,381.24273104)(286.51757602,380.3403882)(286.78710938,379.74664045)
\curveto(287.06054423,379.15288938)(287.47265319,378.85601468)(288.0234375,378.85601545)
\curveto(288.57812084,378.85601468)(288.9902298,379.15288938)(289.25976562,379.74664045)
\curveto(289.53319801,380.3403882)(289.66991662,381.24273104)(289.66992188,382.4536717)
\curveto(289.66991662,383.66850987)(289.53319801,384.57280584)(289.25976562,385.16656232)
\curveto(288.9902298,385.76030465)(288.57812084,386.05717936)(288.0234375,386.05718732)
\moveto(288.0234375,386.99468732)
\curveto(288.95702671,386.99467842)(289.66210413,386.6118663)(290.13867188,385.84624982)
\curveto(290.61913442,385.08061783)(290.85936855,383.94975959)(290.859375,382.4536717)
\curveto(290.85936855,380.96148133)(290.61913442,379.83257621)(290.13867188,379.06695295)
\curveto(289.66210413,378.30132774)(288.95702671,377.91851562)(288.0234375,377.91851545)
\curveto(287.08984107,377.91851562)(286.38476365,378.30132774)(285.90820312,379.06695295)
\curveto(285.43163961,379.83257621)(285.1933586,380.96148133)(285.19335938,382.4536717)
\curveto(285.1933586,383.94975959)(285.43163961,385.08061783)(285.90820312,385.84624982)
\curveto(286.38476365,386.6118663)(287.08984107,386.99467842)(288.0234375,386.99468732)
}
}
{
\newrgbcolor{curcolor}{0 0 0}
\pscustom[linestyle=none,fillstyle=solid,fillcolor=curcolor]
{
\newpath
\moveto(320.69671631,379.14703107)
\lineto(324.71624756,379.14703107)
\lineto(324.71624756,378.15093732)
\lineto(319.40179443,378.15093732)
\lineto(319.40179443,379.14703107)
\curveto(320.13226156,379.91656056)(320.7709328,380.59624738)(321.31781006,381.18609357)
\curveto(321.86468171,381.7759337)(322.24163445,382.19194891)(322.44866943,382.43414045)
\curveto(322.83929011,382.91069819)(323.10296172,383.29546343)(323.23968506,383.58843732)
\curveto(323.37639894,383.88530659)(323.44475825,384.18804066)(323.44476318,384.49664045)
\curveto(323.44475825,384.98491487)(323.30022714,385.36772698)(323.01116943,385.64507795)
\curveto(322.72600897,385.92241393)(322.33343124,386.06108566)(321.83343506,386.06109357)
\curveto(321.47796334,386.06108566)(321.10491684,385.9966326)(320.71429443,385.8677342)
\curveto(320.32366762,385.73882036)(319.90960554,385.54350806)(319.47210693,385.2817967)
\lineto(319.47210693,386.4771092)
\curveto(319.87444932,386.66850693)(320.26898018,386.81303804)(320.65570068,386.91070295)
\curveto(321.04632315,387.00835034)(321.43108839,387.05717842)(321.80999756,387.05718732)
\curveto(322.66546215,387.05717842)(323.35296147,386.82866302)(323.87249756,386.37164045)
\curveto(324.39592917,385.91850768)(324.65764766,385.32280515)(324.65765381,384.58453107)
\curveto(324.65764766,384.20952502)(324.56975713,383.83452539)(324.39398193,383.45953107)
\curveto(324.22210122,383.08452614)(323.9408515,382.67046405)(323.55023193,382.21734357)
\curveto(323.33147711,381.96343351)(323.01311806,381.61187136)(322.59515381,381.16265607)
\curveto(322.18108764,380.71343476)(321.54827577,380.04156043)(320.69671631,379.14703107)
}
}
{
\newrgbcolor{curcolor}{0 0 0}
\pscustom[linestyle=none,fillstyle=solid,fillcolor=curcolor]
{
\newpath
\moveto(222.92411804,361.93365217)
\lineto(226.94364929,361.93365217)
\lineto(226.94364929,360.93755842)
\lineto(221.62919617,360.93755842)
\lineto(221.62919617,361.93365217)
\curveto(222.3596633,362.70318165)(222.99833453,363.38286847)(223.54521179,363.97271467)
\curveto(224.09208344,364.56255479)(224.46903619,364.97857)(224.67607117,365.22076154)
\curveto(225.06669184,365.69731928)(225.33036345,366.08208452)(225.46708679,366.37505842)
\curveto(225.60380068,366.67192768)(225.67215998,366.97466176)(225.67216492,367.28326154)
\curveto(225.67215998,367.77153596)(225.52762888,368.15434808)(225.23857117,368.43169904)
\curveto(224.9534107,368.70903502)(224.56083297,368.84770676)(224.06083679,368.84771467)
\curveto(223.70536508,368.84770676)(223.33231857,368.7832537)(222.94169617,368.65435529)
\curveto(222.55106935,368.52544146)(222.13700727,368.33012915)(221.69950867,368.06841779)
\lineto(221.69950867,369.26373029)
\curveto(222.10185105,369.45512803)(222.49638191,369.59965913)(222.88310242,369.69732404)
\curveto(223.27372488,369.79497144)(223.65849012,369.84379951)(224.03739929,369.84380842)
\curveto(224.89286389,369.84379951)(225.5803632,369.61528412)(226.09989929,369.15826154)
\curveto(226.62333091,368.70512878)(226.8850494,368.10942625)(226.88505554,367.37115217)
\curveto(226.8850494,366.99614611)(226.79715886,366.62114648)(226.62138367,366.24615217)
\curveto(226.44950296,365.87114723)(226.16825324,365.45708515)(225.77763367,365.00396467)
\curveto(225.55887885,364.75005461)(225.24051979,364.39849246)(224.82255554,363.94927717)
\curveto(224.40848937,363.50005586)(223.7756775,362.82818153)(222.92411804,361.93365217)
}
}
{
\newrgbcolor{curcolor}{0 0 0}
\pscustom[linestyle=none,fillstyle=solid,fillcolor=curcolor]
{
\newpath
\moveto(255.52357483,365.75005842)
\curveto(255.523572,365.96489756)(255.59779067,366.15044425)(255.74623108,366.30669904)
\curveto(255.89857162,366.46294394)(256.08021207,366.54106886)(256.29115295,366.54107404)
\curveto(256.50989914,366.54106886)(256.69739895,366.46294394)(256.85365295,366.30669904)
\curveto(257.00989864,366.15044425)(257.08802356,365.96489756)(257.08802795,365.75005842)
\curveto(257.08802356,365.53130424)(257.00989864,365.34575755)(256.85365295,365.19341779)
\curveto(256.7013052,365.04107036)(256.51380538,364.96489856)(256.29115295,364.96490217)
\curveto(256.07239958,364.96489856)(255.88880601,365.03911723)(255.7403717,365.18755842)
\curveto(255.59583755,365.33599194)(255.523572,365.52349175)(255.52357483,365.75005842)
\moveto(256.3028717,369.32427717)
\curveto(255.7520874,369.3242692)(255.33997843,369.0273945)(255.06654358,368.43365217)
\curveto(254.79701023,367.83989568)(254.66224474,366.93559971)(254.6622467,365.72076154)
\curveto(254.66224474,364.50982089)(254.79701023,363.60747804)(255.06654358,363.01373029)
\curveto(255.33997843,362.41997923)(255.7520874,362.12310453)(256.3028717,362.12310529)
\curveto(256.85755504,362.12310453)(257.269664,362.41997923)(257.53919983,363.01373029)
\curveto(257.81263221,363.60747804)(257.94935082,364.50982089)(257.94935608,365.72076154)
\curveto(257.94935082,366.93559971)(257.81263221,367.83989568)(257.53919983,368.43365217)
\curveto(257.269664,369.0273945)(256.85755504,369.3242692)(256.3028717,369.32427717)
\moveto(256.3028717,370.26177717)
\curveto(257.23646091,370.26176826)(257.94153833,369.87895614)(258.41810608,369.11333967)
\curveto(258.89856862,368.34770768)(259.13880276,367.21684943)(259.1388092,365.72076154)
\curveto(259.13880276,364.22857117)(258.89856862,363.09966605)(258.41810608,362.33404279)
\curveto(257.94153833,361.56841758)(257.23646091,361.18560546)(256.3028717,361.18560529)
\curveto(255.36927528,361.18560546)(254.66419786,361.56841758)(254.18763733,362.33404279)
\curveto(253.71107381,363.09966605)(253.4727928,364.22857117)(253.47279358,365.72076154)
\curveto(253.4727928,367.21684943)(253.71107381,368.34770768)(254.18763733,369.11333967)
\curveto(254.66419786,369.87895614)(255.36927528,370.26176826)(256.3028717,370.26177717)
}
}
{
\newrgbcolor{curcolor}{0 0 0}
\pscustom[linestyle=none,fillstyle=solid,fillcolor=curcolor]
{
\newpath
\moveto(287.24414062,365.4647801)
\curveto(287.24413779,365.67961924)(287.31835647,365.86516593)(287.46679688,366.02142072)
\curveto(287.61913742,366.17766562)(287.80077786,366.25579054)(288.01171875,366.25579572)
\curveto(288.23046493,366.25579054)(288.41796475,366.17766562)(288.57421875,366.02142072)
\curveto(288.73046443,365.86516593)(288.80858936,365.67961924)(288.80859375,365.4647801)
\curveto(288.80858936,365.24602592)(288.73046443,365.06047923)(288.57421875,364.90813947)
\curveto(288.42187099,364.75579204)(288.23437118,364.67962024)(288.01171875,364.67962385)
\curveto(287.79296537,364.67962024)(287.6093718,364.75383891)(287.4609375,364.9022801)
\curveto(287.31640335,365.05071362)(287.24413779,365.23821343)(287.24414062,365.4647801)
\moveto(288.0234375,369.03899885)
\curveto(287.47265319,369.03899088)(287.06054423,368.74211618)(286.78710938,368.14837385)
\curveto(286.51757602,367.55461736)(286.38281053,366.65032139)(286.3828125,365.43548322)
\curveto(286.38281053,364.22454257)(286.51757602,363.32219972)(286.78710938,362.72845197)
\curveto(287.06054423,362.13470091)(287.47265319,361.83782621)(288.0234375,361.83782697)
\curveto(288.57812084,361.83782621)(288.9902298,362.13470091)(289.25976562,362.72845197)
\curveto(289.53319801,363.32219972)(289.66991662,364.22454257)(289.66992188,365.43548322)
\curveto(289.66991662,366.65032139)(289.53319801,367.55461736)(289.25976562,368.14837385)
\curveto(288.9902298,368.74211618)(288.57812084,369.03899088)(288.0234375,369.03899885)
\moveto(288.0234375,369.97649885)
\curveto(288.95702671,369.97648994)(289.66210413,369.59367782)(290.13867188,368.82806135)
\curveto(290.61913442,368.06242936)(290.85936855,366.93157111)(290.859375,365.43548322)
\curveto(290.85936855,363.94329285)(290.61913442,362.81438773)(290.13867188,362.04876447)
\curveto(289.66210413,361.28313926)(288.95702671,360.90032714)(288.0234375,360.90032697)
\curveto(287.08984107,360.90032714)(286.38476365,361.28313926)(285.90820312,362.04876447)
\curveto(285.43163961,362.81438773)(285.1933586,363.94329285)(285.19335938,365.43548322)
\curveto(285.1933586,366.93157111)(285.43163961,368.06242936)(285.90820312,368.82806135)
\curveto(286.38476365,369.59367782)(287.08984107,369.97648994)(288.0234375,369.97649885)
}
}
{
\newrgbcolor{curcolor}{0 0 0}
\pscustom[linestyle=none,fillstyle=solid,fillcolor=curcolor]
{
\newpath
\moveto(322.94085693,365.78704572)
\curveto(323.51507056,365.63469744)(323.95452325,365.36321334)(324.25921631,364.9725926)
\curveto(324.56389764,364.58587037)(324.71624124,364.10149585)(324.71624756,363.5194676)
\curveto(324.71624124,362.71477849)(324.44475713,362.08196662)(323.90179443,361.6210301)
\curveto(323.36272696,361.16399879)(322.61468084,360.93548339)(321.65765381,360.93548322)
\curveto(321.2553072,360.93548339)(320.84515136,360.97259273)(320.42718506,361.04681135)
\curveto(320.00921469,361.12103008)(319.59905885,361.22845185)(319.19671631,361.36907697)
\lineto(319.19671631,362.54681135)
\curveto(319.59515261,362.33977886)(319.98773034,362.18548214)(320.37445068,362.08392072)
\curveto(320.76116707,361.98235735)(321.14593231,361.93157615)(321.52874756,361.93157697)
\curveto(322.17718128,361.93157615)(322.67522765,362.07806038)(323.02288818,362.3710301)
\curveto(323.37053946,362.66399729)(323.54436741,363.08587187)(323.54437256,363.6366551)
\curveto(323.54436741,364.14446456)(323.37053946,364.54680791)(323.02288818,364.84368635)
\curveto(322.67522765,365.14446356)(322.204525,365.29485403)(321.61077881,365.29485822)
\lineto(320.70843506,365.29485822)
\lineto(320.70843506,366.26751447)
\lineto(321.61077881,366.26751447)
\curveto(322.1537438,366.26750931)(322.5775715,366.38664982)(322.88226318,366.62493635)
\curveto(323.18694589,366.86321184)(323.33928949,367.19524276)(323.33929443,367.6210301)
\curveto(323.33928949,368.07024188)(323.19671151,368.41399154)(322.91156006,368.6522801)
\curveto(322.63030582,368.89445981)(322.22796247,369.01555344)(321.70452881,369.01556135)
\curveto(321.3568696,369.01555344)(320.99749496,368.97649098)(320.62640381,368.89837385)
\curveto(320.2553082,368.82024113)(319.86663671,368.70305375)(319.46038818,368.54681135)
\lineto(319.46038818,369.6366551)
\curveto(319.93304289,369.76164644)(320.35296435,369.85539635)(320.72015381,369.9179051)
\curveto(321.09124486,369.98039622)(321.41936953,370.01164619)(321.70452881,370.0116551)
\curveto(322.55608715,370.01164619)(323.23577397,369.79680266)(323.74359131,369.36712385)
\curveto(324.2553042,368.94133476)(324.51116332,368.37492908)(324.51116943,367.6679051)
\curveto(324.51116332,367.18743027)(324.37639783,366.78704004)(324.10687256,366.46673322)
\curveto(323.84124211,366.14641568)(323.45257063,365.91985341)(322.94085693,365.78704572)
}
}
{
\newrgbcolor{curcolor}{0 0 0}
\pscustom[linestyle=none,fillstyle=solid,fillcolor=curcolor]
{
\newpath
\moveto(222.11552429,345.21001936)
\lineto(223.95536804,345.21001936)
\lineto(223.95536804,351.89556623)
\lineto(221.97489929,351.45025373)
\lineto(221.97489929,352.52837873)
\lineto(223.94364929,352.96197248)
\lineto(225.12724304,352.96197248)
\lineto(225.12724304,345.21001936)
\lineto(226.94364929,345.21001936)
\lineto(226.94364929,344.21392561)
\lineto(222.11552429,344.21392561)
\lineto(222.11552429,345.21001936)
}
}
{
\newrgbcolor{curcolor}{0 0 0}
\pscustom[linestyle=none,fillstyle=solid,fillcolor=curcolor]
{
\newpath
\moveto(255.52357483,348.7258885)
\curveto(255.523572,348.94072764)(255.59779067,349.12627433)(255.74623108,349.28252912)
\curveto(255.89857162,349.43877401)(256.08021207,349.51689894)(256.29115295,349.51690412)
\curveto(256.50989914,349.51689894)(256.69739895,349.43877401)(256.85365295,349.28252912)
\curveto(257.00989864,349.12627433)(257.08802356,348.94072764)(257.08802795,348.7258885)
\curveto(257.08802356,348.50713432)(257.00989864,348.32158763)(256.85365295,348.16924787)
\curveto(256.7013052,348.01690044)(256.51380538,347.94072864)(256.29115295,347.94073225)
\curveto(256.07239958,347.94072864)(255.88880601,348.01494731)(255.7403717,348.1633885)
\curveto(255.59583755,348.31182202)(255.523572,348.49932183)(255.52357483,348.7258885)
\moveto(256.3028717,352.30010725)
\curveto(255.7520874,352.30009928)(255.33997843,352.00322457)(255.06654358,351.40948225)
\curveto(254.79701023,350.81572576)(254.66224474,349.91142979)(254.6622467,348.69659162)
\curveto(254.66224474,347.48565097)(254.79701023,346.58330812)(255.06654358,345.98956037)
\curveto(255.33997843,345.39580931)(255.7520874,345.0989346)(256.3028717,345.09893537)
\curveto(256.85755504,345.0989346)(257.269664,345.39580931)(257.53919983,345.98956037)
\curveto(257.81263221,346.58330812)(257.94935082,347.48565097)(257.94935608,348.69659162)
\curveto(257.94935082,349.91142979)(257.81263221,350.81572576)(257.53919983,351.40948225)
\curveto(257.269664,352.00322457)(256.85755504,352.30009928)(256.3028717,352.30010725)
\moveto(256.3028717,353.23760725)
\curveto(257.23646091,353.23759834)(257.94153833,352.85478622)(258.41810608,352.08916975)
\curveto(258.89856862,351.32353775)(259.13880276,350.19267951)(259.1388092,348.69659162)
\curveto(259.13880276,347.20440125)(258.89856862,346.07549613)(258.41810608,345.30987287)
\curveto(257.94153833,344.54424766)(257.23646091,344.16143554)(256.3028717,344.16143537)
\curveto(255.36927528,344.16143554)(254.66419786,344.54424766)(254.18763733,345.30987287)
\curveto(253.71107381,346.07549613)(253.4727928,347.20440125)(253.47279358,348.69659162)
\curveto(253.4727928,350.19267951)(253.71107381,351.32353775)(254.18763733,352.08916975)
\curveto(254.66419786,352.85478622)(255.36927528,353.23759834)(256.3028717,353.23760725)
}
}
{
\newrgbcolor{curcolor}{0 0 0}
\pscustom[linestyle=none,fillstyle=solid,fillcolor=curcolor]
{
\newpath
\moveto(287.24414062,348.44659162)
\curveto(287.24413779,348.66143076)(287.31835647,348.84697745)(287.46679688,349.00323225)
\curveto(287.61913742,349.15947714)(287.80077786,349.23760206)(288.01171875,349.23760725)
\curveto(288.23046493,349.23760206)(288.41796475,349.15947714)(288.57421875,349.00323225)
\curveto(288.73046443,348.84697745)(288.80858936,348.66143076)(288.80859375,348.44659162)
\curveto(288.80858936,348.22783745)(288.73046443,348.04229076)(288.57421875,347.889951)
\curveto(288.42187099,347.73760356)(288.23437118,347.66143176)(288.01171875,347.66143537)
\curveto(287.79296537,347.66143176)(287.6093718,347.73565044)(287.4609375,347.88409162)
\curveto(287.31640335,348.03252514)(287.24413779,348.22002495)(287.24414062,348.44659162)
\moveto(288.0234375,352.02081037)
\curveto(287.47265319,352.0208024)(287.06054423,351.7239277)(286.78710938,351.13018537)
\curveto(286.51757602,350.53642889)(286.38281053,349.63213292)(286.3828125,348.41729475)
\curveto(286.38281053,347.20635409)(286.51757602,346.30401124)(286.78710938,345.7102635)
\curveto(287.06054423,345.11651243)(287.47265319,344.81963773)(288.0234375,344.8196385)
\curveto(288.57812084,344.81963773)(288.9902298,345.11651243)(289.25976562,345.7102635)
\curveto(289.53319801,346.30401124)(289.66991662,347.20635409)(289.66992188,348.41729475)
\curveto(289.66991662,349.63213292)(289.53319801,350.53642889)(289.25976562,351.13018537)
\curveto(288.9902298,351.7239277)(288.57812084,352.0208024)(288.0234375,352.02081037)
\moveto(288.0234375,352.95831037)
\curveto(288.95702671,352.95830146)(289.66210413,352.57548935)(290.13867188,351.80987287)
\curveto(290.61913442,351.04424088)(290.85936855,349.91338263)(290.859375,348.41729475)
\curveto(290.85936855,346.92510437)(290.61913442,345.79619925)(290.13867188,345.030576)
\curveto(289.66210413,344.26495078)(288.95702671,343.88213867)(288.0234375,343.8821385)
\curveto(287.08984107,343.88213867)(286.38476365,344.26495078)(285.90820312,345.030576)
\curveto(285.43163961,345.79619925)(285.1933586,346.92510437)(285.19335938,348.41729475)
\curveto(285.1933586,349.91338263)(285.43163961,351.04424088)(285.90820312,351.80987287)
\curveto(286.38476365,352.57548935)(287.08984107,352.95830146)(288.0234375,352.95831037)
}
}
{
\newrgbcolor{curcolor}{0 0 0}
\pscustom[linestyle=none,fillstyle=solid,fillcolor=curcolor]
{
\newpath
\moveto(320.69671631,344.88604475)
\lineto(324.71624756,344.88604475)
\lineto(324.71624756,343.889951)
\lineto(319.40179443,343.889951)
\lineto(319.40179443,344.88604475)
\curveto(320.13226156,345.65557423)(320.7709328,346.33526105)(321.31781006,346.92510725)
\curveto(321.86468171,347.51494737)(322.24163445,347.93096258)(322.44866943,348.17315412)
\curveto(322.83929011,348.64971186)(323.10296172,349.0344771)(323.23968506,349.327451)
\curveto(323.37639894,349.62432026)(323.44475825,349.92705433)(323.44476318,350.23565412)
\curveto(323.44475825,350.72392854)(323.30022714,351.10674065)(323.01116943,351.38409162)
\curveto(322.72600897,351.6614276)(322.33343124,351.80009934)(321.83343506,351.80010725)
\curveto(321.47796334,351.80009934)(321.10491684,351.73564628)(320.71429443,351.60674787)
\curveto(320.32366762,351.47783403)(319.90960554,351.28252173)(319.47210693,351.02081037)
\lineto(319.47210693,352.21612287)
\curveto(319.87444932,352.4075206)(320.26898018,352.55205171)(320.65570068,352.64971662)
\curveto(321.04632315,352.74736401)(321.43108839,352.79619209)(321.80999756,352.796201)
\curveto(322.66546215,352.79619209)(323.35296147,352.56767669)(323.87249756,352.11065412)
\curveto(324.39592917,351.65752135)(324.65764766,351.06181882)(324.65765381,350.32354475)
\curveto(324.65764766,349.94853869)(324.56975713,349.57353906)(324.39398193,349.19854475)
\curveto(324.22210122,348.82353981)(323.9408515,348.40947773)(323.55023193,347.95635725)
\curveto(323.33147711,347.70244718)(323.01311806,347.35088504)(322.59515381,346.90166975)
\curveto(322.18108764,346.45244843)(321.54827577,345.78057411)(320.69671631,344.88604475)
}
}
{
\newrgbcolor{curcolor}{0 0 0}
\pscustom[linestyle=none,fillstyle=solid,fillcolor=curcolor]
{
\newpath
\moveto(223.32841492,331.72662092)
\curveto(223.32841209,331.94146006)(223.40263076,332.12700675)(223.55107117,332.28326154)
\curveto(223.70341171,332.43950644)(223.88505216,332.51763136)(224.09599304,332.51763654)
\curveto(224.31473923,332.51763136)(224.50223904,332.43950644)(224.65849304,332.28326154)
\curveto(224.81473873,332.12700675)(224.89286365,331.94146006)(224.89286804,331.72662092)
\curveto(224.89286365,331.50786674)(224.81473873,331.32232005)(224.65849304,331.16998029)
\curveto(224.50614528,331.01763286)(224.31864547,330.94146106)(224.09599304,330.94146467)
\curveto(223.87723966,330.94146106)(223.6936461,331.01567973)(223.54521179,331.16412092)
\curveto(223.40067764,331.31255444)(223.32841209,331.50005425)(223.32841492,331.72662092)
\moveto(224.10771179,335.30083967)
\curveto(223.55692748,335.3008317)(223.14481852,335.003957)(222.87138367,334.41021467)
\curveto(222.60185031,333.81645818)(222.46708482,332.91216221)(222.46708679,331.69732404)
\curveto(222.46708482,330.48638339)(222.60185031,329.58404054)(222.87138367,328.99029279)
\curveto(223.14481852,328.39654173)(223.55692748,328.09966703)(224.10771179,328.09966779)
\curveto(224.66239513,328.09966703)(225.07450409,328.39654173)(225.34403992,328.99029279)
\curveto(225.6174723,329.58404054)(225.75419091,330.48638339)(225.75419617,331.69732404)
\curveto(225.75419091,332.91216221)(225.6174723,333.81645818)(225.34403992,334.41021467)
\curveto(225.07450409,335.003957)(224.66239513,335.3008317)(224.10771179,335.30083967)
\moveto(224.10771179,336.23833967)
\curveto(225.041301,336.23833076)(225.74637842,335.85551864)(226.22294617,335.08990217)
\curveto(226.70340871,334.32427018)(226.94364285,333.19341193)(226.94364929,331.69732404)
\curveto(226.94364285,330.20513367)(226.70340871,329.07622855)(226.22294617,328.31060529)
\curveto(225.74637842,327.54498008)(225.041301,327.16216796)(224.10771179,327.16216779)
\curveto(223.17411537,327.16216796)(222.46903795,327.54498008)(221.99247742,328.31060529)
\curveto(221.5159139,329.07622855)(221.27763289,330.20513367)(221.27763367,331.69732404)
\curveto(221.27763289,333.19341193)(221.5159139,334.32427018)(221.99247742,335.08990217)
\curveto(222.46903795,335.85551864)(223.17411537,336.23833076)(224.10771179,336.23833967)
}
}
{
\newrgbcolor{curcolor}{0 0 0}
\pscustom[linestyle=none,fillstyle=solid,fillcolor=curcolor]
{
\newpath
\moveto(255.52357483,331.7015965)
\curveto(255.523572,331.91643564)(255.59779067,332.10198233)(255.74623108,332.25823713)
\curveto(255.89857162,332.41448202)(256.08021207,332.49260694)(256.29115295,332.49261213)
\curveto(256.50989914,332.49260694)(256.69739895,332.41448202)(256.85365295,332.25823713)
\curveto(257.00989864,332.10198233)(257.08802356,331.91643564)(257.08802795,331.7015965)
\curveto(257.08802356,331.48284233)(257.00989864,331.29729564)(256.85365295,331.14495588)
\curveto(256.7013052,330.99260844)(256.51380538,330.91643664)(256.29115295,330.91644025)
\curveto(256.07239958,330.91643664)(255.88880601,330.99065532)(255.7403717,331.1390965)
\curveto(255.59583755,331.28753002)(255.523572,331.47502984)(255.52357483,331.7015965)
\moveto(256.3028717,335.27581525)
\curveto(255.7520874,335.27580729)(255.33997843,334.97893258)(255.06654358,334.38519025)
\curveto(254.79701023,333.79143377)(254.66224474,332.8871378)(254.6622467,331.67229963)
\curveto(254.66224474,330.46135897)(254.79701023,329.55901613)(255.06654358,328.96526838)
\curveto(255.33997843,328.37151731)(255.7520874,328.07464261)(256.3028717,328.07464338)
\curveto(256.85755504,328.07464261)(257.269664,328.37151731)(257.53919983,328.96526838)
\curveto(257.81263221,329.55901613)(257.94935082,330.46135897)(257.94935608,331.67229963)
\curveto(257.94935082,332.8871378)(257.81263221,333.79143377)(257.53919983,334.38519025)
\curveto(257.269664,334.97893258)(256.85755504,335.27580729)(256.3028717,335.27581525)
\moveto(256.3028717,336.21331525)
\curveto(257.23646091,336.21330635)(257.94153833,335.83049423)(258.41810608,335.06487775)
\curveto(258.89856862,334.29924576)(259.13880276,333.16838752)(259.1388092,331.67229963)
\curveto(259.13880276,330.18010926)(258.89856862,329.05120413)(258.41810608,328.28558088)
\curveto(257.94153833,327.51995567)(257.23646091,327.13714355)(256.3028717,327.13714338)
\curveto(255.36927528,327.13714355)(254.66419786,327.51995567)(254.18763733,328.28558088)
\curveto(253.71107381,329.05120413)(253.4727928,330.18010926)(253.47279358,331.67229963)
\curveto(253.4727928,333.16838752)(253.71107381,334.29924576)(254.18763733,335.06487775)
\curveto(254.66419786,335.83049423)(255.36927528,336.21330635)(256.3028717,336.21331525)
}
}
{
\newrgbcolor{curcolor}{0 0 0}
\pscustom[linestyle=none,fillstyle=solid,fillcolor=curcolor]
{
\newpath
\moveto(287.24414062,331.42840314)
\curveto(287.24413779,331.64324229)(287.31835647,331.82878897)(287.46679688,331.98504377)
\curveto(287.61913742,332.14128866)(287.80077786,332.21941358)(288.01171875,332.21941877)
\curveto(288.23046493,332.21941358)(288.41796475,332.14128866)(288.57421875,331.98504377)
\curveto(288.73046443,331.82878897)(288.80858936,331.64324229)(288.80859375,331.42840314)
\curveto(288.80858936,331.20964897)(288.73046443,331.02410228)(288.57421875,330.87176252)
\curveto(288.42187099,330.71941508)(288.23437118,330.64324329)(288.01171875,330.64324689)
\curveto(287.79296537,330.64324329)(287.6093718,330.71746196)(287.4609375,330.86590314)
\curveto(287.31640335,331.01433666)(287.24413779,331.20183648)(287.24414062,331.42840314)
\moveto(288.0234375,335.00262189)
\curveto(287.47265319,335.00261393)(287.06054423,334.70573922)(286.78710938,334.11199689)
\curveto(286.51757602,333.51824041)(286.38281053,332.61394444)(286.3828125,331.39910627)
\curveto(286.38281053,330.18816562)(286.51757602,329.28582277)(286.78710938,328.69207502)
\curveto(287.06054423,328.09832396)(287.47265319,327.80144925)(288.0234375,327.80145002)
\curveto(288.57812084,327.80144925)(288.9902298,328.09832396)(289.25976562,328.69207502)
\curveto(289.53319801,329.28582277)(289.66991662,330.18816562)(289.66992188,331.39910627)
\curveto(289.66991662,332.61394444)(289.53319801,333.51824041)(289.25976562,334.11199689)
\curveto(288.9902298,334.70573922)(288.57812084,335.00261393)(288.0234375,335.00262189)
\moveto(288.0234375,335.94012189)
\curveto(288.95702671,335.94011299)(289.66210413,335.55730087)(290.13867188,334.79168439)
\curveto(290.61913442,334.0260524)(290.85936855,332.89519416)(290.859375,331.39910627)
\curveto(290.85936855,329.9069159)(290.61913442,328.77801078)(290.13867188,328.01238752)
\curveto(289.66210413,327.24676231)(288.95702671,326.86395019)(288.0234375,326.86395002)
\curveto(287.08984107,326.86395019)(286.38476365,327.24676231)(285.90820312,328.01238752)
\curveto(285.43163961,328.77801078)(285.1933586,329.9069159)(285.19335938,331.39910627)
\curveto(285.1933586,332.89519416)(285.43163961,334.0260524)(285.90820312,334.79168439)
\curveto(286.38476365,335.55730087)(287.08984107,335.94011299)(288.0234375,335.94012189)
}
}
{
\newrgbcolor{curcolor}{0 0 0}
\pscustom[linestyle=none,fillstyle=solid,fillcolor=curcolor]
{
\newpath
\moveto(319.88812256,327.99871564)
\lineto(321.72796631,327.99871564)
\lineto(321.72796631,334.68426252)
\lineto(319.74749756,334.23895002)
\lineto(319.74749756,335.31707502)
\lineto(321.71624756,335.75066877)
\lineto(322.89984131,335.75066877)
\lineto(322.89984131,327.99871564)
\lineto(324.71624756,327.99871564)
\lineto(324.71624756,327.00262189)
\lineto(319.88812256,327.00262189)
\lineto(319.88812256,327.99871564)
}
}
{
\newrgbcolor{curcolor}{0 0 0}
\pscustom[linestyle=none,fillstyle=solid,fillcolor=curcolor]
{
\newpath
\moveto(223.32841492,314.67486311)
\curveto(223.32841209,314.88970225)(223.40263076,315.07524894)(223.55107117,315.23150373)
\curveto(223.70341171,315.38774862)(223.88505216,315.46587354)(224.09599304,315.46587873)
\curveto(224.31473923,315.46587354)(224.50223904,315.38774862)(224.65849304,315.23150373)
\curveto(224.81473873,315.07524894)(224.89286365,314.88970225)(224.89286804,314.67486311)
\curveto(224.89286365,314.45610893)(224.81473873,314.27056224)(224.65849304,314.11822248)
\curveto(224.50614528,313.96587504)(224.31864547,313.88970325)(224.09599304,313.88970686)
\curveto(223.87723966,313.88970325)(223.6936461,313.96392192)(223.54521179,314.11236311)
\curveto(223.40067764,314.26079663)(223.32841209,314.44829644)(223.32841492,314.67486311)
\moveto(224.10771179,318.24908186)
\curveto(223.55692748,318.24907389)(223.14481852,317.95219918)(222.87138367,317.35845686)
\curveto(222.60185031,316.76470037)(222.46708482,315.8604044)(222.46708679,314.64556623)
\curveto(222.46708482,313.43462558)(222.60185031,312.53228273)(222.87138367,311.93853498)
\curveto(223.14481852,311.34478392)(223.55692748,311.04790921)(224.10771179,311.04790998)
\curveto(224.66239513,311.04790921)(225.07450409,311.34478392)(225.34403992,311.93853498)
\curveto(225.6174723,312.53228273)(225.75419091,313.43462558)(225.75419617,314.64556623)
\curveto(225.75419091,315.8604044)(225.6174723,316.76470037)(225.34403992,317.35845686)
\curveto(225.07450409,317.95219918)(224.66239513,318.24907389)(224.10771179,318.24908186)
\moveto(224.10771179,319.18658186)
\curveto(225.041301,319.18657295)(225.74637842,318.80376083)(226.22294617,318.03814436)
\curveto(226.70340871,317.27251236)(226.94364285,316.14165412)(226.94364929,314.64556623)
\curveto(226.94364285,313.15337586)(226.70340871,312.02447074)(226.22294617,311.25884748)
\curveto(225.74637842,310.49322227)(225.041301,310.11041015)(224.10771179,310.11040998)
\curveto(223.17411537,310.11041015)(222.46903795,310.49322227)(221.99247742,311.25884748)
\curveto(221.5159139,312.02447074)(221.27763289,313.15337586)(221.27763367,314.64556623)
\curveto(221.27763289,316.14165412)(221.5159139,317.27251236)(221.99247742,318.03814436)
\curveto(222.46903795,318.80376083)(223.17411537,319.18657295)(224.10771179,319.18658186)
}
}
{
\newrgbcolor{curcolor}{0 0 0}
\pscustom[linestyle=none,fillstyle=solid,fillcolor=curcolor]
{
\newpath
\moveto(255.52357483,314.67742658)
\curveto(255.523572,314.89226572)(255.59779067,315.07781241)(255.74623108,315.23406721)
\curveto(255.89857162,315.3903121)(256.08021207,315.46843702)(256.29115295,315.46844221)
\curveto(256.50989914,315.46843702)(256.69739895,315.3903121)(256.85365295,315.23406721)
\curveto(257.00989864,315.07781241)(257.08802356,314.89226572)(257.08802795,314.67742658)
\curveto(257.08802356,314.45867241)(257.00989864,314.27312572)(256.85365295,314.12078596)
\curveto(256.7013052,313.96843852)(256.51380538,313.89226672)(256.29115295,313.89227033)
\curveto(256.07239958,313.89226672)(255.88880601,313.9664854)(255.7403717,314.11492658)
\curveto(255.59583755,314.2633601)(255.523572,314.45085991)(255.52357483,314.67742658)
\moveto(256.3028717,318.25164533)
\curveto(255.7520874,318.25163736)(255.33997843,317.95476266)(255.06654358,317.36102033)
\curveto(254.79701023,316.76726385)(254.66224474,315.86296788)(254.6622467,314.64812971)
\curveto(254.66224474,313.43718905)(254.79701023,312.53484621)(255.06654358,311.94109846)
\curveto(255.33997843,311.34734739)(255.7520874,311.05047269)(256.3028717,311.05047346)
\curveto(256.85755504,311.05047269)(257.269664,311.34734739)(257.53919983,311.94109846)
\curveto(257.81263221,312.53484621)(257.94935082,313.43718905)(257.94935608,314.64812971)
\curveto(257.94935082,315.86296788)(257.81263221,316.76726385)(257.53919983,317.36102033)
\curveto(257.269664,317.95476266)(256.85755504,318.25163736)(256.3028717,318.25164533)
\moveto(256.3028717,319.18914533)
\curveto(257.23646091,319.18913643)(257.94153833,318.80632431)(258.41810608,318.04070783)
\curveto(258.89856862,317.27507584)(259.13880276,316.1442176)(259.1388092,314.64812971)
\curveto(259.13880276,313.15593933)(258.89856862,312.02703421)(258.41810608,311.26141096)
\curveto(257.94153833,310.49578574)(257.23646091,310.11297363)(256.3028717,310.11297346)
\curveto(255.36927528,310.11297363)(254.66419786,310.49578574)(254.18763733,311.26141096)
\curveto(253.71107381,312.02703421)(253.4727928,313.15593933)(253.47279358,314.64812971)
\curveto(253.4727928,316.1442176)(253.71107381,317.27507584)(254.18763733,318.04070783)
\curveto(254.66419786,318.80632431)(255.36927528,319.18913643)(256.3028717,319.18914533)
}
}
{
\newrgbcolor{curcolor}{0 0 0}
\pscustom[linestyle=none,fillstyle=solid,fillcolor=curcolor]
{
\newpath
\moveto(287.24414062,314.41021467)
\curveto(287.24413779,314.62505381)(287.31835647,314.8106005)(287.46679688,314.96685529)
\curveto(287.61913742,315.12310019)(287.80077786,315.20122511)(288.01171875,315.20123029)
\curveto(288.23046493,315.20122511)(288.41796475,315.12310019)(288.57421875,314.96685529)
\curveto(288.73046443,314.8106005)(288.80858936,314.62505381)(288.80859375,314.41021467)
\curveto(288.80858936,314.19146049)(288.73046443,314.0059138)(288.57421875,313.85357404)
\curveto(288.42187099,313.70122661)(288.23437118,313.62505481)(288.01171875,313.62505842)
\curveto(287.79296537,313.62505481)(287.6093718,313.69927348)(287.4609375,313.84771467)
\curveto(287.31640335,313.99614819)(287.24413779,314.183648)(287.24414062,314.41021467)
\moveto(288.0234375,317.98443342)
\curveto(287.47265319,317.98442545)(287.06054423,317.68755075)(286.78710938,317.09380842)
\curveto(286.51757602,316.50005193)(286.38281053,315.59575596)(286.3828125,314.38091779)
\curveto(286.38281053,313.16997714)(286.51757602,312.26763429)(286.78710938,311.67388654)
\curveto(287.06054423,311.08013548)(287.47265319,310.78326078)(288.0234375,310.78326154)
\curveto(288.57812084,310.78326078)(288.9902298,311.08013548)(289.25976562,311.67388654)
\curveto(289.53319801,312.26763429)(289.66991662,313.16997714)(289.66992188,314.38091779)
\curveto(289.66991662,315.59575596)(289.53319801,316.50005193)(289.25976562,317.09380842)
\curveto(288.9902298,317.68755075)(288.57812084,317.98442545)(288.0234375,317.98443342)
\moveto(288.0234375,318.92193342)
\curveto(288.95702671,318.92192451)(289.66210413,318.53911239)(290.13867188,317.77349592)
\curveto(290.61913442,317.00786393)(290.85936855,315.87700568)(290.859375,314.38091779)
\curveto(290.85936855,312.88872742)(290.61913442,311.7598223)(290.13867188,310.99419904)
\curveto(289.66210413,310.22857383)(288.95702671,309.84576171)(288.0234375,309.84576154)
\curveto(287.08984107,309.84576171)(286.38476365,310.22857383)(285.90820312,310.99419904)
\curveto(285.43163961,311.7598223)(285.1933586,312.88872742)(285.19335938,314.38091779)
\curveto(285.1933586,315.87700568)(285.43163961,317.00786393)(285.90820312,317.77349592)
\curveto(286.38476365,318.53911239)(287.08984107,318.92192451)(288.0234375,318.92193342)
}
}
{
\newrgbcolor{curcolor}{0 0 0}
\pscustom[linestyle=none,fillstyle=solid,fillcolor=curcolor]
{
\newpath
\moveto(319.88812256,311.11138654)
\lineto(321.72796631,311.11138654)
\lineto(321.72796631,317.79693342)
\lineto(319.74749756,317.35162092)
\lineto(319.74749756,318.42974592)
\lineto(321.71624756,318.86333967)
\lineto(322.89984131,318.86333967)
\lineto(322.89984131,311.11138654)
\lineto(324.71624756,311.11138654)
\lineto(324.71624756,310.11529279)
\lineto(319.88812256,310.11529279)
\lineto(319.88812256,311.11138654)
}
}
{
\newrgbcolor{curcolor}{0 0 0}
\pscustom[linestyle=none,fillstyle=solid,fillcolor=curcolor]
{
\newpath
\moveto(215.58818054,294.22466779)
\lineto(219.60771179,294.22466779)
\lineto(219.60771179,293.22857404)
\lineto(214.29325867,293.22857404)
\lineto(214.29325867,294.22466779)
\curveto(215.0237258,294.99419728)(215.66239703,295.6738841)(216.20927429,296.26373029)
\curveto(216.75614594,296.85357042)(217.13309869,297.26958563)(217.34013367,297.51177717)
\curveto(217.73075434,297.98833491)(217.99442595,298.37310015)(218.13114929,298.66607404)
\curveto(218.26786318,298.96294331)(218.33622248,299.26567738)(218.33622742,299.57427717)
\curveto(218.33622248,300.06255158)(218.19169138,300.4453637)(217.90263367,300.72271467)
\curveto(217.6174732,301.00005065)(217.22489547,301.13872238)(216.72489929,301.13873029)
\curveto(216.36942758,301.13872238)(215.99638107,301.07426932)(215.60575867,300.94537092)
\curveto(215.21513185,300.81645708)(214.80106977,300.62114478)(214.36357117,300.35943342)
\lineto(214.36357117,301.55474592)
\curveto(214.76591355,301.74614365)(215.16044441,301.89067476)(215.54716492,301.98833967)
\curveto(215.93778738,302.08598706)(216.32255262,302.13481514)(216.70146179,302.13482404)
\curveto(217.55692639,302.13481514)(218.2444257,301.90629974)(218.76396179,301.44927717)
\curveto(219.28739341,300.9961444)(219.5491119,300.40044187)(219.54911804,299.66216779)
\curveto(219.5491119,299.28716173)(219.46122136,298.91216211)(219.28544617,298.53716779)
\curveto(219.11356546,298.16216286)(218.83231574,297.74810077)(218.44169617,297.29498029)
\curveto(218.22294135,297.04107023)(217.90458229,296.68950808)(217.48661804,296.24029279)
\curveto(217.07255187,295.79107148)(216.43974,295.11919715)(215.58818054,294.22466779)
}
}
{
\newrgbcolor{curcolor}{0 0 0}
\pscustom[linestyle=none,fillstyle=solid,fillcolor=curcolor]
{
\newpath
\moveto(221.43583679,301.97662092)
\lineto(226.94364929,301.97662092)
\lineto(226.94364929,301.47271467)
\lineto(223.81474304,293.22857404)
\lineto(222.57841492,293.22857404)
\lineto(225.62528992,300.98052717)
\lineto(221.43583679,300.98052717)
\lineto(221.43583679,301.97662092)
}
}
{
\newrgbcolor{curcolor}{0 0 0}
\pscustom[linestyle=none,fillstyle=solid,fillcolor=curcolor]
{
\newpath
\moveto(255.52357483,297.65313459)
\curveto(255.523572,297.86797373)(255.59779067,298.05352042)(255.74623108,298.20977521)
\curveto(255.89857162,298.36602011)(256.08021207,298.44414503)(256.29115295,298.44415021)
\curveto(256.50989914,298.44414503)(256.69739895,298.36602011)(256.85365295,298.20977521)
\curveto(257.00989864,298.05352042)(257.08802356,297.86797373)(257.08802795,297.65313459)
\curveto(257.08802356,297.43438041)(257.00989864,297.24883372)(256.85365295,297.09649396)
\curveto(256.7013052,296.94414653)(256.51380538,296.86797473)(256.29115295,296.86797834)
\curveto(256.07239958,296.86797473)(255.88880601,296.94219341)(255.7403717,297.09063459)
\curveto(255.59583755,297.23906811)(255.523572,297.42656792)(255.52357483,297.65313459)
\moveto(256.3028717,301.22735334)
\curveto(255.7520874,301.22734537)(255.33997843,300.93047067)(255.06654358,300.33672834)
\curveto(254.79701023,299.74297186)(254.66224474,298.83867588)(254.6622467,297.62383771)
\curveto(254.66224474,296.41289706)(254.79701023,295.51055421)(255.06654358,294.91680646)
\curveto(255.33997843,294.3230554)(255.7520874,294.0261807)(256.3028717,294.02618146)
\curveto(256.85755504,294.0261807)(257.269664,294.3230554)(257.53919983,294.91680646)
\curveto(257.81263221,295.51055421)(257.94935082,296.41289706)(257.94935608,297.62383771)
\curveto(257.94935082,298.83867588)(257.81263221,299.74297186)(257.53919983,300.33672834)
\curveto(257.269664,300.93047067)(256.85755504,301.22734537)(256.3028717,301.22735334)
\moveto(256.3028717,302.16485334)
\curveto(257.23646091,302.16484443)(257.94153833,301.78203232)(258.41810608,301.01641584)
\curveto(258.89856862,300.25078385)(259.13880276,299.1199256)(259.1388092,297.62383771)
\curveto(259.13880276,296.13164734)(258.89856862,295.00274222)(258.41810608,294.23711896)
\curveto(257.94153833,293.47149375)(257.23646091,293.08868163)(256.3028717,293.08868146)
\curveto(255.36927528,293.08868163)(254.66419786,293.47149375)(254.18763733,294.23711896)
\curveto(253.71107381,295.00274222)(253.4727928,296.13164734)(253.47279358,297.62383771)
\curveto(253.4727928,299.1199256)(253.71107381,300.25078385)(254.18763733,301.01641584)
\curveto(254.66419786,301.78203232)(255.36927528,302.16484443)(256.3028717,302.16485334)
}
}
{
\newrgbcolor{curcolor}{0 0 0}
\pscustom[linestyle=none,fillstyle=solid,fillcolor=curcolor]
{
\newpath
\moveto(289.08401489,297.67913557)
\curveto(289.65822852,297.52678729)(290.09768121,297.25530318)(290.40237427,296.86468244)
\curveto(290.7070556,296.47796021)(290.8593992,295.9935857)(290.85940552,295.41155744)
\curveto(290.8593992,294.60686833)(290.58791509,293.97405646)(290.04495239,293.51311994)
\curveto(289.50588492,293.05608863)(288.7578388,292.82757324)(287.80081177,292.82757307)
\curveto(287.39846516,292.82757324)(286.98830932,292.86468257)(286.57034302,292.93890119)
\curveto(286.15237265,293.01311993)(285.74221681,293.12054169)(285.33987427,293.26116682)
\lineto(285.33987427,294.43890119)
\curveto(285.73831057,294.23186871)(286.1308883,294.07757199)(286.51760864,293.97601057)
\curveto(286.90432503,293.87444719)(287.28909027,293.82366599)(287.67190552,293.82366682)
\curveto(288.32033923,293.82366599)(288.81838561,293.97015022)(289.16604614,294.26311994)
\curveto(289.51369742,294.55608713)(289.68752537,294.97796171)(289.68753052,295.52874494)
\curveto(289.68752537,296.0365544)(289.51369742,296.43889775)(289.16604614,296.73577619)
\curveto(288.81838561,297.0365534)(288.34768296,297.18694388)(287.75393677,297.18694807)
\lineto(286.85159302,297.18694807)
\lineto(286.85159302,298.15960432)
\lineto(287.75393677,298.15960432)
\curveto(288.29690176,298.15959915)(288.72072946,298.27873966)(289.02542114,298.51702619)
\curveto(289.33010385,298.75530168)(289.48244745,299.0873326)(289.48245239,299.51311994)
\curveto(289.48244745,299.96233173)(289.33986946,300.30608138)(289.05471802,300.54436994)
\curveto(288.77346378,300.78654965)(288.37112043,300.90764328)(287.84768677,300.90765119)
\curveto(287.50002755,300.90764328)(287.14065291,300.86858082)(286.76956177,300.79046369)
\curveto(286.39846616,300.71233098)(286.00979467,300.59514359)(285.60354614,300.43890119)
\lineto(285.60354614,301.52874494)
\curveto(286.07620085,301.65373629)(286.49612231,301.74748619)(286.86331177,301.80999494)
\curveto(287.23440282,301.87248607)(287.56252749,301.90373604)(287.84768677,301.90374494)
\curveto(288.69924511,301.90373604)(289.37893193,301.6888925)(289.88674927,301.25921369)
\curveto(290.39846216,300.83342461)(290.65432128,300.26701892)(290.65432739,299.55999494)
\curveto(290.65432128,299.07952011)(290.51955579,298.67912988)(290.25003052,298.35882307)
\curveto(289.98440007,298.03850553)(289.59572858,297.81194325)(289.08401489,297.67913557)
}
}
{
\newrgbcolor{curcolor}{0 0 0}
\pscustom[linestyle=none,fillstyle=solid,fillcolor=curcolor]
{
\newpath
\moveto(319.66546631,301.97601057)
\lineto(324.09515381,301.97601057)
\lineto(324.09515381,300.97991682)
\lineto(320.74359131,300.97991682)
\lineto(320.74359131,298.82952619)
\curveto(320.9115576,298.89202053)(321.07952618,298.93694236)(321.24749756,298.96429182)
\curveto(321.41936959,298.99553605)(321.59124442,299.01116103)(321.76312256,299.01116682)
\curveto(322.66936834,299.01116103)(323.38811762,298.74358318)(323.91937256,298.20843244)
\curveto(324.45061656,297.67327175)(324.71624129,296.9486631)(324.71624756,296.03460432)
\curveto(324.71624129,295.11272743)(324.4369447,294.38616566)(323.87835693,293.85491682)
\curveto(323.32366456,293.32366672)(322.5638997,293.05804199)(321.59906006,293.05804182)
\curveto(321.13421363,293.05804199)(320.7084328,293.08929196)(320.32171631,293.15179182)
\curveto(319.93890232,293.21429183)(319.59515267,293.30804174)(319.29046631,293.43304182)
\lineto(319.29046631,294.63421369)
\curveto(319.64984011,294.43889998)(320.01116788,294.29241575)(320.37445068,294.19476057)
\curveto(320.73772965,294.10100969)(321.10882303,294.05413474)(321.48773193,294.05413557)
\curveto(322.140072,294.05413474)(322.64202462,294.22600957)(322.99359131,294.56976057)
\curveto(323.34905516,294.91350888)(323.52678936,295.40178964)(323.52679443,296.03460432)
\curveto(323.52678936,296.65960088)(323.34319579,297.14592852)(322.97601318,297.49358869)
\curveto(322.61272777,297.84124033)(322.10491578,298.01506828)(321.45257568,298.01507307)
\curveto(321.13616675,298.01506828)(320.82757331,297.97795894)(320.52679443,297.90374494)
\curveto(320.22601141,297.83342784)(319.93890232,297.72600607)(319.66546631,297.58147932)
\lineto(319.66546631,301.97601057)
}
}
{
\newrgbcolor{curcolor}{0 0 0}
\pscustom[linestyle=none,fillstyle=solid,fillcolor=curcolor]
{
\newpath
\moveto(223.32841492,280.74126936)
\curveto(223.32841209,280.9561085)(223.40263076,281.14165519)(223.55107117,281.29790998)
\curveto(223.70341171,281.45415487)(223.88505216,281.53227979)(224.09599304,281.53228498)
\curveto(224.31473923,281.53227979)(224.50223904,281.45415487)(224.65849304,281.29790998)
\curveto(224.81473873,281.14165519)(224.89286365,280.9561085)(224.89286804,280.74126936)
\curveto(224.89286365,280.52251518)(224.81473873,280.33696849)(224.65849304,280.18462873)
\curveto(224.50614528,280.03228129)(224.31864547,279.9561095)(224.09599304,279.95611311)
\curveto(223.87723966,279.9561095)(223.6936461,280.03032817)(223.54521179,280.17876936)
\curveto(223.40067764,280.32720288)(223.32841209,280.51470269)(223.32841492,280.74126936)
\moveto(224.10771179,284.31548811)
\curveto(223.55692748,284.31548014)(223.14481852,284.01860543)(222.87138367,283.42486311)
\curveto(222.60185031,282.83110662)(222.46708482,281.92681065)(222.46708679,280.71197248)
\curveto(222.46708482,279.50103183)(222.60185031,278.59868898)(222.87138367,278.00494123)
\curveto(223.14481852,277.41119017)(223.55692748,277.11431546)(224.10771179,277.11431623)
\curveto(224.66239513,277.11431546)(225.07450409,277.41119017)(225.34403992,278.00494123)
\curveto(225.6174723,278.59868898)(225.75419091,279.50103183)(225.75419617,280.71197248)
\curveto(225.75419091,281.92681065)(225.6174723,282.83110662)(225.34403992,283.42486311)
\curveto(225.07450409,284.01860543)(224.66239513,284.31548014)(224.10771179,284.31548811)
\moveto(224.10771179,285.25298811)
\curveto(225.041301,285.2529792)(225.74637842,284.87016708)(226.22294617,284.10455061)
\curveto(226.70340871,283.33891861)(226.94364285,282.20806037)(226.94364929,280.71197248)
\curveto(226.94364285,279.21978211)(226.70340871,278.09087699)(226.22294617,277.32525373)
\curveto(225.74637842,276.55962852)(225.041301,276.1768164)(224.10771179,276.17681623)
\curveto(223.17411537,276.1768164)(222.46903795,276.55962852)(221.99247742,277.32525373)
\curveto(221.5159139,278.09087699)(221.27763289,279.21978211)(221.27763367,280.71197248)
\curveto(221.27763289,282.20806037)(221.5159139,283.33891861)(221.99247742,284.10455061)
\curveto(222.46903795,284.87016708)(223.17411537,285.2529792)(224.10771179,285.25298811)
}
}
{
\newrgbcolor{curcolor}{0 0 0}
\pscustom[linestyle=none,fillstyle=solid,fillcolor=curcolor]
{
\newpath
\moveto(255.52357483,280.62896467)
\curveto(255.523572,280.84380381)(255.59779067,281.0293505)(255.74623108,281.18560529)
\curveto(255.89857162,281.34185019)(256.08021207,281.41997511)(256.29115295,281.41998029)
\curveto(256.50989914,281.41997511)(256.69739895,281.34185019)(256.85365295,281.18560529)
\curveto(257.00989864,281.0293505)(257.08802356,280.84380381)(257.08802795,280.62896467)
\curveto(257.08802356,280.41021049)(257.00989864,280.2246638)(256.85365295,280.07232404)
\curveto(256.7013052,279.91997661)(256.51380538,279.84380481)(256.29115295,279.84380842)
\curveto(256.07239958,279.84380481)(255.88880601,279.91802348)(255.7403717,280.06646467)
\curveto(255.59583755,280.21489819)(255.523572,280.402398)(255.52357483,280.62896467)
\moveto(256.3028717,284.20318342)
\curveto(255.7520874,284.20317545)(255.33997843,283.90630075)(255.06654358,283.31255842)
\curveto(254.79701023,282.71880193)(254.66224474,281.81450596)(254.6622467,280.59966779)
\curveto(254.66224474,279.38872714)(254.79701023,278.48638429)(255.06654358,277.89263654)
\curveto(255.33997843,277.29888548)(255.7520874,277.00201078)(256.3028717,277.00201154)
\curveto(256.85755504,277.00201078)(257.269664,277.29888548)(257.53919983,277.89263654)
\curveto(257.81263221,278.48638429)(257.94935082,279.38872714)(257.94935608,280.59966779)
\curveto(257.94935082,281.81450596)(257.81263221,282.71880193)(257.53919983,283.31255842)
\curveto(257.269664,283.90630075)(256.85755504,284.20317545)(256.3028717,284.20318342)
\moveto(256.3028717,285.14068342)
\curveto(257.23646091,285.14067451)(257.94153833,284.75786239)(258.41810608,283.99224592)
\curveto(258.89856862,283.22661393)(259.13880276,282.09575568)(259.1388092,280.59966779)
\curveto(259.13880276,279.10747742)(258.89856862,277.9785723)(258.41810608,277.21294904)
\curveto(257.94153833,276.44732383)(257.23646091,276.06451171)(256.3028717,276.06451154)
\curveto(255.36927528,276.06451171)(254.66419786,276.44732383)(254.18763733,277.21294904)
\curveto(253.71107381,277.9785723)(253.4727928,279.10747742)(253.47279358,280.59966779)
\curveto(253.4727928,282.09575568)(253.71107381,283.22661393)(254.18763733,283.99224592)
\curveto(254.66419786,284.75786239)(255.36927528,285.14067451)(256.3028717,285.14068342)
}
}
{
\newrgbcolor{curcolor}{0 0 0}
\pscustom[linestyle=none,fillstyle=solid,fillcolor=curcolor]
{
\newpath
\moveto(287.24414062,280.37383771)
\curveto(287.24413779,280.58867686)(287.31835647,280.77422354)(287.46679688,280.93047834)
\curveto(287.61913742,281.08672323)(287.80077786,281.16484815)(288.01171875,281.16485334)
\curveto(288.23046493,281.16484815)(288.41796475,281.08672323)(288.57421875,280.93047834)
\curveto(288.73046443,280.77422354)(288.80858936,280.58867686)(288.80859375,280.37383771)
\curveto(288.80858936,280.15508354)(288.73046443,279.96953685)(288.57421875,279.81719709)
\curveto(288.42187099,279.66484965)(288.23437118,279.58867786)(288.01171875,279.58868146)
\curveto(287.79296537,279.58867786)(287.6093718,279.66289653)(287.4609375,279.81133771)
\curveto(287.31640335,279.95977123)(287.24413779,280.14727105)(287.24414062,280.37383771)
\moveto(288.0234375,283.94805646)
\curveto(287.47265319,283.9480485)(287.06054423,283.65117379)(286.78710938,283.05743146)
\curveto(286.51757602,282.46367498)(286.38281053,281.55937901)(286.3828125,280.34454084)
\curveto(286.38281053,279.13360019)(286.51757602,278.23125734)(286.78710938,277.63750959)
\curveto(287.06054423,277.04375853)(287.47265319,276.74688382)(288.0234375,276.74688459)
\curveto(288.57812084,276.74688382)(288.9902298,277.04375853)(289.25976562,277.63750959)
\curveto(289.53319801,278.23125734)(289.66991662,279.13360019)(289.66992188,280.34454084)
\curveto(289.66991662,281.55937901)(289.53319801,282.46367498)(289.25976562,283.05743146)
\curveto(288.9902298,283.65117379)(288.57812084,283.9480485)(288.0234375,283.94805646)
\moveto(288.0234375,284.88555646)
\curveto(288.95702671,284.88554756)(289.66210413,284.50273544)(290.13867188,283.73711896)
\curveto(290.61913442,282.97148697)(290.85936855,281.84062873)(290.859375,280.34454084)
\curveto(290.85936855,278.85235047)(290.61913442,277.72344535)(290.13867188,276.95782209)
\curveto(289.66210413,276.19219688)(288.95702671,275.80938476)(288.0234375,275.80938459)
\curveto(287.08984107,275.80938476)(286.38476365,276.19219688)(285.90820312,276.95782209)
\curveto(285.43163961,277.72344535)(285.1933586,278.85235047)(285.19335938,280.34454084)
\curveto(285.1933586,281.84062873)(285.43163961,282.97148697)(285.90820312,283.73711896)
\curveto(286.38476365,284.50273544)(287.08984107,284.88554756)(288.0234375,284.88555646)
}
}
{
\newrgbcolor{curcolor}{0 0 0}
\pscustom[linestyle=none,fillstyle=solid,fillcolor=curcolor]
{
\newpath
\moveto(322.37249756,283.84063459)
\lineto(319.61273193,279.21758771)
\lineto(322.37249756,279.21758771)
\lineto(322.37249756,283.84063459)
\moveto(322.17913818,284.91875959)
\lineto(323.55023193,284.91875959)
\lineto(323.55023193,279.21758771)
\lineto(324.71624756,279.21758771)
\lineto(324.71624756,278.25665021)
\lineto(323.55023193,278.25665021)
\lineto(323.55023193,276.17071271)
\lineto(322.37249756,276.17071271)
\lineto(322.37249756,278.25665021)
\lineto(318.66351318,278.25665021)
\lineto(318.66351318,279.37579084)
\lineto(322.17913818,284.91875959)
}
}
{
\newrgbcolor{curcolor}{0 0 0}
\pscustom[linestyle=none,fillstyle=solid,fillcolor=curcolor]
{
\newpath
\moveto(223.32841492,263.68951154)
\curveto(223.32841209,263.90435068)(223.40263076,264.08989737)(223.55107117,264.24615217)
\curveto(223.70341171,264.40239706)(223.88505216,264.48052198)(224.09599304,264.48052717)
\curveto(224.31473923,264.48052198)(224.50223904,264.40239706)(224.65849304,264.24615217)
\curveto(224.81473873,264.08989737)(224.89286365,263.90435068)(224.89286804,263.68951154)
\curveto(224.89286365,263.47075737)(224.81473873,263.28521068)(224.65849304,263.13287092)
\curveto(224.50614528,262.98052348)(224.31864547,262.90435168)(224.09599304,262.90435529)
\curveto(223.87723966,262.90435168)(223.6936461,262.97857036)(223.54521179,263.12701154)
\curveto(223.40067764,263.27544506)(223.32841209,263.46294488)(223.32841492,263.68951154)
\moveto(224.10771179,267.26373029)
\curveto(223.55692748,267.26372232)(223.14481852,266.96684762)(222.87138367,266.37310529)
\curveto(222.60185031,265.77934881)(222.46708482,264.87505284)(222.46708679,263.66021467)
\curveto(222.46708482,262.44927401)(222.60185031,261.54693117)(222.87138367,260.95318342)
\curveto(223.14481852,260.35943235)(223.55692748,260.06255765)(224.10771179,260.06255842)
\curveto(224.66239513,260.06255765)(225.07450409,260.35943235)(225.34403992,260.95318342)
\curveto(225.6174723,261.54693117)(225.75419091,262.44927401)(225.75419617,263.66021467)
\curveto(225.75419091,264.87505284)(225.6174723,265.77934881)(225.34403992,266.37310529)
\curveto(225.07450409,266.96684762)(224.66239513,267.26372232)(224.10771179,267.26373029)
\moveto(224.10771179,268.20123029)
\curveto(225.041301,268.20122139)(225.74637842,267.81840927)(226.22294617,267.05279279)
\curveto(226.70340871,266.2871608)(226.94364285,265.15630256)(226.94364929,263.66021467)
\curveto(226.94364285,262.16802429)(226.70340871,261.03911917)(226.22294617,260.27349592)
\curveto(225.74637842,259.50787071)(225.041301,259.12505859)(224.10771179,259.12505842)
\curveto(223.17411537,259.12505859)(222.46903795,259.50787071)(221.99247742,260.27349592)
\curveto(221.5159139,261.03911917)(221.27763289,262.16802429)(221.27763367,263.66021467)
\curveto(221.27763289,265.15630256)(221.5159139,266.2871608)(221.99247742,267.05279279)
\curveto(222.46903795,267.81840927)(223.17411537,268.20122139)(224.10771179,268.20123029)
}
}
{
\newrgbcolor{curcolor}{0 0 0}
\pscustom[linestyle=none,fillstyle=solid,fillcolor=curcolor]
{
\newpath
\moveto(255.52357483,263.60467268)
\curveto(255.523572,263.81951182)(255.59779067,264.00505851)(255.74623108,264.1613133)
\curveto(255.89857162,264.31755819)(256.08021207,264.39568312)(256.29115295,264.3956883)
\curveto(256.50989914,264.39568312)(256.69739895,264.31755819)(256.85365295,264.1613133)
\curveto(257.00989864,264.00505851)(257.08802356,263.81951182)(257.08802795,263.60467268)
\curveto(257.08802356,263.3859185)(257.00989864,263.20037181)(256.85365295,263.04803205)
\curveto(256.7013052,262.89568462)(256.51380538,262.81951282)(256.29115295,262.81951643)
\curveto(256.07239958,262.81951282)(255.88880601,262.89373149)(255.7403717,263.04217268)
\curveto(255.59583755,263.1906062)(255.523572,263.37810601)(255.52357483,263.60467268)
\moveto(256.3028717,267.17889143)
\curveto(255.7520874,267.17888346)(255.33997843,266.88200875)(255.06654358,266.28826643)
\curveto(254.79701023,265.69450994)(254.66224474,264.79021397)(254.6622467,263.5753758)
\curveto(254.66224474,262.36443515)(254.79701023,261.4620923)(255.06654358,260.86834455)
\curveto(255.33997843,260.27459349)(255.7520874,259.97771878)(256.3028717,259.97771955)
\curveto(256.85755504,259.97771878)(257.269664,260.27459349)(257.53919983,260.86834455)
\curveto(257.81263221,261.4620923)(257.94935082,262.36443515)(257.94935608,263.5753758)
\curveto(257.94935082,264.79021397)(257.81263221,265.69450994)(257.53919983,266.28826643)
\curveto(257.269664,266.88200875)(256.85755504,267.17888346)(256.3028717,267.17889143)
\moveto(256.3028717,268.11639143)
\curveto(257.23646091,268.11638252)(257.94153833,267.7335704)(258.41810608,266.96795393)
\curveto(258.89856862,266.20232193)(259.13880276,265.07146369)(259.1388092,263.5753758)
\curveto(259.13880276,262.08318543)(258.89856862,260.95428031)(258.41810608,260.18865705)
\curveto(257.94153833,259.42303184)(257.23646091,259.04021972)(256.3028717,259.04021955)
\curveto(255.36927528,259.04021972)(254.66419786,259.42303184)(254.18763733,260.18865705)
\curveto(253.71107381,260.95428031)(253.4727928,262.08318543)(253.47279358,263.5753758)
\curveto(253.4727928,265.07146369)(253.71107381,266.20232193)(254.18763733,266.96795393)
\curveto(254.66419786,267.7335704)(255.36927528,268.11638252)(256.3028717,268.11639143)
}
}
{
\newrgbcolor{curcolor}{0 0 0}
\pscustom[linestyle=none,fillstyle=solid,fillcolor=curcolor]
{
\newpath
\moveto(287.24414062,263.35564924)
\curveto(287.24413779,263.57048838)(287.31835647,263.75603507)(287.46679688,263.91228986)
\curveto(287.61913742,264.06853476)(287.80077786,264.14665968)(288.01171875,264.14666486)
\curveto(288.23046493,264.14665968)(288.41796475,264.06853476)(288.57421875,263.91228986)
\curveto(288.73046443,263.75603507)(288.80858936,263.57048838)(288.80859375,263.35564924)
\curveto(288.80858936,263.13689506)(288.73046443,262.95134837)(288.57421875,262.79900861)
\curveto(288.42187099,262.64666118)(288.23437118,262.57048938)(288.01171875,262.57049299)
\curveto(287.79296537,262.57048938)(287.6093718,262.64470805)(287.4609375,262.79314924)
\curveto(287.31640335,262.94158276)(287.24413779,263.12908257)(287.24414062,263.35564924)
\moveto(288.0234375,266.92986799)
\curveto(287.47265319,266.92986002)(287.06054423,266.63298532)(286.78710938,266.03924299)
\curveto(286.51757602,265.4454865)(286.38281053,264.54119053)(286.3828125,263.32635236)
\curveto(286.38281053,262.11541171)(286.51757602,261.21306886)(286.78710938,260.61932111)
\curveto(287.06054423,260.02557005)(287.47265319,259.72869535)(288.0234375,259.72869611)
\curveto(288.57812084,259.72869535)(288.9902298,260.02557005)(289.25976562,260.61932111)
\curveto(289.53319801,261.21306886)(289.66991662,262.11541171)(289.66992188,263.32635236)
\curveto(289.66991662,264.54119053)(289.53319801,265.4454865)(289.25976562,266.03924299)
\curveto(288.9902298,266.63298532)(288.57812084,266.92986002)(288.0234375,266.92986799)
\moveto(288.0234375,267.86736799)
\curveto(288.95702671,267.86735908)(289.66210413,267.48454696)(290.13867188,266.71893049)
\curveto(290.61913442,265.9532985)(290.85936855,264.82244025)(290.859375,263.32635236)
\curveto(290.85936855,261.83416199)(290.61913442,260.70525687)(290.13867188,259.93963361)
\curveto(289.66210413,259.1740084)(288.95702671,258.79119628)(288.0234375,258.79119611)
\curveto(287.08984107,258.79119628)(286.38476365,259.1740084)(285.90820312,259.93963361)
\curveto(285.43163961,260.70525687)(285.1933586,261.83416199)(285.19335938,263.32635236)
\curveto(285.1933586,264.82244025)(285.43163961,265.9532985)(285.90820312,266.71893049)
\curveto(286.38476365,267.48454696)(287.08984107,267.86735908)(288.0234375,267.86736799)
}
}
{
\newrgbcolor{curcolor}{0 0 0}
\pscustom[linestyle=none,fillstyle=solid,fillcolor=curcolor]
{
\newpath
\moveto(319.20843506,268.03143049)
\lineto(324.71624756,268.03143049)
\lineto(324.71624756,267.52752424)
\lineto(321.58734131,259.28338361)
\lineto(320.35101318,259.28338361)
\lineto(323.39788818,267.03533674)
\lineto(319.20843506,267.03533674)
\lineto(319.20843506,268.03143049)
}
}
{
\newrgbcolor{curcolor}{0 0 0}
\pscustom[linestyle=none,fillstyle=solid,fillcolor=curcolor]
{
\newpath
\moveto(223.32841492,246.63775373)
\curveto(223.32841209,246.85259287)(223.40263076,247.03813956)(223.55107117,247.19439436)
\curveto(223.70341171,247.35063925)(223.88505216,247.42876417)(224.09599304,247.42876936)
\curveto(224.31473923,247.42876417)(224.50223904,247.35063925)(224.65849304,247.19439436)
\curveto(224.81473873,247.03813956)(224.89286365,246.85259287)(224.89286804,246.63775373)
\curveto(224.89286365,246.41899955)(224.81473873,246.23345287)(224.65849304,246.08111311)
\curveto(224.50614528,245.92876567)(224.31864547,245.85259387)(224.09599304,245.85259748)
\curveto(223.87723966,245.85259387)(223.6936461,245.92681255)(223.54521179,246.07525373)
\curveto(223.40067764,246.22368725)(223.32841209,246.41118706)(223.32841492,246.63775373)
\moveto(224.10771179,250.21197248)
\curveto(223.55692748,250.21196451)(223.14481852,249.91508981)(222.87138367,249.32134748)
\curveto(222.60185031,248.727591)(222.46708482,247.82329503)(222.46708679,246.60845686)
\curveto(222.46708482,245.3975162)(222.60185031,244.49517335)(222.87138367,243.90142561)
\curveto(223.14481852,243.30767454)(223.55692748,243.01079984)(224.10771179,243.01080061)
\curveto(224.66239513,243.01079984)(225.07450409,243.30767454)(225.34403992,243.90142561)
\curveto(225.6174723,244.49517335)(225.75419091,245.3975162)(225.75419617,246.60845686)
\curveto(225.75419091,247.82329503)(225.6174723,248.727591)(225.34403992,249.32134748)
\curveto(225.07450409,249.91508981)(224.66239513,250.21196451)(224.10771179,250.21197248)
\moveto(224.10771179,251.14947248)
\curveto(225.041301,251.14946357)(225.74637842,250.76665146)(226.22294617,250.00103498)
\curveto(226.70340871,249.23540299)(226.94364285,248.10454474)(226.94364929,246.60845686)
\curveto(226.94364285,245.11626648)(226.70340871,243.98736136)(226.22294617,243.22173811)
\curveto(225.74637842,242.45611289)(225.041301,242.07330078)(224.10771179,242.07330061)
\curveto(223.17411537,242.07330078)(222.46903795,242.45611289)(221.99247742,243.22173811)
\curveto(221.5159139,243.98736136)(221.27763289,245.11626648)(221.27763367,246.60845686)
\curveto(221.27763289,248.10454474)(221.5159139,249.23540299)(221.99247742,250.00103498)
\curveto(222.46903795,250.76665146)(223.17411537,251.14946357)(224.10771179,251.14947248)
}
}
{
\newrgbcolor{curcolor}{0 0 0}
\pscustom[linestyle=none,fillstyle=solid,fillcolor=curcolor]
{
\newpath
\moveto(255.52357483,246.58050275)
\curveto(255.523572,246.79534189)(255.59779067,246.98088858)(255.74623108,247.13714338)
\curveto(255.89857162,247.29338827)(256.08021207,247.37151319)(256.29115295,247.37151838)
\curveto(256.50989914,247.37151319)(256.69739895,247.29338827)(256.85365295,247.13714338)
\curveto(257.00989864,246.98088858)(257.08802356,246.79534189)(257.08802795,246.58050275)
\curveto(257.08802356,246.36174858)(257.00989864,246.17620189)(256.85365295,246.02386213)
\curveto(256.7013052,245.87151469)(256.51380538,245.79534289)(256.29115295,245.7953465)
\curveto(256.07239958,245.79534289)(255.88880601,245.86956157)(255.7403717,246.01800275)
\curveto(255.59583755,246.16643627)(255.523572,246.35393609)(255.52357483,246.58050275)
\moveto(256.3028717,250.1547215)
\curveto(255.7520874,250.15471354)(255.33997843,249.85783883)(255.06654358,249.2640965)
\curveto(254.79701023,248.67034002)(254.66224474,247.76604405)(254.6622467,246.55120588)
\curveto(254.66224474,245.34026522)(254.79701023,244.43792238)(255.06654358,243.84417463)
\curveto(255.33997843,243.25042356)(255.7520874,242.95354886)(256.3028717,242.95354963)
\curveto(256.85755504,242.95354886)(257.269664,243.25042356)(257.53919983,243.84417463)
\curveto(257.81263221,244.43792238)(257.94935082,245.34026522)(257.94935608,246.55120588)
\curveto(257.94935082,247.76604405)(257.81263221,248.67034002)(257.53919983,249.2640965)
\curveto(257.269664,249.85783883)(256.85755504,250.15471354)(256.3028717,250.1547215)
\moveto(256.3028717,251.0922215)
\curveto(257.23646091,251.0922126)(257.94153833,250.70940048)(258.41810608,249.943784)
\curveto(258.89856862,249.17815201)(259.13880276,248.04729377)(259.1388092,246.55120588)
\curveto(259.13880276,245.05901551)(258.89856862,243.93011038)(258.41810608,243.16448713)
\curveto(257.94153833,242.39886192)(257.23646091,242.0160498)(256.3028717,242.01604963)
\curveto(255.36927528,242.0160498)(254.66419786,242.39886192)(254.18763733,243.16448713)
\curveto(253.71107381,243.93011038)(253.4727928,245.05901551)(253.47279358,246.55120588)
\curveto(253.4727928,248.04729377)(253.71107381,249.17815201)(254.18763733,249.943784)
\curveto(254.66419786,250.70940048)(255.36927528,251.0922126)(256.3028717,251.0922215)
}
}
{
\newrgbcolor{curcolor}{0 0 0}
\pscustom[linestyle=none,fillstyle=solid,fillcolor=curcolor]
{
\newpath
\moveto(287.24414062,246.33746076)
\curveto(287.24413779,246.5522999)(287.31835647,246.73784659)(287.46679688,246.89410139)
\curveto(287.61913742,247.05034628)(287.80077786,247.1284712)(288.01171875,247.12847639)
\curveto(288.23046493,247.1284712)(288.41796475,247.05034628)(288.57421875,246.89410139)
\curveto(288.73046443,246.73784659)(288.80858936,246.5522999)(288.80859375,246.33746076)
\curveto(288.80858936,246.11870659)(288.73046443,245.9331599)(288.57421875,245.78082014)
\curveto(288.42187099,245.6284727)(288.23437118,245.5523009)(288.01171875,245.55230451)
\curveto(287.79296537,245.5523009)(287.6093718,245.62651958)(287.4609375,245.77496076)
\curveto(287.31640335,245.92339428)(287.24413779,246.11089409)(287.24414062,246.33746076)
\moveto(288.0234375,249.91167951)
\curveto(287.47265319,249.91167154)(287.06054423,249.61479684)(286.78710938,249.02105451)
\curveto(286.51757602,248.42729803)(286.38281053,247.52300206)(286.3828125,246.30816389)
\curveto(286.38281053,245.09722323)(286.51757602,244.19488038)(286.78710938,243.60113264)
\curveto(287.06054423,243.00738157)(287.47265319,242.71050687)(288.0234375,242.71050764)
\curveto(288.57812084,242.71050687)(288.9902298,243.00738157)(289.25976562,243.60113264)
\curveto(289.53319801,244.19488038)(289.66991662,245.09722323)(289.66992188,246.30816389)
\curveto(289.66991662,247.52300206)(289.53319801,248.42729803)(289.25976562,249.02105451)
\curveto(288.9902298,249.61479684)(288.57812084,249.91167154)(288.0234375,249.91167951)
\moveto(288.0234375,250.84917951)
\curveto(288.95702671,250.84917061)(289.66210413,250.46635849)(290.13867188,249.70074201)
\curveto(290.61913442,248.93511002)(290.85936855,247.80425178)(290.859375,246.30816389)
\curveto(290.85936855,244.81597351)(290.61913442,243.68706839)(290.13867188,242.92144514)
\curveto(289.66210413,242.15581992)(288.95702671,241.77300781)(288.0234375,241.77300764)
\curveto(287.08984107,241.77300781)(286.38476365,242.15581992)(285.90820312,242.92144514)
\curveto(285.43163961,243.68706839)(285.1933586,244.81597351)(285.19335938,246.30816389)
\curveto(285.1933586,247.80425178)(285.43163961,248.93511002)(285.90820312,249.70074201)
\curveto(286.38476365,250.46635849)(287.08984107,250.84917061)(288.0234375,250.84917951)
}
}
{
\newrgbcolor{curcolor}{0 0 0}
\pscustom[linestyle=none,fillstyle=solid,fillcolor=curcolor]
{
\newpath
\moveto(324.07757568,250.79253889)
\lineto(324.07757568,249.70269514)
\curveto(323.83147637,249.84721878)(323.56975788,249.95659367)(323.29241943,250.03082014)
\curveto(323.01507094,250.10893727)(322.72600873,250.14799973)(322.42523193,250.14800764)
\curveto(321.67522853,250.14799973)(321.10686972,249.86479688)(320.72015381,249.29839826)
\curveto(320.333433,248.73589176)(320.14007381,247.90581447)(320.14007568,246.80816389)
\curveto(320.32757363,247.19878393)(320.58733899,247.49761175)(320.91937256,247.70464826)
\curveto(321.25140083,247.91558008)(321.63225982,248.02104873)(322.06195068,248.02105451)
\curveto(322.90569605,248.02104873)(323.55803915,247.76128336)(324.01898193,247.24175764)
\curveto(324.48381947,246.72612815)(324.71624111,245.99370701)(324.71624756,245.04449201)
\curveto(324.71624111,244.09917765)(324.4779601,243.36675651)(324.00140381,242.84722639)
\curveto(323.52483605,242.32769505)(322.85491485,242.06792968)(321.99163818,242.06792951)
\curveto(320.97601048,242.06792968)(320.2318706,242.43121057)(319.75921631,243.15777326)
\curveto(319.28655904,243.88824036)(319.05023115,245.03667671)(319.05023193,246.60308576)
\curveto(319.05023115,248.07964242)(319.333434,249.20464129)(319.89984131,249.97808576)
\curveto(320.47015161,250.75542099)(321.29436954,251.14409248)(322.37249756,251.14410139)
\curveto(322.66155567,251.14409248)(322.95061788,251.11284251)(323.23968506,251.05035139)
\curveto(323.5287423,250.99174888)(323.8080389,250.90581147)(324.07757568,250.79253889)
\moveto(321.96820068,247.09527326)
\curveto(321.46429124,247.0952684)(321.06780726,246.91362796)(320.77874756,246.55035139)
\curveto(320.48968284,246.18706619)(320.34515173,245.68511356)(320.34515381,245.04449201)
\curveto(320.34515173,244.40386485)(320.48968284,243.90191222)(320.77874756,243.53863264)
\curveto(321.06780726,243.17535045)(321.46429124,242.99371001)(321.96820068,242.99371076)
\curveto(322.49163396,242.99371001)(322.88616482,243.16558483)(323.15179443,243.50933576)
\curveto(323.41741429,243.85699039)(323.55022665,244.36870863)(323.55023193,245.04449201)
\curveto(323.55022665,245.72417603)(323.41741429,246.23589426)(323.15179443,246.57964826)
\curveto(322.88616482,246.92339358)(322.49163396,247.0952684)(321.96820068,247.09527326)
}
}
{
\newrgbcolor{curcolor}{0 0 0}
\pscustom[linestyle=none,fillstyle=solid,fillcolor=curcolor]
{
\newpath
\moveto(223.32841492,229.58599592)
\curveto(223.32841209,229.80083506)(223.40263076,229.98638175)(223.55107117,230.14263654)
\curveto(223.70341171,230.29888144)(223.88505216,230.37700636)(224.09599304,230.37701154)
\curveto(224.31473923,230.37700636)(224.50223904,230.29888144)(224.65849304,230.14263654)
\curveto(224.81473873,229.98638175)(224.89286365,229.80083506)(224.89286804,229.58599592)
\curveto(224.89286365,229.36724174)(224.81473873,229.18169505)(224.65849304,229.02935529)
\curveto(224.50614528,228.87700786)(224.31864547,228.80083606)(224.09599304,228.80083967)
\curveto(223.87723966,228.80083606)(223.6936461,228.87505473)(223.54521179,229.02349592)
\curveto(223.40067764,229.17192944)(223.32841209,229.35942925)(223.32841492,229.58599592)
\moveto(224.10771179,233.16021467)
\curveto(223.55692748,233.1602067)(223.14481852,232.863332)(222.87138367,232.26958967)
\curveto(222.60185031,231.67583318)(222.46708482,230.77153721)(222.46708679,229.55669904)
\curveto(222.46708482,228.34575839)(222.60185031,227.44341554)(222.87138367,226.84966779)
\curveto(223.14481852,226.25591673)(223.55692748,225.95904203)(224.10771179,225.95904279)
\curveto(224.66239513,225.95904203)(225.07450409,226.25591673)(225.34403992,226.84966779)
\curveto(225.6174723,227.44341554)(225.75419091,228.34575839)(225.75419617,229.55669904)
\curveto(225.75419091,230.77153721)(225.6174723,231.67583318)(225.34403992,232.26958967)
\curveto(225.07450409,232.863332)(224.66239513,233.1602067)(224.10771179,233.16021467)
\moveto(224.10771179,234.09771467)
\curveto(225.041301,234.09770576)(225.74637842,233.71489364)(226.22294617,232.94927717)
\curveto(226.70340871,232.18364518)(226.94364285,231.05278693)(226.94364929,229.55669904)
\curveto(226.94364285,228.06450867)(226.70340871,226.93560355)(226.22294617,226.16998029)
\curveto(225.74637842,225.40435508)(225.041301,225.02154296)(224.10771179,225.02154279)
\curveto(223.17411537,225.02154296)(222.46903795,225.40435508)(221.99247742,226.16998029)
\curveto(221.5159139,226.93560355)(221.27763289,228.06450867)(221.27763367,229.55669904)
\curveto(221.27763289,231.05278693)(221.5159139,232.18364518)(221.99247742,232.94927717)
\curveto(222.46903795,233.71489364)(223.17411537,234.09770576)(224.10771179,234.09771467)
}
}
{
\newrgbcolor{curcolor}{0 0 0}
\pscustom[linestyle=none,fillstyle=solid,fillcolor=curcolor]
{
\newpath
\moveto(255.52357483,229.55621076)
\curveto(255.523572,229.7710499)(255.59779067,229.95659659)(255.74623108,230.11285139)
\curveto(255.89857162,230.26909628)(256.08021207,230.3472212)(256.29115295,230.34722639)
\curveto(256.50989914,230.3472212)(256.69739895,230.26909628)(256.85365295,230.11285139)
\curveto(257.00989864,229.95659659)(257.08802356,229.7710499)(257.08802795,229.55621076)
\curveto(257.08802356,229.33745659)(257.00989864,229.1519099)(256.85365295,228.99957014)
\curveto(256.7013052,228.8472227)(256.51380538,228.7710509)(256.29115295,228.77105451)
\curveto(256.07239958,228.7710509)(255.88880601,228.84526958)(255.7403717,228.99371076)
\curveto(255.59583755,229.14214428)(255.523572,229.32964409)(255.52357483,229.55621076)
\moveto(256.3028717,233.13042951)
\curveto(255.7520874,233.13042154)(255.33997843,232.83354684)(255.06654358,232.23980451)
\curveto(254.79701023,231.64604803)(254.66224474,230.74175206)(254.6622467,229.52691389)
\curveto(254.66224474,228.31597323)(254.79701023,227.41363038)(255.06654358,226.81988264)
\curveto(255.33997843,226.22613157)(255.7520874,225.92925687)(256.3028717,225.92925764)
\curveto(256.85755504,225.92925687)(257.269664,226.22613157)(257.53919983,226.81988264)
\curveto(257.81263221,227.41363038)(257.94935082,228.31597323)(257.94935608,229.52691389)
\curveto(257.94935082,230.74175206)(257.81263221,231.64604803)(257.53919983,232.23980451)
\curveto(257.269664,232.83354684)(256.85755504,233.13042154)(256.3028717,233.13042951)
\moveto(256.3028717,234.06792951)
\curveto(257.23646091,234.06792061)(257.94153833,233.68510849)(258.41810608,232.91949201)
\curveto(258.89856862,232.15386002)(259.13880276,231.02300178)(259.1388092,229.52691389)
\curveto(259.13880276,228.03472351)(258.89856862,226.90581839)(258.41810608,226.14019514)
\curveto(257.94153833,225.37456992)(257.23646091,224.99175781)(256.3028717,224.99175764)
\curveto(255.36927528,224.99175781)(254.66419786,225.37456992)(254.18763733,226.14019514)
\curveto(253.71107381,226.90581839)(253.4727928,228.03472351)(253.47279358,229.52691389)
\curveto(253.4727928,231.02300178)(253.71107381,232.15386002)(254.18763733,232.91949201)
\curveto(254.66419786,233.68510849)(255.36927528,234.06792061)(256.3028717,234.06792951)
}
}
{
\newrgbcolor{curcolor}{0 0 0}
\pscustom[linestyle=none,fillstyle=solid,fillcolor=curcolor]
{
\newpath
\moveto(287.24414062,229.31927229)
\curveto(287.24413779,229.53411143)(287.31835647,229.71965812)(287.46679688,229.87591291)
\curveto(287.61913742,230.0321578)(287.80077786,230.11028272)(288.01171875,230.11028791)
\curveto(288.23046493,230.11028272)(288.41796475,230.0321578)(288.57421875,229.87591291)
\curveto(288.73046443,229.71965812)(288.80858936,229.53411143)(288.80859375,229.31927229)
\curveto(288.80858936,229.10051811)(288.73046443,228.91497142)(288.57421875,228.76263166)
\curveto(288.42187099,228.61028422)(288.23437118,228.53411243)(288.01171875,228.53411604)
\curveto(287.79296537,228.53411243)(287.6093718,228.6083311)(287.4609375,228.75677229)
\curveto(287.31640335,228.9052058)(287.24413779,229.09270562)(287.24414062,229.31927229)
\moveto(288.0234375,232.89349104)
\curveto(287.47265319,232.89348307)(287.06054423,232.59660836)(286.78710938,232.00286604)
\curveto(286.51757602,231.40910955)(286.38281053,230.50481358)(286.3828125,229.28997541)
\curveto(286.38281053,228.07903476)(286.51757602,227.17669191)(286.78710938,226.58294416)
\curveto(287.06054423,225.9891931)(287.47265319,225.69231839)(288.0234375,225.69231916)
\curveto(288.57812084,225.69231839)(288.9902298,225.9891931)(289.25976562,226.58294416)
\curveto(289.53319801,227.17669191)(289.66991662,228.07903476)(289.66992188,229.28997541)
\curveto(289.66991662,230.50481358)(289.53319801,231.40910955)(289.25976562,232.00286604)
\curveto(288.9902298,232.59660836)(288.57812084,232.89348307)(288.0234375,232.89349104)
\moveto(288.0234375,233.83099104)
\curveto(288.95702671,233.83098213)(289.66210413,233.44817001)(290.13867188,232.68255354)
\curveto(290.61913442,231.91692154)(290.85936855,230.7860633)(290.859375,229.28997541)
\curveto(290.85936855,227.79778504)(290.61913442,226.66887992)(290.13867188,225.90325666)
\curveto(289.66210413,225.13763145)(288.95702671,224.75481933)(288.0234375,224.75481916)
\curveto(287.08984107,224.75481933)(286.38476365,225.13763145)(285.90820312,225.90325666)
\curveto(285.43163961,226.66887992)(285.1933586,227.79778504)(285.19335938,229.28997541)
\curveto(285.1933586,230.7860633)(285.43163961,231.91692154)(285.90820312,232.68255354)
\curveto(286.38476365,233.44817001)(287.08984107,233.83098213)(288.0234375,233.83099104)
}
}
{
\newrgbcolor{curcolor}{0 0 0}
\pscustom[linestyle=none,fillstyle=solid,fillcolor=curcolor]
{
\newpath
\moveto(312.63421631,226.01849104)
\lineto(314.47406006,226.01849104)
\lineto(314.47406006,232.70403791)
\lineto(312.49359131,232.25872541)
\lineto(312.49359131,233.33685041)
\lineto(314.46234131,233.77044416)
\lineto(315.64593506,233.77044416)
\lineto(315.64593506,226.01849104)
\lineto(317.46234131,226.01849104)
\lineto(317.46234131,225.02239729)
\lineto(312.63421631,225.02239729)
\lineto(312.63421631,226.01849104)
}
}
{
\newrgbcolor{curcolor}{0 0 0}
\pscustom[linestyle=none,fillstyle=solid,fillcolor=curcolor]
{
\newpath
\moveto(321.10101318,229.41692854)
\curveto(321.10101035,229.63176768)(321.17522903,229.81731437)(321.32366943,229.97356916)
\curveto(321.47600998,230.12981405)(321.65765042,230.20793897)(321.86859131,230.20794416)
\curveto(322.08733749,230.20793897)(322.2748373,230.12981405)(322.43109131,229.97356916)
\curveto(322.58733699,229.81731437)(322.66546191,229.63176768)(322.66546631,229.41692854)
\curveto(322.66546191,229.19817436)(322.58733699,229.01262767)(322.43109131,228.86028791)
\curveto(322.27874355,228.70794047)(322.09124374,228.63176868)(321.86859131,228.63177229)
\curveto(321.64983793,228.63176868)(321.46624436,228.70598735)(321.31781006,228.85442854)
\curveto(321.17327591,229.00286205)(321.10101035,229.19036187)(321.10101318,229.41692854)
\moveto(321.88031006,232.99114729)
\curveto(321.32952575,232.99113932)(320.91741679,232.69426461)(320.64398193,232.10052229)
\curveto(320.37444858,231.5067658)(320.23968309,230.60246983)(320.23968506,229.38763166)
\curveto(320.23968309,228.17669101)(320.37444858,227.27434816)(320.64398193,226.68060041)
\curveto(320.91741679,226.08684935)(321.32952575,225.78997464)(321.88031006,225.78997541)
\curveto(322.43499339,225.78997464)(322.84710236,226.08684935)(323.11663818,226.68060041)
\curveto(323.39007056,227.27434816)(323.52678918,228.17669101)(323.52679443,229.38763166)
\curveto(323.52678918,230.60246983)(323.39007056,231.5067658)(323.11663818,232.10052229)
\curveto(322.84710236,232.69426461)(322.43499339,232.99113932)(321.88031006,232.99114729)
\moveto(321.88031006,233.92864729)
\curveto(322.81389927,233.92863838)(323.51897669,233.54582626)(323.99554443,232.78020979)
\curveto(324.47600698,232.01457779)(324.71624111,230.88371955)(324.71624756,229.38763166)
\curveto(324.71624111,227.89544129)(324.47600698,226.76653617)(323.99554443,226.00091291)
\curveto(323.51897669,225.2352877)(322.81389927,224.85247558)(321.88031006,224.85247541)
\curveto(320.94671363,224.85247558)(320.24163621,225.2352877)(319.76507568,226.00091291)
\curveto(319.28851217,226.76653617)(319.05023115,227.89544129)(319.05023193,229.38763166)
\curveto(319.05023115,230.88371955)(319.28851217,232.01457779)(319.76507568,232.78020979)
\curveto(320.24163621,233.54582626)(320.94671363,233.92863838)(321.88031006,233.92864729)
}
}
{
\newrgbcolor{curcolor}{0 0 0}
\pscustom[linestyle=none,fillstyle=solid,fillcolor=curcolor]
{
\newpath
\moveto(222.92411804,209.13580061)
\lineto(226.94364929,209.13580061)
\lineto(226.94364929,208.13970686)
\lineto(221.62919617,208.13970686)
\lineto(221.62919617,209.13580061)
\curveto(222.3596633,209.90533009)(222.99833453,210.58501691)(223.54521179,211.17486311)
\curveto(224.09208344,211.76470323)(224.46903619,212.18071844)(224.67607117,212.42290998)
\curveto(225.06669184,212.89946772)(225.33036345,213.28423296)(225.46708679,213.57720686)
\curveto(225.60380068,213.87407612)(225.67215998,214.17681019)(225.67216492,214.48540998)
\curveto(225.67215998,214.9736844)(225.52762888,215.35649651)(225.23857117,215.63384748)
\curveto(224.9534107,215.91118346)(224.56083297,216.0498552)(224.06083679,216.04986311)
\curveto(223.70536508,216.0498552)(223.33231857,215.98540213)(222.94169617,215.85650373)
\curveto(222.55106935,215.72758989)(222.13700727,215.53227759)(221.69950867,215.27056623)
\lineto(221.69950867,216.46587873)
\curveto(222.10185105,216.65727646)(222.49638191,216.80180757)(222.88310242,216.89947248)
\curveto(223.27372488,216.99711987)(223.65849012,217.04594795)(224.03739929,217.04595686)
\curveto(224.89286389,217.04594795)(225.5803632,216.81743255)(226.09989929,216.36040998)
\curveto(226.62333091,215.90727721)(226.8850494,215.31157468)(226.88505554,214.57330061)
\curveto(226.8850494,214.19829455)(226.79715886,213.82329492)(226.62138367,213.44830061)
\curveto(226.44950296,213.07329567)(226.16825324,212.65923359)(225.77763367,212.20611311)
\curveto(225.55887885,211.95220304)(225.24051979,211.60064089)(224.82255554,211.15142561)
\curveto(224.40848937,210.70220429)(223.7756775,210.03032996)(222.92411804,209.13580061)
}
}
{
\newrgbcolor{curcolor}{0 0 0}
\pscustom[linestyle=none,fillstyle=solid,fillcolor=curcolor]
{
\newpath
\moveto(255.52357483,212.53204084)
\curveto(255.523572,212.74687998)(255.59779067,212.93242667)(255.74623108,213.08868146)
\curveto(255.89857162,213.24492636)(256.08021207,213.32305128)(256.29115295,213.32305646)
\curveto(256.50989914,213.32305128)(256.69739895,213.24492636)(256.85365295,213.08868146)
\curveto(257.00989864,212.93242667)(257.08802356,212.74687998)(257.08802795,212.53204084)
\curveto(257.08802356,212.31328666)(257.00989864,212.12773997)(256.85365295,211.97540021)
\curveto(256.7013052,211.82305278)(256.51380538,211.74688098)(256.29115295,211.74688459)
\curveto(256.07239958,211.74688098)(255.88880601,211.82109966)(255.7403717,211.96954084)
\curveto(255.59583755,212.11797436)(255.523572,212.30547417)(255.52357483,212.53204084)
\moveto(256.3028717,216.10625959)
\curveto(255.7520874,216.10625162)(255.33997843,215.80937692)(255.06654358,215.21563459)
\curveto(254.79701023,214.62187811)(254.66224474,213.71758213)(254.6622467,212.50274396)
\curveto(254.66224474,211.29180331)(254.79701023,210.38946046)(255.06654358,209.79571271)
\curveto(255.33997843,209.20196165)(255.7520874,208.90508695)(256.3028717,208.90508771)
\curveto(256.85755504,208.90508695)(257.269664,209.20196165)(257.53919983,209.79571271)
\curveto(257.81263221,210.38946046)(257.94935082,211.29180331)(257.94935608,212.50274396)
\curveto(257.94935082,213.71758213)(257.81263221,214.62187811)(257.53919983,215.21563459)
\curveto(257.269664,215.80937692)(256.85755504,216.10625162)(256.3028717,216.10625959)
\moveto(256.3028717,217.04375959)
\curveto(257.23646091,217.04375068)(257.94153833,216.66093857)(258.41810608,215.89532209)
\curveto(258.89856862,215.1296901)(259.13880276,213.99883185)(259.1388092,212.50274396)
\curveto(259.13880276,211.01055359)(258.89856862,209.88164847)(258.41810608,209.11602521)
\curveto(257.94153833,208.3504)(257.23646091,207.96758788)(256.3028717,207.96758771)
\curveto(255.36927528,207.96758788)(254.66419786,208.3504)(254.18763733,209.11602521)
\curveto(253.71107381,209.88164847)(253.4727928,211.01055359)(253.47279358,212.50274396)
\curveto(253.4727928,213.99883185)(253.71107381,215.1296901)(254.18763733,215.89532209)
\curveto(254.66419786,216.66093857)(255.36927528,217.04375068)(256.3028717,217.04375959)
}
}
{
\newrgbcolor{curcolor}{0 0 0}
\pscustom[linestyle=none,fillstyle=solid,fillcolor=curcolor]
{
\newpath
\moveto(286.03128052,209.06084943)
\lineto(287.87112427,209.06084943)
\lineto(287.87112427,215.74639631)
\lineto(285.89065552,215.30108381)
\lineto(285.89065552,216.37920881)
\lineto(287.85940552,216.81280256)
\lineto(289.04299927,216.81280256)
\lineto(289.04299927,209.06084943)
\lineto(290.85940552,209.06084943)
\lineto(290.85940552,208.06475568)
\lineto(286.03128052,208.06475568)
\lineto(286.03128052,209.06084943)
}
}
{
\newrgbcolor{curcolor}{0 0 0}
\pscustom[linestyle=none,fillstyle=solid,fillcolor=curcolor]
{
\newpath
\moveto(321.86859131,211.96124006)
\curveto(321.34124448,211.9612359)(320.93304176,211.81279855)(320.64398193,211.51592756)
\curveto(320.35882358,211.22295539)(320.2162456,210.80694018)(320.21624756,210.26788068)
\curveto(320.2162456,209.72881626)(320.36077671,209.30889481)(320.64984131,209.00811506)
\curveto(320.94280737,208.71123915)(321.34905697,208.5628018)(321.86859131,208.56280256)
\curveto(322.39983717,208.5628018)(322.80803988,208.70928603)(323.09320068,209.00225568)
\curveto(323.38225806,209.29912919)(323.52678917,209.72100377)(323.52679443,210.26788068)
\curveto(323.52678917,210.80303394)(323.38030494,211.21904915)(323.08734131,211.51592756)
\curveto(322.79827427,211.81279855)(322.39202468,211.9612359)(321.86859131,211.96124006)
\moveto(320.83734131,212.45342756)
\curveto(320.33343298,212.58232903)(319.93890213,212.82256317)(319.65374756,213.17413068)
\curveto(319.37249645,213.52568746)(319.23187159,213.94951517)(319.23187256,214.44561506)
\curveto(319.23187159,215.14092022)(319.46819947,215.69170092)(319.94085693,216.09795881)
\curveto(320.41351103,216.50810636)(321.05608851,216.71318428)(321.86859131,216.71319318)
\curveto(322.68499313,216.71318428)(323.32952374,216.50810636)(323.80218506,216.09795881)
\curveto(324.27483529,215.69170092)(324.51116318,215.14092022)(324.51116943,214.44561506)
\curveto(324.51116318,213.94951517)(324.3685852,213.52568746)(324.08343506,213.17413068)
\curveto(323.80217952,212.82256317)(323.40960178,212.58232903)(322.90570068,212.45342756)
\curveto(323.49163295,212.32451679)(323.93889813,212.06475143)(324.24749756,211.67413068)
\curveto(324.55999126,211.28350221)(324.7162411,210.77764334)(324.71624756,210.15655256)
\curveto(324.7162411,209.3674885)(324.46428823,208.75030162)(323.96038818,208.30499006)
\curveto(323.45647674,207.85967751)(322.75921181,207.63702148)(321.86859131,207.63702131)
\curveto(320.97796359,207.63702148)(320.28069866,207.85772438)(319.77679443,208.29913068)
\curveto(319.27679342,208.74444225)(319.02679367,209.35967601)(319.02679443,210.14483381)
\curveto(319.02679367,210.76983085)(319.18109039,211.27764284)(319.48968506,211.66827131)
\curveto(319.80218352,212.0627983)(320.25140182,212.32451679)(320.83734131,212.45342756)
\moveto(320.40960693,214.33428693)
\curveto(320.40960478,213.86553088)(320.53460466,213.50810936)(320.78460693,213.26202131)
\curveto(321.03460416,213.01592235)(321.39593192,212.8928756)(321.86859131,212.89288068)
\curveto(322.34514972,212.8928756)(322.70843061,213.01592235)(322.95843506,213.26202131)
\curveto(323.20843011,213.50810936)(323.33342998,213.86553088)(323.33343506,214.33428693)
\curveto(323.33342998,214.81084243)(323.20843011,215.17412332)(322.95843506,215.42413068)
\curveto(322.71233686,215.67412282)(322.34905597,215.79912269)(321.86859131,215.79913068)
\curveto(321.39593192,215.79912269)(321.03460416,215.67216969)(320.78460693,215.41827131)
\curveto(320.53460466,215.16826395)(320.40960478,214.80693618)(320.40960693,214.33428693)
}
}
{
\newrgbcolor{curcolor}{0 0 0}
\pscustom[linestyle=none,fillstyle=solid,fillcolor=curcolor]
{
\newpath
\moveto(223.32841492,195.65240217)
\curveto(223.32841209,195.86724131)(223.40263076,196.052788)(223.55107117,196.20904279)
\curveto(223.70341171,196.36528769)(223.88505216,196.44341261)(224.09599304,196.44341779)
\curveto(224.31473923,196.44341261)(224.50223904,196.36528769)(224.65849304,196.20904279)
\curveto(224.81473873,196.052788)(224.89286365,195.86724131)(224.89286804,195.65240217)
\curveto(224.89286365,195.43364799)(224.81473873,195.2481013)(224.65849304,195.09576154)
\curveto(224.50614528,194.94341411)(224.31864547,194.86724231)(224.09599304,194.86724592)
\curveto(223.87723966,194.86724231)(223.6936461,194.94146098)(223.54521179,195.08990217)
\curveto(223.40067764,195.23833569)(223.32841209,195.4258355)(223.32841492,195.65240217)
\moveto(224.10771179,199.22662092)
\curveto(223.55692748,199.22661295)(223.14481852,198.92973825)(222.87138367,198.33599592)
\curveto(222.60185031,197.74223943)(222.46708482,196.83794346)(222.46708679,195.62310529)
\curveto(222.46708482,194.41216464)(222.60185031,193.50982179)(222.87138367,192.91607404)
\curveto(223.14481852,192.32232298)(223.55692748,192.02544828)(224.10771179,192.02544904)
\curveto(224.66239513,192.02544828)(225.07450409,192.32232298)(225.34403992,192.91607404)
\curveto(225.6174723,193.50982179)(225.75419091,194.41216464)(225.75419617,195.62310529)
\curveto(225.75419091,196.83794346)(225.6174723,197.74223943)(225.34403992,198.33599592)
\curveto(225.07450409,198.92973825)(224.66239513,199.22661295)(224.10771179,199.22662092)
\moveto(224.10771179,200.16412092)
\curveto(225.041301,200.16411201)(225.74637842,199.78129989)(226.22294617,199.01568342)
\curveto(226.70340871,198.25005143)(226.94364285,197.11919318)(226.94364929,195.62310529)
\curveto(226.94364285,194.13091492)(226.70340871,193.0020098)(226.22294617,192.23638654)
\curveto(225.74637842,191.47076133)(225.041301,191.08794921)(224.10771179,191.08794904)
\curveto(223.17411537,191.08794921)(222.46903795,191.47076133)(221.99247742,192.23638654)
\curveto(221.5159139,193.0020098)(221.27763289,194.13091492)(221.27763367,195.62310529)
\curveto(221.27763289,197.11919318)(221.5159139,198.25005143)(221.99247742,199.01568342)
\curveto(222.46903795,199.78129989)(223.17411537,200.16411201)(224.10771179,200.16412092)
}
}
{
\newrgbcolor{curcolor}{0 0 0}
\pscustom[linestyle=none,fillstyle=solid,fillcolor=curcolor]
{
\newpath
\moveto(255.52357483,195.50774885)
\curveto(255.523572,195.72258799)(255.59779067,195.90813468)(255.74623108,196.06438947)
\curveto(255.89857162,196.22063437)(256.08021207,196.29875929)(256.29115295,196.29876447)
\curveto(256.50989914,196.29875929)(256.69739895,196.22063437)(256.85365295,196.06438947)
\curveto(257.00989864,195.90813468)(257.08802356,195.72258799)(257.08802795,195.50774885)
\curveto(257.08802356,195.28899467)(257.00989864,195.10344798)(256.85365295,194.95110822)
\curveto(256.7013052,194.79876079)(256.51380538,194.72258899)(256.29115295,194.7225926)
\curveto(256.07239958,194.72258899)(255.88880601,194.79680766)(255.7403717,194.94524885)
\curveto(255.59583755,195.09368237)(255.523572,195.28118218)(255.52357483,195.50774885)
\moveto(256.3028717,199.0819676)
\curveto(255.7520874,199.08195963)(255.33997843,198.78508493)(255.06654358,198.1913426)
\curveto(254.79701023,197.59758611)(254.66224474,196.69329014)(254.6622467,195.47845197)
\curveto(254.66224474,194.26751132)(254.79701023,193.36516847)(255.06654358,192.77142072)
\curveto(255.33997843,192.17766966)(255.7520874,191.88079496)(256.3028717,191.88079572)
\curveto(256.85755504,191.88079496)(257.269664,192.17766966)(257.53919983,192.77142072)
\curveto(257.81263221,193.36516847)(257.94935082,194.26751132)(257.94935608,195.47845197)
\curveto(257.94935082,196.69329014)(257.81263221,197.59758611)(257.53919983,198.1913426)
\curveto(257.269664,198.78508493)(256.85755504,199.08195963)(256.3028717,199.0819676)
\moveto(256.3028717,200.0194676)
\curveto(257.23646091,200.01945869)(257.94153833,199.63664657)(258.41810608,198.8710301)
\curveto(258.89856862,198.10539811)(259.13880276,196.97453986)(259.1388092,195.47845197)
\curveto(259.13880276,193.9862616)(258.89856862,192.85735648)(258.41810608,192.09173322)
\curveto(257.94153833,191.32610801)(257.23646091,190.94329589)(256.3028717,190.94329572)
\curveto(255.36927528,190.94329589)(254.66419786,191.32610801)(254.18763733,192.09173322)
\curveto(253.71107381,192.85735648)(253.4727928,193.9862616)(253.47279358,195.47845197)
\curveto(253.4727928,196.97453986)(253.71107381,198.10539811)(254.18763733,198.8710301)
\curveto(254.66419786,199.63664657)(255.36927528,200.01945869)(256.3028717,200.0194676)
}
}
{
\newrgbcolor{curcolor}{0 0 0}
\pscustom[linestyle=none,fillstyle=solid,fillcolor=curcolor]
{
\newpath
\moveto(287.24414062,195.61102033)
\curveto(287.24413779,195.82585947)(287.31835647,196.01140616)(287.46679688,196.16766096)
\curveto(287.61913742,196.32390585)(287.80077786,196.40203077)(288.01171875,196.40203596)
\curveto(288.23046493,196.40203077)(288.41796475,196.32390585)(288.57421875,196.16766096)
\curveto(288.73046443,196.01140616)(288.80858936,195.82585947)(288.80859375,195.61102033)
\curveto(288.80858936,195.39226616)(288.73046443,195.20671947)(288.57421875,195.05437971)
\curveto(288.42187099,194.90203227)(288.23437118,194.82586047)(288.01171875,194.82586408)
\curveto(287.79296537,194.82586047)(287.6093718,194.90007915)(287.4609375,195.04852033)
\curveto(287.31640335,195.19695385)(287.24413779,195.38445366)(287.24414062,195.61102033)
\moveto(288.0234375,199.18523908)
\curveto(287.47265319,199.18523111)(287.06054423,198.88835641)(286.78710938,198.29461408)
\curveto(286.51757602,197.7008576)(286.38281053,196.79656163)(286.3828125,195.58172346)
\curveto(286.38281053,194.3707828)(286.51757602,193.46843996)(286.78710938,192.87469221)
\curveto(287.06054423,192.28094114)(287.47265319,191.98406644)(288.0234375,191.98406721)
\curveto(288.57812084,191.98406644)(288.9902298,192.28094114)(289.25976562,192.87469221)
\curveto(289.53319801,193.46843996)(289.66991662,194.3707828)(289.66992188,195.58172346)
\curveto(289.66991662,196.79656163)(289.53319801,197.7008576)(289.25976562,198.29461408)
\curveto(288.9902298,198.88835641)(288.57812084,199.18523111)(288.0234375,199.18523908)
\moveto(288.0234375,200.12273908)
\curveto(288.95702671,200.12273018)(289.66210413,199.73991806)(290.13867188,198.97430158)
\curveto(290.61913442,198.20866959)(290.85936855,197.07781135)(290.859375,195.58172346)
\curveto(290.85936855,194.08953308)(290.61913442,192.96062796)(290.13867188,192.19500471)
\curveto(289.66210413,191.42937949)(288.95702671,191.04656738)(288.0234375,191.04656721)
\curveto(287.08984107,191.04656738)(286.38476365,191.42937949)(285.90820312,192.19500471)
\curveto(285.43163961,192.96062796)(285.1933586,194.08953308)(285.19335938,195.58172346)
\curveto(285.1933586,197.07781135)(285.43163961,198.20866959)(285.90820312,198.97430158)
\curveto(286.38476365,199.73991806)(287.08984107,200.12273018)(288.0234375,200.12273908)
}
}
{
\newrgbcolor{curcolor}{0 0 0}
\pscustom[linestyle=none,fillstyle=solid,fillcolor=curcolor]
{
\newpath
\moveto(324.07757568,199.14617658)
\lineto(324.07757568,198.05633283)
\curveto(323.83147637,198.20085647)(323.56975788,198.31023136)(323.29241943,198.38445783)
\curveto(323.01507094,198.46257496)(322.72600873,198.50163742)(322.42523193,198.50164533)
\curveto(321.67522853,198.50163742)(321.10686972,198.21843458)(320.72015381,197.65203596)
\curveto(320.333433,197.08952946)(320.14007381,196.25945216)(320.14007568,195.16180158)
\curveto(320.32757363,195.55242162)(320.58733899,195.85124945)(320.91937256,196.05828596)
\curveto(321.25140083,196.26921778)(321.63225982,196.37468642)(322.06195068,196.37469221)
\curveto(322.90569605,196.37468642)(323.55803915,196.11492106)(324.01898193,195.59539533)
\curveto(324.48381947,195.07976584)(324.71624111,194.3473447)(324.71624756,193.39812971)
\curveto(324.71624111,192.45281535)(324.4779601,191.7203942)(324.00140381,191.20086408)
\curveto(323.52483605,190.68133274)(322.85491485,190.42156738)(321.99163818,190.42156721)
\curveto(320.97601048,190.42156738)(320.2318706,190.78484826)(319.75921631,191.51141096)
\curveto(319.28655904,192.24187806)(319.05023115,193.39031441)(319.05023193,194.95672346)
\curveto(319.05023115,196.43328012)(319.333434,197.55827899)(319.89984131,198.33172346)
\curveto(320.47015161,199.10905869)(321.29436954,199.49773018)(322.37249756,199.49773908)
\curveto(322.66155567,199.49773018)(322.95061788,199.46648021)(323.23968506,199.40398908)
\curveto(323.5287423,199.34538658)(323.8080389,199.25944916)(324.07757568,199.14617658)
\moveto(321.96820068,195.44891096)
\curveto(321.46429124,195.4489061)(321.06780726,195.26726566)(320.77874756,194.90398908)
\curveto(320.48968284,194.54070388)(320.34515173,194.03875126)(320.34515381,193.39812971)
\curveto(320.34515173,192.75750254)(320.48968284,192.25554992)(320.77874756,191.89227033)
\curveto(321.06780726,191.52898814)(321.46429124,191.3473477)(321.96820068,191.34734846)
\curveto(322.49163396,191.3473477)(322.88616482,191.51922253)(323.15179443,191.86297346)
\curveto(323.41741429,192.21062809)(323.55022665,192.72234633)(323.55023193,193.39812971)
\curveto(323.55022665,194.07781372)(323.41741429,194.58953196)(323.15179443,194.93328596)
\curveto(322.88616482,195.27703127)(322.49163396,195.4489061)(321.96820068,195.44891096)
}
}
{
\newrgbcolor{curcolor}{0 0 0}
\pscustom[linestyle=none,fillstyle=solid,fillcolor=curcolor]
{
\newpath
\moveto(223.32841492,178.60064436)
\curveto(223.32841209,178.8154835)(223.40263076,179.00103019)(223.55107117,179.15728498)
\curveto(223.70341171,179.31352987)(223.88505216,179.39165479)(224.09599304,179.39165998)
\curveto(224.31473923,179.39165479)(224.50223904,179.31352987)(224.65849304,179.15728498)
\curveto(224.81473873,179.00103019)(224.89286365,178.8154835)(224.89286804,178.60064436)
\curveto(224.89286365,178.38189018)(224.81473873,178.19634349)(224.65849304,178.04400373)
\curveto(224.50614528,177.89165629)(224.31864547,177.8154845)(224.09599304,177.81548811)
\curveto(223.87723966,177.8154845)(223.6936461,177.88970317)(223.54521179,178.03814436)
\curveto(223.40067764,178.18657788)(223.32841209,178.37407769)(223.32841492,178.60064436)
\moveto(224.10771179,182.17486311)
\curveto(223.55692748,182.17485514)(223.14481852,181.87798043)(222.87138367,181.28423811)
\curveto(222.60185031,180.69048162)(222.46708482,179.78618565)(222.46708679,178.57134748)
\curveto(222.46708482,177.36040683)(222.60185031,176.45806398)(222.87138367,175.86431623)
\curveto(223.14481852,175.27056517)(223.55692748,174.97369046)(224.10771179,174.97369123)
\curveto(224.66239513,174.97369046)(225.07450409,175.27056517)(225.34403992,175.86431623)
\curveto(225.6174723,176.45806398)(225.75419091,177.36040683)(225.75419617,178.57134748)
\curveto(225.75419091,179.78618565)(225.6174723,180.69048162)(225.34403992,181.28423811)
\curveto(225.07450409,181.87798043)(224.66239513,182.17485514)(224.10771179,182.17486311)
\moveto(224.10771179,183.11236311)
\curveto(225.041301,183.1123542)(225.74637842,182.72954208)(226.22294617,181.96392561)
\curveto(226.70340871,181.19829361)(226.94364285,180.06743537)(226.94364929,178.57134748)
\curveto(226.94364285,177.07915711)(226.70340871,175.95025199)(226.22294617,175.18462873)
\curveto(225.74637842,174.41900352)(225.041301,174.0361914)(224.10771179,174.03619123)
\curveto(223.17411537,174.0361914)(222.46903795,174.41900352)(221.99247742,175.18462873)
\curveto(221.5159139,175.95025199)(221.27763289,177.07915711)(221.27763367,178.57134748)
\curveto(221.27763289,180.06743537)(221.5159139,181.19829361)(221.99247742,181.96392561)
\curveto(222.46903795,182.72954208)(223.17411537,183.1123542)(224.10771179,183.11236311)
}
}
{
\newrgbcolor{curcolor}{0 0 0}
\pscustom[linestyle=none,fillstyle=solid,fillcolor=curcolor]
{
\newpath
\moveto(255.52357483,178.48357893)
\curveto(255.523572,178.69841807)(255.59779067,178.88396476)(255.74623108,179.04021955)
\curveto(255.89857162,179.19646444)(256.08021207,179.27458937)(256.29115295,179.27459455)
\curveto(256.50989914,179.27458937)(256.69739895,179.19646444)(256.85365295,179.04021955)
\curveto(257.00989864,178.88396476)(257.08802356,178.69841807)(257.08802795,178.48357893)
\curveto(257.08802356,178.26482475)(257.00989864,178.07927806)(256.85365295,177.9269383)
\curveto(256.7013052,177.77459087)(256.51380538,177.69841907)(256.29115295,177.69842268)
\curveto(256.07239958,177.69841907)(255.88880601,177.77263774)(255.7403717,177.92107893)
\curveto(255.59583755,178.06951245)(255.523572,178.25701226)(255.52357483,178.48357893)
\moveto(256.3028717,182.05779768)
\curveto(255.7520874,182.05778971)(255.33997843,181.760915)(255.06654358,181.16717268)
\curveto(254.79701023,180.57341619)(254.66224474,179.66912022)(254.6622467,178.45428205)
\curveto(254.66224474,177.2433414)(254.79701023,176.34099855)(255.06654358,175.7472508)
\curveto(255.33997843,175.15349974)(255.7520874,174.85662503)(256.3028717,174.8566258)
\curveto(256.85755504,174.85662503)(257.269664,175.15349974)(257.53919983,175.7472508)
\curveto(257.81263221,176.34099855)(257.94935082,177.2433414)(257.94935608,178.45428205)
\curveto(257.94935082,179.66912022)(257.81263221,180.57341619)(257.53919983,181.16717268)
\curveto(257.269664,181.760915)(256.85755504,182.05778971)(256.3028717,182.05779768)
\moveto(256.3028717,182.99529768)
\curveto(257.23646091,182.99528877)(257.94153833,182.61247665)(258.41810608,181.84686018)
\curveto(258.89856862,181.08122818)(259.13880276,179.95036994)(259.1388092,178.45428205)
\curveto(259.13880276,176.96209168)(258.89856862,175.83318656)(258.41810608,175.0675633)
\curveto(257.94153833,174.30193809)(257.23646091,173.91912597)(256.3028717,173.9191258)
\curveto(255.36927528,173.91912597)(254.66419786,174.30193809)(254.18763733,175.0675633)
\curveto(253.71107381,175.83318656)(253.4727928,176.96209168)(253.47279358,178.45428205)
\curveto(253.4727928,179.95036994)(253.71107381,181.08122818)(254.18763733,181.84686018)
\curveto(254.66419786,182.61247665)(255.36927528,182.99528877)(256.3028717,182.99529768)
}
}
{
\newrgbcolor{curcolor}{0 0 0}
\pscustom[linestyle=none,fillstyle=solid,fillcolor=curcolor]
{
\newpath
\moveto(287.24414062,178.59295393)
\curveto(287.24413779,178.80779307)(287.31835647,178.99333976)(287.46679688,179.14959455)
\curveto(287.61913742,179.30583944)(287.80077786,179.38396437)(288.01171875,179.38396955)
\curveto(288.23046493,179.38396437)(288.41796475,179.30583944)(288.57421875,179.14959455)
\curveto(288.73046443,178.99333976)(288.80858936,178.80779307)(288.80859375,178.59295393)
\curveto(288.80858936,178.37419975)(288.73046443,178.18865306)(288.57421875,178.0363133)
\curveto(288.42187099,177.88396587)(288.23437118,177.80779407)(288.01171875,177.80779768)
\curveto(287.79296537,177.80779407)(287.6093718,177.88201274)(287.4609375,178.03045393)
\curveto(287.31640335,178.17888745)(287.24413779,178.36638726)(287.24414062,178.59295393)
\moveto(288.0234375,182.16717268)
\curveto(287.47265319,182.16716471)(287.06054423,181.87029)(286.78710938,181.27654768)
\curveto(286.51757602,180.68279119)(286.38281053,179.77849522)(286.3828125,178.56365705)
\curveto(286.38281053,177.3527164)(286.51757602,176.45037355)(286.78710938,175.8566258)
\curveto(287.06054423,175.26287474)(287.47265319,174.96600003)(288.0234375,174.9660008)
\curveto(288.57812084,174.96600003)(288.9902298,175.26287474)(289.25976562,175.8566258)
\curveto(289.53319801,176.45037355)(289.66991662,177.3527164)(289.66992188,178.56365705)
\curveto(289.66991662,179.77849522)(289.53319801,180.68279119)(289.25976562,181.27654768)
\curveto(288.9902298,181.87029)(288.57812084,182.16716471)(288.0234375,182.16717268)
\moveto(288.0234375,183.10467268)
\curveto(288.95702671,183.10466377)(289.66210413,182.72185165)(290.13867188,181.95623518)
\curveto(290.61913442,181.19060318)(290.85936855,180.05974494)(290.859375,178.56365705)
\curveto(290.85936855,177.07146668)(290.61913442,175.94256156)(290.13867188,175.1769383)
\curveto(289.66210413,174.41131309)(288.95702671,174.02850097)(288.0234375,174.0285008)
\curveto(287.08984107,174.02850097)(286.38476365,174.41131309)(285.90820312,175.1769383)
\curveto(285.43163961,175.94256156)(285.1933586,177.07146668)(285.19335938,178.56365705)
\curveto(285.1933586,180.05974494)(285.43163961,181.19060318)(285.90820312,181.95623518)
\curveto(286.38476365,182.72185165)(287.08984107,183.10466377)(288.0234375,183.10467268)
}
}
{
\newrgbcolor{curcolor}{0 0 0}
\pscustom[linestyle=none,fillstyle=solid,fillcolor=curcolor]
{
\newpath
\moveto(319.88812256,174.53045393)
\lineto(321.72796631,174.53045393)
\lineto(321.72796631,181.2160008)
\lineto(319.74749756,180.7706883)
\lineto(319.74749756,181.8488133)
\lineto(321.71624756,182.28240705)
\lineto(322.89984131,182.28240705)
\lineto(322.89984131,174.53045393)
\lineto(324.71624756,174.53045393)
\lineto(324.71624756,173.53436018)
\lineto(319.88812256,173.53436018)
\lineto(319.88812256,174.53045393)
}
}
{
\newrgbcolor{curcolor}{0 0 0}
\pscustom[linestyle=none,fillstyle=solid,fillcolor=curcolor]
{
\newpath
\moveto(223.32841492,161.54888654)
\curveto(223.32841209,161.76372568)(223.40263076,161.94927237)(223.55107117,162.10552717)
\curveto(223.70341171,162.26177206)(223.88505216,162.33989698)(224.09599304,162.33990217)
\curveto(224.31473923,162.33989698)(224.50223904,162.26177206)(224.65849304,162.10552717)
\curveto(224.81473873,161.94927237)(224.89286365,161.76372568)(224.89286804,161.54888654)
\curveto(224.89286365,161.33013237)(224.81473873,161.14458568)(224.65849304,160.99224592)
\curveto(224.50614528,160.83989848)(224.31864547,160.76372668)(224.09599304,160.76373029)
\curveto(223.87723966,160.76372668)(223.6936461,160.83794536)(223.54521179,160.98638654)
\curveto(223.40067764,161.13482006)(223.32841209,161.32231988)(223.32841492,161.54888654)
\moveto(224.10771179,165.12310529)
\curveto(223.55692748,165.12309732)(223.14481852,164.82622262)(222.87138367,164.23248029)
\curveto(222.60185031,163.63872381)(222.46708482,162.73442784)(222.46708679,161.51958967)
\curveto(222.46708482,160.30864901)(222.60185031,159.40630617)(222.87138367,158.81255842)
\curveto(223.14481852,158.21880735)(223.55692748,157.92193265)(224.10771179,157.92193342)
\curveto(224.66239513,157.92193265)(225.07450409,158.21880735)(225.34403992,158.81255842)
\curveto(225.6174723,159.40630617)(225.75419091,160.30864901)(225.75419617,161.51958967)
\curveto(225.75419091,162.73442784)(225.6174723,163.63872381)(225.34403992,164.23248029)
\curveto(225.07450409,164.82622262)(224.66239513,165.12309732)(224.10771179,165.12310529)
\moveto(224.10771179,166.06060529)
\curveto(225.041301,166.06059639)(225.74637842,165.67778427)(226.22294617,164.91216779)
\curveto(226.70340871,164.1465358)(226.94364285,163.01567756)(226.94364929,161.51958967)
\curveto(226.94364285,160.02739929)(226.70340871,158.89849417)(226.22294617,158.13287092)
\curveto(225.74637842,157.36724571)(225.041301,156.98443359)(224.10771179,156.98443342)
\curveto(223.17411537,156.98443359)(222.46903795,157.36724571)(221.99247742,158.13287092)
\curveto(221.5159139,158.89849417)(221.27763289,160.02739929)(221.27763367,161.51958967)
\curveto(221.27763289,163.01567756)(221.5159139,164.1465358)(221.99247742,164.91216779)
\curveto(222.46903795,165.67778427)(223.17411537,166.06059639)(224.10771179,166.06060529)
}
}
{
\newrgbcolor{curcolor}{0 0 0}
\pscustom[linestyle=none,fillstyle=solid,fillcolor=curcolor]
{
\newpath
\moveto(255.52357483,161.45928693)
\curveto(255.523572,161.67412607)(255.59779067,161.85967276)(255.74623108,162.01592756)
\curveto(255.89857162,162.17217245)(256.08021207,162.25029737)(256.29115295,162.25030256)
\curveto(256.50989914,162.25029737)(256.69739895,162.17217245)(256.85365295,162.01592756)
\curveto(257.00989864,161.85967276)(257.08802356,161.67412607)(257.08802795,161.45928693)
\curveto(257.08802356,161.24053276)(257.00989864,161.05498607)(256.85365295,160.90264631)
\curveto(256.7013052,160.75029887)(256.51380538,160.67412707)(256.29115295,160.67413068)
\curveto(256.07239958,160.67412707)(255.88880601,160.74834575)(255.7403717,160.89678693)
\curveto(255.59583755,161.04522045)(255.523572,161.23272027)(255.52357483,161.45928693)
\moveto(256.3028717,165.03350568)
\curveto(255.7520874,165.03349771)(255.33997843,164.73662301)(255.06654358,164.14288068)
\curveto(254.79701023,163.5491242)(254.66224474,162.64482823)(254.6622467,161.42999006)
\curveto(254.66224474,160.2190494)(254.79701023,159.31670656)(255.06654358,158.72295881)
\curveto(255.33997843,158.12920774)(255.7520874,157.83233304)(256.3028717,157.83233381)
\curveto(256.85755504,157.83233304)(257.269664,158.12920774)(257.53919983,158.72295881)
\curveto(257.81263221,159.31670656)(257.94935082,160.2190494)(257.94935608,161.42999006)
\curveto(257.94935082,162.64482823)(257.81263221,163.5491242)(257.53919983,164.14288068)
\curveto(257.269664,164.73662301)(256.85755504,165.03349771)(256.3028717,165.03350568)
\moveto(256.3028717,165.97100568)
\curveto(257.23646091,165.97099678)(257.94153833,165.58818466)(258.41810608,164.82256818)
\curveto(258.89856862,164.05693619)(259.13880276,162.92607795)(259.1388092,161.42999006)
\curveto(259.13880276,159.93779969)(258.89856862,158.80889456)(258.41810608,158.04327131)
\curveto(257.94153833,157.2776461)(257.23646091,156.89483398)(256.3028717,156.89483381)
\curveto(255.36927528,156.89483398)(254.66419786,157.2776461)(254.18763733,158.04327131)
\curveto(253.71107381,158.80889456)(253.4727928,159.93779969)(253.47279358,161.42999006)
\curveto(253.4727928,162.92607795)(253.71107381,164.05693619)(254.18763733,164.82256818)
\curveto(254.66419786,165.58818466)(255.36927528,165.97099678)(256.3028717,165.97100568)
}
}
{
\newrgbcolor{curcolor}{0 0 0}
\pscustom[linestyle=none,fillstyle=solid,fillcolor=curcolor]
{
\newpath
\moveto(287.24414062,161.57476545)
\curveto(287.24413779,161.78960459)(287.31835647,161.97515128)(287.46679688,162.13140607)
\curveto(287.61913742,162.28765097)(287.80077786,162.36577589)(288.01171875,162.36578107)
\curveto(288.23046493,162.36577589)(288.41796475,162.28765097)(288.57421875,162.13140607)
\curveto(288.73046443,161.97515128)(288.80858936,161.78960459)(288.80859375,161.57476545)
\curveto(288.80858936,161.35601127)(288.73046443,161.17046458)(288.57421875,161.01812482)
\curveto(288.42187099,160.86577739)(288.23437118,160.78960559)(288.01171875,160.7896092)
\curveto(287.79296537,160.78960559)(287.6093718,160.86382427)(287.4609375,161.01226545)
\curveto(287.31640335,161.16069897)(287.24413779,161.34819878)(287.24414062,161.57476545)
\moveto(288.0234375,165.1489842)
\curveto(287.47265319,165.14897623)(287.06054423,164.85210153)(286.78710938,164.2583592)
\curveto(286.51757602,163.66460271)(286.38281053,162.76030674)(286.3828125,161.54546857)
\curveto(286.38281053,160.33452792)(286.51757602,159.43218507)(286.78710938,158.83843732)
\curveto(287.06054423,158.24468626)(287.47265319,157.94781156)(288.0234375,157.94781232)
\curveto(288.57812084,157.94781156)(288.9902298,158.24468626)(289.25976562,158.83843732)
\curveto(289.53319801,159.43218507)(289.66991662,160.33452792)(289.66992188,161.54546857)
\curveto(289.66991662,162.76030674)(289.53319801,163.66460271)(289.25976562,164.2583592)
\curveto(288.9902298,164.85210153)(288.57812084,165.14897623)(288.0234375,165.1489842)
\moveto(288.0234375,166.0864842)
\curveto(288.95702671,166.08647529)(289.66210413,165.70366318)(290.13867188,164.9380467)
\curveto(290.61913442,164.17241471)(290.85936855,163.04155646)(290.859375,161.54546857)
\curveto(290.85936855,160.0532782)(290.61913442,158.92437308)(290.13867188,158.15874982)
\curveto(289.66210413,157.39312461)(288.95702671,157.01031249)(288.0234375,157.01031232)
\curveto(287.08984107,157.01031249)(286.38476365,157.39312461)(285.90820312,158.15874982)
\curveto(285.43163961,158.92437308)(285.1933586,160.0532782)(285.19335938,161.54546857)
\curveto(285.1933586,163.04155646)(285.43163961,164.17241471)(285.90820312,164.9380467)
\curveto(286.38476365,165.70366318)(287.08984107,166.08647529)(288.0234375,166.0864842)
}
}
{
\newrgbcolor{curcolor}{0 0 0}
\pscustom[linestyle=none,fillstyle=solid,fillcolor=curcolor]
{
\newpath
\moveto(319.66546631,165.39507795)
\lineto(324.09515381,165.39507795)
\lineto(324.09515381,164.3989842)
\lineto(320.74359131,164.3989842)
\lineto(320.74359131,162.24859357)
\curveto(320.9115576,162.31108791)(321.07952618,162.35600974)(321.24749756,162.3833592)
\curveto(321.41936959,162.41460343)(321.59124442,162.43022842)(321.76312256,162.4302342)
\curveto(322.66936834,162.43022842)(323.38811762,162.16265056)(323.91937256,161.62749982)
\curveto(324.45061656,161.09233913)(324.71624129,160.36773048)(324.71624756,159.4536717)
\curveto(324.71624129,158.53179481)(324.4369447,157.80523304)(323.87835693,157.2739842)
\curveto(323.32366456,156.7427341)(322.5638997,156.47710937)(321.59906006,156.4771092)
\curveto(321.13421363,156.47710937)(320.7084328,156.50835934)(320.32171631,156.5708592)
\curveto(319.93890232,156.63335921)(319.59515267,156.72710912)(319.29046631,156.8521092)
\lineto(319.29046631,158.05328107)
\curveto(319.64984011,157.85796736)(320.01116788,157.71148313)(320.37445068,157.61382795)
\curveto(320.73772965,157.52007708)(321.10882303,157.47320212)(321.48773193,157.47320295)
\curveto(322.140072,157.47320212)(322.64202462,157.64507695)(322.99359131,157.98882795)
\curveto(323.34905516,158.33257626)(323.52678936,158.82085703)(323.52679443,159.4536717)
\curveto(323.52678936,160.07866827)(323.34319579,160.56499591)(322.97601318,160.91265607)
\curveto(322.61272777,161.26030771)(322.10491578,161.43413566)(321.45257568,161.43414045)
\curveto(321.13616675,161.43413566)(320.82757331,161.39702632)(320.52679443,161.32281232)
\curveto(320.22601141,161.25249522)(319.93890232,161.14507345)(319.66546631,161.0005467)
\lineto(319.66546631,165.39507795)
}
}
{
\newrgbcolor{curcolor}{0 0 0}
\pscustom[linestyle=none,fillstyle=solid,fillcolor=curcolor]
{
\newpath
\moveto(223.32841492,144.49712873)
\curveto(223.32841209,144.71196787)(223.40263076,144.89751456)(223.55107117,145.05376936)
\curveto(223.70341171,145.21001425)(223.88505216,145.28813917)(224.09599304,145.28814436)
\curveto(224.31473923,145.28813917)(224.50223904,145.21001425)(224.65849304,145.05376936)
\curveto(224.81473873,144.89751456)(224.89286365,144.71196787)(224.89286804,144.49712873)
\curveto(224.89286365,144.27837455)(224.81473873,144.09282787)(224.65849304,143.94048811)
\curveto(224.50614528,143.78814067)(224.31864547,143.71196887)(224.09599304,143.71197248)
\curveto(223.87723966,143.71196887)(223.6936461,143.78618755)(223.54521179,143.93462873)
\curveto(223.40067764,144.08306225)(223.32841209,144.27056206)(223.32841492,144.49712873)
\moveto(224.10771179,148.07134748)
\curveto(223.55692748,148.07133951)(223.14481852,147.77446481)(222.87138367,147.18072248)
\curveto(222.60185031,146.586966)(222.46708482,145.68267003)(222.46708679,144.46783186)
\curveto(222.46708482,143.2568912)(222.60185031,142.35454835)(222.87138367,141.76080061)
\curveto(223.14481852,141.16704954)(223.55692748,140.87017484)(224.10771179,140.87017561)
\curveto(224.66239513,140.87017484)(225.07450409,141.16704954)(225.34403992,141.76080061)
\curveto(225.6174723,142.35454835)(225.75419091,143.2568912)(225.75419617,144.46783186)
\curveto(225.75419091,145.68267003)(225.6174723,146.586966)(225.34403992,147.18072248)
\curveto(225.07450409,147.77446481)(224.66239513,148.07133951)(224.10771179,148.07134748)
\moveto(224.10771179,149.00884748)
\curveto(225.041301,149.00883857)(225.74637842,148.62602646)(226.22294617,147.86040998)
\curveto(226.70340871,147.09477799)(226.94364285,145.96391974)(226.94364929,144.46783186)
\curveto(226.94364285,142.97564148)(226.70340871,141.84673636)(226.22294617,141.08111311)
\curveto(225.74637842,140.31548789)(225.041301,139.93267578)(224.10771179,139.93267561)
\curveto(223.17411537,139.93267578)(222.46903795,140.31548789)(221.99247742,141.08111311)
\curveto(221.5159139,141.84673636)(221.27763289,142.97564148)(221.27763367,144.46783186)
\curveto(221.27763289,145.96391974)(221.5159139,147.09477799)(221.99247742,147.86040998)
\curveto(222.46903795,148.62602646)(223.17411537,149.00883857)(224.10771179,149.00884748)
}
}
{
\newrgbcolor{curcolor}{0 0 0}
\pscustom[linestyle=none,fillstyle=solid,fillcolor=curcolor]
{
\newpath
\moveto(255.52357483,144.43511701)
\curveto(255.523572,144.64995615)(255.59779067,144.83550284)(255.74623108,144.99175764)
\curveto(255.89857162,145.14800253)(256.08021207,145.22612745)(256.29115295,145.22613264)
\curveto(256.50989914,145.22612745)(256.69739895,145.14800253)(256.85365295,144.99175764)
\curveto(257.00989864,144.83550284)(257.08802356,144.64995615)(257.08802795,144.43511701)
\curveto(257.08802356,144.21636284)(257.00989864,144.03081615)(256.85365295,143.87847639)
\curveto(256.7013052,143.72612895)(256.51380538,143.64995715)(256.29115295,143.64996076)
\curveto(256.07239958,143.64995715)(255.88880601,143.72417583)(255.7403717,143.87261701)
\curveto(255.59583755,144.02105053)(255.523572,144.20855034)(255.52357483,144.43511701)
\moveto(256.3028717,148.00933576)
\curveto(255.7520874,148.00932779)(255.33997843,147.71245309)(255.06654358,147.11871076)
\curveto(254.79701023,146.52495428)(254.66224474,145.62065831)(254.6622467,144.40582014)
\curveto(254.66224474,143.19487948)(254.79701023,142.29253663)(255.06654358,141.69878889)
\curveto(255.33997843,141.10503782)(255.7520874,140.80816312)(256.3028717,140.80816389)
\curveto(256.85755504,140.80816312)(257.269664,141.10503782)(257.53919983,141.69878889)
\curveto(257.81263221,142.29253663)(257.94935082,143.19487948)(257.94935608,144.40582014)
\curveto(257.94935082,145.62065831)(257.81263221,146.52495428)(257.53919983,147.11871076)
\curveto(257.269664,147.71245309)(256.85755504,148.00932779)(256.3028717,148.00933576)
\moveto(256.3028717,148.94683576)
\curveto(257.23646091,148.94682686)(257.94153833,148.56401474)(258.41810608,147.79839826)
\curveto(258.89856862,147.03276627)(259.13880276,145.90190803)(259.1388092,144.40582014)
\curveto(259.13880276,142.91362976)(258.89856862,141.78472464)(258.41810608,141.01910139)
\curveto(257.94153833,140.25347617)(257.23646091,139.87066406)(256.3028717,139.87066389)
\curveto(255.36927528,139.87066406)(254.66419786,140.25347617)(254.18763733,141.01910139)
\curveto(253.71107381,141.78472464)(253.4727928,142.91362976)(253.47279358,144.40582014)
\curveto(253.4727928,145.90190803)(253.71107381,147.03276627)(254.18763733,147.79839826)
\curveto(254.66419786,148.56401474)(255.36927528,148.94682686)(256.3028717,148.94683576)
}
}
{
\newrgbcolor{curcolor}{0 0 0}
\pscustom[linestyle=none,fillstyle=solid,fillcolor=curcolor]
{
\newpath
\moveto(287.24414062,144.55657697)
\curveto(287.24413779,144.77141611)(287.31835647,144.9569628)(287.46679688,145.1132176)
\curveto(287.61913742,145.26946249)(287.80077786,145.34758741)(288.01171875,145.3475926)
\curveto(288.23046493,145.34758741)(288.41796475,145.26946249)(288.57421875,145.1132176)
\curveto(288.73046443,144.9569628)(288.80858936,144.77141611)(288.80859375,144.55657697)
\curveto(288.80858936,144.3378228)(288.73046443,144.15227611)(288.57421875,143.99993635)
\curveto(288.42187099,143.84758891)(288.23437118,143.77141711)(288.01171875,143.77142072)
\curveto(287.79296537,143.77141711)(287.6093718,143.84563579)(287.4609375,143.99407697)
\curveto(287.31640335,144.14251049)(287.24413779,144.3300103)(287.24414062,144.55657697)
\moveto(288.0234375,148.13079572)
\curveto(287.47265319,148.13078775)(287.06054423,147.83391305)(286.78710938,147.24017072)
\curveto(286.51757602,146.64641424)(286.38281053,145.74211827)(286.3828125,144.5272801)
\curveto(286.38281053,143.31633944)(286.51757602,142.4139966)(286.78710938,141.82024885)
\curveto(287.06054423,141.22649778)(287.47265319,140.92962308)(288.0234375,140.92962385)
\curveto(288.57812084,140.92962308)(288.9902298,141.22649778)(289.25976562,141.82024885)
\curveto(289.53319801,142.4139966)(289.66991662,143.31633944)(289.66992188,144.5272801)
\curveto(289.66991662,145.74211827)(289.53319801,146.64641424)(289.25976562,147.24017072)
\curveto(288.9902298,147.83391305)(288.57812084,148.13078775)(288.0234375,148.13079572)
\moveto(288.0234375,149.06829572)
\curveto(288.95702671,149.06828682)(289.66210413,148.6854747)(290.13867188,147.91985822)
\curveto(290.61913442,147.15422623)(290.85936855,146.02336799)(290.859375,144.5272801)
\curveto(290.85936855,143.03508972)(290.61913442,141.9061846)(290.13867188,141.14056135)
\curveto(289.66210413,140.37493613)(288.95702671,139.99212402)(288.0234375,139.99212385)
\curveto(287.08984107,139.99212402)(286.38476365,140.37493613)(285.90820312,141.14056135)
\curveto(285.43163961,141.9061846)(285.1933586,143.03508972)(285.19335938,144.5272801)
\curveto(285.1933586,146.02336799)(285.43163961,147.15422623)(285.90820312,147.91985822)
\curveto(286.38476365,148.6854747)(287.08984107,149.06828682)(288.0234375,149.06829572)
}
}
{
\newrgbcolor{curcolor}{0 0 0}
\pscustom[linestyle=none,fillstyle=solid,fillcolor=curcolor]
{
\newpath
\moveto(320.69671631,140.42767072)
\lineto(324.71624756,140.42767072)
\lineto(324.71624756,139.43157697)
\lineto(319.40179443,139.43157697)
\lineto(319.40179443,140.42767072)
\curveto(320.13226156,141.19720021)(320.7709328,141.87688703)(321.31781006,142.46673322)
\curveto(321.86468171,143.05657335)(322.24163445,143.47258856)(322.44866943,143.7147801)
\curveto(322.83929011,144.19133784)(323.10296172,144.57610308)(323.23968506,144.86907697)
\curveto(323.37639894,145.16594624)(323.44475825,145.46868031)(323.44476318,145.7772801)
\curveto(323.44475825,146.26555451)(323.30022714,146.64836663)(323.01116943,146.9257176)
\curveto(322.72600897,147.20305358)(322.33343124,147.34172531)(321.83343506,147.34173322)
\curveto(321.47796334,147.34172531)(321.10491684,147.27727225)(320.71429443,147.14837385)
\curveto(320.32366762,147.01946001)(319.90960554,146.82414771)(319.47210693,146.56243635)
\lineto(319.47210693,147.75774885)
\curveto(319.87444932,147.94914658)(320.26898018,148.09367769)(320.65570068,148.1913426)
\curveto(321.04632315,148.28898999)(321.43108839,148.33781807)(321.80999756,148.33782697)
\curveto(322.66546215,148.33781807)(323.35296147,148.10930267)(323.87249756,147.6522801)
\curveto(324.39592917,147.19914733)(324.65764766,146.6034448)(324.65765381,145.86517072)
\curveto(324.65764766,145.49016466)(324.56975713,145.11516504)(324.39398193,144.74017072)
\curveto(324.22210122,144.36516579)(323.9408515,143.9511037)(323.55023193,143.49798322)
\curveto(323.33147711,143.24407316)(323.01311806,142.89251101)(322.59515381,142.44329572)
\curveto(322.18108764,141.99407441)(321.54827577,141.32220008)(320.69671631,140.42767072)
}
}
{
\newrgbcolor{curcolor}{0 0 0}
\pscustom[linestyle=none,fillstyle=solid,fillcolor=curcolor]
{
\newpath
\moveto(223.32841492,127.44537092)
\curveto(223.32841209,127.66021006)(223.40263076,127.84575675)(223.55107117,128.00201154)
\curveto(223.70341171,128.15825644)(223.88505216,128.23638136)(224.09599304,128.23638654)
\curveto(224.31473923,128.23638136)(224.50223904,128.15825644)(224.65849304,128.00201154)
\curveto(224.81473873,127.84575675)(224.89286365,127.66021006)(224.89286804,127.44537092)
\curveto(224.89286365,127.22661674)(224.81473873,127.04107005)(224.65849304,126.88873029)
\curveto(224.50614528,126.73638286)(224.31864547,126.66021106)(224.09599304,126.66021467)
\curveto(223.87723966,126.66021106)(223.6936461,126.73442973)(223.54521179,126.88287092)
\curveto(223.40067764,127.03130444)(223.32841209,127.21880425)(223.32841492,127.44537092)
\moveto(224.10771179,131.01958967)
\curveto(223.55692748,131.0195817)(223.14481852,130.722707)(222.87138367,130.12896467)
\curveto(222.60185031,129.53520818)(222.46708482,128.63091221)(222.46708679,127.41607404)
\curveto(222.46708482,126.20513339)(222.60185031,125.30279054)(222.87138367,124.70904279)
\curveto(223.14481852,124.11529173)(223.55692748,123.81841703)(224.10771179,123.81841779)
\curveto(224.66239513,123.81841703)(225.07450409,124.11529173)(225.34403992,124.70904279)
\curveto(225.6174723,125.30279054)(225.75419091,126.20513339)(225.75419617,127.41607404)
\curveto(225.75419091,128.63091221)(225.6174723,129.53520818)(225.34403992,130.12896467)
\curveto(225.07450409,130.722707)(224.66239513,131.0195817)(224.10771179,131.01958967)
\moveto(224.10771179,131.95708967)
\curveto(225.041301,131.95708076)(225.74637842,131.57426864)(226.22294617,130.80865217)
\curveto(226.70340871,130.04302018)(226.94364285,128.91216193)(226.94364929,127.41607404)
\curveto(226.94364285,125.92388367)(226.70340871,124.79497855)(226.22294617,124.02935529)
\curveto(225.74637842,123.26373008)(225.041301,122.88091796)(224.10771179,122.88091779)
\curveto(223.17411537,122.88091796)(222.46903795,123.26373008)(221.99247742,124.02935529)
\curveto(221.5159139,124.79497855)(221.27763289,125.92388367)(221.27763367,127.41607404)
\curveto(221.27763289,128.91216193)(221.5159139,130.04302018)(221.99247742,130.80865217)
\curveto(222.46903795,131.57426864)(223.17411537,131.95708076)(224.10771179,131.95708967)
}
}
{
\newrgbcolor{curcolor}{0 0 0}
\pscustom[linestyle=none,fillstyle=solid,fillcolor=curcolor]
{
\newpath
\moveto(255.52357483,127.41082502)
\curveto(255.523572,127.62566416)(255.59779067,127.81121085)(255.74623108,127.96746564)
\curveto(255.89857162,128.12371054)(256.08021207,128.20183546)(256.29115295,128.20184064)
\curveto(256.50989914,128.20183546)(256.69739895,128.12371054)(256.85365295,127.96746564)
\curveto(257.00989864,127.81121085)(257.08802356,127.62566416)(257.08802795,127.41082502)
\curveto(257.08802356,127.19207084)(257.00989864,127.00652415)(256.85365295,126.85418439)
\curveto(256.7013052,126.70183696)(256.51380538,126.62566516)(256.29115295,126.62566877)
\curveto(256.07239958,126.62566516)(255.88880601,126.69988384)(255.7403717,126.84832502)
\curveto(255.59583755,126.99675854)(255.523572,127.18425835)(255.52357483,127.41082502)
\moveto(256.3028717,130.98504377)
\curveto(255.7520874,130.9850358)(255.33997843,130.6881611)(255.06654358,130.09441877)
\curveto(254.79701023,129.50066229)(254.66224474,128.59636631)(254.6622467,127.38152814)
\curveto(254.66224474,126.17058749)(254.79701023,125.26824464)(255.06654358,124.67449689)
\curveto(255.33997843,124.08074583)(255.7520874,123.78387113)(256.3028717,123.78387189)
\curveto(256.85755504,123.78387113)(257.269664,124.08074583)(257.53919983,124.67449689)
\curveto(257.81263221,125.26824464)(257.94935082,126.17058749)(257.94935608,127.38152814)
\curveto(257.94935082,128.59636631)(257.81263221,129.50066229)(257.53919983,130.09441877)
\curveto(257.269664,130.6881611)(256.85755504,130.9850358)(256.3028717,130.98504377)
\moveto(256.3028717,131.92254377)
\curveto(257.23646091,131.92253486)(257.94153833,131.53972275)(258.41810608,130.77410627)
\curveto(258.89856862,130.00847428)(259.13880276,128.87761603)(259.1388092,127.38152814)
\curveto(259.13880276,125.88933777)(258.89856862,124.76043265)(258.41810608,123.99480939)
\curveto(257.94153833,123.22918418)(257.23646091,122.84637206)(256.3028717,122.84637189)
\curveto(255.36927528,122.84637206)(254.66419786,123.22918418)(254.18763733,123.99480939)
\curveto(253.71107381,124.76043265)(253.4727928,125.88933777)(253.47279358,127.38152814)
\curveto(253.4727928,128.87761603)(253.71107381,130.00847428)(254.18763733,130.77410627)
\curveto(254.66419786,131.53972275)(255.36927528,131.92253486)(256.3028717,131.92254377)
}
}
{
\newrgbcolor{curcolor}{0 0 0}
\pscustom[linestyle=none,fillstyle=solid,fillcolor=curcolor]
{
\newpath
\moveto(287.24414062,127.5383885)
\curveto(287.24413779,127.75322764)(287.31835647,127.93877433)(287.46679688,128.09502912)
\curveto(287.61913742,128.25127401)(287.80077786,128.32939894)(288.01171875,128.32940412)
\curveto(288.23046493,128.32939894)(288.41796475,128.25127401)(288.57421875,128.09502912)
\curveto(288.73046443,127.93877433)(288.80858936,127.75322764)(288.80859375,127.5383885)
\curveto(288.80858936,127.31963432)(288.73046443,127.13408763)(288.57421875,126.98174787)
\curveto(288.42187099,126.82940044)(288.23437118,126.75322864)(288.01171875,126.75323225)
\curveto(287.79296537,126.75322864)(287.6093718,126.82744731)(287.4609375,126.9758885)
\curveto(287.31640335,127.12432202)(287.24413779,127.31182183)(287.24414062,127.5383885)
\moveto(288.0234375,131.11260725)
\curveto(287.47265319,131.11259928)(287.06054423,130.81572457)(286.78710938,130.22198225)
\curveto(286.51757602,129.62822576)(286.38281053,128.72392979)(286.3828125,127.50909162)
\curveto(286.38281053,126.29815097)(286.51757602,125.39580812)(286.78710938,124.80206037)
\curveto(287.06054423,124.20830931)(287.47265319,123.9114346)(288.0234375,123.91143537)
\curveto(288.57812084,123.9114346)(288.9902298,124.20830931)(289.25976562,124.80206037)
\curveto(289.53319801,125.39580812)(289.66991662,126.29815097)(289.66992188,127.50909162)
\curveto(289.66991662,128.72392979)(289.53319801,129.62822576)(289.25976562,130.22198225)
\curveto(288.9902298,130.81572457)(288.57812084,131.11259928)(288.0234375,131.11260725)
\moveto(288.0234375,132.05010725)
\curveto(288.95702671,132.05009834)(289.66210413,131.66728622)(290.13867188,130.90166975)
\curveto(290.61913442,130.13603775)(290.85936855,129.00517951)(290.859375,127.50909162)
\curveto(290.85936855,126.01690125)(290.61913442,124.88799613)(290.13867188,124.12237287)
\curveto(289.66210413,123.35674766)(288.95702671,122.97393554)(288.0234375,122.97393537)
\curveto(287.08984107,122.97393554)(286.38476365,123.35674766)(285.90820312,124.12237287)
\curveto(285.43163961,124.88799613)(285.1933586,126.01690125)(285.19335938,127.50909162)
\curveto(285.1933586,129.00517951)(285.43163961,130.13603775)(285.90820312,130.90166975)
\curveto(286.38476365,131.66728622)(287.08984107,132.05009834)(288.0234375,132.05010725)
}
}
{
\newrgbcolor{curcolor}{0 0 0}
\pscustom[linestyle=none,fillstyle=solid,fillcolor=curcolor]
{
\newpath
\moveto(319.88812256,123.54034162)
\lineto(321.72796631,123.54034162)
\lineto(321.72796631,130.2258885)
\lineto(319.74749756,129.780576)
\lineto(319.74749756,130.858701)
\lineto(321.71624756,131.29229475)
\lineto(322.89984131,131.29229475)
\lineto(322.89984131,123.54034162)
\lineto(324.71624756,123.54034162)
\lineto(324.71624756,122.54424787)
\lineto(319.88812256,122.54424787)
\lineto(319.88812256,123.54034162)
}
}
{
\newrgbcolor{curcolor}{0 0 0}
\pscustom[linestyle=none,fillstyle=solid,fillcolor=curcolor]
{
\newpath
\moveto(223.32841492,110.39361311)
\curveto(223.32841209,110.60845225)(223.40263076,110.79399894)(223.55107117,110.95025373)
\curveto(223.70341171,111.10649862)(223.88505216,111.18462354)(224.09599304,111.18462873)
\curveto(224.31473923,111.18462354)(224.50223904,111.10649862)(224.65849304,110.95025373)
\curveto(224.81473873,110.79399894)(224.89286365,110.60845225)(224.89286804,110.39361311)
\curveto(224.89286365,110.17485893)(224.81473873,109.98931224)(224.65849304,109.83697248)
\curveto(224.50614528,109.68462504)(224.31864547,109.60845325)(224.09599304,109.60845686)
\curveto(223.87723966,109.60845325)(223.6936461,109.68267192)(223.54521179,109.83111311)
\curveto(223.40067764,109.97954663)(223.32841209,110.16704644)(223.32841492,110.39361311)
\moveto(224.10771179,113.96783186)
\curveto(223.55692748,113.96782389)(223.14481852,113.67094918)(222.87138367,113.07720686)
\curveto(222.60185031,112.48345037)(222.46708482,111.5791544)(222.46708679,110.36431623)
\curveto(222.46708482,109.15337558)(222.60185031,108.25103273)(222.87138367,107.65728498)
\curveto(223.14481852,107.06353392)(223.55692748,106.76665921)(224.10771179,106.76665998)
\curveto(224.66239513,106.76665921)(225.07450409,107.06353392)(225.34403992,107.65728498)
\curveto(225.6174723,108.25103273)(225.75419091,109.15337558)(225.75419617,110.36431623)
\curveto(225.75419091,111.5791544)(225.6174723,112.48345037)(225.34403992,113.07720686)
\curveto(225.07450409,113.67094918)(224.66239513,113.96782389)(224.10771179,113.96783186)
\moveto(224.10771179,114.90533186)
\curveto(225.041301,114.90532295)(225.74637842,114.52251083)(226.22294617,113.75689436)
\curveto(226.70340871,112.99126236)(226.94364285,111.86040412)(226.94364929,110.36431623)
\curveto(226.94364285,108.87212586)(226.70340871,107.74322074)(226.22294617,106.97759748)
\curveto(225.74637842,106.21197227)(225.041301,105.82916015)(224.10771179,105.82915998)
\curveto(223.17411537,105.82916015)(222.46903795,106.21197227)(221.99247742,106.97759748)
\curveto(221.5159139,107.74322074)(221.27763289,108.87212586)(221.27763367,110.36431623)
\curveto(221.27763289,111.86040412)(221.5159139,112.99126236)(221.99247742,113.75689436)
\curveto(222.46903795,114.52251083)(223.17411537,114.90532295)(224.10771179,114.90533186)
}
}
{
\newrgbcolor{curcolor}{0 0 0}
\pscustom[linestyle=none,fillstyle=solid,fillcolor=curcolor]
{
\newpath
\moveto(255.52357483,110.3866551)
\curveto(255.523572,110.60149424)(255.59779067,110.78704093)(255.74623108,110.94329572)
\curveto(255.89857162,111.09954062)(256.08021207,111.17766554)(256.29115295,111.17767072)
\curveto(256.50989914,111.17766554)(256.69739895,111.09954062)(256.85365295,110.94329572)
\curveto(257.00989864,110.78704093)(257.08802356,110.60149424)(257.08802795,110.3866551)
\curveto(257.08802356,110.16790092)(257.00989864,109.98235423)(256.85365295,109.83001447)
\curveto(256.7013052,109.67766704)(256.51380538,109.60149524)(256.29115295,109.60149885)
\curveto(256.07239958,109.60149524)(255.88880601,109.67571391)(255.7403717,109.8241551)
\curveto(255.59583755,109.97258862)(255.523572,110.16008843)(255.52357483,110.3866551)
\moveto(256.3028717,113.96087385)
\curveto(255.7520874,113.96086588)(255.33997843,113.66399118)(255.06654358,113.07024885)
\curveto(254.79701023,112.47649236)(254.66224474,111.57219639)(254.6622467,110.35735822)
\curveto(254.66224474,109.14641757)(254.79701023,108.24407472)(255.06654358,107.65032697)
\curveto(255.33997843,107.05657591)(255.7520874,106.75970121)(256.3028717,106.75970197)
\curveto(256.85755504,106.75970121)(257.269664,107.05657591)(257.53919983,107.65032697)
\curveto(257.81263221,108.24407472)(257.94935082,109.14641757)(257.94935608,110.35735822)
\curveto(257.94935082,111.57219639)(257.81263221,112.47649236)(257.53919983,113.07024885)
\curveto(257.269664,113.66399118)(256.85755504,113.96086588)(256.3028717,113.96087385)
\moveto(256.3028717,114.89837385)
\curveto(257.23646091,114.89836494)(257.94153833,114.51555282)(258.41810608,113.74993635)
\curveto(258.89856862,112.98430436)(259.13880276,111.85344611)(259.1388092,110.35735822)
\curveto(259.13880276,108.86516785)(258.89856862,107.73626273)(258.41810608,106.97063947)
\curveto(257.94153833,106.20501426)(257.23646091,105.82220214)(256.3028717,105.82220197)
\curveto(255.36927528,105.82220214)(254.66419786,106.20501426)(254.18763733,106.97063947)
\curveto(253.71107381,107.73626273)(253.4727928,108.86516785)(253.47279358,110.35735822)
\curveto(253.4727928,111.85344611)(253.71107381,112.98430436)(254.18763733,113.74993635)
\curveto(254.66419786,114.51555282)(255.36927528,114.89836494)(256.3028717,114.89837385)
}
}
{
\newrgbcolor{curcolor}{0 0 0}
\pscustom[linestyle=none,fillstyle=solid,fillcolor=curcolor]
{
\newpath
\moveto(287.24414062,110.52020002)
\curveto(287.24413779,110.73503916)(287.31835647,110.92058585)(287.46679688,111.07684064)
\curveto(287.61913742,111.23308554)(287.80077786,111.31121046)(288.01171875,111.31121564)
\curveto(288.23046493,111.31121046)(288.41796475,111.23308554)(288.57421875,111.07684064)
\curveto(288.73046443,110.92058585)(288.80858936,110.73503916)(288.80859375,110.52020002)
\curveto(288.80858936,110.30144584)(288.73046443,110.11589915)(288.57421875,109.96355939)
\curveto(288.42187099,109.81121196)(288.23437118,109.73504016)(288.01171875,109.73504377)
\curveto(287.79296537,109.73504016)(287.6093718,109.80925884)(287.4609375,109.95770002)
\curveto(287.31640335,110.10613354)(287.24413779,110.29363335)(287.24414062,110.52020002)
\moveto(288.0234375,114.09441877)
\curveto(287.47265319,114.0944108)(287.06054423,113.7975361)(286.78710938,113.20379377)
\curveto(286.51757602,112.61003729)(286.38281053,111.70574131)(286.3828125,110.49090314)
\curveto(286.38281053,109.27996249)(286.51757602,108.37761964)(286.78710938,107.78387189)
\curveto(287.06054423,107.19012083)(287.47265319,106.89324613)(288.0234375,106.89324689)
\curveto(288.57812084,106.89324613)(288.9902298,107.19012083)(289.25976562,107.78387189)
\curveto(289.53319801,108.37761964)(289.66991662,109.27996249)(289.66992188,110.49090314)
\curveto(289.66991662,111.70574131)(289.53319801,112.61003729)(289.25976562,113.20379377)
\curveto(288.9902298,113.7975361)(288.57812084,114.0944108)(288.0234375,114.09441877)
\moveto(288.0234375,115.03191877)
\curveto(288.95702671,115.03190986)(289.66210413,114.64909775)(290.13867188,113.88348127)
\curveto(290.61913442,113.11784928)(290.85936855,111.98699103)(290.859375,110.49090314)
\curveto(290.85936855,108.99871277)(290.61913442,107.86980765)(290.13867188,107.10418439)
\curveto(289.66210413,106.33855918)(288.95702671,105.95574706)(288.0234375,105.95574689)
\curveto(287.08984107,105.95574706)(286.38476365,106.33855918)(285.90820312,107.10418439)
\curveto(285.43163961,107.86980765)(285.1933586,108.99871277)(285.19335938,110.49090314)
\curveto(285.1933586,111.98699103)(285.43163961,113.11784928)(285.90820312,113.88348127)
\curveto(286.38476365,114.64909775)(287.08984107,115.03190986)(288.0234375,115.03191877)
}
}
{
\newrgbcolor{curcolor}{0 0 0}
\pscustom[linestyle=none,fillstyle=solid,fillcolor=curcolor]
{
\newpath
\moveto(319.88812256,106.65301252)
\lineto(321.72796631,106.65301252)
\lineto(321.72796631,113.33855939)
\lineto(319.74749756,112.89324689)
\lineto(319.74749756,113.97137189)
\lineto(321.71624756,114.40496564)
\lineto(322.89984131,114.40496564)
\lineto(322.89984131,106.65301252)
\lineto(324.71624756,106.65301252)
\lineto(324.71624756,105.65691877)
\lineto(319.88812256,105.65691877)
\lineto(319.88812256,106.65301252)
}
}
{
\newrgbcolor{curcolor}{0 0 0}
\pscustom[linestyle=none,fillstyle=solid,fillcolor=curcolor]
{
\newpath
\moveto(222.11552429,90.10162092)
\lineto(223.95536804,90.10162092)
\lineto(223.95536804,96.78716779)
\lineto(221.97489929,96.34185529)
\lineto(221.97489929,97.41998029)
\lineto(223.94364929,97.85357404)
\lineto(225.12724304,97.85357404)
\lineto(225.12724304,90.10162092)
\lineto(226.94364929,90.10162092)
\lineto(226.94364929,89.10552717)
\lineto(222.11552429,89.10552717)
\lineto(222.11552429,90.10162092)
}
}
{
\newrgbcolor{curcolor}{0 0 0}
\pscustom[linestyle=none,fillstyle=solid,fillcolor=curcolor]
{
\newpath
\moveto(255.52357483,93.36236311)
\curveto(255.523572,93.57720225)(255.59779067,93.76274894)(255.74623108,93.91900373)
\curveto(255.89857162,94.07524862)(256.08021207,94.15337354)(256.29115295,94.15337873)
\curveto(256.50989914,94.15337354)(256.69739895,94.07524862)(256.85365295,93.91900373)
\curveto(257.00989864,93.76274894)(257.08802356,93.57720225)(257.08802795,93.36236311)
\curveto(257.08802356,93.14360893)(257.00989864,92.95806224)(256.85365295,92.80572248)
\curveto(256.7013052,92.65337504)(256.51380538,92.57720325)(256.29115295,92.57720686)
\curveto(256.07239958,92.57720325)(255.88880601,92.65142192)(255.7403717,92.79986311)
\curveto(255.59583755,92.94829663)(255.523572,93.13579644)(255.52357483,93.36236311)
\moveto(256.3028717,96.93658186)
\curveto(255.7520874,96.93657389)(255.33997843,96.63969918)(255.06654358,96.04595686)
\curveto(254.79701023,95.45220037)(254.66224474,94.5479044)(254.6622467,93.33306623)
\curveto(254.66224474,92.12212558)(254.79701023,91.21978273)(255.06654358,90.62603498)
\curveto(255.33997843,90.03228392)(255.7520874,89.73540921)(256.3028717,89.73540998)
\curveto(256.85755504,89.73540921)(257.269664,90.03228392)(257.53919983,90.62603498)
\curveto(257.81263221,91.21978273)(257.94935082,92.12212558)(257.94935608,93.33306623)
\curveto(257.94935082,94.5479044)(257.81263221,95.45220037)(257.53919983,96.04595686)
\curveto(257.269664,96.63969918)(256.85755504,96.93657389)(256.3028717,96.93658186)
\moveto(256.3028717,97.87408186)
\curveto(257.23646091,97.87407295)(257.94153833,97.49126083)(258.41810608,96.72564436)
\curveto(258.89856862,95.96001236)(259.13880276,94.82915412)(259.1388092,93.33306623)
\curveto(259.13880276,91.84087586)(258.89856862,90.71197074)(258.41810608,89.94634748)
\curveto(257.94153833,89.18072227)(257.23646091,88.79791015)(256.3028717,88.79790998)
\curveto(255.36927528,88.79791015)(254.66419786,89.18072227)(254.18763733,89.94634748)
\curveto(253.71107381,90.71197074)(253.4727928,91.84087586)(253.47279358,93.33306623)
\curveto(253.4727928,94.82915412)(253.71107381,95.96001236)(254.18763733,96.72564436)
\curveto(254.66419786,97.49126083)(255.36927528,97.87407295)(256.3028717,97.87408186)
}
}
{
\newrgbcolor{curcolor}{0 0 0}
\pscustom[linestyle=none,fillstyle=solid,fillcolor=curcolor]
{
\newpath
\moveto(287.24414062,93.50201154)
\curveto(287.24413779,93.71685068)(287.31835647,93.90239737)(287.46679688,94.05865217)
\curveto(287.61913742,94.21489706)(287.80077786,94.29302198)(288.01171875,94.29302717)
\curveto(288.23046493,94.29302198)(288.41796475,94.21489706)(288.57421875,94.05865217)
\curveto(288.73046443,93.90239737)(288.80858936,93.71685068)(288.80859375,93.50201154)
\curveto(288.80858936,93.28325737)(288.73046443,93.09771068)(288.57421875,92.94537092)
\curveto(288.42187099,92.79302348)(288.23437118,92.71685168)(288.01171875,92.71685529)
\curveto(287.79296537,92.71685168)(287.6093718,92.79107036)(287.4609375,92.93951154)
\curveto(287.31640335,93.08794506)(287.24413779,93.27544488)(287.24414062,93.50201154)
\moveto(288.0234375,97.07623029)
\curveto(287.47265319,97.07622232)(287.06054423,96.77934762)(286.78710938,96.18560529)
\curveto(286.51757602,95.59184881)(286.38281053,94.68755284)(286.3828125,93.47271467)
\curveto(286.38281053,92.26177401)(286.51757602,91.35943117)(286.78710938,90.76568342)
\curveto(287.06054423,90.17193235)(287.47265319,89.87505765)(288.0234375,89.87505842)
\curveto(288.57812084,89.87505765)(288.9902298,90.17193235)(289.25976562,90.76568342)
\curveto(289.53319801,91.35943117)(289.66991662,92.26177401)(289.66992188,93.47271467)
\curveto(289.66991662,94.68755284)(289.53319801,95.59184881)(289.25976562,96.18560529)
\curveto(288.9902298,96.77934762)(288.57812084,97.07622232)(288.0234375,97.07623029)
\moveto(288.0234375,98.01373029)
\curveto(288.95702671,98.01372139)(289.66210413,97.63090927)(290.13867188,96.86529279)
\curveto(290.61913442,96.0996608)(290.85936855,94.96880256)(290.859375,93.47271467)
\curveto(290.85936855,91.98052429)(290.61913442,90.85161917)(290.13867188,90.08599592)
\curveto(289.66210413,89.32037071)(288.95702671,88.93755859)(288.0234375,88.93755842)
\curveto(287.08984107,88.93755859)(286.38476365,89.32037071)(285.90820312,90.08599592)
\curveto(285.43163961,90.85161917)(285.1933586,91.98052429)(285.19335938,93.47271467)
\curveto(285.1933586,94.96880256)(285.43163961,96.0996608)(285.90820312,96.86529279)
\curveto(286.38476365,97.63090927)(287.08984107,98.01372139)(288.0234375,98.01373029)
}
}
{
\newrgbcolor{curcolor}{0 0 0}
\pscustom[linestyle=none,fillstyle=solid,fillcolor=curcolor]
{
\newpath
\moveto(319.88812256,89.76568342)
\lineto(321.72796631,89.76568342)
\lineto(321.72796631,96.45123029)
\lineto(319.74749756,96.00591779)
\lineto(319.74749756,97.08404279)
\lineto(321.71624756,97.51763654)
\lineto(322.89984131,97.51763654)
\lineto(322.89984131,89.76568342)
\lineto(324.71624756,89.76568342)
\lineto(324.71624756,88.76958967)
\lineto(319.88812256,88.76958967)
\lineto(319.88812256,89.76568342)
}
}
{
\newrgbcolor{curcolor}{0 0 0}
\pscustom[linestyle=none,fillstyle=solid,fillcolor=curcolor]
{
\newpath
\moveto(223.32841492,76.61822248)
\curveto(223.32841209,76.83306162)(223.40263076,77.01860831)(223.55107117,77.17486311)
\curveto(223.70341171,77.331108)(223.88505216,77.40923292)(224.09599304,77.40923811)
\curveto(224.31473923,77.40923292)(224.50223904,77.331108)(224.65849304,77.17486311)
\curveto(224.81473873,77.01860831)(224.89286365,76.83306162)(224.89286804,76.61822248)
\curveto(224.89286365,76.3994683)(224.81473873,76.21392162)(224.65849304,76.06158186)
\curveto(224.50614528,75.90923442)(224.31864547,75.83306262)(224.09599304,75.83306623)
\curveto(223.87723966,75.83306262)(223.6936461,75.9072813)(223.54521179,76.05572248)
\curveto(223.40067764,76.204156)(223.32841209,76.39165581)(223.32841492,76.61822248)
\moveto(224.10771179,80.19244123)
\curveto(223.55692748,80.19243326)(223.14481852,79.89555856)(222.87138367,79.30181623)
\curveto(222.60185031,78.70805975)(222.46708482,77.80376378)(222.46708679,76.58892561)
\curveto(222.46708482,75.37798495)(222.60185031,74.4756421)(222.87138367,73.88189436)
\curveto(223.14481852,73.28814329)(223.55692748,72.99126859)(224.10771179,72.99126936)
\curveto(224.66239513,72.99126859)(225.07450409,73.28814329)(225.34403992,73.88189436)
\curveto(225.6174723,74.4756421)(225.75419091,75.37798495)(225.75419617,76.58892561)
\curveto(225.75419091,77.80376378)(225.6174723,78.70805975)(225.34403992,79.30181623)
\curveto(225.07450409,79.89555856)(224.66239513,80.19243326)(224.10771179,80.19244123)
\moveto(224.10771179,81.12994123)
\curveto(225.041301,81.12993232)(225.74637842,80.74712021)(226.22294617,79.98150373)
\curveto(226.70340871,79.21587174)(226.94364285,78.08501349)(226.94364929,76.58892561)
\curveto(226.94364285,75.09673523)(226.70340871,73.96783011)(226.22294617,73.20220686)
\curveto(225.74637842,72.43658164)(225.041301,72.05376953)(224.10771179,72.05376936)
\curveto(223.17411537,72.05376953)(222.46903795,72.43658164)(221.99247742,73.20220686)
\curveto(221.5159139,73.96783011)(221.27763289,75.09673523)(221.27763367,76.58892561)
\curveto(221.27763289,78.08501349)(221.5159139,79.21587174)(221.99247742,79.98150373)
\curveto(222.46903795,80.74712021)(223.17411537,81.12993232)(224.10771179,81.12994123)
}
}
{
\newrgbcolor{curcolor}{0 0 0}
\pscustom[linestyle=none,fillstyle=solid,fillcolor=curcolor]
{
\newpath
\moveto(255.11927795,72.93975568)
\lineto(259.1388092,72.93975568)
\lineto(259.1388092,71.94366193)
\lineto(253.82435608,71.94366193)
\lineto(253.82435608,72.93975568)
\curveto(254.55482321,73.70928517)(255.19349444,74.38897199)(255.7403717,74.97881818)
\curveto(256.28724335,75.56865831)(256.6641961,75.98467352)(256.87123108,76.22686506)
\curveto(257.26185175,76.7034228)(257.52552336,77.08818804)(257.6622467,77.38116193)
\curveto(257.79896059,77.6780312)(257.8673199,77.98076527)(257.86732483,78.28936506)
\curveto(257.8673199,78.77763947)(257.72278879,79.16045159)(257.43373108,79.43780256)
\curveto(257.14857061,79.71513854)(256.75599288,79.85381027)(256.2559967,79.85381818)
\curveto(255.90052499,79.85381027)(255.52747849,79.78935721)(255.13685608,79.66045881)
\curveto(254.74622927,79.53154497)(254.33216718,79.33623267)(253.89466858,79.07452131)
\lineto(253.89466858,80.26983381)
\curveto(254.29701097,80.46123154)(254.69154182,80.60576265)(255.07826233,80.70342756)
\curveto(255.46888479,80.80107495)(255.85365003,80.84990303)(256.2325592,80.84991193)
\curveto(257.0880238,80.84990303)(257.77552311,80.62138763)(258.2950592,80.16436506)
\curveto(258.81849082,79.71123229)(259.08020931,79.11552976)(259.08021545,78.37725568)
\curveto(259.08020931,78.00224963)(258.99231877,77.62725)(258.81654358,77.25225568)
\curveto(258.64466287,76.87725075)(258.36341315,76.46318866)(257.97279358,76.01006818)
\curveto(257.75403876,75.75615812)(257.4356797,75.40459597)(257.01771545,74.95538068)
\curveto(256.60364928,74.50615937)(255.97083742,73.83428504)(255.11927795,72.93975568)
}
}
{
\newrgbcolor{curcolor}{0 0 0}
\pscustom[linestyle=none,fillstyle=solid,fillcolor=curcolor]
{
\newpath
\moveto(287.24414062,76.48382307)
\curveto(287.24413779,76.69866221)(287.31835647,76.8842089)(287.46679688,77.04046369)
\curveto(287.61913742,77.19670858)(287.80077786,77.27483351)(288.01171875,77.27483869)
\curveto(288.23046493,77.27483351)(288.41796475,77.19670858)(288.57421875,77.04046369)
\curveto(288.73046443,76.8842089)(288.80858936,76.69866221)(288.80859375,76.48382307)
\curveto(288.80858936,76.26506889)(288.73046443,76.0795222)(288.57421875,75.92718244)
\curveto(288.42187099,75.77483501)(288.23437118,75.69866321)(288.01171875,75.69866682)
\curveto(287.79296537,75.69866321)(287.6093718,75.77288188)(287.4609375,75.92132307)
\curveto(287.31640335,76.06975659)(287.24413779,76.2572564)(287.24414062,76.48382307)
\moveto(288.0234375,80.05804182)
\curveto(287.47265319,80.05803385)(287.06054423,79.76115914)(286.78710938,79.16741682)
\curveto(286.51757602,78.57366033)(286.38281053,77.66936436)(286.3828125,76.45452619)
\curveto(286.38281053,75.24358554)(286.51757602,74.34124269)(286.78710938,73.74749494)
\curveto(287.06054423,73.15374388)(287.47265319,72.85686917)(288.0234375,72.85686994)
\curveto(288.57812084,72.85686917)(288.9902298,73.15374388)(289.25976562,73.74749494)
\curveto(289.53319801,74.34124269)(289.66991662,75.24358554)(289.66992188,76.45452619)
\curveto(289.66991662,77.66936436)(289.53319801,78.57366033)(289.25976562,79.16741682)
\curveto(288.9902298,79.76115914)(288.57812084,80.05803385)(288.0234375,80.05804182)
\moveto(288.0234375,80.99554182)
\curveto(288.95702671,80.99553291)(289.66210413,80.61272079)(290.13867188,79.84710432)
\curveto(290.61913442,79.08147232)(290.85936855,77.95061408)(290.859375,76.45452619)
\curveto(290.85936855,74.96233582)(290.61913442,73.8334307)(290.13867188,73.06780744)
\curveto(289.66210413,72.30218223)(288.95702671,71.91937011)(288.0234375,71.91936994)
\curveto(287.08984107,71.91937011)(286.38476365,72.30218223)(285.90820312,73.06780744)
\curveto(285.43163961,73.8334307)(285.1933586,74.96233582)(285.19335938,76.45452619)
\curveto(285.1933586,77.95061408)(285.43163961,79.08147232)(285.90820312,79.84710432)
\curveto(286.38476365,80.61272079)(287.08984107,80.99553291)(288.0234375,80.99554182)
}
}
{
\newrgbcolor{curcolor}{0 0 0}
\pscustom[linestyle=none,fillstyle=solid,fillcolor=curcolor]
{
\newpath
\moveto(320.69671631,72.72015119)
\lineto(324.71624756,72.72015119)
\lineto(324.71624756,71.72405744)
\lineto(319.40179443,71.72405744)
\lineto(319.40179443,72.72015119)
\curveto(320.13226156,73.48968068)(320.7709328,74.1693675)(321.31781006,74.75921369)
\curveto(321.86468171,75.34905382)(322.24163445,75.76506903)(322.44866943,76.00726057)
\curveto(322.83929011,76.48381831)(323.10296172,76.86858355)(323.23968506,77.16155744)
\curveto(323.37639894,77.45842671)(323.44475825,77.76116078)(323.44476318,78.06976057)
\curveto(323.44475825,78.55803498)(323.30022714,78.9408471)(323.01116943,79.21819807)
\curveto(322.72600897,79.49553404)(322.33343124,79.63420578)(321.83343506,79.63421369)
\curveto(321.47796334,79.63420578)(321.10491684,79.56975272)(320.71429443,79.44085432)
\curveto(320.32366762,79.31194048)(319.90960554,79.11662817)(319.47210693,78.85491682)
\lineto(319.47210693,80.05022932)
\curveto(319.87444932,80.24162705)(320.26898018,80.38615815)(320.65570068,80.48382307)
\curveto(321.04632315,80.58147046)(321.43108839,80.63029854)(321.80999756,80.63030744)
\curveto(322.66546215,80.63029854)(323.35296147,80.40178314)(323.87249756,79.94476057)
\curveto(324.39592917,79.4916278)(324.65764766,78.89592527)(324.65765381,78.15765119)
\curveto(324.65764766,77.78264513)(324.56975713,77.40764551)(324.39398193,77.03265119)
\curveto(324.22210122,76.65764626)(323.9408515,76.24358417)(323.55023193,75.79046369)
\curveto(323.33147711,75.53655363)(323.01311806,75.18499148)(322.59515381,74.73577619)
\curveto(322.18108764,74.28655488)(321.54827577,73.61468055)(320.69671631,72.72015119)
}
}
{
\newrgbcolor{curcolor}{0 0 0}
\pscustom[linestyle=none,fillstyle=solid,fillcolor=curcolor]
{
\newpath
\moveto(223.32841492,59.56646467)
\curveto(223.32841209,59.78130381)(223.40263076,59.9668505)(223.55107117,60.12310529)
\curveto(223.70341171,60.27935019)(223.88505216,60.35747511)(224.09599304,60.35748029)
\curveto(224.31473923,60.35747511)(224.50223904,60.27935019)(224.65849304,60.12310529)
\curveto(224.81473873,59.9668505)(224.89286365,59.78130381)(224.89286804,59.56646467)
\curveto(224.89286365,59.34771049)(224.81473873,59.1621638)(224.65849304,59.00982404)
\curveto(224.50614528,58.85747661)(224.31864547,58.78130481)(224.09599304,58.78130842)
\curveto(223.87723966,58.78130481)(223.6936461,58.85552348)(223.54521179,59.00396467)
\curveto(223.40067764,59.15239819)(223.32841209,59.339898)(223.32841492,59.56646467)
\moveto(224.10771179,63.14068342)
\curveto(223.55692748,63.14067545)(223.14481852,62.84380075)(222.87138367,62.25005842)
\curveto(222.60185031,61.65630193)(222.46708482,60.75200596)(222.46708679,59.53716779)
\curveto(222.46708482,58.32622714)(222.60185031,57.42388429)(222.87138367,56.83013654)
\curveto(223.14481852,56.23638548)(223.55692748,55.93951078)(224.10771179,55.93951154)
\curveto(224.66239513,55.93951078)(225.07450409,56.23638548)(225.34403992,56.83013654)
\curveto(225.6174723,57.42388429)(225.75419091,58.32622714)(225.75419617,59.53716779)
\curveto(225.75419091,60.75200596)(225.6174723,61.65630193)(225.34403992,62.25005842)
\curveto(225.07450409,62.84380075)(224.66239513,63.14067545)(224.10771179,63.14068342)
\moveto(224.10771179,64.07818342)
\curveto(225.041301,64.07817451)(225.74637842,63.69536239)(226.22294617,62.92974592)
\curveto(226.70340871,62.16411393)(226.94364285,61.03325568)(226.94364929,59.53716779)
\curveto(226.94364285,58.04497742)(226.70340871,56.9160723)(226.22294617,56.15044904)
\curveto(225.74637842,55.38482383)(225.041301,55.00201171)(224.10771179,55.00201154)
\curveto(223.17411537,55.00201171)(222.46903795,55.38482383)(221.99247742,56.15044904)
\curveto(221.5159139,56.9160723)(221.27763289,58.04497742)(221.27763367,59.53716779)
\curveto(221.27763289,61.03325568)(221.5159139,62.16411393)(221.99247742,62.92974592)
\curveto(222.46903795,63.69536239)(223.17411537,64.07817451)(224.10771179,64.07818342)
}
}
{
\newrgbcolor{curcolor}{0 0 0}
\pscustom[linestyle=none,fillstyle=solid,fillcolor=curcolor]
{
\newpath
\moveto(255.52357483,59.48382307)
\curveto(255.523572,59.69866221)(255.59779067,59.8842089)(255.74623108,60.04046369)
\curveto(255.89857162,60.19670858)(256.08021207,60.27483351)(256.29115295,60.27483869)
\curveto(256.50989914,60.27483351)(256.69739895,60.19670858)(256.85365295,60.04046369)
\curveto(257.00989864,59.8842089)(257.08802356,59.69866221)(257.08802795,59.48382307)
\curveto(257.08802356,59.26506889)(257.00989864,59.0795222)(256.85365295,58.92718244)
\curveto(256.7013052,58.77483501)(256.51380538,58.69866321)(256.29115295,58.69866682)
\curveto(256.07239958,58.69866321)(255.88880601,58.77288188)(255.7403717,58.92132307)
\curveto(255.59583755,59.06975659)(255.523572,59.2572564)(255.52357483,59.48382307)
\moveto(256.3028717,63.05804182)
\curveto(255.7520874,63.05803385)(255.33997843,62.76115914)(255.06654358,62.16741682)
\curveto(254.79701023,61.57366033)(254.66224474,60.66936436)(254.6622467,59.45452619)
\curveto(254.66224474,58.24358554)(254.79701023,57.34124269)(255.06654358,56.74749494)
\curveto(255.33997843,56.15374388)(255.7520874,55.85686917)(256.3028717,55.85686994)
\curveto(256.85755504,55.85686917)(257.269664,56.15374388)(257.53919983,56.74749494)
\curveto(257.81263221,57.34124269)(257.94935082,58.24358554)(257.94935608,59.45452619)
\curveto(257.94935082,60.66936436)(257.81263221,61.57366033)(257.53919983,62.16741682)
\curveto(257.269664,62.76115914)(256.85755504,63.05803385)(256.3028717,63.05804182)
\moveto(256.3028717,63.99554182)
\curveto(257.23646091,63.99553291)(257.94153833,63.61272079)(258.41810608,62.84710432)
\curveto(258.89856862,62.08147232)(259.13880276,60.95061408)(259.1388092,59.45452619)
\curveto(259.13880276,57.96233582)(258.89856862,56.8334307)(258.41810608,56.06780744)
\curveto(257.94153833,55.30218223)(257.23646091,54.91937011)(256.3028717,54.91936994)
\curveto(255.36927528,54.91937011)(254.66419786,55.30218223)(254.18763733,56.06780744)
\curveto(253.71107381,56.8334307)(253.4727928,57.96233582)(253.47279358,59.45452619)
\curveto(253.4727928,60.95061408)(253.71107381,62.08147232)(254.18763733,62.84710432)
\curveto(254.66419786,63.61272079)(255.36927528,63.99553291)(256.3028717,63.99554182)
}
}
{
\newrgbcolor{curcolor}{0 0 0}
\pscustom[linestyle=none,fillstyle=solid,fillcolor=curcolor]
{
\newpath
\moveto(287.24414062,59.46563459)
\curveto(287.24413779,59.68047373)(287.31835647,59.86602042)(287.46679688,60.02227521)
\curveto(287.61913742,60.17852011)(287.80077786,60.25664503)(288.01171875,60.25665021)
\curveto(288.23046493,60.25664503)(288.41796475,60.17852011)(288.57421875,60.02227521)
\curveto(288.73046443,59.86602042)(288.80858936,59.68047373)(288.80859375,59.46563459)
\curveto(288.80858936,59.24688041)(288.73046443,59.06133372)(288.57421875,58.90899396)
\curveto(288.42187099,58.75664653)(288.23437118,58.68047473)(288.01171875,58.68047834)
\curveto(287.79296537,58.68047473)(287.6093718,58.75469341)(287.4609375,58.90313459)
\curveto(287.31640335,59.05156811)(287.24413779,59.23906792)(287.24414062,59.46563459)
\moveto(288.0234375,63.03985334)
\curveto(287.47265319,63.03984537)(287.06054423,62.74297067)(286.78710938,62.14922834)
\curveto(286.51757602,61.55547186)(286.38281053,60.65117588)(286.3828125,59.43633771)
\curveto(286.38281053,58.22539706)(286.51757602,57.32305421)(286.78710938,56.72930646)
\curveto(287.06054423,56.1355554)(287.47265319,55.8386807)(288.0234375,55.83868146)
\curveto(288.57812084,55.8386807)(288.9902298,56.1355554)(289.25976562,56.72930646)
\curveto(289.53319801,57.32305421)(289.66991662,58.22539706)(289.66992188,59.43633771)
\curveto(289.66991662,60.65117588)(289.53319801,61.55547186)(289.25976562,62.14922834)
\curveto(288.9902298,62.74297067)(288.57812084,63.03984537)(288.0234375,63.03985334)
\moveto(288.0234375,63.97735334)
\curveto(288.95702671,63.97734443)(289.66210413,63.59453232)(290.13867188,62.82891584)
\curveto(290.61913442,62.06328385)(290.85936855,60.9324256)(290.859375,59.43633771)
\curveto(290.85936855,57.94414734)(290.61913442,56.81524222)(290.13867188,56.04961896)
\curveto(289.66210413,55.28399375)(288.95702671,54.90118163)(288.0234375,54.90118146)
\curveto(287.08984107,54.90118163)(286.38476365,55.28399375)(285.90820312,56.04961896)
\curveto(285.43163961,56.81524222)(285.1933586,57.94414734)(285.19335938,59.43633771)
\curveto(285.1933586,60.9324256)(285.43163961,62.06328385)(285.90820312,62.82891584)
\curveto(286.38476365,63.59453232)(287.08984107,63.97734443)(288.0234375,63.97735334)
}
}
{
\newrgbcolor{curcolor}{0 0 0}
\pscustom[linestyle=none,fillstyle=solid,fillcolor=curcolor]
{
\newpath
\moveto(312.62249756,55.67461896)
\lineto(314.46234131,55.67461896)
\lineto(314.46234131,62.36016584)
\lineto(312.48187256,61.91485334)
\lineto(312.48187256,62.99297834)
\lineto(314.45062256,63.42657209)
\lineto(315.63421631,63.42657209)
\lineto(315.63421631,55.67461896)
\lineto(317.45062256,55.67461896)
\lineto(317.45062256,54.67852521)
\lineto(312.62249756,54.67852521)
\lineto(312.62249756,55.67461896)
}
}
{
\newrgbcolor{curcolor}{0 0 0}
\pscustom[linestyle=none,fillstyle=solid,fillcolor=curcolor]
{
\newpath
\moveto(321.86859131,58.83282209)
\curveto(321.34124448,58.83281794)(320.93304176,58.68438058)(320.64398193,58.38750959)
\curveto(320.35882358,58.09453742)(320.2162456,57.67852221)(320.21624756,57.13946271)
\curveto(320.2162456,56.60039829)(320.36077671,56.18047684)(320.64984131,55.87969709)
\curveto(320.94280737,55.58282119)(321.34905697,55.43438383)(321.86859131,55.43438459)
\curveto(322.39983717,55.43438383)(322.80803988,55.58086806)(323.09320068,55.87383771)
\curveto(323.38225806,56.17071122)(323.52678917,56.5925858)(323.52679443,57.13946271)
\curveto(323.52678917,57.67461597)(323.38030494,58.09063118)(323.08734131,58.38750959)
\curveto(322.79827427,58.68438058)(322.39202468,58.83281794)(321.86859131,58.83282209)
\moveto(320.83734131,59.32500959)
\curveto(320.33343298,59.45391106)(319.93890213,59.6941452)(319.65374756,60.04571271)
\curveto(319.37249645,60.3972695)(319.23187159,60.8210972)(319.23187256,61.31719709)
\curveto(319.23187159,62.01250226)(319.46819947,62.56328296)(319.94085693,62.96954084)
\curveto(320.41351103,63.37968839)(321.05608851,63.58476631)(321.86859131,63.58477521)
\curveto(322.68499313,63.58476631)(323.32952374,63.37968839)(323.80218506,62.96954084)
\curveto(324.27483529,62.56328296)(324.51116318,62.01250226)(324.51116943,61.31719709)
\curveto(324.51116318,60.8210972)(324.3685852,60.3972695)(324.08343506,60.04571271)
\curveto(323.80217952,59.6941452)(323.40960178,59.45391106)(322.90570068,59.32500959)
\curveto(323.49163295,59.19609882)(323.93889813,58.93633346)(324.24749756,58.54571271)
\curveto(324.55999126,58.15508424)(324.7162411,57.64922537)(324.71624756,57.02813459)
\curveto(324.7162411,56.23907053)(324.46428823,55.62188365)(323.96038818,55.17657209)
\curveto(323.45647674,54.73125954)(322.75921181,54.50860351)(321.86859131,54.50860334)
\curveto(320.97796359,54.50860351)(320.28069866,54.72930641)(319.77679443,55.17071271)
\curveto(319.27679342,55.61602428)(319.02679367,56.23125804)(319.02679443,57.01641584)
\curveto(319.02679367,57.64141288)(319.18109039,58.14922487)(319.48968506,58.53985334)
\curveto(319.80218352,58.93438033)(320.25140182,59.19609882)(320.83734131,59.32500959)
\moveto(320.40960693,61.20586896)
\curveto(320.40960478,60.73711291)(320.53460466,60.37969139)(320.78460693,60.13360334)
\curveto(321.03460416,59.88750438)(321.39593192,59.76445763)(321.86859131,59.76446271)
\curveto(322.34514972,59.76445763)(322.70843061,59.88750438)(322.95843506,60.13360334)
\curveto(323.20843011,60.37969139)(323.33342998,60.73711291)(323.33343506,61.20586896)
\curveto(323.33342998,61.68242446)(323.20843011,62.04570535)(322.95843506,62.29571271)
\curveto(322.71233686,62.54570485)(322.34905597,62.67070472)(321.86859131,62.67071271)
\curveto(321.39593192,62.67070472)(321.03460416,62.54375172)(320.78460693,62.28985334)
\curveto(320.53460466,62.03984598)(320.40960478,61.67851821)(320.40960693,61.20586896)
}
}
{
\newrgbcolor{curcolor}{0 0 0}
\pscustom[linestyle=none,fillstyle=solid,fillcolor=curcolor]
{
\newpath
\moveto(223.32841492,42.51470686)
\curveto(223.32841209,42.729546)(223.40263076,42.91509269)(223.55107117,43.07134748)
\curveto(223.70341171,43.22759237)(223.88505216,43.30571729)(224.09599304,43.30572248)
\curveto(224.31473923,43.30571729)(224.50223904,43.22759237)(224.65849304,43.07134748)
\curveto(224.81473873,42.91509269)(224.89286365,42.729546)(224.89286804,42.51470686)
\curveto(224.89286365,42.29595268)(224.81473873,42.11040599)(224.65849304,41.95806623)
\curveto(224.50614528,41.80571879)(224.31864547,41.729547)(224.09599304,41.72955061)
\curveto(223.87723966,41.729547)(223.6936461,41.80376567)(223.54521179,41.95220686)
\curveto(223.40067764,42.10064038)(223.32841209,42.28814019)(223.32841492,42.51470686)
\moveto(224.10771179,46.08892561)
\curveto(223.55692748,46.08891764)(223.14481852,45.79204293)(222.87138367,45.19830061)
\curveto(222.60185031,44.60454412)(222.46708482,43.70024815)(222.46708679,42.48540998)
\curveto(222.46708482,41.27446933)(222.60185031,40.37212648)(222.87138367,39.77837873)
\curveto(223.14481852,39.18462767)(223.55692748,38.88775296)(224.10771179,38.88775373)
\curveto(224.66239513,38.88775296)(225.07450409,39.18462767)(225.34403992,39.77837873)
\curveto(225.6174723,40.37212648)(225.75419091,41.27446933)(225.75419617,42.48540998)
\curveto(225.75419091,43.70024815)(225.6174723,44.60454412)(225.34403992,45.19830061)
\curveto(225.07450409,45.79204293)(224.66239513,46.08891764)(224.10771179,46.08892561)
\moveto(224.10771179,47.02642561)
\curveto(225.041301,47.0264167)(225.74637842,46.64360458)(226.22294617,45.87798811)
\curveto(226.70340871,45.11235611)(226.94364285,43.98149787)(226.94364929,42.48540998)
\curveto(226.94364285,40.99321961)(226.70340871,39.86431449)(226.22294617,39.09869123)
\curveto(225.74637842,38.33306602)(225.041301,37.9502539)(224.10771179,37.95025373)
\curveto(223.17411537,37.9502539)(222.46903795,38.33306602)(221.99247742,39.09869123)
\curveto(221.5159139,39.86431449)(221.27763289,40.99321961)(221.27763367,42.48540998)
\curveto(221.27763289,43.98149787)(221.5159139,45.11235611)(221.99247742,45.87798811)
\curveto(222.46903795,46.64360458)(223.17411537,47.0264167)(224.10771179,47.02642561)
}
}
{
\newrgbcolor{curcolor}{0 0 0}
\pscustom[linestyle=none,fillstyle=solid,fillcolor=curcolor]
{
\newpath
\moveto(255.52357483,42.45959211)
\curveto(255.523572,42.67443125)(255.59779067,42.85997794)(255.74623108,43.01623273)
\curveto(255.89857162,43.17247763)(256.08021207,43.25060255)(256.29115295,43.25060773)
\curveto(256.50989914,43.25060255)(256.69739895,43.17247763)(256.85365295,43.01623273)
\curveto(257.00989864,42.85997794)(257.08802356,42.67443125)(257.08802795,42.45959211)
\curveto(257.08802356,42.24083793)(257.00989864,42.05529124)(256.85365295,41.90295148)
\curveto(256.7013052,41.75060405)(256.51380538,41.67443225)(256.29115295,41.67443586)
\curveto(256.07239958,41.67443225)(255.88880601,41.74865093)(255.7403717,41.89709211)
\curveto(255.59583755,42.04552563)(255.523572,42.23302544)(255.52357483,42.45959211)
\moveto(256.3028717,46.03381086)
\curveto(255.7520874,46.03380289)(255.33997843,45.73692819)(255.06654358,45.14318586)
\curveto(254.79701023,44.54942938)(254.66224474,43.6451334)(254.6622467,42.43029523)
\curveto(254.66224474,41.21935458)(254.79701023,40.31701173)(255.06654358,39.72326398)
\curveto(255.33997843,39.12951292)(255.7520874,38.83263822)(256.3028717,38.83263898)
\curveto(256.85755504,38.83263822)(257.269664,39.12951292)(257.53919983,39.72326398)
\curveto(257.81263221,40.31701173)(257.94935082,41.21935458)(257.94935608,42.43029523)
\curveto(257.94935082,43.6451334)(257.81263221,44.54942938)(257.53919983,45.14318586)
\curveto(257.269664,45.73692819)(256.85755504,46.03380289)(256.3028717,46.03381086)
\moveto(256.3028717,46.97131086)
\curveto(257.23646091,46.97130195)(257.94153833,46.58848984)(258.41810608,45.82287336)
\curveto(258.89856862,45.05724137)(259.13880276,43.92638312)(259.1388092,42.43029523)
\curveto(259.13880276,40.93810486)(258.89856862,39.80919974)(258.41810608,39.04357648)
\curveto(257.94153833,38.27795127)(257.23646091,37.89513915)(256.3028717,37.89513898)
\curveto(255.36927528,37.89513915)(254.66419786,38.27795127)(254.18763733,39.04357648)
\curveto(253.71107381,39.80919974)(253.4727928,40.93810486)(253.47279358,42.43029523)
\curveto(253.4727928,43.92638312)(253.71107381,45.05724137)(254.18763733,45.82287336)
\curveto(254.66419786,46.58848984)(255.36927528,46.97130195)(256.3028717,46.97131086)
}
}
{
\newrgbcolor{curcolor}{0 0 0}
\pscustom[linestyle=none,fillstyle=solid,fillcolor=curcolor]
{
\newpath
\moveto(287.24414062,42.44750715)
\curveto(287.24413779,42.66234629)(287.31835647,42.84789298)(287.46679688,43.00414777)
\curveto(287.61913742,43.16039267)(287.80077786,43.23851759)(288.01171875,43.23852277)
\curveto(288.23046493,43.23851759)(288.41796475,43.16039267)(288.57421875,43.00414777)
\curveto(288.73046443,42.84789298)(288.80858936,42.66234629)(288.80859375,42.44750715)
\curveto(288.80858936,42.22875297)(288.73046443,42.04320628)(288.57421875,41.89086652)
\curveto(288.42187099,41.73851909)(288.23437118,41.66234729)(288.01171875,41.6623509)
\curveto(287.79296537,41.66234729)(287.6093718,41.73656596)(287.4609375,41.88500715)
\curveto(287.31640335,42.03344067)(287.24413779,42.22094048)(287.24414062,42.44750715)
\moveto(288.0234375,46.0217259)
\curveto(287.47265319,46.02171793)(287.06054423,45.72484323)(286.78710938,45.1311009)
\curveto(286.51757602,44.53734441)(286.38281053,43.63304844)(286.3828125,42.41821027)
\curveto(286.38281053,41.20726962)(286.51757602,40.30492677)(286.78710938,39.71117902)
\curveto(287.06054423,39.11742796)(287.47265319,38.82055326)(288.0234375,38.82055402)
\curveto(288.57812084,38.82055326)(288.9902298,39.11742796)(289.25976562,39.71117902)
\curveto(289.53319801,40.30492677)(289.66991662,41.20726962)(289.66992188,42.41821027)
\curveto(289.66991662,43.63304844)(289.53319801,44.53734441)(289.25976562,45.1311009)
\curveto(288.9902298,45.72484323)(288.57812084,46.02171793)(288.0234375,46.0217259)
\moveto(288.0234375,46.9592259)
\curveto(288.95702671,46.95921699)(289.66210413,46.57640488)(290.13867188,45.8107884)
\curveto(290.61913442,45.04515641)(290.85936855,43.91429816)(290.859375,42.41821027)
\curveto(290.85936855,40.9260199)(290.61913442,39.79711478)(290.13867188,39.03149152)
\curveto(289.66210413,38.26586631)(288.95702671,37.88305419)(288.0234375,37.88305402)
\curveto(287.08984107,37.88305419)(286.38476365,38.26586631)(285.90820312,39.03149152)
\curveto(285.43163961,39.79711478)(285.1933586,40.9260199)(285.19335938,42.41821027)
\curveto(285.1933586,43.91429816)(285.43163961,45.04515641)(285.90820312,45.8107884)
\curveto(286.38476365,46.57640488)(287.08984107,46.95921699)(288.0234375,46.9592259)
}
}
{
\newrgbcolor{curcolor}{0 0 0}
\pscustom[linestyle=none,fillstyle=solid,fillcolor=curcolor]
{
\newpath
\moveto(319.88812256,38.61749006)
\lineto(321.72796631,38.61749006)
\lineto(321.72796631,45.30303693)
\lineto(319.74749756,44.85772443)
\lineto(319.74749756,45.93584943)
\lineto(321.71624756,46.36944318)
\lineto(322.89984131,46.36944318)
\lineto(322.89984131,38.61749006)
\lineto(324.71624756,38.61749006)
\lineto(324.71624756,37.62139631)
\lineto(319.88812256,37.62139631)
\lineto(319.88812256,38.61749006)
}
}
{
\newrgbcolor{curcolor}{0 0 0}
\pscustom[linestyle=none,fillstyle=solid,fillcolor=curcolor]
{
\newpath
\moveto(223.32841492,25.46288801)
\curveto(223.32841209,25.67772715)(223.40263076,25.86327384)(223.55107117,26.01952863)
\curveto(223.70341171,26.17577353)(223.88505216,26.25389845)(224.09599304,26.25390363)
\curveto(224.31473923,26.25389845)(224.50223904,26.17577353)(224.65849304,26.01952863)
\curveto(224.81473873,25.86327384)(224.89286365,25.67772715)(224.89286804,25.46288801)
\curveto(224.89286365,25.24413383)(224.81473873,25.05858714)(224.65849304,24.90624738)
\curveto(224.50614528,24.75389995)(224.31864547,24.67772815)(224.09599304,24.67773176)
\curveto(223.87723966,24.67772815)(223.6936461,24.75194682)(223.54521179,24.90038801)
\curveto(223.40067764,25.04882153)(223.32841209,25.23632134)(223.32841492,25.46288801)
\moveto(224.10771179,29.03710676)
\curveto(223.55692748,29.03709879)(223.14481852,28.74022409)(222.87138367,28.14648176)
\curveto(222.60185031,27.55272527)(222.46708482,26.6484293)(222.46708679,25.43359113)
\curveto(222.46708482,24.22265048)(222.60185031,23.32030763)(222.87138367,22.72655988)
\curveto(223.14481852,22.13280882)(223.55692748,21.83593412)(224.10771179,21.83593488)
\curveto(224.66239513,21.83593412)(225.07450409,22.13280882)(225.34403992,22.72655988)
\curveto(225.6174723,23.32030763)(225.75419091,24.22265048)(225.75419617,25.43359113)
\curveto(225.75419091,26.6484293)(225.6174723,27.55272527)(225.34403992,28.14648176)
\curveto(225.07450409,28.74022409)(224.66239513,29.03709879)(224.10771179,29.03710676)
\moveto(224.10771179,29.97460676)
\curveto(225.041301,29.97459785)(225.74637842,29.59178573)(226.22294617,28.82616926)
\curveto(226.70340871,28.06053727)(226.94364285,26.92967902)(226.94364929,25.43359113)
\curveto(226.94364285,23.94140076)(226.70340871,22.81249564)(226.22294617,22.04687238)
\curveto(225.74637842,21.28124717)(225.041301,20.89843505)(224.10771179,20.89843488)
\curveto(223.17411537,20.89843505)(222.46903795,21.28124717)(221.99247742,22.04687238)
\curveto(221.5159139,22.81249564)(221.27763289,23.94140076)(221.27763367,25.43359113)
\curveto(221.27763289,26.92967902)(221.5159139,28.06053727)(221.99247742,28.82616926)
\curveto(222.46903795,29.59178573)(223.17411537,29.97459785)(224.10771179,29.97460676)
}
}
{
\newrgbcolor{curcolor}{0 0 0}
\pscustom[linestyle=none,fillstyle=solid,fillcolor=curcolor]
{
\newpath
\moveto(255.52357483,25.43530012)
\curveto(255.523572,25.65013926)(255.59779067,25.83568595)(255.74623108,25.99194074)
\curveto(255.89857162,26.14818563)(256.08021207,26.22631056)(256.29115295,26.22631574)
\curveto(256.50989914,26.22631056)(256.69739895,26.14818563)(256.85365295,25.99194074)
\curveto(257.00989864,25.83568595)(257.08802356,25.65013926)(257.08802795,25.43530012)
\curveto(257.08802356,25.21654594)(257.00989864,25.03099925)(256.85365295,24.87865949)
\curveto(256.7013052,24.72631206)(256.51380538,24.65014026)(256.29115295,24.65014387)
\curveto(256.07239958,24.65014026)(255.88880601,24.72435893)(255.7403717,24.87280012)
\curveto(255.59583755,25.02123364)(255.523572,25.20873345)(255.52357483,25.43530012)
\moveto(256.3028717,29.00951887)
\curveto(255.7520874,29.0095109)(255.33997843,28.7126362)(255.06654358,28.11889387)
\curveto(254.79701023,27.52513738)(254.66224474,26.62084141)(254.6622467,25.40600324)
\curveto(254.66224474,24.19506259)(254.79701023,23.29271974)(255.06654358,22.69897199)
\curveto(255.33997843,22.10522093)(255.7520874,21.80834622)(256.3028717,21.80834699)
\curveto(256.85755504,21.80834622)(257.269664,22.10522093)(257.53919983,22.69897199)
\curveto(257.81263221,23.29271974)(257.94935082,24.19506259)(257.94935608,25.40600324)
\curveto(257.94935082,26.62084141)(257.81263221,27.52513738)(257.53919983,28.11889387)
\curveto(257.269664,28.7126362)(256.85755504,29.0095109)(256.3028717,29.00951887)
\moveto(256.3028717,29.94701887)
\curveto(257.23646091,29.94700996)(257.94153833,29.56419784)(258.41810608,28.79858137)
\curveto(258.89856862,28.03294938)(259.13880276,26.90209113)(259.1388092,25.40600324)
\curveto(259.13880276,23.91381287)(258.89856862,22.78490775)(258.41810608,22.01928449)
\curveto(257.94153833,21.25365928)(257.23646091,20.87084716)(256.3028717,20.87084699)
\curveto(255.36927528,20.87084716)(254.66419786,21.25365928)(254.18763733,22.01928449)
\curveto(253.71107381,22.78490775)(253.4727928,23.91381287)(253.47279358,25.40600324)
\curveto(253.4727928,26.90209113)(253.71107381,28.03294938)(254.18763733,28.79858137)
\curveto(254.66419786,29.56419784)(255.36927528,29.94700996)(256.3028717,29.94701887)
}
}
{
\newrgbcolor{curcolor}{0 0 0}
\pscustom[linestyle=none,fillstyle=solid,fillcolor=curcolor]
{
\newpath
\moveto(287.24414062,25.42931867)
\curveto(287.24413779,25.64415781)(287.31835647,25.8297045)(287.46679688,25.9859593)
\curveto(287.61913742,26.14220419)(287.80077786,26.22032911)(288.01171875,26.2203343)
\curveto(288.23046493,26.22032911)(288.41796475,26.14220419)(288.57421875,25.9859593)
\curveto(288.73046443,25.8297045)(288.80858936,25.64415781)(288.80859375,25.42931867)
\curveto(288.80858936,25.2105645)(288.73046443,25.02501781)(288.57421875,24.87267805)
\curveto(288.42187099,24.72033061)(288.23437118,24.64415881)(288.01171875,24.64416242)
\curveto(287.79296537,24.64415881)(287.6093718,24.71837749)(287.4609375,24.86681867)
\curveto(287.31640335,25.01525219)(287.24413779,25.202752)(287.24414062,25.42931867)
\moveto(288.0234375,29.00353742)
\curveto(287.47265319,29.00352945)(287.06054423,28.70665475)(286.78710938,28.11291242)
\curveto(286.51757602,27.51915594)(286.38281053,26.61485997)(286.3828125,25.4000218)
\curveto(286.38281053,24.18908114)(286.51757602,23.28673829)(286.78710938,22.69299055)
\curveto(287.06054423,22.09923948)(287.47265319,21.80236478)(288.0234375,21.80236555)
\curveto(288.57812084,21.80236478)(288.9902298,22.09923948)(289.25976562,22.69299055)
\curveto(289.53319801,23.28673829)(289.66991662,24.18908114)(289.66992188,25.4000218)
\curveto(289.66991662,26.61485997)(289.53319801,27.51915594)(289.25976562,28.11291242)
\curveto(288.9902298,28.70665475)(288.57812084,29.00352945)(288.0234375,29.00353742)
\moveto(288.0234375,29.94103742)
\curveto(288.95702671,29.94102852)(289.66210413,29.5582164)(290.13867188,28.79259992)
\curveto(290.61913442,28.02696793)(290.85936855,26.89610969)(290.859375,25.4000218)
\curveto(290.85936855,23.90783142)(290.61913442,22.7789263)(290.13867188,22.01330305)
\curveto(289.66210413,21.24767783)(288.95702671,20.86486572)(288.0234375,20.86486555)
\curveto(287.08984107,20.86486572)(286.38476365,21.24767783)(285.90820312,22.01330305)
\curveto(285.43163961,22.7789263)(285.1933586,23.90783142)(285.19335938,25.4000218)
\curveto(285.1933586,26.89610969)(285.43163961,28.02696793)(285.90820312,28.79259992)
\curveto(286.38476365,29.5582164)(287.08984107,29.94102852)(288.0234375,29.94103742)
}
}
{
\newrgbcolor{curcolor}{0 0 0}
\pscustom[linestyle=none,fillstyle=solid,fillcolor=curcolor]
{
\newpath
\moveto(319.88812256,21.73022199)
\lineto(321.72796631,21.73022199)
\lineto(321.72796631,28.41576887)
\lineto(319.74749756,27.97045637)
\lineto(319.74749756,29.04858137)
\lineto(321.71624756,29.48217512)
\lineto(322.89984131,29.48217512)
\lineto(322.89984131,21.73022199)
\lineto(324.71624756,21.73022199)
\lineto(324.71624756,20.73412824)
\lineto(319.88812256,20.73412824)
\lineto(319.88812256,21.73022199)
}
}
{
\newrgbcolor{curcolor}{0 0 0}
\pscustom[linestyle=none,fillstyle=solid,fillcolor=curcolor]
{
\newpath
\moveto(223.32841492,8.41125227)
\curveto(223.32841209,8.62609141)(223.40263076,8.8116381)(223.55107117,8.96789289)
\curveto(223.70341171,9.12413778)(223.88505216,9.20226271)(224.09599304,9.20226789)
\curveto(224.31473923,9.20226271)(224.50223904,9.12413778)(224.65849304,8.96789289)
\curveto(224.81473873,8.8116381)(224.89286365,8.62609141)(224.89286804,8.41125227)
\curveto(224.89286365,8.19249809)(224.81473873,8.0069514)(224.65849304,7.85461164)
\curveto(224.50614528,7.70226421)(224.31864547,7.62609241)(224.09599304,7.62609602)
\curveto(223.87723966,7.62609241)(223.6936461,7.70031108)(223.54521179,7.84875227)
\curveto(223.40067764,7.99718579)(223.32841209,8.1846856)(223.32841492,8.41125227)
\moveto(224.10771179,11.98547102)
\curveto(223.55692748,11.98546305)(223.14481852,11.68858834)(222.87138367,11.09484602)
\curveto(222.60185031,10.50108953)(222.46708482,9.59679356)(222.46708679,8.38195539)
\curveto(222.46708482,7.17101474)(222.60185031,6.26867189)(222.87138367,5.67492414)
\curveto(223.14481852,5.08117308)(223.55692748,4.78429837)(224.10771179,4.78429914)
\curveto(224.66239513,4.78429837)(225.07450409,5.08117308)(225.34403992,5.67492414)
\curveto(225.6174723,6.26867189)(225.75419091,7.17101474)(225.75419617,8.38195539)
\curveto(225.75419091,9.59679356)(225.6174723,10.50108953)(225.34403992,11.09484602)
\curveto(225.07450409,11.68858834)(224.66239513,11.98546305)(224.10771179,11.98547102)
\moveto(224.10771179,12.92297102)
\curveto(225.041301,12.92296211)(225.74637842,12.54014999)(226.22294617,11.77453352)
\curveto(226.70340871,11.00890152)(226.94364285,9.87804328)(226.94364929,8.38195539)
\curveto(226.94364285,6.88976502)(226.70340871,5.7608599)(226.22294617,4.99523664)
\curveto(225.74637842,4.22961143)(225.041301,3.84679931)(224.10771179,3.84679914)
\curveto(223.17411537,3.84679931)(222.46903795,4.22961143)(221.99247742,4.99523664)
\curveto(221.5159139,5.7608599)(221.27763289,6.88976502)(221.27763367,8.38195539)
\curveto(221.27763289,9.87804328)(221.5159139,11.00890152)(221.99247742,11.77453352)
\curveto(222.46903795,12.54014999)(223.17411537,12.92296211)(224.10771179,12.92297102)
}
}
{
\newrgbcolor{curcolor}{0 0 0}
\pscustom[linestyle=none,fillstyle=solid,fillcolor=curcolor]
{
\newpath
\moveto(255.52357483,8.41125227)
\curveto(255.523572,8.62609141)(255.59779067,8.8116381)(255.74623108,8.96789289)
\curveto(255.89857162,9.12413778)(256.08021207,9.20226271)(256.29115295,9.20226789)
\curveto(256.50989914,9.20226271)(256.69739895,9.12413778)(256.85365295,8.96789289)
\curveto(257.00989864,8.8116381)(257.08802356,8.62609141)(257.08802795,8.41125227)
\curveto(257.08802356,8.19249809)(257.00989864,8.0069514)(256.85365295,7.85461164)
\curveto(256.7013052,7.70226421)(256.51380538,7.62609241)(256.29115295,7.62609602)
\curveto(256.07239958,7.62609241)(255.88880601,7.70031108)(255.7403717,7.84875227)
\curveto(255.59583755,7.99718579)(255.523572,8.1846856)(255.52357483,8.41125227)
\moveto(256.3028717,11.98547102)
\curveto(255.7520874,11.98546305)(255.33997843,11.68858834)(255.06654358,11.09484602)
\curveto(254.79701023,10.50108953)(254.66224474,9.59679356)(254.6622467,8.38195539)
\curveto(254.66224474,7.17101474)(254.79701023,6.26867189)(255.06654358,5.67492414)
\curveto(255.33997843,5.08117308)(255.7520874,4.78429837)(256.3028717,4.78429914)
\curveto(256.85755504,4.78429837)(257.269664,5.08117308)(257.53919983,5.67492414)
\curveto(257.81263221,6.26867189)(257.94935082,7.17101474)(257.94935608,8.38195539)
\curveto(257.94935082,9.59679356)(257.81263221,10.50108953)(257.53919983,11.09484602)
\curveto(257.269664,11.68858834)(256.85755504,11.98546305)(256.3028717,11.98547102)
\moveto(256.3028717,12.92297102)
\curveto(257.23646091,12.92296211)(257.94153833,12.54014999)(258.41810608,11.77453352)
\curveto(258.89856862,11.00890152)(259.13880276,9.87804328)(259.1388092,8.38195539)
\curveto(259.13880276,6.88976502)(258.89856862,5.7608599)(258.41810608,4.99523664)
\curveto(257.94153833,4.22961143)(257.23646091,3.84679931)(256.3028717,3.84679914)
\curveto(255.36927528,3.84679931)(254.66419786,4.22961143)(254.18763733,4.99523664)
\curveto(253.71107381,5.7608599)(253.4727928,6.88976502)(253.47279358,8.38195539)
\curveto(253.4727928,9.87804328)(253.71107381,11.00890152)(254.18763733,11.77453352)
\curveto(254.66419786,12.54014999)(255.36927528,12.92296211)(256.3028717,12.92297102)
}
}
{
\newrgbcolor{curcolor}{0 0 0}
\pscustom[linestyle=none,fillstyle=solid,fillcolor=curcolor]
{
\newpath
\moveto(287.24414062,8.41125227)
\curveto(287.24413779,8.62609141)(287.31835647,8.8116381)(287.46679688,8.96789289)
\curveto(287.61913742,9.12413778)(287.80077786,9.20226271)(288.01171875,9.20226789)
\curveto(288.23046493,9.20226271)(288.41796475,9.12413778)(288.57421875,8.96789289)
\curveto(288.73046443,8.8116381)(288.80858936,8.62609141)(288.80859375,8.41125227)
\curveto(288.80858936,8.19249809)(288.73046443,8.0069514)(288.57421875,7.85461164)
\curveto(288.42187099,7.70226421)(288.23437118,7.62609241)(288.01171875,7.62609602)
\curveto(287.79296537,7.62609241)(287.6093718,7.70031108)(287.4609375,7.84875227)
\curveto(287.31640335,7.99718579)(287.24413779,8.1846856)(287.24414062,8.41125227)
\moveto(288.0234375,11.98547102)
\curveto(287.47265319,11.98546305)(287.06054423,11.68858834)(286.78710938,11.09484602)
\curveto(286.51757602,10.50108953)(286.38281053,9.59679356)(286.3828125,8.38195539)
\curveto(286.38281053,7.17101474)(286.51757602,6.26867189)(286.78710938,5.67492414)
\curveto(287.06054423,5.08117308)(287.47265319,4.78429837)(288.0234375,4.78429914)
\curveto(288.57812084,4.78429837)(288.9902298,5.08117308)(289.25976562,5.67492414)
\curveto(289.53319801,6.26867189)(289.66991662,7.17101474)(289.66992188,8.38195539)
\curveto(289.66991662,9.59679356)(289.53319801,10.50108953)(289.25976562,11.09484602)
\curveto(288.9902298,11.68858834)(288.57812084,11.98546305)(288.0234375,11.98547102)
\moveto(288.0234375,12.92297102)
\curveto(288.95702671,12.92296211)(289.66210413,12.54014999)(290.13867188,11.77453352)
\curveto(290.61913442,11.00890152)(290.85936855,9.87804328)(290.859375,8.38195539)
\curveto(290.85936855,6.88976502)(290.61913442,5.7608599)(290.13867188,4.99523664)
\curveto(289.66210413,4.22961143)(288.95702671,3.84679931)(288.0234375,3.84679914)
\curveto(287.08984107,3.84679931)(286.38476365,4.22961143)(285.90820312,4.99523664)
\curveto(285.43163961,5.7608599)(285.1933586,6.88976502)(285.19335938,8.38195539)
\curveto(285.1933586,9.87804328)(285.43163961,11.00890152)(285.90820312,11.77453352)
\curveto(286.38476365,12.54014999)(287.08984107,12.92296211)(288.0234375,12.92297102)
}
}
{
\newrgbcolor{curcolor}{0 0 0}
\pscustom[linestyle=none,fillstyle=solid,fillcolor=curcolor]
{
\newpath
\moveto(322.37249756,11.51672102)
\lineto(319.61273193,6.89367414)
\lineto(322.37249756,6.89367414)
\lineto(322.37249756,11.51672102)
\moveto(322.17913818,12.59484602)
\lineto(323.55023193,12.59484602)
\lineto(323.55023193,6.89367414)
\lineto(324.71624756,6.89367414)
\lineto(324.71624756,5.93273664)
\lineto(323.55023193,5.93273664)
\lineto(323.55023193,3.84679914)
\lineto(322.37249756,3.84679914)
\lineto(322.37249756,5.93273664)
\lineto(318.66351318,5.93273664)
\lineto(318.66351318,7.05187727)
\lineto(322.17913818,12.59484602)
}
}
{
\newrgbcolor{curcolor}{0 0 0}
\pscustom[linewidth=0.96543622,linecolor=curcolor]
{
\newpath
\moveto(0.48271811,510.48647)
\lineto(879.60683,510.48647)
}
}
{
\newrgbcolor{curcolor}{0 0 0}
\pscustom[linewidth=0.96543622,linecolor=curcolor]
{
\newpath
\moveto(0.48271811,493.4869)
\lineto(879.60683,493.4869)
}
}
{
\newrgbcolor{curcolor}{0 0 0}
\pscustom[linewidth=0.96543622,linecolor=curcolor]
{
\newpath
\moveto(0.48271811,476.48733)
\lineto(879.60683,476.48733)
}
}
{
\newrgbcolor{curcolor}{0 0 0}
\pscustom[linewidth=0.96543622,linecolor=curcolor]
{
\newpath
\moveto(0.48271811,459.48775)
\lineto(879.60683,459.48775)
}
}
{
\newrgbcolor{curcolor}{0 0 0}
\pscustom[linewidth=0.96543622,linecolor=curcolor]
{
\newpath
\moveto(0.48271811,442.48818)
\lineto(879.60683,442.48818)
}
}
{
\newrgbcolor{curcolor}{0 0 0}
\pscustom[linewidth=0.96543622,linecolor=curcolor]
{
\newpath
\moveto(0.48271811,425.48861)
\lineto(879.60683,425.48861)
}
}
{
\newrgbcolor{curcolor}{0 0 0}
\pscustom[linewidth=0.96543622,linecolor=curcolor]
{
\newpath
\moveto(0.48271811,408.48904)
\lineto(879.60683,408.48904)
}
}
{
\newrgbcolor{curcolor}{0 0 0}
\pscustom[linewidth=0.96543622,linecolor=curcolor]
{
\newpath
\moveto(0.48271811,391.48946)
\lineto(879.60683,391.48946)
}
}
{
\newrgbcolor{curcolor}{0 0 0}
\pscustom[linewidth=0.96543622,linecolor=curcolor]
{
\newpath
\moveto(0.48271811,374.48989)
\lineto(879.60683,374.48989)
}
}
{
\newrgbcolor{curcolor}{0 0 0}
\pscustom[linewidth=0.96543622,linecolor=curcolor]
{
\newpath
\moveto(0.48271811,357.49032)
\lineto(879.60683,357.49032)
}
}
{
\newrgbcolor{curcolor}{0 0 0}
\pscustom[linewidth=0.96543622,linecolor=curcolor]
{
\newpath
\moveto(0.48271811,340.49074)
\lineto(879.60683,340.49074)
}
}
{
\newrgbcolor{curcolor}{0 0 0}
\pscustom[linewidth=0.96543622,linecolor=curcolor]
{
\newpath
\moveto(0.48271811,323.49117)
\lineto(879.60683,323.49117)
}
}
{
\newrgbcolor{curcolor}{0 0 0}
\pscustom[linewidth=0.96543622,linecolor=curcolor]
{
\newpath
\moveto(0.48271811,306.4916)
\lineto(879.60683,306.4916)
}
}
{
\newrgbcolor{curcolor}{0 0 0}
\pscustom[linewidth=0.96543622,linecolor=curcolor]
{
\newpath
\moveto(0.48271811,289.49203)
\lineto(879.60683,289.49203)
}
}
{
\newrgbcolor{curcolor}{0 0 0}
\pscustom[linewidth=0.96543622,linecolor=curcolor]
{
\newpath
\moveto(0.48271811,272.49245)
\lineto(879.60683,272.49245)
}
}
{
\newrgbcolor{curcolor}{0 0 0}
\pscustom[linewidth=0.96543622,linecolor=curcolor]
{
\newpath
\moveto(0.48271811,255.49288)
\lineto(879.60683,255.49288)
}
}
{
\newrgbcolor{curcolor}{0 0 0}
\pscustom[linewidth=0.96543622,linecolor=curcolor]
{
\newpath
\moveto(0.48271811,238.49331)
\lineto(879.60683,238.49331)
}
}
{
\newrgbcolor{curcolor}{0 0 0}
\pscustom[linewidth=0.96543622,linecolor=curcolor]
{
\newpath
\moveto(0.48271811,221.49373)
\lineto(879.60683,221.49373)
}
}
{
\newrgbcolor{curcolor}{0 0 0}
\pscustom[linewidth=0.96543622,linecolor=curcolor]
{
\newpath
\moveto(0.48271811,204.49416)
\lineto(879.60683,204.49416)
}
}
{
\newrgbcolor{curcolor}{0 0 0}
\pscustom[linewidth=0.96543622,linecolor=curcolor]
{
\newpath
\moveto(0.48271811,187.49459)
\lineto(879.60683,187.49459)
}
}
{
\newrgbcolor{curcolor}{0 0 0}
\pscustom[linewidth=0.96543622,linecolor=curcolor]
{
\newpath
\moveto(0.48271811,170.49502)
\lineto(879.60683,170.49502)
}
}
{
\newrgbcolor{curcolor}{0 0 0}
\pscustom[linewidth=0.96543622,linecolor=curcolor]
{
\newpath
\moveto(0.48271811,153.49544)
\lineto(879.60683,153.49544)
}
}
{
\newrgbcolor{curcolor}{0 0 0}
\pscustom[linewidth=0.96543622,linecolor=curcolor]
{
\newpath
\moveto(0.48271811,136.49587)
\lineto(879.60683,136.49587)
}
}
{
\newrgbcolor{curcolor}{0 0 0}
\pscustom[linewidth=0.96543622,linecolor=curcolor]
{
\newpath
\moveto(0.48271811,119.49629)
\lineto(879.60683,119.49629)
}
}
{
\newrgbcolor{curcolor}{0 0 0}
\pscustom[linewidth=0.96543622,linecolor=curcolor]
{
\newpath
\moveto(0.48271811,102.49672)
\lineto(879.60683,102.49672)
}
}
{
\newrgbcolor{curcolor}{0 0 0}
\pscustom[linewidth=0.96543622,linecolor=curcolor]
{
\newpath
\moveto(0.48271811,85.49715)
\lineto(879.60683,85.49715)
}
}
{
\newrgbcolor{curcolor}{0 0 0}
\pscustom[linewidth=0.96543622,linecolor=curcolor]
{
\newpath
\moveto(0.48271811,68.49757)
\lineto(879.60683,68.49757)
}
}
{
\newrgbcolor{curcolor}{0 0 0}
\pscustom[linewidth=0.96543622,linecolor=curcolor]
{
\newpath
\moveto(0.48271811,51.49799979)
\lineto(879.60683,51.49799979)
}
}
{
\newrgbcolor{curcolor}{0 0 0}
\pscustom[linewidth=0.96543622,linecolor=curcolor]
{
\newpath
\moveto(0.48271811,34.49838)
\lineto(879.60683,34.49838)
}
}
{
\newrgbcolor{curcolor}{0 0 0}
\pscustom[linewidth=0.96543622,linecolor=curcolor]
{
\newpath
\moveto(0.48271811,17.49885237)
\lineto(879.60683,17.49885237)
}
}
\end{pspicture}

\caption{Matriz de adyacencias de los contactos entre usuarios del sistema}
\label{contactos_matriz}
\end{figure}

\subsection{Espacios Virtuales}
Habiéndose terminado las variables referentes a usuarios del sistema, se pasó a
evaluar los espacios virtuales, en el cuadro \ref{espacios_tabla_1} pueden verse
los distintos tipos de espacios virtuales y sus indicadores propios, estos son:

\begin{table}
\centering
\begin{tabular}{l|c c c c c}
$Tipo$ & $Cantidad$ & $Recursos$ & $Audiencia$ &
$Recursos/Cantidad$ & $Audiencia/Recursos$ \\
\hline
$Portada    $ & $ 1$ & $17$ & $1314$ & $17   $ & $77.29$ \\
$Carreras   $ & $ 4$ & $ 1$ & $  18$ & $ 0.25$ & $18   $ \\
$Areas      $ & $ 5$ & $ 2$ & $  15$ & $ 0.4 $ & $ 7.5 $ \\
$Materias   $ & $ 4$ & $13$ & $  85$ & $ 3.25$ & $ 6.54$ \\
$Grupos     $ & $ 8$ & $ 3$ & $   1$ & $ 0.38$ & $ 0.33$ \\
$Equipos    $ & $ 0$ & $ 0$ & $   0$ & $    -$ & $ -   $ \\
$Comunidades$ & $11$ & $30$ & $  96$ & $ 2.73$ & $ 3.2 $ \\
\end{tabular}
\caption{Clasificación de los espacios y su actividad}
\label{espacios_tabla_1}
\end{table}

\begin{description}
\item [Cantidad] Representa la cantidad de espacios que existen de un tipo
establecido.
\item [Recursos] Representa la cantidad de recursos que un espacio virtual
determinado contiene.
\item [Audiencia] Representa la cantidad total de visualizaciones acumuladas de
los recursos en un espacio virtual determinado.
\item [Recursos/Cantidad] Representa el promedio de recursos por espacio virtual
esta variable puede ser utilizada para ver que tipo de espacio virtual posee
mayores grados de generación de contenido.
\item [Audiencia/Recursos] Representa el promedio de visualizaciones por recurso
que tiene un espacio, con esta variable podemos ver que tipo de espacio virtual
tiene mayor grado de visualizaciones.
\end{description}

En la figura \ref{espacios_bars_1}, podemos apreciar el diagrama de barras de
los datos tabulados en el cuadro \ref{espacios_tabla_1}, es notoria la
diferencia entre espacios formales y espacios informales, viéndose como los
recursos contenidos en espacios de tipo \emph{comunidad}, triplican al numero
total de estos en el sistema; también puede observarse como los espacios de área
y carrera carecen de interés por parte de la audiencia, las posibles causas de
esto, podrían ser la formalidad inherente en el mismo espacio, además de cierto
grado de desconocimiento del uso y aprovechamiento de estos espacios formales;
destaca la supremacía del espacio portada, por sobre cualquier otro espacio,
siendo el que capta mas audiencia de entre los espacios. También es notorio el
ausente uso de espacios para equipos de trabajo en los grupos, cosa que puede
ser debida a las escasez de grupos registrados.

\begin{figure}
\centering
%LaTeX with PSTricks extensions
%%Creator: inkscape 0.48.5
%%Please note this file requires PSTricks extensions
\psset{xunit=.5pt,yunit=.5pt,runit=.5pt}
\begin{pspicture}(854,432)
{
\newrgbcolor{curcolor}{0 0 0}
\pscustom[linestyle=none,fillstyle=solid,fillcolor=curcolor]
{
\newpath
\moveto(352.02975342,25.63030273)
\lineto(356.93475342,25.63030273)
\lineto(358.22475342,25.63030273)
\curveto(358.33474339,25.63029204)(358.44474328,25.63029204)(358.55475342,25.63030273)
\curveto(358.66474306,25.64029203)(358.74974298,25.62029205)(358.80975342,25.57030273)
\curveto(358.8297429,25.55029212)(358.84474288,25.52529214)(358.85475342,25.49530273)
\curveto(358.87474285,25.4652922)(358.89474283,25.43529223)(358.91475342,25.40530273)
\curveto(358.91474281,25.33529233)(358.89974283,25.22029245)(358.86975342,25.06030273)
\curveto(358.83974289,24.91029276)(358.80474292,24.79529287)(358.76475342,24.71530273)
\curveto(358.70474302,24.57529309)(358.60474312,24.49529317)(358.46475342,24.47530273)
\curveto(358.33474339,24.4652932)(358.17974355,24.46029321)(357.99975342,24.46030273)
\lineto(356.49975342,24.46030273)
\lineto(353.97975342,24.46030273)
\lineto(353.40975342,24.46030273)
\curveto(353.19974853,24.4702932)(353.03974869,24.44529322)(352.92975342,24.38530273)
\curveto(352.81974891,24.32529334)(352.74474898,24.22029345)(352.70475342,24.07030273)
\curveto(352.67474905,23.92029375)(352.64474908,23.7652939)(352.61475342,23.60530273)
\lineto(352.29975342,22.07530273)
\curveto(352.27974945,21.9652957)(352.24974948,21.83529583)(352.20975342,21.68530273)
\curveto(352.17974955,21.53529613)(352.16474956,21.41529625)(352.16475342,21.32530273)
\curveto(352.17474955,21.20529646)(352.21974951,21.12529654)(352.29975342,21.08530273)
\curveto(352.33974939,21.0652966)(352.40474932,21.04529662)(352.49475342,21.02530273)
\lineto(352.64475342,21.02530273)
\curveto(352.68474904,21.01529665)(352.724749,21.01029666)(352.76475342,21.01030273)
\curveto(352.81474891,21.02029665)(352.86474886,21.02529664)(352.91475342,21.02530273)
\lineto(353.42475342,21.02530273)
\lineto(356.36475342,21.02530273)
\lineto(356.66475342,21.02530273)
\curveto(356.77474495,21.03529663)(356.88474484,21.03529663)(356.99475342,21.02530273)
\curveto(357.11474461,21.02529664)(357.21974451,21.01529665)(357.30975342,20.99530273)
\curveto(357.40974432,20.98529668)(357.47974425,20.9652967)(357.51975342,20.93530273)
\curveto(357.54974418,20.91529675)(357.56474416,20.8702968)(357.56475342,20.80030273)
\curveto(357.57474415,20.73029694)(357.57474415,20.65529701)(357.56475342,20.57530273)
\curveto(357.56474416,20.49529717)(357.54974418,20.41029726)(357.51975342,20.32030273)
\curveto(357.49974423,20.24029743)(357.47474425,20.1702975)(357.44475342,20.11030273)
\curveto(357.40474432,20.02029765)(357.34474438,19.95529771)(357.26475342,19.91530273)
\curveto(357.24474448,19.89529777)(357.21474451,19.88029779)(357.17475342,19.87030273)
\curveto(357.14474458,19.8702978)(357.11474461,19.8652978)(357.08475342,19.85530273)
\lineto(356.99475342,19.85530273)
\curveto(356.93474479,19.84529782)(356.87974485,19.84029783)(356.82975342,19.84030273)
\curveto(356.78974494,19.85029782)(356.74474498,19.85529781)(356.69475342,19.85530273)
\lineto(356.13975342,19.85530273)
\lineto(352.97475342,19.85530273)
\lineto(352.61475342,19.85530273)
\curveto(352.50474922,19.8652978)(352.39474933,19.86029781)(352.28475342,19.84030273)
\curveto(352.18474954,19.83029784)(352.09474963,19.80529786)(352.01475342,19.76530273)
\curveto(351.93474979,19.72529794)(351.86974986,19.65529801)(351.81975342,19.55530273)
\curveto(351.77974995,19.49529817)(351.75474997,19.42529824)(351.74475342,19.34530273)
\lineto(351.71475342,19.10530273)
\lineto(351.53475342,18.26530273)
\lineto(351.24975342,16.84030273)
\curveto(351.2297505,16.70030097)(351.20975052,16.5703011)(351.18975342,16.45030273)
\curveto(351.17975055,16.34030133)(351.20475052,16.26030141)(351.26475342,16.21030273)
\curveto(351.3247504,16.16030151)(351.39975033,16.13030154)(351.48975342,16.12030273)
\lineto(351.78975342,16.12030273)
\lineto(352.74975342,16.12030273)
\lineto(355.52475342,16.12030273)
\lineto(356.37975342,16.12030273)
\lineto(356.61975342,16.12030273)
\curveto(356.69974503,16.13030154)(356.76974496,16.12530154)(356.82975342,16.10530273)
\curveto(356.93974479,16.0653016)(357.00474472,16.01030166)(357.02475342,15.94030273)
\curveto(357.04474468,15.91030176)(357.04974468,15.86030181)(357.03975342,15.79030273)
\curveto(357.03974469,15.72030195)(357.03474469,15.64530202)(357.02475342,15.56530273)
\curveto(357.01474471,15.49530217)(356.99474473,15.42030225)(356.96475342,15.34030273)
\curveto(356.94474478,15.2703024)(356.9247448,15.21530245)(356.90475342,15.17530273)
\curveto(356.85474487,15.09530257)(356.79974493,15.04030263)(356.73975342,15.01030273)
\curveto(356.66974506,14.9703027)(356.58474514,14.95030272)(356.48475342,14.95030273)
\lineto(356.21475342,14.95030273)
\lineto(355.16475342,14.95030273)
\lineto(351.17475342,14.95030273)
\lineto(350.12475342,14.95030273)
\curveto(349.98475174,14.95030272)(349.86475186,14.95530271)(349.76475342,14.96530273)
\curveto(349.67475205,14.98530268)(349.60975212,15.03530263)(349.56975342,15.11530273)
\curveto(349.54975218,15.17530249)(349.54475218,15.25030242)(349.55475342,15.34030273)
\curveto(349.57475215,15.44030223)(349.59475213,15.53530213)(349.61475342,15.62530273)
\lineto(349.82475342,16.67530273)
\lineto(350.63475342,20.69530273)
\lineto(351.30975342,24.05530273)
\lineto(351.48975342,24.98530273)
\curveto(351.50975022,25.07529259)(351.5247502,25.1652925)(351.53475342,25.25530273)
\curveto(351.55475017,25.34529232)(351.58975014,25.41529225)(351.63975342,25.46530273)
\curveto(351.69975003,25.53529213)(351.78474994,25.58529208)(351.89475342,25.61530273)
\curveto(351.9247498,25.62529204)(351.94474978,25.62529204)(351.95475342,25.61530273)
\curveto(351.97474975,25.61529205)(351.99974973,25.62029205)(352.02975342,25.63030273)
}
}
{
\newrgbcolor{curcolor}{0 0 0}
\pscustom[linestyle=none,fillstyle=solid,fillcolor=curcolor]
{
\newpath
\moveto(362.41467529,22.85530273)
\curveto(363.13466964,22.8652948)(363.71966905,22.78029489)(364.16967529,22.60030273)
\curveto(364.62966814,22.43029524)(364.94966782,22.12529554)(365.12967529,21.68530273)
\curveto(365.17966759,21.57529609)(365.20966756,21.46029621)(365.21967529,21.34030273)
\curveto(365.23966753,21.23029644)(365.25466752,21.10529656)(365.26467529,20.96530273)
\curveto(365.2746675,20.89529677)(365.26466751,20.82029685)(365.23467529,20.74030273)
\curveto(365.21466756,20.670297)(365.18966758,20.61529705)(365.15967529,20.57530273)
\curveto(365.13966763,20.55529711)(365.10966766,20.53529713)(365.06967529,20.51530273)
\curveto(365.03966773,20.50529716)(365.01466776,20.49029718)(364.99467529,20.47030273)
\curveto(364.93466784,20.45029722)(364.87966789,20.44529722)(364.82967529,20.45530273)
\curveto(364.78966798,20.4652972)(364.74466803,20.4652972)(364.69467529,20.45530273)
\curveto(364.60466817,20.43529723)(364.49466828,20.43029724)(364.36467529,20.44030273)
\curveto(364.24466853,20.46029721)(364.15966861,20.48529718)(364.10967529,20.51530273)
\curveto(364.03966873,20.5652971)(363.99966877,20.63029704)(363.98967529,20.71030273)
\curveto(363.98966878,20.80029687)(363.9696688,20.88529678)(363.92967529,20.96530273)
\curveto(363.87966889,21.12529654)(363.78466899,21.2702964)(363.64467529,21.40030273)
\curveto(363.55466922,21.48029619)(363.44466933,21.54029613)(363.31467529,21.58030273)
\curveto(363.19466958,21.62029605)(363.06466971,21.66029601)(362.92467529,21.70030273)
\curveto(362.88466989,21.72029595)(362.83466994,21.72529594)(362.77467529,21.71530273)
\curveto(362.72467005,21.71529595)(362.67967009,21.72029595)(362.63967529,21.73030273)
\curveto(362.57967019,21.75029592)(362.50467027,21.76029591)(362.41467529,21.76030273)
\curveto(362.32467045,21.76029591)(362.24967052,21.75029592)(362.18967529,21.73030273)
\lineto(362.09967529,21.73030273)
\curveto(362.03967073,21.72029595)(361.98467079,21.71029596)(361.93467529,21.70030273)
\curveto(361.88467089,21.70029597)(361.83467094,21.69529597)(361.78467529,21.68530273)
\curveto(361.51467126,21.62529604)(361.27967149,21.54029613)(361.07967529,21.43030273)
\curveto(360.88967188,21.32029635)(360.73967203,21.13529653)(360.62967529,20.87530273)
\curveto(360.59967217,20.80529686)(360.58467219,20.73529693)(360.58467529,20.66530273)
\curveto(360.58467219,20.59529707)(360.58967218,20.53529713)(360.59967529,20.48530273)
\curveto(360.62967214,20.33529733)(360.67967209,20.22529744)(360.74967529,20.15530273)
\curveto(360.81967195,20.09529757)(360.91467186,20.02529764)(361.03467529,19.94530273)
\curveto(361.1746716,19.84529782)(361.33967143,19.7702979)(361.52967529,19.72030273)
\curveto(361.71967105,19.68029799)(361.90967086,19.63029804)(362.09967529,19.57030273)
\curveto(362.21967055,19.53029814)(362.33967043,19.50029817)(362.45967529,19.48030273)
\curveto(362.58967018,19.46029821)(362.71467006,19.43029824)(362.83467529,19.39030273)
\curveto(363.03466974,19.33029834)(363.22966954,19.2702984)(363.41967529,19.21030273)
\curveto(363.60966916,19.16029851)(363.79466898,19.09529857)(363.97467529,19.01530273)
\curveto(364.02466875,18.99529867)(364.0696687,18.97529869)(364.10967529,18.95530273)
\curveto(364.15966861,18.93529873)(364.20966856,18.91029876)(364.25967529,18.88030273)
\curveto(364.42966834,18.76029891)(364.5746682,18.62529904)(364.69467529,18.47530273)
\curveto(364.81466796,18.32529934)(364.90466787,18.13529953)(364.96467529,17.90530273)
\lineto(364.96467529,17.62030273)
\curveto(364.96466781,17.55030012)(364.95966781,17.47530019)(364.94967529,17.39530273)
\curveto(364.93966783,17.32530034)(364.92966784,17.24530042)(364.91967529,17.15530273)
\lineto(364.88967529,17.00530273)
\curveto(364.84966792,16.93530073)(364.81966795,16.8653008)(364.79967529,16.79530273)
\curveto(364.78966798,16.72530094)(364.769668,16.65530101)(364.73967529,16.58530273)
\curveto(364.68966808,16.47530119)(364.63466814,16.3703013)(364.57467529,16.27030273)
\curveto(364.51466826,16.1703015)(364.44966832,16.08030159)(364.37967529,16.00030273)
\curveto(364.1696686,15.74030193)(363.92466885,15.53030214)(363.64467529,15.37030273)
\curveto(363.36466941,15.22030245)(363.05966971,15.09030258)(362.72967529,14.98030273)
\curveto(362.62967014,14.95030272)(362.52967024,14.93030274)(362.42967529,14.92030273)
\curveto(362.32967044,14.90030277)(362.23467054,14.87530279)(362.14467529,14.84530273)
\curveto(362.03467074,14.82530284)(361.92967084,14.81530285)(361.82967529,14.81530273)
\curveto(361.72967104,14.81530285)(361.62967114,14.80530286)(361.52967529,14.78530273)
\lineto(361.37967529,14.78530273)
\curveto(361.32967144,14.77530289)(361.25967151,14.7703029)(361.16967529,14.77030273)
\curveto(361.07967169,14.7703029)(361.00967176,14.77530289)(360.95967529,14.78530273)
\lineto(360.79467529,14.78530273)
\curveto(360.73467204,14.80530286)(360.6696721,14.81530285)(360.59967529,14.81530273)
\curveto(360.52967224,14.80530286)(360.4746723,14.81030286)(360.43467529,14.83030273)
\curveto(360.38467239,14.84030283)(360.31967245,14.84530282)(360.23967529,14.84530273)
\curveto(360.15967261,14.8653028)(360.08467269,14.88530278)(360.01467529,14.90530273)
\curveto(359.94467283,14.91530275)(359.8696729,14.93530273)(359.78967529,14.96530273)
\curveto(359.49967327,15.0653026)(359.25467352,15.19030248)(359.05467529,15.34030273)
\curveto(358.85467392,15.49030218)(358.69467408,15.68530198)(358.57467529,15.92530273)
\curveto(358.51467426,16.05530161)(358.46467431,16.19030148)(358.42467529,16.33030273)
\curveto(358.39467438,16.4703012)(358.3746744,16.62530104)(358.36467529,16.79530273)
\curveto(358.35467442,16.85530081)(358.35967441,16.92530074)(358.37967529,17.00530273)
\curveto(358.39967437,17.09530057)(358.42467435,17.1653005)(358.45467529,17.21530273)
\curveto(358.49467428,17.25530041)(358.55467422,17.29530037)(358.63467529,17.33530273)
\curveto(358.68467409,17.35530031)(358.75467402,17.3653003)(358.84467529,17.36530273)
\curveto(358.94467383,17.37530029)(359.03467374,17.37530029)(359.11467529,17.36530273)
\curveto(359.20467357,17.35530031)(359.28967348,17.34030033)(359.36967529,17.32030273)
\curveto(359.45967331,17.31030036)(359.51467326,17.29530037)(359.53467529,17.27530273)
\curveto(359.59467318,17.22530044)(359.62467315,17.15030052)(359.62467529,17.05030273)
\curveto(359.63467314,16.96030071)(359.65467312,16.87530079)(359.68467529,16.79530273)
\curveto(359.73467304,16.57530109)(359.83467294,16.40530126)(359.98467529,16.28530273)
\curveto(360.08467269,16.19530147)(360.20467257,16.12530154)(360.34467529,16.07530273)
\curveto(360.48467229,16.02530164)(360.63467214,15.97530169)(360.79467529,15.92530273)
\lineto(361.10967529,15.88030273)
\lineto(361.19967529,15.88030273)
\curveto(361.25967151,15.86030181)(361.34467143,15.85030182)(361.45467529,15.85030273)
\curveto(361.5746712,15.85030182)(361.67967109,15.86030181)(361.76967529,15.88030273)
\curveto(361.83967093,15.88030179)(361.89467088,15.88530178)(361.93467529,15.89530273)
\curveto(361.99467078,15.90530176)(362.05467072,15.91030176)(362.11467529,15.91030273)
\curveto(362.1746706,15.92030175)(362.22967054,15.93030174)(362.27967529,15.94030273)
\curveto(362.58967018,16.02030165)(362.83966993,16.12530154)(363.02967529,16.25530273)
\curveto(363.22966954,16.38530128)(363.39466938,16.60530106)(363.52467529,16.91530273)
\curveto(363.55466922,16.9653007)(363.5696692,17.02030065)(363.56967529,17.08030273)
\curveto(363.57966919,17.14030053)(363.57966919,17.18530048)(363.56967529,17.21530273)
\curveto(363.55966921,17.40530026)(363.51966925,17.54530012)(363.44967529,17.63530273)
\curveto(363.37966939,17.73529993)(363.28466949,17.82529984)(363.16467529,17.90530273)
\curveto(363.08466969,17.9652997)(362.98966978,18.01529965)(362.87967529,18.05530273)
\lineto(362.57967529,18.17530273)
\curveto(362.54967022,18.18529948)(362.51967025,18.19029948)(362.48967529,18.19030273)
\curveto(362.4696703,18.19029948)(362.44967032,18.20029947)(362.42967529,18.22030273)
\curveto(362.10967066,18.33029934)(361.769671,18.41029926)(361.40967529,18.46030273)
\curveto(361.05967171,18.52029915)(360.73967203,18.61529905)(360.44967529,18.74530273)
\curveto(360.35967241,18.78529888)(360.2696725,18.82029885)(360.17967529,18.85030273)
\curveto(360.09967267,18.88029879)(360.02467275,18.92029875)(359.95467529,18.97030273)
\curveto(359.78467299,19.08029859)(359.63467314,19.20529846)(359.50467529,19.34530273)
\curveto(359.3746734,19.48529818)(359.28467349,19.66029801)(359.23467529,19.87030273)
\curveto(359.21467356,19.94029773)(359.20467357,20.01029766)(359.20467529,20.08030273)
\lineto(359.20467529,20.30530273)
\curveto(359.19467358,20.42529724)(359.20967356,20.56029711)(359.24967529,20.71030273)
\curveto(359.28967348,20.8702968)(359.32967344,21.00529666)(359.36967529,21.11530273)
\curveto(359.39967337,21.1652965)(359.41967335,21.20529646)(359.42967529,21.23530273)
\curveto(359.44967332,21.27529639)(359.4746733,21.31529635)(359.50467529,21.35530273)
\curveto(359.63467314,21.58529608)(359.79467298,21.78529588)(359.98467529,21.95530273)
\curveto(360.1746726,22.12529554)(360.38467239,22.27529539)(360.61467529,22.40530273)
\curveto(360.774672,22.49529517)(360.94967182,22.5652951)(361.13967529,22.61530273)
\curveto(361.33967143,22.67529499)(361.54467123,22.73029494)(361.75467529,22.78030273)
\curveto(361.82467095,22.79029488)(361.88967088,22.80029487)(361.94967529,22.81030273)
\curveto(362.01967075,22.82029485)(362.09467068,22.83029484)(362.17467529,22.84030273)
\curveto(362.21467056,22.85029482)(362.25467052,22.85029482)(362.29467529,22.84030273)
\curveto(362.34467043,22.83029484)(362.38467039,22.83529483)(362.41467529,22.85530273)
}
}
{
\newrgbcolor{curcolor}{0 0 0}
\pscustom[linestyle=none,fillstyle=solid,fillcolor=curcolor]
{
\newpath
\moveto(374.12967529,19.01530273)
\curveto(374.12966614,18.9652987)(374.11966615,18.90029877)(374.09967529,18.82030273)
\curveto(374.08966618,18.74029893)(374.0746662,18.67529899)(374.05467529,18.62530273)
\curveto(374.02466625,18.57529909)(374.00966626,18.52529914)(374.00967529,18.47530273)
\curveto(374.00966626,18.43529923)(374.00466627,18.39529927)(373.99467529,18.35530273)
\curveto(373.9746663,18.28529938)(373.95466632,18.23029944)(373.93467529,18.19030273)
\lineto(373.84467529,17.92030273)
\curveto(373.82466645,17.83029984)(373.79466648,17.74029993)(373.75467529,17.65030273)
\curveto(373.72466655,17.5703001)(373.68966658,17.49030018)(373.64967529,17.41030273)
\curveto(373.61966665,17.34030033)(373.57966669,17.2653004)(373.52967529,17.18530273)
\curveto(373.33966693,16.81530085)(373.10966716,16.48030119)(372.83967529,16.18030273)
\curveto(372.75966751,16.09030158)(372.6746676,16.00030167)(372.58467529,15.91030273)
\curveto(372.49466778,15.83030184)(372.40466787,15.75530191)(372.31467529,15.68530273)
\lineto(372.22467529,15.61030273)
\curveto(372.14466813,15.56030211)(372.0696682,15.51030216)(371.99967529,15.46030273)
\curveto(371.92966834,15.41030226)(371.84966842,15.36030231)(371.75967529,15.31030273)
\curveto(371.62966864,15.23030244)(371.48966878,15.16030251)(371.33967529,15.10030273)
\curveto(371.19966907,15.05030262)(371.05466922,15.00030267)(370.90467529,14.95030273)
\curveto(370.82466945,14.93030274)(370.74466953,14.91530275)(370.66467529,14.90530273)
\curveto(370.58466969,14.89530277)(370.50466977,14.88030279)(370.42467529,14.86030273)
\lineto(370.36467529,14.86030273)
\curveto(370.35466992,14.85030282)(370.33966993,14.84530282)(370.31967529,14.84530273)
\curveto(370.21967005,14.82530284)(370.07967019,14.81530285)(369.89967529,14.81530273)
\curveto(369.72967054,14.80530286)(369.59967067,14.81030286)(369.50967529,14.83030273)
\lineto(369.43467529,14.83030273)
\curveto(369.36467091,14.84030283)(369.29967097,14.85030282)(369.23967529,14.86030273)
\curveto(369.17967109,14.86030281)(369.11967115,14.8703028)(369.05967529,14.89030273)
\curveto(368.89967137,14.94030273)(368.74967152,14.98530268)(368.60967529,15.02530273)
\curveto(368.4696718,15.0653026)(368.34467193,15.12530254)(368.23467529,15.20530273)
\curveto(368.08467219,15.29530237)(367.95967231,15.39030228)(367.85967529,15.49030273)
\curveto(367.82967244,15.52030215)(367.77967249,15.56030211)(367.70967529,15.61030273)
\curveto(367.63967263,15.670302)(367.56467271,15.67530199)(367.48467529,15.62530273)
\curveto(367.44467283,15.59530207)(367.41967285,15.55530211)(367.40967529,15.50530273)
\curveto(367.39967287,15.45530221)(367.3746729,15.40030227)(367.33467529,15.34030273)
\curveto(367.32467295,15.31030236)(367.31967295,15.27530239)(367.31967529,15.23530273)
\curveto(367.31967295,15.20530246)(367.31467296,15.1703025)(367.30467529,15.13030273)
\curveto(367.26467301,15.0703026)(367.23967303,15.00530266)(367.22967529,14.93530273)
\curveto(367.21967305,14.85530281)(367.20967306,14.78530288)(367.19967529,14.72530273)
\lineto(366.83967529,12.92530273)
\curveto(366.80967346,12.78530488)(366.77967349,12.64030503)(366.74967529,12.49030273)
\curveto(366.71967355,12.34030533)(366.6696736,12.22530544)(366.59967529,12.14530273)
\curveto(366.52967374,12.07530559)(366.43967383,12.04030563)(366.32967529,12.04030273)
\curveto(366.21967405,12.03030564)(366.10967416,12.02530564)(365.99967529,12.02530273)
\lineto(365.75967529,12.02530273)
\curveto(365.69967457,12.04530562)(365.64467463,12.0653056)(365.59467529,12.08530273)
\curveto(365.55467472,12.10530556)(365.52467475,12.14030553)(365.50467529,12.19030273)
\curveto(365.4746748,12.26030541)(365.4746748,12.3703053)(365.50467529,12.52030273)
\curveto(365.54467473,12.670305)(365.5746747,12.80030487)(365.59467529,12.91030273)
\lineto(367.39467529,21.91030273)
\curveto(367.41467286,22.03029564)(367.43967283,22.15029552)(367.46967529,22.27030273)
\curveto(367.49967277,22.40029527)(367.54967272,22.50529516)(367.61967529,22.58530273)
\curveto(367.65967261,22.62529504)(367.73467254,22.65529501)(367.84467529,22.67530273)
\curveto(367.95467232,22.70529496)(368.0696722,22.71529495)(368.18967529,22.70530273)
\curveto(368.30967196,22.70529496)(368.41967185,22.69029498)(368.51967529,22.66030273)
\curveto(368.61967165,22.64029503)(368.67967159,22.61029506)(368.69967529,22.57030273)
\curveto(368.72967154,22.52029515)(368.73967153,22.46029521)(368.72967529,22.39030273)
\curveto(368.71967155,22.32029535)(368.72467155,22.25029542)(368.74467529,22.18030273)
\curveto(368.75467152,22.15029552)(368.76467151,22.12529554)(368.77467529,22.10530273)
\lineto(368.81967529,22.06030273)
\curveto(368.92967134,22.05029562)(369.02967124,22.08529558)(369.11967529,22.16530273)
\curveto(369.20967106,22.24529542)(369.29467098,22.31029536)(369.37467529,22.36030273)
\curveto(369.6746706,22.54029513)(370.01467026,22.68029499)(370.39467529,22.78030273)
\curveto(370.48466979,22.80029487)(370.5746697,22.81529485)(370.66467529,22.82530273)
\curveto(370.76466951,22.83529483)(370.8696694,22.85029482)(370.97967529,22.87030273)
\curveto(371.01966925,22.88029479)(371.0696692,22.88029479)(371.12967529,22.87030273)
\curveto(371.18966908,22.86029481)(371.22966904,22.8652948)(371.24967529,22.88530273)
\curveto(371.67966859,22.89529477)(372.04966822,22.85029482)(372.35967529,22.75030273)
\curveto(372.6696676,22.66029501)(372.93966733,22.53029514)(373.16967529,22.36030273)
\curveto(373.20966706,22.32029535)(373.24966702,22.28029539)(373.28967529,22.24030273)
\curveto(373.33966693,22.21029546)(373.38466689,22.17529549)(373.42467529,22.13530273)
\curveto(373.44466683,22.11529555)(373.45966681,22.09529557)(373.46967529,22.07530273)
\curveto(373.47966679,22.0652956)(373.49466678,22.05029562)(373.51467529,22.03030273)
\curveto(373.55466672,21.98029569)(373.59466668,21.92529574)(373.63467529,21.86530273)
\curveto(373.68466659,21.80529586)(373.72966654,21.74529592)(373.76967529,21.68530273)
\curveto(373.85966641,21.51529615)(373.94966632,21.33029634)(374.03967529,21.13030273)
\curveto(374.08966618,21.00029667)(374.12466615,20.85529681)(374.14467529,20.69530273)
\curveto(374.1746661,20.53529713)(374.19466608,20.37529729)(374.20467529,20.21530273)
\curveto(374.21466606,20.13529753)(374.21466606,20.05029762)(374.20467529,19.96030273)
\curveto(374.20466607,19.8702978)(374.20966606,19.78529788)(374.21967529,19.70530273)
\lineto(374.18967529,19.58530273)
\lineto(374.18967529,19.49530273)
\curveto(374.19966607,19.44529822)(374.19466608,19.39029828)(374.17467529,19.33030273)
\curveto(374.15466612,19.2702984)(374.14966612,19.21529845)(374.15967529,19.16530273)
\lineto(374.12967529,19.01530273)
\moveto(372.71967529,18.61030273)
\curveto(372.74966752,18.66029901)(372.76466751,18.72029895)(372.76467529,18.79030273)
\curveto(372.7746675,18.8702988)(372.78466749,18.94029873)(372.79467529,19.00030273)
\curveto(372.83466744,19.1702985)(372.85966741,19.33029834)(372.86967529,19.48030273)
\curveto(372.88966738,19.63029804)(372.88466739,19.77529789)(372.85467529,19.91530273)
\curveto(372.85466742,19.97529769)(372.84966742,20.03529763)(372.83967529,20.09530273)
\curveto(372.83966743,20.1652975)(372.82966744,20.23029744)(372.80967529,20.29030273)
\curveto(372.75966751,20.56029711)(372.63966763,20.82029685)(372.44967529,21.07030273)
\curveto(372.269668,21.32029635)(372.08466819,21.49029618)(371.89467529,21.58030273)
\curveto(371.81466846,21.62029605)(371.73466854,21.65029602)(371.65467529,21.67030273)
\curveto(371.5746687,21.69029598)(371.49466878,21.71529595)(371.41467529,21.74530273)
\curveto(371.32466895,21.7652959)(371.21966905,21.77529589)(371.09967529,21.77530273)
\lineto(370.76967529,21.77530273)
\curveto(370.74966952,21.75529591)(370.70966956,21.74529592)(370.64967529,21.74530273)
\curveto(370.59966967,21.75529591)(370.55466972,21.75529591)(370.51467529,21.74530273)
\curveto(370.41466986,21.72529594)(370.31966995,21.70529596)(370.22967529,21.68530273)
\curveto(370.14967012,21.665296)(370.06467021,21.63529603)(369.97467529,21.59530273)
\curveto(369.62467065,21.45529621)(369.31967095,21.25029642)(369.05967529,20.98030273)
\curveto(368.79967147,20.72029695)(368.57967169,20.41529725)(368.39967529,20.06530273)
\curveto(368.33967193,19.95529771)(368.28967198,19.84529782)(368.24967529,19.73530273)
\curveto(368.21967205,19.62529804)(368.18467209,19.51529815)(368.14467529,19.40530273)
\curveto(368.12467215,19.3652983)(368.10967216,19.32529834)(368.09967529,19.28530273)
\curveto(368.08967218,19.25529841)(368.07967219,19.22029845)(368.06967529,19.18030273)
\lineto(368.03967529,19.06030273)
\curveto(368.01967225,19.01029866)(367.99967227,18.93529873)(367.97967529,18.83530273)
\curveto(367.95967231,18.74529892)(367.94967232,18.67529899)(367.94967529,18.62530273)
\lineto(367.93467529,18.50530273)
\curveto(367.93467234,18.4652992)(367.92967234,18.42529924)(367.91967529,18.38530273)
\curveto(367.90967236,18.34529932)(367.90967236,18.31029936)(367.91967529,18.28030273)
\curveto(367.91967235,18.25029942)(367.91467236,18.22029945)(367.90467529,18.19030273)
\lineto(367.90467529,18.08530273)
\lineto(367.90467529,17.84530273)
\curveto(367.90467237,17.7652999)(367.90967236,17.68529998)(367.91967529,17.60530273)
\curveto(367.95967231,17.2653004)(368.04967222,16.9653007)(368.18967529,16.70530273)
\curveto(368.33967193,16.45530121)(368.55967171,16.26030141)(368.84967529,16.12030273)
\curveto(369.01967125,16.04030163)(369.19967107,15.98030169)(369.38967529,15.94030273)
\curveto(369.42967084,15.92030175)(369.4696708,15.91030176)(369.50967529,15.91030273)
\curveto(369.54967072,15.92030175)(369.58967068,15.92030175)(369.62967529,15.91030273)
\lineto(369.74967529,15.91030273)
\curveto(369.81967045,15.89030178)(369.88967038,15.89030178)(369.95967529,15.91030273)
\lineto(370.07967529,15.91030273)
\curveto(370.18967008,15.93030174)(370.29466998,15.94530172)(370.39467529,15.95530273)
\curveto(370.50466977,15.9653017)(370.61466966,15.99030168)(370.72467529,16.03030273)
\curveto(371.05466922,16.16030151)(371.33966893,16.33030134)(371.57967529,16.54030273)
\curveto(371.81966845,16.76030091)(372.03466824,17.02530064)(372.22467529,17.33530273)
\curveto(372.30466797,17.47530019)(372.3696679,17.61530005)(372.41967529,17.75530273)
\curveto(372.47966779,17.90529976)(372.54466773,18.06029961)(372.61467529,18.22030273)
\curveto(372.63466764,18.2702994)(372.64466763,18.31529935)(372.64467529,18.35530273)
\curveto(372.65466762,18.39529927)(372.6696676,18.44029923)(372.68967529,18.49030273)
\lineto(372.71967529,18.61030273)
}
}
{
\newrgbcolor{curcolor}{0 0 0}
\pscustom[linestyle=none,fillstyle=solid,fillcolor=curcolor]
{
\newpath
\moveto(381.80592529,15.50530273)
\curveto(381.79591738,15.34530232)(381.75091743,15.21030246)(381.67092529,15.10030273)
\curveto(381.59091759,15.00030267)(381.49591768,14.92530274)(381.38592529,14.87530273)
\curveto(381.33591784,14.85530281)(381.2809179,14.84530282)(381.22092529,14.84530273)
\curveto(381.17091801,14.84530282)(381.11091807,14.83530283)(381.04092529,14.81530273)
\curveto(380.81091837,14.7653029)(380.59591858,14.78030289)(380.39592529,14.86030273)
\curveto(380.19591898,14.93030274)(380.07091911,15.02030265)(380.02092529,15.13030273)
\curveto(379.9809192,15.20030247)(379.95091923,15.28030239)(379.93092529,15.37030273)
\curveto(379.91091927,15.4703022)(379.8759193,15.55030212)(379.82592529,15.61030273)
\lineto(379.76592529,15.67030273)
\curveto(379.74591943,15.69030198)(379.71591946,15.69530197)(379.67592529,15.68530273)
\curveto(379.55591962,15.65530201)(379.44091974,15.60030207)(379.33092529,15.52030273)
\curveto(379.22091996,15.44030223)(379.11592006,15.3703023)(379.01592529,15.31030273)
\curveto(378.86592031,15.23030244)(378.71092047,15.15530251)(378.55092529,15.08530273)
\curveto(378.39092079,15.02530264)(378.22092096,14.9703027)(378.04092529,14.92030273)
\curveto(377.93092125,14.89030278)(377.81592136,14.8703028)(377.69592529,14.86030273)
\curveto(377.58592159,14.85030282)(377.47092171,14.83530283)(377.35092529,14.81530273)
\curveto(377.30092188,14.80530286)(377.25592192,14.80030287)(377.21592529,14.80030273)
\lineto(377.11092529,14.80030273)
\curveto(377.00092218,14.78030289)(376.89592228,14.78030289)(376.79592529,14.80030273)
\lineto(376.66092529,14.80030273)
\curveto(376.61092257,14.81030286)(376.56092262,14.81530285)(376.51092529,14.81530273)
\curveto(376.46092272,14.81530285)(376.42092276,14.82530284)(376.39092529,14.84530273)
\curveto(376.35092283,14.85530281)(376.31592286,14.86030281)(376.28592529,14.86030273)
\curveto(376.26592291,14.85030282)(376.24092294,14.85030282)(376.21092529,14.86030273)
\lineto(375.97092529,14.92030273)
\curveto(375.90092328,14.93030274)(375.83592334,14.95030272)(375.77592529,14.98030273)
\curveto(375.49592368,15.11030256)(375.2809239,15.25530241)(375.13092529,15.41530273)
\curveto(374.9809242,15.58530208)(374.8759243,15.82030185)(374.81592529,16.12030273)
\curveto(374.76592441,16.34030133)(374.77092441,16.60530106)(374.83092529,16.91530273)
\lineto(374.90592529,17.23030273)
\curveto(374.92592425,17.28030039)(374.94092424,17.33030034)(374.95092529,17.38030273)
\lineto(375.01092529,17.56030273)
\lineto(375.19092529,17.89030273)
\curveto(375.26092392,18.00029967)(375.33092385,18.10029957)(375.40092529,18.19030273)
\curveto(375.64092354,18.48029919)(375.93092325,18.69529897)(376.27092529,18.83530273)
\curveto(376.61092257,18.97529869)(376.9759222,19.10029857)(377.36592529,19.21030273)
\curveto(377.51592166,19.25029842)(377.66592151,19.28029839)(377.81592529,19.30030273)
\curveto(377.9759212,19.32029835)(378.13092105,19.34529832)(378.28092529,19.37530273)
\curveto(378.36092082,19.39529827)(378.43092075,19.40529826)(378.49092529,19.40530273)
\curveto(378.56092062,19.40529826)(378.63592054,19.41529825)(378.71592529,19.43530273)
\curveto(378.78592039,19.45529821)(378.85592032,19.4652982)(378.92592529,19.46530273)
\curveto(379.00592017,19.47529819)(379.08592009,19.49029818)(379.16592529,19.51030273)
\curveto(379.42591975,19.5702981)(379.67091951,19.62029805)(379.90092529,19.66030273)
\curveto(380.13091905,19.71029796)(380.33091885,19.82529784)(380.50092529,20.00530273)
\curveto(380.57091861,20.08529758)(380.63591854,20.18529748)(380.69592529,20.30530273)
\curveto(380.76591841,20.43529723)(380.79591838,20.57529709)(380.78592529,20.72530273)
\curveto(380.7759184,20.9652967)(380.72591845,21.15529651)(380.63592529,21.29530273)
\curveto(380.55591862,21.43529623)(380.41591876,21.54529612)(380.21592529,21.62530273)
\curveto(380.10591907,21.67529599)(379.97091921,21.71029596)(379.81092529,21.73030273)
\curveto(379.65091953,21.75029592)(379.4809197,21.76029591)(379.30092529,21.76030273)
\curveto(379.12092006,21.76029591)(378.94092024,21.75029592)(378.76092529,21.73030273)
\curveto(378.59092059,21.71029596)(378.44092074,21.68029599)(378.31092529,21.64030273)
\curveto(378.13092105,21.58029609)(377.95092123,21.49529617)(377.77092529,21.38530273)
\curveto(377.6809215,21.32529634)(377.59092159,21.24529642)(377.50092529,21.14530273)
\curveto(377.42092176,21.05529661)(377.34592183,20.95529671)(377.27592529,20.84530273)
\curveto(377.22592195,20.7652969)(377.180922,20.68029699)(377.14092529,20.59030273)
\curveto(377.10092208,20.50029717)(377.04092214,20.43029724)(376.96092529,20.38030273)
\curveto(376.91092227,20.35029732)(376.83592234,20.32529734)(376.73592529,20.30530273)
\curveto(376.63592254,20.29529737)(376.53592264,20.29029738)(376.43592529,20.29030273)
\curveto(376.33592284,20.29029738)(376.24092294,20.29529737)(376.15092529,20.30530273)
\curveto(376.06092312,20.32529734)(376.00092318,20.35029732)(375.97092529,20.38030273)
\curveto(375.93092325,20.41029726)(375.90592327,20.46029721)(375.89592529,20.53030273)
\curveto(375.89592328,20.60029707)(375.91592326,20.67529699)(375.95592529,20.75530273)
\curveto(376.00592317,20.88529678)(376.06092312,21.00529666)(376.12092529,21.11530273)
\curveto(376.180923,21.23529643)(376.24592293,21.35029632)(376.31592529,21.46030273)
\curveto(376.5759226,21.81029586)(376.87092231,22.08029559)(377.20092529,22.27030273)
\curveto(377.53092165,22.4702952)(377.92092126,22.63029504)(378.37092529,22.75030273)
\curveto(378.4809207,22.7702949)(378.58592059,22.78529488)(378.68592529,22.79530273)
\curveto(378.79592038,22.80529486)(378.90592027,22.82029485)(379.01592529,22.84030273)
\curveto(379.06592011,22.85029482)(379.13092005,22.85029482)(379.21092529,22.84030273)
\curveto(379.30091988,22.84029483)(379.36091982,22.85029482)(379.39092529,22.87030273)
\curveto(380.09091909,22.88029479)(380.6809185,22.80029487)(381.16092529,22.63030273)
\curveto(381.65091753,22.46029521)(381.95591722,22.13529553)(382.07592529,21.65530273)
\curveto(382.12591705,21.45529621)(382.13091705,21.22029645)(382.09092529,20.95030273)
\curveto(382.05091713,20.69029698)(382.00091718,20.41529725)(381.94092529,20.12530273)
\lineto(381.28092529,16.81030273)
\curveto(381.25091793,16.670301)(381.22591795,16.53530113)(381.20592529,16.40530273)
\curveto(381.19591798,16.27530139)(381.20591797,16.1703015)(381.23592529,16.09030273)
\curveto(381.2759179,16.02030165)(381.33091785,15.9703017)(381.40092529,15.94030273)
\curveto(381.49091769,15.90030177)(381.57091761,15.8703018)(381.64092529,15.85030273)
\curveto(381.72091746,15.84030183)(381.77091741,15.79530187)(381.79092529,15.71530273)
\curveto(381.81091737,15.68530198)(381.81591736,15.65530201)(381.80592529,15.62530273)
\lineto(381.80592529,15.50530273)
\moveto(379.99092529,17.17030273)
\curveto(380.0809191,17.31030036)(380.14591903,17.4703002)(380.18592529,17.65030273)
\curveto(380.22591895,17.84029983)(380.26591891,18.03529963)(380.30592529,18.23530273)
\curveto(380.32591885,18.34529932)(380.34091884,18.44529922)(380.35092529,18.53530273)
\curveto(380.36091882,18.62529904)(380.33591884,18.69529897)(380.27592529,18.74530273)
\curveto(380.24591893,18.7652989)(380.175919,18.77529889)(380.06592529,18.77530273)
\curveto(380.04591913,18.75529891)(380.01091917,18.74529892)(379.96092529,18.74530273)
\curveto(379.91091927,18.74529892)(379.86091932,18.73529893)(379.81092529,18.71530273)
\curveto(379.73091945,18.69529897)(379.63591954,18.67529899)(379.52592529,18.65530273)
\lineto(379.22592529,18.59530273)
\curveto(379.19591998,18.59529907)(379.16092002,18.59029908)(379.12092529,18.58030273)
\lineto(379.01592529,18.58030273)
\curveto(378.85592032,18.54029913)(378.68592049,18.51529915)(378.50592529,18.50530273)
\curveto(378.33592084,18.50529916)(378.17092101,18.48529918)(378.01092529,18.44530273)
\curveto(377.92092126,18.42529924)(377.84092134,18.40529926)(377.77092529,18.38530273)
\curveto(377.71092147,18.37529929)(377.63592154,18.36029931)(377.54592529,18.34030273)
\curveto(377.3759218,18.29029938)(377.21092197,18.22529944)(377.05092529,18.14530273)
\curveto(376.90092228,18.07529959)(376.76592241,17.98529968)(376.64592529,17.87530273)
\curveto(376.52592265,17.7652999)(376.42592275,17.63030004)(376.34592529,17.47030273)
\curveto(376.26592291,17.32030035)(376.20592297,17.13530053)(376.16592529,16.91530273)
\curveto(376.14592303,16.81530085)(376.14592303,16.72030095)(376.16592529,16.63030273)
\curveto(376.18592299,16.55030112)(376.21592296,16.47530119)(376.25592529,16.40530273)
\curveto(376.30592287,16.29530137)(376.38592279,16.20030147)(376.49592529,16.12030273)
\curveto(376.61592256,16.05030162)(376.74592243,15.99030168)(376.88592529,15.94030273)
\curveto(376.93592224,15.93030174)(376.98592219,15.92530174)(377.03592529,15.92530273)
\curveto(377.08592209,15.92530174)(377.13592204,15.92030175)(377.18592529,15.91030273)
\curveto(377.25592192,15.89030178)(377.34092184,15.87530179)(377.44092529,15.86530273)
\curveto(377.54092164,15.8653018)(377.63092155,15.87530179)(377.71092529,15.89530273)
\curveto(377.77092141,15.91530175)(377.83092135,15.92030175)(377.89092529,15.91030273)
\curveto(377.95092123,15.91030176)(378.01092117,15.92030175)(378.07092529,15.94030273)
\curveto(378.16092102,15.96030171)(378.24092094,15.97530169)(378.31092529,15.98530273)
\curveto(378.39092079,15.99530167)(378.47092071,16.01530165)(378.55092529,16.04530273)
\curveto(378.86092032,16.1653015)(379.13592004,16.31030136)(379.37592529,16.48030273)
\curveto(379.61591956,16.65030102)(379.82091936,16.88030079)(379.99092529,17.17030273)
}
}
{
\newrgbcolor{curcolor}{0 0 0}
\pscustom[linestyle=none,fillstyle=solid,fillcolor=curcolor]
{
\newpath
\moveto(387.58256592,22.85530273)
\curveto(388.32255954,22.8652948)(388.91755894,22.75529491)(389.36756592,22.52530273)
\curveto(389.81755804,22.30529536)(390.14255772,21.9702957)(390.34256592,21.52030273)
\curveto(390.43255743,21.32029635)(390.49255737,21.07529659)(390.52256592,20.78530273)
\curveto(390.53255733,20.73529693)(390.53255733,20.670297)(390.52256592,20.59030273)
\curveto(390.52255734,20.51029716)(390.50755735,20.44029723)(390.47756592,20.38030273)
\curveto(390.43755742,20.33029734)(390.37755748,20.28529738)(390.29756592,20.24530273)
\curveto(390.2575576,20.22529744)(390.22255764,20.21529745)(390.19256592,20.21530273)
\curveto(390.17255769,20.22529744)(390.13755772,20.22529744)(390.08756592,20.21530273)
\curveto(390.04755781,20.20529746)(390.00755785,20.20029747)(389.96756592,20.20030273)
\curveto(389.92755793,20.21029746)(389.88755797,20.21529745)(389.84756592,20.21530273)
\lineto(389.53256592,20.21530273)
\curveto(389.44255842,20.22529744)(389.36755849,20.25529741)(389.30756592,20.30530273)
\curveto(389.23755862,20.3652973)(389.19755866,20.45029722)(389.18756592,20.56030273)
\curveto(389.17755868,20.670297)(389.1575587,20.7652969)(389.12756592,20.84530273)
\curveto(389.02755883,21.10529656)(388.87255899,21.31029636)(388.66256592,21.46030273)
\curveto(388.59255927,21.51029616)(388.51255935,21.55029612)(388.42256592,21.58030273)
\curveto(388.34255952,21.62029605)(388.2575596,21.65529601)(388.16756592,21.68530273)
\curveto(388.03755982,21.72529594)(387.85756,21.74529592)(387.62756592,21.74530273)
\curveto(387.39756046,21.75529591)(387.20256066,21.73529593)(387.04256592,21.68530273)
\curveto(386.97256089,21.665296)(386.90256096,21.65029602)(386.83256592,21.64030273)
\curveto(386.77256109,21.63029604)(386.70756115,21.61529605)(386.63756592,21.59530273)
\curveto(386.3575615,21.48529618)(386.09756176,21.33529633)(385.85756592,21.14530273)
\curveto(385.61756224,20.95529671)(385.41756244,20.73029694)(385.25756592,20.47030273)
\curveto(385.19756266,20.38029729)(385.14256272,20.28529738)(385.09256592,20.18530273)
\curveto(385.04256282,20.09529757)(384.99256287,19.99529767)(384.94256592,19.88530273)
\lineto(384.77756592,19.48030273)
\curveto(384.7575631,19.43029824)(384.74256312,19.37529829)(384.73256592,19.31530273)
\curveto(384.72256314,19.25529841)(384.70256316,19.20029847)(384.67256592,19.15030273)
\lineto(384.65756592,19.03030273)
\curveto(384.63756322,18.99029868)(384.61256325,18.92529874)(384.58256592,18.83530273)
\curveto(384.5625633,18.74529892)(384.5575633,18.68029899)(384.56756592,18.64030273)
\curveto(384.56756329,18.59029908)(384.5575633,18.54029913)(384.53756592,18.49030273)
\curveto(384.51756334,18.44029923)(384.50756335,18.39029928)(384.50756592,18.34030273)
\curveto(384.51756334,18.30029937)(384.51256335,18.23029944)(384.49256592,18.13030273)
\curveto(384.49256337,18.05029962)(384.48756337,17.9652997)(384.47756592,17.87530273)
\curveto(384.47756338,17.78529988)(384.48256338,17.70029997)(384.49256592,17.62030273)
\curveto(384.53256333,17.30030037)(384.60256326,17.02030065)(384.70256592,16.78030273)
\curveto(384.80256306,16.55030112)(384.96756289,16.35030132)(385.19756592,16.18030273)
\curveto(385.27756258,16.13030154)(385.3575625,16.08530158)(385.43756592,16.04530273)
\curveto(385.52756233,16.00530166)(385.62256224,15.9653017)(385.72256592,15.92530273)
\curveto(385.77256209,15.91530175)(385.81256205,15.91030176)(385.84256592,15.91030273)
\curveto(385.87256199,15.91030176)(385.91256195,15.90530176)(385.96256592,15.89530273)
\curveto(385.99256187,15.88530178)(386.04256182,15.88030179)(386.11256592,15.88030273)
\lineto(386.27756592,15.88030273)
\curveto(386.26756159,15.8703018)(386.28256158,15.8653018)(386.32256592,15.86530273)
\curveto(386.35256151,15.87530179)(386.37756148,15.87530179)(386.39756592,15.86530273)
\curveto(386.42756143,15.8653018)(386.4625614,15.8703018)(386.50256592,15.88030273)
\curveto(386.57256129,15.90030177)(386.63756122,15.90530176)(386.69756592,15.89530273)
\curveto(386.76756109,15.89530177)(386.83756102,15.90530176)(386.90756592,15.92530273)
\curveto(387.18756067,16.00530166)(387.43256043,16.10530156)(387.64256592,16.22530273)
\curveto(387.86256,16.35530131)(388.0575598,16.52030115)(388.22756592,16.72030273)
\curveto(388.28755957,16.80030087)(388.34755951,16.88530078)(388.40756592,16.97530273)
\lineto(388.58756592,17.24530273)
\curveto(388.61755924,17.32530034)(388.65255921,17.40030027)(388.69256592,17.47030273)
\curveto(388.73255913,17.55030012)(388.79255907,17.61530005)(388.87256592,17.66530273)
\curveto(388.91255895,17.69529997)(388.97755888,17.71529995)(389.06756592,17.72530273)
\curveto(389.16755869,17.74529992)(389.26755859,17.75529991)(389.36756592,17.75530273)
\curveto(389.47755838,17.7652999)(389.57755828,17.7652999)(389.66756592,17.75530273)
\curveto(389.7575581,17.74529992)(389.82255804,17.72529994)(389.86256592,17.69530273)
\curveto(389.91255795,17.65530001)(389.93755792,17.59530007)(389.93756592,17.51530273)
\curveto(389.93755792,17.43530023)(389.91755794,17.35030032)(389.87756592,17.26030273)
\curveto(389.79755806,17.11030056)(389.72255814,16.9653007)(389.65256592,16.82530273)
\curveto(389.58255828,16.69530097)(389.49755836,16.5653011)(389.39756592,16.43530273)
\curveto(389.18755867,16.13530153)(388.94755891,15.8703018)(388.67756592,15.64030273)
\curveto(388.40755945,15.41030226)(388.09755976,15.22530244)(387.74756592,15.08530273)
\curveto(387.6575602,15.04530262)(387.5625603,15.01030266)(387.46256592,14.98030273)
\curveto(387.37256049,14.96030271)(387.27756058,14.93530273)(387.17756592,14.90530273)
\curveto(387.0575608,14.8653028)(386.94256092,14.84530282)(386.83256592,14.84530273)
\curveto(386.72256114,14.83530283)(386.60756125,14.82030285)(386.48756592,14.80030273)
\curveto(386.44756141,14.78030289)(386.40756145,14.77530289)(386.36756592,14.78530273)
\curveto(386.32756153,14.79530287)(386.28756157,14.79530287)(386.24756592,14.78530273)
\lineto(386.11256592,14.78530273)
\lineto(385.87256592,14.78530273)
\curveto(385.80256206,14.77530289)(385.73756212,14.78030289)(385.67756592,14.80030273)
\lineto(385.60256592,14.80030273)
\lineto(385.25756592,14.84530273)
\curveto(385.13756272,14.88530278)(385.01756284,14.92030275)(384.89756592,14.95030273)
\curveto(384.78756307,14.98030269)(384.68256318,15.02030265)(384.58256592,15.07030273)
\curveto(384.25256361,15.23030244)(383.99256387,15.42030225)(383.80256592,15.64030273)
\curveto(383.61256425,15.86030181)(383.44756441,16.13030154)(383.30756592,16.45030273)
\curveto(383.27756458,16.53030114)(383.25256461,16.62030105)(383.23256592,16.72030273)
\lineto(383.17256592,17.02030273)
\curveto(383.14256472,17.13030054)(383.12756473,17.24530042)(383.12756592,17.36530273)
\curveto(383.13756472,17.48530018)(383.13756472,17.60530006)(383.12756592,17.72530273)
\curveto(383.12756473,17.7652999)(383.13256473,17.80529986)(383.14256592,17.84530273)
\curveto(383.15256471,17.88529978)(383.15256471,17.92529974)(383.14256592,17.96530273)
\curveto(383.14256472,18.02529964)(383.14756471,18.09029958)(383.15756592,18.16030273)
\curveto(383.17756468,18.23029944)(383.18756467,18.29529937)(383.18756592,18.35530273)
\lineto(383.21756592,18.50530273)
\curveto(383.21756464,18.55529911)(383.22256464,18.62529904)(383.23256592,18.71530273)
\curveto(383.25256461,18.80529886)(383.27256459,18.87529879)(383.29256592,18.92530273)
\curveto(383.31256455,18.97529869)(383.32256454,19.02029865)(383.32256592,19.06030273)
\curveto(383.33256453,19.10029857)(383.34756451,19.14029853)(383.36756592,19.18030273)
\curveto(383.39756446,19.25029842)(383.41756444,19.32029835)(383.42756592,19.39030273)
\curveto(383.43756442,19.46029821)(383.4575644,19.52529814)(383.48756592,19.58530273)
\curveto(383.5575643,19.75529791)(383.62256424,19.92529774)(383.68256592,20.09530273)
\curveto(383.75256411,20.2652974)(383.83256403,20.42529724)(383.92256592,20.57530273)
\curveto(384.24256362,21.09529657)(384.58256328,21.51529615)(384.94256592,21.83530273)
\curveto(385.30256256,22.15529551)(385.76756209,22.42029525)(386.33756592,22.63030273)
\curveto(386.4575614,22.68029499)(386.58256128,22.71529495)(386.71256592,22.73530273)
\curveto(386.84256102,22.75529491)(386.98256088,22.78029489)(387.13256592,22.81030273)
\curveto(387.20256066,22.82029485)(387.27256059,22.82529484)(387.34256592,22.82530273)
\curveto(387.41256045,22.83529483)(387.49256037,22.84529482)(387.58256592,22.85530273)
}
}
{
\newrgbcolor{curcolor}{0 0 0}
\pscustom[linestyle=none,fillstyle=solid,fillcolor=curcolor]
{
\newpath
\moveto(393.04420654,24.17530273)
\curveto(392.97420357,24.23529343)(392.95420359,24.34029333)(392.98420654,24.49030273)
\curveto(393.01420353,24.65029302)(393.0442035,24.80529286)(393.07420654,24.95530273)
\curveto(393.08420346,25.03529263)(393.09920344,25.12029255)(393.11920654,25.21030273)
\curveto(393.1392034,25.30029237)(393.16920337,25.37529229)(393.20920654,25.43530273)
\curveto(393.26920327,25.51529215)(393.35920318,25.57529209)(393.47920654,25.61530273)
\curveto(393.50920303,25.62529204)(393.53420301,25.62529204)(393.55420654,25.61530273)
\curveto(393.57420297,25.61529205)(393.59920294,25.62029205)(393.62920654,25.63030273)
\curveto(393.79920274,25.63029204)(393.95420259,25.62529204)(394.09420654,25.61530273)
\curveto(394.2442023,25.60529206)(394.33420221,25.54529212)(394.36420654,25.43530273)
\curveto(394.38420216,25.37529229)(394.38420216,25.30029237)(394.36420654,25.21030273)
\curveto(394.3442022,25.13029254)(394.32920221,25.04529262)(394.31920654,24.95530273)
\curveto(394.27920226,24.77529289)(394.2392023,24.60529306)(394.19920654,24.44530273)
\curveto(394.16920237,24.28529338)(394.08420246,24.18029349)(393.94420654,24.13030273)
\curveto(393.88420266,24.11029356)(393.82420272,24.10029357)(393.76420654,24.10030273)
\lineto(393.59920654,24.10030273)
\lineto(393.28420654,24.10030273)
\curveto(393.18420336,24.10029357)(393.10420344,24.12529354)(393.04420654,24.17530273)
\moveto(392.45920654,15.67030273)
\curveto(392.4392041,15.5703021)(392.41920412,15.4653022)(392.39920654,15.35530273)
\curveto(392.38920415,15.25530241)(392.34920419,15.17530249)(392.27920654,15.11530273)
\curveto(392.2392043,15.05530261)(392.18920435,15.01530265)(392.12920654,14.99530273)
\curveto(392.06920447,14.98530268)(391.99420455,14.9703027)(391.90420654,14.95030273)
\lineto(391.67920654,14.95030273)
\curveto(391.54920499,14.95030272)(391.4392051,14.95530271)(391.34920654,14.96530273)
\curveto(391.25920528,14.98530268)(391.19420535,15.03530263)(391.15420654,15.11530273)
\curveto(391.13420541,15.17530249)(391.12920541,15.25030242)(391.13920654,15.34030273)
\curveto(391.15920538,15.44030223)(391.17920536,15.53530213)(391.19920654,15.62530273)
\lineto(392.47420654,21.97030273)
\curveto(392.49420405,22.08029559)(392.51420403,22.18529548)(392.53420654,22.28530273)
\curveto(392.55420399,22.39529527)(392.59420395,22.48029519)(392.65420654,22.54030273)
\curveto(392.69420385,22.59029508)(392.7392038,22.62029505)(392.78920654,22.63030273)
\curveto(392.84920369,22.64029503)(392.90920363,22.65529501)(392.96920654,22.67530273)
\curveto(392.98920355,22.67529499)(393.00920353,22.670295)(393.02920654,22.66030273)
\curveto(393.05920348,22.66029501)(393.08420346,22.665295)(393.10420654,22.67530273)
\curveto(393.23420331,22.67529499)(393.36420318,22.670295)(393.49420654,22.66030273)
\curveto(393.63420291,22.66029501)(393.71920282,22.62029505)(393.74920654,22.54030273)
\curveto(393.78920275,22.48029519)(393.79920274,22.40029527)(393.77920654,22.30030273)
\curveto(393.75920278,22.21029546)(393.7392028,22.11529555)(393.71920654,22.01530273)
\lineto(392.45920654,15.67030273)
}
}
{
\newrgbcolor{curcolor}{0 0 0}
\pscustom[linestyle=none,fillstyle=solid,fillcolor=curcolor]
{
\newpath
\moveto(402.21905029,19.15030273)
\curveto(402.2290414,19.09029858)(402.21904141,18.99529867)(402.18905029,18.86530273)
\curveto(402.16904146,18.74529892)(402.14904148,18.66029901)(402.12905029,18.61030273)
\lineto(402.09905029,18.46030273)
\curveto(402.06904156,18.38029929)(402.04404159,18.30529936)(402.02405029,18.23530273)
\curveto(402.01404162,18.17529949)(401.99404164,18.10529956)(401.96405029,18.02530273)
\curveto(401.9340417,17.9652997)(401.90904172,17.90529976)(401.88905029,17.84530273)
\curveto(401.87904175,17.78529988)(401.85404178,17.72529994)(401.81405029,17.66530273)
\lineto(401.63405029,17.27530273)
\curveto(401.58404205,17.14530052)(401.51904211,17.02530064)(401.43905029,16.91530273)
\curveto(401.13904249,16.43530123)(400.77904285,16.03030164)(400.35905029,15.70030273)
\curveto(399.94904368,15.38030229)(399.46904416,15.13530253)(398.91905029,14.96530273)
\curveto(398.80904482,14.92530274)(398.68904494,14.89530277)(398.55905029,14.87530273)
\curveto(398.4290452,14.85530281)(398.29404534,14.83530283)(398.15405029,14.81530273)
\curveto(398.09404554,14.80530286)(398.0290456,14.80030287)(397.95905029,14.80030273)
\curveto(397.89904573,14.79030288)(397.83904579,14.78530288)(397.77905029,14.78530273)
\curveto(397.73904589,14.77530289)(397.67904595,14.7703029)(397.59905029,14.77030273)
\curveto(397.5290461,14.7703029)(397.47904615,14.77530289)(397.44905029,14.78530273)
\curveto(397.40904622,14.79530287)(397.36904626,14.80030287)(397.32905029,14.80030273)
\curveto(397.28904634,14.79030288)(397.25404638,14.79030288)(397.22405029,14.80030273)
\lineto(397.13405029,14.80030273)
\lineto(396.78905029,14.84530273)
\lineto(396.39905029,14.96530273)
\curveto(396.27904735,15.00530266)(396.16404747,15.05030262)(396.05405029,15.10030273)
\curveto(395.64404799,15.30030237)(395.32404831,15.56030211)(395.09405029,15.88030273)
\curveto(394.87404876,16.20030147)(394.71404892,16.59030108)(394.61405029,17.05030273)
\curveto(394.58404905,17.15030052)(394.56404907,17.25030042)(394.55405029,17.35030273)
\lineto(394.55405029,17.66530273)
\curveto(394.54404909,17.70529996)(394.54404909,17.73529993)(394.55405029,17.75530273)
\curveto(394.56404907,17.78529988)(394.56904906,17.82029985)(394.56905029,17.86030273)
\curveto(394.56904906,17.94029973)(394.57404906,18.02029965)(394.58405029,18.10030273)
\curveto(394.59404904,18.19029948)(394.59904903,18.27529939)(394.59905029,18.35530273)
\curveto(394.60904902,18.40529926)(394.61404902,18.44529922)(394.61405029,18.47530273)
\curveto(394.62404901,18.51529915)(394.629049,18.56029911)(394.62905029,18.61030273)
\curveto(394.629049,18.66029901)(394.63904899,18.74529892)(394.65905029,18.86530273)
\curveto(394.68904894,18.99529867)(394.71904891,19.09029858)(394.74905029,19.15030273)
\curveto(394.78904884,19.22029845)(394.80904882,19.29029838)(394.80905029,19.36030273)
\curveto(394.80904882,19.43029824)(394.8290488,19.50029817)(394.86905029,19.57030273)
\curveto(394.88904874,19.62029805)(394.90404873,19.66029801)(394.91405029,19.69030273)
\curveto(394.92404871,19.73029794)(394.93904869,19.77529789)(394.95905029,19.82530273)
\curveto(395.01904861,19.94529772)(395.06904856,20.0652976)(395.10905029,20.18530273)
\curveto(395.15904847,20.30529736)(395.22404841,20.42029725)(395.30405029,20.53030273)
\curveto(395.52404811,20.90029677)(395.76904786,21.23029644)(396.03905029,21.52030273)
\curveto(396.31904731,21.82029585)(396.634047,22.0702956)(396.98405029,22.27030273)
\curveto(397.11404652,22.35029532)(397.24904638,22.41529525)(397.38905029,22.46530273)
\lineto(397.83905029,22.64530273)
\curveto(397.96904566,22.69529497)(398.10404553,22.72529494)(398.24405029,22.73530273)
\curveto(398.38404525,22.75529491)(398.5290451,22.78529488)(398.67905029,22.82530273)
\lineto(398.87405029,22.82530273)
\lineto(399.08405029,22.85530273)
\curveto(399.97404366,22.8652948)(400.67404296,22.68029499)(401.18405029,22.30030273)
\curveto(401.70404193,21.92029575)(402.0290416,21.42529624)(402.15905029,20.81530273)
\curveto(402.18904144,20.71529695)(402.20904142,20.61529705)(402.21905029,20.51530273)
\curveto(402.2290414,20.41529725)(402.24404139,20.31029736)(402.26405029,20.20030273)
\curveto(402.27404136,20.09029758)(402.27404136,19.9702977)(402.26405029,19.84030273)
\lineto(402.26405029,19.46530273)
\curveto(402.26404137,19.41529825)(402.25404138,19.36029831)(402.23405029,19.30030273)
\curveto(402.22404141,19.25029842)(402.21904141,19.20029847)(402.21905029,19.15030273)
\moveto(400.71905029,18.29530273)
\curveto(400.74904288,18.3652993)(400.76904286,18.44529922)(400.77905029,18.53530273)
\curveto(400.79904283,18.62529904)(400.81404282,18.71029896)(400.82405029,18.79030273)
\curveto(400.90404273,19.18029849)(400.93904269,19.51029816)(400.92905029,19.78030273)
\curveto(400.90904272,19.86029781)(400.89404274,19.94029773)(400.88405029,20.02030273)
\curveto(400.88404275,20.10029757)(400.87904275,20.17529749)(400.86905029,20.24530273)
\curveto(400.71904291,20.89529677)(400.36404327,21.34529632)(399.80405029,21.59530273)
\curveto(399.7340439,21.62529604)(399.65904397,21.64529602)(399.57905029,21.65530273)
\curveto(399.50904412,21.67529599)(399.4340442,21.69529597)(399.35405029,21.71530273)
\curveto(399.28404435,21.73529593)(399.20404443,21.74529592)(399.11405029,21.74530273)
\lineto(398.84405029,21.74530273)
\lineto(398.55905029,21.70030273)
\curveto(398.45904517,21.68029599)(398.36404527,21.65529601)(398.27405029,21.62530273)
\curveto(398.18404545,21.60529606)(398.09404554,21.57529609)(398.00405029,21.53530273)
\curveto(397.9340457,21.51529615)(397.86404577,21.48529618)(397.79405029,21.44530273)
\curveto(397.72404591,21.40529626)(397.65904597,21.3652963)(397.59905029,21.32530273)
\curveto(397.3290463,21.15529651)(397.09404654,20.95029672)(396.89405029,20.71030273)
\curveto(396.69404694,20.4702972)(396.50904712,20.19029748)(396.33905029,19.87030273)
\curveto(396.28904734,19.7702979)(396.24904738,19.665298)(396.21905029,19.55530273)
\curveto(396.18904744,19.45529821)(396.14904748,19.35029832)(396.09905029,19.24030273)
\curveto(396.08904754,19.20029847)(396.07404756,19.13529853)(396.05405029,19.04530273)
\curveto(396.0340476,19.01529865)(396.02404761,18.98029869)(396.02405029,18.94030273)
\curveto(396.02404761,18.90029877)(396.01904761,18.85529881)(396.00905029,18.80530273)
\lineto(395.94905029,18.50530273)
\curveto(395.9290477,18.40529926)(395.91904771,18.31529935)(395.91905029,18.23530273)
\lineto(395.91905029,18.05530273)
\curveto(395.91904771,17.95529971)(395.91404772,17.85529981)(395.90405029,17.75530273)
\curveto(395.90404773,17.6653)(395.91404772,17.58030009)(395.93405029,17.50030273)
\curveto(395.98404765,17.26030041)(396.05404758,17.03530063)(396.14405029,16.82530273)
\curveto(396.24404739,16.61530105)(396.37904725,16.44030123)(396.54905029,16.30030273)
\curveto(396.59904703,16.2703014)(396.63904699,16.24530142)(396.66905029,16.22530273)
\curveto(396.70904692,16.20530146)(396.74904688,16.18030149)(396.78905029,16.15030273)
\curveto(396.85904677,16.10030157)(396.93904669,16.05530161)(397.02905029,16.01530273)
\curveto(397.11904651,15.98530168)(397.21404642,15.95530171)(397.31405029,15.92530273)
\curveto(397.36404627,15.90530176)(397.40904622,15.89530177)(397.44905029,15.89530273)
\curveto(397.49904613,15.90530176)(397.54904608,15.90530176)(397.59905029,15.89530273)
\curveto(397.629046,15.88530178)(397.68904594,15.87530179)(397.77905029,15.86530273)
\curveto(397.86904576,15.85530181)(397.94404569,15.86030181)(398.00405029,15.88030273)
\curveto(398.04404559,15.89030178)(398.08404555,15.89030178)(398.12405029,15.88030273)
\curveto(398.16404547,15.88030179)(398.20404543,15.89030178)(398.24405029,15.91030273)
\curveto(398.32404531,15.93030174)(398.40404523,15.94530172)(398.48405029,15.95530273)
\curveto(398.57404506,15.97530169)(398.65904497,16.00030167)(398.73905029,16.03030273)
\curveto(399.09904453,16.1703015)(399.40904422,16.3653013)(399.66905029,16.61530273)
\curveto(399.9290437,16.8653008)(400.16404347,17.16030051)(400.37405029,17.50030273)
\curveto(400.45404318,17.62030005)(400.51404312,17.74529992)(400.55405029,17.87530273)
\curveto(400.59404304,18.01529965)(400.64904298,18.15529951)(400.71905029,18.29530273)
}
}
{
\newrgbcolor{curcolor}{0 0 0}
\pscustom[linestyle=none,fillstyle=solid,fillcolor=curcolor]
{
\newpath
\moveto(406.88733154,22.85530273)
\curveto(407.60732589,22.8652948)(408.1923253,22.78029489)(408.64233154,22.60030273)
\curveto(409.10232439,22.43029524)(409.42232407,22.12529554)(409.60233154,21.68530273)
\curveto(409.65232384,21.57529609)(409.68232381,21.46029621)(409.69233154,21.34030273)
\curveto(409.71232378,21.23029644)(409.72732377,21.10529656)(409.73733154,20.96530273)
\curveto(409.74732375,20.89529677)(409.73732376,20.82029685)(409.70733154,20.74030273)
\curveto(409.68732381,20.670297)(409.66232383,20.61529705)(409.63233154,20.57530273)
\curveto(409.61232388,20.55529711)(409.58232391,20.53529713)(409.54233154,20.51530273)
\curveto(409.51232398,20.50529716)(409.48732401,20.49029718)(409.46733154,20.47030273)
\curveto(409.40732409,20.45029722)(409.35232414,20.44529722)(409.30233154,20.45530273)
\curveto(409.26232423,20.4652972)(409.21732428,20.4652972)(409.16733154,20.45530273)
\curveto(409.07732442,20.43529723)(408.96732453,20.43029724)(408.83733154,20.44030273)
\curveto(408.71732478,20.46029721)(408.63232486,20.48529718)(408.58233154,20.51530273)
\curveto(408.51232498,20.5652971)(408.47232502,20.63029704)(408.46233154,20.71030273)
\curveto(408.46232503,20.80029687)(408.44232505,20.88529678)(408.40233154,20.96530273)
\curveto(408.35232514,21.12529654)(408.25732524,21.2702964)(408.11733154,21.40030273)
\curveto(408.02732547,21.48029619)(407.91732558,21.54029613)(407.78733154,21.58030273)
\curveto(407.66732583,21.62029605)(407.53732596,21.66029601)(407.39733154,21.70030273)
\curveto(407.35732614,21.72029595)(407.30732619,21.72529594)(407.24733154,21.71530273)
\curveto(407.1973263,21.71529595)(407.15232634,21.72029595)(407.11233154,21.73030273)
\curveto(407.05232644,21.75029592)(406.97732652,21.76029591)(406.88733154,21.76030273)
\curveto(406.7973267,21.76029591)(406.72232677,21.75029592)(406.66233154,21.73030273)
\lineto(406.57233154,21.73030273)
\curveto(406.51232698,21.72029595)(406.45732704,21.71029596)(406.40733154,21.70030273)
\curveto(406.35732714,21.70029597)(406.30732719,21.69529597)(406.25733154,21.68530273)
\curveto(405.98732751,21.62529604)(405.75232774,21.54029613)(405.55233154,21.43030273)
\curveto(405.36232813,21.32029635)(405.21232828,21.13529653)(405.10233154,20.87530273)
\curveto(405.07232842,20.80529686)(405.05732844,20.73529693)(405.05733154,20.66530273)
\curveto(405.05732844,20.59529707)(405.06232843,20.53529713)(405.07233154,20.48530273)
\curveto(405.10232839,20.33529733)(405.15232834,20.22529744)(405.22233154,20.15530273)
\curveto(405.2923282,20.09529757)(405.38732811,20.02529764)(405.50733154,19.94530273)
\curveto(405.64732785,19.84529782)(405.81232768,19.7702979)(406.00233154,19.72030273)
\curveto(406.1923273,19.68029799)(406.38232711,19.63029804)(406.57233154,19.57030273)
\curveto(406.6923268,19.53029814)(406.81232668,19.50029817)(406.93233154,19.48030273)
\curveto(407.06232643,19.46029821)(407.18732631,19.43029824)(407.30733154,19.39030273)
\curveto(407.50732599,19.33029834)(407.70232579,19.2702984)(407.89233154,19.21030273)
\curveto(408.08232541,19.16029851)(408.26732523,19.09529857)(408.44733154,19.01530273)
\curveto(408.497325,18.99529867)(408.54232495,18.97529869)(408.58233154,18.95530273)
\curveto(408.63232486,18.93529873)(408.68232481,18.91029876)(408.73233154,18.88030273)
\curveto(408.90232459,18.76029891)(409.04732445,18.62529904)(409.16733154,18.47530273)
\curveto(409.28732421,18.32529934)(409.37732412,18.13529953)(409.43733154,17.90530273)
\lineto(409.43733154,17.62030273)
\curveto(409.43732406,17.55030012)(409.43232406,17.47530019)(409.42233154,17.39530273)
\curveto(409.41232408,17.32530034)(409.40232409,17.24530042)(409.39233154,17.15530273)
\lineto(409.36233154,17.00530273)
\curveto(409.32232417,16.93530073)(409.2923242,16.8653008)(409.27233154,16.79530273)
\curveto(409.26232423,16.72530094)(409.24232425,16.65530101)(409.21233154,16.58530273)
\curveto(409.16232433,16.47530119)(409.10732439,16.3703013)(409.04733154,16.27030273)
\curveto(408.98732451,16.1703015)(408.92232457,16.08030159)(408.85233154,16.00030273)
\curveto(408.64232485,15.74030193)(408.3973251,15.53030214)(408.11733154,15.37030273)
\curveto(407.83732566,15.22030245)(407.53232596,15.09030258)(407.20233154,14.98030273)
\curveto(407.10232639,14.95030272)(407.00232649,14.93030274)(406.90233154,14.92030273)
\curveto(406.80232669,14.90030277)(406.70732679,14.87530279)(406.61733154,14.84530273)
\curveto(406.50732699,14.82530284)(406.40232709,14.81530285)(406.30233154,14.81530273)
\curveto(406.20232729,14.81530285)(406.10232739,14.80530286)(406.00233154,14.78530273)
\lineto(405.85233154,14.78530273)
\curveto(405.80232769,14.77530289)(405.73232776,14.7703029)(405.64233154,14.77030273)
\curveto(405.55232794,14.7703029)(405.48232801,14.77530289)(405.43233154,14.78530273)
\lineto(405.26733154,14.78530273)
\curveto(405.20732829,14.80530286)(405.14232835,14.81530285)(405.07233154,14.81530273)
\curveto(405.00232849,14.80530286)(404.94732855,14.81030286)(404.90733154,14.83030273)
\curveto(404.85732864,14.84030283)(404.7923287,14.84530282)(404.71233154,14.84530273)
\curveto(404.63232886,14.8653028)(404.55732894,14.88530278)(404.48733154,14.90530273)
\curveto(404.41732908,14.91530275)(404.34232915,14.93530273)(404.26233154,14.96530273)
\curveto(403.97232952,15.0653026)(403.72732977,15.19030248)(403.52733154,15.34030273)
\curveto(403.32733017,15.49030218)(403.16733033,15.68530198)(403.04733154,15.92530273)
\curveto(402.98733051,16.05530161)(402.93733056,16.19030148)(402.89733154,16.33030273)
\curveto(402.86733063,16.4703012)(402.84733065,16.62530104)(402.83733154,16.79530273)
\curveto(402.82733067,16.85530081)(402.83233066,16.92530074)(402.85233154,17.00530273)
\curveto(402.87233062,17.09530057)(402.8973306,17.1653005)(402.92733154,17.21530273)
\curveto(402.96733053,17.25530041)(403.02733047,17.29530037)(403.10733154,17.33530273)
\curveto(403.15733034,17.35530031)(403.22733027,17.3653003)(403.31733154,17.36530273)
\curveto(403.41733008,17.37530029)(403.50732999,17.37530029)(403.58733154,17.36530273)
\curveto(403.67732982,17.35530031)(403.76232973,17.34030033)(403.84233154,17.32030273)
\curveto(403.93232956,17.31030036)(403.98732951,17.29530037)(404.00733154,17.27530273)
\curveto(404.06732943,17.22530044)(404.0973294,17.15030052)(404.09733154,17.05030273)
\curveto(404.10732939,16.96030071)(404.12732937,16.87530079)(404.15733154,16.79530273)
\curveto(404.20732929,16.57530109)(404.30732919,16.40530126)(404.45733154,16.28530273)
\curveto(404.55732894,16.19530147)(404.67732882,16.12530154)(404.81733154,16.07530273)
\curveto(404.95732854,16.02530164)(405.10732839,15.97530169)(405.26733154,15.92530273)
\lineto(405.58233154,15.88030273)
\lineto(405.67233154,15.88030273)
\curveto(405.73232776,15.86030181)(405.81732768,15.85030182)(405.92733154,15.85030273)
\curveto(406.04732745,15.85030182)(406.15232734,15.86030181)(406.24233154,15.88030273)
\curveto(406.31232718,15.88030179)(406.36732713,15.88530178)(406.40733154,15.89530273)
\curveto(406.46732703,15.90530176)(406.52732697,15.91030176)(406.58733154,15.91030273)
\curveto(406.64732685,15.92030175)(406.70232679,15.93030174)(406.75233154,15.94030273)
\curveto(407.06232643,16.02030165)(407.31232618,16.12530154)(407.50233154,16.25530273)
\curveto(407.70232579,16.38530128)(407.86732563,16.60530106)(407.99733154,16.91530273)
\curveto(408.02732547,16.9653007)(408.04232545,17.02030065)(408.04233154,17.08030273)
\curveto(408.05232544,17.14030053)(408.05232544,17.18530048)(408.04233154,17.21530273)
\curveto(408.03232546,17.40530026)(407.9923255,17.54530012)(407.92233154,17.63530273)
\curveto(407.85232564,17.73529993)(407.75732574,17.82529984)(407.63733154,17.90530273)
\curveto(407.55732594,17.9652997)(407.46232603,18.01529965)(407.35233154,18.05530273)
\lineto(407.05233154,18.17530273)
\curveto(407.02232647,18.18529948)(406.9923265,18.19029948)(406.96233154,18.19030273)
\curveto(406.94232655,18.19029948)(406.92232657,18.20029947)(406.90233154,18.22030273)
\curveto(406.58232691,18.33029934)(406.24232725,18.41029926)(405.88233154,18.46030273)
\curveto(405.53232796,18.52029915)(405.21232828,18.61529905)(404.92233154,18.74530273)
\curveto(404.83232866,18.78529888)(404.74232875,18.82029885)(404.65233154,18.85030273)
\curveto(404.57232892,18.88029879)(404.497329,18.92029875)(404.42733154,18.97030273)
\curveto(404.25732924,19.08029859)(404.10732939,19.20529846)(403.97733154,19.34530273)
\curveto(403.84732965,19.48529818)(403.75732974,19.66029801)(403.70733154,19.87030273)
\curveto(403.68732981,19.94029773)(403.67732982,20.01029766)(403.67733154,20.08030273)
\lineto(403.67733154,20.30530273)
\curveto(403.66732983,20.42529724)(403.68232981,20.56029711)(403.72233154,20.71030273)
\curveto(403.76232973,20.8702968)(403.80232969,21.00529666)(403.84233154,21.11530273)
\curveto(403.87232962,21.1652965)(403.8923296,21.20529646)(403.90233154,21.23530273)
\curveto(403.92232957,21.27529639)(403.94732955,21.31529635)(403.97733154,21.35530273)
\curveto(404.10732939,21.58529608)(404.26732923,21.78529588)(404.45733154,21.95530273)
\curveto(404.64732885,22.12529554)(404.85732864,22.27529539)(405.08733154,22.40530273)
\curveto(405.24732825,22.49529517)(405.42232807,22.5652951)(405.61233154,22.61530273)
\curveto(405.81232768,22.67529499)(406.01732748,22.73029494)(406.22733154,22.78030273)
\curveto(406.2973272,22.79029488)(406.36232713,22.80029487)(406.42233154,22.81030273)
\curveto(406.492327,22.82029485)(406.56732693,22.83029484)(406.64733154,22.84030273)
\curveto(406.68732681,22.85029482)(406.72732677,22.85029482)(406.76733154,22.84030273)
\curveto(406.81732668,22.83029484)(406.85732664,22.83529483)(406.88733154,22.85530273)
}
}
{
\newrgbcolor{curcolor}{0 0 0}
\pscustom[linestyle=none,fillstyle=solid,fillcolor=curcolor]
{
}
}
{
\newrgbcolor{curcolor}{0 0 0}
\pscustom[linestyle=none,fillstyle=solid,fillcolor=curcolor]
{
\newpath
\moveto(421.65248779,15.76030273)
\lineto(421.56248779,15.37030273)
\curveto(421.54247986,15.25030242)(421.5024799,15.15030252)(421.44248779,15.07030273)
\curveto(421.37248003,15.00030267)(421.27748013,14.96030271)(421.15748779,14.95030273)
\lineto(420.81248779,14.95030273)
\curveto(420.75248065,14.95030272)(420.69248071,14.94530272)(420.63248779,14.93530273)
\curveto(420.58248082,14.93530273)(420.53748087,14.94530272)(420.49748779,14.96530273)
\curveto(420.41748099,14.98530268)(420.36748104,15.02530264)(420.34748779,15.08530273)
\curveto(420.31748109,15.13530253)(420.3074811,15.19530247)(420.31748779,15.26530273)
\curveto(420.32748108,15.33530233)(420.32248108,15.40530226)(420.30248779,15.47530273)
\curveto(420.3024811,15.49530217)(420.29248111,15.51030216)(420.27248779,15.52030273)
\lineto(420.24248779,15.58030273)
\curveto(420.14248126,15.59030208)(420.05748135,15.5703021)(419.98748779,15.52030273)
\curveto(419.92748148,15.4703022)(419.86248154,15.42030225)(419.79248779,15.37030273)
\curveto(419.56248184,15.22030245)(419.33748207,15.10530256)(419.11748779,15.02530273)
\curveto(418.92748248,14.94530272)(418.7074827,14.88530278)(418.45748779,14.84530273)
\curveto(418.21748319,14.80530286)(417.97248343,14.78530288)(417.72248779,14.78530273)
\curveto(417.48248392,14.77530289)(417.24248416,14.79030288)(417.00248779,14.83030273)
\curveto(416.77248463,14.86030281)(416.57748483,14.91530275)(416.41748779,14.99530273)
\curveto(415.93748547,15.21530245)(415.57248583,15.51030216)(415.32248779,15.88030273)
\curveto(415.08248632,16.26030141)(414.92748648,16.73030094)(414.85748779,17.29030273)
\curveto(414.83748657,17.38030029)(414.82748658,17.4703002)(414.82748779,17.56030273)
\curveto(414.83748657,17.66030001)(414.83748657,17.76029991)(414.82748779,17.86030273)
\curveto(414.82748658,17.91029976)(414.83248657,17.96029971)(414.84248779,18.01030273)
\curveto(414.85248655,18.06029961)(414.85748655,18.11029956)(414.85748779,18.16030273)
\curveto(414.84748656,18.21029946)(414.84748656,18.26029941)(414.85748779,18.31030273)
\curveto(414.87748653,18.3702993)(414.88748652,18.42529924)(414.88748779,18.47530273)
\lineto(414.91748779,18.62530273)
\curveto(414.9074865,18.67529899)(414.9074865,18.74029893)(414.91748779,18.82030273)
\curveto(414.93748647,18.90029877)(414.96248644,18.9652987)(414.99248779,19.01530273)
\lineto(415.03748779,19.18030273)
\curveto(415.06748634,19.25029842)(415.08748632,19.32029835)(415.09748779,19.39030273)
\curveto(415.1074863,19.4702982)(415.12748628,19.54529812)(415.15748779,19.61530273)
\curveto(415.17748623,19.665298)(415.19248621,19.71029796)(415.20248779,19.75030273)
\curveto(415.21248619,19.79029788)(415.22748618,19.83529783)(415.24748779,19.88530273)
\curveto(415.29748611,19.98529768)(415.34248606,20.08029759)(415.38248779,20.17030273)
\curveto(415.42248598,20.2702974)(415.46748594,20.3652973)(415.51748779,20.45530273)
\curveto(415.71748569,20.83529683)(415.94748546,21.17529649)(416.20748779,21.47530273)
\curveto(416.47748493,21.78529588)(416.77748463,22.04029563)(417.10748779,22.24030273)
\curveto(417.3074841,22.36029531)(417.5074839,22.46029521)(417.70748779,22.54030273)
\curveto(417.9074835,22.62029505)(418.12248328,22.69029498)(418.35248779,22.75030273)
\lineto(418.56248779,22.78030273)
\curveto(418.63248277,22.79029488)(418.7024827,22.80529486)(418.77248779,22.82530273)
\lineto(418.92248779,22.82530273)
\curveto(419.01248239,22.84529482)(419.13248227,22.85529481)(419.28248779,22.85530273)
\curveto(419.44248196,22.85529481)(419.55748185,22.84529482)(419.62748779,22.82530273)
\curveto(419.66748174,22.81529485)(419.72248168,22.81029486)(419.79248779,22.81030273)
\curveto(419.89248151,22.78029489)(419.99748141,22.75529491)(420.10748779,22.73530273)
\curveto(420.21748119,22.72529494)(420.31748109,22.69529497)(420.40748779,22.64530273)
\curveto(420.54748086,22.58529508)(420.67748073,22.52029515)(420.79748779,22.45030273)
\curveto(420.91748049,22.38029529)(421.02748038,22.30029537)(421.12748779,22.21030273)
\curveto(421.17748023,22.16029551)(421.22748018,22.10529556)(421.27748779,22.04530273)
\curveto(421.33748007,21.99529567)(421.42247998,21.98029569)(421.53248779,22.00030273)
\lineto(421.60748779,22.07530273)
\curveto(421.62747978,22.09529557)(421.64247976,22.12529554)(421.65248779,22.16530273)
\curveto(421.7024797,22.25529541)(421.73747967,22.3702953)(421.75748779,22.51030273)
\curveto(421.78747962,22.65029502)(421.81247959,22.77529489)(421.83248779,22.88530273)
\lineto(422.17748779,24.61030273)
\curveto(422.2074792,24.75029292)(422.23747917,24.90529276)(422.26748779,25.07530273)
\curveto(422.3074791,25.25529241)(422.35747905,25.38529228)(422.41748779,25.46530273)
\curveto(422.47747893,25.53529213)(422.54747886,25.58029209)(422.62748779,25.60030273)
\curveto(422.64747876,25.60029207)(422.67247873,25.60029207)(422.70248779,25.60030273)
\curveto(422.73247867,25.61029206)(422.75747865,25.61529205)(422.77748779,25.61530273)
\curveto(422.92747848,25.62529204)(423.07747833,25.62529204)(423.22748779,25.61530273)
\curveto(423.37747803,25.61529205)(423.47747793,25.57529209)(423.52748779,25.49530273)
\curveto(423.55747785,25.41529225)(423.55747785,25.31529235)(423.52748779,25.19530273)
\curveto(423.5074779,25.07529259)(423.48747792,24.95029272)(423.46748779,24.82030273)
\lineto(421.65248779,15.76030273)
\moveto(421.00748779,18.59530273)
\curveto(421.03748037,18.64529902)(421.05748035,18.71029896)(421.06748779,18.79030273)
\curveto(421.08748032,18.88029879)(421.09248031,18.95029872)(421.08248779,19.00030273)
\lineto(421.12748779,19.22530273)
\curveto(421.12748028,19.31529835)(421.13248027,19.40529826)(421.14248779,19.49530273)
\curveto(421.15248025,19.59529807)(421.14748026,19.68529798)(421.12748779,19.76530273)
\lineto(421.12748779,19.99030273)
\curveto(421.12748028,20.06029761)(421.11748029,20.13029754)(421.09748779,20.20030273)
\curveto(421.03748037,20.50029717)(420.93248047,20.7652969)(420.78248779,20.99530273)
\curveto(420.64248076,21.22529644)(420.44248096,21.40529626)(420.18248779,21.53530273)
\curveto(420.09248131,21.58529608)(419.99748141,21.62029605)(419.89748779,21.64030273)
\curveto(419.79748161,21.670296)(419.68748172,21.69529597)(419.56748779,21.71530273)
\curveto(419.49748191,21.73529593)(419.41248199,21.74529592)(419.31248779,21.74530273)
\lineto(419.04248779,21.74530273)
\lineto(418.89248779,21.71530273)
\lineto(418.75748779,21.71530273)
\curveto(418.67748273,21.69529597)(418.59248281,21.67529599)(418.50248779,21.65530273)
\curveto(418.41248299,21.63529603)(418.32748308,21.61029606)(418.24748779,21.58030273)
\curveto(417.89748351,21.44029623)(417.59748381,21.23529643)(417.34748779,20.96530273)
\curveto(417.09748431,20.70529696)(416.87748453,20.40029727)(416.68748779,20.05030273)
\curveto(416.62748478,19.94029773)(416.57748483,19.82529784)(416.53748779,19.70530273)
\lineto(416.41748779,19.37530273)
\lineto(416.38748779,19.25530273)
\curveto(416.37748503,19.22529844)(416.36748504,19.19029848)(416.35748779,19.15030273)
\curveto(416.32748508,19.10029857)(416.3074851,19.04529862)(416.29748779,18.98530273)
\curveto(416.29748511,18.92529874)(416.29248511,18.8702988)(416.28248779,18.82030273)
\curveto(416.26248514,18.71029896)(416.23748517,18.60029907)(416.20748779,18.49030273)
\curveto(416.18748522,18.39029928)(416.18248522,18.29529937)(416.19248779,18.20530273)
\curveto(416.19248521,18.17529949)(416.18748522,18.12529954)(416.17748779,18.05530273)
\lineto(416.17748779,17.84530273)
\curveto(416.17748523,17.77529989)(416.18248522,17.70529996)(416.19248779,17.63530273)
\curveto(416.23248517,17.28530038)(416.32248508,16.98530068)(416.46248779,16.73530273)
\curveto(416.6024848,16.48530118)(416.8024846,16.28030139)(417.06248779,16.12030273)
\curveto(417.14248426,16.0703016)(417.22248418,16.03030164)(417.30248779,16.00030273)
\curveto(417.39248401,15.9703017)(417.48748392,15.94030173)(417.58748779,15.91030273)
\curveto(417.63748377,15.89030178)(417.68748372,15.88530178)(417.73748779,15.89530273)
\curveto(417.79748361,15.90530176)(417.85248355,15.90030177)(417.90248779,15.88030273)
\curveto(417.93248347,15.8703018)(417.96748344,15.8653018)(418.00748779,15.86530273)
\lineto(418.14248779,15.86530273)
\lineto(418.27748779,15.86530273)
\curveto(418.31748309,15.87530179)(418.37248303,15.88030179)(418.44248779,15.88030273)
\curveto(418.52248288,15.90030177)(418.6024828,15.91530175)(418.68248779,15.92530273)
\curveto(418.77248263,15.94530172)(418.85248255,15.9703017)(418.92248779,16.00030273)
\curveto(419.28248212,16.14030153)(419.58748182,16.31530135)(419.83748779,16.52530273)
\curveto(420.08748132,16.74530092)(420.31248109,17.02030065)(420.51248779,17.35030273)
\curveto(420.58248082,17.46030021)(420.63748077,17.5703001)(420.67748779,17.68030273)
\lineto(420.82748779,18.01030273)
\curveto(420.85748055,18.05029962)(420.87248053,18.08529958)(420.87248779,18.11530273)
\curveto(420.88248052,18.15529951)(420.89748051,18.19529947)(420.91748779,18.23530273)
\curveto(420.93748047,18.29529937)(420.95248045,18.35529931)(420.96248779,18.41530273)
\curveto(420.97248043,18.47529919)(420.98748042,18.53529913)(421.00748779,18.59530273)
}
}
{
\newrgbcolor{curcolor}{0 0 0}
\pscustom[linestyle=none,fillstyle=solid,fillcolor=curcolor]
{
\newpath
\moveto(431.02373779,19.12030273)
\curveto(431.02372929,19.02029865)(431.00372931,18.90529876)(430.96373779,18.77530273)
\curveto(430.92372939,18.65529901)(430.87372944,18.5702991)(430.81373779,18.52030273)
\curveto(430.75372956,18.48029919)(430.67372964,18.45029922)(430.57373779,18.43030273)
\curveto(430.47372984,18.42029925)(430.36372995,18.41529925)(430.24373779,18.41530273)
\lineto(429.88373779,18.41530273)
\curveto(429.77373054,18.42529924)(429.67373064,18.43029924)(429.58373779,18.43030273)
\lineto(425.74373779,18.43030273)
\curveto(425.66373465,18.43029924)(425.57873473,18.42529924)(425.48873779,18.41530273)
\curveto(425.4087349,18.41529925)(425.34373497,18.40029927)(425.29373779,18.37030273)
\curveto(425.24373507,18.35029932)(425.19373512,18.31029936)(425.14373779,18.25030273)
\lineto(425.05373779,18.11530273)
\curveto(425.02373529,18.0652996)(425.0137353,18.01529965)(425.02373779,17.96530273)
\curveto(425.02373529,17.91529975)(425.01873529,17.8702998)(425.00873779,17.83030273)
\lineto(425.00873779,17.71030273)
\lineto(425.00873779,17.45530273)
\curveto(425.01873529,17.37530029)(425.03373528,17.29530037)(425.05373779,17.21530273)
\curveto(425.18373513,16.67530099)(425.48873482,16.29030138)(425.96873779,16.06030273)
\curveto(426.01873429,16.03030164)(426.07873423,16.00530166)(426.14873779,15.98530273)
\curveto(426.21873409,15.9653017)(426.28373403,15.94530172)(426.34373779,15.92530273)
\curveto(426.37373394,15.91530175)(426.42373389,15.91030176)(426.49373779,15.91030273)
\curveto(426.62373369,15.8703018)(426.80373351,15.85030182)(427.03373779,15.85030273)
\curveto(427.26373305,15.85030182)(427.45373286,15.8703018)(427.60373779,15.91030273)
\curveto(427.75373256,15.95030172)(427.88873242,15.99030168)(428.00873779,16.03030273)
\curveto(428.13873217,16.08030159)(428.25873205,16.14030153)(428.36873779,16.21030273)
\curveto(428.48873182,16.28030139)(428.59873171,16.36030131)(428.69873779,16.45030273)
\curveto(428.79873151,16.55030112)(428.88873142,16.65530101)(428.96873779,16.76530273)
\curveto(429.04873126,16.8653008)(429.12373119,16.9703007)(429.19373779,17.08030273)
\curveto(429.26373105,17.19030048)(429.35873095,17.2703004)(429.47873779,17.32030273)
\curveto(429.51873079,17.34030033)(429.58373073,17.35530031)(429.67373779,17.36530273)
\curveto(429.77373054,17.37530029)(429.86373045,17.37530029)(429.94373779,17.36530273)
\curveto(430.03373028,17.3653003)(430.11873019,17.36030031)(430.19873779,17.35030273)
\curveto(430.27873003,17.34030033)(430.32872998,17.32030035)(430.34873779,17.29030273)
\curveto(430.43872987,17.22030045)(430.44372987,17.10530056)(430.36373779,16.94530273)
\curveto(430.22373009,16.67530099)(430.06873024,16.43530123)(429.89873779,16.22530273)
\curveto(429.63873067,15.90530176)(429.35873095,15.64030203)(429.05873779,15.43030273)
\curveto(428.76873154,15.23030244)(428.4137319,15.0653026)(427.99373779,14.93530273)
\curveto(427.88373243,14.89530277)(427.77873253,14.8703028)(427.67873779,14.86030273)
\curveto(427.57873273,14.84030283)(427.46873284,14.82030285)(427.34873779,14.80030273)
\curveto(427.29873301,14.79030288)(427.24873306,14.78530288)(427.19873779,14.78530273)
\curveto(427.15873315,14.78530288)(427.1137332,14.78030289)(427.06373779,14.77030273)
\lineto(426.91373779,14.77030273)
\curveto(426.86373345,14.76030291)(426.80373351,14.75530291)(426.73373779,14.75530273)
\curveto(426.67373364,14.75530291)(426.62373369,14.76030291)(426.58373779,14.77030273)
\lineto(426.44873779,14.77030273)
\curveto(426.39873391,14.78030289)(426.35373396,14.78530288)(426.31373779,14.78530273)
\curveto(426.27373404,14.78530288)(426.23373408,14.79030288)(426.19373779,14.80030273)
\curveto(426.14373417,14.81030286)(426.08873422,14.82030285)(426.02873779,14.83030273)
\curveto(425.97873433,14.83030284)(425.92873438,14.83530283)(425.87873779,14.84530273)
\curveto(425.78873452,14.8653028)(425.69873461,14.89030278)(425.60873779,14.92030273)
\curveto(425.52873478,14.94030273)(425.45373486,14.9653027)(425.38373779,14.99530273)
\curveto(425.34373497,15.01530265)(425.308735,15.02530264)(425.27873779,15.02530273)
\curveto(425.24873506,15.03530263)(425.21873509,15.05030262)(425.18873779,15.07030273)
\curveto(425.04873526,15.14030253)(424.90373541,15.22530244)(424.75373779,15.32530273)
\curveto(424.50373581,15.51530215)(424.30373601,15.74530192)(424.15373779,16.01530273)
\curveto(424.00373631,16.29530137)(423.89373642,16.60530106)(423.82373779,16.94530273)
\curveto(423.79373652,17.05530061)(423.77873653,17.1703005)(423.77873779,17.29030273)
\curveto(423.77873653,17.41030026)(423.76873654,17.53030014)(423.74873779,17.65030273)
\lineto(423.74873779,17.75530273)
\curveto(423.75873655,17.78529988)(423.76373655,17.82529984)(423.76373779,17.87530273)
\lineto(423.76373779,18.13030273)
\curveto(423.77373654,18.22029945)(423.77873653,18.31029936)(423.77873779,18.40030273)
\lineto(423.82373779,18.61030273)
\curveto(423.82373649,18.65029902)(423.82873648,18.70529896)(423.83873779,18.77530273)
\curveto(423.84873646,18.85529881)(423.86373645,18.92029875)(423.88373779,18.97030273)
\lineto(423.91373779,19.13530273)
\curveto(423.94373637,19.18529848)(423.95873635,19.23529843)(423.95873779,19.28530273)
\curveto(423.96873634,19.34529832)(423.98373633,19.40029827)(424.00373779,19.45030273)
\curveto(424.07373624,19.61029806)(424.13873617,19.7702979)(424.19873779,19.93030273)
\curveto(424.25873605,20.09029758)(424.33373598,20.24029743)(424.42373779,20.38030273)
\curveto(424.49373582,20.49029718)(424.55873575,20.60029707)(424.61873779,20.71030273)
\curveto(424.68873562,20.83029684)(424.76873554,20.94529672)(424.85873779,21.05530273)
\curveto(425.14873516,21.40529626)(425.45873485,21.70529596)(425.78873779,21.95530273)
\curveto(426.11873419,22.21529545)(426.50373381,22.43029524)(426.94373779,22.60030273)
\curveto(427.07373324,22.65029502)(427.20373311,22.68529498)(427.33373779,22.70530273)
\curveto(427.46373285,22.73529493)(427.60373271,22.7652949)(427.75373779,22.79530273)
\curveto(427.80373251,22.80529486)(427.84873246,22.81029486)(427.88873779,22.81030273)
\curveto(427.92873238,22.82029485)(427.97373234,22.82529484)(428.02373779,22.82530273)
\curveto(428.04373227,22.83529483)(428.06873224,22.83529483)(428.09873779,22.82530273)
\curveto(428.12873218,22.81529485)(428.15373216,22.82029485)(428.17373779,22.84030273)
\curveto(428.60373171,22.85029482)(428.96373135,22.80529486)(429.25373779,22.70530273)
\curveto(429.54373077,22.61529505)(429.79873051,22.49029518)(430.01873779,22.33030273)
\curveto(430.05873025,22.31029536)(430.08873022,22.28029539)(430.10873779,22.24030273)
\curveto(430.13873017,22.21029546)(430.16873014,22.18529548)(430.19873779,22.16530273)
\curveto(430.26873004,22.10529556)(430.33872997,22.03529563)(430.40873779,21.95530273)
\curveto(430.47872983,21.87529579)(430.53372978,21.79529587)(430.57373779,21.71530273)
\curveto(430.69372962,21.50529616)(430.78872952,21.30529636)(430.85873779,21.11530273)
\curveto(430.9087294,21.00529666)(430.93872937,20.88529678)(430.94873779,20.75530273)
\lineto(431.00873779,20.36530273)
\curveto(431.03872927,20.23529743)(431.04872926,20.10029757)(431.03873779,19.96030273)
\curveto(431.03872927,19.82029785)(431.04372927,19.68029799)(431.05373779,19.54030273)
\curveto(431.05372926,19.4702982)(431.04872926,19.40029827)(431.03873779,19.33030273)
\curveto(431.02872928,19.26029841)(431.02372929,19.19029848)(431.02373779,19.12030273)
\moveto(429.67373779,19.63030273)
\curveto(429.70373061,19.670298)(429.73373058,19.72029795)(429.76373779,19.78030273)
\curveto(429.80373051,19.85029782)(429.81873049,19.92029775)(429.80873779,19.99030273)
\curveto(429.79873051,20.21029746)(429.75873055,20.41529725)(429.68873779,20.60530273)
\curveto(429.58873072,20.83529683)(429.46873084,21.03029664)(429.32873779,21.19030273)
\curveto(429.19873111,21.35029632)(429.0087313,21.48529618)(428.75873779,21.59530273)
\curveto(428.68873162,21.61529605)(428.61873169,21.63029604)(428.54873779,21.64030273)
\curveto(428.48873182,21.66029601)(428.41873189,21.68029599)(428.33873779,21.70030273)
\curveto(428.26873204,21.72029595)(428.18873212,21.73029594)(428.09873779,21.73030273)
\lineto(427.84373779,21.73030273)
\curveto(427.80373251,21.71029596)(427.76373255,21.70029597)(427.72373779,21.70030273)
\curveto(427.68373263,21.71029596)(427.64873266,21.71029596)(427.61873779,21.70030273)
\lineto(427.37873779,21.64030273)
\curveto(427.29873301,21.63029604)(427.22373309,21.61529605)(427.15373779,21.59530273)
\curveto(426.83373348,21.47529619)(426.56873374,21.32529634)(426.35873779,21.14530273)
\curveto(426.14873416,20.9652967)(425.94873436,20.74029693)(425.75873779,20.47030273)
\curveto(425.71873459,20.42029725)(425.67373464,20.35529731)(425.62373779,20.27530273)
\curveto(425.58373473,20.20529746)(425.54373477,20.12529754)(425.50373779,20.03530273)
\curveto(425.46373485,19.94529772)(425.43873487,19.86029781)(425.42873779,19.78030273)
\curveto(425.42873488,19.70029797)(425.45373486,19.64029803)(425.50373779,19.60030273)
\curveto(425.57373474,19.54029813)(425.70373461,19.51029816)(425.89373779,19.51030273)
\curveto(426.09373422,19.52029815)(426.26373405,19.52529814)(426.40373779,19.52530273)
\lineto(428.68373779,19.52530273)
\curveto(428.83373148,19.52529814)(429.0137313,19.52029815)(429.22373779,19.51030273)
\curveto(429.43373088,19.51029816)(429.58373073,19.55029812)(429.67373779,19.63030273)
}
}
{
\newrgbcolor{curcolor}{0 0 0}
\pscustom[linestyle=none,fillstyle=solid,fillcolor=curcolor]
{
\newpath
\moveto(436.09537842,25.75030273)
\curveto(436.27537272,25.76029191)(436.46537253,25.76029191)(436.66537842,25.75030273)
\curveto(436.86537213,25.74029193)(436.995372,25.68029199)(437.05537842,25.57030273)
\curveto(437.08537191,25.51029216)(437.0953719,25.43529223)(437.08537842,25.34530273)
\curveto(437.07537192,25.2652924)(437.06037193,25.17529249)(437.04037842,25.07530273)
\curveto(437.02037197,24.94529272)(436.97537202,24.84029283)(436.90537842,24.76030273)
\curveto(436.85537214,24.71029296)(436.7903722,24.67529299)(436.71037842,24.65530273)
\curveto(436.63037236,24.64529302)(436.54537245,24.64029303)(436.45537842,24.64030273)
\lineto(436.18537842,24.64030273)
\curveto(436.0953729,24.65029302)(436.01037298,24.65029302)(435.93037842,24.64030273)
\curveto(435.64037335,24.56029311)(435.43537356,24.43029324)(435.31537842,24.25030273)
\curveto(435.1953738,24.08029359)(435.10037389,23.82029385)(435.03037842,23.47030273)
\curveto(435.01037398,23.40029427)(434.98537401,23.32529434)(434.95537842,23.24530273)
\curveto(434.93537406,23.17529449)(434.93037406,23.11029456)(434.94037842,23.05030273)
\curveto(434.94037405,22.90029477)(434.98537401,22.79529487)(435.07537842,22.73530273)
\curveto(435.14537385,22.70529496)(435.24037375,22.69029498)(435.36037842,22.69030273)
\lineto(435.72037842,22.69030273)
\lineto(435.94537842,22.69030273)
\curveto(435.97537302,22.670295)(436.00537299,22.665295)(436.03537842,22.67530273)
\curveto(436.06537293,22.68529498)(436.0953729,22.68029499)(436.12537842,22.66030273)
\curveto(436.21537278,22.63029504)(436.26537273,22.5702951)(436.27537842,22.48030273)
\curveto(436.2953727,22.40029527)(436.2903727,22.29529537)(436.26037842,22.16530273)
\lineto(436.23037842,22.04530273)
\lineto(436.20037842,21.92530273)
\curveto(436.14037285,21.77529589)(436.05537294,21.67529599)(435.94537842,21.62530273)
\curveto(435.80537319,21.57529609)(435.63537336,21.56029611)(435.43537842,21.58030273)
\curveto(435.23537376,21.61029606)(435.06037393,21.60529606)(434.91037842,21.56530273)
\curveto(434.83037416,21.54529612)(434.76537423,21.50529616)(434.71537842,21.44530273)
\curveto(434.66537433,21.39529627)(434.62037437,21.32529634)(434.58037842,21.23530273)
\curveto(434.55037444,21.1652965)(434.53037446,21.08529658)(434.52037842,20.99530273)
\curveto(434.51037448,20.90529676)(434.4953745,20.82029685)(434.47537842,20.74030273)
\lineto(434.28037842,19.75030273)
\lineto(433.65037842,16.57030273)
\lineto(433.50037842,15.82030273)
\curveto(433.4903755,15.76030191)(433.48037551,15.69530197)(433.47037842,15.62530273)
\curveto(433.46037553,15.55530211)(433.44037555,15.49530217)(433.41037842,15.44530273)
\lineto(433.38037842,15.32530273)
\lineto(433.32037842,15.20530273)
\curveto(433.31037568,15.1653025)(433.2903757,15.13030254)(433.26037842,15.10030273)
\curveto(433.20037579,15.03030264)(433.11537588,14.99030268)(433.00537842,14.98030273)
\curveto(432.90537609,14.9703027)(432.7953762,14.9653027)(432.67537842,14.96530273)
\lineto(432.39037842,14.96530273)
\curveto(432.35037664,14.98530268)(432.30537669,15.00030267)(432.25537842,15.01030273)
\curveto(432.21537678,15.03030264)(432.18537681,15.0653026)(432.16537842,15.11530273)
\curveto(432.15537684,15.14530252)(432.15037684,15.21030246)(432.15037842,15.31030273)
\lineto(432.16537842,15.41530273)
\curveto(432.15537684,15.4653022)(432.16037683,15.51530215)(432.18037842,15.56530273)
\curveto(432.20037679,15.62530204)(432.21537678,15.68030199)(432.22537842,15.73030273)
\lineto(432.34537842,16.33030273)
\lineto(433.15537842,20.42530273)
\curveto(433.17537582,20.53529713)(433.20037579,20.65029702)(433.23037842,20.77030273)
\curveto(433.26037573,20.89029678)(433.28037571,21.00029667)(433.29037842,21.10030273)
\curveto(433.31037568,21.21029646)(433.31037568,21.30529636)(433.29037842,21.38530273)
\curveto(433.28037571,21.4652962)(433.23537576,21.52029615)(433.15537842,21.55030273)
\curveto(433.10537589,21.58029609)(433.04037595,21.59529607)(432.96037842,21.59530273)
\lineto(432.73537842,21.59530273)
\lineto(432.49537842,21.59530273)
\curveto(432.42537657,21.59529607)(432.36037663,21.60529606)(432.30037842,21.62530273)
\curveto(432.22037677,21.665296)(432.17537682,21.75029592)(432.16537842,21.88030273)
\lineto(432.16537842,22.01530273)
\curveto(432.17537682,22.05529561)(432.18537681,22.10029557)(432.19537842,22.15030273)
\curveto(432.22537677,22.29029538)(432.26037673,22.40029527)(432.30037842,22.48030273)
\curveto(432.35037664,22.5702951)(432.43037656,22.63029504)(432.54037842,22.66030273)
\curveto(432.62037637,22.69029498)(432.70537629,22.70029497)(432.79537842,22.69030273)
\lineto(433.06537842,22.69030273)
\curveto(433.16537583,22.69029498)(433.25537574,22.70029497)(433.33537842,22.72030273)
\curveto(433.41537558,22.74029493)(433.48537551,22.78029489)(433.54537842,22.84030273)
\curveto(433.63537536,22.92029475)(433.6953753,23.04529462)(433.72537842,23.21530273)
\curveto(433.75537524,23.38529428)(433.78537521,23.54529412)(433.81537842,23.69530273)
\curveto(433.85537514,23.89529377)(433.90537509,24.08029359)(433.96537842,24.25030273)
\curveto(434.02537497,24.43029324)(434.10037489,24.59029308)(434.19037842,24.73030273)
\curveto(434.34037465,24.9702927)(434.52037447,25.1652925)(434.73037842,25.31530273)
\curveto(434.95037404,25.4652922)(435.20037379,25.58029209)(435.48037842,25.66030273)
\curveto(435.54037345,25.68029199)(435.60537339,25.69029198)(435.67537842,25.69030273)
\curveto(435.74537325,25.70029197)(435.81537318,25.71529195)(435.88537842,25.73530273)
\curveto(435.90537309,25.74529192)(435.94037305,25.74529192)(435.99037842,25.73530273)
\curveto(436.04037295,25.73529193)(436.07537292,25.74029193)(436.09537842,25.75030273)
\moveto(438.04537842,24.17530273)
\curveto(438.10537089,24.12529354)(438.18537081,24.10029357)(438.28537842,24.10030273)
\lineto(438.60037842,24.10030273)
\lineto(438.76537842,24.10030273)
\curveto(438.82537017,24.10029357)(438.88537011,24.11029356)(438.94537842,24.13030273)
\curveto(439.08536991,24.18029349)(439.17036982,24.28529338)(439.20037842,24.44530273)
\curveto(439.24036975,24.60529306)(439.28036971,24.77529289)(439.32037842,24.95530273)
\curveto(439.33036966,25.04529262)(439.34536965,25.13029254)(439.36537842,25.21030273)
\curveto(439.38536961,25.30029237)(439.38536961,25.37529229)(439.36537842,25.43530273)
\curveto(439.33536966,25.54529212)(439.24536975,25.60529206)(439.09537842,25.61530273)
\curveto(438.95537004,25.62529204)(438.80037019,25.63029204)(438.63037842,25.63030273)
\curveto(438.60037039,25.62029205)(438.57537042,25.61529205)(438.55537842,25.61530273)
\curveto(438.53537046,25.62529204)(438.51037048,25.62529204)(438.48037842,25.61530273)
\curveto(438.36037063,25.57529209)(438.27037072,25.51529215)(438.21037842,25.43530273)
\curveto(438.17037082,25.37529229)(438.14037085,25.30029237)(438.12037842,25.21030273)
\curveto(438.10037089,25.12029255)(438.08537091,25.03529263)(438.07537842,24.95530273)
\curveto(438.04537095,24.80529286)(438.01537098,24.65029302)(437.98537842,24.49030273)
\curveto(437.95537104,24.34029333)(437.97537102,24.23529343)(438.04537842,24.17530273)
\moveto(438.72037842,22.01530273)
\curveto(438.74037025,22.11529555)(438.76037023,22.21029546)(438.78037842,22.30030273)
\curveto(438.80037019,22.40029527)(438.7903702,22.48029519)(438.75037842,22.54030273)
\curveto(438.72037027,22.62029505)(438.63537036,22.66029501)(438.49537842,22.66030273)
\curveto(438.36537063,22.670295)(438.23537076,22.67529499)(438.10537842,22.67530273)
\curveto(438.08537091,22.665295)(438.06037093,22.66029501)(438.03037842,22.66030273)
\curveto(438.01037098,22.670295)(437.990371,22.67529499)(437.97037842,22.67530273)
\curveto(437.91037108,22.65529501)(437.85037114,22.64029503)(437.79037842,22.63030273)
\curveto(437.74037125,22.62029505)(437.6953713,22.59029508)(437.65537842,22.54030273)
\curveto(437.5953714,22.48029519)(437.55537144,22.39529527)(437.53537842,22.28530273)
\curveto(437.51537148,22.18529548)(437.4953715,22.08029559)(437.47537842,21.97030273)
\lineto(436.20037842,15.62530273)
\curveto(436.18037281,15.53530213)(436.16037283,15.44030223)(436.14037842,15.34030273)
\curveto(436.13037286,15.25030242)(436.13537286,15.17530249)(436.15537842,15.11530273)
\curveto(436.1953728,15.03530263)(436.26037273,14.98530268)(436.35037842,14.96530273)
\curveto(436.44037255,14.95530271)(436.55037244,14.95030272)(436.68037842,14.95030273)
\lineto(436.90537842,14.95030273)
\curveto(436.995372,14.9703027)(437.07037192,14.98530268)(437.13037842,14.99530273)
\curveto(437.1903718,15.01530265)(437.24037175,15.05530261)(437.28037842,15.11530273)
\curveto(437.35037164,15.17530249)(437.3903716,15.25530241)(437.40037842,15.35530273)
\curveto(437.42037157,15.4653022)(437.44037155,15.5703021)(437.46037842,15.67030273)
\lineto(438.72037842,22.01530273)
}
}
{
\newrgbcolor{curcolor}{0 0 0}
\pscustom[linestyle=none,fillstyle=solid,fillcolor=curcolor]
{
\newpath
\moveto(444.51905029,22.82530273)
\curveto(445.15904347,22.84529482)(445.64904298,22.76029491)(445.98905029,22.57030273)
\curveto(446.3290423,22.38029529)(446.57404206,22.09529557)(446.72405029,21.71530273)
\curveto(446.76404187,21.61529605)(446.78904184,21.50529616)(446.79905029,21.38530273)
\curveto(446.81904181,21.27529639)(446.8290418,21.16029651)(446.82905029,21.04030273)
\curveto(446.84904178,20.85029682)(446.83904179,20.64529702)(446.79905029,20.42530273)
\curveto(446.76904186,20.20529746)(446.7290419,19.98029769)(446.67905029,19.75030273)
\lineto(446.36405029,18.14530273)
\lineto(445.89905029,15.80530273)
\lineto(445.77905029,15.29530273)
\curveto(445.73904289,15.12530254)(445.64904298,15.01530265)(445.50905029,14.96530273)
\curveto(445.45904317,14.94530272)(445.40404323,14.93530273)(445.34405029,14.93530273)
\curveto(445.29404334,14.92530274)(445.23904339,14.92030275)(445.17905029,14.92030273)
\curveto(445.04904358,14.92030275)(444.92404371,14.92530274)(444.80405029,14.93530273)
\curveto(444.68404395,14.93530273)(444.60904402,14.97530269)(444.57905029,15.05530273)
\curveto(444.53904409,15.12530254)(444.5290441,15.21530245)(444.54905029,15.32530273)
\curveto(444.56904406,15.43530223)(444.59404404,15.54530212)(444.62405029,15.65530273)
\lineto(444.87905029,16.94530273)
\lineto(445.35905029,19.39030273)
\curveto(445.41904321,19.66029801)(445.46904316,19.92529774)(445.50905029,20.18530273)
\curveto(445.54904308,20.45529721)(445.54904308,20.68529698)(445.50905029,20.87530273)
\curveto(445.46904316,21.07529659)(445.37904325,21.23529643)(445.23905029,21.35530273)
\curveto(445.10904352,21.48529618)(444.94904368,21.58529608)(444.75905029,21.65530273)
\curveto(444.69904393,21.67529599)(444.634044,21.68529598)(444.56405029,21.68530273)
\curveto(444.50404413,21.69529597)(444.44904418,21.71029596)(444.39905029,21.73030273)
\curveto(444.34904428,21.74029593)(444.26904436,21.74029593)(444.15905029,21.73030273)
\curveto(444.05904457,21.73029594)(443.98404465,21.72529594)(443.93405029,21.71530273)
\curveto(443.89404474,21.69529597)(443.85904477,21.68529598)(443.82905029,21.68530273)
\curveto(443.79904483,21.69529597)(443.76404487,21.69529597)(443.72405029,21.68530273)
\curveto(443.58404505,21.65529601)(443.45404518,21.62029605)(443.33405029,21.58030273)
\curveto(443.21404542,21.55029612)(443.09904553,21.50529616)(442.98905029,21.44530273)
\curveto(442.93904569,21.42529624)(442.89904573,21.40529626)(442.86905029,21.38530273)
\curveto(442.83904579,21.3652963)(442.79904583,21.34529632)(442.74905029,21.32530273)
\curveto(442.34904628,21.07529659)(442.01904661,20.70029697)(441.75905029,20.20030273)
\curveto(441.71904691,20.12029755)(441.68404695,20.03529763)(441.65405029,19.94530273)
\lineto(441.56405029,19.70530273)
\curveto(441.5340471,19.65529801)(441.51904711,19.60529806)(441.51905029,19.55530273)
\curveto(441.51904711,19.51529815)(441.50404713,19.47529819)(441.47405029,19.43530273)
\lineto(441.41405029,19.12030273)
\curveto(441.39404724,19.09029858)(441.38404725,19.05529861)(441.38405029,19.01530273)
\curveto(441.38404725,18.97529869)(441.37904725,18.93029874)(441.36905029,18.88030273)
\lineto(441.27905029,18.43030273)
\lineto(440.97905029,16.99030273)
\lineto(440.72405029,15.67030273)
\curveto(440.70404793,15.56030211)(440.67904795,15.44530222)(440.64905029,15.32530273)
\curveto(440.629048,15.21530245)(440.58904804,15.12530254)(440.52905029,15.05530273)
\curveto(440.45904817,14.97530269)(440.35904827,14.93530273)(440.22905029,14.93530273)
\curveto(440.10904852,14.92530274)(439.98404865,14.92030275)(439.85405029,14.92030273)
\curveto(439.77404886,14.92030275)(439.69904893,14.92530274)(439.62905029,14.93530273)
\curveto(439.55904907,14.94530272)(439.50404913,14.9703027)(439.46405029,15.01030273)
\curveto(439.39404924,15.06030261)(439.37404926,15.15530251)(439.40405029,15.29530273)
\curveto(439.4340492,15.43530223)(439.45904917,15.5703021)(439.47905029,15.70030273)
\lineto(439.83905029,17.47030273)
\lineto(440.55905029,21.10030273)
\lineto(440.73905029,22.01530273)
\lineto(440.79905029,22.28530273)
\curveto(440.81904781,22.37529529)(440.85404778,22.44529522)(440.90405029,22.49530273)
\curveto(440.94404769,22.55529511)(440.99904763,22.59529507)(441.06905029,22.61530273)
\curveto(441.11904751,22.62529504)(441.17904745,22.63529503)(441.24905029,22.64530273)
\curveto(441.3290473,22.65529501)(441.40904722,22.66029501)(441.48905029,22.66030273)
\curveto(441.56904706,22.66029501)(441.64404699,22.65529501)(441.71405029,22.64530273)
\curveto(441.79404684,22.63529503)(441.84404679,22.62029505)(441.86405029,22.60030273)
\curveto(441.96404667,22.53029514)(441.99904663,22.44029523)(441.96905029,22.33030273)
\curveto(441.93904669,22.23029544)(441.9290467,22.11529555)(441.93905029,21.98530273)
\curveto(441.94904668,21.92529574)(441.97904665,21.87529579)(442.02905029,21.83530273)
\curveto(442.14904648,21.82529584)(442.25404638,21.8702958)(442.34405029,21.97030273)
\curveto(442.44404619,22.0702956)(442.53904609,22.15029552)(442.62905029,22.21030273)
\curveto(442.78904584,22.31029536)(442.94904568,22.40029527)(443.10905029,22.48030273)
\curveto(443.26904536,22.5702951)(443.45404518,22.64529502)(443.66405029,22.70530273)
\curveto(443.74404489,22.73529493)(443.8340448,22.75529491)(443.93405029,22.76530273)
\curveto(444.0340446,22.77529489)(444.1290445,22.79029488)(444.21905029,22.81030273)
\curveto(444.26904436,22.82029485)(444.31904431,22.82529484)(444.36905029,22.82530273)
\lineto(444.51905029,22.82530273)
}
}
{
\newrgbcolor{curcolor}{0 0 0}
\pscustom[linestyle=none,fillstyle=solid,fillcolor=curcolor]
{
\newpath
\moveto(449.73365967,24.17530273)
\curveto(449.66365669,24.23529343)(449.64365671,24.34029333)(449.67365967,24.49030273)
\curveto(449.70365665,24.65029302)(449.73365662,24.80529286)(449.76365967,24.95530273)
\curveto(449.77365658,25.03529263)(449.78865657,25.12029255)(449.80865967,25.21030273)
\curveto(449.82865653,25.30029237)(449.8586565,25.37529229)(449.89865967,25.43530273)
\curveto(449.9586564,25.51529215)(450.04865631,25.57529209)(450.16865967,25.61530273)
\curveto(450.19865616,25.62529204)(450.22365613,25.62529204)(450.24365967,25.61530273)
\curveto(450.26365609,25.61529205)(450.28865607,25.62029205)(450.31865967,25.63030273)
\curveto(450.48865587,25.63029204)(450.64365571,25.62529204)(450.78365967,25.61530273)
\curveto(450.93365542,25.60529206)(451.02365533,25.54529212)(451.05365967,25.43530273)
\curveto(451.07365528,25.37529229)(451.07365528,25.30029237)(451.05365967,25.21030273)
\curveto(451.03365532,25.13029254)(451.01865534,25.04529262)(451.00865967,24.95530273)
\curveto(450.96865539,24.77529289)(450.92865543,24.60529306)(450.88865967,24.44530273)
\curveto(450.8586555,24.28529338)(450.77365558,24.18029349)(450.63365967,24.13030273)
\curveto(450.57365578,24.11029356)(450.51365584,24.10029357)(450.45365967,24.10030273)
\lineto(450.28865967,24.10030273)
\lineto(449.97365967,24.10030273)
\curveto(449.87365648,24.10029357)(449.79365656,24.12529354)(449.73365967,24.17530273)
\moveto(449.14865967,15.67030273)
\curveto(449.12865723,15.5703021)(449.10865725,15.4653022)(449.08865967,15.35530273)
\curveto(449.07865728,15.25530241)(449.03865732,15.17530249)(448.96865967,15.11530273)
\curveto(448.92865743,15.05530261)(448.87865748,15.01530265)(448.81865967,14.99530273)
\curveto(448.7586576,14.98530268)(448.68365767,14.9703027)(448.59365967,14.95030273)
\lineto(448.36865967,14.95030273)
\curveto(448.23865812,14.95030272)(448.12865823,14.95530271)(448.03865967,14.96530273)
\curveto(447.94865841,14.98530268)(447.88365847,15.03530263)(447.84365967,15.11530273)
\curveto(447.82365853,15.17530249)(447.81865854,15.25030242)(447.82865967,15.34030273)
\curveto(447.84865851,15.44030223)(447.86865849,15.53530213)(447.88865967,15.62530273)
\lineto(449.16365967,21.97030273)
\curveto(449.18365717,22.08029559)(449.20365715,22.18529548)(449.22365967,22.28530273)
\curveto(449.24365711,22.39529527)(449.28365707,22.48029519)(449.34365967,22.54030273)
\curveto(449.38365697,22.59029508)(449.42865693,22.62029505)(449.47865967,22.63030273)
\curveto(449.53865682,22.64029503)(449.59865676,22.65529501)(449.65865967,22.67530273)
\curveto(449.67865668,22.67529499)(449.69865666,22.670295)(449.71865967,22.66030273)
\curveto(449.74865661,22.66029501)(449.77365658,22.665295)(449.79365967,22.67530273)
\curveto(449.92365643,22.67529499)(450.0536563,22.670295)(450.18365967,22.66030273)
\curveto(450.32365603,22.66029501)(450.40865595,22.62029505)(450.43865967,22.54030273)
\curveto(450.47865588,22.48029519)(450.48865587,22.40029527)(450.46865967,22.30030273)
\curveto(450.44865591,22.21029546)(450.42865593,22.11529555)(450.40865967,22.01530273)
\lineto(449.14865967,15.67030273)
}
}
{
\newrgbcolor{curcolor}{0 0 0}
\pscustom[linestyle=none,fillstyle=solid,fillcolor=curcolor]
{
\newpath
\moveto(458.06850342,15.76030273)
\lineto(457.97850342,15.37030273)
\curveto(457.95849549,15.25030242)(457.91849553,15.15030252)(457.85850342,15.07030273)
\curveto(457.78849566,15.00030267)(457.69349575,14.96030271)(457.57350342,14.95030273)
\lineto(457.22850342,14.95030273)
\curveto(457.16849628,14.95030272)(457.10849634,14.94530272)(457.04850342,14.93530273)
\curveto(456.99849645,14.93530273)(456.95349649,14.94530272)(456.91350342,14.96530273)
\curveto(456.83349661,14.98530268)(456.78349666,15.02530264)(456.76350342,15.08530273)
\curveto(456.73349671,15.13530253)(456.72349672,15.19530247)(456.73350342,15.26530273)
\curveto(456.7434967,15.33530233)(456.73849671,15.40530226)(456.71850342,15.47530273)
\curveto(456.71849673,15.49530217)(456.70849674,15.51030216)(456.68850342,15.52030273)
\lineto(456.65850342,15.58030273)
\curveto(456.55849689,15.59030208)(456.47349697,15.5703021)(456.40350342,15.52030273)
\curveto(456.3434971,15.4703022)(456.27849717,15.42030225)(456.20850342,15.37030273)
\curveto(455.97849747,15.22030245)(455.75349769,15.10530256)(455.53350342,15.02530273)
\curveto(455.3434981,14.94530272)(455.12349832,14.88530278)(454.87350342,14.84530273)
\curveto(454.63349881,14.80530286)(454.38849906,14.78530288)(454.13850342,14.78530273)
\curveto(453.89849955,14.77530289)(453.65849979,14.79030288)(453.41850342,14.83030273)
\curveto(453.18850026,14.86030281)(452.99350045,14.91530275)(452.83350342,14.99530273)
\curveto(452.35350109,15.21530245)(451.98850146,15.51030216)(451.73850342,15.88030273)
\curveto(451.49850195,16.26030141)(451.3435021,16.73030094)(451.27350342,17.29030273)
\curveto(451.25350219,17.38030029)(451.2435022,17.4703002)(451.24350342,17.56030273)
\curveto(451.25350219,17.66030001)(451.25350219,17.76029991)(451.24350342,17.86030273)
\curveto(451.2435022,17.91029976)(451.2485022,17.96029971)(451.25850342,18.01030273)
\curveto(451.26850218,18.06029961)(451.27350217,18.11029956)(451.27350342,18.16030273)
\curveto(451.26350218,18.21029946)(451.26350218,18.26029941)(451.27350342,18.31030273)
\curveto(451.29350215,18.3702993)(451.30350214,18.42529924)(451.30350342,18.47530273)
\lineto(451.33350342,18.62530273)
\curveto(451.32350212,18.67529899)(451.32350212,18.74029893)(451.33350342,18.82030273)
\curveto(451.35350209,18.90029877)(451.37850207,18.9652987)(451.40850342,19.01530273)
\lineto(451.45350342,19.18030273)
\curveto(451.48350196,19.25029842)(451.50350194,19.32029835)(451.51350342,19.39030273)
\curveto(451.52350192,19.4702982)(451.5435019,19.54529812)(451.57350342,19.61530273)
\curveto(451.59350185,19.665298)(451.60850184,19.71029796)(451.61850342,19.75030273)
\curveto(451.62850182,19.79029788)(451.6435018,19.83529783)(451.66350342,19.88530273)
\curveto(451.71350173,19.98529768)(451.75850169,20.08029759)(451.79850342,20.17030273)
\curveto(451.83850161,20.2702974)(451.88350156,20.3652973)(451.93350342,20.45530273)
\curveto(452.13350131,20.83529683)(452.36350108,21.17529649)(452.62350342,21.47530273)
\curveto(452.89350055,21.78529588)(453.19350025,22.04029563)(453.52350342,22.24030273)
\curveto(453.72349972,22.36029531)(453.92349952,22.46029521)(454.12350342,22.54030273)
\curveto(454.32349912,22.62029505)(454.53849891,22.69029498)(454.76850342,22.75030273)
\lineto(454.97850342,22.78030273)
\curveto(455.0484984,22.79029488)(455.11849833,22.80529486)(455.18850342,22.82530273)
\lineto(455.33850342,22.82530273)
\curveto(455.42849802,22.84529482)(455.5484979,22.85529481)(455.69850342,22.85530273)
\curveto(455.85849759,22.85529481)(455.97349747,22.84529482)(456.04350342,22.82530273)
\curveto(456.08349736,22.81529485)(456.13849731,22.81029486)(456.20850342,22.81030273)
\curveto(456.30849714,22.78029489)(456.41349703,22.75529491)(456.52350342,22.73530273)
\curveto(456.63349681,22.72529494)(456.73349671,22.69529497)(456.82350342,22.64530273)
\curveto(456.96349648,22.58529508)(457.09349635,22.52029515)(457.21350342,22.45030273)
\curveto(457.33349611,22.38029529)(457.443496,22.30029537)(457.54350342,22.21030273)
\curveto(457.59349585,22.16029551)(457.6434958,22.10529556)(457.69350342,22.04530273)
\curveto(457.75349569,21.99529567)(457.83849561,21.98029569)(457.94850342,22.00030273)
\lineto(458.02350342,22.07530273)
\curveto(458.0434954,22.09529557)(458.05849539,22.12529554)(458.06850342,22.16530273)
\curveto(458.11849533,22.25529541)(458.15349529,22.3702953)(458.17350342,22.51030273)
\curveto(458.20349524,22.65029502)(458.22849522,22.77529489)(458.24850342,22.88530273)
\lineto(458.59350342,24.61030273)
\curveto(458.62349482,24.75029292)(458.65349479,24.90529276)(458.68350342,25.07530273)
\curveto(458.72349472,25.25529241)(458.77349467,25.38529228)(458.83350342,25.46530273)
\curveto(458.89349455,25.53529213)(458.96349448,25.58029209)(459.04350342,25.60030273)
\curveto(459.06349438,25.60029207)(459.08849436,25.60029207)(459.11850342,25.60030273)
\curveto(459.1484943,25.61029206)(459.17349427,25.61529205)(459.19350342,25.61530273)
\curveto(459.3434941,25.62529204)(459.49349395,25.62529204)(459.64350342,25.61530273)
\curveto(459.79349365,25.61529205)(459.89349355,25.57529209)(459.94350342,25.49530273)
\curveto(459.97349347,25.41529225)(459.97349347,25.31529235)(459.94350342,25.19530273)
\curveto(459.92349352,25.07529259)(459.90349354,24.95029272)(459.88350342,24.82030273)
\lineto(458.06850342,15.76030273)
\moveto(457.42350342,18.59530273)
\curveto(457.45349599,18.64529902)(457.47349597,18.71029896)(457.48350342,18.79030273)
\curveto(457.50349594,18.88029879)(457.50849594,18.95029872)(457.49850342,19.00030273)
\lineto(457.54350342,19.22530273)
\curveto(457.5434959,19.31529835)(457.5484959,19.40529826)(457.55850342,19.49530273)
\curveto(457.56849588,19.59529807)(457.56349588,19.68529798)(457.54350342,19.76530273)
\lineto(457.54350342,19.99030273)
\curveto(457.5434959,20.06029761)(457.53349591,20.13029754)(457.51350342,20.20030273)
\curveto(457.45349599,20.50029717)(457.3484961,20.7652969)(457.19850342,20.99530273)
\curveto(457.05849639,21.22529644)(456.85849659,21.40529626)(456.59850342,21.53530273)
\curveto(456.50849694,21.58529608)(456.41349703,21.62029605)(456.31350342,21.64030273)
\curveto(456.21349723,21.670296)(456.10349734,21.69529597)(455.98350342,21.71530273)
\curveto(455.91349753,21.73529593)(455.82849762,21.74529592)(455.72850342,21.74530273)
\lineto(455.45850342,21.74530273)
\lineto(455.30850342,21.71530273)
\lineto(455.17350342,21.71530273)
\curveto(455.09349835,21.69529597)(455.00849844,21.67529599)(454.91850342,21.65530273)
\curveto(454.82849862,21.63529603)(454.7434987,21.61029606)(454.66350342,21.58030273)
\curveto(454.31349913,21.44029623)(454.01349943,21.23529643)(453.76350342,20.96530273)
\curveto(453.51349993,20.70529696)(453.29350015,20.40029727)(453.10350342,20.05030273)
\curveto(453.0435004,19.94029773)(452.99350045,19.82529784)(452.95350342,19.70530273)
\lineto(452.83350342,19.37530273)
\lineto(452.80350342,19.25530273)
\curveto(452.79350065,19.22529844)(452.78350066,19.19029848)(452.77350342,19.15030273)
\curveto(452.7435007,19.10029857)(452.72350072,19.04529862)(452.71350342,18.98530273)
\curveto(452.71350073,18.92529874)(452.70850074,18.8702988)(452.69850342,18.82030273)
\curveto(452.67850077,18.71029896)(452.65350079,18.60029907)(452.62350342,18.49030273)
\curveto(452.60350084,18.39029928)(452.59850085,18.29529937)(452.60850342,18.20530273)
\curveto(452.60850084,18.17529949)(452.60350084,18.12529954)(452.59350342,18.05530273)
\lineto(452.59350342,17.84530273)
\curveto(452.59350085,17.77529989)(452.59850085,17.70529996)(452.60850342,17.63530273)
\curveto(452.6485008,17.28530038)(452.73850071,16.98530068)(452.87850342,16.73530273)
\curveto(453.01850043,16.48530118)(453.21850023,16.28030139)(453.47850342,16.12030273)
\curveto(453.55849989,16.0703016)(453.63849981,16.03030164)(453.71850342,16.00030273)
\curveto(453.80849964,15.9703017)(453.90349954,15.94030173)(454.00350342,15.91030273)
\curveto(454.05349939,15.89030178)(454.10349934,15.88530178)(454.15350342,15.89530273)
\curveto(454.21349923,15.90530176)(454.26849918,15.90030177)(454.31850342,15.88030273)
\curveto(454.3484991,15.8703018)(454.38349906,15.8653018)(454.42350342,15.86530273)
\lineto(454.55850342,15.86530273)
\lineto(454.69350342,15.86530273)
\curveto(454.73349871,15.87530179)(454.78849866,15.88030179)(454.85850342,15.88030273)
\curveto(454.93849851,15.90030177)(455.01849843,15.91530175)(455.09850342,15.92530273)
\curveto(455.18849826,15.94530172)(455.26849818,15.9703017)(455.33850342,16.00030273)
\curveto(455.69849775,16.14030153)(456.00349744,16.31530135)(456.25350342,16.52530273)
\curveto(456.50349694,16.74530092)(456.72849672,17.02030065)(456.92850342,17.35030273)
\curveto(456.99849645,17.46030021)(457.05349639,17.5703001)(457.09350342,17.68030273)
\lineto(457.24350342,18.01030273)
\curveto(457.27349617,18.05029962)(457.28849616,18.08529958)(457.28850342,18.11530273)
\curveto(457.29849615,18.15529951)(457.31349613,18.19529947)(457.33350342,18.23530273)
\curveto(457.35349609,18.29529937)(457.36849608,18.35529931)(457.37850342,18.41530273)
\curveto(457.38849606,18.47529919)(457.40349604,18.53529913)(457.42350342,18.59530273)
}
}
{
\newrgbcolor{curcolor}{0 0 0}
\pscustom[linestyle=none,fillstyle=solid,fillcolor=curcolor]
{
\newpath
\moveto(467.81475342,19.15030273)
\curveto(467.82474453,19.09029858)(467.81474454,18.99529867)(467.78475342,18.86530273)
\curveto(467.76474459,18.74529892)(467.74474461,18.66029901)(467.72475342,18.61030273)
\lineto(467.69475342,18.46030273)
\curveto(467.66474469,18.38029929)(467.63974471,18.30529936)(467.61975342,18.23530273)
\curveto(467.60974474,18.17529949)(467.58974476,18.10529956)(467.55975342,18.02530273)
\curveto(467.52974482,17.9652997)(467.50474485,17.90529976)(467.48475342,17.84530273)
\curveto(467.47474488,17.78529988)(467.4497449,17.72529994)(467.40975342,17.66530273)
\lineto(467.22975342,17.27530273)
\curveto(467.17974517,17.14530052)(467.11474524,17.02530064)(467.03475342,16.91530273)
\curveto(466.73474562,16.43530123)(466.37474598,16.03030164)(465.95475342,15.70030273)
\curveto(465.54474681,15.38030229)(465.06474729,15.13530253)(464.51475342,14.96530273)
\curveto(464.40474795,14.92530274)(464.28474807,14.89530277)(464.15475342,14.87530273)
\curveto(464.02474833,14.85530281)(463.88974846,14.83530283)(463.74975342,14.81530273)
\curveto(463.68974866,14.80530286)(463.62474873,14.80030287)(463.55475342,14.80030273)
\curveto(463.49474886,14.79030288)(463.43474892,14.78530288)(463.37475342,14.78530273)
\curveto(463.33474902,14.77530289)(463.27474908,14.7703029)(463.19475342,14.77030273)
\curveto(463.12474923,14.7703029)(463.07474928,14.77530289)(463.04475342,14.78530273)
\curveto(463.00474935,14.79530287)(462.96474939,14.80030287)(462.92475342,14.80030273)
\curveto(462.88474947,14.79030288)(462.8497495,14.79030288)(462.81975342,14.80030273)
\lineto(462.72975342,14.80030273)
\lineto(462.38475342,14.84530273)
\lineto(461.99475342,14.96530273)
\curveto(461.87475048,15.00530266)(461.75975059,15.05030262)(461.64975342,15.10030273)
\curveto(461.23975111,15.30030237)(460.91975143,15.56030211)(460.68975342,15.88030273)
\curveto(460.46975188,16.20030147)(460.30975204,16.59030108)(460.20975342,17.05030273)
\curveto(460.17975217,17.15030052)(460.15975219,17.25030042)(460.14975342,17.35030273)
\lineto(460.14975342,17.66530273)
\curveto(460.13975221,17.70529996)(460.13975221,17.73529993)(460.14975342,17.75530273)
\curveto(460.15975219,17.78529988)(460.16475219,17.82029985)(460.16475342,17.86030273)
\curveto(460.16475219,17.94029973)(460.16975218,18.02029965)(460.17975342,18.10030273)
\curveto(460.18975216,18.19029948)(460.19475216,18.27529939)(460.19475342,18.35530273)
\curveto(460.20475215,18.40529926)(460.20975214,18.44529922)(460.20975342,18.47530273)
\curveto(460.21975213,18.51529915)(460.22475213,18.56029911)(460.22475342,18.61030273)
\curveto(460.22475213,18.66029901)(460.23475212,18.74529892)(460.25475342,18.86530273)
\curveto(460.28475207,18.99529867)(460.31475204,19.09029858)(460.34475342,19.15030273)
\curveto(460.38475197,19.22029845)(460.40475195,19.29029838)(460.40475342,19.36030273)
\curveto(460.40475195,19.43029824)(460.42475193,19.50029817)(460.46475342,19.57030273)
\curveto(460.48475187,19.62029805)(460.49975185,19.66029801)(460.50975342,19.69030273)
\curveto(460.51975183,19.73029794)(460.53475182,19.77529789)(460.55475342,19.82530273)
\curveto(460.61475174,19.94529772)(460.66475169,20.0652976)(460.70475342,20.18530273)
\curveto(460.7547516,20.30529736)(460.81975153,20.42029725)(460.89975342,20.53030273)
\curveto(461.11975123,20.90029677)(461.36475099,21.23029644)(461.63475342,21.52030273)
\curveto(461.91475044,21.82029585)(462.22975012,22.0702956)(462.57975342,22.27030273)
\curveto(462.70974964,22.35029532)(462.84474951,22.41529525)(462.98475342,22.46530273)
\lineto(463.43475342,22.64530273)
\curveto(463.56474879,22.69529497)(463.69974865,22.72529494)(463.83975342,22.73530273)
\curveto(463.97974837,22.75529491)(464.12474823,22.78529488)(464.27475342,22.82530273)
\lineto(464.46975342,22.82530273)
\lineto(464.67975342,22.85530273)
\curveto(465.56974678,22.8652948)(466.26974608,22.68029499)(466.77975342,22.30030273)
\curveto(467.29974505,21.92029575)(467.62474473,21.42529624)(467.75475342,20.81530273)
\curveto(467.78474457,20.71529695)(467.80474455,20.61529705)(467.81475342,20.51530273)
\curveto(467.82474453,20.41529725)(467.83974451,20.31029736)(467.85975342,20.20030273)
\curveto(467.86974448,20.09029758)(467.86974448,19.9702977)(467.85975342,19.84030273)
\lineto(467.85975342,19.46530273)
\curveto(467.85974449,19.41529825)(467.8497445,19.36029831)(467.82975342,19.30030273)
\curveto(467.81974453,19.25029842)(467.81474454,19.20029847)(467.81475342,19.15030273)
\moveto(466.31475342,18.29530273)
\curveto(466.34474601,18.3652993)(466.36474599,18.44529922)(466.37475342,18.53530273)
\curveto(466.39474596,18.62529904)(466.40974594,18.71029896)(466.41975342,18.79030273)
\curveto(466.49974585,19.18029849)(466.53474582,19.51029816)(466.52475342,19.78030273)
\curveto(466.50474585,19.86029781)(466.48974586,19.94029773)(466.47975342,20.02030273)
\curveto(466.47974587,20.10029757)(466.47474588,20.17529749)(466.46475342,20.24530273)
\curveto(466.31474604,20.89529677)(465.95974639,21.34529632)(465.39975342,21.59530273)
\curveto(465.32974702,21.62529604)(465.2547471,21.64529602)(465.17475342,21.65530273)
\curveto(465.10474725,21.67529599)(465.02974732,21.69529597)(464.94975342,21.71530273)
\curveto(464.87974747,21.73529593)(464.79974755,21.74529592)(464.70975342,21.74530273)
\lineto(464.43975342,21.74530273)
\lineto(464.15475342,21.70030273)
\curveto(464.0547483,21.68029599)(463.95974839,21.65529601)(463.86975342,21.62530273)
\curveto(463.77974857,21.60529606)(463.68974866,21.57529609)(463.59975342,21.53530273)
\curveto(463.52974882,21.51529615)(463.45974889,21.48529618)(463.38975342,21.44530273)
\curveto(463.31974903,21.40529626)(463.2547491,21.3652963)(463.19475342,21.32530273)
\curveto(462.92474943,21.15529651)(462.68974966,20.95029672)(462.48975342,20.71030273)
\curveto(462.28975006,20.4702972)(462.10475025,20.19029748)(461.93475342,19.87030273)
\curveto(461.88475047,19.7702979)(461.84475051,19.665298)(461.81475342,19.55530273)
\curveto(461.78475057,19.45529821)(461.74475061,19.35029832)(461.69475342,19.24030273)
\curveto(461.68475067,19.20029847)(461.66975068,19.13529853)(461.64975342,19.04530273)
\curveto(461.62975072,19.01529865)(461.61975073,18.98029869)(461.61975342,18.94030273)
\curveto(461.61975073,18.90029877)(461.61475074,18.85529881)(461.60475342,18.80530273)
\lineto(461.54475342,18.50530273)
\curveto(461.52475083,18.40529926)(461.51475084,18.31529935)(461.51475342,18.23530273)
\lineto(461.51475342,18.05530273)
\curveto(461.51475084,17.95529971)(461.50975084,17.85529981)(461.49975342,17.75530273)
\curveto(461.49975085,17.6653)(461.50975084,17.58030009)(461.52975342,17.50030273)
\curveto(461.57975077,17.26030041)(461.6497507,17.03530063)(461.73975342,16.82530273)
\curveto(461.83975051,16.61530105)(461.97475038,16.44030123)(462.14475342,16.30030273)
\curveto(462.19475016,16.2703014)(462.23475012,16.24530142)(462.26475342,16.22530273)
\curveto(462.30475005,16.20530146)(462.34475001,16.18030149)(462.38475342,16.15030273)
\curveto(462.4547499,16.10030157)(462.53474982,16.05530161)(462.62475342,16.01530273)
\curveto(462.71474964,15.98530168)(462.80974954,15.95530171)(462.90975342,15.92530273)
\curveto(462.95974939,15.90530176)(463.00474935,15.89530177)(463.04475342,15.89530273)
\curveto(463.09474926,15.90530176)(463.14474921,15.90530176)(463.19475342,15.89530273)
\curveto(463.22474913,15.88530178)(463.28474907,15.87530179)(463.37475342,15.86530273)
\curveto(463.46474889,15.85530181)(463.53974881,15.86030181)(463.59975342,15.88030273)
\curveto(463.63974871,15.89030178)(463.67974867,15.89030178)(463.71975342,15.88030273)
\curveto(463.75974859,15.88030179)(463.79974855,15.89030178)(463.83975342,15.91030273)
\curveto(463.91974843,15.93030174)(463.99974835,15.94530172)(464.07975342,15.95530273)
\curveto(464.16974818,15.97530169)(464.2547481,16.00030167)(464.33475342,16.03030273)
\curveto(464.69474766,16.1703015)(465.00474735,16.3653013)(465.26475342,16.61530273)
\curveto(465.52474683,16.8653008)(465.75974659,17.16030051)(465.96975342,17.50030273)
\curveto(466.0497463,17.62030005)(466.10974624,17.74529992)(466.14975342,17.87530273)
\curveto(466.18974616,18.01529965)(466.24474611,18.15529951)(466.31475342,18.29530273)
}
}
{
\newrgbcolor{curcolor}{0 0 0}
\pscustom[linestyle=none,fillstyle=solid,fillcolor=curcolor]
{
\newpath
\moveto(472.48303467,22.85530273)
\curveto(473.20302901,22.8652948)(473.78802843,22.78029489)(474.23803467,22.60030273)
\curveto(474.69802752,22.43029524)(475.0180272,22.12529554)(475.19803467,21.68530273)
\curveto(475.24802697,21.57529609)(475.27802694,21.46029621)(475.28803467,21.34030273)
\curveto(475.30802691,21.23029644)(475.32302689,21.10529656)(475.33303467,20.96530273)
\curveto(475.34302687,20.89529677)(475.33302688,20.82029685)(475.30303467,20.74030273)
\curveto(475.28302693,20.670297)(475.25802696,20.61529705)(475.22803467,20.57530273)
\curveto(475.20802701,20.55529711)(475.17802704,20.53529713)(475.13803467,20.51530273)
\curveto(475.10802711,20.50529716)(475.08302713,20.49029718)(475.06303467,20.47030273)
\curveto(475.00302721,20.45029722)(474.94802727,20.44529722)(474.89803467,20.45530273)
\curveto(474.85802736,20.4652972)(474.8130274,20.4652972)(474.76303467,20.45530273)
\curveto(474.67302754,20.43529723)(474.56302765,20.43029724)(474.43303467,20.44030273)
\curveto(474.3130279,20.46029721)(474.22802799,20.48529718)(474.17803467,20.51530273)
\curveto(474.10802811,20.5652971)(474.06802815,20.63029704)(474.05803467,20.71030273)
\curveto(474.05802816,20.80029687)(474.03802818,20.88529678)(473.99803467,20.96530273)
\curveto(473.94802827,21.12529654)(473.85302836,21.2702964)(473.71303467,21.40030273)
\curveto(473.62302859,21.48029619)(473.5130287,21.54029613)(473.38303467,21.58030273)
\curveto(473.26302895,21.62029605)(473.13302908,21.66029601)(472.99303467,21.70030273)
\curveto(472.95302926,21.72029595)(472.90302931,21.72529594)(472.84303467,21.71530273)
\curveto(472.79302942,21.71529595)(472.74802947,21.72029595)(472.70803467,21.73030273)
\curveto(472.64802957,21.75029592)(472.57302964,21.76029591)(472.48303467,21.76030273)
\curveto(472.39302982,21.76029591)(472.3180299,21.75029592)(472.25803467,21.73030273)
\lineto(472.16803467,21.73030273)
\curveto(472.10803011,21.72029595)(472.05303016,21.71029596)(472.00303467,21.70030273)
\curveto(471.95303026,21.70029597)(471.90303031,21.69529597)(471.85303467,21.68530273)
\curveto(471.58303063,21.62529604)(471.34803087,21.54029613)(471.14803467,21.43030273)
\curveto(470.95803126,21.32029635)(470.80803141,21.13529653)(470.69803467,20.87530273)
\curveto(470.66803155,20.80529686)(470.65303156,20.73529693)(470.65303467,20.66530273)
\curveto(470.65303156,20.59529707)(470.65803156,20.53529713)(470.66803467,20.48530273)
\curveto(470.69803152,20.33529733)(470.74803147,20.22529744)(470.81803467,20.15530273)
\curveto(470.88803133,20.09529757)(470.98303123,20.02529764)(471.10303467,19.94530273)
\curveto(471.24303097,19.84529782)(471.40803081,19.7702979)(471.59803467,19.72030273)
\curveto(471.78803043,19.68029799)(471.97803024,19.63029804)(472.16803467,19.57030273)
\curveto(472.28802993,19.53029814)(472.40802981,19.50029817)(472.52803467,19.48030273)
\curveto(472.65802956,19.46029821)(472.78302943,19.43029824)(472.90303467,19.39030273)
\curveto(473.10302911,19.33029834)(473.29802892,19.2702984)(473.48803467,19.21030273)
\curveto(473.67802854,19.16029851)(473.86302835,19.09529857)(474.04303467,19.01530273)
\curveto(474.09302812,18.99529867)(474.13802808,18.97529869)(474.17803467,18.95530273)
\curveto(474.22802799,18.93529873)(474.27802794,18.91029876)(474.32803467,18.88030273)
\curveto(474.49802772,18.76029891)(474.64302757,18.62529904)(474.76303467,18.47530273)
\curveto(474.88302733,18.32529934)(474.97302724,18.13529953)(475.03303467,17.90530273)
\lineto(475.03303467,17.62030273)
\curveto(475.03302718,17.55030012)(475.02802719,17.47530019)(475.01803467,17.39530273)
\curveto(475.00802721,17.32530034)(474.99802722,17.24530042)(474.98803467,17.15530273)
\lineto(474.95803467,17.00530273)
\curveto(474.9180273,16.93530073)(474.88802733,16.8653008)(474.86803467,16.79530273)
\curveto(474.85802736,16.72530094)(474.83802738,16.65530101)(474.80803467,16.58530273)
\curveto(474.75802746,16.47530119)(474.70302751,16.3703013)(474.64303467,16.27030273)
\curveto(474.58302763,16.1703015)(474.5180277,16.08030159)(474.44803467,16.00030273)
\curveto(474.23802798,15.74030193)(473.99302822,15.53030214)(473.71303467,15.37030273)
\curveto(473.43302878,15.22030245)(473.12802909,15.09030258)(472.79803467,14.98030273)
\curveto(472.69802952,14.95030272)(472.59802962,14.93030274)(472.49803467,14.92030273)
\curveto(472.39802982,14.90030277)(472.30302991,14.87530279)(472.21303467,14.84530273)
\curveto(472.10303011,14.82530284)(471.99803022,14.81530285)(471.89803467,14.81530273)
\curveto(471.79803042,14.81530285)(471.69803052,14.80530286)(471.59803467,14.78530273)
\lineto(471.44803467,14.78530273)
\curveto(471.39803082,14.77530289)(471.32803089,14.7703029)(471.23803467,14.77030273)
\curveto(471.14803107,14.7703029)(471.07803114,14.77530289)(471.02803467,14.78530273)
\lineto(470.86303467,14.78530273)
\curveto(470.80303141,14.80530286)(470.73803148,14.81530285)(470.66803467,14.81530273)
\curveto(470.59803162,14.80530286)(470.54303167,14.81030286)(470.50303467,14.83030273)
\curveto(470.45303176,14.84030283)(470.38803183,14.84530282)(470.30803467,14.84530273)
\curveto(470.22803199,14.8653028)(470.15303206,14.88530278)(470.08303467,14.90530273)
\curveto(470.0130322,14.91530275)(469.93803228,14.93530273)(469.85803467,14.96530273)
\curveto(469.56803265,15.0653026)(469.32303289,15.19030248)(469.12303467,15.34030273)
\curveto(468.92303329,15.49030218)(468.76303345,15.68530198)(468.64303467,15.92530273)
\curveto(468.58303363,16.05530161)(468.53303368,16.19030148)(468.49303467,16.33030273)
\curveto(468.46303375,16.4703012)(468.44303377,16.62530104)(468.43303467,16.79530273)
\curveto(468.42303379,16.85530081)(468.42803379,16.92530074)(468.44803467,17.00530273)
\curveto(468.46803375,17.09530057)(468.49303372,17.1653005)(468.52303467,17.21530273)
\curveto(468.56303365,17.25530041)(468.62303359,17.29530037)(468.70303467,17.33530273)
\curveto(468.75303346,17.35530031)(468.82303339,17.3653003)(468.91303467,17.36530273)
\curveto(469.0130332,17.37530029)(469.10303311,17.37530029)(469.18303467,17.36530273)
\curveto(469.27303294,17.35530031)(469.35803286,17.34030033)(469.43803467,17.32030273)
\curveto(469.52803269,17.31030036)(469.58303263,17.29530037)(469.60303467,17.27530273)
\curveto(469.66303255,17.22530044)(469.69303252,17.15030052)(469.69303467,17.05030273)
\curveto(469.70303251,16.96030071)(469.72303249,16.87530079)(469.75303467,16.79530273)
\curveto(469.80303241,16.57530109)(469.90303231,16.40530126)(470.05303467,16.28530273)
\curveto(470.15303206,16.19530147)(470.27303194,16.12530154)(470.41303467,16.07530273)
\curveto(470.55303166,16.02530164)(470.70303151,15.97530169)(470.86303467,15.92530273)
\lineto(471.17803467,15.88030273)
\lineto(471.26803467,15.88030273)
\curveto(471.32803089,15.86030181)(471.4130308,15.85030182)(471.52303467,15.85030273)
\curveto(471.64303057,15.85030182)(471.74803047,15.86030181)(471.83803467,15.88030273)
\curveto(471.90803031,15.88030179)(471.96303025,15.88530178)(472.00303467,15.89530273)
\curveto(472.06303015,15.90530176)(472.12303009,15.91030176)(472.18303467,15.91030273)
\curveto(472.24302997,15.92030175)(472.29802992,15.93030174)(472.34803467,15.94030273)
\curveto(472.65802956,16.02030165)(472.90802931,16.12530154)(473.09803467,16.25530273)
\curveto(473.29802892,16.38530128)(473.46302875,16.60530106)(473.59303467,16.91530273)
\curveto(473.62302859,16.9653007)(473.63802858,17.02030065)(473.63803467,17.08030273)
\curveto(473.64802857,17.14030053)(473.64802857,17.18530048)(473.63803467,17.21530273)
\curveto(473.62802859,17.40530026)(473.58802863,17.54530012)(473.51803467,17.63530273)
\curveto(473.44802877,17.73529993)(473.35302886,17.82529984)(473.23303467,17.90530273)
\curveto(473.15302906,17.9652997)(473.05802916,18.01529965)(472.94803467,18.05530273)
\lineto(472.64803467,18.17530273)
\curveto(472.6180296,18.18529948)(472.58802963,18.19029948)(472.55803467,18.19030273)
\curveto(472.53802968,18.19029948)(472.5180297,18.20029947)(472.49803467,18.22030273)
\curveto(472.17803004,18.33029934)(471.83803038,18.41029926)(471.47803467,18.46030273)
\curveto(471.12803109,18.52029915)(470.80803141,18.61529905)(470.51803467,18.74530273)
\curveto(470.42803179,18.78529888)(470.33803188,18.82029885)(470.24803467,18.85030273)
\curveto(470.16803205,18.88029879)(470.09303212,18.92029875)(470.02303467,18.97030273)
\curveto(469.85303236,19.08029859)(469.70303251,19.20529846)(469.57303467,19.34530273)
\curveto(469.44303277,19.48529818)(469.35303286,19.66029801)(469.30303467,19.87030273)
\curveto(469.28303293,19.94029773)(469.27303294,20.01029766)(469.27303467,20.08030273)
\lineto(469.27303467,20.30530273)
\curveto(469.26303295,20.42529724)(469.27803294,20.56029711)(469.31803467,20.71030273)
\curveto(469.35803286,20.8702968)(469.39803282,21.00529666)(469.43803467,21.11530273)
\curveto(469.46803275,21.1652965)(469.48803273,21.20529646)(469.49803467,21.23530273)
\curveto(469.5180327,21.27529639)(469.54303267,21.31529635)(469.57303467,21.35530273)
\curveto(469.70303251,21.58529608)(469.86303235,21.78529588)(470.05303467,21.95530273)
\curveto(470.24303197,22.12529554)(470.45303176,22.27529539)(470.68303467,22.40530273)
\curveto(470.84303137,22.49529517)(471.0180312,22.5652951)(471.20803467,22.61530273)
\curveto(471.40803081,22.67529499)(471.6130306,22.73029494)(471.82303467,22.78030273)
\curveto(471.89303032,22.79029488)(471.95803026,22.80029487)(472.01803467,22.81030273)
\curveto(472.08803013,22.82029485)(472.16303005,22.83029484)(472.24303467,22.84030273)
\curveto(472.28302993,22.85029482)(472.32302989,22.85029482)(472.36303467,22.84030273)
\curveto(472.4130298,22.83029484)(472.45302976,22.83529483)(472.48303467,22.85530273)
}
}
{
\newrgbcolor{curcolor}{0 0 0}
\pscustom[linewidth=1,linecolor=curcolor]
{
\newpath
\moveto(89.01786,85.52252)
\lineto(726.99776,85.52252)
}
}
{
\newrgbcolor{curcolor}{0 0 0}
\pscustom[linewidth=1,linecolor=curcolor]
{
\newpath
\moveto(89.01786,159.51623)
\lineto(726.99776,159.51623)
}
}
{
\newrgbcolor{curcolor}{0 0 0}
\pscustom[linewidth=1,linecolor=curcolor]
{
\newpath
\moveto(89.01786,233.55822)
\lineto(726.99776,233.55822)
}
}
{
\newrgbcolor{curcolor}{0 0 0}
\pscustom[linewidth=1,linecolor=curcolor]
{
\newpath
\moveto(89.01786,308.62335)
\lineto(726.99776,308.62335)
}
}
{
\newrgbcolor{curcolor}{0 0 0}
\pscustom[linewidth=1,linecolor=curcolor]
{
\newpath
\moveto(89.01786,382.589017)
\lineto(726.99776,382.589017)
}
}
{
\newrgbcolor{curcolor}{0 0 0}
\pscustom[linestyle=none,fillstyle=solid,fillcolor=curcolor]
{
\newpath
\moveto(94.64285278,419.08210297)
\curveto(95.69284611,419.102092)(96.55784524,418.92209218)(97.23785278,418.54210297)
\curveto(97.91784388,418.16209294)(98.45784334,417.65709344)(98.85785278,417.02710297)
\curveto(98.96784283,416.85709424)(99.05784274,416.68209442)(99.12785278,416.50210297)
\curveto(99.1978426,416.33209477)(99.26284254,416.14209496)(99.32285278,415.93210297)
\curveto(99.34284246,415.86209524)(99.36284244,415.78209532)(99.38285278,415.69210297)
\curveto(99.4028424,415.6020955)(99.3978424,415.51709558)(99.36785278,415.43710297)
\curveto(99.34784245,415.37709572)(99.30784249,415.33709576)(99.24785278,415.31710297)
\curveto(99.1978426,415.30709579)(99.13784266,415.29209581)(99.06785278,415.27210297)
\lineto(98.94785278,415.27210297)
\curveto(98.88784291,415.25209585)(98.81784298,415.24209586)(98.73785278,415.24210297)
\curveto(98.66784313,415.25209585)(98.5978432,415.25709584)(98.52785278,415.25710297)
\curveto(98.43784336,415.25709584)(98.32784347,415.25209585)(98.19785278,415.24210297)
\lineto(97.83785278,415.24210297)
\curveto(97.71784408,415.25209585)(97.60784419,415.26209584)(97.50785278,415.27210297)
\curveto(97.40784439,415.29209581)(97.33284447,415.31709578)(97.28285278,415.34710297)
\curveto(97.2028446,415.41709568)(97.14284466,415.51209559)(97.10285278,415.63210297)
\curveto(97.06284474,415.75209535)(97.01284479,415.85709524)(96.95285278,415.94710297)
\curveto(96.76284504,416.27709482)(96.51284529,416.53709456)(96.20285278,416.72710297)
\curveto(95.99284581,416.85709424)(95.75784604,416.96209414)(95.49785278,417.04210297)
\curveto(95.32784647,417.102094)(95.11284669,417.13209397)(94.85285278,417.13210297)
\curveto(94.6028472,417.13209397)(94.38284742,417.10709399)(94.19285278,417.05710297)
\curveto(94.11284769,417.03709406)(94.03784776,417.01709408)(93.96785278,416.99710297)
\curveto(93.90784789,416.98709411)(93.84284796,416.96709413)(93.77285278,416.93710297)
\curveto(93.09284871,416.64709445)(92.61284919,416.16709493)(92.33285278,415.49710297)
\curveto(92.28284952,415.37709572)(92.23784956,415.25209585)(92.19785278,415.12210297)
\curveto(92.15784964,414.99209611)(92.11284969,414.85709624)(92.06285278,414.71710297)
\curveto(92.05284975,414.64709645)(92.04284976,414.58209652)(92.03285278,414.52210297)
\curveto(92.02284978,414.46209664)(92.01284979,414.3970967)(92.00285278,414.32710297)
\curveto(91.98284982,414.26709683)(91.97284983,414.2020969)(91.97285278,414.13210297)
\curveto(91.98284982,414.07209703)(91.97784982,414.00709709)(91.95785278,413.93710297)
\curveto(91.93784986,413.85709724)(91.92784987,413.77209733)(91.92785278,413.68210297)
\curveto(91.93784986,413.6020975)(91.94284986,413.52209758)(91.94285278,413.44210297)
\curveto(91.94284986,413.4020977)(91.93784986,413.36209774)(91.92785278,413.32210297)
\curveto(91.92784987,413.28209782)(91.93284987,413.24209786)(91.94285278,413.20210297)
\lineto(91.94285278,413.06710297)
\curveto(91.96284984,413.01709808)(91.96784983,412.96709813)(91.95785278,412.91710297)
\curveto(91.95784984,412.86709823)(91.96784983,412.81709828)(91.98785278,412.76710297)
\curveto(91.98784981,412.70709839)(91.9978498,412.62709847)(92.01785278,412.52710297)
\curveto(92.03784976,412.41709868)(92.05784974,412.31209879)(92.07785278,412.21210297)
\curveto(92.0978497,412.12209898)(92.12284968,412.03209907)(92.15285278,411.94210297)
\curveto(92.29284951,411.52209958)(92.47284933,411.16209994)(92.69285278,410.86210297)
\curveto(92.91284889,410.57210053)(93.2028486,410.33210077)(93.56285278,410.14210297)
\curveto(93.67284813,410.09210101)(93.78784801,410.04710105)(93.90785278,410.00710297)
\curveto(94.02784777,409.97710112)(94.15784764,409.94210116)(94.29785278,409.90210297)
\curveto(94.34784745,409.89210121)(94.39284741,409.88210122)(94.43285278,409.87210297)
\lineto(94.58285278,409.87210297)
\lineto(94.70285278,409.87210297)
\curveto(94.75284705,409.85210125)(94.81784698,409.84210126)(94.89785278,409.84210297)
\curveto(94.97784682,409.85210125)(95.04284676,409.86210124)(95.09285278,409.87210297)
\curveto(95.15284665,409.87210123)(95.1978466,409.87710122)(95.22785278,409.88710297)
\curveto(95.34784645,409.90710119)(95.45784634,409.92710117)(95.55785278,409.94710297)
\curveto(95.66784613,409.96710113)(95.77284603,410.0021011)(95.87285278,410.05210297)
\curveto(96.1028457,410.15210095)(96.2978455,410.28210082)(96.45785278,410.44210297)
\curveto(96.62784517,410.61210049)(96.77784502,410.8021003)(96.90785278,411.01210297)
\curveto(96.96784483,411.11209999)(97.01784478,411.22209988)(97.05785278,411.34210297)
\curveto(97.0978447,411.46209964)(97.14284466,411.58209952)(97.19285278,411.70210297)
\curveto(97.22284458,411.81209929)(97.25284455,411.91209919)(97.28285278,412.00210297)
\curveto(97.31284449,412.09209901)(97.37784442,412.16209894)(97.47785278,412.21210297)
\curveto(97.53784426,412.23209887)(97.61284419,412.24209886)(97.70285278,412.24210297)
\lineto(97.95785278,412.24210297)
\lineto(98.87285278,412.24210297)
\lineto(99.14285278,412.24210297)
\curveto(99.24284256,412.24209886)(99.32284248,412.22209888)(99.38285278,412.18210297)
\curveto(99.45284235,412.13209897)(99.48784231,412.05209905)(99.48785278,411.94210297)
\curveto(99.4978423,411.83209927)(99.48784231,411.72709937)(99.45785278,411.62710297)
\lineto(99.36785278,411.22210297)
\curveto(99.31784248,411.07210003)(99.26784253,410.92710017)(99.21785278,410.78710297)
\curveto(99.17784262,410.64710045)(99.12284268,410.51210059)(99.05285278,410.38210297)
\curveto(99.0028428,410.3021008)(98.96284284,410.22210088)(98.93285278,410.14210297)
\curveto(98.9028429,410.07210103)(98.86284294,410.0021011)(98.81285278,409.93210297)
\curveto(98.26284354,409.07210203)(97.46284434,408.47210263)(96.41285278,408.13210297)
\curveto(96.3028455,408.09210301)(96.19284561,408.06210304)(96.08285278,408.04210297)
\lineto(95.75285278,407.98210297)
\curveto(95.7028461,407.96210314)(95.65284615,407.95710314)(95.60285278,407.96710297)
\curveto(95.56284624,407.96710313)(95.51784628,407.95710314)(95.46785278,407.93710297)
\lineto(95.25785278,407.93710297)
\curveto(95.1978466,407.92710317)(95.13284667,407.91710318)(95.06285278,407.90710297)
\lineto(94.82285278,407.90710297)
\lineto(94.55285278,407.90710297)
\curveto(94.46284734,407.90710319)(94.37784742,407.91710318)(94.29785278,407.93710297)
\curveto(94.26784753,407.94710315)(94.21784758,407.95210315)(94.14785278,407.95210297)
\lineto(93.93785278,407.98210297)
\curveto(93.86784793,407.99210311)(93.79284801,408.0021031)(93.71285278,408.01210297)
\curveto(93.61284819,408.04210306)(93.51284829,408.06710303)(93.41285278,408.08710297)
\curveto(93.32284848,408.10710299)(93.22784857,408.13210297)(93.12785278,408.16210297)
\lineto(92.85785278,408.25210297)
\curveto(92.76784903,408.29210281)(92.68284912,408.33210277)(92.60285278,408.37210297)
\curveto(92.0028498,408.63210247)(91.49285031,408.97710212)(91.07285278,409.40710297)
\curveto(90.66285114,409.84710125)(90.32785147,410.36710073)(90.06785278,410.96710297)
\curveto(90.00785179,411.11709998)(89.95785184,411.26709983)(89.91785278,411.41710297)
\curveto(89.87785192,411.56709953)(89.83285197,411.72209938)(89.78285278,411.88210297)
\curveto(89.76285204,411.93209917)(89.75285205,411.97209913)(89.75285278,412.00210297)
\curveto(89.75285205,412.04209906)(89.74785205,412.08209902)(89.73785278,412.12210297)
\curveto(89.71785208,412.21209889)(89.7028521,412.30709879)(89.69285278,412.40710297)
\curveto(89.68285212,412.50709859)(89.66785213,412.6020985)(89.64785278,412.69210297)
\curveto(89.62785217,412.74209836)(89.61785218,412.78209832)(89.61785278,412.81210297)
\curveto(89.62785217,412.85209825)(89.62785217,412.89209821)(89.61785278,412.93210297)
\lineto(89.61785278,413.21710297)
\curveto(89.5978522,413.26709783)(89.58785221,413.34209776)(89.58785278,413.44210297)
\curveto(89.58785221,413.54209756)(89.5978522,413.61709748)(89.61785278,413.66710297)
\curveto(89.62785217,413.6970974)(89.62785217,413.72709737)(89.61785278,413.75710297)
\lineto(89.61785278,413.84710297)
\lineto(89.61785278,413.98210297)
\curveto(89.63785216,414.06209704)(89.64785215,414.14709695)(89.64785278,414.23710297)
\lineto(89.67785278,414.50710297)
\curveto(89.6978521,414.58709651)(89.71285209,414.66209644)(89.72285278,414.73210297)
\curveto(89.73285207,414.81209629)(89.74785205,414.89209621)(89.76785278,414.97210297)
\curveto(89.80785199,415.11209599)(89.84285196,415.24709585)(89.87285278,415.37710297)
\curveto(89.9028519,415.51709558)(89.94285186,415.64709545)(89.99285278,415.76710297)
\lineto(90.14285278,416.15710297)
\curveto(90.2028516,416.28709481)(90.26785153,416.40709469)(90.33785278,416.51710297)
\curveto(90.4978513,416.7970943)(90.66285114,417.04709405)(90.83285278,417.26710297)
\curveto(90.88285092,417.33709376)(90.93785086,417.4020937)(90.99785278,417.46210297)
\lineto(91.17785278,417.64210297)
\curveto(91.42785037,417.89209321)(91.68785011,418.102093)(91.95785278,418.27210297)
\curveto(92.23784956,418.45209265)(92.55284925,418.61209249)(92.90285278,418.75210297)
\curveto(93.02284878,418.8020923)(93.14784865,418.84209226)(93.27785278,418.87210297)
\curveto(93.40784839,418.91209219)(93.54284826,418.94709215)(93.68285278,418.97710297)
\curveto(93.74284806,418.9970921)(93.802848,419.00709209)(93.86285278,419.00710297)
\curveto(93.92284788,419.00709209)(93.97784782,419.01709208)(94.02785278,419.03710297)
\curveto(94.10784769,419.04709205)(94.18284762,419.05209205)(94.25285278,419.05210297)
\curveto(94.33284747,419.06209204)(94.41284739,419.07209203)(94.49285278,419.08210297)
\curveto(94.51284729,419.09209201)(94.53784726,419.09209201)(94.56785278,419.08210297)
\curveto(94.5978472,419.07209203)(94.62284718,419.07209203)(94.64285278,419.08210297)
}
}
{
\newrgbcolor{curcolor}{0 0 0}
\pscustom[linestyle=none,fillstyle=solid,fillcolor=curcolor]
{
\newpath
\moveto(107.95136841,408.73210297)
\curveto(107.97136056,408.62210248)(107.98136055,408.51210259)(107.98136841,408.40210297)
\curveto(107.99136054,408.29210281)(107.94136059,408.21710288)(107.83136841,408.17710297)
\curveto(107.77136076,408.14710295)(107.70136083,408.13210297)(107.62136841,408.13210297)
\lineto(107.38136841,408.13210297)
\lineto(106.57136841,408.13210297)
\lineto(106.30136841,408.13210297)
\curveto(106.22136231,408.14210296)(106.15636237,408.16710293)(106.10636841,408.20710297)
\curveto(106.03636249,408.24710285)(105.98136255,408.3021028)(105.94136841,408.37210297)
\curveto(105.91136262,408.45210265)(105.86636266,408.51710258)(105.80636841,408.56710297)
\curveto(105.78636274,408.58710251)(105.76136277,408.6021025)(105.73136841,408.61210297)
\curveto(105.70136283,408.63210247)(105.66136287,408.63710246)(105.61136841,408.62710297)
\curveto(105.56136297,408.60710249)(105.51136302,408.58210252)(105.46136841,408.55210297)
\curveto(105.42136311,408.52210258)(105.37636315,408.4971026)(105.32636841,408.47710297)
\curveto(105.27636325,408.43710266)(105.22136331,408.4021027)(105.16136841,408.37210297)
\lineto(104.98136841,408.28210297)
\curveto(104.85136368,408.22210288)(104.71636381,408.17210293)(104.57636841,408.13210297)
\curveto(104.43636409,408.102103)(104.29136424,408.06710303)(104.14136841,408.02710297)
\curveto(104.07136446,408.00710309)(104.00136453,407.9971031)(103.93136841,407.99710297)
\curveto(103.87136466,407.98710311)(103.80636472,407.97710312)(103.73636841,407.96710297)
\lineto(103.64636841,407.96710297)
\curveto(103.61636491,407.95710314)(103.58636494,407.95210315)(103.55636841,407.95210297)
\lineto(103.39136841,407.95210297)
\curveto(103.29136524,407.93210317)(103.19136534,407.93210317)(103.09136841,407.95210297)
\lineto(102.95636841,407.95210297)
\curveto(102.88636564,407.97210313)(102.81636571,407.98210312)(102.74636841,407.98210297)
\curveto(102.68636584,407.97210313)(102.6263659,407.97710312)(102.56636841,407.99710297)
\curveto(102.46636606,408.01710308)(102.37136616,408.03710306)(102.28136841,408.05710297)
\curveto(102.19136634,408.06710303)(102.10636642,408.09210301)(102.02636841,408.13210297)
\curveto(101.73636679,408.24210286)(101.48636704,408.38210272)(101.27636841,408.55210297)
\curveto(101.07636745,408.73210237)(100.91636761,408.96710213)(100.79636841,409.25710297)
\curveto(100.76636776,409.32710177)(100.73636779,409.4021017)(100.70636841,409.48210297)
\curveto(100.68636784,409.56210154)(100.66636786,409.64710145)(100.64636841,409.73710297)
\curveto(100.6263679,409.78710131)(100.61636791,409.83710126)(100.61636841,409.88710297)
\curveto(100.6263679,409.93710116)(100.6263679,409.98710111)(100.61636841,410.03710297)
\curveto(100.60636792,410.06710103)(100.59636793,410.12710097)(100.58636841,410.21710297)
\curveto(100.58636794,410.31710078)(100.59136794,410.38710071)(100.60136841,410.42710297)
\curveto(100.62136791,410.52710057)(100.6313679,410.61210049)(100.63136841,410.68210297)
\lineto(100.72136841,411.01210297)
\curveto(100.75136778,411.13209997)(100.79136774,411.23709986)(100.84136841,411.32710297)
\curveto(101.01136752,411.61709948)(101.20636732,411.83709926)(101.42636841,411.98710297)
\curveto(101.64636688,412.13709896)(101.9263666,412.26709883)(102.26636841,412.37710297)
\curveto(102.39636613,412.42709867)(102.531366,412.46209864)(102.67136841,412.48210297)
\curveto(102.81136572,412.5020986)(102.95136558,412.52709857)(103.09136841,412.55710297)
\curveto(103.17136536,412.57709852)(103.25636527,412.58709851)(103.34636841,412.58710297)
\curveto(103.43636509,412.5970985)(103.526365,412.61209849)(103.61636841,412.63210297)
\curveto(103.68636484,412.65209845)(103.75636477,412.65709844)(103.82636841,412.64710297)
\curveto(103.89636463,412.64709845)(103.97136456,412.65709844)(104.05136841,412.67710297)
\curveto(104.12136441,412.6970984)(104.19136434,412.70709839)(104.26136841,412.70710297)
\curveto(104.3313642,412.70709839)(104.40636412,412.71709838)(104.48636841,412.73710297)
\curveto(104.69636383,412.78709831)(104.88636364,412.82709827)(105.05636841,412.85710297)
\curveto(105.23636329,412.8970982)(105.39636313,412.98709811)(105.53636841,413.12710297)
\curveto(105.6263629,413.21709788)(105.68636284,413.31709778)(105.71636841,413.42710297)
\curveto(105.7263628,413.45709764)(105.7263628,413.48209762)(105.71636841,413.50210297)
\curveto(105.71636281,413.52209758)(105.72136281,413.54209756)(105.73136841,413.56210297)
\curveto(105.74136279,413.58209752)(105.74636278,413.61209749)(105.74636841,413.65210297)
\lineto(105.74636841,413.74210297)
\lineto(105.71636841,413.86210297)
\curveto(105.71636281,413.9020972)(105.71136282,413.93709716)(105.70136841,413.96710297)
\curveto(105.60136293,414.26709683)(105.39136314,414.47209663)(105.07136841,414.58210297)
\curveto(104.98136355,414.61209649)(104.87136366,414.63209647)(104.74136841,414.64210297)
\curveto(104.62136391,414.66209644)(104.49636403,414.66709643)(104.36636841,414.65710297)
\curveto(104.23636429,414.65709644)(104.11136442,414.64709645)(103.99136841,414.62710297)
\curveto(103.87136466,414.60709649)(103.76636476,414.58209652)(103.67636841,414.55210297)
\curveto(103.61636491,414.53209657)(103.55636497,414.5020966)(103.49636841,414.46210297)
\curveto(103.44636508,414.43209667)(103.39636513,414.3970967)(103.34636841,414.35710297)
\curveto(103.29636523,414.31709678)(103.24136529,414.26209684)(103.18136841,414.19210297)
\curveto(103.1313654,414.12209698)(103.09636543,414.05709704)(103.07636841,413.99710297)
\curveto(103.0263655,413.8970972)(102.98136555,413.8020973)(102.94136841,413.71210297)
\curveto(102.91136562,413.62209748)(102.84136569,413.56209754)(102.73136841,413.53210297)
\curveto(102.65136588,413.51209759)(102.56636596,413.5020976)(102.47636841,413.50210297)
\lineto(102.20636841,413.50210297)
\lineto(101.63636841,413.50210297)
\curveto(101.58636694,413.5020976)(101.53636699,413.4970976)(101.48636841,413.48710297)
\curveto(101.43636709,413.48709761)(101.39136714,413.49209761)(101.35136841,413.50210297)
\lineto(101.21636841,413.50210297)
\curveto(101.19636733,413.51209759)(101.17136736,413.51709758)(101.14136841,413.51710297)
\curveto(101.11136742,413.51709758)(101.08636744,413.52709757)(101.06636841,413.54710297)
\curveto(100.98636754,413.56709753)(100.9313676,413.63209747)(100.90136841,413.74210297)
\curveto(100.89136764,413.79209731)(100.89136764,413.84209726)(100.90136841,413.89210297)
\curveto(100.91136762,413.94209716)(100.92136761,413.98709711)(100.93136841,414.02710297)
\curveto(100.96136757,414.13709696)(100.99136754,414.23709686)(101.02136841,414.32710297)
\curveto(101.06136747,414.42709667)(101.10636742,414.51709658)(101.15636841,414.59710297)
\lineto(101.24636841,414.74710297)
\lineto(101.33636841,414.89710297)
\curveto(101.41636711,415.00709609)(101.51636701,415.11209599)(101.63636841,415.21210297)
\curveto(101.65636687,415.22209588)(101.68636684,415.24709585)(101.72636841,415.28710297)
\curveto(101.77636675,415.32709577)(101.82136671,415.36209574)(101.86136841,415.39210297)
\curveto(101.90136663,415.42209568)(101.94636658,415.45209565)(101.99636841,415.48210297)
\curveto(102.16636636,415.59209551)(102.34636618,415.67709542)(102.53636841,415.73710297)
\curveto(102.7263658,415.80709529)(102.92136561,415.87209523)(103.12136841,415.93210297)
\curveto(103.24136529,415.96209514)(103.36636516,415.98209512)(103.49636841,415.99210297)
\curveto(103.6263649,416.0020951)(103.75636477,416.02209508)(103.88636841,416.05210297)
\curveto(103.9263646,416.06209504)(103.98636454,416.06209504)(104.06636841,416.05210297)
\curveto(104.15636437,416.04209506)(104.21136432,416.04709505)(104.23136841,416.06710297)
\curveto(104.64136389,416.07709502)(105.0313635,416.06209504)(105.40136841,416.02210297)
\curveto(105.78136275,415.98209512)(106.12136241,415.90709519)(106.42136841,415.79710297)
\curveto(106.7313618,415.68709541)(106.99636153,415.53709556)(107.21636841,415.34710297)
\curveto(107.43636109,415.16709593)(107.60636092,414.93209617)(107.72636841,414.64210297)
\curveto(107.79636073,414.47209663)(107.83636069,414.27709682)(107.84636841,414.05710297)
\curveto(107.85636067,413.83709726)(107.86136067,413.61209749)(107.86136841,413.38210297)
\lineto(107.86136841,410.03710297)
\lineto(107.86136841,409.45210297)
\curveto(107.86136067,409.26210184)(107.88136065,409.08710201)(107.92136841,408.92710297)
\curveto(107.9313606,408.8971022)(107.93636059,408.86210224)(107.93636841,408.82210297)
\curveto(107.93636059,408.79210231)(107.94136059,408.76210234)(107.95136841,408.73210297)
\moveto(105.74636841,411.04210297)
\curveto(105.75636277,411.09210001)(105.76136277,411.14709995)(105.76136841,411.20710297)
\curveto(105.76136277,411.27709982)(105.75636277,411.33709976)(105.74636841,411.38710297)
\curveto(105.7263628,411.44709965)(105.71636281,411.5020996)(105.71636841,411.55210297)
\curveto(105.71636281,411.6020995)(105.69636283,411.64209946)(105.65636841,411.67210297)
\curveto(105.60636292,411.71209939)(105.531363,411.73209937)(105.43136841,411.73210297)
\curveto(105.39136314,411.72209938)(105.35636317,411.71209939)(105.32636841,411.70210297)
\curveto(105.29636323,411.7020994)(105.26136327,411.6970994)(105.22136841,411.68710297)
\curveto(105.15136338,411.66709943)(105.07636345,411.65209945)(104.99636841,411.64210297)
\curveto(104.91636361,411.63209947)(104.83636369,411.61709948)(104.75636841,411.59710297)
\curveto(104.7263638,411.58709951)(104.68136385,411.58209952)(104.62136841,411.58210297)
\curveto(104.49136404,411.55209955)(104.36136417,411.53209957)(104.23136841,411.52210297)
\curveto(104.10136443,411.51209959)(103.97636455,411.48709961)(103.85636841,411.44710297)
\curveto(103.77636475,411.42709967)(103.70136483,411.40709969)(103.63136841,411.38710297)
\curveto(103.56136497,411.37709972)(103.49136504,411.35709974)(103.42136841,411.32710297)
\curveto(103.21136532,411.23709986)(103.0313655,411.1021)(102.88136841,410.92210297)
\curveto(102.74136579,410.74210036)(102.69136584,410.49210061)(102.73136841,410.17210297)
\curveto(102.75136578,410.0021011)(102.80636572,409.86210124)(102.89636841,409.75210297)
\curveto(102.96636556,409.64210146)(103.07136546,409.55210155)(103.21136841,409.48210297)
\curveto(103.35136518,409.42210168)(103.50136503,409.37710172)(103.66136841,409.34710297)
\curveto(103.8313647,409.31710178)(104.00636452,409.30710179)(104.18636841,409.31710297)
\curveto(104.37636415,409.33710176)(104.55136398,409.37210173)(104.71136841,409.42210297)
\curveto(104.97136356,409.5021016)(105.17636335,409.62710147)(105.32636841,409.79710297)
\curveto(105.47636305,409.97710112)(105.59136294,410.1971009)(105.67136841,410.45710297)
\curveto(105.69136284,410.52710057)(105.70136283,410.5971005)(105.70136841,410.66710297)
\curveto(105.71136282,410.74710035)(105.7263628,410.82710027)(105.74636841,410.90710297)
\lineto(105.74636841,411.04210297)
}
}
{
\newrgbcolor{curcolor}{0 0 0}
\pscustom[linestyle=none,fillstyle=solid,fillcolor=curcolor]
{
\newpath
\moveto(113.93964966,416.06710297)
\curveto(114.53964385,416.08709501)(115.03964335,416.0020951)(115.43964966,415.81210297)
\curveto(115.83964255,415.62209548)(116.15464224,415.34209576)(116.38464966,414.97210297)
\curveto(116.45464194,414.86209624)(116.50964188,414.74209636)(116.54964966,414.61210297)
\curveto(116.5896418,414.49209661)(116.62964176,414.36709673)(116.66964966,414.23710297)
\curveto(116.6896417,414.15709694)(116.69964169,414.08209702)(116.69964966,414.01210297)
\curveto(116.70964168,413.94209716)(116.72464167,413.87209723)(116.74464966,413.80210297)
\curveto(116.74464165,413.74209736)(116.74964164,413.7020974)(116.75964966,413.68210297)
\curveto(116.77964161,413.54209756)(116.7896416,413.3970977)(116.78964966,413.24710297)
\lineto(116.78964966,412.81210297)
\lineto(116.78964966,411.47710297)
\lineto(116.78964966,409.04710297)
\curveto(116.7896416,408.85710224)(116.78464161,408.67210243)(116.77464966,408.49210297)
\curveto(116.77464162,408.32210278)(116.70464169,408.21210289)(116.56464966,408.16210297)
\curveto(116.50464189,408.14210296)(116.43464196,408.13210297)(116.35464966,408.13210297)
\lineto(116.11464966,408.13210297)
\lineto(115.30464966,408.13210297)
\curveto(115.18464321,408.13210297)(115.07464332,408.13710296)(114.97464966,408.14710297)
\curveto(114.88464351,408.16710293)(114.81464358,408.21210289)(114.76464966,408.28210297)
\curveto(114.72464367,408.34210276)(114.69964369,408.41710268)(114.68964966,408.50710297)
\lineto(114.68964966,408.82210297)
\lineto(114.68964966,409.87210297)
\lineto(114.68964966,412.10710297)
\curveto(114.6896437,412.47709862)(114.67464372,412.81709828)(114.64464966,413.12710297)
\curveto(114.61464378,413.44709765)(114.52464387,413.71709738)(114.37464966,413.93710297)
\curveto(114.23464416,414.13709696)(114.02964436,414.27709682)(113.75964966,414.35710297)
\curveto(113.70964468,414.37709672)(113.65464474,414.38709671)(113.59464966,414.38710297)
\curveto(113.54464485,414.38709671)(113.4896449,414.3970967)(113.42964966,414.41710297)
\curveto(113.37964501,414.42709667)(113.31464508,414.42709667)(113.23464966,414.41710297)
\curveto(113.16464523,414.41709668)(113.10964528,414.41209669)(113.06964966,414.40210297)
\curveto(113.02964536,414.39209671)(112.9946454,414.38709671)(112.96464966,414.38710297)
\curveto(112.93464546,414.38709671)(112.90464549,414.38209672)(112.87464966,414.37210297)
\curveto(112.64464575,414.31209679)(112.45964593,414.23209687)(112.31964966,414.13210297)
\curveto(111.99964639,413.9020972)(111.80964658,413.56709753)(111.74964966,413.12710297)
\curveto(111.6896467,412.68709841)(111.65964673,412.19209891)(111.65964966,411.64210297)
\lineto(111.65964966,409.76710297)
\lineto(111.65964966,408.85210297)
\lineto(111.65964966,408.58210297)
\curveto(111.65964673,408.49210261)(111.64464675,408.41710268)(111.61464966,408.35710297)
\curveto(111.56464683,408.24710285)(111.48464691,408.18210292)(111.37464966,408.16210297)
\curveto(111.26464713,408.14210296)(111.12964726,408.13210297)(110.96964966,408.13210297)
\lineto(110.21964966,408.13210297)
\curveto(110.10964828,408.13210297)(109.99964839,408.13710296)(109.88964966,408.14710297)
\curveto(109.77964861,408.15710294)(109.69964869,408.19210291)(109.64964966,408.25210297)
\curveto(109.57964881,408.34210276)(109.54464885,408.47210263)(109.54464966,408.64210297)
\curveto(109.55464884,408.81210229)(109.55964883,408.97210213)(109.55964966,409.12210297)
\lineto(109.55964966,411.16210297)
\lineto(109.55964966,414.46210297)
\lineto(109.55964966,415.22710297)
\lineto(109.55964966,415.52710297)
\curveto(109.56964882,415.61709548)(109.59964879,415.69209541)(109.64964966,415.75210297)
\curveto(109.66964872,415.78209532)(109.69964869,415.8020953)(109.73964966,415.81210297)
\curveto(109.7896486,415.83209527)(109.83964855,415.84709525)(109.88964966,415.85710297)
\lineto(109.96464966,415.85710297)
\curveto(110.01464838,415.86709523)(110.06464833,415.87209523)(110.11464966,415.87210297)
\lineto(110.27964966,415.87210297)
\lineto(110.90964966,415.87210297)
\curveto(110.9896474,415.87209523)(111.06464733,415.86709523)(111.13464966,415.85710297)
\curveto(111.21464718,415.85709524)(111.28464711,415.84709525)(111.34464966,415.82710297)
\curveto(111.41464698,415.7970953)(111.45964693,415.75209535)(111.47964966,415.69210297)
\curveto(111.50964688,415.63209547)(111.53464686,415.56209554)(111.55464966,415.48210297)
\curveto(111.56464683,415.44209566)(111.56464683,415.40709569)(111.55464966,415.37710297)
\curveto(111.55464684,415.34709575)(111.56464683,415.31709578)(111.58464966,415.28710297)
\curveto(111.60464679,415.23709586)(111.61964677,415.20709589)(111.62964966,415.19710297)
\curveto(111.64964674,415.18709591)(111.67464672,415.17209593)(111.70464966,415.15210297)
\curveto(111.81464658,415.14209596)(111.90464649,415.17709592)(111.97464966,415.25710297)
\curveto(112.04464635,415.34709575)(112.11964627,415.41709568)(112.19964966,415.46710297)
\curveto(112.46964592,415.66709543)(112.76964562,415.82709527)(113.09964966,415.94710297)
\curveto(113.1896452,415.97709512)(113.27964511,415.9970951)(113.36964966,416.00710297)
\curveto(113.46964492,416.01709508)(113.57464482,416.03209507)(113.68464966,416.05210297)
\curveto(113.71464468,416.06209504)(113.75964463,416.06209504)(113.81964966,416.05210297)
\curveto(113.87964451,416.05209505)(113.91964447,416.05709504)(113.93964966,416.06710297)
}
}
{
\newrgbcolor{curcolor}{0 0 0}
\pscustom[linestyle=none,fillstyle=solid,fillcolor=curcolor]
{
\newpath
\moveto(119.47089966,418.18210297)
\lineto(120.47589966,418.18210297)
\curveto(120.62589667,418.18209292)(120.75589654,418.17209293)(120.86589966,418.15210297)
\curveto(120.98589631,418.14209296)(121.07089623,418.08209302)(121.12089966,417.97210297)
\curveto(121.14089616,417.92209318)(121.15089615,417.86209324)(121.15089966,417.79210297)
\lineto(121.15089966,417.58210297)
\lineto(121.15089966,416.90710297)
\curveto(121.15089615,416.85709424)(121.14589615,416.7970943)(121.13589966,416.72710297)
\curveto(121.13589616,416.66709443)(121.14089616,416.61209449)(121.15089966,416.56210297)
\lineto(121.15089966,416.39710297)
\curveto(121.15089615,416.31709478)(121.15589614,416.24209486)(121.16589966,416.17210297)
\curveto(121.17589612,416.11209499)(121.2008961,416.05709504)(121.24089966,416.00710297)
\curveto(121.31089599,415.91709518)(121.43589586,415.86709523)(121.61589966,415.85710297)
\lineto(122.15589966,415.85710297)
\lineto(122.33589966,415.85710297)
\curveto(122.3958949,415.85709524)(122.45089485,415.84709525)(122.50089966,415.82710297)
\curveto(122.61089469,415.77709532)(122.67089463,415.68709541)(122.68089966,415.55710297)
\curveto(122.7008946,415.42709567)(122.71089459,415.28209582)(122.71089966,415.12210297)
\lineto(122.71089966,414.91210297)
\curveto(122.72089458,414.84209626)(122.71589458,414.78209632)(122.69589966,414.73210297)
\curveto(122.64589465,414.57209653)(122.54089476,414.48709661)(122.38089966,414.47710297)
\curveto(122.22089508,414.46709663)(122.04089526,414.46209664)(121.84089966,414.46210297)
\lineto(121.70589966,414.46210297)
\curveto(121.66589563,414.47209663)(121.63089567,414.47209663)(121.60089966,414.46210297)
\curveto(121.56089574,414.45209665)(121.52589577,414.44709665)(121.49589966,414.44710297)
\curveto(121.46589583,414.45709664)(121.43589586,414.45209665)(121.40589966,414.43210297)
\curveto(121.32589597,414.41209669)(121.26589603,414.36709673)(121.22589966,414.29710297)
\curveto(121.1958961,414.23709686)(121.17089613,414.16209694)(121.15089966,414.07210297)
\curveto(121.14089616,414.02209708)(121.14089616,413.96709713)(121.15089966,413.90710297)
\curveto(121.16089614,413.84709725)(121.16089614,413.79209731)(121.15089966,413.74210297)
\lineto(121.15089966,412.81210297)
\lineto(121.15089966,411.05710297)
\curveto(121.15089615,410.80710029)(121.15589614,410.58710051)(121.16589966,410.39710297)
\curveto(121.18589611,410.21710088)(121.25089605,410.05710104)(121.36089966,409.91710297)
\curveto(121.41089589,409.85710124)(121.47589582,409.81210129)(121.55589966,409.78210297)
\lineto(121.82589966,409.72210297)
\curveto(121.85589544,409.71210139)(121.88589541,409.70710139)(121.91589966,409.70710297)
\curveto(121.95589534,409.71710138)(121.98589531,409.71710138)(122.00589966,409.70710297)
\lineto(122.17089966,409.70710297)
\curveto(122.28089502,409.70710139)(122.37589492,409.7021014)(122.45589966,409.69210297)
\curveto(122.53589476,409.68210142)(122.6008947,409.64210146)(122.65089966,409.57210297)
\curveto(122.69089461,409.51210159)(122.71089459,409.43210167)(122.71089966,409.33210297)
\lineto(122.71089966,409.04710297)
\curveto(122.71089459,408.83710226)(122.70589459,408.64210246)(122.69589966,408.46210297)
\curveto(122.6958946,408.29210281)(122.61589468,408.17710292)(122.45589966,408.11710297)
\curveto(122.40589489,408.097103)(122.36089494,408.09210301)(122.32089966,408.10210297)
\curveto(122.28089502,408.102103)(122.23589506,408.09210301)(122.18589966,408.07210297)
\lineto(122.03589966,408.07210297)
\curveto(122.01589528,408.07210303)(121.98589531,408.07710302)(121.94589966,408.08710297)
\curveto(121.90589539,408.08710301)(121.87089543,408.08210302)(121.84089966,408.07210297)
\curveto(121.79089551,408.06210304)(121.73589556,408.06210304)(121.67589966,408.07210297)
\lineto(121.52589966,408.07210297)
\lineto(121.37589966,408.07210297)
\curveto(121.32589597,408.06210304)(121.28089602,408.06210304)(121.24089966,408.07210297)
\lineto(121.07589966,408.07210297)
\curveto(121.02589627,408.08210302)(120.97089633,408.08710301)(120.91089966,408.08710297)
\curveto(120.85089645,408.08710301)(120.7958965,408.09210301)(120.74589966,408.10210297)
\curveto(120.67589662,408.11210299)(120.61089669,408.12210298)(120.55089966,408.13210297)
\lineto(120.37089966,408.16210297)
\curveto(120.26089704,408.19210291)(120.15589714,408.22710287)(120.05589966,408.26710297)
\curveto(119.95589734,408.30710279)(119.86089744,408.35210275)(119.77089966,408.40210297)
\lineto(119.68089966,408.46210297)
\curveto(119.65089765,408.49210261)(119.61589768,408.52210258)(119.57589966,408.55210297)
\curveto(119.55589774,408.57210253)(119.53089777,408.59210251)(119.50089966,408.61210297)
\lineto(119.42589966,408.68710297)
\curveto(119.28589801,408.87710222)(119.18089812,409.08710201)(119.11089966,409.31710297)
\curveto(119.09089821,409.35710174)(119.08089822,409.39210171)(119.08089966,409.42210297)
\curveto(119.09089821,409.46210164)(119.09089821,409.50710159)(119.08089966,409.55710297)
\curveto(119.07089823,409.57710152)(119.06589823,409.6021015)(119.06589966,409.63210297)
\curveto(119.06589823,409.66210144)(119.06089824,409.68710141)(119.05089966,409.70710297)
\lineto(119.05089966,409.85710297)
\curveto(119.04089826,409.8971012)(119.03589826,409.94210116)(119.03589966,409.99210297)
\curveto(119.04589825,410.04210106)(119.05089825,410.09210101)(119.05089966,410.14210297)
\lineto(119.05089966,410.71210297)
\lineto(119.05089966,412.94710297)
\lineto(119.05089966,413.74210297)
\lineto(119.05089966,413.95210297)
\curveto(119.06089824,414.02209708)(119.05589824,414.08709701)(119.03589966,414.14710297)
\curveto(118.9958983,414.28709681)(118.92589837,414.37709672)(118.82589966,414.41710297)
\curveto(118.71589858,414.46709663)(118.57589872,414.48209662)(118.40589966,414.46210297)
\curveto(118.23589906,414.44209666)(118.09089921,414.45709664)(117.97089966,414.50710297)
\curveto(117.89089941,414.53709656)(117.84089946,414.58209652)(117.82089966,414.64210297)
\curveto(117.8008995,414.7020964)(117.78089952,414.77709632)(117.76089966,414.86710297)
\lineto(117.76089966,415.18210297)
\curveto(117.76089954,415.36209574)(117.77089953,415.50709559)(117.79089966,415.61710297)
\curveto(117.81089949,415.72709537)(117.8958994,415.8020953)(118.04589966,415.84210297)
\curveto(118.08589921,415.86209524)(118.12589917,415.86709523)(118.16589966,415.85710297)
\lineto(118.30089966,415.85710297)
\curveto(118.45089885,415.85709524)(118.59089871,415.86209524)(118.72089966,415.87210297)
\curveto(118.85089845,415.89209521)(118.94089836,415.95209515)(118.99089966,416.05210297)
\curveto(119.02089828,416.12209498)(119.03589826,416.2020949)(119.03589966,416.29210297)
\curveto(119.04589825,416.38209472)(119.05089825,416.47209463)(119.05089966,416.56210297)
\lineto(119.05089966,417.49210297)
\lineto(119.05089966,417.74710297)
\curveto(119.05089825,417.83709326)(119.06089824,417.91209319)(119.08089966,417.97210297)
\curveto(119.13089817,418.07209303)(119.20589809,418.13709296)(119.30589966,418.16710297)
\curveto(119.32589797,418.17709292)(119.35089795,418.17709292)(119.38089966,418.16710297)
\curveto(119.42089788,418.16709293)(119.45089785,418.17209293)(119.47089966,418.18210297)
}
}
{
\newrgbcolor{curcolor}{0 0 0}
\pscustom[linestyle=none,fillstyle=solid,fillcolor=curcolor]
{
\newpath
\moveto(125.79433716,418.72210297)
\curveto(125.86433421,418.64209246)(125.89933417,418.52209258)(125.89933716,418.36210297)
\lineto(125.89933716,417.89710297)
\lineto(125.89933716,417.49210297)
\curveto(125.89933417,417.35209375)(125.86433421,417.25709384)(125.79433716,417.20710297)
\curveto(125.73433434,417.15709394)(125.65433442,417.12709397)(125.55433716,417.11710297)
\curveto(125.46433461,417.10709399)(125.36433471,417.102094)(125.25433716,417.10210297)
\lineto(124.41433716,417.10210297)
\curveto(124.30433577,417.102094)(124.20433587,417.10709399)(124.11433716,417.11710297)
\curveto(124.03433604,417.12709397)(123.96433611,417.15709394)(123.90433716,417.20710297)
\curveto(123.86433621,417.23709386)(123.83433624,417.29209381)(123.81433716,417.37210297)
\curveto(123.80433627,417.46209364)(123.79433628,417.55709354)(123.78433716,417.65710297)
\lineto(123.78433716,417.98710297)
\curveto(123.79433628,418.097093)(123.79933627,418.19209291)(123.79933716,418.27210297)
\lineto(123.79933716,418.48210297)
\curveto(123.80933626,418.55209255)(123.82933624,418.61209249)(123.85933716,418.66210297)
\curveto(123.87933619,418.7020924)(123.90433617,418.73209237)(123.93433716,418.75210297)
\lineto(124.05433716,418.81210297)
\curveto(124.074336,418.81209229)(124.09933597,418.81209229)(124.12933716,418.81210297)
\curveto(124.15933591,418.82209228)(124.18433589,418.82709227)(124.20433716,418.82710297)
\lineto(125.29933716,418.82710297)
\curveto(125.39933467,418.82709227)(125.49433458,418.82209228)(125.58433716,418.81210297)
\curveto(125.6743344,418.8020923)(125.74433433,418.77209233)(125.79433716,418.72210297)
\moveto(125.89933716,408.95710297)
\curveto(125.89933417,408.75710234)(125.89433418,408.58710251)(125.88433716,408.44710297)
\curveto(125.8743342,408.30710279)(125.78433429,408.21210289)(125.61433716,408.16210297)
\curveto(125.55433452,408.14210296)(125.48933458,408.13210297)(125.41933716,408.13210297)
\curveto(125.34933472,408.14210296)(125.2743348,408.14710295)(125.19433716,408.14710297)
\lineto(124.35433716,408.14710297)
\curveto(124.26433581,408.14710295)(124.1743359,408.15210295)(124.08433716,408.16210297)
\curveto(124.00433607,408.17210293)(123.94433613,408.2021029)(123.90433716,408.25210297)
\curveto(123.84433623,408.32210278)(123.80933626,408.40710269)(123.79933716,408.50710297)
\lineto(123.79933716,408.85210297)
\lineto(123.79933716,415.18210297)
\lineto(123.79933716,415.48210297)
\curveto(123.79933627,415.58209552)(123.81933625,415.66209544)(123.85933716,415.72210297)
\curveto(123.91933615,415.79209531)(124.00433607,415.83709526)(124.11433716,415.85710297)
\curveto(124.13433594,415.86709523)(124.15933591,415.86709523)(124.18933716,415.85710297)
\curveto(124.22933584,415.85709524)(124.25933581,415.86209524)(124.27933716,415.87210297)
\lineto(125.02933716,415.87210297)
\lineto(125.22433716,415.87210297)
\curveto(125.30433477,415.88209522)(125.3693347,415.88209522)(125.41933716,415.87210297)
\lineto(125.53933716,415.87210297)
\curveto(125.59933447,415.85209525)(125.65433442,415.83709526)(125.70433716,415.82710297)
\curveto(125.75433432,415.81709528)(125.79433428,415.78709531)(125.82433716,415.73710297)
\curveto(125.86433421,415.68709541)(125.88433419,415.61709548)(125.88433716,415.52710297)
\curveto(125.89433418,415.43709566)(125.89933417,415.34209576)(125.89933716,415.24210297)
\lineto(125.89933716,408.95710297)
}
}
{
\newrgbcolor{curcolor}{0 0 0}
\pscustom[linestyle=none,fillstyle=solid,fillcolor=curcolor]
{
\newpath
\moveto(135.15152466,408.98710297)
\lineto(135.15152466,408.56710297)
\curveto(135.15151629,408.43710266)(135.12151632,408.33210277)(135.06152466,408.25210297)
\curveto(135.01151643,408.2021029)(134.94651649,408.16710293)(134.86652466,408.14710297)
\curveto(134.78651665,408.13710296)(134.69651674,408.13210297)(134.59652466,408.13210297)
\lineto(133.77152466,408.13210297)
\lineto(133.48652466,408.13210297)
\curveto(133.40651803,408.14210296)(133.3415181,408.16710293)(133.29152466,408.20710297)
\curveto(133.22151822,408.25710284)(133.18151826,408.32210278)(133.17152466,408.40210297)
\curveto(133.16151828,408.48210262)(133.1415183,408.56210254)(133.11152466,408.64210297)
\curveto(133.09151835,408.66210244)(133.07151837,408.67710242)(133.05152466,408.68710297)
\curveto(133.0415184,408.70710239)(133.02651841,408.72710237)(133.00652466,408.74710297)
\curveto(132.89651854,408.74710235)(132.81651862,408.72210238)(132.76652466,408.67210297)
\lineto(132.61652466,408.52210297)
\curveto(132.54651889,408.47210263)(132.48151896,408.42710267)(132.42152466,408.38710297)
\curveto(132.36151908,408.35710274)(132.29651914,408.31710278)(132.22652466,408.26710297)
\curveto(132.18651925,408.24710285)(132.1415193,408.22710287)(132.09152466,408.20710297)
\curveto(132.05151939,408.18710291)(132.00651943,408.16710293)(131.95652466,408.14710297)
\curveto(131.81651962,408.097103)(131.66651977,408.05210305)(131.50652466,408.01210297)
\curveto(131.45651998,407.99210311)(131.41152003,407.98210312)(131.37152466,407.98210297)
\curveto(131.33152011,407.98210312)(131.29152015,407.97710312)(131.25152466,407.96710297)
\lineto(131.11652466,407.96710297)
\curveto(131.08652035,407.95710314)(131.04652039,407.95210315)(130.99652466,407.95210297)
\lineto(130.86152466,407.95210297)
\curveto(130.80152064,407.93210317)(130.71152073,407.92710317)(130.59152466,407.93710297)
\curveto(130.47152097,407.93710316)(130.38652105,407.94710315)(130.33652466,407.96710297)
\curveto(130.26652117,407.98710311)(130.20152124,407.9971031)(130.14152466,407.99710297)
\curveto(130.09152135,407.98710311)(130.0365214,407.99210311)(129.97652466,408.01210297)
\lineto(129.61652466,408.13210297)
\curveto(129.50652193,408.16210294)(129.39652204,408.2021029)(129.28652466,408.25210297)
\curveto(128.9365225,408.4021027)(128.62152282,408.63210247)(128.34152466,408.94210297)
\curveto(128.07152337,409.26210184)(127.85652358,409.5971015)(127.69652466,409.94710297)
\curveto(127.64652379,410.05710104)(127.60652383,410.16210094)(127.57652466,410.26210297)
\curveto(127.54652389,410.37210073)(127.51152393,410.48210062)(127.47152466,410.59210297)
\curveto(127.46152398,410.63210047)(127.45652398,410.66710043)(127.45652466,410.69710297)
\curveto(127.45652398,410.73710036)(127.44652399,410.78210032)(127.42652466,410.83210297)
\curveto(127.40652403,410.91210019)(127.38652405,410.9971001)(127.36652466,411.08710297)
\curveto(127.35652408,411.18709991)(127.3415241,411.28709981)(127.32152466,411.38710297)
\curveto(127.31152413,411.41709968)(127.30652413,411.45209965)(127.30652466,411.49210297)
\curveto(127.31652412,411.53209957)(127.31652412,411.56709953)(127.30652466,411.59710297)
\lineto(127.30652466,411.73210297)
\curveto(127.30652413,411.78209932)(127.30152414,411.83209927)(127.29152466,411.88210297)
\curveto(127.28152416,411.93209917)(127.27652416,411.98709911)(127.27652466,412.04710297)
\curveto(127.27652416,412.11709898)(127.28152416,412.17209893)(127.29152466,412.21210297)
\curveto(127.30152414,412.26209884)(127.30652413,412.30709879)(127.30652466,412.34710297)
\lineto(127.30652466,412.49710297)
\curveto(127.31652412,412.54709855)(127.31652412,412.59209851)(127.30652466,412.63210297)
\curveto(127.30652413,412.68209842)(127.31652412,412.73209837)(127.33652466,412.78210297)
\curveto(127.35652408,412.89209821)(127.37152407,412.9970981)(127.38152466,413.09710297)
\curveto(127.40152404,413.1970979)(127.42652401,413.2970978)(127.45652466,413.39710297)
\curveto(127.49652394,413.51709758)(127.53152391,413.63209747)(127.56152466,413.74210297)
\curveto(127.59152385,413.85209725)(127.63152381,413.96209714)(127.68152466,414.07210297)
\curveto(127.82152362,414.37209673)(127.99652344,414.65709644)(128.20652466,414.92710297)
\curveto(128.22652321,414.95709614)(128.25152319,414.98209612)(128.28152466,415.00210297)
\curveto(128.32152312,415.03209607)(128.35152309,415.06209604)(128.37152466,415.09210297)
\curveto(128.41152303,415.14209596)(128.45152299,415.18709591)(128.49152466,415.22710297)
\curveto(128.53152291,415.26709583)(128.57652286,415.30709579)(128.62652466,415.34710297)
\curveto(128.66652277,415.36709573)(128.70152274,415.39209571)(128.73152466,415.42210297)
\curveto(128.76152268,415.46209564)(128.79652264,415.49209561)(128.83652466,415.51210297)
\curveto(129.08652235,415.68209542)(129.37652206,415.82209528)(129.70652466,415.93210297)
\curveto(129.77652166,415.95209515)(129.84652159,415.96709513)(129.91652466,415.97710297)
\curveto(129.99652144,415.98709511)(130.07652136,416.0020951)(130.15652466,416.02210297)
\curveto(130.22652121,416.04209506)(130.31652112,416.05209505)(130.42652466,416.05210297)
\curveto(130.5365209,416.06209504)(130.64652079,416.06709503)(130.75652466,416.06710297)
\curveto(130.86652057,416.06709503)(130.97152047,416.06209504)(131.07152466,416.05210297)
\curveto(131.18152026,416.04209506)(131.27152017,416.02709507)(131.34152466,416.00710297)
\curveto(131.49151995,415.95709514)(131.6365198,415.91209519)(131.77652466,415.87210297)
\curveto(131.91651952,415.83209527)(132.04651939,415.77709532)(132.16652466,415.70710297)
\curveto(132.2365192,415.65709544)(132.30151914,415.60709549)(132.36152466,415.55710297)
\curveto(132.42151902,415.51709558)(132.48651895,415.47209563)(132.55652466,415.42210297)
\curveto(132.59651884,415.39209571)(132.65151879,415.35209575)(132.72152466,415.30210297)
\curveto(132.80151864,415.25209585)(132.87651856,415.25209585)(132.94652466,415.30210297)
\curveto(132.98651845,415.32209578)(133.00651843,415.35709574)(133.00652466,415.40710297)
\curveto(133.00651843,415.45709564)(133.01651842,415.50709559)(133.03652466,415.55710297)
\lineto(133.03652466,415.70710297)
\curveto(133.04651839,415.73709536)(133.05151839,415.77209533)(133.05152466,415.81210297)
\lineto(133.05152466,415.93210297)
\lineto(133.05152466,417.97210297)
\curveto(133.05151839,418.08209302)(133.04651839,418.2020929)(133.03652466,418.33210297)
\curveto(133.0365184,418.47209263)(133.06151838,418.57709252)(133.11152466,418.64710297)
\curveto(133.15151829,418.72709237)(133.22651821,418.77709232)(133.33652466,418.79710297)
\curveto(133.35651808,418.80709229)(133.37651806,418.80709229)(133.39652466,418.79710297)
\curveto(133.41651802,418.7970923)(133.436518,418.8020923)(133.45652466,418.81210297)
\lineto(134.52152466,418.81210297)
\curveto(134.6415168,418.81209229)(134.75151669,418.80709229)(134.85152466,418.79710297)
\curveto(134.95151649,418.78709231)(135.02651641,418.74709235)(135.07652466,418.67710297)
\curveto(135.12651631,418.5970925)(135.15151629,418.49209261)(135.15152466,418.36210297)
\lineto(135.15152466,418.00210297)
\lineto(135.15152466,408.98710297)
\moveto(133.11152466,411.92710297)
\curveto(133.12151832,411.96709913)(133.12151832,412.00709909)(133.11152466,412.04710297)
\lineto(133.11152466,412.18210297)
\curveto(133.11151833,412.28209882)(133.10651833,412.38209872)(133.09652466,412.48210297)
\curveto(133.08651835,412.58209852)(133.07151837,412.67209843)(133.05152466,412.75210297)
\curveto(133.03151841,412.86209824)(133.01151843,412.96209814)(132.99152466,413.05210297)
\curveto(132.98151846,413.14209796)(132.95651848,413.22709787)(132.91652466,413.30710297)
\curveto(132.77651866,413.66709743)(132.57151887,413.95209715)(132.30152466,414.16210297)
\curveto(132.0415194,414.37209673)(131.66151978,414.47709662)(131.16152466,414.47710297)
\curveto(131.10152034,414.47709662)(131.02152042,414.46709663)(130.92152466,414.44710297)
\curveto(130.8415206,414.42709667)(130.76652067,414.40709669)(130.69652466,414.38710297)
\curveto(130.6365208,414.37709672)(130.57652086,414.35709674)(130.51652466,414.32710297)
\curveto(130.24652119,414.21709688)(130.0365214,414.04709705)(129.88652466,413.81710297)
\curveto(129.7365217,413.58709751)(129.61652182,413.32709777)(129.52652466,413.03710297)
\curveto(129.49652194,412.93709816)(129.47652196,412.83709826)(129.46652466,412.73710297)
\curveto(129.45652198,412.63709846)(129.436522,412.53209857)(129.40652466,412.42210297)
\lineto(129.40652466,412.21210297)
\curveto(129.38652205,412.12209898)(129.38152206,411.9970991)(129.39152466,411.83710297)
\curveto(129.40152204,411.68709941)(129.41652202,411.57709952)(129.43652466,411.50710297)
\lineto(129.43652466,411.41710297)
\curveto(129.44652199,411.3970997)(129.45152199,411.37709972)(129.45152466,411.35710297)
\curveto(129.47152197,411.27709982)(129.48652195,411.2020999)(129.49652466,411.13210297)
\curveto(129.51652192,411.06210004)(129.5365219,410.98710011)(129.55652466,410.90710297)
\curveto(129.72652171,410.38710071)(130.01652142,410.0021011)(130.42652466,409.75210297)
\curveto(130.55652088,409.66210144)(130.7365207,409.59210151)(130.96652466,409.54210297)
\curveto(131.00652043,409.53210157)(131.06652037,409.52710157)(131.14652466,409.52710297)
\curveto(131.17652026,409.51710158)(131.22152022,409.50710159)(131.28152466,409.49710297)
\curveto(131.35152009,409.4971016)(131.40652003,409.5021016)(131.44652466,409.51210297)
\curveto(131.52651991,409.53210157)(131.60651983,409.54710155)(131.68652466,409.55710297)
\curveto(131.76651967,409.56710153)(131.84651959,409.58710151)(131.92652466,409.61710297)
\curveto(132.17651926,409.72710137)(132.37651906,409.86710123)(132.52652466,410.03710297)
\curveto(132.67651876,410.20710089)(132.80651863,410.42210068)(132.91652466,410.68210297)
\curveto(132.95651848,410.77210033)(132.98651845,410.86210024)(133.00652466,410.95210297)
\curveto(133.02651841,411.05210005)(133.04651839,411.15709994)(133.06652466,411.26710297)
\curveto(133.07651836,411.31709978)(133.07651836,411.36209974)(133.06652466,411.40210297)
\curveto(133.06651837,411.45209965)(133.07651836,411.5020996)(133.09652466,411.55210297)
\curveto(133.10651833,411.58209952)(133.11151833,411.61709948)(133.11152466,411.65710297)
\lineto(133.11152466,411.79210297)
\lineto(133.11152466,411.92710297)
}
}
{
\newrgbcolor{curcolor}{0 0 0}
\pscustom[linestyle=none,fillstyle=solid,fillcolor=curcolor]
{
\newpath
\moveto(143.78144653,408.73210297)
\curveto(143.80143868,408.62210248)(143.81143867,408.51210259)(143.81144653,408.40210297)
\curveto(143.82143866,408.29210281)(143.77143871,408.21710288)(143.66144653,408.17710297)
\curveto(143.60143888,408.14710295)(143.53143895,408.13210297)(143.45144653,408.13210297)
\lineto(143.21144653,408.13210297)
\lineto(142.40144653,408.13210297)
\lineto(142.13144653,408.13210297)
\curveto(142.05144043,408.14210296)(141.9864405,408.16710293)(141.93644653,408.20710297)
\curveto(141.86644062,408.24710285)(141.81144067,408.3021028)(141.77144653,408.37210297)
\curveto(141.74144074,408.45210265)(141.69644079,408.51710258)(141.63644653,408.56710297)
\curveto(141.61644087,408.58710251)(141.59144089,408.6021025)(141.56144653,408.61210297)
\curveto(141.53144095,408.63210247)(141.49144099,408.63710246)(141.44144653,408.62710297)
\curveto(141.39144109,408.60710249)(141.34144114,408.58210252)(141.29144653,408.55210297)
\curveto(141.25144123,408.52210258)(141.20644128,408.4971026)(141.15644653,408.47710297)
\curveto(141.10644138,408.43710266)(141.05144143,408.4021027)(140.99144653,408.37210297)
\lineto(140.81144653,408.28210297)
\curveto(140.6814418,408.22210288)(140.54644194,408.17210293)(140.40644653,408.13210297)
\curveto(140.26644222,408.102103)(140.12144236,408.06710303)(139.97144653,408.02710297)
\curveto(139.90144258,408.00710309)(139.83144265,407.9971031)(139.76144653,407.99710297)
\curveto(139.70144278,407.98710311)(139.63644285,407.97710312)(139.56644653,407.96710297)
\lineto(139.47644653,407.96710297)
\curveto(139.44644304,407.95710314)(139.41644307,407.95210315)(139.38644653,407.95210297)
\lineto(139.22144653,407.95210297)
\curveto(139.12144336,407.93210317)(139.02144346,407.93210317)(138.92144653,407.95210297)
\lineto(138.78644653,407.95210297)
\curveto(138.71644377,407.97210313)(138.64644384,407.98210312)(138.57644653,407.98210297)
\curveto(138.51644397,407.97210313)(138.45644403,407.97710312)(138.39644653,407.99710297)
\curveto(138.29644419,408.01710308)(138.20144428,408.03710306)(138.11144653,408.05710297)
\curveto(138.02144446,408.06710303)(137.93644455,408.09210301)(137.85644653,408.13210297)
\curveto(137.56644492,408.24210286)(137.31644517,408.38210272)(137.10644653,408.55210297)
\curveto(136.90644558,408.73210237)(136.74644574,408.96710213)(136.62644653,409.25710297)
\curveto(136.59644589,409.32710177)(136.56644592,409.4021017)(136.53644653,409.48210297)
\curveto(136.51644597,409.56210154)(136.49644599,409.64710145)(136.47644653,409.73710297)
\curveto(136.45644603,409.78710131)(136.44644604,409.83710126)(136.44644653,409.88710297)
\curveto(136.45644603,409.93710116)(136.45644603,409.98710111)(136.44644653,410.03710297)
\curveto(136.43644605,410.06710103)(136.42644606,410.12710097)(136.41644653,410.21710297)
\curveto(136.41644607,410.31710078)(136.42144606,410.38710071)(136.43144653,410.42710297)
\curveto(136.45144603,410.52710057)(136.46144602,410.61210049)(136.46144653,410.68210297)
\lineto(136.55144653,411.01210297)
\curveto(136.5814459,411.13209997)(136.62144586,411.23709986)(136.67144653,411.32710297)
\curveto(136.84144564,411.61709948)(137.03644545,411.83709926)(137.25644653,411.98710297)
\curveto(137.47644501,412.13709896)(137.75644473,412.26709883)(138.09644653,412.37710297)
\curveto(138.22644426,412.42709867)(138.36144412,412.46209864)(138.50144653,412.48210297)
\curveto(138.64144384,412.5020986)(138.7814437,412.52709857)(138.92144653,412.55710297)
\curveto(139.00144348,412.57709852)(139.0864434,412.58709851)(139.17644653,412.58710297)
\curveto(139.26644322,412.5970985)(139.35644313,412.61209849)(139.44644653,412.63210297)
\curveto(139.51644297,412.65209845)(139.5864429,412.65709844)(139.65644653,412.64710297)
\curveto(139.72644276,412.64709845)(139.80144268,412.65709844)(139.88144653,412.67710297)
\curveto(139.95144253,412.6970984)(140.02144246,412.70709839)(140.09144653,412.70710297)
\curveto(140.16144232,412.70709839)(140.23644225,412.71709838)(140.31644653,412.73710297)
\curveto(140.52644196,412.78709831)(140.71644177,412.82709827)(140.88644653,412.85710297)
\curveto(141.06644142,412.8970982)(141.22644126,412.98709811)(141.36644653,413.12710297)
\curveto(141.45644103,413.21709788)(141.51644097,413.31709778)(141.54644653,413.42710297)
\curveto(141.55644093,413.45709764)(141.55644093,413.48209762)(141.54644653,413.50210297)
\curveto(141.54644094,413.52209758)(141.55144093,413.54209756)(141.56144653,413.56210297)
\curveto(141.57144091,413.58209752)(141.57644091,413.61209749)(141.57644653,413.65210297)
\lineto(141.57644653,413.74210297)
\lineto(141.54644653,413.86210297)
\curveto(141.54644094,413.9020972)(141.54144094,413.93709716)(141.53144653,413.96710297)
\curveto(141.43144105,414.26709683)(141.22144126,414.47209663)(140.90144653,414.58210297)
\curveto(140.81144167,414.61209649)(140.70144178,414.63209647)(140.57144653,414.64210297)
\curveto(140.45144203,414.66209644)(140.32644216,414.66709643)(140.19644653,414.65710297)
\curveto(140.06644242,414.65709644)(139.94144254,414.64709645)(139.82144653,414.62710297)
\curveto(139.70144278,414.60709649)(139.59644289,414.58209652)(139.50644653,414.55210297)
\curveto(139.44644304,414.53209657)(139.3864431,414.5020966)(139.32644653,414.46210297)
\curveto(139.27644321,414.43209667)(139.22644326,414.3970967)(139.17644653,414.35710297)
\curveto(139.12644336,414.31709678)(139.07144341,414.26209684)(139.01144653,414.19210297)
\curveto(138.96144352,414.12209698)(138.92644356,414.05709704)(138.90644653,413.99710297)
\curveto(138.85644363,413.8970972)(138.81144367,413.8020973)(138.77144653,413.71210297)
\curveto(138.74144374,413.62209748)(138.67144381,413.56209754)(138.56144653,413.53210297)
\curveto(138.481444,413.51209759)(138.39644409,413.5020976)(138.30644653,413.50210297)
\lineto(138.03644653,413.50210297)
\lineto(137.46644653,413.50210297)
\curveto(137.41644507,413.5020976)(137.36644512,413.4970976)(137.31644653,413.48710297)
\curveto(137.26644522,413.48709761)(137.22144526,413.49209761)(137.18144653,413.50210297)
\lineto(137.04644653,413.50210297)
\curveto(137.02644546,413.51209759)(137.00144548,413.51709758)(136.97144653,413.51710297)
\curveto(136.94144554,413.51709758)(136.91644557,413.52709757)(136.89644653,413.54710297)
\curveto(136.81644567,413.56709753)(136.76144572,413.63209747)(136.73144653,413.74210297)
\curveto(136.72144576,413.79209731)(136.72144576,413.84209726)(136.73144653,413.89210297)
\curveto(136.74144574,413.94209716)(136.75144573,413.98709711)(136.76144653,414.02710297)
\curveto(136.79144569,414.13709696)(136.82144566,414.23709686)(136.85144653,414.32710297)
\curveto(136.89144559,414.42709667)(136.93644555,414.51709658)(136.98644653,414.59710297)
\lineto(137.07644653,414.74710297)
\lineto(137.16644653,414.89710297)
\curveto(137.24644524,415.00709609)(137.34644514,415.11209599)(137.46644653,415.21210297)
\curveto(137.486445,415.22209588)(137.51644497,415.24709585)(137.55644653,415.28710297)
\curveto(137.60644488,415.32709577)(137.65144483,415.36209574)(137.69144653,415.39210297)
\curveto(137.73144475,415.42209568)(137.77644471,415.45209565)(137.82644653,415.48210297)
\curveto(137.99644449,415.59209551)(138.17644431,415.67709542)(138.36644653,415.73710297)
\curveto(138.55644393,415.80709529)(138.75144373,415.87209523)(138.95144653,415.93210297)
\curveto(139.07144341,415.96209514)(139.19644329,415.98209512)(139.32644653,415.99210297)
\curveto(139.45644303,416.0020951)(139.5864429,416.02209508)(139.71644653,416.05210297)
\curveto(139.75644273,416.06209504)(139.81644267,416.06209504)(139.89644653,416.05210297)
\curveto(139.9864425,416.04209506)(140.04144244,416.04709505)(140.06144653,416.06710297)
\curveto(140.47144201,416.07709502)(140.86144162,416.06209504)(141.23144653,416.02210297)
\curveto(141.61144087,415.98209512)(141.95144053,415.90709519)(142.25144653,415.79710297)
\curveto(142.56143992,415.68709541)(142.82643966,415.53709556)(143.04644653,415.34710297)
\curveto(143.26643922,415.16709593)(143.43643905,414.93209617)(143.55644653,414.64210297)
\curveto(143.62643886,414.47209663)(143.66643882,414.27709682)(143.67644653,414.05710297)
\curveto(143.6864388,413.83709726)(143.69143879,413.61209749)(143.69144653,413.38210297)
\lineto(143.69144653,410.03710297)
\lineto(143.69144653,409.45210297)
\curveto(143.69143879,409.26210184)(143.71143877,409.08710201)(143.75144653,408.92710297)
\curveto(143.76143872,408.8971022)(143.76643872,408.86210224)(143.76644653,408.82210297)
\curveto(143.76643872,408.79210231)(143.77143871,408.76210234)(143.78144653,408.73210297)
\moveto(141.57644653,411.04210297)
\curveto(141.5864409,411.09210001)(141.59144089,411.14709995)(141.59144653,411.20710297)
\curveto(141.59144089,411.27709982)(141.5864409,411.33709976)(141.57644653,411.38710297)
\curveto(141.55644093,411.44709965)(141.54644094,411.5020996)(141.54644653,411.55210297)
\curveto(141.54644094,411.6020995)(141.52644096,411.64209946)(141.48644653,411.67210297)
\curveto(141.43644105,411.71209939)(141.36144112,411.73209937)(141.26144653,411.73210297)
\curveto(141.22144126,411.72209938)(141.1864413,411.71209939)(141.15644653,411.70210297)
\curveto(141.12644136,411.7020994)(141.09144139,411.6970994)(141.05144653,411.68710297)
\curveto(140.9814415,411.66709943)(140.90644158,411.65209945)(140.82644653,411.64210297)
\curveto(140.74644174,411.63209947)(140.66644182,411.61709948)(140.58644653,411.59710297)
\curveto(140.55644193,411.58709951)(140.51144197,411.58209952)(140.45144653,411.58210297)
\curveto(140.32144216,411.55209955)(140.19144229,411.53209957)(140.06144653,411.52210297)
\curveto(139.93144255,411.51209959)(139.80644268,411.48709961)(139.68644653,411.44710297)
\curveto(139.60644288,411.42709967)(139.53144295,411.40709969)(139.46144653,411.38710297)
\curveto(139.39144309,411.37709972)(139.32144316,411.35709974)(139.25144653,411.32710297)
\curveto(139.04144344,411.23709986)(138.86144362,411.1021)(138.71144653,410.92210297)
\curveto(138.57144391,410.74210036)(138.52144396,410.49210061)(138.56144653,410.17210297)
\curveto(138.5814439,410.0021011)(138.63644385,409.86210124)(138.72644653,409.75210297)
\curveto(138.79644369,409.64210146)(138.90144358,409.55210155)(139.04144653,409.48210297)
\curveto(139.1814433,409.42210168)(139.33144315,409.37710172)(139.49144653,409.34710297)
\curveto(139.66144282,409.31710178)(139.83644265,409.30710179)(140.01644653,409.31710297)
\curveto(140.20644228,409.33710176)(140.3814421,409.37210173)(140.54144653,409.42210297)
\curveto(140.80144168,409.5021016)(141.00644148,409.62710147)(141.15644653,409.79710297)
\curveto(141.30644118,409.97710112)(141.42144106,410.1971009)(141.50144653,410.45710297)
\curveto(141.52144096,410.52710057)(141.53144095,410.5971005)(141.53144653,410.66710297)
\curveto(141.54144094,410.74710035)(141.55644093,410.82710027)(141.57644653,410.90710297)
\lineto(141.57644653,411.04210297)
}
}
{
\newrgbcolor{curcolor}{0 0 0}
\pscustom[linestyle=none,fillstyle=solid,fillcolor=curcolor]
{
\newpath
\moveto(152.93472778,408.98710297)
\lineto(152.93472778,408.56710297)
\curveto(152.93471941,408.43710266)(152.90471944,408.33210277)(152.84472778,408.25210297)
\curveto(152.79471955,408.2021029)(152.72971962,408.16710293)(152.64972778,408.14710297)
\curveto(152.56971978,408.13710296)(152.47971987,408.13210297)(152.37972778,408.13210297)
\lineto(151.55472778,408.13210297)
\lineto(151.26972778,408.13210297)
\curveto(151.18972116,408.14210296)(151.12472122,408.16710293)(151.07472778,408.20710297)
\curveto(151.00472134,408.25710284)(150.96472138,408.32210278)(150.95472778,408.40210297)
\curveto(150.9447214,408.48210262)(150.92472142,408.56210254)(150.89472778,408.64210297)
\curveto(150.87472147,408.66210244)(150.85472149,408.67710242)(150.83472778,408.68710297)
\curveto(150.82472152,408.70710239)(150.80972154,408.72710237)(150.78972778,408.74710297)
\curveto(150.67972167,408.74710235)(150.59972175,408.72210238)(150.54972778,408.67210297)
\lineto(150.39972778,408.52210297)
\curveto(150.32972202,408.47210263)(150.26472208,408.42710267)(150.20472778,408.38710297)
\curveto(150.1447222,408.35710274)(150.07972227,408.31710278)(150.00972778,408.26710297)
\curveto(149.96972238,408.24710285)(149.92472242,408.22710287)(149.87472778,408.20710297)
\curveto(149.83472251,408.18710291)(149.78972256,408.16710293)(149.73972778,408.14710297)
\curveto(149.59972275,408.097103)(149.4497229,408.05210305)(149.28972778,408.01210297)
\curveto(149.23972311,407.99210311)(149.19472315,407.98210312)(149.15472778,407.98210297)
\curveto(149.11472323,407.98210312)(149.07472327,407.97710312)(149.03472778,407.96710297)
\lineto(148.89972778,407.96710297)
\curveto(148.86972348,407.95710314)(148.82972352,407.95210315)(148.77972778,407.95210297)
\lineto(148.64472778,407.95210297)
\curveto(148.58472376,407.93210317)(148.49472385,407.92710317)(148.37472778,407.93710297)
\curveto(148.25472409,407.93710316)(148.16972418,407.94710315)(148.11972778,407.96710297)
\curveto(148.0497243,407.98710311)(147.98472436,407.9971031)(147.92472778,407.99710297)
\curveto(147.87472447,407.98710311)(147.81972453,407.99210311)(147.75972778,408.01210297)
\lineto(147.39972778,408.13210297)
\curveto(147.28972506,408.16210294)(147.17972517,408.2021029)(147.06972778,408.25210297)
\curveto(146.71972563,408.4021027)(146.40472594,408.63210247)(146.12472778,408.94210297)
\curveto(145.85472649,409.26210184)(145.63972671,409.5971015)(145.47972778,409.94710297)
\curveto(145.42972692,410.05710104)(145.38972696,410.16210094)(145.35972778,410.26210297)
\curveto(145.32972702,410.37210073)(145.29472705,410.48210062)(145.25472778,410.59210297)
\curveto(145.2447271,410.63210047)(145.23972711,410.66710043)(145.23972778,410.69710297)
\curveto(145.23972711,410.73710036)(145.22972712,410.78210032)(145.20972778,410.83210297)
\curveto(145.18972716,410.91210019)(145.16972718,410.9971001)(145.14972778,411.08710297)
\curveto(145.13972721,411.18709991)(145.12472722,411.28709981)(145.10472778,411.38710297)
\curveto(145.09472725,411.41709968)(145.08972726,411.45209965)(145.08972778,411.49210297)
\curveto(145.09972725,411.53209957)(145.09972725,411.56709953)(145.08972778,411.59710297)
\lineto(145.08972778,411.73210297)
\curveto(145.08972726,411.78209932)(145.08472726,411.83209927)(145.07472778,411.88210297)
\curveto(145.06472728,411.93209917)(145.05972729,411.98709911)(145.05972778,412.04710297)
\curveto(145.05972729,412.11709898)(145.06472728,412.17209893)(145.07472778,412.21210297)
\curveto(145.08472726,412.26209884)(145.08972726,412.30709879)(145.08972778,412.34710297)
\lineto(145.08972778,412.49710297)
\curveto(145.09972725,412.54709855)(145.09972725,412.59209851)(145.08972778,412.63210297)
\curveto(145.08972726,412.68209842)(145.09972725,412.73209837)(145.11972778,412.78210297)
\curveto(145.13972721,412.89209821)(145.15472719,412.9970981)(145.16472778,413.09710297)
\curveto(145.18472716,413.1970979)(145.20972714,413.2970978)(145.23972778,413.39710297)
\curveto(145.27972707,413.51709758)(145.31472703,413.63209747)(145.34472778,413.74210297)
\curveto(145.37472697,413.85209725)(145.41472693,413.96209714)(145.46472778,414.07210297)
\curveto(145.60472674,414.37209673)(145.77972657,414.65709644)(145.98972778,414.92710297)
\curveto(146.00972634,414.95709614)(146.03472631,414.98209612)(146.06472778,415.00210297)
\curveto(146.10472624,415.03209607)(146.13472621,415.06209604)(146.15472778,415.09210297)
\curveto(146.19472615,415.14209596)(146.23472611,415.18709591)(146.27472778,415.22710297)
\curveto(146.31472603,415.26709583)(146.35972599,415.30709579)(146.40972778,415.34710297)
\curveto(146.4497259,415.36709573)(146.48472586,415.39209571)(146.51472778,415.42210297)
\curveto(146.5447258,415.46209564)(146.57972577,415.49209561)(146.61972778,415.51210297)
\curveto(146.86972548,415.68209542)(147.15972519,415.82209528)(147.48972778,415.93210297)
\curveto(147.55972479,415.95209515)(147.62972472,415.96709513)(147.69972778,415.97710297)
\curveto(147.77972457,415.98709511)(147.85972449,416.0020951)(147.93972778,416.02210297)
\curveto(148.00972434,416.04209506)(148.09972425,416.05209505)(148.20972778,416.05210297)
\curveto(148.31972403,416.06209504)(148.42972392,416.06709503)(148.53972778,416.06710297)
\curveto(148.6497237,416.06709503)(148.75472359,416.06209504)(148.85472778,416.05210297)
\curveto(148.96472338,416.04209506)(149.05472329,416.02709507)(149.12472778,416.00710297)
\curveto(149.27472307,415.95709514)(149.41972293,415.91209519)(149.55972778,415.87210297)
\curveto(149.69972265,415.83209527)(149.82972252,415.77709532)(149.94972778,415.70710297)
\curveto(150.01972233,415.65709544)(150.08472226,415.60709549)(150.14472778,415.55710297)
\curveto(150.20472214,415.51709558)(150.26972208,415.47209563)(150.33972778,415.42210297)
\curveto(150.37972197,415.39209571)(150.43472191,415.35209575)(150.50472778,415.30210297)
\curveto(150.58472176,415.25209585)(150.65972169,415.25209585)(150.72972778,415.30210297)
\curveto(150.76972158,415.32209578)(150.78972156,415.35709574)(150.78972778,415.40710297)
\curveto(150.78972156,415.45709564)(150.79972155,415.50709559)(150.81972778,415.55710297)
\lineto(150.81972778,415.70710297)
\curveto(150.82972152,415.73709536)(150.83472151,415.77209533)(150.83472778,415.81210297)
\lineto(150.83472778,415.93210297)
\lineto(150.83472778,417.97210297)
\curveto(150.83472151,418.08209302)(150.82972152,418.2020929)(150.81972778,418.33210297)
\curveto(150.81972153,418.47209263)(150.8447215,418.57709252)(150.89472778,418.64710297)
\curveto(150.93472141,418.72709237)(151.00972134,418.77709232)(151.11972778,418.79710297)
\curveto(151.13972121,418.80709229)(151.15972119,418.80709229)(151.17972778,418.79710297)
\curveto(151.19972115,418.7970923)(151.21972113,418.8020923)(151.23972778,418.81210297)
\lineto(152.30472778,418.81210297)
\curveto(152.42471992,418.81209229)(152.53471981,418.80709229)(152.63472778,418.79710297)
\curveto(152.73471961,418.78709231)(152.80971954,418.74709235)(152.85972778,418.67710297)
\curveto(152.90971944,418.5970925)(152.93471941,418.49209261)(152.93472778,418.36210297)
\lineto(152.93472778,418.00210297)
\lineto(152.93472778,408.98710297)
\moveto(150.89472778,411.92710297)
\curveto(150.90472144,411.96709913)(150.90472144,412.00709909)(150.89472778,412.04710297)
\lineto(150.89472778,412.18210297)
\curveto(150.89472145,412.28209882)(150.88972146,412.38209872)(150.87972778,412.48210297)
\curveto(150.86972148,412.58209852)(150.85472149,412.67209843)(150.83472778,412.75210297)
\curveto(150.81472153,412.86209824)(150.79472155,412.96209814)(150.77472778,413.05210297)
\curveto(150.76472158,413.14209796)(150.73972161,413.22709787)(150.69972778,413.30710297)
\curveto(150.55972179,413.66709743)(150.35472199,413.95209715)(150.08472778,414.16210297)
\curveto(149.82472252,414.37209673)(149.4447229,414.47709662)(148.94472778,414.47710297)
\curveto(148.88472346,414.47709662)(148.80472354,414.46709663)(148.70472778,414.44710297)
\curveto(148.62472372,414.42709667)(148.5497238,414.40709669)(148.47972778,414.38710297)
\curveto(148.41972393,414.37709672)(148.35972399,414.35709674)(148.29972778,414.32710297)
\curveto(148.02972432,414.21709688)(147.81972453,414.04709705)(147.66972778,413.81710297)
\curveto(147.51972483,413.58709751)(147.39972495,413.32709777)(147.30972778,413.03710297)
\curveto(147.27972507,412.93709816)(147.25972509,412.83709826)(147.24972778,412.73710297)
\curveto(147.23972511,412.63709846)(147.21972513,412.53209857)(147.18972778,412.42210297)
\lineto(147.18972778,412.21210297)
\curveto(147.16972518,412.12209898)(147.16472518,411.9970991)(147.17472778,411.83710297)
\curveto(147.18472516,411.68709941)(147.19972515,411.57709952)(147.21972778,411.50710297)
\lineto(147.21972778,411.41710297)
\curveto(147.22972512,411.3970997)(147.23472511,411.37709972)(147.23472778,411.35710297)
\curveto(147.25472509,411.27709982)(147.26972508,411.2020999)(147.27972778,411.13210297)
\curveto(147.29972505,411.06210004)(147.31972503,410.98710011)(147.33972778,410.90710297)
\curveto(147.50972484,410.38710071)(147.79972455,410.0021011)(148.20972778,409.75210297)
\curveto(148.33972401,409.66210144)(148.51972383,409.59210151)(148.74972778,409.54210297)
\curveto(148.78972356,409.53210157)(148.8497235,409.52710157)(148.92972778,409.52710297)
\curveto(148.95972339,409.51710158)(149.00472334,409.50710159)(149.06472778,409.49710297)
\curveto(149.13472321,409.4971016)(149.18972316,409.5021016)(149.22972778,409.51210297)
\curveto(149.30972304,409.53210157)(149.38972296,409.54710155)(149.46972778,409.55710297)
\curveto(149.5497228,409.56710153)(149.62972272,409.58710151)(149.70972778,409.61710297)
\curveto(149.95972239,409.72710137)(150.15972219,409.86710123)(150.30972778,410.03710297)
\curveto(150.45972189,410.20710089)(150.58972176,410.42210068)(150.69972778,410.68210297)
\curveto(150.73972161,410.77210033)(150.76972158,410.86210024)(150.78972778,410.95210297)
\curveto(150.80972154,411.05210005)(150.82972152,411.15709994)(150.84972778,411.26710297)
\curveto(150.85972149,411.31709978)(150.85972149,411.36209974)(150.84972778,411.40210297)
\curveto(150.8497215,411.45209965)(150.85972149,411.5020996)(150.87972778,411.55210297)
\curveto(150.88972146,411.58209952)(150.89472145,411.61709948)(150.89472778,411.65710297)
\lineto(150.89472778,411.79210297)
\lineto(150.89472778,411.92710297)
}
}
{
\newrgbcolor{curcolor}{0 0 0}
\pscustom[linestyle=none,fillstyle=solid,fillcolor=curcolor]
{
}
}
{
\newrgbcolor{curcolor}{0 0 0}
\pscustom[linestyle=none,fillstyle=solid,fillcolor=curcolor]
{
\newpath
\moveto(166.26480591,408.98710297)
\lineto(166.26480591,408.56710297)
\curveto(166.26479754,408.43710266)(166.23479757,408.33210277)(166.17480591,408.25210297)
\curveto(166.12479768,408.2021029)(166.05979774,408.16710293)(165.97980591,408.14710297)
\curveto(165.8997979,408.13710296)(165.80979799,408.13210297)(165.70980591,408.13210297)
\lineto(164.88480591,408.13210297)
\lineto(164.59980591,408.13210297)
\curveto(164.51979928,408.14210296)(164.45479935,408.16710293)(164.40480591,408.20710297)
\curveto(164.33479947,408.25710284)(164.29479951,408.32210278)(164.28480591,408.40210297)
\curveto(164.27479953,408.48210262)(164.25479955,408.56210254)(164.22480591,408.64210297)
\curveto(164.2047996,408.66210244)(164.18479962,408.67710242)(164.16480591,408.68710297)
\curveto(164.15479965,408.70710239)(164.13979966,408.72710237)(164.11980591,408.74710297)
\curveto(164.00979979,408.74710235)(163.92979987,408.72210238)(163.87980591,408.67210297)
\lineto(163.72980591,408.52210297)
\curveto(163.65980014,408.47210263)(163.59480021,408.42710267)(163.53480591,408.38710297)
\curveto(163.47480033,408.35710274)(163.40980039,408.31710278)(163.33980591,408.26710297)
\curveto(163.2998005,408.24710285)(163.25480055,408.22710287)(163.20480591,408.20710297)
\curveto(163.16480064,408.18710291)(163.11980068,408.16710293)(163.06980591,408.14710297)
\curveto(162.92980087,408.097103)(162.77980102,408.05210305)(162.61980591,408.01210297)
\curveto(162.56980123,407.99210311)(162.52480128,407.98210312)(162.48480591,407.98210297)
\curveto(162.44480136,407.98210312)(162.4048014,407.97710312)(162.36480591,407.96710297)
\lineto(162.22980591,407.96710297)
\curveto(162.1998016,407.95710314)(162.15980164,407.95210315)(162.10980591,407.95210297)
\lineto(161.97480591,407.95210297)
\curveto(161.91480189,407.93210317)(161.82480198,407.92710317)(161.70480591,407.93710297)
\curveto(161.58480222,407.93710316)(161.4998023,407.94710315)(161.44980591,407.96710297)
\curveto(161.37980242,407.98710311)(161.31480249,407.9971031)(161.25480591,407.99710297)
\curveto(161.2048026,407.98710311)(161.14980265,407.99210311)(161.08980591,408.01210297)
\lineto(160.72980591,408.13210297)
\curveto(160.61980318,408.16210294)(160.50980329,408.2021029)(160.39980591,408.25210297)
\curveto(160.04980375,408.4021027)(159.73480407,408.63210247)(159.45480591,408.94210297)
\curveto(159.18480462,409.26210184)(158.96980483,409.5971015)(158.80980591,409.94710297)
\curveto(158.75980504,410.05710104)(158.71980508,410.16210094)(158.68980591,410.26210297)
\curveto(158.65980514,410.37210073)(158.62480518,410.48210062)(158.58480591,410.59210297)
\curveto(158.57480523,410.63210047)(158.56980523,410.66710043)(158.56980591,410.69710297)
\curveto(158.56980523,410.73710036)(158.55980524,410.78210032)(158.53980591,410.83210297)
\curveto(158.51980528,410.91210019)(158.4998053,410.9971001)(158.47980591,411.08710297)
\curveto(158.46980533,411.18709991)(158.45480535,411.28709981)(158.43480591,411.38710297)
\curveto(158.42480538,411.41709968)(158.41980538,411.45209965)(158.41980591,411.49210297)
\curveto(158.42980537,411.53209957)(158.42980537,411.56709953)(158.41980591,411.59710297)
\lineto(158.41980591,411.73210297)
\curveto(158.41980538,411.78209932)(158.41480539,411.83209927)(158.40480591,411.88210297)
\curveto(158.39480541,411.93209917)(158.38980541,411.98709911)(158.38980591,412.04710297)
\curveto(158.38980541,412.11709898)(158.39480541,412.17209893)(158.40480591,412.21210297)
\curveto(158.41480539,412.26209884)(158.41980538,412.30709879)(158.41980591,412.34710297)
\lineto(158.41980591,412.49710297)
\curveto(158.42980537,412.54709855)(158.42980537,412.59209851)(158.41980591,412.63210297)
\curveto(158.41980538,412.68209842)(158.42980537,412.73209837)(158.44980591,412.78210297)
\curveto(158.46980533,412.89209821)(158.48480532,412.9970981)(158.49480591,413.09710297)
\curveto(158.51480529,413.1970979)(158.53980526,413.2970978)(158.56980591,413.39710297)
\curveto(158.60980519,413.51709758)(158.64480516,413.63209747)(158.67480591,413.74210297)
\curveto(158.7048051,413.85209725)(158.74480506,413.96209714)(158.79480591,414.07210297)
\curveto(158.93480487,414.37209673)(159.10980469,414.65709644)(159.31980591,414.92710297)
\curveto(159.33980446,414.95709614)(159.36480444,414.98209612)(159.39480591,415.00210297)
\curveto(159.43480437,415.03209607)(159.46480434,415.06209604)(159.48480591,415.09210297)
\curveto(159.52480428,415.14209596)(159.56480424,415.18709591)(159.60480591,415.22710297)
\curveto(159.64480416,415.26709583)(159.68980411,415.30709579)(159.73980591,415.34710297)
\curveto(159.77980402,415.36709573)(159.81480399,415.39209571)(159.84480591,415.42210297)
\curveto(159.87480393,415.46209564)(159.90980389,415.49209561)(159.94980591,415.51210297)
\curveto(160.1998036,415.68209542)(160.48980331,415.82209528)(160.81980591,415.93210297)
\curveto(160.88980291,415.95209515)(160.95980284,415.96709513)(161.02980591,415.97710297)
\curveto(161.10980269,415.98709511)(161.18980261,416.0020951)(161.26980591,416.02210297)
\curveto(161.33980246,416.04209506)(161.42980237,416.05209505)(161.53980591,416.05210297)
\curveto(161.64980215,416.06209504)(161.75980204,416.06709503)(161.86980591,416.06710297)
\curveto(161.97980182,416.06709503)(162.08480172,416.06209504)(162.18480591,416.05210297)
\curveto(162.29480151,416.04209506)(162.38480142,416.02709507)(162.45480591,416.00710297)
\curveto(162.6048012,415.95709514)(162.74980105,415.91209519)(162.88980591,415.87210297)
\curveto(163.02980077,415.83209527)(163.15980064,415.77709532)(163.27980591,415.70710297)
\curveto(163.34980045,415.65709544)(163.41480039,415.60709549)(163.47480591,415.55710297)
\curveto(163.53480027,415.51709558)(163.5998002,415.47209563)(163.66980591,415.42210297)
\curveto(163.70980009,415.39209571)(163.76480004,415.35209575)(163.83480591,415.30210297)
\curveto(163.91479989,415.25209585)(163.98979981,415.25209585)(164.05980591,415.30210297)
\curveto(164.0997997,415.32209578)(164.11979968,415.35709574)(164.11980591,415.40710297)
\curveto(164.11979968,415.45709564)(164.12979967,415.50709559)(164.14980591,415.55710297)
\lineto(164.14980591,415.70710297)
\curveto(164.15979964,415.73709536)(164.16479964,415.77209533)(164.16480591,415.81210297)
\lineto(164.16480591,415.93210297)
\lineto(164.16480591,417.97210297)
\curveto(164.16479964,418.08209302)(164.15979964,418.2020929)(164.14980591,418.33210297)
\curveto(164.14979965,418.47209263)(164.17479963,418.57709252)(164.22480591,418.64710297)
\curveto(164.26479954,418.72709237)(164.33979946,418.77709232)(164.44980591,418.79710297)
\curveto(164.46979933,418.80709229)(164.48979931,418.80709229)(164.50980591,418.79710297)
\curveto(164.52979927,418.7970923)(164.54979925,418.8020923)(164.56980591,418.81210297)
\lineto(165.63480591,418.81210297)
\curveto(165.75479805,418.81209229)(165.86479794,418.80709229)(165.96480591,418.79710297)
\curveto(166.06479774,418.78709231)(166.13979766,418.74709235)(166.18980591,418.67710297)
\curveto(166.23979756,418.5970925)(166.26479754,418.49209261)(166.26480591,418.36210297)
\lineto(166.26480591,418.00210297)
\lineto(166.26480591,408.98710297)
\moveto(164.22480591,411.92710297)
\curveto(164.23479957,411.96709913)(164.23479957,412.00709909)(164.22480591,412.04710297)
\lineto(164.22480591,412.18210297)
\curveto(164.22479958,412.28209882)(164.21979958,412.38209872)(164.20980591,412.48210297)
\curveto(164.1997996,412.58209852)(164.18479962,412.67209843)(164.16480591,412.75210297)
\curveto(164.14479966,412.86209824)(164.12479968,412.96209814)(164.10480591,413.05210297)
\curveto(164.09479971,413.14209796)(164.06979973,413.22709787)(164.02980591,413.30710297)
\curveto(163.88979991,413.66709743)(163.68480012,413.95209715)(163.41480591,414.16210297)
\curveto(163.15480065,414.37209673)(162.77480103,414.47709662)(162.27480591,414.47710297)
\curveto(162.21480159,414.47709662)(162.13480167,414.46709663)(162.03480591,414.44710297)
\curveto(161.95480185,414.42709667)(161.87980192,414.40709669)(161.80980591,414.38710297)
\curveto(161.74980205,414.37709672)(161.68980211,414.35709674)(161.62980591,414.32710297)
\curveto(161.35980244,414.21709688)(161.14980265,414.04709705)(160.99980591,413.81710297)
\curveto(160.84980295,413.58709751)(160.72980307,413.32709777)(160.63980591,413.03710297)
\curveto(160.60980319,412.93709816)(160.58980321,412.83709826)(160.57980591,412.73710297)
\curveto(160.56980323,412.63709846)(160.54980325,412.53209857)(160.51980591,412.42210297)
\lineto(160.51980591,412.21210297)
\curveto(160.4998033,412.12209898)(160.49480331,411.9970991)(160.50480591,411.83710297)
\curveto(160.51480329,411.68709941)(160.52980327,411.57709952)(160.54980591,411.50710297)
\lineto(160.54980591,411.41710297)
\curveto(160.55980324,411.3970997)(160.56480324,411.37709972)(160.56480591,411.35710297)
\curveto(160.58480322,411.27709982)(160.5998032,411.2020999)(160.60980591,411.13210297)
\curveto(160.62980317,411.06210004)(160.64980315,410.98710011)(160.66980591,410.90710297)
\curveto(160.83980296,410.38710071)(161.12980267,410.0021011)(161.53980591,409.75210297)
\curveto(161.66980213,409.66210144)(161.84980195,409.59210151)(162.07980591,409.54210297)
\curveto(162.11980168,409.53210157)(162.17980162,409.52710157)(162.25980591,409.52710297)
\curveto(162.28980151,409.51710158)(162.33480147,409.50710159)(162.39480591,409.49710297)
\curveto(162.46480134,409.4971016)(162.51980128,409.5021016)(162.55980591,409.51210297)
\curveto(162.63980116,409.53210157)(162.71980108,409.54710155)(162.79980591,409.55710297)
\curveto(162.87980092,409.56710153)(162.95980084,409.58710151)(163.03980591,409.61710297)
\curveto(163.28980051,409.72710137)(163.48980031,409.86710123)(163.63980591,410.03710297)
\curveto(163.78980001,410.20710089)(163.91979988,410.42210068)(164.02980591,410.68210297)
\curveto(164.06979973,410.77210033)(164.0997997,410.86210024)(164.11980591,410.95210297)
\curveto(164.13979966,411.05210005)(164.15979964,411.15709994)(164.17980591,411.26710297)
\curveto(164.18979961,411.31709978)(164.18979961,411.36209974)(164.17980591,411.40210297)
\curveto(164.17979962,411.45209965)(164.18979961,411.5020996)(164.20980591,411.55210297)
\curveto(164.21979958,411.58209952)(164.22479958,411.61709948)(164.22480591,411.65710297)
\lineto(164.22480591,411.79210297)
\lineto(164.22480591,411.92710297)
}
}
{
\newrgbcolor{curcolor}{0 0 0}
\pscustom[linestyle=none,fillstyle=solid,fillcolor=curcolor]
{
\newpath
\moveto(175.20972778,412.07710297)
\curveto(175.22971962,411.9970991)(175.22971962,411.90709919)(175.20972778,411.80710297)
\curveto(175.18971966,411.70709939)(175.15471969,411.64209946)(175.10472778,411.61210297)
\curveto(175.05471979,411.57209953)(174.97971987,411.54209956)(174.87972778,411.52210297)
\curveto(174.78972006,411.51209959)(174.68472016,411.5020996)(174.56472778,411.49210297)
\lineto(174.21972778,411.49210297)
\curveto(174.10972074,411.5020996)(174.00972084,411.50709959)(173.91972778,411.50710297)
\lineto(170.25972778,411.50710297)
\lineto(170.04972778,411.50710297)
\curveto(169.98972486,411.50709959)(169.93472491,411.4970996)(169.88472778,411.47710297)
\curveto(169.80472504,411.43709966)(169.75472509,411.3970997)(169.73472778,411.35710297)
\curveto(169.71472513,411.33709976)(169.69472515,411.2970998)(169.67472778,411.23710297)
\curveto(169.65472519,411.18709991)(169.6497252,411.13709996)(169.65972778,411.08710297)
\curveto(169.67972517,411.02710007)(169.68972516,410.96710013)(169.68972778,410.90710297)
\curveto(169.69972515,410.85710024)(169.71472513,410.8021003)(169.73472778,410.74210297)
\curveto(169.81472503,410.5021006)(169.90972494,410.3021008)(170.01972778,410.14210297)
\curveto(170.13972471,409.99210111)(170.29972455,409.85710124)(170.49972778,409.73710297)
\curveto(170.57972427,409.68710141)(170.65972419,409.65210145)(170.73972778,409.63210297)
\curveto(170.82972402,409.62210148)(170.91972393,409.6021015)(171.00972778,409.57210297)
\curveto(171.08972376,409.55210155)(171.19972365,409.53710156)(171.33972778,409.52710297)
\curveto(171.47972337,409.51710158)(171.59972325,409.52210158)(171.69972778,409.54210297)
\lineto(171.83472778,409.54210297)
\curveto(171.93472291,409.56210154)(172.02472282,409.58210152)(172.10472778,409.60210297)
\curveto(172.19472265,409.63210147)(172.27972257,409.66210144)(172.35972778,409.69210297)
\curveto(172.45972239,409.74210136)(172.56972228,409.80710129)(172.68972778,409.88710297)
\curveto(172.81972203,409.96710113)(172.91472193,410.04710105)(172.97472778,410.12710297)
\curveto(173.02472182,410.1971009)(173.07472177,410.26210084)(173.12472778,410.32210297)
\curveto(173.18472166,410.39210071)(173.25472159,410.44210066)(173.33472778,410.47210297)
\curveto(173.43472141,410.52210058)(173.55972129,410.54210056)(173.70972778,410.53210297)
\lineto(174.14472778,410.53210297)
\lineto(174.32472778,410.53210297)
\curveto(174.39472045,410.54210056)(174.45472039,410.53710056)(174.50472778,410.51710297)
\lineto(174.65472778,410.51710297)
\curveto(174.75472009,410.4971006)(174.82472002,410.47210063)(174.86472778,410.44210297)
\curveto(174.90471994,410.42210068)(174.92471992,410.37710072)(174.92472778,410.30710297)
\curveto(174.93471991,410.23710086)(174.92971992,410.17710092)(174.90972778,410.12710297)
\curveto(174.85971999,409.98710111)(174.80472004,409.86210124)(174.74472778,409.75210297)
\curveto(174.68472016,409.64210146)(174.61472023,409.53210157)(174.53472778,409.42210297)
\curveto(174.31472053,409.09210201)(174.06472078,408.82710227)(173.78472778,408.62710297)
\curveto(173.50472134,408.42710267)(173.15472169,408.25710284)(172.73472778,408.11710297)
\curveto(172.62472222,408.07710302)(172.51472233,408.05210305)(172.40472778,408.04210297)
\curveto(172.29472255,408.03210307)(172.17972267,408.01210309)(172.05972778,407.98210297)
\curveto(172.01972283,407.97210313)(171.97472287,407.97210313)(171.92472778,407.98210297)
\curveto(171.88472296,407.98210312)(171.844723,407.97710312)(171.80472778,407.96710297)
\lineto(171.63972778,407.96710297)
\curveto(171.58972326,407.94710315)(171.52972332,407.94210316)(171.45972778,407.95210297)
\curveto(171.39972345,407.95210315)(171.3447235,407.95710314)(171.29472778,407.96710297)
\curveto(171.21472363,407.97710312)(171.1447237,407.97710312)(171.08472778,407.96710297)
\curveto(171.02472382,407.95710314)(170.95972389,407.96210314)(170.88972778,407.98210297)
\curveto(170.83972401,408.0021031)(170.78472406,408.01210309)(170.72472778,408.01210297)
\curveto(170.66472418,408.01210309)(170.60972424,408.02210308)(170.55972778,408.04210297)
\curveto(170.4497244,408.06210304)(170.33972451,408.08710301)(170.22972778,408.11710297)
\curveto(170.11972473,408.13710296)(170.01972483,408.17210293)(169.92972778,408.22210297)
\curveto(169.81972503,408.26210284)(169.71472513,408.2971028)(169.61472778,408.32710297)
\curveto(169.52472532,408.36710273)(169.43972541,408.41210269)(169.35972778,408.46210297)
\curveto(169.03972581,408.66210244)(168.75472609,408.89210221)(168.50472778,409.15210297)
\curveto(168.25472659,409.42210168)(168.0497268,409.73210137)(167.88972778,410.08210297)
\curveto(167.83972701,410.19210091)(167.79972705,410.3021008)(167.76972778,410.41210297)
\curveto(167.73972711,410.53210057)(167.69972715,410.65210045)(167.64972778,410.77210297)
\curveto(167.63972721,410.81210029)(167.63472721,410.84710025)(167.63472778,410.87710297)
\curveto(167.63472721,410.91710018)(167.62972722,410.95710014)(167.61972778,410.99710297)
\curveto(167.57972727,411.11709998)(167.55472729,411.24709985)(167.54472778,411.38710297)
\lineto(167.51472778,411.80710297)
\curveto(167.51472733,411.85709924)(167.50972734,411.91209919)(167.49972778,411.97210297)
\curveto(167.49972735,412.03209907)(167.50472734,412.08709901)(167.51472778,412.13710297)
\lineto(167.51472778,412.31710297)
\lineto(167.55972778,412.67710297)
\curveto(167.59972725,412.84709825)(167.63472721,413.01209809)(167.66472778,413.17210297)
\curveto(167.69472715,413.33209777)(167.73972711,413.48209762)(167.79972778,413.62210297)
\curveto(168.22972662,414.66209644)(168.95972589,415.3970957)(169.98972778,415.82710297)
\curveto(170.12972472,415.88709521)(170.26972458,415.92709517)(170.40972778,415.94710297)
\curveto(170.55972429,415.97709512)(170.71472413,416.01209509)(170.87472778,416.05210297)
\curveto(170.95472389,416.06209504)(171.02972382,416.06709503)(171.09972778,416.06710297)
\curveto(171.16972368,416.06709503)(171.2447236,416.07209503)(171.32472778,416.08210297)
\curveto(171.83472301,416.09209501)(172.26972258,416.03209507)(172.62972778,415.90210297)
\curveto(172.99972185,415.78209532)(173.32972152,415.62209548)(173.61972778,415.42210297)
\curveto(173.70972114,415.36209574)(173.79972105,415.29209581)(173.88972778,415.21210297)
\curveto(173.97972087,415.14209596)(174.05972079,415.06709603)(174.12972778,414.98710297)
\curveto(174.15972069,414.93709616)(174.19972065,414.8970962)(174.24972778,414.86710297)
\curveto(174.32972052,414.75709634)(174.40472044,414.64209646)(174.47472778,414.52210297)
\curveto(174.5447203,414.41209669)(174.61972023,414.2970968)(174.69972778,414.17710297)
\curveto(174.7497201,414.08709701)(174.78972006,413.99209711)(174.81972778,413.89210297)
\curveto(174.85971999,413.8020973)(174.89971995,413.7020974)(174.93972778,413.59210297)
\curveto(174.98971986,413.46209764)(175.02971982,413.32709777)(175.05972778,413.18710297)
\curveto(175.08971976,413.04709805)(175.12471972,412.90709819)(175.16472778,412.76710297)
\curveto(175.18471966,412.68709841)(175.18971966,412.5970985)(175.17972778,412.49710297)
\curveto(175.17971967,412.40709869)(175.18971966,412.32209878)(175.20972778,412.24210297)
\lineto(175.20972778,412.07710297)
\moveto(172.95972778,412.96210297)
\curveto(173.02972182,413.06209804)(173.03472181,413.18209792)(172.97472778,413.32210297)
\curveto(172.92472192,413.47209763)(172.88472196,413.58209752)(172.85472778,413.65210297)
\curveto(172.71472213,413.92209718)(172.52972232,414.12709697)(172.29972778,414.26710297)
\curveto(172.06972278,414.41709668)(171.7497231,414.4970966)(171.33972778,414.50710297)
\curveto(171.30972354,414.48709661)(171.27472357,414.48209662)(171.23472778,414.49210297)
\curveto(171.19472365,414.5020966)(171.15972369,414.5020966)(171.12972778,414.49210297)
\curveto(171.07972377,414.47209663)(171.02472382,414.45709664)(170.96472778,414.44710297)
\curveto(170.90472394,414.44709665)(170.849724,414.43709666)(170.79972778,414.41710297)
\curveto(170.35972449,414.27709682)(170.03472481,414.0020971)(169.82472778,413.59210297)
\curveto(169.80472504,413.55209755)(169.77972507,413.4970976)(169.74972778,413.42710297)
\curveto(169.72972512,413.36709773)(169.71472513,413.3020978)(169.70472778,413.23210297)
\curveto(169.69472515,413.17209793)(169.69472515,413.11209799)(169.70472778,413.05210297)
\curveto(169.72472512,412.99209811)(169.75972509,412.94209816)(169.80972778,412.90210297)
\curveto(169.88972496,412.85209825)(169.99972485,412.82709827)(170.13972778,412.82710297)
\lineto(170.54472778,412.82710297)
\lineto(172.20972778,412.82710297)
\lineto(172.64472778,412.82710297)
\curveto(172.80472204,412.83709826)(172.90972194,412.88209822)(172.95972778,412.96210297)
}
}
{
\newrgbcolor{curcolor}{0 0 0}
\pscustom[linestyle=none,fillstyle=solid,fillcolor=curcolor]
{
}
}
{
\newrgbcolor{curcolor}{0 0 0}
\pscustom[linestyle=none,fillstyle=solid,fillcolor=curcolor]
{
\newpath
\moveto(185.04316528,416.06710297)
\curveto(185.15315997,416.06709503)(185.24815987,416.05709504)(185.32816528,416.03710297)
\curveto(185.4181597,416.01709508)(185.48815963,415.97209513)(185.53816528,415.90210297)
\curveto(185.59815952,415.82209528)(185.62815949,415.68209542)(185.62816528,415.48210297)
\lineto(185.62816528,414.97210297)
\lineto(185.62816528,414.59710297)
\curveto(185.63815948,414.45709664)(185.6231595,414.34709675)(185.58316528,414.26710297)
\curveto(185.54315958,414.1970969)(185.48315964,414.15209695)(185.40316528,414.13210297)
\curveto(185.33315979,414.11209699)(185.24815987,414.102097)(185.14816528,414.10210297)
\curveto(185.05816006,414.102097)(184.95816016,414.10709699)(184.84816528,414.11710297)
\curveto(184.74816037,414.12709697)(184.65316047,414.12209698)(184.56316528,414.10210297)
\curveto(184.49316063,414.08209702)(184.4231607,414.06709703)(184.35316528,414.05710297)
\curveto(184.28316084,414.05709704)(184.2181609,414.04709705)(184.15816528,414.02710297)
\curveto(183.99816112,413.97709712)(183.83816128,413.9020972)(183.67816528,413.80210297)
\curveto(183.5181616,413.71209739)(183.39316173,413.60709749)(183.30316528,413.48710297)
\curveto(183.25316187,413.40709769)(183.19816192,413.32209778)(183.13816528,413.23210297)
\curveto(183.08816203,413.15209795)(183.03816208,413.06709803)(182.98816528,412.97710297)
\curveto(182.95816216,412.8970982)(182.92816219,412.81209829)(182.89816528,412.72210297)
\lineto(182.83816528,412.48210297)
\curveto(182.8181623,412.41209869)(182.80816231,412.33709876)(182.80816528,412.25710297)
\curveto(182.80816231,412.18709891)(182.79816232,412.11709898)(182.77816528,412.04710297)
\curveto(182.76816235,412.00709909)(182.76316236,411.96709913)(182.76316528,411.92710297)
\curveto(182.77316235,411.8970992)(182.77316235,411.86709923)(182.76316528,411.83710297)
\lineto(182.76316528,411.59710297)
\curveto(182.74316238,411.52709957)(182.73816238,411.44709965)(182.74816528,411.35710297)
\curveto(182.75816236,411.27709982)(182.76316236,411.1970999)(182.76316528,411.11710297)
\lineto(182.76316528,410.15710297)
\lineto(182.76316528,408.88210297)
\curveto(182.76316236,408.75210235)(182.75816236,408.63210247)(182.74816528,408.52210297)
\curveto(182.73816238,408.41210269)(182.70816241,408.32210278)(182.65816528,408.25210297)
\curveto(182.63816248,408.22210288)(182.60316252,408.1971029)(182.55316528,408.17710297)
\curveto(182.51316261,408.16710293)(182.46816265,408.15710294)(182.41816528,408.14710297)
\lineto(182.34316528,408.14710297)
\curveto(182.29316283,408.13710296)(182.23816288,408.13210297)(182.17816528,408.13210297)
\lineto(182.01316528,408.13210297)
\lineto(181.36816528,408.13210297)
\curveto(181.30816381,408.14210296)(181.24316388,408.14710295)(181.17316528,408.14710297)
\lineto(180.97816528,408.14710297)
\curveto(180.92816419,408.16710293)(180.87816424,408.18210292)(180.82816528,408.19210297)
\curveto(180.77816434,408.21210289)(180.74316438,408.24710285)(180.72316528,408.29710297)
\curveto(180.68316444,408.34710275)(180.65816446,408.41710268)(180.64816528,408.50710297)
\lineto(180.64816528,408.80710297)
\lineto(180.64816528,409.82710297)
\lineto(180.64816528,414.05710297)
\lineto(180.64816528,415.16710297)
\lineto(180.64816528,415.45210297)
\curveto(180.64816447,415.55209555)(180.66816445,415.63209547)(180.70816528,415.69210297)
\curveto(180.75816436,415.77209533)(180.83316429,415.82209528)(180.93316528,415.84210297)
\curveto(181.03316409,415.86209524)(181.15316397,415.87209523)(181.29316528,415.87210297)
\lineto(182.05816528,415.87210297)
\curveto(182.17816294,415.87209523)(182.28316284,415.86209524)(182.37316528,415.84210297)
\curveto(182.46316266,415.83209527)(182.53316259,415.78709531)(182.58316528,415.70710297)
\curveto(182.61316251,415.65709544)(182.62816249,415.58709551)(182.62816528,415.49710297)
\lineto(182.65816528,415.22710297)
\curveto(182.66816245,415.14709595)(182.68316244,415.07209603)(182.70316528,415.00210297)
\curveto(182.73316239,414.93209617)(182.78316234,414.8970962)(182.85316528,414.89710297)
\curveto(182.87316225,414.91709618)(182.89316223,414.92709617)(182.91316528,414.92710297)
\curveto(182.93316219,414.92709617)(182.95316217,414.93709616)(182.97316528,414.95710297)
\curveto(183.03316209,415.00709609)(183.08316204,415.06209604)(183.12316528,415.12210297)
\curveto(183.17316195,415.19209591)(183.23316189,415.25209585)(183.30316528,415.30210297)
\curveto(183.34316178,415.33209577)(183.37816174,415.36209574)(183.40816528,415.39210297)
\curveto(183.43816168,415.43209567)(183.47316165,415.46709563)(183.51316528,415.49710297)
\lineto(183.78316528,415.67710297)
\curveto(183.88316124,415.73709536)(183.98316114,415.79209531)(184.08316528,415.84210297)
\curveto(184.18316094,415.88209522)(184.28316084,415.91709518)(184.38316528,415.94710297)
\lineto(184.71316528,416.03710297)
\curveto(184.74316038,416.04709505)(184.79816032,416.04709505)(184.87816528,416.03710297)
\curveto(184.96816015,416.03709506)(185.0231601,416.04709505)(185.04316528,416.06710297)
}
}
{
\newrgbcolor{curcolor}{0 0 0}
\pscustom[linestyle=none,fillstyle=solid,fillcolor=curcolor]
{
\newpath
\moveto(193.54957153,412.07710297)
\curveto(193.56956337,411.9970991)(193.56956337,411.90709919)(193.54957153,411.80710297)
\curveto(193.52956341,411.70709939)(193.49456344,411.64209946)(193.44457153,411.61210297)
\curveto(193.39456354,411.57209953)(193.31956362,411.54209956)(193.21957153,411.52210297)
\curveto(193.12956381,411.51209959)(193.02456391,411.5020996)(192.90457153,411.49210297)
\lineto(192.55957153,411.49210297)
\curveto(192.44956449,411.5020996)(192.34956459,411.50709959)(192.25957153,411.50710297)
\lineto(188.59957153,411.50710297)
\lineto(188.38957153,411.50710297)
\curveto(188.32956861,411.50709959)(188.27456866,411.4970996)(188.22457153,411.47710297)
\curveto(188.14456879,411.43709966)(188.09456884,411.3970997)(188.07457153,411.35710297)
\curveto(188.05456888,411.33709976)(188.0345689,411.2970998)(188.01457153,411.23710297)
\curveto(187.99456894,411.18709991)(187.98956895,411.13709996)(187.99957153,411.08710297)
\curveto(188.01956892,411.02710007)(188.02956891,410.96710013)(188.02957153,410.90710297)
\curveto(188.0395689,410.85710024)(188.05456888,410.8021003)(188.07457153,410.74210297)
\curveto(188.15456878,410.5021006)(188.24956869,410.3021008)(188.35957153,410.14210297)
\curveto(188.47956846,409.99210111)(188.6395683,409.85710124)(188.83957153,409.73710297)
\curveto(188.91956802,409.68710141)(188.99956794,409.65210145)(189.07957153,409.63210297)
\curveto(189.16956777,409.62210148)(189.25956768,409.6021015)(189.34957153,409.57210297)
\curveto(189.42956751,409.55210155)(189.5395674,409.53710156)(189.67957153,409.52710297)
\curveto(189.81956712,409.51710158)(189.939567,409.52210158)(190.03957153,409.54210297)
\lineto(190.17457153,409.54210297)
\curveto(190.27456666,409.56210154)(190.36456657,409.58210152)(190.44457153,409.60210297)
\curveto(190.5345664,409.63210147)(190.61956632,409.66210144)(190.69957153,409.69210297)
\curveto(190.79956614,409.74210136)(190.90956603,409.80710129)(191.02957153,409.88710297)
\curveto(191.15956578,409.96710113)(191.25456568,410.04710105)(191.31457153,410.12710297)
\curveto(191.36456557,410.1971009)(191.41456552,410.26210084)(191.46457153,410.32210297)
\curveto(191.52456541,410.39210071)(191.59456534,410.44210066)(191.67457153,410.47210297)
\curveto(191.77456516,410.52210058)(191.89956504,410.54210056)(192.04957153,410.53210297)
\lineto(192.48457153,410.53210297)
\lineto(192.66457153,410.53210297)
\curveto(192.7345642,410.54210056)(192.79456414,410.53710056)(192.84457153,410.51710297)
\lineto(192.99457153,410.51710297)
\curveto(193.09456384,410.4971006)(193.16456377,410.47210063)(193.20457153,410.44210297)
\curveto(193.24456369,410.42210068)(193.26456367,410.37710072)(193.26457153,410.30710297)
\curveto(193.27456366,410.23710086)(193.26956367,410.17710092)(193.24957153,410.12710297)
\curveto(193.19956374,409.98710111)(193.14456379,409.86210124)(193.08457153,409.75210297)
\curveto(193.02456391,409.64210146)(192.95456398,409.53210157)(192.87457153,409.42210297)
\curveto(192.65456428,409.09210201)(192.40456453,408.82710227)(192.12457153,408.62710297)
\curveto(191.84456509,408.42710267)(191.49456544,408.25710284)(191.07457153,408.11710297)
\curveto(190.96456597,408.07710302)(190.85456608,408.05210305)(190.74457153,408.04210297)
\curveto(190.6345663,408.03210307)(190.51956642,408.01210309)(190.39957153,407.98210297)
\curveto(190.35956658,407.97210313)(190.31456662,407.97210313)(190.26457153,407.98210297)
\curveto(190.22456671,407.98210312)(190.18456675,407.97710312)(190.14457153,407.96710297)
\lineto(189.97957153,407.96710297)
\curveto(189.92956701,407.94710315)(189.86956707,407.94210316)(189.79957153,407.95210297)
\curveto(189.7395672,407.95210315)(189.68456725,407.95710314)(189.63457153,407.96710297)
\curveto(189.55456738,407.97710312)(189.48456745,407.97710312)(189.42457153,407.96710297)
\curveto(189.36456757,407.95710314)(189.29956764,407.96210314)(189.22957153,407.98210297)
\curveto(189.17956776,408.0021031)(189.12456781,408.01210309)(189.06457153,408.01210297)
\curveto(189.00456793,408.01210309)(188.94956799,408.02210308)(188.89957153,408.04210297)
\curveto(188.78956815,408.06210304)(188.67956826,408.08710301)(188.56957153,408.11710297)
\curveto(188.45956848,408.13710296)(188.35956858,408.17210293)(188.26957153,408.22210297)
\curveto(188.15956878,408.26210284)(188.05456888,408.2971028)(187.95457153,408.32710297)
\curveto(187.86456907,408.36710273)(187.77956916,408.41210269)(187.69957153,408.46210297)
\curveto(187.37956956,408.66210244)(187.09456984,408.89210221)(186.84457153,409.15210297)
\curveto(186.59457034,409.42210168)(186.38957055,409.73210137)(186.22957153,410.08210297)
\curveto(186.17957076,410.19210091)(186.1395708,410.3021008)(186.10957153,410.41210297)
\curveto(186.07957086,410.53210057)(186.0395709,410.65210045)(185.98957153,410.77210297)
\curveto(185.97957096,410.81210029)(185.97457096,410.84710025)(185.97457153,410.87710297)
\curveto(185.97457096,410.91710018)(185.96957097,410.95710014)(185.95957153,410.99710297)
\curveto(185.91957102,411.11709998)(185.89457104,411.24709985)(185.88457153,411.38710297)
\lineto(185.85457153,411.80710297)
\curveto(185.85457108,411.85709924)(185.84957109,411.91209919)(185.83957153,411.97210297)
\curveto(185.8395711,412.03209907)(185.84457109,412.08709901)(185.85457153,412.13710297)
\lineto(185.85457153,412.31710297)
\lineto(185.89957153,412.67710297)
\curveto(185.939571,412.84709825)(185.97457096,413.01209809)(186.00457153,413.17210297)
\curveto(186.0345709,413.33209777)(186.07957086,413.48209762)(186.13957153,413.62210297)
\curveto(186.56957037,414.66209644)(187.29956964,415.3970957)(188.32957153,415.82710297)
\curveto(188.46956847,415.88709521)(188.60956833,415.92709517)(188.74957153,415.94710297)
\curveto(188.89956804,415.97709512)(189.05456788,416.01209509)(189.21457153,416.05210297)
\curveto(189.29456764,416.06209504)(189.36956757,416.06709503)(189.43957153,416.06710297)
\curveto(189.50956743,416.06709503)(189.58456735,416.07209503)(189.66457153,416.08210297)
\curveto(190.17456676,416.09209501)(190.60956633,416.03209507)(190.96957153,415.90210297)
\curveto(191.3395656,415.78209532)(191.66956527,415.62209548)(191.95957153,415.42210297)
\curveto(192.04956489,415.36209574)(192.1395648,415.29209581)(192.22957153,415.21210297)
\curveto(192.31956462,415.14209596)(192.39956454,415.06709603)(192.46957153,414.98710297)
\curveto(192.49956444,414.93709616)(192.5395644,414.8970962)(192.58957153,414.86710297)
\curveto(192.66956427,414.75709634)(192.74456419,414.64209646)(192.81457153,414.52210297)
\curveto(192.88456405,414.41209669)(192.95956398,414.2970968)(193.03957153,414.17710297)
\curveto(193.08956385,414.08709701)(193.12956381,413.99209711)(193.15957153,413.89210297)
\curveto(193.19956374,413.8020973)(193.2395637,413.7020974)(193.27957153,413.59210297)
\curveto(193.32956361,413.46209764)(193.36956357,413.32709777)(193.39957153,413.18710297)
\curveto(193.42956351,413.04709805)(193.46456347,412.90709819)(193.50457153,412.76710297)
\curveto(193.52456341,412.68709841)(193.52956341,412.5970985)(193.51957153,412.49710297)
\curveto(193.51956342,412.40709869)(193.52956341,412.32209878)(193.54957153,412.24210297)
\lineto(193.54957153,412.07710297)
\moveto(191.29957153,412.96210297)
\curveto(191.36956557,413.06209804)(191.37456556,413.18209792)(191.31457153,413.32210297)
\curveto(191.26456567,413.47209763)(191.22456571,413.58209752)(191.19457153,413.65210297)
\curveto(191.05456588,413.92209718)(190.86956607,414.12709697)(190.63957153,414.26710297)
\curveto(190.40956653,414.41709668)(190.08956685,414.4970966)(189.67957153,414.50710297)
\curveto(189.64956729,414.48709661)(189.61456732,414.48209662)(189.57457153,414.49210297)
\curveto(189.5345674,414.5020966)(189.49956744,414.5020966)(189.46957153,414.49210297)
\curveto(189.41956752,414.47209663)(189.36456757,414.45709664)(189.30457153,414.44710297)
\curveto(189.24456769,414.44709665)(189.18956775,414.43709666)(189.13957153,414.41710297)
\curveto(188.69956824,414.27709682)(188.37456856,414.0020971)(188.16457153,413.59210297)
\curveto(188.14456879,413.55209755)(188.11956882,413.4970976)(188.08957153,413.42710297)
\curveto(188.06956887,413.36709773)(188.05456888,413.3020978)(188.04457153,413.23210297)
\curveto(188.0345689,413.17209793)(188.0345689,413.11209799)(188.04457153,413.05210297)
\curveto(188.06456887,412.99209811)(188.09956884,412.94209816)(188.14957153,412.90210297)
\curveto(188.22956871,412.85209825)(188.3395686,412.82709827)(188.47957153,412.82710297)
\lineto(188.88457153,412.82710297)
\lineto(190.54957153,412.82710297)
\lineto(190.98457153,412.82710297)
\curveto(191.14456579,412.83709826)(191.24956569,412.88209822)(191.29957153,412.96210297)
}
}
{
\newrgbcolor{curcolor}{0 0 0}
\pscustom[linestyle=none,fillstyle=solid,fillcolor=curcolor]
{
\newpath
\moveto(198.36785278,416.08210297)
\curveto(199.17784762,416.102095)(199.85284695,415.98209512)(200.39285278,415.72210297)
\curveto(200.94284586,415.46209564)(201.37784542,415.09209601)(201.69785278,414.61210297)
\curveto(201.85784494,414.37209673)(201.97784482,414.097097)(202.05785278,413.78710297)
\curveto(202.07784472,413.73709736)(202.09284471,413.67209743)(202.10285278,413.59210297)
\curveto(202.12284468,413.51209759)(202.12284468,413.44209766)(202.10285278,413.38210297)
\curveto(202.06284474,413.27209783)(201.99284481,413.20709789)(201.89285278,413.18710297)
\curveto(201.79284501,413.17709792)(201.67284513,413.17209793)(201.53285278,413.17210297)
\lineto(200.75285278,413.17210297)
\lineto(200.46785278,413.17210297)
\curveto(200.37784642,413.17209793)(200.3028465,413.19209791)(200.24285278,413.23210297)
\curveto(200.16284664,413.27209783)(200.10784669,413.33209777)(200.07785278,413.41210297)
\curveto(200.04784675,413.5020976)(200.00784679,413.59209751)(199.95785278,413.68210297)
\curveto(199.8978469,413.79209731)(199.83284697,413.89209721)(199.76285278,413.98210297)
\curveto(199.69284711,414.07209703)(199.61284719,414.15209695)(199.52285278,414.22210297)
\curveto(199.38284742,414.31209679)(199.22784757,414.38209672)(199.05785278,414.43210297)
\curveto(198.9978478,414.45209665)(198.93784786,414.46209664)(198.87785278,414.46210297)
\curveto(198.81784798,414.46209664)(198.76284804,414.47209663)(198.71285278,414.49210297)
\lineto(198.56285278,414.49210297)
\curveto(198.36284844,414.49209661)(198.2028486,414.47209663)(198.08285278,414.43210297)
\curveto(197.79284901,414.34209676)(197.55784924,414.2020969)(197.37785278,414.01210297)
\curveto(197.1978496,413.83209727)(197.05284975,413.61209749)(196.94285278,413.35210297)
\curveto(196.89284991,413.24209786)(196.85284995,413.12209798)(196.82285278,412.99210297)
\curveto(196.80285,412.87209823)(196.77785002,412.74209836)(196.74785278,412.60210297)
\curveto(196.73785006,412.56209854)(196.73285007,412.52209858)(196.73285278,412.48210297)
\curveto(196.73285007,412.44209866)(196.72785007,412.4020987)(196.71785278,412.36210297)
\curveto(196.6978501,412.26209884)(196.68785011,412.12209898)(196.68785278,411.94210297)
\curveto(196.6978501,411.76209934)(196.71285009,411.62209948)(196.73285278,411.52210297)
\curveto(196.73285007,411.44209966)(196.73785006,411.38709971)(196.74785278,411.35710297)
\curveto(196.76785003,411.28709981)(196.77785002,411.21709988)(196.77785278,411.14710297)
\curveto(196.78785001,411.07710002)(196.80285,411.00710009)(196.82285278,410.93710297)
\curveto(196.9028499,410.70710039)(196.9978498,410.4971006)(197.10785278,410.30710297)
\curveto(197.21784958,410.11710098)(197.35784944,409.95710114)(197.52785278,409.82710297)
\curveto(197.56784923,409.7971013)(197.62784917,409.76210134)(197.70785278,409.72210297)
\curveto(197.81784898,409.65210145)(197.92784887,409.60710149)(198.03785278,409.58710297)
\curveto(198.15784864,409.56710153)(198.3028485,409.54710155)(198.47285278,409.52710297)
\lineto(198.56285278,409.52710297)
\curveto(198.6028482,409.52710157)(198.63284817,409.53210157)(198.65285278,409.54210297)
\lineto(198.78785278,409.54210297)
\curveto(198.85784794,409.56210154)(198.92284788,409.57710152)(198.98285278,409.58710297)
\curveto(199.05284775,409.60710149)(199.11784768,409.62710147)(199.17785278,409.64710297)
\curveto(199.47784732,409.77710132)(199.70784709,409.96710113)(199.86785278,410.21710297)
\curveto(199.90784689,410.26710083)(199.94284686,410.32210078)(199.97285278,410.38210297)
\curveto(200.0028468,410.45210065)(200.02784677,410.51210059)(200.04785278,410.56210297)
\curveto(200.08784671,410.67210043)(200.12284668,410.76710033)(200.15285278,410.84710297)
\curveto(200.18284662,410.93710016)(200.25284655,411.00710009)(200.36285278,411.05710297)
\curveto(200.45284635,411.0971)(200.5978462,411.11209999)(200.79785278,411.10210297)
\lineto(201.29285278,411.10210297)
\lineto(201.50285278,411.10210297)
\curveto(201.58284522,411.11209999)(201.64784515,411.10709999)(201.69785278,411.08710297)
\lineto(201.81785278,411.08710297)
\lineto(201.93785278,411.05710297)
\curveto(201.97784482,411.05710004)(202.00784479,411.04710005)(202.02785278,411.02710297)
\curveto(202.07784472,410.98710011)(202.10784469,410.92710017)(202.11785278,410.84710297)
\curveto(202.13784466,410.77710032)(202.13784466,410.7021004)(202.11785278,410.62210297)
\curveto(202.02784477,410.29210081)(201.91784488,409.9971011)(201.78785278,409.73710297)
\curveto(201.37784542,408.96710213)(200.72284608,408.43210267)(199.82285278,408.13210297)
\curveto(199.72284708,408.102103)(199.61784718,408.08210302)(199.50785278,408.07210297)
\curveto(199.3978474,408.05210305)(199.28784751,408.02710307)(199.17785278,407.99710297)
\curveto(199.11784768,407.98710311)(199.05784774,407.98210312)(198.99785278,407.98210297)
\curveto(198.93784786,407.98210312)(198.87784792,407.97710312)(198.81785278,407.96710297)
\lineto(198.65285278,407.96710297)
\curveto(198.6028482,407.94710315)(198.52784827,407.94210316)(198.42785278,407.95210297)
\curveto(198.32784847,407.95210315)(198.25284855,407.95710314)(198.20285278,407.96710297)
\curveto(198.12284868,407.98710311)(198.04784875,407.9971031)(197.97785278,407.99710297)
\curveto(197.91784888,407.98710311)(197.85284895,407.99210311)(197.78285278,408.01210297)
\lineto(197.63285278,408.04210297)
\curveto(197.58284922,408.04210306)(197.53284927,408.04710305)(197.48285278,408.05710297)
\curveto(197.37284943,408.08710301)(197.26784953,408.11710298)(197.16785278,408.14710297)
\curveto(197.06784973,408.17710292)(196.97284983,408.21210289)(196.88285278,408.25210297)
\curveto(196.41285039,408.45210265)(196.01785078,408.70710239)(195.69785278,409.01710297)
\curveto(195.37785142,409.33710176)(195.11785168,409.73210137)(194.91785278,410.20210297)
\curveto(194.86785193,410.29210081)(194.82785197,410.38710071)(194.79785278,410.48710297)
\lineto(194.70785278,410.81710297)
\curveto(194.6978521,410.85710024)(194.69285211,410.89210021)(194.69285278,410.92210297)
\curveto(194.69285211,410.96210014)(194.68285212,411.00710009)(194.66285278,411.05710297)
\curveto(194.64285216,411.12709997)(194.63285217,411.1970999)(194.63285278,411.26710297)
\curveto(194.63285217,411.34709975)(194.62285218,411.42209968)(194.60285278,411.49210297)
\lineto(194.60285278,411.74710297)
\curveto(194.58285222,411.7970993)(194.57285223,411.85209925)(194.57285278,411.91210297)
\curveto(194.57285223,411.98209912)(194.58285222,412.04209906)(194.60285278,412.09210297)
\curveto(194.61285219,412.14209896)(194.61285219,412.18709891)(194.60285278,412.22710297)
\curveto(194.59285221,412.26709883)(194.59285221,412.30709879)(194.60285278,412.34710297)
\curveto(194.62285218,412.41709868)(194.62785217,412.48209862)(194.61785278,412.54210297)
\curveto(194.61785218,412.6020985)(194.62785217,412.66209844)(194.64785278,412.72210297)
\curveto(194.6978521,412.9020982)(194.73785206,413.07209803)(194.76785278,413.23210297)
\curveto(194.797852,413.4020977)(194.84285196,413.56709753)(194.90285278,413.72710297)
\curveto(195.12285168,414.23709686)(195.3978514,414.66209644)(195.72785278,415.00210297)
\curveto(196.06785073,415.34209576)(196.4978503,415.61709548)(197.01785278,415.82710297)
\curveto(197.15784964,415.88709521)(197.3028495,415.92709517)(197.45285278,415.94710297)
\curveto(197.6028492,415.97709512)(197.75784904,416.01209509)(197.91785278,416.05210297)
\curveto(197.9978488,416.06209504)(198.07284873,416.06709503)(198.14285278,416.06710297)
\curveto(198.21284859,416.06709503)(198.28784851,416.07209503)(198.36785278,416.08210297)
}
}
{
\newrgbcolor{curcolor}{0 0 0}
\pscustom[linestyle=none,fillstyle=solid,fillcolor=curcolor]
{
\newpath
\moveto(203.83113403,415.85710297)
\lineto(204.95613403,415.85710297)
\curveto(205.0661316,415.85709524)(205.1661315,415.85209525)(205.25613403,415.84210297)
\curveto(205.34613132,415.83209527)(205.41113125,415.7970953)(205.45113403,415.73710297)
\curveto(205.50113116,415.67709542)(205.53113113,415.59209551)(205.54113403,415.48210297)
\curveto(205.55113111,415.38209572)(205.55613111,415.27709582)(205.55613403,415.16710297)
\lineto(205.55613403,414.11710297)
\lineto(205.55613403,411.88210297)
\curveto(205.55613111,411.52209958)(205.57113109,411.18209992)(205.60113403,410.86210297)
\curveto(205.63113103,410.54210056)(205.72113094,410.27710082)(205.87113403,410.06710297)
\curveto(206.01113065,409.85710124)(206.23613043,409.70710139)(206.54613403,409.61710297)
\curveto(206.59613007,409.60710149)(206.63613003,409.6021015)(206.66613403,409.60210297)
\curveto(206.70612996,409.6021015)(206.75112991,409.5971015)(206.80113403,409.58710297)
\curveto(206.85112981,409.57710152)(206.90612976,409.57210153)(206.96613403,409.57210297)
\curveto(207.02612964,409.57210153)(207.07112959,409.57710152)(207.10113403,409.58710297)
\curveto(207.15112951,409.60710149)(207.19112947,409.61210149)(207.22113403,409.60210297)
\curveto(207.2611294,409.59210151)(207.30112936,409.5971015)(207.34113403,409.61710297)
\curveto(207.55112911,409.66710143)(207.71612895,409.73210137)(207.83613403,409.81210297)
\curveto(208.01612865,409.92210118)(208.15612851,410.06210104)(208.25613403,410.23210297)
\curveto(208.3661283,410.41210069)(208.44112822,410.60710049)(208.48113403,410.81710297)
\curveto(208.53112813,411.03710006)(208.5611281,411.27709982)(208.57113403,411.53710297)
\curveto(208.58112808,411.80709929)(208.58612808,412.08709901)(208.58613403,412.37710297)
\lineto(208.58613403,414.19210297)
\lineto(208.58613403,415.16710297)
\lineto(208.58613403,415.43710297)
\curveto(208.58612808,415.53709556)(208.60612806,415.61709548)(208.64613403,415.67710297)
\curveto(208.69612797,415.76709533)(208.77112789,415.81709528)(208.87113403,415.82710297)
\curveto(208.97112769,415.84709525)(209.09112757,415.85709524)(209.23113403,415.85710297)
\lineto(210.02613403,415.85710297)
\lineto(210.31113403,415.85710297)
\curveto(210.40112626,415.85709524)(210.47612619,415.83709526)(210.53613403,415.79710297)
\curveto(210.61612605,415.74709535)(210.661126,415.67209543)(210.67113403,415.57210297)
\curveto(210.68112598,415.47209563)(210.68612598,415.35709574)(210.68613403,415.22710297)
\lineto(210.68613403,414.08710297)
\lineto(210.68613403,409.87210297)
\lineto(210.68613403,408.80710297)
\lineto(210.68613403,408.50710297)
\curveto(210.68612598,408.40710269)(210.666126,408.33210277)(210.62613403,408.28210297)
\curveto(210.57612609,408.2021029)(210.50112616,408.15710294)(210.40113403,408.14710297)
\curveto(210.30112636,408.13710296)(210.19612647,408.13210297)(210.08613403,408.13210297)
\lineto(209.27613403,408.13210297)
\curveto(209.1661275,408.13210297)(209.0661276,408.13710296)(208.97613403,408.14710297)
\curveto(208.89612777,408.15710294)(208.83112783,408.1971029)(208.78113403,408.26710297)
\curveto(208.7611279,408.2971028)(208.74112792,408.34210276)(208.72113403,408.40210297)
\curveto(208.71112795,408.46210264)(208.69612797,408.52210258)(208.67613403,408.58210297)
\curveto(208.666128,408.64210246)(208.65112801,408.6971024)(208.63113403,408.74710297)
\curveto(208.61112805,408.7971023)(208.58112808,408.82710227)(208.54113403,408.83710297)
\curveto(208.52112814,408.85710224)(208.49612817,408.86210224)(208.46613403,408.85210297)
\curveto(208.43612823,408.84210226)(208.41112825,408.83210227)(208.39113403,408.82210297)
\curveto(208.32112834,408.78210232)(208.2611284,408.73710236)(208.21113403,408.68710297)
\curveto(208.1611285,408.63710246)(208.10612856,408.59210251)(208.04613403,408.55210297)
\curveto(208.00612866,408.52210258)(207.9661287,408.48710261)(207.92613403,408.44710297)
\curveto(207.89612877,408.41710268)(207.85612881,408.38710271)(207.80613403,408.35710297)
\curveto(207.57612909,408.21710288)(207.30612936,408.10710299)(206.99613403,408.02710297)
\curveto(206.92612974,408.00710309)(206.85612981,407.9971031)(206.78613403,407.99710297)
\curveto(206.71612995,407.98710311)(206.64113002,407.97210313)(206.56113403,407.95210297)
\curveto(206.52113014,407.94210316)(206.47613019,407.94210316)(206.42613403,407.95210297)
\curveto(206.38613028,407.95210315)(206.34613032,407.94710315)(206.30613403,407.93710297)
\curveto(206.27613039,407.92710317)(206.21113045,407.92710317)(206.11113403,407.93710297)
\curveto(206.02113064,407.93710316)(205.9611307,407.94210316)(205.93113403,407.95210297)
\curveto(205.88113078,407.95210315)(205.83113083,407.95710314)(205.78113403,407.96710297)
\lineto(205.63113403,407.96710297)
\curveto(205.51113115,407.9971031)(205.39613127,408.02210308)(205.28613403,408.04210297)
\curveto(205.17613149,408.06210304)(205.0661316,408.09210301)(204.95613403,408.13210297)
\curveto(204.90613176,408.15210295)(204.8611318,408.16710293)(204.82113403,408.17710297)
\curveto(204.79113187,408.1971029)(204.75113191,408.21710288)(204.70113403,408.23710297)
\curveto(204.35113231,408.42710267)(204.07113259,408.69210241)(203.86113403,409.03210297)
\curveto(203.73113293,409.24210186)(203.63613303,409.49210161)(203.57613403,409.78210297)
\curveto(203.51613315,410.08210102)(203.47613319,410.3971007)(203.45613403,410.72710297)
\curveto(203.44613322,411.06710003)(203.44113322,411.41209969)(203.44113403,411.76210297)
\curveto(203.45113321,412.12209898)(203.45613321,412.47709862)(203.45613403,412.82710297)
\lineto(203.45613403,414.86710297)
\curveto(203.45613321,414.9970961)(203.45113321,415.14709595)(203.44113403,415.31710297)
\curveto(203.44113322,415.4970956)(203.4661332,415.62709547)(203.51613403,415.70710297)
\curveto(203.54613312,415.75709534)(203.60613306,415.8020953)(203.69613403,415.84210297)
\curveto(203.75613291,415.84209526)(203.80113286,415.84709525)(203.83113403,415.85710297)
}
}
{
\newrgbcolor{curcolor}{0 0 0}
\pscustom[linestyle=none,fillstyle=solid,fillcolor=curcolor]
{
\newpath
\moveto(216.74238403,416.06710297)
\curveto(216.85237872,416.06709503)(216.94737862,416.05709504)(217.02738403,416.03710297)
\curveto(217.11737845,416.01709508)(217.18737838,415.97209513)(217.23738403,415.90210297)
\curveto(217.29737827,415.82209528)(217.32737824,415.68209542)(217.32738403,415.48210297)
\lineto(217.32738403,414.97210297)
\lineto(217.32738403,414.59710297)
\curveto(217.33737823,414.45709664)(217.32237825,414.34709675)(217.28238403,414.26710297)
\curveto(217.24237833,414.1970969)(217.18237839,414.15209695)(217.10238403,414.13210297)
\curveto(217.03237854,414.11209699)(216.94737862,414.102097)(216.84738403,414.10210297)
\curveto(216.75737881,414.102097)(216.65737891,414.10709699)(216.54738403,414.11710297)
\curveto(216.44737912,414.12709697)(216.35237922,414.12209698)(216.26238403,414.10210297)
\curveto(216.19237938,414.08209702)(216.12237945,414.06709703)(216.05238403,414.05710297)
\curveto(215.98237959,414.05709704)(215.91737965,414.04709705)(215.85738403,414.02710297)
\curveto(215.69737987,413.97709712)(215.53738003,413.9020972)(215.37738403,413.80210297)
\curveto(215.21738035,413.71209739)(215.09238048,413.60709749)(215.00238403,413.48710297)
\curveto(214.95238062,413.40709769)(214.89738067,413.32209778)(214.83738403,413.23210297)
\curveto(214.78738078,413.15209795)(214.73738083,413.06709803)(214.68738403,412.97710297)
\curveto(214.65738091,412.8970982)(214.62738094,412.81209829)(214.59738403,412.72210297)
\lineto(214.53738403,412.48210297)
\curveto(214.51738105,412.41209869)(214.50738106,412.33709876)(214.50738403,412.25710297)
\curveto(214.50738106,412.18709891)(214.49738107,412.11709898)(214.47738403,412.04710297)
\curveto(214.4673811,412.00709909)(214.46238111,411.96709913)(214.46238403,411.92710297)
\curveto(214.4723811,411.8970992)(214.4723811,411.86709923)(214.46238403,411.83710297)
\lineto(214.46238403,411.59710297)
\curveto(214.44238113,411.52709957)(214.43738113,411.44709965)(214.44738403,411.35710297)
\curveto(214.45738111,411.27709982)(214.46238111,411.1970999)(214.46238403,411.11710297)
\lineto(214.46238403,410.15710297)
\lineto(214.46238403,408.88210297)
\curveto(214.46238111,408.75210235)(214.45738111,408.63210247)(214.44738403,408.52210297)
\curveto(214.43738113,408.41210269)(214.40738116,408.32210278)(214.35738403,408.25210297)
\curveto(214.33738123,408.22210288)(214.30238127,408.1971029)(214.25238403,408.17710297)
\curveto(214.21238136,408.16710293)(214.1673814,408.15710294)(214.11738403,408.14710297)
\lineto(214.04238403,408.14710297)
\curveto(213.99238158,408.13710296)(213.93738163,408.13210297)(213.87738403,408.13210297)
\lineto(213.71238403,408.13210297)
\lineto(213.06738403,408.13210297)
\curveto(213.00738256,408.14210296)(212.94238263,408.14710295)(212.87238403,408.14710297)
\lineto(212.67738403,408.14710297)
\curveto(212.62738294,408.16710293)(212.57738299,408.18210292)(212.52738403,408.19210297)
\curveto(212.47738309,408.21210289)(212.44238313,408.24710285)(212.42238403,408.29710297)
\curveto(212.38238319,408.34710275)(212.35738321,408.41710268)(212.34738403,408.50710297)
\lineto(212.34738403,408.80710297)
\lineto(212.34738403,409.82710297)
\lineto(212.34738403,414.05710297)
\lineto(212.34738403,415.16710297)
\lineto(212.34738403,415.45210297)
\curveto(212.34738322,415.55209555)(212.3673832,415.63209547)(212.40738403,415.69210297)
\curveto(212.45738311,415.77209533)(212.53238304,415.82209528)(212.63238403,415.84210297)
\curveto(212.73238284,415.86209524)(212.85238272,415.87209523)(212.99238403,415.87210297)
\lineto(213.75738403,415.87210297)
\curveto(213.87738169,415.87209523)(213.98238159,415.86209524)(214.07238403,415.84210297)
\curveto(214.16238141,415.83209527)(214.23238134,415.78709531)(214.28238403,415.70710297)
\curveto(214.31238126,415.65709544)(214.32738124,415.58709551)(214.32738403,415.49710297)
\lineto(214.35738403,415.22710297)
\curveto(214.3673812,415.14709595)(214.38238119,415.07209603)(214.40238403,415.00210297)
\curveto(214.43238114,414.93209617)(214.48238109,414.8970962)(214.55238403,414.89710297)
\curveto(214.572381,414.91709618)(214.59238098,414.92709617)(214.61238403,414.92710297)
\curveto(214.63238094,414.92709617)(214.65238092,414.93709616)(214.67238403,414.95710297)
\curveto(214.73238084,415.00709609)(214.78238079,415.06209604)(214.82238403,415.12210297)
\curveto(214.8723807,415.19209591)(214.93238064,415.25209585)(215.00238403,415.30210297)
\curveto(215.04238053,415.33209577)(215.07738049,415.36209574)(215.10738403,415.39210297)
\curveto(215.13738043,415.43209567)(215.1723804,415.46709563)(215.21238403,415.49710297)
\lineto(215.48238403,415.67710297)
\curveto(215.58237999,415.73709536)(215.68237989,415.79209531)(215.78238403,415.84210297)
\curveto(215.88237969,415.88209522)(215.98237959,415.91709518)(216.08238403,415.94710297)
\lineto(216.41238403,416.03710297)
\curveto(216.44237913,416.04709505)(216.49737907,416.04709505)(216.57738403,416.03710297)
\curveto(216.6673789,416.03709506)(216.72237885,416.04709505)(216.74238403,416.06710297)
}
}
{
\newrgbcolor{curcolor}{0 0 0}
\pscustom[linestyle=none,fillstyle=solid,fillcolor=curcolor]
{
\newpath
\moveto(221.11746216,416.08210297)
\curveto(221.86745766,416.102095)(222.51745701,416.01709508)(223.06746216,415.82710297)
\curveto(223.6274559,415.64709545)(224.05245547,415.33209577)(224.34246216,414.88210297)
\curveto(224.41245511,414.77209633)(224.47245505,414.65709644)(224.52246216,414.53710297)
\curveto(224.58245494,414.42709667)(224.63245489,414.3020968)(224.67246216,414.16210297)
\curveto(224.69245483,414.102097)(224.70245482,414.03709706)(224.70246216,413.96710297)
\curveto(224.70245482,413.8970972)(224.69245483,413.83709726)(224.67246216,413.78710297)
\curveto(224.63245489,413.72709737)(224.57745495,413.68709741)(224.50746216,413.66710297)
\curveto(224.45745507,413.64709745)(224.39745513,413.63709746)(224.32746216,413.63710297)
\lineto(224.11746216,413.63710297)
\lineto(223.45746216,413.63710297)
\curveto(223.38745614,413.63709746)(223.31745621,413.63209747)(223.24746216,413.62210297)
\curveto(223.17745635,413.62209748)(223.11245641,413.63209747)(223.05246216,413.65210297)
\curveto(222.95245657,413.67209743)(222.87745665,413.71209739)(222.82746216,413.77210297)
\curveto(222.77745675,413.83209727)(222.73245679,413.89209721)(222.69246216,413.95210297)
\lineto(222.57246216,414.16210297)
\curveto(222.54245698,414.24209686)(222.49245703,414.30709679)(222.42246216,414.35710297)
\curveto(222.3224572,414.43709666)(222.2224573,414.4970966)(222.12246216,414.53710297)
\curveto(222.03245749,414.57709652)(221.91745761,414.61209649)(221.77746216,414.64210297)
\curveto(221.70745782,414.66209644)(221.60245792,414.67709642)(221.46246216,414.68710297)
\curveto(221.33245819,414.6970964)(221.23245829,414.69209641)(221.16246216,414.67210297)
\lineto(221.05746216,414.67210297)
\lineto(220.90746216,414.64210297)
\curveto(220.86745866,414.64209646)(220.8224587,414.63709646)(220.77246216,414.62710297)
\curveto(220.60245892,414.57709652)(220.46245906,414.50709659)(220.35246216,414.41710297)
\curveto(220.25245927,414.33709676)(220.18245934,414.21209689)(220.14246216,414.04210297)
\curveto(220.1224594,413.97209713)(220.1224594,413.90709719)(220.14246216,413.84710297)
\curveto(220.16245936,413.78709731)(220.18245934,413.73709736)(220.20246216,413.69710297)
\curveto(220.27245925,413.57709752)(220.35245917,413.48209762)(220.44246216,413.41210297)
\curveto(220.54245898,413.34209776)(220.65745887,413.28209782)(220.78746216,413.23210297)
\curveto(220.97745855,413.15209795)(221.18245834,413.08209802)(221.40246216,413.02210297)
\lineto(222.09246216,412.87210297)
\curveto(222.33245719,412.83209827)(222.56245696,412.78209832)(222.78246216,412.72210297)
\curveto(223.01245651,412.67209843)(223.2274563,412.60709849)(223.42746216,412.52710297)
\curveto(223.51745601,412.48709861)(223.60245592,412.45209865)(223.68246216,412.42210297)
\curveto(223.77245575,412.4020987)(223.85745567,412.36709873)(223.93746216,412.31710297)
\curveto(224.1274554,412.1970989)(224.29745523,412.06709903)(224.44746216,411.92710297)
\curveto(224.60745492,411.78709931)(224.73245479,411.61209949)(224.82246216,411.40210297)
\curveto(224.85245467,411.33209977)(224.87745465,411.26209984)(224.89746216,411.19210297)
\curveto(224.91745461,411.12209998)(224.93745459,411.04710005)(224.95746216,410.96710297)
\curveto(224.96745456,410.90710019)(224.97245455,410.81210029)(224.97246216,410.68210297)
\curveto(224.98245454,410.56210054)(224.98245454,410.46710063)(224.97246216,410.39710297)
\lineto(224.97246216,410.32210297)
\curveto(224.95245457,410.26210084)(224.93745459,410.2021009)(224.92746216,410.14210297)
\curveto(224.9274546,410.09210101)(224.9224546,410.04210106)(224.91246216,409.99210297)
\curveto(224.84245468,409.69210141)(224.73245479,409.42710167)(224.58246216,409.19710297)
\curveto(224.4224551,408.95710214)(224.2274553,408.76210234)(223.99746216,408.61210297)
\curveto(223.76745576,408.46210264)(223.50745602,408.33210277)(223.21746216,408.22210297)
\curveto(223.10745642,408.17210293)(222.98745654,408.13710296)(222.85746216,408.11710297)
\curveto(222.73745679,408.097103)(222.61745691,408.07210303)(222.49746216,408.04210297)
\curveto(222.40745712,408.02210308)(222.31245721,408.01210309)(222.21246216,408.01210297)
\curveto(222.1224574,408.0021031)(222.03245749,407.98710311)(221.94246216,407.96710297)
\lineto(221.67246216,407.96710297)
\curveto(221.61245791,407.94710315)(221.50745802,407.93710316)(221.35746216,407.93710297)
\curveto(221.21745831,407.93710316)(221.11745841,407.94710315)(221.05746216,407.96710297)
\curveto(221.0274585,407.96710313)(220.99245853,407.97210313)(220.95246216,407.98210297)
\lineto(220.84746216,407.98210297)
\curveto(220.7274588,408.0021031)(220.60745892,408.01710308)(220.48746216,408.02710297)
\curveto(220.36745916,408.03710306)(220.25245927,408.05710304)(220.14246216,408.08710297)
\curveto(219.75245977,408.1971029)(219.40746012,408.32210278)(219.10746216,408.46210297)
\curveto(218.80746072,408.61210249)(218.55246097,408.83210227)(218.34246216,409.12210297)
\curveto(218.20246132,409.31210179)(218.08246144,409.53210157)(217.98246216,409.78210297)
\curveto(217.96246156,409.84210126)(217.94246158,409.92210118)(217.92246216,410.02210297)
\curveto(217.90246162,410.07210103)(217.88746164,410.14210096)(217.87746216,410.23210297)
\curveto(217.86746166,410.32210078)(217.87246165,410.3971007)(217.89246216,410.45710297)
\curveto(217.9224616,410.52710057)(217.97246155,410.57710052)(218.04246216,410.60710297)
\curveto(218.09246143,410.62710047)(218.15246137,410.63710046)(218.22246216,410.63710297)
\lineto(218.44746216,410.63710297)
\lineto(219.15246216,410.63710297)
\lineto(219.39246216,410.63710297)
\curveto(219.47246005,410.63710046)(219.54245998,410.62710047)(219.60246216,410.60710297)
\curveto(219.71245981,410.56710053)(219.78245974,410.5021006)(219.81246216,410.41210297)
\curveto(219.85245967,410.32210078)(219.89745963,410.22710087)(219.94746216,410.12710297)
\curveto(219.96745956,410.07710102)(220.00245952,410.01210109)(220.05246216,409.93210297)
\curveto(220.11245941,409.85210125)(220.16245936,409.8021013)(220.20246216,409.78210297)
\curveto(220.3224592,409.68210142)(220.43745909,409.6021015)(220.54746216,409.54210297)
\curveto(220.65745887,409.49210161)(220.79745873,409.44210166)(220.96746216,409.39210297)
\curveto(221.01745851,409.37210173)(221.06745846,409.36210174)(221.11746216,409.36210297)
\curveto(221.16745836,409.37210173)(221.21745831,409.37210173)(221.26746216,409.36210297)
\curveto(221.34745818,409.34210176)(221.43245809,409.33210177)(221.52246216,409.33210297)
\curveto(221.6224579,409.34210176)(221.70745782,409.35710174)(221.77746216,409.37710297)
\curveto(221.8274577,409.38710171)(221.87245765,409.39210171)(221.91246216,409.39210297)
\curveto(221.96245756,409.39210171)(222.01245751,409.4021017)(222.06246216,409.42210297)
\curveto(222.20245732,409.47210163)(222.3274572,409.53210157)(222.43746216,409.60210297)
\curveto(222.55745697,409.67210143)(222.65245687,409.76210134)(222.72246216,409.87210297)
\curveto(222.77245675,409.95210115)(222.81245671,410.07710102)(222.84246216,410.24710297)
\curveto(222.86245666,410.31710078)(222.86245666,410.38210072)(222.84246216,410.44210297)
\curveto(222.8224567,410.5021006)(222.80245672,410.55210055)(222.78246216,410.59210297)
\curveto(222.71245681,410.73210037)(222.6224569,410.83710026)(222.51246216,410.90710297)
\curveto(222.41245711,410.97710012)(222.29245723,411.04210006)(222.15246216,411.10210297)
\curveto(221.96245756,411.18209992)(221.76245776,411.24709985)(221.55246216,411.29710297)
\curveto(221.34245818,411.34709975)(221.13245839,411.4020997)(220.92246216,411.46210297)
\curveto(220.84245868,411.48209962)(220.75745877,411.4970996)(220.66746216,411.50710297)
\curveto(220.58745894,411.51709958)(220.50745902,411.53209957)(220.42746216,411.55210297)
\curveto(220.10745942,411.64209946)(219.80245972,411.72709937)(219.51246216,411.80710297)
\curveto(219.2224603,411.8970992)(218.95746057,412.02709907)(218.71746216,412.19710297)
\curveto(218.43746109,412.3970987)(218.23246129,412.66709843)(218.10246216,413.00710297)
\curveto(218.08246144,413.07709802)(218.06246146,413.17209793)(218.04246216,413.29210297)
\curveto(218.0224615,413.36209774)(218.00746152,413.44709765)(217.99746216,413.54710297)
\curveto(217.98746154,413.64709745)(217.99246153,413.73709736)(218.01246216,413.81710297)
\curveto(218.03246149,413.86709723)(218.03746149,413.90709719)(218.02746216,413.93710297)
\curveto(218.01746151,413.97709712)(218.0224615,414.02209708)(218.04246216,414.07210297)
\curveto(218.06246146,414.18209692)(218.08246144,414.28209682)(218.10246216,414.37210297)
\curveto(218.13246139,414.47209663)(218.16746136,414.56709653)(218.20746216,414.65710297)
\curveto(218.33746119,414.94709615)(218.51746101,415.18209592)(218.74746216,415.36210297)
\curveto(218.97746055,415.54209556)(219.23746029,415.68709541)(219.52746216,415.79710297)
\curveto(219.63745989,415.84709525)(219.75245977,415.88209522)(219.87246216,415.90210297)
\curveto(219.99245953,415.93209517)(220.11745941,415.96209514)(220.24746216,415.99210297)
\curveto(220.30745922,416.01209509)(220.36745916,416.02209508)(220.42746216,416.02210297)
\lineto(220.60746216,416.05210297)
\curveto(220.68745884,416.06209504)(220.77245875,416.06709503)(220.86246216,416.06710297)
\curveto(220.95245857,416.06709503)(221.03745849,416.07209503)(221.11746216,416.08210297)
}
}
{
\newrgbcolor{curcolor}{0 0 0}
\pscustom[linestyle=none,fillstyle=solid,fillcolor=curcolor]
{
\newpath
\moveto(233.97410278,412.31710297)
\curveto(233.99409421,412.25709884)(234.0040942,412.17209893)(234.00410278,412.06210297)
\curveto(234.0040942,411.95209915)(233.99409421,411.86709923)(233.97410278,411.80710297)
\lineto(233.97410278,411.65710297)
\curveto(233.95409425,411.57709952)(233.94409426,411.4970996)(233.94410278,411.41710297)
\curveto(233.95409425,411.33709976)(233.94909426,411.25709984)(233.92910278,411.17710297)
\curveto(233.9090943,411.10709999)(233.89409431,411.04210006)(233.88410278,410.98210297)
\curveto(233.87409433,410.92210018)(233.86409434,410.85710024)(233.85410278,410.78710297)
\curveto(233.81409439,410.67710042)(233.77909443,410.56210054)(233.74910278,410.44210297)
\curveto(233.71909449,410.33210077)(233.67909453,410.22710087)(233.62910278,410.12710297)
\curveto(233.41909479,409.64710145)(233.14409506,409.25710184)(232.80410278,408.95710297)
\curveto(232.46409574,408.65710244)(232.05409615,408.40710269)(231.57410278,408.20710297)
\curveto(231.45409675,408.15710294)(231.32909688,408.12210298)(231.19910278,408.10210297)
\curveto(231.07909713,408.07210303)(230.95409725,408.04210306)(230.82410278,408.01210297)
\curveto(230.77409743,407.99210311)(230.71909749,407.98210312)(230.65910278,407.98210297)
\curveto(230.59909761,407.98210312)(230.54409766,407.97710312)(230.49410278,407.96710297)
\lineto(230.38910278,407.96710297)
\curveto(230.35909785,407.95710314)(230.32909788,407.95210315)(230.29910278,407.95210297)
\curveto(230.24909796,407.94210316)(230.16909804,407.93710316)(230.05910278,407.93710297)
\curveto(229.94909826,407.92710317)(229.86409834,407.93210317)(229.80410278,407.95210297)
\lineto(229.65410278,407.95210297)
\curveto(229.6040986,407.96210314)(229.54909866,407.96710313)(229.48910278,407.96710297)
\curveto(229.43909877,407.95710314)(229.38909882,407.96210314)(229.33910278,407.98210297)
\curveto(229.29909891,407.99210311)(229.25909895,407.9971031)(229.21910278,407.99710297)
\curveto(229.18909902,407.9971031)(229.14909906,408.0021031)(229.09910278,408.01210297)
\curveto(228.99909921,408.04210306)(228.89909931,408.06710303)(228.79910278,408.08710297)
\curveto(228.69909951,408.10710299)(228.6040996,408.13710296)(228.51410278,408.17710297)
\curveto(228.39409981,408.21710288)(228.27909993,408.25710284)(228.16910278,408.29710297)
\curveto(228.06910014,408.33710276)(227.96410024,408.38710271)(227.85410278,408.44710297)
\curveto(227.5041007,408.65710244)(227.204101,408.9021022)(226.95410278,409.18210297)
\curveto(226.7041015,409.46210164)(226.49410171,409.7971013)(226.32410278,410.18710297)
\curveto(226.27410193,410.27710082)(226.23410197,410.37210073)(226.20410278,410.47210297)
\curveto(226.18410202,410.57210053)(226.15910205,410.67710042)(226.12910278,410.78710297)
\curveto(226.1091021,410.83710026)(226.09910211,410.88210022)(226.09910278,410.92210297)
\curveto(226.09910211,410.96210014)(226.08910212,411.00710009)(226.06910278,411.05710297)
\curveto(226.04910216,411.13709996)(226.03910217,411.21709988)(226.03910278,411.29710297)
\curveto(226.03910217,411.38709971)(226.02910218,411.47209963)(226.00910278,411.55210297)
\curveto(225.99910221,411.6020995)(225.99410221,411.64709945)(225.99410278,411.68710297)
\lineto(225.99410278,411.82210297)
\curveto(225.97410223,411.88209922)(225.96410224,411.96709913)(225.96410278,412.07710297)
\curveto(225.97410223,412.18709891)(225.98910222,412.27209883)(226.00910278,412.33210297)
\lineto(226.00910278,412.43710297)
\curveto(226.01910219,412.48709861)(226.01910219,412.53709856)(226.00910278,412.58710297)
\curveto(226.0091022,412.64709845)(226.01910219,412.7020984)(226.03910278,412.75210297)
\curveto(226.04910216,412.8020983)(226.05410215,412.84709825)(226.05410278,412.88710297)
\curveto(226.05410215,412.93709816)(226.06410214,412.98709811)(226.08410278,413.03710297)
\curveto(226.12410208,413.16709793)(226.15910205,413.29209781)(226.18910278,413.41210297)
\curveto(226.21910199,413.54209756)(226.25910195,413.66709743)(226.30910278,413.78710297)
\curveto(226.48910172,414.1970969)(226.7041015,414.53709656)(226.95410278,414.80710297)
\curveto(227.204101,415.08709601)(227.5091007,415.34209576)(227.86910278,415.57210297)
\curveto(227.96910024,415.62209548)(228.07410013,415.66709543)(228.18410278,415.70710297)
\curveto(228.29409991,415.74709535)(228.4040998,415.79209531)(228.51410278,415.84210297)
\curveto(228.64409956,415.89209521)(228.77909943,415.92709517)(228.91910278,415.94710297)
\curveto(229.05909915,415.96709513)(229.204099,415.9970951)(229.35410278,416.03710297)
\curveto(229.43409877,416.04709505)(229.5090987,416.05209505)(229.57910278,416.05210297)
\curveto(229.64909856,416.05209505)(229.71909849,416.05709504)(229.78910278,416.06710297)
\curveto(230.36909784,416.07709502)(230.86909734,416.01709508)(231.28910278,415.88710297)
\curveto(231.71909649,415.75709534)(232.09909611,415.57709552)(232.42910278,415.34710297)
\curveto(232.53909567,415.26709583)(232.64909556,415.17709592)(232.75910278,415.07710297)
\curveto(232.87909533,414.98709611)(232.97909523,414.88709621)(233.05910278,414.77710297)
\curveto(233.13909507,414.67709642)(233.209095,414.57709652)(233.26910278,414.47710297)
\curveto(233.33909487,414.37709672)(233.4090948,414.27209683)(233.47910278,414.16210297)
\curveto(233.54909466,414.05209705)(233.6040946,413.93209717)(233.64410278,413.80210297)
\curveto(233.68409452,413.68209742)(233.72909448,413.55209755)(233.77910278,413.41210297)
\curveto(233.8090944,413.33209777)(233.83409437,413.24709785)(233.85410278,413.15710297)
\lineto(233.91410278,412.88710297)
\curveto(233.92409428,412.84709825)(233.92909428,412.80709829)(233.92910278,412.76710297)
\curveto(233.92909428,412.72709837)(233.93409427,412.68709841)(233.94410278,412.64710297)
\curveto(233.96409424,412.5970985)(233.96909424,412.54209856)(233.95910278,412.48210297)
\curveto(233.94909426,412.42209868)(233.95409425,412.36709873)(233.97410278,412.31710297)
\moveto(231.87410278,411.77710297)
\curveto(231.88409632,411.82709927)(231.88909632,411.8970992)(231.88910278,411.98710297)
\curveto(231.88909632,412.08709901)(231.88409632,412.16209894)(231.87410278,412.21210297)
\lineto(231.87410278,412.33210297)
\curveto(231.85409635,412.38209872)(231.84409636,412.43709866)(231.84410278,412.49710297)
\curveto(231.84409636,412.55709854)(231.83909637,412.61209849)(231.82910278,412.66210297)
\curveto(231.82909638,412.7020984)(231.82409638,412.73209837)(231.81410278,412.75210297)
\lineto(231.75410278,412.99210297)
\curveto(231.74409646,413.08209802)(231.72409648,413.16709793)(231.69410278,413.24710297)
\curveto(231.58409662,413.50709759)(231.45409675,413.72709737)(231.30410278,413.90710297)
\curveto(231.15409705,414.097097)(230.95409725,414.24709685)(230.70410278,414.35710297)
\curveto(230.64409756,414.37709672)(230.58409762,414.39209671)(230.52410278,414.40210297)
\curveto(230.46409774,414.42209668)(230.39909781,414.44209666)(230.32910278,414.46210297)
\curveto(230.24909796,414.48209662)(230.16409804,414.48709661)(230.07410278,414.47710297)
\lineto(229.80410278,414.47710297)
\curveto(229.77409843,414.45709664)(229.73909847,414.44709665)(229.69910278,414.44710297)
\curveto(229.65909855,414.45709664)(229.62409858,414.45709664)(229.59410278,414.44710297)
\lineto(229.38410278,414.38710297)
\curveto(229.32409888,414.37709672)(229.26909894,414.35709674)(229.21910278,414.32710297)
\curveto(228.96909924,414.21709688)(228.76409944,414.05709704)(228.60410278,413.84710297)
\curveto(228.45409975,413.64709745)(228.33409987,413.41209769)(228.24410278,413.14210297)
\curveto(228.21409999,413.04209806)(228.18910002,412.93709816)(228.16910278,412.82710297)
\curveto(228.15910005,412.71709838)(228.14410006,412.60709849)(228.12410278,412.49710297)
\curveto(228.11410009,412.44709865)(228.1091001,412.3970987)(228.10910278,412.34710297)
\lineto(228.10910278,412.19710297)
\curveto(228.08910012,412.12709897)(228.07910013,412.02209908)(228.07910278,411.88210297)
\curveto(228.08910012,411.74209936)(228.1041001,411.63709946)(228.12410278,411.56710297)
\lineto(228.12410278,411.43210297)
\curveto(228.14410006,411.35209975)(228.15910005,411.27209983)(228.16910278,411.19210297)
\curveto(228.17910003,411.12209998)(228.19410001,411.04710005)(228.21410278,410.96710297)
\curveto(228.31409989,410.66710043)(228.41909979,410.42210068)(228.52910278,410.23210297)
\curveto(228.64909956,410.05210105)(228.83409937,409.88710121)(229.08410278,409.73710297)
\curveto(229.15409905,409.68710141)(229.22909898,409.64710145)(229.30910278,409.61710297)
\curveto(229.39909881,409.58710151)(229.48909872,409.56210154)(229.57910278,409.54210297)
\curveto(229.61909859,409.53210157)(229.65409855,409.52710157)(229.68410278,409.52710297)
\curveto(229.71409849,409.53710156)(229.74909846,409.53710156)(229.78910278,409.52710297)
\lineto(229.90910278,409.49710297)
\curveto(229.95909825,409.4971016)(230.0040982,409.5021016)(230.04410278,409.51210297)
\lineto(230.16410278,409.51210297)
\curveto(230.24409796,409.53210157)(230.32409788,409.54710155)(230.40410278,409.55710297)
\curveto(230.48409772,409.56710153)(230.55909765,409.58710151)(230.62910278,409.61710297)
\curveto(230.88909732,409.71710138)(231.09909711,409.85210125)(231.25910278,410.02210297)
\curveto(231.41909679,410.19210091)(231.55409665,410.4021007)(231.66410278,410.65210297)
\curveto(231.7040965,410.75210035)(231.73409647,410.85210025)(231.75410278,410.95210297)
\curveto(231.77409643,411.05210005)(231.79909641,411.15709994)(231.82910278,411.26710297)
\curveto(231.83909637,411.30709979)(231.84409636,411.34209976)(231.84410278,411.37210297)
\curveto(231.84409636,411.41209969)(231.84909636,411.45209965)(231.85910278,411.49210297)
\lineto(231.85910278,411.62710297)
\curveto(231.85909635,411.67709942)(231.86409634,411.72709937)(231.87410278,411.77710297)
}
}
{
\newrgbcolor{curcolor}{0 0 0}
\pscustom[linestyle=none,fillstyle=solid,fillcolor=curcolor]
{
\newpath
\moveto(238.34402466,416.08210297)
\curveto(239.09402016,416.102095)(239.74401951,416.01709508)(240.29402466,415.82710297)
\curveto(240.8540184,415.64709545)(241.27901797,415.33209577)(241.56902466,414.88210297)
\curveto(241.63901761,414.77209633)(241.69901755,414.65709644)(241.74902466,414.53710297)
\curveto(241.80901744,414.42709667)(241.85901739,414.3020968)(241.89902466,414.16210297)
\curveto(241.91901733,414.102097)(241.92901732,414.03709706)(241.92902466,413.96710297)
\curveto(241.92901732,413.8970972)(241.91901733,413.83709726)(241.89902466,413.78710297)
\curveto(241.85901739,413.72709737)(241.80401745,413.68709741)(241.73402466,413.66710297)
\curveto(241.68401757,413.64709745)(241.62401763,413.63709746)(241.55402466,413.63710297)
\lineto(241.34402466,413.63710297)
\lineto(240.68402466,413.63710297)
\curveto(240.61401864,413.63709746)(240.54401871,413.63209747)(240.47402466,413.62210297)
\curveto(240.40401885,413.62209748)(240.33901891,413.63209747)(240.27902466,413.65210297)
\curveto(240.17901907,413.67209743)(240.10401915,413.71209739)(240.05402466,413.77210297)
\curveto(240.00401925,413.83209727)(239.95901929,413.89209721)(239.91902466,413.95210297)
\lineto(239.79902466,414.16210297)
\curveto(239.76901948,414.24209686)(239.71901953,414.30709679)(239.64902466,414.35710297)
\curveto(239.5490197,414.43709666)(239.4490198,414.4970966)(239.34902466,414.53710297)
\curveto(239.25901999,414.57709652)(239.14402011,414.61209649)(239.00402466,414.64210297)
\curveto(238.93402032,414.66209644)(238.82902042,414.67709642)(238.68902466,414.68710297)
\curveto(238.55902069,414.6970964)(238.45902079,414.69209641)(238.38902466,414.67210297)
\lineto(238.28402466,414.67210297)
\lineto(238.13402466,414.64210297)
\curveto(238.09402116,414.64209646)(238.0490212,414.63709646)(237.99902466,414.62710297)
\curveto(237.82902142,414.57709652)(237.68902156,414.50709659)(237.57902466,414.41710297)
\curveto(237.47902177,414.33709676)(237.40902184,414.21209689)(237.36902466,414.04210297)
\curveto(237.3490219,413.97209713)(237.3490219,413.90709719)(237.36902466,413.84710297)
\curveto(237.38902186,413.78709731)(237.40902184,413.73709736)(237.42902466,413.69710297)
\curveto(237.49902175,413.57709752)(237.57902167,413.48209762)(237.66902466,413.41210297)
\curveto(237.76902148,413.34209776)(237.88402137,413.28209782)(238.01402466,413.23210297)
\curveto(238.20402105,413.15209795)(238.40902084,413.08209802)(238.62902466,413.02210297)
\lineto(239.31902466,412.87210297)
\curveto(239.55901969,412.83209827)(239.78901946,412.78209832)(240.00902466,412.72210297)
\curveto(240.23901901,412.67209843)(240.4540188,412.60709849)(240.65402466,412.52710297)
\curveto(240.74401851,412.48709861)(240.82901842,412.45209865)(240.90902466,412.42210297)
\curveto(240.99901825,412.4020987)(241.08401817,412.36709873)(241.16402466,412.31710297)
\curveto(241.3540179,412.1970989)(241.52401773,412.06709903)(241.67402466,411.92710297)
\curveto(241.83401742,411.78709931)(241.95901729,411.61209949)(242.04902466,411.40210297)
\curveto(242.07901717,411.33209977)(242.10401715,411.26209984)(242.12402466,411.19210297)
\curveto(242.14401711,411.12209998)(242.16401709,411.04710005)(242.18402466,410.96710297)
\curveto(242.19401706,410.90710019)(242.19901705,410.81210029)(242.19902466,410.68210297)
\curveto(242.20901704,410.56210054)(242.20901704,410.46710063)(242.19902466,410.39710297)
\lineto(242.19902466,410.32210297)
\curveto(242.17901707,410.26210084)(242.16401709,410.2021009)(242.15402466,410.14210297)
\curveto(242.1540171,410.09210101)(242.1490171,410.04210106)(242.13902466,409.99210297)
\curveto(242.06901718,409.69210141)(241.95901729,409.42710167)(241.80902466,409.19710297)
\curveto(241.6490176,408.95710214)(241.4540178,408.76210234)(241.22402466,408.61210297)
\curveto(240.99401826,408.46210264)(240.73401852,408.33210277)(240.44402466,408.22210297)
\curveto(240.33401892,408.17210293)(240.21401904,408.13710296)(240.08402466,408.11710297)
\curveto(239.96401929,408.097103)(239.84401941,408.07210303)(239.72402466,408.04210297)
\curveto(239.63401962,408.02210308)(239.53901971,408.01210309)(239.43902466,408.01210297)
\curveto(239.3490199,408.0021031)(239.25901999,407.98710311)(239.16902466,407.96710297)
\lineto(238.89902466,407.96710297)
\curveto(238.83902041,407.94710315)(238.73402052,407.93710316)(238.58402466,407.93710297)
\curveto(238.44402081,407.93710316)(238.34402091,407.94710315)(238.28402466,407.96710297)
\curveto(238.254021,407.96710313)(238.21902103,407.97210313)(238.17902466,407.98210297)
\lineto(238.07402466,407.98210297)
\curveto(237.9540213,408.0021031)(237.83402142,408.01710308)(237.71402466,408.02710297)
\curveto(237.59402166,408.03710306)(237.47902177,408.05710304)(237.36902466,408.08710297)
\curveto(236.97902227,408.1971029)(236.63402262,408.32210278)(236.33402466,408.46210297)
\curveto(236.03402322,408.61210249)(235.77902347,408.83210227)(235.56902466,409.12210297)
\curveto(235.42902382,409.31210179)(235.30902394,409.53210157)(235.20902466,409.78210297)
\curveto(235.18902406,409.84210126)(235.16902408,409.92210118)(235.14902466,410.02210297)
\curveto(235.12902412,410.07210103)(235.11402414,410.14210096)(235.10402466,410.23210297)
\curveto(235.09402416,410.32210078)(235.09902415,410.3971007)(235.11902466,410.45710297)
\curveto(235.1490241,410.52710057)(235.19902405,410.57710052)(235.26902466,410.60710297)
\curveto(235.31902393,410.62710047)(235.37902387,410.63710046)(235.44902466,410.63710297)
\lineto(235.67402466,410.63710297)
\lineto(236.37902466,410.63710297)
\lineto(236.61902466,410.63710297)
\curveto(236.69902255,410.63710046)(236.76902248,410.62710047)(236.82902466,410.60710297)
\curveto(236.93902231,410.56710053)(237.00902224,410.5021006)(237.03902466,410.41210297)
\curveto(237.07902217,410.32210078)(237.12402213,410.22710087)(237.17402466,410.12710297)
\curveto(237.19402206,410.07710102)(237.22902202,410.01210109)(237.27902466,409.93210297)
\curveto(237.33902191,409.85210125)(237.38902186,409.8021013)(237.42902466,409.78210297)
\curveto(237.5490217,409.68210142)(237.66402159,409.6021015)(237.77402466,409.54210297)
\curveto(237.88402137,409.49210161)(238.02402123,409.44210166)(238.19402466,409.39210297)
\curveto(238.24402101,409.37210173)(238.29402096,409.36210174)(238.34402466,409.36210297)
\curveto(238.39402086,409.37210173)(238.44402081,409.37210173)(238.49402466,409.36210297)
\curveto(238.57402068,409.34210176)(238.65902059,409.33210177)(238.74902466,409.33210297)
\curveto(238.8490204,409.34210176)(238.93402032,409.35710174)(239.00402466,409.37710297)
\curveto(239.0540202,409.38710171)(239.09902015,409.39210171)(239.13902466,409.39210297)
\curveto(239.18902006,409.39210171)(239.23902001,409.4021017)(239.28902466,409.42210297)
\curveto(239.42901982,409.47210163)(239.5540197,409.53210157)(239.66402466,409.60210297)
\curveto(239.78401947,409.67210143)(239.87901937,409.76210134)(239.94902466,409.87210297)
\curveto(239.99901925,409.95210115)(240.03901921,410.07710102)(240.06902466,410.24710297)
\curveto(240.08901916,410.31710078)(240.08901916,410.38210072)(240.06902466,410.44210297)
\curveto(240.0490192,410.5021006)(240.02901922,410.55210055)(240.00902466,410.59210297)
\curveto(239.93901931,410.73210037)(239.8490194,410.83710026)(239.73902466,410.90710297)
\curveto(239.63901961,410.97710012)(239.51901973,411.04210006)(239.37902466,411.10210297)
\curveto(239.18902006,411.18209992)(238.98902026,411.24709985)(238.77902466,411.29710297)
\curveto(238.56902068,411.34709975)(238.35902089,411.4020997)(238.14902466,411.46210297)
\curveto(238.06902118,411.48209962)(237.98402127,411.4970996)(237.89402466,411.50710297)
\curveto(237.81402144,411.51709958)(237.73402152,411.53209957)(237.65402466,411.55210297)
\curveto(237.33402192,411.64209946)(237.02902222,411.72709937)(236.73902466,411.80710297)
\curveto(236.4490228,411.8970992)(236.18402307,412.02709907)(235.94402466,412.19710297)
\curveto(235.66402359,412.3970987)(235.45902379,412.66709843)(235.32902466,413.00710297)
\curveto(235.30902394,413.07709802)(235.28902396,413.17209793)(235.26902466,413.29210297)
\curveto(235.249024,413.36209774)(235.23402402,413.44709765)(235.22402466,413.54710297)
\curveto(235.21402404,413.64709745)(235.21902403,413.73709736)(235.23902466,413.81710297)
\curveto(235.25902399,413.86709723)(235.26402399,413.90709719)(235.25402466,413.93710297)
\curveto(235.24402401,413.97709712)(235.249024,414.02209708)(235.26902466,414.07210297)
\curveto(235.28902396,414.18209692)(235.30902394,414.28209682)(235.32902466,414.37210297)
\curveto(235.35902389,414.47209663)(235.39402386,414.56709653)(235.43402466,414.65710297)
\curveto(235.56402369,414.94709615)(235.74402351,415.18209592)(235.97402466,415.36210297)
\curveto(236.20402305,415.54209556)(236.46402279,415.68709541)(236.75402466,415.79710297)
\curveto(236.86402239,415.84709525)(236.97902227,415.88209522)(237.09902466,415.90210297)
\curveto(237.21902203,415.93209517)(237.34402191,415.96209514)(237.47402466,415.99210297)
\curveto(237.53402172,416.01209509)(237.59402166,416.02209508)(237.65402466,416.02210297)
\lineto(237.83402466,416.05210297)
\curveto(237.91402134,416.06209504)(237.99902125,416.06709503)(238.08902466,416.06710297)
\curveto(238.17902107,416.06709503)(238.26402099,416.07209503)(238.34402466,416.08210297)
}
}
{
\newrgbcolor{curcolor}{0 0 0}
\pscustom[linestyle=none,fillstyle=solid,fillcolor=curcolor]
{
}
}
{
\newrgbcolor{curcolor}{0 0 0}
\pscustom[linestyle=none,fillstyle=solid,fillcolor=curcolor]
{
\newpath
\moveto(255.48082153,412.09210297)
\curveto(255.49081285,412.03209907)(255.49581285,411.94209916)(255.49582153,411.82210297)
\curveto(255.49581285,411.7020994)(255.48581286,411.61709948)(255.46582153,411.56710297)
\lineto(255.46582153,411.37210297)
\curveto(255.43581291,411.26209984)(255.41581293,411.15709994)(255.40582153,411.05710297)
\curveto(255.40581294,410.95710014)(255.39081295,410.85710024)(255.36082153,410.75710297)
\curveto(255.340813,410.66710043)(255.32081302,410.57210053)(255.30082153,410.47210297)
\curveto(255.28081306,410.38210072)(255.25081309,410.29210081)(255.21082153,410.20210297)
\curveto(255.1408132,410.03210107)(255.07081327,409.87210123)(255.00082153,409.72210297)
\curveto(254.93081341,409.58210152)(254.85081349,409.44210166)(254.76082153,409.30210297)
\curveto(254.70081364,409.21210189)(254.63581371,409.12710197)(254.56582153,409.04710297)
\curveto(254.50581384,408.97710212)(254.43581391,408.9021022)(254.35582153,408.82210297)
\lineto(254.25082153,408.71710297)
\curveto(254.20081414,408.66710243)(254.1458142,408.62210248)(254.08582153,408.58210297)
\lineto(253.93582153,408.46210297)
\curveto(253.85581449,408.4021027)(253.76581458,408.34710275)(253.66582153,408.29710297)
\curveto(253.57581477,408.25710284)(253.48081486,408.21210289)(253.38082153,408.16210297)
\curveto(253.28081506,408.11210299)(253.17581517,408.07710302)(253.06582153,408.05710297)
\curveto(252.96581538,408.03710306)(252.86081548,408.01710308)(252.75082153,407.99710297)
\curveto(252.69081565,407.97710312)(252.62581572,407.96710313)(252.55582153,407.96710297)
\curveto(252.49581585,407.96710313)(252.43081591,407.95710314)(252.36082153,407.93710297)
\lineto(252.22582153,407.93710297)
\curveto(252.1458162,407.91710318)(252.07081627,407.91710318)(252.00082153,407.93710297)
\lineto(251.85082153,407.93710297)
\curveto(251.79081655,407.95710314)(251.72581662,407.96710313)(251.65582153,407.96710297)
\curveto(251.59581675,407.95710314)(251.53581681,407.96210314)(251.47582153,407.98210297)
\curveto(251.31581703,408.03210307)(251.16081718,408.07710302)(251.01082153,408.11710297)
\curveto(250.87081747,408.15710294)(250.7408176,408.21710288)(250.62082153,408.29710297)
\curveto(250.55081779,408.33710276)(250.48581786,408.37710272)(250.42582153,408.41710297)
\curveto(250.36581798,408.46710263)(250.30081804,408.51710258)(250.23082153,408.56710297)
\lineto(250.05082153,408.70210297)
\curveto(249.97081837,408.76210234)(249.90081844,408.76710233)(249.84082153,408.71710297)
\curveto(249.79081855,408.68710241)(249.76581858,408.64710245)(249.76582153,408.59710297)
\curveto(249.76581858,408.55710254)(249.75581859,408.50710259)(249.73582153,408.44710297)
\curveto(249.71581863,408.34710275)(249.70581864,408.23210287)(249.70582153,408.10210297)
\curveto(249.71581863,407.97210313)(249.72081862,407.85210325)(249.72082153,407.74210297)
\lineto(249.72082153,406.21210297)
\curveto(249.72081862,406.08210502)(249.71581863,405.95710514)(249.70582153,405.83710297)
\curveto(249.70581864,405.70710539)(249.68081866,405.6021055)(249.63082153,405.52210297)
\curveto(249.60081874,405.48210562)(249.5458188,405.45210565)(249.46582153,405.43210297)
\curveto(249.38581896,405.41210569)(249.29581905,405.4021057)(249.19582153,405.40210297)
\curveto(249.09581925,405.39210571)(248.99581935,405.39210571)(248.89582153,405.40210297)
\lineto(248.64082153,405.40210297)
\lineto(248.23582153,405.40210297)
\lineto(248.13082153,405.40210297)
\curveto(248.09082025,405.4021057)(248.05582029,405.40710569)(248.02582153,405.41710297)
\lineto(247.90582153,405.41710297)
\curveto(247.73582061,405.46710563)(247.6458207,405.56710553)(247.63582153,405.71710297)
\curveto(247.62582072,405.85710524)(247.62082072,406.02710507)(247.62082153,406.22710297)
\lineto(247.62082153,415.03210297)
\curveto(247.62082072,415.14209596)(247.61582073,415.25709584)(247.60582153,415.37710297)
\curveto(247.60582074,415.50709559)(247.63082071,415.60709549)(247.68082153,415.67710297)
\curveto(247.72082062,415.74709535)(247.77582057,415.79209531)(247.84582153,415.81210297)
\curveto(247.89582045,415.83209527)(247.95582039,415.84209526)(248.02582153,415.84210297)
\lineto(248.25082153,415.84210297)
\lineto(248.97082153,415.84210297)
\lineto(249.25582153,415.84210297)
\curveto(249.345819,415.84209526)(249.42081892,415.81709528)(249.48082153,415.76710297)
\curveto(249.55081879,415.71709538)(249.58581876,415.65209545)(249.58582153,415.57210297)
\curveto(249.59581875,415.5020956)(249.62081872,415.42709567)(249.66082153,415.34710297)
\curveto(249.67081867,415.31709578)(249.68081866,415.29209581)(249.69082153,415.27210297)
\curveto(249.71081863,415.26209584)(249.73081861,415.24709585)(249.75082153,415.22710297)
\curveto(249.86081848,415.21709588)(249.95081839,415.24709585)(250.02082153,415.31710297)
\curveto(250.09081825,415.38709571)(250.16081818,415.44709565)(250.23082153,415.49710297)
\curveto(250.36081798,415.58709551)(250.49581785,415.66709543)(250.63582153,415.73710297)
\curveto(250.77581757,415.81709528)(250.93081741,415.88209522)(251.10082153,415.93210297)
\curveto(251.18081716,415.96209514)(251.26581708,415.98209512)(251.35582153,415.99210297)
\curveto(251.45581689,416.0020951)(251.55081679,416.01709508)(251.64082153,416.03710297)
\curveto(251.68081666,416.04709505)(251.72081662,416.04709505)(251.76082153,416.03710297)
\curveto(251.81081653,416.02709507)(251.85081649,416.03209507)(251.88082153,416.05210297)
\curveto(252.45081589,416.07209503)(252.93081541,415.99209511)(253.32082153,415.81210297)
\curveto(253.72081462,415.64209546)(254.06081428,415.41709568)(254.34082153,415.13710297)
\curveto(254.39081395,415.08709601)(254.43581391,415.03709606)(254.47582153,414.98710297)
\curveto(254.51581383,414.94709615)(254.55581379,414.9020962)(254.59582153,414.85210297)
\curveto(254.66581368,414.76209634)(254.72581362,414.67209643)(254.77582153,414.58210297)
\curveto(254.83581351,414.49209661)(254.89081345,414.4020967)(254.94082153,414.31210297)
\curveto(254.96081338,414.29209681)(254.97081337,414.26709683)(254.97082153,414.23710297)
\curveto(254.98081336,414.20709689)(254.99581335,414.17209693)(255.01582153,414.13210297)
\curveto(255.07581327,414.03209707)(255.13081321,413.91209719)(255.18082153,413.77210297)
\curveto(255.20081314,413.71209739)(255.22081312,413.64709745)(255.24082153,413.57710297)
\curveto(255.26081308,413.51709758)(255.28081306,413.45209765)(255.30082153,413.38210297)
\curveto(255.340813,413.26209784)(255.36581298,413.13709796)(255.37582153,413.00710297)
\curveto(255.39581295,412.87709822)(255.42081292,412.74209836)(255.45082153,412.60210297)
\lineto(255.45082153,412.43710297)
\lineto(255.48082153,412.25710297)
\lineto(255.48082153,412.09210297)
\moveto(253.36582153,411.74710297)
\curveto(253.37581497,411.7970993)(253.38081496,411.86209924)(253.38082153,411.94210297)
\curveto(253.38081496,412.03209907)(253.37581497,412.102099)(253.36582153,412.15210297)
\lineto(253.36582153,412.28710297)
\curveto(253.345815,412.34709875)(253.33581501,412.41209869)(253.33582153,412.48210297)
\curveto(253.33581501,412.55209855)(253.32581502,412.62209848)(253.30582153,412.69210297)
\curveto(253.28581506,412.79209831)(253.26581508,412.88709821)(253.24582153,412.97710297)
\curveto(253.22581512,413.07709802)(253.19581515,413.16709793)(253.15582153,413.24710297)
\curveto(253.03581531,413.56709753)(252.88081546,413.82209728)(252.69082153,414.01210297)
\curveto(252.50081584,414.2020969)(252.23081611,414.34209676)(251.88082153,414.43210297)
\curveto(251.80081654,414.45209665)(251.71081663,414.46209664)(251.61082153,414.46210297)
\lineto(251.34082153,414.46210297)
\curveto(251.30081704,414.45209665)(251.26581708,414.44709665)(251.23582153,414.44710297)
\curveto(251.20581714,414.44709665)(251.17081717,414.44209666)(251.13082153,414.43210297)
\lineto(250.92082153,414.37210297)
\curveto(250.86081748,414.36209674)(250.80081754,414.34209676)(250.74082153,414.31210297)
\curveto(250.48081786,414.2020969)(250.27581807,414.03209707)(250.12582153,413.80210297)
\curveto(249.98581836,413.57209753)(249.87081847,413.31709778)(249.78082153,413.03710297)
\curveto(249.76081858,412.95709814)(249.7458186,412.87209823)(249.73582153,412.78210297)
\curveto(249.72581862,412.7020984)(249.71081863,412.62209848)(249.69082153,412.54210297)
\curveto(249.68081866,412.5020986)(249.67581867,412.43709866)(249.67582153,412.34710297)
\curveto(249.65581869,412.30709879)(249.65081869,412.25709884)(249.66082153,412.19710297)
\curveto(249.67081867,412.14709895)(249.67081867,412.097099)(249.66082153,412.04710297)
\curveto(249.6408187,411.98709911)(249.6408187,411.93209917)(249.66082153,411.88210297)
\lineto(249.66082153,411.70210297)
\lineto(249.66082153,411.56710297)
\curveto(249.66081868,411.52709957)(249.67081867,411.48709961)(249.69082153,411.44710297)
\curveto(249.69081865,411.37709972)(249.69581865,411.32209978)(249.70582153,411.28210297)
\lineto(249.73582153,411.10210297)
\curveto(249.7458186,411.04210006)(249.76081858,410.98210012)(249.78082153,410.92210297)
\curveto(249.87081847,410.63210047)(249.97581837,410.39210071)(250.09582153,410.20210297)
\curveto(250.22581812,410.02210108)(250.40581794,409.86210124)(250.63582153,409.72210297)
\curveto(250.77581757,409.64210146)(250.9408174,409.57710152)(251.13082153,409.52710297)
\curveto(251.17081717,409.51710158)(251.20581714,409.51210159)(251.23582153,409.51210297)
\curveto(251.26581708,409.52210158)(251.30081704,409.52210158)(251.34082153,409.51210297)
\curveto(251.38081696,409.5021016)(251.4408169,409.49210161)(251.52082153,409.48210297)
\curveto(251.60081674,409.48210162)(251.66581668,409.48710161)(251.71582153,409.49710297)
\curveto(251.79581655,409.51710158)(251.87581647,409.53210157)(251.95582153,409.54210297)
\curveto(252.0458163,409.56210154)(252.13081621,409.58710151)(252.21082153,409.61710297)
\curveto(252.45081589,409.71710138)(252.6458157,409.85710124)(252.79582153,410.03710297)
\curveto(252.9458154,410.21710088)(253.07081527,410.42710067)(253.17082153,410.66710297)
\curveto(253.22081512,410.78710031)(253.25581509,410.91210019)(253.27582153,411.04210297)
\curveto(253.29581505,411.17209993)(253.32081502,411.30709979)(253.35082153,411.44710297)
\lineto(253.35082153,411.59710297)
\curveto(253.36081498,411.64709945)(253.36581498,411.6970994)(253.36582153,411.74710297)
}
}
{
\newrgbcolor{curcolor}{0 0 0}
\pscustom[linestyle=none,fillstyle=solid,fillcolor=curcolor]
{
\newpath
\moveto(257.18074341,415.85710297)
\lineto(258.30574341,415.85710297)
\curveto(258.41574097,415.85709524)(258.51574087,415.85209525)(258.60574341,415.84210297)
\curveto(258.69574069,415.83209527)(258.76074063,415.7970953)(258.80074341,415.73710297)
\curveto(258.85074054,415.67709542)(258.88074051,415.59209551)(258.89074341,415.48210297)
\curveto(258.90074049,415.38209572)(258.90574048,415.27709582)(258.90574341,415.16710297)
\lineto(258.90574341,414.11710297)
\lineto(258.90574341,411.88210297)
\curveto(258.90574048,411.52209958)(258.92074047,411.18209992)(258.95074341,410.86210297)
\curveto(258.98074041,410.54210056)(259.07074032,410.27710082)(259.22074341,410.06710297)
\curveto(259.36074003,409.85710124)(259.5857398,409.70710139)(259.89574341,409.61710297)
\curveto(259.94573944,409.60710149)(259.9857394,409.6021015)(260.01574341,409.60210297)
\curveto(260.05573933,409.6021015)(260.10073929,409.5971015)(260.15074341,409.58710297)
\curveto(260.20073919,409.57710152)(260.25573913,409.57210153)(260.31574341,409.57210297)
\curveto(260.37573901,409.57210153)(260.42073897,409.57710152)(260.45074341,409.58710297)
\curveto(260.50073889,409.60710149)(260.54073885,409.61210149)(260.57074341,409.60210297)
\curveto(260.61073878,409.59210151)(260.65073874,409.5971015)(260.69074341,409.61710297)
\curveto(260.90073849,409.66710143)(261.06573832,409.73210137)(261.18574341,409.81210297)
\curveto(261.36573802,409.92210118)(261.50573788,410.06210104)(261.60574341,410.23210297)
\curveto(261.71573767,410.41210069)(261.7907376,410.60710049)(261.83074341,410.81710297)
\curveto(261.88073751,411.03710006)(261.91073748,411.27709982)(261.92074341,411.53710297)
\curveto(261.93073746,411.80709929)(261.93573745,412.08709901)(261.93574341,412.37710297)
\lineto(261.93574341,414.19210297)
\lineto(261.93574341,415.16710297)
\lineto(261.93574341,415.43710297)
\curveto(261.93573745,415.53709556)(261.95573743,415.61709548)(261.99574341,415.67710297)
\curveto(262.04573734,415.76709533)(262.12073727,415.81709528)(262.22074341,415.82710297)
\curveto(262.32073707,415.84709525)(262.44073695,415.85709524)(262.58074341,415.85710297)
\lineto(263.37574341,415.85710297)
\lineto(263.66074341,415.85710297)
\curveto(263.75073564,415.85709524)(263.82573556,415.83709526)(263.88574341,415.79710297)
\curveto(263.96573542,415.74709535)(264.01073538,415.67209543)(264.02074341,415.57210297)
\curveto(264.03073536,415.47209563)(264.03573535,415.35709574)(264.03574341,415.22710297)
\lineto(264.03574341,414.08710297)
\lineto(264.03574341,409.87210297)
\lineto(264.03574341,408.80710297)
\lineto(264.03574341,408.50710297)
\curveto(264.03573535,408.40710269)(264.01573537,408.33210277)(263.97574341,408.28210297)
\curveto(263.92573546,408.2021029)(263.85073554,408.15710294)(263.75074341,408.14710297)
\curveto(263.65073574,408.13710296)(263.54573584,408.13210297)(263.43574341,408.13210297)
\lineto(262.62574341,408.13210297)
\curveto(262.51573687,408.13210297)(262.41573697,408.13710296)(262.32574341,408.14710297)
\curveto(262.24573714,408.15710294)(262.18073721,408.1971029)(262.13074341,408.26710297)
\curveto(262.11073728,408.2971028)(262.0907373,408.34210276)(262.07074341,408.40210297)
\curveto(262.06073733,408.46210264)(262.04573734,408.52210258)(262.02574341,408.58210297)
\curveto(262.01573737,408.64210246)(262.00073739,408.6971024)(261.98074341,408.74710297)
\curveto(261.96073743,408.7971023)(261.93073746,408.82710227)(261.89074341,408.83710297)
\curveto(261.87073752,408.85710224)(261.84573754,408.86210224)(261.81574341,408.85210297)
\curveto(261.7857376,408.84210226)(261.76073763,408.83210227)(261.74074341,408.82210297)
\curveto(261.67073772,408.78210232)(261.61073778,408.73710236)(261.56074341,408.68710297)
\curveto(261.51073788,408.63710246)(261.45573793,408.59210251)(261.39574341,408.55210297)
\curveto(261.35573803,408.52210258)(261.31573807,408.48710261)(261.27574341,408.44710297)
\curveto(261.24573814,408.41710268)(261.20573818,408.38710271)(261.15574341,408.35710297)
\curveto(260.92573846,408.21710288)(260.65573873,408.10710299)(260.34574341,408.02710297)
\curveto(260.27573911,408.00710309)(260.20573918,407.9971031)(260.13574341,407.99710297)
\curveto(260.06573932,407.98710311)(259.9907394,407.97210313)(259.91074341,407.95210297)
\curveto(259.87073952,407.94210316)(259.82573956,407.94210316)(259.77574341,407.95210297)
\curveto(259.73573965,407.95210315)(259.69573969,407.94710315)(259.65574341,407.93710297)
\curveto(259.62573976,407.92710317)(259.56073983,407.92710317)(259.46074341,407.93710297)
\curveto(259.37074002,407.93710316)(259.31074008,407.94210316)(259.28074341,407.95210297)
\curveto(259.23074016,407.95210315)(259.18074021,407.95710314)(259.13074341,407.96710297)
\lineto(258.98074341,407.96710297)
\curveto(258.86074053,407.9971031)(258.74574064,408.02210308)(258.63574341,408.04210297)
\curveto(258.52574086,408.06210304)(258.41574097,408.09210301)(258.30574341,408.13210297)
\curveto(258.25574113,408.15210295)(258.21074118,408.16710293)(258.17074341,408.17710297)
\curveto(258.14074125,408.1971029)(258.10074129,408.21710288)(258.05074341,408.23710297)
\curveto(257.70074169,408.42710267)(257.42074197,408.69210241)(257.21074341,409.03210297)
\curveto(257.08074231,409.24210186)(256.9857424,409.49210161)(256.92574341,409.78210297)
\curveto(256.86574252,410.08210102)(256.82574256,410.3971007)(256.80574341,410.72710297)
\curveto(256.79574259,411.06710003)(256.7907426,411.41209969)(256.79074341,411.76210297)
\curveto(256.80074259,412.12209898)(256.80574258,412.47709862)(256.80574341,412.82710297)
\lineto(256.80574341,414.86710297)
\curveto(256.80574258,414.9970961)(256.80074259,415.14709595)(256.79074341,415.31710297)
\curveto(256.7907426,415.4970956)(256.81574257,415.62709547)(256.86574341,415.70710297)
\curveto(256.89574249,415.75709534)(256.95574243,415.8020953)(257.04574341,415.84210297)
\curveto(257.10574228,415.84209526)(257.15074224,415.84709525)(257.18074341,415.85710297)
}
}
{
\newrgbcolor{curcolor}{0 0 0}
\pscustom[linestyle=none,fillstyle=solid,fillcolor=curcolor]
{
\newpath
\moveto(273.46699341,412.39210297)
\curveto(273.48698481,412.33209877)(273.4969848,412.22709887)(273.49699341,412.07710297)
\curveto(273.4969848,411.93709916)(273.4919848,411.83709926)(273.48199341,411.77710297)
\curveto(273.48198481,411.72709937)(273.47698482,411.68209942)(273.46699341,411.64210297)
\lineto(273.46699341,411.52210297)
\curveto(273.44698485,411.44209966)(273.43698486,411.36209974)(273.43699341,411.28210297)
\curveto(273.43698486,411.21209989)(273.42698487,411.13709996)(273.40699341,411.05710297)
\curveto(273.40698489,411.01710008)(273.3969849,410.94710015)(273.37699341,410.84710297)
\curveto(273.34698495,410.72710037)(273.31698498,410.6021005)(273.28699341,410.47210297)
\curveto(273.26698503,410.35210075)(273.23198506,410.23710086)(273.18199341,410.12710297)
\curveto(273.00198529,409.67710142)(272.77698552,409.28710181)(272.50699341,408.95710297)
\curveto(272.23698606,408.62710247)(271.88198641,408.36710273)(271.44199341,408.17710297)
\curveto(271.35198694,408.13710296)(271.25698704,408.10710299)(271.15699341,408.08710297)
\curveto(271.06698723,408.05710304)(270.96698733,408.02710307)(270.85699341,407.99710297)
\curveto(270.7969875,407.97710312)(270.73198756,407.96710313)(270.66199341,407.96710297)
\curveto(270.60198769,407.96710313)(270.54198775,407.96210314)(270.48199341,407.95210297)
\lineto(270.34699341,407.95210297)
\curveto(270.28698801,407.93210317)(270.20698809,407.92710317)(270.10699341,407.93710297)
\curveto(270.00698829,407.93710316)(269.92698837,407.94710315)(269.86699341,407.96710297)
\lineto(269.77699341,407.96710297)
\curveto(269.72698857,407.97710312)(269.67198862,407.98710311)(269.61199341,407.99710297)
\curveto(269.55198874,407.9971031)(269.4919888,408.0021031)(269.43199341,408.01210297)
\curveto(269.24198905,408.06210304)(269.06698923,408.11210299)(268.90699341,408.16210297)
\curveto(268.74698955,408.21210289)(268.5969897,408.28210282)(268.45699341,408.37210297)
\lineto(268.27699341,408.49210297)
\curveto(268.22699007,408.53210257)(268.17699012,408.57710252)(268.12699341,408.62710297)
\lineto(268.03699341,408.68710297)
\curveto(268.00699029,408.70710239)(267.97699032,408.72210238)(267.94699341,408.73210297)
\curveto(267.85699044,408.76210234)(267.80199049,408.74210236)(267.78199341,408.67210297)
\curveto(267.73199056,408.6021025)(267.6969906,408.51710258)(267.67699341,408.41710297)
\curveto(267.66699063,408.32710277)(267.63199066,408.25710284)(267.57199341,408.20710297)
\curveto(267.51199078,408.16710293)(267.44199085,408.14210296)(267.36199341,408.13210297)
\lineto(267.09199341,408.13210297)
\lineto(266.37199341,408.13210297)
\lineto(266.14699341,408.13210297)
\curveto(266.07699222,408.12210298)(266.01199228,408.12710297)(265.95199341,408.14710297)
\curveto(265.81199248,408.1971029)(265.73199256,408.28710281)(265.71199341,408.41710297)
\curveto(265.70199259,408.55710254)(265.6969926,408.71210239)(265.69699341,408.88210297)
\lineto(265.69699341,418.03210297)
\lineto(265.69699341,418.37710297)
\curveto(265.6969926,418.4970926)(265.72199257,418.59209251)(265.77199341,418.66210297)
\curveto(265.81199248,418.73209237)(265.88199241,418.77709232)(265.98199341,418.79710297)
\curveto(266.00199229,418.80709229)(266.02199227,418.80709229)(266.04199341,418.79710297)
\curveto(266.07199222,418.7970923)(266.0969922,418.8020923)(266.11699341,418.81210297)
\lineto(267.06199341,418.81210297)
\curveto(267.24199105,418.81209229)(267.3969909,418.8020923)(267.52699341,418.78210297)
\curveto(267.65699064,418.77209233)(267.74199055,418.6970924)(267.78199341,418.55710297)
\curveto(267.81199048,418.45709264)(267.82199047,418.32209278)(267.81199341,418.15210297)
\curveto(267.80199049,417.99209311)(267.7969905,417.85209325)(267.79699341,417.73210297)
\lineto(267.79699341,416.09710297)
\lineto(267.79699341,415.76710297)
\curveto(267.7969905,415.65709544)(267.80699049,415.56209554)(267.82699341,415.48210297)
\curveto(267.83699046,415.43209567)(267.84699045,415.38709571)(267.85699341,415.34710297)
\curveto(267.86699043,415.31709578)(267.8919904,415.2970958)(267.93199341,415.28710297)
\curveto(267.95199034,415.26709583)(267.97699032,415.25709584)(268.00699341,415.25710297)
\curveto(268.04699025,415.25709584)(268.07699022,415.26209584)(268.09699341,415.27210297)
\curveto(268.16699013,415.31209579)(268.23199006,415.35209575)(268.29199341,415.39210297)
\curveto(268.35198994,415.44209566)(268.41698988,415.49209561)(268.48699341,415.54210297)
\curveto(268.61698968,415.63209547)(268.75198954,415.70709539)(268.89199341,415.76710297)
\curveto(269.03198926,415.83709526)(269.18698911,415.8970952)(269.35699341,415.94710297)
\curveto(269.43698886,415.97709512)(269.51698878,415.99209511)(269.59699341,415.99210297)
\curveto(269.67698862,416.0020951)(269.75698854,416.01709508)(269.83699341,416.03710297)
\curveto(269.90698839,416.05709504)(269.98198831,416.06709503)(270.06199341,416.06710297)
\lineto(270.30199341,416.06710297)
\lineto(270.45199341,416.06710297)
\curveto(270.48198781,416.05709504)(270.51698778,416.05209505)(270.55699341,416.05210297)
\curveto(270.5969877,416.06209504)(270.63698766,416.06209504)(270.67699341,416.05210297)
\curveto(270.78698751,416.02209508)(270.88698741,415.9970951)(270.97699341,415.97710297)
\curveto(271.07698722,415.96709513)(271.17198712,415.94209516)(271.26199341,415.90210297)
\curveto(271.72198657,415.71209539)(272.0969862,415.46709563)(272.38699341,415.16710297)
\curveto(272.67698562,414.86709623)(272.92198537,414.49209661)(273.12199341,414.04210297)
\curveto(273.17198512,413.92209718)(273.21198508,413.7970973)(273.24199341,413.66710297)
\curveto(273.28198501,413.53709756)(273.32198497,413.4020977)(273.36199341,413.26210297)
\curveto(273.38198491,413.19209791)(273.3919849,413.12209798)(273.39199341,413.05210297)
\curveto(273.40198489,412.99209811)(273.41698488,412.92209818)(273.43699341,412.84210297)
\curveto(273.45698484,412.79209831)(273.46198483,412.73709836)(273.45199341,412.67710297)
\curveto(273.45198484,412.61709848)(273.45698484,412.55709854)(273.46699341,412.49710297)
\lineto(273.46699341,412.39210297)
\moveto(271.24699341,410.98210297)
\curveto(271.27698702,411.08210002)(271.30198699,411.20709989)(271.32199341,411.35710297)
\curveto(271.35198694,411.50709959)(271.36698693,411.65709944)(271.36699341,411.80710297)
\curveto(271.37698692,411.96709913)(271.37698692,412.12209898)(271.36699341,412.27210297)
\curveto(271.36698693,412.43209867)(271.35198694,412.56709853)(271.32199341,412.67710297)
\curveto(271.291987,412.77709832)(271.27198702,412.87209823)(271.26199341,412.96210297)
\curveto(271.25198704,413.05209805)(271.22698707,413.13709796)(271.18699341,413.21710297)
\curveto(271.04698725,413.56709753)(270.84698745,413.86209724)(270.58699341,414.10210297)
\curveto(270.33698796,414.35209675)(269.96698833,414.47709662)(269.47699341,414.47710297)
\curveto(269.43698886,414.47709662)(269.40198889,414.47209663)(269.37199341,414.46210297)
\lineto(269.26699341,414.46210297)
\curveto(269.1969891,414.44209666)(269.13198916,414.42209668)(269.07199341,414.40210297)
\curveto(269.01198928,414.39209671)(268.95198934,414.37709672)(268.89199341,414.35710297)
\curveto(268.60198969,414.22709687)(268.38198991,414.04209706)(268.23199341,413.80210297)
\curveto(268.08199021,413.57209753)(267.95699034,413.30709779)(267.85699341,413.00710297)
\curveto(267.82699047,412.92709817)(267.80699049,412.84209826)(267.79699341,412.75210297)
\curveto(267.7969905,412.67209843)(267.78699051,412.59209851)(267.76699341,412.51210297)
\curveto(267.75699054,412.48209862)(267.75199054,412.43209867)(267.75199341,412.36210297)
\curveto(267.74199055,412.32209878)(267.73699056,412.28209882)(267.73699341,412.24210297)
\curveto(267.74699055,412.2020989)(267.74699055,412.16209894)(267.73699341,412.12210297)
\curveto(267.71699058,412.04209906)(267.71199058,411.93209917)(267.72199341,411.79210297)
\curveto(267.73199056,411.65209945)(267.74699055,411.55209955)(267.76699341,411.49210297)
\curveto(267.78699051,411.4020997)(267.7969905,411.31709978)(267.79699341,411.23710297)
\curveto(267.80699049,411.15709994)(267.82699047,411.07710002)(267.85699341,410.99710297)
\curveto(267.94699035,410.71710038)(268.05199024,410.47210063)(268.17199341,410.26210297)
\curveto(268.30198999,410.06210104)(268.48198981,409.89210121)(268.71199341,409.75210297)
\curveto(268.87198942,409.65210145)(269.03698926,409.58210152)(269.20699341,409.54210297)
\curveto(269.22698907,409.54210156)(269.24698905,409.53710156)(269.26699341,409.52710297)
\lineto(269.35699341,409.52710297)
\curveto(269.38698891,409.51710158)(269.43698886,409.50710159)(269.50699341,409.49710297)
\curveto(269.57698872,409.4971016)(269.63698866,409.5021016)(269.68699341,409.51210297)
\curveto(269.78698851,409.53210157)(269.87698842,409.54710155)(269.95699341,409.55710297)
\curveto(270.04698825,409.57710152)(270.13198816,409.6021015)(270.21199341,409.63210297)
\curveto(270.4919878,409.76210134)(270.70698759,409.94210116)(270.85699341,410.17210297)
\curveto(271.01698728,410.4021007)(271.14698715,410.67210043)(271.24699341,410.98210297)
}
}
{
\newrgbcolor{curcolor}{0 0 0}
\pscustom[linestyle=none,fillstyle=solid,fillcolor=curcolor]
{
\newpath
\moveto(275.34691528,418.82710297)
\lineto(276.44191528,418.82710297)
\curveto(276.5419128,418.82709227)(276.6369127,418.82209228)(276.72691528,418.81210297)
\curveto(276.81691252,418.8020923)(276.88691245,418.77209233)(276.93691528,418.72210297)
\curveto(276.99691234,418.65209245)(277.02691231,418.55709254)(277.02691528,418.43710297)
\curveto(277.0369123,418.32709277)(277.0419123,418.21209289)(277.04191528,418.09210297)
\lineto(277.04191528,416.75710297)
\lineto(277.04191528,411.37210297)
\lineto(277.04191528,409.07710297)
\lineto(277.04191528,408.65710297)
\curveto(277.05191229,408.50710259)(277.03191231,408.39210271)(276.98191528,408.31210297)
\curveto(276.93191241,408.23210287)(276.8419125,408.17710292)(276.71191528,408.14710297)
\curveto(276.65191269,408.12710297)(276.58191276,408.12210298)(276.50191528,408.13210297)
\curveto(276.43191291,408.14210296)(276.36191298,408.14710295)(276.29191528,408.14710297)
\lineto(275.57191528,408.14710297)
\curveto(275.46191388,408.14710295)(275.36191398,408.15210295)(275.27191528,408.16210297)
\curveto(275.18191416,408.17210293)(275.10691423,408.2021029)(275.04691528,408.25210297)
\curveto(274.98691435,408.3021028)(274.95191439,408.37710272)(274.94191528,408.47710297)
\lineto(274.94191528,408.80710297)
\lineto(274.94191528,410.14210297)
\lineto(274.94191528,415.76710297)
\lineto(274.94191528,417.80710297)
\curveto(274.9419144,417.93709316)(274.9369144,418.09209301)(274.92691528,418.27210297)
\curveto(274.92691441,418.45209265)(274.95191439,418.58209252)(275.00191528,418.66210297)
\curveto(275.02191432,418.7020924)(275.04691429,418.73209237)(275.07691528,418.75210297)
\lineto(275.19691528,418.81210297)
\curveto(275.21691412,418.81209229)(275.2419141,418.81209229)(275.27191528,418.81210297)
\curveto(275.30191404,418.82209228)(275.32691401,418.82709227)(275.34691528,418.82710297)
}
}
{
\newrgbcolor{curcolor}{0 0 0}
\pscustom[linestyle=none,fillstyle=solid,fillcolor=curcolor]
{
\newpath
\moveto(280.80410278,418.72210297)
\curveto(280.87409983,418.64209246)(280.9090998,418.52209258)(280.90910278,418.36210297)
\lineto(280.90910278,417.89710297)
\lineto(280.90910278,417.49210297)
\curveto(280.9090998,417.35209375)(280.87409983,417.25709384)(280.80410278,417.20710297)
\curveto(280.74409996,417.15709394)(280.66410004,417.12709397)(280.56410278,417.11710297)
\curveto(280.47410023,417.10709399)(280.37410033,417.102094)(280.26410278,417.10210297)
\lineto(279.42410278,417.10210297)
\curveto(279.31410139,417.102094)(279.21410149,417.10709399)(279.12410278,417.11710297)
\curveto(279.04410166,417.12709397)(278.97410173,417.15709394)(278.91410278,417.20710297)
\curveto(278.87410183,417.23709386)(278.84410186,417.29209381)(278.82410278,417.37210297)
\curveto(278.81410189,417.46209364)(278.8041019,417.55709354)(278.79410278,417.65710297)
\lineto(278.79410278,417.98710297)
\curveto(278.8041019,418.097093)(278.8091019,418.19209291)(278.80910278,418.27210297)
\lineto(278.80910278,418.48210297)
\curveto(278.81910189,418.55209255)(278.83910187,418.61209249)(278.86910278,418.66210297)
\curveto(278.88910182,418.7020924)(278.91410179,418.73209237)(278.94410278,418.75210297)
\lineto(279.06410278,418.81210297)
\curveto(279.08410162,418.81209229)(279.1091016,418.81209229)(279.13910278,418.81210297)
\curveto(279.16910154,418.82209228)(279.19410151,418.82709227)(279.21410278,418.82710297)
\lineto(280.30910278,418.82710297)
\curveto(280.4091003,418.82709227)(280.5041002,418.82209228)(280.59410278,418.81210297)
\curveto(280.68410002,418.8020923)(280.75409995,418.77209233)(280.80410278,418.72210297)
\moveto(280.90910278,408.95710297)
\curveto(280.9090998,408.75710234)(280.9040998,408.58710251)(280.89410278,408.44710297)
\curveto(280.88409982,408.30710279)(280.79409991,408.21210289)(280.62410278,408.16210297)
\curveto(280.56410014,408.14210296)(280.49910021,408.13210297)(280.42910278,408.13210297)
\curveto(280.35910035,408.14210296)(280.28410042,408.14710295)(280.20410278,408.14710297)
\lineto(279.36410278,408.14710297)
\curveto(279.27410143,408.14710295)(279.18410152,408.15210295)(279.09410278,408.16210297)
\curveto(279.01410169,408.17210293)(278.95410175,408.2021029)(278.91410278,408.25210297)
\curveto(278.85410185,408.32210278)(278.81910189,408.40710269)(278.80910278,408.50710297)
\lineto(278.80910278,408.85210297)
\lineto(278.80910278,415.18210297)
\lineto(278.80910278,415.48210297)
\curveto(278.8091019,415.58209552)(278.82910188,415.66209544)(278.86910278,415.72210297)
\curveto(278.92910178,415.79209531)(279.01410169,415.83709526)(279.12410278,415.85710297)
\curveto(279.14410156,415.86709523)(279.16910154,415.86709523)(279.19910278,415.85710297)
\curveto(279.23910147,415.85709524)(279.26910144,415.86209524)(279.28910278,415.87210297)
\lineto(280.03910278,415.87210297)
\lineto(280.23410278,415.87210297)
\curveto(280.31410039,415.88209522)(280.37910033,415.88209522)(280.42910278,415.87210297)
\lineto(280.54910278,415.87210297)
\curveto(280.6091001,415.85209525)(280.66410004,415.83709526)(280.71410278,415.82710297)
\curveto(280.76409994,415.81709528)(280.8040999,415.78709531)(280.83410278,415.73710297)
\curveto(280.87409983,415.68709541)(280.89409981,415.61709548)(280.89410278,415.52710297)
\curveto(280.9040998,415.43709566)(280.9090998,415.34209576)(280.90910278,415.24210297)
\lineto(280.90910278,408.95710297)
}
}
{
\newrgbcolor{curcolor}{0 0 0}
\pscustom[linestyle=none,fillstyle=solid,fillcolor=curcolor]
{
\newpath
\moveto(286.14129028,416.08210297)
\curveto(286.95128512,416.102095)(287.62628445,415.98209512)(288.16629028,415.72210297)
\curveto(288.71628336,415.46209564)(289.15128292,415.09209601)(289.47129028,414.61210297)
\curveto(289.63128244,414.37209673)(289.75128232,414.097097)(289.83129028,413.78710297)
\curveto(289.85128222,413.73709736)(289.86628221,413.67209743)(289.87629028,413.59210297)
\curveto(289.89628218,413.51209759)(289.89628218,413.44209766)(289.87629028,413.38210297)
\curveto(289.83628224,413.27209783)(289.76628231,413.20709789)(289.66629028,413.18710297)
\curveto(289.56628251,413.17709792)(289.44628263,413.17209793)(289.30629028,413.17210297)
\lineto(288.52629028,413.17210297)
\lineto(288.24129028,413.17210297)
\curveto(288.15128392,413.17209793)(288.076284,413.19209791)(288.01629028,413.23210297)
\curveto(287.93628414,413.27209783)(287.88128419,413.33209777)(287.85129028,413.41210297)
\curveto(287.82128425,413.5020976)(287.78128429,413.59209751)(287.73129028,413.68210297)
\curveto(287.6712844,413.79209731)(287.60628447,413.89209721)(287.53629028,413.98210297)
\curveto(287.46628461,414.07209703)(287.38628469,414.15209695)(287.29629028,414.22210297)
\curveto(287.15628492,414.31209679)(287.00128507,414.38209672)(286.83129028,414.43210297)
\curveto(286.7712853,414.45209665)(286.71128536,414.46209664)(286.65129028,414.46210297)
\curveto(286.59128548,414.46209664)(286.53628554,414.47209663)(286.48629028,414.49210297)
\lineto(286.33629028,414.49210297)
\curveto(286.13628594,414.49209661)(285.9762861,414.47209663)(285.85629028,414.43210297)
\curveto(285.56628651,414.34209676)(285.33128674,414.2020969)(285.15129028,414.01210297)
\curveto(284.9712871,413.83209727)(284.82628725,413.61209749)(284.71629028,413.35210297)
\curveto(284.66628741,413.24209786)(284.62628745,413.12209798)(284.59629028,412.99210297)
\curveto(284.5762875,412.87209823)(284.55128752,412.74209836)(284.52129028,412.60210297)
\curveto(284.51128756,412.56209854)(284.50628757,412.52209858)(284.50629028,412.48210297)
\curveto(284.50628757,412.44209866)(284.50128757,412.4020987)(284.49129028,412.36210297)
\curveto(284.4712876,412.26209884)(284.46128761,412.12209898)(284.46129028,411.94210297)
\curveto(284.4712876,411.76209934)(284.48628759,411.62209948)(284.50629028,411.52210297)
\curveto(284.50628757,411.44209966)(284.51128756,411.38709971)(284.52129028,411.35710297)
\curveto(284.54128753,411.28709981)(284.55128752,411.21709988)(284.55129028,411.14710297)
\curveto(284.56128751,411.07710002)(284.5762875,411.00710009)(284.59629028,410.93710297)
\curveto(284.6762874,410.70710039)(284.7712873,410.4971006)(284.88129028,410.30710297)
\curveto(284.99128708,410.11710098)(285.13128694,409.95710114)(285.30129028,409.82710297)
\curveto(285.34128673,409.7971013)(285.40128667,409.76210134)(285.48129028,409.72210297)
\curveto(285.59128648,409.65210145)(285.70128637,409.60710149)(285.81129028,409.58710297)
\curveto(285.93128614,409.56710153)(286.076286,409.54710155)(286.24629028,409.52710297)
\lineto(286.33629028,409.52710297)
\curveto(286.3762857,409.52710157)(286.40628567,409.53210157)(286.42629028,409.54210297)
\lineto(286.56129028,409.54210297)
\curveto(286.63128544,409.56210154)(286.69628538,409.57710152)(286.75629028,409.58710297)
\curveto(286.82628525,409.60710149)(286.89128518,409.62710147)(286.95129028,409.64710297)
\curveto(287.25128482,409.77710132)(287.48128459,409.96710113)(287.64129028,410.21710297)
\curveto(287.68128439,410.26710083)(287.71628436,410.32210078)(287.74629028,410.38210297)
\curveto(287.7762843,410.45210065)(287.80128427,410.51210059)(287.82129028,410.56210297)
\curveto(287.86128421,410.67210043)(287.89628418,410.76710033)(287.92629028,410.84710297)
\curveto(287.95628412,410.93710016)(288.02628405,411.00710009)(288.13629028,411.05710297)
\curveto(288.22628385,411.0971)(288.3712837,411.11209999)(288.57129028,411.10210297)
\lineto(289.06629028,411.10210297)
\lineto(289.27629028,411.10210297)
\curveto(289.35628272,411.11209999)(289.42128265,411.10709999)(289.47129028,411.08710297)
\lineto(289.59129028,411.08710297)
\lineto(289.71129028,411.05710297)
\curveto(289.75128232,411.05710004)(289.78128229,411.04710005)(289.80129028,411.02710297)
\curveto(289.85128222,410.98710011)(289.88128219,410.92710017)(289.89129028,410.84710297)
\curveto(289.91128216,410.77710032)(289.91128216,410.7021004)(289.89129028,410.62210297)
\curveto(289.80128227,410.29210081)(289.69128238,409.9971011)(289.56129028,409.73710297)
\curveto(289.15128292,408.96710213)(288.49628358,408.43210267)(287.59629028,408.13210297)
\curveto(287.49628458,408.102103)(287.39128468,408.08210302)(287.28129028,408.07210297)
\curveto(287.1712849,408.05210305)(287.06128501,408.02710307)(286.95129028,407.99710297)
\curveto(286.89128518,407.98710311)(286.83128524,407.98210312)(286.77129028,407.98210297)
\curveto(286.71128536,407.98210312)(286.65128542,407.97710312)(286.59129028,407.96710297)
\lineto(286.42629028,407.96710297)
\curveto(286.3762857,407.94710315)(286.30128577,407.94210316)(286.20129028,407.95210297)
\curveto(286.10128597,407.95210315)(286.02628605,407.95710314)(285.97629028,407.96710297)
\curveto(285.89628618,407.98710311)(285.82128625,407.9971031)(285.75129028,407.99710297)
\curveto(285.69128638,407.98710311)(285.62628645,407.99210311)(285.55629028,408.01210297)
\lineto(285.40629028,408.04210297)
\curveto(285.35628672,408.04210306)(285.30628677,408.04710305)(285.25629028,408.05710297)
\curveto(285.14628693,408.08710301)(285.04128703,408.11710298)(284.94129028,408.14710297)
\curveto(284.84128723,408.17710292)(284.74628733,408.21210289)(284.65629028,408.25210297)
\curveto(284.18628789,408.45210265)(283.79128828,408.70710239)(283.47129028,409.01710297)
\curveto(283.15128892,409.33710176)(282.89128918,409.73210137)(282.69129028,410.20210297)
\curveto(282.64128943,410.29210081)(282.60128947,410.38710071)(282.57129028,410.48710297)
\lineto(282.48129028,410.81710297)
\curveto(282.4712896,410.85710024)(282.46628961,410.89210021)(282.46629028,410.92210297)
\curveto(282.46628961,410.96210014)(282.45628962,411.00710009)(282.43629028,411.05710297)
\curveto(282.41628966,411.12709997)(282.40628967,411.1970999)(282.40629028,411.26710297)
\curveto(282.40628967,411.34709975)(282.39628968,411.42209968)(282.37629028,411.49210297)
\lineto(282.37629028,411.74710297)
\curveto(282.35628972,411.7970993)(282.34628973,411.85209925)(282.34629028,411.91210297)
\curveto(282.34628973,411.98209912)(282.35628972,412.04209906)(282.37629028,412.09210297)
\curveto(282.38628969,412.14209896)(282.38628969,412.18709891)(282.37629028,412.22710297)
\curveto(282.36628971,412.26709883)(282.36628971,412.30709879)(282.37629028,412.34710297)
\curveto(282.39628968,412.41709868)(282.40128967,412.48209862)(282.39129028,412.54210297)
\curveto(282.39128968,412.6020985)(282.40128967,412.66209844)(282.42129028,412.72210297)
\curveto(282.4712896,412.9020982)(282.51128956,413.07209803)(282.54129028,413.23210297)
\curveto(282.5712895,413.4020977)(282.61628946,413.56709753)(282.67629028,413.72710297)
\curveto(282.89628918,414.23709686)(283.1712889,414.66209644)(283.50129028,415.00210297)
\curveto(283.84128823,415.34209576)(284.2712878,415.61709548)(284.79129028,415.82710297)
\curveto(284.93128714,415.88709521)(285.076287,415.92709517)(285.22629028,415.94710297)
\curveto(285.3762867,415.97709512)(285.53128654,416.01209509)(285.69129028,416.05210297)
\curveto(285.7712863,416.06209504)(285.84628623,416.06709503)(285.91629028,416.06710297)
\curveto(285.98628609,416.06709503)(286.06128601,416.07209503)(286.14129028,416.08210297)
}
}
{
\newrgbcolor{curcolor}{0 0 0}
\pscustom[linestyle=none,fillstyle=solid,fillcolor=curcolor]
{
\newpath
\moveto(298.23457153,408.73210297)
\curveto(298.25456368,408.62210248)(298.26456367,408.51210259)(298.26457153,408.40210297)
\curveto(298.27456366,408.29210281)(298.22456371,408.21710288)(298.11457153,408.17710297)
\curveto(298.05456388,408.14710295)(297.98456395,408.13210297)(297.90457153,408.13210297)
\lineto(297.66457153,408.13210297)
\lineto(296.85457153,408.13210297)
\lineto(296.58457153,408.13210297)
\curveto(296.50456543,408.14210296)(296.4395655,408.16710293)(296.38957153,408.20710297)
\curveto(296.31956562,408.24710285)(296.26456567,408.3021028)(296.22457153,408.37210297)
\curveto(296.19456574,408.45210265)(296.14956579,408.51710258)(296.08957153,408.56710297)
\curveto(296.06956587,408.58710251)(296.04456589,408.6021025)(296.01457153,408.61210297)
\curveto(295.98456595,408.63210247)(295.94456599,408.63710246)(295.89457153,408.62710297)
\curveto(295.84456609,408.60710249)(295.79456614,408.58210252)(295.74457153,408.55210297)
\curveto(295.70456623,408.52210258)(295.65956628,408.4971026)(295.60957153,408.47710297)
\curveto(295.55956638,408.43710266)(295.50456643,408.4021027)(295.44457153,408.37210297)
\lineto(295.26457153,408.28210297)
\curveto(295.1345668,408.22210288)(294.99956694,408.17210293)(294.85957153,408.13210297)
\curveto(294.71956722,408.102103)(294.57456736,408.06710303)(294.42457153,408.02710297)
\curveto(294.35456758,408.00710309)(294.28456765,407.9971031)(294.21457153,407.99710297)
\curveto(294.15456778,407.98710311)(294.08956785,407.97710312)(294.01957153,407.96710297)
\lineto(293.92957153,407.96710297)
\curveto(293.89956804,407.95710314)(293.86956807,407.95210315)(293.83957153,407.95210297)
\lineto(293.67457153,407.95210297)
\curveto(293.57456836,407.93210317)(293.47456846,407.93210317)(293.37457153,407.95210297)
\lineto(293.23957153,407.95210297)
\curveto(293.16956877,407.97210313)(293.09956884,407.98210312)(293.02957153,407.98210297)
\curveto(292.96956897,407.97210313)(292.90956903,407.97710312)(292.84957153,407.99710297)
\curveto(292.74956919,408.01710308)(292.65456928,408.03710306)(292.56457153,408.05710297)
\curveto(292.47456946,408.06710303)(292.38956955,408.09210301)(292.30957153,408.13210297)
\curveto(292.01956992,408.24210286)(291.76957017,408.38210272)(291.55957153,408.55210297)
\curveto(291.35957058,408.73210237)(291.19957074,408.96710213)(291.07957153,409.25710297)
\curveto(291.04957089,409.32710177)(291.01957092,409.4021017)(290.98957153,409.48210297)
\curveto(290.96957097,409.56210154)(290.94957099,409.64710145)(290.92957153,409.73710297)
\curveto(290.90957103,409.78710131)(290.89957104,409.83710126)(290.89957153,409.88710297)
\curveto(290.90957103,409.93710116)(290.90957103,409.98710111)(290.89957153,410.03710297)
\curveto(290.88957105,410.06710103)(290.87957106,410.12710097)(290.86957153,410.21710297)
\curveto(290.86957107,410.31710078)(290.87457106,410.38710071)(290.88457153,410.42710297)
\curveto(290.90457103,410.52710057)(290.91457102,410.61210049)(290.91457153,410.68210297)
\lineto(291.00457153,411.01210297)
\curveto(291.0345709,411.13209997)(291.07457086,411.23709986)(291.12457153,411.32710297)
\curveto(291.29457064,411.61709948)(291.48957045,411.83709926)(291.70957153,411.98710297)
\curveto(291.92957001,412.13709896)(292.20956973,412.26709883)(292.54957153,412.37710297)
\curveto(292.67956926,412.42709867)(292.81456912,412.46209864)(292.95457153,412.48210297)
\curveto(293.09456884,412.5020986)(293.2345687,412.52709857)(293.37457153,412.55710297)
\curveto(293.45456848,412.57709852)(293.5395684,412.58709851)(293.62957153,412.58710297)
\curveto(293.71956822,412.5970985)(293.80956813,412.61209849)(293.89957153,412.63210297)
\curveto(293.96956797,412.65209845)(294.0395679,412.65709844)(294.10957153,412.64710297)
\curveto(294.17956776,412.64709845)(294.25456768,412.65709844)(294.33457153,412.67710297)
\curveto(294.40456753,412.6970984)(294.47456746,412.70709839)(294.54457153,412.70710297)
\curveto(294.61456732,412.70709839)(294.68956725,412.71709838)(294.76957153,412.73710297)
\curveto(294.97956696,412.78709831)(295.16956677,412.82709827)(295.33957153,412.85710297)
\curveto(295.51956642,412.8970982)(295.67956626,412.98709811)(295.81957153,413.12710297)
\curveto(295.90956603,413.21709788)(295.96956597,413.31709778)(295.99957153,413.42710297)
\curveto(296.00956593,413.45709764)(296.00956593,413.48209762)(295.99957153,413.50210297)
\curveto(295.99956594,413.52209758)(296.00456593,413.54209756)(296.01457153,413.56210297)
\curveto(296.02456591,413.58209752)(296.02956591,413.61209749)(296.02957153,413.65210297)
\lineto(296.02957153,413.74210297)
\lineto(295.99957153,413.86210297)
\curveto(295.99956594,413.9020972)(295.99456594,413.93709716)(295.98457153,413.96710297)
\curveto(295.88456605,414.26709683)(295.67456626,414.47209663)(295.35457153,414.58210297)
\curveto(295.26456667,414.61209649)(295.15456678,414.63209647)(295.02457153,414.64210297)
\curveto(294.90456703,414.66209644)(294.77956716,414.66709643)(294.64957153,414.65710297)
\curveto(294.51956742,414.65709644)(294.39456754,414.64709645)(294.27457153,414.62710297)
\curveto(294.15456778,414.60709649)(294.04956789,414.58209652)(293.95957153,414.55210297)
\curveto(293.89956804,414.53209657)(293.8395681,414.5020966)(293.77957153,414.46210297)
\curveto(293.72956821,414.43209667)(293.67956826,414.3970967)(293.62957153,414.35710297)
\curveto(293.57956836,414.31709678)(293.52456841,414.26209684)(293.46457153,414.19210297)
\curveto(293.41456852,414.12209698)(293.37956856,414.05709704)(293.35957153,413.99710297)
\curveto(293.30956863,413.8970972)(293.26456867,413.8020973)(293.22457153,413.71210297)
\curveto(293.19456874,413.62209748)(293.12456881,413.56209754)(293.01457153,413.53210297)
\curveto(292.934569,413.51209759)(292.84956909,413.5020976)(292.75957153,413.50210297)
\lineto(292.48957153,413.50210297)
\lineto(291.91957153,413.50210297)
\curveto(291.86957007,413.5020976)(291.81957012,413.4970976)(291.76957153,413.48710297)
\curveto(291.71957022,413.48709761)(291.67457026,413.49209761)(291.63457153,413.50210297)
\lineto(291.49957153,413.50210297)
\curveto(291.47957046,413.51209759)(291.45457048,413.51709758)(291.42457153,413.51710297)
\curveto(291.39457054,413.51709758)(291.36957057,413.52709757)(291.34957153,413.54710297)
\curveto(291.26957067,413.56709753)(291.21457072,413.63209747)(291.18457153,413.74210297)
\curveto(291.17457076,413.79209731)(291.17457076,413.84209726)(291.18457153,413.89210297)
\curveto(291.19457074,413.94209716)(291.20457073,413.98709711)(291.21457153,414.02710297)
\curveto(291.24457069,414.13709696)(291.27457066,414.23709686)(291.30457153,414.32710297)
\curveto(291.34457059,414.42709667)(291.38957055,414.51709658)(291.43957153,414.59710297)
\lineto(291.52957153,414.74710297)
\lineto(291.61957153,414.89710297)
\curveto(291.69957024,415.00709609)(291.79957014,415.11209599)(291.91957153,415.21210297)
\curveto(291.93957,415.22209588)(291.96956997,415.24709585)(292.00957153,415.28710297)
\curveto(292.05956988,415.32709577)(292.10456983,415.36209574)(292.14457153,415.39210297)
\curveto(292.18456975,415.42209568)(292.22956971,415.45209565)(292.27957153,415.48210297)
\curveto(292.44956949,415.59209551)(292.62956931,415.67709542)(292.81957153,415.73710297)
\curveto(293.00956893,415.80709529)(293.20456873,415.87209523)(293.40457153,415.93210297)
\curveto(293.52456841,415.96209514)(293.64956829,415.98209512)(293.77957153,415.99210297)
\curveto(293.90956803,416.0020951)(294.0395679,416.02209508)(294.16957153,416.05210297)
\curveto(294.20956773,416.06209504)(294.26956767,416.06209504)(294.34957153,416.05210297)
\curveto(294.4395675,416.04209506)(294.49456744,416.04709505)(294.51457153,416.06710297)
\curveto(294.92456701,416.07709502)(295.31456662,416.06209504)(295.68457153,416.02210297)
\curveto(296.06456587,415.98209512)(296.40456553,415.90709519)(296.70457153,415.79710297)
\curveto(297.01456492,415.68709541)(297.27956466,415.53709556)(297.49957153,415.34710297)
\curveto(297.71956422,415.16709593)(297.88956405,414.93209617)(298.00957153,414.64210297)
\curveto(298.07956386,414.47209663)(298.11956382,414.27709682)(298.12957153,414.05710297)
\curveto(298.1395638,413.83709726)(298.14456379,413.61209749)(298.14457153,413.38210297)
\lineto(298.14457153,410.03710297)
\lineto(298.14457153,409.45210297)
\curveto(298.14456379,409.26210184)(298.16456377,409.08710201)(298.20457153,408.92710297)
\curveto(298.21456372,408.8971022)(298.21956372,408.86210224)(298.21957153,408.82210297)
\curveto(298.21956372,408.79210231)(298.22456371,408.76210234)(298.23457153,408.73210297)
\moveto(296.02957153,411.04210297)
\curveto(296.0395659,411.09210001)(296.04456589,411.14709995)(296.04457153,411.20710297)
\curveto(296.04456589,411.27709982)(296.0395659,411.33709976)(296.02957153,411.38710297)
\curveto(296.00956593,411.44709965)(295.99956594,411.5020996)(295.99957153,411.55210297)
\curveto(295.99956594,411.6020995)(295.97956596,411.64209946)(295.93957153,411.67210297)
\curveto(295.88956605,411.71209939)(295.81456612,411.73209937)(295.71457153,411.73210297)
\curveto(295.67456626,411.72209938)(295.6395663,411.71209939)(295.60957153,411.70210297)
\curveto(295.57956636,411.7020994)(295.54456639,411.6970994)(295.50457153,411.68710297)
\curveto(295.4345665,411.66709943)(295.35956658,411.65209945)(295.27957153,411.64210297)
\curveto(295.19956674,411.63209947)(295.11956682,411.61709948)(295.03957153,411.59710297)
\curveto(295.00956693,411.58709951)(294.96456697,411.58209952)(294.90457153,411.58210297)
\curveto(294.77456716,411.55209955)(294.64456729,411.53209957)(294.51457153,411.52210297)
\curveto(294.38456755,411.51209959)(294.25956768,411.48709961)(294.13957153,411.44710297)
\curveto(294.05956788,411.42709967)(293.98456795,411.40709969)(293.91457153,411.38710297)
\curveto(293.84456809,411.37709972)(293.77456816,411.35709974)(293.70457153,411.32710297)
\curveto(293.49456844,411.23709986)(293.31456862,411.1021)(293.16457153,410.92210297)
\curveto(293.02456891,410.74210036)(292.97456896,410.49210061)(293.01457153,410.17210297)
\curveto(293.0345689,410.0021011)(293.08956885,409.86210124)(293.17957153,409.75210297)
\curveto(293.24956869,409.64210146)(293.35456858,409.55210155)(293.49457153,409.48210297)
\curveto(293.6345683,409.42210168)(293.78456815,409.37710172)(293.94457153,409.34710297)
\curveto(294.11456782,409.31710178)(294.28956765,409.30710179)(294.46957153,409.31710297)
\curveto(294.65956728,409.33710176)(294.8345671,409.37210173)(294.99457153,409.42210297)
\curveto(295.25456668,409.5021016)(295.45956648,409.62710147)(295.60957153,409.79710297)
\curveto(295.75956618,409.97710112)(295.87456606,410.1971009)(295.95457153,410.45710297)
\curveto(295.97456596,410.52710057)(295.98456595,410.5971005)(295.98457153,410.66710297)
\curveto(295.99456594,410.74710035)(296.00956593,410.82710027)(296.02957153,410.90710297)
\lineto(296.02957153,411.04210297)
}
}
{
\newrgbcolor{curcolor}{0 0 0}
\pscustom[linestyle=none,fillstyle=solid,fillcolor=curcolor]
{
\newpath
\moveto(307.38785278,408.98710297)
\lineto(307.38785278,408.56710297)
\curveto(307.38784441,408.43710266)(307.35784444,408.33210277)(307.29785278,408.25210297)
\curveto(307.24784455,408.2021029)(307.18284462,408.16710293)(307.10285278,408.14710297)
\curveto(307.02284478,408.13710296)(306.93284487,408.13210297)(306.83285278,408.13210297)
\lineto(306.00785278,408.13210297)
\lineto(305.72285278,408.13210297)
\curveto(305.64284616,408.14210296)(305.57784622,408.16710293)(305.52785278,408.20710297)
\curveto(305.45784634,408.25710284)(305.41784638,408.32210278)(305.40785278,408.40210297)
\curveto(305.3978464,408.48210262)(305.37784642,408.56210254)(305.34785278,408.64210297)
\curveto(305.32784647,408.66210244)(305.30784649,408.67710242)(305.28785278,408.68710297)
\curveto(305.27784652,408.70710239)(305.26284654,408.72710237)(305.24285278,408.74710297)
\curveto(305.13284667,408.74710235)(305.05284675,408.72210238)(305.00285278,408.67210297)
\lineto(304.85285278,408.52210297)
\curveto(304.78284702,408.47210263)(304.71784708,408.42710267)(304.65785278,408.38710297)
\curveto(304.5978472,408.35710274)(304.53284727,408.31710278)(304.46285278,408.26710297)
\curveto(304.42284738,408.24710285)(304.37784742,408.22710287)(304.32785278,408.20710297)
\curveto(304.28784751,408.18710291)(304.24284756,408.16710293)(304.19285278,408.14710297)
\curveto(304.05284775,408.097103)(303.9028479,408.05210305)(303.74285278,408.01210297)
\curveto(303.69284811,407.99210311)(303.64784815,407.98210312)(303.60785278,407.98210297)
\curveto(303.56784823,407.98210312)(303.52784827,407.97710312)(303.48785278,407.96710297)
\lineto(303.35285278,407.96710297)
\curveto(303.32284848,407.95710314)(303.28284852,407.95210315)(303.23285278,407.95210297)
\lineto(303.09785278,407.95210297)
\curveto(303.03784876,407.93210317)(302.94784885,407.92710317)(302.82785278,407.93710297)
\curveto(302.70784909,407.93710316)(302.62284918,407.94710315)(302.57285278,407.96710297)
\curveto(302.5028493,407.98710311)(302.43784936,407.9971031)(302.37785278,407.99710297)
\curveto(302.32784947,407.98710311)(302.27284953,407.99210311)(302.21285278,408.01210297)
\lineto(301.85285278,408.13210297)
\curveto(301.74285006,408.16210294)(301.63285017,408.2021029)(301.52285278,408.25210297)
\curveto(301.17285063,408.4021027)(300.85785094,408.63210247)(300.57785278,408.94210297)
\curveto(300.30785149,409.26210184)(300.09285171,409.5971015)(299.93285278,409.94710297)
\curveto(299.88285192,410.05710104)(299.84285196,410.16210094)(299.81285278,410.26210297)
\curveto(299.78285202,410.37210073)(299.74785205,410.48210062)(299.70785278,410.59210297)
\curveto(299.6978521,410.63210047)(299.69285211,410.66710043)(299.69285278,410.69710297)
\curveto(299.69285211,410.73710036)(299.68285212,410.78210032)(299.66285278,410.83210297)
\curveto(299.64285216,410.91210019)(299.62285218,410.9971001)(299.60285278,411.08710297)
\curveto(299.59285221,411.18709991)(299.57785222,411.28709981)(299.55785278,411.38710297)
\curveto(299.54785225,411.41709968)(299.54285226,411.45209965)(299.54285278,411.49210297)
\curveto(299.55285225,411.53209957)(299.55285225,411.56709953)(299.54285278,411.59710297)
\lineto(299.54285278,411.73210297)
\curveto(299.54285226,411.78209932)(299.53785226,411.83209927)(299.52785278,411.88210297)
\curveto(299.51785228,411.93209917)(299.51285229,411.98709911)(299.51285278,412.04710297)
\curveto(299.51285229,412.11709898)(299.51785228,412.17209893)(299.52785278,412.21210297)
\curveto(299.53785226,412.26209884)(299.54285226,412.30709879)(299.54285278,412.34710297)
\lineto(299.54285278,412.49710297)
\curveto(299.55285225,412.54709855)(299.55285225,412.59209851)(299.54285278,412.63210297)
\curveto(299.54285226,412.68209842)(299.55285225,412.73209837)(299.57285278,412.78210297)
\curveto(299.59285221,412.89209821)(299.60785219,412.9970981)(299.61785278,413.09710297)
\curveto(299.63785216,413.1970979)(299.66285214,413.2970978)(299.69285278,413.39710297)
\curveto(299.73285207,413.51709758)(299.76785203,413.63209747)(299.79785278,413.74210297)
\curveto(299.82785197,413.85209725)(299.86785193,413.96209714)(299.91785278,414.07210297)
\curveto(300.05785174,414.37209673)(300.23285157,414.65709644)(300.44285278,414.92710297)
\curveto(300.46285134,414.95709614)(300.48785131,414.98209612)(300.51785278,415.00210297)
\curveto(300.55785124,415.03209607)(300.58785121,415.06209604)(300.60785278,415.09210297)
\curveto(300.64785115,415.14209596)(300.68785111,415.18709591)(300.72785278,415.22710297)
\curveto(300.76785103,415.26709583)(300.81285099,415.30709579)(300.86285278,415.34710297)
\curveto(300.9028509,415.36709573)(300.93785086,415.39209571)(300.96785278,415.42210297)
\curveto(300.9978508,415.46209564)(301.03285077,415.49209561)(301.07285278,415.51210297)
\curveto(301.32285048,415.68209542)(301.61285019,415.82209528)(301.94285278,415.93210297)
\curveto(302.01284979,415.95209515)(302.08284972,415.96709513)(302.15285278,415.97710297)
\curveto(302.23284957,415.98709511)(302.31284949,416.0020951)(302.39285278,416.02210297)
\curveto(302.46284934,416.04209506)(302.55284925,416.05209505)(302.66285278,416.05210297)
\curveto(302.77284903,416.06209504)(302.88284892,416.06709503)(302.99285278,416.06710297)
\curveto(303.1028487,416.06709503)(303.20784859,416.06209504)(303.30785278,416.05210297)
\curveto(303.41784838,416.04209506)(303.50784829,416.02709507)(303.57785278,416.00710297)
\curveto(303.72784807,415.95709514)(303.87284793,415.91209519)(304.01285278,415.87210297)
\curveto(304.15284765,415.83209527)(304.28284752,415.77709532)(304.40285278,415.70710297)
\curveto(304.47284733,415.65709544)(304.53784726,415.60709549)(304.59785278,415.55710297)
\curveto(304.65784714,415.51709558)(304.72284708,415.47209563)(304.79285278,415.42210297)
\curveto(304.83284697,415.39209571)(304.88784691,415.35209575)(304.95785278,415.30210297)
\curveto(305.03784676,415.25209585)(305.11284669,415.25209585)(305.18285278,415.30210297)
\curveto(305.22284658,415.32209578)(305.24284656,415.35709574)(305.24285278,415.40710297)
\curveto(305.24284656,415.45709564)(305.25284655,415.50709559)(305.27285278,415.55710297)
\lineto(305.27285278,415.70710297)
\curveto(305.28284652,415.73709536)(305.28784651,415.77209533)(305.28785278,415.81210297)
\lineto(305.28785278,415.93210297)
\lineto(305.28785278,417.97210297)
\curveto(305.28784651,418.08209302)(305.28284652,418.2020929)(305.27285278,418.33210297)
\curveto(305.27284653,418.47209263)(305.2978465,418.57709252)(305.34785278,418.64710297)
\curveto(305.38784641,418.72709237)(305.46284634,418.77709232)(305.57285278,418.79710297)
\curveto(305.59284621,418.80709229)(305.61284619,418.80709229)(305.63285278,418.79710297)
\curveto(305.65284615,418.7970923)(305.67284613,418.8020923)(305.69285278,418.81210297)
\lineto(306.75785278,418.81210297)
\curveto(306.87784492,418.81209229)(306.98784481,418.80709229)(307.08785278,418.79710297)
\curveto(307.18784461,418.78709231)(307.26284454,418.74709235)(307.31285278,418.67710297)
\curveto(307.36284444,418.5970925)(307.38784441,418.49209261)(307.38785278,418.36210297)
\lineto(307.38785278,418.00210297)
\lineto(307.38785278,408.98710297)
\moveto(305.34785278,411.92710297)
\curveto(305.35784644,411.96709913)(305.35784644,412.00709909)(305.34785278,412.04710297)
\lineto(305.34785278,412.18210297)
\curveto(305.34784645,412.28209882)(305.34284646,412.38209872)(305.33285278,412.48210297)
\curveto(305.32284648,412.58209852)(305.30784649,412.67209843)(305.28785278,412.75210297)
\curveto(305.26784653,412.86209824)(305.24784655,412.96209814)(305.22785278,413.05210297)
\curveto(305.21784658,413.14209796)(305.19284661,413.22709787)(305.15285278,413.30710297)
\curveto(305.01284679,413.66709743)(304.80784699,413.95209715)(304.53785278,414.16210297)
\curveto(304.27784752,414.37209673)(303.8978479,414.47709662)(303.39785278,414.47710297)
\curveto(303.33784846,414.47709662)(303.25784854,414.46709663)(303.15785278,414.44710297)
\curveto(303.07784872,414.42709667)(303.0028488,414.40709669)(302.93285278,414.38710297)
\curveto(302.87284893,414.37709672)(302.81284899,414.35709674)(302.75285278,414.32710297)
\curveto(302.48284932,414.21709688)(302.27284953,414.04709705)(302.12285278,413.81710297)
\curveto(301.97284983,413.58709751)(301.85284995,413.32709777)(301.76285278,413.03710297)
\curveto(301.73285007,412.93709816)(301.71285009,412.83709826)(301.70285278,412.73710297)
\curveto(301.69285011,412.63709846)(301.67285013,412.53209857)(301.64285278,412.42210297)
\lineto(301.64285278,412.21210297)
\curveto(301.62285018,412.12209898)(301.61785018,411.9970991)(301.62785278,411.83710297)
\curveto(301.63785016,411.68709941)(301.65285015,411.57709952)(301.67285278,411.50710297)
\lineto(301.67285278,411.41710297)
\curveto(301.68285012,411.3970997)(301.68785011,411.37709972)(301.68785278,411.35710297)
\curveto(301.70785009,411.27709982)(301.72285008,411.2020999)(301.73285278,411.13210297)
\curveto(301.75285005,411.06210004)(301.77285003,410.98710011)(301.79285278,410.90710297)
\curveto(301.96284984,410.38710071)(302.25284955,410.0021011)(302.66285278,409.75210297)
\curveto(302.79284901,409.66210144)(302.97284883,409.59210151)(303.20285278,409.54210297)
\curveto(303.24284856,409.53210157)(303.3028485,409.52710157)(303.38285278,409.52710297)
\curveto(303.41284839,409.51710158)(303.45784834,409.50710159)(303.51785278,409.49710297)
\curveto(303.58784821,409.4971016)(303.64284816,409.5021016)(303.68285278,409.51210297)
\curveto(303.76284804,409.53210157)(303.84284796,409.54710155)(303.92285278,409.55710297)
\curveto(304.0028478,409.56710153)(304.08284772,409.58710151)(304.16285278,409.61710297)
\curveto(304.41284739,409.72710137)(304.61284719,409.86710123)(304.76285278,410.03710297)
\curveto(304.91284689,410.20710089)(305.04284676,410.42210068)(305.15285278,410.68210297)
\curveto(305.19284661,410.77210033)(305.22284658,410.86210024)(305.24285278,410.95210297)
\curveto(305.26284654,411.05210005)(305.28284652,411.15709994)(305.30285278,411.26710297)
\curveto(305.31284649,411.31709978)(305.31284649,411.36209974)(305.30285278,411.40210297)
\curveto(305.3028465,411.45209965)(305.31284649,411.5020996)(305.33285278,411.55210297)
\curveto(305.34284646,411.58209952)(305.34784645,411.61709948)(305.34785278,411.65710297)
\lineto(305.34785278,411.79210297)
\lineto(305.34785278,411.92710297)
}
}
{
\newrgbcolor{curcolor}{0 0 0}
\pscustom[linestyle=none,fillstyle=solid,fillcolor=curcolor]
{
\newpath
\moveto(316.73777466,412.31710297)
\curveto(316.75776609,412.25709884)(316.76776608,412.17209893)(316.76777466,412.06210297)
\curveto(316.76776608,411.95209915)(316.75776609,411.86709923)(316.73777466,411.80710297)
\lineto(316.73777466,411.65710297)
\curveto(316.71776613,411.57709952)(316.70776614,411.4970996)(316.70777466,411.41710297)
\curveto(316.71776613,411.33709976)(316.71276613,411.25709984)(316.69277466,411.17710297)
\curveto(316.67276617,411.10709999)(316.65776619,411.04210006)(316.64777466,410.98210297)
\curveto(316.63776621,410.92210018)(316.62776622,410.85710024)(316.61777466,410.78710297)
\curveto(316.57776627,410.67710042)(316.5427663,410.56210054)(316.51277466,410.44210297)
\curveto(316.48276636,410.33210077)(316.4427664,410.22710087)(316.39277466,410.12710297)
\curveto(316.18276666,409.64710145)(315.90776694,409.25710184)(315.56777466,408.95710297)
\curveto(315.22776762,408.65710244)(314.81776803,408.40710269)(314.33777466,408.20710297)
\curveto(314.21776863,408.15710294)(314.09276875,408.12210298)(313.96277466,408.10210297)
\curveto(313.842769,408.07210303)(313.71776913,408.04210306)(313.58777466,408.01210297)
\curveto(313.53776931,407.99210311)(313.48276936,407.98210312)(313.42277466,407.98210297)
\curveto(313.36276948,407.98210312)(313.30776954,407.97710312)(313.25777466,407.96710297)
\lineto(313.15277466,407.96710297)
\curveto(313.12276972,407.95710314)(313.09276975,407.95210315)(313.06277466,407.95210297)
\curveto(313.01276983,407.94210316)(312.93276991,407.93710316)(312.82277466,407.93710297)
\curveto(312.71277013,407.92710317)(312.62777022,407.93210317)(312.56777466,407.95210297)
\lineto(312.41777466,407.95210297)
\curveto(312.36777048,407.96210314)(312.31277053,407.96710313)(312.25277466,407.96710297)
\curveto(312.20277064,407.95710314)(312.15277069,407.96210314)(312.10277466,407.98210297)
\curveto(312.06277078,407.99210311)(312.02277082,407.9971031)(311.98277466,407.99710297)
\curveto(311.95277089,407.9971031)(311.91277093,408.0021031)(311.86277466,408.01210297)
\curveto(311.76277108,408.04210306)(311.66277118,408.06710303)(311.56277466,408.08710297)
\curveto(311.46277138,408.10710299)(311.36777148,408.13710296)(311.27777466,408.17710297)
\curveto(311.15777169,408.21710288)(311.0427718,408.25710284)(310.93277466,408.29710297)
\curveto(310.83277201,408.33710276)(310.72777212,408.38710271)(310.61777466,408.44710297)
\curveto(310.26777258,408.65710244)(309.96777288,408.9021022)(309.71777466,409.18210297)
\curveto(309.46777338,409.46210164)(309.25777359,409.7971013)(309.08777466,410.18710297)
\curveto(309.03777381,410.27710082)(308.99777385,410.37210073)(308.96777466,410.47210297)
\curveto(308.9477739,410.57210053)(308.92277392,410.67710042)(308.89277466,410.78710297)
\curveto(308.87277397,410.83710026)(308.86277398,410.88210022)(308.86277466,410.92210297)
\curveto(308.86277398,410.96210014)(308.85277399,411.00710009)(308.83277466,411.05710297)
\curveto(308.81277403,411.13709996)(308.80277404,411.21709988)(308.80277466,411.29710297)
\curveto(308.80277404,411.38709971)(308.79277405,411.47209963)(308.77277466,411.55210297)
\curveto(308.76277408,411.6020995)(308.75777409,411.64709945)(308.75777466,411.68710297)
\lineto(308.75777466,411.82210297)
\curveto(308.73777411,411.88209922)(308.72777412,411.96709913)(308.72777466,412.07710297)
\curveto(308.73777411,412.18709891)(308.75277409,412.27209883)(308.77277466,412.33210297)
\lineto(308.77277466,412.43710297)
\curveto(308.78277406,412.48709861)(308.78277406,412.53709856)(308.77277466,412.58710297)
\curveto(308.77277407,412.64709845)(308.78277406,412.7020984)(308.80277466,412.75210297)
\curveto(308.81277403,412.8020983)(308.81777403,412.84709825)(308.81777466,412.88710297)
\curveto(308.81777403,412.93709816)(308.82777402,412.98709811)(308.84777466,413.03710297)
\curveto(308.88777396,413.16709793)(308.92277392,413.29209781)(308.95277466,413.41210297)
\curveto(308.98277386,413.54209756)(309.02277382,413.66709743)(309.07277466,413.78710297)
\curveto(309.25277359,414.1970969)(309.46777338,414.53709656)(309.71777466,414.80710297)
\curveto(309.96777288,415.08709601)(310.27277257,415.34209576)(310.63277466,415.57210297)
\curveto(310.73277211,415.62209548)(310.83777201,415.66709543)(310.94777466,415.70710297)
\curveto(311.05777179,415.74709535)(311.16777168,415.79209531)(311.27777466,415.84210297)
\curveto(311.40777144,415.89209521)(311.5427713,415.92709517)(311.68277466,415.94710297)
\curveto(311.82277102,415.96709513)(311.96777088,415.9970951)(312.11777466,416.03710297)
\curveto(312.19777065,416.04709505)(312.27277057,416.05209505)(312.34277466,416.05210297)
\curveto(312.41277043,416.05209505)(312.48277036,416.05709504)(312.55277466,416.06710297)
\curveto(313.13276971,416.07709502)(313.63276921,416.01709508)(314.05277466,415.88710297)
\curveto(314.48276836,415.75709534)(314.86276798,415.57709552)(315.19277466,415.34710297)
\curveto(315.30276754,415.26709583)(315.41276743,415.17709592)(315.52277466,415.07710297)
\curveto(315.6427672,414.98709611)(315.7427671,414.88709621)(315.82277466,414.77710297)
\curveto(315.90276694,414.67709642)(315.97276687,414.57709652)(316.03277466,414.47710297)
\curveto(316.10276674,414.37709672)(316.17276667,414.27209683)(316.24277466,414.16210297)
\curveto(316.31276653,414.05209705)(316.36776648,413.93209717)(316.40777466,413.80210297)
\curveto(316.4477664,413.68209742)(316.49276635,413.55209755)(316.54277466,413.41210297)
\curveto(316.57276627,413.33209777)(316.59776625,413.24709785)(316.61777466,413.15710297)
\lineto(316.67777466,412.88710297)
\curveto(316.68776616,412.84709825)(316.69276615,412.80709829)(316.69277466,412.76710297)
\curveto(316.69276615,412.72709837)(316.69776615,412.68709841)(316.70777466,412.64710297)
\curveto(316.72776612,412.5970985)(316.73276611,412.54209856)(316.72277466,412.48210297)
\curveto(316.71276613,412.42209868)(316.71776613,412.36709873)(316.73777466,412.31710297)
\moveto(314.63777466,411.77710297)
\curveto(314.6477682,411.82709927)(314.65276819,411.8970992)(314.65277466,411.98710297)
\curveto(314.65276819,412.08709901)(314.6477682,412.16209894)(314.63777466,412.21210297)
\lineto(314.63777466,412.33210297)
\curveto(314.61776823,412.38209872)(314.60776824,412.43709866)(314.60777466,412.49710297)
\curveto(314.60776824,412.55709854)(314.60276824,412.61209849)(314.59277466,412.66210297)
\curveto(314.59276825,412.7020984)(314.58776826,412.73209837)(314.57777466,412.75210297)
\lineto(314.51777466,412.99210297)
\curveto(314.50776834,413.08209802)(314.48776836,413.16709793)(314.45777466,413.24710297)
\curveto(314.3477685,413.50709759)(314.21776863,413.72709737)(314.06777466,413.90710297)
\curveto(313.91776893,414.097097)(313.71776913,414.24709685)(313.46777466,414.35710297)
\curveto(313.40776944,414.37709672)(313.3477695,414.39209671)(313.28777466,414.40210297)
\curveto(313.22776962,414.42209668)(313.16276968,414.44209666)(313.09277466,414.46210297)
\curveto(313.01276983,414.48209662)(312.92776992,414.48709661)(312.83777466,414.47710297)
\lineto(312.56777466,414.47710297)
\curveto(312.53777031,414.45709664)(312.50277034,414.44709665)(312.46277466,414.44710297)
\curveto(312.42277042,414.45709664)(312.38777046,414.45709664)(312.35777466,414.44710297)
\lineto(312.14777466,414.38710297)
\curveto(312.08777076,414.37709672)(312.03277081,414.35709674)(311.98277466,414.32710297)
\curveto(311.73277111,414.21709688)(311.52777132,414.05709704)(311.36777466,413.84710297)
\curveto(311.21777163,413.64709745)(311.09777175,413.41209769)(311.00777466,413.14210297)
\curveto(310.97777187,413.04209806)(310.95277189,412.93709816)(310.93277466,412.82710297)
\curveto(310.92277192,412.71709838)(310.90777194,412.60709849)(310.88777466,412.49710297)
\curveto(310.87777197,412.44709865)(310.87277197,412.3970987)(310.87277466,412.34710297)
\lineto(310.87277466,412.19710297)
\curveto(310.85277199,412.12709897)(310.842772,412.02209908)(310.84277466,411.88210297)
\curveto(310.85277199,411.74209936)(310.86777198,411.63709946)(310.88777466,411.56710297)
\lineto(310.88777466,411.43210297)
\curveto(310.90777194,411.35209975)(310.92277192,411.27209983)(310.93277466,411.19210297)
\curveto(310.9427719,411.12209998)(310.95777189,411.04710005)(310.97777466,410.96710297)
\curveto(311.07777177,410.66710043)(311.18277166,410.42210068)(311.29277466,410.23210297)
\curveto(311.41277143,410.05210105)(311.59777125,409.88710121)(311.84777466,409.73710297)
\curveto(311.91777093,409.68710141)(311.99277085,409.64710145)(312.07277466,409.61710297)
\curveto(312.16277068,409.58710151)(312.25277059,409.56210154)(312.34277466,409.54210297)
\curveto(312.38277046,409.53210157)(312.41777043,409.52710157)(312.44777466,409.52710297)
\curveto(312.47777037,409.53710156)(312.51277033,409.53710156)(312.55277466,409.52710297)
\lineto(312.67277466,409.49710297)
\curveto(312.72277012,409.4971016)(312.76777008,409.5021016)(312.80777466,409.51210297)
\lineto(312.92777466,409.51210297)
\curveto(313.00776984,409.53210157)(313.08776976,409.54710155)(313.16777466,409.55710297)
\curveto(313.2477696,409.56710153)(313.32276952,409.58710151)(313.39277466,409.61710297)
\curveto(313.65276919,409.71710138)(313.86276898,409.85210125)(314.02277466,410.02210297)
\curveto(314.18276866,410.19210091)(314.31776853,410.4021007)(314.42777466,410.65210297)
\curveto(314.46776838,410.75210035)(314.49776835,410.85210025)(314.51777466,410.95210297)
\curveto(314.53776831,411.05210005)(314.56276828,411.15709994)(314.59277466,411.26710297)
\curveto(314.60276824,411.30709979)(314.60776824,411.34209976)(314.60777466,411.37210297)
\curveto(314.60776824,411.41209969)(314.61276823,411.45209965)(314.62277466,411.49210297)
\lineto(314.62277466,411.62710297)
\curveto(314.62276822,411.67709942)(314.62776822,411.72709937)(314.63777466,411.77710297)
}
}
{
\newrgbcolor{curcolor}{0 0 0}
\pscustom[linestyle=none,fillstyle=solid,fillcolor=curcolor]
{
\newpath
\moveto(321.10769653,416.08210297)
\curveto(321.85769203,416.102095)(322.50769138,416.01709508)(323.05769653,415.82710297)
\curveto(323.61769027,415.64709545)(324.04268985,415.33209577)(324.33269653,414.88210297)
\curveto(324.40268949,414.77209633)(324.46268943,414.65709644)(324.51269653,414.53710297)
\curveto(324.57268932,414.42709667)(324.62268927,414.3020968)(324.66269653,414.16210297)
\curveto(324.68268921,414.102097)(324.6926892,414.03709706)(324.69269653,413.96710297)
\curveto(324.6926892,413.8970972)(324.68268921,413.83709726)(324.66269653,413.78710297)
\curveto(324.62268927,413.72709737)(324.56768932,413.68709741)(324.49769653,413.66710297)
\curveto(324.44768944,413.64709745)(324.3876895,413.63709746)(324.31769653,413.63710297)
\lineto(324.10769653,413.63710297)
\lineto(323.44769653,413.63710297)
\curveto(323.37769051,413.63709746)(323.30769058,413.63209747)(323.23769653,413.62210297)
\curveto(323.16769072,413.62209748)(323.10269079,413.63209747)(323.04269653,413.65210297)
\curveto(322.94269095,413.67209743)(322.86769102,413.71209739)(322.81769653,413.77210297)
\curveto(322.76769112,413.83209727)(322.72269117,413.89209721)(322.68269653,413.95210297)
\lineto(322.56269653,414.16210297)
\curveto(322.53269136,414.24209686)(322.48269141,414.30709679)(322.41269653,414.35710297)
\curveto(322.31269158,414.43709666)(322.21269168,414.4970966)(322.11269653,414.53710297)
\curveto(322.02269187,414.57709652)(321.90769198,414.61209649)(321.76769653,414.64210297)
\curveto(321.69769219,414.66209644)(321.5926923,414.67709642)(321.45269653,414.68710297)
\curveto(321.32269257,414.6970964)(321.22269267,414.69209641)(321.15269653,414.67210297)
\lineto(321.04769653,414.67210297)
\lineto(320.89769653,414.64210297)
\curveto(320.85769303,414.64209646)(320.81269308,414.63709646)(320.76269653,414.62710297)
\curveto(320.5926933,414.57709652)(320.45269344,414.50709659)(320.34269653,414.41710297)
\curveto(320.24269365,414.33709676)(320.17269372,414.21209689)(320.13269653,414.04210297)
\curveto(320.11269378,413.97209713)(320.11269378,413.90709719)(320.13269653,413.84710297)
\curveto(320.15269374,413.78709731)(320.17269372,413.73709736)(320.19269653,413.69710297)
\curveto(320.26269363,413.57709752)(320.34269355,413.48209762)(320.43269653,413.41210297)
\curveto(320.53269336,413.34209776)(320.64769324,413.28209782)(320.77769653,413.23210297)
\curveto(320.96769292,413.15209795)(321.17269272,413.08209802)(321.39269653,413.02210297)
\lineto(322.08269653,412.87210297)
\curveto(322.32269157,412.83209827)(322.55269134,412.78209832)(322.77269653,412.72210297)
\curveto(323.00269089,412.67209843)(323.21769067,412.60709849)(323.41769653,412.52710297)
\curveto(323.50769038,412.48709861)(323.5926903,412.45209865)(323.67269653,412.42210297)
\curveto(323.76269013,412.4020987)(323.84769004,412.36709873)(323.92769653,412.31710297)
\curveto(324.11768977,412.1970989)(324.2876896,412.06709903)(324.43769653,411.92710297)
\curveto(324.59768929,411.78709931)(324.72268917,411.61209949)(324.81269653,411.40210297)
\curveto(324.84268905,411.33209977)(324.86768902,411.26209984)(324.88769653,411.19210297)
\curveto(324.90768898,411.12209998)(324.92768896,411.04710005)(324.94769653,410.96710297)
\curveto(324.95768893,410.90710019)(324.96268893,410.81210029)(324.96269653,410.68210297)
\curveto(324.97268892,410.56210054)(324.97268892,410.46710063)(324.96269653,410.39710297)
\lineto(324.96269653,410.32210297)
\curveto(324.94268895,410.26210084)(324.92768896,410.2021009)(324.91769653,410.14210297)
\curveto(324.91768897,410.09210101)(324.91268898,410.04210106)(324.90269653,409.99210297)
\curveto(324.83268906,409.69210141)(324.72268917,409.42710167)(324.57269653,409.19710297)
\curveto(324.41268948,408.95710214)(324.21768967,408.76210234)(323.98769653,408.61210297)
\curveto(323.75769013,408.46210264)(323.49769039,408.33210277)(323.20769653,408.22210297)
\curveto(323.09769079,408.17210293)(322.97769091,408.13710296)(322.84769653,408.11710297)
\curveto(322.72769116,408.097103)(322.60769128,408.07210303)(322.48769653,408.04210297)
\curveto(322.39769149,408.02210308)(322.30269159,408.01210309)(322.20269653,408.01210297)
\curveto(322.11269178,408.0021031)(322.02269187,407.98710311)(321.93269653,407.96710297)
\lineto(321.66269653,407.96710297)
\curveto(321.60269229,407.94710315)(321.49769239,407.93710316)(321.34769653,407.93710297)
\curveto(321.20769268,407.93710316)(321.10769278,407.94710315)(321.04769653,407.96710297)
\curveto(321.01769287,407.96710313)(320.98269291,407.97210313)(320.94269653,407.98210297)
\lineto(320.83769653,407.98210297)
\curveto(320.71769317,408.0021031)(320.59769329,408.01710308)(320.47769653,408.02710297)
\curveto(320.35769353,408.03710306)(320.24269365,408.05710304)(320.13269653,408.08710297)
\curveto(319.74269415,408.1971029)(319.39769449,408.32210278)(319.09769653,408.46210297)
\curveto(318.79769509,408.61210249)(318.54269535,408.83210227)(318.33269653,409.12210297)
\curveto(318.1926957,409.31210179)(318.07269582,409.53210157)(317.97269653,409.78210297)
\curveto(317.95269594,409.84210126)(317.93269596,409.92210118)(317.91269653,410.02210297)
\curveto(317.892696,410.07210103)(317.87769601,410.14210096)(317.86769653,410.23210297)
\curveto(317.85769603,410.32210078)(317.86269603,410.3971007)(317.88269653,410.45710297)
\curveto(317.91269598,410.52710057)(317.96269593,410.57710052)(318.03269653,410.60710297)
\curveto(318.08269581,410.62710047)(318.14269575,410.63710046)(318.21269653,410.63710297)
\lineto(318.43769653,410.63710297)
\lineto(319.14269653,410.63710297)
\lineto(319.38269653,410.63710297)
\curveto(319.46269443,410.63710046)(319.53269436,410.62710047)(319.59269653,410.60710297)
\curveto(319.70269419,410.56710053)(319.77269412,410.5021006)(319.80269653,410.41210297)
\curveto(319.84269405,410.32210078)(319.887694,410.22710087)(319.93769653,410.12710297)
\curveto(319.95769393,410.07710102)(319.9926939,410.01210109)(320.04269653,409.93210297)
\curveto(320.10269379,409.85210125)(320.15269374,409.8021013)(320.19269653,409.78210297)
\curveto(320.31269358,409.68210142)(320.42769346,409.6021015)(320.53769653,409.54210297)
\curveto(320.64769324,409.49210161)(320.7876931,409.44210166)(320.95769653,409.39210297)
\curveto(321.00769288,409.37210173)(321.05769283,409.36210174)(321.10769653,409.36210297)
\curveto(321.15769273,409.37210173)(321.20769268,409.37210173)(321.25769653,409.36210297)
\curveto(321.33769255,409.34210176)(321.42269247,409.33210177)(321.51269653,409.33210297)
\curveto(321.61269228,409.34210176)(321.69769219,409.35710174)(321.76769653,409.37710297)
\curveto(321.81769207,409.38710171)(321.86269203,409.39210171)(321.90269653,409.39210297)
\curveto(321.95269194,409.39210171)(322.00269189,409.4021017)(322.05269653,409.42210297)
\curveto(322.1926917,409.47210163)(322.31769157,409.53210157)(322.42769653,409.60210297)
\curveto(322.54769134,409.67210143)(322.64269125,409.76210134)(322.71269653,409.87210297)
\curveto(322.76269113,409.95210115)(322.80269109,410.07710102)(322.83269653,410.24710297)
\curveto(322.85269104,410.31710078)(322.85269104,410.38210072)(322.83269653,410.44210297)
\curveto(322.81269108,410.5021006)(322.7926911,410.55210055)(322.77269653,410.59210297)
\curveto(322.70269119,410.73210037)(322.61269128,410.83710026)(322.50269653,410.90710297)
\curveto(322.40269149,410.97710012)(322.28269161,411.04210006)(322.14269653,411.10210297)
\curveto(321.95269194,411.18209992)(321.75269214,411.24709985)(321.54269653,411.29710297)
\curveto(321.33269256,411.34709975)(321.12269277,411.4020997)(320.91269653,411.46210297)
\curveto(320.83269306,411.48209962)(320.74769314,411.4970996)(320.65769653,411.50710297)
\curveto(320.57769331,411.51709958)(320.49769339,411.53209957)(320.41769653,411.55210297)
\curveto(320.09769379,411.64209946)(319.7926941,411.72709937)(319.50269653,411.80710297)
\curveto(319.21269468,411.8970992)(318.94769494,412.02709907)(318.70769653,412.19710297)
\curveto(318.42769546,412.3970987)(318.22269567,412.66709843)(318.09269653,413.00710297)
\curveto(318.07269582,413.07709802)(318.05269584,413.17209793)(318.03269653,413.29210297)
\curveto(318.01269588,413.36209774)(317.99769589,413.44709765)(317.98769653,413.54710297)
\curveto(317.97769591,413.64709745)(317.98269591,413.73709736)(318.00269653,413.81710297)
\curveto(318.02269587,413.86709723)(318.02769586,413.90709719)(318.01769653,413.93710297)
\curveto(318.00769588,413.97709712)(318.01269588,414.02209708)(318.03269653,414.07210297)
\curveto(318.05269584,414.18209692)(318.07269582,414.28209682)(318.09269653,414.37210297)
\curveto(318.12269577,414.47209663)(318.15769573,414.56709653)(318.19769653,414.65710297)
\curveto(318.32769556,414.94709615)(318.50769538,415.18209592)(318.73769653,415.36210297)
\curveto(318.96769492,415.54209556)(319.22769466,415.68709541)(319.51769653,415.79710297)
\curveto(319.62769426,415.84709525)(319.74269415,415.88209522)(319.86269653,415.90210297)
\curveto(319.98269391,415.93209517)(320.10769378,415.96209514)(320.23769653,415.99210297)
\curveto(320.29769359,416.01209509)(320.35769353,416.02209508)(320.41769653,416.02210297)
\lineto(320.59769653,416.05210297)
\curveto(320.67769321,416.06209504)(320.76269313,416.06709503)(320.85269653,416.06710297)
\curveto(320.94269295,416.06709503)(321.02769286,416.07209503)(321.10769653,416.08210297)
}
}
{
\newrgbcolor{curcolor}{0 0 0}
\pscustom[linestyle=none,fillstyle=solid,fillcolor=curcolor]
{
}
}
{
\newrgbcolor{curcolor}{0 0 0}
\pscustom[linestyle=none,fillstyle=solid,fillcolor=curcolor]
{
\newpath
\moveto(338.24449341,412.09210297)
\curveto(338.25448473,412.03209907)(338.25948472,411.94209916)(338.25949341,411.82210297)
\curveto(338.25948472,411.7020994)(338.24948473,411.61709948)(338.22949341,411.56710297)
\lineto(338.22949341,411.37210297)
\curveto(338.19948478,411.26209984)(338.1794848,411.15709994)(338.16949341,411.05710297)
\curveto(338.16948481,410.95710014)(338.15448483,410.85710024)(338.12449341,410.75710297)
\curveto(338.10448488,410.66710043)(338.0844849,410.57210053)(338.06449341,410.47210297)
\curveto(338.04448494,410.38210072)(338.01448497,410.29210081)(337.97449341,410.20210297)
\curveto(337.90448508,410.03210107)(337.83448515,409.87210123)(337.76449341,409.72210297)
\curveto(337.69448529,409.58210152)(337.61448537,409.44210166)(337.52449341,409.30210297)
\curveto(337.46448552,409.21210189)(337.39948558,409.12710197)(337.32949341,409.04710297)
\curveto(337.26948571,408.97710212)(337.19948578,408.9021022)(337.11949341,408.82210297)
\lineto(337.01449341,408.71710297)
\curveto(336.96448602,408.66710243)(336.90948607,408.62210248)(336.84949341,408.58210297)
\lineto(336.69949341,408.46210297)
\curveto(336.61948636,408.4021027)(336.52948645,408.34710275)(336.42949341,408.29710297)
\curveto(336.33948664,408.25710284)(336.24448674,408.21210289)(336.14449341,408.16210297)
\curveto(336.04448694,408.11210299)(335.93948704,408.07710302)(335.82949341,408.05710297)
\curveto(335.72948725,408.03710306)(335.62448736,408.01710308)(335.51449341,407.99710297)
\curveto(335.45448753,407.97710312)(335.38948759,407.96710313)(335.31949341,407.96710297)
\curveto(335.25948772,407.96710313)(335.19448779,407.95710314)(335.12449341,407.93710297)
\lineto(334.98949341,407.93710297)
\curveto(334.90948807,407.91710318)(334.83448815,407.91710318)(334.76449341,407.93710297)
\lineto(334.61449341,407.93710297)
\curveto(334.55448843,407.95710314)(334.48948849,407.96710313)(334.41949341,407.96710297)
\curveto(334.35948862,407.95710314)(334.29948868,407.96210314)(334.23949341,407.98210297)
\curveto(334.0794889,408.03210307)(333.92448906,408.07710302)(333.77449341,408.11710297)
\curveto(333.63448935,408.15710294)(333.50448948,408.21710288)(333.38449341,408.29710297)
\curveto(333.31448967,408.33710276)(333.24948973,408.37710272)(333.18949341,408.41710297)
\curveto(333.12948985,408.46710263)(333.06448992,408.51710258)(332.99449341,408.56710297)
\lineto(332.81449341,408.70210297)
\curveto(332.73449025,408.76210234)(332.66449032,408.76710233)(332.60449341,408.71710297)
\curveto(332.55449043,408.68710241)(332.52949045,408.64710245)(332.52949341,408.59710297)
\curveto(332.52949045,408.55710254)(332.51949046,408.50710259)(332.49949341,408.44710297)
\curveto(332.4794905,408.34710275)(332.46949051,408.23210287)(332.46949341,408.10210297)
\curveto(332.4794905,407.97210313)(332.4844905,407.85210325)(332.48449341,407.74210297)
\lineto(332.48449341,406.21210297)
\curveto(332.4844905,406.08210502)(332.4794905,405.95710514)(332.46949341,405.83710297)
\curveto(332.46949051,405.70710539)(332.44449054,405.6021055)(332.39449341,405.52210297)
\curveto(332.36449062,405.48210562)(332.30949067,405.45210565)(332.22949341,405.43210297)
\curveto(332.14949083,405.41210569)(332.05949092,405.4021057)(331.95949341,405.40210297)
\curveto(331.85949112,405.39210571)(331.75949122,405.39210571)(331.65949341,405.40210297)
\lineto(331.40449341,405.40210297)
\lineto(330.99949341,405.40210297)
\lineto(330.89449341,405.40210297)
\curveto(330.85449213,405.4021057)(330.81949216,405.40710569)(330.78949341,405.41710297)
\lineto(330.66949341,405.41710297)
\curveto(330.49949248,405.46710563)(330.40949257,405.56710553)(330.39949341,405.71710297)
\curveto(330.38949259,405.85710524)(330.3844926,406.02710507)(330.38449341,406.22710297)
\lineto(330.38449341,415.03210297)
\curveto(330.3844926,415.14209596)(330.3794926,415.25709584)(330.36949341,415.37710297)
\curveto(330.36949261,415.50709559)(330.39449259,415.60709549)(330.44449341,415.67710297)
\curveto(330.4844925,415.74709535)(330.53949244,415.79209531)(330.60949341,415.81210297)
\curveto(330.65949232,415.83209527)(330.71949226,415.84209526)(330.78949341,415.84210297)
\lineto(331.01449341,415.84210297)
\lineto(331.73449341,415.84210297)
\lineto(332.01949341,415.84210297)
\curveto(332.10949087,415.84209526)(332.1844908,415.81709528)(332.24449341,415.76710297)
\curveto(332.31449067,415.71709538)(332.34949063,415.65209545)(332.34949341,415.57210297)
\curveto(332.35949062,415.5020956)(332.3844906,415.42709567)(332.42449341,415.34710297)
\curveto(332.43449055,415.31709578)(332.44449054,415.29209581)(332.45449341,415.27210297)
\curveto(332.47449051,415.26209584)(332.49449049,415.24709585)(332.51449341,415.22710297)
\curveto(332.62449036,415.21709588)(332.71449027,415.24709585)(332.78449341,415.31710297)
\curveto(332.85449013,415.38709571)(332.92449006,415.44709565)(332.99449341,415.49710297)
\curveto(333.12448986,415.58709551)(333.25948972,415.66709543)(333.39949341,415.73710297)
\curveto(333.53948944,415.81709528)(333.69448929,415.88209522)(333.86449341,415.93210297)
\curveto(333.94448904,415.96209514)(334.02948895,415.98209512)(334.11949341,415.99210297)
\curveto(334.21948876,416.0020951)(334.31448867,416.01709508)(334.40449341,416.03710297)
\curveto(334.44448854,416.04709505)(334.4844885,416.04709505)(334.52449341,416.03710297)
\curveto(334.57448841,416.02709507)(334.61448837,416.03209507)(334.64449341,416.05210297)
\curveto(335.21448777,416.07209503)(335.69448729,415.99209511)(336.08449341,415.81210297)
\curveto(336.4844865,415.64209546)(336.82448616,415.41709568)(337.10449341,415.13710297)
\curveto(337.15448583,415.08709601)(337.19948578,415.03709606)(337.23949341,414.98710297)
\curveto(337.2794857,414.94709615)(337.31948566,414.9020962)(337.35949341,414.85210297)
\curveto(337.42948555,414.76209634)(337.48948549,414.67209643)(337.53949341,414.58210297)
\curveto(337.59948538,414.49209661)(337.65448533,414.4020967)(337.70449341,414.31210297)
\curveto(337.72448526,414.29209681)(337.73448525,414.26709683)(337.73449341,414.23710297)
\curveto(337.74448524,414.20709689)(337.75948522,414.17209693)(337.77949341,414.13210297)
\curveto(337.83948514,414.03209707)(337.89448509,413.91209719)(337.94449341,413.77210297)
\curveto(337.96448502,413.71209739)(337.984485,413.64709745)(338.00449341,413.57710297)
\curveto(338.02448496,413.51709758)(338.04448494,413.45209765)(338.06449341,413.38210297)
\curveto(338.10448488,413.26209784)(338.12948485,413.13709796)(338.13949341,413.00710297)
\curveto(338.15948482,412.87709822)(338.1844848,412.74209836)(338.21449341,412.60210297)
\lineto(338.21449341,412.43710297)
\lineto(338.24449341,412.25710297)
\lineto(338.24449341,412.09210297)
\moveto(336.12949341,411.74710297)
\curveto(336.13948684,411.7970993)(336.14448684,411.86209924)(336.14449341,411.94210297)
\curveto(336.14448684,412.03209907)(336.13948684,412.102099)(336.12949341,412.15210297)
\lineto(336.12949341,412.28710297)
\curveto(336.10948687,412.34709875)(336.09948688,412.41209869)(336.09949341,412.48210297)
\curveto(336.09948688,412.55209855)(336.08948689,412.62209848)(336.06949341,412.69210297)
\curveto(336.04948693,412.79209831)(336.02948695,412.88709821)(336.00949341,412.97710297)
\curveto(335.98948699,413.07709802)(335.95948702,413.16709793)(335.91949341,413.24710297)
\curveto(335.79948718,413.56709753)(335.64448734,413.82209728)(335.45449341,414.01210297)
\curveto(335.26448772,414.2020969)(334.99448799,414.34209676)(334.64449341,414.43210297)
\curveto(334.56448842,414.45209665)(334.47448851,414.46209664)(334.37449341,414.46210297)
\lineto(334.10449341,414.46210297)
\curveto(334.06448892,414.45209665)(334.02948895,414.44709665)(333.99949341,414.44710297)
\curveto(333.96948901,414.44709665)(333.93448905,414.44209666)(333.89449341,414.43210297)
\lineto(333.68449341,414.37210297)
\curveto(333.62448936,414.36209674)(333.56448942,414.34209676)(333.50449341,414.31210297)
\curveto(333.24448974,414.2020969)(333.03948994,414.03209707)(332.88949341,413.80210297)
\curveto(332.74949023,413.57209753)(332.63449035,413.31709778)(332.54449341,413.03710297)
\curveto(332.52449046,412.95709814)(332.50949047,412.87209823)(332.49949341,412.78210297)
\curveto(332.48949049,412.7020984)(332.47449051,412.62209848)(332.45449341,412.54210297)
\curveto(332.44449054,412.5020986)(332.43949054,412.43709866)(332.43949341,412.34710297)
\curveto(332.41949056,412.30709879)(332.41449057,412.25709884)(332.42449341,412.19710297)
\curveto(332.43449055,412.14709895)(332.43449055,412.097099)(332.42449341,412.04710297)
\curveto(332.40449058,411.98709911)(332.40449058,411.93209917)(332.42449341,411.88210297)
\lineto(332.42449341,411.70210297)
\lineto(332.42449341,411.56710297)
\curveto(332.42449056,411.52709957)(332.43449055,411.48709961)(332.45449341,411.44710297)
\curveto(332.45449053,411.37709972)(332.45949052,411.32209978)(332.46949341,411.28210297)
\lineto(332.49949341,411.10210297)
\curveto(332.50949047,411.04210006)(332.52449046,410.98210012)(332.54449341,410.92210297)
\curveto(332.63449035,410.63210047)(332.73949024,410.39210071)(332.85949341,410.20210297)
\curveto(332.98948999,410.02210108)(333.16948981,409.86210124)(333.39949341,409.72210297)
\curveto(333.53948944,409.64210146)(333.70448928,409.57710152)(333.89449341,409.52710297)
\curveto(333.93448905,409.51710158)(333.96948901,409.51210159)(333.99949341,409.51210297)
\curveto(334.02948895,409.52210158)(334.06448892,409.52210158)(334.10449341,409.51210297)
\curveto(334.14448884,409.5021016)(334.20448878,409.49210161)(334.28449341,409.48210297)
\curveto(334.36448862,409.48210162)(334.42948855,409.48710161)(334.47949341,409.49710297)
\curveto(334.55948842,409.51710158)(334.63948834,409.53210157)(334.71949341,409.54210297)
\curveto(334.80948817,409.56210154)(334.89448809,409.58710151)(334.97449341,409.61710297)
\curveto(335.21448777,409.71710138)(335.40948757,409.85710124)(335.55949341,410.03710297)
\curveto(335.70948727,410.21710088)(335.83448715,410.42710067)(335.93449341,410.66710297)
\curveto(335.984487,410.78710031)(336.01948696,410.91210019)(336.03949341,411.04210297)
\curveto(336.05948692,411.17209993)(336.0844869,411.30709979)(336.11449341,411.44710297)
\lineto(336.11449341,411.59710297)
\curveto(336.12448686,411.64709945)(336.12948685,411.6970994)(336.12949341,411.74710297)
}
}
{
\newrgbcolor{curcolor}{0 0 0}
\pscustom[linestyle=none,fillstyle=solid,fillcolor=curcolor]
{
\newpath
\moveto(347.29441528,412.31710297)
\curveto(347.31440671,412.25709884)(347.3244067,412.17209893)(347.32441528,412.06210297)
\curveto(347.3244067,411.95209915)(347.31440671,411.86709923)(347.29441528,411.80710297)
\lineto(347.29441528,411.65710297)
\curveto(347.27440675,411.57709952)(347.26440676,411.4970996)(347.26441528,411.41710297)
\curveto(347.27440675,411.33709976)(347.26940676,411.25709984)(347.24941528,411.17710297)
\curveto(347.2294068,411.10709999)(347.21440681,411.04210006)(347.20441528,410.98210297)
\curveto(347.19440683,410.92210018)(347.18440684,410.85710024)(347.17441528,410.78710297)
\curveto(347.13440689,410.67710042)(347.09940693,410.56210054)(347.06941528,410.44210297)
\curveto(347.03940699,410.33210077)(346.99940703,410.22710087)(346.94941528,410.12710297)
\curveto(346.73940729,409.64710145)(346.46440756,409.25710184)(346.12441528,408.95710297)
\curveto(345.78440824,408.65710244)(345.37440865,408.40710269)(344.89441528,408.20710297)
\curveto(344.77440925,408.15710294)(344.64940938,408.12210298)(344.51941528,408.10210297)
\curveto(344.39940963,408.07210303)(344.27440975,408.04210306)(344.14441528,408.01210297)
\curveto(344.09440993,407.99210311)(344.03940999,407.98210312)(343.97941528,407.98210297)
\curveto(343.91941011,407.98210312)(343.86441016,407.97710312)(343.81441528,407.96710297)
\lineto(343.70941528,407.96710297)
\curveto(343.67941035,407.95710314)(343.64941038,407.95210315)(343.61941528,407.95210297)
\curveto(343.56941046,407.94210316)(343.48941054,407.93710316)(343.37941528,407.93710297)
\curveto(343.26941076,407.92710317)(343.18441084,407.93210317)(343.12441528,407.95210297)
\lineto(342.97441528,407.95210297)
\curveto(342.9244111,407.96210314)(342.86941116,407.96710313)(342.80941528,407.96710297)
\curveto(342.75941127,407.95710314)(342.70941132,407.96210314)(342.65941528,407.98210297)
\curveto(342.61941141,407.99210311)(342.57941145,407.9971031)(342.53941528,407.99710297)
\curveto(342.50941152,407.9971031)(342.46941156,408.0021031)(342.41941528,408.01210297)
\curveto(342.31941171,408.04210306)(342.21941181,408.06710303)(342.11941528,408.08710297)
\curveto(342.01941201,408.10710299)(341.9244121,408.13710296)(341.83441528,408.17710297)
\curveto(341.71441231,408.21710288)(341.59941243,408.25710284)(341.48941528,408.29710297)
\curveto(341.38941264,408.33710276)(341.28441274,408.38710271)(341.17441528,408.44710297)
\curveto(340.8244132,408.65710244)(340.5244135,408.9021022)(340.27441528,409.18210297)
\curveto(340.024414,409.46210164)(339.81441421,409.7971013)(339.64441528,410.18710297)
\curveto(339.59441443,410.27710082)(339.55441447,410.37210073)(339.52441528,410.47210297)
\curveto(339.50441452,410.57210053)(339.47941455,410.67710042)(339.44941528,410.78710297)
\curveto(339.4294146,410.83710026)(339.41941461,410.88210022)(339.41941528,410.92210297)
\curveto(339.41941461,410.96210014)(339.40941462,411.00710009)(339.38941528,411.05710297)
\curveto(339.36941466,411.13709996)(339.35941467,411.21709988)(339.35941528,411.29710297)
\curveto(339.35941467,411.38709971)(339.34941468,411.47209963)(339.32941528,411.55210297)
\curveto(339.31941471,411.6020995)(339.31441471,411.64709945)(339.31441528,411.68710297)
\lineto(339.31441528,411.82210297)
\curveto(339.29441473,411.88209922)(339.28441474,411.96709913)(339.28441528,412.07710297)
\curveto(339.29441473,412.18709891)(339.30941472,412.27209883)(339.32941528,412.33210297)
\lineto(339.32941528,412.43710297)
\curveto(339.33941469,412.48709861)(339.33941469,412.53709856)(339.32941528,412.58710297)
\curveto(339.3294147,412.64709845)(339.33941469,412.7020984)(339.35941528,412.75210297)
\curveto(339.36941466,412.8020983)(339.37441465,412.84709825)(339.37441528,412.88710297)
\curveto(339.37441465,412.93709816)(339.38441464,412.98709811)(339.40441528,413.03710297)
\curveto(339.44441458,413.16709793)(339.47941455,413.29209781)(339.50941528,413.41210297)
\curveto(339.53941449,413.54209756)(339.57941445,413.66709743)(339.62941528,413.78710297)
\curveto(339.80941422,414.1970969)(340.024414,414.53709656)(340.27441528,414.80710297)
\curveto(340.5244135,415.08709601)(340.8294132,415.34209576)(341.18941528,415.57210297)
\curveto(341.28941274,415.62209548)(341.39441263,415.66709543)(341.50441528,415.70710297)
\curveto(341.61441241,415.74709535)(341.7244123,415.79209531)(341.83441528,415.84210297)
\curveto(341.96441206,415.89209521)(342.09941193,415.92709517)(342.23941528,415.94710297)
\curveto(342.37941165,415.96709513)(342.5244115,415.9970951)(342.67441528,416.03710297)
\curveto(342.75441127,416.04709505)(342.8294112,416.05209505)(342.89941528,416.05210297)
\curveto(342.96941106,416.05209505)(343.03941099,416.05709504)(343.10941528,416.06710297)
\curveto(343.68941034,416.07709502)(344.18940984,416.01709508)(344.60941528,415.88710297)
\curveto(345.03940899,415.75709534)(345.41940861,415.57709552)(345.74941528,415.34710297)
\curveto(345.85940817,415.26709583)(345.96940806,415.17709592)(346.07941528,415.07710297)
\curveto(346.19940783,414.98709611)(346.29940773,414.88709621)(346.37941528,414.77710297)
\curveto(346.45940757,414.67709642)(346.5294075,414.57709652)(346.58941528,414.47710297)
\curveto(346.65940737,414.37709672)(346.7294073,414.27209683)(346.79941528,414.16210297)
\curveto(346.86940716,414.05209705)(346.9244071,413.93209717)(346.96441528,413.80210297)
\curveto(347.00440702,413.68209742)(347.04940698,413.55209755)(347.09941528,413.41210297)
\curveto(347.1294069,413.33209777)(347.15440687,413.24709785)(347.17441528,413.15710297)
\lineto(347.23441528,412.88710297)
\curveto(347.24440678,412.84709825)(347.24940678,412.80709829)(347.24941528,412.76710297)
\curveto(347.24940678,412.72709837)(347.25440677,412.68709841)(347.26441528,412.64710297)
\curveto(347.28440674,412.5970985)(347.28940674,412.54209856)(347.27941528,412.48210297)
\curveto(347.26940676,412.42209868)(347.27440675,412.36709873)(347.29441528,412.31710297)
\moveto(345.19441528,411.77710297)
\curveto(345.20440882,411.82709927)(345.20940882,411.8970992)(345.20941528,411.98710297)
\curveto(345.20940882,412.08709901)(345.20440882,412.16209894)(345.19441528,412.21210297)
\lineto(345.19441528,412.33210297)
\curveto(345.17440885,412.38209872)(345.16440886,412.43709866)(345.16441528,412.49710297)
\curveto(345.16440886,412.55709854)(345.15940887,412.61209849)(345.14941528,412.66210297)
\curveto(345.14940888,412.7020984)(345.14440888,412.73209837)(345.13441528,412.75210297)
\lineto(345.07441528,412.99210297)
\curveto(345.06440896,413.08209802)(345.04440898,413.16709793)(345.01441528,413.24710297)
\curveto(344.90440912,413.50709759)(344.77440925,413.72709737)(344.62441528,413.90710297)
\curveto(344.47440955,414.097097)(344.27440975,414.24709685)(344.02441528,414.35710297)
\curveto(343.96441006,414.37709672)(343.90441012,414.39209671)(343.84441528,414.40210297)
\curveto(343.78441024,414.42209668)(343.71941031,414.44209666)(343.64941528,414.46210297)
\curveto(343.56941046,414.48209662)(343.48441054,414.48709661)(343.39441528,414.47710297)
\lineto(343.12441528,414.47710297)
\curveto(343.09441093,414.45709664)(343.05941097,414.44709665)(343.01941528,414.44710297)
\curveto(342.97941105,414.45709664)(342.94441108,414.45709664)(342.91441528,414.44710297)
\lineto(342.70441528,414.38710297)
\curveto(342.64441138,414.37709672)(342.58941144,414.35709674)(342.53941528,414.32710297)
\curveto(342.28941174,414.21709688)(342.08441194,414.05709704)(341.92441528,413.84710297)
\curveto(341.77441225,413.64709745)(341.65441237,413.41209769)(341.56441528,413.14210297)
\curveto(341.53441249,413.04209806)(341.50941252,412.93709816)(341.48941528,412.82710297)
\curveto(341.47941255,412.71709838)(341.46441256,412.60709849)(341.44441528,412.49710297)
\curveto(341.43441259,412.44709865)(341.4294126,412.3970987)(341.42941528,412.34710297)
\lineto(341.42941528,412.19710297)
\curveto(341.40941262,412.12709897)(341.39941263,412.02209908)(341.39941528,411.88210297)
\curveto(341.40941262,411.74209936)(341.4244126,411.63709946)(341.44441528,411.56710297)
\lineto(341.44441528,411.43210297)
\curveto(341.46441256,411.35209975)(341.47941255,411.27209983)(341.48941528,411.19210297)
\curveto(341.49941253,411.12209998)(341.51441251,411.04710005)(341.53441528,410.96710297)
\curveto(341.63441239,410.66710043)(341.73941229,410.42210068)(341.84941528,410.23210297)
\curveto(341.96941206,410.05210105)(342.15441187,409.88710121)(342.40441528,409.73710297)
\curveto(342.47441155,409.68710141)(342.54941148,409.64710145)(342.62941528,409.61710297)
\curveto(342.71941131,409.58710151)(342.80941122,409.56210154)(342.89941528,409.54210297)
\curveto(342.93941109,409.53210157)(342.97441105,409.52710157)(343.00441528,409.52710297)
\curveto(343.03441099,409.53710156)(343.06941096,409.53710156)(343.10941528,409.52710297)
\lineto(343.22941528,409.49710297)
\curveto(343.27941075,409.4971016)(343.3244107,409.5021016)(343.36441528,409.51210297)
\lineto(343.48441528,409.51210297)
\curveto(343.56441046,409.53210157)(343.64441038,409.54710155)(343.72441528,409.55710297)
\curveto(343.80441022,409.56710153)(343.87941015,409.58710151)(343.94941528,409.61710297)
\curveto(344.20940982,409.71710138)(344.41940961,409.85210125)(344.57941528,410.02210297)
\curveto(344.73940929,410.19210091)(344.87440915,410.4021007)(344.98441528,410.65210297)
\curveto(345.024409,410.75210035)(345.05440897,410.85210025)(345.07441528,410.95210297)
\curveto(345.09440893,411.05210005)(345.11940891,411.15709994)(345.14941528,411.26710297)
\curveto(345.15940887,411.30709979)(345.16440886,411.34209976)(345.16441528,411.37210297)
\curveto(345.16440886,411.41209969)(345.16940886,411.45209965)(345.17941528,411.49210297)
\lineto(345.17941528,411.62710297)
\curveto(345.17940885,411.67709942)(345.18440884,411.72709937)(345.19441528,411.77710297)
}
}
{
\newrgbcolor{curcolor}{0 0 0}
\pscustom[linestyle=none,fillstyle=solid,fillcolor=curcolor]
{
\newpath
\moveto(353.11933716,416.06710297)
\curveto(353.22933184,416.06709503)(353.32433175,416.05709504)(353.40433716,416.03710297)
\curveto(353.49433158,416.01709508)(353.56433151,415.97209513)(353.61433716,415.90210297)
\curveto(353.6743314,415.82209528)(353.70433137,415.68209542)(353.70433716,415.48210297)
\lineto(353.70433716,414.97210297)
\lineto(353.70433716,414.59710297)
\curveto(353.71433136,414.45709664)(353.69933137,414.34709675)(353.65933716,414.26710297)
\curveto(353.61933145,414.1970969)(353.55933151,414.15209695)(353.47933716,414.13210297)
\curveto(353.40933166,414.11209699)(353.32433175,414.102097)(353.22433716,414.10210297)
\curveto(353.13433194,414.102097)(353.03433204,414.10709699)(352.92433716,414.11710297)
\curveto(352.82433225,414.12709697)(352.72933234,414.12209698)(352.63933716,414.10210297)
\curveto(352.5693325,414.08209702)(352.49933257,414.06709703)(352.42933716,414.05710297)
\curveto(352.35933271,414.05709704)(352.29433278,414.04709705)(352.23433716,414.02710297)
\curveto(352.074333,413.97709712)(351.91433316,413.9020972)(351.75433716,413.80210297)
\curveto(351.59433348,413.71209739)(351.4693336,413.60709749)(351.37933716,413.48710297)
\curveto(351.32933374,413.40709769)(351.2743338,413.32209778)(351.21433716,413.23210297)
\curveto(351.16433391,413.15209795)(351.11433396,413.06709803)(351.06433716,412.97710297)
\curveto(351.03433404,412.8970982)(351.00433407,412.81209829)(350.97433716,412.72210297)
\lineto(350.91433716,412.48210297)
\curveto(350.89433418,412.41209869)(350.88433419,412.33709876)(350.88433716,412.25710297)
\curveto(350.88433419,412.18709891)(350.8743342,412.11709898)(350.85433716,412.04710297)
\curveto(350.84433423,412.00709909)(350.83933423,411.96709913)(350.83933716,411.92710297)
\curveto(350.84933422,411.8970992)(350.84933422,411.86709923)(350.83933716,411.83710297)
\lineto(350.83933716,411.59710297)
\curveto(350.81933425,411.52709957)(350.81433426,411.44709965)(350.82433716,411.35710297)
\curveto(350.83433424,411.27709982)(350.83933423,411.1970999)(350.83933716,411.11710297)
\lineto(350.83933716,410.15710297)
\lineto(350.83933716,408.88210297)
\curveto(350.83933423,408.75210235)(350.83433424,408.63210247)(350.82433716,408.52210297)
\curveto(350.81433426,408.41210269)(350.78433429,408.32210278)(350.73433716,408.25210297)
\curveto(350.71433436,408.22210288)(350.67933439,408.1971029)(350.62933716,408.17710297)
\curveto(350.58933448,408.16710293)(350.54433453,408.15710294)(350.49433716,408.14710297)
\lineto(350.41933716,408.14710297)
\curveto(350.3693347,408.13710296)(350.31433476,408.13210297)(350.25433716,408.13210297)
\lineto(350.08933716,408.13210297)
\lineto(349.44433716,408.13210297)
\curveto(349.38433569,408.14210296)(349.31933575,408.14710295)(349.24933716,408.14710297)
\lineto(349.05433716,408.14710297)
\curveto(349.00433607,408.16710293)(348.95433612,408.18210292)(348.90433716,408.19210297)
\curveto(348.85433622,408.21210289)(348.81933625,408.24710285)(348.79933716,408.29710297)
\curveto(348.75933631,408.34710275)(348.73433634,408.41710268)(348.72433716,408.50710297)
\lineto(348.72433716,408.80710297)
\lineto(348.72433716,409.82710297)
\lineto(348.72433716,414.05710297)
\lineto(348.72433716,415.16710297)
\lineto(348.72433716,415.45210297)
\curveto(348.72433635,415.55209555)(348.74433633,415.63209547)(348.78433716,415.69210297)
\curveto(348.83433624,415.77209533)(348.90933616,415.82209528)(349.00933716,415.84210297)
\curveto(349.10933596,415.86209524)(349.22933584,415.87209523)(349.36933716,415.87210297)
\lineto(350.13433716,415.87210297)
\curveto(350.25433482,415.87209523)(350.35933471,415.86209524)(350.44933716,415.84210297)
\curveto(350.53933453,415.83209527)(350.60933446,415.78709531)(350.65933716,415.70710297)
\curveto(350.68933438,415.65709544)(350.70433437,415.58709551)(350.70433716,415.49710297)
\lineto(350.73433716,415.22710297)
\curveto(350.74433433,415.14709595)(350.75933431,415.07209603)(350.77933716,415.00210297)
\curveto(350.80933426,414.93209617)(350.85933421,414.8970962)(350.92933716,414.89710297)
\curveto(350.94933412,414.91709618)(350.9693341,414.92709617)(350.98933716,414.92710297)
\curveto(351.00933406,414.92709617)(351.02933404,414.93709616)(351.04933716,414.95710297)
\curveto(351.10933396,415.00709609)(351.15933391,415.06209604)(351.19933716,415.12210297)
\curveto(351.24933382,415.19209591)(351.30933376,415.25209585)(351.37933716,415.30210297)
\curveto(351.41933365,415.33209577)(351.45433362,415.36209574)(351.48433716,415.39210297)
\curveto(351.51433356,415.43209567)(351.54933352,415.46709563)(351.58933716,415.49710297)
\lineto(351.85933716,415.67710297)
\curveto(351.95933311,415.73709536)(352.05933301,415.79209531)(352.15933716,415.84210297)
\curveto(352.25933281,415.88209522)(352.35933271,415.91709518)(352.45933716,415.94710297)
\lineto(352.78933716,416.03710297)
\curveto(352.81933225,416.04709505)(352.8743322,416.04709505)(352.95433716,416.03710297)
\curveto(353.04433203,416.03709506)(353.09933197,416.04709505)(353.11933716,416.06710297)
}
}
{
\newrgbcolor{curcolor}{0 0 0}
\pscustom[linestyle=none,fillstyle=solid,fillcolor=curcolor]
{
}
}
{
\newrgbcolor{curcolor}{0 0 0}
\pscustom[linestyle=none,fillstyle=solid,fillcolor=curcolor]
{
\newpath
\moveto(359.73457153,418.18210297)
\lineto(360.73957153,418.18210297)
\curveto(360.88956855,418.18209292)(361.01956842,418.17209293)(361.12957153,418.15210297)
\curveto(361.24956819,418.14209296)(361.3345681,418.08209302)(361.38457153,417.97210297)
\curveto(361.40456803,417.92209318)(361.41456802,417.86209324)(361.41457153,417.79210297)
\lineto(361.41457153,417.58210297)
\lineto(361.41457153,416.90710297)
\curveto(361.41456802,416.85709424)(361.40956803,416.7970943)(361.39957153,416.72710297)
\curveto(361.39956804,416.66709443)(361.40456803,416.61209449)(361.41457153,416.56210297)
\lineto(361.41457153,416.39710297)
\curveto(361.41456802,416.31709478)(361.41956802,416.24209486)(361.42957153,416.17210297)
\curveto(361.439568,416.11209499)(361.46456797,416.05709504)(361.50457153,416.00710297)
\curveto(361.57456786,415.91709518)(361.69956774,415.86709523)(361.87957153,415.85710297)
\lineto(362.41957153,415.85710297)
\lineto(362.59957153,415.85710297)
\curveto(362.65956678,415.85709524)(362.71456672,415.84709525)(362.76457153,415.82710297)
\curveto(362.87456656,415.77709532)(362.9345665,415.68709541)(362.94457153,415.55710297)
\curveto(362.96456647,415.42709567)(362.97456646,415.28209582)(362.97457153,415.12210297)
\lineto(362.97457153,414.91210297)
\curveto(362.98456645,414.84209626)(362.97956646,414.78209632)(362.95957153,414.73210297)
\curveto(362.90956653,414.57209653)(362.80456663,414.48709661)(362.64457153,414.47710297)
\curveto(362.48456695,414.46709663)(362.30456713,414.46209664)(362.10457153,414.46210297)
\lineto(361.96957153,414.46210297)
\curveto(361.92956751,414.47209663)(361.89456754,414.47209663)(361.86457153,414.46210297)
\curveto(361.82456761,414.45209665)(361.78956765,414.44709665)(361.75957153,414.44710297)
\curveto(361.72956771,414.45709664)(361.69956774,414.45209665)(361.66957153,414.43210297)
\curveto(361.58956785,414.41209669)(361.52956791,414.36709673)(361.48957153,414.29710297)
\curveto(361.45956798,414.23709686)(361.434568,414.16209694)(361.41457153,414.07210297)
\curveto(361.40456803,414.02209708)(361.40456803,413.96709713)(361.41457153,413.90710297)
\curveto(361.42456801,413.84709725)(361.42456801,413.79209731)(361.41457153,413.74210297)
\lineto(361.41457153,412.81210297)
\lineto(361.41457153,411.05710297)
\curveto(361.41456802,410.80710029)(361.41956802,410.58710051)(361.42957153,410.39710297)
\curveto(361.44956799,410.21710088)(361.51456792,410.05710104)(361.62457153,409.91710297)
\curveto(361.67456776,409.85710124)(361.7395677,409.81210129)(361.81957153,409.78210297)
\lineto(362.08957153,409.72210297)
\curveto(362.11956732,409.71210139)(362.14956729,409.70710139)(362.17957153,409.70710297)
\curveto(362.21956722,409.71710138)(362.24956719,409.71710138)(362.26957153,409.70710297)
\lineto(362.43457153,409.70710297)
\curveto(362.54456689,409.70710139)(362.6395668,409.7021014)(362.71957153,409.69210297)
\curveto(362.79956664,409.68210142)(362.86456657,409.64210146)(362.91457153,409.57210297)
\curveto(362.95456648,409.51210159)(362.97456646,409.43210167)(362.97457153,409.33210297)
\lineto(362.97457153,409.04710297)
\curveto(362.97456646,408.83710226)(362.96956647,408.64210246)(362.95957153,408.46210297)
\curveto(362.95956648,408.29210281)(362.87956656,408.17710292)(362.71957153,408.11710297)
\curveto(362.66956677,408.097103)(362.62456681,408.09210301)(362.58457153,408.10210297)
\curveto(362.54456689,408.102103)(362.49956694,408.09210301)(362.44957153,408.07210297)
\lineto(362.29957153,408.07210297)
\curveto(362.27956716,408.07210303)(362.24956719,408.07710302)(362.20957153,408.08710297)
\curveto(362.16956727,408.08710301)(362.1345673,408.08210302)(362.10457153,408.07210297)
\curveto(362.05456738,408.06210304)(361.99956744,408.06210304)(361.93957153,408.07210297)
\lineto(361.78957153,408.07210297)
\lineto(361.63957153,408.07210297)
\curveto(361.58956785,408.06210304)(361.54456789,408.06210304)(361.50457153,408.07210297)
\lineto(361.33957153,408.07210297)
\curveto(361.28956815,408.08210302)(361.2345682,408.08710301)(361.17457153,408.08710297)
\curveto(361.11456832,408.08710301)(361.05956838,408.09210301)(361.00957153,408.10210297)
\curveto(360.9395685,408.11210299)(360.87456856,408.12210298)(360.81457153,408.13210297)
\lineto(360.63457153,408.16210297)
\curveto(360.52456891,408.19210291)(360.41956902,408.22710287)(360.31957153,408.26710297)
\curveto(360.21956922,408.30710279)(360.12456931,408.35210275)(360.03457153,408.40210297)
\lineto(359.94457153,408.46210297)
\curveto(359.91456952,408.49210261)(359.87956956,408.52210258)(359.83957153,408.55210297)
\curveto(359.81956962,408.57210253)(359.79456964,408.59210251)(359.76457153,408.61210297)
\lineto(359.68957153,408.68710297)
\curveto(359.54956989,408.87710222)(359.44456999,409.08710201)(359.37457153,409.31710297)
\curveto(359.35457008,409.35710174)(359.34457009,409.39210171)(359.34457153,409.42210297)
\curveto(359.35457008,409.46210164)(359.35457008,409.50710159)(359.34457153,409.55710297)
\curveto(359.3345701,409.57710152)(359.32957011,409.6021015)(359.32957153,409.63210297)
\curveto(359.32957011,409.66210144)(359.32457011,409.68710141)(359.31457153,409.70710297)
\lineto(359.31457153,409.85710297)
\curveto(359.30457013,409.8971012)(359.29957014,409.94210116)(359.29957153,409.99210297)
\curveto(359.30957013,410.04210106)(359.31457012,410.09210101)(359.31457153,410.14210297)
\lineto(359.31457153,410.71210297)
\lineto(359.31457153,412.94710297)
\lineto(359.31457153,413.74210297)
\lineto(359.31457153,413.95210297)
\curveto(359.32457011,414.02209708)(359.31957012,414.08709701)(359.29957153,414.14710297)
\curveto(359.25957018,414.28709681)(359.18957025,414.37709672)(359.08957153,414.41710297)
\curveto(358.97957046,414.46709663)(358.8395706,414.48209662)(358.66957153,414.46210297)
\curveto(358.49957094,414.44209666)(358.35457108,414.45709664)(358.23457153,414.50710297)
\curveto(358.15457128,414.53709656)(358.10457133,414.58209652)(358.08457153,414.64210297)
\curveto(358.06457137,414.7020964)(358.04457139,414.77709632)(358.02457153,414.86710297)
\lineto(358.02457153,415.18210297)
\curveto(358.02457141,415.36209574)(358.0345714,415.50709559)(358.05457153,415.61710297)
\curveto(358.07457136,415.72709537)(358.15957128,415.8020953)(358.30957153,415.84210297)
\curveto(358.34957109,415.86209524)(358.38957105,415.86709523)(358.42957153,415.85710297)
\lineto(358.56457153,415.85710297)
\curveto(358.71457072,415.85709524)(358.85457058,415.86209524)(358.98457153,415.87210297)
\curveto(359.11457032,415.89209521)(359.20457023,415.95209515)(359.25457153,416.05210297)
\curveto(359.28457015,416.12209498)(359.29957014,416.2020949)(359.29957153,416.29210297)
\curveto(359.30957013,416.38209472)(359.31457012,416.47209463)(359.31457153,416.56210297)
\lineto(359.31457153,417.49210297)
\lineto(359.31457153,417.74710297)
\curveto(359.31457012,417.83709326)(359.32457011,417.91209319)(359.34457153,417.97210297)
\curveto(359.39457004,418.07209303)(359.46956997,418.13709296)(359.56957153,418.16710297)
\curveto(359.58956985,418.17709292)(359.61456982,418.17709292)(359.64457153,418.16710297)
\curveto(359.68456975,418.16709293)(359.71456972,418.17209293)(359.73457153,418.18210297)
}
}
{
\newrgbcolor{curcolor}{0 0 0}
\pscustom[linestyle=none,fillstyle=solid,fillcolor=curcolor]
{
\newpath
\moveto(366.05800903,418.72210297)
\curveto(366.12800608,418.64209246)(366.16300605,418.52209258)(366.16300903,418.36210297)
\lineto(366.16300903,417.89710297)
\lineto(366.16300903,417.49210297)
\curveto(366.16300605,417.35209375)(366.12800608,417.25709384)(366.05800903,417.20710297)
\curveto(365.99800621,417.15709394)(365.91800629,417.12709397)(365.81800903,417.11710297)
\curveto(365.72800648,417.10709399)(365.62800658,417.102094)(365.51800903,417.10210297)
\lineto(364.67800903,417.10210297)
\curveto(364.56800764,417.102094)(364.46800774,417.10709399)(364.37800903,417.11710297)
\curveto(364.29800791,417.12709397)(364.22800798,417.15709394)(364.16800903,417.20710297)
\curveto(364.12800808,417.23709386)(364.09800811,417.29209381)(364.07800903,417.37210297)
\curveto(364.06800814,417.46209364)(364.05800815,417.55709354)(364.04800903,417.65710297)
\lineto(364.04800903,417.98710297)
\curveto(364.05800815,418.097093)(364.06300815,418.19209291)(364.06300903,418.27210297)
\lineto(364.06300903,418.48210297)
\curveto(364.07300814,418.55209255)(364.09300812,418.61209249)(364.12300903,418.66210297)
\curveto(364.14300807,418.7020924)(364.16800804,418.73209237)(364.19800903,418.75210297)
\lineto(364.31800903,418.81210297)
\curveto(364.33800787,418.81209229)(364.36300785,418.81209229)(364.39300903,418.81210297)
\curveto(364.42300779,418.82209228)(364.44800776,418.82709227)(364.46800903,418.82710297)
\lineto(365.56300903,418.82710297)
\curveto(365.66300655,418.82709227)(365.75800645,418.82209228)(365.84800903,418.81210297)
\curveto(365.93800627,418.8020923)(366.0080062,418.77209233)(366.05800903,418.72210297)
\moveto(366.16300903,408.95710297)
\curveto(366.16300605,408.75710234)(366.15800605,408.58710251)(366.14800903,408.44710297)
\curveto(366.13800607,408.30710279)(366.04800616,408.21210289)(365.87800903,408.16210297)
\curveto(365.81800639,408.14210296)(365.75300646,408.13210297)(365.68300903,408.13210297)
\curveto(365.6130066,408.14210296)(365.53800667,408.14710295)(365.45800903,408.14710297)
\lineto(364.61800903,408.14710297)
\curveto(364.52800768,408.14710295)(364.43800777,408.15210295)(364.34800903,408.16210297)
\curveto(364.26800794,408.17210293)(364.208008,408.2021029)(364.16800903,408.25210297)
\curveto(364.1080081,408.32210278)(364.07300814,408.40710269)(364.06300903,408.50710297)
\lineto(364.06300903,408.85210297)
\lineto(364.06300903,415.18210297)
\lineto(364.06300903,415.48210297)
\curveto(364.06300815,415.58209552)(364.08300813,415.66209544)(364.12300903,415.72210297)
\curveto(364.18300803,415.79209531)(364.26800794,415.83709526)(364.37800903,415.85710297)
\curveto(364.39800781,415.86709523)(364.42300779,415.86709523)(364.45300903,415.85710297)
\curveto(364.49300772,415.85709524)(364.52300769,415.86209524)(364.54300903,415.87210297)
\lineto(365.29300903,415.87210297)
\lineto(365.48800903,415.87210297)
\curveto(365.56800664,415.88209522)(365.63300658,415.88209522)(365.68300903,415.87210297)
\lineto(365.80300903,415.87210297)
\curveto(365.86300635,415.85209525)(365.91800629,415.83709526)(365.96800903,415.82710297)
\curveto(366.01800619,415.81709528)(366.05800615,415.78709531)(366.08800903,415.73710297)
\curveto(366.12800608,415.68709541)(366.14800606,415.61709548)(366.14800903,415.52710297)
\curveto(366.15800605,415.43709566)(366.16300605,415.34209576)(366.16300903,415.24210297)
\lineto(366.16300903,408.95710297)
}
}
{
\newrgbcolor{curcolor}{0 0 0}
\pscustom[linestyle=none,fillstyle=solid,fillcolor=curcolor]
{
\newpath
\moveto(375.71519653,412.09210297)
\curveto(375.72518785,412.03209907)(375.73018785,411.94209916)(375.73019653,411.82210297)
\curveto(375.73018785,411.7020994)(375.72018786,411.61709948)(375.70019653,411.56710297)
\lineto(375.70019653,411.37210297)
\curveto(375.67018791,411.26209984)(375.65018793,411.15709994)(375.64019653,411.05710297)
\curveto(375.64018794,410.95710014)(375.62518795,410.85710024)(375.59519653,410.75710297)
\curveto(375.575188,410.66710043)(375.55518802,410.57210053)(375.53519653,410.47210297)
\curveto(375.51518806,410.38210072)(375.48518809,410.29210081)(375.44519653,410.20210297)
\curveto(375.3751882,410.03210107)(375.30518827,409.87210123)(375.23519653,409.72210297)
\curveto(375.16518841,409.58210152)(375.08518849,409.44210166)(374.99519653,409.30210297)
\curveto(374.93518864,409.21210189)(374.87018871,409.12710197)(374.80019653,409.04710297)
\curveto(374.74018884,408.97710212)(374.67018891,408.9021022)(374.59019653,408.82210297)
\lineto(374.48519653,408.71710297)
\curveto(374.43518914,408.66710243)(374.3801892,408.62210248)(374.32019653,408.58210297)
\lineto(374.17019653,408.46210297)
\curveto(374.09018949,408.4021027)(374.00018958,408.34710275)(373.90019653,408.29710297)
\curveto(373.81018977,408.25710284)(373.71518986,408.21210289)(373.61519653,408.16210297)
\curveto(373.51519006,408.11210299)(373.41019017,408.07710302)(373.30019653,408.05710297)
\curveto(373.20019038,408.03710306)(373.09519048,408.01710308)(372.98519653,407.99710297)
\curveto(372.92519065,407.97710312)(372.86019072,407.96710313)(372.79019653,407.96710297)
\curveto(372.73019085,407.96710313)(372.66519091,407.95710314)(372.59519653,407.93710297)
\lineto(372.46019653,407.93710297)
\curveto(372.3801912,407.91710318)(372.30519127,407.91710318)(372.23519653,407.93710297)
\lineto(372.08519653,407.93710297)
\curveto(372.02519155,407.95710314)(371.96019162,407.96710313)(371.89019653,407.96710297)
\curveto(371.83019175,407.95710314)(371.77019181,407.96210314)(371.71019653,407.98210297)
\curveto(371.55019203,408.03210307)(371.39519218,408.07710302)(371.24519653,408.11710297)
\curveto(371.10519247,408.15710294)(370.9751926,408.21710288)(370.85519653,408.29710297)
\curveto(370.78519279,408.33710276)(370.72019286,408.37710272)(370.66019653,408.41710297)
\curveto(370.60019298,408.46710263)(370.53519304,408.51710258)(370.46519653,408.56710297)
\lineto(370.28519653,408.70210297)
\curveto(370.20519337,408.76210234)(370.13519344,408.76710233)(370.07519653,408.71710297)
\curveto(370.02519355,408.68710241)(370.00019358,408.64710245)(370.00019653,408.59710297)
\curveto(370.00019358,408.55710254)(369.99019359,408.50710259)(369.97019653,408.44710297)
\curveto(369.95019363,408.34710275)(369.94019364,408.23210287)(369.94019653,408.10210297)
\curveto(369.95019363,407.97210313)(369.95519362,407.85210325)(369.95519653,407.74210297)
\lineto(369.95519653,406.21210297)
\curveto(369.95519362,406.08210502)(369.95019363,405.95710514)(369.94019653,405.83710297)
\curveto(369.94019364,405.70710539)(369.91519366,405.6021055)(369.86519653,405.52210297)
\curveto(369.83519374,405.48210562)(369.7801938,405.45210565)(369.70019653,405.43210297)
\curveto(369.62019396,405.41210569)(369.53019405,405.4021057)(369.43019653,405.40210297)
\curveto(369.33019425,405.39210571)(369.23019435,405.39210571)(369.13019653,405.40210297)
\lineto(368.87519653,405.40210297)
\lineto(368.47019653,405.40210297)
\lineto(368.36519653,405.40210297)
\curveto(368.32519525,405.4021057)(368.29019529,405.40710569)(368.26019653,405.41710297)
\lineto(368.14019653,405.41710297)
\curveto(367.97019561,405.46710563)(367.8801957,405.56710553)(367.87019653,405.71710297)
\curveto(367.86019572,405.85710524)(367.85519572,406.02710507)(367.85519653,406.22710297)
\lineto(367.85519653,415.03210297)
\curveto(367.85519572,415.14209596)(367.85019573,415.25709584)(367.84019653,415.37710297)
\curveto(367.84019574,415.50709559)(367.86519571,415.60709549)(367.91519653,415.67710297)
\curveto(367.95519562,415.74709535)(368.01019557,415.79209531)(368.08019653,415.81210297)
\curveto(368.13019545,415.83209527)(368.19019539,415.84209526)(368.26019653,415.84210297)
\lineto(368.48519653,415.84210297)
\lineto(369.20519653,415.84210297)
\lineto(369.49019653,415.84210297)
\curveto(369.580194,415.84209526)(369.65519392,415.81709528)(369.71519653,415.76710297)
\curveto(369.78519379,415.71709538)(369.82019376,415.65209545)(369.82019653,415.57210297)
\curveto(369.83019375,415.5020956)(369.85519372,415.42709567)(369.89519653,415.34710297)
\curveto(369.90519367,415.31709578)(369.91519366,415.29209581)(369.92519653,415.27210297)
\curveto(369.94519363,415.26209584)(369.96519361,415.24709585)(369.98519653,415.22710297)
\curveto(370.09519348,415.21709588)(370.18519339,415.24709585)(370.25519653,415.31710297)
\curveto(370.32519325,415.38709571)(370.39519318,415.44709565)(370.46519653,415.49710297)
\curveto(370.59519298,415.58709551)(370.73019285,415.66709543)(370.87019653,415.73710297)
\curveto(371.01019257,415.81709528)(371.16519241,415.88209522)(371.33519653,415.93210297)
\curveto(371.41519216,415.96209514)(371.50019208,415.98209512)(371.59019653,415.99210297)
\curveto(371.69019189,416.0020951)(371.78519179,416.01709508)(371.87519653,416.03710297)
\curveto(371.91519166,416.04709505)(371.95519162,416.04709505)(371.99519653,416.03710297)
\curveto(372.04519153,416.02709507)(372.08519149,416.03209507)(372.11519653,416.05210297)
\curveto(372.68519089,416.07209503)(373.16519041,415.99209511)(373.55519653,415.81210297)
\curveto(373.95518962,415.64209546)(374.29518928,415.41709568)(374.57519653,415.13710297)
\curveto(374.62518895,415.08709601)(374.67018891,415.03709606)(374.71019653,414.98710297)
\curveto(374.75018883,414.94709615)(374.79018879,414.9020962)(374.83019653,414.85210297)
\curveto(374.90018868,414.76209634)(374.96018862,414.67209643)(375.01019653,414.58210297)
\curveto(375.07018851,414.49209661)(375.12518845,414.4020967)(375.17519653,414.31210297)
\curveto(375.19518838,414.29209681)(375.20518837,414.26709683)(375.20519653,414.23710297)
\curveto(375.21518836,414.20709689)(375.23018835,414.17209693)(375.25019653,414.13210297)
\curveto(375.31018827,414.03209707)(375.36518821,413.91209719)(375.41519653,413.77210297)
\curveto(375.43518814,413.71209739)(375.45518812,413.64709745)(375.47519653,413.57710297)
\curveto(375.49518808,413.51709758)(375.51518806,413.45209765)(375.53519653,413.38210297)
\curveto(375.575188,413.26209784)(375.60018798,413.13709796)(375.61019653,413.00710297)
\curveto(375.63018795,412.87709822)(375.65518792,412.74209836)(375.68519653,412.60210297)
\lineto(375.68519653,412.43710297)
\lineto(375.71519653,412.25710297)
\lineto(375.71519653,412.09210297)
\moveto(373.60019653,411.74710297)
\curveto(373.61018997,411.7970993)(373.61518996,411.86209924)(373.61519653,411.94210297)
\curveto(373.61518996,412.03209907)(373.61018997,412.102099)(373.60019653,412.15210297)
\lineto(373.60019653,412.28710297)
\curveto(373.58019,412.34709875)(373.57019001,412.41209869)(373.57019653,412.48210297)
\curveto(373.57019001,412.55209855)(373.56019002,412.62209848)(373.54019653,412.69210297)
\curveto(373.52019006,412.79209831)(373.50019008,412.88709821)(373.48019653,412.97710297)
\curveto(373.46019012,413.07709802)(373.43019015,413.16709793)(373.39019653,413.24710297)
\curveto(373.27019031,413.56709753)(373.11519046,413.82209728)(372.92519653,414.01210297)
\curveto(372.73519084,414.2020969)(372.46519111,414.34209676)(372.11519653,414.43210297)
\curveto(372.03519154,414.45209665)(371.94519163,414.46209664)(371.84519653,414.46210297)
\lineto(371.57519653,414.46210297)
\curveto(371.53519204,414.45209665)(371.50019208,414.44709665)(371.47019653,414.44710297)
\curveto(371.44019214,414.44709665)(371.40519217,414.44209666)(371.36519653,414.43210297)
\lineto(371.15519653,414.37210297)
\curveto(371.09519248,414.36209674)(371.03519254,414.34209676)(370.97519653,414.31210297)
\curveto(370.71519286,414.2020969)(370.51019307,414.03209707)(370.36019653,413.80210297)
\curveto(370.22019336,413.57209753)(370.10519347,413.31709778)(370.01519653,413.03710297)
\curveto(369.99519358,412.95709814)(369.9801936,412.87209823)(369.97019653,412.78210297)
\curveto(369.96019362,412.7020984)(369.94519363,412.62209848)(369.92519653,412.54210297)
\curveto(369.91519366,412.5020986)(369.91019367,412.43709866)(369.91019653,412.34710297)
\curveto(369.89019369,412.30709879)(369.88519369,412.25709884)(369.89519653,412.19710297)
\curveto(369.90519367,412.14709895)(369.90519367,412.097099)(369.89519653,412.04710297)
\curveto(369.8751937,411.98709911)(369.8751937,411.93209917)(369.89519653,411.88210297)
\lineto(369.89519653,411.70210297)
\lineto(369.89519653,411.56710297)
\curveto(369.89519368,411.52709957)(369.90519367,411.48709961)(369.92519653,411.44710297)
\curveto(369.92519365,411.37709972)(369.93019365,411.32209978)(369.94019653,411.28210297)
\lineto(369.97019653,411.10210297)
\curveto(369.9801936,411.04210006)(369.99519358,410.98210012)(370.01519653,410.92210297)
\curveto(370.10519347,410.63210047)(370.21019337,410.39210071)(370.33019653,410.20210297)
\curveto(370.46019312,410.02210108)(370.64019294,409.86210124)(370.87019653,409.72210297)
\curveto(371.01019257,409.64210146)(371.1751924,409.57710152)(371.36519653,409.52710297)
\curveto(371.40519217,409.51710158)(371.44019214,409.51210159)(371.47019653,409.51210297)
\curveto(371.50019208,409.52210158)(371.53519204,409.52210158)(371.57519653,409.51210297)
\curveto(371.61519196,409.5021016)(371.6751919,409.49210161)(371.75519653,409.48210297)
\curveto(371.83519174,409.48210162)(371.90019168,409.48710161)(371.95019653,409.49710297)
\curveto(372.03019155,409.51710158)(372.11019147,409.53210157)(372.19019653,409.54210297)
\curveto(372.2801913,409.56210154)(372.36519121,409.58710151)(372.44519653,409.61710297)
\curveto(372.68519089,409.71710138)(372.8801907,409.85710124)(373.03019653,410.03710297)
\curveto(373.1801904,410.21710088)(373.30519027,410.42710067)(373.40519653,410.66710297)
\curveto(373.45519012,410.78710031)(373.49019009,410.91210019)(373.51019653,411.04210297)
\curveto(373.53019005,411.17209993)(373.55519002,411.30709979)(373.58519653,411.44710297)
\lineto(373.58519653,411.59710297)
\curveto(373.59518998,411.64709945)(373.60018998,411.6970994)(373.60019653,411.74710297)
}
}
{
\newrgbcolor{curcolor}{0 0 0}
\pscustom[linestyle=none,fillstyle=solid,fillcolor=curcolor]
{
\newpath
\moveto(384.76511841,412.31710297)
\curveto(384.78510984,412.25709884)(384.79510983,412.17209893)(384.79511841,412.06210297)
\curveto(384.79510983,411.95209915)(384.78510984,411.86709923)(384.76511841,411.80710297)
\lineto(384.76511841,411.65710297)
\curveto(384.74510988,411.57709952)(384.73510989,411.4970996)(384.73511841,411.41710297)
\curveto(384.74510988,411.33709976)(384.74010988,411.25709984)(384.72011841,411.17710297)
\curveto(384.70010992,411.10709999)(384.68510994,411.04210006)(384.67511841,410.98210297)
\curveto(384.66510996,410.92210018)(384.65510997,410.85710024)(384.64511841,410.78710297)
\curveto(384.60511002,410.67710042)(384.57011005,410.56210054)(384.54011841,410.44210297)
\curveto(384.51011011,410.33210077)(384.47011015,410.22710087)(384.42011841,410.12710297)
\curveto(384.21011041,409.64710145)(383.93511069,409.25710184)(383.59511841,408.95710297)
\curveto(383.25511137,408.65710244)(382.84511178,408.40710269)(382.36511841,408.20710297)
\curveto(382.24511238,408.15710294)(382.1201125,408.12210298)(381.99011841,408.10210297)
\curveto(381.87011275,408.07210303)(381.74511288,408.04210306)(381.61511841,408.01210297)
\curveto(381.56511306,407.99210311)(381.51011311,407.98210312)(381.45011841,407.98210297)
\curveto(381.39011323,407.98210312)(381.33511329,407.97710312)(381.28511841,407.96710297)
\lineto(381.18011841,407.96710297)
\curveto(381.15011347,407.95710314)(381.1201135,407.95210315)(381.09011841,407.95210297)
\curveto(381.04011358,407.94210316)(380.96011366,407.93710316)(380.85011841,407.93710297)
\curveto(380.74011388,407.92710317)(380.65511397,407.93210317)(380.59511841,407.95210297)
\lineto(380.44511841,407.95210297)
\curveto(380.39511423,407.96210314)(380.34011428,407.96710313)(380.28011841,407.96710297)
\curveto(380.23011439,407.95710314)(380.18011444,407.96210314)(380.13011841,407.98210297)
\curveto(380.09011453,407.99210311)(380.05011457,407.9971031)(380.01011841,407.99710297)
\curveto(379.98011464,407.9971031)(379.94011468,408.0021031)(379.89011841,408.01210297)
\curveto(379.79011483,408.04210306)(379.69011493,408.06710303)(379.59011841,408.08710297)
\curveto(379.49011513,408.10710299)(379.39511523,408.13710296)(379.30511841,408.17710297)
\curveto(379.18511544,408.21710288)(379.07011555,408.25710284)(378.96011841,408.29710297)
\curveto(378.86011576,408.33710276)(378.75511587,408.38710271)(378.64511841,408.44710297)
\curveto(378.29511633,408.65710244)(377.99511663,408.9021022)(377.74511841,409.18210297)
\curveto(377.49511713,409.46210164)(377.28511734,409.7971013)(377.11511841,410.18710297)
\curveto(377.06511756,410.27710082)(377.0251176,410.37210073)(376.99511841,410.47210297)
\curveto(376.97511765,410.57210053)(376.95011767,410.67710042)(376.92011841,410.78710297)
\curveto(376.90011772,410.83710026)(376.89011773,410.88210022)(376.89011841,410.92210297)
\curveto(376.89011773,410.96210014)(376.88011774,411.00710009)(376.86011841,411.05710297)
\curveto(376.84011778,411.13709996)(376.83011779,411.21709988)(376.83011841,411.29710297)
\curveto(376.83011779,411.38709971)(376.8201178,411.47209963)(376.80011841,411.55210297)
\curveto(376.79011783,411.6020995)(376.78511784,411.64709945)(376.78511841,411.68710297)
\lineto(376.78511841,411.82210297)
\curveto(376.76511786,411.88209922)(376.75511787,411.96709913)(376.75511841,412.07710297)
\curveto(376.76511786,412.18709891)(376.78011784,412.27209883)(376.80011841,412.33210297)
\lineto(376.80011841,412.43710297)
\curveto(376.81011781,412.48709861)(376.81011781,412.53709856)(376.80011841,412.58710297)
\curveto(376.80011782,412.64709845)(376.81011781,412.7020984)(376.83011841,412.75210297)
\curveto(376.84011778,412.8020983)(376.84511778,412.84709825)(376.84511841,412.88710297)
\curveto(376.84511778,412.93709816)(376.85511777,412.98709811)(376.87511841,413.03710297)
\curveto(376.91511771,413.16709793)(376.95011767,413.29209781)(376.98011841,413.41210297)
\curveto(377.01011761,413.54209756)(377.05011757,413.66709743)(377.10011841,413.78710297)
\curveto(377.28011734,414.1970969)(377.49511713,414.53709656)(377.74511841,414.80710297)
\curveto(377.99511663,415.08709601)(378.30011632,415.34209576)(378.66011841,415.57210297)
\curveto(378.76011586,415.62209548)(378.86511576,415.66709543)(378.97511841,415.70710297)
\curveto(379.08511554,415.74709535)(379.19511543,415.79209531)(379.30511841,415.84210297)
\curveto(379.43511519,415.89209521)(379.57011505,415.92709517)(379.71011841,415.94710297)
\curveto(379.85011477,415.96709513)(379.99511463,415.9970951)(380.14511841,416.03710297)
\curveto(380.2251144,416.04709505)(380.30011432,416.05209505)(380.37011841,416.05210297)
\curveto(380.44011418,416.05209505)(380.51011411,416.05709504)(380.58011841,416.06710297)
\curveto(381.16011346,416.07709502)(381.66011296,416.01709508)(382.08011841,415.88710297)
\curveto(382.51011211,415.75709534)(382.89011173,415.57709552)(383.22011841,415.34710297)
\curveto(383.33011129,415.26709583)(383.44011118,415.17709592)(383.55011841,415.07710297)
\curveto(383.67011095,414.98709611)(383.77011085,414.88709621)(383.85011841,414.77710297)
\curveto(383.93011069,414.67709642)(384.00011062,414.57709652)(384.06011841,414.47710297)
\curveto(384.13011049,414.37709672)(384.20011042,414.27209683)(384.27011841,414.16210297)
\curveto(384.34011028,414.05209705)(384.39511023,413.93209717)(384.43511841,413.80210297)
\curveto(384.47511015,413.68209742)(384.5201101,413.55209755)(384.57011841,413.41210297)
\curveto(384.60011002,413.33209777)(384.62511,413.24709785)(384.64511841,413.15710297)
\lineto(384.70511841,412.88710297)
\curveto(384.71510991,412.84709825)(384.7201099,412.80709829)(384.72011841,412.76710297)
\curveto(384.7201099,412.72709837)(384.7251099,412.68709841)(384.73511841,412.64710297)
\curveto(384.75510987,412.5970985)(384.76010986,412.54209856)(384.75011841,412.48210297)
\curveto(384.74010988,412.42209868)(384.74510988,412.36709873)(384.76511841,412.31710297)
\moveto(382.66511841,411.77710297)
\curveto(382.67511195,411.82709927)(382.68011194,411.8970992)(382.68011841,411.98710297)
\curveto(382.68011194,412.08709901)(382.67511195,412.16209894)(382.66511841,412.21210297)
\lineto(382.66511841,412.33210297)
\curveto(382.64511198,412.38209872)(382.63511199,412.43709866)(382.63511841,412.49710297)
\curveto(382.63511199,412.55709854)(382.63011199,412.61209849)(382.62011841,412.66210297)
\curveto(382.620112,412.7020984)(382.61511201,412.73209837)(382.60511841,412.75210297)
\lineto(382.54511841,412.99210297)
\curveto(382.53511209,413.08209802)(382.51511211,413.16709793)(382.48511841,413.24710297)
\curveto(382.37511225,413.50709759)(382.24511238,413.72709737)(382.09511841,413.90710297)
\curveto(381.94511268,414.097097)(381.74511288,414.24709685)(381.49511841,414.35710297)
\curveto(381.43511319,414.37709672)(381.37511325,414.39209671)(381.31511841,414.40210297)
\curveto(381.25511337,414.42209668)(381.19011343,414.44209666)(381.12011841,414.46210297)
\curveto(381.04011358,414.48209662)(380.95511367,414.48709661)(380.86511841,414.47710297)
\lineto(380.59511841,414.47710297)
\curveto(380.56511406,414.45709664)(380.53011409,414.44709665)(380.49011841,414.44710297)
\curveto(380.45011417,414.45709664)(380.41511421,414.45709664)(380.38511841,414.44710297)
\lineto(380.17511841,414.38710297)
\curveto(380.11511451,414.37709672)(380.06011456,414.35709674)(380.01011841,414.32710297)
\curveto(379.76011486,414.21709688)(379.55511507,414.05709704)(379.39511841,413.84710297)
\curveto(379.24511538,413.64709745)(379.1251155,413.41209769)(379.03511841,413.14210297)
\curveto(379.00511562,413.04209806)(378.98011564,412.93709816)(378.96011841,412.82710297)
\curveto(378.95011567,412.71709838)(378.93511569,412.60709849)(378.91511841,412.49710297)
\curveto(378.90511572,412.44709865)(378.90011572,412.3970987)(378.90011841,412.34710297)
\lineto(378.90011841,412.19710297)
\curveto(378.88011574,412.12709897)(378.87011575,412.02209908)(378.87011841,411.88210297)
\curveto(378.88011574,411.74209936)(378.89511573,411.63709946)(378.91511841,411.56710297)
\lineto(378.91511841,411.43210297)
\curveto(378.93511569,411.35209975)(378.95011567,411.27209983)(378.96011841,411.19210297)
\curveto(378.97011565,411.12209998)(378.98511564,411.04710005)(379.00511841,410.96710297)
\curveto(379.10511552,410.66710043)(379.21011541,410.42210068)(379.32011841,410.23210297)
\curveto(379.44011518,410.05210105)(379.625115,409.88710121)(379.87511841,409.73710297)
\curveto(379.94511468,409.68710141)(380.0201146,409.64710145)(380.10011841,409.61710297)
\curveto(380.19011443,409.58710151)(380.28011434,409.56210154)(380.37011841,409.54210297)
\curveto(380.41011421,409.53210157)(380.44511418,409.52710157)(380.47511841,409.52710297)
\curveto(380.50511412,409.53710156)(380.54011408,409.53710156)(380.58011841,409.52710297)
\lineto(380.70011841,409.49710297)
\curveto(380.75011387,409.4971016)(380.79511383,409.5021016)(380.83511841,409.51210297)
\lineto(380.95511841,409.51210297)
\curveto(381.03511359,409.53210157)(381.11511351,409.54710155)(381.19511841,409.55710297)
\curveto(381.27511335,409.56710153)(381.35011327,409.58710151)(381.42011841,409.61710297)
\curveto(381.68011294,409.71710138)(381.89011273,409.85210125)(382.05011841,410.02210297)
\curveto(382.21011241,410.19210091)(382.34511228,410.4021007)(382.45511841,410.65210297)
\curveto(382.49511213,410.75210035)(382.5251121,410.85210025)(382.54511841,410.95210297)
\curveto(382.56511206,411.05210005)(382.59011203,411.15709994)(382.62011841,411.26710297)
\curveto(382.63011199,411.30709979)(382.63511199,411.34209976)(382.63511841,411.37210297)
\curveto(382.63511199,411.41209969)(382.64011198,411.45209965)(382.65011841,411.49210297)
\lineto(382.65011841,411.62710297)
\curveto(382.65011197,411.67709942)(382.65511197,411.72709937)(382.66511841,411.77710297)
}
}
{
\newrgbcolor{curcolor}{0 0 0}
\pscustom[linestyle=none,fillstyle=solid,fillcolor=curcolor]
{
}
}
{
\newrgbcolor{curcolor}{0 0 0}
\pscustom[linestyle=none,fillstyle=solid,fillcolor=curcolor]
{
\newpath
\moveto(397.91519653,408.98710297)
\lineto(397.91519653,408.56710297)
\curveto(397.91518816,408.43710266)(397.88518819,408.33210277)(397.82519653,408.25210297)
\curveto(397.7751883,408.2021029)(397.71018837,408.16710293)(397.63019653,408.14710297)
\curveto(397.55018853,408.13710296)(397.46018862,408.13210297)(397.36019653,408.13210297)
\lineto(396.53519653,408.13210297)
\lineto(396.25019653,408.13210297)
\curveto(396.17018991,408.14210296)(396.10518997,408.16710293)(396.05519653,408.20710297)
\curveto(395.98519009,408.25710284)(395.94519013,408.32210278)(395.93519653,408.40210297)
\curveto(395.92519015,408.48210262)(395.90519017,408.56210254)(395.87519653,408.64210297)
\curveto(395.85519022,408.66210244)(395.83519024,408.67710242)(395.81519653,408.68710297)
\curveto(395.80519027,408.70710239)(395.79019029,408.72710237)(395.77019653,408.74710297)
\curveto(395.66019042,408.74710235)(395.5801905,408.72210238)(395.53019653,408.67210297)
\lineto(395.38019653,408.52210297)
\curveto(395.31019077,408.47210263)(395.24519083,408.42710267)(395.18519653,408.38710297)
\curveto(395.12519095,408.35710274)(395.06019102,408.31710278)(394.99019653,408.26710297)
\curveto(394.95019113,408.24710285)(394.90519117,408.22710287)(394.85519653,408.20710297)
\curveto(394.81519126,408.18710291)(394.77019131,408.16710293)(394.72019653,408.14710297)
\curveto(394.5801915,408.097103)(394.43019165,408.05210305)(394.27019653,408.01210297)
\curveto(394.22019186,407.99210311)(394.1751919,407.98210312)(394.13519653,407.98210297)
\curveto(394.09519198,407.98210312)(394.05519202,407.97710312)(394.01519653,407.96710297)
\lineto(393.88019653,407.96710297)
\curveto(393.85019223,407.95710314)(393.81019227,407.95210315)(393.76019653,407.95210297)
\lineto(393.62519653,407.95210297)
\curveto(393.56519251,407.93210317)(393.4751926,407.92710317)(393.35519653,407.93710297)
\curveto(393.23519284,407.93710316)(393.15019293,407.94710315)(393.10019653,407.96710297)
\curveto(393.03019305,407.98710311)(392.96519311,407.9971031)(392.90519653,407.99710297)
\curveto(392.85519322,407.98710311)(392.80019328,407.99210311)(392.74019653,408.01210297)
\lineto(392.38019653,408.13210297)
\curveto(392.27019381,408.16210294)(392.16019392,408.2021029)(392.05019653,408.25210297)
\curveto(391.70019438,408.4021027)(391.38519469,408.63210247)(391.10519653,408.94210297)
\curveto(390.83519524,409.26210184)(390.62019546,409.5971015)(390.46019653,409.94710297)
\curveto(390.41019567,410.05710104)(390.37019571,410.16210094)(390.34019653,410.26210297)
\curveto(390.31019577,410.37210073)(390.2751958,410.48210062)(390.23519653,410.59210297)
\curveto(390.22519585,410.63210047)(390.22019586,410.66710043)(390.22019653,410.69710297)
\curveto(390.22019586,410.73710036)(390.21019587,410.78210032)(390.19019653,410.83210297)
\curveto(390.17019591,410.91210019)(390.15019593,410.9971001)(390.13019653,411.08710297)
\curveto(390.12019596,411.18709991)(390.10519597,411.28709981)(390.08519653,411.38710297)
\curveto(390.075196,411.41709968)(390.07019601,411.45209965)(390.07019653,411.49210297)
\curveto(390.080196,411.53209957)(390.080196,411.56709953)(390.07019653,411.59710297)
\lineto(390.07019653,411.73210297)
\curveto(390.07019601,411.78209932)(390.06519601,411.83209927)(390.05519653,411.88210297)
\curveto(390.04519603,411.93209917)(390.04019604,411.98709911)(390.04019653,412.04710297)
\curveto(390.04019604,412.11709898)(390.04519603,412.17209893)(390.05519653,412.21210297)
\curveto(390.06519601,412.26209884)(390.07019601,412.30709879)(390.07019653,412.34710297)
\lineto(390.07019653,412.49710297)
\curveto(390.080196,412.54709855)(390.080196,412.59209851)(390.07019653,412.63210297)
\curveto(390.07019601,412.68209842)(390.080196,412.73209837)(390.10019653,412.78210297)
\curveto(390.12019596,412.89209821)(390.13519594,412.9970981)(390.14519653,413.09710297)
\curveto(390.16519591,413.1970979)(390.19019589,413.2970978)(390.22019653,413.39710297)
\curveto(390.26019582,413.51709758)(390.29519578,413.63209747)(390.32519653,413.74210297)
\curveto(390.35519572,413.85209725)(390.39519568,413.96209714)(390.44519653,414.07210297)
\curveto(390.58519549,414.37209673)(390.76019532,414.65709644)(390.97019653,414.92710297)
\curveto(390.99019509,414.95709614)(391.01519506,414.98209612)(391.04519653,415.00210297)
\curveto(391.08519499,415.03209607)(391.11519496,415.06209604)(391.13519653,415.09210297)
\curveto(391.1751949,415.14209596)(391.21519486,415.18709591)(391.25519653,415.22710297)
\curveto(391.29519478,415.26709583)(391.34019474,415.30709579)(391.39019653,415.34710297)
\curveto(391.43019465,415.36709573)(391.46519461,415.39209571)(391.49519653,415.42210297)
\curveto(391.52519455,415.46209564)(391.56019452,415.49209561)(391.60019653,415.51210297)
\curveto(391.85019423,415.68209542)(392.14019394,415.82209528)(392.47019653,415.93210297)
\curveto(392.54019354,415.95209515)(392.61019347,415.96709513)(392.68019653,415.97710297)
\curveto(392.76019332,415.98709511)(392.84019324,416.0020951)(392.92019653,416.02210297)
\curveto(392.99019309,416.04209506)(393.080193,416.05209505)(393.19019653,416.05210297)
\curveto(393.30019278,416.06209504)(393.41019267,416.06709503)(393.52019653,416.06710297)
\curveto(393.63019245,416.06709503)(393.73519234,416.06209504)(393.83519653,416.05210297)
\curveto(393.94519213,416.04209506)(394.03519204,416.02709507)(394.10519653,416.00710297)
\curveto(394.25519182,415.95709514)(394.40019168,415.91209519)(394.54019653,415.87210297)
\curveto(394.6801914,415.83209527)(394.81019127,415.77709532)(394.93019653,415.70710297)
\curveto(395.00019108,415.65709544)(395.06519101,415.60709549)(395.12519653,415.55710297)
\curveto(395.18519089,415.51709558)(395.25019083,415.47209563)(395.32019653,415.42210297)
\curveto(395.36019072,415.39209571)(395.41519066,415.35209575)(395.48519653,415.30210297)
\curveto(395.56519051,415.25209585)(395.64019044,415.25209585)(395.71019653,415.30210297)
\curveto(395.75019033,415.32209578)(395.77019031,415.35709574)(395.77019653,415.40710297)
\curveto(395.77019031,415.45709564)(395.7801903,415.50709559)(395.80019653,415.55710297)
\lineto(395.80019653,415.70710297)
\curveto(395.81019027,415.73709536)(395.81519026,415.77209533)(395.81519653,415.81210297)
\lineto(395.81519653,415.93210297)
\lineto(395.81519653,417.97210297)
\curveto(395.81519026,418.08209302)(395.81019027,418.2020929)(395.80019653,418.33210297)
\curveto(395.80019028,418.47209263)(395.82519025,418.57709252)(395.87519653,418.64710297)
\curveto(395.91519016,418.72709237)(395.99019009,418.77709232)(396.10019653,418.79710297)
\curveto(396.12018996,418.80709229)(396.14018994,418.80709229)(396.16019653,418.79710297)
\curveto(396.1801899,418.7970923)(396.20018988,418.8020923)(396.22019653,418.81210297)
\lineto(397.28519653,418.81210297)
\curveto(397.40518867,418.81209229)(397.51518856,418.80709229)(397.61519653,418.79710297)
\curveto(397.71518836,418.78709231)(397.79018829,418.74709235)(397.84019653,418.67710297)
\curveto(397.89018819,418.5970925)(397.91518816,418.49209261)(397.91519653,418.36210297)
\lineto(397.91519653,418.00210297)
\lineto(397.91519653,408.98710297)
\moveto(395.87519653,411.92710297)
\curveto(395.88519019,411.96709913)(395.88519019,412.00709909)(395.87519653,412.04710297)
\lineto(395.87519653,412.18210297)
\curveto(395.8751902,412.28209882)(395.87019021,412.38209872)(395.86019653,412.48210297)
\curveto(395.85019023,412.58209852)(395.83519024,412.67209843)(395.81519653,412.75210297)
\curveto(395.79519028,412.86209824)(395.7751903,412.96209814)(395.75519653,413.05210297)
\curveto(395.74519033,413.14209796)(395.72019036,413.22709787)(395.68019653,413.30710297)
\curveto(395.54019054,413.66709743)(395.33519074,413.95209715)(395.06519653,414.16210297)
\curveto(394.80519127,414.37209673)(394.42519165,414.47709662)(393.92519653,414.47710297)
\curveto(393.86519221,414.47709662)(393.78519229,414.46709663)(393.68519653,414.44710297)
\curveto(393.60519247,414.42709667)(393.53019255,414.40709669)(393.46019653,414.38710297)
\curveto(393.40019268,414.37709672)(393.34019274,414.35709674)(393.28019653,414.32710297)
\curveto(393.01019307,414.21709688)(392.80019328,414.04709705)(392.65019653,413.81710297)
\curveto(392.50019358,413.58709751)(392.3801937,413.32709777)(392.29019653,413.03710297)
\curveto(392.26019382,412.93709816)(392.24019384,412.83709826)(392.23019653,412.73710297)
\curveto(392.22019386,412.63709846)(392.20019388,412.53209857)(392.17019653,412.42210297)
\lineto(392.17019653,412.21210297)
\curveto(392.15019393,412.12209898)(392.14519393,411.9970991)(392.15519653,411.83710297)
\curveto(392.16519391,411.68709941)(392.1801939,411.57709952)(392.20019653,411.50710297)
\lineto(392.20019653,411.41710297)
\curveto(392.21019387,411.3970997)(392.21519386,411.37709972)(392.21519653,411.35710297)
\curveto(392.23519384,411.27709982)(392.25019383,411.2020999)(392.26019653,411.13210297)
\curveto(392.2801938,411.06210004)(392.30019378,410.98710011)(392.32019653,410.90710297)
\curveto(392.49019359,410.38710071)(392.7801933,410.0021011)(393.19019653,409.75210297)
\curveto(393.32019276,409.66210144)(393.50019258,409.59210151)(393.73019653,409.54210297)
\curveto(393.77019231,409.53210157)(393.83019225,409.52710157)(393.91019653,409.52710297)
\curveto(393.94019214,409.51710158)(393.98519209,409.50710159)(394.04519653,409.49710297)
\curveto(394.11519196,409.4971016)(394.17019191,409.5021016)(394.21019653,409.51210297)
\curveto(394.29019179,409.53210157)(394.37019171,409.54710155)(394.45019653,409.55710297)
\curveto(394.53019155,409.56710153)(394.61019147,409.58710151)(394.69019653,409.61710297)
\curveto(394.94019114,409.72710137)(395.14019094,409.86710123)(395.29019653,410.03710297)
\curveto(395.44019064,410.20710089)(395.57019051,410.42210068)(395.68019653,410.68210297)
\curveto(395.72019036,410.77210033)(395.75019033,410.86210024)(395.77019653,410.95210297)
\curveto(395.79019029,411.05210005)(395.81019027,411.15709994)(395.83019653,411.26710297)
\curveto(395.84019024,411.31709978)(395.84019024,411.36209974)(395.83019653,411.40210297)
\curveto(395.83019025,411.45209965)(395.84019024,411.5020996)(395.86019653,411.55210297)
\curveto(395.87019021,411.58209952)(395.8751902,411.61709948)(395.87519653,411.65710297)
\lineto(395.87519653,411.79210297)
\lineto(395.87519653,411.92710297)
}
}
{
\newrgbcolor{curcolor}{0 0 0}
\pscustom[linestyle=none,fillstyle=solid,fillcolor=curcolor]
{
\newpath
\moveto(406.86011841,412.07710297)
\curveto(406.88011024,411.9970991)(406.88011024,411.90709919)(406.86011841,411.80710297)
\curveto(406.84011028,411.70709939)(406.80511032,411.64209946)(406.75511841,411.61210297)
\curveto(406.70511042,411.57209953)(406.63011049,411.54209956)(406.53011841,411.52210297)
\curveto(406.44011068,411.51209959)(406.33511079,411.5020996)(406.21511841,411.49210297)
\lineto(405.87011841,411.49210297)
\curveto(405.76011136,411.5020996)(405.66011146,411.50709959)(405.57011841,411.50710297)
\lineto(401.91011841,411.50710297)
\lineto(401.70011841,411.50710297)
\curveto(401.64011548,411.50709959)(401.58511554,411.4970996)(401.53511841,411.47710297)
\curveto(401.45511567,411.43709966)(401.40511572,411.3970997)(401.38511841,411.35710297)
\curveto(401.36511576,411.33709976)(401.34511578,411.2970998)(401.32511841,411.23710297)
\curveto(401.30511582,411.18709991)(401.30011582,411.13709996)(401.31011841,411.08710297)
\curveto(401.33011579,411.02710007)(401.34011578,410.96710013)(401.34011841,410.90710297)
\curveto(401.35011577,410.85710024)(401.36511576,410.8021003)(401.38511841,410.74210297)
\curveto(401.46511566,410.5021006)(401.56011556,410.3021008)(401.67011841,410.14210297)
\curveto(401.79011533,409.99210111)(401.95011517,409.85710124)(402.15011841,409.73710297)
\curveto(402.23011489,409.68710141)(402.31011481,409.65210145)(402.39011841,409.63210297)
\curveto(402.48011464,409.62210148)(402.57011455,409.6021015)(402.66011841,409.57210297)
\curveto(402.74011438,409.55210155)(402.85011427,409.53710156)(402.99011841,409.52710297)
\curveto(403.13011399,409.51710158)(403.25011387,409.52210158)(403.35011841,409.54210297)
\lineto(403.48511841,409.54210297)
\curveto(403.58511354,409.56210154)(403.67511345,409.58210152)(403.75511841,409.60210297)
\curveto(403.84511328,409.63210147)(403.93011319,409.66210144)(404.01011841,409.69210297)
\curveto(404.11011301,409.74210136)(404.2201129,409.80710129)(404.34011841,409.88710297)
\curveto(404.47011265,409.96710113)(404.56511256,410.04710105)(404.62511841,410.12710297)
\curveto(404.67511245,410.1971009)(404.7251124,410.26210084)(404.77511841,410.32210297)
\curveto(404.83511229,410.39210071)(404.90511222,410.44210066)(404.98511841,410.47210297)
\curveto(405.08511204,410.52210058)(405.21011191,410.54210056)(405.36011841,410.53210297)
\lineto(405.79511841,410.53210297)
\lineto(405.97511841,410.53210297)
\curveto(406.04511108,410.54210056)(406.10511102,410.53710056)(406.15511841,410.51710297)
\lineto(406.30511841,410.51710297)
\curveto(406.40511072,410.4971006)(406.47511065,410.47210063)(406.51511841,410.44210297)
\curveto(406.55511057,410.42210068)(406.57511055,410.37710072)(406.57511841,410.30710297)
\curveto(406.58511054,410.23710086)(406.58011054,410.17710092)(406.56011841,410.12710297)
\curveto(406.51011061,409.98710111)(406.45511067,409.86210124)(406.39511841,409.75210297)
\curveto(406.33511079,409.64210146)(406.26511086,409.53210157)(406.18511841,409.42210297)
\curveto(405.96511116,409.09210201)(405.71511141,408.82710227)(405.43511841,408.62710297)
\curveto(405.15511197,408.42710267)(404.80511232,408.25710284)(404.38511841,408.11710297)
\curveto(404.27511285,408.07710302)(404.16511296,408.05210305)(404.05511841,408.04210297)
\curveto(403.94511318,408.03210307)(403.83011329,408.01210309)(403.71011841,407.98210297)
\curveto(403.67011345,407.97210313)(403.6251135,407.97210313)(403.57511841,407.98210297)
\curveto(403.53511359,407.98210312)(403.49511363,407.97710312)(403.45511841,407.96710297)
\lineto(403.29011841,407.96710297)
\curveto(403.24011388,407.94710315)(403.18011394,407.94210316)(403.11011841,407.95210297)
\curveto(403.05011407,407.95210315)(402.99511413,407.95710314)(402.94511841,407.96710297)
\curveto(402.86511426,407.97710312)(402.79511433,407.97710312)(402.73511841,407.96710297)
\curveto(402.67511445,407.95710314)(402.61011451,407.96210314)(402.54011841,407.98210297)
\curveto(402.49011463,408.0021031)(402.43511469,408.01210309)(402.37511841,408.01210297)
\curveto(402.31511481,408.01210309)(402.26011486,408.02210308)(402.21011841,408.04210297)
\curveto(402.10011502,408.06210304)(401.99011513,408.08710301)(401.88011841,408.11710297)
\curveto(401.77011535,408.13710296)(401.67011545,408.17210293)(401.58011841,408.22210297)
\curveto(401.47011565,408.26210284)(401.36511576,408.2971028)(401.26511841,408.32710297)
\curveto(401.17511595,408.36710273)(401.09011603,408.41210269)(401.01011841,408.46210297)
\curveto(400.69011643,408.66210244)(400.40511672,408.89210221)(400.15511841,409.15210297)
\curveto(399.90511722,409.42210168)(399.70011742,409.73210137)(399.54011841,410.08210297)
\curveto(399.49011763,410.19210091)(399.45011767,410.3021008)(399.42011841,410.41210297)
\curveto(399.39011773,410.53210057)(399.35011777,410.65210045)(399.30011841,410.77210297)
\curveto(399.29011783,410.81210029)(399.28511784,410.84710025)(399.28511841,410.87710297)
\curveto(399.28511784,410.91710018)(399.28011784,410.95710014)(399.27011841,410.99710297)
\curveto(399.23011789,411.11709998)(399.20511792,411.24709985)(399.19511841,411.38710297)
\lineto(399.16511841,411.80710297)
\curveto(399.16511796,411.85709924)(399.16011796,411.91209919)(399.15011841,411.97210297)
\curveto(399.15011797,412.03209907)(399.15511797,412.08709901)(399.16511841,412.13710297)
\lineto(399.16511841,412.31710297)
\lineto(399.21011841,412.67710297)
\curveto(399.25011787,412.84709825)(399.28511784,413.01209809)(399.31511841,413.17210297)
\curveto(399.34511778,413.33209777)(399.39011773,413.48209762)(399.45011841,413.62210297)
\curveto(399.88011724,414.66209644)(400.61011651,415.3970957)(401.64011841,415.82710297)
\curveto(401.78011534,415.88709521)(401.9201152,415.92709517)(402.06011841,415.94710297)
\curveto(402.21011491,415.97709512)(402.36511476,416.01209509)(402.52511841,416.05210297)
\curveto(402.60511452,416.06209504)(402.68011444,416.06709503)(402.75011841,416.06710297)
\curveto(402.8201143,416.06709503)(402.89511423,416.07209503)(402.97511841,416.08210297)
\curveto(403.48511364,416.09209501)(403.9201132,416.03209507)(404.28011841,415.90210297)
\curveto(404.65011247,415.78209532)(404.98011214,415.62209548)(405.27011841,415.42210297)
\curveto(405.36011176,415.36209574)(405.45011167,415.29209581)(405.54011841,415.21210297)
\curveto(405.63011149,415.14209596)(405.71011141,415.06709603)(405.78011841,414.98710297)
\curveto(405.81011131,414.93709616)(405.85011127,414.8970962)(405.90011841,414.86710297)
\curveto(405.98011114,414.75709634)(406.05511107,414.64209646)(406.12511841,414.52210297)
\curveto(406.19511093,414.41209669)(406.27011085,414.2970968)(406.35011841,414.17710297)
\curveto(406.40011072,414.08709701)(406.44011068,413.99209711)(406.47011841,413.89210297)
\curveto(406.51011061,413.8020973)(406.55011057,413.7020974)(406.59011841,413.59210297)
\curveto(406.64011048,413.46209764)(406.68011044,413.32709777)(406.71011841,413.18710297)
\curveto(406.74011038,413.04709805)(406.77511035,412.90709819)(406.81511841,412.76710297)
\curveto(406.83511029,412.68709841)(406.84011028,412.5970985)(406.83011841,412.49710297)
\curveto(406.83011029,412.40709869)(406.84011028,412.32209878)(406.86011841,412.24210297)
\lineto(406.86011841,412.07710297)
\moveto(404.61011841,412.96210297)
\curveto(404.68011244,413.06209804)(404.68511244,413.18209792)(404.62511841,413.32210297)
\curveto(404.57511255,413.47209763)(404.53511259,413.58209752)(404.50511841,413.65210297)
\curveto(404.36511276,413.92209718)(404.18011294,414.12709697)(403.95011841,414.26710297)
\curveto(403.7201134,414.41709668)(403.40011372,414.4970966)(402.99011841,414.50710297)
\curveto(402.96011416,414.48709661)(402.9251142,414.48209662)(402.88511841,414.49210297)
\curveto(402.84511428,414.5020966)(402.81011431,414.5020966)(402.78011841,414.49210297)
\curveto(402.73011439,414.47209663)(402.67511445,414.45709664)(402.61511841,414.44710297)
\curveto(402.55511457,414.44709665)(402.50011462,414.43709666)(402.45011841,414.41710297)
\curveto(402.01011511,414.27709682)(401.68511544,414.0020971)(401.47511841,413.59210297)
\curveto(401.45511567,413.55209755)(401.43011569,413.4970976)(401.40011841,413.42710297)
\curveto(401.38011574,413.36709773)(401.36511576,413.3020978)(401.35511841,413.23210297)
\curveto(401.34511578,413.17209793)(401.34511578,413.11209799)(401.35511841,413.05210297)
\curveto(401.37511575,412.99209811)(401.41011571,412.94209816)(401.46011841,412.90210297)
\curveto(401.54011558,412.85209825)(401.65011547,412.82709827)(401.79011841,412.82710297)
\lineto(402.19511841,412.82710297)
\lineto(403.86011841,412.82710297)
\lineto(404.29511841,412.82710297)
\curveto(404.45511267,412.83709826)(404.56011256,412.88209822)(404.61011841,412.96210297)
}
}
{
\newrgbcolor{curcolor}{0 0 0}
\pscustom[linestyle=none,fillstyle=solid,fillcolor=curcolor]
{
}
}
{
\newrgbcolor{curcolor}{0 0 0}
\pscustom[linestyle=none,fillstyle=solid,fillcolor=curcolor]
{
\newpath
\moveto(419.63355591,412.07710297)
\curveto(419.65354774,411.9970991)(419.65354774,411.90709919)(419.63355591,411.80710297)
\curveto(419.61354778,411.70709939)(419.57854782,411.64209946)(419.52855591,411.61210297)
\curveto(419.47854792,411.57209953)(419.40354799,411.54209956)(419.30355591,411.52210297)
\curveto(419.21354818,411.51209959)(419.10854829,411.5020996)(418.98855591,411.49210297)
\lineto(418.64355591,411.49210297)
\curveto(418.53354886,411.5020996)(418.43354896,411.50709959)(418.34355591,411.50710297)
\lineto(414.68355591,411.50710297)
\lineto(414.47355591,411.50710297)
\curveto(414.41355298,411.50709959)(414.35855304,411.4970996)(414.30855591,411.47710297)
\curveto(414.22855317,411.43709966)(414.17855322,411.3970997)(414.15855591,411.35710297)
\curveto(414.13855326,411.33709976)(414.11855328,411.2970998)(414.09855591,411.23710297)
\curveto(414.07855332,411.18709991)(414.07355332,411.13709996)(414.08355591,411.08710297)
\curveto(414.10355329,411.02710007)(414.11355328,410.96710013)(414.11355591,410.90710297)
\curveto(414.12355327,410.85710024)(414.13855326,410.8021003)(414.15855591,410.74210297)
\curveto(414.23855316,410.5021006)(414.33355306,410.3021008)(414.44355591,410.14210297)
\curveto(414.56355283,409.99210111)(414.72355267,409.85710124)(414.92355591,409.73710297)
\curveto(415.00355239,409.68710141)(415.08355231,409.65210145)(415.16355591,409.63210297)
\curveto(415.25355214,409.62210148)(415.34355205,409.6021015)(415.43355591,409.57210297)
\curveto(415.51355188,409.55210155)(415.62355177,409.53710156)(415.76355591,409.52710297)
\curveto(415.90355149,409.51710158)(416.02355137,409.52210158)(416.12355591,409.54210297)
\lineto(416.25855591,409.54210297)
\curveto(416.35855104,409.56210154)(416.44855095,409.58210152)(416.52855591,409.60210297)
\curveto(416.61855078,409.63210147)(416.70355069,409.66210144)(416.78355591,409.69210297)
\curveto(416.88355051,409.74210136)(416.9935504,409.80710129)(417.11355591,409.88710297)
\curveto(417.24355015,409.96710113)(417.33855006,410.04710105)(417.39855591,410.12710297)
\curveto(417.44854995,410.1971009)(417.4985499,410.26210084)(417.54855591,410.32210297)
\curveto(417.60854979,410.39210071)(417.67854972,410.44210066)(417.75855591,410.47210297)
\curveto(417.85854954,410.52210058)(417.98354941,410.54210056)(418.13355591,410.53210297)
\lineto(418.56855591,410.53210297)
\lineto(418.74855591,410.53210297)
\curveto(418.81854858,410.54210056)(418.87854852,410.53710056)(418.92855591,410.51710297)
\lineto(419.07855591,410.51710297)
\curveto(419.17854822,410.4971006)(419.24854815,410.47210063)(419.28855591,410.44210297)
\curveto(419.32854807,410.42210068)(419.34854805,410.37710072)(419.34855591,410.30710297)
\curveto(419.35854804,410.23710086)(419.35354804,410.17710092)(419.33355591,410.12710297)
\curveto(419.28354811,409.98710111)(419.22854817,409.86210124)(419.16855591,409.75210297)
\curveto(419.10854829,409.64210146)(419.03854836,409.53210157)(418.95855591,409.42210297)
\curveto(418.73854866,409.09210201)(418.48854891,408.82710227)(418.20855591,408.62710297)
\curveto(417.92854947,408.42710267)(417.57854982,408.25710284)(417.15855591,408.11710297)
\curveto(417.04855035,408.07710302)(416.93855046,408.05210305)(416.82855591,408.04210297)
\curveto(416.71855068,408.03210307)(416.60355079,408.01210309)(416.48355591,407.98210297)
\curveto(416.44355095,407.97210313)(416.398551,407.97210313)(416.34855591,407.98210297)
\curveto(416.30855109,407.98210312)(416.26855113,407.97710312)(416.22855591,407.96710297)
\lineto(416.06355591,407.96710297)
\curveto(416.01355138,407.94710315)(415.95355144,407.94210316)(415.88355591,407.95210297)
\curveto(415.82355157,407.95210315)(415.76855163,407.95710314)(415.71855591,407.96710297)
\curveto(415.63855176,407.97710312)(415.56855183,407.97710312)(415.50855591,407.96710297)
\curveto(415.44855195,407.95710314)(415.38355201,407.96210314)(415.31355591,407.98210297)
\curveto(415.26355213,408.0021031)(415.20855219,408.01210309)(415.14855591,408.01210297)
\curveto(415.08855231,408.01210309)(415.03355236,408.02210308)(414.98355591,408.04210297)
\curveto(414.87355252,408.06210304)(414.76355263,408.08710301)(414.65355591,408.11710297)
\curveto(414.54355285,408.13710296)(414.44355295,408.17210293)(414.35355591,408.22210297)
\curveto(414.24355315,408.26210284)(414.13855326,408.2971028)(414.03855591,408.32710297)
\curveto(413.94855345,408.36710273)(413.86355353,408.41210269)(413.78355591,408.46210297)
\curveto(413.46355393,408.66210244)(413.17855422,408.89210221)(412.92855591,409.15210297)
\curveto(412.67855472,409.42210168)(412.47355492,409.73210137)(412.31355591,410.08210297)
\curveto(412.26355513,410.19210091)(412.22355517,410.3021008)(412.19355591,410.41210297)
\curveto(412.16355523,410.53210057)(412.12355527,410.65210045)(412.07355591,410.77210297)
\curveto(412.06355533,410.81210029)(412.05855534,410.84710025)(412.05855591,410.87710297)
\curveto(412.05855534,410.91710018)(412.05355534,410.95710014)(412.04355591,410.99710297)
\curveto(412.00355539,411.11709998)(411.97855542,411.24709985)(411.96855591,411.38710297)
\lineto(411.93855591,411.80710297)
\curveto(411.93855546,411.85709924)(411.93355546,411.91209919)(411.92355591,411.97210297)
\curveto(411.92355547,412.03209907)(411.92855547,412.08709901)(411.93855591,412.13710297)
\lineto(411.93855591,412.31710297)
\lineto(411.98355591,412.67710297)
\curveto(412.02355537,412.84709825)(412.05855534,413.01209809)(412.08855591,413.17210297)
\curveto(412.11855528,413.33209777)(412.16355523,413.48209762)(412.22355591,413.62210297)
\curveto(412.65355474,414.66209644)(413.38355401,415.3970957)(414.41355591,415.82710297)
\curveto(414.55355284,415.88709521)(414.6935527,415.92709517)(414.83355591,415.94710297)
\curveto(414.98355241,415.97709512)(415.13855226,416.01209509)(415.29855591,416.05210297)
\curveto(415.37855202,416.06209504)(415.45355194,416.06709503)(415.52355591,416.06710297)
\curveto(415.5935518,416.06709503)(415.66855173,416.07209503)(415.74855591,416.08210297)
\curveto(416.25855114,416.09209501)(416.6935507,416.03209507)(417.05355591,415.90210297)
\curveto(417.42354997,415.78209532)(417.75354964,415.62209548)(418.04355591,415.42210297)
\curveto(418.13354926,415.36209574)(418.22354917,415.29209581)(418.31355591,415.21210297)
\curveto(418.40354899,415.14209596)(418.48354891,415.06709603)(418.55355591,414.98710297)
\curveto(418.58354881,414.93709616)(418.62354877,414.8970962)(418.67355591,414.86710297)
\curveto(418.75354864,414.75709634)(418.82854857,414.64209646)(418.89855591,414.52210297)
\curveto(418.96854843,414.41209669)(419.04354835,414.2970968)(419.12355591,414.17710297)
\curveto(419.17354822,414.08709701)(419.21354818,413.99209711)(419.24355591,413.89210297)
\curveto(419.28354811,413.8020973)(419.32354807,413.7020974)(419.36355591,413.59210297)
\curveto(419.41354798,413.46209764)(419.45354794,413.32709777)(419.48355591,413.18710297)
\curveto(419.51354788,413.04709805)(419.54854785,412.90709819)(419.58855591,412.76710297)
\curveto(419.60854779,412.68709841)(419.61354778,412.5970985)(419.60355591,412.49710297)
\curveto(419.60354779,412.40709869)(419.61354778,412.32209878)(419.63355591,412.24210297)
\lineto(419.63355591,412.07710297)
\moveto(417.38355591,412.96210297)
\curveto(417.45354994,413.06209804)(417.45854994,413.18209792)(417.39855591,413.32210297)
\curveto(417.34855005,413.47209763)(417.30855009,413.58209752)(417.27855591,413.65210297)
\curveto(417.13855026,413.92209718)(416.95355044,414.12709697)(416.72355591,414.26710297)
\curveto(416.4935509,414.41709668)(416.17355122,414.4970966)(415.76355591,414.50710297)
\curveto(415.73355166,414.48709661)(415.6985517,414.48209662)(415.65855591,414.49210297)
\curveto(415.61855178,414.5020966)(415.58355181,414.5020966)(415.55355591,414.49210297)
\curveto(415.50355189,414.47209663)(415.44855195,414.45709664)(415.38855591,414.44710297)
\curveto(415.32855207,414.44709665)(415.27355212,414.43709666)(415.22355591,414.41710297)
\curveto(414.78355261,414.27709682)(414.45855294,414.0020971)(414.24855591,413.59210297)
\curveto(414.22855317,413.55209755)(414.20355319,413.4970976)(414.17355591,413.42710297)
\curveto(414.15355324,413.36709773)(414.13855326,413.3020978)(414.12855591,413.23210297)
\curveto(414.11855328,413.17209793)(414.11855328,413.11209799)(414.12855591,413.05210297)
\curveto(414.14855325,412.99209811)(414.18355321,412.94209816)(414.23355591,412.90210297)
\curveto(414.31355308,412.85209825)(414.42355297,412.82709827)(414.56355591,412.82710297)
\lineto(414.96855591,412.82710297)
\lineto(416.63355591,412.82710297)
\lineto(417.06855591,412.82710297)
\curveto(417.22855017,412.83709826)(417.33355006,412.88209822)(417.38355591,412.96210297)
}
}
{
\newrgbcolor{curcolor}{0 0 0}
\pscustom[linestyle=none,fillstyle=solid,fillcolor=curcolor]
{
\newpath
\moveto(423.85183716,416.08210297)
\curveto(424.60183266,416.102095)(425.25183201,416.01709508)(425.80183716,415.82710297)
\curveto(426.3618309,415.64709545)(426.78683047,415.33209577)(427.07683716,414.88210297)
\curveto(427.14683011,414.77209633)(427.20683005,414.65709644)(427.25683716,414.53710297)
\curveto(427.31682994,414.42709667)(427.36682989,414.3020968)(427.40683716,414.16210297)
\curveto(427.42682983,414.102097)(427.43682982,414.03709706)(427.43683716,413.96710297)
\curveto(427.43682982,413.8970972)(427.42682983,413.83709726)(427.40683716,413.78710297)
\curveto(427.36682989,413.72709737)(427.31182995,413.68709741)(427.24183716,413.66710297)
\curveto(427.19183007,413.64709745)(427.13183013,413.63709746)(427.06183716,413.63710297)
\lineto(426.85183716,413.63710297)
\lineto(426.19183716,413.63710297)
\curveto(426.12183114,413.63709746)(426.05183121,413.63209747)(425.98183716,413.62210297)
\curveto(425.91183135,413.62209748)(425.84683141,413.63209747)(425.78683716,413.65210297)
\curveto(425.68683157,413.67209743)(425.61183165,413.71209739)(425.56183716,413.77210297)
\curveto(425.51183175,413.83209727)(425.46683179,413.89209721)(425.42683716,413.95210297)
\lineto(425.30683716,414.16210297)
\curveto(425.27683198,414.24209686)(425.22683203,414.30709679)(425.15683716,414.35710297)
\curveto(425.0568322,414.43709666)(424.9568323,414.4970966)(424.85683716,414.53710297)
\curveto(424.76683249,414.57709652)(424.65183261,414.61209649)(424.51183716,414.64210297)
\curveto(424.44183282,414.66209644)(424.33683292,414.67709642)(424.19683716,414.68710297)
\curveto(424.06683319,414.6970964)(423.96683329,414.69209641)(423.89683716,414.67210297)
\lineto(423.79183716,414.67210297)
\lineto(423.64183716,414.64210297)
\curveto(423.60183366,414.64209646)(423.5568337,414.63709646)(423.50683716,414.62710297)
\curveto(423.33683392,414.57709652)(423.19683406,414.50709659)(423.08683716,414.41710297)
\curveto(422.98683427,414.33709676)(422.91683434,414.21209689)(422.87683716,414.04210297)
\curveto(422.8568344,413.97209713)(422.8568344,413.90709719)(422.87683716,413.84710297)
\curveto(422.89683436,413.78709731)(422.91683434,413.73709736)(422.93683716,413.69710297)
\curveto(423.00683425,413.57709752)(423.08683417,413.48209762)(423.17683716,413.41210297)
\curveto(423.27683398,413.34209776)(423.39183387,413.28209782)(423.52183716,413.23210297)
\curveto(423.71183355,413.15209795)(423.91683334,413.08209802)(424.13683716,413.02210297)
\lineto(424.82683716,412.87210297)
\curveto(425.06683219,412.83209827)(425.29683196,412.78209832)(425.51683716,412.72210297)
\curveto(425.74683151,412.67209843)(425.9618313,412.60709849)(426.16183716,412.52710297)
\curveto(426.25183101,412.48709861)(426.33683092,412.45209865)(426.41683716,412.42210297)
\curveto(426.50683075,412.4020987)(426.59183067,412.36709873)(426.67183716,412.31710297)
\curveto(426.8618304,412.1970989)(427.03183023,412.06709903)(427.18183716,411.92710297)
\curveto(427.34182992,411.78709931)(427.46682979,411.61209949)(427.55683716,411.40210297)
\curveto(427.58682967,411.33209977)(427.61182965,411.26209984)(427.63183716,411.19210297)
\curveto(427.65182961,411.12209998)(427.67182959,411.04710005)(427.69183716,410.96710297)
\curveto(427.70182956,410.90710019)(427.70682955,410.81210029)(427.70683716,410.68210297)
\curveto(427.71682954,410.56210054)(427.71682954,410.46710063)(427.70683716,410.39710297)
\lineto(427.70683716,410.32210297)
\curveto(427.68682957,410.26210084)(427.67182959,410.2021009)(427.66183716,410.14210297)
\curveto(427.6618296,410.09210101)(427.6568296,410.04210106)(427.64683716,409.99210297)
\curveto(427.57682968,409.69210141)(427.46682979,409.42710167)(427.31683716,409.19710297)
\curveto(427.1568301,408.95710214)(426.9618303,408.76210234)(426.73183716,408.61210297)
\curveto(426.50183076,408.46210264)(426.24183102,408.33210277)(425.95183716,408.22210297)
\curveto(425.84183142,408.17210293)(425.72183154,408.13710296)(425.59183716,408.11710297)
\curveto(425.47183179,408.097103)(425.35183191,408.07210303)(425.23183716,408.04210297)
\curveto(425.14183212,408.02210308)(425.04683221,408.01210309)(424.94683716,408.01210297)
\curveto(424.8568324,408.0021031)(424.76683249,407.98710311)(424.67683716,407.96710297)
\lineto(424.40683716,407.96710297)
\curveto(424.34683291,407.94710315)(424.24183302,407.93710316)(424.09183716,407.93710297)
\curveto(423.95183331,407.93710316)(423.85183341,407.94710315)(423.79183716,407.96710297)
\curveto(423.7618335,407.96710313)(423.72683353,407.97210313)(423.68683716,407.98210297)
\lineto(423.58183716,407.98210297)
\curveto(423.4618338,408.0021031)(423.34183392,408.01710308)(423.22183716,408.02710297)
\curveto(423.10183416,408.03710306)(422.98683427,408.05710304)(422.87683716,408.08710297)
\curveto(422.48683477,408.1971029)(422.14183512,408.32210278)(421.84183716,408.46210297)
\curveto(421.54183572,408.61210249)(421.28683597,408.83210227)(421.07683716,409.12210297)
\curveto(420.93683632,409.31210179)(420.81683644,409.53210157)(420.71683716,409.78210297)
\curveto(420.69683656,409.84210126)(420.67683658,409.92210118)(420.65683716,410.02210297)
\curveto(420.63683662,410.07210103)(420.62183664,410.14210096)(420.61183716,410.23210297)
\curveto(420.60183666,410.32210078)(420.60683665,410.3971007)(420.62683716,410.45710297)
\curveto(420.6568366,410.52710057)(420.70683655,410.57710052)(420.77683716,410.60710297)
\curveto(420.82683643,410.62710047)(420.88683637,410.63710046)(420.95683716,410.63710297)
\lineto(421.18183716,410.63710297)
\lineto(421.88683716,410.63710297)
\lineto(422.12683716,410.63710297)
\curveto(422.20683505,410.63710046)(422.27683498,410.62710047)(422.33683716,410.60710297)
\curveto(422.44683481,410.56710053)(422.51683474,410.5021006)(422.54683716,410.41210297)
\curveto(422.58683467,410.32210078)(422.63183463,410.22710087)(422.68183716,410.12710297)
\curveto(422.70183456,410.07710102)(422.73683452,410.01210109)(422.78683716,409.93210297)
\curveto(422.84683441,409.85210125)(422.89683436,409.8021013)(422.93683716,409.78210297)
\curveto(423.0568342,409.68210142)(423.17183409,409.6021015)(423.28183716,409.54210297)
\curveto(423.39183387,409.49210161)(423.53183373,409.44210166)(423.70183716,409.39210297)
\curveto(423.75183351,409.37210173)(423.80183346,409.36210174)(423.85183716,409.36210297)
\curveto(423.90183336,409.37210173)(423.95183331,409.37210173)(424.00183716,409.36210297)
\curveto(424.08183318,409.34210176)(424.16683309,409.33210177)(424.25683716,409.33210297)
\curveto(424.3568329,409.34210176)(424.44183282,409.35710174)(424.51183716,409.37710297)
\curveto(424.5618327,409.38710171)(424.60683265,409.39210171)(424.64683716,409.39210297)
\curveto(424.69683256,409.39210171)(424.74683251,409.4021017)(424.79683716,409.42210297)
\curveto(424.93683232,409.47210163)(425.0618322,409.53210157)(425.17183716,409.60210297)
\curveto(425.29183197,409.67210143)(425.38683187,409.76210134)(425.45683716,409.87210297)
\curveto(425.50683175,409.95210115)(425.54683171,410.07710102)(425.57683716,410.24710297)
\curveto(425.59683166,410.31710078)(425.59683166,410.38210072)(425.57683716,410.44210297)
\curveto(425.5568317,410.5021006)(425.53683172,410.55210055)(425.51683716,410.59210297)
\curveto(425.44683181,410.73210037)(425.3568319,410.83710026)(425.24683716,410.90710297)
\curveto(425.14683211,410.97710012)(425.02683223,411.04210006)(424.88683716,411.10210297)
\curveto(424.69683256,411.18209992)(424.49683276,411.24709985)(424.28683716,411.29710297)
\curveto(424.07683318,411.34709975)(423.86683339,411.4020997)(423.65683716,411.46210297)
\curveto(423.57683368,411.48209962)(423.49183377,411.4970996)(423.40183716,411.50710297)
\curveto(423.32183394,411.51709958)(423.24183402,411.53209957)(423.16183716,411.55210297)
\curveto(422.84183442,411.64209946)(422.53683472,411.72709937)(422.24683716,411.80710297)
\curveto(421.9568353,411.8970992)(421.69183557,412.02709907)(421.45183716,412.19710297)
\curveto(421.17183609,412.3970987)(420.96683629,412.66709843)(420.83683716,413.00710297)
\curveto(420.81683644,413.07709802)(420.79683646,413.17209793)(420.77683716,413.29210297)
\curveto(420.7568365,413.36209774)(420.74183652,413.44709765)(420.73183716,413.54710297)
\curveto(420.72183654,413.64709745)(420.72683653,413.73709736)(420.74683716,413.81710297)
\curveto(420.76683649,413.86709723)(420.77183649,413.90709719)(420.76183716,413.93710297)
\curveto(420.75183651,413.97709712)(420.7568365,414.02209708)(420.77683716,414.07210297)
\curveto(420.79683646,414.18209692)(420.81683644,414.28209682)(420.83683716,414.37210297)
\curveto(420.86683639,414.47209663)(420.90183636,414.56709653)(420.94183716,414.65710297)
\curveto(421.07183619,414.94709615)(421.25183601,415.18209592)(421.48183716,415.36210297)
\curveto(421.71183555,415.54209556)(421.97183529,415.68709541)(422.26183716,415.79710297)
\curveto(422.37183489,415.84709525)(422.48683477,415.88209522)(422.60683716,415.90210297)
\curveto(422.72683453,415.93209517)(422.85183441,415.96209514)(422.98183716,415.99210297)
\curveto(423.04183422,416.01209509)(423.10183416,416.02209508)(423.16183716,416.02210297)
\lineto(423.34183716,416.05210297)
\curveto(423.42183384,416.06209504)(423.50683375,416.06709503)(423.59683716,416.06710297)
\curveto(423.68683357,416.06709503)(423.77183349,416.07209503)(423.85183716,416.08210297)
}
}
{
\newrgbcolor{curcolor}{0 0 0}
\pscustom[linestyle=none,fillstyle=solid,fillcolor=curcolor]
{
\newpath
\moveto(436.82847778,412.09210297)
\curveto(436.8384691,412.03209907)(436.8434691,411.94209916)(436.84347778,411.82210297)
\curveto(436.8434691,411.7020994)(436.83346911,411.61709948)(436.81347778,411.56710297)
\lineto(436.81347778,411.37210297)
\curveto(436.78346916,411.26209984)(436.76346918,411.15709994)(436.75347778,411.05710297)
\curveto(436.75346919,410.95710014)(436.7384692,410.85710024)(436.70847778,410.75710297)
\curveto(436.68846925,410.66710043)(436.66846927,410.57210053)(436.64847778,410.47210297)
\curveto(436.62846931,410.38210072)(436.59846934,410.29210081)(436.55847778,410.20210297)
\curveto(436.48846945,410.03210107)(436.41846952,409.87210123)(436.34847778,409.72210297)
\curveto(436.27846966,409.58210152)(436.19846974,409.44210166)(436.10847778,409.30210297)
\curveto(436.04846989,409.21210189)(435.98346996,409.12710197)(435.91347778,409.04710297)
\curveto(435.85347009,408.97710212)(435.78347016,408.9021022)(435.70347778,408.82210297)
\lineto(435.59847778,408.71710297)
\curveto(435.54847039,408.66710243)(435.49347045,408.62210248)(435.43347778,408.58210297)
\lineto(435.28347778,408.46210297)
\curveto(435.20347074,408.4021027)(435.11347083,408.34710275)(435.01347778,408.29710297)
\curveto(434.92347102,408.25710284)(434.82847111,408.21210289)(434.72847778,408.16210297)
\curveto(434.62847131,408.11210299)(434.52347142,408.07710302)(434.41347778,408.05710297)
\curveto(434.31347163,408.03710306)(434.20847173,408.01710308)(434.09847778,407.99710297)
\curveto(434.0384719,407.97710312)(433.97347197,407.96710313)(433.90347778,407.96710297)
\curveto(433.8434721,407.96710313)(433.77847216,407.95710314)(433.70847778,407.93710297)
\lineto(433.57347778,407.93710297)
\curveto(433.49347245,407.91710318)(433.41847252,407.91710318)(433.34847778,407.93710297)
\lineto(433.19847778,407.93710297)
\curveto(433.1384728,407.95710314)(433.07347287,407.96710313)(433.00347778,407.96710297)
\curveto(432.943473,407.95710314)(432.88347306,407.96210314)(432.82347778,407.98210297)
\curveto(432.66347328,408.03210307)(432.50847343,408.07710302)(432.35847778,408.11710297)
\curveto(432.21847372,408.15710294)(432.08847385,408.21710288)(431.96847778,408.29710297)
\curveto(431.89847404,408.33710276)(431.83347411,408.37710272)(431.77347778,408.41710297)
\curveto(431.71347423,408.46710263)(431.64847429,408.51710258)(431.57847778,408.56710297)
\lineto(431.39847778,408.70210297)
\curveto(431.31847462,408.76210234)(431.24847469,408.76710233)(431.18847778,408.71710297)
\curveto(431.1384748,408.68710241)(431.11347483,408.64710245)(431.11347778,408.59710297)
\curveto(431.11347483,408.55710254)(431.10347484,408.50710259)(431.08347778,408.44710297)
\curveto(431.06347488,408.34710275)(431.05347489,408.23210287)(431.05347778,408.10210297)
\curveto(431.06347488,407.97210313)(431.06847487,407.85210325)(431.06847778,407.74210297)
\lineto(431.06847778,406.21210297)
\curveto(431.06847487,406.08210502)(431.06347488,405.95710514)(431.05347778,405.83710297)
\curveto(431.05347489,405.70710539)(431.02847491,405.6021055)(430.97847778,405.52210297)
\curveto(430.94847499,405.48210562)(430.89347505,405.45210565)(430.81347778,405.43210297)
\curveto(430.73347521,405.41210569)(430.6434753,405.4021057)(430.54347778,405.40210297)
\curveto(430.4434755,405.39210571)(430.3434756,405.39210571)(430.24347778,405.40210297)
\lineto(429.98847778,405.40210297)
\lineto(429.58347778,405.40210297)
\lineto(429.47847778,405.40210297)
\curveto(429.4384765,405.4021057)(429.40347654,405.40710569)(429.37347778,405.41710297)
\lineto(429.25347778,405.41710297)
\curveto(429.08347686,405.46710563)(428.99347695,405.56710553)(428.98347778,405.71710297)
\curveto(428.97347697,405.85710524)(428.96847697,406.02710507)(428.96847778,406.22710297)
\lineto(428.96847778,415.03210297)
\curveto(428.96847697,415.14209596)(428.96347698,415.25709584)(428.95347778,415.37710297)
\curveto(428.95347699,415.50709559)(428.97847696,415.60709549)(429.02847778,415.67710297)
\curveto(429.06847687,415.74709535)(429.12347682,415.79209531)(429.19347778,415.81210297)
\curveto(429.2434767,415.83209527)(429.30347664,415.84209526)(429.37347778,415.84210297)
\lineto(429.59847778,415.84210297)
\lineto(430.31847778,415.84210297)
\lineto(430.60347778,415.84210297)
\curveto(430.69347525,415.84209526)(430.76847517,415.81709528)(430.82847778,415.76710297)
\curveto(430.89847504,415.71709538)(430.93347501,415.65209545)(430.93347778,415.57210297)
\curveto(430.943475,415.5020956)(430.96847497,415.42709567)(431.00847778,415.34710297)
\curveto(431.01847492,415.31709578)(431.02847491,415.29209581)(431.03847778,415.27210297)
\curveto(431.05847488,415.26209584)(431.07847486,415.24709585)(431.09847778,415.22710297)
\curveto(431.20847473,415.21709588)(431.29847464,415.24709585)(431.36847778,415.31710297)
\curveto(431.4384745,415.38709571)(431.50847443,415.44709565)(431.57847778,415.49710297)
\curveto(431.70847423,415.58709551)(431.8434741,415.66709543)(431.98347778,415.73710297)
\curveto(432.12347382,415.81709528)(432.27847366,415.88209522)(432.44847778,415.93210297)
\curveto(432.52847341,415.96209514)(432.61347333,415.98209512)(432.70347778,415.99210297)
\curveto(432.80347314,416.0020951)(432.89847304,416.01709508)(432.98847778,416.03710297)
\curveto(433.02847291,416.04709505)(433.06847287,416.04709505)(433.10847778,416.03710297)
\curveto(433.15847278,416.02709507)(433.19847274,416.03209507)(433.22847778,416.05210297)
\curveto(433.79847214,416.07209503)(434.27847166,415.99209511)(434.66847778,415.81210297)
\curveto(435.06847087,415.64209546)(435.40847053,415.41709568)(435.68847778,415.13710297)
\curveto(435.7384702,415.08709601)(435.78347016,415.03709606)(435.82347778,414.98710297)
\curveto(435.86347008,414.94709615)(435.90347004,414.9020962)(435.94347778,414.85210297)
\curveto(436.01346993,414.76209634)(436.07346987,414.67209643)(436.12347778,414.58210297)
\curveto(436.18346976,414.49209661)(436.2384697,414.4020967)(436.28847778,414.31210297)
\curveto(436.30846963,414.29209681)(436.31846962,414.26709683)(436.31847778,414.23710297)
\curveto(436.32846961,414.20709689)(436.3434696,414.17209693)(436.36347778,414.13210297)
\curveto(436.42346952,414.03209707)(436.47846946,413.91209719)(436.52847778,413.77210297)
\curveto(436.54846939,413.71209739)(436.56846937,413.64709745)(436.58847778,413.57710297)
\curveto(436.60846933,413.51709758)(436.62846931,413.45209765)(436.64847778,413.38210297)
\curveto(436.68846925,413.26209784)(436.71346923,413.13709796)(436.72347778,413.00710297)
\curveto(436.7434692,412.87709822)(436.76846917,412.74209836)(436.79847778,412.60210297)
\lineto(436.79847778,412.43710297)
\lineto(436.82847778,412.25710297)
\lineto(436.82847778,412.09210297)
\moveto(434.71347778,411.74710297)
\curveto(434.72347122,411.7970993)(434.72847121,411.86209924)(434.72847778,411.94210297)
\curveto(434.72847121,412.03209907)(434.72347122,412.102099)(434.71347778,412.15210297)
\lineto(434.71347778,412.28710297)
\curveto(434.69347125,412.34709875)(434.68347126,412.41209869)(434.68347778,412.48210297)
\curveto(434.68347126,412.55209855)(434.67347127,412.62209848)(434.65347778,412.69210297)
\curveto(434.63347131,412.79209831)(434.61347133,412.88709821)(434.59347778,412.97710297)
\curveto(434.57347137,413.07709802)(434.5434714,413.16709793)(434.50347778,413.24710297)
\curveto(434.38347156,413.56709753)(434.22847171,413.82209728)(434.03847778,414.01210297)
\curveto(433.84847209,414.2020969)(433.57847236,414.34209676)(433.22847778,414.43210297)
\curveto(433.14847279,414.45209665)(433.05847288,414.46209664)(432.95847778,414.46210297)
\lineto(432.68847778,414.46210297)
\curveto(432.64847329,414.45209665)(432.61347333,414.44709665)(432.58347778,414.44710297)
\curveto(432.55347339,414.44709665)(432.51847342,414.44209666)(432.47847778,414.43210297)
\lineto(432.26847778,414.37210297)
\curveto(432.20847373,414.36209674)(432.14847379,414.34209676)(432.08847778,414.31210297)
\curveto(431.82847411,414.2020969)(431.62347432,414.03209707)(431.47347778,413.80210297)
\curveto(431.33347461,413.57209753)(431.21847472,413.31709778)(431.12847778,413.03710297)
\curveto(431.10847483,412.95709814)(431.09347485,412.87209823)(431.08347778,412.78210297)
\curveto(431.07347487,412.7020984)(431.05847488,412.62209848)(431.03847778,412.54210297)
\curveto(431.02847491,412.5020986)(431.02347492,412.43709866)(431.02347778,412.34710297)
\curveto(431.00347494,412.30709879)(430.99847494,412.25709884)(431.00847778,412.19710297)
\curveto(431.01847492,412.14709895)(431.01847492,412.097099)(431.00847778,412.04710297)
\curveto(430.98847495,411.98709911)(430.98847495,411.93209917)(431.00847778,411.88210297)
\lineto(431.00847778,411.70210297)
\lineto(431.00847778,411.56710297)
\curveto(431.00847493,411.52709957)(431.01847492,411.48709961)(431.03847778,411.44710297)
\curveto(431.0384749,411.37709972)(431.0434749,411.32209978)(431.05347778,411.28210297)
\lineto(431.08347778,411.10210297)
\curveto(431.09347485,411.04210006)(431.10847483,410.98210012)(431.12847778,410.92210297)
\curveto(431.21847472,410.63210047)(431.32347462,410.39210071)(431.44347778,410.20210297)
\curveto(431.57347437,410.02210108)(431.75347419,409.86210124)(431.98347778,409.72210297)
\curveto(432.12347382,409.64210146)(432.28847365,409.57710152)(432.47847778,409.52710297)
\curveto(432.51847342,409.51710158)(432.55347339,409.51210159)(432.58347778,409.51210297)
\curveto(432.61347333,409.52210158)(432.64847329,409.52210158)(432.68847778,409.51210297)
\curveto(432.72847321,409.5021016)(432.78847315,409.49210161)(432.86847778,409.48210297)
\curveto(432.94847299,409.48210162)(433.01347293,409.48710161)(433.06347778,409.49710297)
\curveto(433.1434728,409.51710158)(433.22347272,409.53210157)(433.30347778,409.54210297)
\curveto(433.39347255,409.56210154)(433.47847246,409.58710151)(433.55847778,409.61710297)
\curveto(433.79847214,409.71710138)(433.99347195,409.85710124)(434.14347778,410.03710297)
\curveto(434.29347165,410.21710088)(434.41847152,410.42710067)(434.51847778,410.66710297)
\curveto(434.56847137,410.78710031)(434.60347134,410.91210019)(434.62347778,411.04210297)
\curveto(434.6434713,411.17209993)(434.66847127,411.30709979)(434.69847778,411.44710297)
\lineto(434.69847778,411.59710297)
\curveto(434.70847123,411.64709945)(434.71347123,411.6970994)(434.71347778,411.74710297)
}
}
{
\newrgbcolor{curcolor}{0 0 0}
\pscustom[linestyle=none,fillstyle=solid,fillcolor=curcolor]
{
\newpath
\moveto(445.15839966,408.73210297)
\curveto(445.17839181,408.62210248)(445.1883918,408.51210259)(445.18839966,408.40210297)
\curveto(445.19839179,408.29210281)(445.14839184,408.21710288)(445.03839966,408.17710297)
\curveto(444.97839201,408.14710295)(444.90839208,408.13210297)(444.82839966,408.13210297)
\lineto(444.58839966,408.13210297)
\lineto(443.77839966,408.13210297)
\lineto(443.50839966,408.13210297)
\curveto(443.42839356,408.14210296)(443.36339362,408.16710293)(443.31339966,408.20710297)
\curveto(443.24339374,408.24710285)(443.1883938,408.3021028)(443.14839966,408.37210297)
\curveto(443.11839387,408.45210265)(443.07339391,408.51710258)(443.01339966,408.56710297)
\curveto(442.99339399,408.58710251)(442.96839402,408.6021025)(442.93839966,408.61210297)
\curveto(442.90839408,408.63210247)(442.86839412,408.63710246)(442.81839966,408.62710297)
\curveto(442.76839422,408.60710249)(442.71839427,408.58210252)(442.66839966,408.55210297)
\curveto(442.62839436,408.52210258)(442.5833944,408.4971026)(442.53339966,408.47710297)
\curveto(442.4833945,408.43710266)(442.42839456,408.4021027)(442.36839966,408.37210297)
\lineto(442.18839966,408.28210297)
\curveto(442.05839493,408.22210288)(441.92339506,408.17210293)(441.78339966,408.13210297)
\curveto(441.64339534,408.102103)(441.49839549,408.06710303)(441.34839966,408.02710297)
\curveto(441.27839571,408.00710309)(441.20839578,407.9971031)(441.13839966,407.99710297)
\curveto(441.07839591,407.98710311)(441.01339597,407.97710312)(440.94339966,407.96710297)
\lineto(440.85339966,407.96710297)
\curveto(440.82339616,407.95710314)(440.79339619,407.95210315)(440.76339966,407.95210297)
\lineto(440.59839966,407.95210297)
\curveto(440.49839649,407.93210317)(440.39839659,407.93210317)(440.29839966,407.95210297)
\lineto(440.16339966,407.95210297)
\curveto(440.09339689,407.97210313)(440.02339696,407.98210312)(439.95339966,407.98210297)
\curveto(439.89339709,407.97210313)(439.83339715,407.97710312)(439.77339966,407.99710297)
\curveto(439.67339731,408.01710308)(439.57839741,408.03710306)(439.48839966,408.05710297)
\curveto(439.39839759,408.06710303)(439.31339767,408.09210301)(439.23339966,408.13210297)
\curveto(438.94339804,408.24210286)(438.69339829,408.38210272)(438.48339966,408.55210297)
\curveto(438.2833987,408.73210237)(438.12339886,408.96710213)(438.00339966,409.25710297)
\curveto(437.97339901,409.32710177)(437.94339904,409.4021017)(437.91339966,409.48210297)
\curveto(437.89339909,409.56210154)(437.87339911,409.64710145)(437.85339966,409.73710297)
\curveto(437.83339915,409.78710131)(437.82339916,409.83710126)(437.82339966,409.88710297)
\curveto(437.83339915,409.93710116)(437.83339915,409.98710111)(437.82339966,410.03710297)
\curveto(437.81339917,410.06710103)(437.80339918,410.12710097)(437.79339966,410.21710297)
\curveto(437.79339919,410.31710078)(437.79839919,410.38710071)(437.80839966,410.42710297)
\curveto(437.82839916,410.52710057)(437.83839915,410.61210049)(437.83839966,410.68210297)
\lineto(437.92839966,411.01210297)
\curveto(437.95839903,411.13209997)(437.99839899,411.23709986)(438.04839966,411.32710297)
\curveto(438.21839877,411.61709948)(438.41339857,411.83709926)(438.63339966,411.98710297)
\curveto(438.85339813,412.13709896)(439.13339785,412.26709883)(439.47339966,412.37710297)
\curveto(439.60339738,412.42709867)(439.73839725,412.46209864)(439.87839966,412.48210297)
\curveto(440.01839697,412.5020986)(440.15839683,412.52709857)(440.29839966,412.55710297)
\curveto(440.37839661,412.57709852)(440.46339652,412.58709851)(440.55339966,412.58710297)
\curveto(440.64339634,412.5970985)(440.73339625,412.61209849)(440.82339966,412.63210297)
\curveto(440.89339609,412.65209845)(440.96339602,412.65709844)(441.03339966,412.64710297)
\curveto(441.10339588,412.64709845)(441.17839581,412.65709844)(441.25839966,412.67710297)
\curveto(441.32839566,412.6970984)(441.39839559,412.70709839)(441.46839966,412.70710297)
\curveto(441.53839545,412.70709839)(441.61339537,412.71709838)(441.69339966,412.73710297)
\curveto(441.90339508,412.78709831)(442.09339489,412.82709827)(442.26339966,412.85710297)
\curveto(442.44339454,412.8970982)(442.60339438,412.98709811)(442.74339966,413.12710297)
\curveto(442.83339415,413.21709788)(442.89339409,413.31709778)(442.92339966,413.42710297)
\curveto(442.93339405,413.45709764)(442.93339405,413.48209762)(442.92339966,413.50210297)
\curveto(442.92339406,413.52209758)(442.92839406,413.54209756)(442.93839966,413.56210297)
\curveto(442.94839404,413.58209752)(442.95339403,413.61209749)(442.95339966,413.65210297)
\lineto(442.95339966,413.74210297)
\lineto(442.92339966,413.86210297)
\curveto(442.92339406,413.9020972)(442.91839407,413.93709716)(442.90839966,413.96710297)
\curveto(442.80839418,414.26709683)(442.59839439,414.47209663)(442.27839966,414.58210297)
\curveto(442.1883948,414.61209649)(442.07839491,414.63209647)(441.94839966,414.64210297)
\curveto(441.82839516,414.66209644)(441.70339528,414.66709643)(441.57339966,414.65710297)
\curveto(441.44339554,414.65709644)(441.31839567,414.64709645)(441.19839966,414.62710297)
\curveto(441.07839591,414.60709649)(440.97339601,414.58209652)(440.88339966,414.55210297)
\curveto(440.82339616,414.53209657)(440.76339622,414.5020966)(440.70339966,414.46210297)
\curveto(440.65339633,414.43209667)(440.60339638,414.3970967)(440.55339966,414.35710297)
\curveto(440.50339648,414.31709678)(440.44839654,414.26209684)(440.38839966,414.19210297)
\curveto(440.33839665,414.12209698)(440.30339668,414.05709704)(440.28339966,413.99710297)
\curveto(440.23339675,413.8970972)(440.1883968,413.8020973)(440.14839966,413.71210297)
\curveto(440.11839687,413.62209748)(440.04839694,413.56209754)(439.93839966,413.53210297)
\curveto(439.85839713,413.51209759)(439.77339721,413.5020976)(439.68339966,413.50210297)
\lineto(439.41339966,413.50210297)
\lineto(438.84339966,413.50210297)
\curveto(438.79339819,413.5020976)(438.74339824,413.4970976)(438.69339966,413.48710297)
\curveto(438.64339834,413.48709761)(438.59839839,413.49209761)(438.55839966,413.50210297)
\lineto(438.42339966,413.50210297)
\curveto(438.40339858,413.51209759)(438.37839861,413.51709758)(438.34839966,413.51710297)
\curveto(438.31839867,413.51709758)(438.29339869,413.52709757)(438.27339966,413.54710297)
\curveto(438.19339879,413.56709753)(438.13839885,413.63209747)(438.10839966,413.74210297)
\curveto(438.09839889,413.79209731)(438.09839889,413.84209726)(438.10839966,413.89210297)
\curveto(438.11839887,413.94209716)(438.12839886,413.98709711)(438.13839966,414.02710297)
\curveto(438.16839882,414.13709696)(438.19839879,414.23709686)(438.22839966,414.32710297)
\curveto(438.26839872,414.42709667)(438.31339867,414.51709658)(438.36339966,414.59710297)
\lineto(438.45339966,414.74710297)
\lineto(438.54339966,414.89710297)
\curveto(438.62339836,415.00709609)(438.72339826,415.11209599)(438.84339966,415.21210297)
\curveto(438.86339812,415.22209588)(438.89339809,415.24709585)(438.93339966,415.28710297)
\curveto(438.983398,415.32709577)(439.02839796,415.36209574)(439.06839966,415.39210297)
\curveto(439.10839788,415.42209568)(439.15339783,415.45209565)(439.20339966,415.48210297)
\curveto(439.37339761,415.59209551)(439.55339743,415.67709542)(439.74339966,415.73710297)
\curveto(439.93339705,415.80709529)(440.12839686,415.87209523)(440.32839966,415.93210297)
\curveto(440.44839654,415.96209514)(440.57339641,415.98209512)(440.70339966,415.99210297)
\curveto(440.83339615,416.0020951)(440.96339602,416.02209508)(441.09339966,416.05210297)
\curveto(441.13339585,416.06209504)(441.19339579,416.06209504)(441.27339966,416.05210297)
\curveto(441.36339562,416.04209506)(441.41839557,416.04709505)(441.43839966,416.06710297)
\curveto(441.84839514,416.07709502)(442.23839475,416.06209504)(442.60839966,416.02210297)
\curveto(442.988394,415.98209512)(443.32839366,415.90709519)(443.62839966,415.79710297)
\curveto(443.93839305,415.68709541)(444.20339278,415.53709556)(444.42339966,415.34710297)
\curveto(444.64339234,415.16709593)(444.81339217,414.93209617)(444.93339966,414.64210297)
\curveto(445.00339198,414.47209663)(445.04339194,414.27709682)(445.05339966,414.05710297)
\curveto(445.06339192,413.83709726)(445.06839192,413.61209749)(445.06839966,413.38210297)
\lineto(445.06839966,410.03710297)
\lineto(445.06839966,409.45210297)
\curveto(445.06839192,409.26210184)(445.0883919,409.08710201)(445.12839966,408.92710297)
\curveto(445.13839185,408.8971022)(445.14339184,408.86210224)(445.14339966,408.82210297)
\curveto(445.14339184,408.79210231)(445.14839184,408.76210234)(445.15839966,408.73210297)
\moveto(442.95339966,411.04210297)
\curveto(442.96339402,411.09210001)(442.96839402,411.14709995)(442.96839966,411.20710297)
\curveto(442.96839402,411.27709982)(442.96339402,411.33709976)(442.95339966,411.38710297)
\curveto(442.93339405,411.44709965)(442.92339406,411.5020996)(442.92339966,411.55210297)
\curveto(442.92339406,411.6020995)(442.90339408,411.64209946)(442.86339966,411.67210297)
\curveto(442.81339417,411.71209939)(442.73839425,411.73209937)(442.63839966,411.73210297)
\curveto(442.59839439,411.72209938)(442.56339442,411.71209939)(442.53339966,411.70210297)
\curveto(442.50339448,411.7020994)(442.46839452,411.6970994)(442.42839966,411.68710297)
\curveto(442.35839463,411.66709943)(442.2833947,411.65209945)(442.20339966,411.64210297)
\curveto(442.12339486,411.63209947)(442.04339494,411.61709948)(441.96339966,411.59710297)
\curveto(441.93339505,411.58709951)(441.8883951,411.58209952)(441.82839966,411.58210297)
\curveto(441.69839529,411.55209955)(441.56839542,411.53209957)(441.43839966,411.52210297)
\curveto(441.30839568,411.51209959)(441.1833958,411.48709961)(441.06339966,411.44710297)
\curveto(440.983396,411.42709967)(440.90839608,411.40709969)(440.83839966,411.38710297)
\curveto(440.76839622,411.37709972)(440.69839629,411.35709974)(440.62839966,411.32710297)
\curveto(440.41839657,411.23709986)(440.23839675,411.1021)(440.08839966,410.92210297)
\curveto(439.94839704,410.74210036)(439.89839709,410.49210061)(439.93839966,410.17210297)
\curveto(439.95839703,410.0021011)(440.01339697,409.86210124)(440.10339966,409.75210297)
\curveto(440.17339681,409.64210146)(440.27839671,409.55210155)(440.41839966,409.48210297)
\curveto(440.55839643,409.42210168)(440.70839628,409.37710172)(440.86839966,409.34710297)
\curveto(441.03839595,409.31710178)(441.21339577,409.30710179)(441.39339966,409.31710297)
\curveto(441.5833954,409.33710176)(441.75839523,409.37210173)(441.91839966,409.42210297)
\curveto(442.17839481,409.5021016)(442.3833946,409.62710147)(442.53339966,409.79710297)
\curveto(442.6833943,409.97710112)(442.79839419,410.1971009)(442.87839966,410.45710297)
\curveto(442.89839409,410.52710057)(442.90839408,410.5971005)(442.90839966,410.66710297)
\curveto(442.91839407,410.74710035)(442.93339405,410.82710027)(442.95339966,410.90710297)
\lineto(442.95339966,411.04210297)
}
}
{
\newrgbcolor{curcolor}{0 0 0}
\pscustom[linestyle=none,fillstyle=solid,fillcolor=curcolor]
{
\newpath
\moveto(450.29168091,416.08210297)
\curveto(451.10167575,416.102095)(451.77667507,415.98209512)(452.31668091,415.72210297)
\curveto(452.86667398,415.46209564)(453.30167355,415.09209601)(453.62168091,414.61210297)
\curveto(453.78167307,414.37209673)(453.90167295,414.097097)(453.98168091,413.78710297)
\curveto(454.00167285,413.73709736)(454.01667283,413.67209743)(454.02668091,413.59210297)
\curveto(454.0466728,413.51209759)(454.0466728,413.44209766)(454.02668091,413.38210297)
\curveto(453.98667286,413.27209783)(453.91667293,413.20709789)(453.81668091,413.18710297)
\curveto(453.71667313,413.17709792)(453.59667325,413.17209793)(453.45668091,413.17210297)
\lineto(452.67668091,413.17210297)
\lineto(452.39168091,413.17210297)
\curveto(452.30167455,413.17209793)(452.22667462,413.19209791)(452.16668091,413.23210297)
\curveto(452.08667476,413.27209783)(452.03167482,413.33209777)(452.00168091,413.41210297)
\curveto(451.97167488,413.5020976)(451.93167492,413.59209751)(451.88168091,413.68210297)
\curveto(451.82167503,413.79209731)(451.75667509,413.89209721)(451.68668091,413.98210297)
\curveto(451.61667523,414.07209703)(451.53667531,414.15209695)(451.44668091,414.22210297)
\curveto(451.30667554,414.31209679)(451.1516757,414.38209672)(450.98168091,414.43210297)
\curveto(450.92167593,414.45209665)(450.86167599,414.46209664)(450.80168091,414.46210297)
\curveto(450.74167611,414.46209664)(450.68667616,414.47209663)(450.63668091,414.49210297)
\lineto(450.48668091,414.49210297)
\curveto(450.28667656,414.49209661)(450.12667672,414.47209663)(450.00668091,414.43210297)
\curveto(449.71667713,414.34209676)(449.48167737,414.2020969)(449.30168091,414.01210297)
\curveto(449.12167773,413.83209727)(448.97667787,413.61209749)(448.86668091,413.35210297)
\curveto(448.81667803,413.24209786)(448.77667807,413.12209798)(448.74668091,412.99210297)
\curveto(448.72667812,412.87209823)(448.70167815,412.74209836)(448.67168091,412.60210297)
\curveto(448.66167819,412.56209854)(448.65667819,412.52209858)(448.65668091,412.48210297)
\curveto(448.65667819,412.44209866)(448.6516782,412.4020987)(448.64168091,412.36210297)
\curveto(448.62167823,412.26209884)(448.61167824,412.12209898)(448.61168091,411.94210297)
\curveto(448.62167823,411.76209934)(448.63667821,411.62209948)(448.65668091,411.52210297)
\curveto(448.65667819,411.44209966)(448.66167819,411.38709971)(448.67168091,411.35710297)
\curveto(448.69167816,411.28709981)(448.70167815,411.21709988)(448.70168091,411.14710297)
\curveto(448.71167814,411.07710002)(448.72667812,411.00710009)(448.74668091,410.93710297)
\curveto(448.82667802,410.70710039)(448.92167793,410.4971006)(449.03168091,410.30710297)
\curveto(449.14167771,410.11710098)(449.28167757,409.95710114)(449.45168091,409.82710297)
\curveto(449.49167736,409.7971013)(449.5516773,409.76210134)(449.63168091,409.72210297)
\curveto(449.74167711,409.65210145)(449.851677,409.60710149)(449.96168091,409.58710297)
\curveto(450.08167677,409.56710153)(450.22667662,409.54710155)(450.39668091,409.52710297)
\lineto(450.48668091,409.52710297)
\curveto(450.52667632,409.52710157)(450.55667629,409.53210157)(450.57668091,409.54210297)
\lineto(450.71168091,409.54210297)
\curveto(450.78167607,409.56210154)(450.846676,409.57710152)(450.90668091,409.58710297)
\curveto(450.97667587,409.60710149)(451.04167581,409.62710147)(451.10168091,409.64710297)
\curveto(451.40167545,409.77710132)(451.63167522,409.96710113)(451.79168091,410.21710297)
\curveto(451.83167502,410.26710083)(451.86667498,410.32210078)(451.89668091,410.38210297)
\curveto(451.92667492,410.45210065)(451.9516749,410.51210059)(451.97168091,410.56210297)
\curveto(452.01167484,410.67210043)(452.0466748,410.76710033)(452.07668091,410.84710297)
\curveto(452.10667474,410.93710016)(452.17667467,411.00710009)(452.28668091,411.05710297)
\curveto(452.37667447,411.0971)(452.52167433,411.11209999)(452.72168091,411.10210297)
\lineto(453.21668091,411.10210297)
\lineto(453.42668091,411.10210297)
\curveto(453.50667334,411.11209999)(453.57167328,411.10709999)(453.62168091,411.08710297)
\lineto(453.74168091,411.08710297)
\lineto(453.86168091,411.05710297)
\curveto(453.90167295,411.05710004)(453.93167292,411.04710005)(453.95168091,411.02710297)
\curveto(454.00167285,410.98710011)(454.03167282,410.92710017)(454.04168091,410.84710297)
\curveto(454.06167279,410.77710032)(454.06167279,410.7021004)(454.04168091,410.62210297)
\curveto(453.9516729,410.29210081)(453.84167301,409.9971011)(453.71168091,409.73710297)
\curveto(453.30167355,408.96710213)(452.6466742,408.43210267)(451.74668091,408.13210297)
\curveto(451.6466752,408.102103)(451.54167531,408.08210302)(451.43168091,408.07210297)
\curveto(451.32167553,408.05210305)(451.21167564,408.02710307)(451.10168091,407.99710297)
\curveto(451.04167581,407.98710311)(450.98167587,407.98210312)(450.92168091,407.98210297)
\curveto(450.86167599,407.98210312)(450.80167605,407.97710312)(450.74168091,407.96710297)
\lineto(450.57668091,407.96710297)
\curveto(450.52667632,407.94710315)(450.4516764,407.94210316)(450.35168091,407.95210297)
\curveto(450.2516766,407.95210315)(450.17667667,407.95710314)(450.12668091,407.96710297)
\curveto(450.0466768,407.98710311)(449.97167688,407.9971031)(449.90168091,407.99710297)
\curveto(449.84167701,407.98710311)(449.77667707,407.99210311)(449.70668091,408.01210297)
\lineto(449.55668091,408.04210297)
\curveto(449.50667734,408.04210306)(449.45667739,408.04710305)(449.40668091,408.05710297)
\curveto(449.29667755,408.08710301)(449.19167766,408.11710298)(449.09168091,408.14710297)
\curveto(448.99167786,408.17710292)(448.89667795,408.21210289)(448.80668091,408.25210297)
\curveto(448.33667851,408.45210265)(447.94167891,408.70710239)(447.62168091,409.01710297)
\curveto(447.30167955,409.33710176)(447.04167981,409.73210137)(446.84168091,410.20210297)
\curveto(446.79168006,410.29210081)(446.7516801,410.38710071)(446.72168091,410.48710297)
\lineto(446.63168091,410.81710297)
\curveto(446.62168023,410.85710024)(446.61668023,410.89210021)(446.61668091,410.92210297)
\curveto(446.61668023,410.96210014)(446.60668024,411.00710009)(446.58668091,411.05710297)
\curveto(446.56668028,411.12709997)(446.55668029,411.1970999)(446.55668091,411.26710297)
\curveto(446.55668029,411.34709975)(446.5466803,411.42209968)(446.52668091,411.49210297)
\lineto(446.52668091,411.74710297)
\curveto(446.50668034,411.7970993)(446.49668035,411.85209925)(446.49668091,411.91210297)
\curveto(446.49668035,411.98209912)(446.50668034,412.04209906)(446.52668091,412.09210297)
\curveto(446.53668031,412.14209896)(446.53668031,412.18709891)(446.52668091,412.22710297)
\curveto(446.51668033,412.26709883)(446.51668033,412.30709879)(446.52668091,412.34710297)
\curveto(446.5466803,412.41709868)(446.5516803,412.48209862)(446.54168091,412.54210297)
\curveto(446.54168031,412.6020985)(446.5516803,412.66209844)(446.57168091,412.72210297)
\curveto(446.62168023,412.9020982)(446.66168019,413.07209803)(446.69168091,413.23210297)
\curveto(446.72168013,413.4020977)(446.76668008,413.56709753)(446.82668091,413.72710297)
\curveto(447.0466798,414.23709686)(447.32167953,414.66209644)(447.65168091,415.00210297)
\curveto(447.99167886,415.34209576)(448.42167843,415.61709548)(448.94168091,415.82710297)
\curveto(449.08167777,415.88709521)(449.22667762,415.92709517)(449.37668091,415.94710297)
\curveto(449.52667732,415.97709512)(449.68167717,416.01209509)(449.84168091,416.05210297)
\curveto(449.92167693,416.06209504)(449.99667685,416.06709503)(450.06668091,416.06710297)
\curveto(450.13667671,416.06709503)(450.21167664,416.07209503)(450.29168091,416.08210297)
}
}
{
\newrgbcolor{curcolor}{0 0 0}
\pscustom[linestyle=none,fillstyle=solid,fillcolor=curcolor]
{
\newpath
\moveto(457.43496216,418.72210297)
\curveto(457.50495921,418.64209246)(457.53995917,418.52209258)(457.53996216,418.36210297)
\lineto(457.53996216,417.89710297)
\lineto(457.53996216,417.49210297)
\curveto(457.53995917,417.35209375)(457.50495921,417.25709384)(457.43496216,417.20710297)
\curveto(457.37495934,417.15709394)(457.29495942,417.12709397)(457.19496216,417.11710297)
\curveto(457.10495961,417.10709399)(457.00495971,417.102094)(456.89496216,417.10210297)
\lineto(456.05496216,417.10210297)
\curveto(455.94496077,417.102094)(455.84496087,417.10709399)(455.75496216,417.11710297)
\curveto(455.67496104,417.12709397)(455.60496111,417.15709394)(455.54496216,417.20710297)
\curveto(455.50496121,417.23709386)(455.47496124,417.29209381)(455.45496216,417.37210297)
\curveto(455.44496127,417.46209364)(455.43496128,417.55709354)(455.42496216,417.65710297)
\lineto(455.42496216,417.98710297)
\curveto(455.43496128,418.097093)(455.43996127,418.19209291)(455.43996216,418.27210297)
\lineto(455.43996216,418.48210297)
\curveto(455.44996126,418.55209255)(455.46996124,418.61209249)(455.49996216,418.66210297)
\curveto(455.51996119,418.7020924)(455.54496117,418.73209237)(455.57496216,418.75210297)
\lineto(455.69496216,418.81210297)
\curveto(455.714961,418.81209229)(455.73996097,418.81209229)(455.76996216,418.81210297)
\curveto(455.79996091,418.82209228)(455.82496089,418.82709227)(455.84496216,418.82710297)
\lineto(456.93996216,418.82710297)
\curveto(457.03995967,418.82709227)(457.13495958,418.82209228)(457.22496216,418.81210297)
\curveto(457.3149594,418.8020923)(457.38495933,418.77209233)(457.43496216,418.72210297)
\moveto(457.53996216,408.95710297)
\curveto(457.53995917,408.75710234)(457.53495918,408.58710251)(457.52496216,408.44710297)
\curveto(457.5149592,408.30710279)(457.42495929,408.21210289)(457.25496216,408.16210297)
\curveto(457.19495952,408.14210296)(457.12995958,408.13210297)(457.05996216,408.13210297)
\curveto(456.98995972,408.14210296)(456.9149598,408.14710295)(456.83496216,408.14710297)
\lineto(455.99496216,408.14710297)
\curveto(455.90496081,408.14710295)(455.8149609,408.15210295)(455.72496216,408.16210297)
\curveto(455.64496107,408.17210293)(455.58496113,408.2021029)(455.54496216,408.25210297)
\curveto(455.48496123,408.32210278)(455.44996126,408.40710269)(455.43996216,408.50710297)
\lineto(455.43996216,408.85210297)
\lineto(455.43996216,415.18210297)
\lineto(455.43996216,415.48210297)
\curveto(455.43996127,415.58209552)(455.45996125,415.66209544)(455.49996216,415.72210297)
\curveto(455.55996115,415.79209531)(455.64496107,415.83709526)(455.75496216,415.85710297)
\curveto(455.77496094,415.86709523)(455.79996091,415.86709523)(455.82996216,415.85710297)
\curveto(455.86996084,415.85709524)(455.89996081,415.86209524)(455.91996216,415.87210297)
\lineto(456.66996216,415.87210297)
\lineto(456.86496216,415.87210297)
\curveto(456.94495977,415.88209522)(457.0099597,415.88209522)(457.05996216,415.87210297)
\lineto(457.17996216,415.87210297)
\curveto(457.23995947,415.85209525)(457.29495942,415.83709526)(457.34496216,415.82710297)
\curveto(457.39495932,415.81709528)(457.43495928,415.78709531)(457.46496216,415.73710297)
\curveto(457.50495921,415.68709541)(457.52495919,415.61709548)(457.52496216,415.52710297)
\curveto(457.53495918,415.43709566)(457.53995917,415.34209576)(457.53996216,415.24210297)
\lineto(457.53996216,408.95710297)
}
}
{
\newrgbcolor{curcolor}{0 0 0}
\pscustom[linestyle=none,fillstyle=solid,fillcolor=curcolor]
{
\newpath
\moveto(466.97214966,412.31710297)
\curveto(466.99214109,412.25709884)(467.00214108,412.17209893)(467.00214966,412.06210297)
\curveto(467.00214108,411.95209915)(466.99214109,411.86709923)(466.97214966,411.80710297)
\lineto(466.97214966,411.65710297)
\curveto(466.95214113,411.57709952)(466.94214114,411.4970996)(466.94214966,411.41710297)
\curveto(466.95214113,411.33709976)(466.94714113,411.25709984)(466.92714966,411.17710297)
\curveto(466.90714117,411.10709999)(466.89214119,411.04210006)(466.88214966,410.98210297)
\curveto(466.87214121,410.92210018)(466.86214122,410.85710024)(466.85214966,410.78710297)
\curveto(466.81214127,410.67710042)(466.7771413,410.56210054)(466.74714966,410.44210297)
\curveto(466.71714136,410.33210077)(466.6771414,410.22710087)(466.62714966,410.12710297)
\curveto(466.41714166,409.64710145)(466.14214194,409.25710184)(465.80214966,408.95710297)
\curveto(465.46214262,408.65710244)(465.05214303,408.40710269)(464.57214966,408.20710297)
\curveto(464.45214363,408.15710294)(464.32714375,408.12210298)(464.19714966,408.10210297)
\curveto(464.077144,408.07210303)(463.95214413,408.04210306)(463.82214966,408.01210297)
\curveto(463.77214431,407.99210311)(463.71714436,407.98210312)(463.65714966,407.98210297)
\curveto(463.59714448,407.98210312)(463.54214454,407.97710312)(463.49214966,407.96710297)
\lineto(463.38714966,407.96710297)
\curveto(463.35714472,407.95710314)(463.32714475,407.95210315)(463.29714966,407.95210297)
\curveto(463.24714483,407.94210316)(463.16714491,407.93710316)(463.05714966,407.93710297)
\curveto(462.94714513,407.92710317)(462.86214522,407.93210317)(462.80214966,407.95210297)
\lineto(462.65214966,407.95210297)
\curveto(462.60214548,407.96210314)(462.54714553,407.96710313)(462.48714966,407.96710297)
\curveto(462.43714564,407.95710314)(462.38714569,407.96210314)(462.33714966,407.98210297)
\curveto(462.29714578,407.99210311)(462.25714582,407.9971031)(462.21714966,407.99710297)
\curveto(462.18714589,407.9971031)(462.14714593,408.0021031)(462.09714966,408.01210297)
\curveto(461.99714608,408.04210306)(461.89714618,408.06710303)(461.79714966,408.08710297)
\curveto(461.69714638,408.10710299)(461.60214648,408.13710296)(461.51214966,408.17710297)
\curveto(461.39214669,408.21710288)(461.2771468,408.25710284)(461.16714966,408.29710297)
\curveto(461.06714701,408.33710276)(460.96214712,408.38710271)(460.85214966,408.44710297)
\curveto(460.50214758,408.65710244)(460.20214788,408.9021022)(459.95214966,409.18210297)
\curveto(459.70214838,409.46210164)(459.49214859,409.7971013)(459.32214966,410.18710297)
\curveto(459.27214881,410.27710082)(459.23214885,410.37210073)(459.20214966,410.47210297)
\curveto(459.1821489,410.57210053)(459.15714892,410.67710042)(459.12714966,410.78710297)
\curveto(459.10714897,410.83710026)(459.09714898,410.88210022)(459.09714966,410.92210297)
\curveto(459.09714898,410.96210014)(459.08714899,411.00710009)(459.06714966,411.05710297)
\curveto(459.04714903,411.13709996)(459.03714904,411.21709988)(459.03714966,411.29710297)
\curveto(459.03714904,411.38709971)(459.02714905,411.47209963)(459.00714966,411.55210297)
\curveto(458.99714908,411.6020995)(458.99214909,411.64709945)(458.99214966,411.68710297)
\lineto(458.99214966,411.82210297)
\curveto(458.97214911,411.88209922)(458.96214912,411.96709913)(458.96214966,412.07710297)
\curveto(458.97214911,412.18709891)(458.98714909,412.27209883)(459.00714966,412.33210297)
\lineto(459.00714966,412.43710297)
\curveto(459.01714906,412.48709861)(459.01714906,412.53709856)(459.00714966,412.58710297)
\curveto(459.00714907,412.64709845)(459.01714906,412.7020984)(459.03714966,412.75210297)
\curveto(459.04714903,412.8020983)(459.05214903,412.84709825)(459.05214966,412.88710297)
\curveto(459.05214903,412.93709816)(459.06214902,412.98709811)(459.08214966,413.03710297)
\curveto(459.12214896,413.16709793)(459.15714892,413.29209781)(459.18714966,413.41210297)
\curveto(459.21714886,413.54209756)(459.25714882,413.66709743)(459.30714966,413.78710297)
\curveto(459.48714859,414.1970969)(459.70214838,414.53709656)(459.95214966,414.80710297)
\curveto(460.20214788,415.08709601)(460.50714757,415.34209576)(460.86714966,415.57210297)
\curveto(460.96714711,415.62209548)(461.07214701,415.66709543)(461.18214966,415.70710297)
\curveto(461.29214679,415.74709535)(461.40214668,415.79209531)(461.51214966,415.84210297)
\curveto(461.64214644,415.89209521)(461.7771463,415.92709517)(461.91714966,415.94710297)
\curveto(462.05714602,415.96709513)(462.20214588,415.9970951)(462.35214966,416.03710297)
\curveto(462.43214565,416.04709505)(462.50714557,416.05209505)(462.57714966,416.05210297)
\curveto(462.64714543,416.05209505)(462.71714536,416.05709504)(462.78714966,416.06710297)
\curveto(463.36714471,416.07709502)(463.86714421,416.01709508)(464.28714966,415.88710297)
\curveto(464.71714336,415.75709534)(465.09714298,415.57709552)(465.42714966,415.34710297)
\curveto(465.53714254,415.26709583)(465.64714243,415.17709592)(465.75714966,415.07710297)
\curveto(465.8771422,414.98709611)(465.9771421,414.88709621)(466.05714966,414.77710297)
\curveto(466.13714194,414.67709642)(466.20714187,414.57709652)(466.26714966,414.47710297)
\curveto(466.33714174,414.37709672)(466.40714167,414.27209683)(466.47714966,414.16210297)
\curveto(466.54714153,414.05209705)(466.60214148,413.93209717)(466.64214966,413.80210297)
\curveto(466.6821414,413.68209742)(466.72714135,413.55209755)(466.77714966,413.41210297)
\curveto(466.80714127,413.33209777)(466.83214125,413.24709785)(466.85214966,413.15710297)
\lineto(466.91214966,412.88710297)
\curveto(466.92214116,412.84709825)(466.92714115,412.80709829)(466.92714966,412.76710297)
\curveto(466.92714115,412.72709837)(466.93214115,412.68709841)(466.94214966,412.64710297)
\curveto(466.96214112,412.5970985)(466.96714111,412.54209856)(466.95714966,412.48210297)
\curveto(466.94714113,412.42209868)(466.95214113,412.36709873)(466.97214966,412.31710297)
\moveto(464.87214966,411.77710297)
\curveto(464.8821432,411.82709927)(464.88714319,411.8970992)(464.88714966,411.98710297)
\curveto(464.88714319,412.08709901)(464.8821432,412.16209894)(464.87214966,412.21210297)
\lineto(464.87214966,412.33210297)
\curveto(464.85214323,412.38209872)(464.84214324,412.43709866)(464.84214966,412.49710297)
\curveto(464.84214324,412.55709854)(464.83714324,412.61209849)(464.82714966,412.66210297)
\curveto(464.82714325,412.7020984)(464.82214326,412.73209837)(464.81214966,412.75210297)
\lineto(464.75214966,412.99210297)
\curveto(464.74214334,413.08209802)(464.72214336,413.16709793)(464.69214966,413.24710297)
\curveto(464.5821435,413.50709759)(464.45214363,413.72709737)(464.30214966,413.90710297)
\curveto(464.15214393,414.097097)(463.95214413,414.24709685)(463.70214966,414.35710297)
\curveto(463.64214444,414.37709672)(463.5821445,414.39209671)(463.52214966,414.40210297)
\curveto(463.46214462,414.42209668)(463.39714468,414.44209666)(463.32714966,414.46210297)
\curveto(463.24714483,414.48209662)(463.16214492,414.48709661)(463.07214966,414.47710297)
\lineto(462.80214966,414.47710297)
\curveto(462.77214531,414.45709664)(462.73714534,414.44709665)(462.69714966,414.44710297)
\curveto(462.65714542,414.45709664)(462.62214546,414.45709664)(462.59214966,414.44710297)
\lineto(462.38214966,414.38710297)
\curveto(462.32214576,414.37709672)(462.26714581,414.35709674)(462.21714966,414.32710297)
\curveto(461.96714611,414.21709688)(461.76214632,414.05709704)(461.60214966,413.84710297)
\curveto(461.45214663,413.64709745)(461.33214675,413.41209769)(461.24214966,413.14210297)
\curveto(461.21214687,413.04209806)(461.18714689,412.93709816)(461.16714966,412.82710297)
\curveto(461.15714692,412.71709838)(461.14214694,412.60709849)(461.12214966,412.49710297)
\curveto(461.11214697,412.44709865)(461.10714697,412.3970987)(461.10714966,412.34710297)
\lineto(461.10714966,412.19710297)
\curveto(461.08714699,412.12709897)(461.077147,412.02209908)(461.07714966,411.88210297)
\curveto(461.08714699,411.74209936)(461.10214698,411.63709946)(461.12214966,411.56710297)
\lineto(461.12214966,411.43210297)
\curveto(461.14214694,411.35209975)(461.15714692,411.27209983)(461.16714966,411.19210297)
\curveto(461.1771469,411.12209998)(461.19214689,411.04710005)(461.21214966,410.96710297)
\curveto(461.31214677,410.66710043)(461.41714666,410.42210068)(461.52714966,410.23210297)
\curveto(461.64714643,410.05210105)(461.83214625,409.88710121)(462.08214966,409.73710297)
\curveto(462.15214593,409.68710141)(462.22714585,409.64710145)(462.30714966,409.61710297)
\curveto(462.39714568,409.58710151)(462.48714559,409.56210154)(462.57714966,409.54210297)
\curveto(462.61714546,409.53210157)(462.65214543,409.52710157)(462.68214966,409.52710297)
\curveto(462.71214537,409.53710156)(462.74714533,409.53710156)(462.78714966,409.52710297)
\lineto(462.90714966,409.49710297)
\curveto(462.95714512,409.4971016)(463.00214508,409.5021016)(463.04214966,409.51210297)
\lineto(463.16214966,409.51210297)
\curveto(463.24214484,409.53210157)(463.32214476,409.54710155)(463.40214966,409.55710297)
\curveto(463.4821446,409.56710153)(463.55714452,409.58710151)(463.62714966,409.61710297)
\curveto(463.88714419,409.71710138)(464.09714398,409.85210125)(464.25714966,410.02210297)
\curveto(464.41714366,410.19210091)(464.55214353,410.4021007)(464.66214966,410.65210297)
\curveto(464.70214338,410.75210035)(464.73214335,410.85210025)(464.75214966,410.95210297)
\curveto(464.77214331,411.05210005)(464.79714328,411.15709994)(464.82714966,411.26710297)
\curveto(464.83714324,411.30709979)(464.84214324,411.34209976)(464.84214966,411.37210297)
\curveto(464.84214324,411.41209969)(464.84714323,411.45209965)(464.85714966,411.49210297)
\lineto(464.85714966,411.62710297)
\curveto(464.85714322,411.67709942)(464.86214322,411.72709937)(464.87214966,411.77710297)
}
}
{
\newrgbcolor{curcolor}{0 0 0}
\pscustom[linestyle=none,fillstyle=solid,fillcolor=curcolor]
{
\newpath
\moveto(12.5808255,206.61524658)
\curveto(12.5808362,206.75524277)(12.5808362,206.9202426)(12.5808255,207.11024658)
\curveto(12.57083621,207.31024221)(12.60583617,207.43024209)(12.6858255,207.47024658)
\curveto(12.775836,207.53024199)(12.91083587,207.54024198)(13.0908255,207.50024658)
\curveto(13.27083551,207.46024206)(13.44083534,207.4252421)(13.6008255,207.39524658)
\lineto(15.8508255,206.94524658)
\lineto(18.6408255,206.39024658)
\curveto(18.99082979,206.3202432)(19.33582944,206.25524327)(19.6758255,206.19524658)
\curveto(20.00582877,206.13524339)(20.30082848,206.1252434)(20.5608255,206.16524658)
\curveto(20.95082783,206.21524331)(21.27082751,206.33524319)(21.5208255,206.52524658)
\curveto(21.77082701,206.71524281)(21.9758268,206.98024254)(22.1358255,207.32024658)
\curveto(22.18582659,207.4202421)(22.22082656,207.53024199)(22.2408255,207.65024658)
\curveto(22.25082653,207.77024175)(22.27082651,207.88524164)(22.3008255,207.99524658)
\lineto(22.3308255,208.17524658)
\curveto(22.33082645,208.24524128)(22.33582644,208.30524122)(22.3458255,208.35524658)
\lineto(22.3458255,208.50524658)
\curveto(22.34582643,208.56524096)(22.35082643,208.6252409)(22.3608255,208.68524658)
\curveto(22.36082642,208.75524077)(22.35082643,208.8202407)(22.3308255,208.88024658)
\curveto(22.32082646,208.93024059)(22.32082646,208.98024054)(22.3308255,209.03024658)
\curveto(22.33082645,209.09024043)(22.32582645,209.14524038)(22.3158255,209.19524658)
\curveto(22.28582649,209.31524021)(22.26082652,209.43024009)(22.2408255,209.54024658)
\curveto(22.22082656,209.66023986)(22.18582659,209.78023974)(22.1358255,209.90024658)
\curveto(21.98582679,210.31023921)(21.79082699,210.64523888)(21.5508255,210.90524658)
\curveto(21.31082747,211.17523835)(20.99082779,211.41523811)(20.5908255,211.62524658)
\curveto(20.34082844,211.75523777)(20.06082872,211.85523767)(19.7508255,211.92524658)
\curveto(19.43082935,212.00523752)(19.10582967,212.08023744)(18.7758255,212.15024658)
\lineto(16.3158255,212.64524658)
\lineto(13.7208255,213.15524658)
\curveto(13.55083523,213.19523633)(13.36083542,213.23023629)(13.1508255,213.26024658)
\curveto(12.93083585,213.30023622)(12.775836,213.37023615)(12.6858255,213.47024658)
\curveto(12.61583616,213.54023598)(12.5808362,213.64023588)(12.5808255,213.77024658)
\curveto(12.5808362,213.91023561)(12.5808362,214.04523548)(12.5808255,214.17524658)
\curveto(12.5808362,214.2252353)(12.58583619,214.27023525)(12.5958255,214.31024658)
\curveto(12.59583618,214.35023517)(12.59583618,214.39523513)(12.5958255,214.44524658)
\curveto(12.62583615,214.58523494)(12.6758361,214.66523486)(12.7458255,214.68524658)
\curveto(12.82583595,214.7252348)(12.94083584,214.7252348)(13.0908255,214.68524658)
\curveto(13.24083554,214.65523487)(13.3758354,214.6252349)(13.4958255,214.59524658)
\lineto(15.5658255,214.19024658)
\lineto(18.6858255,213.56024658)
\curveto(19.05582972,213.49023603)(19.42082936,213.41023611)(19.7808255,213.32024658)
\curveto(20.14082864,213.24023628)(20.46582831,213.13523639)(20.7558255,213.00524658)
\curveto(21.24582753,212.76523676)(21.66582711,212.49523703)(22.0158255,212.19524658)
\curveto(22.35582642,211.90523762)(22.64582613,211.54523798)(22.8858255,211.11524658)
\curveto(22.9758258,210.94523858)(23.05582572,210.77023875)(23.1258255,210.59024658)
\curveto(23.19582558,210.41023911)(23.26082552,210.21523931)(23.3208255,210.00524658)
\curveto(23.35082543,209.91523961)(23.37082541,209.8202397)(23.3808255,209.72024658)
\curveto(23.40082538,209.63023989)(23.42082536,209.53523999)(23.4408255,209.43524658)
\curveto(23.46082532,209.3252402)(23.47082531,209.21524031)(23.4708255,209.10524658)
\curveto(23.47082531,209.00524052)(23.4808253,208.90524062)(23.5008255,208.80524658)
\lineto(23.5008255,208.62524658)
\curveto(23.51082527,208.57524095)(23.51582526,208.49524103)(23.5158255,208.38524658)
\curveto(23.51582526,208.27524125)(23.51082527,208.19524133)(23.5008255,208.14524658)
\lineto(23.5008255,207.96524658)
\curveto(23.4808253,207.89524163)(23.47082531,207.8202417)(23.4708255,207.74024658)
\curveto(23.4808253,207.67024185)(23.4758253,207.60524192)(23.4558255,207.54524658)
\lineto(23.4558255,207.42524658)
\curveto(23.43582534,207.34524218)(23.42082536,207.26524226)(23.4108255,207.18524658)
\curveto(23.40082538,207.11524241)(23.38582539,207.04524248)(23.3658255,206.97524658)
\curveto(23.31582546,206.81524271)(23.27082551,206.65524287)(23.2308255,206.49524658)
\curveto(23.1808256,206.34524318)(23.12082566,206.20024332)(23.0508255,206.06024658)
\curveto(23.02082576,206.00024352)(22.98582579,205.94024358)(22.9458255,205.88024658)
\curveto(22.90582587,205.8202437)(22.86082592,205.76024376)(22.8108255,205.70024658)
\curveto(22.62082616,205.44024408)(22.39582638,205.23524429)(22.1358255,205.08524658)
\curveto(21.8758269,204.93524459)(21.57082721,204.8252447)(21.2208255,204.75524658)
\curveto(21.07082771,204.7252448)(20.91582786,204.71024481)(20.7558255,204.71024658)
\curveto(20.59582818,204.7202448)(20.43082835,204.7202448)(20.2608255,204.71024658)
\curveto(20.1808286,204.7202448)(20.10582867,204.73024479)(20.0358255,204.74024658)
\curveto(19.95582882,204.75024477)(19.8758289,204.75524477)(19.7958255,204.75524658)
\lineto(19.6458255,204.78524658)
\curveto(19.56582921,204.78524474)(19.48582929,204.79524473)(19.4058255,204.81524658)
\curveto(19.31582946,204.84524468)(19.23082955,204.87024465)(19.1508255,204.89024658)
\lineto(18.1758255,205.08524658)
\lineto(14.2608255,205.86524658)
\lineto(13.2408255,206.07524658)
\curveto(13.15083563,206.09524343)(13.06083572,206.11024341)(12.9708255,206.12024658)
\curveto(12.8808359,206.14024338)(12.80583597,206.17524335)(12.7458255,206.22524658)
\curveto(12.6758361,206.28524324)(12.62583615,206.37024315)(12.5958255,206.48024658)
\curveto(12.59583618,206.54024298)(12.59083619,206.58524294)(12.5808255,206.61524658)
}
}
{
\newrgbcolor{curcolor}{0 0 0}
\pscustom[linestyle=none,fillstyle=solid,fillcolor=curcolor]
{
\newpath
\moveto(15.3858255,219.96009033)
\curveto(15.36583341,220.60008351)(15.45083333,221.09008302)(15.6408255,221.43009033)
\curveto(15.83083295,221.77008234)(16.11583266,222.0150821)(16.4958255,222.16509033)
\curveto(16.59583218,222.20508191)(16.70583207,222.23008188)(16.8258255,222.24009033)
\curveto(16.93583184,222.26008185)(17.05083173,222.27008184)(17.1708255,222.27009033)
\curveto(17.36083142,222.29008182)(17.56583121,222.28008183)(17.7858255,222.24009033)
\curveto(18.00583077,222.2100819)(18.23083055,222.17008194)(18.4608255,222.12009033)
\lineto(20.0658255,221.80509033)
\lineto(22.4058255,221.34009033)
\lineto(22.9158255,221.22009033)
\curveto(23.08582569,221.18008293)(23.19582558,221.09008302)(23.2458255,220.95009033)
\curveto(23.26582551,220.90008321)(23.2758255,220.84508327)(23.2758255,220.78509033)
\curveto(23.28582549,220.73508338)(23.29082549,220.68008343)(23.2908255,220.62009033)
\curveto(23.29082549,220.49008362)(23.28582549,220.36508375)(23.2758255,220.24509033)
\curveto(23.2758255,220.12508399)(23.23582554,220.05008406)(23.1558255,220.02009033)
\curveto(23.08582569,219.98008413)(22.99582578,219.97008414)(22.8858255,219.99009033)
\curveto(22.775826,220.0100841)(22.66582611,220.03508408)(22.5558255,220.06509033)
\lineto(21.2658255,220.32009033)
\lineto(18.8208255,220.80009033)
\curveto(18.55083023,220.86008325)(18.28583049,220.9100832)(18.0258255,220.95009033)
\curveto(17.75583102,220.99008312)(17.52583125,220.99008312)(17.3358255,220.95009033)
\curveto(17.13583164,220.9100832)(16.9758318,220.82008329)(16.8558255,220.68009033)
\curveto(16.72583205,220.55008356)(16.62583215,220.39008372)(16.5558255,220.20009033)
\curveto(16.53583224,220.14008397)(16.52583225,220.07508404)(16.5258255,220.00509033)
\curveto(16.51583226,219.94508417)(16.50083228,219.89008422)(16.4808255,219.84009033)
\curveto(16.47083231,219.79008432)(16.47083231,219.7100844)(16.4808255,219.60009033)
\curveto(16.4808323,219.50008461)(16.48583229,219.42508469)(16.4958255,219.37509033)
\curveto(16.51583226,219.33508478)(16.52583225,219.30008481)(16.5258255,219.27009033)
\curveto(16.51583226,219.24008487)(16.51583226,219.20508491)(16.5258255,219.16509033)
\curveto(16.55583222,219.02508509)(16.59083219,218.89508522)(16.6308255,218.77509033)
\curveto(16.66083212,218.65508546)(16.70583207,218.54008557)(16.7658255,218.43009033)
\curveto(16.78583199,218.38008573)(16.80583197,218.34008577)(16.8258255,218.31009033)
\curveto(16.84583193,218.28008583)(16.86583191,218.24008587)(16.8858255,218.19009033)
\curveto(17.13583164,217.79008632)(17.51083127,217.46008665)(18.0108255,217.20009033)
\curveto(18.09083069,217.16008695)(18.1758306,217.12508699)(18.2658255,217.09509033)
\lineto(18.5058255,217.00509033)
\curveto(18.55583022,216.97508714)(18.60583017,216.96008715)(18.6558255,216.96009033)
\curveto(18.69583008,216.96008715)(18.73583004,216.94508717)(18.7758255,216.91509033)
\lineto(19.0908255,216.85509033)
\curveto(19.12082966,216.83508728)(19.15582962,216.82508729)(19.1958255,216.82509033)
\curveto(19.23582954,216.82508729)(19.2808295,216.82008729)(19.3308255,216.81009033)
\lineto(19.7808255,216.72009033)
\lineto(21.2208255,216.42009033)
\lineto(22.5408255,216.16509033)
\curveto(22.65082613,216.14508797)(22.76582601,216.12008799)(22.8858255,216.09009033)
\curveto(22.99582578,216.07008804)(23.08582569,216.03008808)(23.1558255,215.97009033)
\curveto(23.23582554,215.90008821)(23.2758255,215.80008831)(23.2758255,215.67009033)
\curveto(23.28582549,215.55008856)(23.29082549,215.42508869)(23.2908255,215.29509033)
\curveto(23.29082549,215.2150889)(23.28582549,215.14008897)(23.2758255,215.07009033)
\curveto(23.26582551,215.00008911)(23.24082554,214.94508917)(23.2008255,214.90509033)
\curveto(23.15082563,214.83508928)(23.05582572,214.8150893)(22.9158255,214.84509033)
\curveto(22.775826,214.87508924)(22.64082614,214.90008921)(22.5108255,214.92009033)
\lineto(20.7408255,215.28009033)
\lineto(17.1108255,216.00009033)
\lineto(16.1958255,216.18009033)
\lineto(15.9258255,216.24009033)
\curveto(15.83583294,216.26008785)(15.76583301,216.29508782)(15.7158255,216.34509033)
\curveto(15.65583312,216.38508773)(15.61583316,216.44008767)(15.5958255,216.51009033)
\curveto(15.58583319,216.56008755)(15.5758332,216.62008749)(15.5658255,216.69009033)
\curveto(15.55583322,216.77008734)(15.55083323,216.85008726)(15.5508255,216.93009033)
\curveto(15.55083323,217.0100871)(15.55583322,217.08508703)(15.5658255,217.15509033)
\curveto(15.5758332,217.23508688)(15.59083319,217.28508683)(15.6108255,217.30509033)
\curveto(15.6808331,217.40508671)(15.77083301,217.44008667)(15.8808255,217.41009033)
\curveto(15.9808328,217.38008673)(16.09583268,217.37008674)(16.2258255,217.38009033)
\curveto(16.28583249,217.39008672)(16.33583244,217.42008669)(16.3758255,217.47009033)
\curveto(16.38583239,217.59008652)(16.34083244,217.69508642)(16.2408255,217.78509033)
\curveto(16.14083264,217.88508623)(16.06083272,217.98008613)(16.0008255,218.07009033)
\curveto(15.90083288,218.23008588)(15.81083297,218.39008572)(15.7308255,218.55009033)
\curveto(15.64083314,218.7100854)(15.56583321,218.89508522)(15.5058255,219.10509033)
\curveto(15.4758333,219.18508493)(15.45583332,219.27508484)(15.4458255,219.37509033)
\curveto(15.43583334,219.47508464)(15.42083336,219.57008454)(15.4008255,219.66009033)
\curveto(15.39083339,219.7100844)(15.38583339,219.76008435)(15.3858255,219.81009033)
\lineto(15.3858255,219.96009033)
}
}
{
\newrgbcolor{curcolor}{0 0 0}
\pscustom[linestyle=none,fillstyle=solid,fillcolor=curcolor]
{
\newpath
\moveto(14.0358255,225.17469971)
\curveto(13.9758348,225.10469673)(13.87083491,225.08469675)(13.7208255,225.11469971)
\curveto(13.56083522,225.14469669)(13.40583537,225.17469666)(13.2558255,225.20469971)
\curveto(13.1758356,225.21469662)(13.09083569,225.22969661)(13.0008255,225.24969971)
\curveto(12.91083587,225.26969657)(12.83583594,225.29969654)(12.7758255,225.33969971)
\curveto(12.69583608,225.39969644)(12.63583614,225.48969635)(12.5958255,225.60969971)
\curveto(12.58583619,225.6396962)(12.58583619,225.66469617)(12.5958255,225.68469971)
\curveto(12.59583618,225.70469613)(12.59083619,225.72969611)(12.5808255,225.75969971)
\curveto(12.5808362,225.92969591)(12.58583619,226.08469575)(12.5958255,226.22469971)
\curveto(12.60583617,226.37469546)(12.66583611,226.46469537)(12.7758255,226.49469971)
\curveto(12.83583594,226.51469532)(12.91083587,226.51469532)(13.0008255,226.49469971)
\curveto(13.0808357,226.47469536)(13.16583561,226.45969538)(13.2558255,226.44969971)
\curveto(13.43583534,226.40969543)(13.60583517,226.36969547)(13.7658255,226.32969971)
\curveto(13.92583485,226.29969554)(14.03083475,226.21469562)(14.0808255,226.07469971)
\curveto(14.10083468,226.01469582)(14.11083467,225.95469588)(14.1108255,225.89469971)
\lineto(14.1108255,225.72969971)
\lineto(14.1108255,225.41469971)
\curveto(14.11083467,225.31469652)(14.08583469,225.2346966)(14.0358255,225.17469971)
\moveto(22.5408255,224.58969971)
\curveto(22.64082614,224.56969727)(22.74582603,224.54969729)(22.8558255,224.52969971)
\curveto(22.95582582,224.51969732)(23.03582574,224.47969736)(23.0958255,224.40969971)
\curveto(23.15582562,224.36969747)(23.19582558,224.31969752)(23.2158255,224.25969971)
\curveto(23.22582555,224.19969764)(23.24082554,224.12469771)(23.2608255,224.03469971)
\lineto(23.2608255,223.80969971)
\curveto(23.26082552,223.67969816)(23.25582552,223.56969827)(23.2458255,223.47969971)
\curveto(23.22582555,223.38969845)(23.1758256,223.32469851)(23.0958255,223.28469971)
\curveto(23.03582574,223.26469857)(22.96082582,223.25969858)(22.8708255,223.26969971)
\curveto(22.77082601,223.28969855)(22.6758261,223.30969853)(22.5858255,223.32969971)
\lineto(16.2408255,224.60469971)
\curveto(16.13083265,224.62469721)(16.02583275,224.64469719)(15.9258255,224.66469971)
\curveto(15.81583296,224.68469715)(15.73083305,224.72469711)(15.6708255,224.78469971)
\curveto(15.62083316,224.82469701)(15.59083319,224.86969697)(15.5808255,224.91969971)
\curveto(15.57083321,224.97969686)(15.55583322,225.0396968)(15.5358255,225.09969971)
\curveto(15.53583324,225.11969672)(15.54083324,225.1396967)(15.5508255,225.15969971)
\curveto(15.55083323,225.18969665)(15.54583323,225.21469662)(15.5358255,225.23469971)
\curveto(15.53583324,225.36469647)(15.54083324,225.49469634)(15.5508255,225.62469971)
\curveto(15.55083323,225.76469607)(15.59083319,225.84969599)(15.6708255,225.87969971)
\curveto(15.73083305,225.91969592)(15.81083297,225.92969591)(15.9108255,225.90969971)
\curveto(16.00083278,225.88969595)(16.09583268,225.86969597)(16.1958255,225.84969971)
\lineto(22.5408255,224.58969971)
}
}
{
\newrgbcolor{curcolor}{0 0 0}
\pscustom[linestyle=none,fillstyle=solid,fillcolor=curcolor]
{
\newpath
\moveto(22.4508255,233.50954346)
\lineto(22.8408255,233.41954346)
\curveto(22.96082582,233.39953553)(23.06082572,233.35953557)(23.1408255,233.29954346)
\curveto(23.21082557,233.2295357)(23.25082553,233.13453579)(23.2608255,233.01454346)
\lineto(23.2608255,232.66954346)
\curveto(23.26082552,232.60953632)(23.26582551,232.54953638)(23.2758255,232.48954346)
\curveto(23.2758255,232.43953649)(23.26582551,232.39453653)(23.2458255,232.35454346)
\curveto(23.22582555,232.27453665)(23.18582559,232.2245367)(23.1258255,232.20454346)
\curveto(23.0758257,232.17453675)(23.01582576,232.16453676)(22.9458255,232.17454346)
\curveto(22.8758259,232.18453674)(22.80582597,232.17953675)(22.7358255,232.15954346)
\curveto(22.71582606,232.15953677)(22.70082608,232.14953678)(22.6908255,232.12954346)
\lineto(22.6308255,232.09954346)
\curveto(22.62082616,231.99953693)(22.64082614,231.91453701)(22.6908255,231.84454346)
\curveto(22.74082604,231.78453714)(22.79082599,231.71953721)(22.8408255,231.64954346)
\curveto(22.99082579,231.41953751)(23.10582567,231.19453773)(23.1858255,230.97454346)
\curveto(23.26582551,230.78453814)(23.32582545,230.56453836)(23.3658255,230.31454346)
\curveto(23.40582537,230.07453885)(23.42582535,229.8295391)(23.4258255,229.57954346)
\curveto(23.43582534,229.33953959)(23.42082536,229.09953983)(23.3808255,228.85954346)
\curveto(23.35082543,228.6295403)(23.29582548,228.43454049)(23.2158255,228.27454346)
\curveto(22.99582578,227.79454113)(22.70082608,227.4295415)(22.3308255,227.17954346)
\curveto(21.95082683,226.93954199)(21.4808273,226.78454214)(20.9208255,226.71454346)
\curveto(20.83082795,226.69454223)(20.74082804,226.68454224)(20.6508255,226.68454346)
\curveto(20.55082823,226.69454223)(20.45082833,226.69454223)(20.3508255,226.68454346)
\curveto(20.30082848,226.68454224)(20.25082853,226.68954224)(20.2008255,226.69954346)
\curveto(20.15082863,226.70954222)(20.10082868,226.71454221)(20.0508255,226.71454346)
\curveto(20.00082878,226.70454222)(19.95082883,226.70454222)(19.9008255,226.71454346)
\curveto(19.84082894,226.73454219)(19.78582899,226.74454218)(19.7358255,226.74454346)
\lineto(19.5858255,226.77454346)
\curveto(19.53582924,226.76454216)(19.47082931,226.76454216)(19.3908255,226.77454346)
\curveto(19.31082947,226.79454213)(19.24582953,226.81954211)(19.1958255,226.84954346)
\lineto(19.0308255,226.89454346)
\curveto(18.96082982,226.924542)(18.89082989,226.94454198)(18.8208255,226.95454346)
\curveto(18.74083004,226.96454196)(18.66583011,226.98454194)(18.5958255,227.01454346)
\curveto(18.54583023,227.03454189)(18.50083028,227.04954188)(18.4608255,227.05954346)
\curveto(18.42083036,227.06954186)(18.3758304,227.08454184)(18.3258255,227.10454346)
\curveto(18.22583055,227.15454177)(18.13083065,227.19954173)(18.0408255,227.23954346)
\curveto(17.94083084,227.27954165)(17.84583093,227.3245416)(17.7558255,227.37454346)
\curveto(17.3758314,227.57454135)(17.03583174,227.80454112)(16.7358255,228.06454346)
\curveto(16.42583235,228.33454059)(16.17083261,228.63454029)(15.9708255,228.96454346)
\curveto(15.85083293,229.16453976)(15.75083303,229.36453956)(15.6708255,229.56454346)
\curveto(15.59083319,229.76453916)(15.52083326,229.97953895)(15.4608255,230.20954346)
\lineto(15.4308255,230.41954346)
\curveto(15.42083336,230.48953844)(15.40583337,230.55953837)(15.3858255,230.62954346)
\lineto(15.3858255,230.77954346)
\curveto(15.36583341,230.86953806)(15.35583342,230.98953794)(15.3558255,231.13954346)
\curveto(15.35583342,231.29953763)(15.36583341,231.41453751)(15.3858255,231.48454346)
\curveto(15.39583338,231.5245374)(15.40083338,231.57953735)(15.4008255,231.64954346)
\curveto(15.43083335,231.74953718)(15.45583332,231.85453707)(15.4758255,231.96454346)
\curveto(15.48583329,232.07453685)(15.51583326,232.17453675)(15.5658255,232.26454346)
\curveto(15.62583315,232.40453652)(15.69083309,232.53453639)(15.7608255,232.65454346)
\curveto(15.83083295,232.77453615)(15.91083287,232.88453604)(16.0008255,232.98454346)
\curveto(16.05083273,233.03453589)(16.10583267,233.08453584)(16.1658255,233.13454346)
\curveto(16.21583256,233.19453573)(16.23083255,233.27953565)(16.2108255,233.38954346)
\lineto(16.1358255,233.46454346)
\curveto(16.11583266,233.48453544)(16.08583269,233.49953543)(16.0458255,233.50954346)
\curveto(15.95583282,233.55953537)(15.84083294,233.59453533)(15.7008255,233.61454346)
\curveto(15.56083322,233.64453528)(15.43583334,233.66953526)(15.3258255,233.68954346)
\lineto(13.6008255,234.03454346)
\curveto(13.46083532,234.06453486)(13.30583547,234.09453483)(13.1358255,234.12454346)
\curveto(12.95583582,234.16453476)(12.82583595,234.21453471)(12.7458255,234.27454346)
\curveto(12.6758361,234.33453459)(12.63083615,234.40453452)(12.6108255,234.48454346)
\curveto(12.61083617,234.50453442)(12.61083617,234.5295344)(12.6108255,234.55954346)
\curveto(12.60083618,234.58953434)(12.59583618,234.61453431)(12.5958255,234.63454346)
\curveto(12.58583619,234.78453414)(12.58583619,234.93453399)(12.5958255,235.08454346)
\curveto(12.59583618,235.23453369)(12.63583614,235.33453359)(12.7158255,235.38454346)
\curveto(12.79583598,235.41453351)(12.89583588,235.41453351)(13.0158255,235.38454346)
\curveto(13.13583564,235.36453356)(13.26083552,235.34453358)(13.3908255,235.32454346)
\lineto(22.4508255,233.50954346)
\moveto(19.6158255,232.86454346)
\curveto(19.56582921,232.89453603)(19.50082928,232.91453601)(19.4208255,232.92454346)
\curveto(19.33082945,232.94453598)(19.26082952,232.94953598)(19.2108255,232.93954346)
\lineto(18.9858255,232.98454346)
\curveto(18.89582988,232.98453594)(18.80582997,232.98953594)(18.7158255,232.99954346)
\curveto(18.61583016,233.00953592)(18.52583025,233.00453592)(18.4458255,232.98454346)
\lineto(18.2208255,232.98454346)
\curveto(18.15083063,232.98453594)(18.0808307,232.97453595)(18.0108255,232.95454346)
\curveto(17.71083107,232.89453603)(17.44583133,232.78953614)(17.2158255,232.63954346)
\curveto(16.98583179,232.49953643)(16.80583197,232.29953663)(16.6758255,232.03954346)
\curveto(16.62583215,231.94953698)(16.59083219,231.85453707)(16.5708255,231.75454346)
\curveto(16.54083224,231.65453727)(16.51583226,231.54453738)(16.4958255,231.42454346)
\curveto(16.4758323,231.35453757)(16.46583231,231.26953766)(16.4658255,231.16954346)
\lineto(16.4658255,230.89954346)
\lineto(16.4958255,230.74954346)
\lineto(16.4958255,230.61454346)
\curveto(16.51583226,230.53453839)(16.53583224,230.44953848)(16.5558255,230.35954346)
\curveto(16.5758322,230.26953866)(16.60083218,230.18453874)(16.6308255,230.10454346)
\curveto(16.77083201,229.75453917)(16.9758318,229.45453947)(17.2458255,229.20454346)
\curveto(17.50583127,228.95453997)(17.81083097,228.73454019)(18.1608255,228.54454346)
\curveto(18.27083051,228.48454044)(18.38583039,228.43454049)(18.5058255,228.39454346)
\lineto(18.8358255,228.27454346)
\lineto(18.9558255,228.24454346)
\curveto(18.98582979,228.23454069)(19.02082976,228.2245407)(19.0608255,228.21454346)
\curveto(19.11082967,228.18454074)(19.16582961,228.16454076)(19.2258255,228.15454346)
\curveto(19.28582949,228.15454077)(19.34082944,228.14954078)(19.3908255,228.13954346)
\curveto(19.50082928,228.11954081)(19.61082917,228.09454083)(19.7208255,228.06454346)
\curveto(19.82082896,228.04454088)(19.91582886,228.03954089)(20.0058255,228.04954346)
\curveto(20.03582874,228.04954088)(20.08582869,228.04454088)(20.1558255,228.03454346)
\lineto(20.3658255,228.03454346)
\curveto(20.43582834,228.03454089)(20.50582827,228.03954089)(20.5758255,228.04954346)
\curveto(20.92582785,228.08954084)(21.22582755,228.17954075)(21.4758255,228.31954346)
\curveto(21.72582705,228.45954047)(21.93082685,228.65954027)(22.0908255,228.91954346)
\curveto(22.14082664,228.99953993)(22.1808266,229.07953985)(22.2108255,229.15954346)
\curveto(22.24082654,229.24953968)(22.27082651,229.34453958)(22.3008255,229.44454346)
\curveto(22.32082646,229.49453943)(22.32582645,229.54453938)(22.3158255,229.59454346)
\curveto(22.30582647,229.65453927)(22.31082647,229.70953922)(22.3308255,229.75954346)
\curveto(22.34082644,229.78953914)(22.34582643,229.8245391)(22.3458255,229.86454346)
\lineto(22.3458255,229.99954346)
\lineto(22.3458255,230.13454346)
\curveto(22.33582644,230.17453875)(22.33082645,230.2295387)(22.3308255,230.29954346)
\curveto(22.31082647,230.37953855)(22.29582648,230.45953847)(22.2858255,230.53954346)
\curveto(22.26582651,230.6295383)(22.24082654,230.70953822)(22.2108255,230.77954346)
\curveto(22.07082671,231.13953779)(21.89582688,231.44453748)(21.6858255,231.69454346)
\curveto(21.46582731,231.94453698)(21.19082759,232.16953676)(20.8608255,232.36954346)
\curveto(20.75082803,232.43953649)(20.64082814,232.49453643)(20.5308255,232.53454346)
\lineto(20.2008255,232.68454346)
\curveto(20.16082862,232.71453621)(20.12582865,232.7295362)(20.0958255,232.72954346)
\curveto(20.05582872,232.73953619)(20.01582876,232.75453617)(19.9758255,232.77454346)
\curveto(19.91582886,232.79453613)(19.85582892,232.80953612)(19.7958255,232.81954346)
\curveto(19.73582904,232.8295361)(19.6758291,232.84453608)(19.6158255,232.86454346)
}
}
{
\newrgbcolor{curcolor}{0 0 0}
\pscustom[linestyle=none,fillstyle=solid,fillcolor=curcolor]
{
\newpath
\moveto(22.7058255,242.29579346)
\curveto(22.86582591,242.28578555)(23.00082578,242.24078559)(23.1108255,242.16079346)
\curveto(23.21082557,242.08078575)(23.28582549,241.98578585)(23.3358255,241.87579346)
\curveto(23.35582542,241.82578601)(23.36582541,241.77078606)(23.3658255,241.71079346)
\curveto(23.36582541,241.66078617)(23.3758254,241.60078623)(23.3958255,241.53079346)
\curveto(23.44582533,241.30078653)(23.43082535,241.08578675)(23.3508255,240.88579346)
\curveto(23.2808255,240.68578715)(23.19082559,240.56078727)(23.0808255,240.51079346)
\curveto(23.01082577,240.47078736)(22.93082585,240.44078739)(22.8408255,240.42079346)
\curveto(22.74082604,240.40078743)(22.66082612,240.36578747)(22.6008255,240.31579346)
\lineto(22.5408255,240.25579346)
\curveto(22.52082626,240.2357876)(22.51582626,240.20578763)(22.5258255,240.16579346)
\curveto(22.55582622,240.04578779)(22.61082617,239.9307879)(22.6908255,239.82079346)
\curveto(22.77082601,239.71078812)(22.84082594,239.60578823)(22.9008255,239.50579346)
\curveto(22.9808258,239.35578848)(23.05582572,239.20078863)(23.1258255,239.04079346)
\curveto(23.18582559,238.88078895)(23.24082554,238.71078912)(23.2908255,238.53079346)
\curveto(23.32082546,238.42078941)(23.34082544,238.30578953)(23.3508255,238.18579346)
\curveto(23.36082542,238.07578976)(23.3758254,237.96078987)(23.3958255,237.84079346)
\curveto(23.40582537,237.79079004)(23.41082537,237.74579009)(23.4108255,237.70579346)
\lineto(23.4108255,237.60079346)
\curveto(23.43082535,237.49079034)(23.43082535,237.38579045)(23.4108255,237.28579346)
\lineto(23.4108255,237.15079346)
\curveto(23.40082538,237.10079073)(23.39582538,237.05079078)(23.3958255,237.00079346)
\curveto(23.39582538,236.95079088)(23.38582539,236.91079092)(23.3658255,236.88079346)
\curveto(23.35582542,236.84079099)(23.35082543,236.80579103)(23.3508255,236.77579346)
\curveto(23.36082542,236.75579108)(23.36082542,236.7307911)(23.3508255,236.70079346)
\lineto(23.2908255,236.46079346)
\curveto(23.2808255,236.39079144)(23.26082552,236.32579151)(23.2308255,236.26579346)
\curveto(23.10082568,235.98579185)(22.95582582,235.77079206)(22.7958255,235.62079346)
\curveto(22.62582615,235.47079236)(22.39082639,235.36579247)(22.0908255,235.30579346)
\curveto(21.87082691,235.25579258)(21.60582717,235.26079257)(21.2958255,235.32079346)
\lineto(20.9808255,235.39579346)
\curveto(20.93082785,235.41579242)(20.8808279,235.4307924)(20.8308255,235.44079346)
\lineto(20.6508255,235.50079346)
\lineto(20.3208255,235.68079346)
\curveto(20.21082857,235.75079208)(20.11082867,235.82079201)(20.0208255,235.89079346)
\curveto(19.73082905,236.1307917)(19.51582926,236.42079141)(19.3758255,236.76079346)
\curveto(19.23582954,237.10079073)(19.11082967,237.46579037)(19.0008255,237.85579346)
\curveto(18.96082982,238.00578983)(18.93082985,238.15578968)(18.9108255,238.30579346)
\curveto(18.89082989,238.46578937)(18.86582991,238.62078921)(18.8358255,238.77079346)
\curveto(18.81582996,238.85078898)(18.80582997,238.92078891)(18.8058255,238.98079346)
\curveto(18.80582997,239.05078878)(18.79582998,239.12578871)(18.7758255,239.20579346)
\curveto(18.75583002,239.27578856)(18.74583003,239.34578849)(18.7458255,239.41579346)
\curveto(18.73583004,239.49578834)(18.72083006,239.57578826)(18.7008255,239.65579346)
\curveto(18.64083014,239.91578792)(18.59083019,240.16078767)(18.5508255,240.39079346)
\curveto(18.50083028,240.62078721)(18.38583039,240.82078701)(18.2058255,240.99079346)
\curveto(18.12583065,241.06078677)(18.02583075,241.12578671)(17.9058255,241.18579346)
\curveto(17.775831,241.25578658)(17.63583114,241.28578655)(17.4858255,241.27579346)
\curveto(17.24583153,241.26578657)(17.05583172,241.21578662)(16.9158255,241.12579346)
\curveto(16.775832,241.04578679)(16.66583211,240.90578693)(16.5858255,240.70579346)
\curveto(16.53583224,240.59578724)(16.50083228,240.46078737)(16.4808255,240.30079346)
\curveto(16.46083232,240.14078769)(16.45083233,239.97078786)(16.4508255,239.79079346)
\curveto(16.45083233,239.61078822)(16.46083232,239.4307884)(16.4808255,239.25079346)
\curveto(16.50083228,239.08078875)(16.53083225,238.9307889)(16.5708255,238.80079346)
\curveto(16.63083215,238.62078921)(16.71583206,238.44078939)(16.8258255,238.26079346)
\curveto(16.88583189,238.17078966)(16.96583181,238.08078975)(17.0658255,237.99079346)
\curveto(17.15583162,237.91078992)(17.25583152,237.83579)(17.3658255,237.76579346)
\curveto(17.44583133,237.71579012)(17.53083125,237.67079016)(17.6208255,237.63079346)
\curveto(17.71083107,237.59079024)(17.780831,237.5307903)(17.8308255,237.45079346)
\curveto(17.86083092,237.40079043)(17.88583089,237.32579051)(17.9058255,237.22579346)
\curveto(17.91583086,237.12579071)(17.92083086,237.02579081)(17.9208255,236.92579346)
\curveto(17.92083086,236.82579101)(17.91583086,236.7307911)(17.9058255,236.64079346)
\curveto(17.88583089,236.55079128)(17.86083092,236.49079134)(17.8308255,236.46079346)
\curveto(17.80083098,236.42079141)(17.75083103,236.39579144)(17.6808255,236.38579346)
\curveto(17.61083117,236.38579145)(17.53583124,236.40579143)(17.4558255,236.44579346)
\curveto(17.32583145,236.49579134)(17.20583157,236.55079128)(17.0958255,236.61079346)
\curveto(16.9758318,236.67079116)(16.86083192,236.7357911)(16.7508255,236.80579346)
\curveto(16.40083238,237.06579077)(16.13083265,237.36079047)(15.9408255,237.69079346)
\curveto(15.74083304,238.02078981)(15.5808332,238.41078942)(15.4608255,238.86079346)
\curveto(15.44083334,238.97078886)(15.42583335,239.07578876)(15.4158255,239.17579346)
\curveto(15.40583337,239.28578855)(15.39083339,239.39578844)(15.3708255,239.50579346)
\curveto(15.36083342,239.55578828)(15.36083342,239.62078821)(15.3708255,239.70079346)
\curveto(15.37083341,239.79078804)(15.36083342,239.85078798)(15.3408255,239.88079346)
\curveto(15.33083345,240.58078725)(15.41083337,241.17078666)(15.5808255,241.65079346)
\curveto(15.75083303,242.14078569)(16.0758327,242.44578539)(16.5558255,242.56579346)
\curveto(16.75583202,242.61578522)(16.99083179,242.62078521)(17.2608255,242.58079346)
\curveto(17.52083126,242.54078529)(17.79583098,242.49078534)(18.0858255,242.43079346)
\lineto(21.4008255,241.77079346)
\curveto(21.54082724,241.74078609)(21.6758271,241.71578612)(21.8058255,241.69579346)
\curveto(21.93582684,241.68578615)(22.04082674,241.69578614)(22.1208255,241.72579346)
\curveto(22.19082659,241.76578607)(22.24082654,241.82078601)(22.2708255,241.89079346)
\curveto(22.31082647,241.98078585)(22.34082644,242.06078577)(22.3608255,242.13079346)
\curveto(22.37082641,242.21078562)(22.41582636,242.26078557)(22.4958255,242.28079346)
\curveto(22.52582625,242.30078553)(22.55582622,242.30578553)(22.5858255,242.29579346)
\lineto(22.7058255,242.29579346)
\moveto(21.0408255,240.48079346)
\curveto(20.90082788,240.57078726)(20.74082804,240.6357872)(20.5608255,240.67579346)
\curveto(20.37082841,240.71578712)(20.1758286,240.75578708)(19.9758255,240.79579346)
\curveto(19.86582891,240.81578702)(19.76582901,240.830787)(19.6758255,240.84079346)
\curveto(19.58582919,240.85078698)(19.51582926,240.82578701)(19.4658255,240.76579346)
\curveto(19.44582933,240.7357871)(19.43582934,240.66578717)(19.4358255,240.55579346)
\curveto(19.45582932,240.5357873)(19.46582931,240.50078733)(19.4658255,240.45079346)
\curveto(19.46582931,240.40078743)(19.4758293,240.35078748)(19.4958255,240.30079346)
\curveto(19.51582926,240.22078761)(19.53582924,240.12578771)(19.5558255,240.01579346)
\lineto(19.6158255,239.71579346)
\curveto(19.61582916,239.68578815)(19.62082916,239.65078818)(19.6308255,239.61079346)
\lineto(19.6308255,239.50579346)
\curveto(19.67082911,239.34578849)(19.69582908,239.17578866)(19.7058255,238.99579346)
\curveto(19.70582907,238.82578901)(19.72582905,238.66078917)(19.7658255,238.50079346)
\curveto(19.78582899,238.41078942)(19.80582897,238.3307895)(19.8258255,238.26079346)
\curveto(19.83582894,238.20078963)(19.85082893,238.12578971)(19.8708255,238.03579346)
\curveto(19.92082886,237.86578997)(19.98582879,237.70079013)(20.0658255,237.54079346)
\curveto(20.13582864,237.39079044)(20.22582855,237.25579058)(20.3358255,237.13579346)
\curveto(20.44582833,237.01579082)(20.5808282,236.91579092)(20.7408255,236.83579346)
\curveto(20.89082789,236.75579108)(21.0758277,236.69579114)(21.2958255,236.65579346)
\curveto(21.39582738,236.6357912)(21.49082729,236.6357912)(21.5808255,236.65579346)
\curveto(21.66082712,236.67579116)(21.73582704,236.70579113)(21.8058255,236.74579346)
\curveto(21.91582686,236.79579104)(22.01082677,236.87579096)(22.0908255,236.98579346)
\curveto(22.16082662,237.10579073)(22.22082656,237.2357906)(22.2708255,237.37579346)
\curveto(22.2808265,237.42579041)(22.28582649,237.47579036)(22.2858255,237.52579346)
\curveto(22.28582649,237.57579026)(22.29082649,237.62579021)(22.3008255,237.67579346)
\curveto(22.32082646,237.74579009)(22.33582644,237.83079)(22.3458255,237.93079346)
\curveto(22.34582643,238.0307898)(22.33582644,238.12078971)(22.3158255,238.20079346)
\curveto(22.29582648,238.26078957)(22.29082649,238.32078951)(22.3008255,238.38079346)
\curveto(22.30082648,238.44078939)(22.29082649,238.50078933)(22.2708255,238.56079346)
\curveto(22.25082653,238.65078918)(22.23582654,238.7307891)(22.2258255,238.80079346)
\curveto(22.21582656,238.88078895)(22.19582658,238.96078887)(22.1658255,239.04079346)
\curveto(22.04582673,239.35078848)(21.90082688,239.62578821)(21.7308255,239.86579346)
\curveto(21.56082722,240.10578773)(21.33082745,240.31078752)(21.0408255,240.48079346)
}
}
{
\newrgbcolor{curcolor}{0 0 0}
\pscustom[linestyle=none,fillstyle=solid,fillcolor=curcolor]
{
\newpath
\moveto(22.4508255,250.47243408)
\lineto(22.8408255,250.38243408)
\curveto(22.96082582,250.36242615)(23.06082572,250.32242619)(23.1408255,250.26243408)
\curveto(23.21082557,250.19242632)(23.25082553,250.09742642)(23.2608255,249.97743408)
\lineto(23.2608255,249.63243408)
\curveto(23.26082552,249.57242694)(23.26582551,249.512427)(23.2758255,249.45243408)
\curveto(23.2758255,249.40242711)(23.26582551,249.35742716)(23.2458255,249.31743408)
\curveto(23.22582555,249.23742728)(23.18582559,249.18742733)(23.1258255,249.16743408)
\curveto(23.0758257,249.13742738)(23.01582576,249.12742739)(22.9458255,249.13743408)
\curveto(22.8758259,249.14742737)(22.80582597,249.14242737)(22.7358255,249.12243408)
\curveto(22.71582606,249.12242739)(22.70082608,249.1124274)(22.6908255,249.09243408)
\lineto(22.6308255,249.06243408)
\curveto(22.62082616,248.96242755)(22.64082614,248.87742764)(22.6908255,248.80743408)
\curveto(22.74082604,248.74742777)(22.79082599,248.68242783)(22.8408255,248.61243408)
\curveto(22.99082579,248.38242813)(23.10582567,248.15742836)(23.1858255,247.93743408)
\curveto(23.26582551,247.74742877)(23.32582545,247.52742899)(23.3658255,247.27743408)
\curveto(23.40582537,247.03742948)(23.42582535,246.79242972)(23.4258255,246.54243408)
\curveto(23.43582534,246.30243021)(23.42082536,246.06243045)(23.3808255,245.82243408)
\curveto(23.35082543,245.59243092)(23.29582548,245.39743112)(23.2158255,245.23743408)
\curveto(22.99582578,244.75743176)(22.70082608,244.39243212)(22.3308255,244.14243408)
\curveto(21.95082683,243.90243261)(21.4808273,243.74743277)(20.9208255,243.67743408)
\curveto(20.83082795,243.65743286)(20.74082804,243.64743287)(20.6508255,243.64743408)
\curveto(20.55082823,243.65743286)(20.45082833,243.65743286)(20.3508255,243.64743408)
\curveto(20.30082848,243.64743287)(20.25082853,243.65243286)(20.2008255,243.66243408)
\curveto(20.15082863,243.67243284)(20.10082868,243.67743284)(20.0508255,243.67743408)
\curveto(20.00082878,243.66743285)(19.95082883,243.66743285)(19.9008255,243.67743408)
\curveto(19.84082894,243.69743282)(19.78582899,243.70743281)(19.7358255,243.70743408)
\lineto(19.5858255,243.73743408)
\curveto(19.53582924,243.72743279)(19.47082931,243.72743279)(19.3908255,243.73743408)
\curveto(19.31082947,243.75743276)(19.24582953,243.78243273)(19.1958255,243.81243408)
\lineto(19.0308255,243.85743408)
\curveto(18.96082982,243.88743263)(18.89082989,243.90743261)(18.8208255,243.91743408)
\curveto(18.74083004,243.92743259)(18.66583011,243.94743257)(18.5958255,243.97743408)
\curveto(18.54583023,243.99743252)(18.50083028,244.0124325)(18.4608255,244.02243408)
\curveto(18.42083036,244.03243248)(18.3758304,244.04743247)(18.3258255,244.06743408)
\curveto(18.22583055,244.1174324)(18.13083065,244.16243235)(18.0408255,244.20243408)
\curveto(17.94083084,244.24243227)(17.84583093,244.28743223)(17.7558255,244.33743408)
\curveto(17.3758314,244.53743198)(17.03583174,244.76743175)(16.7358255,245.02743408)
\curveto(16.42583235,245.29743122)(16.17083261,245.59743092)(15.9708255,245.92743408)
\curveto(15.85083293,246.12743039)(15.75083303,246.32743019)(15.6708255,246.52743408)
\curveto(15.59083319,246.72742979)(15.52083326,246.94242957)(15.4608255,247.17243408)
\lineto(15.4308255,247.38243408)
\curveto(15.42083336,247.45242906)(15.40583337,247.52242899)(15.3858255,247.59243408)
\lineto(15.3858255,247.74243408)
\curveto(15.36583341,247.83242868)(15.35583342,247.95242856)(15.3558255,248.10243408)
\curveto(15.35583342,248.26242825)(15.36583341,248.37742814)(15.3858255,248.44743408)
\curveto(15.39583338,248.48742803)(15.40083338,248.54242797)(15.4008255,248.61243408)
\curveto(15.43083335,248.7124278)(15.45583332,248.8174277)(15.4758255,248.92743408)
\curveto(15.48583329,249.03742748)(15.51583326,249.13742738)(15.5658255,249.22743408)
\curveto(15.62583315,249.36742715)(15.69083309,249.49742702)(15.7608255,249.61743408)
\curveto(15.83083295,249.73742678)(15.91083287,249.84742667)(16.0008255,249.94743408)
\curveto(16.05083273,249.99742652)(16.10583267,250.04742647)(16.1658255,250.09743408)
\curveto(16.21583256,250.15742636)(16.23083255,250.24242627)(16.2108255,250.35243408)
\lineto(16.1358255,250.42743408)
\curveto(16.11583266,250.44742607)(16.08583269,250.46242605)(16.0458255,250.47243408)
\curveto(15.95583282,250.52242599)(15.84083294,250.55742596)(15.7008255,250.57743408)
\curveto(15.56083322,250.60742591)(15.43583334,250.63242588)(15.3258255,250.65243408)
\lineto(13.6008255,250.99743408)
\curveto(13.46083532,251.02742549)(13.30583547,251.05742546)(13.1358255,251.08743408)
\curveto(12.95583582,251.12742539)(12.82583595,251.17742534)(12.7458255,251.23743408)
\curveto(12.6758361,251.29742522)(12.63083615,251.36742515)(12.6108255,251.44743408)
\curveto(12.61083617,251.46742505)(12.61083617,251.49242502)(12.6108255,251.52243408)
\curveto(12.60083618,251.55242496)(12.59583618,251.57742494)(12.5958255,251.59743408)
\curveto(12.58583619,251.74742477)(12.58583619,251.89742462)(12.5958255,252.04743408)
\curveto(12.59583618,252.19742432)(12.63583614,252.29742422)(12.7158255,252.34743408)
\curveto(12.79583598,252.37742414)(12.89583588,252.37742414)(13.0158255,252.34743408)
\curveto(13.13583564,252.32742419)(13.26083552,252.30742421)(13.3908255,252.28743408)
\lineto(22.4508255,250.47243408)
\moveto(19.6158255,249.82743408)
\curveto(19.56582921,249.85742666)(19.50082928,249.87742664)(19.4208255,249.88743408)
\curveto(19.33082945,249.90742661)(19.26082952,249.9124266)(19.2108255,249.90243408)
\lineto(18.9858255,249.94743408)
\curveto(18.89582988,249.94742657)(18.80582997,249.95242656)(18.7158255,249.96243408)
\curveto(18.61583016,249.97242654)(18.52583025,249.96742655)(18.4458255,249.94743408)
\lineto(18.2208255,249.94743408)
\curveto(18.15083063,249.94742657)(18.0808307,249.93742658)(18.0108255,249.91743408)
\curveto(17.71083107,249.85742666)(17.44583133,249.75242676)(17.2158255,249.60243408)
\curveto(16.98583179,249.46242705)(16.80583197,249.26242725)(16.6758255,249.00243408)
\curveto(16.62583215,248.9124276)(16.59083219,248.8174277)(16.5708255,248.71743408)
\curveto(16.54083224,248.6174279)(16.51583226,248.50742801)(16.4958255,248.38743408)
\curveto(16.4758323,248.3174282)(16.46583231,248.23242828)(16.4658255,248.13243408)
\lineto(16.4658255,247.86243408)
\lineto(16.4958255,247.71243408)
\lineto(16.4958255,247.57743408)
\curveto(16.51583226,247.49742902)(16.53583224,247.4124291)(16.5558255,247.32243408)
\curveto(16.5758322,247.23242928)(16.60083218,247.14742937)(16.6308255,247.06743408)
\curveto(16.77083201,246.7174298)(16.9758318,246.4174301)(17.2458255,246.16743408)
\curveto(17.50583127,245.9174306)(17.81083097,245.69743082)(18.1608255,245.50743408)
\curveto(18.27083051,245.44743107)(18.38583039,245.39743112)(18.5058255,245.35743408)
\lineto(18.8358255,245.23743408)
\lineto(18.9558255,245.20743408)
\curveto(18.98582979,245.19743132)(19.02082976,245.18743133)(19.0608255,245.17743408)
\curveto(19.11082967,245.14743137)(19.16582961,245.12743139)(19.2258255,245.11743408)
\curveto(19.28582949,245.1174314)(19.34082944,245.1124314)(19.3908255,245.10243408)
\curveto(19.50082928,245.08243143)(19.61082917,245.05743146)(19.7208255,245.02743408)
\curveto(19.82082896,245.00743151)(19.91582886,245.00243151)(20.0058255,245.01243408)
\curveto(20.03582874,245.0124315)(20.08582869,245.00743151)(20.1558255,244.99743408)
\lineto(20.3658255,244.99743408)
\curveto(20.43582834,244.99743152)(20.50582827,245.00243151)(20.5758255,245.01243408)
\curveto(20.92582785,245.05243146)(21.22582755,245.14243137)(21.4758255,245.28243408)
\curveto(21.72582705,245.42243109)(21.93082685,245.62243089)(22.0908255,245.88243408)
\curveto(22.14082664,245.96243055)(22.1808266,246.04243047)(22.2108255,246.12243408)
\curveto(22.24082654,246.2124303)(22.27082651,246.30743021)(22.3008255,246.40743408)
\curveto(22.32082646,246.45743006)(22.32582645,246.50743001)(22.3158255,246.55743408)
\curveto(22.30582647,246.6174299)(22.31082647,246.67242984)(22.3308255,246.72243408)
\curveto(22.34082644,246.75242976)(22.34582643,246.78742973)(22.3458255,246.82743408)
\lineto(22.3458255,246.96243408)
\lineto(22.3458255,247.09743408)
\curveto(22.33582644,247.13742938)(22.33082645,247.19242932)(22.3308255,247.26243408)
\curveto(22.31082647,247.34242917)(22.29582648,247.42242909)(22.2858255,247.50243408)
\curveto(22.26582651,247.59242892)(22.24082654,247.67242884)(22.2108255,247.74243408)
\curveto(22.07082671,248.10242841)(21.89582688,248.40742811)(21.6858255,248.65743408)
\curveto(21.46582731,248.90742761)(21.19082759,249.13242738)(20.8608255,249.33243408)
\curveto(20.75082803,249.40242711)(20.64082814,249.45742706)(20.5308255,249.49743408)
\lineto(20.2008255,249.64743408)
\curveto(20.16082862,249.67742684)(20.12582865,249.69242682)(20.0958255,249.69243408)
\curveto(20.05582872,249.70242681)(20.01582876,249.7174268)(19.9758255,249.73743408)
\curveto(19.91582886,249.75742676)(19.85582892,249.77242674)(19.7958255,249.78243408)
\curveto(19.73582904,249.79242672)(19.6758291,249.80742671)(19.6158255,249.82743408)
}
}
{
\newrgbcolor{curcolor}{0 0 0}
\pscustom[linestyle=none,fillstyle=solid,fillcolor=curcolor]
{
\newpath
\moveto(19.0908255,259.84368408)
\curveto(19.19082959,259.84367558)(19.30582947,259.8236756)(19.4358255,259.78368408)
\curveto(19.55582922,259.74367568)(19.64082914,259.69367573)(19.6908255,259.63368408)
\curveto(19.73082905,259.57367585)(19.76082902,259.49367593)(19.7808255,259.39368408)
\curveto(19.79082899,259.29367613)(19.79582898,259.18367624)(19.7958255,259.06368408)
\lineto(19.7958255,258.70368408)
\curveto(19.78582899,258.59367683)(19.780829,258.49367693)(19.7808255,258.40368408)
\lineto(19.7808255,254.56368408)
\curveto(19.780829,254.48368094)(19.78582899,254.39868102)(19.7958255,254.30868408)
\curveto(19.79582898,254.22868119)(19.81082897,254.16368126)(19.8408255,254.11368408)
\curveto(19.86082892,254.06368136)(19.90082888,254.01368141)(19.9608255,253.96368408)
\lineto(20.0958255,253.87368408)
\curveto(20.14582863,253.84368158)(20.19582858,253.83368159)(20.2458255,253.84368408)
\curveto(20.29582848,253.84368158)(20.34082844,253.83868158)(20.3808255,253.82868408)
\lineto(20.5008255,253.82868408)
\lineto(20.7558255,253.82868408)
\curveto(20.83582794,253.83868158)(20.91582786,253.85368157)(20.9958255,253.87368408)
\curveto(21.53582724,254.00368142)(21.92082686,254.30868111)(22.1508255,254.78868408)
\curveto(22.1808266,254.83868058)(22.20582657,254.89868052)(22.2258255,254.96868408)
\curveto(22.24582653,255.03868038)(22.26582651,255.10368032)(22.2858255,255.16368408)
\curveto(22.29582648,255.19368023)(22.30082648,255.24368018)(22.3008255,255.31368408)
\curveto(22.34082644,255.44367998)(22.36082642,255.6236798)(22.3608255,255.85368408)
\curveto(22.36082642,256.08367934)(22.34082644,256.27367915)(22.3008255,256.42368408)
\curveto(22.26082652,256.57367885)(22.22082656,256.70867871)(22.1808255,256.82868408)
\curveto(22.13082665,256.95867846)(22.07082671,257.07867834)(22.0008255,257.18868408)
\curveto(21.93082685,257.30867811)(21.85082693,257.418678)(21.7608255,257.51868408)
\curveto(21.66082712,257.6186778)(21.55582722,257.70867771)(21.4458255,257.78868408)
\curveto(21.34582743,257.86867755)(21.24082754,257.94367748)(21.1308255,258.01368408)
\curveto(21.02082776,258.08367734)(20.94082784,258.17867724)(20.8908255,258.29868408)
\curveto(20.87082791,258.33867708)(20.85582792,258.40367702)(20.8458255,258.49368408)
\curveto(20.83582794,258.59367683)(20.83582794,258.68367674)(20.8458255,258.76368408)
\curveto(20.84582793,258.85367657)(20.85082793,258.93867648)(20.8608255,259.01868408)
\curveto(20.87082791,259.09867632)(20.89082789,259.14867627)(20.9208255,259.16868408)
\curveto(20.99082779,259.25867616)(21.10582767,259.26367616)(21.2658255,259.18368408)
\curveto(21.53582724,259.04367638)(21.775827,258.88867653)(21.9858255,258.71868408)
\curveto(22.30582647,258.45867696)(22.57082621,258.17867724)(22.7808255,257.87868408)
\curveto(22.9808258,257.58867783)(23.14582563,257.23367819)(23.2758255,256.81368408)
\curveto(23.31582546,256.70367872)(23.34082544,256.59867882)(23.3508255,256.49868408)
\curveto(23.37082541,256.39867902)(23.39082539,256.28867913)(23.4108255,256.16868408)
\curveto(23.42082536,256.1186793)(23.42582535,256.06867935)(23.4258255,256.01868408)
\curveto(23.42582535,255.97867944)(23.43082535,255.93367949)(23.4408255,255.88368408)
\lineto(23.4408255,255.73368408)
\curveto(23.45082533,255.68367974)(23.45582532,255.6236798)(23.4558255,255.55368408)
\curveto(23.45582532,255.49367993)(23.45082533,255.44367998)(23.4408255,255.40368408)
\lineto(23.4408255,255.26868408)
\curveto(23.43082535,255.2186802)(23.42582535,255.17368025)(23.4258255,255.13368408)
\curveto(23.42582535,255.09368033)(23.42082536,255.05368037)(23.4108255,255.01368408)
\curveto(23.40082538,254.96368046)(23.39082539,254.90868051)(23.3808255,254.84868408)
\curveto(23.3808254,254.79868062)(23.3758254,254.74868067)(23.3658255,254.69868408)
\curveto(23.34582543,254.60868081)(23.32082546,254.5186809)(23.2908255,254.42868408)
\curveto(23.27082551,254.34868107)(23.24582553,254.27368115)(23.2158255,254.20368408)
\curveto(23.19582558,254.16368126)(23.18582559,254.12868129)(23.1858255,254.09868408)
\curveto(23.1758256,254.06868135)(23.16082562,254.03868138)(23.1408255,254.00868408)
\curveto(23.07082571,253.86868155)(22.98582579,253.7236817)(22.8858255,253.57368408)
\curveto(22.69582608,253.3236821)(22.46582631,253.1236823)(22.1958255,252.97368408)
\curveto(21.91582686,252.8236826)(21.60582717,252.71368271)(21.2658255,252.64368408)
\curveto(21.15582762,252.61368281)(21.04082774,252.59868282)(20.9208255,252.59868408)
\curveto(20.80082798,252.59868282)(20.6808281,252.58868283)(20.5608255,252.56868408)
\lineto(20.4558255,252.56868408)
\curveto(20.42582835,252.57868284)(20.38582839,252.58368284)(20.3358255,252.58368408)
\lineto(20.0808255,252.58368408)
\curveto(19.99082879,252.59368283)(19.90082888,252.59868282)(19.8108255,252.59868408)
\lineto(19.6008255,252.64368408)
\curveto(19.56082922,252.64368278)(19.50582927,252.64868277)(19.4358255,252.65868408)
\curveto(19.35582942,252.66868275)(19.29082949,252.68368274)(19.2408255,252.70368408)
\lineto(19.0758255,252.73368408)
\curveto(19.02582975,252.76368266)(18.9758298,252.77868264)(18.9258255,252.77868408)
\curveto(18.86582991,252.78868263)(18.81082997,252.80368262)(18.7608255,252.82368408)
\curveto(18.60083018,252.89368253)(18.44083034,252.95868246)(18.2808255,253.01868408)
\curveto(18.12083066,253.07868234)(17.97083081,253.15368227)(17.8308255,253.24368408)
\curveto(17.72083106,253.31368211)(17.61083117,253.37868204)(17.5008255,253.43868408)
\curveto(17.3808314,253.50868191)(17.26583151,253.58868183)(17.1558255,253.67868408)
\curveto(16.80583197,253.96868145)(16.50583227,254.27868114)(16.2558255,254.60868408)
\curveto(15.99583278,254.93868048)(15.780833,255.3236801)(15.6108255,255.76368408)
\curveto(15.56083322,255.89367953)(15.52583325,256.0236794)(15.5058255,256.15368408)
\curveto(15.4758333,256.28367914)(15.44583333,256.423679)(15.4158255,256.57368408)
\curveto(15.40583337,256.6236788)(15.40083338,256.66867875)(15.4008255,256.70868408)
\curveto(15.39083339,256.74867867)(15.38583339,256.79367863)(15.3858255,256.84368408)
\curveto(15.3758334,256.86367856)(15.3758334,256.88867853)(15.3858255,256.91868408)
\curveto(15.39583338,256.94867847)(15.39083339,256.97367845)(15.3708255,256.99368408)
\curveto(15.36083342,257.423678)(15.40583337,257.78367764)(15.5058255,258.07368408)
\curveto(15.59583318,258.36367706)(15.72083306,258.6186768)(15.8808255,258.83868408)
\curveto(15.90083288,258.87867654)(15.93083285,258.90867651)(15.9708255,258.92868408)
\curveto(16.00083278,258.95867646)(16.02583275,258.98867643)(16.0458255,259.01868408)
\curveto(16.10583267,259.08867633)(16.1758326,259.15867626)(16.2558255,259.22868408)
\curveto(16.33583244,259.29867612)(16.41583236,259.35367607)(16.4958255,259.39368408)
\curveto(16.70583207,259.51367591)(16.90583187,259.60867581)(17.0958255,259.67868408)
\curveto(17.20583157,259.72867569)(17.32583145,259.75867566)(17.4558255,259.76868408)
\lineto(17.8458255,259.82868408)
\curveto(17.9758308,259.85867556)(18.11083067,259.86867555)(18.2508255,259.85868408)
\curveto(18.39083039,259.85867556)(18.53083025,259.86367556)(18.6708255,259.87368408)
\curveto(18.74083004,259.87367555)(18.81082997,259.86867555)(18.8808255,259.85868408)
\curveto(18.95082983,259.84867557)(19.02082976,259.84367558)(19.0908255,259.84368408)
\moveto(18.5808255,258.49368408)
\curveto(18.54083024,258.5236769)(18.49083029,258.55367687)(18.4308255,258.58368408)
\curveto(18.36083042,258.6236768)(18.29083049,258.63867678)(18.2208255,258.62868408)
\curveto(18.00083078,258.6186768)(17.79583098,258.57867684)(17.6058255,258.50868408)
\curveto(17.3758314,258.40867701)(17.1808316,258.28867713)(17.0208255,258.14868408)
\curveto(16.86083192,258.0186774)(16.72583205,257.82867759)(16.6158255,257.57868408)
\curveto(16.59583218,257.50867791)(16.5808322,257.43867798)(16.5708255,257.36868408)
\curveto(16.55083223,257.30867811)(16.53083225,257.23867818)(16.5108255,257.15868408)
\curveto(16.49083229,257.08867833)(16.4808323,257.00867841)(16.4808255,256.91868408)
\lineto(16.4808255,256.66368408)
\curveto(16.50083228,256.6236788)(16.51083227,256.58367884)(16.5108255,256.54368408)
\curveto(16.50083228,256.50367892)(16.50083228,256.46867895)(16.5108255,256.43868408)
\lineto(16.5708255,256.19868408)
\curveto(16.5808322,256.1186793)(16.59583218,256.04367938)(16.6158255,255.97368408)
\curveto(16.73583204,255.65367977)(16.88583189,255.38868003)(17.0658255,255.17868408)
\curveto(17.24583153,254.96868045)(17.47083131,254.76868065)(17.7408255,254.57868408)
\curveto(17.79083099,254.53868088)(17.85583092,254.49368093)(17.9358255,254.44368408)
\curveto(18.00583077,254.40368102)(18.08583069,254.36368106)(18.1758255,254.32368408)
\curveto(18.26583051,254.28368114)(18.35083043,254.25868116)(18.4308255,254.24868408)
\curveto(18.51083027,254.24868117)(18.57083021,254.27368115)(18.6108255,254.32368408)
\curveto(18.67083011,254.39368103)(18.70083008,254.5236809)(18.7008255,254.71368408)
\curveto(18.69083009,254.91368051)(18.68583009,255.08368034)(18.6858255,255.22368408)
\lineto(18.6858255,257.50368408)
\curveto(18.68583009,257.65367777)(18.69083009,257.83367759)(18.7008255,258.04368408)
\curveto(18.70083008,258.25367717)(18.66083012,258.40367702)(18.5808255,258.49368408)
}
}
{
\newrgbcolor{curcolor}{0 0 0}
\pscustom[linestyle=none,fillstyle=solid,fillcolor=curcolor]
{
\newpath
\moveto(15.3558255,264.33032471)
\curveto(15.34583343,265.05031905)(15.43083335,265.63531847)(15.6108255,266.08532471)
\curveto(15.780833,266.54531756)(16.08583269,266.86531724)(16.5258255,267.04532471)
\curveto(16.63583214,267.09531701)(16.75083203,267.12531698)(16.8708255,267.13532471)
\curveto(16.9808318,267.15531695)(17.10583167,267.17031693)(17.2458255,267.18032471)
\curveto(17.31583146,267.19031691)(17.39083139,267.18031692)(17.4708255,267.15032471)
\curveto(17.54083124,267.13031697)(17.59583118,267.105317)(17.6358255,267.07532471)
\curveto(17.65583112,267.05531705)(17.6758311,267.02531708)(17.6958255,266.98532471)
\curveto(17.70583107,266.95531715)(17.72083106,266.93031717)(17.7408255,266.91032471)
\curveto(17.76083102,266.85031725)(17.76583101,266.79531731)(17.7558255,266.74532471)
\curveto(17.74583103,266.7053174)(17.74583103,266.66031744)(17.7558255,266.61032471)
\curveto(17.775831,266.52031758)(17.780831,266.41031769)(17.7708255,266.28032471)
\curveto(17.75083103,266.16031794)(17.72583105,266.07531803)(17.6958255,266.02532471)
\curveto(17.64583113,265.95531815)(17.5808312,265.91531819)(17.5008255,265.90532471)
\curveto(17.41083137,265.9053182)(17.32583145,265.88531822)(17.2458255,265.84532471)
\curveto(17.08583169,265.79531831)(16.94083184,265.7003184)(16.8108255,265.56032471)
\curveto(16.73083205,265.47031863)(16.67083211,265.36031874)(16.6308255,265.23032471)
\curveto(16.59083219,265.11031899)(16.55083223,264.98031912)(16.5108255,264.84032471)
\curveto(16.49083229,264.8003193)(16.48583229,264.75031935)(16.4958255,264.69032471)
\curveto(16.49583228,264.64031946)(16.49083229,264.59531951)(16.4808255,264.55532471)
\curveto(16.46083232,264.49531961)(16.45083233,264.42031968)(16.4508255,264.33032471)
\curveto(16.45083233,264.24031986)(16.46083232,264.16531994)(16.4808255,264.10532471)
\lineto(16.4808255,264.01532471)
\curveto(16.49083229,263.95532015)(16.50083228,263.9003202)(16.5108255,263.85032471)
\curveto(16.51083227,263.8003203)(16.51583226,263.75032035)(16.5258255,263.70032471)
\curveto(16.58583219,263.43032067)(16.67083211,263.19532091)(16.7808255,262.99532471)
\curveto(16.89083189,262.8053213)(17.0758317,262.65532145)(17.3358255,262.54532471)
\curveto(17.40583137,262.51532159)(17.4758313,262.5003216)(17.5458255,262.50032471)
\curveto(17.61583116,262.5003216)(17.6758311,262.5053216)(17.7258255,262.51532471)
\curveto(17.8758309,262.54532156)(17.98583079,262.59532151)(18.0558255,262.66532471)
\curveto(18.11583066,262.73532137)(18.18583059,262.83032127)(18.2658255,262.95032471)
\curveto(18.36583041,263.09032101)(18.44083034,263.25532085)(18.4908255,263.44532471)
\curveto(18.53083025,263.63532047)(18.5808302,263.82532028)(18.6408255,264.01532471)
\curveto(18.6808301,264.13531997)(18.71083007,264.25531985)(18.7308255,264.37532471)
\curveto(18.75083003,264.5053196)(18.78083,264.63031947)(18.8208255,264.75032471)
\curveto(18.8808299,264.95031915)(18.94082984,265.14531896)(19.0008255,265.33532471)
\curveto(19.05082973,265.52531858)(19.11582966,265.71031839)(19.1958255,265.89032471)
\curveto(19.21582956,265.94031816)(19.23582954,265.98531812)(19.2558255,266.02532471)
\curveto(19.2758295,266.07531803)(19.30082948,266.12531798)(19.3308255,266.17532471)
\curveto(19.45082933,266.34531776)(19.58582919,266.49031761)(19.7358255,266.61032471)
\curveto(19.88582889,266.73031737)(20.0758287,266.82031728)(20.3058255,266.88032471)
\lineto(20.5908255,266.88032471)
\curveto(20.66082812,266.88031722)(20.73582804,266.87531723)(20.8158255,266.86532471)
\curveto(20.88582789,266.85531725)(20.96582781,266.84531726)(21.0558255,266.83532471)
\lineto(21.2058255,266.80532471)
\curveto(21.2758275,266.76531734)(21.34582743,266.73531737)(21.4158255,266.71532471)
\curveto(21.48582729,266.7053174)(21.55582722,266.68531742)(21.6258255,266.65532471)
\curveto(21.73582704,266.6053175)(21.84082694,266.55031755)(21.9408255,266.49032471)
\curveto(22.04082674,266.43031767)(22.13082665,266.36531774)(22.2108255,266.29532471)
\curveto(22.47082631,266.08531802)(22.6808261,265.84031826)(22.8408255,265.56032471)
\curveto(22.99082579,265.28031882)(23.12082566,264.97531913)(23.2308255,264.64532471)
\curveto(23.26082552,264.54531956)(23.2808255,264.44531966)(23.2908255,264.34532471)
\curveto(23.31082547,264.24531986)(23.33582544,264.15031995)(23.3658255,264.06032471)
\curveto(23.38582539,263.95032015)(23.39582538,263.84532026)(23.3958255,263.74532471)
\curveto(23.39582538,263.64532046)(23.40582537,263.54532056)(23.4258255,263.44532471)
\lineto(23.4258255,263.29532471)
\curveto(23.43582534,263.24532086)(23.44082534,263.17532093)(23.4408255,263.08532471)
\curveto(23.44082534,262.99532111)(23.43582534,262.92532118)(23.4258255,262.87532471)
\lineto(23.4258255,262.71032471)
\curveto(23.40582537,262.65032145)(23.39582538,262.58532152)(23.3958255,262.51532471)
\curveto(23.40582537,262.44532166)(23.40082538,262.39032171)(23.3808255,262.35032471)
\curveto(23.37082541,262.3003218)(23.36582541,262.23532187)(23.3658255,262.15532471)
\curveto(23.34582543,262.07532203)(23.32582545,262.0003221)(23.3058255,261.93032471)
\curveto(23.29582548,261.86032224)(23.2758255,261.78532232)(23.2458255,261.70532471)
\curveto(23.14582563,261.41532269)(23.02082576,261.17032293)(22.8708255,260.97032471)
\curveto(22.72082606,260.77032333)(22.52582625,260.61032349)(22.2858255,260.49032471)
\curveto(22.15582662,260.43032367)(22.02082676,260.38032372)(21.8808255,260.34032471)
\curveto(21.74082704,260.31032379)(21.58582719,260.29032381)(21.4158255,260.28032471)
\curveto(21.35582742,260.27032383)(21.28582749,260.27532383)(21.2058255,260.29532471)
\curveto(21.11582766,260.31532379)(21.04582773,260.34032376)(20.9958255,260.37032471)
\curveto(20.95582782,260.41032369)(20.91582786,260.47032363)(20.8758255,260.55032471)
\curveto(20.85582792,260.6003235)(20.84582793,260.67032343)(20.8458255,260.76032471)
\curveto(20.83582794,260.86032324)(20.83582794,260.95032315)(20.8458255,261.03032471)
\curveto(20.85582792,261.12032298)(20.87082791,261.2053229)(20.8908255,261.28532471)
\curveto(20.90082788,261.37532273)(20.91582786,261.43032267)(20.9358255,261.45032471)
\curveto(20.98582779,261.51032259)(21.06082772,261.54032256)(21.1608255,261.54032471)
\curveto(21.25082753,261.55032255)(21.33582744,261.57032253)(21.4158255,261.60032471)
\curveto(21.63582714,261.65032245)(21.80582697,261.75032235)(21.9258255,261.90032471)
\curveto(22.01582676,262.0003221)(22.08582669,262.12032198)(22.1358255,262.26032471)
\curveto(22.18582659,262.4003217)(22.23582654,262.55032155)(22.2858255,262.71032471)
\lineto(22.3308255,263.02532471)
\lineto(22.3308255,263.11532471)
\curveto(22.35082643,263.17532093)(22.36082642,263.26032084)(22.3608255,263.37032471)
\curveto(22.36082642,263.49032061)(22.35082643,263.59532051)(22.3308255,263.68532471)
\curveto(22.33082645,263.75532035)(22.32582645,263.81032029)(22.3158255,263.85032471)
\curveto(22.30582647,263.91032019)(22.30082648,263.97032013)(22.3008255,264.03032471)
\curveto(22.29082649,264.09032001)(22.2808265,264.14531996)(22.2708255,264.19532471)
\curveto(22.19082659,264.5053196)(22.08582669,264.75531935)(21.9558255,264.94532471)
\curveto(21.82582695,265.14531896)(21.60582717,265.31031879)(21.2958255,265.44032471)
\curveto(21.24582753,265.47031863)(21.19082759,265.48531862)(21.1308255,265.48532471)
\curveto(21.07082771,265.49531861)(21.02582775,265.49531861)(20.9958255,265.48532471)
\curveto(20.80582797,265.47531863)(20.66582811,265.43531867)(20.5758255,265.36532471)
\curveto(20.4758283,265.29531881)(20.38582839,265.2003189)(20.3058255,265.08032471)
\curveto(20.24582853,265.0003191)(20.19582858,264.9053192)(20.1558255,264.79532471)
\lineto(20.0358255,264.49532471)
\curveto(20.02582875,264.46531964)(20.02082876,264.43531967)(20.0208255,264.40532471)
\curveto(20.02082876,264.38531972)(20.01082877,264.36531974)(19.9908255,264.34532471)
\curveto(19.8808289,264.02532008)(19.80082898,263.68532042)(19.7508255,263.32532471)
\curveto(19.69082909,262.97532113)(19.59582918,262.65532145)(19.4658255,262.36532471)
\curveto(19.42582935,262.27532183)(19.39082939,262.18532192)(19.3608255,262.09532471)
\curveto(19.33082945,262.01532209)(19.29082949,261.94032216)(19.2408255,261.87032471)
\curveto(19.13082965,261.7003224)(19.00582977,261.55032255)(18.8658255,261.42032471)
\curveto(18.72583005,261.29032281)(18.55083023,261.2003229)(18.3408255,261.15032471)
\curveto(18.27083051,261.13032297)(18.20083058,261.12032298)(18.1308255,261.12032471)
\lineto(17.9058255,261.12032471)
\curveto(17.78583099,261.11032299)(17.65083113,261.12532298)(17.5008255,261.16532471)
\curveto(17.34083144,261.2053229)(17.20583157,261.24532286)(17.0958255,261.28532471)
\curveto(17.04583173,261.31532279)(17.00583177,261.33532277)(16.9758255,261.34532471)
\curveto(16.93583184,261.36532274)(16.89583188,261.39032271)(16.8558255,261.42032471)
\curveto(16.62583215,261.55032255)(16.42583235,261.71032239)(16.2558255,261.90032471)
\curveto(16.08583269,262.09032201)(15.93583284,262.3003218)(15.8058255,262.53032471)
\curveto(15.71583306,262.69032141)(15.64583313,262.86532124)(15.5958255,263.05532471)
\curveto(15.53583324,263.25532085)(15.4808333,263.46032064)(15.4308255,263.67032471)
\curveto(15.42083336,263.74032036)(15.41083337,263.8053203)(15.4008255,263.86532471)
\curveto(15.39083339,263.93532017)(15.3808334,264.01032009)(15.3708255,264.09032471)
\curveto(15.36083342,264.13031997)(15.36083342,264.17031993)(15.3708255,264.21032471)
\curveto(15.3808334,264.26031984)(15.3758334,264.3003198)(15.3558255,264.33032471)
}
}
{
\newrgbcolor{curcolor}{0 0 0}
\pscustom[linestyle=none,fillstyle=solid,fillcolor=curcolor]
{
\newpath
\moveto(118.89923523,72.61)
\curveto(118.9192261,72.52999222)(118.92922609,72.41999233)(118.92923523,72.28)
\curveto(118.92922609,72.1499926)(118.9192261,72.0499927)(118.89923523,71.98)
\curveto(118.87922614,71.90999284)(118.87422615,71.84499291)(118.88423523,71.785)
\curveto(118.89422613,71.72499303)(118.88922613,71.65999309)(118.86923523,71.59)
\curveto(118.84922617,71.52999322)(118.83422619,71.46499328)(118.82423523,71.395)
\curveto(118.81422621,71.33499341)(118.79922622,71.27499347)(118.77923523,71.215)
\curveto(118.75922626,71.13499361)(118.73422629,71.05999369)(118.70423523,70.99)
\curveto(118.68422634,70.91999383)(118.65922636,70.8499939)(118.62923523,70.78)
\curveto(118.60922641,70.749994)(118.59422643,70.71999403)(118.58423523,70.69)
\curveto(118.58422644,70.66999408)(118.57422645,70.6499941)(118.55423523,70.63)
\curveto(118.44422658,70.42999432)(118.3242267,70.2499945)(118.19423523,70.09)
\curveto(118.17422685,70.0499947)(118.13922688,70.00999474)(118.08923523,69.97)
\curveto(118.04922697,69.92999482)(118.01422701,69.89999485)(117.98423523,69.88)
\curveto(117.94422708,69.85999489)(117.90922711,69.82999492)(117.87923523,69.79)
\curveto(117.84922717,69.75999499)(117.8192272,69.73499501)(117.78923523,69.715)
\lineto(117.47423523,69.535)
\curveto(117.36422766,69.4549953)(117.23422779,69.39499535)(117.08423523,69.355)
\lineto(116.63423523,69.235)
\curveto(116.55422847,69.21499554)(116.47422855,69.19999555)(116.39423523,69.19)
\curveto(116.31422871,69.18999556)(116.23422879,69.17999557)(116.15423523,69.16)
\curveto(116.11422891,69.1499956)(116.07422895,69.14499561)(116.03423523,69.145)
\curveto(116.00422902,69.1549956)(115.97422905,69.1549956)(115.94423523,69.145)
\curveto(115.89422913,69.13499561)(115.84422918,69.13499561)(115.79423523,69.145)
\curveto(115.75422927,69.1549956)(115.70922931,69.1549956)(115.65923523,69.145)
\lineto(113.39423523,69.145)
\lineto(112.89923523,69.145)
\curveto(112.72923229,69.1549956)(112.59923242,69.12499562)(112.50923523,69.055)
\curveto(112.39923262,68.97499578)(112.34423268,68.82999592)(112.34423523,68.62)
\curveto(112.35423267,68.40999634)(112.35923266,68.21499654)(112.35923523,68.035)
\lineto(112.35923523,65.83)
\lineto(112.35923523,65.335)
\curveto(112.36923265,65.14499961)(112.34923267,65.00999974)(112.29923523,64.93)
\curveto(112.25923276,64.86999988)(112.20923281,64.82999992)(112.14923523,64.81)
\curveto(112.09923292,64.79999995)(112.03423299,64.78499996)(111.95423523,64.765)
\lineto(111.68423523,64.765)
\curveto(111.53423349,64.76499998)(111.39923362,64.76999998)(111.27923523,64.78)
\curveto(111.15923386,64.78999996)(111.07423395,64.83999991)(111.02423523,64.93)
\curveto(110.98423404,64.98999976)(110.96423406,65.06999968)(110.96423523,65.17)
\lineto(110.96423523,65.485)
\lineto(110.96423523,74.59)
\curveto(110.96423406,74.69999005)(110.95923406,74.81998993)(110.94923523,74.95)
\curveto(110.94923407,75.08998966)(110.97423405,75.19998955)(111.02423523,75.28)
\curveto(111.06423396,75.33998941)(111.13923388,75.38998936)(111.24923523,75.43)
\curveto(111.26923375,75.43998931)(111.28923373,75.43998931)(111.30923523,75.43)
\curveto(111.32923369,75.42998932)(111.34923367,75.43498931)(111.36923523,75.445)
\lineto(114.77423523,75.445)
\curveto(115.15422987,75.4449893)(115.5242295,75.43998931)(115.88423523,75.43)
\curveto(116.25422877,75.42998932)(116.58422844,75.38498937)(116.87423523,75.295)
\curveto(117.3242277,75.1449896)(117.68922733,74.9499898)(117.96923523,74.71)
\curveto(118.24922677,74.46999028)(118.47922654,74.13999061)(118.65923523,73.72)
\curveto(118.70922631,73.60999114)(118.74422628,73.49499126)(118.76423523,73.375)
\curveto(118.79422623,73.2549915)(118.82922619,73.12999162)(118.86923523,73)
\curveto(118.88922613,72.92999182)(118.89422613,72.86499188)(118.88423523,72.805)
\curveto(118.87422615,72.744992)(118.87922614,72.67999207)(118.89923523,72.61)
\moveto(117.48923523,72.07)
\curveto(117.52922749,72.20999254)(117.53422749,72.36999238)(117.50423523,72.55)
\curveto(117.47422755,72.73999201)(117.44422758,72.88999186)(117.41423523,73)
\curveto(117.31422771,73.27999147)(117.17922784,73.49999125)(117.00923523,73.66)
\curveto(116.84922817,73.82999092)(116.63922838,73.96999078)(116.37923523,74.08)
\curveto(116.15922886,74.16999058)(115.90422912,74.22499053)(115.61423523,74.245)
\curveto(115.33422969,74.26499048)(115.03922998,74.27499047)(114.72923523,74.275)
\lineto(112.79423523,74.275)
\curveto(112.77423225,74.26499048)(112.74923227,74.25999049)(112.71923523,74.26)
\curveto(112.69923232,74.25999049)(112.67423235,74.2549905)(112.64423523,74.245)
\curveto(112.5242325,74.21499054)(112.44423258,74.1499906)(112.40423523,74.05)
\curveto(112.36423266,73.9499908)(112.34423268,73.81499094)(112.34423523,73.645)
\curveto(112.35423267,73.48499127)(112.35923266,73.33499141)(112.35923523,73.195)
\lineto(112.35923523,71.395)
\curveto(112.35923266,71.24499351)(112.35423267,71.07999367)(112.34423523,70.9)
\curveto(112.34423268,70.71999403)(112.37423265,70.57999417)(112.43423523,70.48)
\curveto(112.48423254,70.39999435)(112.55923246,70.3499944)(112.65923523,70.33)
\curveto(112.76923225,70.31999443)(112.88923213,70.31499444)(113.01923523,70.315)
\lineto(115.04423523,70.315)
\lineto(115.50923523,70.315)
\curveto(115.66922935,70.32499442)(115.80922921,70.34499441)(115.92923523,70.375)
\curveto(116.19922882,70.44499431)(116.43422859,70.52499423)(116.63423523,70.615)
\curveto(116.84422818,70.71499404)(117.019228,70.86499388)(117.15923523,71.065)
\curveto(117.23922778,71.18499357)(117.29922772,71.30999344)(117.33923523,71.44)
\curveto(117.38922763,71.56999318)(117.43422759,71.71499304)(117.47423523,71.875)
\curveto(117.48422754,71.91499284)(117.48922753,71.97999277)(117.48923523,72.07)
}
}
{
\newrgbcolor{curcolor}{0 0 0}
\pscustom[linestyle=none,fillstyle=solid,fillcolor=curcolor]
{
\newpath
\moveto(127.56079773,68.965)
\curveto(127.58078967,68.90499584)(127.59078966,68.80999594)(127.59079773,68.68)
\curveto(127.59078966,68.55999619)(127.58578966,68.47499628)(127.57579773,68.425)
\lineto(127.57579773,68.275)
\curveto(127.56578968,68.19499655)(127.55578969,68.11999663)(127.54579773,68.05)
\curveto(127.5457897,67.98999676)(127.54078971,67.91999683)(127.53079773,67.84)
\curveto(127.51078974,67.77999697)(127.49578975,67.71999703)(127.48579773,67.66)
\curveto(127.48578976,67.59999715)(127.47578977,67.53999721)(127.45579773,67.48)
\curveto(127.41578983,67.3499974)(127.38078987,67.21999753)(127.35079773,67.09)
\curveto(127.32078993,66.95999779)(127.28078997,66.83999791)(127.23079773,66.73)
\curveto(127.02079023,66.2499985)(126.74079051,65.84499891)(126.39079773,65.515)
\curveto(126.04079121,65.19499955)(125.61079164,64.9499998)(125.10079773,64.78)
\curveto(124.99079226,64.74000001)(124.87079238,64.71000004)(124.74079773,64.69)
\curveto(124.62079263,64.67000008)(124.49579275,64.6500001)(124.36579773,64.63)
\curveto(124.30579294,64.62000013)(124.24079301,64.61500014)(124.17079773,64.615)
\curveto(124.11079314,64.60500015)(124.0507932,64.60000015)(123.99079773,64.6)
\curveto(123.9507933,64.59000016)(123.89079336,64.58500016)(123.81079773,64.585)
\curveto(123.74079351,64.58500016)(123.69079356,64.59000016)(123.66079773,64.6)
\curveto(123.62079363,64.61000014)(123.58079367,64.61500014)(123.54079773,64.615)
\curveto(123.50079375,64.60500015)(123.46579378,64.60500015)(123.43579773,64.615)
\lineto(123.34579773,64.615)
\lineto(122.98579773,64.66)
\curveto(122.8457944,64.70000005)(122.71079454,64.74000001)(122.58079773,64.78)
\curveto(122.4507948,64.81999993)(122.32579492,64.86499988)(122.20579773,64.915)
\curveto(121.75579549,65.11499964)(121.38579586,65.37499938)(121.09579773,65.695)
\curveto(120.80579644,66.01499874)(120.56579668,66.40499835)(120.37579773,66.865)
\curveto(120.32579692,66.96499778)(120.28579696,67.06499768)(120.25579773,67.165)
\curveto(120.23579701,67.26499748)(120.21579703,67.36999738)(120.19579773,67.48)
\curveto(120.17579707,67.51999723)(120.16579708,67.5499972)(120.16579773,67.57)
\curveto(120.17579707,67.59999715)(120.17579707,67.63499711)(120.16579773,67.675)
\curveto(120.1457971,67.75499699)(120.13079712,67.83499691)(120.12079773,67.915)
\curveto(120.12079713,68.00499675)(120.11079714,68.08999666)(120.09079773,68.17)
\lineto(120.09079773,68.29)
\curveto(120.09079716,68.32999642)(120.08579716,68.37499638)(120.07579773,68.425)
\curveto(120.06579718,68.47499628)(120.06079719,68.55999619)(120.06079773,68.68)
\curveto(120.06079719,68.80999594)(120.07079718,68.90499584)(120.09079773,68.965)
\curveto(120.11079714,69.03499571)(120.11579713,69.10499564)(120.10579773,69.175)
\curveto(120.09579715,69.24499551)(120.10079715,69.31499544)(120.12079773,69.385)
\curveto(120.13079712,69.43499531)(120.13579711,69.47499528)(120.13579773,69.505)
\curveto(120.1457971,69.54499521)(120.15579709,69.58999516)(120.16579773,69.64)
\curveto(120.19579705,69.75999499)(120.22079703,69.87999487)(120.24079773,70)
\curveto(120.27079698,70.11999463)(120.31079694,70.23499451)(120.36079773,70.345)
\curveto(120.51079674,70.71499404)(120.69079656,71.04499371)(120.90079773,71.335)
\curveto(121.12079613,71.63499311)(121.38579586,71.88499287)(121.69579773,72.085)
\curveto(121.81579543,72.16499258)(121.94079531,72.22999252)(122.07079773,72.28)
\curveto(122.20079505,72.33999241)(122.33579491,72.39999235)(122.47579773,72.46)
\curveto(122.59579465,72.50999224)(122.72579452,72.53999221)(122.86579773,72.55)
\curveto(123.00579424,72.56999218)(123.1457941,72.59999215)(123.28579773,72.64)
\lineto(123.48079773,72.64)
\curveto(123.5507937,72.6499921)(123.61579363,72.65999209)(123.67579773,72.67)
\curveto(124.56579268,72.67999207)(125.30579194,72.49499225)(125.89579773,72.115)
\curveto(126.48579076,71.73499301)(126.91079034,71.23999351)(127.17079773,70.63)
\curveto(127.22079003,70.52999422)(127.26078999,70.42999432)(127.29079773,70.33)
\curveto(127.32078993,70.22999452)(127.35578989,70.12499462)(127.39579773,70.015)
\curveto(127.42578982,69.90499484)(127.4507898,69.78499497)(127.47079773,69.655)
\curveto(127.49078976,69.53499521)(127.51578973,69.40999534)(127.54579773,69.28)
\curveto(127.55578969,69.22999552)(127.55578969,69.17499558)(127.54579773,69.115)
\curveto(127.5457897,69.06499568)(127.5507897,69.01499574)(127.56079773,68.965)
\moveto(126.22579773,68.11)
\curveto(126.245791,68.17999657)(126.250791,68.25999649)(126.24079773,68.35)
\lineto(126.24079773,68.605)
\curveto(126.24079101,68.99499575)(126.20579104,69.32499542)(126.13579773,69.595)
\curveto(126.10579114,69.67499508)(126.08079117,69.754995)(126.06079773,69.835)
\curveto(126.04079121,69.91499484)(126.01579123,69.98999476)(125.98579773,70.06)
\curveto(125.70579154,70.70999404)(125.26079199,71.15999359)(124.65079773,71.41)
\curveto(124.58079267,71.43999331)(124.50579274,71.45999329)(124.42579773,71.47)
\lineto(124.18579773,71.53)
\curveto(124.10579314,71.5499932)(124.02079323,71.55999319)(123.93079773,71.56)
\lineto(123.66079773,71.56)
\lineto(123.39079773,71.515)
\curveto(123.29079396,71.49499325)(123.19579405,71.46999328)(123.10579773,71.44)
\curveto(123.02579422,71.41999333)(122.9457943,71.38999336)(122.86579773,71.35)
\curveto(122.79579445,71.32999342)(122.73079452,71.29999345)(122.67079773,71.26)
\curveto(122.61079464,71.21999353)(122.55579469,71.17999357)(122.50579773,71.14)
\curveto(122.26579498,70.96999378)(122.07079518,70.76499398)(121.92079773,70.525)
\curveto(121.77079548,70.28499447)(121.64079561,70.00499474)(121.53079773,69.685)
\curveto(121.50079575,69.58499517)(121.48079577,69.47999527)(121.47079773,69.37)
\curveto(121.46079579,69.26999548)(121.4457958,69.16499558)(121.42579773,69.055)
\curveto(121.41579583,69.01499574)(121.41079584,68.9499958)(121.41079773,68.86)
\curveto(121.40079585,68.82999592)(121.39579585,68.79499595)(121.39579773,68.755)
\curveto(121.40579584,68.71499604)(121.41079584,68.66999608)(121.41079773,68.62)
\lineto(121.41079773,68.32)
\curveto(121.41079584,68.21999653)(121.42079583,68.12999662)(121.44079773,68.05)
\lineto(121.47079773,67.87)
\curveto(121.49079576,67.76999698)(121.50579574,67.66999708)(121.51579773,67.57)
\curveto(121.53579571,67.47999727)(121.56579568,67.39499735)(121.60579773,67.315)
\curveto(121.70579554,67.07499768)(121.82079543,66.8499979)(121.95079773,66.64)
\curveto(122.09079516,66.42999832)(122.26079499,66.25499849)(122.46079773,66.115)
\curveto(122.51079474,66.08499867)(122.55579469,66.05999869)(122.59579773,66.04)
\curveto(122.63579461,66.01999873)(122.68079457,65.99499875)(122.73079773,65.965)
\curveto(122.81079444,65.91499884)(122.89579435,65.86999888)(122.98579773,65.83)
\curveto(123.08579416,65.79999895)(123.19079406,65.76999898)(123.30079773,65.74)
\curveto(123.3507939,65.71999903)(123.39579385,65.70999904)(123.43579773,65.71)
\curveto(123.48579376,65.71999903)(123.53579371,65.71999903)(123.58579773,65.71)
\curveto(123.61579363,65.69999905)(123.67579357,65.68999906)(123.76579773,65.68)
\curveto(123.86579338,65.66999908)(123.94079331,65.67499908)(123.99079773,65.695)
\curveto(124.03079322,65.70499905)(124.07079318,65.70499905)(124.11079773,65.695)
\curveto(124.1507931,65.69499905)(124.19079306,65.70499905)(124.23079773,65.725)
\curveto(124.31079294,65.74499901)(124.39079286,65.75999899)(124.47079773,65.77)
\curveto(124.5507927,65.78999896)(124.62579262,65.81499894)(124.69579773,65.845)
\curveto(125.03579221,65.98499877)(125.31079194,66.17999857)(125.52079773,66.43)
\curveto(125.73079152,66.67999807)(125.90579134,66.97499778)(126.04579773,67.315)
\curveto(126.09579115,67.43499731)(126.12579112,67.55999719)(126.13579773,67.69)
\curveto(126.15579109,67.82999692)(126.18579106,67.96999678)(126.22579773,68.11)
}
}
{
\newrgbcolor{curcolor}{0 0 0}
\pscustom[linestyle=none,fillstyle=solid,fillcolor=curcolor]
{
\newpath
\moveto(132.69407898,72.67)
\curveto(132.92407419,72.66999208)(133.05407406,72.60999214)(133.08407898,72.49)
\curveto(133.114074,72.37999237)(133.12907398,72.21499254)(133.12907898,71.995)
\lineto(133.12907898,71.71)
\curveto(133.12907398,71.61999313)(133.10407401,71.54499321)(133.05407898,71.485)
\curveto(132.99407412,71.40499334)(132.9090742,71.35999339)(132.79907898,71.35)
\curveto(132.68907442,71.3499934)(132.57907453,71.33499341)(132.46907898,71.305)
\curveto(132.32907478,71.27499347)(132.19407492,71.24499351)(132.06407898,71.215)
\curveto(131.94407517,71.18499357)(131.82907528,71.14499361)(131.71907898,71.095)
\curveto(131.42907568,70.96499378)(131.19407592,70.78499397)(131.01407898,70.555)
\curveto(130.83407628,70.33499441)(130.67907643,70.07999467)(130.54907898,69.79)
\curveto(130.5090766,69.67999507)(130.47907663,69.56499518)(130.45907898,69.445)
\curveto(130.43907667,69.33499541)(130.4140767,69.21999553)(130.38407898,69.1)
\curveto(130.37407674,69.0499957)(130.36907674,68.99999575)(130.36907898,68.95)
\curveto(130.37907673,68.89999585)(130.37907673,68.8499959)(130.36907898,68.8)
\curveto(130.33907677,68.67999607)(130.32407679,68.53999621)(130.32407898,68.38)
\curveto(130.33407678,68.22999652)(130.33907677,68.08499667)(130.33907898,67.945)
\lineto(130.33907898,66.1)
\lineto(130.33907898,65.755)
\curveto(130.33907677,65.63499912)(130.33407678,65.51999923)(130.32407898,65.41)
\curveto(130.3140768,65.29999945)(130.3090768,65.20499955)(130.30907898,65.125)
\curveto(130.31907679,65.04499971)(130.29907681,64.97499978)(130.24907898,64.915)
\curveto(130.19907691,64.84499991)(130.11907699,64.80499995)(130.00907898,64.795)
\curveto(129.9090772,64.78499996)(129.79907731,64.77999997)(129.67907898,64.78)
\lineto(129.40907898,64.78)
\curveto(129.35907775,64.79999995)(129.3090778,64.81499994)(129.25907898,64.825)
\curveto(129.21907789,64.84499991)(129.18907792,64.86999988)(129.16907898,64.9)
\curveto(129.11907799,64.96999978)(129.08907802,65.05499969)(129.07907898,65.155)
\lineto(129.07907898,65.485)
\lineto(129.07907898,66.64)
\lineto(129.07907898,70.795)
\lineto(129.07907898,71.83)
\lineto(129.07907898,72.13)
\curveto(129.08907802,72.22999252)(129.11907799,72.31499244)(129.16907898,72.385)
\curveto(129.19907791,72.42499233)(129.24907786,72.4549923)(129.31907898,72.475)
\curveto(129.39907771,72.49499225)(129.48407763,72.50499224)(129.57407898,72.505)
\curveto(129.66407745,72.51499224)(129.75407736,72.51499224)(129.84407898,72.505)
\curveto(129.93407718,72.49499225)(130.00407711,72.47999227)(130.05407898,72.46)
\curveto(130.13407698,72.42999232)(130.18407693,72.36999238)(130.20407898,72.28)
\curveto(130.23407688,72.19999255)(130.24907686,72.10999264)(130.24907898,72.01)
\lineto(130.24907898,71.71)
\curveto(130.24907686,71.60999314)(130.26907684,71.51999323)(130.30907898,71.44)
\curveto(130.31907679,71.41999333)(130.32907678,71.40499334)(130.33907898,71.395)
\lineto(130.38407898,71.35)
\curveto(130.49407662,71.3499934)(130.58407653,71.39499335)(130.65407898,71.485)
\curveto(130.72407639,71.58499317)(130.78407633,71.66499308)(130.83407898,71.725)
\lineto(130.92407898,71.815)
\curveto(131.0140761,71.92499283)(131.13907597,72.03999271)(131.29907898,72.16)
\curveto(131.45907565,72.27999247)(131.6090755,72.36999238)(131.74907898,72.43)
\curveto(131.83907527,72.47999227)(131.93407518,72.51499224)(132.03407898,72.535)
\curveto(132.13407498,72.56499218)(132.23907487,72.59499215)(132.34907898,72.625)
\curveto(132.4090747,72.63499211)(132.46907464,72.63999211)(132.52907898,72.64)
\curveto(132.58907452,72.6499921)(132.64407447,72.65999209)(132.69407898,72.67)
}
}
{
\newrgbcolor{curcolor}{0 0 0}
\pscustom[linestyle=none,fillstyle=solid,fillcolor=curcolor]
{
\newpath
\moveto(135.0038446,74.83)
\curveto(135.15384259,74.82998992)(135.30384244,74.82498993)(135.4538446,74.815)
\curveto(135.60384214,74.81498993)(135.70884204,74.77498997)(135.7688446,74.695)
\curveto(135.81884193,74.63499011)(135.8438419,74.5499902)(135.8438446,74.44)
\curveto(135.85384189,74.33999041)(135.85884189,74.23499051)(135.8588446,74.125)
\lineto(135.8588446,73.255)
\curveto(135.85884189,73.17499157)(135.85384189,73.08999166)(135.8438446,73)
\curveto(135.8438419,72.91999183)(135.85384189,72.8499919)(135.8738446,72.79)
\curveto(135.91384183,72.6499921)(136.00384174,72.55999219)(136.1438446,72.52)
\curveto(136.19384155,72.50999224)(136.23884151,72.50499224)(136.2788446,72.505)
\lineto(136.4288446,72.505)
\lineto(136.8338446,72.505)
\curveto(136.99384075,72.51499224)(137.10884064,72.50499224)(137.1788446,72.475)
\curveto(137.26884048,72.41499234)(137.32884042,72.3549924)(137.3588446,72.295)
\curveto(137.37884037,72.2549925)(137.38884036,72.20999254)(137.3888446,72.16)
\lineto(137.3888446,72.01)
\curveto(137.38884036,71.89999285)(137.38384036,71.79499295)(137.3738446,71.695)
\curveto(137.36384038,71.60499314)(137.32884042,71.53499321)(137.2688446,71.485)
\curveto(137.20884054,71.43499331)(137.12384062,71.40499334)(137.0138446,71.395)
\lineto(136.6838446,71.395)
\curveto(136.57384117,71.40499334)(136.46384128,71.40999334)(136.3538446,71.41)
\curveto(136.2438415,71.40999334)(136.1488416,71.39499335)(136.0688446,71.365)
\curveto(135.99884175,71.33499341)(135.9488418,71.28499347)(135.9188446,71.215)
\curveto(135.88884186,71.14499361)(135.86884188,71.05999369)(135.8588446,70.96)
\curveto(135.8488419,70.86999388)(135.8438419,70.76999398)(135.8438446,70.66)
\curveto(135.85384189,70.55999419)(135.85884189,70.45999429)(135.8588446,70.36)
\lineto(135.8588446,67.39)
\curveto(135.85884189,67.16999758)(135.85384189,66.93499781)(135.8438446,66.685)
\curveto(135.8438419,66.44499831)(135.88884186,66.25999849)(135.9788446,66.13)
\curveto(136.02884172,66.0499987)(136.09384165,65.99499875)(136.1738446,65.965)
\curveto(136.25384149,65.93499881)(136.3488414,65.90999884)(136.4588446,65.89)
\curveto(136.48884126,65.87999887)(136.51884123,65.87499888)(136.5488446,65.875)
\curveto(136.58884116,65.88499887)(136.62384112,65.88499887)(136.6538446,65.875)
\lineto(136.8488446,65.875)
\curveto(136.9488408,65.87499888)(137.03884071,65.86499888)(137.1188446,65.845)
\curveto(137.20884054,65.83499892)(137.27384047,65.79999895)(137.3138446,65.74)
\curveto(137.33384041,65.70999904)(137.3488404,65.65499909)(137.3588446,65.575)
\curveto(137.37884037,65.50499925)(137.38884036,65.42999932)(137.3888446,65.35)
\curveto(137.39884035,65.26999948)(137.39884035,65.18999956)(137.3888446,65.11)
\curveto(137.37884037,65.03999971)(137.35884039,64.98499976)(137.3288446,64.945)
\curveto(137.28884046,64.87499988)(137.21384053,64.82499992)(137.1038446,64.795)
\curveto(137.02384072,64.77499998)(136.93384081,64.76499998)(136.8338446,64.765)
\curveto(136.73384101,64.77499998)(136.6438411,64.77999997)(136.5638446,64.78)
\curveto(136.50384124,64.77999997)(136.4438413,64.77499998)(136.3838446,64.765)
\curveto(136.32384142,64.76499998)(136.26884148,64.76999998)(136.2188446,64.78)
\lineto(136.0388446,64.78)
\curveto(135.98884176,64.78999996)(135.93884181,64.79499995)(135.8888446,64.795)
\curveto(135.8488419,64.80499995)(135.80384194,64.80999994)(135.7538446,64.81)
\curveto(135.55384219,64.85999989)(135.37884237,64.91499984)(135.2288446,64.975)
\curveto(135.08884266,65.03499972)(134.96884278,65.13999961)(134.8688446,65.29)
\curveto(134.72884302,65.48999926)(134.6488431,65.73999901)(134.6288446,66.04)
\curveto(134.60884314,66.3499984)(134.59884315,66.67999807)(134.5988446,67.03)
\lineto(134.5988446,70.96)
\curveto(134.56884318,71.08999366)(134.53884321,71.18499357)(134.5088446,71.245)
\curveto(134.48884326,71.30499344)(134.41884333,71.3549934)(134.2988446,71.395)
\curveto(134.25884349,71.40499334)(134.21884353,71.40499334)(134.1788446,71.395)
\curveto(134.13884361,71.38499337)(134.09884365,71.38999336)(134.0588446,71.41)
\lineto(133.8188446,71.41)
\curveto(133.68884406,71.40999334)(133.57884417,71.41999333)(133.4888446,71.44)
\curveto(133.40884434,71.46999328)(133.35384439,71.52999322)(133.3238446,71.62)
\curveto(133.30384444,71.65999309)(133.28884446,71.70499304)(133.2788446,71.755)
\lineto(133.2788446,71.905)
\curveto(133.27884447,72.04499271)(133.28884446,72.15999259)(133.3088446,72.25)
\curveto(133.32884442,72.3499924)(133.38884436,72.42499233)(133.4888446,72.475)
\curveto(133.59884415,72.51499224)(133.73884401,72.52499223)(133.9088446,72.505)
\curveto(134.08884366,72.48499227)(134.23884351,72.49499225)(134.3588446,72.535)
\curveto(134.4488433,72.58499217)(134.51884323,72.6549921)(134.5688446,72.745)
\curveto(134.58884316,72.80499194)(134.59884315,72.87999187)(134.5988446,72.97)
\lineto(134.5988446,73.225)
\lineto(134.5988446,74.155)
\lineto(134.5988446,74.395)
\curveto(134.59884315,74.48499027)(134.60884314,74.55999019)(134.6288446,74.62)
\curveto(134.66884308,74.69999005)(134.743843,74.76498998)(134.8538446,74.815)
\curveto(134.88384286,74.81498993)(134.90884284,74.81498993)(134.9288446,74.815)
\curveto(134.95884279,74.82498993)(134.98384276,74.82998992)(135.0038446,74.83)
}
}
{
\newrgbcolor{curcolor}{0 0 0}
\pscustom[linestyle=none,fillstyle=solid,fillcolor=curcolor]
{
\newpath
\moveto(145.66064148,65.32)
\curveto(145.69063365,65.15999959)(145.67563366,65.02499972)(145.61564148,64.915)
\curveto(145.55563378,64.81499994)(145.47563386,64.74000001)(145.37564148,64.69)
\curveto(145.32563401,64.67000008)(145.27063407,64.66000009)(145.21064148,64.66)
\curveto(145.16063418,64.66000009)(145.10563423,64.6500001)(145.04564148,64.63)
\curveto(144.82563451,64.58000017)(144.60563473,64.59500015)(144.38564148,64.675)
\curveto(144.17563516,64.74500001)(144.03063531,64.83499992)(143.95064148,64.945)
\curveto(143.90063544,65.01499974)(143.85563548,65.09499965)(143.81564148,65.185)
\curveto(143.77563556,65.28499946)(143.72563561,65.36499938)(143.66564148,65.425)
\curveto(143.64563569,65.44499931)(143.62063572,65.46499928)(143.59064148,65.485)
\curveto(143.57063577,65.50499925)(143.5406358,65.50999924)(143.50064148,65.5)
\curveto(143.39063595,65.46999928)(143.28563605,65.41499934)(143.18564148,65.335)
\curveto(143.09563624,65.25499949)(143.00563633,65.18499956)(142.91564148,65.125)
\curveto(142.78563655,65.04499971)(142.64563669,64.96999978)(142.49564148,64.9)
\curveto(142.34563699,64.83999991)(142.18563715,64.78499996)(142.01564148,64.735)
\curveto(141.91563742,64.70500005)(141.80563753,64.68500006)(141.68564148,64.675)
\curveto(141.57563776,64.66500008)(141.46563787,64.6500001)(141.35564148,64.63)
\curveto(141.30563803,64.62000013)(141.26063808,64.61500014)(141.22064148,64.615)
\lineto(141.11564148,64.615)
\curveto(141.00563833,64.59500015)(140.90063844,64.59500015)(140.80064148,64.615)
\lineto(140.66564148,64.615)
\curveto(140.61563872,64.62500012)(140.56563877,64.63000012)(140.51564148,64.63)
\curveto(140.46563887,64.63000012)(140.42063892,64.64000011)(140.38064148,64.66)
\curveto(140.340639,64.67000008)(140.30563903,64.67500008)(140.27564148,64.675)
\curveto(140.25563908,64.66500008)(140.23063911,64.66500008)(140.20064148,64.675)
\lineto(139.96064148,64.735)
\curveto(139.88063946,64.74500001)(139.80563953,64.76499998)(139.73564148,64.795)
\curveto(139.4356399,64.92499982)(139.19064015,65.06999968)(139.00064148,65.23)
\curveto(138.82064052,65.39999935)(138.67064067,65.63499912)(138.55064148,65.935)
\curveto(138.46064088,66.15499859)(138.41564092,66.41999833)(138.41564148,66.73)
\lineto(138.41564148,67.045)
\curveto(138.42564091,67.09499765)(138.43064091,67.14499761)(138.43064148,67.195)
\lineto(138.46064148,67.375)
\lineto(138.58064148,67.705)
\curveto(138.62064072,67.81499694)(138.67064067,67.91499684)(138.73064148,68.005)
\curveto(138.91064043,68.29499645)(139.15564018,68.50999624)(139.46564148,68.65)
\curveto(139.77563956,68.78999596)(140.11563922,68.91499584)(140.48564148,69.025)
\curveto(140.62563871,69.06499568)(140.77063857,69.09499565)(140.92064148,69.115)
\curveto(141.07063827,69.13499561)(141.22063812,69.15999559)(141.37064148,69.19)
\curveto(141.4406379,69.20999554)(141.50563783,69.21999553)(141.56564148,69.22)
\curveto(141.6356377,69.21999553)(141.71063763,69.22999552)(141.79064148,69.25)
\curveto(141.86063748,69.26999548)(141.93063741,69.27999547)(142.00064148,69.28)
\curveto(142.07063727,69.28999546)(142.14563719,69.30499544)(142.22564148,69.325)
\curveto(142.47563686,69.38499537)(142.71063663,69.43499531)(142.93064148,69.475)
\curveto(143.15063619,69.52499522)(143.32563601,69.63999511)(143.45564148,69.82)
\curveto(143.51563582,69.89999485)(143.56563577,69.99999475)(143.60564148,70.12)
\curveto(143.64563569,70.2499945)(143.64563569,70.38999436)(143.60564148,70.54)
\curveto(143.54563579,70.77999397)(143.45563588,70.96999378)(143.33564148,71.11)
\curveto(143.22563611,71.2499935)(143.06563627,71.35999339)(142.85564148,71.44)
\curveto(142.7356366,71.48999326)(142.59063675,71.52499323)(142.42064148,71.545)
\curveto(142.26063708,71.56499318)(142.09063725,71.57499317)(141.91064148,71.575)
\curveto(141.73063761,71.57499317)(141.55563778,71.56499318)(141.38564148,71.545)
\curveto(141.21563812,71.52499323)(141.07063827,71.49499325)(140.95064148,71.455)
\curveto(140.78063856,71.39499335)(140.61563872,71.30999344)(140.45564148,71.2)
\curveto(140.37563896,71.13999361)(140.30063904,71.05999369)(140.23064148,70.96)
\curveto(140.17063917,70.86999388)(140.11563922,70.76999398)(140.06564148,70.66)
\curveto(140.0356393,70.57999417)(140.00563933,70.49499425)(139.97564148,70.405)
\curveto(139.95563938,70.31499444)(139.91063943,70.24499451)(139.84064148,70.195)
\curveto(139.80063954,70.16499458)(139.73063961,70.13999461)(139.63064148,70.12)
\curveto(139.5406398,70.10999464)(139.44563989,70.10499464)(139.34564148,70.105)
\curveto(139.24564009,70.10499464)(139.14564019,70.10999464)(139.04564148,70.12)
\curveto(138.95564038,70.13999461)(138.89064045,70.16499458)(138.85064148,70.195)
\curveto(138.81064053,70.22499452)(138.78064056,70.27499447)(138.76064148,70.345)
\curveto(138.7406406,70.41499434)(138.7406406,70.48999426)(138.76064148,70.57)
\curveto(138.79064055,70.69999405)(138.82064052,70.81999393)(138.85064148,70.93)
\curveto(138.89064045,71.0499937)(138.9356404,71.16499358)(138.98564148,71.275)
\curveto(139.17564016,71.62499313)(139.41563992,71.89499285)(139.70564148,72.085)
\curveto(139.99563934,72.28499247)(140.35563898,72.44499231)(140.78564148,72.565)
\curveto(140.88563845,72.58499217)(140.98563835,72.59999215)(141.08564148,72.61)
\curveto(141.19563814,72.61999213)(141.30563803,72.63499211)(141.41564148,72.655)
\curveto(141.45563788,72.66499208)(141.52063782,72.66499208)(141.61064148,72.655)
\curveto(141.70063764,72.6549921)(141.75563758,72.66499208)(141.77564148,72.685)
\curveto(142.47563686,72.69499206)(143.08563625,72.61499214)(143.60564148,72.445)
\curveto(144.12563521,72.27499247)(144.49063485,71.9499928)(144.70064148,71.47)
\curveto(144.79063455,71.26999348)(144.8406345,71.03499371)(144.85064148,70.765)
\curveto(144.87063447,70.50499424)(144.88063446,70.22999452)(144.88064148,69.94)
\lineto(144.88064148,66.625)
\curveto(144.88063446,66.48499827)(144.88563445,66.3499984)(144.89564148,66.22)
\curveto(144.90563443,66.08999866)(144.9356344,65.98499877)(144.98564148,65.905)
\curveto(145.0356343,65.83499892)(145.10063424,65.78499896)(145.18064148,65.755)
\curveto(145.27063407,65.71499904)(145.35563398,65.68499906)(145.43564148,65.665)
\curveto(145.51563382,65.65499909)(145.57563376,65.60999914)(145.61564148,65.53)
\curveto(145.6356337,65.49999925)(145.64563369,65.46999928)(145.64564148,65.44)
\curveto(145.64563369,65.40999934)(145.65063369,65.36999938)(145.66064148,65.32)
\moveto(143.51564148,66.985)
\curveto(143.57563576,67.12499762)(143.60563573,67.28499747)(143.60564148,67.465)
\curveto(143.61563572,67.65499709)(143.62063572,67.8499969)(143.62064148,68.05)
\curveto(143.62063572,68.15999659)(143.61563572,68.25999649)(143.60564148,68.35)
\curveto(143.59563574,68.43999631)(143.55563578,68.50999624)(143.48564148,68.56)
\curveto(143.45563588,68.57999617)(143.38563595,68.58999616)(143.27564148,68.59)
\curveto(143.25563608,68.56999618)(143.22063612,68.55999619)(143.17064148,68.56)
\curveto(143.12063622,68.55999619)(143.07563626,68.5499962)(143.03564148,68.53)
\curveto(142.95563638,68.50999624)(142.86563647,68.48999626)(142.76564148,68.47)
\lineto(142.46564148,68.41)
\curveto(142.4356369,68.40999634)(142.40063694,68.40499634)(142.36064148,68.395)
\lineto(142.25564148,68.395)
\curveto(142.10563723,68.3549964)(141.9406374,68.32999642)(141.76064148,68.32)
\curveto(141.59063775,68.31999643)(141.43063791,68.29999645)(141.28064148,68.26)
\curveto(141.20063814,68.23999651)(141.12563821,68.21999653)(141.05564148,68.2)
\curveto(140.99563834,68.18999656)(140.92563841,68.17499658)(140.84564148,68.155)
\curveto(140.68563865,68.10499665)(140.5356388,68.03999671)(140.39564148,67.96)
\curveto(140.25563908,67.88999686)(140.1356392,67.79999695)(140.03564148,67.69)
\curveto(139.9356394,67.57999717)(139.86063948,67.44499731)(139.81064148,67.285)
\curveto(139.76063958,67.13499761)(139.7406396,66.9499978)(139.75064148,66.73)
\curveto(139.75063959,66.62999812)(139.76563957,66.53499821)(139.79564148,66.445)
\curveto(139.8356395,66.36499838)(139.88063946,66.28999846)(139.93064148,66.22)
\curveto(140.01063933,66.10999864)(140.11563922,66.01499874)(140.24564148,65.935)
\curveto(140.37563896,65.86499888)(140.51563882,65.80499895)(140.66564148,65.755)
\curveto(140.71563862,65.74499901)(140.76563857,65.73999901)(140.81564148,65.74)
\curveto(140.86563847,65.73999901)(140.91563842,65.73499902)(140.96564148,65.725)
\curveto(141.0356383,65.70499905)(141.12063822,65.68999906)(141.22064148,65.68)
\curveto(141.33063801,65.67999907)(141.42063792,65.68999906)(141.49064148,65.71)
\curveto(141.55063779,65.72999902)(141.61063773,65.73499902)(141.67064148,65.725)
\curveto(141.73063761,65.72499902)(141.79063755,65.73499902)(141.85064148,65.755)
\curveto(141.93063741,65.77499898)(142.00563733,65.78999896)(142.07564148,65.8)
\curveto(142.15563718,65.80999894)(142.23063711,65.82999892)(142.30064148,65.86)
\curveto(142.59063675,65.97999877)(142.8356365,66.12499862)(143.03564148,66.295)
\curveto(143.24563609,66.46499828)(143.40563593,66.69499805)(143.51564148,66.985)
}
}
{
\newrgbcolor{curcolor}{0 0 0}
\pscustom[linestyle=none,fillstyle=solid,fillcolor=curcolor]
{
\newpath
\moveto(153.7922821,65.575)
\lineto(153.7922821,65.185)
\curveto(153.79227423,65.06499968)(153.76727425,64.96499978)(153.7172821,64.885)
\curveto(153.66727435,64.81499994)(153.58227444,64.77499998)(153.4622821,64.765)
\lineto(153.1172821,64.765)
\curveto(153.05727496,64.76499998)(152.99727502,64.75999999)(152.9372821,64.75)
\curveto(152.88727513,64.75)(152.84227518,64.75999999)(152.8022821,64.78)
\curveto(152.71227531,64.79999995)(152.65227537,64.83999991)(152.6222821,64.9)
\curveto(152.58227544,64.9499998)(152.55727546,65.00999974)(152.5472821,65.08)
\curveto(152.54727547,65.1499996)(152.53227549,65.21999953)(152.5022821,65.29)
\curveto(152.49227553,65.30999944)(152.47727554,65.32499942)(152.4572821,65.335)
\curveto(152.44727557,65.35499939)(152.43227559,65.37499938)(152.4122821,65.395)
\curveto(152.31227571,65.40499935)(152.23227579,65.38499936)(152.1722821,65.335)
\curveto(152.1222759,65.28499946)(152.06727595,65.23499952)(152.0072821,65.185)
\curveto(151.80727621,65.03499972)(151.60727641,64.91999983)(151.4072821,64.84)
\curveto(151.22727679,64.75999999)(151.017277,64.70000005)(150.7772821,64.66)
\curveto(150.54727747,64.62000013)(150.30727771,64.60000015)(150.0572821,64.6)
\curveto(149.8172782,64.59000016)(149.57727844,64.60500015)(149.3372821,64.645)
\curveto(149.09727892,64.67500008)(148.88727913,64.73000002)(148.7072821,64.81)
\curveto(148.18727983,65.02999972)(147.76728025,65.32499942)(147.4472821,65.695)
\curveto(147.12728089,66.07499868)(146.87728114,66.54499821)(146.6972821,67.105)
\curveto(146.65728136,67.19499755)(146.62728139,67.28499747)(146.6072821,67.375)
\curveto(146.59728142,67.47499728)(146.57728144,67.57499718)(146.5472821,67.675)
\curveto(146.53728148,67.72499702)(146.53228149,67.77499698)(146.5322821,67.825)
\curveto(146.53228149,67.87499688)(146.52728149,67.92499682)(146.5172821,67.975)
\curveto(146.49728152,68.02499672)(146.48728153,68.07499668)(146.4872821,68.125)
\curveto(146.49728152,68.18499657)(146.49728152,68.23999651)(146.4872821,68.29)
\lineto(146.4872821,68.44)
\curveto(146.46728155,68.48999626)(146.45728156,68.5549962)(146.4572821,68.635)
\curveto(146.45728156,68.71499604)(146.46728155,68.77999597)(146.4872821,68.83)
\lineto(146.4872821,68.995)
\curveto(146.50728151,69.06499568)(146.51228151,69.13499561)(146.5022821,69.205)
\curveto(146.50228152,69.28499547)(146.51228151,69.35999539)(146.5322821,69.43)
\curveto(146.54228148,69.47999527)(146.54728147,69.52499522)(146.5472821,69.565)
\curveto(146.54728147,69.60499514)(146.55228147,69.6499951)(146.5622821,69.7)
\curveto(146.59228143,69.79999495)(146.6172814,69.89499485)(146.6372821,69.985)
\curveto(146.65728136,70.08499467)(146.68228134,70.17999457)(146.7122821,70.27)
\curveto(146.84228118,70.6499941)(147.00728101,70.98999376)(147.2072821,71.29)
\curveto(147.4172806,71.59999315)(147.66728035,71.8549929)(147.9572821,72.055)
\curveto(148.12727989,72.17499257)(148.30227972,72.27499247)(148.4822821,72.355)
\curveto(148.67227935,72.43499231)(148.87727914,72.50499224)(149.0972821,72.565)
\curveto(149.16727885,72.57499217)(149.23227879,72.58499217)(149.2922821,72.595)
\curveto(149.36227866,72.60499214)(149.43227859,72.61999213)(149.5022821,72.64)
\lineto(149.6522821,72.64)
\curveto(149.73227829,72.65999209)(149.84727817,72.66999208)(149.9972821,72.67)
\curveto(150.15727786,72.66999208)(150.27727774,72.65999209)(150.3572821,72.64)
\curveto(150.39727762,72.62999212)(150.45227757,72.62499213)(150.5222821,72.625)
\curveto(150.63227739,72.59499215)(150.74227728,72.56999218)(150.8522821,72.55)
\curveto(150.96227706,72.53999221)(151.06727695,72.50999224)(151.1672821,72.46)
\curveto(151.3172767,72.39999235)(151.45727656,72.33499241)(151.5872821,72.265)
\curveto(151.72727629,72.19499255)(151.85727616,72.11499264)(151.9772821,72.025)
\curveto(152.03727598,71.97499277)(152.09727592,71.91999283)(152.1572821,71.86)
\curveto(152.22727579,71.80999294)(152.3172757,71.79499295)(152.4272821,71.815)
\curveto(152.44727557,71.84499291)(152.46227556,71.86999288)(152.4722821,71.89)
\curveto(152.49227553,71.90999284)(152.50727551,71.93999281)(152.5172821,71.98)
\curveto(152.54727547,72.06999268)(152.55727546,72.18499257)(152.5472821,72.325)
\lineto(152.5472821,72.7)
\lineto(152.5472821,74.425)
\lineto(152.5472821,74.89)
\curveto(152.54727547,75.06998968)(152.57227545,75.19998955)(152.6222821,75.28)
\curveto(152.66227536,75.3499894)(152.7222753,75.39498936)(152.8022821,75.415)
\curveto(152.8222752,75.41498933)(152.84727517,75.41498933)(152.8772821,75.415)
\curveto(152.90727511,75.42498933)(152.93227509,75.42998932)(152.9522821,75.43)
\curveto(153.09227493,75.43998931)(153.23727478,75.43998931)(153.3872821,75.43)
\curveto(153.54727447,75.42998932)(153.65727436,75.38998936)(153.7172821,75.31)
\curveto(153.76727425,75.22998952)(153.79227423,75.12998962)(153.7922821,75.01)
\lineto(153.7922821,74.635)
\lineto(153.7922821,65.575)
\moveto(152.5772821,68.41)
\curveto(152.59727542,68.45999629)(152.60727541,68.52499622)(152.6072821,68.605)
\curveto(152.60727541,68.69499605)(152.59727542,68.76499598)(152.5772821,68.815)
\lineto(152.5772821,69.04)
\curveto(152.55727546,69.12999562)(152.54227548,69.21999553)(152.5322821,69.31)
\curveto(152.5222755,69.40999534)(152.50227552,69.49999525)(152.4722821,69.58)
\curveto(152.45227557,69.65999509)(152.43227559,69.73499501)(152.4122821,69.805)
\curveto(152.40227562,69.87499488)(152.38227564,69.94499481)(152.3522821,70.015)
\curveto(152.23227579,70.31499444)(152.07727594,70.57999417)(151.8872821,70.81)
\curveto(151.69727632,71.03999371)(151.45727656,71.21999353)(151.1672821,71.35)
\curveto(151.06727695,71.39999335)(150.96227706,71.43499331)(150.8522821,71.455)
\curveto(150.75227727,71.48499327)(150.64227738,71.50999324)(150.5222821,71.53)
\curveto(150.44227758,71.5499932)(150.35227767,71.55999319)(150.2522821,71.56)
\lineto(149.9822821,71.56)
\curveto(149.93227809,71.5499932)(149.88727813,71.53999321)(149.8472821,71.53)
\lineto(149.7122821,71.53)
\curveto(149.63227839,71.50999324)(149.54727847,71.48999326)(149.4572821,71.47)
\curveto(149.37727864,71.4499933)(149.29727872,71.42499333)(149.2172821,71.395)
\curveto(148.89727912,71.2549935)(148.63727938,71.0499937)(148.4372821,70.78)
\curveto(148.24727977,70.51999423)(148.09227993,70.21499454)(147.9722821,69.865)
\curveto(147.93228009,69.754995)(147.90228012,69.63999511)(147.8822821,69.52)
\curveto(147.87228015,69.40999534)(147.85728016,69.29999545)(147.8372821,69.19)
\curveto(147.83728018,69.1499956)(147.83228019,69.10999564)(147.8222821,69.07)
\lineto(147.8222821,68.965)
\curveto(147.80228022,68.91499584)(147.79228023,68.85999589)(147.7922821,68.8)
\curveto(147.80228022,68.73999601)(147.80728021,68.68499607)(147.8072821,68.635)
\lineto(147.8072821,68.305)
\curveto(147.80728021,68.20499654)(147.8172802,68.10999664)(147.8372821,68.02)
\curveto(147.84728017,67.98999676)(147.85228017,67.93999681)(147.8522821,67.87)
\curveto(147.87228015,67.79999695)(147.88728013,67.72999702)(147.8972821,67.66)
\lineto(147.9572821,67.45)
\curveto(148.06727995,67.09999765)(148.2172798,66.79999795)(148.4072821,66.55)
\curveto(148.59727942,66.29999845)(148.83727918,66.09499865)(149.1272821,65.935)
\curveto(149.2172788,65.88499887)(149.30727871,65.84499891)(149.3972821,65.815)
\curveto(149.48727853,65.78499896)(149.58727843,65.75499899)(149.6972821,65.725)
\curveto(149.74727827,65.70499905)(149.79727822,65.69999905)(149.8472821,65.71)
\curveto(149.90727811,65.71999903)(149.96227806,65.71499904)(150.0122821,65.695)
\curveto(150.05227797,65.68499906)(150.09227793,65.67999907)(150.1322821,65.68)
\lineto(150.2672821,65.68)
\lineto(150.4022821,65.68)
\curveto(150.43227759,65.68999906)(150.48227754,65.69499905)(150.5522821,65.695)
\curveto(150.63227739,65.71499904)(150.71227731,65.72999902)(150.7922821,65.74)
\curveto(150.87227715,65.75999899)(150.94727707,65.78499896)(151.0172821,65.815)
\curveto(151.34727667,65.95499879)(151.61227641,66.12999862)(151.8122821,66.34)
\curveto(152.022276,66.55999819)(152.19727582,66.83499791)(152.3372821,67.165)
\curveto(152.38727563,67.27499748)(152.4222756,67.38499737)(152.4422821,67.495)
\curveto(152.46227556,67.60499715)(152.48727553,67.71499704)(152.5172821,67.825)
\curveto(152.53727548,67.86499688)(152.54727547,67.89999685)(152.5472821,67.93)
\curveto(152.54727547,67.96999678)(152.55227547,68.00999674)(152.5622821,68.05)
\curveto(152.57227545,68.10999664)(152.57227545,68.16999658)(152.5622821,68.23)
\curveto(152.56227546,68.28999646)(152.56727545,68.3499964)(152.5772821,68.41)
}
}
{
\newrgbcolor{curcolor}{0 0 0}
\pscustom[linestyle=none,fillstyle=solid,fillcolor=curcolor]
{
\newpath
\moveto(162.6235321,65.32)
\curveto(162.65352427,65.15999959)(162.63852429,65.02499972)(162.5785321,64.915)
\curveto(162.51852441,64.81499994)(162.43852449,64.74000001)(162.3385321,64.69)
\curveto(162.28852464,64.67000008)(162.23352469,64.66000009)(162.1735321,64.66)
\curveto(162.1235248,64.66000009)(162.06852486,64.6500001)(162.0085321,64.63)
\curveto(161.78852514,64.58000017)(161.56852536,64.59500015)(161.3485321,64.675)
\curveto(161.13852579,64.74500001)(160.99352593,64.83499992)(160.9135321,64.945)
\curveto(160.86352606,65.01499974)(160.81852611,65.09499965)(160.7785321,65.185)
\curveto(160.73852619,65.28499946)(160.68852624,65.36499938)(160.6285321,65.425)
\curveto(160.60852632,65.44499931)(160.58352634,65.46499928)(160.5535321,65.485)
\curveto(160.53352639,65.50499925)(160.50352642,65.50999924)(160.4635321,65.5)
\curveto(160.35352657,65.46999928)(160.24852668,65.41499934)(160.1485321,65.335)
\curveto(160.05852687,65.25499949)(159.96852696,65.18499956)(159.8785321,65.125)
\curveto(159.74852718,65.04499971)(159.60852732,64.96999978)(159.4585321,64.9)
\curveto(159.30852762,64.83999991)(159.14852778,64.78499996)(158.9785321,64.735)
\curveto(158.87852805,64.70500005)(158.76852816,64.68500006)(158.6485321,64.675)
\curveto(158.53852839,64.66500008)(158.4285285,64.6500001)(158.3185321,64.63)
\curveto(158.26852866,64.62000013)(158.2235287,64.61500014)(158.1835321,64.615)
\lineto(158.0785321,64.615)
\curveto(157.96852896,64.59500015)(157.86352906,64.59500015)(157.7635321,64.615)
\lineto(157.6285321,64.615)
\curveto(157.57852935,64.62500012)(157.5285294,64.63000012)(157.4785321,64.63)
\curveto(157.4285295,64.63000012)(157.38352954,64.64000011)(157.3435321,64.66)
\curveto(157.30352962,64.67000008)(157.26852966,64.67500008)(157.2385321,64.675)
\curveto(157.21852971,64.66500008)(157.19352973,64.66500008)(157.1635321,64.675)
\lineto(156.9235321,64.735)
\curveto(156.84353008,64.74500001)(156.76853016,64.76499998)(156.6985321,64.795)
\curveto(156.39853053,64.92499982)(156.15353077,65.06999968)(155.9635321,65.23)
\curveto(155.78353114,65.39999935)(155.63353129,65.63499912)(155.5135321,65.935)
\curveto(155.4235315,66.15499859)(155.37853155,66.41999833)(155.3785321,66.73)
\lineto(155.3785321,67.045)
\curveto(155.38853154,67.09499765)(155.39353153,67.14499761)(155.3935321,67.195)
\lineto(155.4235321,67.375)
\lineto(155.5435321,67.705)
\curveto(155.58353134,67.81499694)(155.63353129,67.91499684)(155.6935321,68.005)
\curveto(155.87353105,68.29499645)(156.11853081,68.50999624)(156.4285321,68.65)
\curveto(156.73853019,68.78999596)(157.07852985,68.91499584)(157.4485321,69.025)
\curveto(157.58852934,69.06499568)(157.73352919,69.09499565)(157.8835321,69.115)
\curveto(158.03352889,69.13499561)(158.18352874,69.15999559)(158.3335321,69.19)
\curveto(158.40352852,69.20999554)(158.46852846,69.21999553)(158.5285321,69.22)
\curveto(158.59852833,69.21999553)(158.67352825,69.22999552)(158.7535321,69.25)
\curveto(158.8235281,69.26999548)(158.89352803,69.27999547)(158.9635321,69.28)
\curveto(159.03352789,69.28999546)(159.10852782,69.30499544)(159.1885321,69.325)
\curveto(159.43852749,69.38499537)(159.67352725,69.43499531)(159.8935321,69.475)
\curveto(160.11352681,69.52499522)(160.28852664,69.63999511)(160.4185321,69.82)
\curveto(160.47852645,69.89999485)(160.5285264,69.99999475)(160.5685321,70.12)
\curveto(160.60852632,70.2499945)(160.60852632,70.38999436)(160.5685321,70.54)
\curveto(160.50852642,70.77999397)(160.41852651,70.96999378)(160.2985321,71.11)
\curveto(160.18852674,71.2499935)(160.0285269,71.35999339)(159.8185321,71.44)
\curveto(159.69852723,71.48999326)(159.55352737,71.52499323)(159.3835321,71.545)
\curveto(159.2235277,71.56499318)(159.05352787,71.57499317)(158.8735321,71.575)
\curveto(158.69352823,71.57499317)(158.51852841,71.56499318)(158.3485321,71.545)
\curveto(158.17852875,71.52499323)(158.03352889,71.49499325)(157.9135321,71.455)
\curveto(157.74352918,71.39499335)(157.57852935,71.30999344)(157.4185321,71.2)
\curveto(157.33852959,71.13999361)(157.26352966,71.05999369)(157.1935321,70.96)
\curveto(157.13352979,70.86999388)(157.07852985,70.76999398)(157.0285321,70.66)
\curveto(156.99852993,70.57999417)(156.96852996,70.49499425)(156.9385321,70.405)
\curveto(156.91853001,70.31499444)(156.87353005,70.24499451)(156.8035321,70.195)
\curveto(156.76353016,70.16499458)(156.69353023,70.13999461)(156.5935321,70.12)
\curveto(156.50353042,70.10999464)(156.40853052,70.10499464)(156.3085321,70.105)
\curveto(156.20853072,70.10499464)(156.10853082,70.10999464)(156.0085321,70.12)
\curveto(155.91853101,70.13999461)(155.85353107,70.16499458)(155.8135321,70.195)
\curveto(155.77353115,70.22499452)(155.74353118,70.27499447)(155.7235321,70.345)
\curveto(155.70353122,70.41499434)(155.70353122,70.48999426)(155.7235321,70.57)
\curveto(155.75353117,70.69999405)(155.78353114,70.81999393)(155.8135321,70.93)
\curveto(155.85353107,71.0499937)(155.89853103,71.16499358)(155.9485321,71.275)
\curveto(156.13853079,71.62499313)(156.37853055,71.89499285)(156.6685321,72.085)
\curveto(156.95852997,72.28499247)(157.31852961,72.44499231)(157.7485321,72.565)
\curveto(157.84852908,72.58499217)(157.94852898,72.59999215)(158.0485321,72.61)
\curveto(158.15852877,72.61999213)(158.26852866,72.63499211)(158.3785321,72.655)
\curveto(158.41852851,72.66499208)(158.48352844,72.66499208)(158.5735321,72.655)
\curveto(158.66352826,72.6549921)(158.71852821,72.66499208)(158.7385321,72.685)
\curveto(159.43852749,72.69499206)(160.04852688,72.61499214)(160.5685321,72.445)
\curveto(161.08852584,72.27499247)(161.45352547,71.9499928)(161.6635321,71.47)
\curveto(161.75352517,71.26999348)(161.80352512,71.03499371)(161.8135321,70.765)
\curveto(161.83352509,70.50499424)(161.84352508,70.22999452)(161.8435321,69.94)
\lineto(161.8435321,66.625)
\curveto(161.84352508,66.48499827)(161.84852508,66.3499984)(161.8585321,66.22)
\curveto(161.86852506,66.08999866)(161.89852503,65.98499877)(161.9485321,65.905)
\curveto(161.99852493,65.83499892)(162.06352486,65.78499896)(162.1435321,65.755)
\curveto(162.23352469,65.71499904)(162.31852461,65.68499906)(162.3985321,65.665)
\curveto(162.47852445,65.65499909)(162.53852439,65.60999914)(162.5785321,65.53)
\curveto(162.59852433,65.49999925)(162.60852432,65.46999928)(162.6085321,65.44)
\curveto(162.60852432,65.40999934)(162.61352431,65.36999938)(162.6235321,65.32)
\moveto(160.4785321,66.985)
\curveto(160.53852639,67.12499762)(160.56852636,67.28499747)(160.5685321,67.465)
\curveto(160.57852635,67.65499709)(160.58352634,67.8499969)(160.5835321,68.05)
\curveto(160.58352634,68.15999659)(160.57852635,68.25999649)(160.5685321,68.35)
\curveto(160.55852637,68.43999631)(160.51852641,68.50999624)(160.4485321,68.56)
\curveto(160.41852651,68.57999617)(160.34852658,68.58999616)(160.2385321,68.59)
\curveto(160.21852671,68.56999618)(160.18352674,68.55999619)(160.1335321,68.56)
\curveto(160.08352684,68.55999619)(160.03852689,68.5499962)(159.9985321,68.53)
\curveto(159.91852701,68.50999624)(159.8285271,68.48999626)(159.7285321,68.47)
\lineto(159.4285321,68.41)
\curveto(159.39852753,68.40999634)(159.36352756,68.40499634)(159.3235321,68.395)
\lineto(159.2185321,68.395)
\curveto(159.06852786,68.3549964)(158.90352802,68.32999642)(158.7235321,68.32)
\curveto(158.55352837,68.31999643)(158.39352853,68.29999645)(158.2435321,68.26)
\curveto(158.16352876,68.23999651)(158.08852884,68.21999653)(158.0185321,68.2)
\curveto(157.95852897,68.18999656)(157.88852904,68.17499658)(157.8085321,68.155)
\curveto(157.64852928,68.10499665)(157.49852943,68.03999671)(157.3585321,67.96)
\curveto(157.21852971,67.88999686)(157.09852983,67.79999695)(156.9985321,67.69)
\curveto(156.89853003,67.57999717)(156.8235301,67.44499731)(156.7735321,67.285)
\curveto(156.7235302,67.13499761)(156.70353022,66.9499978)(156.7135321,66.73)
\curveto(156.71353021,66.62999812)(156.7285302,66.53499821)(156.7585321,66.445)
\curveto(156.79853013,66.36499838)(156.84353008,66.28999846)(156.8935321,66.22)
\curveto(156.97352995,66.10999864)(157.07852985,66.01499874)(157.2085321,65.935)
\curveto(157.33852959,65.86499888)(157.47852945,65.80499895)(157.6285321,65.755)
\curveto(157.67852925,65.74499901)(157.7285292,65.73999901)(157.7785321,65.74)
\curveto(157.8285291,65.73999901)(157.87852905,65.73499902)(157.9285321,65.725)
\curveto(157.99852893,65.70499905)(158.08352884,65.68999906)(158.1835321,65.68)
\curveto(158.29352863,65.67999907)(158.38352854,65.68999906)(158.4535321,65.71)
\curveto(158.51352841,65.72999902)(158.57352835,65.73499902)(158.6335321,65.725)
\curveto(158.69352823,65.72499902)(158.75352817,65.73499902)(158.8135321,65.755)
\curveto(158.89352803,65.77499898)(158.96852796,65.78999896)(159.0385321,65.8)
\curveto(159.11852781,65.80999894)(159.19352773,65.82999892)(159.2635321,65.86)
\curveto(159.55352737,65.97999877)(159.79852713,66.12499862)(159.9985321,66.295)
\curveto(160.20852672,66.46499828)(160.36852656,66.69499805)(160.4785321,66.985)
}
}
{
\newrgbcolor{curcolor}{0 0 0}
\pscustom[linestyle=none,fillstyle=solid,fillcolor=curcolor]
{
\newpath
\moveto(308.24115601,65.52900879)
\curveto(308.26114646,65.47900804)(308.28614644,65.4190081)(308.31615601,65.34900879)
\curveto(308.34614638,65.27900824)(308.36614636,65.20400832)(308.37615601,65.12400879)
\curveto(308.39614633,65.05400847)(308.39614633,64.98400854)(308.37615601,64.91400879)
\curveto(308.36614636,64.85400867)(308.3261464,64.80900871)(308.25615601,64.77900879)
\curveto(308.20614652,64.75900876)(308.14614658,64.74900877)(308.07615601,64.74900879)
\lineto(307.86615601,64.74900879)
\lineto(307.41615601,64.74900879)
\curveto(307.26614746,64.74900877)(307.14614758,64.77400875)(307.05615601,64.82400879)
\curveto(306.95614777,64.88400864)(306.88114784,64.98900853)(306.83115601,65.13900879)
\curveto(306.79114793,65.28900823)(306.74614798,65.4240081)(306.69615601,65.54400879)
\curveto(306.58614814,65.80400772)(306.48614824,66.07400745)(306.39615601,66.35400879)
\curveto(306.30614842,66.63400689)(306.20614852,66.90900661)(306.09615601,67.17900879)
\curveto(306.06614866,67.26900625)(306.03614869,67.35400617)(306.00615601,67.43400879)
\curveto(305.98614874,67.51400601)(305.95614877,67.58900593)(305.91615601,67.65900879)
\curveto(305.88614884,67.72900579)(305.84114888,67.78900573)(305.78115601,67.83900879)
\curveto(305.721149,67.88900563)(305.64114908,67.92900559)(305.54115601,67.95900879)
\curveto(305.49114923,67.97900554)(305.43114929,67.98400554)(305.36115601,67.97400879)
\lineto(305.16615601,67.97400879)
\lineto(302.33115601,67.97400879)
\lineto(302.03115601,67.97400879)
\curveto(301.9211528,67.98400554)(301.81615291,67.98400554)(301.71615601,67.97400879)
\curveto(301.61615311,67.96400556)(301.5211532,67.94900557)(301.43115601,67.92900879)
\curveto(301.35115337,67.90900561)(301.29115343,67.86900565)(301.25115601,67.80900879)
\curveto(301.17115355,67.70900581)(301.11115361,67.59400593)(301.07115601,67.46400879)
\curveto(301.04115368,67.34400618)(301.00115372,67.2190063)(300.95115601,67.08900879)
\curveto(300.85115387,66.85900666)(300.75615397,66.6190069)(300.66615601,66.36900879)
\curveto(300.58615414,66.1190074)(300.49615423,65.87900764)(300.39615601,65.64900879)
\curveto(300.37615435,65.58900793)(300.35115437,65.519008)(300.32115601,65.43900879)
\curveto(300.30115442,65.36900815)(300.27615445,65.29400823)(300.24615601,65.21400879)
\curveto(300.21615451,65.13400839)(300.18115454,65.05900846)(300.14115601,64.98900879)
\curveto(300.11115461,64.92900859)(300.07615465,64.88400864)(300.03615601,64.85400879)
\curveto(299.95615477,64.79400873)(299.84615488,64.75900876)(299.70615601,64.74900879)
\lineto(299.28615601,64.74900879)
\lineto(299.04615601,64.74900879)
\curveto(298.97615575,64.75900876)(298.91615581,64.78400874)(298.86615601,64.82400879)
\curveto(298.81615591,64.85400867)(298.78615594,64.89900862)(298.77615601,64.95900879)
\curveto(298.77615595,65.0190085)(298.78115594,65.07900844)(298.79115601,65.13900879)
\curveto(298.81115591,65.20900831)(298.83115589,65.27400825)(298.85115601,65.33400879)
\curveto(298.88115584,65.40400812)(298.90615582,65.45400807)(298.92615601,65.48400879)
\curveto(299.06615566,65.80400772)(299.19115553,66.1190074)(299.30115601,66.42900879)
\curveto(299.41115531,66.74900677)(299.53115519,67.06900645)(299.66115601,67.38900879)
\curveto(299.75115497,67.60900591)(299.83615489,67.8240057)(299.91615601,68.03400879)
\curveto(299.99615473,68.25400527)(300.08115464,68.47400505)(300.17115601,68.69400879)
\curveto(300.47115425,69.41400411)(300.75615397,70.13900338)(301.02615601,70.86900879)
\curveto(301.29615343,71.60900191)(301.58115314,72.34400118)(301.88115601,73.07400879)
\curveto(301.99115273,73.33400019)(302.09115263,73.59899992)(302.18115601,73.86900879)
\curveto(302.28115244,74.13899938)(302.38615234,74.40399912)(302.49615601,74.66400879)
\curveto(302.54615218,74.77399875)(302.59115213,74.89399863)(302.63115601,75.02400879)
\curveto(302.68115204,75.16399836)(302.75115197,75.26399826)(302.84115601,75.32400879)
\curveto(302.88115184,75.36399816)(302.94615178,75.39399813)(303.03615601,75.41400879)
\curveto(303.05615167,75.4239981)(303.07615165,75.4239981)(303.09615601,75.41400879)
\curveto(303.1261516,75.41399811)(303.15115157,75.4189981)(303.17115601,75.42900879)
\curveto(303.35115137,75.42899809)(303.56115116,75.42899809)(303.80115601,75.42900879)
\curveto(304.04115068,75.43899808)(304.21615051,75.40399812)(304.32615601,75.32400879)
\curveto(304.40615032,75.26399826)(304.46615026,75.16399836)(304.50615601,75.02400879)
\curveto(304.55615017,74.89399863)(304.60615012,74.77399875)(304.65615601,74.66400879)
\curveto(304.75614997,74.43399909)(304.84614988,74.20399932)(304.92615601,73.97400879)
\curveto(305.00614972,73.74399978)(305.09614963,73.51400001)(305.19615601,73.28400879)
\curveto(305.27614945,73.08400044)(305.35114937,72.87900064)(305.42115601,72.66900879)
\curveto(305.50114922,72.45900106)(305.58614914,72.25400127)(305.67615601,72.05400879)
\curveto(305.97614875,71.3240022)(306.26114846,70.58400294)(306.53115601,69.83400879)
\curveto(306.81114791,69.09400443)(307.10614762,68.35900516)(307.41615601,67.62900879)
\curveto(307.45614727,67.53900598)(307.48614724,67.45400607)(307.50615601,67.37400879)
\curveto(307.53614719,67.29400623)(307.56614716,67.20900631)(307.59615601,67.11900879)
\curveto(307.70614702,66.85900666)(307.81114691,66.59400693)(307.91115601,66.32400879)
\curveto(308.0211467,66.05400747)(308.13114659,65.78900773)(308.24115601,65.52900879)
\moveto(305.03115601,69.17400879)
\curveto(305.1211496,69.20400432)(305.17614955,69.25400427)(305.19615601,69.32400879)
\curveto(305.2261495,69.39400413)(305.23114949,69.46900405)(305.21115601,69.54900879)
\curveto(305.20114952,69.63900388)(305.17614955,69.7240038)(305.13615601,69.80400879)
\curveto(305.10614962,69.89400363)(305.07614965,69.96900355)(305.04615601,70.02900879)
\curveto(305.0261497,70.06900345)(305.01614971,70.10400342)(305.01615601,70.13400879)
\curveto(305.01614971,70.16400336)(305.00614972,70.19900332)(304.98615601,70.23900879)
\lineto(304.89615601,70.47900879)
\curveto(304.87614985,70.56900295)(304.84614988,70.65900286)(304.80615601,70.74900879)
\curveto(304.65615007,71.10900241)(304.5211502,71.47400205)(304.40115601,71.84400879)
\curveto(304.29115043,72.2240013)(304.16115056,72.59400093)(304.01115601,72.95400879)
\curveto(303.96115076,73.06400046)(303.91615081,73.17400035)(303.87615601,73.28400879)
\curveto(303.84615088,73.39400013)(303.80615092,73.49900002)(303.75615601,73.59900879)
\curveto(303.73615099,73.64899987)(303.71115101,73.69399983)(303.68115601,73.73400879)
\curveto(303.66115106,73.78399974)(303.61115111,73.80899971)(303.53115601,73.80900879)
\curveto(303.51115121,73.78899973)(303.49115123,73.77399975)(303.47115601,73.76400879)
\curveto(303.45115127,73.75399977)(303.43115129,73.73899978)(303.41115601,73.71900879)
\curveto(303.37115135,73.66899985)(303.34115138,73.61399991)(303.32115601,73.55400879)
\curveto(303.30115142,73.50400002)(303.28115144,73.44900007)(303.26115601,73.38900879)
\curveto(303.21115151,73.27900024)(303.17115155,73.16900035)(303.14115601,73.05900879)
\curveto(303.11115161,72.94900057)(303.07115165,72.83900068)(303.02115601,72.72900879)
\curveto(302.85115187,72.33900118)(302.70115202,71.94400158)(302.57115601,71.54400879)
\curveto(302.45115227,71.14400238)(302.31115241,70.75400277)(302.15115601,70.37400879)
\lineto(302.09115601,70.22400879)
\curveto(302.08115264,70.17400335)(302.06615266,70.1240034)(302.04615601,70.07400879)
\lineto(301.95615601,69.83400879)
\curveto(301.9261528,69.75400377)(301.90115282,69.67400385)(301.88115601,69.59400879)
\curveto(301.86115286,69.54400398)(301.85115287,69.48900403)(301.85115601,69.42900879)
\curveto(301.86115286,69.36900415)(301.87615285,69.3190042)(301.89615601,69.27900879)
\curveto(301.94615278,69.19900432)(302.05115267,69.15400437)(302.21115601,69.14400879)
\lineto(302.66115601,69.14400879)
\lineto(304.26615601,69.14400879)
\curveto(304.37615035,69.14400438)(304.51115021,69.13900438)(304.67115601,69.12900879)
\curveto(304.83114989,69.12900439)(304.95114977,69.14400438)(305.03115601,69.17400879)
}
}
{
\newrgbcolor{curcolor}{0 0 0}
\pscustom[linestyle=none,fillstyle=solid,fillcolor=curcolor]
{
\newpath
\moveto(313.00271851,72.65400879)
\curveto(313.23271372,72.65400087)(313.36271359,72.59400093)(313.39271851,72.47400879)
\curveto(313.42271353,72.36400116)(313.43771351,72.19900132)(313.43771851,71.97900879)
\lineto(313.43771851,71.69400879)
\curveto(313.43771351,71.60400192)(313.41271354,71.52900199)(313.36271851,71.46900879)
\curveto(313.30271365,71.38900213)(313.21771373,71.34400218)(313.10771851,71.33400879)
\curveto(312.99771395,71.33400219)(312.88771406,71.3190022)(312.77771851,71.28900879)
\curveto(312.63771431,71.25900226)(312.50271445,71.22900229)(312.37271851,71.19900879)
\curveto(312.2527147,71.16900235)(312.13771481,71.12900239)(312.02771851,71.07900879)
\curveto(311.73771521,70.94900257)(311.50271545,70.76900275)(311.32271851,70.53900879)
\curveto(311.14271581,70.3190032)(310.98771596,70.06400346)(310.85771851,69.77400879)
\curveto(310.81771613,69.66400386)(310.78771616,69.54900397)(310.76771851,69.42900879)
\curveto(310.7477162,69.3190042)(310.72271623,69.20400432)(310.69271851,69.08400879)
\curveto(310.68271627,69.03400449)(310.67771627,68.98400454)(310.67771851,68.93400879)
\curveto(310.68771626,68.88400464)(310.68771626,68.83400469)(310.67771851,68.78400879)
\curveto(310.6477163,68.66400486)(310.63271632,68.524005)(310.63271851,68.36400879)
\curveto(310.64271631,68.21400531)(310.6477163,68.06900545)(310.64771851,67.92900879)
\lineto(310.64771851,66.08400879)
\lineto(310.64771851,65.73900879)
\curveto(310.6477163,65.6190079)(310.64271631,65.50400802)(310.63271851,65.39400879)
\curveto(310.62271633,65.28400824)(310.61771633,65.18900833)(310.61771851,65.10900879)
\curveto(310.62771632,65.02900849)(310.60771634,64.95900856)(310.55771851,64.89900879)
\curveto(310.50771644,64.82900869)(310.42771652,64.78900873)(310.31771851,64.77900879)
\curveto(310.21771673,64.76900875)(310.10771684,64.76400876)(309.98771851,64.76400879)
\lineto(309.71771851,64.76400879)
\curveto(309.66771728,64.78400874)(309.61771733,64.79900872)(309.56771851,64.80900879)
\curveto(309.52771742,64.82900869)(309.49771745,64.85400867)(309.47771851,64.88400879)
\curveto(309.42771752,64.95400857)(309.39771755,65.03900848)(309.38771851,65.13900879)
\lineto(309.38771851,65.46900879)
\lineto(309.38771851,66.62400879)
\lineto(309.38771851,70.77900879)
\lineto(309.38771851,71.81400879)
\lineto(309.38771851,72.11400879)
\curveto(309.39771755,72.21400131)(309.42771752,72.29900122)(309.47771851,72.36900879)
\curveto(309.50771744,72.40900111)(309.55771739,72.43900108)(309.62771851,72.45900879)
\curveto(309.70771724,72.47900104)(309.79271716,72.48900103)(309.88271851,72.48900879)
\curveto(309.97271698,72.49900102)(310.06271689,72.49900102)(310.15271851,72.48900879)
\curveto(310.24271671,72.47900104)(310.31271664,72.46400106)(310.36271851,72.44400879)
\curveto(310.44271651,72.41400111)(310.49271646,72.35400117)(310.51271851,72.26400879)
\curveto(310.54271641,72.18400134)(310.55771639,72.09400143)(310.55771851,71.99400879)
\lineto(310.55771851,71.69400879)
\curveto(310.55771639,71.59400193)(310.57771637,71.50400202)(310.61771851,71.42400879)
\curveto(310.62771632,71.40400212)(310.63771631,71.38900213)(310.64771851,71.37900879)
\lineto(310.69271851,71.33400879)
\curveto(310.80271615,71.33400219)(310.89271606,71.37900214)(310.96271851,71.46900879)
\curveto(311.03271592,71.56900195)(311.09271586,71.64900187)(311.14271851,71.70900879)
\lineto(311.23271851,71.79900879)
\curveto(311.32271563,71.90900161)(311.4477155,72.0240015)(311.60771851,72.14400879)
\curveto(311.76771518,72.26400126)(311.91771503,72.35400117)(312.05771851,72.41400879)
\curveto(312.1477148,72.46400106)(312.24271471,72.49900102)(312.34271851,72.51900879)
\curveto(312.44271451,72.54900097)(312.5477144,72.57900094)(312.65771851,72.60900879)
\curveto(312.71771423,72.6190009)(312.77771417,72.6240009)(312.83771851,72.62400879)
\curveto(312.89771405,72.63400089)(312.952714,72.64400088)(313.00271851,72.65400879)
}
}
{
\newrgbcolor{curcolor}{0 0 0}
\pscustom[linestyle=none,fillstyle=solid,fillcolor=curcolor]
{
\newpath
\moveto(321.11748413,68.91900879)
\curveto(321.13747645,68.8190047)(321.13747645,68.70400482)(321.11748413,68.57400879)
\curveto(321.10747648,68.45400507)(321.07747651,68.36900515)(321.02748413,68.31900879)
\curveto(320.97747661,68.27900524)(320.90247668,68.24900527)(320.80248413,68.22900879)
\curveto(320.71247687,68.2190053)(320.60747698,68.21400531)(320.48748413,68.21400879)
\lineto(320.12748413,68.21400879)
\curveto(320.00747758,68.2240053)(319.90247768,68.22900529)(319.81248413,68.22900879)
\lineto(315.97248413,68.22900879)
\curveto(315.89248169,68.22900529)(315.81248177,68.2240053)(315.73248413,68.21400879)
\curveto(315.65248193,68.21400531)(315.587482,68.19900532)(315.53748413,68.16900879)
\curveto(315.49748209,68.14900537)(315.45748213,68.10900541)(315.41748413,68.04900879)
\curveto(315.39748219,68.0190055)(315.37748221,67.97400555)(315.35748413,67.91400879)
\curveto(315.33748225,67.86400566)(315.33748225,67.81400571)(315.35748413,67.76400879)
\curveto(315.36748222,67.71400581)(315.37248221,67.66900585)(315.37248413,67.62900879)
\curveto(315.37248221,67.58900593)(315.37748221,67.54900597)(315.38748413,67.50900879)
\curveto(315.40748218,67.42900609)(315.42748216,67.34400618)(315.44748413,67.25400879)
\curveto(315.46748212,67.17400635)(315.49748209,67.09400643)(315.53748413,67.01400879)
\curveto(315.76748182,66.47400705)(316.14748144,66.08900743)(316.67748413,65.85900879)
\curveto(316.73748085,65.82900769)(316.80248078,65.80400772)(316.87248413,65.78400879)
\lineto(317.08248413,65.72400879)
\curveto(317.11248047,65.71400781)(317.16248042,65.70900781)(317.23248413,65.70900879)
\curveto(317.37248021,65.66900785)(317.55748003,65.64900787)(317.78748413,65.64900879)
\curveto(318.01747957,65.64900787)(318.20247938,65.66900785)(318.34248413,65.70900879)
\curveto(318.4824791,65.74900777)(318.60747898,65.78900773)(318.71748413,65.82900879)
\curveto(318.83747875,65.87900764)(318.94747864,65.93900758)(319.04748413,66.00900879)
\curveto(319.15747843,66.07900744)(319.25247833,66.15900736)(319.33248413,66.24900879)
\curveto(319.41247817,66.34900717)(319.4824781,66.45400707)(319.54248413,66.56400879)
\curveto(319.60247798,66.66400686)(319.65247793,66.76900675)(319.69248413,66.87900879)
\curveto(319.74247784,66.98900653)(319.82247776,67.06900645)(319.93248413,67.11900879)
\curveto(319.97247761,67.13900638)(320.03747755,67.15400637)(320.12748413,67.16400879)
\curveto(320.21747737,67.17400635)(320.30747728,67.17400635)(320.39748413,67.16400879)
\curveto(320.4874771,67.16400636)(320.57247701,67.15900636)(320.65248413,67.14900879)
\curveto(320.73247685,67.13900638)(320.7874768,67.1190064)(320.81748413,67.08900879)
\curveto(320.91747667,67.0190065)(320.94247664,66.90400662)(320.89248413,66.74400879)
\curveto(320.81247677,66.47400705)(320.70747688,66.23400729)(320.57748413,66.02400879)
\curveto(320.37747721,65.70400782)(320.14747744,65.43900808)(319.88748413,65.22900879)
\curveto(319.63747795,65.02900849)(319.31747827,64.86400866)(318.92748413,64.73400879)
\curveto(318.82747876,64.69400883)(318.72747886,64.66900885)(318.62748413,64.65900879)
\curveto(318.52747906,64.63900888)(318.42247916,64.6190089)(318.31248413,64.59900879)
\curveto(318.26247932,64.58900893)(318.21247937,64.58400894)(318.16248413,64.58400879)
\curveto(318.12247946,64.58400894)(318.07747951,64.57900894)(318.02748413,64.56900879)
\lineto(317.87748413,64.56900879)
\curveto(317.82747976,64.55900896)(317.76747982,64.55400897)(317.69748413,64.55400879)
\curveto(317.63747995,64.55400897)(317.58748,64.55900896)(317.54748413,64.56900879)
\lineto(317.41248413,64.56900879)
\curveto(317.36248022,64.57900894)(317.31748027,64.58400894)(317.27748413,64.58400879)
\curveto(317.23748035,64.58400894)(317.19748039,64.58900893)(317.15748413,64.59900879)
\curveto(317.10748048,64.60900891)(317.05248053,64.6190089)(316.99248413,64.62900879)
\curveto(316.93248065,64.62900889)(316.87748071,64.63400889)(316.82748413,64.64400879)
\curveto(316.73748085,64.66400886)(316.64748094,64.68900883)(316.55748413,64.71900879)
\curveto(316.46748112,64.73900878)(316.3824812,64.76400876)(316.30248413,64.79400879)
\curveto(316.26248132,64.81400871)(316.22748136,64.8240087)(316.19748413,64.82400879)
\curveto(316.16748142,64.83400869)(316.13248145,64.84900867)(316.09248413,64.86900879)
\curveto(315.94248164,64.93900858)(315.7824818,65.0240085)(315.61248413,65.12400879)
\curveto(315.32248226,65.31400821)(315.07248251,65.54400798)(314.86248413,65.81400879)
\curveto(314.66248292,66.09400743)(314.49248309,66.40400712)(314.35248413,66.74400879)
\curveto(314.30248328,66.85400667)(314.26248332,66.96900655)(314.23248413,67.08900879)
\curveto(314.21248337,67.20900631)(314.1824834,67.32900619)(314.14248413,67.44900879)
\curveto(314.13248345,67.48900603)(314.12748346,67.524006)(314.12748413,67.55400879)
\curveto(314.12748346,67.58400594)(314.12248346,67.6240059)(314.11248413,67.67400879)
\curveto(314.09248349,67.75400577)(314.07748351,67.83900568)(314.06748413,67.92900879)
\curveto(314.05748353,68.0190055)(314.04248354,68.10900541)(314.02248413,68.19900879)
\lineto(314.02248413,68.40900879)
\curveto(314.01248357,68.44900507)(314.00248358,68.50400502)(313.99248413,68.57400879)
\curveto(313.99248359,68.65400487)(313.99748359,68.7190048)(314.00748413,68.76900879)
\lineto(314.00748413,68.93400879)
\curveto(314.02748356,68.98400454)(314.03248355,69.03400449)(314.02248413,69.08400879)
\curveto(314.02248356,69.14400438)(314.02748356,69.19900432)(314.03748413,69.24900879)
\curveto(314.07748351,69.40900411)(314.10748348,69.56900395)(314.12748413,69.72900879)
\curveto(314.15748343,69.88900363)(314.20248338,70.03900348)(314.26248413,70.17900879)
\curveto(314.31248327,70.28900323)(314.35748323,70.39900312)(314.39748413,70.50900879)
\curveto(314.44748314,70.62900289)(314.50248308,70.74400278)(314.56248413,70.85400879)
\curveto(314.7824828,71.20400232)(315.03248255,71.50400202)(315.31248413,71.75400879)
\curveto(315.59248199,72.01400151)(315.93748165,72.22900129)(316.34748413,72.39900879)
\curveto(316.46748112,72.44900107)(316.587481,72.48400104)(316.70748413,72.50400879)
\curveto(316.83748075,72.53400099)(316.97248061,72.56400096)(317.11248413,72.59400879)
\curveto(317.16248042,72.60400092)(317.20748038,72.60900091)(317.24748413,72.60900879)
\curveto(317.2874803,72.6190009)(317.33248025,72.6240009)(317.38248413,72.62400879)
\curveto(317.40248018,72.63400089)(317.42748016,72.63400089)(317.45748413,72.62400879)
\curveto(317.4874801,72.61400091)(317.51248007,72.6190009)(317.53248413,72.63900879)
\curveto(317.95247963,72.64900087)(318.31747927,72.60400092)(318.62748413,72.50400879)
\curveto(318.93747865,72.41400111)(319.21747837,72.28900123)(319.46748413,72.12900879)
\curveto(319.51747807,72.10900141)(319.55747803,72.07900144)(319.58748413,72.03900879)
\curveto(319.61747797,72.00900151)(319.65247793,71.98400154)(319.69248413,71.96400879)
\curveto(319.77247781,71.90400162)(319.85247773,71.83400169)(319.93248413,71.75400879)
\curveto(320.02247756,71.67400185)(320.09747749,71.59400193)(320.15748413,71.51400879)
\curveto(320.31747727,71.30400222)(320.45247713,71.10400242)(320.56248413,70.91400879)
\curveto(320.63247695,70.80400272)(320.6874769,70.68400284)(320.72748413,70.55400879)
\curveto(320.76747682,70.4240031)(320.81247677,70.29400323)(320.86248413,70.16400879)
\curveto(320.91247667,70.03400349)(320.94747664,69.89900362)(320.96748413,69.75900879)
\curveto(320.99747659,69.6190039)(321.03247655,69.47900404)(321.07248413,69.33900879)
\curveto(321.0824765,69.26900425)(321.0874765,69.19900432)(321.08748413,69.12900879)
\lineto(321.11748413,68.91900879)
\moveto(319.66248413,69.42900879)
\curveto(319.69247789,69.46900405)(319.71747787,69.519004)(319.73748413,69.57900879)
\curveto(319.75747783,69.64900387)(319.75747783,69.7190038)(319.73748413,69.78900879)
\curveto(319.67747791,70.00900351)(319.59247799,70.21400331)(319.48248413,70.40400879)
\curveto(319.34247824,70.63400289)(319.1874784,70.82900269)(319.01748413,70.98900879)
\curveto(318.84747874,71.14900237)(318.62747896,71.28400224)(318.35748413,71.39400879)
\curveto(318.2874793,71.41400211)(318.21747937,71.42900209)(318.14748413,71.43900879)
\curveto(318.07747951,71.45900206)(318.00247958,71.47900204)(317.92248413,71.49900879)
\curveto(317.84247974,71.519002)(317.75747983,71.52900199)(317.66748413,71.52900879)
\lineto(317.41248413,71.52900879)
\curveto(317.3824802,71.50900201)(317.34748024,71.49900202)(317.30748413,71.49900879)
\curveto(317.26748032,71.50900201)(317.23248035,71.50900201)(317.20248413,71.49900879)
\lineto(316.96248413,71.43900879)
\curveto(316.89248069,71.42900209)(316.82248076,71.41400211)(316.75248413,71.39400879)
\curveto(316.46248112,71.27400225)(316.22748136,71.1240024)(316.04748413,70.94400879)
\curveto(315.87748171,70.76400276)(315.72248186,70.53900298)(315.58248413,70.26900879)
\curveto(315.55248203,70.2190033)(315.52248206,70.15400337)(315.49248413,70.07400879)
\curveto(315.46248212,70.00400352)(315.43748215,69.9240036)(315.41748413,69.83400879)
\curveto(315.39748219,69.74400378)(315.39248219,69.65900386)(315.40248413,69.57900879)
\curveto(315.41248217,69.49900402)(315.44748214,69.43900408)(315.50748413,69.39900879)
\curveto(315.587482,69.33900418)(315.72248186,69.30900421)(315.91248413,69.30900879)
\curveto(316.11248147,69.3190042)(316.2824813,69.3240042)(316.42248413,69.32400879)
\lineto(318.70248413,69.32400879)
\curveto(318.85247873,69.3240042)(319.03247855,69.3190042)(319.24248413,69.30900879)
\curveto(319.45247813,69.30900421)(319.59247799,69.34900417)(319.66248413,69.42900879)
}
}
{
\newrgbcolor{curcolor}{0 0 0}
\pscustom[linestyle=none,fillstyle=solid,fillcolor=curcolor]
{
\newpath
\moveto(329.30912476,65.30400879)
\curveto(329.33911693,65.14400838)(329.32411694,65.00900851)(329.26412476,64.89900879)
\curveto(329.20411706,64.79900872)(329.12411714,64.7240088)(329.02412476,64.67400879)
\curveto(328.97411729,64.65400887)(328.91911735,64.64400888)(328.85912476,64.64400879)
\curveto(328.80911746,64.64400888)(328.75411751,64.63400889)(328.69412476,64.61400879)
\curveto(328.47411779,64.56400896)(328.25411801,64.57900894)(328.03412476,64.65900879)
\curveto(327.82411844,64.72900879)(327.67911859,64.8190087)(327.59912476,64.92900879)
\curveto(327.54911872,64.99900852)(327.50411876,65.07900844)(327.46412476,65.16900879)
\curveto(327.42411884,65.26900825)(327.37411889,65.34900817)(327.31412476,65.40900879)
\curveto(327.29411897,65.42900809)(327.269119,65.44900807)(327.23912476,65.46900879)
\curveto(327.21911905,65.48900803)(327.18911908,65.49400803)(327.14912476,65.48400879)
\curveto(327.03911923,65.45400807)(326.93411933,65.39900812)(326.83412476,65.31900879)
\curveto(326.74411952,65.23900828)(326.65411961,65.16900835)(326.56412476,65.10900879)
\curveto(326.43411983,65.02900849)(326.29411997,64.95400857)(326.14412476,64.88400879)
\curveto(325.99412027,64.8240087)(325.83412043,64.76900875)(325.66412476,64.71900879)
\curveto(325.5641207,64.68900883)(325.45412081,64.66900885)(325.33412476,64.65900879)
\curveto(325.22412104,64.64900887)(325.11412115,64.63400889)(325.00412476,64.61400879)
\curveto(324.95412131,64.60400892)(324.90912136,64.59900892)(324.86912476,64.59900879)
\lineto(324.76412476,64.59900879)
\curveto(324.65412161,64.57900894)(324.54912172,64.57900894)(324.44912476,64.59900879)
\lineto(324.31412476,64.59900879)
\curveto(324.264122,64.60900891)(324.21412205,64.61400891)(324.16412476,64.61400879)
\curveto(324.11412215,64.61400891)(324.0691222,64.6240089)(324.02912476,64.64400879)
\curveto(323.98912228,64.65400887)(323.95412231,64.65900886)(323.92412476,64.65900879)
\curveto(323.90412236,64.64900887)(323.87912239,64.64900887)(323.84912476,64.65900879)
\lineto(323.60912476,64.71900879)
\curveto(323.52912274,64.72900879)(323.45412281,64.74900877)(323.38412476,64.77900879)
\curveto(323.08412318,64.90900861)(322.83912343,65.05400847)(322.64912476,65.21400879)
\curveto(322.4691238,65.38400814)(322.31912395,65.6190079)(322.19912476,65.91900879)
\curveto(322.10912416,66.13900738)(322.0641242,66.40400712)(322.06412476,66.71400879)
\lineto(322.06412476,67.02900879)
\curveto(322.07412419,67.07900644)(322.07912419,67.12900639)(322.07912476,67.17900879)
\lineto(322.10912476,67.35900879)
\lineto(322.22912476,67.68900879)
\curveto(322.269124,67.79900572)(322.31912395,67.89900562)(322.37912476,67.98900879)
\curveto(322.55912371,68.27900524)(322.80412346,68.49400503)(323.11412476,68.63400879)
\curveto(323.42412284,68.77400475)(323.7641225,68.89900462)(324.13412476,69.00900879)
\curveto(324.27412199,69.04900447)(324.41912185,69.07900444)(324.56912476,69.09900879)
\curveto(324.71912155,69.1190044)(324.8691214,69.14400438)(325.01912476,69.17400879)
\curveto(325.08912118,69.19400433)(325.15412111,69.20400432)(325.21412476,69.20400879)
\curveto(325.28412098,69.20400432)(325.35912091,69.21400431)(325.43912476,69.23400879)
\curveto(325.50912076,69.25400427)(325.57912069,69.26400426)(325.64912476,69.26400879)
\curveto(325.71912055,69.27400425)(325.79412047,69.28900423)(325.87412476,69.30900879)
\curveto(326.12412014,69.36900415)(326.35911991,69.4190041)(326.57912476,69.45900879)
\curveto(326.79911947,69.50900401)(326.97411929,69.6240039)(327.10412476,69.80400879)
\curveto(327.1641191,69.88400364)(327.21411905,69.98400354)(327.25412476,70.10400879)
\curveto(327.29411897,70.23400329)(327.29411897,70.37400315)(327.25412476,70.52400879)
\curveto(327.19411907,70.76400276)(327.10411916,70.95400257)(326.98412476,71.09400879)
\curveto(326.87411939,71.23400229)(326.71411955,71.34400218)(326.50412476,71.42400879)
\curveto(326.38411988,71.47400205)(326.23912003,71.50900201)(326.06912476,71.52900879)
\curveto(325.90912036,71.54900197)(325.73912053,71.55900196)(325.55912476,71.55900879)
\curveto(325.37912089,71.55900196)(325.20412106,71.54900197)(325.03412476,71.52900879)
\curveto(324.8641214,71.50900201)(324.71912155,71.47900204)(324.59912476,71.43900879)
\curveto(324.42912184,71.37900214)(324.264122,71.29400223)(324.10412476,71.18400879)
\curveto(324.02412224,71.1240024)(323.94912232,71.04400248)(323.87912476,70.94400879)
\curveto(323.81912245,70.85400267)(323.7641225,70.75400277)(323.71412476,70.64400879)
\curveto(323.68412258,70.56400296)(323.65412261,70.47900304)(323.62412476,70.38900879)
\curveto(323.60412266,70.29900322)(323.55912271,70.22900329)(323.48912476,70.17900879)
\curveto(323.44912282,70.14900337)(323.37912289,70.1240034)(323.27912476,70.10400879)
\curveto(323.18912308,70.09400343)(323.09412317,70.08900343)(322.99412476,70.08900879)
\curveto(322.89412337,70.08900343)(322.79412347,70.09400343)(322.69412476,70.10400879)
\curveto(322.60412366,70.1240034)(322.53912373,70.14900337)(322.49912476,70.17900879)
\curveto(322.45912381,70.20900331)(322.42912384,70.25900326)(322.40912476,70.32900879)
\curveto(322.38912388,70.39900312)(322.38912388,70.47400305)(322.40912476,70.55400879)
\curveto(322.43912383,70.68400284)(322.4691238,70.80400272)(322.49912476,70.91400879)
\curveto(322.53912373,71.03400249)(322.58412368,71.14900237)(322.63412476,71.25900879)
\curveto(322.82412344,71.60900191)(323.0641232,71.87900164)(323.35412476,72.06900879)
\curveto(323.64412262,72.26900125)(324.00412226,72.42900109)(324.43412476,72.54900879)
\curveto(324.53412173,72.56900095)(324.63412163,72.58400094)(324.73412476,72.59400879)
\curveto(324.84412142,72.60400092)(324.95412131,72.6190009)(325.06412476,72.63900879)
\curveto(325.10412116,72.64900087)(325.1691211,72.64900087)(325.25912476,72.63900879)
\curveto(325.34912092,72.63900088)(325.40412086,72.64900087)(325.42412476,72.66900879)
\curveto(326.12412014,72.67900084)(326.73411953,72.59900092)(327.25412476,72.42900879)
\curveto(327.77411849,72.25900126)(328.13911813,71.93400159)(328.34912476,71.45400879)
\curveto(328.43911783,71.25400227)(328.48911778,71.0190025)(328.49912476,70.74900879)
\curveto(328.51911775,70.48900303)(328.52911774,70.21400331)(328.52912476,69.92400879)
\lineto(328.52912476,66.60900879)
\curveto(328.52911774,66.46900705)(328.53411773,66.33400719)(328.54412476,66.20400879)
\curveto(328.55411771,66.07400745)(328.58411768,65.96900755)(328.63412476,65.88900879)
\curveto(328.68411758,65.8190077)(328.74911752,65.76900775)(328.82912476,65.73900879)
\curveto(328.91911735,65.69900782)(329.00411726,65.66900785)(329.08412476,65.64900879)
\curveto(329.1641171,65.63900788)(329.22411704,65.59400793)(329.26412476,65.51400879)
\curveto(329.28411698,65.48400804)(329.29411697,65.45400807)(329.29412476,65.42400879)
\curveto(329.29411697,65.39400813)(329.29911697,65.35400817)(329.30912476,65.30400879)
\moveto(327.16412476,66.96900879)
\curveto(327.22411904,67.10900641)(327.25411901,67.26900625)(327.25412476,67.44900879)
\curveto(327.264119,67.63900588)(327.269119,67.83400569)(327.26912476,68.03400879)
\curveto(327.269119,68.14400538)(327.264119,68.24400528)(327.25412476,68.33400879)
\curveto(327.24411902,68.4240051)(327.20411906,68.49400503)(327.13412476,68.54400879)
\curveto(327.10411916,68.56400496)(327.03411923,68.57400495)(326.92412476,68.57400879)
\curveto(326.90411936,68.55400497)(326.8691194,68.54400498)(326.81912476,68.54400879)
\curveto(326.7691195,68.54400498)(326.72411954,68.53400499)(326.68412476,68.51400879)
\curveto(326.60411966,68.49400503)(326.51411975,68.47400505)(326.41412476,68.45400879)
\lineto(326.11412476,68.39400879)
\curveto(326.08412018,68.39400513)(326.04912022,68.38900513)(326.00912476,68.37900879)
\lineto(325.90412476,68.37900879)
\curveto(325.75412051,68.33900518)(325.58912068,68.31400521)(325.40912476,68.30400879)
\curveto(325.23912103,68.30400522)(325.07912119,68.28400524)(324.92912476,68.24400879)
\curveto(324.84912142,68.2240053)(324.77412149,68.20400532)(324.70412476,68.18400879)
\curveto(324.64412162,68.17400535)(324.57412169,68.15900536)(324.49412476,68.13900879)
\curveto(324.33412193,68.08900543)(324.18412208,68.0240055)(324.04412476,67.94400879)
\curveto(323.90412236,67.87400565)(323.78412248,67.78400574)(323.68412476,67.67400879)
\curveto(323.58412268,67.56400596)(323.50912276,67.42900609)(323.45912476,67.26900879)
\curveto(323.40912286,67.1190064)(323.38912288,66.93400659)(323.39912476,66.71400879)
\curveto(323.39912287,66.61400691)(323.41412285,66.519007)(323.44412476,66.42900879)
\curveto(323.48412278,66.34900717)(323.52912274,66.27400725)(323.57912476,66.20400879)
\curveto(323.65912261,66.09400743)(323.7641225,65.99900752)(323.89412476,65.91900879)
\curveto(324.02412224,65.84900767)(324.1641221,65.78900773)(324.31412476,65.73900879)
\curveto(324.3641219,65.72900779)(324.41412185,65.7240078)(324.46412476,65.72400879)
\curveto(324.51412175,65.7240078)(324.5641217,65.7190078)(324.61412476,65.70900879)
\curveto(324.68412158,65.68900783)(324.7691215,65.67400785)(324.86912476,65.66400879)
\curveto(324.97912129,65.66400786)(325.0691212,65.67400785)(325.13912476,65.69400879)
\curveto(325.19912107,65.71400781)(325.25912101,65.7190078)(325.31912476,65.70900879)
\curveto(325.37912089,65.70900781)(325.43912083,65.7190078)(325.49912476,65.73900879)
\curveto(325.57912069,65.75900776)(325.65412061,65.77400775)(325.72412476,65.78400879)
\curveto(325.80412046,65.79400773)(325.87912039,65.81400771)(325.94912476,65.84400879)
\curveto(326.23912003,65.96400756)(326.48411978,66.10900741)(326.68412476,66.27900879)
\curveto(326.89411937,66.44900707)(327.05411921,66.67900684)(327.16412476,66.96900879)
}
}
{
\newrgbcolor{curcolor}{0 0 0}
\pscustom[linestyle=none,fillstyle=solid,fillcolor=curcolor]
{
\newpath
\moveto(332.91076538,72.65400879)
\curveto(333.63076132,72.66400086)(334.23576071,72.57900094)(334.72576538,72.39900879)
\curveto(335.21575973,72.22900129)(335.59575935,71.9240016)(335.86576538,71.48400879)
\curveto(335.93575901,71.37400215)(335.99075896,71.25900226)(336.03076538,71.13900879)
\curveto(336.07075888,71.02900249)(336.11075884,70.90400262)(336.15076538,70.76400879)
\curveto(336.17075878,70.69400283)(336.17575877,70.6190029)(336.16576538,70.53900879)
\curveto(336.15575879,70.46900305)(336.14075881,70.41400311)(336.12076538,70.37400879)
\curveto(336.10075885,70.35400317)(336.07575887,70.33400319)(336.04576538,70.31400879)
\curveto(336.01575893,70.30400322)(335.99075896,70.28900323)(335.97076538,70.26900879)
\curveto(335.92075903,70.24900327)(335.87075908,70.24400328)(335.82076538,70.25400879)
\curveto(335.77075918,70.26400326)(335.72075923,70.26400326)(335.67076538,70.25400879)
\curveto(335.59075936,70.23400329)(335.48575946,70.22900329)(335.35576538,70.23900879)
\curveto(335.22575972,70.25900326)(335.13575981,70.28400324)(335.08576538,70.31400879)
\curveto(335.00575994,70.36400316)(334.95076,70.42900309)(334.92076538,70.50900879)
\curveto(334.90076005,70.59900292)(334.86576008,70.68400284)(334.81576538,70.76400879)
\curveto(334.72576022,70.9240026)(334.60076035,71.06900245)(334.44076538,71.19900879)
\curveto(334.33076062,71.27900224)(334.21076074,71.33900218)(334.08076538,71.37900879)
\curveto(333.950761,71.4190021)(333.81076114,71.45900206)(333.66076538,71.49900879)
\curveto(333.61076134,71.519002)(333.56076139,71.524002)(333.51076538,71.51400879)
\curveto(333.46076149,71.51400201)(333.41076154,71.519002)(333.36076538,71.52900879)
\curveto(333.30076165,71.54900197)(333.22576172,71.55900196)(333.13576538,71.55900879)
\curveto(333.0457619,71.55900196)(332.97076198,71.54900197)(332.91076538,71.52900879)
\lineto(332.82076538,71.52900879)
\lineto(332.67076538,71.49900879)
\curveto(332.62076233,71.49900202)(332.57076238,71.49400203)(332.52076538,71.48400879)
\curveto(332.26076269,71.4240021)(332.0457629,71.33900218)(331.87576538,71.22900879)
\curveto(331.70576324,71.1190024)(331.59076336,70.93400259)(331.53076538,70.67400879)
\curveto(331.51076344,70.60400292)(331.50576344,70.53400299)(331.51576538,70.46400879)
\curveto(331.53576341,70.39400313)(331.55576339,70.33400319)(331.57576538,70.28400879)
\curveto(331.63576331,70.13400339)(331.70576324,70.0240035)(331.78576538,69.95400879)
\curveto(331.87576307,69.89400363)(331.98576296,69.8240037)(332.11576538,69.74400879)
\curveto(332.27576267,69.64400388)(332.45576249,69.56900395)(332.65576538,69.51900879)
\curveto(332.85576209,69.47900404)(333.05576189,69.42900409)(333.25576538,69.36900879)
\curveto(333.38576156,69.32900419)(333.51576143,69.29900422)(333.64576538,69.27900879)
\curveto(333.77576117,69.25900426)(333.90576104,69.22900429)(334.03576538,69.18900879)
\curveto(334.2457607,69.12900439)(334.4507605,69.06900445)(334.65076538,69.00900879)
\curveto(334.8507601,68.95900456)(335.0507599,68.89400463)(335.25076538,68.81400879)
\lineto(335.40076538,68.75400879)
\curveto(335.4507595,68.73400479)(335.50075945,68.70900481)(335.55076538,68.67900879)
\curveto(335.7507592,68.55900496)(335.92575902,68.4240051)(336.07576538,68.27400879)
\curveto(336.22575872,68.1240054)(336.3507586,67.93400559)(336.45076538,67.70400879)
\curveto(336.47075848,67.63400589)(336.49075846,67.53900598)(336.51076538,67.41900879)
\curveto(336.53075842,67.34900617)(336.54075841,67.27400625)(336.54076538,67.19400879)
\curveto(336.5507584,67.1240064)(336.55575839,67.04400648)(336.55576538,66.95400879)
\lineto(336.55576538,66.80400879)
\curveto(336.53575841,66.73400679)(336.52575842,66.66400686)(336.52576538,66.59400879)
\curveto(336.52575842,66.524007)(336.51575843,66.45400707)(336.49576538,66.38400879)
\curveto(336.46575848,66.27400725)(336.43075852,66.16900735)(336.39076538,66.06900879)
\curveto(336.3507586,65.96900755)(336.30575864,65.87900764)(336.25576538,65.79900879)
\curveto(336.09575885,65.53900798)(335.89075906,65.32900819)(335.64076538,65.16900879)
\curveto(335.39075956,65.0190085)(335.11075984,64.88900863)(334.80076538,64.77900879)
\curveto(334.71076024,64.74900877)(334.61576033,64.72900879)(334.51576538,64.71900879)
\curveto(334.42576052,64.69900882)(334.33576061,64.67400885)(334.24576538,64.64400879)
\curveto(334.1457608,64.6240089)(334.0457609,64.61400891)(333.94576538,64.61400879)
\curveto(333.8457611,64.61400891)(333.7457612,64.60400892)(333.64576538,64.58400879)
\lineto(333.49576538,64.58400879)
\curveto(333.4457615,64.57400895)(333.37576157,64.56900895)(333.28576538,64.56900879)
\curveto(333.19576175,64.56900895)(333.12576182,64.57400895)(333.07576538,64.58400879)
\lineto(332.91076538,64.58400879)
\curveto(332.8507621,64.60400892)(332.78576216,64.61400891)(332.71576538,64.61400879)
\curveto(332.6457623,64.60400892)(332.58576236,64.60900891)(332.53576538,64.62900879)
\curveto(332.48576246,64.63900888)(332.42076253,64.64400888)(332.34076538,64.64400879)
\lineto(332.10076538,64.70400879)
\curveto(332.03076292,64.71400881)(331.95576299,64.73400879)(331.87576538,64.76400879)
\curveto(331.56576338,64.86400866)(331.29576365,64.98900853)(331.06576538,65.13900879)
\curveto(330.83576411,65.28900823)(330.63576431,65.48400804)(330.46576538,65.72400879)
\curveto(330.37576457,65.85400767)(330.30076465,65.98900753)(330.24076538,66.12900879)
\curveto(330.18076477,66.26900725)(330.12576482,66.4240071)(330.07576538,66.59400879)
\curveto(330.05576489,66.65400687)(330.0457649,66.7240068)(330.04576538,66.80400879)
\curveto(330.05576489,66.89400663)(330.07076488,66.96400656)(330.09076538,67.01400879)
\curveto(330.12076483,67.05400647)(330.17076478,67.09400643)(330.24076538,67.13400879)
\curveto(330.29076466,67.15400637)(330.36076459,67.16400636)(330.45076538,67.16400879)
\curveto(330.54076441,67.17400635)(330.63076432,67.17400635)(330.72076538,67.16400879)
\curveto(330.81076414,67.15400637)(330.89576405,67.13900638)(330.97576538,67.11900879)
\curveto(331.06576388,67.10900641)(331.12576382,67.09400643)(331.15576538,67.07400879)
\curveto(331.22576372,67.0240065)(331.27076368,66.94900657)(331.29076538,66.84900879)
\curveto(331.32076363,66.75900676)(331.35576359,66.67400685)(331.39576538,66.59400879)
\curveto(331.49576345,66.37400715)(331.63076332,66.20400732)(331.80076538,66.08400879)
\curveto(331.92076303,65.99400753)(332.05576289,65.9240076)(332.20576538,65.87400879)
\curveto(332.35576259,65.8240077)(332.51576243,65.77400775)(332.68576538,65.72400879)
\lineto(333.00076538,65.67900879)
\lineto(333.09076538,65.67900879)
\curveto(333.16076179,65.65900786)(333.2507617,65.64900787)(333.36076538,65.64900879)
\curveto(333.48076147,65.64900787)(333.58076137,65.65900786)(333.66076538,65.67900879)
\curveto(333.73076122,65.67900784)(333.78576116,65.68400784)(333.82576538,65.69400879)
\curveto(333.88576106,65.70400782)(333.945761,65.70900781)(334.00576538,65.70900879)
\curveto(334.06576088,65.7190078)(334.12076083,65.72900779)(334.17076538,65.73900879)
\curveto(334.46076049,65.8190077)(334.69076026,65.9240076)(334.86076538,66.05400879)
\curveto(335.03075992,66.18400734)(335.1507598,66.40400712)(335.22076538,66.71400879)
\curveto(335.24075971,66.76400676)(335.2457597,66.8190067)(335.23576538,66.87900879)
\curveto(335.22575972,66.93900658)(335.21575973,66.98400654)(335.20576538,67.01400879)
\curveto(335.15575979,67.20400632)(335.08575986,67.34400618)(334.99576538,67.43400879)
\curveto(334.90576004,67.53400599)(334.79076016,67.6240059)(334.65076538,67.70400879)
\curveto(334.56076039,67.76400576)(334.46076049,67.81400571)(334.35076538,67.85400879)
\lineto(334.02076538,67.97400879)
\curveto(333.99076096,67.98400554)(333.96076099,67.98900553)(333.93076538,67.98900879)
\curveto(333.91076104,67.98900553)(333.88576106,67.99900552)(333.85576538,68.01900879)
\curveto(333.51576143,68.12900539)(333.16076179,68.20900531)(332.79076538,68.25900879)
\curveto(332.43076252,68.3190052)(332.09076286,68.41400511)(331.77076538,68.54400879)
\curveto(331.67076328,68.58400494)(331.57576337,68.6190049)(331.48576538,68.64900879)
\curveto(331.39576355,68.67900484)(331.31076364,68.7190048)(331.23076538,68.76900879)
\curveto(331.04076391,68.87900464)(330.86576408,69.00400452)(330.70576538,69.14400879)
\curveto(330.5457644,69.28400424)(330.42076453,69.45900406)(330.33076538,69.66900879)
\curveto(330.30076465,69.73900378)(330.27576467,69.80900371)(330.25576538,69.87900879)
\curveto(330.2457647,69.94900357)(330.23076472,70.0240035)(330.21076538,70.10400879)
\curveto(330.18076477,70.2240033)(330.17076478,70.35900316)(330.18076538,70.50900879)
\curveto(330.19076476,70.66900285)(330.20576474,70.80400272)(330.22576538,70.91400879)
\curveto(330.2457647,70.96400256)(330.25576469,71.00400252)(330.25576538,71.03400879)
\curveto(330.26576468,71.07400245)(330.28076467,71.11400241)(330.30076538,71.15400879)
\curveto(330.39076456,71.38400214)(330.51076444,71.58400194)(330.66076538,71.75400879)
\curveto(330.82076413,71.9240016)(331.00076395,72.07400145)(331.20076538,72.20400879)
\curveto(331.3507636,72.29400123)(331.51576343,72.36400116)(331.69576538,72.41400879)
\curveto(331.87576307,72.47400105)(332.06576288,72.52900099)(332.26576538,72.57900879)
\curveto(332.33576261,72.58900093)(332.40076255,72.59900092)(332.46076538,72.60900879)
\curveto(332.53076242,72.6190009)(332.60576234,72.62900089)(332.68576538,72.63900879)
\curveto(332.71576223,72.64900087)(332.75576219,72.64900087)(332.80576538,72.63900879)
\curveto(332.85576209,72.62900089)(332.89076206,72.63400089)(332.91076538,72.65400879)
}
}
{
\newrgbcolor{curcolor}{0 0 0}
\pscustom[linestyle=none,fillstyle=solid,fillcolor=curcolor]
{
\newpath
\moveto(482.74775391,75.85150391)
\curveto(483.43774767,75.87149295)(484.04274706,75.80649302)(484.56275391,75.65650391)
\curveto(485.08274602,75.51649331)(485.54274556,75.30649352)(485.94275391,75.02650391)
\curveto(486.12274498,74.90649392)(486.28774482,74.77149405)(486.43775391,74.62150391)
\curveto(486.45774465,74.60149422)(486.47774463,74.58149424)(486.49775391,74.56150391)
\curveto(486.51774459,74.54149428)(486.53774457,74.5214943)(486.55775391,74.50150391)
\curveto(486.6077445,74.4214944)(486.66274444,74.34649448)(486.72275391,74.27650391)
\curveto(486.78274432,74.21649461)(486.83774427,74.14649468)(486.88775391,74.06650391)
\curveto(486.99774411,73.89649493)(487.09274401,73.71649511)(487.17275391,73.52650391)
\curveto(487.25274385,73.33649549)(487.33274377,73.14149568)(487.41275391,72.94150391)
\curveto(487.44274366,72.84149598)(487.45774365,72.73649609)(487.45775391,72.62650391)
\curveto(487.45774365,72.51649631)(487.41774369,72.43649639)(487.33775391,72.38650391)
\curveto(487.31774379,72.36649646)(487.28774382,72.35649647)(487.24775391,72.35650391)
\curveto(487.21774389,72.35649647)(487.18774392,72.35149647)(487.15775391,72.34150391)
\lineto(487.05275391,72.34150391)
\curveto(487.0027441,72.3214965)(486.91774419,72.31149651)(486.79775391,72.31150391)
\curveto(486.68774442,72.31149651)(486.6077445,72.3214965)(486.55775391,72.34150391)
\curveto(486.52774458,72.35149647)(486.49774461,72.35149647)(486.46775391,72.34150391)
\curveto(486.43774467,72.33149649)(486.4027447,72.33649649)(486.36275391,72.35650391)
\curveto(486.3027448,72.36649646)(486.24774486,72.39149643)(486.19775391,72.43150391)
\curveto(486.13774497,72.48149634)(486.09274501,72.55149627)(486.06275391,72.64150391)
\curveto(486.04274506,72.73149609)(486.01274509,72.81649601)(485.97275391,72.89650391)
\curveto(485.92274518,73.0264958)(485.86274524,73.14649568)(485.79275391,73.25650391)
\curveto(485.73274537,73.37649545)(485.66274544,73.49149533)(485.58275391,73.60150391)
\curveto(485.56274554,73.6214952)(485.53774557,73.64149518)(485.50775391,73.66150391)
\curveto(485.47774563,73.69149513)(485.45274565,73.7214951)(485.43275391,73.75150391)
\curveto(485.34274576,73.85149497)(485.25274585,73.93649489)(485.16275391,74.00650391)
\curveto(485.11274599,74.03649479)(485.06774604,74.06649476)(485.02775391,74.09650391)
\curveto(484.98774612,74.13649469)(484.94274616,74.17149465)(484.89275391,74.20150391)
\curveto(484.75274635,74.28149454)(484.6027465,74.35149447)(484.44275391,74.41150391)
\curveto(484.28274682,74.47149435)(484.11774699,74.5264943)(483.94775391,74.57650391)
\curveto(483.85774725,74.59649423)(483.76774734,74.61149421)(483.67775391,74.62150391)
\curveto(483.58774752,74.63149419)(483.49774761,74.64649418)(483.40775391,74.66650391)
\curveto(483.36774774,74.67649415)(483.32774778,74.67649415)(483.28775391,74.66650391)
\curveto(483.25774785,74.66649416)(483.22774788,74.67149415)(483.19775391,74.68150391)
\lineto(483.01775391,74.68150391)
\lineto(482.80775391,74.68150391)
\curveto(482.73774837,74.68149414)(482.67274843,74.67649415)(482.61275391,74.66650391)
\curveto(482.59274851,74.66649416)(482.56774854,74.66149416)(482.53775391,74.65150391)
\lineto(482.46275391,74.65150391)
\curveto(482.4027487,74.64149418)(482.33774877,74.63149419)(482.26775391,74.62150391)
\lineto(482.08775391,74.59150391)
\curveto(481.73774937,74.50149432)(481.43274967,74.37649445)(481.17275391,74.21650391)
\curveto(480.75275035,73.95649487)(480.41275069,73.64649518)(480.15275391,73.28650391)
\curveto(479.9027512,72.93649589)(479.69275141,72.50649632)(479.52275391,71.99650391)
\curveto(479.48275162,71.88649694)(479.45275165,71.77149705)(479.43275391,71.65150391)
\curveto(479.41275169,71.53149729)(479.38775172,71.41149741)(479.35775391,71.29150391)
\curveto(479.33775177,71.24149758)(479.32775178,71.19149763)(479.32775391,71.14150391)
\curveto(479.33775177,71.10149772)(479.33275177,71.05649777)(479.31275391,71.00650391)
\curveto(479.29275181,70.93649789)(479.28275182,70.86149796)(479.28275391,70.78150391)
\curveto(479.29275181,70.71149811)(479.28775182,70.63649819)(479.26775391,70.55650391)
\lineto(479.26775391,70.39150391)
\curveto(479.25775185,70.33149849)(479.25275185,70.24649858)(479.25275391,70.13650391)
\curveto(479.25275185,70.0264988)(479.25775185,69.94649888)(479.26775391,69.89650391)
\lineto(479.26775391,69.74650391)
\curveto(479.27775183,69.7264991)(479.28275182,69.69649913)(479.28275391,69.65650391)
\curveto(479.28275182,69.6264992)(479.28775182,69.60149922)(479.29775391,69.58150391)
\curveto(479.31775179,69.51149931)(479.32275178,69.44649938)(479.31275391,69.38650391)
\curveto(479.3027518,69.3264995)(479.3077518,69.26149956)(479.32775391,69.19150391)
\curveto(479.34775176,69.11149971)(479.36275174,69.03149979)(479.37275391,68.95150391)
\curveto(479.39275171,68.88149994)(479.41275169,68.80650002)(479.43275391,68.72650391)
\lineto(479.58275391,68.27650391)
\lineto(479.76275391,67.85650391)
\curveto(479.81275129,67.73650109)(479.87275123,67.6215012)(479.94275391,67.51150391)
\curveto(480.02275108,67.40150142)(480.102751,67.29650153)(480.18275391,67.19650391)
\curveto(480.55275055,66.69650213)(481.03775007,66.3265025)(481.63775391,66.08650391)
\curveto(481.7077494,66.05650277)(481.77774933,66.03150279)(481.84775391,66.01150391)
\curveto(481.91774919,65.99150283)(481.99274911,65.97150285)(482.07275391,65.95150391)
\curveto(482.31274879,65.88150294)(482.59274851,65.84650298)(482.91275391,65.84650391)
\lineto(483.10775391,65.84650391)
\curveto(483.17774793,65.84650298)(483.24274786,65.85150297)(483.30275391,65.86150391)
\curveto(483.35274775,65.88150294)(483.4027477,65.88650294)(483.45275391,65.87650391)
\curveto(483.51274759,65.86650296)(483.56774754,65.87150295)(483.61775391,65.89150391)
\curveto(483.75774735,65.93150289)(483.88774722,65.96150286)(484.00775391,65.98150391)
\curveto(484.13774697,66.01150281)(484.25774685,66.05150277)(484.36775391,66.10150391)
\curveto(485.29774581,66.47150235)(485.92274518,67.13650169)(486.24275391,68.09650391)
\curveto(486.26274484,68.17650065)(486.27774483,68.25650057)(486.28775391,68.33650391)
\lineto(486.34775391,68.57650391)
\curveto(486.37774473,68.69650013)(486.38774472,68.8265)(486.37775391,68.96650391)
\curveto(486.36774474,69.11649971)(486.32274478,69.2214996)(486.24275391,69.28150391)
\curveto(486.16274494,69.33149949)(486.05274505,69.35649947)(485.91275391,69.35650391)
\lineto(485.50775391,69.35650391)
\lineto(483.84275391,69.35650391)
\lineto(483.48275391,69.35650391)
\curveto(483.35274775,69.35649947)(483.24774786,69.37149945)(483.16775391,69.40150391)
\curveto(483.08774802,69.44149938)(483.03774807,69.49649933)(483.01775391,69.56650391)
\curveto(482.99774811,69.60649922)(482.98274812,69.66149916)(482.97275391,69.73150391)
\curveto(482.96274814,69.81149901)(482.95774815,69.89149893)(482.95775391,69.97150391)
\curveto(482.95774815,70.05149877)(482.96274814,70.1264987)(482.97275391,70.19650391)
\curveto(482.99274811,70.27649855)(483.01274809,70.33149849)(483.03275391,70.36150391)
\curveto(483.07274803,70.43149839)(483.14274796,70.48149834)(483.24275391,70.51150391)
\curveto(483.29274781,70.53149829)(483.35274775,70.54149828)(483.42275391,70.54150391)
\lineto(483.63275391,70.54150391)
\lineto(484.30775391,70.54150391)
\lineto(486.58775391,70.54150391)
\lineto(486.91775391,70.54150391)
\curveto(487.02774408,70.55149827)(487.12774398,70.54649828)(487.21775391,70.52650391)
\curveto(487.31774379,70.51649831)(487.4027437,70.49149833)(487.47275391,70.45150391)
\curveto(487.54274356,70.4214984)(487.59274351,70.36149846)(487.62275391,70.27150391)
\curveto(487.64274346,70.21149861)(487.64774346,70.14149868)(487.63775391,70.06150391)
\curveto(487.63774347,69.98149884)(487.63774347,69.90149892)(487.63775391,69.82150391)
\lineto(487.63775391,68.99650391)
\lineto(487.63775391,66.22150391)
\lineto(487.63775391,65.53150391)
\curveto(487.63774347,65.46150336)(487.63774347,65.39150343)(487.63775391,65.32150391)
\curveto(487.63774347,65.25150357)(487.62774348,65.19150363)(487.60775391,65.14150391)
\curveto(487.57774353,65.06150376)(487.51774359,65.00150382)(487.42775391,64.96150391)
\curveto(487.39774371,64.94150388)(487.33274377,64.93150389)(487.23275391,64.93150391)
\curveto(487.09274401,64.93150389)(486.98774412,64.94650388)(486.91775391,64.97650391)
\curveto(486.84774426,65.00650382)(486.79274431,65.05150377)(486.75275391,65.11150391)
\curveto(486.71274439,65.17150365)(486.67774443,65.24150358)(486.64775391,65.32150391)
\curveto(486.62774448,65.40150342)(486.6027445,65.49150333)(486.57275391,65.59150391)
\curveto(486.55274455,65.66150316)(486.51774459,65.75150307)(486.46775391,65.86150391)
\curveto(486.42774468,65.97150285)(486.35274475,66.01150281)(486.24275391,65.98150391)
\curveto(486.17274493,65.96150286)(486.11774499,65.93150289)(486.07775391,65.89150391)
\curveto(486.03774507,65.85150297)(485.99274511,65.81150301)(485.94275391,65.77150391)
\curveto(485.86274524,65.71150311)(485.78774532,65.64650318)(485.71775391,65.57650391)
\curveto(485.64774546,65.51650331)(485.57274553,65.46150336)(485.49275391,65.41150391)
\curveto(485.28274582,65.27150355)(485.05774605,65.15650367)(484.81775391,65.06650391)
\curveto(484.58774652,64.97650385)(484.33774677,64.89150393)(484.06775391,64.81150391)
\curveto(483.99774711,64.79150403)(483.92774718,64.77650405)(483.85775391,64.76650391)
\curveto(483.78774732,64.75650407)(483.71274739,64.74150408)(483.63275391,64.72150391)
\curveto(483.55274755,64.7215041)(483.49274761,64.71650411)(483.45275391,64.70650391)
\lineto(483.34775391,64.70650391)
\curveto(483.31774779,64.70650412)(483.28774782,64.70150412)(483.25775391,64.69150391)
\lineto(483.10775391,64.69150391)
\curveto(483.06774804,64.68150414)(483.01274809,64.67650415)(482.94275391,64.67650391)
\curveto(482.87274823,64.67650415)(482.81274829,64.68150414)(482.76275391,64.69150391)
\lineto(482.50775391,64.69150391)
\curveto(482.39774871,64.71150411)(482.29274881,64.7265041)(482.19275391,64.73650391)
\curveto(482.102749,64.73650409)(482.0077491,64.75150407)(481.90775391,64.78150391)
\curveto(481.72774938,64.83150399)(481.55274955,64.87650395)(481.38275391,64.91650391)
\curveto(481.21274989,64.95650387)(481.04775006,65.01150381)(480.88775391,65.08150391)
\curveto(480.31775079,65.34150348)(479.81775129,65.68650314)(479.38775391,66.11650391)
\curveto(478.95775215,66.55650227)(478.61275249,67.06650176)(478.35275391,67.64650391)
\curveto(478.3027528,67.76650106)(478.25275285,67.89150093)(478.20275391,68.02150391)
\curveto(478.16275294,68.15150067)(478.11775299,68.28650054)(478.06775391,68.42650391)
\curveto(478.05775305,68.46650036)(478.05275305,68.50150032)(478.05275391,68.53150391)
\curveto(478.05275305,68.57150025)(478.04275306,68.61150021)(478.02275391,68.65150391)
\curveto(477.99275311,68.76150006)(477.96775314,68.87649995)(477.94775391,68.99650391)
\curveto(477.93775317,69.11649971)(477.91775319,69.23649959)(477.88775391,69.35650391)
\curveto(477.87775323,69.39649943)(477.87275323,69.43649939)(477.87275391,69.47650391)
\curveto(477.88275322,69.51649931)(477.88275322,69.55149927)(477.87275391,69.58150391)
\lineto(477.87275391,69.71650391)
\curveto(477.85275325,69.76649906)(477.84275326,69.86149896)(477.84275391,70.00150391)
\curveto(477.84275326,70.14149868)(477.85275325,70.24149858)(477.87275391,70.30150391)
\lineto(477.87275391,70.43650391)
\lineto(477.87275391,70.61650391)
\curveto(477.87275323,70.67649815)(477.87775323,70.73649809)(477.88775391,70.79650391)
\curveto(477.89775321,70.84649798)(477.9027532,70.89649793)(477.90275391,70.94650391)
\curveto(477.9027532,71.00649782)(477.9077532,71.06149776)(477.91775391,71.11150391)
\curveto(477.94775316,71.23149759)(477.96775314,71.35149747)(477.97775391,71.47150391)
\curveto(477.99775311,71.59149723)(478.02275308,71.71149711)(478.05275391,71.83150391)
\curveto(478.15275295,72.16149666)(478.25275285,72.47649635)(478.35275391,72.77650391)
\curveto(478.46275264,73.08649574)(478.6027525,73.37149545)(478.77275391,73.63150391)
\curveto(479.07275203,74.10149472)(479.42775168,74.50149432)(479.83775391,74.83150391)
\curveto(480.25775085,75.17149365)(480.75275035,75.43649339)(481.32275391,75.62650391)
\curveto(481.43274967,75.66649316)(481.53774957,75.69149313)(481.63775391,75.70150391)
\curveto(481.74774936,75.7214931)(481.85774925,75.74649308)(481.96775391,75.77650391)
\curveto(482.01774909,75.79649303)(482.06274904,75.80649302)(482.10275391,75.80650391)
\curveto(482.15274895,75.80649302)(482.2027489,75.81149301)(482.25275391,75.82150391)
\curveto(482.33274877,75.83149299)(482.41274869,75.83649299)(482.49275391,75.83650391)
\curveto(482.58274852,75.84649298)(482.66774844,75.85149297)(482.74775391,75.85150391)
}
}
{
\newrgbcolor{curcolor}{0 0 0}
\pscustom[linestyle=none,fillstyle=solid,fillcolor=curcolor]
{
\newpath
\moveto(493.15423828,72.83650391)
\curveto(493.38423349,72.83649599)(493.51423336,72.77649605)(493.54423828,72.65650391)
\curveto(493.5742333,72.54649628)(493.58923329,72.38149644)(493.58923828,72.16150391)
\lineto(493.58923828,71.87650391)
\curveto(493.58923329,71.78649704)(493.56423331,71.71149711)(493.51423828,71.65150391)
\curveto(493.45423342,71.57149725)(493.36923351,71.5264973)(493.25923828,71.51650391)
\curveto(493.14923373,71.51649731)(493.03923384,71.50149732)(492.92923828,71.47150391)
\curveto(492.78923409,71.44149738)(492.65423422,71.41149741)(492.52423828,71.38150391)
\curveto(492.40423447,71.35149747)(492.28923459,71.31149751)(492.17923828,71.26150391)
\curveto(491.88923499,71.13149769)(491.65423522,70.95149787)(491.47423828,70.72150391)
\curveto(491.29423558,70.50149832)(491.13923574,70.24649858)(491.00923828,69.95650391)
\curveto(490.96923591,69.84649898)(490.93923594,69.73149909)(490.91923828,69.61150391)
\curveto(490.89923598,69.50149932)(490.874236,69.38649944)(490.84423828,69.26650391)
\curveto(490.83423604,69.21649961)(490.82923605,69.16649966)(490.82923828,69.11650391)
\curveto(490.83923604,69.06649976)(490.83923604,69.01649981)(490.82923828,68.96650391)
\curveto(490.79923608,68.84649998)(490.78423609,68.70650012)(490.78423828,68.54650391)
\curveto(490.79423608,68.39650043)(490.79923608,68.25150057)(490.79923828,68.11150391)
\lineto(490.79923828,66.26650391)
\lineto(490.79923828,65.92150391)
\curveto(490.79923608,65.80150302)(490.79423608,65.68650314)(490.78423828,65.57650391)
\curveto(490.7742361,65.46650336)(490.76923611,65.37150345)(490.76923828,65.29150391)
\curveto(490.7792361,65.21150361)(490.75923612,65.14150368)(490.70923828,65.08150391)
\curveto(490.65923622,65.01150381)(490.5792363,64.97150385)(490.46923828,64.96150391)
\curveto(490.36923651,64.95150387)(490.25923662,64.94650388)(490.13923828,64.94650391)
\lineto(489.86923828,64.94650391)
\curveto(489.81923706,64.96650386)(489.76923711,64.98150384)(489.71923828,64.99150391)
\curveto(489.6792372,65.01150381)(489.64923723,65.03650379)(489.62923828,65.06650391)
\curveto(489.5792373,65.13650369)(489.54923733,65.2215036)(489.53923828,65.32150391)
\lineto(489.53923828,65.65150391)
\lineto(489.53923828,66.80650391)
\lineto(489.53923828,70.96150391)
\lineto(489.53923828,71.99650391)
\lineto(489.53923828,72.29650391)
\curveto(489.54923733,72.39649643)(489.5792373,72.48149634)(489.62923828,72.55150391)
\curveto(489.65923722,72.59149623)(489.70923717,72.6214962)(489.77923828,72.64150391)
\curveto(489.85923702,72.66149616)(489.94423693,72.67149615)(490.03423828,72.67150391)
\curveto(490.12423675,72.68149614)(490.21423666,72.68149614)(490.30423828,72.67150391)
\curveto(490.39423648,72.66149616)(490.46423641,72.64649618)(490.51423828,72.62650391)
\curveto(490.59423628,72.59649623)(490.64423623,72.53649629)(490.66423828,72.44650391)
\curveto(490.69423618,72.36649646)(490.70923617,72.27649655)(490.70923828,72.17650391)
\lineto(490.70923828,71.87650391)
\curveto(490.70923617,71.77649705)(490.72923615,71.68649714)(490.76923828,71.60650391)
\curveto(490.7792361,71.58649724)(490.78923609,71.57149725)(490.79923828,71.56150391)
\lineto(490.84423828,71.51650391)
\curveto(490.95423592,71.51649731)(491.04423583,71.56149726)(491.11423828,71.65150391)
\curveto(491.18423569,71.75149707)(491.24423563,71.83149699)(491.29423828,71.89150391)
\lineto(491.38423828,71.98150391)
\curveto(491.4742354,72.09149673)(491.59923528,72.20649662)(491.75923828,72.32650391)
\curveto(491.91923496,72.44649638)(492.06923481,72.53649629)(492.20923828,72.59650391)
\curveto(492.29923458,72.64649618)(492.39423448,72.68149614)(492.49423828,72.70150391)
\curveto(492.59423428,72.73149609)(492.69923418,72.76149606)(492.80923828,72.79150391)
\curveto(492.86923401,72.80149602)(492.92923395,72.80649602)(492.98923828,72.80650391)
\curveto(493.04923383,72.81649601)(493.10423377,72.826496)(493.15423828,72.83650391)
}
}
{
\newrgbcolor{curcolor}{0 0 0}
\pscustom[linestyle=none,fillstyle=solid,fillcolor=curcolor]
{
\newpath
\moveto(494.98400391,72.65650391)
\lineto(495.41900391,72.65650391)
\curveto(495.56900194,72.65649617)(495.67400184,72.61649621)(495.73400391,72.53650391)
\curveto(495.78400173,72.45649637)(495.8090017,72.35649647)(495.80900391,72.23650391)
\curveto(495.81900169,72.11649671)(495.82400169,71.99649683)(495.82400391,71.87650391)
\lineto(495.82400391,70.45150391)
\lineto(495.82400391,68.18650391)
\lineto(495.82400391,67.49650391)
\curveto(495.82400169,67.26650156)(495.84900166,67.06650176)(495.89900391,66.89650391)
\curveto(496.05900145,66.44650238)(496.35900115,66.13150269)(496.79900391,65.95150391)
\curveto(497.01900049,65.86150296)(497.28400023,65.826503)(497.59400391,65.84650391)
\curveto(497.90399961,65.87650295)(498.15399936,65.93150289)(498.34400391,66.01150391)
\curveto(498.67399884,66.15150267)(498.93399858,66.3265025)(499.12400391,66.53650391)
\curveto(499.32399819,66.75650207)(499.47899803,67.04150178)(499.58900391,67.39150391)
\curveto(499.61899789,67.47150135)(499.63899787,67.55150127)(499.64900391,67.63150391)
\curveto(499.65899785,67.71150111)(499.67399784,67.79650103)(499.69400391,67.88650391)
\curveto(499.70399781,67.93650089)(499.70399781,67.98150084)(499.69400391,68.02150391)
\curveto(499.69399782,68.06150076)(499.70399781,68.10650072)(499.72400391,68.15650391)
\lineto(499.72400391,68.47150391)
\curveto(499.74399777,68.55150027)(499.74899776,68.64150018)(499.73900391,68.74150391)
\curveto(499.72899778,68.85149997)(499.72399779,68.95149987)(499.72400391,69.04150391)
\lineto(499.72400391,70.21150391)
\lineto(499.72400391,71.80150391)
\curveto(499.72399779,71.9214969)(499.71899779,72.04649678)(499.70900391,72.17650391)
\curveto(499.7089978,72.31649651)(499.73399778,72.4264964)(499.78400391,72.50650391)
\curveto(499.82399769,72.55649627)(499.86899764,72.58649624)(499.91900391,72.59650391)
\curveto(499.97899753,72.61649621)(500.04899746,72.63649619)(500.12900391,72.65650391)
\lineto(500.35400391,72.65650391)
\curveto(500.47399704,72.65649617)(500.57899693,72.65149617)(500.66900391,72.64150391)
\curveto(500.76899674,72.63149619)(500.84399667,72.58649624)(500.89400391,72.50650391)
\curveto(500.94399657,72.45649637)(500.96899654,72.38149644)(500.96900391,72.28150391)
\lineto(500.96900391,71.99650391)
\lineto(500.96900391,70.97650391)
\lineto(500.96900391,66.94150391)
\lineto(500.96900391,65.59150391)
\curveto(500.96899654,65.47150335)(500.96399655,65.35650347)(500.95400391,65.24650391)
\curveto(500.95399656,65.14650368)(500.91899659,65.07150375)(500.84900391,65.02150391)
\curveto(500.8089967,64.99150383)(500.74899676,64.96650386)(500.66900391,64.94650391)
\curveto(500.58899692,64.93650389)(500.49899701,64.9265039)(500.39900391,64.91650391)
\curveto(500.3089972,64.91650391)(500.21899729,64.9215039)(500.12900391,64.93150391)
\curveto(500.04899746,64.94150388)(499.98899752,64.96150386)(499.94900391,64.99150391)
\curveto(499.89899761,65.03150379)(499.85399766,65.09650373)(499.81400391,65.18650391)
\curveto(499.80399771,65.2265036)(499.79399772,65.28150354)(499.78400391,65.35150391)
\curveto(499.78399773,65.4215034)(499.77899773,65.48650334)(499.76900391,65.54650391)
\curveto(499.75899775,65.61650321)(499.73899777,65.67150315)(499.70900391,65.71150391)
\curveto(499.67899783,65.75150307)(499.63399788,65.76650306)(499.57400391,65.75650391)
\curveto(499.49399802,65.73650309)(499.4139981,65.67650315)(499.33400391,65.57650391)
\curveto(499.25399826,65.48650334)(499.17899833,65.41650341)(499.10900391,65.36650391)
\curveto(498.88899862,65.20650362)(498.63899887,65.06650376)(498.35900391,64.94650391)
\curveto(498.24899926,64.89650393)(498.13399938,64.86650396)(498.01400391,64.85650391)
\curveto(497.90399961,64.83650399)(497.78899972,64.81150401)(497.66900391,64.78150391)
\curveto(497.61899989,64.77150405)(497.56399995,64.77150405)(497.50400391,64.78150391)
\curveto(497.45400006,64.79150403)(497.40400011,64.78650404)(497.35400391,64.76650391)
\curveto(497.25400026,64.74650408)(497.16400035,64.74650408)(497.08400391,64.76650391)
\lineto(496.93400391,64.76650391)
\curveto(496.88400063,64.78650404)(496.82400069,64.79650403)(496.75400391,64.79650391)
\curveto(496.69400082,64.79650403)(496.63900087,64.80150402)(496.58900391,64.81150391)
\curveto(496.54900096,64.83150399)(496.509001,64.84150398)(496.46900391,64.84150391)
\curveto(496.43900107,64.83150399)(496.39900111,64.83650399)(496.34900391,64.85650391)
\lineto(496.10900391,64.91650391)
\curveto(496.03900147,64.93650389)(495.96400155,64.96650386)(495.88400391,65.00650391)
\curveto(495.62400189,65.11650371)(495.40400211,65.26150356)(495.22400391,65.44150391)
\curveto(495.05400246,65.63150319)(494.9140026,65.85650297)(494.80400391,66.11650391)
\curveto(494.76400275,66.20650262)(494.73400278,66.29650253)(494.71400391,66.38650391)
\lineto(494.65400391,66.68650391)
\curveto(494.63400288,66.74650208)(494.62400289,66.80150202)(494.62400391,66.85150391)
\curveto(494.63400288,66.91150191)(494.62900288,66.97650185)(494.60900391,67.04650391)
\curveto(494.59900291,67.06650176)(494.59400292,67.09150173)(494.59400391,67.12150391)
\curveto(494.59400292,67.16150166)(494.58900292,67.19650163)(494.57900391,67.22650391)
\lineto(494.57900391,67.37650391)
\curveto(494.56900294,67.41650141)(494.56400295,67.46150136)(494.56400391,67.51150391)
\curveto(494.57400294,67.57150125)(494.57900293,67.6265012)(494.57900391,67.67650391)
\lineto(494.57900391,68.27650391)
\lineto(494.57900391,71.03650391)
\lineto(494.57900391,71.99650391)
\lineto(494.57900391,72.26650391)
\curveto(494.57900293,72.35649647)(494.59900291,72.43149639)(494.63900391,72.49150391)
\curveto(494.67900283,72.56149626)(494.75400276,72.61149621)(494.86400391,72.64150391)
\curveto(494.88400263,72.65149617)(494.90400261,72.65149617)(494.92400391,72.64150391)
\curveto(494.94400257,72.64149618)(494.96400255,72.64649618)(494.98400391,72.65650391)
}
}
{
\newrgbcolor{curcolor}{0 0 0}
\pscustom[linestyle=none,fillstyle=solid,fillcolor=curcolor]
{
\newpath
\moveto(510.29361328,68.99650391)
\curveto(510.30360493,68.94649988)(510.30860493,68.88149994)(510.30861328,68.80150391)
\curveto(510.30860493,68.7215001)(510.30360493,68.65650017)(510.29361328,68.60650391)
\curveto(510.27360496,68.55650027)(510.26860497,68.50650032)(510.27861328,68.45650391)
\curveto(510.28860495,68.41650041)(510.28860495,68.37650045)(510.27861328,68.33650391)
\curveto(510.27860496,68.26650056)(510.27360496,68.21150061)(510.26361328,68.17150391)
\curveto(510.24360499,68.08150074)(510.22860501,67.99150083)(510.21861328,67.90150391)
\curveto(510.21860502,67.81150101)(510.20860503,67.7215011)(510.18861328,67.63150391)
\lineto(510.12861328,67.39150391)
\curveto(510.10860513,67.3215015)(510.08360515,67.24650158)(510.05361328,67.16650391)
\curveto(509.9336053,66.79650203)(509.76860547,66.46150236)(509.55861328,66.16150391)
\curveto(509.49860574,66.07150275)(509.4336058,65.98150284)(509.36361328,65.89150391)
\curveto(509.29360594,65.81150301)(509.21860602,65.73650309)(509.13861328,65.66650391)
\lineto(509.06361328,65.59150391)
\curveto(508.99360624,65.54150328)(508.92860631,65.49150333)(508.86861328,65.44150391)
\curveto(508.80860643,65.39150343)(508.7386065,65.34150348)(508.65861328,65.29150391)
\curveto(508.54860669,65.21150361)(508.42360681,65.14150368)(508.28361328,65.08150391)
\curveto(508.15360708,65.03150379)(508.01860722,64.98150384)(507.87861328,64.93150391)
\curveto(507.79860744,64.91150391)(507.71860752,64.89650393)(507.63861328,64.88650391)
\curveto(507.56860767,64.87650395)(507.49360774,64.86150396)(507.41361328,64.84150391)
\lineto(507.35361328,64.84150391)
\curveto(507.34360789,64.83150399)(507.32860791,64.826504)(507.30861328,64.82650391)
\curveto(507.21860802,64.80650402)(507.08360815,64.79650403)(506.90361328,64.79650391)
\curveto(506.7336085,64.78650404)(506.59860864,64.79150403)(506.49861328,64.81150391)
\lineto(506.42361328,64.81150391)
\curveto(506.35360888,64.821504)(506.28860895,64.83150399)(506.22861328,64.84150391)
\curveto(506.16860907,64.84150398)(506.10860913,64.85150397)(506.04861328,64.87150391)
\curveto(505.87860936,64.9215039)(505.71860952,64.96650386)(505.56861328,65.00650391)
\curveto(505.41860982,65.04650378)(505.27860996,65.10650372)(505.14861328,65.18650391)
\curveto(504.98861025,65.27650355)(504.84861039,65.37150345)(504.72861328,65.47150391)
\curveto(504.68861055,65.50150332)(504.62861061,65.54150328)(504.54861328,65.59150391)
\curveto(504.46861077,65.65150317)(504.39361084,65.65650317)(504.32361328,65.60650391)
\curveto(504.28361095,65.57650325)(504.26361097,65.53650329)(504.26361328,65.48650391)
\curveto(504.26361097,65.43650339)(504.25361098,65.38150344)(504.23361328,65.32150391)
\curveto(504.22361101,65.29150353)(504.22361101,65.25650357)(504.23361328,65.21650391)
\curveto(504.24361099,65.18650364)(504.24361099,65.15150367)(504.23361328,65.11150391)
\curveto(504.21361102,65.05150377)(504.20361103,64.98650384)(504.20361328,64.91650391)
\curveto(504.21361102,64.83650399)(504.21861102,64.76650406)(504.21861328,64.70650391)
\lineto(504.21861328,62.90650391)
\lineto(504.21861328,62.47150391)
\curveto(504.21861102,62.3215065)(504.18861105,62.20650662)(504.12861328,62.12650391)
\curveto(504.07861116,62.05650677)(503.99861124,62.0215068)(503.88861328,62.02150391)
\curveto(503.77861146,62.01150681)(503.66861157,62.00650682)(503.55861328,62.00650391)
\lineto(503.31861328,62.00650391)
\curveto(503.24861199,62.0265068)(503.18861205,62.04650678)(503.13861328,62.06650391)
\curveto(503.09861214,62.08650674)(503.06361217,62.1215067)(503.03361328,62.17150391)
\curveto(502.98361225,62.24150658)(502.95861228,62.35150647)(502.95861328,62.50150391)
\curveto(502.96861227,62.65150617)(502.97361226,62.78150604)(502.97361328,62.89150391)
\lineto(502.97361328,71.89150391)
\lineto(502.97361328,72.25150391)
\curveto(502.98361225,72.38149644)(503.01361222,72.48649634)(503.06361328,72.56650391)
\curveto(503.09361214,72.60649622)(503.15861208,72.63649619)(503.25861328,72.65650391)
\curveto(503.36861187,72.68649614)(503.48361175,72.69649613)(503.60361328,72.68650391)
\curveto(503.72361151,72.68649614)(503.8336114,72.67149615)(503.93361328,72.64150391)
\curveto(504.04361119,72.6214962)(504.11361112,72.59149623)(504.14361328,72.55150391)
\curveto(504.18361105,72.50149632)(504.20361103,72.44149638)(504.20361328,72.37150391)
\curveto(504.21361102,72.30149652)(504.233611,72.23149659)(504.26361328,72.16150391)
\curveto(504.28361095,72.13149669)(504.29861094,72.10649672)(504.30861328,72.08650391)
\curveto(504.32861091,72.07649675)(504.34861089,72.06149676)(504.36861328,72.04150391)
\curveto(504.47861076,72.03149679)(504.56861067,72.06649676)(504.63861328,72.14650391)
\curveto(504.71861052,72.2264966)(504.79361044,72.29149653)(504.86361328,72.34150391)
\curveto(505.12361011,72.5214963)(505.4336098,72.66149616)(505.79361328,72.76150391)
\curveto(505.88360935,72.78149604)(505.97360926,72.79649603)(506.06361328,72.80650391)
\curveto(506.16360907,72.81649601)(506.26360897,72.83149599)(506.36361328,72.85150391)
\curveto(506.40360883,72.86149596)(506.45360878,72.86149596)(506.51361328,72.85150391)
\curveto(506.57360866,72.84149598)(506.61360862,72.84649598)(506.63361328,72.86650391)
\curveto(507.06360817,72.87649595)(507.44360779,72.83149599)(507.77361328,72.73150391)
\curveto(508.10360713,72.64149618)(508.39860684,72.51149631)(508.65861328,72.34150391)
\lineto(508.80861328,72.22150391)
\curveto(508.85860638,72.19149663)(508.90860633,72.15649667)(508.95861328,72.11650391)
\curveto(508.97860626,72.09649673)(508.99360624,72.07649675)(509.00361328,72.05650391)
\curveto(509.02360621,72.04649678)(509.04360619,72.03149679)(509.06361328,72.01150391)
\curveto(509.11360612,71.96149686)(509.16860607,71.90649692)(509.22861328,71.84650391)
\curveto(509.28860595,71.78649704)(509.34360589,71.7264971)(509.39361328,71.66650391)
\curveto(509.51360572,71.49649733)(509.6386056,71.31149751)(509.76861328,71.11150391)
\curveto(509.84860539,70.98149784)(509.91360532,70.83649799)(509.96361328,70.67650391)
\curveto(510.02360521,70.51649831)(510.07860516,70.35649847)(510.12861328,70.19650391)
\curveto(510.14860509,70.11649871)(510.16360507,70.03149879)(510.17361328,69.94150391)
\curveto(510.19360504,69.85149897)(510.21360502,69.76649906)(510.23361328,69.68650391)
\lineto(510.23361328,69.56650391)
\curveto(510.24360499,69.53649929)(510.24860499,69.50649932)(510.24861328,69.47650391)
\curveto(510.26860497,69.4264994)(510.27360496,69.37149945)(510.26361328,69.31150391)
\curveto(510.26360497,69.25149957)(510.27360496,69.19649963)(510.29361328,69.14650391)
\lineto(510.29361328,68.99650391)
\moveto(508.95861328,68.59150391)
\curveto(508.97860626,68.64150018)(508.98360625,68.70150012)(508.97361328,68.77150391)
\curveto(508.96360627,68.85149997)(508.95860628,68.9214999)(508.95861328,68.98150391)
\curveto(508.95860628,69.15149967)(508.94860629,69.31149951)(508.92861328,69.46150391)
\curveto(508.91860632,69.61149921)(508.88860635,69.75649907)(508.83861328,69.89650391)
\lineto(508.77861328,70.07650391)
\curveto(508.76860647,70.14649868)(508.74860649,70.21149861)(508.71861328,70.27150391)
\curveto(508.60860663,70.54149828)(508.4336068,70.80149802)(508.19361328,71.05150391)
\curveto(507.96360727,71.30149752)(507.74360749,71.47149735)(507.53361328,71.56150391)
\curveto(507.45360778,71.60149722)(507.36860787,71.63149719)(507.27861328,71.65150391)
\curveto(507.19860804,71.67149715)(507.11360812,71.69649713)(507.02361328,71.72650391)
\curveto(506.9336083,71.74649708)(506.82860841,71.75649707)(506.70861328,71.75650391)
\lineto(506.37861328,71.75650391)
\curveto(506.35860888,71.73649709)(506.31860892,71.7264971)(506.25861328,71.72650391)
\curveto(506.20860903,71.73649709)(506.16360907,71.73649709)(506.12361328,71.72650391)
\lineto(505.85361328,71.66650391)
\curveto(505.77360946,71.64649718)(505.69360954,71.61649721)(505.61361328,71.57650391)
\curveto(505.29360994,71.43649739)(505.02861021,71.23149759)(504.81861328,70.96150391)
\curveto(504.61861062,70.70149812)(504.46361077,70.39649843)(504.35361328,70.04650391)
\curveto(504.31361092,69.93649889)(504.28361095,69.826499)(504.26361328,69.71650391)
\curveto(504.25361098,69.60649922)(504.238611,69.49649933)(504.21861328,69.38650391)
\curveto(504.20861103,69.34649948)(504.20361103,69.30649952)(504.20361328,69.26650391)
\curveto(504.20361103,69.23649959)(504.19861104,69.20149962)(504.18861328,69.16150391)
\lineto(504.18861328,69.04150391)
\curveto(504.17861106,68.99149983)(504.17361106,68.91649991)(504.17361328,68.81650391)
\curveto(504.17361106,68.7265001)(504.17861106,68.65650017)(504.18861328,68.60650391)
\lineto(504.18861328,68.48650391)
\curveto(504.19861104,68.44650038)(504.20361103,68.40650042)(504.20361328,68.36650391)
\curveto(504.20361103,68.3265005)(504.20861103,68.29150053)(504.21861328,68.26150391)
\curveto(504.22861101,68.23150059)(504.233611,68.20150062)(504.23361328,68.17150391)
\curveto(504.233611,68.14150068)(504.238611,68.10650072)(504.24861328,68.06650391)
\curveto(504.26861097,67.98650084)(504.28361095,67.90650092)(504.29361328,67.82650391)
\lineto(504.35361328,67.58650391)
\curveto(504.46361077,67.24650158)(504.61361062,66.94650188)(504.80361328,66.68650391)
\curveto(505.00361023,66.43650239)(505.26360997,66.24150258)(505.58361328,66.10150391)
\curveto(505.77360946,66.0215028)(505.96860927,65.96150286)(506.16861328,65.92150391)
\curveto(506.20860903,65.90150292)(506.24860899,65.89150293)(506.28861328,65.89150391)
\curveto(506.32860891,65.90150292)(506.36860887,65.90150292)(506.40861328,65.89150391)
\lineto(506.52861328,65.89150391)
\curveto(506.59860864,65.87150295)(506.66860857,65.87150295)(506.73861328,65.89150391)
\lineto(506.85861328,65.89150391)
\curveto(506.96860827,65.91150291)(507.07360816,65.9265029)(507.17361328,65.93650391)
\curveto(507.27360796,65.94650288)(507.37360786,65.97150285)(507.47361328,66.01150391)
\curveto(507.78360745,66.14150268)(508.0336072,66.31150251)(508.22361328,66.52150391)
\curveto(508.42360681,66.74150208)(508.58860665,67.00650182)(508.71861328,67.31650391)
\curveto(508.76860647,67.45650137)(508.80360643,67.59650123)(508.82361328,67.73650391)
\curveto(508.85360638,67.88650094)(508.88860635,68.04150078)(508.92861328,68.20150391)
\curveto(508.9386063,68.25150057)(508.94360629,68.29650053)(508.94361328,68.33650391)
\curveto(508.94360629,68.37650045)(508.94860629,68.4215004)(508.95861328,68.47150391)
\lineto(508.95861328,68.59150391)
}
}
{
\newrgbcolor{curcolor}{0 0 0}
\pscustom[linestyle=none,fillstyle=solid,fillcolor=curcolor]
{
\newpath
\moveto(518.89986328,69.13150391)
\curveto(518.91985522,69.07149975)(518.92985521,68.97649985)(518.92986328,68.84650391)
\curveto(518.92985521,68.7265001)(518.92485522,68.64150018)(518.91486328,68.59150391)
\lineto(518.91486328,68.44150391)
\curveto(518.90485524,68.36150046)(518.89485525,68.28650054)(518.88486328,68.21650391)
\curveto(518.88485526,68.15650067)(518.87985526,68.08650074)(518.86986328,68.00650391)
\curveto(518.84985529,67.94650088)(518.83485531,67.88650094)(518.82486328,67.82650391)
\curveto(518.82485532,67.76650106)(518.81485533,67.70650112)(518.79486328,67.64650391)
\curveto(518.75485539,67.51650131)(518.71985542,67.38650144)(518.68986328,67.25650391)
\curveto(518.65985548,67.1265017)(518.61985552,67.00650182)(518.56986328,66.89650391)
\curveto(518.35985578,66.41650241)(518.07985606,66.01150281)(517.72986328,65.68150391)
\curveto(517.37985676,65.36150346)(516.94985719,65.11650371)(516.43986328,64.94650391)
\curveto(516.32985781,64.90650392)(516.20985793,64.87650395)(516.07986328,64.85650391)
\curveto(515.95985818,64.83650399)(515.83485831,64.81650401)(515.70486328,64.79650391)
\curveto(515.6448585,64.78650404)(515.57985856,64.78150404)(515.50986328,64.78150391)
\curveto(515.44985869,64.77150405)(515.38985875,64.76650406)(515.32986328,64.76650391)
\curveto(515.28985885,64.75650407)(515.22985891,64.75150407)(515.14986328,64.75150391)
\curveto(515.07985906,64.75150407)(515.02985911,64.75650407)(514.99986328,64.76650391)
\curveto(514.95985918,64.77650405)(514.91985922,64.78150404)(514.87986328,64.78150391)
\curveto(514.8398593,64.77150405)(514.80485934,64.77150405)(514.77486328,64.78150391)
\lineto(514.68486328,64.78150391)
\lineto(514.32486328,64.82650391)
\curveto(514.18485996,64.86650396)(514.04986009,64.90650392)(513.91986328,64.94650391)
\curveto(513.78986035,64.98650384)(513.66486048,65.03150379)(513.54486328,65.08150391)
\curveto(513.09486105,65.28150354)(512.72486142,65.54150328)(512.43486328,65.86150391)
\curveto(512.144862,66.18150264)(511.90486224,66.57150225)(511.71486328,67.03150391)
\curveto(511.66486248,67.13150169)(511.62486252,67.23150159)(511.59486328,67.33150391)
\curveto(511.57486257,67.43150139)(511.55486259,67.53650129)(511.53486328,67.64650391)
\curveto(511.51486263,67.68650114)(511.50486264,67.71650111)(511.50486328,67.73650391)
\curveto(511.51486263,67.76650106)(511.51486263,67.80150102)(511.50486328,67.84150391)
\curveto(511.48486266,67.9215009)(511.46986267,68.00150082)(511.45986328,68.08150391)
\curveto(511.45986268,68.17150065)(511.44986269,68.25650057)(511.42986328,68.33650391)
\lineto(511.42986328,68.45650391)
\curveto(511.42986271,68.49650033)(511.42486272,68.54150028)(511.41486328,68.59150391)
\curveto(511.40486274,68.64150018)(511.39986274,68.7265001)(511.39986328,68.84650391)
\curveto(511.39986274,68.97649985)(511.40986273,69.07149975)(511.42986328,69.13150391)
\curveto(511.44986269,69.20149962)(511.45486269,69.27149955)(511.44486328,69.34150391)
\curveto(511.43486271,69.41149941)(511.4398627,69.48149934)(511.45986328,69.55150391)
\curveto(511.46986267,69.60149922)(511.47486267,69.64149918)(511.47486328,69.67150391)
\curveto(511.48486266,69.71149911)(511.49486265,69.75649907)(511.50486328,69.80650391)
\curveto(511.53486261,69.9264989)(511.55986258,70.04649878)(511.57986328,70.16650391)
\curveto(511.60986253,70.28649854)(511.64986249,70.40149842)(511.69986328,70.51150391)
\curveto(511.84986229,70.88149794)(512.02986211,71.21149761)(512.23986328,71.50150391)
\curveto(512.45986168,71.80149702)(512.72486142,72.05149677)(513.03486328,72.25150391)
\curveto(513.15486099,72.33149649)(513.27986086,72.39649643)(513.40986328,72.44650391)
\curveto(513.5398606,72.50649632)(513.67486047,72.56649626)(513.81486328,72.62650391)
\curveto(513.93486021,72.67649615)(514.06486008,72.70649612)(514.20486328,72.71650391)
\curveto(514.3448598,72.73649609)(514.48485966,72.76649606)(514.62486328,72.80650391)
\lineto(514.81986328,72.80650391)
\curveto(514.88985925,72.81649601)(514.95485919,72.826496)(515.01486328,72.83650391)
\curveto(515.90485824,72.84649598)(516.6448575,72.66149616)(517.23486328,72.28150391)
\curveto(517.82485632,71.90149692)(518.24985589,71.40649742)(518.50986328,70.79650391)
\curveto(518.55985558,70.69649813)(518.59985554,70.59649823)(518.62986328,70.49650391)
\curveto(518.65985548,70.39649843)(518.69485545,70.29149853)(518.73486328,70.18150391)
\curveto(518.76485538,70.07149875)(518.78985535,69.95149887)(518.80986328,69.82150391)
\curveto(518.82985531,69.70149912)(518.85485529,69.57649925)(518.88486328,69.44650391)
\curveto(518.89485525,69.39649943)(518.89485525,69.34149948)(518.88486328,69.28150391)
\curveto(518.88485526,69.23149959)(518.88985525,69.18149964)(518.89986328,69.13150391)
\moveto(517.56486328,68.27650391)
\curveto(517.58485656,68.34650048)(517.58985655,68.4265004)(517.57986328,68.51650391)
\lineto(517.57986328,68.77150391)
\curveto(517.57985656,69.16149966)(517.5448566,69.49149933)(517.47486328,69.76150391)
\curveto(517.4448567,69.84149898)(517.41985672,69.9214989)(517.39986328,70.00150391)
\curveto(517.37985676,70.08149874)(517.35485679,70.15649867)(517.32486328,70.22650391)
\curveto(517.0448571,70.87649795)(516.59985754,71.3264975)(515.98986328,71.57650391)
\curveto(515.91985822,71.60649722)(515.8448583,71.6264972)(515.76486328,71.63650391)
\lineto(515.52486328,71.69650391)
\curveto(515.4448587,71.71649711)(515.35985878,71.7264971)(515.26986328,71.72650391)
\lineto(514.99986328,71.72650391)
\lineto(514.72986328,71.68150391)
\curveto(514.62985951,71.66149716)(514.53485961,71.63649719)(514.44486328,71.60650391)
\curveto(514.36485978,71.58649724)(514.28485986,71.55649727)(514.20486328,71.51650391)
\curveto(514.13486001,71.49649733)(514.06986007,71.46649736)(514.00986328,71.42650391)
\curveto(513.94986019,71.38649744)(513.89486025,71.34649748)(513.84486328,71.30650391)
\curveto(513.60486054,71.13649769)(513.40986073,70.93149789)(513.25986328,70.69150391)
\curveto(513.10986103,70.45149837)(512.97986116,70.17149865)(512.86986328,69.85150391)
\curveto(512.8398613,69.75149907)(512.81986132,69.64649918)(512.80986328,69.53650391)
\curveto(512.79986134,69.43649939)(512.78486136,69.33149949)(512.76486328,69.22150391)
\curveto(512.75486139,69.18149964)(512.74986139,69.11649971)(512.74986328,69.02650391)
\curveto(512.7398614,68.99649983)(512.73486141,68.96149986)(512.73486328,68.92150391)
\curveto(512.7448614,68.88149994)(512.74986139,68.83649999)(512.74986328,68.78650391)
\lineto(512.74986328,68.48650391)
\curveto(512.74986139,68.38650044)(512.75986138,68.29650053)(512.77986328,68.21650391)
\lineto(512.80986328,68.03650391)
\curveto(512.82986131,67.93650089)(512.8448613,67.83650099)(512.85486328,67.73650391)
\curveto(512.87486127,67.64650118)(512.90486124,67.56150126)(512.94486328,67.48150391)
\curveto(513.0448611,67.24150158)(513.15986098,67.01650181)(513.28986328,66.80650391)
\curveto(513.42986071,66.59650223)(513.59986054,66.4215024)(513.79986328,66.28150391)
\curveto(513.84986029,66.25150257)(513.89486025,66.2265026)(513.93486328,66.20650391)
\curveto(513.97486017,66.18650264)(514.01986012,66.16150266)(514.06986328,66.13150391)
\curveto(514.14985999,66.08150274)(514.23485991,66.03650279)(514.32486328,65.99650391)
\curveto(514.42485972,65.96650286)(514.52985961,65.93650289)(514.63986328,65.90650391)
\curveto(514.68985945,65.88650294)(514.73485941,65.87650295)(514.77486328,65.87650391)
\curveto(514.82485932,65.88650294)(514.87485927,65.88650294)(514.92486328,65.87650391)
\curveto(514.95485919,65.86650296)(515.01485913,65.85650297)(515.10486328,65.84650391)
\curveto(515.20485894,65.83650299)(515.27985886,65.84150298)(515.32986328,65.86150391)
\curveto(515.36985877,65.87150295)(515.40985873,65.87150295)(515.44986328,65.86150391)
\curveto(515.48985865,65.86150296)(515.52985861,65.87150295)(515.56986328,65.89150391)
\curveto(515.64985849,65.91150291)(515.72985841,65.9265029)(515.80986328,65.93650391)
\curveto(515.88985825,65.95650287)(515.96485818,65.98150284)(516.03486328,66.01150391)
\curveto(516.37485777,66.15150267)(516.64985749,66.34650248)(516.85986328,66.59650391)
\curveto(517.06985707,66.84650198)(517.2448569,67.14150168)(517.38486328,67.48150391)
\curveto(517.43485671,67.60150122)(517.46485668,67.7265011)(517.47486328,67.85650391)
\curveto(517.49485665,67.99650083)(517.52485662,68.13650069)(517.56486328,68.27650391)
}
}
{
\newrgbcolor{curcolor}{0 0 0}
\pscustom[linestyle=none,fillstyle=solid,fillcolor=curcolor]
{
\newpath
\moveto(522.81814453,72.83650391)
\curveto(523.53814047,72.84649598)(524.14313986,72.76149606)(524.63314453,72.58150391)
\curveto(525.12313888,72.41149641)(525.5031385,72.10649672)(525.77314453,71.66650391)
\curveto(525.84313816,71.55649727)(525.89813811,71.44149738)(525.93814453,71.32150391)
\curveto(525.97813803,71.21149761)(526.01813799,71.08649774)(526.05814453,70.94650391)
\curveto(526.07813793,70.87649795)(526.08313792,70.80149802)(526.07314453,70.72150391)
\curveto(526.06313794,70.65149817)(526.04813796,70.59649823)(526.02814453,70.55650391)
\curveto(526.008138,70.53649829)(525.98313802,70.51649831)(525.95314453,70.49650391)
\curveto(525.92313808,70.48649834)(525.89813811,70.47149835)(525.87814453,70.45150391)
\curveto(525.82813818,70.43149839)(525.77813823,70.4264984)(525.72814453,70.43650391)
\curveto(525.67813833,70.44649838)(525.62813838,70.44649838)(525.57814453,70.43650391)
\curveto(525.49813851,70.41649841)(525.39313861,70.41149841)(525.26314453,70.42150391)
\curveto(525.13313887,70.44149838)(525.04313896,70.46649836)(524.99314453,70.49650391)
\curveto(524.91313909,70.54649828)(524.85813915,70.61149821)(524.82814453,70.69150391)
\curveto(524.8081392,70.78149804)(524.77313923,70.86649796)(524.72314453,70.94650391)
\curveto(524.63313937,71.10649772)(524.5081395,71.25149757)(524.34814453,71.38150391)
\curveto(524.23813977,71.46149736)(524.11813989,71.5214973)(523.98814453,71.56150391)
\curveto(523.85814015,71.60149722)(523.71814029,71.64149718)(523.56814453,71.68150391)
\curveto(523.51814049,71.70149712)(523.46814054,71.70649712)(523.41814453,71.69650391)
\curveto(523.36814064,71.69649713)(523.31814069,71.70149712)(523.26814453,71.71150391)
\curveto(523.2081408,71.73149709)(523.13314087,71.74149708)(523.04314453,71.74150391)
\curveto(522.95314105,71.74149708)(522.87814113,71.73149709)(522.81814453,71.71150391)
\lineto(522.72814453,71.71150391)
\lineto(522.57814453,71.68150391)
\curveto(522.52814148,71.68149714)(522.47814153,71.67649715)(522.42814453,71.66650391)
\curveto(522.16814184,71.60649722)(521.95314205,71.5214973)(521.78314453,71.41150391)
\curveto(521.61314239,71.30149752)(521.49814251,71.11649771)(521.43814453,70.85650391)
\curveto(521.41814259,70.78649804)(521.41314259,70.71649811)(521.42314453,70.64650391)
\curveto(521.44314256,70.57649825)(521.46314254,70.51649831)(521.48314453,70.46650391)
\curveto(521.54314246,70.31649851)(521.61314239,70.20649862)(521.69314453,70.13650391)
\curveto(521.78314222,70.07649875)(521.89314211,70.00649882)(522.02314453,69.92650391)
\curveto(522.18314182,69.826499)(522.36314164,69.75149907)(522.56314453,69.70150391)
\curveto(522.76314124,69.66149916)(522.96314104,69.61149921)(523.16314453,69.55150391)
\curveto(523.29314071,69.51149931)(523.42314058,69.48149934)(523.55314453,69.46150391)
\curveto(523.68314032,69.44149938)(523.81314019,69.41149941)(523.94314453,69.37150391)
\curveto(524.15313985,69.31149951)(524.35813965,69.25149957)(524.55814453,69.19150391)
\curveto(524.75813925,69.14149968)(524.95813905,69.07649975)(525.15814453,68.99650391)
\lineto(525.30814453,68.93650391)
\curveto(525.35813865,68.91649991)(525.4081386,68.89149993)(525.45814453,68.86150391)
\curveto(525.65813835,68.74150008)(525.83313817,68.60650022)(525.98314453,68.45650391)
\curveto(526.13313787,68.30650052)(526.25813775,68.11650071)(526.35814453,67.88650391)
\curveto(526.37813763,67.81650101)(526.39813761,67.7215011)(526.41814453,67.60150391)
\curveto(526.43813757,67.53150129)(526.44813756,67.45650137)(526.44814453,67.37650391)
\curveto(526.45813755,67.30650152)(526.46313754,67.2265016)(526.46314453,67.13650391)
\lineto(526.46314453,66.98650391)
\curveto(526.44313756,66.91650191)(526.43313757,66.84650198)(526.43314453,66.77650391)
\curveto(526.43313757,66.70650212)(526.42313758,66.63650219)(526.40314453,66.56650391)
\curveto(526.37313763,66.45650237)(526.33813767,66.35150247)(526.29814453,66.25150391)
\curveto(526.25813775,66.15150267)(526.21313779,66.06150276)(526.16314453,65.98150391)
\curveto(526.003138,65.7215031)(525.79813821,65.51150331)(525.54814453,65.35150391)
\curveto(525.29813871,65.20150362)(525.01813899,65.07150375)(524.70814453,64.96150391)
\curveto(524.61813939,64.93150389)(524.52313948,64.91150391)(524.42314453,64.90150391)
\curveto(524.33313967,64.88150394)(524.24313976,64.85650397)(524.15314453,64.82650391)
\curveto(524.05313995,64.80650402)(523.95314005,64.79650403)(523.85314453,64.79650391)
\curveto(523.75314025,64.79650403)(523.65314035,64.78650404)(523.55314453,64.76650391)
\lineto(523.40314453,64.76650391)
\curveto(523.35314065,64.75650407)(523.28314072,64.75150407)(523.19314453,64.75150391)
\curveto(523.1031409,64.75150407)(523.03314097,64.75650407)(522.98314453,64.76650391)
\lineto(522.81814453,64.76650391)
\curveto(522.75814125,64.78650404)(522.69314131,64.79650403)(522.62314453,64.79650391)
\curveto(522.55314145,64.78650404)(522.49314151,64.79150403)(522.44314453,64.81150391)
\curveto(522.39314161,64.821504)(522.32814168,64.826504)(522.24814453,64.82650391)
\lineto(522.00814453,64.88650391)
\curveto(521.93814207,64.89650393)(521.86314214,64.91650391)(521.78314453,64.94650391)
\curveto(521.47314253,65.04650378)(521.2031428,65.17150365)(520.97314453,65.32150391)
\curveto(520.74314326,65.47150335)(520.54314346,65.66650316)(520.37314453,65.90650391)
\curveto(520.28314372,66.03650279)(520.2081438,66.17150265)(520.14814453,66.31150391)
\curveto(520.08814392,66.45150237)(520.03314397,66.60650222)(519.98314453,66.77650391)
\curveto(519.96314404,66.83650199)(519.95314405,66.90650192)(519.95314453,66.98650391)
\curveto(519.96314404,67.07650175)(519.97814403,67.14650168)(519.99814453,67.19650391)
\curveto(520.02814398,67.23650159)(520.07814393,67.27650155)(520.14814453,67.31650391)
\curveto(520.19814381,67.33650149)(520.26814374,67.34650148)(520.35814453,67.34650391)
\curveto(520.44814356,67.35650147)(520.53814347,67.35650147)(520.62814453,67.34650391)
\curveto(520.71814329,67.33650149)(520.8031432,67.3215015)(520.88314453,67.30150391)
\curveto(520.97314303,67.29150153)(521.03314297,67.27650155)(521.06314453,67.25650391)
\curveto(521.13314287,67.20650162)(521.17814283,67.13150169)(521.19814453,67.03150391)
\curveto(521.22814278,66.94150188)(521.26314274,66.85650197)(521.30314453,66.77650391)
\curveto(521.4031426,66.55650227)(521.53814247,66.38650244)(521.70814453,66.26650391)
\curveto(521.82814218,66.17650265)(521.96314204,66.10650272)(522.11314453,66.05650391)
\curveto(522.26314174,66.00650282)(522.42314158,65.95650287)(522.59314453,65.90650391)
\lineto(522.90814453,65.86150391)
\lineto(522.99814453,65.86150391)
\curveto(523.06814094,65.84150298)(523.15814085,65.83150299)(523.26814453,65.83150391)
\curveto(523.38814062,65.83150299)(523.48814052,65.84150298)(523.56814453,65.86150391)
\curveto(523.63814037,65.86150296)(523.69314031,65.86650296)(523.73314453,65.87650391)
\curveto(523.79314021,65.88650294)(523.85314015,65.89150293)(523.91314453,65.89150391)
\curveto(523.97314003,65.90150292)(524.02813998,65.91150291)(524.07814453,65.92150391)
\curveto(524.36813964,66.00150282)(524.59813941,66.10650272)(524.76814453,66.23650391)
\curveto(524.93813907,66.36650246)(525.05813895,66.58650224)(525.12814453,66.89650391)
\curveto(525.14813886,66.94650188)(525.15313885,67.00150182)(525.14314453,67.06150391)
\curveto(525.13313887,67.1215017)(525.12313888,67.16650166)(525.11314453,67.19650391)
\curveto(525.06313894,67.38650144)(524.99313901,67.5265013)(524.90314453,67.61650391)
\curveto(524.81313919,67.71650111)(524.69813931,67.80650102)(524.55814453,67.88650391)
\curveto(524.46813954,67.94650088)(524.36813964,67.99650083)(524.25814453,68.03650391)
\lineto(523.92814453,68.15650391)
\curveto(523.89814011,68.16650066)(523.86814014,68.17150065)(523.83814453,68.17150391)
\curveto(523.81814019,68.17150065)(523.79314021,68.18150064)(523.76314453,68.20150391)
\curveto(523.42314058,68.31150051)(523.06814094,68.39150043)(522.69814453,68.44150391)
\curveto(522.33814167,68.50150032)(521.99814201,68.59650023)(521.67814453,68.72650391)
\curveto(521.57814243,68.76650006)(521.48314252,68.80150002)(521.39314453,68.83150391)
\curveto(521.3031427,68.86149996)(521.21814279,68.90149992)(521.13814453,68.95150391)
\curveto(520.94814306,69.06149976)(520.77314323,69.18649964)(520.61314453,69.32650391)
\curveto(520.45314355,69.46649936)(520.32814368,69.64149918)(520.23814453,69.85150391)
\curveto(520.2081438,69.9214989)(520.18314382,69.99149883)(520.16314453,70.06150391)
\curveto(520.15314385,70.13149869)(520.13814387,70.20649862)(520.11814453,70.28650391)
\curveto(520.08814392,70.40649842)(520.07814393,70.54149828)(520.08814453,70.69150391)
\curveto(520.09814391,70.85149797)(520.11314389,70.98649784)(520.13314453,71.09650391)
\curveto(520.15314385,71.14649768)(520.16314384,71.18649764)(520.16314453,71.21650391)
\curveto(520.17314383,71.25649757)(520.18814382,71.29649753)(520.20814453,71.33650391)
\curveto(520.29814371,71.56649726)(520.41814359,71.76649706)(520.56814453,71.93650391)
\curveto(520.72814328,72.10649672)(520.9081431,72.25649657)(521.10814453,72.38650391)
\curveto(521.25814275,72.47649635)(521.42314258,72.54649628)(521.60314453,72.59650391)
\curveto(521.78314222,72.65649617)(521.97314203,72.71149611)(522.17314453,72.76150391)
\curveto(522.24314176,72.77149605)(522.3081417,72.78149604)(522.36814453,72.79150391)
\curveto(522.43814157,72.80149602)(522.51314149,72.81149601)(522.59314453,72.82150391)
\curveto(522.62314138,72.83149599)(522.66314134,72.83149599)(522.71314453,72.82150391)
\curveto(522.76314124,72.81149601)(522.79814121,72.81649601)(522.81814453,72.83650391)
}
}
{
\newrgbcolor{curcolor}{0 0 0}
\pscustom[linestyle=none,fillstyle=solid,fillcolor=curcolor]
{
\newpath
\moveto(644.0099292,75.92651611)
\curveto(644.9899227,75.94650516)(645.80992188,75.78650532)(646.4699292,75.44651611)
\curveto(647.13992055,75.11650599)(647.65992003,74.65650645)(648.0299292,74.06651611)
\curveto(648.12991956,73.9065072)(648.20991948,73.75150735)(648.2699292,73.60151611)
\curveto(648.33991935,73.46150764)(648.40491928,73.29150781)(648.4649292,73.09151611)
\curveto(648.4849192,73.04150806)(648.50491918,72.97150813)(648.5249292,72.88151611)
\curveto(648.54491914,72.8015083)(648.53991915,72.72650838)(648.5099292,72.65651611)
\curveto(648.4899192,72.59650851)(648.44991924,72.55650855)(648.3899292,72.53651611)
\curveto(648.33991935,72.52650858)(648.2849194,72.51150859)(648.2249292,72.49151611)
\lineto(648.0749292,72.49151611)
\curveto(648.04491964,72.48150862)(648.00491968,72.47650863)(647.9549292,72.47651611)
\lineto(647.8349292,72.47651611)
\curveto(647.69491999,72.47650863)(647.56492012,72.48150862)(647.4449292,72.49151611)
\curveto(647.33492035,72.51150859)(647.25492043,72.56150854)(647.2049292,72.64151611)
\curveto(647.13492055,72.74150836)(647.07992061,72.85650825)(647.0399292,72.98651611)
\curveto(646.99992069,73.11650799)(646.94492074,73.23650787)(646.8749292,73.34651611)
\curveto(646.74492094,73.56650754)(646.59492109,73.75650735)(646.4249292,73.91651611)
\curveto(646.26492142,74.07650703)(646.07492161,74.22650688)(645.8549292,74.36651611)
\curveto(645.73492195,74.44650666)(645.59992209,74.5065066)(645.4499292,74.54651611)
\curveto(645.30992238,74.58650652)(645.16492252,74.62650648)(645.0149292,74.66651611)
\curveto(644.90492278,74.69650641)(644.77992291,74.71650639)(644.6399292,74.72651611)
\curveto(644.49992319,74.74650636)(644.34992334,74.75650635)(644.1899292,74.75651611)
\curveto(644.03992365,74.75650635)(643.8899238,74.74650636)(643.7399292,74.72651611)
\curveto(643.59992409,74.71650639)(643.47992421,74.69650641)(643.3799292,74.66651611)
\curveto(643.27992441,74.64650646)(643.1849245,74.62650648)(643.0949292,74.60651611)
\curveto(643.00492468,74.58650652)(642.91492477,74.55650655)(642.8249292,74.51651611)
\curveto(641.9849257,74.16650694)(641.37992631,73.56650754)(641.0099292,72.71651611)
\curveto(640.93992675,72.57650853)(640.87992681,72.42650868)(640.8299292,72.26651611)
\curveto(640.7899269,72.11650899)(640.74492694,71.96150914)(640.6949292,71.80151611)
\curveto(640.67492701,71.74150936)(640.66492702,71.67650943)(640.6649292,71.60651611)
\curveto(640.66492702,71.54650956)(640.65492703,71.48650962)(640.6349292,71.42651611)
\curveto(640.62492706,71.38650972)(640.61992707,71.35150975)(640.6199292,71.32151611)
\curveto(640.61992707,71.29150981)(640.61492707,71.25650985)(640.6049292,71.21651611)
\curveto(640.5849271,71.10651)(640.56992712,70.99151011)(640.5599292,70.87151611)
\lineto(640.5599292,70.52651611)
\curveto(640.55992713,70.45651065)(640.55492713,70.38151072)(640.5449292,70.30151611)
\curveto(640.54492714,70.23151087)(640.54992714,70.16651094)(640.5599292,70.10651611)
\lineto(640.5599292,69.95651611)
\curveto(640.57992711,69.88651122)(640.5849271,69.81651129)(640.5749292,69.74651611)
\curveto(640.57492711,69.67651143)(640.5849271,69.6065115)(640.6049292,69.53651611)
\curveto(640.62492706,69.47651163)(640.62992706,69.41651169)(640.6199292,69.35651611)
\curveto(640.61992707,69.29651181)(640.62992706,69.24151186)(640.6499292,69.19151611)
\curveto(640.67992701,69.06151204)(640.70492698,68.93151217)(640.7249292,68.80151611)
\curveto(640.75492693,68.68151242)(640.7899269,68.56151254)(640.8299292,68.44151611)
\curveto(640.99992669,67.94151316)(641.21992647,67.51151359)(641.4899292,67.15151611)
\curveto(641.75992593,66.8015143)(642.11492557,66.51151459)(642.5549292,66.28151611)
\curveto(642.69492499,66.21151489)(642.83492485,66.15651495)(642.9749292,66.11651611)
\curveto(643.12492456,66.07651503)(643.2849244,66.03151507)(643.4549292,65.98151611)
\curveto(643.52492416,65.96151514)(643.5899241,65.95151515)(643.6499292,65.95151611)
\curveto(643.70992398,65.96151514)(643.77992391,65.95651515)(643.8599292,65.93651611)
\curveto(643.90992378,65.92651518)(643.99992369,65.91651519)(644.1299292,65.90651611)
\curveto(644.25992343,65.9065152)(644.35492333,65.91651519)(644.4149292,65.93651611)
\lineto(644.5199292,65.93651611)
\curveto(644.55992313,65.94651516)(644.59992309,65.94651516)(644.6399292,65.93651611)
\curveto(644.67992301,65.93651517)(644.71992297,65.94651516)(644.7599292,65.96651611)
\curveto(644.85992283,65.98651512)(644.95492273,66.0015151)(645.0449292,66.01151611)
\curveto(645.14492254,66.03151507)(645.23992245,66.06151504)(645.3299292,66.10151611)
\curveto(646.10992158,66.42151468)(646.65992103,66.94651416)(646.9799292,67.67651611)
\curveto(647.05992063,67.85651325)(647.13492055,68.07151303)(647.2049292,68.32151611)
\curveto(647.22492046,68.41151269)(647.23992045,68.5015126)(647.2499292,68.59151611)
\curveto(647.26992042,68.69151241)(647.30492038,68.78151232)(647.3549292,68.86151611)
\curveto(647.40492028,68.94151216)(647.4849202,68.98651212)(647.5949292,68.99651611)
\curveto(647.70491998,69.0065121)(647.82491986,69.01151209)(647.9549292,69.01151611)
\lineto(648.1049292,69.01151611)
\curveto(648.15491953,69.01151209)(648.19991949,69.0065121)(648.2399292,68.99651611)
\lineto(648.3449292,68.99651611)
\lineto(648.4349292,68.96651611)
\curveto(648.47491921,68.96651214)(648.50491918,68.95651215)(648.5249292,68.93651611)
\curveto(648.59491909,68.89651221)(648.63491905,68.82151228)(648.6449292,68.71151611)
\curveto(648.65491903,68.61151249)(648.64491904,68.51151259)(648.6149292,68.41151611)
\curveto(648.55491913,68.18151292)(648.49991919,67.96151314)(648.4499292,67.75151611)
\curveto(648.39991929,67.54151356)(648.32491936,67.34151376)(648.2249292,67.15151611)
\curveto(648.14491954,67.02151408)(648.06991962,66.89651421)(647.9999292,66.77651611)
\curveto(647.93991975,66.65651445)(647.86991982,66.53651457)(647.7899292,66.41651611)
\curveto(647.60992008,66.15651495)(647.3849203,65.91651519)(647.1149292,65.69651611)
\curveto(646.85492083,65.48651562)(646.56992112,65.31151579)(646.2599292,65.17151611)
\curveto(646.14992154,65.12151598)(646.03992165,65.08151602)(645.9299292,65.05151611)
\curveto(645.82992186,65.02151608)(645.72492196,64.98651612)(645.6149292,64.94651611)
\curveto(645.50492218,64.9065162)(645.3899223,64.88151622)(645.2699292,64.87151611)
\curveto(645.15992253,64.85151625)(645.04492264,64.83151627)(644.9249292,64.81151611)
\curveto(644.87492281,64.79151631)(644.82992286,64.78651632)(644.7899292,64.79651611)
\curveto(644.74992294,64.79651631)(644.70992298,64.79151631)(644.6699292,64.78151611)
\curveto(644.60992308,64.77151633)(644.54992314,64.76651634)(644.4899292,64.76651611)
\curveto(644.42992326,64.76651634)(644.36492332,64.76151634)(644.2949292,64.75151611)
\curveto(644.26492342,64.74151636)(644.19492349,64.74151636)(644.0849292,64.75151611)
\curveto(643.9849237,64.75151635)(643.91992377,64.75651635)(643.8899292,64.76651611)
\curveto(643.83992385,64.77651633)(643.7899239,64.78151632)(643.7399292,64.78151611)
\curveto(643.69992399,64.77151633)(643.65492403,64.77151633)(643.6049292,64.78151611)
\lineto(643.4549292,64.78151611)
\curveto(643.37492431,64.8015163)(643.29992439,64.81651629)(643.2299292,64.82651611)
\curveto(643.15992453,64.82651628)(643.0849246,64.83651627)(643.0049292,64.85651611)
\lineto(642.7349292,64.91651611)
\curveto(642.64492504,64.92651618)(642.55992513,64.94651616)(642.4799292,64.97651611)
\curveto(642.26992542,65.03651607)(642.07992561,65.11151599)(641.9099292,65.20151611)
\curveto(641.27992641,65.47151563)(640.76992692,65.85651525)(640.3799292,66.35651611)
\curveto(639.9899277,66.85651425)(639.67992801,67.44651366)(639.4499292,68.12651611)
\curveto(639.40992828,68.24651286)(639.37492831,68.37151273)(639.3449292,68.50151611)
\curveto(639.32492836,68.63151247)(639.29992839,68.76651234)(639.2699292,68.90651611)
\curveto(639.24992844,68.95651215)(639.23992845,69.0015121)(639.2399292,69.04151611)
\curveto(639.24992844,69.08151202)(639.24992844,69.12651198)(639.2399292,69.17651611)
\curveto(639.21992847,69.26651184)(639.20492848,69.36151174)(639.1949292,69.46151611)
\curveto(639.19492849,69.56151154)(639.1849285,69.65651145)(639.1649292,69.74651611)
\lineto(639.1649292,70.03151611)
\curveto(639.14492854,70.08151102)(639.13492855,70.16651094)(639.1349292,70.28651611)
\curveto(639.13492855,70.4065107)(639.14492854,70.49151061)(639.1649292,70.54151611)
\curveto(639.17492851,70.57151053)(639.17492851,70.6015105)(639.1649292,70.63151611)
\curveto(639.15492853,70.67151043)(639.15492853,70.7015104)(639.1649292,70.72151611)
\lineto(639.1649292,70.85651611)
\curveto(639.17492851,70.93651017)(639.17992851,71.01651009)(639.1799292,71.09651611)
\curveto(639.1899285,71.18650992)(639.20492848,71.27150983)(639.2249292,71.35151611)
\curveto(639.24492844,71.41150969)(639.25492843,71.47150963)(639.2549292,71.53151611)
\curveto(639.25492843,71.6015095)(639.26492842,71.67150943)(639.2849292,71.74151611)
\curveto(639.33492835,71.91150919)(639.37492831,72.07650903)(639.4049292,72.23651611)
\curveto(639.43492825,72.39650871)(639.47992821,72.54650856)(639.5399292,72.68651611)
\lineto(639.6899292,73.07651611)
\curveto(639.74992794,73.21650789)(639.81492787,73.34150776)(639.8849292,73.45151611)
\curveto(640.03492765,73.71150739)(640.1849275,73.94650716)(640.3349292,74.15651611)
\curveto(640.36492732,74.2065069)(640.39992729,74.24650686)(640.4399292,74.27651611)
\curveto(640.4899272,74.31650679)(640.52992716,74.36150674)(640.5599292,74.41151611)
\curveto(640.61992707,74.49150661)(640.67992701,74.56150654)(640.7399292,74.62151611)
\lineto(640.9499292,74.80151611)
\curveto(641.00992668,74.85150625)(641.06492662,74.89650621)(641.1149292,74.93651611)
\curveto(641.17492651,74.98650612)(641.23992645,75.03650607)(641.3099292,75.08651611)
\curveto(641.45992623,75.19650591)(641.61492607,75.29150581)(641.7749292,75.37151611)
\curveto(641.94492574,75.45150565)(642.11992557,75.53150557)(642.2999292,75.61151611)
\curveto(642.40992528,75.66150544)(642.52492516,75.69650541)(642.6449292,75.71651611)
\curveto(642.77492491,75.74650536)(642.89992479,75.78150532)(643.0199292,75.82151611)
\curveto(643.0899246,75.83150527)(643.15492453,75.84150526)(643.2149292,75.85151611)
\lineto(643.3949292,75.88151611)
\curveto(643.47492421,75.89150521)(643.54992414,75.89650521)(643.6199292,75.89651611)
\curveto(643.69992399,75.9065052)(643.77992391,75.91650519)(643.8599292,75.92651611)
\curveto(643.87992381,75.93650517)(643.90492378,75.93650517)(643.9349292,75.92651611)
\curveto(643.96492372,75.91650519)(643.9899237,75.91650519)(644.0099292,75.92651611)
}
}
{
\newrgbcolor{curcolor}{0 0 0}
\pscustom[linestyle=none,fillstyle=solid,fillcolor=curcolor]
{
\newpath
\moveto(657.36977295,69.20651611)
\curveto(657.38976489,69.14651196)(657.39976488,69.05151205)(657.39977295,68.92151611)
\curveto(657.39976488,68.8015123)(657.39476488,68.71651239)(657.38477295,68.66651611)
\lineto(657.38477295,68.51651611)
\curveto(657.3747649,68.43651267)(657.36476491,68.36151274)(657.35477295,68.29151611)
\curveto(657.35476492,68.23151287)(657.34976493,68.16151294)(657.33977295,68.08151611)
\curveto(657.31976496,68.02151308)(657.30476497,67.96151314)(657.29477295,67.90151611)
\curveto(657.29476498,67.84151326)(657.28476499,67.78151332)(657.26477295,67.72151611)
\curveto(657.22476505,67.59151351)(657.18976509,67.46151364)(657.15977295,67.33151611)
\curveto(657.12976515,67.2015139)(657.08976519,67.08151402)(657.03977295,66.97151611)
\curveto(656.82976545,66.49151461)(656.54976573,66.08651502)(656.19977295,65.75651611)
\curveto(655.84976643,65.43651567)(655.41976686,65.19151591)(654.90977295,65.02151611)
\curveto(654.79976748,64.98151612)(654.6797676,64.95151615)(654.54977295,64.93151611)
\curveto(654.42976785,64.91151619)(654.30476797,64.89151621)(654.17477295,64.87151611)
\curveto(654.11476816,64.86151624)(654.04976823,64.85651625)(653.97977295,64.85651611)
\curveto(653.91976836,64.84651626)(653.85976842,64.84151626)(653.79977295,64.84151611)
\curveto(653.75976852,64.83151627)(653.69976858,64.82651628)(653.61977295,64.82651611)
\curveto(653.54976873,64.82651628)(653.49976878,64.83151627)(653.46977295,64.84151611)
\curveto(653.42976885,64.85151625)(653.38976889,64.85651625)(653.34977295,64.85651611)
\curveto(653.30976897,64.84651626)(653.274769,64.84651626)(653.24477295,64.85651611)
\lineto(653.15477295,64.85651611)
\lineto(652.79477295,64.90151611)
\curveto(652.65476962,64.94151616)(652.51976976,64.98151612)(652.38977295,65.02151611)
\curveto(652.25977002,65.06151604)(652.13477014,65.106516)(652.01477295,65.15651611)
\curveto(651.56477071,65.35651575)(651.19477108,65.61651549)(650.90477295,65.93651611)
\curveto(650.61477166,66.25651485)(650.3747719,66.64651446)(650.18477295,67.10651611)
\curveto(650.13477214,67.2065139)(650.09477218,67.3065138)(650.06477295,67.40651611)
\curveto(650.04477223,67.5065136)(650.02477225,67.61151349)(650.00477295,67.72151611)
\curveto(649.98477229,67.76151334)(649.9747723,67.79151331)(649.97477295,67.81151611)
\curveto(649.98477229,67.84151326)(649.98477229,67.87651323)(649.97477295,67.91651611)
\curveto(649.95477232,67.99651311)(649.93977234,68.07651303)(649.92977295,68.15651611)
\curveto(649.92977235,68.24651286)(649.91977236,68.33151277)(649.89977295,68.41151611)
\lineto(649.89977295,68.53151611)
\curveto(649.89977238,68.57151253)(649.89477238,68.61651249)(649.88477295,68.66651611)
\curveto(649.8747724,68.71651239)(649.86977241,68.8015123)(649.86977295,68.92151611)
\curveto(649.86977241,69.05151205)(649.8797724,69.14651196)(649.89977295,69.20651611)
\curveto(649.91977236,69.27651183)(649.92477235,69.34651176)(649.91477295,69.41651611)
\curveto(649.90477237,69.48651162)(649.90977237,69.55651155)(649.92977295,69.62651611)
\curveto(649.93977234,69.67651143)(649.94477233,69.71651139)(649.94477295,69.74651611)
\curveto(649.95477232,69.78651132)(649.96477231,69.83151127)(649.97477295,69.88151611)
\curveto(650.00477227,70.0015111)(650.02977225,70.12151098)(650.04977295,70.24151611)
\curveto(650.0797722,70.36151074)(650.11977216,70.47651063)(650.16977295,70.58651611)
\curveto(650.31977196,70.95651015)(650.49977178,71.28650982)(650.70977295,71.57651611)
\curveto(650.92977135,71.87650923)(651.19477108,72.12650898)(651.50477295,72.32651611)
\curveto(651.62477065,72.4065087)(651.74977053,72.47150863)(651.87977295,72.52151611)
\curveto(652.00977027,72.58150852)(652.14477013,72.64150846)(652.28477295,72.70151611)
\curveto(652.40476987,72.75150835)(652.53476974,72.78150832)(652.67477295,72.79151611)
\curveto(652.81476946,72.81150829)(652.95476932,72.84150826)(653.09477295,72.88151611)
\lineto(653.28977295,72.88151611)
\curveto(653.35976892,72.89150821)(653.42476885,72.9015082)(653.48477295,72.91151611)
\curveto(654.3747679,72.92150818)(655.11476716,72.73650837)(655.70477295,72.35651611)
\curveto(656.29476598,71.97650913)(656.71976556,71.48150962)(656.97977295,70.87151611)
\curveto(657.02976525,70.77151033)(657.06976521,70.67151043)(657.09977295,70.57151611)
\curveto(657.12976515,70.47151063)(657.16476511,70.36651074)(657.20477295,70.25651611)
\curveto(657.23476504,70.14651096)(657.25976502,70.02651108)(657.27977295,69.89651611)
\curveto(657.29976498,69.77651133)(657.32476495,69.65151145)(657.35477295,69.52151611)
\curveto(657.36476491,69.47151163)(657.36476491,69.41651169)(657.35477295,69.35651611)
\curveto(657.35476492,69.3065118)(657.35976492,69.25651185)(657.36977295,69.20651611)
\moveto(656.03477295,68.35151611)
\curveto(656.05476622,68.42151268)(656.05976622,68.5015126)(656.04977295,68.59151611)
\lineto(656.04977295,68.84651611)
\curveto(656.04976623,69.23651187)(656.01476626,69.56651154)(655.94477295,69.83651611)
\curveto(655.91476636,69.91651119)(655.88976639,69.99651111)(655.86977295,70.07651611)
\curveto(655.84976643,70.15651095)(655.82476645,70.23151087)(655.79477295,70.30151611)
\curveto(655.51476676,70.95151015)(655.06976721,71.4015097)(654.45977295,71.65151611)
\curveto(654.38976789,71.68150942)(654.31476796,71.7015094)(654.23477295,71.71151611)
\lineto(653.99477295,71.77151611)
\curveto(653.91476836,71.79150931)(653.82976845,71.8015093)(653.73977295,71.80151611)
\lineto(653.46977295,71.80151611)
\lineto(653.19977295,71.75651611)
\curveto(653.09976918,71.73650937)(653.00476927,71.71150939)(652.91477295,71.68151611)
\curveto(652.83476944,71.66150944)(652.75476952,71.63150947)(652.67477295,71.59151611)
\curveto(652.60476967,71.57150953)(652.53976974,71.54150956)(652.47977295,71.50151611)
\curveto(652.41976986,71.46150964)(652.36476991,71.42150968)(652.31477295,71.38151611)
\curveto(652.0747702,71.21150989)(651.8797704,71.0065101)(651.72977295,70.76651611)
\curveto(651.5797707,70.52651058)(651.44977083,70.24651086)(651.33977295,69.92651611)
\curveto(651.30977097,69.82651128)(651.28977099,69.72151138)(651.27977295,69.61151611)
\curveto(651.26977101,69.51151159)(651.25477102,69.4065117)(651.23477295,69.29651611)
\curveto(651.22477105,69.25651185)(651.21977106,69.19151191)(651.21977295,69.10151611)
\curveto(651.20977107,69.07151203)(651.20477107,69.03651207)(651.20477295,68.99651611)
\curveto(651.21477106,68.95651215)(651.21977106,68.91151219)(651.21977295,68.86151611)
\lineto(651.21977295,68.56151611)
\curveto(651.21977106,68.46151264)(651.22977105,68.37151273)(651.24977295,68.29151611)
\lineto(651.27977295,68.11151611)
\curveto(651.29977098,68.01151309)(651.31477096,67.91151319)(651.32477295,67.81151611)
\curveto(651.34477093,67.72151338)(651.3747709,67.63651347)(651.41477295,67.55651611)
\curveto(651.51477076,67.31651379)(651.62977065,67.09151401)(651.75977295,66.88151611)
\curveto(651.89977038,66.67151443)(652.06977021,66.49651461)(652.26977295,66.35651611)
\curveto(652.31976996,66.32651478)(652.36476991,66.3015148)(652.40477295,66.28151611)
\curveto(652.44476983,66.26151484)(652.48976979,66.23651487)(652.53977295,66.20651611)
\curveto(652.61976966,66.15651495)(652.70476957,66.11151499)(652.79477295,66.07151611)
\curveto(652.89476938,66.04151506)(652.99976928,66.01151509)(653.10977295,65.98151611)
\curveto(653.15976912,65.96151514)(653.20476907,65.95151515)(653.24477295,65.95151611)
\curveto(653.29476898,65.96151514)(653.34476893,65.96151514)(653.39477295,65.95151611)
\curveto(653.42476885,65.94151516)(653.48476879,65.93151517)(653.57477295,65.92151611)
\curveto(653.6747686,65.91151519)(653.74976853,65.91651519)(653.79977295,65.93651611)
\curveto(653.83976844,65.94651516)(653.8797684,65.94651516)(653.91977295,65.93651611)
\curveto(653.95976832,65.93651517)(653.99976828,65.94651516)(654.03977295,65.96651611)
\curveto(654.11976816,65.98651512)(654.19976808,66.0015151)(654.27977295,66.01151611)
\curveto(654.35976792,66.03151507)(654.43476784,66.05651505)(654.50477295,66.08651611)
\curveto(654.84476743,66.22651488)(655.11976716,66.42151468)(655.32977295,66.67151611)
\curveto(655.53976674,66.92151418)(655.71476656,67.21651389)(655.85477295,67.55651611)
\curveto(655.90476637,67.67651343)(655.93476634,67.8015133)(655.94477295,67.93151611)
\curveto(655.96476631,68.07151303)(655.99476628,68.21151289)(656.03477295,68.35151611)
}
}
{
\newrgbcolor{curcolor}{0 0 0}
\pscustom[linestyle=none,fillstyle=solid,fillcolor=curcolor]
{
\newpath
\moveto(662.5480542,72.91151611)
\curveto(662.92804921,72.92150818)(663.24804889,72.88150822)(663.5080542,72.79151611)
\curveto(663.77804836,72.7015084)(664.02304812,72.57150853)(664.2430542,72.40151611)
\curveto(664.32304782,72.35150875)(664.38804775,72.28150882)(664.4380542,72.19151611)
\curveto(664.49804764,72.11150899)(664.56304758,72.03650907)(664.6330542,71.96651611)
\curveto(664.65304749,71.94650916)(664.68304746,71.92150918)(664.7230542,71.89151611)
\curveto(664.76304738,71.86150924)(664.81304733,71.85150925)(664.8730542,71.86151611)
\curveto(664.97304717,71.89150921)(665.05804708,71.95150915)(665.1280542,72.04151611)
\curveto(665.20804693,72.14150896)(665.28804685,72.21650889)(665.3680542,72.26651611)
\curveto(665.50804663,72.37650873)(665.65304649,72.47150863)(665.8030542,72.55151611)
\curveto(665.95304619,72.64150846)(666.11804602,72.71650839)(666.2980542,72.77651611)
\curveto(666.37804576,72.8065083)(666.46304568,72.82650828)(666.5530542,72.83651611)
\curveto(666.65304549,72.85650825)(666.74804539,72.87650823)(666.8380542,72.89651611)
\curveto(666.88804525,72.9065082)(666.93304521,72.91150819)(666.9730542,72.91151611)
\lineto(667.1230542,72.91151611)
\curveto(667.17304497,72.93150817)(667.2430449,72.93650817)(667.3330542,72.92651611)
\curveto(667.42304472,72.92650818)(667.48804465,72.92150818)(667.5280542,72.91151611)
\curveto(667.57804456,72.9015082)(667.65304449,72.89650821)(667.7530542,72.89651611)
\curveto(667.8430443,72.87650823)(667.92804421,72.85650825)(668.0080542,72.83651611)
\curveto(668.09804404,72.82650828)(668.18304396,72.8065083)(668.2630542,72.77651611)
\curveto(668.31304383,72.75650835)(668.35804378,72.74150836)(668.3980542,72.73151611)
\curveto(668.44804369,72.73150837)(668.49804364,72.72150838)(668.5480542,72.70151611)
\curveto(669.04804309,72.48150862)(669.39304275,72.14150896)(669.5830542,71.68151611)
\curveto(669.62304252,71.6015095)(669.65304249,71.51150959)(669.6730542,71.41151611)
\curveto(669.69304245,71.32150978)(669.71304243,71.22150988)(669.7330542,71.11151611)
\curveto(669.75304239,71.08151002)(669.75804238,71.04651006)(669.7480542,71.00651611)
\curveto(669.74804239,70.97651013)(669.75304239,70.94651016)(669.7630542,70.91651611)
\lineto(669.7630542,70.78151611)
\curveto(669.77304237,70.74151036)(669.77304237,70.69651041)(669.7630542,70.64651611)
\curveto(669.76304238,70.59651051)(669.76304238,70.54651056)(669.7630542,70.49651611)
\lineto(669.7630542,69.91151611)
\lineto(669.7630542,68.95151611)
\lineto(669.7630542,66.10151611)
\curveto(669.76304238,65.94151516)(669.76304238,65.75151535)(669.7630542,65.53151611)
\curveto(669.77304237,65.31151579)(669.73304241,65.16651594)(669.6430542,65.09651611)
\curveto(669.60304254,65.06651604)(669.5380426,65.04151606)(669.4480542,65.02151611)
\curveto(669.35804278,65.01151609)(669.26304288,65.0065161)(669.1630542,65.00651611)
\curveto(669.06304308,65.0065161)(668.96304318,65.01151609)(668.8630542,65.02151611)
\curveto(668.77304337,65.03151607)(668.70804343,65.05151605)(668.6680542,65.08151611)
\curveto(668.60804353,65.11151599)(668.56804357,65.17151593)(668.5480542,65.26151611)
\curveto(668.52804361,65.32151578)(668.52304362,65.38151572)(668.5330542,65.44151611)
\curveto(668.5430436,65.51151559)(668.5380436,65.57651553)(668.5180542,65.63651611)
\curveto(668.50804363,65.68651542)(668.50304364,65.74151536)(668.5030542,65.80151611)
\curveto(668.51304363,65.87151523)(668.51804362,65.93651517)(668.5180542,65.99651611)
\lineto(668.5180542,66.67151611)
\lineto(668.5180542,69.53651611)
\curveto(668.51804362,69.86651124)(668.50804363,70.17651093)(668.4880542,70.46651611)
\curveto(668.47804366,70.76651034)(668.40804373,71.01651009)(668.2780542,71.21651611)
\curveto(668.12804401,71.45650965)(667.89804424,71.63150947)(667.5880542,71.74151611)
\curveto(667.52804461,71.76150934)(667.46304468,71.77150933)(667.3930542,71.77151611)
\curveto(667.33304481,71.78150932)(667.26804487,71.79650931)(667.1980542,71.81651611)
\curveto(667.15804498,71.82650928)(667.09304505,71.82650928)(667.0030542,71.81651611)
\curveto(666.91304523,71.81650929)(666.85304529,71.81150929)(666.8230542,71.80151611)
\curveto(666.77304537,71.79150931)(666.72304542,71.78650932)(666.6730542,71.78651611)
\curveto(666.62304552,71.79650931)(666.57304557,71.79150931)(666.5230542,71.77151611)
\curveto(666.38304576,71.74150936)(666.24804589,71.7015094)(666.1180542,71.65151611)
\curveto(665.59804654,71.43150967)(665.24804689,71.04651006)(665.0680542,70.49651611)
\curveto(665.01804712,70.32651078)(664.98804715,70.13151097)(664.9780542,69.91151611)
\lineto(664.9780542,69.23651611)
\lineto(664.9780542,67.27151611)
\lineto(664.9780542,65.81651611)
\lineto(664.9780542,65.44151611)
\curveto(664.97804716,65.32151578)(664.95304719,65.22651588)(664.9030542,65.15651611)
\curveto(664.85304729,65.07651603)(664.76804737,65.03151607)(664.6480542,65.02151611)
\curveto(664.52804761,65.01151609)(664.40304774,65.0065161)(664.2730542,65.00651611)
\curveto(664.10304804,65.0065161)(663.97804816,65.02651608)(663.8980542,65.06651611)
\curveto(663.80804833,65.11651599)(663.75304839,65.19651591)(663.7330542,65.30651611)
\curveto(663.72304842,65.42651568)(663.71804842,65.55651555)(663.7180542,65.69651611)
\lineto(663.7180542,67.12151611)
\lineto(663.7180542,69.59651611)
\curveto(663.71804842,69.91651119)(663.70804843,70.21151089)(663.6880542,70.48151611)
\curveto(663.66804847,70.76151034)(663.59804854,71.0015101)(663.4780542,71.20151611)
\curveto(663.36804877,71.38150972)(663.2430489,71.51150959)(663.1030542,71.59151611)
\curveto(662.96304918,71.68150942)(662.77304937,71.75150935)(662.5330542,71.80151611)
\curveto(662.49304965,71.81150929)(662.44804969,71.81650929)(662.3980542,71.81651611)
\lineto(662.2630542,71.81651611)
\curveto(662.0430501,71.81650929)(661.84805029,71.79150931)(661.6780542,71.74151611)
\curveto(661.51805062,71.69150941)(661.37305077,71.62650948)(661.2430542,71.54651611)
\curveto(660.73305141,71.23650987)(660.39305175,70.77151033)(660.2230542,70.15151611)
\curveto(660.18305196,70.02151108)(660.16305198,69.87151123)(660.1630542,69.70151611)
\curveto(660.17305197,69.54151156)(660.17805196,69.38151172)(660.1780542,69.22151611)
\lineto(660.1780542,67.52651611)
\lineto(660.1780542,65.87651611)
\lineto(660.1780542,65.47151611)
\curveto(660.17805196,65.33151577)(660.14805199,65.22151588)(660.0880542,65.14151611)
\curveto(660.0380521,65.07151603)(659.96305218,65.03151607)(659.8630542,65.02151611)
\curveto(659.76305238,65.01151609)(659.65805248,65.0065161)(659.5480542,65.00651611)
\lineto(659.3230542,65.00651611)
\curveto(659.26305288,65.02651608)(659.20305294,65.04151606)(659.1430542,65.05151611)
\curveto(659.09305305,65.06151604)(659.04805309,65.09151601)(659.0080542,65.14151611)
\curveto(658.95805318,65.2015159)(658.93305321,65.27651583)(658.9330542,65.36651611)
\lineto(658.9330542,65.68151611)
\lineto(658.9330542,66.65651611)
\lineto(658.9330542,70.94651611)
\lineto(658.9330542,72.05651611)
\lineto(658.9330542,72.34151611)
\curveto(658.93305321,72.44150866)(658.95305319,72.52150858)(658.9930542,72.58151611)
\curveto(659.02305312,72.64150846)(659.06805307,72.68150842)(659.1280542,72.70151611)
\curveto(659.20805293,72.73150837)(659.33305281,72.74650836)(659.5030542,72.74651611)
\curveto(659.68305246,72.74650836)(659.81305233,72.73150837)(659.8930542,72.70151611)
\curveto(659.97305217,72.66150844)(660.02805211,72.61150849)(660.0580542,72.55151611)
\curveto(660.07805206,72.5015086)(660.08805205,72.44150866)(660.0880542,72.37151611)
\curveto(660.09805204,72.3015088)(660.10805203,72.23650887)(660.1180542,72.17651611)
\curveto(660.12805201,72.11650899)(660.14805199,72.06650904)(660.1780542,72.02651611)
\curveto(660.20805193,71.98650912)(660.25805188,71.96650914)(660.3280542,71.96651611)
\curveto(660.34805179,71.98650912)(660.36805177,71.99650911)(660.3880542,71.99651611)
\curveto(660.41805172,71.99650911)(660.4430517,72.0065091)(660.4630542,72.02651611)
\curveto(660.52305162,72.07650903)(660.57805156,72.12650898)(660.6280542,72.17651611)
\lineto(660.8080542,72.32651611)
\curveto(661.02805111,72.48650862)(661.27805086,72.62650848)(661.5580542,72.74651611)
\curveto(661.65805048,72.78650832)(661.75805038,72.81150829)(661.8580542,72.82151611)
\curveto(661.95805018,72.84150826)(662.06305008,72.86650824)(662.1730542,72.89651611)
\lineto(662.3530542,72.89651611)
\curveto(662.42304972,72.9065082)(662.48804965,72.91150819)(662.5480542,72.91151611)
}
}
{
\newrgbcolor{curcolor}{0 0 0}
\pscustom[linestyle=none,fillstyle=solid,fillcolor=curcolor]
{
\newpath
\moveto(672.12578857,72.73151611)
\lineto(672.56078857,72.73151611)
\curveto(672.71078661,72.73150837)(672.8157865,72.69150841)(672.87578857,72.61151611)
\curveto(672.92578639,72.53150857)(672.95078637,72.43150867)(672.95078857,72.31151611)
\curveto(672.96078636,72.19150891)(672.96578635,72.07150903)(672.96578857,71.95151611)
\lineto(672.96578857,70.52651611)
\lineto(672.96578857,68.26151611)
\lineto(672.96578857,67.57151611)
\curveto(672.96578635,67.34151376)(672.99078633,67.14151396)(673.04078857,66.97151611)
\curveto(673.20078612,66.52151458)(673.50078582,66.2065149)(673.94078857,66.02651611)
\curveto(674.16078516,65.93651517)(674.42578489,65.9015152)(674.73578857,65.92151611)
\curveto(675.04578427,65.95151515)(675.29578402,66.0065151)(675.48578857,66.08651611)
\curveto(675.8157835,66.22651488)(676.07578324,66.4015147)(676.26578857,66.61151611)
\curveto(676.46578285,66.83151427)(676.6207827,67.11651399)(676.73078857,67.46651611)
\curveto(676.76078256,67.54651356)(676.78078254,67.62651348)(676.79078857,67.70651611)
\curveto(676.80078252,67.78651332)(676.8157825,67.87151323)(676.83578857,67.96151611)
\curveto(676.84578247,68.01151309)(676.84578247,68.05651305)(676.83578857,68.09651611)
\curveto(676.83578248,68.13651297)(676.84578247,68.18151292)(676.86578857,68.23151611)
\lineto(676.86578857,68.54651611)
\curveto(676.88578243,68.62651248)(676.89078243,68.71651239)(676.88078857,68.81651611)
\curveto(676.87078245,68.92651218)(676.86578245,69.02651208)(676.86578857,69.11651611)
\lineto(676.86578857,70.28651611)
\lineto(676.86578857,71.87651611)
\curveto(676.86578245,71.99650911)(676.86078246,72.12150898)(676.85078857,72.25151611)
\curveto(676.85078247,72.39150871)(676.87578244,72.5015086)(676.92578857,72.58151611)
\curveto(676.96578235,72.63150847)(677.01078231,72.66150844)(677.06078857,72.67151611)
\curveto(677.1207822,72.69150841)(677.19078213,72.71150839)(677.27078857,72.73151611)
\lineto(677.49578857,72.73151611)
\curveto(677.6157817,72.73150837)(677.7207816,72.72650838)(677.81078857,72.71651611)
\curveto(677.91078141,72.7065084)(677.98578133,72.66150844)(678.03578857,72.58151611)
\curveto(678.08578123,72.53150857)(678.11078121,72.45650865)(678.11078857,72.35651611)
\lineto(678.11078857,72.07151611)
\lineto(678.11078857,71.05151611)
\lineto(678.11078857,67.01651611)
\lineto(678.11078857,65.66651611)
\curveto(678.11078121,65.54651556)(678.10578121,65.43151567)(678.09578857,65.32151611)
\curveto(678.09578122,65.22151588)(678.06078126,65.14651596)(677.99078857,65.09651611)
\curveto(677.95078137,65.06651604)(677.89078143,65.04151606)(677.81078857,65.02151611)
\curveto(677.73078159,65.01151609)(677.64078168,65.0015161)(677.54078857,64.99151611)
\curveto(677.45078187,64.99151611)(677.36078196,64.99651611)(677.27078857,65.00651611)
\curveto(677.19078213,65.01651609)(677.13078219,65.03651607)(677.09078857,65.06651611)
\curveto(677.04078228,65.106516)(676.99578232,65.17151593)(676.95578857,65.26151611)
\curveto(676.94578237,65.3015158)(676.93578238,65.35651575)(676.92578857,65.42651611)
\curveto(676.92578239,65.49651561)(676.9207824,65.56151554)(676.91078857,65.62151611)
\curveto(676.90078242,65.69151541)(676.88078244,65.74651536)(676.85078857,65.78651611)
\curveto(676.8207825,65.82651528)(676.77578254,65.84151526)(676.71578857,65.83151611)
\curveto(676.63578268,65.81151529)(676.55578276,65.75151535)(676.47578857,65.65151611)
\curveto(676.39578292,65.56151554)(676.320783,65.49151561)(676.25078857,65.44151611)
\curveto(676.03078329,65.28151582)(675.78078354,65.14151596)(675.50078857,65.02151611)
\curveto(675.39078393,64.97151613)(675.27578404,64.94151616)(675.15578857,64.93151611)
\curveto(675.04578427,64.91151619)(674.93078439,64.88651622)(674.81078857,64.85651611)
\curveto(674.76078456,64.84651626)(674.70578461,64.84651626)(674.64578857,64.85651611)
\curveto(674.59578472,64.86651624)(674.54578477,64.86151624)(674.49578857,64.84151611)
\curveto(674.39578492,64.82151628)(674.30578501,64.82151628)(674.22578857,64.84151611)
\lineto(674.07578857,64.84151611)
\curveto(674.02578529,64.86151624)(673.96578535,64.87151623)(673.89578857,64.87151611)
\curveto(673.83578548,64.87151623)(673.78078554,64.87651623)(673.73078857,64.88651611)
\curveto(673.69078563,64.9065162)(673.65078567,64.91651619)(673.61078857,64.91651611)
\curveto(673.58078574,64.9065162)(673.54078578,64.91151619)(673.49078857,64.93151611)
\lineto(673.25078857,64.99151611)
\curveto(673.18078614,65.01151609)(673.10578621,65.04151606)(673.02578857,65.08151611)
\curveto(672.76578655,65.19151591)(672.54578677,65.33651577)(672.36578857,65.51651611)
\curveto(672.19578712,65.7065154)(672.05578726,65.93151517)(671.94578857,66.19151611)
\curveto(671.90578741,66.28151482)(671.87578744,66.37151473)(671.85578857,66.46151611)
\lineto(671.79578857,66.76151611)
\curveto(671.77578754,66.82151428)(671.76578755,66.87651423)(671.76578857,66.92651611)
\curveto(671.77578754,66.98651412)(671.77078755,67.05151405)(671.75078857,67.12151611)
\curveto(671.74078758,67.14151396)(671.73578758,67.16651394)(671.73578857,67.19651611)
\curveto(671.73578758,67.23651387)(671.73078759,67.27151383)(671.72078857,67.30151611)
\lineto(671.72078857,67.45151611)
\curveto(671.71078761,67.49151361)(671.70578761,67.53651357)(671.70578857,67.58651611)
\curveto(671.7157876,67.64651346)(671.7207876,67.7015134)(671.72078857,67.75151611)
\lineto(671.72078857,68.35151611)
\lineto(671.72078857,71.11151611)
\lineto(671.72078857,72.07151611)
\lineto(671.72078857,72.34151611)
\curveto(671.7207876,72.43150867)(671.74078758,72.5065086)(671.78078857,72.56651611)
\curveto(671.8207875,72.63650847)(671.89578742,72.68650842)(672.00578857,72.71651611)
\curveto(672.02578729,72.72650838)(672.04578727,72.72650838)(672.06578857,72.71651611)
\curveto(672.08578723,72.71650839)(672.10578721,72.72150838)(672.12578857,72.73151611)
}
}
{
\newrgbcolor{curcolor}{0 0 0}
\pscustom[linestyle=none,fillstyle=solid,fillcolor=curcolor]
{
\newpath
\moveto(683.70039795,72.88151611)
\curveto(684.33039271,72.9015082)(684.83539221,72.81650829)(685.21539795,72.62651611)
\curveto(685.59539145,72.43650867)(685.90039114,72.15150895)(686.13039795,71.77151611)
\curveto(686.19039085,71.67150943)(686.23539081,71.56150954)(686.26539795,71.44151611)
\curveto(686.30539074,71.33150977)(686.3403907,71.21650989)(686.37039795,71.09651611)
\curveto(686.42039062,70.9065102)(686.45039059,70.7015104)(686.46039795,70.48151611)
\curveto(686.47039057,70.26151084)(686.47539057,70.03651107)(686.47539795,69.80651611)
\lineto(686.47539795,68.20151611)
\lineto(686.47539795,65.86151611)
\curveto(686.47539057,65.69151541)(686.47039057,65.52151558)(686.46039795,65.35151611)
\curveto(686.46039058,65.18151592)(686.39539065,65.07151603)(686.26539795,65.02151611)
\curveto(686.21539083,65.0015161)(686.16039088,64.99151611)(686.10039795,64.99151611)
\curveto(686.05039099,64.98151612)(685.99539105,64.97651613)(685.93539795,64.97651611)
\curveto(685.80539124,64.97651613)(685.68039136,64.98151612)(685.56039795,64.99151611)
\curveto(685.4403916,64.99151611)(685.35539169,65.03151607)(685.30539795,65.11151611)
\curveto(685.25539179,65.18151592)(685.23039181,65.27151583)(685.23039795,65.38151611)
\lineto(685.23039795,65.71151611)
\lineto(685.23039795,67.00151611)
\lineto(685.23039795,69.44651611)
\curveto(685.23039181,69.71651139)(685.22539182,69.98151112)(685.21539795,70.24151611)
\curveto(685.20539184,70.51151059)(685.16039188,70.74151036)(685.08039795,70.93151611)
\curveto(685.00039204,71.13150997)(684.88039216,71.29150981)(684.72039795,71.41151611)
\curveto(684.56039248,71.54150956)(684.37539267,71.64150946)(684.16539795,71.71151611)
\curveto(684.10539294,71.73150937)(684.040393,71.74150936)(683.97039795,71.74151611)
\curveto(683.91039313,71.75150935)(683.85039319,71.76650934)(683.79039795,71.78651611)
\curveto(683.7403933,71.79650931)(683.66039338,71.79650931)(683.55039795,71.78651611)
\curveto(683.45039359,71.78650932)(683.38039366,71.78150932)(683.34039795,71.77151611)
\curveto(683.30039374,71.75150935)(683.26539378,71.74150936)(683.23539795,71.74151611)
\curveto(683.20539384,71.75150935)(683.17039387,71.75150935)(683.13039795,71.74151611)
\curveto(683.00039404,71.71150939)(682.87539417,71.67650943)(682.75539795,71.63651611)
\curveto(682.6453944,71.6065095)(682.5403945,71.56150954)(682.44039795,71.50151611)
\curveto(682.40039464,71.48150962)(682.36539468,71.46150964)(682.33539795,71.44151611)
\curveto(682.30539474,71.42150968)(682.27039477,71.4015097)(682.23039795,71.38151611)
\curveto(681.88039516,71.13150997)(681.62539542,70.75651035)(681.46539795,70.25651611)
\curveto(681.43539561,70.17651093)(681.41539563,70.09151101)(681.40539795,70.00151611)
\curveto(681.39539565,69.92151118)(681.38039566,69.84151126)(681.36039795,69.76151611)
\curveto(681.3403957,69.71151139)(681.33539571,69.66151144)(681.34539795,69.61151611)
\curveto(681.35539569,69.57151153)(681.35039569,69.53151157)(681.33039795,69.49151611)
\lineto(681.33039795,69.17651611)
\curveto(681.32039572,69.14651196)(681.31539573,69.11151199)(681.31539795,69.07151611)
\curveto(681.32539572,69.03151207)(681.33039571,68.98651212)(681.33039795,68.93651611)
\lineto(681.33039795,68.48651611)
\lineto(681.33039795,67.04651611)
\lineto(681.33039795,65.72651611)
\lineto(681.33039795,65.38151611)
\curveto(681.33039571,65.27151583)(681.30539574,65.18151592)(681.25539795,65.11151611)
\curveto(681.20539584,65.03151607)(681.11539593,64.99151611)(680.98539795,64.99151611)
\curveto(680.86539618,64.98151612)(680.7403963,64.97651613)(680.61039795,64.97651611)
\curveto(680.53039651,64.97651613)(680.45539659,64.98151612)(680.38539795,64.99151611)
\curveto(680.31539673,65.0015161)(680.25539679,65.02651608)(680.20539795,65.06651611)
\curveto(680.12539692,65.11651599)(680.08539696,65.21151589)(680.08539795,65.35151611)
\lineto(680.08539795,65.75651611)
\lineto(680.08539795,67.52651611)
\lineto(680.08539795,71.15651611)
\lineto(680.08539795,72.07151611)
\lineto(680.08539795,72.34151611)
\curveto(680.08539696,72.43150867)(680.10539694,72.5015086)(680.14539795,72.55151611)
\curveto(680.17539687,72.61150849)(680.22539682,72.65150845)(680.29539795,72.67151611)
\curveto(680.33539671,72.68150842)(680.39039665,72.69150841)(680.46039795,72.70151611)
\curveto(680.5403965,72.71150839)(680.62039642,72.71650839)(680.70039795,72.71651611)
\curveto(680.78039626,72.71650839)(680.85539619,72.71150839)(680.92539795,72.70151611)
\curveto(681.00539604,72.69150841)(681.06039598,72.67650843)(681.09039795,72.65651611)
\curveto(681.20039584,72.58650852)(681.25039579,72.49650861)(681.24039795,72.38651611)
\curveto(681.23039581,72.28650882)(681.2453958,72.17150893)(681.28539795,72.04151611)
\curveto(681.30539574,71.98150912)(681.3453957,71.93150917)(681.40539795,71.89151611)
\curveto(681.52539552,71.88150922)(681.62039542,71.92650918)(681.69039795,72.02651611)
\curveto(681.77039527,72.12650898)(681.85039519,72.2065089)(681.93039795,72.26651611)
\curveto(682.07039497,72.36650874)(682.21039483,72.45650865)(682.35039795,72.53651611)
\curveto(682.50039454,72.62650848)(682.67039437,72.7015084)(682.86039795,72.76151611)
\curveto(682.9403941,72.79150831)(683.02539402,72.81150829)(683.11539795,72.82151611)
\curveto(683.21539383,72.83150827)(683.31039373,72.84650826)(683.40039795,72.86651611)
\curveto(683.45039359,72.87650823)(683.50039354,72.88150822)(683.55039795,72.88151611)
\lineto(683.70039795,72.88151611)
}
}
{
\newrgbcolor{curcolor}{0 0 0}
\pscustom[linestyle=none,fillstyle=solid,fillcolor=curcolor]
{
\newpath
\moveto(688.64500732,74.23151611)
\curveto(688.5650062,74.29150681)(688.52000625,74.39650671)(688.51000732,74.54651611)
\lineto(688.51000732,75.01151611)
\lineto(688.51000732,75.26651611)
\curveto(688.51000626,75.35650575)(688.52500624,75.43150567)(688.55500732,75.49151611)
\curveto(688.59500617,75.57150553)(688.67500609,75.63150547)(688.79500732,75.67151611)
\curveto(688.81500595,75.68150542)(688.83500593,75.68150542)(688.85500732,75.67151611)
\curveto(688.88500588,75.67150543)(688.91000586,75.67650543)(688.93000732,75.68651611)
\curveto(689.10000567,75.68650542)(689.26000551,75.68150542)(689.41000732,75.67151611)
\curveto(689.56000521,75.66150544)(689.66000511,75.6015055)(689.71000732,75.49151611)
\curveto(689.74000503,75.43150567)(689.75500501,75.35650575)(689.75500732,75.26651611)
\lineto(689.75500732,75.01151611)
\curveto(689.75500501,74.83150627)(689.75000502,74.66150644)(689.74000732,74.50151611)
\curveto(689.74000503,74.34150676)(689.67500509,74.23650687)(689.54500732,74.18651611)
\curveto(689.49500527,74.16650694)(689.44000533,74.15650695)(689.38000732,74.15651611)
\lineto(689.21500732,74.15651611)
\lineto(688.90000732,74.15651611)
\curveto(688.80000597,74.15650695)(688.71500605,74.18150692)(688.64500732,74.23151611)
\moveto(689.75500732,65.72651611)
\lineto(689.75500732,65.41151611)
\curveto(689.765005,65.31151579)(689.74500502,65.23151587)(689.69500732,65.17151611)
\curveto(689.6650051,65.11151599)(689.62000515,65.07151603)(689.56000732,65.05151611)
\curveto(689.50000527,65.04151606)(689.43000534,65.02651608)(689.35000732,65.00651611)
\lineto(689.12500732,65.00651611)
\curveto(688.99500577,65.0065161)(688.88000589,65.01151609)(688.78000732,65.02151611)
\curveto(688.69000608,65.04151606)(688.62000615,65.09151601)(688.57000732,65.17151611)
\curveto(688.53000624,65.23151587)(688.51000626,65.3065158)(688.51000732,65.39651611)
\lineto(688.51000732,65.68151611)
\lineto(688.51000732,72.02651611)
\lineto(688.51000732,72.34151611)
\curveto(688.51000626,72.45150865)(688.53500623,72.53650857)(688.58500732,72.59651611)
\curveto(688.61500615,72.64650846)(688.65500611,72.67650843)(688.70500732,72.68651611)
\curveto(688.75500601,72.69650841)(688.81000596,72.71150839)(688.87000732,72.73151611)
\curveto(688.89000588,72.73150837)(688.91000586,72.72650838)(688.93000732,72.71651611)
\curveto(688.96000581,72.71650839)(688.98500578,72.72150838)(689.00500732,72.73151611)
\curveto(689.13500563,72.73150837)(689.2650055,72.72650838)(689.39500732,72.71651611)
\curveto(689.53500523,72.71650839)(689.63000514,72.67650843)(689.68000732,72.59651611)
\curveto(689.73000504,72.53650857)(689.75500501,72.45650865)(689.75500732,72.35651611)
\lineto(689.75500732,72.07151611)
\lineto(689.75500732,65.72651611)
}
}
{
\newrgbcolor{curcolor}{0 0 0}
\pscustom[linestyle=none,fillstyle=solid,fillcolor=curcolor]
{
\newpath
\moveto(698.65985107,65.81651611)
\lineto(698.65985107,65.42651611)
\curveto(698.6598432,65.3065158)(698.63484322,65.2065159)(698.58485107,65.12651611)
\curveto(698.53484332,65.05651605)(698.44984341,65.01651609)(698.32985107,65.00651611)
\lineto(697.98485107,65.00651611)
\curveto(697.92484393,65.0065161)(697.86484399,65.0015161)(697.80485107,64.99151611)
\curveto(697.7548441,64.99151611)(697.70984415,65.0015161)(697.66985107,65.02151611)
\curveto(697.57984428,65.04151606)(697.51984434,65.08151602)(697.48985107,65.14151611)
\curveto(697.44984441,65.19151591)(697.42484443,65.25151585)(697.41485107,65.32151611)
\curveto(697.41484444,65.39151571)(697.39984446,65.46151564)(697.36985107,65.53151611)
\curveto(697.3598445,65.55151555)(697.34484451,65.56651554)(697.32485107,65.57651611)
\curveto(697.31484454,65.59651551)(697.29984456,65.61651549)(697.27985107,65.63651611)
\curveto(697.17984468,65.64651546)(697.09984476,65.62651548)(697.03985107,65.57651611)
\curveto(696.98984487,65.52651558)(696.93484492,65.47651563)(696.87485107,65.42651611)
\curveto(696.67484518,65.27651583)(696.47484538,65.16151594)(696.27485107,65.08151611)
\curveto(696.09484576,65.0015161)(695.88484597,64.94151616)(695.64485107,64.90151611)
\curveto(695.41484644,64.86151624)(695.17484668,64.84151626)(694.92485107,64.84151611)
\curveto(694.68484717,64.83151627)(694.44484741,64.84651626)(694.20485107,64.88651611)
\curveto(693.96484789,64.91651619)(693.7548481,64.97151613)(693.57485107,65.05151611)
\curveto(693.0548488,65.27151583)(692.63484922,65.56651554)(692.31485107,65.93651611)
\curveto(691.99484986,66.31651479)(691.74485011,66.78651432)(691.56485107,67.34651611)
\curveto(691.52485033,67.43651367)(691.49485036,67.52651358)(691.47485107,67.61651611)
\curveto(691.46485039,67.71651339)(691.44485041,67.81651329)(691.41485107,67.91651611)
\curveto(691.40485045,67.96651314)(691.39985046,68.01651309)(691.39985107,68.06651611)
\curveto(691.39985046,68.11651299)(691.39485046,68.16651294)(691.38485107,68.21651611)
\curveto(691.36485049,68.26651284)(691.3548505,68.31651279)(691.35485107,68.36651611)
\curveto(691.36485049,68.42651268)(691.36485049,68.48151262)(691.35485107,68.53151611)
\lineto(691.35485107,68.68151611)
\curveto(691.33485052,68.73151237)(691.32485053,68.79651231)(691.32485107,68.87651611)
\curveto(691.32485053,68.95651215)(691.33485052,69.02151208)(691.35485107,69.07151611)
\lineto(691.35485107,69.23651611)
\curveto(691.37485048,69.3065118)(691.37985048,69.37651173)(691.36985107,69.44651611)
\curveto(691.36985049,69.52651158)(691.37985048,69.6015115)(691.39985107,69.67151611)
\curveto(691.40985045,69.72151138)(691.41485044,69.76651134)(691.41485107,69.80651611)
\curveto(691.41485044,69.84651126)(691.41985044,69.89151121)(691.42985107,69.94151611)
\curveto(691.4598504,70.04151106)(691.48485037,70.13651097)(691.50485107,70.22651611)
\curveto(691.52485033,70.32651078)(691.54985031,70.42151068)(691.57985107,70.51151611)
\curveto(691.70985015,70.89151021)(691.87484998,71.23150987)(692.07485107,71.53151611)
\curveto(692.28484957,71.84150926)(692.53484932,72.09650901)(692.82485107,72.29651611)
\curveto(692.99484886,72.41650869)(693.16984869,72.51650859)(693.34985107,72.59651611)
\curveto(693.53984832,72.67650843)(693.74484811,72.74650836)(693.96485107,72.80651611)
\curveto(694.03484782,72.81650829)(694.09984776,72.82650828)(694.15985107,72.83651611)
\curveto(694.22984763,72.84650826)(694.29984756,72.86150824)(694.36985107,72.88151611)
\lineto(694.51985107,72.88151611)
\curveto(694.59984726,72.9015082)(694.71484714,72.91150819)(694.86485107,72.91151611)
\curveto(695.02484683,72.91150819)(695.14484671,72.9015082)(695.22485107,72.88151611)
\curveto(695.26484659,72.87150823)(695.31984654,72.86650824)(695.38985107,72.86651611)
\curveto(695.49984636,72.83650827)(695.60984625,72.81150829)(695.71985107,72.79151611)
\curveto(695.82984603,72.78150832)(695.93484592,72.75150835)(696.03485107,72.70151611)
\curveto(696.18484567,72.64150846)(696.32484553,72.57650853)(696.45485107,72.50651611)
\curveto(696.59484526,72.43650867)(696.72484513,72.35650875)(696.84485107,72.26651611)
\curveto(696.90484495,72.21650889)(696.96484489,72.16150894)(697.02485107,72.10151611)
\curveto(697.09484476,72.05150905)(697.18484467,72.03650907)(697.29485107,72.05651611)
\curveto(697.31484454,72.08650902)(697.32984453,72.11150899)(697.33985107,72.13151611)
\curveto(697.3598445,72.15150895)(697.37484448,72.18150892)(697.38485107,72.22151611)
\curveto(697.41484444,72.31150879)(697.42484443,72.42650868)(697.41485107,72.56651611)
\lineto(697.41485107,72.94151611)
\lineto(697.41485107,74.66651611)
\lineto(697.41485107,75.13151611)
\curveto(697.41484444,75.31150579)(697.43984442,75.44150566)(697.48985107,75.52151611)
\curveto(697.52984433,75.59150551)(697.58984427,75.63650547)(697.66985107,75.65651611)
\curveto(697.68984417,75.65650545)(697.71484414,75.65650545)(697.74485107,75.65651611)
\curveto(697.77484408,75.66650544)(697.79984406,75.67150543)(697.81985107,75.67151611)
\curveto(697.9598439,75.68150542)(698.10484375,75.68150542)(698.25485107,75.67151611)
\curveto(698.41484344,75.67150543)(698.52484333,75.63150547)(698.58485107,75.55151611)
\curveto(698.63484322,75.47150563)(698.6598432,75.37150573)(698.65985107,75.25151611)
\lineto(698.65985107,74.87651611)
\lineto(698.65985107,65.81651611)
\moveto(697.44485107,68.65151611)
\curveto(697.46484439,68.7015124)(697.47484438,68.76651234)(697.47485107,68.84651611)
\curveto(697.47484438,68.93651217)(697.46484439,69.0065121)(697.44485107,69.05651611)
\lineto(697.44485107,69.28151611)
\curveto(697.42484443,69.37151173)(697.40984445,69.46151164)(697.39985107,69.55151611)
\curveto(697.38984447,69.65151145)(697.36984449,69.74151136)(697.33985107,69.82151611)
\curveto(697.31984454,69.9015112)(697.29984456,69.97651113)(697.27985107,70.04651611)
\curveto(697.26984459,70.11651099)(697.24984461,70.18651092)(697.21985107,70.25651611)
\curveto(697.09984476,70.55651055)(696.94484491,70.82151028)(696.75485107,71.05151611)
\curveto(696.56484529,71.28150982)(696.32484553,71.46150964)(696.03485107,71.59151611)
\curveto(695.93484592,71.64150946)(695.82984603,71.67650943)(695.71985107,71.69651611)
\curveto(695.61984624,71.72650938)(695.50984635,71.75150935)(695.38985107,71.77151611)
\curveto(695.30984655,71.79150931)(695.21984664,71.8015093)(695.11985107,71.80151611)
\lineto(694.84985107,71.80151611)
\curveto(694.79984706,71.79150931)(694.7548471,71.78150932)(694.71485107,71.77151611)
\lineto(694.57985107,71.77151611)
\curveto(694.49984736,71.75150935)(694.41484744,71.73150937)(694.32485107,71.71151611)
\curveto(694.24484761,71.69150941)(694.16484769,71.66650944)(694.08485107,71.63651611)
\curveto(693.76484809,71.49650961)(693.50484835,71.29150981)(693.30485107,71.02151611)
\curveto(693.11484874,70.76151034)(692.9598489,70.45651065)(692.83985107,70.10651611)
\curveto(692.79984906,69.99651111)(692.76984909,69.88151122)(692.74985107,69.76151611)
\curveto(692.73984912,69.65151145)(692.72484913,69.54151156)(692.70485107,69.43151611)
\curveto(692.70484915,69.39151171)(692.69984916,69.35151175)(692.68985107,69.31151611)
\lineto(692.68985107,69.20651611)
\curveto(692.66984919,69.15651195)(692.6598492,69.101512)(692.65985107,69.04151611)
\curveto(692.66984919,68.98151212)(692.67484918,68.92651218)(692.67485107,68.87651611)
\lineto(692.67485107,68.54651611)
\curveto(692.67484918,68.44651266)(692.68484917,68.35151275)(692.70485107,68.26151611)
\curveto(692.71484914,68.23151287)(692.71984914,68.18151292)(692.71985107,68.11151611)
\curveto(692.73984912,68.04151306)(692.7548491,67.97151313)(692.76485107,67.90151611)
\lineto(692.82485107,67.69151611)
\curveto(692.93484892,67.34151376)(693.08484877,67.04151406)(693.27485107,66.79151611)
\curveto(693.46484839,66.54151456)(693.70484815,66.33651477)(693.99485107,66.17651611)
\curveto(694.08484777,66.12651498)(694.17484768,66.08651502)(694.26485107,66.05651611)
\curveto(694.3548475,66.02651508)(694.4548474,65.99651511)(694.56485107,65.96651611)
\curveto(694.61484724,65.94651516)(694.66484719,65.94151516)(694.71485107,65.95151611)
\curveto(694.77484708,65.96151514)(694.82984703,65.95651515)(694.87985107,65.93651611)
\curveto(694.91984694,65.92651518)(694.9598469,65.92151518)(694.99985107,65.92151611)
\lineto(695.13485107,65.92151611)
\lineto(695.26985107,65.92151611)
\curveto(695.29984656,65.93151517)(695.34984651,65.93651517)(695.41985107,65.93651611)
\curveto(695.49984636,65.95651515)(695.57984628,65.97151513)(695.65985107,65.98151611)
\curveto(695.73984612,66.0015151)(695.81484604,66.02651508)(695.88485107,66.05651611)
\curveto(696.21484564,66.19651491)(696.47984538,66.37151473)(696.67985107,66.58151611)
\curveto(696.88984497,66.8015143)(697.06484479,67.07651403)(697.20485107,67.40651611)
\curveto(697.2548446,67.51651359)(697.28984457,67.62651348)(697.30985107,67.73651611)
\curveto(697.32984453,67.84651326)(697.3548445,67.95651315)(697.38485107,68.06651611)
\curveto(697.40484445,68.106513)(697.41484444,68.14151296)(697.41485107,68.17151611)
\curveto(697.41484444,68.21151289)(697.41984444,68.25151285)(697.42985107,68.29151611)
\curveto(697.43984442,68.35151275)(697.43984442,68.41151269)(697.42985107,68.47151611)
\curveto(697.42984443,68.53151257)(697.43484442,68.59151251)(697.44485107,68.65151611)
}
}
{
\newrgbcolor{curcolor}{0 0 0}
\pscustom[linestyle=none,fillstyle=solid,fillcolor=curcolor]
{
\newpath
\moveto(707.49110107,65.56151611)
\curveto(707.52109324,65.4015157)(707.50609326,65.26651584)(707.44610107,65.15651611)
\curveto(707.38609338,65.05651605)(707.30609346,64.98151612)(707.20610107,64.93151611)
\curveto(707.15609361,64.91151619)(707.10109366,64.9015162)(707.04110107,64.90151611)
\curveto(706.99109377,64.9015162)(706.93609383,64.89151621)(706.87610107,64.87151611)
\curveto(706.65609411,64.82151628)(706.43609433,64.83651627)(706.21610107,64.91651611)
\curveto(706.00609476,64.98651612)(705.8610949,65.07651603)(705.78110107,65.18651611)
\curveto(705.73109503,65.25651585)(705.68609508,65.33651577)(705.64610107,65.42651611)
\curveto(705.60609516,65.52651558)(705.55609521,65.6065155)(705.49610107,65.66651611)
\curveto(705.47609529,65.68651542)(705.45109531,65.7065154)(705.42110107,65.72651611)
\curveto(705.40109536,65.74651536)(705.37109539,65.75151535)(705.33110107,65.74151611)
\curveto(705.22109554,65.71151539)(705.11609565,65.65651545)(705.01610107,65.57651611)
\curveto(704.92609584,65.49651561)(704.83609593,65.42651568)(704.74610107,65.36651611)
\curveto(704.61609615,65.28651582)(704.47609629,65.21151589)(704.32610107,65.14151611)
\curveto(704.17609659,65.08151602)(704.01609675,65.02651608)(703.84610107,64.97651611)
\curveto(703.74609702,64.94651616)(703.63609713,64.92651618)(703.51610107,64.91651611)
\curveto(703.40609736,64.9065162)(703.29609747,64.89151621)(703.18610107,64.87151611)
\curveto(703.13609763,64.86151624)(703.09109767,64.85651625)(703.05110107,64.85651611)
\lineto(702.94610107,64.85651611)
\curveto(702.83609793,64.83651627)(702.73109803,64.83651627)(702.63110107,64.85651611)
\lineto(702.49610107,64.85651611)
\curveto(702.44609832,64.86651624)(702.39609837,64.87151623)(702.34610107,64.87151611)
\curveto(702.29609847,64.87151623)(702.25109851,64.88151622)(702.21110107,64.90151611)
\curveto(702.17109859,64.91151619)(702.13609863,64.91651619)(702.10610107,64.91651611)
\curveto(702.08609868,64.9065162)(702.0610987,64.9065162)(702.03110107,64.91651611)
\lineto(701.79110107,64.97651611)
\curveto(701.71109905,64.98651612)(701.63609913,65.0065161)(701.56610107,65.03651611)
\curveto(701.2660995,65.16651594)(701.02109974,65.31151579)(700.83110107,65.47151611)
\curveto(700.65110011,65.64151546)(700.50110026,65.87651523)(700.38110107,66.17651611)
\curveto(700.29110047,66.39651471)(700.24610052,66.66151444)(700.24610107,66.97151611)
\lineto(700.24610107,67.28651611)
\curveto(700.25610051,67.33651377)(700.2611005,67.38651372)(700.26110107,67.43651611)
\lineto(700.29110107,67.61651611)
\lineto(700.41110107,67.94651611)
\curveto(700.45110031,68.05651305)(700.50110026,68.15651295)(700.56110107,68.24651611)
\curveto(700.74110002,68.53651257)(700.98609978,68.75151235)(701.29610107,68.89151611)
\curveto(701.60609916,69.03151207)(701.94609882,69.15651195)(702.31610107,69.26651611)
\curveto(702.45609831,69.3065118)(702.60109816,69.33651177)(702.75110107,69.35651611)
\curveto(702.90109786,69.37651173)(703.05109771,69.4015117)(703.20110107,69.43151611)
\curveto(703.27109749,69.45151165)(703.33609743,69.46151164)(703.39610107,69.46151611)
\curveto(703.4660973,69.46151164)(703.54109722,69.47151163)(703.62110107,69.49151611)
\curveto(703.69109707,69.51151159)(703.761097,69.52151158)(703.83110107,69.52151611)
\curveto(703.90109686,69.53151157)(703.97609679,69.54651156)(704.05610107,69.56651611)
\curveto(704.30609646,69.62651148)(704.54109622,69.67651143)(704.76110107,69.71651611)
\curveto(704.98109578,69.76651134)(705.15609561,69.88151122)(705.28610107,70.06151611)
\curveto(705.34609542,70.14151096)(705.39609537,70.24151086)(705.43610107,70.36151611)
\curveto(705.47609529,70.49151061)(705.47609529,70.63151047)(705.43610107,70.78151611)
\curveto(705.37609539,71.02151008)(705.28609548,71.21150989)(705.16610107,71.35151611)
\curveto(705.05609571,71.49150961)(704.89609587,71.6015095)(704.68610107,71.68151611)
\curveto(704.5660962,71.73150937)(704.42109634,71.76650934)(704.25110107,71.78651611)
\curveto(704.09109667,71.8065093)(703.92109684,71.81650929)(703.74110107,71.81651611)
\curveto(703.5610972,71.81650929)(703.38609738,71.8065093)(703.21610107,71.78651611)
\curveto(703.04609772,71.76650934)(702.90109786,71.73650937)(702.78110107,71.69651611)
\curveto(702.61109815,71.63650947)(702.44609832,71.55150955)(702.28610107,71.44151611)
\curveto(702.20609856,71.38150972)(702.13109863,71.3015098)(702.06110107,71.20151611)
\curveto(702.00109876,71.11150999)(701.94609882,71.01151009)(701.89610107,70.90151611)
\curveto(701.8660989,70.82151028)(701.83609893,70.73651037)(701.80610107,70.64651611)
\curveto(701.78609898,70.55651055)(701.74109902,70.48651062)(701.67110107,70.43651611)
\curveto(701.63109913,70.4065107)(701.5610992,70.38151072)(701.46110107,70.36151611)
\curveto(701.37109939,70.35151075)(701.27609949,70.34651076)(701.17610107,70.34651611)
\curveto(701.07609969,70.34651076)(700.97609979,70.35151075)(700.87610107,70.36151611)
\curveto(700.78609998,70.38151072)(700.72110004,70.4065107)(700.68110107,70.43651611)
\curveto(700.64110012,70.46651064)(700.61110015,70.51651059)(700.59110107,70.58651611)
\curveto(700.57110019,70.65651045)(700.57110019,70.73151037)(700.59110107,70.81151611)
\curveto(700.62110014,70.94151016)(700.65110011,71.06151004)(700.68110107,71.17151611)
\curveto(700.72110004,71.29150981)(700.7661,71.4065097)(700.81610107,71.51651611)
\curveto(701.00609976,71.86650924)(701.24609952,72.13650897)(701.53610107,72.32651611)
\curveto(701.82609894,72.52650858)(702.18609858,72.68650842)(702.61610107,72.80651611)
\curveto(702.71609805,72.82650828)(702.81609795,72.84150826)(702.91610107,72.85151611)
\curveto(703.02609774,72.86150824)(703.13609763,72.87650823)(703.24610107,72.89651611)
\curveto(703.28609748,72.9065082)(703.35109741,72.9065082)(703.44110107,72.89651611)
\curveto(703.53109723,72.89650821)(703.58609718,72.9065082)(703.60610107,72.92651611)
\curveto(704.30609646,72.93650817)(704.91609585,72.85650825)(705.43610107,72.68651611)
\curveto(705.95609481,72.51650859)(706.32109444,72.19150891)(706.53110107,71.71151611)
\curveto(706.62109414,71.51150959)(706.67109409,71.27650983)(706.68110107,71.00651611)
\curveto(706.70109406,70.74651036)(706.71109405,70.47151063)(706.71110107,70.18151611)
\lineto(706.71110107,66.86651611)
\curveto(706.71109405,66.72651438)(706.71609405,66.59151451)(706.72610107,66.46151611)
\curveto(706.73609403,66.33151477)(706.766094,66.22651488)(706.81610107,66.14651611)
\curveto(706.8660939,66.07651503)(706.93109383,66.02651508)(707.01110107,65.99651611)
\curveto(707.10109366,65.95651515)(707.18609358,65.92651518)(707.26610107,65.90651611)
\curveto(707.34609342,65.89651521)(707.40609336,65.85151525)(707.44610107,65.77151611)
\curveto(707.4660933,65.74151536)(707.47609329,65.71151539)(707.47610107,65.68151611)
\curveto(707.47609329,65.65151545)(707.48109328,65.61151549)(707.49110107,65.56151611)
\moveto(705.34610107,67.22651611)
\curveto(705.40609536,67.36651374)(705.43609533,67.52651358)(705.43610107,67.70651611)
\curveto(705.44609532,67.89651321)(705.45109531,68.09151301)(705.45110107,68.29151611)
\curveto(705.45109531,68.4015127)(705.44609532,68.5015126)(705.43610107,68.59151611)
\curveto(705.42609534,68.68151242)(705.38609538,68.75151235)(705.31610107,68.80151611)
\curveto(705.28609548,68.82151228)(705.21609555,68.83151227)(705.10610107,68.83151611)
\curveto(705.08609568,68.81151229)(705.05109571,68.8015123)(705.00110107,68.80151611)
\curveto(704.95109581,68.8015123)(704.90609586,68.79151231)(704.86610107,68.77151611)
\curveto(704.78609598,68.75151235)(704.69609607,68.73151237)(704.59610107,68.71151611)
\lineto(704.29610107,68.65151611)
\curveto(704.2660965,68.65151245)(704.23109653,68.64651246)(704.19110107,68.63651611)
\lineto(704.08610107,68.63651611)
\curveto(703.93609683,68.59651251)(703.77109699,68.57151253)(703.59110107,68.56151611)
\curveto(703.42109734,68.56151254)(703.2610975,68.54151256)(703.11110107,68.50151611)
\curveto(703.03109773,68.48151262)(702.95609781,68.46151264)(702.88610107,68.44151611)
\curveto(702.82609794,68.43151267)(702.75609801,68.41651269)(702.67610107,68.39651611)
\curveto(702.51609825,68.34651276)(702.3660984,68.28151282)(702.22610107,68.20151611)
\curveto(702.08609868,68.13151297)(701.9660988,68.04151306)(701.86610107,67.93151611)
\curveto(701.766099,67.82151328)(701.69109907,67.68651342)(701.64110107,67.52651611)
\curveto(701.59109917,67.37651373)(701.57109919,67.19151391)(701.58110107,66.97151611)
\curveto(701.58109918,66.87151423)(701.59609917,66.77651433)(701.62610107,66.68651611)
\curveto(701.6660991,66.6065145)(701.71109905,66.53151457)(701.76110107,66.46151611)
\curveto(701.84109892,66.35151475)(701.94609882,66.25651485)(702.07610107,66.17651611)
\curveto(702.20609856,66.106515)(702.34609842,66.04651506)(702.49610107,65.99651611)
\curveto(702.54609822,65.98651512)(702.59609817,65.98151512)(702.64610107,65.98151611)
\curveto(702.69609807,65.98151512)(702.74609802,65.97651513)(702.79610107,65.96651611)
\curveto(702.8660979,65.94651516)(702.95109781,65.93151517)(703.05110107,65.92151611)
\curveto(703.1610976,65.92151518)(703.25109751,65.93151517)(703.32110107,65.95151611)
\curveto(703.38109738,65.97151513)(703.44109732,65.97651513)(703.50110107,65.96651611)
\curveto(703.5610972,65.96651514)(703.62109714,65.97651513)(703.68110107,65.99651611)
\curveto(703.761097,66.01651509)(703.83609693,66.03151507)(703.90610107,66.04151611)
\curveto(703.98609678,66.05151505)(704.0610967,66.07151503)(704.13110107,66.10151611)
\curveto(704.42109634,66.22151488)(704.6660961,66.36651474)(704.86610107,66.53651611)
\curveto(705.07609569,66.7065144)(705.23609553,66.93651417)(705.34610107,67.22651611)
}
}
{
\newrgbcolor{curcolor}{0 0 0}
\pscustom[linestyle=none,fillstyle=solid,fillcolor=curcolor]
{
\newpath
\moveto(715.6227417,65.81651611)
\lineto(715.6227417,65.42651611)
\curveto(715.62273382,65.3065158)(715.59773385,65.2065159)(715.5477417,65.12651611)
\curveto(715.49773395,65.05651605)(715.41273403,65.01651609)(715.2927417,65.00651611)
\lineto(714.9477417,65.00651611)
\curveto(714.88773456,65.0065161)(714.82773462,65.0015161)(714.7677417,64.99151611)
\curveto(714.71773473,64.99151611)(714.67273477,65.0015161)(714.6327417,65.02151611)
\curveto(714.5427349,65.04151606)(714.48273496,65.08151602)(714.4527417,65.14151611)
\curveto(714.41273503,65.19151591)(714.38773506,65.25151585)(714.3777417,65.32151611)
\curveto(714.37773507,65.39151571)(714.36273508,65.46151564)(714.3327417,65.53151611)
\curveto(714.32273512,65.55151555)(714.30773514,65.56651554)(714.2877417,65.57651611)
\curveto(714.27773517,65.59651551)(714.26273518,65.61651549)(714.2427417,65.63651611)
\curveto(714.1427353,65.64651546)(714.06273538,65.62651548)(714.0027417,65.57651611)
\curveto(713.95273549,65.52651558)(713.89773555,65.47651563)(713.8377417,65.42651611)
\curveto(713.63773581,65.27651583)(713.43773601,65.16151594)(713.2377417,65.08151611)
\curveto(713.05773639,65.0015161)(712.8477366,64.94151616)(712.6077417,64.90151611)
\curveto(712.37773707,64.86151624)(712.13773731,64.84151626)(711.8877417,64.84151611)
\curveto(711.6477378,64.83151627)(711.40773804,64.84651626)(711.1677417,64.88651611)
\curveto(710.92773852,64.91651619)(710.71773873,64.97151613)(710.5377417,65.05151611)
\curveto(710.01773943,65.27151583)(709.59773985,65.56651554)(709.2777417,65.93651611)
\curveto(708.95774049,66.31651479)(708.70774074,66.78651432)(708.5277417,67.34651611)
\curveto(708.48774096,67.43651367)(708.45774099,67.52651358)(708.4377417,67.61651611)
\curveto(708.42774102,67.71651339)(708.40774104,67.81651329)(708.3777417,67.91651611)
\curveto(708.36774108,67.96651314)(708.36274108,68.01651309)(708.3627417,68.06651611)
\curveto(708.36274108,68.11651299)(708.35774109,68.16651294)(708.3477417,68.21651611)
\curveto(708.32774112,68.26651284)(708.31774113,68.31651279)(708.3177417,68.36651611)
\curveto(708.32774112,68.42651268)(708.32774112,68.48151262)(708.3177417,68.53151611)
\lineto(708.3177417,68.68151611)
\curveto(708.29774115,68.73151237)(708.28774116,68.79651231)(708.2877417,68.87651611)
\curveto(708.28774116,68.95651215)(708.29774115,69.02151208)(708.3177417,69.07151611)
\lineto(708.3177417,69.23651611)
\curveto(708.33774111,69.3065118)(708.3427411,69.37651173)(708.3327417,69.44651611)
\curveto(708.33274111,69.52651158)(708.3427411,69.6015115)(708.3627417,69.67151611)
\curveto(708.37274107,69.72151138)(708.37774107,69.76651134)(708.3777417,69.80651611)
\curveto(708.37774107,69.84651126)(708.38274106,69.89151121)(708.3927417,69.94151611)
\curveto(708.42274102,70.04151106)(708.447741,70.13651097)(708.4677417,70.22651611)
\curveto(708.48774096,70.32651078)(708.51274093,70.42151068)(708.5427417,70.51151611)
\curveto(708.67274077,70.89151021)(708.83774061,71.23150987)(709.0377417,71.53151611)
\curveto(709.2477402,71.84150926)(709.49773995,72.09650901)(709.7877417,72.29651611)
\curveto(709.95773949,72.41650869)(710.13273931,72.51650859)(710.3127417,72.59651611)
\curveto(710.50273894,72.67650843)(710.70773874,72.74650836)(710.9277417,72.80651611)
\curveto(710.99773845,72.81650829)(711.06273838,72.82650828)(711.1227417,72.83651611)
\curveto(711.19273825,72.84650826)(711.26273818,72.86150824)(711.3327417,72.88151611)
\lineto(711.4827417,72.88151611)
\curveto(711.56273788,72.9015082)(711.67773777,72.91150819)(711.8277417,72.91151611)
\curveto(711.98773746,72.91150819)(712.10773734,72.9015082)(712.1877417,72.88151611)
\curveto(712.22773722,72.87150823)(712.28273716,72.86650824)(712.3527417,72.86651611)
\curveto(712.46273698,72.83650827)(712.57273687,72.81150829)(712.6827417,72.79151611)
\curveto(712.79273665,72.78150832)(712.89773655,72.75150835)(712.9977417,72.70151611)
\curveto(713.1477363,72.64150846)(713.28773616,72.57650853)(713.4177417,72.50651611)
\curveto(713.55773589,72.43650867)(713.68773576,72.35650875)(713.8077417,72.26651611)
\curveto(713.86773558,72.21650889)(713.92773552,72.16150894)(713.9877417,72.10151611)
\curveto(714.05773539,72.05150905)(714.1477353,72.03650907)(714.2577417,72.05651611)
\curveto(714.27773517,72.08650902)(714.29273515,72.11150899)(714.3027417,72.13151611)
\curveto(714.32273512,72.15150895)(714.33773511,72.18150892)(714.3477417,72.22151611)
\curveto(714.37773507,72.31150879)(714.38773506,72.42650868)(714.3777417,72.56651611)
\lineto(714.3777417,72.94151611)
\lineto(714.3777417,74.66651611)
\lineto(714.3777417,75.13151611)
\curveto(714.37773507,75.31150579)(714.40273504,75.44150566)(714.4527417,75.52151611)
\curveto(714.49273495,75.59150551)(714.55273489,75.63650547)(714.6327417,75.65651611)
\curveto(714.65273479,75.65650545)(714.67773477,75.65650545)(714.7077417,75.65651611)
\curveto(714.73773471,75.66650544)(714.76273468,75.67150543)(714.7827417,75.67151611)
\curveto(714.92273452,75.68150542)(715.06773438,75.68150542)(715.2177417,75.67151611)
\curveto(715.37773407,75.67150543)(715.48773396,75.63150547)(715.5477417,75.55151611)
\curveto(715.59773385,75.47150563)(715.62273382,75.37150573)(715.6227417,75.25151611)
\lineto(715.6227417,74.87651611)
\lineto(715.6227417,65.81651611)
\moveto(714.4077417,68.65151611)
\curveto(714.42773502,68.7015124)(714.43773501,68.76651234)(714.4377417,68.84651611)
\curveto(714.43773501,68.93651217)(714.42773502,69.0065121)(714.4077417,69.05651611)
\lineto(714.4077417,69.28151611)
\curveto(714.38773506,69.37151173)(714.37273507,69.46151164)(714.3627417,69.55151611)
\curveto(714.35273509,69.65151145)(714.33273511,69.74151136)(714.3027417,69.82151611)
\curveto(714.28273516,69.9015112)(714.26273518,69.97651113)(714.2427417,70.04651611)
\curveto(714.23273521,70.11651099)(714.21273523,70.18651092)(714.1827417,70.25651611)
\curveto(714.06273538,70.55651055)(713.90773554,70.82151028)(713.7177417,71.05151611)
\curveto(713.52773592,71.28150982)(713.28773616,71.46150964)(712.9977417,71.59151611)
\curveto(712.89773655,71.64150946)(712.79273665,71.67650943)(712.6827417,71.69651611)
\curveto(712.58273686,71.72650938)(712.47273697,71.75150935)(712.3527417,71.77151611)
\curveto(712.27273717,71.79150931)(712.18273726,71.8015093)(712.0827417,71.80151611)
\lineto(711.8127417,71.80151611)
\curveto(711.76273768,71.79150931)(711.71773773,71.78150932)(711.6777417,71.77151611)
\lineto(711.5427417,71.77151611)
\curveto(711.46273798,71.75150935)(711.37773807,71.73150937)(711.2877417,71.71151611)
\curveto(711.20773824,71.69150941)(711.12773832,71.66650944)(711.0477417,71.63651611)
\curveto(710.72773872,71.49650961)(710.46773898,71.29150981)(710.2677417,71.02151611)
\curveto(710.07773937,70.76151034)(709.92273952,70.45651065)(709.8027417,70.10651611)
\curveto(709.76273968,69.99651111)(709.73273971,69.88151122)(709.7127417,69.76151611)
\curveto(709.70273974,69.65151145)(709.68773976,69.54151156)(709.6677417,69.43151611)
\curveto(709.66773978,69.39151171)(709.66273978,69.35151175)(709.6527417,69.31151611)
\lineto(709.6527417,69.20651611)
\curveto(709.63273981,69.15651195)(709.62273982,69.101512)(709.6227417,69.04151611)
\curveto(709.63273981,68.98151212)(709.63773981,68.92651218)(709.6377417,68.87651611)
\lineto(709.6377417,68.54651611)
\curveto(709.63773981,68.44651266)(709.6477398,68.35151275)(709.6677417,68.26151611)
\curveto(709.67773977,68.23151287)(709.68273976,68.18151292)(709.6827417,68.11151611)
\curveto(709.70273974,68.04151306)(709.71773973,67.97151313)(709.7277417,67.90151611)
\lineto(709.7877417,67.69151611)
\curveto(709.89773955,67.34151376)(710.0477394,67.04151406)(710.2377417,66.79151611)
\curveto(710.42773902,66.54151456)(710.66773878,66.33651477)(710.9577417,66.17651611)
\curveto(711.0477384,66.12651498)(711.13773831,66.08651502)(711.2277417,66.05651611)
\curveto(711.31773813,66.02651508)(711.41773803,65.99651511)(711.5277417,65.96651611)
\curveto(711.57773787,65.94651516)(711.62773782,65.94151516)(711.6777417,65.95151611)
\curveto(711.73773771,65.96151514)(711.79273765,65.95651515)(711.8427417,65.93651611)
\curveto(711.88273756,65.92651518)(711.92273752,65.92151518)(711.9627417,65.92151611)
\lineto(712.0977417,65.92151611)
\lineto(712.2327417,65.92151611)
\curveto(712.26273718,65.93151517)(712.31273713,65.93651517)(712.3827417,65.93651611)
\curveto(712.46273698,65.95651515)(712.5427369,65.97151513)(712.6227417,65.98151611)
\curveto(712.70273674,66.0015151)(712.77773667,66.02651508)(712.8477417,66.05651611)
\curveto(713.17773627,66.19651491)(713.442736,66.37151473)(713.6427417,66.58151611)
\curveto(713.85273559,66.8015143)(714.02773542,67.07651403)(714.1677417,67.40651611)
\curveto(714.21773523,67.51651359)(714.25273519,67.62651348)(714.2727417,67.73651611)
\curveto(714.29273515,67.84651326)(714.31773513,67.95651315)(714.3477417,68.06651611)
\curveto(714.36773508,68.106513)(714.37773507,68.14151296)(714.3777417,68.17151611)
\curveto(714.37773507,68.21151289)(714.38273506,68.25151285)(714.3927417,68.29151611)
\curveto(714.40273504,68.35151275)(714.40273504,68.41151269)(714.3927417,68.47151611)
\curveto(714.39273505,68.53151257)(714.39773505,68.59151251)(714.4077417,68.65151611)
}
}
{
\newrgbcolor{curcolor}{0 0 0}
\pscustom[linestyle=none,fillstyle=solid,fillcolor=curcolor]
{
\newpath
\moveto(724.3189917,69.17651611)
\curveto(724.33898401,69.07651203)(724.33898401,68.96151214)(724.3189917,68.83151611)
\curveto(724.30898404,68.71151239)(724.27898407,68.62651248)(724.2289917,68.57651611)
\curveto(724.17898417,68.53651257)(724.10398425,68.5065126)(724.0039917,68.48651611)
\curveto(723.91398444,68.47651263)(723.80898454,68.47151263)(723.6889917,68.47151611)
\lineto(723.3289917,68.47151611)
\curveto(723.20898514,68.48151262)(723.10398525,68.48651262)(723.0139917,68.48651611)
\lineto(719.1739917,68.48651611)
\curveto(719.09398926,68.48651262)(719.01398934,68.48151262)(718.9339917,68.47151611)
\curveto(718.8539895,68.47151263)(718.78898956,68.45651265)(718.7389917,68.42651611)
\curveto(718.69898965,68.4065127)(718.65898969,68.36651274)(718.6189917,68.30651611)
\curveto(718.59898975,68.27651283)(718.57898977,68.23151287)(718.5589917,68.17151611)
\curveto(718.53898981,68.12151298)(718.53898981,68.07151303)(718.5589917,68.02151611)
\curveto(718.56898978,67.97151313)(718.57398978,67.92651318)(718.5739917,67.88651611)
\curveto(718.57398978,67.84651326)(718.57898977,67.8065133)(718.5889917,67.76651611)
\curveto(718.60898974,67.68651342)(718.62898972,67.6015135)(718.6489917,67.51151611)
\curveto(718.66898968,67.43151367)(718.69898965,67.35151375)(718.7389917,67.27151611)
\curveto(718.96898938,66.73151437)(719.348989,66.34651476)(719.8789917,66.11651611)
\curveto(719.93898841,66.08651502)(720.00398835,66.06151504)(720.0739917,66.04151611)
\lineto(720.2839917,65.98151611)
\curveto(720.31398804,65.97151513)(720.36398799,65.96651514)(720.4339917,65.96651611)
\curveto(720.57398778,65.92651518)(720.75898759,65.9065152)(720.9889917,65.90651611)
\curveto(721.21898713,65.9065152)(721.40398695,65.92651518)(721.5439917,65.96651611)
\curveto(721.68398667,66.0065151)(721.80898654,66.04651506)(721.9189917,66.08651611)
\curveto(722.03898631,66.13651497)(722.1489862,66.19651491)(722.2489917,66.26651611)
\curveto(722.35898599,66.33651477)(722.4539859,66.41651469)(722.5339917,66.50651611)
\curveto(722.61398574,66.6065145)(722.68398567,66.71151439)(722.7439917,66.82151611)
\curveto(722.80398555,66.92151418)(722.8539855,67.02651408)(722.8939917,67.13651611)
\curveto(722.94398541,67.24651386)(723.02398533,67.32651378)(723.1339917,67.37651611)
\curveto(723.17398518,67.39651371)(723.23898511,67.41151369)(723.3289917,67.42151611)
\curveto(723.41898493,67.43151367)(723.50898484,67.43151367)(723.5989917,67.42151611)
\curveto(723.68898466,67.42151368)(723.77398458,67.41651369)(723.8539917,67.40651611)
\curveto(723.93398442,67.39651371)(723.98898436,67.37651373)(724.0189917,67.34651611)
\curveto(724.11898423,67.27651383)(724.14398421,67.16151394)(724.0939917,67.00151611)
\curveto(724.01398434,66.73151437)(723.90898444,66.49151461)(723.7789917,66.28151611)
\curveto(723.57898477,65.96151514)(723.348985,65.69651541)(723.0889917,65.48651611)
\curveto(722.83898551,65.28651582)(722.51898583,65.12151598)(722.1289917,64.99151611)
\curveto(722.02898632,64.95151615)(721.92898642,64.92651618)(721.8289917,64.91651611)
\curveto(721.72898662,64.89651621)(721.62398673,64.87651623)(721.5139917,64.85651611)
\curveto(721.46398689,64.84651626)(721.41398694,64.84151626)(721.3639917,64.84151611)
\curveto(721.32398703,64.84151626)(721.27898707,64.83651627)(721.2289917,64.82651611)
\lineto(721.0789917,64.82651611)
\curveto(721.02898732,64.81651629)(720.96898738,64.81151629)(720.8989917,64.81151611)
\curveto(720.83898751,64.81151629)(720.78898756,64.81651629)(720.7489917,64.82651611)
\lineto(720.6139917,64.82651611)
\curveto(720.56398779,64.83651627)(720.51898783,64.84151626)(720.4789917,64.84151611)
\curveto(720.43898791,64.84151626)(720.39898795,64.84651626)(720.3589917,64.85651611)
\curveto(720.30898804,64.86651624)(720.2539881,64.87651623)(720.1939917,64.88651611)
\curveto(720.13398822,64.88651622)(720.07898827,64.89151621)(720.0289917,64.90151611)
\curveto(719.93898841,64.92151618)(719.8489885,64.94651616)(719.7589917,64.97651611)
\curveto(719.66898868,64.99651611)(719.58398877,65.02151608)(719.5039917,65.05151611)
\curveto(719.46398889,65.07151603)(719.42898892,65.08151602)(719.3989917,65.08151611)
\curveto(719.36898898,65.09151601)(719.33398902,65.106516)(719.2939917,65.12651611)
\curveto(719.14398921,65.19651591)(718.98398937,65.28151582)(718.8139917,65.38151611)
\curveto(718.52398983,65.57151553)(718.27399008,65.8015153)(718.0639917,66.07151611)
\curveto(717.86399049,66.35151475)(717.69399066,66.66151444)(717.5539917,67.00151611)
\curveto(717.50399085,67.11151399)(717.46399089,67.22651388)(717.4339917,67.34651611)
\curveto(717.41399094,67.46651364)(717.38399097,67.58651352)(717.3439917,67.70651611)
\curveto(717.33399102,67.74651336)(717.32899102,67.78151332)(717.3289917,67.81151611)
\curveto(717.32899102,67.84151326)(717.32399103,67.88151322)(717.3139917,67.93151611)
\curveto(717.29399106,68.01151309)(717.27899107,68.09651301)(717.2689917,68.18651611)
\curveto(717.25899109,68.27651283)(717.24399111,68.36651274)(717.2239917,68.45651611)
\lineto(717.2239917,68.66651611)
\curveto(717.21399114,68.7065124)(717.20399115,68.76151234)(717.1939917,68.83151611)
\curveto(717.19399116,68.91151219)(717.19899115,68.97651213)(717.2089917,69.02651611)
\lineto(717.2089917,69.19151611)
\curveto(717.22899112,69.24151186)(717.23399112,69.29151181)(717.2239917,69.34151611)
\curveto(717.22399113,69.4015117)(717.22899112,69.45651165)(717.2389917,69.50651611)
\curveto(717.27899107,69.66651144)(717.30899104,69.82651128)(717.3289917,69.98651611)
\curveto(717.35899099,70.14651096)(717.40399095,70.29651081)(717.4639917,70.43651611)
\curveto(717.51399084,70.54651056)(717.55899079,70.65651045)(717.5989917,70.76651611)
\curveto(717.6489907,70.88651022)(717.70399065,71.0015101)(717.7639917,71.11151611)
\curveto(717.98399037,71.46150964)(718.23399012,71.76150934)(718.5139917,72.01151611)
\curveto(718.79398956,72.27150883)(719.13898921,72.48650862)(719.5489917,72.65651611)
\curveto(719.66898868,72.7065084)(719.78898856,72.74150836)(719.9089917,72.76151611)
\curveto(720.03898831,72.79150831)(720.17398818,72.82150828)(720.3139917,72.85151611)
\curveto(720.36398799,72.86150824)(720.40898794,72.86650824)(720.4489917,72.86651611)
\curveto(720.48898786,72.87650823)(720.53398782,72.88150822)(720.5839917,72.88151611)
\curveto(720.60398775,72.89150821)(720.62898772,72.89150821)(720.6589917,72.88151611)
\curveto(720.68898766,72.87150823)(720.71398764,72.87650823)(720.7339917,72.89651611)
\curveto(721.1539872,72.9065082)(721.51898683,72.86150824)(721.8289917,72.76151611)
\curveto(722.13898621,72.67150843)(722.41898593,72.54650856)(722.6689917,72.38651611)
\curveto(722.71898563,72.36650874)(722.75898559,72.33650877)(722.7889917,72.29651611)
\curveto(722.81898553,72.26650884)(722.8539855,72.24150886)(722.8939917,72.22151611)
\curveto(722.97398538,72.16150894)(723.0539853,72.09150901)(723.1339917,72.01151611)
\curveto(723.22398513,71.93150917)(723.29898505,71.85150925)(723.3589917,71.77151611)
\curveto(723.51898483,71.56150954)(723.6539847,71.36150974)(723.7639917,71.17151611)
\curveto(723.83398452,71.06151004)(723.88898446,70.94151016)(723.9289917,70.81151611)
\curveto(723.96898438,70.68151042)(724.01398434,70.55151055)(724.0639917,70.42151611)
\curveto(724.11398424,70.29151081)(724.1489842,70.15651095)(724.1689917,70.01651611)
\curveto(724.19898415,69.87651123)(724.23398412,69.73651137)(724.2739917,69.59651611)
\curveto(724.28398407,69.52651158)(724.28898406,69.45651165)(724.2889917,69.38651611)
\lineto(724.3189917,69.17651611)
\moveto(722.8639917,69.68651611)
\curveto(722.89398546,69.72651138)(722.91898543,69.77651133)(722.9389917,69.83651611)
\curveto(722.95898539,69.9065112)(722.95898539,69.97651113)(722.9389917,70.04651611)
\curveto(722.87898547,70.26651084)(722.79398556,70.47151063)(722.6839917,70.66151611)
\curveto(722.54398581,70.89151021)(722.38898596,71.08651002)(722.2189917,71.24651611)
\curveto(722.0489863,71.4065097)(721.82898652,71.54150956)(721.5589917,71.65151611)
\curveto(721.48898686,71.67150943)(721.41898693,71.68650942)(721.3489917,71.69651611)
\curveto(721.27898707,71.71650939)(721.20398715,71.73650937)(721.1239917,71.75651611)
\curveto(721.04398731,71.77650933)(720.95898739,71.78650932)(720.8689917,71.78651611)
\lineto(720.6139917,71.78651611)
\curveto(720.58398777,71.76650934)(720.5489878,71.75650935)(720.5089917,71.75651611)
\curveto(720.46898788,71.76650934)(720.43398792,71.76650934)(720.4039917,71.75651611)
\lineto(720.1639917,71.69651611)
\curveto(720.09398826,71.68650942)(720.02398833,71.67150943)(719.9539917,71.65151611)
\curveto(719.66398869,71.53150957)(719.42898892,71.38150972)(719.2489917,71.20151611)
\curveto(719.07898927,71.02151008)(718.92398943,70.79651031)(718.7839917,70.52651611)
\curveto(718.7539896,70.47651063)(718.72398963,70.41151069)(718.6939917,70.33151611)
\curveto(718.66398969,70.26151084)(718.63898971,70.18151092)(718.6189917,70.09151611)
\curveto(718.59898975,70.0015111)(718.59398976,69.91651119)(718.6039917,69.83651611)
\curveto(718.61398974,69.75651135)(718.6489897,69.69651141)(718.7089917,69.65651611)
\curveto(718.78898956,69.59651151)(718.92398943,69.56651154)(719.1139917,69.56651611)
\curveto(719.31398904,69.57651153)(719.48398887,69.58151152)(719.6239917,69.58151611)
\lineto(721.9039917,69.58151611)
\curveto(722.0539863,69.58151152)(722.23398612,69.57651153)(722.4439917,69.56651611)
\curveto(722.6539857,69.56651154)(722.79398556,69.6065115)(722.8639917,69.68651611)
}
}
{
\newrgbcolor{curcolor}{0 0 0}
\pscustom[linestyle=none,fillstyle=solid,fillcolor=curcolor]
{
\newpath
\moveto(728.05563232,72.91151611)
\curveto(728.77562826,72.92150818)(729.38062765,72.83650827)(729.87063232,72.65651611)
\curveto(730.36062667,72.48650862)(730.74062629,72.18150892)(731.01063232,71.74151611)
\curveto(731.08062595,71.63150947)(731.1356259,71.51650959)(731.17563232,71.39651611)
\curveto(731.21562582,71.28650982)(731.25562578,71.16150994)(731.29563232,71.02151611)
\curveto(731.31562572,70.95151015)(731.32062571,70.87651023)(731.31063232,70.79651611)
\curveto(731.30062573,70.72651038)(731.28562575,70.67151043)(731.26563232,70.63151611)
\curveto(731.24562579,70.61151049)(731.22062581,70.59151051)(731.19063232,70.57151611)
\curveto(731.16062587,70.56151054)(731.1356259,70.54651056)(731.11563232,70.52651611)
\curveto(731.06562597,70.5065106)(731.01562602,70.5015106)(730.96563232,70.51151611)
\curveto(730.91562612,70.52151058)(730.86562617,70.52151058)(730.81563232,70.51151611)
\curveto(730.7356263,70.49151061)(730.6306264,70.48651062)(730.50063232,70.49651611)
\curveto(730.37062666,70.51651059)(730.28062675,70.54151056)(730.23063232,70.57151611)
\curveto(730.15062688,70.62151048)(730.09562694,70.68651042)(730.06563232,70.76651611)
\curveto(730.04562699,70.85651025)(730.01062702,70.94151016)(729.96063232,71.02151611)
\curveto(729.87062716,71.18150992)(729.74562729,71.32650978)(729.58563232,71.45651611)
\curveto(729.47562756,71.53650957)(729.35562768,71.59650951)(729.22563232,71.63651611)
\curveto(729.09562794,71.67650943)(728.95562808,71.71650939)(728.80563232,71.75651611)
\curveto(728.75562828,71.77650933)(728.70562833,71.78150932)(728.65563232,71.77151611)
\curveto(728.60562843,71.77150933)(728.55562848,71.77650933)(728.50563232,71.78651611)
\curveto(728.44562859,71.8065093)(728.37062866,71.81650929)(728.28063232,71.81651611)
\curveto(728.19062884,71.81650929)(728.11562892,71.8065093)(728.05563232,71.78651611)
\lineto(727.96563232,71.78651611)
\lineto(727.81563232,71.75651611)
\curveto(727.76562927,71.75650935)(727.71562932,71.75150935)(727.66563232,71.74151611)
\curveto(727.40562963,71.68150942)(727.19062984,71.59650951)(727.02063232,71.48651611)
\curveto(726.85063018,71.37650973)(726.7356303,71.19150991)(726.67563232,70.93151611)
\curveto(726.65563038,70.86151024)(726.65063038,70.79151031)(726.66063232,70.72151611)
\curveto(726.68063035,70.65151045)(726.70063033,70.59151051)(726.72063232,70.54151611)
\curveto(726.78063025,70.39151071)(726.85063018,70.28151082)(726.93063232,70.21151611)
\curveto(727.02063001,70.15151095)(727.1306299,70.08151102)(727.26063232,70.00151611)
\curveto(727.42062961,69.9015112)(727.60062943,69.82651128)(727.80063232,69.77651611)
\curveto(728.00062903,69.73651137)(728.20062883,69.68651142)(728.40063232,69.62651611)
\curveto(728.5306285,69.58651152)(728.66062837,69.55651155)(728.79063232,69.53651611)
\curveto(728.92062811,69.51651159)(729.05062798,69.48651162)(729.18063232,69.44651611)
\curveto(729.39062764,69.38651172)(729.59562744,69.32651178)(729.79563232,69.26651611)
\curveto(729.99562704,69.21651189)(730.19562684,69.15151195)(730.39563232,69.07151611)
\lineto(730.54563232,69.01151611)
\curveto(730.59562644,68.99151211)(730.64562639,68.96651214)(730.69563232,68.93651611)
\curveto(730.89562614,68.81651229)(731.07062596,68.68151242)(731.22063232,68.53151611)
\curveto(731.37062566,68.38151272)(731.49562554,68.19151291)(731.59563232,67.96151611)
\curveto(731.61562542,67.89151321)(731.6356254,67.79651331)(731.65563232,67.67651611)
\curveto(731.67562536,67.6065135)(731.68562535,67.53151357)(731.68563232,67.45151611)
\curveto(731.69562534,67.38151372)(731.70062533,67.3015138)(731.70063232,67.21151611)
\lineto(731.70063232,67.06151611)
\curveto(731.68062535,66.99151411)(731.67062536,66.92151418)(731.67063232,66.85151611)
\curveto(731.67062536,66.78151432)(731.66062537,66.71151439)(731.64063232,66.64151611)
\curveto(731.61062542,66.53151457)(731.57562546,66.42651468)(731.53563232,66.32651611)
\curveto(731.49562554,66.22651488)(731.45062558,66.13651497)(731.40063232,66.05651611)
\curveto(731.24062579,65.79651531)(731.035626,65.58651552)(730.78563232,65.42651611)
\curveto(730.5356265,65.27651583)(730.25562678,65.14651596)(729.94563232,65.03651611)
\curveto(729.85562718,65.0065161)(729.76062727,64.98651612)(729.66063232,64.97651611)
\curveto(729.57062746,64.95651615)(729.48062755,64.93151617)(729.39063232,64.90151611)
\curveto(729.29062774,64.88151622)(729.19062784,64.87151623)(729.09063232,64.87151611)
\curveto(728.99062804,64.87151623)(728.89062814,64.86151624)(728.79063232,64.84151611)
\lineto(728.64063232,64.84151611)
\curveto(728.59062844,64.83151627)(728.52062851,64.82651628)(728.43063232,64.82651611)
\curveto(728.34062869,64.82651628)(728.27062876,64.83151627)(728.22063232,64.84151611)
\lineto(728.05563232,64.84151611)
\curveto(727.99562904,64.86151624)(727.9306291,64.87151623)(727.86063232,64.87151611)
\curveto(727.79062924,64.86151624)(727.7306293,64.86651624)(727.68063232,64.88651611)
\curveto(727.6306294,64.89651621)(727.56562947,64.9015162)(727.48563232,64.90151611)
\lineto(727.24563232,64.96151611)
\curveto(727.17562986,64.97151613)(727.10062993,64.99151611)(727.02063232,65.02151611)
\curveto(726.71063032,65.12151598)(726.44063059,65.24651586)(726.21063232,65.39651611)
\curveto(725.98063105,65.54651556)(725.78063125,65.74151536)(725.61063232,65.98151611)
\curveto(725.52063151,66.11151499)(725.44563159,66.24651486)(725.38563232,66.38651611)
\curveto(725.32563171,66.52651458)(725.27063176,66.68151442)(725.22063232,66.85151611)
\curveto(725.20063183,66.91151419)(725.19063184,66.98151412)(725.19063232,67.06151611)
\curveto(725.20063183,67.15151395)(725.21563182,67.22151388)(725.23563232,67.27151611)
\curveto(725.26563177,67.31151379)(725.31563172,67.35151375)(725.38563232,67.39151611)
\curveto(725.4356316,67.41151369)(725.50563153,67.42151368)(725.59563232,67.42151611)
\curveto(725.68563135,67.43151367)(725.77563126,67.43151367)(725.86563232,67.42151611)
\curveto(725.95563108,67.41151369)(726.04063099,67.39651371)(726.12063232,67.37651611)
\curveto(726.21063082,67.36651374)(726.27063076,67.35151375)(726.30063232,67.33151611)
\curveto(726.37063066,67.28151382)(726.41563062,67.2065139)(726.43563232,67.10651611)
\curveto(726.46563057,67.01651409)(726.50063053,66.93151417)(726.54063232,66.85151611)
\curveto(726.64063039,66.63151447)(726.77563026,66.46151464)(726.94563232,66.34151611)
\curveto(727.06562997,66.25151485)(727.20062983,66.18151492)(727.35063232,66.13151611)
\curveto(727.50062953,66.08151502)(727.66062937,66.03151507)(727.83063232,65.98151611)
\lineto(728.14563232,65.93651611)
\lineto(728.23563232,65.93651611)
\curveto(728.30562873,65.91651519)(728.39562864,65.9065152)(728.50563232,65.90651611)
\curveto(728.62562841,65.9065152)(728.72562831,65.91651519)(728.80563232,65.93651611)
\curveto(728.87562816,65.93651517)(728.9306281,65.94151516)(728.97063232,65.95151611)
\curveto(729.030628,65.96151514)(729.09062794,65.96651514)(729.15063232,65.96651611)
\curveto(729.21062782,65.97651513)(729.26562777,65.98651512)(729.31563232,65.99651611)
\curveto(729.60562743,66.07651503)(729.8356272,66.18151492)(730.00563232,66.31151611)
\curveto(730.17562686,66.44151466)(730.29562674,66.66151444)(730.36563232,66.97151611)
\curveto(730.38562665,67.02151408)(730.39062664,67.07651403)(730.38063232,67.13651611)
\curveto(730.37062666,67.19651391)(730.36062667,67.24151386)(730.35063232,67.27151611)
\curveto(730.30062673,67.46151364)(730.2306268,67.6015135)(730.14063232,67.69151611)
\curveto(730.05062698,67.79151331)(729.9356271,67.88151322)(729.79563232,67.96151611)
\curveto(729.70562733,68.02151308)(729.60562743,68.07151303)(729.49563232,68.11151611)
\lineto(729.16563232,68.23151611)
\curveto(729.1356279,68.24151286)(729.10562793,68.24651286)(729.07563232,68.24651611)
\curveto(729.05562798,68.24651286)(729.030628,68.25651285)(729.00063232,68.27651611)
\curveto(728.66062837,68.38651272)(728.30562873,68.46651264)(727.93563232,68.51651611)
\curveto(727.57562946,68.57651253)(727.2356298,68.67151243)(726.91563232,68.80151611)
\curveto(726.81563022,68.84151226)(726.72063031,68.87651223)(726.63063232,68.90651611)
\curveto(726.54063049,68.93651217)(726.45563058,68.97651213)(726.37563232,69.02651611)
\curveto(726.18563085,69.13651197)(726.01063102,69.26151184)(725.85063232,69.40151611)
\curveto(725.69063134,69.54151156)(725.56563147,69.71651139)(725.47563232,69.92651611)
\curveto(725.44563159,69.99651111)(725.42063161,70.06651104)(725.40063232,70.13651611)
\curveto(725.39063164,70.2065109)(725.37563166,70.28151082)(725.35563232,70.36151611)
\curveto(725.32563171,70.48151062)(725.31563172,70.61651049)(725.32563232,70.76651611)
\curveto(725.3356317,70.92651018)(725.35063168,71.06151004)(725.37063232,71.17151611)
\curveto(725.39063164,71.22150988)(725.40063163,71.26150984)(725.40063232,71.29151611)
\curveto(725.41063162,71.33150977)(725.42563161,71.37150973)(725.44563232,71.41151611)
\curveto(725.5356315,71.64150946)(725.65563138,71.84150926)(725.80563232,72.01151611)
\curveto(725.96563107,72.18150892)(726.14563089,72.33150877)(726.34563232,72.46151611)
\curveto(726.49563054,72.55150855)(726.66063037,72.62150848)(726.84063232,72.67151611)
\curveto(727.02063001,72.73150837)(727.21062982,72.78650832)(727.41063232,72.83651611)
\curveto(727.48062955,72.84650826)(727.54562949,72.85650825)(727.60563232,72.86651611)
\curveto(727.67562936,72.87650823)(727.75062928,72.88650822)(727.83063232,72.89651611)
\curveto(727.86062917,72.9065082)(727.90062913,72.9065082)(727.95063232,72.89651611)
\curveto(728.00062903,72.88650822)(728.035629,72.89150821)(728.05563232,72.91151611)
}
}
{
\newrgbcolor{curcolor}{0 0 0}
\pscustom[linestyle=none,fillstyle=solid,fillcolor=curcolor]
{
\newpath
\moveto(203.55070068,57.435)
\curveto(204.53069418,57.45498904)(205.35069336,57.2949892)(206.01070068,56.955)
\curveto(206.68069203,56.62498987)(207.20069151,56.16499033)(207.57070068,55.575)
\curveto(207.67069104,55.41499108)(207.75069096,55.25999124)(207.81070068,55.11)
\curveto(207.88069083,54.96999153)(207.94569077,54.7999917)(208.00570068,54.6)
\curveto(208.02569069,54.54999195)(208.04569067,54.47999202)(208.06570068,54.39)
\curveto(208.08569063,54.30999219)(208.08069063,54.23499227)(208.05070068,54.165)
\curveto(208.03069068,54.1049924)(207.99069072,54.06499244)(207.93070068,54.045)
\curveto(207.88069083,54.03499247)(207.82569089,54.01999248)(207.76570068,54)
\lineto(207.61570068,54)
\curveto(207.58569113,53.98999251)(207.54569117,53.98499251)(207.49570068,53.985)
\lineto(207.37570068,53.985)
\curveto(207.23569148,53.98499251)(207.10569161,53.98999251)(206.98570068,54)
\curveto(206.87569184,54.01999248)(206.79569192,54.06999243)(206.74570068,54.15)
\curveto(206.67569204,54.24999225)(206.62069209,54.36499214)(206.58070068,54.495)
\curveto(206.54069217,54.62499187)(206.48569223,54.74499176)(206.41570068,54.855)
\curveto(206.28569243,55.07499143)(206.13569258,55.26499124)(205.96570068,55.425)
\curveto(205.80569291,55.58499091)(205.6156931,55.73499077)(205.39570068,55.875)
\curveto(205.27569344,55.95499054)(205.14069357,56.01499048)(204.99070068,56.055)
\curveto(204.85069386,56.0949904)(204.70569401,56.13499037)(204.55570068,56.175)
\curveto(204.44569427,56.2049903)(204.32069439,56.22499027)(204.18070068,56.235)
\curveto(204.04069467,56.25499024)(203.89069482,56.26499024)(203.73070068,56.265)
\curveto(203.58069513,56.26499024)(203.43069528,56.25499024)(203.28070068,56.235)
\curveto(203.14069557,56.22499027)(203.02069569,56.2049903)(202.92070068,56.175)
\curveto(202.82069589,56.15499034)(202.72569599,56.13499037)(202.63570068,56.115)
\curveto(202.54569617,56.0949904)(202.45569626,56.06499044)(202.36570068,56.025)
\curveto(201.52569719,55.67499083)(200.92069779,55.07499143)(200.55070068,54.225)
\curveto(200.48069823,54.08499241)(200.42069829,53.93499257)(200.37070068,53.775)
\curveto(200.33069838,53.62499287)(200.28569843,53.46999303)(200.23570068,53.31)
\curveto(200.2156985,53.24999325)(200.20569851,53.18499331)(200.20570068,53.115)
\curveto(200.20569851,53.05499344)(200.19569852,52.99499351)(200.17570068,52.935)
\curveto(200.16569855,52.89499361)(200.16069855,52.85999364)(200.16070068,52.83)
\curveto(200.16069855,52.7999937)(200.15569856,52.76499374)(200.14570068,52.725)
\curveto(200.12569859,52.61499388)(200.1106986,52.499994)(200.10070068,52.38)
\lineto(200.10070068,52.035)
\curveto(200.10069861,51.96499454)(200.09569862,51.88999461)(200.08570068,51.81)
\curveto(200.08569863,51.73999476)(200.09069862,51.67499482)(200.10070068,51.615)
\lineto(200.10070068,51.465)
\curveto(200.12069859,51.39499511)(200.12569859,51.32499518)(200.11570068,51.255)
\curveto(200.1156986,51.18499531)(200.12569859,51.11499538)(200.14570068,51.045)
\curveto(200.16569855,50.98499551)(200.17069854,50.92499558)(200.16070068,50.865)
\curveto(200.16069855,50.8049957)(200.17069854,50.74999575)(200.19070068,50.7)
\curveto(200.22069849,50.56999593)(200.24569847,50.43999606)(200.26570068,50.31)
\curveto(200.29569842,50.18999631)(200.33069838,50.06999643)(200.37070068,49.95)
\curveto(200.54069817,49.44999705)(200.76069795,49.01999748)(201.03070068,48.66)
\curveto(201.30069741,48.30999819)(201.65569706,48.01999848)(202.09570068,47.79)
\curveto(202.23569648,47.71999878)(202.37569634,47.66499884)(202.51570068,47.625)
\curveto(202.66569605,47.58499892)(202.82569589,47.53999896)(202.99570068,47.49)
\curveto(203.06569565,47.46999903)(203.13069558,47.45999904)(203.19070068,47.46)
\curveto(203.25069546,47.46999903)(203.32069539,47.46499904)(203.40070068,47.445)
\curveto(203.45069526,47.43499906)(203.54069517,47.42499908)(203.67070068,47.415)
\curveto(203.80069491,47.41499908)(203.89569482,47.42499908)(203.95570068,47.445)
\lineto(204.06070068,47.445)
\curveto(204.10069461,47.45499905)(204.14069457,47.45499905)(204.18070068,47.445)
\curveto(204.22069449,47.44499905)(204.26069445,47.45499905)(204.30070068,47.475)
\curveto(204.40069431,47.49499901)(204.49569422,47.50999899)(204.58570068,47.52)
\curveto(204.68569403,47.53999896)(204.78069393,47.56999893)(204.87070068,47.61)
\curveto(205.65069306,47.92999857)(206.20069251,48.45499805)(206.52070068,49.185)
\curveto(206.60069211,49.36499714)(206.67569204,49.57999692)(206.74570068,49.83)
\curveto(206.76569195,49.91999658)(206.78069193,50.00999649)(206.79070068,50.1)
\curveto(206.8106919,50.1999963)(206.84569187,50.28999621)(206.89570068,50.37)
\curveto(206.94569177,50.44999605)(207.02569169,50.49499601)(207.13570068,50.505)
\curveto(207.24569147,50.51499598)(207.36569135,50.51999598)(207.49570068,50.52)
\lineto(207.64570068,50.52)
\curveto(207.69569102,50.51999598)(207.74069097,50.51499598)(207.78070068,50.505)
\lineto(207.88570068,50.505)
\lineto(207.97570068,50.475)
\curveto(208.0156907,50.47499602)(208.04569067,50.46499604)(208.06570068,50.445)
\curveto(208.13569058,50.4049961)(208.17569054,50.32999617)(208.18570068,50.22)
\curveto(208.19569052,50.11999638)(208.18569053,50.01999648)(208.15570068,49.92)
\curveto(208.09569062,49.68999681)(208.04069067,49.46999703)(207.99070068,49.26)
\curveto(207.94069077,49.04999745)(207.86569085,48.84999765)(207.76570068,48.66)
\curveto(207.68569103,48.52999797)(207.6106911,48.40499809)(207.54070068,48.285)
\curveto(207.48069123,48.16499834)(207.4106913,48.04499845)(207.33070068,47.925)
\curveto(207.15069156,47.66499884)(206.92569179,47.42499908)(206.65570068,47.205)
\curveto(206.39569232,46.99499951)(206.1106926,46.81999968)(205.80070068,46.68)
\curveto(205.69069302,46.62999987)(205.58069313,46.58999991)(205.47070068,46.56)
\curveto(205.37069334,46.52999997)(205.26569345,46.49500001)(205.15570068,46.455)
\curveto(205.04569367,46.41500008)(204.93069378,46.39000011)(204.81070068,46.38)
\curveto(204.70069401,46.36000014)(204.58569413,46.34000016)(204.46570068,46.32)
\curveto(204.4156943,46.3000002)(204.37069434,46.29500021)(204.33070068,46.305)
\curveto(204.29069442,46.30500019)(204.25069446,46.3000002)(204.21070068,46.29)
\curveto(204.15069456,46.28000022)(204.09069462,46.27500022)(204.03070068,46.275)
\curveto(203.97069474,46.27500022)(203.90569481,46.27000023)(203.83570068,46.26)
\curveto(203.80569491,46.25000025)(203.73569498,46.25000025)(203.62570068,46.26)
\curveto(203.52569519,46.26000024)(203.46069525,46.26500024)(203.43070068,46.275)
\curveto(203.38069533,46.28500022)(203.33069538,46.29000021)(203.28070068,46.29)
\curveto(203.24069547,46.28000022)(203.19569552,46.28000022)(203.14570068,46.29)
\lineto(202.99570068,46.29)
\curveto(202.9156958,46.31000019)(202.84069587,46.32500018)(202.77070068,46.335)
\curveto(202.70069601,46.33500016)(202.62569609,46.34500015)(202.54570068,46.365)
\lineto(202.27570068,46.425)
\curveto(202.18569653,46.43500006)(202.10069661,46.45500005)(202.02070068,46.485)
\curveto(201.8106969,46.54499995)(201.62069709,46.61999988)(201.45070068,46.71)
\curveto(200.82069789,46.97999952)(200.3106984,47.36499914)(199.92070068,47.865)
\curveto(199.53069918,48.36499814)(199.22069949,48.95499755)(198.99070068,49.635)
\curveto(198.95069976,49.75499675)(198.9156998,49.87999662)(198.88570068,50.01)
\curveto(198.86569985,50.13999636)(198.84069987,50.27499622)(198.81070068,50.415)
\curveto(198.79069992,50.46499604)(198.78069993,50.50999599)(198.78070068,50.55)
\curveto(198.79069992,50.58999591)(198.79069992,50.63499587)(198.78070068,50.685)
\curveto(198.76069995,50.77499572)(198.74569997,50.86999563)(198.73570068,50.97)
\curveto(198.73569998,51.06999543)(198.72569999,51.16499534)(198.70570068,51.255)
\lineto(198.70570068,51.54)
\curveto(198.68570003,51.58999491)(198.67570004,51.67499482)(198.67570068,51.795)
\curveto(198.67570004,51.91499458)(198.68570003,51.9999945)(198.70570068,52.05)
\curveto(198.7157,52.07999442)(198.7157,52.10999439)(198.70570068,52.14)
\curveto(198.69570002,52.17999432)(198.69570002,52.20999429)(198.70570068,52.23)
\lineto(198.70570068,52.365)
\curveto(198.7157,52.44499405)(198.72069999,52.52499397)(198.72070068,52.605)
\curveto(198.73069998,52.69499381)(198.74569997,52.77999372)(198.76570068,52.86)
\curveto(198.78569993,52.91999358)(198.79569992,52.97999352)(198.79570068,53.04)
\curveto(198.79569992,53.10999339)(198.80569991,53.17999332)(198.82570068,53.25)
\curveto(198.87569984,53.41999308)(198.9156998,53.58499291)(198.94570068,53.745)
\curveto(198.97569974,53.9049926)(199.02069969,54.05499244)(199.08070068,54.195)
\lineto(199.23070068,54.585)
\curveto(199.29069942,54.72499177)(199.35569936,54.84999165)(199.42570068,54.96)
\curveto(199.57569914,55.21999128)(199.72569899,55.45499104)(199.87570068,55.665)
\curveto(199.90569881,55.71499078)(199.94069877,55.75499074)(199.98070068,55.785)
\curveto(200.03069868,55.82499067)(200.07069864,55.86999063)(200.10070068,55.92)
\curveto(200.16069855,55.9999905)(200.22069849,56.06999043)(200.28070068,56.13)
\lineto(200.49070068,56.31)
\curveto(200.55069816,56.35999014)(200.60569811,56.4049901)(200.65570068,56.445)
\curveto(200.715698,56.49499)(200.78069793,56.54498996)(200.85070068,56.595)
\curveto(201.00069771,56.7049898)(201.15569756,56.7999897)(201.31570068,56.88)
\curveto(201.48569723,56.95998954)(201.66069705,57.03998946)(201.84070068,57.12)
\curveto(201.95069676,57.16998933)(202.06569665,57.2049893)(202.18570068,57.225)
\curveto(202.3156964,57.25498924)(202.44069627,57.28998921)(202.56070068,57.33)
\curveto(202.63069608,57.33998916)(202.69569602,57.34998915)(202.75570068,57.36)
\lineto(202.93570068,57.39)
\curveto(203.0156957,57.3999891)(203.09069562,57.4049891)(203.16070068,57.405)
\curveto(203.24069547,57.41498909)(203.32069539,57.42498907)(203.40070068,57.435)
\curveto(203.42069529,57.44498906)(203.44569527,57.44498906)(203.47570068,57.435)
\curveto(203.50569521,57.42498907)(203.53069518,57.42498907)(203.55070068,57.435)
}
}
{
\newrgbcolor{curcolor}{0 0 0}
\pscustom[linestyle=none,fillstyle=solid,fillcolor=curcolor]
{
\newpath
\moveto(216.67054443,47.07)
\curveto(216.7005366,46.90999959)(216.68553662,46.77499972)(216.62554443,46.665)
\curveto(216.56553674,46.56499994)(216.48553682,46.49000001)(216.38554443,46.44)
\curveto(216.33553697,46.42000008)(216.28053702,46.41000009)(216.22054443,46.41)
\curveto(216.17053713,46.41000009)(216.11553719,46.4000001)(216.05554443,46.38)
\curveto(215.83553747,46.33000017)(215.61553769,46.34500015)(215.39554443,46.425)
\curveto(215.18553812,46.49500001)(215.04053826,46.58499992)(214.96054443,46.695)
\curveto(214.91053839,46.76499974)(214.86553844,46.84499965)(214.82554443,46.935)
\curveto(214.78553852,47.03499946)(214.73553857,47.11499938)(214.67554443,47.175)
\curveto(214.65553865,47.19499931)(214.63053867,47.21499928)(214.60054443,47.235)
\curveto(214.58053872,47.25499925)(214.55053875,47.25999924)(214.51054443,47.25)
\curveto(214.4005389,47.21999928)(214.29553901,47.16499934)(214.19554443,47.085)
\curveto(214.1055392,47.00499949)(214.01553929,46.93499956)(213.92554443,46.875)
\curveto(213.79553951,46.79499971)(213.65553965,46.71999978)(213.50554443,46.65)
\curveto(213.35553995,46.58999991)(213.19554011,46.53499996)(213.02554443,46.485)
\curveto(212.92554038,46.45500005)(212.81554049,46.43500006)(212.69554443,46.425)
\curveto(212.58554072,46.41500008)(212.47554083,46.4000001)(212.36554443,46.38)
\curveto(212.31554099,46.37000013)(212.27054103,46.36500014)(212.23054443,46.365)
\lineto(212.12554443,46.365)
\curveto(212.01554129,46.34500015)(211.91054139,46.34500015)(211.81054443,46.365)
\lineto(211.67554443,46.365)
\curveto(211.62554168,46.37500012)(211.57554173,46.38000012)(211.52554443,46.38)
\curveto(211.47554183,46.38000012)(211.43054187,46.39000011)(211.39054443,46.41)
\curveto(211.35054195,46.42000008)(211.31554199,46.42500008)(211.28554443,46.425)
\curveto(211.26554204,46.41500008)(211.24054206,46.41500008)(211.21054443,46.425)
\lineto(210.97054443,46.485)
\curveto(210.89054241,46.49500001)(210.81554249,46.51499998)(210.74554443,46.545)
\curveto(210.44554286,46.67499982)(210.2005431,46.81999968)(210.01054443,46.98)
\curveto(209.83054347,47.14999935)(209.68054362,47.38499912)(209.56054443,47.685)
\curveto(209.47054383,47.90499859)(209.42554388,48.16999833)(209.42554443,48.48)
\lineto(209.42554443,48.795)
\curveto(209.43554387,48.84499765)(209.44054386,48.89499761)(209.44054443,48.945)
\lineto(209.47054443,49.125)
\lineto(209.59054443,49.455)
\curveto(209.63054367,49.56499694)(209.68054362,49.66499684)(209.74054443,49.755)
\curveto(209.92054338,50.04499645)(210.16554314,50.25999624)(210.47554443,50.4)
\curveto(210.78554252,50.53999596)(211.12554218,50.66499584)(211.49554443,50.775)
\curveto(211.63554167,50.81499568)(211.78054152,50.84499565)(211.93054443,50.865)
\curveto(212.08054122,50.88499561)(212.23054107,50.90999559)(212.38054443,50.94)
\curveto(212.45054085,50.95999554)(212.51554079,50.96999553)(212.57554443,50.97)
\curveto(212.64554066,50.96999553)(212.72054058,50.97999552)(212.80054443,51)
\curveto(212.87054043,51.01999548)(212.94054036,51.02999547)(213.01054443,51.03)
\curveto(213.08054022,51.03999546)(213.15554015,51.05499544)(213.23554443,51.075)
\curveto(213.48553982,51.13499537)(213.72053958,51.18499531)(213.94054443,51.225)
\curveto(214.16053914,51.27499522)(214.33553897,51.38999511)(214.46554443,51.57)
\curveto(214.52553878,51.64999485)(214.57553873,51.74999475)(214.61554443,51.87)
\curveto(214.65553865,51.9999945)(214.65553865,52.13999436)(214.61554443,52.29)
\curveto(214.55553875,52.52999397)(214.46553884,52.71999378)(214.34554443,52.86)
\curveto(214.23553907,52.9999935)(214.07553923,53.10999339)(213.86554443,53.19)
\curveto(213.74553956,53.23999326)(213.6005397,53.27499323)(213.43054443,53.295)
\curveto(213.27054003,53.31499318)(213.1005402,53.32499317)(212.92054443,53.325)
\curveto(212.74054056,53.32499317)(212.56554074,53.31499318)(212.39554443,53.295)
\curveto(212.22554108,53.27499323)(212.08054122,53.24499325)(211.96054443,53.205)
\curveto(211.79054151,53.14499335)(211.62554168,53.05999344)(211.46554443,52.95)
\curveto(211.38554192,52.88999361)(211.31054199,52.80999369)(211.24054443,52.71)
\curveto(211.18054212,52.61999388)(211.12554218,52.51999398)(211.07554443,52.41)
\curveto(211.04554226,52.32999417)(211.01554229,52.24499425)(210.98554443,52.155)
\curveto(210.96554234,52.06499444)(210.92054238,51.99499451)(210.85054443,51.945)
\curveto(210.81054249,51.91499458)(210.74054256,51.88999461)(210.64054443,51.87)
\curveto(210.55054275,51.85999464)(210.45554285,51.85499464)(210.35554443,51.855)
\curveto(210.25554305,51.85499464)(210.15554315,51.85999464)(210.05554443,51.87)
\curveto(209.96554334,51.88999461)(209.9005434,51.91499458)(209.86054443,51.945)
\curveto(209.82054348,51.97499452)(209.79054351,52.02499447)(209.77054443,52.095)
\curveto(209.75054355,52.16499434)(209.75054355,52.23999426)(209.77054443,52.32)
\curveto(209.8005435,52.44999405)(209.83054347,52.56999393)(209.86054443,52.68)
\curveto(209.9005434,52.7999937)(209.94554336,52.91499358)(209.99554443,53.025)
\curveto(210.18554312,53.37499313)(210.42554288,53.64499285)(210.71554443,53.835)
\curveto(211.0055423,54.03499247)(211.36554194,54.19499231)(211.79554443,54.315)
\curveto(211.89554141,54.33499217)(211.99554131,54.34999215)(212.09554443,54.36)
\curveto(212.2055411,54.36999213)(212.31554099,54.38499211)(212.42554443,54.405)
\curveto(212.46554084,54.41499208)(212.53054077,54.41499208)(212.62054443,54.405)
\curveto(212.71054059,54.4049921)(212.76554054,54.41499208)(212.78554443,54.435)
\curveto(213.48553982,54.44499206)(214.09553921,54.36499214)(214.61554443,54.195)
\curveto(215.13553817,54.02499247)(215.5005378,53.6999928)(215.71054443,53.22)
\curveto(215.8005375,53.01999348)(215.85053745,52.78499371)(215.86054443,52.515)
\curveto(215.88053742,52.25499424)(215.89053741,51.97999452)(215.89054443,51.69)
\lineto(215.89054443,48.375)
\curveto(215.89053741,48.23499827)(215.89553741,48.0999984)(215.90554443,47.97)
\curveto(215.91553739,47.83999866)(215.94553736,47.73499877)(215.99554443,47.655)
\curveto(216.04553726,47.58499892)(216.11053719,47.53499896)(216.19054443,47.505)
\curveto(216.28053702,47.46499904)(216.36553694,47.43499906)(216.44554443,47.415)
\curveto(216.52553678,47.40499909)(216.58553672,47.35999914)(216.62554443,47.28)
\curveto(216.64553666,47.24999925)(216.65553665,47.21999928)(216.65554443,47.19)
\curveto(216.65553665,47.15999934)(216.66053664,47.11999938)(216.67054443,47.07)
\moveto(214.52554443,48.735)
\curveto(214.58553872,48.87499762)(214.61553869,49.03499747)(214.61554443,49.215)
\curveto(214.62553868,49.40499709)(214.63053867,49.5999969)(214.63054443,49.8)
\curveto(214.63053867,49.90999659)(214.62553868,50.00999649)(214.61554443,50.1)
\curveto(214.6055387,50.18999631)(214.56553874,50.25999624)(214.49554443,50.31)
\curveto(214.46553884,50.32999617)(214.39553891,50.33999616)(214.28554443,50.34)
\curveto(214.26553904,50.31999618)(214.23053907,50.30999619)(214.18054443,50.31)
\curveto(214.13053917,50.30999619)(214.08553922,50.2999962)(214.04554443,50.28)
\curveto(213.96553934,50.25999624)(213.87553943,50.23999626)(213.77554443,50.22)
\lineto(213.47554443,50.16)
\curveto(213.44553986,50.15999634)(213.41053989,50.15499634)(213.37054443,50.145)
\lineto(213.26554443,50.145)
\curveto(213.11554019,50.1049964)(212.95054035,50.07999642)(212.77054443,50.07)
\curveto(212.6005407,50.06999643)(212.44054086,50.04999645)(212.29054443,50.01)
\curveto(212.21054109,49.98999651)(212.13554117,49.96999653)(212.06554443,49.95)
\curveto(212.0055413,49.93999656)(211.93554137,49.92499658)(211.85554443,49.905)
\curveto(211.69554161,49.85499665)(211.54554176,49.78999671)(211.40554443,49.71)
\curveto(211.26554204,49.63999686)(211.14554216,49.54999695)(211.04554443,49.44)
\curveto(210.94554236,49.32999717)(210.87054243,49.19499731)(210.82054443,49.035)
\curveto(210.77054253,48.88499761)(210.75054255,48.6999978)(210.76054443,48.48)
\curveto(210.76054254,48.37999812)(210.77554253,48.28499821)(210.80554443,48.195)
\curveto(210.84554246,48.11499838)(210.89054241,48.03999846)(210.94054443,47.97)
\curveto(211.02054228,47.85999864)(211.12554218,47.76499874)(211.25554443,47.685)
\curveto(211.38554192,47.61499888)(211.52554178,47.55499895)(211.67554443,47.505)
\curveto(211.72554158,47.49499901)(211.77554153,47.48999901)(211.82554443,47.49)
\curveto(211.87554143,47.48999901)(211.92554138,47.48499902)(211.97554443,47.475)
\curveto(212.04554126,47.45499905)(212.13054117,47.43999906)(212.23054443,47.43)
\curveto(212.34054096,47.42999907)(212.43054087,47.43999906)(212.50054443,47.46)
\curveto(212.56054074,47.47999902)(212.62054068,47.48499902)(212.68054443,47.475)
\curveto(212.74054056,47.47499902)(212.8005405,47.48499902)(212.86054443,47.505)
\curveto(212.94054036,47.52499898)(213.01554029,47.53999896)(213.08554443,47.55)
\curveto(213.16554014,47.55999894)(213.24054006,47.57999892)(213.31054443,47.61)
\curveto(213.6005397,47.72999877)(213.84553946,47.87499862)(214.04554443,48.045)
\curveto(214.25553905,48.21499828)(214.41553889,48.44499805)(214.52554443,48.735)
}
}
{
\newrgbcolor{curcolor}{0 0 0}
\pscustom[linestyle=none,fillstyle=solid,fillcolor=curcolor]
{
\newpath
\moveto(221.48718506,54.42)
\curveto(221.71718027,54.41999208)(221.84718014,54.35999214)(221.87718506,54.24)
\curveto(221.90718008,54.12999237)(221.92218006,53.96499254)(221.92218506,53.745)
\lineto(221.92218506,53.46)
\curveto(221.92218006,53.36999313)(221.89718009,53.29499321)(221.84718506,53.235)
\curveto(221.7871802,53.15499334)(221.70218028,53.10999339)(221.59218506,53.1)
\curveto(221.4821805,53.0999934)(221.37218061,53.08499341)(221.26218506,53.055)
\curveto(221.12218086,53.02499347)(220.987181,52.99499351)(220.85718506,52.965)
\curveto(220.73718125,52.93499357)(220.62218136,52.89499361)(220.51218506,52.845)
\curveto(220.22218176,52.71499378)(219.987182,52.53499397)(219.80718506,52.305)
\curveto(219.62718236,52.08499441)(219.47218251,51.82999467)(219.34218506,51.54)
\curveto(219.30218268,51.42999507)(219.27218271,51.31499518)(219.25218506,51.195)
\curveto(219.23218275,51.08499541)(219.20718278,50.96999553)(219.17718506,50.85)
\curveto(219.16718282,50.7999957)(219.16218282,50.74999575)(219.16218506,50.7)
\curveto(219.17218281,50.64999585)(219.17218281,50.5999959)(219.16218506,50.55)
\curveto(219.13218285,50.42999607)(219.11718287,50.28999621)(219.11718506,50.13)
\curveto(219.12718286,49.97999652)(219.13218285,49.83499667)(219.13218506,49.695)
\lineto(219.13218506,47.85)
\lineto(219.13218506,47.505)
\curveto(219.13218285,47.38499912)(219.12718286,47.26999923)(219.11718506,47.16)
\curveto(219.10718288,47.04999945)(219.10218288,46.95499955)(219.10218506,46.875)
\curveto(219.11218287,46.79499971)(219.09218289,46.72499978)(219.04218506,46.665)
\curveto(218.99218299,46.59499991)(218.91218307,46.55499995)(218.80218506,46.545)
\curveto(218.70218328,46.53499996)(218.59218339,46.52999997)(218.47218506,46.53)
\lineto(218.20218506,46.53)
\curveto(218.15218383,46.54999995)(218.10218388,46.56499994)(218.05218506,46.575)
\curveto(218.01218397,46.59499991)(217.982184,46.61999988)(217.96218506,46.65)
\curveto(217.91218407,46.71999978)(217.8821841,46.80499969)(217.87218506,46.905)
\lineto(217.87218506,47.235)
\lineto(217.87218506,48.39)
\lineto(217.87218506,52.545)
\lineto(217.87218506,53.58)
\lineto(217.87218506,53.88)
\curveto(217.8821841,53.97999252)(217.91218407,54.06499244)(217.96218506,54.135)
\curveto(217.99218399,54.17499233)(218.04218394,54.2049923)(218.11218506,54.225)
\curveto(218.19218379,54.24499225)(218.27718371,54.25499224)(218.36718506,54.255)
\curveto(218.45718353,54.26499224)(218.54718344,54.26499224)(218.63718506,54.255)
\curveto(218.72718326,54.24499225)(218.79718319,54.22999227)(218.84718506,54.21)
\curveto(218.92718306,54.17999232)(218.97718301,54.11999238)(218.99718506,54.03)
\curveto(219.02718296,53.94999255)(219.04218294,53.85999264)(219.04218506,53.76)
\lineto(219.04218506,53.46)
\curveto(219.04218294,53.35999314)(219.06218292,53.26999323)(219.10218506,53.19)
\curveto(219.11218287,53.16999333)(219.12218286,53.15499334)(219.13218506,53.145)
\lineto(219.17718506,53.1)
\curveto(219.2871827,53.0999934)(219.37718261,53.14499335)(219.44718506,53.235)
\curveto(219.51718247,53.33499317)(219.57718241,53.41499308)(219.62718506,53.475)
\lineto(219.71718506,53.565)
\curveto(219.80718218,53.67499283)(219.93218205,53.78999271)(220.09218506,53.91)
\curveto(220.25218173,54.02999247)(220.40218158,54.11999238)(220.54218506,54.18)
\curveto(220.63218135,54.22999227)(220.72718126,54.26499224)(220.82718506,54.285)
\curveto(220.92718106,54.31499218)(221.03218095,54.34499215)(221.14218506,54.375)
\curveto(221.20218078,54.38499211)(221.26218072,54.38999211)(221.32218506,54.39)
\curveto(221.3821806,54.3999921)(221.43718055,54.40999209)(221.48718506,54.42)
}
}
{
\newrgbcolor{curcolor}{0 0 0}
\pscustom[linestyle=none,fillstyle=solid,fillcolor=curcolor]
{
\newpath
\moveto(226.49695068,54.42)
\curveto(226.72694589,54.41999208)(226.85694576,54.35999214)(226.88695068,54.24)
\curveto(226.9169457,54.12999237)(226.93194569,53.96499254)(226.93195068,53.745)
\lineto(226.93195068,53.46)
\curveto(226.93194569,53.36999313)(226.90694571,53.29499321)(226.85695068,53.235)
\curveto(226.79694582,53.15499334)(226.71194591,53.10999339)(226.60195068,53.1)
\curveto(226.49194613,53.0999934)(226.38194624,53.08499341)(226.27195068,53.055)
\curveto(226.13194649,53.02499347)(225.99694662,52.99499351)(225.86695068,52.965)
\curveto(225.74694687,52.93499357)(225.63194699,52.89499361)(225.52195068,52.845)
\curveto(225.23194739,52.71499378)(224.99694762,52.53499397)(224.81695068,52.305)
\curveto(224.63694798,52.08499441)(224.48194814,51.82999467)(224.35195068,51.54)
\curveto(224.31194831,51.42999507)(224.28194834,51.31499518)(224.26195068,51.195)
\curveto(224.24194838,51.08499541)(224.2169484,50.96999553)(224.18695068,50.85)
\curveto(224.17694844,50.7999957)(224.17194845,50.74999575)(224.17195068,50.7)
\curveto(224.18194844,50.64999585)(224.18194844,50.5999959)(224.17195068,50.55)
\curveto(224.14194848,50.42999607)(224.12694849,50.28999621)(224.12695068,50.13)
\curveto(224.13694848,49.97999652)(224.14194848,49.83499667)(224.14195068,49.695)
\lineto(224.14195068,47.85)
\lineto(224.14195068,47.505)
\curveto(224.14194848,47.38499912)(224.13694848,47.26999923)(224.12695068,47.16)
\curveto(224.1169485,47.04999945)(224.11194851,46.95499955)(224.11195068,46.875)
\curveto(224.1219485,46.79499971)(224.10194852,46.72499978)(224.05195068,46.665)
\curveto(224.00194862,46.59499991)(223.9219487,46.55499995)(223.81195068,46.545)
\curveto(223.71194891,46.53499996)(223.60194902,46.52999997)(223.48195068,46.53)
\lineto(223.21195068,46.53)
\curveto(223.16194946,46.54999995)(223.11194951,46.56499994)(223.06195068,46.575)
\curveto(223.0219496,46.59499991)(222.99194963,46.61999988)(222.97195068,46.65)
\curveto(222.9219497,46.71999978)(222.89194973,46.80499969)(222.88195068,46.905)
\lineto(222.88195068,47.235)
\lineto(222.88195068,48.39)
\lineto(222.88195068,52.545)
\lineto(222.88195068,53.58)
\lineto(222.88195068,53.88)
\curveto(222.89194973,53.97999252)(222.9219497,54.06499244)(222.97195068,54.135)
\curveto(223.00194962,54.17499233)(223.05194957,54.2049923)(223.12195068,54.225)
\curveto(223.20194942,54.24499225)(223.28694933,54.25499224)(223.37695068,54.255)
\curveto(223.46694915,54.26499224)(223.55694906,54.26499224)(223.64695068,54.255)
\curveto(223.73694888,54.24499225)(223.80694881,54.22999227)(223.85695068,54.21)
\curveto(223.93694868,54.17999232)(223.98694863,54.11999238)(224.00695068,54.03)
\curveto(224.03694858,53.94999255)(224.05194857,53.85999264)(224.05195068,53.76)
\lineto(224.05195068,53.46)
\curveto(224.05194857,53.35999314)(224.07194855,53.26999323)(224.11195068,53.19)
\curveto(224.1219485,53.16999333)(224.13194849,53.15499334)(224.14195068,53.145)
\lineto(224.18695068,53.1)
\curveto(224.29694832,53.0999934)(224.38694823,53.14499335)(224.45695068,53.235)
\curveto(224.52694809,53.33499317)(224.58694803,53.41499308)(224.63695068,53.475)
\lineto(224.72695068,53.565)
\curveto(224.8169478,53.67499283)(224.94194768,53.78999271)(225.10195068,53.91)
\curveto(225.26194736,54.02999247)(225.41194721,54.11999238)(225.55195068,54.18)
\curveto(225.64194698,54.22999227)(225.73694688,54.26499224)(225.83695068,54.285)
\curveto(225.93694668,54.31499218)(226.04194658,54.34499215)(226.15195068,54.375)
\curveto(226.21194641,54.38499211)(226.27194635,54.38999211)(226.33195068,54.39)
\curveto(226.39194623,54.3999921)(226.44694617,54.40999209)(226.49695068,54.42)
}
}
{
\newrgbcolor{curcolor}{0 0 0}
\pscustom[linestyle=none,fillstyle=solid,fillcolor=curcolor]
{
\newpath
\moveto(234.61171631,50.685)
\curveto(234.63170862,50.58499591)(234.63170862,50.46999603)(234.61171631,50.34)
\curveto(234.60170865,50.21999628)(234.57170868,50.13499637)(234.52171631,50.085)
\curveto(234.47170878,50.04499645)(234.39670886,50.01499648)(234.29671631,49.995)
\curveto(234.20670905,49.98499651)(234.10170915,49.97999652)(233.98171631,49.98)
\lineto(233.62171631,49.98)
\curveto(233.50170975,49.98999651)(233.39670986,49.99499651)(233.30671631,49.995)
\lineto(229.46671631,49.995)
\curveto(229.38671387,49.99499651)(229.30671395,49.98999651)(229.22671631,49.98)
\curveto(229.14671411,49.97999652)(229.08171417,49.96499654)(229.03171631,49.935)
\curveto(228.99171426,49.91499658)(228.9517143,49.87499662)(228.91171631,49.815)
\curveto(228.89171436,49.78499671)(228.87171438,49.73999676)(228.85171631,49.68)
\curveto(228.83171442,49.62999687)(228.83171442,49.57999692)(228.85171631,49.53)
\curveto(228.86171439,49.47999702)(228.86671439,49.43499707)(228.86671631,49.395)
\curveto(228.86671439,49.35499715)(228.87171438,49.31499718)(228.88171631,49.275)
\curveto(228.90171435,49.19499731)(228.92171433,49.10999739)(228.94171631,49.02)
\curveto(228.96171429,48.93999756)(228.99171426,48.85999764)(229.03171631,48.78)
\curveto(229.26171399,48.23999826)(229.64171361,47.85499865)(230.17171631,47.625)
\curveto(230.23171302,47.59499891)(230.29671296,47.56999893)(230.36671631,47.55)
\lineto(230.57671631,47.49)
\curveto(230.60671265,47.47999902)(230.6567126,47.47499902)(230.72671631,47.475)
\curveto(230.86671239,47.43499906)(231.0517122,47.41499908)(231.28171631,47.415)
\curveto(231.51171174,47.41499908)(231.69671156,47.43499906)(231.83671631,47.475)
\curveto(231.97671128,47.51499898)(232.10171115,47.55499895)(232.21171631,47.595)
\curveto(232.33171092,47.64499885)(232.44171081,47.70499879)(232.54171631,47.775)
\curveto(232.6517106,47.84499865)(232.74671051,47.92499858)(232.82671631,48.015)
\curveto(232.90671035,48.11499838)(232.97671028,48.21999828)(233.03671631,48.33)
\curveto(233.09671016,48.42999807)(233.14671011,48.53499797)(233.18671631,48.645)
\curveto(233.23671002,48.75499775)(233.31670994,48.83499767)(233.42671631,48.885)
\curveto(233.46670979,48.90499759)(233.53170972,48.91999758)(233.62171631,48.93)
\curveto(233.71170954,48.93999756)(233.80170945,48.93999756)(233.89171631,48.93)
\curveto(233.98170927,48.92999757)(234.06670919,48.92499758)(234.14671631,48.915)
\curveto(234.22670903,48.90499759)(234.28170897,48.88499761)(234.31171631,48.855)
\curveto(234.41170884,48.78499771)(234.43670882,48.66999783)(234.38671631,48.51)
\curveto(234.30670895,48.23999826)(234.20170905,47.9999985)(234.07171631,47.79)
\curveto(233.87170938,47.46999903)(233.64170961,47.20499929)(233.38171631,46.995)
\curveto(233.13171012,46.79499971)(232.81171044,46.62999987)(232.42171631,46.5)
\curveto(232.32171093,46.46000004)(232.22171103,46.43500006)(232.12171631,46.425)
\curveto(232.02171123,46.40500009)(231.91671134,46.38500012)(231.80671631,46.365)
\curveto(231.7567115,46.35500015)(231.70671155,46.35000015)(231.65671631,46.35)
\curveto(231.61671164,46.35000015)(231.57171168,46.34500015)(231.52171631,46.335)
\lineto(231.37171631,46.335)
\curveto(231.32171193,46.32500018)(231.26171199,46.32000018)(231.19171631,46.32)
\curveto(231.13171212,46.32000018)(231.08171217,46.32500018)(231.04171631,46.335)
\lineto(230.90671631,46.335)
\curveto(230.8567124,46.34500015)(230.81171244,46.35000015)(230.77171631,46.35)
\curveto(230.73171252,46.35000015)(230.69171256,46.35500015)(230.65171631,46.365)
\curveto(230.60171265,46.37500012)(230.54671271,46.38500012)(230.48671631,46.395)
\curveto(230.42671283,46.39500011)(230.37171288,46.4000001)(230.32171631,46.41)
\curveto(230.23171302,46.43000007)(230.14171311,46.45500005)(230.05171631,46.485)
\curveto(229.96171329,46.50499999)(229.87671338,46.52999997)(229.79671631,46.56)
\curveto(229.7567135,46.57999992)(229.72171353,46.58999991)(229.69171631,46.59)
\curveto(229.66171359,46.5999999)(229.62671363,46.61499988)(229.58671631,46.635)
\curveto(229.43671382,46.70499979)(229.27671398,46.78999971)(229.10671631,46.89)
\curveto(228.81671444,47.07999942)(228.56671469,47.30999919)(228.35671631,47.58)
\curveto(228.1567151,47.85999864)(227.98671527,48.16999833)(227.84671631,48.51)
\curveto(227.79671546,48.61999788)(227.7567155,48.73499777)(227.72671631,48.855)
\curveto(227.70671555,48.97499752)(227.67671558,49.09499741)(227.63671631,49.215)
\curveto(227.62671563,49.25499725)(227.62171563,49.28999721)(227.62171631,49.32)
\curveto(227.62171563,49.34999715)(227.61671564,49.38999711)(227.60671631,49.44)
\curveto(227.58671567,49.51999698)(227.57171568,49.60499689)(227.56171631,49.695)
\curveto(227.5517157,49.78499671)(227.53671572,49.87499662)(227.51671631,49.965)
\lineto(227.51671631,50.175)
\curveto(227.50671575,50.21499628)(227.49671576,50.26999623)(227.48671631,50.34)
\curveto(227.48671577,50.41999608)(227.49171576,50.48499601)(227.50171631,50.535)
\lineto(227.50171631,50.7)
\curveto(227.52171573,50.74999575)(227.52671573,50.7999957)(227.51671631,50.85)
\curveto(227.51671574,50.90999559)(227.52171573,50.96499554)(227.53171631,51.015)
\curveto(227.57171568,51.17499532)(227.60171565,51.33499517)(227.62171631,51.495)
\curveto(227.6517156,51.65499484)(227.69671556,51.8049947)(227.75671631,51.945)
\curveto(227.80671545,52.05499444)(227.8517154,52.16499434)(227.89171631,52.275)
\curveto(227.94171531,52.39499411)(227.99671526,52.50999399)(228.05671631,52.62)
\curveto(228.27671498,52.96999353)(228.52671473,53.26999323)(228.80671631,53.52)
\curveto(229.08671417,53.77999272)(229.43171382,53.99499251)(229.84171631,54.165)
\curveto(229.96171329,54.21499228)(230.08171317,54.24999225)(230.20171631,54.27)
\curveto(230.33171292,54.2999922)(230.46671279,54.32999217)(230.60671631,54.36)
\curveto(230.6567126,54.36999213)(230.70171255,54.37499213)(230.74171631,54.375)
\curveto(230.78171247,54.38499211)(230.82671243,54.38999211)(230.87671631,54.39)
\curveto(230.89671236,54.3999921)(230.92171233,54.3999921)(230.95171631,54.39)
\curveto(230.98171227,54.37999212)(231.00671225,54.38499211)(231.02671631,54.405)
\curveto(231.44671181,54.41499208)(231.81171144,54.36999213)(232.12171631,54.27)
\curveto(232.43171082,54.17999232)(232.71171054,54.05499244)(232.96171631,53.895)
\curveto(233.01171024,53.87499263)(233.0517102,53.84499265)(233.08171631,53.805)
\curveto(233.11171014,53.77499273)(233.14671011,53.74999275)(233.18671631,53.73)
\curveto(233.26670999,53.66999283)(233.34670991,53.5999929)(233.42671631,53.52)
\curveto(233.51670974,53.43999306)(233.59170966,53.35999314)(233.65171631,53.28)
\curveto(233.81170944,53.06999343)(233.94670931,52.86999363)(234.05671631,52.68)
\curveto(234.12670913,52.56999393)(234.18170907,52.44999405)(234.22171631,52.32)
\curveto(234.26170899,52.18999431)(234.30670895,52.05999444)(234.35671631,51.93)
\curveto(234.40670885,51.7999947)(234.44170881,51.66499484)(234.46171631,51.525)
\curveto(234.49170876,51.38499511)(234.52670873,51.24499525)(234.56671631,51.105)
\curveto(234.57670868,51.03499547)(234.58170867,50.96499554)(234.58171631,50.895)
\lineto(234.61171631,50.685)
\moveto(233.15671631,51.195)
\curveto(233.18671007,51.23499527)(233.21171004,51.28499521)(233.23171631,51.345)
\curveto(233.25171,51.41499508)(233.25171,51.48499501)(233.23171631,51.555)
\curveto(233.17171008,51.77499472)(233.08671017,51.97999452)(232.97671631,52.17)
\curveto(232.83671042,52.3999941)(232.68171057,52.59499391)(232.51171631,52.755)
\curveto(232.34171091,52.91499358)(232.12171113,53.04999345)(231.85171631,53.16)
\curveto(231.78171147,53.17999332)(231.71171154,53.19499331)(231.64171631,53.205)
\curveto(231.57171168,53.22499327)(231.49671176,53.24499325)(231.41671631,53.265)
\curveto(231.33671192,53.28499321)(231.251712,53.29499321)(231.16171631,53.295)
\lineto(230.90671631,53.295)
\curveto(230.87671238,53.27499323)(230.84171241,53.26499324)(230.80171631,53.265)
\curveto(230.76171249,53.27499323)(230.72671253,53.27499323)(230.69671631,53.265)
\lineto(230.45671631,53.205)
\curveto(230.38671287,53.19499331)(230.31671294,53.17999332)(230.24671631,53.16)
\curveto(229.9567133,53.03999346)(229.72171353,52.88999361)(229.54171631,52.71)
\curveto(229.37171388,52.52999397)(229.21671404,52.3049942)(229.07671631,52.035)
\curveto(229.04671421,51.98499451)(229.01671424,51.91999458)(228.98671631,51.84)
\curveto(228.9567143,51.76999473)(228.93171432,51.68999481)(228.91171631,51.6)
\curveto(228.89171436,51.50999499)(228.88671437,51.42499508)(228.89671631,51.345)
\curveto(228.90671435,51.26499524)(228.94171431,51.2049953)(229.00171631,51.165)
\curveto(229.08171417,51.1049954)(229.21671404,51.07499542)(229.40671631,51.075)
\curveto(229.60671365,51.08499541)(229.77671348,51.08999541)(229.91671631,51.09)
\lineto(232.19671631,51.09)
\curveto(232.34671091,51.08999541)(232.52671073,51.08499541)(232.73671631,51.075)
\curveto(232.94671031,51.07499542)(233.08671017,51.11499538)(233.15671631,51.195)
}
}
{
\newrgbcolor{curcolor}{0 0 0}
\pscustom[linestyle=none,fillstyle=solid,fillcolor=curcolor]
{
\newpath
\moveto(239.56335693,54.42)
\curveto(239.79335214,54.41999208)(239.92335201,54.35999214)(239.95335693,54.24)
\curveto(239.98335195,54.12999237)(239.99835194,53.96499254)(239.99835693,53.745)
\lineto(239.99835693,53.46)
\curveto(239.99835194,53.36999313)(239.97335196,53.29499321)(239.92335693,53.235)
\curveto(239.86335207,53.15499334)(239.77835216,53.10999339)(239.66835693,53.1)
\curveto(239.55835238,53.0999934)(239.44835249,53.08499341)(239.33835693,53.055)
\curveto(239.19835274,53.02499347)(239.06335287,52.99499351)(238.93335693,52.965)
\curveto(238.81335312,52.93499357)(238.69835324,52.89499361)(238.58835693,52.845)
\curveto(238.29835364,52.71499378)(238.06335387,52.53499397)(237.88335693,52.305)
\curveto(237.70335423,52.08499441)(237.54835439,51.82999467)(237.41835693,51.54)
\curveto(237.37835456,51.42999507)(237.34835459,51.31499518)(237.32835693,51.195)
\curveto(237.30835463,51.08499541)(237.28335465,50.96999553)(237.25335693,50.85)
\curveto(237.24335469,50.7999957)(237.2383547,50.74999575)(237.23835693,50.7)
\curveto(237.24835469,50.64999585)(237.24835469,50.5999959)(237.23835693,50.55)
\curveto(237.20835473,50.42999607)(237.19335474,50.28999621)(237.19335693,50.13)
\curveto(237.20335473,49.97999652)(237.20835473,49.83499667)(237.20835693,49.695)
\lineto(237.20835693,47.85)
\lineto(237.20835693,47.505)
\curveto(237.20835473,47.38499912)(237.20335473,47.26999923)(237.19335693,47.16)
\curveto(237.18335475,47.04999945)(237.17835476,46.95499955)(237.17835693,46.875)
\curveto(237.18835475,46.79499971)(237.16835477,46.72499978)(237.11835693,46.665)
\curveto(237.06835487,46.59499991)(236.98835495,46.55499995)(236.87835693,46.545)
\curveto(236.77835516,46.53499996)(236.66835527,46.52999997)(236.54835693,46.53)
\lineto(236.27835693,46.53)
\curveto(236.22835571,46.54999995)(236.17835576,46.56499994)(236.12835693,46.575)
\curveto(236.08835585,46.59499991)(236.05835588,46.61999988)(236.03835693,46.65)
\curveto(235.98835595,46.71999978)(235.95835598,46.80499969)(235.94835693,46.905)
\lineto(235.94835693,47.235)
\lineto(235.94835693,48.39)
\lineto(235.94835693,52.545)
\lineto(235.94835693,53.58)
\lineto(235.94835693,53.88)
\curveto(235.95835598,53.97999252)(235.98835595,54.06499244)(236.03835693,54.135)
\curveto(236.06835587,54.17499233)(236.11835582,54.2049923)(236.18835693,54.225)
\curveto(236.26835567,54.24499225)(236.35335558,54.25499224)(236.44335693,54.255)
\curveto(236.5333554,54.26499224)(236.62335531,54.26499224)(236.71335693,54.255)
\curveto(236.80335513,54.24499225)(236.87335506,54.22999227)(236.92335693,54.21)
\curveto(237.00335493,54.17999232)(237.05335488,54.11999238)(237.07335693,54.03)
\curveto(237.10335483,53.94999255)(237.11835482,53.85999264)(237.11835693,53.76)
\lineto(237.11835693,53.46)
\curveto(237.11835482,53.35999314)(237.1383548,53.26999323)(237.17835693,53.19)
\curveto(237.18835475,53.16999333)(237.19835474,53.15499334)(237.20835693,53.145)
\lineto(237.25335693,53.1)
\curveto(237.36335457,53.0999934)(237.45335448,53.14499335)(237.52335693,53.235)
\curveto(237.59335434,53.33499317)(237.65335428,53.41499308)(237.70335693,53.475)
\lineto(237.79335693,53.565)
\curveto(237.88335405,53.67499283)(238.00835393,53.78999271)(238.16835693,53.91)
\curveto(238.32835361,54.02999247)(238.47835346,54.11999238)(238.61835693,54.18)
\curveto(238.70835323,54.22999227)(238.80335313,54.26499224)(238.90335693,54.285)
\curveto(239.00335293,54.31499218)(239.10835283,54.34499215)(239.21835693,54.375)
\curveto(239.27835266,54.38499211)(239.3383526,54.38999211)(239.39835693,54.39)
\curveto(239.45835248,54.3999921)(239.51335242,54.40999209)(239.56335693,54.42)
}
}
{
\newrgbcolor{curcolor}{0 0 0}
\pscustom[linestyle=none,fillstyle=solid,fillcolor=curcolor]
{
\newpath
\moveto(247.81312256,47.07)
\curveto(247.84311473,46.90999959)(247.82811474,46.77499972)(247.76812256,46.665)
\curveto(247.70811486,46.56499994)(247.62811494,46.49000001)(247.52812256,46.44)
\curveto(247.47811509,46.42000008)(247.42311515,46.41000009)(247.36312256,46.41)
\curveto(247.31311526,46.41000009)(247.25811531,46.4000001)(247.19812256,46.38)
\curveto(246.97811559,46.33000017)(246.75811581,46.34500015)(246.53812256,46.425)
\curveto(246.32811624,46.49500001)(246.18311639,46.58499992)(246.10312256,46.695)
\curveto(246.05311652,46.76499974)(246.00811656,46.84499965)(245.96812256,46.935)
\curveto(245.92811664,47.03499946)(245.87811669,47.11499938)(245.81812256,47.175)
\curveto(245.79811677,47.19499931)(245.7731168,47.21499928)(245.74312256,47.235)
\curveto(245.72311685,47.25499925)(245.69311688,47.25999924)(245.65312256,47.25)
\curveto(245.54311703,47.21999928)(245.43811713,47.16499934)(245.33812256,47.085)
\curveto(245.24811732,47.00499949)(245.15811741,46.93499956)(245.06812256,46.875)
\curveto(244.93811763,46.79499971)(244.79811777,46.71999978)(244.64812256,46.65)
\curveto(244.49811807,46.58999991)(244.33811823,46.53499996)(244.16812256,46.485)
\curveto(244.0681185,46.45500005)(243.95811861,46.43500006)(243.83812256,46.425)
\curveto(243.72811884,46.41500008)(243.61811895,46.4000001)(243.50812256,46.38)
\curveto(243.45811911,46.37000013)(243.41311916,46.36500014)(243.37312256,46.365)
\lineto(243.26812256,46.365)
\curveto(243.15811941,46.34500015)(243.05311952,46.34500015)(242.95312256,46.365)
\lineto(242.81812256,46.365)
\curveto(242.7681198,46.37500012)(242.71811985,46.38000012)(242.66812256,46.38)
\curveto(242.61811995,46.38000012)(242.57312,46.39000011)(242.53312256,46.41)
\curveto(242.49312008,46.42000008)(242.45812011,46.42500008)(242.42812256,46.425)
\curveto(242.40812016,46.41500008)(242.38312019,46.41500008)(242.35312256,46.425)
\lineto(242.11312256,46.485)
\curveto(242.03312054,46.49500001)(241.95812061,46.51499998)(241.88812256,46.545)
\curveto(241.58812098,46.67499982)(241.34312123,46.81999968)(241.15312256,46.98)
\curveto(240.9731216,47.14999935)(240.82312175,47.38499912)(240.70312256,47.685)
\curveto(240.61312196,47.90499859)(240.568122,48.16999833)(240.56812256,48.48)
\lineto(240.56812256,48.795)
\curveto(240.57812199,48.84499765)(240.58312199,48.89499761)(240.58312256,48.945)
\lineto(240.61312256,49.125)
\lineto(240.73312256,49.455)
\curveto(240.7731218,49.56499694)(240.82312175,49.66499684)(240.88312256,49.755)
\curveto(241.06312151,50.04499645)(241.30812126,50.25999624)(241.61812256,50.4)
\curveto(241.92812064,50.53999596)(242.2681203,50.66499584)(242.63812256,50.775)
\curveto(242.77811979,50.81499568)(242.92311965,50.84499565)(243.07312256,50.865)
\curveto(243.22311935,50.88499561)(243.3731192,50.90999559)(243.52312256,50.94)
\curveto(243.59311898,50.95999554)(243.65811891,50.96999553)(243.71812256,50.97)
\curveto(243.78811878,50.96999553)(243.86311871,50.97999552)(243.94312256,51)
\curveto(244.01311856,51.01999548)(244.08311849,51.02999547)(244.15312256,51.03)
\curveto(244.22311835,51.03999546)(244.29811827,51.05499544)(244.37812256,51.075)
\curveto(244.62811794,51.13499537)(244.86311771,51.18499531)(245.08312256,51.225)
\curveto(245.30311727,51.27499522)(245.47811709,51.38999511)(245.60812256,51.57)
\curveto(245.6681169,51.64999485)(245.71811685,51.74999475)(245.75812256,51.87)
\curveto(245.79811677,51.9999945)(245.79811677,52.13999436)(245.75812256,52.29)
\curveto(245.69811687,52.52999397)(245.60811696,52.71999378)(245.48812256,52.86)
\curveto(245.37811719,52.9999935)(245.21811735,53.10999339)(245.00812256,53.19)
\curveto(244.88811768,53.23999326)(244.74311783,53.27499323)(244.57312256,53.295)
\curveto(244.41311816,53.31499318)(244.24311833,53.32499317)(244.06312256,53.325)
\curveto(243.88311869,53.32499317)(243.70811886,53.31499318)(243.53812256,53.295)
\curveto(243.3681192,53.27499323)(243.22311935,53.24499325)(243.10312256,53.205)
\curveto(242.93311964,53.14499335)(242.7681198,53.05999344)(242.60812256,52.95)
\curveto(242.52812004,52.88999361)(242.45312012,52.80999369)(242.38312256,52.71)
\curveto(242.32312025,52.61999388)(242.2681203,52.51999398)(242.21812256,52.41)
\curveto(242.18812038,52.32999417)(242.15812041,52.24499425)(242.12812256,52.155)
\curveto(242.10812046,52.06499444)(242.06312051,51.99499451)(241.99312256,51.945)
\curveto(241.95312062,51.91499458)(241.88312069,51.88999461)(241.78312256,51.87)
\curveto(241.69312088,51.85999464)(241.59812097,51.85499464)(241.49812256,51.855)
\curveto(241.39812117,51.85499464)(241.29812127,51.85999464)(241.19812256,51.87)
\curveto(241.10812146,51.88999461)(241.04312153,51.91499458)(241.00312256,51.945)
\curveto(240.96312161,51.97499452)(240.93312164,52.02499447)(240.91312256,52.095)
\curveto(240.89312168,52.16499434)(240.89312168,52.23999426)(240.91312256,52.32)
\curveto(240.94312163,52.44999405)(240.9731216,52.56999393)(241.00312256,52.68)
\curveto(241.04312153,52.7999937)(241.08812148,52.91499358)(241.13812256,53.025)
\curveto(241.32812124,53.37499313)(241.568121,53.64499285)(241.85812256,53.835)
\curveto(242.14812042,54.03499247)(242.50812006,54.19499231)(242.93812256,54.315)
\curveto(243.03811953,54.33499217)(243.13811943,54.34999215)(243.23812256,54.36)
\curveto(243.34811922,54.36999213)(243.45811911,54.38499211)(243.56812256,54.405)
\curveto(243.60811896,54.41499208)(243.6731189,54.41499208)(243.76312256,54.405)
\curveto(243.85311872,54.4049921)(243.90811866,54.41499208)(243.92812256,54.435)
\curveto(244.62811794,54.44499206)(245.23811733,54.36499214)(245.75812256,54.195)
\curveto(246.27811629,54.02499247)(246.64311593,53.6999928)(246.85312256,53.22)
\curveto(246.94311563,53.01999348)(246.99311558,52.78499371)(247.00312256,52.515)
\curveto(247.02311555,52.25499424)(247.03311554,51.97999452)(247.03312256,51.69)
\lineto(247.03312256,48.375)
\curveto(247.03311554,48.23499827)(247.03811553,48.0999984)(247.04812256,47.97)
\curveto(247.05811551,47.83999866)(247.08811548,47.73499877)(247.13812256,47.655)
\curveto(247.18811538,47.58499892)(247.25311532,47.53499896)(247.33312256,47.505)
\curveto(247.42311515,47.46499904)(247.50811506,47.43499906)(247.58812256,47.415)
\curveto(247.6681149,47.40499909)(247.72811484,47.35999914)(247.76812256,47.28)
\curveto(247.78811478,47.24999925)(247.79811477,47.21999928)(247.79812256,47.19)
\curveto(247.79811477,47.15999934)(247.80311477,47.11999938)(247.81312256,47.07)
\moveto(245.66812256,48.735)
\curveto(245.72811684,48.87499762)(245.75811681,49.03499747)(245.75812256,49.215)
\curveto(245.7681168,49.40499709)(245.7731168,49.5999969)(245.77312256,49.8)
\curveto(245.7731168,49.90999659)(245.7681168,50.00999649)(245.75812256,50.1)
\curveto(245.74811682,50.18999631)(245.70811686,50.25999624)(245.63812256,50.31)
\curveto(245.60811696,50.32999617)(245.53811703,50.33999616)(245.42812256,50.34)
\curveto(245.40811716,50.31999618)(245.3731172,50.30999619)(245.32312256,50.31)
\curveto(245.2731173,50.30999619)(245.22811734,50.2999962)(245.18812256,50.28)
\curveto(245.10811746,50.25999624)(245.01811755,50.23999626)(244.91812256,50.22)
\lineto(244.61812256,50.16)
\curveto(244.58811798,50.15999634)(244.55311802,50.15499634)(244.51312256,50.145)
\lineto(244.40812256,50.145)
\curveto(244.25811831,50.1049964)(244.09311848,50.07999642)(243.91312256,50.07)
\curveto(243.74311883,50.06999643)(243.58311899,50.04999645)(243.43312256,50.01)
\curveto(243.35311922,49.98999651)(243.27811929,49.96999653)(243.20812256,49.95)
\curveto(243.14811942,49.93999656)(243.07811949,49.92499658)(242.99812256,49.905)
\curveto(242.83811973,49.85499665)(242.68811988,49.78999671)(242.54812256,49.71)
\curveto(242.40812016,49.63999686)(242.28812028,49.54999695)(242.18812256,49.44)
\curveto(242.08812048,49.32999717)(242.01312056,49.19499731)(241.96312256,49.035)
\curveto(241.91312066,48.88499761)(241.89312068,48.6999978)(241.90312256,48.48)
\curveto(241.90312067,48.37999812)(241.91812065,48.28499821)(241.94812256,48.195)
\curveto(241.98812058,48.11499838)(242.03312054,48.03999846)(242.08312256,47.97)
\curveto(242.16312041,47.85999864)(242.2681203,47.76499874)(242.39812256,47.685)
\curveto(242.52812004,47.61499888)(242.6681199,47.55499895)(242.81812256,47.505)
\curveto(242.8681197,47.49499901)(242.91811965,47.48999901)(242.96812256,47.49)
\curveto(243.01811955,47.48999901)(243.0681195,47.48499902)(243.11812256,47.475)
\curveto(243.18811938,47.45499905)(243.2731193,47.43999906)(243.37312256,47.43)
\curveto(243.48311909,47.42999907)(243.573119,47.43999906)(243.64312256,47.46)
\curveto(243.70311887,47.47999902)(243.76311881,47.48499902)(243.82312256,47.475)
\curveto(243.88311869,47.47499902)(243.94311863,47.48499902)(244.00312256,47.505)
\curveto(244.08311849,47.52499898)(244.15811841,47.53999896)(244.22812256,47.55)
\curveto(244.30811826,47.55999894)(244.38311819,47.57999892)(244.45312256,47.61)
\curveto(244.74311783,47.72999877)(244.98811758,47.87499862)(245.18812256,48.045)
\curveto(245.39811717,48.21499828)(245.55811701,48.44499805)(245.66812256,48.735)
}
}
{
\newrgbcolor{curcolor}{0 0 0}
\pscustom[linestyle=none,fillstyle=solid,fillcolor=curcolor]
{
\newpath
\moveto(251.41476318,54.42)
\curveto(252.13475912,54.42999207)(252.73975851,54.34499215)(253.22976318,54.165)
\curveto(253.71975753,53.99499251)(254.09975715,53.68999281)(254.36976318,53.25)
\curveto(254.43975681,53.13999336)(254.49475676,53.02499347)(254.53476318,52.905)
\curveto(254.57475668,52.79499371)(254.61475664,52.66999383)(254.65476318,52.53)
\curveto(254.67475658,52.45999404)(254.67975657,52.38499411)(254.66976318,52.305)
\curveto(254.65975659,52.23499427)(254.64475661,52.17999432)(254.62476318,52.14)
\curveto(254.60475665,52.11999438)(254.57975667,52.0999944)(254.54976318,52.08)
\curveto(254.51975673,52.06999443)(254.49475676,52.05499444)(254.47476318,52.035)
\curveto(254.42475683,52.01499448)(254.37475688,52.00999449)(254.32476318,52.02)
\curveto(254.27475698,52.02999447)(254.22475703,52.02999447)(254.17476318,52.02)
\curveto(254.09475716,51.9999945)(253.98975726,51.99499451)(253.85976318,52.005)
\curveto(253.72975752,52.02499447)(253.63975761,52.04999445)(253.58976318,52.08)
\curveto(253.50975774,52.12999437)(253.4547578,52.19499431)(253.42476318,52.275)
\curveto(253.40475785,52.36499414)(253.36975788,52.44999405)(253.31976318,52.53)
\curveto(253.22975802,52.68999381)(253.10475815,52.83499367)(252.94476318,52.965)
\curveto(252.83475842,53.04499345)(252.71475854,53.1049934)(252.58476318,53.145)
\curveto(252.4547588,53.18499331)(252.31475894,53.22499327)(252.16476318,53.265)
\curveto(252.11475914,53.28499321)(252.06475919,53.28999321)(252.01476318,53.28)
\curveto(251.96475929,53.27999322)(251.91475934,53.28499321)(251.86476318,53.295)
\curveto(251.80475945,53.31499318)(251.72975952,53.32499317)(251.63976318,53.325)
\curveto(251.5497597,53.32499317)(251.47475978,53.31499318)(251.41476318,53.295)
\lineto(251.32476318,53.295)
\lineto(251.17476318,53.265)
\curveto(251.12476013,53.26499324)(251.07476018,53.25999324)(251.02476318,53.25)
\curveto(250.76476049,53.18999331)(250.5497607,53.1049934)(250.37976318,52.995)
\curveto(250.20976104,52.88499361)(250.09476116,52.6999938)(250.03476318,52.44)
\curveto(250.01476124,52.36999413)(250.00976124,52.2999942)(250.01976318,52.23)
\curveto(250.03976121,52.15999434)(250.05976119,52.0999944)(250.07976318,52.05)
\curveto(250.13976111,51.8999946)(250.20976104,51.78999471)(250.28976318,51.72)
\curveto(250.37976087,51.65999484)(250.48976076,51.58999491)(250.61976318,51.51)
\curveto(250.77976047,51.40999509)(250.95976029,51.33499517)(251.15976318,51.285)
\curveto(251.35975989,51.24499525)(251.55975969,51.19499531)(251.75976318,51.135)
\curveto(251.88975936,51.09499541)(252.01975923,51.06499544)(252.14976318,51.045)
\curveto(252.27975897,51.02499548)(252.40975884,50.99499551)(252.53976318,50.955)
\curveto(252.7497585,50.89499561)(252.9547583,50.83499567)(253.15476318,50.775)
\curveto(253.3547579,50.72499578)(253.5547577,50.65999584)(253.75476318,50.58)
\lineto(253.90476318,50.52)
\curveto(253.9547573,50.499996)(254.00475725,50.47499602)(254.05476318,50.445)
\curveto(254.254757,50.32499618)(254.42975682,50.18999631)(254.57976318,50.04)
\curveto(254.72975652,49.88999661)(254.8547564,49.6999968)(254.95476318,49.47)
\curveto(254.97475628,49.3999971)(254.99475626,49.30499719)(255.01476318,49.185)
\curveto(255.03475622,49.11499738)(255.04475621,49.03999746)(255.04476318,48.96)
\curveto(255.0547562,48.88999761)(255.05975619,48.80999769)(255.05976318,48.72)
\lineto(255.05976318,48.57)
\curveto(255.03975621,48.499998)(255.02975622,48.42999807)(255.02976318,48.36)
\curveto(255.02975622,48.28999821)(255.01975623,48.21999828)(254.99976318,48.15)
\curveto(254.96975628,48.03999846)(254.93475632,47.93499857)(254.89476318,47.835)
\curveto(254.8547564,47.73499877)(254.80975644,47.64499885)(254.75976318,47.565)
\curveto(254.59975665,47.30499919)(254.39475686,47.09499941)(254.14476318,46.935)
\curveto(253.89475736,46.78499972)(253.61475764,46.65499985)(253.30476318,46.545)
\curveto(253.21475804,46.51499998)(253.11975813,46.49500001)(253.01976318,46.485)
\curveto(252.92975832,46.46500004)(252.83975841,46.44000006)(252.74976318,46.41)
\curveto(252.6497586,46.39000011)(252.5497587,46.38000012)(252.44976318,46.38)
\curveto(252.3497589,46.38000012)(252.249759,46.37000013)(252.14976318,46.35)
\lineto(251.99976318,46.35)
\curveto(251.9497593,46.34000016)(251.87975937,46.33500016)(251.78976318,46.335)
\curveto(251.69975955,46.33500016)(251.62975962,46.34000016)(251.57976318,46.35)
\lineto(251.41476318,46.35)
\curveto(251.3547599,46.37000013)(251.28975996,46.38000012)(251.21976318,46.38)
\curveto(251.1497601,46.37000013)(251.08976016,46.37500012)(251.03976318,46.395)
\curveto(250.98976026,46.40500009)(250.92476033,46.41000009)(250.84476318,46.41)
\lineto(250.60476318,46.47)
\curveto(250.53476072,46.48000002)(250.45976079,46.5)(250.37976318,46.53)
\curveto(250.06976118,46.62999987)(249.79976145,46.75499975)(249.56976318,46.905)
\curveto(249.33976191,47.05499945)(249.13976211,47.24999925)(248.96976318,47.49)
\curveto(248.87976237,47.61999888)(248.80476245,47.75499875)(248.74476318,47.895)
\curveto(248.68476257,48.03499847)(248.62976262,48.18999831)(248.57976318,48.36)
\curveto(248.55976269,48.41999808)(248.5497627,48.48999801)(248.54976318,48.57)
\curveto(248.55976269,48.65999784)(248.57476268,48.72999777)(248.59476318,48.78)
\curveto(248.62476263,48.81999768)(248.67476258,48.85999764)(248.74476318,48.9)
\curveto(248.79476246,48.91999758)(248.86476239,48.92999757)(248.95476318,48.93)
\curveto(249.04476221,48.93999756)(249.13476212,48.93999756)(249.22476318,48.93)
\curveto(249.31476194,48.91999758)(249.39976185,48.90499759)(249.47976318,48.885)
\curveto(249.56976168,48.87499762)(249.62976162,48.85999764)(249.65976318,48.84)
\curveto(249.72976152,48.78999771)(249.77476148,48.71499778)(249.79476318,48.615)
\curveto(249.82476143,48.52499798)(249.85976139,48.43999806)(249.89976318,48.36)
\curveto(249.99976125,48.13999836)(250.13476112,47.96999853)(250.30476318,47.85)
\curveto(250.42476083,47.75999874)(250.55976069,47.68999881)(250.70976318,47.64)
\curveto(250.85976039,47.58999891)(251.01976023,47.53999896)(251.18976318,47.49)
\lineto(251.50476318,47.445)
\lineto(251.59476318,47.445)
\curveto(251.66475959,47.42499908)(251.7547595,47.41499908)(251.86476318,47.415)
\curveto(251.98475927,47.41499908)(252.08475917,47.42499908)(252.16476318,47.445)
\curveto(252.23475902,47.44499905)(252.28975896,47.44999905)(252.32976318,47.46)
\curveto(252.38975886,47.46999903)(252.4497588,47.47499902)(252.50976318,47.475)
\curveto(252.56975868,47.48499902)(252.62475863,47.49499901)(252.67476318,47.505)
\curveto(252.96475829,47.58499892)(253.19475806,47.68999881)(253.36476318,47.82)
\curveto(253.53475772,47.94999855)(253.6547576,48.16999833)(253.72476318,48.48)
\curveto(253.74475751,48.52999797)(253.7497575,48.58499791)(253.73976318,48.645)
\curveto(253.72975752,48.70499779)(253.71975753,48.74999775)(253.70976318,48.78)
\curveto(253.65975759,48.96999753)(253.58975766,49.10999739)(253.49976318,49.2)
\curveto(253.40975784,49.2999972)(253.29475796,49.38999711)(253.15476318,49.47)
\curveto(253.06475819,49.52999697)(252.96475829,49.57999692)(252.85476318,49.62)
\lineto(252.52476318,49.74)
\curveto(252.49475876,49.74999675)(252.46475879,49.75499675)(252.43476318,49.755)
\curveto(252.41475884,49.75499675)(252.38975886,49.76499674)(252.35976318,49.785)
\curveto(252.01975923,49.89499661)(251.66475959,49.97499652)(251.29476318,50.025)
\curveto(250.93476032,50.08499641)(250.59476066,50.17999632)(250.27476318,50.31)
\curveto(250.17476108,50.34999615)(250.07976117,50.38499611)(249.98976318,50.415)
\curveto(249.89976135,50.44499605)(249.81476144,50.48499601)(249.73476318,50.535)
\curveto(249.54476171,50.64499585)(249.36976188,50.76999573)(249.20976318,50.91)
\curveto(249.0497622,51.04999545)(248.92476233,51.22499528)(248.83476318,51.435)
\curveto(248.80476245,51.504995)(248.77976247,51.57499492)(248.75976318,51.645)
\curveto(248.7497625,51.71499478)(248.73476252,51.78999471)(248.71476318,51.87)
\curveto(248.68476257,51.98999451)(248.67476258,52.12499438)(248.68476318,52.275)
\curveto(248.69476256,52.43499407)(248.70976254,52.56999393)(248.72976318,52.68)
\curveto(248.7497625,52.72999377)(248.75976249,52.76999373)(248.75976318,52.8)
\curveto(248.76976248,52.83999366)(248.78476247,52.87999362)(248.80476318,52.92)
\curveto(248.89476236,53.14999335)(249.01476224,53.34999315)(249.16476318,53.52)
\curveto(249.32476193,53.68999281)(249.50476175,53.83999266)(249.70476318,53.97)
\curveto(249.8547614,54.05999244)(250.01976123,54.12999237)(250.19976318,54.18)
\curveto(250.37976087,54.23999226)(250.56976068,54.29499221)(250.76976318,54.345)
\curveto(250.83976041,54.35499214)(250.90476035,54.36499214)(250.96476318,54.375)
\curveto(251.03476022,54.38499211)(251.10976014,54.39499211)(251.18976318,54.405)
\curveto(251.21976003,54.41499208)(251.25975999,54.41499208)(251.30976318,54.405)
\curveto(251.35975989,54.39499211)(251.39475986,54.3999921)(251.41476318,54.42)
}
}
{
\newrgbcolor{curcolor}{0 0 0}
\pscustom[linestyle=none,fillstyle=solid,fillcolor=curcolor]
{
\newpath
\moveto(382.46584106,57.195)
\lineto(383.38084106,57.195)
\curveto(383.48083841,57.1949893)(383.57583832,57.1949893)(383.66584106,57.195)
\curveto(383.75583814,57.1949893)(383.83083806,57.17498933)(383.89084106,57.135)
\curveto(383.98083791,57.07498943)(384.04083785,56.9949895)(384.07084106,56.895)
\curveto(384.11083778,56.7949897)(384.15583774,56.68998981)(384.20584106,56.58)
\curveto(384.28583761,56.38999011)(384.35583754,56.1999903)(384.41584106,56.01)
\curveto(384.48583741,55.81999068)(384.56083733,55.62999087)(384.64084106,55.44)
\curveto(384.71083718,55.25999124)(384.77583712,55.07499143)(384.83584106,54.885)
\curveto(384.895837,54.7049918)(384.96583693,54.52499197)(385.04584106,54.345)
\curveto(385.10583679,54.2049923)(385.16083673,54.05999244)(385.21084106,53.91)
\curveto(385.26083663,53.75999274)(385.31583658,53.61499288)(385.37584106,53.475)
\curveto(385.55583634,53.02499347)(385.72583617,52.56999393)(385.88584106,52.11)
\curveto(386.04583585,51.65999484)(386.21583568,51.20999529)(386.39584106,50.76)
\curveto(386.41583548,50.70999579)(386.43083546,50.65999584)(386.44084106,50.61)
\lineto(386.50084106,50.46)
\curveto(386.5908353,50.23999626)(386.67583522,50.01499648)(386.75584106,49.785)
\curveto(386.83583506,49.56499694)(386.92083497,49.34499715)(387.01084106,49.125)
\curveto(387.05083484,49.03499747)(387.0908348,48.92499758)(387.13084106,48.795)
\curveto(387.17083472,48.67499782)(387.23583466,48.60499789)(387.32584106,48.585)
\curveto(387.36583453,48.57499792)(387.3958345,48.57499792)(387.41584106,48.585)
\lineto(387.47584106,48.645)
\curveto(387.52583437,48.69499781)(387.56083433,48.74999775)(387.58084106,48.81)
\curveto(387.61083428,48.86999763)(387.64083425,48.93499757)(387.67084106,49.005)
\lineto(387.91084106,49.635)
\curveto(387.9908339,49.85499665)(388.07083382,50.06999643)(388.15084106,50.28)
\lineto(388.21084106,50.43)
\lineto(388.27084106,50.61)
\curveto(388.35083354,50.7999957)(388.42083347,50.98999551)(388.48084106,51.18)
\curveto(388.55083334,51.37999512)(388.62583327,51.57999492)(388.70584106,51.78)
\curveto(388.94583295,52.35999414)(389.16583273,52.94499355)(389.36584106,53.535)
\curveto(389.57583232,54.12499237)(389.80083209,54.70999179)(390.04084106,55.29)
\curveto(390.12083177,55.48999101)(390.1958317,55.6949908)(390.26584106,55.905)
\curveto(390.34583155,56.11499038)(390.42583147,56.31999018)(390.50584106,56.52)
\curveto(390.54583135,56.5999899)(390.58083131,56.6999898)(390.61084106,56.82)
\curveto(390.65083124,56.93998956)(390.70583119,57.02498947)(390.77584106,57.075)
\curveto(390.83583106,57.11498939)(390.91083098,57.14498936)(391.00084106,57.165)
\curveto(391.10083079,57.18498931)(391.21083068,57.1949893)(391.33084106,57.195)
\curveto(391.45083044,57.2049893)(391.57083032,57.2049893)(391.69084106,57.195)
\curveto(391.81083008,57.1949893)(391.92082997,57.1949893)(392.02084106,57.195)
\curveto(392.11082978,57.1949893)(392.20082969,57.1949893)(392.29084106,57.195)
\curveto(392.3908295,57.1949893)(392.46582943,57.17498933)(392.51584106,57.135)
\curveto(392.60582929,57.08498941)(392.65582924,56.9949895)(392.66584106,56.865)
\curveto(392.67582922,56.73498977)(392.68082921,56.5949899)(392.68084106,56.445)
\lineto(392.68084106,54.795)
\lineto(392.68084106,48.525)
\lineto(392.68084106,47.265)
\curveto(392.68082921,47.15499935)(392.68082921,47.04499945)(392.68084106,46.935)
\curveto(392.6908292,46.82499968)(392.67082922,46.73999976)(392.62084106,46.68)
\curveto(392.5908293,46.61999988)(392.54582935,46.57999992)(392.48584106,46.56)
\curveto(392.42582947,46.54999995)(392.35582954,46.53499996)(392.27584106,46.515)
\lineto(392.03584106,46.515)
\lineto(391.67584106,46.515)
\curveto(391.56583033,46.52499998)(391.48583041,46.56999993)(391.43584106,46.65)
\curveto(391.41583048,46.67999982)(391.40083049,46.70999979)(391.39084106,46.74)
\curveto(391.3908305,46.77999972)(391.38083051,46.82499968)(391.36084106,46.875)
\lineto(391.36084106,47.04)
\curveto(391.35083054,47.0999994)(391.34583055,47.16999933)(391.34584106,47.25)
\curveto(391.35583054,47.32999917)(391.36083053,47.40499909)(391.36084106,47.475)
\lineto(391.36084106,48.315)
\lineto(391.36084106,52.74)
\curveto(391.36083053,52.98999351)(391.36083053,53.23999326)(391.36084106,53.49)
\curveto(391.36083053,53.74999275)(391.35583054,53.9999925)(391.34584106,54.24)
\curveto(391.34583055,54.33999216)(391.34083055,54.44999205)(391.33084106,54.57)
\curveto(391.32083057,54.68999181)(391.26583063,54.74999175)(391.16584106,54.75)
\lineto(391.16584106,54.735)
\curveto(391.0958308,54.71499178)(391.03583086,54.64999185)(390.98584106,54.54)
\curveto(390.94583095,54.42999207)(390.91083098,54.33499217)(390.88084106,54.255)
\curveto(390.81083108,54.08499241)(390.74583115,53.90999259)(390.68584106,53.73)
\curveto(390.62583127,53.55999294)(390.55583134,53.38999311)(390.47584106,53.22)
\curveto(390.45583144,53.16999333)(390.44083145,53.12499337)(390.43084106,53.085)
\curveto(390.42083147,53.04499345)(390.40583149,52.9999935)(390.38584106,52.95)
\curveto(390.30583159,52.76999373)(390.23583166,52.58499391)(390.17584106,52.395)
\curveto(390.12583177,52.21499428)(390.06083183,52.03499447)(389.98084106,51.855)
\curveto(389.91083198,51.7049948)(389.85083204,51.54999495)(389.80084106,51.39)
\curveto(389.75083214,51.23999526)(389.6958322,51.08999541)(389.63584106,50.94)
\curveto(389.43583246,50.46999603)(389.25583264,49.99499651)(389.09584106,49.515)
\curveto(388.93583296,49.04499745)(388.76083313,48.57999792)(388.57084106,48.12)
\curveto(388.4908334,47.93999856)(388.42083347,47.75999874)(388.36084106,47.58)
\curveto(388.30083359,47.3999991)(388.23583366,47.21999928)(388.16584106,47.04)
\curveto(388.11583378,46.92999957)(388.06583383,46.82499968)(388.01584106,46.725)
\curveto(387.97583392,46.63499986)(387.890834,46.56999993)(387.76084106,46.53)
\curveto(387.74083415,46.51999998)(387.71583418,46.51499998)(387.68584106,46.515)
\curveto(387.66583423,46.52499998)(387.64083425,46.52499998)(387.61084106,46.515)
\curveto(387.58083431,46.50499999)(387.54583435,46.5)(387.50584106,46.5)
\curveto(387.46583443,46.50999999)(387.42583447,46.51499998)(387.38584106,46.515)
\lineto(387.08584106,46.515)
\curveto(386.98583491,46.51499998)(386.90583499,46.53999996)(386.84584106,46.59)
\curveto(386.76583513,46.63999986)(386.70583519,46.70999979)(386.66584106,46.8)
\curveto(386.63583526,46.8999996)(386.5958353,46.9999995)(386.54584106,47.1)
\curveto(386.46583543,47.2999992)(386.38583551,47.50499899)(386.30584106,47.715)
\curveto(386.23583566,47.93499857)(386.16083573,48.14499835)(386.08084106,48.345)
\curveto(386.00083589,48.52499798)(385.93083596,48.70499779)(385.87084106,48.885)
\curveto(385.82083607,49.07499742)(385.75583614,49.25999724)(385.67584106,49.44)
\curveto(385.44583645,49.9999965)(385.23083666,50.56499594)(385.03084106,51.135)
\curveto(384.83083706,51.7049948)(384.61583728,52.26999423)(384.38584106,52.83)
\lineto(384.14584106,53.46)
\curveto(384.07583782,53.67999282)(384.00083789,53.88999261)(383.92084106,54.09)
\curveto(383.87083802,54.1999923)(383.82583807,54.3049922)(383.78584106,54.405)
\curveto(383.75583814,54.51499198)(383.70583819,54.60999189)(383.63584106,54.69)
\curveto(383.62583827,54.70999179)(383.61583828,54.71999178)(383.60584106,54.72)
\lineto(383.57584106,54.75)
\lineto(383.50084106,54.75)
\lineto(383.47084106,54.72)
\curveto(383.46083843,54.71999178)(383.45083844,54.71499178)(383.44084106,54.705)
\curveto(383.42083847,54.65499184)(383.41083848,54.5999919)(383.41084106,54.54)
\curveto(383.41083848,54.47999202)(383.40083849,54.41999208)(383.38084106,54.36)
\lineto(383.38084106,54.195)
\curveto(383.36083853,54.13499237)(383.35583854,54.06999243)(383.36584106,54)
\curveto(383.37583852,53.92999257)(383.38083851,53.85999264)(383.38084106,53.79)
\lineto(383.38084106,52.98)
\lineto(383.38084106,48.42)
\lineto(383.38084106,47.235)
\curveto(383.38083851,47.12499938)(383.37583852,47.01499948)(383.36584106,46.905)
\curveto(383.36583853,46.79499971)(383.34083855,46.70999979)(383.29084106,46.65)
\curveto(383.24083865,46.56999993)(383.15083874,46.52499998)(383.02084106,46.515)
\lineto(382.63084106,46.515)
\lineto(382.43584106,46.515)
\curveto(382.38583951,46.51499998)(382.33583956,46.52499998)(382.28584106,46.545)
\curveto(382.15583974,46.58499992)(382.08083981,46.66999983)(382.06084106,46.8)
\curveto(382.05083984,46.92999957)(382.04583985,47.07999942)(382.04584106,47.25)
\lineto(382.04584106,48.99)
\lineto(382.04584106,54.99)
\lineto(382.04584106,56.4)
\curveto(382.04583985,56.50998999)(382.04083985,56.62498987)(382.03084106,56.745)
\curveto(382.03083986,56.86498964)(382.05583984,56.95998954)(382.10584106,57.03)
\curveto(382.14583975,57.08998941)(382.22083967,57.13998936)(382.33084106,57.18)
\curveto(382.35083954,57.18998931)(382.37083952,57.18998931)(382.39084106,57.18)
\curveto(382.42083947,57.17998932)(382.44583945,57.18498931)(382.46584106,57.195)
}
}
{
\newrgbcolor{curcolor}{0 0 0}
\pscustom[linestyle=none,fillstyle=solid,fillcolor=curcolor]
{
\newpath
\moveto(401.66795044,47.07)
\curveto(401.69794261,46.90999959)(401.68294262,46.77499972)(401.62295044,46.665)
\curveto(401.56294274,46.56499994)(401.48294282,46.49000001)(401.38295044,46.44)
\curveto(401.33294297,46.42000008)(401.27794303,46.41000009)(401.21795044,46.41)
\curveto(401.16794314,46.41000009)(401.11294319,46.4000001)(401.05295044,46.38)
\curveto(400.83294347,46.33000017)(400.61294369,46.34500015)(400.39295044,46.425)
\curveto(400.18294412,46.49500001)(400.03794427,46.58499992)(399.95795044,46.695)
\curveto(399.9079444,46.76499974)(399.86294444,46.84499965)(399.82295044,46.935)
\curveto(399.78294452,47.03499946)(399.73294457,47.11499938)(399.67295044,47.175)
\curveto(399.65294465,47.19499931)(399.62794468,47.21499928)(399.59795044,47.235)
\curveto(399.57794473,47.25499925)(399.54794476,47.25999924)(399.50795044,47.25)
\curveto(399.39794491,47.21999928)(399.29294501,47.16499934)(399.19295044,47.085)
\curveto(399.1029452,47.00499949)(399.01294529,46.93499956)(398.92295044,46.875)
\curveto(398.79294551,46.79499971)(398.65294565,46.71999978)(398.50295044,46.65)
\curveto(398.35294595,46.58999991)(398.19294611,46.53499996)(398.02295044,46.485)
\curveto(397.92294638,46.45500005)(397.81294649,46.43500006)(397.69295044,46.425)
\curveto(397.58294672,46.41500008)(397.47294683,46.4000001)(397.36295044,46.38)
\curveto(397.31294699,46.37000013)(397.26794704,46.36500014)(397.22795044,46.365)
\lineto(397.12295044,46.365)
\curveto(397.01294729,46.34500015)(396.9079474,46.34500015)(396.80795044,46.365)
\lineto(396.67295044,46.365)
\curveto(396.62294768,46.37500012)(396.57294773,46.38000012)(396.52295044,46.38)
\curveto(396.47294783,46.38000012)(396.42794788,46.39000011)(396.38795044,46.41)
\curveto(396.34794796,46.42000008)(396.31294799,46.42500008)(396.28295044,46.425)
\curveto(396.26294804,46.41500008)(396.23794807,46.41500008)(396.20795044,46.425)
\lineto(395.96795044,46.485)
\curveto(395.88794842,46.49500001)(395.81294849,46.51499998)(395.74295044,46.545)
\curveto(395.44294886,46.67499982)(395.19794911,46.81999968)(395.00795044,46.98)
\curveto(394.82794948,47.14999935)(394.67794963,47.38499912)(394.55795044,47.685)
\curveto(394.46794984,47.90499859)(394.42294988,48.16999833)(394.42295044,48.48)
\lineto(394.42295044,48.795)
\curveto(394.43294987,48.84499765)(394.43794987,48.89499761)(394.43795044,48.945)
\lineto(394.46795044,49.125)
\lineto(394.58795044,49.455)
\curveto(394.62794968,49.56499694)(394.67794963,49.66499684)(394.73795044,49.755)
\curveto(394.91794939,50.04499645)(395.16294914,50.25999624)(395.47295044,50.4)
\curveto(395.78294852,50.53999596)(396.12294818,50.66499584)(396.49295044,50.775)
\curveto(396.63294767,50.81499568)(396.77794753,50.84499565)(396.92795044,50.865)
\curveto(397.07794723,50.88499561)(397.22794708,50.90999559)(397.37795044,50.94)
\curveto(397.44794686,50.95999554)(397.51294679,50.96999553)(397.57295044,50.97)
\curveto(397.64294666,50.96999553)(397.71794659,50.97999552)(397.79795044,51)
\curveto(397.86794644,51.01999548)(397.93794637,51.02999547)(398.00795044,51.03)
\curveto(398.07794623,51.03999546)(398.15294615,51.05499544)(398.23295044,51.075)
\curveto(398.48294582,51.13499537)(398.71794559,51.18499531)(398.93795044,51.225)
\curveto(399.15794515,51.27499522)(399.33294497,51.38999511)(399.46295044,51.57)
\curveto(399.52294478,51.64999485)(399.57294473,51.74999475)(399.61295044,51.87)
\curveto(399.65294465,51.9999945)(399.65294465,52.13999436)(399.61295044,52.29)
\curveto(399.55294475,52.52999397)(399.46294484,52.71999378)(399.34295044,52.86)
\curveto(399.23294507,52.9999935)(399.07294523,53.10999339)(398.86295044,53.19)
\curveto(398.74294556,53.23999326)(398.59794571,53.27499323)(398.42795044,53.295)
\curveto(398.26794604,53.31499318)(398.09794621,53.32499317)(397.91795044,53.325)
\curveto(397.73794657,53.32499317)(397.56294674,53.31499318)(397.39295044,53.295)
\curveto(397.22294708,53.27499323)(397.07794723,53.24499325)(396.95795044,53.205)
\curveto(396.78794752,53.14499335)(396.62294768,53.05999344)(396.46295044,52.95)
\curveto(396.38294792,52.88999361)(396.307948,52.80999369)(396.23795044,52.71)
\curveto(396.17794813,52.61999388)(396.12294818,52.51999398)(396.07295044,52.41)
\curveto(396.04294826,52.32999417)(396.01294829,52.24499425)(395.98295044,52.155)
\curveto(395.96294834,52.06499444)(395.91794839,51.99499451)(395.84795044,51.945)
\curveto(395.8079485,51.91499458)(395.73794857,51.88999461)(395.63795044,51.87)
\curveto(395.54794876,51.85999464)(395.45294885,51.85499464)(395.35295044,51.855)
\curveto(395.25294905,51.85499464)(395.15294915,51.85999464)(395.05295044,51.87)
\curveto(394.96294934,51.88999461)(394.89794941,51.91499458)(394.85795044,51.945)
\curveto(394.81794949,51.97499452)(394.78794952,52.02499447)(394.76795044,52.095)
\curveto(394.74794956,52.16499434)(394.74794956,52.23999426)(394.76795044,52.32)
\curveto(394.79794951,52.44999405)(394.82794948,52.56999393)(394.85795044,52.68)
\curveto(394.89794941,52.7999937)(394.94294936,52.91499358)(394.99295044,53.025)
\curveto(395.18294912,53.37499313)(395.42294888,53.64499285)(395.71295044,53.835)
\curveto(396.0029483,54.03499247)(396.36294794,54.19499231)(396.79295044,54.315)
\curveto(396.89294741,54.33499217)(396.99294731,54.34999215)(397.09295044,54.36)
\curveto(397.2029471,54.36999213)(397.31294699,54.38499211)(397.42295044,54.405)
\curveto(397.46294684,54.41499208)(397.52794678,54.41499208)(397.61795044,54.405)
\curveto(397.7079466,54.4049921)(397.76294654,54.41499208)(397.78295044,54.435)
\curveto(398.48294582,54.44499206)(399.09294521,54.36499214)(399.61295044,54.195)
\curveto(400.13294417,54.02499247)(400.49794381,53.6999928)(400.70795044,53.22)
\curveto(400.79794351,53.01999348)(400.84794346,52.78499371)(400.85795044,52.515)
\curveto(400.87794343,52.25499424)(400.88794342,51.97999452)(400.88795044,51.69)
\lineto(400.88795044,48.375)
\curveto(400.88794342,48.23499827)(400.89294341,48.0999984)(400.90295044,47.97)
\curveto(400.91294339,47.83999866)(400.94294336,47.73499877)(400.99295044,47.655)
\curveto(401.04294326,47.58499892)(401.1079432,47.53499896)(401.18795044,47.505)
\curveto(401.27794303,47.46499904)(401.36294294,47.43499906)(401.44295044,47.415)
\curveto(401.52294278,47.40499909)(401.58294272,47.35999914)(401.62295044,47.28)
\curveto(401.64294266,47.24999925)(401.65294265,47.21999928)(401.65295044,47.19)
\curveto(401.65294265,47.15999934)(401.65794265,47.11999938)(401.66795044,47.07)
\moveto(399.52295044,48.735)
\curveto(399.58294472,48.87499762)(399.61294469,49.03499747)(399.61295044,49.215)
\curveto(399.62294468,49.40499709)(399.62794468,49.5999969)(399.62795044,49.8)
\curveto(399.62794468,49.90999659)(399.62294468,50.00999649)(399.61295044,50.1)
\curveto(399.6029447,50.18999631)(399.56294474,50.25999624)(399.49295044,50.31)
\curveto(399.46294484,50.32999617)(399.39294491,50.33999616)(399.28295044,50.34)
\curveto(399.26294504,50.31999618)(399.22794508,50.30999619)(399.17795044,50.31)
\curveto(399.12794518,50.30999619)(399.08294522,50.2999962)(399.04295044,50.28)
\curveto(398.96294534,50.25999624)(398.87294543,50.23999626)(398.77295044,50.22)
\lineto(398.47295044,50.16)
\curveto(398.44294586,50.15999634)(398.4079459,50.15499634)(398.36795044,50.145)
\lineto(398.26295044,50.145)
\curveto(398.11294619,50.1049964)(397.94794636,50.07999642)(397.76795044,50.07)
\curveto(397.59794671,50.06999643)(397.43794687,50.04999645)(397.28795044,50.01)
\curveto(397.2079471,49.98999651)(397.13294717,49.96999653)(397.06295044,49.95)
\curveto(397.0029473,49.93999656)(396.93294737,49.92499658)(396.85295044,49.905)
\curveto(396.69294761,49.85499665)(396.54294776,49.78999671)(396.40295044,49.71)
\curveto(396.26294804,49.63999686)(396.14294816,49.54999695)(396.04295044,49.44)
\curveto(395.94294836,49.32999717)(395.86794844,49.19499731)(395.81795044,49.035)
\curveto(395.76794854,48.88499761)(395.74794856,48.6999978)(395.75795044,48.48)
\curveto(395.75794855,48.37999812)(395.77294853,48.28499821)(395.80295044,48.195)
\curveto(395.84294846,48.11499838)(395.88794842,48.03999846)(395.93795044,47.97)
\curveto(396.01794829,47.85999864)(396.12294818,47.76499874)(396.25295044,47.685)
\curveto(396.38294792,47.61499888)(396.52294778,47.55499895)(396.67295044,47.505)
\curveto(396.72294758,47.49499901)(396.77294753,47.48999901)(396.82295044,47.49)
\curveto(396.87294743,47.48999901)(396.92294738,47.48499902)(396.97295044,47.475)
\curveto(397.04294726,47.45499905)(397.12794718,47.43999906)(397.22795044,47.43)
\curveto(397.33794697,47.42999907)(397.42794688,47.43999906)(397.49795044,47.46)
\curveto(397.55794675,47.47999902)(397.61794669,47.48499902)(397.67795044,47.475)
\curveto(397.73794657,47.47499902)(397.79794651,47.48499902)(397.85795044,47.505)
\curveto(397.93794637,47.52499898)(398.01294629,47.53999896)(398.08295044,47.55)
\curveto(398.16294614,47.55999894)(398.23794607,47.57999892)(398.30795044,47.61)
\curveto(398.59794571,47.72999877)(398.84294546,47.87499862)(399.04295044,48.045)
\curveto(399.25294505,48.21499828)(399.41294489,48.44499805)(399.52295044,48.735)
}
}
{
\newrgbcolor{curcolor}{0 0 0}
\pscustom[linestyle=none,fillstyle=solid,fillcolor=curcolor]
{
\newpath
\moveto(403.78459106,56.58)
\curveto(403.93458905,56.57998992)(404.0845889,56.57498993)(404.23459106,56.565)
\curveto(404.3845886,56.56498993)(404.4895885,56.52498997)(404.54959106,56.445)
\curveto(404.59958839,56.38499011)(404.62458836,56.2999902)(404.62459106,56.19)
\curveto(404.63458835,56.08999041)(404.63958835,55.98499051)(404.63959106,55.875)
\lineto(404.63959106,55.005)
\curveto(404.63958835,54.92499157)(404.63458835,54.83999166)(404.62459106,54.75)
\curveto(404.62458836,54.66999183)(404.63458835,54.5999919)(404.65459106,54.54)
\curveto(404.69458829,54.3999921)(404.7845882,54.30999219)(404.92459106,54.27)
\curveto(404.97458801,54.25999224)(405.01958797,54.25499224)(405.05959106,54.255)
\lineto(405.20959106,54.255)
\lineto(405.61459106,54.255)
\curveto(405.77458721,54.26499224)(405.8895871,54.25499224)(405.95959106,54.225)
\curveto(406.04958694,54.16499234)(406.10958688,54.1049924)(406.13959106,54.045)
\curveto(406.15958683,54.0049925)(406.16958682,53.95999254)(406.16959106,53.91)
\lineto(406.16959106,53.76)
\curveto(406.16958682,53.64999285)(406.16458682,53.54499295)(406.15459106,53.445)
\curveto(406.14458684,53.35499314)(406.10958688,53.28499321)(406.04959106,53.235)
\curveto(405.989587,53.18499331)(405.90458708,53.15499334)(405.79459106,53.145)
\lineto(405.46459106,53.145)
\curveto(405.35458763,53.15499334)(405.24458774,53.15999334)(405.13459106,53.16)
\curveto(405.02458796,53.15999334)(404.92958806,53.14499335)(404.84959106,53.115)
\curveto(404.77958821,53.08499341)(404.72958826,53.03499347)(404.69959106,52.965)
\curveto(404.66958832,52.89499361)(404.64958834,52.80999369)(404.63959106,52.71)
\curveto(404.62958836,52.61999388)(404.62458836,52.51999398)(404.62459106,52.41)
\curveto(404.63458835,52.30999419)(404.63958835,52.20999429)(404.63959106,52.11)
\lineto(404.63959106,49.14)
\curveto(404.63958835,48.91999758)(404.63458835,48.68499781)(404.62459106,48.435)
\curveto(404.62458836,48.19499831)(404.66958832,48.00999849)(404.75959106,47.88)
\curveto(404.80958818,47.7999987)(404.87458811,47.74499875)(404.95459106,47.715)
\curveto(405.03458795,47.68499881)(405.12958786,47.65999884)(405.23959106,47.64)
\curveto(405.26958772,47.62999887)(405.29958769,47.62499888)(405.32959106,47.625)
\curveto(405.36958762,47.63499887)(405.40458758,47.63499887)(405.43459106,47.625)
\lineto(405.62959106,47.625)
\curveto(405.72958726,47.62499888)(405.81958717,47.61499888)(405.89959106,47.595)
\curveto(405.989587,47.58499892)(406.05458693,47.54999895)(406.09459106,47.49)
\curveto(406.11458687,47.45999904)(406.12958686,47.40499909)(406.13959106,47.325)
\curveto(406.15958683,47.25499925)(406.16958682,47.17999932)(406.16959106,47.1)
\curveto(406.17958681,47.01999948)(406.17958681,46.93999956)(406.16959106,46.86)
\curveto(406.15958683,46.78999971)(406.13958685,46.73499976)(406.10959106,46.695)
\curveto(406.06958692,46.62499988)(405.99458699,46.57499992)(405.88459106,46.545)
\curveto(405.80458718,46.52499998)(405.71458727,46.51499998)(405.61459106,46.515)
\curveto(405.51458747,46.52499998)(405.42458756,46.52999997)(405.34459106,46.53)
\curveto(405.2845877,46.52999997)(405.22458776,46.52499998)(405.16459106,46.515)
\curveto(405.10458788,46.51499998)(405.04958794,46.51999998)(404.99959106,46.53)
\lineto(404.81959106,46.53)
\curveto(404.76958822,46.53999996)(404.71958827,46.54499995)(404.66959106,46.545)
\curveto(404.62958836,46.55499995)(404.5845884,46.55999994)(404.53459106,46.56)
\curveto(404.33458865,46.60999989)(404.15958883,46.66499984)(404.00959106,46.725)
\curveto(403.86958912,46.78499972)(403.74958924,46.88999961)(403.64959106,47.04)
\curveto(403.50958948,47.23999926)(403.42958956,47.48999901)(403.40959106,47.79)
\curveto(403.3895896,48.0999984)(403.37958961,48.42999807)(403.37959106,48.78)
\lineto(403.37959106,52.71)
\curveto(403.34958964,52.83999366)(403.31958967,52.93499357)(403.28959106,52.995)
\curveto(403.26958972,53.05499344)(403.19958979,53.1049934)(403.07959106,53.145)
\curveto(403.03958995,53.15499334)(402.99958999,53.15499334)(402.95959106,53.145)
\curveto(402.91959007,53.13499337)(402.87959011,53.13999336)(402.83959106,53.16)
\lineto(402.59959106,53.16)
\curveto(402.46959052,53.15999334)(402.35959063,53.16999333)(402.26959106,53.19)
\curveto(402.1895908,53.21999328)(402.13459085,53.27999322)(402.10459106,53.37)
\curveto(402.0845909,53.40999309)(402.06959092,53.45499304)(402.05959106,53.505)
\lineto(402.05959106,53.655)
\curveto(402.05959093,53.79499271)(402.06959092,53.90999259)(402.08959106,54)
\curveto(402.10959088,54.0999924)(402.16959082,54.17499233)(402.26959106,54.225)
\curveto(402.37959061,54.26499224)(402.51959047,54.27499223)(402.68959106,54.255)
\curveto(402.86959012,54.23499227)(403.01958997,54.24499225)(403.13959106,54.285)
\curveto(403.22958976,54.33499217)(403.29958969,54.4049921)(403.34959106,54.495)
\curveto(403.36958962,54.55499194)(403.37958961,54.62999187)(403.37959106,54.72)
\lineto(403.37959106,54.975)
\lineto(403.37959106,55.905)
\lineto(403.37959106,56.145)
\curveto(403.37958961,56.23499027)(403.3895896,56.30999019)(403.40959106,56.37)
\curveto(403.44958954,56.44999005)(403.52458946,56.51498998)(403.63459106,56.565)
\curveto(403.66458932,56.56498993)(403.6895893,56.56498993)(403.70959106,56.565)
\curveto(403.73958925,56.57498993)(403.76458922,56.57998992)(403.78459106,56.58)
}
}
{
\newrgbcolor{curcolor}{0 0 0}
\pscustom[linestyle=none,fillstyle=solid,fillcolor=curcolor]
{
\newpath
\moveto(414.30638794,50.685)
\curveto(414.32638025,50.58499591)(414.32638025,50.46999603)(414.30638794,50.34)
\curveto(414.29638028,50.21999628)(414.26638031,50.13499637)(414.21638794,50.085)
\curveto(414.16638041,50.04499645)(414.09138049,50.01499648)(413.99138794,49.995)
\curveto(413.90138068,49.98499651)(413.79638078,49.97999652)(413.67638794,49.98)
\lineto(413.31638794,49.98)
\curveto(413.19638138,49.98999651)(413.09138149,49.99499651)(413.00138794,49.995)
\lineto(409.16138794,49.995)
\curveto(409.0813855,49.99499651)(409.00138558,49.98999651)(408.92138794,49.98)
\curveto(408.84138574,49.97999652)(408.7763858,49.96499654)(408.72638794,49.935)
\curveto(408.68638589,49.91499658)(408.64638593,49.87499662)(408.60638794,49.815)
\curveto(408.58638599,49.78499671)(408.56638601,49.73999676)(408.54638794,49.68)
\curveto(408.52638605,49.62999687)(408.52638605,49.57999692)(408.54638794,49.53)
\curveto(408.55638602,49.47999702)(408.56138602,49.43499707)(408.56138794,49.395)
\curveto(408.56138602,49.35499715)(408.56638601,49.31499718)(408.57638794,49.275)
\curveto(408.59638598,49.19499731)(408.61638596,49.10999739)(408.63638794,49.02)
\curveto(408.65638592,48.93999756)(408.68638589,48.85999764)(408.72638794,48.78)
\curveto(408.95638562,48.23999826)(409.33638524,47.85499865)(409.86638794,47.625)
\curveto(409.92638465,47.59499891)(409.99138459,47.56999893)(410.06138794,47.55)
\lineto(410.27138794,47.49)
\curveto(410.30138428,47.47999902)(410.35138423,47.47499902)(410.42138794,47.475)
\curveto(410.56138402,47.43499906)(410.74638383,47.41499908)(410.97638794,47.415)
\curveto(411.20638337,47.41499908)(411.39138319,47.43499906)(411.53138794,47.475)
\curveto(411.67138291,47.51499898)(411.79638278,47.55499895)(411.90638794,47.595)
\curveto(412.02638255,47.64499885)(412.13638244,47.70499879)(412.23638794,47.775)
\curveto(412.34638223,47.84499865)(412.44138214,47.92499858)(412.52138794,48.015)
\curveto(412.60138198,48.11499838)(412.67138191,48.21999828)(412.73138794,48.33)
\curveto(412.79138179,48.42999807)(412.84138174,48.53499797)(412.88138794,48.645)
\curveto(412.93138165,48.75499775)(413.01138157,48.83499767)(413.12138794,48.885)
\curveto(413.16138142,48.90499759)(413.22638135,48.91999758)(413.31638794,48.93)
\curveto(413.40638117,48.93999756)(413.49638108,48.93999756)(413.58638794,48.93)
\curveto(413.6763809,48.92999757)(413.76138082,48.92499758)(413.84138794,48.915)
\curveto(413.92138066,48.90499759)(413.9763806,48.88499761)(414.00638794,48.855)
\curveto(414.10638047,48.78499771)(414.13138045,48.66999783)(414.08138794,48.51)
\curveto(414.00138058,48.23999826)(413.89638068,47.9999985)(413.76638794,47.79)
\curveto(413.56638101,47.46999903)(413.33638124,47.20499929)(413.07638794,46.995)
\curveto(412.82638175,46.79499971)(412.50638207,46.62999987)(412.11638794,46.5)
\curveto(412.01638256,46.46000004)(411.91638266,46.43500006)(411.81638794,46.425)
\curveto(411.71638286,46.40500009)(411.61138297,46.38500012)(411.50138794,46.365)
\curveto(411.45138313,46.35500015)(411.40138318,46.35000015)(411.35138794,46.35)
\curveto(411.31138327,46.35000015)(411.26638331,46.34500015)(411.21638794,46.335)
\lineto(411.06638794,46.335)
\curveto(411.01638356,46.32500018)(410.95638362,46.32000018)(410.88638794,46.32)
\curveto(410.82638375,46.32000018)(410.7763838,46.32500018)(410.73638794,46.335)
\lineto(410.60138794,46.335)
\curveto(410.55138403,46.34500015)(410.50638407,46.35000015)(410.46638794,46.35)
\curveto(410.42638415,46.35000015)(410.38638419,46.35500015)(410.34638794,46.365)
\curveto(410.29638428,46.37500012)(410.24138434,46.38500012)(410.18138794,46.395)
\curveto(410.12138446,46.39500011)(410.06638451,46.4000001)(410.01638794,46.41)
\curveto(409.92638465,46.43000007)(409.83638474,46.45500005)(409.74638794,46.485)
\curveto(409.65638492,46.50499999)(409.57138501,46.52999997)(409.49138794,46.56)
\curveto(409.45138513,46.57999992)(409.41638516,46.58999991)(409.38638794,46.59)
\curveto(409.35638522,46.5999999)(409.32138526,46.61499988)(409.28138794,46.635)
\curveto(409.13138545,46.70499979)(408.97138561,46.78999971)(408.80138794,46.89)
\curveto(408.51138607,47.07999942)(408.26138632,47.30999919)(408.05138794,47.58)
\curveto(407.85138673,47.85999864)(407.6813869,48.16999833)(407.54138794,48.51)
\curveto(407.49138709,48.61999788)(407.45138713,48.73499777)(407.42138794,48.855)
\curveto(407.40138718,48.97499752)(407.37138721,49.09499741)(407.33138794,49.215)
\curveto(407.32138726,49.25499725)(407.31638726,49.28999721)(407.31638794,49.32)
\curveto(407.31638726,49.34999715)(407.31138727,49.38999711)(407.30138794,49.44)
\curveto(407.2813873,49.51999698)(407.26638731,49.60499689)(407.25638794,49.695)
\curveto(407.24638733,49.78499671)(407.23138735,49.87499662)(407.21138794,49.965)
\lineto(407.21138794,50.175)
\curveto(407.20138738,50.21499628)(407.19138739,50.26999623)(407.18138794,50.34)
\curveto(407.1813874,50.41999608)(407.18638739,50.48499601)(407.19638794,50.535)
\lineto(407.19638794,50.7)
\curveto(407.21638736,50.74999575)(407.22138736,50.7999957)(407.21138794,50.85)
\curveto(407.21138737,50.90999559)(407.21638736,50.96499554)(407.22638794,51.015)
\curveto(407.26638731,51.17499532)(407.29638728,51.33499517)(407.31638794,51.495)
\curveto(407.34638723,51.65499484)(407.39138719,51.8049947)(407.45138794,51.945)
\curveto(407.50138708,52.05499444)(407.54638703,52.16499434)(407.58638794,52.275)
\curveto(407.63638694,52.39499411)(407.69138689,52.50999399)(407.75138794,52.62)
\curveto(407.97138661,52.96999353)(408.22138636,53.26999323)(408.50138794,53.52)
\curveto(408.7813858,53.77999272)(409.12638545,53.99499251)(409.53638794,54.165)
\curveto(409.65638492,54.21499228)(409.7763848,54.24999225)(409.89638794,54.27)
\curveto(410.02638455,54.2999922)(410.16138442,54.32999217)(410.30138794,54.36)
\curveto(410.35138423,54.36999213)(410.39638418,54.37499213)(410.43638794,54.375)
\curveto(410.4763841,54.38499211)(410.52138406,54.38999211)(410.57138794,54.39)
\curveto(410.59138399,54.3999921)(410.61638396,54.3999921)(410.64638794,54.39)
\curveto(410.6763839,54.37999212)(410.70138388,54.38499211)(410.72138794,54.405)
\curveto(411.14138344,54.41499208)(411.50638307,54.36999213)(411.81638794,54.27)
\curveto(412.12638245,54.17999232)(412.40638217,54.05499244)(412.65638794,53.895)
\curveto(412.70638187,53.87499263)(412.74638183,53.84499265)(412.77638794,53.805)
\curveto(412.80638177,53.77499273)(412.84138174,53.74999275)(412.88138794,53.73)
\curveto(412.96138162,53.66999283)(413.04138154,53.5999929)(413.12138794,53.52)
\curveto(413.21138137,53.43999306)(413.28638129,53.35999314)(413.34638794,53.28)
\curveto(413.50638107,53.06999343)(413.64138094,52.86999363)(413.75138794,52.68)
\curveto(413.82138076,52.56999393)(413.8763807,52.44999405)(413.91638794,52.32)
\curveto(413.95638062,52.18999431)(414.00138058,52.05999444)(414.05138794,51.93)
\curveto(414.10138048,51.7999947)(414.13638044,51.66499484)(414.15638794,51.525)
\curveto(414.18638039,51.38499511)(414.22138036,51.24499525)(414.26138794,51.105)
\curveto(414.27138031,51.03499547)(414.2763803,50.96499554)(414.27638794,50.895)
\lineto(414.30638794,50.685)
\moveto(412.85138794,51.195)
\curveto(412.8813817,51.23499527)(412.90638167,51.28499521)(412.92638794,51.345)
\curveto(412.94638163,51.41499508)(412.94638163,51.48499501)(412.92638794,51.555)
\curveto(412.86638171,51.77499472)(412.7813818,51.97999452)(412.67138794,52.17)
\curveto(412.53138205,52.3999941)(412.3763822,52.59499391)(412.20638794,52.755)
\curveto(412.03638254,52.91499358)(411.81638276,53.04999345)(411.54638794,53.16)
\curveto(411.4763831,53.17999332)(411.40638317,53.19499331)(411.33638794,53.205)
\curveto(411.26638331,53.22499327)(411.19138339,53.24499325)(411.11138794,53.265)
\curveto(411.03138355,53.28499321)(410.94638363,53.29499321)(410.85638794,53.295)
\lineto(410.60138794,53.295)
\curveto(410.57138401,53.27499323)(410.53638404,53.26499324)(410.49638794,53.265)
\curveto(410.45638412,53.27499323)(410.42138416,53.27499323)(410.39138794,53.265)
\lineto(410.15138794,53.205)
\curveto(410.0813845,53.19499331)(410.01138457,53.17999332)(409.94138794,53.16)
\curveto(409.65138493,53.03999346)(409.41638516,52.88999361)(409.23638794,52.71)
\curveto(409.06638551,52.52999397)(408.91138567,52.3049942)(408.77138794,52.035)
\curveto(408.74138584,51.98499451)(408.71138587,51.91999458)(408.68138794,51.84)
\curveto(408.65138593,51.76999473)(408.62638595,51.68999481)(408.60638794,51.6)
\curveto(408.58638599,51.50999499)(408.581386,51.42499508)(408.59138794,51.345)
\curveto(408.60138598,51.26499524)(408.63638594,51.2049953)(408.69638794,51.165)
\curveto(408.7763858,51.1049954)(408.91138567,51.07499542)(409.10138794,51.075)
\curveto(409.30138528,51.08499541)(409.47138511,51.08999541)(409.61138794,51.09)
\lineto(411.89138794,51.09)
\curveto(412.04138254,51.08999541)(412.22138236,51.08499541)(412.43138794,51.075)
\curveto(412.64138194,51.07499542)(412.7813818,51.11499538)(412.85138794,51.195)
}
}
{
\newrgbcolor{curcolor}{0 0 0}
\pscustom[linestyle=none,fillstyle=solid,fillcolor=curcolor]
{
\newpath
\moveto(419.25802856,54.42)
\curveto(419.48802377,54.41999208)(419.61802364,54.35999214)(419.64802856,54.24)
\curveto(419.67802358,54.12999237)(419.69302357,53.96499254)(419.69302856,53.745)
\lineto(419.69302856,53.46)
\curveto(419.69302357,53.36999313)(419.66802359,53.29499321)(419.61802856,53.235)
\curveto(419.5580237,53.15499334)(419.47302379,53.10999339)(419.36302856,53.1)
\curveto(419.25302401,53.0999934)(419.14302412,53.08499341)(419.03302856,53.055)
\curveto(418.89302437,53.02499347)(418.7580245,52.99499351)(418.62802856,52.965)
\curveto(418.50802475,52.93499357)(418.39302487,52.89499361)(418.28302856,52.845)
\curveto(417.99302527,52.71499378)(417.7580255,52.53499397)(417.57802856,52.305)
\curveto(417.39802586,52.08499441)(417.24302602,51.82999467)(417.11302856,51.54)
\curveto(417.07302619,51.42999507)(417.04302622,51.31499518)(417.02302856,51.195)
\curveto(417.00302626,51.08499541)(416.97802628,50.96999553)(416.94802856,50.85)
\curveto(416.93802632,50.7999957)(416.93302633,50.74999575)(416.93302856,50.7)
\curveto(416.94302632,50.64999585)(416.94302632,50.5999959)(416.93302856,50.55)
\curveto(416.90302636,50.42999607)(416.88802637,50.28999621)(416.88802856,50.13)
\curveto(416.89802636,49.97999652)(416.90302636,49.83499667)(416.90302856,49.695)
\lineto(416.90302856,47.85)
\lineto(416.90302856,47.505)
\curveto(416.90302636,47.38499912)(416.89802636,47.26999923)(416.88802856,47.16)
\curveto(416.87802638,47.04999945)(416.87302639,46.95499955)(416.87302856,46.875)
\curveto(416.88302638,46.79499971)(416.8630264,46.72499978)(416.81302856,46.665)
\curveto(416.7630265,46.59499991)(416.68302658,46.55499995)(416.57302856,46.545)
\curveto(416.47302679,46.53499996)(416.3630269,46.52999997)(416.24302856,46.53)
\lineto(415.97302856,46.53)
\curveto(415.92302734,46.54999995)(415.87302739,46.56499994)(415.82302856,46.575)
\curveto(415.78302748,46.59499991)(415.75302751,46.61999988)(415.73302856,46.65)
\curveto(415.68302758,46.71999978)(415.65302761,46.80499969)(415.64302856,46.905)
\lineto(415.64302856,47.235)
\lineto(415.64302856,48.39)
\lineto(415.64302856,52.545)
\lineto(415.64302856,53.58)
\lineto(415.64302856,53.88)
\curveto(415.65302761,53.97999252)(415.68302758,54.06499244)(415.73302856,54.135)
\curveto(415.7630275,54.17499233)(415.81302745,54.2049923)(415.88302856,54.225)
\curveto(415.9630273,54.24499225)(416.04802721,54.25499224)(416.13802856,54.255)
\curveto(416.22802703,54.26499224)(416.31802694,54.26499224)(416.40802856,54.255)
\curveto(416.49802676,54.24499225)(416.56802669,54.22999227)(416.61802856,54.21)
\curveto(416.69802656,54.17999232)(416.74802651,54.11999238)(416.76802856,54.03)
\curveto(416.79802646,53.94999255)(416.81302645,53.85999264)(416.81302856,53.76)
\lineto(416.81302856,53.46)
\curveto(416.81302645,53.35999314)(416.83302643,53.26999323)(416.87302856,53.19)
\curveto(416.88302638,53.16999333)(416.89302637,53.15499334)(416.90302856,53.145)
\lineto(416.94802856,53.1)
\curveto(417.0580262,53.0999934)(417.14802611,53.14499335)(417.21802856,53.235)
\curveto(417.28802597,53.33499317)(417.34802591,53.41499308)(417.39802856,53.475)
\lineto(417.48802856,53.565)
\curveto(417.57802568,53.67499283)(417.70302556,53.78999271)(417.86302856,53.91)
\curveto(418.02302524,54.02999247)(418.17302509,54.11999238)(418.31302856,54.18)
\curveto(418.40302486,54.22999227)(418.49802476,54.26499224)(418.59802856,54.285)
\curveto(418.69802456,54.31499218)(418.80302446,54.34499215)(418.91302856,54.375)
\curveto(418.97302429,54.38499211)(419.03302423,54.38999211)(419.09302856,54.39)
\curveto(419.15302411,54.3999921)(419.20802405,54.40999209)(419.25802856,54.42)
}
}
{
\newrgbcolor{curcolor}{0 0 0}
\pscustom[linestyle=none,fillstyle=solid,fillcolor=curcolor]
{
\newpath
\moveto(420.90779419,55.74)
\curveto(420.82779307,55.7999907)(420.78279311,55.9049906)(420.77279419,56.055)
\lineto(420.77279419,56.52)
\lineto(420.77279419,56.775)
\curveto(420.77279312,56.86498964)(420.78779311,56.93998956)(420.81779419,57)
\curveto(420.85779304,57.07998942)(420.93779296,57.13998936)(421.05779419,57.18)
\curveto(421.07779282,57.18998931)(421.0977928,57.18998931)(421.11779419,57.18)
\curveto(421.14779275,57.17998932)(421.17279272,57.18498931)(421.19279419,57.195)
\curveto(421.36279253,57.1949893)(421.52279237,57.18998931)(421.67279419,57.18)
\curveto(421.82279207,57.16998933)(421.92279197,57.10998939)(421.97279419,57)
\curveto(422.00279189,56.93998956)(422.01779188,56.86498964)(422.01779419,56.775)
\lineto(422.01779419,56.52)
\curveto(422.01779188,56.33999016)(422.01279188,56.16999033)(422.00279419,56.01)
\curveto(422.00279189,55.84999065)(421.93779196,55.74499076)(421.80779419,55.695)
\curveto(421.75779214,55.67499083)(421.70279219,55.66499084)(421.64279419,55.665)
\lineto(421.47779419,55.665)
\lineto(421.16279419,55.665)
\curveto(421.06279283,55.66499084)(420.97779292,55.68999081)(420.90779419,55.74)
\moveto(422.01779419,47.235)
\lineto(422.01779419,46.92)
\curveto(422.02779187,46.81999968)(422.00779189,46.73999976)(421.95779419,46.68)
\curveto(421.92779197,46.61999988)(421.88279201,46.57999992)(421.82279419,46.56)
\curveto(421.76279213,46.54999995)(421.6927922,46.53499996)(421.61279419,46.515)
\lineto(421.38779419,46.515)
\curveto(421.25779264,46.51499998)(421.14279275,46.51999998)(421.04279419,46.53)
\curveto(420.95279294,46.54999995)(420.88279301,46.5999999)(420.83279419,46.68)
\curveto(420.7927931,46.73999976)(420.77279312,46.81499968)(420.77279419,46.905)
\lineto(420.77279419,47.19)
\lineto(420.77279419,53.535)
\lineto(420.77279419,53.85)
\curveto(420.77279312,53.95999254)(420.7977931,54.04499245)(420.84779419,54.105)
\curveto(420.87779302,54.15499234)(420.91779298,54.18499231)(420.96779419,54.195)
\curveto(421.01779288,54.2049923)(421.07279282,54.21999228)(421.13279419,54.24)
\curveto(421.15279274,54.23999226)(421.17279272,54.23499227)(421.19279419,54.225)
\curveto(421.22279267,54.22499227)(421.24779265,54.22999227)(421.26779419,54.24)
\curveto(421.3977925,54.23999226)(421.52779237,54.23499227)(421.65779419,54.225)
\curveto(421.7977921,54.22499227)(421.892792,54.18499231)(421.94279419,54.105)
\curveto(421.9927919,54.04499245)(422.01779188,53.96499254)(422.01779419,53.865)
\lineto(422.01779419,53.58)
\lineto(422.01779419,47.235)
}
}
{
\newrgbcolor{curcolor}{0 0 0}
\pscustom[linestyle=none,fillstyle=solid,fillcolor=curcolor]
{
\newpath
\moveto(430.84763794,47.07)
\curveto(430.87763011,46.90999959)(430.86263012,46.77499972)(430.80263794,46.665)
\curveto(430.74263024,46.56499994)(430.66263032,46.49000001)(430.56263794,46.44)
\curveto(430.51263047,46.42000008)(430.45763053,46.41000009)(430.39763794,46.41)
\curveto(430.34763064,46.41000009)(430.29263069,46.4000001)(430.23263794,46.38)
\curveto(430.01263097,46.33000017)(429.79263119,46.34500015)(429.57263794,46.425)
\curveto(429.36263162,46.49500001)(429.21763177,46.58499992)(429.13763794,46.695)
\curveto(429.0876319,46.76499974)(429.04263194,46.84499965)(429.00263794,46.935)
\curveto(428.96263202,47.03499946)(428.91263207,47.11499938)(428.85263794,47.175)
\curveto(428.83263215,47.19499931)(428.80763218,47.21499928)(428.77763794,47.235)
\curveto(428.75763223,47.25499925)(428.72763226,47.25999924)(428.68763794,47.25)
\curveto(428.57763241,47.21999928)(428.47263251,47.16499934)(428.37263794,47.085)
\curveto(428.2826327,47.00499949)(428.19263279,46.93499956)(428.10263794,46.875)
\curveto(427.97263301,46.79499971)(427.83263315,46.71999978)(427.68263794,46.65)
\curveto(427.53263345,46.58999991)(427.37263361,46.53499996)(427.20263794,46.485)
\curveto(427.10263388,46.45500005)(426.99263399,46.43500006)(426.87263794,46.425)
\curveto(426.76263422,46.41500008)(426.65263433,46.4000001)(426.54263794,46.38)
\curveto(426.49263449,46.37000013)(426.44763454,46.36500014)(426.40763794,46.365)
\lineto(426.30263794,46.365)
\curveto(426.19263479,46.34500015)(426.0876349,46.34500015)(425.98763794,46.365)
\lineto(425.85263794,46.365)
\curveto(425.80263518,46.37500012)(425.75263523,46.38000012)(425.70263794,46.38)
\curveto(425.65263533,46.38000012)(425.60763538,46.39000011)(425.56763794,46.41)
\curveto(425.52763546,46.42000008)(425.49263549,46.42500008)(425.46263794,46.425)
\curveto(425.44263554,46.41500008)(425.41763557,46.41500008)(425.38763794,46.425)
\lineto(425.14763794,46.485)
\curveto(425.06763592,46.49500001)(424.99263599,46.51499998)(424.92263794,46.545)
\curveto(424.62263636,46.67499982)(424.37763661,46.81999968)(424.18763794,46.98)
\curveto(424.00763698,47.14999935)(423.85763713,47.38499912)(423.73763794,47.685)
\curveto(423.64763734,47.90499859)(423.60263738,48.16999833)(423.60263794,48.48)
\lineto(423.60263794,48.795)
\curveto(423.61263737,48.84499765)(423.61763737,48.89499761)(423.61763794,48.945)
\lineto(423.64763794,49.125)
\lineto(423.76763794,49.455)
\curveto(423.80763718,49.56499694)(423.85763713,49.66499684)(423.91763794,49.755)
\curveto(424.09763689,50.04499645)(424.34263664,50.25999624)(424.65263794,50.4)
\curveto(424.96263602,50.53999596)(425.30263568,50.66499584)(425.67263794,50.775)
\curveto(425.81263517,50.81499568)(425.95763503,50.84499565)(426.10763794,50.865)
\curveto(426.25763473,50.88499561)(426.40763458,50.90999559)(426.55763794,50.94)
\curveto(426.62763436,50.95999554)(426.69263429,50.96999553)(426.75263794,50.97)
\curveto(426.82263416,50.96999553)(426.89763409,50.97999552)(426.97763794,51)
\curveto(427.04763394,51.01999548)(427.11763387,51.02999547)(427.18763794,51.03)
\curveto(427.25763373,51.03999546)(427.33263365,51.05499544)(427.41263794,51.075)
\curveto(427.66263332,51.13499537)(427.89763309,51.18499531)(428.11763794,51.225)
\curveto(428.33763265,51.27499522)(428.51263247,51.38999511)(428.64263794,51.57)
\curveto(428.70263228,51.64999485)(428.75263223,51.74999475)(428.79263794,51.87)
\curveto(428.83263215,51.9999945)(428.83263215,52.13999436)(428.79263794,52.29)
\curveto(428.73263225,52.52999397)(428.64263234,52.71999378)(428.52263794,52.86)
\curveto(428.41263257,52.9999935)(428.25263273,53.10999339)(428.04263794,53.19)
\curveto(427.92263306,53.23999326)(427.77763321,53.27499323)(427.60763794,53.295)
\curveto(427.44763354,53.31499318)(427.27763371,53.32499317)(427.09763794,53.325)
\curveto(426.91763407,53.32499317)(426.74263424,53.31499318)(426.57263794,53.295)
\curveto(426.40263458,53.27499323)(426.25763473,53.24499325)(426.13763794,53.205)
\curveto(425.96763502,53.14499335)(425.80263518,53.05999344)(425.64263794,52.95)
\curveto(425.56263542,52.88999361)(425.4876355,52.80999369)(425.41763794,52.71)
\curveto(425.35763563,52.61999388)(425.30263568,52.51999398)(425.25263794,52.41)
\curveto(425.22263576,52.32999417)(425.19263579,52.24499425)(425.16263794,52.155)
\curveto(425.14263584,52.06499444)(425.09763589,51.99499451)(425.02763794,51.945)
\curveto(424.987636,51.91499458)(424.91763607,51.88999461)(424.81763794,51.87)
\curveto(424.72763626,51.85999464)(424.63263635,51.85499464)(424.53263794,51.855)
\curveto(424.43263655,51.85499464)(424.33263665,51.85999464)(424.23263794,51.87)
\curveto(424.14263684,51.88999461)(424.07763691,51.91499458)(424.03763794,51.945)
\curveto(423.99763699,51.97499452)(423.96763702,52.02499447)(423.94763794,52.095)
\curveto(423.92763706,52.16499434)(423.92763706,52.23999426)(423.94763794,52.32)
\curveto(423.97763701,52.44999405)(424.00763698,52.56999393)(424.03763794,52.68)
\curveto(424.07763691,52.7999937)(424.12263686,52.91499358)(424.17263794,53.025)
\curveto(424.36263662,53.37499313)(424.60263638,53.64499285)(424.89263794,53.835)
\curveto(425.1826358,54.03499247)(425.54263544,54.19499231)(425.97263794,54.315)
\curveto(426.07263491,54.33499217)(426.17263481,54.34999215)(426.27263794,54.36)
\curveto(426.3826346,54.36999213)(426.49263449,54.38499211)(426.60263794,54.405)
\curveto(426.64263434,54.41499208)(426.70763428,54.41499208)(426.79763794,54.405)
\curveto(426.8876341,54.4049921)(426.94263404,54.41499208)(426.96263794,54.435)
\curveto(427.66263332,54.44499206)(428.27263271,54.36499214)(428.79263794,54.195)
\curveto(429.31263167,54.02499247)(429.67763131,53.6999928)(429.88763794,53.22)
\curveto(429.97763101,53.01999348)(430.02763096,52.78499371)(430.03763794,52.515)
\curveto(430.05763093,52.25499424)(430.06763092,51.97999452)(430.06763794,51.69)
\lineto(430.06763794,48.375)
\curveto(430.06763092,48.23499827)(430.07263091,48.0999984)(430.08263794,47.97)
\curveto(430.09263089,47.83999866)(430.12263086,47.73499877)(430.17263794,47.655)
\curveto(430.22263076,47.58499892)(430.2876307,47.53499896)(430.36763794,47.505)
\curveto(430.45763053,47.46499904)(430.54263044,47.43499906)(430.62263794,47.415)
\curveto(430.70263028,47.40499909)(430.76263022,47.35999914)(430.80263794,47.28)
\curveto(430.82263016,47.24999925)(430.83263015,47.21999928)(430.83263794,47.19)
\curveto(430.83263015,47.15999934)(430.83763015,47.11999938)(430.84763794,47.07)
\moveto(428.70263794,48.735)
\curveto(428.76263222,48.87499762)(428.79263219,49.03499747)(428.79263794,49.215)
\curveto(428.80263218,49.40499709)(428.80763218,49.5999969)(428.80763794,49.8)
\curveto(428.80763218,49.90999659)(428.80263218,50.00999649)(428.79263794,50.1)
\curveto(428.7826322,50.18999631)(428.74263224,50.25999624)(428.67263794,50.31)
\curveto(428.64263234,50.32999617)(428.57263241,50.33999616)(428.46263794,50.34)
\curveto(428.44263254,50.31999618)(428.40763258,50.30999619)(428.35763794,50.31)
\curveto(428.30763268,50.30999619)(428.26263272,50.2999962)(428.22263794,50.28)
\curveto(428.14263284,50.25999624)(428.05263293,50.23999626)(427.95263794,50.22)
\lineto(427.65263794,50.16)
\curveto(427.62263336,50.15999634)(427.5876334,50.15499634)(427.54763794,50.145)
\lineto(427.44263794,50.145)
\curveto(427.29263369,50.1049964)(427.12763386,50.07999642)(426.94763794,50.07)
\curveto(426.77763421,50.06999643)(426.61763437,50.04999645)(426.46763794,50.01)
\curveto(426.3876346,49.98999651)(426.31263467,49.96999653)(426.24263794,49.95)
\curveto(426.1826348,49.93999656)(426.11263487,49.92499658)(426.03263794,49.905)
\curveto(425.87263511,49.85499665)(425.72263526,49.78999671)(425.58263794,49.71)
\curveto(425.44263554,49.63999686)(425.32263566,49.54999695)(425.22263794,49.44)
\curveto(425.12263586,49.32999717)(425.04763594,49.19499731)(424.99763794,49.035)
\curveto(424.94763604,48.88499761)(424.92763606,48.6999978)(424.93763794,48.48)
\curveto(424.93763605,48.37999812)(424.95263603,48.28499821)(424.98263794,48.195)
\curveto(425.02263596,48.11499838)(425.06763592,48.03999846)(425.11763794,47.97)
\curveto(425.19763579,47.85999864)(425.30263568,47.76499874)(425.43263794,47.685)
\curveto(425.56263542,47.61499888)(425.70263528,47.55499895)(425.85263794,47.505)
\curveto(425.90263508,47.49499901)(425.95263503,47.48999901)(426.00263794,47.49)
\curveto(426.05263493,47.48999901)(426.10263488,47.48499902)(426.15263794,47.475)
\curveto(426.22263476,47.45499905)(426.30763468,47.43999906)(426.40763794,47.43)
\curveto(426.51763447,47.42999907)(426.60763438,47.43999906)(426.67763794,47.46)
\curveto(426.73763425,47.47999902)(426.79763419,47.48499902)(426.85763794,47.475)
\curveto(426.91763407,47.47499902)(426.97763401,47.48499902)(427.03763794,47.505)
\curveto(427.11763387,47.52499898)(427.19263379,47.53999896)(427.26263794,47.55)
\curveto(427.34263364,47.55999894)(427.41763357,47.57999892)(427.48763794,47.61)
\curveto(427.77763321,47.72999877)(428.02263296,47.87499862)(428.22263794,48.045)
\curveto(428.43263255,48.21499828)(428.59263239,48.44499805)(428.70263794,48.735)
}
}
{
\newrgbcolor{curcolor}{0 0 0}
\pscustom[linestyle=none,fillstyle=solid,fillcolor=curcolor]
{
\newpath
\moveto(434.44927856,54.42)
\curveto(435.1692745,54.42999207)(435.77427389,54.34499215)(436.26427856,54.165)
\curveto(436.75427291,53.99499251)(437.13427253,53.68999281)(437.40427856,53.25)
\curveto(437.47427219,53.13999336)(437.52927214,53.02499347)(437.56927856,52.905)
\curveto(437.60927206,52.79499371)(437.64927202,52.66999383)(437.68927856,52.53)
\curveto(437.70927196,52.45999404)(437.71427195,52.38499411)(437.70427856,52.305)
\curveto(437.69427197,52.23499427)(437.67927199,52.17999432)(437.65927856,52.14)
\curveto(437.63927203,52.11999438)(437.61427205,52.0999944)(437.58427856,52.08)
\curveto(437.55427211,52.06999443)(437.52927214,52.05499444)(437.50927856,52.035)
\curveto(437.45927221,52.01499448)(437.40927226,52.00999449)(437.35927856,52.02)
\curveto(437.30927236,52.02999447)(437.25927241,52.02999447)(437.20927856,52.02)
\curveto(437.12927254,51.9999945)(437.02427264,51.99499451)(436.89427856,52.005)
\curveto(436.7642729,52.02499447)(436.67427299,52.04999445)(436.62427856,52.08)
\curveto(436.54427312,52.12999437)(436.48927318,52.19499431)(436.45927856,52.275)
\curveto(436.43927323,52.36499414)(436.40427326,52.44999405)(436.35427856,52.53)
\curveto(436.2642734,52.68999381)(436.13927353,52.83499367)(435.97927856,52.965)
\curveto(435.8692738,53.04499345)(435.74927392,53.1049934)(435.61927856,53.145)
\curveto(435.48927418,53.18499331)(435.34927432,53.22499327)(435.19927856,53.265)
\curveto(435.14927452,53.28499321)(435.09927457,53.28999321)(435.04927856,53.28)
\curveto(434.99927467,53.27999322)(434.94927472,53.28499321)(434.89927856,53.295)
\curveto(434.83927483,53.31499318)(434.7642749,53.32499317)(434.67427856,53.325)
\curveto(434.58427508,53.32499317)(434.50927516,53.31499318)(434.44927856,53.295)
\lineto(434.35927856,53.295)
\lineto(434.20927856,53.265)
\curveto(434.15927551,53.26499324)(434.10927556,53.25999324)(434.05927856,53.25)
\curveto(433.79927587,53.18999331)(433.58427608,53.1049934)(433.41427856,52.995)
\curveto(433.24427642,52.88499361)(433.12927654,52.6999938)(433.06927856,52.44)
\curveto(433.04927662,52.36999413)(433.04427662,52.2999942)(433.05427856,52.23)
\curveto(433.07427659,52.15999434)(433.09427657,52.0999944)(433.11427856,52.05)
\curveto(433.17427649,51.8999946)(433.24427642,51.78999471)(433.32427856,51.72)
\curveto(433.41427625,51.65999484)(433.52427614,51.58999491)(433.65427856,51.51)
\curveto(433.81427585,51.40999509)(433.99427567,51.33499517)(434.19427856,51.285)
\curveto(434.39427527,51.24499525)(434.59427507,51.19499531)(434.79427856,51.135)
\curveto(434.92427474,51.09499541)(435.05427461,51.06499544)(435.18427856,51.045)
\curveto(435.31427435,51.02499548)(435.44427422,50.99499551)(435.57427856,50.955)
\curveto(435.78427388,50.89499561)(435.98927368,50.83499567)(436.18927856,50.775)
\curveto(436.38927328,50.72499578)(436.58927308,50.65999584)(436.78927856,50.58)
\lineto(436.93927856,50.52)
\curveto(436.98927268,50.499996)(437.03927263,50.47499602)(437.08927856,50.445)
\curveto(437.28927238,50.32499618)(437.4642722,50.18999631)(437.61427856,50.04)
\curveto(437.7642719,49.88999661)(437.88927178,49.6999968)(437.98927856,49.47)
\curveto(438.00927166,49.3999971)(438.02927164,49.30499719)(438.04927856,49.185)
\curveto(438.0692716,49.11499738)(438.07927159,49.03999746)(438.07927856,48.96)
\curveto(438.08927158,48.88999761)(438.09427157,48.80999769)(438.09427856,48.72)
\lineto(438.09427856,48.57)
\curveto(438.07427159,48.499998)(438.0642716,48.42999807)(438.06427856,48.36)
\curveto(438.0642716,48.28999821)(438.05427161,48.21999828)(438.03427856,48.15)
\curveto(438.00427166,48.03999846)(437.9692717,47.93499857)(437.92927856,47.835)
\curveto(437.88927178,47.73499877)(437.84427182,47.64499885)(437.79427856,47.565)
\curveto(437.63427203,47.30499919)(437.42927224,47.09499941)(437.17927856,46.935)
\curveto(436.92927274,46.78499972)(436.64927302,46.65499985)(436.33927856,46.545)
\curveto(436.24927342,46.51499998)(436.15427351,46.49500001)(436.05427856,46.485)
\curveto(435.9642737,46.46500004)(435.87427379,46.44000006)(435.78427856,46.41)
\curveto(435.68427398,46.39000011)(435.58427408,46.38000012)(435.48427856,46.38)
\curveto(435.38427428,46.38000012)(435.28427438,46.37000013)(435.18427856,46.35)
\lineto(435.03427856,46.35)
\curveto(434.98427468,46.34000016)(434.91427475,46.33500016)(434.82427856,46.335)
\curveto(434.73427493,46.33500016)(434.664275,46.34000016)(434.61427856,46.35)
\lineto(434.44927856,46.35)
\curveto(434.38927528,46.37000013)(434.32427534,46.38000012)(434.25427856,46.38)
\curveto(434.18427548,46.37000013)(434.12427554,46.37500012)(434.07427856,46.395)
\curveto(434.02427564,46.40500009)(433.95927571,46.41000009)(433.87927856,46.41)
\lineto(433.63927856,46.47)
\curveto(433.5692761,46.48000002)(433.49427617,46.5)(433.41427856,46.53)
\curveto(433.10427656,46.62999987)(432.83427683,46.75499975)(432.60427856,46.905)
\curveto(432.37427729,47.05499945)(432.17427749,47.24999925)(432.00427856,47.49)
\curveto(431.91427775,47.61999888)(431.83927783,47.75499875)(431.77927856,47.895)
\curveto(431.71927795,48.03499847)(431.664278,48.18999831)(431.61427856,48.36)
\curveto(431.59427807,48.41999808)(431.58427808,48.48999801)(431.58427856,48.57)
\curveto(431.59427807,48.65999784)(431.60927806,48.72999777)(431.62927856,48.78)
\curveto(431.65927801,48.81999768)(431.70927796,48.85999764)(431.77927856,48.9)
\curveto(431.82927784,48.91999758)(431.89927777,48.92999757)(431.98927856,48.93)
\curveto(432.07927759,48.93999756)(432.1692775,48.93999756)(432.25927856,48.93)
\curveto(432.34927732,48.91999758)(432.43427723,48.90499759)(432.51427856,48.885)
\curveto(432.60427706,48.87499762)(432.664277,48.85999764)(432.69427856,48.84)
\curveto(432.7642769,48.78999771)(432.80927686,48.71499778)(432.82927856,48.615)
\curveto(432.85927681,48.52499798)(432.89427677,48.43999806)(432.93427856,48.36)
\curveto(433.03427663,48.13999836)(433.1692765,47.96999853)(433.33927856,47.85)
\curveto(433.45927621,47.75999874)(433.59427607,47.68999881)(433.74427856,47.64)
\curveto(433.89427577,47.58999891)(434.05427561,47.53999896)(434.22427856,47.49)
\lineto(434.53927856,47.445)
\lineto(434.62927856,47.445)
\curveto(434.69927497,47.42499908)(434.78927488,47.41499908)(434.89927856,47.415)
\curveto(435.01927465,47.41499908)(435.11927455,47.42499908)(435.19927856,47.445)
\curveto(435.2692744,47.44499905)(435.32427434,47.44999905)(435.36427856,47.46)
\curveto(435.42427424,47.46999903)(435.48427418,47.47499902)(435.54427856,47.475)
\curveto(435.60427406,47.48499902)(435.65927401,47.49499901)(435.70927856,47.505)
\curveto(435.99927367,47.58499892)(436.22927344,47.68999881)(436.39927856,47.82)
\curveto(436.5692731,47.94999855)(436.68927298,48.16999833)(436.75927856,48.48)
\curveto(436.77927289,48.52999797)(436.78427288,48.58499791)(436.77427856,48.645)
\curveto(436.7642729,48.70499779)(436.75427291,48.74999775)(436.74427856,48.78)
\curveto(436.69427297,48.96999753)(436.62427304,49.10999739)(436.53427856,49.2)
\curveto(436.44427322,49.2999972)(436.32927334,49.38999711)(436.18927856,49.47)
\curveto(436.09927357,49.52999697)(435.99927367,49.57999692)(435.88927856,49.62)
\lineto(435.55927856,49.74)
\curveto(435.52927414,49.74999675)(435.49927417,49.75499675)(435.46927856,49.755)
\curveto(435.44927422,49.75499675)(435.42427424,49.76499674)(435.39427856,49.785)
\curveto(435.05427461,49.89499661)(434.69927497,49.97499652)(434.32927856,50.025)
\curveto(433.9692757,50.08499641)(433.62927604,50.17999632)(433.30927856,50.31)
\curveto(433.20927646,50.34999615)(433.11427655,50.38499611)(433.02427856,50.415)
\curveto(432.93427673,50.44499605)(432.84927682,50.48499601)(432.76927856,50.535)
\curveto(432.57927709,50.64499585)(432.40427726,50.76999573)(432.24427856,50.91)
\curveto(432.08427758,51.04999545)(431.95927771,51.22499528)(431.86927856,51.435)
\curveto(431.83927783,51.504995)(431.81427785,51.57499492)(431.79427856,51.645)
\curveto(431.78427788,51.71499478)(431.7692779,51.78999471)(431.74927856,51.87)
\curveto(431.71927795,51.98999451)(431.70927796,52.12499438)(431.71927856,52.275)
\curveto(431.72927794,52.43499407)(431.74427792,52.56999393)(431.76427856,52.68)
\curveto(431.78427788,52.72999377)(431.79427787,52.76999373)(431.79427856,52.8)
\curveto(431.80427786,52.83999366)(431.81927785,52.87999362)(431.83927856,52.92)
\curveto(431.92927774,53.14999335)(432.04927762,53.34999315)(432.19927856,53.52)
\curveto(432.35927731,53.68999281)(432.53927713,53.83999266)(432.73927856,53.97)
\curveto(432.88927678,54.05999244)(433.05427661,54.12999237)(433.23427856,54.18)
\curveto(433.41427625,54.23999226)(433.60427606,54.29499221)(433.80427856,54.345)
\curveto(433.87427579,54.35499214)(433.93927573,54.36499214)(433.99927856,54.375)
\curveto(434.0692756,54.38499211)(434.14427552,54.39499211)(434.22427856,54.405)
\curveto(434.25427541,54.41499208)(434.29427537,54.41499208)(434.34427856,54.405)
\curveto(434.39427527,54.39499211)(434.42927524,54.3999921)(434.44927856,54.42)
}
}
{
\newrgbcolor{curcolor}{0 0 0}
\pscustom[linestyle=none,fillstyle=solid,fillcolor=curcolor]
{
\newpath
\moveto(73.44210083,86.16295776)
\lineto(73.44210083,85.90795776)
\curveto(73.45209313,85.827953)(73.44709313,85.75295307)(73.42710083,85.68295776)
\lineto(73.42710083,85.44295776)
\lineto(73.42710083,85.27795776)
\curveto(73.40709317,85.17795365)(73.39709318,85.07295375)(73.39710083,84.96295776)
\curveto(73.39709318,84.86295396)(73.38709319,84.76295406)(73.36710083,84.66295776)
\lineto(73.36710083,84.51295776)
\curveto(73.33709324,84.37295445)(73.31709326,84.23295459)(73.30710083,84.09295776)
\curveto(73.29709328,83.96295486)(73.27209331,83.83295499)(73.23210083,83.70295776)
\curveto(73.21209337,83.6229552)(73.19209339,83.53795529)(73.17210083,83.44795776)
\lineto(73.11210083,83.20795776)
\lineto(72.99210083,82.90795776)
\curveto(72.96209362,82.81795601)(72.92709365,82.7279561)(72.88710083,82.63795776)
\curveto(72.78709379,82.41795641)(72.65209393,82.20295662)(72.48210083,81.99295776)
\curveto(72.32209426,81.78295704)(72.14709443,81.61295721)(71.95710083,81.48295776)
\curveto(71.90709467,81.44295738)(71.84709473,81.40295742)(71.77710083,81.36295776)
\curveto(71.71709486,81.33295749)(71.65709492,81.29795753)(71.59710083,81.25795776)
\curveto(71.51709506,81.20795762)(71.42209516,81.16795766)(71.31210083,81.13795776)
\curveto(71.20209538,81.10795772)(71.09709548,81.07795775)(70.99710083,81.04795776)
\curveto(70.88709569,81.00795782)(70.7770958,80.98295784)(70.66710083,80.97295776)
\curveto(70.55709602,80.96295786)(70.44209614,80.94795788)(70.32210083,80.92795776)
\curveto(70.2820963,80.91795791)(70.23709634,80.91795791)(70.18710083,80.92795776)
\curveto(70.14709643,80.9279579)(70.10709647,80.9229579)(70.06710083,80.91295776)
\curveto(70.02709655,80.90295792)(69.97209661,80.89795793)(69.90210083,80.89795776)
\curveto(69.83209675,80.89795793)(69.7820968,80.90295792)(69.75210083,80.91295776)
\curveto(69.70209688,80.93295789)(69.65709692,80.93795789)(69.61710083,80.92795776)
\curveto(69.577097,80.91795791)(69.54209704,80.91795791)(69.51210083,80.92795776)
\lineto(69.42210083,80.92795776)
\curveto(69.36209722,80.94795788)(69.29709728,80.96295786)(69.22710083,80.97295776)
\curveto(69.16709741,80.97295785)(69.10209748,80.97795785)(69.03210083,80.98795776)
\curveto(68.86209772,81.03795779)(68.70209788,81.08795774)(68.55210083,81.13795776)
\curveto(68.40209818,81.18795764)(68.25709832,81.25295757)(68.11710083,81.33295776)
\curveto(68.06709851,81.37295745)(68.01209857,81.40295742)(67.95210083,81.42295776)
\curveto(67.90209868,81.45295737)(67.85209873,81.48795734)(67.80210083,81.52795776)
\curveto(67.56209902,81.70795712)(67.36209922,81.9279569)(67.20210083,82.18795776)
\curveto(67.04209954,82.44795638)(66.90209968,82.73295609)(66.78210083,83.04295776)
\curveto(66.72209986,83.18295564)(66.6770999,83.3229555)(66.64710083,83.46295776)
\curveto(66.61709996,83.61295521)(66.5821,83.76795506)(66.54210083,83.92795776)
\curveto(66.52210006,84.03795479)(66.50710007,84.14795468)(66.49710083,84.25795776)
\curveto(66.48710009,84.36795446)(66.47210011,84.47795435)(66.45210083,84.58795776)
\curveto(66.44210014,84.6279542)(66.43710014,84.66795416)(66.43710083,84.70795776)
\curveto(66.44710013,84.74795408)(66.44710013,84.78795404)(66.43710083,84.82795776)
\curveto(66.42710015,84.87795395)(66.42210016,84.9279539)(66.42210083,84.97795776)
\lineto(66.42210083,85.14295776)
\curveto(66.40210018,85.19295363)(66.39710018,85.24295358)(66.40710083,85.29295776)
\curveto(66.41710016,85.35295347)(66.41710016,85.40795342)(66.40710083,85.45795776)
\curveto(66.39710018,85.49795333)(66.39710018,85.54295328)(66.40710083,85.59295776)
\curveto(66.41710016,85.64295318)(66.41210017,85.69295313)(66.39210083,85.74295776)
\curveto(66.37210021,85.81295301)(66.36710021,85.88795294)(66.37710083,85.96795776)
\curveto(66.38710019,86.05795277)(66.39210019,86.14295268)(66.39210083,86.22295776)
\curveto(66.39210019,86.31295251)(66.38710019,86.41295241)(66.37710083,86.52295776)
\curveto(66.36710021,86.64295218)(66.37210021,86.74295208)(66.39210083,86.82295776)
\lineto(66.39210083,87.10795776)
\lineto(66.43710083,87.73795776)
\curveto(66.44710013,87.83795099)(66.45710012,87.93295089)(66.46710083,88.02295776)
\lineto(66.49710083,88.32295776)
\curveto(66.51710006,88.37295045)(66.52210006,88.4229504)(66.51210083,88.47295776)
\curveto(66.51210007,88.53295029)(66.52210006,88.58795024)(66.54210083,88.63795776)
\curveto(66.59209999,88.80795002)(66.63209995,88.97294985)(66.66210083,89.13295776)
\curveto(66.69209989,89.30294952)(66.74209984,89.46294936)(66.81210083,89.61295776)
\curveto(67.00209958,90.07294875)(67.22209936,90.44794838)(67.47210083,90.73795776)
\curveto(67.73209885,91.0279478)(68.09209849,91.27294755)(68.55210083,91.47295776)
\curveto(68.6820979,91.5229473)(68.81209777,91.55794727)(68.94210083,91.57795776)
\curveto(69.0820975,91.59794723)(69.22209736,91.6229472)(69.36210083,91.65295776)
\curveto(69.43209715,91.66294716)(69.49709708,91.66794716)(69.55710083,91.66795776)
\curveto(69.61709696,91.66794716)(69.6820969,91.67294715)(69.75210083,91.68295776)
\curveto(70.582096,91.70294712)(71.25209533,91.55294727)(71.76210083,91.23295776)
\curveto(72.27209431,90.9229479)(72.65209393,90.48294834)(72.90210083,89.91295776)
\curveto(72.95209363,89.79294903)(72.99709358,89.66794916)(73.03710083,89.53795776)
\curveto(73.0770935,89.40794942)(73.12209346,89.27294955)(73.17210083,89.13295776)
\curveto(73.19209339,89.05294977)(73.20709337,88.96794986)(73.21710083,88.87795776)
\lineto(73.27710083,88.63795776)
\curveto(73.30709327,88.5279503)(73.32209326,88.41795041)(73.32210083,88.30795776)
\curveto(73.33209325,88.19795063)(73.34709323,88.08795074)(73.36710083,87.97795776)
\curveto(73.38709319,87.9279509)(73.39209319,87.88295094)(73.38210083,87.84295776)
\curveto(73.3820932,87.80295102)(73.38709319,87.76295106)(73.39710083,87.72295776)
\curveto(73.40709317,87.67295115)(73.40709317,87.61795121)(73.39710083,87.55795776)
\curveto(73.39709318,87.50795132)(73.40209318,87.45795137)(73.41210083,87.40795776)
\lineto(73.41210083,87.27295776)
\curveto(73.43209315,87.21295161)(73.43209315,87.14295168)(73.41210083,87.06295776)
\curveto(73.40209318,86.99295183)(73.40709317,86.9279519)(73.42710083,86.86795776)
\curveto(73.43709314,86.83795199)(73.44209314,86.79795203)(73.44210083,86.74795776)
\lineto(73.44210083,86.62795776)
\lineto(73.44210083,86.16295776)
\moveto(71.89710083,83.83795776)
\curveto(71.99709458,84.15795467)(72.05709452,84.5229543)(72.07710083,84.93295776)
\curveto(72.09709448,85.34295348)(72.10709447,85.75295307)(72.10710083,86.16295776)
\curveto(72.10709447,86.59295223)(72.09709448,87.01295181)(72.07710083,87.42295776)
\curveto(72.05709452,87.83295099)(72.01209457,88.21795061)(71.94210083,88.57795776)
\curveto(71.87209471,88.93794989)(71.76209482,89.25794957)(71.61210083,89.53795776)
\curveto(71.47209511,89.827949)(71.2770953,90.06294876)(71.02710083,90.24295776)
\curveto(70.86709571,90.35294847)(70.68709589,90.43294839)(70.48710083,90.48295776)
\curveto(70.28709629,90.54294828)(70.04209654,90.57294825)(69.75210083,90.57295776)
\curveto(69.73209685,90.55294827)(69.69709688,90.54294828)(69.64710083,90.54295776)
\curveto(69.59709698,90.55294827)(69.55709702,90.55294827)(69.52710083,90.54295776)
\curveto(69.44709713,90.5229483)(69.37209721,90.50294832)(69.30210083,90.48295776)
\curveto(69.24209734,90.47294835)(69.1770974,90.45294837)(69.10710083,90.42295776)
\curveto(68.83709774,90.30294852)(68.61709796,90.13294869)(68.44710083,89.91295776)
\curveto(68.28709829,89.70294912)(68.15209843,89.45794937)(68.04210083,89.17795776)
\curveto(67.99209859,89.06794976)(67.95209863,88.94794988)(67.92210083,88.81795776)
\curveto(67.90209868,88.69795013)(67.8770987,88.57295025)(67.84710083,88.44295776)
\curveto(67.82709875,88.39295043)(67.81709876,88.33795049)(67.81710083,88.27795776)
\curveto(67.81709876,88.2279506)(67.81209877,88.17795065)(67.80210083,88.12795776)
\curveto(67.79209879,88.03795079)(67.7820988,87.94295088)(67.77210083,87.84295776)
\curveto(67.76209882,87.75295107)(67.75209883,87.65795117)(67.74210083,87.55795776)
\curveto(67.74209884,87.47795135)(67.73709884,87.39295143)(67.72710083,87.30295776)
\lineto(67.72710083,87.06295776)
\lineto(67.72710083,86.88295776)
\curveto(67.71709886,86.85295197)(67.71209887,86.81795201)(67.71210083,86.77795776)
\lineto(67.71210083,86.64295776)
\lineto(67.71210083,86.19295776)
\curveto(67.71209887,86.11295271)(67.70709887,86.0279528)(67.69710083,85.93795776)
\curveto(67.69709888,85.85795297)(67.70709887,85.78295304)(67.72710083,85.71295776)
\lineto(67.72710083,85.44295776)
\curveto(67.72709885,85.4229534)(67.72209886,85.39295343)(67.71210083,85.35295776)
\curveto(67.71209887,85.3229535)(67.71709886,85.29795353)(67.72710083,85.27795776)
\curveto(67.73709884,85.17795365)(67.74209884,85.07795375)(67.74210083,84.97795776)
\curveto(67.75209883,84.88795394)(67.76209882,84.78795404)(67.77210083,84.67795776)
\curveto(67.80209878,84.55795427)(67.81709876,84.43295439)(67.81710083,84.30295776)
\curveto(67.82709875,84.18295464)(67.85209873,84.06795476)(67.89210083,83.95795776)
\curveto(67.97209861,83.65795517)(68.05709852,83.39295543)(68.14710083,83.16295776)
\curveto(68.24709833,82.93295589)(68.39209819,82.71795611)(68.58210083,82.51795776)
\curveto(68.79209779,82.31795651)(69.05709752,82.16795666)(69.37710083,82.06795776)
\curveto(69.41709716,82.04795678)(69.45209713,82.03795679)(69.48210083,82.03795776)
\curveto(69.52209706,82.04795678)(69.56709701,82.04295678)(69.61710083,82.02295776)
\curveto(69.65709692,82.01295681)(69.72709685,82.00295682)(69.82710083,81.99295776)
\curveto(69.93709664,81.98295684)(70.02209656,81.98795684)(70.08210083,82.00795776)
\curveto(70.15209643,82.0279568)(70.22209636,82.03795679)(70.29210083,82.03795776)
\curveto(70.36209622,82.04795678)(70.42709615,82.06295676)(70.48710083,82.08295776)
\curveto(70.68709589,82.14295668)(70.86709571,82.2279566)(71.02710083,82.33795776)
\curveto(71.05709552,82.35795647)(71.0820955,82.37795645)(71.10210083,82.39795776)
\lineto(71.16210083,82.45795776)
\curveto(71.20209538,82.47795635)(71.25209533,82.51795631)(71.31210083,82.57795776)
\curveto(71.41209517,82.71795611)(71.49709508,82.84795598)(71.56710083,82.96795776)
\curveto(71.63709494,83.08795574)(71.70709487,83.23295559)(71.77710083,83.40295776)
\curveto(71.80709477,83.47295535)(71.82709475,83.54295528)(71.83710083,83.61295776)
\curveto(71.85709472,83.68295514)(71.8770947,83.75795507)(71.89710083,83.83795776)
}
}
{
\newrgbcolor{curcolor}{0 0 0}
\pscustom[linestyle=none,fillstyle=solid,fillcolor=curcolor]
{
\newpath
\moveto(62.1524762,165.96866333)
\curveto(62.25247134,165.96865271)(62.34747125,165.95865272)(62.4374762,165.93866333)
\curveto(62.52747107,165.92865275)(62.592471,165.89865278)(62.6324762,165.84866333)
\curveto(62.6924709,165.76865291)(62.72247087,165.66365302)(62.7224762,165.53366333)
\lineto(62.7224762,165.14366333)
\lineto(62.7224762,163.64366333)
\lineto(62.7224762,157.25366333)
\lineto(62.7224762,156.08366333)
\lineto(62.7224762,155.76866333)
\curveto(62.73247086,155.66866301)(62.71747088,155.58866309)(62.6774762,155.52866333)
\curveto(62.62747097,155.44866323)(62.55247104,155.39866328)(62.4524762,155.37866333)
\curveto(62.36247123,155.36866331)(62.25247134,155.36366332)(62.1224762,155.36366333)
\lineto(61.8974762,155.36366333)
\curveto(61.81747178,155.3836633)(61.74747185,155.39866328)(61.6874762,155.40866333)
\curveto(61.62747197,155.42866325)(61.57747202,155.46866321)(61.5374762,155.52866333)
\curveto(61.4974721,155.58866309)(61.47747212,155.66366302)(61.4774762,155.75366333)
\lineto(61.4774762,156.05366333)
\lineto(61.4774762,157.14866333)
\lineto(61.4774762,162.48866333)
\curveto(61.45747214,162.5786561)(61.44247215,162.65365603)(61.4324762,162.71366333)
\curveto(61.43247216,162.7836559)(61.40247219,162.84365584)(61.3424762,162.89366333)
\curveto(61.27247232,162.94365574)(61.18247241,162.96865571)(61.0724762,162.96866333)
\curveto(60.97247262,162.9786557)(60.86247273,162.9836557)(60.7424762,162.98366333)
\lineto(59.6024762,162.98366333)
\lineto(59.1074762,162.98366333)
\curveto(58.94747465,162.99365569)(58.83747476,163.05365563)(58.7774762,163.16366333)
\curveto(58.75747484,163.19365549)(58.74747485,163.22365546)(58.7474762,163.25366333)
\curveto(58.74747485,163.29365539)(58.74247485,163.33865534)(58.7324762,163.38866333)
\curveto(58.71247488,163.50865517)(58.71747488,163.61865506)(58.7474762,163.71866333)
\curveto(58.78747481,163.81865486)(58.84247475,163.88865479)(58.9124762,163.92866333)
\curveto(58.9924746,163.9786547)(59.11247448,164.00365468)(59.2724762,164.00366333)
\curveto(59.43247416,164.00365468)(59.56747403,164.01865466)(59.6774762,164.04866333)
\curveto(59.72747387,164.05865462)(59.78247381,164.06365462)(59.8424762,164.06366333)
\curveto(59.90247369,164.07365461)(59.96247363,164.08865459)(60.0224762,164.10866333)
\curveto(60.17247342,164.15865452)(60.31747328,164.20865447)(60.4574762,164.25866333)
\curveto(60.597473,164.31865436)(60.73247286,164.38865429)(60.8624762,164.46866333)
\curveto(61.00247259,164.55865412)(61.12247247,164.66365402)(61.2224762,164.78366333)
\curveto(61.32247227,164.90365378)(61.41747218,165.03365365)(61.5074762,165.17366333)
\curveto(61.56747203,165.27365341)(61.61247198,165.3836533)(61.6424762,165.50366333)
\curveto(61.68247191,165.62365306)(61.73247186,165.72865295)(61.7924762,165.81866333)
\curveto(61.84247175,165.8786528)(61.91247168,165.91865276)(62.0024762,165.93866333)
\curveto(62.02247157,165.94865273)(62.04747155,165.95365273)(62.0774762,165.95366333)
\curveto(62.10747149,165.95365273)(62.13247146,165.95865272)(62.1524762,165.96866333)
}
}
{
\newrgbcolor{curcolor}{0 0 0}
\pscustom[linestyle=none,fillstyle=solid,fillcolor=curcolor]
{
\newpath
\moveto(73.44208557,160.44866333)
\lineto(73.44208557,160.19366333)
\curveto(73.45207787,160.11365857)(73.44707787,160.03865864)(73.42708557,159.96866333)
\lineto(73.42708557,159.72866333)
\lineto(73.42708557,159.56366333)
\curveto(73.40707791,159.46365922)(73.39707792,159.35865932)(73.39708557,159.24866333)
\curveto(73.39707792,159.14865953)(73.38707793,159.04865963)(73.36708557,158.94866333)
\lineto(73.36708557,158.79866333)
\curveto(73.33707798,158.65866002)(73.317078,158.51866016)(73.30708557,158.37866333)
\curveto(73.29707802,158.24866043)(73.27207805,158.11866056)(73.23208557,157.98866333)
\curveto(73.21207811,157.90866077)(73.19207813,157.82366086)(73.17208557,157.73366333)
\lineto(73.11208557,157.49366333)
\lineto(72.99208557,157.19366333)
\curveto(72.96207836,157.10366158)(72.92707839,157.01366167)(72.88708557,156.92366333)
\curveto(72.78707853,156.70366198)(72.65207867,156.48866219)(72.48208557,156.27866333)
\curveto(72.322079,156.06866261)(72.14707917,155.89866278)(71.95708557,155.76866333)
\curveto(71.90707941,155.72866295)(71.84707947,155.68866299)(71.77708557,155.64866333)
\curveto(71.7170796,155.61866306)(71.65707966,155.5836631)(71.59708557,155.54366333)
\curveto(71.5170798,155.49366319)(71.4220799,155.45366323)(71.31208557,155.42366333)
\curveto(71.20208012,155.39366329)(71.09708022,155.36366332)(70.99708557,155.33366333)
\curveto(70.88708043,155.29366339)(70.77708054,155.26866341)(70.66708557,155.25866333)
\curveto(70.55708076,155.24866343)(70.44208088,155.23366345)(70.32208557,155.21366333)
\curveto(70.28208104,155.20366348)(70.23708108,155.20366348)(70.18708557,155.21366333)
\curveto(70.14708117,155.21366347)(70.10708121,155.20866347)(70.06708557,155.19866333)
\curveto(70.02708129,155.18866349)(69.97208135,155.1836635)(69.90208557,155.18366333)
\curveto(69.83208149,155.1836635)(69.78208154,155.18866349)(69.75208557,155.19866333)
\curveto(69.70208162,155.21866346)(69.65708166,155.22366346)(69.61708557,155.21366333)
\curveto(69.57708174,155.20366348)(69.54208178,155.20366348)(69.51208557,155.21366333)
\lineto(69.42208557,155.21366333)
\curveto(69.36208196,155.23366345)(69.29708202,155.24866343)(69.22708557,155.25866333)
\curveto(69.16708215,155.25866342)(69.10208222,155.26366342)(69.03208557,155.27366333)
\curveto(68.86208246,155.32366336)(68.70208262,155.37366331)(68.55208557,155.42366333)
\curveto(68.40208292,155.47366321)(68.25708306,155.53866314)(68.11708557,155.61866333)
\curveto(68.06708325,155.65866302)(68.01208331,155.68866299)(67.95208557,155.70866333)
\curveto(67.90208342,155.73866294)(67.85208347,155.77366291)(67.80208557,155.81366333)
\curveto(67.56208376,155.99366269)(67.36208396,156.21366247)(67.20208557,156.47366333)
\curveto(67.04208428,156.73366195)(66.90208442,157.01866166)(66.78208557,157.32866333)
\curveto(66.7220846,157.46866121)(66.67708464,157.60866107)(66.64708557,157.74866333)
\curveto(66.6170847,157.89866078)(66.58208474,158.05366063)(66.54208557,158.21366333)
\curveto(66.5220848,158.32366036)(66.50708481,158.43366025)(66.49708557,158.54366333)
\curveto(66.48708483,158.65366003)(66.47208485,158.76365992)(66.45208557,158.87366333)
\curveto(66.44208488,158.91365977)(66.43708488,158.95365973)(66.43708557,158.99366333)
\curveto(66.44708487,159.03365965)(66.44708487,159.07365961)(66.43708557,159.11366333)
\curveto(66.42708489,159.16365952)(66.4220849,159.21365947)(66.42208557,159.26366333)
\lineto(66.42208557,159.42866333)
\curveto(66.40208492,159.4786592)(66.39708492,159.52865915)(66.40708557,159.57866333)
\curveto(66.4170849,159.63865904)(66.4170849,159.69365899)(66.40708557,159.74366333)
\curveto(66.39708492,159.7836589)(66.39708492,159.82865885)(66.40708557,159.87866333)
\curveto(66.4170849,159.92865875)(66.41208491,159.9786587)(66.39208557,160.02866333)
\curveto(66.37208495,160.09865858)(66.36708495,160.17365851)(66.37708557,160.25366333)
\curveto(66.38708493,160.34365834)(66.39208493,160.42865825)(66.39208557,160.50866333)
\curveto(66.39208493,160.59865808)(66.38708493,160.69865798)(66.37708557,160.80866333)
\curveto(66.36708495,160.92865775)(66.37208495,161.02865765)(66.39208557,161.10866333)
\lineto(66.39208557,161.39366333)
\lineto(66.43708557,162.02366333)
\curveto(66.44708487,162.12365656)(66.45708486,162.21865646)(66.46708557,162.30866333)
\lineto(66.49708557,162.60866333)
\curveto(66.5170848,162.65865602)(66.5220848,162.70865597)(66.51208557,162.75866333)
\curveto(66.51208481,162.81865586)(66.5220848,162.87365581)(66.54208557,162.92366333)
\curveto(66.59208473,163.09365559)(66.63208469,163.25865542)(66.66208557,163.41866333)
\curveto(66.69208463,163.58865509)(66.74208458,163.74865493)(66.81208557,163.89866333)
\curveto(67.00208432,164.35865432)(67.2220841,164.73365395)(67.47208557,165.02366333)
\curveto(67.73208359,165.31365337)(68.09208323,165.55865312)(68.55208557,165.75866333)
\curveto(68.68208264,165.80865287)(68.81208251,165.84365284)(68.94208557,165.86366333)
\curveto(69.08208224,165.8836528)(69.2220821,165.90865277)(69.36208557,165.93866333)
\curveto(69.43208189,165.94865273)(69.49708182,165.95365273)(69.55708557,165.95366333)
\curveto(69.6170817,165.95365273)(69.68208164,165.95865272)(69.75208557,165.96866333)
\curveto(70.58208074,165.98865269)(71.25208007,165.83865284)(71.76208557,165.51866333)
\curveto(72.27207905,165.20865347)(72.65207867,164.76865391)(72.90208557,164.19866333)
\curveto(72.95207837,164.0786546)(72.99707832,163.95365473)(73.03708557,163.82366333)
\curveto(73.07707824,163.69365499)(73.1220782,163.55865512)(73.17208557,163.41866333)
\curveto(73.19207813,163.33865534)(73.20707811,163.25365543)(73.21708557,163.16366333)
\lineto(73.27708557,162.92366333)
\curveto(73.30707801,162.81365587)(73.322078,162.70365598)(73.32208557,162.59366333)
\curveto(73.33207799,162.4836562)(73.34707797,162.37365631)(73.36708557,162.26366333)
\curveto(73.38707793,162.21365647)(73.39207793,162.16865651)(73.38208557,162.12866333)
\curveto(73.38207794,162.08865659)(73.38707793,162.04865663)(73.39708557,162.00866333)
\curveto(73.40707791,161.95865672)(73.40707791,161.90365678)(73.39708557,161.84366333)
\curveto(73.39707792,161.79365689)(73.40207792,161.74365694)(73.41208557,161.69366333)
\lineto(73.41208557,161.55866333)
\curveto(73.43207789,161.49865718)(73.43207789,161.42865725)(73.41208557,161.34866333)
\curveto(73.40207792,161.2786574)(73.40707791,161.21365747)(73.42708557,161.15366333)
\curveto(73.43707788,161.12365756)(73.44207788,161.0836576)(73.44208557,161.03366333)
\lineto(73.44208557,160.91366333)
\lineto(73.44208557,160.44866333)
\moveto(71.89708557,158.12366333)
\curveto(71.99707932,158.44366024)(72.05707926,158.80865987)(72.07708557,159.21866333)
\curveto(72.09707922,159.62865905)(72.10707921,160.03865864)(72.10708557,160.44866333)
\curveto(72.10707921,160.8786578)(72.09707922,161.29865738)(72.07708557,161.70866333)
\curveto(72.05707926,162.11865656)(72.01207931,162.50365618)(71.94208557,162.86366333)
\curveto(71.87207945,163.22365546)(71.76207956,163.54365514)(71.61208557,163.82366333)
\curveto(71.47207985,164.11365457)(71.27708004,164.34865433)(71.02708557,164.52866333)
\curveto(70.86708045,164.63865404)(70.68708063,164.71865396)(70.48708557,164.76866333)
\curveto(70.28708103,164.82865385)(70.04208128,164.85865382)(69.75208557,164.85866333)
\curveto(69.73208159,164.83865384)(69.69708162,164.82865385)(69.64708557,164.82866333)
\curveto(69.59708172,164.83865384)(69.55708176,164.83865384)(69.52708557,164.82866333)
\curveto(69.44708187,164.80865387)(69.37208195,164.78865389)(69.30208557,164.76866333)
\curveto(69.24208208,164.75865392)(69.17708214,164.73865394)(69.10708557,164.70866333)
\curveto(68.83708248,164.58865409)(68.6170827,164.41865426)(68.44708557,164.19866333)
\curveto(68.28708303,163.98865469)(68.15208317,163.74365494)(68.04208557,163.46366333)
\curveto(67.99208333,163.35365533)(67.95208337,163.23365545)(67.92208557,163.10366333)
\curveto(67.90208342,162.9836557)(67.87708344,162.85865582)(67.84708557,162.72866333)
\curveto(67.82708349,162.678656)(67.8170835,162.62365606)(67.81708557,162.56366333)
\curveto(67.8170835,162.51365617)(67.81208351,162.46365622)(67.80208557,162.41366333)
\curveto(67.79208353,162.32365636)(67.78208354,162.22865645)(67.77208557,162.12866333)
\curveto(67.76208356,162.03865664)(67.75208357,161.94365674)(67.74208557,161.84366333)
\curveto(67.74208358,161.76365692)(67.73708358,161.678657)(67.72708557,161.58866333)
\lineto(67.72708557,161.34866333)
\lineto(67.72708557,161.16866333)
\curveto(67.7170836,161.13865754)(67.71208361,161.10365758)(67.71208557,161.06366333)
\lineto(67.71208557,160.92866333)
\lineto(67.71208557,160.47866333)
\curveto(67.71208361,160.39865828)(67.70708361,160.31365837)(67.69708557,160.22366333)
\curveto(67.69708362,160.14365854)(67.70708361,160.06865861)(67.72708557,159.99866333)
\lineto(67.72708557,159.72866333)
\curveto(67.72708359,159.70865897)(67.7220836,159.678659)(67.71208557,159.63866333)
\curveto(67.71208361,159.60865907)(67.7170836,159.5836591)(67.72708557,159.56366333)
\curveto(67.73708358,159.46365922)(67.74208358,159.36365932)(67.74208557,159.26366333)
\curveto(67.75208357,159.17365951)(67.76208356,159.07365961)(67.77208557,158.96366333)
\curveto(67.80208352,158.84365984)(67.8170835,158.71865996)(67.81708557,158.58866333)
\curveto(67.82708349,158.46866021)(67.85208347,158.35366033)(67.89208557,158.24366333)
\curveto(67.97208335,157.94366074)(68.05708326,157.678661)(68.14708557,157.44866333)
\curveto(68.24708307,157.21866146)(68.39208293,157.00366168)(68.58208557,156.80366333)
\curveto(68.79208253,156.60366208)(69.05708226,156.45366223)(69.37708557,156.35366333)
\curveto(69.4170819,156.33366235)(69.45208187,156.32366236)(69.48208557,156.32366333)
\curveto(69.5220818,156.33366235)(69.56708175,156.32866235)(69.61708557,156.30866333)
\curveto(69.65708166,156.29866238)(69.72708159,156.28866239)(69.82708557,156.27866333)
\curveto(69.93708138,156.26866241)(70.0220813,156.27366241)(70.08208557,156.29366333)
\curveto(70.15208117,156.31366237)(70.2220811,156.32366236)(70.29208557,156.32366333)
\curveto(70.36208096,156.33366235)(70.42708089,156.34866233)(70.48708557,156.36866333)
\curveto(70.68708063,156.42866225)(70.86708045,156.51366217)(71.02708557,156.62366333)
\curveto(71.05708026,156.64366204)(71.08208024,156.66366202)(71.10208557,156.68366333)
\lineto(71.16208557,156.74366333)
\curveto(71.20208012,156.76366192)(71.25208007,156.80366188)(71.31208557,156.86366333)
\curveto(71.41207991,157.00366168)(71.49707982,157.13366155)(71.56708557,157.25366333)
\curveto(71.63707968,157.37366131)(71.70707961,157.51866116)(71.77708557,157.68866333)
\curveto(71.80707951,157.75866092)(71.82707949,157.82866085)(71.83708557,157.89866333)
\curveto(71.85707946,157.96866071)(71.87707944,158.04366064)(71.89708557,158.12366333)
}
}
{
\newrgbcolor{curcolor}{0 0 0}
\pscustom[linestyle=none,fillstyle=solid,fillcolor=curcolor]
{
\newpath
\moveto(61.3424762,239.89723694)
\curveto(62.03247156,239.90722631)(62.63247096,239.78722643)(63.1424762,239.53723694)
\curveto(63.66246993,239.28722693)(64.05746954,238.95222726)(64.3274762,238.53223694)
\curveto(64.37746922,238.45222776)(64.42246917,238.36222785)(64.4624762,238.26223694)
\curveto(64.50246909,238.17222804)(64.54746905,238.07722814)(64.5974762,237.97723694)
\curveto(64.63746896,237.87722834)(64.66746893,237.77722844)(64.6874762,237.67723694)
\curveto(64.70746889,237.57722864)(64.72746887,237.47222874)(64.7474762,237.36223694)
\curveto(64.76746883,237.3122289)(64.77246882,237.26722895)(64.7624762,237.22723694)
\curveto(64.75246884,237.18722903)(64.75746884,237.14222907)(64.7774762,237.09223694)
\curveto(64.78746881,237.04222917)(64.7924688,236.95722926)(64.7924762,236.83723694)
\curveto(64.7924688,236.72722949)(64.78746881,236.64222957)(64.7774762,236.58223694)
\curveto(64.75746884,236.52222969)(64.74746885,236.46222975)(64.7474762,236.40223694)
\curveto(64.75746884,236.34222987)(64.75246884,236.28222993)(64.7324762,236.22223694)
\curveto(64.6924689,236.08223013)(64.65746894,235.94723027)(64.6274762,235.81723694)
\curveto(64.597469,235.68723053)(64.55746904,235.56223065)(64.5074762,235.44223694)
\curveto(64.44746915,235.30223091)(64.37746922,235.17723104)(64.2974762,235.06723694)
\curveto(64.22746937,234.95723126)(64.15246944,234.84723137)(64.0724762,234.73723694)
\lineto(64.0124762,234.67723694)
\curveto(64.00246959,234.65723156)(63.98746961,234.63723158)(63.9674762,234.61723694)
\curveto(63.84746975,234.45723176)(63.71246988,234.3122319)(63.5624762,234.18223694)
\curveto(63.41247018,234.05223216)(63.25247034,233.92723229)(63.0824762,233.80723694)
\curveto(62.77247082,233.58723263)(62.47747112,233.38223283)(62.1974762,233.19223694)
\curveto(61.96747163,233.05223316)(61.73747186,232.9172333)(61.5074762,232.78723694)
\curveto(61.28747231,232.65723356)(61.06747253,232.52223369)(60.8474762,232.38223694)
\curveto(60.597473,232.212234)(60.35747324,232.03223418)(60.1274762,231.84223694)
\curveto(59.90747369,231.65223456)(59.71747388,231.42723479)(59.5574762,231.16723694)
\curveto(59.51747408,231.10723511)(59.48247411,231.04723517)(59.4524762,230.98723694)
\curveto(59.42247417,230.93723528)(59.3924742,230.87223534)(59.3624762,230.79223694)
\curveto(59.34247425,230.72223549)(59.33747426,230.66223555)(59.3474762,230.61223694)
\curveto(59.36747423,230.54223567)(59.40247419,230.48723573)(59.4524762,230.44723694)
\curveto(59.50247409,230.4172358)(59.56247403,230.39723582)(59.6324762,230.38723694)
\lineto(59.8724762,230.38723694)
\lineto(60.6224762,230.38723694)
\lineto(63.4274762,230.38723694)
\lineto(64.0874762,230.38723694)
\curveto(64.17746942,230.38723583)(64.26246933,230.38223583)(64.3424762,230.37223694)
\curveto(64.42246917,230.37223584)(64.48746911,230.35223586)(64.5374762,230.31223694)
\curveto(64.58746901,230.27223594)(64.62746897,230.19723602)(64.6574762,230.08723694)
\curveto(64.6974689,229.98723623)(64.70746889,229.88723633)(64.6874762,229.78723694)
\lineto(64.6874762,229.65223694)
\curveto(64.66746893,229.58223663)(64.64746895,229.52223669)(64.6274762,229.47223694)
\curveto(64.60746899,229.42223679)(64.57246902,229.38223683)(64.5224762,229.35223694)
\curveto(64.47246912,229.3122369)(64.40246919,229.29223692)(64.3124762,229.29223694)
\lineto(64.0424762,229.29223694)
\lineto(63.1424762,229.29223694)
\lineto(59.6324762,229.29223694)
\lineto(58.5674762,229.29223694)
\curveto(58.48747511,229.29223692)(58.3974752,229.28723693)(58.2974762,229.27723694)
\curveto(58.1974754,229.27723694)(58.11247548,229.28723693)(58.0424762,229.30723694)
\curveto(57.83247576,229.37723684)(57.76747583,229.55723666)(57.8474762,229.84723694)
\curveto(57.85747574,229.88723633)(57.85747574,229.92223629)(57.8474762,229.95223694)
\curveto(57.84747575,229.99223622)(57.85747574,230.03723618)(57.8774762,230.08723694)
\curveto(57.8974757,230.16723605)(57.91747568,230.25223596)(57.9374762,230.34223694)
\curveto(57.95747564,230.43223578)(57.98247561,230.5172357)(58.0124762,230.59723694)
\curveto(58.17247542,231.08723513)(58.37247522,231.50223471)(58.6124762,231.84223694)
\curveto(58.7924748,232.09223412)(58.9974746,232.3172339)(59.2274762,232.51723694)
\curveto(59.45747414,232.72723349)(59.6974739,232.92223329)(59.9474762,233.10223694)
\curveto(60.20747339,233.28223293)(60.47247312,233.45223276)(60.7424762,233.61223694)
\curveto(61.02247257,233.78223243)(61.2924723,233.95723226)(61.5524762,234.13723694)
\curveto(61.66247193,234.217232)(61.76747183,234.29223192)(61.8674762,234.36223694)
\curveto(61.97747162,234.43223178)(62.08747151,234.50723171)(62.1974762,234.58723694)
\curveto(62.23747136,234.6172316)(62.27247132,234.64723157)(62.3024762,234.67723694)
\curveto(62.34247125,234.7172315)(62.38247121,234.74723147)(62.4224762,234.76723694)
\curveto(62.56247103,234.87723134)(62.68747091,235.00223121)(62.7974762,235.14223694)
\curveto(62.81747078,235.17223104)(62.84247075,235.19723102)(62.8724762,235.21723694)
\curveto(62.90247069,235.24723097)(62.92747067,235.27723094)(62.9474762,235.30723694)
\curveto(63.02747057,235.40723081)(63.0924705,235.50723071)(63.1424762,235.60723694)
\curveto(63.20247039,235.70723051)(63.25747034,235.8172304)(63.3074762,235.93723694)
\curveto(63.33747026,236.00723021)(63.35747024,236.08223013)(63.3674762,236.16223694)
\lineto(63.4274762,236.40223694)
\lineto(63.4274762,236.49223694)
\curveto(63.43747016,236.52222969)(63.44247015,236.55222966)(63.4424762,236.58223694)
\curveto(63.46247013,236.65222956)(63.46747013,236.74722947)(63.4574762,236.86723694)
\curveto(63.45747014,236.99722922)(63.44747015,237.09722912)(63.4274762,237.16723694)
\curveto(63.40747019,237.24722897)(63.38747021,237.32222889)(63.3674762,237.39223694)
\curveto(63.35747024,237.47222874)(63.33747026,237.55222866)(63.3074762,237.63223694)
\curveto(63.1974704,237.87222834)(63.04747055,238.07222814)(62.8574762,238.23223694)
\curveto(62.67747092,238.40222781)(62.45747114,238.54222767)(62.1974762,238.65223694)
\curveto(62.12747147,238.67222754)(62.05747154,238.68722753)(61.9874762,238.69723694)
\curveto(61.91747168,238.7172275)(61.84247175,238.73722748)(61.7624762,238.75723694)
\curveto(61.68247191,238.77722744)(61.57247202,238.78722743)(61.4324762,238.78723694)
\curveto(61.30247229,238.78722743)(61.1974724,238.77722744)(61.1174762,238.75723694)
\curveto(61.05747254,238.74722747)(61.00247259,238.74222747)(60.9524762,238.74223694)
\curveto(60.90247269,238.74222747)(60.85247274,238.73222748)(60.8024762,238.71223694)
\curveto(60.70247289,238.67222754)(60.60747299,238.63222758)(60.5174762,238.59223694)
\curveto(60.43747316,238.55222766)(60.35747324,238.50722771)(60.2774762,238.45723694)
\curveto(60.24747335,238.43722778)(60.21747338,238.4122278)(60.1874762,238.38223694)
\curveto(60.16747343,238.35222786)(60.14247345,238.32722789)(60.1124762,238.30723694)
\lineto(60.0374762,238.23223694)
\curveto(60.00747359,238.212228)(59.98247361,238.19222802)(59.9624762,238.17223694)
\lineto(59.8124762,237.96223694)
\curveto(59.77247382,237.90222831)(59.72747387,237.83722838)(59.6774762,237.76723694)
\curveto(59.61747398,237.67722854)(59.56747403,237.57222864)(59.5274762,237.45223694)
\curveto(59.4974741,237.34222887)(59.46247413,237.23222898)(59.4224762,237.12223694)
\curveto(59.38247421,237.0122292)(59.35747424,236.86722935)(59.3474762,236.68723694)
\curveto(59.33747426,236.5172297)(59.30747429,236.39222982)(59.2574762,236.31223694)
\curveto(59.20747439,236.23222998)(59.13247446,236.18723003)(59.0324762,236.17723694)
\curveto(58.93247466,236.16723005)(58.82247477,236.16223005)(58.7024762,236.16223694)
\curveto(58.66247493,236.16223005)(58.62247497,236.15723006)(58.5824762,236.14723694)
\curveto(58.54247505,236.14723007)(58.50747509,236.15223006)(58.4774762,236.16223694)
\curveto(58.42747517,236.18223003)(58.37747522,236.19223002)(58.3274762,236.19223694)
\curveto(58.28747531,236.19223002)(58.24747535,236.20223001)(58.2074762,236.22223694)
\curveto(58.11747548,236.28222993)(58.07247552,236.4172298)(58.0724762,236.62723694)
\lineto(58.0724762,236.74723694)
\curveto(58.08247551,236.80722941)(58.08747551,236.86722935)(58.0874762,236.92723694)
\curveto(58.0974755,236.99722922)(58.10747549,237.06222915)(58.1174762,237.12223694)
\curveto(58.13747546,237.23222898)(58.15747544,237.33222888)(58.1774762,237.42223694)
\curveto(58.1974754,237.52222869)(58.22747537,237.6172286)(58.2674762,237.70723694)
\curveto(58.28747531,237.77722844)(58.30747529,237.83722838)(58.3274762,237.88723694)
\lineto(58.3874762,238.06723694)
\curveto(58.50747509,238.32722789)(58.66247493,238.57222764)(58.8524762,238.80223694)
\curveto(59.05247454,239.03222718)(59.26747433,239.217227)(59.4974762,239.35723694)
\curveto(59.60747399,239.43722678)(59.72247387,239.50222671)(59.8424762,239.55223694)
\lineto(60.2324762,239.70223694)
\curveto(60.34247325,239.75222646)(60.45747314,239.78222643)(60.5774762,239.79223694)
\curveto(60.6974729,239.8122264)(60.82247277,239.83722638)(60.9524762,239.86723694)
\curveto(61.02247257,239.86722635)(61.08747251,239.86722635)(61.1474762,239.86723694)
\curveto(61.20747239,239.87722634)(61.27247232,239.88722633)(61.3424762,239.89723694)
}
}
{
\newrgbcolor{curcolor}{0 0 0}
\pscustom[linestyle=none,fillstyle=solid,fillcolor=curcolor]
{
\newpath
\moveto(73.44208557,234.37723694)
\lineto(73.44208557,234.12223694)
\curveto(73.45207787,234.04223217)(73.44707787,233.96723225)(73.42708557,233.89723694)
\lineto(73.42708557,233.65723694)
\lineto(73.42708557,233.49223694)
\curveto(73.40707791,233.39223282)(73.39707792,233.28723293)(73.39708557,233.17723694)
\curveto(73.39707792,233.07723314)(73.38707793,232.97723324)(73.36708557,232.87723694)
\lineto(73.36708557,232.72723694)
\curveto(73.33707798,232.58723363)(73.317078,232.44723377)(73.30708557,232.30723694)
\curveto(73.29707802,232.17723404)(73.27207805,232.04723417)(73.23208557,231.91723694)
\curveto(73.21207811,231.83723438)(73.19207813,231.75223446)(73.17208557,231.66223694)
\lineto(73.11208557,231.42223694)
\lineto(72.99208557,231.12223694)
\curveto(72.96207836,231.03223518)(72.92707839,230.94223527)(72.88708557,230.85223694)
\curveto(72.78707853,230.63223558)(72.65207867,230.4172358)(72.48208557,230.20723694)
\curveto(72.322079,229.99723622)(72.14707917,229.82723639)(71.95708557,229.69723694)
\curveto(71.90707941,229.65723656)(71.84707947,229.6172366)(71.77708557,229.57723694)
\curveto(71.7170796,229.54723667)(71.65707966,229.5122367)(71.59708557,229.47223694)
\curveto(71.5170798,229.42223679)(71.4220799,229.38223683)(71.31208557,229.35223694)
\curveto(71.20208012,229.32223689)(71.09708022,229.29223692)(70.99708557,229.26223694)
\curveto(70.88708043,229.22223699)(70.77708054,229.19723702)(70.66708557,229.18723694)
\curveto(70.55708076,229.17723704)(70.44208088,229.16223705)(70.32208557,229.14223694)
\curveto(70.28208104,229.13223708)(70.23708108,229.13223708)(70.18708557,229.14223694)
\curveto(70.14708117,229.14223707)(70.10708121,229.13723708)(70.06708557,229.12723694)
\curveto(70.02708129,229.1172371)(69.97208135,229.1122371)(69.90208557,229.11223694)
\curveto(69.83208149,229.1122371)(69.78208154,229.1172371)(69.75208557,229.12723694)
\curveto(69.70208162,229.14723707)(69.65708166,229.15223706)(69.61708557,229.14223694)
\curveto(69.57708174,229.13223708)(69.54208178,229.13223708)(69.51208557,229.14223694)
\lineto(69.42208557,229.14223694)
\curveto(69.36208196,229.16223705)(69.29708202,229.17723704)(69.22708557,229.18723694)
\curveto(69.16708215,229.18723703)(69.10208222,229.19223702)(69.03208557,229.20223694)
\curveto(68.86208246,229.25223696)(68.70208262,229.30223691)(68.55208557,229.35223694)
\curveto(68.40208292,229.40223681)(68.25708306,229.46723675)(68.11708557,229.54723694)
\curveto(68.06708325,229.58723663)(68.01208331,229.6172366)(67.95208557,229.63723694)
\curveto(67.90208342,229.66723655)(67.85208347,229.70223651)(67.80208557,229.74223694)
\curveto(67.56208376,229.92223629)(67.36208396,230.14223607)(67.20208557,230.40223694)
\curveto(67.04208428,230.66223555)(66.90208442,230.94723527)(66.78208557,231.25723694)
\curveto(66.7220846,231.39723482)(66.67708464,231.53723468)(66.64708557,231.67723694)
\curveto(66.6170847,231.82723439)(66.58208474,231.98223423)(66.54208557,232.14223694)
\curveto(66.5220848,232.25223396)(66.50708481,232.36223385)(66.49708557,232.47223694)
\curveto(66.48708483,232.58223363)(66.47208485,232.69223352)(66.45208557,232.80223694)
\curveto(66.44208488,232.84223337)(66.43708488,232.88223333)(66.43708557,232.92223694)
\curveto(66.44708487,232.96223325)(66.44708487,233.00223321)(66.43708557,233.04223694)
\curveto(66.42708489,233.09223312)(66.4220849,233.14223307)(66.42208557,233.19223694)
\lineto(66.42208557,233.35723694)
\curveto(66.40208492,233.40723281)(66.39708492,233.45723276)(66.40708557,233.50723694)
\curveto(66.4170849,233.56723265)(66.4170849,233.62223259)(66.40708557,233.67223694)
\curveto(66.39708492,233.7122325)(66.39708492,233.75723246)(66.40708557,233.80723694)
\curveto(66.4170849,233.85723236)(66.41208491,233.90723231)(66.39208557,233.95723694)
\curveto(66.37208495,234.02723219)(66.36708495,234.10223211)(66.37708557,234.18223694)
\curveto(66.38708493,234.27223194)(66.39208493,234.35723186)(66.39208557,234.43723694)
\curveto(66.39208493,234.52723169)(66.38708493,234.62723159)(66.37708557,234.73723694)
\curveto(66.36708495,234.85723136)(66.37208495,234.95723126)(66.39208557,235.03723694)
\lineto(66.39208557,235.32223694)
\lineto(66.43708557,235.95223694)
\curveto(66.44708487,236.05223016)(66.45708486,236.14723007)(66.46708557,236.23723694)
\lineto(66.49708557,236.53723694)
\curveto(66.5170848,236.58722963)(66.5220848,236.63722958)(66.51208557,236.68723694)
\curveto(66.51208481,236.74722947)(66.5220848,236.80222941)(66.54208557,236.85223694)
\curveto(66.59208473,237.02222919)(66.63208469,237.18722903)(66.66208557,237.34723694)
\curveto(66.69208463,237.5172287)(66.74208458,237.67722854)(66.81208557,237.82723694)
\curveto(67.00208432,238.28722793)(67.2220841,238.66222755)(67.47208557,238.95223694)
\curveto(67.73208359,239.24222697)(68.09208323,239.48722673)(68.55208557,239.68723694)
\curveto(68.68208264,239.73722648)(68.81208251,239.77222644)(68.94208557,239.79223694)
\curveto(69.08208224,239.8122264)(69.2220821,239.83722638)(69.36208557,239.86723694)
\curveto(69.43208189,239.87722634)(69.49708182,239.88222633)(69.55708557,239.88223694)
\curveto(69.6170817,239.88222633)(69.68208164,239.88722633)(69.75208557,239.89723694)
\curveto(70.58208074,239.9172263)(71.25208007,239.76722645)(71.76208557,239.44723694)
\curveto(72.27207905,239.13722708)(72.65207867,238.69722752)(72.90208557,238.12723694)
\curveto(72.95207837,238.00722821)(72.99707832,237.88222833)(73.03708557,237.75223694)
\curveto(73.07707824,237.62222859)(73.1220782,237.48722873)(73.17208557,237.34723694)
\curveto(73.19207813,237.26722895)(73.20707811,237.18222903)(73.21708557,237.09223694)
\lineto(73.27708557,236.85223694)
\curveto(73.30707801,236.74222947)(73.322078,236.63222958)(73.32208557,236.52223694)
\curveto(73.33207799,236.4122298)(73.34707797,236.30222991)(73.36708557,236.19223694)
\curveto(73.38707793,236.14223007)(73.39207793,236.09723012)(73.38208557,236.05723694)
\curveto(73.38207794,236.0172302)(73.38707793,235.97723024)(73.39708557,235.93723694)
\curveto(73.40707791,235.88723033)(73.40707791,235.83223038)(73.39708557,235.77223694)
\curveto(73.39707792,235.72223049)(73.40207792,235.67223054)(73.41208557,235.62223694)
\lineto(73.41208557,235.48723694)
\curveto(73.43207789,235.42723079)(73.43207789,235.35723086)(73.41208557,235.27723694)
\curveto(73.40207792,235.20723101)(73.40707791,235.14223107)(73.42708557,235.08223694)
\curveto(73.43707788,235.05223116)(73.44207788,235.0122312)(73.44208557,234.96223694)
\lineto(73.44208557,234.84223694)
\lineto(73.44208557,234.37723694)
\moveto(71.89708557,232.05223694)
\curveto(71.99707932,232.37223384)(72.05707926,232.73723348)(72.07708557,233.14723694)
\curveto(72.09707922,233.55723266)(72.10707921,233.96723225)(72.10708557,234.37723694)
\curveto(72.10707921,234.80723141)(72.09707922,235.22723099)(72.07708557,235.63723694)
\curveto(72.05707926,236.04723017)(72.01207931,236.43222978)(71.94208557,236.79223694)
\curveto(71.87207945,237.15222906)(71.76207956,237.47222874)(71.61208557,237.75223694)
\curveto(71.47207985,238.04222817)(71.27708004,238.27722794)(71.02708557,238.45723694)
\curveto(70.86708045,238.56722765)(70.68708063,238.64722757)(70.48708557,238.69723694)
\curveto(70.28708103,238.75722746)(70.04208128,238.78722743)(69.75208557,238.78723694)
\curveto(69.73208159,238.76722745)(69.69708162,238.75722746)(69.64708557,238.75723694)
\curveto(69.59708172,238.76722745)(69.55708176,238.76722745)(69.52708557,238.75723694)
\curveto(69.44708187,238.73722748)(69.37208195,238.7172275)(69.30208557,238.69723694)
\curveto(69.24208208,238.68722753)(69.17708214,238.66722755)(69.10708557,238.63723694)
\curveto(68.83708248,238.5172277)(68.6170827,238.34722787)(68.44708557,238.12723694)
\curveto(68.28708303,237.9172283)(68.15208317,237.67222854)(68.04208557,237.39223694)
\curveto(67.99208333,237.28222893)(67.95208337,237.16222905)(67.92208557,237.03223694)
\curveto(67.90208342,236.9122293)(67.87708344,236.78722943)(67.84708557,236.65723694)
\curveto(67.82708349,236.60722961)(67.8170835,236.55222966)(67.81708557,236.49223694)
\curveto(67.8170835,236.44222977)(67.81208351,236.39222982)(67.80208557,236.34223694)
\curveto(67.79208353,236.25222996)(67.78208354,236.15723006)(67.77208557,236.05723694)
\curveto(67.76208356,235.96723025)(67.75208357,235.87223034)(67.74208557,235.77223694)
\curveto(67.74208358,235.69223052)(67.73708358,235.60723061)(67.72708557,235.51723694)
\lineto(67.72708557,235.27723694)
\lineto(67.72708557,235.09723694)
\curveto(67.7170836,235.06723115)(67.71208361,235.03223118)(67.71208557,234.99223694)
\lineto(67.71208557,234.85723694)
\lineto(67.71208557,234.40723694)
\curveto(67.71208361,234.32723189)(67.70708361,234.24223197)(67.69708557,234.15223694)
\curveto(67.69708362,234.07223214)(67.70708361,233.99723222)(67.72708557,233.92723694)
\lineto(67.72708557,233.65723694)
\curveto(67.72708359,233.63723258)(67.7220836,233.60723261)(67.71208557,233.56723694)
\curveto(67.71208361,233.53723268)(67.7170836,233.5122327)(67.72708557,233.49223694)
\curveto(67.73708358,233.39223282)(67.74208358,233.29223292)(67.74208557,233.19223694)
\curveto(67.75208357,233.10223311)(67.76208356,233.00223321)(67.77208557,232.89223694)
\curveto(67.80208352,232.77223344)(67.8170835,232.64723357)(67.81708557,232.51723694)
\curveto(67.82708349,232.39723382)(67.85208347,232.28223393)(67.89208557,232.17223694)
\curveto(67.97208335,231.87223434)(68.05708326,231.60723461)(68.14708557,231.37723694)
\curveto(68.24708307,231.14723507)(68.39208293,230.93223528)(68.58208557,230.73223694)
\curveto(68.79208253,230.53223568)(69.05708226,230.38223583)(69.37708557,230.28223694)
\curveto(69.4170819,230.26223595)(69.45208187,230.25223596)(69.48208557,230.25223694)
\curveto(69.5220818,230.26223595)(69.56708175,230.25723596)(69.61708557,230.23723694)
\curveto(69.65708166,230.22723599)(69.72708159,230.217236)(69.82708557,230.20723694)
\curveto(69.93708138,230.19723602)(70.0220813,230.20223601)(70.08208557,230.22223694)
\curveto(70.15208117,230.24223597)(70.2220811,230.25223596)(70.29208557,230.25223694)
\curveto(70.36208096,230.26223595)(70.42708089,230.27723594)(70.48708557,230.29723694)
\curveto(70.68708063,230.35723586)(70.86708045,230.44223577)(71.02708557,230.55223694)
\curveto(71.05708026,230.57223564)(71.08208024,230.59223562)(71.10208557,230.61223694)
\lineto(71.16208557,230.67223694)
\curveto(71.20208012,230.69223552)(71.25208007,230.73223548)(71.31208557,230.79223694)
\curveto(71.41207991,230.93223528)(71.49707982,231.06223515)(71.56708557,231.18223694)
\curveto(71.63707968,231.30223491)(71.70707961,231.44723477)(71.77708557,231.61723694)
\curveto(71.80707951,231.68723453)(71.82707949,231.75723446)(71.83708557,231.82723694)
\curveto(71.85707946,231.89723432)(71.87707944,231.97223424)(71.89708557,232.05223694)
}
}
{
\newrgbcolor{curcolor}{0 0 0}
\pscustom[linestyle=none,fillstyle=solid,fillcolor=curcolor]
{
\newpath
\moveto(61.2074762,314.89723694)
\curveto(62.83747076,314.92722629)(63.88746971,314.37222684)(64.3574762,313.23223694)
\curveto(64.45746914,313.00222821)(64.52246907,312.7122285)(64.5524762,312.36223694)
\curveto(64.592469,312.02222919)(64.56746903,311.7122295)(64.4774762,311.43223694)
\curveto(64.38746921,311.17223004)(64.26746933,310.94723027)(64.1174762,310.75723694)
\curveto(64.0974695,310.7172305)(64.07246952,310.68223053)(64.0424762,310.65223694)
\curveto(64.01246958,310.63223058)(63.98746961,310.60723061)(63.9674762,310.57723694)
\lineto(63.8774762,310.45723694)
\curveto(63.84746975,310.42723079)(63.81246978,310.40223081)(63.7724762,310.38223694)
\curveto(63.72246987,310.33223088)(63.66746993,310.28723093)(63.6074762,310.24723694)
\curveto(63.55747004,310.20723101)(63.51247008,310.15723106)(63.4724762,310.09723694)
\curveto(63.43247016,310.05723116)(63.41747018,310.00723121)(63.4274762,309.94723694)
\curveto(63.43747016,309.89723132)(63.46747013,309.85223136)(63.5174762,309.81223694)
\curveto(63.56747003,309.77223144)(63.62246997,309.73223148)(63.6824762,309.69223694)
\curveto(63.75246984,309.66223155)(63.81746978,309.63223158)(63.8774762,309.60223694)
\curveto(63.93746966,309.57223164)(63.98746961,309.54223167)(64.0274762,309.51223694)
\curveto(64.34746925,309.29223192)(64.60246899,308.98223223)(64.7924762,308.58223694)
\curveto(64.83246876,308.49223272)(64.86246873,308.39723282)(64.8824762,308.29723694)
\curveto(64.91246868,308.20723301)(64.93746866,308.1172331)(64.9574762,308.02723694)
\curveto(64.96746863,307.97723324)(64.97246862,307.92723329)(64.9724762,307.87723694)
\curveto(64.98246861,307.83723338)(64.9924686,307.79223342)(65.0024762,307.74223694)
\curveto(65.01246858,307.69223352)(65.01246858,307.64223357)(65.0024762,307.59223694)
\curveto(64.9924686,307.54223367)(64.9974686,307.49223372)(65.0174762,307.44223694)
\curveto(65.02746857,307.39223382)(65.03246856,307.33223388)(65.0324762,307.26223694)
\curveto(65.03246856,307.19223402)(65.02246857,307.13223408)(65.0024762,307.08223694)
\lineto(65.0024762,306.85723694)
\lineto(64.9424762,306.61723694)
\curveto(64.93246866,306.54723467)(64.91746868,306.47723474)(64.8974762,306.40723694)
\curveto(64.86746873,306.3172349)(64.83746876,306.23223498)(64.8074762,306.15223694)
\curveto(64.78746881,306.07223514)(64.75746884,305.99223522)(64.7174762,305.91223694)
\curveto(64.6974689,305.85223536)(64.66746893,305.79223542)(64.6274762,305.73223694)
\curveto(64.597469,305.68223553)(64.56246903,305.63223558)(64.5224762,305.58223694)
\curveto(64.32246927,305.27223594)(64.07246952,305.0122362)(63.7724762,304.80223694)
\curveto(63.47247012,304.60223661)(63.12747047,304.43723678)(62.7374762,304.30723694)
\curveto(62.61747098,304.26723695)(62.48747111,304.24223697)(62.3474762,304.23223694)
\curveto(62.21747138,304.212237)(62.08247151,304.18723703)(61.9424762,304.15723694)
\curveto(61.87247172,304.14723707)(61.80247179,304.14223707)(61.7324762,304.14223694)
\curveto(61.67247192,304.14223707)(61.60747199,304.13723708)(61.5374762,304.12723694)
\curveto(61.4974721,304.1172371)(61.43747216,304.1122371)(61.3574762,304.11223694)
\curveto(61.28747231,304.1122371)(61.23747236,304.1172371)(61.2074762,304.12723694)
\curveto(61.15747244,304.13723708)(61.11247248,304.14223707)(61.0724762,304.14223694)
\lineto(60.9524762,304.14223694)
\curveto(60.85247274,304.16223705)(60.75247284,304.17723704)(60.6524762,304.18723694)
\curveto(60.55247304,304.19723702)(60.45747314,304.212237)(60.3674762,304.23223694)
\curveto(60.25747334,304.26223695)(60.14747345,304.28723693)(60.0374762,304.30723694)
\curveto(59.93747366,304.33723688)(59.83247376,304.37723684)(59.7224762,304.42723694)
\curveto(59.35247424,304.58723663)(59.03747456,304.78723643)(58.7774762,305.02723694)
\curveto(58.51747508,305.27723594)(58.30747529,305.58723563)(58.1474762,305.95723694)
\curveto(58.10747549,306.04723517)(58.07247552,306.14223507)(58.0424762,306.24223694)
\curveto(58.01247558,306.34223487)(57.98247561,306.44723477)(57.9524762,306.55723694)
\curveto(57.93247566,306.60723461)(57.92247567,306.65723456)(57.9224762,306.70723694)
\curveto(57.92247567,306.76723445)(57.91247568,306.82723439)(57.8924762,306.88723694)
\curveto(57.87247572,306.94723427)(57.86247573,307.02723419)(57.8624762,307.12723694)
\curveto(57.86247573,307.22723399)(57.87747572,307.30223391)(57.9074762,307.35223694)
\curveto(57.91747568,307.38223383)(57.93247566,307.40723381)(57.9524762,307.42723694)
\lineto(58.0124762,307.48723694)
\curveto(58.05247554,307.50723371)(58.11247548,307.52223369)(58.1924762,307.53223694)
\curveto(58.28247531,307.54223367)(58.37247522,307.54723367)(58.4624762,307.54723694)
\curveto(58.55247504,307.54723367)(58.63747496,307.54223367)(58.7174762,307.53223694)
\curveto(58.80747479,307.52223369)(58.87247472,307.5122337)(58.9124762,307.50223694)
\curveto(58.93247466,307.48223373)(58.95247464,307.46723375)(58.9724762,307.45723694)
\curveto(58.9924746,307.45723376)(59.01247458,307.44723377)(59.0324762,307.42723694)
\curveto(59.10247449,307.33723388)(59.14247445,307.22223399)(59.1524762,307.08223694)
\curveto(59.17247442,306.94223427)(59.20247439,306.8172344)(59.2424762,306.70723694)
\lineto(59.3924762,306.34723694)
\curveto(59.44247415,306.23723498)(59.50747409,306.13223508)(59.5874762,306.03223694)
\curveto(59.60747399,306.00223521)(59.62747397,305.97723524)(59.6474762,305.95723694)
\curveto(59.67747392,305.93723528)(59.70247389,305.9122353)(59.7224762,305.88223694)
\curveto(59.76247383,305.82223539)(59.7974738,305.77723544)(59.8274762,305.74723694)
\curveto(59.86747373,305.7172355)(59.90247369,305.68723553)(59.9324762,305.65723694)
\curveto(59.97247362,305.62723559)(60.01747358,305.59723562)(60.0674762,305.56723694)
\curveto(60.15747344,305.50723571)(60.25247334,305.45723576)(60.3524762,305.41723694)
\lineto(60.6824762,305.29723694)
\curveto(60.83247276,305.24723597)(61.03247256,305.217236)(61.2824762,305.20723694)
\curveto(61.53247206,305.19723602)(61.74247185,305.217236)(61.9124762,305.26723694)
\curveto(61.9924716,305.28723593)(62.06247153,305.30223591)(62.1224762,305.31223694)
\lineto(62.3324762,305.37223694)
\curveto(62.61247098,305.49223572)(62.85247074,305.64223557)(63.0524762,305.82223694)
\curveto(63.26247033,306.00223521)(63.42747017,306.23223498)(63.5474762,306.51223694)
\curveto(63.57747002,306.58223463)(63.59747,306.65223456)(63.6074762,306.72223694)
\lineto(63.6674762,306.96223694)
\curveto(63.70746989,307.10223411)(63.71746988,307.26223395)(63.6974762,307.44223694)
\curveto(63.67746992,307.63223358)(63.64746995,307.78223343)(63.6074762,307.89223694)
\curveto(63.47747012,308.27223294)(63.2924703,308.56223265)(63.0524762,308.76223694)
\curveto(62.82247077,308.96223225)(62.51247108,309.12223209)(62.1224762,309.24223694)
\curveto(62.01247158,309.27223194)(61.8924717,309.29223192)(61.7624762,309.30223694)
\curveto(61.64247195,309.3122319)(61.51747208,309.3172319)(61.3874762,309.31723694)
\curveto(61.22747237,309.3172319)(61.08747251,309.32223189)(60.9674762,309.33223694)
\curveto(60.84747275,309.34223187)(60.76247283,309.40223181)(60.7124762,309.51223694)
\curveto(60.6924729,309.54223167)(60.68247291,309.57723164)(60.6824762,309.61723694)
\lineto(60.6824762,309.75223694)
\curveto(60.67247292,309.85223136)(60.67247292,309.94723127)(60.6824762,310.03723694)
\curveto(60.70247289,310.12723109)(60.74247285,310.19223102)(60.8024762,310.23223694)
\curveto(60.84247275,310.26223095)(60.88247271,310.28223093)(60.9224762,310.29223694)
\curveto(60.97247262,310.30223091)(61.02747257,310.3122309)(61.0874762,310.32223694)
\curveto(61.10747249,310.33223088)(61.13247246,310.33223088)(61.1624762,310.32223694)
\curveto(61.1924724,310.32223089)(61.21747238,310.32723089)(61.2374762,310.33723694)
\lineto(61.3724762,310.33723694)
\curveto(61.48247211,310.35723086)(61.58247201,310.36723085)(61.6724762,310.36723694)
\curveto(61.77247182,310.37723084)(61.86747173,310.39723082)(61.9574762,310.42723694)
\curveto(62.27747132,310.53723068)(62.53247106,310.68223053)(62.7224762,310.86223694)
\curveto(62.91247068,311.04223017)(63.06247053,311.29222992)(63.1724762,311.61223694)
\curveto(63.20247039,311.7122295)(63.22247037,311.83722938)(63.2324762,311.98723694)
\curveto(63.25247034,312.14722907)(63.24747035,312.29222892)(63.2174762,312.42223694)
\curveto(63.1974704,312.49222872)(63.17747042,312.55722866)(63.1574762,312.61723694)
\curveto(63.14747045,312.68722853)(63.12747047,312.75222846)(63.0974762,312.81223694)
\curveto(62.9974706,313.05222816)(62.85247074,313.24222797)(62.6624762,313.38223694)
\curveto(62.47247112,313.52222769)(62.24747135,313.63222758)(61.9874762,313.71223694)
\curveto(61.92747167,313.73222748)(61.86747173,313.74222747)(61.8074762,313.74223694)
\curveto(61.74747185,313.74222747)(61.68247191,313.75222746)(61.6124762,313.77223694)
\curveto(61.53247206,313.79222742)(61.43747216,313.80222741)(61.3274762,313.80223694)
\curveto(61.21747238,313.80222741)(61.12247247,313.79222742)(61.0424762,313.77223694)
\curveto(60.9924726,313.75222746)(60.94247265,313.74222747)(60.8924762,313.74223694)
\curveto(60.85247274,313.74222747)(60.80747279,313.73222748)(60.7574762,313.71223694)
\curveto(60.57747302,313.66222755)(60.40747319,313.58722763)(60.2474762,313.48723694)
\curveto(60.0974735,313.39722782)(59.96747363,313.28222793)(59.8574762,313.14223694)
\curveto(59.76747383,313.02222819)(59.68747391,312.89222832)(59.6174762,312.75223694)
\curveto(59.54747405,312.6122286)(59.48247411,312.45722876)(59.4224762,312.28723694)
\curveto(59.3924742,312.17722904)(59.37247422,312.05722916)(59.3624762,311.92723694)
\curveto(59.35247424,311.80722941)(59.31747428,311.70722951)(59.2574762,311.62723694)
\curveto(59.23747436,311.58722963)(59.17747442,311.54722967)(59.0774762,311.50723694)
\curveto(59.03747456,311.49722972)(58.97747462,311.48722973)(58.8974762,311.47723694)
\lineto(58.6424762,311.47723694)
\curveto(58.55247504,311.48722973)(58.46747513,311.49722972)(58.3874762,311.50723694)
\curveto(58.31747528,311.5172297)(58.26747533,311.53222968)(58.2374762,311.55223694)
\curveto(58.1974754,311.58222963)(58.16247543,311.63722958)(58.1324762,311.71723694)
\curveto(58.10247549,311.79722942)(58.0974755,311.88222933)(58.1174762,311.97223694)
\curveto(58.12747547,312.02222919)(58.13247546,312.07222914)(58.1324762,312.12223694)
\lineto(58.1624762,312.30223694)
\curveto(58.1924754,312.40222881)(58.21747538,312.50222871)(58.2374762,312.60223694)
\curveto(58.26747533,312.70222851)(58.30247529,312.79222842)(58.3424762,312.87223694)
\curveto(58.3924752,312.98222823)(58.43747516,313.08722813)(58.4774762,313.18723694)
\curveto(58.51747508,313.29722792)(58.56747503,313.40222781)(58.6274762,313.50223694)
\curveto(58.95747464,314.04222717)(59.42747417,314.43722678)(60.0374762,314.68723694)
\curveto(60.15747344,314.73722648)(60.28247331,314.77222644)(60.4124762,314.79223694)
\curveto(60.55247304,314.8122264)(60.6924729,314.83722638)(60.8324762,314.86723694)
\curveto(60.8924727,314.87722634)(60.95247264,314.88222633)(61.0124762,314.88223694)
\curveto(61.08247251,314.88222633)(61.14747245,314.88722633)(61.2074762,314.89723694)
}
}
{
\newrgbcolor{curcolor}{0 0 0}
\pscustom[linestyle=none,fillstyle=solid,fillcolor=curcolor]
{
\newpath
\moveto(73.44208557,309.37723694)
\lineto(73.44208557,309.12223694)
\curveto(73.45207787,309.04223217)(73.44707787,308.96723225)(73.42708557,308.89723694)
\lineto(73.42708557,308.65723694)
\lineto(73.42708557,308.49223694)
\curveto(73.40707791,308.39223282)(73.39707792,308.28723293)(73.39708557,308.17723694)
\curveto(73.39707792,308.07723314)(73.38707793,307.97723324)(73.36708557,307.87723694)
\lineto(73.36708557,307.72723694)
\curveto(73.33707798,307.58723363)(73.317078,307.44723377)(73.30708557,307.30723694)
\curveto(73.29707802,307.17723404)(73.27207805,307.04723417)(73.23208557,306.91723694)
\curveto(73.21207811,306.83723438)(73.19207813,306.75223446)(73.17208557,306.66223694)
\lineto(73.11208557,306.42223694)
\lineto(72.99208557,306.12223694)
\curveto(72.96207836,306.03223518)(72.92707839,305.94223527)(72.88708557,305.85223694)
\curveto(72.78707853,305.63223558)(72.65207867,305.4172358)(72.48208557,305.20723694)
\curveto(72.322079,304.99723622)(72.14707917,304.82723639)(71.95708557,304.69723694)
\curveto(71.90707941,304.65723656)(71.84707947,304.6172366)(71.77708557,304.57723694)
\curveto(71.7170796,304.54723667)(71.65707966,304.5122367)(71.59708557,304.47223694)
\curveto(71.5170798,304.42223679)(71.4220799,304.38223683)(71.31208557,304.35223694)
\curveto(71.20208012,304.32223689)(71.09708022,304.29223692)(70.99708557,304.26223694)
\curveto(70.88708043,304.22223699)(70.77708054,304.19723702)(70.66708557,304.18723694)
\curveto(70.55708076,304.17723704)(70.44208088,304.16223705)(70.32208557,304.14223694)
\curveto(70.28208104,304.13223708)(70.23708108,304.13223708)(70.18708557,304.14223694)
\curveto(70.14708117,304.14223707)(70.10708121,304.13723708)(70.06708557,304.12723694)
\curveto(70.02708129,304.1172371)(69.97208135,304.1122371)(69.90208557,304.11223694)
\curveto(69.83208149,304.1122371)(69.78208154,304.1172371)(69.75208557,304.12723694)
\curveto(69.70208162,304.14723707)(69.65708166,304.15223706)(69.61708557,304.14223694)
\curveto(69.57708174,304.13223708)(69.54208178,304.13223708)(69.51208557,304.14223694)
\lineto(69.42208557,304.14223694)
\curveto(69.36208196,304.16223705)(69.29708202,304.17723704)(69.22708557,304.18723694)
\curveto(69.16708215,304.18723703)(69.10208222,304.19223702)(69.03208557,304.20223694)
\curveto(68.86208246,304.25223696)(68.70208262,304.30223691)(68.55208557,304.35223694)
\curveto(68.40208292,304.40223681)(68.25708306,304.46723675)(68.11708557,304.54723694)
\curveto(68.06708325,304.58723663)(68.01208331,304.6172366)(67.95208557,304.63723694)
\curveto(67.90208342,304.66723655)(67.85208347,304.70223651)(67.80208557,304.74223694)
\curveto(67.56208376,304.92223629)(67.36208396,305.14223607)(67.20208557,305.40223694)
\curveto(67.04208428,305.66223555)(66.90208442,305.94723527)(66.78208557,306.25723694)
\curveto(66.7220846,306.39723482)(66.67708464,306.53723468)(66.64708557,306.67723694)
\curveto(66.6170847,306.82723439)(66.58208474,306.98223423)(66.54208557,307.14223694)
\curveto(66.5220848,307.25223396)(66.50708481,307.36223385)(66.49708557,307.47223694)
\curveto(66.48708483,307.58223363)(66.47208485,307.69223352)(66.45208557,307.80223694)
\curveto(66.44208488,307.84223337)(66.43708488,307.88223333)(66.43708557,307.92223694)
\curveto(66.44708487,307.96223325)(66.44708487,308.00223321)(66.43708557,308.04223694)
\curveto(66.42708489,308.09223312)(66.4220849,308.14223307)(66.42208557,308.19223694)
\lineto(66.42208557,308.35723694)
\curveto(66.40208492,308.40723281)(66.39708492,308.45723276)(66.40708557,308.50723694)
\curveto(66.4170849,308.56723265)(66.4170849,308.62223259)(66.40708557,308.67223694)
\curveto(66.39708492,308.7122325)(66.39708492,308.75723246)(66.40708557,308.80723694)
\curveto(66.4170849,308.85723236)(66.41208491,308.90723231)(66.39208557,308.95723694)
\curveto(66.37208495,309.02723219)(66.36708495,309.10223211)(66.37708557,309.18223694)
\curveto(66.38708493,309.27223194)(66.39208493,309.35723186)(66.39208557,309.43723694)
\curveto(66.39208493,309.52723169)(66.38708493,309.62723159)(66.37708557,309.73723694)
\curveto(66.36708495,309.85723136)(66.37208495,309.95723126)(66.39208557,310.03723694)
\lineto(66.39208557,310.32223694)
\lineto(66.43708557,310.95223694)
\curveto(66.44708487,311.05223016)(66.45708486,311.14723007)(66.46708557,311.23723694)
\lineto(66.49708557,311.53723694)
\curveto(66.5170848,311.58722963)(66.5220848,311.63722958)(66.51208557,311.68723694)
\curveto(66.51208481,311.74722947)(66.5220848,311.80222941)(66.54208557,311.85223694)
\curveto(66.59208473,312.02222919)(66.63208469,312.18722903)(66.66208557,312.34723694)
\curveto(66.69208463,312.5172287)(66.74208458,312.67722854)(66.81208557,312.82723694)
\curveto(67.00208432,313.28722793)(67.2220841,313.66222755)(67.47208557,313.95223694)
\curveto(67.73208359,314.24222697)(68.09208323,314.48722673)(68.55208557,314.68723694)
\curveto(68.68208264,314.73722648)(68.81208251,314.77222644)(68.94208557,314.79223694)
\curveto(69.08208224,314.8122264)(69.2220821,314.83722638)(69.36208557,314.86723694)
\curveto(69.43208189,314.87722634)(69.49708182,314.88222633)(69.55708557,314.88223694)
\curveto(69.6170817,314.88222633)(69.68208164,314.88722633)(69.75208557,314.89723694)
\curveto(70.58208074,314.9172263)(71.25208007,314.76722645)(71.76208557,314.44723694)
\curveto(72.27207905,314.13722708)(72.65207867,313.69722752)(72.90208557,313.12723694)
\curveto(72.95207837,313.00722821)(72.99707832,312.88222833)(73.03708557,312.75223694)
\curveto(73.07707824,312.62222859)(73.1220782,312.48722873)(73.17208557,312.34723694)
\curveto(73.19207813,312.26722895)(73.20707811,312.18222903)(73.21708557,312.09223694)
\lineto(73.27708557,311.85223694)
\curveto(73.30707801,311.74222947)(73.322078,311.63222958)(73.32208557,311.52223694)
\curveto(73.33207799,311.4122298)(73.34707797,311.30222991)(73.36708557,311.19223694)
\curveto(73.38707793,311.14223007)(73.39207793,311.09723012)(73.38208557,311.05723694)
\curveto(73.38207794,311.0172302)(73.38707793,310.97723024)(73.39708557,310.93723694)
\curveto(73.40707791,310.88723033)(73.40707791,310.83223038)(73.39708557,310.77223694)
\curveto(73.39707792,310.72223049)(73.40207792,310.67223054)(73.41208557,310.62223694)
\lineto(73.41208557,310.48723694)
\curveto(73.43207789,310.42723079)(73.43207789,310.35723086)(73.41208557,310.27723694)
\curveto(73.40207792,310.20723101)(73.40707791,310.14223107)(73.42708557,310.08223694)
\curveto(73.43707788,310.05223116)(73.44207788,310.0122312)(73.44208557,309.96223694)
\lineto(73.44208557,309.84223694)
\lineto(73.44208557,309.37723694)
\moveto(71.89708557,307.05223694)
\curveto(71.99707932,307.37223384)(72.05707926,307.73723348)(72.07708557,308.14723694)
\curveto(72.09707922,308.55723266)(72.10707921,308.96723225)(72.10708557,309.37723694)
\curveto(72.10707921,309.80723141)(72.09707922,310.22723099)(72.07708557,310.63723694)
\curveto(72.05707926,311.04723017)(72.01207931,311.43222978)(71.94208557,311.79223694)
\curveto(71.87207945,312.15222906)(71.76207956,312.47222874)(71.61208557,312.75223694)
\curveto(71.47207985,313.04222817)(71.27708004,313.27722794)(71.02708557,313.45723694)
\curveto(70.86708045,313.56722765)(70.68708063,313.64722757)(70.48708557,313.69723694)
\curveto(70.28708103,313.75722746)(70.04208128,313.78722743)(69.75208557,313.78723694)
\curveto(69.73208159,313.76722745)(69.69708162,313.75722746)(69.64708557,313.75723694)
\curveto(69.59708172,313.76722745)(69.55708176,313.76722745)(69.52708557,313.75723694)
\curveto(69.44708187,313.73722748)(69.37208195,313.7172275)(69.30208557,313.69723694)
\curveto(69.24208208,313.68722753)(69.17708214,313.66722755)(69.10708557,313.63723694)
\curveto(68.83708248,313.5172277)(68.6170827,313.34722787)(68.44708557,313.12723694)
\curveto(68.28708303,312.9172283)(68.15208317,312.67222854)(68.04208557,312.39223694)
\curveto(67.99208333,312.28222893)(67.95208337,312.16222905)(67.92208557,312.03223694)
\curveto(67.90208342,311.9122293)(67.87708344,311.78722943)(67.84708557,311.65723694)
\curveto(67.82708349,311.60722961)(67.8170835,311.55222966)(67.81708557,311.49223694)
\curveto(67.8170835,311.44222977)(67.81208351,311.39222982)(67.80208557,311.34223694)
\curveto(67.79208353,311.25222996)(67.78208354,311.15723006)(67.77208557,311.05723694)
\curveto(67.76208356,310.96723025)(67.75208357,310.87223034)(67.74208557,310.77223694)
\curveto(67.74208358,310.69223052)(67.73708358,310.60723061)(67.72708557,310.51723694)
\lineto(67.72708557,310.27723694)
\lineto(67.72708557,310.09723694)
\curveto(67.7170836,310.06723115)(67.71208361,310.03223118)(67.71208557,309.99223694)
\lineto(67.71208557,309.85723694)
\lineto(67.71208557,309.40723694)
\curveto(67.71208361,309.32723189)(67.70708361,309.24223197)(67.69708557,309.15223694)
\curveto(67.69708362,309.07223214)(67.70708361,308.99723222)(67.72708557,308.92723694)
\lineto(67.72708557,308.65723694)
\curveto(67.72708359,308.63723258)(67.7220836,308.60723261)(67.71208557,308.56723694)
\curveto(67.71208361,308.53723268)(67.7170836,308.5122327)(67.72708557,308.49223694)
\curveto(67.73708358,308.39223282)(67.74208358,308.29223292)(67.74208557,308.19223694)
\curveto(67.75208357,308.10223311)(67.76208356,308.00223321)(67.77208557,307.89223694)
\curveto(67.80208352,307.77223344)(67.8170835,307.64723357)(67.81708557,307.51723694)
\curveto(67.82708349,307.39723382)(67.85208347,307.28223393)(67.89208557,307.17223694)
\curveto(67.97208335,306.87223434)(68.05708326,306.60723461)(68.14708557,306.37723694)
\curveto(68.24708307,306.14723507)(68.39208293,305.93223528)(68.58208557,305.73223694)
\curveto(68.79208253,305.53223568)(69.05708226,305.38223583)(69.37708557,305.28223694)
\curveto(69.4170819,305.26223595)(69.45208187,305.25223596)(69.48208557,305.25223694)
\curveto(69.5220818,305.26223595)(69.56708175,305.25723596)(69.61708557,305.23723694)
\curveto(69.65708166,305.22723599)(69.72708159,305.217236)(69.82708557,305.20723694)
\curveto(69.93708138,305.19723602)(70.0220813,305.20223601)(70.08208557,305.22223694)
\curveto(70.15208117,305.24223597)(70.2220811,305.25223596)(70.29208557,305.25223694)
\curveto(70.36208096,305.26223595)(70.42708089,305.27723594)(70.48708557,305.29723694)
\curveto(70.68708063,305.35723586)(70.86708045,305.44223577)(71.02708557,305.55223694)
\curveto(71.05708026,305.57223564)(71.08208024,305.59223562)(71.10208557,305.61223694)
\lineto(71.16208557,305.67223694)
\curveto(71.20208012,305.69223552)(71.25208007,305.73223548)(71.31208557,305.79223694)
\curveto(71.41207991,305.93223528)(71.49707982,306.06223515)(71.56708557,306.18223694)
\curveto(71.63707968,306.30223491)(71.70707961,306.44723477)(71.77708557,306.61723694)
\curveto(71.80707951,306.68723453)(71.82707949,306.75723446)(71.83708557,306.82723694)
\curveto(71.85707946,306.89723432)(71.87707944,306.97223424)(71.89708557,307.05223694)
}
}
{
\newrgbcolor{curcolor}{0 0 0}
\pscustom[linestyle=none,fillstyle=solid,fillcolor=curcolor]
{
\newpath
\moveto(64.9724762,382.07295013)
\curveto(65.04246855,382.02294667)(65.08246851,381.95294674)(65.0924762,381.86295013)
\curveto(65.11246848,381.77294692)(65.12246847,381.66794703)(65.1224762,381.54795013)
\curveto(65.12246847,381.4979472)(65.11746848,381.44794725)(65.1074762,381.39795013)
\curveto(65.10746849,381.34794735)(65.0974685,381.30294739)(65.0774762,381.26295013)
\curveto(65.04746855,381.17294752)(64.98746861,381.11294758)(64.8974762,381.08295013)
\curveto(64.81746878,381.06294763)(64.72246887,381.05294764)(64.6124762,381.05295013)
\lineto(64.2974762,381.05295013)
\curveto(64.18746941,381.06294763)(64.08246951,381.05294764)(63.9824762,381.02295013)
\curveto(63.84246975,380.9929477)(63.75246984,380.91294778)(63.7124762,380.78295013)
\curveto(63.6924699,380.71294798)(63.68246991,380.62794807)(63.6824762,380.52795013)
\lineto(63.6824762,380.25795013)
\lineto(63.6824762,379.31295013)
\lineto(63.6824762,378.98295013)
\curveto(63.68246991,378.87294982)(63.66246993,378.78794991)(63.6224762,378.72795013)
\curveto(63.58247001,378.66795003)(63.53247006,378.62795007)(63.4724762,378.60795013)
\curveto(63.42247017,378.5979501)(63.35747024,378.58295011)(63.2774762,378.56295013)
\lineto(63.0824762,378.56295013)
\curveto(62.96247063,378.56295013)(62.85747074,378.56795013)(62.7674762,378.57795013)
\curveto(62.67747092,378.5979501)(62.60747099,378.64795005)(62.5574762,378.72795013)
\curveto(62.52747107,378.77794992)(62.51247108,378.84794985)(62.5124762,378.93795013)
\lineto(62.5124762,379.23795013)
\lineto(62.5124762,380.27295013)
\curveto(62.51247108,380.43294826)(62.50247109,380.57794812)(62.4824762,380.70795013)
\curveto(62.47247112,380.84794785)(62.41747118,380.94294775)(62.3174762,380.99295013)
\curveto(62.26747133,381.01294768)(62.1974714,381.02794767)(62.1074762,381.03795013)
\curveto(62.02747157,381.04794765)(61.93747166,381.05294764)(61.8374762,381.05295013)
\lineto(61.5524762,381.05295013)
\lineto(61.3124762,381.05295013)
\lineto(59.0474762,381.05295013)
\curveto(58.95747464,381.05294764)(58.85247474,381.04794765)(58.7324762,381.03795013)
\lineto(58.4024762,381.03795013)
\curveto(58.2924753,381.03794766)(58.1924754,381.04794765)(58.1024762,381.06795013)
\curveto(58.01247558,381.08794761)(57.95247564,381.12294757)(57.9224762,381.17295013)
\curveto(57.87247572,381.24294745)(57.84747575,381.33794736)(57.8474762,381.45795013)
\lineto(57.8474762,381.80295013)
\lineto(57.8474762,382.07295013)
\curveto(57.88747571,382.24294645)(57.94247565,382.38294631)(58.0124762,382.49295013)
\curveto(58.08247551,382.60294609)(58.16247543,382.71794598)(58.2524762,382.83795013)
\lineto(58.6124762,383.37795013)
\curveto(59.05247454,384.00794469)(59.48747411,384.62794407)(59.9174762,385.23795013)
\lineto(61.2374762,387.09795013)
\curveto(61.3974722,387.32794137)(61.55247204,387.54794115)(61.7024762,387.75795013)
\curveto(61.85247174,387.97794072)(62.00747159,388.20294049)(62.1674762,388.43295013)
\curveto(62.21747138,388.50294019)(62.26747133,388.56794013)(62.3174762,388.62795013)
\curveto(62.36747123,388.69794)(62.41747118,388.77293992)(62.4674762,388.85295013)
\lineto(62.5274762,388.94295013)
\curveto(62.55747104,388.98293971)(62.58747101,389.01293968)(62.6174762,389.03295013)
\curveto(62.65747094,389.06293963)(62.6974709,389.08293961)(62.7374762,389.09295013)
\curveto(62.77747082,389.11293958)(62.82247077,389.13293956)(62.8724762,389.15295013)
\curveto(62.8924707,389.15293954)(62.91247068,389.14793955)(62.9324762,389.13795013)
\curveto(62.96247063,389.13793956)(62.98747061,389.14793955)(63.0074762,389.16795013)
\curveto(63.13747046,389.16793953)(63.25747034,389.16293953)(63.3674762,389.15295013)
\curveto(63.47747012,389.14293955)(63.55747004,389.0979396)(63.6074762,389.01795013)
\curveto(63.64746995,388.96793973)(63.66746993,388.8979398)(63.6674762,388.80795013)
\curveto(63.67746992,388.71793998)(63.68246991,388.62294007)(63.6824762,388.52295013)
\lineto(63.6824762,383.06295013)
\curveto(63.68246991,382.9929457)(63.67746992,382.91794578)(63.6674762,382.83795013)
\curveto(63.66746993,382.76794593)(63.67246992,382.697946)(63.6824762,382.62795013)
\lineto(63.6824762,382.52295013)
\curveto(63.70246989,382.47294622)(63.71746988,382.41794628)(63.7274762,382.35795013)
\curveto(63.73746986,382.30794639)(63.76246983,382.26794643)(63.8024762,382.23795013)
\curveto(63.87246972,382.18794651)(63.95746964,382.15794654)(64.0574762,382.14795013)
\lineto(64.3874762,382.14795013)
\curveto(64.4974691,382.14794655)(64.60246899,382.14294655)(64.7024762,382.13295013)
\curveto(64.81246878,382.13294656)(64.90246869,382.11294658)(64.9724762,382.07295013)
\moveto(62.4074762,382.26795013)
\curveto(62.48747111,382.37794632)(62.52247107,382.54794615)(62.5124762,382.77795013)
\lineto(62.5124762,383.39295013)
\lineto(62.5124762,385.86795013)
\lineto(62.5124762,386.18295013)
\curveto(62.52247107,386.30294239)(62.51747108,386.40294229)(62.4974762,386.48295013)
\lineto(62.4974762,386.63295013)
\curveto(62.4974711,386.72294197)(62.48247111,386.80794189)(62.4524762,386.88795013)
\curveto(62.44247115,386.90794179)(62.43247116,386.91794178)(62.4224762,386.91795013)
\lineto(62.3774762,386.96295013)
\curveto(62.35747124,386.97294172)(62.32747127,386.97794172)(62.2874762,386.97795013)
\curveto(62.26747133,386.95794174)(62.24747135,386.94294175)(62.2274762,386.93295013)
\curveto(62.21747138,386.93294176)(62.20247139,386.92794177)(62.1824762,386.91795013)
\curveto(62.12247147,386.86794183)(62.06247153,386.7979419)(62.0024762,386.70795013)
\curveto(61.94247165,386.61794208)(61.88747171,386.53794216)(61.8374762,386.46795013)
\curveto(61.73747186,386.32794237)(61.64247195,386.18294251)(61.5524762,386.03295013)
\curveto(61.46247213,385.8929428)(61.36747223,385.75294294)(61.2674762,385.61295013)
\lineto(60.7274762,384.83295013)
\curveto(60.55747304,384.57294412)(60.38247321,384.31294438)(60.2024762,384.05295013)
\curveto(60.12247347,383.94294475)(60.04747355,383.83794486)(59.9774762,383.73795013)
\lineto(59.7674762,383.43795013)
\curveto(59.71747388,383.35794534)(59.66747393,383.28294541)(59.6174762,383.21295013)
\curveto(59.57747402,383.14294555)(59.53247406,383.06794563)(59.4824762,382.98795013)
\curveto(59.43247416,382.92794577)(59.38247421,382.86294583)(59.3324762,382.79295013)
\curveto(59.2924743,382.73294596)(59.25247434,382.66294603)(59.2124762,382.58295013)
\curveto(59.17247442,382.52294617)(59.14747445,382.45294624)(59.1374762,382.37295013)
\curveto(59.12747447,382.30294639)(59.16247443,382.24794645)(59.2424762,382.20795013)
\curveto(59.31247428,382.15794654)(59.42247417,382.13294656)(59.5724762,382.13295013)
\curveto(59.73247386,382.14294655)(59.86747373,382.14794655)(59.9774762,382.14795013)
\lineto(61.6574762,382.14795013)
\lineto(62.0924762,382.14795013)
\curveto(62.24247135,382.14794655)(62.34747125,382.18794651)(62.4074762,382.26795013)
}
}
{
\newrgbcolor{curcolor}{0 0 0}
\pscustom[linestyle=none,fillstyle=solid,fillcolor=curcolor]
{
\newpath
\moveto(73.44208557,383.66295013)
\lineto(73.44208557,383.40795013)
\curveto(73.45207787,383.32794537)(73.44707787,383.25294544)(73.42708557,383.18295013)
\lineto(73.42708557,382.94295013)
\lineto(73.42708557,382.77795013)
\curveto(73.40707791,382.67794602)(73.39707792,382.57294612)(73.39708557,382.46295013)
\curveto(73.39707792,382.36294633)(73.38707793,382.26294643)(73.36708557,382.16295013)
\lineto(73.36708557,382.01295013)
\curveto(73.33707798,381.87294682)(73.317078,381.73294696)(73.30708557,381.59295013)
\curveto(73.29707802,381.46294723)(73.27207805,381.33294736)(73.23208557,381.20295013)
\curveto(73.21207811,381.12294757)(73.19207813,381.03794766)(73.17208557,380.94795013)
\lineto(73.11208557,380.70795013)
\lineto(72.99208557,380.40795013)
\curveto(72.96207836,380.31794838)(72.92707839,380.22794847)(72.88708557,380.13795013)
\curveto(72.78707853,379.91794878)(72.65207867,379.70294899)(72.48208557,379.49295013)
\curveto(72.322079,379.28294941)(72.14707917,379.11294958)(71.95708557,378.98295013)
\curveto(71.90707941,378.94294975)(71.84707947,378.90294979)(71.77708557,378.86295013)
\curveto(71.7170796,378.83294986)(71.65707966,378.7979499)(71.59708557,378.75795013)
\curveto(71.5170798,378.70794999)(71.4220799,378.66795003)(71.31208557,378.63795013)
\curveto(71.20208012,378.60795009)(71.09708022,378.57795012)(70.99708557,378.54795013)
\curveto(70.88708043,378.50795019)(70.77708054,378.48295021)(70.66708557,378.47295013)
\curveto(70.55708076,378.46295023)(70.44208088,378.44795025)(70.32208557,378.42795013)
\curveto(70.28208104,378.41795028)(70.23708108,378.41795028)(70.18708557,378.42795013)
\curveto(70.14708117,378.42795027)(70.10708121,378.42295027)(70.06708557,378.41295013)
\curveto(70.02708129,378.40295029)(69.97208135,378.3979503)(69.90208557,378.39795013)
\curveto(69.83208149,378.3979503)(69.78208154,378.40295029)(69.75208557,378.41295013)
\curveto(69.70208162,378.43295026)(69.65708166,378.43795026)(69.61708557,378.42795013)
\curveto(69.57708174,378.41795028)(69.54208178,378.41795028)(69.51208557,378.42795013)
\lineto(69.42208557,378.42795013)
\curveto(69.36208196,378.44795025)(69.29708202,378.46295023)(69.22708557,378.47295013)
\curveto(69.16708215,378.47295022)(69.10208222,378.47795022)(69.03208557,378.48795013)
\curveto(68.86208246,378.53795016)(68.70208262,378.58795011)(68.55208557,378.63795013)
\curveto(68.40208292,378.68795001)(68.25708306,378.75294994)(68.11708557,378.83295013)
\curveto(68.06708325,378.87294982)(68.01208331,378.90294979)(67.95208557,378.92295013)
\curveto(67.90208342,378.95294974)(67.85208347,378.98794971)(67.80208557,379.02795013)
\curveto(67.56208376,379.20794949)(67.36208396,379.42794927)(67.20208557,379.68795013)
\curveto(67.04208428,379.94794875)(66.90208442,380.23294846)(66.78208557,380.54295013)
\curveto(66.7220846,380.68294801)(66.67708464,380.82294787)(66.64708557,380.96295013)
\curveto(66.6170847,381.11294758)(66.58208474,381.26794743)(66.54208557,381.42795013)
\curveto(66.5220848,381.53794716)(66.50708481,381.64794705)(66.49708557,381.75795013)
\curveto(66.48708483,381.86794683)(66.47208485,381.97794672)(66.45208557,382.08795013)
\curveto(66.44208488,382.12794657)(66.43708488,382.16794653)(66.43708557,382.20795013)
\curveto(66.44708487,382.24794645)(66.44708487,382.28794641)(66.43708557,382.32795013)
\curveto(66.42708489,382.37794632)(66.4220849,382.42794627)(66.42208557,382.47795013)
\lineto(66.42208557,382.64295013)
\curveto(66.40208492,382.692946)(66.39708492,382.74294595)(66.40708557,382.79295013)
\curveto(66.4170849,382.85294584)(66.4170849,382.90794579)(66.40708557,382.95795013)
\curveto(66.39708492,382.9979457)(66.39708492,383.04294565)(66.40708557,383.09295013)
\curveto(66.4170849,383.14294555)(66.41208491,383.1929455)(66.39208557,383.24295013)
\curveto(66.37208495,383.31294538)(66.36708495,383.38794531)(66.37708557,383.46795013)
\curveto(66.38708493,383.55794514)(66.39208493,383.64294505)(66.39208557,383.72295013)
\curveto(66.39208493,383.81294488)(66.38708493,383.91294478)(66.37708557,384.02295013)
\curveto(66.36708495,384.14294455)(66.37208495,384.24294445)(66.39208557,384.32295013)
\lineto(66.39208557,384.60795013)
\lineto(66.43708557,385.23795013)
\curveto(66.44708487,385.33794336)(66.45708486,385.43294326)(66.46708557,385.52295013)
\lineto(66.49708557,385.82295013)
\curveto(66.5170848,385.87294282)(66.5220848,385.92294277)(66.51208557,385.97295013)
\curveto(66.51208481,386.03294266)(66.5220848,386.08794261)(66.54208557,386.13795013)
\curveto(66.59208473,386.30794239)(66.63208469,386.47294222)(66.66208557,386.63295013)
\curveto(66.69208463,386.80294189)(66.74208458,386.96294173)(66.81208557,387.11295013)
\curveto(67.00208432,387.57294112)(67.2220841,387.94794075)(67.47208557,388.23795013)
\curveto(67.73208359,388.52794017)(68.09208323,388.77293992)(68.55208557,388.97295013)
\curveto(68.68208264,389.02293967)(68.81208251,389.05793964)(68.94208557,389.07795013)
\curveto(69.08208224,389.0979396)(69.2220821,389.12293957)(69.36208557,389.15295013)
\curveto(69.43208189,389.16293953)(69.49708182,389.16793953)(69.55708557,389.16795013)
\curveto(69.6170817,389.16793953)(69.68208164,389.17293952)(69.75208557,389.18295013)
\curveto(70.58208074,389.20293949)(71.25208007,389.05293964)(71.76208557,388.73295013)
\curveto(72.27207905,388.42294027)(72.65207867,387.98294071)(72.90208557,387.41295013)
\curveto(72.95207837,387.2929414)(72.99707832,387.16794153)(73.03708557,387.03795013)
\curveto(73.07707824,386.90794179)(73.1220782,386.77294192)(73.17208557,386.63295013)
\curveto(73.19207813,386.55294214)(73.20707811,386.46794223)(73.21708557,386.37795013)
\lineto(73.27708557,386.13795013)
\curveto(73.30707801,386.02794267)(73.322078,385.91794278)(73.32208557,385.80795013)
\curveto(73.33207799,385.697943)(73.34707797,385.58794311)(73.36708557,385.47795013)
\curveto(73.38707793,385.42794327)(73.39207793,385.38294331)(73.38208557,385.34295013)
\curveto(73.38207794,385.30294339)(73.38707793,385.26294343)(73.39708557,385.22295013)
\curveto(73.40707791,385.17294352)(73.40707791,385.11794358)(73.39708557,385.05795013)
\curveto(73.39707792,385.00794369)(73.40207792,384.95794374)(73.41208557,384.90795013)
\lineto(73.41208557,384.77295013)
\curveto(73.43207789,384.71294398)(73.43207789,384.64294405)(73.41208557,384.56295013)
\curveto(73.40207792,384.4929442)(73.40707791,384.42794427)(73.42708557,384.36795013)
\curveto(73.43707788,384.33794436)(73.44207788,384.2979444)(73.44208557,384.24795013)
\lineto(73.44208557,384.12795013)
\lineto(73.44208557,383.66295013)
\moveto(71.89708557,381.33795013)
\curveto(71.99707932,381.65794704)(72.05707926,382.02294667)(72.07708557,382.43295013)
\curveto(72.09707922,382.84294585)(72.10707921,383.25294544)(72.10708557,383.66295013)
\curveto(72.10707921,384.0929446)(72.09707922,384.51294418)(72.07708557,384.92295013)
\curveto(72.05707926,385.33294336)(72.01207931,385.71794298)(71.94208557,386.07795013)
\curveto(71.87207945,386.43794226)(71.76207956,386.75794194)(71.61208557,387.03795013)
\curveto(71.47207985,387.32794137)(71.27708004,387.56294113)(71.02708557,387.74295013)
\curveto(70.86708045,387.85294084)(70.68708063,387.93294076)(70.48708557,387.98295013)
\curveto(70.28708103,388.04294065)(70.04208128,388.07294062)(69.75208557,388.07295013)
\curveto(69.73208159,388.05294064)(69.69708162,388.04294065)(69.64708557,388.04295013)
\curveto(69.59708172,388.05294064)(69.55708176,388.05294064)(69.52708557,388.04295013)
\curveto(69.44708187,388.02294067)(69.37208195,388.00294069)(69.30208557,387.98295013)
\curveto(69.24208208,387.97294072)(69.17708214,387.95294074)(69.10708557,387.92295013)
\curveto(68.83708248,387.80294089)(68.6170827,387.63294106)(68.44708557,387.41295013)
\curveto(68.28708303,387.20294149)(68.15208317,386.95794174)(68.04208557,386.67795013)
\curveto(67.99208333,386.56794213)(67.95208337,386.44794225)(67.92208557,386.31795013)
\curveto(67.90208342,386.1979425)(67.87708344,386.07294262)(67.84708557,385.94295013)
\curveto(67.82708349,385.8929428)(67.8170835,385.83794286)(67.81708557,385.77795013)
\curveto(67.8170835,385.72794297)(67.81208351,385.67794302)(67.80208557,385.62795013)
\curveto(67.79208353,385.53794316)(67.78208354,385.44294325)(67.77208557,385.34295013)
\curveto(67.76208356,385.25294344)(67.75208357,385.15794354)(67.74208557,385.05795013)
\curveto(67.74208358,384.97794372)(67.73708358,384.8929438)(67.72708557,384.80295013)
\lineto(67.72708557,384.56295013)
\lineto(67.72708557,384.38295013)
\curveto(67.7170836,384.35294434)(67.71208361,384.31794438)(67.71208557,384.27795013)
\lineto(67.71208557,384.14295013)
\lineto(67.71208557,383.69295013)
\curveto(67.71208361,383.61294508)(67.70708361,383.52794517)(67.69708557,383.43795013)
\curveto(67.69708362,383.35794534)(67.70708361,383.28294541)(67.72708557,383.21295013)
\lineto(67.72708557,382.94295013)
\curveto(67.72708359,382.92294577)(67.7220836,382.8929458)(67.71208557,382.85295013)
\curveto(67.71208361,382.82294587)(67.7170836,382.7979459)(67.72708557,382.77795013)
\curveto(67.73708358,382.67794602)(67.74208358,382.57794612)(67.74208557,382.47795013)
\curveto(67.75208357,382.38794631)(67.76208356,382.28794641)(67.77208557,382.17795013)
\curveto(67.80208352,382.05794664)(67.8170835,381.93294676)(67.81708557,381.80295013)
\curveto(67.82708349,381.68294701)(67.85208347,381.56794713)(67.89208557,381.45795013)
\curveto(67.97208335,381.15794754)(68.05708326,380.8929478)(68.14708557,380.66295013)
\curveto(68.24708307,380.43294826)(68.39208293,380.21794848)(68.58208557,380.01795013)
\curveto(68.79208253,379.81794888)(69.05708226,379.66794903)(69.37708557,379.56795013)
\curveto(69.4170819,379.54794915)(69.45208187,379.53794916)(69.48208557,379.53795013)
\curveto(69.5220818,379.54794915)(69.56708175,379.54294915)(69.61708557,379.52295013)
\curveto(69.65708166,379.51294918)(69.72708159,379.50294919)(69.82708557,379.49295013)
\curveto(69.93708138,379.48294921)(70.0220813,379.48794921)(70.08208557,379.50795013)
\curveto(70.15208117,379.52794917)(70.2220811,379.53794916)(70.29208557,379.53795013)
\curveto(70.36208096,379.54794915)(70.42708089,379.56294913)(70.48708557,379.58295013)
\curveto(70.68708063,379.64294905)(70.86708045,379.72794897)(71.02708557,379.83795013)
\curveto(71.05708026,379.85794884)(71.08208024,379.87794882)(71.10208557,379.89795013)
\lineto(71.16208557,379.95795013)
\curveto(71.20208012,379.97794872)(71.25208007,380.01794868)(71.31208557,380.07795013)
\curveto(71.41207991,380.21794848)(71.49707982,380.34794835)(71.56708557,380.46795013)
\curveto(71.63707968,380.58794811)(71.70707961,380.73294796)(71.77708557,380.90295013)
\curveto(71.80707951,380.97294772)(71.82707949,381.04294765)(71.83708557,381.11295013)
\curveto(71.85707946,381.18294751)(71.87707944,381.25794744)(71.89708557,381.33795013)
}
}
{
\newrgbcolor{curcolor}{0 0 0}
\pscustom[linestyle=none,fillstyle=solid,fillcolor=curcolor]
{
\newpath
\moveto(765.90647461,382.23397308)
\curveto(766.88646811,382.25396213)(767.70646729,382.09396229)(768.36647461,381.75397308)
\curveto(769.03646596,381.42396296)(769.55646544,380.96396342)(769.92647461,380.37397308)
\curveto(770.02646497,380.21396417)(770.10646489,380.05896432)(770.16647461,379.90897308)
\curveto(770.23646476,379.76896461)(770.30146469,379.59896478)(770.36147461,379.39897308)
\curveto(770.38146461,379.34896503)(770.40146459,379.2789651)(770.42147461,379.18897308)
\curveto(770.44146455,379.10896527)(770.43646456,379.03396535)(770.40647461,378.96397308)
\curveto(770.38646461,378.90396548)(770.34646465,378.86396552)(770.28647461,378.84397308)
\curveto(770.23646476,378.83396555)(770.18146481,378.81896556)(770.12147461,378.79897308)
\lineto(769.97147461,378.79897308)
\curveto(769.94146505,378.78896559)(769.90146509,378.7839656)(769.85147461,378.78397308)
\lineto(769.73147461,378.78397308)
\curveto(769.5914654,378.7839656)(769.46146553,378.78896559)(769.34147461,378.79897308)
\curveto(769.23146576,378.81896556)(769.15146584,378.86896551)(769.10147461,378.94897308)
\curveto(769.03146596,379.04896533)(768.97646602,379.16396522)(768.93647461,379.29397308)
\curveto(768.8964661,379.42396496)(768.84146615,379.54396484)(768.77147461,379.65397308)
\curveto(768.64146635,379.87396451)(768.4914665,380.06396432)(768.32147461,380.22397308)
\curveto(768.16146683,380.383964)(767.97146702,380.53396385)(767.75147461,380.67397308)
\curveto(767.63146736,380.75396363)(767.4964675,380.81396357)(767.34647461,380.85397308)
\curveto(767.20646779,380.89396349)(767.06146793,380.93396345)(766.91147461,380.97397308)
\curveto(766.80146819,381.00396338)(766.67646832,381.02396336)(766.53647461,381.03397308)
\curveto(766.3964686,381.05396333)(766.24646875,381.06396332)(766.08647461,381.06397308)
\curveto(765.93646906,381.06396332)(765.78646921,381.05396333)(765.63647461,381.03397308)
\curveto(765.4964695,381.02396336)(765.37646962,381.00396338)(765.27647461,380.97397308)
\curveto(765.17646982,380.95396343)(765.08146991,380.93396345)(764.99147461,380.91397308)
\curveto(764.90147009,380.89396349)(764.81147018,380.86396352)(764.72147461,380.82397308)
\curveto(763.88147111,380.47396391)(763.27647172,379.87396451)(762.90647461,379.02397308)
\curveto(762.83647216,378.8839655)(762.77647222,378.73396565)(762.72647461,378.57397308)
\curveto(762.68647231,378.42396596)(762.64147235,378.26896611)(762.59147461,378.10897308)
\curveto(762.57147242,378.04896633)(762.56147243,377.9839664)(762.56147461,377.91397308)
\curveto(762.56147243,377.85396653)(762.55147244,377.79396659)(762.53147461,377.73397308)
\curveto(762.52147247,377.69396669)(762.51647248,377.65896672)(762.51647461,377.62897308)
\curveto(762.51647248,377.59896678)(762.51147248,377.56396682)(762.50147461,377.52397308)
\curveto(762.48147251,377.41396697)(762.46647253,377.29896708)(762.45647461,377.17897308)
\lineto(762.45647461,376.83397308)
\curveto(762.45647254,376.76396762)(762.45147254,376.68896769)(762.44147461,376.60897308)
\curveto(762.44147255,376.53896784)(762.44647255,376.47396791)(762.45647461,376.41397308)
\lineto(762.45647461,376.26397308)
\curveto(762.47647252,376.19396819)(762.48147251,376.12396826)(762.47147461,376.05397308)
\curveto(762.47147252,375.9839684)(762.48147251,375.91396847)(762.50147461,375.84397308)
\curveto(762.52147247,375.7839686)(762.52647247,375.72396866)(762.51647461,375.66397308)
\curveto(762.51647248,375.60396878)(762.52647247,375.54896883)(762.54647461,375.49897308)
\curveto(762.57647242,375.36896901)(762.60147239,375.23896914)(762.62147461,375.10897308)
\curveto(762.65147234,374.98896939)(762.68647231,374.86896951)(762.72647461,374.74897308)
\curveto(762.8964721,374.24897013)(763.11647188,373.81897056)(763.38647461,373.45897308)
\curveto(763.65647134,373.10897127)(764.01147098,372.81897156)(764.45147461,372.58897308)
\curveto(764.5914704,372.51897186)(764.73147026,372.46397192)(764.87147461,372.42397308)
\curveto(765.02146997,372.383972)(765.18146981,372.33897204)(765.35147461,372.28897308)
\curveto(765.42146957,372.26897211)(765.48646951,372.25897212)(765.54647461,372.25897308)
\curveto(765.60646939,372.26897211)(765.67646932,372.26397212)(765.75647461,372.24397308)
\curveto(765.80646919,372.23397215)(765.8964691,372.22397216)(766.02647461,372.21397308)
\curveto(766.15646884,372.21397217)(766.25146874,372.22397216)(766.31147461,372.24397308)
\lineto(766.41647461,372.24397308)
\curveto(766.45646854,372.25397213)(766.4964685,372.25397213)(766.53647461,372.24397308)
\curveto(766.57646842,372.24397214)(766.61646838,372.25397213)(766.65647461,372.27397308)
\curveto(766.75646824,372.29397209)(766.85146814,372.30897207)(766.94147461,372.31897308)
\curveto(767.04146795,372.33897204)(767.13646786,372.36897201)(767.22647461,372.40897308)
\curveto(768.00646699,372.72897165)(768.55646644,373.25397113)(768.87647461,373.98397308)
\curveto(768.95646604,374.16397022)(769.03146596,374.37897)(769.10147461,374.62897308)
\curveto(769.12146587,374.71896966)(769.13646586,374.80896957)(769.14647461,374.89897308)
\curveto(769.16646583,374.99896938)(769.20146579,375.08896929)(769.25147461,375.16897308)
\curveto(769.30146569,375.24896913)(769.38146561,375.29396909)(769.49147461,375.30397308)
\curveto(769.60146539,375.31396907)(769.72146527,375.31896906)(769.85147461,375.31897308)
\lineto(770.00147461,375.31897308)
\curveto(770.05146494,375.31896906)(770.0964649,375.31396907)(770.13647461,375.30397308)
\lineto(770.24147461,375.30397308)
\lineto(770.33147461,375.27397308)
\curveto(770.37146462,375.27396911)(770.40146459,375.26396912)(770.42147461,375.24397308)
\curveto(770.4914645,375.20396918)(770.53146446,375.12896925)(770.54147461,375.01897308)
\curveto(770.55146444,374.91896946)(770.54146445,374.81896956)(770.51147461,374.71897308)
\curveto(770.45146454,374.48896989)(770.3964646,374.26897011)(770.34647461,374.05897308)
\curveto(770.2964647,373.84897053)(770.22146477,373.64897073)(770.12147461,373.45897308)
\curveto(770.04146495,373.32897105)(769.96646503,373.20397118)(769.89647461,373.08397308)
\curveto(769.83646516,372.96397142)(769.76646523,372.84397154)(769.68647461,372.72397308)
\curveto(769.50646549,372.46397192)(769.28146571,372.22397216)(769.01147461,372.00397308)
\curveto(768.75146624,371.79397259)(768.46646653,371.61897276)(768.15647461,371.47897308)
\curveto(768.04646695,371.42897295)(767.93646706,371.38897299)(767.82647461,371.35897308)
\curveto(767.72646727,371.32897305)(767.62146737,371.29397309)(767.51147461,371.25397308)
\curveto(767.40146759,371.21397317)(767.28646771,371.18897319)(767.16647461,371.17897308)
\curveto(767.05646794,371.15897322)(766.94146805,371.13897324)(766.82147461,371.11897308)
\curveto(766.77146822,371.09897328)(766.72646827,371.09397329)(766.68647461,371.10397308)
\curveto(766.64646835,371.10397328)(766.60646839,371.09897328)(766.56647461,371.08897308)
\curveto(766.50646849,371.0789733)(766.44646855,371.07397331)(766.38647461,371.07397308)
\curveto(766.32646867,371.07397331)(766.26146873,371.06897331)(766.19147461,371.05897308)
\curveto(766.16146883,371.04897333)(766.0914689,371.04897333)(765.98147461,371.05897308)
\curveto(765.88146911,371.05897332)(765.81646918,371.06397332)(765.78647461,371.07397308)
\curveto(765.73646926,371.0839733)(765.68646931,371.08897329)(765.63647461,371.08897308)
\curveto(765.5964694,371.0789733)(765.55146944,371.0789733)(765.50147461,371.08897308)
\lineto(765.35147461,371.08897308)
\curveto(765.27146972,371.10897327)(765.1964698,371.12397326)(765.12647461,371.13397308)
\curveto(765.05646994,371.13397325)(764.98147001,371.14397324)(764.90147461,371.16397308)
\lineto(764.63147461,371.22397308)
\curveto(764.54147045,371.23397315)(764.45647054,371.25397313)(764.37647461,371.28397308)
\curveto(764.16647083,371.34397304)(763.97647102,371.41897296)(763.80647461,371.50897308)
\curveto(763.17647182,371.7789726)(762.66647233,372.16397222)(762.27647461,372.66397308)
\curveto(761.88647311,373.16397122)(761.57647342,373.75397063)(761.34647461,374.43397308)
\curveto(761.30647369,374.55396983)(761.27147372,374.6789697)(761.24147461,374.80897308)
\curveto(761.22147377,374.93896944)(761.1964738,375.07396931)(761.16647461,375.21397308)
\curveto(761.14647385,375.26396912)(761.13647386,375.30896907)(761.13647461,375.34897308)
\curveto(761.14647385,375.38896899)(761.14647385,375.43396895)(761.13647461,375.48397308)
\curveto(761.11647388,375.57396881)(761.10147389,375.66896871)(761.09147461,375.76897308)
\curveto(761.0914739,375.86896851)(761.08147391,375.96396842)(761.06147461,376.05397308)
\lineto(761.06147461,376.33897308)
\curveto(761.04147395,376.38896799)(761.03147396,376.47396791)(761.03147461,376.59397308)
\curveto(761.03147396,376.71396767)(761.04147395,376.79896758)(761.06147461,376.84897308)
\curveto(761.07147392,376.8789675)(761.07147392,376.90896747)(761.06147461,376.93897308)
\curveto(761.05147394,376.9789674)(761.05147394,377.00896737)(761.06147461,377.02897308)
\lineto(761.06147461,377.16397308)
\curveto(761.07147392,377.24396714)(761.07647392,377.32396706)(761.07647461,377.40397308)
\curveto(761.08647391,377.49396689)(761.10147389,377.5789668)(761.12147461,377.65897308)
\curveto(761.14147385,377.71896666)(761.15147384,377.7789666)(761.15147461,377.83897308)
\curveto(761.15147384,377.90896647)(761.16147383,377.9789664)(761.18147461,378.04897308)
\curveto(761.23147376,378.21896616)(761.27147372,378.383966)(761.30147461,378.54397308)
\curveto(761.33147366,378.70396568)(761.37647362,378.85396553)(761.43647461,378.99397308)
\lineto(761.58647461,379.38397308)
\curveto(761.64647335,379.52396486)(761.71147328,379.64896473)(761.78147461,379.75897308)
\curveto(761.93147306,380.01896436)(762.08147291,380.25396413)(762.23147461,380.46397308)
\curveto(762.26147273,380.51396387)(762.2964727,380.55396383)(762.33647461,380.58397308)
\curveto(762.38647261,380.62396376)(762.42647257,380.66896371)(762.45647461,380.71897308)
\curveto(762.51647248,380.79896358)(762.57647242,380.86896351)(762.63647461,380.92897308)
\lineto(762.84647461,381.10897308)
\curveto(762.90647209,381.15896322)(762.96147203,381.20396318)(763.01147461,381.24397308)
\curveto(763.07147192,381.29396309)(763.13647186,381.34396304)(763.20647461,381.39397308)
\curveto(763.35647164,381.50396288)(763.51147148,381.59896278)(763.67147461,381.67897308)
\curveto(763.84147115,381.75896262)(764.01647098,381.83896254)(764.19647461,381.91897308)
\curveto(764.30647069,381.96896241)(764.42147057,382.00396238)(764.54147461,382.02397308)
\curveto(764.67147032,382.05396233)(764.7964702,382.08896229)(764.91647461,382.12897308)
\curveto(764.98647001,382.13896224)(765.05146994,382.14896223)(765.11147461,382.15897308)
\lineto(765.29147461,382.18897308)
\curveto(765.37146962,382.19896218)(765.44646955,382.20396218)(765.51647461,382.20397308)
\curveto(765.5964694,382.21396217)(765.67646932,382.22396216)(765.75647461,382.23397308)
\curveto(765.77646922,382.24396214)(765.80146919,382.24396214)(765.83147461,382.23397308)
\curveto(765.86146913,382.22396216)(765.88646911,382.22396216)(765.90647461,382.23397308)
}
}
{
\newrgbcolor{curcolor}{0 0 0}
\pscustom[linestyle=none,fillstyle=solid,fillcolor=curcolor]
{
\newpath
\moveto(779.02631836,371.86897308)
\curveto(779.05631053,371.70897267)(779.04131054,371.57397281)(778.98131836,371.46397308)
\curveto(778.92131066,371.36397302)(778.84131074,371.28897309)(778.74131836,371.23897308)
\curveto(778.69131089,371.21897316)(778.63631095,371.20897317)(778.57631836,371.20897308)
\curveto(778.52631106,371.20897317)(778.47131111,371.19897318)(778.41131836,371.17897308)
\curveto(778.19131139,371.12897325)(777.97131161,371.14397324)(777.75131836,371.22397308)
\curveto(777.54131204,371.29397309)(777.39631219,371.383973)(777.31631836,371.49397308)
\curveto(777.26631232,371.56397282)(777.22131236,371.64397274)(777.18131836,371.73397308)
\curveto(777.14131244,371.83397255)(777.09131249,371.91397247)(777.03131836,371.97397308)
\curveto(777.01131257,371.99397239)(776.9863126,372.01397237)(776.95631836,372.03397308)
\curveto(776.93631265,372.05397233)(776.90631268,372.05897232)(776.86631836,372.04897308)
\curveto(776.75631283,372.01897236)(776.65131293,371.96397242)(776.55131836,371.88397308)
\curveto(776.46131312,371.80397258)(776.37131321,371.73397265)(776.28131836,371.67397308)
\curveto(776.15131343,371.59397279)(776.01131357,371.51897286)(775.86131836,371.44897308)
\curveto(775.71131387,371.38897299)(775.55131403,371.33397305)(775.38131836,371.28397308)
\curveto(775.2813143,371.25397313)(775.17131441,371.23397315)(775.05131836,371.22397308)
\curveto(774.94131464,371.21397317)(774.83131475,371.19897318)(774.72131836,371.17897308)
\curveto(774.67131491,371.16897321)(774.62631496,371.16397322)(774.58631836,371.16397308)
\lineto(774.48131836,371.16397308)
\curveto(774.37131521,371.14397324)(774.26631532,371.14397324)(774.16631836,371.16397308)
\lineto(774.03131836,371.16397308)
\curveto(773.9813156,371.17397321)(773.93131565,371.1789732)(773.88131836,371.17897308)
\curveto(773.83131575,371.1789732)(773.7863158,371.18897319)(773.74631836,371.20897308)
\curveto(773.70631588,371.21897316)(773.67131591,371.22397316)(773.64131836,371.22397308)
\curveto(773.62131596,371.21397317)(773.59631599,371.21397317)(773.56631836,371.22397308)
\lineto(773.32631836,371.28397308)
\curveto(773.24631634,371.29397309)(773.17131641,371.31397307)(773.10131836,371.34397308)
\curveto(772.80131678,371.47397291)(772.55631703,371.61897276)(772.36631836,371.77897308)
\curveto(772.1863174,371.94897243)(772.03631755,372.1839722)(771.91631836,372.48397308)
\curveto(771.82631776,372.70397168)(771.7813178,372.96897141)(771.78131836,373.27897308)
\lineto(771.78131836,373.59397308)
\curveto(771.79131779,373.64397074)(771.79631779,373.69397069)(771.79631836,373.74397308)
\lineto(771.82631836,373.92397308)
\lineto(771.94631836,374.25397308)
\curveto(771.9863176,374.36397002)(772.03631755,374.46396992)(772.09631836,374.55397308)
\curveto(772.27631731,374.84396954)(772.52131706,375.05896932)(772.83131836,375.19897308)
\curveto(773.14131644,375.33896904)(773.4813161,375.46396892)(773.85131836,375.57397308)
\curveto(773.99131559,375.61396877)(774.13631545,375.64396874)(774.28631836,375.66397308)
\curveto(774.43631515,375.6839687)(774.586315,375.70896867)(774.73631836,375.73897308)
\curveto(774.80631478,375.75896862)(774.87131471,375.76896861)(774.93131836,375.76897308)
\curveto(775.00131458,375.76896861)(775.07631451,375.7789686)(775.15631836,375.79897308)
\curveto(775.22631436,375.81896856)(775.29631429,375.82896855)(775.36631836,375.82897308)
\curveto(775.43631415,375.83896854)(775.51131407,375.85396853)(775.59131836,375.87397308)
\curveto(775.84131374,375.93396845)(776.07631351,375.9839684)(776.29631836,376.02397308)
\curveto(776.51631307,376.07396831)(776.69131289,376.18896819)(776.82131836,376.36897308)
\curveto(776.8813127,376.44896793)(776.93131265,376.54896783)(776.97131836,376.66897308)
\curveto(777.01131257,376.79896758)(777.01131257,376.93896744)(776.97131836,377.08897308)
\curveto(776.91131267,377.32896705)(776.82131276,377.51896686)(776.70131836,377.65897308)
\curveto(776.59131299,377.79896658)(776.43131315,377.90896647)(776.22131836,377.98897308)
\curveto(776.10131348,378.03896634)(775.95631363,378.07396631)(775.78631836,378.09397308)
\curveto(775.62631396,378.11396627)(775.45631413,378.12396626)(775.27631836,378.12397308)
\curveto(775.09631449,378.12396626)(774.92131466,378.11396627)(774.75131836,378.09397308)
\curveto(774.581315,378.07396631)(774.43631515,378.04396634)(774.31631836,378.00397308)
\curveto(774.14631544,377.94396644)(773.9813156,377.85896652)(773.82131836,377.74897308)
\curveto(773.74131584,377.68896669)(773.66631592,377.60896677)(773.59631836,377.50897308)
\curveto(773.53631605,377.41896696)(773.4813161,377.31896706)(773.43131836,377.20897308)
\curveto(773.40131618,377.12896725)(773.37131621,377.04396734)(773.34131836,376.95397308)
\curveto(773.32131626,376.86396752)(773.27631631,376.79396759)(773.20631836,376.74397308)
\curveto(773.16631642,376.71396767)(773.09631649,376.68896769)(772.99631836,376.66897308)
\curveto(772.90631668,376.65896772)(772.81131677,376.65396773)(772.71131836,376.65397308)
\curveto(772.61131697,376.65396773)(772.51131707,376.65896772)(772.41131836,376.66897308)
\curveto(772.32131726,376.68896769)(772.25631733,376.71396767)(772.21631836,376.74397308)
\curveto(772.17631741,376.77396761)(772.14631744,376.82396756)(772.12631836,376.89397308)
\curveto(772.10631748,376.96396742)(772.10631748,377.03896734)(772.12631836,377.11897308)
\curveto(772.15631743,377.24896713)(772.1863174,377.36896701)(772.21631836,377.47897308)
\curveto(772.25631733,377.59896678)(772.30131728,377.71396667)(772.35131836,377.82397308)
\curveto(772.54131704,378.17396621)(772.7813168,378.44396594)(773.07131836,378.63397308)
\curveto(773.36131622,378.83396555)(773.72131586,378.99396539)(774.15131836,379.11397308)
\curveto(774.25131533,379.13396525)(774.35131523,379.14896523)(774.45131836,379.15897308)
\curveto(774.56131502,379.16896521)(774.67131491,379.1839652)(774.78131836,379.20397308)
\curveto(774.82131476,379.21396517)(774.8863147,379.21396517)(774.97631836,379.20397308)
\curveto(775.06631452,379.20396518)(775.12131446,379.21396517)(775.14131836,379.23397308)
\curveto(775.84131374,379.24396514)(776.45131313,379.16396522)(776.97131836,378.99397308)
\curveto(777.49131209,378.82396556)(777.85631173,378.49896588)(778.06631836,378.01897308)
\curveto(778.15631143,377.81896656)(778.20631138,377.5839668)(778.21631836,377.31397308)
\curveto(778.23631135,377.05396733)(778.24631134,376.7789676)(778.24631836,376.48897308)
\lineto(778.24631836,373.17397308)
\curveto(778.24631134,373.03397135)(778.25131133,372.89897148)(778.26131836,372.76897308)
\curveto(778.27131131,372.63897174)(778.30131128,372.53397185)(778.35131836,372.45397308)
\curveto(778.40131118,372.383972)(778.46631112,372.33397205)(778.54631836,372.30397308)
\curveto(778.63631095,372.26397212)(778.72131086,372.23397215)(778.80131836,372.21397308)
\curveto(778.8813107,372.20397218)(778.94131064,372.15897222)(778.98131836,372.07897308)
\curveto(779.00131058,372.04897233)(779.01131057,372.01897236)(779.01131836,371.98897308)
\curveto(779.01131057,371.95897242)(779.01631057,371.91897246)(779.02631836,371.86897308)
\moveto(776.88131836,373.53397308)
\curveto(776.94131264,373.67397071)(776.97131261,373.83397055)(776.97131836,374.01397308)
\curveto(776.9813126,374.20397018)(776.9863126,374.39896998)(776.98631836,374.59897308)
\curveto(776.9863126,374.70896967)(776.9813126,374.80896957)(776.97131836,374.89897308)
\curveto(776.96131262,374.98896939)(776.92131266,375.05896932)(776.85131836,375.10897308)
\curveto(776.82131276,375.12896925)(776.75131283,375.13896924)(776.64131836,375.13897308)
\curveto(776.62131296,375.11896926)(776.586313,375.10896927)(776.53631836,375.10897308)
\curveto(776.4863131,375.10896927)(776.44131314,375.09896928)(776.40131836,375.07897308)
\curveto(776.32131326,375.05896932)(776.23131335,375.03896934)(776.13131836,375.01897308)
\lineto(775.83131836,374.95897308)
\curveto(775.80131378,374.95896942)(775.76631382,374.95396943)(775.72631836,374.94397308)
\lineto(775.62131836,374.94397308)
\curveto(775.47131411,374.90396948)(775.30631428,374.8789695)(775.12631836,374.86897308)
\curveto(774.95631463,374.86896951)(774.79631479,374.84896953)(774.64631836,374.80897308)
\curveto(774.56631502,374.78896959)(774.49131509,374.76896961)(774.42131836,374.74897308)
\curveto(774.36131522,374.73896964)(774.29131529,374.72396966)(774.21131836,374.70397308)
\curveto(774.05131553,374.65396973)(773.90131568,374.58896979)(773.76131836,374.50897308)
\curveto(773.62131596,374.43896994)(773.50131608,374.34897003)(773.40131836,374.23897308)
\curveto(773.30131628,374.12897025)(773.22631636,373.99397039)(773.17631836,373.83397308)
\curveto(773.12631646,373.6839707)(773.10631648,373.49897088)(773.11631836,373.27897308)
\curveto(773.11631647,373.1789712)(773.13131645,373.0839713)(773.16131836,372.99397308)
\curveto(773.20131638,372.91397147)(773.24631634,372.83897154)(773.29631836,372.76897308)
\curveto(773.37631621,372.65897172)(773.4813161,372.56397182)(773.61131836,372.48397308)
\curveto(773.74131584,372.41397197)(773.8813157,372.35397203)(774.03131836,372.30397308)
\curveto(774.0813155,372.29397209)(774.13131545,372.28897209)(774.18131836,372.28897308)
\curveto(774.23131535,372.28897209)(774.2813153,372.2839721)(774.33131836,372.27397308)
\curveto(774.40131518,372.25397213)(774.4863151,372.23897214)(774.58631836,372.22897308)
\curveto(774.69631489,372.22897215)(774.7863148,372.23897214)(774.85631836,372.25897308)
\curveto(774.91631467,372.2789721)(774.97631461,372.2839721)(775.03631836,372.27397308)
\curveto(775.09631449,372.27397211)(775.15631443,372.2839721)(775.21631836,372.30397308)
\curveto(775.29631429,372.32397206)(775.37131421,372.33897204)(775.44131836,372.34897308)
\curveto(775.52131406,372.35897202)(775.59631399,372.378972)(775.66631836,372.40897308)
\curveto(775.95631363,372.52897185)(776.20131338,372.67397171)(776.40131836,372.84397308)
\curveto(776.61131297,373.01397137)(776.77131281,373.24397114)(776.88131836,373.53397308)
}
}
{
\newrgbcolor{curcolor}{0 0 0}
\pscustom[linestyle=none,fillstyle=solid,fillcolor=curcolor]
{
\newpath
\moveto(783.88795898,379.18897308)
\curveto(784.51795375,379.20896517)(785.02295324,379.12396526)(785.40295898,378.93397308)
\curveto(785.78295248,378.74396564)(786.08795218,378.45896592)(786.31795898,378.07897308)
\curveto(786.37795189,377.9789664)(786.42295184,377.86896651)(786.45295898,377.74897308)
\curveto(786.49295177,377.63896674)(786.52795174,377.52396686)(786.55795898,377.40397308)
\curveto(786.60795166,377.21396717)(786.63795163,377.00896737)(786.64795898,376.78897308)
\curveto(786.65795161,376.56896781)(786.6629516,376.34396804)(786.66295898,376.11397308)
\lineto(786.66295898,374.50897308)
\lineto(786.66295898,372.16897308)
\curveto(786.6629516,371.99897238)(786.65795161,371.82897255)(786.64795898,371.65897308)
\curveto(786.64795162,371.48897289)(786.58295168,371.378973)(786.45295898,371.32897308)
\curveto(786.40295186,371.30897307)(786.34795192,371.29897308)(786.28795898,371.29897308)
\curveto(786.23795203,371.28897309)(786.18295208,371.2839731)(786.12295898,371.28397308)
\curveto(785.99295227,371.2839731)(785.8679524,371.28897309)(785.74795898,371.29897308)
\curveto(785.62795264,371.29897308)(785.54295272,371.33897304)(785.49295898,371.41897308)
\curveto(785.44295282,371.48897289)(785.41795285,371.5789728)(785.41795898,371.68897308)
\lineto(785.41795898,372.01897308)
\lineto(785.41795898,373.30897308)
\lineto(785.41795898,375.75397308)
\curveto(785.41795285,376.02396836)(785.41295285,376.28896809)(785.40295898,376.54897308)
\curveto(785.39295287,376.81896756)(785.34795292,377.04896733)(785.26795898,377.23897308)
\curveto(785.18795308,377.43896694)(785.0679532,377.59896678)(784.90795898,377.71897308)
\curveto(784.74795352,377.84896653)(784.5629537,377.94896643)(784.35295898,378.01897308)
\curveto(784.29295397,378.03896634)(784.22795404,378.04896633)(784.15795898,378.04897308)
\curveto(784.09795417,378.05896632)(784.03795423,378.07396631)(783.97795898,378.09397308)
\curveto(783.92795434,378.10396628)(783.84795442,378.10396628)(783.73795898,378.09397308)
\curveto(783.63795463,378.09396629)(783.5679547,378.08896629)(783.52795898,378.07897308)
\curveto(783.48795478,378.05896632)(783.45295481,378.04896633)(783.42295898,378.04897308)
\curveto(783.39295487,378.05896632)(783.35795491,378.05896632)(783.31795898,378.04897308)
\curveto(783.18795508,378.01896636)(783.0629552,377.9839664)(782.94295898,377.94397308)
\curveto(782.83295543,377.91396647)(782.72795554,377.86896651)(782.62795898,377.80897308)
\curveto(782.58795568,377.78896659)(782.55295571,377.76896661)(782.52295898,377.74897308)
\curveto(782.49295577,377.72896665)(782.45795581,377.70896667)(782.41795898,377.68897308)
\curveto(782.0679562,377.43896694)(781.81295645,377.06396732)(781.65295898,376.56397308)
\curveto(781.62295664,376.4839679)(781.60295666,376.39896798)(781.59295898,376.30897308)
\curveto(781.58295668,376.22896815)(781.5679567,376.14896823)(781.54795898,376.06897308)
\curveto(781.52795674,376.01896836)(781.52295674,375.96896841)(781.53295898,375.91897308)
\curveto(781.54295672,375.8789685)(781.53795673,375.83896854)(781.51795898,375.79897308)
\lineto(781.51795898,375.48397308)
\curveto(781.50795676,375.45396893)(781.50295676,375.41896896)(781.50295898,375.37897308)
\curveto(781.51295675,375.33896904)(781.51795675,375.29396909)(781.51795898,375.24397308)
\lineto(781.51795898,374.79397308)
\lineto(781.51795898,373.35397308)
\lineto(781.51795898,372.03397308)
\lineto(781.51795898,371.68897308)
\curveto(781.51795675,371.5789728)(781.49295677,371.48897289)(781.44295898,371.41897308)
\curveto(781.39295687,371.33897304)(781.30295696,371.29897308)(781.17295898,371.29897308)
\curveto(781.05295721,371.28897309)(780.92795734,371.2839731)(780.79795898,371.28397308)
\curveto(780.71795755,371.2839731)(780.64295762,371.28897309)(780.57295898,371.29897308)
\curveto(780.50295776,371.30897307)(780.44295782,371.33397305)(780.39295898,371.37397308)
\curveto(780.31295795,371.42397296)(780.27295799,371.51897286)(780.27295898,371.65897308)
\lineto(780.27295898,372.06397308)
\lineto(780.27295898,373.83397308)
\lineto(780.27295898,377.46397308)
\lineto(780.27295898,378.37897308)
\lineto(780.27295898,378.64897308)
\curveto(780.27295799,378.73896564)(780.29295797,378.80896557)(780.33295898,378.85897308)
\curveto(780.3629579,378.91896546)(780.41295785,378.95896542)(780.48295898,378.97897308)
\curveto(780.52295774,378.98896539)(780.57795769,378.99896538)(780.64795898,379.00897308)
\curveto(780.72795754,379.01896536)(780.80795746,379.02396536)(780.88795898,379.02397308)
\curveto(780.9679573,379.02396536)(781.04295722,379.01896536)(781.11295898,379.00897308)
\curveto(781.19295707,378.99896538)(781.24795702,378.9839654)(781.27795898,378.96397308)
\curveto(781.38795688,378.89396549)(781.43795683,378.80396558)(781.42795898,378.69397308)
\curveto(781.41795685,378.59396579)(781.43295683,378.4789659)(781.47295898,378.34897308)
\curveto(781.49295677,378.28896609)(781.53295673,378.23896614)(781.59295898,378.19897308)
\curveto(781.71295655,378.18896619)(781.80795646,378.23396615)(781.87795898,378.33397308)
\curveto(781.95795631,378.43396595)(782.03795623,378.51396587)(782.11795898,378.57397308)
\curveto(782.25795601,378.67396571)(782.39795587,378.76396562)(782.53795898,378.84397308)
\curveto(782.68795558,378.93396545)(782.85795541,379.00896537)(783.04795898,379.06897308)
\curveto(783.12795514,379.09896528)(783.21295505,379.11896526)(783.30295898,379.12897308)
\curveto(783.40295486,379.13896524)(783.49795477,379.15396523)(783.58795898,379.17397308)
\curveto(783.63795463,379.1839652)(783.68795458,379.18896519)(783.73795898,379.18897308)
\lineto(783.88795898,379.18897308)
}
}
{
\newrgbcolor{curcolor}{0 0 0}
\pscustom[linestyle=none,fillstyle=solid,fillcolor=curcolor]
{
\newpath
\moveto(789.49256836,381.37897308)
\curveto(789.64256635,381.378963)(789.7925662,381.37396301)(789.94256836,381.36397308)
\curveto(790.0925659,381.36396302)(790.19756579,381.32396306)(790.25756836,381.24397308)
\curveto(790.30756568,381.1839632)(790.33256566,381.09896328)(790.33256836,380.98897308)
\curveto(790.34256565,380.88896349)(790.34756564,380.7839636)(790.34756836,380.67397308)
\lineto(790.34756836,379.80397308)
\curveto(790.34756564,379.72396466)(790.34256565,379.63896474)(790.33256836,379.54897308)
\curveto(790.33256566,379.46896491)(790.34256565,379.39896498)(790.36256836,379.33897308)
\curveto(790.40256559,379.19896518)(790.4925655,379.10896527)(790.63256836,379.06897308)
\curveto(790.68256531,379.05896532)(790.72756526,379.05396533)(790.76756836,379.05397308)
\lineto(790.91756836,379.05397308)
\lineto(791.32256836,379.05397308)
\curveto(791.48256451,379.06396532)(791.59756439,379.05396533)(791.66756836,379.02397308)
\curveto(791.75756423,378.96396542)(791.81756417,378.90396548)(791.84756836,378.84397308)
\curveto(791.86756412,378.80396558)(791.87756411,378.75896562)(791.87756836,378.70897308)
\lineto(791.87756836,378.55897308)
\curveto(791.87756411,378.44896593)(791.87256412,378.34396604)(791.86256836,378.24397308)
\curveto(791.85256414,378.15396623)(791.81756417,378.0839663)(791.75756836,378.03397308)
\curveto(791.69756429,377.9839664)(791.61256438,377.95396643)(791.50256836,377.94397308)
\lineto(791.17256836,377.94397308)
\curveto(791.06256493,377.95396643)(790.95256504,377.95896642)(790.84256836,377.95897308)
\curveto(790.73256526,377.95896642)(790.63756535,377.94396644)(790.55756836,377.91397308)
\curveto(790.4875655,377.8839665)(790.43756555,377.83396655)(790.40756836,377.76397308)
\curveto(790.37756561,377.69396669)(790.35756563,377.60896677)(790.34756836,377.50897308)
\curveto(790.33756565,377.41896696)(790.33256566,377.31896706)(790.33256836,377.20897308)
\curveto(790.34256565,377.10896727)(790.34756564,377.00896737)(790.34756836,376.90897308)
\lineto(790.34756836,373.93897308)
\curveto(790.34756564,373.71897066)(790.34256565,373.4839709)(790.33256836,373.23397308)
\curveto(790.33256566,372.99397139)(790.37756561,372.80897157)(790.46756836,372.67897308)
\curveto(790.51756547,372.59897178)(790.58256541,372.54397184)(790.66256836,372.51397308)
\curveto(790.74256525,372.4839719)(790.83756515,372.45897192)(790.94756836,372.43897308)
\curveto(790.97756501,372.42897195)(791.00756498,372.42397196)(791.03756836,372.42397308)
\curveto(791.07756491,372.43397195)(791.11256488,372.43397195)(791.14256836,372.42397308)
\lineto(791.33756836,372.42397308)
\curveto(791.43756455,372.42397196)(791.52756446,372.41397197)(791.60756836,372.39397308)
\curveto(791.69756429,372.383972)(791.76256423,372.34897203)(791.80256836,372.28897308)
\curveto(791.82256417,372.25897212)(791.83756415,372.20397218)(791.84756836,372.12397308)
\curveto(791.86756412,372.05397233)(791.87756411,371.9789724)(791.87756836,371.89897308)
\curveto(791.8875641,371.81897256)(791.8875641,371.73897264)(791.87756836,371.65897308)
\curveto(791.86756412,371.58897279)(791.84756414,371.53397285)(791.81756836,371.49397308)
\curveto(791.77756421,371.42397296)(791.70256429,371.37397301)(791.59256836,371.34397308)
\curveto(791.51256448,371.32397306)(791.42256457,371.31397307)(791.32256836,371.31397308)
\curveto(791.22256477,371.32397306)(791.13256486,371.32897305)(791.05256836,371.32897308)
\curveto(790.992565,371.32897305)(790.93256506,371.32397306)(790.87256836,371.31397308)
\curveto(790.81256518,371.31397307)(790.75756523,371.31897306)(790.70756836,371.32897308)
\lineto(790.52756836,371.32897308)
\curveto(790.47756551,371.33897304)(790.42756556,371.34397304)(790.37756836,371.34397308)
\curveto(790.33756565,371.35397303)(790.2925657,371.35897302)(790.24256836,371.35897308)
\curveto(790.04256595,371.40897297)(789.86756612,371.46397292)(789.71756836,371.52397308)
\curveto(789.57756641,371.5839728)(789.45756653,371.68897269)(789.35756836,371.83897308)
\curveto(789.21756677,372.03897234)(789.13756685,372.28897209)(789.11756836,372.58897308)
\curveto(789.09756689,372.89897148)(789.0875669,373.22897115)(789.08756836,373.57897308)
\lineto(789.08756836,377.50897308)
\curveto(789.05756693,377.63896674)(789.02756696,377.73396665)(788.99756836,377.79397308)
\curveto(788.97756701,377.85396653)(788.90756708,377.90396648)(788.78756836,377.94397308)
\curveto(788.74756724,377.95396643)(788.70756728,377.95396643)(788.66756836,377.94397308)
\curveto(788.62756736,377.93396645)(788.5875674,377.93896644)(788.54756836,377.95897308)
\lineto(788.30756836,377.95897308)
\curveto(788.17756781,377.95896642)(788.06756792,377.96896641)(787.97756836,377.98897308)
\curveto(787.89756809,378.01896636)(787.84256815,378.0789663)(787.81256836,378.16897308)
\curveto(787.7925682,378.20896617)(787.77756821,378.25396613)(787.76756836,378.30397308)
\lineto(787.76756836,378.45397308)
\curveto(787.76756822,378.59396579)(787.77756821,378.70896567)(787.79756836,378.79897308)
\curveto(787.81756817,378.89896548)(787.87756811,378.97396541)(787.97756836,379.02397308)
\curveto(788.0875679,379.06396532)(788.22756776,379.07396531)(788.39756836,379.05397308)
\curveto(788.57756741,379.03396535)(788.72756726,379.04396534)(788.84756836,379.08397308)
\curveto(788.93756705,379.13396525)(789.00756698,379.20396518)(789.05756836,379.29397308)
\curveto(789.07756691,379.35396503)(789.0875669,379.42896495)(789.08756836,379.51897308)
\lineto(789.08756836,379.77397308)
\lineto(789.08756836,380.70397308)
\lineto(789.08756836,380.94397308)
\curveto(789.0875669,381.03396335)(789.09756689,381.10896327)(789.11756836,381.16897308)
\curveto(789.15756683,381.24896313)(789.23256676,381.31396307)(789.34256836,381.36397308)
\curveto(789.37256662,381.36396302)(789.39756659,381.36396302)(789.41756836,381.36397308)
\curveto(789.44756654,381.37396301)(789.47256652,381.378963)(789.49256836,381.37897308)
}
}
{
\newrgbcolor{curcolor}{0 0 0}
\pscustom[linestyle=none,fillstyle=solid,fillcolor=curcolor]
{
\newpath
\moveto(793.54936523,380.53897308)
\curveto(793.46936411,380.59896378)(793.42436416,380.70396368)(793.41436523,380.85397308)
\lineto(793.41436523,381.31897308)
\lineto(793.41436523,381.57397308)
\curveto(793.41436417,381.66396272)(793.42936415,381.73896264)(793.45936523,381.79897308)
\curveto(793.49936408,381.8789625)(793.579364,381.93896244)(793.69936523,381.97897308)
\curveto(793.71936386,381.98896239)(793.73936384,381.98896239)(793.75936523,381.97897308)
\curveto(793.78936379,381.9789624)(793.81436377,381.9839624)(793.83436523,381.99397308)
\curveto(794.00436358,381.99396239)(794.16436342,381.98896239)(794.31436523,381.97897308)
\curveto(794.46436312,381.96896241)(794.56436302,381.90896247)(794.61436523,381.79897308)
\curveto(794.64436294,381.73896264)(794.65936292,381.66396272)(794.65936523,381.57397308)
\lineto(794.65936523,381.31897308)
\curveto(794.65936292,381.13896324)(794.65436293,380.96896341)(794.64436523,380.80897308)
\curveto(794.64436294,380.64896373)(794.579363,380.54396384)(794.44936523,380.49397308)
\curveto(794.39936318,380.47396391)(794.34436324,380.46396392)(794.28436523,380.46397308)
\lineto(794.11936523,380.46397308)
\lineto(793.80436523,380.46397308)
\curveto(793.70436388,380.46396392)(793.61936396,380.48896389)(793.54936523,380.53897308)
\moveto(794.65936523,372.03397308)
\lineto(794.65936523,371.71897308)
\curveto(794.66936291,371.61897276)(794.64936293,371.53897284)(794.59936523,371.47897308)
\curveto(794.56936301,371.41897296)(794.52436306,371.378973)(794.46436523,371.35897308)
\curveto(794.40436318,371.34897303)(794.33436325,371.33397305)(794.25436523,371.31397308)
\lineto(794.02936523,371.31397308)
\curveto(793.89936368,371.31397307)(793.7843638,371.31897306)(793.68436523,371.32897308)
\curveto(793.59436399,371.34897303)(793.52436406,371.39897298)(793.47436523,371.47897308)
\curveto(793.43436415,371.53897284)(793.41436417,371.61397277)(793.41436523,371.70397308)
\lineto(793.41436523,371.98897308)
\lineto(793.41436523,378.33397308)
\lineto(793.41436523,378.64897308)
\curveto(793.41436417,378.75896562)(793.43936414,378.84396554)(793.48936523,378.90397308)
\curveto(793.51936406,378.95396543)(793.55936402,378.9839654)(793.60936523,378.99397308)
\curveto(793.65936392,379.00396538)(793.71436387,379.01896536)(793.77436523,379.03897308)
\curveto(793.79436379,379.03896534)(793.81436377,379.03396535)(793.83436523,379.02397308)
\curveto(793.86436372,379.02396536)(793.88936369,379.02896535)(793.90936523,379.03897308)
\curveto(794.03936354,379.03896534)(794.16936341,379.03396535)(794.29936523,379.02397308)
\curveto(794.43936314,379.02396536)(794.53436305,378.9839654)(794.58436523,378.90397308)
\curveto(794.63436295,378.84396554)(794.65936292,378.76396562)(794.65936523,378.66397308)
\lineto(794.65936523,378.37897308)
\lineto(794.65936523,372.03397308)
}
}
{
\newrgbcolor{curcolor}{0 0 0}
\pscustom[linestyle=none,fillstyle=solid,fillcolor=curcolor]
{
\newpath
\moveto(803.56420898,372.12397308)
\lineto(803.56420898,371.73397308)
\curveto(803.56420111,371.61397277)(803.53920113,371.51397287)(803.48920898,371.43397308)
\curveto(803.43920123,371.36397302)(803.35420132,371.32397306)(803.23420898,371.31397308)
\lineto(802.88920898,371.31397308)
\curveto(802.82920184,371.31397307)(802.7692019,371.30897307)(802.70920898,371.29897308)
\curveto(802.65920201,371.29897308)(802.61420206,371.30897307)(802.57420898,371.32897308)
\curveto(802.48420219,371.34897303)(802.42420225,371.38897299)(802.39420898,371.44897308)
\curveto(802.35420232,371.49897288)(802.32920234,371.55897282)(802.31920898,371.62897308)
\curveto(802.31920235,371.69897268)(802.30420237,371.76897261)(802.27420898,371.83897308)
\curveto(802.26420241,371.85897252)(802.24920242,371.87397251)(802.22920898,371.88397308)
\curveto(802.21920245,371.90397248)(802.20420247,371.92397246)(802.18420898,371.94397308)
\curveto(802.08420259,371.95397243)(802.00420267,371.93397245)(801.94420898,371.88397308)
\curveto(801.89420278,371.83397255)(801.83920283,371.7839726)(801.77920898,371.73397308)
\curveto(801.57920309,371.5839728)(801.37920329,371.46897291)(801.17920898,371.38897308)
\curveto(800.99920367,371.30897307)(800.78920388,371.24897313)(800.54920898,371.20897308)
\curveto(800.31920435,371.16897321)(800.07920459,371.14897323)(799.82920898,371.14897308)
\curveto(799.58920508,371.13897324)(799.34920532,371.15397323)(799.10920898,371.19397308)
\curveto(798.8692058,371.22397316)(798.65920601,371.2789731)(798.47920898,371.35897308)
\curveto(797.95920671,371.5789728)(797.53920713,371.87397251)(797.21920898,372.24397308)
\curveto(796.89920777,372.62397176)(796.64920802,373.09397129)(796.46920898,373.65397308)
\curveto(796.42920824,373.74397064)(796.39920827,373.83397055)(796.37920898,373.92397308)
\curveto(796.3692083,374.02397036)(796.34920832,374.12397026)(796.31920898,374.22397308)
\curveto(796.30920836,374.27397011)(796.30420837,374.32397006)(796.30420898,374.37397308)
\curveto(796.30420837,374.42396996)(796.29920837,374.47396991)(796.28920898,374.52397308)
\curveto(796.2692084,374.57396981)(796.25920841,374.62396976)(796.25920898,374.67397308)
\curveto(796.2692084,374.73396965)(796.2692084,374.78896959)(796.25920898,374.83897308)
\lineto(796.25920898,374.98897308)
\curveto(796.23920843,375.03896934)(796.22920844,375.10396928)(796.22920898,375.18397308)
\curveto(796.22920844,375.26396912)(796.23920843,375.32896905)(796.25920898,375.37897308)
\lineto(796.25920898,375.54397308)
\curveto(796.27920839,375.61396877)(796.28420839,375.6839687)(796.27420898,375.75397308)
\curveto(796.2742084,375.83396855)(796.28420839,375.90896847)(796.30420898,375.97897308)
\curveto(796.31420836,376.02896835)(796.31920835,376.07396831)(796.31920898,376.11397308)
\curveto(796.31920835,376.15396823)(796.32420835,376.19896818)(796.33420898,376.24897308)
\curveto(796.36420831,376.34896803)(796.38920828,376.44396794)(796.40920898,376.53397308)
\curveto(796.42920824,376.63396775)(796.45420822,376.72896765)(796.48420898,376.81897308)
\curveto(796.61420806,377.19896718)(796.77920789,377.53896684)(796.97920898,377.83897308)
\curveto(797.18920748,378.14896623)(797.43920723,378.40396598)(797.72920898,378.60397308)
\curveto(797.89920677,378.72396566)(798.0742066,378.82396556)(798.25420898,378.90397308)
\curveto(798.44420623,378.9839654)(798.64920602,379.05396533)(798.86920898,379.11397308)
\curveto(798.93920573,379.12396526)(799.00420567,379.13396525)(799.06420898,379.14397308)
\curveto(799.13420554,379.15396523)(799.20420547,379.16896521)(799.27420898,379.18897308)
\lineto(799.42420898,379.18897308)
\curveto(799.50420517,379.20896517)(799.61920505,379.21896516)(799.76920898,379.21897308)
\curveto(799.92920474,379.21896516)(800.04920462,379.20896517)(800.12920898,379.18897308)
\curveto(800.1692045,379.1789652)(800.22420445,379.17396521)(800.29420898,379.17397308)
\curveto(800.40420427,379.14396524)(800.51420416,379.11896526)(800.62420898,379.09897308)
\curveto(800.73420394,379.08896529)(800.83920383,379.05896532)(800.93920898,379.00897308)
\curveto(801.08920358,378.94896543)(801.22920344,378.8839655)(801.35920898,378.81397308)
\curveto(801.49920317,378.74396564)(801.62920304,378.66396572)(801.74920898,378.57397308)
\curveto(801.80920286,378.52396586)(801.8692028,378.46896591)(801.92920898,378.40897308)
\curveto(801.99920267,378.35896602)(802.08920258,378.34396604)(802.19920898,378.36397308)
\curveto(802.21920245,378.39396599)(802.23420244,378.41896596)(802.24420898,378.43897308)
\curveto(802.26420241,378.45896592)(802.27920239,378.48896589)(802.28920898,378.52897308)
\curveto(802.31920235,378.61896576)(802.32920234,378.73396565)(802.31920898,378.87397308)
\lineto(802.31920898,379.24897308)
\lineto(802.31920898,380.97397308)
\lineto(802.31920898,381.43897308)
\curveto(802.31920235,381.61896276)(802.34420233,381.74896263)(802.39420898,381.82897308)
\curveto(802.43420224,381.89896248)(802.49420218,381.94396244)(802.57420898,381.96397308)
\curveto(802.59420208,381.96396242)(802.61920205,381.96396242)(802.64920898,381.96397308)
\curveto(802.67920199,381.97396241)(802.70420197,381.9789624)(802.72420898,381.97897308)
\curveto(802.86420181,381.98896239)(803.00920166,381.98896239)(803.15920898,381.97897308)
\curveto(803.31920135,381.9789624)(803.42920124,381.93896244)(803.48920898,381.85897308)
\curveto(803.53920113,381.7789626)(803.56420111,381.6789627)(803.56420898,381.55897308)
\lineto(803.56420898,381.18397308)
\lineto(803.56420898,372.12397308)
\moveto(802.34920898,374.95897308)
\curveto(802.3692023,375.00896937)(802.37920229,375.07396931)(802.37920898,375.15397308)
\curveto(802.37920229,375.24396914)(802.3692023,375.31396907)(802.34920898,375.36397308)
\lineto(802.34920898,375.58897308)
\curveto(802.32920234,375.6789687)(802.31420236,375.76896861)(802.30420898,375.85897308)
\curveto(802.29420238,375.95896842)(802.2742024,376.04896833)(802.24420898,376.12897308)
\curveto(802.22420245,376.20896817)(802.20420247,376.2839681)(802.18420898,376.35397308)
\curveto(802.1742025,376.42396796)(802.15420252,376.49396789)(802.12420898,376.56397308)
\curveto(802.00420267,376.86396752)(801.84920282,377.12896725)(801.65920898,377.35897308)
\curveto(801.4692032,377.58896679)(801.22920344,377.76896661)(800.93920898,377.89897308)
\curveto(800.83920383,377.94896643)(800.73420394,377.9839664)(800.62420898,378.00397308)
\curveto(800.52420415,378.03396635)(800.41420426,378.05896632)(800.29420898,378.07897308)
\curveto(800.21420446,378.09896628)(800.12420455,378.10896627)(800.02420898,378.10897308)
\lineto(799.75420898,378.10897308)
\curveto(799.70420497,378.09896628)(799.65920501,378.08896629)(799.61920898,378.07897308)
\lineto(799.48420898,378.07897308)
\curveto(799.40420527,378.05896632)(799.31920535,378.03896634)(799.22920898,378.01897308)
\curveto(799.14920552,377.99896638)(799.0692056,377.97396641)(798.98920898,377.94397308)
\curveto(798.669206,377.80396658)(798.40920626,377.59896678)(798.20920898,377.32897308)
\curveto(798.01920665,377.06896731)(797.86420681,376.76396762)(797.74420898,376.41397308)
\curveto(797.70420697,376.30396808)(797.674207,376.18896819)(797.65420898,376.06897308)
\curveto(797.64420703,375.95896842)(797.62920704,375.84896853)(797.60920898,375.73897308)
\curveto(797.60920706,375.69896868)(797.60420707,375.65896872)(797.59420898,375.61897308)
\lineto(797.59420898,375.51397308)
\curveto(797.5742071,375.46396892)(797.56420711,375.40896897)(797.56420898,375.34897308)
\curveto(797.5742071,375.28896909)(797.57920709,375.23396915)(797.57920898,375.18397308)
\lineto(797.57920898,374.85397308)
\curveto(797.57920709,374.75396963)(797.58920708,374.65896972)(797.60920898,374.56897308)
\curveto(797.61920705,374.53896984)(797.62420705,374.48896989)(797.62420898,374.41897308)
\curveto(797.64420703,374.34897003)(797.65920701,374.2789701)(797.66920898,374.20897308)
\lineto(797.72920898,373.99897308)
\curveto(797.83920683,373.64897073)(797.98920668,373.34897103)(798.17920898,373.09897308)
\curveto(798.3692063,372.84897153)(798.60920606,372.64397174)(798.89920898,372.48397308)
\curveto(798.98920568,372.43397195)(799.07920559,372.39397199)(799.16920898,372.36397308)
\curveto(799.25920541,372.33397205)(799.35920531,372.30397208)(799.46920898,372.27397308)
\curveto(799.51920515,372.25397213)(799.5692051,372.24897213)(799.61920898,372.25897308)
\curveto(799.67920499,372.26897211)(799.73420494,372.26397212)(799.78420898,372.24397308)
\curveto(799.82420485,372.23397215)(799.86420481,372.22897215)(799.90420898,372.22897308)
\lineto(800.03920898,372.22897308)
\lineto(800.17420898,372.22897308)
\curveto(800.20420447,372.23897214)(800.25420442,372.24397214)(800.32420898,372.24397308)
\curveto(800.40420427,372.26397212)(800.48420419,372.2789721)(800.56420898,372.28897308)
\curveto(800.64420403,372.30897207)(800.71920395,372.33397205)(800.78920898,372.36397308)
\curveto(801.11920355,372.50397188)(801.38420329,372.6789717)(801.58420898,372.88897308)
\curveto(801.79420288,373.10897127)(801.9692027,373.383971)(802.10920898,373.71397308)
\curveto(802.15920251,373.82397056)(802.19420248,373.93397045)(802.21420898,374.04397308)
\curveto(802.23420244,374.15397023)(802.25920241,374.26397012)(802.28920898,374.37397308)
\curveto(802.30920236,374.41396997)(802.31920235,374.44896993)(802.31920898,374.47897308)
\curveto(802.31920235,374.51896986)(802.32420235,374.55896982)(802.33420898,374.59897308)
\curveto(802.34420233,374.65896972)(802.34420233,374.71896966)(802.33420898,374.77897308)
\curveto(802.33420234,374.83896954)(802.33920233,374.89896948)(802.34920898,374.95897308)
}
}
{
\newrgbcolor{curcolor}{0 0 0}
\pscustom[linestyle=none,fillstyle=solid,fillcolor=curcolor]
{
\newpath
\moveto(812.39545898,371.86897308)
\curveto(812.42545115,371.70897267)(812.41045117,371.57397281)(812.35045898,371.46397308)
\curveto(812.29045129,371.36397302)(812.21045137,371.28897309)(812.11045898,371.23897308)
\curveto(812.06045152,371.21897316)(812.00545157,371.20897317)(811.94545898,371.20897308)
\curveto(811.89545168,371.20897317)(811.84045174,371.19897318)(811.78045898,371.17897308)
\curveto(811.56045202,371.12897325)(811.34045224,371.14397324)(811.12045898,371.22397308)
\curveto(810.91045267,371.29397309)(810.76545281,371.383973)(810.68545898,371.49397308)
\curveto(810.63545294,371.56397282)(810.59045299,371.64397274)(810.55045898,371.73397308)
\curveto(810.51045307,371.83397255)(810.46045312,371.91397247)(810.40045898,371.97397308)
\curveto(810.3804532,371.99397239)(810.35545322,372.01397237)(810.32545898,372.03397308)
\curveto(810.30545327,372.05397233)(810.2754533,372.05897232)(810.23545898,372.04897308)
\curveto(810.12545345,372.01897236)(810.02045356,371.96397242)(809.92045898,371.88397308)
\curveto(809.83045375,371.80397258)(809.74045384,371.73397265)(809.65045898,371.67397308)
\curveto(809.52045406,371.59397279)(809.3804542,371.51897286)(809.23045898,371.44897308)
\curveto(809.0804545,371.38897299)(808.92045466,371.33397305)(808.75045898,371.28397308)
\curveto(808.65045493,371.25397313)(808.54045504,371.23397315)(808.42045898,371.22397308)
\curveto(808.31045527,371.21397317)(808.20045538,371.19897318)(808.09045898,371.17897308)
\curveto(808.04045554,371.16897321)(807.99545558,371.16397322)(807.95545898,371.16397308)
\lineto(807.85045898,371.16397308)
\curveto(807.74045584,371.14397324)(807.63545594,371.14397324)(807.53545898,371.16397308)
\lineto(807.40045898,371.16397308)
\curveto(807.35045623,371.17397321)(807.30045628,371.1789732)(807.25045898,371.17897308)
\curveto(807.20045638,371.1789732)(807.15545642,371.18897319)(807.11545898,371.20897308)
\curveto(807.0754565,371.21897316)(807.04045654,371.22397316)(807.01045898,371.22397308)
\curveto(806.99045659,371.21397317)(806.96545661,371.21397317)(806.93545898,371.22397308)
\lineto(806.69545898,371.28397308)
\curveto(806.61545696,371.29397309)(806.54045704,371.31397307)(806.47045898,371.34397308)
\curveto(806.17045741,371.47397291)(805.92545765,371.61897276)(805.73545898,371.77897308)
\curveto(805.55545802,371.94897243)(805.40545817,372.1839722)(805.28545898,372.48397308)
\curveto(805.19545838,372.70397168)(805.15045843,372.96897141)(805.15045898,373.27897308)
\lineto(805.15045898,373.59397308)
\curveto(805.16045842,373.64397074)(805.16545841,373.69397069)(805.16545898,373.74397308)
\lineto(805.19545898,373.92397308)
\lineto(805.31545898,374.25397308)
\curveto(805.35545822,374.36397002)(805.40545817,374.46396992)(805.46545898,374.55397308)
\curveto(805.64545793,374.84396954)(805.89045769,375.05896932)(806.20045898,375.19897308)
\curveto(806.51045707,375.33896904)(806.85045673,375.46396892)(807.22045898,375.57397308)
\curveto(807.36045622,375.61396877)(807.50545607,375.64396874)(807.65545898,375.66397308)
\curveto(807.80545577,375.6839687)(807.95545562,375.70896867)(808.10545898,375.73897308)
\curveto(808.1754554,375.75896862)(808.24045534,375.76896861)(808.30045898,375.76897308)
\curveto(808.37045521,375.76896861)(808.44545513,375.7789686)(808.52545898,375.79897308)
\curveto(808.59545498,375.81896856)(808.66545491,375.82896855)(808.73545898,375.82897308)
\curveto(808.80545477,375.83896854)(808.8804547,375.85396853)(808.96045898,375.87397308)
\curveto(809.21045437,375.93396845)(809.44545413,375.9839684)(809.66545898,376.02397308)
\curveto(809.88545369,376.07396831)(810.06045352,376.18896819)(810.19045898,376.36897308)
\curveto(810.25045333,376.44896793)(810.30045328,376.54896783)(810.34045898,376.66897308)
\curveto(810.3804532,376.79896758)(810.3804532,376.93896744)(810.34045898,377.08897308)
\curveto(810.2804533,377.32896705)(810.19045339,377.51896686)(810.07045898,377.65897308)
\curveto(809.96045362,377.79896658)(809.80045378,377.90896647)(809.59045898,377.98897308)
\curveto(809.47045411,378.03896634)(809.32545425,378.07396631)(809.15545898,378.09397308)
\curveto(808.99545458,378.11396627)(808.82545475,378.12396626)(808.64545898,378.12397308)
\curveto(808.46545511,378.12396626)(808.29045529,378.11396627)(808.12045898,378.09397308)
\curveto(807.95045563,378.07396631)(807.80545577,378.04396634)(807.68545898,378.00397308)
\curveto(807.51545606,377.94396644)(807.35045623,377.85896652)(807.19045898,377.74897308)
\curveto(807.11045647,377.68896669)(807.03545654,377.60896677)(806.96545898,377.50897308)
\curveto(806.90545667,377.41896696)(806.85045673,377.31896706)(806.80045898,377.20897308)
\curveto(806.77045681,377.12896725)(806.74045684,377.04396734)(806.71045898,376.95397308)
\curveto(806.69045689,376.86396752)(806.64545693,376.79396759)(806.57545898,376.74397308)
\curveto(806.53545704,376.71396767)(806.46545711,376.68896769)(806.36545898,376.66897308)
\curveto(806.2754573,376.65896772)(806.1804574,376.65396773)(806.08045898,376.65397308)
\curveto(805.9804576,376.65396773)(805.8804577,376.65896772)(805.78045898,376.66897308)
\curveto(805.69045789,376.68896769)(805.62545795,376.71396767)(805.58545898,376.74397308)
\curveto(805.54545803,376.77396761)(805.51545806,376.82396756)(805.49545898,376.89397308)
\curveto(805.4754581,376.96396742)(805.4754581,377.03896734)(805.49545898,377.11897308)
\curveto(805.52545805,377.24896713)(805.55545802,377.36896701)(805.58545898,377.47897308)
\curveto(805.62545795,377.59896678)(805.67045791,377.71396667)(805.72045898,377.82397308)
\curveto(805.91045767,378.17396621)(806.15045743,378.44396594)(806.44045898,378.63397308)
\curveto(806.73045685,378.83396555)(807.09045649,378.99396539)(807.52045898,379.11397308)
\curveto(807.62045596,379.13396525)(807.72045586,379.14896523)(807.82045898,379.15897308)
\curveto(807.93045565,379.16896521)(808.04045554,379.1839652)(808.15045898,379.20397308)
\curveto(808.19045539,379.21396517)(808.25545532,379.21396517)(808.34545898,379.20397308)
\curveto(808.43545514,379.20396518)(808.49045509,379.21396517)(808.51045898,379.23397308)
\curveto(809.21045437,379.24396514)(809.82045376,379.16396522)(810.34045898,378.99397308)
\curveto(810.86045272,378.82396556)(811.22545235,378.49896588)(811.43545898,378.01897308)
\curveto(811.52545205,377.81896656)(811.575452,377.5839668)(811.58545898,377.31397308)
\curveto(811.60545197,377.05396733)(811.61545196,376.7789676)(811.61545898,376.48897308)
\lineto(811.61545898,373.17397308)
\curveto(811.61545196,373.03397135)(811.62045196,372.89897148)(811.63045898,372.76897308)
\curveto(811.64045194,372.63897174)(811.67045191,372.53397185)(811.72045898,372.45397308)
\curveto(811.77045181,372.383972)(811.83545174,372.33397205)(811.91545898,372.30397308)
\curveto(812.00545157,372.26397212)(812.09045149,372.23397215)(812.17045898,372.21397308)
\curveto(812.25045133,372.20397218)(812.31045127,372.15897222)(812.35045898,372.07897308)
\curveto(812.37045121,372.04897233)(812.3804512,372.01897236)(812.38045898,371.98897308)
\curveto(812.3804512,371.95897242)(812.38545119,371.91897246)(812.39545898,371.86897308)
\moveto(810.25045898,373.53397308)
\curveto(810.31045327,373.67397071)(810.34045324,373.83397055)(810.34045898,374.01397308)
\curveto(810.35045323,374.20397018)(810.35545322,374.39896998)(810.35545898,374.59897308)
\curveto(810.35545322,374.70896967)(810.35045323,374.80896957)(810.34045898,374.89897308)
\curveto(810.33045325,374.98896939)(810.29045329,375.05896932)(810.22045898,375.10897308)
\curveto(810.19045339,375.12896925)(810.12045346,375.13896924)(810.01045898,375.13897308)
\curveto(809.99045359,375.11896926)(809.95545362,375.10896927)(809.90545898,375.10897308)
\curveto(809.85545372,375.10896927)(809.81045377,375.09896928)(809.77045898,375.07897308)
\curveto(809.69045389,375.05896932)(809.60045398,375.03896934)(809.50045898,375.01897308)
\lineto(809.20045898,374.95897308)
\curveto(809.17045441,374.95896942)(809.13545444,374.95396943)(809.09545898,374.94397308)
\lineto(808.99045898,374.94397308)
\curveto(808.84045474,374.90396948)(808.6754549,374.8789695)(808.49545898,374.86897308)
\curveto(808.32545525,374.86896951)(808.16545541,374.84896953)(808.01545898,374.80897308)
\curveto(807.93545564,374.78896959)(807.86045572,374.76896961)(807.79045898,374.74897308)
\curveto(807.73045585,374.73896964)(807.66045592,374.72396966)(807.58045898,374.70397308)
\curveto(807.42045616,374.65396973)(807.27045631,374.58896979)(807.13045898,374.50897308)
\curveto(806.99045659,374.43896994)(806.87045671,374.34897003)(806.77045898,374.23897308)
\curveto(806.67045691,374.12897025)(806.59545698,373.99397039)(806.54545898,373.83397308)
\curveto(806.49545708,373.6839707)(806.4754571,373.49897088)(806.48545898,373.27897308)
\curveto(806.48545709,373.1789712)(806.50045708,373.0839713)(806.53045898,372.99397308)
\curveto(806.57045701,372.91397147)(806.61545696,372.83897154)(806.66545898,372.76897308)
\curveto(806.74545683,372.65897172)(806.85045673,372.56397182)(806.98045898,372.48397308)
\curveto(807.11045647,372.41397197)(807.25045633,372.35397203)(807.40045898,372.30397308)
\curveto(807.45045613,372.29397209)(807.50045608,372.28897209)(807.55045898,372.28897308)
\curveto(807.60045598,372.28897209)(807.65045593,372.2839721)(807.70045898,372.27397308)
\curveto(807.77045581,372.25397213)(807.85545572,372.23897214)(807.95545898,372.22897308)
\curveto(808.06545551,372.22897215)(808.15545542,372.23897214)(808.22545898,372.25897308)
\curveto(808.28545529,372.2789721)(808.34545523,372.2839721)(808.40545898,372.27397308)
\curveto(808.46545511,372.27397211)(808.52545505,372.2839721)(808.58545898,372.30397308)
\curveto(808.66545491,372.32397206)(808.74045484,372.33897204)(808.81045898,372.34897308)
\curveto(808.89045469,372.35897202)(808.96545461,372.378972)(809.03545898,372.40897308)
\curveto(809.32545425,372.52897185)(809.57045401,372.67397171)(809.77045898,372.84397308)
\curveto(809.9804536,373.01397137)(810.14045344,373.24397114)(810.25045898,373.53397308)
}
}
{
\newrgbcolor{curcolor}{0 0 0}
\pscustom[linestyle=none,fillstyle=solid,fillcolor=curcolor]
{
\newpath
\moveto(820.52709961,372.12397308)
\lineto(820.52709961,371.73397308)
\curveto(820.52709173,371.61397277)(820.50209176,371.51397287)(820.45209961,371.43397308)
\curveto(820.40209186,371.36397302)(820.31709194,371.32397306)(820.19709961,371.31397308)
\lineto(819.85209961,371.31397308)
\curveto(819.79209247,371.31397307)(819.73209253,371.30897307)(819.67209961,371.29897308)
\curveto(819.62209264,371.29897308)(819.57709268,371.30897307)(819.53709961,371.32897308)
\curveto(819.44709281,371.34897303)(819.38709287,371.38897299)(819.35709961,371.44897308)
\curveto(819.31709294,371.49897288)(819.29209297,371.55897282)(819.28209961,371.62897308)
\curveto(819.28209298,371.69897268)(819.26709299,371.76897261)(819.23709961,371.83897308)
\curveto(819.22709303,371.85897252)(819.21209305,371.87397251)(819.19209961,371.88397308)
\curveto(819.18209308,371.90397248)(819.16709309,371.92397246)(819.14709961,371.94397308)
\curveto(819.04709321,371.95397243)(818.96709329,371.93397245)(818.90709961,371.88397308)
\curveto(818.8570934,371.83397255)(818.80209346,371.7839726)(818.74209961,371.73397308)
\curveto(818.54209372,371.5839728)(818.34209392,371.46897291)(818.14209961,371.38897308)
\curveto(817.9620943,371.30897307)(817.75209451,371.24897313)(817.51209961,371.20897308)
\curveto(817.28209498,371.16897321)(817.04209522,371.14897323)(816.79209961,371.14897308)
\curveto(816.55209571,371.13897324)(816.31209595,371.15397323)(816.07209961,371.19397308)
\curveto(815.83209643,371.22397316)(815.62209664,371.2789731)(815.44209961,371.35897308)
\curveto(814.92209734,371.5789728)(814.50209776,371.87397251)(814.18209961,372.24397308)
\curveto(813.8620984,372.62397176)(813.61209865,373.09397129)(813.43209961,373.65397308)
\curveto(813.39209887,373.74397064)(813.3620989,373.83397055)(813.34209961,373.92397308)
\curveto(813.33209893,374.02397036)(813.31209895,374.12397026)(813.28209961,374.22397308)
\curveto(813.27209899,374.27397011)(813.26709899,374.32397006)(813.26709961,374.37397308)
\curveto(813.26709899,374.42396996)(813.262099,374.47396991)(813.25209961,374.52397308)
\curveto(813.23209903,374.57396981)(813.22209904,374.62396976)(813.22209961,374.67397308)
\curveto(813.23209903,374.73396965)(813.23209903,374.78896959)(813.22209961,374.83897308)
\lineto(813.22209961,374.98897308)
\curveto(813.20209906,375.03896934)(813.19209907,375.10396928)(813.19209961,375.18397308)
\curveto(813.19209907,375.26396912)(813.20209906,375.32896905)(813.22209961,375.37897308)
\lineto(813.22209961,375.54397308)
\curveto(813.24209902,375.61396877)(813.24709901,375.6839687)(813.23709961,375.75397308)
\curveto(813.23709902,375.83396855)(813.24709901,375.90896847)(813.26709961,375.97897308)
\curveto(813.27709898,376.02896835)(813.28209898,376.07396831)(813.28209961,376.11397308)
\curveto(813.28209898,376.15396823)(813.28709897,376.19896818)(813.29709961,376.24897308)
\curveto(813.32709893,376.34896803)(813.35209891,376.44396794)(813.37209961,376.53397308)
\curveto(813.39209887,376.63396775)(813.41709884,376.72896765)(813.44709961,376.81897308)
\curveto(813.57709868,377.19896718)(813.74209852,377.53896684)(813.94209961,377.83897308)
\curveto(814.15209811,378.14896623)(814.40209786,378.40396598)(814.69209961,378.60397308)
\curveto(814.8620974,378.72396566)(815.03709722,378.82396556)(815.21709961,378.90397308)
\curveto(815.40709685,378.9839654)(815.61209665,379.05396533)(815.83209961,379.11397308)
\curveto(815.90209636,379.12396526)(815.96709629,379.13396525)(816.02709961,379.14397308)
\curveto(816.09709616,379.15396523)(816.16709609,379.16896521)(816.23709961,379.18897308)
\lineto(816.38709961,379.18897308)
\curveto(816.46709579,379.20896517)(816.58209568,379.21896516)(816.73209961,379.21897308)
\curveto(816.89209537,379.21896516)(817.01209525,379.20896517)(817.09209961,379.18897308)
\curveto(817.13209513,379.1789652)(817.18709507,379.17396521)(817.25709961,379.17397308)
\curveto(817.36709489,379.14396524)(817.47709478,379.11896526)(817.58709961,379.09897308)
\curveto(817.69709456,379.08896529)(817.80209446,379.05896532)(817.90209961,379.00897308)
\curveto(818.05209421,378.94896543)(818.19209407,378.8839655)(818.32209961,378.81397308)
\curveto(818.4620938,378.74396564)(818.59209367,378.66396572)(818.71209961,378.57397308)
\curveto(818.77209349,378.52396586)(818.83209343,378.46896591)(818.89209961,378.40897308)
\curveto(818.9620933,378.35896602)(819.05209321,378.34396604)(819.16209961,378.36397308)
\curveto(819.18209308,378.39396599)(819.19709306,378.41896596)(819.20709961,378.43897308)
\curveto(819.22709303,378.45896592)(819.24209302,378.48896589)(819.25209961,378.52897308)
\curveto(819.28209298,378.61896576)(819.29209297,378.73396565)(819.28209961,378.87397308)
\lineto(819.28209961,379.24897308)
\lineto(819.28209961,380.97397308)
\lineto(819.28209961,381.43897308)
\curveto(819.28209298,381.61896276)(819.30709295,381.74896263)(819.35709961,381.82897308)
\curveto(819.39709286,381.89896248)(819.4570928,381.94396244)(819.53709961,381.96397308)
\curveto(819.5570927,381.96396242)(819.58209268,381.96396242)(819.61209961,381.96397308)
\curveto(819.64209262,381.97396241)(819.66709259,381.9789624)(819.68709961,381.97897308)
\curveto(819.82709243,381.98896239)(819.97209229,381.98896239)(820.12209961,381.97897308)
\curveto(820.28209198,381.9789624)(820.39209187,381.93896244)(820.45209961,381.85897308)
\curveto(820.50209176,381.7789626)(820.52709173,381.6789627)(820.52709961,381.55897308)
\lineto(820.52709961,381.18397308)
\lineto(820.52709961,372.12397308)
\moveto(819.31209961,374.95897308)
\curveto(819.33209293,375.00896937)(819.34209292,375.07396931)(819.34209961,375.15397308)
\curveto(819.34209292,375.24396914)(819.33209293,375.31396907)(819.31209961,375.36397308)
\lineto(819.31209961,375.58897308)
\curveto(819.29209297,375.6789687)(819.27709298,375.76896861)(819.26709961,375.85897308)
\curveto(819.257093,375.95896842)(819.23709302,376.04896833)(819.20709961,376.12897308)
\curveto(819.18709307,376.20896817)(819.16709309,376.2839681)(819.14709961,376.35397308)
\curveto(819.13709312,376.42396796)(819.11709314,376.49396789)(819.08709961,376.56397308)
\curveto(818.96709329,376.86396752)(818.81209345,377.12896725)(818.62209961,377.35897308)
\curveto(818.43209383,377.58896679)(818.19209407,377.76896661)(817.90209961,377.89897308)
\curveto(817.80209446,377.94896643)(817.69709456,377.9839664)(817.58709961,378.00397308)
\curveto(817.48709477,378.03396635)(817.37709488,378.05896632)(817.25709961,378.07897308)
\curveto(817.17709508,378.09896628)(817.08709517,378.10896627)(816.98709961,378.10897308)
\lineto(816.71709961,378.10897308)
\curveto(816.66709559,378.09896628)(816.62209564,378.08896629)(816.58209961,378.07897308)
\lineto(816.44709961,378.07897308)
\curveto(816.36709589,378.05896632)(816.28209598,378.03896634)(816.19209961,378.01897308)
\curveto(816.11209615,377.99896638)(816.03209623,377.97396641)(815.95209961,377.94397308)
\curveto(815.63209663,377.80396658)(815.37209689,377.59896678)(815.17209961,377.32897308)
\curveto(814.98209728,377.06896731)(814.82709743,376.76396762)(814.70709961,376.41397308)
\curveto(814.66709759,376.30396808)(814.63709762,376.18896819)(814.61709961,376.06897308)
\curveto(814.60709765,375.95896842)(814.59209767,375.84896853)(814.57209961,375.73897308)
\curveto(814.57209769,375.69896868)(814.56709769,375.65896872)(814.55709961,375.61897308)
\lineto(814.55709961,375.51397308)
\curveto(814.53709772,375.46396892)(814.52709773,375.40896897)(814.52709961,375.34897308)
\curveto(814.53709772,375.28896909)(814.54209772,375.23396915)(814.54209961,375.18397308)
\lineto(814.54209961,374.85397308)
\curveto(814.54209772,374.75396963)(814.55209771,374.65896972)(814.57209961,374.56897308)
\curveto(814.58209768,374.53896984)(814.58709767,374.48896989)(814.58709961,374.41897308)
\curveto(814.60709765,374.34897003)(814.62209764,374.2789701)(814.63209961,374.20897308)
\lineto(814.69209961,373.99897308)
\curveto(814.80209746,373.64897073)(814.95209731,373.34897103)(815.14209961,373.09897308)
\curveto(815.33209693,372.84897153)(815.57209669,372.64397174)(815.86209961,372.48397308)
\curveto(815.95209631,372.43397195)(816.04209622,372.39397199)(816.13209961,372.36397308)
\curveto(816.22209604,372.33397205)(816.32209594,372.30397208)(816.43209961,372.27397308)
\curveto(816.48209578,372.25397213)(816.53209573,372.24897213)(816.58209961,372.25897308)
\curveto(816.64209562,372.26897211)(816.69709556,372.26397212)(816.74709961,372.24397308)
\curveto(816.78709547,372.23397215)(816.82709543,372.22897215)(816.86709961,372.22897308)
\lineto(817.00209961,372.22897308)
\lineto(817.13709961,372.22897308)
\curveto(817.16709509,372.23897214)(817.21709504,372.24397214)(817.28709961,372.24397308)
\curveto(817.36709489,372.26397212)(817.44709481,372.2789721)(817.52709961,372.28897308)
\curveto(817.60709465,372.30897207)(817.68209458,372.33397205)(817.75209961,372.36397308)
\curveto(818.08209418,372.50397188)(818.34709391,372.6789717)(818.54709961,372.88897308)
\curveto(818.7570935,373.10897127)(818.93209333,373.383971)(819.07209961,373.71397308)
\curveto(819.12209314,373.82397056)(819.1570931,373.93397045)(819.17709961,374.04397308)
\curveto(819.19709306,374.15397023)(819.22209304,374.26397012)(819.25209961,374.37397308)
\curveto(819.27209299,374.41396997)(819.28209298,374.44896993)(819.28209961,374.47897308)
\curveto(819.28209298,374.51896986)(819.28709297,374.55896982)(819.29709961,374.59897308)
\curveto(819.30709295,374.65896972)(819.30709295,374.71896966)(819.29709961,374.77897308)
\curveto(819.29709296,374.83896954)(819.30209296,374.89896948)(819.31209961,374.95897308)
}
}
{
\newrgbcolor{curcolor}{0 0 0}
\pscustom[linestyle=none,fillstyle=solid,fillcolor=curcolor]
{
}
}
{
\newrgbcolor{curcolor}{0 0 0}
\pscustom[linestyle=none,fillstyle=solid,fillcolor=curcolor]
{
\newpath
\moveto(833.59350586,372.12397308)
\lineto(833.59350586,371.73397308)
\curveto(833.59349798,371.61397277)(833.56849801,371.51397287)(833.51850586,371.43397308)
\curveto(833.46849811,371.36397302)(833.38349819,371.32397306)(833.26350586,371.31397308)
\lineto(832.91850586,371.31397308)
\curveto(832.85849872,371.31397307)(832.79849878,371.30897307)(832.73850586,371.29897308)
\curveto(832.68849889,371.29897308)(832.64349893,371.30897307)(832.60350586,371.32897308)
\curveto(832.51349906,371.34897303)(832.45349912,371.38897299)(832.42350586,371.44897308)
\curveto(832.38349919,371.49897288)(832.35849922,371.55897282)(832.34850586,371.62897308)
\curveto(832.34849923,371.69897268)(832.33349924,371.76897261)(832.30350586,371.83897308)
\curveto(832.29349928,371.85897252)(832.2784993,371.87397251)(832.25850586,371.88397308)
\curveto(832.24849933,371.90397248)(832.23349934,371.92397246)(832.21350586,371.94397308)
\curveto(832.11349946,371.95397243)(832.03349954,371.93397245)(831.97350586,371.88397308)
\curveto(831.92349965,371.83397255)(831.86849971,371.7839726)(831.80850586,371.73397308)
\curveto(831.60849997,371.5839728)(831.40850017,371.46897291)(831.20850586,371.38897308)
\curveto(831.02850055,371.30897307)(830.81850076,371.24897313)(830.57850586,371.20897308)
\curveto(830.34850123,371.16897321)(830.10850147,371.14897323)(829.85850586,371.14897308)
\curveto(829.61850196,371.13897324)(829.3785022,371.15397323)(829.13850586,371.19397308)
\curveto(828.89850268,371.22397316)(828.68850289,371.2789731)(828.50850586,371.35897308)
\curveto(827.98850359,371.5789728)(827.56850401,371.87397251)(827.24850586,372.24397308)
\curveto(826.92850465,372.62397176)(826.6785049,373.09397129)(826.49850586,373.65397308)
\curveto(826.45850512,373.74397064)(826.42850515,373.83397055)(826.40850586,373.92397308)
\curveto(826.39850518,374.02397036)(826.3785052,374.12397026)(826.34850586,374.22397308)
\curveto(826.33850524,374.27397011)(826.33350524,374.32397006)(826.33350586,374.37397308)
\curveto(826.33350524,374.42396996)(826.32850525,374.47396991)(826.31850586,374.52397308)
\curveto(826.29850528,374.57396981)(826.28850529,374.62396976)(826.28850586,374.67397308)
\curveto(826.29850528,374.73396965)(826.29850528,374.78896959)(826.28850586,374.83897308)
\lineto(826.28850586,374.98897308)
\curveto(826.26850531,375.03896934)(826.25850532,375.10396928)(826.25850586,375.18397308)
\curveto(826.25850532,375.26396912)(826.26850531,375.32896905)(826.28850586,375.37897308)
\lineto(826.28850586,375.54397308)
\curveto(826.30850527,375.61396877)(826.31350526,375.6839687)(826.30350586,375.75397308)
\curveto(826.30350527,375.83396855)(826.31350526,375.90896847)(826.33350586,375.97897308)
\curveto(826.34350523,376.02896835)(826.34850523,376.07396831)(826.34850586,376.11397308)
\curveto(826.34850523,376.15396823)(826.35350522,376.19896818)(826.36350586,376.24897308)
\curveto(826.39350518,376.34896803)(826.41850516,376.44396794)(826.43850586,376.53397308)
\curveto(826.45850512,376.63396775)(826.48350509,376.72896765)(826.51350586,376.81897308)
\curveto(826.64350493,377.19896718)(826.80850477,377.53896684)(827.00850586,377.83897308)
\curveto(827.21850436,378.14896623)(827.46850411,378.40396598)(827.75850586,378.60397308)
\curveto(827.92850365,378.72396566)(828.10350347,378.82396556)(828.28350586,378.90397308)
\curveto(828.4735031,378.9839654)(828.6785029,379.05396533)(828.89850586,379.11397308)
\curveto(828.96850261,379.12396526)(829.03350254,379.13396525)(829.09350586,379.14397308)
\curveto(829.16350241,379.15396523)(829.23350234,379.16896521)(829.30350586,379.18897308)
\lineto(829.45350586,379.18897308)
\curveto(829.53350204,379.20896517)(829.64850193,379.21896516)(829.79850586,379.21897308)
\curveto(829.95850162,379.21896516)(830.0785015,379.20896517)(830.15850586,379.18897308)
\curveto(830.19850138,379.1789652)(830.25350132,379.17396521)(830.32350586,379.17397308)
\curveto(830.43350114,379.14396524)(830.54350103,379.11896526)(830.65350586,379.09897308)
\curveto(830.76350081,379.08896529)(830.86850071,379.05896532)(830.96850586,379.00897308)
\curveto(831.11850046,378.94896543)(831.25850032,378.8839655)(831.38850586,378.81397308)
\curveto(831.52850005,378.74396564)(831.65849992,378.66396572)(831.77850586,378.57397308)
\curveto(831.83849974,378.52396586)(831.89849968,378.46896591)(831.95850586,378.40897308)
\curveto(832.02849955,378.35896602)(832.11849946,378.34396604)(832.22850586,378.36397308)
\curveto(832.24849933,378.39396599)(832.26349931,378.41896596)(832.27350586,378.43897308)
\curveto(832.29349928,378.45896592)(832.30849927,378.48896589)(832.31850586,378.52897308)
\curveto(832.34849923,378.61896576)(832.35849922,378.73396565)(832.34850586,378.87397308)
\lineto(832.34850586,379.24897308)
\lineto(832.34850586,380.97397308)
\lineto(832.34850586,381.43897308)
\curveto(832.34849923,381.61896276)(832.3734992,381.74896263)(832.42350586,381.82897308)
\curveto(832.46349911,381.89896248)(832.52349905,381.94396244)(832.60350586,381.96397308)
\curveto(832.62349895,381.96396242)(832.64849893,381.96396242)(832.67850586,381.96397308)
\curveto(832.70849887,381.97396241)(832.73349884,381.9789624)(832.75350586,381.97897308)
\curveto(832.89349868,381.98896239)(833.03849854,381.98896239)(833.18850586,381.97897308)
\curveto(833.34849823,381.9789624)(833.45849812,381.93896244)(833.51850586,381.85897308)
\curveto(833.56849801,381.7789626)(833.59349798,381.6789627)(833.59350586,381.55897308)
\lineto(833.59350586,381.18397308)
\lineto(833.59350586,372.12397308)
\moveto(832.37850586,374.95897308)
\curveto(832.39849918,375.00896937)(832.40849917,375.07396931)(832.40850586,375.15397308)
\curveto(832.40849917,375.24396914)(832.39849918,375.31396907)(832.37850586,375.36397308)
\lineto(832.37850586,375.58897308)
\curveto(832.35849922,375.6789687)(832.34349923,375.76896861)(832.33350586,375.85897308)
\curveto(832.32349925,375.95896842)(832.30349927,376.04896833)(832.27350586,376.12897308)
\curveto(832.25349932,376.20896817)(832.23349934,376.2839681)(832.21350586,376.35397308)
\curveto(832.20349937,376.42396796)(832.18349939,376.49396789)(832.15350586,376.56397308)
\curveto(832.03349954,376.86396752)(831.8784997,377.12896725)(831.68850586,377.35897308)
\curveto(831.49850008,377.58896679)(831.25850032,377.76896661)(830.96850586,377.89897308)
\curveto(830.86850071,377.94896643)(830.76350081,377.9839664)(830.65350586,378.00397308)
\curveto(830.55350102,378.03396635)(830.44350113,378.05896632)(830.32350586,378.07897308)
\curveto(830.24350133,378.09896628)(830.15350142,378.10896627)(830.05350586,378.10897308)
\lineto(829.78350586,378.10897308)
\curveto(829.73350184,378.09896628)(829.68850189,378.08896629)(829.64850586,378.07897308)
\lineto(829.51350586,378.07897308)
\curveto(829.43350214,378.05896632)(829.34850223,378.03896634)(829.25850586,378.01897308)
\curveto(829.1785024,377.99896638)(829.09850248,377.97396641)(829.01850586,377.94397308)
\curveto(828.69850288,377.80396658)(828.43850314,377.59896678)(828.23850586,377.32897308)
\curveto(828.04850353,377.06896731)(827.89350368,376.76396762)(827.77350586,376.41397308)
\curveto(827.73350384,376.30396808)(827.70350387,376.18896819)(827.68350586,376.06897308)
\curveto(827.6735039,375.95896842)(827.65850392,375.84896853)(827.63850586,375.73897308)
\curveto(827.63850394,375.69896868)(827.63350394,375.65896872)(827.62350586,375.61897308)
\lineto(827.62350586,375.51397308)
\curveto(827.60350397,375.46396892)(827.59350398,375.40896897)(827.59350586,375.34897308)
\curveto(827.60350397,375.28896909)(827.60850397,375.23396915)(827.60850586,375.18397308)
\lineto(827.60850586,374.85397308)
\curveto(827.60850397,374.75396963)(827.61850396,374.65896972)(827.63850586,374.56897308)
\curveto(827.64850393,374.53896984)(827.65350392,374.48896989)(827.65350586,374.41897308)
\curveto(827.6735039,374.34897003)(827.68850389,374.2789701)(827.69850586,374.20897308)
\lineto(827.75850586,373.99897308)
\curveto(827.86850371,373.64897073)(828.01850356,373.34897103)(828.20850586,373.09897308)
\curveto(828.39850318,372.84897153)(828.63850294,372.64397174)(828.92850586,372.48397308)
\curveto(829.01850256,372.43397195)(829.10850247,372.39397199)(829.19850586,372.36397308)
\curveto(829.28850229,372.33397205)(829.38850219,372.30397208)(829.49850586,372.27397308)
\curveto(829.54850203,372.25397213)(829.59850198,372.24897213)(829.64850586,372.25897308)
\curveto(829.70850187,372.26897211)(829.76350181,372.26397212)(829.81350586,372.24397308)
\curveto(829.85350172,372.23397215)(829.89350168,372.22897215)(829.93350586,372.22897308)
\lineto(830.06850586,372.22897308)
\lineto(830.20350586,372.22897308)
\curveto(830.23350134,372.23897214)(830.28350129,372.24397214)(830.35350586,372.24397308)
\curveto(830.43350114,372.26397212)(830.51350106,372.2789721)(830.59350586,372.28897308)
\curveto(830.6735009,372.30897207)(830.74850083,372.33397205)(830.81850586,372.36397308)
\curveto(831.14850043,372.50397188)(831.41350016,372.6789717)(831.61350586,372.88897308)
\curveto(831.82349975,373.10897127)(831.99849958,373.383971)(832.13850586,373.71397308)
\curveto(832.18849939,373.82397056)(832.22349935,373.93397045)(832.24350586,374.04397308)
\curveto(832.26349931,374.15397023)(832.28849929,374.26397012)(832.31850586,374.37397308)
\curveto(832.33849924,374.41396997)(832.34849923,374.44896993)(832.34850586,374.47897308)
\curveto(832.34849923,374.51896986)(832.35349922,374.55896982)(832.36350586,374.59897308)
\curveto(832.3734992,374.65896972)(832.3734992,374.71896966)(832.36350586,374.77897308)
\curveto(832.36349921,374.83896954)(832.36849921,374.89896948)(832.37850586,374.95897308)
}
}
{
\newrgbcolor{curcolor}{0 0 0}
\pscustom[linestyle=none,fillstyle=solid,fillcolor=curcolor]
{
\newpath
\moveto(842.28975586,375.48397308)
\curveto(842.30974817,375.383969)(842.30974817,375.26896911)(842.28975586,375.13897308)
\curveto(842.2797482,375.01896936)(842.24974823,374.93396945)(842.19975586,374.88397308)
\curveto(842.14974833,374.84396954)(842.07474841,374.81396957)(841.97475586,374.79397308)
\curveto(841.8847486,374.7839696)(841.7797487,374.7789696)(841.65975586,374.77897308)
\lineto(841.29975586,374.77897308)
\curveto(841.1797493,374.78896959)(841.07474941,374.79396959)(840.98475586,374.79397308)
\lineto(837.14475586,374.79397308)
\curveto(837.06475342,374.79396959)(836.9847535,374.78896959)(836.90475586,374.77897308)
\curveto(836.82475366,374.7789696)(836.75975372,374.76396962)(836.70975586,374.73397308)
\curveto(836.66975381,374.71396967)(836.62975385,374.67396971)(836.58975586,374.61397308)
\curveto(836.56975391,374.5839698)(836.54975393,374.53896984)(836.52975586,374.47897308)
\curveto(836.50975397,374.42896995)(836.50975397,374.37897)(836.52975586,374.32897308)
\curveto(836.53975394,374.2789701)(836.54475394,374.23397015)(836.54475586,374.19397308)
\curveto(836.54475394,374.15397023)(836.54975393,374.11397027)(836.55975586,374.07397308)
\curveto(836.5797539,373.99397039)(836.59975388,373.90897047)(836.61975586,373.81897308)
\curveto(836.63975384,373.73897064)(836.66975381,373.65897072)(836.70975586,373.57897308)
\curveto(836.93975354,373.03897134)(837.31975316,372.65397173)(837.84975586,372.42397308)
\curveto(837.90975257,372.39397199)(837.97475251,372.36897201)(838.04475586,372.34897308)
\lineto(838.25475586,372.28897308)
\curveto(838.2847522,372.2789721)(838.33475215,372.27397211)(838.40475586,372.27397308)
\curveto(838.54475194,372.23397215)(838.72975175,372.21397217)(838.95975586,372.21397308)
\curveto(839.18975129,372.21397217)(839.37475111,372.23397215)(839.51475586,372.27397308)
\curveto(839.65475083,372.31397207)(839.7797507,372.35397203)(839.88975586,372.39397308)
\curveto(840.00975047,372.44397194)(840.11975036,372.50397188)(840.21975586,372.57397308)
\curveto(840.32975015,372.64397174)(840.42475006,372.72397166)(840.50475586,372.81397308)
\curveto(840.5847499,372.91397147)(840.65474983,373.01897136)(840.71475586,373.12897308)
\curveto(840.77474971,373.22897115)(840.82474966,373.33397105)(840.86475586,373.44397308)
\curveto(840.91474957,373.55397083)(840.99474949,373.63397075)(841.10475586,373.68397308)
\curveto(841.14474934,373.70397068)(841.20974927,373.71897066)(841.29975586,373.72897308)
\curveto(841.38974909,373.73897064)(841.479749,373.73897064)(841.56975586,373.72897308)
\curveto(841.65974882,373.72897065)(841.74474874,373.72397066)(841.82475586,373.71397308)
\curveto(841.90474858,373.70397068)(841.95974852,373.6839707)(841.98975586,373.65397308)
\curveto(842.08974839,373.5839708)(842.11474837,373.46897091)(842.06475586,373.30897308)
\curveto(841.9847485,373.03897134)(841.8797486,372.79897158)(841.74975586,372.58897308)
\curveto(841.54974893,372.26897211)(841.31974916,372.00397238)(841.05975586,371.79397308)
\curveto(840.80974967,371.59397279)(840.48974999,371.42897295)(840.09975586,371.29897308)
\curveto(839.99975048,371.25897312)(839.89975058,371.23397315)(839.79975586,371.22397308)
\curveto(839.69975078,371.20397318)(839.59475089,371.1839732)(839.48475586,371.16397308)
\curveto(839.43475105,371.15397323)(839.3847511,371.14897323)(839.33475586,371.14897308)
\curveto(839.29475119,371.14897323)(839.24975123,371.14397324)(839.19975586,371.13397308)
\lineto(839.04975586,371.13397308)
\curveto(838.99975148,371.12397326)(838.93975154,371.11897326)(838.86975586,371.11897308)
\curveto(838.80975167,371.11897326)(838.75975172,371.12397326)(838.71975586,371.13397308)
\lineto(838.58475586,371.13397308)
\curveto(838.53475195,371.14397324)(838.48975199,371.14897323)(838.44975586,371.14897308)
\curveto(838.40975207,371.14897323)(838.36975211,371.15397323)(838.32975586,371.16397308)
\curveto(838.2797522,371.17397321)(838.22475226,371.1839732)(838.16475586,371.19397308)
\curveto(838.10475238,371.19397319)(838.04975243,371.19897318)(837.99975586,371.20897308)
\curveto(837.90975257,371.22897315)(837.81975266,371.25397313)(837.72975586,371.28397308)
\curveto(837.63975284,371.30397308)(837.55475293,371.32897305)(837.47475586,371.35897308)
\curveto(837.43475305,371.378973)(837.39975308,371.38897299)(837.36975586,371.38897308)
\curveto(837.33975314,371.39897298)(837.30475318,371.41397297)(837.26475586,371.43397308)
\curveto(837.11475337,371.50397288)(836.95475353,371.58897279)(836.78475586,371.68897308)
\curveto(836.49475399,371.8789725)(836.24475424,372.10897227)(836.03475586,372.37897308)
\curveto(835.83475465,372.65897172)(835.66475482,372.96897141)(835.52475586,373.30897308)
\curveto(835.47475501,373.41897096)(835.43475505,373.53397085)(835.40475586,373.65397308)
\curveto(835.3847551,373.77397061)(835.35475513,373.89397049)(835.31475586,374.01397308)
\curveto(835.30475518,374.05397033)(835.29975518,374.08897029)(835.29975586,374.11897308)
\curveto(835.29975518,374.14897023)(835.29475519,374.18897019)(835.28475586,374.23897308)
\curveto(835.26475522,374.31897006)(835.24975523,374.40396998)(835.23975586,374.49397308)
\curveto(835.22975525,374.5839698)(835.21475527,374.67396971)(835.19475586,374.76397308)
\lineto(835.19475586,374.97397308)
\curveto(835.1847553,375.01396937)(835.17475531,375.06896931)(835.16475586,375.13897308)
\curveto(835.16475532,375.21896916)(835.16975531,375.2839691)(835.17975586,375.33397308)
\lineto(835.17975586,375.49897308)
\curveto(835.19975528,375.54896883)(835.20475528,375.59896878)(835.19475586,375.64897308)
\curveto(835.19475529,375.70896867)(835.19975528,375.76396862)(835.20975586,375.81397308)
\curveto(835.24975523,375.97396841)(835.2797552,376.13396825)(835.29975586,376.29397308)
\curveto(835.32975515,376.45396793)(835.37475511,376.60396778)(835.43475586,376.74397308)
\curveto(835.484755,376.85396753)(835.52975495,376.96396742)(835.56975586,377.07397308)
\curveto(835.61975486,377.19396719)(835.67475481,377.30896707)(835.73475586,377.41897308)
\curveto(835.95475453,377.76896661)(836.20475428,378.06896631)(836.48475586,378.31897308)
\curveto(836.76475372,378.5789658)(837.10975337,378.79396559)(837.51975586,378.96397308)
\curveto(837.63975284,379.01396537)(837.75975272,379.04896533)(837.87975586,379.06897308)
\curveto(838.00975247,379.09896528)(838.14475234,379.12896525)(838.28475586,379.15897308)
\curveto(838.33475215,379.16896521)(838.3797521,379.17396521)(838.41975586,379.17397308)
\curveto(838.45975202,379.1839652)(838.50475198,379.18896519)(838.55475586,379.18897308)
\curveto(838.57475191,379.19896518)(838.59975188,379.19896518)(838.62975586,379.18897308)
\curveto(838.65975182,379.1789652)(838.6847518,379.1839652)(838.70475586,379.20397308)
\curveto(839.12475136,379.21396517)(839.48975099,379.16896521)(839.79975586,379.06897308)
\curveto(840.10975037,378.9789654)(840.38975009,378.85396553)(840.63975586,378.69397308)
\curveto(840.68974979,378.67396571)(840.72974975,378.64396574)(840.75975586,378.60397308)
\curveto(840.78974969,378.57396581)(840.82474966,378.54896583)(840.86475586,378.52897308)
\curveto(840.94474954,378.46896591)(841.02474946,378.39896598)(841.10475586,378.31897308)
\curveto(841.19474929,378.23896614)(841.26974921,378.15896622)(841.32975586,378.07897308)
\curveto(841.48974899,377.86896651)(841.62474886,377.66896671)(841.73475586,377.47897308)
\curveto(841.80474868,377.36896701)(841.85974862,377.24896713)(841.89975586,377.11897308)
\curveto(841.93974854,376.98896739)(841.9847485,376.85896752)(842.03475586,376.72897308)
\curveto(842.0847484,376.59896778)(842.11974836,376.46396792)(842.13975586,376.32397308)
\curveto(842.16974831,376.1839682)(842.20474828,376.04396834)(842.24475586,375.90397308)
\curveto(842.25474823,375.83396855)(842.25974822,375.76396862)(842.25975586,375.69397308)
\lineto(842.28975586,375.48397308)
\moveto(840.83475586,375.99397308)
\curveto(840.86474962,376.03396835)(840.88974959,376.0839683)(840.90975586,376.14397308)
\curveto(840.92974955,376.21396817)(840.92974955,376.2839681)(840.90975586,376.35397308)
\curveto(840.84974963,376.57396781)(840.76474972,376.7789676)(840.65475586,376.96897308)
\curveto(840.51474997,377.19896718)(840.35975012,377.39396699)(840.18975586,377.55397308)
\curveto(840.01975046,377.71396667)(839.79975068,377.84896653)(839.52975586,377.95897308)
\curveto(839.45975102,377.9789664)(839.38975109,377.99396639)(839.31975586,378.00397308)
\curveto(839.24975123,378.02396636)(839.17475131,378.04396634)(839.09475586,378.06397308)
\curveto(839.01475147,378.0839663)(838.92975155,378.09396629)(838.83975586,378.09397308)
\lineto(838.58475586,378.09397308)
\curveto(838.55475193,378.07396631)(838.51975196,378.06396632)(838.47975586,378.06397308)
\curveto(838.43975204,378.07396631)(838.40475208,378.07396631)(838.37475586,378.06397308)
\lineto(838.13475586,378.00397308)
\curveto(838.06475242,377.99396639)(837.99475249,377.9789664)(837.92475586,377.95897308)
\curveto(837.63475285,377.83896654)(837.39975308,377.68896669)(837.21975586,377.50897308)
\curveto(837.04975343,377.32896705)(836.89475359,377.10396728)(836.75475586,376.83397308)
\curveto(836.72475376,376.7839676)(836.69475379,376.71896766)(836.66475586,376.63897308)
\curveto(836.63475385,376.56896781)(836.60975387,376.48896789)(836.58975586,376.39897308)
\curveto(836.56975391,376.30896807)(836.56475392,376.22396816)(836.57475586,376.14397308)
\curveto(836.5847539,376.06396832)(836.61975386,376.00396838)(836.67975586,375.96397308)
\curveto(836.75975372,375.90396848)(836.89475359,375.87396851)(837.08475586,375.87397308)
\curveto(837.2847532,375.8839685)(837.45475303,375.88896849)(837.59475586,375.88897308)
\lineto(839.87475586,375.88897308)
\curveto(840.02475046,375.88896849)(840.20475028,375.8839685)(840.41475586,375.87397308)
\curveto(840.62474986,375.87396851)(840.76474972,375.91396847)(840.83475586,375.99397308)
}
}
{
\newrgbcolor{curcolor}{0 0 0}
\pscustom[linestyle=none,fillstyle=solid,fillcolor=curcolor]
{
\newpath
\moveto(768.02144531,357.41765076)
\curveto(768.04143763,357.31764667)(768.04143763,357.20264679)(768.02144531,357.07265076)
\curveto(768.01143766,356.95264704)(767.98143769,356.86764712)(767.93144531,356.81765076)
\curveto(767.88143779,356.77764721)(767.80643786,356.74764724)(767.70644531,356.72765076)
\curveto(767.61643805,356.71764727)(767.51143816,356.71264728)(767.39144531,356.71265076)
\lineto(767.03144531,356.71265076)
\curveto(766.91143876,356.72264727)(766.80643886,356.72764726)(766.71644531,356.72765076)
\lineto(762.87644531,356.72765076)
\curveto(762.79644287,356.72764726)(762.71644295,356.72264727)(762.63644531,356.71265076)
\curveto(762.55644311,356.71264728)(762.49144318,356.69764729)(762.44144531,356.66765076)
\curveto(762.40144327,356.64764734)(762.36144331,356.60764738)(762.32144531,356.54765076)
\curveto(762.30144337,356.51764747)(762.28144339,356.47264752)(762.26144531,356.41265076)
\curveto(762.24144343,356.36264763)(762.24144343,356.31264768)(762.26144531,356.26265076)
\curveto(762.2714434,356.21264778)(762.27644339,356.16764782)(762.27644531,356.12765076)
\curveto(762.27644339,356.0876479)(762.28144339,356.04764794)(762.29144531,356.00765076)
\curveto(762.31144336,355.92764806)(762.33144334,355.84264815)(762.35144531,355.75265076)
\curveto(762.3714433,355.67264832)(762.40144327,355.5926484)(762.44144531,355.51265076)
\curveto(762.671443,354.97264902)(763.05144262,354.5876494)(763.58144531,354.35765076)
\curveto(763.64144203,354.32764966)(763.70644196,354.30264969)(763.77644531,354.28265076)
\lineto(763.98644531,354.22265076)
\curveto(764.01644165,354.21264978)(764.0664416,354.20764978)(764.13644531,354.20765076)
\curveto(764.27644139,354.16764982)(764.46144121,354.14764984)(764.69144531,354.14765076)
\curveto(764.92144075,354.14764984)(765.10644056,354.16764982)(765.24644531,354.20765076)
\curveto(765.38644028,354.24764974)(765.51144016,354.2876497)(765.62144531,354.32765076)
\curveto(765.74143993,354.37764961)(765.85143982,354.43764955)(765.95144531,354.50765076)
\curveto(766.06143961,354.57764941)(766.15643951,354.65764933)(766.23644531,354.74765076)
\curveto(766.31643935,354.84764914)(766.38643928,354.95264904)(766.44644531,355.06265076)
\curveto(766.50643916,355.16264883)(766.55643911,355.26764872)(766.59644531,355.37765076)
\curveto(766.64643902,355.4876485)(766.72643894,355.56764842)(766.83644531,355.61765076)
\curveto(766.87643879,355.63764835)(766.94143873,355.65264834)(767.03144531,355.66265076)
\curveto(767.12143855,355.67264832)(767.21143846,355.67264832)(767.30144531,355.66265076)
\curveto(767.39143828,355.66264833)(767.47643819,355.65764833)(767.55644531,355.64765076)
\curveto(767.63643803,355.63764835)(767.69143798,355.61764837)(767.72144531,355.58765076)
\curveto(767.82143785,355.51764847)(767.84643782,355.40264859)(767.79644531,355.24265076)
\curveto(767.71643795,354.97264902)(767.61143806,354.73264926)(767.48144531,354.52265076)
\curveto(767.28143839,354.20264979)(767.05143862,353.93765005)(766.79144531,353.72765076)
\curveto(766.54143913,353.52765046)(766.22143945,353.36265063)(765.83144531,353.23265076)
\curveto(765.73143994,353.1926508)(765.63144004,353.16765082)(765.53144531,353.15765076)
\curveto(765.43144024,353.13765085)(765.32644034,353.11765087)(765.21644531,353.09765076)
\curveto(765.1664405,353.0876509)(765.11644055,353.08265091)(765.06644531,353.08265076)
\curveto(765.02644064,353.08265091)(764.98144069,353.07765091)(764.93144531,353.06765076)
\lineto(764.78144531,353.06765076)
\curveto(764.73144094,353.05765093)(764.671441,353.05265094)(764.60144531,353.05265076)
\curveto(764.54144113,353.05265094)(764.49144118,353.05765093)(764.45144531,353.06765076)
\lineto(764.31644531,353.06765076)
\curveto(764.2664414,353.07765091)(764.22144145,353.08265091)(764.18144531,353.08265076)
\curveto(764.14144153,353.08265091)(764.10144157,353.0876509)(764.06144531,353.09765076)
\curveto(764.01144166,353.10765088)(763.95644171,353.11765087)(763.89644531,353.12765076)
\curveto(763.83644183,353.12765086)(763.78144189,353.13265086)(763.73144531,353.14265076)
\curveto(763.64144203,353.16265083)(763.55144212,353.1876508)(763.46144531,353.21765076)
\curveto(763.3714423,353.23765075)(763.28644238,353.26265073)(763.20644531,353.29265076)
\curveto(763.1664425,353.31265068)(763.13144254,353.32265067)(763.10144531,353.32265076)
\curveto(763.0714426,353.33265066)(763.03644263,353.34765064)(762.99644531,353.36765076)
\curveto(762.84644282,353.43765055)(762.68644298,353.52265047)(762.51644531,353.62265076)
\curveto(762.22644344,353.81265018)(761.97644369,354.04264995)(761.76644531,354.31265076)
\curveto(761.5664441,354.5926494)(761.39644427,354.90264909)(761.25644531,355.24265076)
\curveto(761.20644446,355.35264864)(761.1664445,355.46764852)(761.13644531,355.58765076)
\curveto(761.11644455,355.70764828)(761.08644458,355.82764816)(761.04644531,355.94765076)
\curveto(761.03644463,355.987648)(761.03144464,356.02264797)(761.03144531,356.05265076)
\curveto(761.03144464,356.08264791)(761.02644464,356.12264787)(761.01644531,356.17265076)
\curveto(760.99644467,356.25264774)(760.98144469,356.33764765)(760.97144531,356.42765076)
\curveto(760.96144471,356.51764747)(760.94644472,356.60764738)(760.92644531,356.69765076)
\lineto(760.92644531,356.90765076)
\curveto(760.91644475,356.94764704)(760.90644476,357.00264699)(760.89644531,357.07265076)
\curveto(760.89644477,357.15264684)(760.90144477,357.21764677)(760.91144531,357.26765076)
\lineto(760.91144531,357.43265076)
\curveto(760.93144474,357.48264651)(760.93644473,357.53264646)(760.92644531,357.58265076)
\curveto(760.92644474,357.64264635)(760.93144474,357.69764629)(760.94144531,357.74765076)
\curveto(760.98144469,357.90764608)(761.01144466,358.06764592)(761.03144531,358.22765076)
\curveto(761.06144461,358.3876456)(761.10644456,358.53764545)(761.16644531,358.67765076)
\curveto(761.21644445,358.7876452)(761.26144441,358.89764509)(761.30144531,359.00765076)
\curveto(761.35144432,359.12764486)(761.40644426,359.24264475)(761.46644531,359.35265076)
\curveto(761.68644398,359.70264429)(761.93644373,360.00264399)(762.21644531,360.25265076)
\curveto(762.49644317,360.51264348)(762.84144283,360.72764326)(763.25144531,360.89765076)
\curveto(763.3714423,360.94764304)(763.49144218,360.98264301)(763.61144531,361.00265076)
\curveto(763.74144193,361.03264296)(763.87644179,361.06264293)(764.01644531,361.09265076)
\curveto(764.0664416,361.10264289)(764.11144156,361.10764288)(764.15144531,361.10765076)
\curveto(764.19144148,361.11764287)(764.23644143,361.12264287)(764.28644531,361.12265076)
\curveto(764.30644136,361.13264286)(764.33144134,361.13264286)(764.36144531,361.12265076)
\curveto(764.39144128,361.11264288)(764.41644125,361.11764287)(764.43644531,361.13765076)
\curveto(764.85644081,361.14764284)(765.22144045,361.10264289)(765.53144531,361.00265076)
\curveto(765.84143983,360.91264308)(766.12143955,360.7876432)(766.37144531,360.62765076)
\curveto(766.42143925,360.60764338)(766.46143921,360.57764341)(766.49144531,360.53765076)
\curveto(766.52143915,360.50764348)(766.55643911,360.48264351)(766.59644531,360.46265076)
\curveto(766.67643899,360.40264359)(766.75643891,360.33264366)(766.83644531,360.25265076)
\curveto(766.92643874,360.17264382)(767.00143867,360.0926439)(767.06144531,360.01265076)
\curveto(767.22143845,359.80264419)(767.35643831,359.60264439)(767.46644531,359.41265076)
\curveto(767.53643813,359.30264469)(767.59143808,359.18264481)(767.63144531,359.05265076)
\curveto(767.671438,358.92264507)(767.71643795,358.7926452)(767.76644531,358.66265076)
\curveto(767.81643785,358.53264546)(767.85143782,358.39764559)(767.87144531,358.25765076)
\curveto(767.90143777,358.11764587)(767.93643773,357.97764601)(767.97644531,357.83765076)
\curveto(767.98643768,357.76764622)(767.99143768,357.69764629)(767.99144531,357.62765076)
\lineto(768.02144531,357.41765076)
\moveto(766.56644531,357.92765076)
\curveto(766.59643907,357.96764602)(766.62143905,358.01764597)(766.64144531,358.07765076)
\curveto(766.66143901,358.14764584)(766.66143901,358.21764577)(766.64144531,358.28765076)
\curveto(766.58143909,358.50764548)(766.49643917,358.71264528)(766.38644531,358.90265076)
\curveto(766.24643942,359.13264486)(766.09143958,359.32764466)(765.92144531,359.48765076)
\curveto(765.75143992,359.64764434)(765.53144014,359.78264421)(765.26144531,359.89265076)
\curveto(765.19144048,359.91264408)(765.12144055,359.92764406)(765.05144531,359.93765076)
\curveto(764.98144069,359.95764403)(764.90644076,359.97764401)(764.82644531,359.99765076)
\curveto(764.74644092,360.01764397)(764.66144101,360.02764396)(764.57144531,360.02765076)
\lineto(764.31644531,360.02765076)
\curveto(764.28644138,360.00764398)(764.25144142,359.99764399)(764.21144531,359.99765076)
\curveto(764.1714415,360.00764398)(764.13644153,360.00764398)(764.10644531,359.99765076)
\lineto(763.86644531,359.93765076)
\curveto(763.79644187,359.92764406)(763.72644194,359.91264408)(763.65644531,359.89265076)
\curveto(763.3664423,359.77264422)(763.13144254,359.62264437)(762.95144531,359.44265076)
\curveto(762.78144289,359.26264473)(762.62644304,359.03764495)(762.48644531,358.76765076)
\curveto(762.45644321,358.71764527)(762.42644324,358.65264534)(762.39644531,358.57265076)
\curveto(762.3664433,358.50264549)(762.34144333,358.42264557)(762.32144531,358.33265076)
\curveto(762.30144337,358.24264575)(762.29644337,358.15764583)(762.30644531,358.07765076)
\curveto(762.31644335,357.99764599)(762.35144332,357.93764605)(762.41144531,357.89765076)
\curveto(762.49144318,357.83764615)(762.62644304,357.80764618)(762.81644531,357.80765076)
\curveto(763.01644265,357.81764617)(763.18644248,357.82264617)(763.32644531,357.82265076)
\lineto(765.60644531,357.82265076)
\curveto(765.75643991,357.82264617)(765.93643973,357.81764617)(766.14644531,357.80765076)
\curveto(766.35643931,357.80764618)(766.49643917,357.84764614)(766.56644531,357.92765076)
}
}
{
\newrgbcolor{curcolor}{0 0 0}
\pscustom[linestyle=none,fillstyle=solid,fillcolor=curcolor]
{
\newpath
\moveto(771.75808594,361.15265076)
\curveto(772.47808187,361.16264283)(773.08308127,361.07764291)(773.57308594,360.89765076)
\curveto(774.06308029,360.72764326)(774.44307991,360.42264357)(774.71308594,359.98265076)
\curveto(774.78307957,359.87264412)(774.83807951,359.75764423)(774.87808594,359.63765076)
\curveto(774.91807943,359.52764446)(774.95807939,359.40264459)(774.99808594,359.26265076)
\curveto(775.01807933,359.1926448)(775.02307933,359.11764487)(775.01308594,359.03765076)
\curveto(775.00307935,358.96764502)(774.98807936,358.91264508)(774.96808594,358.87265076)
\curveto(774.9480794,358.85264514)(774.92307943,358.83264516)(774.89308594,358.81265076)
\curveto(774.86307949,358.80264519)(774.83807951,358.7876452)(774.81808594,358.76765076)
\curveto(774.76807958,358.74764524)(774.71807963,358.74264525)(774.66808594,358.75265076)
\curveto(774.61807973,358.76264523)(774.56807978,358.76264523)(774.51808594,358.75265076)
\curveto(774.43807991,358.73264526)(774.33308002,358.72764526)(774.20308594,358.73765076)
\curveto(774.07308028,358.75764523)(773.98308037,358.78264521)(773.93308594,358.81265076)
\curveto(773.8530805,358.86264513)(773.79808055,358.92764506)(773.76808594,359.00765076)
\curveto(773.7480806,359.09764489)(773.71308064,359.18264481)(773.66308594,359.26265076)
\curveto(773.57308078,359.42264457)(773.4480809,359.56764442)(773.28808594,359.69765076)
\curveto(773.17808117,359.77764421)(773.05808129,359.83764415)(772.92808594,359.87765076)
\curveto(772.79808155,359.91764407)(772.65808169,359.95764403)(772.50808594,359.99765076)
\curveto(772.45808189,360.01764397)(772.40808194,360.02264397)(772.35808594,360.01265076)
\curveto(772.30808204,360.01264398)(772.25808209,360.01764397)(772.20808594,360.02765076)
\curveto(772.1480822,360.04764394)(772.07308228,360.05764393)(771.98308594,360.05765076)
\curveto(771.89308246,360.05764393)(771.81808253,360.04764394)(771.75808594,360.02765076)
\lineto(771.66808594,360.02765076)
\lineto(771.51808594,359.99765076)
\curveto(771.46808288,359.99764399)(771.41808293,359.992644)(771.36808594,359.98265076)
\curveto(771.10808324,359.92264407)(770.89308346,359.83764415)(770.72308594,359.72765076)
\curveto(770.5530838,359.61764437)(770.43808391,359.43264456)(770.37808594,359.17265076)
\curveto(770.35808399,359.10264489)(770.353084,359.03264496)(770.36308594,358.96265076)
\curveto(770.38308397,358.8926451)(770.40308395,358.83264516)(770.42308594,358.78265076)
\curveto(770.48308387,358.63264536)(770.5530838,358.52264547)(770.63308594,358.45265076)
\curveto(770.72308363,358.3926456)(770.83308352,358.32264567)(770.96308594,358.24265076)
\curveto(771.12308323,358.14264585)(771.30308305,358.06764592)(771.50308594,358.01765076)
\curveto(771.70308265,357.97764601)(771.90308245,357.92764606)(772.10308594,357.86765076)
\curveto(772.23308212,357.82764616)(772.36308199,357.79764619)(772.49308594,357.77765076)
\curveto(772.62308173,357.75764623)(772.7530816,357.72764626)(772.88308594,357.68765076)
\curveto(773.09308126,357.62764636)(773.29808105,357.56764642)(773.49808594,357.50765076)
\curveto(773.69808065,357.45764653)(773.89808045,357.3926466)(774.09808594,357.31265076)
\lineto(774.24808594,357.25265076)
\curveto(774.29808005,357.23264676)(774.34808,357.20764678)(774.39808594,357.17765076)
\curveto(774.59807975,357.05764693)(774.77307958,356.92264707)(774.92308594,356.77265076)
\curveto(775.07307928,356.62264737)(775.19807915,356.43264756)(775.29808594,356.20265076)
\curveto(775.31807903,356.13264786)(775.33807901,356.03764795)(775.35808594,355.91765076)
\curveto(775.37807897,355.84764814)(775.38807896,355.77264822)(775.38808594,355.69265076)
\curveto(775.39807895,355.62264837)(775.40307895,355.54264845)(775.40308594,355.45265076)
\lineto(775.40308594,355.30265076)
\curveto(775.38307897,355.23264876)(775.37307898,355.16264883)(775.37308594,355.09265076)
\curveto(775.37307898,355.02264897)(775.36307899,354.95264904)(775.34308594,354.88265076)
\curveto(775.31307904,354.77264922)(775.27807907,354.66764932)(775.23808594,354.56765076)
\curveto(775.19807915,354.46764952)(775.1530792,354.37764961)(775.10308594,354.29765076)
\curveto(774.94307941,354.03764995)(774.73807961,353.82765016)(774.48808594,353.66765076)
\curveto(774.23808011,353.51765047)(773.95808039,353.3876506)(773.64808594,353.27765076)
\curveto(773.55808079,353.24765074)(773.46308089,353.22765076)(773.36308594,353.21765076)
\curveto(773.27308108,353.19765079)(773.18308117,353.17265082)(773.09308594,353.14265076)
\curveto(772.99308136,353.12265087)(772.89308146,353.11265088)(772.79308594,353.11265076)
\curveto(772.69308166,353.11265088)(772.59308176,353.10265089)(772.49308594,353.08265076)
\lineto(772.34308594,353.08265076)
\curveto(772.29308206,353.07265092)(772.22308213,353.06765092)(772.13308594,353.06765076)
\curveto(772.04308231,353.06765092)(771.97308238,353.07265092)(771.92308594,353.08265076)
\lineto(771.75808594,353.08265076)
\curveto(771.69808265,353.10265089)(771.63308272,353.11265088)(771.56308594,353.11265076)
\curveto(771.49308286,353.10265089)(771.43308292,353.10765088)(771.38308594,353.12765076)
\curveto(771.33308302,353.13765085)(771.26808308,353.14265085)(771.18808594,353.14265076)
\lineto(770.94808594,353.20265076)
\curveto(770.87808347,353.21265078)(770.80308355,353.23265076)(770.72308594,353.26265076)
\curveto(770.41308394,353.36265063)(770.14308421,353.4876505)(769.91308594,353.63765076)
\curveto(769.68308467,353.7876502)(769.48308487,353.98265001)(769.31308594,354.22265076)
\curveto(769.22308513,354.35264964)(769.1480852,354.4876495)(769.08808594,354.62765076)
\curveto(769.02808532,354.76764922)(768.97308538,354.92264907)(768.92308594,355.09265076)
\curveto(768.90308545,355.15264884)(768.89308546,355.22264877)(768.89308594,355.30265076)
\curveto(768.90308545,355.3926486)(768.91808543,355.46264853)(768.93808594,355.51265076)
\curveto(768.96808538,355.55264844)(769.01808533,355.5926484)(769.08808594,355.63265076)
\curveto(769.13808521,355.65264834)(769.20808514,355.66264833)(769.29808594,355.66265076)
\curveto(769.38808496,355.67264832)(769.47808487,355.67264832)(769.56808594,355.66265076)
\curveto(769.65808469,355.65264834)(769.74308461,355.63764835)(769.82308594,355.61765076)
\curveto(769.91308444,355.60764838)(769.97308438,355.5926484)(770.00308594,355.57265076)
\curveto(770.07308428,355.52264847)(770.11808423,355.44764854)(770.13808594,355.34765076)
\curveto(770.16808418,355.25764873)(770.20308415,355.17264882)(770.24308594,355.09265076)
\curveto(770.34308401,354.87264912)(770.47808387,354.70264929)(770.64808594,354.58265076)
\curveto(770.76808358,354.4926495)(770.90308345,354.42264957)(771.05308594,354.37265076)
\curveto(771.20308315,354.32264967)(771.36308299,354.27264972)(771.53308594,354.22265076)
\lineto(771.84808594,354.17765076)
\lineto(771.93808594,354.17765076)
\curveto(772.00808234,354.15764983)(772.09808225,354.14764984)(772.20808594,354.14765076)
\curveto(772.32808202,354.14764984)(772.42808192,354.15764983)(772.50808594,354.17765076)
\curveto(772.57808177,354.17764981)(772.63308172,354.18264981)(772.67308594,354.19265076)
\curveto(772.73308162,354.20264979)(772.79308156,354.20764978)(772.85308594,354.20765076)
\curveto(772.91308144,354.21764977)(772.96808138,354.22764976)(773.01808594,354.23765076)
\curveto(773.30808104,354.31764967)(773.53808081,354.42264957)(773.70808594,354.55265076)
\curveto(773.87808047,354.68264931)(773.99808035,354.90264909)(774.06808594,355.21265076)
\curveto(774.08808026,355.26264873)(774.09308026,355.31764867)(774.08308594,355.37765076)
\curveto(774.07308028,355.43764855)(774.06308029,355.48264851)(774.05308594,355.51265076)
\curveto(774.00308035,355.70264829)(773.93308042,355.84264815)(773.84308594,355.93265076)
\curveto(773.7530806,356.03264796)(773.63808071,356.12264787)(773.49808594,356.20265076)
\curveto(773.40808094,356.26264773)(773.30808104,356.31264768)(773.19808594,356.35265076)
\lineto(772.86808594,356.47265076)
\curveto(772.83808151,356.48264751)(772.80808154,356.4876475)(772.77808594,356.48765076)
\curveto(772.75808159,356.4876475)(772.73308162,356.49764749)(772.70308594,356.51765076)
\curveto(772.36308199,356.62764736)(772.00808234,356.70764728)(771.63808594,356.75765076)
\curveto(771.27808307,356.81764717)(770.93808341,356.91264708)(770.61808594,357.04265076)
\curveto(770.51808383,357.08264691)(770.42308393,357.11764687)(770.33308594,357.14765076)
\curveto(770.24308411,357.17764681)(770.15808419,357.21764677)(770.07808594,357.26765076)
\curveto(769.88808446,357.37764661)(769.71308464,357.50264649)(769.55308594,357.64265076)
\curveto(769.39308496,357.78264621)(769.26808508,357.95764603)(769.17808594,358.16765076)
\curveto(769.1480852,358.23764575)(769.12308523,358.30764568)(769.10308594,358.37765076)
\curveto(769.09308526,358.44764554)(769.07808527,358.52264547)(769.05808594,358.60265076)
\curveto(769.02808532,358.72264527)(769.01808533,358.85764513)(769.02808594,359.00765076)
\curveto(769.03808531,359.16764482)(769.0530853,359.30264469)(769.07308594,359.41265076)
\curveto(769.09308526,359.46264453)(769.10308525,359.50264449)(769.10308594,359.53265076)
\curveto(769.11308524,359.57264442)(769.12808522,359.61264438)(769.14808594,359.65265076)
\curveto(769.23808511,359.88264411)(769.35808499,360.08264391)(769.50808594,360.25265076)
\curveto(769.66808468,360.42264357)(769.8480845,360.57264342)(770.04808594,360.70265076)
\curveto(770.19808415,360.7926432)(770.36308399,360.86264313)(770.54308594,360.91265076)
\curveto(770.72308363,360.97264302)(770.91308344,361.02764296)(771.11308594,361.07765076)
\curveto(771.18308317,361.0876429)(771.2480831,361.09764289)(771.30808594,361.10765076)
\curveto(771.37808297,361.11764287)(771.4530829,361.12764286)(771.53308594,361.13765076)
\curveto(771.56308279,361.14764284)(771.60308275,361.14764284)(771.65308594,361.13765076)
\curveto(771.70308265,361.12764286)(771.73808261,361.13264286)(771.75808594,361.15265076)
}
}
{
\newrgbcolor{curcolor}{0 0 0}
\pscustom[linestyle=none,fillstyle=solid,fillcolor=curcolor]
{
\newpath
\moveto(784.25308594,357.31265076)
\curveto(784.26307759,357.26264673)(784.26807758,357.19764679)(784.26808594,357.11765076)
\curveto(784.26807758,357.03764695)(784.26307759,356.97264702)(784.25308594,356.92265076)
\curveto(784.23307762,356.87264712)(784.22807762,356.82264717)(784.23808594,356.77265076)
\curveto(784.2480776,356.73264726)(784.2480776,356.6926473)(784.23808594,356.65265076)
\curveto(784.23807761,356.58264741)(784.23307762,356.52764746)(784.22308594,356.48765076)
\curveto(784.20307765,356.39764759)(784.18807766,356.30764768)(784.17808594,356.21765076)
\curveto(784.17807767,356.12764786)(784.16807768,356.03764795)(784.14808594,355.94765076)
\lineto(784.08808594,355.70765076)
\curveto(784.06807778,355.63764835)(784.04307781,355.56264843)(784.01308594,355.48265076)
\curveto(783.89307796,355.11264888)(783.72807812,354.77764921)(783.51808594,354.47765076)
\curveto(783.45807839,354.3876496)(783.39307846,354.29764969)(783.32308594,354.20765076)
\curveto(783.2530786,354.12764986)(783.17807867,354.05264994)(783.09808594,353.98265076)
\lineto(783.02308594,353.90765076)
\curveto(782.9530789,353.85765013)(782.88807896,353.80765018)(782.82808594,353.75765076)
\curveto(782.76807908,353.70765028)(782.69807915,353.65765033)(782.61808594,353.60765076)
\curveto(782.50807934,353.52765046)(782.38307947,353.45765053)(782.24308594,353.39765076)
\curveto(782.11307974,353.34765064)(781.97807987,353.29765069)(781.83808594,353.24765076)
\curveto(781.75808009,353.22765076)(781.67808017,353.21265078)(781.59808594,353.20265076)
\curveto(781.52808032,353.1926508)(781.4530804,353.17765081)(781.37308594,353.15765076)
\lineto(781.31308594,353.15765076)
\curveto(781.30308055,353.14765084)(781.28808056,353.14265085)(781.26808594,353.14265076)
\curveto(781.17808067,353.12265087)(781.04308081,353.11265088)(780.86308594,353.11265076)
\curveto(780.69308116,353.10265089)(780.55808129,353.10765088)(780.45808594,353.12765076)
\lineto(780.38308594,353.12765076)
\curveto(780.31308154,353.13765085)(780.2480816,353.14765084)(780.18808594,353.15765076)
\curveto(780.12808172,353.15765083)(780.06808178,353.16765082)(780.00808594,353.18765076)
\curveto(779.83808201,353.23765075)(779.67808217,353.28265071)(779.52808594,353.32265076)
\curveto(779.37808247,353.36265063)(779.23808261,353.42265057)(779.10808594,353.50265076)
\curveto(778.9480829,353.5926504)(778.80808304,353.6876503)(778.68808594,353.78765076)
\curveto(778.6480832,353.81765017)(778.58808326,353.85765013)(778.50808594,353.90765076)
\curveto(778.42808342,353.96765002)(778.3530835,353.97265002)(778.28308594,353.92265076)
\curveto(778.24308361,353.8926501)(778.22308363,353.85265014)(778.22308594,353.80265076)
\curveto(778.22308363,353.75265024)(778.21308364,353.69765029)(778.19308594,353.63765076)
\curveto(778.18308367,353.60765038)(778.18308367,353.57265042)(778.19308594,353.53265076)
\curveto(778.20308365,353.50265049)(778.20308365,353.46765052)(778.19308594,353.42765076)
\curveto(778.17308368,353.36765062)(778.16308369,353.30265069)(778.16308594,353.23265076)
\curveto(778.17308368,353.15265084)(778.17808367,353.08265091)(778.17808594,353.02265076)
\lineto(778.17808594,351.22265076)
\lineto(778.17808594,350.78765076)
\curveto(778.17808367,350.63765335)(778.1480837,350.52265347)(778.08808594,350.44265076)
\curveto(778.03808381,350.37265362)(777.95808389,350.33765365)(777.84808594,350.33765076)
\curveto(777.73808411,350.32765366)(777.62808422,350.32265367)(777.51808594,350.32265076)
\lineto(777.27808594,350.32265076)
\curveto(777.20808464,350.34265365)(777.1480847,350.36265363)(777.09808594,350.38265076)
\curveto(777.05808479,350.40265359)(777.02308483,350.43765355)(776.99308594,350.48765076)
\curveto(776.94308491,350.55765343)(776.91808493,350.66765332)(776.91808594,350.81765076)
\curveto(776.92808492,350.96765302)(776.93308492,351.09765289)(776.93308594,351.20765076)
\lineto(776.93308594,360.20765076)
\lineto(776.93308594,360.56765076)
\curveto(776.94308491,360.69764329)(776.97308488,360.80264319)(777.02308594,360.88265076)
\curveto(777.0530848,360.92264307)(777.11808473,360.95264304)(777.21808594,360.97265076)
\curveto(777.32808452,361.00264299)(777.44308441,361.01264298)(777.56308594,361.00265076)
\curveto(777.68308417,361.00264299)(777.79308406,360.987643)(777.89308594,360.95765076)
\curveto(778.00308385,360.93764305)(778.07308378,360.90764308)(778.10308594,360.86765076)
\curveto(778.14308371,360.81764317)(778.16308369,360.75764323)(778.16308594,360.68765076)
\curveto(778.17308368,360.61764337)(778.19308366,360.54764344)(778.22308594,360.47765076)
\curveto(778.24308361,360.44764354)(778.25808359,360.42264357)(778.26808594,360.40265076)
\curveto(778.28808356,360.3926436)(778.30808354,360.37764361)(778.32808594,360.35765076)
\curveto(778.43808341,360.34764364)(778.52808332,360.38264361)(778.59808594,360.46265076)
\curveto(778.67808317,360.54264345)(778.7530831,360.60764338)(778.82308594,360.65765076)
\curveto(779.08308277,360.83764315)(779.39308246,360.97764301)(779.75308594,361.07765076)
\curveto(779.84308201,361.09764289)(779.93308192,361.11264288)(780.02308594,361.12265076)
\curveto(780.12308173,361.13264286)(780.22308163,361.14764284)(780.32308594,361.16765076)
\curveto(780.36308149,361.17764281)(780.41308144,361.17764281)(780.47308594,361.16765076)
\curveto(780.53308132,361.15764283)(780.57308128,361.16264283)(780.59308594,361.18265076)
\curveto(781.02308083,361.1926428)(781.40308045,361.14764284)(781.73308594,361.04765076)
\curveto(782.06307979,360.95764303)(782.35807949,360.82764316)(782.61808594,360.65765076)
\lineto(782.76808594,360.53765076)
\curveto(782.81807903,360.50764348)(782.86807898,360.47264352)(782.91808594,360.43265076)
\curveto(782.93807891,360.41264358)(782.9530789,360.3926436)(782.96308594,360.37265076)
\curveto(782.98307887,360.36264363)(783.00307885,360.34764364)(783.02308594,360.32765076)
\curveto(783.07307878,360.27764371)(783.12807872,360.22264377)(783.18808594,360.16265076)
\curveto(783.2480786,360.10264389)(783.30307855,360.04264395)(783.35308594,359.98265076)
\curveto(783.47307838,359.81264418)(783.59807825,359.62764436)(783.72808594,359.42765076)
\curveto(783.80807804,359.29764469)(783.87307798,359.15264484)(783.92308594,358.99265076)
\curveto(783.98307787,358.83264516)(784.03807781,358.67264532)(784.08808594,358.51265076)
\curveto(784.10807774,358.43264556)(784.12307773,358.34764564)(784.13308594,358.25765076)
\curveto(784.1530777,358.16764582)(784.17307768,358.08264591)(784.19308594,358.00265076)
\lineto(784.19308594,357.88265076)
\curveto(784.20307765,357.85264614)(784.20807764,357.82264617)(784.20808594,357.79265076)
\curveto(784.22807762,357.74264625)(784.23307762,357.6876463)(784.22308594,357.62765076)
\curveto(784.22307763,357.56764642)(784.23307762,357.51264648)(784.25308594,357.46265076)
\lineto(784.25308594,357.31265076)
\moveto(782.91808594,356.90765076)
\curveto(782.93807891,356.95764703)(782.94307891,357.01764697)(782.93308594,357.08765076)
\curveto(782.92307893,357.16764682)(782.91807893,357.23764675)(782.91808594,357.29765076)
\curveto(782.91807893,357.46764652)(782.90807894,357.62764636)(782.88808594,357.77765076)
\curveto(782.87807897,357.92764606)(782.848079,358.07264592)(782.79808594,358.21265076)
\lineto(782.73808594,358.39265076)
\curveto(782.72807912,358.46264553)(782.70807914,358.52764546)(782.67808594,358.58765076)
\curveto(782.56807928,358.85764513)(782.39307946,359.11764487)(782.15308594,359.36765076)
\curveto(781.92307993,359.61764437)(781.70308015,359.7876442)(781.49308594,359.87765076)
\curveto(781.41308044,359.91764407)(781.32808052,359.94764404)(781.23808594,359.96765076)
\curveto(781.15808069,359.987644)(781.07308078,360.01264398)(780.98308594,360.04265076)
\curveto(780.89308096,360.06264393)(780.78808106,360.07264392)(780.66808594,360.07265076)
\lineto(780.33808594,360.07265076)
\curveto(780.31808153,360.05264394)(780.27808157,360.04264395)(780.21808594,360.04265076)
\curveto(780.16808168,360.05264394)(780.12308173,360.05264394)(780.08308594,360.04265076)
\lineto(779.81308594,359.98265076)
\curveto(779.73308212,359.96264403)(779.6530822,359.93264406)(779.57308594,359.89265076)
\curveto(779.2530826,359.75264424)(778.98808286,359.54764444)(778.77808594,359.27765076)
\curveto(778.57808327,359.01764497)(778.42308343,358.71264528)(778.31308594,358.36265076)
\curveto(778.27308358,358.25264574)(778.24308361,358.14264585)(778.22308594,358.03265076)
\curveto(778.21308364,357.92264607)(778.19808365,357.81264618)(778.17808594,357.70265076)
\curveto(778.16808368,357.66264633)(778.16308369,357.62264637)(778.16308594,357.58265076)
\curveto(778.16308369,357.55264644)(778.15808369,357.51764647)(778.14808594,357.47765076)
\lineto(778.14808594,357.35765076)
\curveto(778.13808371,357.30764668)(778.13308372,357.23264676)(778.13308594,357.13265076)
\curveto(778.13308372,357.04264695)(778.13808371,356.97264702)(778.14808594,356.92265076)
\lineto(778.14808594,356.80265076)
\curveto(778.15808369,356.76264723)(778.16308369,356.72264727)(778.16308594,356.68265076)
\curveto(778.16308369,356.64264735)(778.16808368,356.60764738)(778.17808594,356.57765076)
\curveto(778.18808366,356.54764744)(778.19308366,356.51764747)(778.19308594,356.48765076)
\curveto(778.19308366,356.45764753)(778.19808365,356.42264757)(778.20808594,356.38265076)
\curveto(778.22808362,356.30264769)(778.24308361,356.22264777)(778.25308594,356.14265076)
\lineto(778.31308594,355.90265076)
\curveto(778.42308343,355.56264843)(778.57308328,355.26264873)(778.76308594,355.00265076)
\curveto(778.96308289,354.75264924)(779.22308263,354.55764943)(779.54308594,354.41765076)
\curveto(779.73308212,354.33764965)(779.92808192,354.27764971)(780.12808594,354.23765076)
\curveto(780.16808168,354.21764977)(780.20808164,354.20764978)(780.24808594,354.20765076)
\curveto(780.28808156,354.21764977)(780.32808152,354.21764977)(780.36808594,354.20765076)
\lineto(780.48808594,354.20765076)
\curveto(780.55808129,354.1876498)(780.62808122,354.1876498)(780.69808594,354.20765076)
\lineto(780.81808594,354.20765076)
\curveto(780.92808092,354.22764976)(781.03308082,354.24264975)(781.13308594,354.25265076)
\curveto(781.23308062,354.26264973)(781.33308052,354.2876497)(781.43308594,354.32765076)
\curveto(781.74308011,354.45764953)(781.99307986,354.62764936)(782.18308594,354.83765076)
\curveto(782.38307947,355.05764893)(782.5480793,355.32264867)(782.67808594,355.63265076)
\curveto(782.72807912,355.77264822)(782.76307909,355.91264808)(782.78308594,356.05265076)
\curveto(782.81307904,356.20264779)(782.848079,356.35764763)(782.88808594,356.51765076)
\curveto(782.89807895,356.56764742)(782.90307895,356.61264738)(782.90308594,356.65265076)
\curveto(782.90307895,356.6926473)(782.90807894,356.73764725)(782.91808594,356.78765076)
\lineto(782.91808594,356.90765076)
}
}
{
\newrgbcolor{curcolor}{0 0 0}
\pscustom[linestyle=none,fillstyle=solid,fillcolor=curcolor]
{
\newpath
\moveto(792.61933594,353.80265076)
\curveto(792.64932811,353.64265035)(792.63432812,353.50765048)(792.57433594,353.39765076)
\curveto(792.51432824,353.29765069)(792.43432832,353.22265077)(792.33433594,353.17265076)
\curveto(792.28432847,353.15265084)(792.22932853,353.14265085)(792.16933594,353.14265076)
\curveto(792.11932864,353.14265085)(792.06432869,353.13265086)(792.00433594,353.11265076)
\curveto(791.78432897,353.06265093)(791.56432919,353.07765091)(791.34433594,353.15765076)
\curveto(791.13432962,353.22765076)(790.98932977,353.31765067)(790.90933594,353.42765076)
\curveto(790.8593299,353.49765049)(790.81432994,353.57765041)(790.77433594,353.66765076)
\curveto(790.73433002,353.76765022)(790.68433007,353.84765014)(790.62433594,353.90765076)
\curveto(790.60433015,353.92765006)(790.57933018,353.94765004)(790.54933594,353.96765076)
\curveto(790.52933023,353.98765)(790.49933026,353.99265)(790.45933594,353.98265076)
\curveto(790.34933041,353.95265004)(790.24433051,353.89765009)(790.14433594,353.81765076)
\curveto(790.0543307,353.73765025)(789.96433079,353.66765032)(789.87433594,353.60765076)
\curveto(789.74433101,353.52765046)(789.60433115,353.45265054)(789.45433594,353.38265076)
\curveto(789.30433145,353.32265067)(789.14433161,353.26765072)(788.97433594,353.21765076)
\curveto(788.87433188,353.1876508)(788.76433199,353.16765082)(788.64433594,353.15765076)
\curveto(788.53433222,353.14765084)(788.42433233,353.13265086)(788.31433594,353.11265076)
\curveto(788.26433249,353.10265089)(788.21933254,353.09765089)(788.17933594,353.09765076)
\lineto(788.07433594,353.09765076)
\curveto(787.96433279,353.07765091)(787.8593329,353.07765091)(787.75933594,353.09765076)
\lineto(787.62433594,353.09765076)
\curveto(787.57433318,353.10765088)(787.52433323,353.11265088)(787.47433594,353.11265076)
\curveto(787.42433333,353.11265088)(787.37933338,353.12265087)(787.33933594,353.14265076)
\curveto(787.29933346,353.15265084)(787.26433349,353.15765083)(787.23433594,353.15765076)
\curveto(787.21433354,353.14765084)(787.18933357,353.14765084)(787.15933594,353.15765076)
\lineto(786.91933594,353.21765076)
\curveto(786.83933392,353.22765076)(786.76433399,353.24765074)(786.69433594,353.27765076)
\curveto(786.39433436,353.40765058)(786.14933461,353.55265044)(785.95933594,353.71265076)
\curveto(785.77933498,353.88265011)(785.62933513,354.11764987)(785.50933594,354.41765076)
\curveto(785.41933534,354.63764935)(785.37433538,354.90264909)(785.37433594,355.21265076)
\lineto(785.37433594,355.52765076)
\curveto(785.38433537,355.57764841)(785.38933537,355.62764836)(785.38933594,355.67765076)
\lineto(785.41933594,355.85765076)
\lineto(785.53933594,356.18765076)
\curveto(785.57933518,356.29764769)(785.62933513,356.39764759)(785.68933594,356.48765076)
\curveto(785.86933489,356.77764721)(786.11433464,356.992647)(786.42433594,357.13265076)
\curveto(786.73433402,357.27264672)(787.07433368,357.39764659)(787.44433594,357.50765076)
\curveto(787.58433317,357.54764644)(787.72933303,357.57764641)(787.87933594,357.59765076)
\curveto(788.02933273,357.61764637)(788.17933258,357.64264635)(788.32933594,357.67265076)
\curveto(788.39933236,357.6926463)(788.46433229,357.70264629)(788.52433594,357.70265076)
\curveto(788.59433216,357.70264629)(788.66933209,357.71264628)(788.74933594,357.73265076)
\curveto(788.81933194,357.75264624)(788.88933187,357.76264623)(788.95933594,357.76265076)
\curveto(789.02933173,357.77264622)(789.10433165,357.7876462)(789.18433594,357.80765076)
\curveto(789.43433132,357.86764612)(789.66933109,357.91764607)(789.88933594,357.95765076)
\curveto(790.10933065,358.00764598)(790.28433047,358.12264587)(790.41433594,358.30265076)
\curveto(790.47433028,358.38264561)(790.52433023,358.48264551)(790.56433594,358.60265076)
\curveto(790.60433015,358.73264526)(790.60433015,358.87264512)(790.56433594,359.02265076)
\curveto(790.50433025,359.26264473)(790.41433034,359.45264454)(790.29433594,359.59265076)
\curveto(790.18433057,359.73264426)(790.02433073,359.84264415)(789.81433594,359.92265076)
\curveto(789.69433106,359.97264402)(789.54933121,360.00764398)(789.37933594,360.02765076)
\curveto(789.21933154,360.04764394)(789.04933171,360.05764393)(788.86933594,360.05765076)
\curveto(788.68933207,360.05764393)(788.51433224,360.04764394)(788.34433594,360.02765076)
\curveto(788.17433258,360.00764398)(788.02933273,359.97764401)(787.90933594,359.93765076)
\curveto(787.73933302,359.87764411)(787.57433318,359.7926442)(787.41433594,359.68265076)
\curveto(787.33433342,359.62264437)(787.2593335,359.54264445)(787.18933594,359.44265076)
\curveto(787.12933363,359.35264464)(787.07433368,359.25264474)(787.02433594,359.14265076)
\curveto(786.99433376,359.06264493)(786.96433379,358.97764501)(786.93433594,358.88765076)
\curveto(786.91433384,358.79764519)(786.86933389,358.72764526)(786.79933594,358.67765076)
\curveto(786.759334,358.64764534)(786.68933407,358.62264537)(786.58933594,358.60265076)
\curveto(786.49933426,358.5926454)(786.40433435,358.5876454)(786.30433594,358.58765076)
\curveto(786.20433455,358.5876454)(786.10433465,358.5926454)(786.00433594,358.60265076)
\curveto(785.91433484,358.62264537)(785.84933491,358.64764534)(785.80933594,358.67765076)
\curveto(785.76933499,358.70764528)(785.73933502,358.75764523)(785.71933594,358.82765076)
\curveto(785.69933506,358.89764509)(785.69933506,358.97264502)(785.71933594,359.05265076)
\curveto(785.74933501,359.18264481)(785.77933498,359.30264469)(785.80933594,359.41265076)
\curveto(785.84933491,359.53264446)(785.89433486,359.64764434)(785.94433594,359.75765076)
\curveto(786.13433462,360.10764388)(786.37433438,360.37764361)(786.66433594,360.56765076)
\curveto(786.9543338,360.76764322)(787.31433344,360.92764306)(787.74433594,361.04765076)
\curveto(787.84433291,361.06764292)(787.94433281,361.08264291)(788.04433594,361.09265076)
\curveto(788.1543326,361.10264289)(788.26433249,361.11764287)(788.37433594,361.13765076)
\curveto(788.41433234,361.14764284)(788.47933228,361.14764284)(788.56933594,361.13765076)
\curveto(788.6593321,361.13764285)(788.71433204,361.14764284)(788.73433594,361.16765076)
\curveto(789.43433132,361.17764281)(790.04433071,361.09764289)(790.56433594,360.92765076)
\curveto(791.08432967,360.75764323)(791.44932931,360.43264356)(791.65933594,359.95265076)
\curveto(791.74932901,359.75264424)(791.79932896,359.51764447)(791.80933594,359.24765076)
\curveto(791.82932893,358.987645)(791.83932892,358.71264528)(791.83933594,358.42265076)
\lineto(791.83933594,355.10765076)
\curveto(791.83932892,354.96764902)(791.84432891,354.83264916)(791.85433594,354.70265076)
\curveto(791.86432889,354.57264942)(791.89432886,354.46764952)(791.94433594,354.38765076)
\curveto(791.99432876,354.31764967)(792.0593287,354.26764972)(792.13933594,354.23765076)
\curveto(792.22932853,354.19764979)(792.31432844,354.16764982)(792.39433594,354.14765076)
\curveto(792.47432828,354.13764985)(792.53432822,354.0926499)(792.57433594,354.01265076)
\curveto(792.59432816,353.98265001)(792.60432815,353.95265004)(792.60433594,353.92265076)
\curveto(792.60432815,353.8926501)(792.60932815,353.85265014)(792.61933594,353.80265076)
\moveto(790.47433594,355.46765076)
\curveto(790.53433022,355.60764838)(790.56433019,355.76764822)(790.56433594,355.94765076)
\curveto(790.57433018,356.13764785)(790.57933018,356.33264766)(790.57933594,356.53265076)
\curveto(790.57933018,356.64264735)(790.57433018,356.74264725)(790.56433594,356.83265076)
\curveto(790.5543302,356.92264707)(790.51433024,356.992647)(790.44433594,357.04265076)
\curveto(790.41433034,357.06264693)(790.34433041,357.07264692)(790.23433594,357.07265076)
\curveto(790.21433054,357.05264694)(790.17933058,357.04264695)(790.12933594,357.04265076)
\curveto(790.07933068,357.04264695)(790.03433072,357.03264696)(789.99433594,357.01265076)
\curveto(789.91433084,356.992647)(789.82433093,356.97264702)(789.72433594,356.95265076)
\lineto(789.42433594,356.89265076)
\curveto(789.39433136,356.8926471)(789.3593314,356.8876471)(789.31933594,356.87765076)
\lineto(789.21433594,356.87765076)
\curveto(789.06433169,356.83764715)(788.89933186,356.81264718)(788.71933594,356.80265076)
\curveto(788.54933221,356.80264719)(788.38933237,356.78264721)(788.23933594,356.74265076)
\curveto(788.1593326,356.72264727)(788.08433267,356.70264729)(788.01433594,356.68265076)
\curveto(787.9543328,356.67264732)(787.88433287,356.65764733)(787.80433594,356.63765076)
\curveto(787.64433311,356.5876474)(787.49433326,356.52264747)(787.35433594,356.44265076)
\curveto(787.21433354,356.37264762)(787.09433366,356.28264771)(786.99433594,356.17265076)
\curveto(786.89433386,356.06264793)(786.81933394,355.92764806)(786.76933594,355.76765076)
\curveto(786.71933404,355.61764837)(786.69933406,355.43264856)(786.70933594,355.21265076)
\curveto(786.70933405,355.11264888)(786.72433403,355.01764897)(786.75433594,354.92765076)
\curveto(786.79433396,354.84764914)(786.83933392,354.77264922)(786.88933594,354.70265076)
\curveto(786.96933379,354.5926494)(787.07433368,354.49764949)(787.20433594,354.41765076)
\curveto(787.33433342,354.34764964)(787.47433328,354.2876497)(787.62433594,354.23765076)
\curveto(787.67433308,354.22764976)(787.72433303,354.22264977)(787.77433594,354.22265076)
\curveto(787.82433293,354.22264977)(787.87433288,354.21764977)(787.92433594,354.20765076)
\curveto(787.99433276,354.1876498)(788.07933268,354.17264982)(788.17933594,354.16265076)
\curveto(788.28933247,354.16264983)(788.37933238,354.17264982)(788.44933594,354.19265076)
\curveto(788.50933225,354.21264978)(788.56933219,354.21764977)(788.62933594,354.20765076)
\curveto(788.68933207,354.20764978)(788.74933201,354.21764977)(788.80933594,354.23765076)
\curveto(788.88933187,354.25764973)(788.96433179,354.27264972)(789.03433594,354.28265076)
\curveto(789.11433164,354.2926497)(789.18933157,354.31264968)(789.25933594,354.34265076)
\curveto(789.54933121,354.46264953)(789.79433096,354.60764938)(789.99433594,354.77765076)
\curveto(790.20433055,354.94764904)(790.36433039,355.17764881)(790.47433594,355.46765076)
}
}
{
\newrgbcolor{curcolor}{0 0 0}
\pscustom[linestyle=none,fillstyle=solid,fillcolor=curcolor]
{
\newpath
\moveto(796.92597656,361.15265076)
\curveto(797.66597177,361.16264283)(798.28097116,361.05264294)(798.77097656,360.82265076)
\curveto(799.27097017,360.60264339)(799.66596977,360.26764372)(799.95597656,359.81765076)
\curveto(800.08596935,359.61764437)(800.19596924,359.37264462)(800.28597656,359.08265076)
\curveto(800.30596913,359.03264496)(800.32096912,358.96764502)(800.33097656,358.88765076)
\curveto(800.3409691,358.80764518)(800.3359691,358.73764525)(800.31597656,358.67765076)
\curveto(800.28596915,358.62764536)(800.2359692,358.58264541)(800.16597656,358.54265076)
\curveto(800.1359693,358.52264547)(800.10596933,358.51264548)(800.07597656,358.51265076)
\curveto(800.04596939,358.52264547)(800.01096943,358.52264547)(799.97097656,358.51265076)
\curveto(799.93096951,358.50264549)(799.89096955,358.49764549)(799.85097656,358.49765076)
\curveto(799.81096963,358.50764548)(799.77096967,358.51264548)(799.73097656,358.51265076)
\lineto(799.41597656,358.51265076)
\curveto(799.31597012,358.52264547)(799.23097021,358.55264544)(799.16097656,358.60265076)
\curveto(799.08097036,358.66264533)(799.02597041,358.74764524)(798.99597656,358.85765076)
\curveto(798.96597047,358.96764502)(798.92597051,359.06264493)(798.87597656,359.14265076)
\curveto(798.72597071,359.40264459)(798.53097091,359.60764438)(798.29097656,359.75765076)
\curveto(798.21097123,359.80764418)(798.12597131,359.84764414)(798.03597656,359.87765076)
\curveto(797.94597149,359.91764407)(797.85097159,359.95264404)(797.75097656,359.98265076)
\curveto(797.61097183,360.02264397)(797.42597201,360.04264395)(797.19597656,360.04265076)
\curveto(796.96597247,360.05264394)(796.77597266,360.03264396)(796.62597656,359.98265076)
\curveto(796.55597288,359.96264403)(796.49097295,359.94764404)(796.43097656,359.93765076)
\curveto(796.37097307,359.92764406)(796.30597313,359.91264408)(796.23597656,359.89265076)
\curveto(795.97597346,359.78264421)(795.74597369,359.63264436)(795.54597656,359.44265076)
\curveto(795.34597409,359.25264474)(795.19097425,359.02764496)(795.08097656,358.76765076)
\curveto(795.0409744,358.67764531)(795.00597443,358.58264541)(794.97597656,358.48265076)
\curveto(794.94597449,358.3926456)(794.91597452,358.2926457)(794.88597656,358.18265076)
\lineto(794.79597656,357.77765076)
\curveto(794.78597465,357.72764626)(794.78097466,357.67264632)(794.78097656,357.61265076)
\curveto(794.79097465,357.55264644)(794.78597465,357.49764649)(794.76597656,357.44765076)
\lineto(794.76597656,357.32765076)
\curveto(794.75597468,357.2876467)(794.74597469,357.22264677)(794.73597656,357.13265076)
\curveto(794.7359747,357.04264695)(794.74597469,356.97764701)(794.76597656,356.93765076)
\curveto(794.77597466,356.8876471)(794.77597466,356.83764715)(794.76597656,356.78765076)
\curveto(794.75597468,356.73764725)(794.75597468,356.6876473)(794.76597656,356.63765076)
\curveto(794.77597466,356.59764739)(794.78097466,356.52764746)(794.78097656,356.42765076)
\curveto(794.80097464,356.34764764)(794.81597462,356.26264773)(794.82597656,356.17265076)
\curveto(794.84597459,356.08264791)(794.86597457,355.99764799)(794.88597656,355.91765076)
\curveto(794.99597444,355.59764839)(795.12097432,355.31764867)(795.26097656,355.07765076)
\curveto(795.41097403,354.84764914)(795.61597382,354.64764934)(795.87597656,354.47765076)
\curveto(795.96597347,354.42764956)(796.05597338,354.38264961)(796.14597656,354.34265076)
\curveto(796.24597319,354.30264969)(796.35097309,354.26264973)(796.46097656,354.22265076)
\curveto(796.51097293,354.21264978)(796.55097289,354.20764978)(796.58097656,354.20765076)
\curveto(796.61097283,354.20764978)(796.65097279,354.20264979)(796.70097656,354.19265076)
\curveto(796.73097271,354.18264981)(796.78097266,354.17764981)(796.85097656,354.17765076)
\lineto(797.01597656,354.17765076)
\curveto(797.01597242,354.16764982)(797.0359724,354.16264983)(797.07597656,354.16265076)
\curveto(797.09597234,354.17264982)(797.12097232,354.17264982)(797.15097656,354.16265076)
\curveto(797.18097226,354.16264983)(797.21097223,354.16764982)(797.24097656,354.17765076)
\curveto(797.31097213,354.19764979)(797.37597206,354.20264979)(797.43597656,354.19265076)
\curveto(797.50597193,354.1926498)(797.57597186,354.20264979)(797.64597656,354.22265076)
\curveto(797.90597153,354.30264969)(798.13097131,354.40264959)(798.32097656,354.52265076)
\curveto(798.51097093,354.65264934)(798.67097077,354.81764917)(798.80097656,355.01765076)
\curveto(798.85097059,355.09764889)(798.89597054,355.18264881)(798.93597656,355.27265076)
\lineto(799.05597656,355.54265076)
\curveto(799.07597036,355.62264837)(799.09597034,355.69764829)(799.11597656,355.76765076)
\curveto(799.14597029,355.84764814)(799.19597024,355.91264808)(799.26597656,355.96265076)
\curveto(799.29597014,355.992648)(799.35597008,356.01264798)(799.44597656,356.02265076)
\curveto(799.5359699,356.04264795)(799.63096981,356.05264794)(799.73097656,356.05265076)
\curveto(799.8409696,356.06264793)(799.9409695,356.06264793)(800.03097656,356.05265076)
\curveto(800.13096931,356.04264795)(800.20096924,356.02264797)(800.24097656,355.99265076)
\curveto(800.30096914,355.95264804)(800.3359691,355.8926481)(800.34597656,355.81265076)
\curveto(800.36596907,355.73264826)(800.36596907,355.64764834)(800.34597656,355.55765076)
\curveto(800.29596914,355.40764858)(800.24596919,355.26264873)(800.19597656,355.12265076)
\curveto(800.15596928,354.992649)(800.10096934,354.86264913)(800.03097656,354.73265076)
\curveto(799.88096956,354.43264956)(799.69096975,354.16764982)(799.46097656,353.93765076)
\curveto(799.2409702,353.70765028)(798.97097047,353.52265047)(798.65097656,353.38265076)
\curveto(798.57097087,353.34265065)(798.48597095,353.30765068)(798.39597656,353.27765076)
\curveto(798.30597113,353.25765073)(798.21097123,353.23265076)(798.11097656,353.20265076)
\curveto(798.00097144,353.16265083)(797.89097155,353.14265085)(797.78097656,353.14265076)
\curveto(797.67097177,353.13265086)(797.56097188,353.11765087)(797.45097656,353.09765076)
\curveto(797.41097203,353.07765091)(797.37097207,353.07265092)(797.33097656,353.08265076)
\curveto(797.29097215,353.0926509)(797.25097219,353.0926509)(797.21097656,353.08265076)
\lineto(797.07597656,353.08265076)
\lineto(796.83597656,353.08265076)
\curveto(796.76597267,353.07265092)(796.70097274,353.07765091)(796.64097656,353.09765076)
\lineto(796.56597656,353.09765076)
\lineto(796.20597656,353.14265076)
\curveto(796.07597336,353.18265081)(795.95097349,353.21765077)(795.83097656,353.24765076)
\curveto(795.71097373,353.27765071)(795.59597384,353.31765067)(795.48597656,353.36765076)
\curveto(795.12597431,353.52765046)(794.82597461,353.71765027)(794.58597656,353.93765076)
\curveto(794.35597508,354.15764983)(794.1409753,354.42764956)(793.94097656,354.74765076)
\curveto(793.89097555,354.82764916)(793.84597559,354.91764907)(793.80597656,355.01765076)
\lineto(793.68597656,355.31765076)
\curveto(793.6359758,355.42764856)(793.60097584,355.54264845)(793.58097656,355.66265076)
\curveto(793.56097588,355.78264821)(793.5359759,355.90264809)(793.50597656,356.02265076)
\curveto(793.49597594,356.06264793)(793.49097595,356.10264789)(793.49097656,356.14265076)
\curveto(793.49097595,356.18264781)(793.48597595,356.22264777)(793.47597656,356.26265076)
\curveto(793.45597598,356.32264767)(793.44597599,356.3876476)(793.44597656,356.45765076)
\curveto(793.45597598,356.52764746)(793.45097599,356.5926474)(793.43097656,356.65265076)
\lineto(793.43097656,356.80265076)
\curveto(793.42097602,356.85264714)(793.41597602,356.92264707)(793.41597656,357.01265076)
\curveto(793.41597602,357.10264689)(793.42097602,357.17264682)(793.43097656,357.22265076)
\curveto(793.440976,357.27264672)(793.440976,357.31764667)(793.43097656,357.35765076)
\curveto(793.43097601,357.39764659)(793.435976,357.43764655)(793.44597656,357.47765076)
\curveto(793.46597597,357.54764644)(793.47097597,357.61764637)(793.46097656,357.68765076)
\curveto(793.46097598,357.75764623)(793.47097597,357.82264617)(793.49097656,357.88265076)
\curveto(793.53097591,358.05264594)(793.56597587,358.22264577)(793.59597656,358.39265076)
\curveto(793.62597581,358.56264543)(793.67097577,358.72264527)(793.73097656,358.87265076)
\curveto(793.9409755,359.3926446)(794.19597524,359.81264418)(794.49597656,360.13265076)
\curveto(794.79597464,360.45264354)(795.20597423,360.71764327)(795.72597656,360.92765076)
\curveto(795.8359736,360.97764301)(795.95597348,361.01264298)(796.08597656,361.03265076)
\curveto(796.21597322,361.05264294)(796.35097309,361.07764291)(796.49097656,361.10765076)
\curveto(796.56097288,361.11764287)(796.63097281,361.12264287)(796.70097656,361.12265076)
\curveto(796.77097267,361.13264286)(796.84597259,361.14264285)(796.92597656,361.15265076)
}
}
{
\newrgbcolor{curcolor}{0 0 0}
\pscustom[linestyle=none,fillstyle=solid,fillcolor=curcolor]
{
\newpath
\moveto(802.13261719,362.47265076)
\curveto(802.05261607,362.53264146)(802.00761611,362.63764135)(801.99761719,362.78765076)
\lineto(801.99761719,363.25265076)
\lineto(801.99761719,363.50765076)
\curveto(801.99761612,363.59764039)(802.01261611,363.67264032)(802.04261719,363.73265076)
\curveto(802.08261604,363.81264018)(802.16261596,363.87264012)(802.28261719,363.91265076)
\curveto(802.30261582,363.92264007)(802.3226158,363.92264007)(802.34261719,363.91265076)
\curveto(802.37261575,363.91264008)(802.39761572,363.91764007)(802.41761719,363.92765076)
\curveto(802.58761553,363.92764006)(802.74761537,363.92264007)(802.89761719,363.91265076)
\curveto(803.04761507,363.90264009)(803.14761497,363.84264015)(803.19761719,363.73265076)
\curveto(803.22761489,363.67264032)(803.24261488,363.59764039)(803.24261719,363.50765076)
\lineto(803.24261719,363.25265076)
\curveto(803.24261488,363.07264092)(803.23761488,362.90264109)(803.22761719,362.74265076)
\curveto(803.22761489,362.58264141)(803.16261496,362.47764151)(803.03261719,362.42765076)
\curveto(802.98261514,362.40764158)(802.92761519,362.39764159)(802.86761719,362.39765076)
\lineto(802.70261719,362.39765076)
\lineto(802.38761719,362.39765076)
\curveto(802.28761583,362.39764159)(802.20261592,362.42264157)(802.13261719,362.47265076)
\moveto(803.24261719,353.96765076)
\lineto(803.24261719,353.65265076)
\curveto(803.25261487,353.55265044)(803.23261489,353.47265052)(803.18261719,353.41265076)
\curveto(803.15261497,353.35265064)(803.10761501,353.31265068)(803.04761719,353.29265076)
\curveto(802.98761513,353.28265071)(802.9176152,353.26765072)(802.83761719,353.24765076)
\lineto(802.61261719,353.24765076)
\curveto(802.48261564,353.24765074)(802.36761575,353.25265074)(802.26761719,353.26265076)
\curveto(802.17761594,353.28265071)(802.10761601,353.33265066)(802.05761719,353.41265076)
\curveto(802.0176161,353.47265052)(801.99761612,353.54765044)(801.99761719,353.63765076)
\lineto(801.99761719,353.92265076)
\lineto(801.99761719,360.26765076)
\lineto(801.99761719,360.58265076)
\curveto(801.99761612,360.6926433)(802.0226161,360.77764321)(802.07261719,360.83765076)
\curveto(802.10261602,360.8876431)(802.14261598,360.91764307)(802.19261719,360.92765076)
\curveto(802.24261588,360.93764305)(802.29761582,360.95264304)(802.35761719,360.97265076)
\curveto(802.37761574,360.97264302)(802.39761572,360.96764302)(802.41761719,360.95765076)
\curveto(802.44761567,360.95764303)(802.47261565,360.96264303)(802.49261719,360.97265076)
\curveto(802.6226155,360.97264302)(802.75261537,360.96764302)(802.88261719,360.95765076)
\curveto(803.0226151,360.95764303)(803.117615,360.91764307)(803.16761719,360.83765076)
\curveto(803.2176149,360.77764321)(803.24261488,360.69764329)(803.24261719,360.59765076)
\lineto(803.24261719,360.31265076)
\lineto(803.24261719,353.96765076)
}
}
{
\newrgbcolor{curcolor}{0 0 0}
\pscustom[linestyle=none,fillstyle=solid,fillcolor=curcolor]
{
\newpath
\moveto(812.31246094,357.44765076)
\curveto(812.33245288,357.3876466)(812.34245287,357.2926467)(812.34246094,357.16265076)
\curveto(812.34245287,357.04264695)(812.33745287,356.95764703)(812.32746094,356.90765076)
\lineto(812.32746094,356.75765076)
\curveto(812.31745289,356.67764731)(812.3074529,356.60264739)(812.29746094,356.53265076)
\curveto(812.29745291,356.47264752)(812.29245292,356.40264759)(812.28246094,356.32265076)
\curveto(812.26245295,356.26264773)(812.24745296,356.20264779)(812.23746094,356.14265076)
\curveto(812.23745297,356.08264791)(812.22745298,356.02264797)(812.20746094,355.96265076)
\curveto(812.16745304,355.83264816)(812.13245308,355.70264829)(812.10246094,355.57265076)
\curveto(812.07245314,355.44264855)(812.03245318,355.32264867)(811.98246094,355.21265076)
\curveto(811.77245344,354.73264926)(811.49245372,354.32764966)(811.14246094,353.99765076)
\curveto(810.79245442,353.67765031)(810.36245485,353.43265056)(809.85246094,353.26265076)
\curveto(809.74245547,353.22265077)(809.62245559,353.1926508)(809.49246094,353.17265076)
\curveto(809.37245584,353.15265084)(809.24745596,353.13265086)(809.11746094,353.11265076)
\curveto(809.05745615,353.10265089)(808.99245622,353.09765089)(808.92246094,353.09765076)
\curveto(808.86245635,353.0876509)(808.80245641,353.08265091)(808.74246094,353.08265076)
\curveto(808.70245651,353.07265092)(808.64245657,353.06765092)(808.56246094,353.06765076)
\curveto(808.49245672,353.06765092)(808.44245677,353.07265092)(808.41246094,353.08265076)
\curveto(808.37245684,353.0926509)(808.33245688,353.09765089)(808.29246094,353.09765076)
\curveto(808.25245696,353.0876509)(808.21745699,353.0876509)(808.18746094,353.09765076)
\lineto(808.09746094,353.09765076)
\lineto(807.73746094,353.14265076)
\curveto(807.59745761,353.18265081)(807.46245775,353.22265077)(807.33246094,353.26265076)
\curveto(807.20245801,353.30265069)(807.07745813,353.34765064)(806.95746094,353.39765076)
\curveto(806.5074587,353.59765039)(806.13745907,353.85765013)(805.84746094,354.17765076)
\curveto(805.55745965,354.49764949)(805.31745989,354.8876491)(805.12746094,355.34765076)
\curveto(805.07746013,355.44764854)(805.03746017,355.54764844)(805.00746094,355.64765076)
\curveto(804.98746022,355.74764824)(804.96746024,355.85264814)(804.94746094,355.96265076)
\curveto(804.92746028,356.00264799)(804.91746029,356.03264796)(804.91746094,356.05265076)
\curveto(804.92746028,356.08264791)(804.92746028,356.11764787)(804.91746094,356.15765076)
\curveto(804.89746031,356.23764775)(804.88246033,356.31764767)(804.87246094,356.39765076)
\curveto(804.87246034,356.4876475)(804.86246035,356.57264742)(804.84246094,356.65265076)
\lineto(804.84246094,356.77265076)
\curveto(804.84246037,356.81264718)(804.83746037,356.85764713)(804.82746094,356.90765076)
\curveto(804.81746039,356.95764703)(804.8124604,357.04264695)(804.81246094,357.16265076)
\curveto(804.8124604,357.2926467)(804.82246039,357.3876466)(804.84246094,357.44765076)
\curveto(804.86246035,357.51764647)(804.86746034,357.5876464)(804.85746094,357.65765076)
\curveto(804.84746036,357.72764626)(804.85246036,357.79764619)(804.87246094,357.86765076)
\curveto(804.88246033,357.91764607)(804.88746032,357.95764603)(804.88746094,357.98765076)
\curveto(804.89746031,358.02764596)(804.9074603,358.07264592)(804.91746094,358.12265076)
\curveto(804.94746026,358.24264575)(804.97246024,358.36264563)(804.99246094,358.48265076)
\curveto(805.02246019,358.60264539)(805.06246015,358.71764527)(805.11246094,358.82765076)
\curveto(805.26245995,359.19764479)(805.44245977,359.52764446)(805.65246094,359.81765076)
\curveto(805.87245934,360.11764387)(806.13745907,360.36764362)(806.44746094,360.56765076)
\curveto(806.56745864,360.64764334)(806.69245852,360.71264328)(806.82246094,360.76265076)
\curveto(806.95245826,360.82264317)(807.08745812,360.88264311)(807.22746094,360.94265076)
\curveto(807.34745786,360.992643)(807.47745773,361.02264297)(807.61746094,361.03265076)
\curveto(807.75745745,361.05264294)(807.89745731,361.08264291)(808.03746094,361.12265076)
\lineto(808.23246094,361.12265076)
\curveto(808.30245691,361.13264286)(808.36745684,361.14264285)(808.42746094,361.15265076)
\curveto(809.31745589,361.16264283)(810.05745515,360.97764301)(810.64746094,360.59765076)
\curveto(811.23745397,360.21764377)(811.66245355,359.72264427)(811.92246094,359.11265076)
\curveto(811.97245324,359.01264498)(812.0124532,358.91264508)(812.04246094,358.81265076)
\curveto(812.07245314,358.71264528)(812.1074531,358.60764538)(812.14746094,358.49765076)
\curveto(812.17745303,358.3876456)(812.20245301,358.26764572)(812.22246094,358.13765076)
\curveto(812.24245297,358.01764597)(812.26745294,357.8926461)(812.29746094,357.76265076)
\curveto(812.3074529,357.71264628)(812.3074529,357.65764633)(812.29746094,357.59765076)
\curveto(812.29745291,357.54764644)(812.30245291,357.49764649)(812.31246094,357.44765076)
\moveto(810.97746094,356.59265076)
\curveto(810.99745421,356.66264733)(811.00245421,356.74264725)(810.99246094,356.83265076)
\lineto(810.99246094,357.08765076)
\curveto(810.99245422,357.47764651)(810.95745425,357.80764618)(810.88746094,358.07765076)
\curveto(810.85745435,358.15764583)(810.83245438,358.23764575)(810.81246094,358.31765076)
\curveto(810.79245442,358.39764559)(810.76745444,358.47264552)(810.73746094,358.54265076)
\curveto(810.45745475,359.1926448)(810.0124552,359.64264435)(809.40246094,359.89265076)
\curveto(809.33245588,359.92264407)(809.25745595,359.94264405)(809.17746094,359.95265076)
\lineto(808.93746094,360.01265076)
\curveto(808.85745635,360.03264396)(808.77245644,360.04264395)(808.68246094,360.04265076)
\lineto(808.41246094,360.04265076)
\lineto(808.14246094,359.99765076)
\curveto(808.04245717,359.97764401)(807.94745726,359.95264404)(807.85746094,359.92265076)
\curveto(807.77745743,359.90264409)(807.69745751,359.87264412)(807.61746094,359.83265076)
\curveto(807.54745766,359.81264418)(807.48245773,359.78264421)(807.42246094,359.74265076)
\curveto(807.36245785,359.70264429)(807.3074579,359.66264433)(807.25746094,359.62265076)
\curveto(807.01745819,359.45264454)(806.82245839,359.24764474)(806.67246094,359.00765076)
\curveto(806.52245869,358.76764522)(806.39245882,358.4876455)(806.28246094,358.16765076)
\curveto(806.25245896,358.06764592)(806.23245898,357.96264603)(806.22246094,357.85265076)
\curveto(806.212459,357.75264624)(806.19745901,357.64764634)(806.17746094,357.53765076)
\curveto(806.16745904,357.49764649)(806.16245905,357.43264656)(806.16246094,357.34265076)
\curveto(806.15245906,357.31264668)(806.14745906,357.27764671)(806.14746094,357.23765076)
\curveto(806.15745905,357.19764679)(806.16245905,357.15264684)(806.16246094,357.10265076)
\lineto(806.16246094,356.80265076)
\curveto(806.16245905,356.70264729)(806.17245904,356.61264738)(806.19246094,356.53265076)
\lineto(806.22246094,356.35265076)
\curveto(806.24245897,356.25264774)(806.25745895,356.15264784)(806.26746094,356.05265076)
\curveto(806.28745892,355.96264803)(806.31745889,355.87764811)(806.35746094,355.79765076)
\curveto(806.45745875,355.55764843)(806.57245864,355.33264866)(806.70246094,355.12265076)
\curveto(806.84245837,354.91264908)(807.0124582,354.73764925)(807.21246094,354.59765076)
\curveto(807.26245795,354.56764942)(807.3074579,354.54264945)(807.34746094,354.52265076)
\curveto(807.38745782,354.50264949)(807.43245778,354.47764951)(807.48246094,354.44765076)
\curveto(807.56245765,354.39764959)(807.64745756,354.35264964)(807.73746094,354.31265076)
\curveto(807.83745737,354.28264971)(807.94245727,354.25264974)(808.05246094,354.22265076)
\curveto(808.10245711,354.20264979)(808.14745706,354.1926498)(808.18746094,354.19265076)
\curveto(808.23745697,354.20264979)(808.28745692,354.20264979)(808.33746094,354.19265076)
\curveto(808.36745684,354.18264981)(808.42745678,354.17264982)(808.51746094,354.16265076)
\curveto(808.61745659,354.15264984)(808.69245652,354.15764983)(808.74246094,354.17765076)
\curveto(808.78245643,354.1876498)(808.82245639,354.1876498)(808.86246094,354.17765076)
\curveto(808.90245631,354.17764981)(808.94245627,354.1876498)(808.98246094,354.20765076)
\curveto(809.06245615,354.22764976)(809.14245607,354.24264975)(809.22246094,354.25265076)
\curveto(809.30245591,354.27264972)(809.37745583,354.29764969)(809.44746094,354.32765076)
\curveto(809.78745542,354.46764952)(810.06245515,354.66264933)(810.27246094,354.91265076)
\curveto(810.48245473,355.16264883)(810.65745455,355.45764853)(810.79746094,355.79765076)
\curveto(810.84745436,355.91764807)(810.87745433,356.04264795)(810.88746094,356.17265076)
\curveto(810.9074543,356.31264768)(810.93745427,356.45264754)(810.97746094,356.59265076)
}
}
{
\newrgbcolor{curcolor}{0 0 0}
\pscustom[linestyle=none,fillstyle=solid,fillcolor=curcolor]
{
\newpath
\moveto(816.23074219,361.15265076)
\curveto(816.95073812,361.16264283)(817.55573752,361.07764291)(818.04574219,360.89765076)
\curveto(818.53573654,360.72764326)(818.91573616,360.42264357)(819.18574219,359.98265076)
\curveto(819.25573582,359.87264412)(819.31073576,359.75764423)(819.35074219,359.63765076)
\curveto(819.39073568,359.52764446)(819.43073564,359.40264459)(819.47074219,359.26265076)
\curveto(819.49073558,359.1926448)(819.49573558,359.11764487)(819.48574219,359.03765076)
\curveto(819.4757356,358.96764502)(819.46073561,358.91264508)(819.44074219,358.87265076)
\curveto(819.42073565,358.85264514)(819.39573568,358.83264516)(819.36574219,358.81265076)
\curveto(819.33573574,358.80264519)(819.31073576,358.7876452)(819.29074219,358.76765076)
\curveto(819.24073583,358.74764524)(819.19073588,358.74264525)(819.14074219,358.75265076)
\curveto(819.09073598,358.76264523)(819.04073603,358.76264523)(818.99074219,358.75265076)
\curveto(818.91073616,358.73264526)(818.80573627,358.72764526)(818.67574219,358.73765076)
\curveto(818.54573653,358.75764523)(818.45573662,358.78264521)(818.40574219,358.81265076)
\curveto(818.32573675,358.86264513)(818.2707368,358.92764506)(818.24074219,359.00765076)
\curveto(818.22073685,359.09764489)(818.18573689,359.18264481)(818.13574219,359.26265076)
\curveto(818.04573703,359.42264457)(817.92073715,359.56764442)(817.76074219,359.69765076)
\curveto(817.65073742,359.77764421)(817.53073754,359.83764415)(817.40074219,359.87765076)
\curveto(817.2707378,359.91764407)(817.13073794,359.95764403)(816.98074219,359.99765076)
\curveto(816.93073814,360.01764397)(816.88073819,360.02264397)(816.83074219,360.01265076)
\curveto(816.78073829,360.01264398)(816.73073834,360.01764397)(816.68074219,360.02765076)
\curveto(816.62073845,360.04764394)(816.54573853,360.05764393)(816.45574219,360.05765076)
\curveto(816.36573871,360.05764393)(816.29073878,360.04764394)(816.23074219,360.02765076)
\lineto(816.14074219,360.02765076)
\lineto(815.99074219,359.99765076)
\curveto(815.94073913,359.99764399)(815.89073918,359.992644)(815.84074219,359.98265076)
\curveto(815.58073949,359.92264407)(815.36573971,359.83764415)(815.19574219,359.72765076)
\curveto(815.02574005,359.61764437)(814.91074016,359.43264456)(814.85074219,359.17265076)
\curveto(814.83074024,359.10264489)(814.82574025,359.03264496)(814.83574219,358.96265076)
\curveto(814.85574022,358.8926451)(814.8757402,358.83264516)(814.89574219,358.78265076)
\curveto(814.95574012,358.63264536)(815.02574005,358.52264547)(815.10574219,358.45265076)
\curveto(815.19573988,358.3926456)(815.30573977,358.32264567)(815.43574219,358.24265076)
\curveto(815.59573948,358.14264585)(815.7757393,358.06764592)(815.97574219,358.01765076)
\curveto(816.1757389,357.97764601)(816.3757387,357.92764606)(816.57574219,357.86765076)
\curveto(816.70573837,357.82764616)(816.83573824,357.79764619)(816.96574219,357.77765076)
\curveto(817.09573798,357.75764623)(817.22573785,357.72764626)(817.35574219,357.68765076)
\curveto(817.56573751,357.62764636)(817.7707373,357.56764642)(817.97074219,357.50765076)
\curveto(818.1707369,357.45764653)(818.3707367,357.3926466)(818.57074219,357.31265076)
\lineto(818.72074219,357.25265076)
\curveto(818.7707363,357.23264676)(818.82073625,357.20764678)(818.87074219,357.17765076)
\curveto(819.070736,357.05764693)(819.24573583,356.92264707)(819.39574219,356.77265076)
\curveto(819.54573553,356.62264737)(819.6707354,356.43264756)(819.77074219,356.20265076)
\curveto(819.79073528,356.13264786)(819.81073526,356.03764795)(819.83074219,355.91765076)
\curveto(819.85073522,355.84764814)(819.86073521,355.77264822)(819.86074219,355.69265076)
\curveto(819.8707352,355.62264837)(819.8757352,355.54264845)(819.87574219,355.45265076)
\lineto(819.87574219,355.30265076)
\curveto(819.85573522,355.23264876)(819.84573523,355.16264883)(819.84574219,355.09265076)
\curveto(819.84573523,355.02264897)(819.83573524,354.95264904)(819.81574219,354.88265076)
\curveto(819.78573529,354.77264922)(819.75073532,354.66764932)(819.71074219,354.56765076)
\curveto(819.6707354,354.46764952)(819.62573545,354.37764961)(819.57574219,354.29765076)
\curveto(819.41573566,354.03764995)(819.21073586,353.82765016)(818.96074219,353.66765076)
\curveto(818.71073636,353.51765047)(818.43073664,353.3876506)(818.12074219,353.27765076)
\curveto(818.03073704,353.24765074)(817.93573714,353.22765076)(817.83574219,353.21765076)
\curveto(817.74573733,353.19765079)(817.65573742,353.17265082)(817.56574219,353.14265076)
\curveto(817.46573761,353.12265087)(817.36573771,353.11265088)(817.26574219,353.11265076)
\curveto(817.16573791,353.11265088)(817.06573801,353.10265089)(816.96574219,353.08265076)
\lineto(816.81574219,353.08265076)
\curveto(816.76573831,353.07265092)(816.69573838,353.06765092)(816.60574219,353.06765076)
\curveto(816.51573856,353.06765092)(816.44573863,353.07265092)(816.39574219,353.08265076)
\lineto(816.23074219,353.08265076)
\curveto(816.1707389,353.10265089)(816.10573897,353.11265088)(816.03574219,353.11265076)
\curveto(815.96573911,353.10265089)(815.90573917,353.10765088)(815.85574219,353.12765076)
\curveto(815.80573927,353.13765085)(815.74073933,353.14265085)(815.66074219,353.14265076)
\lineto(815.42074219,353.20265076)
\curveto(815.35073972,353.21265078)(815.2757398,353.23265076)(815.19574219,353.26265076)
\curveto(814.88574019,353.36265063)(814.61574046,353.4876505)(814.38574219,353.63765076)
\curveto(814.15574092,353.7876502)(813.95574112,353.98265001)(813.78574219,354.22265076)
\curveto(813.69574138,354.35264964)(813.62074145,354.4876495)(813.56074219,354.62765076)
\curveto(813.50074157,354.76764922)(813.44574163,354.92264907)(813.39574219,355.09265076)
\curveto(813.3757417,355.15264884)(813.36574171,355.22264877)(813.36574219,355.30265076)
\curveto(813.3757417,355.3926486)(813.39074168,355.46264853)(813.41074219,355.51265076)
\curveto(813.44074163,355.55264844)(813.49074158,355.5926484)(813.56074219,355.63265076)
\curveto(813.61074146,355.65264834)(813.68074139,355.66264833)(813.77074219,355.66265076)
\curveto(813.86074121,355.67264832)(813.95074112,355.67264832)(814.04074219,355.66265076)
\curveto(814.13074094,355.65264834)(814.21574086,355.63764835)(814.29574219,355.61765076)
\curveto(814.38574069,355.60764838)(814.44574063,355.5926484)(814.47574219,355.57265076)
\curveto(814.54574053,355.52264847)(814.59074048,355.44764854)(814.61074219,355.34765076)
\curveto(814.64074043,355.25764873)(814.6757404,355.17264882)(814.71574219,355.09265076)
\curveto(814.81574026,354.87264912)(814.95074012,354.70264929)(815.12074219,354.58265076)
\curveto(815.24073983,354.4926495)(815.3757397,354.42264957)(815.52574219,354.37265076)
\curveto(815.6757394,354.32264967)(815.83573924,354.27264972)(816.00574219,354.22265076)
\lineto(816.32074219,354.17765076)
\lineto(816.41074219,354.17765076)
\curveto(816.48073859,354.15764983)(816.5707385,354.14764984)(816.68074219,354.14765076)
\curveto(816.80073827,354.14764984)(816.90073817,354.15764983)(816.98074219,354.17765076)
\curveto(817.05073802,354.17764981)(817.10573797,354.18264981)(817.14574219,354.19265076)
\curveto(817.20573787,354.20264979)(817.26573781,354.20764978)(817.32574219,354.20765076)
\curveto(817.38573769,354.21764977)(817.44073763,354.22764976)(817.49074219,354.23765076)
\curveto(817.78073729,354.31764967)(818.01073706,354.42264957)(818.18074219,354.55265076)
\curveto(818.35073672,354.68264931)(818.4707366,354.90264909)(818.54074219,355.21265076)
\curveto(818.56073651,355.26264873)(818.56573651,355.31764867)(818.55574219,355.37765076)
\curveto(818.54573653,355.43764855)(818.53573654,355.48264851)(818.52574219,355.51265076)
\curveto(818.4757366,355.70264829)(818.40573667,355.84264815)(818.31574219,355.93265076)
\curveto(818.22573685,356.03264796)(818.11073696,356.12264787)(817.97074219,356.20265076)
\curveto(817.88073719,356.26264773)(817.78073729,356.31264768)(817.67074219,356.35265076)
\lineto(817.34074219,356.47265076)
\curveto(817.31073776,356.48264751)(817.28073779,356.4876475)(817.25074219,356.48765076)
\curveto(817.23073784,356.4876475)(817.20573787,356.49764749)(817.17574219,356.51765076)
\curveto(816.83573824,356.62764736)(816.48073859,356.70764728)(816.11074219,356.75765076)
\curveto(815.75073932,356.81764717)(815.41073966,356.91264708)(815.09074219,357.04265076)
\curveto(814.99074008,357.08264691)(814.89574018,357.11764687)(814.80574219,357.14765076)
\curveto(814.71574036,357.17764681)(814.63074044,357.21764677)(814.55074219,357.26765076)
\curveto(814.36074071,357.37764661)(814.18574089,357.50264649)(814.02574219,357.64265076)
\curveto(813.86574121,357.78264621)(813.74074133,357.95764603)(813.65074219,358.16765076)
\curveto(813.62074145,358.23764575)(813.59574148,358.30764568)(813.57574219,358.37765076)
\curveto(813.56574151,358.44764554)(813.55074152,358.52264547)(813.53074219,358.60265076)
\curveto(813.50074157,358.72264527)(813.49074158,358.85764513)(813.50074219,359.00765076)
\curveto(813.51074156,359.16764482)(813.52574155,359.30264469)(813.54574219,359.41265076)
\curveto(813.56574151,359.46264453)(813.5757415,359.50264449)(813.57574219,359.53265076)
\curveto(813.58574149,359.57264442)(813.60074147,359.61264438)(813.62074219,359.65265076)
\curveto(813.71074136,359.88264411)(813.83074124,360.08264391)(813.98074219,360.25265076)
\curveto(814.14074093,360.42264357)(814.32074075,360.57264342)(814.52074219,360.70265076)
\curveto(814.6707404,360.7926432)(814.83574024,360.86264313)(815.01574219,360.91265076)
\curveto(815.19573988,360.97264302)(815.38573969,361.02764296)(815.58574219,361.07765076)
\curveto(815.65573942,361.0876429)(815.72073935,361.09764289)(815.78074219,361.10765076)
\curveto(815.85073922,361.11764287)(815.92573915,361.12764286)(816.00574219,361.13765076)
\curveto(816.03573904,361.14764284)(816.075739,361.14764284)(816.12574219,361.13765076)
\curveto(816.1757389,361.12764286)(816.21073886,361.13264286)(816.23074219,361.15265076)
}
}
{
\newrgbcolor{curcolor}{0 0 0}
\pscustom[linestyle=none,fillstyle=solid,fillcolor=curcolor]
{
\newpath
\moveto(770.46642334,330.80134644)
\curveto(770.51641372,330.67134618)(770.49641374,330.57134628)(770.40642334,330.50134644)
\curveto(770.35641388,330.47134638)(770.29141394,330.4513464)(770.21142334,330.44134644)
\lineto(769.98642334,330.44134644)
\lineto(769.50642334,330.44134644)
\curveto(769.34641489,330.44134641)(769.22141501,330.47634637)(769.13142334,330.54634644)
\curveto(769.05141518,330.59634625)(768.99641524,330.67134618)(768.96642334,330.77134644)
\lineto(768.90642334,331.10134644)
\curveto(768.89641534,331.14134571)(768.89141534,331.17634567)(768.89142334,331.20634644)
\lineto(768.89142334,331.31134644)
\curveto(768.87141536,331.36134549)(768.86641537,331.40634544)(768.87642334,331.44634644)
\curveto(768.88641535,331.48634536)(768.88641535,331.52634532)(768.87642334,331.56634644)
\curveto(768.86641537,331.62634522)(768.86141537,331.68634516)(768.86142334,331.74634644)
\lineto(768.86142334,331.92634644)
\lineto(768.81642334,332.60134644)
\curveto(768.79641544,332.67134418)(768.78641545,332.74134411)(768.78642334,332.81134644)
\curveto(768.78641545,332.88134397)(768.77641546,332.95634389)(768.75642334,333.03634644)
\curveto(768.70641553,333.21634363)(768.66641557,333.39634345)(768.63642334,333.57634644)
\curveto(768.61641562,333.75634309)(768.57141566,333.92634292)(768.50142334,334.08634644)
\curveto(768.31141592,334.50634234)(767.99641624,334.78634206)(767.55642334,334.92634644)
\curveto(767.42641681,334.97634187)(767.28141695,335.00134185)(767.12142334,335.00134644)
\curveto(766.97141726,335.01134184)(766.81141742,335.01634183)(766.64142334,335.01634644)
\lineto(763.88142334,335.01634644)
\curveto(763.81142042,334.99634185)(763.74642049,334.97634187)(763.68642334,334.95634644)
\curveto(763.6364206,334.9463419)(763.59142064,334.91634193)(763.55142334,334.86634644)
\curveto(763.48142075,334.76634208)(763.44642079,334.60134225)(763.44642334,334.37134644)
\curveto(763.45642078,334.1513427)(763.46142077,333.95634289)(763.46142334,333.78634644)
\lineto(763.46142334,331.61134644)
\curveto(763.46142077,331.47134538)(763.46642077,331.29634555)(763.47642334,331.08634644)
\curveto(763.48642075,330.88634596)(763.46642077,330.73634611)(763.41642334,330.63634644)
\curveto(763.39642084,330.56634628)(763.35642088,330.52134633)(763.29642334,330.50134644)
\curveto(763.25642098,330.48134637)(763.21642102,330.47134638)(763.17642334,330.47134644)
\curveto(763.14642109,330.47134638)(763.10642113,330.46134639)(763.05642334,330.44134644)
\curveto(763.01642122,330.43134642)(762.97142126,330.42634642)(762.92142334,330.42634644)
\curveto(762.87142136,330.43634641)(762.82142141,330.44134641)(762.77142334,330.44134644)
\lineto(762.44142334,330.44134644)
\curveto(762.34142189,330.4513464)(762.25642198,330.48134637)(762.18642334,330.53134644)
\curveto(762.10642213,330.58134627)(762.06642217,330.67134618)(762.06642334,330.80134644)
\lineto(762.06642334,331.20634644)
\lineto(762.06642334,340.32634644)
\curveto(762.06642217,340.43633641)(762.06142217,340.5513363)(762.05142334,340.67134644)
\curveto(762.05142218,340.79133606)(762.07642216,340.88633596)(762.12642334,340.95634644)
\curveto(762.16642207,341.01633583)(762.24142199,341.06633578)(762.35142334,341.10634644)
\curveto(762.37142186,341.11633573)(762.39142184,341.11633573)(762.41142334,341.10634644)
\curveto(762.4314218,341.10633574)(762.45142178,341.11133574)(762.47142334,341.12134644)
\lineto(766.82142334,341.12134644)
\curveto(766.89141734,341.12133573)(766.96641727,341.12133573)(767.04642334,341.12134644)
\curveto(767.12641711,341.13133572)(767.19641704,341.13133572)(767.25642334,341.12134644)
\lineto(767.42142334,341.12134644)
\curveto(767.48141675,341.11133574)(767.54141669,341.10133575)(767.60142334,341.09134644)
\curveto(767.66141657,341.09133576)(767.72641651,341.08633576)(767.79642334,341.07634644)
\curveto(767.87641636,341.05633579)(767.95641628,341.04133581)(768.03642334,341.03134644)
\curveto(768.12641611,341.02133583)(768.21141602,341.00633584)(768.29142334,340.98634644)
\curveto(768.48141575,340.92633592)(768.65641558,340.86133599)(768.81642334,340.79134644)
\curveto(768.97641526,340.72133613)(769.12641511,340.63633621)(769.26642334,340.53634644)
\curveto(769.51641472,340.36633648)(769.71641452,340.15633669)(769.86642334,339.90634644)
\curveto(770.02641421,339.66633718)(770.15641408,339.38133747)(770.25642334,339.05134644)
\curveto(770.27641396,338.97133788)(770.28641395,338.88633796)(770.28642334,338.79634644)
\curveto(770.29641394,338.71633813)(770.31141392,338.63633821)(770.33142334,338.55634644)
\lineto(770.33142334,338.40634644)
\curveto(770.34141389,338.35633849)(770.34141389,338.29633855)(770.33142334,338.22634644)
\curveto(770.3314139,338.16633868)(770.32641391,338.11133874)(770.31642334,338.06134644)
\lineto(770.31642334,337.89634644)
\curveto(770.29641394,337.81633903)(770.28141395,337.74133911)(770.27142334,337.67134644)
\curveto(770.27141396,337.60133925)(770.26141397,337.53133932)(770.24142334,337.46134644)
\curveto(770.19141404,337.31133954)(770.14141409,337.16633968)(770.09142334,337.02634644)
\curveto(770.05141418,336.89633995)(769.99141424,336.77134008)(769.91142334,336.65134644)
\curveto(769.88141435,336.60134025)(769.84641439,336.55634029)(769.80642334,336.51634644)
\curveto(769.77641446,336.47634037)(769.74641449,336.43134042)(769.71642334,336.38134644)
\lineto(769.68642334,336.35134644)
\curveto(769.67641456,336.3513405)(769.66641457,336.3463405)(769.65642334,336.33634644)
\lineto(769.58142334,336.26134644)
\curveto(769.56141467,336.23134062)(769.54141469,336.20634064)(769.52142334,336.18634644)
\curveto(769.44141479,336.12634072)(769.36641487,336.06634078)(769.29642334,336.00634644)
\curveto(769.22641501,335.95634089)(769.15141508,335.90634094)(769.07142334,335.85634644)
\curveto(769.02141521,335.82634102)(768.97641526,335.79134106)(768.93642334,335.75134644)
\curveto(768.89641534,335.72134113)(768.87141536,335.67634117)(768.86142334,335.61634644)
\curveto(768.85141538,335.55634129)(768.87141536,335.50634134)(768.92142334,335.46634644)
\curveto(768.98141525,335.42634142)(769.0314152,335.39634145)(769.07142334,335.37634644)
\curveto(769.18141505,335.30634154)(769.28141495,335.23134162)(769.37142334,335.15134644)
\curveto(769.47141476,335.07134178)(769.55641468,334.97634187)(769.62642334,334.86634644)
\curveto(769.7364145,334.72634212)(769.81641442,334.56634228)(769.86642334,334.38634644)
\curveto(769.91641432,334.21634263)(769.96641427,334.03134282)(770.01642334,333.83134644)
\lineto(770.04642334,333.59134644)
\curveto(770.05641418,333.52134333)(770.06641417,333.4463434)(770.07642334,333.36634644)
\curveto(770.09641414,333.29634355)(770.10141413,333.22634362)(770.09142334,333.15634644)
\curveto(770.08141415,333.08634376)(770.08641415,333.01634383)(770.10642334,332.94634644)
\lineto(770.10642334,332.81134644)
\curveto(770.12641411,332.74134411)(770.1314141,332.66634418)(770.12142334,332.58634644)
\curveto(770.11141412,332.50634434)(770.11641412,332.42634442)(770.13642334,332.34634644)
\curveto(770.14641409,332.30634454)(770.14641409,332.26634458)(770.13642334,332.22634644)
\curveto(770.1364141,332.19634465)(770.14141409,332.15634469)(770.15142334,332.10634644)
\curveto(770.17141406,332.00634484)(770.18641405,331.90134495)(770.19642334,331.79134644)
\curveto(770.20641403,331.69134516)(770.22641401,331.59634525)(770.25642334,331.50634644)
\curveto(770.27641396,331.4463454)(770.28641395,331.38634546)(770.28642334,331.32634644)
\curveto(770.29641394,331.27634557)(770.31141392,331.22134563)(770.33142334,331.16134644)
\lineto(770.46642334,330.80134644)
\moveto(768.65142334,337.02634644)
\curveto(768.72141551,337.13633971)(768.77141546,337.2513396)(768.80142334,337.37134644)
\curveto(768.84141539,337.49133936)(768.87641536,337.62133923)(768.90642334,337.76134644)
\lineto(768.90642334,337.89634644)
\curveto(768.9364153,338.03633881)(768.94141529,338.18633866)(768.92142334,338.34634644)
\curveto(768.90141533,338.51633833)(768.87141536,338.65633819)(768.83142334,338.76634644)
\curveto(768.67141556,339.26633758)(768.35641588,339.61133724)(767.88642334,339.80134644)
\curveto(767.68641655,339.88133697)(767.45141678,339.92633692)(767.18142334,339.93634644)
\curveto(766.92141731,339.9463369)(766.65141758,339.9513369)(766.37142334,339.95134644)
\lineto(763.89642334,339.95134644)
\curveto(763.87642036,339.94133691)(763.85142038,339.93633691)(763.82142334,339.93634644)
\curveto(763.80142043,339.93633691)(763.77642046,339.93133692)(763.74642334,339.92134644)
\curveto(763.62642061,339.89133696)(763.54642069,339.82633702)(763.50642334,339.72634644)
\curveto(763.46642077,339.63633721)(763.44642079,339.51133734)(763.44642334,339.35134644)
\curveto(763.45642078,339.19133766)(763.46142077,339.0463378)(763.46142334,338.91634644)
\lineto(763.46142334,337.19134644)
\curveto(763.46142077,337.04133981)(763.45642078,336.88133997)(763.44642334,336.71134644)
\curveto(763.44642079,336.5513403)(763.48142075,336.42634042)(763.55142334,336.33634644)
\curveto(763.60142063,336.26634058)(763.67642056,336.22134063)(763.77642334,336.20134644)
\curveto(763.87642036,336.19134066)(763.98642025,336.18634066)(764.10642334,336.18634644)
\lineto(765.03642334,336.18634644)
\curveto(765.42641881,336.18634066)(765.80641843,336.18134067)(766.17642334,336.17134644)
\curveto(766.54641769,336.17134068)(766.88641735,336.19134066)(767.19642334,336.23134644)
\curveto(767.51641672,336.28134057)(767.80141643,336.36634048)(768.05142334,336.48634644)
\curveto(768.30141593,336.60634024)(768.50141573,336.78634006)(768.65142334,337.02634644)
}
}
{
\newrgbcolor{curcolor}{0 0 0}
\pscustom[linestyle=none,fillstyle=solid,fillcolor=curcolor]
{
\newpath
\moveto(778.84462646,334.59634644)
\curveto(778.86461878,334.49634235)(778.86461878,334.38134247)(778.84462646,334.25134644)
\curveto(778.83461881,334.13134272)(778.80461884,334.0463428)(778.75462646,333.99634644)
\curveto(778.70461894,333.95634289)(778.62961901,333.92634292)(778.52962646,333.90634644)
\curveto(778.4396192,333.89634295)(778.33461931,333.89134296)(778.21462646,333.89134644)
\lineto(777.85462646,333.89134644)
\curveto(777.73461991,333.90134295)(777.62962001,333.90634294)(777.53962646,333.90634644)
\lineto(773.69962646,333.90634644)
\curveto(773.61962402,333.90634294)(773.5396241,333.90134295)(773.45962646,333.89134644)
\curveto(773.37962426,333.89134296)(773.31462433,333.87634297)(773.26462646,333.84634644)
\curveto(773.22462442,333.82634302)(773.18462446,333.78634306)(773.14462646,333.72634644)
\curveto(773.12462452,333.69634315)(773.10462454,333.6513432)(773.08462646,333.59134644)
\curveto(773.06462458,333.54134331)(773.06462458,333.49134336)(773.08462646,333.44134644)
\curveto(773.09462455,333.39134346)(773.09962454,333.3463435)(773.09962646,333.30634644)
\curveto(773.09962454,333.26634358)(773.10462454,333.22634362)(773.11462646,333.18634644)
\curveto(773.13462451,333.10634374)(773.15462449,333.02134383)(773.17462646,332.93134644)
\curveto(773.19462445,332.851344)(773.22462442,332.77134408)(773.26462646,332.69134644)
\curveto(773.49462415,332.1513447)(773.87462377,331.76634508)(774.40462646,331.53634644)
\curveto(774.46462318,331.50634534)(774.52962311,331.48134537)(774.59962646,331.46134644)
\lineto(774.80962646,331.40134644)
\curveto(774.8396228,331.39134546)(774.88962275,331.38634546)(774.95962646,331.38634644)
\curveto(775.09962254,331.3463455)(775.28462236,331.32634552)(775.51462646,331.32634644)
\curveto(775.7446219,331.32634552)(775.92962171,331.3463455)(776.06962646,331.38634644)
\curveto(776.20962143,331.42634542)(776.33462131,331.46634538)(776.44462646,331.50634644)
\curveto(776.56462108,331.55634529)(776.67462097,331.61634523)(776.77462646,331.68634644)
\curveto(776.88462076,331.75634509)(776.97962066,331.83634501)(777.05962646,331.92634644)
\curveto(777.1396205,332.02634482)(777.20962043,332.13134472)(777.26962646,332.24134644)
\curveto(777.32962031,332.34134451)(777.37962026,332.4463444)(777.41962646,332.55634644)
\curveto(777.46962017,332.66634418)(777.54962009,332.7463441)(777.65962646,332.79634644)
\curveto(777.69961994,332.81634403)(777.76461988,332.83134402)(777.85462646,332.84134644)
\curveto(777.9446197,332.851344)(778.03461961,332.851344)(778.12462646,332.84134644)
\curveto(778.21461943,332.84134401)(778.29961934,332.83634401)(778.37962646,332.82634644)
\curveto(778.45961918,332.81634403)(778.51461913,332.79634405)(778.54462646,332.76634644)
\curveto(778.644619,332.69634415)(778.66961897,332.58134427)(778.61962646,332.42134644)
\curveto(778.5396191,332.1513447)(778.43461921,331.91134494)(778.30462646,331.70134644)
\curveto(778.10461954,331.38134547)(777.87461977,331.11634573)(777.61462646,330.90634644)
\curveto(777.36462028,330.70634614)(777.0446206,330.54134631)(776.65462646,330.41134644)
\curveto(776.55462109,330.37134648)(776.45462119,330.3463465)(776.35462646,330.33634644)
\curveto(776.25462139,330.31634653)(776.14962149,330.29634655)(776.03962646,330.27634644)
\curveto(775.98962165,330.26634658)(775.9396217,330.26134659)(775.88962646,330.26134644)
\curveto(775.84962179,330.26134659)(775.80462184,330.25634659)(775.75462646,330.24634644)
\lineto(775.60462646,330.24634644)
\curveto(775.55462209,330.23634661)(775.49462215,330.23134662)(775.42462646,330.23134644)
\curveto(775.36462228,330.23134662)(775.31462233,330.23634661)(775.27462646,330.24634644)
\lineto(775.13962646,330.24634644)
\curveto(775.08962255,330.25634659)(775.0446226,330.26134659)(775.00462646,330.26134644)
\curveto(774.96462268,330.26134659)(774.92462272,330.26634658)(774.88462646,330.27634644)
\curveto(774.83462281,330.28634656)(774.77962286,330.29634655)(774.71962646,330.30634644)
\curveto(774.65962298,330.30634654)(774.60462304,330.31134654)(774.55462646,330.32134644)
\curveto(774.46462318,330.34134651)(774.37462327,330.36634648)(774.28462646,330.39634644)
\curveto(774.19462345,330.41634643)(774.10962353,330.44134641)(774.02962646,330.47134644)
\curveto(773.98962365,330.49134636)(773.95462369,330.50134635)(773.92462646,330.50134644)
\curveto(773.89462375,330.51134634)(773.85962378,330.52634632)(773.81962646,330.54634644)
\curveto(773.66962397,330.61634623)(773.50962413,330.70134615)(773.33962646,330.80134644)
\curveto(773.04962459,330.99134586)(772.79962484,331.22134563)(772.58962646,331.49134644)
\curveto(772.38962525,331.77134508)(772.21962542,332.08134477)(772.07962646,332.42134644)
\curveto(772.02962561,332.53134432)(771.98962565,332.6463442)(771.95962646,332.76634644)
\curveto(771.9396257,332.88634396)(771.90962573,333.00634384)(771.86962646,333.12634644)
\curveto(771.85962578,333.16634368)(771.85462579,333.20134365)(771.85462646,333.23134644)
\curveto(771.85462579,333.26134359)(771.84962579,333.30134355)(771.83962646,333.35134644)
\curveto(771.81962582,333.43134342)(771.80462584,333.51634333)(771.79462646,333.60634644)
\curveto(771.78462586,333.69634315)(771.76962587,333.78634306)(771.74962646,333.87634644)
\lineto(771.74962646,334.08634644)
\curveto(771.7396259,334.12634272)(771.72962591,334.18134267)(771.71962646,334.25134644)
\curveto(771.71962592,334.33134252)(771.72462592,334.39634245)(771.73462646,334.44634644)
\lineto(771.73462646,334.61134644)
\curveto(771.75462589,334.66134219)(771.75962588,334.71134214)(771.74962646,334.76134644)
\curveto(771.74962589,334.82134203)(771.75462589,334.87634197)(771.76462646,334.92634644)
\curveto(771.80462584,335.08634176)(771.83462581,335.2463416)(771.85462646,335.40634644)
\curveto(771.88462576,335.56634128)(771.92962571,335.71634113)(771.98962646,335.85634644)
\curveto(772.0396256,335.96634088)(772.08462556,336.07634077)(772.12462646,336.18634644)
\curveto(772.17462547,336.30634054)(772.22962541,336.42134043)(772.28962646,336.53134644)
\curveto(772.50962513,336.88133997)(772.75962488,337.18133967)(773.03962646,337.43134644)
\curveto(773.31962432,337.69133916)(773.66462398,337.90633894)(774.07462646,338.07634644)
\curveto(774.19462345,338.12633872)(774.31462333,338.16133869)(774.43462646,338.18134644)
\curveto(774.56462308,338.21133864)(774.69962294,338.24133861)(774.83962646,338.27134644)
\curveto(774.88962275,338.28133857)(774.93462271,338.28633856)(774.97462646,338.28634644)
\curveto(775.01462263,338.29633855)(775.05962258,338.30133855)(775.10962646,338.30134644)
\curveto(775.12962251,338.31133854)(775.15462249,338.31133854)(775.18462646,338.30134644)
\curveto(775.21462243,338.29133856)(775.2396224,338.29633855)(775.25962646,338.31634644)
\curveto(775.67962196,338.32633852)(776.0446216,338.28133857)(776.35462646,338.18134644)
\curveto(776.66462098,338.09133876)(776.9446207,337.96633888)(777.19462646,337.80634644)
\curveto(777.2446204,337.78633906)(777.28462036,337.75633909)(777.31462646,337.71634644)
\curveto(777.3446203,337.68633916)(777.37962026,337.66133919)(777.41962646,337.64134644)
\curveto(777.49962014,337.58133927)(777.57962006,337.51133934)(777.65962646,337.43134644)
\curveto(777.74961989,337.3513395)(777.82461982,337.27133958)(777.88462646,337.19134644)
\curveto(778.0446196,336.98133987)(778.17961946,336.78134007)(778.28962646,336.59134644)
\curveto(778.35961928,336.48134037)(778.41461923,336.36134049)(778.45462646,336.23134644)
\curveto(778.49461915,336.10134075)(778.5396191,335.97134088)(778.58962646,335.84134644)
\curveto(778.639619,335.71134114)(778.67461897,335.57634127)(778.69462646,335.43634644)
\curveto(778.72461892,335.29634155)(778.75961888,335.15634169)(778.79962646,335.01634644)
\curveto(778.80961883,334.9463419)(778.81461883,334.87634197)(778.81462646,334.80634644)
\lineto(778.84462646,334.59634644)
\moveto(777.38962646,335.10634644)
\curveto(777.41962022,335.1463417)(777.4446202,335.19634165)(777.46462646,335.25634644)
\curveto(777.48462016,335.32634152)(777.48462016,335.39634145)(777.46462646,335.46634644)
\curveto(777.40462024,335.68634116)(777.31962032,335.89134096)(777.20962646,336.08134644)
\curveto(777.06962057,336.31134054)(776.91462073,336.50634034)(776.74462646,336.66634644)
\curveto(776.57462107,336.82634002)(776.35462129,336.96133989)(776.08462646,337.07134644)
\curveto(776.01462163,337.09133976)(775.9446217,337.10633974)(775.87462646,337.11634644)
\curveto(775.80462184,337.13633971)(775.72962191,337.15633969)(775.64962646,337.17634644)
\curveto(775.56962207,337.19633965)(775.48462216,337.20633964)(775.39462646,337.20634644)
\lineto(775.13962646,337.20634644)
\curveto(775.10962253,337.18633966)(775.07462257,337.17633967)(775.03462646,337.17634644)
\curveto(774.99462265,337.18633966)(774.95962268,337.18633966)(774.92962646,337.17634644)
\lineto(774.68962646,337.11634644)
\curveto(774.61962302,337.10633974)(774.54962309,337.09133976)(774.47962646,337.07134644)
\curveto(774.18962345,336.9513399)(773.95462369,336.80134005)(773.77462646,336.62134644)
\curveto(773.60462404,336.44134041)(773.44962419,336.21634063)(773.30962646,335.94634644)
\curveto(773.27962436,335.89634095)(773.24962439,335.83134102)(773.21962646,335.75134644)
\curveto(773.18962445,335.68134117)(773.16462448,335.60134125)(773.14462646,335.51134644)
\curveto(773.12462452,335.42134143)(773.11962452,335.33634151)(773.12962646,335.25634644)
\curveto(773.1396245,335.17634167)(773.17462447,335.11634173)(773.23462646,335.07634644)
\curveto(773.31462433,335.01634183)(773.44962419,334.98634186)(773.63962646,334.98634644)
\curveto(773.8396238,334.99634185)(774.00962363,335.00134185)(774.14962646,335.00134644)
\lineto(776.42962646,335.00134644)
\curveto(776.57962106,335.00134185)(776.75962088,334.99634185)(776.96962646,334.98634644)
\curveto(777.17962046,334.98634186)(777.31962032,335.02634182)(777.38962646,335.10634644)
}
}
{
\newrgbcolor{curcolor}{0 0 0}
\pscustom[linestyle=none,fillstyle=solid,fillcolor=curcolor]
{
\newpath
\moveto(783.28626709,338.33134644)
\curveto(784.0262623,338.34133851)(784.64126168,338.23133862)(785.13126709,338.00134644)
\curveto(785.63126069,337.78133907)(786.0262603,337.4463394)(786.31626709,336.99634644)
\curveto(786.44625988,336.79634005)(786.55625977,336.5513403)(786.64626709,336.26134644)
\curveto(786.66625966,336.21134064)(786.68125964,336.1463407)(786.69126709,336.06634644)
\curveto(786.70125962,335.98634086)(786.69625963,335.91634093)(786.67626709,335.85634644)
\curveto(786.64625968,335.80634104)(786.59625973,335.76134109)(786.52626709,335.72134644)
\curveto(786.49625983,335.70134115)(786.46625986,335.69134116)(786.43626709,335.69134644)
\curveto(786.40625992,335.70134115)(786.37125995,335.70134115)(786.33126709,335.69134644)
\curveto(786.29126003,335.68134117)(786.25126007,335.67634117)(786.21126709,335.67634644)
\curveto(786.17126015,335.68634116)(786.13126019,335.69134116)(786.09126709,335.69134644)
\lineto(785.77626709,335.69134644)
\curveto(785.67626065,335.70134115)(785.59126073,335.73134112)(785.52126709,335.78134644)
\curveto(785.44126088,335.84134101)(785.38626094,335.92634092)(785.35626709,336.03634644)
\curveto(785.326261,336.1463407)(785.28626104,336.24134061)(785.23626709,336.32134644)
\curveto(785.08626124,336.58134027)(784.89126143,336.78634006)(784.65126709,336.93634644)
\curveto(784.57126175,336.98633986)(784.48626184,337.02633982)(784.39626709,337.05634644)
\curveto(784.30626202,337.09633975)(784.21126211,337.13133972)(784.11126709,337.16134644)
\curveto(783.97126235,337.20133965)(783.78626254,337.22133963)(783.55626709,337.22134644)
\curveto(783.326263,337.23133962)(783.13626319,337.21133964)(782.98626709,337.16134644)
\curveto(782.91626341,337.14133971)(782.85126347,337.12633972)(782.79126709,337.11634644)
\curveto(782.73126359,337.10633974)(782.66626366,337.09133976)(782.59626709,337.07134644)
\curveto(782.33626399,336.96133989)(782.10626422,336.81134004)(781.90626709,336.62134644)
\curveto(781.70626462,336.43134042)(781.55126477,336.20634064)(781.44126709,335.94634644)
\curveto(781.40126492,335.85634099)(781.36626496,335.76134109)(781.33626709,335.66134644)
\curveto(781.30626502,335.57134128)(781.27626505,335.47134138)(781.24626709,335.36134644)
\lineto(781.15626709,334.95634644)
\curveto(781.14626518,334.90634194)(781.14126518,334.851342)(781.14126709,334.79134644)
\curveto(781.15126517,334.73134212)(781.14626518,334.67634217)(781.12626709,334.62634644)
\lineto(781.12626709,334.50634644)
\curveto(781.11626521,334.46634238)(781.10626522,334.40134245)(781.09626709,334.31134644)
\curveto(781.09626523,334.22134263)(781.10626522,334.15634269)(781.12626709,334.11634644)
\curveto(781.13626519,334.06634278)(781.13626519,334.01634283)(781.12626709,333.96634644)
\curveto(781.11626521,333.91634293)(781.11626521,333.86634298)(781.12626709,333.81634644)
\curveto(781.13626519,333.77634307)(781.14126518,333.70634314)(781.14126709,333.60634644)
\curveto(781.16126516,333.52634332)(781.17626515,333.44134341)(781.18626709,333.35134644)
\curveto(781.20626512,333.26134359)(781.2262651,333.17634367)(781.24626709,333.09634644)
\curveto(781.35626497,332.77634407)(781.48126484,332.49634435)(781.62126709,332.25634644)
\curveto(781.77126455,332.02634482)(781.97626435,331.82634502)(782.23626709,331.65634644)
\curveto(782.326264,331.60634524)(782.41626391,331.56134529)(782.50626709,331.52134644)
\curveto(782.60626372,331.48134537)(782.71126361,331.44134541)(782.82126709,331.40134644)
\curveto(782.87126345,331.39134546)(782.91126341,331.38634546)(782.94126709,331.38634644)
\curveto(782.97126335,331.38634546)(783.01126331,331.38134547)(783.06126709,331.37134644)
\curveto(783.09126323,331.36134549)(783.14126318,331.35634549)(783.21126709,331.35634644)
\lineto(783.37626709,331.35634644)
\curveto(783.37626295,331.3463455)(783.39626293,331.34134551)(783.43626709,331.34134644)
\curveto(783.45626287,331.3513455)(783.48126284,331.3513455)(783.51126709,331.34134644)
\curveto(783.54126278,331.34134551)(783.57126275,331.3463455)(783.60126709,331.35634644)
\curveto(783.67126265,331.37634547)(783.73626259,331.38134547)(783.79626709,331.37134644)
\curveto(783.86626246,331.37134548)(783.93626239,331.38134547)(784.00626709,331.40134644)
\curveto(784.26626206,331.48134537)(784.49126183,331.58134527)(784.68126709,331.70134644)
\curveto(784.87126145,331.83134502)(785.03126129,331.99634485)(785.16126709,332.19634644)
\curveto(785.21126111,332.27634457)(785.25626107,332.36134449)(785.29626709,332.45134644)
\lineto(785.41626709,332.72134644)
\curveto(785.43626089,332.80134405)(785.45626087,332.87634397)(785.47626709,332.94634644)
\curveto(785.50626082,333.02634382)(785.55626077,333.09134376)(785.62626709,333.14134644)
\curveto(785.65626067,333.17134368)(785.71626061,333.19134366)(785.80626709,333.20134644)
\curveto(785.89626043,333.22134363)(785.99126033,333.23134362)(786.09126709,333.23134644)
\curveto(786.20126012,333.24134361)(786.30126002,333.24134361)(786.39126709,333.23134644)
\curveto(786.49125983,333.22134363)(786.56125976,333.20134365)(786.60126709,333.17134644)
\curveto(786.66125966,333.13134372)(786.69625963,333.07134378)(786.70626709,332.99134644)
\curveto(786.7262596,332.91134394)(786.7262596,332.82634402)(786.70626709,332.73634644)
\curveto(786.65625967,332.58634426)(786.60625972,332.44134441)(786.55626709,332.30134644)
\curveto(786.51625981,332.17134468)(786.46125986,332.04134481)(786.39126709,331.91134644)
\curveto(786.24126008,331.61134524)(786.05126027,331.3463455)(785.82126709,331.11634644)
\curveto(785.60126072,330.88634596)(785.33126099,330.70134615)(785.01126709,330.56134644)
\curveto(784.93126139,330.52134633)(784.84626148,330.48634636)(784.75626709,330.45634644)
\curveto(784.66626166,330.43634641)(784.57126175,330.41134644)(784.47126709,330.38134644)
\curveto(784.36126196,330.34134651)(784.25126207,330.32134653)(784.14126709,330.32134644)
\curveto(784.03126229,330.31134654)(783.9212624,330.29634655)(783.81126709,330.27634644)
\curveto(783.77126255,330.25634659)(783.73126259,330.2513466)(783.69126709,330.26134644)
\curveto(783.65126267,330.27134658)(783.61126271,330.27134658)(783.57126709,330.26134644)
\lineto(783.43626709,330.26134644)
\lineto(783.19626709,330.26134644)
\curveto(783.1262632,330.2513466)(783.06126326,330.25634659)(783.00126709,330.27634644)
\lineto(782.92626709,330.27634644)
\lineto(782.56626709,330.32134644)
\curveto(782.43626389,330.36134649)(782.31126401,330.39634645)(782.19126709,330.42634644)
\curveto(782.07126425,330.45634639)(781.95626437,330.49634635)(781.84626709,330.54634644)
\curveto(781.48626484,330.70634614)(781.18626514,330.89634595)(780.94626709,331.11634644)
\curveto(780.71626561,331.33634551)(780.50126582,331.60634524)(780.30126709,331.92634644)
\curveto(780.25126607,332.00634484)(780.20626612,332.09634475)(780.16626709,332.19634644)
\lineto(780.04626709,332.49634644)
\curveto(779.99626633,332.60634424)(779.96126636,332.72134413)(779.94126709,332.84134644)
\curveto(779.9212664,332.96134389)(779.89626643,333.08134377)(779.86626709,333.20134644)
\curveto(779.85626647,333.24134361)(779.85126647,333.28134357)(779.85126709,333.32134644)
\curveto(779.85126647,333.36134349)(779.84626648,333.40134345)(779.83626709,333.44134644)
\curveto(779.81626651,333.50134335)(779.80626652,333.56634328)(779.80626709,333.63634644)
\curveto(779.81626651,333.70634314)(779.81126651,333.77134308)(779.79126709,333.83134644)
\lineto(779.79126709,333.98134644)
\curveto(779.78126654,334.03134282)(779.77626655,334.10134275)(779.77626709,334.19134644)
\curveto(779.77626655,334.28134257)(779.78126654,334.3513425)(779.79126709,334.40134644)
\curveto(779.80126652,334.4513424)(779.80126652,334.49634235)(779.79126709,334.53634644)
\curveto(779.79126653,334.57634227)(779.79626653,334.61634223)(779.80626709,334.65634644)
\curveto(779.8262665,334.72634212)(779.83126649,334.79634205)(779.82126709,334.86634644)
\curveto(779.8212665,334.93634191)(779.83126649,335.00134185)(779.85126709,335.06134644)
\curveto(779.89126643,335.23134162)(779.9262664,335.40134145)(779.95626709,335.57134644)
\curveto(779.98626634,335.74134111)(780.03126629,335.90134095)(780.09126709,336.05134644)
\curveto(780.30126602,336.57134028)(780.55626577,336.99133986)(780.85626709,337.31134644)
\curveto(781.15626517,337.63133922)(781.56626476,337.89633895)(782.08626709,338.10634644)
\curveto(782.19626413,338.15633869)(782.31626401,338.19133866)(782.44626709,338.21134644)
\curveto(782.57626375,338.23133862)(782.71126361,338.25633859)(782.85126709,338.28634644)
\curveto(782.9212634,338.29633855)(782.99126333,338.30133855)(783.06126709,338.30134644)
\curveto(783.13126319,338.31133854)(783.20626312,338.32133853)(783.28626709,338.33134644)
}
}
{
\newrgbcolor{curcolor}{0 0 0}
\pscustom[linestyle=none,fillstyle=solid,fillcolor=curcolor]
{
\newpath
\moveto(788.67290771,338.15134644)
\lineto(789.10790771,338.15134644)
\curveto(789.25790575,338.1513387)(789.36290564,338.11133874)(789.42290771,338.03134644)
\curveto(789.47290553,337.9513389)(789.49790551,337.851339)(789.49790771,337.73134644)
\curveto(789.5079055,337.61133924)(789.51290549,337.49133936)(789.51290771,337.37134644)
\lineto(789.51290771,335.94634644)
\lineto(789.51290771,333.68134644)
\lineto(789.51290771,332.99134644)
\curveto(789.51290549,332.76134409)(789.53790547,332.56134429)(789.58790771,332.39134644)
\curveto(789.74790526,331.94134491)(790.04790496,331.62634522)(790.48790771,331.44634644)
\curveto(790.7079043,331.35634549)(790.97290403,331.32134553)(791.28290771,331.34134644)
\curveto(791.59290341,331.37134548)(791.84290316,331.42634542)(792.03290771,331.50634644)
\curveto(792.36290264,331.6463452)(792.62290238,331.82134503)(792.81290771,332.03134644)
\curveto(793.01290199,332.2513446)(793.16790184,332.53634431)(793.27790771,332.88634644)
\curveto(793.3079017,332.96634388)(793.32790168,333.0463438)(793.33790771,333.12634644)
\curveto(793.34790166,333.20634364)(793.36290164,333.29134356)(793.38290771,333.38134644)
\curveto(793.39290161,333.43134342)(793.39290161,333.47634337)(793.38290771,333.51634644)
\curveto(793.38290162,333.55634329)(793.39290161,333.60134325)(793.41290771,333.65134644)
\lineto(793.41290771,333.96634644)
\curveto(793.43290157,334.0463428)(793.43790157,334.13634271)(793.42790771,334.23634644)
\curveto(793.41790159,334.3463425)(793.41290159,334.4463424)(793.41290771,334.53634644)
\lineto(793.41290771,335.70634644)
\lineto(793.41290771,337.29634644)
\curveto(793.41290159,337.41633943)(793.4079016,337.54133931)(793.39790771,337.67134644)
\curveto(793.39790161,337.81133904)(793.42290158,337.92133893)(793.47290771,338.00134644)
\curveto(793.51290149,338.0513388)(793.55790145,338.08133877)(793.60790771,338.09134644)
\curveto(793.66790134,338.11133874)(793.73790127,338.13133872)(793.81790771,338.15134644)
\lineto(794.04290771,338.15134644)
\curveto(794.16290084,338.1513387)(794.26790074,338.1463387)(794.35790771,338.13634644)
\curveto(794.45790055,338.12633872)(794.53290047,338.08133877)(794.58290771,338.00134644)
\curveto(794.63290037,337.9513389)(794.65790035,337.87633897)(794.65790771,337.77634644)
\lineto(794.65790771,337.49134644)
\lineto(794.65790771,336.47134644)
\lineto(794.65790771,332.43634644)
\lineto(794.65790771,331.08634644)
\curveto(794.65790035,330.96634588)(794.65290035,330.851346)(794.64290771,330.74134644)
\curveto(794.64290036,330.64134621)(794.6079004,330.56634628)(794.53790771,330.51634644)
\curveto(794.49790051,330.48634636)(794.43790057,330.46134639)(794.35790771,330.44134644)
\curveto(794.27790073,330.43134642)(794.18790082,330.42134643)(794.08790771,330.41134644)
\curveto(793.99790101,330.41134644)(793.9079011,330.41634643)(793.81790771,330.42634644)
\curveto(793.73790127,330.43634641)(793.67790133,330.45634639)(793.63790771,330.48634644)
\curveto(793.58790142,330.52634632)(793.54290146,330.59134626)(793.50290771,330.68134644)
\curveto(793.49290151,330.72134613)(793.48290152,330.77634607)(793.47290771,330.84634644)
\curveto(793.47290153,330.91634593)(793.46790154,330.98134587)(793.45790771,331.04134644)
\curveto(793.44790156,331.11134574)(793.42790158,331.16634568)(793.39790771,331.20634644)
\curveto(793.36790164,331.2463456)(793.32290168,331.26134559)(793.26290771,331.25134644)
\curveto(793.18290182,331.23134562)(793.1029019,331.17134568)(793.02290771,331.07134644)
\curveto(792.94290206,330.98134587)(792.86790214,330.91134594)(792.79790771,330.86134644)
\curveto(792.57790243,330.70134615)(792.32790268,330.56134629)(792.04790771,330.44134644)
\curveto(791.93790307,330.39134646)(791.82290318,330.36134649)(791.70290771,330.35134644)
\curveto(791.59290341,330.33134652)(791.47790353,330.30634654)(791.35790771,330.27634644)
\curveto(791.3079037,330.26634658)(791.25290375,330.26634658)(791.19290771,330.27634644)
\curveto(791.14290386,330.28634656)(791.09290391,330.28134657)(791.04290771,330.26134644)
\curveto(790.94290406,330.24134661)(790.85290415,330.24134661)(790.77290771,330.26134644)
\lineto(790.62290771,330.26134644)
\curveto(790.57290443,330.28134657)(790.51290449,330.29134656)(790.44290771,330.29134644)
\curveto(790.38290462,330.29134656)(790.32790468,330.29634655)(790.27790771,330.30634644)
\curveto(790.23790477,330.32634652)(790.19790481,330.33634651)(790.15790771,330.33634644)
\curveto(790.12790488,330.32634652)(790.08790492,330.33134652)(790.03790771,330.35134644)
\lineto(789.79790771,330.41134644)
\curveto(789.72790528,330.43134642)(789.65290535,330.46134639)(789.57290771,330.50134644)
\curveto(789.31290569,330.61134624)(789.09290591,330.75634609)(788.91290771,330.93634644)
\curveto(788.74290626,331.12634572)(788.6029064,331.3513455)(788.49290771,331.61134644)
\curveto(788.45290655,331.70134515)(788.42290658,331.79134506)(788.40290771,331.88134644)
\lineto(788.34290771,332.18134644)
\curveto(788.32290668,332.24134461)(788.31290669,332.29634455)(788.31290771,332.34634644)
\curveto(788.32290668,332.40634444)(788.31790669,332.47134438)(788.29790771,332.54134644)
\curveto(788.28790672,332.56134429)(788.28290672,332.58634426)(788.28290771,332.61634644)
\curveto(788.28290672,332.65634419)(788.27790673,332.69134416)(788.26790771,332.72134644)
\lineto(788.26790771,332.87134644)
\curveto(788.25790675,332.91134394)(788.25290675,332.95634389)(788.25290771,333.00634644)
\curveto(788.26290674,333.06634378)(788.26790674,333.12134373)(788.26790771,333.17134644)
\lineto(788.26790771,333.77134644)
\lineto(788.26790771,336.53134644)
\lineto(788.26790771,337.49134644)
\lineto(788.26790771,337.76134644)
\curveto(788.26790674,337.851339)(788.28790672,337.92633892)(788.32790771,337.98634644)
\curveto(788.36790664,338.05633879)(788.44290656,338.10633874)(788.55290771,338.13634644)
\curveto(788.57290643,338.1463387)(788.59290641,338.1463387)(788.61290771,338.13634644)
\curveto(788.63290637,338.13633871)(788.65290635,338.14133871)(788.67290771,338.15134644)
}
}
{
\newrgbcolor{curcolor}{0 0 0}
\pscustom[linestyle=none,fillstyle=solid,fillcolor=curcolor]
{
\newpath
\moveto(800.20251709,338.33134644)
\curveto(800.4325123,338.33133852)(800.56251217,338.27133858)(800.59251709,338.15134644)
\curveto(800.62251211,338.04133881)(800.63751209,337.87633897)(800.63751709,337.65634644)
\lineto(800.63751709,337.37134644)
\curveto(800.63751209,337.28133957)(800.61251212,337.20633964)(800.56251709,337.14634644)
\curveto(800.50251223,337.06633978)(800.41751231,337.02133983)(800.30751709,337.01134644)
\curveto(800.19751253,337.01133984)(800.08751264,336.99633985)(799.97751709,336.96634644)
\curveto(799.83751289,336.93633991)(799.70251303,336.90633994)(799.57251709,336.87634644)
\curveto(799.45251328,336.84634)(799.33751339,336.80634004)(799.22751709,336.75634644)
\curveto(798.93751379,336.62634022)(798.70251403,336.4463404)(798.52251709,336.21634644)
\curveto(798.34251439,335.99634085)(798.18751454,335.74134111)(798.05751709,335.45134644)
\curveto(798.01751471,335.34134151)(797.98751474,335.22634162)(797.96751709,335.10634644)
\curveto(797.94751478,334.99634185)(797.92251481,334.88134197)(797.89251709,334.76134644)
\curveto(797.88251485,334.71134214)(797.87751485,334.66134219)(797.87751709,334.61134644)
\curveto(797.88751484,334.56134229)(797.88751484,334.51134234)(797.87751709,334.46134644)
\curveto(797.84751488,334.34134251)(797.8325149,334.20134265)(797.83251709,334.04134644)
\curveto(797.84251489,333.89134296)(797.84751488,333.7463431)(797.84751709,333.60634644)
\lineto(797.84751709,331.76134644)
\lineto(797.84751709,331.41634644)
\curveto(797.84751488,331.29634555)(797.84251489,331.18134567)(797.83251709,331.07134644)
\curveto(797.82251491,330.96134589)(797.81751491,330.86634598)(797.81751709,330.78634644)
\curveto(797.8275149,330.70634614)(797.80751492,330.63634621)(797.75751709,330.57634644)
\curveto(797.70751502,330.50634634)(797.6275151,330.46634638)(797.51751709,330.45634644)
\curveto(797.41751531,330.4463464)(797.30751542,330.44134641)(797.18751709,330.44134644)
\lineto(796.91751709,330.44134644)
\curveto(796.86751586,330.46134639)(796.81751591,330.47634637)(796.76751709,330.48634644)
\curveto(796.727516,330.50634634)(796.69751603,330.53134632)(796.67751709,330.56134644)
\curveto(796.6275161,330.63134622)(796.59751613,330.71634613)(796.58751709,330.81634644)
\lineto(796.58751709,331.14634644)
\lineto(796.58751709,332.30134644)
\lineto(796.58751709,336.45634644)
\lineto(796.58751709,337.49134644)
\lineto(796.58751709,337.79134644)
\curveto(796.59751613,337.89133896)(796.6275161,337.97633887)(796.67751709,338.04634644)
\curveto(796.70751602,338.08633876)(796.75751597,338.11633873)(796.82751709,338.13634644)
\curveto(796.90751582,338.15633869)(796.99251574,338.16633868)(797.08251709,338.16634644)
\curveto(797.17251556,338.17633867)(797.26251547,338.17633867)(797.35251709,338.16634644)
\curveto(797.44251529,338.15633869)(797.51251522,338.14133871)(797.56251709,338.12134644)
\curveto(797.64251509,338.09133876)(797.69251504,338.03133882)(797.71251709,337.94134644)
\curveto(797.74251499,337.86133899)(797.75751497,337.77133908)(797.75751709,337.67134644)
\lineto(797.75751709,337.37134644)
\curveto(797.75751497,337.27133958)(797.77751495,337.18133967)(797.81751709,337.10134644)
\curveto(797.8275149,337.08133977)(797.83751489,337.06633978)(797.84751709,337.05634644)
\lineto(797.89251709,337.01134644)
\curveto(798.00251473,337.01133984)(798.09251464,337.05633979)(798.16251709,337.14634644)
\curveto(798.2325145,337.2463396)(798.29251444,337.32633952)(798.34251709,337.38634644)
\lineto(798.43251709,337.47634644)
\curveto(798.52251421,337.58633926)(798.64751408,337.70133915)(798.80751709,337.82134644)
\curveto(798.96751376,337.94133891)(799.11751361,338.03133882)(799.25751709,338.09134644)
\curveto(799.34751338,338.14133871)(799.44251329,338.17633867)(799.54251709,338.19634644)
\curveto(799.64251309,338.22633862)(799.74751298,338.25633859)(799.85751709,338.28634644)
\curveto(799.91751281,338.29633855)(799.97751275,338.30133855)(800.03751709,338.30134644)
\curveto(800.09751263,338.31133854)(800.15251258,338.32133853)(800.20251709,338.33134644)
}
}
{
\newrgbcolor{curcolor}{0 0 0}
\pscustom[linestyle=none,fillstyle=solid,fillcolor=curcolor]
{
\newpath
\moveto(803.99728271,338.33134644)
\curveto(804.71727865,338.34133851)(805.32227804,338.25633859)(805.81228271,338.07634644)
\curveto(806.30227706,337.90633894)(806.68227668,337.60133925)(806.95228271,337.16134644)
\curveto(807.02227634,337.0513398)(807.07727629,336.93633991)(807.11728271,336.81634644)
\curveto(807.15727621,336.70634014)(807.19727617,336.58134027)(807.23728271,336.44134644)
\curveto(807.25727611,336.37134048)(807.2622761,336.29634055)(807.25228271,336.21634644)
\curveto(807.24227612,336.1463407)(807.22727614,336.09134076)(807.20728271,336.05134644)
\curveto(807.18727618,336.03134082)(807.1622762,336.01134084)(807.13228271,335.99134644)
\curveto(807.10227626,335.98134087)(807.07727629,335.96634088)(807.05728271,335.94634644)
\curveto(807.00727636,335.92634092)(806.95727641,335.92134093)(806.90728271,335.93134644)
\curveto(806.85727651,335.94134091)(806.80727656,335.94134091)(806.75728271,335.93134644)
\curveto(806.67727669,335.91134094)(806.57227679,335.90634094)(806.44228271,335.91634644)
\curveto(806.31227705,335.93634091)(806.22227714,335.96134089)(806.17228271,335.99134644)
\curveto(806.09227727,336.04134081)(806.03727733,336.10634074)(806.00728271,336.18634644)
\curveto(805.98727738,336.27634057)(805.95227741,336.36134049)(805.90228271,336.44134644)
\curveto(805.81227755,336.60134025)(805.68727768,336.7463401)(805.52728271,336.87634644)
\curveto(805.41727795,336.95633989)(805.29727807,337.01633983)(805.16728271,337.05634644)
\curveto(805.03727833,337.09633975)(804.89727847,337.13633971)(804.74728271,337.17634644)
\curveto(804.69727867,337.19633965)(804.64727872,337.20133965)(804.59728271,337.19134644)
\curveto(804.54727882,337.19133966)(804.49727887,337.19633965)(804.44728271,337.20634644)
\curveto(804.38727898,337.22633962)(804.31227905,337.23633961)(804.22228271,337.23634644)
\curveto(804.13227923,337.23633961)(804.05727931,337.22633962)(803.99728271,337.20634644)
\lineto(803.90728271,337.20634644)
\lineto(803.75728271,337.17634644)
\curveto(803.70727966,337.17633967)(803.65727971,337.17133968)(803.60728271,337.16134644)
\curveto(803.34728002,337.10133975)(803.13228023,337.01633983)(802.96228271,336.90634644)
\curveto(802.79228057,336.79634005)(802.67728069,336.61134024)(802.61728271,336.35134644)
\curveto(802.59728077,336.28134057)(802.59228077,336.21134064)(802.60228271,336.14134644)
\curveto(802.62228074,336.07134078)(802.64228072,336.01134084)(802.66228271,335.96134644)
\curveto(802.72228064,335.81134104)(802.79228057,335.70134115)(802.87228271,335.63134644)
\curveto(802.9622804,335.57134128)(803.07228029,335.50134135)(803.20228271,335.42134644)
\curveto(803.36228,335.32134153)(803.54227982,335.2463416)(803.74228271,335.19634644)
\curveto(803.94227942,335.15634169)(804.14227922,335.10634174)(804.34228271,335.04634644)
\curveto(804.47227889,335.00634184)(804.60227876,334.97634187)(804.73228271,334.95634644)
\curveto(804.8622785,334.93634191)(804.99227837,334.90634194)(805.12228271,334.86634644)
\curveto(805.33227803,334.80634204)(805.53727783,334.7463421)(805.73728271,334.68634644)
\curveto(805.93727743,334.63634221)(806.13727723,334.57134228)(806.33728271,334.49134644)
\lineto(806.48728271,334.43134644)
\curveto(806.53727683,334.41134244)(806.58727678,334.38634246)(806.63728271,334.35634644)
\curveto(806.83727653,334.23634261)(807.01227635,334.10134275)(807.16228271,333.95134644)
\curveto(807.31227605,333.80134305)(807.43727593,333.61134324)(807.53728271,333.38134644)
\curveto(807.55727581,333.31134354)(807.57727579,333.21634363)(807.59728271,333.09634644)
\curveto(807.61727575,333.02634382)(807.62727574,332.9513439)(807.62728271,332.87134644)
\curveto(807.63727573,332.80134405)(807.64227572,332.72134413)(807.64228271,332.63134644)
\lineto(807.64228271,332.48134644)
\curveto(807.62227574,332.41134444)(807.61227575,332.34134451)(807.61228271,332.27134644)
\curveto(807.61227575,332.20134465)(807.60227576,332.13134472)(807.58228271,332.06134644)
\curveto(807.55227581,331.9513449)(807.51727585,331.846345)(807.47728271,331.74634644)
\curveto(807.43727593,331.6463452)(807.39227597,331.55634529)(807.34228271,331.47634644)
\curveto(807.18227618,331.21634563)(806.97727639,331.00634584)(806.72728271,330.84634644)
\curveto(806.47727689,330.69634615)(806.19727717,330.56634628)(805.88728271,330.45634644)
\curveto(805.79727757,330.42634642)(805.70227766,330.40634644)(805.60228271,330.39634644)
\curveto(805.51227785,330.37634647)(805.42227794,330.3513465)(805.33228271,330.32134644)
\curveto(805.23227813,330.30134655)(805.13227823,330.29134656)(805.03228271,330.29134644)
\curveto(804.93227843,330.29134656)(804.83227853,330.28134657)(804.73228271,330.26134644)
\lineto(804.58228271,330.26134644)
\curveto(804.53227883,330.2513466)(804.4622789,330.2463466)(804.37228271,330.24634644)
\curveto(804.28227908,330.2463466)(804.21227915,330.2513466)(804.16228271,330.26134644)
\lineto(803.99728271,330.26134644)
\curveto(803.93727943,330.28134657)(803.87227949,330.29134656)(803.80228271,330.29134644)
\curveto(803.73227963,330.28134657)(803.67227969,330.28634656)(803.62228271,330.30634644)
\curveto(803.57227979,330.31634653)(803.50727986,330.32134653)(803.42728271,330.32134644)
\lineto(803.18728271,330.38134644)
\curveto(803.11728025,330.39134646)(803.04228032,330.41134644)(802.96228271,330.44134644)
\curveto(802.65228071,330.54134631)(802.38228098,330.66634618)(802.15228271,330.81634644)
\curveto(801.92228144,330.96634588)(801.72228164,331.16134569)(801.55228271,331.40134644)
\curveto(801.4622819,331.53134532)(801.38728198,331.66634518)(801.32728271,331.80634644)
\curveto(801.2672821,331.9463449)(801.21228215,332.10134475)(801.16228271,332.27134644)
\curveto(801.14228222,332.33134452)(801.13228223,332.40134445)(801.13228271,332.48134644)
\curveto(801.14228222,332.57134428)(801.15728221,332.64134421)(801.17728271,332.69134644)
\curveto(801.20728216,332.73134412)(801.25728211,332.77134408)(801.32728271,332.81134644)
\curveto(801.37728199,332.83134402)(801.44728192,332.84134401)(801.53728271,332.84134644)
\curveto(801.62728174,332.851344)(801.71728165,332.851344)(801.80728271,332.84134644)
\curveto(801.89728147,332.83134402)(801.98228138,332.81634403)(802.06228271,332.79634644)
\curveto(802.15228121,332.78634406)(802.21228115,332.77134408)(802.24228271,332.75134644)
\curveto(802.31228105,332.70134415)(802.35728101,332.62634422)(802.37728271,332.52634644)
\curveto(802.40728096,332.43634441)(802.44228092,332.3513445)(802.48228271,332.27134644)
\curveto(802.58228078,332.0513448)(802.71728065,331.88134497)(802.88728271,331.76134644)
\curveto(803.00728036,331.67134518)(803.14228022,331.60134525)(803.29228271,331.55134644)
\curveto(803.44227992,331.50134535)(803.60227976,331.4513454)(803.77228271,331.40134644)
\lineto(804.08728271,331.35634644)
\lineto(804.17728271,331.35634644)
\curveto(804.24727912,331.33634551)(804.33727903,331.32634552)(804.44728271,331.32634644)
\curveto(804.5672788,331.32634552)(804.6672787,331.33634551)(804.74728271,331.35634644)
\curveto(804.81727855,331.35634549)(804.87227849,331.36134549)(804.91228271,331.37134644)
\curveto(804.97227839,331.38134547)(805.03227833,331.38634546)(805.09228271,331.38634644)
\curveto(805.15227821,331.39634545)(805.20727816,331.40634544)(805.25728271,331.41634644)
\curveto(805.54727782,331.49634535)(805.77727759,331.60134525)(805.94728271,331.73134644)
\curveto(806.11727725,331.86134499)(806.23727713,332.08134477)(806.30728271,332.39134644)
\curveto(806.32727704,332.44134441)(806.33227703,332.49634435)(806.32228271,332.55634644)
\curveto(806.31227705,332.61634423)(806.30227706,332.66134419)(806.29228271,332.69134644)
\curveto(806.24227712,332.88134397)(806.17227719,333.02134383)(806.08228271,333.11134644)
\curveto(805.99227737,333.21134364)(805.87727749,333.30134355)(805.73728271,333.38134644)
\curveto(805.64727772,333.44134341)(805.54727782,333.49134336)(805.43728271,333.53134644)
\lineto(805.10728271,333.65134644)
\curveto(805.07727829,333.66134319)(805.04727832,333.66634318)(805.01728271,333.66634644)
\curveto(804.99727837,333.66634318)(804.97227839,333.67634317)(804.94228271,333.69634644)
\curveto(804.60227876,333.80634304)(804.24727912,333.88634296)(803.87728271,333.93634644)
\curveto(803.51727985,333.99634285)(803.17728019,334.09134276)(802.85728271,334.22134644)
\curveto(802.75728061,334.26134259)(802.6622807,334.29634255)(802.57228271,334.32634644)
\curveto(802.48228088,334.35634249)(802.39728097,334.39634245)(802.31728271,334.44634644)
\curveto(802.12728124,334.55634229)(801.95228141,334.68134217)(801.79228271,334.82134644)
\curveto(801.63228173,334.96134189)(801.50728186,335.13634171)(801.41728271,335.34634644)
\curveto(801.38728198,335.41634143)(801.362282,335.48634136)(801.34228271,335.55634644)
\curveto(801.33228203,335.62634122)(801.31728205,335.70134115)(801.29728271,335.78134644)
\curveto(801.2672821,335.90134095)(801.25728211,336.03634081)(801.26728271,336.18634644)
\curveto(801.27728209,336.3463405)(801.29228207,336.48134037)(801.31228271,336.59134644)
\curveto(801.33228203,336.64134021)(801.34228202,336.68134017)(801.34228271,336.71134644)
\curveto(801.35228201,336.7513401)(801.367282,336.79134006)(801.38728271,336.83134644)
\curveto(801.47728189,337.06133979)(801.59728177,337.26133959)(801.74728271,337.43134644)
\curveto(801.90728146,337.60133925)(802.08728128,337.7513391)(802.28728271,337.88134644)
\curveto(802.43728093,337.97133888)(802.60228076,338.04133881)(802.78228271,338.09134644)
\curveto(802.9622804,338.1513387)(803.15228021,338.20633864)(803.35228271,338.25634644)
\curveto(803.42227994,338.26633858)(803.48727988,338.27633857)(803.54728271,338.28634644)
\curveto(803.61727975,338.29633855)(803.69227967,338.30633854)(803.77228271,338.31634644)
\curveto(803.80227956,338.32633852)(803.84227952,338.32633852)(803.89228271,338.31634644)
\curveto(803.94227942,338.30633854)(803.97727939,338.31133854)(803.99728271,338.33134644)
}
}
{
\newrgbcolor{curcolor}{0 0 0}
\pscustom[linestyle=none,fillstyle=solid,fillcolor=curcolor]
{
\newpath
\moveto(816.19228271,334.62634644)
\curveto(816.21227465,334.56634228)(816.22227464,334.47134238)(816.22228271,334.34134644)
\curveto(816.22227464,334.22134263)(816.21727465,334.13634271)(816.20728271,334.08634644)
\lineto(816.20728271,333.93634644)
\curveto(816.19727467,333.85634299)(816.18727468,333.78134307)(816.17728271,333.71134644)
\curveto(816.17727469,333.6513432)(816.17227469,333.58134327)(816.16228271,333.50134644)
\curveto(816.14227472,333.44134341)(816.12727474,333.38134347)(816.11728271,333.32134644)
\curveto(816.11727475,333.26134359)(816.10727476,333.20134365)(816.08728271,333.14134644)
\curveto(816.04727482,333.01134384)(816.01227485,332.88134397)(815.98228271,332.75134644)
\curveto(815.95227491,332.62134423)(815.91227495,332.50134435)(815.86228271,332.39134644)
\curveto(815.65227521,331.91134494)(815.37227549,331.50634534)(815.02228271,331.17634644)
\curveto(814.67227619,330.85634599)(814.24227662,330.61134624)(813.73228271,330.44134644)
\curveto(813.62227724,330.40134645)(813.50227736,330.37134648)(813.37228271,330.35134644)
\curveto(813.25227761,330.33134652)(813.12727774,330.31134654)(812.99728271,330.29134644)
\curveto(812.93727793,330.28134657)(812.87227799,330.27634657)(812.80228271,330.27634644)
\curveto(812.74227812,330.26634658)(812.68227818,330.26134659)(812.62228271,330.26134644)
\curveto(812.58227828,330.2513466)(812.52227834,330.2463466)(812.44228271,330.24634644)
\curveto(812.37227849,330.2463466)(812.32227854,330.2513466)(812.29228271,330.26134644)
\curveto(812.25227861,330.27134658)(812.21227865,330.27634657)(812.17228271,330.27634644)
\curveto(812.13227873,330.26634658)(812.09727877,330.26634658)(812.06728271,330.27634644)
\lineto(811.97728271,330.27634644)
\lineto(811.61728271,330.32134644)
\curveto(811.47727939,330.36134649)(811.34227952,330.40134645)(811.21228271,330.44134644)
\curveto(811.08227978,330.48134637)(810.95727991,330.52634632)(810.83728271,330.57634644)
\curveto(810.38728048,330.77634607)(810.01728085,331.03634581)(809.72728271,331.35634644)
\curveto(809.43728143,331.67634517)(809.19728167,332.06634478)(809.00728271,332.52634644)
\curveto(808.95728191,332.62634422)(808.91728195,332.72634412)(808.88728271,332.82634644)
\curveto(808.867282,332.92634392)(808.84728202,333.03134382)(808.82728271,333.14134644)
\curveto(808.80728206,333.18134367)(808.79728207,333.21134364)(808.79728271,333.23134644)
\curveto(808.80728206,333.26134359)(808.80728206,333.29634355)(808.79728271,333.33634644)
\curveto(808.77728209,333.41634343)(808.7622821,333.49634335)(808.75228271,333.57634644)
\curveto(808.75228211,333.66634318)(808.74228212,333.7513431)(808.72228271,333.83134644)
\lineto(808.72228271,333.95134644)
\curveto(808.72228214,333.99134286)(808.71728215,334.03634281)(808.70728271,334.08634644)
\curveto(808.69728217,334.13634271)(808.69228217,334.22134263)(808.69228271,334.34134644)
\curveto(808.69228217,334.47134238)(808.70228216,334.56634228)(808.72228271,334.62634644)
\curveto(808.74228212,334.69634215)(808.74728212,334.76634208)(808.73728271,334.83634644)
\curveto(808.72728214,334.90634194)(808.73228213,334.97634187)(808.75228271,335.04634644)
\curveto(808.7622821,335.09634175)(808.7672821,335.13634171)(808.76728271,335.16634644)
\curveto(808.77728209,335.20634164)(808.78728208,335.2513416)(808.79728271,335.30134644)
\curveto(808.82728204,335.42134143)(808.85228201,335.54134131)(808.87228271,335.66134644)
\curveto(808.90228196,335.78134107)(808.94228192,335.89634095)(808.99228271,336.00634644)
\curveto(809.14228172,336.37634047)(809.32228154,336.70634014)(809.53228271,336.99634644)
\curveto(809.75228111,337.29633955)(810.01728085,337.5463393)(810.32728271,337.74634644)
\curveto(810.44728042,337.82633902)(810.57228029,337.89133896)(810.70228271,337.94134644)
\curveto(810.83228003,338.00133885)(810.9672799,338.06133879)(811.10728271,338.12134644)
\curveto(811.22727964,338.17133868)(811.35727951,338.20133865)(811.49728271,338.21134644)
\curveto(811.63727923,338.23133862)(811.77727909,338.26133859)(811.91728271,338.30134644)
\lineto(812.11228271,338.30134644)
\curveto(812.18227868,338.31133854)(812.24727862,338.32133853)(812.30728271,338.33134644)
\curveto(813.19727767,338.34133851)(813.93727693,338.15633869)(814.52728271,337.77634644)
\curveto(815.11727575,337.39633945)(815.54227532,336.90133995)(815.80228271,336.29134644)
\curveto(815.85227501,336.19134066)(815.89227497,336.09134076)(815.92228271,335.99134644)
\curveto(815.95227491,335.89134096)(815.98727488,335.78634106)(816.02728271,335.67634644)
\curveto(816.05727481,335.56634128)(816.08227478,335.4463414)(816.10228271,335.31634644)
\curveto(816.12227474,335.19634165)(816.14727472,335.07134178)(816.17728271,334.94134644)
\curveto(816.18727468,334.89134196)(816.18727468,334.83634201)(816.17728271,334.77634644)
\curveto(816.17727469,334.72634212)(816.18227468,334.67634217)(816.19228271,334.62634644)
\moveto(814.85728271,333.77134644)
\curveto(814.87727599,333.84134301)(814.88227598,333.92134293)(814.87228271,334.01134644)
\lineto(814.87228271,334.26634644)
\curveto(814.87227599,334.65634219)(814.83727603,334.98634186)(814.76728271,335.25634644)
\curveto(814.73727613,335.33634151)(814.71227615,335.41634143)(814.69228271,335.49634644)
\curveto(814.67227619,335.57634127)(814.64727622,335.6513412)(814.61728271,335.72134644)
\curveto(814.33727653,336.37134048)(813.89227697,336.82134003)(813.28228271,337.07134644)
\curveto(813.21227765,337.10133975)(813.13727773,337.12133973)(813.05728271,337.13134644)
\lineto(812.81728271,337.19134644)
\curveto(812.73727813,337.21133964)(812.65227821,337.22133963)(812.56228271,337.22134644)
\lineto(812.29228271,337.22134644)
\lineto(812.02228271,337.17634644)
\curveto(811.92227894,337.15633969)(811.82727904,337.13133972)(811.73728271,337.10134644)
\curveto(811.65727921,337.08133977)(811.57727929,337.0513398)(811.49728271,337.01134644)
\curveto(811.42727944,336.99133986)(811.3622795,336.96133989)(811.30228271,336.92134644)
\curveto(811.24227962,336.88133997)(811.18727968,336.84134001)(811.13728271,336.80134644)
\curveto(810.89727997,336.63134022)(810.70228016,336.42634042)(810.55228271,336.18634644)
\curveto(810.40228046,335.9463409)(810.27228059,335.66634118)(810.16228271,335.34634644)
\curveto(810.13228073,335.2463416)(810.11228075,335.14134171)(810.10228271,335.03134644)
\curveto(810.09228077,334.93134192)(810.07728079,334.82634202)(810.05728271,334.71634644)
\curveto(810.04728082,334.67634217)(810.04228082,334.61134224)(810.04228271,334.52134644)
\curveto(810.03228083,334.49134236)(810.02728084,334.45634239)(810.02728271,334.41634644)
\curveto(810.03728083,334.37634247)(810.04228082,334.33134252)(810.04228271,334.28134644)
\lineto(810.04228271,333.98134644)
\curveto(810.04228082,333.88134297)(810.05228081,333.79134306)(810.07228271,333.71134644)
\lineto(810.10228271,333.53134644)
\curveto(810.12228074,333.43134342)(810.13728073,333.33134352)(810.14728271,333.23134644)
\curveto(810.1672807,333.14134371)(810.19728067,333.05634379)(810.23728271,332.97634644)
\curveto(810.33728053,332.73634411)(810.45228041,332.51134434)(810.58228271,332.30134644)
\curveto(810.72228014,332.09134476)(810.89227997,331.91634493)(811.09228271,331.77634644)
\curveto(811.14227972,331.7463451)(811.18727968,331.72134513)(811.22728271,331.70134644)
\curveto(811.2672796,331.68134517)(811.31227955,331.65634519)(811.36228271,331.62634644)
\curveto(811.44227942,331.57634527)(811.52727934,331.53134532)(811.61728271,331.49134644)
\curveto(811.71727915,331.46134539)(811.82227904,331.43134542)(811.93228271,331.40134644)
\curveto(811.98227888,331.38134547)(812.02727884,331.37134548)(812.06728271,331.37134644)
\curveto(812.11727875,331.38134547)(812.1672787,331.38134547)(812.21728271,331.37134644)
\curveto(812.24727862,331.36134549)(812.30727856,331.3513455)(812.39728271,331.34134644)
\curveto(812.49727837,331.33134552)(812.57227829,331.33634551)(812.62228271,331.35634644)
\curveto(812.6622782,331.36634548)(812.70227816,331.36634548)(812.74228271,331.35634644)
\curveto(812.78227808,331.35634549)(812.82227804,331.36634548)(812.86228271,331.38634644)
\curveto(812.94227792,331.40634544)(813.02227784,331.42134543)(813.10228271,331.43134644)
\curveto(813.18227768,331.4513454)(813.25727761,331.47634537)(813.32728271,331.50634644)
\curveto(813.6672772,331.6463452)(813.94227692,331.84134501)(814.15228271,332.09134644)
\curveto(814.3622765,332.34134451)(814.53727633,332.63634421)(814.67728271,332.97634644)
\curveto(814.72727614,333.09634375)(814.75727611,333.22134363)(814.76728271,333.35134644)
\curveto(814.78727608,333.49134336)(814.81727605,333.63134322)(814.85728271,333.77134644)
}
}
{
\newrgbcolor{curcolor}{0 0 0}
\pscustom[linestyle=none,fillstyle=solid,fillcolor=curcolor]
{
\newpath
\moveto(820.11056396,338.33134644)
\curveto(820.8305599,338.34133851)(821.43555929,338.25633859)(821.92556396,338.07634644)
\curveto(822.41555831,337.90633894)(822.79555793,337.60133925)(823.06556396,337.16134644)
\curveto(823.13555759,337.0513398)(823.19055754,336.93633991)(823.23056396,336.81634644)
\curveto(823.27055746,336.70634014)(823.31055742,336.58134027)(823.35056396,336.44134644)
\curveto(823.37055736,336.37134048)(823.37555735,336.29634055)(823.36556396,336.21634644)
\curveto(823.35555737,336.1463407)(823.34055739,336.09134076)(823.32056396,336.05134644)
\curveto(823.30055743,336.03134082)(823.27555745,336.01134084)(823.24556396,335.99134644)
\curveto(823.21555751,335.98134087)(823.19055754,335.96634088)(823.17056396,335.94634644)
\curveto(823.12055761,335.92634092)(823.07055766,335.92134093)(823.02056396,335.93134644)
\curveto(822.97055776,335.94134091)(822.92055781,335.94134091)(822.87056396,335.93134644)
\curveto(822.79055794,335.91134094)(822.68555804,335.90634094)(822.55556396,335.91634644)
\curveto(822.4255583,335.93634091)(822.33555839,335.96134089)(822.28556396,335.99134644)
\curveto(822.20555852,336.04134081)(822.15055858,336.10634074)(822.12056396,336.18634644)
\curveto(822.10055863,336.27634057)(822.06555866,336.36134049)(822.01556396,336.44134644)
\curveto(821.9255588,336.60134025)(821.80055893,336.7463401)(821.64056396,336.87634644)
\curveto(821.5305592,336.95633989)(821.41055932,337.01633983)(821.28056396,337.05634644)
\curveto(821.15055958,337.09633975)(821.01055972,337.13633971)(820.86056396,337.17634644)
\curveto(820.81055992,337.19633965)(820.76055997,337.20133965)(820.71056396,337.19134644)
\curveto(820.66056007,337.19133966)(820.61056012,337.19633965)(820.56056396,337.20634644)
\curveto(820.50056023,337.22633962)(820.4255603,337.23633961)(820.33556396,337.23634644)
\curveto(820.24556048,337.23633961)(820.17056056,337.22633962)(820.11056396,337.20634644)
\lineto(820.02056396,337.20634644)
\lineto(819.87056396,337.17634644)
\curveto(819.82056091,337.17633967)(819.77056096,337.17133968)(819.72056396,337.16134644)
\curveto(819.46056127,337.10133975)(819.24556148,337.01633983)(819.07556396,336.90634644)
\curveto(818.90556182,336.79634005)(818.79056194,336.61134024)(818.73056396,336.35134644)
\curveto(818.71056202,336.28134057)(818.70556202,336.21134064)(818.71556396,336.14134644)
\curveto(818.73556199,336.07134078)(818.75556197,336.01134084)(818.77556396,335.96134644)
\curveto(818.83556189,335.81134104)(818.90556182,335.70134115)(818.98556396,335.63134644)
\curveto(819.07556165,335.57134128)(819.18556154,335.50134135)(819.31556396,335.42134644)
\curveto(819.47556125,335.32134153)(819.65556107,335.2463416)(819.85556396,335.19634644)
\curveto(820.05556067,335.15634169)(820.25556047,335.10634174)(820.45556396,335.04634644)
\curveto(820.58556014,335.00634184)(820.71556001,334.97634187)(820.84556396,334.95634644)
\curveto(820.97555975,334.93634191)(821.10555962,334.90634194)(821.23556396,334.86634644)
\curveto(821.44555928,334.80634204)(821.65055908,334.7463421)(821.85056396,334.68634644)
\curveto(822.05055868,334.63634221)(822.25055848,334.57134228)(822.45056396,334.49134644)
\lineto(822.60056396,334.43134644)
\curveto(822.65055808,334.41134244)(822.70055803,334.38634246)(822.75056396,334.35634644)
\curveto(822.95055778,334.23634261)(823.1255576,334.10134275)(823.27556396,333.95134644)
\curveto(823.4255573,333.80134305)(823.55055718,333.61134324)(823.65056396,333.38134644)
\curveto(823.67055706,333.31134354)(823.69055704,333.21634363)(823.71056396,333.09634644)
\curveto(823.730557,333.02634382)(823.74055699,332.9513439)(823.74056396,332.87134644)
\curveto(823.75055698,332.80134405)(823.75555697,332.72134413)(823.75556396,332.63134644)
\lineto(823.75556396,332.48134644)
\curveto(823.73555699,332.41134444)(823.725557,332.34134451)(823.72556396,332.27134644)
\curveto(823.725557,332.20134465)(823.71555701,332.13134472)(823.69556396,332.06134644)
\curveto(823.66555706,331.9513449)(823.6305571,331.846345)(823.59056396,331.74634644)
\curveto(823.55055718,331.6463452)(823.50555722,331.55634529)(823.45556396,331.47634644)
\curveto(823.29555743,331.21634563)(823.09055764,331.00634584)(822.84056396,330.84634644)
\curveto(822.59055814,330.69634615)(822.31055842,330.56634628)(822.00056396,330.45634644)
\curveto(821.91055882,330.42634642)(821.81555891,330.40634644)(821.71556396,330.39634644)
\curveto(821.6255591,330.37634647)(821.53555919,330.3513465)(821.44556396,330.32134644)
\curveto(821.34555938,330.30134655)(821.24555948,330.29134656)(821.14556396,330.29134644)
\curveto(821.04555968,330.29134656)(820.94555978,330.28134657)(820.84556396,330.26134644)
\lineto(820.69556396,330.26134644)
\curveto(820.64556008,330.2513466)(820.57556015,330.2463466)(820.48556396,330.24634644)
\curveto(820.39556033,330.2463466)(820.3255604,330.2513466)(820.27556396,330.26134644)
\lineto(820.11056396,330.26134644)
\curveto(820.05056068,330.28134657)(819.98556074,330.29134656)(819.91556396,330.29134644)
\curveto(819.84556088,330.28134657)(819.78556094,330.28634656)(819.73556396,330.30634644)
\curveto(819.68556104,330.31634653)(819.62056111,330.32134653)(819.54056396,330.32134644)
\lineto(819.30056396,330.38134644)
\curveto(819.2305615,330.39134646)(819.15556157,330.41134644)(819.07556396,330.44134644)
\curveto(818.76556196,330.54134631)(818.49556223,330.66634618)(818.26556396,330.81634644)
\curveto(818.03556269,330.96634588)(817.83556289,331.16134569)(817.66556396,331.40134644)
\curveto(817.57556315,331.53134532)(817.50056323,331.66634518)(817.44056396,331.80634644)
\curveto(817.38056335,331.9463449)(817.3255634,332.10134475)(817.27556396,332.27134644)
\curveto(817.25556347,332.33134452)(817.24556348,332.40134445)(817.24556396,332.48134644)
\curveto(817.25556347,332.57134428)(817.27056346,332.64134421)(817.29056396,332.69134644)
\curveto(817.32056341,332.73134412)(817.37056336,332.77134408)(817.44056396,332.81134644)
\curveto(817.49056324,332.83134402)(817.56056317,332.84134401)(817.65056396,332.84134644)
\curveto(817.74056299,332.851344)(817.8305629,332.851344)(817.92056396,332.84134644)
\curveto(818.01056272,332.83134402)(818.09556263,332.81634403)(818.17556396,332.79634644)
\curveto(818.26556246,332.78634406)(818.3255624,332.77134408)(818.35556396,332.75134644)
\curveto(818.4255623,332.70134415)(818.47056226,332.62634422)(818.49056396,332.52634644)
\curveto(818.52056221,332.43634441)(818.55556217,332.3513445)(818.59556396,332.27134644)
\curveto(818.69556203,332.0513448)(818.8305619,331.88134497)(819.00056396,331.76134644)
\curveto(819.12056161,331.67134518)(819.25556147,331.60134525)(819.40556396,331.55134644)
\curveto(819.55556117,331.50134535)(819.71556101,331.4513454)(819.88556396,331.40134644)
\lineto(820.20056396,331.35634644)
\lineto(820.29056396,331.35634644)
\curveto(820.36056037,331.33634551)(820.45056028,331.32634552)(820.56056396,331.32634644)
\curveto(820.68056005,331.32634552)(820.78055995,331.33634551)(820.86056396,331.35634644)
\curveto(820.9305598,331.35634549)(820.98555974,331.36134549)(821.02556396,331.37134644)
\curveto(821.08555964,331.38134547)(821.14555958,331.38634546)(821.20556396,331.38634644)
\curveto(821.26555946,331.39634545)(821.32055941,331.40634544)(821.37056396,331.41634644)
\curveto(821.66055907,331.49634535)(821.89055884,331.60134525)(822.06056396,331.73134644)
\curveto(822.2305585,331.86134499)(822.35055838,332.08134477)(822.42056396,332.39134644)
\curveto(822.44055829,332.44134441)(822.44555828,332.49634435)(822.43556396,332.55634644)
\curveto(822.4255583,332.61634423)(822.41555831,332.66134419)(822.40556396,332.69134644)
\curveto(822.35555837,332.88134397)(822.28555844,333.02134383)(822.19556396,333.11134644)
\curveto(822.10555862,333.21134364)(821.99055874,333.30134355)(821.85056396,333.38134644)
\curveto(821.76055897,333.44134341)(821.66055907,333.49134336)(821.55056396,333.53134644)
\lineto(821.22056396,333.65134644)
\curveto(821.19055954,333.66134319)(821.16055957,333.66634318)(821.13056396,333.66634644)
\curveto(821.11055962,333.66634318)(821.08555964,333.67634317)(821.05556396,333.69634644)
\curveto(820.71556001,333.80634304)(820.36056037,333.88634296)(819.99056396,333.93634644)
\curveto(819.6305611,333.99634285)(819.29056144,334.09134276)(818.97056396,334.22134644)
\curveto(818.87056186,334.26134259)(818.77556195,334.29634255)(818.68556396,334.32634644)
\curveto(818.59556213,334.35634249)(818.51056222,334.39634245)(818.43056396,334.44634644)
\curveto(818.24056249,334.55634229)(818.06556266,334.68134217)(817.90556396,334.82134644)
\curveto(817.74556298,334.96134189)(817.62056311,335.13634171)(817.53056396,335.34634644)
\curveto(817.50056323,335.41634143)(817.47556325,335.48634136)(817.45556396,335.55634644)
\curveto(817.44556328,335.62634122)(817.4305633,335.70134115)(817.41056396,335.78134644)
\curveto(817.38056335,335.90134095)(817.37056336,336.03634081)(817.38056396,336.18634644)
\curveto(817.39056334,336.3463405)(817.40556332,336.48134037)(817.42556396,336.59134644)
\curveto(817.44556328,336.64134021)(817.45556327,336.68134017)(817.45556396,336.71134644)
\curveto(817.46556326,336.7513401)(817.48056325,336.79134006)(817.50056396,336.83134644)
\curveto(817.59056314,337.06133979)(817.71056302,337.26133959)(817.86056396,337.43134644)
\curveto(818.02056271,337.60133925)(818.20056253,337.7513391)(818.40056396,337.88134644)
\curveto(818.55056218,337.97133888)(818.71556201,338.04133881)(818.89556396,338.09134644)
\curveto(819.07556165,338.1513387)(819.26556146,338.20633864)(819.46556396,338.25634644)
\curveto(819.53556119,338.26633858)(819.60056113,338.27633857)(819.66056396,338.28634644)
\curveto(819.730561,338.29633855)(819.80556092,338.30633854)(819.88556396,338.31634644)
\curveto(819.91556081,338.32633852)(819.95556077,338.32633852)(820.00556396,338.31634644)
\curveto(820.05556067,338.30633854)(820.09056064,338.31133854)(820.11056396,338.33134644)
}
}
{
\newrgbcolor{curcolor}{0.80000001 0.80000001 0.80000001}
\pscustom[linestyle=none,fillstyle=solid,fillcolor=curcolor]
{
\newpath
\moveto(741.9732666,382.02397156)
\lineto(756.9732666,382.02397156)
\lineto(756.9732666,367.02397156)
\lineto(741.9732666,367.02397156)
\closepath
}
}
{
\newrgbcolor{curcolor}{0.7019608 0.7019608 0.7019608}
\pscustom[linestyle=none,fillstyle=solid,fillcolor=curcolor]
{
\newpath
\moveto(741.9732666,340.94264221)
\lineto(756.9732666,340.94264221)
\lineto(756.9732666,325.94264221)
\lineto(741.9732666,325.94264221)
\closepath
}
}
{
\newrgbcolor{curcolor}{0.80000001 0.80000001 0.80000001}
\pscustom[linestyle=none,fillstyle=solid,fillcolor=curcolor]
{
\newpath
\moveto(106.02189636,92.00222778)
\lineto(133.94314194,92.00222778)
\lineto(133.94314194,86.0827961)
\lineto(106.02189636,86.0827961)
\closepath
}
}
{
\newrgbcolor{curcolor}{0 0 0}
\pscustom[linestyle=none,fillstyle=solid,fillcolor=curcolor]
{
\newpath
\moveto(567.4097168,57.195)
\lineto(572.3147168,57.195)
\lineto(573.6047168,57.195)
\curveto(573.71470892,57.1949893)(573.82470881,57.1949893)(573.9347168,57.195)
\curveto(574.04470859,57.2049893)(574.1347085,57.18498931)(574.2047168,57.135)
\curveto(574.2347084,57.11498939)(574.25970837,57.08998941)(574.2797168,57.06)
\curveto(574.29970833,57.02998947)(574.31970831,56.9999895)(574.3397168,56.97)
\curveto(574.35970827,56.8999896)(574.36970826,56.78498971)(574.3697168,56.625)
\curveto(574.36970826,56.47499003)(574.35970827,56.35999014)(574.3397168,56.28)
\curveto(574.29970833,56.13999036)(574.21470842,56.05999044)(574.0847168,56.04)
\curveto(573.95470868,56.02999047)(573.79970883,56.02499047)(573.6197168,56.025)
\lineto(572.1197168,56.025)
\lineto(569.5997168,56.025)
\lineto(569.0297168,56.025)
\curveto(568.81971381,56.03499047)(568.66471397,56.00999049)(568.5647168,55.95)
\curveto(568.46471417,55.88999061)(568.40971422,55.78499071)(568.3997168,55.635)
\lineto(568.3997168,55.17)
\lineto(568.3997168,53.64)
\curveto(568.39971423,53.52999297)(568.39471424,53.3999931)(568.3847168,53.25)
\curveto(568.38471425,53.0999934)(568.39471424,52.97999352)(568.4147168,52.89)
\curveto(568.44471419,52.76999373)(568.50471413,52.68999381)(568.5947168,52.65)
\curveto(568.634714,52.62999387)(568.70471393,52.60999389)(568.8047168,52.59)
\lineto(568.9547168,52.59)
\curveto(568.99471364,52.57999392)(569.0347136,52.57499393)(569.0747168,52.575)
\curveto(569.12471351,52.58499391)(569.17471346,52.58999391)(569.2247168,52.59)
\lineto(569.7347168,52.59)
\lineto(572.6747168,52.59)
\lineto(572.9747168,52.59)
\curveto(573.08470955,52.5999939)(573.19470944,52.5999939)(573.3047168,52.59)
\curveto(573.42470921,52.58999391)(573.5297091,52.57999392)(573.6197168,52.56)
\curveto(573.71970891,52.54999395)(573.79470884,52.52999397)(573.8447168,52.5)
\curveto(573.87470876,52.47999402)(573.89970873,52.43499407)(573.9197168,52.365)
\curveto(573.93970869,52.29499421)(573.95470868,52.21999428)(573.9647168,52.14)
\curveto(573.97470866,52.05999444)(573.97470866,51.97499452)(573.9647168,51.885)
\curveto(573.96470867,51.8049947)(573.95470868,51.73499477)(573.9347168,51.675)
\curveto(573.91470872,51.58499491)(573.86970876,51.51999498)(573.7997168,51.48)
\curveto(573.77970885,51.45999504)(573.74970888,51.44499505)(573.7097168,51.435)
\curveto(573.67970895,51.43499507)(573.64970898,51.42999507)(573.6197168,51.42)
\lineto(573.5297168,51.42)
\curveto(573.47970915,51.40999509)(573.4297092,51.4049951)(573.3797168,51.405)
\curveto(573.3297093,51.41499508)(573.27970935,51.41999508)(573.2297168,51.42)
\lineto(572.6747168,51.42)
\lineto(569.5097168,51.42)
\lineto(569.1497168,51.42)
\curveto(569.03971359,51.42999507)(568.9347137,51.42499508)(568.8347168,51.405)
\curveto(568.7347139,51.39499511)(568.64471399,51.36999513)(568.5647168,51.33)
\curveto(568.49471414,51.28999521)(568.44471419,51.21999528)(568.4147168,51.12)
\curveto(568.39471424,51.05999544)(568.38471425,50.98999551)(568.3847168,50.91)
\curveto(568.39471424,50.82999567)(568.39971423,50.74999575)(568.3997168,50.67)
\lineto(568.3997168,49.83)
\lineto(568.3997168,48.405)
\curveto(568.39971423,48.26499824)(568.40471423,48.13499837)(568.4147168,48.015)
\curveto(568.42471421,47.90499859)(568.46471417,47.82499868)(568.5347168,47.775)
\curveto(568.60471403,47.72499878)(568.68471395,47.69499881)(568.7747168,47.685)
\lineto(569.0747168,47.685)
\lineto(570.0347168,47.685)
\lineto(572.8097168,47.685)
\lineto(573.6647168,47.685)
\lineto(573.9047168,47.685)
\curveto(573.98470865,47.69499881)(574.05470858,47.68999881)(574.1147168,47.67)
\curveto(574.2347084,47.62999887)(574.31470832,47.57499892)(574.3547168,47.505)
\curveto(574.37470826,47.47499902)(574.38970824,47.42499908)(574.3997168,47.355)
\curveto(574.40970822,47.28499922)(574.41470822,47.20999929)(574.4147168,47.13)
\curveto(574.42470821,47.05999944)(574.42470821,46.98499952)(574.4147168,46.905)
\curveto(574.40470823,46.83499966)(574.39470824,46.77999972)(574.3847168,46.74)
\curveto(574.34470829,46.65999984)(574.29970833,46.60499989)(574.2497168,46.575)
\curveto(574.18970844,46.53499996)(574.10970852,46.51499998)(574.0097168,46.515)
\lineto(573.7397168,46.515)
\lineto(572.6897168,46.515)
\lineto(568.6997168,46.515)
\lineto(567.6497168,46.515)
\curveto(567.50971512,46.51499998)(567.38971524,46.51999998)(567.2897168,46.53)
\curveto(567.18971544,46.54999995)(567.11471552,46.5999999)(567.0647168,46.68)
\curveto(567.02471561,46.73999976)(567.00471563,46.81499968)(567.0047168,46.905)
\lineto(567.0047168,47.19)
\lineto(567.0047168,48.24)
\lineto(567.0047168,52.26)
\lineto(567.0047168,55.62)
\lineto(567.0047168,56.55)
\lineto(567.0047168,56.82)
\curveto(567.00471563,56.90998959)(567.02471561,56.97998952)(567.0647168,57.03)
\curveto(567.10471553,57.0999894)(567.17971545,57.14998935)(567.2897168,57.18)
\curveto(567.30971532,57.18998931)(567.3297153,57.18998931)(567.3497168,57.18)
\curveto(567.36971526,57.17998932)(567.38971524,57.18498931)(567.4097168,57.195)
}
}
{
\newrgbcolor{curcolor}{0 0 0}
\pscustom[linestyle=none,fillstyle=solid,fillcolor=curcolor]
{
\newpath
\moveto(582.87963867,44.31)
\lineto(582.87963867,43.98)
\curveto(582.88963079,43.87000263)(582.86963081,43.78500272)(582.81963867,43.725)
\curveto(582.79963088,43.69500281)(582.7746309,43.67000283)(582.74463867,43.65)
\lineto(582.65463867,43.59)
\curveto(582.62463105,43.58000292)(582.5746311,43.57500292)(582.50463867,43.575)
\curveto(582.43463124,43.56500293)(582.35963132,43.56000294)(582.27963867,43.56)
\curveto(582.20963147,43.56000294)(582.13963154,43.56500293)(582.06963867,43.575)
\curveto(581.99963168,43.57500292)(581.94963173,43.58000292)(581.91963867,43.59)
\curveto(581.81963186,43.61000289)(581.74963193,43.65500285)(581.70963867,43.725)
\curveto(581.65963202,43.80500269)(581.63463204,43.93000257)(581.63463867,44.1)
\lineto(581.63463867,44.52)
\lineto(581.63463867,46.365)
\lineto(581.63463867,46.725)
\curveto(581.64463203,46.86499964)(581.62963205,46.97999952)(581.58963867,47.07)
\curveto(581.56963211,47.08999941)(581.54963213,47.10499939)(581.52963867,47.115)
\curveto(581.51963216,47.13499936)(581.50463217,47.15499935)(581.48463867,47.175)
\curveto(581.38463229,47.17499932)(581.30463237,47.14999935)(581.24463867,47.1)
\lineto(581.09463867,46.95)
\curveto(581.01463266,46.88999961)(580.92963275,46.82999967)(580.83963867,46.77)
\curveto(580.74963293,46.71999978)(580.64963303,46.66999983)(580.53963867,46.62)
\curveto(580.38963329,46.55999994)(580.21463346,46.50999999)(580.01463867,46.47)
\curveto(579.82463385,46.42000008)(579.61963406,46.39000011)(579.39963867,46.38)
\curveto(579.18963449,46.36000014)(578.9746347,46.36000014)(578.75463867,46.38)
\curveto(578.54463513,46.39000011)(578.34963533,46.42000008)(578.16963867,46.47)
\curveto(578.11963556,46.49000001)(578.06963561,46.50499999)(578.01963867,46.515)
\curveto(577.96963571,46.52499998)(577.91963576,46.53999996)(577.86963867,46.56)
\curveto(577.7796359,46.5999999)(577.68963599,46.63499986)(577.59963867,46.665)
\curveto(577.50963617,46.70499979)(577.42463625,46.74999975)(577.34463867,46.8)
\curveto(576.99463668,47.01999948)(576.69963698,47.26999923)(576.45963867,47.55)
\curveto(576.22963745,47.83999866)(576.02963765,48.19499831)(575.85963867,48.615)
\curveto(575.81963786,48.71499778)(575.78463789,48.81999768)(575.75463867,48.93)
\curveto(575.73463794,49.03999746)(575.70963797,49.14999735)(575.67963867,49.26)
\curveto(575.66963801,49.27999722)(575.66463801,49.2999972)(575.66463867,49.32)
\curveto(575.66463801,49.34999715)(575.65963802,49.37999712)(575.64963867,49.41)
\curveto(575.62963805,49.48999701)(575.61463806,49.57999692)(575.60463867,49.68)
\curveto(575.60463807,49.77999672)(575.59463808,49.87499662)(575.57463867,49.965)
\lineto(575.57463867,50.22)
\curveto(575.55463812,50.26999623)(575.54463813,50.33499617)(575.54463867,50.415)
\curveto(575.54463813,50.49499601)(575.55463812,50.55999594)(575.57463867,50.61)
\lineto(575.57463867,50.775)
\curveto(575.5746381,50.83499567)(575.5796381,50.89499561)(575.58963867,50.955)
\curveto(575.59963808,50.99499551)(575.59963808,51.03499547)(575.58963867,51.075)
\curveto(575.58963809,51.11499538)(575.59463808,51.15999534)(575.60463867,51.21)
\curveto(575.63463804,51.31999518)(575.65463802,51.42499508)(575.66463867,51.525)
\curveto(575.68463799,51.63499487)(575.70963797,51.73999476)(575.73963867,51.84)
\curveto(575.7796379,51.96999453)(575.81963786,52.08999441)(575.85963867,52.2)
\curveto(575.89963778,52.31999418)(575.94463773,52.43499407)(575.99463867,52.545)
\curveto(576.06463761,52.68499381)(576.13963754,52.81499368)(576.21963867,52.935)
\curveto(576.30963737,53.06499344)(576.39963728,53.18999331)(576.48963867,53.31)
\curveto(576.49963718,53.30999319)(576.51463716,53.31999318)(576.53463867,53.34)
\curveto(576.58463709,53.41999308)(576.65963702,53.499993)(576.75963867,53.58)
\curveto(576.76963691,53.58999291)(576.7746369,53.5999929)(576.77463867,53.61)
\curveto(576.78463689,53.61999288)(576.79963688,53.62999287)(576.81963867,53.64)
\curveto(576.85963682,53.66999283)(576.89463678,53.6999928)(576.92463867,53.73)
\curveto(576.96463671,53.76999273)(577.00963667,53.8049927)(577.05963867,53.835)
\curveto(577.19963648,53.94499255)(577.35463632,54.03499247)(577.52463867,54.105)
\curveto(577.69463598,54.17499233)(577.8746358,54.23999226)(578.06463867,54.3)
\curveto(578.16463551,54.33999216)(578.26963541,54.36499214)(578.37963867,54.375)
\curveto(578.48963519,54.38499211)(578.59963508,54.3999921)(578.70963867,54.42)
\curveto(578.74963493,54.42999207)(578.80463487,54.42999207)(578.87463867,54.42)
\curveto(578.95463472,54.40999209)(579.00463467,54.41499208)(579.02463867,54.435)
\curveto(579.35463432,54.43499207)(579.674634,54.39499211)(579.98463867,54.315)
\curveto(580.29463338,54.23499227)(580.54963313,54.13499237)(580.74963867,54.015)
\lineto(580.92963867,53.895)
\curveto(580.98963269,53.85499264)(581.04963263,53.80999269)(581.10963867,53.76)
\lineto(581.25963867,53.64)
\curveto(581.30963237,53.5999929)(581.38463229,53.57999292)(581.48463867,53.58)
\curveto(581.50463217,53.5999929)(581.52463215,53.61499288)(581.54463867,53.625)
\curveto(581.56463211,53.64499285)(581.5796321,53.66999283)(581.58963867,53.7)
\curveto(581.61963206,53.76999273)(581.63463204,53.84499265)(581.63463867,53.925)
\curveto(581.64463203,54.0049925)(581.674632,54.06999243)(581.72463867,54.12)
\curveto(581.75463192,54.15999234)(581.81463186,54.18999231)(581.90463867,54.21)
\curveto(582.00463167,54.23999226)(582.10963157,54.25499224)(582.21963867,54.255)
\curveto(582.32963135,54.26499224)(582.43463124,54.25999224)(582.53463867,54.24)
\curveto(582.63463104,54.21999228)(582.70963097,54.19499231)(582.75963867,54.165)
\curveto(582.82963085,54.11499238)(582.86463081,54.02999247)(582.86463867,53.91)
\curveto(582.8746308,53.78999271)(582.8796308,53.66999283)(582.87963867,53.55)
\lineto(582.87963867,44.31)
\moveto(581.66463867,50.175)
\curveto(581.674632,50.22499628)(581.679632,50.29499621)(581.67963867,50.385)
\curveto(581.68963199,50.47499602)(581.68463199,50.54499595)(581.66463867,50.595)
\lineto(581.66463867,50.805)
\lineto(581.60463867,51.105)
\curveto(581.59463208,51.2049953)(581.5796321,51.29499521)(581.55963867,51.375)
\curveto(581.53963214,51.45499504)(581.51963216,51.52499498)(581.49963867,51.585)
\curveto(581.48963219,51.65499484)(581.46963221,51.72499478)(581.43963867,51.795)
\curveto(581.32963235,52.06499444)(581.15463252,52.32999417)(580.91463867,52.59)
\curveto(580.674633,52.84999365)(580.44463323,53.02999347)(580.22463867,53.13)
\curveto(580.14463353,53.16999333)(580.05963362,53.1999933)(579.96963867,53.22)
\curveto(579.88963379,53.23999326)(579.80463387,53.26499324)(579.71463867,53.295)
\curveto(579.61463406,53.31499318)(579.50463417,53.32499317)(579.38463867,53.325)
\lineto(579.03963867,53.325)
\lineto(578.88963867,53.295)
\lineto(578.75463867,53.295)
\lineto(578.51463867,53.235)
\curveto(578.43463524,53.21499328)(578.35963532,53.18499331)(578.28963867,53.145)
\curveto(577.96963571,53.0049935)(577.70963597,52.8049937)(577.50963867,52.545)
\curveto(577.31963636,52.28499421)(577.16963651,51.97999452)(577.05963867,51.63)
\curveto(577.01963666,51.51999498)(576.98963669,51.3999951)(576.96963867,51.27)
\curveto(576.95963672,51.14999535)(576.93963674,51.02499548)(576.90963867,50.895)
\lineto(576.90963867,50.76)
\curveto(576.90963677,50.71999578)(576.90463677,50.67499582)(576.89463867,50.625)
\curveto(576.88463679,50.58499591)(576.8796368,50.53999596)(576.87963867,50.49)
\curveto(576.88963679,50.43999606)(576.89463678,50.38999611)(576.89463867,50.34)
\lineto(576.89463867,50.04)
\curveto(576.89463678,49.94999655)(576.90463677,49.86499664)(576.92463867,49.785)
\curveto(576.93463674,49.75499675)(576.93963674,49.70999679)(576.93963867,49.65)
\curveto(576.95963672,49.57999692)(576.9746367,49.50999699)(576.98463867,49.44)
\lineto(577.04463867,49.23)
\curveto(577.13463654,48.93999756)(577.25463642,48.67499782)(577.40463867,48.435)
\curveto(577.55463612,48.20499829)(577.74463593,48.00999849)(577.97463867,47.85)
\lineto(578.06463867,47.79)
\curveto(578.10463557,47.76999873)(578.13963554,47.74999875)(578.16963867,47.73)
\curveto(578.26963541,47.66999883)(578.3746353,47.61999888)(578.48463867,47.58)
\lineto(578.84463867,47.49)
\curveto(578.89463478,47.46999903)(578.93463474,47.45999904)(578.96463867,47.46)
\curveto(578.99463468,47.46999903)(579.03463464,47.46999903)(579.08463867,47.46)
\curveto(579.12463455,47.44999905)(579.1746345,47.43999906)(579.23463867,47.43)
\curveto(579.29463438,47.42999907)(579.34963433,47.43999906)(579.39963867,47.46)
\lineto(579.51963867,47.46)
\curveto(579.54963413,47.46999903)(579.5796341,47.46999903)(579.60963867,47.46)
\curveto(579.63963404,47.45999904)(579.66963401,47.46499904)(579.69963867,47.475)
\curveto(579.7796339,47.49499901)(579.85963382,47.50999899)(579.93963867,47.52)
\curveto(580.01963366,47.53999896)(580.09463358,47.56499894)(580.16463867,47.595)
\curveto(580.4746332,47.72499878)(580.72963295,47.8999986)(580.92963867,48.12)
\curveto(581.12963255,48.34999815)(581.29463238,48.61499788)(581.42463867,48.915)
\curveto(581.4746322,49.02499748)(581.50963217,49.13499737)(581.52963867,49.245)
\curveto(581.54963213,49.35499715)(581.5746321,49.46999703)(581.60463867,49.59)
\curveto(581.62463205,49.62999687)(581.63463204,49.66999683)(581.63463867,49.71)
\curveto(581.63463204,49.74999675)(581.63963204,49.78999671)(581.64963867,49.83)
\curveto(581.65963202,49.87999662)(581.65963202,49.93499657)(581.64963867,49.995)
\curveto(581.64963203,50.05499645)(581.65463202,50.11499638)(581.66463867,50.175)
}
}
{
\newrgbcolor{curcolor}{0 0 0}
\pscustom[linestyle=none,fillstyle=solid,fillcolor=curcolor]
{
\newpath
\moveto(585.29088867,54.24)
\lineto(585.72588867,54.24)
\curveto(585.87588671,54.23999226)(585.9808866,54.1999923)(586.04088867,54.12)
\curveto(586.09088649,54.03999246)(586.11588647,53.93999256)(586.11588867,53.82)
\curveto(586.12588646,53.6999928)(586.13088645,53.57999292)(586.13088867,53.46)
\lineto(586.13088867,52.035)
\lineto(586.13088867,49.77)
\lineto(586.13088867,49.08)
\curveto(586.13088645,48.84999765)(586.15588643,48.64999785)(586.20588867,48.48)
\curveto(586.36588622,48.02999847)(586.66588592,47.71499878)(587.10588867,47.535)
\curveto(587.32588526,47.44499905)(587.59088499,47.40999909)(587.90088867,47.43)
\curveto(588.21088437,47.45999904)(588.46088412,47.51499898)(588.65088867,47.595)
\curveto(588.9808836,47.73499877)(589.24088334,47.90999859)(589.43088867,48.12)
\curveto(589.63088295,48.33999816)(589.7858828,48.62499788)(589.89588867,48.975)
\curveto(589.92588266,49.05499745)(589.94588264,49.13499737)(589.95588867,49.215)
\curveto(589.96588262,49.29499721)(589.9808826,49.37999712)(590.00088867,49.47)
\curveto(590.01088257,49.51999698)(590.01088257,49.56499694)(590.00088867,49.605)
\curveto(590.00088258,49.64499685)(590.01088257,49.68999681)(590.03088867,49.74)
\lineto(590.03088867,50.055)
\curveto(590.05088253,50.13499637)(590.05588253,50.22499628)(590.04588867,50.325)
\curveto(590.03588255,50.43499607)(590.03088255,50.53499597)(590.03088867,50.625)
\lineto(590.03088867,51.795)
\lineto(590.03088867,53.385)
\curveto(590.03088255,53.504993)(590.02588256,53.62999287)(590.01588867,53.76)
\curveto(590.01588257,53.8999926)(590.04088254,54.00999249)(590.09088867,54.09)
\curveto(590.13088245,54.13999236)(590.17588241,54.16999233)(590.22588867,54.18)
\curveto(590.2858823,54.1999923)(590.35588223,54.21999228)(590.43588867,54.24)
\lineto(590.66088867,54.24)
\curveto(590.7808818,54.23999226)(590.8858817,54.23499227)(590.97588867,54.225)
\curveto(591.07588151,54.21499228)(591.15088143,54.16999233)(591.20088867,54.09)
\curveto(591.25088133,54.03999246)(591.27588131,53.96499254)(591.27588867,53.865)
\lineto(591.27588867,53.58)
\lineto(591.27588867,52.56)
\lineto(591.27588867,48.525)
\lineto(591.27588867,47.175)
\curveto(591.27588131,47.05499945)(591.27088131,46.93999956)(591.26088867,46.83)
\curveto(591.26088132,46.72999977)(591.22588136,46.65499985)(591.15588867,46.605)
\curveto(591.11588147,46.57499992)(591.05588153,46.54999995)(590.97588867,46.53)
\curveto(590.89588169,46.51999998)(590.80588178,46.50999999)(590.70588867,46.5)
\curveto(590.61588197,46.5)(590.52588206,46.50499999)(590.43588867,46.515)
\curveto(590.35588223,46.52499998)(590.29588229,46.54499995)(590.25588867,46.575)
\curveto(590.20588238,46.61499988)(590.16088242,46.67999982)(590.12088867,46.77)
\curveto(590.11088247,46.80999969)(590.10088248,46.86499964)(590.09088867,46.935)
\curveto(590.09088249,47.00499949)(590.0858825,47.06999943)(590.07588867,47.13)
\curveto(590.06588252,47.1999993)(590.04588254,47.25499925)(590.01588867,47.295)
\curveto(589.9858826,47.33499916)(589.94088264,47.34999915)(589.88088867,47.34)
\curveto(589.80088278,47.31999918)(589.72088286,47.25999924)(589.64088867,47.16)
\curveto(589.56088302,47.06999943)(589.4858831,46.9999995)(589.41588867,46.95)
\curveto(589.19588339,46.78999971)(588.94588364,46.64999985)(588.66588867,46.53)
\curveto(588.55588403,46.48000002)(588.44088414,46.45000005)(588.32088867,46.44)
\curveto(588.21088437,46.42000008)(588.09588449,46.39500011)(587.97588867,46.365)
\curveto(587.92588466,46.35500015)(587.87088471,46.35500015)(587.81088867,46.365)
\curveto(587.76088482,46.37500012)(587.71088487,46.37000013)(587.66088867,46.35)
\curveto(587.56088502,46.33000017)(587.47088511,46.33000017)(587.39088867,46.35)
\lineto(587.24088867,46.35)
\curveto(587.19088539,46.37000013)(587.13088545,46.38000012)(587.06088867,46.38)
\curveto(587.00088558,46.38000012)(586.94588564,46.38500012)(586.89588867,46.395)
\curveto(586.85588573,46.41500008)(586.81588577,46.42500008)(586.77588867,46.425)
\curveto(586.74588584,46.41500008)(586.70588588,46.42000008)(586.65588867,46.44)
\lineto(586.41588867,46.5)
\curveto(586.34588624,46.51999998)(586.27088631,46.54999995)(586.19088867,46.59)
\curveto(585.93088665,46.6999998)(585.71088687,46.84499965)(585.53088867,47.025)
\curveto(585.36088722,47.21499928)(585.22088736,47.43999906)(585.11088867,47.7)
\curveto(585.07088751,47.78999871)(585.04088754,47.87999862)(585.02088867,47.97)
\lineto(584.96088867,48.27)
\curveto(584.94088764,48.32999817)(584.93088765,48.38499811)(584.93088867,48.435)
\curveto(584.94088764,48.49499801)(584.93588765,48.55999794)(584.91588867,48.63)
\curveto(584.90588768,48.64999785)(584.90088768,48.67499782)(584.90088867,48.705)
\curveto(584.90088768,48.74499775)(584.89588769,48.77999772)(584.88588867,48.81)
\lineto(584.88588867,48.96)
\curveto(584.87588771,48.9999975)(584.87088771,49.04499745)(584.87088867,49.095)
\curveto(584.8808877,49.15499735)(584.8858877,49.20999729)(584.88588867,49.26)
\lineto(584.88588867,49.86)
\lineto(584.88588867,52.62)
\lineto(584.88588867,53.58)
\lineto(584.88588867,53.85)
\curveto(584.8858877,53.93999256)(584.90588768,54.01499248)(584.94588867,54.075)
\curveto(584.9858876,54.14499235)(585.06088752,54.19499231)(585.17088867,54.225)
\curveto(585.19088739,54.23499227)(585.21088737,54.23499227)(585.23088867,54.225)
\curveto(585.25088733,54.22499227)(585.27088731,54.22999227)(585.29088867,54.24)
}
}
{
\newrgbcolor{curcolor}{0 0 0}
\pscustom[linestyle=none,fillstyle=solid,fillcolor=curcolor]
{
\newpath
\moveto(593.46049805,55.74)
\curveto(593.38049693,55.7999907)(593.33549697,55.9049906)(593.32549805,56.055)
\lineto(593.32549805,56.52)
\lineto(593.32549805,56.775)
\curveto(593.32549698,56.86498964)(593.34049697,56.93998956)(593.37049805,57)
\curveto(593.4104969,57.07998942)(593.49049682,57.13998936)(593.61049805,57.18)
\curveto(593.63049668,57.18998931)(593.65049666,57.18998931)(593.67049805,57.18)
\curveto(593.70049661,57.17998932)(593.72549658,57.18498931)(593.74549805,57.195)
\curveto(593.91549639,57.1949893)(594.07549623,57.18998931)(594.22549805,57.18)
\curveto(594.37549593,57.16998933)(594.47549583,57.10998939)(594.52549805,57)
\curveto(594.55549575,56.93998956)(594.57049574,56.86498964)(594.57049805,56.775)
\lineto(594.57049805,56.52)
\curveto(594.57049574,56.33999016)(594.56549574,56.16999033)(594.55549805,56.01)
\curveto(594.55549575,55.84999065)(594.49049582,55.74499076)(594.36049805,55.695)
\curveto(594.310496,55.67499083)(594.25549605,55.66499084)(594.19549805,55.665)
\lineto(594.03049805,55.665)
\lineto(593.71549805,55.665)
\curveto(593.61549669,55.66499084)(593.53049678,55.68999081)(593.46049805,55.74)
\moveto(594.57049805,47.235)
\lineto(594.57049805,46.92)
\curveto(594.58049573,46.81999968)(594.56049575,46.73999976)(594.51049805,46.68)
\curveto(594.48049583,46.61999988)(594.43549587,46.57999992)(594.37549805,46.56)
\curveto(594.31549599,46.54999995)(594.24549606,46.53499996)(594.16549805,46.515)
\lineto(593.94049805,46.515)
\curveto(593.8104965,46.51499998)(593.69549661,46.51999998)(593.59549805,46.53)
\curveto(593.5054968,46.54999995)(593.43549687,46.5999999)(593.38549805,46.68)
\curveto(593.34549696,46.73999976)(593.32549698,46.81499968)(593.32549805,46.905)
\lineto(593.32549805,47.19)
\lineto(593.32549805,53.535)
\lineto(593.32549805,53.85)
\curveto(593.32549698,53.95999254)(593.35049696,54.04499245)(593.40049805,54.105)
\curveto(593.43049688,54.15499234)(593.47049684,54.18499231)(593.52049805,54.195)
\curveto(593.57049674,54.2049923)(593.62549668,54.21999228)(593.68549805,54.24)
\curveto(593.7054966,54.23999226)(593.72549658,54.23499227)(593.74549805,54.225)
\curveto(593.77549653,54.22499227)(593.80049651,54.22999227)(593.82049805,54.24)
\curveto(593.95049636,54.23999226)(594.08049623,54.23499227)(594.21049805,54.225)
\curveto(594.35049596,54.22499227)(594.44549586,54.18499231)(594.49549805,54.105)
\curveto(594.54549576,54.04499245)(594.57049574,53.96499254)(594.57049805,53.865)
\lineto(594.57049805,53.58)
\lineto(594.57049805,47.235)
}
}
{
\newrgbcolor{curcolor}{0 0 0}
\pscustom[linestyle=none,fillstyle=solid,fillcolor=curcolor]
{
\newpath
\moveto(603.9403418,50.58)
\curveto(603.95033345,50.52999597)(603.95533344,50.46499604)(603.9553418,50.385)
\curveto(603.95533344,50.3049962)(603.95033345,50.23999626)(603.9403418,50.19)
\curveto(603.92033348,50.13999636)(603.91533348,50.08999641)(603.9253418,50.04)
\curveto(603.93533346,49.9999965)(603.93533346,49.95999654)(603.9253418,49.92)
\curveto(603.92533347,49.84999665)(603.92033348,49.79499671)(603.9103418,49.755)
\curveto(603.89033351,49.66499684)(603.87533352,49.57499692)(603.8653418,49.485)
\curveto(603.86533353,49.39499711)(603.85533354,49.30499719)(603.8353418,49.215)
\lineto(603.7753418,48.975)
\curveto(603.75533364,48.90499759)(603.73033367,48.82999767)(603.7003418,48.75)
\curveto(603.58033382,48.37999812)(603.41533398,48.04499845)(603.2053418,47.745)
\curveto(603.14533425,47.65499885)(603.08033432,47.56499894)(603.0103418,47.475)
\curveto(602.94033446,47.39499911)(602.86533453,47.31999918)(602.7853418,47.25)
\lineto(602.7103418,47.175)
\curveto(602.64033476,47.12499938)(602.57533482,47.07499942)(602.5153418,47.025)
\curveto(602.45533494,46.97499952)(602.38533501,46.92499958)(602.3053418,46.875)
\curveto(602.1953352,46.79499971)(602.07033533,46.72499978)(601.9303418,46.665)
\curveto(601.8003356,46.61499988)(601.66533573,46.56499994)(601.5253418,46.515)
\curveto(601.44533595,46.49500001)(601.36533603,46.48000002)(601.2853418,46.47)
\curveto(601.21533618,46.46000004)(601.14033626,46.44500005)(601.0603418,46.425)
\lineto(601.0003418,46.425)
\curveto(600.99033641,46.41500008)(600.97533642,46.41000009)(600.9553418,46.41)
\curveto(600.86533653,46.39000011)(600.73033667,46.38000012)(600.5503418,46.38)
\curveto(600.38033702,46.37000013)(600.24533715,46.37500012)(600.1453418,46.395)
\lineto(600.0703418,46.395)
\curveto(600.0003374,46.40500009)(599.93533746,46.41500008)(599.8753418,46.425)
\curveto(599.81533758,46.42500008)(599.75533764,46.43500006)(599.6953418,46.455)
\curveto(599.52533787,46.50499999)(599.36533803,46.54999995)(599.2153418,46.59)
\curveto(599.06533833,46.62999987)(598.92533847,46.68999981)(598.7953418,46.77)
\curveto(598.63533876,46.85999964)(598.4953389,46.95499955)(598.3753418,47.055)
\curveto(598.33533906,47.08499942)(598.27533912,47.12499938)(598.1953418,47.175)
\curveto(598.11533928,47.23499926)(598.04033936,47.23999926)(597.9703418,47.19)
\curveto(597.93033947,47.15999934)(597.91033949,47.11999938)(597.9103418,47.07)
\curveto(597.91033949,47.01999948)(597.9003395,46.96499954)(597.8803418,46.905)
\curveto(597.87033953,46.87499962)(597.87033953,46.83999966)(597.8803418,46.8)
\curveto(597.89033951,46.76999973)(597.89033951,46.73499976)(597.8803418,46.695)
\curveto(597.86033954,46.63499986)(597.85033955,46.56999993)(597.8503418,46.5)
\curveto(597.86033954,46.42000008)(597.86533953,46.35000015)(597.8653418,46.29)
\lineto(597.8653418,44.49)
\lineto(597.8653418,44.055)
\curveto(597.86533953,43.90500259)(597.83533956,43.79000271)(597.7753418,43.71)
\curveto(597.72533967,43.64000286)(597.64533975,43.60500289)(597.5353418,43.605)
\curveto(597.42533997,43.59500291)(597.31534008,43.59000291)(597.2053418,43.59)
\lineto(596.9653418,43.59)
\curveto(596.8953405,43.61000289)(596.83534056,43.63000287)(596.7853418,43.65)
\curveto(596.74534065,43.67000283)(596.71034069,43.70500279)(596.6803418,43.755)
\curveto(596.63034077,43.82500268)(596.60534079,43.93500256)(596.6053418,44.085)
\curveto(596.61534078,44.23500226)(596.62034078,44.36500213)(596.6203418,44.475)
\lineto(596.6203418,53.475)
\lineto(596.6203418,53.835)
\curveto(596.63034077,53.96499254)(596.66034074,54.06999243)(596.7103418,54.15)
\curveto(596.74034066,54.18999231)(596.80534059,54.21999228)(596.9053418,54.24)
\curveto(597.01534038,54.26999223)(597.13034027,54.27999222)(597.2503418,54.27)
\curveto(597.37034003,54.26999223)(597.48033992,54.25499224)(597.5803418,54.225)
\curveto(597.69033971,54.2049923)(597.76033964,54.17499233)(597.7903418,54.135)
\curveto(597.83033957,54.08499241)(597.85033955,54.02499247)(597.8503418,53.955)
\curveto(597.86033954,53.88499261)(597.88033952,53.81499268)(597.9103418,53.745)
\curveto(597.93033947,53.71499278)(597.94533945,53.68999281)(597.9553418,53.67)
\curveto(597.97533942,53.65999284)(597.9953394,53.64499285)(598.0153418,53.625)
\curveto(598.12533927,53.61499288)(598.21533918,53.64999285)(598.2853418,53.73)
\curveto(598.36533903,53.80999269)(598.44033896,53.87499263)(598.5103418,53.925)
\curveto(598.77033863,54.1049924)(599.08033832,54.24499225)(599.4403418,54.345)
\curveto(599.53033787,54.36499214)(599.62033778,54.37999212)(599.7103418,54.39)
\curveto(599.81033759,54.3999921)(599.91033749,54.41499208)(600.0103418,54.435)
\curveto(600.05033735,54.44499206)(600.1003373,54.44499206)(600.1603418,54.435)
\curveto(600.22033718,54.42499207)(600.26033714,54.42999207)(600.2803418,54.45)
\curveto(600.71033669,54.45999204)(601.09033631,54.41499208)(601.4203418,54.315)
\curveto(601.75033565,54.22499227)(602.04533535,54.09499241)(602.3053418,53.925)
\lineto(602.4553418,53.805)
\curveto(602.50533489,53.77499273)(602.55533484,53.73999276)(602.6053418,53.7)
\curveto(602.62533477,53.67999282)(602.64033476,53.65999284)(602.6503418,53.64)
\curveto(602.67033473,53.62999287)(602.69033471,53.61499288)(602.7103418,53.595)
\curveto(602.76033464,53.54499295)(602.81533458,53.48999301)(602.8753418,53.43)
\curveto(602.93533446,53.36999313)(602.99033441,53.30999319)(603.0403418,53.25)
\curveto(603.16033424,53.07999342)(603.28533411,52.89499361)(603.4153418,52.695)
\curveto(603.4953339,52.56499394)(603.56033384,52.41999408)(603.6103418,52.26)
\curveto(603.67033373,52.0999944)(603.72533367,51.93999456)(603.7753418,51.78)
\curveto(603.7953336,51.6999948)(603.81033359,51.61499488)(603.8203418,51.525)
\curveto(603.84033356,51.43499507)(603.86033354,51.34999515)(603.8803418,51.27)
\lineto(603.8803418,51.15)
\curveto(603.89033351,51.11999538)(603.8953335,51.08999541)(603.8953418,51.06)
\curveto(603.91533348,51.00999549)(603.92033348,50.95499554)(603.9103418,50.895)
\curveto(603.91033349,50.83499567)(603.92033348,50.77999572)(603.9403418,50.73)
\lineto(603.9403418,50.58)
\moveto(602.6053418,50.175)
\curveto(602.62533477,50.22499628)(602.63033477,50.28499621)(602.6203418,50.355)
\curveto(602.61033479,50.43499607)(602.60533479,50.504996)(602.6053418,50.565)
\curveto(602.60533479,50.73499577)(602.5953348,50.89499561)(602.5753418,51.045)
\curveto(602.56533483,51.19499531)(602.53533486,51.33999516)(602.4853418,51.48)
\lineto(602.4253418,51.66)
\curveto(602.41533498,51.72999477)(602.395335,51.79499471)(602.3653418,51.855)
\curveto(602.25533514,52.12499438)(602.08033532,52.38499411)(601.8403418,52.635)
\curveto(601.61033579,52.88499361)(601.39033601,53.05499344)(601.1803418,53.145)
\curveto(601.1003363,53.18499331)(601.01533638,53.21499328)(600.9253418,53.235)
\curveto(600.84533655,53.25499324)(600.76033664,53.27999322)(600.6703418,53.31)
\curveto(600.58033682,53.32999317)(600.47533692,53.33999316)(600.3553418,53.34)
\lineto(600.0253418,53.34)
\curveto(600.00533739,53.31999318)(599.96533743,53.30999319)(599.9053418,53.31)
\curveto(599.85533754,53.31999318)(599.81033759,53.31999318)(599.7703418,53.31)
\lineto(599.5003418,53.25)
\curveto(599.42033798,53.22999327)(599.34033806,53.1999933)(599.2603418,53.16)
\curveto(598.94033846,53.01999348)(598.67533872,52.81499368)(598.4653418,52.545)
\curveto(598.26533913,52.28499421)(598.11033929,51.97999452)(598.0003418,51.63)
\curveto(597.96033944,51.51999498)(597.93033947,51.40999509)(597.9103418,51.3)
\curveto(597.9003395,51.18999531)(597.88533951,51.07999542)(597.8653418,50.97)
\curveto(597.85533954,50.92999557)(597.85033955,50.88999561)(597.8503418,50.85)
\curveto(597.85033955,50.81999568)(597.84533955,50.78499571)(597.8353418,50.745)
\lineto(597.8353418,50.625)
\curveto(597.82533957,50.57499592)(597.82033958,50.499996)(597.8203418,50.4)
\curveto(597.82033958,50.30999619)(597.82533957,50.23999626)(597.8353418,50.19)
\lineto(597.8353418,50.07)
\curveto(597.84533955,50.02999647)(597.85033955,49.98999651)(597.8503418,49.95)
\curveto(597.85033955,49.90999659)(597.85533954,49.87499662)(597.8653418,49.845)
\curveto(597.87533952,49.81499668)(597.88033952,49.78499671)(597.8803418,49.755)
\curveto(597.88033952,49.72499678)(597.88533951,49.68999681)(597.8953418,49.65)
\curveto(597.91533948,49.56999693)(597.93033947,49.48999701)(597.9403418,49.41)
\lineto(598.0003418,49.17)
\curveto(598.11033929,48.82999767)(598.26033914,48.52999797)(598.4503418,48.27)
\curveto(598.65033875,48.01999848)(598.91033849,47.82499868)(599.2303418,47.685)
\curveto(599.42033798,47.60499889)(599.61533778,47.54499895)(599.8153418,47.505)
\curveto(599.85533754,47.48499902)(599.8953375,47.47499902)(599.9353418,47.475)
\curveto(599.97533742,47.48499902)(600.01533738,47.48499902)(600.0553418,47.475)
\lineto(600.1753418,47.475)
\curveto(600.24533715,47.45499905)(600.31533708,47.45499905)(600.3853418,47.475)
\lineto(600.5053418,47.475)
\curveto(600.61533678,47.49499901)(600.72033668,47.50999899)(600.8203418,47.52)
\curveto(600.92033648,47.52999897)(601.02033638,47.55499895)(601.1203418,47.595)
\curveto(601.43033597,47.72499878)(601.68033572,47.89499861)(601.8703418,48.105)
\curveto(602.07033533,48.32499818)(602.23533516,48.58999791)(602.3653418,48.9)
\curveto(602.41533498,49.03999746)(602.45033495,49.17999732)(602.4703418,49.32)
\curveto(602.5003349,49.46999703)(602.53533486,49.62499688)(602.5753418,49.785)
\curveto(602.58533481,49.83499667)(602.59033481,49.87999662)(602.5903418,49.92)
\curveto(602.59033481,49.95999654)(602.5953348,50.0049965)(602.6053418,50.055)
\lineto(602.6053418,50.175)
}
}
{
\newrgbcolor{curcolor}{0 0 0}
\pscustom[linestyle=none,fillstyle=solid,fillcolor=curcolor]
{
\newpath
\moveto(612.5465918,50.715)
\curveto(612.56658374,50.65499584)(612.57658373,50.55999594)(612.5765918,50.43)
\curveto(612.57658373,50.30999619)(612.57158373,50.22499628)(612.5615918,50.175)
\lineto(612.5615918,50.025)
\curveto(612.55158375,49.94499655)(612.54158376,49.86999663)(612.5315918,49.8)
\curveto(612.53158377,49.73999676)(612.52658378,49.66999683)(612.5165918,49.59)
\curveto(612.49658381,49.52999697)(612.48158382,49.46999703)(612.4715918,49.41)
\curveto(612.47158383,49.34999715)(612.46158384,49.28999721)(612.4415918,49.23)
\curveto(612.4015839,49.0999974)(612.36658394,48.96999753)(612.3365918,48.84)
\curveto(612.306584,48.70999779)(612.26658404,48.58999791)(612.2165918,48.48)
\curveto(612.0065843,47.9999985)(611.72658458,47.59499891)(611.3765918,47.265)
\curveto(611.02658528,46.94499955)(610.59658571,46.6999998)(610.0865918,46.53)
\curveto(609.97658633,46.49000001)(609.85658645,46.46000004)(609.7265918,46.44)
\curveto(609.6065867,46.42000008)(609.48158682,46.4000001)(609.3515918,46.38)
\curveto(609.29158701,46.37000013)(609.22658708,46.36500014)(609.1565918,46.365)
\curveto(609.09658721,46.35500015)(609.03658727,46.35000015)(608.9765918,46.35)
\curveto(608.93658737,46.34000016)(608.87658743,46.33500016)(608.7965918,46.335)
\curveto(608.72658758,46.33500016)(608.67658763,46.34000016)(608.6465918,46.35)
\curveto(608.6065877,46.36000014)(608.56658774,46.36500014)(608.5265918,46.365)
\curveto(608.48658782,46.35500015)(608.45158785,46.35500015)(608.4215918,46.365)
\lineto(608.3315918,46.365)
\lineto(607.9715918,46.41)
\curveto(607.83158847,46.45000005)(607.69658861,46.49000001)(607.5665918,46.53)
\curveto(607.43658887,46.56999993)(607.31158899,46.61499988)(607.1915918,46.665)
\curveto(606.74158956,46.86499964)(606.37158993,47.12499938)(606.0815918,47.445)
\curveto(605.79159051,47.76499874)(605.55159075,48.15499835)(605.3615918,48.615)
\curveto(605.31159099,48.71499778)(605.27159103,48.81499768)(605.2415918,48.915)
\curveto(605.22159108,49.01499748)(605.2015911,49.11999738)(605.1815918,49.23)
\curveto(605.16159114,49.26999723)(605.15159115,49.2999972)(605.1515918,49.32)
\curveto(605.16159114,49.34999715)(605.16159114,49.38499711)(605.1515918,49.425)
\curveto(605.13159117,49.50499699)(605.11659119,49.58499691)(605.1065918,49.665)
\curveto(605.1065912,49.75499675)(605.09659121,49.83999666)(605.0765918,49.92)
\lineto(605.0765918,50.04)
\curveto(605.07659123,50.07999642)(605.07159123,50.12499638)(605.0615918,50.175)
\curveto(605.05159125,50.22499628)(605.04659126,50.30999619)(605.0465918,50.43)
\curveto(605.04659126,50.55999594)(605.05659125,50.65499584)(605.0765918,50.715)
\curveto(605.09659121,50.78499571)(605.1015912,50.85499564)(605.0915918,50.925)
\curveto(605.08159122,50.99499551)(605.08659122,51.06499544)(605.1065918,51.135)
\curveto(605.11659119,51.18499531)(605.12159118,51.22499528)(605.1215918,51.255)
\curveto(605.13159117,51.29499521)(605.14159116,51.33999516)(605.1515918,51.39)
\curveto(605.18159112,51.50999499)(605.2065911,51.62999487)(605.2265918,51.75)
\curveto(605.25659105,51.86999463)(605.29659101,51.98499451)(605.3465918,52.095)
\curveto(605.49659081,52.46499404)(605.67659063,52.79499371)(605.8865918,53.085)
\curveto(606.1065902,53.38499311)(606.37158993,53.63499287)(606.6815918,53.835)
\curveto(606.8015895,53.91499258)(606.92658938,53.97999252)(607.0565918,54.03)
\curveto(607.18658912,54.08999241)(607.32158898,54.14999235)(607.4615918,54.21)
\curveto(607.58158872,54.25999224)(607.71158859,54.28999221)(607.8515918,54.3)
\curveto(607.99158831,54.31999218)(608.13158817,54.34999215)(608.2715918,54.39)
\lineto(608.4665918,54.39)
\curveto(608.53658777,54.3999921)(608.6015877,54.40999209)(608.6615918,54.42)
\curveto(609.55158675,54.42999207)(610.29158601,54.24499225)(610.8815918,53.865)
\curveto(611.47158483,53.48499301)(611.89658441,52.98999351)(612.1565918,52.38)
\curveto(612.2065841,52.27999422)(612.24658406,52.17999432)(612.2765918,52.08)
\curveto(612.306584,51.97999452)(612.34158396,51.87499462)(612.3815918,51.765)
\curveto(612.41158389,51.65499484)(612.43658387,51.53499497)(612.4565918,51.405)
\curveto(612.47658383,51.28499521)(612.5015838,51.15999534)(612.5315918,51.03)
\curveto(612.54158376,50.97999552)(612.54158376,50.92499558)(612.5315918,50.865)
\curveto(612.53158377,50.81499568)(612.53658377,50.76499574)(612.5465918,50.715)
\moveto(611.2115918,49.86)
\curveto(611.23158507,49.92999657)(611.23658507,50.00999649)(611.2265918,50.1)
\lineto(611.2265918,50.355)
\curveto(611.22658508,50.74499575)(611.19158511,51.07499542)(611.1215918,51.345)
\curveto(611.09158521,51.42499508)(611.06658524,51.504995)(611.0465918,51.585)
\curveto(611.02658528,51.66499484)(611.0015853,51.73999476)(610.9715918,51.81)
\curveto(610.69158561,52.45999404)(610.24658606,52.90999359)(609.6365918,53.16)
\curveto(609.56658674,53.18999331)(609.49158681,53.20999329)(609.4115918,53.22)
\lineto(609.1715918,53.28)
\curveto(609.09158721,53.2999932)(609.0065873,53.30999319)(608.9165918,53.31)
\lineto(608.6465918,53.31)
\lineto(608.3765918,53.265)
\curveto(608.27658803,53.24499325)(608.18158812,53.21999328)(608.0915918,53.19)
\curveto(608.01158829,53.16999333)(607.93158837,53.13999336)(607.8515918,53.1)
\curveto(607.78158852,53.07999342)(607.71658859,53.04999345)(607.6565918,53.01)
\curveto(607.59658871,52.96999353)(607.54158876,52.92999357)(607.4915918,52.89)
\curveto(607.25158905,52.71999378)(607.05658925,52.51499398)(606.9065918,52.275)
\curveto(606.75658955,52.03499447)(606.62658968,51.75499474)(606.5165918,51.435)
\curveto(606.48658982,51.33499517)(606.46658984,51.22999527)(606.4565918,51.12)
\curveto(606.44658986,51.01999548)(606.43158987,50.91499558)(606.4115918,50.805)
\curveto(606.4015899,50.76499574)(606.39658991,50.6999958)(606.3965918,50.61)
\curveto(606.38658992,50.57999592)(606.38158992,50.54499595)(606.3815918,50.505)
\curveto(606.39158991,50.46499604)(606.39658991,50.41999608)(606.3965918,50.37)
\lineto(606.3965918,50.07)
\curveto(606.39658991,49.96999653)(606.4065899,49.87999662)(606.4265918,49.8)
\lineto(606.4565918,49.62)
\curveto(606.47658983,49.51999698)(606.49158981,49.41999708)(606.5015918,49.32)
\curveto(606.52158978,49.22999727)(606.55158975,49.14499735)(606.5915918,49.065)
\curveto(606.69158961,48.82499768)(606.8065895,48.5999979)(606.9365918,48.39)
\curveto(607.07658923,48.17999832)(607.24658906,48.00499849)(607.4465918,47.865)
\curveto(607.49658881,47.83499867)(607.54158876,47.80999869)(607.5815918,47.79)
\curveto(607.62158868,47.76999873)(607.66658864,47.74499875)(607.7165918,47.715)
\curveto(607.79658851,47.66499884)(607.88158842,47.61999888)(607.9715918,47.58)
\curveto(608.07158823,47.54999895)(608.17658813,47.51999898)(608.2865918,47.49)
\curveto(608.33658797,47.46999903)(608.38158792,47.45999904)(608.4215918,47.46)
\curveto(608.47158783,47.46999903)(608.52158778,47.46999903)(608.5715918,47.46)
\curveto(608.6015877,47.44999905)(608.66158764,47.43999906)(608.7515918,47.43)
\curveto(608.85158745,47.41999908)(608.92658738,47.42499908)(608.9765918,47.445)
\curveto(609.01658729,47.45499905)(609.05658725,47.45499905)(609.0965918,47.445)
\curveto(609.13658717,47.44499905)(609.17658713,47.45499905)(609.2165918,47.475)
\curveto(609.29658701,47.49499901)(609.37658693,47.50999899)(609.4565918,47.52)
\curveto(609.53658677,47.53999896)(609.61158669,47.56499894)(609.6815918,47.595)
\curveto(610.02158628,47.73499877)(610.29658601,47.92999857)(610.5065918,48.18)
\curveto(610.71658559,48.42999807)(610.89158541,48.72499778)(611.0315918,49.065)
\curveto(611.08158522,49.18499731)(611.11158519,49.30999719)(611.1215918,49.44)
\curveto(611.14158516,49.57999692)(611.17158513,49.71999678)(611.2115918,49.86)
}
}
{
\newrgbcolor{curcolor}{0 0 0}
\pscustom[linestyle=none,fillstyle=solid,fillcolor=curcolor]
{
\newpath
\moveto(616.46487305,54.42)
\curveto(617.18486898,54.42999207)(617.78986838,54.34499215)(618.27987305,54.165)
\curveto(618.7698674,53.99499251)(619.14986702,53.68999281)(619.41987305,53.25)
\curveto(619.48986668,53.13999336)(619.54486662,53.02499347)(619.58487305,52.905)
\curveto(619.62486654,52.79499371)(619.6648665,52.66999383)(619.70487305,52.53)
\curveto(619.72486644,52.45999404)(619.72986644,52.38499411)(619.71987305,52.305)
\curveto(619.70986646,52.23499427)(619.69486647,52.17999432)(619.67487305,52.14)
\curveto(619.65486651,52.11999438)(619.62986654,52.0999944)(619.59987305,52.08)
\curveto(619.5698666,52.06999443)(619.54486662,52.05499444)(619.52487305,52.035)
\curveto(619.47486669,52.01499448)(619.42486674,52.00999449)(619.37487305,52.02)
\curveto(619.32486684,52.02999447)(619.27486689,52.02999447)(619.22487305,52.02)
\curveto(619.14486702,51.9999945)(619.03986713,51.99499451)(618.90987305,52.005)
\curveto(618.77986739,52.02499447)(618.68986748,52.04999445)(618.63987305,52.08)
\curveto(618.55986761,52.12999437)(618.50486766,52.19499431)(618.47487305,52.275)
\curveto(618.45486771,52.36499414)(618.41986775,52.44999405)(618.36987305,52.53)
\curveto(618.27986789,52.68999381)(618.15486801,52.83499367)(617.99487305,52.965)
\curveto(617.88486828,53.04499345)(617.7648684,53.1049934)(617.63487305,53.145)
\curveto(617.50486866,53.18499331)(617.3648688,53.22499327)(617.21487305,53.265)
\curveto(617.164869,53.28499321)(617.11486905,53.28999321)(617.06487305,53.28)
\curveto(617.01486915,53.27999322)(616.9648692,53.28499321)(616.91487305,53.295)
\curveto(616.85486931,53.31499318)(616.77986939,53.32499317)(616.68987305,53.325)
\curveto(616.59986957,53.32499317)(616.52486964,53.31499318)(616.46487305,53.295)
\lineto(616.37487305,53.295)
\lineto(616.22487305,53.265)
\curveto(616.17486999,53.26499324)(616.12487004,53.25999324)(616.07487305,53.25)
\curveto(615.81487035,53.18999331)(615.59987057,53.1049934)(615.42987305,52.995)
\curveto(615.25987091,52.88499361)(615.14487102,52.6999938)(615.08487305,52.44)
\curveto(615.0648711,52.36999413)(615.05987111,52.2999942)(615.06987305,52.23)
\curveto(615.08987108,52.15999434)(615.10987106,52.0999944)(615.12987305,52.05)
\curveto(615.18987098,51.8999946)(615.25987091,51.78999471)(615.33987305,51.72)
\curveto(615.42987074,51.65999484)(615.53987063,51.58999491)(615.66987305,51.51)
\curveto(615.82987034,51.40999509)(616.00987016,51.33499517)(616.20987305,51.285)
\curveto(616.40986976,51.24499525)(616.60986956,51.19499531)(616.80987305,51.135)
\curveto(616.93986923,51.09499541)(617.0698691,51.06499544)(617.19987305,51.045)
\curveto(617.32986884,51.02499548)(617.45986871,50.99499551)(617.58987305,50.955)
\curveto(617.79986837,50.89499561)(618.00486816,50.83499567)(618.20487305,50.775)
\curveto(618.40486776,50.72499578)(618.60486756,50.65999584)(618.80487305,50.58)
\lineto(618.95487305,50.52)
\curveto(619.00486716,50.499996)(619.05486711,50.47499602)(619.10487305,50.445)
\curveto(619.30486686,50.32499618)(619.47986669,50.18999631)(619.62987305,50.04)
\curveto(619.77986639,49.88999661)(619.90486626,49.6999968)(620.00487305,49.47)
\curveto(620.02486614,49.3999971)(620.04486612,49.30499719)(620.06487305,49.185)
\curveto(620.08486608,49.11499738)(620.09486607,49.03999746)(620.09487305,48.96)
\curveto(620.10486606,48.88999761)(620.10986606,48.80999769)(620.10987305,48.72)
\lineto(620.10987305,48.57)
\curveto(620.08986608,48.499998)(620.07986609,48.42999807)(620.07987305,48.36)
\curveto(620.07986609,48.28999821)(620.0698661,48.21999828)(620.04987305,48.15)
\curveto(620.01986615,48.03999846)(619.98486618,47.93499857)(619.94487305,47.835)
\curveto(619.90486626,47.73499877)(619.85986631,47.64499885)(619.80987305,47.565)
\curveto(619.64986652,47.30499919)(619.44486672,47.09499941)(619.19487305,46.935)
\curveto(618.94486722,46.78499972)(618.6648675,46.65499985)(618.35487305,46.545)
\curveto(618.2648679,46.51499998)(618.169868,46.49500001)(618.06987305,46.485)
\curveto(617.97986819,46.46500004)(617.88986828,46.44000006)(617.79987305,46.41)
\curveto(617.69986847,46.39000011)(617.59986857,46.38000012)(617.49987305,46.38)
\curveto(617.39986877,46.38000012)(617.29986887,46.37000013)(617.19987305,46.35)
\lineto(617.04987305,46.35)
\curveto(616.99986917,46.34000016)(616.92986924,46.33500016)(616.83987305,46.335)
\curveto(616.74986942,46.33500016)(616.67986949,46.34000016)(616.62987305,46.35)
\lineto(616.46487305,46.35)
\curveto(616.40486976,46.37000013)(616.33986983,46.38000012)(616.26987305,46.38)
\curveto(616.19986997,46.37000013)(616.13987003,46.37500012)(616.08987305,46.395)
\curveto(616.03987013,46.40500009)(615.97487019,46.41000009)(615.89487305,46.41)
\lineto(615.65487305,46.47)
\curveto(615.58487058,46.48000002)(615.50987066,46.5)(615.42987305,46.53)
\curveto(615.11987105,46.62999987)(614.84987132,46.75499975)(614.61987305,46.905)
\curveto(614.38987178,47.05499945)(614.18987198,47.24999925)(614.01987305,47.49)
\curveto(613.92987224,47.61999888)(613.85487231,47.75499875)(613.79487305,47.895)
\curveto(613.73487243,48.03499847)(613.67987249,48.18999831)(613.62987305,48.36)
\curveto(613.60987256,48.41999808)(613.59987257,48.48999801)(613.59987305,48.57)
\curveto(613.60987256,48.65999784)(613.62487254,48.72999777)(613.64487305,48.78)
\curveto(613.67487249,48.81999768)(613.72487244,48.85999764)(613.79487305,48.9)
\curveto(613.84487232,48.91999758)(613.91487225,48.92999757)(614.00487305,48.93)
\curveto(614.09487207,48.93999756)(614.18487198,48.93999756)(614.27487305,48.93)
\curveto(614.3648718,48.91999758)(614.44987172,48.90499759)(614.52987305,48.885)
\curveto(614.61987155,48.87499762)(614.67987149,48.85999764)(614.70987305,48.84)
\curveto(614.77987139,48.78999771)(614.82487134,48.71499778)(614.84487305,48.615)
\curveto(614.87487129,48.52499798)(614.90987126,48.43999806)(614.94987305,48.36)
\curveto(615.04987112,48.13999836)(615.18487098,47.96999853)(615.35487305,47.85)
\curveto(615.47487069,47.75999874)(615.60987056,47.68999881)(615.75987305,47.64)
\curveto(615.90987026,47.58999891)(616.0698701,47.53999896)(616.23987305,47.49)
\lineto(616.55487305,47.445)
\lineto(616.64487305,47.445)
\curveto(616.71486945,47.42499908)(616.80486936,47.41499908)(616.91487305,47.415)
\curveto(617.03486913,47.41499908)(617.13486903,47.42499908)(617.21487305,47.445)
\curveto(617.28486888,47.44499905)(617.33986883,47.44999905)(617.37987305,47.46)
\curveto(617.43986873,47.46999903)(617.49986867,47.47499902)(617.55987305,47.475)
\curveto(617.61986855,47.48499902)(617.67486849,47.49499901)(617.72487305,47.505)
\curveto(618.01486815,47.58499892)(618.24486792,47.68999881)(618.41487305,47.82)
\curveto(618.58486758,47.94999855)(618.70486746,48.16999833)(618.77487305,48.48)
\curveto(618.79486737,48.52999797)(618.79986737,48.58499791)(618.78987305,48.645)
\curveto(618.77986739,48.70499779)(618.7698674,48.74999775)(618.75987305,48.78)
\curveto(618.70986746,48.96999753)(618.63986753,49.10999739)(618.54987305,49.2)
\curveto(618.45986771,49.2999972)(618.34486782,49.38999711)(618.20487305,49.47)
\curveto(618.11486805,49.52999697)(618.01486815,49.57999692)(617.90487305,49.62)
\lineto(617.57487305,49.74)
\curveto(617.54486862,49.74999675)(617.51486865,49.75499675)(617.48487305,49.755)
\curveto(617.4648687,49.75499675)(617.43986873,49.76499674)(617.40987305,49.785)
\curveto(617.0698691,49.89499661)(616.71486945,49.97499652)(616.34487305,50.025)
\curveto(615.98487018,50.08499641)(615.64487052,50.17999632)(615.32487305,50.31)
\curveto(615.22487094,50.34999615)(615.12987104,50.38499611)(615.03987305,50.415)
\curveto(614.94987122,50.44499605)(614.8648713,50.48499601)(614.78487305,50.535)
\curveto(614.59487157,50.64499585)(614.41987175,50.76999573)(614.25987305,50.91)
\curveto(614.09987207,51.04999545)(613.97487219,51.22499528)(613.88487305,51.435)
\curveto(613.85487231,51.504995)(613.82987234,51.57499492)(613.80987305,51.645)
\curveto(613.79987237,51.71499478)(613.78487238,51.78999471)(613.76487305,51.87)
\curveto(613.73487243,51.98999451)(613.72487244,52.12499438)(613.73487305,52.275)
\curveto(613.74487242,52.43499407)(613.75987241,52.56999393)(613.77987305,52.68)
\curveto(613.79987237,52.72999377)(613.80987236,52.76999373)(613.80987305,52.8)
\curveto(613.81987235,52.83999366)(613.83487233,52.87999362)(613.85487305,52.92)
\curveto(613.94487222,53.14999335)(614.0648721,53.34999315)(614.21487305,53.52)
\curveto(614.37487179,53.68999281)(614.55487161,53.83999266)(614.75487305,53.97)
\curveto(614.90487126,54.05999244)(615.0698711,54.12999237)(615.24987305,54.18)
\curveto(615.42987074,54.23999226)(615.61987055,54.29499221)(615.81987305,54.345)
\curveto(615.88987028,54.35499214)(615.95487021,54.36499214)(616.01487305,54.375)
\curveto(616.08487008,54.38499211)(616.15987001,54.39499211)(616.23987305,54.405)
\curveto(616.2698699,54.41499208)(616.30986986,54.41499208)(616.35987305,54.405)
\curveto(616.40986976,54.39499211)(616.44486972,54.3999921)(616.46487305,54.42)
}
}
{
\newrgbcolor{curcolor}{0.7019608 0.7019608 0.7019608}
\pscustom[linestyle=none,fillstyle=solid,fillcolor=curcolor]
{
\newpath
\moveto(134.08439636,211.00222778)
\lineto(162.00564194,211.00222778)
\lineto(162.00564194,86.08279419)
\lineto(134.08439636,86.08279419)
\closepath
}
}
{
\newrgbcolor{curcolor}{0.80000001 0.80000001 0.80000001}
\pscustom[linestyle=none,fillstyle=solid,fillcolor=curcolor]
{
\newpath
\moveto(197.2718811,114.98318481)
\lineto(225.19312668,114.98318481)
\lineto(225.19312668,86.08278275)
\lineto(197.2718811,86.08278275)
\closepath
}
}
{
\newrgbcolor{curcolor}{0.7019608 0.7019608 0.7019608}
\pscustom[linestyle=none,fillstyle=solid,fillcolor=curcolor]
{
\newpath
\moveto(225.3343811,92.00222778)
\lineto(253.25562668,92.00222778)
\lineto(253.25562668,86.0827961)
\lineto(225.3343811,86.0827961)
\closepath
}
}
{
\newrgbcolor{curcolor}{0.80000001 0.80000001 0.80000001}
\pscustom[linestyle=none,fillstyle=solid,fillcolor=curcolor]
{
\newpath
\moveto(288.99200439,121.96588135)
\lineto(316.91324997,121.96588135)
\lineto(316.91324997,86.08280182)
\lineto(288.99200439,86.08280182)
\closepath
}
}
{
\newrgbcolor{curcolor}{0.7019608 0.7019608 0.7019608}
\pscustom[linestyle=none,fillstyle=solid,fillcolor=curcolor]
{
\newpath
\moveto(317.05450439,99.86880493)
\lineto(344.97574997,99.86880493)
\lineto(344.97574997,86.0828104)
\lineto(317.05450439,86.0828104)
\closepath
}
}
{
\newrgbcolor{curcolor}{0.80000001 0.80000001 0.80000001}
\pscustom[linestyle=none,fillstyle=solid,fillcolor=curcolor]
{
\newpath
\moveto(380.11700439,115.51351929)
\lineto(408.03824997,115.51351929)
\lineto(408.03824997,86.08278847)
\lineto(380.11700439,86.08278847)
\closepath
}
}
{
\newrgbcolor{curcolor}{0.7019608 0.7019608 0.7019608}
\pscustom[linestyle=none,fillstyle=solid,fillcolor=curcolor]
{
\newpath
\moveto(408.17950439,181.98156738)
\lineto(436.10074997,181.98156738)
\lineto(436.10074997,86.08280182)
\lineto(408.17950439,86.08280182)
\closepath
}
}
{
\newrgbcolor{curcolor}{0.80000001 0.80000001 0.80000001}
\pscustom[linestyle=none,fillstyle=solid,fillcolor=curcolor]
{
\newpath
\moveto(471.15698242,143.97457886)
\lineto(499.078228,143.97457886)
\lineto(499.078228,86.08280182)
\lineto(471.15698242,86.08280182)
\closepath
}
}
{
\newrgbcolor{curcolor}{0.7019608 0.7019608 0.7019608}
\pscustom[linestyle=none,fillstyle=solid,fillcolor=curcolor]
{
\newpath
\moveto(499.21948242,107.20501709)
\lineto(527.140728,107.20501709)
\lineto(527.140728,86.08279037)
\lineto(499.21948242,86.08279037)
\closepath
}
}
{
\newrgbcolor{curcolor}{0.80000001 0.80000001 0.80000001}
\pscustom[linestyle=none,fillstyle=solid,fillcolor=curcolor]
{
\newpath
\moveto(654.12084961,167.13232422)
\lineto(682.04209518,167.13232422)
\lineto(682.04209518,86.08280182)
\lineto(654.12084961,86.08280182)
\closepath
}
}
{
\newrgbcolor{curcolor}{0.7019608 0.7019608 0.7019608}
\pscustom[linestyle=none,fillstyle=solid,fillcolor=curcolor]
{
\newpath
\moveto(682.18334961,308.02334595)
\lineto(710.10459518,308.02334595)
\lineto(710.10459518,86.08280945)
\lineto(682.18334961,86.08280945)
\closepath
}
}
\end{pspicture}

\caption{Diagrama de barras de los espacios y sus recursos}
\label{espacios_bars_1}
\end{figure}

Para finalizar con los espacios virtuales, podemos ver en la figura
\ref{espacios_pie_1} las gráficas circulares correspondientes a los datos sobre
espacios virtuales, podemos ver que si bien la portada acapara la mayor
audiencia, esta no acapara la mayor cantidad de recursos, llevándose los
espacios de comunidades un 45\% del contenido del sitio, reforzando la teoría de
fomento hacia los espacios menos formales.

\begin{figure}
\centering
%LaTeX with PSTricks extensions
%%Creator: inkscape 0.48.5
%%Please note this file requires PSTricks extensions
\psset{xunit=.5pt,yunit=.5pt,runit=.5pt}
\begin{pspicture}(927,369)
{
\newrgbcolor{curcolor}{0 0 0}
\pscustom[linestyle=none,fillstyle=solid,fillcolor=curcolor]
{
\newpath
\moveto(28.17753982,344.30956697)
\curveto(28.17752934,344.2795613)(28.17752934,344.23956134)(28.17753982,344.18956697)
\curveto(28.18752933,344.13956144)(28.19252932,344.0845615)(28.19253982,344.02456697)
\curveto(28.19252932,343.96456162)(28.18752933,343.90956167)(28.17753982,343.85956697)
\curveto(28.17752934,343.80956177)(28.17752934,343.77456181)(28.17753982,343.75456697)
\curveto(28.17752934,343.6845619)(28.17252934,343.61456197)(28.16253982,343.54456697)
\curveto(28.16252935,343.4845621)(28.16252935,343.42456216)(28.16253982,343.36456697)
\curveto(28.14252937,343.31456227)(28.13252938,343.26456232)(28.13253982,343.21456697)
\curveto(28.14252937,343.16456242)(28.14252937,343.11456247)(28.13253982,343.06456697)
\curveto(28.1125294,342.95456263)(28.09752942,342.84456274)(28.08753982,342.73456697)
\curveto(28.07752944,342.62456296)(28.05752946,342.51456307)(28.02753982,342.40456697)
\curveto(27.97752954,342.23456335)(27.93252958,342.06956351)(27.89253982,341.90956697)
\curveto(27.85252966,341.75956382)(27.80252971,341.60956397)(27.74253982,341.45956697)
\curveto(27.57252994,341.03956454)(27.36253015,340.65956492)(27.11253982,340.31956697)
\curveto(26.86253065,339.9795656)(26.56253095,339.68956589)(26.21253982,339.44956697)
\curveto(26.0125315,339.30956627)(25.80253171,339.18956639)(25.58253982,339.08956697)
\curveto(25.37253214,338.98956659)(25.14253237,338.89956668)(24.89253982,338.81956697)
\curveto(24.79253272,338.78956679)(24.68753283,338.76456682)(24.57753982,338.74456697)
\curveto(24.47753304,338.73456685)(24.37253314,338.71456687)(24.26253982,338.68456697)
\curveto(24.2125333,338.67456691)(24.16253335,338.66956691)(24.11253982,338.66956697)
\curveto(24.07253344,338.66956691)(24.02753349,338.66456692)(23.97753982,338.65456697)
\curveto(23.93753358,338.64456694)(23.89753362,338.63956694)(23.85753982,338.63956697)
\curveto(23.8175337,338.64956693)(23.77253374,338.64956693)(23.72253982,338.63956697)
\curveto(23.70253381,338.62956695)(23.67253384,338.62456696)(23.63253982,338.62456697)
\curveto(23.59253392,338.63456695)(23.56253395,338.63456695)(23.54253982,338.62456697)
\curveto(23.46253405,338.60456698)(23.36253415,338.59956698)(23.24253982,338.60956697)
\curveto(23.12253439,338.61956696)(23.0175345,338.62456696)(22.92753982,338.62456697)
\lineto(19.43253982,338.62456697)
\curveto(19.26253825,338.62456696)(19.1175384,338.62956695)(18.99753982,338.63956697)
\curveto(18.88753863,338.65956692)(18.80753871,338.72956685)(18.75753982,338.84956697)
\curveto(18.72753879,338.92956665)(18.7125388,339.04956653)(18.71253982,339.20956697)
\curveto(18.72253879,339.3795662)(18.72753879,339.51956606)(18.72753982,339.62956697)
\lineto(18.72753982,348.43456697)
\curveto(18.72753879,348.55455703)(18.72253879,348.6795569)(18.71253982,348.80956697)
\curveto(18.7125388,348.94955663)(18.73753878,349.05955652)(18.78753982,349.13956697)
\curveto(18.82753869,349.19955638)(18.90253861,349.24955633)(19.01253982,349.28956697)
\curveto(19.03253848,349.29955628)(19.05253846,349.29955628)(19.07253982,349.28956697)
\curveto(19.09253842,349.28955629)(19.1125384,349.29455629)(19.13253982,349.30456697)
\lineto(23.16753982,349.30456697)
\curveto(23.22753429,349.30455628)(23.28753423,349.30455628)(23.34753982,349.30456697)
\curveto(23.4175341,349.31455627)(23.47753404,349.31455627)(23.52753982,349.30456697)
\lineto(23.70753982,349.30456697)
\curveto(23.75753376,349.2845563)(23.8125337,349.27455631)(23.87253982,349.27456697)
\curveto(23.93253358,349.2845563)(23.98753353,349.2795563)(24.03753982,349.25956697)
\curveto(24.09753342,349.23955634)(24.15253336,349.22955635)(24.20253982,349.22956697)
\curveto(24.26253325,349.23955634)(24.32253319,349.23455635)(24.38253982,349.21456697)
\curveto(24.52253299,349.1845564)(24.65753286,349.15455643)(24.78753982,349.12456697)
\curveto(24.9175326,349.10455648)(25.04253247,349.06955651)(25.16253982,349.01956697)
\curveto(25.27253224,348.96955661)(25.38253213,348.92455666)(25.49253982,348.88456697)
\curveto(25.60253191,348.84455674)(25.70753181,348.79455679)(25.80753982,348.73456697)
\curveto(26.05753146,348.57455701)(26.28753123,348.41955716)(26.49753982,348.26956697)
\lineto(26.58753982,348.17956697)
\curveto(26.68753083,348.09955748)(26.77753074,348.00955757)(26.85753982,347.90956697)
\lineto(26.99253982,347.78956697)
\curveto(27.04253047,347.70955787)(27.09753042,347.62955795)(27.15753982,347.54956697)
\curveto(27.22753029,347.4795581)(27.28753023,347.40455818)(27.33753982,347.32456697)
\curveto(27.46753005,347.11455847)(27.58252993,346.88955869)(27.68253982,346.64956697)
\curveto(27.78252973,346.41955916)(27.87252964,346.17455941)(27.95253982,345.91456697)
\curveto(28.00252951,345.7845598)(28.03252948,345.64955993)(28.04253982,345.50956697)
\curveto(28.06252945,345.36956021)(28.08752943,345.22956035)(28.11753982,345.08956697)
\curveto(28.1175294,345.03956054)(28.1175294,344.99456059)(28.11753982,344.95456697)
\curveto(28.12752939,344.92456066)(28.13252938,344.88956069)(28.13253982,344.84956697)
\curveto(28.15252936,344.78956079)(28.15752936,344.72456086)(28.14753982,344.65456697)
\curveto(28.14752937,344.584561)(28.15752936,344.52456106)(28.17753982,344.47456697)
\lineto(28.17753982,344.30956697)
\moveto(25.83753982,343.58956697)
\curveto(25.85753166,343.63956194)(25.86753165,343.71956186)(25.86753982,343.82956697)
\curveto(25.86753165,343.93956164)(25.85753166,344.01956156)(25.83753982,344.06956697)
\lineto(25.83753982,344.35456697)
\curveto(25.8175317,344.44456114)(25.80253171,344.53956104)(25.79253982,344.63956697)
\curveto(25.79253172,344.73956084)(25.78253173,344.82956075)(25.76253982,344.90956697)
\curveto(25.74253177,344.95956062)(25.73253178,345.00456058)(25.73253982,345.04456697)
\curveto(25.74253177,345.09456049)(25.73753178,345.14456044)(25.71753982,345.19456697)
\curveto(25.66753185,345.35456023)(25.6175319,345.50456008)(25.56753982,345.64456697)
\curveto(25.52753199,345.79455979)(25.46753205,345.93455965)(25.38753982,346.06456697)
\curveto(25.23753228,346.30455928)(25.06253245,346.50955907)(24.86253982,346.67956697)
\curveto(24.67253284,346.85955872)(24.43753308,347.00955857)(24.15753982,347.12956697)
\curveto(24.06753345,347.15955842)(23.97753354,347.1845584)(23.88753982,347.20456697)
\curveto(23.79753372,347.23455835)(23.70753381,347.25955832)(23.61753982,347.27956697)
\curveto(23.53753398,347.28955829)(23.46253405,347.29455829)(23.39253982,347.29456697)
\curveto(23.33253418,347.30455828)(23.26253425,347.31955826)(23.18253982,347.33956697)
\curveto(23.14253437,347.34955823)(23.10253441,347.34955823)(23.06253982,347.33956697)
\curveto(23.02253449,347.33955824)(22.98753453,347.34455824)(22.95753982,347.35456697)
\lineto(22.62753982,347.35456697)
\curveto(22.57753494,347.36455822)(22.52253499,347.36455822)(22.46253982,347.35456697)
\lineto(22.28253982,347.35456697)
\lineto(21.60753982,347.35456697)
\curveto(21.58753593,347.33455825)(21.55253596,347.32955825)(21.50253982,347.33956697)
\curveto(21.46253605,347.34955823)(21.42753609,347.34955823)(21.39753982,347.33956697)
\lineto(21.24753982,347.27956697)
\curveto(21.19753632,347.26955831)(21.15753636,347.23955834)(21.12753982,347.18956697)
\curveto(21.08753643,347.13955844)(21.06753645,347.06955851)(21.06753982,346.97956697)
\lineto(21.06753982,346.67956697)
\curveto(21.06753645,346.54955903)(21.06253645,346.41455917)(21.05253982,346.27456697)
\lineto(21.05253982,345.85456697)
\lineto(21.05253982,341.66956697)
\curveto(21.05253646,341.60956397)(21.04753647,341.54456404)(21.03753982,341.47456697)
\curveto(21.03753648,341.40456418)(21.04753647,341.34456424)(21.06753982,341.29456697)
\lineto(21.06753982,341.14456697)
\lineto(21.06753982,340.93456697)
\curveto(21.07753644,340.87456471)(21.09253642,340.81956476)(21.11253982,340.76956697)
\curveto(21.17253634,340.64956493)(21.28753623,340.584565)(21.45753982,340.57456697)
\lineto(21.98253982,340.57456697)
\lineto(23.16753982,340.57456697)
\curveto(23.56753395,340.584565)(23.90753361,340.64456494)(24.18753982,340.75456697)
\curveto(24.55753296,340.90456468)(24.84753267,341.10456448)(25.05753982,341.35456697)
\curveto(25.27753224,341.60456398)(25.46253205,341.91456367)(25.61253982,342.28456697)
\curveto(25.65253186,342.36456322)(25.68253183,342.45456313)(25.70253982,342.55456697)
\curveto(25.72253179,342.65456293)(25.74753177,342.75456283)(25.77753982,342.85456697)
\lineto(25.77753982,342.97456697)
\curveto(25.79753172,343.04456254)(25.80753171,343.11956246)(25.80753982,343.19956697)
\curveto(25.80753171,343.2795623)(25.8175317,343.35956222)(25.83753982,343.43956697)
\lineto(25.83753982,343.58956697)
}
}
{
\newrgbcolor{curcolor}{0 0 0}
\pscustom[linestyle=none,fillstyle=solid,fillcolor=curcolor]
{
\newpath
\moveto(31.67605545,349.19956697)
\curveto(31.7460525,349.11955646)(31.78105246,348.99955658)(31.78105545,348.83956697)
\lineto(31.78105545,348.37456697)
\lineto(31.78105545,347.96956697)
\curveto(31.78105246,347.82955775)(31.7460525,347.73455785)(31.67605545,347.68456697)
\curveto(31.61605263,347.63455795)(31.53605271,347.60455798)(31.43605545,347.59456697)
\curveto(31.3460529,347.584558)(31.246053,347.579558)(31.13605545,347.57956697)
\lineto(30.29605545,347.57956697)
\curveto(30.18605406,347.579558)(30.08605416,347.584558)(29.99605545,347.59456697)
\curveto(29.91605433,347.60455798)(29.8460544,347.63455795)(29.78605545,347.68456697)
\curveto(29.7460545,347.71455787)(29.71605453,347.76955781)(29.69605545,347.84956697)
\curveto(29.68605456,347.93955764)(29.67605457,348.03455755)(29.66605545,348.13456697)
\lineto(29.66605545,348.46456697)
\curveto(29.67605457,348.57455701)(29.68105456,348.66955691)(29.68105545,348.74956697)
\lineto(29.68105545,348.95956697)
\curveto(29.69105455,349.02955655)(29.71105453,349.08955649)(29.74105545,349.13956697)
\curveto(29.76105448,349.1795564)(29.78605446,349.20955637)(29.81605545,349.22956697)
\lineto(29.93605545,349.28956697)
\curveto(29.95605429,349.28955629)(29.98105426,349.28955629)(30.01105545,349.28956697)
\curveto(30.0410542,349.29955628)(30.06605418,349.30455628)(30.08605545,349.30456697)
\lineto(31.18105545,349.30456697)
\curveto(31.28105296,349.30455628)(31.37605287,349.29955628)(31.46605545,349.28956697)
\curveto(31.55605269,349.2795563)(31.62605262,349.24955633)(31.67605545,349.19956697)
\moveto(31.78105545,339.43456697)
\curveto(31.78105246,339.23456635)(31.77605247,339.06456652)(31.76605545,338.92456697)
\curveto(31.75605249,338.7845668)(31.66605258,338.68956689)(31.49605545,338.63956697)
\curveto(31.43605281,338.61956696)(31.37105287,338.60956697)(31.30105545,338.60956697)
\curveto(31.23105301,338.61956696)(31.15605309,338.62456696)(31.07605545,338.62456697)
\lineto(30.23605545,338.62456697)
\curveto(30.1460541,338.62456696)(30.05605419,338.62956695)(29.96605545,338.63956697)
\curveto(29.88605436,338.64956693)(29.82605442,338.6795669)(29.78605545,338.72956697)
\curveto(29.72605452,338.79956678)(29.69105455,338.8845667)(29.68105545,338.98456697)
\lineto(29.68105545,339.32956697)
\lineto(29.68105545,345.65956697)
\lineto(29.68105545,345.95956697)
\curveto(29.68105456,346.05955952)(29.70105454,346.13955944)(29.74105545,346.19956697)
\curveto(29.80105444,346.26955931)(29.88605436,346.31455927)(29.99605545,346.33456697)
\curveto(30.01605423,346.34455924)(30.0410542,346.34455924)(30.07105545,346.33456697)
\curveto(30.11105413,346.33455925)(30.1410541,346.33955924)(30.16105545,346.34956697)
\lineto(30.91105545,346.34956697)
\lineto(31.10605545,346.34956697)
\curveto(31.18605306,346.35955922)(31.25105299,346.35955922)(31.30105545,346.34956697)
\lineto(31.42105545,346.34956697)
\curveto(31.48105276,346.32955925)(31.53605271,346.31455927)(31.58605545,346.30456697)
\curveto(31.63605261,346.29455929)(31.67605257,346.26455932)(31.70605545,346.21456697)
\curveto(31.7460525,346.16455942)(31.76605248,346.09455949)(31.76605545,346.00456697)
\curveto(31.77605247,345.91455967)(31.78105246,345.81955976)(31.78105545,345.71956697)
\lineto(31.78105545,339.43456697)
}
}
{
\newrgbcolor{curcolor}{0 0 0}
\pscustom[linestyle=none,fillstyle=solid,fillcolor=curcolor]
{
\newpath
\moveto(36.41324295,346.55956697)
\curveto(37.16323845,346.579559)(37.8132378,346.49455909)(38.36324295,346.30456697)
\curveto(38.92323669,346.12455946)(39.34823626,345.80955977)(39.63824295,345.35956697)
\curveto(39.7082359,345.24956033)(39.76823584,345.13456045)(39.81824295,345.01456697)
\curveto(39.87823573,344.90456068)(39.92823568,344.7795608)(39.96824295,344.63956697)
\curveto(39.98823562,344.579561)(39.99823561,344.51456107)(39.99824295,344.44456697)
\curveto(39.99823561,344.37456121)(39.98823562,344.31456127)(39.96824295,344.26456697)
\curveto(39.92823568,344.20456138)(39.87323574,344.16456142)(39.80324295,344.14456697)
\curveto(39.75323586,344.12456146)(39.69323592,344.11456147)(39.62324295,344.11456697)
\lineto(39.41324295,344.11456697)
\lineto(38.75324295,344.11456697)
\curveto(38.68323693,344.11456147)(38.613237,344.10956147)(38.54324295,344.09956697)
\curveto(38.47323714,344.09956148)(38.4082372,344.10956147)(38.34824295,344.12956697)
\curveto(38.24823736,344.14956143)(38.17323744,344.18956139)(38.12324295,344.24956697)
\curveto(38.07323754,344.30956127)(38.02823758,344.36956121)(37.98824295,344.42956697)
\lineto(37.86824295,344.63956697)
\curveto(37.83823777,344.71956086)(37.78823782,344.7845608)(37.71824295,344.83456697)
\curveto(37.61823799,344.91456067)(37.51823809,344.97456061)(37.41824295,345.01456697)
\curveto(37.32823828,345.05456053)(37.2132384,345.08956049)(37.07324295,345.11956697)
\curveto(37.00323861,345.13956044)(36.89823871,345.15456043)(36.75824295,345.16456697)
\curveto(36.62823898,345.17456041)(36.52823908,345.16956041)(36.45824295,345.14956697)
\lineto(36.35324295,345.14956697)
\lineto(36.20324295,345.11956697)
\curveto(36.16323945,345.11956046)(36.11823949,345.11456047)(36.06824295,345.10456697)
\curveto(35.89823971,345.05456053)(35.75823985,344.9845606)(35.64824295,344.89456697)
\curveto(35.54824006,344.81456077)(35.47824013,344.68956089)(35.43824295,344.51956697)
\curveto(35.41824019,344.44956113)(35.41824019,344.3845612)(35.43824295,344.32456697)
\curveto(35.45824015,344.26456132)(35.47824013,344.21456137)(35.49824295,344.17456697)
\curveto(35.56824004,344.05456153)(35.64823996,343.95956162)(35.73824295,343.88956697)
\curveto(35.83823977,343.81956176)(35.95323966,343.75956182)(36.08324295,343.70956697)
\curveto(36.27323934,343.62956195)(36.47823913,343.55956202)(36.69824295,343.49956697)
\lineto(37.38824295,343.34956697)
\curveto(37.62823798,343.30956227)(37.85823775,343.25956232)(38.07824295,343.19956697)
\curveto(38.3082373,343.14956243)(38.52323709,343.0845625)(38.72324295,343.00456697)
\curveto(38.8132368,342.96456262)(38.89823671,342.92956265)(38.97824295,342.89956697)
\curveto(39.06823654,342.8795627)(39.15323646,342.84456274)(39.23324295,342.79456697)
\curveto(39.42323619,342.67456291)(39.59323602,342.54456304)(39.74324295,342.40456697)
\curveto(39.90323571,342.26456332)(40.02823558,342.08956349)(40.11824295,341.87956697)
\curveto(40.14823546,341.80956377)(40.17323544,341.73956384)(40.19324295,341.66956697)
\curveto(40.2132354,341.59956398)(40.23323538,341.52456406)(40.25324295,341.44456697)
\curveto(40.26323535,341.3845642)(40.26823534,341.28956429)(40.26824295,341.15956697)
\curveto(40.27823533,341.03956454)(40.27823533,340.94456464)(40.26824295,340.87456697)
\lineto(40.26824295,340.79956697)
\curveto(40.24823536,340.73956484)(40.23323538,340.6795649)(40.22324295,340.61956697)
\curveto(40.22323539,340.56956501)(40.21823539,340.51956506)(40.20824295,340.46956697)
\curveto(40.13823547,340.16956541)(40.02823558,339.90456568)(39.87824295,339.67456697)
\curveto(39.71823589,339.43456615)(39.52323609,339.23956634)(39.29324295,339.08956697)
\curveto(39.06323655,338.93956664)(38.80323681,338.80956677)(38.51324295,338.69956697)
\curveto(38.40323721,338.64956693)(38.28323733,338.61456697)(38.15324295,338.59456697)
\curveto(38.03323758,338.57456701)(37.9132377,338.54956703)(37.79324295,338.51956697)
\curveto(37.70323791,338.49956708)(37.608238,338.48956709)(37.50824295,338.48956697)
\curveto(37.41823819,338.4795671)(37.32823828,338.46456712)(37.23824295,338.44456697)
\lineto(36.96824295,338.44456697)
\curveto(36.9082387,338.42456716)(36.80323881,338.41456717)(36.65324295,338.41456697)
\curveto(36.5132391,338.41456717)(36.4132392,338.42456716)(36.35324295,338.44456697)
\curveto(36.32323929,338.44456714)(36.28823932,338.44956713)(36.24824295,338.45956697)
\lineto(36.14324295,338.45956697)
\curveto(36.02323959,338.4795671)(35.90323971,338.49456709)(35.78324295,338.50456697)
\curveto(35.66323995,338.51456707)(35.54824006,338.53456705)(35.43824295,338.56456697)
\curveto(35.04824056,338.67456691)(34.70324091,338.79956678)(34.40324295,338.93956697)
\curveto(34.10324151,339.08956649)(33.84824176,339.30956627)(33.63824295,339.59956697)
\curveto(33.49824211,339.78956579)(33.37824223,340.00956557)(33.27824295,340.25956697)
\curveto(33.25824235,340.31956526)(33.23824237,340.39956518)(33.21824295,340.49956697)
\curveto(33.19824241,340.54956503)(33.18324243,340.61956496)(33.17324295,340.70956697)
\curveto(33.16324245,340.79956478)(33.16824244,340.87456471)(33.18824295,340.93456697)
\curveto(33.21824239,341.00456458)(33.26824234,341.05456453)(33.33824295,341.08456697)
\curveto(33.38824222,341.10456448)(33.44824216,341.11456447)(33.51824295,341.11456697)
\lineto(33.74324295,341.11456697)
\lineto(34.44824295,341.11456697)
\lineto(34.68824295,341.11456697)
\curveto(34.76824084,341.11456447)(34.83824077,341.10456448)(34.89824295,341.08456697)
\curveto(35.0082406,341.04456454)(35.07824053,340.9795646)(35.10824295,340.88956697)
\curveto(35.14824046,340.79956478)(35.19324042,340.70456488)(35.24324295,340.60456697)
\curveto(35.26324035,340.55456503)(35.29824031,340.48956509)(35.34824295,340.40956697)
\curveto(35.4082402,340.32956525)(35.45824015,340.2795653)(35.49824295,340.25956697)
\curveto(35.61823999,340.15956542)(35.73323988,340.0795655)(35.84324295,340.01956697)
\curveto(35.95323966,339.96956561)(36.09323952,339.91956566)(36.26324295,339.86956697)
\curveto(36.3132393,339.84956573)(36.36323925,339.83956574)(36.41324295,339.83956697)
\curveto(36.46323915,339.84956573)(36.5132391,339.84956573)(36.56324295,339.83956697)
\curveto(36.64323897,339.81956576)(36.72823888,339.80956577)(36.81824295,339.80956697)
\curveto(36.91823869,339.81956576)(37.00323861,339.83456575)(37.07324295,339.85456697)
\curveto(37.12323849,339.86456572)(37.16823844,339.86956571)(37.20824295,339.86956697)
\curveto(37.25823835,339.86956571)(37.3082383,339.8795657)(37.35824295,339.89956697)
\curveto(37.49823811,339.94956563)(37.62323799,340.00956557)(37.73324295,340.07956697)
\curveto(37.85323776,340.14956543)(37.94823766,340.23956534)(38.01824295,340.34956697)
\curveto(38.06823754,340.42956515)(38.1082375,340.55456503)(38.13824295,340.72456697)
\curveto(38.15823745,340.79456479)(38.15823745,340.85956472)(38.13824295,340.91956697)
\curveto(38.11823749,340.9795646)(38.09823751,341.02956455)(38.07824295,341.06956697)
\curveto(38.0082376,341.20956437)(37.91823769,341.31456427)(37.80824295,341.38456697)
\curveto(37.7082379,341.45456413)(37.58823802,341.51956406)(37.44824295,341.57956697)
\curveto(37.25823835,341.65956392)(37.05823855,341.72456386)(36.84824295,341.77456697)
\curveto(36.63823897,341.82456376)(36.42823918,341.8795637)(36.21824295,341.93956697)
\curveto(36.13823947,341.95956362)(36.05323956,341.97456361)(35.96324295,341.98456697)
\curveto(35.88323973,341.99456359)(35.80323981,342.00956357)(35.72324295,342.02956697)
\curveto(35.40324021,342.11956346)(35.09824051,342.20456338)(34.80824295,342.28456697)
\curveto(34.51824109,342.37456321)(34.25324136,342.50456308)(34.01324295,342.67456697)
\curveto(33.73324188,342.87456271)(33.52824208,343.14456244)(33.39824295,343.48456697)
\curveto(33.37824223,343.55456203)(33.35824225,343.64956193)(33.33824295,343.76956697)
\curveto(33.31824229,343.83956174)(33.30324231,343.92456166)(33.29324295,344.02456697)
\curveto(33.28324233,344.12456146)(33.28824232,344.21456137)(33.30824295,344.29456697)
\curveto(33.32824228,344.34456124)(33.33324228,344.3845612)(33.32324295,344.41456697)
\curveto(33.3132423,344.45456113)(33.31824229,344.49956108)(33.33824295,344.54956697)
\curveto(33.35824225,344.65956092)(33.37824223,344.75956082)(33.39824295,344.84956697)
\curveto(33.42824218,344.94956063)(33.46324215,345.04456054)(33.50324295,345.13456697)
\curveto(33.63324198,345.42456016)(33.8132418,345.65955992)(34.04324295,345.83956697)
\curveto(34.27324134,346.01955956)(34.53324108,346.16455942)(34.82324295,346.27456697)
\curveto(34.93324068,346.32455926)(35.04824056,346.35955922)(35.16824295,346.37956697)
\curveto(35.28824032,346.40955917)(35.4132402,346.43955914)(35.54324295,346.46956697)
\curveto(35.60324001,346.48955909)(35.66323995,346.49955908)(35.72324295,346.49956697)
\lineto(35.90324295,346.52956697)
\curveto(35.98323963,346.53955904)(36.06823954,346.54455904)(36.15824295,346.54456697)
\curveto(36.24823936,346.54455904)(36.33323928,346.54955903)(36.41324295,346.55956697)
}
}
{
\newrgbcolor{curcolor}{0 0 0}
\pscustom[linestyle=none,fillstyle=solid,fillcolor=curcolor]
{
\newpath
\moveto(42.54988357,348.65956697)
\lineto(43.55488357,348.65956697)
\curveto(43.70488059,348.65955692)(43.83488046,348.64955693)(43.94488357,348.62956697)
\curveto(44.06488023,348.61955696)(44.14988014,348.55955702)(44.19988357,348.44956697)
\curveto(44.21988007,348.39955718)(44.22988006,348.33955724)(44.22988357,348.26956697)
\lineto(44.22988357,348.05956697)
\lineto(44.22988357,347.38456697)
\curveto(44.22988006,347.33455825)(44.22488007,347.27455831)(44.21488357,347.20456697)
\curveto(44.21488008,347.14455844)(44.21988007,347.08955849)(44.22988357,347.03956697)
\lineto(44.22988357,346.87456697)
\curveto(44.22988006,346.79455879)(44.23488006,346.71955886)(44.24488357,346.64956697)
\curveto(44.25488004,346.58955899)(44.27988001,346.53455905)(44.31988357,346.48456697)
\curveto(44.3898799,346.39455919)(44.51487978,346.34455924)(44.69488357,346.33456697)
\lineto(45.23488357,346.33456697)
\lineto(45.41488357,346.33456697)
\curveto(45.47487882,346.33455925)(45.52987876,346.32455926)(45.57988357,346.30456697)
\curveto(45.6898786,346.25455933)(45.74987854,346.16455942)(45.75988357,346.03456697)
\curveto(45.77987851,345.90455968)(45.7898785,345.75955982)(45.78988357,345.59956697)
\lineto(45.78988357,345.38956697)
\curveto(45.79987849,345.31956026)(45.7948785,345.25956032)(45.77488357,345.20956697)
\curveto(45.72487857,345.04956053)(45.61987867,344.96456062)(45.45988357,344.95456697)
\curveto(45.29987899,344.94456064)(45.11987917,344.93956064)(44.91988357,344.93956697)
\lineto(44.78488357,344.93956697)
\curveto(44.74487955,344.94956063)(44.70987958,344.94956063)(44.67988357,344.93956697)
\curveto(44.63987965,344.92956065)(44.60487969,344.92456066)(44.57488357,344.92456697)
\curveto(44.54487975,344.93456065)(44.51487978,344.92956065)(44.48488357,344.90956697)
\curveto(44.40487989,344.88956069)(44.34487995,344.84456074)(44.30488357,344.77456697)
\curveto(44.27488002,344.71456087)(44.24988004,344.63956094)(44.22988357,344.54956697)
\curveto(44.21988007,344.49956108)(44.21988007,344.44456114)(44.22988357,344.38456697)
\curveto(44.23988005,344.32456126)(44.23988005,344.26956131)(44.22988357,344.21956697)
\lineto(44.22988357,343.28956697)
\lineto(44.22988357,341.53456697)
\curveto(44.22988006,341.2845643)(44.23488006,341.06456452)(44.24488357,340.87456697)
\curveto(44.26488003,340.69456489)(44.32987996,340.53456505)(44.43988357,340.39456697)
\curveto(44.4898798,340.33456525)(44.55487974,340.28956529)(44.63488357,340.25956697)
\lineto(44.90488357,340.19956697)
\curveto(44.93487936,340.18956539)(44.96487933,340.1845654)(44.99488357,340.18456697)
\curveto(45.03487926,340.19456539)(45.06487923,340.19456539)(45.08488357,340.18456697)
\lineto(45.24988357,340.18456697)
\curveto(45.35987893,340.1845654)(45.45487884,340.1795654)(45.53488357,340.16956697)
\curveto(45.61487868,340.15956542)(45.67987861,340.11956546)(45.72988357,340.04956697)
\curveto(45.76987852,339.98956559)(45.7898785,339.90956567)(45.78988357,339.80956697)
\lineto(45.78988357,339.52456697)
\curveto(45.7898785,339.31456627)(45.78487851,339.11956646)(45.77488357,338.93956697)
\curveto(45.77487852,338.76956681)(45.6948786,338.65456693)(45.53488357,338.59456697)
\curveto(45.48487881,338.57456701)(45.43987885,338.56956701)(45.39988357,338.57956697)
\curveto(45.35987893,338.579567)(45.31487898,338.56956701)(45.26488357,338.54956697)
\lineto(45.11488357,338.54956697)
\curveto(45.0948792,338.54956703)(45.06487923,338.55456703)(45.02488357,338.56456697)
\curveto(44.98487931,338.56456702)(44.94987934,338.55956702)(44.91988357,338.54956697)
\curveto(44.86987942,338.53956704)(44.81487948,338.53956704)(44.75488357,338.54956697)
\lineto(44.60488357,338.54956697)
\lineto(44.45488357,338.54956697)
\curveto(44.40487989,338.53956704)(44.35987993,338.53956704)(44.31988357,338.54956697)
\lineto(44.15488357,338.54956697)
\curveto(44.10488019,338.55956702)(44.04988024,338.56456702)(43.98988357,338.56456697)
\curveto(43.92988036,338.56456702)(43.87488042,338.56956701)(43.82488357,338.57956697)
\curveto(43.75488054,338.58956699)(43.6898806,338.59956698)(43.62988357,338.60956697)
\lineto(43.44988357,338.63956697)
\curveto(43.33988095,338.66956691)(43.23488106,338.70456688)(43.13488357,338.74456697)
\curveto(43.03488126,338.7845668)(42.93988135,338.82956675)(42.84988357,338.87956697)
\lineto(42.75988357,338.93956697)
\curveto(42.72988156,338.96956661)(42.6948816,338.99956658)(42.65488357,339.02956697)
\curveto(42.63488166,339.04956653)(42.60988168,339.06956651)(42.57988357,339.08956697)
\lineto(42.50488357,339.16456697)
\curveto(42.36488193,339.35456623)(42.25988203,339.56456602)(42.18988357,339.79456697)
\curveto(42.16988212,339.83456575)(42.15988213,339.86956571)(42.15988357,339.89956697)
\curveto(42.16988212,339.93956564)(42.16988212,339.9845656)(42.15988357,340.03456697)
\curveto(42.14988214,340.05456553)(42.14488215,340.0795655)(42.14488357,340.10956697)
\curveto(42.14488215,340.13956544)(42.13988215,340.16456542)(42.12988357,340.18456697)
\lineto(42.12988357,340.33456697)
\curveto(42.11988217,340.37456521)(42.11488218,340.41956516)(42.11488357,340.46956697)
\curveto(42.12488217,340.51956506)(42.12988216,340.56956501)(42.12988357,340.61956697)
\lineto(42.12988357,341.18956697)
\lineto(42.12988357,343.42456697)
\lineto(42.12988357,344.21956697)
\lineto(42.12988357,344.42956697)
\curveto(42.13988215,344.49956108)(42.13488216,344.56456102)(42.11488357,344.62456697)
\curveto(42.07488222,344.76456082)(42.00488229,344.85456073)(41.90488357,344.89456697)
\curveto(41.7948825,344.94456064)(41.65488264,344.95956062)(41.48488357,344.93956697)
\curveto(41.31488298,344.91956066)(41.16988312,344.93456065)(41.04988357,344.98456697)
\curveto(40.96988332,345.01456057)(40.91988337,345.05956052)(40.89988357,345.11956697)
\curveto(40.87988341,345.1795604)(40.85988343,345.25456033)(40.83988357,345.34456697)
\lineto(40.83988357,345.65956697)
\curveto(40.83988345,345.83955974)(40.84988344,345.9845596)(40.86988357,346.09456697)
\curveto(40.8898834,346.20455938)(40.97488332,346.2795593)(41.12488357,346.31956697)
\curveto(41.16488313,346.33955924)(41.20488309,346.34455924)(41.24488357,346.33456697)
\lineto(41.37988357,346.33456697)
\curveto(41.52988276,346.33455925)(41.66988262,346.33955924)(41.79988357,346.34956697)
\curveto(41.92988236,346.36955921)(42.01988227,346.42955915)(42.06988357,346.52956697)
\curveto(42.09988219,346.59955898)(42.11488218,346.6795589)(42.11488357,346.76956697)
\curveto(42.12488217,346.85955872)(42.12988216,346.94955863)(42.12988357,347.03956697)
\lineto(42.12988357,347.96956697)
\lineto(42.12988357,348.22456697)
\curveto(42.12988216,348.31455727)(42.13988215,348.38955719)(42.15988357,348.44956697)
\curveto(42.20988208,348.54955703)(42.28488201,348.61455697)(42.38488357,348.64456697)
\curveto(42.40488189,348.65455693)(42.42988186,348.65455693)(42.45988357,348.64456697)
\curveto(42.49988179,348.64455694)(42.52988176,348.64955693)(42.54988357,348.65956697)
}
}
{
\newrgbcolor{curcolor}{0 0 0}
\pscustom[linestyle=none,fillstyle=solid,fillcolor=curcolor]
{
\newpath
\moveto(51.19832107,346.54456697)
\curveto(51.30831576,346.54455904)(51.40331566,346.53455905)(51.48332107,346.51456697)
\curveto(51.57331549,346.49455909)(51.64331542,346.44955913)(51.69332107,346.37956697)
\curveto(51.75331531,346.29955928)(51.78331528,346.15955942)(51.78332107,345.95956697)
\lineto(51.78332107,345.44956697)
\lineto(51.78332107,345.07456697)
\curveto(51.79331527,344.93456065)(51.77831529,344.82456076)(51.73832107,344.74456697)
\curveto(51.69831537,344.67456091)(51.63831543,344.62956095)(51.55832107,344.60956697)
\curveto(51.48831558,344.58956099)(51.40331566,344.579561)(51.30332107,344.57956697)
\curveto(51.21331585,344.579561)(51.11331595,344.584561)(51.00332107,344.59456697)
\curveto(50.90331616,344.60456098)(50.80831626,344.59956098)(50.71832107,344.57956697)
\curveto(50.64831642,344.55956102)(50.57831649,344.54456104)(50.50832107,344.53456697)
\curveto(50.43831663,344.53456105)(50.37331669,344.52456106)(50.31332107,344.50456697)
\curveto(50.15331691,344.45456113)(49.99331707,344.3795612)(49.83332107,344.27956697)
\curveto(49.67331739,344.18956139)(49.54831752,344.0845615)(49.45832107,343.96456697)
\curveto(49.40831766,343.8845617)(49.35331771,343.79956178)(49.29332107,343.70956697)
\curveto(49.24331782,343.62956195)(49.19331787,343.54456204)(49.14332107,343.45456697)
\curveto(49.11331795,343.37456221)(49.08331798,343.28956229)(49.05332107,343.19956697)
\lineto(48.99332107,342.95956697)
\curveto(48.97331809,342.88956269)(48.9633181,342.81456277)(48.96332107,342.73456697)
\curveto(48.9633181,342.66456292)(48.95331811,342.59456299)(48.93332107,342.52456697)
\curveto(48.92331814,342.4845631)(48.91831815,342.44456314)(48.91832107,342.40456697)
\curveto(48.92831814,342.37456321)(48.92831814,342.34456324)(48.91832107,342.31456697)
\lineto(48.91832107,342.07456697)
\curveto(48.89831817,342.00456358)(48.89331817,341.92456366)(48.90332107,341.83456697)
\curveto(48.91331815,341.75456383)(48.91831815,341.67456391)(48.91832107,341.59456697)
\lineto(48.91832107,340.63456697)
\lineto(48.91832107,339.35956697)
\curveto(48.91831815,339.22956635)(48.91331815,339.10956647)(48.90332107,338.99956697)
\curveto(48.89331817,338.88956669)(48.8633182,338.79956678)(48.81332107,338.72956697)
\curveto(48.79331827,338.69956688)(48.75831831,338.67456691)(48.70832107,338.65456697)
\curveto(48.6683184,338.64456694)(48.62331844,338.63456695)(48.57332107,338.62456697)
\lineto(48.49832107,338.62456697)
\curveto(48.44831862,338.61456697)(48.39331867,338.60956697)(48.33332107,338.60956697)
\lineto(48.16832107,338.60956697)
\lineto(47.52332107,338.60956697)
\curveto(47.4633196,338.61956696)(47.39831967,338.62456696)(47.32832107,338.62456697)
\lineto(47.13332107,338.62456697)
\curveto(47.08331998,338.64456694)(47.03332003,338.65956692)(46.98332107,338.66956697)
\curveto(46.93332013,338.68956689)(46.89832017,338.72456686)(46.87832107,338.77456697)
\curveto(46.83832023,338.82456676)(46.81332025,338.89456669)(46.80332107,338.98456697)
\lineto(46.80332107,339.28456697)
\lineto(46.80332107,340.30456697)
\lineto(46.80332107,344.53456697)
\lineto(46.80332107,345.64456697)
\lineto(46.80332107,345.92956697)
\curveto(46.80332026,346.02955955)(46.82332024,346.10955947)(46.86332107,346.16956697)
\curveto(46.91332015,346.24955933)(46.98832008,346.29955928)(47.08832107,346.31956697)
\curveto(47.18831988,346.33955924)(47.30831976,346.34955923)(47.44832107,346.34956697)
\lineto(48.21332107,346.34956697)
\curveto(48.33331873,346.34955923)(48.43831863,346.33955924)(48.52832107,346.31956697)
\curveto(48.61831845,346.30955927)(48.68831838,346.26455932)(48.73832107,346.18456697)
\curveto(48.7683183,346.13455945)(48.78331828,346.06455952)(48.78332107,345.97456697)
\lineto(48.81332107,345.70456697)
\curveto(48.82331824,345.62455996)(48.83831823,345.54956003)(48.85832107,345.47956697)
\curveto(48.88831818,345.40956017)(48.93831813,345.37456021)(49.00832107,345.37456697)
\curveto(49.02831804,345.39456019)(49.04831802,345.40456018)(49.06832107,345.40456697)
\curveto(49.08831798,345.40456018)(49.10831796,345.41456017)(49.12832107,345.43456697)
\curveto(49.18831788,345.4845601)(49.23831783,345.53956004)(49.27832107,345.59956697)
\curveto(49.32831774,345.66955991)(49.38831768,345.72955985)(49.45832107,345.77956697)
\curveto(49.49831757,345.80955977)(49.53331753,345.83955974)(49.56332107,345.86956697)
\curveto(49.59331747,345.90955967)(49.62831744,345.94455964)(49.66832107,345.97456697)
\lineto(49.93832107,346.15456697)
\curveto(50.03831703,346.21455937)(50.13831693,346.26955931)(50.23832107,346.31956697)
\curveto(50.33831673,346.35955922)(50.43831663,346.39455919)(50.53832107,346.42456697)
\lineto(50.86832107,346.51456697)
\curveto(50.89831617,346.52455906)(50.95331611,346.52455906)(51.03332107,346.51456697)
\curveto(51.12331594,346.51455907)(51.17831589,346.52455906)(51.19832107,346.54456697)
}
}
{
\newrgbcolor{curcolor}{0 0 0}
\pscustom[linestyle=none,fillstyle=solid,fillcolor=curcolor]
{
\newpath
\moveto(54.7033992,349.19956697)
\curveto(54.77339625,349.11955646)(54.80839621,348.99955658)(54.8083992,348.83956697)
\lineto(54.8083992,348.37456697)
\lineto(54.8083992,347.96956697)
\curveto(54.80839621,347.82955775)(54.77339625,347.73455785)(54.7033992,347.68456697)
\curveto(54.64339638,347.63455795)(54.56339646,347.60455798)(54.4633992,347.59456697)
\curveto(54.37339665,347.584558)(54.27339675,347.579558)(54.1633992,347.57956697)
\lineto(53.3233992,347.57956697)
\curveto(53.21339781,347.579558)(53.11339791,347.584558)(53.0233992,347.59456697)
\curveto(52.94339808,347.60455798)(52.87339815,347.63455795)(52.8133992,347.68456697)
\curveto(52.77339825,347.71455787)(52.74339828,347.76955781)(52.7233992,347.84956697)
\curveto(52.71339831,347.93955764)(52.70339832,348.03455755)(52.6933992,348.13456697)
\lineto(52.6933992,348.46456697)
\curveto(52.70339832,348.57455701)(52.70839831,348.66955691)(52.7083992,348.74956697)
\lineto(52.7083992,348.95956697)
\curveto(52.7183983,349.02955655)(52.73839828,349.08955649)(52.7683992,349.13956697)
\curveto(52.78839823,349.1795564)(52.81339821,349.20955637)(52.8433992,349.22956697)
\lineto(52.9633992,349.28956697)
\curveto(52.98339804,349.28955629)(53.00839801,349.28955629)(53.0383992,349.28956697)
\curveto(53.06839795,349.29955628)(53.09339793,349.30455628)(53.1133992,349.30456697)
\lineto(54.2083992,349.30456697)
\curveto(54.30839671,349.30455628)(54.40339662,349.29955628)(54.4933992,349.28956697)
\curveto(54.58339644,349.2795563)(54.65339637,349.24955633)(54.7033992,349.19956697)
\moveto(54.8083992,339.43456697)
\curveto(54.80839621,339.23456635)(54.80339622,339.06456652)(54.7933992,338.92456697)
\curveto(54.78339624,338.7845668)(54.69339633,338.68956689)(54.5233992,338.63956697)
\curveto(54.46339656,338.61956696)(54.39839662,338.60956697)(54.3283992,338.60956697)
\curveto(54.25839676,338.61956696)(54.18339684,338.62456696)(54.1033992,338.62456697)
\lineto(53.2633992,338.62456697)
\curveto(53.17339785,338.62456696)(53.08339794,338.62956695)(52.9933992,338.63956697)
\curveto(52.91339811,338.64956693)(52.85339817,338.6795669)(52.8133992,338.72956697)
\curveto(52.75339827,338.79956678)(52.7183983,338.8845667)(52.7083992,338.98456697)
\lineto(52.7083992,339.32956697)
\lineto(52.7083992,345.65956697)
\lineto(52.7083992,345.95956697)
\curveto(52.70839831,346.05955952)(52.72839829,346.13955944)(52.7683992,346.19956697)
\curveto(52.82839819,346.26955931)(52.91339811,346.31455927)(53.0233992,346.33456697)
\curveto(53.04339798,346.34455924)(53.06839795,346.34455924)(53.0983992,346.33456697)
\curveto(53.13839788,346.33455925)(53.16839785,346.33955924)(53.1883992,346.34956697)
\lineto(53.9383992,346.34956697)
\lineto(54.1333992,346.34956697)
\curveto(54.21339681,346.35955922)(54.27839674,346.35955922)(54.3283992,346.34956697)
\lineto(54.4483992,346.34956697)
\curveto(54.50839651,346.32955925)(54.56339646,346.31455927)(54.6133992,346.30456697)
\curveto(54.66339636,346.29455929)(54.70339632,346.26455932)(54.7333992,346.21456697)
\curveto(54.77339625,346.16455942)(54.79339623,346.09455949)(54.7933992,346.00456697)
\curveto(54.80339622,345.91455967)(54.80839621,345.81955976)(54.8083992,345.71956697)
\lineto(54.8083992,339.43456697)
}
}
{
\newrgbcolor{curcolor}{0 0 0}
\pscustom[linestyle=none,fillstyle=solid,fillcolor=curcolor]
{
\newpath
\moveto(64.2705867,342.86956697)
\curveto(64.2905781,342.80956277)(64.30057809,342.70456288)(64.3005867,342.55456697)
\curveto(64.30057809,342.41456317)(64.29557809,342.31456327)(64.2855867,342.25456697)
\curveto(64.2855781,342.20456338)(64.28057811,342.15956342)(64.2705867,342.11956697)
\lineto(64.2705867,341.99956697)
\curveto(64.25057814,341.91956366)(64.24057815,341.83956374)(64.2405867,341.75956697)
\curveto(64.24057815,341.68956389)(64.23057816,341.61456397)(64.2105867,341.53456697)
\curveto(64.21057818,341.49456409)(64.20057819,341.42456416)(64.1805867,341.32456697)
\curveto(64.15057824,341.20456438)(64.12057827,341.0795645)(64.0905867,340.94956697)
\curveto(64.07057832,340.82956475)(64.03557835,340.71456487)(63.9855867,340.60456697)
\curveto(63.80557858,340.15456543)(63.58057881,339.76456582)(63.3105867,339.43456697)
\curveto(63.04057935,339.10456648)(62.6855797,338.84456674)(62.2455867,338.65456697)
\curveto(62.15558023,338.61456697)(62.06058033,338.584567)(61.9605867,338.56456697)
\curveto(61.87058052,338.53456705)(61.77058062,338.50456708)(61.6605867,338.47456697)
\curveto(61.60058079,338.45456713)(61.53558085,338.44456714)(61.4655867,338.44456697)
\curveto(61.40558098,338.44456714)(61.34558104,338.43956714)(61.2855867,338.42956697)
\lineto(61.1505867,338.42956697)
\curveto(61.0905813,338.40956717)(61.01058138,338.40456718)(60.9105867,338.41456697)
\curveto(60.81058158,338.41456717)(60.73058166,338.42456716)(60.6705867,338.44456697)
\lineto(60.5805867,338.44456697)
\curveto(60.53058186,338.45456713)(60.47558191,338.46456712)(60.4155867,338.47456697)
\curveto(60.35558203,338.47456711)(60.29558209,338.4795671)(60.2355867,338.48956697)
\curveto(60.04558234,338.53956704)(59.87058252,338.58956699)(59.7105867,338.63956697)
\curveto(59.55058284,338.68956689)(59.40058299,338.75956682)(59.2605867,338.84956697)
\lineto(59.0805867,338.96956697)
\curveto(59.03058336,339.00956657)(58.98058341,339.05456653)(58.9305867,339.10456697)
\lineto(58.8405867,339.16456697)
\curveto(58.81058358,339.1845664)(58.78058361,339.19956638)(58.7505867,339.20956697)
\curveto(58.66058373,339.23956634)(58.60558378,339.21956636)(58.5855867,339.14956697)
\curveto(58.53558385,339.0795665)(58.50058389,338.99456659)(58.4805867,338.89456697)
\curveto(58.47058392,338.80456678)(58.43558395,338.73456685)(58.3755867,338.68456697)
\curveto(58.31558407,338.64456694)(58.24558414,338.61956696)(58.1655867,338.60956697)
\lineto(57.8955867,338.60956697)
\lineto(57.1755867,338.60956697)
\lineto(56.9505867,338.60956697)
\curveto(56.88058551,338.59956698)(56.81558557,338.60456698)(56.7555867,338.62456697)
\curveto(56.61558577,338.67456691)(56.53558585,338.76456682)(56.5155867,338.89456697)
\curveto(56.50558588,339.03456655)(56.50058589,339.18956639)(56.5005867,339.35956697)
\lineto(56.5005867,348.50956697)
\lineto(56.5005867,348.85456697)
\curveto(56.50058589,348.97455661)(56.52558586,349.06955651)(56.5755867,349.13956697)
\curveto(56.61558577,349.20955637)(56.6855857,349.25455633)(56.7855867,349.27456697)
\curveto(56.80558558,349.2845563)(56.82558556,349.2845563)(56.8455867,349.27456697)
\curveto(56.87558551,349.27455631)(56.90058549,349.2795563)(56.9205867,349.28956697)
\lineto(57.8655867,349.28956697)
\curveto(58.04558434,349.28955629)(58.20058419,349.2795563)(58.3305867,349.25956697)
\curveto(58.46058393,349.24955633)(58.54558384,349.17455641)(58.5855867,349.03456697)
\curveto(58.61558377,348.93455665)(58.62558376,348.79955678)(58.6155867,348.62956697)
\curveto(58.60558378,348.46955711)(58.60058379,348.32955725)(58.6005867,348.20956697)
\lineto(58.6005867,346.57456697)
\lineto(58.6005867,346.24456697)
\curveto(58.60058379,346.13455945)(58.61058378,346.03955954)(58.6305867,345.95956697)
\curveto(58.64058375,345.90955967)(58.65058374,345.86455972)(58.6605867,345.82456697)
\curveto(58.67058372,345.79455979)(58.69558369,345.77455981)(58.7355867,345.76456697)
\curveto(58.75558363,345.74455984)(58.78058361,345.73455985)(58.8105867,345.73456697)
\curveto(58.85058354,345.73455985)(58.88058351,345.73955984)(58.9005867,345.74956697)
\curveto(58.97058342,345.78955979)(59.03558335,345.82955975)(59.0955867,345.86956697)
\curveto(59.15558323,345.91955966)(59.22058317,345.96955961)(59.2905867,346.01956697)
\curveto(59.42058297,346.10955947)(59.55558283,346.1845594)(59.6955867,346.24456697)
\curveto(59.83558255,346.31455927)(59.9905824,346.37455921)(60.1605867,346.42456697)
\curveto(60.24058215,346.45455913)(60.32058207,346.46955911)(60.4005867,346.46956697)
\curveto(60.48058191,346.4795591)(60.56058183,346.49455909)(60.6405867,346.51456697)
\curveto(60.71058168,346.53455905)(60.7855816,346.54455904)(60.8655867,346.54456697)
\lineto(61.1055867,346.54456697)
\lineto(61.2555867,346.54456697)
\curveto(61.2855811,346.53455905)(61.32058107,346.52955905)(61.3605867,346.52956697)
\curveto(61.40058099,346.53955904)(61.44058095,346.53955904)(61.4805867,346.52956697)
\curveto(61.5905808,346.49955908)(61.6905807,346.47455911)(61.7805867,346.45456697)
\curveto(61.88058051,346.44455914)(61.97558041,346.41955916)(62.0655867,346.37956697)
\curveto(62.52557986,346.18955939)(62.90057949,345.94455964)(63.1905867,345.64456697)
\curveto(63.48057891,345.34456024)(63.72557866,344.96956061)(63.9255867,344.51956697)
\curveto(63.97557841,344.39956118)(64.01557837,344.27456131)(64.0455867,344.14456697)
\curveto(64.0855783,344.01456157)(64.12557826,343.8795617)(64.1655867,343.73956697)
\curveto(64.1855782,343.66956191)(64.19557819,343.59956198)(64.1955867,343.52956697)
\curveto(64.20557818,343.46956211)(64.22057817,343.39956218)(64.2405867,343.31956697)
\curveto(64.26057813,343.26956231)(64.26557812,343.21456237)(64.2555867,343.15456697)
\curveto(64.25557813,343.09456249)(64.26057813,343.03456255)(64.2705867,342.97456697)
\lineto(64.2705867,342.86956697)
\moveto(62.0505867,341.45956697)
\curveto(62.08058031,341.55956402)(62.10558028,341.6845639)(62.1255867,341.83456697)
\curveto(62.15558023,341.9845636)(62.17058022,342.13456345)(62.1705867,342.28456697)
\curveto(62.18058021,342.44456314)(62.18058021,342.59956298)(62.1705867,342.74956697)
\curveto(62.17058022,342.90956267)(62.15558023,343.04456254)(62.1255867,343.15456697)
\curveto(62.09558029,343.25456233)(62.07558031,343.34956223)(62.0655867,343.43956697)
\curveto(62.05558033,343.52956205)(62.03058036,343.61456197)(61.9905867,343.69456697)
\curveto(61.85058054,344.04456154)(61.65058074,344.33956124)(61.3905867,344.57956697)
\curveto(61.14058125,344.82956075)(60.77058162,344.95456063)(60.2805867,344.95456697)
\curveto(60.24058215,344.95456063)(60.20558218,344.94956063)(60.1755867,344.93956697)
\lineto(60.0705867,344.93956697)
\curveto(60.00058239,344.91956066)(59.93558245,344.89956068)(59.8755867,344.87956697)
\curveto(59.81558257,344.86956071)(59.75558263,344.85456073)(59.6955867,344.83456697)
\curveto(59.40558298,344.70456088)(59.1855832,344.51956106)(59.0355867,344.27956697)
\curveto(58.8855835,344.04956153)(58.76058363,343.7845618)(58.6605867,343.48456697)
\curveto(58.63058376,343.40456218)(58.61058378,343.31956226)(58.6005867,343.22956697)
\curveto(58.60058379,343.14956243)(58.5905838,343.06956251)(58.5705867,342.98956697)
\curveto(58.56058383,342.95956262)(58.55558383,342.90956267)(58.5555867,342.83956697)
\curveto(58.54558384,342.79956278)(58.54058385,342.75956282)(58.5405867,342.71956697)
\curveto(58.55058384,342.6795629)(58.55058384,342.63956294)(58.5405867,342.59956697)
\curveto(58.52058387,342.51956306)(58.51558387,342.40956317)(58.5255867,342.26956697)
\curveto(58.53558385,342.12956345)(58.55058384,342.02956355)(58.5705867,341.96956697)
\curveto(58.5905838,341.8795637)(58.60058379,341.79456379)(58.6005867,341.71456697)
\curveto(58.61058378,341.63456395)(58.63058376,341.55456403)(58.6605867,341.47456697)
\curveto(58.75058364,341.19456439)(58.85558353,340.94956463)(58.9755867,340.73956697)
\curveto(59.10558328,340.53956504)(59.2855831,340.36956521)(59.5155867,340.22956697)
\curveto(59.67558271,340.12956545)(59.84058255,340.05956552)(60.0105867,340.01956697)
\curveto(60.03058236,340.01956556)(60.05058234,340.01456557)(60.0705867,340.00456697)
\lineto(60.1605867,340.00456697)
\curveto(60.1905822,339.99456559)(60.24058215,339.9845656)(60.3105867,339.97456697)
\curveto(60.38058201,339.97456561)(60.44058195,339.9795656)(60.4905867,339.98956697)
\curveto(60.5905818,340.00956557)(60.68058171,340.02456556)(60.7605867,340.03456697)
\curveto(60.85058154,340.05456553)(60.93558145,340.0795655)(61.0155867,340.10956697)
\curveto(61.29558109,340.23956534)(61.51058088,340.41956516)(61.6605867,340.64956697)
\curveto(61.82058057,340.8795647)(61.95058044,341.14956443)(62.0505867,341.45956697)
}
}
{
\newrgbcolor{curcolor}{0 0 0}
\pscustom[linestyle=none,fillstyle=solid,fillcolor=curcolor]
{
\newpath
\moveto(66.06050857,346.33456697)
\lineto(67.18550857,346.33456697)
\curveto(67.29550614,346.33455925)(67.39550604,346.32955925)(67.48550857,346.31956697)
\curveto(67.57550586,346.30955927)(67.64050579,346.27455931)(67.68050857,346.21456697)
\curveto(67.7305057,346.15455943)(67.76050567,346.06955951)(67.77050857,345.95956697)
\curveto(67.78050565,345.85955972)(67.78550565,345.75455983)(67.78550857,345.64456697)
\lineto(67.78550857,344.59456697)
\lineto(67.78550857,342.35956697)
\curveto(67.78550565,341.99956358)(67.80050563,341.65956392)(67.83050857,341.33956697)
\curveto(67.86050557,341.01956456)(67.95050548,340.75456483)(68.10050857,340.54456697)
\curveto(68.24050519,340.33456525)(68.46550497,340.1845654)(68.77550857,340.09456697)
\curveto(68.82550461,340.0845655)(68.86550457,340.0795655)(68.89550857,340.07956697)
\curveto(68.9355045,340.0795655)(68.98050445,340.07456551)(69.03050857,340.06456697)
\curveto(69.08050435,340.05456553)(69.1355043,340.04956553)(69.19550857,340.04956697)
\curveto(69.25550418,340.04956553)(69.30050413,340.05456553)(69.33050857,340.06456697)
\curveto(69.38050405,340.0845655)(69.42050401,340.08956549)(69.45050857,340.07956697)
\curveto(69.49050394,340.06956551)(69.5305039,340.07456551)(69.57050857,340.09456697)
\curveto(69.78050365,340.14456544)(69.94550349,340.20956537)(70.06550857,340.28956697)
\curveto(70.24550319,340.39956518)(70.38550305,340.53956504)(70.48550857,340.70956697)
\curveto(70.59550284,340.88956469)(70.67050276,341.0845645)(70.71050857,341.29456697)
\curveto(70.76050267,341.51456407)(70.79050264,341.75456383)(70.80050857,342.01456697)
\curveto(70.81050262,342.2845633)(70.81550262,342.56456302)(70.81550857,342.85456697)
\lineto(70.81550857,344.66956697)
\lineto(70.81550857,345.64456697)
\lineto(70.81550857,345.91456697)
\curveto(70.81550262,346.01455957)(70.8355026,346.09455949)(70.87550857,346.15456697)
\curveto(70.92550251,346.24455934)(71.00050243,346.29455929)(71.10050857,346.30456697)
\curveto(71.20050223,346.32455926)(71.32050211,346.33455925)(71.46050857,346.33456697)
\lineto(72.25550857,346.33456697)
\lineto(72.54050857,346.33456697)
\curveto(72.6305008,346.33455925)(72.70550073,346.31455927)(72.76550857,346.27456697)
\curveto(72.84550059,346.22455936)(72.89050054,346.14955943)(72.90050857,346.04956697)
\curveto(72.91050052,345.94955963)(72.91550052,345.83455975)(72.91550857,345.70456697)
\lineto(72.91550857,344.56456697)
\lineto(72.91550857,340.34956697)
\lineto(72.91550857,339.28456697)
\lineto(72.91550857,338.98456697)
\curveto(72.91550052,338.8845667)(72.89550054,338.80956677)(72.85550857,338.75956697)
\curveto(72.80550063,338.6795669)(72.7305007,338.63456695)(72.63050857,338.62456697)
\curveto(72.5305009,338.61456697)(72.42550101,338.60956697)(72.31550857,338.60956697)
\lineto(71.50550857,338.60956697)
\curveto(71.39550204,338.60956697)(71.29550214,338.61456697)(71.20550857,338.62456697)
\curveto(71.12550231,338.63456695)(71.06050237,338.67456691)(71.01050857,338.74456697)
\curveto(70.99050244,338.77456681)(70.97050246,338.81956676)(70.95050857,338.87956697)
\curveto(70.94050249,338.93956664)(70.92550251,338.99956658)(70.90550857,339.05956697)
\curveto(70.89550254,339.11956646)(70.88050255,339.17456641)(70.86050857,339.22456697)
\curveto(70.84050259,339.27456631)(70.81050262,339.30456628)(70.77050857,339.31456697)
\curveto(70.75050268,339.33456625)(70.72550271,339.33956624)(70.69550857,339.32956697)
\curveto(70.66550277,339.31956626)(70.64050279,339.30956627)(70.62050857,339.29956697)
\curveto(70.55050288,339.25956632)(70.49050294,339.21456637)(70.44050857,339.16456697)
\curveto(70.39050304,339.11456647)(70.3355031,339.06956651)(70.27550857,339.02956697)
\curveto(70.2355032,338.99956658)(70.19550324,338.96456662)(70.15550857,338.92456697)
\curveto(70.12550331,338.89456669)(70.08550335,338.86456672)(70.03550857,338.83456697)
\curveto(69.80550363,338.69456689)(69.5355039,338.584567)(69.22550857,338.50456697)
\curveto(69.15550428,338.4845671)(69.08550435,338.47456711)(69.01550857,338.47456697)
\curveto(68.94550449,338.46456712)(68.87050456,338.44956713)(68.79050857,338.42956697)
\curveto(68.75050468,338.41956716)(68.70550473,338.41956716)(68.65550857,338.42956697)
\curveto(68.61550482,338.42956715)(68.57550486,338.42456716)(68.53550857,338.41456697)
\curveto(68.50550493,338.40456718)(68.44050499,338.40456718)(68.34050857,338.41456697)
\curveto(68.25050518,338.41456717)(68.19050524,338.41956716)(68.16050857,338.42956697)
\curveto(68.11050532,338.42956715)(68.06050537,338.43456715)(68.01050857,338.44456697)
\lineto(67.86050857,338.44456697)
\curveto(67.74050569,338.47456711)(67.62550581,338.49956708)(67.51550857,338.51956697)
\curveto(67.40550603,338.53956704)(67.29550614,338.56956701)(67.18550857,338.60956697)
\curveto(67.1355063,338.62956695)(67.09050634,338.64456694)(67.05050857,338.65456697)
\curveto(67.02050641,338.67456691)(66.98050645,338.69456689)(66.93050857,338.71456697)
\curveto(66.58050685,338.90456668)(66.30050713,339.16956641)(66.09050857,339.50956697)
\curveto(65.96050747,339.71956586)(65.86550757,339.96956561)(65.80550857,340.25956697)
\curveto(65.74550769,340.55956502)(65.70550773,340.87456471)(65.68550857,341.20456697)
\curveto(65.67550776,341.54456404)(65.67050776,341.88956369)(65.67050857,342.23956697)
\curveto(65.68050775,342.59956298)(65.68550775,342.95456263)(65.68550857,343.30456697)
\lineto(65.68550857,345.34456697)
\curveto(65.68550775,345.47456011)(65.68050775,345.62455996)(65.67050857,345.79456697)
\curveto(65.67050776,345.97455961)(65.69550774,346.10455948)(65.74550857,346.18456697)
\curveto(65.77550766,346.23455935)(65.8355076,346.2795593)(65.92550857,346.31956697)
\curveto(65.98550745,346.31955926)(66.0305074,346.32455926)(66.06050857,346.33456697)
}
}
{
\newrgbcolor{curcolor}{0 0 0}
\pscustom[linestyle=none,fillstyle=solid,fillcolor=curcolor]
{
\newpath
\moveto(78.11675857,346.55956697)
\curveto(78.92675341,346.579559)(79.60175274,346.45955912)(80.14175857,346.19956697)
\curveto(80.69175165,345.93955964)(81.12675121,345.56956001)(81.44675857,345.08956697)
\curveto(81.60675073,344.84956073)(81.72675061,344.57456101)(81.80675857,344.26456697)
\curveto(81.82675051,344.21456137)(81.8417505,344.14956143)(81.85175857,344.06956697)
\curveto(81.87175047,343.98956159)(81.87175047,343.91956166)(81.85175857,343.85956697)
\curveto(81.81175053,343.74956183)(81.7417506,343.6845619)(81.64175857,343.66456697)
\curveto(81.5417508,343.65456193)(81.42175092,343.64956193)(81.28175857,343.64956697)
\lineto(80.50175857,343.64956697)
\lineto(80.21675857,343.64956697)
\curveto(80.12675221,343.64956193)(80.05175229,343.66956191)(79.99175857,343.70956697)
\curveto(79.91175243,343.74956183)(79.85675248,343.80956177)(79.82675857,343.88956697)
\curveto(79.79675254,343.9795616)(79.75675258,344.06956151)(79.70675857,344.15956697)
\curveto(79.64675269,344.26956131)(79.58175276,344.36956121)(79.51175857,344.45956697)
\curveto(79.4417529,344.54956103)(79.36175298,344.62956095)(79.27175857,344.69956697)
\curveto(79.13175321,344.78956079)(78.97675336,344.85956072)(78.80675857,344.90956697)
\curveto(78.74675359,344.92956065)(78.68675365,344.93956064)(78.62675857,344.93956697)
\curveto(78.56675377,344.93956064)(78.51175383,344.94956063)(78.46175857,344.96956697)
\lineto(78.31175857,344.96956697)
\curveto(78.11175423,344.96956061)(77.95175439,344.94956063)(77.83175857,344.90956697)
\curveto(77.5417548,344.81956076)(77.30675503,344.6795609)(77.12675857,344.48956697)
\curveto(76.94675539,344.30956127)(76.80175554,344.08956149)(76.69175857,343.82956697)
\curveto(76.6417557,343.71956186)(76.60175574,343.59956198)(76.57175857,343.46956697)
\curveto(76.55175579,343.34956223)(76.52675581,343.21956236)(76.49675857,343.07956697)
\curveto(76.48675585,343.03956254)(76.48175586,342.99956258)(76.48175857,342.95956697)
\curveto(76.48175586,342.91956266)(76.47675586,342.8795627)(76.46675857,342.83956697)
\curveto(76.44675589,342.73956284)(76.4367559,342.59956298)(76.43675857,342.41956697)
\curveto(76.44675589,342.23956334)(76.46175588,342.09956348)(76.48175857,341.99956697)
\curveto(76.48175586,341.91956366)(76.48675585,341.86456372)(76.49675857,341.83456697)
\curveto(76.51675582,341.76456382)(76.52675581,341.69456389)(76.52675857,341.62456697)
\curveto(76.5367558,341.55456403)(76.55175579,341.4845641)(76.57175857,341.41456697)
\curveto(76.65175569,341.1845644)(76.74675559,340.97456461)(76.85675857,340.78456697)
\curveto(76.96675537,340.59456499)(77.10675523,340.43456515)(77.27675857,340.30456697)
\curveto(77.31675502,340.27456531)(77.37675496,340.23956534)(77.45675857,340.19956697)
\curveto(77.56675477,340.12956545)(77.67675466,340.0845655)(77.78675857,340.06456697)
\curveto(77.90675443,340.04456554)(78.05175429,340.02456556)(78.22175857,340.00456697)
\lineto(78.31175857,340.00456697)
\curveto(78.35175399,340.00456558)(78.38175396,340.00956557)(78.40175857,340.01956697)
\lineto(78.53675857,340.01956697)
\curveto(78.60675373,340.03956554)(78.67175367,340.05456553)(78.73175857,340.06456697)
\curveto(78.80175354,340.0845655)(78.86675347,340.10456548)(78.92675857,340.12456697)
\curveto(79.22675311,340.25456533)(79.45675288,340.44456514)(79.61675857,340.69456697)
\curveto(79.65675268,340.74456484)(79.69175265,340.79956478)(79.72175857,340.85956697)
\curveto(79.75175259,340.92956465)(79.77675256,340.98956459)(79.79675857,341.03956697)
\curveto(79.8367525,341.14956443)(79.87175247,341.24456434)(79.90175857,341.32456697)
\curveto(79.93175241,341.41456417)(80.00175234,341.4845641)(80.11175857,341.53456697)
\curveto(80.20175214,341.57456401)(80.34675199,341.58956399)(80.54675857,341.57956697)
\lineto(81.04175857,341.57956697)
\lineto(81.25175857,341.57956697)
\curveto(81.33175101,341.58956399)(81.39675094,341.584564)(81.44675857,341.56456697)
\lineto(81.56675857,341.56456697)
\lineto(81.68675857,341.53456697)
\curveto(81.72675061,341.53456405)(81.75675058,341.52456406)(81.77675857,341.50456697)
\curveto(81.82675051,341.46456412)(81.85675048,341.40456418)(81.86675857,341.32456697)
\curveto(81.88675045,341.25456433)(81.88675045,341.1795644)(81.86675857,341.09956697)
\curveto(81.77675056,340.76956481)(81.66675067,340.47456511)(81.53675857,340.21456697)
\curveto(81.12675121,339.44456614)(80.47175187,338.90956667)(79.57175857,338.60956697)
\curveto(79.47175287,338.579567)(79.36675297,338.55956702)(79.25675857,338.54956697)
\curveto(79.14675319,338.52956705)(79.0367533,338.50456708)(78.92675857,338.47456697)
\curveto(78.86675347,338.46456712)(78.80675353,338.45956712)(78.74675857,338.45956697)
\curveto(78.68675365,338.45956712)(78.62675371,338.45456713)(78.56675857,338.44456697)
\lineto(78.40175857,338.44456697)
\curveto(78.35175399,338.42456716)(78.27675406,338.41956716)(78.17675857,338.42956697)
\curveto(78.07675426,338.42956715)(78.00175434,338.43456715)(77.95175857,338.44456697)
\curveto(77.87175447,338.46456712)(77.79675454,338.47456711)(77.72675857,338.47456697)
\curveto(77.66675467,338.46456712)(77.60175474,338.46956711)(77.53175857,338.48956697)
\lineto(77.38175857,338.51956697)
\curveto(77.33175501,338.51956706)(77.28175506,338.52456706)(77.23175857,338.53456697)
\curveto(77.12175522,338.56456702)(77.01675532,338.59456699)(76.91675857,338.62456697)
\curveto(76.81675552,338.65456693)(76.72175562,338.68956689)(76.63175857,338.72956697)
\curveto(76.16175618,338.92956665)(75.76675657,339.1845664)(75.44675857,339.49456697)
\curveto(75.12675721,339.81456577)(74.86675747,340.20956537)(74.66675857,340.67956697)
\curveto(74.61675772,340.76956481)(74.57675776,340.86456472)(74.54675857,340.96456697)
\lineto(74.45675857,341.29456697)
\curveto(74.44675789,341.33456425)(74.4417579,341.36956421)(74.44175857,341.39956697)
\curveto(74.4417579,341.43956414)(74.43175791,341.4845641)(74.41175857,341.53456697)
\curveto(74.39175795,341.60456398)(74.38175796,341.67456391)(74.38175857,341.74456697)
\curveto(74.38175796,341.82456376)(74.37175797,341.89956368)(74.35175857,341.96956697)
\lineto(74.35175857,342.22456697)
\curveto(74.33175801,342.27456331)(74.32175802,342.32956325)(74.32175857,342.38956697)
\curveto(74.32175802,342.45956312)(74.33175801,342.51956306)(74.35175857,342.56956697)
\curveto(74.36175798,342.61956296)(74.36175798,342.66456292)(74.35175857,342.70456697)
\curveto(74.341758,342.74456284)(74.341758,342.7845628)(74.35175857,342.82456697)
\curveto(74.37175797,342.89456269)(74.37675796,342.95956262)(74.36675857,343.01956697)
\curveto(74.36675797,343.0795625)(74.37675796,343.13956244)(74.39675857,343.19956697)
\curveto(74.44675789,343.3795622)(74.48675785,343.54956203)(74.51675857,343.70956697)
\curveto(74.54675779,343.8795617)(74.59175775,344.04456154)(74.65175857,344.20456697)
\curveto(74.87175747,344.71456087)(75.14675719,345.13956044)(75.47675857,345.47956697)
\curveto(75.81675652,345.81955976)(76.24675609,346.09455949)(76.76675857,346.30456697)
\curveto(76.90675543,346.36455922)(77.05175529,346.40455918)(77.20175857,346.42456697)
\curveto(77.35175499,346.45455913)(77.50675483,346.48955909)(77.66675857,346.52956697)
\curveto(77.74675459,346.53955904)(77.82175452,346.54455904)(77.89175857,346.54456697)
\curveto(77.96175438,346.54455904)(78.0367543,346.54955903)(78.11675857,346.55956697)
}
}
{
\newrgbcolor{curcolor}{0 0 0}
\pscustom[linestyle=none,fillstyle=solid,fillcolor=curcolor]
{
\newpath
\moveto(85.26003982,349.19956697)
\curveto(85.33003687,349.11955646)(85.36503684,348.99955658)(85.36503982,348.83956697)
\lineto(85.36503982,348.37456697)
\lineto(85.36503982,347.96956697)
\curveto(85.36503684,347.82955775)(85.33003687,347.73455785)(85.26003982,347.68456697)
\curveto(85.200037,347.63455795)(85.12003708,347.60455798)(85.02003982,347.59456697)
\curveto(84.93003727,347.584558)(84.83003737,347.579558)(84.72003982,347.57956697)
\lineto(83.88003982,347.57956697)
\curveto(83.77003843,347.579558)(83.67003853,347.584558)(83.58003982,347.59456697)
\curveto(83.5000387,347.60455798)(83.43003877,347.63455795)(83.37003982,347.68456697)
\curveto(83.33003887,347.71455787)(83.3000389,347.76955781)(83.28003982,347.84956697)
\curveto(83.27003893,347.93955764)(83.26003894,348.03455755)(83.25003982,348.13456697)
\lineto(83.25003982,348.46456697)
\curveto(83.26003894,348.57455701)(83.26503894,348.66955691)(83.26503982,348.74956697)
\lineto(83.26503982,348.95956697)
\curveto(83.27503893,349.02955655)(83.29503891,349.08955649)(83.32503982,349.13956697)
\curveto(83.34503886,349.1795564)(83.37003883,349.20955637)(83.40003982,349.22956697)
\lineto(83.52003982,349.28956697)
\curveto(83.54003866,349.28955629)(83.56503864,349.28955629)(83.59503982,349.28956697)
\curveto(83.62503858,349.29955628)(83.65003855,349.30455628)(83.67003982,349.30456697)
\lineto(84.76503982,349.30456697)
\curveto(84.86503734,349.30455628)(84.96003724,349.29955628)(85.05003982,349.28956697)
\curveto(85.14003706,349.2795563)(85.21003699,349.24955633)(85.26003982,349.19956697)
\moveto(85.36503982,339.43456697)
\curveto(85.36503684,339.23456635)(85.36003684,339.06456652)(85.35003982,338.92456697)
\curveto(85.34003686,338.7845668)(85.25003695,338.68956689)(85.08003982,338.63956697)
\curveto(85.02003718,338.61956696)(84.95503725,338.60956697)(84.88503982,338.60956697)
\curveto(84.81503739,338.61956696)(84.74003746,338.62456696)(84.66003982,338.62456697)
\lineto(83.82003982,338.62456697)
\curveto(83.73003847,338.62456696)(83.64003856,338.62956695)(83.55003982,338.63956697)
\curveto(83.47003873,338.64956693)(83.41003879,338.6795669)(83.37003982,338.72956697)
\curveto(83.31003889,338.79956678)(83.27503893,338.8845667)(83.26503982,338.98456697)
\lineto(83.26503982,339.32956697)
\lineto(83.26503982,345.65956697)
\lineto(83.26503982,345.95956697)
\curveto(83.26503894,346.05955952)(83.28503892,346.13955944)(83.32503982,346.19956697)
\curveto(83.38503882,346.26955931)(83.47003873,346.31455927)(83.58003982,346.33456697)
\curveto(83.6000386,346.34455924)(83.62503858,346.34455924)(83.65503982,346.33456697)
\curveto(83.69503851,346.33455925)(83.72503848,346.33955924)(83.74503982,346.34956697)
\lineto(84.49503982,346.34956697)
\lineto(84.69003982,346.34956697)
\curveto(84.77003743,346.35955922)(84.83503737,346.35955922)(84.88503982,346.34956697)
\lineto(85.00503982,346.34956697)
\curveto(85.06503714,346.32955925)(85.12003708,346.31455927)(85.17003982,346.30456697)
\curveto(85.22003698,346.29455929)(85.26003694,346.26455932)(85.29003982,346.21456697)
\curveto(85.33003687,346.16455942)(85.35003685,346.09455949)(85.35003982,346.00456697)
\curveto(85.36003684,345.91455967)(85.36503684,345.81955976)(85.36503982,345.71956697)
\lineto(85.36503982,339.43456697)
}
}
{
\newrgbcolor{curcolor}{0 0 0}
\pscustom[linestyle=none,fillstyle=solid,fillcolor=curcolor]
{
\newpath
\moveto(94.79722732,342.79456697)
\curveto(94.77721879,342.84456274)(94.7722188,342.89956268)(94.78222732,342.95956697)
\curveto(94.79221878,343.01956256)(94.78721878,343.07456251)(94.76722732,343.12456697)
\curveto(94.75721881,343.16456242)(94.75221882,343.20456238)(94.75222732,343.24456697)
\curveto(94.75221882,343.2845623)(94.74721882,343.32456226)(94.73722732,343.36456697)
\lineto(94.67722732,343.63456697)
\curveto(94.65721891,343.72456186)(94.63221894,343.80956177)(94.60222732,343.88956697)
\curveto(94.55221902,344.02956155)(94.50721906,344.15956142)(94.46722732,344.27956697)
\curveto(94.42721914,344.40956117)(94.3722192,344.52956105)(94.30222732,344.63956697)
\curveto(94.23221934,344.74956083)(94.16221941,344.85456073)(94.09222732,344.95456697)
\curveto(94.03221954,345.05456053)(93.96221961,345.15456043)(93.88222732,345.25456697)
\curveto(93.80221977,345.36456022)(93.70221987,345.46456012)(93.58222732,345.55456697)
\curveto(93.4722201,345.65455993)(93.36222021,345.74455984)(93.25222732,345.82456697)
\curveto(92.92222065,346.05455953)(92.54222103,346.23455935)(92.11222732,346.36456697)
\curveto(91.69222188,346.49455909)(91.19222238,346.55455903)(90.61222732,346.54456697)
\curveto(90.54222303,346.53455905)(90.4722231,346.52955905)(90.40222732,346.52956697)
\curveto(90.33222324,346.52955905)(90.25722331,346.52455906)(90.17722732,346.51456697)
\curveto(90.02722354,346.47455911)(89.88222369,346.44455914)(89.74222732,346.42456697)
\curveto(89.60222397,346.40455918)(89.4672241,346.36955921)(89.33722732,346.31956697)
\curveto(89.22722434,346.26955931)(89.11722445,346.22455936)(89.00722732,346.18456697)
\curveto(88.89722467,346.14455944)(88.79222478,346.09955948)(88.69222732,346.04956697)
\curveto(88.33222524,345.81955976)(88.02722554,345.56456002)(87.77722732,345.28456697)
\curveto(87.52722604,345.01456057)(87.31222626,344.67456091)(87.13222732,344.26456697)
\curveto(87.08222649,344.14456144)(87.04222653,344.01956156)(87.01222732,343.88956697)
\curveto(86.98222659,343.76956181)(86.94722662,343.64456194)(86.90722732,343.51456697)
\curveto(86.88722668,343.46456212)(86.87722669,343.41456217)(86.87722732,343.36456697)
\curveto(86.87722669,343.32456226)(86.8722267,343.2795623)(86.86222732,343.22956697)
\curveto(86.84222673,343.1795624)(86.83222674,343.12456246)(86.83222732,343.06456697)
\curveto(86.84222673,343.01456257)(86.84222673,342.96456262)(86.83222732,342.91456697)
\lineto(86.83222732,342.80956697)
\curveto(86.81222676,342.74956283)(86.79722677,342.66456292)(86.78722732,342.55456697)
\curveto(86.78722678,342.44456314)(86.79722677,342.35956322)(86.81722732,342.29956697)
\lineto(86.81722732,342.16456697)
\curveto(86.81722675,342.12456346)(86.82222675,342.0795635)(86.83222732,342.02956697)
\curveto(86.85222672,341.94956363)(86.86222671,341.86456372)(86.86222732,341.77456697)
\curveto(86.86222671,341.69456389)(86.8722267,341.61456397)(86.89222732,341.53456697)
\curveto(86.91222666,341.4845641)(86.92222665,341.43956414)(86.92222732,341.39956697)
\curveto(86.92222665,341.35956422)(86.93222664,341.31456427)(86.95222732,341.26456697)
\curveto(86.98222659,341.15456443)(87.00722656,341.04956453)(87.02722732,340.94956697)
\curveto(87.05722651,340.84956473)(87.09722647,340.75456483)(87.14722732,340.66456697)
\curveto(87.31722625,340.27456531)(87.52722604,339.93956564)(87.77722732,339.65956697)
\curveto(88.02722554,339.3795662)(88.32722524,339.13456645)(88.67722732,338.92456697)
\curveto(88.78722478,338.86456672)(88.89222468,338.81456677)(88.99222732,338.77456697)
\curveto(89.10222447,338.73456685)(89.21722435,338.69456689)(89.33722732,338.65456697)
\curveto(89.42722414,338.61456697)(89.52222405,338.584567)(89.62222732,338.56456697)
\curveto(89.72222385,338.54456704)(89.82222375,338.51956706)(89.92222732,338.48956697)
\curveto(89.9722236,338.4795671)(90.01222356,338.47456711)(90.04222732,338.47456697)
\curveto(90.08222349,338.47456711)(90.12222345,338.46956711)(90.16222732,338.45956697)
\curveto(90.21222336,338.43956714)(90.26222331,338.43456715)(90.31222732,338.44456697)
\curveto(90.3722232,338.44456714)(90.42722314,338.43956714)(90.47722732,338.42956697)
\lineto(90.62722732,338.42956697)
\curveto(90.68722288,338.40956717)(90.7722228,338.40456718)(90.88222732,338.41456697)
\curveto(90.99222258,338.41456717)(91.0722225,338.41956716)(91.12222732,338.42956697)
\curveto(91.15222242,338.42956715)(91.18222239,338.43456715)(91.21222732,338.44456697)
\lineto(91.31722732,338.44456697)
\curveto(91.3672222,338.45456713)(91.42222215,338.45956712)(91.48222732,338.45956697)
\curveto(91.54222203,338.45956712)(91.59722197,338.46956711)(91.64722732,338.48956697)
\curveto(91.77722179,338.51956706)(91.90222167,338.54956703)(92.02222732,338.57956697)
\curveto(92.15222142,338.59956698)(92.27722129,338.63456695)(92.39722732,338.68456697)
\curveto(92.87722069,338.8845667)(93.28722028,339.13456645)(93.62722732,339.43456697)
\curveto(93.9672196,339.73456585)(94.24221933,340.12456546)(94.45222732,340.60456697)
\curveto(94.50221907,340.70456488)(94.54221903,340.80956477)(94.57222732,340.91956697)
\curveto(94.60221897,341.03956454)(94.63721893,341.15456443)(94.67722732,341.26456697)
\curveto(94.68721888,341.33456425)(94.69721887,341.39956418)(94.70722732,341.45956697)
\curveto(94.71721885,341.51956406)(94.73221884,341.584564)(94.75222732,341.65456697)
\curveto(94.7722188,341.73456385)(94.77721879,341.81456377)(94.76722732,341.89456697)
\curveto(94.7672188,341.97456361)(94.77721879,342.05456353)(94.79722732,342.13456697)
\lineto(94.79722732,342.28456697)
\curveto(94.81721875,342.34456324)(94.82721874,342.42956315)(94.82722732,342.53956697)
\curveto(94.82721874,342.64956293)(94.81721875,342.73456285)(94.79722732,342.79456697)
\moveto(92.69722732,342.25456697)
\curveto(92.68722088,342.20456338)(92.68222089,342.15456343)(92.68222732,342.10456697)
\lineto(92.68222732,341.96956697)
\curveto(92.6722209,341.92956365)(92.6672209,341.88956369)(92.66722732,341.84956697)
\curveto(92.6672209,341.81956376)(92.66222091,341.7845638)(92.65222732,341.74456697)
\curveto(92.62222095,341.63456395)(92.59722097,341.52956405)(92.57722732,341.42956697)
\curveto(92.55722101,341.32956425)(92.52722104,341.22956435)(92.48722732,341.12956697)
\curveto(92.37722119,340.8795647)(92.24222133,340.66956491)(92.08222732,340.49956697)
\curveto(91.92222165,340.32956525)(91.71222186,340.19456539)(91.45222732,340.09456697)
\curveto(91.38222219,340.06456552)(91.30722226,340.04456554)(91.22722732,340.03456697)
\curveto(91.14722242,340.02456556)(91.0672225,340.00956557)(90.98722732,339.98956697)
\lineto(90.86722732,339.98956697)
\curveto(90.82722274,339.9795656)(90.78222279,339.97456561)(90.73222732,339.97456697)
\lineto(90.61222732,340.00456697)
\curveto(90.572223,340.01456557)(90.53722303,340.01456557)(90.50722732,340.00456697)
\curveto(90.47722309,340.00456558)(90.44222313,340.00956557)(90.40222732,340.01956697)
\curveto(90.31222326,340.03956554)(90.22222335,340.06456552)(90.13222732,340.09456697)
\curveto(90.05222352,340.12456546)(89.97722359,340.16456542)(89.90722732,340.21456697)
\curveto(89.65722391,340.36456522)(89.4722241,340.52956505)(89.35222732,340.70956697)
\curveto(89.24222433,340.89956468)(89.13722443,341.14456444)(89.03722732,341.44456697)
\curveto(89.01722455,341.52456406)(89.00222457,341.59956398)(88.99222732,341.66956697)
\curveto(88.98222459,341.74956383)(88.9672246,341.82956375)(88.94722732,341.90956697)
\lineto(88.94722732,342.04456697)
\curveto(88.92722464,342.11456347)(88.91222466,342.21956336)(88.90222732,342.35956697)
\curveto(88.90222467,342.49956308)(88.91222466,342.60456298)(88.93222732,342.67456697)
\lineto(88.93222732,342.82456697)
\curveto(88.93222464,342.87456271)(88.93722463,342.92456266)(88.94722732,342.97456697)
\curveto(88.9672246,343.0845625)(88.98222459,343.19456239)(88.99222732,343.30456697)
\curveto(89.01222456,343.41456217)(89.03722453,343.51956206)(89.06722732,343.61956697)
\curveto(89.15722441,343.88956169)(89.27722429,344.12456146)(89.42722732,344.32456697)
\curveto(89.58722398,344.53456105)(89.79222378,344.69456089)(90.04222732,344.80456697)
\curveto(90.09222348,344.83456075)(90.14722342,344.85456073)(90.20722732,344.86456697)
\lineto(90.41722732,344.92456697)
\curveto(90.44722312,344.93456065)(90.48222309,344.93456065)(90.52222732,344.92456697)
\curveto(90.56222301,344.92456066)(90.59722297,344.93456065)(90.62722732,344.95456697)
\lineto(90.89722732,344.95456697)
\curveto(90.98722258,344.96456062)(91.0722225,344.95956062)(91.15222732,344.93956697)
\curveto(91.22222235,344.91956066)(91.28722228,344.89956068)(91.34722732,344.87956697)
\curveto(91.40722216,344.86956071)(91.4672221,344.85456073)(91.52722732,344.83456697)
\curveto(91.77722179,344.72456086)(91.97722159,344.57456101)(92.12722732,344.38456697)
\curveto(92.27722129,344.20456138)(92.40722116,343.9845616)(92.51722732,343.72456697)
\curveto(92.54722102,343.64456194)(92.567221,343.55956202)(92.57722732,343.46956697)
\lineto(92.63722732,343.22956697)
\curveto(92.64722092,343.20956237)(92.65222092,343.1795624)(92.65222732,343.13956697)
\curveto(92.66222091,343.08956249)(92.6672209,343.03456255)(92.66722732,342.97456697)
\curveto(92.6672209,342.91456267)(92.67722089,342.85956272)(92.69722732,342.80956697)
\lineto(92.69722732,342.68956697)
\curveto(92.70722086,342.63956294)(92.71222086,342.56456302)(92.71222732,342.46456697)
\curveto(92.71222086,342.37456321)(92.70722086,342.30456328)(92.69722732,342.25456697)
\moveto(91.46722732,349.42456697)
\lineto(92.53222732,349.42456697)
\curveto(92.61222096,349.42455616)(92.70722086,349.42455616)(92.81722732,349.42456697)
\curveto(92.92722064,349.42455616)(93.00722056,349.40955617)(93.05722732,349.37956697)
\curveto(93.07722049,349.36955621)(93.08722048,349.35455623)(93.08722732,349.33456697)
\curveto(93.09722047,349.32455626)(93.11222046,349.31455627)(93.13222732,349.30456697)
\curveto(93.14222043,349.1845564)(93.09222048,349.0795565)(92.98222732,348.98956697)
\curveto(92.88222069,348.89955668)(92.79722077,348.81955676)(92.72722732,348.74956697)
\curveto(92.64722092,348.6795569)(92.567221,348.60455698)(92.48722732,348.52456697)
\curveto(92.41722115,348.45455713)(92.34222123,348.38955719)(92.26222732,348.32956697)
\curveto(92.22222135,348.29955728)(92.18722138,348.26455732)(92.15722732,348.22456697)
\curveto(92.13722143,348.19455739)(92.10722146,348.16955741)(92.06722732,348.14956697)
\curveto(92.04722152,348.11955746)(92.02222155,348.09455749)(91.99222732,348.07456697)
\lineto(91.84222732,347.92456697)
\lineto(91.69222732,347.80456697)
\lineto(91.64722732,347.75956697)
\curveto(91.64722192,347.74955783)(91.63722193,347.73455785)(91.61722732,347.71456697)
\curveto(91.53722203,347.65455793)(91.45722211,347.58955799)(91.37722732,347.51956697)
\curveto(91.30722226,347.44955813)(91.21722235,347.39455819)(91.10722732,347.35456697)
\curveto(91.0672225,347.34455824)(91.02722254,347.33955824)(90.98722732,347.33956697)
\curveto(90.95722261,347.33955824)(90.91722265,347.33455825)(90.86722732,347.32456697)
\curveto(90.83722273,347.31455827)(90.79722277,347.30955827)(90.74722732,347.30956697)
\curveto(90.69722287,347.31955826)(90.65222292,347.32455826)(90.61222732,347.32456697)
\lineto(90.26722732,347.32456697)
\curveto(90.14722342,347.32455826)(90.05722351,347.34955823)(89.99722732,347.39956697)
\curveto(89.93722363,347.43955814)(89.92222365,347.50955807)(89.95222732,347.60956697)
\curveto(89.9722236,347.68955789)(90.00722356,347.75955782)(90.05722732,347.81956697)
\curveto(90.10722346,347.88955769)(90.15222342,347.95955762)(90.19222732,348.02956697)
\curveto(90.29222328,348.16955741)(90.38722318,348.30455728)(90.47722732,348.43456697)
\curveto(90.567223,348.56455702)(90.65722291,348.69955688)(90.74722732,348.83956697)
\curveto(90.79722277,348.91955666)(90.84722272,349.00455658)(90.89722732,349.09456697)
\curveto(90.95722261,349.1845564)(91.02222255,349.25455633)(91.09222732,349.30456697)
\curveto(91.13222244,349.33455625)(91.20222237,349.36955621)(91.30222732,349.40956697)
\curveto(91.32222225,349.41955616)(91.34722222,349.41955616)(91.37722732,349.40956697)
\curveto(91.41722215,349.40955617)(91.44722212,349.41455617)(91.46722732,349.42456697)
}
}
{
\newrgbcolor{curcolor}{0 0 0}
\pscustom[linestyle=none,fillstyle=solid,fillcolor=curcolor]
{
\newpath
\moveto(100.6221492,346.54456697)
\curveto(101.22214339,346.56455902)(101.72214289,346.4795591)(102.1221492,346.28956697)
\curveto(102.52214209,346.09955948)(102.83714178,345.81955976)(103.0671492,345.44956697)
\curveto(103.13714148,345.33956024)(103.19214142,345.21956036)(103.2321492,345.08956697)
\curveto(103.27214134,344.96956061)(103.3121413,344.84456074)(103.3521492,344.71456697)
\curveto(103.37214124,344.63456095)(103.38214123,344.55956102)(103.3821492,344.48956697)
\curveto(103.39214122,344.41956116)(103.40714121,344.34956123)(103.4271492,344.27956697)
\curveto(103.42714119,344.21956136)(103.43214118,344.1795614)(103.4421492,344.15956697)
\curveto(103.46214115,344.01956156)(103.47214114,343.87456171)(103.4721492,343.72456697)
\lineto(103.4721492,343.28956697)
\lineto(103.4721492,341.95456697)
\lineto(103.4721492,339.52456697)
\curveto(103.47214114,339.33456625)(103.46714115,339.14956643)(103.4571492,338.96956697)
\curveto(103.45714116,338.79956678)(103.38714123,338.68956689)(103.2471492,338.63956697)
\curveto(103.18714143,338.61956696)(103.1171415,338.60956697)(103.0371492,338.60956697)
\lineto(102.7971492,338.60956697)
\lineto(101.9871492,338.60956697)
\curveto(101.86714275,338.60956697)(101.75714286,338.61456697)(101.6571492,338.62456697)
\curveto(101.56714305,338.64456694)(101.49714312,338.68956689)(101.4471492,338.75956697)
\curveto(101.40714321,338.81956676)(101.38214323,338.89456669)(101.3721492,338.98456697)
\lineto(101.3721492,339.29956697)
\lineto(101.3721492,340.34956697)
\lineto(101.3721492,342.58456697)
\curveto(101.37214324,342.95456263)(101.35714326,343.29456229)(101.3271492,343.60456697)
\curveto(101.29714332,343.92456166)(101.20714341,344.19456139)(101.0571492,344.41456697)
\curveto(100.9171437,344.61456097)(100.7121439,344.75456083)(100.4421492,344.83456697)
\curveto(100.39214422,344.85456073)(100.33714428,344.86456072)(100.2771492,344.86456697)
\curveto(100.22714439,344.86456072)(100.17214444,344.87456071)(100.1121492,344.89456697)
\curveto(100.06214455,344.90456068)(99.99714462,344.90456068)(99.9171492,344.89456697)
\curveto(99.84714477,344.89456069)(99.79214482,344.88956069)(99.7521492,344.87956697)
\curveto(99.7121449,344.86956071)(99.67714494,344.86456072)(99.6471492,344.86456697)
\curveto(99.617145,344.86456072)(99.58714503,344.85956072)(99.5571492,344.84956697)
\curveto(99.32714529,344.78956079)(99.14214547,344.70956087)(99.0021492,344.60956697)
\curveto(98.68214593,344.3795612)(98.49214612,344.04456154)(98.4321492,343.60456697)
\curveto(98.37214624,343.16456242)(98.34214627,342.66956291)(98.3421492,342.11956697)
\lineto(98.3421492,340.24456697)
\lineto(98.3421492,339.32956697)
\lineto(98.3421492,339.05956697)
\curveto(98.34214627,338.96956661)(98.32714629,338.89456669)(98.2971492,338.83456697)
\curveto(98.24714637,338.72456686)(98.16714645,338.65956692)(98.0571492,338.63956697)
\curveto(97.94714667,338.61956696)(97.8121468,338.60956697)(97.6521492,338.60956697)
\lineto(96.9021492,338.60956697)
\curveto(96.79214782,338.60956697)(96.68214793,338.61456697)(96.5721492,338.62456697)
\curveto(96.46214815,338.63456695)(96.38214823,338.66956691)(96.3321492,338.72956697)
\curveto(96.26214835,338.81956676)(96.22714839,338.94956663)(96.2271492,339.11956697)
\curveto(96.23714838,339.28956629)(96.24214837,339.44956613)(96.2421492,339.59956697)
\lineto(96.2421492,341.63956697)
\lineto(96.2421492,344.93956697)
\lineto(96.2421492,345.70456697)
\lineto(96.2421492,346.00456697)
\curveto(96.25214836,346.09455949)(96.28214833,346.16955941)(96.3321492,346.22956697)
\curveto(96.35214826,346.25955932)(96.38214823,346.2795593)(96.4221492,346.28956697)
\curveto(96.47214814,346.30955927)(96.52214809,346.32455926)(96.5721492,346.33456697)
\lineto(96.6471492,346.33456697)
\curveto(96.69714792,346.34455924)(96.74714787,346.34955923)(96.7971492,346.34956697)
\lineto(96.9621492,346.34956697)
\lineto(97.5921492,346.34956697)
\curveto(97.67214694,346.34955923)(97.74714687,346.34455924)(97.8171492,346.33456697)
\curveto(97.89714672,346.33455925)(97.96714665,346.32455926)(98.0271492,346.30456697)
\curveto(98.09714652,346.27455931)(98.14214647,346.22955935)(98.1621492,346.16956697)
\curveto(98.19214642,346.10955947)(98.2171464,346.03955954)(98.2371492,345.95956697)
\curveto(98.24714637,345.91955966)(98.24714637,345.8845597)(98.2371492,345.85456697)
\curveto(98.23714638,345.82455976)(98.24714637,345.79455979)(98.2671492,345.76456697)
\curveto(98.28714633,345.71455987)(98.30214631,345.6845599)(98.3121492,345.67456697)
\curveto(98.33214628,345.66455992)(98.35714626,345.64955993)(98.3871492,345.62956697)
\curveto(98.49714612,345.61955996)(98.58714603,345.65455993)(98.6571492,345.73456697)
\curveto(98.72714589,345.82455976)(98.80214581,345.89455969)(98.8821492,345.94456697)
\curveto(99.15214546,346.14455944)(99.45214516,346.30455928)(99.7821492,346.42456697)
\curveto(99.87214474,346.45455913)(99.96214465,346.47455911)(100.0521492,346.48456697)
\curveto(100.15214446,346.49455909)(100.25714436,346.50955907)(100.3671492,346.52956697)
\curveto(100.39714422,346.53955904)(100.44214417,346.53955904)(100.5021492,346.52956697)
\curveto(100.56214405,346.52955905)(100.60214401,346.53455905)(100.6221492,346.54456697)
}
}
{
\newrgbcolor{curcolor}{0 0 0}
\pscustom[linestyle=none,fillstyle=solid,fillcolor=curcolor]
{
}
}
{
\newrgbcolor{curcolor}{0 0 0}
\pscustom[linestyle=none,fillstyle=solid,fillcolor=curcolor]
{
\newpath
\moveto(116.85355545,339.46456697)
\lineto(116.85355545,339.04456697)
\curveto(116.85354708,338.91456667)(116.82354711,338.80956677)(116.76355545,338.72956697)
\curveto(116.71354722,338.6795669)(116.64854728,338.64456694)(116.56855545,338.62456697)
\curveto(116.48854744,338.61456697)(116.39854753,338.60956697)(116.29855545,338.60956697)
\lineto(115.47355545,338.60956697)
\lineto(115.18855545,338.60956697)
\curveto(115.10854882,338.61956696)(115.04354889,338.64456694)(114.99355545,338.68456697)
\curveto(114.92354901,338.73456685)(114.88354905,338.79956678)(114.87355545,338.87956697)
\curveto(114.86354907,338.95956662)(114.84354909,339.03956654)(114.81355545,339.11956697)
\curveto(114.79354914,339.13956644)(114.77354916,339.15456643)(114.75355545,339.16456697)
\curveto(114.74354919,339.1845664)(114.7285492,339.20456638)(114.70855545,339.22456697)
\curveto(114.59854933,339.22456636)(114.51854941,339.19956638)(114.46855545,339.14956697)
\lineto(114.31855545,338.99956697)
\curveto(114.24854968,338.94956663)(114.18354975,338.90456668)(114.12355545,338.86456697)
\curveto(114.06354987,338.83456675)(113.99854993,338.79456679)(113.92855545,338.74456697)
\curveto(113.88855004,338.72456686)(113.84355009,338.70456688)(113.79355545,338.68456697)
\curveto(113.75355018,338.66456692)(113.70855022,338.64456694)(113.65855545,338.62456697)
\curveto(113.51855041,338.57456701)(113.36855056,338.52956705)(113.20855545,338.48956697)
\curveto(113.15855077,338.46956711)(113.11355082,338.45956712)(113.07355545,338.45956697)
\curveto(113.0335509,338.45956712)(112.99355094,338.45456713)(112.95355545,338.44456697)
\lineto(112.81855545,338.44456697)
\curveto(112.78855114,338.43456715)(112.74855118,338.42956715)(112.69855545,338.42956697)
\lineto(112.56355545,338.42956697)
\curveto(112.50355143,338.40956717)(112.41355152,338.40456718)(112.29355545,338.41456697)
\curveto(112.17355176,338.41456717)(112.08855184,338.42456716)(112.03855545,338.44456697)
\curveto(111.96855196,338.46456712)(111.90355203,338.47456711)(111.84355545,338.47456697)
\curveto(111.79355214,338.46456712)(111.73855219,338.46956711)(111.67855545,338.48956697)
\lineto(111.31855545,338.60956697)
\curveto(111.20855272,338.63956694)(111.09855283,338.6795669)(110.98855545,338.72956697)
\curveto(110.63855329,338.8795667)(110.32355361,339.10956647)(110.04355545,339.41956697)
\curveto(109.77355416,339.73956584)(109.55855437,340.07456551)(109.39855545,340.42456697)
\curveto(109.34855458,340.53456505)(109.30855462,340.63956494)(109.27855545,340.73956697)
\curveto(109.24855468,340.84956473)(109.21355472,340.95956462)(109.17355545,341.06956697)
\curveto(109.16355477,341.10956447)(109.15855477,341.14456444)(109.15855545,341.17456697)
\curveto(109.15855477,341.21456437)(109.14855478,341.25956432)(109.12855545,341.30956697)
\curveto(109.10855482,341.38956419)(109.08855484,341.47456411)(109.06855545,341.56456697)
\curveto(109.05855487,341.66456392)(109.04355489,341.76456382)(109.02355545,341.86456697)
\curveto(109.01355492,341.89456369)(109.00855492,341.92956365)(109.00855545,341.96956697)
\curveto(109.01855491,342.00956357)(109.01855491,342.04456354)(109.00855545,342.07456697)
\lineto(109.00855545,342.20956697)
\curveto(109.00855492,342.25956332)(109.00355493,342.30956327)(108.99355545,342.35956697)
\curveto(108.98355495,342.40956317)(108.97855495,342.46456312)(108.97855545,342.52456697)
\curveto(108.97855495,342.59456299)(108.98355495,342.64956293)(108.99355545,342.68956697)
\curveto(109.00355493,342.73956284)(109.00855492,342.7845628)(109.00855545,342.82456697)
\lineto(109.00855545,342.97456697)
\curveto(109.01855491,343.02456256)(109.01855491,343.06956251)(109.00855545,343.10956697)
\curveto(109.00855492,343.15956242)(109.01855491,343.20956237)(109.03855545,343.25956697)
\curveto(109.05855487,343.36956221)(109.07355486,343.47456211)(109.08355545,343.57456697)
\curveto(109.10355483,343.67456191)(109.1285548,343.77456181)(109.15855545,343.87456697)
\curveto(109.19855473,343.99456159)(109.2335547,344.10956147)(109.26355545,344.21956697)
\curveto(109.29355464,344.32956125)(109.3335546,344.43956114)(109.38355545,344.54956697)
\curveto(109.52355441,344.84956073)(109.69855423,345.13456045)(109.90855545,345.40456697)
\curveto(109.928554,345.43456015)(109.95355398,345.45956012)(109.98355545,345.47956697)
\curveto(110.02355391,345.50956007)(110.05355388,345.53956004)(110.07355545,345.56956697)
\curveto(110.11355382,345.61955996)(110.15355378,345.66455992)(110.19355545,345.70456697)
\curveto(110.2335537,345.74455984)(110.27855365,345.7845598)(110.32855545,345.82456697)
\curveto(110.36855356,345.84455974)(110.40355353,345.86955971)(110.43355545,345.89956697)
\curveto(110.46355347,345.93955964)(110.49855343,345.96955961)(110.53855545,345.98956697)
\curveto(110.78855314,346.15955942)(111.07855285,346.29955928)(111.40855545,346.40956697)
\curveto(111.47855245,346.42955915)(111.54855238,346.44455914)(111.61855545,346.45456697)
\curveto(111.69855223,346.46455912)(111.77855215,346.4795591)(111.85855545,346.49956697)
\curveto(111.928552,346.51955906)(112.01855191,346.52955905)(112.12855545,346.52956697)
\curveto(112.23855169,346.53955904)(112.34855158,346.54455904)(112.45855545,346.54456697)
\curveto(112.56855136,346.54455904)(112.67355126,346.53955904)(112.77355545,346.52956697)
\curveto(112.88355105,346.51955906)(112.97355096,346.50455908)(113.04355545,346.48456697)
\curveto(113.19355074,346.43455915)(113.33855059,346.38955919)(113.47855545,346.34956697)
\curveto(113.61855031,346.30955927)(113.74855018,346.25455933)(113.86855545,346.18456697)
\curveto(113.93854999,346.13455945)(114.00354993,346.0845595)(114.06355545,346.03456697)
\curveto(114.12354981,345.99455959)(114.18854974,345.94955963)(114.25855545,345.89956697)
\curveto(114.29854963,345.86955971)(114.35354958,345.82955975)(114.42355545,345.77956697)
\curveto(114.50354943,345.72955985)(114.57854935,345.72955985)(114.64855545,345.77956697)
\curveto(114.68854924,345.79955978)(114.70854922,345.83455975)(114.70855545,345.88456697)
\curveto(114.70854922,345.93455965)(114.71854921,345.9845596)(114.73855545,346.03456697)
\lineto(114.73855545,346.18456697)
\curveto(114.74854918,346.21455937)(114.75354918,346.24955933)(114.75355545,346.28956697)
\lineto(114.75355545,346.40956697)
\lineto(114.75355545,348.44956697)
\curveto(114.75354918,348.55955702)(114.74854918,348.6795569)(114.73855545,348.80956697)
\curveto(114.73854919,348.94955663)(114.76354917,349.05455653)(114.81355545,349.12456697)
\curveto(114.85354908,349.20455638)(114.928549,349.25455633)(115.03855545,349.27456697)
\curveto(115.05854887,349.2845563)(115.07854885,349.2845563)(115.09855545,349.27456697)
\curveto(115.11854881,349.27455631)(115.13854879,349.2795563)(115.15855545,349.28956697)
\lineto(116.22355545,349.28956697)
\curveto(116.34354759,349.28955629)(116.45354748,349.2845563)(116.55355545,349.27456697)
\curveto(116.65354728,349.26455632)(116.7285472,349.22455636)(116.77855545,349.15456697)
\curveto(116.8285471,349.07455651)(116.85354708,348.96955661)(116.85355545,348.83956697)
\lineto(116.85355545,348.47956697)
\lineto(116.85355545,339.46456697)
\moveto(114.81355545,342.40456697)
\curveto(114.82354911,342.44456314)(114.82354911,342.4845631)(114.81355545,342.52456697)
\lineto(114.81355545,342.65956697)
\curveto(114.81354912,342.75956282)(114.80854912,342.85956272)(114.79855545,342.95956697)
\curveto(114.78854914,343.05956252)(114.77354916,343.14956243)(114.75355545,343.22956697)
\curveto(114.7335492,343.33956224)(114.71354922,343.43956214)(114.69355545,343.52956697)
\curveto(114.68354925,343.61956196)(114.65854927,343.70456188)(114.61855545,343.78456697)
\curveto(114.47854945,344.14456144)(114.27354966,344.42956115)(114.00355545,344.63956697)
\curveto(113.74355019,344.84956073)(113.36355057,344.95456063)(112.86355545,344.95456697)
\curveto(112.80355113,344.95456063)(112.72355121,344.94456064)(112.62355545,344.92456697)
\curveto(112.54355139,344.90456068)(112.46855146,344.8845607)(112.39855545,344.86456697)
\curveto(112.33855159,344.85456073)(112.27855165,344.83456075)(112.21855545,344.80456697)
\curveto(111.94855198,344.69456089)(111.73855219,344.52456106)(111.58855545,344.29456697)
\curveto(111.43855249,344.06456152)(111.31855261,343.80456178)(111.22855545,343.51456697)
\curveto(111.19855273,343.41456217)(111.17855275,343.31456227)(111.16855545,343.21456697)
\curveto(111.15855277,343.11456247)(111.13855279,343.00956257)(111.10855545,342.89956697)
\lineto(111.10855545,342.68956697)
\curveto(111.08855284,342.59956298)(111.08355285,342.47456311)(111.09355545,342.31456697)
\curveto(111.10355283,342.16456342)(111.11855281,342.05456353)(111.13855545,341.98456697)
\lineto(111.13855545,341.89456697)
\curveto(111.14855278,341.87456371)(111.15355278,341.85456373)(111.15355545,341.83456697)
\curveto(111.17355276,341.75456383)(111.18855274,341.6795639)(111.19855545,341.60956697)
\curveto(111.21855271,341.53956404)(111.23855269,341.46456412)(111.25855545,341.38456697)
\curveto(111.4285525,340.86456472)(111.71855221,340.4795651)(112.12855545,340.22956697)
\curveto(112.25855167,340.13956544)(112.43855149,340.06956551)(112.66855545,340.01956697)
\curveto(112.70855122,340.00956557)(112.76855116,340.00456558)(112.84855545,340.00456697)
\curveto(112.87855105,339.99456559)(112.92355101,339.9845656)(112.98355545,339.97456697)
\curveto(113.05355088,339.97456561)(113.10855082,339.9795656)(113.14855545,339.98956697)
\curveto(113.2285507,340.00956557)(113.30855062,340.02456556)(113.38855545,340.03456697)
\curveto(113.46855046,340.04456554)(113.54855038,340.06456552)(113.62855545,340.09456697)
\curveto(113.87855005,340.20456538)(114.07854985,340.34456524)(114.22855545,340.51456697)
\curveto(114.37854955,340.6845649)(114.50854942,340.89956468)(114.61855545,341.15956697)
\curveto(114.65854927,341.24956433)(114.68854924,341.33956424)(114.70855545,341.42956697)
\curveto(114.7285492,341.52956405)(114.74854918,341.63456395)(114.76855545,341.74456697)
\curveto(114.77854915,341.79456379)(114.77854915,341.83956374)(114.76855545,341.87956697)
\curveto(114.76854916,341.92956365)(114.77854915,341.9795636)(114.79855545,342.02956697)
\curveto(114.80854912,342.05956352)(114.81354912,342.09456349)(114.81355545,342.13456697)
\lineto(114.81355545,342.26956697)
\lineto(114.81355545,342.40456697)
}
}
{
\newrgbcolor{curcolor}{0 0 0}
\pscustom[linestyle=none,fillstyle=solid,fillcolor=curcolor]
{
\newpath
\moveto(125.79847732,342.55456697)
\curveto(125.81846916,342.47456311)(125.81846916,342.3845632)(125.79847732,342.28456697)
\curveto(125.7784692,342.1845634)(125.74346923,342.11956346)(125.69347732,342.08956697)
\curveto(125.64346933,342.04956353)(125.56846941,342.01956356)(125.46847732,341.99956697)
\curveto(125.3784696,341.98956359)(125.2734697,341.9795636)(125.15347732,341.96956697)
\lineto(124.80847732,341.96956697)
\curveto(124.69847028,341.9795636)(124.59847038,341.9845636)(124.50847732,341.98456697)
\lineto(120.84847732,341.98456697)
\lineto(120.63847732,341.98456697)
\curveto(120.5784744,341.9845636)(120.52347445,341.97456361)(120.47347732,341.95456697)
\curveto(120.39347458,341.91456367)(120.34347463,341.87456371)(120.32347732,341.83456697)
\curveto(120.30347467,341.81456377)(120.28347469,341.77456381)(120.26347732,341.71456697)
\curveto(120.24347473,341.66456392)(120.23847474,341.61456397)(120.24847732,341.56456697)
\curveto(120.26847471,341.50456408)(120.2784747,341.44456414)(120.27847732,341.38456697)
\curveto(120.28847469,341.33456425)(120.30347467,341.2795643)(120.32347732,341.21956697)
\curveto(120.40347457,340.9795646)(120.49847448,340.7795648)(120.60847732,340.61956697)
\curveto(120.72847425,340.46956511)(120.88847409,340.33456525)(121.08847732,340.21456697)
\curveto(121.16847381,340.16456542)(121.24847373,340.12956545)(121.32847732,340.10956697)
\curveto(121.41847356,340.09956548)(121.50847347,340.0795655)(121.59847732,340.04956697)
\curveto(121.6784733,340.02956555)(121.78847319,340.01456557)(121.92847732,340.00456697)
\curveto(122.06847291,339.99456559)(122.18847279,339.99956558)(122.28847732,340.01956697)
\lineto(122.42347732,340.01956697)
\curveto(122.52347245,340.03956554)(122.61347236,340.05956552)(122.69347732,340.07956697)
\curveto(122.78347219,340.10956547)(122.86847211,340.13956544)(122.94847732,340.16956697)
\curveto(123.04847193,340.21956536)(123.15847182,340.2845653)(123.27847732,340.36456697)
\curveto(123.40847157,340.44456514)(123.50347147,340.52456506)(123.56347732,340.60456697)
\curveto(123.61347136,340.67456491)(123.66347131,340.73956484)(123.71347732,340.79956697)
\curveto(123.7734712,340.86956471)(123.84347113,340.91956466)(123.92347732,340.94956697)
\curveto(124.02347095,340.99956458)(124.14847083,341.01956456)(124.29847732,341.00956697)
\lineto(124.73347732,341.00956697)
\lineto(124.91347732,341.00956697)
\curveto(124.98346999,341.01956456)(125.04346993,341.01456457)(125.09347732,340.99456697)
\lineto(125.24347732,340.99456697)
\curveto(125.34346963,340.97456461)(125.41346956,340.94956463)(125.45347732,340.91956697)
\curveto(125.49346948,340.89956468)(125.51346946,340.85456473)(125.51347732,340.78456697)
\curveto(125.52346945,340.71456487)(125.51846946,340.65456493)(125.49847732,340.60456697)
\curveto(125.44846953,340.46456512)(125.39346958,340.33956524)(125.33347732,340.22956697)
\curveto(125.2734697,340.11956546)(125.20346977,340.00956557)(125.12347732,339.89956697)
\curveto(124.90347007,339.56956601)(124.65347032,339.30456628)(124.37347732,339.10456697)
\curveto(124.09347088,338.90456668)(123.74347123,338.73456685)(123.32347732,338.59456697)
\curveto(123.21347176,338.55456703)(123.10347187,338.52956705)(122.99347732,338.51956697)
\curveto(122.88347209,338.50956707)(122.76847221,338.48956709)(122.64847732,338.45956697)
\curveto(122.60847237,338.44956713)(122.56347241,338.44956713)(122.51347732,338.45956697)
\curveto(122.4734725,338.45956712)(122.43347254,338.45456713)(122.39347732,338.44456697)
\lineto(122.22847732,338.44456697)
\curveto(122.1784728,338.42456716)(122.11847286,338.41956716)(122.04847732,338.42956697)
\curveto(121.98847299,338.42956715)(121.93347304,338.43456715)(121.88347732,338.44456697)
\curveto(121.80347317,338.45456713)(121.73347324,338.45456713)(121.67347732,338.44456697)
\curveto(121.61347336,338.43456715)(121.54847343,338.43956714)(121.47847732,338.45956697)
\curveto(121.42847355,338.4795671)(121.3734736,338.48956709)(121.31347732,338.48956697)
\curveto(121.25347372,338.48956709)(121.19847378,338.49956708)(121.14847732,338.51956697)
\curveto(121.03847394,338.53956704)(120.92847405,338.56456702)(120.81847732,338.59456697)
\curveto(120.70847427,338.61456697)(120.60847437,338.64956693)(120.51847732,338.69956697)
\curveto(120.40847457,338.73956684)(120.30347467,338.77456681)(120.20347732,338.80456697)
\curveto(120.11347486,338.84456674)(120.02847495,338.88956669)(119.94847732,338.93956697)
\curveto(119.62847535,339.13956644)(119.34347563,339.36956621)(119.09347732,339.62956697)
\curveto(118.84347613,339.89956568)(118.63847634,340.20956537)(118.47847732,340.55956697)
\curveto(118.42847655,340.66956491)(118.38847659,340.7795648)(118.35847732,340.88956697)
\curveto(118.32847665,341.00956457)(118.28847669,341.12956445)(118.23847732,341.24956697)
\curveto(118.22847675,341.28956429)(118.22347675,341.32456426)(118.22347732,341.35456697)
\curveto(118.22347675,341.39456419)(118.21847676,341.43456415)(118.20847732,341.47456697)
\curveto(118.16847681,341.59456399)(118.14347683,341.72456386)(118.13347732,341.86456697)
\lineto(118.10347732,342.28456697)
\curveto(118.10347687,342.33456325)(118.09847688,342.38956319)(118.08847732,342.44956697)
\curveto(118.08847689,342.50956307)(118.09347688,342.56456302)(118.10347732,342.61456697)
\lineto(118.10347732,342.79456697)
\lineto(118.14847732,343.15456697)
\curveto(118.18847679,343.32456226)(118.22347675,343.48956209)(118.25347732,343.64956697)
\curveto(118.28347669,343.80956177)(118.32847665,343.95956162)(118.38847732,344.09956697)
\curveto(118.81847616,345.13956044)(119.54847543,345.87455971)(120.57847732,346.30456697)
\curveto(120.71847426,346.36455922)(120.85847412,346.40455918)(120.99847732,346.42456697)
\curveto(121.14847383,346.45455913)(121.30347367,346.48955909)(121.46347732,346.52956697)
\curveto(121.54347343,346.53955904)(121.61847336,346.54455904)(121.68847732,346.54456697)
\curveto(121.75847322,346.54455904)(121.83347314,346.54955903)(121.91347732,346.55956697)
\curveto(122.42347255,346.56955901)(122.85847212,346.50955907)(123.21847732,346.37956697)
\curveto(123.58847139,346.25955932)(123.91847106,346.09955948)(124.20847732,345.89956697)
\curveto(124.29847068,345.83955974)(124.38847059,345.76955981)(124.47847732,345.68956697)
\curveto(124.56847041,345.61955996)(124.64847033,345.54456004)(124.71847732,345.46456697)
\curveto(124.74847023,345.41456017)(124.78847019,345.37456021)(124.83847732,345.34456697)
\curveto(124.91847006,345.23456035)(124.99346998,345.11956046)(125.06347732,344.99956697)
\curveto(125.13346984,344.88956069)(125.20846977,344.77456081)(125.28847732,344.65456697)
\curveto(125.33846964,344.56456102)(125.3784696,344.46956111)(125.40847732,344.36956697)
\curveto(125.44846953,344.2795613)(125.48846949,344.1795614)(125.52847732,344.06956697)
\curveto(125.5784694,343.93956164)(125.61846936,343.80456178)(125.64847732,343.66456697)
\curveto(125.6784693,343.52456206)(125.71346926,343.3845622)(125.75347732,343.24456697)
\curveto(125.7734692,343.16456242)(125.7784692,343.07456251)(125.76847732,342.97456697)
\curveto(125.76846921,342.8845627)(125.7784692,342.79956278)(125.79847732,342.71956697)
\lineto(125.79847732,342.55456697)
\moveto(123.54847732,343.43956697)
\curveto(123.61847136,343.53956204)(123.62347135,343.65956192)(123.56347732,343.79956697)
\curveto(123.51347146,343.94956163)(123.4734715,344.05956152)(123.44347732,344.12956697)
\curveto(123.30347167,344.39956118)(123.11847186,344.60456098)(122.88847732,344.74456697)
\curveto(122.65847232,344.89456069)(122.33847264,344.97456061)(121.92847732,344.98456697)
\curveto(121.89847308,344.96456062)(121.86347311,344.95956062)(121.82347732,344.96956697)
\curveto(121.78347319,344.9795606)(121.74847323,344.9795606)(121.71847732,344.96956697)
\curveto(121.66847331,344.94956063)(121.61347336,344.93456065)(121.55347732,344.92456697)
\curveto(121.49347348,344.92456066)(121.43847354,344.91456067)(121.38847732,344.89456697)
\curveto(120.94847403,344.75456083)(120.62347435,344.4795611)(120.41347732,344.06956697)
\curveto(120.39347458,344.02956155)(120.36847461,343.97456161)(120.33847732,343.90456697)
\curveto(120.31847466,343.84456174)(120.30347467,343.7795618)(120.29347732,343.70956697)
\curveto(120.28347469,343.64956193)(120.28347469,343.58956199)(120.29347732,343.52956697)
\curveto(120.31347466,343.46956211)(120.34847463,343.41956216)(120.39847732,343.37956697)
\curveto(120.4784745,343.32956225)(120.58847439,343.30456228)(120.72847732,343.30456697)
\lineto(121.13347732,343.30456697)
\lineto(122.79847732,343.30456697)
\lineto(123.23347732,343.30456697)
\curveto(123.39347158,343.31456227)(123.49847148,343.35956222)(123.54847732,343.43956697)
}
}
{
\newrgbcolor{curcolor}{0 0 0}
\pscustom[linestyle=none,fillstyle=solid,fillcolor=curcolor]
{
}
}
{
\newrgbcolor{curcolor}{0 0 0}
\pscustom[linestyle=none,fillstyle=solid,fillcolor=curcolor]
{
\newpath
\moveto(138.57191482,342.55456697)
\curveto(138.59190666,342.47456311)(138.59190666,342.3845632)(138.57191482,342.28456697)
\curveto(138.5519067,342.1845634)(138.51690673,342.11956346)(138.46691482,342.08956697)
\curveto(138.41690683,342.04956353)(138.34190691,342.01956356)(138.24191482,341.99956697)
\curveto(138.1519071,341.98956359)(138.0469072,341.9795636)(137.92691482,341.96956697)
\lineto(137.58191482,341.96956697)
\curveto(137.47190778,341.9795636)(137.37190788,341.9845636)(137.28191482,341.98456697)
\lineto(133.62191482,341.98456697)
\lineto(133.41191482,341.98456697)
\curveto(133.3519119,341.9845636)(133.29691195,341.97456361)(133.24691482,341.95456697)
\curveto(133.16691208,341.91456367)(133.11691213,341.87456371)(133.09691482,341.83456697)
\curveto(133.07691217,341.81456377)(133.05691219,341.77456381)(133.03691482,341.71456697)
\curveto(133.01691223,341.66456392)(133.01191224,341.61456397)(133.02191482,341.56456697)
\curveto(133.04191221,341.50456408)(133.0519122,341.44456414)(133.05191482,341.38456697)
\curveto(133.06191219,341.33456425)(133.07691217,341.2795643)(133.09691482,341.21956697)
\curveto(133.17691207,340.9795646)(133.27191198,340.7795648)(133.38191482,340.61956697)
\curveto(133.50191175,340.46956511)(133.66191159,340.33456525)(133.86191482,340.21456697)
\curveto(133.94191131,340.16456542)(134.02191123,340.12956545)(134.10191482,340.10956697)
\curveto(134.19191106,340.09956548)(134.28191097,340.0795655)(134.37191482,340.04956697)
\curveto(134.4519108,340.02956555)(134.56191069,340.01456557)(134.70191482,340.00456697)
\curveto(134.84191041,339.99456559)(134.96191029,339.99956558)(135.06191482,340.01956697)
\lineto(135.19691482,340.01956697)
\curveto(135.29690995,340.03956554)(135.38690986,340.05956552)(135.46691482,340.07956697)
\curveto(135.55690969,340.10956547)(135.64190961,340.13956544)(135.72191482,340.16956697)
\curveto(135.82190943,340.21956536)(135.93190932,340.2845653)(136.05191482,340.36456697)
\curveto(136.18190907,340.44456514)(136.27690897,340.52456506)(136.33691482,340.60456697)
\curveto(136.38690886,340.67456491)(136.43690881,340.73956484)(136.48691482,340.79956697)
\curveto(136.5469087,340.86956471)(136.61690863,340.91956466)(136.69691482,340.94956697)
\curveto(136.79690845,340.99956458)(136.92190833,341.01956456)(137.07191482,341.00956697)
\lineto(137.50691482,341.00956697)
\lineto(137.68691482,341.00956697)
\curveto(137.75690749,341.01956456)(137.81690743,341.01456457)(137.86691482,340.99456697)
\lineto(138.01691482,340.99456697)
\curveto(138.11690713,340.97456461)(138.18690706,340.94956463)(138.22691482,340.91956697)
\curveto(138.26690698,340.89956468)(138.28690696,340.85456473)(138.28691482,340.78456697)
\curveto(138.29690695,340.71456487)(138.29190696,340.65456493)(138.27191482,340.60456697)
\curveto(138.22190703,340.46456512)(138.16690708,340.33956524)(138.10691482,340.22956697)
\curveto(138.0469072,340.11956546)(137.97690727,340.00956557)(137.89691482,339.89956697)
\curveto(137.67690757,339.56956601)(137.42690782,339.30456628)(137.14691482,339.10456697)
\curveto(136.86690838,338.90456668)(136.51690873,338.73456685)(136.09691482,338.59456697)
\curveto(135.98690926,338.55456703)(135.87690937,338.52956705)(135.76691482,338.51956697)
\curveto(135.65690959,338.50956707)(135.54190971,338.48956709)(135.42191482,338.45956697)
\curveto(135.38190987,338.44956713)(135.33690991,338.44956713)(135.28691482,338.45956697)
\curveto(135.24691,338.45956712)(135.20691004,338.45456713)(135.16691482,338.44456697)
\lineto(135.00191482,338.44456697)
\curveto(134.9519103,338.42456716)(134.89191036,338.41956716)(134.82191482,338.42956697)
\curveto(134.76191049,338.42956715)(134.70691054,338.43456715)(134.65691482,338.44456697)
\curveto(134.57691067,338.45456713)(134.50691074,338.45456713)(134.44691482,338.44456697)
\curveto(134.38691086,338.43456715)(134.32191093,338.43956714)(134.25191482,338.45956697)
\curveto(134.20191105,338.4795671)(134.1469111,338.48956709)(134.08691482,338.48956697)
\curveto(134.02691122,338.48956709)(133.97191128,338.49956708)(133.92191482,338.51956697)
\curveto(133.81191144,338.53956704)(133.70191155,338.56456702)(133.59191482,338.59456697)
\curveto(133.48191177,338.61456697)(133.38191187,338.64956693)(133.29191482,338.69956697)
\curveto(133.18191207,338.73956684)(133.07691217,338.77456681)(132.97691482,338.80456697)
\curveto(132.88691236,338.84456674)(132.80191245,338.88956669)(132.72191482,338.93956697)
\curveto(132.40191285,339.13956644)(132.11691313,339.36956621)(131.86691482,339.62956697)
\curveto(131.61691363,339.89956568)(131.41191384,340.20956537)(131.25191482,340.55956697)
\curveto(131.20191405,340.66956491)(131.16191409,340.7795648)(131.13191482,340.88956697)
\curveto(131.10191415,341.00956457)(131.06191419,341.12956445)(131.01191482,341.24956697)
\curveto(131.00191425,341.28956429)(130.99691425,341.32456426)(130.99691482,341.35456697)
\curveto(130.99691425,341.39456419)(130.99191426,341.43456415)(130.98191482,341.47456697)
\curveto(130.94191431,341.59456399)(130.91691433,341.72456386)(130.90691482,341.86456697)
\lineto(130.87691482,342.28456697)
\curveto(130.87691437,342.33456325)(130.87191438,342.38956319)(130.86191482,342.44956697)
\curveto(130.86191439,342.50956307)(130.86691438,342.56456302)(130.87691482,342.61456697)
\lineto(130.87691482,342.79456697)
\lineto(130.92191482,343.15456697)
\curveto(130.96191429,343.32456226)(130.99691425,343.48956209)(131.02691482,343.64956697)
\curveto(131.05691419,343.80956177)(131.10191415,343.95956162)(131.16191482,344.09956697)
\curveto(131.59191366,345.13956044)(132.32191293,345.87455971)(133.35191482,346.30456697)
\curveto(133.49191176,346.36455922)(133.63191162,346.40455918)(133.77191482,346.42456697)
\curveto(133.92191133,346.45455913)(134.07691117,346.48955909)(134.23691482,346.52956697)
\curveto(134.31691093,346.53955904)(134.39191086,346.54455904)(134.46191482,346.54456697)
\curveto(134.53191072,346.54455904)(134.60691064,346.54955903)(134.68691482,346.55956697)
\curveto(135.19691005,346.56955901)(135.63190962,346.50955907)(135.99191482,346.37956697)
\curveto(136.36190889,346.25955932)(136.69190856,346.09955948)(136.98191482,345.89956697)
\curveto(137.07190818,345.83955974)(137.16190809,345.76955981)(137.25191482,345.68956697)
\curveto(137.34190791,345.61955996)(137.42190783,345.54456004)(137.49191482,345.46456697)
\curveto(137.52190773,345.41456017)(137.56190769,345.37456021)(137.61191482,345.34456697)
\curveto(137.69190756,345.23456035)(137.76690748,345.11956046)(137.83691482,344.99956697)
\curveto(137.90690734,344.88956069)(137.98190727,344.77456081)(138.06191482,344.65456697)
\curveto(138.11190714,344.56456102)(138.1519071,344.46956111)(138.18191482,344.36956697)
\curveto(138.22190703,344.2795613)(138.26190699,344.1795614)(138.30191482,344.06956697)
\curveto(138.3519069,343.93956164)(138.39190686,343.80456178)(138.42191482,343.66456697)
\curveto(138.4519068,343.52456206)(138.48690676,343.3845622)(138.52691482,343.24456697)
\curveto(138.5469067,343.16456242)(138.5519067,343.07456251)(138.54191482,342.97456697)
\curveto(138.54190671,342.8845627)(138.5519067,342.79956278)(138.57191482,342.71956697)
\lineto(138.57191482,342.55456697)
\moveto(136.32191482,343.43956697)
\curveto(136.39190886,343.53956204)(136.39690885,343.65956192)(136.33691482,343.79956697)
\curveto(136.28690896,343.94956163)(136.246909,344.05956152)(136.21691482,344.12956697)
\curveto(136.07690917,344.39956118)(135.89190936,344.60456098)(135.66191482,344.74456697)
\curveto(135.43190982,344.89456069)(135.11191014,344.97456061)(134.70191482,344.98456697)
\curveto(134.67191058,344.96456062)(134.63691061,344.95956062)(134.59691482,344.96956697)
\curveto(134.55691069,344.9795606)(134.52191073,344.9795606)(134.49191482,344.96956697)
\curveto(134.44191081,344.94956063)(134.38691086,344.93456065)(134.32691482,344.92456697)
\curveto(134.26691098,344.92456066)(134.21191104,344.91456067)(134.16191482,344.89456697)
\curveto(133.72191153,344.75456083)(133.39691185,344.4795611)(133.18691482,344.06956697)
\curveto(133.16691208,344.02956155)(133.14191211,343.97456161)(133.11191482,343.90456697)
\curveto(133.09191216,343.84456174)(133.07691217,343.7795618)(133.06691482,343.70956697)
\curveto(133.05691219,343.64956193)(133.05691219,343.58956199)(133.06691482,343.52956697)
\curveto(133.08691216,343.46956211)(133.12191213,343.41956216)(133.17191482,343.37956697)
\curveto(133.251912,343.32956225)(133.36191189,343.30456228)(133.50191482,343.30456697)
\lineto(133.90691482,343.30456697)
\lineto(135.57191482,343.30456697)
\lineto(136.00691482,343.30456697)
\curveto(136.16690908,343.31456227)(136.27190898,343.35956222)(136.32191482,343.43956697)
}
}
{
\newrgbcolor{curcolor}{0 0 0}
\pscustom[linestyle=none,fillstyle=solid,fillcolor=curcolor]
{
\newpath
\moveto(142.79019607,346.55956697)
\curveto(143.54019157,346.579559)(144.19019092,346.49455909)(144.74019607,346.30456697)
\curveto(145.30018981,346.12455946)(145.72518939,345.80955977)(146.01519607,345.35956697)
\curveto(146.08518903,345.24956033)(146.14518897,345.13456045)(146.19519607,345.01456697)
\curveto(146.25518886,344.90456068)(146.30518881,344.7795608)(146.34519607,344.63956697)
\curveto(146.36518875,344.579561)(146.37518874,344.51456107)(146.37519607,344.44456697)
\curveto(146.37518874,344.37456121)(146.36518875,344.31456127)(146.34519607,344.26456697)
\curveto(146.30518881,344.20456138)(146.25018886,344.16456142)(146.18019607,344.14456697)
\curveto(146.13018898,344.12456146)(146.07018904,344.11456147)(146.00019607,344.11456697)
\lineto(145.79019607,344.11456697)
\lineto(145.13019607,344.11456697)
\curveto(145.06019005,344.11456147)(144.99019012,344.10956147)(144.92019607,344.09956697)
\curveto(144.85019026,344.09956148)(144.78519033,344.10956147)(144.72519607,344.12956697)
\curveto(144.62519049,344.14956143)(144.55019056,344.18956139)(144.50019607,344.24956697)
\curveto(144.45019066,344.30956127)(144.40519071,344.36956121)(144.36519607,344.42956697)
\lineto(144.24519607,344.63956697)
\curveto(144.2151909,344.71956086)(144.16519095,344.7845608)(144.09519607,344.83456697)
\curveto(143.99519112,344.91456067)(143.89519122,344.97456061)(143.79519607,345.01456697)
\curveto(143.70519141,345.05456053)(143.59019152,345.08956049)(143.45019607,345.11956697)
\curveto(143.38019173,345.13956044)(143.27519184,345.15456043)(143.13519607,345.16456697)
\curveto(143.00519211,345.17456041)(142.90519221,345.16956041)(142.83519607,345.14956697)
\lineto(142.73019607,345.14956697)
\lineto(142.58019607,345.11956697)
\curveto(142.54019257,345.11956046)(142.49519262,345.11456047)(142.44519607,345.10456697)
\curveto(142.27519284,345.05456053)(142.13519298,344.9845606)(142.02519607,344.89456697)
\curveto(141.92519319,344.81456077)(141.85519326,344.68956089)(141.81519607,344.51956697)
\curveto(141.79519332,344.44956113)(141.79519332,344.3845612)(141.81519607,344.32456697)
\curveto(141.83519328,344.26456132)(141.85519326,344.21456137)(141.87519607,344.17456697)
\curveto(141.94519317,344.05456153)(142.02519309,343.95956162)(142.11519607,343.88956697)
\curveto(142.2151929,343.81956176)(142.33019278,343.75956182)(142.46019607,343.70956697)
\curveto(142.65019246,343.62956195)(142.85519226,343.55956202)(143.07519607,343.49956697)
\lineto(143.76519607,343.34956697)
\curveto(144.00519111,343.30956227)(144.23519088,343.25956232)(144.45519607,343.19956697)
\curveto(144.68519043,343.14956243)(144.90019021,343.0845625)(145.10019607,343.00456697)
\curveto(145.19018992,342.96456262)(145.27518984,342.92956265)(145.35519607,342.89956697)
\curveto(145.44518967,342.8795627)(145.53018958,342.84456274)(145.61019607,342.79456697)
\curveto(145.80018931,342.67456291)(145.97018914,342.54456304)(146.12019607,342.40456697)
\curveto(146.28018883,342.26456332)(146.40518871,342.08956349)(146.49519607,341.87956697)
\curveto(146.52518859,341.80956377)(146.55018856,341.73956384)(146.57019607,341.66956697)
\curveto(146.59018852,341.59956398)(146.6101885,341.52456406)(146.63019607,341.44456697)
\curveto(146.64018847,341.3845642)(146.64518847,341.28956429)(146.64519607,341.15956697)
\curveto(146.65518846,341.03956454)(146.65518846,340.94456464)(146.64519607,340.87456697)
\lineto(146.64519607,340.79956697)
\curveto(146.62518849,340.73956484)(146.6101885,340.6795649)(146.60019607,340.61956697)
\curveto(146.60018851,340.56956501)(146.59518852,340.51956506)(146.58519607,340.46956697)
\curveto(146.5151886,340.16956541)(146.40518871,339.90456568)(146.25519607,339.67456697)
\curveto(146.09518902,339.43456615)(145.90018921,339.23956634)(145.67019607,339.08956697)
\curveto(145.44018967,338.93956664)(145.18018993,338.80956677)(144.89019607,338.69956697)
\curveto(144.78019033,338.64956693)(144.66019045,338.61456697)(144.53019607,338.59456697)
\curveto(144.4101907,338.57456701)(144.29019082,338.54956703)(144.17019607,338.51956697)
\curveto(144.08019103,338.49956708)(143.98519113,338.48956709)(143.88519607,338.48956697)
\curveto(143.79519132,338.4795671)(143.70519141,338.46456712)(143.61519607,338.44456697)
\lineto(143.34519607,338.44456697)
\curveto(143.28519183,338.42456716)(143.18019193,338.41456717)(143.03019607,338.41456697)
\curveto(142.89019222,338.41456717)(142.79019232,338.42456716)(142.73019607,338.44456697)
\curveto(142.70019241,338.44456714)(142.66519245,338.44956713)(142.62519607,338.45956697)
\lineto(142.52019607,338.45956697)
\curveto(142.40019271,338.4795671)(142.28019283,338.49456709)(142.16019607,338.50456697)
\curveto(142.04019307,338.51456707)(141.92519319,338.53456705)(141.81519607,338.56456697)
\curveto(141.42519369,338.67456691)(141.08019403,338.79956678)(140.78019607,338.93956697)
\curveto(140.48019463,339.08956649)(140.22519489,339.30956627)(140.01519607,339.59956697)
\curveto(139.87519524,339.78956579)(139.75519536,340.00956557)(139.65519607,340.25956697)
\curveto(139.63519548,340.31956526)(139.6151955,340.39956518)(139.59519607,340.49956697)
\curveto(139.57519554,340.54956503)(139.56019555,340.61956496)(139.55019607,340.70956697)
\curveto(139.54019557,340.79956478)(139.54519557,340.87456471)(139.56519607,340.93456697)
\curveto(139.59519552,341.00456458)(139.64519547,341.05456453)(139.71519607,341.08456697)
\curveto(139.76519535,341.10456448)(139.82519529,341.11456447)(139.89519607,341.11456697)
\lineto(140.12019607,341.11456697)
\lineto(140.82519607,341.11456697)
\lineto(141.06519607,341.11456697)
\curveto(141.14519397,341.11456447)(141.2151939,341.10456448)(141.27519607,341.08456697)
\curveto(141.38519373,341.04456454)(141.45519366,340.9795646)(141.48519607,340.88956697)
\curveto(141.52519359,340.79956478)(141.57019354,340.70456488)(141.62019607,340.60456697)
\curveto(141.64019347,340.55456503)(141.67519344,340.48956509)(141.72519607,340.40956697)
\curveto(141.78519333,340.32956525)(141.83519328,340.2795653)(141.87519607,340.25956697)
\curveto(141.99519312,340.15956542)(142.110193,340.0795655)(142.22019607,340.01956697)
\curveto(142.33019278,339.96956561)(142.47019264,339.91956566)(142.64019607,339.86956697)
\curveto(142.69019242,339.84956573)(142.74019237,339.83956574)(142.79019607,339.83956697)
\curveto(142.84019227,339.84956573)(142.89019222,339.84956573)(142.94019607,339.83956697)
\curveto(143.02019209,339.81956576)(143.10519201,339.80956577)(143.19519607,339.80956697)
\curveto(143.29519182,339.81956576)(143.38019173,339.83456575)(143.45019607,339.85456697)
\curveto(143.50019161,339.86456572)(143.54519157,339.86956571)(143.58519607,339.86956697)
\curveto(143.63519148,339.86956571)(143.68519143,339.8795657)(143.73519607,339.89956697)
\curveto(143.87519124,339.94956563)(144.00019111,340.00956557)(144.11019607,340.07956697)
\curveto(144.23019088,340.14956543)(144.32519079,340.23956534)(144.39519607,340.34956697)
\curveto(144.44519067,340.42956515)(144.48519063,340.55456503)(144.51519607,340.72456697)
\curveto(144.53519058,340.79456479)(144.53519058,340.85956472)(144.51519607,340.91956697)
\curveto(144.49519062,340.9795646)(144.47519064,341.02956455)(144.45519607,341.06956697)
\curveto(144.38519073,341.20956437)(144.29519082,341.31456427)(144.18519607,341.38456697)
\curveto(144.08519103,341.45456413)(143.96519115,341.51956406)(143.82519607,341.57956697)
\curveto(143.63519148,341.65956392)(143.43519168,341.72456386)(143.22519607,341.77456697)
\curveto(143.0151921,341.82456376)(142.80519231,341.8795637)(142.59519607,341.93956697)
\curveto(142.5151926,341.95956362)(142.43019268,341.97456361)(142.34019607,341.98456697)
\curveto(142.26019285,341.99456359)(142.18019293,342.00956357)(142.10019607,342.02956697)
\curveto(141.78019333,342.11956346)(141.47519364,342.20456338)(141.18519607,342.28456697)
\curveto(140.89519422,342.37456321)(140.63019448,342.50456308)(140.39019607,342.67456697)
\curveto(140.110195,342.87456271)(139.90519521,343.14456244)(139.77519607,343.48456697)
\curveto(139.75519536,343.55456203)(139.73519538,343.64956193)(139.71519607,343.76956697)
\curveto(139.69519542,343.83956174)(139.68019543,343.92456166)(139.67019607,344.02456697)
\curveto(139.66019545,344.12456146)(139.66519545,344.21456137)(139.68519607,344.29456697)
\curveto(139.70519541,344.34456124)(139.7101954,344.3845612)(139.70019607,344.41456697)
\curveto(139.69019542,344.45456113)(139.69519542,344.49956108)(139.71519607,344.54956697)
\curveto(139.73519538,344.65956092)(139.75519536,344.75956082)(139.77519607,344.84956697)
\curveto(139.80519531,344.94956063)(139.84019527,345.04456054)(139.88019607,345.13456697)
\curveto(140.0101951,345.42456016)(140.19019492,345.65955992)(140.42019607,345.83956697)
\curveto(140.65019446,346.01955956)(140.9101942,346.16455942)(141.20019607,346.27456697)
\curveto(141.3101938,346.32455926)(141.42519369,346.35955922)(141.54519607,346.37956697)
\curveto(141.66519345,346.40955917)(141.79019332,346.43955914)(141.92019607,346.46956697)
\curveto(141.98019313,346.48955909)(142.04019307,346.49955908)(142.10019607,346.49956697)
\lineto(142.28019607,346.52956697)
\curveto(142.36019275,346.53955904)(142.44519267,346.54455904)(142.53519607,346.54456697)
\curveto(142.62519249,346.54455904)(142.7101924,346.54955903)(142.79019607,346.55956697)
}
}
{
\newrgbcolor{curcolor}{0 0 0}
\pscustom[linestyle=none,fillstyle=solid,fillcolor=curcolor]
{
\newpath
\moveto(155.7668367,342.56956697)
\curveto(155.77682802,342.50956307)(155.78182801,342.41956316)(155.7818367,342.29956697)
\curveto(155.78182801,342.1795634)(155.77182802,342.09456349)(155.7518367,342.04456697)
\lineto(155.7518367,341.84956697)
\curveto(155.72182807,341.73956384)(155.70182809,341.63456395)(155.6918367,341.53456697)
\curveto(155.6918281,341.43456415)(155.67682812,341.33456425)(155.6468367,341.23456697)
\curveto(155.62682817,341.14456444)(155.60682819,341.04956453)(155.5868367,340.94956697)
\curveto(155.56682823,340.85956472)(155.53682826,340.76956481)(155.4968367,340.67956697)
\curveto(155.42682837,340.50956507)(155.35682844,340.34956523)(155.2868367,340.19956697)
\curveto(155.21682858,340.05956552)(155.13682866,339.91956566)(155.0468367,339.77956697)
\curveto(154.98682881,339.68956589)(154.92182887,339.60456598)(154.8518367,339.52456697)
\curveto(154.791829,339.45456613)(154.72182907,339.3795662)(154.6418367,339.29956697)
\lineto(154.5368367,339.19456697)
\curveto(154.48682931,339.14456644)(154.43182936,339.09956648)(154.3718367,339.05956697)
\lineto(154.2218367,338.93956697)
\curveto(154.14182965,338.8795667)(154.05182974,338.82456676)(153.9518367,338.77456697)
\curveto(153.86182993,338.73456685)(153.76683003,338.68956689)(153.6668367,338.63956697)
\curveto(153.56683023,338.58956699)(153.46183033,338.55456703)(153.3518367,338.53456697)
\curveto(153.25183054,338.51456707)(153.14683065,338.49456709)(153.0368367,338.47456697)
\curveto(152.97683082,338.45456713)(152.91183088,338.44456714)(152.8418367,338.44456697)
\curveto(152.78183101,338.44456714)(152.71683108,338.43456715)(152.6468367,338.41456697)
\lineto(152.5118367,338.41456697)
\curveto(152.43183136,338.39456719)(152.35683144,338.39456719)(152.2868367,338.41456697)
\lineto(152.1368367,338.41456697)
\curveto(152.07683172,338.43456715)(152.01183178,338.44456714)(151.9418367,338.44456697)
\curveto(151.88183191,338.43456715)(151.82183197,338.43956714)(151.7618367,338.45956697)
\curveto(151.60183219,338.50956707)(151.44683235,338.55456703)(151.2968367,338.59456697)
\curveto(151.15683264,338.63456695)(151.02683277,338.69456689)(150.9068367,338.77456697)
\curveto(150.83683296,338.81456677)(150.77183302,338.85456673)(150.7118367,338.89456697)
\curveto(150.65183314,338.94456664)(150.58683321,338.99456659)(150.5168367,339.04456697)
\lineto(150.3368367,339.17956697)
\curveto(150.25683354,339.23956634)(150.18683361,339.24456634)(150.1268367,339.19456697)
\curveto(150.07683372,339.16456642)(150.05183374,339.12456646)(150.0518367,339.07456697)
\curveto(150.05183374,339.03456655)(150.04183375,338.9845666)(150.0218367,338.92456697)
\curveto(150.00183379,338.82456676)(149.9918338,338.70956687)(149.9918367,338.57956697)
\curveto(150.00183379,338.44956713)(150.00683379,338.32956725)(150.0068367,338.21956697)
\lineto(150.0068367,336.68956697)
\curveto(150.00683379,336.55956902)(150.00183379,336.43456915)(149.9918367,336.31456697)
\curveto(149.9918338,336.1845694)(149.96683383,336.0795695)(149.9168367,335.99956697)
\curveto(149.88683391,335.95956962)(149.83183396,335.92956965)(149.7518367,335.90956697)
\curveto(149.67183412,335.88956969)(149.58183421,335.8795697)(149.4818367,335.87956697)
\curveto(149.38183441,335.86956971)(149.28183451,335.86956971)(149.1818367,335.87956697)
\lineto(148.9268367,335.87956697)
\lineto(148.5218367,335.87956697)
\lineto(148.4168367,335.87956697)
\curveto(148.37683542,335.8795697)(148.34183545,335.8845697)(148.3118367,335.89456697)
\lineto(148.1918367,335.89456697)
\curveto(148.02183577,335.94456964)(147.93183586,336.04456954)(147.9218367,336.19456697)
\curveto(147.91183588,336.33456925)(147.90683589,336.50456908)(147.9068367,336.70456697)
\lineto(147.9068367,345.50956697)
\curveto(147.90683589,345.61955996)(147.90183589,345.73455985)(147.8918367,345.85456697)
\curveto(147.8918359,345.9845596)(147.91683588,346.0845595)(147.9668367,346.15456697)
\curveto(148.00683579,346.22455936)(148.06183573,346.26955931)(148.1318367,346.28956697)
\curveto(148.18183561,346.30955927)(148.24183555,346.31955926)(148.3118367,346.31956697)
\lineto(148.5368367,346.31956697)
\lineto(149.2568367,346.31956697)
\lineto(149.5418367,346.31956697)
\curveto(149.63183416,346.31955926)(149.70683409,346.29455929)(149.7668367,346.24456697)
\curveto(149.83683396,346.19455939)(149.87183392,346.12955945)(149.8718367,346.04956697)
\curveto(149.88183391,345.9795596)(149.90683389,345.90455968)(149.9468367,345.82456697)
\curveto(149.95683384,345.79455979)(149.96683383,345.76955981)(149.9768367,345.74956697)
\curveto(149.9968338,345.73955984)(150.01683378,345.72455986)(150.0368367,345.70456697)
\curveto(150.14683365,345.69455989)(150.23683356,345.72455986)(150.3068367,345.79456697)
\curveto(150.37683342,345.86455972)(150.44683335,345.92455966)(150.5168367,345.97456697)
\curveto(150.64683315,346.06455952)(150.78183301,346.14455944)(150.9218367,346.21456697)
\curveto(151.06183273,346.29455929)(151.21683258,346.35955922)(151.3868367,346.40956697)
\curveto(151.46683233,346.43955914)(151.55183224,346.45955912)(151.6418367,346.46956697)
\curveto(151.74183205,346.4795591)(151.83683196,346.49455909)(151.9268367,346.51456697)
\curveto(151.96683183,346.52455906)(152.00683179,346.52455906)(152.0468367,346.51456697)
\curveto(152.0968317,346.50455908)(152.13683166,346.50955907)(152.1668367,346.52956697)
\curveto(152.73683106,346.54955903)(153.21683058,346.46955911)(153.6068367,346.28956697)
\curveto(154.00682979,346.11955946)(154.34682945,345.89455969)(154.6268367,345.61456697)
\curveto(154.67682912,345.56456002)(154.72182907,345.51456007)(154.7618367,345.46456697)
\curveto(154.80182899,345.42456016)(154.84182895,345.3795602)(154.8818367,345.32956697)
\curveto(154.95182884,345.23956034)(155.01182878,345.14956043)(155.0618367,345.05956697)
\curveto(155.12182867,344.96956061)(155.17682862,344.8795607)(155.2268367,344.78956697)
\curveto(155.24682855,344.76956081)(155.25682854,344.74456084)(155.2568367,344.71456697)
\curveto(155.26682853,344.6845609)(155.28182851,344.64956093)(155.3018367,344.60956697)
\curveto(155.36182843,344.50956107)(155.41682838,344.38956119)(155.4668367,344.24956697)
\curveto(155.48682831,344.18956139)(155.50682829,344.12456146)(155.5268367,344.05456697)
\curveto(155.54682825,343.99456159)(155.56682823,343.92956165)(155.5868367,343.85956697)
\curveto(155.62682817,343.73956184)(155.65182814,343.61456197)(155.6618367,343.48456697)
\curveto(155.68182811,343.35456223)(155.70682809,343.21956236)(155.7368367,343.07956697)
\lineto(155.7368367,342.91456697)
\lineto(155.7668367,342.73456697)
\lineto(155.7668367,342.56956697)
\moveto(153.6518367,342.22456697)
\curveto(153.66183013,342.27456331)(153.66683013,342.33956324)(153.6668367,342.41956697)
\curveto(153.66683013,342.50956307)(153.66183013,342.579563)(153.6518367,342.62956697)
\lineto(153.6518367,342.76456697)
\curveto(153.63183016,342.82456276)(153.62183017,342.88956269)(153.6218367,342.95956697)
\curveto(153.62183017,343.02956255)(153.61183018,343.09956248)(153.5918367,343.16956697)
\curveto(153.57183022,343.26956231)(153.55183024,343.36456222)(153.5318367,343.45456697)
\curveto(153.51183028,343.55456203)(153.48183031,343.64456194)(153.4418367,343.72456697)
\curveto(153.32183047,344.04456154)(153.16683063,344.29956128)(152.9768367,344.48956697)
\curveto(152.78683101,344.6795609)(152.51683128,344.81956076)(152.1668367,344.90956697)
\curveto(152.08683171,344.92956065)(151.9968318,344.93956064)(151.8968367,344.93956697)
\lineto(151.6268367,344.93956697)
\curveto(151.58683221,344.92956065)(151.55183224,344.92456066)(151.5218367,344.92456697)
\curveto(151.4918323,344.92456066)(151.45683234,344.91956066)(151.4168367,344.90956697)
\lineto(151.2068367,344.84956697)
\curveto(151.14683265,344.83956074)(151.08683271,344.81956076)(151.0268367,344.78956697)
\curveto(150.76683303,344.6795609)(150.56183323,344.50956107)(150.4118367,344.27956697)
\curveto(150.27183352,344.04956153)(150.15683364,343.79456179)(150.0668367,343.51456697)
\curveto(150.04683375,343.43456215)(150.03183376,343.34956223)(150.0218367,343.25956697)
\curveto(150.01183378,343.1795624)(149.9968338,343.09956248)(149.9768367,343.01956697)
\curveto(149.96683383,342.9795626)(149.96183383,342.91456267)(149.9618367,342.82456697)
\curveto(149.94183385,342.7845628)(149.93683386,342.73456285)(149.9468367,342.67456697)
\curveto(149.95683384,342.62456296)(149.95683384,342.57456301)(149.9468367,342.52456697)
\curveto(149.92683387,342.46456312)(149.92683387,342.40956317)(149.9468367,342.35956697)
\lineto(149.9468367,342.17956697)
\lineto(149.9468367,342.04456697)
\curveto(149.94683385,342.00456358)(149.95683384,341.96456362)(149.9768367,341.92456697)
\curveto(149.97683382,341.85456373)(149.98183381,341.79956378)(149.9918367,341.75956697)
\lineto(150.0218367,341.57956697)
\curveto(150.03183376,341.51956406)(150.04683375,341.45956412)(150.0668367,341.39956697)
\curveto(150.15683364,341.10956447)(150.26183353,340.86956471)(150.3818367,340.67956697)
\curveto(150.51183328,340.49956508)(150.6918331,340.33956524)(150.9218367,340.19956697)
\curveto(151.06183273,340.11956546)(151.22683257,340.05456553)(151.4168367,340.00456697)
\curveto(151.45683234,339.99456559)(151.4918323,339.98956559)(151.5218367,339.98956697)
\curveto(151.55183224,339.99956558)(151.58683221,339.99956558)(151.6268367,339.98956697)
\curveto(151.66683213,339.9795656)(151.72683207,339.96956561)(151.8068367,339.95956697)
\curveto(151.88683191,339.95956562)(151.95183184,339.96456562)(152.0018367,339.97456697)
\curveto(152.08183171,339.99456559)(152.16183163,340.00956557)(152.2418367,340.01956697)
\curveto(152.33183146,340.03956554)(152.41683138,340.06456552)(152.4968367,340.09456697)
\curveto(152.73683106,340.19456539)(152.93183086,340.33456525)(153.0818367,340.51456697)
\curveto(153.23183056,340.69456489)(153.35683044,340.90456468)(153.4568367,341.14456697)
\curveto(153.50683029,341.26456432)(153.54183025,341.38956419)(153.5618367,341.51956697)
\curveto(153.58183021,341.64956393)(153.60683019,341.7845638)(153.6368367,341.92456697)
\lineto(153.6368367,342.07456697)
\curveto(153.64683015,342.12456346)(153.65183014,342.17456341)(153.6518367,342.22456697)
}
}
{
\newrgbcolor{curcolor}{0 0 0}
\pscustom[linestyle=none,fillstyle=solid,fillcolor=curcolor]
{
\newpath
\moveto(164.09675857,339.20956697)
\curveto(164.11675072,339.09956648)(164.12675071,338.98956659)(164.12675857,338.87956697)
\curveto(164.1367507,338.76956681)(164.08675075,338.69456689)(163.97675857,338.65456697)
\curveto(163.91675092,338.62456696)(163.84675099,338.60956697)(163.76675857,338.60956697)
\lineto(163.52675857,338.60956697)
\lineto(162.71675857,338.60956697)
\lineto(162.44675857,338.60956697)
\curveto(162.36675247,338.61956696)(162.30175254,338.64456694)(162.25175857,338.68456697)
\curveto(162.18175266,338.72456686)(162.12675271,338.7795668)(162.08675857,338.84956697)
\curveto(162.05675278,338.92956665)(162.01175283,338.99456659)(161.95175857,339.04456697)
\curveto(161.93175291,339.06456652)(161.90675293,339.0795665)(161.87675857,339.08956697)
\curveto(161.84675299,339.10956647)(161.80675303,339.11456647)(161.75675857,339.10456697)
\curveto(161.70675313,339.0845665)(161.65675318,339.05956652)(161.60675857,339.02956697)
\curveto(161.56675327,338.99956658)(161.52175332,338.97456661)(161.47175857,338.95456697)
\curveto(161.42175342,338.91456667)(161.36675347,338.8795667)(161.30675857,338.84956697)
\lineto(161.12675857,338.75956697)
\curveto(160.99675384,338.69956688)(160.86175398,338.64956693)(160.72175857,338.60956697)
\curveto(160.58175426,338.579567)(160.4367544,338.54456704)(160.28675857,338.50456697)
\curveto(160.21675462,338.4845671)(160.14675469,338.47456711)(160.07675857,338.47456697)
\curveto(160.01675482,338.46456712)(159.95175489,338.45456713)(159.88175857,338.44456697)
\lineto(159.79175857,338.44456697)
\curveto(159.76175508,338.43456715)(159.73175511,338.42956715)(159.70175857,338.42956697)
\lineto(159.53675857,338.42956697)
\curveto(159.4367554,338.40956717)(159.3367555,338.40956717)(159.23675857,338.42956697)
\lineto(159.10175857,338.42956697)
\curveto(159.03175581,338.44956713)(158.96175588,338.45956712)(158.89175857,338.45956697)
\curveto(158.83175601,338.44956713)(158.77175607,338.45456713)(158.71175857,338.47456697)
\curveto(158.61175623,338.49456709)(158.51675632,338.51456707)(158.42675857,338.53456697)
\curveto(158.3367565,338.54456704)(158.25175659,338.56956701)(158.17175857,338.60956697)
\curveto(157.88175696,338.71956686)(157.63175721,338.85956672)(157.42175857,339.02956697)
\curveto(157.22175762,339.20956637)(157.06175778,339.44456614)(156.94175857,339.73456697)
\curveto(156.91175793,339.80456578)(156.88175796,339.8795657)(156.85175857,339.95956697)
\curveto(156.83175801,340.03956554)(156.81175803,340.12456546)(156.79175857,340.21456697)
\curveto(156.77175807,340.26456532)(156.76175808,340.31456527)(156.76175857,340.36456697)
\curveto(156.77175807,340.41456517)(156.77175807,340.46456512)(156.76175857,340.51456697)
\curveto(156.75175809,340.54456504)(156.7417581,340.60456498)(156.73175857,340.69456697)
\curveto(156.73175811,340.79456479)(156.7367581,340.86456472)(156.74675857,340.90456697)
\curveto(156.76675807,341.00456458)(156.77675806,341.08956449)(156.77675857,341.15956697)
\lineto(156.86675857,341.48956697)
\curveto(156.89675794,341.60956397)(156.9367579,341.71456387)(156.98675857,341.80456697)
\curveto(157.15675768,342.09456349)(157.35175749,342.31456327)(157.57175857,342.46456697)
\curveto(157.79175705,342.61456297)(158.07175677,342.74456284)(158.41175857,342.85456697)
\curveto(158.5417563,342.90456268)(158.67675616,342.93956264)(158.81675857,342.95956697)
\curveto(158.95675588,342.9795626)(159.09675574,343.00456258)(159.23675857,343.03456697)
\curveto(159.31675552,343.05456253)(159.40175544,343.06456252)(159.49175857,343.06456697)
\curveto(159.58175526,343.07456251)(159.67175517,343.08956249)(159.76175857,343.10956697)
\curveto(159.83175501,343.12956245)(159.90175494,343.13456245)(159.97175857,343.12456697)
\curveto(160.0417548,343.12456246)(160.11675472,343.13456245)(160.19675857,343.15456697)
\curveto(160.26675457,343.17456241)(160.3367545,343.1845624)(160.40675857,343.18456697)
\curveto(160.47675436,343.1845624)(160.55175429,343.19456239)(160.63175857,343.21456697)
\curveto(160.841754,343.26456232)(161.03175381,343.30456228)(161.20175857,343.33456697)
\curveto(161.38175346,343.37456221)(161.5417533,343.46456212)(161.68175857,343.60456697)
\curveto(161.77175307,343.69456189)(161.83175301,343.79456179)(161.86175857,343.90456697)
\curveto(161.87175297,343.93456165)(161.87175297,343.95956162)(161.86175857,343.97956697)
\curveto(161.86175298,343.99956158)(161.86675297,344.01956156)(161.87675857,344.03956697)
\curveto(161.88675295,344.05956152)(161.89175295,344.08956149)(161.89175857,344.12956697)
\lineto(161.89175857,344.21956697)
\lineto(161.86175857,344.33956697)
\curveto(161.86175298,344.3795612)(161.85675298,344.41456117)(161.84675857,344.44456697)
\curveto(161.74675309,344.74456084)(161.5367533,344.94956063)(161.21675857,345.05956697)
\curveto(161.12675371,345.08956049)(161.01675382,345.10956047)(160.88675857,345.11956697)
\curveto(160.76675407,345.13956044)(160.6417542,345.14456044)(160.51175857,345.13456697)
\curveto(160.38175446,345.13456045)(160.25675458,345.12456046)(160.13675857,345.10456697)
\curveto(160.01675482,345.0845605)(159.91175493,345.05956052)(159.82175857,345.02956697)
\curveto(159.76175508,345.00956057)(159.70175514,344.9795606)(159.64175857,344.93956697)
\curveto(159.59175525,344.90956067)(159.5417553,344.87456071)(159.49175857,344.83456697)
\curveto(159.4417554,344.79456079)(159.38675545,344.73956084)(159.32675857,344.66956697)
\curveto(159.27675556,344.59956098)(159.2417556,344.53456105)(159.22175857,344.47456697)
\curveto(159.17175567,344.37456121)(159.12675571,344.2795613)(159.08675857,344.18956697)
\curveto(159.05675578,344.09956148)(158.98675585,344.03956154)(158.87675857,344.00956697)
\curveto(158.79675604,343.98956159)(158.71175613,343.9795616)(158.62175857,343.97956697)
\lineto(158.35175857,343.97956697)
\lineto(157.78175857,343.97956697)
\curveto(157.73175711,343.9795616)(157.68175716,343.97456161)(157.63175857,343.96456697)
\curveto(157.58175726,343.96456162)(157.5367573,343.96956161)(157.49675857,343.97956697)
\lineto(157.36175857,343.97956697)
\curveto(157.3417575,343.98956159)(157.31675752,343.99456159)(157.28675857,343.99456697)
\curveto(157.25675758,343.99456159)(157.23175761,344.00456158)(157.21175857,344.02456697)
\curveto(157.13175771,344.04456154)(157.07675776,344.10956147)(157.04675857,344.21956697)
\curveto(157.0367578,344.26956131)(157.0367578,344.31956126)(157.04675857,344.36956697)
\curveto(157.05675778,344.41956116)(157.06675777,344.46456112)(157.07675857,344.50456697)
\curveto(157.10675773,344.61456097)(157.1367577,344.71456087)(157.16675857,344.80456697)
\curveto(157.20675763,344.90456068)(157.25175759,344.99456059)(157.30175857,345.07456697)
\lineto(157.39175857,345.22456697)
\lineto(157.48175857,345.37456697)
\curveto(157.56175728,345.4845601)(157.66175718,345.58955999)(157.78175857,345.68956697)
\curveto(157.80175704,345.69955988)(157.83175701,345.72455986)(157.87175857,345.76456697)
\curveto(157.92175692,345.80455978)(157.96675687,345.83955974)(158.00675857,345.86956697)
\curveto(158.04675679,345.89955968)(158.09175675,345.92955965)(158.14175857,345.95956697)
\curveto(158.31175653,346.06955951)(158.49175635,346.15455943)(158.68175857,346.21456697)
\curveto(158.87175597,346.2845593)(159.06675577,346.34955923)(159.26675857,346.40956697)
\curveto(159.38675545,346.43955914)(159.51175533,346.45955912)(159.64175857,346.46956697)
\curveto(159.77175507,346.4795591)(159.90175494,346.49955908)(160.03175857,346.52956697)
\curveto(160.07175477,346.53955904)(160.13175471,346.53955904)(160.21175857,346.52956697)
\curveto(160.30175454,346.51955906)(160.35675448,346.52455906)(160.37675857,346.54456697)
\curveto(160.78675405,346.55455903)(161.17675366,346.53955904)(161.54675857,346.49956697)
\curveto(161.92675291,346.45955912)(162.26675257,346.3845592)(162.56675857,346.27456697)
\curveto(162.87675196,346.16455942)(163.1417517,346.01455957)(163.36175857,345.82456697)
\curveto(163.58175126,345.64455994)(163.75175109,345.40956017)(163.87175857,345.11956697)
\curveto(163.9417509,344.94956063)(163.98175086,344.75456083)(163.99175857,344.53456697)
\curveto(164.00175084,344.31456127)(164.00675083,344.08956149)(164.00675857,343.85956697)
\lineto(164.00675857,340.51456697)
\lineto(164.00675857,339.92956697)
\curveto(164.00675083,339.73956584)(164.02675081,339.56456602)(164.06675857,339.40456697)
\curveto(164.07675076,339.37456621)(164.08175076,339.33956624)(164.08175857,339.29956697)
\curveto(164.08175076,339.26956631)(164.08675075,339.23956634)(164.09675857,339.20956697)
\moveto(161.89175857,341.51956697)
\curveto(161.90175294,341.56956401)(161.90675293,341.62456396)(161.90675857,341.68456697)
\curveto(161.90675293,341.75456383)(161.90175294,341.81456377)(161.89175857,341.86456697)
\curveto(161.87175297,341.92456366)(161.86175298,341.9795636)(161.86175857,342.02956697)
\curveto(161.86175298,342.0795635)(161.841753,342.11956346)(161.80175857,342.14956697)
\curveto(161.75175309,342.18956339)(161.67675316,342.20956337)(161.57675857,342.20956697)
\curveto(161.5367533,342.19956338)(161.50175334,342.18956339)(161.47175857,342.17956697)
\curveto(161.4417534,342.1795634)(161.40675343,342.17456341)(161.36675857,342.16456697)
\curveto(161.29675354,342.14456344)(161.22175362,342.12956345)(161.14175857,342.11956697)
\curveto(161.06175378,342.10956347)(160.98175386,342.09456349)(160.90175857,342.07456697)
\curveto(160.87175397,342.06456352)(160.82675401,342.05956352)(160.76675857,342.05956697)
\curveto(160.6367542,342.02956355)(160.50675433,342.00956357)(160.37675857,341.99956697)
\curveto(160.24675459,341.98956359)(160.12175472,341.96456362)(160.00175857,341.92456697)
\curveto(159.92175492,341.90456368)(159.84675499,341.8845637)(159.77675857,341.86456697)
\curveto(159.70675513,341.85456373)(159.6367552,341.83456375)(159.56675857,341.80456697)
\curveto(159.35675548,341.71456387)(159.17675566,341.579564)(159.02675857,341.39956697)
\curveto(158.88675595,341.21956436)(158.836756,340.96956461)(158.87675857,340.64956697)
\curveto(158.89675594,340.4795651)(158.95175589,340.33956524)(159.04175857,340.22956697)
\curveto(159.11175573,340.11956546)(159.21675562,340.02956555)(159.35675857,339.95956697)
\curveto(159.49675534,339.89956568)(159.64675519,339.85456573)(159.80675857,339.82456697)
\curveto(159.97675486,339.79456579)(160.15175469,339.7845658)(160.33175857,339.79456697)
\curveto(160.52175432,339.81456577)(160.69675414,339.84956573)(160.85675857,339.89956697)
\curveto(161.11675372,339.9795656)(161.32175352,340.10456548)(161.47175857,340.27456697)
\curveto(161.62175322,340.45456513)(161.7367531,340.67456491)(161.81675857,340.93456697)
\curveto(161.836753,341.00456458)(161.84675299,341.07456451)(161.84675857,341.14456697)
\curveto(161.85675298,341.22456436)(161.87175297,341.30456428)(161.89175857,341.38456697)
\lineto(161.89175857,341.51956697)
}
}
{
\newrgbcolor{curcolor}{0 0 0}
\pscustom[linestyle=none,fillstyle=solid,fillcolor=curcolor]
{
\newpath
\moveto(169.23003982,346.55956697)
\curveto(170.04003466,346.579559)(170.71503399,346.45955912)(171.25503982,346.19956697)
\curveto(171.8050329,345.93955964)(172.24003246,345.56956001)(172.56003982,345.08956697)
\curveto(172.72003198,344.84956073)(172.84003186,344.57456101)(172.92003982,344.26456697)
\curveto(172.94003176,344.21456137)(172.95503175,344.14956143)(172.96503982,344.06956697)
\curveto(172.98503172,343.98956159)(172.98503172,343.91956166)(172.96503982,343.85956697)
\curveto(172.92503178,343.74956183)(172.85503185,343.6845619)(172.75503982,343.66456697)
\curveto(172.65503205,343.65456193)(172.53503217,343.64956193)(172.39503982,343.64956697)
\lineto(171.61503982,343.64956697)
\lineto(171.33003982,343.64956697)
\curveto(171.24003346,343.64956193)(171.16503354,343.66956191)(171.10503982,343.70956697)
\curveto(171.02503368,343.74956183)(170.97003373,343.80956177)(170.94003982,343.88956697)
\curveto(170.91003379,343.9795616)(170.87003383,344.06956151)(170.82003982,344.15956697)
\curveto(170.76003394,344.26956131)(170.69503401,344.36956121)(170.62503982,344.45956697)
\curveto(170.55503415,344.54956103)(170.47503423,344.62956095)(170.38503982,344.69956697)
\curveto(170.24503446,344.78956079)(170.09003461,344.85956072)(169.92003982,344.90956697)
\curveto(169.86003484,344.92956065)(169.8000349,344.93956064)(169.74003982,344.93956697)
\curveto(169.68003502,344.93956064)(169.62503508,344.94956063)(169.57503982,344.96956697)
\lineto(169.42503982,344.96956697)
\curveto(169.22503548,344.96956061)(169.06503564,344.94956063)(168.94503982,344.90956697)
\curveto(168.65503605,344.81956076)(168.42003628,344.6795609)(168.24003982,344.48956697)
\curveto(168.06003664,344.30956127)(167.91503679,344.08956149)(167.80503982,343.82956697)
\curveto(167.75503695,343.71956186)(167.71503699,343.59956198)(167.68503982,343.46956697)
\curveto(167.66503704,343.34956223)(167.64003706,343.21956236)(167.61003982,343.07956697)
\curveto(167.6000371,343.03956254)(167.59503711,342.99956258)(167.59503982,342.95956697)
\curveto(167.59503711,342.91956266)(167.59003711,342.8795627)(167.58003982,342.83956697)
\curveto(167.56003714,342.73956284)(167.55003715,342.59956298)(167.55003982,342.41956697)
\curveto(167.56003714,342.23956334)(167.57503713,342.09956348)(167.59503982,341.99956697)
\curveto(167.59503711,341.91956366)(167.6000371,341.86456372)(167.61003982,341.83456697)
\curveto(167.63003707,341.76456382)(167.64003706,341.69456389)(167.64003982,341.62456697)
\curveto(167.65003705,341.55456403)(167.66503704,341.4845641)(167.68503982,341.41456697)
\curveto(167.76503694,341.1845644)(167.86003684,340.97456461)(167.97003982,340.78456697)
\curveto(168.08003662,340.59456499)(168.22003648,340.43456515)(168.39003982,340.30456697)
\curveto(168.43003627,340.27456531)(168.49003621,340.23956534)(168.57003982,340.19956697)
\curveto(168.68003602,340.12956545)(168.79003591,340.0845655)(168.90003982,340.06456697)
\curveto(169.02003568,340.04456554)(169.16503554,340.02456556)(169.33503982,340.00456697)
\lineto(169.42503982,340.00456697)
\curveto(169.46503524,340.00456558)(169.49503521,340.00956557)(169.51503982,340.01956697)
\lineto(169.65003982,340.01956697)
\curveto(169.72003498,340.03956554)(169.78503492,340.05456553)(169.84503982,340.06456697)
\curveto(169.91503479,340.0845655)(169.98003472,340.10456548)(170.04003982,340.12456697)
\curveto(170.34003436,340.25456533)(170.57003413,340.44456514)(170.73003982,340.69456697)
\curveto(170.77003393,340.74456484)(170.8050339,340.79956478)(170.83503982,340.85956697)
\curveto(170.86503384,340.92956465)(170.89003381,340.98956459)(170.91003982,341.03956697)
\curveto(170.95003375,341.14956443)(170.98503372,341.24456434)(171.01503982,341.32456697)
\curveto(171.04503366,341.41456417)(171.11503359,341.4845641)(171.22503982,341.53456697)
\curveto(171.31503339,341.57456401)(171.46003324,341.58956399)(171.66003982,341.57956697)
\lineto(172.15503982,341.57956697)
\lineto(172.36503982,341.57956697)
\curveto(172.44503226,341.58956399)(172.51003219,341.584564)(172.56003982,341.56456697)
\lineto(172.68003982,341.56456697)
\lineto(172.80003982,341.53456697)
\curveto(172.84003186,341.53456405)(172.87003183,341.52456406)(172.89003982,341.50456697)
\curveto(172.94003176,341.46456412)(172.97003173,341.40456418)(172.98003982,341.32456697)
\curveto(173.0000317,341.25456433)(173.0000317,341.1795644)(172.98003982,341.09956697)
\curveto(172.89003181,340.76956481)(172.78003192,340.47456511)(172.65003982,340.21456697)
\curveto(172.24003246,339.44456614)(171.58503312,338.90956667)(170.68503982,338.60956697)
\curveto(170.58503412,338.579567)(170.48003422,338.55956702)(170.37003982,338.54956697)
\curveto(170.26003444,338.52956705)(170.15003455,338.50456708)(170.04003982,338.47456697)
\curveto(169.98003472,338.46456712)(169.92003478,338.45956712)(169.86003982,338.45956697)
\curveto(169.8000349,338.45956712)(169.74003496,338.45456713)(169.68003982,338.44456697)
\lineto(169.51503982,338.44456697)
\curveto(169.46503524,338.42456716)(169.39003531,338.41956716)(169.29003982,338.42956697)
\curveto(169.19003551,338.42956715)(169.11503559,338.43456715)(169.06503982,338.44456697)
\curveto(168.98503572,338.46456712)(168.91003579,338.47456711)(168.84003982,338.47456697)
\curveto(168.78003592,338.46456712)(168.71503599,338.46956711)(168.64503982,338.48956697)
\lineto(168.49503982,338.51956697)
\curveto(168.44503626,338.51956706)(168.39503631,338.52456706)(168.34503982,338.53456697)
\curveto(168.23503647,338.56456702)(168.13003657,338.59456699)(168.03003982,338.62456697)
\curveto(167.93003677,338.65456693)(167.83503687,338.68956689)(167.74503982,338.72956697)
\curveto(167.27503743,338.92956665)(166.88003782,339.1845664)(166.56003982,339.49456697)
\curveto(166.24003846,339.81456577)(165.98003872,340.20956537)(165.78003982,340.67956697)
\curveto(165.73003897,340.76956481)(165.69003901,340.86456472)(165.66003982,340.96456697)
\lineto(165.57003982,341.29456697)
\curveto(165.56003914,341.33456425)(165.55503915,341.36956421)(165.55503982,341.39956697)
\curveto(165.55503915,341.43956414)(165.54503916,341.4845641)(165.52503982,341.53456697)
\curveto(165.5050392,341.60456398)(165.49503921,341.67456391)(165.49503982,341.74456697)
\curveto(165.49503921,341.82456376)(165.48503922,341.89956368)(165.46503982,341.96956697)
\lineto(165.46503982,342.22456697)
\curveto(165.44503926,342.27456331)(165.43503927,342.32956325)(165.43503982,342.38956697)
\curveto(165.43503927,342.45956312)(165.44503926,342.51956306)(165.46503982,342.56956697)
\curveto(165.47503923,342.61956296)(165.47503923,342.66456292)(165.46503982,342.70456697)
\curveto(165.45503925,342.74456284)(165.45503925,342.7845628)(165.46503982,342.82456697)
\curveto(165.48503922,342.89456269)(165.49003921,342.95956262)(165.48003982,343.01956697)
\curveto(165.48003922,343.0795625)(165.49003921,343.13956244)(165.51003982,343.19956697)
\curveto(165.56003914,343.3795622)(165.6000391,343.54956203)(165.63003982,343.70956697)
\curveto(165.66003904,343.8795617)(165.705039,344.04456154)(165.76503982,344.20456697)
\curveto(165.98503872,344.71456087)(166.26003844,345.13956044)(166.59003982,345.47956697)
\curveto(166.93003777,345.81955976)(167.36003734,346.09455949)(167.88003982,346.30456697)
\curveto(168.02003668,346.36455922)(168.16503654,346.40455918)(168.31503982,346.42456697)
\curveto(168.46503624,346.45455913)(168.62003608,346.48955909)(168.78003982,346.52956697)
\curveto(168.86003584,346.53955904)(168.93503577,346.54455904)(169.00503982,346.54456697)
\curveto(169.07503563,346.54455904)(169.15003555,346.54955903)(169.23003982,346.55956697)
}
}
{
\newrgbcolor{curcolor}{0 0 0}
\pscustom[linestyle=none,fillstyle=solid,fillcolor=curcolor]
{
\newpath
\moveto(176.37332107,349.19956697)
\curveto(176.44331812,349.11955646)(176.47831809,348.99955658)(176.47832107,348.83956697)
\lineto(176.47832107,348.37456697)
\lineto(176.47832107,347.96956697)
\curveto(176.47831809,347.82955775)(176.44331812,347.73455785)(176.37332107,347.68456697)
\curveto(176.31331825,347.63455795)(176.23331833,347.60455798)(176.13332107,347.59456697)
\curveto(176.04331852,347.584558)(175.94331862,347.579558)(175.83332107,347.57956697)
\lineto(174.99332107,347.57956697)
\curveto(174.88331968,347.579558)(174.78331978,347.584558)(174.69332107,347.59456697)
\curveto(174.61331995,347.60455798)(174.54332002,347.63455795)(174.48332107,347.68456697)
\curveto(174.44332012,347.71455787)(174.41332015,347.76955781)(174.39332107,347.84956697)
\curveto(174.38332018,347.93955764)(174.37332019,348.03455755)(174.36332107,348.13456697)
\lineto(174.36332107,348.46456697)
\curveto(174.37332019,348.57455701)(174.37832019,348.66955691)(174.37832107,348.74956697)
\lineto(174.37832107,348.95956697)
\curveto(174.38832018,349.02955655)(174.40832016,349.08955649)(174.43832107,349.13956697)
\curveto(174.45832011,349.1795564)(174.48332008,349.20955637)(174.51332107,349.22956697)
\lineto(174.63332107,349.28956697)
\curveto(174.65331991,349.28955629)(174.67831989,349.28955629)(174.70832107,349.28956697)
\curveto(174.73831983,349.29955628)(174.7633198,349.30455628)(174.78332107,349.30456697)
\lineto(175.87832107,349.30456697)
\curveto(175.97831859,349.30455628)(176.07331849,349.29955628)(176.16332107,349.28956697)
\curveto(176.25331831,349.2795563)(176.32331824,349.24955633)(176.37332107,349.19956697)
\moveto(176.47832107,339.43456697)
\curveto(176.47831809,339.23456635)(176.47331809,339.06456652)(176.46332107,338.92456697)
\curveto(176.45331811,338.7845668)(176.3633182,338.68956689)(176.19332107,338.63956697)
\curveto(176.13331843,338.61956696)(176.0683185,338.60956697)(175.99832107,338.60956697)
\curveto(175.92831864,338.61956696)(175.85331871,338.62456696)(175.77332107,338.62456697)
\lineto(174.93332107,338.62456697)
\curveto(174.84331972,338.62456696)(174.75331981,338.62956695)(174.66332107,338.63956697)
\curveto(174.58331998,338.64956693)(174.52332004,338.6795669)(174.48332107,338.72956697)
\curveto(174.42332014,338.79956678)(174.38832018,338.8845667)(174.37832107,338.98456697)
\lineto(174.37832107,339.32956697)
\lineto(174.37832107,345.65956697)
\lineto(174.37832107,345.95956697)
\curveto(174.37832019,346.05955952)(174.39832017,346.13955944)(174.43832107,346.19956697)
\curveto(174.49832007,346.26955931)(174.58331998,346.31455927)(174.69332107,346.33456697)
\curveto(174.71331985,346.34455924)(174.73831983,346.34455924)(174.76832107,346.33456697)
\curveto(174.80831976,346.33455925)(174.83831973,346.33955924)(174.85832107,346.34956697)
\lineto(175.60832107,346.34956697)
\lineto(175.80332107,346.34956697)
\curveto(175.88331868,346.35955922)(175.94831862,346.35955922)(175.99832107,346.34956697)
\lineto(176.11832107,346.34956697)
\curveto(176.17831839,346.32955925)(176.23331833,346.31455927)(176.28332107,346.30456697)
\curveto(176.33331823,346.29455929)(176.37331819,346.26455932)(176.40332107,346.21456697)
\curveto(176.44331812,346.16455942)(176.4633181,346.09455949)(176.46332107,346.00456697)
\curveto(176.47331809,345.91455967)(176.47831809,345.81955976)(176.47832107,345.71956697)
\lineto(176.47832107,339.43456697)
}
}
{
\newrgbcolor{curcolor}{0 0 0}
\pscustom[linestyle=none,fillstyle=solid,fillcolor=curcolor]
{
\newpath
\moveto(185.91050857,342.79456697)
\curveto(185.9305,342.73456285)(185.94049999,342.64956293)(185.94050857,342.53956697)
\curveto(185.94049999,342.42956315)(185.9305,342.34456324)(185.91050857,342.28456697)
\lineto(185.91050857,342.13456697)
\curveto(185.89050004,342.05456353)(185.88050005,341.97456361)(185.88050857,341.89456697)
\curveto(185.89050004,341.81456377)(185.88550005,341.73456385)(185.86550857,341.65456697)
\curveto(185.84550009,341.584564)(185.8305001,341.51956406)(185.82050857,341.45956697)
\curveto(185.81050012,341.39956418)(185.80050013,341.33456425)(185.79050857,341.26456697)
\curveto(185.75050018,341.15456443)(185.71550022,341.03956454)(185.68550857,340.91956697)
\curveto(185.65550028,340.80956477)(185.61550032,340.70456488)(185.56550857,340.60456697)
\curveto(185.35550058,340.12456546)(185.08050085,339.73456585)(184.74050857,339.43456697)
\curveto(184.40050153,339.13456645)(183.99050194,338.8845667)(183.51050857,338.68456697)
\curveto(183.39050254,338.63456695)(183.26550267,338.59956698)(183.13550857,338.57956697)
\curveto(183.01550292,338.54956703)(182.89050304,338.51956706)(182.76050857,338.48956697)
\curveto(182.71050322,338.46956711)(182.65550328,338.45956712)(182.59550857,338.45956697)
\curveto(182.5355034,338.45956712)(182.48050345,338.45456713)(182.43050857,338.44456697)
\lineto(182.32550857,338.44456697)
\curveto(182.29550364,338.43456715)(182.26550367,338.42956715)(182.23550857,338.42956697)
\curveto(182.18550375,338.41956716)(182.10550383,338.41456717)(181.99550857,338.41456697)
\curveto(181.88550405,338.40456718)(181.80050413,338.40956717)(181.74050857,338.42956697)
\lineto(181.59050857,338.42956697)
\curveto(181.54050439,338.43956714)(181.48550445,338.44456714)(181.42550857,338.44456697)
\curveto(181.37550456,338.43456715)(181.32550461,338.43956714)(181.27550857,338.45956697)
\curveto(181.2355047,338.46956711)(181.19550474,338.47456711)(181.15550857,338.47456697)
\curveto(181.12550481,338.47456711)(181.08550485,338.4795671)(181.03550857,338.48956697)
\curveto(180.935505,338.51956706)(180.8355051,338.54456704)(180.73550857,338.56456697)
\curveto(180.6355053,338.584567)(180.54050539,338.61456697)(180.45050857,338.65456697)
\curveto(180.3305056,338.69456689)(180.21550572,338.73456685)(180.10550857,338.77456697)
\curveto(180.00550593,338.81456677)(179.90050603,338.86456672)(179.79050857,338.92456697)
\curveto(179.44050649,339.13456645)(179.14050679,339.3795662)(178.89050857,339.65956697)
\curveto(178.64050729,339.93956564)(178.4305075,340.27456531)(178.26050857,340.66456697)
\curveto(178.21050772,340.75456483)(178.17050776,340.84956473)(178.14050857,340.94956697)
\curveto(178.12050781,341.04956453)(178.09550784,341.15456443)(178.06550857,341.26456697)
\curveto(178.04550789,341.31456427)(178.0355079,341.35956422)(178.03550857,341.39956697)
\curveto(178.0355079,341.43956414)(178.02550791,341.4845641)(178.00550857,341.53456697)
\curveto(177.98550795,341.61456397)(177.97550796,341.69456389)(177.97550857,341.77456697)
\curveto(177.97550796,341.86456372)(177.96550797,341.94956363)(177.94550857,342.02956697)
\curveto(177.935508,342.0795635)(177.930508,342.12456346)(177.93050857,342.16456697)
\lineto(177.93050857,342.29956697)
\curveto(177.91050802,342.35956322)(177.90050803,342.44456314)(177.90050857,342.55456697)
\curveto(177.91050802,342.66456292)(177.92550801,342.74956283)(177.94550857,342.80956697)
\lineto(177.94550857,342.91456697)
\curveto(177.95550798,342.96456262)(177.95550798,343.01456257)(177.94550857,343.06456697)
\curveto(177.94550799,343.12456246)(177.95550798,343.1795624)(177.97550857,343.22956697)
\curveto(177.98550795,343.2795623)(177.99050794,343.32456226)(177.99050857,343.36456697)
\curveto(177.99050794,343.41456217)(178.00050793,343.46456212)(178.02050857,343.51456697)
\curveto(178.06050787,343.64456194)(178.09550784,343.76956181)(178.12550857,343.88956697)
\curveto(178.15550778,344.01956156)(178.19550774,344.14456144)(178.24550857,344.26456697)
\curveto(178.42550751,344.67456091)(178.64050729,345.01456057)(178.89050857,345.28456697)
\curveto(179.14050679,345.56456002)(179.44550649,345.81955976)(179.80550857,346.04956697)
\curveto(179.90550603,346.09955948)(180.01050592,346.14455944)(180.12050857,346.18456697)
\curveto(180.2305057,346.22455936)(180.34050559,346.26955931)(180.45050857,346.31956697)
\curveto(180.58050535,346.36955921)(180.71550522,346.40455918)(180.85550857,346.42456697)
\curveto(180.99550494,346.44455914)(181.14050479,346.47455911)(181.29050857,346.51456697)
\curveto(181.37050456,346.52455906)(181.44550449,346.52955905)(181.51550857,346.52956697)
\curveto(181.58550435,346.52955905)(181.65550428,346.53455905)(181.72550857,346.54456697)
\curveto(182.30550363,346.55455903)(182.80550313,346.49455909)(183.22550857,346.36456697)
\curveto(183.65550228,346.23455935)(184.0355019,346.05455953)(184.36550857,345.82456697)
\curveto(184.47550146,345.74455984)(184.58550135,345.65455993)(184.69550857,345.55456697)
\curveto(184.81550112,345.46456012)(184.91550102,345.36456022)(184.99550857,345.25456697)
\curveto(185.07550086,345.15456043)(185.14550079,345.05456053)(185.20550857,344.95456697)
\curveto(185.27550066,344.85456073)(185.34550059,344.74956083)(185.41550857,344.63956697)
\curveto(185.48550045,344.52956105)(185.54050039,344.40956117)(185.58050857,344.27956697)
\curveto(185.62050031,344.15956142)(185.66550027,344.02956155)(185.71550857,343.88956697)
\curveto(185.74550019,343.80956177)(185.77050016,343.72456186)(185.79050857,343.63456697)
\lineto(185.85050857,343.36456697)
\curveto(185.86050007,343.32456226)(185.86550007,343.2845623)(185.86550857,343.24456697)
\curveto(185.86550007,343.20456238)(185.87050006,343.16456242)(185.88050857,343.12456697)
\curveto(185.90050003,343.07456251)(185.90550003,343.01956256)(185.89550857,342.95956697)
\curveto(185.88550005,342.89956268)(185.89050004,342.84456274)(185.91050857,342.79456697)
\moveto(183.81050857,342.25456697)
\curveto(183.82050211,342.30456328)(183.82550211,342.37456321)(183.82550857,342.46456697)
\curveto(183.82550211,342.56456302)(183.82050211,342.63956294)(183.81050857,342.68956697)
\lineto(183.81050857,342.80956697)
\curveto(183.79050214,342.85956272)(183.78050215,342.91456267)(183.78050857,342.97456697)
\curveto(183.78050215,343.03456255)(183.77550216,343.08956249)(183.76550857,343.13956697)
\curveto(183.76550217,343.1795624)(183.76050217,343.20956237)(183.75050857,343.22956697)
\lineto(183.69050857,343.46956697)
\curveto(183.68050225,343.55956202)(183.66050227,343.64456194)(183.63050857,343.72456697)
\curveto(183.52050241,343.9845616)(183.39050254,344.20456138)(183.24050857,344.38456697)
\curveto(183.09050284,344.57456101)(182.89050304,344.72456086)(182.64050857,344.83456697)
\curveto(182.58050335,344.85456073)(182.52050341,344.86956071)(182.46050857,344.87956697)
\curveto(182.40050353,344.89956068)(182.3355036,344.91956066)(182.26550857,344.93956697)
\curveto(182.18550375,344.95956062)(182.10050383,344.96456062)(182.01050857,344.95456697)
\lineto(181.74050857,344.95456697)
\curveto(181.71050422,344.93456065)(181.67550426,344.92456066)(181.63550857,344.92456697)
\curveto(181.59550434,344.93456065)(181.56050437,344.93456065)(181.53050857,344.92456697)
\lineto(181.32050857,344.86456697)
\curveto(181.26050467,344.85456073)(181.20550473,344.83456075)(181.15550857,344.80456697)
\curveto(180.90550503,344.69456089)(180.70050523,344.53456105)(180.54050857,344.32456697)
\curveto(180.39050554,344.12456146)(180.27050566,343.88956169)(180.18050857,343.61956697)
\curveto(180.15050578,343.51956206)(180.12550581,343.41456217)(180.10550857,343.30456697)
\curveto(180.09550584,343.19456239)(180.08050585,343.0845625)(180.06050857,342.97456697)
\curveto(180.05050588,342.92456266)(180.04550589,342.87456271)(180.04550857,342.82456697)
\lineto(180.04550857,342.67456697)
\curveto(180.02550591,342.60456298)(180.01550592,342.49956308)(180.01550857,342.35956697)
\curveto(180.02550591,342.21956336)(180.04050589,342.11456347)(180.06050857,342.04456697)
\lineto(180.06050857,341.90956697)
\curveto(180.08050585,341.82956375)(180.09550584,341.74956383)(180.10550857,341.66956697)
\curveto(180.11550582,341.59956398)(180.1305058,341.52456406)(180.15050857,341.44456697)
\curveto(180.25050568,341.14456444)(180.35550558,340.89956468)(180.46550857,340.70956697)
\curveto(180.58550535,340.52956505)(180.77050516,340.36456522)(181.02050857,340.21456697)
\curveto(181.09050484,340.16456542)(181.16550477,340.12456546)(181.24550857,340.09456697)
\curveto(181.3355046,340.06456552)(181.42550451,340.03956554)(181.51550857,340.01956697)
\curveto(181.55550438,340.00956557)(181.59050434,340.00456558)(181.62050857,340.00456697)
\curveto(181.65050428,340.01456557)(181.68550425,340.01456557)(181.72550857,340.00456697)
\lineto(181.84550857,339.97456697)
\curveto(181.89550404,339.97456561)(181.94050399,339.9795656)(181.98050857,339.98956697)
\lineto(182.10050857,339.98956697)
\curveto(182.18050375,340.00956557)(182.26050367,340.02456556)(182.34050857,340.03456697)
\curveto(182.42050351,340.04456554)(182.49550344,340.06456552)(182.56550857,340.09456697)
\curveto(182.82550311,340.19456539)(183.0355029,340.32956525)(183.19550857,340.49956697)
\curveto(183.35550258,340.66956491)(183.49050244,340.8795647)(183.60050857,341.12956697)
\curveto(183.64050229,341.22956435)(183.67050226,341.32956425)(183.69050857,341.42956697)
\curveto(183.71050222,341.52956405)(183.7355022,341.63456395)(183.76550857,341.74456697)
\curveto(183.77550216,341.7845638)(183.78050215,341.81956376)(183.78050857,341.84956697)
\curveto(183.78050215,341.88956369)(183.78550215,341.92956365)(183.79550857,341.96956697)
\lineto(183.79550857,342.10456697)
\curveto(183.79550214,342.15456343)(183.80050213,342.20456338)(183.81050857,342.25456697)
}
}
{
\newrgbcolor{curcolor}{0 0 0}
\pscustom[linestyle=none,fillstyle=solid,fillcolor=curcolor]
{
\newpath
\moveto(190.28043045,346.55956697)
\curveto(191.03042595,346.579559)(191.6804253,346.49455909)(192.23043045,346.30456697)
\curveto(192.79042419,346.12455946)(193.21542376,345.80955977)(193.50543045,345.35956697)
\curveto(193.5754234,345.24956033)(193.63542334,345.13456045)(193.68543045,345.01456697)
\curveto(193.74542323,344.90456068)(193.79542318,344.7795608)(193.83543045,344.63956697)
\curveto(193.85542312,344.579561)(193.86542311,344.51456107)(193.86543045,344.44456697)
\curveto(193.86542311,344.37456121)(193.85542312,344.31456127)(193.83543045,344.26456697)
\curveto(193.79542318,344.20456138)(193.74042324,344.16456142)(193.67043045,344.14456697)
\curveto(193.62042336,344.12456146)(193.56042342,344.11456147)(193.49043045,344.11456697)
\lineto(193.28043045,344.11456697)
\lineto(192.62043045,344.11456697)
\curveto(192.55042443,344.11456147)(192.4804245,344.10956147)(192.41043045,344.09956697)
\curveto(192.34042464,344.09956148)(192.2754247,344.10956147)(192.21543045,344.12956697)
\curveto(192.11542486,344.14956143)(192.04042494,344.18956139)(191.99043045,344.24956697)
\curveto(191.94042504,344.30956127)(191.89542508,344.36956121)(191.85543045,344.42956697)
\lineto(191.73543045,344.63956697)
\curveto(191.70542527,344.71956086)(191.65542532,344.7845608)(191.58543045,344.83456697)
\curveto(191.48542549,344.91456067)(191.38542559,344.97456061)(191.28543045,345.01456697)
\curveto(191.19542578,345.05456053)(191.0804259,345.08956049)(190.94043045,345.11956697)
\curveto(190.87042611,345.13956044)(190.76542621,345.15456043)(190.62543045,345.16456697)
\curveto(190.49542648,345.17456041)(190.39542658,345.16956041)(190.32543045,345.14956697)
\lineto(190.22043045,345.14956697)
\lineto(190.07043045,345.11956697)
\curveto(190.03042695,345.11956046)(189.98542699,345.11456047)(189.93543045,345.10456697)
\curveto(189.76542721,345.05456053)(189.62542735,344.9845606)(189.51543045,344.89456697)
\curveto(189.41542756,344.81456077)(189.34542763,344.68956089)(189.30543045,344.51956697)
\curveto(189.28542769,344.44956113)(189.28542769,344.3845612)(189.30543045,344.32456697)
\curveto(189.32542765,344.26456132)(189.34542763,344.21456137)(189.36543045,344.17456697)
\curveto(189.43542754,344.05456153)(189.51542746,343.95956162)(189.60543045,343.88956697)
\curveto(189.70542727,343.81956176)(189.82042716,343.75956182)(189.95043045,343.70956697)
\curveto(190.14042684,343.62956195)(190.34542663,343.55956202)(190.56543045,343.49956697)
\lineto(191.25543045,343.34956697)
\curveto(191.49542548,343.30956227)(191.72542525,343.25956232)(191.94543045,343.19956697)
\curveto(192.1754248,343.14956243)(192.39042459,343.0845625)(192.59043045,343.00456697)
\curveto(192.6804243,342.96456262)(192.76542421,342.92956265)(192.84543045,342.89956697)
\curveto(192.93542404,342.8795627)(193.02042396,342.84456274)(193.10043045,342.79456697)
\curveto(193.29042369,342.67456291)(193.46042352,342.54456304)(193.61043045,342.40456697)
\curveto(193.77042321,342.26456332)(193.89542308,342.08956349)(193.98543045,341.87956697)
\curveto(194.01542296,341.80956377)(194.04042294,341.73956384)(194.06043045,341.66956697)
\curveto(194.0804229,341.59956398)(194.10042288,341.52456406)(194.12043045,341.44456697)
\curveto(194.13042285,341.3845642)(194.13542284,341.28956429)(194.13543045,341.15956697)
\curveto(194.14542283,341.03956454)(194.14542283,340.94456464)(194.13543045,340.87456697)
\lineto(194.13543045,340.79956697)
\curveto(194.11542286,340.73956484)(194.10042288,340.6795649)(194.09043045,340.61956697)
\curveto(194.09042289,340.56956501)(194.08542289,340.51956506)(194.07543045,340.46956697)
\curveto(194.00542297,340.16956541)(193.89542308,339.90456568)(193.74543045,339.67456697)
\curveto(193.58542339,339.43456615)(193.39042359,339.23956634)(193.16043045,339.08956697)
\curveto(192.93042405,338.93956664)(192.67042431,338.80956677)(192.38043045,338.69956697)
\curveto(192.27042471,338.64956693)(192.15042483,338.61456697)(192.02043045,338.59456697)
\curveto(191.90042508,338.57456701)(191.7804252,338.54956703)(191.66043045,338.51956697)
\curveto(191.57042541,338.49956708)(191.4754255,338.48956709)(191.37543045,338.48956697)
\curveto(191.28542569,338.4795671)(191.19542578,338.46456712)(191.10543045,338.44456697)
\lineto(190.83543045,338.44456697)
\curveto(190.7754262,338.42456716)(190.67042631,338.41456717)(190.52043045,338.41456697)
\curveto(190.3804266,338.41456717)(190.2804267,338.42456716)(190.22043045,338.44456697)
\curveto(190.19042679,338.44456714)(190.15542682,338.44956713)(190.11543045,338.45956697)
\lineto(190.01043045,338.45956697)
\curveto(189.89042709,338.4795671)(189.77042721,338.49456709)(189.65043045,338.50456697)
\curveto(189.53042745,338.51456707)(189.41542756,338.53456705)(189.30543045,338.56456697)
\curveto(188.91542806,338.67456691)(188.57042841,338.79956678)(188.27043045,338.93956697)
\curveto(187.97042901,339.08956649)(187.71542926,339.30956627)(187.50543045,339.59956697)
\curveto(187.36542961,339.78956579)(187.24542973,340.00956557)(187.14543045,340.25956697)
\curveto(187.12542985,340.31956526)(187.10542987,340.39956518)(187.08543045,340.49956697)
\curveto(187.06542991,340.54956503)(187.05042993,340.61956496)(187.04043045,340.70956697)
\curveto(187.03042995,340.79956478)(187.03542994,340.87456471)(187.05543045,340.93456697)
\curveto(187.08542989,341.00456458)(187.13542984,341.05456453)(187.20543045,341.08456697)
\curveto(187.25542972,341.10456448)(187.31542966,341.11456447)(187.38543045,341.11456697)
\lineto(187.61043045,341.11456697)
\lineto(188.31543045,341.11456697)
\lineto(188.55543045,341.11456697)
\curveto(188.63542834,341.11456447)(188.70542827,341.10456448)(188.76543045,341.08456697)
\curveto(188.8754281,341.04456454)(188.94542803,340.9795646)(188.97543045,340.88956697)
\curveto(189.01542796,340.79956478)(189.06042792,340.70456488)(189.11043045,340.60456697)
\curveto(189.13042785,340.55456503)(189.16542781,340.48956509)(189.21543045,340.40956697)
\curveto(189.2754277,340.32956525)(189.32542765,340.2795653)(189.36543045,340.25956697)
\curveto(189.48542749,340.15956542)(189.60042738,340.0795655)(189.71043045,340.01956697)
\curveto(189.82042716,339.96956561)(189.96042702,339.91956566)(190.13043045,339.86956697)
\curveto(190.1804268,339.84956573)(190.23042675,339.83956574)(190.28043045,339.83956697)
\curveto(190.33042665,339.84956573)(190.3804266,339.84956573)(190.43043045,339.83956697)
\curveto(190.51042647,339.81956576)(190.59542638,339.80956577)(190.68543045,339.80956697)
\curveto(190.78542619,339.81956576)(190.87042611,339.83456575)(190.94043045,339.85456697)
\curveto(190.99042599,339.86456572)(191.03542594,339.86956571)(191.07543045,339.86956697)
\curveto(191.12542585,339.86956571)(191.1754258,339.8795657)(191.22543045,339.89956697)
\curveto(191.36542561,339.94956563)(191.49042549,340.00956557)(191.60043045,340.07956697)
\curveto(191.72042526,340.14956543)(191.81542516,340.23956534)(191.88543045,340.34956697)
\curveto(191.93542504,340.42956515)(191.975425,340.55456503)(192.00543045,340.72456697)
\curveto(192.02542495,340.79456479)(192.02542495,340.85956472)(192.00543045,340.91956697)
\curveto(191.98542499,340.9795646)(191.96542501,341.02956455)(191.94543045,341.06956697)
\curveto(191.8754251,341.20956437)(191.78542519,341.31456427)(191.67543045,341.38456697)
\curveto(191.5754254,341.45456413)(191.45542552,341.51956406)(191.31543045,341.57956697)
\curveto(191.12542585,341.65956392)(190.92542605,341.72456386)(190.71543045,341.77456697)
\curveto(190.50542647,341.82456376)(190.29542668,341.8795637)(190.08543045,341.93956697)
\curveto(190.00542697,341.95956362)(189.92042706,341.97456361)(189.83043045,341.98456697)
\curveto(189.75042723,341.99456359)(189.67042731,342.00956357)(189.59043045,342.02956697)
\curveto(189.27042771,342.11956346)(188.96542801,342.20456338)(188.67543045,342.28456697)
\curveto(188.38542859,342.37456321)(188.12042886,342.50456308)(187.88043045,342.67456697)
\curveto(187.60042938,342.87456271)(187.39542958,343.14456244)(187.26543045,343.48456697)
\curveto(187.24542973,343.55456203)(187.22542975,343.64956193)(187.20543045,343.76956697)
\curveto(187.18542979,343.83956174)(187.17042981,343.92456166)(187.16043045,344.02456697)
\curveto(187.15042983,344.12456146)(187.15542982,344.21456137)(187.17543045,344.29456697)
\curveto(187.19542978,344.34456124)(187.20042978,344.3845612)(187.19043045,344.41456697)
\curveto(187.1804298,344.45456113)(187.18542979,344.49956108)(187.20543045,344.54956697)
\curveto(187.22542975,344.65956092)(187.24542973,344.75956082)(187.26543045,344.84956697)
\curveto(187.29542968,344.94956063)(187.33042965,345.04456054)(187.37043045,345.13456697)
\curveto(187.50042948,345.42456016)(187.6804293,345.65955992)(187.91043045,345.83956697)
\curveto(188.14042884,346.01955956)(188.40042858,346.16455942)(188.69043045,346.27456697)
\curveto(188.80042818,346.32455926)(188.91542806,346.35955922)(189.03543045,346.37956697)
\curveto(189.15542782,346.40955917)(189.2804277,346.43955914)(189.41043045,346.46956697)
\curveto(189.47042751,346.48955909)(189.53042745,346.49955908)(189.59043045,346.49956697)
\lineto(189.77043045,346.52956697)
\curveto(189.85042713,346.53955904)(189.93542704,346.54455904)(190.02543045,346.54456697)
\curveto(190.11542686,346.54455904)(190.20042678,346.54955903)(190.28043045,346.55956697)
}
}
{
\newrgbcolor{curcolor}{0 0 0}
\pscustom[linestyle=none,fillstyle=solid,fillcolor=curcolor]
{
}
}
{
\newrgbcolor{curcolor}{0 0 0}
\pscustom[linestyle=none,fillstyle=solid,fillcolor=curcolor]
{
\newpath
\moveto(202.49722732,346.55956697)
\curveto(203.24722282,346.579559)(203.89722217,346.49455909)(204.44722732,346.30456697)
\curveto(205.00722106,346.12455946)(205.43222064,345.80955977)(205.72222732,345.35956697)
\curveto(205.79222028,345.24956033)(205.85222022,345.13456045)(205.90222732,345.01456697)
\curveto(205.96222011,344.90456068)(206.01222006,344.7795608)(206.05222732,344.63956697)
\curveto(206.07222,344.579561)(206.08221999,344.51456107)(206.08222732,344.44456697)
\curveto(206.08221999,344.37456121)(206.07222,344.31456127)(206.05222732,344.26456697)
\curveto(206.01222006,344.20456138)(205.95722011,344.16456142)(205.88722732,344.14456697)
\curveto(205.83722023,344.12456146)(205.77722029,344.11456147)(205.70722732,344.11456697)
\lineto(205.49722732,344.11456697)
\lineto(204.83722732,344.11456697)
\curveto(204.7672213,344.11456147)(204.69722137,344.10956147)(204.62722732,344.09956697)
\curveto(204.55722151,344.09956148)(204.49222158,344.10956147)(204.43222732,344.12956697)
\curveto(204.33222174,344.14956143)(204.25722181,344.18956139)(204.20722732,344.24956697)
\curveto(204.15722191,344.30956127)(204.11222196,344.36956121)(204.07222732,344.42956697)
\lineto(203.95222732,344.63956697)
\curveto(203.92222215,344.71956086)(203.8722222,344.7845608)(203.80222732,344.83456697)
\curveto(203.70222237,344.91456067)(203.60222247,344.97456061)(203.50222732,345.01456697)
\curveto(203.41222266,345.05456053)(203.29722277,345.08956049)(203.15722732,345.11956697)
\curveto(203.08722298,345.13956044)(202.98222309,345.15456043)(202.84222732,345.16456697)
\curveto(202.71222336,345.17456041)(202.61222346,345.16956041)(202.54222732,345.14956697)
\lineto(202.43722732,345.14956697)
\lineto(202.28722732,345.11956697)
\curveto(202.24722382,345.11956046)(202.20222387,345.11456047)(202.15222732,345.10456697)
\curveto(201.98222409,345.05456053)(201.84222423,344.9845606)(201.73222732,344.89456697)
\curveto(201.63222444,344.81456077)(201.56222451,344.68956089)(201.52222732,344.51956697)
\curveto(201.50222457,344.44956113)(201.50222457,344.3845612)(201.52222732,344.32456697)
\curveto(201.54222453,344.26456132)(201.56222451,344.21456137)(201.58222732,344.17456697)
\curveto(201.65222442,344.05456153)(201.73222434,343.95956162)(201.82222732,343.88956697)
\curveto(201.92222415,343.81956176)(202.03722403,343.75956182)(202.16722732,343.70956697)
\curveto(202.35722371,343.62956195)(202.56222351,343.55956202)(202.78222732,343.49956697)
\lineto(203.47222732,343.34956697)
\curveto(203.71222236,343.30956227)(203.94222213,343.25956232)(204.16222732,343.19956697)
\curveto(204.39222168,343.14956243)(204.60722146,343.0845625)(204.80722732,343.00456697)
\curveto(204.89722117,342.96456262)(204.98222109,342.92956265)(205.06222732,342.89956697)
\curveto(205.15222092,342.8795627)(205.23722083,342.84456274)(205.31722732,342.79456697)
\curveto(205.50722056,342.67456291)(205.67722039,342.54456304)(205.82722732,342.40456697)
\curveto(205.98722008,342.26456332)(206.11221996,342.08956349)(206.20222732,341.87956697)
\curveto(206.23221984,341.80956377)(206.25721981,341.73956384)(206.27722732,341.66956697)
\curveto(206.29721977,341.59956398)(206.31721975,341.52456406)(206.33722732,341.44456697)
\curveto(206.34721972,341.3845642)(206.35221972,341.28956429)(206.35222732,341.15956697)
\curveto(206.36221971,341.03956454)(206.36221971,340.94456464)(206.35222732,340.87456697)
\lineto(206.35222732,340.79956697)
\curveto(206.33221974,340.73956484)(206.31721975,340.6795649)(206.30722732,340.61956697)
\curveto(206.30721976,340.56956501)(206.30221977,340.51956506)(206.29222732,340.46956697)
\curveto(206.22221985,340.16956541)(206.11221996,339.90456568)(205.96222732,339.67456697)
\curveto(205.80222027,339.43456615)(205.60722046,339.23956634)(205.37722732,339.08956697)
\curveto(205.14722092,338.93956664)(204.88722118,338.80956677)(204.59722732,338.69956697)
\curveto(204.48722158,338.64956693)(204.3672217,338.61456697)(204.23722732,338.59456697)
\curveto(204.11722195,338.57456701)(203.99722207,338.54956703)(203.87722732,338.51956697)
\curveto(203.78722228,338.49956708)(203.69222238,338.48956709)(203.59222732,338.48956697)
\curveto(203.50222257,338.4795671)(203.41222266,338.46456712)(203.32222732,338.44456697)
\lineto(203.05222732,338.44456697)
\curveto(202.99222308,338.42456716)(202.88722318,338.41456717)(202.73722732,338.41456697)
\curveto(202.59722347,338.41456717)(202.49722357,338.42456716)(202.43722732,338.44456697)
\curveto(202.40722366,338.44456714)(202.3722237,338.44956713)(202.33222732,338.45956697)
\lineto(202.22722732,338.45956697)
\curveto(202.10722396,338.4795671)(201.98722408,338.49456709)(201.86722732,338.50456697)
\curveto(201.74722432,338.51456707)(201.63222444,338.53456705)(201.52222732,338.56456697)
\curveto(201.13222494,338.67456691)(200.78722528,338.79956678)(200.48722732,338.93956697)
\curveto(200.18722588,339.08956649)(199.93222614,339.30956627)(199.72222732,339.59956697)
\curveto(199.58222649,339.78956579)(199.46222661,340.00956557)(199.36222732,340.25956697)
\curveto(199.34222673,340.31956526)(199.32222675,340.39956518)(199.30222732,340.49956697)
\curveto(199.28222679,340.54956503)(199.2672268,340.61956496)(199.25722732,340.70956697)
\curveto(199.24722682,340.79956478)(199.25222682,340.87456471)(199.27222732,340.93456697)
\curveto(199.30222677,341.00456458)(199.35222672,341.05456453)(199.42222732,341.08456697)
\curveto(199.4722266,341.10456448)(199.53222654,341.11456447)(199.60222732,341.11456697)
\lineto(199.82722732,341.11456697)
\lineto(200.53222732,341.11456697)
\lineto(200.77222732,341.11456697)
\curveto(200.85222522,341.11456447)(200.92222515,341.10456448)(200.98222732,341.08456697)
\curveto(201.09222498,341.04456454)(201.16222491,340.9795646)(201.19222732,340.88956697)
\curveto(201.23222484,340.79956478)(201.27722479,340.70456488)(201.32722732,340.60456697)
\curveto(201.34722472,340.55456503)(201.38222469,340.48956509)(201.43222732,340.40956697)
\curveto(201.49222458,340.32956525)(201.54222453,340.2795653)(201.58222732,340.25956697)
\curveto(201.70222437,340.15956542)(201.81722425,340.0795655)(201.92722732,340.01956697)
\curveto(202.03722403,339.96956561)(202.17722389,339.91956566)(202.34722732,339.86956697)
\curveto(202.39722367,339.84956573)(202.44722362,339.83956574)(202.49722732,339.83956697)
\curveto(202.54722352,339.84956573)(202.59722347,339.84956573)(202.64722732,339.83956697)
\curveto(202.72722334,339.81956576)(202.81222326,339.80956577)(202.90222732,339.80956697)
\curveto(203.00222307,339.81956576)(203.08722298,339.83456575)(203.15722732,339.85456697)
\curveto(203.20722286,339.86456572)(203.25222282,339.86956571)(203.29222732,339.86956697)
\curveto(203.34222273,339.86956571)(203.39222268,339.8795657)(203.44222732,339.89956697)
\curveto(203.58222249,339.94956563)(203.70722236,340.00956557)(203.81722732,340.07956697)
\curveto(203.93722213,340.14956543)(204.03222204,340.23956534)(204.10222732,340.34956697)
\curveto(204.15222192,340.42956515)(204.19222188,340.55456503)(204.22222732,340.72456697)
\curveto(204.24222183,340.79456479)(204.24222183,340.85956472)(204.22222732,340.91956697)
\curveto(204.20222187,340.9795646)(204.18222189,341.02956455)(204.16222732,341.06956697)
\curveto(204.09222198,341.20956437)(204.00222207,341.31456427)(203.89222732,341.38456697)
\curveto(203.79222228,341.45456413)(203.6722224,341.51956406)(203.53222732,341.57956697)
\curveto(203.34222273,341.65956392)(203.14222293,341.72456386)(202.93222732,341.77456697)
\curveto(202.72222335,341.82456376)(202.51222356,341.8795637)(202.30222732,341.93956697)
\curveto(202.22222385,341.95956362)(202.13722393,341.97456361)(202.04722732,341.98456697)
\curveto(201.9672241,341.99456359)(201.88722418,342.00956357)(201.80722732,342.02956697)
\curveto(201.48722458,342.11956346)(201.18222489,342.20456338)(200.89222732,342.28456697)
\curveto(200.60222547,342.37456321)(200.33722573,342.50456308)(200.09722732,342.67456697)
\curveto(199.81722625,342.87456271)(199.61222646,343.14456244)(199.48222732,343.48456697)
\curveto(199.46222661,343.55456203)(199.44222663,343.64956193)(199.42222732,343.76956697)
\curveto(199.40222667,343.83956174)(199.38722668,343.92456166)(199.37722732,344.02456697)
\curveto(199.3672267,344.12456146)(199.3722267,344.21456137)(199.39222732,344.29456697)
\curveto(199.41222666,344.34456124)(199.41722665,344.3845612)(199.40722732,344.41456697)
\curveto(199.39722667,344.45456113)(199.40222667,344.49956108)(199.42222732,344.54956697)
\curveto(199.44222663,344.65956092)(199.46222661,344.75956082)(199.48222732,344.84956697)
\curveto(199.51222656,344.94956063)(199.54722652,345.04456054)(199.58722732,345.13456697)
\curveto(199.71722635,345.42456016)(199.89722617,345.65955992)(200.12722732,345.83956697)
\curveto(200.35722571,346.01955956)(200.61722545,346.16455942)(200.90722732,346.27456697)
\curveto(201.01722505,346.32455926)(201.13222494,346.35955922)(201.25222732,346.37956697)
\curveto(201.3722247,346.40955917)(201.49722457,346.43955914)(201.62722732,346.46956697)
\curveto(201.68722438,346.48955909)(201.74722432,346.49955908)(201.80722732,346.49956697)
\lineto(201.98722732,346.52956697)
\curveto(202.067224,346.53955904)(202.15222392,346.54455904)(202.24222732,346.54456697)
\curveto(202.33222374,346.54455904)(202.41722365,346.54955903)(202.49722732,346.55956697)
}
}
{
\newrgbcolor{curcolor}{0 0 0}
\pscustom[linestyle=none,fillstyle=solid,fillcolor=curcolor]
{
\newpath
\moveto(214.94886795,342.55456697)
\curveto(214.96885978,342.47456311)(214.96885978,342.3845632)(214.94886795,342.28456697)
\curveto(214.92885982,342.1845634)(214.89385986,342.11956346)(214.84386795,342.08956697)
\curveto(214.79385996,342.04956353)(214.71886003,342.01956356)(214.61886795,341.99956697)
\curveto(214.52886022,341.98956359)(214.42386033,341.9795636)(214.30386795,341.96956697)
\lineto(213.95886795,341.96956697)
\curveto(213.8488609,341.9795636)(213.748861,341.9845636)(213.65886795,341.98456697)
\lineto(209.99886795,341.98456697)
\lineto(209.78886795,341.98456697)
\curveto(209.72886502,341.9845636)(209.67386508,341.97456361)(209.62386795,341.95456697)
\curveto(209.54386521,341.91456367)(209.49386526,341.87456371)(209.47386795,341.83456697)
\curveto(209.4538653,341.81456377)(209.43386532,341.77456381)(209.41386795,341.71456697)
\curveto(209.39386536,341.66456392)(209.38886536,341.61456397)(209.39886795,341.56456697)
\curveto(209.41886533,341.50456408)(209.42886532,341.44456414)(209.42886795,341.38456697)
\curveto(209.43886531,341.33456425)(209.4538653,341.2795643)(209.47386795,341.21956697)
\curveto(209.5538652,340.9795646)(209.6488651,340.7795648)(209.75886795,340.61956697)
\curveto(209.87886487,340.46956511)(210.03886471,340.33456525)(210.23886795,340.21456697)
\curveto(210.31886443,340.16456542)(210.39886435,340.12956545)(210.47886795,340.10956697)
\curveto(210.56886418,340.09956548)(210.65886409,340.0795655)(210.74886795,340.04956697)
\curveto(210.82886392,340.02956555)(210.93886381,340.01456557)(211.07886795,340.00456697)
\curveto(211.21886353,339.99456559)(211.33886341,339.99956558)(211.43886795,340.01956697)
\lineto(211.57386795,340.01956697)
\curveto(211.67386308,340.03956554)(211.76386299,340.05956552)(211.84386795,340.07956697)
\curveto(211.93386282,340.10956547)(212.01886273,340.13956544)(212.09886795,340.16956697)
\curveto(212.19886255,340.21956536)(212.30886244,340.2845653)(212.42886795,340.36456697)
\curveto(212.55886219,340.44456514)(212.6538621,340.52456506)(212.71386795,340.60456697)
\curveto(212.76386199,340.67456491)(212.81386194,340.73956484)(212.86386795,340.79956697)
\curveto(212.92386183,340.86956471)(212.99386176,340.91956466)(213.07386795,340.94956697)
\curveto(213.17386158,340.99956458)(213.29886145,341.01956456)(213.44886795,341.00956697)
\lineto(213.88386795,341.00956697)
\lineto(214.06386795,341.00956697)
\curveto(214.13386062,341.01956456)(214.19386056,341.01456457)(214.24386795,340.99456697)
\lineto(214.39386795,340.99456697)
\curveto(214.49386026,340.97456461)(214.56386019,340.94956463)(214.60386795,340.91956697)
\curveto(214.64386011,340.89956468)(214.66386009,340.85456473)(214.66386795,340.78456697)
\curveto(214.67386008,340.71456487)(214.66886008,340.65456493)(214.64886795,340.60456697)
\curveto(214.59886015,340.46456512)(214.54386021,340.33956524)(214.48386795,340.22956697)
\curveto(214.42386033,340.11956546)(214.3538604,340.00956557)(214.27386795,339.89956697)
\curveto(214.0538607,339.56956601)(213.80386095,339.30456628)(213.52386795,339.10456697)
\curveto(213.24386151,338.90456668)(212.89386186,338.73456685)(212.47386795,338.59456697)
\curveto(212.36386239,338.55456703)(212.2538625,338.52956705)(212.14386795,338.51956697)
\curveto(212.03386272,338.50956707)(211.91886283,338.48956709)(211.79886795,338.45956697)
\curveto(211.75886299,338.44956713)(211.71386304,338.44956713)(211.66386795,338.45956697)
\curveto(211.62386313,338.45956712)(211.58386317,338.45456713)(211.54386795,338.44456697)
\lineto(211.37886795,338.44456697)
\curveto(211.32886342,338.42456716)(211.26886348,338.41956716)(211.19886795,338.42956697)
\curveto(211.13886361,338.42956715)(211.08386367,338.43456715)(211.03386795,338.44456697)
\curveto(210.9538638,338.45456713)(210.88386387,338.45456713)(210.82386795,338.44456697)
\curveto(210.76386399,338.43456715)(210.69886405,338.43956714)(210.62886795,338.45956697)
\curveto(210.57886417,338.4795671)(210.52386423,338.48956709)(210.46386795,338.48956697)
\curveto(210.40386435,338.48956709)(210.3488644,338.49956708)(210.29886795,338.51956697)
\curveto(210.18886456,338.53956704)(210.07886467,338.56456702)(209.96886795,338.59456697)
\curveto(209.85886489,338.61456697)(209.75886499,338.64956693)(209.66886795,338.69956697)
\curveto(209.55886519,338.73956684)(209.4538653,338.77456681)(209.35386795,338.80456697)
\curveto(209.26386549,338.84456674)(209.17886557,338.88956669)(209.09886795,338.93956697)
\curveto(208.77886597,339.13956644)(208.49386626,339.36956621)(208.24386795,339.62956697)
\curveto(207.99386676,339.89956568)(207.78886696,340.20956537)(207.62886795,340.55956697)
\curveto(207.57886717,340.66956491)(207.53886721,340.7795648)(207.50886795,340.88956697)
\curveto(207.47886727,341.00956457)(207.43886731,341.12956445)(207.38886795,341.24956697)
\curveto(207.37886737,341.28956429)(207.37386738,341.32456426)(207.37386795,341.35456697)
\curveto(207.37386738,341.39456419)(207.36886738,341.43456415)(207.35886795,341.47456697)
\curveto(207.31886743,341.59456399)(207.29386746,341.72456386)(207.28386795,341.86456697)
\lineto(207.25386795,342.28456697)
\curveto(207.2538675,342.33456325)(207.2488675,342.38956319)(207.23886795,342.44956697)
\curveto(207.23886751,342.50956307)(207.24386751,342.56456302)(207.25386795,342.61456697)
\lineto(207.25386795,342.79456697)
\lineto(207.29886795,343.15456697)
\curveto(207.33886741,343.32456226)(207.37386738,343.48956209)(207.40386795,343.64956697)
\curveto(207.43386732,343.80956177)(207.47886727,343.95956162)(207.53886795,344.09956697)
\curveto(207.96886678,345.13956044)(208.69886605,345.87455971)(209.72886795,346.30456697)
\curveto(209.86886488,346.36455922)(210.00886474,346.40455918)(210.14886795,346.42456697)
\curveto(210.29886445,346.45455913)(210.4538643,346.48955909)(210.61386795,346.52956697)
\curveto(210.69386406,346.53955904)(210.76886398,346.54455904)(210.83886795,346.54456697)
\curveto(210.90886384,346.54455904)(210.98386377,346.54955903)(211.06386795,346.55956697)
\curveto(211.57386318,346.56955901)(212.00886274,346.50955907)(212.36886795,346.37956697)
\curveto(212.73886201,346.25955932)(213.06886168,346.09955948)(213.35886795,345.89956697)
\curveto(213.4488613,345.83955974)(213.53886121,345.76955981)(213.62886795,345.68956697)
\curveto(213.71886103,345.61955996)(213.79886095,345.54456004)(213.86886795,345.46456697)
\curveto(213.89886085,345.41456017)(213.93886081,345.37456021)(213.98886795,345.34456697)
\curveto(214.06886068,345.23456035)(214.14386061,345.11956046)(214.21386795,344.99956697)
\curveto(214.28386047,344.88956069)(214.35886039,344.77456081)(214.43886795,344.65456697)
\curveto(214.48886026,344.56456102)(214.52886022,344.46956111)(214.55886795,344.36956697)
\curveto(214.59886015,344.2795613)(214.63886011,344.1795614)(214.67886795,344.06956697)
\curveto(214.72886002,343.93956164)(214.76885998,343.80456178)(214.79886795,343.66456697)
\curveto(214.82885992,343.52456206)(214.86385989,343.3845622)(214.90386795,343.24456697)
\curveto(214.92385983,343.16456242)(214.92885982,343.07456251)(214.91886795,342.97456697)
\curveto(214.91885983,342.8845627)(214.92885982,342.79956278)(214.94886795,342.71956697)
\lineto(214.94886795,342.55456697)
\moveto(212.69886795,343.43956697)
\curveto(212.76886198,343.53956204)(212.77386198,343.65956192)(212.71386795,343.79956697)
\curveto(212.66386209,343.94956163)(212.62386213,344.05956152)(212.59386795,344.12956697)
\curveto(212.4538623,344.39956118)(212.26886248,344.60456098)(212.03886795,344.74456697)
\curveto(211.80886294,344.89456069)(211.48886326,344.97456061)(211.07886795,344.98456697)
\curveto(211.0488637,344.96456062)(211.01386374,344.95956062)(210.97386795,344.96956697)
\curveto(210.93386382,344.9795606)(210.89886385,344.9795606)(210.86886795,344.96956697)
\curveto(210.81886393,344.94956063)(210.76386399,344.93456065)(210.70386795,344.92456697)
\curveto(210.64386411,344.92456066)(210.58886416,344.91456067)(210.53886795,344.89456697)
\curveto(210.09886465,344.75456083)(209.77386498,344.4795611)(209.56386795,344.06956697)
\curveto(209.54386521,344.02956155)(209.51886523,343.97456161)(209.48886795,343.90456697)
\curveto(209.46886528,343.84456174)(209.4538653,343.7795618)(209.44386795,343.70956697)
\curveto(209.43386532,343.64956193)(209.43386532,343.58956199)(209.44386795,343.52956697)
\curveto(209.46386529,343.46956211)(209.49886525,343.41956216)(209.54886795,343.37956697)
\curveto(209.62886512,343.32956225)(209.73886501,343.30456228)(209.87886795,343.30456697)
\lineto(210.28386795,343.30456697)
\lineto(211.94886795,343.30456697)
\lineto(212.38386795,343.30456697)
\curveto(212.54386221,343.31456227)(212.6488621,343.35956222)(212.69886795,343.43956697)
}
}
{
\newrgbcolor{curcolor}{0 0 0}
\pscustom[linestyle=none,fillstyle=solid,fillcolor=curcolor]
{
\newpath
\moveto(223.5471492,346.25956697)
\curveto(223.617141,346.20955937)(223.65214096,346.12455946)(223.6521492,346.00456697)
\curveto(223.66214095,345.89455969)(223.66714095,345.7795598)(223.6671492,345.65956697)
\lineto(223.6671492,339.25456697)
\curveto(223.66714095,339.17456641)(223.66214095,339.09456649)(223.6521492,339.01456697)
\lineto(223.6521492,338.78956697)
\curveto(223.64214097,338.70956687)(223.63214098,338.63956694)(223.6221492,338.57956697)
\curveto(223.62214099,338.50956707)(223.617141,338.43456715)(223.6071492,338.35456697)
\curveto(223.56714105,338.21456737)(223.53214108,338.0845675)(223.5021492,337.96456697)
\curveto(223.48214113,337.83456775)(223.44714117,337.71456787)(223.3971492,337.60456697)
\curveto(223.22714139,337.22456836)(223.00714161,336.90956867)(222.7371492,336.65956697)
\curveto(222.47714214,336.40956917)(222.15714246,336.20456938)(221.7771492,336.04456697)
\curveto(221.66714295,335.99456959)(221.55714306,335.95456963)(221.4471492,335.92456697)
\curveto(221.33714328,335.89456969)(221.22214339,335.86456972)(221.1021492,335.83456697)
\curveto(220.99214362,335.80456978)(220.88214373,335.7845698)(220.7721492,335.77456697)
\curveto(220.66214395,335.76456982)(220.55214406,335.74956983)(220.4421492,335.72956697)
\lineto(220.3221492,335.72956697)
\curveto(220.28214433,335.71956986)(220.23714438,335.71456987)(220.1871492,335.71456697)
\curveto(220.14714447,335.70456988)(220.10214451,335.70456988)(220.0521492,335.71456697)
\curveto(220.00214461,335.71456987)(219.95214466,335.70956987)(219.9021492,335.69956697)
\curveto(219.85214476,335.68956989)(219.78714483,335.6845699)(219.7071492,335.68456697)
\curveto(219.62714499,335.6845699)(219.56214505,335.68956989)(219.5121492,335.69956697)
\lineto(219.3771492,335.69956697)
\curveto(219.33714528,335.69956988)(219.29714532,335.70456988)(219.2571492,335.71456697)
\curveto(219.17714544,335.73456985)(219.09214552,335.74456984)(219.0021492,335.74456697)
\curveto(218.92214569,335.74456984)(218.84714577,335.75456983)(218.7771492,335.77456697)
\curveto(218.75714586,335.7845698)(218.73214588,335.78956979)(218.7021492,335.78956697)
\curveto(218.67214594,335.78956979)(218.64714597,335.79456979)(218.6271492,335.80456697)
\curveto(218.52714609,335.82456976)(218.42714619,335.84956973)(218.3271492,335.87956697)
\curveto(218.23714638,335.89956968)(218.14714647,335.92956965)(218.0571492,335.96956697)
\curveto(217.67714694,336.12956945)(217.33714728,336.33456925)(217.0371492,336.58456697)
\curveto(216.73714788,336.82456876)(216.5171481,337.14956843)(216.3771492,337.55956697)
\curveto(216.35714826,337.58956799)(216.34714827,337.61956796)(216.3471492,337.64956697)
\curveto(216.34714827,337.6795679)(216.34214827,337.70456788)(216.3321492,337.72456697)
\curveto(216.30214831,337.85456773)(216.3121483,337.95456763)(216.3621492,338.02456697)
\curveto(216.42214819,338.0845675)(216.50214811,338.12456746)(216.6021492,338.14456697)
\curveto(216.70214791,338.16456742)(216.8121478,338.17456741)(216.9321492,338.17456697)
\curveto(217.06214755,338.16456742)(217.18214743,338.15956742)(217.2921492,338.15956697)
\lineto(217.8021492,338.15956697)
\lineto(217.9221492,338.15956697)
\curveto(217.96214665,338.14956743)(218.00714661,338.14456744)(218.0571492,338.14456697)
\curveto(218.2171464,338.10456748)(218.3171463,338.05456753)(218.3571492,337.99456697)
\curveto(218.39714622,337.92456766)(218.45714616,337.83456775)(218.5371492,337.72456697)
\curveto(218.56714605,337.6845679)(218.612146,337.63456795)(218.6721492,337.57456697)
\curveto(218.68214593,337.55456803)(218.69214592,337.53956804)(218.7021492,337.52956697)
\curveto(218.7121459,337.51956806)(218.72214589,337.50456808)(218.7321492,337.48456697)
\curveto(218.8121458,337.42456816)(218.89714572,337.36956821)(218.9871492,337.31956697)
\curveto(219.07714554,337.26956831)(219.17714544,337.22456836)(219.2871492,337.18456697)
\curveto(219.35714526,337.16456842)(219.42714519,337.15456843)(219.4971492,337.15456697)
\curveto(219.56714505,337.14456844)(219.64214497,337.12956845)(219.7221492,337.10956697)
\lineto(219.8871492,337.10956697)
\curveto(219.95714466,337.08956849)(220.04714457,337.08956849)(220.1571492,337.10956697)
\curveto(220.26714435,337.11956846)(220.35214426,337.13456845)(220.4121492,337.15456697)
\curveto(220.46214415,337.17456841)(220.50214411,337.1845684)(220.5321492,337.18456697)
\curveto(220.57214404,337.1845684)(220.612144,337.19456839)(220.6521492,337.21456697)
\curveto(220.86214375,337.30456828)(221.03714358,337.42456816)(221.1771492,337.57456697)
\curveto(221.3171433,337.72456786)(221.43214318,337.89956768)(221.5221492,338.09956697)
\curveto(221.54214307,338.15956742)(221.55714306,338.21956736)(221.5671492,338.27956697)
\curveto(221.57714304,338.33956724)(221.59214302,338.40456718)(221.6121492,338.47456697)
\curveto(221.63214298,338.56456702)(221.64214297,338.65956692)(221.6421492,338.75956697)
\curveto(221.65214296,338.86956671)(221.65714296,338.9795666)(221.6571492,339.08956697)
\lineto(221.6571492,339.20956697)
\curveto(221.66714295,339.24956633)(221.66714295,339.2845663)(221.6571492,339.31456697)
\curveto(221.63714298,339.36456622)(221.62714299,339.40956617)(221.6271492,339.44956697)
\curveto(221.63714298,339.48956609)(221.63214298,339.52956605)(221.6121492,339.56956697)
\curveto(221.60214301,339.58956599)(221.58714303,339.60456598)(221.5671492,339.61456697)
\lineto(221.5221492,339.65956697)
\curveto(221.43214318,339.66956591)(221.35714326,339.64956593)(221.2971492,339.59956697)
\curveto(221.24714337,339.54956603)(221.19714342,339.50456608)(221.1471492,339.46456697)
\curveto(221.05714356,339.39456619)(220.96714365,339.32956625)(220.8771492,339.26956697)
\curveto(220.78714383,339.20956637)(220.68714393,339.15456643)(220.5771492,339.10456697)
\curveto(220.46714415,339.05456653)(220.35714426,339.01456657)(220.2471492,338.98456697)
\curveto(220.13714448,338.95456663)(220.02214459,338.92456666)(219.9021492,338.89456697)
\lineto(219.7221492,338.86456697)
\curveto(219.67214494,338.86456672)(219.62214499,338.85956672)(219.5721492,338.84956697)
\curveto(219.52214509,338.83956674)(219.44214517,338.83456675)(219.3321492,338.83456697)
\curveto(219.22214539,338.83456675)(219.14214547,338.83956674)(219.0921492,338.84956697)
\lineto(218.9721492,338.84956697)
\curveto(218.94214567,338.85956672)(218.90714571,338.86456672)(218.8671492,338.86456697)
\curveto(218.83714578,338.86456672)(218.80214581,338.86956671)(218.7621492,338.87956697)
\curveto(218.62214599,338.90956667)(218.48714613,338.93456665)(218.3571492,338.95456697)
\curveto(218.22714639,338.9845666)(218.10714651,339.02456656)(217.9971492,339.07456697)
\curveto(217.56714705,339.24456634)(217.2171474,339.4795661)(216.9471492,339.77956697)
\curveto(216.68714793,340.08956549)(216.46714815,340.45956512)(216.2871492,340.88956697)
\curveto(216.23714838,340.99956458)(216.20214841,341.11456447)(216.1821492,341.23456697)
\curveto(216.16214845,341.35456423)(216.13214848,341.47456411)(216.0921492,341.59456697)
\curveto(216.09214852,341.64456394)(216.08714853,341.6845639)(216.0771492,341.71456697)
\curveto(216.05714856,341.79456379)(216.04714857,341.8795637)(216.0471492,341.96956697)
\curveto(216.04714857,342.06956351)(216.03714858,342.15956342)(216.0171492,342.23956697)
\curveto(216.00714861,342.28956329)(216.00214861,342.33456325)(216.0021492,342.37456697)
\lineto(216.0021492,342.52456697)
\curveto(215.99214862,342.57456301)(215.98714863,342.63456295)(215.9871492,342.70456697)
\curveto(215.98714863,342.7845628)(215.99214862,342.84956273)(216.0021492,342.89956697)
\lineto(216.0021492,343.04956697)
\curveto(216.0121486,343.08956249)(216.0121486,343.12956245)(216.0021492,343.16956697)
\curveto(216.00214861,343.20956237)(216.0121486,343.24956233)(216.0321492,343.28956697)
\curveto(216.05214856,343.38956219)(216.06714855,343.4845621)(216.0771492,343.57456697)
\curveto(216.08714853,343.67456191)(216.10214851,343.77456181)(216.1221492,343.87456697)
\curveto(216.18214843,344.07456151)(216.24214837,344.26456132)(216.3021492,344.44456697)
\curveto(216.37214824,344.62456096)(216.45714816,344.79456079)(216.5571492,344.95456697)
\curveto(216.60714801,345.05456053)(216.66214795,345.14456044)(216.7221492,345.22456697)
\lineto(216.9321492,345.49456697)
\curveto(216.96214765,345.54456004)(217.00214761,345.59455999)(217.0521492,345.64456697)
\curveto(217.1121475,345.69455989)(217.16714745,345.73955984)(217.2171492,345.77956697)
\lineto(217.3071492,345.86956697)
\curveto(217.35714726,345.90955967)(217.40714721,345.94455964)(217.4571492,345.97456697)
\curveto(217.50714711,346.01455957)(217.55714706,346.04955953)(217.6071492,346.07956697)
\curveto(217.73714688,346.15955942)(217.87214674,346.22955935)(218.0121492,346.28956697)
\curveto(218.15214646,346.34955923)(218.30714631,346.40455918)(218.4771492,346.45456697)
\curveto(218.55714606,346.4845591)(218.63714598,346.49955908)(218.7171492,346.49956697)
\curveto(218.80714581,346.50955907)(218.89214572,346.52455906)(218.9721492,346.54456697)
\curveto(219.0121456,346.55455903)(219.06714555,346.55455903)(219.1371492,346.54456697)
\curveto(219.20714541,346.53455905)(219.25214536,346.53955904)(219.2721492,346.55956697)
\curveto(219.59214502,346.56955901)(219.87714474,346.53955904)(220.1271492,346.46956697)
\curveto(220.38714423,346.39955918)(220.617144,346.29955928)(220.8171492,346.16956697)
\curveto(220.84714377,346.14955943)(220.87714374,346.12455946)(220.9071492,346.09456697)
\curveto(220.93714368,346.07455951)(220.97214364,346.04955953)(221.0121492,346.01956697)
\curveto(221.07214354,345.96955961)(221.12714349,345.91955966)(221.1771492,345.86956697)
\curveto(221.22714339,345.81955976)(221.28714333,345.77455981)(221.3571492,345.73456697)
\curveto(221.37714324,345.72455986)(221.40214321,345.71455987)(221.4321492,345.70456697)
\curveto(221.47214314,345.69455989)(221.50214311,345.69955988)(221.5221492,345.71956697)
\curveto(221.57214304,345.73955984)(221.60214301,345.77455981)(221.6121492,345.82456697)
\curveto(221.62214299,345.87455971)(221.63714298,345.92455966)(221.6571492,345.97456697)
\curveto(221.67714294,346.02455956)(221.69214292,346.07455951)(221.7021492,346.12456697)
\curveto(221.72214289,346.1845594)(221.75214286,346.23455935)(221.7921492,346.27456697)
\curveto(221.85214276,346.31455927)(221.92214269,346.33455925)(222.0021492,346.33456697)
\curveto(222.09214252,346.34455924)(222.18214243,346.34955923)(222.2721492,346.34956697)
\lineto(223.0371492,346.34956697)
\curveto(223.14714147,346.34955923)(223.24214137,346.34455924)(223.3221492,346.33456697)
\curveto(223.4121412,346.33455925)(223.48714113,346.30955927)(223.5471492,346.25956697)
\moveto(221.4921492,341.62456697)
\curveto(221.53214308,341.71456387)(221.56714305,341.82956375)(221.5971492,341.96956697)
\curveto(221.62714299,342.10956347)(221.64714297,342.25456333)(221.6571492,342.40456697)
\curveto(221.66714295,342.56456302)(221.66714295,342.71956286)(221.6571492,342.86956697)
\curveto(221.65714296,343.01956256)(221.64214297,343.15456243)(221.6121492,343.27456697)
\curveto(221.59214302,343.31456227)(221.58214303,343.34456224)(221.5821492,343.36456697)
\curveto(221.59214302,343.39456219)(221.59214302,343.42956215)(221.5821492,343.46956697)
\lineto(221.5221492,343.67956697)
\curveto(221.50214311,343.74956183)(221.47714314,343.81456177)(221.4471492,343.87456697)
\curveto(221.30714331,344.22456136)(221.10714351,344.49456109)(220.8471492,344.68456697)
\curveto(220.58714403,344.87456071)(220.20714441,344.96956061)(219.7071492,344.96956697)
\curveto(219.68714493,344.94956063)(219.65714496,344.93956064)(219.6171492,344.93956697)
\curveto(219.58714503,344.94956063)(219.55714506,344.94956063)(219.5271492,344.93956697)
\curveto(219.45714516,344.91956066)(219.39214522,344.89956068)(219.3321492,344.87956697)
\curveto(219.27214534,344.86956071)(219.2121454,344.85456073)(219.1521492,344.83456697)
\curveto(218.89214572,344.72456086)(218.69214592,344.55956102)(218.5521492,344.33956697)
\curveto(218.4121462,344.11956146)(218.29714632,343.87456171)(218.2071492,343.60456697)
\curveto(218.18714643,343.55456203)(218.17714644,343.51456207)(218.1771492,343.48456697)
\curveto(218.17714644,343.45456213)(218.17214644,343.41456217)(218.1621492,343.36456697)
\curveto(218.13214648,343.25456233)(218.1121465,343.09456249)(218.1021492,342.88456697)
\curveto(218.09214652,342.67456291)(218.10214651,342.50456308)(218.1321492,342.37456697)
\lineto(218.1321492,342.22456697)
\curveto(218.15214646,342.14456344)(218.16714645,342.06456352)(218.1771492,341.98456697)
\curveto(218.18714643,341.91456367)(218.20214641,341.83956374)(218.2221492,341.75956697)
\curveto(218.3121463,341.49956408)(218.42214619,341.26956431)(218.5521492,341.06956697)
\curveto(218.68214593,340.8795647)(218.86214575,340.72456486)(219.0921492,340.60456697)
\curveto(219.19214542,340.55456503)(219.33214528,340.50456508)(219.5121492,340.45456697)
\curveto(219.58214503,340.45456513)(219.63714498,340.44956513)(219.6771492,340.43956697)
\curveto(219.69714492,340.43956514)(219.72714489,340.43456515)(219.7671492,340.42456697)
\curveto(219.80714481,340.42456516)(219.83714478,340.42956515)(219.8571492,340.43956697)
\lineto(220.0071492,340.43956697)
\curveto(220.09714452,340.45956512)(220.18214443,340.47456511)(220.2621492,340.48456697)
\curveto(220.34214427,340.49456509)(220.42214419,340.51956506)(220.5021492,340.55956697)
\curveto(220.75214386,340.65956492)(220.95214366,340.79956478)(221.1021492,340.97956697)
\curveto(221.26214335,341.15956442)(221.39214322,341.37456421)(221.4921492,341.62456697)
}
}
{
\newrgbcolor{curcolor}{0 0 0}
\pscustom[linestyle=none,fillstyle=solid,fillcolor=curcolor]
{
\newpath
\moveto(225.78707107,346.33456697)
\lineto(226.91207107,346.33456697)
\curveto(227.02206864,346.33455925)(227.12206854,346.32955925)(227.21207107,346.31956697)
\curveto(227.30206836,346.30955927)(227.36706829,346.27455931)(227.40707107,346.21456697)
\curveto(227.4570682,346.15455943)(227.48706817,346.06955951)(227.49707107,345.95956697)
\curveto(227.50706815,345.85955972)(227.51206815,345.75455983)(227.51207107,345.64456697)
\lineto(227.51207107,344.59456697)
\lineto(227.51207107,342.35956697)
\curveto(227.51206815,341.99956358)(227.52706813,341.65956392)(227.55707107,341.33956697)
\curveto(227.58706807,341.01956456)(227.67706798,340.75456483)(227.82707107,340.54456697)
\curveto(227.96706769,340.33456525)(228.19206747,340.1845654)(228.50207107,340.09456697)
\curveto(228.55206711,340.0845655)(228.59206707,340.0795655)(228.62207107,340.07956697)
\curveto(228.662067,340.0795655)(228.70706695,340.07456551)(228.75707107,340.06456697)
\curveto(228.80706685,340.05456553)(228.8620668,340.04956553)(228.92207107,340.04956697)
\curveto(228.98206668,340.04956553)(229.02706663,340.05456553)(229.05707107,340.06456697)
\curveto(229.10706655,340.0845655)(229.14706651,340.08956549)(229.17707107,340.07956697)
\curveto(229.21706644,340.06956551)(229.2570664,340.07456551)(229.29707107,340.09456697)
\curveto(229.50706615,340.14456544)(229.67206599,340.20956537)(229.79207107,340.28956697)
\curveto(229.97206569,340.39956518)(230.11206555,340.53956504)(230.21207107,340.70956697)
\curveto(230.32206534,340.88956469)(230.39706526,341.0845645)(230.43707107,341.29456697)
\curveto(230.48706517,341.51456407)(230.51706514,341.75456383)(230.52707107,342.01456697)
\curveto(230.53706512,342.2845633)(230.54206512,342.56456302)(230.54207107,342.85456697)
\lineto(230.54207107,344.66956697)
\lineto(230.54207107,345.64456697)
\lineto(230.54207107,345.91456697)
\curveto(230.54206512,346.01455957)(230.5620651,346.09455949)(230.60207107,346.15456697)
\curveto(230.65206501,346.24455934)(230.72706493,346.29455929)(230.82707107,346.30456697)
\curveto(230.92706473,346.32455926)(231.04706461,346.33455925)(231.18707107,346.33456697)
\lineto(231.98207107,346.33456697)
\lineto(232.26707107,346.33456697)
\curveto(232.3570633,346.33455925)(232.43206323,346.31455927)(232.49207107,346.27456697)
\curveto(232.57206309,346.22455936)(232.61706304,346.14955943)(232.62707107,346.04956697)
\curveto(232.63706302,345.94955963)(232.64206302,345.83455975)(232.64207107,345.70456697)
\lineto(232.64207107,344.56456697)
\lineto(232.64207107,340.34956697)
\lineto(232.64207107,339.28456697)
\lineto(232.64207107,338.98456697)
\curveto(232.64206302,338.8845667)(232.62206304,338.80956677)(232.58207107,338.75956697)
\curveto(232.53206313,338.6795669)(232.4570632,338.63456695)(232.35707107,338.62456697)
\curveto(232.2570634,338.61456697)(232.15206351,338.60956697)(232.04207107,338.60956697)
\lineto(231.23207107,338.60956697)
\curveto(231.12206454,338.60956697)(231.02206464,338.61456697)(230.93207107,338.62456697)
\curveto(230.85206481,338.63456695)(230.78706487,338.67456691)(230.73707107,338.74456697)
\curveto(230.71706494,338.77456681)(230.69706496,338.81956676)(230.67707107,338.87956697)
\curveto(230.66706499,338.93956664)(230.65206501,338.99956658)(230.63207107,339.05956697)
\curveto(230.62206504,339.11956646)(230.60706505,339.17456641)(230.58707107,339.22456697)
\curveto(230.56706509,339.27456631)(230.53706512,339.30456628)(230.49707107,339.31456697)
\curveto(230.47706518,339.33456625)(230.45206521,339.33956624)(230.42207107,339.32956697)
\curveto(230.39206527,339.31956626)(230.36706529,339.30956627)(230.34707107,339.29956697)
\curveto(230.27706538,339.25956632)(230.21706544,339.21456637)(230.16707107,339.16456697)
\curveto(230.11706554,339.11456647)(230.0620656,339.06956651)(230.00207107,339.02956697)
\curveto(229.9620657,338.99956658)(229.92206574,338.96456662)(229.88207107,338.92456697)
\curveto(229.85206581,338.89456669)(229.81206585,338.86456672)(229.76207107,338.83456697)
\curveto(229.53206613,338.69456689)(229.2620664,338.584567)(228.95207107,338.50456697)
\curveto(228.88206678,338.4845671)(228.81206685,338.47456711)(228.74207107,338.47456697)
\curveto(228.67206699,338.46456712)(228.59706706,338.44956713)(228.51707107,338.42956697)
\curveto(228.47706718,338.41956716)(228.43206723,338.41956716)(228.38207107,338.42956697)
\curveto(228.34206732,338.42956715)(228.30206736,338.42456716)(228.26207107,338.41456697)
\curveto(228.23206743,338.40456718)(228.16706749,338.40456718)(228.06707107,338.41456697)
\curveto(227.97706768,338.41456717)(227.91706774,338.41956716)(227.88707107,338.42956697)
\curveto(227.83706782,338.42956715)(227.78706787,338.43456715)(227.73707107,338.44456697)
\lineto(227.58707107,338.44456697)
\curveto(227.46706819,338.47456711)(227.35206831,338.49956708)(227.24207107,338.51956697)
\curveto(227.13206853,338.53956704)(227.02206864,338.56956701)(226.91207107,338.60956697)
\curveto(226.8620688,338.62956695)(226.81706884,338.64456694)(226.77707107,338.65456697)
\curveto(226.74706891,338.67456691)(226.70706895,338.69456689)(226.65707107,338.71456697)
\curveto(226.30706935,338.90456668)(226.02706963,339.16956641)(225.81707107,339.50956697)
\curveto(225.68706997,339.71956586)(225.59207007,339.96956561)(225.53207107,340.25956697)
\curveto(225.47207019,340.55956502)(225.43207023,340.87456471)(225.41207107,341.20456697)
\curveto(225.40207026,341.54456404)(225.39707026,341.88956369)(225.39707107,342.23956697)
\curveto(225.40707025,342.59956298)(225.41207025,342.95456263)(225.41207107,343.30456697)
\lineto(225.41207107,345.34456697)
\curveto(225.41207025,345.47456011)(225.40707025,345.62455996)(225.39707107,345.79456697)
\curveto(225.39707026,345.97455961)(225.42207024,346.10455948)(225.47207107,346.18456697)
\curveto(225.50207016,346.23455935)(225.5620701,346.2795593)(225.65207107,346.31956697)
\curveto(225.71206995,346.31955926)(225.7570699,346.32455926)(225.78707107,346.33456697)
\moveto(229.65707107,349.42456697)
\lineto(230.72207107,349.42456697)
\curveto(230.80206486,349.42455616)(230.89706476,349.42455616)(231.00707107,349.42456697)
\curveto(231.11706454,349.42455616)(231.19706446,349.40955617)(231.24707107,349.37956697)
\curveto(231.26706439,349.36955621)(231.27706438,349.35455623)(231.27707107,349.33456697)
\curveto(231.28706437,349.32455626)(231.30206436,349.31455627)(231.32207107,349.30456697)
\curveto(231.33206433,349.1845564)(231.28206438,349.0795565)(231.17207107,348.98956697)
\curveto(231.07206459,348.89955668)(230.98706467,348.81955676)(230.91707107,348.74956697)
\curveto(230.83706482,348.6795569)(230.7570649,348.60455698)(230.67707107,348.52456697)
\curveto(230.60706505,348.45455713)(230.53206513,348.38955719)(230.45207107,348.32956697)
\curveto(230.41206525,348.29955728)(230.37706528,348.26455732)(230.34707107,348.22456697)
\curveto(230.32706533,348.19455739)(230.29706536,348.16955741)(230.25707107,348.14956697)
\curveto(230.23706542,348.11955746)(230.21206545,348.09455749)(230.18207107,348.07456697)
\lineto(230.03207107,347.92456697)
\lineto(229.88207107,347.80456697)
\lineto(229.83707107,347.75956697)
\curveto(229.83706582,347.74955783)(229.82706583,347.73455785)(229.80707107,347.71456697)
\curveto(229.72706593,347.65455793)(229.64706601,347.58955799)(229.56707107,347.51956697)
\curveto(229.49706616,347.44955813)(229.40706625,347.39455819)(229.29707107,347.35456697)
\curveto(229.2570664,347.34455824)(229.21706644,347.33955824)(229.17707107,347.33956697)
\curveto(229.14706651,347.33955824)(229.10706655,347.33455825)(229.05707107,347.32456697)
\curveto(229.02706663,347.31455827)(228.98706667,347.30955827)(228.93707107,347.30956697)
\curveto(228.88706677,347.31955826)(228.84206682,347.32455826)(228.80207107,347.32456697)
\lineto(228.45707107,347.32456697)
\curveto(228.33706732,347.32455826)(228.24706741,347.34955823)(228.18707107,347.39956697)
\curveto(228.12706753,347.43955814)(228.11206755,347.50955807)(228.14207107,347.60956697)
\curveto(228.1620675,347.68955789)(228.19706746,347.75955782)(228.24707107,347.81956697)
\curveto(228.29706736,347.88955769)(228.34206732,347.95955762)(228.38207107,348.02956697)
\curveto(228.48206718,348.16955741)(228.57706708,348.30455728)(228.66707107,348.43456697)
\curveto(228.7570669,348.56455702)(228.84706681,348.69955688)(228.93707107,348.83956697)
\curveto(228.98706667,348.91955666)(229.03706662,349.00455658)(229.08707107,349.09456697)
\curveto(229.14706651,349.1845564)(229.21206645,349.25455633)(229.28207107,349.30456697)
\curveto(229.32206634,349.33455625)(229.39206627,349.36955621)(229.49207107,349.40956697)
\curveto(229.51206615,349.41955616)(229.53706612,349.41955616)(229.56707107,349.40956697)
\curveto(229.60706605,349.40955617)(229.63706602,349.41455617)(229.65707107,349.42456697)
}
}
{
\newrgbcolor{curcolor}{0 0 0}
\pscustom[linestyle=none,fillstyle=solid,fillcolor=curcolor]
{
\newpath
\moveto(238.69832107,346.54456697)
\curveto(239.29831527,346.56455902)(239.79831477,346.4795591)(240.19832107,346.28956697)
\curveto(240.59831397,346.09955948)(240.91331365,345.81955976)(241.14332107,345.44956697)
\curveto(241.21331335,345.33956024)(241.2683133,345.21956036)(241.30832107,345.08956697)
\curveto(241.34831322,344.96956061)(241.38831318,344.84456074)(241.42832107,344.71456697)
\curveto(241.44831312,344.63456095)(241.45831311,344.55956102)(241.45832107,344.48956697)
\curveto(241.4683131,344.41956116)(241.48331308,344.34956123)(241.50332107,344.27956697)
\curveto(241.50331306,344.21956136)(241.50831306,344.1795614)(241.51832107,344.15956697)
\curveto(241.53831303,344.01956156)(241.54831302,343.87456171)(241.54832107,343.72456697)
\lineto(241.54832107,343.28956697)
\lineto(241.54832107,341.95456697)
\lineto(241.54832107,339.52456697)
\curveto(241.54831302,339.33456625)(241.54331302,339.14956643)(241.53332107,338.96956697)
\curveto(241.53331303,338.79956678)(241.4633131,338.68956689)(241.32332107,338.63956697)
\curveto(241.2633133,338.61956696)(241.19331337,338.60956697)(241.11332107,338.60956697)
\lineto(240.87332107,338.60956697)
\lineto(240.06332107,338.60956697)
\curveto(239.94331462,338.60956697)(239.83331473,338.61456697)(239.73332107,338.62456697)
\curveto(239.64331492,338.64456694)(239.57331499,338.68956689)(239.52332107,338.75956697)
\curveto(239.48331508,338.81956676)(239.45831511,338.89456669)(239.44832107,338.98456697)
\lineto(239.44832107,339.29956697)
\lineto(239.44832107,340.34956697)
\lineto(239.44832107,342.58456697)
\curveto(239.44831512,342.95456263)(239.43331513,343.29456229)(239.40332107,343.60456697)
\curveto(239.37331519,343.92456166)(239.28331528,344.19456139)(239.13332107,344.41456697)
\curveto(238.99331557,344.61456097)(238.78831578,344.75456083)(238.51832107,344.83456697)
\curveto(238.4683161,344.85456073)(238.41331615,344.86456072)(238.35332107,344.86456697)
\curveto(238.30331626,344.86456072)(238.24831632,344.87456071)(238.18832107,344.89456697)
\curveto(238.13831643,344.90456068)(238.07331649,344.90456068)(237.99332107,344.89456697)
\curveto(237.92331664,344.89456069)(237.8683167,344.88956069)(237.82832107,344.87956697)
\curveto(237.78831678,344.86956071)(237.75331681,344.86456072)(237.72332107,344.86456697)
\curveto(237.69331687,344.86456072)(237.6633169,344.85956072)(237.63332107,344.84956697)
\curveto(237.40331716,344.78956079)(237.21831735,344.70956087)(237.07832107,344.60956697)
\curveto(236.75831781,344.3795612)(236.568318,344.04456154)(236.50832107,343.60456697)
\curveto(236.44831812,343.16456242)(236.41831815,342.66956291)(236.41832107,342.11956697)
\lineto(236.41832107,340.24456697)
\lineto(236.41832107,339.32956697)
\lineto(236.41832107,339.05956697)
\curveto(236.41831815,338.96956661)(236.40331816,338.89456669)(236.37332107,338.83456697)
\curveto(236.32331824,338.72456686)(236.24331832,338.65956692)(236.13332107,338.63956697)
\curveto(236.02331854,338.61956696)(235.88831868,338.60956697)(235.72832107,338.60956697)
\lineto(234.97832107,338.60956697)
\curveto(234.8683197,338.60956697)(234.75831981,338.61456697)(234.64832107,338.62456697)
\curveto(234.53832003,338.63456695)(234.45832011,338.66956691)(234.40832107,338.72956697)
\curveto(234.33832023,338.81956676)(234.30332026,338.94956663)(234.30332107,339.11956697)
\curveto(234.31332025,339.28956629)(234.31832025,339.44956613)(234.31832107,339.59956697)
\lineto(234.31832107,341.63956697)
\lineto(234.31832107,344.93956697)
\lineto(234.31832107,345.70456697)
\lineto(234.31832107,346.00456697)
\curveto(234.32832024,346.09455949)(234.35832021,346.16955941)(234.40832107,346.22956697)
\curveto(234.42832014,346.25955932)(234.45832011,346.2795593)(234.49832107,346.28956697)
\curveto(234.54832002,346.30955927)(234.59831997,346.32455926)(234.64832107,346.33456697)
\lineto(234.72332107,346.33456697)
\curveto(234.77331979,346.34455924)(234.82331974,346.34955923)(234.87332107,346.34956697)
\lineto(235.03832107,346.34956697)
\lineto(235.66832107,346.34956697)
\curveto(235.74831882,346.34955923)(235.82331874,346.34455924)(235.89332107,346.33456697)
\curveto(235.97331859,346.33455925)(236.04331852,346.32455926)(236.10332107,346.30456697)
\curveto(236.17331839,346.27455931)(236.21831835,346.22955935)(236.23832107,346.16956697)
\curveto(236.2683183,346.10955947)(236.29331827,346.03955954)(236.31332107,345.95956697)
\curveto(236.32331824,345.91955966)(236.32331824,345.8845597)(236.31332107,345.85456697)
\curveto(236.31331825,345.82455976)(236.32331824,345.79455979)(236.34332107,345.76456697)
\curveto(236.3633182,345.71455987)(236.37831819,345.6845599)(236.38832107,345.67456697)
\curveto(236.40831816,345.66455992)(236.43331813,345.64955993)(236.46332107,345.62956697)
\curveto(236.57331799,345.61955996)(236.6633179,345.65455993)(236.73332107,345.73456697)
\curveto(236.80331776,345.82455976)(236.87831769,345.89455969)(236.95832107,345.94456697)
\curveto(237.22831734,346.14455944)(237.52831704,346.30455928)(237.85832107,346.42456697)
\curveto(237.94831662,346.45455913)(238.03831653,346.47455911)(238.12832107,346.48456697)
\curveto(238.22831634,346.49455909)(238.33331623,346.50955907)(238.44332107,346.52956697)
\curveto(238.47331609,346.53955904)(238.51831605,346.53955904)(238.57832107,346.52956697)
\curveto(238.63831593,346.52955905)(238.67831589,346.53455905)(238.69832107,346.54456697)
}
}
{
\newrgbcolor{curcolor}{0 0 0}
\pscustom[linestyle=none,fillstyle=solid,fillcolor=curcolor]
{
}
}
{
\newrgbcolor{curcolor}{0 0 0}
\pscustom[linestyle=none,fillstyle=solid,fillcolor=curcolor]
{
\newpath
\moveto(250.30972732,346.55956697)
\curveto(251.05972282,346.579559)(251.70972217,346.49455909)(252.25972732,346.30456697)
\curveto(252.81972106,346.12455946)(253.24472064,345.80955977)(253.53472732,345.35956697)
\curveto(253.60472028,345.24956033)(253.66472022,345.13456045)(253.71472732,345.01456697)
\curveto(253.77472011,344.90456068)(253.82472006,344.7795608)(253.86472732,344.63956697)
\curveto(253.88472,344.579561)(253.89471999,344.51456107)(253.89472732,344.44456697)
\curveto(253.89471999,344.37456121)(253.88472,344.31456127)(253.86472732,344.26456697)
\curveto(253.82472006,344.20456138)(253.76972011,344.16456142)(253.69972732,344.14456697)
\curveto(253.64972023,344.12456146)(253.58972029,344.11456147)(253.51972732,344.11456697)
\lineto(253.30972732,344.11456697)
\lineto(252.64972732,344.11456697)
\curveto(252.5797213,344.11456147)(252.50972137,344.10956147)(252.43972732,344.09956697)
\curveto(252.36972151,344.09956148)(252.30472158,344.10956147)(252.24472732,344.12956697)
\curveto(252.14472174,344.14956143)(252.06972181,344.18956139)(252.01972732,344.24956697)
\curveto(251.96972191,344.30956127)(251.92472196,344.36956121)(251.88472732,344.42956697)
\lineto(251.76472732,344.63956697)
\curveto(251.73472215,344.71956086)(251.6847222,344.7845608)(251.61472732,344.83456697)
\curveto(251.51472237,344.91456067)(251.41472247,344.97456061)(251.31472732,345.01456697)
\curveto(251.22472266,345.05456053)(251.10972277,345.08956049)(250.96972732,345.11956697)
\curveto(250.89972298,345.13956044)(250.79472309,345.15456043)(250.65472732,345.16456697)
\curveto(250.52472336,345.17456041)(250.42472346,345.16956041)(250.35472732,345.14956697)
\lineto(250.24972732,345.14956697)
\lineto(250.09972732,345.11956697)
\curveto(250.05972382,345.11956046)(250.01472387,345.11456047)(249.96472732,345.10456697)
\curveto(249.79472409,345.05456053)(249.65472423,344.9845606)(249.54472732,344.89456697)
\curveto(249.44472444,344.81456077)(249.37472451,344.68956089)(249.33472732,344.51956697)
\curveto(249.31472457,344.44956113)(249.31472457,344.3845612)(249.33472732,344.32456697)
\curveto(249.35472453,344.26456132)(249.37472451,344.21456137)(249.39472732,344.17456697)
\curveto(249.46472442,344.05456153)(249.54472434,343.95956162)(249.63472732,343.88956697)
\curveto(249.73472415,343.81956176)(249.84972403,343.75956182)(249.97972732,343.70956697)
\curveto(250.16972371,343.62956195)(250.37472351,343.55956202)(250.59472732,343.49956697)
\lineto(251.28472732,343.34956697)
\curveto(251.52472236,343.30956227)(251.75472213,343.25956232)(251.97472732,343.19956697)
\curveto(252.20472168,343.14956243)(252.41972146,343.0845625)(252.61972732,343.00456697)
\curveto(252.70972117,342.96456262)(252.79472109,342.92956265)(252.87472732,342.89956697)
\curveto(252.96472092,342.8795627)(253.04972083,342.84456274)(253.12972732,342.79456697)
\curveto(253.31972056,342.67456291)(253.48972039,342.54456304)(253.63972732,342.40456697)
\curveto(253.79972008,342.26456332)(253.92471996,342.08956349)(254.01472732,341.87956697)
\curveto(254.04471984,341.80956377)(254.06971981,341.73956384)(254.08972732,341.66956697)
\curveto(254.10971977,341.59956398)(254.12971975,341.52456406)(254.14972732,341.44456697)
\curveto(254.15971972,341.3845642)(254.16471972,341.28956429)(254.16472732,341.15956697)
\curveto(254.17471971,341.03956454)(254.17471971,340.94456464)(254.16472732,340.87456697)
\lineto(254.16472732,340.79956697)
\curveto(254.14471974,340.73956484)(254.12971975,340.6795649)(254.11972732,340.61956697)
\curveto(254.11971976,340.56956501)(254.11471977,340.51956506)(254.10472732,340.46956697)
\curveto(254.03471985,340.16956541)(253.92471996,339.90456568)(253.77472732,339.67456697)
\curveto(253.61472027,339.43456615)(253.41972046,339.23956634)(253.18972732,339.08956697)
\curveto(252.95972092,338.93956664)(252.69972118,338.80956677)(252.40972732,338.69956697)
\curveto(252.29972158,338.64956693)(252.1797217,338.61456697)(252.04972732,338.59456697)
\curveto(251.92972195,338.57456701)(251.80972207,338.54956703)(251.68972732,338.51956697)
\curveto(251.59972228,338.49956708)(251.50472238,338.48956709)(251.40472732,338.48956697)
\curveto(251.31472257,338.4795671)(251.22472266,338.46456712)(251.13472732,338.44456697)
\lineto(250.86472732,338.44456697)
\curveto(250.80472308,338.42456716)(250.69972318,338.41456717)(250.54972732,338.41456697)
\curveto(250.40972347,338.41456717)(250.30972357,338.42456716)(250.24972732,338.44456697)
\curveto(250.21972366,338.44456714)(250.1847237,338.44956713)(250.14472732,338.45956697)
\lineto(250.03972732,338.45956697)
\curveto(249.91972396,338.4795671)(249.79972408,338.49456709)(249.67972732,338.50456697)
\curveto(249.55972432,338.51456707)(249.44472444,338.53456705)(249.33472732,338.56456697)
\curveto(248.94472494,338.67456691)(248.59972528,338.79956678)(248.29972732,338.93956697)
\curveto(247.99972588,339.08956649)(247.74472614,339.30956627)(247.53472732,339.59956697)
\curveto(247.39472649,339.78956579)(247.27472661,340.00956557)(247.17472732,340.25956697)
\curveto(247.15472673,340.31956526)(247.13472675,340.39956518)(247.11472732,340.49956697)
\curveto(247.09472679,340.54956503)(247.0797268,340.61956496)(247.06972732,340.70956697)
\curveto(247.05972682,340.79956478)(247.06472682,340.87456471)(247.08472732,340.93456697)
\curveto(247.11472677,341.00456458)(247.16472672,341.05456453)(247.23472732,341.08456697)
\curveto(247.2847266,341.10456448)(247.34472654,341.11456447)(247.41472732,341.11456697)
\lineto(247.63972732,341.11456697)
\lineto(248.34472732,341.11456697)
\lineto(248.58472732,341.11456697)
\curveto(248.66472522,341.11456447)(248.73472515,341.10456448)(248.79472732,341.08456697)
\curveto(248.90472498,341.04456454)(248.97472491,340.9795646)(249.00472732,340.88956697)
\curveto(249.04472484,340.79956478)(249.08972479,340.70456488)(249.13972732,340.60456697)
\curveto(249.15972472,340.55456503)(249.19472469,340.48956509)(249.24472732,340.40956697)
\curveto(249.30472458,340.32956525)(249.35472453,340.2795653)(249.39472732,340.25956697)
\curveto(249.51472437,340.15956542)(249.62972425,340.0795655)(249.73972732,340.01956697)
\curveto(249.84972403,339.96956561)(249.98972389,339.91956566)(250.15972732,339.86956697)
\curveto(250.20972367,339.84956573)(250.25972362,339.83956574)(250.30972732,339.83956697)
\curveto(250.35972352,339.84956573)(250.40972347,339.84956573)(250.45972732,339.83956697)
\curveto(250.53972334,339.81956576)(250.62472326,339.80956577)(250.71472732,339.80956697)
\curveto(250.81472307,339.81956576)(250.89972298,339.83456575)(250.96972732,339.85456697)
\curveto(251.01972286,339.86456572)(251.06472282,339.86956571)(251.10472732,339.86956697)
\curveto(251.15472273,339.86956571)(251.20472268,339.8795657)(251.25472732,339.89956697)
\curveto(251.39472249,339.94956563)(251.51972236,340.00956557)(251.62972732,340.07956697)
\curveto(251.74972213,340.14956543)(251.84472204,340.23956534)(251.91472732,340.34956697)
\curveto(251.96472192,340.42956515)(252.00472188,340.55456503)(252.03472732,340.72456697)
\curveto(252.05472183,340.79456479)(252.05472183,340.85956472)(252.03472732,340.91956697)
\curveto(252.01472187,340.9795646)(251.99472189,341.02956455)(251.97472732,341.06956697)
\curveto(251.90472198,341.20956437)(251.81472207,341.31456427)(251.70472732,341.38456697)
\curveto(251.60472228,341.45456413)(251.4847224,341.51956406)(251.34472732,341.57956697)
\curveto(251.15472273,341.65956392)(250.95472293,341.72456386)(250.74472732,341.77456697)
\curveto(250.53472335,341.82456376)(250.32472356,341.8795637)(250.11472732,341.93956697)
\curveto(250.03472385,341.95956362)(249.94972393,341.97456361)(249.85972732,341.98456697)
\curveto(249.7797241,341.99456359)(249.69972418,342.00956357)(249.61972732,342.02956697)
\curveto(249.29972458,342.11956346)(248.99472489,342.20456338)(248.70472732,342.28456697)
\curveto(248.41472547,342.37456321)(248.14972573,342.50456308)(247.90972732,342.67456697)
\curveto(247.62972625,342.87456271)(247.42472646,343.14456244)(247.29472732,343.48456697)
\curveto(247.27472661,343.55456203)(247.25472663,343.64956193)(247.23472732,343.76956697)
\curveto(247.21472667,343.83956174)(247.19972668,343.92456166)(247.18972732,344.02456697)
\curveto(247.1797267,344.12456146)(247.1847267,344.21456137)(247.20472732,344.29456697)
\curveto(247.22472666,344.34456124)(247.22972665,344.3845612)(247.21972732,344.41456697)
\curveto(247.20972667,344.45456113)(247.21472667,344.49956108)(247.23472732,344.54956697)
\curveto(247.25472663,344.65956092)(247.27472661,344.75956082)(247.29472732,344.84956697)
\curveto(247.32472656,344.94956063)(247.35972652,345.04456054)(247.39972732,345.13456697)
\curveto(247.52972635,345.42456016)(247.70972617,345.65955992)(247.93972732,345.83956697)
\curveto(248.16972571,346.01955956)(248.42972545,346.16455942)(248.71972732,346.27456697)
\curveto(248.82972505,346.32455926)(248.94472494,346.35955922)(249.06472732,346.37956697)
\curveto(249.1847247,346.40955917)(249.30972457,346.43955914)(249.43972732,346.46956697)
\curveto(249.49972438,346.48955909)(249.55972432,346.49955908)(249.61972732,346.49956697)
\lineto(249.79972732,346.52956697)
\curveto(249.879724,346.53955904)(249.96472392,346.54455904)(250.05472732,346.54456697)
\curveto(250.14472374,346.54455904)(250.22972365,346.54955903)(250.30972732,346.55956697)
}
}
{
\newrgbcolor{curcolor}{0 0 0}
\pscustom[linestyle=none,fillstyle=solid,fillcolor=curcolor]
{
\newpath
\moveto(255.81636795,346.33456697)
\lineto(256.94136795,346.33456697)
\curveto(257.05136551,346.33455925)(257.15136541,346.32955925)(257.24136795,346.31956697)
\curveto(257.33136523,346.30955927)(257.39636517,346.27455931)(257.43636795,346.21456697)
\curveto(257.48636508,346.15455943)(257.51636505,346.06955951)(257.52636795,345.95956697)
\curveto(257.53636503,345.85955972)(257.54136502,345.75455983)(257.54136795,345.64456697)
\lineto(257.54136795,344.59456697)
\lineto(257.54136795,342.35956697)
\curveto(257.54136502,341.99956358)(257.55636501,341.65956392)(257.58636795,341.33956697)
\curveto(257.61636495,341.01956456)(257.70636486,340.75456483)(257.85636795,340.54456697)
\curveto(257.99636457,340.33456525)(258.22136434,340.1845654)(258.53136795,340.09456697)
\curveto(258.58136398,340.0845655)(258.62136394,340.0795655)(258.65136795,340.07956697)
\curveto(258.69136387,340.0795655)(258.73636383,340.07456551)(258.78636795,340.06456697)
\curveto(258.83636373,340.05456553)(258.89136367,340.04956553)(258.95136795,340.04956697)
\curveto(259.01136355,340.04956553)(259.05636351,340.05456553)(259.08636795,340.06456697)
\curveto(259.13636343,340.0845655)(259.17636339,340.08956549)(259.20636795,340.07956697)
\curveto(259.24636332,340.06956551)(259.28636328,340.07456551)(259.32636795,340.09456697)
\curveto(259.53636303,340.14456544)(259.70136286,340.20956537)(259.82136795,340.28956697)
\curveto(260.00136256,340.39956518)(260.14136242,340.53956504)(260.24136795,340.70956697)
\curveto(260.35136221,340.88956469)(260.42636214,341.0845645)(260.46636795,341.29456697)
\curveto(260.51636205,341.51456407)(260.54636202,341.75456383)(260.55636795,342.01456697)
\curveto(260.566362,342.2845633)(260.57136199,342.56456302)(260.57136795,342.85456697)
\lineto(260.57136795,344.66956697)
\lineto(260.57136795,345.64456697)
\lineto(260.57136795,345.91456697)
\curveto(260.57136199,346.01455957)(260.59136197,346.09455949)(260.63136795,346.15456697)
\curveto(260.68136188,346.24455934)(260.75636181,346.29455929)(260.85636795,346.30456697)
\curveto(260.95636161,346.32455926)(261.07636149,346.33455925)(261.21636795,346.33456697)
\lineto(262.01136795,346.33456697)
\lineto(262.29636795,346.33456697)
\curveto(262.38636018,346.33455925)(262.4613601,346.31455927)(262.52136795,346.27456697)
\curveto(262.60135996,346.22455936)(262.64635992,346.14955943)(262.65636795,346.04956697)
\curveto(262.6663599,345.94955963)(262.67135989,345.83455975)(262.67136795,345.70456697)
\lineto(262.67136795,344.56456697)
\lineto(262.67136795,340.34956697)
\lineto(262.67136795,339.28456697)
\lineto(262.67136795,338.98456697)
\curveto(262.67135989,338.8845667)(262.65135991,338.80956677)(262.61136795,338.75956697)
\curveto(262.56136,338.6795669)(262.48636008,338.63456695)(262.38636795,338.62456697)
\curveto(262.28636028,338.61456697)(262.18136038,338.60956697)(262.07136795,338.60956697)
\lineto(261.26136795,338.60956697)
\curveto(261.15136141,338.60956697)(261.05136151,338.61456697)(260.96136795,338.62456697)
\curveto(260.88136168,338.63456695)(260.81636175,338.67456691)(260.76636795,338.74456697)
\curveto(260.74636182,338.77456681)(260.72636184,338.81956676)(260.70636795,338.87956697)
\curveto(260.69636187,338.93956664)(260.68136188,338.99956658)(260.66136795,339.05956697)
\curveto(260.65136191,339.11956646)(260.63636193,339.17456641)(260.61636795,339.22456697)
\curveto(260.59636197,339.27456631)(260.566362,339.30456628)(260.52636795,339.31456697)
\curveto(260.50636206,339.33456625)(260.48136208,339.33956624)(260.45136795,339.32956697)
\curveto(260.42136214,339.31956626)(260.39636217,339.30956627)(260.37636795,339.29956697)
\curveto(260.30636226,339.25956632)(260.24636232,339.21456637)(260.19636795,339.16456697)
\curveto(260.14636242,339.11456647)(260.09136247,339.06956651)(260.03136795,339.02956697)
\curveto(259.99136257,338.99956658)(259.95136261,338.96456662)(259.91136795,338.92456697)
\curveto(259.88136268,338.89456669)(259.84136272,338.86456672)(259.79136795,338.83456697)
\curveto(259.561363,338.69456689)(259.29136327,338.584567)(258.98136795,338.50456697)
\curveto(258.91136365,338.4845671)(258.84136372,338.47456711)(258.77136795,338.47456697)
\curveto(258.70136386,338.46456712)(258.62636394,338.44956713)(258.54636795,338.42956697)
\curveto(258.50636406,338.41956716)(258.4613641,338.41956716)(258.41136795,338.42956697)
\curveto(258.37136419,338.42956715)(258.33136423,338.42456716)(258.29136795,338.41456697)
\curveto(258.2613643,338.40456718)(258.19636437,338.40456718)(258.09636795,338.41456697)
\curveto(258.00636456,338.41456717)(257.94636462,338.41956716)(257.91636795,338.42956697)
\curveto(257.8663647,338.42956715)(257.81636475,338.43456715)(257.76636795,338.44456697)
\lineto(257.61636795,338.44456697)
\curveto(257.49636507,338.47456711)(257.38136518,338.49956708)(257.27136795,338.51956697)
\curveto(257.1613654,338.53956704)(257.05136551,338.56956701)(256.94136795,338.60956697)
\curveto(256.89136567,338.62956695)(256.84636572,338.64456694)(256.80636795,338.65456697)
\curveto(256.77636579,338.67456691)(256.73636583,338.69456689)(256.68636795,338.71456697)
\curveto(256.33636623,338.90456668)(256.05636651,339.16956641)(255.84636795,339.50956697)
\curveto(255.71636685,339.71956586)(255.62136694,339.96956561)(255.56136795,340.25956697)
\curveto(255.50136706,340.55956502)(255.4613671,340.87456471)(255.44136795,341.20456697)
\curveto(255.43136713,341.54456404)(255.42636714,341.88956369)(255.42636795,342.23956697)
\curveto(255.43636713,342.59956298)(255.44136712,342.95456263)(255.44136795,343.30456697)
\lineto(255.44136795,345.34456697)
\curveto(255.44136712,345.47456011)(255.43636713,345.62455996)(255.42636795,345.79456697)
\curveto(255.42636714,345.97455961)(255.45136711,346.10455948)(255.50136795,346.18456697)
\curveto(255.53136703,346.23455935)(255.59136697,346.2795593)(255.68136795,346.31956697)
\curveto(255.74136682,346.31955926)(255.78636678,346.32455926)(255.81636795,346.33456697)
}
}
{
\newrgbcolor{curcolor}{0 0 0}
\pscustom[linestyle=none,fillstyle=solid,fillcolor=curcolor]
{
}
}
{
\newrgbcolor{curcolor}{0 0 0}
\pscustom[linestyle=none,fillstyle=solid,fillcolor=curcolor]
{
\newpath
\moveto(269.5127742,348.65956697)
\lineto(270.5177742,348.65956697)
\curveto(270.66777121,348.65955692)(270.79777108,348.64955693)(270.9077742,348.62956697)
\curveto(271.02777085,348.61955696)(271.11277077,348.55955702)(271.1627742,348.44956697)
\curveto(271.1827707,348.39955718)(271.19277069,348.33955724)(271.1927742,348.26956697)
\lineto(271.1927742,348.05956697)
\lineto(271.1927742,347.38456697)
\curveto(271.19277069,347.33455825)(271.18777069,347.27455831)(271.1777742,347.20456697)
\curveto(271.1777707,347.14455844)(271.1827707,347.08955849)(271.1927742,347.03956697)
\lineto(271.1927742,346.87456697)
\curveto(271.19277069,346.79455879)(271.19777068,346.71955886)(271.2077742,346.64956697)
\curveto(271.21777066,346.58955899)(271.24277064,346.53455905)(271.2827742,346.48456697)
\curveto(271.35277053,346.39455919)(271.4777704,346.34455924)(271.6577742,346.33456697)
\lineto(272.1977742,346.33456697)
\lineto(272.3777742,346.33456697)
\curveto(272.43776944,346.33455925)(272.49276939,346.32455926)(272.5427742,346.30456697)
\curveto(272.65276923,346.25455933)(272.71276917,346.16455942)(272.7227742,346.03456697)
\curveto(272.74276914,345.90455968)(272.75276913,345.75955982)(272.7527742,345.59956697)
\lineto(272.7527742,345.38956697)
\curveto(272.76276912,345.31956026)(272.75776912,345.25956032)(272.7377742,345.20956697)
\curveto(272.68776919,345.04956053)(272.5827693,344.96456062)(272.4227742,344.95456697)
\curveto(272.26276962,344.94456064)(272.0827698,344.93956064)(271.8827742,344.93956697)
\lineto(271.7477742,344.93956697)
\curveto(271.70777017,344.94956063)(271.67277021,344.94956063)(271.6427742,344.93956697)
\curveto(271.60277028,344.92956065)(271.56777031,344.92456066)(271.5377742,344.92456697)
\curveto(271.50777037,344.93456065)(271.4777704,344.92956065)(271.4477742,344.90956697)
\curveto(271.36777051,344.88956069)(271.30777057,344.84456074)(271.2677742,344.77456697)
\curveto(271.23777064,344.71456087)(271.21277067,344.63956094)(271.1927742,344.54956697)
\curveto(271.1827707,344.49956108)(271.1827707,344.44456114)(271.1927742,344.38456697)
\curveto(271.20277068,344.32456126)(271.20277068,344.26956131)(271.1927742,344.21956697)
\lineto(271.1927742,343.28956697)
\lineto(271.1927742,341.53456697)
\curveto(271.19277069,341.2845643)(271.19777068,341.06456452)(271.2077742,340.87456697)
\curveto(271.22777065,340.69456489)(271.29277059,340.53456505)(271.4027742,340.39456697)
\curveto(271.45277043,340.33456525)(271.51777036,340.28956529)(271.5977742,340.25956697)
\lineto(271.8677742,340.19956697)
\curveto(271.89776998,340.18956539)(271.92776995,340.1845654)(271.9577742,340.18456697)
\curveto(271.99776988,340.19456539)(272.02776985,340.19456539)(272.0477742,340.18456697)
\lineto(272.2127742,340.18456697)
\curveto(272.32276956,340.1845654)(272.41776946,340.1795654)(272.4977742,340.16956697)
\curveto(272.5777693,340.15956542)(272.64276924,340.11956546)(272.6927742,340.04956697)
\curveto(272.73276915,339.98956559)(272.75276913,339.90956567)(272.7527742,339.80956697)
\lineto(272.7527742,339.52456697)
\curveto(272.75276913,339.31456627)(272.74776913,339.11956646)(272.7377742,338.93956697)
\curveto(272.73776914,338.76956681)(272.65776922,338.65456693)(272.4977742,338.59456697)
\curveto(272.44776943,338.57456701)(272.40276948,338.56956701)(272.3627742,338.57956697)
\curveto(272.32276956,338.579567)(272.2777696,338.56956701)(272.2277742,338.54956697)
\lineto(272.0777742,338.54956697)
\curveto(272.05776982,338.54956703)(272.02776985,338.55456703)(271.9877742,338.56456697)
\curveto(271.94776993,338.56456702)(271.91276997,338.55956702)(271.8827742,338.54956697)
\curveto(271.83277005,338.53956704)(271.7777701,338.53956704)(271.7177742,338.54956697)
\lineto(271.5677742,338.54956697)
\lineto(271.4177742,338.54956697)
\curveto(271.36777051,338.53956704)(271.32277056,338.53956704)(271.2827742,338.54956697)
\lineto(271.1177742,338.54956697)
\curveto(271.06777081,338.55956702)(271.01277087,338.56456702)(270.9527742,338.56456697)
\curveto(270.89277099,338.56456702)(270.83777104,338.56956701)(270.7877742,338.57956697)
\curveto(270.71777116,338.58956699)(270.65277123,338.59956698)(270.5927742,338.60956697)
\lineto(270.4127742,338.63956697)
\curveto(270.30277158,338.66956691)(270.19777168,338.70456688)(270.0977742,338.74456697)
\curveto(269.99777188,338.7845668)(269.90277198,338.82956675)(269.8127742,338.87956697)
\lineto(269.7227742,338.93956697)
\curveto(269.69277219,338.96956661)(269.65777222,338.99956658)(269.6177742,339.02956697)
\curveto(269.59777228,339.04956653)(269.57277231,339.06956651)(269.5427742,339.08956697)
\lineto(269.4677742,339.16456697)
\curveto(269.32777255,339.35456623)(269.22277266,339.56456602)(269.1527742,339.79456697)
\curveto(269.13277275,339.83456575)(269.12277276,339.86956571)(269.1227742,339.89956697)
\curveto(269.13277275,339.93956564)(269.13277275,339.9845656)(269.1227742,340.03456697)
\curveto(269.11277277,340.05456553)(269.10777277,340.0795655)(269.1077742,340.10956697)
\curveto(269.10777277,340.13956544)(269.10277278,340.16456542)(269.0927742,340.18456697)
\lineto(269.0927742,340.33456697)
\curveto(269.0827728,340.37456521)(269.0777728,340.41956516)(269.0777742,340.46956697)
\curveto(269.08777279,340.51956506)(269.09277279,340.56956501)(269.0927742,340.61956697)
\lineto(269.0927742,341.18956697)
\lineto(269.0927742,343.42456697)
\lineto(269.0927742,344.21956697)
\lineto(269.0927742,344.42956697)
\curveto(269.10277278,344.49956108)(269.09777278,344.56456102)(269.0777742,344.62456697)
\curveto(269.03777284,344.76456082)(268.96777291,344.85456073)(268.8677742,344.89456697)
\curveto(268.75777312,344.94456064)(268.61777326,344.95956062)(268.4477742,344.93956697)
\curveto(268.2777736,344.91956066)(268.13277375,344.93456065)(268.0127742,344.98456697)
\curveto(267.93277395,345.01456057)(267.882774,345.05956052)(267.8627742,345.11956697)
\curveto(267.84277404,345.1795604)(267.82277406,345.25456033)(267.8027742,345.34456697)
\lineto(267.8027742,345.65956697)
\curveto(267.80277408,345.83955974)(267.81277407,345.9845596)(267.8327742,346.09456697)
\curveto(267.85277403,346.20455938)(267.93777394,346.2795593)(268.0877742,346.31956697)
\curveto(268.12777375,346.33955924)(268.16777371,346.34455924)(268.2077742,346.33456697)
\lineto(268.3427742,346.33456697)
\curveto(268.49277339,346.33455925)(268.63277325,346.33955924)(268.7627742,346.34956697)
\curveto(268.89277299,346.36955921)(268.9827729,346.42955915)(269.0327742,346.52956697)
\curveto(269.06277282,346.59955898)(269.0777728,346.6795589)(269.0777742,346.76956697)
\curveto(269.08777279,346.85955872)(269.09277279,346.94955863)(269.0927742,347.03956697)
\lineto(269.0927742,347.96956697)
\lineto(269.0927742,348.22456697)
\curveto(269.09277279,348.31455727)(269.10277278,348.38955719)(269.1227742,348.44956697)
\curveto(269.17277271,348.54955703)(269.24777263,348.61455697)(269.3477742,348.64456697)
\curveto(269.36777251,348.65455693)(269.39277249,348.65455693)(269.4227742,348.64456697)
\curveto(269.46277242,348.64455694)(269.49277239,348.64955693)(269.5127742,348.65956697)
}
}
{
\newrgbcolor{curcolor}{0 0 0}
\pscustom[linestyle=none,fillstyle=solid,fillcolor=curcolor]
{
\newpath
\moveto(275.8362117,349.19956697)
\curveto(275.90620875,349.11955646)(275.94120871,348.99955658)(275.9412117,348.83956697)
\lineto(275.9412117,348.37456697)
\lineto(275.9412117,347.96956697)
\curveto(275.94120871,347.82955775)(275.90620875,347.73455785)(275.8362117,347.68456697)
\curveto(275.77620888,347.63455795)(275.69620896,347.60455798)(275.5962117,347.59456697)
\curveto(275.50620915,347.584558)(275.40620925,347.579558)(275.2962117,347.57956697)
\lineto(274.4562117,347.57956697)
\curveto(274.34621031,347.579558)(274.24621041,347.584558)(274.1562117,347.59456697)
\curveto(274.07621058,347.60455798)(274.00621065,347.63455795)(273.9462117,347.68456697)
\curveto(273.90621075,347.71455787)(273.87621078,347.76955781)(273.8562117,347.84956697)
\curveto(273.84621081,347.93955764)(273.83621082,348.03455755)(273.8262117,348.13456697)
\lineto(273.8262117,348.46456697)
\curveto(273.83621082,348.57455701)(273.84121081,348.66955691)(273.8412117,348.74956697)
\lineto(273.8412117,348.95956697)
\curveto(273.8512108,349.02955655)(273.87121078,349.08955649)(273.9012117,349.13956697)
\curveto(273.92121073,349.1795564)(273.94621071,349.20955637)(273.9762117,349.22956697)
\lineto(274.0962117,349.28956697)
\curveto(274.11621054,349.28955629)(274.14121051,349.28955629)(274.1712117,349.28956697)
\curveto(274.20121045,349.29955628)(274.22621043,349.30455628)(274.2462117,349.30456697)
\lineto(275.3412117,349.30456697)
\curveto(275.44120921,349.30455628)(275.53620912,349.29955628)(275.6262117,349.28956697)
\curveto(275.71620894,349.2795563)(275.78620887,349.24955633)(275.8362117,349.19956697)
\moveto(275.9412117,339.43456697)
\curveto(275.94120871,339.23456635)(275.93620872,339.06456652)(275.9262117,338.92456697)
\curveto(275.91620874,338.7845668)(275.82620883,338.68956689)(275.6562117,338.63956697)
\curveto(275.59620906,338.61956696)(275.53120912,338.60956697)(275.4612117,338.60956697)
\curveto(275.39120926,338.61956696)(275.31620934,338.62456696)(275.2362117,338.62456697)
\lineto(274.3962117,338.62456697)
\curveto(274.30621035,338.62456696)(274.21621044,338.62956695)(274.1262117,338.63956697)
\curveto(274.04621061,338.64956693)(273.98621067,338.6795669)(273.9462117,338.72956697)
\curveto(273.88621077,338.79956678)(273.8512108,338.8845667)(273.8412117,338.98456697)
\lineto(273.8412117,339.32956697)
\lineto(273.8412117,345.65956697)
\lineto(273.8412117,345.95956697)
\curveto(273.84121081,346.05955952)(273.86121079,346.13955944)(273.9012117,346.19956697)
\curveto(273.96121069,346.26955931)(274.04621061,346.31455927)(274.1562117,346.33456697)
\curveto(274.17621048,346.34455924)(274.20121045,346.34455924)(274.2312117,346.33456697)
\curveto(274.27121038,346.33455925)(274.30121035,346.33955924)(274.3212117,346.34956697)
\lineto(275.0712117,346.34956697)
\lineto(275.2662117,346.34956697)
\curveto(275.34620931,346.35955922)(275.41120924,346.35955922)(275.4612117,346.34956697)
\lineto(275.5812117,346.34956697)
\curveto(275.64120901,346.32955925)(275.69620896,346.31455927)(275.7462117,346.30456697)
\curveto(275.79620886,346.29455929)(275.83620882,346.26455932)(275.8662117,346.21456697)
\curveto(275.90620875,346.16455942)(275.92620873,346.09455949)(275.9262117,346.00456697)
\curveto(275.93620872,345.91455967)(275.94120871,345.81955976)(275.9412117,345.71956697)
\lineto(275.9412117,339.43456697)
}
}
{
\newrgbcolor{curcolor}{0 0 0}
\pscustom[linestyle=none,fillstyle=solid,fillcolor=curcolor]
{
\newpath
\moveto(285.4933992,342.56956697)
\curveto(285.50339052,342.50956307)(285.50839051,342.41956316)(285.5083992,342.29956697)
\curveto(285.50839051,342.1795634)(285.49839052,342.09456349)(285.4783992,342.04456697)
\lineto(285.4783992,341.84956697)
\curveto(285.44839057,341.73956384)(285.42839059,341.63456395)(285.4183992,341.53456697)
\curveto(285.4183906,341.43456415)(285.40339062,341.33456425)(285.3733992,341.23456697)
\curveto(285.35339067,341.14456444)(285.33339069,341.04956453)(285.3133992,340.94956697)
\curveto(285.29339073,340.85956472)(285.26339076,340.76956481)(285.2233992,340.67956697)
\curveto(285.15339087,340.50956507)(285.08339094,340.34956523)(285.0133992,340.19956697)
\curveto(284.94339108,340.05956552)(284.86339116,339.91956566)(284.7733992,339.77956697)
\curveto(284.71339131,339.68956589)(284.64839137,339.60456598)(284.5783992,339.52456697)
\curveto(284.5183915,339.45456613)(284.44839157,339.3795662)(284.3683992,339.29956697)
\lineto(284.2633992,339.19456697)
\curveto(284.21339181,339.14456644)(284.15839186,339.09956648)(284.0983992,339.05956697)
\lineto(283.9483992,338.93956697)
\curveto(283.86839215,338.8795667)(283.77839224,338.82456676)(283.6783992,338.77456697)
\curveto(283.58839243,338.73456685)(283.49339253,338.68956689)(283.3933992,338.63956697)
\curveto(283.29339273,338.58956699)(283.18839283,338.55456703)(283.0783992,338.53456697)
\curveto(282.97839304,338.51456707)(282.87339315,338.49456709)(282.7633992,338.47456697)
\curveto(282.70339332,338.45456713)(282.63839338,338.44456714)(282.5683992,338.44456697)
\curveto(282.50839351,338.44456714)(282.44339358,338.43456715)(282.3733992,338.41456697)
\lineto(282.2383992,338.41456697)
\curveto(282.15839386,338.39456719)(282.08339394,338.39456719)(282.0133992,338.41456697)
\lineto(281.8633992,338.41456697)
\curveto(281.80339422,338.43456715)(281.73839428,338.44456714)(281.6683992,338.44456697)
\curveto(281.60839441,338.43456715)(281.54839447,338.43956714)(281.4883992,338.45956697)
\curveto(281.32839469,338.50956707)(281.17339485,338.55456703)(281.0233992,338.59456697)
\curveto(280.88339514,338.63456695)(280.75339527,338.69456689)(280.6333992,338.77456697)
\curveto(280.56339546,338.81456677)(280.49839552,338.85456673)(280.4383992,338.89456697)
\curveto(280.37839564,338.94456664)(280.31339571,338.99456659)(280.2433992,339.04456697)
\lineto(280.0633992,339.17956697)
\curveto(279.98339604,339.23956634)(279.91339611,339.24456634)(279.8533992,339.19456697)
\curveto(279.80339622,339.16456642)(279.77839624,339.12456646)(279.7783992,339.07456697)
\curveto(279.77839624,339.03456655)(279.76839625,338.9845666)(279.7483992,338.92456697)
\curveto(279.72839629,338.82456676)(279.7183963,338.70956687)(279.7183992,338.57956697)
\curveto(279.72839629,338.44956713)(279.73339629,338.32956725)(279.7333992,338.21956697)
\lineto(279.7333992,336.68956697)
\curveto(279.73339629,336.55956902)(279.72839629,336.43456915)(279.7183992,336.31456697)
\curveto(279.7183963,336.1845694)(279.69339633,336.0795695)(279.6433992,335.99956697)
\curveto(279.61339641,335.95956962)(279.55839646,335.92956965)(279.4783992,335.90956697)
\curveto(279.39839662,335.88956969)(279.30839671,335.8795697)(279.2083992,335.87956697)
\curveto(279.10839691,335.86956971)(279.00839701,335.86956971)(278.9083992,335.87956697)
\lineto(278.6533992,335.87956697)
\lineto(278.2483992,335.87956697)
\lineto(278.1433992,335.87956697)
\curveto(278.10339792,335.8795697)(278.06839795,335.8845697)(278.0383992,335.89456697)
\lineto(277.9183992,335.89456697)
\curveto(277.74839827,335.94456964)(277.65839836,336.04456954)(277.6483992,336.19456697)
\curveto(277.63839838,336.33456925)(277.63339839,336.50456908)(277.6333992,336.70456697)
\lineto(277.6333992,345.50956697)
\curveto(277.63339839,345.61955996)(277.62839839,345.73455985)(277.6183992,345.85456697)
\curveto(277.6183984,345.9845596)(277.64339838,346.0845595)(277.6933992,346.15456697)
\curveto(277.73339829,346.22455936)(277.78839823,346.26955931)(277.8583992,346.28956697)
\curveto(277.90839811,346.30955927)(277.96839805,346.31955926)(278.0383992,346.31956697)
\lineto(278.2633992,346.31956697)
\lineto(278.9833992,346.31956697)
\lineto(279.2683992,346.31956697)
\curveto(279.35839666,346.31955926)(279.43339659,346.29455929)(279.4933992,346.24456697)
\curveto(279.56339646,346.19455939)(279.59839642,346.12955945)(279.5983992,346.04956697)
\curveto(279.60839641,345.9795596)(279.63339639,345.90455968)(279.6733992,345.82456697)
\curveto(279.68339634,345.79455979)(279.69339633,345.76955981)(279.7033992,345.74956697)
\curveto(279.7233963,345.73955984)(279.74339628,345.72455986)(279.7633992,345.70456697)
\curveto(279.87339615,345.69455989)(279.96339606,345.72455986)(280.0333992,345.79456697)
\curveto(280.10339592,345.86455972)(280.17339585,345.92455966)(280.2433992,345.97456697)
\curveto(280.37339565,346.06455952)(280.50839551,346.14455944)(280.6483992,346.21456697)
\curveto(280.78839523,346.29455929)(280.94339508,346.35955922)(281.1133992,346.40956697)
\curveto(281.19339483,346.43955914)(281.27839474,346.45955912)(281.3683992,346.46956697)
\curveto(281.46839455,346.4795591)(281.56339446,346.49455909)(281.6533992,346.51456697)
\curveto(281.69339433,346.52455906)(281.73339429,346.52455906)(281.7733992,346.51456697)
\curveto(281.8233942,346.50455908)(281.86339416,346.50955907)(281.8933992,346.52956697)
\curveto(282.46339356,346.54955903)(282.94339308,346.46955911)(283.3333992,346.28956697)
\curveto(283.73339229,346.11955946)(284.07339195,345.89455969)(284.3533992,345.61456697)
\curveto(284.40339162,345.56456002)(284.44839157,345.51456007)(284.4883992,345.46456697)
\curveto(284.52839149,345.42456016)(284.56839145,345.3795602)(284.6083992,345.32956697)
\curveto(284.67839134,345.23956034)(284.73839128,345.14956043)(284.7883992,345.05956697)
\curveto(284.84839117,344.96956061)(284.90339112,344.8795607)(284.9533992,344.78956697)
\curveto(284.97339105,344.76956081)(284.98339104,344.74456084)(284.9833992,344.71456697)
\curveto(284.99339103,344.6845609)(285.00839101,344.64956093)(285.0283992,344.60956697)
\curveto(285.08839093,344.50956107)(285.14339088,344.38956119)(285.1933992,344.24956697)
\curveto(285.21339081,344.18956139)(285.23339079,344.12456146)(285.2533992,344.05456697)
\curveto(285.27339075,343.99456159)(285.29339073,343.92956165)(285.3133992,343.85956697)
\curveto(285.35339067,343.73956184)(285.37839064,343.61456197)(285.3883992,343.48456697)
\curveto(285.40839061,343.35456223)(285.43339059,343.21956236)(285.4633992,343.07956697)
\lineto(285.4633992,342.91456697)
\lineto(285.4933992,342.73456697)
\lineto(285.4933992,342.56956697)
\moveto(283.3783992,342.22456697)
\curveto(283.38839263,342.27456331)(283.39339263,342.33956324)(283.3933992,342.41956697)
\curveto(283.39339263,342.50956307)(283.38839263,342.579563)(283.3783992,342.62956697)
\lineto(283.3783992,342.76456697)
\curveto(283.35839266,342.82456276)(283.34839267,342.88956269)(283.3483992,342.95956697)
\curveto(283.34839267,343.02956255)(283.33839268,343.09956248)(283.3183992,343.16956697)
\curveto(283.29839272,343.26956231)(283.27839274,343.36456222)(283.2583992,343.45456697)
\curveto(283.23839278,343.55456203)(283.20839281,343.64456194)(283.1683992,343.72456697)
\curveto(283.04839297,344.04456154)(282.89339313,344.29956128)(282.7033992,344.48956697)
\curveto(282.51339351,344.6795609)(282.24339378,344.81956076)(281.8933992,344.90956697)
\curveto(281.81339421,344.92956065)(281.7233943,344.93956064)(281.6233992,344.93956697)
\lineto(281.3533992,344.93956697)
\curveto(281.31339471,344.92956065)(281.27839474,344.92456066)(281.2483992,344.92456697)
\curveto(281.2183948,344.92456066)(281.18339484,344.91956066)(281.1433992,344.90956697)
\lineto(280.9333992,344.84956697)
\curveto(280.87339515,344.83956074)(280.81339521,344.81956076)(280.7533992,344.78956697)
\curveto(280.49339553,344.6795609)(280.28839573,344.50956107)(280.1383992,344.27956697)
\curveto(279.99839602,344.04956153)(279.88339614,343.79456179)(279.7933992,343.51456697)
\curveto(279.77339625,343.43456215)(279.75839626,343.34956223)(279.7483992,343.25956697)
\curveto(279.73839628,343.1795624)(279.7233963,343.09956248)(279.7033992,343.01956697)
\curveto(279.69339633,342.9795626)(279.68839633,342.91456267)(279.6883992,342.82456697)
\curveto(279.66839635,342.7845628)(279.66339636,342.73456285)(279.6733992,342.67456697)
\curveto(279.68339634,342.62456296)(279.68339634,342.57456301)(279.6733992,342.52456697)
\curveto(279.65339637,342.46456312)(279.65339637,342.40956317)(279.6733992,342.35956697)
\lineto(279.6733992,342.17956697)
\lineto(279.6733992,342.04456697)
\curveto(279.67339635,342.00456358)(279.68339634,341.96456362)(279.7033992,341.92456697)
\curveto(279.70339632,341.85456373)(279.70839631,341.79956378)(279.7183992,341.75956697)
\lineto(279.7483992,341.57956697)
\curveto(279.75839626,341.51956406)(279.77339625,341.45956412)(279.7933992,341.39956697)
\curveto(279.88339614,341.10956447)(279.98839603,340.86956471)(280.1083992,340.67956697)
\curveto(280.23839578,340.49956508)(280.4183956,340.33956524)(280.6483992,340.19956697)
\curveto(280.78839523,340.11956546)(280.95339507,340.05456553)(281.1433992,340.00456697)
\curveto(281.18339484,339.99456559)(281.2183948,339.98956559)(281.2483992,339.98956697)
\curveto(281.27839474,339.99956558)(281.31339471,339.99956558)(281.3533992,339.98956697)
\curveto(281.39339463,339.9795656)(281.45339457,339.96956561)(281.5333992,339.95956697)
\curveto(281.61339441,339.95956562)(281.67839434,339.96456562)(281.7283992,339.97456697)
\curveto(281.80839421,339.99456559)(281.88839413,340.00956557)(281.9683992,340.01956697)
\curveto(282.05839396,340.03956554)(282.14339388,340.06456552)(282.2233992,340.09456697)
\curveto(282.46339356,340.19456539)(282.65839336,340.33456525)(282.8083992,340.51456697)
\curveto(282.95839306,340.69456489)(283.08339294,340.90456468)(283.1833992,341.14456697)
\curveto(283.23339279,341.26456432)(283.26839275,341.38956419)(283.2883992,341.51956697)
\curveto(283.30839271,341.64956393)(283.33339269,341.7845638)(283.3633992,341.92456697)
\lineto(283.3633992,342.07456697)
\curveto(283.37339265,342.12456346)(283.37839264,342.17456341)(283.3783992,342.22456697)
}
}
{
\newrgbcolor{curcolor}{0 0 0}
\pscustom[linestyle=none,fillstyle=solid,fillcolor=curcolor]
{
\newpath
\moveto(294.54332107,342.79456697)
\curveto(294.5633125,342.73456285)(294.57331249,342.64956293)(294.57332107,342.53956697)
\curveto(294.57331249,342.42956315)(294.5633125,342.34456324)(294.54332107,342.28456697)
\lineto(294.54332107,342.13456697)
\curveto(294.52331254,342.05456353)(294.51331255,341.97456361)(294.51332107,341.89456697)
\curveto(294.52331254,341.81456377)(294.51831255,341.73456385)(294.49832107,341.65456697)
\curveto(294.47831259,341.584564)(294.4633126,341.51956406)(294.45332107,341.45956697)
\curveto(294.44331262,341.39956418)(294.43331263,341.33456425)(294.42332107,341.26456697)
\curveto(294.38331268,341.15456443)(294.34831272,341.03956454)(294.31832107,340.91956697)
\curveto(294.28831278,340.80956477)(294.24831282,340.70456488)(294.19832107,340.60456697)
\curveto(293.98831308,340.12456546)(293.71331335,339.73456585)(293.37332107,339.43456697)
\curveto(293.03331403,339.13456645)(292.62331444,338.8845667)(292.14332107,338.68456697)
\curveto(292.02331504,338.63456695)(291.89831517,338.59956698)(291.76832107,338.57956697)
\curveto(291.64831542,338.54956703)(291.52331554,338.51956706)(291.39332107,338.48956697)
\curveto(291.34331572,338.46956711)(291.28831578,338.45956712)(291.22832107,338.45956697)
\curveto(291.1683159,338.45956712)(291.11331595,338.45456713)(291.06332107,338.44456697)
\lineto(290.95832107,338.44456697)
\curveto(290.92831614,338.43456715)(290.89831617,338.42956715)(290.86832107,338.42956697)
\curveto(290.81831625,338.41956716)(290.73831633,338.41456717)(290.62832107,338.41456697)
\curveto(290.51831655,338.40456718)(290.43331663,338.40956717)(290.37332107,338.42956697)
\lineto(290.22332107,338.42956697)
\curveto(290.17331689,338.43956714)(290.11831695,338.44456714)(290.05832107,338.44456697)
\curveto(290.00831706,338.43456715)(289.95831711,338.43956714)(289.90832107,338.45956697)
\curveto(289.8683172,338.46956711)(289.82831724,338.47456711)(289.78832107,338.47456697)
\curveto(289.75831731,338.47456711)(289.71831735,338.4795671)(289.66832107,338.48956697)
\curveto(289.5683175,338.51956706)(289.4683176,338.54456704)(289.36832107,338.56456697)
\curveto(289.2683178,338.584567)(289.17331789,338.61456697)(289.08332107,338.65456697)
\curveto(288.9633181,338.69456689)(288.84831822,338.73456685)(288.73832107,338.77456697)
\curveto(288.63831843,338.81456677)(288.53331853,338.86456672)(288.42332107,338.92456697)
\curveto(288.07331899,339.13456645)(287.77331929,339.3795662)(287.52332107,339.65956697)
\curveto(287.27331979,339.93956564)(287.06332,340.27456531)(286.89332107,340.66456697)
\curveto(286.84332022,340.75456483)(286.80332026,340.84956473)(286.77332107,340.94956697)
\curveto(286.75332031,341.04956453)(286.72832034,341.15456443)(286.69832107,341.26456697)
\curveto(286.67832039,341.31456427)(286.6683204,341.35956422)(286.66832107,341.39956697)
\curveto(286.6683204,341.43956414)(286.65832041,341.4845641)(286.63832107,341.53456697)
\curveto(286.61832045,341.61456397)(286.60832046,341.69456389)(286.60832107,341.77456697)
\curveto(286.60832046,341.86456372)(286.59832047,341.94956363)(286.57832107,342.02956697)
\curveto(286.5683205,342.0795635)(286.5633205,342.12456346)(286.56332107,342.16456697)
\lineto(286.56332107,342.29956697)
\curveto(286.54332052,342.35956322)(286.53332053,342.44456314)(286.53332107,342.55456697)
\curveto(286.54332052,342.66456292)(286.55832051,342.74956283)(286.57832107,342.80956697)
\lineto(286.57832107,342.91456697)
\curveto(286.58832048,342.96456262)(286.58832048,343.01456257)(286.57832107,343.06456697)
\curveto(286.57832049,343.12456246)(286.58832048,343.1795624)(286.60832107,343.22956697)
\curveto(286.61832045,343.2795623)(286.62332044,343.32456226)(286.62332107,343.36456697)
\curveto(286.62332044,343.41456217)(286.63332043,343.46456212)(286.65332107,343.51456697)
\curveto(286.69332037,343.64456194)(286.72832034,343.76956181)(286.75832107,343.88956697)
\curveto(286.78832028,344.01956156)(286.82832024,344.14456144)(286.87832107,344.26456697)
\curveto(287.05832001,344.67456091)(287.27331979,345.01456057)(287.52332107,345.28456697)
\curveto(287.77331929,345.56456002)(288.07831899,345.81955976)(288.43832107,346.04956697)
\curveto(288.53831853,346.09955948)(288.64331842,346.14455944)(288.75332107,346.18456697)
\curveto(288.8633182,346.22455936)(288.97331809,346.26955931)(289.08332107,346.31956697)
\curveto(289.21331785,346.36955921)(289.34831772,346.40455918)(289.48832107,346.42456697)
\curveto(289.62831744,346.44455914)(289.77331729,346.47455911)(289.92332107,346.51456697)
\curveto(290.00331706,346.52455906)(290.07831699,346.52955905)(290.14832107,346.52956697)
\curveto(290.21831685,346.52955905)(290.28831678,346.53455905)(290.35832107,346.54456697)
\curveto(290.93831613,346.55455903)(291.43831563,346.49455909)(291.85832107,346.36456697)
\curveto(292.28831478,346.23455935)(292.6683144,346.05455953)(292.99832107,345.82456697)
\curveto(293.10831396,345.74455984)(293.21831385,345.65455993)(293.32832107,345.55456697)
\curveto(293.44831362,345.46456012)(293.54831352,345.36456022)(293.62832107,345.25456697)
\curveto(293.70831336,345.15456043)(293.77831329,345.05456053)(293.83832107,344.95456697)
\curveto(293.90831316,344.85456073)(293.97831309,344.74956083)(294.04832107,344.63956697)
\curveto(294.11831295,344.52956105)(294.17331289,344.40956117)(294.21332107,344.27956697)
\curveto(294.25331281,344.15956142)(294.29831277,344.02956155)(294.34832107,343.88956697)
\curveto(294.37831269,343.80956177)(294.40331266,343.72456186)(294.42332107,343.63456697)
\lineto(294.48332107,343.36456697)
\curveto(294.49331257,343.32456226)(294.49831257,343.2845623)(294.49832107,343.24456697)
\curveto(294.49831257,343.20456238)(294.50331256,343.16456242)(294.51332107,343.12456697)
\curveto(294.53331253,343.07456251)(294.53831253,343.01956256)(294.52832107,342.95956697)
\curveto(294.51831255,342.89956268)(294.52331254,342.84456274)(294.54332107,342.79456697)
\moveto(292.44332107,342.25456697)
\curveto(292.45331461,342.30456328)(292.45831461,342.37456321)(292.45832107,342.46456697)
\curveto(292.45831461,342.56456302)(292.45331461,342.63956294)(292.44332107,342.68956697)
\lineto(292.44332107,342.80956697)
\curveto(292.42331464,342.85956272)(292.41331465,342.91456267)(292.41332107,342.97456697)
\curveto(292.41331465,343.03456255)(292.40831466,343.08956249)(292.39832107,343.13956697)
\curveto(292.39831467,343.1795624)(292.39331467,343.20956237)(292.38332107,343.22956697)
\lineto(292.32332107,343.46956697)
\curveto(292.31331475,343.55956202)(292.29331477,343.64456194)(292.26332107,343.72456697)
\curveto(292.15331491,343.9845616)(292.02331504,344.20456138)(291.87332107,344.38456697)
\curveto(291.72331534,344.57456101)(291.52331554,344.72456086)(291.27332107,344.83456697)
\curveto(291.21331585,344.85456073)(291.15331591,344.86956071)(291.09332107,344.87956697)
\curveto(291.03331603,344.89956068)(290.9683161,344.91956066)(290.89832107,344.93956697)
\curveto(290.81831625,344.95956062)(290.73331633,344.96456062)(290.64332107,344.95456697)
\lineto(290.37332107,344.95456697)
\curveto(290.34331672,344.93456065)(290.30831676,344.92456066)(290.26832107,344.92456697)
\curveto(290.22831684,344.93456065)(290.19331687,344.93456065)(290.16332107,344.92456697)
\lineto(289.95332107,344.86456697)
\curveto(289.89331717,344.85456073)(289.83831723,344.83456075)(289.78832107,344.80456697)
\curveto(289.53831753,344.69456089)(289.33331773,344.53456105)(289.17332107,344.32456697)
\curveto(289.02331804,344.12456146)(288.90331816,343.88956169)(288.81332107,343.61956697)
\curveto(288.78331828,343.51956206)(288.75831831,343.41456217)(288.73832107,343.30456697)
\curveto(288.72831834,343.19456239)(288.71331835,343.0845625)(288.69332107,342.97456697)
\curveto(288.68331838,342.92456266)(288.67831839,342.87456271)(288.67832107,342.82456697)
\lineto(288.67832107,342.67456697)
\curveto(288.65831841,342.60456298)(288.64831842,342.49956308)(288.64832107,342.35956697)
\curveto(288.65831841,342.21956336)(288.67331839,342.11456347)(288.69332107,342.04456697)
\lineto(288.69332107,341.90956697)
\curveto(288.71331835,341.82956375)(288.72831834,341.74956383)(288.73832107,341.66956697)
\curveto(288.74831832,341.59956398)(288.7633183,341.52456406)(288.78332107,341.44456697)
\curveto(288.88331818,341.14456444)(288.98831808,340.89956468)(289.09832107,340.70956697)
\curveto(289.21831785,340.52956505)(289.40331766,340.36456522)(289.65332107,340.21456697)
\curveto(289.72331734,340.16456542)(289.79831727,340.12456546)(289.87832107,340.09456697)
\curveto(289.9683171,340.06456552)(290.05831701,340.03956554)(290.14832107,340.01956697)
\curveto(290.18831688,340.00956557)(290.22331684,340.00456558)(290.25332107,340.00456697)
\curveto(290.28331678,340.01456557)(290.31831675,340.01456557)(290.35832107,340.00456697)
\lineto(290.47832107,339.97456697)
\curveto(290.52831654,339.97456561)(290.57331649,339.9795656)(290.61332107,339.98956697)
\lineto(290.73332107,339.98956697)
\curveto(290.81331625,340.00956557)(290.89331617,340.02456556)(290.97332107,340.03456697)
\curveto(291.05331601,340.04456554)(291.12831594,340.06456552)(291.19832107,340.09456697)
\curveto(291.45831561,340.19456539)(291.6683154,340.32956525)(291.82832107,340.49956697)
\curveto(291.98831508,340.66956491)(292.12331494,340.8795647)(292.23332107,341.12956697)
\curveto(292.27331479,341.22956435)(292.30331476,341.32956425)(292.32332107,341.42956697)
\curveto(292.34331472,341.52956405)(292.3683147,341.63456395)(292.39832107,341.74456697)
\curveto(292.40831466,341.7845638)(292.41331465,341.81956376)(292.41332107,341.84956697)
\curveto(292.41331465,341.88956369)(292.41831465,341.92956365)(292.42832107,341.96956697)
\lineto(292.42832107,342.10456697)
\curveto(292.42831464,342.15456343)(292.43331463,342.20456338)(292.44332107,342.25456697)
}
}
{
\newrgbcolor{curcolor}{0 0 0}
\pscustom[linestyle=none,fillstyle=solid,fillcolor=curcolor]
{
\newpath
\moveto(826.0368663,330.52791658)
\curveto(826.05685717,330.4479088)(826.06685716,330.33790891)(826.0668663,330.19791658)
\curveto(826.06685716,330.06790918)(826.05685717,329.96790928)(826.0368663,329.89791658)
\curveto(826.01685721,329.82790942)(826.01185722,329.76290949)(826.0218663,329.70291658)
\curveto(826.0318572,329.64290961)(826.0268572,329.57790967)(826.0068663,329.50791658)
\curveto(825.98685724,329.4479098)(825.97185726,329.38290987)(825.9618663,329.31291658)
\curveto(825.95185728,329.25291)(825.93685729,329.19291006)(825.9168663,329.13291658)
\curveto(825.89685733,329.0529102)(825.87185736,328.97791027)(825.8418663,328.90791658)
\curveto(825.82185741,328.83791041)(825.79685743,328.76791048)(825.7668663,328.69791658)
\curveto(825.74685748,328.66791058)(825.7318575,328.63791061)(825.7218663,328.60791658)
\curveto(825.72185751,328.58791066)(825.71185752,328.56791068)(825.6918663,328.54791658)
\curveto(825.58185765,328.3479109)(825.46185777,328.16791108)(825.3318663,328.00791658)
\curveto(825.31185792,327.96791128)(825.27685795,327.92791132)(825.2268663,327.88791658)
\curveto(825.18685804,327.8479114)(825.15185808,327.81791143)(825.1218663,327.79791658)
\curveto(825.08185815,327.77791147)(825.04685818,327.7479115)(825.0168663,327.70791658)
\curveto(824.98685824,327.67791157)(824.95685827,327.6529116)(824.9268663,327.63291658)
\lineto(824.6118663,327.45291658)
\curveto(824.50185873,327.37291188)(824.37185886,327.31291194)(824.2218663,327.27291658)
\lineto(823.7718663,327.15291658)
\curveto(823.69185954,327.13291212)(823.61185962,327.11791213)(823.5318663,327.10791658)
\curveto(823.45185978,327.10791214)(823.37185986,327.09791215)(823.2918663,327.07791658)
\curveto(823.25185998,327.06791218)(823.21186002,327.06291219)(823.1718663,327.06291658)
\curveto(823.14186009,327.07291218)(823.11186012,327.07291218)(823.0818663,327.06291658)
\curveto(823.0318602,327.0529122)(822.98186025,327.0529122)(822.9318663,327.06291658)
\curveto(822.89186034,327.07291218)(822.84686038,327.07291218)(822.7968663,327.06291658)
\lineto(820.5318663,327.06291658)
\lineto(820.0368663,327.06291658)
\curveto(819.86686336,327.07291218)(819.73686349,327.04291221)(819.6468663,326.97291658)
\curveto(819.53686369,326.89291236)(819.48186375,326.7479125)(819.4818663,326.53791658)
\curveto(819.49186374,326.32791292)(819.49686373,326.13291312)(819.4968663,325.95291658)
\lineto(819.4968663,323.74791658)
\lineto(819.4968663,323.25291658)
\curveto(819.50686372,323.06291619)(819.48686374,322.92791632)(819.4368663,322.84791658)
\curveto(819.39686383,322.78791646)(819.34686388,322.7479165)(819.2868663,322.72791658)
\curveto(819.23686399,322.71791653)(819.17186406,322.70291655)(819.0918663,322.68291658)
\lineto(818.8218663,322.68291658)
\curveto(818.67186456,322.68291657)(818.53686469,322.68791656)(818.4168663,322.69791658)
\curveto(818.29686493,322.70791654)(818.21186502,322.75791649)(818.1618663,322.84791658)
\curveto(818.12186511,322.90791634)(818.10186513,322.98791626)(818.1018663,323.08791658)
\lineto(818.1018663,323.40291658)
\lineto(818.1018663,332.50791658)
\curveto(818.10186513,332.61790663)(818.09686513,332.73790651)(818.0868663,332.86791658)
\curveto(818.08686514,333.00790624)(818.11186512,333.11790613)(818.1618663,333.19791658)
\curveto(818.20186503,333.25790599)(818.27686495,333.30790594)(818.3868663,333.34791658)
\curveto(818.40686482,333.35790589)(818.4268648,333.35790589)(818.4468663,333.34791658)
\curveto(818.46686476,333.3479059)(818.48686474,333.3529059)(818.5068663,333.36291658)
\lineto(821.9118663,333.36291658)
\curveto(822.29186094,333.36290589)(822.66186057,333.35790589)(823.0218663,333.34791658)
\curveto(823.39185984,333.3479059)(823.72185951,333.30290595)(824.0118663,333.21291658)
\curveto(824.46185877,333.06290619)(824.8268584,332.86790638)(825.1068663,332.62791658)
\curveto(825.38685784,332.38790686)(825.61685761,332.05790719)(825.7968663,331.63791658)
\curveto(825.84685738,331.52790772)(825.88185735,331.41290784)(825.9018663,331.29291658)
\curveto(825.9318573,331.17290808)(825.96685726,331.0479082)(826.0068663,330.91791658)
\curveto(826.0268572,330.8479084)(826.0318572,330.78290847)(826.0218663,330.72291658)
\curveto(826.01185722,330.66290859)(826.01685721,330.59790865)(826.0368663,330.52791658)
\moveto(824.6268663,329.98791658)
\curveto(824.66685856,330.12790912)(824.67185856,330.28790896)(824.6418663,330.46791658)
\curveto(824.61185862,330.65790859)(824.58185865,330.80790844)(824.5518663,330.91791658)
\curveto(824.45185878,331.19790805)(824.31685891,331.41790783)(824.1468663,331.57791658)
\curveto(823.98685924,331.7479075)(823.77685945,331.88790736)(823.5168663,331.99791658)
\curveto(823.29685993,332.08790716)(823.04186019,332.14290711)(822.7518663,332.16291658)
\curveto(822.47186076,332.18290707)(822.17686105,332.19290706)(821.8668663,332.19291658)
\lineto(819.9318663,332.19291658)
\curveto(819.91186332,332.18290707)(819.88686334,332.17790707)(819.8568663,332.17791658)
\curveto(819.83686339,332.17790707)(819.81186342,332.17290708)(819.7818663,332.16291658)
\curveto(819.66186357,332.13290712)(819.58186365,332.06790718)(819.5418663,331.96791658)
\curveto(819.50186373,331.86790738)(819.48186375,331.73290752)(819.4818663,331.56291658)
\curveto(819.49186374,331.40290785)(819.49686373,331.252908)(819.4968663,331.11291658)
\lineto(819.4968663,329.31291658)
\curveto(819.49686373,329.16291009)(819.49186374,328.99791025)(819.4818663,328.81791658)
\curveto(819.48186375,328.63791061)(819.51186372,328.49791075)(819.5718663,328.39791658)
\curveto(819.62186361,328.31791093)(819.69686353,328.26791098)(819.7968663,328.24791658)
\curveto(819.90686332,328.23791101)(820.0268632,328.23291102)(820.1568663,328.23291658)
\lineto(822.1818663,328.23291658)
\lineto(822.6468663,328.23291658)
\curveto(822.80686042,328.24291101)(822.94686028,328.26291099)(823.0668663,328.29291658)
\curveto(823.33685989,328.36291089)(823.57185966,328.44291081)(823.7718663,328.53291658)
\curveto(823.98185925,328.63291062)(824.15685907,328.78291047)(824.2968663,328.98291658)
\curveto(824.37685885,329.10291015)(824.43685879,329.22791002)(824.4768663,329.35791658)
\curveto(824.5268587,329.48790976)(824.57185866,329.63290962)(824.6118663,329.79291658)
\curveto(824.62185861,329.83290942)(824.6268586,329.89790935)(824.6268663,329.98791658)
}
}
{
\newrgbcolor{curcolor}{0 0 0}
\pscustom[linestyle=none,fillstyle=solid,fillcolor=curcolor]
{
\newpath
\moveto(834.6984288,326.88291658)
\curveto(834.71842074,326.82291243)(834.72842073,326.72791252)(834.7284288,326.59791658)
\curveto(834.72842073,326.47791277)(834.72342073,326.39291286)(834.7134288,326.34291658)
\lineto(834.7134288,326.19291658)
\curveto(834.70342075,326.11291314)(834.69342076,326.03791321)(834.6834288,325.96791658)
\curveto(834.68342077,325.90791334)(834.67842078,325.83791341)(834.6684288,325.75791658)
\curveto(834.64842081,325.69791355)(834.63342082,325.63791361)(834.6234288,325.57791658)
\curveto(834.62342083,325.51791373)(834.61342084,325.45791379)(834.5934288,325.39791658)
\curveto(834.5534209,325.26791398)(834.51842094,325.13791411)(834.4884288,325.00791658)
\curveto(834.458421,324.87791437)(834.41842104,324.75791449)(834.3684288,324.64791658)
\curveto(834.1584213,324.16791508)(833.87842158,323.76291549)(833.5284288,323.43291658)
\curveto(833.17842228,323.11291614)(832.74842271,322.86791638)(832.2384288,322.69791658)
\curveto(832.12842333,322.65791659)(832.00842345,322.62791662)(831.8784288,322.60791658)
\curveto(831.7584237,322.58791666)(831.63342382,322.56791668)(831.5034288,322.54791658)
\curveto(831.44342401,322.53791671)(831.37842408,322.53291672)(831.3084288,322.53291658)
\curveto(831.24842421,322.52291673)(831.18842427,322.51791673)(831.1284288,322.51791658)
\curveto(831.08842437,322.50791674)(831.02842443,322.50291675)(830.9484288,322.50291658)
\curveto(830.87842458,322.50291675)(830.82842463,322.50791674)(830.7984288,322.51791658)
\curveto(830.7584247,322.52791672)(830.71842474,322.53291672)(830.6784288,322.53291658)
\curveto(830.63842482,322.52291673)(830.60342485,322.52291673)(830.5734288,322.53291658)
\lineto(830.4834288,322.53291658)
\lineto(830.1234288,322.57791658)
\curveto(829.98342547,322.61791663)(829.84842561,322.65791659)(829.7184288,322.69791658)
\curveto(829.58842587,322.73791651)(829.46342599,322.78291647)(829.3434288,322.83291658)
\curveto(828.89342656,323.03291622)(828.52342693,323.29291596)(828.2334288,323.61291658)
\curveto(827.94342751,323.93291532)(827.70342775,324.32291493)(827.5134288,324.78291658)
\curveto(827.46342799,324.88291437)(827.42342803,324.98291427)(827.3934288,325.08291658)
\curveto(827.37342808,325.18291407)(827.3534281,325.28791396)(827.3334288,325.39791658)
\curveto(827.31342814,325.43791381)(827.30342815,325.46791378)(827.3034288,325.48791658)
\curveto(827.31342814,325.51791373)(827.31342814,325.5529137)(827.3034288,325.59291658)
\curveto(827.28342817,325.67291358)(827.26842819,325.7529135)(827.2584288,325.83291658)
\curveto(827.2584282,325.92291333)(827.24842821,326.00791324)(827.2284288,326.08791658)
\lineto(827.2284288,326.20791658)
\curveto(827.22842823,326.247913)(827.22342823,326.29291296)(827.2134288,326.34291658)
\curveto(827.20342825,326.39291286)(827.19842826,326.47791277)(827.1984288,326.59791658)
\curveto(827.19842826,326.72791252)(827.20842825,326.82291243)(827.2284288,326.88291658)
\curveto(827.24842821,326.9529123)(827.2534282,327.02291223)(827.2434288,327.09291658)
\curveto(827.23342822,327.16291209)(827.23842822,327.23291202)(827.2584288,327.30291658)
\curveto(827.26842819,327.3529119)(827.27342818,327.39291186)(827.2734288,327.42291658)
\curveto(827.28342817,327.46291179)(827.29342816,327.50791174)(827.3034288,327.55791658)
\curveto(827.33342812,327.67791157)(827.3584281,327.79791145)(827.3784288,327.91791658)
\curveto(827.40842805,328.03791121)(827.44842801,328.1529111)(827.4984288,328.26291658)
\curveto(827.64842781,328.63291062)(827.82842763,328.96291029)(828.0384288,329.25291658)
\curveto(828.2584272,329.5529097)(828.52342693,329.80290945)(828.8334288,330.00291658)
\curveto(828.9534265,330.08290917)(829.07842638,330.1479091)(829.2084288,330.19791658)
\curveto(829.33842612,330.25790899)(829.47342598,330.31790893)(829.6134288,330.37791658)
\curveto(829.73342572,330.42790882)(829.86342559,330.45790879)(830.0034288,330.46791658)
\curveto(830.14342531,330.48790876)(830.28342517,330.51790873)(830.4234288,330.55791658)
\lineto(830.6184288,330.55791658)
\curveto(830.68842477,330.56790868)(830.7534247,330.57790867)(830.8134288,330.58791658)
\curveto(831.70342375,330.59790865)(832.44342301,330.41290884)(833.0334288,330.03291658)
\curveto(833.62342183,329.6529096)(834.04842141,329.15791009)(834.3084288,328.54791658)
\curveto(834.3584211,328.4479108)(834.39842106,328.3479109)(834.4284288,328.24791658)
\curveto(834.458421,328.1479111)(834.49342096,328.04291121)(834.5334288,327.93291658)
\curveto(834.56342089,327.82291143)(834.58842087,327.70291155)(834.6084288,327.57291658)
\curveto(834.62842083,327.4529118)(834.6534208,327.32791192)(834.6834288,327.19791658)
\curveto(834.69342076,327.1479121)(834.69342076,327.09291216)(834.6834288,327.03291658)
\curveto(834.68342077,326.98291227)(834.68842077,326.93291232)(834.6984288,326.88291658)
\moveto(833.3634288,326.02791658)
\curveto(833.38342207,326.09791315)(833.38842207,326.17791307)(833.3784288,326.26791658)
\lineto(833.3784288,326.52291658)
\curveto(833.37842208,326.91291234)(833.34342211,327.24291201)(833.2734288,327.51291658)
\curveto(833.24342221,327.59291166)(833.21842224,327.67291158)(833.1984288,327.75291658)
\curveto(833.17842228,327.83291142)(833.1534223,327.90791134)(833.1234288,327.97791658)
\curveto(832.84342261,328.62791062)(832.39842306,329.07791017)(831.7884288,329.32791658)
\curveto(831.71842374,329.35790989)(831.64342381,329.37790987)(831.5634288,329.38791658)
\lineto(831.3234288,329.44791658)
\curveto(831.24342421,329.46790978)(831.1584243,329.47790977)(831.0684288,329.47791658)
\lineto(830.7984288,329.47791658)
\lineto(830.5284288,329.43291658)
\curveto(830.42842503,329.41290984)(830.33342512,329.38790986)(830.2434288,329.35791658)
\curveto(830.16342529,329.33790991)(830.08342537,329.30790994)(830.0034288,329.26791658)
\curveto(829.93342552,329.24791)(829.86842559,329.21791003)(829.8084288,329.17791658)
\curveto(829.74842571,329.13791011)(829.69342576,329.09791015)(829.6434288,329.05791658)
\curveto(829.40342605,328.88791036)(829.20842625,328.68291057)(829.0584288,328.44291658)
\curveto(828.90842655,328.20291105)(828.77842668,327.92291133)(828.6684288,327.60291658)
\curveto(828.63842682,327.50291175)(828.61842684,327.39791185)(828.6084288,327.28791658)
\curveto(828.59842686,327.18791206)(828.58342687,327.08291217)(828.5634288,326.97291658)
\curveto(828.5534269,326.93291232)(828.54842691,326.86791238)(828.5484288,326.77791658)
\curveto(828.53842692,326.7479125)(828.53342692,326.71291254)(828.5334288,326.67291658)
\curveto(828.54342691,326.63291262)(828.54842691,326.58791266)(828.5484288,326.53791658)
\lineto(828.5484288,326.23791658)
\curveto(828.54842691,326.13791311)(828.5584269,326.0479132)(828.5784288,325.96791658)
\lineto(828.6084288,325.78791658)
\curveto(828.62842683,325.68791356)(828.64342681,325.58791366)(828.6534288,325.48791658)
\curveto(828.67342678,325.39791385)(828.70342675,325.31291394)(828.7434288,325.23291658)
\curveto(828.84342661,324.99291426)(828.9584265,324.76791448)(829.0884288,324.55791658)
\curveto(829.22842623,324.3479149)(829.39842606,324.17291508)(829.5984288,324.03291658)
\curveto(829.64842581,324.00291525)(829.69342576,323.97791527)(829.7334288,323.95791658)
\curveto(829.77342568,323.93791531)(829.81842564,323.91291534)(829.8684288,323.88291658)
\curveto(829.94842551,323.83291542)(830.03342542,323.78791546)(830.1234288,323.74791658)
\curveto(830.22342523,323.71791553)(830.32842513,323.68791556)(830.4384288,323.65791658)
\curveto(830.48842497,323.63791561)(830.53342492,323.62791562)(830.5734288,323.62791658)
\curveto(830.62342483,323.63791561)(830.67342478,323.63791561)(830.7234288,323.62791658)
\curveto(830.7534247,323.61791563)(830.81342464,323.60791564)(830.9034288,323.59791658)
\curveto(831.00342445,323.58791566)(831.07842438,323.59291566)(831.1284288,323.61291658)
\curveto(831.16842429,323.62291563)(831.20842425,323.62291563)(831.2484288,323.61291658)
\curveto(831.28842417,323.61291564)(831.32842413,323.62291563)(831.3684288,323.64291658)
\curveto(831.44842401,323.66291559)(831.52842393,323.67791557)(831.6084288,323.68791658)
\curveto(831.68842377,323.70791554)(831.76342369,323.73291552)(831.8334288,323.76291658)
\curveto(832.17342328,323.90291535)(832.44842301,324.09791515)(832.6584288,324.34791658)
\curveto(832.86842259,324.59791465)(833.04342241,324.89291436)(833.1834288,325.23291658)
\curveto(833.23342222,325.3529139)(833.26342219,325.47791377)(833.2734288,325.60791658)
\curveto(833.29342216,325.7479135)(833.32342213,325.88791336)(833.3634288,326.02791658)
}
}
{
\newrgbcolor{curcolor}{0 0 0}
\pscustom[linestyle=none,fillstyle=solid,fillcolor=curcolor]
{
\newpath
\moveto(839.83171005,330.58791658)
\curveto(840.06170526,330.58790866)(840.19170513,330.52790872)(840.22171005,330.40791658)
\curveto(840.25170507,330.29790895)(840.26670505,330.13290912)(840.26671005,329.91291658)
\lineto(840.26671005,329.62791658)
\curveto(840.26670505,329.53790971)(840.24170508,329.46290979)(840.19171005,329.40291658)
\curveto(840.13170519,329.32290993)(840.04670527,329.27790997)(839.93671005,329.26791658)
\curveto(839.82670549,329.26790998)(839.7167056,329.25291)(839.60671005,329.22291658)
\curveto(839.46670585,329.19291006)(839.33170599,329.16291009)(839.20171005,329.13291658)
\curveto(839.08170624,329.10291015)(838.96670635,329.06291019)(838.85671005,329.01291658)
\curveto(838.56670675,328.88291037)(838.33170699,328.70291055)(838.15171005,328.47291658)
\curveto(837.97170735,328.252911)(837.8167075,327.99791125)(837.68671005,327.70791658)
\curveto(837.64670767,327.59791165)(837.6167077,327.48291177)(837.59671005,327.36291658)
\curveto(837.57670774,327.252912)(837.55170777,327.13791211)(837.52171005,327.01791658)
\curveto(837.51170781,326.96791228)(837.50670781,326.91791233)(837.50671005,326.86791658)
\curveto(837.5167078,326.81791243)(837.5167078,326.76791248)(837.50671005,326.71791658)
\curveto(837.47670784,326.59791265)(837.46170786,326.45791279)(837.46171005,326.29791658)
\curveto(837.47170785,326.1479131)(837.47670784,326.00291325)(837.47671005,325.86291658)
\lineto(837.47671005,324.01791658)
\lineto(837.47671005,323.67291658)
\curveto(837.47670784,323.5529157)(837.47170785,323.43791581)(837.46171005,323.32791658)
\curveto(837.45170787,323.21791603)(837.44670787,323.12291613)(837.44671005,323.04291658)
\curveto(837.45670786,322.96291629)(837.43670788,322.89291636)(837.38671005,322.83291658)
\curveto(837.33670798,322.76291649)(837.25670806,322.72291653)(837.14671005,322.71291658)
\curveto(837.04670827,322.70291655)(836.93670838,322.69791655)(836.81671005,322.69791658)
\lineto(836.54671005,322.69791658)
\curveto(836.49670882,322.71791653)(836.44670887,322.73291652)(836.39671005,322.74291658)
\curveto(836.35670896,322.76291649)(836.32670899,322.78791646)(836.30671005,322.81791658)
\curveto(836.25670906,322.88791636)(836.22670909,322.97291628)(836.21671005,323.07291658)
\lineto(836.21671005,323.40291658)
\lineto(836.21671005,324.55791658)
\lineto(836.21671005,328.71291658)
\lineto(836.21671005,329.74791658)
\lineto(836.21671005,330.04791658)
\curveto(836.22670909,330.1479091)(836.25670906,330.23290902)(836.30671005,330.30291658)
\curveto(836.33670898,330.34290891)(836.38670893,330.37290888)(836.45671005,330.39291658)
\curveto(836.53670878,330.41290884)(836.6217087,330.42290883)(836.71171005,330.42291658)
\curveto(836.80170852,330.43290882)(836.89170843,330.43290882)(836.98171005,330.42291658)
\curveto(837.07170825,330.41290884)(837.14170818,330.39790885)(837.19171005,330.37791658)
\curveto(837.27170805,330.3479089)(837.321708,330.28790896)(837.34171005,330.19791658)
\curveto(837.37170795,330.11790913)(837.38670793,330.02790922)(837.38671005,329.92791658)
\lineto(837.38671005,329.62791658)
\curveto(837.38670793,329.52790972)(837.40670791,329.43790981)(837.44671005,329.35791658)
\curveto(837.45670786,329.33790991)(837.46670785,329.32290993)(837.47671005,329.31291658)
\lineto(837.52171005,329.26791658)
\curveto(837.63170769,329.26790998)(837.7217076,329.31290994)(837.79171005,329.40291658)
\curveto(837.86170746,329.50290975)(837.9217074,329.58290967)(837.97171005,329.64291658)
\lineto(838.06171005,329.73291658)
\curveto(838.15170717,329.84290941)(838.27670704,329.95790929)(838.43671005,330.07791658)
\curveto(838.59670672,330.19790905)(838.74670657,330.28790896)(838.88671005,330.34791658)
\curveto(838.97670634,330.39790885)(839.07170625,330.43290882)(839.17171005,330.45291658)
\curveto(839.27170605,330.48290877)(839.37670594,330.51290874)(839.48671005,330.54291658)
\curveto(839.54670577,330.5529087)(839.60670571,330.55790869)(839.66671005,330.55791658)
\curveto(839.72670559,330.56790868)(839.78170554,330.57790867)(839.83171005,330.58791658)
}
}
{
\newrgbcolor{curcolor}{0 0 0}
\pscustom[linestyle=none,fillstyle=solid,fillcolor=curcolor]
{
\newpath
\moveto(842.14147567,332.74791658)
\curveto(842.29147366,332.7479065)(842.44147351,332.74290651)(842.59147567,332.73291658)
\curveto(842.74147321,332.73290652)(842.84647311,332.69290656)(842.90647567,332.61291658)
\curveto(842.956473,332.5529067)(842.98147297,332.46790678)(842.98147567,332.35791658)
\curveto(842.99147296,332.25790699)(842.99647296,332.1529071)(842.99647567,332.04291658)
\lineto(842.99647567,331.17291658)
\curveto(842.99647296,331.09290816)(842.99147296,331.00790824)(842.98147567,330.91791658)
\curveto(842.98147297,330.83790841)(842.99147296,330.76790848)(843.01147567,330.70791658)
\curveto(843.0514729,330.56790868)(843.14147281,330.47790877)(843.28147567,330.43791658)
\curveto(843.33147262,330.42790882)(843.37647258,330.42290883)(843.41647567,330.42291658)
\lineto(843.56647567,330.42291658)
\lineto(843.97147567,330.42291658)
\curveto(844.13147182,330.43290882)(844.24647171,330.42290883)(844.31647567,330.39291658)
\curveto(844.40647155,330.33290892)(844.46647149,330.27290898)(844.49647567,330.21291658)
\curveto(844.51647144,330.17290908)(844.52647143,330.12790912)(844.52647567,330.07791658)
\lineto(844.52647567,329.92791658)
\curveto(844.52647143,329.81790943)(844.52147143,329.71290954)(844.51147567,329.61291658)
\curveto(844.50147145,329.52290973)(844.46647149,329.4529098)(844.40647567,329.40291658)
\curveto(844.34647161,329.3529099)(844.26147169,329.32290993)(844.15147567,329.31291658)
\lineto(843.82147567,329.31291658)
\curveto(843.71147224,329.32290993)(843.60147235,329.32790992)(843.49147567,329.32791658)
\curveto(843.38147257,329.32790992)(843.28647267,329.31290994)(843.20647567,329.28291658)
\curveto(843.13647282,329.25291)(843.08647287,329.20291005)(843.05647567,329.13291658)
\curveto(843.02647293,329.06291019)(843.00647295,328.97791027)(842.99647567,328.87791658)
\curveto(842.98647297,328.78791046)(842.98147297,328.68791056)(842.98147567,328.57791658)
\curveto(842.99147296,328.47791077)(842.99647296,328.37791087)(842.99647567,328.27791658)
\lineto(842.99647567,325.30791658)
\curveto(842.99647296,325.08791416)(842.99147296,324.8529144)(842.98147567,324.60291658)
\curveto(842.98147297,324.36291489)(843.02647293,324.17791507)(843.11647567,324.04791658)
\curveto(843.16647279,323.96791528)(843.23147272,323.91291534)(843.31147567,323.88291658)
\curveto(843.39147256,323.8529154)(843.48647247,323.82791542)(843.59647567,323.80791658)
\curveto(843.62647233,323.79791545)(843.6564723,323.79291546)(843.68647567,323.79291658)
\curveto(843.72647223,323.80291545)(843.76147219,323.80291545)(843.79147567,323.79291658)
\lineto(843.98647567,323.79291658)
\curveto(844.08647187,323.79291546)(844.17647178,323.78291547)(844.25647567,323.76291658)
\curveto(844.34647161,323.7529155)(844.41147154,323.71791553)(844.45147567,323.65791658)
\curveto(844.47147148,323.62791562)(844.48647147,323.57291568)(844.49647567,323.49291658)
\curveto(844.51647144,323.42291583)(844.52647143,323.3479159)(844.52647567,323.26791658)
\curveto(844.53647142,323.18791606)(844.53647142,323.10791614)(844.52647567,323.02791658)
\curveto(844.51647144,322.95791629)(844.49647146,322.90291635)(844.46647567,322.86291658)
\curveto(844.42647153,322.79291646)(844.3514716,322.74291651)(844.24147567,322.71291658)
\curveto(844.16147179,322.69291656)(844.07147188,322.68291657)(843.97147567,322.68291658)
\curveto(843.87147208,322.69291656)(843.78147217,322.69791655)(843.70147567,322.69791658)
\curveto(843.64147231,322.69791655)(843.58147237,322.69291656)(843.52147567,322.68291658)
\curveto(843.46147249,322.68291657)(843.40647255,322.68791656)(843.35647567,322.69791658)
\lineto(843.17647567,322.69791658)
\curveto(843.12647283,322.70791654)(843.07647288,322.71291654)(843.02647567,322.71291658)
\curveto(842.98647297,322.72291653)(842.94147301,322.72791652)(842.89147567,322.72791658)
\curveto(842.69147326,322.77791647)(842.51647344,322.83291642)(842.36647567,322.89291658)
\curveto(842.22647373,322.9529163)(842.10647385,323.05791619)(842.00647567,323.20791658)
\curveto(841.86647409,323.40791584)(841.78647417,323.65791559)(841.76647567,323.95791658)
\curveto(841.74647421,324.26791498)(841.73647422,324.59791465)(841.73647567,324.94791658)
\lineto(841.73647567,328.87791658)
\curveto(841.70647425,329.00791024)(841.67647428,329.10291015)(841.64647567,329.16291658)
\curveto(841.62647433,329.22291003)(841.5564744,329.27290998)(841.43647567,329.31291658)
\curveto(841.39647456,329.32290993)(841.3564746,329.32290993)(841.31647567,329.31291658)
\curveto(841.27647468,329.30290995)(841.23647472,329.30790994)(841.19647567,329.32791658)
\lineto(840.95647567,329.32791658)
\curveto(840.82647513,329.32790992)(840.71647524,329.33790991)(840.62647567,329.35791658)
\curveto(840.54647541,329.38790986)(840.49147546,329.4479098)(840.46147567,329.53791658)
\curveto(840.44147551,329.57790967)(840.42647553,329.62290963)(840.41647567,329.67291658)
\lineto(840.41647567,329.82291658)
\curveto(840.41647554,329.96290929)(840.42647553,330.07790917)(840.44647567,330.16791658)
\curveto(840.46647549,330.26790898)(840.52647543,330.34290891)(840.62647567,330.39291658)
\curveto(840.73647522,330.43290882)(840.87647508,330.44290881)(841.04647567,330.42291658)
\curveto(841.22647473,330.40290885)(841.37647458,330.41290884)(841.49647567,330.45291658)
\curveto(841.58647437,330.50290875)(841.6564743,330.57290868)(841.70647567,330.66291658)
\curveto(841.72647423,330.72290853)(841.73647422,330.79790845)(841.73647567,330.88791658)
\lineto(841.73647567,331.14291658)
\lineto(841.73647567,332.07291658)
\lineto(841.73647567,332.31291658)
\curveto(841.73647422,332.40290685)(841.74647421,332.47790677)(841.76647567,332.53791658)
\curveto(841.80647415,332.61790663)(841.88147407,332.68290657)(841.99147567,332.73291658)
\curveto(842.02147393,332.73290652)(842.04647391,332.73290652)(842.06647567,332.73291658)
\curveto(842.09647386,332.74290651)(842.12147383,332.7479065)(842.14147567,332.74791658)
}
}
{
\newrgbcolor{curcolor}{0 0 0}
\pscustom[linestyle=none,fillstyle=solid,fillcolor=curcolor]
{
\newpath
\moveto(852.79827255,323.23791658)
\curveto(852.82826472,323.07791617)(852.81326473,322.94291631)(852.75327255,322.83291658)
\curveto(852.69326485,322.73291652)(852.61326493,322.65791659)(852.51327255,322.60791658)
\curveto(852.46326508,322.58791666)(852.40826514,322.57791667)(852.34827255,322.57791658)
\curveto(852.29826525,322.57791667)(852.2432653,322.56791668)(852.18327255,322.54791658)
\curveto(851.96326558,322.49791675)(851.7432658,322.51291674)(851.52327255,322.59291658)
\curveto(851.31326623,322.66291659)(851.16826638,322.7529165)(851.08827255,322.86291658)
\curveto(851.03826651,322.93291632)(850.99326655,323.01291624)(850.95327255,323.10291658)
\curveto(850.91326663,323.20291605)(850.86326668,323.28291597)(850.80327255,323.34291658)
\curveto(850.78326676,323.36291589)(850.75826679,323.38291587)(850.72827255,323.40291658)
\curveto(850.70826684,323.42291583)(850.67826687,323.42791582)(850.63827255,323.41791658)
\curveto(850.52826702,323.38791586)(850.42326712,323.33291592)(850.32327255,323.25291658)
\curveto(850.23326731,323.17291608)(850.1432674,323.10291615)(850.05327255,323.04291658)
\curveto(849.92326762,322.96291629)(849.78326776,322.88791636)(849.63327255,322.81791658)
\curveto(849.48326806,322.75791649)(849.32326822,322.70291655)(849.15327255,322.65291658)
\curveto(849.05326849,322.62291663)(848.9432686,322.60291665)(848.82327255,322.59291658)
\curveto(848.71326883,322.58291667)(848.60326894,322.56791668)(848.49327255,322.54791658)
\curveto(848.4432691,322.53791671)(848.39826915,322.53291672)(848.35827255,322.53291658)
\lineto(848.25327255,322.53291658)
\curveto(848.1432694,322.51291674)(848.03826951,322.51291674)(847.93827255,322.53291658)
\lineto(847.80327255,322.53291658)
\curveto(847.75326979,322.54291671)(847.70326984,322.5479167)(847.65327255,322.54791658)
\curveto(847.60326994,322.5479167)(847.55826999,322.55791669)(847.51827255,322.57791658)
\curveto(847.47827007,322.58791666)(847.4432701,322.59291666)(847.41327255,322.59291658)
\curveto(847.39327015,322.58291667)(847.36827018,322.58291667)(847.33827255,322.59291658)
\lineto(847.09827255,322.65291658)
\curveto(847.01827053,322.66291659)(846.9432706,322.68291657)(846.87327255,322.71291658)
\curveto(846.57327097,322.84291641)(846.32827122,322.98791626)(846.13827255,323.14791658)
\curveto(845.95827159,323.31791593)(845.80827174,323.5529157)(845.68827255,323.85291658)
\curveto(845.59827195,324.07291518)(845.55327199,324.33791491)(845.55327255,324.64791658)
\lineto(845.55327255,324.96291658)
\curveto(845.56327198,325.01291424)(845.56827198,325.06291419)(845.56827255,325.11291658)
\lineto(845.59827255,325.29291658)
\lineto(845.71827255,325.62291658)
\curveto(845.75827179,325.73291352)(845.80827174,325.83291342)(845.86827255,325.92291658)
\curveto(846.0482715,326.21291304)(846.29327125,326.42791282)(846.60327255,326.56791658)
\curveto(846.91327063,326.70791254)(847.25327029,326.83291242)(847.62327255,326.94291658)
\curveto(847.76326978,326.98291227)(847.90826964,327.01291224)(848.05827255,327.03291658)
\curveto(848.20826934,327.0529122)(848.35826919,327.07791217)(848.50827255,327.10791658)
\curveto(848.57826897,327.12791212)(848.6432689,327.13791211)(848.70327255,327.13791658)
\curveto(848.77326877,327.13791211)(848.8482687,327.1479121)(848.92827255,327.16791658)
\curveto(848.99826855,327.18791206)(849.06826848,327.19791205)(849.13827255,327.19791658)
\curveto(849.20826834,327.20791204)(849.28326826,327.22291203)(849.36327255,327.24291658)
\curveto(849.61326793,327.30291195)(849.8482677,327.3529119)(850.06827255,327.39291658)
\curveto(850.28826726,327.44291181)(850.46326708,327.55791169)(850.59327255,327.73791658)
\curveto(850.65326689,327.81791143)(850.70326684,327.91791133)(850.74327255,328.03791658)
\curveto(850.78326676,328.16791108)(850.78326676,328.30791094)(850.74327255,328.45791658)
\curveto(850.68326686,328.69791055)(850.59326695,328.88791036)(850.47327255,329.02791658)
\curveto(850.36326718,329.16791008)(850.20326734,329.27790997)(849.99327255,329.35791658)
\curveto(849.87326767,329.40790984)(849.72826782,329.44290981)(849.55827255,329.46291658)
\curveto(849.39826815,329.48290977)(849.22826832,329.49290976)(849.04827255,329.49291658)
\curveto(848.86826868,329.49290976)(848.69326885,329.48290977)(848.52327255,329.46291658)
\curveto(848.35326919,329.44290981)(848.20826934,329.41290984)(848.08827255,329.37291658)
\curveto(847.91826963,329.31290994)(847.75326979,329.22791002)(847.59327255,329.11791658)
\curveto(847.51327003,329.05791019)(847.43827011,328.97791027)(847.36827255,328.87791658)
\curveto(847.30827024,328.78791046)(847.25327029,328.68791056)(847.20327255,328.57791658)
\curveto(847.17327037,328.49791075)(847.1432704,328.41291084)(847.11327255,328.32291658)
\curveto(847.09327045,328.23291102)(847.0482705,328.16291109)(846.97827255,328.11291658)
\curveto(846.93827061,328.08291117)(846.86827068,328.05791119)(846.76827255,328.03791658)
\curveto(846.67827087,328.02791122)(846.58327096,328.02291123)(846.48327255,328.02291658)
\curveto(846.38327116,328.02291123)(846.28327126,328.02791122)(846.18327255,328.03791658)
\curveto(846.09327145,328.05791119)(846.02827152,328.08291117)(845.98827255,328.11291658)
\curveto(845.9482716,328.14291111)(845.91827163,328.19291106)(845.89827255,328.26291658)
\curveto(845.87827167,328.33291092)(845.87827167,328.40791084)(845.89827255,328.48791658)
\curveto(845.92827162,328.61791063)(845.95827159,328.73791051)(845.98827255,328.84791658)
\curveto(846.02827152,328.96791028)(846.07327147,329.08291017)(846.12327255,329.19291658)
\curveto(846.31327123,329.54290971)(846.55327099,329.81290944)(846.84327255,330.00291658)
\curveto(847.13327041,330.20290905)(847.49327005,330.36290889)(847.92327255,330.48291658)
\curveto(848.02326952,330.50290875)(848.12326942,330.51790873)(848.22327255,330.52791658)
\curveto(848.33326921,330.53790871)(848.4432691,330.5529087)(848.55327255,330.57291658)
\curveto(848.59326895,330.58290867)(848.65826889,330.58290867)(848.74827255,330.57291658)
\curveto(848.83826871,330.57290868)(848.89326865,330.58290867)(848.91327255,330.60291658)
\curveto(849.61326793,330.61290864)(850.22326732,330.53290872)(850.74327255,330.36291658)
\curveto(851.26326628,330.19290906)(851.62826592,329.86790938)(851.83827255,329.38791658)
\curveto(851.92826562,329.18791006)(851.97826557,328.9529103)(851.98827255,328.68291658)
\curveto(852.00826554,328.42291083)(852.01826553,328.1479111)(852.01827255,327.85791658)
\lineto(852.01827255,324.54291658)
\curveto(852.01826553,324.40291485)(852.02326552,324.26791498)(852.03327255,324.13791658)
\curveto(852.0432655,324.00791524)(852.07326547,323.90291535)(852.12327255,323.82291658)
\curveto(852.17326537,323.7529155)(852.23826531,323.70291555)(852.31827255,323.67291658)
\curveto(852.40826514,323.63291562)(852.49326505,323.60291565)(852.57327255,323.58291658)
\curveto(852.65326489,323.57291568)(852.71326483,323.52791572)(852.75327255,323.44791658)
\curveto(852.77326477,323.41791583)(852.78326476,323.38791586)(852.78327255,323.35791658)
\curveto(852.78326476,323.32791592)(852.78826476,323.28791596)(852.79827255,323.23791658)
\moveto(850.65327255,324.90291658)
\curveto(850.71326683,325.04291421)(850.7432668,325.20291405)(850.74327255,325.38291658)
\curveto(850.75326679,325.57291368)(850.75826679,325.76791348)(850.75827255,325.96791658)
\curveto(850.75826679,326.07791317)(850.75326679,326.17791307)(850.74327255,326.26791658)
\curveto(850.73326681,326.35791289)(850.69326685,326.42791282)(850.62327255,326.47791658)
\curveto(850.59326695,326.49791275)(850.52326702,326.50791274)(850.41327255,326.50791658)
\curveto(850.39326715,326.48791276)(850.35826719,326.47791277)(850.30827255,326.47791658)
\curveto(850.25826729,326.47791277)(850.21326733,326.46791278)(850.17327255,326.44791658)
\curveto(850.09326745,326.42791282)(850.00326754,326.40791284)(849.90327255,326.38791658)
\lineto(849.60327255,326.32791658)
\curveto(849.57326797,326.32791292)(849.53826801,326.32291293)(849.49827255,326.31291658)
\lineto(849.39327255,326.31291658)
\curveto(849.2432683,326.27291298)(849.07826847,326.247913)(848.89827255,326.23791658)
\curveto(848.72826882,326.23791301)(848.56826898,326.21791303)(848.41827255,326.17791658)
\curveto(848.33826921,326.15791309)(848.26326928,326.13791311)(848.19327255,326.11791658)
\curveto(848.13326941,326.10791314)(848.06326948,326.09291316)(847.98327255,326.07291658)
\curveto(847.82326972,326.02291323)(847.67326987,325.95791329)(847.53327255,325.87791658)
\curveto(847.39327015,325.80791344)(847.27327027,325.71791353)(847.17327255,325.60791658)
\curveto(847.07327047,325.49791375)(846.99827055,325.36291389)(846.94827255,325.20291658)
\curveto(846.89827065,325.0529142)(846.87827067,324.86791438)(846.88827255,324.64791658)
\curveto(846.88827066,324.5479147)(846.90327064,324.4529148)(846.93327255,324.36291658)
\curveto(846.97327057,324.28291497)(847.01827053,324.20791504)(847.06827255,324.13791658)
\curveto(847.1482704,324.02791522)(847.25327029,323.93291532)(847.38327255,323.85291658)
\curveto(847.51327003,323.78291547)(847.65326989,323.72291553)(847.80327255,323.67291658)
\curveto(847.85326969,323.66291559)(847.90326964,323.65791559)(847.95327255,323.65791658)
\curveto(848.00326954,323.65791559)(848.05326949,323.6529156)(848.10327255,323.64291658)
\curveto(848.17326937,323.62291563)(848.25826929,323.60791564)(848.35827255,323.59791658)
\curveto(848.46826908,323.59791565)(848.55826899,323.60791564)(848.62827255,323.62791658)
\curveto(848.68826886,323.6479156)(848.7482688,323.6529156)(848.80827255,323.64291658)
\curveto(848.86826868,323.64291561)(848.92826862,323.6529156)(848.98827255,323.67291658)
\curveto(849.06826848,323.69291556)(849.1432684,323.70791554)(849.21327255,323.71791658)
\curveto(849.29326825,323.72791552)(849.36826818,323.7479155)(849.43827255,323.77791658)
\curveto(849.72826782,323.89791535)(849.97326757,324.04291521)(850.17327255,324.21291658)
\curveto(850.38326716,324.38291487)(850.543267,324.61291464)(850.65327255,324.90291658)
}
}
{
\newrgbcolor{curcolor}{0 0 0}
\pscustom[linestyle=none,fillstyle=solid,fillcolor=curcolor]
{
\newpath
\moveto(860.92991317,323.49291658)
\lineto(860.92991317,323.10291658)
\curveto(860.9299053,322.98291627)(860.90490532,322.88291637)(860.85491317,322.80291658)
\curveto(860.80490542,322.73291652)(860.71990551,322.69291656)(860.59991317,322.68291658)
\lineto(860.25491317,322.68291658)
\curveto(860.19490603,322.68291657)(860.13490609,322.67791657)(860.07491317,322.66791658)
\curveto(860.0249062,322.66791658)(859.97990625,322.67791657)(859.93991317,322.69791658)
\curveto(859.84990638,322.71791653)(859.78990644,322.75791649)(859.75991317,322.81791658)
\curveto(859.71990651,322.86791638)(859.69490653,322.92791632)(859.68491317,322.99791658)
\curveto(859.68490654,323.06791618)(859.66990656,323.13791611)(859.63991317,323.20791658)
\curveto(859.6299066,323.22791602)(859.61490661,323.24291601)(859.59491317,323.25291658)
\curveto(859.58490664,323.27291598)(859.56990666,323.29291596)(859.54991317,323.31291658)
\curveto(859.44990678,323.32291593)(859.36990686,323.30291595)(859.30991317,323.25291658)
\curveto(859.25990697,323.20291605)(859.20490702,323.1529161)(859.14491317,323.10291658)
\curveto(858.94490728,322.9529163)(858.74490748,322.83791641)(858.54491317,322.75791658)
\curveto(858.36490786,322.67791657)(858.15490807,322.61791663)(857.91491317,322.57791658)
\curveto(857.68490854,322.53791671)(857.44490878,322.51791673)(857.19491317,322.51791658)
\curveto(856.95490927,322.50791674)(856.71490951,322.52291673)(856.47491317,322.56291658)
\curveto(856.23490999,322.59291666)(856.0249102,322.6479166)(855.84491317,322.72791658)
\curveto(855.3249109,322.9479163)(854.90491132,323.24291601)(854.58491317,323.61291658)
\curveto(854.26491196,323.99291526)(854.01491221,324.46291479)(853.83491317,325.02291658)
\curveto(853.79491243,325.11291414)(853.76491246,325.20291405)(853.74491317,325.29291658)
\curveto(853.73491249,325.39291386)(853.71491251,325.49291376)(853.68491317,325.59291658)
\curveto(853.67491255,325.64291361)(853.66991256,325.69291356)(853.66991317,325.74291658)
\curveto(853.66991256,325.79291346)(853.66491256,325.84291341)(853.65491317,325.89291658)
\curveto(853.63491259,325.94291331)(853.6249126,325.99291326)(853.62491317,326.04291658)
\curveto(853.63491259,326.10291315)(853.63491259,326.15791309)(853.62491317,326.20791658)
\lineto(853.62491317,326.35791658)
\curveto(853.60491262,326.40791284)(853.59491263,326.47291278)(853.59491317,326.55291658)
\curveto(853.59491263,326.63291262)(853.60491262,326.69791255)(853.62491317,326.74791658)
\lineto(853.62491317,326.91291658)
\curveto(853.64491258,326.98291227)(853.64991258,327.0529122)(853.63991317,327.12291658)
\curveto(853.63991259,327.20291205)(853.64991258,327.27791197)(853.66991317,327.34791658)
\curveto(853.67991255,327.39791185)(853.68491254,327.44291181)(853.68491317,327.48291658)
\curveto(853.68491254,327.52291173)(853.68991254,327.56791168)(853.69991317,327.61791658)
\curveto(853.7299125,327.71791153)(853.75491247,327.81291144)(853.77491317,327.90291658)
\curveto(853.79491243,328.00291125)(853.81991241,328.09791115)(853.84991317,328.18791658)
\curveto(853.97991225,328.56791068)(854.14491208,328.90791034)(854.34491317,329.20791658)
\curveto(854.55491167,329.51790973)(854.80491142,329.77290948)(855.09491317,329.97291658)
\curveto(855.26491096,330.09290916)(855.43991079,330.19290906)(855.61991317,330.27291658)
\curveto(855.80991042,330.3529089)(856.01491021,330.42290883)(856.23491317,330.48291658)
\curveto(856.30490992,330.49290876)(856.36990986,330.50290875)(856.42991317,330.51291658)
\curveto(856.49990973,330.52290873)(856.56990966,330.53790871)(856.63991317,330.55791658)
\lineto(856.78991317,330.55791658)
\curveto(856.86990936,330.57790867)(856.98490924,330.58790866)(857.13491317,330.58791658)
\curveto(857.29490893,330.58790866)(857.41490881,330.57790867)(857.49491317,330.55791658)
\curveto(857.53490869,330.5479087)(857.58990864,330.54290871)(857.65991317,330.54291658)
\curveto(857.76990846,330.51290874)(857.87990835,330.48790876)(857.98991317,330.46791658)
\curveto(858.09990813,330.45790879)(858.20490802,330.42790882)(858.30491317,330.37791658)
\curveto(858.45490777,330.31790893)(858.59490763,330.252909)(858.72491317,330.18291658)
\curveto(858.86490736,330.11290914)(858.99490723,330.03290922)(859.11491317,329.94291658)
\curveto(859.17490705,329.89290936)(859.23490699,329.83790941)(859.29491317,329.77791658)
\curveto(859.36490686,329.72790952)(859.45490677,329.71290954)(859.56491317,329.73291658)
\curveto(859.58490664,329.76290949)(859.59990663,329.78790946)(859.60991317,329.80791658)
\curveto(859.6299066,329.82790942)(859.64490658,329.85790939)(859.65491317,329.89791658)
\curveto(859.68490654,329.98790926)(859.69490653,330.10290915)(859.68491317,330.24291658)
\lineto(859.68491317,330.61791658)
\lineto(859.68491317,332.34291658)
\lineto(859.68491317,332.80791658)
\curveto(859.68490654,332.98790626)(859.70990652,333.11790613)(859.75991317,333.19791658)
\curveto(859.79990643,333.26790598)(859.85990637,333.31290594)(859.93991317,333.33291658)
\curveto(859.95990627,333.33290592)(859.98490624,333.33290592)(860.01491317,333.33291658)
\curveto(860.04490618,333.34290591)(860.06990616,333.3479059)(860.08991317,333.34791658)
\curveto(860.229906,333.35790589)(860.37490585,333.35790589)(860.52491317,333.34791658)
\curveto(860.68490554,333.3479059)(860.79490543,333.30790594)(860.85491317,333.22791658)
\curveto(860.90490532,333.1479061)(860.9299053,333.0479062)(860.92991317,332.92791658)
\lineto(860.92991317,332.55291658)
\lineto(860.92991317,323.49291658)
\moveto(859.71491317,326.32791658)
\curveto(859.73490649,326.37791287)(859.74490648,326.44291281)(859.74491317,326.52291658)
\curveto(859.74490648,326.61291264)(859.73490649,326.68291257)(859.71491317,326.73291658)
\lineto(859.71491317,326.95791658)
\curveto(859.69490653,327.0479122)(859.67990655,327.13791211)(859.66991317,327.22791658)
\curveto(859.65990657,327.32791192)(859.63990659,327.41791183)(859.60991317,327.49791658)
\curveto(859.58990664,327.57791167)(859.56990666,327.6529116)(859.54991317,327.72291658)
\curveto(859.53990669,327.79291146)(859.51990671,327.86291139)(859.48991317,327.93291658)
\curveto(859.36990686,328.23291102)(859.21490701,328.49791075)(859.02491317,328.72791658)
\curveto(858.83490739,328.95791029)(858.59490763,329.13791011)(858.30491317,329.26791658)
\curveto(858.20490802,329.31790993)(858.09990813,329.3529099)(857.98991317,329.37291658)
\curveto(857.88990834,329.40290985)(857.77990845,329.42790982)(857.65991317,329.44791658)
\curveto(857.57990865,329.46790978)(857.48990874,329.47790977)(857.38991317,329.47791658)
\lineto(857.11991317,329.47791658)
\curveto(857.06990916,329.46790978)(857.0249092,329.45790979)(856.98491317,329.44791658)
\lineto(856.84991317,329.44791658)
\curveto(856.76990946,329.42790982)(856.68490954,329.40790984)(856.59491317,329.38791658)
\curveto(856.51490971,329.36790988)(856.43490979,329.34290991)(856.35491317,329.31291658)
\curveto(856.03491019,329.17291008)(855.77491045,328.96791028)(855.57491317,328.69791658)
\curveto(855.38491084,328.43791081)(855.229911,328.13291112)(855.10991317,327.78291658)
\curveto(855.06991116,327.67291158)(855.03991119,327.55791169)(855.01991317,327.43791658)
\curveto(855.00991122,327.32791192)(854.99491123,327.21791203)(854.97491317,327.10791658)
\curveto(854.97491125,327.06791218)(854.96991126,327.02791222)(854.95991317,326.98791658)
\lineto(854.95991317,326.88291658)
\curveto(854.93991129,326.83291242)(854.9299113,326.77791247)(854.92991317,326.71791658)
\curveto(854.93991129,326.65791259)(854.94491128,326.60291265)(854.94491317,326.55291658)
\lineto(854.94491317,326.22291658)
\curveto(854.94491128,326.12291313)(854.95491127,326.02791322)(854.97491317,325.93791658)
\curveto(854.98491124,325.90791334)(854.98991124,325.85791339)(854.98991317,325.78791658)
\curveto(855.00991122,325.71791353)(855.0249112,325.6479136)(855.03491317,325.57791658)
\lineto(855.09491317,325.36791658)
\curveto(855.20491102,325.01791423)(855.35491087,324.71791453)(855.54491317,324.46791658)
\curveto(855.73491049,324.21791503)(855.97491025,324.01291524)(856.26491317,323.85291658)
\curveto(856.35490987,323.80291545)(856.44490978,323.76291549)(856.53491317,323.73291658)
\curveto(856.6249096,323.70291555)(856.7249095,323.67291558)(856.83491317,323.64291658)
\curveto(856.88490934,323.62291563)(856.93490929,323.61791563)(856.98491317,323.62791658)
\curveto(857.04490918,323.63791561)(857.09990913,323.63291562)(857.14991317,323.61291658)
\curveto(857.18990904,323.60291565)(857.229909,323.59791565)(857.26991317,323.59791658)
\lineto(857.40491317,323.59791658)
\lineto(857.53991317,323.59791658)
\curveto(857.56990866,323.60791564)(857.61990861,323.61291564)(857.68991317,323.61291658)
\curveto(857.76990846,323.63291562)(857.84990838,323.6479156)(857.92991317,323.65791658)
\curveto(858.00990822,323.67791557)(858.08490814,323.70291555)(858.15491317,323.73291658)
\curveto(858.48490774,323.87291538)(858.74990748,324.0479152)(858.94991317,324.25791658)
\curveto(859.15990707,324.47791477)(859.33490689,324.7529145)(859.47491317,325.08291658)
\curveto(859.5249067,325.19291406)(859.55990667,325.30291395)(859.57991317,325.41291658)
\curveto(859.59990663,325.52291373)(859.6249066,325.63291362)(859.65491317,325.74291658)
\curveto(859.67490655,325.78291347)(859.68490654,325.81791343)(859.68491317,325.84791658)
\curveto(859.68490654,325.88791336)(859.68990654,325.92791332)(859.69991317,325.96791658)
\curveto(859.70990652,326.02791322)(859.70990652,326.08791316)(859.69991317,326.14791658)
\curveto(859.69990653,326.20791304)(859.70490652,326.26791298)(859.71491317,326.32791658)
}
}
{
\newrgbcolor{curcolor}{0 0 0}
\pscustom[linestyle=none,fillstyle=solid,fillcolor=curcolor]
{
\newpath
\moveto(869.76116317,323.23791658)
\curveto(869.79115534,323.07791617)(869.77615536,322.94291631)(869.71616317,322.83291658)
\curveto(869.65615548,322.73291652)(869.57615556,322.65791659)(869.47616317,322.60791658)
\curveto(869.42615571,322.58791666)(869.37115576,322.57791667)(869.31116317,322.57791658)
\curveto(869.26115587,322.57791667)(869.20615593,322.56791668)(869.14616317,322.54791658)
\curveto(868.92615621,322.49791675)(868.70615643,322.51291674)(868.48616317,322.59291658)
\curveto(868.27615686,322.66291659)(868.131157,322.7529165)(868.05116317,322.86291658)
\curveto(868.00115713,322.93291632)(867.95615718,323.01291624)(867.91616317,323.10291658)
\curveto(867.87615726,323.20291605)(867.82615731,323.28291597)(867.76616317,323.34291658)
\curveto(867.74615739,323.36291589)(867.72115741,323.38291587)(867.69116317,323.40291658)
\curveto(867.67115746,323.42291583)(867.64115749,323.42791582)(867.60116317,323.41791658)
\curveto(867.49115764,323.38791586)(867.38615775,323.33291592)(867.28616317,323.25291658)
\curveto(867.19615794,323.17291608)(867.10615803,323.10291615)(867.01616317,323.04291658)
\curveto(866.88615825,322.96291629)(866.74615839,322.88791636)(866.59616317,322.81791658)
\curveto(866.44615869,322.75791649)(866.28615885,322.70291655)(866.11616317,322.65291658)
\curveto(866.01615912,322.62291663)(865.90615923,322.60291665)(865.78616317,322.59291658)
\curveto(865.67615946,322.58291667)(865.56615957,322.56791668)(865.45616317,322.54791658)
\curveto(865.40615973,322.53791671)(865.36115977,322.53291672)(865.32116317,322.53291658)
\lineto(865.21616317,322.53291658)
\curveto(865.10616003,322.51291674)(865.00116013,322.51291674)(864.90116317,322.53291658)
\lineto(864.76616317,322.53291658)
\curveto(864.71616042,322.54291671)(864.66616047,322.5479167)(864.61616317,322.54791658)
\curveto(864.56616057,322.5479167)(864.52116061,322.55791669)(864.48116317,322.57791658)
\curveto(864.44116069,322.58791666)(864.40616073,322.59291666)(864.37616317,322.59291658)
\curveto(864.35616078,322.58291667)(864.3311608,322.58291667)(864.30116317,322.59291658)
\lineto(864.06116317,322.65291658)
\curveto(863.98116115,322.66291659)(863.90616123,322.68291657)(863.83616317,322.71291658)
\curveto(863.5361616,322.84291641)(863.29116184,322.98791626)(863.10116317,323.14791658)
\curveto(862.92116221,323.31791593)(862.77116236,323.5529157)(862.65116317,323.85291658)
\curveto(862.56116257,324.07291518)(862.51616262,324.33791491)(862.51616317,324.64791658)
\lineto(862.51616317,324.96291658)
\curveto(862.52616261,325.01291424)(862.5311626,325.06291419)(862.53116317,325.11291658)
\lineto(862.56116317,325.29291658)
\lineto(862.68116317,325.62291658)
\curveto(862.72116241,325.73291352)(862.77116236,325.83291342)(862.83116317,325.92291658)
\curveto(863.01116212,326.21291304)(863.25616188,326.42791282)(863.56616317,326.56791658)
\curveto(863.87616126,326.70791254)(864.21616092,326.83291242)(864.58616317,326.94291658)
\curveto(864.72616041,326.98291227)(864.87116026,327.01291224)(865.02116317,327.03291658)
\curveto(865.17115996,327.0529122)(865.32115981,327.07791217)(865.47116317,327.10791658)
\curveto(865.54115959,327.12791212)(865.60615953,327.13791211)(865.66616317,327.13791658)
\curveto(865.7361594,327.13791211)(865.81115932,327.1479121)(865.89116317,327.16791658)
\curveto(865.96115917,327.18791206)(866.0311591,327.19791205)(866.10116317,327.19791658)
\curveto(866.17115896,327.20791204)(866.24615889,327.22291203)(866.32616317,327.24291658)
\curveto(866.57615856,327.30291195)(866.81115832,327.3529119)(867.03116317,327.39291658)
\curveto(867.25115788,327.44291181)(867.42615771,327.55791169)(867.55616317,327.73791658)
\curveto(867.61615752,327.81791143)(867.66615747,327.91791133)(867.70616317,328.03791658)
\curveto(867.74615739,328.16791108)(867.74615739,328.30791094)(867.70616317,328.45791658)
\curveto(867.64615749,328.69791055)(867.55615758,328.88791036)(867.43616317,329.02791658)
\curveto(867.32615781,329.16791008)(867.16615797,329.27790997)(866.95616317,329.35791658)
\curveto(866.8361583,329.40790984)(866.69115844,329.44290981)(866.52116317,329.46291658)
\curveto(866.36115877,329.48290977)(866.19115894,329.49290976)(866.01116317,329.49291658)
\curveto(865.8311593,329.49290976)(865.65615948,329.48290977)(865.48616317,329.46291658)
\curveto(865.31615982,329.44290981)(865.17115996,329.41290984)(865.05116317,329.37291658)
\curveto(864.88116025,329.31290994)(864.71616042,329.22791002)(864.55616317,329.11791658)
\curveto(864.47616066,329.05791019)(864.40116073,328.97791027)(864.33116317,328.87791658)
\curveto(864.27116086,328.78791046)(864.21616092,328.68791056)(864.16616317,328.57791658)
\curveto(864.136161,328.49791075)(864.10616103,328.41291084)(864.07616317,328.32291658)
\curveto(864.05616108,328.23291102)(864.01116112,328.16291109)(863.94116317,328.11291658)
\curveto(863.90116123,328.08291117)(863.8311613,328.05791119)(863.73116317,328.03791658)
\curveto(863.64116149,328.02791122)(863.54616159,328.02291123)(863.44616317,328.02291658)
\curveto(863.34616179,328.02291123)(863.24616189,328.02791122)(863.14616317,328.03791658)
\curveto(863.05616208,328.05791119)(862.99116214,328.08291117)(862.95116317,328.11291658)
\curveto(862.91116222,328.14291111)(862.88116225,328.19291106)(862.86116317,328.26291658)
\curveto(862.84116229,328.33291092)(862.84116229,328.40791084)(862.86116317,328.48791658)
\curveto(862.89116224,328.61791063)(862.92116221,328.73791051)(862.95116317,328.84791658)
\curveto(862.99116214,328.96791028)(863.0361621,329.08291017)(863.08616317,329.19291658)
\curveto(863.27616186,329.54290971)(863.51616162,329.81290944)(863.80616317,330.00291658)
\curveto(864.09616104,330.20290905)(864.45616068,330.36290889)(864.88616317,330.48291658)
\curveto(864.98616015,330.50290875)(865.08616005,330.51790873)(865.18616317,330.52791658)
\curveto(865.29615984,330.53790871)(865.40615973,330.5529087)(865.51616317,330.57291658)
\curveto(865.55615958,330.58290867)(865.62115951,330.58290867)(865.71116317,330.57291658)
\curveto(865.80115933,330.57290868)(865.85615928,330.58290867)(865.87616317,330.60291658)
\curveto(866.57615856,330.61290864)(867.18615795,330.53290872)(867.70616317,330.36291658)
\curveto(868.22615691,330.19290906)(868.59115654,329.86790938)(868.80116317,329.38791658)
\curveto(868.89115624,329.18791006)(868.94115619,328.9529103)(868.95116317,328.68291658)
\curveto(868.97115616,328.42291083)(868.98115615,328.1479111)(868.98116317,327.85791658)
\lineto(868.98116317,324.54291658)
\curveto(868.98115615,324.40291485)(868.98615615,324.26791498)(868.99616317,324.13791658)
\curveto(869.00615613,324.00791524)(869.0361561,323.90291535)(869.08616317,323.82291658)
\curveto(869.136156,323.7529155)(869.20115593,323.70291555)(869.28116317,323.67291658)
\curveto(869.37115576,323.63291562)(869.45615568,323.60291565)(869.53616317,323.58291658)
\curveto(869.61615552,323.57291568)(869.67615546,323.52791572)(869.71616317,323.44791658)
\curveto(869.7361554,323.41791583)(869.74615539,323.38791586)(869.74616317,323.35791658)
\curveto(869.74615539,323.32791592)(869.75115538,323.28791596)(869.76116317,323.23791658)
\moveto(867.61616317,324.90291658)
\curveto(867.67615746,325.04291421)(867.70615743,325.20291405)(867.70616317,325.38291658)
\curveto(867.71615742,325.57291368)(867.72115741,325.76791348)(867.72116317,325.96791658)
\curveto(867.72115741,326.07791317)(867.71615742,326.17791307)(867.70616317,326.26791658)
\curveto(867.69615744,326.35791289)(867.65615748,326.42791282)(867.58616317,326.47791658)
\curveto(867.55615758,326.49791275)(867.48615765,326.50791274)(867.37616317,326.50791658)
\curveto(867.35615778,326.48791276)(867.32115781,326.47791277)(867.27116317,326.47791658)
\curveto(867.22115791,326.47791277)(867.17615796,326.46791278)(867.13616317,326.44791658)
\curveto(867.05615808,326.42791282)(866.96615817,326.40791284)(866.86616317,326.38791658)
\lineto(866.56616317,326.32791658)
\curveto(866.5361586,326.32791292)(866.50115863,326.32291293)(866.46116317,326.31291658)
\lineto(866.35616317,326.31291658)
\curveto(866.20615893,326.27291298)(866.04115909,326.247913)(865.86116317,326.23791658)
\curveto(865.69115944,326.23791301)(865.5311596,326.21791303)(865.38116317,326.17791658)
\curveto(865.30115983,326.15791309)(865.22615991,326.13791311)(865.15616317,326.11791658)
\curveto(865.09616004,326.10791314)(865.02616011,326.09291316)(864.94616317,326.07291658)
\curveto(864.78616035,326.02291323)(864.6361605,325.95791329)(864.49616317,325.87791658)
\curveto(864.35616078,325.80791344)(864.2361609,325.71791353)(864.13616317,325.60791658)
\curveto(864.0361611,325.49791375)(863.96116117,325.36291389)(863.91116317,325.20291658)
\curveto(863.86116127,325.0529142)(863.84116129,324.86791438)(863.85116317,324.64791658)
\curveto(863.85116128,324.5479147)(863.86616127,324.4529148)(863.89616317,324.36291658)
\curveto(863.9361612,324.28291497)(863.98116115,324.20791504)(864.03116317,324.13791658)
\curveto(864.11116102,324.02791522)(864.21616092,323.93291532)(864.34616317,323.85291658)
\curveto(864.47616066,323.78291547)(864.61616052,323.72291553)(864.76616317,323.67291658)
\curveto(864.81616032,323.66291559)(864.86616027,323.65791559)(864.91616317,323.65791658)
\curveto(864.96616017,323.65791559)(865.01616012,323.6529156)(865.06616317,323.64291658)
\curveto(865.13616,323.62291563)(865.22115991,323.60791564)(865.32116317,323.59791658)
\curveto(865.4311597,323.59791565)(865.52115961,323.60791564)(865.59116317,323.62791658)
\curveto(865.65115948,323.6479156)(865.71115942,323.6529156)(865.77116317,323.64291658)
\curveto(865.8311593,323.64291561)(865.89115924,323.6529156)(865.95116317,323.67291658)
\curveto(866.0311591,323.69291556)(866.10615903,323.70791554)(866.17616317,323.71791658)
\curveto(866.25615888,323.72791552)(866.3311588,323.7479155)(866.40116317,323.77791658)
\curveto(866.69115844,323.89791535)(866.9361582,324.04291521)(867.13616317,324.21291658)
\curveto(867.34615779,324.38291487)(867.50615763,324.61291464)(867.61616317,324.90291658)
}
}
{
\newrgbcolor{curcolor}{0.90196079 0.90196079 0.90196079}
\pscustom[linestyle=none,fillstyle=solid,fillcolor=curcolor]
{
\newpath
\moveto(798.51865829,333.3929532)
\lineto(813.51865829,333.3929532)
\lineto(813.51865829,318.3929532)
\lineto(798.51865829,318.3929532)
\closepath
}
}
{
\newrgbcolor{curcolor}{0 0 0}
\pscustom[linestyle=none,fillstyle=solid,fillcolor=curcolor]
{
\newpath
\moveto(822.4518663,310.56721102)
\curveto(823.4318598,310.58720006)(824.25185898,310.42720022)(824.9118663,310.08721102)
\curveto(825.58185765,309.75720089)(826.10185713,309.29720135)(826.4718663,308.70721102)
\curveto(826.57185666,308.5472021)(826.65185658,308.39220226)(826.7118663,308.24221102)
\curveto(826.78185645,308.10220255)(826.84685638,307.93220272)(826.9068663,307.73221102)
\curveto(826.9268563,307.68220297)(826.94685628,307.61220304)(826.9668663,307.52221102)
\curveto(826.98685624,307.44220321)(826.98185625,307.36720328)(826.9518663,307.29721102)
\curveto(826.9318563,307.23720341)(826.89185634,307.19720345)(826.8318663,307.17721102)
\curveto(826.78185645,307.16720348)(826.7268565,307.1522035)(826.6668663,307.13221102)
\lineto(826.5168663,307.13221102)
\curveto(826.48685674,307.12220353)(826.44685678,307.11720353)(826.3968663,307.11721102)
\lineto(826.2768663,307.11721102)
\curveto(826.13685709,307.11720353)(826.00685722,307.12220353)(825.8868663,307.13221102)
\curveto(825.77685745,307.1522035)(825.69685753,307.20220345)(825.6468663,307.28221102)
\curveto(825.57685765,307.38220327)(825.52185771,307.49720315)(825.4818663,307.62721102)
\curveto(825.44185779,307.75720289)(825.38685784,307.87720277)(825.3168663,307.98721102)
\curveto(825.18685804,308.20720244)(825.03685819,308.39720225)(824.8668663,308.55721102)
\curveto(824.70685852,308.71720193)(824.51685871,308.86720178)(824.2968663,309.00721102)
\curveto(824.17685905,309.08720156)(824.04185919,309.1472015)(823.8918663,309.18721102)
\curveto(823.75185948,309.22720142)(823.60685962,309.26720138)(823.4568663,309.30721102)
\curveto(823.34685988,309.33720131)(823.22186001,309.35720129)(823.0818663,309.36721102)
\curveto(822.94186029,309.38720126)(822.79186044,309.39720125)(822.6318663,309.39721102)
\curveto(822.48186075,309.39720125)(822.3318609,309.38720126)(822.1818663,309.36721102)
\curveto(822.04186119,309.35720129)(821.92186131,309.33720131)(821.8218663,309.30721102)
\curveto(821.72186151,309.28720136)(821.6268616,309.26720138)(821.5368663,309.24721102)
\curveto(821.44686178,309.22720142)(821.35686187,309.19720145)(821.2668663,309.15721102)
\curveto(820.4268628,308.80720184)(819.82186341,308.20720244)(819.4518663,307.35721102)
\curveto(819.38186385,307.21720343)(819.32186391,307.06720358)(819.2718663,306.90721102)
\curveto(819.231864,306.75720389)(819.18686404,306.60220405)(819.1368663,306.44221102)
\curveto(819.11686411,306.38220427)(819.10686412,306.31720433)(819.1068663,306.24721102)
\curveto(819.10686412,306.18720446)(819.09686413,306.12720452)(819.0768663,306.06721102)
\curveto(819.06686416,306.02720462)(819.06186417,305.99220466)(819.0618663,305.96221102)
\curveto(819.06186417,305.93220472)(819.05686417,305.89720475)(819.0468663,305.85721102)
\curveto(819.0268642,305.7472049)(819.01186422,305.63220502)(819.0018663,305.51221102)
\lineto(819.0018663,305.16721102)
\curveto(819.00186423,305.09720555)(818.99686423,305.02220563)(818.9868663,304.94221102)
\curveto(818.98686424,304.87220578)(818.99186424,304.80720584)(819.0018663,304.74721102)
\lineto(819.0018663,304.59721102)
\curveto(819.02186421,304.52720612)(819.0268642,304.45720619)(819.0168663,304.38721102)
\curveto(819.01686421,304.31720633)(819.0268642,304.2472064)(819.0468663,304.17721102)
\curveto(819.06686416,304.11720653)(819.07186416,304.05720659)(819.0618663,303.99721102)
\curveto(819.06186417,303.93720671)(819.07186416,303.88220677)(819.0918663,303.83221102)
\curveto(819.12186411,303.70220695)(819.14686408,303.57220708)(819.1668663,303.44221102)
\curveto(819.19686403,303.32220733)(819.231864,303.20220745)(819.2718663,303.08221102)
\curveto(819.44186379,302.58220807)(819.66186357,302.1522085)(819.9318663,301.79221102)
\curveto(820.20186303,301.44220921)(820.55686267,301.1522095)(820.9968663,300.92221102)
\curveto(821.13686209,300.8522098)(821.27686195,300.79720985)(821.4168663,300.75721102)
\curveto(821.56686166,300.71720993)(821.7268615,300.67220998)(821.8968663,300.62221102)
\curveto(821.96686126,300.60221005)(822.0318612,300.59221006)(822.0918663,300.59221102)
\curveto(822.15186108,300.60221005)(822.22186101,300.59721005)(822.3018663,300.57721102)
\curveto(822.35186088,300.56721008)(822.44186079,300.55721009)(822.5718663,300.54721102)
\curveto(822.70186053,300.5472101)(822.79686043,300.55721009)(822.8568663,300.57721102)
\lineto(822.9618663,300.57721102)
\curveto(823.00186023,300.58721006)(823.04186019,300.58721006)(823.0818663,300.57721102)
\curveto(823.12186011,300.57721007)(823.16186007,300.58721006)(823.2018663,300.60721102)
\curveto(823.30185993,300.62721002)(823.39685983,300.64221001)(823.4868663,300.65221102)
\curveto(823.58685964,300.67220998)(823.68185955,300.70220995)(823.7718663,300.74221102)
\curveto(824.55185868,301.06220959)(825.10185813,301.58720906)(825.4218663,302.31721102)
\curveto(825.50185773,302.49720815)(825.57685765,302.71220794)(825.6468663,302.96221102)
\curveto(825.66685756,303.0522076)(825.68185755,303.14220751)(825.6918663,303.23221102)
\curveto(825.71185752,303.33220732)(825.74685748,303.42220723)(825.7968663,303.50221102)
\curveto(825.84685738,303.58220707)(825.9268573,303.62720702)(826.0368663,303.63721102)
\curveto(826.14685708,303.647207)(826.26685696,303.652207)(826.3968663,303.65221102)
\lineto(826.5468663,303.65221102)
\curveto(826.59685663,303.652207)(826.64185659,303.647207)(826.6818663,303.63721102)
\lineto(826.7868663,303.63721102)
\lineto(826.8768663,303.60721102)
\curveto(826.91685631,303.60720704)(826.94685628,303.59720705)(826.9668663,303.57721102)
\curveto(827.03685619,303.53720711)(827.07685615,303.46220719)(827.0868663,303.35221102)
\curveto(827.09685613,303.2522074)(827.08685614,303.1522075)(827.0568663,303.05221102)
\curveto(826.99685623,302.82220783)(826.94185629,302.60220805)(826.8918663,302.39221102)
\curveto(826.84185639,302.18220847)(826.76685646,301.98220867)(826.6668663,301.79221102)
\curveto(826.58685664,301.66220899)(826.51185672,301.53720911)(826.4418663,301.41721102)
\curveto(826.38185685,301.29720935)(826.31185692,301.17720947)(826.2318663,301.05721102)
\curveto(826.05185718,300.79720985)(825.8268574,300.55721009)(825.5568663,300.33721102)
\curveto(825.29685793,300.12721052)(825.01185822,299.9522107)(824.7018663,299.81221102)
\curveto(824.59185864,299.76221089)(824.48185875,299.72221093)(824.3718663,299.69221102)
\curveto(824.27185896,299.66221099)(824.16685906,299.62721102)(824.0568663,299.58721102)
\curveto(823.94685928,299.5472111)(823.8318594,299.52221113)(823.7118663,299.51221102)
\curveto(823.60185963,299.49221116)(823.48685974,299.47221118)(823.3668663,299.45221102)
\curveto(823.31685991,299.43221122)(823.27185996,299.42721122)(823.2318663,299.43721102)
\curveto(823.19186004,299.43721121)(823.15186008,299.43221122)(823.1118663,299.42221102)
\curveto(823.05186018,299.41221124)(822.99186024,299.40721124)(822.9318663,299.40721102)
\curveto(822.87186036,299.40721124)(822.80686042,299.40221125)(822.7368663,299.39221102)
\curveto(822.70686052,299.38221127)(822.63686059,299.38221127)(822.5268663,299.39221102)
\curveto(822.4268608,299.39221126)(822.36186087,299.39721125)(822.3318663,299.40721102)
\curveto(822.28186095,299.41721123)(822.231861,299.42221123)(822.1818663,299.42221102)
\curveto(822.14186109,299.41221124)(822.09686113,299.41221124)(822.0468663,299.42221102)
\lineto(821.8968663,299.42221102)
\curveto(821.81686141,299.44221121)(821.74186149,299.45721119)(821.6718663,299.46721102)
\curveto(821.60186163,299.46721118)(821.5268617,299.47721117)(821.4468663,299.49721102)
\lineto(821.1768663,299.55721102)
\curveto(821.08686214,299.56721108)(821.00186223,299.58721106)(820.9218663,299.61721102)
\curveto(820.71186252,299.67721097)(820.52186271,299.7522109)(820.3518663,299.84221102)
\curveto(819.72186351,300.11221054)(819.21186402,300.49721015)(818.8218663,300.99721102)
\curveto(818.4318648,301.49720915)(818.12186511,302.08720856)(817.8918663,302.76721102)
\curveto(817.85186538,302.88720776)(817.81686541,303.01220764)(817.7868663,303.14221102)
\curveto(817.76686546,303.27220738)(817.74186549,303.40720724)(817.7118663,303.54721102)
\curveto(817.69186554,303.59720705)(817.68186555,303.64220701)(817.6818663,303.68221102)
\curveto(817.69186554,303.72220693)(817.69186554,303.76720688)(817.6818663,303.81721102)
\curveto(817.66186557,303.90720674)(817.64686558,304.00220665)(817.6368663,304.10221102)
\curveto(817.63686559,304.20220645)(817.6268656,304.29720635)(817.6068663,304.38721102)
\lineto(817.6068663,304.67221102)
\curveto(817.58686564,304.72220593)(817.57686565,304.80720584)(817.5768663,304.92721102)
\curveto(817.57686565,305.0472056)(817.58686564,305.13220552)(817.6068663,305.18221102)
\curveto(817.61686561,305.21220544)(817.61686561,305.24220541)(817.6068663,305.27221102)
\curveto(817.59686563,305.31220534)(817.59686563,305.34220531)(817.6068663,305.36221102)
\lineto(817.6068663,305.49721102)
\curveto(817.61686561,305.57720507)(817.62186561,305.65720499)(817.6218663,305.73721102)
\curveto(817.6318656,305.82720482)(817.64686558,305.91220474)(817.6668663,305.99221102)
\curveto(817.68686554,306.0522046)(817.69686553,306.11220454)(817.6968663,306.17221102)
\curveto(817.69686553,306.24220441)(817.70686552,306.31220434)(817.7268663,306.38221102)
\curveto(817.77686545,306.5522041)(817.81686541,306.71720393)(817.8468663,306.87721102)
\curveto(817.87686535,307.03720361)(817.92186531,307.18720346)(817.9818663,307.32721102)
\lineto(818.1318663,307.71721102)
\curveto(818.19186504,307.85720279)(818.25686497,307.98220267)(818.3268663,308.09221102)
\curveto(818.47686475,308.3522023)(818.6268646,308.58720206)(818.7768663,308.79721102)
\curveto(818.80686442,308.8472018)(818.84186439,308.88720176)(818.8818663,308.91721102)
\curveto(818.9318643,308.95720169)(818.97186426,309.00220165)(819.0018663,309.05221102)
\curveto(819.06186417,309.13220152)(819.12186411,309.20220145)(819.1818663,309.26221102)
\lineto(819.3918663,309.44221102)
\curveto(819.45186378,309.49220116)(819.50686372,309.53720111)(819.5568663,309.57721102)
\curveto(819.61686361,309.62720102)(819.68186355,309.67720097)(819.7518663,309.72721102)
\curveto(819.90186333,309.83720081)(820.05686317,309.93220072)(820.2168663,310.01221102)
\curveto(820.38686284,310.09220056)(820.56186267,310.17220048)(820.7418663,310.25221102)
\curveto(820.85186238,310.30220035)(820.96686226,310.33720031)(821.0868663,310.35721102)
\curveto(821.21686201,310.38720026)(821.34186189,310.42220023)(821.4618663,310.46221102)
\curveto(821.5318617,310.47220018)(821.59686163,310.48220017)(821.6568663,310.49221102)
\lineto(821.8368663,310.52221102)
\curveto(821.91686131,310.53220012)(821.99186124,310.53720011)(822.0618663,310.53721102)
\curveto(822.14186109,310.5472001)(822.22186101,310.55720009)(822.3018663,310.56721102)
\curveto(822.32186091,310.57720007)(822.34686088,310.57720007)(822.3768663,310.56721102)
\curveto(822.40686082,310.55720009)(822.4318608,310.55720009)(822.4518663,310.56721102)
}
}
{
\newrgbcolor{curcolor}{0 0 0}
\pscustom[linestyle=none,fillstyle=solid,fillcolor=curcolor]
{
\newpath
\moveto(835.57171005,300.20221102)
\curveto(835.60170222,300.04221061)(835.58670223,299.90721074)(835.52671005,299.79721102)
\curveto(835.46670235,299.69721095)(835.38670243,299.62221103)(835.28671005,299.57221102)
\curveto(835.23670258,299.5522111)(835.18170264,299.54221111)(835.12171005,299.54221102)
\curveto(835.07170275,299.54221111)(835.0167028,299.53221112)(834.95671005,299.51221102)
\curveto(834.73670308,299.46221119)(834.5167033,299.47721117)(834.29671005,299.55721102)
\curveto(834.08670373,299.62721102)(833.94170388,299.71721093)(833.86171005,299.82721102)
\curveto(833.81170401,299.89721075)(833.76670405,299.97721067)(833.72671005,300.06721102)
\curveto(833.68670413,300.16721048)(833.63670418,300.2472104)(833.57671005,300.30721102)
\curveto(833.55670426,300.32721032)(833.53170429,300.3472103)(833.50171005,300.36721102)
\curveto(833.48170434,300.38721026)(833.45170437,300.39221026)(833.41171005,300.38221102)
\curveto(833.30170452,300.3522103)(833.19670462,300.29721035)(833.09671005,300.21721102)
\curveto(833.00670481,300.13721051)(832.9167049,300.06721058)(832.82671005,300.00721102)
\curveto(832.69670512,299.92721072)(832.55670526,299.8522108)(832.40671005,299.78221102)
\curveto(832.25670556,299.72221093)(832.09670572,299.66721098)(831.92671005,299.61721102)
\curveto(831.82670599,299.58721106)(831.7167061,299.56721108)(831.59671005,299.55721102)
\curveto(831.48670633,299.5472111)(831.37670644,299.53221112)(831.26671005,299.51221102)
\curveto(831.2167066,299.50221115)(831.17170665,299.49721115)(831.13171005,299.49721102)
\lineto(831.02671005,299.49721102)
\curveto(830.9167069,299.47721117)(830.81170701,299.47721117)(830.71171005,299.49721102)
\lineto(830.57671005,299.49721102)
\curveto(830.52670729,299.50721114)(830.47670734,299.51221114)(830.42671005,299.51221102)
\curveto(830.37670744,299.51221114)(830.33170749,299.52221113)(830.29171005,299.54221102)
\curveto(830.25170757,299.5522111)(830.2167076,299.55721109)(830.18671005,299.55721102)
\curveto(830.16670765,299.5472111)(830.14170768,299.5472111)(830.11171005,299.55721102)
\lineto(829.87171005,299.61721102)
\curveto(829.79170803,299.62721102)(829.7167081,299.647211)(829.64671005,299.67721102)
\curveto(829.34670847,299.80721084)(829.10170872,299.9522107)(828.91171005,300.11221102)
\curveto(828.73170909,300.28221037)(828.58170924,300.51721013)(828.46171005,300.81721102)
\curveto(828.37170945,301.03720961)(828.32670949,301.30220935)(828.32671005,301.61221102)
\lineto(828.32671005,301.92721102)
\curveto(828.33670948,301.97720867)(828.34170948,302.02720862)(828.34171005,302.07721102)
\lineto(828.37171005,302.25721102)
\lineto(828.49171005,302.58721102)
\curveto(828.53170929,302.69720795)(828.58170924,302.79720785)(828.64171005,302.88721102)
\curveto(828.821709,303.17720747)(829.06670875,303.39220726)(829.37671005,303.53221102)
\curveto(829.68670813,303.67220698)(830.02670779,303.79720685)(830.39671005,303.90721102)
\curveto(830.53670728,303.9472067)(830.68170714,303.97720667)(830.83171005,303.99721102)
\curveto(830.98170684,304.01720663)(831.13170669,304.04220661)(831.28171005,304.07221102)
\curveto(831.35170647,304.09220656)(831.4167064,304.10220655)(831.47671005,304.10221102)
\curveto(831.54670627,304.10220655)(831.6217062,304.11220654)(831.70171005,304.13221102)
\curveto(831.77170605,304.1522065)(831.84170598,304.16220649)(831.91171005,304.16221102)
\curveto(831.98170584,304.17220648)(832.05670576,304.18720646)(832.13671005,304.20721102)
\curveto(832.38670543,304.26720638)(832.6217052,304.31720633)(832.84171005,304.35721102)
\curveto(833.06170476,304.40720624)(833.23670458,304.52220613)(833.36671005,304.70221102)
\curveto(833.42670439,304.78220587)(833.47670434,304.88220577)(833.51671005,305.00221102)
\curveto(833.55670426,305.13220552)(833.55670426,305.27220538)(833.51671005,305.42221102)
\curveto(833.45670436,305.66220499)(833.36670445,305.8522048)(833.24671005,305.99221102)
\curveto(833.13670468,306.13220452)(832.97670484,306.24220441)(832.76671005,306.32221102)
\curveto(832.64670517,306.37220428)(832.50170532,306.40720424)(832.33171005,306.42721102)
\curveto(832.17170565,306.4472042)(832.00170582,306.45720419)(831.82171005,306.45721102)
\curveto(831.64170618,306.45720419)(831.46670635,306.4472042)(831.29671005,306.42721102)
\curveto(831.12670669,306.40720424)(830.98170684,306.37720427)(830.86171005,306.33721102)
\curveto(830.69170713,306.27720437)(830.52670729,306.19220446)(830.36671005,306.08221102)
\curveto(830.28670753,306.02220463)(830.21170761,305.94220471)(830.14171005,305.84221102)
\curveto(830.08170774,305.7522049)(830.02670779,305.652205)(829.97671005,305.54221102)
\curveto(829.94670787,305.46220519)(829.9167079,305.37720527)(829.88671005,305.28721102)
\curveto(829.86670795,305.19720545)(829.821708,305.12720552)(829.75171005,305.07721102)
\curveto(829.71170811,305.0472056)(829.64170818,305.02220563)(829.54171005,305.00221102)
\curveto(829.45170837,304.99220566)(829.35670846,304.98720566)(829.25671005,304.98721102)
\curveto(829.15670866,304.98720566)(829.05670876,304.99220566)(828.95671005,305.00221102)
\curveto(828.86670895,305.02220563)(828.80170902,305.0472056)(828.76171005,305.07721102)
\curveto(828.7217091,305.10720554)(828.69170913,305.15720549)(828.67171005,305.22721102)
\curveto(828.65170917,305.29720535)(828.65170917,305.37220528)(828.67171005,305.45221102)
\curveto(828.70170912,305.58220507)(828.73170909,305.70220495)(828.76171005,305.81221102)
\curveto(828.80170902,305.93220472)(828.84670897,306.0472046)(828.89671005,306.15721102)
\curveto(829.08670873,306.50720414)(829.32670849,306.77720387)(829.61671005,306.96721102)
\curveto(829.90670791,307.16720348)(830.26670755,307.32720332)(830.69671005,307.44721102)
\curveto(830.79670702,307.46720318)(830.89670692,307.48220317)(830.99671005,307.49221102)
\curveto(831.10670671,307.50220315)(831.2167066,307.51720313)(831.32671005,307.53721102)
\curveto(831.36670645,307.5472031)(831.43170639,307.5472031)(831.52171005,307.53721102)
\curveto(831.61170621,307.53720311)(831.66670615,307.5472031)(831.68671005,307.56721102)
\curveto(832.38670543,307.57720307)(832.99670482,307.49720315)(833.51671005,307.32721102)
\curveto(834.03670378,307.15720349)(834.40170342,306.83220382)(834.61171005,306.35221102)
\curveto(834.70170312,306.1522045)(834.75170307,305.91720473)(834.76171005,305.64721102)
\curveto(834.78170304,305.38720526)(834.79170303,305.11220554)(834.79171005,304.82221102)
\lineto(834.79171005,301.50721102)
\curveto(834.79170303,301.36720928)(834.79670302,301.23220942)(834.80671005,301.10221102)
\curveto(834.816703,300.97220968)(834.84670297,300.86720978)(834.89671005,300.78721102)
\curveto(834.94670287,300.71720993)(835.01170281,300.66720998)(835.09171005,300.63721102)
\curveto(835.18170264,300.59721005)(835.26670255,300.56721008)(835.34671005,300.54721102)
\curveto(835.42670239,300.53721011)(835.48670233,300.49221016)(835.52671005,300.41221102)
\curveto(835.54670227,300.38221027)(835.55670226,300.3522103)(835.55671005,300.32221102)
\curveto(835.55670226,300.29221036)(835.56170226,300.2522104)(835.57171005,300.20221102)
\moveto(833.42671005,301.86721102)
\curveto(833.48670433,302.00720864)(833.5167043,302.16720848)(833.51671005,302.34721102)
\curveto(833.52670429,302.53720811)(833.53170429,302.73220792)(833.53171005,302.93221102)
\curveto(833.53170429,303.04220761)(833.52670429,303.14220751)(833.51671005,303.23221102)
\curveto(833.50670431,303.32220733)(833.46670435,303.39220726)(833.39671005,303.44221102)
\curveto(833.36670445,303.46220719)(833.29670452,303.47220718)(833.18671005,303.47221102)
\curveto(833.16670465,303.4522072)(833.13170469,303.44220721)(833.08171005,303.44221102)
\curveto(833.03170479,303.44220721)(832.98670483,303.43220722)(832.94671005,303.41221102)
\curveto(832.86670495,303.39220726)(832.77670504,303.37220728)(832.67671005,303.35221102)
\lineto(832.37671005,303.29221102)
\curveto(832.34670547,303.29220736)(832.31170551,303.28720736)(832.27171005,303.27721102)
\lineto(832.16671005,303.27721102)
\curveto(832.0167058,303.23720741)(831.85170597,303.21220744)(831.67171005,303.20221102)
\curveto(831.50170632,303.20220745)(831.34170648,303.18220747)(831.19171005,303.14221102)
\curveto(831.11170671,303.12220753)(831.03670678,303.10220755)(830.96671005,303.08221102)
\curveto(830.90670691,303.07220758)(830.83670698,303.05720759)(830.75671005,303.03721102)
\curveto(830.59670722,302.98720766)(830.44670737,302.92220773)(830.30671005,302.84221102)
\curveto(830.16670765,302.77220788)(830.04670777,302.68220797)(829.94671005,302.57221102)
\curveto(829.84670797,302.46220819)(829.77170805,302.32720832)(829.72171005,302.16721102)
\curveto(829.67170815,302.01720863)(829.65170817,301.83220882)(829.66171005,301.61221102)
\curveto(829.66170816,301.51220914)(829.67670814,301.41720923)(829.70671005,301.32721102)
\curveto(829.74670807,301.2472094)(829.79170803,301.17220948)(829.84171005,301.10221102)
\curveto(829.9217079,300.99220966)(830.02670779,300.89720975)(830.15671005,300.81721102)
\curveto(830.28670753,300.7472099)(830.42670739,300.68720996)(830.57671005,300.63721102)
\curveto(830.62670719,300.62721002)(830.67670714,300.62221003)(830.72671005,300.62221102)
\curveto(830.77670704,300.62221003)(830.82670699,300.61721003)(830.87671005,300.60721102)
\curveto(830.94670687,300.58721006)(831.03170679,300.57221008)(831.13171005,300.56221102)
\curveto(831.24170658,300.56221009)(831.33170649,300.57221008)(831.40171005,300.59221102)
\curveto(831.46170636,300.61221004)(831.5217063,300.61721003)(831.58171005,300.60721102)
\curveto(831.64170618,300.60721004)(831.70170612,300.61721003)(831.76171005,300.63721102)
\curveto(831.84170598,300.65720999)(831.9167059,300.67220998)(831.98671005,300.68221102)
\curveto(832.06670575,300.69220996)(832.14170568,300.71220994)(832.21171005,300.74221102)
\curveto(832.50170532,300.86220979)(832.74670507,301.00720964)(832.94671005,301.17721102)
\curveto(833.15670466,301.3472093)(833.3167045,301.57720907)(833.42671005,301.86721102)
}
}
{
\newrgbcolor{curcolor}{0 0 0}
\pscustom[linestyle=none,fillstyle=solid,fillcolor=curcolor]
{
\newpath
\moveto(840.38835067,307.55221102)
\curveto(840.61834588,307.5522031)(840.74834575,307.49220316)(840.77835067,307.37221102)
\curveto(840.80834569,307.26220339)(840.82334568,307.09720355)(840.82335067,306.87721102)
\lineto(840.82335067,306.59221102)
\curveto(840.82334568,306.50220415)(840.7983457,306.42720422)(840.74835067,306.36721102)
\curveto(840.68834581,306.28720436)(840.6033459,306.24220441)(840.49335067,306.23221102)
\curveto(840.38334612,306.23220442)(840.27334623,306.21720443)(840.16335067,306.18721102)
\curveto(840.02334648,306.15720449)(839.88834661,306.12720452)(839.75835067,306.09721102)
\curveto(839.63834686,306.06720458)(839.52334698,306.02720462)(839.41335067,305.97721102)
\curveto(839.12334738,305.8472048)(838.88834761,305.66720498)(838.70835067,305.43721102)
\curveto(838.52834797,305.21720543)(838.37334813,304.96220569)(838.24335067,304.67221102)
\curveto(838.2033483,304.56220609)(838.17334833,304.4472062)(838.15335067,304.32721102)
\curveto(838.13334837,304.21720643)(838.10834839,304.10220655)(838.07835067,303.98221102)
\curveto(838.06834843,303.93220672)(838.06334844,303.88220677)(838.06335067,303.83221102)
\curveto(838.07334843,303.78220687)(838.07334843,303.73220692)(838.06335067,303.68221102)
\curveto(838.03334847,303.56220709)(838.01834848,303.42220723)(838.01835067,303.26221102)
\curveto(838.02834847,303.11220754)(838.03334847,302.96720768)(838.03335067,302.82721102)
\lineto(838.03335067,300.98221102)
\lineto(838.03335067,300.63721102)
\curveto(838.03334847,300.51721013)(838.02834847,300.40221025)(838.01835067,300.29221102)
\curveto(838.00834849,300.18221047)(838.0033485,300.08721056)(838.00335067,300.00721102)
\curveto(838.01334849,299.92721072)(837.99334851,299.85721079)(837.94335067,299.79721102)
\curveto(837.89334861,299.72721092)(837.81334869,299.68721096)(837.70335067,299.67721102)
\curveto(837.6033489,299.66721098)(837.49334901,299.66221099)(837.37335067,299.66221102)
\lineto(837.10335067,299.66221102)
\curveto(837.05334945,299.68221097)(837.0033495,299.69721095)(836.95335067,299.70721102)
\curveto(836.91334959,299.72721092)(836.88334962,299.7522109)(836.86335067,299.78221102)
\curveto(836.81334969,299.8522108)(836.78334972,299.93721071)(836.77335067,300.03721102)
\lineto(836.77335067,300.36721102)
\lineto(836.77335067,301.52221102)
\lineto(836.77335067,305.67721102)
\lineto(836.77335067,306.71221102)
\lineto(836.77335067,307.01221102)
\curveto(836.78334972,307.11220354)(836.81334969,307.19720345)(836.86335067,307.26721102)
\curveto(836.89334961,307.30720334)(836.94334956,307.33720331)(837.01335067,307.35721102)
\curveto(837.09334941,307.37720327)(837.17834932,307.38720326)(837.26835067,307.38721102)
\curveto(837.35834914,307.39720325)(837.44834905,307.39720325)(837.53835067,307.38721102)
\curveto(837.62834887,307.37720327)(837.6983488,307.36220329)(837.74835067,307.34221102)
\curveto(837.82834867,307.31220334)(837.87834862,307.2522034)(837.89835067,307.16221102)
\curveto(837.92834857,307.08220357)(837.94334856,306.99220366)(837.94335067,306.89221102)
\lineto(837.94335067,306.59221102)
\curveto(837.94334856,306.49220416)(837.96334854,306.40220425)(838.00335067,306.32221102)
\curveto(838.01334849,306.30220435)(838.02334848,306.28720436)(838.03335067,306.27721102)
\lineto(838.07835067,306.23221102)
\curveto(838.18834831,306.23220442)(838.27834822,306.27720437)(838.34835067,306.36721102)
\curveto(838.41834808,306.46720418)(838.47834802,306.5472041)(838.52835067,306.60721102)
\lineto(838.61835067,306.69721102)
\curveto(838.70834779,306.80720384)(838.83334767,306.92220373)(838.99335067,307.04221102)
\curveto(839.15334735,307.16220349)(839.3033472,307.2522034)(839.44335067,307.31221102)
\curveto(839.53334697,307.36220329)(839.62834687,307.39720325)(839.72835067,307.41721102)
\curveto(839.82834667,307.4472032)(839.93334657,307.47720317)(840.04335067,307.50721102)
\curveto(840.1033464,307.51720313)(840.16334634,307.52220313)(840.22335067,307.52221102)
\curveto(840.28334622,307.53220312)(840.33834616,307.54220311)(840.38835067,307.55221102)
}
}
{
\newrgbcolor{curcolor}{0 0 0}
\pscustom[linestyle=none,fillstyle=solid,fillcolor=curcolor]
{
\newpath
\moveto(845.3981163,307.55221102)
\curveto(845.62811151,307.5522031)(845.75811138,307.49220316)(845.7881163,307.37221102)
\curveto(845.81811132,307.26220339)(845.8331113,307.09720355)(845.8331163,306.87721102)
\lineto(845.8331163,306.59221102)
\curveto(845.8331113,306.50220415)(845.80811133,306.42720422)(845.7581163,306.36721102)
\curveto(845.69811144,306.28720436)(845.61311152,306.24220441)(845.5031163,306.23221102)
\curveto(845.39311174,306.23220442)(845.28311185,306.21720443)(845.1731163,306.18721102)
\curveto(845.0331121,306.15720449)(844.89811224,306.12720452)(844.7681163,306.09721102)
\curveto(844.64811249,306.06720458)(844.5331126,306.02720462)(844.4231163,305.97721102)
\curveto(844.133113,305.8472048)(843.89811324,305.66720498)(843.7181163,305.43721102)
\curveto(843.5381136,305.21720543)(843.38311375,304.96220569)(843.2531163,304.67221102)
\curveto(843.21311392,304.56220609)(843.18311395,304.4472062)(843.1631163,304.32721102)
\curveto(843.14311399,304.21720643)(843.11811402,304.10220655)(843.0881163,303.98221102)
\curveto(843.07811406,303.93220672)(843.07311406,303.88220677)(843.0731163,303.83221102)
\curveto(843.08311405,303.78220687)(843.08311405,303.73220692)(843.0731163,303.68221102)
\curveto(843.04311409,303.56220709)(843.02811411,303.42220723)(843.0281163,303.26221102)
\curveto(843.0381141,303.11220754)(843.04311409,302.96720768)(843.0431163,302.82721102)
\lineto(843.0431163,300.98221102)
\lineto(843.0431163,300.63721102)
\curveto(843.04311409,300.51721013)(843.0381141,300.40221025)(843.0281163,300.29221102)
\curveto(843.01811412,300.18221047)(843.01311412,300.08721056)(843.0131163,300.00721102)
\curveto(843.02311411,299.92721072)(843.00311413,299.85721079)(842.9531163,299.79721102)
\curveto(842.90311423,299.72721092)(842.82311431,299.68721096)(842.7131163,299.67721102)
\curveto(842.61311452,299.66721098)(842.50311463,299.66221099)(842.3831163,299.66221102)
\lineto(842.1131163,299.66221102)
\curveto(842.06311507,299.68221097)(842.01311512,299.69721095)(841.9631163,299.70721102)
\curveto(841.92311521,299.72721092)(841.89311524,299.7522109)(841.8731163,299.78221102)
\curveto(841.82311531,299.8522108)(841.79311534,299.93721071)(841.7831163,300.03721102)
\lineto(841.7831163,300.36721102)
\lineto(841.7831163,301.52221102)
\lineto(841.7831163,305.67721102)
\lineto(841.7831163,306.71221102)
\lineto(841.7831163,307.01221102)
\curveto(841.79311534,307.11220354)(841.82311531,307.19720345)(841.8731163,307.26721102)
\curveto(841.90311523,307.30720334)(841.95311518,307.33720331)(842.0231163,307.35721102)
\curveto(842.10311503,307.37720327)(842.18811495,307.38720326)(842.2781163,307.38721102)
\curveto(842.36811477,307.39720325)(842.45811468,307.39720325)(842.5481163,307.38721102)
\curveto(842.6381145,307.37720327)(842.70811443,307.36220329)(842.7581163,307.34221102)
\curveto(842.8381143,307.31220334)(842.88811425,307.2522034)(842.9081163,307.16221102)
\curveto(842.9381142,307.08220357)(842.95311418,306.99220366)(842.9531163,306.89221102)
\lineto(842.9531163,306.59221102)
\curveto(842.95311418,306.49220416)(842.97311416,306.40220425)(843.0131163,306.32221102)
\curveto(843.02311411,306.30220435)(843.0331141,306.28720436)(843.0431163,306.27721102)
\lineto(843.0881163,306.23221102)
\curveto(843.19811394,306.23220442)(843.28811385,306.27720437)(843.3581163,306.36721102)
\curveto(843.42811371,306.46720418)(843.48811365,306.5472041)(843.5381163,306.60721102)
\lineto(843.6281163,306.69721102)
\curveto(843.71811342,306.80720384)(843.84311329,306.92220373)(844.0031163,307.04221102)
\curveto(844.16311297,307.16220349)(844.31311282,307.2522034)(844.4531163,307.31221102)
\curveto(844.54311259,307.36220329)(844.6381125,307.39720325)(844.7381163,307.41721102)
\curveto(844.8381123,307.4472032)(844.94311219,307.47720317)(845.0531163,307.50721102)
\curveto(845.11311202,307.51720313)(845.17311196,307.52220313)(845.2331163,307.52221102)
\curveto(845.29311184,307.53220312)(845.34811179,307.54220311)(845.3981163,307.55221102)
}
}
{
\newrgbcolor{curcolor}{0 0 0}
\pscustom[linestyle=none,fillstyle=solid,fillcolor=curcolor]
{
\newpath
\moveto(853.51288192,303.81721102)
\curveto(853.53287424,303.71720693)(853.53287424,303.60220705)(853.51288192,303.47221102)
\curveto(853.50287427,303.3522073)(853.4728743,303.26720738)(853.42288192,303.21721102)
\curveto(853.3728744,303.17720747)(853.29787447,303.1472075)(853.19788192,303.12721102)
\curveto(853.10787466,303.11720753)(853.00287477,303.11220754)(852.88288192,303.11221102)
\lineto(852.52288192,303.11221102)
\curveto(852.40287537,303.12220753)(852.29787547,303.12720752)(852.20788192,303.12721102)
\lineto(848.36788192,303.12721102)
\curveto(848.28787948,303.12720752)(848.20787956,303.12220753)(848.12788192,303.11221102)
\curveto(848.04787972,303.11220754)(847.98287979,303.09720755)(847.93288192,303.06721102)
\curveto(847.89287988,303.0472076)(847.85287992,303.00720764)(847.81288192,302.94721102)
\curveto(847.79287998,302.91720773)(847.77288,302.87220778)(847.75288192,302.81221102)
\curveto(847.73288004,302.76220789)(847.73288004,302.71220794)(847.75288192,302.66221102)
\curveto(847.76288001,302.61220804)(847.76788,302.56720808)(847.76788192,302.52721102)
\curveto(847.76788,302.48720816)(847.77288,302.4472082)(847.78288192,302.40721102)
\curveto(847.80287997,302.32720832)(847.82287995,302.24220841)(847.84288192,302.15221102)
\curveto(847.86287991,302.07220858)(847.89287988,301.99220866)(847.93288192,301.91221102)
\curveto(848.16287961,301.37220928)(848.54287923,300.98720966)(849.07288192,300.75721102)
\curveto(849.13287864,300.72720992)(849.19787857,300.70220995)(849.26788192,300.68221102)
\lineto(849.47788192,300.62221102)
\curveto(849.50787826,300.61221004)(849.55787821,300.60721004)(849.62788192,300.60721102)
\curveto(849.767878,300.56721008)(849.95287782,300.5472101)(850.18288192,300.54721102)
\curveto(850.41287736,300.5472101)(850.59787717,300.56721008)(850.73788192,300.60721102)
\curveto(850.87787689,300.64721)(851.00287677,300.68720996)(851.11288192,300.72721102)
\curveto(851.23287654,300.77720987)(851.34287643,300.83720981)(851.44288192,300.90721102)
\curveto(851.55287622,300.97720967)(851.64787612,301.05720959)(851.72788192,301.14721102)
\curveto(851.80787596,301.2472094)(851.87787589,301.3522093)(851.93788192,301.46221102)
\curveto(851.99787577,301.56220909)(852.04787572,301.66720898)(852.08788192,301.77721102)
\curveto(852.13787563,301.88720876)(852.21787555,301.96720868)(852.32788192,302.01721102)
\curveto(852.3678754,302.03720861)(852.43287534,302.0522086)(852.52288192,302.06221102)
\curveto(852.61287516,302.07220858)(852.70287507,302.07220858)(852.79288192,302.06221102)
\curveto(852.88287489,302.06220859)(852.9678748,302.05720859)(853.04788192,302.04721102)
\curveto(853.12787464,302.03720861)(853.18287459,302.01720863)(853.21288192,301.98721102)
\curveto(853.31287446,301.91720873)(853.33787443,301.80220885)(853.28788192,301.64221102)
\curveto(853.20787456,301.37220928)(853.10287467,301.13220952)(852.97288192,300.92221102)
\curveto(852.772875,300.60221005)(852.54287523,300.33721031)(852.28288192,300.12721102)
\curveto(852.03287574,299.92721072)(851.71287606,299.76221089)(851.32288192,299.63221102)
\curveto(851.22287655,299.59221106)(851.12287665,299.56721108)(851.02288192,299.55721102)
\curveto(850.92287685,299.53721111)(850.81787695,299.51721113)(850.70788192,299.49721102)
\curveto(850.65787711,299.48721116)(850.60787716,299.48221117)(850.55788192,299.48221102)
\curveto(850.51787725,299.48221117)(850.4728773,299.47721117)(850.42288192,299.46721102)
\lineto(850.27288192,299.46721102)
\curveto(850.22287755,299.45721119)(850.16287761,299.4522112)(850.09288192,299.45221102)
\curveto(850.03287774,299.4522112)(849.98287779,299.45721119)(849.94288192,299.46721102)
\lineto(849.80788192,299.46721102)
\curveto(849.75787801,299.47721117)(849.71287806,299.48221117)(849.67288192,299.48221102)
\curveto(849.63287814,299.48221117)(849.59287818,299.48721116)(849.55288192,299.49721102)
\curveto(849.50287827,299.50721114)(849.44787832,299.51721113)(849.38788192,299.52721102)
\curveto(849.32787844,299.52721112)(849.2728785,299.53221112)(849.22288192,299.54221102)
\curveto(849.13287864,299.56221109)(849.04287873,299.58721106)(848.95288192,299.61721102)
\curveto(848.86287891,299.63721101)(848.77787899,299.66221099)(848.69788192,299.69221102)
\curveto(848.65787911,299.71221094)(848.62287915,299.72221093)(848.59288192,299.72221102)
\curveto(848.56287921,299.73221092)(848.52787924,299.7472109)(848.48788192,299.76721102)
\curveto(848.33787943,299.83721081)(848.17787959,299.92221073)(848.00788192,300.02221102)
\curveto(847.71788005,300.21221044)(847.4678803,300.44221021)(847.25788192,300.71221102)
\curveto(847.05788071,300.99220966)(846.88788088,301.30220935)(846.74788192,301.64221102)
\curveto(846.69788107,301.7522089)(846.65788111,301.86720878)(846.62788192,301.98721102)
\curveto(846.60788116,302.10720854)(846.57788119,302.22720842)(846.53788192,302.34721102)
\curveto(846.52788124,302.38720826)(846.52288125,302.42220823)(846.52288192,302.45221102)
\curveto(846.52288125,302.48220817)(846.51788125,302.52220813)(846.50788192,302.57221102)
\curveto(846.48788128,302.652208)(846.4728813,302.73720791)(846.46288192,302.82721102)
\curveto(846.45288132,302.91720773)(846.43788133,303.00720764)(846.41788192,303.09721102)
\lineto(846.41788192,303.30721102)
\curveto(846.40788136,303.3472073)(846.39788137,303.40220725)(846.38788192,303.47221102)
\curveto(846.38788138,303.5522071)(846.39288138,303.61720703)(846.40288192,303.66721102)
\lineto(846.40288192,303.83221102)
\curveto(846.42288135,303.88220677)(846.42788134,303.93220672)(846.41788192,303.98221102)
\curveto(846.41788135,304.04220661)(846.42288135,304.09720655)(846.43288192,304.14721102)
\curveto(846.4728813,304.30720634)(846.50288127,304.46720618)(846.52288192,304.62721102)
\curveto(846.55288122,304.78720586)(846.59788117,304.93720571)(846.65788192,305.07721102)
\curveto(846.70788106,305.18720546)(846.75288102,305.29720535)(846.79288192,305.40721102)
\curveto(846.84288093,305.52720512)(846.89788087,305.64220501)(846.95788192,305.75221102)
\curveto(847.17788059,306.10220455)(847.42788034,306.40220425)(847.70788192,306.65221102)
\curveto(847.98787978,306.91220374)(848.33287944,307.12720352)(848.74288192,307.29721102)
\curveto(848.86287891,307.3472033)(848.98287879,307.38220327)(849.10288192,307.40221102)
\curveto(849.23287854,307.43220322)(849.3678784,307.46220319)(849.50788192,307.49221102)
\curveto(849.55787821,307.50220315)(849.60287817,307.50720314)(849.64288192,307.50721102)
\curveto(849.68287809,307.51720313)(849.72787804,307.52220313)(849.77788192,307.52221102)
\curveto(849.79787797,307.53220312)(849.82287795,307.53220312)(849.85288192,307.52221102)
\curveto(849.88287789,307.51220314)(849.90787786,307.51720313)(849.92788192,307.53721102)
\curveto(850.34787742,307.5472031)(850.71287706,307.50220315)(851.02288192,307.40221102)
\curveto(851.33287644,307.31220334)(851.61287616,307.18720346)(851.86288192,307.02721102)
\curveto(851.91287586,307.00720364)(851.95287582,306.97720367)(851.98288192,306.93721102)
\curveto(852.01287576,306.90720374)(852.04787572,306.88220377)(852.08788192,306.86221102)
\curveto(852.1678756,306.80220385)(852.24787552,306.73220392)(852.32788192,306.65221102)
\curveto(852.41787535,306.57220408)(852.49287528,306.49220416)(852.55288192,306.41221102)
\curveto(852.71287506,306.20220445)(852.84787492,306.00220465)(852.95788192,305.81221102)
\curveto(853.02787474,305.70220495)(853.08287469,305.58220507)(853.12288192,305.45221102)
\curveto(853.16287461,305.32220533)(853.20787456,305.19220546)(853.25788192,305.06221102)
\curveto(853.30787446,304.93220572)(853.34287443,304.79720585)(853.36288192,304.65721102)
\curveto(853.39287438,304.51720613)(853.42787434,304.37720627)(853.46788192,304.23721102)
\curveto(853.47787429,304.16720648)(853.48287429,304.09720655)(853.48288192,304.02721102)
\lineto(853.51288192,303.81721102)
\moveto(852.05788192,304.32721102)
\curveto(852.08787568,304.36720628)(852.11287566,304.41720623)(852.13288192,304.47721102)
\curveto(852.15287562,304.5472061)(852.15287562,304.61720603)(852.13288192,304.68721102)
\curveto(852.0728757,304.90720574)(851.98787578,305.11220554)(851.87788192,305.30221102)
\curveto(851.73787603,305.53220512)(851.58287619,305.72720492)(851.41288192,305.88721102)
\curveto(851.24287653,306.0472046)(851.02287675,306.18220447)(850.75288192,306.29221102)
\curveto(850.68287709,306.31220434)(850.61287716,306.32720432)(850.54288192,306.33721102)
\curveto(850.4728773,306.35720429)(850.39787737,306.37720427)(850.31788192,306.39721102)
\curveto(850.23787753,306.41720423)(850.15287762,306.42720422)(850.06288192,306.42721102)
\lineto(849.80788192,306.42721102)
\curveto(849.77787799,306.40720424)(849.74287803,306.39720425)(849.70288192,306.39721102)
\curveto(849.66287811,306.40720424)(849.62787814,306.40720424)(849.59788192,306.39721102)
\lineto(849.35788192,306.33721102)
\curveto(849.28787848,306.32720432)(849.21787855,306.31220434)(849.14788192,306.29221102)
\curveto(848.85787891,306.17220448)(848.62287915,306.02220463)(848.44288192,305.84221102)
\curveto(848.2728795,305.66220499)(848.11787965,305.43720521)(847.97788192,305.16721102)
\curveto(847.94787982,305.11720553)(847.91787985,305.0522056)(847.88788192,304.97221102)
\curveto(847.85787991,304.90220575)(847.83287994,304.82220583)(847.81288192,304.73221102)
\curveto(847.79287998,304.64220601)(847.78787998,304.55720609)(847.79788192,304.47721102)
\curveto(847.80787996,304.39720625)(847.84287993,304.33720631)(847.90288192,304.29721102)
\curveto(847.98287979,304.23720641)(848.11787965,304.20720644)(848.30788192,304.20721102)
\curveto(848.50787926,304.21720643)(848.67787909,304.22220643)(848.81788192,304.22221102)
\lineto(851.09788192,304.22221102)
\curveto(851.24787652,304.22220643)(851.42787634,304.21720643)(851.63788192,304.20721102)
\curveto(851.84787592,304.20720644)(851.98787578,304.2472064)(852.05788192,304.32721102)
}
}
{
\newrgbcolor{curcolor}{0 0 0}
\pscustom[linestyle=none,fillstyle=solid,fillcolor=curcolor]
{
\newpath
\moveto(858.46452255,307.55221102)
\curveto(858.69451776,307.5522031)(858.82451763,307.49220316)(858.85452255,307.37221102)
\curveto(858.88451757,307.26220339)(858.89951755,307.09720355)(858.89952255,306.87721102)
\lineto(858.89952255,306.59221102)
\curveto(858.89951755,306.50220415)(858.87451758,306.42720422)(858.82452255,306.36721102)
\curveto(858.76451769,306.28720436)(858.67951777,306.24220441)(858.56952255,306.23221102)
\curveto(858.45951799,306.23220442)(858.3495181,306.21720443)(858.23952255,306.18721102)
\curveto(858.09951835,306.15720449)(857.96451849,306.12720452)(857.83452255,306.09721102)
\curveto(857.71451874,306.06720458)(857.59951885,306.02720462)(857.48952255,305.97721102)
\curveto(857.19951925,305.8472048)(856.96451949,305.66720498)(856.78452255,305.43721102)
\curveto(856.60451985,305.21720543)(856.44952,304.96220569)(856.31952255,304.67221102)
\curveto(856.27952017,304.56220609)(856.2495202,304.4472062)(856.22952255,304.32721102)
\curveto(856.20952024,304.21720643)(856.18452027,304.10220655)(856.15452255,303.98221102)
\curveto(856.14452031,303.93220672)(856.13952031,303.88220677)(856.13952255,303.83221102)
\curveto(856.1495203,303.78220687)(856.1495203,303.73220692)(856.13952255,303.68221102)
\curveto(856.10952034,303.56220709)(856.09452036,303.42220723)(856.09452255,303.26221102)
\curveto(856.10452035,303.11220754)(856.10952034,302.96720768)(856.10952255,302.82721102)
\lineto(856.10952255,300.98221102)
\lineto(856.10952255,300.63721102)
\curveto(856.10952034,300.51721013)(856.10452035,300.40221025)(856.09452255,300.29221102)
\curveto(856.08452037,300.18221047)(856.07952037,300.08721056)(856.07952255,300.00721102)
\curveto(856.08952036,299.92721072)(856.06952038,299.85721079)(856.01952255,299.79721102)
\curveto(855.96952048,299.72721092)(855.88952056,299.68721096)(855.77952255,299.67721102)
\curveto(855.67952077,299.66721098)(855.56952088,299.66221099)(855.44952255,299.66221102)
\lineto(855.17952255,299.66221102)
\curveto(855.12952132,299.68221097)(855.07952137,299.69721095)(855.02952255,299.70721102)
\curveto(854.98952146,299.72721092)(854.95952149,299.7522109)(854.93952255,299.78221102)
\curveto(854.88952156,299.8522108)(854.85952159,299.93721071)(854.84952255,300.03721102)
\lineto(854.84952255,300.36721102)
\lineto(854.84952255,301.52221102)
\lineto(854.84952255,305.67721102)
\lineto(854.84952255,306.71221102)
\lineto(854.84952255,307.01221102)
\curveto(854.85952159,307.11220354)(854.88952156,307.19720345)(854.93952255,307.26721102)
\curveto(854.96952148,307.30720334)(855.01952143,307.33720331)(855.08952255,307.35721102)
\curveto(855.16952128,307.37720327)(855.2545212,307.38720326)(855.34452255,307.38721102)
\curveto(855.43452102,307.39720325)(855.52452093,307.39720325)(855.61452255,307.38721102)
\curveto(855.70452075,307.37720327)(855.77452068,307.36220329)(855.82452255,307.34221102)
\curveto(855.90452055,307.31220334)(855.9545205,307.2522034)(855.97452255,307.16221102)
\curveto(856.00452045,307.08220357)(856.01952043,306.99220366)(856.01952255,306.89221102)
\lineto(856.01952255,306.59221102)
\curveto(856.01952043,306.49220416)(856.03952041,306.40220425)(856.07952255,306.32221102)
\curveto(856.08952036,306.30220435)(856.09952035,306.28720436)(856.10952255,306.27721102)
\lineto(856.15452255,306.23221102)
\curveto(856.26452019,306.23220442)(856.3545201,306.27720437)(856.42452255,306.36721102)
\curveto(856.49451996,306.46720418)(856.5545199,306.5472041)(856.60452255,306.60721102)
\lineto(856.69452255,306.69721102)
\curveto(856.78451967,306.80720384)(856.90951954,306.92220373)(857.06952255,307.04221102)
\curveto(857.22951922,307.16220349)(857.37951907,307.2522034)(857.51952255,307.31221102)
\curveto(857.60951884,307.36220329)(857.70451875,307.39720325)(857.80452255,307.41721102)
\curveto(857.90451855,307.4472032)(858.00951844,307.47720317)(858.11952255,307.50721102)
\curveto(858.17951827,307.51720313)(858.23951821,307.52220313)(858.29952255,307.52221102)
\curveto(858.35951809,307.53220312)(858.41451804,307.54220311)(858.46452255,307.55221102)
}
}
{
\newrgbcolor{curcolor}{0 0 0}
\pscustom[linestyle=none,fillstyle=solid,fillcolor=curcolor]
{
\newpath
\moveto(866.71428817,300.20221102)
\curveto(866.74428034,300.04221061)(866.72928036,299.90721074)(866.66928817,299.79721102)
\curveto(866.60928048,299.69721095)(866.52928056,299.62221103)(866.42928817,299.57221102)
\curveto(866.37928071,299.5522111)(866.32428076,299.54221111)(866.26428817,299.54221102)
\curveto(866.21428087,299.54221111)(866.15928093,299.53221112)(866.09928817,299.51221102)
\curveto(865.87928121,299.46221119)(865.65928143,299.47721117)(865.43928817,299.55721102)
\curveto(865.22928186,299.62721102)(865.084282,299.71721093)(865.00428817,299.82721102)
\curveto(864.95428213,299.89721075)(864.90928218,299.97721067)(864.86928817,300.06721102)
\curveto(864.82928226,300.16721048)(864.77928231,300.2472104)(864.71928817,300.30721102)
\curveto(864.69928239,300.32721032)(864.67428241,300.3472103)(864.64428817,300.36721102)
\curveto(864.62428246,300.38721026)(864.59428249,300.39221026)(864.55428817,300.38221102)
\curveto(864.44428264,300.3522103)(864.33928275,300.29721035)(864.23928817,300.21721102)
\curveto(864.14928294,300.13721051)(864.05928303,300.06721058)(863.96928817,300.00721102)
\curveto(863.83928325,299.92721072)(863.69928339,299.8522108)(863.54928817,299.78221102)
\curveto(863.39928369,299.72221093)(863.23928385,299.66721098)(863.06928817,299.61721102)
\curveto(862.96928412,299.58721106)(862.85928423,299.56721108)(862.73928817,299.55721102)
\curveto(862.62928446,299.5472111)(862.51928457,299.53221112)(862.40928817,299.51221102)
\curveto(862.35928473,299.50221115)(862.31428477,299.49721115)(862.27428817,299.49721102)
\lineto(862.16928817,299.49721102)
\curveto(862.05928503,299.47721117)(861.95428513,299.47721117)(861.85428817,299.49721102)
\lineto(861.71928817,299.49721102)
\curveto(861.66928542,299.50721114)(861.61928547,299.51221114)(861.56928817,299.51221102)
\curveto(861.51928557,299.51221114)(861.47428561,299.52221113)(861.43428817,299.54221102)
\curveto(861.39428569,299.5522111)(861.35928573,299.55721109)(861.32928817,299.55721102)
\curveto(861.30928578,299.5472111)(861.2842858,299.5472111)(861.25428817,299.55721102)
\lineto(861.01428817,299.61721102)
\curveto(860.93428615,299.62721102)(860.85928623,299.647211)(860.78928817,299.67721102)
\curveto(860.4892866,299.80721084)(860.24428684,299.9522107)(860.05428817,300.11221102)
\curveto(859.87428721,300.28221037)(859.72428736,300.51721013)(859.60428817,300.81721102)
\curveto(859.51428757,301.03720961)(859.46928762,301.30220935)(859.46928817,301.61221102)
\lineto(859.46928817,301.92721102)
\curveto(859.47928761,301.97720867)(859.4842876,302.02720862)(859.48428817,302.07721102)
\lineto(859.51428817,302.25721102)
\lineto(859.63428817,302.58721102)
\curveto(859.67428741,302.69720795)(859.72428736,302.79720785)(859.78428817,302.88721102)
\curveto(859.96428712,303.17720747)(860.20928688,303.39220726)(860.51928817,303.53221102)
\curveto(860.82928626,303.67220698)(861.16928592,303.79720685)(861.53928817,303.90721102)
\curveto(861.67928541,303.9472067)(861.82428526,303.97720667)(861.97428817,303.99721102)
\curveto(862.12428496,304.01720663)(862.27428481,304.04220661)(862.42428817,304.07221102)
\curveto(862.49428459,304.09220656)(862.55928453,304.10220655)(862.61928817,304.10221102)
\curveto(862.6892844,304.10220655)(862.76428432,304.11220654)(862.84428817,304.13221102)
\curveto(862.91428417,304.1522065)(862.9842841,304.16220649)(863.05428817,304.16221102)
\curveto(863.12428396,304.17220648)(863.19928389,304.18720646)(863.27928817,304.20721102)
\curveto(863.52928356,304.26720638)(863.76428332,304.31720633)(863.98428817,304.35721102)
\curveto(864.20428288,304.40720624)(864.37928271,304.52220613)(864.50928817,304.70221102)
\curveto(864.56928252,304.78220587)(864.61928247,304.88220577)(864.65928817,305.00221102)
\curveto(864.69928239,305.13220552)(864.69928239,305.27220538)(864.65928817,305.42221102)
\curveto(864.59928249,305.66220499)(864.50928258,305.8522048)(864.38928817,305.99221102)
\curveto(864.27928281,306.13220452)(864.11928297,306.24220441)(863.90928817,306.32221102)
\curveto(863.7892833,306.37220428)(863.64428344,306.40720424)(863.47428817,306.42721102)
\curveto(863.31428377,306.4472042)(863.14428394,306.45720419)(862.96428817,306.45721102)
\curveto(862.7842843,306.45720419)(862.60928448,306.4472042)(862.43928817,306.42721102)
\curveto(862.26928482,306.40720424)(862.12428496,306.37720427)(862.00428817,306.33721102)
\curveto(861.83428525,306.27720437)(861.66928542,306.19220446)(861.50928817,306.08221102)
\curveto(861.42928566,306.02220463)(861.35428573,305.94220471)(861.28428817,305.84221102)
\curveto(861.22428586,305.7522049)(861.16928592,305.652205)(861.11928817,305.54221102)
\curveto(861.089286,305.46220519)(861.05928603,305.37720527)(861.02928817,305.28721102)
\curveto(861.00928608,305.19720545)(860.96428612,305.12720552)(860.89428817,305.07721102)
\curveto(860.85428623,305.0472056)(860.7842863,305.02220563)(860.68428817,305.00221102)
\curveto(860.59428649,304.99220566)(860.49928659,304.98720566)(860.39928817,304.98721102)
\curveto(860.29928679,304.98720566)(860.19928689,304.99220566)(860.09928817,305.00221102)
\curveto(860.00928708,305.02220563)(859.94428714,305.0472056)(859.90428817,305.07721102)
\curveto(859.86428722,305.10720554)(859.83428725,305.15720549)(859.81428817,305.22721102)
\curveto(859.79428729,305.29720535)(859.79428729,305.37220528)(859.81428817,305.45221102)
\curveto(859.84428724,305.58220507)(859.87428721,305.70220495)(859.90428817,305.81221102)
\curveto(859.94428714,305.93220472)(859.9892871,306.0472046)(860.03928817,306.15721102)
\curveto(860.22928686,306.50720414)(860.46928662,306.77720387)(860.75928817,306.96721102)
\curveto(861.04928604,307.16720348)(861.40928568,307.32720332)(861.83928817,307.44721102)
\curveto(861.93928515,307.46720318)(862.03928505,307.48220317)(862.13928817,307.49221102)
\curveto(862.24928484,307.50220315)(862.35928473,307.51720313)(862.46928817,307.53721102)
\curveto(862.50928458,307.5472031)(862.57428451,307.5472031)(862.66428817,307.53721102)
\curveto(862.75428433,307.53720311)(862.80928428,307.5472031)(862.82928817,307.56721102)
\curveto(863.52928356,307.57720307)(864.13928295,307.49720315)(864.65928817,307.32721102)
\curveto(865.17928191,307.15720349)(865.54428154,306.83220382)(865.75428817,306.35221102)
\curveto(865.84428124,306.1522045)(865.89428119,305.91720473)(865.90428817,305.64721102)
\curveto(865.92428116,305.38720526)(865.93428115,305.11220554)(865.93428817,304.82221102)
\lineto(865.93428817,301.50721102)
\curveto(865.93428115,301.36720928)(865.93928115,301.23220942)(865.94928817,301.10221102)
\curveto(865.95928113,300.97220968)(865.9892811,300.86720978)(866.03928817,300.78721102)
\curveto(866.089281,300.71720993)(866.15428093,300.66720998)(866.23428817,300.63721102)
\curveto(866.32428076,300.59721005)(866.40928068,300.56721008)(866.48928817,300.54721102)
\curveto(866.56928052,300.53721011)(866.62928046,300.49221016)(866.66928817,300.41221102)
\curveto(866.6892804,300.38221027)(866.69928039,300.3522103)(866.69928817,300.32221102)
\curveto(866.69928039,300.29221036)(866.70428038,300.2522104)(866.71428817,300.20221102)
\moveto(864.56928817,301.86721102)
\curveto(864.62928246,302.00720864)(864.65928243,302.16720848)(864.65928817,302.34721102)
\curveto(864.66928242,302.53720811)(864.67428241,302.73220792)(864.67428817,302.93221102)
\curveto(864.67428241,303.04220761)(864.66928242,303.14220751)(864.65928817,303.23221102)
\curveto(864.64928244,303.32220733)(864.60928248,303.39220726)(864.53928817,303.44221102)
\curveto(864.50928258,303.46220719)(864.43928265,303.47220718)(864.32928817,303.47221102)
\curveto(864.30928278,303.4522072)(864.27428281,303.44220721)(864.22428817,303.44221102)
\curveto(864.17428291,303.44220721)(864.12928296,303.43220722)(864.08928817,303.41221102)
\curveto(864.00928308,303.39220726)(863.91928317,303.37220728)(863.81928817,303.35221102)
\lineto(863.51928817,303.29221102)
\curveto(863.4892836,303.29220736)(863.45428363,303.28720736)(863.41428817,303.27721102)
\lineto(863.30928817,303.27721102)
\curveto(863.15928393,303.23720741)(862.99428409,303.21220744)(862.81428817,303.20221102)
\curveto(862.64428444,303.20220745)(862.4842846,303.18220747)(862.33428817,303.14221102)
\curveto(862.25428483,303.12220753)(862.17928491,303.10220755)(862.10928817,303.08221102)
\curveto(862.04928504,303.07220758)(861.97928511,303.05720759)(861.89928817,303.03721102)
\curveto(861.73928535,302.98720766)(861.5892855,302.92220773)(861.44928817,302.84221102)
\curveto(861.30928578,302.77220788)(861.1892859,302.68220797)(861.08928817,302.57221102)
\curveto(860.9892861,302.46220819)(860.91428617,302.32720832)(860.86428817,302.16721102)
\curveto(860.81428627,302.01720863)(860.79428629,301.83220882)(860.80428817,301.61221102)
\curveto(860.80428628,301.51220914)(860.81928627,301.41720923)(860.84928817,301.32721102)
\curveto(860.8892862,301.2472094)(860.93428615,301.17220948)(860.98428817,301.10221102)
\curveto(861.06428602,300.99220966)(861.16928592,300.89720975)(861.29928817,300.81721102)
\curveto(861.42928566,300.7472099)(861.56928552,300.68720996)(861.71928817,300.63721102)
\curveto(861.76928532,300.62721002)(861.81928527,300.62221003)(861.86928817,300.62221102)
\curveto(861.91928517,300.62221003)(861.96928512,300.61721003)(862.01928817,300.60721102)
\curveto(862.089285,300.58721006)(862.17428491,300.57221008)(862.27428817,300.56221102)
\curveto(862.3842847,300.56221009)(862.47428461,300.57221008)(862.54428817,300.59221102)
\curveto(862.60428448,300.61221004)(862.66428442,300.61721003)(862.72428817,300.60721102)
\curveto(862.7842843,300.60721004)(862.84428424,300.61721003)(862.90428817,300.63721102)
\curveto(862.9842841,300.65720999)(863.05928403,300.67220998)(863.12928817,300.68221102)
\curveto(863.20928388,300.69220996)(863.2842838,300.71220994)(863.35428817,300.74221102)
\curveto(863.64428344,300.86220979)(863.8892832,301.00720964)(864.08928817,301.17721102)
\curveto(864.29928279,301.3472093)(864.45928263,301.57720907)(864.56928817,301.86721102)
}
}
{
\newrgbcolor{curcolor}{0 0 0}
\pscustom[linestyle=none,fillstyle=solid,fillcolor=curcolor]
{
\newpath
\moveto(870.3159288,307.55221102)
\curveto(871.03592473,307.56220309)(871.64092413,307.47720317)(872.1309288,307.29721102)
\curveto(872.62092315,307.12720352)(873.00092277,306.82220383)(873.2709288,306.38221102)
\curveto(873.34092243,306.27220438)(873.39592237,306.15720449)(873.4359288,306.03721102)
\curveto(873.47592229,305.92720472)(873.51592225,305.80220485)(873.5559288,305.66221102)
\curveto(873.57592219,305.59220506)(873.58092219,305.51720513)(873.5709288,305.43721102)
\curveto(873.56092221,305.36720528)(873.54592222,305.31220534)(873.5259288,305.27221102)
\curveto(873.50592226,305.2522054)(873.48092229,305.23220542)(873.4509288,305.21221102)
\curveto(873.42092235,305.20220545)(873.39592237,305.18720546)(873.3759288,305.16721102)
\curveto(873.32592244,305.1472055)(873.27592249,305.14220551)(873.2259288,305.15221102)
\curveto(873.17592259,305.16220549)(873.12592264,305.16220549)(873.0759288,305.15221102)
\curveto(872.99592277,305.13220552)(872.89092288,305.12720552)(872.7609288,305.13721102)
\curveto(872.63092314,305.15720549)(872.54092323,305.18220547)(872.4909288,305.21221102)
\curveto(872.41092336,305.26220539)(872.35592341,305.32720532)(872.3259288,305.40721102)
\curveto(872.30592346,305.49720515)(872.2709235,305.58220507)(872.2209288,305.66221102)
\curveto(872.13092364,305.82220483)(872.00592376,305.96720468)(871.8459288,306.09721102)
\curveto(871.73592403,306.17720447)(871.61592415,306.23720441)(871.4859288,306.27721102)
\curveto(871.35592441,306.31720433)(871.21592455,306.35720429)(871.0659288,306.39721102)
\curveto(871.01592475,306.41720423)(870.9659248,306.42220423)(870.9159288,306.41221102)
\curveto(870.8659249,306.41220424)(870.81592495,306.41720423)(870.7659288,306.42721102)
\curveto(870.70592506,306.4472042)(870.63092514,306.45720419)(870.5409288,306.45721102)
\curveto(870.45092532,306.45720419)(870.37592539,306.4472042)(870.3159288,306.42721102)
\lineto(870.2259288,306.42721102)
\lineto(870.0759288,306.39721102)
\curveto(870.02592574,306.39720425)(869.97592579,306.39220426)(869.9259288,306.38221102)
\curveto(869.6659261,306.32220433)(869.45092632,306.23720441)(869.2809288,306.12721102)
\curveto(869.11092666,306.01720463)(868.99592677,305.83220482)(868.9359288,305.57221102)
\curveto(868.91592685,305.50220515)(868.91092686,305.43220522)(868.9209288,305.36221102)
\curveto(868.94092683,305.29220536)(868.96092681,305.23220542)(868.9809288,305.18221102)
\curveto(869.04092673,305.03220562)(869.11092666,304.92220573)(869.1909288,304.85221102)
\curveto(869.28092649,304.79220586)(869.39092638,304.72220593)(869.5209288,304.64221102)
\curveto(869.68092609,304.54220611)(869.86092591,304.46720618)(870.0609288,304.41721102)
\curveto(870.26092551,304.37720627)(870.46092531,304.32720632)(870.6609288,304.26721102)
\curveto(870.79092498,304.22720642)(870.92092485,304.19720645)(871.0509288,304.17721102)
\curveto(871.18092459,304.15720649)(871.31092446,304.12720652)(871.4409288,304.08721102)
\curveto(871.65092412,304.02720662)(871.85592391,303.96720668)(872.0559288,303.90721102)
\curveto(872.25592351,303.85720679)(872.45592331,303.79220686)(872.6559288,303.71221102)
\lineto(872.8059288,303.65221102)
\curveto(872.85592291,303.63220702)(872.90592286,303.60720704)(872.9559288,303.57721102)
\curveto(873.15592261,303.45720719)(873.33092244,303.32220733)(873.4809288,303.17221102)
\curveto(873.63092214,303.02220763)(873.75592201,302.83220782)(873.8559288,302.60221102)
\curveto(873.87592189,302.53220812)(873.89592187,302.43720821)(873.9159288,302.31721102)
\curveto(873.93592183,302.2472084)(873.94592182,302.17220848)(873.9459288,302.09221102)
\curveto(873.95592181,302.02220863)(873.96092181,301.94220871)(873.9609288,301.85221102)
\lineto(873.9609288,301.70221102)
\curveto(873.94092183,301.63220902)(873.93092184,301.56220909)(873.9309288,301.49221102)
\curveto(873.93092184,301.42220923)(873.92092185,301.3522093)(873.9009288,301.28221102)
\curveto(873.8709219,301.17220948)(873.83592193,301.06720958)(873.7959288,300.96721102)
\curveto(873.75592201,300.86720978)(873.71092206,300.77720987)(873.6609288,300.69721102)
\curveto(873.50092227,300.43721021)(873.29592247,300.22721042)(873.0459288,300.06721102)
\curveto(872.79592297,299.91721073)(872.51592325,299.78721086)(872.2059288,299.67721102)
\curveto(872.11592365,299.647211)(872.02092375,299.62721102)(871.9209288,299.61721102)
\curveto(871.83092394,299.59721105)(871.74092403,299.57221108)(871.6509288,299.54221102)
\curveto(871.55092422,299.52221113)(871.45092432,299.51221114)(871.3509288,299.51221102)
\curveto(871.25092452,299.51221114)(871.15092462,299.50221115)(871.0509288,299.48221102)
\lineto(870.9009288,299.48221102)
\curveto(870.85092492,299.47221118)(870.78092499,299.46721118)(870.6909288,299.46721102)
\curveto(870.60092517,299.46721118)(870.53092524,299.47221118)(870.4809288,299.48221102)
\lineto(870.3159288,299.48221102)
\curveto(870.25592551,299.50221115)(870.19092558,299.51221114)(870.1209288,299.51221102)
\curveto(870.05092572,299.50221115)(869.99092578,299.50721114)(869.9409288,299.52721102)
\curveto(869.89092588,299.53721111)(869.82592594,299.54221111)(869.7459288,299.54221102)
\lineto(869.5059288,299.60221102)
\curveto(869.43592633,299.61221104)(869.36092641,299.63221102)(869.2809288,299.66221102)
\curveto(868.9709268,299.76221089)(868.70092707,299.88721076)(868.4709288,300.03721102)
\curveto(868.24092753,300.18721046)(868.04092773,300.38221027)(867.8709288,300.62221102)
\curveto(867.78092799,300.7522099)(867.70592806,300.88720976)(867.6459288,301.02721102)
\curveto(867.58592818,301.16720948)(867.53092824,301.32220933)(867.4809288,301.49221102)
\curveto(867.46092831,301.5522091)(867.45092832,301.62220903)(867.4509288,301.70221102)
\curveto(867.46092831,301.79220886)(867.47592829,301.86220879)(867.4959288,301.91221102)
\curveto(867.52592824,301.9522087)(867.57592819,301.99220866)(867.6459288,302.03221102)
\curveto(867.69592807,302.0522086)(867.765928,302.06220859)(867.8559288,302.06221102)
\curveto(867.94592782,302.07220858)(868.03592773,302.07220858)(868.1259288,302.06221102)
\curveto(868.21592755,302.0522086)(868.30092747,302.03720861)(868.3809288,302.01721102)
\curveto(868.4709273,302.00720864)(868.53092724,301.99220866)(868.5609288,301.97221102)
\curveto(868.63092714,301.92220873)(868.67592709,301.8472088)(868.6959288,301.74721102)
\curveto(868.72592704,301.65720899)(868.76092701,301.57220908)(868.8009288,301.49221102)
\curveto(868.90092687,301.27220938)(869.03592673,301.10220955)(869.2059288,300.98221102)
\curveto(869.32592644,300.89220976)(869.46092631,300.82220983)(869.6109288,300.77221102)
\curveto(869.76092601,300.72220993)(869.92092585,300.67220998)(870.0909288,300.62221102)
\lineto(870.4059288,300.57721102)
\lineto(870.4959288,300.57721102)
\curveto(870.5659252,300.55721009)(870.65592511,300.5472101)(870.7659288,300.54721102)
\curveto(870.88592488,300.5472101)(870.98592478,300.55721009)(871.0659288,300.57721102)
\curveto(871.13592463,300.57721007)(871.19092458,300.58221007)(871.2309288,300.59221102)
\curveto(871.29092448,300.60221005)(871.35092442,300.60721004)(871.4109288,300.60721102)
\curveto(871.4709243,300.61721003)(871.52592424,300.62721002)(871.5759288,300.63721102)
\curveto(871.8659239,300.71720993)(872.09592367,300.82220983)(872.2659288,300.95221102)
\curveto(872.43592333,301.08220957)(872.55592321,301.30220935)(872.6259288,301.61221102)
\curveto(872.64592312,301.66220899)(872.65092312,301.71720893)(872.6409288,301.77721102)
\curveto(872.63092314,301.83720881)(872.62092315,301.88220877)(872.6109288,301.91221102)
\curveto(872.56092321,302.10220855)(872.49092328,302.24220841)(872.4009288,302.33221102)
\curveto(872.31092346,302.43220822)(872.19592357,302.52220813)(872.0559288,302.60221102)
\curveto(871.9659238,302.66220799)(871.8659239,302.71220794)(871.7559288,302.75221102)
\lineto(871.4259288,302.87221102)
\curveto(871.39592437,302.88220777)(871.3659244,302.88720776)(871.3359288,302.88721102)
\curveto(871.31592445,302.88720776)(871.29092448,302.89720775)(871.2609288,302.91721102)
\curveto(870.92092485,303.02720762)(870.5659252,303.10720754)(870.1959288,303.15721102)
\curveto(869.83592593,303.21720743)(869.49592627,303.31220734)(869.1759288,303.44221102)
\curveto(869.07592669,303.48220717)(868.98092679,303.51720713)(868.8909288,303.54721102)
\curveto(868.80092697,303.57720707)(868.71592705,303.61720703)(868.6359288,303.66721102)
\curveto(868.44592732,303.77720687)(868.2709275,303.90220675)(868.1109288,304.04221102)
\curveto(867.95092782,304.18220647)(867.82592794,304.35720629)(867.7359288,304.56721102)
\curveto(867.70592806,304.63720601)(867.68092809,304.70720594)(867.6609288,304.77721102)
\curveto(867.65092812,304.8472058)(867.63592813,304.92220573)(867.6159288,305.00221102)
\curveto(867.58592818,305.12220553)(867.57592819,305.25720539)(867.5859288,305.40721102)
\curveto(867.59592817,305.56720508)(867.61092816,305.70220495)(867.6309288,305.81221102)
\curveto(867.65092812,305.86220479)(867.66092811,305.90220475)(867.6609288,305.93221102)
\curveto(867.6709281,305.97220468)(867.68592808,306.01220464)(867.7059288,306.05221102)
\curveto(867.79592797,306.28220437)(867.91592785,306.48220417)(868.0659288,306.65221102)
\curveto(868.22592754,306.82220383)(868.40592736,306.97220368)(868.6059288,307.10221102)
\curveto(868.75592701,307.19220346)(868.92092685,307.26220339)(869.1009288,307.31221102)
\curveto(869.28092649,307.37220328)(869.4709263,307.42720322)(869.6709288,307.47721102)
\curveto(869.74092603,307.48720316)(869.80592596,307.49720315)(869.8659288,307.50721102)
\curveto(869.93592583,307.51720313)(870.01092576,307.52720312)(870.0909288,307.53721102)
\curveto(870.12092565,307.5472031)(870.16092561,307.5472031)(870.2109288,307.53721102)
\curveto(870.26092551,307.52720312)(870.29592547,307.53220312)(870.3159288,307.55221102)
}
}
{
\newrgbcolor{curcolor}{0.80000001 0.80000001 0.80000001}
\pscustom[linestyle=none,fillstyle=solid,fillcolor=curcolor]
{
\newpath
\moveto(798.51865829,310.35724764)
\lineto(813.51865829,310.35724764)
\lineto(813.51865829,295.35724764)
\lineto(798.51865829,295.35724764)
\closepath
}
}
{
\newrgbcolor{curcolor}{0 0 0}
\pscustom[linestyle=none,fillstyle=solid,fillcolor=curcolor]
{
\newpath
\moveto(826.4568663,277.57003328)
\curveto(826.47685675,277.52003254)(826.50185673,277.4600326)(826.5318663,277.39003328)
\curveto(826.56185667,277.32003274)(826.58185665,277.24503281)(826.5918663,277.16503328)
\curveto(826.61185662,277.09503296)(826.61185662,277.02503303)(826.5918663,276.95503328)
\curveto(826.58185665,276.89503316)(826.54185669,276.85003321)(826.4718663,276.82003328)
\curveto(826.42185681,276.80003326)(826.36185687,276.79003327)(826.2918663,276.79003328)
\lineto(826.0818663,276.79003328)
\lineto(825.6318663,276.79003328)
\curveto(825.48185775,276.79003327)(825.36185787,276.81503324)(825.2718663,276.86503328)
\curveto(825.17185806,276.92503313)(825.09685813,277.03003303)(825.0468663,277.18003328)
\curveto(825.00685822,277.33003273)(824.96185827,277.46503259)(824.9118663,277.58503328)
\curveto(824.80185843,277.84503221)(824.70185853,278.11503194)(824.6118663,278.39503328)
\curveto(824.52185871,278.67503138)(824.42185881,278.95003111)(824.3118663,279.22003328)
\curveto(824.28185895,279.31003075)(824.25185898,279.39503066)(824.2218663,279.47503328)
\curveto(824.20185903,279.5550305)(824.17185906,279.63003043)(824.1318663,279.70003328)
\curveto(824.10185913,279.77003029)(824.05685917,279.83003023)(823.9968663,279.88003328)
\curveto(823.93685929,279.93003013)(823.85685937,279.97003009)(823.7568663,280.00003328)
\curveto(823.70685952,280.02003004)(823.64685958,280.02503003)(823.5768663,280.01503328)
\lineto(823.3818663,280.01503328)
\lineto(820.5468663,280.01503328)
\lineto(820.2468663,280.01503328)
\curveto(820.13686309,280.02503003)(820.0318632,280.02503003)(819.9318663,280.01503328)
\curveto(819.8318634,280.00503005)(819.73686349,279.99003007)(819.6468663,279.97003328)
\curveto(819.56686366,279.95003011)(819.50686372,279.91003015)(819.4668663,279.85003328)
\curveto(819.38686384,279.75003031)(819.3268639,279.63503042)(819.2868663,279.50503328)
\curveto(819.25686397,279.38503067)(819.21686401,279.2600308)(819.1668663,279.13003328)
\curveto(819.06686416,278.90003116)(818.97186426,278.6600314)(818.8818663,278.41003328)
\curveto(818.80186443,278.1600319)(818.71186452,277.92003214)(818.6118663,277.69003328)
\curveto(818.59186464,277.63003243)(818.56686466,277.5600325)(818.5368663,277.48003328)
\curveto(818.51686471,277.41003265)(818.49186474,277.33503272)(818.4618663,277.25503328)
\curveto(818.4318648,277.17503288)(818.39686483,277.10003296)(818.3568663,277.03003328)
\curveto(818.3268649,276.97003309)(818.29186494,276.92503313)(818.2518663,276.89503328)
\curveto(818.17186506,276.83503322)(818.06186517,276.80003326)(817.9218663,276.79003328)
\lineto(817.5018663,276.79003328)
\lineto(817.2618663,276.79003328)
\curveto(817.19186604,276.80003326)(817.1318661,276.82503323)(817.0818663,276.86503328)
\curveto(817.0318662,276.89503316)(817.00186623,276.94003312)(816.9918663,277.00003328)
\curveto(816.99186624,277.060033)(816.99686623,277.12003294)(817.0068663,277.18003328)
\curveto(817.0268662,277.25003281)(817.04686618,277.31503274)(817.0668663,277.37503328)
\curveto(817.09686613,277.44503261)(817.12186611,277.49503256)(817.1418663,277.52503328)
\curveto(817.28186595,277.84503221)(817.40686582,278.1600319)(817.5168663,278.47003328)
\curveto(817.6268656,278.79003127)(817.74686548,279.11003095)(817.8768663,279.43003328)
\curveto(817.96686526,279.65003041)(818.05186518,279.86503019)(818.1318663,280.07503328)
\curveto(818.21186502,280.29502976)(818.29686493,280.51502954)(818.3868663,280.73503328)
\curveto(818.68686454,281.4550286)(818.97186426,282.18002788)(819.2418663,282.91003328)
\curveto(819.51186372,283.65002641)(819.79686343,284.38502567)(820.0968663,285.11503328)
\curveto(820.20686302,285.37502468)(820.30686292,285.64002442)(820.3968663,285.91003328)
\curveto(820.49686273,286.18002388)(820.60186263,286.44502361)(820.7118663,286.70503328)
\curveto(820.76186247,286.81502324)(820.80686242,286.93502312)(820.8468663,287.06503328)
\curveto(820.89686233,287.20502285)(820.96686226,287.30502275)(821.0568663,287.36503328)
\curveto(821.09686213,287.40502265)(821.16186207,287.43502262)(821.2518663,287.45503328)
\curveto(821.27186196,287.46502259)(821.29186194,287.46502259)(821.3118663,287.45503328)
\curveto(821.34186189,287.4550226)(821.36686186,287.4600226)(821.3868663,287.47003328)
\curveto(821.56686166,287.47002259)(821.77686145,287.47002259)(822.0168663,287.47003328)
\curveto(822.25686097,287.48002258)(822.4318608,287.44502261)(822.5418663,287.36503328)
\curveto(822.62186061,287.30502275)(822.68186055,287.20502285)(822.7218663,287.06503328)
\curveto(822.77186046,286.93502312)(822.82186041,286.81502324)(822.8718663,286.70503328)
\curveto(822.97186026,286.47502358)(823.06186017,286.24502381)(823.1418663,286.01503328)
\curveto(823.22186001,285.78502427)(823.31185992,285.5550245)(823.4118663,285.32503328)
\curveto(823.49185974,285.12502493)(823.56685966,284.92002514)(823.6368663,284.71003328)
\curveto(823.71685951,284.50002556)(823.80185943,284.29502576)(823.8918663,284.09503328)
\curveto(824.19185904,283.36502669)(824.47685875,282.62502743)(824.7468663,281.87503328)
\curveto(825.0268582,281.13502892)(825.32185791,280.40002966)(825.6318663,279.67003328)
\curveto(825.67185756,279.58003048)(825.70185753,279.49503056)(825.7218663,279.41503328)
\curveto(825.75185748,279.33503072)(825.78185745,279.25003081)(825.8118663,279.16003328)
\curveto(825.92185731,278.90003116)(826.0268572,278.63503142)(826.1268663,278.36503328)
\curveto(826.23685699,278.09503196)(826.34685688,277.83003223)(826.4568663,277.57003328)
\moveto(823.2468663,281.21503328)
\curveto(823.33685989,281.24502881)(823.39185984,281.29502876)(823.4118663,281.36503328)
\curveto(823.44185979,281.43502862)(823.44685978,281.51002855)(823.4268663,281.59003328)
\curveto(823.41685981,281.68002838)(823.39185984,281.76502829)(823.3518663,281.84503328)
\curveto(823.32185991,281.93502812)(823.29185994,282.01002805)(823.2618663,282.07003328)
\curveto(823.24185999,282.11002795)(823.23186,282.14502791)(823.2318663,282.17503328)
\curveto(823.23186,282.20502785)(823.22186001,282.24002782)(823.2018663,282.28003328)
\lineto(823.1118663,282.52003328)
\curveto(823.09186014,282.61002745)(823.06186017,282.70002736)(823.0218663,282.79003328)
\curveto(822.87186036,283.15002691)(822.73686049,283.51502654)(822.6168663,283.88503328)
\curveto(822.50686072,284.26502579)(822.37686085,284.63502542)(822.2268663,284.99503328)
\curveto(822.17686105,285.10502495)(822.1318611,285.21502484)(822.0918663,285.32503328)
\curveto(822.06186117,285.43502462)(822.02186121,285.54002452)(821.9718663,285.64003328)
\curveto(821.95186128,285.69002437)(821.9268613,285.73502432)(821.8968663,285.77503328)
\curveto(821.87686135,285.82502423)(821.8268614,285.85002421)(821.7468663,285.85003328)
\curveto(821.7268615,285.83002423)(821.70686152,285.81502424)(821.6868663,285.80503328)
\curveto(821.66686156,285.79502426)(821.64686158,285.78002428)(821.6268663,285.76003328)
\curveto(821.58686164,285.71002435)(821.55686167,285.6550244)(821.5368663,285.59503328)
\curveto(821.51686171,285.54502451)(821.49686173,285.49002457)(821.4768663,285.43003328)
\curveto(821.4268618,285.32002474)(821.38686184,285.21002485)(821.3568663,285.10003328)
\curveto(821.3268619,284.99002507)(821.28686194,284.88002518)(821.2368663,284.77003328)
\curveto(821.06686216,284.38002568)(820.91686231,283.98502607)(820.7868663,283.58503328)
\curveto(820.66686256,283.18502687)(820.5268627,282.79502726)(820.3668663,282.41503328)
\lineto(820.3068663,282.26503328)
\curveto(820.29686293,282.21502784)(820.28186295,282.16502789)(820.2618663,282.11503328)
\lineto(820.1718663,281.87503328)
\curveto(820.14186309,281.79502826)(820.11686311,281.71502834)(820.0968663,281.63503328)
\curveto(820.07686315,281.58502847)(820.06686316,281.53002853)(820.0668663,281.47003328)
\curveto(820.07686315,281.41002865)(820.09186314,281.3600287)(820.1118663,281.32003328)
\curveto(820.16186307,281.24002882)(820.26686296,281.19502886)(820.4268663,281.18503328)
\lineto(820.8768663,281.18503328)
\lineto(822.4818663,281.18503328)
\curveto(822.59186064,281.18502887)(822.7268605,281.18002888)(822.8868663,281.17003328)
\curveto(823.04686018,281.17002889)(823.16686006,281.18502887)(823.2468663,281.21503328)
}
}
{
\newrgbcolor{curcolor}{0 0 0}
\pscustom[linestyle=none,fillstyle=solid,fillcolor=curcolor]
{
\newpath
\moveto(831.2184288,284.69503328)
\curveto(831.44842401,284.69502536)(831.57842388,284.63502542)(831.6084288,284.51503328)
\curveto(831.63842382,284.40502565)(831.6534238,284.24002582)(831.6534288,284.02003328)
\lineto(831.6534288,283.73503328)
\curveto(831.6534238,283.64502641)(831.62842383,283.57002649)(831.5784288,283.51003328)
\curveto(831.51842394,283.43002663)(831.43342402,283.38502667)(831.3234288,283.37503328)
\curveto(831.21342424,283.37502668)(831.10342435,283.3600267)(830.9934288,283.33003328)
\curveto(830.8534246,283.30002676)(830.71842474,283.27002679)(830.5884288,283.24003328)
\curveto(830.46842499,283.21002685)(830.3534251,283.17002689)(830.2434288,283.12003328)
\curveto(829.9534255,282.99002707)(829.71842574,282.81002725)(829.5384288,282.58003328)
\curveto(829.3584261,282.3600277)(829.20342625,282.10502795)(829.0734288,281.81503328)
\curveto(829.03342642,281.70502835)(829.00342645,281.59002847)(828.9834288,281.47003328)
\curveto(828.96342649,281.3600287)(828.93842652,281.24502881)(828.9084288,281.12503328)
\curveto(828.89842656,281.07502898)(828.89342656,281.02502903)(828.8934288,280.97503328)
\curveto(828.90342655,280.92502913)(828.90342655,280.87502918)(828.8934288,280.82503328)
\curveto(828.86342659,280.70502935)(828.84842661,280.56502949)(828.8484288,280.40503328)
\curveto(828.8584266,280.2550298)(828.86342659,280.11002995)(828.8634288,279.97003328)
\lineto(828.8634288,278.12503328)
\lineto(828.8634288,277.78003328)
\curveto(828.86342659,277.6600324)(828.8584266,277.54503251)(828.8484288,277.43503328)
\curveto(828.83842662,277.32503273)(828.83342662,277.23003283)(828.8334288,277.15003328)
\curveto(828.84342661,277.07003299)(828.82342663,277.00003306)(828.7734288,276.94003328)
\curveto(828.72342673,276.87003319)(828.64342681,276.83003323)(828.5334288,276.82003328)
\curveto(828.43342702,276.81003325)(828.32342713,276.80503325)(828.2034288,276.80503328)
\lineto(827.9334288,276.80503328)
\curveto(827.88342757,276.82503323)(827.83342762,276.84003322)(827.7834288,276.85003328)
\curveto(827.74342771,276.87003319)(827.71342774,276.89503316)(827.6934288,276.92503328)
\curveto(827.64342781,276.99503306)(827.61342784,277.08003298)(827.6034288,277.18003328)
\lineto(827.6034288,277.51003328)
\lineto(827.6034288,278.66503328)
\lineto(827.6034288,282.82003328)
\lineto(827.6034288,283.85503328)
\lineto(827.6034288,284.15503328)
\curveto(827.61342784,284.2550258)(827.64342781,284.34002572)(827.6934288,284.41003328)
\curveto(827.72342773,284.45002561)(827.77342768,284.48002558)(827.8434288,284.50003328)
\curveto(827.92342753,284.52002554)(828.00842745,284.53002553)(828.0984288,284.53003328)
\curveto(828.18842727,284.54002552)(828.27842718,284.54002552)(828.3684288,284.53003328)
\curveto(828.458427,284.52002554)(828.52842693,284.50502555)(828.5784288,284.48503328)
\curveto(828.6584268,284.4550256)(828.70842675,284.39502566)(828.7284288,284.30503328)
\curveto(828.7584267,284.22502583)(828.77342668,284.13502592)(828.7734288,284.03503328)
\lineto(828.7734288,283.73503328)
\curveto(828.77342668,283.63502642)(828.79342666,283.54502651)(828.8334288,283.46503328)
\curveto(828.84342661,283.44502661)(828.8534266,283.43002663)(828.8634288,283.42003328)
\lineto(828.9084288,283.37503328)
\curveto(829.01842644,283.37502668)(829.10842635,283.42002664)(829.1784288,283.51003328)
\curveto(829.24842621,283.61002645)(829.30842615,283.69002637)(829.3584288,283.75003328)
\lineto(829.4484288,283.84003328)
\curveto(829.53842592,283.95002611)(829.66342579,284.06502599)(829.8234288,284.18503328)
\curveto(829.98342547,284.30502575)(830.13342532,284.39502566)(830.2734288,284.45503328)
\curveto(830.36342509,284.50502555)(830.458425,284.54002552)(830.5584288,284.56003328)
\curveto(830.6584248,284.59002547)(830.76342469,284.62002544)(830.8734288,284.65003328)
\curveto(830.93342452,284.6600254)(830.99342446,284.66502539)(831.0534288,284.66503328)
\curveto(831.11342434,284.67502538)(831.16842429,284.68502537)(831.2184288,284.69503328)
}
}
{
\newrgbcolor{curcolor}{0 0 0}
\pscustom[linestyle=none,fillstyle=solid,fillcolor=curcolor]
{
\newpath
\moveto(839.33319442,280.96003328)
\curveto(839.35318674,280.8600292)(839.35318674,280.74502931)(839.33319442,280.61503328)
\curveto(839.32318677,280.49502956)(839.2931868,280.41002965)(839.24319442,280.36003328)
\curveto(839.1931869,280.32002974)(839.11818697,280.29002977)(839.01819442,280.27003328)
\curveto(838.92818716,280.2600298)(838.82318727,280.2550298)(838.70319442,280.25503328)
\lineto(838.34319442,280.25503328)
\curveto(838.22318787,280.26502979)(838.11818797,280.27002979)(838.02819442,280.27003328)
\lineto(834.18819442,280.27003328)
\curveto(834.10819198,280.27002979)(834.02819206,280.26502979)(833.94819442,280.25503328)
\curveto(833.86819222,280.2550298)(833.80319229,280.24002982)(833.75319442,280.21003328)
\curveto(833.71319238,280.19002987)(833.67319242,280.15002991)(833.63319442,280.09003328)
\curveto(833.61319248,280.06003)(833.5931925,280.01503004)(833.57319442,279.95503328)
\curveto(833.55319254,279.90503015)(833.55319254,279.8550302)(833.57319442,279.80503328)
\curveto(833.58319251,279.7550303)(833.5881925,279.71003035)(833.58819442,279.67003328)
\curveto(833.5881925,279.63003043)(833.5931925,279.59003047)(833.60319442,279.55003328)
\curveto(833.62319247,279.47003059)(833.64319245,279.38503067)(833.66319442,279.29503328)
\curveto(833.68319241,279.21503084)(833.71319238,279.13503092)(833.75319442,279.05503328)
\curveto(833.98319211,278.51503154)(834.36319173,278.13003193)(834.89319442,277.90003328)
\curveto(834.95319114,277.87003219)(835.01819107,277.84503221)(835.08819442,277.82503328)
\lineto(835.29819442,277.76503328)
\curveto(835.32819076,277.7550323)(835.37819071,277.75003231)(835.44819442,277.75003328)
\curveto(835.5881905,277.71003235)(835.77319032,277.69003237)(836.00319442,277.69003328)
\curveto(836.23318986,277.69003237)(836.41818967,277.71003235)(836.55819442,277.75003328)
\curveto(836.69818939,277.79003227)(836.82318927,277.83003223)(836.93319442,277.87003328)
\curveto(837.05318904,277.92003214)(837.16318893,277.98003208)(837.26319442,278.05003328)
\curveto(837.37318872,278.12003194)(837.46818862,278.20003186)(837.54819442,278.29003328)
\curveto(837.62818846,278.39003167)(837.69818839,278.49503156)(837.75819442,278.60503328)
\curveto(837.81818827,278.70503135)(837.86818822,278.81003125)(837.90819442,278.92003328)
\curveto(837.95818813,279.03003103)(838.03818805,279.11003095)(838.14819442,279.16003328)
\curveto(838.1881879,279.18003088)(838.25318784,279.19503086)(838.34319442,279.20503328)
\curveto(838.43318766,279.21503084)(838.52318757,279.21503084)(838.61319442,279.20503328)
\curveto(838.70318739,279.20503085)(838.7881873,279.20003086)(838.86819442,279.19003328)
\curveto(838.94818714,279.18003088)(839.00318709,279.1600309)(839.03319442,279.13003328)
\curveto(839.13318696,279.060031)(839.15818693,278.94503111)(839.10819442,278.78503328)
\curveto(839.02818706,278.51503154)(838.92318717,278.27503178)(838.79319442,278.06503328)
\curveto(838.5931875,277.74503231)(838.36318773,277.48003258)(838.10319442,277.27003328)
\curveto(837.85318824,277.07003299)(837.53318856,276.90503315)(837.14319442,276.77503328)
\curveto(837.04318905,276.73503332)(836.94318915,276.71003335)(836.84319442,276.70003328)
\curveto(836.74318935,276.68003338)(836.63818945,276.6600334)(836.52819442,276.64003328)
\curveto(836.47818961,276.63003343)(836.42818966,276.62503343)(836.37819442,276.62503328)
\curveto(836.33818975,276.62503343)(836.2931898,276.62003344)(836.24319442,276.61003328)
\lineto(836.09319442,276.61003328)
\curveto(836.04319005,276.60003346)(835.98319011,276.59503346)(835.91319442,276.59503328)
\curveto(835.85319024,276.59503346)(835.80319029,276.60003346)(835.76319442,276.61003328)
\lineto(835.62819442,276.61003328)
\curveto(835.57819051,276.62003344)(835.53319056,276.62503343)(835.49319442,276.62503328)
\curveto(835.45319064,276.62503343)(835.41319068,276.63003343)(835.37319442,276.64003328)
\curveto(835.32319077,276.65003341)(835.26819082,276.6600334)(835.20819442,276.67003328)
\curveto(835.14819094,276.67003339)(835.093191,276.67503338)(835.04319442,276.68503328)
\curveto(834.95319114,276.70503335)(834.86319123,276.73003333)(834.77319442,276.76003328)
\curveto(834.68319141,276.78003328)(834.59819149,276.80503325)(834.51819442,276.83503328)
\curveto(834.47819161,276.8550332)(834.44319165,276.86503319)(834.41319442,276.86503328)
\curveto(834.38319171,276.87503318)(834.34819174,276.89003317)(834.30819442,276.91003328)
\curveto(834.15819193,276.98003308)(833.99819209,277.06503299)(833.82819442,277.16503328)
\curveto(833.53819255,277.3550327)(833.2881928,277.58503247)(833.07819442,277.85503328)
\curveto(832.87819321,278.13503192)(832.70819338,278.44503161)(832.56819442,278.78503328)
\curveto(832.51819357,278.89503116)(832.47819361,279.01003105)(832.44819442,279.13003328)
\curveto(832.42819366,279.25003081)(832.39819369,279.37003069)(832.35819442,279.49003328)
\curveto(832.34819374,279.53003053)(832.34319375,279.56503049)(832.34319442,279.59503328)
\curveto(832.34319375,279.62503043)(832.33819375,279.66503039)(832.32819442,279.71503328)
\curveto(832.30819378,279.79503026)(832.2931938,279.88003018)(832.28319442,279.97003328)
\curveto(832.27319382,280.06003)(832.25819383,280.15002991)(832.23819442,280.24003328)
\lineto(832.23819442,280.45003328)
\curveto(832.22819386,280.49002957)(832.21819387,280.54502951)(832.20819442,280.61503328)
\curveto(832.20819388,280.69502936)(832.21319388,280.7600293)(832.22319442,280.81003328)
\lineto(832.22319442,280.97503328)
\curveto(832.24319385,281.02502903)(832.24819384,281.07502898)(832.23819442,281.12503328)
\curveto(832.23819385,281.18502887)(832.24319385,281.24002882)(832.25319442,281.29003328)
\curveto(832.2931938,281.45002861)(832.32319377,281.61002845)(832.34319442,281.77003328)
\curveto(832.37319372,281.93002813)(832.41819367,282.08002798)(832.47819442,282.22003328)
\curveto(832.52819356,282.33002773)(832.57319352,282.44002762)(832.61319442,282.55003328)
\curveto(832.66319343,282.67002739)(832.71819337,282.78502727)(832.77819442,282.89503328)
\curveto(832.99819309,283.24502681)(833.24819284,283.54502651)(833.52819442,283.79503328)
\curveto(833.80819228,284.055026)(834.15319194,284.27002579)(834.56319442,284.44003328)
\curveto(834.68319141,284.49002557)(834.80319129,284.52502553)(834.92319442,284.54503328)
\curveto(835.05319104,284.57502548)(835.1881909,284.60502545)(835.32819442,284.63503328)
\curveto(835.37819071,284.64502541)(835.42319067,284.65002541)(835.46319442,284.65003328)
\curveto(835.50319059,284.6600254)(835.54819054,284.66502539)(835.59819442,284.66503328)
\curveto(835.61819047,284.67502538)(835.64319045,284.67502538)(835.67319442,284.66503328)
\curveto(835.70319039,284.6550254)(835.72819036,284.6600254)(835.74819442,284.68003328)
\curveto(836.16818992,284.69002537)(836.53318956,284.64502541)(836.84319442,284.54503328)
\curveto(837.15318894,284.4550256)(837.43318866,284.33002573)(837.68319442,284.17003328)
\curveto(837.73318836,284.15002591)(837.77318832,284.12002594)(837.80319442,284.08003328)
\curveto(837.83318826,284.05002601)(837.86818822,284.02502603)(837.90819442,284.00503328)
\curveto(837.9881881,283.94502611)(838.06818802,283.87502618)(838.14819442,283.79503328)
\curveto(838.23818785,283.71502634)(838.31318778,283.63502642)(838.37319442,283.55503328)
\curveto(838.53318756,283.34502671)(838.66818742,283.14502691)(838.77819442,282.95503328)
\curveto(838.84818724,282.84502721)(838.90318719,282.72502733)(838.94319442,282.59503328)
\curveto(838.98318711,282.46502759)(839.02818706,282.33502772)(839.07819442,282.20503328)
\curveto(839.12818696,282.07502798)(839.16318693,281.94002812)(839.18319442,281.80003328)
\curveto(839.21318688,281.6600284)(839.24818684,281.52002854)(839.28819442,281.38003328)
\curveto(839.29818679,281.31002875)(839.30318679,281.24002882)(839.30319442,281.17003328)
\lineto(839.33319442,280.96003328)
\moveto(837.87819442,281.47003328)
\curveto(837.90818818,281.51002855)(837.93318816,281.5600285)(837.95319442,281.62003328)
\curveto(837.97318812,281.69002837)(837.97318812,281.7600283)(837.95319442,281.83003328)
\curveto(837.8931882,282.05002801)(837.80818828,282.2550278)(837.69819442,282.44503328)
\curveto(837.55818853,282.67502738)(837.40318869,282.87002719)(837.23319442,283.03003328)
\curveto(837.06318903,283.19002687)(836.84318925,283.32502673)(836.57319442,283.43503328)
\curveto(836.50318959,283.4550266)(836.43318966,283.47002659)(836.36319442,283.48003328)
\curveto(836.2931898,283.50002656)(836.21818987,283.52002654)(836.13819442,283.54003328)
\curveto(836.05819003,283.5600265)(835.97319012,283.57002649)(835.88319442,283.57003328)
\lineto(835.62819442,283.57003328)
\curveto(835.59819049,283.55002651)(835.56319053,283.54002652)(835.52319442,283.54003328)
\curveto(835.48319061,283.55002651)(835.44819064,283.55002651)(835.41819442,283.54003328)
\lineto(835.17819442,283.48003328)
\curveto(835.10819098,283.47002659)(835.03819105,283.4550266)(834.96819442,283.43503328)
\curveto(834.67819141,283.31502674)(834.44319165,283.16502689)(834.26319442,282.98503328)
\curveto(834.093192,282.80502725)(833.93819215,282.58002748)(833.79819442,282.31003328)
\curveto(833.76819232,282.2600278)(833.73819235,282.19502786)(833.70819442,282.11503328)
\curveto(833.67819241,282.04502801)(833.65319244,281.96502809)(833.63319442,281.87503328)
\curveto(833.61319248,281.78502827)(833.60819248,281.70002836)(833.61819442,281.62003328)
\curveto(833.62819246,281.54002852)(833.66319243,281.48002858)(833.72319442,281.44003328)
\curveto(833.80319229,281.38002868)(833.93819215,281.35002871)(834.12819442,281.35003328)
\curveto(834.32819176,281.3600287)(834.49819159,281.36502869)(834.63819442,281.36503328)
\lineto(836.91819442,281.36503328)
\curveto(837.06818902,281.36502869)(837.24818884,281.3600287)(837.45819442,281.35003328)
\curveto(837.66818842,281.35002871)(837.80818828,281.39002867)(837.87819442,281.47003328)
}
}
{
\newrgbcolor{curcolor}{0 0 0}
\pscustom[linestyle=none,fillstyle=solid,fillcolor=curcolor]
{
\newpath
\moveto(847.52483505,277.34503328)
\curveto(847.55482722,277.18503287)(847.53982723,277.05003301)(847.47983505,276.94003328)
\curveto(847.41982735,276.84003322)(847.33982743,276.76503329)(847.23983505,276.71503328)
\curveto(847.18982758,276.69503336)(847.13482764,276.68503337)(847.07483505,276.68503328)
\curveto(847.02482775,276.68503337)(846.9698278,276.67503338)(846.90983505,276.65503328)
\curveto(846.68982808,276.60503345)(846.4698283,276.62003344)(846.24983505,276.70003328)
\curveto(846.03982873,276.77003329)(845.89482888,276.8600332)(845.81483505,276.97003328)
\curveto(845.76482901,277.04003302)(845.71982905,277.12003294)(845.67983505,277.21003328)
\curveto(845.63982913,277.31003275)(845.58982918,277.39003267)(845.52983505,277.45003328)
\curveto(845.50982926,277.47003259)(845.48482929,277.49003257)(845.45483505,277.51003328)
\curveto(845.43482934,277.53003253)(845.40482937,277.53503252)(845.36483505,277.52503328)
\curveto(845.25482952,277.49503256)(845.14982962,277.44003262)(845.04983505,277.36003328)
\curveto(844.95982981,277.28003278)(844.8698299,277.21003285)(844.77983505,277.15003328)
\curveto(844.64983012,277.07003299)(844.50983026,276.99503306)(844.35983505,276.92503328)
\curveto(844.20983056,276.86503319)(844.04983072,276.81003325)(843.87983505,276.76003328)
\curveto(843.77983099,276.73003333)(843.6698311,276.71003335)(843.54983505,276.70003328)
\curveto(843.43983133,276.69003337)(843.32983144,276.67503338)(843.21983505,276.65503328)
\curveto(843.1698316,276.64503341)(843.12483165,276.64003342)(843.08483505,276.64003328)
\lineto(842.97983505,276.64003328)
\curveto(842.8698319,276.62003344)(842.76483201,276.62003344)(842.66483505,276.64003328)
\lineto(842.52983505,276.64003328)
\curveto(842.47983229,276.65003341)(842.42983234,276.6550334)(842.37983505,276.65503328)
\curveto(842.32983244,276.6550334)(842.28483249,276.66503339)(842.24483505,276.68503328)
\curveto(842.20483257,276.69503336)(842.1698326,276.70003336)(842.13983505,276.70003328)
\curveto(842.11983265,276.69003337)(842.09483268,276.69003337)(842.06483505,276.70003328)
\lineto(841.82483505,276.76003328)
\curveto(841.74483303,276.77003329)(841.6698331,276.79003327)(841.59983505,276.82003328)
\curveto(841.29983347,276.95003311)(841.05483372,277.09503296)(840.86483505,277.25503328)
\curveto(840.68483409,277.42503263)(840.53483424,277.6600324)(840.41483505,277.96003328)
\curveto(840.32483445,278.18003188)(840.27983449,278.44503161)(840.27983505,278.75503328)
\lineto(840.27983505,279.07003328)
\curveto(840.28983448,279.12003094)(840.29483448,279.17003089)(840.29483505,279.22003328)
\lineto(840.32483505,279.40003328)
\lineto(840.44483505,279.73003328)
\curveto(840.48483429,279.84003022)(840.53483424,279.94003012)(840.59483505,280.03003328)
\curveto(840.774834,280.32002974)(841.01983375,280.53502952)(841.32983505,280.67503328)
\curveto(841.63983313,280.81502924)(841.97983279,280.94002912)(842.34983505,281.05003328)
\curveto(842.48983228,281.09002897)(842.63483214,281.12002894)(842.78483505,281.14003328)
\curveto(842.93483184,281.1600289)(843.08483169,281.18502887)(843.23483505,281.21503328)
\curveto(843.30483147,281.23502882)(843.3698314,281.24502881)(843.42983505,281.24503328)
\curveto(843.49983127,281.24502881)(843.5748312,281.2550288)(843.65483505,281.27503328)
\curveto(843.72483105,281.29502876)(843.79483098,281.30502875)(843.86483505,281.30503328)
\curveto(843.93483084,281.31502874)(844.00983076,281.33002873)(844.08983505,281.35003328)
\curveto(844.33983043,281.41002865)(844.5748302,281.4600286)(844.79483505,281.50003328)
\curveto(845.01482976,281.55002851)(845.18982958,281.66502839)(845.31983505,281.84503328)
\curveto(845.37982939,281.92502813)(845.42982934,282.02502803)(845.46983505,282.14503328)
\curveto(845.50982926,282.27502778)(845.50982926,282.41502764)(845.46983505,282.56503328)
\curveto(845.40982936,282.80502725)(845.31982945,282.99502706)(845.19983505,283.13503328)
\curveto(845.08982968,283.27502678)(844.92982984,283.38502667)(844.71983505,283.46503328)
\curveto(844.59983017,283.51502654)(844.45483032,283.55002651)(844.28483505,283.57003328)
\curveto(844.12483065,283.59002647)(843.95483082,283.60002646)(843.77483505,283.60003328)
\curveto(843.59483118,283.60002646)(843.41983135,283.59002647)(843.24983505,283.57003328)
\curveto(843.07983169,283.55002651)(842.93483184,283.52002654)(842.81483505,283.48003328)
\curveto(842.64483213,283.42002664)(842.47983229,283.33502672)(842.31983505,283.22503328)
\curveto(842.23983253,283.16502689)(842.16483261,283.08502697)(842.09483505,282.98503328)
\curveto(842.03483274,282.89502716)(841.97983279,282.79502726)(841.92983505,282.68503328)
\curveto(841.89983287,282.60502745)(841.8698329,282.52002754)(841.83983505,282.43003328)
\curveto(841.81983295,282.34002772)(841.774833,282.27002779)(841.70483505,282.22003328)
\curveto(841.66483311,282.19002787)(841.59483318,282.16502789)(841.49483505,282.14503328)
\curveto(841.40483337,282.13502792)(841.30983346,282.13002793)(841.20983505,282.13003328)
\curveto(841.10983366,282.13002793)(841.00983376,282.13502792)(840.90983505,282.14503328)
\curveto(840.81983395,282.16502789)(840.75483402,282.19002787)(840.71483505,282.22003328)
\curveto(840.6748341,282.25002781)(840.64483413,282.30002776)(840.62483505,282.37003328)
\curveto(840.60483417,282.44002762)(840.60483417,282.51502754)(840.62483505,282.59503328)
\curveto(840.65483412,282.72502733)(840.68483409,282.84502721)(840.71483505,282.95503328)
\curveto(840.75483402,283.07502698)(840.79983397,283.19002687)(840.84983505,283.30003328)
\curveto(841.03983373,283.65002641)(841.27983349,283.92002614)(841.56983505,284.11003328)
\curveto(841.85983291,284.31002575)(842.21983255,284.47002559)(842.64983505,284.59003328)
\curveto(842.74983202,284.61002545)(842.84983192,284.62502543)(842.94983505,284.63503328)
\curveto(843.05983171,284.64502541)(843.1698316,284.6600254)(843.27983505,284.68003328)
\curveto(843.31983145,284.69002537)(843.38483139,284.69002537)(843.47483505,284.68003328)
\curveto(843.56483121,284.68002538)(843.61983115,284.69002537)(843.63983505,284.71003328)
\curveto(844.33983043,284.72002534)(844.94982982,284.64002542)(845.46983505,284.47003328)
\curveto(845.98982878,284.30002576)(846.35482842,283.97502608)(846.56483505,283.49503328)
\curveto(846.65482812,283.29502676)(846.70482807,283.060027)(846.71483505,282.79003328)
\curveto(846.73482804,282.53002753)(846.74482803,282.2550278)(846.74483505,281.96503328)
\lineto(846.74483505,278.65003328)
\curveto(846.74482803,278.51003155)(846.74982802,278.37503168)(846.75983505,278.24503328)
\curveto(846.769828,278.11503194)(846.79982797,278.01003205)(846.84983505,277.93003328)
\curveto(846.89982787,277.8600322)(846.96482781,277.81003225)(847.04483505,277.78003328)
\curveto(847.13482764,277.74003232)(847.21982755,277.71003235)(847.29983505,277.69003328)
\curveto(847.37982739,277.68003238)(847.43982733,277.63503242)(847.47983505,277.55503328)
\curveto(847.49982727,277.52503253)(847.50982726,277.49503256)(847.50983505,277.46503328)
\curveto(847.50982726,277.43503262)(847.51482726,277.39503266)(847.52483505,277.34503328)
\moveto(845.37983505,279.01003328)
\curveto(845.43982933,279.15003091)(845.4698293,279.31003075)(845.46983505,279.49003328)
\curveto(845.47982929,279.68003038)(845.48482929,279.87503018)(845.48483505,280.07503328)
\curveto(845.48482929,280.18502987)(845.47982929,280.28502977)(845.46983505,280.37503328)
\curveto(845.45982931,280.46502959)(845.41982935,280.53502952)(845.34983505,280.58503328)
\curveto(845.31982945,280.60502945)(845.24982952,280.61502944)(845.13983505,280.61503328)
\curveto(845.11982965,280.59502946)(845.08482969,280.58502947)(845.03483505,280.58503328)
\curveto(844.98482979,280.58502947)(844.93982983,280.57502948)(844.89983505,280.55503328)
\curveto(844.81982995,280.53502952)(844.72983004,280.51502954)(844.62983505,280.49503328)
\lineto(844.32983505,280.43503328)
\curveto(844.29983047,280.43502962)(844.26483051,280.43002963)(844.22483505,280.42003328)
\lineto(844.11983505,280.42003328)
\curveto(843.9698308,280.38002968)(843.80483097,280.3550297)(843.62483505,280.34503328)
\curveto(843.45483132,280.34502971)(843.29483148,280.32502973)(843.14483505,280.28503328)
\curveto(843.06483171,280.26502979)(842.98983178,280.24502981)(842.91983505,280.22503328)
\curveto(842.85983191,280.21502984)(842.78983198,280.20002986)(842.70983505,280.18003328)
\curveto(842.54983222,280.13002993)(842.39983237,280.06502999)(842.25983505,279.98503328)
\curveto(842.11983265,279.91503014)(841.99983277,279.82503023)(841.89983505,279.71503328)
\curveto(841.79983297,279.60503045)(841.72483305,279.47003059)(841.67483505,279.31003328)
\curveto(841.62483315,279.1600309)(841.60483317,278.97503108)(841.61483505,278.75503328)
\curveto(841.61483316,278.6550314)(841.62983314,278.5600315)(841.65983505,278.47003328)
\curveto(841.69983307,278.39003167)(841.74483303,278.31503174)(841.79483505,278.24503328)
\curveto(841.8748329,278.13503192)(841.97983279,278.04003202)(842.10983505,277.96003328)
\curveto(842.23983253,277.89003217)(842.37983239,277.83003223)(842.52983505,277.78003328)
\curveto(842.57983219,277.77003229)(842.62983214,277.76503229)(842.67983505,277.76503328)
\curveto(842.72983204,277.76503229)(842.77983199,277.7600323)(842.82983505,277.75003328)
\curveto(842.89983187,277.73003233)(842.98483179,277.71503234)(843.08483505,277.70503328)
\curveto(843.19483158,277.70503235)(843.28483149,277.71503234)(843.35483505,277.73503328)
\curveto(843.41483136,277.7550323)(843.4748313,277.7600323)(843.53483505,277.75003328)
\curveto(843.59483118,277.75003231)(843.65483112,277.7600323)(843.71483505,277.78003328)
\curveto(843.79483098,277.80003226)(843.8698309,277.81503224)(843.93983505,277.82503328)
\curveto(844.01983075,277.83503222)(844.09483068,277.8550322)(844.16483505,277.88503328)
\curveto(844.45483032,278.00503205)(844.69983007,278.15003191)(844.89983505,278.32003328)
\curveto(845.10982966,278.49003157)(845.2698295,278.72003134)(845.37983505,279.01003328)
}
}
{
\newrgbcolor{curcolor}{0 0 0}
\pscustom[linestyle=none,fillstyle=solid,fillcolor=curcolor]
{
\newpath
\moveto(851.12647567,284.69503328)
\curveto(851.84647161,284.70502535)(852.451471,284.62002544)(852.94147567,284.44003328)
\curveto(853.43147002,284.27002579)(853.81146964,283.96502609)(854.08147567,283.52503328)
\curveto(854.1514693,283.41502664)(854.20646925,283.30002676)(854.24647567,283.18003328)
\curveto(854.28646917,283.07002699)(854.32646913,282.94502711)(854.36647567,282.80503328)
\curveto(854.38646907,282.73502732)(854.39146906,282.6600274)(854.38147567,282.58003328)
\curveto(854.37146908,282.51002755)(854.3564691,282.4550276)(854.33647567,282.41503328)
\curveto(854.31646914,282.39502766)(854.29146916,282.37502768)(854.26147567,282.35503328)
\curveto(854.23146922,282.34502771)(854.20646925,282.33002773)(854.18647567,282.31003328)
\curveto(854.13646932,282.29002777)(854.08646937,282.28502777)(854.03647567,282.29503328)
\curveto(853.98646947,282.30502775)(853.93646952,282.30502775)(853.88647567,282.29503328)
\curveto(853.80646965,282.27502778)(853.70146975,282.27002779)(853.57147567,282.28003328)
\curveto(853.44147001,282.30002776)(853.3514701,282.32502773)(853.30147567,282.35503328)
\curveto(853.22147023,282.40502765)(853.16647029,282.47002759)(853.13647567,282.55003328)
\curveto(853.11647034,282.64002742)(853.08147037,282.72502733)(853.03147567,282.80503328)
\curveto(852.94147051,282.96502709)(852.81647064,283.11002695)(852.65647567,283.24003328)
\curveto(852.54647091,283.32002674)(852.42647103,283.38002668)(852.29647567,283.42003328)
\curveto(852.16647129,283.4600266)(852.02647143,283.50002656)(851.87647567,283.54003328)
\curveto(851.82647163,283.5600265)(851.77647168,283.56502649)(851.72647567,283.55503328)
\curveto(851.67647178,283.5550265)(851.62647183,283.5600265)(851.57647567,283.57003328)
\curveto(851.51647194,283.59002647)(851.44147201,283.60002646)(851.35147567,283.60003328)
\curveto(851.26147219,283.60002646)(851.18647227,283.59002647)(851.12647567,283.57003328)
\lineto(851.03647567,283.57003328)
\lineto(850.88647567,283.54003328)
\curveto(850.83647262,283.54002652)(850.78647267,283.53502652)(850.73647567,283.52503328)
\curveto(850.47647298,283.46502659)(850.26147319,283.38002668)(850.09147567,283.27003328)
\curveto(849.92147353,283.1600269)(849.80647365,282.97502708)(849.74647567,282.71503328)
\curveto(849.72647373,282.64502741)(849.72147373,282.57502748)(849.73147567,282.50503328)
\curveto(849.7514737,282.43502762)(849.77147368,282.37502768)(849.79147567,282.32503328)
\curveto(849.8514736,282.17502788)(849.92147353,282.06502799)(850.00147567,281.99503328)
\curveto(850.09147336,281.93502812)(850.20147325,281.86502819)(850.33147567,281.78503328)
\curveto(850.49147296,281.68502837)(850.67147278,281.61002845)(850.87147567,281.56003328)
\curveto(851.07147238,281.52002854)(851.27147218,281.47002859)(851.47147567,281.41003328)
\curveto(851.60147185,281.37002869)(851.73147172,281.34002872)(851.86147567,281.32003328)
\curveto(851.99147146,281.30002876)(852.12147133,281.27002879)(852.25147567,281.23003328)
\curveto(852.46147099,281.17002889)(852.66647079,281.11002895)(852.86647567,281.05003328)
\curveto(853.06647039,281.00002906)(853.26647019,280.93502912)(853.46647567,280.85503328)
\lineto(853.61647567,280.79503328)
\curveto(853.66646979,280.77502928)(853.71646974,280.75002931)(853.76647567,280.72003328)
\curveto(853.96646949,280.60002946)(854.14146931,280.46502959)(854.29147567,280.31503328)
\curveto(854.44146901,280.16502989)(854.56646889,279.97503008)(854.66647567,279.74503328)
\curveto(854.68646877,279.67503038)(854.70646875,279.58003048)(854.72647567,279.46003328)
\curveto(854.74646871,279.39003067)(854.7564687,279.31503074)(854.75647567,279.23503328)
\curveto(854.76646869,279.16503089)(854.77146868,279.08503097)(854.77147567,278.99503328)
\lineto(854.77147567,278.84503328)
\curveto(854.7514687,278.77503128)(854.74146871,278.70503135)(854.74147567,278.63503328)
\curveto(854.74146871,278.56503149)(854.73146872,278.49503156)(854.71147567,278.42503328)
\curveto(854.68146877,278.31503174)(854.64646881,278.21003185)(854.60647567,278.11003328)
\curveto(854.56646889,278.01003205)(854.52146893,277.92003214)(854.47147567,277.84003328)
\curveto(854.31146914,277.58003248)(854.10646935,277.37003269)(853.85647567,277.21003328)
\curveto(853.60646985,277.060033)(853.32647013,276.93003313)(853.01647567,276.82003328)
\curveto(852.92647053,276.79003327)(852.83147062,276.77003329)(852.73147567,276.76003328)
\curveto(852.64147081,276.74003332)(852.5514709,276.71503334)(852.46147567,276.68503328)
\curveto(852.36147109,276.66503339)(852.26147119,276.6550334)(852.16147567,276.65503328)
\curveto(852.06147139,276.6550334)(851.96147149,276.64503341)(851.86147567,276.62503328)
\lineto(851.71147567,276.62503328)
\curveto(851.66147179,276.61503344)(851.59147186,276.61003345)(851.50147567,276.61003328)
\curveto(851.41147204,276.61003345)(851.34147211,276.61503344)(851.29147567,276.62503328)
\lineto(851.12647567,276.62503328)
\curveto(851.06647239,276.64503341)(851.00147245,276.6550334)(850.93147567,276.65503328)
\curveto(850.86147259,276.64503341)(850.80147265,276.65003341)(850.75147567,276.67003328)
\curveto(850.70147275,276.68003338)(850.63647282,276.68503337)(850.55647567,276.68503328)
\lineto(850.31647567,276.74503328)
\curveto(850.24647321,276.7550333)(850.17147328,276.77503328)(850.09147567,276.80503328)
\curveto(849.78147367,276.90503315)(849.51147394,277.03003303)(849.28147567,277.18003328)
\curveto(849.0514744,277.33003273)(848.8514746,277.52503253)(848.68147567,277.76503328)
\curveto(848.59147486,277.89503216)(848.51647494,278.03003203)(848.45647567,278.17003328)
\curveto(848.39647506,278.31003175)(848.34147511,278.46503159)(848.29147567,278.63503328)
\curveto(848.27147518,278.69503136)(848.26147519,278.76503129)(848.26147567,278.84503328)
\curveto(848.27147518,278.93503112)(848.28647517,279.00503105)(848.30647567,279.05503328)
\curveto(848.33647512,279.09503096)(848.38647507,279.13503092)(848.45647567,279.17503328)
\curveto(848.50647495,279.19503086)(848.57647488,279.20503085)(848.66647567,279.20503328)
\curveto(848.7564747,279.21503084)(848.84647461,279.21503084)(848.93647567,279.20503328)
\curveto(849.02647443,279.19503086)(849.11147434,279.18003088)(849.19147567,279.16003328)
\curveto(849.28147417,279.15003091)(849.34147411,279.13503092)(849.37147567,279.11503328)
\curveto(849.44147401,279.06503099)(849.48647397,278.99003107)(849.50647567,278.89003328)
\curveto(849.53647392,278.80003126)(849.57147388,278.71503134)(849.61147567,278.63503328)
\curveto(849.71147374,278.41503164)(849.84647361,278.24503181)(850.01647567,278.12503328)
\curveto(850.13647332,278.03503202)(850.27147318,277.96503209)(850.42147567,277.91503328)
\curveto(850.57147288,277.86503219)(850.73147272,277.81503224)(850.90147567,277.76503328)
\lineto(851.21647567,277.72003328)
\lineto(851.30647567,277.72003328)
\curveto(851.37647208,277.70003236)(851.46647199,277.69003237)(851.57647567,277.69003328)
\curveto(851.69647176,277.69003237)(851.79647166,277.70003236)(851.87647567,277.72003328)
\curveto(851.94647151,277.72003234)(852.00147145,277.72503233)(852.04147567,277.73503328)
\curveto(852.10147135,277.74503231)(852.16147129,277.75003231)(852.22147567,277.75003328)
\curveto(852.28147117,277.7600323)(852.33647112,277.77003229)(852.38647567,277.78003328)
\curveto(852.67647078,277.8600322)(852.90647055,277.96503209)(853.07647567,278.09503328)
\curveto(853.24647021,278.22503183)(853.36647009,278.44503161)(853.43647567,278.75503328)
\curveto(853.45647,278.80503125)(853.46146999,278.8600312)(853.45147567,278.92003328)
\curveto(853.44147001,278.98003108)(853.43147002,279.02503103)(853.42147567,279.05503328)
\curveto(853.37147008,279.24503081)(853.30147015,279.38503067)(853.21147567,279.47503328)
\curveto(853.12147033,279.57503048)(853.00647045,279.66503039)(852.86647567,279.74503328)
\curveto(852.77647068,279.80503025)(852.67647078,279.8550302)(852.56647567,279.89503328)
\lineto(852.23647567,280.01503328)
\curveto(852.20647125,280.02503003)(852.17647128,280.03003003)(852.14647567,280.03003328)
\curveto(852.12647133,280.03003003)(852.10147135,280.04003002)(852.07147567,280.06003328)
\curveto(851.73147172,280.17002989)(851.37647208,280.25002981)(851.00647567,280.30003328)
\curveto(850.64647281,280.3600297)(850.30647315,280.4550296)(849.98647567,280.58503328)
\curveto(849.88647357,280.62502943)(849.79147366,280.6600294)(849.70147567,280.69003328)
\curveto(849.61147384,280.72002934)(849.52647393,280.7600293)(849.44647567,280.81003328)
\curveto(849.2564742,280.92002914)(849.08147437,281.04502901)(848.92147567,281.18503328)
\curveto(848.76147469,281.32502873)(848.63647482,281.50002856)(848.54647567,281.71003328)
\curveto(848.51647494,281.78002828)(848.49147496,281.85002821)(848.47147567,281.92003328)
\curveto(848.46147499,281.99002807)(848.44647501,282.06502799)(848.42647567,282.14503328)
\curveto(848.39647506,282.26502779)(848.38647507,282.40002766)(848.39647567,282.55003328)
\curveto(848.40647505,282.71002735)(848.42147503,282.84502721)(848.44147567,282.95503328)
\curveto(848.46147499,283.00502705)(848.47147498,283.04502701)(848.47147567,283.07503328)
\curveto(848.48147497,283.11502694)(848.49647496,283.1550269)(848.51647567,283.19503328)
\curveto(848.60647485,283.42502663)(848.72647473,283.62502643)(848.87647567,283.79503328)
\curveto(849.03647442,283.96502609)(849.21647424,284.11502594)(849.41647567,284.24503328)
\curveto(849.56647389,284.33502572)(849.73147372,284.40502565)(849.91147567,284.45503328)
\curveto(850.09147336,284.51502554)(850.28147317,284.57002549)(850.48147567,284.62003328)
\curveto(850.5514729,284.63002543)(850.61647284,284.64002542)(850.67647567,284.65003328)
\curveto(850.74647271,284.6600254)(850.82147263,284.67002539)(850.90147567,284.68003328)
\curveto(850.93147252,284.69002537)(850.97147248,284.69002537)(851.02147567,284.68003328)
\curveto(851.07147238,284.67002539)(851.10647235,284.67502538)(851.12647567,284.69503328)
}
}
{
\newrgbcolor{curcolor}{0.7019608 0.7019608 0.7019608}
\pscustom[linestyle=none,fillstyle=solid,fillcolor=curcolor]
{
\newpath
\moveto(798.51865829,287.5000699)
\lineto(813.51865829,287.5000699)
\lineto(813.51865829,272.5000699)
\lineto(798.51865829,272.5000699)
\closepath
}
}
{
\newrgbcolor{curcolor}{0 0 0}
\pscustom[linestyle=none,fillstyle=solid,fillcolor=curcolor]
{
\newpath
\moveto(818.5668663,264.43426668)
\lineto(819.4818663,264.43426668)
\curveto(819.58186365,264.43425598)(819.67686355,264.43425598)(819.7668663,264.43426668)
\curveto(819.85686337,264.43425598)(819.9318633,264.414256)(819.9918663,264.37426668)
\curveto(820.08186315,264.3142561)(820.14186309,264.23425618)(820.1718663,264.13426668)
\curveto(820.21186302,264.03425638)(820.25686297,263.92925649)(820.3068663,263.81926668)
\curveto(820.38686284,263.62925679)(820.45686277,263.43925698)(820.5168663,263.24926668)
\curveto(820.58686264,263.05925736)(820.66186257,262.86925755)(820.7418663,262.67926668)
\curveto(820.81186242,262.49925792)(820.87686235,262.3142581)(820.9368663,262.12426668)
\curveto(820.99686223,261.94425847)(821.06686216,261.76425865)(821.1468663,261.58426668)
\curveto(821.20686202,261.44425897)(821.26186197,261.29925912)(821.3118663,261.14926668)
\curveto(821.36186187,260.99925942)(821.41686181,260.85425956)(821.4768663,260.71426668)
\curveto(821.65686157,260.26426015)(821.8268614,259.80926061)(821.9868663,259.34926668)
\curveto(822.14686108,258.89926152)(822.31686091,258.44926197)(822.4968663,257.99926668)
\curveto(822.51686071,257.94926247)(822.5318607,257.89926252)(822.5418663,257.84926668)
\lineto(822.6018663,257.69926668)
\curveto(822.69186054,257.47926294)(822.77686045,257.25426316)(822.8568663,257.02426668)
\curveto(822.93686029,256.80426361)(823.02186021,256.58426383)(823.1118663,256.36426668)
\curveto(823.15186008,256.27426414)(823.19186004,256.16426425)(823.2318663,256.03426668)
\curveto(823.27185996,255.9142645)(823.33685989,255.84426457)(823.4268663,255.82426668)
\curveto(823.46685976,255.8142646)(823.49685973,255.8142646)(823.5168663,255.82426668)
\lineto(823.5768663,255.88426668)
\curveto(823.6268596,255.93426448)(823.66185957,255.98926443)(823.6818663,256.04926668)
\curveto(823.71185952,256.10926431)(823.74185949,256.17426424)(823.7718663,256.24426668)
\lineto(824.0118663,256.87426668)
\curveto(824.09185914,257.09426332)(824.17185906,257.30926311)(824.2518663,257.51926668)
\lineto(824.3118663,257.66926668)
\lineto(824.3718663,257.84926668)
\curveto(824.45185878,258.03926238)(824.52185871,258.22926219)(824.5818663,258.41926668)
\curveto(824.65185858,258.6192618)(824.7268585,258.8192616)(824.8068663,259.01926668)
\curveto(825.04685818,259.59926082)(825.26685796,260.18426023)(825.4668663,260.77426668)
\curveto(825.67685755,261.36425905)(825.90185733,261.94925847)(826.1418663,262.52926668)
\curveto(826.22185701,262.72925769)(826.29685693,262.93425748)(826.3668663,263.14426668)
\curveto(826.44685678,263.35425706)(826.5268567,263.55925686)(826.6068663,263.75926668)
\curveto(826.64685658,263.83925658)(826.68185655,263.93925648)(826.7118663,264.05926668)
\curveto(826.75185648,264.17925624)(826.80685642,264.26425615)(826.8768663,264.31426668)
\curveto(826.93685629,264.35425606)(827.01185622,264.38425603)(827.1018663,264.40426668)
\curveto(827.20185603,264.42425599)(827.31185592,264.43425598)(827.4318663,264.43426668)
\curveto(827.55185568,264.44425597)(827.67185556,264.44425597)(827.7918663,264.43426668)
\curveto(827.91185532,264.43425598)(828.02185521,264.43425598)(828.1218663,264.43426668)
\curveto(828.21185502,264.43425598)(828.30185493,264.43425598)(828.3918663,264.43426668)
\curveto(828.49185474,264.43425598)(828.56685466,264.414256)(828.6168663,264.37426668)
\curveto(828.70685452,264.32425609)(828.75685447,264.23425618)(828.7668663,264.10426668)
\curveto(828.77685445,263.97425644)(828.78185445,263.83425658)(828.7818663,263.68426668)
\lineto(828.7818663,262.03426668)
\lineto(828.7818663,255.76426668)
\lineto(828.7818663,254.50426668)
\curveto(828.78185445,254.39426602)(828.78185445,254.28426613)(828.7818663,254.17426668)
\curveto(828.79185444,254.06426635)(828.77185446,253.97926644)(828.7218663,253.91926668)
\curveto(828.69185454,253.85926656)(828.64685458,253.8192666)(828.5868663,253.79926668)
\curveto(828.5268547,253.78926663)(828.45685477,253.77426664)(828.3768663,253.75426668)
\lineto(828.1368663,253.75426668)
\lineto(827.7768663,253.75426668)
\curveto(827.66685556,253.76426665)(827.58685564,253.80926661)(827.5368663,253.88926668)
\curveto(827.51685571,253.9192665)(827.50185573,253.94926647)(827.4918663,253.97926668)
\curveto(827.49185574,254.0192664)(827.48185575,254.06426635)(827.4618663,254.11426668)
\lineto(827.4618663,254.27926668)
\curveto(827.45185578,254.33926608)(827.44685578,254.40926601)(827.4468663,254.48926668)
\curveto(827.45685577,254.56926585)(827.46185577,254.64426577)(827.4618663,254.71426668)
\lineto(827.4618663,255.55426668)
\lineto(827.4618663,259.97926668)
\curveto(827.46185577,260.22926019)(827.46185577,260.47925994)(827.4618663,260.72926668)
\curveto(827.46185577,260.98925943)(827.45685577,261.23925918)(827.4468663,261.47926668)
\curveto(827.44685578,261.57925884)(827.44185579,261.68925873)(827.4318663,261.80926668)
\curveto(827.42185581,261.92925849)(827.36685586,261.98925843)(827.2668663,261.98926668)
\lineto(827.2668663,261.97426668)
\curveto(827.19685603,261.95425846)(827.13685609,261.88925853)(827.0868663,261.77926668)
\curveto(827.04685618,261.66925875)(827.01185622,261.57425884)(826.9818663,261.49426668)
\curveto(826.91185632,261.32425909)(826.84685638,261.14925927)(826.7868663,260.96926668)
\curveto(826.7268565,260.79925962)(826.65685657,260.62925979)(826.5768663,260.45926668)
\curveto(826.55685667,260.40926001)(826.54185669,260.36426005)(826.5318663,260.32426668)
\curveto(826.52185671,260.28426013)(826.50685672,260.23926018)(826.4868663,260.18926668)
\curveto(826.40685682,260.00926041)(826.33685689,259.82426059)(826.2768663,259.63426668)
\curveto(826.226857,259.45426096)(826.16185707,259.27426114)(826.0818663,259.09426668)
\curveto(826.01185722,258.94426147)(825.95185728,258.78926163)(825.9018663,258.62926668)
\curveto(825.85185738,258.47926194)(825.79685743,258.32926209)(825.7368663,258.17926668)
\curveto(825.53685769,257.70926271)(825.35685787,257.23426318)(825.1968663,256.75426668)
\curveto(825.03685819,256.28426413)(824.86185837,255.8192646)(824.6718663,255.35926668)
\curveto(824.59185864,255.17926524)(824.52185871,254.99926542)(824.4618663,254.81926668)
\curveto(824.40185883,254.63926578)(824.33685889,254.45926596)(824.2668663,254.27926668)
\curveto(824.21685901,254.16926625)(824.16685906,254.06426635)(824.1168663,253.96426668)
\curveto(824.07685915,253.87426654)(823.99185924,253.80926661)(823.8618663,253.76926668)
\curveto(823.84185939,253.75926666)(823.81685941,253.75426666)(823.7868663,253.75426668)
\curveto(823.76685946,253.76426665)(823.74185949,253.76426665)(823.7118663,253.75426668)
\curveto(823.68185955,253.74426667)(823.64685958,253.73926668)(823.6068663,253.73926668)
\curveto(823.56685966,253.74926667)(823.5268597,253.75426666)(823.4868663,253.75426668)
\lineto(823.1868663,253.75426668)
\curveto(823.08686014,253.75426666)(823.00686022,253.77926664)(822.9468663,253.82926668)
\curveto(822.86686036,253.87926654)(822.80686042,253.94926647)(822.7668663,254.03926668)
\curveto(822.73686049,254.13926628)(822.69686053,254.23926618)(822.6468663,254.33926668)
\curveto(822.56686066,254.53926588)(822.48686074,254.74426567)(822.4068663,254.95426668)
\curveto(822.33686089,255.17426524)(822.26186097,255.38426503)(822.1818663,255.58426668)
\curveto(822.10186113,255.76426465)(822.0318612,255.94426447)(821.9718663,256.12426668)
\curveto(821.92186131,256.3142641)(821.85686137,256.49926392)(821.7768663,256.67926668)
\curveto(821.54686168,257.23926318)(821.3318619,257.80426261)(821.1318663,258.37426668)
\curveto(820.9318623,258.94426147)(820.71686251,259.50926091)(820.4868663,260.06926668)
\lineto(820.2468663,260.69926668)
\curveto(820.17686305,260.9192595)(820.10186313,261.12925929)(820.0218663,261.32926668)
\curveto(819.97186326,261.43925898)(819.9268633,261.54425887)(819.8868663,261.64426668)
\curveto(819.85686337,261.75425866)(819.80686342,261.84925857)(819.7368663,261.92926668)
\curveto(819.7268635,261.94925847)(819.71686351,261.95925846)(819.7068663,261.95926668)
\lineto(819.6768663,261.98926668)
\lineto(819.6018663,261.98926668)
\lineto(819.5718663,261.95926668)
\curveto(819.56186367,261.95925846)(819.55186368,261.95425846)(819.5418663,261.94426668)
\curveto(819.52186371,261.89425852)(819.51186372,261.83925858)(819.5118663,261.77926668)
\curveto(819.51186372,261.7192587)(819.50186373,261.65925876)(819.4818663,261.59926668)
\lineto(819.4818663,261.43426668)
\curveto(819.46186377,261.37425904)(819.45686377,261.30925911)(819.4668663,261.23926668)
\curveto(819.47686375,261.16925925)(819.48186375,261.09925932)(819.4818663,261.02926668)
\lineto(819.4818663,260.21926668)
\lineto(819.4818663,255.65926668)
\lineto(819.4818663,254.47426668)
\curveto(819.48186375,254.36426605)(819.47686375,254.25426616)(819.4668663,254.14426668)
\curveto(819.46686376,254.03426638)(819.44186379,253.94926647)(819.3918663,253.88926668)
\curveto(819.34186389,253.80926661)(819.25186398,253.76426665)(819.1218663,253.75426668)
\lineto(818.7318663,253.75426668)
\lineto(818.5368663,253.75426668)
\curveto(818.48686474,253.75426666)(818.43686479,253.76426665)(818.3868663,253.78426668)
\curveto(818.25686497,253.82426659)(818.18186505,253.90926651)(818.1618663,254.03926668)
\curveto(818.15186508,254.16926625)(818.14686508,254.3192661)(818.1468663,254.48926668)
\lineto(818.1468663,256.22926668)
\lineto(818.1468663,262.22926668)
\lineto(818.1468663,263.63926668)
\curveto(818.14686508,263.74925667)(818.14186509,263.86425655)(818.1318663,263.98426668)
\curveto(818.1318651,264.10425631)(818.15686507,264.19925622)(818.2068663,264.26926668)
\curveto(818.24686498,264.32925609)(818.32186491,264.37925604)(818.4318663,264.41926668)
\curveto(818.45186478,264.42925599)(818.47186476,264.42925599)(818.4918663,264.41926668)
\curveto(818.52186471,264.419256)(818.54686468,264.42425599)(818.5668663,264.43426668)
}
}
{
\newrgbcolor{curcolor}{0 0 0}
\pscustom[linestyle=none,fillstyle=solid,fillcolor=curcolor]
{
\newpath
\moveto(837.76897567,254.30926668)
\curveto(837.79896784,254.14926627)(837.78396786,254.0142664)(837.72397567,253.90426668)
\curveto(837.66396798,253.80426661)(837.58396806,253.72926669)(837.48397567,253.67926668)
\curveto(837.43396821,253.65926676)(837.37896826,253.64926677)(837.31897567,253.64926668)
\curveto(837.26896837,253.64926677)(837.21396843,253.63926678)(837.15397567,253.61926668)
\curveto(836.93396871,253.56926685)(836.71396893,253.58426683)(836.49397567,253.66426668)
\curveto(836.28396936,253.73426668)(836.1389695,253.82426659)(836.05897567,253.93426668)
\curveto(836.00896963,254.00426641)(835.96396968,254.08426633)(835.92397567,254.17426668)
\curveto(835.88396976,254.27426614)(835.83396981,254.35426606)(835.77397567,254.41426668)
\curveto(835.75396989,254.43426598)(835.72896991,254.45426596)(835.69897567,254.47426668)
\curveto(835.67896996,254.49426592)(835.64896999,254.49926592)(835.60897567,254.48926668)
\curveto(835.49897014,254.45926596)(835.39397025,254.40426601)(835.29397567,254.32426668)
\curveto(835.20397044,254.24426617)(835.11397053,254.17426624)(835.02397567,254.11426668)
\curveto(834.89397075,254.03426638)(834.75397089,253.95926646)(834.60397567,253.88926668)
\curveto(834.45397119,253.82926659)(834.29397135,253.77426664)(834.12397567,253.72426668)
\curveto(834.02397162,253.69426672)(833.91397173,253.67426674)(833.79397567,253.66426668)
\curveto(833.68397196,253.65426676)(833.57397207,253.63926678)(833.46397567,253.61926668)
\curveto(833.41397223,253.60926681)(833.36897227,253.60426681)(833.32897567,253.60426668)
\lineto(833.22397567,253.60426668)
\curveto(833.11397253,253.58426683)(833.00897263,253.58426683)(832.90897567,253.60426668)
\lineto(832.77397567,253.60426668)
\curveto(832.72397292,253.6142668)(832.67397297,253.6192668)(832.62397567,253.61926668)
\curveto(832.57397307,253.6192668)(832.52897311,253.62926679)(832.48897567,253.64926668)
\curveto(832.44897319,253.65926676)(832.41397323,253.66426675)(832.38397567,253.66426668)
\curveto(832.36397328,253.65426676)(832.3389733,253.65426676)(832.30897567,253.66426668)
\lineto(832.06897567,253.72426668)
\curveto(831.98897365,253.73426668)(831.91397373,253.75426666)(831.84397567,253.78426668)
\curveto(831.5439741,253.9142665)(831.29897434,254.05926636)(831.10897567,254.21926668)
\curveto(830.92897471,254.38926603)(830.77897486,254.62426579)(830.65897567,254.92426668)
\curveto(830.56897507,255.14426527)(830.52397512,255.40926501)(830.52397567,255.71926668)
\lineto(830.52397567,256.03426668)
\curveto(830.53397511,256.08426433)(830.5389751,256.13426428)(830.53897567,256.18426668)
\lineto(830.56897567,256.36426668)
\lineto(830.68897567,256.69426668)
\curveto(830.72897491,256.80426361)(830.77897486,256.90426351)(830.83897567,256.99426668)
\curveto(831.01897462,257.28426313)(831.26397438,257.49926292)(831.57397567,257.63926668)
\curveto(831.88397376,257.77926264)(832.22397342,257.90426251)(832.59397567,258.01426668)
\curveto(832.73397291,258.05426236)(832.87897276,258.08426233)(833.02897567,258.10426668)
\curveto(833.17897246,258.12426229)(833.32897231,258.14926227)(833.47897567,258.17926668)
\curveto(833.54897209,258.19926222)(833.61397203,258.20926221)(833.67397567,258.20926668)
\curveto(833.7439719,258.20926221)(833.81897182,258.2192622)(833.89897567,258.23926668)
\curveto(833.96897167,258.25926216)(834.0389716,258.26926215)(834.10897567,258.26926668)
\curveto(834.17897146,258.27926214)(834.25397139,258.29426212)(834.33397567,258.31426668)
\curveto(834.58397106,258.37426204)(834.81897082,258.42426199)(835.03897567,258.46426668)
\curveto(835.25897038,258.5142619)(835.43397021,258.62926179)(835.56397567,258.80926668)
\curveto(835.62397002,258.88926153)(835.67396997,258.98926143)(835.71397567,259.10926668)
\curveto(835.75396989,259.23926118)(835.75396989,259.37926104)(835.71397567,259.52926668)
\curveto(835.65396999,259.76926065)(835.56397008,259.95926046)(835.44397567,260.09926668)
\curveto(835.33397031,260.23926018)(835.17397047,260.34926007)(834.96397567,260.42926668)
\curveto(834.8439708,260.47925994)(834.69897094,260.5142599)(834.52897567,260.53426668)
\curveto(834.36897127,260.55425986)(834.19897144,260.56425985)(834.01897567,260.56426668)
\curveto(833.8389718,260.56425985)(833.66397198,260.55425986)(833.49397567,260.53426668)
\curveto(833.32397232,260.5142599)(833.17897246,260.48425993)(833.05897567,260.44426668)
\curveto(832.88897275,260.38426003)(832.72397292,260.29926012)(832.56397567,260.18926668)
\curveto(832.48397316,260.12926029)(832.40897323,260.04926037)(832.33897567,259.94926668)
\curveto(832.27897336,259.85926056)(832.22397342,259.75926066)(832.17397567,259.64926668)
\curveto(832.1439735,259.56926085)(832.11397353,259.48426093)(832.08397567,259.39426668)
\curveto(832.06397358,259.30426111)(832.01897362,259.23426118)(831.94897567,259.18426668)
\curveto(831.90897373,259.15426126)(831.8389738,259.12926129)(831.73897567,259.10926668)
\curveto(831.64897399,259.09926132)(831.55397409,259.09426132)(831.45397567,259.09426668)
\curveto(831.35397429,259.09426132)(831.25397439,259.09926132)(831.15397567,259.10926668)
\curveto(831.06397458,259.12926129)(830.99897464,259.15426126)(830.95897567,259.18426668)
\curveto(830.91897472,259.2142612)(830.88897475,259.26426115)(830.86897567,259.33426668)
\curveto(830.84897479,259.40426101)(830.84897479,259.47926094)(830.86897567,259.55926668)
\curveto(830.89897474,259.68926073)(830.92897471,259.80926061)(830.95897567,259.91926668)
\curveto(830.99897464,260.03926038)(831.0439746,260.15426026)(831.09397567,260.26426668)
\curveto(831.28397436,260.6142598)(831.52397412,260.88425953)(831.81397567,261.07426668)
\curveto(832.10397354,261.27425914)(832.46397318,261.43425898)(832.89397567,261.55426668)
\curveto(832.99397265,261.57425884)(833.09397255,261.58925883)(833.19397567,261.59926668)
\curveto(833.30397234,261.60925881)(833.41397223,261.62425879)(833.52397567,261.64426668)
\curveto(833.56397208,261.65425876)(833.62897201,261.65425876)(833.71897567,261.64426668)
\curveto(833.80897183,261.64425877)(833.86397178,261.65425876)(833.88397567,261.67426668)
\curveto(834.58397106,261.68425873)(835.19397045,261.60425881)(835.71397567,261.43426668)
\curveto(836.23396941,261.26425915)(836.59896904,260.93925948)(836.80897567,260.45926668)
\curveto(836.89896874,260.25926016)(836.94896869,260.02426039)(836.95897567,259.75426668)
\curveto(836.97896866,259.49426092)(836.98896865,259.2192612)(836.98897567,258.92926668)
\lineto(836.98897567,255.61426668)
\curveto(836.98896865,255.47426494)(836.99396865,255.33926508)(837.00397567,255.20926668)
\curveto(837.01396863,255.07926534)(837.0439686,254.97426544)(837.09397567,254.89426668)
\curveto(837.1439685,254.82426559)(837.20896843,254.77426564)(837.28897567,254.74426668)
\curveto(837.37896826,254.70426571)(837.46396818,254.67426574)(837.54397567,254.65426668)
\curveto(837.62396802,254.64426577)(837.68396796,254.59926582)(837.72397567,254.51926668)
\curveto(837.7439679,254.48926593)(837.75396789,254.45926596)(837.75397567,254.42926668)
\curveto(837.75396789,254.39926602)(837.75896788,254.35926606)(837.76897567,254.30926668)
\moveto(835.62397567,255.97426668)
\curveto(835.68396996,256.1142643)(835.71396993,256.27426414)(835.71397567,256.45426668)
\curveto(835.72396992,256.64426377)(835.72896991,256.83926358)(835.72897567,257.03926668)
\curveto(835.72896991,257.14926327)(835.72396992,257.24926317)(835.71397567,257.33926668)
\curveto(835.70396994,257.42926299)(835.66396998,257.49926292)(835.59397567,257.54926668)
\curveto(835.56397008,257.56926285)(835.49397015,257.57926284)(835.38397567,257.57926668)
\curveto(835.36397028,257.55926286)(835.32897031,257.54926287)(835.27897567,257.54926668)
\curveto(835.22897041,257.54926287)(835.18397046,257.53926288)(835.14397567,257.51926668)
\curveto(835.06397058,257.49926292)(834.97397067,257.47926294)(834.87397567,257.45926668)
\lineto(834.57397567,257.39926668)
\curveto(834.5439711,257.39926302)(834.50897113,257.39426302)(834.46897567,257.38426668)
\lineto(834.36397567,257.38426668)
\curveto(834.21397143,257.34426307)(834.04897159,257.3192631)(833.86897567,257.30926668)
\curveto(833.69897194,257.30926311)(833.5389721,257.28926313)(833.38897567,257.24926668)
\curveto(833.30897233,257.22926319)(833.23397241,257.20926321)(833.16397567,257.18926668)
\curveto(833.10397254,257.17926324)(833.03397261,257.16426325)(832.95397567,257.14426668)
\curveto(832.79397285,257.09426332)(832.643973,257.02926339)(832.50397567,256.94926668)
\curveto(832.36397328,256.87926354)(832.2439734,256.78926363)(832.14397567,256.67926668)
\curveto(832.0439736,256.56926385)(831.96897367,256.43426398)(831.91897567,256.27426668)
\curveto(831.86897377,256.12426429)(831.84897379,255.93926448)(831.85897567,255.71926668)
\curveto(831.85897378,255.6192648)(831.87397377,255.52426489)(831.90397567,255.43426668)
\curveto(831.9439737,255.35426506)(831.98897365,255.27926514)(832.03897567,255.20926668)
\curveto(832.11897352,255.09926532)(832.22397342,255.00426541)(832.35397567,254.92426668)
\curveto(832.48397316,254.85426556)(832.62397302,254.79426562)(832.77397567,254.74426668)
\curveto(832.82397282,254.73426568)(832.87397277,254.72926569)(832.92397567,254.72926668)
\curveto(832.97397267,254.72926569)(833.02397262,254.72426569)(833.07397567,254.71426668)
\curveto(833.1439725,254.69426572)(833.22897241,254.67926574)(833.32897567,254.66926668)
\curveto(833.4389722,254.66926575)(833.52897211,254.67926574)(833.59897567,254.69926668)
\curveto(833.65897198,254.7192657)(833.71897192,254.72426569)(833.77897567,254.71426668)
\curveto(833.8389718,254.7142657)(833.89897174,254.72426569)(833.95897567,254.74426668)
\curveto(834.0389716,254.76426565)(834.11397153,254.77926564)(834.18397567,254.78926668)
\curveto(834.26397138,254.79926562)(834.3389713,254.8192656)(834.40897567,254.84926668)
\curveto(834.69897094,254.96926545)(834.9439707,255.1142653)(835.14397567,255.28426668)
\curveto(835.35397029,255.45426496)(835.51397013,255.68426473)(835.62397567,255.97426668)
}
}
{
\newrgbcolor{curcolor}{0 0 0}
\pscustom[linestyle=none,fillstyle=solid,fillcolor=curcolor]
{
\newpath
\moveto(839.8856163,263.81926668)
\curveto(840.03561429,263.8192566)(840.18561414,263.8142566)(840.3356163,263.80426668)
\curveto(840.48561384,263.80425661)(840.59061373,263.76425665)(840.6506163,263.68426668)
\curveto(840.70061362,263.62425679)(840.7256136,263.53925688)(840.7256163,263.42926668)
\curveto(840.73561359,263.32925709)(840.74061358,263.22425719)(840.7406163,263.11426668)
\lineto(840.7406163,262.24426668)
\curveto(840.74061358,262.16425825)(840.73561359,262.07925834)(840.7256163,261.98926668)
\curveto(840.7256136,261.90925851)(840.73561359,261.83925858)(840.7556163,261.77926668)
\curveto(840.79561353,261.63925878)(840.88561344,261.54925887)(841.0256163,261.50926668)
\curveto(841.07561325,261.49925892)(841.1206132,261.49425892)(841.1606163,261.49426668)
\lineto(841.3106163,261.49426668)
\lineto(841.7156163,261.49426668)
\curveto(841.87561245,261.50425891)(841.99061233,261.49425892)(842.0606163,261.46426668)
\curveto(842.15061217,261.40425901)(842.21061211,261.34425907)(842.2406163,261.28426668)
\curveto(842.26061206,261.24425917)(842.27061205,261.19925922)(842.2706163,261.14926668)
\lineto(842.2706163,260.99926668)
\curveto(842.27061205,260.88925953)(842.26561206,260.78425963)(842.2556163,260.68426668)
\curveto(842.24561208,260.59425982)(842.21061211,260.52425989)(842.1506163,260.47426668)
\curveto(842.09061223,260.42425999)(842.00561232,260.39426002)(841.8956163,260.38426668)
\lineto(841.5656163,260.38426668)
\curveto(841.45561287,260.39426002)(841.34561298,260.39926002)(841.2356163,260.39926668)
\curveto(841.1256132,260.39926002)(841.03061329,260.38426003)(840.9506163,260.35426668)
\curveto(840.88061344,260.32426009)(840.83061349,260.27426014)(840.8006163,260.20426668)
\curveto(840.77061355,260.13426028)(840.75061357,260.04926037)(840.7406163,259.94926668)
\curveto(840.73061359,259.85926056)(840.7256136,259.75926066)(840.7256163,259.64926668)
\curveto(840.73561359,259.54926087)(840.74061358,259.44926097)(840.7406163,259.34926668)
\lineto(840.7406163,256.37926668)
\curveto(840.74061358,256.15926426)(840.73561359,255.92426449)(840.7256163,255.67426668)
\curveto(840.7256136,255.43426498)(840.77061355,255.24926517)(840.8606163,255.11926668)
\curveto(840.91061341,255.03926538)(840.97561335,254.98426543)(841.0556163,254.95426668)
\curveto(841.13561319,254.92426549)(841.23061309,254.89926552)(841.3406163,254.87926668)
\curveto(841.37061295,254.86926555)(841.40061292,254.86426555)(841.4306163,254.86426668)
\curveto(841.47061285,254.87426554)(841.50561282,254.87426554)(841.5356163,254.86426668)
\lineto(841.7306163,254.86426668)
\curveto(841.83061249,254.86426555)(841.9206124,254.85426556)(842.0006163,254.83426668)
\curveto(842.09061223,254.82426559)(842.15561217,254.78926563)(842.1956163,254.72926668)
\curveto(842.21561211,254.69926572)(842.23061209,254.64426577)(842.2406163,254.56426668)
\curveto(842.26061206,254.49426592)(842.27061205,254.419266)(842.2706163,254.33926668)
\curveto(842.28061204,254.25926616)(842.28061204,254.17926624)(842.2706163,254.09926668)
\curveto(842.26061206,254.02926639)(842.24061208,253.97426644)(842.2106163,253.93426668)
\curveto(842.17061215,253.86426655)(842.09561223,253.8142666)(841.9856163,253.78426668)
\curveto(841.90561242,253.76426665)(841.81561251,253.75426666)(841.7156163,253.75426668)
\curveto(841.61561271,253.76426665)(841.5256128,253.76926665)(841.4456163,253.76926668)
\curveto(841.38561294,253.76926665)(841.325613,253.76426665)(841.2656163,253.75426668)
\curveto(841.20561312,253.75426666)(841.15061317,253.75926666)(841.1006163,253.76926668)
\lineto(840.9206163,253.76926668)
\curveto(840.87061345,253.77926664)(840.8206135,253.78426663)(840.7706163,253.78426668)
\curveto(840.73061359,253.79426662)(840.68561364,253.79926662)(840.6356163,253.79926668)
\curveto(840.43561389,253.84926657)(840.26061406,253.90426651)(840.1106163,253.96426668)
\curveto(839.97061435,254.02426639)(839.85061447,254.12926629)(839.7506163,254.27926668)
\curveto(839.61061471,254.47926594)(839.53061479,254.72926569)(839.5106163,255.02926668)
\curveto(839.49061483,255.33926508)(839.48061484,255.66926475)(839.4806163,256.01926668)
\lineto(839.4806163,259.94926668)
\curveto(839.45061487,260.07926034)(839.4206149,260.17426024)(839.3906163,260.23426668)
\curveto(839.37061495,260.29426012)(839.30061502,260.34426007)(839.1806163,260.38426668)
\curveto(839.14061518,260.39426002)(839.10061522,260.39426002)(839.0606163,260.38426668)
\curveto(839.0206153,260.37426004)(838.98061534,260.37926004)(838.9406163,260.39926668)
\lineto(838.7006163,260.39926668)
\curveto(838.57061575,260.39926002)(838.46061586,260.40926001)(838.3706163,260.42926668)
\curveto(838.29061603,260.45925996)(838.23561609,260.5192599)(838.2056163,260.60926668)
\curveto(838.18561614,260.64925977)(838.17061615,260.69425972)(838.1606163,260.74426668)
\lineto(838.1606163,260.89426668)
\curveto(838.16061616,261.03425938)(838.17061615,261.14925927)(838.1906163,261.23926668)
\curveto(838.21061611,261.33925908)(838.27061605,261.414259)(838.3706163,261.46426668)
\curveto(838.48061584,261.50425891)(838.6206157,261.5142589)(838.7906163,261.49426668)
\curveto(838.97061535,261.47425894)(839.1206152,261.48425893)(839.2406163,261.52426668)
\curveto(839.33061499,261.57425884)(839.40061492,261.64425877)(839.4506163,261.73426668)
\curveto(839.47061485,261.79425862)(839.48061484,261.86925855)(839.4806163,261.95926668)
\lineto(839.4806163,262.21426668)
\lineto(839.4806163,263.14426668)
\lineto(839.4806163,263.38426668)
\curveto(839.48061484,263.47425694)(839.49061483,263.54925687)(839.5106163,263.60926668)
\curveto(839.55061477,263.68925673)(839.6256147,263.75425666)(839.7356163,263.80426668)
\curveto(839.76561456,263.80425661)(839.79061453,263.80425661)(839.8106163,263.80426668)
\curveto(839.84061448,263.8142566)(839.86561446,263.8192566)(839.8856163,263.81926668)
}
}
{
\newrgbcolor{curcolor}{0 0 0}
\pscustom[linestyle=none,fillstyle=solid,fillcolor=curcolor]
{
\newpath
\moveto(850.40741317,257.92426668)
\curveto(850.42740549,257.82426259)(850.42740549,257.70926271)(850.40741317,257.57926668)
\curveto(850.39740552,257.45926296)(850.36740555,257.37426304)(850.31741317,257.32426668)
\curveto(850.26740565,257.28426313)(850.19240572,257.25426316)(850.09241317,257.23426668)
\curveto(850.00240591,257.22426319)(849.89740602,257.2192632)(849.77741317,257.21926668)
\lineto(849.41741317,257.21926668)
\curveto(849.29740662,257.22926319)(849.19240672,257.23426318)(849.10241317,257.23426668)
\lineto(845.26241317,257.23426668)
\curveto(845.18241073,257.23426318)(845.10241081,257.22926319)(845.02241317,257.21926668)
\curveto(844.94241097,257.2192632)(844.87741104,257.20426321)(844.82741317,257.17426668)
\curveto(844.78741113,257.15426326)(844.74741117,257.1142633)(844.70741317,257.05426668)
\curveto(844.68741123,257.02426339)(844.66741125,256.97926344)(844.64741317,256.91926668)
\curveto(844.62741129,256.86926355)(844.62741129,256.8192636)(844.64741317,256.76926668)
\curveto(844.65741126,256.7192637)(844.66241125,256.67426374)(844.66241317,256.63426668)
\curveto(844.66241125,256.59426382)(844.66741125,256.55426386)(844.67741317,256.51426668)
\curveto(844.69741122,256.43426398)(844.7174112,256.34926407)(844.73741317,256.25926668)
\curveto(844.75741116,256.17926424)(844.78741113,256.09926432)(844.82741317,256.01926668)
\curveto(845.05741086,255.47926494)(845.43741048,255.09426532)(845.96741317,254.86426668)
\curveto(846.02740989,254.83426558)(846.09240982,254.80926561)(846.16241317,254.78926668)
\lineto(846.37241317,254.72926668)
\curveto(846.40240951,254.7192657)(846.45240946,254.7142657)(846.52241317,254.71426668)
\curveto(846.66240925,254.67426574)(846.84740907,254.65426576)(847.07741317,254.65426668)
\curveto(847.30740861,254.65426576)(847.49240842,254.67426574)(847.63241317,254.71426668)
\curveto(847.77240814,254.75426566)(847.89740802,254.79426562)(848.00741317,254.83426668)
\curveto(848.12740779,254.88426553)(848.23740768,254.94426547)(848.33741317,255.01426668)
\curveto(848.44740747,255.08426533)(848.54240737,255.16426525)(848.62241317,255.25426668)
\curveto(848.70240721,255.35426506)(848.77240714,255.45926496)(848.83241317,255.56926668)
\curveto(848.89240702,255.66926475)(848.94240697,255.77426464)(848.98241317,255.88426668)
\curveto(849.03240688,255.99426442)(849.1124068,256.07426434)(849.22241317,256.12426668)
\curveto(849.26240665,256.14426427)(849.32740659,256.15926426)(849.41741317,256.16926668)
\curveto(849.50740641,256.17926424)(849.59740632,256.17926424)(849.68741317,256.16926668)
\curveto(849.77740614,256.16926425)(849.86240605,256.16426425)(849.94241317,256.15426668)
\curveto(850.02240589,256.14426427)(850.07740584,256.12426429)(850.10741317,256.09426668)
\curveto(850.20740571,256.02426439)(850.23240568,255.90926451)(850.18241317,255.74926668)
\curveto(850.10240581,255.47926494)(849.99740592,255.23926518)(849.86741317,255.02926668)
\curveto(849.66740625,254.70926571)(849.43740648,254.44426597)(849.17741317,254.23426668)
\curveto(848.92740699,254.03426638)(848.60740731,253.86926655)(848.21741317,253.73926668)
\curveto(848.1174078,253.69926672)(848.0174079,253.67426674)(847.91741317,253.66426668)
\curveto(847.8174081,253.64426677)(847.7124082,253.62426679)(847.60241317,253.60426668)
\curveto(847.55240836,253.59426682)(847.50240841,253.58926683)(847.45241317,253.58926668)
\curveto(847.4124085,253.58926683)(847.36740855,253.58426683)(847.31741317,253.57426668)
\lineto(847.16741317,253.57426668)
\curveto(847.1174088,253.56426685)(847.05740886,253.55926686)(846.98741317,253.55926668)
\curveto(846.92740899,253.55926686)(846.87740904,253.56426685)(846.83741317,253.57426668)
\lineto(846.70241317,253.57426668)
\curveto(846.65240926,253.58426683)(846.60740931,253.58926683)(846.56741317,253.58926668)
\curveto(846.52740939,253.58926683)(846.48740943,253.59426682)(846.44741317,253.60426668)
\curveto(846.39740952,253.6142668)(846.34240957,253.62426679)(846.28241317,253.63426668)
\curveto(846.22240969,253.63426678)(846.16740975,253.63926678)(846.11741317,253.64926668)
\curveto(846.02740989,253.66926675)(845.93740998,253.69426672)(845.84741317,253.72426668)
\curveto(845.75741016,253.74426667)(845.67241024,253.76926665)(845.59241317,253.79926668)
\curveto(845.55241036,253.8192666)(845.5174104,253.82926659)(845.48741317,253.82926668)
\curveto(845.45741046,253.83926658)(845.42241049,253.85426656)(845.38241317,253.87426668)
\curveto(845.23241068,253.94426647)(845.07241084,254.02926639)(844.90241317,254.12926668)
\curveto(844.6124113,254.3192661)(844.36241155,254.54926587)(844.15241317,254.81926668)
\curveto(843.95241196,255.09926532)(843.78241213,255.40926501)(843.64241317,255.74926668)
\curveto(843.59241232,255.85926456)(843.55241236,255.97426444)(843.52241317,256.09426668)
\curveto(843.50241241,256.2142642)(843.47241244,256.33426408)(843.43241317,256.45426668)
\curveto(843.42241249,256.49426392)(843.4174125,256.52926389)(843.41741317,256.55926668)
\curveto(843.4174125,256.58926383)(843.4124125,256.62926379)(843.40241317,256.67926668)
\curveto(843.38241253,256.75926366)(843.36741255,256.84426357)(843.35741317,256.93426668)
\curveto(843.34741257,257.02426339)(843.33241258,257.1142633)(843.31241317,257.20426668)
\lineto(843.31241317,257.41426668)
\curveto(843.30241261,257.45426296)(843.29241262,257.50926291)(843.28241317,257.57926668)
\curveto(843.28241263,257.65926276)(843.28741263,257.72426269)(843.29741317,257.77426668)
\lineto(843.29741317,257.93926668)
\curveto(843.3174126,257.98926243)(843.32241259,258.03926238)(843.31241317,258.08926668)
\curveto(843.3124126,258.14926227)(843.3174126,258.20426221)(843.32741317,258.25426668)
\curveto(843.36741255,258.414262)(843.39741252,258.57426184)(843.41741317,258.73426668)
\curveto(843.44741247,258.89426152)(843.49241242,259.04426137)(843.55241317,259.18426668)
\curveto(843.60241231,259.29426112)(843.64741227,259.40426101)(843.68741317,259.51426668)
\curveto(843.73741218,259.63426078)(843.79241212,259.74926067)(843.85241317,259.85926668)
\curveto(844.07241184,260.20926021)(844.32241159,260.50925991)(844.60241317,260.75926668)
\curveto(844.88241103,261.0192594)(845.22741069,261.23425918)(845.63741317,261.40426668)
\curveto(845.75741016,261.45425896)(845.87741004,261.48925893)(845.99741317,261.50926668)
\curveto(846.12740979,261.53925888)(846.26240965,261.56925885)(846.40241317,261.59926668)
\curveto(846.45240946,261.60925881)(846.49740942,261.6142588)(846.53741317,261.61426668)
\curveto(846.57740934,261.62425879)(846.62240929,261.62925879)(846.67241317,261.62926668)
\curveto(846.69240922,261.63925878)(846.7174092,261.63925878)(846.74741317,261.62926668)
\curveto(846.77740914,261.6192588)(846.80240911,261.62425879)(846.82241317,261.64426668)
\curveto(847.24240867,261.65425876)(847.60740831,261.60925881)(847.91741317,261.50926668)
\curveto(848.22740769,261.419259)(848.50740741,261.29425912)(848.75741317,261.13426668)
\curveto(848.80740711,261.1142593)(848.84740707,261.08425933)(848.87741317,261.04426668)
\curveto(848.90740701,261.0142594)(848.94240697,260.98925943)(848.98241317,260.96926668)
\curveto(849.06240685,260.90925951)(849.14240677,260.83925958)(849.22241317,260.75926668)
\curveto(849.3124066,260.67925974)(849.38740653,260.59925982)(849.44741317,260.51926668)
\curveto(849.60740631,260.30926011)(849.74240617,260.10926031)(849.85241317,259.91926668)
\curveto(849.92240599,259.80926061)(849.97740594,259.68926073)(850.01741317,259.55926668)
\curveto(850.05740586,259.42926099)(850.10240581,259.29926112)(850.15241317,259.16926668)
\curveto(850.20240571,259.03926138)(850.23740568,258.90426151)(850.25741317,258.76426668)
\curveto(850.28740563,258.62426179)(850.32240559,258.48426193)(850.36241317,258.34426668)
\curveto(850.37240554,258.27426214)(850.37740554,258.20426221)(850.37741317,258.13426668)
\lineto(850.40741317,257.92426668)
\moveto(848.95241317,258.43426668)
\curveto(848.98240693,258.47426194)(849.00740691,258.52426189)(849.02741317,258.58426668)
\curveto(849.04740687,258.65426176)(849.04740687,258.72426169)(849.02741317,258.79426668)
\curveto(848.96740695,259.0142614)(848.88240703,259.2192612)(848.77241317,259.40926668)
\curveto(848.63240728,259.63926078)(848.47740744,259.83426058)(848.30741317,259.99426668)
\curveto(848.13740778,260.15426026)(847.917408,260.28926013)(847.64741317,260.39926668)
\curveto(847.57740834,260.41926)(847.50740841,260.43425998)(847.43741317,260.44426668)
\curveto(847.36740855,260.46425995)(847.29240862,260.48425993)(847.21241317,260.50426668)
\curveto(847.13240878,260.52425989)(847.04740887,260.53425988)(846.95741317,260.53426668)
\lineto(846.70241317,260.53426668)
\curveto(846.67240924,260.5142599)(846.63740928,260.50425991)(846.59741317,260.50426668)
\curveto(846.55740936,260.5142599)(846.52240939,260.5142599)(846.49241317,260.50426668)
\lineto(846.25241317,260.44426668)
\curveto(846.18240973,260.43425998)(846.1124098,260.41926)(846.04241317,260.39926668)
\curveto(845.75241016,260.27926014)(845.5174104,260.12926029)(845.33741317,259.94926668)
\curveto(845.16741075,259.76926065)(845.0124109,259.54426087)(844.87241317,259.27426668)
\curveto(844.84241107,259.22426119)(844.8124111,259.15926126)(844.78241317,259.07926668)
\curveto(844.75241116,259.00926141)(844.72741119,258.92926149)(844.70741317,258.83926668)
\curveto(844.68741123,258.74926167)(844.68241123,258.66426175)(844.69241317,258.58426668)
\curveto(844.70241121,258.50426191)(844.73741118,258.44426197)(844.79741317,258.40426668)
\curveto(844.87741104,258.34426207)(845.0124109,258.3142621)(845.20241317,258.31426668)
\curveto(845.40241051,258.32426209)(845.57241034,258.32926209)(845.71241317,258.32926668)
\lineto(847.99241317,258.32926668)
\curveto(848.14240777,258.32926209)(848.32240759,258.32426209)(848.53241317,258.31426668)
\curveto(848.74240717,258.3142621)(848.88240703,258.35426206)(848.95241317,258.43426668)
}
}
{
\newrgbcolor{curcolor}{0 0 0}
\pscustom[linestyle=none,fillstyle=solid,fillcolor=curcolor]
{
\newpath
\moveto(855.3590538,261.65926668)
\curveto(855.58904901,261.65925876)(855.71904888,261.59925882)(855.7490538,261.47926668)
\curveto(855.77904882,261.36925905)(855.7940488,261.20425921)(855.7940538,260.98426668)
\lineto(855.7940538,260.69926668)
\curveto(855.7940488,260.60925981)(855.76904883,260.53425988)(855.7190538,260.47426668)
\curveto(855.65904894,260.39426002)(855.57404902,260.34926007)(855.4640538,260.33926668)
\curveto(855.35404924,260.33926008)(855.24404935,260.32426009)(855.1340538,260.29426668)
\curveto(854.9940496,260.26426015)(854.85904974,260.23426018)(854.7290538,260.20426668)
\curveto(854.60904999,260.17426024)(854.4940501,260.13426028)(854.3840538,260.08426668)
\curveto(854.0940505,259.95426046)(853.85905074,259.77426064)(853.6790538,259.54426668)
\curveto(853.4990511,259.32426109)(853.34405125,259.06926135)(853.2140538,258.77926668)
\curveto(853.17405142,258.66926175)(853.14405145,258.55426186)(853.1240538,258.43426668)
\curveto(853.10405149,258.32426209)(853.07905152,258.20926221)(853.0490538,258.08926668)
\curveto(853.03905156,258.03926238)(853.03405156,257.98926243)(853.0340538,257.93926668)
\curveto(853.04405155,257.88926253)(853.04405155,257.83926258)(853.0340538,257.78926668)
\curveto(853.00405159,257.66926275)(852.98905161,257.52926289)(852.9890538,257.36926668)
\curveto(852.9990516,257.2192632)(853.00405159,257.07426334)(853.0040538,256.93426668)
\lineto(853.0040538,255.08926668)
\lineto(853.0040538,254.74426668)
\curveto(853.00405159,254.62426579)(852.9990516,254.50926591)(852.9890538,254.39926668)
\curveto(852.97905162,254.28926613)(852.97405162,254.19426622)(852.9740538,254.11426668)
\curveto(852.98405161,254.03426638)(852.96405163,253.96426645)(852.9140538,253.90426668)
\curveto(852.86405173,253.83426658)(852.78405181,253.79426662)(852.6740538,253.78426668)
\curveto(852.57405202,253.77426664)(852.46405213,253.76926665)(852.3440538,253.76926668)
\lineto(852.0740538,253.76926668)
\curveto(852.02405257,253.78926663)(851.97405262,253.80426661)(851.9240538,253.81426668)
\curveto(851.88405271,253.83426658)(851.85405274,253.85926656)(851.8340538,253.88926668)
\curveto(851.78405281,253.95926646)(851.75405284,254.04426637)(851.7440538,254.14426668)
\lineto(851.7440538,254.47426668)
\lineto(851.7440538,255.62926668)
\lineto(851.7440538,259.78426668)
\lineto(851.7440538,260.81926668)
\lineto(851.7440538,261.11926668)
\curveto(851.75405284,261.2192592)(851.78405281,261.30425911)(851.8340538,261.37426668)
\curveto(851.86405273,261.414259)(851.91405268,261.44425897)(851.9840538,261.46426668)
\curveto(852.06405253,261.48425893)(852.14905245,261.49425892)(852.2390538,261.49426668)
\curveto(852.32905227,261.50425891)(852.41905218,261.50425891)(852.5090538,261.49426668)
\curveto(852.599052,261.48425893)(852.66905193,261.46925895)(852.7190538,261.44926668)
\curveto(852.7990518,261.419259)(852.84905175,261.35925906)(852.8690538,261.26926668)
\curveto(852.8990517,261.18925923)(852.91405168,261.09925932)(852.9140538,260.99926668)
\lineto(852.9140538,260.69926668)
\curveto(852.91405168,260.59925982)(852.93405166,260.50925991)(852.9740538,260.42926668)
\curveto(852.98405161,260.40926001)(852.9940516,260.39426002)(853.0040538,260.38426668)
\lineto(853.0490538,260.33926668)
\curveto(853.15905144,260.33926008)(853.24905135,260.38426003)(853.3190538,260.47426668)
\curveto(853.38905121,260.57425984)(853.44905115,260.65425976)(853.4990538,260.71426668)
\lineto(853.5890538,260.80426668)
\curveto(853.67905092,260.9142595)(853.80405079,261.02925939)(853.9640538,261.14926668)
\curveto(854.12405047,261.26925915)(854.27405032,261.35925906)(854.4140538,261.41926668)
\curveto(854.50405009,261.46925895)(854.59905,261.50425891)(854.6990538,261.52426668)
\curveto(854.7990498,261.55425886)(854.90404969,261.58425883)(855.0140538,261.61426668)
\curveto(855.07404952,261.62425879)(855.13404946,261.62925879)(855.1940538,261.62926668)
\curveto(855.25404934,261.63925878)(855.30904929,261.64925877)(855.3590538,261.65926668)
}
}
{
\newrgbcolor{curcolor}{0 0 0}
\pscustom[linestyle=none,fillstyle=solid,fillcolor=curcolor]
{
\newpath
\moveto(857.00881942,262.97926668)
\curveto(856.9288183,263.03925738)(856.88381835,263.14425727)(856.87381942,263.29426668)
\lineto(856.87381942,263.75926668)
\lineto(856.87381942,264.01426668)
\curveto(856.87381836,264.10425631)(856.88881834,264.17925624)(856.91881942,264.23926668)
\curveto(856.95881827,264.3192561)(857.03881819,264.37925604)(857.15881942,264.41926668)
\curveto(857.17881805,264.42925599)(857.19881803,264.42925599)(857.21881942,264.41926668)
\curveto(857.24881798,264.419256)(857.27381796,264.42425599)(857.29381942,264.43426668)
\curveto(857.46381777,264.43425598)(857.62381761,264.42925599)(857.77381942,264.41926668)
\curveto(857.92381731,264.40925601)(858.02381721,264.34925607)(858.07381942,264.23926668)
\curveto(858.10381713,264.17925624)(858.11881711,264.10425631)(858.11881942,264.01426668)
\lineto(858.11881942,263.75926668)
\curveto(858.11881711,263.57925684)(858.11381712,263.40925701)(858.10381942,263.24926668)
\curveto(858.10381713,263.08925733)(858.03881719,262.98425743)(857.90881942,262.93426668)
\curveto(857.85881737,262.9142575)(857.80381743,262.90425751)(857.74381942,262.90426668)
\lineto(857.57881942,262.90426668)
\lineto(857.26381942,262.90426668)
\curveto(857.16381807,262.90425751)(857.07881815,262.92925749)(857.00881942,262.97926668)
\moveto(858.11881942,254.47426668)
\lineto(858.11881942,254.15926668)
\curveto(858.1288171,254.05926636)(858.10881712,253.97926644)(858.05881942,253.91926668)
\curveto(858.0288172,253.85926656)(857.98381725,253.8192666)(857.92381942,253.79926668)
\curveto(857.86381737,253.78926663)(857.79381744,253.77426664)(857.71381942,253.75426668)
\lineto(857.48881942,253.75426668)
\curveto(857.35881787,253.75426666)(857.24381799,253.75926666)(857.14381942,253.76926668)
\curveto(857.05381818,253.78926663)(856.98381825,253.83926658)(856.93381942,253.91926668)
\curveto(856.89381834,253.97926644)(856.87381836,254.05426636)(856.87381942,254.14426668)
\lineto(856.87381942,254.42926668)
\lineto(856.87381942,260.77426668)
\lineto(856.87381942,261.08926668)
\curveto(856.87381836,261.19925922)(856.89881833,261.28425913)(856.94881942,261.34426668)
\curveto(856.97881825,261.39425902)(857.01881821,261.42425899)(857.06881942,261.43426668)
\curveto(857.11881811,261.44425897)(857.17381806,261.45925896)(857.23381942,261.47926668)
\curveto(857.25381798,261.47925894)(857.27381796,261.47425894)(857.29381942,261.46426668)
\curveto(857.32381791,261.46425895)(857.34881788,261.46925895)(857.36881942,261.47926668)
\curveto(857.49881773,261.47925894)(857.6288176,261.47425894)(857.75881942,261.46426668)
\curveto(857.89881733,261.46425895)(857.99381724,261.42425899)(858.04381942,261.34426668)
\curveto(858.09381714,261.28425913)(858.11881711,261.20425921)(858.11881942,261.10426668)
\lineto(858.11881942,260.81926668)
\lineto(858.11881942,254.47426668)
}
}
{
\newrgbcolor{curcolor}{0 0 0}
\pscustom[linestyle=none,fillstyle=solid,fillcolor=curcolor]
{
\newpath
\moveto(866.94866317,254.30926668)
\curveto(866.97865534,254.14926627)(866.96365536,254.0142664)(866.90366317,253.90426668)
\curveto(866.84365548,253.80426661)(866.76365556,253.72926669)(866.66366317,253.67926668)
\curveto(866.61365571,253.65926676)(866.55865576,253.64926677)(866.49866317,253.64926668)
\curveto(866.44865587,253.64926677)(866.39365593,253.63926678)(866.33366317,253.61926668)
\curveto(866.11365621,253.56926685)(865.89365643,253.58426683)(865.67366317,253.66426668)
\curveto(865.46365686,253.73426668)(865.318657,253.82426659)(865.23866317,253.93426668)
\curveto(865.18865713,254.00426641)(865.14365718,254.08426633)(865.10366317,254.17426668)
\curveto(865.06365726,254.27426614)(865.01365731,254.35426606)(864.95366317,254.41426668)
\curveto(864.93365739,254.43426598)(864.90865741,254.45426596)(864.87866317,254.47426668)
\curveto(864.85865746,254.49426592)(864.82865749,254.49926592)(864.78866317,254.48926668)
\curveto(864.67865764,254.45926596)(864.57365775,254.40426601)(864.47366317,254.32426668)
\curveto(864.38365794,254.24426617)(864.29365803,254.17426624)(864.20366317,254.11426668)
\curveto(864.07365825,254.03426638)(863.93365839,253.95926646)(863.78366317,253.88926668)
\curveto(863.63365869,253.82926659)(863.47365885,253.77426664)(863.30366317,253.72426668)
\curveto(863.20365912,253.69426672)(863.09365923,253.67426674)(862.97366317,253.66426668)
\curveto(862.86365946,253.65426676)(862.75365957,253.63926678)(862.64366317,253.61926668)
\curveto(862.59365973,253.60926681)(862.54865977,253.60426681)(862.50866317,253.60426668)
\lineto(862.40366317,253.60426668)
\curveto(862.29366003,253.58426683)(862.18866013,253.58426683)(862.08866317,253.60426668)
\lineto(861.95366317,253.60426668)
\curveto(861.90366042,253.6142668)(861.85366047,253.6192668)(861.80366317,253.61926668)
\curveto(861.75366057,253.6192668)(861.70866061,253.62926679)(861.66866317,253.64926668)
\curveto(861.62866069,253.65926676)(861.59366073,253.66426675)(861.56366317,253.66426668)
\curveto(861.54366078,253.65426676)(861.5186608,253.65426676)(861.48866317,253.66426668)
\lineto(861.24866317,253.72426668)
\curveto(861.16866115,253.73426668)(861.09366123,253.75426666)(861.02366317,253.78426668)
\curveto(860.7236616,253.9142665)(860.47866184,254.05926636)(860.28866317,254.21926668)
\curveto(860.10866221,254.38926603)(859.95866236,254.62426579)(859.83866317,254.92426668)
\curveto(859.74866257,255.14426527)(859.70366262,255.40926501)(859.70366317,255.71926668)
\lineto(859.70366317,256.03426668)
\curveto(859.71366261,256.08426433)(859.7186626,256.13426428)(859.71866317,256.18426668)
\lineto(859.74866317,256.36426668)
\lineto(859.86866317,256.69426668)
\curveto(859.90866241,256.80426361)(859.95866236,256.90426351)(860.01866317,256.99426668)
\curveto(860.19866212,257.28426313)(860.44366188,257.49926292)(860.75366317,257.63926668)
\curveto(861.06366126,257.77926264)(861.40366092,257.90426251)(861.77366317,258.01426668)
\curveto(861.91366041,258.05426236)(862.05866026,258.08426233)(862.20866317,258.10426668)
\curveto(862.35865996,258.12426229)(862.50865981,258.14926227)(862.65866317,258.17926668)
\curveto(862.72865959,258.19926222)(862.79365953,258.20926221)(862.85366317,258.20926668)
\curveto(862.9236594,258.20926221)(862.99865932,258.2192622)(863.07866317,258.23926668)
\curveto(863.14865917,258.25926216)(863.2186591,258.26926215)(863.28866317,258.26926668)
\curveto(863.35865896,258.27926214)(863.43365889,258.29426212)(863.51366317,258.31426668)
\curveto(863.76365856,258.37426204)(863.99865832,258.42426199)(864.21866317,258.46426668)
\curveto(864.43865788,258.5142619)(864.61365771,258.62926179)(864.74366317,258.80926668)
\curveto(864.80365752,258.88926153)(864.85365747,258.98926143)(864.89366317,259.10926668)
\curveto(864.93365739,259.23926118)(864.93365739,259.37926104)(864.89366317,259.52926668)
\curveto(864.83365749,259.76926065)(864.74365758,259.95926046)(864.62366317,260.09926668)
\curveto(864.51365781,260.23926018)(864.35365797,260.34926007)(864.14366317,260.42926668)
\curveto(864.0236583,260.47925994)(863.87865844,260.5142599)(863.70866317,260.53426668)
\curveto(863.54865877,260.55425986)(863.37865894,260.56425985)(863.19866317,260.56426668)
\curveto(863.0186593,260.56425985)(862.84365948,260.55425986)(862.67366317,260.53426668)
\curveto(862.50365982,260.5142599)(862.35865996,260.48425993)(862.23866317,260.44426668)
\curveto(862.06866025,260.38426003)(861.90366042,260.29926012)(861.74366317,260.18926668)
\curveto(861.66366066,260.12926029)(861.58866073,260.04926037)(861.51866317,259.94926668)
\curveto(861.45866086,259.85926056)(861.40366092,259.75926066)(861.35366317,259.64926668)
\curveto(861.323661,259.56926085)(861.29366103,259.48426093)(861.26366317,259.39426668)
\curveto(861.24366108,259.30426111)(861.19866112,259.23426118)(861.12866317,259.18426668)
\curveto(861.08866123,259.15426126)(861.0186613,259.12926129)(860.91866317,259.10926668)
\curveto(860.82866149,259.09926132)(860.73366159,259.09426132)(860.63366317,259.09426668)
\curveto(860.53366179,259.09426132)(860.43366189,259.09926132)(860.33366317,259.10926668)
\curveto(860.24366208,259.12926129)(860.17866214,259.15426126)(860.13866317,259.18426668)
\curveto(860.09866222,259.2142612)(860.06866225,259.26426115)(860.04866317,259.33426668)
\curveto(860.02866229,259.40426101)(860.02866229,259.47926094)(860.04866317,259.55926668)
\curveto(860.07866224,259.68926073)(860.10866221,259.80926061)(860.13866317,259.91926668)
\curveto(860.17866214,260.03926038)(860.2236621,260.15426026)(860.27366317,260.26426668)
\curveto(860.46366186,260.6142598)(860.70366162,260.88425953)(860.99366317,261.07426668)
\curveto(861.28366104,261.27425914)(861.64366068,261.43425898)(862.07366317,261.55426668)
\curveto(862.17366015,261.57425884)(862.27366005,261.58925883)(862.37366317,261.59926668)
\curveto(862.48365984,261.60925881)(862.59365973,261.62425879)(862.70366317,261.64426668)
\curveto(862.74365958,261.65425876)(862.80865951,261.65425876)(862.89866317,261.64426668)
\curveto(862.98865933,261.64425877)(863.04365928,261.65425876)(863.06366317,261.67426668)
\curveto(863.76365856,261.68425873)(864.37365795,261.60425881)(864.89366317,261.43426668)
\curveto(865.41365691,261.26425915)(865.77865654,260.93925948)(865.98866317,260.45926668)
\curveto(866.07865624,260.25926016)(866.12865619,260.02426039)(866.13866317,259.75426668)
\curveto(866.15865616,259.49426092)(866.16865615,259.2192612)(866.16866317,258.92926668)
\lineto(866.16866317,255.61426668)
\curveto(866.16865615,255.47426494)(866.17365615,255.33926508)(866.18366317,255.20926668)
\curveto(866.19365613,255.07926534)(866.2236561,254.97426544)(866.27366317,254.89426668)
\curveto(866.323656,254.82426559)(866.38865593,254.77426564)(866.46866317,254.74426668)
\curveto(866.55865576,254.70426571)(866.64365568,254.67426574)(866.72366317,254.65426668)
\curveto(866.80365552,254.64426577)(866.86365546,254.59926582)(866.90366317,254.51926668)
\curveto(866.9236554,254.48926593)(866.93365539,254.45926596)(866.93366317,254.42926668)
\curveto(866.93365539,254.39926602)(866.93865538,254.35926606)(866.94866317,254.30926668)
\moveto(864.80366317,255.97426668)
\curveto(864.86365746,256.1142643)(864.89365743,256.27426414)(864.89366317,256.45426668)
\curveto(864.90365742,256.64426377)(864.90865741,256.83926358)(864.90866317,257.03926668)
\curveto(864.90865741,257.14926327)(864.90365742,257.24926317)(864.89366317,257.33926668)
\curveto(864.88365744,257.42926299)(864.84365748,257.49926292)(864.77366317,257.54926668)
\curveto(864.74365758,257.56926285)(864.67365765,257.57926284)(864.56366317,257.57926668)
\curveto(864.54365778,257.55926286)(864.50865781,257.54926287)(864.45866317,257.54926668)
\curveto(864.40865791,257.54926287)(864.36365796,257.53926288)(864.32366317,257.51926668)
\curveto(864.24365808,257.49926292)(864.15365817,257.47926294)(864.05366317,257.45926668)
\lineto(863.75366317,257.39926668)
\curveto(863.7236586,257.39926302)(863.68865863,257.39426302)(863.64866317,257.38426668)
\lineto(863.54366317,257.38426668)
\curveto(863.39365893,257.34426307)(863.22865909,257.3192631)(863.04866317,257.30926668)
\curveto(862.87865944,257.30926311)(862.7186596,257.28926313)(862.56866317,257.24926668)
\curveto(862.48865983,257.22926319)(862.41365991,257.20926321)(862.34366317,257.18926668)
\curveto(862.28366004,257.17926324)(862.21366011,257.16426325)(862.13366317,257.14426668)
\curveto(861.97366035,257.09426332)(861.8236605,257.02926339)(861.68366317,256.94926668)
\curveto(861.54366078,256.87926354)(861.4236609,256.78926363)(861.32366317,256.67926668)
\curveto(861.2236611,256.56926385)(861.14866117,256.43426398)(861.09866317,256.27426668)
\curveto(861.04866127,256.12426429)(861.02866129,255.93926448)(861.03866317,255.71926668)
\curveto(861.03866128,255.6192648)(861.05366127,255.52426489)(861.08366317,255.43426668)
\curveto(861.1236612,255.35426506)(861.16866115,255.27926514)(861.21866317,255.20926668)
\curveto(861.29866102,255.09926532)(861.40366092,255.00426541)(861.53366317,254.92426668)
\curveto(861.66366066,254.85426556)(861.80366052,254.79426562)(861.95366317,254.74426668)
\curveto(862.00366032,254.73426568)(862.05366027,254.72926569)(862.10366317,254.72926668)
\curveto(862.15366017,254.72926569)(862.20366012,254.72426569)(862.25366317,254.71426668)
\curveto(862.32366,254.69426572)(862.40865991,254.67926574)(862.50866317,254.66926668)
\curveto(862.6186597,254.66926575)(862.70865961,254.67926574)(862.77866317,254.69926668)
\curveto(862.83865948,254.7192657)(862.89865942,254.72426569)(862.95866317,254.71426668)
\curveto(863.0186593,254.7142657)(863.07865924,254.72426569)(863.13866317,254.74426668)
\curveto(863.2186591,254.76426565)(863.29365903,254.77926564)(863.36366317,254.78926668)
\curveto(863.44365888,254.79926562)(863.5186588,254.8192656)(863.58866317,254.84926668)
\curveto(863.87865844,254.96926545)(864.1236582,255.1142653)(864.32366317,255.28426668)
\curveto(864.53365779,255.45426496)(864.69365763,255.68426473)(864.80366317,255.97426668)
}
}
{
\newrgbcolor{curcolor}{0 0 0}
\pscustom[linestyle=none,fillstyle=solid,fillcolor=curcolor]
{
\newpath
\moveto(870.5503038,261.65926668)
\curveto(871.27029973,261.66925875)(871.87529913,261.58425883)(872.3653038,261.40426668)
\curveto(872.85529815,261.23425918)(873.23529777,260.92925949)(873.5053038,260.48926668)
\curveto(873.57529743,260.37926004)(873.63029737,260.26426015)(873.6703038,260.14426668)
\curveto(873.71029729,260.03426038)(873.75029725,259.90926051)(873.7903038,259.76926668)
\curveto(873.81029719,259.69926072)(873.81529719,259.62426079)(873.8053038,259.54426668)
\curveto(873.79529721,259.47426094)(873.78029722,259.419261)(873.7603038,259.37926668)
\curveto(873.74029726,259.35926106)(873.71529729,259.33926108)(873.6853038,259.31926668)
\curveto(873.65529735,259.30926111)(873.63029737,259.29426112)(873.6103038,259.27426668)
\curveto(873.56029744,259.25426116)(873.51029749,259.24926117)(873.4603038,259.25926668)
\curveto(873.41029759,259.26926115)(873.36029764,259.26926115)(873.3103038,259.25926668)
\curveto(873.23029777,259.23926118)(873.12529788,259.23426118)(872.9953038,259.24426668)
\curveto(872.86529814,259.26426115)(872.77529823,259.28926113)(872.7253038,259.31926668)
\curveto(872.64529836,259.36926105)(872.59029841,259.43426098)(872.5603038,259.51426668)
\curveto(872.54029846,259.60426081)(872.5052985,259.68926073)(872.4553038,259.76926668)
\curveto(872.36529864,259.92926049)(872.24029876,260.07426034)(872.0803038,260.20426668)
\curveto(871.97029903,260.28426013)(871.85029915,260.34426007)(871.7203038,260.38426668)
\curveto(871.59029941,260.42425999)(871.45029955,260.46425995)(871.3003038,260.50426668)
\curveto(871.25029975,260.52425989)(871.2002998,260.52925989)(871.1503038,260.51926668)
\curveto(871.1002999,260.5192599)(871.05029995,260.52425989)(871.0003038,260.53426668)
\curveto(870.94030006,260.55425986)(870.86530014,260.56425985)(870.7753038,260.56426668)
\curveto(870.68530032,260.56425985)(870.61030039,260.55425986)(870.5503038,260.53426668)
\lineto(870.4603038,260.53426668)
\lineto(870.3103038,260.50426668)
\curveto(870.26030074,260.50425991)(870.21030079,260.49925992)(870.1603038,260.48926668)
\curveto(869.9003011,260.42925999)(869.68530132,260.34426007)(869.5153038,260.23426668)
\curveto(869.34530166,260.12426029)(869.23030177,259.93926048)(869.1703038,259.67926668)
\curveto(869.15030185,259.60926081)(869.14530186,259.53926088)(869.1553038,259.46926668)
\curveto(869.17530183,259.39926102)(869.19530181,259.33926108)(869.2153038,259.28926668)
\curveto(869.27530173,259.13926128)(869.34530166,259.02926139)(869.4253038,258.95926668)
\curveto(869.51530149,258.89926152)(869.62530138,258.82926159)(869.7553038,258.74926668)
\curveto(869.91530109,258.64926177)(870.09530091,258.57426184)(870.2953038,258.52426668)
\curveto(870.49530051,258.48426193)(870.69530031,258.43426198)(870.8953038,258.37426668)
\curveto(871.02529998,258.33426208)(871.15529985,258.30426211)(871.2853038,258.28426668)
\curveto(871.41529959,258.26426215)(871.54529946,258.23426218)(871.6753038,258.19426668)
\curveto(871.88529912,258.13426228)(872.09029891,258.07426234)(872.2903038,258.01426668)
\curveto(872.49029851,257.96426245)(872.69029831,257.89926252)(872.8903038,257.81926668)
\lineto(873.0403038,257.75926668)
\curveto(873.09029791,257.73926268)(873.14029786,257.7142627)(873.1903038,257.68426668)
\curveto(873.39029761,257.56426285)(873.56529744,257.42926299)(873.7153038,257.27926668)
\curveto(873.86529714,257.12926329)(873.99029701,256.93926348)(874.0903038,256.70926668)
\curveto(874.11029689,256.63926378)(874.13029687,256.54426387)(874.1503038,256.42426668)
\curveto(874.17029683,256.35426406)(874.18029682,256.27926414)(874.1803038,256.19926668)
\curveto(874.19029681,256.12926429)(874.19529681,256.04926437)(874.1953038,255.95926668)
\lineto(874.1953038,255.80926668)
\curveto(874.17529683,255.73926468)(874.16529684,255.66926475)(874.1653038,255.59926668)
\curveto(874.16529684,255.52926489)(874.15529685,255.45926496)(874.1353038,255.38926668)
\curveto(874.1052969,255.27926514)(874.07029693,255.17426524)(874.0303038,255.07426668)
\curveto(873.99029701,254.97426544)(873.94529706,254.88426553)(873.8953038,254.80426668)
\curveto(873.73529727,254.54426587)(873.53029747,254.33426608)(873.2803038,254.17426668)
\curveto(873.03029797,254.02426639)(872.75029825,253.89426652)(872.4403038,253.78426668)
\curveto(872.35029865,253.75426666)(872.25529875,253.73426668)(872.1553038,253.72426668)
\curveto(872.06529894,253.70426671)(871.97529903,253.67926674)(871.8853038,253.64926668)
\curveto(871.78529922,253.62926679)(871.68529932,253.6192668)(871.5853038,253.61926668)
\curveto(871.48529952,253.6192668)(871.38529962,253.60926681)(871.2853038,253.58926668)
\lineto(871.1353038,253.58926668)
\curveto(871.08529992,253.57926684)(871.01529999,253.57426684)(870.9253038,253.57426668)
\curveto(870.83530017,253.57426684)(870.76530024,253.57926684)(870.7153038,253.58926668)
\lineto(870.5503038,253.58926668)
\curveto(870.49030051,253.60926681)(870.42530058,253.6192668)(870.3553038,253.61926668)
\curveto(870.28530072,253.60926681)(870.22530078,253.6142668)(870.1753038,253.63426668)
\curveto(870.12530088,253.64426677)(870.06030094,253.64926677)(869.9803038,253.64926668)
\lineto(869.7403038,253.70926668)
\curveto(869.67030133,253.7192667)(869.59530141,253.73926668)(869.5153038,253.76926668)
\curveto(869.2053018,253.86926655)(868.93530207,253.99426642)(868.7053038,254.14426668)
\curveto(868.47530253,254.29426612)(868.27530273,254.48926593)(868.1053038,254.72926668)
\curveto(868.01530299,254.85926556)(867.94030306,254.99426542)(867.8803038,255.13426668)
\curveto(867.82030318,255.27426514)(867.76530324,255.42926499)(867.7153038,255.59926668)
\curveto(867.69530331,255.65926476)(867.68530332,255.72926469)(867.6853038,255.80926668)
\curveto(867.69530331,255.89926452)(867.71030329,255.96926445)(867.7303038,256.01926668)
\curveto(867.76030324,256.05926436)(867.81030319,256.09926432)(867.8803038,256.13926668)
\curveto(867.93030307,256.15926426)(868.000303,256.16926425)(868.0903038,256.16926668)
\curveto(868.18030282,256.17926424)(868.27030273,256.17926424)(868.3603038,256.16926668)
\curveto(868.45030255,256.15926426)(868.53530247,256.14426427)(868.6153038,256.12426668)
\curveto(868.7053023,256.1142643)(868.76530224,256.09926432)(868.7953038,256.07926668)
\curveto(868.86530214,256.02926439)(868.91030209,255.95426446)(868.9303038,255.85426668)
\curveto(868.96030204,255.76426465)(868.99530201,255.67926474)(869.0353038,255.59926668)
\curveto(869.13530187,255.37926504)(869.27030173,255.20926521)(869.4403038,255.08926668)
\curveto(869.56030144,254.99926542)(869.69530131,254.92926549)(869.8453038,254.87926668)
\curveto(869.99530101,254.82926559)(870.15530085,254.77926564)(870.3253038,254.72926668)
\lineto(870.6403038,254.68426668)
\lineto(870.7303038,254.68426668)
\curveto(870.8003002,254.66426575)(870.89030011,254.65426576)(871.0003038,254.65426668)
\curveto(871.12029988,254.65426576)(871.22029978,254.66426575)(871.3003038,254.68426668)
\curveto(871.37029963,254.68426573)(871.42529958,254.68926573)(871.4653038,254.69926668)
\curveto(871.52529948,254.70926571)(871.58529942,254.7142657)(871.6453038,254.71426668)
\curveto(871.7052993,254.72426569)(871.76029924,254.73426568)(871.8103038,254.74426668)
\curveto(872.1002989,254.82426559)(872.33029867,254.92926549)(872.5003038,255.05926668)
\curveto(872.67029833,255.18926523)(872.79029821,255.40926501)(872.8603038,255.71926668)
\curveto(872.88029812,255.76926465)(872.88529812,255.82426459)(872.8753038,255.88426668)
\curveto(872.86529814,255.94426447)(872.85529815,255.98926443)(872.8453038,256.01926668)
\curveto(872.79529821,256.20926421)(872.72529828,256.34926407)(872.6353038,256.43926668)
\curveto(872.54529846,256.53926388)(872.43029857,256.62926379)(872.2903038,256.70926668)
\curveto(872.2002988,256.76926365)(872.1002989,256.8192636)(871.9903038,256.85926668)
\lineto(871.6603038,256.97926668)
\curveto(871.63029937,256.98926343)(871.6002994,256.99426342)(871.5703038,256.99426668)
\curveto(871.55029945,256.99426342)(871.52529948,257.00426341)(871.4953038,257.02426668)
\curveto(871.15529985,257.13426328)(870.8003002,257.2142632)(870.4303038,257.26426668)
\curveto(870.07030093,257.32426309)(869.73030127,257.419263)(869.4103038,257.54926668)
\curveto(869.31030169,257.58926283)(869.21530179,257.62426279)(869.1253038,257.65426668)
\curveto(869.03530197,257.68426273)(868.95030205,257.72426269)(868.8703038,257.77426668)
\curveto(868.68030232,257.88426253)(868.5053025,258.00926241)(868.3453038,258.14926668)
\curveto(868.18530282,258.28926213)(868.06030294,258.46426195)(867.9703038,258.67426668)
\curveto(867.94030306,258.74426167)(867.91530309,258.8142616)(867.8953038,258.88426668)
\curveto(867.88530312,258.95426146)(867.87030313,259.02926139)(867.8503038,259.10926668)
\curveto(867.82030318,259.22926119)(867.81030319,259.36426105)(867.8203038,259.51426668)
\curveto(867.83030317,259.67426074)(867.84530316,259.80926061)(867.8653038,259.91926668)
\curveto(867.88530312,259.96926045)(867.89530311,260.00926041)(867.8953038,260.03926668)
\curveto(867.9053031,260.07926034)(867.92030308,260.1192603)(867.9403038,260.15926668)
\curveto(868.03030297,260.38926003)(868.15030285,260.58925983)(868.3003038,260.75926668)
\curveto(868.46030254,260.92925949)(868.64030236,261.07925934)(868.8403038,261.20926668)
\curveto(868.99030201,261.29925912)(869.15530185,261.36925905)(869.3353038,261.41926668)
\curveto(869.51530149,261.47925894)(869.7053013,261.53425888)(869.9053038,261.58426668)
\curveto(869.97530103,261.59425882)(870.04030096,261.60425881)(870.1003038,261.61426668)
\curveto(870.17030083,261.62425879)(870.24530076,261.63425878)(870.3253038,261.64426668)
\curveto(870.35530065,261.65425876)(870.39530061,261.65425876)(870.4453038,261.64426668)
\curveto(870.49530051,261.63425878)(870.53030047,261.63925878)(870.5503038,261.65926668)
}
}
{
\newrgbcolor{curcolor}{0.60000002 0.60000002 0.60000002}
\pscustom[linestyle=none,fillstyle=solid,fillcolor=curcolor]
{
\newpath
\moveto(798.51865829,264.4643033)
\lineto(813.51865829,264.4643033)
\lineto(813.51865829,249.4643033)
\lineto(798.51865829,249.4643033)
\closepath
}
}
{
\newrgbcolor{curcolor}{0 0 0}
\pscustom[linestyle=none,fillstyle=solid,fillcolor=curcolor]
{
\newpath
\moveto(822.4818663,241.63856111)
\curveto(823.17186006,241.65855016)(823.77685945,241.59355022)(824.2968663,241.44356111)
\curveto(824.81685841,241.30355051)(825.27685795,241.09355072)(825.6768663,240.81356111)
\curveto(825.85685737,240.69355112)(826.02185721,240.55855126)(826.1718663,240.40856111)
\curveto(826.19185704,240.38855143)(826.21185702,240.36855145)(826.2318663,240.34856111)
\curveto(826.25185698,240.32855149)(826.27185696,240.30855151)(826.2918663,240.28856111)
\curveto(826.34185689,240.20855161)(826.39685683,240.13355168)(826.4568663,240.06356111)
\curveto(826.51685671,240.00355181)(826.57185666,239.93355188)(826.6218663,239.85356111)
\curveto(826.7318565,239.68355213)(826.8268564,239.50355231)(826.9068663,239.31356111)
\curveto(826.98685624,239.12355269)(827.06685616,238.92855289)(827.1468663,238.72856111)
\curveto(827.17685605,238.62855319)(827.19185604,238.52355329)(827.1918663,238.41356111)
\curveto(827.19185604,238.30355351)(827.15185608,238.22355359)(827.0718663,238.17356111)
\curveto(827.05185618,238.15355366)(827.02185621,238.14355367)(826.9818663,238.14356111)
\curveto(826.95185628,238.14355367)(826.92185631,238.13855368)(826.8918663,238.12856111)
\lineto(826.7868663,238.12856111)
\curveto(826.73685649,238.10855371)(826.65185658,238.09855372)(826.5318663,238.09856111)
\curveto(826.42185681,238.09855372)(826.34185689,238.10855371)(826.2918663,238.12856111)
\curveto(826.26185697,238.13855368)(826.231857,238.13855368)(826.2018663,238.12856111)
\curveto(826.17185706,238.1185537)(826.13685709,238.12355369)(826.0968663,238.14356111)
\curveto(826.03685719,238.15355366)(825.98185725,238.17855364)(825.9318663,238.21856111)
\curveto(825.87185736,238.26855355)(825.8268574,238.33855348)(825.7968663,238.42856111)
\curveto(825.77685745,238.5185533)(825.74685748,238.60355321)(825.7068663,238.68356111)
\curveto(825.65685757,238.813553)(825.59685763,238.93355288)(825.5268663,239.04356111)
\curveto(825.46685776,239.16355265)(825.39685783,239.27855254)(825.3168663,239.38856111)
\curveto(825.29685793,239.40855241)(825.27185796,239.42855239)(825.2418663,239.44856111)
\curveto(825.21185802,239.47855234)(825.18685804,239.50855231)(825.1668663,239.53856111)
\curveto(825.07685815,239.63855218)(824.98685824,239.72355209)(824.8968663,239.79356111)
\curveto(824.84685838,239.82355199)(824.80185843,239.85355196)(824.7618663,239.88356111)
\curveto(824.72185851,239.92355189)(824.67685855,239.95855186)(824.6268663,239.98856111)
\curveto(824.48685874,240.06855175)(824.33685889,240.13855168)(824.1768663,240.19856111)
\curveto(824.01685921,240.25855156)(823.85185938,240.3135515)(823.6818663,240.36356111)
\curveto(823.59185964,240.38355143)(823.50185973,240.39855142)(823.4118663,240.40856111)
\curveto(823.32185991,240.4185514)(823.23186,240.43355138)(823.1418663,240.45356111)
\curveto(823.10186013,240.46355135)(823.06186017,240.46355135)(823.0218663,240.45356111)
\curveto(822.99186024,240.45355136)(822.96186027,240.45855136)(822.9318663,240.46856111)
\lineto(822.7518663,240.46856111)
\lineto(822.5418663,240.46856111)
\curveto(822.47186076,240.46855135)(822.40686082,240.46355135)(822.3468663,240.45356111)
\curveto(822.3268609,240.45355136)(822.30186093,240.44855137)(822.2718663,240.43856111)
\lineto(822.1968663,240.43856111)
\curveto(822.13686109,240.42855139)(822.07186116,240.4185514)(822.0018663,240.40856111)
\lineto(821.8218663,240.37856111)
\curveto(821.47186176,240.28855153)(821.16686206,240.16355165)(820.9068663,240.00356111)
\curveto(820.48686274,239.74355207)(820.14686308,239.43355238)(819.8868663,239.07356111)
\curveto(819.63686359,238.72355309)(819.4268638,238.29355352)(819.2568663,237.78356111)
\curveto(819.21686401,237.67355414)(819.18686404,237.55855426)(819.1668663,237.43856111)
\curveto(819.14686408,237.3185545)(819.12186411,237.19855462)(819.0918663,237.07856111)
\curveto(819.07186416,237.02855479)(819.06186417,236.97855484)(819.0618663,236.92856111)
\curveto(819.07186416,236.88855493)(819.06686416,236.84355497)(819.0468663,236.79356111)
\curveto(819.0268642,236.72355509)(819.01686421,236.64855517)(819.0168663,236.56856111)
\curveto(819.0268642,236.49855532)(819.02186421,236.42355539)(819.0018663,236.34356111)
\lineto(819.0018663,236.17856111)
\curveto(818.99186424,236.1185557)(818.98686424,236.03355578)(818.9868663,235.92356111)
\curveto(818.98686424,235.813556)(818.99186424,235.73355608)(819.0018663,235.68356111)
\lineto(819.0018663,235.53356111)
\curveto(819.01186422,235.5135563)(819.01686421,235.48355633)(819.0168663,235.44356111)
\curveto(819.01686421,235.4135564)(819.02186421,235.38855643)(819.0318663,235.36856111)
\curveto(819.05186418,235.29855652)(819.05686417,235.23355658)(819.0468663,235.17356111)
\curveto(819.03686419,235.1135567)(819.04186419,235.04855677)(819.0618663,234.97856111)
\curveto(819.08186415,234.89855692)(819.09686413,234.818557)(819.1068663,234.73856111)
\curveto(819.1268641,234.66855715)(819.14686408,234.59355722)(819.1668663,234.51356111)
\lineto(819.3168663,234.06356111)
\lineto(819.4968663,233.64356111)
\curveto(819.54686368,233.52355829)(819.60686362,233.40855841)(819.6768663,233.29856111)
\curveto(819.75686347,233.18855863)(819.83686339,233.08355873)(819.9168663,232.98356111)
\curveto(820.28686294,232.48355933)(820.77186246,232.1135597)(821.3718663,231.87356111)
\curveto(821.44186179,231.84355997)(821.51186172,231.81856)(821.5818663,231.79856111)
\curveto(821.65186158,231.77856004)(821.7268615,231.75856006)(821.8068663,231.73856111)
\curveto(822.04686118,231.66856015)(822.3268609,231.63356018)(822.6468663,231.63356111)
\lineto(822.8418663,231.63356111)
\curveto(822.91186032,231.63356018)(822.97686025,231.63856018)(823.0368663,231.64856111)
\curveto(823.08686014,231.66856015)(823.13686009,231.67356014)(823.1868663,231.66356111)
\curveto(823.24685998,231.65356016)(823.30185993,231.65856016)(823.3518663,231.67856111)
\curveto(823.49185974,231.7185601)(823.62185961,231.74856007)(823.7418663,231.76856111)
\curveto(823.87185936,231.79856002)(823.99185924,231.83855998)(824.1018663,231.88856111)
\curveto(825.0318582,232.25855956)(825.65685757,232.92355889)(825.9768663,233.88356111)
\curveto(825.99685723,233.96355785)(826.01185722,234.04355777)(826.0218663,234.12356111)
\lineto(826.0818663,234.36356111)
\curveto(826.11185712,234.48355733)(826.12185711,234.6135572)(826.1118663,234.75356111)
\curveto(826.10185713,234.90355691)(826.05685717,235.00855681)(825.9768663,235.06856111)
\curveto(825.89685733,235.1185567)(825.78685744,235.14355667)(825.6468663,235.14356111)
\lineto(825.2418663,235.14356111)
\lineto(823.5768663,235.14356111)
\lineto(823.2168663,235.14356111)
\curveto(823.08686014,235.14355667)(822.98186025,235.15855666)(822.9018663,235.18856111)
\curveto(822.82186041,235.22855659)(822.77186046,235.28355653)(822.7518663,235.35356111)
\curveto(822.7318605,235.39355642)(822.71686051,235.44855637)(822.7068663,235.51856111)
\curveto(822.69686053,235.59855622)(822.69186054,235.67855614)(822.6918663,235.75856111)
\curveto(822.69186054,235.83855598)(822.69686053,235.9135559)(822.7068663,235.98356111)
\curveto(822.7268605,236.06355575)(822.74686048,236.1185557)(822.7668663,236.14856111)
\curveto(822.80686042,236.2185556)(822.87686035,236.26855555)(822.9768663,236.29856111)
\curveto(823.0268602,236.3185555)(823.08686014,236.32855549)(823.1568663,236.32856111)
\lineto(823.3668663,236.32856111)
\lineto(824.0418663,236.32856111)
\lineto(826.3218663,236.32856111)
\lineto(826.6518663,236.32856111)
\curveto(826.76185647,236.33855548)(826.86185637,236.33355548)(826.9518663,236.31356111)
\curveto(827.05185618,236.30355551)(827.13685609,236.27855554)(827.2068663,236.23856111)
\curveto(827.27685595,236.20855561)(827.3268559,236.14855567)(827.3568663,236.05856111)
\curveto(827.37685585,235.99855582)(827.38185585,235.92855589)(827.3718663,235.84856111)
\curveto(827.37185586,235.76855605)(827.37185586,235.68855613)(827.3718663,235.60856111)
\lineto(827.3718663,234.78356111)
\lineto(827.3718663,232.00856111)
\lineto(827.3718663,231.31856111)
\curveto(827.37185586,231.24856057)(827.37185586,231.17856064)(827.3718663,231.10856111)
\curveto(827.37185586,231.03856078)(827.36185587,230.97856084)(827.3418663,230.92856111)
\curveto(827.31185592,230.84856097)(827.25185598,230.78856103)(827.1618663,230.74856111)
\curveto(827.1318561,230.72856109)(827.06685616,230.7185611)(826.9668663,230.71856111)
\curveto(826.8268564,230.7185611)(826.72185651,230.73356108)(826.6518663,230.76356111)
\curveto(826.58185665,230.79356102)(826.5268567,230.83856098)(826.4868663,230.89856111)
\curveto(826.44685678,230.95856086)(826.41185682,231.02856079)(826.3818663,231.10856111)
\curveto(826.36185687,231.18856063)(826.33685689,231.27856054)(826.3068663,231.37856111)
\curveto(826.28685694,231.44856037)(826.25185698,231.53856028)(826.2018663,231.64856111)
\curveto(826.16185707,231.75856006)(826.08685714,231.79856002)(825.9768663,231.76856111)
\curveto(825.90685732,231.74856007)(825.85185738,231.7185601)(825.8118663,231.67856111)
\curveto(825.77185746,231.63856018)(825.7268575,231.59856022)(825.6768663,231.55856111)
\curveto(825.59685763,231.49856032)(825.52185771,231.43356038)(825.4518663,231.36356111)
\curveto(825.38185785,231.30356051)(825.30685792,231.24856057)(825.2268663,231.19856111)
\curveto(825.01685821,231.05856076)(824.79185844,230.94356087)(824.5518663,230.85356111)
\curveto(824.32185891,230.76356105)(824.07185916,230.67856114)(823.8018663,230.59856111)
\curveto(823.7318595,230.57856124)(823.66185957,230.56356125)(823.5918663,230.55356111)
\curveto(823.52185971,230.54356127)(823.44685978,230.52856129)(823.3668663,230.50856111)
\curveto(823.28685994,230.50856131)(823.22686,230.50356131)(823.1868663,230.49356111)
\lineto(823.0818663,230.49356111)
\curveto(823.05186018,230.49356132)(823.02186021,230.48856133)(822.9918663,230.47856111)
\lineto(822.8418663,230.47856111)
\curveto(822.80186043,230.46856135)(822.74686048,230.46356135)(822.6768663,230.46356111)
\curveto(822.60686062,230.46356135)(822.54686068,230.46856135)(822.4968663,230.47856111)
\lineto(822.2418663,230.47856111)
\curveto(822.1318611,230.49856132)(822.0268612,230.5135613)(821.9268663,230.52356111)
\curveto(821.83686139,230.52356129)(821.74186149,230.53856128)(821.6418663,230.56856111)
\curveto(821.46186177,230.6185612)(821.28686194,230.66356115)(821.1168663,230.70356111)
\curveto(820.94686228,230.74356107)(820.78186245,230.79856102)(820.6218663,230.86856111)
\curveto(820.05186318,231.12856069)(819.55186368,231.47356034)(819.1218663,231.90356111)
\curveto(818.69186454,232.34355947)(818.34686488,232.85355896)(818.0868663,233.43356111)
\curveto(818.03686519,233.55355826)(817.98686524,233.67855814)(817.9368663,233.80856111)
\curveto(817.89686533,233.93855788)(817.85186538,234.07355774)(817.8018663,234.21356111)
\curveto(817.79186544,234.25355756)(817.78686544,234.28855753)(817.7868663,234.31856111)
\curveto(817.78686544,234.35855746)(817.77686545,234.39855742)(817.7568663,234.43856111)
\curveto(817.7268655,234.54855727)(817.70186553,234.66355715)(817.6818663,234.78356111)
\curveto(817.67186556,234.90355691)(817.65186558,235.02355679)(817.6218663,235.14356111)
\curveto(817.61186562,235.18355663)(817.60686562,235.22355659)(817.6068663,235.26356111)
\curveto(817.61686561,235.30355651)(817.61686561,235.33855648)(817.6068663,235.36856111)
\lineto(817.6068663,235.50356111)
\curveto(817.58686564,235.55355626)(817.57686565,235.64855617)(817.5768663,235.78856111)
\curveto(817.57686565,235.92855589)(817.58686564,236.02855579)(817.6068663,236.08856111)
\lineto(817.6068663,236.22356111)
\lineto(817.6068663,236.40356111)
\curveto(817.60686562,236.46355535)(817.61186562,236.52355529)(817.6218663,236.58356111)
\curveto(817.6318656,236.63355518)(817.63686559,236.68355513)(817.6368663,236.73356111)
\curveto(817.63686559,236.79355502)(817.64186559,236.84855497)(817.6518663,236.89856111)
\curveto(817.68186555,237.0185548)(817.70186553,237.13855468)(817.7118663,237.25856111)
\curveto(817.7318655,237.37855444)(817.75686547,237.49855432)(817.7868663,237.61856111)
\curveto(817.88686534,237.94855387)(817.98686524,238.26355355)(818.0868663,238.56356111)
\curveto(818.19686503,238.87355294)(818.33686489,239.15855266)(818.5068663,239.41856111)
\curveto(818.80686442,239.88855193)(819.16186407,240.28855153)(819.5718663,240.61856111)
\curveto(819.99186324,240.95855086)(820.48686274,241.22355059)(821.0568663,241.41356111)
\curveto(821.16686206,241.45355036)(821.27186196,241.47855034)(821.3718663,241.48856111)
\curveto(821.48186175,241.50855031)(821.59186164,241.53355028)(821.7018663,241.56356111)
\curveto(821.75186148,241.58355023)(821.79686143,241.59355022)(821.8368663,241.59356111)
\curveto(821.88686134,241.59355022)(821.93686129,241.59855022)(821.9868663,241.60856111)
\curveto(822.06686116,241.6185502)(822.14686108,241.62355019)(822.2268663,241.62356111)
\curveto(822.31686091,241.63355018)(822.40186083,241.63855018)(822.4818663,241.63856111)
}
}
{
\newrgbcolor{curcolor}{0 0 0}
\pscustom[linestyle=none,fillstyle=solid,fillcolor=curcolor]
{
\newpath
\moveto(832.88835067,238.62356111)
\curveto(833.11834588,238.62355319)(833.24834575,238.56355325)(833.27835067,238.44356111)
\curveto(833.30834569,238.33355348)(833.32334568,238.16855365)(833.32335067,237.94856111)
\lineto(833.32335067,237.66356111)
\curveto(833.32334568,237.57355424)(833.2983457,237.49855432)(833.24835067,237.43856111)
\curveto(833.18834581,237.35855446)(833.1033459,237.3135545)(832.99335067,237.30356111)
\curveto(832.88334612,237.30355451)(832.77334623,237.28855453)(832.66335067,237.25856111)
\curveto(832.52334648,237.22855459)(832.38834661,237.19855462)(832.25835067,237.16856111)
\curveto(832.13834686,237.13855468)(832.02334698,237.09855472)(831.91335067,237.04856111)
\curveto(831.62334738,236.9185549)(831.38834761,236.73855508)(831.20835067,236.50856111)
\curveto(831.02834797,236.28855553)(830.87334813,236.03355578)(830.74335067,235.74356111)
\curveto(830.7033483,235.63355618)(830.67334833,235.5185563)(830.65335067,235.39856111)
\curveto(830.63334837,235.28855653)(830.60834839,235.17355664)(830.57835067,235.05356111)
\curveto(830.56834843,235.00355681)(830.56334844,234.95355686)(830.56335067,234.90356111)
\curveto(830.57334843,234.85355696)(830.57334843,234.80355701)(830.56335067,234.75356111)
\curveto(830.53334847,234.63355718)(830.51834848,234.49355732)(830.51835067,234.33356111)
\curveto(830.52834847,234.18355763)(830.53334847,234.03855778)(830.53335067,233.89856111)
\lineto(830.53335067,232.05356111)
\lineto(830.53335067,231.70856111)
\curveto(830.53334847,231.58856023)(830.52834847,231.47356034)(830.51835067,231.36356111)
\curveto(830.50834849,231.25356056)(830.5033485,231.15856066)(830.50335067,231.07856111)
\curveto(830.51334849,230.99856082)(830.49334851,230.92856089)(830.44335067,230.86856111)
\curveto(830.39334861,230.79856102)(830.31334869,230.75856106)(830.20335067,230.74856111)
\curveto(830.1033489,230.73856108)(829.99334901,230.73356108)(829.87335067,230.73356111)
\lineto(829.60335067,230.73356111)
\curveto(829.55334945,230.75356106)(829.5033495,230.76856105)(829.45335067,230.77856111)
\curveto(829.41334959,230.79856102)(829.38334962,230.82356099)(829.36335067,230.85356111)
\curveto(829.31334969,230.92356089)(829.28334972,231.00856081)(829.27335067,231.10856111)
\lineto(829.27335067,231.43856111)
\lineto(829.27335067,232.59356111)
\lineto(829.27335067,236.74856111)
\lineto(829.27335067,237.78356111)
\lineto(829.27335067,238.08356111)
\curveto(829.28334972,238.18355363)(829.31334969,238.26855355)(829.36335067,238.33856111)
\curveto(829.39334961,238.37855344)(829.44334956,238.40855341)(829.51335067,238.42856111)
\curveto(829.59334941,238.44855337)(829.67834932,238.45855336)(829.76835067,238.45856111)
\curveto(829.85834914,238.46855335)(829.94834905,238.46855335)(830.03835067,238.45856111)
\curveto(830.12834887,238.44855337)(830.1983488,238.43355338)(830.24835067,238.41356111)
\curveto(830.32834867,238.38355343)(830.37834862,238.32355349)(830.39835067,238.23356111)
\curveto(830.42834857,238.15355366)(830.44334856,238.06355375)(830.44335067,237.96356111)
\lineto(830.44335067,237.66356111)
\curveto(830.44334856,237.56355425)(830.46334854,237.47355434)(830.50335067,237.39356111)
\curveto(830.51334849,237.37355444)(830.52334848,237.35855446)(830.53335067,237.34856111)
\lineto(830.57835067,237.30356111)
\curveto(830.68834831,237.30355451)(830.77834822,237.34855447)(830.84835067,237.43856111)
\curveto(830.91834808,237.53855428)(830.97834802,237.6185542)(831.02835067,237.67856111)
\lineto(831.11835067,237.76856111)
\curveto(831.20834779,237.87855394)(831.33334767,237.99355382)(831.49335067,238.11356111)
\curveto(831.65334735,238.23355358)(831.8033472,238.32355349)(831.94335067,238.38356111)
\curveto(832.03334697,238.43355338)(832.12834687,238.46855335)(832.22835067,238.48856111)
\curveto(832.32834667,238.5185533)(832.43334657,238.54855327)(832.54335067,238.57856111)
\curveto(832.6033464,238.58855323)(832.66334634,238.59355322)(832.72335067,238.59356111)
\curveto(832.78334622,238.60355321)(832.83834616,238.6135532)(832.88835067,238.62356111)
}
}
{
\newrgbcolor{curcolor}{0 0 0}
\pscustom[linestyle=none,fillstyle=solid,fillcolor=curcolor]
{
\newpath
\moveto(834.7181163,238.44356111)
\lineto(835.1531163,238.44356111)
\curveto(835.30311433,238.44355337)(835.40811423,238.40355341)(835.4681163,238.32356111)
\curveto(835.51811412,238.24355357)(835.54311409,238.14355367)(835.5431163,238.02356111)
\curveto(835.55311408,237.90355391)(835.55811408,237.78355403)(835.5581163,237.66356111)
\lineto(835.5581163,236.23856111)
\lineto(835.5581163,233.97356111)
\lineto(835.5581163,233.28356111)
\curveto(835.55811408,233.05355876)(835.58311405,232.85355896)(835.6331163,232.68356111)
\curveto(835.79311384,232.23355958)(836.09311354,231.9185599)(836.5331163,231.73856111)
\curveto(836.75311288,231.64856017)(837.01811262,231.6135602)(837.3281163,231.63356111)
\curveto(837.638112,231.66356015)(837.88811175,231.7185601)(838.0781163,231.79856111)
\curveto(838.40811123,231.93855988)(838.66811097,232.1135597)(838.8581163,232.32356111)
\curveto(839.05811058,232.54355927)(839.21311042,232.82855899)(839.3231163,233.17856111)
\curveto(839.35311028,233.25855856)(839.37311026,233.33855848)(839.3831163,233.41856111)
\curveto(839.39311024,233.49855832)(839.40811023,233.58355823)(839.4281163,233.67356111)
\curveto(839.4381102,233.72355809)(839.4381102,233.76855805)(839.4281163,233.80856111)
\curveto(839.42811021,233.84855797)(839.4381102,233.89355792)(839.4581163,233.94356111)
\lineto(839.4581163,234.25856111)
\curveto(839.47811016,234.33855748)(839.48311015,234.42855739)(839.4731163,234.52856111)
\curveto(839.46311017,234.63855718)(839.45811018,234.73855708)(839.4581163,234.82856111)
\lineto(839.4581163,235.99856111)
\lineto(839.4581163,237.58856111)
\curveto(839.45811018,237.70855411)(839.45311018,237.83355398)(839.4431163,237.96356111)
\curveto(839.44311019,238.10355371)(839.46811017,238.2135536)(839.5181163,238.29356111)
\curveto(839.55811008,238.34355347)(839.60311003,238.37355344)(839.6531163,238.38356111)
\curveto(839.71310992,238.40355341)(839.78310985,238.42355339)(839.8631163,238.44356111)
\lineto(840.0881163,238.44356111)
\curveto(840.20810943,238.44355337)(840.31310932,238.43855338)(840.4031163,238.42856111)
\curveto(840.50310913,238.4185534)(840.57810906,238.37355344)(840.6281163,238.29356111)
\curveto(840.67810896,238.24355357)(840.70310893,238.16855365)(840.7031163,238.06856111)
\lineto(840.7031163,237.78356111)
\lineto(840.7031163,236.76356111)
\lineto(840.7031163,232.72856111)
\lineto(840.7031163,231.37856111)
\curveto(840.70310893,231.25856056)(840.69810894,231.14356067)(840.6881163,231.03356111)
\curveto(840.68810895,230.93356088)(840.65310898,230.85856096)(840.5831163,230.80856111)
\curveto(840.54310909,230.77856104)(840.48310915,230.75356106)(840.4031163,230.73356111)
\curveto(840.32310931,230.72356109)(840.2331094,230.7135611)(840.1331163,230.70356111)
\curveto(840.04310959,230.70356111)(839.95310968,230.70856111)(839.8631163,230.71856111)
\curveto(839.78310985,230.72856109)(839.72310991,230.74856107)(839.6831163,230.77856111)
\curveto(839.63311,230.818561)(839.58811005,230.88356093)(839.5481163,230.97356111)
\curveto(839.5381101,231.0135608)(839.52811011,231.06856075)(839.5181163,231.13856111)
\curveto(839.51811012,231.20856061)(839.51311012,231.27356054)(839.5031163,231.33356111)
\curveto(839.49311014,231.40356041)(839.47311016,231.45856036)(839.4431163,231.49856111)
\curveto(839.41311022,231.53856028)(839.36811027,231.55356026)(839.3081163,231.54356111)
\curveto(839.22811041,231.52356029)(839.14811049,231.46356035)(839.0681163,231.36356111)
\curveto(838.98811065,231.27356054)(838.91311072,231.20356061)(838.8431163,231.15356111)
\curveto(838.62311101,230.99356082)(838.37311126,230.85356096)(838.0931163,230.73356111)
\curveto(837.98311165,230.68356113)(837.86811177,230.65356116)(837.7481163,230.64356111)
\curveto(837.638112,230.62356119)(837.52311211,230.59856122)(837.4031163,230.56856111)
\curveto(837.35311228,230.55856126)(837.29811234,230.55856126)(837.2381163,230.56856111)
\curveto(837.18811245,230.57856124)(837.1381125,230.57356124)(837.0881163,230.55356111)
\curveto(836.98811265,230.53356128)(836.89811274,230.53356128)(836.8181163,230.55356111)
\lineto(836.6681163,230.55356111)
\curveto(836.61811302,230.57356124)(836.55811308,230.58356123)(836.4881163,230.58356111)
\curveto(836.42811321,230.58356123)(836.37311326,230.58856123)(836.3231163,230.59856111)
\curveto(836.28311335,230.6185612)(836.24311339,230.62856119)(836.2031163,230.62856111)
\curveto(836.17311346,230.6185612)(836.1331135,230.62356119)(836.0831163,230.64356111)
\lineto(835.8431163,230.70356111)
\curveto(835.77311386,230.72356109)(835.69811394,230.75356106)(835.6181163,230.79356111)
\curveto(835.35811428,230.90356091)(835.1381145,231.04856077)(834.9581163,231.22856111)
\curveto(834.78811485,231.4185604)(834.64811499,231.64356017)(834.5381163,231.90356111)
\curveto(834.49811514,231.99355982)(834.46811517,232.08355973)(834.4481163,232.17356111)
\lineto(834.3881163,232.47356111)
\curveto(834.36811527,232.53355928)(834.35811528,232.58855923)(834.3581163,232.63856111)
\curveto(834.36811527,232.69855912)(834.36311527,232.76355905)(834.3431163,232.83356111)
\curveto(834.3331153,232.85355896)(834.32811531,232.87855894)(834.3281163,232.90856111)
\curveto(834.32811531,232.94855887)(834.32311531,232.98355883)(834.3131163,233.01356111)
\lineto(834.3131163,233.16356111)
\curveto(834.30311533,233.20355861)(834.29811534,233.24855857)(834.2981163,233.29856111)
\curveto(834.30811533,233.35855846)(834.31311532,233.4135584)(834.3131163,233.46356111)
\lineto(834.3131163,234.06356111)
\lineto(834.3131163,236.82356111)
\lineto(834.3131163,237.78356111)
\lineto(834.3131163,238.05356111)
\curveto(834.31311532,238.14355367)(834.3331153,238.2185536)(834.3731163,238.27856111)
\curveto(834.41311522,238.34855347)(834.48811515,238.39855342)(834.5981163,238.42856111)
\curveto(834.61811502,238.43855338)(834.638115,238.43855338)(834.6581163,238.42856111)
\curveto(834.67811496,238.42855339)(834.69811494,238.43355338)(834.7181163,238.44356111)
}
}
{
\newrgbcolor{curcolor}{0 0 0}
\pscustom[linestyle=none,fillstyle=solid,fillcolor=curcolor]
{
\newpath
\moveto(850.02772567,234.78356111)
\curveto(850.03771732,234.73355708)(850.04271732,234.66855715)(850.04272567,234.58856111)
\curveto(850.04271732,234.50855731)(850.03771732,234.44355737)(850.02772567,234.39356111)
\curveto(850.00771735,234.34355747)(850.00271736,234.29355752)(850.01272567,234.24356111)
\curveto(850.02271734,234.20355761)(850.02271734,234.16355765)(850.01272567,234.12356111)
\curveto(850.01271735,234.05355776)(850.00771735,233.99855782)(849.99772567,233.95856111)
\curveto(849.97771738,233.86855795)(849.9627174,233.77855804)(849.95272567,233.68856111)
\curveto(849.95271741,233.59855822)(849.94271742,233.50855831)(849.92272567,233.41856111)
\lineto(849.86272567,233.17856111)
\curveto(849.84271752,233.10855871)(849.81771754,233.03355878)(849.78772567,232.95356111)
\curveto(849.66771769,232.58355923)(849.50271786,232.24855957)(849.29272567,231.94856111)
\curveto(849.23271813,231.85855996)(849.16771819,231.76856005)(849.09772567,231.67856111)
\curveto(849.02771833,231.59856022)(848.95271841,231.52356029)(848.87272567,231.45356111)
\lineto(848.79772567,231.37856111)
\curveto(848.72771863,231.32856049)(848.6627187,231.27856054)(848.60272567,231.22856111)
\curveto(848.54271882,231.17856064)(848.47271889,231.12856069)(848.39272567,231.07856111)
\curveto(848.28271908,230.99856082)(848.1577192,230.92856089)(848.01772567,230.86856111)
\curveto(847.88771947,230.818561)(847.75271961,230.76856105)(847.61272567,230.71856111)
\curveto(847.53271983,230.69856112)(847.45271991,230.68356113)(847.37272567,230.67356111)
\curveto(847.30272006,230.66356115)(847.22772013,230.64856117)(847.14772567,230.62856111)
\lineto(847.08772567,230.62856111)
\curveto(847.07772028,230.6185612)(847.0627203,230.6135612)(847.04272567,230.61356111)
\curveto(846.95272041,230.59356122)(846.81772054,230.58356123)(846.63772567,230.58356111)
\curveto(846.46772089,230.57356124)(846.33272103,230.57856124)(846.23272567,230.59856111)
\lineto(846.15772567,230.59856111)
\curveto(846.08772127,230.60856121)(846.02272134,230.6185612)(845.96272567,230.62856111)
\curveto(845.90272146,230.62856119)(845.84272152,230.63856118)(845.78272567,230.65856111)
\curveto(845.61272175,230.70856111)(845.45272191,230.75356106)(845.30272567,230.79356111)
\curveto(845.15272221,230.83356098)(845.01272235,230.89356092)(844.88272567,230.97356111)
\curveto(844.72272264,231.06356075)(844.58272278,231.15856066)(844.46272567,231.25856111)
\curveto(844.42272294,231.28856053)(844.362723,231.32856049)(844.28272567,231.37856111)
\curveto(844.20272316,231.43856038)(844.12772323,231.44356037)(844.05772567,231.39356111)
\curveto(844.01772334,231.36356045)(843.99772336,231.32356049)(843.99772567,231.27356111)
\curveto(843.99772336,231.22356059)(843.98772337,231.16856065)(843.96772567,231.10856111)
\curveto(843.9577234,231.07856074)(843.9577234,231.04356077)(843.96772567,231.00356111)
\curveto(843.97772338,230.97356084)(843.97772338,230.93856088)(843.96772567,230.89856111)
\curveto(843.94772341,230.83856098)(843.93772342,230.77356104)(843.93772567,230.70356111)
\curveto(843.94772341,230.62356119)(843.95272341,230.55356126)(843.95272567,230.49356111)
\lineto(843.95272567,228.69356111)
\lineto(843.95272567,228.25856111)
\curveto(843.95272341,228.10856371)(843.92272344,227.99356382)(843.86272567,227.91356111)
\curveto(843.81272355,227.84356397)(843.73272363,227.80856401)(843.62272567,227.80856111)
\curveto(843.51272385,227.79856402)(843.40272396,227.79356402)(843.29272567,227.79356111)
\lineto(843.05272567,227.79356111)
\curveto(842.98272438,227.813564)(842.92272444,227.83356398)(842.87272567,227.85356111)
\curveto(842.83272453,227.87356394)(842.79772456,227.90856391)(842.76772567,227.95856111)
\curveto(842.71772464,228.02856379)(842.69272467,228.13856368)(842.69272567,228.28856111)
\curveto(842.70272466,228.43856338)(842.70772465,228.56856325)(842.70772567,228.67856111)
\lineto(842.70772567,237.67856111)
\lineto(842.70772567,238.03856111)
\curveto(842.71772464,238.16855365)(842.74772461,238.27355354)(842.79772567,238.35356111)
\curveto(842.82772453,238.39355342)(842.89272447,238.42355339)(842.99272567,238.44356111)
\curveto(843.10272426,238.47355334)(843.21772414,238.48355333)(843.33772567,238.47356111)
\curveto(843.4577239,238.47355334)(843.56772379,238.45855336)(843.66772567,238.42856111)
\curveto(843.77772358,238.40855341)(843.84772351,238.37855344)(843.87772567,238.33856111)
\curveto(843.91772344,238.28855353)(843.93772342,238.22855359)(843.93772567,238.15856111)
\curveto(843.94772341,238.08855373)(843.96772339,238.0185538)(843.99772567,237.94856111)
\curveto(844.01772334,237.9185539)(844.03272333,237.89355392)(844.04272567,237.87356111)
\curveto(844.0627233,237.86355395)(844.08272328,237.84855397)(844.10272567,237.82856111)
\curveto(844.21272315,237.818554)(844.30272306,237.85355396)(844.37272567,237.93356111)
\curveto(844.45272291,238.0135538)(844.52772283,238.07855374)(844.59772567,238.12856111)
\curveto(844.8577225,238.30855351)(845.16772219,238.44855337)(845.52772567,238.54856111)
\curveto(845.61772174,238.56855325)(845.70772165,238.58355323)(845.79772567,238.59356111)
\curveto(845.89772146,238.60355321)(845.99772136,238.6185532)(846.09772567,238.63856111)
\curveto(846.13772122,238.64855317)(846.18772117,238.64855317)(846.24772567,238.63856111)
\curveto(846.30772105,238.62855319)(846.34772101,238.63355318)(846.36772567,238.65356111)
\curveto(846.79772056,238.66355315)(847.17772018,238.6185532)(847.50772567,238.51856111)
\curveto(847.83771952,238.42855339)(848.13271923,238.29855352)(848.39272567,238.12856111)
\lineto(848.54272567,238.00856111)
\curveto(848.59271877,237.97855384)(848.64271872,237.94355387)(848.69272567,237.90356111)
\curveto(848.71271865,237.88355393)(848.72771863,237.86355395)(848.73772567,237.84356111)
\curveto(848.7577186,237.83355398)(848.77771858,237.818554)(848.79772567,237.79856111)
\curveto(848.84771851,237.74855407)(848.90271846,237.69355412)(848.96272567,237.63356111)
\curveto(849.02271834,237.57355424)(849.07771828,237.5135543)(849.12772567,237.45356111)
\curveto(849.24771811,237.28355453)(849.37271799,237.09855472)(849.50272567,236.89856111)
\curveto(849.58271778,236.76855505)(849.64771771,236.62355519)(849.69772567,236.46356111)
\curveto(849.7577176,236.30355551)(849.81271755,236.14355567)(849.86272567,235.98356111)
\curveto(849.88271748,235.90355591)(849.89771746,235.818556)(849.90772567,235.72856111)
\curveto(849.92771743,235.63855618)(849.94771741,235.55355626)(849.96772567,235.47356111)
\lineto(849.96772567,235.35356111)
\curveto(849.97771738,235.32355649)(849.98271738,235.29355652)(849.98272567,235.26356111)
\curveto(850.00271736,235.2135566)(850.00771735,235.15855666)(849.99772567,235.09856111)
\curveto(849.99771736,235.03855678)(850.00771735,234.98355683)(850.02772567,234.93356111)
\lineto(850.02772567,234.78356111)
\moveto(848.69272567,234.37856111)
\curveto(848.71271865,234.42855739)(848.71771864,234.48855733)(848.70772567,234.55856111)
\curveto(848.69771866,234.63855718)(848.69271867,234.70855711)(848.69272567,234.76856111)
\curveto(848.69271867,234.93855688)(848.68271868,235.09855672)(848.66272567,235.24856111)
\curveto(848.65271871,235.39855642)(848.62271874,235.54355627)(848.57272567,235.68356111)
\lineto(848.51272567,235.86356111)
\curveto(848.50271886,235.93355588)(848.48271888,235.99855582)(848.45272567,236.05856111)
\curveto(848.34271902,236.32855549)(848.16771919,236.58855523)(847.92772567,236.83856111)
\curveto(847.69771966,237.08855473)(847.47771988,237.25855456)(847.26772567,237.34856111)
\curveto(847.18772017,237.38855443)(847.10272026,237.4185544)(847.01272567,237.43856111)
\curveto(846.93272043,237.45855436)(846.84772051,237.48355433)(846.75772567,237.51356111)
\curveto(846.66772069,237.53355428)(846.5627208,237.54355427)(846.44272567,237.54356111)
\lineto(846.11272567,237.54356111)
\curveto(846.09272127,237.52355429)(846.05272131,237.5135543)(845.99272567,237.51356111)
\curveto(845.94272142,237.52355429)(845.89772146,237.52355429)(845.85772567,237.51356111)
\lineto(845.58772567,237.45356111)
\curveto(845.50772185,237.43355438)(845.42772193,237.40355441)(845.34772567,237.36356111)
\curveto(845.02772233,237.22355459)(844.7627226,237.0185548)(844.55272567,236.74856111)
\curveto(844.35272301,236.48855533)(844.19772316,236.18355563)(844.08772567,235.83356111)
\curveto(844.04772331,235.72355609)(844.01772334,235.6135562)(843.99772567,235.50356111)
\curveto(843.98772337,235.39355642)(843.97272339,235.28355653)(843.95272567,235.17356111)
\curveto(843.94272342,235.13355668)(843.93772342,235.09355672)(843.93772567,235.05356111)
\curveto(843.93772342,235.02355679)(843.93272343,234.98855683)(843.92272567,234.94856111)
\lineto(843.92272567,234.82856111)
\curveto(843.91272345,234.77855704)(843.90772345,234.70355711)(843.90772567,234.60356111)
\curveto(843.90772345,234.5135573)(843.91272345,234.44355737)(843.92272567,234.39356111)
\lineto(843.92272567,234.27356111)
\curveto(843.93272343,234.23355758)(843.93772342,234.19355762)(843.93772567,234.15356111)
\curveto(843.93772342,234.1135577)(843.94272342,234.07855774)(843.95272567,234.04856111)
\curveto(843.9627234,234.0185578)(843.96772339,233.98855783)(843.96772567,233.95856111)
\curveto(843.96772339,233.92855789)(843.97272339,233.89355792)(843.98272567,233.85356111)
\curveto(844.00272336,233.77355804)(844.01772334,233.69355812)(844.02772567,233.61356111)
\lineto(844.08772567,233.37356111)
\curveto(844.19772316,233.03355878)(844.34772301,232.73355908)(844.53772567,232.47356111)
\curveto(844.73772262,232.22355959)(844.99772236,232.02855979)(845.31772567,231.88856111)
\curveto(845.50772185,231.80856001)(845.70272166,231.74856007)(845.90272567,231.70856111)
\curveto(845.94272142,231.68856013)(845.98272138,231.67856014)(846.02272567,231.67856111)
\curveto(846.0627213,231.68856013)(846.10272126,231.68856013)(846.14272567,231.67856111)
\lineto(846.26272567,231.67856111)
\curveto(846.33272103,231.65856016)(846.40272096,231.65856016)(846.47272567,231.67856111)
\lineto(846.59272567,231.67856111)
\curveto(846.70272066,231.69856012)(846.80772055,231.7135601)(846.90772567,231.72356111)
\curveto(847.00772035,231.73356008)(847.10772025,231.75856006)(847.20772567,231.79856111)
\curveto(847.51771984,231.92855989)(847.76771959,232.09855972)(847.95772567,232.30856111)
\curveto(848.1577192,232.52855929)(848.32271904,232.79355902)(848.45272567,233.10356111)
\curveto(848.50271886,233.24355857)(848.53771882,233.38355843)(848.55772567,233.52356111)
\curveto(848.58771877,233.67355814)(848.62271874,233.82855799)(848.66272567,233.98856111)
\curveto(848.67271869,234.03855778)(848.67771868,234.08355773)(848.67772567,234.12356111)
\curveto(848.67771868,234.16355765)(848.68271868,234.20855761)(848.69272567,234.25856111)
\lineto(848.69272567,234.37856111)
}
}
{
\newrgbcolor{curcolor}{0 0 0}
\pscustom[linestyle=none,fillstyle=solid,fillcolor=curcolor]
{
\newpath
\moveto(858.63397567,234.91856111)
\curveto(858.65396761,234.85855696)(858.6639676,234.76355705)(858.66397567,234.63356111)
\curveto(858.6639676,234.5135573)(858.65896761,234.42855739)(858.64897567,234.37856111)
\lineto(858.64897567,234.22856111)
\curveto(858.63896763,234.14855767)(858.62896764,234.07355774)(858.61897567,234.00356111)
\curveto(858.61896765,233.94355787)(858.61396765,233.87355794)(858.60397567,233.79356111)
\curveto(858.58396768,233.73355808)(858.5689677,233.67355814)(858.55897567,233.61356111)
\curveto(858.55896771,233.55355826)(858.54896772,233.49355832)(858.52897567,233.43356111)
\curveto(858.48896778,233.30355851)(858.45396781,233.17355864)(858.42397567,233.04356111)
\curveto(858.39396787,232.9135589)(858.35396791,232.79355902)(858.30397567,232.68356111)
\curveto(858.09396817,232.20355961)(857.81396845,231.79856002)(857.46397567,231.46856111)
\curveto(857.11396915,231.14856067)(856.68396958,230.90356091)(856.17397567,230.73356111)
\curveto(856.0639702,230.69356112)(855.94397032,230.66356115)(855.81397567,230.64356111)
\curveto(855.69397057,230.62356119)(855.5689707,230.60356121)(855.43897567,230.58356111)
\curveto(855.37897089,230.57356124)(855.31397095,230.56856125)(855.24397567,230.56856111)
\curveto(855.18397108,230.55856126)(855.12397114,230.55356126)(855.06397567,230.55356111)
\curveto(855.02397124,230.54356127)(854.9639713,230.53856128)(854.88397567,230.53856111)
\curveto(854.81397145,230.53856128)(854.7639715,230.54356127)(854.73397567,230.55356111)
\curveto(854.69397157,230.56356125)(854.65397161,230.56856125)(854.61397567,230.56856111)
\curveto(854.57397169,230.55856126)(854.53897173,230.55856126)(854.50897567,230.56856111)
\lineto(854.41897567,230.56856111)
\lineto(854.05897567,230.61356111)
\curveto(853.91897235,230.65356116)(853.78397248,230.69356112)(853.65397567,230.73356111)
\curveto(853.52397274,230.77356104)(853.39897287,230.818561)(853.27897567,230.86856111)
\curveto(852.82897344,231.06856075)(852.45897381,231.32856049)(852.16897567,231.64856111)
\curveto(851.87897439,231.96855985)(851.63897463,232.35855946)(851.44897567,232.81856111)
\curveto(851.39897487,232.9185589)(851.35897491,233.0185588)(851.32897567,233.11856111)
\curveto(851.30897496,233.2185586)(851.28897498,233.32355849)(851.26897567,233.43356111)
\curveto(851.24897502,233.47355834)(851.23897503,233.50355831)(851.23897567,233.52356111)
\curveto(851.24897502,233.55355826)(851.24897502,233.58855823)(851.23897567,233.62856111)
\curveto(851.21897505,233.70855811)(851.20397506,233.78855803)(851.19397567,233.86856111)
\curveto(851.19397507,233.95855786)(851.18397508,234.04355777)(851.16397567,234.12356111)
\lineto(851.16397567,234.24356111)
\curveto(851.1639751,234.28355753)(851.15897511,234.32855749)(851.14897567,234.37856111)
\curveto(851.13897513,234.42855739)(851.13397513,234.5135573)(851.13397567,234.63356111)
\curveto(851.13397513,234.76355705)(851.14397512,234.85855696)(851.16397567,234.91856111)
\curveto(851.18397508,234.98855683)(851.18897508,235.05855676)(851.17897567,235.12856111)
\curveto(851.1689751,235.19855662)(851.17397509,235.26855655)(851.19397567,235.33856111)
\curveto(851.20397506,235.38855643)(851.20897506,235.42855639)(851.20897567,235.45856111)
\curveto(851.21897505,235.49855632)(851.22897504,235.54355627)(851.23897567,235.59356111)
\curveto(851.268975,235.7135561)(851.29397497,235.83355598)(851.31397567,235.95356111)
\curveto(851.34397492,236.07355574)(851.38397488,236.18855563)(851.43397567,236.29856111)
\curveto(851.58397468,236.66855515)(851.7639745,236.99855482)(851.97397567,237.28856111)
\curveto(852.19397407,237.58855423)(852.45897381,237.83855398)(852.76897567,238.03856111)
\curveto(852.88897338,238.1185537)(853.01397325,238.18355363)(853.14397567,238.23356111)
\curveto(853.27397299,238.29355352)(853.40897286,238.35355346)(853.54897567,238.41356111)
\curveto(853.6689726,238.46355335)(853.79897247,238.49355332)(853.93897567,238.50356111)
\curveto(854.07897219,238.52355329)(854.21897205,238.55355326)(854.35897567,238.59356111)
\lineto(854.55397567,238.59356111)
\curveto(854.62397164,238.60355321)(854.68897158,238.6135532)(854.74897567,238.62356111)
\curveto(855.63897063,238.63355318)(856.37896989,238.44855337)(856.96897567,238.06856111)
\curveto(857.55896871,237.68855413)(857.98396828,237.19355462)(858.24397567,236.58356111)
\curveto(858.29396797,236.48355533)(858.33396793,236.38355543)(858.36397567,236.28356111)
\curveto(858.39396787,236.18355563)(858.42896784,236.07855574)(858.46897567,235.96856111)
\curveto(858.49896777,235.85855596)(858.52396774,235.73855608)(858.54397567,235.60856111)
\curveto(858.5639677,235.48855633)(858.58896768,235.36355645)(858.61897567,235.23356111)
\curveto(858.62896764,235.18355663)(858.62896764,235.12855669)(858.61897567,235.06856111)
\curveto(858.61896765,235.0185568)(858.62396764,234.96855685)(858.63397567,234.91856111)
\moveto(857.29897567,234.06356111)
\curveto(857.31896895,234.13355768)(857.32396894,234.2135576)(857.31397567,234.30356111)
\lineto(857.31397567,234.55856111)
\curveto(857.31396895,234.94855687)(857.27896899,235.27855654)(857.20897567,235.54856111)
\curveto(857.17896909,235.62855619)(857.15396911,235.70855611)(857.13397567,235.78856111)
\curveto(857.11396915,235.86855595)(857.08896918,235.94355587)(857.05897567,236.01356111)
\curveto(856.77896949,236.66355515)(856.33396993,237.1135547)(855.72397567,237.36356111)
\curveto(855.65397061,237.39355442)(855.57897069,237.4135544)(855.49897567,237.42356111)
\lineto(855.25897567,237.48356111)
\curveto(855.17897109,237.50355431)(855.09397117,237.5135543)(855.00397567,237.51356111)
\lineto(854.73397567,237.51356111)
\lineto(854.46397567,237.46856111)
\curveto(854.3639719,237.44855437)(854.268972,237.42355439)(854.17897567,237.39356111)
\curveto(854.09897217,237.37355444)(854.01897225,237.34355447)(853.93897567,237.30356111)
\curveto(853.8689724,237.28355453)(853.80397246,237.25355456)(853.74397567,237.21356111)
\curveto(853.68397258,237.17355464)(853.62897264,237.13355468)(853.57897567,237.09356111)
\curveto(853.33897293,236.92355489)(853.14397312,236.7185551)(852.99397567,236.47856111)
\curveto(852.84397342,236.23855558)(852.71397355,235.95855586)(852.60397567,235.63856111)
\curveto(852.57397369,235.53855628)(852.55397371,235.43355638)(852.54397567,235.32356111)
\curveto(852.53397373,235.22355659)(852.51897375,235.1185567)(852.49897567,235.00856111)
\curveto(852.48897378,234.96855685)(852.48397378,234.90355691)(852.48397567,234.81356111)
\curveto(852.47397379,234.78355703)(852.4689738,234.74855707)(852.46897567,234.70856111)
\curveto(852.47897379,234.66855715)(852.48397378,234.62355719)(852.48397567,234.57356111)
\lineto(852.48397567,234.27356111)
\curveto(852.48397378,234.17355764)(852.49397377,234.08355773)(852.51397567,234.00356111)
\lineto(852.54397567,233.82356111)
\curveto(852.5639737,233.72355809)(852.57897369,233.62355819)(852.58897567,233.52356111)
\curveto(852.60897366,233.43355838)(852.63897363,233.34855847)(852.67897567,233.26856111)
\curveto(852.77897349,233.02855879)(852.89397337,232.80355901)(853.02397567,232.59356111)
\curveto(853.1639731,232.38355943)(853.33397293,232.20855961)(853.53397567,232.06856111)
\curveto(853.58397268,232.03855978)(853.62897264,232.0135598)(853.66897567,231.99356111)
\curveto(853.70897256,231.97355984)(853.75397251,231.94855987)(853.80397567,231.91856111)
\curveto(853.88397238,231.86855995)(853.9689723,231.82355999)(854.05897567,231.78356111)
\curveto(854.15897211,231.75356006)(854.263972,231.72356009)(854.37397567,231.69356111)
\curveto(854.42397184,231.67356014)(854.4689718,231.66356015)(854.50897567,231.66356111)
\curveto(854.55897171,231.67356014)(854.60897166,231.67356014)(854.65897567,231.66356111)
\curveto(854.68897158,231.65356016)(854.74897152,231.64356017)(854.83897567,231.63356111)
\curveto(854.93897133,231.62356019)(855.01397125,231.62856019)(855.06397567,231.64856111)
\curveto(855.10397116,231.65856016)(855.14397112,231.65856016)(855.18397567,231.64856111)
\curveto(855.22397104,231.64856017)(855.263971,231.65856016)(855.30397567,231.67856111)
\curveto(855.38397088,231.69856012)(855.4639708,231.7135601)(855.54397567,231.72356111)
\curveto(855.62397064,231.74356007)(855.69897057,231.76856005)(855.76897567,231.79856111)
\curveto(856.10897016,231.93855988)(856.38396988,232.13355968)(856.59397567,232.38356111)
\curveto(856.80396946,232.63355918)(856.97896929,232.92855889)(857.11897567,233.26856111)
\curveto(857.1689691,233.38855843)(857.19896907,233.5135583)(857.20897567,233.64356111)
\curveto(857.22896904,233.78355803)(857.25896901,233.92355789)(857.29897567,234.06356111)
}
}
{
\newrgbcolor{curcolor}{0 0 0}
\pscustom[linestyle=none,fillstyle=solid,fillcolor=curcolor]
{
\newpath
\moveto(862.55225692,238.62356111)
\curveto(863.27225286,238.63355318)(863.87725225,238.54855327)(864.36725692,238.36856111)
\curveto(864.85725127,238.19855362)(865.23725089,237.89355392)(865.50725692,237.45356111)
\curveto(865.57725055,237.34355447)(865.6322505,237.22855459)(865.67225692,237.10856111)
\curveto(865.71225042,236.99855482)(865.75225038,236.87355494)(865.79225692,236.73356111)
\curveto(865.81225032,236.66355515)(865.81725031,236.58855523)(865.80725692,236.50856111)
\curveto(865.79725033,236.43855538)(865.78225035,236.38355543)(865.76225692,236.34356111)
\curveto(865.74225039,236.32355549)(865.71725041,236.30355551)(865.68725692,236.28356111)
\curveto(865.65725047,236.27355554)(865.6322505,236.25855556)(865.61225692,236.23856111)
\curveto(865.56225057,236.2185556)(865.51225062,236.2135556)(865.46225692,236.22356111)
\curveto(865.41225072,236.23355558)(865.36225077,236.23355558)(865.31225692,236.22356111)
\curveto(865.2322509,236.20355561)(865.127251,236.19855562)(864.99725692,236.20856111)
\curveto(864.86725126,236.22855559)(864.77725135,236.25355556)(864.72725692,236.28356111)
\curveto(864.64725148,236.33355548)(864.59225154,236.39855542)(864.56225692,236.47856111)
\curveto(864.54225159,236.56855525)(864.50725162,236.65355516)(864.45725692,236.73356111)
\curveto(864.36725176,236.89355492)(864.24225189,237.03855478)(864.08225692,237.16856111)
\curveto(863.97225216,237.24855457)(863.85225228,237.30855451)(863.72225692,237.34856111)
\curveto(863.59225254,237.38855443)(863.45225268,237.42855439)(863.30225692,237.46856111)
\curveto(863.25225288,237.48855433)(863.20225293,237.49355432)(863.15225692,237.48356111)
\curveto(863.10225303,237.48355433)(863.05225308,237.48855433)(863.00225692,237.49856111)
\curveto(862.94225319,237.5185543)(862.86725326,237.52855429)(862.77725692,237.52856111)
\curveto(862.68725344,237.52855429)(862.61225352,237.5185543)(862.55225692,237.49856111)
\lineto(862.46225692,237.49856111)
\lineto(862.31225692,237.46856111)
\curveto(862.26225387,237.46855435)(862.21225392,237.46355435)(862.16225692,237.45356111)
\curveto(861.90225423,237.39355442)(861.68725444,237.30855451)(861.51725692,237.19856111)
\curveto(861.34725478,237.08855473)(861.2322549,236.90355491)(861.17225692,236.64356111)
\curveto(861.15225498,236.57355524)(861.14725498,236.50355531)(861.15725692,236.43356111)
\curveto(861.17725495,236.36355545)(861.19725493,236.30355551)(861.21725692,236.25356111)
\curveto(861.27725485,236.10355571)(861.34725478,235.99355582)(861.42725692,235.92356111)
\curveto(861.51725461,235.86355595)(861.6272545,235.79355602)(861.75725692,235.71356111)
\curveto(861.91725421,235.6135562)(862.09725403,235.53855628)(862.29725692,235.48856111)
\curveto(862.49725363,235.44855637)(862.69725343,235.39855642)(862.89725692,235.33856111)
\curveto(863.0272531,235.29855652)(863.15725297,235.26855655)(863.28725692,235.24856111)
\curveto(863.41725271,235.22855659)(863.54725258,235.19855662)(863.67725692,235.15856111)
\curveto(863.88725224,235.09855672)(864.09225204,235.03855678)(864.29225692,234.97856111)
\curveto(864.49225164,234.92855689)(864.69225144,234.86355695)(864.89225692,234.78356111)
\lineto(865.04225692,234.72356111)
\curveto(865.09225104,234.70355711)(865.14225099,234.67855714)(865.19225692,234.64856111)
\curveto(865.39225074,234.52855729)(865.56725056,234.39355742)(865.71725692,234.24356111)
\curveto(865.86725026,234.09355772)(865.99225014,233.90355791)(866.09225692,233.67356111)
\curveto(866.11225002,233.60355821)(866.13225,233.50855831)(866.15225692,233.38856111)
\curveto(866.17224996,233.3185585)(866.18224995,233.24355857)(866.18225692,233.16356111)
\curveto(866.19224994,233.09355872)(866.19724993,233.0135588)(866.19725692,232.92356111)
\lineto(866.19725692,232.77356111)
\curveto(866.17724995,232.70355911)(866.16724996,232.63355918)(866.16725692,232.56356111)
\curveto(866.16724996,232.49355932)(866.15724997,232.42355939)(866.13725692,232.35356111)
\curveto(866.10725002,232.24355957)(866.07225006,232.13855968)(866.03225692,232.03856111)
\curveto(865.99225014,231.93855988)(865.94725018,231.84855997)(865.89725692,231.76856111)
\curveto(865.73725039,231.50856031)(865.5322506,231.29856052)(865.28225692,231.13856111)
\curveto(865.0322511,230.98856083)(864.75225138,230.85856096)(864.44225692,230.74856111)
\curveto(864.35225178,230.7185611)(864.25725187,230.69856112)(864.15725692,230.68856111)
\curveto(864.06725206,230.66856115)(863.97725215,230.64356117)(863.88725692,230.61356111)
\curveto(863.78725234,230.59356122)(863.68725244,230.58356123)(863.58725692,230.58356111)
\curveto(863.48725264,230.58356123)(863.38725274,230.57356124)(863.28725692,230.55356111)
\lineto(863.13725692,230.55356111)
\curveto(863.08725304,230.54356127)(863.01725311,230.53856128)(862.92725692,230.53856111)
\curveto(862.83725329,230.53856128)(862.76725336,230.54356127)(862.71725692,230.55356111)
\lineto(862.55225692,230.55356111)
\curveto(862.49225364,230.57356124)(862.4272537,230.58356123)(862.35725692,230.58356111)
\curveto(862.28725384,230.57356124)(862.2272539,230.57856124)(862.17725692,230.59856111)
\curveto(862.127254,230.60856121)(862.06225407,230.6135612)(861.98225692,230.61356111)
\lineto(861.74225692,230.67356111)
\curveto(861.67225446,230.68356113)(861.59725453,230.70356111)(861.51725692,230.73356111)
\curveto(861.20725492,230.83356098)(860.93725519,230.95856086)(860.70725692,231.10856111)
\curveto(860.47725565,231.25856056)(860.27725585,231.45356036)(860.10725692,231.69356111)
\curveto(860.01725611,231.82355999)(859.94225619,231.95855986)(859.88225692,232.09856111)
\curveto(859.82225631,232.23855958)(859.76725636,232.39355942)(859.71725692,232.56356111)
\curveto(859.69725643,232.62355919)(859.68725644,232.69355912)(859.68725692,232.77356111)
\curveto(859.69725643,232.86355895)(859.71225642,232.93355888)(859.73225692,232.98356111)
\curveto(859.76225637,233.02355879)(859.81225632,233.06355875)(859.88225692,233.10356111)
\curveto(859.9322562,233.12355869)(860.00225613,233.13355868)(860.09225692,233.13356111)
\curveto(860.18225595,233.14355867)(860.27225586,233.14355867)(860.36225692,233.13356111)
\curveto(860.45225568,233.12355869)(860.53725559,233.10855871)(860.61725692,233.08856111)
\curveto(860.70725542,233.07855874)(860.76725536,233.06355875)(860.79725692,233.04356111)
\curveto(860.86725526,232.99355882)(860.91225522,232.9185589)(860.93225692,232.81856111)
\curveto(860.96225517,232.72855909)(860.99725513,232.64355917)(861.03725692,232.56356111)
\curveto(861.13725499,232.34355947)(861.27225486,232.17355964)(861.44225692,232.05356111)
\curveto(861.56225457,231.96355985)(861.69725443,231.89355992)(861.84725692,231.84356111)
\curveto(861.99725413,231.79356002)(862.15725397,231.74356007)(862.32725692,231.69356111)
\lineto(862.64225692,231.64856111)
\lineto(862.73225692,231.64856111)
\curveto(862.80225333,231.62856019)(862.89225324,231.6185602)(863.00225692,231.61856111)
\curveto(863.12225301,231.6185602)(863.22225291,231.62856019)(863.30225692,231.64856111)
\curveto(863.37225276,231.64856017)(863.4272527,231.65356016)(863.46725692,231.66356111)
\curveto(863.5272526,231.67356014)(863.58725254,231.67856014)(863.64725692,231.67856111)
\curveto(863.70725242,231.68856013)(863.76225237,231.69856012)(863.81225692,231.70856111)
\curveto(864.10225203,231.78856003)(864.3322518,231.89355992)(864.50225692,232.02356111)
\curveto(864.67225146,232.15355966)(864.79225134,232.37355944)(864.86225692,232.68356111)
\curveto(864.88225125,232.73355908)(864.88725124,232.78855903)(864.87725692,232.84856111)
\curveto(864.86725126,232.90855891)(864.85725127,232.95355886)(864.84725692,232.98356111)
\curveto(864.79725133,233.17355864)(864.7272514,233.3135585)(864.63725692,233.40356111)
\curveto(864.54725158,233.50355831)(864.4322517,233.59355822)(864.29225692,233.67356111)
\curveto(864.20225193,233.73355808)(864.10225203,233.78355803)(863.99225692,233.82356111)
\lineto(863.66225692,233.94356111)
\curveto(863.6322525,233.95355786)(863.60225253,233.95855786)(863.57225692,233.95856111)
\curveto(863.55225258,233.95855786)(863.5272526,233.96855785)(863.49725692,233.98856111)
\curveto(863.15725297,234.09855772)(862.80225333,234.17855764)(862.43225692,234.22856111)
\curveto(862.07225406,234.28855753)(861.7322544,234.38355743)(861.41225692,234.51356111)
\curveto(861.31225482,234.55355726)(861.21725491,234.58855723)(861.12725692,234.61856111)
\curveto(861.03725509,234.64855717)(860.95225518,234.68855713)(860.87225692,234.73856111)
\curveto(860.68225545,234.84855697)(860.50725562,234.97355684)(860.34725692,235.11356111)
\curveto(860.18725594,235.25355656)(860.06225607,235.42855639)(859.97225692,235.63856111)
\curveto(859.94225619,235.70855611)(859.91725621,235.77855604)(859.89725692,235.84856111)
\curveto(859.88725624,235.9185559)(859.87225626,235.99355582)(859.85225692,236.07356111)
\curveto(859.82225631,236.19355562)(859.81225632,236.32855549)(859.82225692,236.47856111)
\curveto(859.8322563,236.63855518)(859.84725628,236.77355504)(859.86725692,236.88356111)
\curveto(859.88725624,236.93355488)(859.89725623,236.97355484)(859.89725692,237.00356111)
\curveto(859.90725622,237.04355477)(859.92225621,237.08355473)(859.94225692,237.12356111)
\curveto(860.0322561,237.35355446)(860.15225598,237.55355426)(860.30225692,237.72356111)
\curveto(860.46225567,237.89355392)(860.64225549,238.04355377)(860.84225692,238.17356111)
\curveto(860.99225514,238.26355355)(861.15725497,238.33355348)(861.33725692,238.38356111)
\curveto(861.51725461,238.44355337)(861.70725442,238.49855332)(861.90725692,238.54856111)
\curveto(861.97725415,238.55855326)(862.04225409,238.56855325)(862.10225692,238.57856111)
\curveto(862.17225396,238.58855323)(862.24725388,238.59855322)(862.32725692,238.60856111)
\curveto(862.35725377,238.6185532)(862.39725373,238.6185532)(862.44725692,238.60856111)
\curveto(862.49725363,238.59855322)(862.5322536,238.60355321)(862.55225692,238.62356111)
}
}
{
\newrgbcolor{curcolor}{0.50196081 0.50196081 0.50196081}
\pscustom[linestyle=none,fillstyle=solid,fillcolor=curcolor]
{
\newpath
\moveto(798.51865829,241.42859773)
\lineto(813.51865829,241.42859773)
\lineto(813.51865829,226.42859773)
\lineto(798.51865829,226.42859773)
\closepath
}
}
{
\newrgbcolor{curcolor}{0 0 0}
\pscustom[linestyle=none,fillstyle=solid,fillcolor=curcolor]
{
\newpath
\moveto(822.4518663,218.78138338)
\curveto(823.4318598,218.80137242)(824.25185898,218.64137258)(824.9118663,218.30138338)
\curveto(825.58185765,217.97137325)(826.10185713,217.51137371)(826.4718663,216.92138338)
\curveto(826.57185666,216.76137446)(826.65185658,216.60637462)(826.7118663,216.45638338)
\curveto(826.78185645,216.31637491)(826.84685638,216.14637508)(826.9068663,215.94638338)
\curveto(826.9268563,215.89637533)(826.94685628,215.8263754)(826.9668663,215.73638338)
\curveto(826.98685624,215.65637557)(826.98185625,215.58137564)(826.9518663,215.51138338)
\curveto(826.9318563,215.45137577)(826.89185634,215.41137581)(826.8318663,215.39138338)
\curveto(826.78185645,215.38137584)(826.7268565,215.36637586)(826.6668663,215.34638338)
\lineto(826.5168663,215.34638338)
\curveto(826.48685674,215.33637589)(826.44685678,215.33137589)(826.3968663,215.33138338)
\lineto(826.2768663,215.33138338)
\curveto(826.13685709,215.33137589)(826.00685722,215.33637589)(825.8868663,215.34638338)
\curveto(825.77685745,215.36637586)(825.69685753,215.41637581)(825.6468663,215.49638338)
\curveto(825.57685765,215.59637563)(825.52185771,215.71137551)(825.4818663,215.84138338)
\curveto(825.44185779,215.97137525)(825.38685784,216.09137513)(825.3168663,216.20138338)
\curveto(825.18685804,216.4213748)(825.03685819,216.61137461)(824.8668663,216.77138338)
\curveto(824.70685852,216.93137429)(824.51685871,217.08137414)(824.2968663,217.22138338)
\curveto(824.17685905,217.30137392)(824.04185919,217.36137386)(823.8918663,217.40138338)
\curveto(823.75185948,217.44137378)(823.60685962,217.48137374)(823.4568663,217.52138338)
\curveto(823.34685988,217.55137367)(823.22186001,217.57137365)(823.0818663,217.58138338)
\curveto(822.94186029,217.60137362)(822.79186044,217.61137361)(822.6318663,217.61138338)
\curveto(822.48186075,217.61137361)(822.3318609,217.60137362)(822.1818663,217.58138338)
\curveto(822.04186119,217.57137365)(821.92186131,217.55137367)(821.8218663,217.52138338)
\curveto(821.72186151,217.50137372)(821.6268616,217.48137374)(821.5368663,217.46138338)
\curveto(821.44686178,217.44137378)(821.35686187,217.41137381)(821.2668663,217.37138338)
\curveto(820.4268628,217.0213742)(819.82186341,216.4213748)(819.4518663,215.57138338)
\curveto(819.38186385,215.43137579)(819.32186391,215.28137594)(819.2718663,215.12138338)
\curveto(819.231864,214.97137625)(819.18686404,214.81637641)(819.1368663,214.65638338)
\curveto(819.11686411,214.59637663)(819.10686412,214.53137669)(819.1068663,214.46138338)
\curveto(819.10686412,214.40137682)(819.09686413,214.34137688)(819.0768663,214.28138338)
\curveto(819.06686416,214.24137698)(819.06186417,214.20637702)(819.0618663,214.17638338)
\curveto(819.06186417,214.14637708)(819.05686417,214.11137711)(819.0468663,214.07138338)
\curveto(819.0268642,213.96137726)(819.01186422,213.84637738)(819.0018663,213.72638338)
\lineto(819.0018663,213.38138338)
\curveto(819.00186423,213.31137791)(818.99686423,213.23637799)(818.9868663,213.15638338)
\curveto(818.98686424,213.08637814)(818.99186424,213.0213782)(819.0018663,212.96138338)
\lineto(819.0018663,212.81138338)
\curveto(819.02186421,212.74137848)(819.0268642,212.67137855)(819.0168663,212.60138338)
\curveto(819.01686421,212.53137869)(819.0268642,212.46137876)(819.0468663,212.39138338)
\curveto(819.06686416,212.33137889)(819.07186416,212.27137895)(819.0618663,212.21138338)
\curveto(819.06186417,212.15137907)(819.07186416,212.09637913)(819.0918663,212.04638338)
\curveto(819.12186411,211.91637931)(819.14686408,211.78637944)(819.1668663,211.65638338)
\curveto(819.19686403,211.53637969)(819.231864,211.41637981)(819.2718663,211.29638338)
\curveto(819.44186379,210.79638043)(819.66186357,210.36638086)(819.9318663,210.00638338)
\curveto(820.20186303,209.65638157)(820.55686267,209.36638186)(820.9968663,209.13638338)
\curveto(821.13686209,209.06638216)(821.27686195,209.01138221)(821.4168663,208.97138338)
\curveto(821.56686166,208.93138229)(821.7268615,208.88638234)(821.8968663,208.83638338)
\curveto(821.96686126,208.81638241)(822.0318612,208.80638242)(822.0918663,208.80638338)
\curveto(822.15186108,208.81638241)(822.22186101,208.81138241)(822.3018663,208.79138338)
\curveto(822.35186088,208.78138244)(822.44186079,208.77138245)(822.5718663,208.76138338)
\curveto(822.70186053,208.76138246)(822.79686043,208.77138245)(822.8568663,208.79138338)
\lineto(822.9618663,208.79138338)
\curveto(823.00186023,208.80138242)(823.04186019,208.80138242)(823.0818663,208.79138338)
\curveto(823.12186011,208.79138243)(823.16186007,208.80138242)(823.2018663,208.82138338)
\curveto(823.30185993,208.84138238)(823.39685983,208.85638237)(823.4868663,208.86638338)
\curveto(823.58685964,208.88638234)(823.68185955,208.91638231)(823.7718663,208.95638338)
\curveto(824.55185868,209.27638195)(825.10185813,209.80138142)(825.4218663,210.53138338)
\curveto(825.50185773,210.71138051)(825.57685765,210.9263803)(825.6468663,211.17638338)
\curveto(825.66685756,211.26637996)(825.68185755,211.35637987)(825.6918663,211.44638338)
\curveto(825.71185752,211.54637968)(825.74685748,211.63637959)(825.7968663,211.71638338)
\curveto(825.84685738,211.79637943)(825.9268573,211.84137938)(826.0368663,211.85138338)
\curveto(826.14685708,211.86137936)(826.26685696,211.86637936)(826.3968663,211.86638338)
\lineto(826.5468663,211.86638338)
\curveto(826.59685663,211.86637936)(826.64185659,211.86137936)(826.6818663,211.85138338)
\lineto(826.7868663,211.85138338)
\lineto(826.8768663,211.82138338)
\curveto(826.91685631,211.8213794)(826.94685628,211.81137941)(826.9668663,211.79138338)
\curveto(827.03685619,211.75137947)(827.07685615,211.67637955)(827.0868663,211.56638338)
\curveto(827.09685613,211.46637976)(827.08685614,211.36637986)(827.0568663,211.26638338)
\curveto(826.99685623,211.03638019)(826.94185629,210.81638041)(826.8918663,210.60638338)
\curveto(826.84185639,210.39638083)(826.76685646,210.19638103)(826.6668663,210.00638338)
\curveto(826.58685664,209.87638135)(826.51185672,209.75138147)(826.4418663,209.63138338)
\curveto(826.38185685,209.51138171)(826.31185692,209.39138183)(826.2318663,209.27138338)
\curveto(826.05185718,209.01138221)(825.8268574,208.77138245)(825.5568663,208.55138338)
\curveto(825.29685793,208.34138288)(825.01185822,208.16638306)(824.7018663,208.02638338)
\curveto(824.59185864,207.97638325)(824.48185875,207.93638329)(824.3718663,207.90638338)
\curveto(824.27185896,207.87638335)(824.16685906,207.84138338)(824.0568663,207.80138338)
\curveto(823.94685928,207.76138346)(823.8318594,207.73638349)(823.7118663,207.72638338)
\curveto(823.60185963,207.70638352)(823.48685974,207.68638354)(823.3668663,207.66638338)
\curveto(823.31685991,207.64638358)(823.27185996,207.64138358)(823.2318663,207.65138338)
\curveto(823.19186004,207.65138357)(823.15186008,207.64638358)(823.1118663,207.63638338)
\curveto(823.05186018,207.6263836)(822.99186024,207.6213836)(822.9318663,207.62138338)
\curveto(822.87186036,207.6213836)(822.80686042,207.61638361)(822.7368663,207.60638338)
\curveto(822.70686052,207.59638363)(822.63686059,207.59638363)(822.5268663,207.60638338)
\curveto(822.4268608,207.60638362)(822.36186087,207.61138361)(822.3318663,207.62138338)
\curveto(822.28186095,207.63138359)(822.231861,207.63638359)(822.1818663,207.63638338)
\curveto(822.14186109,207.6263836)(822.09686113,207.6263836)(822.0468663,207.63638338)
\lineto(821.8968663,207.63638338)
\curveto(821.81686141,207.65638357)(821.74186149,207.67138355)(821.6718663,207.68138338)
\curveto(821.60186163,207.68138354)(821.5268617,207.69138353)(821.4468663,207.71138338)
\lineto(821.1768663,207.77138338)
\curveto(821.08686214,207.78138344)(821.00186223,207.80138342)(820.9218663,207.83138338)
\curveto(820.71186252,207.89138333)(820.52186271,207.96638326)(820.3518663,208.05638338)
\curveto(819.72186351,208.3263829)(819.21186402,208.71138251)(818.8218663,209.21138338)
\curveto(818.4318648,209.71138151)(818.12186511,210.30138092)(817.8918663,210.98138338)
\curveto(817.85186538,211.10138012)(817.81686541,211.22638)(817.7868663,211.35638338)
\curveto(817.76686546,211.48637974)(817.74186549,211.6213796)(817.7118663,211.76138338)
\curveto(817.69186554,211.81137941)(817.68186555,211.85637937)(817.6818663,211.89638338)
\curveto(817.69186554,211.93637929)(817.69186554,211.98137924)(817.6818663,212.03138338)
\curveto(817.66186557,212.1213791)(817.64686558,212.21637901)(817.6368663,212.31638338)
\curveto(817.63686559,212.41637881)(817.6268656,212.51137871)(817.6068663,212.60138338)
\lineto(817.6068663,212.88638338)
\curveto(817.58686564,212.93637829)(817.57686565,213.0213782)(817.5768663,213.14138338)
\curveto(817.57686565,213.26137796)(817.58686564,213.34637788)(817.6068663,213.39638338)
\curveto(817.61686561,213.4263778)(817.61686561,213.45637777)(817.6068663,213.48638338)
\curveto(817.59686563,213.5263777)(817.59686563,213.55637767)(817.6068663,213.57638338)
\lineto(817.6068663,213.71138338)
\curveto(817.61686561,213.79137743)(817.62186561,213.87137735)(817.6218663,213.95138338)
\curveto(817.6318656,214.04137718)(817.64686558,214.1263771)(817.6668663,214.20638338)
\curveto(817.68686554,214.26637696)(817.69686553,214.3263769)(817.6968663,214.38638338)
\curveto(817.69686553,214.45637677)(817.70686552,214.5263767)(817.7268663,214.59638338)
\curveto(817.77686545,214.76637646)(817.81686541,214.93137629)(817.8468663,215.09138338)
\curveto(817.87686535,215.25137597)(817.92186531,215.40137582)(817.9818663,215.54138338)
\lineto(818.1318663,215.93138338)
\curveto(818.19186504,216.07137515)(818.25686497,216.19637503)(818.3268663,216.30638338)
\curveto(818.47686475,216.56637466)(818.6268646,216.80137442)(818.7768663,217.01138338)
\curveto(818.80686442,217.06137416)(818.84186439,217.10137412)(818.8818663,217.13138338)
\curveto(818.9318643,217.17137405)(818.97186426,217.21637401)(819.0018663,217.26638338)
\curveto(819.06186417,217.34637388)(819.12186411,217.41637381)(819.1818663,217.47638338)
\lineto(819.3918663,217.65638338)
\curveto(819.45186378,217.70637352)(819.50686372,217.75137347)(819.5568663,217.79138338)
\curveto(819.61686361,217.84137338)(819.68186355,217.89137333)(819.7518663,217.94138338)
\curveto(819.90186333,218.05137317)(820.05686317,218.14637308)(820.2168663,218.22638338)
\curveto(820.38686284,218.30637292)(820.56186267,218.38637284)(820.7418663,218.46638338)
\curveto(820.85186238,218.51637271)(820.96686226,218.55137267)(821.0868663,218.57138338)
\curveto(821.21686201,218.60137262)(821.34186189,218.63637259)(821.4618663,218.67638338)
\curveto(821.5318617,218.68637254)(821.59686163,218.69637253)(821.6568663,218.70638338)
\lineto(821.8368663,218.73638338)
\curveto(821.91686131,218.74637248)(821.99186124,218.75137247)(822.0618663,218.75138338)
\curveto(822.14186109,218.76137246)(822.22186101,218.77137245)(822.3018663,218.78138338)
\curveto(822.32186091,218.79137243)(822.34686088,218.79137243)(822.3768663,218.78138338)
\curveto(822.40686082,218.77137245)(822.4318608,218.77137245)(822.4518663,218.78138338)
}
}
{
\newrgbcolor{curcolor}{0 0 0}
\pscustom[linestyle=none,fillstyle=solid,fillcolor=curcolor]
{
\newpath
\moveto(835.81171005,212.06138338)
\curveto(835.83170199,212.00137922)(835.84170198,211.90637932)(835.84171005,211.77638338)
\curveto(835.84170198,211.65637957)(835.83670198,211.57137965)(835.82671005,211.52138338)
\lineto(835.82671005,211.37138338)
\curveto(835.816702,211.29137993)(835.80670201,211.21638001)(835.79671005,211.14638338)
\curveto(835.79670202,211.08638014)(835.79170203,211.01638021)(835.78171005,210.93638338)
\curveto(835.76170206,210.87638035)(835.74670207,210.81638041)(835.73671005,210.75638338)
\curveto(835.73670208,210.69638053)(835.72670209,210.63638059)(835.70671005,210.57638338)
\curveto(835.66670215,210.44638078)(835.63170219,210.31638091)(835.60171005,210.18638338)
\curveto(835.57170225,210.05638117)(835.53170229,209.93638129)(835.48171005,209.82638338)
\curveto(835.27170255,209.34638188)(834.99170283,208.94138228)(834.64171005,208.61138338)
\curveto(834.29170353,208.29138293)(833.86170396,208.04638318)(833.35171005,207.87638338)
\curveto(833.24170458,207.83638339)(833.1217047,207.80638342)(832.99171005,207.78638338)
\curveto(832.87170495,207.76638346)(832.74670507,207.74638348)(832.61671005,207.72638338)
\curveto(832.55670526,207.71638351)(832.49170533,207.71138351)(832.42171005,207.71138338)
\curveto(832.36170546,207.70138352)(832.30170552,207.69638353)(832.24171005,207.69638338)
\curveto(832.20170562,207.68638354)(832.14170568,207.68138354)(832.06171005,207.68138338)
\curveto(831.99170583,207.68138354)(831.94170588,207.68638354)(831.91171005,207.69638338)
\curveto(831.87170595,207.70638352)(831.83170599,207.71138351)(831.79171005,207.71138338)
\curveto(831.75170607,207.70138352)(831.7167061,207.70138352)(831.68671005,207.71138338)
\lineto(831.59671005,207.71138338)
\lineto(831.23671005,207.75638338)
\curveto(831.09670672,207.79638343)(830.96170686,207.83638339)(830.83171005,207.87638338)
\curveto(830.70170712,207.91638331)(830.57670724,207.96138326)(830.45671005,208.01138338)
\curveto(830.00670781,208.21138301)(829.63670818,208.47138275)(829.34671005,208.79138338)
\curveto(829.05670876,209.11138211)(828.816709,209.50138172)(828.62671005,209.96138338)
\curveto(828.57670924,210.06138116)(828.53670928,210.16138106)(828.50671005,210.26138338)
\curveto(828.48670933,210.36138086)(828.46670935,210.46638076)(828.44671005,210.57638338)
\curveto(828.42670939,210.61638061)(828.4167094,210.64638058)(828.41671005,210.66638338)
\curveto(828.42670939,210.69638053)(828.42670939,210.73138049)(828.41671005,210.77138338)
\curveto(828.39670942,210.85138037)(828.38170944,210.93138029)(828.37171005,211.01138338)
\curveto(828.37170945,211.10138012)(828.36170946,211.18638004)(828.34171005,211.26638338)
\lineto(828.34171005,211.38638338)
\curveto(828.34170948,211.4263798)(828.33670948,211.47137975)(828.32671005,211.52138338)
\curveto(828.3167095,211.57137965)(828.31170951,211.65637957)(828.31171005,211.77638338)
\curveto(828.31170951,211.90637932)(828.3217095,212.00137922)(828.34171005,212.06138338)
\curveto(828.36170946,212.13137909)(828.36670945,212.20137902)(828.35671005,212.27138338)
\curveto(828.34670947,212.34137888)(828.35170947,212.41137881)(828.37171005,212.48138338)
\curveto(828.38170944,212.53137869)(828.38670943,212.57137865)(828.38671005,212.60138338)
\curveto(828.39670942,212.64137858)(828.40670941,212.68637854)(828.41671005,212.73638338)
\curveto(828.44670937,212.85637837)(828.47170935,212.97637825)(828.49171005,213.09638338)
\curveto(828.5217093,213.21637801)(828.56170926,213.33137789)(828.61171005,213.44138338)
\curveto(828.76170906,213.81137741)(828.94170888,214.14137708)(829.15171005,214.43138338)
\curveto(829.37170845,214.73137649)(829.63670818,214.98137624)(829.94671005,215.18138338)
\curveto(830.06670775,215.26137596)(830.19170763,215.3263759)(830.32171005,215.37638338)
\curveto(830.45170737,215.43637579)(830.58670723,215.49637573)(830.72671005,215.55638338)
\curveto(830.84670697,215.60637562)(830.97670684,215.63637559)(831.11671005,215.64638338)
\curveto(831.25670656,215.66637556)(831.39670642,215.69637553)(831.53671005,215.73638338)
\lineto(831.73171005,215.73638338)
\curveto(831.80170602,215.74637548)(831.86670595,215.75637547)(831.92671005,215.76638338)
\curveto(832.816705,215.77637545)(833.55670426,215.59137563)(834.14671005,215.21138338)
\curveto(834.73670308,214.83137639)(835.16170266,214.33637689)(835.42171005,213.72638338)
\curveto(835.47170235,213.6263776)(835.51170231,213.5263777)(835.54171005,213.42638338)
\curveto(835.57170225,213.3263779)(835.60670221,213.221378)(835.64671005,213.11138338)
\curveto(835.67670214,213.00137822)(835.70170212,212.88137834)(835.72171005,212.75138338)
\curveto(835.74170208,212.63137859)(835.76670205,212.50637872)(835.79671005,212.37638338)
\curveto(835.80670201,212.3263789)(835.80670201,212.27137895)(835.79671005,212.21138338)
\curveto(835.79670202,212.16137906)(835.80170202,212.11137911)(835.81171005,212.06138338)
\moveto(834.47671005,211.20638338)
\curveto(834.49670332,211.27637995)(834.50170332,211.35637987)(834.49171005,211.44638338)
\lineto(834.49171005,211.70138338)
\curveto(834.49170333,212.09137913)(834.45670336,212.4213788)(834.38671005,212.69138338)
\curveto(834.35670346,212.77137845)(834.33170349,212.85137837)(834.31171005,212.93138338)
\curveto(834.29170353,213.01137821)(834.26670355,213.08637814)(834.23671005,213.15638338)
\curveto(833.95670386,213.80637742)(833.51170431,214.25637697)(832.90171005,214.50638338)
\curveto(832.83170499,214.53637669)(832.75670506,214.55637667)(832.67671005,214.56638338)
\lineto(832.43671005,214.62638338)
\curveto(832.35670546,214.64637658)(832.27170555,214.65637657)(832.18171005,214.65638338)
\lineto(831.91171005,214.65638338)
\lineto(831.64171005,214.61138338)
\curveto(831.54170628,214.59137663)(831.44670637,214.56637666)(831.35671005,214.53638338)
\curveto(831.27670654,214.51637671)(831.19670662,214.48637674)(831.11671005,214.44638338)
\curveto(831.04670677,214.4263768)(830.98170684,214.39637683)(830.92171005,214.35638338)
\curveto(830.86170696,214.31637691)(830.80670701,214.27637695)(830.75671005,214.23638338)
\curveto(830.5167073,214.06637716)(830.3217075,213.86137736)(830.17171005,213.62138338)
\curveto(830.0217078,213.38137784)(829.89170793,213.10137812)(829.78171005,212.78138338)
\curveto(829.75170807,212.68137854)(829.73170809,212.57637865)(829.72171005,212.46638338)
\curveto(829.71170811,212.36637886)(829.69670812,212.26137896)(829.67671005,212.15138338)
\curveto(829.66670815,212.11137911)(829.66170816,212.04637918)(829.66171005,211.95638338)
\curveto(829.65170817,211.9263793)(829.64670817,211.89137933)(829.64671005,211.85138338)
\curveto(829.65670816,211.81137941)(829.66170816,211.76637946)(829.66171005,211.71638338)
\lineto(829.66171005,211.41638338)
\curveto(829.66170816,211.31637991)(829.67170815,211.22638)(829.69171005,211.14638338)
\lineto(829.72171005,210.96638338)
\curveto(829.74170808,210.86638036)(829.75670806,210.76638046)(829.76671005,210.66638338)
\curveto(829.78670803,210.57638065)(829.816708,210.49138073)(829.85671005,210.41138338)
\curveto(829.95670786,210.17138105)(830.07170775,209.94638128)(830.20171005,209.73638338)
\curveto(830.34170748,209.5263817)(830.51170731,209.35138187)(830.71171005,209.21138338)
\curveto(830.76170706,209.18138204)(830.80670701,209.15638207)(830.84671005,209.13638338)
\curveto(830.88670693,209.11638211)(830.93170689,209.09138213)(830.98171005,209.06138338)
\curveto(831.06170676,209.01138221)(831.14670667,208.96638226)(831.23671005,208.92638338)
\curveto(831.33670648,208.89638233)(831.44170638,208.86638236)(831.55171005,208.83638338)
\curveto(831.60170622,208.81638241)(831.64670617,208.80638242)(831.68671005,208.80638338)
\curveto(831.73670608,208.81638241)(831.78670603,208.81638241)(831.83671005,208.80638338)
\curveto(831.86670595,208.79638243)(831.92670589,208.78638244)(832.01671005,208.77638338)
\curveto(832.1167057,208.76638246)(832.19170563,208.77138245)(832.24171005,208.79138338)
\curveto(832.28170554,208.80138242)(832.3217055,208.80138242)(832.36171005,208.79138338)
\curveto(832.40170542,208.79138243)(832.44170538,208.80138242)(832.48171005,208.82138338)
\curveto(832.56170526,208.84138238)(832.64170518,208.85638237)(832.72171005,208.86638338)
\curveto(832.80170502,208.88638234)(832.87670494,208.91138231)(832.94671005,208.94138338)
\curveto(833.28670453,209.08138214)(833.56170426,209.27638195)(833.77171005,209.52638338)
\curveto(833.98170384,209.77638145)(834.15670366,210.07138115)(834.29671005,210.41138338)
\curveto(834.34670347,210.53138069)(834.37670344,210.65638057)(834.38671005,210.78638338)
\curveto(834.40670341,210.9263803)(834.43670338,211.06638016)(834.47671005,211.20638338)
}
}
{
\newrgbcolor{curcolor}{0 0 0}
\pscustom[linestyle=none,fillstyle=solid,fillcolor=curcolor]
{
\newpath
\moveto(840.9899913,215.76638338)
\curveto(841.36998631,215.77637545)(841.68998599,215.73637549)(841.9499913,215.64638338)
\curveto(842.21998546,215.55637567)(842.46498522,215.4263758)(842.6849913,215.25638338)
\curveto(842.76498492,215.20637602)(842.82998485,215.13637609)(842.8799913,215.04638338)
\curveto(842.93998474,214.96637626)(843.00498468,214.89137633)(843.0749913,214.82138338)
\curveto(843.09498459,214.80137642)(843.12498456,214.77637645)(843.1649913,214.74638338)
\curveto(843.20498448,214.71637651)(843.25498443,214.70637652)(843.3149913,214.71638338)
\curveto(843.41498427,214.74637648)(843.49998418,214.80637642)(843.5699913,214.89638338)
\curveto(843.64998403,214.99637623)(843.72998395,215.07137615)(843.8099913,215.12138338)
\curveto(843.94998373,215.23137599)(844.09498359,215.3263759)(844.2449913,215.40638338)
\curveto(844.39498329,215.49637573)(844.55998312,215.57137565)(844.7399913,215.63138338)
\curveto(844.81998286,215.66137556)(844.90498278,215.68137554)(844.9949913,215.69138338)
\curveto(845.09498259,215.71137551)(845.18998249,215.73137549)(845.2799913,215.75138338)
\curveto(845.32998235,215.76137546)(845.37498231,215.76637546)(845.4149913,215.76638338)
\lineto(845.5649913,215.76638338)
\curveto(845.61498207,215.78637544)(845.684982,215.79137543)(845.7749913,215.78138338)
\curveto(845.86498182,215.78137544)(845.92998175,215.77637545)(845.9699913,215.76638338)
\curveto(846.01998166,215.75637547)(846.09498159,215.75137547)(846.1949913,215.75138338)
\curveto(846.2849814,215.73137549)(846.36998131,215.71137551)(846.4499913,215.69138338)
\curveto(846.53998114,215.68137554)(846.62498106,215.66137556)(846.7049913,215.63138338)
\curveto(846.75498093,215.61137561)(846.79998088,215.59637563)(846.8399913,215.58638338)
\curveto(846.88998079,215.58637564)(846.93998074,215.57637565)(846.9899913,215.55638338)
\curveto(847.48998019,215.33637589)(847.83497985,214.99637623)(848.0249913,214.53638338)
\curveto(848.06497962,214.45637677)(848.09497959,214.36637686)(848.1149913,214.26638338)
\curveto(848.13497955,214.17637705)(848.15497953,214.07637715)(848.1749913,213.96638338)
\curveto(848.19497949,213.93637729)(848.19997948,213.90137732)(848.1899913,213.86138338)
\curveto(848.18997949,213.83137739)(848.19497949,213.80137742)(848.2049913,213.77138338)
\lineto(848.2049913,213.63638338)
\curveto(848.21497947,213.59637763)(848.21497947,213.55137767)(848.2049913,213.50138338)
\curveto(848.20497948,213.45137777)(848.20497948,213.40137782)(848.2049913,213.35138338)
\lineto(848.2049913,212.76638338)
\lineto(848.2049913,211.80638338)
\lineto(848.2049913,208.95638338)
\curveto(848.20497948,208.79638243)(848.20497948,208.60638262)(848.2049913,208.38638338)
\curveto(848.21497947,208.16638306)(848.17497951,208.0213832)(848.0849913,207.95138338)
\curveto(848.04497964,207.9213833)(847.9799797,207.89638333)(847.8899913,207.87638338)
\curveto(847.79997988,207.86638336)(847.70497998,207.86138336)(847.6049913,207.86138338)
\curveto(847.50498018,207.86138336)(847.40498028,207.86638336)(847.3049913,207.87638338)
\curveto(847.21498047,207.88638334)(847.14998053,207.90638332)(847.1099913,207.93638338)
\curveto(847.04998063,207.96638326)(847.00998067,208.0263832)(846.9899913,208.11638338)
\curveto(846.96998071,208.17638305)(846.96498072,208.23638299)(846.9749913,208.29638338)
\curveto(846.9849807,208.36638286)(846.9799807,208.43138279)(846.9599913,208.49138338)
\curveto(846.94998073,208.54138268)(846.94498074,208.59638263)(846.9449913,208.65638338)
\curveto(846.95498073,208.7263825)(846.95998072,208.79138243)(846.9599913,208.85138338)
\lineto(846.9599913,209.52638338)
\lineto(846.9599913,212.39138338)
\curveto(846.95998072,212.7213785)(846.94998073,213.03137819)(846.9299913,213.32138338)
\curveto(846.91998076,213.6213776)(846.84998083,213.87137735)(846.7199913,214.07138338)
\curveto(846.56998111,214.31137691)(846.33998134,214.48637674)(846.0299913,214.59638338)
\curveto(845.96998171,214.61637661)(845.90498178,214.6263766)(845.8349913,214.62638338)
\curveto(845.77498191,214.63637659)(845.70998197,214.65137657)(845.6399913,214.67138338)
\curveto(845.59998208,214.68137654)(845.53498215,214.68137654)(845.4449913,214.67138338)
\curveto(845.35498233,214.67137655)(845.29498239,214.66637656)(845.2649913,214.65638338)
\curveto(845.21498247,214.64637658)(845.16498252,214.64137658)(845.1149913,214.64138338)
\curveto(845.06498262,214.65137657)(845.01498267,214.64637658)(844.9649913,214.62638338)
\curveto(844.82498286,214.59637663)(844.68998299,214.55637667)(844.5599913,214.50638338)
\curveto(844.03998364,214.28637694)(843.68998399,213.90137732)(843.5099913,213.35138338)
\curveto(843.45998422,213.18137804)(843.42998425,212.98637824)(843.4199913,212.76638338)
\lineto(843.4199913,212.09138338)
\lineto(843.4199913,210.12638338)
\lineto(843.4199913,208.67138338)
\lineto(843.4199913,208.29638338)
\curveto(843.41998426,208.17638305)(843.39498429,208.08138314)(843.3449913,208.01138338)
\curveto(843.29498439,207.93138329)(843.20998447,207.88638334)(843.0899913,207.87638338)
\curveto(842.96998471,207.86638336)(842.84498484,207.86138336)(842.7149913,207.86138338)
\curveto(842.54498514,207.86138336)(842.41998526,207.88138334)(842.3399913,207.92138338)
\curveto(842.24998543,207.97138325)(842.19498549,208.05138317)(842.1749913,208.16138338)
\curveto(842.16498552,208.28138294)(842.15998552,208.41138281)(842.1599913,208.55138338)
\lineto(842.1599913,209.97638338)
\lineto(842.1599913,212.45138338)
\curveto(842.15998552,212.77137845)(842.14998553,213.06637816)(842.1299913,213.33638338)
\curveto(842.10998557,213.61637761)(842.03998564,213.85637737)(841.9199913,214.05638338)
\curveto(841.80998587,214.23637699)(841.684986,214.36637686)(841.5449913,214.44638338)
\curveto(841.40498628,214.53637669)(841.21498647,214.60637662)(840.9749913,214.65638338)
\curveto(840.93498675,214.66637656)(840.88998679,214.67137655)(840.8399913,214.67138338)
\lineto(840.7049913,214.67138338)
\curveto(840.4849872,214.67137655)(840.28998739,214.64637658)(840.1199913,214.59638338)
\curveto(839.95998772,214.54637668)(839.81498787,214.48137674)(839.6849913,214.40138338)
\curveto(839.17498851,214.09137713)(838.83498885,213.6263776)(838.6649913,213.00638338)
\curveto(838.62498906,212.87637835)(838.60498908,212.7263785)(838.6049913,212.55638338)
\curveto(838.61498907,212.39637883)(838.61998906,212.23637899)(838.6199913,212.07638338)
\lineto(838.6199913,210.38138338)
\lineto(838.6199913,208.73138338)
\lineto(838.6199913,208.32638338)
\curveto(838.61998906,208.18638304)(838.58998909,208.07638315)(838.5299913,207.99638338)
\curveto(838.4799892,207.9263833)(838.40498928,207.88638334)(838.3049913,207.87638338)
\curveto(838.20498948,207.86638336)(838.09998958,207.86138336)(837.9899913,207.86138338)
\lineto(837.7649913,207.86138338)
\curveto(837.70498998,207.88138334)(837.64499004,207.89638333)(837.5849913,207.90638338)
\curveto(837.53499015,207.91638331)(837.48999019,207.94638328)(837.4499913,207.99638338)
\curveto(837.39999028,208.05638317)(837.37499031,208.13138309)(837.3749913,208.22138338)
\lineto(837.3749913,208.53638338)
\lineto(837.3749913,209.51138338)
\lineto(837.3749913,213.80138338)
\lineto(837.3749913,214.91138338)
\lineto(837.3749913,215.19638338)
\curveto(837.37499031,215.29637593)(837.39499029,215.37637585)(837.4349913,215.43638338)
\curveto(837.46499022,215.49637573)(837.50999017,215.53637569)(837.5699913,215.55638338)
\curveto(837.64999003,215.58637564)(837.77498991,215.60137562)(837.9449913,215.60138338)
\curveto(838.12498956,215.60137562)(838.25498943,215.58637564)(838.3349913,215.55638338)
\curveto(838.41498927,215.51637571)(838.46998921,215.46637576)(838.4999913,215.40638338)
\curveto(838.51998916,215.35637587)(838.52998915,215.29637593)(838.5299913,215.22638338)
\curveto(838.53998914,215.15637607)(838.54998913,215.09137613)(838.5599913,215.03138338)
\curveto(838.56998911,214.97137625)(838.58998909,214.9213763)(838.6199913,214.88138338)
\curveto(838.64998903,214.84137638)(838.69998898,214.8213764)(838.7699913,214.82138338)
\curveto(838.78998889,214.84137638)(838.80998887,214.85137637)(838.8299913,214.85138338)
\curveto(838.85998882,214.85137637)(838.8849888,214.86137636)(838.9049913,214.88138338)
\curveto(838.96498872,214.93137629)(839.01998866,214.98137624)(839.0699913,215.03138338)
\lineto(839.2499913,215.18138338)
\curveto(839.46998821,215.34137588)(839.71998796,215.48137574)(839.9999913,215.60138338)
\curveto(840.09998758,215.64137558)(840.19998748,215.66637556)(840.2999913,215.67638338)
\curveto(840.39998728,215.69637553)(840.50498718,215.7213755)(840.6149913,215.75138338)
\lineto(840.7949913,215.75138338)
\curveto(840.86498682,215.76137546)(840.92998675,215.76637546)(840.9899913,215.76638338)
}
}
{
\newrgbcolor{curcolor}{0 0 0}
\pscustom[linestyle=none,fillstyle=solid,fillcolor=curcolor]
{
\newpath
\moveto(850.56772567,215.58638338)
\lineto(851.00272567,215.58638338)
\curveto(851.15272371,215.58637564)(851.2577236,215.54637568)(851.31772567,215.46638338)
\curveto(851.36772349,215.38637584)(851.39272347,215.28637594)(851.39272567,215.16638338)
\curveto(851.40272346,215.04637618)(851.40772345,214.9263763)(851.40772567,214.80638338)
\lineto(851.40772567,213.38138338)
\lineto(851.40772567,211.11638338)
\lineto(851.40772567,210.42638338)
\curveto(851.40772345,210.19638103)(851.43272343,209.99638123)(851.48272567,209.82638338)
\curveto(851.64272322,209.37638185)(851.94272292,209.06138216)(852.38272567,208.88138338)
\curveto(852.60272226,208.79138243)(852.86772199,208.75638247)(853.17772567,208.77638338)
\curveto(853.48772137,208.80638242)(853.73772112,208.86138236)(853.92772567,208.94138338)
\curveto(854.2577206,209.08138214)(854.51772034,209.25638197)(854.70772567,209.46638338)
\curveto(854.90771995,209.68638154)(855.0627198,209.97138125)(855.17272567,210.32138338)
\curveto(855.20271966,210.40138082)(855.22271964,210.48138074)(855.23272567,210.56138338)
\curveto(855.24271962,210.64138058)(855.2577196,210.7263805)(855.27772567,210.81638338)
\curveto(855.28771957,210.86638036)(855.28771957,210.91138031)(855.27772567,210.95138338)
\curveto(855.27771958,210.99138023)(855.28771957,211.03638019)(855.30772567,211.08638338)
\lineto(855.30772567,211.40138338)
\curveto(855.32771953,211.48137974)(855.33271953,211.57137965)(855.32272567,211.67138338)
\curveto(855.31271955,211.78137944)(855.30771955,211.88137934)(855.30772567,211.97138338)
\lineto(855.30772567,213.14138338)
\lineto(855.30772567,214.73138338)
\curveto(855.30771955,214.85137637)(855.30271956,214.97637625)(855.29272567,215.10638338)
\curveto(855.29271957,215.24637598)(855.31771954,215.35637587)(855.36772567,215.43638338)
\curveto(855.40771945,215.48637574)(855.45271941,215.51637571)(855.50272567,215.52638338)
\curveto(855.5627193,215.54637568)(855.63271923,215.56637566)(855.71272567,215.58638338)
\lineto(855.93772567,215.58638338)
\curveto(856.0577188,215.58637564)(856.1627187,215.58137564)(856.25272567,215.57138338)
\curveto(856.35271851,215.56137566)(856.42771843,215.51637571)(856.47772567,215.43638338)
\curveto(856.52771833,215.38637584)(856.55271831,215.31137591)(856.55272567,215.21138338)
\lineto(856.55272567,214.92638338)
\lineto(856.55272567,213.90638338)
\lineto(856.55272567,209.87138338)
\lineto(856.55272567,208.52138338)
\curveto(856.55271831,208.40138282)(856.54771831,208.28638294)(856.53772567,208.17638338)
\curveto(856.53771832,208.07638315)(856.50271836,208.00138322)(856.43272567,207.95138338)
\curveto(856.39271847,207.9213833)(856.33271853,207.89638333)(856.25272567,207.87638338)
\curveto(856.17271869,207.86638336)(856.08271878,207.85638337)(855.98272567,207.84638338)
\curveto(855.89271897,207.84638338)(855.80271906,207.85138337)(855.71272567,207.86138338)
\curveto(855.63271923,207.87138335)(855.57271929,207.89138333)(855.53272567,207.92138338)
\curveto(855.48271938,207.96138326)(855.43771942,208.0263832)(855.39772567,208.11638338)
\curveto(855.38771947,208.15638307)(855.37771948,208.21138301)(855.36772567,208.28138338)
\curveto(855.36771949,208.35138287)(855.3627195,208.41638281)(855.35272567,208.47638338)
\curveto(855.34271952,208.54638268)(855.32271954,208.60138262)(855.29272567,208.64138338)
\curveto(855.2627196,208.68138254)(855.21771964,208.69638253)(855.15772567,208.68638338)
\curveto(855.07771978,208.66638256)(854.99771986,208.60638262)(854.91772567,208.50638338)
\curveto(854.83772002,208.41638281)(854.7627201,208.34638288)(854.69272567,208.29638338)
\curveto(854.47272039,208.13638309)(854.22272064,207.99638323)(853.94272567,207.87638338)
\curveto(853.83272103,207.8263834)(853.71772114,207.79638343)(853.59772567,207.78638338)
\curveto(853.48772137,207.76638346)(853.37272149,207.74138348)(853.25272567,207.71138338)
\curveto(853.20272166,207.70138352)(853.14772171,207.70138352)(853.08772567,207.71138338)
\curveto(853.03772182,207.7213835)(852.98772187,207.71638351)(852.93772567,207.69638338)
\curveto(852.83772202,207.67638355)(852.74772211,207.67638355)(852.66772567,207.69638338)
\lineto(852.51772567,207.69638338)
\curveto(852.46772239,207.71638351)(852.40772245,207.7263835)(852.33772567,207.72638338)
\curveto(852.27772258,207.7263835)(852.22272264,207.73138349)(852.17272567,207.74138338)
\curveto(852.13272273,207.76138346)(852.09272277,207.77138345)(852.05272567,207.77138338)
\curveto(852.02272284,207.76138346)(851.98272288,207.76638346)(851.93272567,207.78638338)
\lineto(851.69272567,207.84638338)
\curveto(851.62272324,207.86638336)(851.54772331,207.89638333)(851.46772567,207.93638338)
\curveto(851.20772365,208.04638318)(850.98772387,208.19138303)(850.80772567,208.37138338)
\curveto(850.63772422,208.56138266)(850.49772436,208.78638244)(850.38772567,209.04638338)
\curveto(850.34772451,209.13638209)(850.31772454,209.226382)(850.29772567,209.31638338)
\lineto(850.23772567,209.61638338)
\curveto(850.21772464,209.67638155)(850.20772465,209.73138149)(850.20772567,209.78138338)
\curveto(850.21772464,209.84138138)(850.21272465,209.90638132)(850.19272567,209.97638338)
\curveto(850.18272468,209.99638123)(850.17772468,210.0213812)(850.17772567,210.05138338)
\curveto(850.17772468,210.09138113)(850.17272469,210.1263811)(850.16272567,210.15638338)
\lineto(850.16272567,210.30638338)
\curveto(850.15272471,210.34638088)(850.14772471,210.39138083)(850.14772567,210.44138338)
\curveto(850.1577247,210.50138072)(850.1627247,210.55638067)(850.16272567,210.60638338)
\lineto(850.16272567,211.20638338)
\lineto(850.16272567,213.96638338)
\lineto(850.16272567,214.92638338)
\lineto(850.16272567,215.19638338)
\curveto(850.1627247,215.28637594)(850.18272468,215.36137586)(850.22272567,215.42138338)
\curveto(850.2627246,215.49137573)(850.33772452,215.54137568)(850.44772567,215.57138338)
\curveto(850.46772439,215.58137564)(850.48772437,215.58137564)(850.50772567,215.57138338)
\curveto(850.52772433,215.57137565)(850.54772431,215.57637565)(850.56772567,215.58638338)
}
}
{
\newrgbcolor{curcolor}{0 0 0}
\pscustom[linestyle=none,fillstyle=solid,fillcolor=curcolor]
{
\newpath
\moveto(862.14233505,215.73638338)
\curveto(862.77232981,215.75637547)(863.27732931,215.67137555)(863.65733505,215.48138338)
\curveto(864.03732855,215.29137593)(864.34232824,215.00637622)(864.57233505,214.62638338)
\curveto(864.63232795,214.5263767)(864.67732791,214.41637681)(864.70733505,214.29638338)
\curveto(864.74732784,214.18637704)(864.7823278,214.07137715)(864.81233505,213.95138338)
\curveto(864.86232772,213.76137746)(864.89232769,213.55637767)(864.90233505,213.33638338)
\curveto(864.91232767,213.11637811)(864.91732767,212.89137833)(864.91733505,212.66138338)
\lineto(864.91733505,211.05638338)
\lineto(864.91733505,208.71638338)
\curveto(864.91732767,208.54638268)(864.91232767,208.37638285)(864.90233505,208.20638338)
\curveto(864.90232768,208.03638319)(864.83732775,207.9263833)(864.70733505,207.87638338)
\curveto(864.65732793,207.85638337)(864.60232798,207.84638338)(864.54233505,207.84638338)
\curveto(864.49232809,207.83638339)(864.43732815,207.83138339)(864.37733505,207.83138338)
\curveto(864.24732834,207.83138339)(864.12232846,207.83638339)(864.00233505,207.84638338)
\curveto(863.8823287,207.84638338)(863.79732879,207.88638334)(863.74733505,207.96638338)
\curveto(863.69732889,208.03638319)(863.67232891,208.1263831)(863.67233505,208.23638338)
\lineto(863.67233505,208.56638338)
\lineto(863.67233505,209.85638338)
\lineto(863.67233505,212.30138338)
\curveto(863.67232891,212.57137865)(863.66732892,212.83637839)(863.65733505,213.09638338)
\curveto(863.64732894,213.36637786)(863.60232898,213.59637763)(863.52233505,213.78638338)
\curveto(863.44232914,213.98637724)(863.32232926,214.14637708)(863.16233505,214.26638338)
\curveto(863.00232958,214.39637683)(862.81732977,214.49637673)(862.60733505,214.56638338)
\curveto(862.54733004,214.58637664)(862.4823301,214.59637663)(862.41233505,214.59638338)
\curveto(862.35233023,214.60637662)(862.29233029,214.6213766)(862.23233505,214.64138338)
\curveto(862.1823304,214.65137657)(862.10233048,214.65137657)(861.99233505,214.64138338)
\curveto(861.89233069,214.64137658)(861.82233076,214.63637659)(861.78233505,214.62638338)
\curveto(861.74233084,214.60637662)(861.70733088,214.59637663)(861.67733505,214.59638338)
\curveto(861.64733094,214.60637662)(861.61233097,214.60637662)(861.57233505,214.59638338)
\curveto(861.44233114,214.56637666)(861.31733127,214.53137669)(861.19733505,214.49138338)
\curveto(861.0873315,214.46137676)(860.9823316,214.41637681)(860.88233505,214.35638338)
\curveto(860.84233174,214.33637689)(860.80733178,214.31637691)(860.77733505,214.29638338)
\curveto(860.74733184,214.27637695)(860.71233187,214.25637697)(860.67233505,214.23638338)
\curveto(860.32233226,213.98637724)(860.06733252,213.61137761)(859.90733505,213.11138338)
\curveto(859.87733271,213.03137819)(859.85733273,212.94637828)(859.84733505,212.85638338)
\curveto(859.83733275,212.77637845)(859.82233276,212.69637853)(859.80233505,212.61638338)
\curveto(859.7823328,212.56637866)(859.77733281,212.51637871)(859.78733505,212.46638338)
\curveto(859.79733279,212.4263788)(859.79233279,212.38637884)(859.77233505,212.34638338)
\lineto(859.77233505,212.03138338)
\curveto(859.76233282,212.00137922)(859.75733283,211.96637926)(859.75733505,211.92638338)
\curveto(859.76733282,211.88637934)(859.77233281,211.84137938)(859.77233505,211.79138338)
\lineto(859.77233505,211.34138338)
\lineto(859.77233505,209.90138338)
\lineto(859.77233505,208.58138338)
\lineto(859.77233505,208.23638338)
\curveto(859.77233281,208.1263831)(859.74733284,208.03638319)(859.69733505,207.96638338)
\curveto(859.64733294,207.88638334)(859.55733303,207.84638338)(859.42733505,207.84638338)
\curveto(859.30733328,207.83638339)(859.1823334,207.83138339)(859.05233505,207.83138338)
\curveto(858.97233361,207.83138339)(858.89733369,207.83638339)(858.82733505,207.84638338)
\curveto(858.75733383,207.85638337)(858.69733389,207.88138334)(858.64733505,207.92138338)
\curveto(858.56733402,207.97138325)(858.52733406,208.06638316)(858.52733505,208.20638338)
\lineto(858.52733505,208.61138338)
\lineto(858.52733505,210.38138338)
\lineto(858.52733505,214.01138338)
\lineto(858.52733505,214.92638338)
\lineto(858.52733505,215.19638338)
\curveto(858.52733406,215.28637594)(858.54733404,215.35637587)(858.58733505,215.40638338)
\curveto(858.61733397,215.46637576)(858.66733392,215.50637572)(858.73733505,215.52638338)
\curveto(858.77733381,215.53637569)(858.83233375,215.54637568)(858.90233505,215.55638338)
\curveto(858.9823336,215.56637566)(859.06233352,215.57137565)(859.14233505,215.57138338)
\curveto(859.22233336,215.57137565)(859.29733329,215.56637566)(859.36733505,215.55638338)
\curveto(859.44733314,215.54637568)(859.50233308,215.53137569)(859.53233505,215.51138338)
\curveto(859.64233294,215.44137578)(859.69233289,215.35137587)(859.68233505,215.24138338)
\curveto(859.67233291,215.14137608)(859.6873329,215.0263762)(859.72733505,214.89638338)
\curveto(859.74733284,214.83637639)(859.7873328,214.78637644)(859.84733505,214.74638338)
\curveto(859.96733262,214.73637649)(860.06233252,214.78137644)(860.13233505,214.88138338)
\curveto(860.21233237,214.98137624)(860.29233229,215.06137616)(860.37233505,215.12138338)
\curveto(860.51233207,215.221376)(860.65233193,215.31137591)(860.79233505,215.39138338)
\curveto(860.94233164,215.48137574)(861.11233147,215.55637567)(861.30233505,215.61638338)
\curveto(861.3823312,215.64637558)(861.46733112,215.66637556)(861.55733505,215.67638338)
\curveto(861.65733093,215.68637554)(861.75233083,215.70137552)(861.84233505,215.72138338)
\curveto(861.89233069,215.73137549)(861.94233064,215.73637549)(861.99233505,215.73638338)
\lineto(862.14233505,215.73638338)
}
}
{
\newrgbcolor{curcolor}{0 0 0}
\pscustom[linestyle=none,fillstyle=solid,fillcolor=curcolor]
{
\newpath
\moveto(867.08694442,217.08638338)
\curveto(867.0069433,217.14637408)(866.96194335,217.25137397)(866.95194442,217.40138338)
\lineto(866.95194442,217.86638338)
\lineto(866.95194442,218.12138338)
\curveto(866.95194336,218.21137301)(866.96694334,218.28637294)(866.99694442,218.34638338)
\curveto(867.03694327,218.4263728)(867.11694319,218.48637274)(867.23694442,218.52638338)
\curveto(867.25694305,218.53637269)(867.27694303,218.53637269)(867.29694442,218.52638338)
\curveto(867.32694298,218.5263727)(867.35194296,218.53137269)(867.37194442,218.54138338)
\curveto(867.54194277,218.54137268)(867.70194261,218.53637269)(867.85194442,218.52638338)
\curveto(868.00194231,218.51637271)(868.10194221,218.45637277)(868.15194442,218.34638338)
\curveto(868.18194213,218.28637294)(868.19694211,218.21137301)(868.19694442,218.12138338)
\lineto(868.19694442,217.86638338)
\curveto(868.19694211,217.68637354)(868.19194212,217.51637371)(868.18194442,217.35638338)
\curveto(868.18194213,217.19637403)(868.11694219,217.09137413)(867.98694442,217.04138338)
\curveto(867.93694237,217.0213742)(867.88194243,217.01137421)(867.82194442,217.01138338)
\lineto(867.65694442,217.01138338)
\lineto(867.34194442,217.01138338)
\curveto(867.24194307,217.01137421)(867.15694315,217.03637419)(867.08694442,217.08638338)
\moveto(868.19694442,208.58138338)
\lineto(868.19694442,208.26638338)
\curveto(868.2069421,208.16638306)(868.18694212,208.08638314)(868.13694442,208.02638338)
\curveto(868.1069422,207.96638326)(868.06194225,207.9263833)(868.00194442,207.90638338)
\curveto(867.94194237,207.89638333)(867.87194244,207.88138334)(867.79194442,207.86138338)
\lineto(867.56694442,207.86138338)
\curveto(867.43694287,207.86138336)(867.32194299,207.86638336)(867.22194442,207.87638338)
\curveto(867.13194318,207.89638333)(867.06194325,207.94638328)(867.01194442,208.02638338)
\curveto(866.97194334,208.08638314)(866.95194336,208.16138306)(866.95194442,208.25138338)
\lineto(866.95194442,208.53638338)
\lineto(866.95194442,214.88138338)
\lineto(866.95194442,215.19638338)
\curveto(866.95194336,215.30637592)(866.97694333,215.39137583)(867.02694442,215.45138338)
\curveto(867.05694325,215.50137572)(867.09694321,215.53137569)(867.14694442,215.54138338)
\curveto(867.19694311,215.55137567)(867.25194306,215.56637566)(867.31194442,215.58638338)
\curveto(867.33194298,215.58637564)(867.35194296,215.58137564)(867.37194442,215.57138338)
\curveto(867.40194291,215.57137565)(867.42694288,215.57637565)(867.44694442,215.58638338)
\curveto(867.57694273,215.58637564)(867.7069426,215.58137564)(867.83694442,215.57138338)
\curveto(867.97694233,215.57137565)(868.07194224,215.53137569)(868.12194442,215.45138338)
\curveto(868.17194214,215.39137583)(868.19694211,215.31137591)(868.19694442,215.21138338)
\lineto(868.19694442,214.92638338)
\lineto(868.19694442,208.58138338)
}
}
{
\newrgbcolor{curcolor}{0 0 0}
\pscustom[linestyle=none,fillstyle=solid,fillcolor=curcolor]
{
\newpath
\moveto(877.10178817,208.67138338)
\lineto(877.10178817,208.28138338)
\curveto(877.1017803,208.16138306)(877.07678032,208.06138316)(877.02678817,207.98138338)
\curveto(876.97678042,207.91138331)(876.89178051,207.87138335)(876.77178817,207.86138338)
\lineto(876.42678817,207.86138338)
\curveto(876.36678103,207.86138336)(876.30678109,207.85638337)(876.24678817,207.84638338)
\curveto(876.1967812,207.84638338)(876.15178125,207.85638337)(876.11178817,207.87638338)
\curveto(876.02178138,207.89638333)(875.96178144,207.93638329)(875.93178817,207.99638338)
\curveto(875.89178151,208.04638318)(875.86678153,208.10638312)(875.85678817,208.17638338)
\curveto(875.85678154,208.24638298)(875.84178156,208.31638291)(875.81178817,208.38638338)
\curveto(875.8017816,208.40638282)(875.78678161,208.4213828)(875.76678817,208.43138338)
\curveto(875.75678164,208.45138277)(875.74178166,208.47138275)(875.72178817,208.49138338)
\curveto(875.62178178,208.50138272)(875.54178186,208.48138274)(875.48178817,208.43138338)
\curveto(875.43178197,208.38138284)(875.37678202,208.33138289)(875.31678817,208.28138338)
\curveto(875.11678228,208.13138309)(874.91678248,208.01638321)(874.71678817,207.93638338)
\curveto(874.53678286,207.85638337)(874.32678307,207.79638343)(874.08678817,207.75638338)
\curveto(873.85678354,207.71638351)(873.61678378,207.69638353)(873.36678817,207.69638338)
\curveto(873.12678427,207.68638354)(872.88678451,207.70138352)(872.64678817,207.74138338)
\curveto(872.40678499,207.77138345)(872.1967852,207.8263834)(872.01678817,207.90638338)
\curveto(871.4967859,208.1263831)(871.07678632,208.4213828)(870.75678817,208.79138338)
\curveto(870.43678696,209.17138205)(870.18678721,209.64138158)(870.00678817,210.20138338)
\curveto(869.96678743,210.29138093)(869.93678746,210.38138084)(869.91678817,210.47138338)
\curveto(869.90678749,210.57138065)(869.88678751,210.67138055)(869.85678817,210.77138338)
\curveto(869.84678755,210.8213804)(869.84178756,210.87138035)(869.84178817,210.92138338)
\curveto(869.84178756,210.97138025)(869.83678756,211.0213802)(869.82678817,211.07138338)
\curveto(869.80678759,211.1213801)(869.7967876,211.17138005)(869.79678817,211.22138338)
\curveto(869.80678759,211.28137994)(869.80678759,211.33637989)(869.79678817,211.38638338)
\lineto(869.79678817,211.53638338)
\curveto(869.77678762,211.58637964)(869.76678763,211.65137957)(869.76678817,211.73138338)
\curveto(869.76678763,211.81137941)(869.77678762,211.87637935)(869.79678817,211.92638338)
\lineto(869.79678817,212.09138338)
\curveto(869.81678758,212.16137906)(869.82178758,212.23137899)(869.81178817,212.30138338)
\curveto(869.81178759,212.38137884)(869.82178758,212.45637877)(869.84178817,212.52638338)
\curveto(869.85178755,212.57637865)(869.85678754,212.6213786)(869.85678817,212.66138338)
\curveto(869.85678754,212.70137852)(869.86178754,212.74637848)(869.87178817,212.79638338)
\curveto(869.9017875,212.89637833)(869.92678747,212.99137823)(869.94678817,213.08138338)
\curveto(869.96678743,213.18137804)(869.99178741,213.27637795)(870.02178817,213.36638338)
\curveto(870.15178725,213.74637748)(870.31678708,214.08637714)(870.51678817,214.38638338)
\curveto(870.72678667,214.69637653)(870.97678642,214.95137627)(871.26678817,215.15138338)
\curveto(871.43678596,215.27137595)(871.61178579,215.37137585)(871.79178817,215.45138338)
\curveto(871.98178542,215.53137569)(872.18678521,215.60137562)(872.40678817,215.66138338)
\curveto(872.47678492,215.67137555)(872.54178486,215.68137554)(872.60178817,215.69138338)
\curveto(872.67178473,215.70137552)(872.74178466,215.71637551)(872.81178817,215.73638338)
\lineto(872.96178817,215.73638338)
\curveto(873.04178436,215.75637547)(873.15678424,215.76637546)(873.30678817,215.76638338)
\curveto(873.46678393,215.76637546)(873.58678381,215.75637547)(873.66678817,215.73638338)
\curveto(873.70678369,215.7263755)(873.76178364,215.7213755)(873.83178817,215.72138338)
\curveto(873.94178346,215.69137553)(874.05178335,215.66637556)(874.16178817,215.64638338)
\curveto(874.27178313,215.63637559)(874.37678302,215.60637562)(874.47678817,215.55638338)
\curveto(874.62678277,215.49637573)(874.76678263,215.43137579)(874.89678817,215.36138338)
\curveto(875.03678236,215.29137593)(875.16678223,215.21137601)(875.28678817,215.12138338)
\curveto(875.34678205,215.07137615)(875.40678199,215.01637621)(875.46678817,214.95638338)
\curveto(875.53678186,214.90637632)(875.62678177,214.89137633)(875.73678817,214.91138338)
\curveto(875.75678164,214.94137628)(875.77178163,214.96637626)(875.78178817,214.98638338)
\curveto(875.8017816,215.00637622)(875.81678158,215.03637619)(875.82678817,215.07638338)
\curveto(875.85678154,215.16637606)(875.86678153,215.28137594)(875.85678817,215.42138338)
\lineto(875.85678817,215.79638338)
\lineto(875.85678817,217.52138338)
\lineto(875.85678817,217.98638338)
\curveto(875.85678154,218.16637306)(875.88178152,218.29637293)(875.93178817,218.37638338)
\curveto(875.97178143,218.44637278)(876.03178137,218.49137273)(876.11178817,218.51138338)
\curveto(876.13178127,218.51137271)(876.15678124,218.51137271)(876.18678817,218.51138338)
\curveto(876.21678118,218.5213727)(876.24178116,218.5263727)(876.26178817,218.52638338)
\curveto(876.401781,218.53637269)(876.54678085,218.53637269)(876.69678817,218.52638338)
\curveto(876.85678054,218.5263727)(876.96678043,218.48637274)(877.02678817,218.40638338)
\curveto(877.07678032,218.3263729)(877.1017803,218.226373)(877.10178817,218.10638338)
\lineto(877.10178817,217.73138338)
\lineto(877.10178817,208.67138338)
\moveto(875.88678817,211.50638338)
\curveto(875.90678149,211.55637967)(875.91678148,211.6213796)(875.91678817,211.70138338)
\curveto(875.91678148,211.79137943)(875.90678149,211.86137936)(875.88678817,211.91138338)
\lineto(875.88678817,212.13638338)
\curveto(875.86678153,212.226379)(875.85178155,212.31637891)(875.84178817,212.40638338)
\curveto(875.83178157,212.50637872)(875.81178159,212.59637863)(875.78178817,212.67638338)
\curveto(875.76178164,212.75637847)(875.74178166,212.83137839)(875.72178817,212.90138338)
\curveto(875.71178169,212.97137825)(875.69178171,213.04137818)(875.66178817,213.11138338)
\curveto(875.54178186,213.41137781)(875.38678201,213.67637755)(875.19678817,213.90638338)
\curveto(875.00678239,214.13637709)(874.76678263,214.31637691)(874.47678817,214.44638338)
\curveto(874.37678302,214.49637673)(874.27178313,214.53137669)(874.16178817,214.55138338)
\curveto(874.06178334,214.58137664)(873.95178345,214.60637662)(873.83178817,214.62638338)
\curveto(873.75178365,214.64637658)(873.66178374,214.65637657)(873.56178817,214.65638338)
\lineto(873.29178817,214.65638338)
\curveto(873.24178416,214.64637658)(873.1967842,214.63637659)(873.15678817,214.62638338)
\lineto(873.02178817,214.62638338)
\curveto(872.94178446,214.60637662)(872.85678454,214.58637664)(872.76678817,214.56638338)
\curveto(872.68678471,214.54637668)(872.60678479,214.5213767)(872.52678817,214.49138338)
\curveto(872.20678519,214.35137687)(871.94678545,214.14637708)(871.74678817,213.87638338)
\curveto(871.55678584,213.61637761)(871.401786,213.31137791)(871.28178817,212.96138338)
\curveto(871.24178616,212.85137837)(871.21178619,212.73637849)(871.19178817,212.61638338)
\curveto(871.18178622,212.50637872)(871.16678623,212.39637883)(871.14678817,212.28638338)
\curveto(871.14678625,212.24637898)(871.14178626,212.20637902)(871.13178817,212.16638338)
\lineto(871.13178817,212.06138338)
\curveto(871.11178629,212.01137921)(871.1017863,211.95637927)(871.10178817,211.89638338)
\curveto(871.11178629,211.83637939)(871.11678628,211.78137944)(871.11678817,211.73138338)
\lineto(871.11678817,211.40138338)
\curveto(871.11678628,211.30137992)(871.12678627,211.20638002)(871.14678817,211.11638338)
\curveto(871.15678624,211.08638014)(871.16178624,211.03638019)(871.16178817,210.96638338)
\curveto(871.18178622,210.89638033)(871.1967862,210.8263804)(871.20678817,210.75638338)
\lineto(871.26678817,210.54638338)
\curveto(871.37678602,210.19638103)(871.52678587,209.89638133)(871.71678817,209.64638338)
\curveto(871.90678549,209.39638183)(872.14678525,209.19138203)(872.43678817,209.03138338)
\curveto(872.52678487,208.98138224)(872.61678478,208.94138228)(872.70678817,208.91138338)
\curveto(872.7967846,208.88138234)(872.8967845,208.85138237)(873.00678817,208.82138338)
\curveto(873.05678434,208.80138242)(873.10678429,208.79638243)(873.15678817,208.80638338)
\curveto(873.21678418,208.81638241)(873.27178413,208.81138241)(873.32178817,208.79138338)
\curveto(873.36178404,208.78138244)(873.401784,208.77638245)(873.44178817,208.77638338)
\lineto(873.57678817,208.77638338)
\lineto(873.71178817,208.77638338)
\curveto(873.74178366,208.78638244)(873.79178361,208.79138243)(873.86178817,208.79138338)
\curveto(873.94178346,208.81138241)(874.02178338,208.8263824)(874.10178817,208.83638338)
\curveto(874.18178322,208.85638237)(874.25678314,208.88138234)(874.32678817,208.91138338)
\curveto(874.65678274,209.05138217)(874.92178248,209.226382)(875.12178817,209.43638338)
\curveto(875.33178207,209.65638157)(875.50678189,209.93138129)(875.64678817,210.26138338)
\curveto(875.6967817,210.37138085)(875.73178167,210.48138074)(875.75178817,210.59138338)
\curveto(875.77178163,210.70138052)(875.7967816,210.81138041)(875.82678817,210.92138338)
\curveto(875.84678155,210.96138026)(875.85678154,210.99638023)(875.85678817,211.02638338)
\curveto(875.85678154,211.06638016)(875.86178154,211.10638012)(875.87178817,211.14638338)
\curveto(875.88178152,211.20638002)(875.88178152,211.26637996)(875.87178817,211.32638338)
\curveto(875.87178153,211.38637984)(875.87678152,211.44637978)(875.88678817,211.50638338)
}
}
{
\newrgbcolor{curcolor}{0 0 0}
\pscustom[linestyle=none,fillstyle=solid,fillcolor=curcolor]
{
\newpath
\moveto(885.93303817,208.41638338)
\curveto(885.96303034,208.25638297)(885.94803036,208.1213831)(885.88803817,208.01138338)
\curveto(885.82803048,207.91138331)(885.74803056,207.83638339)(885.64803817,207.78638338)
\curveto(885.59803071,207.76638346)(885.54303076,207.75638347)(885.48303817,207.75638338)
\curveto(885.43303087,207.75638347)(885.37803093,207.74638348)(885.31803817,207.72638338)
\curveto(885.09803121,207.67638355)(884.87803143,207.69138353)(884.65803817,207.77138338)
\curveto(884.44803186,207.84138338)(884.303032,207.93138329)(884.22303817,208.04138338)
\curveto(884.17303213,208.11138311)(884.12803218,208.19138303)(884.08803817,208.28138338)
\curveto(884.04803226,208.38138284)(883.99803231,208.46138276)(883.93803817,208.52138338)
\curveto(883.91803239,208.54138268)(883.89303241,208.56138266)(883.86303817,208.58138338)
\curveto(883.84303246,208.60138262)(883.81303249,208.60638262)(883.77303817,208.59638338)
\curveto(883.66303264,208.56638266)(883.55803275,208.51138271)(883.45803817,208.43138338)
\curveto(883.36803294,208.35138287)(883.27803303,208.28138294)(883.18803817,208.22138338)
\curveto(883.05803325,208.14138308)(882.91803339,208.06638316)(882.76803817,207.99638338)
\curveto(882.61803369,207.93638329)(882.45803385,207.88138334)(882.28803817,207.83138338)
\curveto(882.18803412,207.80138342)(882.07803423,207.78138344)(881.95803817,207.77138338)
\curveto(881.84803446,207.76138346)(881.73803457,207.74638348)(881.62803817,207.72638338)
\curveto(881.57803473,207.71638351)(881.53303477,207.71138351)(881.49303817,207.71138338)
\lineto(881.38803817,207.71138338)
\curveto(881.27803503,207.69138353)(881.17303513,207.69138353)(881.07303817,207.71138338)
\lineto(880.93803817,207.71138338)
\curveto(880.88803542,207.7213835)(880.83803547,207.7263835)(880.78803817,207.72638338)
\curveto(880.73803557,207.7263835)(880.69303561,207.73638349)(880.65303817,207.75638338)
\curveto(880.61303569,207.76638346)(880.57803573,207.77138345)(880.54803817,207.77138338)
\curveto(880.52803578,207.76138346)(880.5030358,207.76138346)(880.47303817,207.77138338)
\lineto(880.23303817,207.83138338)
\curveto(880.15303615,207.84138338)(880.07803623,207.86138336)(880.00803817,207.89138338)
\curveto(879.7080366,208.0213832)(879.46303684,208.16638306)(879.27303817,208.32638338)
\curveto(879.09303721,208.49638273)(878.94303736,208.73138249)(878.82303817,209.03138338)
\curveto(878.73303757,209.25138197)(878.68803762,209.51638171)(878.68803817,209.82638338)
\lineto(878.68803817,210.14138338)
\curveto(878.69803761,210.19138103)(878.7030376,210.24138098)(878.70303817,210.29138338)
\lineto(878.73303817,210.47138338)
\lineto(878.85303817,210.80138338)
\curveto(878.89303741,210.91138031)(878.94303736,211.01138021)(879.00303817,211.10138338)
\curveto(879.18303712,211.39137983)(879.42803688,211.60637962)(879.73803817,211.74638338)
\curveto(880.04803626,211.88637934)(880.38803592,212.01137921)(880.75803817,212.12138338)
\curveto(880.89803541,212.16137906)(881.04303526,212.19137903)(881.19303817,212.21138338)
\curveto(881.34303496,212.23137899)(881.49303481,212.25637897)(881.64303817,212.28638338)
\curveto(881.71303459,212.30637892)(881.77803453,212.31637891)(881.83803817,212.31638338)
\curveto(881.9080344,212.31637891)(881.98303432,212.3263789)(882.06303817,212.34638338)
\curveto(882.13303417,212.36637886)(882.2030341,212.37637885)(882.27303817,212.37638338)
\curveto(882.34303396,212.38637884)(882.41803389,212.40137882)(882.49803817,212.42138338)
\curveto(882.74803356,212.48137874)(882.98303332,212.53137869)(883.20303817,212.57138338)
\curveto(883.42303288,212.6213786)(883.59803271,212.73637849)(883.72803817,212.91638338)
\curveto(883.78803252,212.99637823)(883.83803247,213.09637813)(883.87803817,213.21638338)
\curveto(883.91803239,213.34637788)(883.91803239,213.48637774)(883.87803817,213.63638338)
\curveto(883.81803249,213.87637735)(883.72803258,214.06637716)(883.60803817,214.20638338)
\curveto(883.49803281,214.34637688)(883.33803297,214.45637677)(883.12803817,214.53638338)
\curveto(883.0080333,214.58637664)(882.86303344,214.6213766)(882.69303817,214.64138338)
\curveto(882.53303377,214.66137656)(882.36303394,214.67137655)(882.18303817,214.67138338)
\curveto(882.0030343,214.67137655)(881.82803448,214.66137656)(881.65803817,214.64138338)
\curveto(881.48803482,214.6213766)(881.34303496,214.59137663)(881.22303817,214.55138338)
\curveto(881.05303525,214.49137673)(880.88803542,214.40637682)(880.72803817,214.29638338)
\curveto(880.64803566,214.23637699)(880.57303573,214.15637707)(880.50303817,214.05638338)
\curveto(880.44303586,213.96637726)(880.38803592,213.86637736)(880.33803817,213.75638338)
\curveto(880.308036,213.67637755)(880.27803603,213.59137763)(880.24803817,213.50138338)
\curveto(880.22803608,213.41137781)(880.18303612,213.34137788)(880.11303817,213.29138338)
\curveto(880.07303623,213.26137796)(880.0030363,213.23637799)(879.90303817,213.21638338)
\curveto(879.81303649,213.20637802)(879.71803659,213.20137802)(879.61803817,213.20138338)
\curveto(879.51803679,213.20137802)(879.41803689,213.20637802)(879.31803817,213.21638338)
\curveto(879.22803708,213.23637799)(879.16303714,213.26137796)(879.12303817,213.29138338)
\curveto(879.08303722,213.3213779)(879.05303725,213.37137785)(879.03303817,213.44138338)
\curveto(879.01303729,213.51137771)(879.01303729,213.58637764)(879.03303817,213.66638338)
\curveto(879.06303724,213.79637743)(879.09303721,213.91637731)(879.12303817,214.02638338)
\curveto(879.16303714,214.14637708)(879.2080371,214.26137696)(879.25803817,214.37138338)
\curveto(879.44803686,214.7213765)(879.68803662,214.99137623)(879.97803817,215.18138338)
\curveto(880.26803604,215.38137584)(880.62803568,215.54137568)(881.05803817,215.66138338)
\curveto(881.15803515,215.68137554)(881.25803505,215.69637553)(881.35803817,215.70638338)
\curveto(881.46803484,215.71637551)(881.57803473,215.73137549)(881.68803817,215.75138338)
\curveto(881.72803458,215.76137546)(881.79303451,215.76137546)(881.88303817,215.75138338)
\curveto(881.97303433,215.75137547)(882.02803428,215.76137546)(882.04803817,215.78138338)
\curveto(882.74803356,215.79137543)(883.35803295,215.71137551)(883.87803817,215.54138338)
\curveto(884.39803191,215.37137585)(884.76303154,215.04637618)(884.97303817,214.56638338)
\curveto(885.06303124,214.36637686)(885.11303119,214.13137709)(885.12303817,213.86138338)
\curveto(885.14303116,213.60137762)(885.15303115,213.3263779)(885.15303817,213.03638338)
\lineto(885.15303817,209.72138338)
\curveto(885.15303115,209.58138164)(885.15803115,209.44638178)(885.16803817,209.31638338)
\curveto(885.17803113,209.18638204)(885.2080311,209.08138214)(885.25803817,209.00138338)
\curveto(885.308031,208.93138229)(885.37303093,208.88138234)(885.45303817,208.85138338)
\curveto(885.54303076,208.81138241)(885.62803068,208.78138244)(885.70803817,208.76138338)
\curveto(885.78803052,208.75138247)(885.84803046,208.70638252)(885.88803817,208.62638338)
\curveto(885.9080304,208.59638263)(885.91803039,208.56638266)(885.91803817,208.53638338)
\curveto(885.91803039,208.50638272)(885.92303038,208.46638276)(885.93303817,208.41638338)
\moveto(883.78803817,210.08138338)
\curveto(883.84803246,210.221381)(883.87803243,210.38138084)(883.87803817,210.56138338)
\curveto(883.88803242,210.75138047)(883.89303241,210.94638028)(883.89303817,211.14638338)
\curveto(883.89303241,211.25637997)(883.88803242,211.35637987)(883.87803817,211.44638338)
\curveto(883.86803244,211.53637969)(883.82803248,211.60637962)(883.75803817,211.65638338)
\curveto(883.72803258,211.67637955)(883.65803265,211.68637954)(883.54803817,211.68638338)
\curveto(883.52803278,211.66637956)(883.49303281,211.65637957)(883.44303817,211.65638338)
\curveto(883.39303291,211.65637957)(883.34803296,211.64637958)(883.30803817,211.62638338)
\curveto(883.22803308,211.60637962)(883.13803317,211.58637964)(883.03803817,211.56638338)
\lineto(882.73803817,211.50638338)
\curveto(882.7080336,211.50637972)(882.67303363,211.50137972)(882.63303817,211.49138338)
\lineto(882.52803817,211.49138338)
\curveto(882.37803393,211.45137977)(882.21303409,211.4263798)(882.03303817,211.41638338)
\curveto(881.86303444,211.41637981)(881.7030346,211.39637983)(881.55303817,211.35638338)
\curveto(881.47303483,211.33637989)(881.39803491,211.31637991)(881.32803817,211.29638338)
\curveto(881.26803504,211.28637994)(881.19803511,211.27137995)(881.11803817,211.25138338)
\curveto(880.95803535,211.20138002)(880.8080355,211.13638009)(880.66803817,211.05638338)
\curveto(880.52803578,210.98638024)(880.4080359,210.89638033)(880.30803817,210.78638338)
\curveto(880.2080361,210.67638055)(880.13303617,210.54138068)(880.08303817,210.38138338)
\curveto(880.03303627,210.23138099)(880.01303629,210.04638118)(880.02303817,209.82638338)
\curveto(880.02303628,209.7263815)(880.03803627,209.63138159)(880.06803817,209.54138338)
\curveto(880.1080362,209.46138176)(880.15303615,209.38638184)(880.20303817,209.31638338)
\curveto(880.28303602,209.20638202)(880.38803592,209.11138211)(880.51803817,209.03138338)
\curveto(880.64803566,208.96138226)(880.78803552,208.90138232)(880.93803817,208.85138338)
\curveto(880.98803532,208.84138238)(881.03803527,208.83638239)(881.08803817,208.83638338)
\curveto(881.13803517,208.83638239)(881.18803512,208.83138239)(881.23803817,208.82138338)
\curveto(881.308035,208.80138242)(881.39303491,208.78638244)(881.49303817,208.77638338)
\curveto(881.6030347,208.77638245)(881.69303461,208.78638244)(881.76303817,208.80638338)
\curveto(881.82303448,208.8263824)(881.88303442,208.83138239)(881.94303817,208.82138338)
\curveto(882.0030343,208.8213824)(882.06303424,208.83138239)(882.12303817,208.85138338)
\curveto(882.2030341,208.87138235)(882.27803403,208.88638234)(882.34803817,208.89638338)
\curveto(882.42803388,208.90638232)(882.5030338,208.9263823)(882.57303817,208.95638338)
\curveto(882.86303344,209.07638215)(883.1080332,209.221382)(883.30803817,209.39138338)
\curveto(883.51803279,209.56138166)(883.67803263,209.79138143)(883.78803817,210.08138338)
}
}
{
\newrgbcolor{curcolor}{0 0 0}
\pscustom[linestyle=none,fillstyle=solid,fillcolor=curcolor]
{
\newpath
\moveto(894.0646788,208.67138338)
\lineto(894.0646788,208.28138338)
\curveto(894.06467092,208.16138306)(894.03967095,208.06138316)(893.9896788,207.98138338)
\curveto(893.93967105,207.91138331)(893.85467113,207.87138335)(893.7346788,207.86138338)
\lineto(893.3896788,207.86138338)
\curveto(893.32967166,207.86138336)(893.26967172,207.85638337)(893.2096788,207.84638338)
\curveto(893.15967183,207.84638338)(893.11467187,207.85638337)(893.0746788,207.87638338)
\curveto(892.984672,207.89638333)(892.92467206,207.93638329)(892.8946788,207.99638338)
\curveto(892.85467213,208.04638318)(892.82967216,208.10638312)(892.8196788,208.17638338)
\curveto(892.81967217,208.24638298)(892.80467218,208.31638291)(892.7746788,208.38638338)
\curveto(892.76467222,208.40638282)(892.74967224,208.4213828)(892.7296788,208.43138338)
\curveto(892.71967227,208.45138277)(892.70467228,208.47138275)(892.6846788,208.49138338)
\curveto(892.5846724,208.50138272)(892.50467248,208.48138274)(892.4446788,208.43138338)
\curveto(892.39467259,208.38138284)(892.33967265,208.33138289)(892.2796788,208.28138338)
\curveto(892.07967291,208.13138309)(891.87967311,208.01638321)(891.6796788,207.93638338)
\curveto(891.49967349,207.85638337)(891.2896737,207.79638343)(891.0496788,207.75638338)
\curveto(890.81967417,207.71638351)(890.57967441,207.69638353)(890.3296788,207.69638338)
\curveto(890.0896749,207.68638354)(889.84967514,207.70138352)(889.6096788,207.74138338)
\curveto(889.36967562,207.77138345)(889.15967583,207.8263834)(888.9796788,207.90638338)
\curveto(888.45967653,208.1263831)(888.03967695,208.4213828)(887.7196788,208.79138338)
\curveto(887.39967759,209.17138205)(887.14967784,209.64138158)(886.9696788,210.20138338)
\curveto(886.92967806,210.29138093)(886.89967809,210.38138084)(886.8796788,210.47138338)
\curveto(886.86967812,210.57138065)(886.84967814,210.67138055)(886.8196788,210.77138338)
\curveto(886.80967818,210.8213804)(886.80467818,210.87138035)(886.8046788,210.92138338)
\curveto(886.80467818,210.97138025)(886.79967819,211.0213802)(886.7896788,211.07138338)
\curveto(886.76967822,211.1213801)(886.75967823,211.17138005)(886.7596788,211.22138338)
\curveto(886.76967822,211.28137994)(886.76967822,211.33637989)(886.7596788,211.38638338)
\lineto(886.7596788,211.53638338)
\curveto(886.73967825,211.58637964)(886.72967826,211.65137957)(886.7296788,211.73138338)
\curveto(886.72967826,211.81137941)(886.73967825,211.87637935)(886.7596788,211.92638338)
\lineto(886.7596788,212.09138338)
\curveto(886.77967821,212.16137906)(886.7846782,212.23137899)(886.7746788,212.30138338)
\curveto(886.77467821,212.38137884)(886.7846782,212.45637877)(886.8046788,212.52638338)
\curveto(886.81467817,212.57637865)(886.81967817,212.6213786)(886.8196788,212.66138338)
\curveto(886.81967817,212.70137852)(886.82467816,212.74637848)(886.8346788,212.79638338)
\curveto(886.86467812,212.89637833)(886.8896781,212.99137823)(886.9096788,213.08138338)
\curveto(886.92967806,213.18137804)(886.95467803,213.27637795)(886.9846788,213.36638338)
\curveto(887.11467787,213.74637748)(887.27967771,214.08637714)(887.4796788,214.38638338)
\curveto(887.6896773,214.69637653)(887.93967705,214.95137627)(888.2296788,215.15138338)
\curveto(888.39967659,215.27137595)(888.57467641,215.37137585)(888.7546788,215.45138338)
\curveto(888.94467604,215.53137569)(889.14967584,215.60137562)(889.3696788,215.66138338)
\curveto(889.43967555,215.67137555)(889.50467548,215.68137554)(889.5646788,215.69138338)
\curveto(889.63467535,215.70137552)(889.70467528,215.71637551)(889.7746788,215.73638338)
\lineto(889.9246788,215.73638338)
\curveto(890.00467498,215.75637547)(890.11967487,215.76637546)(890.2696788,215.76638338)
\curveto(890.42967456,215.76637546)(890.54967444,215.75637547)(890.6296788,215.73638338)
\curveto(890.66967432,215.7263755)(890.72467426,215.7213755)(890.7946788,215.72138338)
\curveto(890.90467408,215.69137553)(891.01467397,215.66637556)(891.1246788,215.64638338)
\curveto(891.23467375,215.63637559)(891.33967365,215.60637562)(891.4396788,215.55638338)
\curveto(891.5896734,215.49637573)(891.72967326,215.43137579)(891.8596788,215.36138338)
\curveto(891.99967299,215.29137593)(892.12967286,215.21137601)(892.2496788,215.12138338)
\curveto(892.30967268,215.07137615)(892.36967262,215.01637621)(892.4296788,214.95638338)
\curveto(892.49967249,214.90637632)(892.5896724,214.89137633)(892.6996788,214.91138338)
\curveto(892.71967227,214.94137628)(892.73467225,214.96637626)(892.7446788,214.98638338)
\curveto(892.76467222,215.00637622)(892.77967221,215.03637619)(892.7896788,215.07638338)
\curveto(892.81967217,215.16637606)(892.82967216,215.28137594)(892.8196788,215.42138338)
\lineto(892.8196788,215.79638338)
\lineto(892.8196788,217.52138338)
\lineto(892.8196788,217.98638338)
\curveto(892.81967217,218.16637306)(892.84467214,218.29637293)(892.8946788,218.37638338)
\curveto(892.93467205,218.44637278)(892.99467199,218.49137273)(893.0746788,218.51138338)
\curveto(893.09467189,218.51137271)(893.11967187,218.51137271)(893.1496788,218.51138338)
\curveto(893.17967181,218.5213727)(893.20467178,218.5263727)(893.2246788,218.52638338)
\curveto(893.36467162,218.53637269)(893.50967148,218.53637269)(893.6596788,218.52638338)
\curveto(893.81967117,218.5263727)(893.92967106,218.48637274)(893.9896788,218.40638338)
\curveto(894.03967095,218.3263729)(894.06467092,218.226373)(894.0646788,218.10638338)
\lineto(894.0646788,217.73138338)
\lineto(894.0646788,208.67138338)
\moveto(892.8496788,211.50638338)
\curveto(892.86967212,211.55637967)(892.87967211,211.6213796)(892.8796788,211.70138338)
\curveto(892.87967211,211.79137943)(892.86967212,211.86137936)(892.8496788,211.91138338)
\lineto(892.8496788,212.13638338)
\curveto(892.82967216,212.226379)(892.81467217,212.31637891)(892.8046788,212.40638338)
\curveto(892.79467219,212.50637872)(892.77467221,212.59637863)(892.7446788,212.67638338)
\curveto(892.72467226,212.75637847)(892.70467228,212.83137839)(892.6846788,212.90138338)
\curveto(892.67467231,212.97137825)(892.65467233,213.04137818)(892.6246788,213.11138338)
\curveto(892.50467248,213.41137781)(892.34967264,213.67637755)(892.1596788,213.90638338)
\curveto(891.96967302,214.13637709)(891.72967326,214.31637691)(891.4396788,214.44638338)
\curveto(891.33967365,214.49637673)(891.23467375,214.53137669)(891.1246788,214.55138338)
\curveto(891.02467396,214.58137664)(890.91467407,214.60637662)(890.7946788,214.62638338)
\curveto(890.71467427,214.64637658)(890.62467436,214.65637657)(890.5246788,214.65638338)
\lineto(890.2546788,214.65638338)
\curveto(890.20467478,214.64637658)(890.15967483,214.63637659)(890.1196788,214.62638338)
\lineto(889.9846788,214.62638338)
\curveto(889.90467508,214.60637662)(889.81967517,214.58637664)(889.7296788,214.56638338)
\curveto(889.64967534,214.54637668)(889.56967542,214.5213767)(889.4896788,214.49138338)
\curveto(889.16967582,214.35137687)(888.90967608,214.14637708)(888.7096788,213.87638338)
\curveto(888.51967647,213.61637761)(888.36467662,213.31137791)(888.2446788,212.96138338)
\curveto(888.20467678,212.85137837)(888.17467681,212.73637849)(888.1546788,212.61638338)
\curveto(888.14467684,212.50637872)(888.12967686,212.39637883)(888.1096788,212.28638338)
\curveto(888.10967688,212.24637898)(888.10467688,212.20637902)(888.0946788,212.16638338)
\lineto(888.0946788,212.06138338)
\curveto(888.07467691,212.01137921)(888.06467692,211.95637927)(888.0646788,211.89638338)
\curveto(888.07467691,211.83637939)(888.07967691,211.78137944)(888.0796788,211.73138338)
\lineto(888.0796788,211.40138338)
\curveto(888.07967691,211.30137992)(888.0896769,211.20638002)(888.1096788,211.11638338)
\curveto(888.11967687,211.08638014)(888.12467686,211.03638019)(888.1246788,210.96638338)
\curveto(888.14467684,210.89638033)(888.15967683,210.8263804)(888.1696788,210.75638338)
\lineto(888.2296788,210.54638338)
\curveto(888.33967665,210.19638103)(888.4896765,209.89638133)(888.6796788,209.64638338)
\curveto(888.86967612,209.39638183)(889.10967588,209.19138203)(889.3996788,209.03138338)
\curveto(889.4896755,208.98138224)(889.57967541,208.94138228)(889.6696788,208.91138338)
\curveto(889.75967523,208.88138234)(889.85967513,208.85138237)(889.9696788,208.82138338)
\curveto(890.01967497,208.80138242)(890.06967492,208.79638243)(890.1196788,208.80638338)
\curveto(890.17967481,208.81638241)(890.23467475,208.81138241)(890.2846788,208.79138338)
\curveto(890.32467466,208.78138244)(890.36467462,208.77638245)(890.4046788,208.77638338)
\lineto(890.5396788,208.77638338)
\lineto(890.6746788,208.77638338)
\curveto(890.70467428,208.78638244)(890.75467423,208.79138243)(890.8246788,208.79138338)
\curveto(890.90467408,208.81138241)(890.984674,208.8263824)(891.0646788,208.83638338)
\curveto(891.14467384,208.85638237)(891.21967377,208.88138234)(891.2896788,208.91138338)
\curveto(891.61967337,209.05138217)(891.8846731,209.226382)(892.0846788,209.43638338)
\curveto(892.29467269,209.65638157)(892.46967252,209.93138129)(892.6096788,210.26138338)
\curveto(892.65967233,210.37138085)(892.69467229,210.48138074)(892.7146788,210.59138338)
\curveto(892.73467225,210.70138052)(892.75967223,210.81138041)(892.7896788,210.92138338)
\curveto(892.80967218,210.96138026)(892.81967217,210.99638023)(892.8196788,211.02638338)
\curveto(892.81967217,211.06638016)(892.82467216,211.10638012)(892.8346788,211.14638338)
\curveto(892.84467214,211.20638002)(892.84467214,211.26637996)(892.8346788,211.32638338)
\curveto(892.83467215,211.38637984)(892.83967215,211.44637978)(892.8496788,211.50638338)
}
}
{
\newrgbcolor{curcolor}{0 0 0}
\pscustom[linestyle=none,fillstyle=solid,fillcolor=curcolor]
{
\newpath
\moveto(902.7609288,212.03138338)
\curveto(902.78092111,211.93137929)(902.78092111,211.81637941)(902.7609288,211.68638338)
\curveto(902.75092114,211.56637966)(902.72092117,211.48137974)(902.6709288,211.43138338)
\curveto(902.62092127,211.39137983)(902.54592135,211.36137986)(902.4459288,211.34138338)
\curveto(902.35592154,211.33137989)(902.25092164,211.3263799)(902.1309288,211.32638338)
\lineto(901.7709288,211.32638338)
\curveto(901.65092224,211.33637989)(901.54592235,211.34137988)(901.4559288,211.34138338)
\lineto(897.6159288,211.34138338)
\curveto(897.53592636,211.34137988)(897.45592644,211.33637989)(897.3759288,211.32638338)
\curveto(897.2959266,211.3263799)(897.23092666,211.31137991)(897.1809288,211.28138338)
\curveto(897.14092675,211.26137996)(897.10092679,211.22138)(897.0609288,211.16138338)
\curveto(897.04092685,211.13138009)(897.02092687,211.08638014)(897.0009288,211.02638338)
\curveto(896.98092691,210.97638025)(896.98092691,210.9263803)(897.0009288,210.87638338)
\curveto(897.01092688,210.8263804)(897.01592688,210.78138044)(897.0159288,210.74138338)
\curveto(897.01592688,210.70138052)(897.02092687,210.66138056)(897.0309288,210.62138338)
\curveto(897.05092684,210.54138068)(897.07092682,210.45638077)(897.0909288,210.36638338)
\curveto(897.11092678,210.28638094)(897.14092675,210.20638102)(897.1809288,210.12638338)
\curveto(897.41092648,209.58638164)(897.7909261,209.20138202)(898.3209288,208.97138338)
\curveto(898.38092551,208.94138228)(898.44592545,208.91638231)(898.5159288,208.89638338)
\lineto(898.7259288,208.83638338)
\curveto(898.75592514,208.8263824)(898.80592509,208.8213824)(898.8759288,208.82138338)
\curveto(899.01592488,208.78138244)(899.20092469,208.76138246)(899.4309288,208.76138338)
\curveto(899.66092423,208.76138246)(899.84592405,208.78138244)(899.9859288,208.82138338)
\curveto(900.12592377,208.86138236)(900.25092364,208.90138232)(900.3609288,208.94138338)
\curveto(900.48092341,208.99138223)(900.5909233,209.05138217)(900.6909288,209.12138338)
\curveto(900.80092309,209.19138203)(900.895923,209.27138195)(900.9759288,209.36138338)
\curveto(901.05592284,209.46138176)(901.12592277,209.56638166)(901.1859288,209.67638338)
\curveto(901.24592265,209.77638145)(901.2959226,209.88138134)(901.3359288,209.99138338)
\curveto(901.38592251,210.10138112)(901.46592243,210.18138104)(901.5759288,210.23138338)
\curveto(901.61592228,210.25138097)(901.68092221,210.26638096)(901.7709288,210.27638338)
\curveto(901.86092203,210.28638094)(901.95092194,210.28638094)(902.0409288,210.27638338)
\curveto(902.13092176,210.27638095)(902.21592168,210.27138095)(902.2959288,210.26138338)
\curveto(902.37592152,210.25138097)(902.43092146,210.23138099)(902.4609288,210.20138338)
\curveto(902.56092133,210.13138109)(902.58592131,210.01638121)(902.5359288,209.85638338)
\curveto(902.45592144,209.58638164)(902.35092154,209.34638188)(902.2209288,209.13638338)
\curveto(902.02092187,208.81638241)(901.7909221,208.55138267)(901.5309288,208.34138338)
\curveto(901.28092261,208.14138308)(900.96092293,207.97638325)(900.5709288,207.84638338)
\curveto(900.47092342,207.80638342)(900.37092352,207.78138344)(900.2709288,207.77138338)
\curveto(900.17092372,207.75138347)(900.06592383,207.73138349)(899.9559288,207.71138338)
\curveto(899.90592399,207.70138352)(899.85592404,207.69638353)(899.8059288,207.69638338)
\curveto(899.76592413,207.69638353)(899.72092417,207.69138353)(899.6709288,207.68138338)
\lineto(899.5209288,207.68138338)
\curveto(899.47092442,207.67138355)(899.41092448,207.66638356)(899.3409288,207.66638338)
\curveto(899.28092461,207.66638356)(899.23092466,207.67138355)(899.1909288,207.68138338)
\lineto(899.0559288,207.68138338)
\curveto(899.00592489,207.69138353)(898.96092493,207.69638353)(898.9209288,207.69638338)
\curveto(898.88092501,207.69638353)(898.84092505,207.70138352)(898.8009288,207.71138338)
\curveto(898.75092514,207.7213835)(898.6959252,207.73138349)(898.6359288,207.74138338)
\curveto(898.57592532,207.74138348)(898.52092537,207.74638348)(898.4709288,207.75638338)
\curveto(898.38092551,207.77638345)(898.2909256,207.80138342)(898.2009288,207.83138338)
\curveto(898.11092578,207.85138337)(898.02592587,207.87638335)(897.9459288,207.90638338)
\curveto(897.90592599,207.9263833)(897.87092602,207.93638329)(897.8409288,207.93638338)
\curveto(897.81092608,207.94638328)(897.77592612,207.96138326)(897.7359288,207.98138338)
\curveto(897.58592631,208.05138317)(897.42592647,208.13638309)(897.2559288,208.23638338)
\curveto(896.96592693,208.4263828)(896.71592718,208.65638257)(896.5059288,208.92638338)
\curveto(896.30592759,209.20638202)(896.13592776,209.51638171)(895.9959288,209.85638338)
\curveto(895.94592795,209.96638126)(895.90592799,210.08138114)(895.8759288,210.20138338)
\curveto(895.85592804,210.3213809)(895.82592807,210.44138078)(895.7859288,210.56138338)
\curveto(895.77592812,210.60138062)(895.77092812,210.63638059)(895.7709288,210.66638338)
\curveto(895.77092812,210.69638053)(895.76592813,210.73638049)(895.7559288,210.78638338)
\curveto(895.73592816,210.86638036)(895.72092817,210.95138027)(895.7109288,211.04138338)
\curveto(895.70092819,211.13138009)(895.68592821,211.22138)(895.6659288,211.31138338)
\lineto(895.6659288,211.52138338)
\curveto(895.65592824,211.56137966)(895.64592825,211.61637961)(895.6359288,211.68638338)
\curveto(895.63592826,211.76637946)(895.64092825,211.83137939)(895.6509288,211.88138338)
\lineto(895.6509288,212.04638338)
\curveto(895.67092822,212.09637913)(895.67592822,212.14637908)(895.6659288,212.19638338)
\curveto(895.66592823,212.25637897)(895.67092822,212.31137891)(895.6809288,212.36138338)
\curveto(895.72092817,212.5213787)(895.75092814,212.68137854)(895.7709288,212.84138338)
\curveto(895.80092809,213.00137822)(895.84592805,213.15137807)(895.9059288,213.29138338)
\curveto(895.95592794,213.40137782)(896.00092789,213.51137771)(896.0409288,213.62138338)
\curveto(896.0909278,213.74137748)(896.14592775,213.85637737)(896.2059288,213.96638338)
\curveto(896.42592747,214.31637691)(896.67592722,214.61637661)(896.9559288,214.86638338)
\curveto(897.23592666,215.1263761)(897.58092631,215.34137588)(897.9909288,215.51138338)
\curveto(898.11092578,215.56137566)(898.23092566,215.59637563)(898.3509288,215.61638338)
\curveto(898.48092541,215.64637558)(898.61592528,215.67637555)(898.7559288,215.70638338)
\curveto(898.80592509,215.71637551)(898.85092504,215.7213755)(898.8909288,215.72138338)
\curveto(898.93092496,215.73137549)(898.97592492,215.73637549)(899.0259288,215.73638338)
\curveto(899.04592485,215.74637548)(899.07092482,215.74637548)(899.1009288,215.73638338)
\curveto(899.13092476,215.7263755)(899.15592474,215.73137549)(899.1759288,215.75138338)
\curveto(899.5959243,215.76137546)(899.96092393,215.71637551)(900.2709288,215.61638338)
\curveto(900.58092331,215.5263757)(900.86092303,215.40137582)(901.1109288,215.24138338)
\curveto(901.16092273,215.221376)(901.20092269,215.19137603)(901.2309288,215.15138338)
\curveto(901.26092263,215.1213761)(901.2959226,215.09637613)(901.3359288,215.07638338)
\curveto(901.41592248,215.01637621)(901.4959224,214.94637628)(901.5759288,214.86638338)
\curveto(901.66592223,214.78637644)(901.74092215,214.70637652)(901.8009288,214.62638338)
\curveto(901.96092193,214.41637681)(902.0959218,214.21637701)(902.2059288,214.02638338)
\curveto(902.27592162,213.91637731)(902.33092156,213.79637743)(902.3709288,213.66638338)
\curveto(902.41092148,213.53637769)(902.45592144,213.40637782)(902.5059288,213.27638338)
\curveto(902.55592134,213.14637808)(902.5909213,213.01137821)(902.6109288,212.87138338)
\curveto(902.64092125,212.73137849)(902.67592122,212.59137863)(902.7159288,212.45138338)
\curveto(902.72592117,212.38137884)(902.73092116,212.31137891)(902.7309288,212.24138338)
\lineto(902.7609288,212.03138338)
\moveto(901.3059288,212.54138338)
\curveto(901.33592256,212.58137864)(901.36092253,212.63137859)(901.3809288,212.69138338)
\curveto(901.40092249,212.76137846)(901.40092249,212.83137839)(901.3809288,212.90138338)
\curveto(901.32092257,213.1213781)(901.23592266,213.3263779)(901.1259288,213.51638338)
\curveto(900.98592291,213.74637748)(900.83092306,213.94137728)(900.6609288,214.10138338)
\curveto(900.4909234,214.26137696)(900.27092362,214.39637683)(900.0009288,214.50638338)
\curveto(899.93092396,214.5263767)(899.86092403,214.54137668)(899.7909288,214.55138338)
\curveto(899.72092417,214.57137665)(899.64592425,214.59137663)(899.5659288,214.61138338)
\curveto(899.48592441,214.63137659)(899.40092449,214.64137658)(899.3109288,214.64138338)
\lineto(899.0559288,214.64138338)
\curveto(899.02592487,214.6213766)(898.9909249,214.61137661)(898.9509288,214.61138338)
\curveto(898.91092498,214.6213766)(898.87592502,214.6213766)(898.8459288,214.61138338)
\lineto(898.6059288,214.55138338)
\curveto(898.53592536,214.54137668)(898.46592543,214.5263767)(898.3959288,214.50638338)
\curveto(898.10592579,214.38637684)(897.87092602,214.23637699)(897.6909288,214.05638338)
\curveto(897.52092637,213.87637735)(897.36592653,213.65137757)(897.2259288,213.38138338)
\curveto(897.1959267,213.33137789)(897.16592673,213.26637796)(897.1359288,213.18638338)
\curveto(897.10592679,213.11637811)(897.08092681,213.03637819)(897.0609288,212.94638338)
\curveto(897.04092685,212.85637837)(897.03592686,212.77137845)(897.0459288,212.69138338)
\curveto(897.05592684,212.61137861)(897.0909268,212.55137867)(897.1509288,212.51138338)
\curveto(897.23092666,212.45137877)(897.36592653,212.4213788)(897.5559288,212.42138338)
\curveto(897.75592614,212.43137879)(897.92592597,212.43637879)(898.0659288,212.43638338)
\lineto(900.3459288,212.43638338)
\curveto(900.4959234,212.43637879)(900.67592322,212.43137879)(900.8859288,212.42138338)
\curveto(901.0959228,212.4213788)(901.23592266,212.46137876)(901.3059288,212.54138338)
}
}
{
\newrgbcolor{curcolor}{0 0 0}
\pscustom[linestyle=none,fillstyle=solid,fillcolor=curcolor]
{
\newpath
\moveto(906.49756942,215.76638338)
\curveto(907.21756536,215.77637545)(907.82256475,215.69137553)(908.31256942,215.51138338)
\curveto(908.80256377,215.34137588)(909.18256339,215.03637619)(909.45256942,214.59638338)
\curveto(909.52256305,214.48637674)(909.577563,214.37137685)(909.61756942,214.25138338)
\curveto(909.65756292,214.14137708)(909.69756288,214.01637721)(909.73756942,213.87638338)
\curveto(909.75756282,213.80637742)(909.76256281,213.73137749)(909.75256942,213.65138338)
\curveto(909.74256283,213.58137764)(909.72756285,213.5263777)(909.70756942,213.48638338)
\curveto(909.68756289,213.46637776)(909.66256291,213.44637778)(909.63256942,213.42638338)
\curveto(909.60256297,213.41637781)(909.577563,213.40137782)(909.55756942,213.38138338)
\curveto(909.50756307,213.36137786)(909.45756312,213.35637787)(909.40756942,213.36638338)
\curveto(909.35756322,213.37637785)(909.30756327,213.37637785)(909.25756942,213.36638338)
\curveto(909.1775634,213.34637788)(909.0725635,213.34137788)(908.94256942,213.35138338)
\curveto(908.81256376,213.37137785)(908.72256385,213.39637783)(908.67256942,213.42638338)
\curveto(908.59256398,213.47637775)(908.53756404,213.54137768)(908.50756942,213.62138338)
\curveto(908.48756409,213.71137751)(908.45256412,213.79637743)(908.40256942,213.87638338)
\curveto(908.31256426,214.03637719)(908.18756439,214.18137704)(908.02756942,214.31138338)
\curveto(907.91756466,214.39137683)(907.79756478,214.45137677)(907.66756942,214.49138338)
\curveto(907.53756504,214.53137669)(907.39756518,214.57137665)(907.24756942,214.61138338)
\curveto(907.19756538,214.63137659)(907.14756543,214.63637659)(907.09756942,214.62638338)
\curveto(907.04756553,214.6263766)(906.99756558,214.63137659)(906.94756942,214.64138338)
\curveto(906.88756569,214.66137656)(906.81256576,214.67137655)(906.72256942,214.67138338)
\curveto(906.63256594,214.67137655)(906.55756602,214.66137656)(906.49756942,214.64138338)
\lineto(906.40756942,214.64138338)
\lineto(906.25756942,214.61138338)
\curveto(906.20756637,214.61137661)(906.15756642,214.60637662)(906.10756942,214.59638338)
\curveto(905.84756673,214.53637669)(905.63256694,214.45137677)(905.46256942,214.34138338)
\curveto(905.29256728,214.23137699)(905.1775674,214.04637718)(905.11756942,213.78638338)
\curveto(905.09756748,213.71637751)(905.09256748,213.64637758)(905.10256942,213.57638338)
\curveto(905.12256745,213.50637772)(905.14256743,213.44637778)(905.16256942,213.39638338)
\curveto(905.22256735,213.24637798)(905.29256728,213.13637809)(905.37256942,213.06638338)
\curveto(905.46256711,213.00637822)(905.572567,212.93637829)(905.70256942,212.85638338)
\curveto(905.86256671,212.75637847)(906.04256653,212.68137854)(906.24256942,212.63138338)
\curveto(906.44256613,212.59137863)(906.64256593,212.54137868)(906.84256942,212.48138338)
\curveto(906.9725656,212.44137878)(907.10256547,212.41137881)(907.23256942,212.39138338)
\curveto(907.36256521,212.37137885)(907.49256508,212.34137888)(907.62256942,212.30138338)
\curveto(907.83256474,212.24137898)(908.03756454,212.18137904)(908.23756942,212.12138338)
\curveto(908.43756414,212.07137915)(908.63756394,212.00637922)(908.83756942,211.92638338)
\lineto(908.98756942,211.86638338)
\curveto(909.03756354,211.84637938)(909.08756349,211.8213794)(909.13756942,211.79138338)
\curveto(909.33756324,211.67137955)(909.51256306,211.53637969)(909.66256942,211.38638338)
\curveto(909.81256276,211.23637999)(909.93756264,211.04638018)(910.03756942,210.81638338)
\curveto(910.05756252,210.74638048)(910.0775625,210.65138057)(910.09756942,210.53138338)
\curveto(910.11756246,210.46138076)(910.12756245,210.38638084)(910.12756942,210.30638338)
\curveto(910.13756244,210.23638099)(910.14256243,210.15638107)(910.14256942,210.06638338)
\lineto(910.14256942,209.91638338)
\curveto(910.12256245,209.84638138)(910.11256246,209.77638145)(910.11256942,209.70638338)
\curveto(910.11256246,209.63638159)(910.10256247,209.56638166)(910.08256942,209.49638338)
\curveto(910.05256252,209.38638184)(910.01756256,209.28138194)(909.97756942,209.18138338)
\curveto(909.93756264,209.08138214)(909.89256268,208.99138223)(909.84256942,208.91138338)
\curveto(909.68256289,208.65138257)(909.4775631,208.44138278)(909.22756942,208.28138338)
\curveto(908.9775636,208.13138309)(908.69756388,208.00138322)(908.38756942,207.89138338)
\curveto(908.29756428,207.86138336)(908.20256437,207.84138338)(908.10256942,207.83138338)
\curveto(908.01256456,207.81138341)(907.92256465,207.78638344)(907.83256942,207.75638338)
\curveto(907.73256484,207.73638349)(907.63256494,207.7263835)(907.53256942,207.72638338)
\curveto(907.43256514,207.7263835)(907.33256524,207.71638351)(907.23256942,207.69638338)
\lineto(907.08256942,207.69638338)
\curveto(907.03256554,207.68638354)(906.96256561,207.68138354)(906.87256942,207.68138338)
\curveto(906.78256579,207.68138354)(906.71256586,207.68638354)(906.66256942,207.69638338)
\lineto(906.49756942,207.69638338)
\curveto(906.43756614,207.71638351)(906.3725662,207.7263835)(906.30256942,207.72638338)
\curveto(906.23256634,207.71638351)(906.1725664,207.7213835)(906.12256942,207.74138338)
\curveto(906.0725665,207.75138347)(906.00756657,207.75638347)(905.92756942,207.75638338)
\lineto(905.68756942,207.81638338)
\curveto(905.61756696,207.8263834)(905.54256703,207.84638338)(905.46256942,207.87638338)
\curveto(905.15256742,207.97638325)(904.88256769,208.10138312)(904.65256942,208.25138338)
\curveto(904.42256815,208.40138282)(904.22256835,208.59638263)(904.05256942,208.83638338)
\curveto(903.96256861,208.96638226)(903.88756869,209.10138212)(903.82756942,209.24138338)
\curveto(903.76756881,209.38138184)(903.71256886,209.53638169)(903.66256942,209.70638338)
\curveto(903.64256893,209.76638146)(903.63256894,209.83638139)(903.63256942,209.91638338)
\curveto(903.64256893,210.00638122)(903.65756892,210.07638115)(903.67756942,210.12638338)
\curveto(903.70756887,210.16638106)(903.75756882,210.20638102)(903.82756942,210.24638338)
\curveto(903.8775687,210.26638096)(903.94756863,210.27638095)(904.03756942,210.27638338)
\curveto(904.12756845,210.28638094)(904.21756836,210.28638094)(904.30756942,210.27638338)
\curveto(904.39756818,210.26638096)(904.48256809,210.25138097)(904.56256942,210.23138338)
\curveto(904.65256792,210.221381)(904.71256786,210.20638102)(904.74256942,210.18638338)
\curveto(904.81256776,210.13638109)(904.85756772,210.06138116)(904.87756942,209.96138338)
\curveto(904.90756767,209.87138135)(904.94256763,209.78638144)(904.98256942,209.70638338)
\curveto(905.08256749,209.48638174)(905.21756736,209.31638191)(905.38756942,209.19638338)
\curveto(905.50756707,209.10638212)(905.64256693,209.03638219)(905.79256942,208.98638338)
\curveto(905.94256663,208.93638229)(906.10256647,208.88638234)(906.27256942,208.83638338)
\lineto(906.58756942,208.79138338)
\lineto(906.67756942,208.79138338)
\curveto(906.74756583,208.77138245)(906.83756574,208.76138246)(906.94756942,208.76138338)
\curveto(907.06756551,208.76138246)(907.16756541,208.77138245)(907.24756942,208.79138338)
\curveto(907.31756526,208.79138243)(907.3725652,208.79638243)(907.41256942,208.80638338)
\curveto(907.4725651,208.81638241)(907.53256504,208.8213824)(907.59256942,208.82138338)
\curveto(907.65256492,208.83138239)(907.70756487,208.84138238)(907.75756942,208.85138338)
\curveto(908.04756453,208.93138229)(908.2775643,209.03638219)(908.44756942,209.16638338)
\curveto(908.61756396,209.29638193)(908.73756384,209.51638171)(908.80756942,209.82638338)
\curveto(908.82756375,209.87638135)(908.83256374,209.93138129)(908.82256942,209.99138338)
\curveto(908.81256376,210.05138117)(908.80256377,210.09638113)(908.79256942,210.12638338)
\curveto(908.74256383,210.31638091)(908.6725639,210.45638077)(908.58256942,210.54638338)
\curveto(908.49256408,210.64638058)(908.3775642,210.73638049)(908.23756942,210.81638338)
\curveto(908.14756443,210.87638035)(908.04756453,210.9263803)(907.93756942,210.96638338)
\lineto(907.60756942,211.08638338)
\curveto(907.577565,211.09638013)(907.54756503,211.10138012)(907.51756942,211.10138338)
\curveto(907.49756508,211.10138012)(907.4725651,211.11138011)(907.44256942,211.13138338)
\curveto(907.10256547,211.24137998)(906.74756583,211.3213799)(906.37756942,211.37138338)
\curveto(906.01756656,211.43137979)(905.6775669,211.5263797)(905.35756942,211.65638338)
\curveto(905.25756732,211.69637953)(905.16256741,211.73137949)(905.07256942,211.76138338)
\curveto(904.98256759,211.79137943)(904.89756768,211.83137939)(904.81756942,211.88138338)
\curveto(904.62756795,211.99137923)(904.45256812,212.11637911)(904.29256942,212.25638338)
\curveto(904.13256844,212.39637883)(904.00756857,212.57137865)(903.91756942,212.78138338)
\curveto(903.88756869,212.85137837)(903.86256871,212.9213783)(903.84256942,212.99138338)
\curveto(903.83256874,213.06137816)(903.81756876,213.13637809)(903.79756942,213.21638338)
\curveto(903.76756881,213.33637789)(903.75756882,213.47137775)(903.76756942,213.62138338)
\curveto(903.7775688,213.78137744)(903.79256878,213.91637731)(903.81256942,214.02638338)
\curveto(903.83256874,214.07637715)(903.84256873,214.11637711)(903.84256942,214.14638338)
\curveto(903.85256872,214.18637704)(903.86756871,214.226377)(903.88756942,214.26638338)
\curveto(903.9775686,214.49637673)(904.09756848,214.69637653)(904.24756942,214.86638338)
\curveto(904.40756817,215.03637619)(904.58756799,215.18637604)(904.78756942,215.31638338)
\curveto(904.93756764,215.40637582)(905.10256747,215.47637575)(905.28256942,215.52638338)
\curveto(905.46256711,215.58637564)(905.65256692,215.64137558)(905.85256942,215.69138338)
\curveto(905.92256665,215.70137552)(905.98756659,215.71137551)(906.04756942,215.72138338)
\curveto(906.11756646,215.73137549)(906.19256638,215.74137548)(906.27256942,215.75138338)
\curveto(906.30256627,215.76137546)(906.34256623,215.76137546)(906.39256942,215.75138338)
\curveto(906.44256613,215.74137548)(906.4775661,215.74637548)(906.49756942,215.76638338)
}
}
{
\newrgbcolor{curcolor}{0.40000001 0.40000001 0.40000001}
\pscustom[linestyle=none,fillstyle=solid,fillcolor=curcolor]
{
\newpath
\moveto(798.51865829,218.57142)
\lineto(813.51865829,218.57142)
\lineto(813.51865829,203.57142)
\lineto(798.51865829,203.57142)
\closepath
}
}
{
\newrgbcolor{curcolor}{0.80000001 0.80000001 0.80000001}
\pscustom[linestyle=none,fillstyle=solid,fillcolor=curcolor]
{
\newpath
\moveto(244.39332872,308.43968984)
\curveto(281.08391762,301.37128393)(313.70500034,280.59395891)(335.62333163,250.33262335)
\lineto(216.58956691,164.11628684)
\closepath
}
}
{
\newrgbcolor{curcolor}{0.90196079 0.90196079 0.90196079}
\pscustom[linestyle=none,fillstyle=solid,fillcolor=curcolor]
{
\newpath
\moveto(216.58956979,311.09347495)
\curveto(225.99075994,311.09347476)(235.37026491,310.19147838)(244.59916327,308.39988374)
\lineto(216.58956691,164.11628684)
\closepath
}
}
{
\newrgbcolor{curcolor}{0.7019608 0.7019608 0.7019608}
\pscustom[linestyle=none,fillstyle=solid,fillcolor=curcolor]
{
\newpath
\moveto(335.58833524,250.38092016)
\curveto(363.94781985,211.26009382)(371.26767784,160.699381)(355.16737212,115.14194938)
\lineto(216.58956691,164.11628684)
\closepath
}
}
{
\newrgbcolor{curcolor}{0.60000002 0.60000002 0.60000002}
\pscustom[linestyle=none,fillstyle=solid,fillcolor=curcolor]
{
\newpath
\moveto(355.2232877,115.30045656)
\curveto(342.75455511,79.89008286)(317.20846897,50.59318685)(283.82539029,33.41954671)
\lineto(216.58956691,164.11628684)
\closepath
}
}
{
\newrgbcolor{curcolor}{0.50196081 0.50196081 0.50196081}
\pscustom[linestyle=none,fillstyle=solid,fillcolor=curcolor]
{
\newpath
\moveto(284.02254811,33.52116108)
\curveto(213.9783754,-2.64624834)(127.87414614,22.88549849)(88.86711607,91.38892351)
\lineto(216.58956691,164.11628684)
\closepath
}
}
{
\newrgbcolor{curcolor}{0.40000001 0.40000001 0.40000001}
\pscustom[linestyle=none,fillstyle=solid,fillcolor=curcolor]
{
\newpath
\moveto(89.0560104,91.05818812)
\curveto(48.70713663,161.49302647)(73.09662984,251.30096957)(143.53146819,291.64984335)
\curveto(165.77217293,304.3905184)(190.95807066,311.09347545)(216.58956979,311.09347495)
\lineto(216.58956691,164.11628684)
\closepath
}
}
{
\newrgbcolor{curcolor}{0 0 0}
\pscustom[linestyle=none,fillstyle=solid,fillcolor=curcolor]
{
\newpath
\moveto(392.3377037,344.1295743)
\curveto(392.33769322,344.09956863)(392.33769322,344.05956867)(392.3377037,344.0095743)
\curveto(392.34769321,343.95956877)(392.3526932,343.90456882)(392.3527037,343.8445743)
\curveto(392.3526932,343.78456894)(392.34769321,343.729569)(392.3377037,343.6795743)
\curveto(392.33769322,343.6295691)(392.33769322,343.59456913)(392.3377037,343.5745743)
\curveto(392.33769322,343.50456922)(392.33269322,343.43456929)(392.3227037,343.3645743)
\curveto(392.32269323,343.30456942)(392.32269323,343.24456948)(392.3227037,343.1845743)
\curveto(392.30269325,343.13456959)(392.29269326,343.08456964)(392.2927037,343.0345743)
\curveto(392.30269325,342.98456974)(392.30269325,342.93456979)(392.2927037,342.8845743)
\curveto(392.27269328,342.77456995)(392.2576933,342.66457006)(392.2477037,342.5545743)
\curveto(392.23769332,342.44457028)(392.21769334,342.33457039)(392.1877037,342.2245743)
\curveto(392.13769342,342.05457067)(392.09269346,341.88957084)(392.0527037,341.7295743)
\curveto(392.01269354,341.57957115)(391.96269359,341.4295713)(391.9027037,341.2795743)
\curveto(391.73269382,340.85957187)(391.52269403,340.47957225)(391.2727037,340.1395743)
\curveto(391.02269453,339.79957293)(390.72269483,339.50957322)(390.3727037,339.2695743)
\curveto(390.17269538,339.1295736)(389.96269559,339.00957372)(389.7427037,338.9095743)
\curveto(389.53269602,338.80957392)(389.30269625,338.71957401)(389.0527037,338.6395743)
\curveto(388.9526966,338.60957412)(388.84769671,338.58457414)(388.7377037,338.5645743)
\curveto(388.63769692,338.55457417)(388.53269702,338.53457419)(388.4227037,338.5045743)
\curveto(388.37269718,338.49457423)(388.32269723,338.48957424)(388.2727037,338.4895743)
\curveto(388.23269732,338.48957424)(388.18769737,338.48457424)(388.1377037,338.4745743)
\curveto(388.09769746,338.46457426)(388.0576975,338.45957427)(388.0177037,338.4595743)
\curveto(387.97769758,338.46957426)(387.93269762,338.46957426)(387.8827037,338.4595743)
\curveto(387.86269769,338.44957428)(387.83269772,338.44457428)(387.7927037,338.4445743)
\curveto(387.7526978,338.45457427)(387.72269783,338.45457427)(387.7027037,338.4445743)
\curveto(387.62269793,338.4245743)(387.52269803,338.41957431)(387.4027037,338.4295743)
\curveto(387.28269827,338.43957429)(387.17769838,338.44457428)(387.0877037,338.4445743)
\lineto(383.5927037,338.4445743)
\curveto(383.42270213,338.44457428)(383.27770228,338.44957428)(383.1577037,338.4595743)
\curveto(383.04770251,338.47957425)(382.96770259,338.54957418)(382.9177037,338.6695743)
\curveto(382.88770267,338.74957398)(382.87270268,338.86957386)(382.8727037,339.0295743)
\curveto(382.88270267,339.19957353)(382.88770267,339.33957339)(382.8877037,339.4495743)
\lineto(382.8877037,348.2545743)
\curveto(382.88770267,348.37456435)(382.88270267,348.49956423)(382.8727037,348.6295743)
\curveto(382.87270268,348.76956396)(382.89770266,348.87956385)(382.9477037,348.9595743)
\curveto(382.98770257,349.01956371)(383.06270249,349.06956366)(383.1727037,349.1095743)
\curveto(383.19270236,349.11956361)(383.21270234,349.11956361)(383.2327037,349.1095743)
\curveto(383.2527023,349.10956362)(383.27270228,349.11456361)(383.2927037,349.1245743)
\lineto(387.3277037,349.1245743)
\curveto(387.38769817,349.1245636)(387.44769811,349.1245636)(387.5077037,349.1245743)
\curveto(387.57769798,349.13456359)(387.63769792,349.13456359)(387.6877037,349.1245743)
\lineto(387.8677037,349.1245743)
\curveto(387.91769764,349.10456362)(387.97269758,349.09456363)(388.0327037,349.0945743)
\curveto(388.09269746,349.10456362)(388.14769741,349.09956363)(388.1977037,349.0795743)
\curveto(388.2576973,349.05956367)(388.31269724,349.04956368)(388.3627037,349.0495743)
\curveto(388.42269713,349.05956367)(388.48269707,349.05456367)(388.5427037,349.0345743)
\curveto(388.68269687,349.00456372)(388.81769674,348.97456375)(388.9477037,348.9445743)
\curveto(389.07769648,348.9245638)(389.20269635,348.88956384)(389.3227037,348.8395743)
\curveto(389.43269612,348.78956394)(389.54269601,348.74456398)(389.6527037,348.7045743)
\curveto(389.76269579,348.66456406)(389.86769569,348.61456411)(389.9677037,348.5545743)
\curveto(390.21769534,348.39456433)(390.44769511,348.23956449)(390.6577037,348.0895743)
\lineto(390.7477037,347.9995743)
\curveto(390.84769471,347.91956481)(390.93769462,347.8295649)(391.0177037,347.7295743)
\lineto(391.1527037,347.6095743)
\curveto(391.20269435,347.5295652)(391.2576943,347.44956528)(391.3177037,347.3695743)
\curveto(391.38769417,347.29956543)(391.44769411,347.2245655)(391.4977037,347.1445743)
\curveto(391.62769393,346.93456579)(391.74269381,346.70956602)(391.8427037,346.4695743)
\curveto(391.94269361,346.23956649)(392.03269352,345.99456673)(392.1127037,345.7345743)
\curveto(392.16269339,345.60456712)(392.19269336,345.46956726)(392.2027037,345.3295743)
\curveto(392.22269333,345.18956754)(392.24769331,345.04956768)(392.2777037,344.9095743)
\curveto(392.27769328,344.85956787)(392.27769328,344.81456791)(392.2777037,344.7745743)
\curveto(392.28769327,344.74456798)(392.29269326,344.70956802)(392.2927037,344.6695743)
\curveto(392.31269324,344.60956812)(392.31769324,344.54456818)(392.3077037,344.4745743)
\curveto(392.30769325,344.40456832)(392.31769324,344.34456838)(392.3377037,344.2945743)
\lineto(392.3377037,344.1295743)
\moveto(389.9977037,343.4095743)
\curveto(390.01769554,343.45956927)(390.02769553,343.53956919)(390.0277037,343.6495743)
\curveto(390.02769553,343.75956897)(390.01769554,343.83956889)(389.9977037,343.8895743)
\lineto(389.9977037,344.1745743)
\curveto(389.97769558,344.26456846)(389.96269559,344.35956837)(389.9527037,344.4595743)
\curveto(389.9526956,344.55956817)(389.94269561,344.64956808)(389.9227037,344.7295743)
\curveto(389.90269565,344.77956795)(389.89269566,344.8245679)(389.8927037,344.8645743)
\curveto(389.90269565,344.91456781)(389.89769566,344.96456776)(389.8777037,345.0145743)
\curveto(389.82769573,345.17456755)(389.77769578,345.3245674)(389.7277037,345.4645743)
\curveto(389.68769587,345.61456711)(389.62769593,345.75456697)(389.5477037,345.8845743)
\curveto(389.39769616,346.1245666)(389.22269633,346.3295664)(389.0227037,346.4995743)
\curveto(388.83269672,346.67956605)(388.59769696,346.8295659)(388.3177037,346.9495743)
\curveto(388.22769733,346.97956575)(388.13769742,347.00456572)(388.0477037,347.0245743)
\curveto(387.9576976,347.05456567)(387.86769769,347.07956565)(387.7777037,347.0995743)
\curveto(387.69769786,347.10956562)(387.62269793,347.11456561)(387.5527037,347.1145743)
\curveto(387.49269806,347.1245656)(387.42269813,347.13956559)(387.3427037,347.1595743)
\curveto(387.30269825,347.16956556)(387.26269829,347.16956556)(387.2227037,347.1595743)
\curveto(387.18269837,347.15956557)(387.14769841,347.16456556)(387.1177037,347.1745743)
\lineto(386.7877037,347.1745743)
\curveto(386.73769882,347.18456554)(386.68269887,347.18456554)(386.6227037,347.1745743)
\lineto(386.4427037,347.1745743)
\lineto(385.7677037,347.1745743)
\curveto(385.74769981,347.15456557)(385.71269984,347.14956558)(385.6627037,347.1595743)
\curveto(385.62269993,347.16956556)(385.58769997,347.16956556)(385.5577037,347.1595743)
\lineto(385.4077037,347.0995743)
\curveto(385.3577002,347.08956564)(385.31770024,347.05956567)(385.2877037,347.0095743)
\curveto(385.24770031,346.95956577)(385.22770033,346.88956584)(385.2277037,346.7995743)
\lineto(385.2277037,346.4995743)
\curveto(385.22770033,346.36956636)(385.22270033,346.23456649)(385.2127037,346.0945743)
\lineto(385.2127037,345.6745743)
\lineto(385.2127037,341.4895743)
\curveto(385.21270034,341.4295713)(385.20770035,341.36457136)(385.1977037,341.2945743)
\curveto(385.19770036,341.2245715)(385.20770035,341.16457156)(385.2277037,341.1145743)
\lineto(385.2277037,340.9645743)
\lineto(385.2277037,340.7545743)
\curveto(385.23770032,340.69457203)(385.2527003,340.63957209)(385.2727037,340.5895743)
\curveto(385.33270022,340.46957226)(385.44770011,340.40457232)(385.6177037,340.3945743)
\lineto(386.1427037,340.3945743)
\lineto(387.3277037,340.3945743)
\curveto(387.72769783,340.40457232)(388.06769749,340.46457226)(388.3477037,340.5745743)
\curveto(388.71769684,340.724572)(389.00769655,340.9245718)(389.2177037,341.1745743)
\curveto(389.43769612,341.4245713)(389.62269593,341.73457099)(389.7727037,342.1045743)
\curveto(389.81269574,342.18457054)(389.84269571,342.27457045)(389.8627037,342.3745743)
\curveto(389.88269567,342.47457025)(389.90769565,342.57457015)(389.9377037,342.6745743)
\lineto(389.9377037,342.7945743)
\curveto(389.9576956,342.86456986)(389.96769559,342.93956979)(389.9677037,343.0195743)
\curveto(389.96769559,343.09956963)(389.97769558,343.17956955)(389.9977037,343.2595743)
\lineto(389.9977037,343.4095743)
}
}
{
\newrgbcolor{curcolor}{0 0 0}
\pscustom[linestyle=none,fillstyle=solid,fillcolor=curcolor]
{
\newpath
\moveto(395.83621933,349.0195743)
\curveto(395.90621638,348.93956379)(395.94121634,348.81956391)(395.94121933,348.6595743)
\lineto(395.94121933,348.1945743)
\lineto(395.94121933,347.7895743)
\curveto(395.94121634,347.64956508)(395.90621638,347.55456517)(395.83621933,347.5045743)
\curveto(395.77621651,347.45456527)(395.69621659,347.4245653)(395.59621933,347.4145743)
\curveto(395.50621678,347.40456532)(395.40621688,347.39956533)(395.29621933,347.3995743)
\lineto(394.45621933,347.3995743)
\curveto(394.34621794,347.39956533)(394.24621804,347.40456532)(394.15621933,347.4145743)
\curveto(394.07621821,347.4245653)(394.00621828,347.45456527)(393.94621933,347.5045743)
\curveto(393.90621838,347.53456519)(393.87621841,347.58956514)(393.85621933,347.6695743)
\curveto(393.84621844,347.75956497)(393.83621845,347.85456487)(393.82621933,347.9545743)
\lineto(393.82621933,348.2845743)
\curveto(393.83621845,348.39456433)(393.84121844,348.48956424)(393.84121933,348.5695743)
\lineto(393.84121933,348.7795743)
\curveto(393.85121843,348.84956388)(393.87121841,348.90956382)(393.90121933,348.9595743)
\curveto(393.92121836,348.99956373)(393.94621834,349.0295637)(393.97621933,349.0495743)
\lineto(394.09621933,349.1095743)
\curveto(394.11621817,349.10956362)(394.14121814,349.10956362)(394.17121933,349.1095743)
\curveto(394.20121808,349.11956361)(394.22621806,349.1245636)(394.24621933,349.1245743)
\lineto(395.34121933,349.1245743)
\curveto(395.44121684,349.1245636)(395.53621675,349.11956361)(395.62621933,349.1095743)
\curveto(395.71621657,349.09956363)(395.7862165,349.06956366)(395.83621933,349.0195743)
\moveto(395.94121933,339.2545743)
\curveto(395.94121634,339.05457367)(395.93621635,338.88457384)(395.92621933,338.7445743)
\curveto(395.91621637,338.60457412)(395.82621646,338.50957422)(395.65621933,338.4595743)
\curveto(395.59621669,338.43957429)(395.53121675,338.4295743)(395.46121933,338.4295743)
\curveto(395.39121689,338.43957429)(395.31621697,338.44457428)(395.23621933,338.4445743)
\lineto(394.39621933,338.4445743)
\curveto(394.30621798,338.44457428)(394.21621807,338.44957428)(394.12621933,338.4595743)
\curveto(394.04621824,338.46957426)(393.9862183,338.49957423)(393.94621933,338.5495743)
\curveto(393.8862184,338.61957411)(393.85121843,338.70457402)(393.84121933,338.8045743)
\lineto(393.84121933,339.1495743)
\lineto(393.84121933,345.4795743)
\lineto(393.84121933,345.7795743)
\curveto(393.84121844,345.87956685)(393.86121842,345.95956677)(393.90121933,346.0195743)
\curveto(393.96121832,346.08956664)(394.04621824,346.13456659)(394.15621933,346.1545743)
\curveto(394.17621811,346.16456656)(394.20121808,346.16456656)(394.23121933,346.1545743)
\curveto(394.27121801,346.15456657)(394.30121798,346.15956657)(394.32121933,346.1695743)
\lineto(395.07121933,346.1695743)
\lineto(395.26621933,346.1695743)
\curveto(395.34621694,346.17956655)(395.41121687,346.17956655)(395.46121933,346.1695743)
\lineto(395.58121933,346.1695743)
\curveto(395.64121664,346.14956658)(395.69621659,346.13456659)(395.74621933,346.1245743)
\curveto(395.79621649,346.11456661)(395.83621645,346.08456664)(395.86621933,346.0345743)
\curveto(395.90621638,345.98456674)(395.92621636,345.91456681)(395.92621933,345.8245743)
\curveto(395.93621635,345.73456699)(395.94121634,345.63956709)(395.94121933,345.5395743)
\lineto(395.94121933,339.2545743)
}
}
{
\newrgbcolor{curcolor}{0 0 0}
\pscustom[linestyle=none,fillstyle=solid,fillcolor=curcolor]
{
\newpath
\moveto(400.57340683,346.3795743)
\curveto(401.32340233,346.39956633)(401.97340168,346.31456641)(402.52340683,346.1245743)
\curveto(403.08340057,345.94456678)(403.50840014,345.6295671)(403.79840683,345.1795743)
\curveto(403.86839978,345.06956766)(403.92839972,344.95456777)(403.97840683,344.8345743)
\curveto(404.03839961,344.724568)(404.08839956,344.59956813)(404.12840683,344.4595743)
\curveto(404.1483995,344.39956833)(404.15839949,344.33456839)(404.15840683,344.2645743)
\curveto(404.15839949,344.19456853)(404.1483995,344.13456859)(404.12840683,344.0845743)
\curveto(404.08839956,344.0245687)(404.03339962,343.98456874)(403.96340683,343.9645743)
\curveto(403.91339974,343.94456878)(403.8533998,343.93456879)(403.78340683,343.9345743)
\lineto(403.57340683,343.9345743)
\lineto(402.91340683,343.9345743)
\curveto(402.84340081,343.93456879)(402.77340088,343.9295688)(402.70340683,343.9195743)
\curveto(402.63340102,343.91956881)(402.56840108,343.9295688)(402.50840683,343.9495743)
\curveto(402.40840124,343.96956876)(402.33340132,344.00956872)(402.28340683,344.0695743)
\curveto(402.23340142,344.1295686)(402.18840146,344.18956854)(402.14840683,344.2495743)
\lineto(402.02840683,344.4595743)
\curveto(401.99840165,344.53956819)(401.9484017,344.60456812)(401.87840683,344.6545743)
\curveto(401.77840187,344.73456799)(401.67840197,344.79456793)(401.57840683,344.8345743)
\curveto(401.48840216,344.87456785)(401.37340228,344.90956782)(401.23340683,344.9395743)
\curveto(401.16340249,344.95956777)(401.05840259,344.97456775)(400.91840683,344.9845743)
\curveto(400.78840286,344.99456773)(400.68840296,344.98956774)(400.61840683,344.9695743)
\lineto(400.51340683,344.9695743)
\lineto(400.36340683,344.9395743)
\curveto(400.32340333,344.93956779)(400.27840337,344.93456779)(400.22840683,344.9245743)
\curveto(400.05840359,344.87456785)(399.91840373,344.80456792)(399.80840683,344.7145743)
\curveto(399.70840394,344.63456809)(399.63840401,344.50956822)(399.59840683,344.3395743)
\curveto(399.57840407,344.26956846)(399.57840407,344.20456852)(399.59840683,344.1445743)
\curveto(399.61840403,344.08456864)(399.63840401,344.03456869)(399.65840683,343.9945743)
\curveto(399.72840392,343.87456885)(399.80840384,343.77956895)(399.89840683,343.7095743)
\curveto(399.99840365,343.63956909)(400.11340354,343.57956915)(400.24340683,343.5295743)
\curveto(400.43340322,343.44956928)(400.63840301,343.37956935)(400.85840683,343.3195743)
\lineto(401.54840683,343.1695743)
\curveto(401.78840186,343.1295696)(402.01840163,343.07956965)(402.23840683,343.0195743)
\curveto(402.46840118,342.96956976)(402.68340097,342.90456982)(402.88340683,342.8245743)
\curveto(402.97340068,342.78456994)(403.05840059,342.74956998)(403.13840683,342.7195743)
\curveto(403.22840042,342.69957003)(403.31340034,342.66457006)(403.39340683,342.6145743)
\curveto(403.58340007,342.49457023)(403.7533999,342.36457036)(403.90340683,342.2245743)
\curveto(404.06339959,342.08457064)(404.18839946,341.90957082)(404.27840683,341.6995743)
\curveto(404.30839934,341.6295711)(404.33339932,341.55957117)(404.35340683,341.4895743)
\curveto(404.37339928,341.41957131)(404.39339926,341.34457138)(404.41340683,341.2645743)
\curveto(404.42339923,341.20457152)(404.42839922,341.10957162)(404.42840683,340.9795743)
\curveto(404.43839921,340.85957187)(404.43839921,340.76457196)(404.42840683,340.6945743)
\lineto(404.42840683,340.6195743)
\curveto(404.40839924,340.55957217)(404.39339926,340.49957223)(404.38340683,340.4395743)
\curveto(404.38339927,340.38957234)(404.37839927,340.33957239)(404.36840683,340.2895743)
\curveto(404.29839935,339.98957274)(404.18839946,339.724573)(404.03840683,339.4945743)
\curveto(403.87839977,339.25457347)(403.68339997,339.05957367)(403.45340683,338.9095743)
\curveto(403.22340043,338.75957397)(402.96340069,338.6295741)(402.67340683,338.5195743)
\curveto(402.56340109,338.46957426)(402.44340121,338.43457429)(402.31340683,338.4145743)
\curveto(402.19340146,338.39457433)(402.07340158,338.36957436)(401.95340683,338.3395743)
\curveto(401.86340179,338.31957441)(401.76840188,338.30957442)(401.66840683,338.3095743)
\curveto(401.57840207,338.29957443)(401.48840216,338.28457444)(401.39840683,338.2645743)
\lineto(401.12840683,338.2645743)
\curveto(401.06840258,338.24457448)(400.96340269,338.23457449)(400.81340683,338.2345743)
\curveto(400.67340298,338.23457449)(400.57340308,338.24457448)(400.51340683,338.2645743)
\curveto(400.48340317,338.26457446)(400.4484032,338.26957446)(400.40840683,338.2795743)
\lineto(400.30340683,338.2795743)
\curveto(400.18340347,338.29957443)(400.06340359,338.31457441)(399.94340683,338.3245743)
\curveto(399.82340383,338.33457439)(399.70840394,338.35457437)(399.59840683,338.3845743)
\curveto(399.20840444,338.49457423)(398.86340479,338.61957411)(398.56340683,338.7595743)
\curveto(398.26340539,338.90957382)(398.00840564,339.1295736)(397.79840683,339.4195743)
\curveto(397.65840599,339.60957312)(397.53840611,339.8295729)(397.43840683,340.0795743)
\curveto(397.41840623,340.13957259)(397.39840625,340.21957251)(397.37840683,340.3195743)
\curveto(397.35840629,340.36957236)(397.34340631,340.43957229)(397.33340683,340.5295743)
\curveto(397.32340633,340.61957211)(397.32840632,340.69457203)(397.34840683,340.7545743)
\curveto(397.37840627,340.8245719)(397.42840622,340.87457185)(397.49840683,340.9045743)
\curveto(397.5484061,340.9245718)(397.60840604,340.93457179)(397.67840683,340.9345743)
\lineto(397.90340683,340.9345743)
\lineto(398.60840683,340.9345743)
\lineto(398.84840683,340.9345743)
\curveto(398.92840472,340.93457179)(398.99840465,340.9245718)(399.05840683,340.9045743)
\curveto(399.16840448,340.86457186)(399.23840441,340.79957193)(399.26840683,340.7095743)
\curveto(399.30840434,340.61957211)(399.3534043,340.5245722)(399.40340683,340.4245743)
\curveto(399.42340423,340.37457235)(399.45840419,340.30957242)(399.50840683,340.2295743)
\curveto(399.56840408,340.14957258)(399.61840403,340.09957263)(399.65840683,340.0795743)
\curveto(399.77840387,339.97957275)(399.89340376,339.89957283)(400.00340683,339.8395743)
\curveto(400.11340354,339.78957294)(400.2534034,339.73957299)(400.42340683,339.6895743)
\curveto(400.47340318,339.66957306)(400.52340313,339.65957307)(400.57340683,339.6595743)
\curveto(400.62340303,339.66957306)(400.67340298,339.66957306)(400.72340683,339.6595743)
\curveto(400.80340285,339.63957309)(400.88840276,339.6295731)(400.97840683,339.6295743)
\curveto(401.07840257,339.63957309)(401.16340249,339.65457307)(401.23340683,339.6745743)
\curveto(401.28340237,339.68457304)(401.32840232,339.68957304)(401.36840683,339.6895743)
\curveto(401.41840223,339.68957304)(401.46840218,339.69957303)(401.51840683,339.7195743)
\curveto(401.65840199,339.76957296)(401.78340187,339.8295729)(401.89340683,339.8995743)
\curveto(402.01340164,339.96957276)(402.10840154,340.05957267)(402.17840683,340.1695743)
\curveto(402.22840142,340.24957248)(402.26840138,340.37457235)(402.29840683,340.5445743)
\curveto(402.31840133,340.61457211)(402.31840133,340.67957205)(402.29840683,340.7395743)
\curveto(402.27840137,340.79957193)(402.25840139,340.84957188)(402.23840683,340.8895743)
\curveto(402.16840148,341.0295717)(402.07840157,341.13457159)(401.96840683,341.2045743)
\curveto(401.86840178,341.27457145)(401.7484019,341.33957139)(401.60840683,341.3995743)
\curveto(401.41840223,341.47957125)(401.21840243,341.54457118)(401.00840683,341.5945743)
\curveto(400.79840285,341.64457108)(400.58840306,341.69957103)(400.37840683,341.7595743)
\curveto(400.29840335,341.77957095)(400.21340344,341.79457093)(400.12340683,341.8045743)
\curveto(400.04340361,341.81457091)(399.96340369,341.8295709)(399.88340683,341.8495743)
\curveto(399.56340409,341.93957079)(399.25840439,342.0245707)(398.96840683,342.1045743)
\curveto(398.67840497,342.19457053)(398.41340524,342.3245704)(398.17340683,342.4945743)
\curveto(397.89340576,342.69457003)(397.68840596,342.96456976)(397.55840683,343.3045743)
\curveto(397.53840611,343.37456935)(397.51840613,343.46956926)(397.49840683,343.5895743)
\curveto(397.47840617,343.65956907)(397.46340619,343.74456898)(397.45340683,343.8445743)
\curveto(397.44340621,343.94456878)(397.4484062,344.03456869)(397.46840683,344.1145743)
\curveto(397.48840616,344.16456856)(397.49340616,344.20456852)(397.48340683,344.2345743)
\curveto(397.47340618,344.27456845)(397.47840617,344.31956841)(397.49840683,344.3695743)
\curveto(397.51840613,344.47956825)(397.53840611,344.57956815)(397.55840683,344.6695743)
\curveto(397.58840606,344.76956796)(397.62340603,344.86456786)(397.66340683,344.9545743)
\curveto(397.79340586,345.24456748)(397.97340568,345.47956725)(398.20340683,345.6595743)
\curveto(398.43340522,345.83956689)(398.69340496,345.98456674)(398.98340683,346.0945743)
\curveto(399.09340456,346.14456658)(399.20840444,346.17956655)(399.32840683,346.1995743)
\curveto(399.4484042,346.2295665)(399.57340408,346.25956647)(399.70340683,346.2895743)
\curveto(399.76340389,346.30956642)(399.82340383,346.31956641)(399.88340683,346.3195743)
\lineto(400.06340683,346.3495743)
\curveto(400.14340351,346.35956637)(400.22840342,346.36456636)(400.31840683,346.3645743)
\curveto(400.40840324,346.36456636)(400.49340316,346.36956636)(400.57340683,346.3795743)
}
}
{
\newrgbcolor{curcolor}{0 0 0}
\pscustom[linestyle=none,fillstyle=solid,fillcolor=curcolor]
{
\newpath
\moveto(406.71004745,348.4795743)
\lineto(407.71504745,348.4795743)
\curveto(407.86504447,348.47956425)(407.99504434,348.46956426)(408.10504745,348.4495743)
\curveto(408.22504411,348.43956429)(408.31004402,348.37956435)(408.36004745,348.2695743)
\curveto(408.38004395,348.21956451)(408.39004394,348.15956457)(408.39004745,348.0895743)
\lineto(408.39004745,347.8795743)
\lineto(408.39004745,347.2045743)
\curveto(408.39004394,347.15456557)(408.38504395,347.09456563)(408.37504745,347.0245743)
\curveto(408.37504396,346.96456576)(408.38004395,346.90956582)(408.39004745,346.8595743)
\lineto(408.39004745,346.6945743)
\curveto(408.39004394,346.61456611)(408.39504394,346.53956619)(408.40504745,346.4695743)
\curveto(408.41504392,346.40956632)(408.44004389,346.35456637)(408.48004745,346.3045743)
\curveto(408.55004378,346.21456651)(408.67504366,346.16456656)(408.85504745,346.1545743)
\lineto(409.39504745,346.1545743)
\lineto(409.57504745,346.1545743)
\curveto(409.6350427,346.15456657)(409.69004264,346.14456658)(409.74004745,346.1245743)
\curveto(409.85004248,346.07456665)(409.91004242,345.98456674)(409.92004745,345.8545743)
\curveto(409.94004239,345.724567)(409.95004238,345.57956715)(409.95004745,345.4195743)
\lineto(409.95004745,345.2095743)
\curveto(409.96004237,345.13956759)(409.95504238,345.07956765)(409.93504745,345.0295743)
\curveto(409.88504245,344.86956786)(409.78004255,344.78456794)(409.62004745,344.7745743)
\curveto(409.46004287,344.76456796)(409.28004305,344.75956797)(409.08004745,344.7595743)
\lineto(408.94504745,344.7595743)
\curveto(408.90504343,344.76956796)(408.87004346,344.76956796)(408.84004745,344.7595743)
\curveto(408.80004353,344.74956798)(408.76504357,344.74456798)(408.73504745,344.7445743)
\curveto(408.70504363,344.75456797)(408.67504366,344.74956798)(408.64504745,344.7295743)
\curveto(408.56504377,344.70956802)(408.50504383,344.66456806)(408.46504745,344.5945743)
\curveto(408.4350439,344.53456819)(408.41004392,344.45956827)(408.39004745,344.3695743)
\curveto(408.38004395,344.31956841)(408.38004395,344.26456846)(408.39004745,344.2045743)
\curveto(408.40004393,344.14456858)(408.40004393,344.08956864)(408.39004745,344.0395743)
\lineto(408.39004745,343.1095743)
\lineto(408.39004745,341.3545743)
\curveto(408.39004394,341.10457162)(408.39504394,340.88457184)(408.40504745,340.6945743)
\curveto(408.42504391,340.51457221)(408.49004384,340.35457237)(408.60004745,340.2145743)
\curveto(408.65004368,340.15457257)(408.71504362,340.10957262)(408.79504745,340.0795743)
\lineto(409.06504745,340.0195743)
\curveto(409.09504324,340.00957272)(409.12504321,340.00457272)(409.15504745,340.0045743)
\curveto(409.19504314,340.01457271)(409.22504311,340.01457271)(409.24504745,340.0045743)
\lineto(409.41004745,340.0045743)
\curveto(409.52004281,340.00457272)(409.61504272,339.99957273)(409.69504745,339.9895743)
\curveto(409.77504256,339.97957275)(409.84004249,339.93957279)(409.89004745,339.8695743)
\curveto(409.9300424,339.80957292)(409.95004238,339.729573)(409.95004745,339.6295743)
\lineto(409.95004745,339.3445743)
\curveto(409.95004238,339.13457359)(409.94504239,338.93957379)(409.93504745,338.7595743)
\curveto(409.9350424,338.58957414)(409.85504248,338.47457425)(409.69504745,338.4145743)
\curveto(409.64504269,338.39457433)(409.60004273,338.38957434)(409.56004745,338.3995743)
\curveto(409.52004281,338.39957433)(409.47504286,338.38957434)(409.42504745,338.3695743)
\lineto(409.27504745,338.3695743)
\curveto(409.25504308,338.36957436)(409.22504311,338.37457435)(409.18504745,338.3845743)
\curveto(409.14504319,338.38457434)(409.11004322,338.37957435)(409.08004745,338.3695743)
\curveto(409.0300433,338.35957437)(408.97504336,338.35957437)(408.91504745,338.3695743)
\lineto(408.76504745,338.3695743)
\lineto(408.61504745,338.3695743)
\curveto(408.56504377,338.35957437)(408.52004381,338.35957437)(408.48004745,338.3695743)
\lineto(408.31504745,338.3695743)
\curveto(408.26504407,338.37957435)(408.21004412,338.38457434)(408.15004745,338.3845743)
\curveto(408.09004424,338.38457434)(408.0350443,338.38957434)(407.98504745,338.3995743)
\curveto(407.91504442,338.40957432)(407.85004448,338.41957431)(407.79004745,338.4295743)
\lineto(407.61004745,338.4595743)
\curveto(407.50004483,338.48957424)(407.39504494,338.5245742)(407.29504745,338.5645743)
\curveto(407.19504514,338.60457412)(407.10004523,338.64957408)(407.01004745,338.6995743)
\lineto(406.92004745,338.7595743)
\curveto(406.89004544,338.78957394)(406.85504548,338.81957391)(406.81504745,338.8495743)
\curveto(406.79504554,338.86957386)(406.77004556,338.88957384)(406.74004745,338.9095743)
\lineto(406.66504745,338.9845743)
\curveto(406.52504581,339.17457355)(406.42004591,339.38457334)(406.35004745,339.6145743)
\curveto(406.330046,339.65457307)(406.32004601,339.68957304)(406.32004745,339.7195743)
\curveto(406.330046,339.75957297)(406.330046,339.80457292)(406.32004745,339.8545743)
\curveto(406.31004602,339.87457285)(406.30504603,339.89957283)(406.30504745,339.9295743)
\curveto(406.30504603,339.95957277)(406.30004603,339.98457274)(406.29004745,340.0045743)
\lineto(406.29004745,340.1545743)
\curveto(406.28004605,340.19457253)(406.27504606,340.23957249)(406.27504745,340.2895743)
\curveto(406.28504605,340.33957239)(406.29004604,340.38957234)(406.29004745,340.4395743)
\lineto(406.29004745,341.0095743)
\lineto(406.29004745,343.2445743)
\lineto(406.29004745,344.0395743)
\lineto(406.29004745,344.2495743)
\curveto(406.30004603,344.31956841)(406.29504604,344.38456834)(406.27504745,344.4445743)
\curveto(406.2350461,344.58456814)(406.16504617,344.67456805)(406.06504745,344.7145743)
\curveto(405.95504638,344.76456796)(405.81504652,344.77956795)(405.64504745,344.7595743)
\curveto(405.47504686,344.73956799)(405.330047,344.75456797)(405.21004745,344.8045743)
\curveto(405.1300472,344.83456789)(405.08004725,344.87956785)(405.06004745,344.9395743)
\curveto(405.04004729,344.99956773)(405.02004731,345.07456765)(405.00004745,345.1645743)
\lineto(405.00004745,345.4795743)
\curveto(405.00004733,345.65956707)(405.01004732,345.80456692)(405.03004745,345.9145743)
\curveto(405.05004728,346.0245667)(405.1350472,346.09956663)(405.28504745,346.1395743)
\curveto(405.32504701,346.15956657)(405.36504697,346.16456656)(405.40504745,346.1545743)
\lineto(405.54004745,346.1545743)
\curveto(405.69004664,346.15456657)(405.8300465,346.15956657)(405.96004745,346.1695743)
\curveto(406.09004624,346.18956654)(406.18004615,346.24956648)(406.23004745,346.3495743)
\curveto(406.26004607,346.41956631)(406.27504606,346.49956623)(406.27504745,346.5895743)
\curveto(406.28504605,346.67956605)(406.29004604,346.76956596)(406.29004745,346.8595743)
\lineto(406.29004745,347.7895743)
\lineto(406.29004745,348.0445743)
\curveto(406.29004604,348.13456459)(406.30004603,348.20956452)(406.32004745,348.2695743)
\curveto(406.37004596,348.36956436)(406.44504589,348.43456429)(406.54504745,348.4645743)
\curveto(406.56504577,348.47456425)(406.59004574,348.47456425)(406.62004745,348.4645743)
\curveto(406.66004567,348.46456426)(406.69004564,348.46956426)(406.71004745,348.4795743)
}
}
{
\newrgbcolor{curcolor}{0 0 0}
\pscustom[linestyle=none,fillstyle=solid,fillcolor=curcolor]
{
\newpath
\moveto(415.35848495,346.3645743)
\curveto(415.46847964,346.36456636)(415.56347954,346.35456637)(415.64348495,346.3345743)
\curveto(415.73347937,346.31456641)(415.8034793,346.26956646)(415.85348495,346.1995743)
\curveto(415.91347919,346.11956661)(415.94347916,345.97956675)(415.94348495,345.7795743)
\lineto(415.94348495,345.2695743)
\lineto(415.94348495,344.8945743)
\curveto(415.95347915,344.75456797)(415.93847917,344.64456808)(415.89848495,344.5645743)
\curveto(415.85847925,344.49456823)(415.79847931,344.44956828)(415.71848495,344.4295743)
\curveto(415.64847946,344.40956832)(415.56347954,344.39956833)(415.46348495,344.3995743)
\curveto(415.37347973,344.39956833)(415.27347983,344.40456832)(415.16348495,344.4145743)
\curveto(415.06348004,344.4245683)(414.96848014,344.41956831)(414.87848495,344.3995743)
\curveto(414.8084803,344.37956835)(414.73848037,344.36456836)(414.66848495,344.3545743)
\curveto(414.59848051,344.35456837)(414.53348057,344.34456838)(414.47348495,344.3245743)
\curveto(414.31348079,344.27456845)(414.15348095,344.19956853)(413.99348495,344.0995743)
\curveto(413.83348127,344.00956872)(413.7084814,343.90456882)(413.61848495,343.7845743)
\curveto(413.56848154,343.70456902)(413.51348159,343.61956911)(413.45348495,343.5295743)
\curveto(413.4034817,343.44956928)(413.35348175,343.36456936)(413.30348495,343.2745743)
\curveto(413.27348183,343.19456953)(413.24348186,343.10956962)(413.21348495,343.0195743)
\lineto(413.15348495,342.7795743)
\curveto(413.13348197,342.70957002)(413.12348198,342.63457009)(413.12348495,342.5545743)
\curveto(413.12348198,342.48457024)(413.11348199,342.41457031)(413.09348495,342.3445743)
\curveto(413.08348202,342.30457042)(413.07848203,342.26457046)(413.07848495,342.2245743)
\curveto(413.08848202,342.19457053)(413.08848202,342.16457056)(413.07848495,342.1345743)
\lineto(413.07848495,341.8945743)
\curveto(413.05848205,341.8245709)(413.05348205,341.74457098)(413.06348495,341.6545743)
\curveto(413.07348203,341.57457115)(413.07848203,341.49457123)(413.07848495,341.4145743)
\lineto(413.07848495,340.4545743)
\lineto(413.07848495,339.1795743)
\curveto(413.07848203,339.04957368)(413.07348203,338.9295738)(413.06348495,338.8195743)
\curveto(413.05348205,338.70957402)(413.02348208,338.61957411)(412.97348495,338.5495743)
\curveto(412.95348215,338.51957421)(412.91848219,338.49457423)(412.86848495,338.4745743)
\curveto(412.82848228,338.46457426)(412.78348232,338.45457427)(412.73348495,338.4445743)
\lineto(412.65848495,338.4445743)
\curveto(412.6084825,338.43457429)(412.55348255,338.4295743)(412.49348495,338.4295743)
\lineto(412.32848495,338.4295743)
\lineto(411.68348495,338.4295743)
\curveto(411.62348348,338.43957429)(411.55848355,338.44457428)(411.48848495,338.4445743)
\lineto(411.29348495,338.4445743)
\curveto(411.24348386,338.46457426)(411.19348391,338.47957425)(411.14348495,338.4895743)
\curveto(411.09348401,338.50957422)(411.05848405,338.54457418)(411.03848495,338.5945743)
\curveto(410.99848411,338.64457408)(410.97348413,338.71457401)(410.96348495,338.8045743)
\lineto(410.96348495,339.1045743)
\lineto(410.96348495,340.1245743)
\lineto(410.96348495,344.3545743)
\lineto(410.96348495,345.4645743)
\lineto(410.96348495,345.7495743)
\curveto(410.96348414,345.84956688)(410.98348412,345.9295668)(411.02348495,345.9895743)
\curveto(411.07348403,346.06956666)(411.14848396,346.11956661)(411.24848495,346.1395743)
\curveto(411.34848376,346.15956657)(411.46848364,346.16956656)(411.60848495,346.1695743)
\lineto(412.37348495,346.1695743)
\curveto(412.49348261,346.16956656)(412.59848251,346.15956657)(412.68848495,346.1395743)
\curveto(412.77848233,346.1295666)(412.84848226,346.08456664)(412.89848495,346.0045743)
\curveto(412.92848218,345.95456677)(412.94348216,345.88456684)(412.94348495,345.7945743)
\lineto(412.97348495,345.5245743)
\curveto(412.98348212,345.44456728)(412.99848211,345.36956736)(413.01848495,345.2995743)
\curveto(413.04848206,345.2295675)(413.09848201,345.19456753)(413.16848495,345.1945743)
\curveto(413.18848192,345.21456751)(413.2084819,345.2245675)(413.22848495,345.2245743)
\curveto(413.24848186,345.2245675)(413.26848184,345.23456749)(413.28848495,345.2545743)
\curveto(413.34848176,345.30456742)(413.39848171,345.35956737)(413.43848495,345.4195743)
\curveto(413.48848162,345.48956724)(413.54848156,345.54956718)(413.61848495,345.5995743)
\curveto(413.65848145,345.6295671)(413.69348141,345.65956707)(413.72348495,345.6895743)
\curveto(413.75348135,345.729567)(413.78848132,345.76456696)(413.82848495,345.7945743)
\lineto(414.09848495,345.9745743)
\curveto(414.19848091,346.03456669)(414.29848081,346.08956664)(414.39848495,346.1395743)
\curveto(414.49848061,346.17956655)(414.59848051,346.21456651)(414.69848495,346.2445743)
\lineto(415.02848495,346.3345743)
\curveto(415.05848005,346.34456638)(415.11347999,346.34456638)(415.19348495,346.3345743)
\curveto(415.28347982,346.33456639)(415.33847977,346.34456638)(415.35848495,346.3645743)
}
}
{
\newrgbcolor{curcolor}{0 0 0}
\pscustom[linestyle=none,fillstyle=solid,fillcolor=curcolor]
{
\newpath
\moveto(418.86356308,349.0195743)
\curveto(418.93356013,348.93956379)(418.96856009,348.81956391)(418.96856308,348.6595743)
\lineto(418.96856308,348.1945743)
\lineto(418.96856308,347.7895743)
\curveto(418.96856009,347.64956508)(418.93356013,347.55456517)(418.86356308,347.5045743)
\curveto(418.80356026,347.45456527)(418.72356034,347.4245653)(418.62356308,347.4145743)
\curveto(418.53356053,347.40456532)(418.43356063,347.39956533)(418.32356308,347.3995743)
\lineto(417.48356308,347.3995743)
\curveto(417.37356169,347.39956533)(417.27356179,347.40456532)(417.18356308,347.4145743)
\curveto(417.10356196,347.4245653)(417.03356203,347.45456527)(416.97356308,347.5045743)
\curveto(416.93356213,347.53456519)(416.90356216,347.58956514)(416.88356308,347.6695743)
\curveto(416.87356219,347.75956497)(416.8635622,347.85456487)(416.85356308,347.9545743)
\lineto(416.85356308,348.2845743)
\curveto(416.8635622,348.39456433)(416.86856219,348.48956424)(416.86856308,348.5695743)
\lineto(416.86856308,348.7795743)
\curveto(416.87856218,348.84956388)(416.89856216,348.90956382)(416.92856308,348.9595743)
\curveto(416.94856211,348.99956373)(416.97356209,349.0295637)(417.00356308,349.0495743)
\lineto(417.12356308,349.1095743)
\curveto(417.14356192,349.10956362)(417.16856189,349.10956362)(417.19856308,349.1095743)
\curveto(417.22856183,349.11956361)(417.25356181,349.1245636)(417.27356308,349.1245743)
\lineto(418.36856308,349.1245743)
\curveto(418.46856059,349.1245636)(418.5635605,349.11956361)(418.65356308,349.1095743)
\curveto(418.74356032,349.09956363)(418.81356025,349.06956366)(418.86356308,349.0195743)
\moveto(418.96856308,339.2545743)
\curveto(418.96856009,339.05457367)(418.9635601,338.88457384)(418.95356308,338.7445743)
\curveto(418.94356012,338.60457412)(418.85356021,338.50957422)(418.68356308,338.4595743)
\curveto(418.62356044,338.43957429)(418.5585605,338.4295743)(418.48856308,338.4295743)
\curveto(418.41856064,338.43957429)(418.34356072,338.44457428)(418.26356308,338.4445743)
\lineto(417.42356308,338.4445743)
\curveto(417.33356173,338.44457428)(417.24356182,338.44957428)(417.15356308,338.4595743)
\curveto(417.07356199,338.46957426)(417.01356205,338.49957423)(416.97356308,338.5495743)
\curveto(416.91356215,338.61957411)(416.87856218,338.70457402)(416.86856308,338.8045743)
\lineto(416.86856308,339.1495743)
\lineto(416.86856308,345.4795743)
\lineto(416.86856308,345.7795743)
\curveto(416.86856219,345.87956685)(416.88856217,345.95956677)(416.92856308,346.0195743)
\curveto(416.98856207,346.08956664)(417.07356199,346.13456659)(417.18356308,346.1545743)
\curveto(417.20356186,346.16456656)(417.22856183,346.16456656)(417.25856308,346.1545743)
\curveto(417.29856176,346.15456657)(417.32856173,346.15956657)(417.34856308,346.1695743)
\lineto(418.09856308,346.1695743)
\lineto(418.29356308,346.1695743)
\curveto(418.37356069,346.17956655)(418.43856062,346.17956655)(418.48856308,346.1695743)
\lineto(418.60856308,346.1695743)
\curveto(418.66856039,346.14956658)(418.72356034,346.13456659)(418.77356308,346.1245743)
\curveto(418.82356024,346.11456661)(418.8635602,346.08456664)(418.89356308,346.0345743)
\curveto(418.93356013,345.98456674)(418.95356011,345.91456681)(418.95356308,345.8245743)
\curveto(418.9635601,345.73456699)(418.96856009,345.63956709)(418.96856308,345.5395743)
\lineto(418.96856308,339.2545743)
}
}
{
\newrgbcolor{curcolor}{0 0 0}
\pscustom[linestyle=none,fillstyle=solid,fillcolor=curcolor]
{
\newpath
\moveto(428.43075058,342.6895743)
\curveto(428.45074198,342.6295701)(428.46074197,342.5245702)(428.46075058,342.3745743)
\curveto(428.46074197,342.23457049)(428.45574197,342.13457059)(428.44575058,342.0745743)
\curveto(428.44574198,342.0245707)(428.44074199,341.97957075)(428.43075058,341.9395743)
\lineto(428.43075058,341.8195743)
\curveto(428.41074202,341.73957099)(428.40074203,341.65957107)(428.40075058,341.5795743)
\curveto(428.40074203,341.50957122)(428.39074204,341.43457129)(428.37075058,341.3545743)
\curveto(428.37074206,341.31457141)(428.36074207,341.24457148)(428.34075058,341.1445743)
\curveto(428.31074212,341.0245717)(428.28074215,340.89957183)(428.25075058,340.7695743)
\curveto(428.2307422,340.64957208)(428.19574223,340.53457219)(428.14575058,340.4245743)
\curveto(427.96574246,339.97457275)(427.74074269,339.58457314)(427.47075058,339.2545743)
\curveto(427.20074323,338.9245738)(426.84574358,338.66457406)(426.40575058,338.4745743)
\curveto(426.31574411,338.43457429)(426.22074421,338.40457432)(426.12075058,338.3845743)
\curveto(426.0307444,338.35457437)(425.9307445,338.3245744)(425.82075058,338.2945743)
\curveto(425.76074467,338.27457445)(425.69574473,338.26457446)(425.62575058,338.2645743)
\curveto(425.56574486,338.26457446)(425.50574492,338.25957447)(425.44575058,338.2495743)
\lineto(425.31075058,338.2495743)
\curveto(425.25074518,338.2295745)(425.17074526,338.2245745)(425.07075058,338.2345743)
\curveto(424.97074546,338.23457449)(424.89074554,338.24457448)(424.83075058,338.2645743)
\lineto(424.74075058,338.2645743)
\curveto(424.69074574,338.27457445)(424.63574579,338.28457444)(424.57575058,338.2945743)
\curveto(424.51574591,338.29457443)(424.45574597,338.29957443)(424.39575058,338.3095743)
\curveto(424.20574622,338.35957437)(424.0307464,338.40957432)(423.87075058,338.4595743)
\curveto(423.71074672,338.50957422)(423.56074687,338.57957415)(423.42075058,338.6695743)
\lineto(423.24075058,338.7895743)
\curveto(423.19074724,338.8295739)(423.14074729,338.87457385)(423.09075058,338.9245743)
\lineto(423.00075058,338.9845743)
\curveto(422.97074746,339.00457372)(422.94074749,339.01957371)(422.91075058,339.0295743)
\curveto(422.82074761,339.05957367)(422.76574766,339.03957369)(422.74575058,338.9695743)
\curveto(422.69574773,338.89957383)(422.66074777,338.81457391)(422.64075058,338.7145743)
\curveto(422.6307478,338.6245741)(422.59574783,338.55457417)(422.53575058,338.5045743)
\curveto(422.47574795,338.46457426)(422.40574802,338.43957429)(422.32575058,338.4295743)
\lineto(422.05575058,338.4295743)
\lineto(421.33575058,338.4295743)
\lineto(421.11075058,338.4295743)
\curveto(421.04074939,338.41957431)(420.97574945,338.4245743)(420.91575058,338.4445743)
\curveto(420.77574965,338.49457423)(420.69574973,338.58457414)(420.67575058,338.7145743)
\curveto(420.66574976,338.85457387)(420.66074977,339.00957372)(420.66075058,339.1795743)
\lineto(420.66075058,348.3295743)
\lineto(420.66075058,348.6745743)
\curveto(420.66074977,348.79456393)(420.68574974,348.88956384)(420.73575058,348.9595743)
\curveto(420.77574965,349.0295637)(420.84574958,349.07456365)(420.94575058,349.0945743)
\curveto(420.96574946,349.10456362)(420.98574944,349.10456362)(421.00575058,349.0945743)
\curveto(421.03574939,349.09456363)(421.06074937,349.09956363)(421.08075058,349.1095743)
\lineto(422.02575058,349.1095743)
\curveto(422.20574822,349.10956362)(422.36074807,349.09956363)(422.49075058,349.0795743)
\curveto(422.62074781,349.06956366)(422.70574772,348.99456373)(422.74575058,348.8545743)
\curveto(422.77574765,348.75456397)(422.78574764,348.61956411)(422.77575058,348.4495743)
\curveto(422.76574766,348.28956444)(422.76074767,348.14956458)(422.76075058,348.0295743)
\lineto(422.76075058,346.3945743)
\lineto(422.76075058,346.0645743)
\curveto(422.76074767,345.95456677)(422.77074766,345.85956687)(422.79075058,345.7795743)
\curveto(422.80074763,345.729567)(422.81074762,345.68456704)(422.82075058,345.6445743)
\curveto(422.8307476,345.61456711)(422.85574757,345.59456713)(422.89575058,345.5845743)
\curveto(422.91574751,345.56456716)(422.94074749,345.55456717)(422.97075058,345.5545743)
\curveto(423.01074742,345.55456717)(423.04074739,345.55956717)(423.06075058,345.5695743)
\curveto(423.1307473,345.60956712)(423.19574723,345.64956708)(423.25575058,345.6895743)
\curveto(423.31574711,345.73956699)(423.38074705,345.78956694)(423.45075058,345.8395743)
\curveto(423.58074685,345.9295668)(423.71574671,346.00456672)(423.85575058,346.0645743)
\curveto(423.99574643,346.13456659)(424.15074628,346.19456653)(424.32075058,346.2445743)
\curveto(424.40074603,346.27456645)(424.48074595,346.28956644)(424.56075058,346.2895743)
\curveto(424.64074579,346.29956643)(424.72074571,346.31456641)(424.80075058,346.3345743)
\curveto(424.87074556,346.35456637)(424.94574548,346.36456636)(425.02575058,346.3645743)
\lineto(425.26575058,346.3645743)
\lineto(425.41575058,346.3645743)
\curveto(425.44574498,346.35456637)(425.48074495,346.34956638)(425.52075058,346.3495743)
\curveto(425.56074487,346.35956637)(425.60074483,346.35956637)(425.64075058,346.3495743)
\curveto(425.75074468,346.31956641)(425.85074458,346.29456643)(425.94075058,346.2745743)
\curveto(426.04074439,346.26456646)(426.13574429,346.23956649)(426.22575058,346.1995743)
\curveto(426.68574374,346.00956672)(427.06074337,345.76456696)(427.35075058,345.4645743)
\curveto(427.64074279,345.16456756)(427.88574254,344.78956794)(428.08575058,344.3395743)
\curveto(428.13574229,344.21956851)(428.17574225,344.09456863)(428.20575058,343.9645743)
\curveto(428.24574218,343.83456889)(428.28574214,343.69956903)(428.32575058,343.5595743)
\curveto(428.34574208,343.48956924)(428.35574207,343.41956931)(428.35575058,343.3495743)
\curveto(428.36574206,343.28956944)(428.38074205,343.21956951)(428.40075058,343.1395743)
\curveto(428.42074201,343.08956964)(428.425742,343.03456969)(428.41575058,342.9745743)
\curveto(428.41574201,342.91456981)(428.42074201,342.85456987)(428.43075058,342.7945743)
\lineto(428.43075058,342.6895743)
\moveto(426.21075058,341.2795743)
\curveto(426.24074419,341.37957135)(426.26574416,341.50457122)(426.28575058,341.6545743)
\curveto(426.31574411,341.80457092)(426.3307441,341.95457077)(426.33075058,342.1045743)
\curveto(426.34074409,342.26457046)(426.34074409,342.41957031)(426.33075058,342.5695743)
\curveto(426.3307441,342.72957)(426.31574411,342.86456986)(426.28575058,342.9745743)
\curveto(426.25574417,343.07456965)(426.23574419,343.16956956)(426.22575058,343.2595743)
\curveto(426.21574421,343.34956938)(426.19074424,343.43456929)(426.15075058,343.5145743)
\curveto(426.01074442,343.86456886)(425.81074462,344.15956857)(425.55075058,344.3995743)
\curveto(425.30074513,344.64956808)(424.9307455,344.77456795)(424.44075058,344.7745743)
\curveto(424.40074603,344.77456795)(424.36574606,344.76956796)(424.33575058,344.7595743)
\lineto(424.23075058,344.7595743)
\curveto(424.16074627,344.73956799)(424.09574633,344.71956801)(424.03575058,344.6995743)
\curveto(423.97574645,344.68956804)(423.91574651,344.67456805)(423.85575058,344.6545743)
\curveto(423.56574686,344.5245682)(423.34574708,344.33956839)(423.19575058,344.0995743)
\curveto(423.04574738,343.86956886)(422.92074751,343.60456912)(422.82075058,343.3045743)
\curveto(422.79074764,343.2245695)(422.77074766,343.13956959)(422.76075058,343.0495743)
\curveto(422.76074767,342.96956976)(422.75074768,342.88956984)(422.73075058,342.8095743)
\curveto(422.72074771,342.77956995)(422.71574771,342.72957)(422.71575058,342.6595743)
\curveto(422.70574772,342.61957011)(422.70074773,342.57957015)(422.70075058,342.5395743)
\curveto(422.71074772,342.49957023)(422.71074772,342.45957027)(422.70075058,342.4195743)
\curveto(422.68074775,342.33957039)(422.67574775,342.2295705)(422.68575058,342.0895743)
\curveto(422.69574773,341.94957078)(422.71074772,341.84957088)(422.73075058,341.7895743)
\curveto(422.75074768,341.69957103)(422.76074767,341.61457111)(422.76075058,341.5345743)
\curveto(422.77074766,341.45457127)(422.79074764,341.37457135)(422.82075058,341.2945743)
\curveto(422.91074752,341.01457171)(423.01574741,340.76957196)(423.13575058,340.5595743)
\curveto(423.26574716,340.35957237)(423.44574698,340.18957254)(423.67575058,340.0495743)
\curveto(423.83574659,339.94957278)(424.00074643,339.87957285)(424.17075058,339.8395743)
\curveto(424.19074624,339.83957289)(424.21074622,339.83457289)(424.23075058,339.8245743)
\lineto(424.32075058,339.8245743)
\curveto(424.35074608,339.81457291)(424.40074603,339.80457292)(424.47075058,339.7945743)
\curveto(424.54074589,339.79457293)(424.60074583,339.79957293)(424.65075058,339.8095743)
\curveto(424.75074568,339.8295729)(424.84074559,339.84457288)(424.92075058,339.8545743)
\curveto(425.01074542,339.87457285)(425.09574533,339.89957283)(425.17575058,339.9295743)
\curveto(425.45574497,340.05957267)(425.67074476,340.23957249)(425.82075058,340.4695743)
\curveto(425.98074445,340.69957203)(426.11074432,340.96957176)(426.21075058,341.2795743)
}
}
{
\newrgbcolor{curcolor}{0 0 0}
\pscustom[linestyle=none,fillstyle=solid,fillcolor=curcolor]
{
\newpath
\moveto(430.22067245,346.1545743)
\lineto(431.34567245,346.1545743)
\curveto(431.45567002,346.15456657)(431.55566992,346.14956658)(431.64567245,346.1395743)
\curveto(431.73566974,346.1295666)(431.80066967,346.09456663)(431.84067245,346.0345743)
\curveto(431.89066958,345.97456675)(431.92066955,345.88956684)(431.93067245,345.7795743)
\curveto(431.94066953,345.67956705)(431.94566953,345.57456715)(431.94567245,345.4645743)
\lineto(431.94567245,344.4145743)
\lineto(431.94567245,342.1795743)
\curveto(431.94566953,341.81957091)(431.96066951,341.47957125)(431.99067245,341.1595743)
\curveto(432.02066945,340.83957189)(432.11066936,340.57457215)(432.26067245,340.3645743)
\curveto(432.40066907,340.15457257)(432.62566885,340.00457272)(432.93567245,339.9145743)
\curveto(432.98566849,339.90457282)(433.02566845,339.89957283)(433.05567245,339.8995743)
\curveto(433.09566838,339.89957283)(433.14066833,339.89457283)(433.19067245,339.8845743)
\curveto(433.24066823,339.87457285)(433.29566818,339.86957286)(433.35567245,339.8695743)
\curveto(433.41566806,339.86957286)(433.46066801,339.87457285)(433.49067245,339.8845743)
\curveto(433.54066793,339.90457282)(433.58066789,339.90957282)(433.61067245,339.8995743)
\curveto(433.65066782,339.88957284)(433.69066778,339.89457283)(433.73067245,339.9145743)
\curveto(433.94066753,339.96457276)(434.10566737,340.0295727)(434.22567245,340.1095743)
\curveto(434.40566707,340.21957251)(434.54566693,340.35957237)(434.64567245,340.5295743)
\curveto(434.75566672,340.70957202)(434.83066664,340.90457182)(434.87067245,341.1145743)
\curveto(434.92066655,341.33457139)(434.95066652,341.57457115)(434.96067245,341.8345743)
\curveto(434.9706665,342.10457062)(434.9756665,342.38457034)(434.97567245,342.6745743)
\lineto(434.97567245,344.4895743)
\lineto(434.97567245,345.4645743)
\lineto(434.97567245,345.7345743)
\curveto(434.9756665,345.83456689)(434.99566648,345.91456681)(435.03567245,345.9745743)
\curveto(435.08566639,346.06456666)(435.16066631,346.11456661)(435.26067245,346.1245743)
\curveto(435.36066611,346.14456658)(435.48066599,346.15456657)(435.62067245,346.1545743)
\lineto(436.41567245,346.1545743)
\lineto(436.70067245,346.1545743)
\curveto(436.79066468,346.15456657)(436.86566461,346.13456659)(436.92567245,346.0945743)
\curveto(437.00566447,346.04456668)(437.05066442,345.96956676)(437.06067245,345.8695743)
\curveto(437.0706644,345.76956696)(437.0756644,345.65456707)(437.07567245,345.5245743)
\lineto(437.07567245,344.3845743)
\lineto(437.07567245,340.1695743)
\lineto(437.07567245,339.1045743)
\lineto(437.07567245,338.8045743)
\curveto(437.0756644,338.70457402)(437.05566442,338.6295741)(437.01567245,338.5795743)
\curveto(436.96566451,338.49957423)(436.89066458,338.45457427)(436.79067245,338.4445743)
\curveto(436.69066478,338.43457429)(436.58566489,338.4295743)(436.47567245,338.4295743)
\lineto(435.66567245,338.4295743)
\curveto(435.55566592,338.4295743)(435.45566602,338.43457429)(435.36567245,338.4445743)
\curveto(435.28566619,338.45457427)(435.22066625,338.49457423)(435.17067245,338.5645743)
\curveto(435.15066632,338.59457413)(435.13066634,338.63957409)(435.11067245,338.6995743)
\curveto(435.10066637,338.75957397)(435.08566639,338.81957391)(435.06567245,338.8795743)
\curveto(435.05566642,338.93957379)(435.04066643,338.99457373)(435.02067245,339.0445743)
\curveto(435.00066647,339.09457363)(434.9706665,339.1245736)(434.93067245,339.1345743)
\curveto(434.91066656,339.15457357)(434.88566659,339.15957357)(434.85567245,339.1495743)
\curveto(434.82566665,339.13957359)(434.80066667,339.1295736)(434.78067245,339.1195743)
\curveto(434.71066676,339.07957365)(434.65066682,339.03457369)(434.60067245,338.9845743)
\curveto(434.55066692,338.93457379)(434.49566698,338.88957384)(434.43567245,338.8495743)
\curveto(434.39566708,338.81957391)(434.35566712,338.78457394)(434.31567245,338.7445743)
\curveto(434.28566719,338.71457401)(434.24566723,338.68457404)(434.19567245,338.6545743)
\curveto(433.96566751,338.51457421)(433.69566778,338.40457432)(433.38567245,338.3245743)
\curveto(433.31566816,338.30457442)(433.24566823,338.29457443)(433.17567245,338.2945743)
\curveto(433.10566837,338.28457444)(433.03066844,338.26957446)(432.95067245,338.2495743)
\curveto(432.91066856,338.23957449)(432.86566861,338.23957449)(432.81567245,338.2495743)
\curveto(432.7756687,338.24957448)(432.73566874,338.24457448)(432.69567245,338.2345743)
\curveto(432.66566881,338.2245745)(432.60066887,338.2245745)(432.50067245,338.2345743)
\curveto(432.41066906,338.23457449)(432.35066912,338.23957449)(432.32067245,338.2495743)
\curveto(432.2706692,338.24957448)(432.22066925,338.25457447)(432.17067245,338.2645743)
\lineto(432.02067245,338.2645743)
\curveto(431.90066957,338.29457443)(431.78566969,338.31957441)(431.67567245,338.3395743)
\curveto(431.56566991,338.35957437)(431.45567002,338.38957434)(431.34567245,338.4295743)
\curveto(431.29567018,338.44957428)(431.25067022,338.46457426)(431.21067245,338.4745743)
\curveto(431.18067029,338.49457423)(431.14067033,338.51457421)(431.09067245,338.5345743)
\curveto(430.74067073,338.724574)(430.46067101,338.98957374)(430.25067245,339.3295743)
\curveto(430.12067135,339.53957319)(430.02567145,339.78957294)(429.96567245,340.0795743)
\curveto(429.90567157,340.37957235)(429.86567161,340.69457203)(429.84567245,341.0245743)
\curveto(429.83567164,341.36457136)(429.83067164,341.70957102)(429.83067245,342.0595743)
\curveto(429.84067163,342.41957031)(429.84567163,342.77456995)(429.84567245,343.1245743)
\lineto(429.84567245,345.1645743)
\curveto(429.84567163,345.29456743)(429.84067163,345.44456728)(429.83067245,345.6145743)
\curveto(429.83067164,345.79456693)(429.85567162,345.9245668)(429.90567245,346.0045743)
\curveto(429.93567154,346.05456667)(429.99567148,346.09956663)(430.08567245,346.1395743)
\curveto(430.14567133,346.13956659)(430.19067128,346.14456658)(430.22067245,346.1545743)
}
}
{
\newrgbcolor{curcolor}{0 0 0}
\pscustom[linestyle=none,fillstyle=solid,fillcolor=curcolor]
{
\newpath
\moveto(442.27692245,346.3795743)
\curveto(443.08691729,346.39956633)(443.76191662,346.27956645)(444.30192245,346.0195743)
\curveto(444.85191553,345.75956697)(445.28691509,345.38956734)(445.60692245,344.9095743)
\curveto(445.76691461,344.66956806)(445.88691449,344.39456833)(445.96692245,344.0845743)
\curveto(445.98691439,344.03456869)(446.00191438,343.96956876)(446.01192245,343.8895743)
\curveto(446.03191435,343.80956892)(446.03191435,343.73956899)(446.01192245,343.6795743)
\curveto(445.97191441,343.56956916)(445.90191448,343.50456922)(445.80192245,343.4845743)
\curveto(445.70191468,343.47456925)(445.5819148,343.46956926)(445.44192245,343.4695743)
\lineto(444.66192245,343.4695743)
\lineto(444.37692245,343.4695743)
\curveto(444.28691609,343.46956926)(444.21191617,343.48956924)(444.15192245,343.5295743)
\curveto(444.07191631,343.56956916)(444.01691636,343.6295691)(443.98692245,343.7095743)
\curveto(443.95691642,343.79956893)(443.91691646,343.88956884)(443.86692245,343.9795743)
\curveto(443.80691657,344.08956864)(443.74191664,344.18956854)(443.67192245,344.2795743)
\curveto(443.60191678,344.36956836)(443.52191686,344.44956828)(443.43192245,344.5195743)
\curveto(443.29191709,344.60956812)(443.13691724,344.67956805)(442.96692245,344.7295743)
\curveto(442.90691747,344.74956798)(442.84691753,344.75956797)(442.78692245,344.7595743)
\curveto(442.72691765,344.75956797)(442.67191771,344.76956796)(442.62192245,344.7895743)
\lineto(442.47192245,344.7895743)
\curveto(442.27191811,344.78956794)(442.11191827,344.76956796)(441.99192245,344.7295743)
\curveto(441.70191868,344.63956809)(441.46691891,344.49956823)(441.28692245,344.3095743)
\curveto(441.10691927,344.1295686)(440.96191942,343.90956882)(440.85192245,343.6495743)
\curveto(440.80191958,343.53956919)(440.76191962,343.41956931)(440.73192245,343.2895743)
\curveto(440.71191967,343.16956956)(440.68691969,343.03956969)(440.65692245,342.8995743)
\curveto(440.64691973,342.85956987)(440.64191974,342.81956991)(440.64192245,342.7795743)
\curveto(440.64191974,342.73956999)(440.63691974,342.69957003)(440.62692245,342.6595743)
\curveto(440.60691977,342.55957017)(440.59691978,342.41957031)(440.59692245,342.2395743)
\curveto(440.60691977,342.05957067)(440.62191976,341.91957081)(440.64192245,341.8195743)
\curveto(440.64191974,341.73957099)(440.64691973,341.68457104)(440.65692245,341.6545743)
\curveto(440.6769197,341.58457114)(440.68691969,341.51457121)(440.68692245,341.4445743)
\curveto(440.69691968,341.37457135)(440.71191967,341.30457142)(440.73192245,341.2345743)
\curveto(440.81191957,341.00457172)(440.90691947,340.79457193)(441.01692245,340.6045743)
\curveto(441.12691925,340.41457231)(441.26691911,340.25457247)(441.43692245,340.1245743)
\curveto(441.4769189,340.09457263)(441.53691884,340.05957267)(441.61692245,340.0195743)
\curveto(441.72691865,339.94957278)(441.83691854,339.90457282)(441.94692245,339.8845743)
\curveto(442.06691831,339.86457286)(442.21191817,339.84457288)(442.38192245,339.8245743)
\lineto(442.47192245,339.8245743)
\curveto(442.51191787,339.8245729)(442.54191784,339.8295729)(442.56192245,339.8395743)
\lineto(442.69692245,339.8395743)
\curveto(442.76691761,339.85957287)(442.83191755,339.87457285)(442.89192245,339.8845743)
\curveto(442.96191742,339.90457282)(443.02691735,339.9245728)(443.08692245,339.9445743)
\curveto(443.38691699,340.07457265)(443.61691676,340.26457246)(443.77692245,340.5145743)
\curveto(443.81691656,340.56457216)(443.85191653,340.61957211)(443.88192245,340.6795743)
\curveto(443.91191647,340.74957198)(443.93691644,340.80957192)(443.95692245,340.8595743)
\curveto(443.99691638,340.96957176)(444.03191635,341.06457166)(444.06192245,341.1445743)
\curveto(444.09191629,341.23457149)(444.16191622,341.30457142)(444.27192245,341.3545743)
\curveto(444.36191602,341.39457133)(444.50691587,341.40957132)(444.70692245,341.3995743)
\lineto(445.20192245,341.3995743)
\lineto(445.41192245,341.3995743)
\curveto(445.49191489,341.40957132)(445.55691482,341.40457132)(445.60692245,341.3845743)
\lineto(445.72692245,341.3845743)
\lineto(445.84692245,341.3545743)
\curveto(445.88691449,341.35457137)(445.91691446,341.34457138)(445.93692245,341.3245743)
\curveto(445.98691439,341.28457144)(446.01691436,341.2245715)(446.02692245,341.1445743)
\curveto(446.04691433,341.07457165)(446.04691433,340.99957173)(446.02692245,340.9195743)
\curveto(445.93691444,340.58957214)(445.82691455,340.29457243)(445.69692245,340.0345743)
\curveto(445.28691509,339.26457346)(444.63191575,338.729574)(443.73192245,338.4295743)
\curveto(443.63191675,338.39957433)(443.52691685,338.37957435)(443.41692245,338.3695743)
\curveto(443.30691707,338.34957438)(443.19691718,338.3245744)(443.08692245,338.2945743)
\curveto(443.02691735,338.28457444)(442.96691741,338.27957445)(442.90692245,338.2795743)
\curveto(442.84691753,338.27957445)(442.78691759,338.27457445)(442.72692245,338.2645743)
\lineto(442.56192245,338.2645743)
\curveto(442.51191787,338.24457448)(442.43691794,338.23957449)(442.33692245,338.2495743)
\curveto(442.23691814,338.24957448)(442.16191822,338.25457447)(442.11192245,338.2645743)
\curveto(442.03191835,338.28457444)(441.95691842,338.29457443)(441.88692245,338.2945743)
\curveto(441.82691855,338.28457444)(441.76191862,338.28957444)(441.69192245,338.3095743)
\lineto(441.54192245,338.3395743)
\curveto(441.49191889,338.33957439)(441.44191894,338.34457438)(441.39192245,338.3545743)
\curveto(441.2819191,338.38457434)(441.1769192,338.41457431)(441.07692245,338.4445743)
\curveto(440.9769194,338.47457425)(440.8819195,338.50957422)(440.79192245,338.5495743)
\curveto(440.32192006,338.74957398)(439.92692045,339.00457372)(439.60692245,339.3145743)
\curveto(439.28692109,339.63457309)(439.02692135,340.0295727)(438.82692245,340.4995743)
\curveto(438.7769216,340.58957214)(438.73692164,340.68457204)(438.70692245,340.7845743)
\lineto(438.61692245,341.1145743)
\curveto(438.60692177,341.15457157)(438.60192178,341.18957154)(438.60192245,341.2195743)
\curveto(438.60192178,341.25957147)(438.59192179,341.30457142)(438.57192245,341.3545743)
\curveto(438.55192183,341.4245713)(438.54192184,341.49457123)(438.54192245,341.5645743)
\curveto(438.54192184,341.64457108)(438.53192185,341.71957101)(438.51192245,341.7895743)
\lineto(438.51192245,342.0445743)
\curveto(438.49192189,342.09457063)(438.4819219,342.14957058)(438.48192245,342.2095743)
\curveto(438.4819219,342.27957045)(438.49192189,342.33957039)(438.51192245,342.3895743)
\curveto(438.52192186,342.43957029)(438.52192186,342.48457024)(438.51192245,342.5245743)
\curveto(438.50192188,342.56457016)(438.50192188,342.60457012)(438.51192245,342.6445743)
\curveto(438.53192185,342.71457001)(438.53692184,342.77956995)(438.52692245,342.8395743)
\curveto(438.52692185,342.89956983)(438.53692184,342.95956977)(438.55692245,343.0195743)
\curveto(438.60692177,343.19956953)(438.64692173,343.36956936)(438.67692245,343.5295743)
\curveto(438.70692167,343.69956903)(438.75192163,343.86456886)(438.81192245,344.0245743)
\curveto(439.03192135,344.53456819)(439.30692107,344.95956777)(439.63692245,345.2995743)
\curveto(439.9769204,345.63956709)(440.40691997,345.91456681)(440.92692245,346.1245743)
\curveto(441.06691931,346.18456654)(441.21191917,346.2245665)(441.36192245,346.2445743)
\curveto(441.51191887,346.27456645)(441.66691871,346.30956642)(441.82692245,346.3495743)
\curveto(441.90691847,346.35956637)(441.9819184,346.36456636)(442.05192245,346.3645743)
\curveto(442.12191826,346.36456636)(442.19691818,346.36956636)(442.27692245,346.3795743)
}
}
{
\newrgbcolor{curcolor}{0 0 0}
\pscustom[linestyle=none,fillstyle=solid,fillcolor=curcolor]
{
\newpath
\moveto(449.4202037,349.0195743)
\curveto(449.49020075,348.93956379)(449.52520072,348.81956391)(449.5252037,348.6595743)
\lineto(449.5252037,348.1945743)
\lineto(449.5252037,347.7895743)
\curveto(449.52520072,347.64956508)(449.49020075,347.55456517)(449.4202037,347.5045743)
\curveto(449.36020088,347.45456527)(449.28020096,347.4245653)(449.1802037,347.4145743)
\curveto(449.09020115,347.40456532)(448.99020125,347.39956533)(448.8802037,347.3995743)
\lineto(448.0402037,347.3995743)
\curveto(447.93020231,347.39956533)(447.83020241,347.40456532)(447.7402037,347.4145743)
\curveto(447.66020258,347.4245653)(447.59020265,347.45456527)(447.5302037,347.5045743)
\curveto(447.49020275,347.53456519)(447.46020278,347.58956514)(447.4402037,347.6695743)
\curveto(447.43020281,347.75956497)(447.42020282,347.85456487)(447.4102037,347.9545743)
\lineto(447.4102037,348.2845743)
\curveto(447.42020282,348.39456433)(447.42520282,348.48956424)(447.4252037,348.5695743)
\lineto(447.4252037,348.7795743)
\curveto(447.43520281,348.84956388)(447.45520279,348.90956382)(447.4852037,348.9595743)
\curveto(447.50520274,348.99956373)(447.53020271,349.0295637)(447.5602037,349.0495743)
\lineto(447.6802037,349.1095743)
\curveto(447.70020254,349.10956362)(447.72520252,349.10956362)(447.7552037,349.1095743)
\curveto(447.78520246,349.11956361)(447.81020243,349.1245636)(447.8302037,349.1245743)
\lineto(448.9252037,349.1245743)
\curveto(449.02520122,349.1245636)(449.12020112,349.11956361)(449.2102037,349.1095743)
\curveto(449.30020094,349.09956363)(449.37020087,349.06956366)(449.4202037,349.0195743)
\moveto(449.5252037,339.2545743)
\curveto(449.52520072,339.05457367)(449.52020072,338.88457384)(449.5102037,338.7445743)
\curveto(449.50020074,338.60457412)(449.41020083,338.50957422)(449.2402037,338.4595743)
\curveto(449.18020106,338.43957429)(449.11520113,338.4295743)(449.0452037,338.4295743)
\curveto(448.97520127,338.43957429)(448.90020134,338.44457428)(448.8202037,338.4445743)
\lineto(447.9802037,338.4445743)
\curveto(447.89020235,338.44457428)(447.80020244,338.44957428)(447.7102037,338.4595743)
\curveto(447.63020261,338.46957426)(447.57020267,338.49957423)(447.5302037,338.5495743)
\curveto(447.47020277,338.61957411)(447.43520281,338.70457402)(447.4252037,338.8045743)
\lineto(447.4252037,339.1495743)
\lineto(447.4252037,345.4795743)
\lineto(447.4252037,345.7795743)
\curveto(447.42520282,345.87956685)(447.4452028,345.95956677)(447.4852037,346.0195743)
\curveto(447.5452027,346.08956664)(447.63020261,346.13456659)(447.7402037,346.1545743)
\curveto(447.76020248,346.16456656)(447.78520246,346.16456656)(447.8152037,346.1545743)
\curveto(447.85520239,346.15456657)(447.88520236,346.15956657)(447.9052037,346.1695743)
\lineto(448.6552037,346.1695743)
\lineto(448.8502037,346.1695743)
\curveto(448.93020131,346.17956655)(448.99520125,346.17956655)(449.0452037,346.1695743)
\lineto(449.1652037,346.1695743)
\curveto(449.22520102,346.14956658)(449.28020096,346.13456659)(449.3302037,346.1245743)
\curveto(449.38020086,346.11456661)(449.42020082,346.08456664)(449.4502037,346.0345743)
\curveto(449.49020075,345.98456674)(449.51020073,345.91456681)(449.5102037,345.8245743)
\curveto(449.52020072,345.73456699)(449.52520072,345.63956709)(449.5252037,345.5395743)
\lineto(449.5252037,339.2545743)
}
}
{
\newrgbcolor{curcolor}{0 0 0}
\pscustom[linestyle=none,fillstyle=solid,fillcolor=curcolor]
{
\newpath
\moveto(458.9573912,342.6145743)
\curveto(458.93738267,342.66457006)(458.93238268,342.71957001)(458.9423912,342.7795743)
\curveto(458.95238266,342.83956989)(458.94738266,342.89456983)(458.9273912,342.9445743)
\curveto(458.91738269,342.98456974)(458.9123827,343.0245697)(458.9123912,343.0645743)
\curveto(458.9123827,343.10456962)(458.9073827,343.14456958)(458.8973912,343.1845743)
\lineto(458.8373912,343.4545743)
\curveto(458.81738279,343.54456918)(458.79238282,343.6295691)(458.7623912,343.7095743)
\curveto(458.7123829,343.84956888)(458.66738294,343.97956875)(458.6273912,344.0995743)
\curveto(458.58738302,344.2295685)(458.53238308,344.34956838)(458.4623912,344.4595743)
\curveto(458.39238322,344.56956816)(458.32238329,344.67456805)(458.2523912,344.7745743)
\curveto(458.19238342,344.87456785)(458.12238349,344.97456775)(458.0423912,345.0745743)
\curveto(457.96238365,345.18456754)(457.86238375,345.28456744)(457.7423912,345.3745743)
\curveto(457.63238398,345.47456725)(457.52238409,345.56456716)(457.4123912,345.6445743)
\curveto(457.08238453,345.87456685)(456.70238491,346.05456667)(456.2723912,346.1845743)
\curveto(455.85238576,346.31456641)(455.35238626,346.37456635)(454.7723912,346.3645743)
\curveto(454.70238691,346.35456637)(454.63238698,346.34956638)(454.5623912,346.3495743)
\curveto(454.49238712,346.34956638)(454.41738719,346.34456638)(454.3373912,346.3345743)
\curveto(454.18738742,346.29456643)(454.04238757,346.26456646)(453.9023912,346.2445743)
\curveto(453.76238785,346.2245665)(453.62738798,346.18956654)(453.4973912,346.1395743)
\curveto(453.38738822,346.08956664)(453.27738833,346.04456668)(453.1673912,346.0045743)
\curveto(453.05738855,345.96456676)(452.95238866,345.91956681)(452.8523912,345.8695743)
\curveto(452.49238912,345.63956709)(452.18738942,345.38456734)(451.9373912,345.1045743)
\curveto(451.68738992,344.83456789)(451.47239014,344.49456823)(451.2923912,344.0845743)
\curveto(451.24239037,343.96456876)(451.20239041,343.83956889)(451.1723912,343.7095743)
\curveto(451.14239047,343.58956914)(451.1073905,343.46456926)(451.0673912,343.3345743)
\curveto(451.04739056,343.28456944)(451.03739057,343.23456949)(451.0373912,343.1845743)
\curveto(451.03739057,343.14456958)(451.03239058,343.09956963)(451.0223912,343.0495743)
\curveto(451.00239061,342.99956973)(450.99239062,342.94456978)(450.9923912,342.8845743)
\curveto(451.00239061,342.83456989)(451.00239061,342.78456994)(450.9923912,342.7345743)
\lineto(450.9923912,342.6295743)
\curveto(450.97239064,342.56957016)(450.95739065,342.48457024)(450.9473912,342.3745743)
\curveto(450.94739066,342.26457046)(450.95739065,342.17957055)(450.9773912,342.1195743)
\lineto(450.9773912,341.9845743)
\curveto(450.97739063,341.94457078)(450.98239063,341.89957083)(450.9923912,341.8495743)
\curveto(451.0123906,341.76957096)(451.02239059,341.68457104)(451.0223912,341.5945743)
\curveto(451.02239059,341.51457121)(451.03239058,341.43457129)(451.0523912,341.3545743)
\curveto(451.07239054,341.30457142)(451.08239053,341.25957147)(451.0823912,341.2195743)
\curveto(451.08239053,341.17957155)(451.09239052,341.13457159)(451.1123912,341.0845743)
\curveto(451.14239047,340.97457175)(451.16739044,340.86957186)(451.1873912,340.7695743)
\curveto(451.21739039,340.66957206)(451.25739035,340.57457215)(451.3073912,340.4845743)
\curveto(451.47739013,340.09457263)(451.68738992,339.75957297)(451.9373912,339.4795743)
\curveto(452.18738942,339.19957353)(452.48738912,338.95457377)(452.8373912,338.7445743)
\curveto(452.94738866,338.68457404)(453.05238856,338.63457409)(453.1523912,338.5945743)
\curveto(453.26238835,338.55457417)(453.37738823,338.51457421)(453.4973912,338.4745743)
\curveto(453.58738802,338.43457429)(453.68238793,338.40457432)(453.7823912,338.3845743)
\curveto(453.88238773,338.36457436)(453.98238763,338.33957439)(454.0823912,338.3095743)
\curveto(454.13238748,338.29957443)(454.17238744,338.29457443)(454.2023912,338.2945743)
\curveto(454.24238737,338.29457443)(454.28238733,338.28957444)(454.3223912,338.2795743)
\curveto(454.37238724,338.25957447)(454.42238719,338.25457447)(454.4723912,338.2645743)
\curveto(454.53238708,338.26457446)(454.58738702,338.25957447)(454.6373912,338.2495743)
\lineto(454.7873912,338.2495743)
\curveto(454.84738676,338.2295745)(454.93238668,338.2245745)(455.0423912,338.2345743)
\curveto(455.15238646,338.23457449)(455.23238638,338.23957449)(455.2823912,338.2495743)
\curveto(455.3123863,338.24957448)(455.34238627,338.25457447)(455.3723912,338.2645743)
\lineto(455.4773912,338.2645743)
\curveto(455.52738608,338.27457445)(455.58238603,338.27957445)(455.6423912,338.2795743)
\curveto(455.70238591,338.27957445)(455.75738585,338.28957444)(455.8073912,338.3095743)
\curveto(455.93738567,338.33957439)(456.06238555,338.36957436)(456.1823912,338.3995743)
\curveto(456.3123853,338.41957431)(456.43738517,338.45457427)(456.5573912,338.5045743)
\curveto(457.03738457,338.70457402)(457.44738416,338.95457377)(457.7873912,339.2545743)
\curveto(458.12738348,339.55457317)(458.40238321,339.94457278)(458.6123912,340.4245743)
\curveto(458.66238295,340.5245722)(458.70238291,340.6295721)(458.7323912,340.7395743)
\curveto(458.76238285,340.85957187)(458.79738281,340.97457175)(458.8373912,341.0845743)
\curveto(458.84738276,341.15457157)(458.85738275,341.21957151)(458.8673912,341.2795743)
\curveto(458.87738273,341.33957139)(458.89238272,341.40457132)(458.9123912,341.4745743)
\curveto(458.93238268,341.55457117)(458.93738267,341.63457109)(458.9273912,341.7145743)
\curveto(458.92738268,341.79457093)(458.93738267,341.87457085)(458.9573912,341.9545743)
\lineto(458.9573912,342.1045743)
\curveto(458.97738263,342.16457056)(458.98738262,342.24957048)(458.9873912,342.3595743)
\curveto(458.98738262,342.46957026)(458.97738263,342.55457017)(458.9573912,342.6145743)
\moveto(456.8573912,342.0745743)
\curveto(456.84738476,342.0245707)(456.84238477,341.97457075)(456.8423912,341.9245743)
\lineto(456.8423912,341.7895743)
\curveto(456.83238478,341.74957098)(456.82738478,341.70957102)(456.8273912,341.6695743)
\curveto(456.82738478,341.63957109)(456.82238479,341.60457112)(456.8123912,341.5645743)
\curveto(456.78238483,341.45457127)(456.75738485,341.34957138)(456.7373912,341.2495743)
\curveto(456.71738489,341.14957158)(456.68738492,341.04957168)(456.6473912,340.9495743)
\curveto(456.53738507,340.69957203)(456.40238521,340.48957224)(456.2423912,340.3195743)
\curveto(456.08238553,340.14957258)(455.87238574,340.01457271)(455.6123912,339.9145743)
\curveto(455.54238607,339.88457284)(455.46738614,339.86457286)(455.3873912,339.8545743)
\curveto(455.3073863,339.84457288)(455.22738638,339.8295729)(455.1473912,339.8095743)
\lineto(455.0273912,339.8095743)
\curveto(454.98738662,339.79957293)(454.94238667,339.79457293)(454.8923912,339.7945743)
\lineto(454.7723912,339.8245743)
\curveto(454.73238688,339.83457289)(454.69738691,339.83457289)(454.6673912,339.8245743)
\curveto(454.63738697,339.8245729)(454.60238701,339.8295729)(454.5623912,339.8395743)
\curveto(454.47238714,339.85957287)(454.38238723,339.88457284)(454.2923912,339.9145743)
\curveto(454.2123874,339.94457278)(454.13738747,339.98457274)(454.0673912,340.0345743)
\curveto(453.81738779,340.18457254)(453.63238798,340.34957238)(453.5123912,340.5295743)
\curveto(453.40238821,340.71957201)(453.29738831,340.96457176)(453.1973912,341.2645743)
\curveto(453.17738843,341.34457138)(453.16238845,341.41957131)(453.1523912,341.4895743)
\curveto(453.14238847,341.56957116)(453.12738848,341.64957108)(453.1073912,341.7295743)
\lineto(453.1073912,341.8645743)
\curveto(453.08738852,341.93457079)(453.07238854,342.03957069)(453.0623912,342.1795743)
\curveto(453.06238855,342.31957041)(453.07238854,342.4245703)(453.0923912,342.4945743)
\lineto(453.0923912,342.6445743)
\curveto(453.09238852,342.69457003)(453.09738851,342.74456998)(453.1073912,342.7945743)
\curveto(453.12738848,342.90456982)(453.14238847,343.01456971)(453.1523912,343.1245743)
\curveto(453.17238844,343.23456949)(453.19738841,343.33956939)(453.2273912,343.4395743)
\curveto(453.31738829,343.70956902)(453.43738817,343.94456878)(453.5873912,344.1445743)
\curveto(453.74738786,344.35456837)(453.95238766,344.51456821)(454.2023912,344.6245743)
\curveto(454.25238736,344.65456807)(454.3073873,344.67456805)(454.3673912,344.6845743)
\lineto(454.5773912,344.7445743)
\curveto(454.607387,344.75456797)(454.64238697,344.75456797)(454.6823912,344.7445743)
\curveto(454.72238689,344.74456798)(454.75738685,344.75456797)(454.7873912,344.7745743)
\lineto(455.0573912,344.7745743)
\curveto(455.14738646,344.78456794)(455.23238638,344.77956795)(455.3123912,344.7595743)
\curveto(455.38238623,344.73956799)(455.44738616,344.71956801)(455.5073912,344.6995743)
\curveto(455.56738604,344.68956804)(455.62738598,344.67456805)(455.6873912,344.6545743)
\curveto(455.93738567,344.54456818)(456.13738547,344.39456833)(456.2873912,344.2045743)
\curveto(456.43738517,344.0245687)(456.56738504,343.80456892)(456.6773912,343.5445743)
\curveto(456.7073849,343.46456926)(456.72738488,343.37956935)(456.7373912,343.2895743)
\lineto(456.7973912,343.0495743)
\curveto(456.8073848,343.0295697)(456.8123848,342.99956973)(456.8123912,342.9595743)
\curveto(456.82238479,342.90956982)(456.82738478,342.85456987)(456.8273912,342.7945743)
\curveto(456.82738478,342.73456999)(456.83738477,342.67957005)(456.8573912,342.6295743)
\lineto(456.8573912,342.5095743)
\curveto(456.86738474,342.45957027)(456.87238474,342.38457034)(456.8723912,342.2845743)
\curveto(456.87238474,342.19457053)(456.86738474,342.1245706)(456.8573912,342.0745743)
\moveto(455.6273912,349.2445743)
\lineto(456.6923912,349.2445743)
\curveto(456.77238484,349.24456348)(456.86738474,349.24456348)(456.9773912,349.2445743)
\curveto(457.08738452,349.24456348)(457.16738444,349.2295635)(457.2173912,349.1995743)
\curveto(457.23738437,349.18956354)(457.24738436,349.17456355)(457.2473912,349.1545743)
\curveto(457.25738435,349.14456358)(457.27238434,349.13456359)(457.2923912,349.1245743)
\curveto(457.30238431,349.00456372)(457.25238436,348.89956383)(457.1423912,348.8095743)
\curveto(457.04238457,348.71956401)(456.95738465,348.63956409)(456.8873912,348.5695743)
\curveto(456.8073848,348.49956423)(456.72738488,348.4245643)(456.6473912,348.3445743)
\curveto(456.57738503,348.27456445)(456.50238511,348.20956452)(456.4223912,348.1495743)
\curveto(456.38238523,348.11956461)(456.34738526,348.08456464)(456.3173912,348.0445743)
\curveto(456.29738531,348.01456471)(456.26738534,347.98956474)(456.2273912,347.9695743)
\curveto(456.2073854,347.93956479)(456.18238543,347.91456481)(456.1523912,347.8945743)
\lineto(456.0023912,347.7445743)
\lineto(455.8523912,347.6245743)
\lineto(455.8073912,347.5795743)
\curveto(455.8073858,347.56956516)(455.79738581,347.55456517)(455.7773912,347.5345743)
\curveto(455.69738591,347.47456525)(455.61738599,347.40956532)(455.5373912,347.3395743)
\curveto(455.46738614,347.26956546)(455.37738623,347.21456551)(455.2673912,347.1745743)
\curveto(455.22738638,347.16456556)(455.18738642,347.15956557)(455.1473912,347.1595743)
\curveto(455.11738649,347.15956557)(455.07738653,347.15456557)(455.0273912,347.1445743)
\curveto(454.99738661,347.13456559)(454.95738665,347.1295656)(454.9073912,347.1295743)
\curveto(454.85738675,347.13956559)(454.8123868,347.14456558)(454.7723912,347.1445743)
\lineto(454.4273912,347.1445743)
\curveto(454.3073873,347.14456558)(454.21738739,347.16956556)(454.1573912,347.2195743)
\curveto(454.09738751,347.25956547)(454.08238753,347.3295654)(454.1123912,347.4295743)
\curveto(454.13238748,347.50956522)(454.16738744,347.57956515)(454.2173912,347.6395743)
\curveto(454.26738734,347.70956502)(454.3123873,347.77956495)(454.3523912,347.8495743)
\curveto(454.45238716,347.98956474)(454.54738706,348.1245646)(454.6373912,348.2545743)
\curveto(454.72738688,348.38456434)(454.81738679,348.51956421)(454.9073912,348.6595743)
\curveto(454.95738665,348.73956399)(455.0073866,348.8245639)(455.0573912,348.9145743)
\curveto(455.11738649,349.00456372)(455.18238643,349.07456365)(455.2523912,349.1245743)
\curveto(455.29238632,349.15456357)(455.36238625,349.18956354)(455.4623912,349.2295743)
\curveto(455.48238613,349.23956349)(455.5073861,349.23956349)(455.5373912,349.2295743)
\curveto(455.57738603,349.2295635)(455.607386,349.23456349)(455.6273912,349.2445743)
}
}
{
\newrgbcolor{curcolor}{0 0 0}
\pscustom[linestyle=none,fillstyle=solid,fillcolor=curcolor]
{
\newpath
\moveto(464.78231308,346.3645743)
\curveto(465.38230727,346.38456634)(465.88230677,346.29956643)(466.28231308,346.1095743)
\curveto(466.68230597,345.91956681)(466.99730566,345.63956709)(467.22731308,345.2695743)
\curveto(467.29730536,345.15956757)(467.3523053,345.03956769)(467.39231308,344.9095743)
\curveto(467.43230522,344.78956794)(467.47230518,344.66456806)(467.51231308,344.5345743)
\curveto(467.53230512,344.45456827)(467.54230511,344.37956835)(467.54231308,344.3095743)
\curveto(467.5523051,344.23956849)(467.56730509,344.16956856)(467.58731308,344.0995743)
\curveto(467.58730507,344.03956869)(467.59230506,343.99956873)(467.60231308,343.9795743)
\curveto(467.62230503,343.83956889)(467.63230502,343.69456903)(467.63231308,343.5445743)
\lineto(467.63231308,343.1095743)
\lineto(467.63231308,341.7745743)
\lineto(467.63231308,339.3445743)
\curveto(467.63230502,339.15457357)(467.62730503,338.96957376)(467.61731308,338.7895743)
\curveto(467.61730504,338.61957411)(467.54730511,338.50957422)(467.40731308,338.4595743)
\curveto(467.34730531,338.43957429)(467.27730538,338.4295743)(467.19731308,338.4295743)
\lineto(466.95731308,338.4295743)
\lineto(466.14731308,338.4295743)
\curveto(466.02730663,338.4295743)(465.91730674,338.43457429)(465.81731308,338.4445743)
\curveto(465.72730693,338.46457426)(465.657307,338.50957422)(465.60731308,338.5795743)
\curveto(465.56730709,338.63957409)(465.54230711,338.71457401)(465.53231308,338.8045743)
\lineto(465.53231308,339.1195743)
\lineto(465.53231308,340.1695743)
\lineto(465.53231308,342.4045743)
\curveto(465.53230712,342.77456995)(465.51730714,343.11456961)(465.48731308,343.4245743)
\curveto(465.4573072,343.74456898)(465.36730729,344.01456871)(465.21731308,344.2345743)
\curveto(465.07730758,344.43456829)(464.87230778,344.57456815)(464.60231308,344.6545743)
\curveto(464.5523081,344.67456805)(464.49730816,344.68456804)(464.43731308,344.6845743)
\curveto(464.38730827,344.68456804)(464.33230832,344.69456803)(464.27231308,344.7145743)
\curveto(464.22230843,344.724568)(464.1573085,344.724568)(464.07731308,344.7145743)
\curveto(464.00730865,344.71456801)(463.9523087,344.70956802)(463.91231308,344.6995743)
\curveto(463.87230878,344.68956804)(463.83730882,344.68456804)(463.80731308,344.6845743)
\curveto(463.77730888,344.68456804)(463.74730891,344.67956805)(463.71731308,344.6695743)
\curveto(463.48730917,344.60956812)(463.30230935,344.5295682)(463.16231308,344.4295743)
\curveto(462.84230981,344.19956853)(462.65231,343.86456886)(462.59231308,343.4245743)
\curveto(462.53231012,342.98456974)(462.50231015,342.48957024)(462.50231308,341.9395743)
\lineto(462.50231308,340.0645743)
\lineto(462.50231308,339.1495743)
\lineto(462.50231308,338.8795743)
\curveto(462.50231015,338.78957394)(462.48731017,338.71457401)(462.45731308,338.6545743)
\curveto(462.40731025,338.54457418)(462.32731033,338.47957425)(462.21731308,338.4595743)
\curveto(462.10731055,338.43957429)(461.97231068,338.4295743)(461.81231308,338.4295743)
\lineto(461.06231308,338.4295743)
\curveto(460.9523117,338.4295743)(460.84231181,338.43457429)(460.73231308,338.4445743)
\curveto(460.62231203,338.45457427)(460.54231211,338.48957424)(460.49231308,338.5495743)
\curveto(460.42231223,338.63957409)(460.38731227,338.76957396)(460.38731308,338.9395743)
\curveto(460.39731226,339.10957362)(460.40231225,339.26957346)(460.40231308,339.4195743)
\lineto(460.40231308,341.4595743)
\lineto(460.40231308,344.7595743)
\lineto(460.40231308,345.5245743)
\lineto(460.40231308,345.8245743)
\curveto(460.41231224,345.91456681)(460.44231221,345.98956674)(460.49231308,346.0495743)
\curveto(460.51231214,346.07956665)(460.54231211,346.09956663)(460.58231308,346.1095743)
\curveto(460.63231202,346.1295666)(460.68231197,346.14456658)(460.73231308,346.1545743)
\lineto(460.80731308,346.1545743)
\curveto(460.8573118,346.16456656)(460.90731175,346.16956656)(460.95731308,346.1695743)
\lineto(461.12231308,346.1695743)
\lineto(461.75231308,346.1695743)
\curveto(461.83231082,346.16956656)(461.90731075,346.16456656)(461.97731308,346.1545743)
\curveto(462.0573106,346.15456657)(462.12731053,346.14456658)(462.18731308,346.1245743)
\curveto(462.2573104,346.09456663)(462.30231035,346.04956668)(462.32231308,345.9895743)
\curveto(462.3523103,345.9295668)(462.37731028,345.85956687)(462.39731308,345.7795743)
\curveto(462.40731025,345.73956699)(462.40731025,345.70456702)(462.39731308,345.6745743)
\curveto(462.39731026,345.64456708)(462.40731025,345.61456711)(462.42731308,345.5845743)
\curveto(462.44731021,345.53456719)(462.46231019,345.50456722)(462.47231308,345.4945743)
\curveto(462.49231016,345.48456724)(462.51731014,345.46956726)(462.54731308,345.4495743)
\curveto(462.65731,345.43956729)(462.74730991,345.47456725)(462.81731308,345.5545743)
\curveto(462.88730977,345.64456708)(462.96230969,345.71456701)(463.04231308,345.7645743)
\curveto(463.31230934,345.96456676)(463.61230904,346.1245666)(463.94231308,346.2445743)
\curveto(464.03230862,346.27456645)(464.12230853,346.29456643)(464.21231308,346.3045743)
\curveto(464.31230834,346.31456641)(464.41730824,346.3295664)(464.52731308,346.3495743)
\curveto(464.5573081,346.35956637)(464.60230805,346.35956637)(464.66231308,346.3495743)
\curveto(464.72230793,346.34956638)(464.76230789,346.35456637)(464.78231308,346.3645743)
}
}
{
\newrgbcolor{curcolor}{0 0 0}
\pscustom[linestyle=none,fillstyle=solid,fillcolor=curcolor]
{
}
}
{
\newrgbcolor{curcolor}{0 0 0}
\pscustom[linestyle=none,fillstyle=solid,fillcolor=curcolor]
{
\newpath
\moveto(481.01371933,339.2845743)
\lineto(481.01371933,338.8645743)
\curveto(481.01371096,338.73457399)(480.98371099,338.6295741)(480.92371933,338.5495743)
\curveto(480.8737111,338.49957423)(480.80871116,338.46457426)(480.72871933,338.4445743)
\curveto(480.64871132,338.43457429)(480.55871141,338.4295743)(480.45871933,338.4295743)
\lineto(479.63371933,338.4295743)
\lineto(479.34871933,338.4295743)
\curveto(479.2687127,338.43957429)(479.20371277,338.46457426)(479.15371933,338.5045743)
\curveto(479.08371289,338.55457417)(479.04371293,338.61957411)(479.03371933,338.6995743)
\curveto(479.02371295,338.77957395)(479.00371297,338.85957387)(478.97371933,338.9395743)
\curveto(478.95371302,338.95957377)(478.93371304,338.97457375)(478.91371933,338.9845743)
\curveto(478.90371307,339.00457372)(478.88871308,339.0245737)(478.86871933,339.0445743)
\curveto(478.75871321,339.04457368)(478.67871329,339.01957371)(478.62871933,338.9695743)
\lineto(478.47871933,338.8195743)
\curveto(478.40871356,338.76957396)(478.34371363,338.724574)(478.28371933,338.6845743)
\curveto(478.22371375,338.65457407)(478.15871381,338.61457411)(478.08871933,338.5645743)
\curveto(478.04871392,338.54457418)(478.00371397,338.5245742)(477.95371933,338.5045743)
\curveto(477.91371406,338.48457424)(477.8687141,338.46457426)(477.81871933,338.4445743)
\curveto(477.67871429,338.39457433)(477.52871444,338.34957438)(477.36871933,338.3095743)
\curveto(477.31871465,338.28957444)(477.2737147,338.27957445)(477.23371933,338.2795743)
\curveto(477.19371478,338.27957445)(477.15371482,338.27457445)(477.11371933,338.2645743)
\lineto(476.97871933,338.2645743)
\curveto(476.94871502,338.25457447)(476.90871506,338.24957448)(476.85871933,338.2495743)
\lineto(476.72371933,338.2495743)
\curveto(476.66371531,338.2295745)(476.5737154,338.2245745)(476.45371933,338.2345743)
\curveto(476.33371564,338.23457449)(476.24871572,338.24457448)(476.19871933,338.2645743)
\curveto(476.12871584,338.28457444)(476.06371591,338.29457443)(476.00371933,338.2945743)
\curveto(475.95371602,338.28457444)(475.89871607,338.28957444)(475.83871933,338.3095743)
\lineto(475.47871933,338.4295743)
\curveto(475.3687166,338.45957427)(475.25871671,338.49957423)(475.14871933,338.5495743)
\curveto(474.79871717,338.69957403)(474.48371749,338.9295738)(474.20371933,339.2395743)
\curveto(473.93371804,339.55957317)(473.71871825,339.89457283)(473.55871933,340.2445743)
\curveto(473.50871846,340.35457237)(473.4687185,340.45957227)(473.43871933,340.5595743)
\curveto(473.40871856,340.66957206)(473.3737186,340.77957195)(473.33371933,340.8895743)
\curveto(473.32371865,340.9295718)(473.31871865,340.96457176)(473.31871933,340.9945743)
\curveto(473.31871865,341.03457169)(473.30871866,341.07957165)(473.28871933,341.1295743)
\curveto(473.2687187,341.20957152)(473.24871872,341.29457143)(473.22871933,341.3845743)
\curveto(473.21871875,341.48457124)(473.20371877,341.58457114)(473.18371933,341.6845743)
\curveto(473.1737188,341.71457101)(473.1687188,341.74957098)(473.16871933,341.7895743)
\curveto(473.17871879,341.8295709)(473.17871879,341.86457086)(473.16871933,341.8945743)
\lineto(473.16871933,342.0295743)
\curveto(473.1687188,342.07957065)(473.16371881,342.1295706)(473.15371933,342.1795743)
\curveto(473.14371883,342.2295705)(473.13871883,342.28457044)(473.13871933,342.3445743)
\curveto(473.13871883,342.41457031)(473.14371883,342.46957026)(473.15371933,342.5095743)
\curveto(473.16371881,342.55957017)(473.1687188,342.60457012)(473.16871933,342.6445743)
\lineto(473.16871933,342.7945743)
\curveto(473.17871879,342.84456988)(473.17871879,342.88956984)(473.16871933,342.9295743)
\curveto(473.1687188,342.97956975)(473.17871879,343.0295697)(473.19871933,343.0795743)
\curveto(473.21871875,343.18956954)(473.23371874,343.29456943)(473.24371933,343.3945743)
\curveto(473.26371871,343.49456923)(473.28871868,343.59456913)(473.31871933,343.6945743)
\curveto(473.35871861,343.81456891)(473.39371858,343.9295688)(473.42371933,344.0395743)
\curveto(473.45371852,344.14956858)(473.49371848,344.25956847)(473.54371933,344.3695743)
\curveto(473.68371829,344.66956806)(473.85871811,344.95456777)(474.06871933,345.2245743)
\curveto(474.08871788,345.25456747)(474.11371786,345.27956745)(474.14371933,345.2995743)
\curveto(474.18371779,345.3295674)(474.21371776,345.35956737)(474.23371933,345.3895743)
\curveto(474.2737177,345.43956729)(474.31371766,345.48456724)(474.35371933,345.5245743)
\curveto(474.39371758,345.56456716)(474.43871753,345.60456712)(474.48871933,345.6445743)
\curveto(474.52871744,345.66456706)(474.56371741,345.68956704)(474.59371933,345.7195743)
\curveto(474.62371735,345.75956697)(474.65871731,345.78956694)(474.69871933,345.8095743)
\curveto(474.94871702,345.97956675)(475.23871673,346.11956661)(475.56871933,346.2295743)
\curveto(475.63871633,346.24956648)(475.70871626,346.26456646)(475.77871933,346.2745743)
\curveto(475.85871611,346.28456644)(475.93871603,346.29956643)(476.01871933,346.3195743)
\curveto(476.08871588,346.33956639)(476.17871579,346.34956638)(476.28871933,346.3495743)
\curveto(476.39871557,346.35956637)(476.50871546,346.36456636)(476.61871933,346.3645743)
\curveto(476.72871524,346.36456636)(476.83371514,346.35956637)(476.93371933,346.3495743)
\curveto(477.04371493,346.33956639)(477.13371484,346.3245664)(477.20371933,346.3045743)
\curveto(477.35371462,346.25456647)(477.49871447,346.20956652)(477.63871933,346.1695743)
\curveto(477.77871419,346.1295666)(477.90871406,346.07456665)(478.02871933,346.0045743)
\curveto(478.09871387,345.95456677)(478.16371381,345.90456682)(478.22371933,345.8545743)
\curveto(478.28371369,345.81456691)(478.34871362,345.76956696)(478.41871933,345.7195743)
\curveto(478.45871351,345.68956704)(478.51371346,345.64956708)(478.58371933,345.5995743)
\curveto(478.66371331,345.54956718)(478.73871323,345.54956718)(478.80871933,345.5995743)
\curveto(478.84871312,345.61956711)(478.8687131,345.65456707)(478.86871933,345.7045743)
\curveto(478.8687131,345.75456697)(478.87871309,345.80456692)(478.89871933,345.8545743)
\lineto(478.89871933,346.0045743)
\curveto(478.90871306,346.03456669)(478.91371306,346.06956666)(478.91371933,346.1095743)
\lineto(478.91371933,346.2295743)
\lineto(478.91371933,348.2695743)
\curveto(478.91371306,348.37956435)(478.90871306,348.49956423)(478.89871933,348.6295743)
\curveto(478.89871307,348.76956396)(478.92371305,348.87456385)(478.97371933,348.9445743)
\curveto(479.01371296,349.0245637)(479.08871288,349.07456365)(479.19871933,349.0945743)
\curveto(479.21871275,349.10456362)(479.23871273,349.10456362)(479.25871933,349.0945743)
\curveto(479.27871269,349.09456363)(479.29871267,349.09956363)(479.31871933,349.1095743)
\lineto(480.38371933,349.1095743)
\curveto(480.50371147,349.10956362)(480.61371136,349.10456362)(480.71371933,349.0945743)
\curveto(480.81371116,349.08456364)(480.88871108,349.04456368)(480.93871933,348.9745743)
\curveto(480.98871098,348.89456383)(481.01371096,348.78956394)(481.01371933,348.6595743)
\lineto(481.01371933,348.2995743)
\lineto(481.01371933,339.2845743)
\moveto(478.97371933,342.2245743)
\curveto(478.98371299,342.26457046)(478.98371299,342.30457042)(478.97371933,342.3445743)
\lineto(478.97371933,342.4795743)
\curveto(478.973713,342.57957015)(478.968713,342.67957005)(478.95871933,342.7795743)
\curveto(478.94871302,342.87956985)(478.93371304,342.96956976)(478.91371933,343.0495743)
\curveto(478.89371308,343.15956957)(478.8737131,343.25956947)(478.85371933,343.3495743)
\curveto(478.84371313,343.43956929)(478.81871315,343.5245692)(478.77871933,343.6045743)
\curveto(478.63871333,343.96456876)(478.43371354,344.24956848)(478.16371933,344.4595743)
\curveto(477.90371407,344.66956806)(477.52371445,344.77456795)(477.02371933,344.7745743)
\curveto(476.96371501,344.77456795)(476.88371509,344.76456796)(476.78371933,344.7445743)
\curveto(476.70371527,344.724568)(476.62871534,344.70456802)(476.55871933,344.6845743)
\curveto(476.49871547,344.67456805)(476.43871553,344.65456807)(476.37871933,344.6245743)
\curveto(476.10871586,344.51456821)(475.89871607,344.34456838)(475.74871933,344.1145743)
\curveto(475.59871637,343.88456884)(475.47871649,343.6245691)(475.38871933,343.3345743)
\curveto(475.35871661,343.23456949)(475.33871663,343.13456959)(475.32871933,343.0345743)
\curveto(475.31871665,342.93456979)(475.29871667,342.8295699)(475.26871933,342.7195743)
\lineto(475.26871933,342.5095743)
\curveto(475.24871672,342.41957031)(475.24371673,342.29457043)(475.25371933,342.1345743)
\curveto(475.26371671,341.98457074)(475.27871669,341.87457085)(475.29871933,341.8045743)
\lineto(475.29871933,341.7145743)
\curveto(475.30871666,341.69457103)(475.31371666,341.67457105)(475.31371933,341.6545743)
\curveto(475.33371664,341.57457115)(475.34871662,341.49957123)(475.35871933,341.4295743)
\curveto(475.37871659,341.35957137)(475.39871657,341.28457144)(475.41871933,341.2045743)
\curveto(475.58871638,340.68457204)(475.87871609,340.29957243)(476.28871933,340.0495743)
\curveto(476.41871555,339.95957277)(476.59871537,339.88957284)(476.82871933,339.8395743)
\curveto(476.8687151,339.8295729)(476.92871504,339.8245729)(477.00871933,339.8245743)
\curveto(477.03871493,339.81457291)(477.08371489,339.80457292)(477.14371933,339.7945743)
\curveto(477.21371476,339.79457293)(477.2687147,339.79957293)(477.30871933,339.8095743)
\curveto(477.38871458,339.8295729)(477.4687145,339.84457288)(477.54871933,339.8545743)
\curveto(477.62871434,339.86457286)(477.70871426,339.88457284)(477.78871933,339.9145743)
\curveto(478.03871393,340.0245727)(478.23871373,340.16457256)(478.38871933,340.3345743)
\curveto(478.53871343,340.50457222)(478.6687133,340.71957201)(478.77871933,340.9795743)
\curveto(478.81871315,341.06957166)(478.84871312,341.15957157)(478.86871933,341.2495743)
\curveto(478.88871308,341.34957138)(478.90871306,341.45457127)(478.92871933,341.5645743)
\curveto(478.93871303,341.61457111)(478.93871303,341.65957107)(478.92871933,341.6995743)
\curveto(478.92871304,341.74957098)(478.93871303,341.79957093)(478.95871933,341.8495743)
\curveto(478.968713,341.87957085)(478.973713,341.91457081)(478.97371933,341.9545743)
\lineto(478.97371933,342.0895743)
\lineto(478.97371933,342.2245743)
}
}
{
\newrgbcolor{curcolor}{0 0 0}
\pscustom[linestyle=none,fillstyle=solid,fillcolor=curcolor]
{
\newpath
\moveto(489.9586412,342.3745743)
\curveto(489.97863304,342.29457043)(489.97863304,342.20457052)(489.9586412,342.1045743)
\curveto(489.93863308,342.00457072)(489.90363311,341.93957079)(489.8536412,341.9095743)
\curveto(489.80363321,341.86957086)(489.72863329,341.83957089)(489.6286412,341.8195743)
\curveto(489.53863348,341.80957092)(489.43363358,341.79957093)(489.3136412,341.7895743)
\lineto(488.9686412,341.7895743)
\curveto(488.85863416,341.79957093)(488.75863426,341.80457092)(488.6686412,341.8045743)
\lineto(485.0086412,341.8045743)
\lineto(484.7986412,341.8045743)
\curveto(484.73863828,341.80457092)(484.68363833,341.79457093)(484.6336412,341.7745743)
\curveto(484.55363846,341.73457099)(484.50363851,341.69457103)(484.4836412,341.6545743)
\curveto(484.46363855,341.63457109)(484.44363857,341.59457113)(484.4236412,341.5345743)
\curveto(484.40363861,341.48457124)(484.39863862,341.43457129)(484.4086412,341.3845743)
\curveto(484.42863859,341.3245714)(484.43863858,341.26457146)(484.4386412,341.2045743)
\curveto(484.44863857,341.15457157)(484.46363855,341.09957163)(484.4836412,341.0395743)
\curveto(484.56363845,340.79957193)(484.65863836,340.59957213)(484.7686412,340.4395743)
\curveto(484.88863813,340.28957244)(485.04863797,340.15457257)(485.2486412,340.0345743)
\curveto(485.32863769,339.98457274)(485.40863761,339.94957278)(485.4886412,339.9295743)
\curveto(485.57863744,339.91957281)(485.66863735,339.89957283)(485.7586412,339.8695743)
\curveto(485.83863718,339.84957288)(485.94863707,339.83457289)(486.0886412,339.8245743)
\curveto(486.22863679,339.81457291)(486.34863667,339.81957291)(486.4486412,339.8395743)
\lineto(486.5836412,339.8395743)
\curveto(486.68363633,339.85957287)(486.77363624,339.87957285)(486.8536412,339.8995743)
\curveto(486.94363607,339.9295728)(487.02863599,339.95957277)(487.1086412,339.9895743)
\curveto(487.20863581,340.03957269)(487.3186357,340.10457262)(487.4386412,340.1845743)
\curveto(487.56863545,340.26457246)(487.66363535,340.34457238)(487.7236412,340.4245743)
\curveto(487.77363524,340.49457223)(487.82363519,340.55957217)(487.8736412,340.6195743)
\curveto(487.93363508,340.68957204)(488.00363501,340.73957199)(488.0836412,340.7695743)
\curveto(488.18363483,340.81957191)(488.30863471,340.83957189)(488.4586412,340.8295743)
\lineto(488.8936412,340.8295743)
\lineto(489.0736412,340.8295743)
\curveto(489.14363387,340.83957189)(489.20363381,340.83457189)(489.2536412,340.8145743)
\lineto(489.4036412,340.8145743)
\curveto(489.50363351,340.79457193)(489.57363344,340.76957196)(489.6136412,340.7395743)
\curveto(489.65363336,340.71957201)(489.67363334,340.67457205)(489.6736412,340.6045743)
\curveto(489.68363333,340.53457219)(489.67863334,340.47457225)(489.6586412,340.4245743)
\curveto(489.60863341,340.28457244)(489.55363346,340.15957257)(489.4936412,340.0495743)
\curveto(489.43363358,339.93957279)(489.36363365,339.8295729)(489.2836412,339.7195743)
\curveto(489.06363395,339.38957334)(488.8136342,339.1245736)(488.5336412,338.9245743)
\curveto(488.25363476,338.724574)(487.90363511,338.55457417)(487.4836412,338.4145743)
\curveto(487.37363564,338.37457435)(487.26363575,338.34957438)(487.1536412,338.3395743)
\curveto(487.04363597,338.3295744)(486.92863609,338.30957442)(486.8086412,338.2795743)
\curveto(486.76863625,338.26957446)(486.72363629,338.26957446)(486.6736412,338.2795743)
\curveto(486.63363638,338.27957445)(486.59363642,338.27457445)(486.5536412,338.2645743)
\lineto(486.3886412,338.2645743)
\curveto(486.33863668,338.24457448)(486.27863674,338.23957449)(486.2086412,338.2495743)
\curveto(486.14863687,338.24957448)(486.09363692,338.25457447)(486.0436412,338.2645743)
\curveto(485.96363705,338.27457445)(485.89363712,338.27457445)(485.8336412,338.2645743)
\curveto(485.77363724,338.25457447)(485.70863731,338.25957447)(485.6386412,338.2795743)
\curveto(485.58863743,338.29957443)(485.53363748,338.30957442)(485.4736412,338.3095743)
\curveto(485.4136376,338.30957442)(485.35863766,338.31957441)(485.3086412,338.3395743)
\curveto(485.19863782,338.35957437)(485.08863793,338.38457434)(484.9786412,338.4145743)
\curveto(484.86863815,338.43457429)(484.76863825,338.46957426)(484.6786412,338.5195743)
\curveto(484.56863845,338.55957417)(484.46363855,338.59457413)(484.3636412,338.6245743)
\curveto(484.27363874,338.66457406)(484.18863883,338.70957402)(484.1086412,338.7595743)
\curveto(483.78863923,338.95957377)(483.50363951,339.18957354)(483.2536412,339.4495743)
\curveto(483.00364001,339.71957301)(482.79864022,340.0295727)(482.6386412,340.3795743)
\curveto(482.58864043,340.48957224)(482.54864047,340.59957213)(482.5186412,340.7095743)
\curveto(482.48864053,340.8295719)(482.44864057,340.94957178)(482.3986412,341.0695743)
\curveto(482.38864063,341.10957162)(482.38364063,341.14457158)(482.3836412,341.1745743)
\curveto(482.38364063,341.21457151)(482.37864064,341.25457147)(482.3686412,341.2945743)
\curveto(482.32864069,341.41457131)(482.30364071,341.54457118)(482.2936412,341.6845743)
\lineto(482.2636412,342.1045743)
\curveto(482.26364075,342.15457057)(482.25864076,342.20957052)(482.2486412,342.2695743)
\curveto(482.24864077,342.3295704)(482.25364076,342.38457034)(482.2636412,342.4345743)
\lineto(482.2636412,342.6145743)
\lineto(482.3086412,342.9745743)
\curveto(482.34864067,343.14456958)(482.38364063,343.30956942)(482.4136412,343.4695743)
\curveto(482.44364057,343.6295691)(482.48864053,343.77956895)(482.5486412,343.9195743)
\curveto(482.97864004,344.95956777)(483.70863931,345.69456703)(484.7386412,346.1245743)
\curveto(484.87863814,346.18456654)(485.018638,346.2245665)(485.1586412,346.2445743)
\curveto(485.30863771,346.27456645)(485.46363755,346.30956642)(485.6236412,346.3495743)
\curveto(485.70363731,346.35956637)(485.77863724,346.36456636)(485.8486412,346.3645743)
\curveto(485.9186371,346.36456636)(485.99363702,346.36956636)(486.0736412,346.3795743)
\curveto(486.58363643,346.38956634)(487.018636,346.3295664)(487.3786412,346.1995743)
\curveto(487.74863527,346.07956665)(488.07863494,345.91956681)(488.3686412,345.7195743)
\curveto(488.45863456,345.65956707)(488.54863447,345.58956714)(488.6386412,345.5095743)
\curveto(488.72863429,345.43956729)(488.80863421,345.36456736)(488.8786412,345.2845743)
\curveto(488.90863411,345.23456749)(488.94863407,345.19456753)(488.9986412,345.1645743)
\curveto(489.07863394,345.05456767)(489.15363386,344.93956779)(489.2236412,344.8195743)
\curveto(489.29363372,344.70956802)(489.36863365,344.59456813)(489.4486412,344.4745743)
\curveto(489.49863352,344.38456834)(489.53863348,344.28956844)(489.5686412,344.1895743)
\curveto(489.60863341,344.09956863)(489.64863337,343.99956873)(489.6886412,343.8895743)
\curveto(489.73863328,343.75956897)(489.77863324,343.6245691)(489.8086412,343.4845743)
\curveto(489.83863318,343.34456938)(489.87363314,343.20456952)(489.9136412,343.0645743)
\curveto(489.93363308,342.98456974)(489.93863308,342.89456983)(489.9286412,342.7945743)
\curveto(489.92863309,342.70457002)(489.93863308,342.61957011)(489.9586412,342.5395743)
\lineto(489.9586412,342.3745743)
\moveto(487.7086412,343.2595743)
\curveto(487.77863524,343.35956937)(487.78363523,343.47956925)(487.7236412,343.6195743)
\curveto(487.67363534,343.76956896)(487.63363538,343.87956885)(487.6036412,343.9495743)
\curveto(487.46363555,344.21956851)(487.27863574,344.4245683)(487.0486412,344.5645743)
\curveto(486.8186362,344.71456801)(486.49863652,344.79456793)(486.0886412,344.8045743)
\curveto(486.05863696,344.78456794)(486.02363699,344.77956795)(485.9836412,344.7895743)
\curveto(485.94363707,344.79956793)(485.90863711,344.79956793)(485.8786412,344.7895743)
\curveto(485.82863719,344.76956796)(485.77363724,344.75456797)(485.7136412,344.7445743)
\curveto(485.65363736,344.74456798)(485.59863742,344.73456799)(485.5486412,344.7145743)
\curveto(485.10863791,344.57456815)(484.78363823,344.29956843)(484.5736412,343.8895743)
\curveto(484.55363846,343.84956888)(484.52863849,343.79456893)(484.4986412,343.7245743)
\curveto(484.47863854,343.66456906)(484.46363855,343.59956913)(484.4536412,343.5295743)
\curveto(484.44363857,343.46956926)(484.44363857,343.40956932)(484.4536412,343.3495743)
\curveto(484.47363854,343.28956944)(484.50863851,343.23956949)(484.5586412,343.1995743)
\curveto(484.63863838,343.14956958)(484.74863827,343.1245696)(484.8886412,343.1245743)
\lineto(485.2936412,343.1245743)
\lineto(486.9586412,343.1245743)
\lineto(487.3936412,343.1245743)
\curveto(487.55363546,343.13456959)(487.65863536,343.17956955)(487.7086412,343.2595743)
}
}
{
\newrgbcolor{curcolor}{0 0 0}
\pscustom[linestyle=none,fillstyle=solid,fillcolor=curcolor]
{
}
}
{
\newrgbcolor{curcolor}{0 0 0}
\pscustom[linestyle=none,fillstyle=solid,fillcolor=curcolor]
{
\newpath
\moveto(495.8770787,349.1245743)
\lineto(496.9720787,349.1245743)
\curveto(497.07207622,349.1245636)(497.16707612,349.11956361)(497.2570787,349.1095743)
\curveto(497.34707594,349.09956363)(497.41707587,349.06956366)(497.4670787,349.0195743)
\curveto(497.52707576,348.94956378)(497.55707573,348.85456387)(497.5570787,348.7345743)
\curveto(497.56707572,348.6245641)(497.57207572,348.50956422)(497.5720787,348.3895743)
\lineto(497.5720787,347.0545743)
\lineto(497.5720787,341.6695743)
\lineto(497.5720787,339.3745743)
\lineto(497.5720787,338.9545743)
\curveto(497.58207571,338.80457392)(497.56207573,338.68957404)(497.5120787,338.6095743)
\curveto(497.46207583,338.5295742)(497.37207592,338.47457425)(497.2420787,338.4445743)
\curveto(497.18207611,338.4245743)(497.11207618,338.41957431)(497.0320787,338.4295743)
\curveto(496.96207633,338.43957429)(496.8920764,338.44457428)(496.8220787,338.4445743)
\lineto(496.1020787,338.4445743)
\curveto(495.9920773,338.44457428)(495.8920774,338.44957428)(495.8020787,338.4595743)
\curveto(495.71207758,338.46957426)(495.63707765,338.49957423)(495.5770787,338.5495743)
\curveto(495.51707777,338.59957413)(495.48207781,338.67457405)(495.4720787,338.7745743)
\lineto(495.4720787,339.1045743)
\lineto(495.4720787,340.4395743)
\lineto(495.4720787,346.0645743)
\lineto(495.4720787,348.1045743)
\curveto(495.47207782,348.23456449)(495.46707782,348.38956434)(495.4570787,348.5695743)
\curveto(495.45707783,348.74956398)(495.48207781,348.87956385)(495.5320787,348.9595743)
\curveto(495.55207774,348.99956373)(495.57707771,349.0295637)(495.6070787,349.0495743)
\lineto(495.7270787,349.1095743)
\curveto(495.74707754,349.10956362)(495.77207752,349.10956362)(495.8020787,349.1095743)
\curveto(495.83207746,349.11956361)(495.85707743,349.1245636)(495.8770787,349.1245743)
}
}
{
\newrgbcolor{curcolor}{0 0 0}
\pscustom[linestyle=none,fillstyle=solid,fillcolor=curcolor]
{
\newpath
\moveto(507.0042662,342.6145743)
\curveto(507.02425763,342.55457017)(507.03425762,342.46957026)(507.0342662,342.3595743)
\curveto(507.03425762,342.24957048)(507.02425763,342.16457056)(507.0042662,342.1045743)
\lineto(507.0042662,341.9545743)
\curveto(506.98425767,341.87457085)(506.97425768,341.79457093)(506.9742662,341.7145743)
\curveto(506.98425767,341.63457109)(506.97925768,341.55457117)(506.9592662,341.4745743)
\curveto(506.93925772,341.40457132)(506.92425773,341.33957139)(506.9142662,341.2795743)
\curveto(506.90425775,341.21957151)(506.89425776,341.15457157)(506.8842662,341.0845743)
\curveto(506.84425781,340.97457175)(506.80925785,340.85957187)(506.7792662,340.7395743)
\curveto(506.74925791,340.6295721)(506.70925795,340.5245722)(506.6592662,340.4245743)
\curveto(506.44925821,339.94457278)(506.17425848,339.55457317)(505.8342662,339.2545743)
\curveto(505.49425916,338.95457377)(505.08425957,338.70457402)(504.6042662,338.5045743)
\curveto(504.48426017,338.45457427)(504.3592603,338.41957431)(504.2292662,338.3995743)
\curveto(504.10926055,338.36957436)(503.98426067,338.33957439)(503.8542662,338.3095743)
\curveto(503.80426085,338.28957444)(503.74926091,338.27957445)(503.6892662,338.2795743)
\curveto(503.62926103,338.27957445)(503.57426108,338.27457445)(503.5242662,338.2645743)
\lineto(503.4192662,338.2645743)
\curveto(503.38926127,338.25457447)(503.3592613,338.24957448)(503.3292662,338.2495743)
\curveto(503.27926138,338.23957449)(503.19926146,338.23457449)(503.0892662,338.2345743)
\curveto(502.97926168,338.2245745)(502.89426176,338.2295745)(502.8342662,338.2495743)
\lineto(502.6842662,338.2495743)
\curveto(502.63426202,338.25957447)(502.57926208,338.26457446)(502.5192662,338.2645743)
\curveto(502.46926219,338.25457447)(502.41926224,338.25957447)(502.3692662,338.2795743)
\curveto(502.32926233,338.28957444)(502.28926237,338.29457443)(502.2492662,338.2945743)
\curveto(502.21926244,338.29457443)(502.17926248,338.29957443)(502.1292662,338.3095743)
\curveto(502.02926263,338.33957439)(501.92926273,338.36457436)(501.8292662,338.3845743)
\curveto(501.72926293,338.40457432)(501.63426302,338.43457429)(501.5442662,338.4745743)
\curveto(501.42426323,338.51457421)(501.30926335,338.55457417)(501.1992662,338.5945743)
\curveto(501.09926356,338.63457409)(500.99426366,338.68457404)(500.8842662,338.7445743)
\curveto(500.53426412,338.95457377)(500.23426442,339.19957353)(499.9842662,339.4795743)
\curveto(499.73426492,339.75957297)(499.52426513,340.09457263)(499.3542662,340.4845743)
\curveto(499.30426535,340.57457215)(499.26426539,340.66957206)(499.2342662,340.7695743)
\curveto(499.21426544,340.86957186)(499.18926547,340.97457175)(499.1592662,341.0845743)
\curveto(499.13926552,341.13457159)(499.12926553,341.17957155)(499.1292662,341.2195743)
\curveto(499.12926553,341.25957147)(499.11926554,341.30457142)(499.0992662,341.3545743)
\curveto(499.07926558,341.43457129)(499.06926559,341.51457121)(499.0692662,341.5945743)
\curveto(499.06926559,341.68457104)(499.0592656,341.76957096)(499.0392662,341.8495743)
\curveto(499.02926563,341.89957083)(499.02426563,341.94457078)(499.0242662,341.9845743)
\lineto(499.0242662,342.1195743)
\curveto(499.00426565,342.17957055)(498.99426566,342.26457046)(498.9942662,342.3745743)
\curveto(499.00426565,342.48457024)(499.01926564,342.56957016)(499.0392662,342.6295743)
\lineto(499.0392662,342.7345743)
\curveto(499.04926561,342.78456994)(499.04926561,342.83456989)(499.0392662,342.8845743)
\curveto(499.03926562,342.94456978)(499.04926561,342.99956973)(499.0692662,343.0495743)
\curveto(499.07926558,343.09956963)(499.08426557,343.14456958)(499.0842662,343.1845743)
\curveto(499.08426557,343.23456949)(499.09426556,343.28456944)(499.1142662,343.3345743)
\curveto(499.1542655,343.46456926)(499.18926547,343.58956914)(499.2192662,343.7095743)
\curveto(499.24926541,343.83956889)(499.28926537,343.96456876)(499.3392662,344.0845743)
\curveto(499.51926514,344.49456823)(499.73426492,344.83456789)(499.9842662,345.1045743)
\curveto(500.23426442,345.38456734)(500.53926412,345.63956709)(500.8992662,345.8695743)
\curveto(500.99926366,345.91956681)(501.10426355,345.96456676)(501.2142662,346.0045743)
\curveto(501.32426333,346.04456668)(501.43426322,346.08956664)(501.5442662,346.1395743)
\curveto(501.67426298,346.18956654)(501.80926285,346.2245665)(501.9492662,346.2445743)
\curveto(502.08926257,346.26456646)(502.23426242,346.29456643)(502.3842662,346.3345743)
\curveto(502.46426219,346.34456638)(502.53926212,346.34956638)(502.6092662,346.3495743)
\curveto(502.67926198,346.34956638)(502.74926191,346.35456637)(502.8192662,346.3645743)
\curveto(503.39926126,346.37456635)(503.89926076,346.31456641)(504.3192662,346.1845743)
\curveto(504.74925991,346.05456667)(505.12925953,345.87456685)(505.4592662,345.6445743)
\curveto(505.56925909,345.56456716)(505.67925898,345.47456725)(505.7892662,345.3745743)
\curveto(505.90925875,345.28456744)(506.00925865,345.18456754)(506.0892662,345.0745743)
\curveto(506.16925849,344.97456775)(506.23925842,344.87456785)(506.2992662,344.7745743)
\curveto(506.36925829,344.67456805)(506.43925822,344.56956816)(506.5092662,344.4595743)
\curveto(506.57925808,344.34956838)(506.63425802,344.2295685)(506.6742662,344.0995743)
\curveto(506.71425794,343.97956875)(506.7592579,343.84956888)(506.8092662,343.7095743)
\curveto(506.83925782,343.6295691)(506.86425779,343.54456918)(506.8842662,343.4545743)
\lineto(506.9442662,343.1845743)
\curveto(506.9542577,343.14456958)(506.9592577,343.10456962)(506.9592662,343.0645743)
\curveto(506.9592577,343.0245697)(506.96425769,342.98456974)(506.9742662,342.9445743)
\curveto(506.99425766,342.89456983)(506.99925766,342.83956989)(506.9892662,342.7795743)
\curveto(506.97925768,342.71957001)(506.98425767,342.66457006)(507.0042662,342.6145743)
\moveto(504.9042662,342.0745743)
\curveto(504.91425974,342.1245706)(504.91925974,342.19457053)(504.9192662,342.2845743)
\curveto(504.91925974,342.38457034)(504.91425974,342.45957027)(504.9042662,342.5095743)
\lineto(504.9042662,342.6295743)
\curveto(504.88425977,342.67957005)(504.87425978,342.73456999)(504.8742662,342.7945743)
\curveto(504.87425978,342.85456987)(504.86925979,342.90956982)(504.8592662,342.9595743)
\curveto(504.8592598,342.99956973)(504.8542598,343.0295697)(504.8442662,343.0495743)
\lineto(504.7842662,343.2895743)
\curveto(504.77425988,343.37956935)(504.7542599,343.46456926)(504.7242662,343.5445743)
\curveto(504.61426004,343.80456892)(504.48426017,344.0245687)(504.3342662,344.2045743)
\curveto(504.18426047,344.39456833)(503.98426067,344.54456818)(503.7342662,344.6545743)
\curveto(503.67426098,344.67456805)(503.61426104,344.68956804)(503.5542662,344.6995743)
\curveto(503.49426116,344.71956801)(503.42926123,344.73956799)(503.3592662,344.7595743)
\curveto(503.27926138,344.77956795)(503.19426146,344.78456794)(503.1042662,344.7745743)
\lineto(502.8342662,344.7745743)
\curveto(502.80426185,344.75456797)(502.76926189,344.74456798)(502.7292662,344.7445743)
\curveto(502.68926197,344.75456797)(502.654262,344.75456797)(502.6242662,344.7445743)
\lineto(502.4142662,344.6845743)
\curveto(502.3542623,344.67456805)(502.29926236,344.65456807)(502.2492662,344.6245743)
\curveto(501.99926266,344.51456821)(501.79426286,344.35456837)(501.6342662,344.1445743)
\curveto(501.48426317,343.94456878)(501.36426329,343.70956902)(501.2742662,343.4395743)
\curveto(501.24426341,343.33956939)(501.21926344,343.23456949)(501.1992662,343.1245743)
\curveto(501.18926347,343.01456971)(501.17426348,342.90456982)(501.1542662,342.7945743)
\curveto(501.14426351,342.74456998)(501.13926352,342.69457003)(501.1392662,342.6445743)
\lineto(501.1392662,342.4945743)
\curveto(501.11926354,342.4245703)(501.10926355,342.31957041)(501.1092662,342.1795743)
\curveto(501.11926354,342.03957069)(501.13426352,341.93457079)(501.1542662,341.8645743)
\lineto(501.1542662,341.7295743)
\curveto(501.17426348,341.64957108)(501.18926347,341.56957116)(501.1992662,341.4895743)
\curveto(501.20926345,341.41957131)(501.22426343,341.34457138)(501.2442662,341.2645743)
\curveto(501.34426331,340.96457176)(501.44926321,340.71957201)(501.5592662,340.5295743)
\curveto(501.67926298,340.34957238)(501.86426279,340.18457254)(502.1142662,340.0345743)
\curveto(502.18426247,339.98457274)(502.2592624,339.94457278)(502.3392662,339.9145743)
\curveto(502.42926223,339.88457284)(502.51926214,339.85957287)(502.6092662,339.8395743)
\curveto(502.64926201,339.8295729)(502.68426197,339.8245729)(502.7142662,339.8245743)
\curveto(502.74426191,339.83457289)(502.77926188,339.83457289)(502.8192662,339.8245743)
\lineto(502.9392662,339.7945743)
\curveto(502.98926167,339.79457293)(503.03426162,339.79957293)(503.0742662,339.8095743)
\lineto(503.1942662,339.8095743)
\curveto(503.27426138,339.8295729)(503.3542613,339.84457288)(503.4342662,339.8545743)
\curveto(503.51426114,339.86457286)(503.58926107,339.88457284)(503.6592662,339.9145743)
\curveto(503.91926074,340.01457271)(504.12926053,340.14957258)(504.2892662,340.3195743)
\curveto(504.44926021,340.48957224)(504.58426007,340.69957203)(504.6942662,340.9495743)
\curveto(504.73425992,341.04957168)(504.76425989,341.14957158)(504.7842662,341.2495743)
\curveto(504.80425985,341.34957138)(504.82925983,341.45457127)(504.8592662,341.5645743)
\curveto(504.86925979,341.60457112)(504.87425978,341.63957109)(504.8742662,341.6695743)
\curveto(504.87425978,341.70957102)(504.87925978,341.74957098)(504.8892662,341.7895743)
\lineto(504.8892662,341.9245743)
\curveto(504.88925977,341.97457075)(504.89425976,342.0245707)(504.9042662,342.0745743)
}
}
{
\newrgbcolor{curcolor}{0 0 0}
\pscustom[linestyle=none,fillstyle=solid,fillcolor=curcolor]
{
\newpath
\moveto(511.37418808,346.3795743)
\curveto(512.12418358,346.39956633)(512.77418293,346.31456641)(513.32418808,346.1245743)
\curveto(513.88418182,345.94456678)(514.30918139,345.6295671)(514.59918808,345.1795743)
\curveto(514.66918103,345.06956766)(514.72918097,344.95456777)(514.77918808,344.8345743)
\curveto(514.83918086,344.724568)(514.88918081,344.59956813)(514.92918808,344.4595743)
\curveto(514.94918075,344.39956833)(514.95918074,344.33456839)(514.95918808,344.2645743)
\curveto(514.95918074,344.19456853)(514.94918075,344.13456859)(514.92918808,344.0845743)
\curveto(514.88918081,344.0245687)(514.83418087,343.98456874)(514.76418808,343.9645743)
\curveto(514.71418099,343.94456878)(514.65418105,343.93456879)(514.58418808,343.9345743)
\lineto(514.37418808,343.9345743)
\lineto(513.71418808,343.9345743)
\curveto(513.64418206,343.93456879)(513.57418213,343.9295688)(513.50418808,343.9195743)
\curveto(513.43418227,343.91956881)(513.36918233,343.9295688)(513.30918808,343.9495743)
\curveto(513.20918249,343.96956876)(513.13418257,344.00956872)(513.08418808,344.0695743)
\curveto(513.03418267,344.1295686)(512.98918271,344.18956854)(512.94918808,344.2495743)
\lineto(512.82918808,344.4595743)
\curveto(512.7991829,344.53956819)(512.74918295,344.60456812)(512.67918808,344.6545743)
\curveto(512.57918312,344.73456799)(512.47918322,344.79456793)(512.37918808,344.8345743)
\curveto(512.28918341,344.87456785)(512.17418353,344.90956782)(512.03418808,344.9395743)
\curveto(511.96418374,344.95956777)(511.85918384,344.97456775)(511.71918808,344.9845743)
\curveto(511.58918411,344.99456773)(511.48918421,344.98956774)(511.41918808,344.9695743)
\lineto(511.31418808,344.9695743)
\lineto(511.16418808,344.9395743)
\curveto(511.12418458,344.93956779)(511.07918462,344.93456779)(511.02918808,344.9245743)
\curveto(510.85918484,344.87456785)(510.71918498,344.80456792)(510.60918808,344.7145743)
\curveto(510.50918519,344.63456809)(510.43918526,344.50956822)(510.39918808,344.3395743)
\curveto(510.37918532,344.26956846)(510.37918532,344.20456852)(510.39918808,344.1445743)
\curveto(510.41918528,344.08456864)(510.43918526,344.03456869)(510.45918808,343.9945743)
\curveto(510.52918517,343.87456885)(510.60918509,343.77956895)(510.69918808,343.7095743)
\curveto(510.7991849,343.63956909)(510.91418479,343.57956915)(511.04418808,343.5295743)
\curveto(511.23418447,343.44956928)(511.43918426,343.37956935)(511.65918808,343.3195743)
\lineto(512.34918808,343.1695743)
\curveto(512.58918311,343.1295696)(512.81918288,343.07956965)(513.03918808,343.0195743)
\curveto(513.26918243,342.96956976)(513.48418222,342.90456982)(513.68418808,342.8245743)
\curveto(513.77418193,342.78456994)(513.85918184,342.74956998)(513.93918808,342.7195743)
\curveto(514.02918167,342.69957003)(514.11418159,342.66457006)(514.19418808,342.6145743)
\curveto(514.38418132,342.49457023)(514.55418115,342.36457036)(514.70418808,342.2245743)
\curveto(514.86418084,342.08457064)(514.98918071,341.90957082)(515.07918808,341.6995743)
\curveto(515.10918059,341.6295711)(515.13418057,341.55957117)(515.15418808,341.4895743)
\curveto(515.17418053,341.41957131)(515.19418051,341.34457138)(515.21418808,341.2645743)
\curveto(515.22418048,341.20457152)(515.22918047,341.10957162)(515.22918808,340.9795743)
\curveto(515.23918046,340.85957187)(515.23918046,340.76457196)(515.22918808,340.6945743)
\lineto(515.22918808,340.6195743)
\curveto(515.20918049,340.55957217)(515.19418051,340.49957223)(515.18418808,340.4395743)
\curveto(515.18418052,340.38957234)(515.17918052,340.33957239)(515.16918808,340.2895743)
\curveto(515.0991806,339.98957274)(514.98918071,339.724573)(514.83918808,339.4945743)
\curveto(514.67918102,339.25457347)(514.48418122,339.05957367)(514.25418808,338.9095743)
\curveto(514.02418168,338.75957397)(513.76418194,338.6295741)(513.47418808,338.5195743)
\curveto(513.36418234,338.46957426)(513.24418246,338.43457429)(513.11418808,338.4145743)
\curveto(512.99418271,338.39457433)(512.87418283,338.36957436)(512.75418808,338.3395743)
\curveto(512.66418304,338.31957441)(512.56918313,338.30957442)(512.46918808,338.3095743)
\curveto(512.37918332,338.29957443)(512.28918341,338.28457444)(512.19918808,338.2645743)
\lineto(511.92918808,338.2645743)
\curveto(511.86918383,338.24457448)(511.76418394,338.23457449)(511.61418808,338.2345743)
\curveto(511.47418423,338.23457449)(511.37418433,338.24457448)(511.31418808,338.2645743)
\curveto(511.28418442,338.26457446)(511.24918445,338.26957446)(511.20918808,338.2795743)
\lineto(511.10418808,338.2795743)
\curveto(510.98418472,338.29957443)(510.86418484,338.31457441)(510.74418808,338.3245743)
\curveto(510.62418508,338.33457439)(510.50918519,338.35457437)(510.39918808,338.3845743)
\curveto(510.00918569,338.49457423)(509.66418604,338.61957411)(509.36418808,338.7595743)
\curveto(509.06418664,338.90957382)(508.80918689,339.1295736)(508.59918808,339.4195743)
\curveto(508.45918724,339.60957312)(508.33918736,339.8295729)(508.23918808,340.0795743)
\curveto(508.21918748,340.13957259)(508.1991875,340.21957251)(508.17918808,340.3195743)
\curveto(508.15918754,340.36957236)(508.14418756,340.43957229)(508.13418808,340.5295743)
\curveto(508.12418758,340.61957211)(508.12918757,340.69457203)(508.14918808,340.7545743)
\curveto(508.17918752,340.8245719)(508.22918747,340.87457185)(508.29918808,340.9045743)
\curveto(508.34918735,340.9245718)(508.40918729,340.93457179)(508.47918808,340.9345743)
\lineto(508.70418808,340.9345743)
\lineto(509.40918808,340.9345743)
\lineto(509.64918808,340.9345743)
\curveto(509.72918597,340.93457179)(509.7991859,340.9245718)(509.85918808,340.9045743)
\curveto(509.96918573,340.86457186)(510.03918566,340.79957193)(510.06918808,340.7095743)
\curveto(510.10918559,340.61957211)(510.15418555,340.5245722)(510.20418808,340.4245743)
\curveto(510.22418548,340.37457235)(510.25918544,340.30957242)(510.30918808,340.2295743)
\curveto(510.36918533,340.14957258)(510.41918528,340.09957263)(510.45918808,340.0795743)
\curveto(510.57918512,339.97957275)(510.69418501,339.89957283)(510.80418808,339.8395743)
\curveto(510.91418479,339.78957294)(511.05418465,339.73957299)(511.22418808,339.6895743)
\curveto(511.27418443,339.66957306)(511.32418438,339.65957307)(511.37418808,339.6595743)
\curveto(511.42418428,339.66957306)(511.47418423,339.66957306)(511.52418808,339.6595743)
\curveto(511.6041841,339.63957309)(511.68918401,339.6295731)(511.77918808,339.6295743)
\curveto(511.87918382,339.63957309)(511.96418374,339.65457307)(512.03418808,339.6745743)
\curveto(512.08418362,339.68457304)(512.12918357,339.68957304)(512.16918808,339.6895743)
\curveto(512.21918348,339.68957304)(512.26918343,339.69957303)(512.31918808,339.7195743)
\curveto(512.45918324,339.76957296)(512.58418312,339.8295729)(512.69418808,339.8995743)
\curveto(512.81418289,339.96957276)(512.90918279,340.05957267)(512.97918808,340.1695743)
\curveto(513.02918267,340.24957248)(513.06918263,340.37457235)(513.09918808,340.5445743)
\curveto(513.11918258,340.61457211)(513.11918258,340.67957205)(513.09918808,340.7395743)
\curveto(513.07918262,340.79957193)(513.05918264,340.84957188)(513.03918808,340.8895743)
\curveto(512.96918273,341.0295717)(512.87918282,341.13457159)(512.76918808,341.2045743)
\curveto(512.66918303,341.27457145)(512.54918315,341.33957139)(512.40918808,341.3995743)
\curveto(512.21918348,341.47957125)(512.01918368,341.54457118)(511.80918808,341.5945743)
\curveto(511.5991841,341.64457108)(511.38918431,341.69957103)(511.17918808,341.7595743)
\curveto(511.0991846,341.77957095)(511.01418469,341.79457093)(510.92418808,341.8045743)
\curveto(510.84418486,341.81457091)(510.76418494,341.8295709)(510.68418808,341.8495743)
\curveto(510.36418534,341.93957079)(510.05918564,342.0245707)(509.76918808,342.1045743)
\curveto(509.47918622,342.19457053)(509.21418649,342.3245704)(508.97418808,342.4945743)
\curveto(508.69418701,342.69457003)(508.48918721,342.96456976)(508.35918808,343.3045743)
\curveto(508.33918736,343.37456935)(508.31918738,343.46956926)(508.29918808,343.5895743)
\curveto(508.27918742,343.65956907)(508.26418744,343.74456898)(508.25418808,343.8445743)
\curveto(508.24418746,343.94456878)(508.24918745,344.03456869)(508.26918808,344.1145743)
\curveto(508.28918741,344.16456856)(508.29418741,344.20456852)(508.28418808,344.2345743)
\curveto(508.27418743,344.27456845)(508.27918742,344.31956841)(508.29918808,344.3695743)
\curveto(508.31918738,344.47956825)(508.33918736,344.57956815)(508.35918808,344.6695743)
\curveto(508.38918731,344.76956796)(508.42418728,344.86456786)(508.46418808,344.9545743)
\curveto(508.59418711,345.24456748)(508.77418693,345.47956725)(509.00418808,345.6595743)
\curveto(509.23418647,345.83956689)(509.49418621,345.98456674)(509.78418808,346.0945743)
\curveto(509.89418581,346.14456658)(510.00918569,346.17956655)(510.12918808,346.1995743)
\curveto(510.24918545,346.2295665)(510.37418533,346.25956647)(510.50418808,346.2895743)
\curveto(510.56418514,346.30956642)(510.62418508,346.31956641)(510.68418808,346.3195743)
\lineto(510.86418808,346.3495743)
\curveto(510.94418476,346.35956637)(511.02918467,346.36456636)(511.11918808,346.3645743)
\curveto(511.20918449,346.36456636)(511.29418441,346.36956636)(511.37418808,346.3795743)
}
}
{
\newrgbcolor{curcolor}{0 0 0}
\pscustom[linestyle=none,fillstyle=solid,fillcolor=curcolor]
{
}
}
{
\newrgbcolor{curcolor}{0 0 0}
\pscustom[linestyle=none,fillstyle=solid,fillcolor=curcolor]
{
\newpath
\moveto(525.04598495,346.3645743)
\curveto(525.15597964,346.36456636)(525.25097954,346.35456637)(525.33098495,346.3345743)
\curveto(525.42097937,346.31456641)(525.4909793,346.26956646)(525.54098495,346.1995743)
\curveto(525.60097919,346.11956661)(525.63097916,345.97956675)(525.63098495,345.7795743)
\lineto(525.63098495,345.2695743)
\lineto(525.63098495,344.8945743)
\curveto(525.64097915,344.75456797)(525.62597917,344.64456808)(525.58598495,344.5645743)
\curveto(525.54597925,344.49456823)(525.48597931,344.44956828)(525.40598495,344.4295743)
\curveto(525.33597946,344.40956832)(525.25097954,344.39956833)(525.15098495,344.3995743)
\curveto(525.06097973,344.39956833)(524.96097983,344.40456832)(524.85098495,344.4145743)
\curveto(524.75098004,344.4245683)(524.65598014,344.41956831)(524.56598495,344.3995743)
\curveto(524.4959803,344.37956835)(524.42598037,344.36456836)(524.35598495,344.3545743)
\curveto(524.28598051,344.35456837)(524.22098057,344.34456838)(524.16098495,344.3245743)
\curveto(524.00098079,344.27456845)(523.84098095,344.19956853)(523.68098495,344.0995743)
\curveto(523.52098127,344.00956872)(523.3959814,343.90456882)(523.30598495,343.7845743)
\curveto(523.25598154,343.70456902)(523.20098159,343.61956911)(523.14098495,343.5295743)
\curveto(523.0909817,343.44956928)(523.04098175,343.36456936)(522.99098495,343.2745743)
\curveto(522.96098183,343.19456953)(522.93098186,343.10956962)(522.90098495,343.0195743)
\lineto(522.84098495,342.7795743)
\curveto(522.82098197,342.70957002)(522.81098198,342.63457009)(522.81098495,342.5545743)
\curveto(522.81098198,342.48457024)(522.80098199,342.41457031)(522.78098495,342.3445743)
\curveto(522.77098202,342.30457042)(522.76598203,342.26457046)(522.76598495,342.2245743)
\curveto(522.77598202,342.19457053)(522.77598202,342.16457056)(522.76598495,342.1345743)
\lineto(522.76598495,341.8945743)
\curveto(522.74598205,341.8245709)(522.74098205,341.74457098)(522.75098495,341.6545743)
\curveto(522.76098203,341.57457115)(522.76598203,341.49457123)(522.76598495,341.4145743)
\lineto(522.76598495,340.4545743)
\lineto(522.76598495,339.1795743)
\curveto(522.76598203,339.04957368)(522.76098203,338.9295738)(522.75098495,338.8195743)
\curveto(522.74098205,338.70957402)(522.71098208,338.61957411)(522.66098495,338.5495743)
\curveto(522.64098215,338.51957421)(522.60598219,338.49457423)(522.55598495,338.4745743)
\curveto(522.51598228,338.46457426)(522.47098232,338.45457427)(522.42098495,338.4445743)
\lineto(522.34598495,338.4445743)
\curveto(522.2959825,338.43457429)(522.24098255,338.4295743)(522.18098495,338.4295743)
\lineto(522.01598495,338.4295743)
\lineto(521.37098495,338.4295743)
\curveto(521.31098348,338.43957429)(521.24598355,338.44457428)(521.17598495,338.4445743)
\lineto(520.98098495,338.4445743)
\curveto(520.93098386,338.46457426)(520.88098391,338.47957425)(520.83098495,338.4895743)
\curveto(520.78098401,338.50957422)(520.74598405,338.54457418)(520.72598495,338.5945743)
\curveto(520.68598411,338.64457408)(520.66098413,338.71457401)(520.65098495,338.8045743)
\lineto(520.65098495,339.1045743)
\lineto(520.65098495,340.1245743)
\lineto(520.65098495,344.3545743)
\lineto(520.65098495,345.4645743)
\lineto(520.65098495,345.7495743)
\curveto(520.65098414,345.84956688)(520.67098412,345.9295668)(520.71098495,345.9895743)
\curveto(520.76098403,346.06956666)(520.83598396,346.11956661)(520.93598495,346.1395743)
\curveto(521.03598376,346.15956657)(521.15598364,346.16956656)(521.29598495,346.1695743)
\lineto(522.06098495,346.1695743)
\curveto(522.18098261,346.16956656)(522.28598251,346.15956657)(522.37598495,346.1395743)
\curveto(522.46598233,346.1295666)(522.53598226,346.08456664)(522.58598495,346.0045743)
\curveto(522.61598218,345.95456677)(522.63098216,345.88456684)(522.63098495,345.7945743)
\lineto(522.66098495,345.5245743)
\curveto(522.67098212,345.44456728)(522.68598211,345.36956736)(522.70598495,345.2995743)
\curveto(522.73598206,345.2295675)(522.78598201,345.19456753)(522.85598495,345.1945743)
\curveto(522.87598192,345.21456751)(522.8959819,345.2245675)(522.91598495,345.2245743)
\curveto(522.93598186,345.2245675)(522.95598184,345.23456749)(522.97598495,345.2545743)
\curveto(523.03598176,345.30456742)(523.08598171,345.35956737)(523.12598495,345.4195743)
\curveto(523.17598162,345.48956724)(523.23598156,345.54956718)(523.30598495,345.5995743)
\curveto(523.34598145,345.6295671)(523.38098141,345.65956707)(523.41098495,345.6895743)
\curveto(523.44098135,345.729567)(523.47598132,345.76456696)(523.51598495,345.7945743)
\lineto(523.78598495,345.9745743)
\curveto(523.88598091,346.03456669)(523.98598081,346.08956664)(524.08598495,346.1395743)
\curveto(524.18598061,346.17956655)(524.28598051,346.21456651)(524.38598495,346.2445743)
\lineto(524.71598495,346.3345743)
\curveto(524.74598005,346.34456638)(524.80097999,346.34456638)(524.88098495,346.3345743)
\curveto(524.97097982,346.33456639)(525.02597977,346.34456638)(525.04598495,346.3645743)
}
}
{
\newrgbcolor{curcolor}{0 0 0}
\pscustom[linestyle=none,fillstyle=solid,fillcolor=curcolor]
{
\newpath
\moveto(533.5523912,342.3745743)
\curveto(533.57238304,342.29457043)(533.57238304,342.20457052)(533.5523912,342.1045743)
\curveto(533.53238308,342.00457072)(533.49738311,341.93957079)(533.4473912,341.9095743)
\curveto(533.39738321,341.86957086)(533.32238329,341.83957089)(533.2223912,341.8195743)
\curveto(533.13238348,341.80957092)(533.02738358,341.79957093)(532.9073912,341.7895743)
\lineto(532.5623912,341.7895743)
\curveto(532.45238416,341.79957093)(532.35238426,341.80457092)(532.2623912,341.8045743)
\lineto(528.6023912,341.8045743)
\lineto(528.3923912,341.8045743)
\curveto(528.33238828,341.80457092)(528.27738833,341.79457093)(528.2273912,341.7745743)
\curveto(528.14738846,341.73457099)(528.09738851,341.69457103)(528.0773912,341.6545743)
\curveto(528.05738855,341.63457109)(528.03738857,341.59457113)(528.0173912,341.5345743)
\curveto(527.99738861,341.48457124)(527.99238862,341.43457129)(528.0023912,341.3845743)
\curveto(528.02238859,341.3245714)(528.03238858,341.26457146)(528.0323912,341.2045743)
\curveto(528.04238857,341.15457157)(528.05738855,341.09957163)(528.0773912,341.0395743)
\curveto(528.15738845,340.79957193)(528.25238836,340.59957213)(528.3623912,340.4395743)
\curveto(528.48238813,340.28957244)(528.64238797,340.15457257)(528.8423912,340.0345743)
\curveto(528.92238769,339.98457274)(529.00238761,339.94957278)(529.0823912,339.9295743)
\curveto(529.17238744,339.91957281)(529.26238735,339.89957283)(529.3523912,339.8695743)
\curveto(529.43238718,339.84957288)(529.54238707,339.83457289)(529.6823912,339.8245743)
\curveto(529.82238679,339.81457291)(529.94238667,339.81957291)(530.0423912,339.8395743)
\lineto(530.1773912,339.8395743)
\curveto(530.27738633,339.85957287)(530.36738624,339.87957285)(530.4473912,339.8995743)
\curveto(530.53738607,339.9295728)(530.62238599,339.95957277)(530.7023912,339.9895743)
\curveto(530.80238581,340.03957269)(530.9123857,340.10457262)(531.0323912,340.1845743)
\curveto(531.16238545,340.26457246)(531.25738535,340.34457238)(531.3173912,340.4245743)
\curveto(531.36738524,340.49457223)(531.41738519,340.55957217)(531.4673912,340.6195743)
\curveto(531.52738508,340.68957204)(531.59738501,340.73957199)(531.6773912,340.7695743)
\curveto(531.77738483,340.81957191)(531.90238471,340.83957189)(532.0523912,340.8295743)
\lineto(532.4873912,340.8295743)
\lineto(532.6673912,340.8295743)
\curveto(532.73738387,340.83957189)(532.79738381,340.83457189)(532.8473912,340.8145743)
\lineto(532.9973912,340.8145743)
\curveto(533.09738351,340.79457193)(533.16738344,340.76957196)(533.2073912,340.7395743)
\curveto(533.24738336,340.71957201)(533.26738334,340.67457205)(533.2673912,340.6045743)
\curveto(533.27738333,340.53457219)(533.27238334,340.47457225)(533.2523912,340.4245743)
\curveto(533.20238341,340.28457244)(533.14738346,340.15957257)(533.0873912,340.0495743)
\curveto(533.02738358,339.93957279)(532.95738365,339.8295729)(532.8773912,339.7195743)
\curveto(532.65738395,339.38957334)(532.4073842,339.1245736)(532.1273912,338.9245743)
\curveto(531.84738476,338.724574)(531.49738511,338.55457417)(531.0773912,338.4145743)
\curveto(530.96738564,338.37457435)(530.85738575,338.34957438)(530.7473912,338.3395743)
\curveto(530.63738597,338.3295744)(530.52238609,338.30957442)(530.4023912,338.2795743)
\curveto(530.36238625,338.26957446)(530.31738629,338.26957446)(530.2673912,338.2795743)
\curveto(530.22738638,338.27957445)(530.18738642,338.27457445)(530.1473912,338.2645743)
\lineto(529.9823912,338.2645743)
\curveto(529.93238668,338.24457448)(529.87238674,338.23957449)(529.8023912,338.2495743)
\curveto(529.74238687,338.24957448)(529.68738692,338.25457447)(529.6373912,338.2645743)
\curveto(529.55738705,338.27457445)(529.48738712,338.27457445)(529.4273912,338.2645743)
\curveto(529.36738724,338.25457447)(529.30238731,338.25957447)(529.2323912,338.2795743)
\curveto(529.18238743,338.29957443)(529.12738748,338.30957442)(529.0673912,338.3095743)
\curveto(529.0073876,338.30957442)(528.95238766,338.31957441)(528.9023912,338.3395743)
\curveto(528.79238782,338.35957437)(528.68238793,338.38457434)(528.5723912,338.4145743)
\curveto(528.46238815,338.43457429)(528.36238825,338.46957426)(528.2723912,338.5195743)
\curveto(528.16238845,338.55957417)(528.05738855,338.59457413)(527.9573912,338.6245743)
\curveto(527.86738874,338.66457406)(527.78238883,338.70957402)(527.7023912,338.7595743)
\curveto(527.38238923,338.95957377)(527.09738951,339.18957354)(526.8473912,339.4495743)
\curveto(526.59739001,339.71957301)(526.39239022,340.0295727)(526.2323912,340.3795743)
\curveto(526.18239043,340.48957224)(526.14239047,340.59957213)(526.1123912,340.7095743)
\curveto(526.08239053,340.8295719)(526.04239057,340.94957178)(525.9923912,341.0695743)
\curveto(525.98239063,341.10957162)(525.97739063,341.14457158)(525.9773912,341.1745743)
\curveto(525.97739063,341.21457151)(525.97239064,341.25457147)(525.9623912,341.2945743)
\curveto(525.92239069,341.41457131)(525.89739071,341.54457118)(525.8873912,341.6845743)
\lineto(525.8573912,342.1045743)
\curveto(525.85739075,342.15457057)(525.85239076,342.20957052)(525.8423912,342.2695743)
\curveto(525.84239077,342.3295704)(525.84739076,342.38457034)(525.8573912,342.4345743)
\lineto(525.8573912,342.6145743)
\lineto(525.9023912,342.9745743)
\curveto(525.94239067,343.14456958)(525.97739063,343.30956942)(526.0073912,343.4695743)
\curveto(526.03739057,343.6295691)(526.08239053,343.77956895)(526.1423912,343.9195743)
\curveto(526.57239004,344.95956777)(527.30238931,345.69456703)(528.3323912,346.1245743)
\curveto(528.47238814,346.18456654)(528.612388,346.2245665)(528.7523912,346.2445743)
\curveto(528.90238771,346.27456645)(529.05738755,346.30956642)(529.2173912,346.3495743)
\curveto(529.29738731,346.35956637)(529.37238724,346.36456636)(529.4423912,346.3645743)
\curveto(529.5123871,346.36456636)(529.58738702,346.36956636)(529.6673912,346.3795743)
\curveto(530.17738643,346.38956634)(530.612386,346.3295664)(530.9723912,346.1995743)
\curveto(531.34238527,346.07956665)(531.67238494,345.91956681)(531.9623912,345.7195743)
\curveto(532.05238456,345.65956707)(532.14238447,345.58956714)(532.2323912,345.5095743)
\curveto(532.32238429,345.43956729)(532.40238421,345.36456736)(532.4723912,345.2845743)
\curveto(532.50238411,345.23456749)(532.54238407,345.19456753)(532.5923912,345.1645743)
\curveto(532.67238394,345.05456767)(532.74738386,344.93956779)(532.8173912,344.8195743)
\curveto(532.88738372,344.70956802)(532.96238365,344.59456813)(533.0423912,344.4745743)
\curveto(533.09238352,344.38456834)(533.13238348,344.28956844)(533.1623912,344.1895743)
\curveto(533.20238341,344.09956863)(533.24238337,343.99956873)(533.2823912,343.8895743)
\curveto(533.33238328,343.75956897)(533.37238324,343.6245691)(533.4023912,343.4845743)
\curveto(533.43238318,343.34456938)(533.46738314,343.20456952)(533.5073912,343.0645743)
\curveto(533.52738308,342.98456974)(533.53238308,342.89456983)(533.5223912,342.7945743)
\curveto(533.52238309,342.70457002)(533.53238308,342.61957011)(533.5523912,342.5395743)
\lineto(533.5523912,342.3745743)
\moveto(531.3023912,343.2595743)
\curveto(531.37238524,343.35956937)(531.37738523,343.47956925)(531.3173912,343.6195743)
\curveto(531.26738534,343.76956896)(531.22738538,343.87956885)(531.1973912,343.9495743)
\curveto(531.05738555,344.21956851)(530.87238574,344.4245683)(530.6423912,344.5645743)
\curveto(530.4123862,344.71456801)(530.09238652,344.79456793)(529.6823912,344.8045743)
\curveto(529.65238696,344.78456794)(529.61738699,344.77956795)(529.5773912,344.7895743)
\curveto(529.53738707,344.79956793)(529.50238711,344.79956793)(529.4723912,344.7895743)
\curveto(529.42238719,344.76956796)(529.36738724,344.75456797)(529.3073912,344.7445743)
\curveto(529.24738736,344.74456798)(529.19238742,344.73456799)(529.1423912,344.7145743)
\curveto(528.70238791,344.57456815)(528.37738823,344.29956843)(528.1673912,343.8895743)
\curveto(528.14738846,343.84956888)(528.12238849,343.79456893)(528.0923912,343.7245743)
\curveto(528.07238854,343.66456906)(528.05738855,343.59956913)(528.0473912,343.5295743)
\curveto(528.03738857,343.46956926)(528.03738857,343.40956932)(528.0473912,343.3495743)
\curveto(528.06738854,343.28956944)(528.10238851,343.23956949)(528.1523912,343.1995743)
\curveto(528.23238838,343.14956958)(528.34238827,343.1245696)(528.4823912,343.1245743)
\lineto(528.8873912,343.1245743)
\lineto(530.5523912,343.1245743)
\lineto(530.9873912,343.1245743)
\curveto(531.14738546,343.13456959)(531.25238536,343.17956955)(531.3023912,343.2595743)
}
}
{
\newrgbcolor{curcolor}{0 0 0}
\pscustom[linestyle=none,fillstyle=solid,fillcolor=curcolor]
{
\newpath
\moveto(538.37067245,346.3795743)
\curveto(539.18066729,346.39956633)(539.85566662,346.27956645)(540.39567245,346.0195743)
\curveto(540.94566553,345.75956697)(541.38066509,345.38956734)(541.70067245,344.9095743)
\curveto(541.86066461,344.66956806)(541.98066449,344.39456833)(542.06067245,344.0845743)
\curveto(542.08066439,344.03456869)(542.09566438,343.96956876)(542.10567245,343.8895743)
\curveto(542.12566435,343.80956892)(542.12566435,343.73956899)(542.10567245,343.6795743)
\curveto(542.06566441,343.56956916)(541.99566448,343.50456922)(541.89567245,343.4845743)
\curveto(541.79566468,343.47456925)(541.6756648,343.46956926)(541.53567245,343.4695743)
\lineto(540.75567245,343.4695743)
\lineto(540.47067245,343.4695743)
\curveto(540.38066609,343.46956926)(540.30566617,343.48956924)(540.24567245,343.5295743)
\curveto(540.16566631,343.56956916)(540.11066636,343.6295691)(540.08067245,343.7095743)
\curveto(540.05066642,343.79956893)(540.01066646,343.88956884)(539.96067245,343.9795743)
\curveto(539.90066657,344.08956864)(539.83566664,344.18956854)(539.76567245,344.2795743)
\curveto(539.69566678,344.36956836)(539.61566686,344.44956828)(539.52567245,344.5195743)
\curveto(539.38566709,344.60956812)(539.23066724,344.67956805)(539.06067245,344.7295743)
\curveto(539.00066747,344.74956798)(538.94066753,344.75956797)(538.88067245,344.7595743)
\curveto(538.82066765,344.75956797)(538.76566771,344.76956796)(538.71567245,344.7895743)
\lineto(538.56567245,344.7895743)
\curveto(538.36566811,344.78956794)(538.20566827,344.76956796)(538.08567245,344.7295743)
\curveto(537.79566868,344.63956809)(537.56066891,344.49956823)(537.38067245,344.3095743)
\curveto(537.20066927,344.1295686)(537.05566942,343.90956882)(536.94567245,343.6495743)
\curveto(536.89566958,343.53956919)(536.85566962,343.41956931)(536.82567245,343.2895743)
\curveto(536.80566967,343.16956956)(536.78066969,343.03956969)(536.75067245,342.8995743)
\curveto(536.74066973,342.85956987)(536.73566974,342.81956991)(536.73567245,342.7795743)
\curveto(536.73566974,342.73956999)(536.73066974,342.69957003)(536.72067245,342.6595743)
\curveto(536.70066977,342.55957017)(536.69066978,342.41957031)(536.69067245,342.2395743)
\curveto(536.70066977,342.05957067)(536.71566976,341.91957081)(536.73567245,341.8195743)
\curveto(536.73566974,341.73957099)(536.74066973,341.68457104)(536.75067245,341.6545743)
\curveto(536.7706697,341.58457114)(536.78066969,341.51457121)(536.78067245,341.4445743)
\curveto(536.79066968,341.37457135)(536.80566967,341.30457142)(536.82567245,341.2345743)
\curveto(536.90566957,341.00457172)(537.00066947,340.79457193)(537.11067245,340.6045743)
\curveto(537.22066925,340.41457231)(537.36066911,340.25457247)(537.53067245,340.1245743)
\curveto(537.5706689,340.09457263)(537.63066884,340.05957267)(537.71067245,340.0195743)
\curveto(537.82066865,339.94957278)(537.93066854,339.90457282)(538.04067245,339.8845743)
\curveto(538.16066831,339.86457286)(538.30566817,339.84457288)(538.47567245,339.8245743)
\lineto(538.56567245,339.8245743)
\curveto(538.60566787,339.8245729)(538.63566784,339.8295729)(538.65567245,339.8395743)
\lineto(538.79067245,339.8395743)
\curveto(538.86066761,339.85957287)(538.92566755,339.87457285)(538.98567245,339.8845743)
\curveto(539.05566742,339.90457282)(539.12066735,339.9245728)(539.18067245,339.9445743)
\curveto(539.48066699,340.07457265)(539.71066676,340.26457246)(539.87067245,340.5145743)
\curveto(539.91066656,340.56457216)(539.94566653,340.61957211)(539.97567245,340.6795743)
\curveto(540.00566647,340.74957198)(540.03066644,340.80957192)(540.05067245,340.8595743)
\curveto(540.09066638,340.96957176)(540.12566635,341.06457166)(540.15567245,341.1445743)
\curveto(540.18566629,341.23457149)(540.25566622,341.30457142)(540.36567245,341.3545743)
\curveto(540.45566602,341.39457133)(540.60066587,341.40957132)(540.80067245,341.3995743)
\lineto(541.29567245,341.3995743)
\lineto(541.50567245,341.3995743)
\curveto(541.58566489,341.40957132)(541.65066482,341.40457132)(541.70067245,341.3845743)
\lineto(541.82067245,341.3845743)
\lineto(541.94067245,341.3545743)
\curveto(541.98066449,341.35457137)(542.01066446,341.34457138)(542.03067245,341.3245743)
\curveto(542.08066439,341.28457144)(542.11066436,341.2245715)(542.12067245,341.1445743)
\curveto(542.14066433,341.07457165)(542.14066433,340.99957173)(542.12067245,340.9195743)
\curveto(542.03066444,340.58957214)(541.92066455,340.29457243)(541.79067245,340.0345743)
\curveto(541.38066509,339.26457346)(540.72566575,338.729574)(539.82567245,338.4295743)
\curveto(539.72566675,338.39957433)(539.62066685,338.37957435)(539.51067245,338.3695743)
\curveto(539.40066707,338.34957438)(539.29066718,338.3245744)(539.18067245,338.2945743)
\curveto(539.12066735,338.28457444)(539.06066741,338.27957445)(539.00067245,338.2795743)
\curveto(538.94066753,338.27957445)(538.88066759,338.27457445)(538.82067245,338.2645743)
\lineto(538.65567245,338.2645743)
\curveto(538.60566787,338.24457448)(538.53066794,338.23957449)(538.43067245,338.2495743)
\curveto(538.33066814,338.24957448)(538.25566822,338.25457447)(538.20567245,338.2645743)
\curveto(538.12566835,338.28457444)(538.05066842,338.29457443)(537.98067245,338.2945743)
\curveto(537.92066855,338.28457444)(537.85566862,338.28957444)(537.78567245,338.3095743)
\lineto(537.63567245,338.3395743)
\curveto(537.58566889,338.33957439)(537.53566894,338.34457438)(537.48567245,338.3545743)
\curveto(537.3756691,338.38457434)(537.2706692,338.41457431)(537.17067245,338.4445743)
\curveto(537.0706694,338.47457425)(536.9756695,338.50957422)(536.88567245,338.5495743)
\curveto(536.41567006,338.74957398)(536.02067045,339.00457372)(535.70067245,339.3145743)
\curveto(535.38067109,339.63457309)(535.12067135,340.0295727)(534.92067245,340.4995743)
\curveto(534.8706716,340.58957214)(534.83067164,340.68457204)(534.80067245,340.7845743)
\lineto(534.71067245,341.1145743)
\curveto(534.70067177,341.15457157)(534.69567178,341.18957154)(534.69567245,341.2195743)
\curveto(534.69567178,341.25957147)(534.68567179,341.30457142)(534.66567245,341.3545743)
\curveto(534.64567183,341.4245713)(534.63567184,341.49457123)(534.63567245,341.5645743)
\curveto(534.63567184,341.64457108)(534.62567185,341.71957101)(534.60567245,341.7895743)
\lineto(534.60567245,342.0445743)
\curveto(534.58567189,342.09457063)(534.5756719,342.14957058)(534.57567245,342.2095743)
\curveto(534.5756719,342.27957045)(534.58567189,342.33957039)(534.60567245,342.3895743)
\curveto(534.61567186,342.43957029)(534.61567186,342.48457024)(534.60567245,342.5245743)
\curveto(534.59567188,342.56457016)(534.59567188,342.60457012)(534.60567245,342.6445743)
\curveto(534.62567185,342.71457001)(534.63067184,342.77956995)(534.62067245,342.8395743)
\curveto(534.62067185,342.89956983)(534.63067184,342.95956977)(534.65067245,343.0195743)
\curveto(534.70067177,343.19956953)(534.74067173,343.36956936)(534.77067245,343.5295743)
\curveto(534.80067167,343.69956903)(534.84567163,343.86456886)(534.90567245,344.0245743)
\curveto(535.12567135,344.53456819)(535.40067107,344.95956777)(535.73067245,345.2995743)
\curveto(536.0706704,345.63956709)(536.50066997,345.91456681)(537.02067245,346.1245743)
\curveto(537.16066931,346.18456654)(537.30566917,346.2245665)(537.45567245,346.2445743)
\curveto(537.60566887,346.27456645)(537.76066871,346.30956642)(537.92067245,346.3495743)
\curveto(538.00066847,346.35956637)(538.0756684,346.36456636)(538.14567245,346.3645743)
\curveto(538.21566826,346.36456636)(538.29066818,346.36956636)(538.37067245,346.3795743)
}
}
{
\newrgbcolor{curcolor}{0 0 0}
\pscustom[linestyle=none,fillstyle=solid,fillcolor=curcolor]
{
\newpath
\moveto(543.8339537,346.1545743)
\lineto(544.9589537,346.1545743)
\curveto(545.06895127,346.15456657)(545.16895117,346.14956658)(545.2589537,346.1395743)
\curveto(545.34895099,346.1295666)(545.41395092,346.09456663)(545.4539537,346.0345743)
\curveto(545.50395083,345.97456675)(545.5339508,345.88956684)(545.5439537,345.7795743)
\curveto(545.55395078,345.67956705)(545.55895078,345.57456715)(545.5589537,345.4645743)
\lineto(545.5589537,344.4145743)
\lineto(545.5589537,342.1795743)
\curveto(545.55895078,341.81957091)(545.57395076,341.47957125)(545.6039537,341.1595743)
\curveto(545.6339507,340.83957189)(545.72395061,340.57457215)(545.8739537,340.3645743)
\curveto(546.01395032,340.15457257)(546.2389501,340.00457272)(546.5489537,339.9145743)
\curveto(546.59894974,339.90457282)(546.6389497,339.89957283)(546.6689537,339.8995743)
\curveto(546.70894963,339.89957283)(546.75394958,339.89457283)(546.8039537,339.8845743)
\curveto(546.85394948,339.87457285)(546.90894943,339.86957286)(546.9689537,339.8695743)
\curveto(547.02894931,339.86957286)(547.07394926,339.87457285)(547.1039537,339.8845743)
\curveto(547.15394918,339.90457282)(547.19394914,339.90957282)(547.2239537,339.8995743)
\curveto(547.26394907,339.88957284)(547.30394903,339.89457283)(547.3439537,339.9145743)
\curveto(547.55394878,339.96457276)(547.71894862,340.0295727)(547.8389537,340.1095743)
\curveto(548.01894832,340.21957251)(548.15894818,340.35957237)(548.2589537,340.5295743)
\curveto(548.36894797,340.70957202)(548.44394789,340.90457182)(548.4839537,341.1145743)
\curveto(548.5339478,341.33457139)(548.56394777,341.57457115)(548.5739537,341.8345743)
\curveto(548.58394775,342.10457062)(548.58894775,342.38457034)(548.5889537,342.6745743)
\lineto(548.5889537,344.4895743)
\lineto(548.5889537,345.4645743)
\lineto(548.5889537,345.7345743)
\curveto(548.58894775,345.83456689)(548.60894773,345.91456681)(548.6489537,345.9745743)
\curveto(548.69894764,346.06456666)(548.77394756,346.11456661)(548.8739537,346.1245743)
\curveto(548.97394736,346.14456658)(549.09394724,346.15456657)(549.2339537,346.1545743)
\lineto(550.0289537,346.1545743)
\lineto(550.3139537,346.1545743)
\curveto(550.40394593,346.15456657)(550.47894586,346.13456659)(550.5389537,346.0945743)
\curveto(550.61894572,346.04456668)(550.66394567,345.96956676)(550.6739537,345.8695743)
\curveto(550.68394565,345.76956696)(550.68894565,345.65456707)(550.6889537,345.5245743)
\lineto(550.6889537,344.3845743)
\lineto(550.6889537,340.1695743)
\lineto(550.6889537,339.1045743)
\lineto(550.6889537,338.8045743)
\curveto(550.68894565,338.70457402)(550.66894567,338.6295741)(550.6289537,338.5795743)
\curveto(550.57894576,338.49957423)(550.50394583,338.45457427)(550.4039537,338.4445743)
\curveto(550.30394603,338.43457429)(550.19894614,338.4295743)(550.0889537,338.4295743)
\lineto(549.2789537,338.4295743)
\curveto(549.16894717,338.4295743)(549.06894727,338.43457429)(548.9789537,338.4445743)
\curveto(548.89894744,338.45457427)(548.8339475,338.49457423)(548.7839537,338.5645743)
\curveto(548.76394757,338.59457413)(548.74394759,338.63957409)(548.7239537,338.6995743)
\curveto(548.71394762,338.75957397)(548.69894764,338.81957391)(548.6789537,338.8795743)
\curveto(548.66894767,338.93957379)(548.65394768,338.99457373)(548.6339537,339.0445743)
\curveto(548.61394772,339.09457363)(548.58394775,339.1245736)(548.5439537,339.1345743)
\curveto(548.52394781,339.15457357)(548.49894784,339.15957357)(548.4689537,339.1495743)
\curveto(548.4389479,339.13957359)(548.41394792,339.1295736)(548.3939537,339.1195743)
\curveto(548.32394801,339.07957365)(548.26394807,339.03457369)(548.2139537,338.9845743)
\curveto(548.16394817,338.93457379)(548.10894823,338.88957384)(548.0489537,338.8495743)
\curveto(548.00894833,338.81957391)(547.96894837,338.78457394)(547.9289537,338.7445743)
\curveto(547.89894844,338.71457401)(547.85894848,338.68457404)(547.8089537,338.6545743)
\curveto(547.57894876,338.51457421)(547.30894903,338.40457432)(546.9989537,338.3245743)
\curveto(546.92894941,338.30457442)(546.85894948,338.29457443)(546.7889537,338.2945743)
\curveto(546.71894962,338.28457444)(546.64394969,338.26957446)(546.5639537,338.2495743)
\curveto(546.52394981,338.23957449)(546.47894986,338.23957449)(546.4289537,338.2495743)
\curveto(546.38894995,338.24957448)(546.34894999,338.24457448)(546.3089537,338.2345743)
\curveto(546.27895006,338.2245745)(546.21395012,338.2245745)(546.1139537,338.2345743)
\curveto(546.02395031,338.23457449)(545.96395037,338.23957449)(545.9339537,338.2495743)
\curveto(545.88395045,338.24957448)(545.8339505,338.25457447)(545.7839537,338.2645743)
\lineto(545.6339537,338.2645743)
\curveto(545.51395082,338.29457443)(545.39895094,338.31957441)(545.2889537,338.3395743)
\curveto(545.17895116,338.35957437)(545.06895127,338.38957434)(544.9589537,338.4295743)
\curveto(544.90895143,338.44957428)(544.86395147,338.46457426)(544.8239537,338.4745743)
\curveto(544.79395154,338.49457423)(544.75395158,338.51457421)(544.7039537,338.5345743)
\curveto(544.35395198,338.724574)(544.07395226,338.98957374)(543.8639537,339.3295743)
\curveto(543.7339526,339.53957319)(543.6389527,339.78957294)(543.5789537,340.0795743)
\curveto(543.51895282,340.37957235)(543.47895286,340.69457203)(543.4589537,341.0245743)
\curveto(543.44895289,341.36457136)(543.44395289,341.70957102)(543.4439537,342.0595743)
\curveto(543.45395288,342.41957031)(543.45895288,342.77456995)(543.4589537,343.1245743)
\lineto(543.4589537,345.1645743)
\curveto(543.45895288,345.29456743)(543.45395288,345.44456728)(543.4439537,345.6145743)
\curveto(543.44395289,345.79456693)(543.46895287,345.9245668)(543.5189537,346.0045743)
\curveto(543.54895279,346.05456667)(543.60895273,346.09956663)(543.6989537,346.1395743)
\curveto(543.75895258,346.13956659)(543.80395253,346.14456658)(543.8339537,346.1545743)
}
}
{
\newrgbcolor{curcolor}{0 0 0}
\pscustom[linestyle=none,fillstyle=solid,fillcolor=curcolor]
{
\newpath
\moveto(556.7452037,346.3645743)
\curveto(556.85519839,346.36456636)(556.95019829,346.35456637)(557.0302037,346.3345743)
\curveto(557.12019812,346.31456641)(557.19019805,346.26956646)(557.2402037,346.1995743)
\curveto(557.30019794,346.11956661)(557.33019791,345.97956675)(557.3302037,345.7795743)
\lineto(557.3302037,345.2695743)
\lineto(557.3302037,344.8945743)
\curveto(557.3401979,344.75456797)(557.32519792,344.64456808)(557.2852037,344.5645743)
\curveto(557.245198,344.49456823)(557.18519806,344.44956828)(557.1052037,344.4295743)
\curveto(557.03519821,344.40956832)(556.95019829,344.39956833)(556.8502037,344.3995743)
\curveto(556.76019848,344.39956833)(556.66019858,344.40456832)(556.5502037,344.4145743)
\curveto(556.45019879,344.4245683)(556.35519889,344.41956831)(556.2652037,344.3995743)
\curveto(556.19519905,344.37956835)(556.12519912,344.36456836)(556.0552037,344.3545743)
\curveto(555.98519926,344.35456837)(555.92019932,344.34456838)(555.8602037,344.3245743)
\curveto(555.70019954,344.27456845)(555.5401997,344.19956853)(555.3802037,344.0995743)
\curveto(555.22020002,344.00956872)(555.09520015,343.90456882)(555.0052037,343.7845743)
\curveto(554.95520029,343.70456902)(554.90020034,343.61956911)(554.8402037,343.5295743)
\curveto(554.79020045,343.44956928)(554.7402005,343.36456936)(554.6902037,343.2745743)
\curveto(554.66020058,343.19456953)(554.63020061,343.10956962)(554.6002037,343.0195743)
\lineto(554.5402037,342.7795743)
\curveto(554.52020072,342.70957002)(554.51020073,342.63457009)(554.5102037,342.5545743)
\curveto(554.51020073,342.48457024)(554.50020074,342.41457031)(554.4802037,342.3445743)
\curveto(554.47020077,342.30457042)(554.46520078,342.26457046)(554.4652037,342.2245743)
\curveto(554.47520077,342.19457053)(554.47520077,342.16457056)(554.4652037,342.1345743)
\lineto(554.4652037,341.8945743)
\curveto(554.4452008,341.8245709)(554.4402008,341.74457098)(554.4502037,341.6545743)
\curveto(554.46020078,341.57457115)(554.46520078,341.49457123)(554.4652037,341.4145743)
\lineto(554.4652037,340.4545743)
\lineto(554.4652037,339.1795743)
\curveto(554.46520078,339.04957368)(554.46020078,338.9295738)(554.4502037,338.8195743)
\curveto(554.4402008,338.70957402)(554.41020083,338.61957411)(554.3602037,338.5495743)
\curveto(554.3402009,338.51957421)(554.30520094,338.49457423)(554.2552037,338.4745743)
\curveto(554.21520103,338.46457426)(554.17020107,338.45457427)(554.1202037,338.4445743)
\lineto(554.0452037,338.4445743)
\curveto(553.99520125,338.43457429)(553.9402013,338.4295743)(553.8802037,338.4295743)
\lineto(553.7152037,338.4295743)
\lineto(553.0702037,338.4295743)
\curveto(553.01020223,338.43957429)(552.9452023,338.44457428)(552.8752037,338.4445743)
\lineto(552.6802037,338.4445743)
\curveto(552.63020261,338.46457426)(552.58020266,338.47957425)(552.5302037,338.4895743)
\curveto(552.48020276,338.50957422)(552.4452028,338.54457418)(552.4252037,338.5945743)
\curveto(552.38520286,338.64457408)(552.36020288,338.71457401)(552.3502037,338.8045743)
\lineto(552.3502037,339.1045743)
\lineto(552.3502037,340.1245743)
\lineto(552.3502037,344.3545743)
\lineto(552.3502037,345.4645743)
\lineto(552.3502037,345.7495743)
\curveto(552.35020289,345.84956688)(552.37020287,345.9295668)(552.4102037,345.9895743)
\curveto(552.46020278,346.06956666)(552.53520271,346.11956661)(552.6352037,346.1395743)
\curveto(552.73520251,346.15956657)(552.85520239,346.16956656)(552.9952037,346.1695743)
\lineto(553.7602037,346.1695743)
\curveto(553.88020136,346.16956656)(553.98520126,346.15956657)(554.0752037,346.1395743)
\curveto(554.16520108,346.1295666)(554.23520101,346.08456664)(554.2852037,346.0045743)
\curveto(554.31520093,345.95456677)(554.33020091,345.88456684)(554.3302037,345.7945743)
\lineto(554.3602037,345.5245743)
\curveto(554.37020087,345.44456728)(554.38520086,345.36956736)(554.4052037,345.2995743)
\curveto(554.43520081,345.2295675)(554.48520076,345.19456753)(554.5552037,345.1945743)
\curveto(554.57520067,345.21456751)(554.59520065,345.2245675)(554.6152037,345.2245743)
\curveto(554.63520061,345.2245675)(554.65520059,345.23456749)(554.6752037,345.2545743)
\curveto(554.73520051,345.30456742)(554.78520046,345.35956737)(554.8252037,345.4195743)
\curveto(554.87520037,345.48956724)(554.93520031,345.54956718)(555.0052037,345.5995743)
\curveto(555.0452002,345.6295671)(555.08020016,345.65956707)(555.1102037,345.6895743)
\curveto(555.1402001,345.729567)(555.17520007,345.76456696)(555.2152037,345.7945743)
\lineto(555.4852037,345.9745743)
\curveto(555.58519966,346.03456669)(555.68519956,346.08956664)(555.7852037,346.1395743)
\curveto(555.88519936,346.17956655)(555.98519926,346.21456651)(556.0852037,346.2445743)
\lineto(556.4152037,346.3345743)
\curveto(556.4451988,346.34456638)(556.50019874,346.34456638)(556.5802037,346.3345743)
\curveto(556.67019857,346.33456639)(556.72519852,346.34456638)(556.7452037,346.3645743)
}
}
{
\newrgbcolor{curcolor}{0 0 0}
\pscustom[linestyle=none,fillstyle=solid,fillcolor=curcolor]
{
\newpath
\moveto(561.12028183,346.3795743)
\curveto(561.87027733,346.39956633)(562.52027668,346.31456641)(563.07028183,346.1245743)
\curveto(563.63027557,345.94456678)(564.05527514,345.6295671)(564.34528183,345.1795743)
\curveto(564.41527478,345.06956766)(564.47527472,344.95456777)(564.52528183,344.8345743)
\curveto(564.58527461,344.724568)(564.63527456,344.59956813)(564.67528183,344.4595743)
\curveto(564.6952745,344.39956833)(564.70527449,344.33456839)(564.70528183,344.2645743)
\curveto(564.70527449,344.19456853)(564.6952745,344.13456859)(564.67528183,344.0845743)
\curveto(564.63527456,344.0245687)(564.58027462,343.98456874)(564.51028183,343.9645743)
\curveto(564.46027474,343.94456878)(564.4002748,343.93456879)(564.33028183,343.9345743)
\lineto(564.12028183,343.9345743)
\lineto(563.46028183,343.9345743)
\curveto(563.39027581,343.93456879)(563.32027588,343.9295688)(563.25028183,343.9195743)
\curveto(563.18027602,343.91956881)(563.11527608,343.9295688)(563.05528183,343.9495743)
\curveto(562.95527624,343.96956876)(562.88027632,344.00956872)(562.83028183,344.0695743)
\curveto(562.78027642,344.1295686)(562.73527646,344.18956854)(562.69528183,344.2495743)
\lineto(562.57528183,344.4595743)
\curveto(562.54527665,344.53956819)(562.4952767,344.60456812)(562.42528183,344.6545743)
\curveto(562.32527687,344.73456799)(562.22527697,344.79456793)(562.12528183,344.8345743)
\curveto(562.03527716,344.87456785)(561.92027728,344.90956782)(561.78028183,344.9395743)
\curveto(561.71027749,344.95956777)(561.60527759,344.97456775)(561.46528183,344.9845743)
\curveto(561.33527786,344.99456773)(561.23527796,344.98956774)(561.16528183,344.9695743)
\lineto(561.06028183,344.9695743)
\lineto(560.91028183,344.9395743)
\curveto(560.87027833,344.93956779)(560.82527837,344.93456779)(560.77528183,344.9245743)
\curveto(560.60527859,344.87456785)(560.46527873,344.80456792)(560.35528183,344.7145743)
\curveto(560.25527894,344.63456809)(560.18527901,344.50956822)(560.14528183,344.3395743)
\curveto(560.12527907,344.26956846)(560.12527907,344.20456852)(560.14528183,344.1445743)
\curveto(560.16527903,344.08456864)(560.18527901,344.03456869)(560.20528183,343.9945743)
\curveto(560.27527892,343.87456885)(560.35527884,343.77956895)(560.44528183,343.7095743)
\curveto(560.54527865,343.63956909)(560.66027854,343.57956915)(560.79028183,343.5295743)
\curveto(560.98027822,343.44956928)(561.18527801,343.37956935)(561.40528183,343.3195743)
\lineto(562.09528183,343.1695743)
\curveto(562.33527686,343.1295696)(562.56527663,343.07956965)(562.78528183,343.0195743)
\curveto(563.01527618,342.96956976)(563.23027597,342.90456982)(563.43028183,342.8245743)
\curveto(563.52027568,342.78456994)(563.60527559,342.74956998)(563.68528183,342.7195743)
\curveto(563.77527542,342.69957003)(563.86027534,342.66457006)(563.94028183,342.6145743)
\curveto(564.13027507,342.49457023)(564.3002749,342.36457036)(564.45028183,342.2245743)
\curveto(564.61027459,342.08457064)(564.73527446,341.90957082)(564.82528183,341.6995743)
\curveto(564.85527434,341.6295711)(564.88027432,341.55957117)(564.90028183,341.4895743)
\curveto(564.92027428,341.41957131)(564.94027426,341.34457138)(564.96028183,341.2645743)
\curveto(564.97027423,341.20457152)(564.97527422,341.10957162)(564.97528183,340.9795743)
\curveto(564.98527421,340.85957187)(564.98527421,340.76457196)(564.97528183,340.6945743)
\lineto(564.97528183,340.6195743)
\curveto(564.95527424,340.55957217)(564.94027426,340.49957223)(564.93028183,340.4395743)
\curveto(564.93027427,340.38957234)(564.92527427,340.33957239)(564.91528183,340.2895743)
\curveto(564.84527435,339.98957274)(564.73527446,339.724573)(564.58528183,339.4945743)
\curveto(564.42527477,339.25457347)(564.23027497,339.05957367)(564.00028183,338.9095743)
\curveto(563.77027543,338.75957397)(563.51027569,338.6295741)(563.22028183,338.5195743)
\curveto(563.11027609,338.46957426)(562.99027621,338.43457429)(562.86028183,338.4145743)
\curveto(562.74027646,338.39457433)(562.62027658,338.36957436)(562.50028183,338.3395743)
\curveto(562.41027679,338.31957441)(562.31527688,338.30957442)(562.21528183,338.3095743)
\curveto(562.12527707,338.29957443)(562.03527716,338.28457444)(561.94528183,338.2645743)
\lineto(561.67528183,338.2645743)
\curveto(561.61527758,338.24457448)(561.51027769,338.23457449)(561.36028183,338.2345743)
\curveto(561.22027798,338.23457449)(561.12027808,338.24457448)(561.06028183,338.2645743)
\curveto(561.03027817,338.26457446)(560.9952782,338.26957446)(560.95528183,338.2795743)
\lineto(560.85028183,338.2795743)
\curveto(560.73027847,338.29957443)(560.61027859,338.31457441)(560.49028183,338.3245743)
\curveto(560.37027883,338.33457439)(560.25527894,338.35457437)(560.14528183,338.3845743)
\curveto(559.75527944,338.49457423)(559.41027979,338.61957411)(559.11028183,338.7595743)
\curveto(558.81028039,338.90957382)(558.55528064,339.1295736)(558.34528183,339.4195743)
\curveto(558.20528099,339.60957312)(558.08528111,339.8295729)(557.98528183,340.0795743)
\curveto(557.96528123,340.13957259)(557.94528125,340.21957251)(557.92528183,340.3195743)
\curveto(557.90528129,340.36957236)(557.89028131,340.43957229)(557.88028183,340.5295743)
\curveto(557.87028133,340.61957211)(557.87528132,340.69457203)(557.89528183,340.7545743)
\curveto(557.92528127,340.8245719)(557.97528122,340.87457185)(558.04528183,340.9045743)
\curveto(558.0952811,340.9245718)(558.15528104,340.93457179)(558.22528183,340.9345743)
\lineto(558.45028183,340.9345743)
\lineto(559.15528183,340.9345743)
\lineto(559.39528183,340.9345743)
\curveto(559.47527972,340.93457179)(559.54527965,340.9245718)(559.60528183,340.9045743)
\curveto(559.71527948,340.86457186)(559.78527941,340.79957193)(559.81528183,340.7095743)
\curveto(559.85527934,340.61957211)(559.9002793,340.5245722)(559.95028183,340.4245743)
\curveto(559.97027923,340.37457235)(560.00527919,340.30957242)(560.05528183,340.2295743)
\curveto(560.11527908,340.14957258)(560.16527903,340.09957263)(560.20528183,340.0795743)
\curveto(560.32527887,339.97957275)(560.44027876,339.89957283)(560.55028183,339.8395743)
\curveto(560.66027854,339.78957294)(560.8002784,339.73957299)(560.97028183,339.6895743)
\curveto(561.02027818,339.66957306)(561.07027813,339.65957307)(561.12028183,339.6595743)
\curveto(561.17027803,339.66957306)(561.22027798,339.66957306)(561.27028183,339.6595743)
\curveto(561.35027785,339.63957309)(561.43527776,339.6295731)(561.52528183,339.6295743)
\curveto(561.62527757,339.63957309)(561.71027749,339.65457307)(561.78028183,339.6745743)
\curveto(561.83027737,339.68457304)(561.87527732,339.68957304)(561.91528183,339.6895743)
\curveto(561.96527723,339.68957304)(562.01527718,339.69957303)(562.06528183,339.7195743)
\curveto(562.20527699,339.76957296)(562.33027687,339.8295729)(562.44028183,339.8995743)
\curveto(562.56027664,339.96957276)(562.65527654,340.05957267)(562.72528183,340.1695743)
\curveto(562.77527642,340.24957248)(562.81527638,340.37457235)(562.84528183,340.5445743)
\curveto(562.86527633,340.61457211)(562.86527633,340.67957205)(562.84528183,340.7395743)
\curveto(562.82527637,340.79957193)(562.80527639,340.84957188)(562.78528183,340.8895743)
\curveto(562.71527648,341.0295717)(562.62527657,341.13457159)(562.51528183,341.2045743)
\curveto(562.41527678,341.27457145)(562.2952769,341.33957139)(562.15528183,341.3995743)
\curveto(561.96527723,341.47957125)(561.76527743,341.54457118)(561.55528183,341.5945743)
\curveto(561.34527785,341.64457108)(561.13527806,341.69957103)(560.92528183,341.7595743)
\curveto(560.84527835,341.77957095)(560.76027844,341.79457093)(560.67028183,341.8045743)
\curveto(560.59027861,341.81457091)(560.51027869,341.8295709)(560.43028183,341.8495743)
\curveto(560.11027909,341.93957079)(559.80527939,342.0245707)(559.51528183,342.1045743)
\curveto(559.22527997,342.19457053)(558.96028024,342.3245704)(558.72028183,342.4945743)
\curveto(558.44028076,342.69457003)(558.23528096,342.96456976)(558.10528183,343.3045743)
\curveto(558.08528111,343.37456935)(558.06528113,343.46956926)(558.04528183,343.5895743)
\curveto(558.02528117,343.65956907)(558.01028119,343.74456898)(558.00028183,343.8445743)
\curveto(557.99028121,343.94456878)(557.9952812,344.03456869)(558.01528183,344.1145743)
\curveto(558.03528116,344.16456856)(558.04028116,344.20456852)(558.03028183,344.2345743)
\curveto(558.02028118,344.27456845)(558.02528117,344.31956841)(558.04528183,344.3695743)
\curveto(558.06528113,344.47956825)(558.08528111,344.57956815)(558.10528183,344.6695743)
\curveto(558.13528106,344.76956796)(558.17028103,344.86456786)(558.21028183,344.9545743)
\curveto(558.34028086,345.24456748)(558.52028068,345.47956725)(558.75028183,345.6595743)
\curveto(558.98028022,345.83956689)(559.24027996,345.98456674)(559.53028183,346.0945743)
\curveto(559.64027956,346.14456658)(559.75527944,346.17956655)(559.87528183,346.1995743)
\curveto(559.9952792,346.2295665)(560.12027908,346.25956647)(560.25028183,346.2895743)
\curveto(560.31027889,346.30956642)(560.37027883,346.31956641)(560.43028183,346.3195743)
\lineto(560.61028183,346.3495743)
\curveto(560.69027851,346.35956637)(560.77527842,346.36456636)(560.86528183,346.3645743)
\curveto(560.95527824,346.36456636)(561.04027816,346.36956636)(561.12028183,346.3795743)
}
}
{
\newrgbcolor{curcolor}{0 0 0}
\pscustom[linestyle=none,fillstyle=solid,fillcolor=curcolor]
{
\newpath
\moveto(573.97692245,342.6145743)
\curveto(573.99691388,342.55457017)(574.00691387,342.46957026)(574.00692245,342.3595743)
\curveto(574.00691387,342.24957048)(573.99691388,342.16457056)(573.97692245,342.1045743)
\lineto(573.97692245,341.9545743)
\curveto(573.95691392,341.87457085)(573.94691393,341.79457093)(573.94692245,341.7145743)
\curveto(573.95691392,341.63457109)(573.95191393,341.55457117)(573.93192245,341.4745743)
\curveto(573.91191397,341.40457132)(573.89691398,341.33957139)(573.88692245,341.2795743)
\curveto(573.876914,341.21957151)(573.86691401,341.15457157)(573.85692245,341.0845743)
\curveto(573.81691406,340.97457175)(573.7819141,340.85957187)(573.75192245,340.7395743)
\curveto(573.72191416,340.6295721)(573.6819142,340.5245722)(573.63192245,340.4245743)
\curveto(573.42191446,339.94457278)(573.14691473,339.55457317)(572.80692245,339.2545743)
\curveto(572.46691541,338.95457377)(572.05691582,338.70457402)(571.57692245,338.5045743)
\curveto(571.45691642,338.45457427)(571.33191655,338.41957431)(571.20192245,338.3995743)
\curveto(571.0819168,338.36957436)(570.95691692,338.33957439)(570.82692245,338.3095743)
\curveto(570.7769171,338.28957444)(570.72191716,338.27957445)(570.66192245,338.2795743)
\curveto(570.60191728,338.27957445)(570.54691733,338.27457445)(570.49692245,338.2645743)
\lineto(570.39192245,338.2645743)
\curveto(570.36191752,338.25457447)(570.33191755,338.24957448)(570.30192245,338.2495743)
\curveto(570.25191763,338.23957449)(570.17191771,338.23457449)(570.06192245,338.2345743)
\curveto(569.95191793,338.2245745)(569.86691801,338.2295745)(569.80692245,338.2495743)
\lineto(569.65692245,338.2495743)
\curveto(569.60691827,338.25957447)(569.55191833,338.26457446)(569.49192245,338.2645743)
\curveto(569.44191844,338.25457447)(569.39191849,338.25957447)(569.34192245,338.2795743)
\curveto(569.30191858,338.28957444)(569.26191862,338.29457443)(569.22192245,338.2945743)
\curveto(569.19191869,338.29457443)(569.15191873,338.29957443)(569.10192245,338.3095743)
\curveto(569.00191888,338.33957439)(568.90191898,338.36457436)(568.80192245,338.3845743)
\curveto(568.70191918,338.40457432)(568.60691927,338.43457429)(568.51692245,338.4745743)
\curveto(568.39691948,338.51457421)(568.2819196,338.55457417)(568.17192245,338.5945743)
\curveto(568.07191981,338.63457409)(567.96691991,338.68457404)(567.85692245,338.7445743)
\curveto(567.50692037,338.95457377)(567.20692067,339.19957353)(566.95692245,339.4795743)
\curveto(566.70692117,339.75957297)(566.49692138,340.09457263)(566.32692245,340.4845743)
\curveto(566.2769216,340.57457215)(566.23692164,340.66957206)(566.20692245,340.7695743)
\curveto(566.18692169,340.86957186)(566.16192172,340.97457175)(566.13192245,341.0845743)
\curveto(566.11192177,341.13457159)(566.10192178,341.17957155)(566.10192245,341.2195743)
\curveto(566.10192178,341.25957147)(566.09192179,341.30457142)(566.07192245,341.3545743)
\curveto(566.05192183,341.43457129)(566.04192184,341.51457121)(566.04192245,341.5945743)
\curveto(566.04192184,341.68457104)(566.03192185,341.76957096)(566.01192245,341.8495743)
\curveto(566.00192188,341.89957083)(565.99692188,341.94457078)(565.99692245,341.9845743)
\lineto(565.99692245,342.1195743)
\curveto(565.9769219,342.17957055)(565.96692191,342.26457046)(565.96692245,342.3745743)
\curveto(565.9769219,342.48457024)(565.99192189,342.56957016)(566.01192245,342.6295743)
\lineto(566.01192245,342.7345743)
\curveto(566.02192186,342.78456994)(566.02192186,342.83456989)(566.01192245,342.8845743)
\curveto(566.01192187,342.94456978)(566.02192186,342.99956973)(566.04192245,343.0495743)
\curveto(566.05192183,343.09956963)(566.05692182,343.14456958)(566.05692245,343.1845743)
\curveto(566.05692182,343.23456949)(566.06692181,343.28456944)(566.08692245,343.3345743)
\curveto(566.12692175,343.46456926)(566.16192172,343.58956914)(566.19192245,343.7095743)
\curveto(566.22192166,343.83956889)(566.26192162,343.96456876)(566.31192245,344.0845743)
\curveto(566.49192139,344.49456823)(566.70692117,344.83456789)(566.95692245,345.1045743)
\curveto(567.20692067,345.38456734)(567.51192037,345.63956709)(567.87192245,345.8695743)
\curveto(567.97191991,345.91956681)(568.0769198,345.96456676)(568.18692245,346.0045743)
\curveto(568.29691958,346.04456668)(568.40691947,346.08956664)(568.51692245,346.1395743)
\curveto(568.64691923,346.18956654)(568.7819191,346.2245665)(568.92192245,346.2445743)
\curveto(569.06191882,346.26456646)(569.20691867,346.29456643)(569.35692245,346.3345743)
\curveto(569.43691844,346.34456638)(569.51191837,346.34956638)(569.58192245,346.3495743)
\curveto(569.65191823,346.34956638)(569.72191816,346.35456637)(569.79192245,346.3645743)
\curveto(570.37191751,346.37456635)(570.87191701,346.31456641)(571.29192245,346.1845743)
\curveto(571.72191616,346.05456667)(572.10191578,345.87456685)(572.43192245,345.6445743)
\curveto(572.54191534,345.56456716)(572.65191523,345.47456725)(572.76192245,345.3745743)
\curveto(572.881915,345.28456744)(572.9819149,345.18456754)(573.06192245,345.0745743)
\curveto(573.14191474,344.97456775)(573.21191467,344.87456785)(573.27192245,344.7745743)
\curveto(573.34191454,344.67456805)(573.41191447,344.56956816)(573.48192245,344.4595743)
\curveto(573.55191433,344.34956838)(573.60691427,344.2295685)(573.64692245,344.0995743)
\curveto(573.68691419,343.97956875)(573.73191415,343.84956888)(573.78192245,343.7095743)
\curveto(573.81191407,343.6295691)(573.83691404,343.54456918)(573.85692245,343.4545743)
\lineto(573.91692245,343.1845743)
\curveto(573.92691395,343.14456958)(573.93191395,343.10456962)(573.93192245,343.0645743)
\curveto(573.93191395,343.0245697)(573.93691394,342.98456974)(573.94692245,342.9445743)
\curveto(573.96691391,342.89456983)(573.97191391,342.83956989)(573.96192245,342.7795743)
\curveto(573.95191393,342.71957001)(573.95691392,342.66457006)(573.97692245,342.6145743)
\moveto(571.87692245,342.0745743)
\curveto(571.88691599,342.1245706)(571.89191599,342.19457053)(571.89192245,342.2845743)
\curveto(571.89191599,342.38457034)(571.88691599,342.45957027)(571.87692245,342.5095743)
\lineto(571.87692245,342.6295743)
\curveto(571.85691602,342.67957005)(571.84691603,342.73456999)(571.84692245,342.7945743)
\curveto(571.84691603,342.85456987)(571.84191604,342.90956982)(571.83192245,342.9595743)
\curveto(571.83191605,342.99956973)(571.82691605,343.0295697)(571.81692245,343.0495743)
\lineto(571.75692245,343.2895743)
\curveto(571.74691613,343.37956935)(571.72691615,343.46456926)(571.69692245,343.5445743)
\curveto(571.58691629,343.80456892)(571.45691642,344.0245687)(571.30692245,344.2045743)
\curveto(571.15691672,344.39456833)(570.95691692,344.54456818)(570.70692245,344.6545743)
\curveto(570.64691723,344.67456805)(570.58691729,344.68956804)(570.52692245,344.6995743)
\curveto(570.46691741,344.71956801)(570.40191748,344.73956799)(570.33192245,344.7595743)
\curveto(570.25191763,344.77956795)(570.16691771,344.78456794)(570.07692245,344.7745743)
\lineto(569.80692245,344.7745743)
\curveto(569.7769181,344.75456797)(569.74191814,344.74456798)(569.70192245,344.7445743)
\curveto(569.66191822,344.75456797)(569.62691825,344.75456797)(569.59692245,344.7445743)
\lineto(569.38692245,344.6845743)
\curveto(569.32691855,344.67456805)(569.27191861,344.65456807)(569.22192245,344.6245743)
\curveto(568.97191891,344.51456821)(568.76691911,344.35456837)(568.60692245,344.1445743)
\curveto(568.45691942,343.94456878)(568.33691954,343.70956902)(568.24692245,343.4395743)
\curveto(568.21691966,343.33956939)(568.19191969,343.23456949)(568.17192245,343.1245743)
\curveto(568.16191972,343.01456971)(568.14691973,342.90456982)(568.12692245,342.7945743)
\curveto(568.11691976,342.74456998)(568.11191977,342.69457003)(568.11192245,342.6445743)
\lineto(568.11192245,342.4945743)
\curveto(568.09191979,342.4245703)(568.0819198,342.31957041)(568.08192245,342.1795743)
\curveto(568.09191979,342.03957069)(568.10691977,341.93457079)(568.12692245,341.8645743)
\lineto(568.12692245,341.7295743)
\curveto(568.14691973,341.64957108)(568.16191972,341.56957116)(568.17192245,341.4895743)
\curveto(568.1819197,341.41957131)(568.19691968,341.34457138)(568.21692245,341.2645743)
\curveto(568.31691956,340.96457176)(568.42191946,340.71957201)(568.53192245,340.5295743)
\curveto(568.65191923,340.34957238)(568.83691904,340.18457254)(569.08692245,340.0345743)
\curveto(569.15691872,339.98457274)(569.23191865,339.94457278)(569.31192245,339.9145743)
\curveto(569.40191848,339.88457284)(569.49191839,339.85957287)(569.58192245,339.8395743)
\curveto(569.62191826,339.8295729)(569.65691822,339.8245729)(569.68692245,339.8245743)
\curveto(569.71691816,339.83457289)(569.75191813,339.83457289)(569.79192245,339.8245743)
\lineto(569.91192245,339.7945743)
\curveto(569.96191792,339.79457293)(570.00691787,339.79957293)(570.04692245,339.8095743)
\lineto(570.16692245,339.8095743)
\curveto(570.24691763,339.8295729)(570.32691755,339.84457288)(570.40692245,339.8545743)
\curveto(570.48691739,339.86457286)(570.56191732,339.88457284)(570.63192245,339.9145743)
\curveto(570.89191699,340.01457271)(571.10191678,340.14957258)(571.26192245,340.3195743)
\curveto(571.42191646,340.48957224)(571.55691632,340.69957203)(571.66692245,340.9495743)
\curveto(571.70691617,341.04957168)(571.73691614,341.14957158)(571.75692245,341.2495743)
\curveto(571.7769161,341.34957138)(571.80191608,341.45457127)(571.83192245,341.5645743)
\curveto(571.84191604,341.60457112)(571.84691603,341.63957109)(571.84692245,341.6695743)
\curveto(571.84691603,341.70957102)(571.85191603,341.74957098)(571.86192245,341.7895743)
\lineto(571.86192245,341.9245743)
\curveto(571.86191602,341.97457075)(571.86691601,342.0245707)(571.87692245,342.0745743)
}
}
{
\newrgbcolor{curcolor}{0 0 0}
\pscustom[linestyle=none,fillstyle=solid,fillcolor=curcolor]
{
\newpath
\moveto(578.34684433,346.3795743)
\curveto(579.09683983,346.39956633)(579.74683918,346.31456641)(580.29684433,346.1245743)
\curveto(580.85683807,345.94456678)(581.28183764,345.6295671)(581.57184433,345.1795743)
\curveto(581.64183728,345.06956766)(581.70183722,344.95456777)(581.75184433,344.8345743)
\curveto(581.81183711,344.724568)(581.86183706,344.59956813)(581.90184433,344.4595743)
\curveto(581.921837,344.39956833)(581.93183699,344.33456839)(581.93184433,344.2645743)
\curveto(581.93183699,344.19456853)(581.921837,344.13456859)(581.90184433,344.0845743)
\curveto(581.86183706,344.0245687)(581.80683712,343.98456874)(581.73684433,343.9645743)
\curveto(581.68683724,343.94456878)(581.6268373,343.93456879)(581.55684433,343.9345743)
\lineto(581.34684433,343.9345743)
\lineto(580.68684433,343.9345743)
\curveto(580.61683831,343.93456879)(580.54683838,343.9295688)(580.47684433,343.9195743)
\curveto(580.40683852,343.91956881)(580.34183858,343.9295688)(580.28184433,343.9495743)
\curveto(580.18183874,343.96956876)(580.10683882,344.00956872)(580.05684433,344.0695743)
\curveto(580.00683892,344.1295686)(579.96183896,344.18956854)(579.92184433,344.2495743)
\lineto(579.80184433,344.4595743)
\curveto(579.77183915,344.53956819)(579.7218392,344.60456812)(579.65184433,344.6545743)
\curveto(579.55183937,344.73456799)(579.45183947,344.79456793)(579.35184433,344.8345743)
\curveto(579.26183966,344.87456785)(579.14683978,344.90956782)(579.00684433,344.9395743)
\curveto(578.93683999,344.95956777)(578.83184009,344.97456775)(578.69184433,344.9845743)
\curveto(578.56184036,344.99456773)(578.46184046,344.98956774)(578.39184433,344.9695743)
\lineto(578.28684433,344.9695743)
\lineto(578.13684433,344.9395743)
\curveto(578.09684083,344.93956779)(578.05184087,344.93456779)(578.00184433,344.9245743)
\curveto(577.83184109,344.87456785)(577.69184123,344.80456792)(577.58184433,344.7145743)
\curveto(577.48184144,344.63456809)(577.41184151,344.50956822)(577.37184433,344.3395743)
\curveto(577.35184157,344.26956846)(577.35184157,344.20456852)(577.37184433,344.1445743)
\curveto(577.39184153,344.08456864)(577.41184151,344.03456869)(577.43184433,343.9945743)
\curveto(577.50184142,343.87456885)(577.58184134,343.77956895)(577.67184433,343.7095743)
\curveto(577.77184115,343.63956909)(577.88684104,343.57956915)(578.01684433,343.5295743)
\curveto(578.20684072,343.44956928)(578.41184051,343.37956935)(578.63184433,343.3195743)
\lineto(579.32184433,343.1695743)
\curveto(579.56183936,343.1295696)(579.79183913,343.07956965)(580.01184433,343.0195743)
\curveto(580.24183868,342.96956976)(580.45683847,342.90456982)(580.65684433,342.8245743)
\curveto(580.74683818,342.78456994)(580.83183809,342.74956998)(580.91184433,342.7195743)
\curveto(581.00183792,342.69957003)(581.08683784,342.66457006)(581.16684433,342.6145743)
\curveto(581.35683757,342.49457023)(581.5268374,342.36457036)(581.67684433,342.2245743)
\curveto(581.83683709,342.08457064)(581.96183696,341.90957082)(582.05184433,341.6995743)
\curveto(582.08183684,341.6295711)(582.10683682,341.55957117)(582.12684433,341.4895743)
\curveto(582.14683678,341.41957131)(582.16683676,341.34457138)(582.18684433,341.2645743)
\curveto(582.19683673,341.20457152)(582.20183672,341.10957162)(582.20184433,340.9795743)
\curveto(582.21183671,340.85957187)(582.21183671,340.76457196)(582.20184433,340.6945743)
\lineto(582.20184433,340.6195743)
\curveto(582.18183674,340.55957217)(582.16683676,340.49957223)(582.15684433,340.4395743)
\curveto(582.15683677,340.38957234)(582.15183677,340.33957239)(582.14184433,340.2895743)
\curveto(582.07183685,339.98957274)(581.96183696,339.724573)(581.81184433,339.4945743)
\curveto(581.65183727,339.25457347)(581.45683747,339.05957367)(581.22684433,338.9095743)
\curveto(580.99683793,338.75957397)(580.73683819,338.6295741)(580.44684433,338.5195743)
\curveto(580.33683859,338.46957426)(580.21683871,338.43457429)(580.08684433,338.4145743)
\curveto(579.96683896,338.39457433)(579.84683908,338.36957436)(579.72684433,338.3395743)
\curveto(579.63683929,338.31957441)(579.54183938,338.30957442)(579.44184433,338.3095743)
\curveto(579.35183957,338.29957443)(579.26183966,338.28457444)(579.17184433,338.2645743)
\lineto(578.90184433,338.2645743)
\curveto(578.84184008,338.24457448)(578.73684019,338.23457449)(578.58684433,338.2345743)
\curveto(578.44684048,338.23457449)(578.34684058,338.24457448)(578.28684433,338.2645743)
\curveto(578.25684067,338.26457446)(578.2218407,338.26957446)(578.18184433,338.2795743)
\lineto(578.07684433,338.2795743)
\curveto(577.95684097,338.29957443)(577.83684109,338.31457441)(577.71684433,338.3245743)
\curveto(577.59684133,338.33457439)(577.48184144,338.35457437)(577.37184433,338.3845743)
\curveto(576.98184194,338.49457423)(576.63684229,338.61957411)(576.33684433,338.7595743)
\curveto(576.03684289,338.90957382)(575.78184314,339.1295736)(575.57184433,339.4195743)
\curveto(575.43184349,339.60957312)(575.31184361,339.8295729)(575.21184433,340.0795743)
\curveto(575.19184373,340.13957259)(575.17184375,340.21957251)(575.15184433,340.3195743)
\curveto(575.13184379,340.36957236)(575.11684381,340.43957229)(575.10684433,340.5295743)
\curveto(575.09684383,340.61957211)(575.10184382,340.69457203)(575.12184433,340.7545743)
\curveto(575.15184377,340.8245719)(575.20184372,340.87457185)(575.27184433,340.9045743)
\curveto(575.3218436,340.9245718)(575.38184354,340.93457179)(575.45184433,340.9345743)
\lineto(575.67684433,340.9345743)
\lineto(576.38184433,340.9345743)
\lineto(576.62184433,340.9345743)
\curveto(576.70184222,340.93457179)(576.77184215,340.9245718)(576.83184433,340.9045743)
\curveto(576.94184198,340.86457186)(577.01184191,340.79957193)(577.04184433,340.7095743)
\curveto(577.08184184,340.61957211)(577.1268418,340.5245722)(577.17684433,340.4245743)
\curveto(577.19684173,340.37457235)(577.23184169,340.30957242)(577.28184433,340.2295743)
\curveto(577.34184158,340.14957258)(577.39184153,340.09957263)(577.43184433,340.0795743)
\curveto(577.55184137,339.97957275)(577.66684126,339.89957283)(577.77684433,339.8395743)
\curveto(577.88684104,339.78957294)(578.0268409,339.73957299)(578.19684433,339.6895743)
\curveto(578.24684068,339.66957306)(578.29684063,339.65957307)(578.34684433,339.6595743)
\curveto(578.39684053,339.66957306)(578.44684048,339.66957306)(578.49684433,339.6595743)
\curveto(578.57684035,339.63957309)(578.66184026,339.6295731)(578.75184433,339.6295743)
\curveto(578.85184007,339.63957309)(578.93683999,339.65457307)(579.00684433,339.6745743)
\curveto(579.05683987,339.68457304)(579.10183982,339.68957304)(579.14184433,339.6895743)
\curveto(579.19183973,339.68957304)(579.24183968,339.69957303)(579.29184433,339.7195743)
\curveto(579.43183949,339.76957296)(579.55683937,339.8295729)(579.66684433,339.8995743)
\curveto(579.78683914,339.96957276)(579.88183904,340.05957267)(579.95184433,340.1695743)
\curveto(580.00183892,340.24957248)(580.04183888,340.37457235)(580.07184433,340.5445743)
\curveto(580.09183883,340.61457211)(580.09183883,340.67957205)(580.07184433,340.7395743)
\curveto(580.05183887,340.79957193)(580.03183889,340.84957188)(580.01184433,340.8895743)
\curveto(579.94183898,341.0295717)(579.85183907,341.13457159)(579.74184433,341.2045743)
\curveto(579.64183928,341.27457145)(579.5218394,341.33957139)(579.38184433,341.3995743)
\curveto(579.19183973,341.47957125)(578.99183993,341.54457118)(578.78184433,341.5945743)
\curveto(578.57184035,341.64457108)(578.36184056,341.69957103)(578.15184433,341.7595743)
\curveto(578.07184085,341.77957095)(577.98684094,341.79457093)(577.89684433,341.8045743)
\curveto(577.81684111,341.81457091)(577.73684119,341.8295709)(577.65684433,341.8495743)
\curveto(577.33684159,341.93957079)(577.03184189,342.0245707)(576.74184433,342.1045743)
\curveto(576.45184247,342.19457053)(576.18684274,342.3245704)(575.94684433,342.4945743)
\curveto(575.66684326,342.69457003)(575.46184346,342.96456976)(575.33184433,343.3045743)
\curveto(575.31184361,343.37456935)(575.29184363,343.46956926)(575.27184433,343.5895743)
\curveto(575.25184367,343.65956907)(575.23684369,343.74456898)(575.22684433,343.8445743)
\curveto(575.21684371,343.94456878)(575.2218437,344.03456869)(575.24184433,344.1145743)
\curveto(575.26184366,344.16456856)(575.26684366,344.20456852)(575.25684433,344.2345743)
\curveto(575.24684368,344.27456845)(575.25184367,344.31956841)(575.27184433,344.3695743)
\curveto(575.29184363,344.47956825)(575.31184361,344.57956815)(575.33184433,344.6695743)
\curveto(575.36184356,344.76956796)(575.39684353,344.86456786)(575.43684433,344.9545743)
\curveto(575.56684336,345.24456748)(575.74684318,345.47956725)(575.97684433,345.6595743)
\curveto(576.20684272,345.83956689)(576.46684246,345.98456674)(576.75684433,346.0945743)
\curveto(576.86684206,346.14456658)(576.98184194,346.17956655)(577.10184433,346.1995743)
\curveto(577.2218417,346.2295665)(577.34684158,346.25956647)(577.47684433,346.2895743)
\curveto(577.53684139,346.30956642)(577.59684133,346.31956641)(577.65684433,346.3195743)
\lineto(577.83684433,346.3495743)
\curveto(577.91684101,346.35956637)(578.00184092,346.36456636)(578.09184433,346.3645743)
\curveto(578.18184074,346.36456636)(578.26684066,346.36956636)(578.34684433,346.3795743)
}
}
{
\newrgbcolor{curcolor}{0 0 0}
\pscustom[linestyle=none,fillstyle=solid,fillcolor=curcolor]
{
}
}
{
\newrgbcolor{curcolor}{0 0 0}
\pscustom[linestyle=none,fillstyle=solid,fillcolor=curcolor]
{
\newpath
\moveto(595.4836412,342.3895743)
\curveto(595.49363252,342.3295704)(595.49863252,342.23957049)(595.4986412,342.1195743)
\curveto(595.49863252,341.99957073)(595.48863253,341.91457081)(595.4686412,341.8645743)
\lineto(595.4686412,341.6695743)
\curveto(595.43863258,341.55957117)(595.4186326,341.45457127)(595.4086412,341.3545743)
\curveto(595.40863261,341.25457147)(595.39363262,341.15457157)(595.3636412,341.0545743)
\curveto(595.34363267,340.96457176)(595.32363269,340.86957186)(595.3036412,340.7695743)
\curveto(595.28363273,340.67957205)(595.25363276,340.58957214)(595.2136412,340.4995743)
\curveto(595.14363287,340.3295724)(595.07363294,340.16957256)(595.0036412,340.0195743)
\curveto(594.93363308,339.87957285)(594.85363316,339.73957299)(594.7636412,339.5995743)
\curveto(594.70363331,339.50957322)(594.63863338,339.4245733)(594.5686412,339.3445743)
\curveto(594.50863351,339.27457345)(594.43863358,339.19957353)(594.3586412,339.1195743)
\lineto(594.2536412,339.0145743)
\curveto(594.20363381,338.96457376)(594.14863387,338.91957381)(594.0886412,338.8795743)
\lineto(593.9386412,338.7595743)
\curveto(593.85863416,338.69957403)(593.76863425,338.64457408)(593.6686412,338.5945743)
\curveto(593.57863444,338.55457417)(593.48363453,338.50957422)(593.3836412,338.4595743)
\curveto(593.28363473,338.40957432)(593.17863484,338.37457435)(593.0686412,338.3545743)
\curveto(592.96863505,338.33457439)(592.86363515,338.31457441)(592.7536412,338.2945743)
\curveto(592.69363532,338.27457445)(592.62863539,338.26457446)(592.5586412,338.2645743)
\curveto(592.49863552,338.26457446)(592.43363558,338.25457447)(592.3636412,338.2345743)
\lineto(592.2286412,338.2345743)
\curveto(592.14863587,338.21457451)(592.07363594,338.21457451)(592.0036412,338.2345743)
\lineto(591.8536412,338.2345743)
\curveto(591.79363622,338.25457447)(591.72863629,338.26457446)(591.6586412,338.2645743)
\curveto(591.59863642,338.25457447)(591.53863648,338.25957447)(591.4786412,338.2795743)
\curveto(591.3186367,338.3295744)(591.16363685,338.37457435)(591.0136412,338.4145743)
\curveto(590.87363714,338.45457427)(590.74363727,338.51457421)(590.6236412,338.5945743)
\curveto(590.55363746,338.63457409)(590.48863753,338.67457405)(590.4286412,338.7145743)
\curveto(590.36863765,338.76457396)(590.30363771,338.81457391)(590.2336412,338.8645743)
\lineto(590.0536412,338.9995743)
\curveto(589.97363804,339.05957367)(589.90363811,339.06457366)(589.8436412,339.0145743)
\curveto(589.79363822,338.98457374)(589.76863825,338.94457378)(589.7686412,338.8945743)
\curveto(589.76863825,338.85457387)(589.75863826,338.80457392)(589.7386412,338.7445743)
\curveto(589.7186383,338.64457408)(589.70863831,338.5295742)(589.7086412,338.3995743)
\curveto(589.7186383,338.26957446)(589.72363829,338.14957458)(589.7236412,338.0395743)
\lineto(589.7236412,336.5095743)
\curveto(589.72363829,336.37957635)(589.7186383,336.25457647)(589.7086412,336.1345743)
\curveto(589.70863831,336.00457672)(589.68363833,335.89957683)(589.6336412,335.8195743)
\curveto(589.60363841,335.77957695)(589.54863847,335.74957698)(589.4686412,335.7295743)
\curveto(589.38863863,335.70957702)(589.29863872,335.69957703)(589.1986412,335.6995743)
\curveto(589.09863892,335.68957704)(588.99863902,335.68957704)(588.8986412,335.6995743)
\lineto(588.6436412,335.6995743)
\lineto(588.2386412,335.6995743)
\lineto(588.1336412,335.6995743)
\curveto(588.09363992,335.69957703)(588.05863996,335.70457702)(588.0286412,335.7145743)
\lineto(587.9086412,335.7145743)
\curveto(587.73864028,335.76457696)(587.64864037,335.86457686)(587.6386412,336.0145743)
\curveto(587.62864039,336.15457657)(587.62364039,336.3245764)(587.6236412,336.5245743)
\lineto(587.6236412,345.3295743)
\curveto(587.62364039,345.43956729)(587.6186404,345.55456717)(587.6086412,345.6745743)
\curveto(587.60864041,345.80456692)(587.63364038,345.90456682)(587.6836412,345.9745743)
\curveto(587.72364029,346.04456668)(587.77864024,346.08956664)(587.8486412,346.1095743)
\curveto(587.89864012,346.1295666)(587.95864006,346.13956659)(588.0286412,346.1395743)
\lineto(588.2536412,346.1395743)
\lineto(588.9736412,346.1395743)
\lineto(589.2586412,346.1395743)
\curveto(589.34863867,346.13956659)(589.42363859,346.11456661)(589.4836412,346.0645743)
\curveto(589.55363846,346.01456671)(589.58863843,345.94956678)(589.5886412,345.8695743)
\curveto(589.59863842,345.79956693)(589.62363839,345.724567)(589.6636412,345.6445743)
\curveto(589.67363834,345.61456711)(589.68363833,345.58956714)(589.6936412,345.5695743)
\curveto(589.7136383,345.55956717)(589.73363828,345.54456718)(589.7536412,345.5245743)
\curveto(589.86363815,345.51456721)(589.95363806,345.54456718)(590.0236412,345.6145743)
\curveto(590.09363792,345.68456704)(590.16363785,345.74456698)(590.2336412,345.7945743)
\curveto(590.36363765,345.88456684)(590.49863752,345.96456676)(590.6386412,346.0345743)
\curveto(590.77863724,346.11456661)(590.93363708,346.17956655)(591.1036412,346.2295743)
\curveto(591.18363683,346.25956647)(591.26863675,346.27956645)(591.3586412,346.2895743)
\curveto(591.45863656,346.29956643)(591.55363646,346.31456641)(591.6436412,346.3345743)
\curveto(591.68363633,346.34456638)(591.72363629,346.34456638)(591.7636412,346.3345743)
\curveto(591.8136362,346.3245664)(591.85363616,346.3295664)(591.8836412,346.3495743)
\curveto(592.45363556,346.36956636)(592.93363508,346.28956644)(593.3236412,346.1095743)
\curveto(593.72363429,345.93956679)(594.06363395,345.71456701)(594.3436412,345.4345743)
\curveto(594.39363362,345.38456734)(594.43863358,345.33456739)(594.4786412,345.2845743)
\curveto(594.5186335,345.24456748)(594.55863346,345.19956753)(594.5986412,345.1495743)
\curveto(594.66863335,345.05956767)(594.72863329,344.96956776)(594.7786412,344.8795743)
\curveto(594.83863318,344.78956794)(594.89363312,344.69956803)(594.9436412,344.6095743)
\curveto(594.96363305,344.58956814)(594.97363304,344.56456816)(594.9736412,344.5345743)
\curveto(594.98363303,344.50456822)(594.99863302,344.46956826)(595.0186412,344.4295743)
\curveto(595.07863294,344.3295684)(595.13363288,344.20956852)(595.1836412,344.0695743)
\curveto(595.20363281,344.00956872)(595.22363279,343.94456878)(595.2436412,343.8745743)
\curveto(595.26363275,343.81456891)(595.28363273,343.74956898)(595.3036412,343.6795743)
\curveto(595.34363267,343.55956917)(595.36863265,343.43456929)(595.3786412,343.3045743)
\curveto(595.39863262,343.17456955)(595.42363259,343.03956969)(595.4536412,342.8995743)
\lineto(595.4536412,342.7345743)
\lineto(595.4836412,342.5545743)
\lineto(595.4836412,342.3895743)
\moveto(593.3686412,342.0445743)
\curveto(593.37863464,342.09457063)(593.38363463,342.15957057)(593.3836412,342.2395743)
\curveto(593.38363463,342.3295704)(593.37863464,342.39957033)(593.3686412,342.4495743)
\lineto(593.3686412,342.5845743)
\curveto(593.34863467,342.64457008)(593.33863468,342.70957002)(593.3386412,342.7795743)
\curveto(593.33863468,342.84956988)(593.32863469,342.91956981)(593.3086412,342.9895743)
\curveto(593.28863473,343.08956964)(593.26863475,343.18456954)(593.2486412,343.2745743)
\curveto(593.22863479,343.37456935)(593.19863482,343.46456926)(593.1586412,343.5445743)
\curveto(593.03863498,343.86456886)(592.88363513,344.11956861)(592.6936412,344.3095743)
\curveto(592.50363551,344.49956823)(592.23363578,344.63956809)(591.8836412,344.7295743)
\curveto(591.80363621,344.74956798)(591.7136363,344.75956797)(591.6136412,344.7595743)
\lineto(591.3436412,344.7595743)
\curveto(591.30363671,344.74956798)(591.26863675,344.74456798)(591.2386412,344.7445743)
\curveto(591.20863681,344.74456798)(591.17363684,344.73956799)(591.1336412,344.7295743)
\lineto(590.9236412,344.6695743)
\curveto(590.86363715,344.65956807)(590.80363721,344.63956809)(590.7436412,344.6095743)
\curveto(590.48363753,344.49956823)(590.27863774,344.3295684)(590.1286412,344.0995743)
\curveto(589.98863803,343.86956886)(589.87363814,343.61456911)(589.7836412,343.3345743)
\curveto(589.76363825,343.25456947)(589.74863827,343.16956956)(589.7386412,343.0795743)
\curveto(589.72863829,342.99956973)(589.7136383,342.91956981)(589.6936412,342.8395743)
\curveto(589.68363833,342.79956993)(589.67863834,342.73456999)(589.6786412,342.6445743)
\curveto(589.65863836,342.60457012)(589.65363836,342.55457017)(589.6636412,342.4945743)
\curveto(589.67363834,342.44457028)(589.67363834,342.39457033)(589.6636412,342.3445743)
\curveto(589.64363837,342.28457044)(589.64363837,342.2295705)(589.6636412,342.1795743)
\lineto(589.6636412,341.9995743)
\lineto(589.6636412,341.8645743)
\curveto(589.66363835,341.8245709)(589.67363834,341.78457094)(589.6936412,341.7445743)
\curveto(589.69363832,341.67457105)(589.69863832,341.61957111)(589.7086412,341.5795743)
\lineto(589.7386412,341.3995743)
\curveto(589.74863827,341.33957139)(589.76363825,341.27957145)(589.7836412,341.2195743)
\curveto(589.87363814,340.9295718)(589.97863804,340.68957204)(590.0986412,340.4995743)
\curveto(590.22863779,340.31957241)(590.40863761,340.15957257)(590.6386412,340.0195743)
\curveto(590.77863724,339.93957279)(590.94363707,339.87457285)(591.1336412,339.8245743)
\curveto(591.17363684,339.81457291)(591.20863681,339.80957292)(591.2386412,339.8095743)
\curveto(591.26863675,339.81957291)(591.30363671,339.81957291)(591.3436412,339.8095743)
\curveto(591.38363663,339.79957293)(591.44363657,339.78957294)(591.5236412,339.7795743)
\curveto(591.60363641,339.77957295)(591.66863635,339.78457294)(591.7186412,339.7945743)
\curveto(591.79863622,339.81457291)(591.87863614,339.8295729)(591.9586412,339.8395743)
\curveto(592.04863597,339.85957287)(592.13363588,339.88457284)(592.2136412,339.9145743)
\curveto(592.45363556,340.01457271)(592.64863537,340.15457257)(592.7986412,340.3345743)
\curveto(592.94863507,340.51457221)(593.07363494,340.724572)(593.1736412,340.9645743)
\curveto(593.22363479,341.08457164)(593.25863476,341.20957152)(593.2786412,341.3395743)
\curveto(593.29863472,341.46957126)(593.32363469,341.60457112)(593.3536412,341.7445743)
\lineto(593.3536412,341.8945743)
\curveto(593.36363465,341.94457078)(593.36863465,341.99457073)(593.3686412,342.0445743)
}
}
{
\newrgbcolor{curcolor}{0 0 0}
\pscustom[linestyle=none,fillstyle=solid,fillcolor=curcolor]
{
\newpath
\moveto(604.53356308,342.6145743)
\curveto(604.55355451,342.55457017)(604.5635545,342.46957026)(604.56356308,342.3595743)
\curveto(604.5635545,342.24957048)(604.55355451,342.16457056)(604.53356308,342.1045743)
\lineto(604.53356308,341.9545743)
\curveto(604.51355455,341.87457085)(604.50355456,341.79457093)(604.50356308,341.7145743)
\curveto(604.51355455,341.63457109)(604.50855455,341.55457117)(604.48856308,341.4745743)
\curveto(604.46855459,341.40457132)(604.45355461,341.33957139)(604.44356308,341.2795743)
\curveto(604.43355463,341.21957151)(604.42355464,341.15457157)(604.41356308,341.0845743)
\curveto(604.37355469,340.97457175)(604.33855472,340.85957187)(604.30856308,340.7395743)
\curveto(604.27855478,340.6295721)(604.23855482,340.5245722)(604.18856308,340.4245743)
\curveto(603.97855508,339.94457278)(603.70355536,339.55457317)(603.36356308,339.2545743)
\curveto(603.02355604,338.95457377)(602.61355645,338.70457402)(602.13356308,338.5045743)
\curveto(602.01355705,338.45457427)(601.88855717,338.41957431)(601.75856308,338.3995743)
\curveto(601.63855742,338.36957436)(601.51355755,338.33957439)(601.38356308,338.3095743)
\curveto(601.33355773,338.28957444)(601.27855778,338.27957445)(601.21856308,338.2795743)
\curveto(601.1585579,338.27957445)(601.10355796,338.27457445)(601.05356308,338.2645743)
\lineto(600.94856308,338.2645743)
\curveto(600.91855814,338.25457447)(600.88855817,338.24957448)(600.85856308,338.2495743)
\curveto(600.80855825,338.23957449)(600.72855833,338.23457449)(600.61856308,338.2345743)
\curveto(600.50855855,338.2245745)(600.42355864,338.2295745)(600.36356308,338.2495743)
\lineto(600.21356308,338.2495743)
\curveto(600.1635589,338.25957447)(600.10855895,338.26457446)(600.04856308,338.2645743)
\curveto(599.99855906,338.25457447)(599.94855911,338.25957447)(599.89856308,338.2795743)
\curveto(599.8585592,338.28957444)(599.81855924,338.29457443)(599.77856308,338.2945743)
\curveto(599.74855931,338.29457443)(599.70855935,338.29957443)(599.65856308,338.3095743)
\curveto(599.5585595,338.33957439)(599.4585596,338.36457436)(599.35856308,338.3845743)
\curveto(599.2585598,338.40457432)(599.1635599,338.43457429)(599.07356308,338.4745743)
\curveto(598.95356011,338.51457421)(598.83856022,338.55457417)(598.72856308,338.5945743)
\curveto(598.62856043,338.63457409)(598.52356054,338.68457404)(598.41356308,338.7445743)
\curveto(598.063561,338.95457377)(597.7635613,339.19957353)(597.51356308,339.4795743)
\curveto(597.2635618,339.75957297)(597.05356201,340.09457263)(596.88356308,340.4845743)
\curveto(596.83356223,340.57457215)(596.79356227,340.66957206)(596.76356308,340.7695743)
\curveto(596.74356232,340.86957186)(596.71856234,340.97457175)(596.68856308,341.0845743)
\curveto(596.66856239,341.13457159)(596.6585624,341.17957155)(596.65856308,341.2195743)
\curveto(596.6585624,341.25957147)(596.64856241,341.30457142)(596.62856308,341.3545743)
\curveto(596.60856245,341.43457129)(596.59856246,341.51457121)(596.59856308,341.5945743)
\curveto(596.59856246,341.68457104)(596.58856247,341.76957096)(596.56856308,341.8495743)
\curveto(596.5585625,341.89957083)(596.55356251,341.94457078)(596.55356308,341.9845743)
\lineto(596.55356308,342.1195743)
\curveto(596.53356253,342.17957055)(596.52356254,342.26457046)(596.52356308,342.3745743)
\curveto(596.53356253,342.48457024)(596.54856251,342.56957016)(596.56856308,342.6295743)
\lineto(596.56856308,342.7345743)
\curveto(596.57856248,342.78456994)(596.57856248,342.83456989)(596.56856308,342.8845743)
\curveto(596.56856249,342.94456978)(596.57856248,342.99956973)(596.59856308,343.0495743)
\curveto(596.60856245,343.09956963)(596.61356245,343.14456958)(596.61356308,343.1845743)
\curveto(596.61356245,343.23456949)(596.62356244,343.28456944)(596.64356308,343.3345743)
\curveto(596.68356238,343.46456926)(596.71856234,343.58956914)(596.74856308,343.7095743)
\curveto(596.77856228,343.83956889)(596.81856224,343.96456876)(596.86856308,344.0845743)
\curveto(597.04856201,344.49456823)(597.2635618,344.83456789)(597.51356308,345.1045743)
\curveto(597.7635613,345.38456734)(598.06856099,345.63956709)(598.42856308,345.8695743)
\curveto(598.52856053,345.91956681)(598.63356043,345.96456676)(598.74356308,346.0045743)
\curveto(598.85356021,346.04456668)(598.9635601,346.08956664)(599.07356308,346.1395743)
\curveto(599.20355986,346.18956654)(599.33855972,346.2245665)(599.47856308,346.2445743)
\curveto(599.61855944,346.26456646)(599.7635593,346.29456643)(599.91356308,346.3345743)
\curveto(599.99355907,346.34456638)(600.06855899,346.34956638)(600.13856308,346.3495743)
\curveto(600.20855885,346.34956638)(600.27855878,346.35456637)(600.34856308,346.3645743)
\curveto(600.92855813,346.37456635)(601.42855763,346.31456641)(601.84856308,346.1845743)
\curveto(602.27855678,346.05456667)(602.6585564,345.87456685)(602.98856308,345.6445743)
\curveto(603.09855596,345.56456716)(603.20855585,345.47456725)(603.31856308,345.3745743)
\curveto(603.43855562,345.28456744)(603.53855552,345.18456754)(603.61856308,345.0745743)
\curveto(603.69855536,344.97456775)(603.76855529,344.87456785)(603.82856308,344.7745743)
\curveto(603.89855516,344.67456805)(603.96855509,344.56956816)(604.03856308,344.4595743)
\curveto(604.10855495,344.34956838)(604.1635549,344.2295685)(604.20356308,344.0995743)
\curveto(604.24355482,343.97956875)(604.28855477,343.84956888)(604.33856308,343.7095743)
\curveto(604.36855469,343.6295691)(604.39355467,343.54456918)(604.41356308,343.4545743)
\lineto(604.47356308,343.1845743)
\curveto(604.48355458,343.14456958)(604.48855457,343.10456962)(604.48856308,343.0645743)
\curveto(604.48855457,343.0245697)(604.49355457,342.98456974)(604.50356308,342.9445743)
\curveto(604.52355454,342.89456983)(604.52855453,342.83956989)(604.51856308,342.7795743)
\curveto(604.50855455,342.71957001)(604.51355455,342.66457006)(604.53356308,342.6145743)
\moveto(602.43356308,342.0745743)
\curveto(602.44355662,342.1245706)(602.44855661,342.19457053)(602.44856308,342.2845743)
\curveto(602.44855661,342.38457034)(602.44355662,342.45957027)(602.43356308,342.5095743)
\lineto(602.43356308,342.6295743)
\curveto(602.41355665,342.67957005)(602.40355666,342.73456999)(602.40356308,342.7945743)
\curveto(602.40355666,342.85456987)(602.39855666,342.90956982)(602.38856308,342.9595743)
\curveto(602.38855667,342.99956973)(602.38355668,343.0295697)(602.37356308,343.0495743)
\lineto(602.31356308,343.2895743)
\curveto(602.30355676,343.37956935)(602.28355678,343.46456926)(602.25356308,343.5445743)
\curveto(602.14355692,343.80456892)(602.01355705,344.0245687)(601.86356308,344.2045743)
\curveto(601.71355735,344.39456833)(601.51355755,344.54456818)(601.26356308,344.6545743)
\curveto(601.20355786,344.67456805)(601.14355792,344.68956804)(601.08356308,344.6995743)
\curveto(601.02355804,344.71956801)(600.9585581,344.73956799)(600.88856308,344.7595743)
\curveto(600.80855825,344.77956795)(600.72355834,344.78456794)(600.63356308,344.7745743)
\lineto(600.36356308,344.7745743)
\curveto(600.33355873,344.75456797)(600.29855876,344.74456798)(600.25856308,344.7445743)
\curveto(600.21855884,344.75456797)(600.18355888,344.75456797)(600.15356308,344.7445743)
\lineto(599.94356308,344.6845743)
\curveto(599.88355918,344.67456805)(599.82855923,344.65456807)(599.77856308,344.6245743)
\curveto(599.52855953,344.51456821)(599.32355974,344.35456837)(599.16356308,344.1445743)
\curveto(599.01356005,343.94456878)(598.89356017,343.70956902)(598.80356308,343.4395743)
\curveto(598.77356029,343.33956939)(598.74856031,343.23456949)(598.72856308,343.1245743)
\curveto(598.71856034,343.01456971)(598.70356036,342.90456982)(598.68356308,342.7945743)
\curveto(598.67356039,342.74456998)(598.66856039,342.69457003)(598.66856308,342.6445743)
\lineto(598.66856308,342.4945743)
\curveto(598.64856041,342.4245703)(598.63856042,342.31957041)(598.63856308,342.1795743)
\curveto(598.64856041,342.03957069)(598.6635604,341.93457079)(598.68356308,341.8645743)
\lineto(598.68356308,341.7295743)
\curveto(598.70356036,341.64957108)(598.71856034,341.56957116)(598.72856308,341.4895743)
\curveto(598.73856032,341.41957131)(598.75356031,341.34457138)(598.77356308,341.2645743)
\curveto(598.87356019,340.96457176)(598.97856008,340.71957201)(599.08856308,340.5295743)
\curveto(599.20855985,340.34957238)(599.39355967,340.18457254)(599.64356308,340.0345743)
\curveto(599.71355935,339.98457274)(599.78855927,339.94457278)(599.86856308,339.9145743)
\curveto(599.9585591,339.88457284)(600.04855901,339.85957287)(600.13856308,339.8395743)
\curveto(600.17855888,339.8295729)(600.21355885,339.8245729)(600.24356308,339.8245743)
\curveto(600.27355879,339.83457289)(600.30855875,339.83457289)(600.34856308,339.8245743)
\lineto(600.46856308,339.7945743)
\curveto(600.51855854,339.79457293)(600.5635585,339.79957293)(600.60356308,339.8095743)
\lineto(600.72356308,339.8095743)
\curveto(600.80355826,339.8295729)(600.88355818,339.84457288)(600.96356308,339.8545743)
\curveto(601.04355802,339.86457286)(601.11855794,339.88457284)(601.18856308,339.9145743)
\curveto(601.44855761,340.01457271)(601.6585574,340.14957258)(601.81856308,340.3195743)
\curveto(601.97855708,340.48957224)(602.11355695,340.69957203)(602.22356308,340.9495743)
\curveto(602.2635568,341.04957168)(602.29355677,341.14957158)(602.31356308,341.2495743)
\curveto(602.33355673,341.34957138)(602.3585567,341.45457127)(602.38856308,341.5645743)
\curveto(602.39855666,341.60457112)(602.40355666,341.63957109)(602.40356308,341.6695743)
\curveto(602.40355666,341.70957102)(602.40855665,341.74957098)(602.41856308,341.7895743)
\lineto(602.41856308,341.9245743)
\curveto(602.41855664,341.97457075)(602.42355664,342.0245707)(602.43356308,342.0745743)
}
}
{
\newrgbcolor{curcolor}{0 0 0}
\pscustom[linestyle=none,fillstyle=solid,fillcolor=curcolor]
{
\newpath
\moveto(610.35848495,346.3645743)
\curveto(610.46847964,346.36456636)(610.56347954,346.35456637)(610.64348495,346.3345743)
\curveto(610.73347937,346.31456641)(610.8034793,346.26956646)(610.85348495,346.1995743)
\curveto(610.91347919,346.11956661)(610.94347916,345.97956675)(610.94348495,345.7795743)
\lineto(610.94348495,345.2695743)
\lineto(610.94348495,344.8945743)
\curveto(610.95347915,344.75456797)(610.93847917,344.64456808)(610.89848495,344.5645743)
\curveto(610.85847925,344.49456823)(610.79847931,344.44956828)(610.71848495,344.4295743)
\curveto(610.64847946,344.40956832)(610.56347954,344.39956833)(610.46348495,344.3995743)
\curveto(610.37347973,344.39956833)(610.27347983,344.40456832)(610.16348495,344.4145743)
\curveto(610.06348004,344.4245683)(609.96848014,344.41956831)(609.87848495,344.3995743)
\curveto(609.8084803,344.37956835)(609.73848037,344.36456836)(609.66848495,344.3545743)
\curveto(609.59848051,344.35456837)(609.53348057,344.34456838)(609.47348495,344.3245743)
\curveto(609.31348079,344.27456845)(609.15348095,344.19956853)(608.99348495,344.0995743)
\curveto(608.83348127,344.00956872)(608.7084814,343.90456882)(608.61848495,343.7845743)
\curveto(608.56848154,343.70456902)(608.51348159,343.61956911)(608.45348495,343.5295743)
\curveto(608.4034817,343.44956928)(608.35348175,343.36456936)(608.30348495,343.2745743)
\curveto(608.27348183,343.19456953)(608.24348186,343.10956962)(608.21348495,343.0195743)
\lineto(608.15348495,342.7795743)
\curveto(608.13348197,342.70957002)(608.12348198,342.63457009)(608.12348495,342.5545743)
\curveto(608.12348198,342.48457024)(608.11348199,342.41457031)(608.09348495,342.3445743)
\curveto(608.08348202,342.30457042)(608.07848203,342.26457046)(608.07848495,342.2245743)
\curveto(608.08848202,342.19457053)(608.08848202,342.16457056)(608.07848495,342.1345743)
\lineto(608.07848495,341.8945743)
\curveto(608.05848205,341.8245709)(608.05348205,341.74457098)(608.06348495,341.6545743)
\curveto(608.07348203,341.57457115)(608.07848203,341.49457123)(608.07848495,341.4145743)
\lineto(608.07848495,340.4545743)
\lineto(608.07848495,339.1795743)
\curveto(608.07848203,339.04957368)(608.07348203,338.9295738)(608.06348495,338.8195743)
\curveto(608.05348205,338.70957402)(608.02348208,338.61957411)(607.97348495,338.5495743)
\curveto(607.95348215,338.51957421)(607.91848219,338.49457423)(607.86848495,338.4745743)
\curveto(607.82848228,338.46457426)(607.78348232,338.45457427)(607.73348495,338.4445743)
\lineto(607.65848495,338.4445743)
\curveto(607.6084825,338.43457429)(607.55348255,338.4295743)(607.49348495,338.4295743)
\lineto(607.32848495,338.4295743)
\lineto(606.68348495,338.4295743)
\curveto(606.62348348,338.43957429)(606.55848355,338.44457428)(606.48848495,338.4445743)
\lineto(606.29348495,338.4445743)
\curveto(606.24348386,338.46457426)(606.19348391,338.47957425)(606.14348495,338.4895743)
\curveto(606.09348401,338.50957422)(606.05848405,338.54457418)(606.03848495,338.5945743)
\curveto(605.99848411,338.64457408)(605.97348413,338.71457401)(605.96348495,338.8045743)
\lineto(605.96348495,339.1045743)
\lineto(605.96348495,340.1245743)
\lineto(605.96348495,344.3545743)
\lineto(605.96348495,345.4645743)
\lineto(605.96348495,345.7495743)
\curveto(605.96348414,345.84956688)(605.98348412,345.9295668)(606.02348495,345.9895743)
\curveto(606.07348403,346.06956666)(606.14848396,346.11956661)(606.24848495,346.1395743)
\curveto(606.34848376,346.15956657)(606.46848364,346.16956656)(606.60848495,346.1695743)
\lineto(607.37348495,346.1695743)
\curveto(607.49348261,346.16956656)(607.59848251,346.15956657)(607.68848495,346.1395743)
\curveto(607.77848233,346.1295666)(607.84848226,346.08456664)(607.89848495,346.0045743)
\curveto(607.92848218,345.95456677)(607.94348216,345.88456684)(607.94348495,345.7945743)
\lineto(607.97348495,345.5245743)
\curveto(607.98348212,345.44456728)(607.99848211,345.36956736)(608.01848495,345.2995743)
\curveto(608.04848206,345.2295675)(608.09848201,345.19456753)(608.16848495,345.1945743)
\curveto(608.18848192,345.21456751)(608.2084819,345.2245675)(608.22848495,345.2245743)
\curveto(608.24848186,345.2245675)(608.26848184,345.23456749)(608.28848495,345.2545743)
\curveto(608.34848176,345.30456742)(608.39848171,345.35956737)(608.43848495,345.4195743)
\curveto(608.48848162,345.48956724)(608.54848156,345.54956718)(608.61848495,345.5995743)
\curveto(608.65848145,345.6295671)(608.69348141,345.65956707)(608.72348495,345.6895743)
\curveto(608.75348135,345.729567)(608.78848132,345.76456696)(608.82848495,345.7945743)
\lineto(609.09848495,345.9745743)
\curveto(609.19848091,346.03456669)(609.29848081,346.08956664)(609.39848495,346.1395743)
\curveto(609.49848061,346.17956655)(609.59848051,346.21456651)(609.69848495,346.2445743)
\lineto(610.02848495,346.3345743)
\curveto(610.05848005,346.34456638)(610.11347999,346.34456638)(610.19348495,346.3345743)
\curveto(610.28347982,346.33456639)(610.33847977,346.34456638)(610.35848495,346.3645743)
}
}
{
\newrgbcolor{curcolor}{0 0 0}
\pscustom[linestyle=none,fillstyle=solid,fillcolor=curcolor]
{
}
}
{
\newrgbcolor{curcolor}{0 0 0}
\pscustom[linestyle=none,fillstyle=solid,fillcolor=curcolor]
{
\newpath
\moveto(616.97371933,348.4795743)
\lineto(617.97871933,348.4795743)
\curveto(618.12871634,348.47956425)(618.25871621,348.46956426)(618.36871933,348.4495743)
\curveto(618.48871598,348.43956429)(618.5737159,348.37956435)(618.62371933,348.2695743)
\curveto(618.64371583,348.21956451)(618.65371582,348.15956457)(618.65371933,348.0895743)
\lineto(618.65371933,347.8795743)
\lineto(618.65371933,347.2045743)
\curveto(618.65371582,347.15456557)(618.64871582,347.09456563)(618.63871933,347.0245743)
\curveto(618.63871583,346.96456576)(618.64371583,346.90956582)(618.65371933,346.8595743)
\lineto(618.65371933,346.6945743)
\curveto(618.65371582,346.61456611)(618.65871581,346.53956619)(618.66871933,346.4695743)
\curveto(618.67871579,346.40956632)(618.70371577,346.35456637)(618.74371933,346.3045743)
\curveto(618.81371566,346.21456651)(618.93871553,346.16456656)(619.11871933,346.1545743)
\lineto(619.65871933,346.1545743)
\lineto(619.83871933,346.1545743)
\curveto(619.89871457,346.15456657)(619.95371452,346.14456658)(620.00371933,346.1245743)
\curveto(620.11371436,346.07456665)(620.1737143,345.98456674)(620.18371933,345.8545743)
\curveto(620.20371427,345.724567)(620.21371426,345.57956715)(620.21371933,345.4195743)
\lineto(620.21371933,345.2095743)
\curveto(620.22371425,345.13956759)(620.21871425,345.07956765)(620.19871933,345.0295743)
\curveto(620.14871432,344.86956786)(620.04371443,344.78456794)(619.88371933,344.7745743)
\curveto(619.72371475,344.76456796)(619.54371493,344.75956797)(619.34371933,344.7595743)
\lineto(619.20871933,344.7595743)
\curveto(619.1687153,344.76956796)(619.13371534,344.76956796)(619.10371933,344.7595743)
\curveto(619.06371541,344.74956798)(619.02871544,344.74456798)(618.99871933,344.7445743)
\curveto(618.9687155,344.75456797)(618.93871553,344.74956798)(618.90871933,344.7295743)
\curveto(618.82871564,344.70956802)(618.7687157,344.66456806)(618.72871933,344.5945743)
\curveto(618.69871577,344.53456819)(618.6737158,344.45956827)(618.65371933,344.3695743)
\curveto(618.64371583,344.31956841)(618.64371583,344.26456846)(618.65371933,344.2045743)
\curveto(618.66371581,344.14456858)(618.66371581,344.08956864)(618.65371933,344.0395743)
\lineto(618.65371933,343.1095743)
\lineto(618.65371933,341.3545743)
\curveto(618.65371582,341.10457162)(618.65871581,340.88457184)(618.66871933,340.6945743)
\curveto(618.68871578,340.51457221)(618.75371572,340.35457237)(618.86371933,340.2145743)
\curveto(618.91371556,340.15457257)(618.97871549,340.10957262)(619.05871933,340.0795743)
\lineto(619.32871933,340.0195743)
\curveto(619.35871511,340.00957272)(619.38871508,340.00457272)(619.41871933,340.0045743)
\curveto(619.45871501,340.01457271)(619.48871498,340.01457271)(619.50871933,340.0045743)
\lineto(619.67371933,340.0045743)
\curveto(619.78371469,340.00457272)(619.87871459,339.99957273)(619.95871933,339.9895743)
\curveto(620.03871443,339.97957275)(620.10371437,339.93957279)(620.15371933,339.8695743)
\curveto(620.19371428,339.80957292)(620.21371426,339.729573)(620.21371933,339.6295743)
\lineto(620.21371933,339.3445743)
\curveto(620.21371426,339.13457359)(620.20871426,338.93957379)(620.19871933,338.7595743)
\curveto(620.19871427,338.58957414)(620.11871435,338.47457425)(619.95871933,338.4145743)
\curveto(619.90871456,338.39457433)(619.86371461,338.38957434)(619.82371933,338.3995743)
\curveto(619.78371469,338.39957433)(619.73871473,338.38957434)(619.68871933,338.3695743)
\lineto(619.53871933,338.3695743)
\curveto(619.51871495,338.36957436)(619.48871498,338.37457435)(619.44871933,338.3845743)
\curveto(619.40871506,338.38457434)(619.3737151,338.37957435)(619.34371933,338.3695743)
\curveto(619.29371518,338.35957437)(619.23871523,338.35957437)(619.17871933,338.3695743)
\lineto(619.02871933,338.3695743)
\lineto(618.87871933,338.3695743)
\curveto(618.82871564,338.35957437)(618.78371569,338.35957437)(618.74371933,338.3695743)
\lineto(618.57871933,338.3695743)
\curveto(618.52871594,338.37957435)(618.473716,338.38457434)(618.41371933,338.3845743)
\curveto(618.35371612,338.38457434)(618.29871617,338.38957434)(618.24871933,338.3995743)
\curveto(618.17871629,338.40957432)(618.11371636,338.41957431)(618.05371933,338.4295743)
\lineto(617.87371933,338.4595743)
\curveto(617.76371671,338.48957424)(617.65871681,338.5245742)(617.55871933,338.5645743)
\curveto(617.45871701,338.60457412)(617.36371711,338.64957408)(617.27371933,338.6995743)
\lineto(617.18371933,338.7595743)
\curveto(617.15371732,338.78957394)(617.11871735,338.81957391)(617.07871933,338.8495743)
\curveto(617.05871741,338.86957386)(617.03371744,338.88957384)(617.00371933,338.9095743)
\lineto(616.92871933,338.9845743)
\curveto(616.78871768,339.17457355)(616.68371779,339.38457334)(616.61371933,339.6145743)
\curveto(616.59371788,339.65457307)(616.58371789,339.68957304)(616.58371933,339.7195743)
\curveto(616.59371788,339.75957297)(616.59371788,339.80457292)(616.58371933,339.8545743)
\curveto(616.5737179,339.87457285)(616.5687179,339.89957283)(616.56871933,339.9295743)
\curveto(616.5687179,339.95957277)(616.56371791,339.98457274)(616.55371933,340.0045743)
\lineto(616.55371933,340.1545743)
\curveto(616.54371793,340.19457253)(616.53871793,340.23957249)(616.53871933,340.2895743)
\curveto(616.54871792,340.33957239)(616.55371792,340.38957234)(616.55371933,340.4395743)
\lineto(616.55371933,341.0095743)
\lineto(616.55371933,343.2445743)
\lineto(616.55371933,344.0395743)
\lineto(616.55371933,344.2495743)
\curveto(616.56371791,344.31956841)(616.55871791,344.38456834)(616.53871933,344.4445743)
\curveto(616.49871797,344.58456814)(616.42871804,344.67456805)(616.32871933,344.7145743)
\curveto(616.21871825,344.76456796)(616.07871839,344.77956795)(615.90871933,344.7595743)
\curveto(615.73871873,344.73956799)(615.59371888,344.75456797)(615.47371933,344.8045743)
\curveto(615.39371908,344.83456789)(615.34371913,344.87956785)(615.32371933,344.9395743)
\curveto(615.30371917,344.99956773)(615.28371919,345.07456765)(615.26371933,345.1645743)
\lineto(615.26371933,345.4795743)
\curveto(615.26371921,345.65956707)(615.2737192,345.80456692)(615.29371933,345.9145743)
\curveto(615.31371916,346.0245667)(615.39871907,346.09956663)(615.54871933,346.1395743)
\curveto(615.58871888,346.15956657)(615.62871884,346.16456656)(615.66871933,346.1545743)
\lineto(615.80371933,346.1545743)
\curveto(615.95371852,346.15456657)(616.09371838,346.15956657)(616.22371933,346.1695743)
\curveto(616.35371812,346.18956654)(616.44371803,346.24956648)(616.49371933,346.3495743)
\curveto(616.52371795,346.41956631)(616.53871793,346.49956623)(616.53871933,346.5895743)
\curveto(616.54871792,346.67956605)(616.55371792,346.76956596)(616.55371933,346.8595743)
\lineto(616.55371933,347.7895743)
\lineto(616.55371933,348.0445743)
\curveto(616.55371792,348.13456459)(616.56371791,348.20956452)(616.58371933,348.2695743)
\curveto(616.63371784,348.36956436)(616.70871776,348.43456429)(616.80871933,348.4645743)
\curveto(616.82871764,348.47456425)(616.85371762,348.47456425)(616.88371933,348.4645743)
\curveto(616.92371755,348.46456426)(616.95371752,348.46956426)(616.97371933,348.4795743)
}
}
{
\newrgbcolor{curcolor}{0 0 0}
\pscustom[linestyle=none,fillstyle=solid,fillcolor=curcolor]
{
\newpath
\moveto(623.29715683,349.0195743)
\curveto(623.36715388,348.93956379)(623.40215384,348.81956391)(623.40215683,348.6595743)
\lineto(623.40215683,348.1945743)
\lineto(623.40215683,347.7895743)
\curveto(623.40215384,347.64956508)(623.36715388,347.55456517)(623.29715683,347.5045743)
\curveto(623.23715401,347.45456527)(623.15715409,347.4245653)(623.05715683,347.4145743)
\curveto(622.96715428,347.40456532)(622.86715438,347.39956533)(622.75715683,347.3995743)
\lineto(621.91715683,347.3995743)
\curveto(621.80715544,347.39956533)(621.70715554,347.40456532)(621.61715683,347.4145743)
\curveto(621.53715571,347.4245653)(621.46715578,347.45456527)(621.40715683,347.5045743)
\curveto(621.36715588,347.53456519)(621.33715591,347.58956514)(621.31715683,347.6695743)
\curveto(621.30715594,347.75956497)(621.29715595,347.85456487)(621.28715683,347.9545743)
\lineto(621.28715683,348.2845743)
\curveto(621.29715595,348.39456433)(621.30215594,348.48956424)(621.30215683,348.5695743)
\lineto(621.30215683,348.7795743)
\curveto(621.31215593,348.84956388)(621.33215591,348.90956382)(621.36215683,348.9595743)
\curveto(621.38215586,348.99956373)(621.40715584,349.0295637)(621.43715683,349.0495743)
\lineto(621.55715683,349.1095743)
\curveto(621.57715567,349.10956362)(621.60215564,349.10956362)(621.63215683,349.1095743)
\curveto(621.66215558,349.11956361)(621.68715556,349.1245636)(621.70715683,349.1245743)
\lineto(622.80215683,349.1245743)
\curveto(622.90215434,349.1245636)(622.99715425,349.11956361)(623.08715683,349.1095743)
\curveto(623.17715407,349.09956363)(623.247154,349.06956366)(623.29715683,349.0195743)
\moveto(623.40215683,339.2545743)
\curveto(623.40215384,339.05457367)(623.39715385,338.88457384)(623.38715683,338.7445743)
\curveto(623.37715387,338.60457412)(623.28715396,338.50957422)(623.11715683,338.4595743)
\curveto(623.05715419,338.43957429)(622.99215425,338.4295743)(622.92215683,338.4295743)
\curveto(622.85215439,338.43957429)(622.77715447,338.44457428)(622.69715683,338.4445743)
\lineto(621.85715683,338.4445743)
\curveto(621.76715548,338.44457428)(621.67715557,338.44957428)(621.58715683,338.4595743)
\curveto(621.50715574,338.46957426)(621.4471558,338.49957423)(621.40715683,338.5495743)
\curveto(621.3471559,338.61957411)(621.31215593,338.70457402)(621.30215683,338.8045743)
\lineto(621.30215683,339.1495743)
\lineto(621.30215683,345.4795743)
\lineto(621.30215683,345.7795743)
\curveto(621.30215594,345.87956685)(621.32215592,345.95956677)(621.36215683,346.0195743)
\curveto(621.42215582,346.08956664)(621.50715574,346.13456659)(621.61715683,346.1545743)
\curveto(621.63715561,346.16456656)(621.66215558,346.16456656)(621.69215683,346.1545743)
\curveto(621.73215551,346.15456657)(621.76215548,346.15956657)(621.78215683,346.1695743)
\lineto(622.53215683,346.1695743)
\lineto(622.72715683,346.1695743)
\curveto(622.80715444,346.17956655)(622.87215437,346.17956655)(622.92215683,346.1695743)
\lineto(623.04215683,346.1695743)
\curveto(623.10215414,346.14956658)(623.15715409,346.13456659)(623.20715683,346.1245743)
\curveto(623.25715399,346.11456661)(623.29715395,346.08456664)(623.32715683,346.0345743)
\curveto(623.36715388,345.98456674)(623.38715386,345.91456681)(623.38715683,345.8245743)
\curveto(623.39715385,345.73456699)(623.40215384,345.63956709)(623.40215683,345.5395743)
\lineto(623.40215683,339.2545743)
}
}
{
\newrgbcolor{curcolor}{0 0 0}
\pscustom[linestyle=none,fillstyle=solid,fillcolor=curcolor]
{
\newpath
\moveto(632.95434433,342.3895743)
\curveto(632.96433565,342.3295704)(632.96933564,342.23957049)(632.96934433,342.1195743)
\curveto(632.96933564,341.99957073)(632.95933565,341.91457081)(632.93934433,341.8645743)
\lineto(632.93934433,341.6695743)
\curveto(632.9093357,341.55957117)(632.88933572,341.45457127)(632.87934433,341.3545743)
\curveto(632.87933573,341.25457147)(632.86433575,341.15457157)(632.83434433,341.0545743)
\curveto(632.8143358,340.96457176)(632.79433582,340.86957186)(632.77434433,340.7695743)
\curveto(632.75433586,340.67957205)(632.72433589,340.58957214)(632.68434433,340.4995743)
\curveto(632.614336,340.3295724)(632.54433607,340.16957256)(632.47434433,340.0195743)
\curveto(632.40433621,339.87957285)(632.32433629,339.73957299)(632.23434433,339.5995743)
\curveto(632.17433644,339.50957322)(632.1093365,339.4245733)(632.03934433,339.3445743)
\curveto(631.97933663,339.27457345)(631.9093367,339.19957353)(631.82934433,339.1195743)
\lineto(631.72434433,339.0145743)
\curveto(631.67433694,338.96457376)(631.61933699,338.91957381)(631.55934433,338.8795743)
\lineto(631.40934433,338.7595743)
\curveto(631.32933728,338.69957403)(631.23933737,338.64457408)(631.13934433,338.5945743)
\curveto(631.04933756,338.55457417)(630.95433766,338.50957422)(630.85434433,338.4595743)
\curveto(630.75433786,338.40957432)(630.64933796,338.37457435)(630.53934433,338.3545743)
\curveto(630.43933817,338.33457439)(630.33433828,338.31457441)(630.22434433,338.2945743)
\curveto(630.16433845,338.27457445)(630.09933851,338.26457446)(630.02934433,338.2645743)
\curveto(629.96933864,338.26457446)(629.90433871,338.25457447)(629.83434433,338.2345743)
\lineto(629.69934433,338.2345743)
\curveto(629.61933899,338.21457451)(629.54433907,338.21457451)(629.47434433,338.2345743)
\lineto(629.32434433,338.2345743)
\curveto(629.26433935,338.25457447)(629.19933941,338.26457446)(629.12934433,338.2645743)
\curveto(629.06933954,338.25457447)(629.0093396,338.25957447)(628.94934433,338.2795743)
\curveto(628.78933982,338.3295744)(628.63433998,338.37457435)(628.48434433,338.4145743)
\curveto(628.34434027,338.45457427)(628.2143404,338.51457421)(628.09434433,338.5945743)
\curveto(628.02434059,338.63457409)(627.95934065,338.67457405)(627.89934433,338.7145743)
\curveto(627.83934077,338.76457396)(627.77434084,338.81457391)(627.70434433,338.8645743)
\lineto(627.52434433,338.9995743)
\curveto(627.44434117,339.05957367)(627.37434124,339.06457366)(627.31434433,339.0145743)
\curveto(627.26434135,338.98457374)(627.23934137,338.94457378)(627.23934433,338.8945743)
\curveto(627.23934137,338.85457387)(627.22934138,338.80457392)(627.20934433,338.7445743)
\curveto(627.18934142,338.64457408)(627.17934143,338.5295742)(627.17934433,338.3995743)
\curveto(627.18934142,338.26957446)(627.19434142,338.14957458)(627.19434433,338.0395743)
\lineto(627.19434433,336.5095743)
\curveto(627.19434142,336.37957635)(627.18934142,336.25457647)(627.17934433,336.1345743)
\curveto(627.17934143,336.00457672)(627.15434146,335.89957683)(627.10434433,335.8195743)
\curveto(627.07434154,335.77957695)(627.01934159,335.74957698)(626.93934433,335.7295743)
\curveto(626.85934175,335.70957702)(626.76934184,335.69957703)(626.66934433,335.6995743)
\curveto(626.56934204,335.68957704)(626.46934214,335.68957704)(626.36934433,335.6995743)
\lineto(626.11434433,335.6995743)
\lineto(625.70934433,335.6995743)
\lineto(625.60434433,335.6995743)
\curveto(625.56434305,335.69957703)(625.52934308,335.70457702)(625.49934433,335.7145743)
\lineto(625.37934433,335.7145743)
\curveto(625.2093434,335.76457696)(625.11934349,335.86457686)(625.10934433,336.0145743)
\curveto(625.09934351,336.15457657)(625.09434352,336.3245764)(625.09434433,336.5245743)
\lineto(625.09434433,345.3295743)
\curveto(625.09434352,345.43956729)(625.08934352,345.55456717)(625.07934433,345.6745743)
\curveto(625.07934353,345.80456692)(625.10434351,345.90456682)(625.15434433,345.9745743)
\curveto(625.19434342,346.04456668)(625.24934336,346.08956664)(625.31934433,346.1095743)
\curveto(625.36934324,346.1295666)(625.42934318,346.13956659)(625.49934433,346.1395743)
\lineto(625.72434433,346.1395743)
\lineto(626.44434433,346.1395743)
\lineto(626.72934433,346.1395743)
\curveto(626.81934179,346.13956659)(626.89434172,346.11456661)(626.95434433,346.0645743)
\curveto(627.02434159,346.01456671)(627.05934155,345.94956678)(627.05934433,345.8695743)
\curveto(627.06934154,345.79956693)(627.09434152,345.724567)(627.13434433,345.6445743)
\curveto(627.14434147,345.61456711)(627.15434146,345.58956714)(627.16434433,345.5695743)
\curveto(627.18434143,345.55956717)(627.20434141,345.54456718)(627.22434433,345.5245743)
\curveto(627.33434128,345.51456721)(627.42434119,345.54456718)(627.49434433,345.6145743)
\curveto(627.56434105,345.68456704)(627.63434098,345.74456698)(627.70434433,345.7945743)
\curveto(627.83434078,345.88456684)(627.96934064,345.96456676)(628.10934433,346.0345743)
\curveto(628.24934036,346.11456661)(628.40434021,346.17956655)(628.57434433,346.2295743)
\curveto(628.65433996,346.25956647)(628.73933987,346.27956645)(628.82934433,346.2895743)
\curveto(628.92933968,346.29956643)(629.02433959,346.31456641)(629.11434433,346.3345743)
\curveto(629.15433946,346.34456638)(629.19433942,346.34456638)(629.23434433,346.3345743)
\curveto(629.28433933,346.3245664)(629.32433929,346.3295664)(629.35434433,346.3495743)
\curveto(629.92433869,346.36956636)(630.40433821,346.28956644)(630.79434433,346.1095743)
\curveto(631.19433742,345.93956679)(631.53433708,345.71456701)(631.81434433,345.4345743)
\curveto(631.86433675,345.38456734)(631.9093367,345.33456739)(631.94934433,345.2845743)
\curveto(631.98933662,345.24456748)(632.02933658,345.19956753)(632.06934433,345.1495743)
\curveto(632.13933647,345.05956767)(632.19933641,344.96956776)(632.24934433,344.8795743)
\curveto(632.3093363,344.78956794)(632.36433625,344.69956803)(632.41434433,344.6095743)
\curveto(632.43433618,344.58956814)(632.44433617,344.56456816)(632.44434433,344.5345743)
\curveto(632.45433616,344.50456822)(632.46933614,344.46956826)(632.48934433,344.4295743)
\curveto(632.54933606,344.3295684)(632.60433601,344.20956852)(632.65434433,344.0695743)
\curveto(632.67433594,344.00956872)(632.69433592,343.94456878)(632.71434433,343.8745743)
\curveto(632.73433588,343.81456891)(632.75433586,343.74956898)(632.77434433,343.6795743)
\curveto(632.8143358,343.55956917)(632.83933577,343.43456929)(632.84934433,343.3045743)
\curveto(632.86933574,343.17456955)(632.89433572,343.03956969)(632.92434433,342.8995743)
\lineto(632.92434433,342.7345743)
\lineto(632.95434433,342.5545743)
\lineto(632.95434433,342.3895743)
\moveto(630.83934433,342.0445743)
\curveto(630.84933776,342.09457063)(630.85433776,342.15957057)(630.85434433,342.2395743)
\curveto(630.85433776,342.3295704)(630.84933776,342.39957033)(630.83934433,342.4495743)
\lineto(630.83934433,342.5845743)
\curveto(630.81933779,342.64457008)(630.8093378,342.70957002)(630.80934433,342.7795743)
\curveto(630.8093378,342.84956988)(630.79933781,342.91956981)(630.77934433,342.9895743)
\curveto(630.75933785,343.08956964)(630.73933787,343.18456954)(630.71934433,343.2745743)
\curveto(630.69933791,343.37456935)(630.66933794,343.46456926)(630.62934433,343.5445743)
\curveto(630.5093381,343.86456886)(630.35433826,344.11956861)(630.16434433,344.3095743)
\curveto(629.97433864,344.49956823)(629.70433891,344.63956809)(629.35434433,344.7295743)
\curveto(629.27433934,344.74956798)(629.18433943,344.75956797)(629.08434433,344.7595743)
\lineto(628.81434433,344.7595743)
\curveto(628.77433984,344.74956798)(628.73933987,344.74456798)(628.70934433,344.7445743)
\curveto(628.67933993,344.74456798)(628.64433997,344.73956799)(628.60434433,344.7295743)
\lineto(628.39434433,344.6695743)
\curveto(628.33434028,344.65956807)(628.27434034,344.63956809)(628.21434433,344.6095743)
\curveto(627.95434066,344.49956823)(627.74934086,344.3295684)(627.59934433,344.0995743)
\curveto(627.45934115,343.86956886)(627.34434127,343.61456911)(627.25434433,343.3345743)
\curveto(627.23434138,343.25456947)(627.21934139,343.16956956)(627.20934433,343.0795743)
\curveto(627.19934141,342.99956973)(627.18434143,342.91956981)(627.16434433,342.8395743)
\curveto(627.15434146,342.79956993)(627.14934146,342.73456999)(627.14934433,342.6445743)
\curveto(627.12934148,342.60457012)(627.12434149,342.55457017)(627.13434433,342.4945743)
\curveto(627.14434147,342.44457028)(627.14434147,342.39457033)(627.13434433,342.3445743)
\curveto(627.1143415,342.28457044)(627.1143415,342.2295705)(627.13434433,342.1795743)
\lineto(627.13434433,341.9995743)
\lineto(627.13434433,341.8645743)
\curveto(627.13434148,341.8245709)(627.14434147,341.78457094)(627.16434433,341.7445743)
\curveto(627.16434145,341.67457105)(627.16934144,341.61957111)(627.17934433,341.5795743)
\lineto(627.20934433,341.3995743)
\curveto(627.21934139,341.33957139)(627.23434138,341.27957145)(627.25434433,341.2195743)
\curveto(627.34434127,340.9295718)(627.44934116,340.68957204)(627.56934433,340.4995743)
\curveto(627.69934091,340.31957241)(627.87934073,340.15957257)(628.10934433,340.0195743)
\curveto(628.24934036,339.93957279)(628.4143402,339.87457285)(628.60434433,339.8245743)
\curveto(628.64433997,339.81457291)(628.67933993,339.80957292)(628.70934433,339.8095743)
\curveto(628.73933987,339.81957291)(628.77433984,339.81957291)(628.81434433,339.8095743)
\curveto(628.85433976,339.79957293)(628.9143397,339.78957294)(628.99434433,339.7795743)
\curveto(629.07433954,339.77957295)(629.13933947,339.78457294)(629.18934433,339.7945743)
\curveto(629.26933934,339.81457291)(629.34933926,339.8295729)(629.42934433,339.8395743)
\curveto(629.51933909,339.85957287)(629.60433901,339.88457284)(629.68434433,339.9145743)
\curveto(629.92433869,340.01457271)(630.11933849,340.15457257)(630.26934433,340.3345743)
\curveto(630.41933819,340.51457221)(630.54433807,340.724572)(630.64434433,340.9645743)
\curveto(630.69433792,341.08457164)(630.72933788,341.20957152)(630.74934433,341.3395743)
\curveto(630.76933784,341.46957126)(630.79433782,341.60457112)(630.82434433,341.7445743)
\lineto(630.82434433,341.8945743)
\curveto(630.83433778,341.94457078)(630.83933777,341.99457073)(630.83934433,342.0445743)
}
}
{
\newrgbcolor{curcolor}{0 0 0}
\pscustom[linestyle=none,fillstyle=solid,fillcolor=curcolor]
{
\newpath
\moveto(642.0042662,342.6145743)
\curveto(642.02425763,342.55457017)(642.03425762,342.46957026)(642.0342662,342.3595743)
\curveto(642.03425762,342.24957048)(642.02425763,342.16457056)(642.0042662,342.1045743)
\lineto(642.0042662,341.9545743)
\curveto(641.98425767,341.87457085)(641.97425768,341.79457093)(641.9742662,341.7145743)
\curveto(641.98425767,341.63457109)(641.97925768,341.55457117)(641.9592662,341.4745743)
\curveto(641.93925772,341.40457132)(641.92425773,341.33957139)(641.9142662,341.2795743)
\curveto(641.90425775,341.21957151)(641.89425776,341.15457157)(641.8842662,341.0845743)
\curveto(641.84425781,340.97457175)(641.80925785,340.85957187)(641.7792662,340.7395743)
\curveto(641.74925791,340.6295721)(641.70925795,340.5245722)(641.6592662,340.4245743)
\curveto(641.44925821,339.94457278)(641.17425848,339.55457317)(640.8342662,339.2545743)
\curveto(640.49425916,338.95457377)(640.08425957,338.70457402)(639.6042662,338.5045743)
\curveto(639.48426017,338.45457427)(639.3592603,338.41957431)(639.2292662,338.3995743)
\curveto(639.10926055,338.36957436)(638.98426067,338.33957439)(638.8542662,338.3095743)
\curveto(638.80426085,338.28957444)(638.74926091,338.27957445)(638.6892662,338.2795743)
\curveto(638.62926103,338.27957445)(638.57426108,338.27457445)(638.5242662,338.2645743)
\lineto(638.4192662,338.2645743)
\curveto(638.38926127,338.25457447)(638.3592613,338.24957448)(638.3292662,338.2495743)
\curveto(638.27926138,338.23957449)(638.19926146,338.23457449)(638.0892662,338.2345743)
\curveto(637.97926168,338.2245745)(637.89426176,338.2295745)(637.8342662,338.2495743)
\lineto(637.6842662,338.2495743)
\curveto(637.63426202,338.25957447)(637.57926208,338.26457446)(637.5192662,338.2645743)
\curveto(637.46926219,338.25457447)(637.41926224,338.25957447)(637.3692662,338.2795743)
\curveto(637.32926233,338.28957444)(637.28926237,338.29457443)(637.2492662,338.2945743)
\curveto(637.21926244,338.29457443)(637.17926248,338.29957443)(637.1292662,338.3095743)
\curveto(637.02926263,338.33957439)(636.92926273,338.36457436)(636.8292662,338.3845743)
\curveto(636.72926293,338.40457432)(636.63426302,338.43457429)(636.5442662,338.4745743)
\curveto(636.42426323,338.51457421)(636.30926335,338.55457417)(636.1992662,338.5945743)
\curveto(636.09926356,338.63457409)(635.99426366,338.68457404)(635.8842662,338.7445743)
\curveto(635.53426412,338.95457377)(635.23426442,339.19957353)(634.9842662,339.4795743)
\curveto(634.73426492,339.75957297)(634.52426513,340.09457263)(634.3542662,340.4845743)
\curveto(634.30426535,340.57457215)(634.26426539,340.66957206)(634.2342662,340.7695743)
\curveto(634.21426544,340.86957186)(634.18926547,340.97457175)(634.1592662,341.0845743)
\curveto(634.13926552,341.13457159)(634.12926553,341.17957155)(634.1292662,341.2195743)
\curveto(634.12926553,341.25957147)(634.11926554,341.30457142)(634.0992662,341.3545743)
\curveto(634.07926558,341.43457129)(634.06926559,341.51457121)(634.0692662,341.5945743)
\curveto(634.06926559,341.68457104)(634.0592656,341.76957096)(634.0392662,341.8495743)
\curveto(634.02926563,341.89957083)(634.02426563,341.94457078)(634.0242662,341.9845743)
\lineto(634.0242662,342.1195743)
\curveto(634.00426565,342.17957055)(633.99426566,342.26457046)(633.9942662,342.3745743)
\curveto(634.00426565,342.48457024)(634.01926564,342.56957016)(634.0392662,342.6295743)
\lineto(634.0392662,342.7345743)
\curveto(634.04926561,342.78456994)(634.04926561,342.83456989)(634.0392662,342.8845743)
\curveto(634.03926562,342.94456978)(634.04926561,342.99956973)(634.0692662,343.0495743)
\curveto(634.07926558,343.09956963)(634.08426557,343.14456958)(634.0842662,343.1845743)
\curveto(634.08426557,343.23456949)(634.09426556,343.28456944)(634.1142662,343.3345743)
\curveto(634.1542655,343.46456926)(634.18926547,343.58956914)(634.2192662,343.7095743)
\curveto(634.24926541,343.83956889)(634.28926537,343.96456876)(634.3392662,344.0845743)
\curveto(634.51926514,344.49456823)(634.73426492,344.83456789)(634.9842662,345.1045743)
\curveto(635.23426442,345.38456734)(635.53926412,345.63956709)(635.8992662,345.8695743)
\curveto(635.99926366,345.91956681)(636.10426355,345.96456676)(636.2142662,346.0045743)
\curveto(636.32426333,346.04456668)(636.43426322,346.08956664)(636.5442662,346.1395743)
\curveto(636.67426298,346.18956654)(636.80926285,346.2245665)(636.9492662,346.2445743)
\curveto(637.08926257,346.26456646)(637.23426242,346.29456643)(637.3842662,346.3345743)
\curveto(637.46426219,346.34456638)(637.53926212,346.34956638)(637.6092662,346.3495743)
\curveto(637.67926198,346.34956638)(637.74926191,346.35456637)(637.8192662,346.3645743)
\curveto(638.39926126,346.37456635)(638.89926076,346.31456641)(639.3192662,346.1845743)
\curveto(639.74925991,346.05456667)(640.12925953,345.87456685)(640.4592662,345.6445743)
\curveto(640.56925909,345.56456716)(640.67925898,345.47456725)(640.7892662,345.3745743)
\curveto(640.90925875,345.28456744)(641.00925865,345.18456754)(641.0892662,345.0745743)
\curveto(641.16925849,344.97456775)(641.23925842,344.87456785)(641.2992662,344.7745743)
\curveto(641.36925829,344.67456805)(641.43925822,344.56956816)(641.5092662,344.4595743)
\curveto(641.57925808,344.34956838)(641.63425802,344.2295685)(641.6742662,344.0995743)
\curveto(641.71425794,343.97956875)(641.7592579,343.84956888)(641.8092662,343.7095743)
\curveto(641.83925782,343.6295691)(641.86425779,343.54456918)(641.8842662,343.4545743)
\lineto(641.9442662,343.1845743)
\curveto(641.9542577,343.14456958)(641.9592577,343.10456962)(641.9592662,343.0645743)
\curveto(641.9592577,343.0245697)(641.96425769,342.98456974)(641.9742662,342.9445743)
\curveto(641.99425766,342.89456983)(641.99925766,342.83956989)(641.9892662,342.7795743)
\curveto(641.97925768,342.71957001)(641.98425767,342.66457006)(642.0042662,342.6145743)
\moveto(639.9042662,342.0745743)
\curveto(639.91425974,342.1245706)(639.91925974,342.19457053)(639.9192662,342.2845743)
\curveto(639.91925974,342.38457034)(639.91425974,342.45957027)(639.9042662,342.5095743)
\lineto(639.9042662,342.6295743)
\curveto(639.88425977,342.67957005)(639.87425978,342.73456999)(639.8742662,342.7945743)
\curveto(639.87425978,342.85456987)(639.86925979,342.90956982)(639.8592662,342.9595743)
\curveto(639.8592598,342.99956973)(639.8542598,343.0295697)(639.8442662,343.0495743)
\lineto(639.7842662,343.2895743)
\curveto(639.77425988,343.37956935)(639.7542599,343.46456926)(639.7242662,343.5445743)
\curveto(639.61426004,343.80456892)(639.48426017,344.0245687)(639.3342662,344.2045743)
\curveto(639.18426047,344.39456833)(638.98426067,344.54456818)(638.7342662,344.6545743)
\curveto(638.67426098,344.67456805)(638.61426104,344.68956804)(638.5542662,344.6995743)
\curveto(638.49426116,344.71956801)(638.42926123,344.73956799)(638.3592662,344.7595743)
\curveto(638.27926138,344.77956795)(638.19426146,344.78456794)(638.1042662,344.7745743)
\lineto(637.8342662,344.7745743)
\curveto(637.80426185,344.75456797)(637.76926189,344.74456798)(637.7292662,344.7445743)
\curveto(637.68926197,344.75456797)(637.654262,344.75456797)(637.6242662,344.7445743)
\lineto(637.4142662,344.6845743)
\curveto(637.3542623,344.67456805)(637.29926236,344.65456807)(637.2492662,344.6245743)
\curveto(636.99926266,344.51456821)(636.79426286,344.35456837)(636.6342662,344.1445743)
\curveto(636.48426317,343.94456878)(636.36426329,343.70956902)(636.2742662,343.4395743)
\curveto(636.24426341,343.33956939)(636.21926344,343.23456949)(636.1992662,343.1245743)
\curveto(636.18926347,343.01456971)(636.17426348,342.90456982)(636.1542662,342.7945743)
\curveto(636.14426351,342.74456998)(636.13926352,342.69457003)(636.1392662,342.6445743)
\lineto(636.1392662,342.4945743)
\curveto(636.11926354,342.4245703)(636.10926355,342.31957041)(636.1092662,342.1795743)
\curveto(636.11926354,342.03957069)(636.13426352,341.93457079)(636.1542662,341.8645743)
\lineto(636.1542662,341.7295743)
\curveto(636.17426348,341.64957108)(636.18926347,341.56957116)(636.1992662,341.4895743)
\curveto(636.20926345,341.41957131)(636.22426343,341.34457138)(636.2442662,341.2645743)
\curveto(636.34426331,340.96457176)(636.44926321,340.71957201)(636.5592662,340.5295743)
\curveto(636.67926298,340.34957238)(636.86426279,340.18457254)(637.1142662,340.0345743)
\curveto(637.18426247,339.98457274)(637.2592624,339.94457278)(637.3392662,339.9145743)
\curveto(637.42926223,339.88457284)(637.51926214,339.85957287)(637.6092662,339.8395743)
\curveto(637.64926201,339.8295729)(637.68426197,339.8245729)(637.7142662,339.8245743)
\curveto(637.74426191,339.83457289)(637.77926188,339.83457289)(637.8192662,339.8245743)
\lineto(637.9392662,339.7945743)
\curveto(637.98926167,339.79457293)(638.03426162,339.79957293)(638.0742662,339.8095743)
\lineto(638.1942662,339.8095743)
\curveto(638.27426138,339.8295729)(638.3542613,339.84457288)(638.4342662,339.8545743)
\curveto(638.51426114,339.86457286)(638.58926107,339.88457284)(638.6592662,339.9145743)
\curveto(638.91926074,340.01457271)(639.12926053,340.14957258)(639.2892662,340.3195743)
\curveto(639.44926021,340.48957224)(639.58426007,340.69957203)(639.6942662,340.9495743)
\curveto(639.73425992,341.04957168)(639.76425989,341.14957158)(639.7842662,341.2495743)
\curveto(639.80425985,341.34957138)(639.82925983,341.45457127)(639.8592662,341.5645743)
\curveto(639.86925979,341.60457112)(639.87425978,341.63957109)(639.8742662,341.6695743)
\curveto(639.87425978,341.70957102)(639.87925978,341.74957098)(639.8892662,341.7895743)
\lineto(639.8892662,341.9245743)
\curveto(639.88925977,341.97457075)(639.89425976,342.0245707)(639.9042662,342.0745743)
}
}
{
\newrgbcolor{curcolor}{0 0 0}
\pscustom[linestyle=none,fillstyle=solid,fillcolor=curcolor]
{
}
}
{
\newrgbcolor{curcolor}{0 0 0}
\pscustom[linestyle=none,fillstyle=solid,fillcolor=curcolor]
{
\newpath
\moveto(655.15434433,339.2845743)
\lineto(655.15434433,338.8645743)
\curveto(655.15433596,338.73457399)(655.12433599,338.6295741)(655.06434433,338.5495743)
\curveto(655.0143361,338.49957423)(654.94933616,338.46457426)(654.86934433,338.4445743)
\curveto(654.78933632,338.43457429)(654.69933641,338.4295743)(654.59934433,338.4295743)
\lineto(653.77434433,338.4295743)
\lineto(653.48934433,338.4295743)
\curveto(653.4093377,338.43957429)(653.34433777,338.46457426)(653.29434433,338.5045743)
\curveto(653.22433789,338.55457417)(653.18433793,338.61957411)(653.17434433,338.6995743)
\curveto(653.16433795,338.77957395)(653.14433797,338.85957387)(653.11434433,338.9395743)
\curveto(653.09433802,338.95957377)(653.07433804,338.97457375)(653.05434433,338.9845743)
\curveto(653.04433807,339.00457372)(653.02933808,339.0245737)(653.00934433,339.0445743)
\curveto(652.89933821,339.04457368)(652.81933829,339.01957371)(652.76934433,338.9695743)
\lineto(652.61934433,338.8195743)
\curveto(652.54933856,338.76957396)(652.48433863,338.724574)(652.42434433,338.6845743)
\curveto(652.36433875,338.65457407)(652.29933881,338.61457411)(652.22934433,338.5645743)
\curveto(652.18933892,338.54457418)(652.14433897,338.5245742)(652.09434433,338.5045743)
\curveto(652.05433906,338.48457424)(652.0093391,338.46457426)(651.95934433,338.4445743)
\curveto(651.81933929,338.39457433)(651.66933944,338.34957438)(651.50934433,338.3095743)
\curveto(651.45933965,338.28957444)(651.4143397,338.27957445)(651.37434433,338.2795743)
\curveto(651.33433978,338.27957445)(651.29433982,338.27457445)(651.25434433,338.2645743)
\lineto(651.11934433,338.2645743)
\curveto(651.08934002,338.25457447)(651.04934006,338.24957448)(650.99934433,338.2495743)
\lineto(650.86434433,338.2495743)
\curveto(650.80434031,338.2295745)(650.7143404,338.2245745)(650.59434433,338.2345743)
\curveto(650.47434064,338.23457449)(650.38934072,338.24457448)(650.33934433,338.2645743)
\curveto(650.26934084,338.28457444)(650.20434091,338.29457443)(650.14434433,338.2945743)
\curveto(650.09434102,338.28457444)(650.03934107,338.28957444)(649.97934433,338.3095743)
\lineto(649.61934433,338.4295743)
\curveto(649.5093416,338.45957427)(649.39934171,338.49957423)(649.28934433,338.5495743)
\curveto(648.93934217,338.69957403)(648.62434249,338.9295738)(648.34434433,339.2395743)
\curveto(648.07434304,339.55957317)(647.85934325,339.89457283)(647.69934433,340.2445743)
\curveto(647.64934346,340.35457237)(647.6093435,340.45957227)(647.57934433,340.5595743)
\curveto(647.54934356,340.66957206)(647.5143436,340.77957195)(647.47434433,340.8895743)
\curveto(647.46434365,340.9295718)(647.45934365,340.96457176)(647.45934433,340.9945743)
\curveto(647.45934365,341.03457169)(647.44934366,341.07957165)(647.42934433,341.1295743)
\curveto(647.4093437,341.20957152)(647.38934372,341.29457143)(647.36934433,341.3845743)
\curveto(647.35934375,341.48457124)(647.34434377,341.58457114)(647.32434433,341.6845743)
\curveto(647.3143438,341.71457101)(647.3093438,341.74957098)(647.30934433,341.7895743)
\curveto(647.31934379,341.8295709)(647.31934379,341.86457086)(647.30934433,341.8945743)
\lineto(647.30934433,342.0295743)
\curveto(647.3093438,342.07957065)(647.30434381,342.1295706)(647.29434433,342.1795743)
\curveto(647.28434383,342.2295705)(647.27934383,342.28457044)(647.27934433,342.3445743)
\curveto(647.27934383,342.41457031)(647.28434383,342.46957026)(647.29434433,342.5095743)
\curveto(647.30434381,342.55957017)(647.3093438,342.60457012)(647.30934433,342.6445743)
\lineto(647.30934433,342.7945743)
\curveto(647.31934379,342.84456988)(647.31934379,342.88956984)(647.30934433,342.9295743)
\curveto(647.3093438,342.97956975)(647.31934379,343.0295697)(647.33934433,343.0795743)
\curveto(647.35934375,343.18956954)(647.37434374,343.29456943)(647.38434433,343.3945743)
\curveto(647.40434371,343.49456923)(647.42934368,343.59456913)(647.45934433,343.6945743)
\curveto(647.49934361,343.81456891)(647.53434358,343.9295688)(647.56434433,344.0395743)
\curveto(647.59434352,344.14956858)(647.63434348,344.25956847)(647.68434433,344.3695743)
\curveto(647.82434329,344.66956806)(647.99934311,344.95456777)(648.20934433,345.2245743)
\curveto(648.22934288,345.25456747)(648.25434286,345.27956745)(648.28434433,345.2995743)
\curveto(648.32434279,345.3295674)(648.35434276,345.35956737)(648.37434433,345.3895743)
\curveto(648.4143427,345.43956729)(648.45434266,345.48456724)(648.49434433,345.5245743)
\curveto(648.53434258,345.56456716)(648.57934253,345.60456712)(648.62934433,345.6445743)
\curveto(648.66934244,345.66456706)(648.70434241,345.68956704)(648.73434433,345.7195743)
\curveto(648.76434235,345.75956697)(648.79934231,345.78956694)(648.83934433,345.8095743)
\curveto(649.08934202,345.97956675)(649.37934173,346.11956661)(649.70934433,346.2295743)
\curveto(649.77934133,346.24956648)(649.84934126,346.26456646)(649.91934433,346.2745743)
\curveto(649.99934111,346.28456644)(650.07934103,346.29956643)(650.15934433,346.3195743)
\curveto(650.22934088,346.33956639)(650.31934079,346.34956638)(650.42934433,346.3495743)
\curveto(650.53934057,346.35956637)(650.64934046,346.36456636)(650.75934433,346.3645743)
\curveto(650.86934024,346.36456636)(650.97434014,346.35956637)(651.07434433,346.3495743)
\curveto(651.18433993,346.33956639)(651.27433984,346.3245664)(651.34434433,346.3045743)
\curveto(651.49433962,346.25456647)(651.63933947,346.20956652)(651.77934433,346.1695743)
\curveto(651.91933919,346.1295666)(652.04933906,346.07456665)(652.16934433,346.0045743)
\curveto(652.23933887,345.95456677)(652.30433881,345.90456682)(652.36434433,345.8545743)
\curveto(652.42433869,345.81456691)(652.48933862,345.76956696)(652.55934433,345.7195743)
\curveto(652.59933851,345.68956704)(652.65433846,345.64956708)(652.72434433,345.5995743)
\curveto(652.80433831,345.54956718)(652.87933823,345.54956718)(652.94934433,345.5995743)
\curveto(652.98933812,345.61956711)(653.0093381,345.65456707)(653.00934433,345.7045743)
\curveto(653.0093381,345.75456697)(653.01933809,345.80456692)(653.03934433,345.8545743)
\lineto(653.03934433,346.0045743)
\curveto(653.04933806,346.03456669)(653.05433806,346.06956666)(653.05434433,346.1095743)
\lineto(653.05434433,346.2295743)
\lineto(653.05434433,348.2695743)
\curveto(653.05433806,348.37956435)(653.04933806,348.49956423)(653.03934433,348.6295743)
\curveto(653.03933807,348.76956396)(653.06433805,348.87456385)(653.11434433,348.9445743)
\curveto(653.15433796,349.0245637)(653.22933788,349.07456365)(653.33934433,349.0945743)
\curveto(653.35933775,349.10456362)(653.37933773,349.10456362)(653.39934433,349.0945743)
\curveto(653.41933769,349.09456363)(653.43933767,349.09956363)(653.45934433,349.1095743)
\lineto(654.52434433,349.1095743)
\curveto(654.64433647,349.10956362)(654.75433636,349.10456362)(654.85434433,349.0945743)
\curveto(654.95433616,349.08456364)(655.02933608,349.04456368)(655.07934433,348.9745743)
\curveto(655.12933598,348.89456383)(655.15433596,348.78956394)(655.15434433,348.6595743)
\lineto(655.15434433,348.2995743)
\lineto(655.15434433,339.2845743)
\moveto(653.11434433,342.2245743)
\curveto(653.12433799,342.26457046)(653.12433799,342.30457042)(653.11434433,342.3445743)
\lineto(653.11434433,342.4795743)
\curveto(653.114338,342.57957015)(653.109338,342.67957005)(653.09934433,342.7795743)
\curveto(653.08933802,342.87956985)(653.07433804,342.96956976)(653.05434433,343.0495743)
\curveto(653.03433808,343.15956957)(653.0143381,343.25956947)(652.99434433,343.3495743)
\curveto(652.98433813,343.43956929)(652.95933815,343.5245692)(652.91934433,343.6045743)
\curveto(652.77933833,343.96456876)(652.57433854,344.24956848)(652.30434433,344.4595743)
\curveto(652.04433907,344.66956806)(651.66433945,344.77456795)(651.16434433,344.7745743)
\curveto(651.10434001,344.77456795)(651.02434009,344.76456796)(650.92434433,344.7445743)
\curveto(650.84434027,344.724568)(650.76934034,344.70456802)(650.69934433,344.6845743)
\curveto(650.63934047,344.67456805)(650.57934053,344.65456807)(650.51934433,344.6245743)
\curveto(650.24934086,344.51456821)(650.03934107,344.34456838)(649.88934433,344.1145743)
\curveto(649.73934137,343.88456884)(649.61934149,343.6245691)(649.52934433,343.3345743)
\curveto(649.49934161,343.23456949)(649.47934163,343.13456959)(649.46934433,343.0345743)
\curveto(649.45934165,342.93456979)(649.43934167,342.8295699)(649.40934433,342.7195743)
\lineto(649.40934433,342.5095743)
\curveto(649.38934172,342.41957031)(649.38434173,342.29457043)(649.39434433,342.1345743)
\curveto(649.40434171,341.98457074)(649.41934169,341.87457085)(649.43934433,341.8045743)
\lineto(649.43934433,341.7145743)
\curveto(649.44934166,341.69457103)(649.45434166,341.67457105)(649.45434433,341.6545743)
\curveto(649.47434164,341.57457115)(649.48934162,341.49957123)(649.49934433,341.4295743)
\curveto(649.51934159,341.35957137)(649.53934157,341.28457144)(649.55934433,341.2045743)
\curveto(649.72934138,340.68457204)(650.01934109,340.29957243)(650.42934433,340.0495743)
\curveto(650.55934055,339.95957277)(650.73934037,339.88957284)(650.96934433,339.8395743)
\curveto(651.0093401,339.8295729)(651.06934004,339.8245729)(651.14934433,339.8245743)
\curveto(651.17933993,339.81457291)(651.22433989,339.80457292)(651.28434433,339.7945743)
\curveto(651.35433976,339.79457293)(651.4093397,339.79957293)(651.44934433,339.8095743)
\curveto(651.52933958,339.8295729)(651.6093395,339.84457288)(651.68934433,339.8545743)
\curveto(651.76933934,339.86457286)(651.84933926,339.88457284)(651.92934433,339.9145743)
\curveto(652.17933893,340.0245727)(652.37933873,340.16457256)(652.52934433,340.3345743)
\curveto(652.67933843,340.50457222)(652.8093383,340.71957201)(652.91934433,340.9795743)
\curveto(652.95933815,341.06957166)(652.98933812,341.15957157)(653.00934433,341.2495743)
\curveto(653.02933808,341.34957138)(653.04933806,341.45457127)(653.06934433,341.5645743)
\curveto(653.07933803,341.61457111)(653.07933803,341.65957107)(653.06934433,341.6995743)
\curveto(653.06933804,341.74957098)(653.07933803,341.79957093)(653.09934433,341.8495743)
\curveto(653.109338,341.87957085)(653.114338,341.91457081)(653.11434433,341.9545743)
\lineto(653.11434433,342.0895743)
\lineto(653.11434433,342.2245743)
}
}
{
\newrgbcolor{curcolor}{0 0 0}
\pscustom[linestyle=none,fillstyle=solid,fillcolor=curcolor]
{
\newpath
\moveto(664.0992662,342.3745743)
\curveto(664.11925804,342.29457043)(664.11925804,342.20457052)(664.0992662,342.1045743)
\curveto(664.07925808,342.00457072)(664.04425811,341.93957079)(663.9942662,341.9095743)
\curveto(663.94425821,341.86957086)(663.86925829,341.83957089)(663.7692662,341.8195743)
\curveto(663.67925848,341.80957092)(663.57425858,341.79957093)(663.4542662,341.7895743)
\lineto(663.1092662,341.7895743)
\curveto(662.99925916,341.79957093)(662.89925926,341.80457092)(662.8092662,341.8045743)
\lineto(659.1492662,341.8045743)
\lineto(658.9392662,341.8045743)
\curveto(658.87926328,341.80457092)(658.82426333,341.79457093)(658.7742662,341.7745743)
\curveto(658.69426346,341.73457099)(658.64426351,341.69457103)(658.6242662,341.6545743)
\curveto(658.60426355,341.63457109)(658.58426357,341.59457113)(658.5642662,341.5345743)
\curveto(658.54426361,341.48457124)(658.53926362,341.43457129)(658.5492662,341.3845743)
\curveto(658.56926359,341.3245714)(658.57926358,341.26457146)(658.5792662,341.2045743)
\curveto(658.58926357,341.15457157)(658.60426355,341.09957163)(658.6242662,341.0395743)
\curveto(658.70426345,340.79957193)(658.79926336,340.59957213)(658.9092662,340.4395743)
\curveto(659.02926313,340.28957244)(659.18926297,340.15457257)(659.3892662,340.0345743)
\curveto(659.46926269,339.98457274)(659.54926261,339.94957278)(659.6292662,339.9295743)
\curveto(659.71926244,339.91957281)(659.80926235,339.89957283)(659.8992662,339.8695743)
\curveto(659.97926218,339.84957288)(660.08926207,339.83457289)(660.2292662,339.8245743)
\curveto(660.36926179,339.81457291)(660.48926167,339.81957291)(660.5892662,339.8395743)
\lineto(660.7242662,339.8395743)
\curveto(660.82426133,339.85957287)(660.91426124,339.87957285)(660.9942662,339.8995743)
\curveto(661.08426107,339.9295728)(661.16926099,339.95957277)(661.2492662,339.9895743)
\curveto(661.34926081,340.03957269)(661.4592607,340.10457262)(661.5792662,340.1845743)
\curveto(661.70926045,340.26457246)(661.80426035,340.34457238)(661.8642662,340.4245743)
\curveto(661.91426024,340.49457223)(661.96426019,340.55957217)(662.0142662,340.6195743)
\curveto(662.07426008,340.68957204)(662.14426001,340.73957199)(662.2242662,340.7695743)
\curveto(662.32425983,340.81957191)(662.44925971,340.83957189)(662.5992662,340.8295743)
\lineto(663.0342662,340.8295743)
\lineto(663.2142662,340.8295743)
\curveto(663.28425887,340.83957189)(663.34425881,340.83457189)(663.3942662,340.8145743)
\lineto(663.5442662,340.8145743)
\curveto(663.64425851,340.79457193)(663.71425844,340.76957196)(663.7542662,340.7395743)
\curveto(663.79425836,340.71957201)(663.81425834,340.67457205)(663.8142662,340.6045743)
\curveto(663.82425833,340.53457219)(663.81925834,340.47457225)(663.7992662,340.4245743)
\curveto(663.74925841,340.28457244)(663.69425846,340.15957257)(663.6342662,340.0495743)
\curveto(663.57425858,339.93957279)(663.50425865,339.8295729)(663.4242662,339.7195743)
\curveto(663.20425895,339.38957334)(662.9542592,339.1245736)(662.6742662,338.9245743)
\curveto(662.39425976,338.724574)(662.04426011,338.55457417)(661.6242662,338.4145743)
\curveto(661.51426064,338.37457435)(661.40426075,338.34957438)(661.2942662,338.3395743)
\curveto(661.18426097,338.3295744)(661.06926109,338.30957442)(660.9492662,338.2795743)
\curveto(660.90926125,338.26957446)(660.86426129,338.26957446)(660.8142662,338.2795743)
\curveto(660.77426138,338.27957445)(660.73426142,338.27457445)(660.6942662,338.2645743)
\lineto(660.5292662,338.2645743)
\curveto(660.47926168,338.24457448)(660.41926174,338.23957449)(660.3492662,338.2495743)
\curveto(660.28926187,338.24957448)(660.23426192,338.25457447)(660.1842662,338.2645743)
\curveto(660.10426205,338.27457445)(660.03426212,338.27457445)(659.9742662,338.2645743)
\curveto(659.91426224,338.25457447)(659.84926231,338.25957447)(659.7792662,338.2795743)
\curveto(659.72926243,338.29957443)(659.67426248,338.30957442)(659.6142662,338.3095743)
\curveto(659.5542626,338.30957442)(659.49926266,338.31957441)(659.4492662,338.3395743)
\curveto(659.33926282,338.35957437)(659.22926293,338.38457434)(659.1192662,338.4145743)
\curveto(659.00926315,338.43457429)(658.90926325,338.46957426)(658.8192662,338.5195743)
\curveto(658.70926345,338.55957417)(658.60426355,338.59457413)(658.5042662,338.6245743)
\curveto(658.41426374,338.66457406)(658.32926383,338.70957402)(658.2492662,338.7595743)
\curveto(657.92926423,338.95957377)(657.64426451,339.18957354)(657.3942662,339.4495743)
\curveto(657.14426501,339.71957301)(656.93926522,340.0295727)(656.7792662,340.3795743)
\curveto(656.72926543,340.48957224)(656.68926547,340.59957213)(656.6592662,340.7095743)
\curveto(656.62926553,340.8295719)(656.58926557,340.94957178)(656.5392662,341.0695743)
\curveto(656.52926563,341.10957162)(656.52426563,341.14457158)(656.5242662,341.1745743)
\curveto(656.52426563,341.21457151)(656.51926564,341.25457147)(656.5092662,341.2945743)
\curveto(656.46926569,341.41457131)(656.44426571,341.54457118)(656.4342662,341.6845743)
\lineto(656.4042662,342.1045743)
\curveto(656.40426575,342.15457057)(656.39926576,342.20957052)(656.3892662,342.2695743)
\curveto(656.38926577,342.3295704)(656.39426576,342.38457034)(656.4042662,342.4345743)
\lineto(656.4042662,342.6145743)
\lineto(656.4492662,342.9745743)
\curveto(656.48926567,343.14456958)(656.52426563,343.30956942)(656.5542662,343.4695743)
\curveto(656.58426557,343.6295691)(656.62926553,343.77956895)(656.6892662,343.9195743)
\curveto(657.11926504,344.95956777)(657.84926431,345.69456703)(658.8792662,346.1245743)
\curveto(659.01926314,346.18456654)(659.159263,346.2245665)(659.2992662,346.2445743)
\curveto(659.44926271,346.27456645)(659.60426255,346.30956642)(659.7642662,346.3495743)
\curveto(659.84426231,346.35956637)(659.91926224,346.36456636)(659.9892662,346.3645743)
\curveto(660.0592621,346.36456636)(660.13426202,346.36956636)(660.2142662,346.3795743)
\curveto(660.72426143,346.38956634)(661.159261,346.3295664)(661.5192662,346.1995743)
\curveto(661.88926027,346.07956665)(662.21925994,345.91956681)(662.5092662,345.7195743)
\curveto(662.59925956,345.65956707)(662.68925947,345.58956714)(662.7792662,345.5095743)
\curveto(662.86925929,345.43956729)(662.94925921,345.36456736)(663.0192662,345.2845743)
\curveto(663.04925911,345.23456749)(663.08925907,345.19456753)(663.1392662,345.1645743)
\curveto(663.21925894,345.05456767)(663.29425886,344.93956779)(663.3642662,344.8195743)
\curveto(663.43425872,344.70956802)(663.50925865,344.59456813)(663.5892662,344.4745743)
\curveto(663.63925852,344.38456834)(663.67925848,344.28956844)(663.7092662,344.1895743)
\curveto(663.74925841,344.09956863)(663.78925837,343.99956873)(663.8292662,343.8895743)
\curveto(663.87925828,343.75956897)(663.91925824,343.6245691)(663.9492662,343.4845743)
\curveto(663.97925818,343.34456938)(664.01425814,343.20456952)(664.0542662,343.0645743)
\curveto(664.07425808,342.98456974)(664.07925808,342.89456983)(664.0692662,342.7945743)
\curveto(664.06925809,342.70457002)(664.07925808,342.61957011)(664.0992662,342.5395743)
\lineto(664.0992662,342.3745743)
\moveto(661.8492662,343.2595743)
\curveto(661.91926024,343.35956937)(661.92426023,343.47956925)(661.8642662,343.6195743)
\curveto(661.81426034,343.76956896)(661.77426038,343.87956885)(661.7442662,343.9495743)
\curveto(661.60426055,344.21956851)(661.41926074,344.4245683)(661.1892662,344.5645743)
\curveto(660.9592612,344.71456801)(660.63926152,344.79456793)(660.2292662,344.8045743)
\curveto(660.19926196,344.78456794)(660.16426199,344.77956795)(660.1242662,344.7895743)
\curveto(660.08426207,344.79956793)(660.04926211,344.79956793)(660.0192662,344.7895743)
\curveto(659.96926219,344.76956796)(659.91426224,344.75456797)(659.8542662,344.7445743)
\curveto(659.79426236,344.74456798)(659.73926242,344.73456799)(659.6892662,344.7145743)
\curveto(659.24926291,344.57456815)(658.92426323,344.29956843)(658.7142662,343.8895743)
\curveto(658.69426346,343.84956888)(658.66926349,343.79456893)(658.6392662,343.7245743)
\curveto(658.61926354,343.66456906)(658.60426355,343.59956913)(658.5942662,343.5295743)
\curveto(658.58426357,343.46956926)(658.58426357,343.40956932)(658.5942662,343.3495743)
\curveto(658.61426354,343.28956944)(658.64926351,343.23956949)(658.6992662,343.1995743)
\curveto(658.77926338,343.14956958)(658.88926327,343.1245696)(659.0292662,343.1245743)
\lineto(659.4342662,343.1245743)
\lineto(661.0992662,343.1245743)
\lineto(661.5342662,343.1245743)
\curveto(661.69426046,343.13456959)(661.79926036,343.17956955)(661.8492662,343.2595743)
}
}
{
\newrgbcolor{curcolor}{0 0 0}
\pscustom[linestyle=none,fillstyle=solid,fillcolor=curcolor]
{
}
}
{
\newrgbcolor{curcolor}{0 0 0}
\pscustom[linestyle=none,fillstyle=solid,fillcolor=curcolor]
{
\newpath
\moveto(676.8727037,342.3745743)
\curveto(676.89269554,342.29457043)(676.89269554,342.20457052)(676.8727037,342.1045743)
\curveto(676.85269558,342.00457072)(676.81769561,341.93957079)(676.7677037,341.9095743)
\curveto(676.71769571,341.86957086)(676.64269579,341.83957089)(676.5427037,341.8195743)
\curveto(676.45269598,341.80957092)(676.34769608,341.79957093)(676.2277037,341.7895743)
\lineto(675.8827037,341.7895743)
\curveto(675.77269666,341.79957093)(675.67269676,341.80457092)(675.5827037,341.8045743)
\lineto(671.9227037,341.8045743)
\lineto(671.7127037,341.8045743)
\curveto(671.65270078,341.80457092)(671.59770083,341.79457093)(671.5477037,341.7745743)
\curveto(671.46770096,341.73457099)(671.41770101,341.69457103)(671.3977037,341.6545743)
\curveto(671.37770105,341.63457109)(671.35770107,341.59457113)(671.3377037,341.5345743)
\curveto(671.31770111,341.48457124)(671.31270112,341.43457129)(671.3227037,341.3845743)
\curveto(671.34270109,341.3245714)(671.35270108,341.26457146)(671.3527037,341.2045743)
\curveto(671.36270107,341.15457157)(671.37770105,341.09957163)(671.3977037,341.0395743)
\curveto(671.47770095,340.79957193)(671.57270086,340.59957213)(671.6827037,340.4395743)
\curveto(671.80270063,340.28957244)(671.96270047,340.15457257)(672.1627037,340.0345743)
\curveto(672.24270019,339.98457274)(672.32270011,339.94957278)(672.4027037,339.9295743)
\curveto(672.49269994,339.91957281)(672.58269985,339.89957283)(672.6727037,339.8695743)
\curveto(672.75269968,339.84957288)(672.86269957,339.83457289)(673.0027037,339.8245743)
\curveto(673.14269929,339.81457291)(673.26269917,339.81957291)(673.3627037,339.8395743)
\lineto(673.4977037,339.8395743)
\curveto(673.59769883,339.85957287)(673.68769874,339.87957285)(673.7677037,339.8995743)
\curveto(673.85769857,339.9295728)(673.94269849,339.95957277)(674.0227037,339.9895743)
\curveto(674.12269831,340.03957269)(674.2326982,340.10457262)(674.3527037,340.1845743)
\curveto(674.48269795,340.26457246)(674.57769785,340.34457238)(674.6377037,340.4245743)
\curveto(674.68769774,340.49457223)(674.73769769,340.55957217)(674.7877037,340.6195743)
\curveto(674.84769758,340.68957204)(674.91769751,340.73957199)(674.9977037,340.7695743)
\curveto(675.09769733,340.81957191)(675.22269721,340.83957189)(675.3727037,340.8295743)
\lineto(675.8077037,340.8295743)
\lineto(675.9877037,340.8295743)
\curveto(676.05769637,340.83957189)(676.11769631,340.83457189)(676.1677037,340.8145743)
\lineto(676.3177037,340.8145743)
\curveto(676.41769601,340.79457193)(676.48769594,340.76957196)(676.5277037,340.7395743)
\curveto(676.56769586,340.71957201)(676.58769584,340.67457205)(676.5877037,340.6045743)
\curveto(676.59769583,340.53457219)(676.59269584,340.47457225)(676.5727037,340.4245743)
\curveto(676.52269591,340.28457244)(676.46769596,340.15957257)(676.4077037,340.0495743)
\curveto(676.34769608,339.93957279)(676.27769615,339.8295729)(676.1977037,339.7195743)
\curveto(675.97769645,339.38957334)(675.7276967,339.1245736)(675.4477037,338.9245743)
\curveto(675.16769726,338.724574)(674.81769761,338.55457417)(674.3977037,338.4145743)
\curveto(674.28769814,338.37457435)(674.17769825,338.34957438)(674.0677037,338.3395743)
\curveto(673.95769847,338.3295744)(673.84269859,338.30957442)(673.7227037,338.2795743)
\curveto(673.68269875,338.26957446)(673.63769879,338.26957446)(673.5877037,338.2795743)
\curveto(673.54769888,338.27957445)(673.50769892,338.27457445)(673.4677037,338.2645743)
\lineto(673.3027037,338.2645743)
\curveto(673.25269918,338.24457448)(673.19269924,338.23957449)(673.1227037,338.2495743)
\curveto(673.06269937,338.24957448)(673.00769942,338.25457447)(672.9577037,338.2645743)
\curveto(672.87769955,338.27457445)(672.80769962,338.27457445)(672.7477037,338.2645743)
\curveto(672.68769974,338.25457447)(672.62269981,338.25957447)(672.5527037,338.2795743)
\curveto(672.50269993,338.29957443)(672.44769998,338.30957442)(672.3877037,338.3095743)
\curveto(672.3277001,338.30957442)(672.27270016,338.31957441)(672.2227037,338.3395743)
\curveto(672.11270032,338.35957437)(672.00270043,338.38457434)(671.8927037,338.4145743)
\curveto(671.78270065,338.43457429)(671.68270075,338.46957426)(671.5927037,338.5195743)
\curveto(671.48270095,338.55957417)(671.37770105,338.59457413)(671.2777037,338.6245743)
\curveto(671.18770124,338.66457406)(671.10270133,338.70957402)(671.0227037,338.7595743)
\curveto(670.70270173,338.95957377)(670.41770201,339.18957354)(670.1677037,339.4495743)
\curveto(669.91770251,339.71957301)(669.71270272,340.0295727)(669.5527037,340.3795743)
\curveto(669.50270293,340.48957224)(669.46270297,340.59957213)(669.4327037,340.7095743)
\curveto(669.40270303,340.8295719)(669.36270307,340.94957178)(669.3127037,341.0695743)
\curveto(669.30270313,341.10957162)(669.29770313,341.14457158)(669.2977037,341.1745743)
\curveto(669.29770313,341.21457151)(669.29270314,341.25457147)(669.2827037,341.2945743)
\curveto(669.24270319,341.41457131)(669.21770321,341.54457118)(669.2077037,341.6845743)
\lineto(669.1777037,342.1045743)
\curveto(669.17770325,342.15457057)(669.17270326,342.20957052)(669.1627037,342.2695743)
\curveto(669.16270327,342.3295704)(669.16770326,342.38457034)(669.1777037,342.4345743)
\lineto(669.1777037,342.6145743)
\lineto(669.2227037,342.9745743)
\curveto(669.26270317,343.14456958)(669.29770313,343.30956942)(669.3277037,343.4695743)
\curveto(669.35770307,343.6295691)(669.40270303,343.77956895)(669.4627037,343.9195743)
\curveto(669.89270254,344.95956777)(670.62270181,345.69456703)(671.6527037,346.1245743)
\curveto(671.79270064,346.18456654)(671.9327005,346.2245665)(672.0727037,346.2445743)
\curveto(672.22270021,346.27456645)(672.37770005,346.30956642)(672.5377037,346.3495743)
\curveto(672.61769981,346.35956637)(672.69269974,346.36456636)(672.7627037,346.3645743)
\curveto(672.8326996,346.36456636)(672.90769952,346.36956636)(672.9877037,346.3795743)
\curveto(673.49769893,346.38956634)(673.9326985,346.3295664)(674.2927037,346.1995743)
\curveto(674.66269777,346.07956665)(674.99269744,345.91956681)(675.2827037,345.7195743)
\curveto(675.37269706,345.65956707)(675.46269697,345.58956714)(675.5527037,345.5095743)
\curveto(675.64269679,345.43956729)(675.72269671,345.36456736)(675.7927037,345.2845743)
\curveto(675.82269661,345.23456749)(675.86269657,345.19456753)(675.9127037,345.1645743)
\curveto(675.99269644,345.05456767)(676.06769636,344.93956779)(676.1377037,344.8195743)
\curveto(676.20769622,344.70956802)(676.28269615,344.59456813)(676.3627037,344.4745743)
\curveto(676.41269602,344.38456834)(676.45269598,344.28956844)(676.4827037,344.1895743)
\curveto(676.52269591,344.09956863)(676.56269587,343.99956873)(676.6027037,343.8895743)
\curveto(676.65269578,343.75956897)(676.69269574,343.6245691)(676.7227037,343.4845743)
\curveto(676.75269568,343.34456938)(676.78769564,343.20456952)(676.8277037,343.0645743)
\curveto(676.84769558,342.98456974)(676.85269558,342.89456983)(676.8427037,342.7945743)
\curveto(676.84269559,342.70457002)(676.85269558,342.61957011)(676.8727037,342.5395743)
\lineto(676.8727037,342.3745743)
\moveto(674.6227037,343.2595743)
\curveto(674.69269774,343.35956937)(674.69769773,343.47956925)(674.6377037,343.6195743)
\curveto(674.58769784,343.76956896)(674.54769788,343.87956885)(674.5177037,343.9495743)
\curveto(674.37769805,344.21956851)(674.19269824,344.4245683)(673.9627037,344.5645743)
\curveto(673.7326987,344.71456801)(673.41269902,344.79456793)(673.0027037,344.8045743)
\curveto(672.97269946,344.78456794)(672.93769949,344.77956795)(672.8977037,344.7895743)
\curveto(672.85769957,344.79956793)(672.82269961,344.79956793)(672.7927037,344.7895743)
\curveto(672.74269969,344.76956796)(672.68769974,344.75456797)(672.6277037,344.7445743)
\curveto(672.56769986,344.74456798)(672.51269992,344.73456799)(672.4627037,344.7145743)
\curveto(672.02270041,344.57456815)(671.69770073,344.29956843)(671.4877037,343.8895743)
\curveto(671.46770096,343.84956888)(671.44270099,343.79456893)(671.4127037,343.7245743)
\curveto(671.39270104,343.66456906)(671.37770105,343.59956913)(671.3677037,343.5295743)
\curveto(671.35770107,343.46956926)(671.35770107,343.40956932)(671.3677037,343.3495743)
\curveto(671.38770104,343.28956944)(671.42270101,343.23956949)(671.4727037,343.1995743)
\curveto(671.55270088,343.14956958)(671.66270077,343.1245696)(671.8027037,343.1245743)
\lineto(672.2077037,343.1245743)
\lineto(673.8727037,343.1245743)
\lineto(674.3077037,343.1245743)
\curveto(674.46769796,343.13456959)(674.57269786,343.17956955)(674.6227037,343.2595743)
}
}
{
\newrgbcolor{curcolor}{0 0 0}
\pscustom[linestyle=none,fillstyle=solid,fillcolor=curcolor]
{
\newpath
\moveto(681.09098495,346.3795743)
\curveto(681.84098045,346.39956633)(682.4909798,346.31456641)(683.04098495,346.1245743)
\curveto(683.60097869,345.94456678)(684.02597827,345.6295671)(684.31598495,345.1795743)
\curveto(684.38597791,345.06956766)(684.44597785,344.95456777)(684.49598495,344.8345743)
\curveto(684.55597774,344.724568)(684.60597769,344.59956813)(684.64598495,344.4595743)
\curveto(684.66597763,344.39956833)(684.67597762,344.33456839)(684.67598495,344.2645743)
\curveto(684.67597762,344.19456853)(684.66597763,344.13456859)(684.64598495,344.0845743)
\curveto(684.60597769,344.0245687)(684.55097774,343.98456874)(684.48098495,343.9645743)
\curveto(684.43097786,343.94456878)(684.37097792,343.93456879)(684.30098495,343.9345743)
\lineto(684.09098495,343.9345743)
\lineto(683.43098495,343.9345743)
\curveto(683.36097893,343.93456879)(683.290979,343.9295688)(683.22098495,343.9195743)
\curveto(683.15097914,343.91956881)(683.08597921,343.9295688)(683.02598495,343.9495743)
\curveto(682.92597937,343.96956876)(682.85097944,344.00956872)(682.80098495,344.0695743)
\curveto(682.75097954,344.1295686)(682.70597959,344.18956854)(682.66598495,344.2495743)
\lineto(682.54598495,344.4595743)
\curveto(682.51597978,344.53956819)(682.46597983,344.60456812)(682.39598495,344.6545743)
\curveto(682.29598,344.73456799)(682.1959801,344.79456793)(682.09598495,344.8345743)
\curveto(682.00598029,344.87456785)(681.8909804,344.90956782)(681.75098495,344.9395743)
\curveto(681.68098061,344.95956777)(681.57598072,344.97456775)(681.43598495,344.9845743)
\curveto(681.30598099,344.99456773)(681.20598109,344.98956774)(681.13598495,344.9695743)
\lineto(681.03098495,344.9695743)
\lineto(680.88098495,344.9395743)
\curveto(680.84098145,344.93956779)(680.7959815,344.93456779)(680.74598495,344.9245743)
\curveto(680.57598172,344.87456785)(680.43598186,344.80456792)(680.32598495,344.7145743)
\curveto(680.22598207,344.63456809)(680.15598214,344.50956822)(680.11598495,344.3395743)
\curveto(680.0959822,344.26956846)(680.0959822,344.20456852)(680.11598495,344.1445743)
\curveto(680.13598216,344.08456864)(680.15598214,344.03456869)(680.17598495,343.9945743)
\curveto(680.24598205,343.87456885)(680.32598197,343.77956895)(680.41598495,343.7095743)
\curveto(680.51598178,343.63956909)(680.63098166,343.57956915)(680.76098495,343.5295743)
\curveto(680.95098134,343.44956928)(681.15598114,343.37956935)(681.37598495,343.3195743)
\lineto(682.06598495,343.1695743)
\curveto(682.30597999,343.1295696)(682.53597976,343.07956965)(682.75598495,343.0195743)
\curveto(682.98597931,342.96956976)(683.20097909,342.90456982)(683.40098495,342.8245743)
\curveto(683.4909788,342.78456994)(683.57597872,342.74956998)(683.65598495,342.7195743)
\curveto(683.74597855,342.69957003)(683.83097846,342.66457006)(683.91098495,342.6145743)
\curveto(684.10097819,342.49457023)(684.27097802,342.36457036)(684.42098495,342.2245743)
\curveto(684.58097771,342.08457064)(684.70597759,341.90957082)(684.79598495,341.6995743)
\curveto(684.82597747,341.6295711)(684.85097744,341.55957117)(684.87098495,341.4895743)
\curveto(684.8909774,341.41957131)(684.91097738,341.34457138)(684.93098495,341.2645743)
\curveto(684.94097735,341.20457152)(684.94597735,341.10957162)(684.94598495,340.9795743)
\curveto(684.95597734,340.85957187)(684.95597734,340.76457196)(684.94598495,340.6945743)
\lineto(684.94598495,340.6195743)
\curveto(684.92597737,340.55957217)(684.91097738,340.49957223)(684.90098495,340.4395743)
\curveto(684.90097739,340.38957234)(684.8959774,340.33957239)(684.88598495,340.2895743)
\curveto(684.81597748,339.98957274)(684.70597759,339.724573)(684.55598495,339.4945743)
\curveto(684.3959779,339.25457347)(684.20097809,339.05957367)(683.97098495,338.9095743)
\curveto(683.74097855,338.75957397)(683.48097881,338.6295741)(683.19098495,338.5195743)
\curveto(683.08097921,338.46957426)(682.96097933,338.43457429)(682.83098495,338.4145743)
\curveto(682.71097958,338.39457433)(682.5909797,338.36957436)(682.47098495,338.3395743)
\curveto(682.38097991,338.31957441)(682.28598001,338.30957442)(682.18598495,338.3095743)
\curveto(682.0959802,338.29957443)(682.00598029,338.28457444)(681.91598495,338.2645743)
\lineto(681.64598495,338.2645743)
\curveto(681.58598071,338.24457448)(681.48098081,338.23457449)(681.33098495,338.2345743)
\curveto(681.1909811,338.23457449)(681.0909812,338.24457448)(681.03098495,338.2645743)
\curveto(681.00098129,338.26457446)(680.96598133,338.26957446)(680.92598495,338.2795743)
\lineto(680.82098495,338.2795743)
\curveto(680.70098159,338.29957443)(680.58098171,338.31457441)(680.46098495,338.3245743)
\curveto(680.34098195,338.33457439)(680.22598207,338.35457437)(680.11598495,338.3845743)
\curveto(679.72598257,338.49457423)(679.38098291,338.61957411)(679.08098495,338.7595743)
\curveto(678.78098351,338.90957382)(678.52598377,339.1295736)(678.31598495,339.4195743)
\curveto(678.17598412,339.60957312)(678.05598424,339.8295729)(677.95598495,340.0795743)
\curveto(677.93598436,340.13957259)(677.91598438,340.21957251)(677.89598495,340.3195743)
\curveto(677.87598442,340.36957236)(677.86098443,340.43957229)(677.85098495,340.5295743)
\curveto(677.84098445,340.61957211)(677.84598445,340.69457203)(677.86598495,340.7545743)
\curveto(677.8959844,340.8245719)(677.94598435,340.87457185)(678.01598495,340.9045743)
\curveto(678.06598423,340.9245718)(678.12598417,340.93457179)(678.19598495,340.9345743)
\lineto(678.42098495,340.9345743)
\lineto(679.12598495,340.9345743)
\lineto(679.36598495,340.9345743)
\curveto(679.44598285,340.93457179)(679.51598278,340.9245718)(679.57598495,340.9045743)
\curveto(679.68598261,340.86457186)(679.75598254,340.79957193)(679.78598495,340.7095743)
\curveto(679.82598247,340.61957211)(679.87098242,340.5245722)(679.92098495,340.4245743)
\curveto(679.94098235,340.37457235)(679.97598232,340.30957242)(680.02598495,340.2295743)
\curveto(680.08598221,340.14957258)(680.13598216,340.09957263)(680.17598495,340.0795743)
\curveto(680.295982,339.97957275)(680.41098188,339.89957283)(680.52098495,339.8395743)
\curveto(680.63098166,339.78957294)(680.77098152,339.73957299)(680.94098495,339.6895743)
\curveto(680.9909813,339.66957306)(681.04098125,339.65957307)(681.09098495,339.6595743)
\curveto(681.14098115,339.66957306)(681.1909811,339.66957306)(681.24098495,339.6595743)
\curveto(681.32098097,339.63957309)(681.40598089,339.6295731)(681.49598495,339.6295743)
\curveto(681.5959807,339.63957309)(681.68098061,339.65457307)(681.75098495,339.6745743)
\curveto(681.80098049,339.68457304)(681.84598045,339.68957304)(681.88598495,339.6895743)
\curveto(681.93598036,339.68957304)(681.98598031,339.69957303)(682.03598495,339.7195743)
\curveto(682.17598012,339.76957296)(682.30097999,339.8295729)(682.41098495,339.8995743)
\curveto(682.53097976,339.96957276)(682.62597967,340.05957267)(682.69598495,340.1695743)
\curveto(682.74597955,340.24957248)(682.78597951,340.37457235)(682.81598495,340.5445743)
\curveto(682.83597946,340.61457211)(682.83597946,340.67957205)(682.81598495,340.7395743)
\curveto(682.7959795,340.79957193)(682.77597952,340.84957188)(682.75598495,340.8895743)
\curveto(682.68597961,341.0295717)(682.5959797,341.13457159)(682.48598495,341.2045743)
\curveto(682.38597991,341.27457145)(682.26598003,341.33957139)(682.12598495,341.3995743)
\curveto(681.93598036,341.47957125)(681.73598056,341.54457118)(681.52598495,341.5945743)
\curveto(681.31598098,341.64457108)(681.10598119,341.69957103)(680.89598495,341.7595743)
\curveto(680.81598148,341.77957095)(680.73098156,341.79457093)(680.64098495,341.8045743)
\curveto(680.56098173,341.81457091)(680.48098181,341.8295709)(680.40098495,341.8495743)
\curveto(680.08098221,341.93957079)(679.77598252,342.0245707)(679.48598495,342.1045743)
\curveto(679.1959831,342.19457053)(678.93098336,342.3245704)(678.69098495,342.4945743)
\curveto(678.41098388,342.69457003)(678.20598409,342.96456976)(678.07598495,343.3045743)
\curveto(678.05598424,343.37456935)(678.03598426,343.46956926)(678.01598495,343.5895743)
\curveto(677.9959843,343.65956907)(677.98098431,343.74456898)(677.97098495,343.8445743)
\curveto(677.96098433,343.94456878)(677.96598433,344.03456869)(677.98598495,344.1145743)
\curveto(678.00598429,344.16456856)(678.01098428,344.20456852)(678.00098495,344.2345743)
\curveto(677.9909843,344.27456845)(677.9959843,344.31956841)(678.01598495,344.3695743)
\curveto(678.03598426,344.47956825)(678.05598424,344.57956815)(678.07598495,344.6695743)
\curveto(678.10598419,344.76956796)(678.14098415,344.86456786)(678.18098495,344.9545743)
\curveto(678.31098398,345.24456748)(678.4909838,345.47956725)(678.72098495,345.6595743)
\curveto(678.95098334,345.83956689)(679.21098308,345.98456674)(679.50098495,346.0945743)
\curveto(679.61098268,346.14456658)(679.72598257,346.17956655)(679.84598495,346.1995743)
\curveto(679.96598233,346.2295665)(680.0909822,346.25956647)(680.22098495,346.2895743)
\curveto(680.28098201,346.30956642)(680.34098195,346.31956641)(680.40098495,346.3195743)
\lineto(680.58098495,346.3495743)
\curveto(680.66098163,346.35956637)(680.74598155,346.36456636)(680.83598495,346.3645743)
\curveto(680.92598137,346.36456636)(681.01098128,346.36956636)(681.09098495,346.3795743)
}
}
{
\newrgbcolor{curcolor}{0 0 0}
\pscustom[linestyle=none,fillstyle=solid,fillcolor=curcolor]
{
\newpath
\moveto(694.06762558,342.3895743)
\curveto(694.0776169,342.3295704)(694.08261689,342.23957049)(694.08262558,342.1195743)
\curveto(694.08261689,341.99957073)(694.0726169,341.91457081)(694.05262558,341.8645743)
\lineto(694.05262558,341.6695743)
\curveto(694.02261695,341.55957117)(694.00261697,341.45457127)(693.99262558,341.3545743)
\curveto(693.99261698,341.25457147)(693.977617,341.15457157)(693.94762558,341.0545743)
\curveto(693.92761705,340.96457176)(693.90761707,340.86957186)(693.88762558,340.7695743)
\curveto(693.86761711,340.67957205)(693.83761714,340.58957214)(693.79762558,340.4995743)
\curveto(693.72761725,340.3295724)(693.65761732,340.16957256)(693.58762558,340.0195743)
\curveto(693.51761746,339.87957285)(693.43761754,339.73957299)(693.34762558,339.5995743)
\curveto(693.28761769,339.50957322)(693.22261775,339.4245733)(693.15262558,339.3445743)
\curveto(693.09261788,339.27457345)(693.02261795,339.19957353)(692.94262558,339.1195743)
\lineto(692.83762558,339.0145743)
\curveto(692.78761819,338.96457376)(692.73261824,338.91957381)(692.67262558,338.8795743)
\lineto(692.52262558,338.7595743)
\curveto(692.44261853,338.69957403)(692.35261862,338.64457408)(692.25262558,338.5945743)
\curveto(692.16261881,338.55457417)(692.06761891,338.50957422)(691.96762558,338.4595743)
\curveto(691.86761911,338.40957432)(691.76261921,338.37457435)(691.65262558,338.3545743)
\curveto(691.55261942,338.33457439)(691.44761953,338.31457441)(691.33762558,338.2945743)
\curveto(691.2776197,338.27457445)(691.21261976,338.26457446)(691.14262558,338.2645743)
\curveto(691.08261989,338.26457446)(691.01761996,338.25457447)(690.94762558,338.2345743)
\lineto(690.81262558,338.2345743)
\curveto(690.73262024,338.21457451)(690.65762032,338.21457451)(690.58762558,338.2345743)
\lineto(690.43762558,338.2345743)
\curveto(690.3776206,338.25457447)(690.31262066,338.26457446)(690.24262558,338.2645743)
\curveto(690.18262079,338.25457447)(690.12262085,338.25957447)(690.06262558,338.2795743)
\curveto(689.90262107,338.3295744)(689.74762123,338.37457435)(689.59762558,338.4145743)
\curveto(689.45762152,338.45457427)(689.32762165,338.51457421)(689.20762558,338.5945743)
\curveto(689.13762184,338.63457409)(689.0726219,338.67457405)(689.01262558,338.7145743)
\curveto(688.95262202,338.76457396)(688.88762209,338.81457391)(688.81762558,338.8645743)
\lineto(688.63762558,338.9995743)
\curveto(688.55762242,339.05957367)(688.48762249,339.06457366)(688.42762558,339.0145743)
\curveto(688.3776226,338.98457374)(688.35262262,338.94457378)(688.35262558,338.8945743)
\curveto(688.35262262,338.85457387)(688.34262263,338.80457392)(688.32262558,338.7445743)
\curveto(688.30262267,338.64457408)(688.29262268,338.5295742)(688.29262558,338.3995743)
\curveto(688.30262267,338.26957446)(688.30762267,338.14957458)(688.30762558,338.0395743)
\lineto(688.30762558,336.5095743)
\curveto(688.30762267,336.37957635)(688.30262267,336.25457647)(688.29262558,336.1345743)
\curveto(688.29262268,336.00457672)(688.26762271,335.89957683)(688.21762558,335.8195743)
\curveto(688.18762279,335.77957695)(688.13262284,335.74957698)(688.05262558,335.7295743)
\curveto(687.972623,335.70957702)(687.88262309,335.69957703)(687.78262558,335.6995743)
\curveto(687.68262329,335.68957704)(687.58262339,335.68957704)(687.48262558,335.6995743)
\lineto(687.22762558,335.6995743)
\lineto(686.82262558,335.6995743)
\lineto(686.71762558,335.6995743)
\curveto(686.6776243,335.69957703)(686.64262433,335.70457702)(686.61262558,335.7145743)
\lineto(686.49262558,335.7145743)
\curveto(686.32262465,335.76457696)(686.23262474,335.86457686)(686.22262558,336.0145743)
\curveto(686.21262476,336.15457657)(686.20762477,336.3245764)(686.20762558,336.5245743)
\lineto(686.20762558,345.3295743)
\curveto(686.20762477,345.43956729)(686.20262477,345.55456717)(686.19262558,345.6745743)
\curveto(686.19262478,345.80456692)(686.21762476,345.90456682)(686.26762558,345.9745743)
\curveto(686.30762467,346.04456668)(686.36262461,346.08956664)(686.43262558,346.1095743)
\curveto(686.48262449,346.1295666)(686.54262443,346.13956659)(686.61262558,346.1395743)
\lineto(686.83762558,346.1395743)
\lineto(687.55762558,346.1395743)
\lineto(687.84262558,346.1395743)
\curveto(687.93262304,346.13956659)(688.00762297,346.11456661)(688.06762558,346.0645743)
\curveto(688.13762284,346.01456671)(688.1726228,345.94956678)(688.17262558,345.8695743)
\curveto(688.18262279,345.79956693)(688.20762277,345.724567)(688.24762558,345.6445743)
\curveto(688.25762272,345.61456711)(688.26762271,345.58956714)(688.27762558,345.5695743)
\curveto(688.29762268,345.55956717)(688.31762266,345.54456718)(688.33762558,345.5245743)
\curveto(688.44762253,345.51456721)(688.53762244,345.54456718)(688.60762558,345.6145743)
\curveto(688.6776223,345.68456704)(688.74762223,345.74456698)(688.81762558,345.7945743)
\curveto(688.94762203,345.88456684)(689.08262189,345.96456676)(689.22262558,346.0345743)
\curveto(689.36262161,346.11456661)(689.51762146,346.17956655)(689.68762558,346.2295743)
\curveto(689.76762121,346.25956647)(689.85262112,346.27956645)(689.94262558,346.2895743)
\curveto(690.04262093,346.29956643)(690.13762084,346.31456641)(690.22762558,346.3345743)
\curveto(690.26762071,346.34456638)(690.30762067,346.34456638)(690.34762558,346.3345743)
\curveto(690.39762058,346.3245664)(690.43762054,346.3295664)(690.46762558,346.3495743)
\curveto(691.03761994,346.36956636)(691.51761946,346.28956644)(691.90762558,346.1095743)
\curveto(692.30761867,345.93956679)(692.64761833,345.71456701)(692.92762558,345.4345743)
\curveto(692.977618,345.38456734)(693.02261795,345.33456739)(693.06262558,345.2845743)
\curveto(693.10261787,345.24456748)(693.14261783,345.19956753)(693.18262558,345.1495743)
\curveto(693.25261772,345.05956767)(693.31261766,344.96956776)(693.36262558,344.8795743)
\curveto(693.42261755,344.78956794)(693.4776175,344.69956803)(693.52762558,344.6095743)
\curveto(693.54761743,344.58956814)(693.55761742,344.56456816)(693.55762558,344.5345743)
\curveto(693.56761741,344.50456822)(693.58261739,344.46956826)(693.60262558,344.4295743)
\curveto(693.66261731,344.3295684)(693.71761726,344.20956852)(693.76762558,344.0695743)
\curveto(693.78761719,344.00956872)(693.80761717,343.94456878)(693.82762558,343.8745743)
\curveto(693.84761713,343.81456891)(693.86761711,343.74956898)(693.88762558,343.6795743)
\curveto(693.92761705,343.55956917)(693.95261702,343.43456929)(693.96262558,343.3045743)
\curveto(693.98261699,343.17456955)(694.00761697,343.03956969)(694.03762558,342.8995743)
\lineto(694.03762558,342.7345743)
\lineto(694.06762558,342.5545743)
\lineto(694.06762558,342.3895743)
\moveto(691.95262558,342.0445743)
\curveto(691.96261901,342.09457063)(691.96761901,342.15957057)(691.96762558,342.2395743)
\curveto(691.96761901,342.3295704)(691.96261901,342.39957033)(691.95262558,342.4495743)
\lineto(691.95262558,342.5845743)
\curveto(691.93261904,342.64457008)(691.92261905,342.70957002)(691.92262558,342.7795743)
\curveto(691.92261905,342.84956988)(691.91261906,342.91956981)(691.89262558,342.9895743)
\curveto(691.8726191,343.08956964)(691.85261912,343.18456954)(691.83262558,343.2745743)
\curveto(691.81261916,343.37456935)(691.78261919,343.46456926)(691.74262558,343.5445743)
\curveto(691.62261935,343.86456886)(691.46761951,344.11956861)(691.27762558,344.3095743)
\curveto(691.08761989,344.49956823)(690.81762016,344.63956809)(690.46762558,344.7295743)
\curveto(690.38762059,344.74956798)(690.29762068,344.75956797)(690.19762558,344.7595743)
\lineto(689.92762558,344.7595743)
\curveto(689.88762109,344.74956798)(689.85262112,344.74456798)(689.82262558,344.7445743)
\curveto(689.79262118,344.74456798)(689.75762122,344.73956799)(689.71762558,344.7295743)
\lineto(689.50762558,344.6695743)
\curveto(689.44762153,344.65956807)(689.38762159,344.63956809)(689.32762558,344.6095743)
\curveto(689.06762191,344.49956823)(688.86262211,344.3295684)(688.71262558,344.0995743)
\curveto(688.5726224,343.86956886)(688.45762252,343.61456911)(688.36762558,343.3345743)
\curveto(688.34762263,343.25456947)(688.33262264,343.16956956)(688.32262558,343.0795743)
\curveto(688.31262266,342.99956973)(688.29762268,342.91956981)(688.27762558,342.8395743)
\curveto(688.26762271,342.79956993)(688.26262271,342.73456999)(688.26262558,342.6445743)
\curveto(688.24262273,342.60457012)(688.23762274,342.55457017)(688.24762558,342.4945743)
\curveto(688.25762272,342.44457028)(688.25762272,342.39457033)(688.24762558,342.3445743)
\curveto(688.22762275,342.28457044)(688.22762275,342.2295705)(688.24762558,342.1795743)
\lineto(688.24762558,341.9995743)
\lineto(688.24762558,341.8645743)
\curveto(688.24762273,341.8245709)(688.25762272,341.78457094)(688.27762558,341.7445743)
\curveto(688.2776227,341.67457105)(688.28262269,341.61957111)(688.29262558,341.5795743)
\lineto(688.32262558,341.3995743)
\curveto(688.33262264,341.33957139)(688.34762263,341.27957145)(688.36762558,341.2195743)
\curveto(688.45762252,340.9295718)(688.56262241,340.68957204)(688.68262558,340.4995743)
\curveto(688.81262216,340.31957241)(688.99262198,340.15957257)(689.22262558,340.0195743)
\curveto(689.36262161,339.93957279)(689.52762145,339.87457285)(689.71762558,339.8245743)
\curveto(689.75762122,339.81457291)(689.79262118,339.80957292)(689.82262558,339.8095743)
\curveto(689.85262112,339.81957291)(689.88762109,339.81957291)(689.92762558,339.8095743)
\curveto(689.96762101,339.79957293)(690.02762095,339.78957294)(690.10762558,339.7795743)
\curveto(690.18762079,339.77957295)(690.25262072,339.78457294)(690.30262558,339.7945743)
\curveto(690.38262059,339.81457291)(690.46262051,339.8295729)(690.54262558,339.8395743)
\curveto(690.63262034,339.85957287)(690.71762026,339.88457284)(690.79762558,339.9145743)
\curveto(691.03761994,340.01457271)(691.23261974,340.15457257)(691.38262558,340.3345743)
\curveto(691.53261944,340.51457221)(691.65761932,340.724572)(691.75762558,340.9645743)
\curveto(691.80761917,341.08457164)(691.84261913,341.20957152)(691.86262558,341.3395743)
\curveto(691.88261909,341.46957126)(691.90761907,341.60457112)(691.93762558,341.7445743)
\lineto(691.93762558,341.8945743)
\curveto(691.94761903,341.94457078)(691.95261902,341.99457073)(691.95262558,342.0445743)
}
}
{
\newrgbcolor{curcolor}{0 0 0}
\pscustom[linestyle=none,fillstyle=solid,fillcolor=curcolor]
{
\newpath
\moveto(702.39754745,339.0295743)
\curveto(702.4175396,338.91957381)(702.42753959,338.80957392)(702.42754745,338.6995743)
\curveto(702.43753958,338.58957414)(702.38753963,338.51457421)(702.27754745,338.4745743)
\curveto(702.2175398,338.44457428)(702.14753987,338.4295743)(702.06754745,338.4295743)
\lineto(701.82754745,338.4295743)
\lineto(701.01754745,338.4295743)
\lineto(700.74754745,338.4295743)
\curveto(700.66754135,338.43957429)(700.60254142,338.46457426)(700.55254745,338.5045743)
\curveto(700.48254154,338.54457418)(700.42754159,338.59957413)(700.38754745,338.6695743)
\curveto(700.35754166,338.74957398)(700.31254171,338.81457391)(700.25254745,338.8645743)
\curveto(700.23254179,338.88457384)(700.20754181,338.89957383)(700.17754745,338.9095743)
\curveto(700.14754187,338.9295738)(700.10754191,338.93457379)(700.05754745,338.9245743)
\curveto(700.00754201,338.90457382)(699.95754206,338.87957385)(699.90754745,338.8495743)
\curveto(699.86754215,338.81957391)(699.8225422,338.79457393)(699.77254745,338.7745743)
\curveto(699.7225423,338.73457399)(699.66754235,338.69957403)(699.60754745,338.6695743)
\lineto(699.42754745,338.5795743)
\curveto(699.29754272,338.51957421)(699.16254286,338.46957426)(699.02254745,338.4295743)
\curveto(698.88254314,338.39957433)(698.73754328,338.36457436)(698.58754745,338.3245743)
\curveto(698.5175435,338.30457442)(698.44754357,338.29457443)(698.37754745,338.2945743)
\curveto(698.3175437,338.28457444)(698.25254377,338.27457445)(698.18254745,338.2645743)
\lineto(698.09254745,338.2645743)
\curveto(698.06254396,338.25457447)(698.03254399,338.24957448)(698.00254745,338.2495743)
\lineto(697.83754745,338.2495743)
\curveto(697.73754428,338.2295745)(697.63754438,338.2295745)(697.53754745,338.2495743)
\lineto(697.40254745,338.2495743)
\curveto(697.33254469,338.26957446)(697.26254476,338.27957445)(697.19254745,338.2795743)
\curveto(697.13254489,338.26957446)(697.07254495,338.27457445)(697.01254745,338.2945743)
\curveto(696.91254511,338.31457441)(696.8175452,338.33457439)(696.72754745,338.3545743)
\curveto(696.63754538,338.36457436)(696.55254547,338.38957434)(696.47254745,338.4295743)
\curveto(696.18254584,338.53957419)(695.93254609,338.67957405)(695.72254745,338.8495743)
\curveto(695.5225465,339.0295737)(695.36254666,339.26457346)(695.24254745,339.5545743)
\curveto(695.21254681,339.6245731)(695.18254684,339.69957303)(695.15254745,339.7795743)
\curveto(695.13254689,339.85957287)(695.11254691,339.94457278)(695.09254745,340.0345743)
\curveto(695.07254695,340.08457264)(695.06254696,340.13457259)(695.06254745,340.1845743)
\curveto(695.07254695,340.23457249)(695.07254695,340.28457244)(695.06254745,340.3345743)
\curveto(695.05254697,340.36457236)(695.04254698,340.4245723)(695.03254745,340.5145743)
\curveto(695.03254699,340.61457211)(695.03754698,340.68457204)(695.04754745,340.7245743)
\curveto(695.06754695,340.8245719)(695.07754694,340.90957182)(695.07754745,340.9795743)
\lineto(695.16754745,341.3095743)
\curveto(695.19754682,341.4295713)(695.23754678,341.53457119)(695.28754745,341.6245743)
\curveto(695.45754656,341.91457081)(695.65254637,342.13457059)(695.87254745,342.2845743)
\curveto(696.09254593,342.43457029)(696.37254565,342.56457016)(696.71254745,342.6745743)
\curveto(696.84254518,342.72457)(696.97754504,342.75956997)(697.11754745,342.7795743)
\curveto(697.25754476,342.79956993)(697.39754462,342.8245699)(697.53754745,342.8545743)
\curveto(697.6175444,342.87456985)(697.70254432,342.88456984)(697.79254745,342.8845743)
\curveto(697.88254414,342.89456983)(697.97254405,342.90956982)(698.06254745,342.9295743)
\curveto(698.13254389,342.94956978)(698.20254382,342.95456977)(698.27254745,342.9445743)
\curveto(698.34254368,342.94456978)(698.4175436,342.95456977)(698.49754745,342.9745743)
\curveto(698.56754345,342.99456973)(698.63754338,343.00456972)(698.70754745,343.0045743)
\curveto(698.77754324,343.00456972)(698.85254317,343.01456971)(698.93254745,343.0345743)
\curveto(699.14254288,343.08456964)(699.33254269,343.1245696)(699.50254745,343.1545743)
\curveto(699.68254234,343.19456953)(699.84254218,343.28456944)(699.98254745,343.4245743)
\curveto(700.07254195,343.51456921)(700.13254189,343.61456911)(700.16254745,343.7245743)
\curveto(700.17254185,343.75456897)(700.17254185,343.77956895)(700.16254745,343.7995743)
\curveto(700.16254186,343.81956891)(700.16754185,343.83956889)(700.17754745,343.8595743)
\curveto(700.18754183,343.87956885)(700.19254183,343.90956882)(700.19254745,343.9495743)
\lineto(700.19254745,344.0395743)
\lineto(700.16254745,344.1595743)
\curveto(700.16254186,344.19956853)(700.15754186,344.23456849)(700.14754745,344.2645743)
\curveto(700.04754197,344.56456816)(699.83754218,344.76956796)(699.51754745,344.8795743)
\curveto(699.42754259,344.90956782)(699.3175427,344.9295678)(699.18754745,344.9395743)
\curveto(699.06754295,344.95956777)(698.94254308,344.96456776)(698.81254745,344.9545743)
\curveto(698.68254334,344.95456777)(698.55754346,344.94456778)(698.43754745,344.9245743)
\curveto(698.3175437,344.90456782)(698.21254381,344.87956785)(698.12254745,344.8495743)
\curveto(698.06254396,344.8295679)(698.00254402,344.79956793)(697.94254745,344.7595743)
\curveto(697.89254413,344.729568)(697.84254418,344.69456803)(697.79254745,344.6545743)
\curveto(697.74254428,344.61456811)(697.68754433,344.55956817)(697.62754745,344.4895743)
\curveto(697.57754444,344.41956831)(697.54254448,344.35456837)(697.52254745,344.2945743)
\curveto(697.47254455,344.19456853)(697.42754459,344.09956863)(697.38754745,344.0095743)
\curveto(697.35754466,343.91956881)(697.28754473,343.85956887)(697.17754745,343.8295743)
\curveto(697.09754492,343.80956892)(697.01254501,343.79956893)(696.92254745,343.7995743)
\lineto(696.65254745,343.7995743)
\lineto(696.08254745,343.7995743)
\curveto(696.03254599,343.79956893)(695.98254604,343.79456893)(695.93254745,343.7845743)
\curveto(695.88254614,343.78456894)(695.83754618,343.78956894)(695.79754745,343.7995743)
\lineto(695.66254745,343.7995743)
\curveto(695.64254638,343.80956892)(695.6175464,343.81456891)(695.58754745,343.8145743)
\curveto(695.55754646,343.81456891)(695.53254649,343.8245689)(695.51254745,343.8445743)
\curveto(695.43254659,343.86456886)(695.37754664,343.9295688)(695.34754745,344.0395743)
\curveto(695.33754668,344.08956864)(695.33754668,344.13956859)(695.34754745,344.1895743)
\curveto(695.35754666,344.23956849)(695.36754665,344.28456844)(695.37754745,344.3245743)
\curveto(695.40754661,344.43456829)(695.43754658,344.53456819)(695.46754745,344.6245743)
\curveto(695.50754651,344.724568)(695.55254647,344.81456791)(695.60254745,344.8945743)
\lineto(695.69254745,345.0445743)
\lineto(695.78254745,345.1945743)
\curveto(695.86254616,345.30456742)(695.96254606,345.40956732)(696.08254745,345.5095743)
\curveto(696.10254592,345.51956721)(696.13254589,345.54456718)(696.17254745,345.5845743)
\curveto(696.2225458,345.6245671)(696.26754575,345.65956707)(696.30754745,345.6895743)
\curveto(696.34754567,345.71956701)(696.39254563,345.74956698)(696.44254745,345.7795743)
\curveto(696.61254541,345.88956684)(696.79254523,345.97456675)(696.98254745,346.0345743)
\curveto(697.17254485,346.10456662)(697.36754465,346.16956656)(697.56754745,346.2295743)
\curveto(697.68754433,346.25956647)(697.81254421,346.27956645)(697.94254745,346.2895743)
\curveto(698.07254395,346.29956643)(698.20254382,346.31956641)(698.33254745,346.3495743)
\curveto(698.37254365,346.35956637)(698.43254359,346.35956637)(698.51254745,346.3495743)
\curveto(698.60254342,346.33956639)(698.65754336,346.34456638)(698.67754745,346.3645743)
\curveto(699.08754293,346.37456635)(699.47754254,346.35956637)(699.84754745,346.3195743)
\curveto(700.22754179,346.27956645)(700.56754145,346.20456652)(700.86754745,346.0945743)
\curveto(701.17754084,345.98456674)(701.44254058,345.83456689)(701.66254745,345.6445743)
\curveto(701.88254014,345.46456726)(702.05253997,345.2295675)(702.17254745,344.9395743)
\curveto(702.24253978,344.76956796)(702.28253974,344.57456815)(702.29254745,344.3545743)
\curveto(702.30253972,344.13456859)(702.30753971,343.90956882)(702.30754745,343.6795743)
\lineto(702.30754745,340.3345743)
\lineto(702.30754745,339.7495743)
\curveto(702.30753971,339.55957317)(702.32753969,339.38457334)(702.36754745,339.2245743)
\curveto(702.37753964,339.19457353)(702.38253964,339.15957357)(702.38254745,339.1195743)
\curveto(702.38253964,339.08957364)(702.38753963,339.05957367)(702.39754745,339.0295743)
\moveto(700.19254745,341.3395743)
\curveto(700.20254182,341.38957134)(700.20754181,341.44457128)(700.20754745,341.5045743)
\curveto(700.20754181,341.57457115)(700.20254182,341.63457109)(700.19254745,341.6845743)
\curveto(700.17254185,341.74457098)(700.16254186,341.79957093)(700.16254745,341.8495743)
\curveto(700.16254186,341.89957083)(700.14254188,341.93957079)(700.10254745,341.9695743)
\curveto(700.05254197,342.00957072)(699.97754204,342.0295707)(699.87754745,342.0295743)
\curveto(699.83754218,342.01957071)(699.80254222,342.00957072)(699.77254745,341.9995743)
\curveto(699.74254228,341.99957073)(699.70754231,341.99457073)(699.66754745,341.9845743)
\curveto(699.59754242,341.96457076)(699.5225425,341.94957078)(699.44254745,341.9395743)
\curveto(699.36254266,341.9295708)(699.28254274,341.91457081)(699.20254745,341.8945743)
\curveto(699.17254285,341.88457084)(699.12754289,341.87957085)(699.06754745,341.8795743)
\curveto(698.93754308,341.84957088)(698.80754321,341.8295709)(698.67754745,341.8195743)
\curveto(698.54754347,341.80957092)(698.4225436,341.78457094)(698.30254745,341.7445743)
\curveto(698.2225438,341.724571)(698.14754387,341.70457102)(698.07754745,341.6845743)
\curveto(698.00754401,341.67457105)(697.93754408,341.65457107)(697.86754745,341.6245743)
\curveto(697.65754436,341.53457119)(697.47754454,341.39957133)(697.32754745,341.2195743)
\curveto(697.18754483,341.03957169)(697.13754488,340.78957194)(697.17754745,340.4695743)
\curveto(697.19754482,340.29957243)(697.25254477,340.15957257)(697.34254745,340.0495743)
\curveto(697.41254461,339.93957279)(697.5175445,339.84957288)(697.65754745,339.7795743)
\curveto(697.79754422,339.71957301)(697.94754407,339.67457305)(698.10754745,339.6445743)
\curveto(698.27754374,339.61457311)(698.45254357,339.60457312)(698.63254745,339.6145743)
\curveto(698.8225432,339.63457309)(698.99754302,339.66957306)(699.15754745,339.7195743)
\curveto(699.4175426,339.79957293)(699.6225424,339.9245728)(699.77254745,340.0945743)
\curveto(699.9225421,340.27457245)(700.03754198,340.49457223)(700.11754745,340.7545743)
\curveto(700.13754188,340.8245719)(700.14754187,340.89457183)(700.14754745,340.9645743)
\curveto(700.15754186,341.04457168)(700.17254185,341.1245716)(700.19254745,341.2045743)
\lineto(700.19254745,341.3395743)
}
}
{
\newrgbcolor{curcolor}{0 0 0}
\pscustom[linestyle=none,fillstyle=solid,fillcolor=curcolor]
{
\newpath
\moveto(707.5308287,346.3795743)
\curveto(708.34082354,346.39956633)(709.01582287,346.27956645)(709.5558287,346.0195743)
\curveto(710.10582178,345.75956697)(710.54082134,345.38956734)(710.8608287,344.9095743)
\curveto(711.02082086,344.66956806)(711.14082074,344.39456833)(711.2208287,344.0845743)
\curveto(711.24082064,344.03456869)(711.25582063,343.96956876)(711.2658287,343.8895743)
\curveto(711.2858206,343.80956892)(711.2858206,343.73956899)(711.2658287,343.6795743)
\curveto(711.22582066,343.56956916)(711.15582073,343.50456922)(711.0558287,343.4845743)
\curveto(710.95582093,343.47456925)(710.83582105,343.46956926)(710.6958287,343.4695743)
\lineto(709.9158287,343.4695743)
\lineto(709.6308287,343.4695743)
\curveto(709.54082234,343.46956926)(709.46582242,343.48956924)(709.4058287,343.5295743)
\curveto(709.32582256,343.56956916)(709.27082261,343.6295691)(709.2408287,343.7095743)
\curveto(709.21082267,343.79956893)(709.17082271,343.88956884)(709.1208287,343.9795743)
\curveto(709.06082282,344.08956864)(708.99582289,344.18956854)(708.9258287,344.2795743)
\curveto(708.85582303,344.36956836)(708.77582311,344.44956828)(708.6858287,344.5195743)
\curveto(708.54582334,344.60956812)(708.39082349,344.67956805)(708.2208287,344.7295743)
\curveto(708.16082372,344.74956798)(708.10082378,344.75956797)(708.0408287,344.7595743)
\curveto(707.9808239,344.75956797)(707.92582396,344.76956796)(707.8758287,344.7895743)
\lineto(707.7258287,344.7895743)
\curveto(707.52582436,344.78956794)(707.36582452,344.76956796)(707.2458287,344.7295743)
\curveto(706.95582493,344.63956809)(706.72082516,344.49956823)(706.5408287,344.3095743)
\curveto(706.36082552,344.1295686)(706.21582567,343.90956882)(706.1058287,343.6495743)
\curveto(706.05582583,343.53956919)(706.01582587,343.41956931)(705.9858287,343.2895743)
\curveto(705.96582592,343.16956956)(705.94082594,343.03956969)(705.9108287,342.8995743)
\curveto(705.90082598,342.85956987)(705.89582599,342.81956991)(705.8958287,342.7795743)
\curveto(705.89582599,342.73956999)(705.89082599,342.69957003)(705.8808287,342.6595743)
\curveto(705.86082602,342.55957017)(705.85082603,342.41957031)(705.8508287,342.2395743)
\curveto(705.86082602,342.05957067)(705.87582601,341.91957081)(705.8958287,341.8195743)
\curveto(705.89582599,341.73957099)(705.90082598,341.68457104)(705.9108287,341.6545743)
\curveto(705.93082595,341.58457114)(705.94082594,341.51457121)(705.9408287,341.4445743)
\curveto(705.95082593,341.37457135)(705.96582592,341.30457142)(705.9858287,341.2345743)
\curveto(706.06582582,341.00457172)(706.16082572,340.79457193)(706.2708287,340.6045743)
\curveto(706.3808255,340.41457231)(706.52082536,340.25457247)(706.6908287,340.1245743)
\curveto(706.73082515,340.09457263)(706.79082509,340.05957267)(706.8708287,340.0195743)
\curveto(706.9808249,339.94957278)(707.09082479,339.90457282)(707.2008287,339.8845743)
\curveto(707.32082456,339.86457286)(707.46582442,339.84457288)(707.6358287,339.8245743)
\lineto(707.7258287,339.8245743)
\curveto(707.76582412,339.8245729)(707.79582409,339.8295729)(707.8158287,339.8395743)
\lineto(707.9508287,339.8395743)
\curveto(708.02082386,339.85957287)(708.0858238,339.87457285)(708.1458287,339.8845743)
\curveto(708.21582367,339.90457282)(708.2808236,339.9245728)(708.3408287,339.9445743)
\curveto(708.64082324,340.07457265)(708.87082301,340.26457246)(709.0308287,340.5145743)
\curveto(709.07082281,340.56457216)(709.10582278,340.61957211)(709.1358287,340.6795743)
\curveto(709.16582272,340.74957198)(709.19082269,340.80957192)(709.2108287,340.8595743)
\curveto(709.25082263,340.96957176)(709.2858226,341.06457166)(709.3158287,341.1445743)
\curveto(709.34582254,341.23457149)(709.41582247,341.30457142)(709.5258287,341.3545743)
\curveto(709.61582227,341.39457133)(709.76082212,341.40957132)(709.9608287,341.3995743)
\lineto(710.4558287,341.3995743)
\lineto(710.6658287,341.3995743)
\curveto(710.74582114,341.40957132)(710.81082107,341.40457132)(710.8608287,341.3845743)
\lineto(710.9808287,341.3845743)
\lineto(711.1008287,341.3545743)
\curveto(711.14082074,341.35457137)(711.17082071,341.34457138)(711.1908287,341.3245743)
\curveto(711.24082064,341.28457144)(711.27082061,341.2245715)(711.2808287,341.1445743)
\curveto(711.30082058,341.07457165)(711.30082058,340.99957173)(711.2808287,340.9195743)
\curveto(711.19082069,340.58957214)(711.0808208,340.29457243)(710.9508287,340.0345743)
\curveto(710.54082134,339.26457346)(709.885822,338.729574)(708.9858287,338.4295743)
\curveto(708.885823,338.39957433)(708.7808231,338.37957435)(708.6708287,338.3695743)
\curveto(708.56082332,338.34957438)(708.45082343,338.3245744)(708.3408287,338.2945743)
\curveto(708.2808236,338.28457444)(708.22082366,338.27957445)(708.1608287,338.2795743)
\curveto(708.10082378,338.27957445)(708.04082384,338.27457445)(707.9808287,338.2645743)
\lineto(707.8158287,338.2645743)
\curveto(707.76582412,338.24457448)(707.69082419,338.23957449)(707.5908287,338.2495743)
\curveto(707.49082439,338.24957448)(707.41582447,338.25457447)(707.3658287,338.2645743)
\curveto(707.2858246,338.28457444)(707.21082467,338.29457443)(707.1408287,338.2945743)
\curveto(707.0808248,338.28457444)(707.01582487,338.28957444)(706.9458287,338.3095743)
\lineto(706.7958287,338.3395743)
\curveto(706.74582514,338.33957439)(706.69582519,338.34457438)(706.6458287,338.3545743)
\curveto(706.53582535,338.38457434)(706.43082545,338.41457431)(706.3308287,338.4445743)
\curveto(706.23082565,338.47457425)(706.13582575,338.50957422)(706.0458287,338.5495743)
\curveto(705.57582631,338.74957398)(705.1808267,339.00457372)(704.8608287,339.3145743)
\curveto(704.54082734,339.63457309)(704.2808276,340.0295727)(704.0808287,340.4995743)
\curveto(704.03082785,340.58957214)(703.99082789,340.68457204)(703.9608287,340.7845743)
\lineto(703.8708287,341.1145743)
\curveto(703.86082802,341.15457157)(703.85582803,341.18957154)(703.8558287,341.2195743)
\curveto(703.85582803,341.25957147)(703.84582804,341.30457142)(703.8258287,341.3545743)
\curveto(703.80582808,341.4245713)(703.79582809,341.49457123)(703.7958287,341.5645743)
\curveto(703.79582809,341.64457108)(703.7858281,341.71957101)(703.7658287,341.7895743)
\lineto(703.7658287,342.0445743)
\curveto(703.74582814,342.09457063)(703.73582815,342.14957058)(703.7358287,342.2095743)
\curveto(703.73582815,342.27957045)(703.74582814,342.33957039)(703.7658287,342.3895743)
\curveto(703.77582811,342.43957029)(703.77582811,342.48457024)(703.7658287,342.5245743)
\curveto(703.75582813,342.56457016)(703.75582813,342.60457012)(703.7658287,342.6445743)
\curveto(703.7858281,342.71457001)(703.79082809,342.77956995)(703.7808287,342.8395743)
\curveto(703.7808281,342.89956983)(703.79082809,342.95956977)(703.8108287,343.0195743)
\curveto(703.86082802,343.19956953)(703.90082798,343.36956936)(703.9308287,343.5295743)
\curveto(703.96082792,343.69956903)(704.00582788,343.86456886)(704.0658287,344.0245743)
\curveto(704.2858276,344.53456819)(704.56082732,344.95956777)(704.8908287,345.2995743)
\curveto(705.23082665,345.63956709)(705.66082622,345.91456681)(706.1808287,346.1245743)
\curveto(706.32082556,346.18456654)(706.46582542,346.2245665)(706.6158287,346.2445743)
\curveto(706.76582512,346.27456645)(706.92082496,346.30956642)(707.0808287,346.3495743)
\curveto(707.16082472,346.35956637)(707.23582465,346.36456636)(707.3058287,346.3645743)
\curveto(707.37582451,346.36456636)(707.45082443,346.36956636)(707.5308287,346.3795743)
}
}
{
\newrgbcolor{curcolor}{0 0 0}
\pscustom[linestyle=none,fillstyle=solid,fillcolor=curcolor]
{
\newpath
\moveto(714.67410995,349.0195743)
\curveto(714.744107,348.93956379)(714.77910697,348.81956391)(714.77910995,348.6595743)
\lineto(714.77910995,348.1945743)
\lineto(714.77910995,347.7895743)
\curveto(714.77910697,347.64956508)(714.744107,347.55456517)(714.67410995,347.5045743)
\curveto(714.61410713,347.45456527)(714.53410721,347.4245653)(714.43410995,347.4145743)
\curveto(714.3441074,347.40456532)(714.2441075,347.39956533)(714.13410995,347.3995743)
\lineto(713.29410995,347.3995743)
\curveto(713.18410856,347.39956533)(713.08410866,347.40456532)(712.99410995,347.4145743)
\curveto(712.91410883,347.4245653)(712.8441089,347.45456527)(712.78410995,347.5045743)
\curveto(712.744109,347.53456519)(712.71410903,347.58956514)(712.69410995,347.6695743)
\curveto(712.68410906,347.75956497)(712.67410907,347.85456487)(712.66410995,347.9545743)
\lineto(712.66410995,348.2845743)
\curveto(712.67410907,348.39456433)(712.67910907,348.48956424)(712.67910995,348.5695743)
\lineto(712.67910995,348.7795743)
\curveto(712.68910906,348.84956388)(712.70910904,348.90956382)(712.73910995,348.9595743)
\curveto(712.75910899,348.99956373)(712.78410896,349.0295637)(712.81410995,349.0495743)
\lineto(712.93410995,349.1095743)
\curveto(712.95410879,349.10956362)(712.97910877,349.10956362)(713.00910995,349.1095743)
\curveto(713.03910871,349.11956361)(713.06410868,349.1245636)(713.08410995,349.1245743)
\lineto(714.17910995,349.1245743)
\curveto(714.27910747,349.1245636)(714.37410737,349.11956361)(714.46410995,349.1095743)
\curveto(714.55410719,349.09956363)(714.62410712,349.06956366)(714.67410995,349.0195743)
\moveto(714.77910995,339.2545743)
\curveto(714.77910697,339.05457367)(714.77410697,338.88457384)(714.76410995,338.7445743)
\curveto(714.75410699,338.60457412)(714.66410708,338.50957422)(714.49410995,338.4595743)
\curveto(714.43410731,338.43957429)(714.36910738,338.4295743)(714.29910995,338.4295743)
\curveto(714.22910752,338.43957429)(714.15410759,338.44457428)(714.07410995,338.4445743)
\lineto(713.23410995,338.4445743)
\curveto(713.1441086,338.44457428)(713.05410869,338.44957428)(712.96410995,338.4595743)
\curveto(712.88410886,338.46957426)(712.82410892,338.49957423)(712.78410995,338.5495743)
\curveto(712.72410902,338.61957411)(712.68910906,338.70457402)(712.67910995,338.8045743)
\lineto(712.67910995,339.1495743)
\lineto(712.67910995,345.4795743)
\lineto(712.67910995,345.7795743)
\curveto(712.67910907,345.87956685)(712.69910905,345.95956677)(712.73910995,346.0195743)
\curveto(712.79910895,346.08956664)(712.88410886,346.13456659)(712.99410995,346.1545743)
\curveto(713.01410873,346.16456656)(713.03910871,346.16456656)(713.06910995,346.1545743)
\curveto(713.10910864,346.15456657)(713.13910861,346.15956657)(713.15910995,346.1695743)
\lineto(713.90910995,346.1695743)
\lineto(714.10410995,346.1695743)
\curveto(714.18410756,346.17956655)(714.2491075,346.17956655)(714.29910995,346.1695743)
\lineto(714.41910995,346.1695743)
\curveto(714.47910727,346.14956658)(714.53410721,346.13456659)(714.58410995,346.1245743)
\curveto(714.63410711,346.11456661)(714.67410707,346.08456664)(714.70410995,346.0345743)
\curveto(714.744107,345.98456674)(714.76410698,345.91456681)(714.76410995,345.8245743)
\curveto(714.77410697,345.73456699)(714.77910697,345.63956709)(714.77910995,345.5395743)
\lineto(714.77910995,339.2545743)
}
}
{
\newrgbcolor{curcolor}{0 0 0}
\pscustom[linestyle=none,fillstyle=solid,fillcolor=curcolor]
{
\newpath
\moveto(724.21129745,342.6145743)
\curveto(724.23128888,342.55457017)(724.24128887,342.46957026)(724.24129745,342.3595743)
\curveto(724.24128887,342.24957048)(724.23128888,342.16457056)(724.21129745,342.1045743)
\lineto(724.21129745,341.9545743)
\curveto(724.19128892,341.87457085)(724.18128893,341.79457093)(724.18129745,341.7145743)
\curveto(724.19128892,341.63457109)(724.18628893,341.55457117)(724.16629745,341.4745743)
\curveto(724.14628897,341.40457132)(724.13128898,341.33957139)(724.12129745,341.2795743)
\curveto(724.111289,341.21957151)(724.10128901,341.15457157)(724.09129745,341.0845743)
\curveto(724.05128906,340.97457175)(724.0162891,340.85957187)(723.98629745,340.7395743)
\curveto(723.95628916,340.6295721)(723.9162892,340.5245722)(723.86629745,340.4245743)
\curveto(723.65628946,339.94457278)(723.38128973,339.55457317)(723.04129745,339.2545743)
\curveto(722.70129041,338.95457377)(722.29129082,338.70457402)(721.81129745,338.5045743)
\curveto(721.69129142,338.45457427)(721.56629155,338.41957431)(721.43629745,338.3995743)
\curveto(721.3162918,338.36957436)(721.19129192,338.33957439)(721.06129745,338.3095743)
\curveto(721.0112921,338.28957444)(720.95629216,338.27957445)(720.89629745,338.2795743)
\curveto(720.83629228,338.27957445)(720.78129233,338.27457445)(720.73129745,338.2645743)
\lineto(720.62629745,338.2645743)
\curveto(720.59629252,338.25457447)(720.56629255,338.24957448)(720.53629745,338.2495743)
\curveto(720.48629263,338.23957449)(720.40629271,338.23457449)(720.29629745,338.2345743)
\curveto(720.18629293,338.2245745)(720.10129301,338.2295745)(720.04129745,338.2495743)
\lineto(719.89129745,338.2495743)
\curveto(719.84129327,338.25957447)(719.78629333,338.26457446)(719.72629745,338.2645743)
\curveto(719.67629344,338.25457447)(719.62629349,338.25957447)(719.57629745,338.2795743)
\curveto(719.53629358,338.28957444)(719.49629362,338.29457443)(719.45629745,338.2945743)
\curveto(719.42629369,338.29457443)(719.38629373,338.29957443)(719.33629745,338.3095743)
\curveto(719.23629388,338.33957439)(719.13629398,338.36457436)(719.03629745,338.3845743)
\curveto(718.93629418,338.40457432)(718.84129427,338.43457429)(718.75129745,338.4745743)
\curveto(718.63129448,338.51457421)(718.5162946,338.55457417)(718.40629745,338.5945743)
\curveto(718.30629481,338.63457409)(718.20129491,338.68457404)(718.09129745,338.7445743)
\curveto(717.74129537,338.95457377)(717.44129567,339.19957353)(717.19129745,339.4795743)
\curveto(716.94129617,339.75957297)(716.73129638,340.09457263)(716.56129745,340.4845743)
\curveto(716.5112966,340.57457215)(716.47129664,340.66957206)(716.44129745,340.7695743)
\curveto(716.42129669,340.86957186)(716.39629672,340.97457175)(716.36629745,341.0845743)
\curveto(716.34629677,341.13457159)(716.33629678,341.17957155)(716.33629745,341.2195743)
\curveto(716.33629678,341.25957147)(716.32629679,341.30457142)(716.30629745,341.3545743)
\curveto(716.28629683,341.43457129)(716.27629684,341.51457121)(716.27629745,341.5945743)
\curveto(716.27629684,341.68457104)(716.26629685,341.76957096)(716.24629745,341.8495743)
\curveto(716.23629688,341.89957083)(716.23129688,341.94457078)(716.23129745,341.9845743)
\lineto(716.23129745,342.1195743)
\curveto(716.2112969,342.17957055)(716.20129691,342.26457046)(716.20129745,342.3745743)
\curveto(716.2112969,342.48457024)(716.22629689,342.56957016)(716.24629745,342.6295743)
\lineto(716.24629745,342.7345743)
\curveto(716.25629686,342.78456994)(716.25629686,342.83456989)(716.24629745,342.8845743)
\curveto(716.24629687,342.94456978)(716.25629686,342.99956973)(716.27629745,343.0495743)
\curveto(716.28629683,343.09956963)(716.29129682,343.14456958)(716.29129745,343.1845743)
\curveto(716.29129682,343.23456949)(716.30129681,343.28456944)(716.32129745,343.3345743)
\curveto(716.36129675,343.46456926)(716.39629672,343.58956914)(716.42629745,343.7095743)
\curveto(716.45629666,343.83956889)(716.49629662,343.96456876)(716.54629745,344.0845743)
\curveto(716.72629639,344.49456823)(716.94129617,344.83456789)(717.19129745,345.1045743)
\curveto(717.44129567,345.38456734)(717.74629537,345.63956709)(718.10629745,345.8695743)
\curveto(718.20629491,345.91956681)(718.3112948,345.96456676)(718.42129745,346.0045743)
\curveto(718.53129458,346.04456668)(718.64129447,346.08956664)(718.75129745,346.1395743)
\curveto(718.88129423,346.18956654)(719.0162941,346.2245665)(719.15629745,346.2445743)
\curveto(719.29629382,346.26456646)(719.44129367,346.29456643)(719.59129745,346.3345743)
\curveto(719.67129344,346.34456638)(719.74629337,346.34956638)(719.81629745,346.3495743)
\curveto(719.88629323,346.34956638)(719.95629316,346.35456637)(720.02629745,346.3645743)
\curveto(720.60629251,346.37456635)(721.10629201,346.31456641)(721.52629745,346.1845743)
\curveto(721.95629116,346.05456667)(722.33629078,345.87456685)(722.66629745,345.6445743)
\curveto(722.77629034,345.56456716)(722.88629023,345.47456725)(722.99629745,345.3745743)
\curveto(723.11629,345.28456744)(723.2162899,345.18456754)(723.29629745,345.0745743)
\curveto(723.37628974,344.97456775)(723.44628967,344.87456785)(723.50629745,344.7745743)
\curveto(723.57628954,344.67456805)(723.64628947,344.56956816)(723.71629745,344.4595743)
\curveto(723.78628933,344.34956838)(723.84128927,344.2295685)(723.88129745,344.0995743)
\curveto(723.92128919,343.97956875)(723.96628915,343.84956888)(724.01629745,343.7095743)
\curveto(724.04628907,343.6295691)(724.07128904,343.54456918)(724.09129745,343.4545743)
\lineto(724.15129745,343.1845743)
\curveto(724.16128895,343.14456958)(724.16628895,343.10456962)(724.16629745,343.0645743)
\curveto(724.16628895,343.0245697)(724.17128894,342.98456974)(724.18129745,342.9445743)
\curveto(724.20128891,342.89456983)(724.20628891,342.83956989)(724.19629745,342.7795743)
\curveto(724.18628893,342.71957001)(724.19128892,342.66457006)(724.21129745,342.6145743)
\moveto(722.11129745,342.0745743)
\curveto(722.12129099,342.1245706)(722.12629099,342.19457053)(722.12629745,342.2845743)
\curveto(722.12629099,342.38457034)(722.12129099,342.45957027)(722.11129745,342.5095743)
\lineto(722.11129745,342.6295743)
\curveto(722.09129102,342.67957005)(722.08129103,342.73456999)(722.08129745,342.7945743)
\curveto(722.08129103,342.85456987)(722.07629104,342.90956982)(722.06629745,342.9595743)
\curveto(722.06629105,342.99956973)(722.06129105,343.0295697)(722.05129745,343.0495743)
\lineto(721.99129745,343.2895743)
\curveto(721.98129113,343.37956935)(721.96129115,343.46456926)(721.93129745,343.5445743)
\curveto(721.82129129,343.80456892)(721.69129142,344.0245687)(721.54129745,344.2045743)
\curveto(721.39129172,344.39456833)(721.19129192,344.54456818)(720.94129745,344.6545743)
\curveto(720.88129223,344.67456805)(720.82129229,344.68956804)(720.76129745,344.6995743)
\curveto(720.70129241,344.71956801)(720.63629248,344.73956799)(720.56629745,344.7595743)
\curveto(720.48629263,344.77956795)(720.40129271,344.78456794)(720.31129745,344.7745743)
\lineto(720.04129745,344.7745743)
\curveto(720.0112931,344.75456797)(719.97629314,344.74456798)(719.93629745,344.7445743)
\curveto(719.89629322,344.75456797)(719.86129325,344.75456797)(719.83129745,344.7445743)
\lineto(719.62129745,344.6845743)
\curveto(719.56129355,344.67456805)(719.50629361,344.65456807)(719.45629745,344.6245743)
\curveto(719.20629391,344.51456821)(719.00129411,344.35456837)(718.84129745,344.1445743)
\curveto(718.69129442,343.94456878)(718.57129454,343.70956902)(718.48129745,343.4395743)
\curveto(718.45129466,343.33956939)(718.42629469,343.23456949)(718.40629745,343.1245743)
\curveto(718.39629472,343.01456971)(718.38129473,342.90456982)(718.36129745,342.7945743)
\curveto(718.35129476,342.74456998)(718.34629477,342.69457003)(718.34629745,342.6445743)
\lineto(718.34629745,342.4945743)
\curveto(718.32629479,342.4245703)(718.3162948,342.31957041)(718.31629745,342.1795743)
\curveto(718.32629479,342.03957069)(718.34129477,341.93457079)(718.36129745,341.8645743)
\lineto(718.36129745,341.7295743)
\curveto(718.38129473,341.64957108)(718.39629472,341.56957116)(718.40629745,341.4895743)
\curveto(718.4162947,341.41957131)(718.43129468,341.34457138)(718.45129745,341.2645743)
\curveto(718.55129456,340.96457176)(718.65629446,340.71957201)(718.76629745,340.5295743)
\curveto(718.88629423,340.34957238)(719.07129404,340.18457254)(719.32129745,340.0345743)
\curveto(719.39129372,339.98457274)(719.46629365,339.94457278)(719.54629745,339.9145743)
\curveto(719.63629348,339.88457284)(719.72629339,339.85957287)(719.81629745,339.8395743)
\curveto(719.85629326,339.8295729)(719.89129322,339.8245729)(719.92129745,339.8245743)
\curveto(719.95129316,339.83457289)(719.98629313,339.83457289)(720.02629745,339.8245743)
\lineto(720.14629745,339.7945743)
\curveto(720.19629292,339.79457293)(720.24129287,339.79957293)(720.28129745,339.8095743)
\lineto(720.40129745,339.8095743)
\curveto(720.48129263,339.8295729)(720.56129255,339.84457288)(720.64129745,339.8545743)
\curveto(720.72129239,339.86457286)(720.79629232,339.88457284)(720.86629745,339.9145743)
\curveto(721.12629199,340.01457271)(721.33629178,340.14957258)(721.49629745,340.3195743)
\curveto(721.65629146,340.48957224)(721.79129132,340.69957203)(721.90129745,340.9495743)
\curveto(721.94129117,341.04957168)(721.97129114,341.14957158)(721.99129745,341.2495743)
\curveto(722.0112911,341.34957138)(722.03629108,341.45457127)(722.06629745,341.5645743)
\curveto(722.07629104,341.60457112)(722.08129103,341.63957109)(722.08129745,341.6695743)
\curveto(722.08129103,341.70957102)(722.08629103,341.74957098)(722.09629745,341.7895743)
\lineto(722.09629745,341.9245743)
\curveto(722.09629102,341.97457075)(722.10129101,342.0245707)(722.11129745,342.0745743)
}
}
{
\newrgbcolor{curcolor}{0 0 0}
\pscustom[linestyle=none,fillstyle=solid,fillcolor=curcolor]
{
\newpath
\moveto(263.08072616,269.42545931)
\curveto(263.18072131,269.42544869)(263.27572121,269.4154487)(263.36572616,269.39545931)
\curveto(263.45572103,269.38544873)(263.52072097,269.35544876)(263.56072616,269.30545931)
\curveto(263.62072087,269.22544889)(263.65072084,269.12044899)(263.65072616,268.99045931)
\lineto(263.65072616,268.60045931)
\lineto(263.65072616,267.10045931)
\lineto(263.65072616,260.71045931)
\lineto(263.65072616,259.54045931)
\lineto(263.65072616,259.22545931)
\curveto(263.66072083,259.12545899)(263.64572084,259.04545907)(263.60572616,258.98545931)
\curveto(263.55572093,258.90545921)(263.48072101,258.85545926)(263.38072616,258.83545931)
\curveto(263.2907212,258.82545929)(263.18072131,258.82045929)(263.05072616,258.82045931)
\lineto(262.82572616,258.82045931)
\curveto(262.74572174,258.84045927)(262.67572181,258.85545926)(262.61572616,258.86545931)
\curveto(262.55572193,258.88545923)(262.50572198,258.92545919)(262.46572616,258.98545931)
\curveto(262.42572206,259.04545907)(262.40572208,259.12045899)(262.40572616,259.21045931)
\lineto(262.40572616,259.51045931)
\lineto(262.40572616,260.60545931)
\lineto(262.40572616,265.94545931)
\curveto(262.3857221,266.03545208)(262.37072212,266.110452)(262.36072616,266.17045931)
\curveto(262.36072213,266.24045187)(262.33072216,266.30045181)(262.27072616,266.35045931)
\curveto(262.20072229,266.40045171)(262.11072238,266.42545169)(262.00072616,266.42545931)
\curveto(261.90072259,266.43545168)(261.7907227,266.44045167)(261.67072616,266.44045931)
\lineto(260.53072616,266.44045931)
\lineto(260.03572616,266.44045931)
\curveto(259.87572461,266.45045166)(259.76572472,266.5104516)(259.70572616,266.62045931)
\curveto(259.6857248,266.65045146)(259.67572481,266.68045143)(259.67572616,266.71045931)
\curveto(259.67572481,266.75045136)(259.67072482,266.79545132)(259.66072616,266.84545931)
\curveto(259.64072485,266.96545115)(259.64572484,267.07545104)(259.67572616,267.17545931)
\curveto(259.71572477,267.27545084)(259.77072472,267.34545077)(259.84072616,267.38545931)
\curveto(259.92072457,267.43545068)(260.04072445,267.46045065)(260.20072616,267.46045931)
\curveto(260.36072413,267.46045065)(260.49572399,267.47545064)(260.60572616,267.50545931)
\curveto(260.65572383,267.5154506)(260.71072378,267.52045059)(260.77072616,267.52045931)
\curveto(260.83072366,267.53045058)(260.8907236,267.54545057)(260.95072616,267.56545931)
\curveto(261.10072339,267.6154505)(261.24572324,267.66545045)(261.38572616,267.71545931)
\curveto(261.52572296,267.77545034)(261.66072283,267.84545027)(261.79072616,267.92545931)
\curveto(261.93072256,268.0154501)(262.05072244,268.12044999)(262.15072616,268.24045931)
\curveto(262.25072224,268.36044975)(262.34572214,268.49044962)(262.43572616,268.63045931)
\curveto(262.49572199,268.73044938)(262.54072195,268.84044927)(262.57072616,268.96045931)
\curveto(262.61072188,269.08044903)(262.66072183,269.18544893)(262.72072616,269.27545931)
\curveto(262.77072172,269.33544878)(262.84072165,269.37544874)(262.93072616,269.39545931)
\curveto(262.95072154,269.40544871)(262.97572151,269.4104487)(263.00572616,269.41045931)
\curveto(263.03572145,269.4104487)(263.06072143,269.4154487)(263.08072616,269.42545931)
}
}
{
\newrgbcolor{curcolor}{0 0 0}
\pscustom[linestyle=none,fillstyle=solid,fillcolor=curcolor]
{
\newpath
\moveto(270.62033554,269.42545931)
\curveto(271.3103309,269.43544868)(271.9103303,269.3154488)(272.42033554,269.06545931)
\curveto(272.94032927,268.8154493)(273.33532888,268.48044963)(273.60533554,268.06045931)
\curveto(273.65532856,267.98045013)(273.70032851,267.89045022)(273.74033554,267.79045931)
\curveto(273.78032843,267.70045041)(273.82532839,267.60545051)(273.87533554,267.50545931)
\curveto(273.9153283,267.40545071)(273.94532827,267.30545081)(273.96533554,267.20545931)
\curveto(273.98532823,267.10545101)(274.00532821,267.00045111)(274.02533554,266.89045931)
\curveto(274.04532817,266.84045127)(274.05032816,266.79545132)(274.04033554,266.75545931)
\curveto(274.03032818,266.7154514)(274.03532818,266.67045144)(274.05533554,266.62045931)
\curveto(274.06532815,266.57045154)(274.07032814,266.48545163)(274.07033554,266.36545931)
\curveto(274.07032814,266.25545186)(274.06532815,266.17045194)(274.05533554,266.11045931)
\curveto(274.03532818,266.05045206)(274.02532819,265.99045212)(274.02533554,265.93045931)
\curveto(274.03532818,265.87045224)(274.03032818,265.8104523)(274.01033554,265.75045931)
\curveto(273.97032824,265.6104525)(273.93532828,265.47545264)(273.90533554,265.34545931)
\curveto(273.87532834,265.2154529)(273.83532838,265.09045302)(273.78533554,264.97045931)
\curveto(273.72532849,264.83045328)(273.65532856,264.70545341)(273.57533554,264.59545931)
\curveto(273.50532871,264.48545363)(273.43032878,264.37545374)(273.35033554,264.26545931)
\lineto(273.29033554,264.20545931)
\curveto(273.28032893,264.18545393)(273.26532895,264.16545395)(273.24533554,264.14545931)
\curveto(273.12532909,263.98545413)(272.99032922,263.84045427)(272.84033554,263.71045931)
\curveto(272.69032952,263.58045453)(272.53032968,263.45545466)(272.36033554,263.33545931)
\curveto(272.05033016,263.115455)(271.75533046,262.9104552)(271.47533554,262.72045931)
\curveto(271.24533097,262.58045553)(271.0153312,262.44545567)(270.78533554,262.31545931)
\curveto(270.56533165,262.18545593)(270.34533187,262.05045606)(270.12533554,261.91045931)
\curveto(269.87533234,261.74045637)(269.63533258,261.56045655)(269.40533554,261.37045931)
\curveto(269.18533303,261.18045693)(268.99533322,260.95545716)(268.83533554,260.69545931)
\curveto(268.79533342,260.63545748)(268.76033345,260.57545754)(268.73033554,260.51545931)
\curveto(268.70033351,260.46545765)(268.67033354,260.40045771)(268.64033554,260.32045931)
\curveto(268.62033359,260.25045786)(268.6153336,260.19045792)(268.62533554,260.14045931)
\curveto(268.64533357,260.07045804)(268.68033353,260.0154581)(268.73033554,259.97545931)
\curveto(268.78033343,259.94545817)(268.84033337,259.92545819)(268.91033554,259.91545931)
\lineto(269.15033554,259.91545931)
\lineto(269.90033554,259.91545931)
\lineto(272.70533554,259.91545931)
\lineto(273.36533554,259.91545931)
\curveto(273.45532876,259.9154582)(273.54032867,259.9104582)(273.62033554,259.90045931)
\curveto(273.70032851,259.90045821)(273.76532845,259.88045823)(273.81533554,259.84045931)
\curveto(273.86532835,259.80045831)(273.90532831,259.72545839)(273.93533554,259.61545931)
\curveto(273.97532824,259.5154586)(273.98532823,259.4154587)(273.96533554,259.31545931)
\lineto(273.96533554,259.18045931)
\curveto(273.94532827,259.110459)(273.92532829,259.05045906)(273.90533554,259.00045931)
\curveto(273.88532833,258.95045916)(273.85032836,258.9104592)(273.80033554,258.88045931)
\curveto(273.75032846,258.84045927)(273.68032853,258.82045929)(273.59033554,258.82045931)
\lineto(273.32033554,258.82045931)
\lineto(272.42033554,258.82045931)
\lineto(268.91033554,258.82045931)
\lineto(267.84533554,258.82045931)
\curveto(267.76533445,258.82045929)(267.67533454,258.8154593)(267.57533554,258.80545931)
\curveto(267.47533474,258.80545931)(267.39033482,258.8154593)(267.32033554,258.83545931)
\curveto(267.1103351,258.90545921)(267.04533517,259.08545903)(267.12533554,259.37545931)
\curveto(267.13533508,259.4154587)(267.13533508,259.45045866)(267.12533554,259.48045931)
\curveto(267.12533509,259.52045859)(267.13533508,259.56545855)(267.15533554,259.61545931)
\curveto(267.17533504,259.69545842)(267.19533502,259.78045833)(267.21533554,259.87045931)
\curveto(267.23533498,259.96045815)(267.26033495,260.04545807)(267.29033554,260.12545931)
\curveto(267.45033476,260.6154575)(267.65033456,261.03045708)(267.89033554,261.37045931)
\curveto(268.07033414,261.62045649)(268.27533394,261.84545627)(268.50533554,262.04545931)
\curveto(268.73533348,262.25545586)(268.97533324,262.45045566)(269.22533554,262.63045931)
\curveto(269.48533273,262.8104553)(269.75033246,262.98045513)(270.02033554,263.14045931)
\curveto(270.30033191,263.3104548)(270.57033164,263.48545463)(270.83033554,263.66545931)
\curveto(270.94033127,263.74545437)(271.04533117,263.82045429)(271.14533554,263.89045931)
\curveto(271.25533096,263.96045415)(271.36533085,264.03545408)(271.47533554,264.11545931)
\curveto(271.5153307,264.14545397)(271.55033066,264.17545394)(271.58033554,264.20545931)
\curveto(271.62033059,264.24545387)(271.66033055,264.27545384)(271.70033554,264.29545931)
\curveto(271.84033037,264.40545371)(271.96533025,264.53045358)(272.07533554,264.67045931)
\curveto(272.09533012,264.70045341)(272.12033009,264.72545339)(272.15033554,264.74545931)
\curveto(272.18033003,264.77545334)(272.20533001,264.80545331)(272.22533554,264.83545931)
\curveto(272.30532991,264.93545318)(272.37032984,265.03545308)(272.42033554,265.13545931)
\curveto(272.48032973,265.23545288)(272.53532968,265.34545277)(272.58533554,265.46545931)
\curveto(272.6153296,265.53545258)(272.63532958,265.6104525)(272.64533554,265.69045931)
\lineto(272.70533554,265.93045931)
\lineto(272.70533554,266.02045931)
\curveto(272.7153295,266.05045206)(272.72032949,266.08045203)(272.72033554,266.11045931)
\curveto(272.74032947,266.18045193)(272.74532947,266.27545184)(272.73533554,266.39545931)
\curveto(272.73532948,266.52545159)(272.72532949,266.62545149)(272.70533554,266.69545931)
\curveto(272.68532953,266.77545134)(272.66532955,266.85045126)(272.64533554,266.92045931)
\curveto(272.63532958,267.00045111)(272.6153296,267.08045103)(272.58533554,267.16045931)
\curveto(272.47532974,267.40045071)(272.32532989,267.60045051)(272.13533554,267.76045931)
\curveto(271.95533026,267.93045018)(271.73533048,268.07045004)(271.47533554,268.18045931)
\curveto(271.40533081,268.20044991)(271.33533088,268.2154499)(271.26533554,268.22545931)
\curveto(271.19533102,268.24544987)(271.12033109,268.26544985)(271.04033554,268.28545931)
\curveto(270.96033125,268.30544981)(270.85033136,268.3154498)(270.71033554,268.31545931)
\curveto(270.58033163,268.3154498)(270.47533174,268.30544981)(270.39533554,268.28545931)
\curveto(270.33533188,268.27544984)(270.28033193,268.27044984)(270.23033554,268.27045931)
\curveto(270.18033203,268.27044984)(270.13033208,268.26044985)(270.08033554,268.24045931)
\curveto(269.98033223,268.20044991)(269.88533233,268.16044995)(269.79533554,268.12045931)
\curveto(269.7153325,268.08045003)(269.63533258,268.03545008)(269.55533554,267.98545931)
\curveto(269.52533269,267.96545015)(269.49533272,267.94045017)(269.46533554,267.91045931)
\curveto(269.44533277,267.88045023)(269.42033279,267.85545026)(269.39033554,267.83545931)
\lineto(269.31533554,267.76045931)
\curveto(269.28533293,267.74045037)(269.26033295,267.72045039)(269.24033554,267.70045931)
\lineto(269.09033554,267.49045931)
\curveto(269.05033316,267.43045068)(269.00533321,267.36545075)(268.95533554,267.29545931)
\curveto(268.89533332,267.20545091)(268.84533337,267.10045101)(268.80533554,266.98045931)
\curveto(268.77533344,266.87045124)(268.74033347,266.76045135)(268.70033554,266.65045931)
\curveto(268.66033355,266.54045157)(268.63533358,266.39545172)(268.62533554,266.21545931)
\curveto(268.6153336,266.04545207)(268.58533363,265.92045219)(268.53533554,265.84045931)
\curveto(268.48533373,265.76045235)(268.4103338,265.7154524)(268.31033554,265.70545931)
\curveto(268.210334,265.69545242)(268.10033411,265.69045242)(267.98033554,265.69045931)
\curveto(267.94033427,265.69045242)(267.90033431,265.68545243)(267.86033554,265.67545931)
\curveto(267.82033439,265.67545244)(267.78533443,265.68045243)(267.75533554,265.69045931)
\curveto(267.70533451,265.7104524)(267.65533456,265.72045239)(267.60533554,265.72045931)
\curveto(267.56533465,265.72045239)(267.52533469,265.73045238)(267.48533554,265.75045931)
\curveto(267.39533482,265.8104523)(267.35033486,265.94545217)(267.35033554,266.15545931)
\lineto(267.35033554,266.27545931)
\curveto(267.36033485,266.33545178)(267.36533485,266.39545172)(267.36533554,266.45545931)
\curveto(267.37533484,266.52545159)(267.38533483,266.59045152)(267.39533554,266.65045931)
\curveto(267.4153348,266.76045135)(267.43533478,266.86045125)(267.45533554,266.95045931)
\curveto(267.47533474,267.05045106)(267.50533471,267.14545097)(267.54533554,267.23545931)
\curveto(267.56533465,267.30545081)(267.58533463,267.36545075)(267.60533554,267.41545931)
\lineto(267.66533554,267.59545931)
\curveto(267.78533443,267.85545026)(267.94033427,268.10045001)(268.13033554,268.33045931)
\curveto(268.33033388,268.56044955)(268.54533367,268.74544937)(268.77533554,268.88545931)
\curveto(268.88533333,268.96544915)(269.00033321,269.03044908)(269.12033554,269.08045931)
\lineto(269.51033554,269.23045931)
\curveto(269.62033259,269.28044883)(269.73533248,269.3104488)(269.85533554,269.32045931)
\curveto(269.97533224,269.34044877)(270.10033211,269.36544875)(270.23033554,269.39545931)
\curveto(270.30033191,269.39544872)(270.36533185,269.39544872)(270.42533554,269.39545931)
\curveto(270.48533173,269.40544871)(270.55033166,269.4154487)(270.62033554,269.42545931)
}
}
{
\newrgbcolor{curcolor}{0 0 0}
\pscustom[linestyle=none,fillstyle=solid,fillcolor=curcolor]
{
\newpath
\moveto(276.67494491,260.45545931)
\lineto(276.97494491,260.45545931)
\curveto(277.08494285,260.46545765)(277.18994275,260.46545765)(277.28994491,260.45545931)
\curveto(277.39994254,260.45545766)(277.49994244,260.44545767)(277.58994491,260.42545931)
\curveto(277.67994226,260.4154577)(277.74994219,260.39045772)(277.79994491,260.35045931)
\curveto(277.81994212,260.33045778)(277.8349421,260.30045781)(277.84494491,260.26045931)
\curveto(277.86494207,260.22045789)(277.88494205,260.17545794)(277.90494491,260.12545931)
\lineto(277.90494491,260.05045931)
\curveto(277.91494202,260.00045811)(277.91494202,259.94545817)(277.90494491,259.88545931)
\lineto(277.90494491,259.73545931)
\lineto(277.90494491,259.25545931)
\curveto(277.90494203,259.08545903)(277.86494207,258.96545915)(277.78494491,258.89545931)
\curveto(277.71494222,258.84545927)(277.62494231,258.82045929)(277.51494491,258.82045931)
\lineto(277.18494491,258.82045931)
\lineto(276.73494491,258.82045931)
\curveto(276.58494335,258.82045929)(276.46994347,258.85045926)(276.38994491,258.91045931)
\curveto(276.34994359,258.94045917)(276.31994362,258.99045912)(276.29994491,259.06045931)
\curveto(276.27994366,259.14045897)(276.26494367,259.22545889)(276.25494491,259.31545931)
\lineto(276.25494491,259.60045931)
\curveto(276.26494367,259.70045841)(276.26994367,259.78545833)(276.26994491,259.85545931)
\lineto(276.26994491,260.05045931)
\curveto(276.26994367,260.110458)(276.27994366,260.16545795)(276.29994491,260.21545931)
\curveto(276.3399436,260.32545779)(276.40994353,260.39545772)(276.50994491,260.42545931)
\curveto(276.5399434,260.42545769)(276.59494334,260.43545768)(276.67494491,260.45545931)
}
}
{
\newrgbcolor{curcolor}{0 0 0}
\pscustom[linestyle=none,fillstyle=solid,fillcolor=curcolor]
{
\newpath
\moveto(283.94010116,269.42545931)
\curveto(284.04009631,269.42544869)(284.13509621,269.4154487)(284.22510116,269.39545931)
\curveto(284.31509603,269.38544873)(284.38009597,269.35544876)(284.42010116,269.30545931)
\curveto(284.48009587,269.22544889)(284.51009584,269.12044899)(284.51010116,268.99045931)
\lineto(284.51010116,268.60045931)
\lineto(284.51010116,267.10045931)
\lineto(284.51010116,260.71045931)
\lineto(284.51010116,259.54045931)
\lineto(284.51010116,259.22545931)
\curveto(284.52009583,259.12545899)(284.50509584,259.04545907)(284.46510116,258.98545931)
\curveto(284.41509593,258.90545921)(284.34009601,258.85545926)(284.24010116,258.83545931)
\curveto(284.1500962,258.82545929)(284.04009631,258.82045929)(283.91010116,258.82045931)
\lineto(283.68510116,258.82045931)
\curveto(283.60509674,258.84045927)(283.53509681,258.85545926)(283.47510116,258.86545931)
\curveto(283.41509693,258.88545923)(283.36509698,258.92545919)(283.32510116,258.98545931)
\curveto(283.28509706,259.04545907)(283.26509708,259.12045899)(283.26510116,259.21045931)
\lineto(283.26510116,259.51045931)
\lineto(283.26510116,260.60545931)
\lineto(283.26510116,265.94545931)
\curveto(283.2450971,266.03545208)(283.23009712,266.110452)(283.22010116,266.17045931)
\curveto(283.22009713,266.24045187)(283.19009716,266.30045181)(283.13010116,266.35045931)
\curveto(283.06009729,266.40045171)(282.97009738,266.42545169)(282.86010116,266.42545931)
\curveto(282.76009759,266.43545168)(282.6500977,266.44045167)(282.53010116,266.44045931)
\lineto(281.39010116,266.44045931)
\lineto(280.89510116,266.44045931)
\curveto(280.73509961,266.45045166)(280.62509972,266.5104516)(280.56510116,266.62045931)
\curveto(280.5450998,266.65045146)(280.53509981,266.68045143)(280.53510116,266.71045931)
\curveto(280.53509981,266.75045136)(280.53009982,266.79545132)(280.52010116,266.84545931)
\curveto(280.50009985,266.96545115)(280.50509984,267.07545104)(280.53510116,267.17545931)
\curveto(280.57509977,267.27545084)(280.63009972,267.34545077)(280.70010116,267.38545931)
\curveto(280.78009957,267.43545068)(280.90009945,267.46045065)(281.06010116,267.46045931)
\curveto(281.22009913,267.46045065)(281.35509899,267.47545064)(281.46510116,267.50545931)
\curveto(281.51509883,267.5154506)(281.57009878,267.52045059)(281.63010116,267.52045931)
\curveto(281.69009866,267.53045058)(281.7500986,267.54545057)(281.81010116,267.56545931)
\curveto(281.96009839,267.6154505)(282.10509824,267.66545045)(282.24510116,267.71545931)
\curveto(282.38509796,267.77545034)(282.52009783,267.84545027)(282.65010116,267.92545931)
\curveto(282.79009756,268.0154501)(282.91009744,268.12044999)(283.01010116,268.24045931)
\curveto(283.11009724,268.36044975)(283.20509714,268.49044962)(283.29510116,268.63045931)
\curveto(283.35509699,268.73044938)(283.40009695,268.84044927)(283.43010116,268.96045931)
\curveto(283.47009688,269.08044903)(283.52009683,269.18544893)(283.58010116,269.27545931)
\curveto(283.63009672,269.33544878)(283.70009665,269.37544874)(283.79010116,269.39545931)
\curveto(283.81009654,269.40544871)(283.83509651,269.4104487)(283.86510116,269.41045931)
\curveto(283.89509645,269.4104487)(283.92009643,269.4154487)(283.94010116,269.42545931)
}
}
{
\newrgbcolor{curcolor}{0 0 0}
\pscustom[linestyle=none,fillstyle=solid,fillcolor=curcolor]
{
\newpath
\moveto(298.03471054,267.34045931)
\curveto(297.83470024,267.05045106)(297.62470045,266.76545135)(297.40471054,266.48545931)
\curveto(297.19470088,266.20545191)(296.98970108,265.92045219)(296.78971054,265.63045931)
\curveto(296.18970188,264.78045333)(295.58470249,263.94045417)(294.97471054,263.11045931)
\curveto(294.36470371,262.29045582)(293.75970431,261.45545666)(293.15971054,260.60545931)
\lineto(292.64971054,259.88545931)
\lineto(292.13971054,259.19545931)
\curveto(292.05970601,259.08545903)(291.97970609,258.97045914)(291.89971054,258.85045931)
\curveto(291.81970625,258.73045938)(291.72470635,258.63545948)(291.61471054,258.56545931)
\curveto(291.5747065,258.54545957)(291.50970656,258.53045958)(291.41971054,258.52045931)
\curveto(291.33970673,258.50045961)(291.24970682,258.49045962)(291.14971054,258.49045931)
\curveto(291.04970702,258.49045962)(290.95470712,258.49545962)(290.86471054,258.50545931)
\curveto(290.78470729,258.5154596)(290.72470735,258.53545958)(290.68471054,258.56545931)
\curveto(290.65470742,258.58545953)(290.62970744,258.62045949)(290.60971054,258.67045931)
\curveto(290.59970747,258.7104594)(290.60470747,258.75545936)(290.62471054,258.80545931)
\curveto(290.66470741,258.88545923)(290.70970736,258.96045915)(290.75971054,259.03045931)
\curveto(290.81970725,259.110459)(290.8747072,259.19045892)(290.92471054,259.27045931)
\curveto(291.16470691,259.6104585)(291.40970666,259.94545817)(291.65971054,260.27545931)
\curveto(291.90970616,260.60545751)(292.14970592,260.94045717)(292.37971054,261.28045931)
\curveto(292.53970553,261.50045661)(292.69970537,261.7154564)(292.85971054,261.92545931)
\curveto(293.01970505,262.13545598)(293.17970489,262.35045576)(293.33971054,262.57045931)
\curveto(293.69970437,263.09045502)(294.06470401,263.60045451)(294.43471054,264.10045931)
\curveto(294.80470327,264.60045351)(295.1747029,265.110453)(295.54471054,265.63045931)
\curveto(295.68470239,265.83045228)(295.82470225,266.02545209)(295.96471054,266.21545931)
\curveto(296.11470196,266.40545171)(296.25970181,266.60045151)(296.39971054,266.80045931)
\curveto(296.60970146,267.10045101)(296.82470125,267.40045071)(297.04471054,267.70045931)
\lineto(297.70471054,268.60045931)
\lineto(297.88471054,268.87045931)
\lineto(298.09471054,269.14045931)
\lineto(298.21471054,269.32045931)
\curveto(298.26469981,269.38044873)(298.31469976,269.43544868)(298.36471054,269.48545931)
\curveto(298.43469964,269.53544858)(298.50969956,269.57044854)(298.58971054,269.59045931)
\curveto(298.60969946,269.60044851)(298.63469944,269.60044851)(298.66471054,269.59045931)
\curveto(298.70469937,269.59044852)(298.73469934,269.60044851)(298.75471054,269.62045931)
\curveto(298.8746992,269.62044849)(299.00969906,269.6154485)(299.15971054,269.60545931)
\curveto(299.30969876,269.60544851)(299.39969867,269.56044855)(299.42971054,269.47045931)
\curveto(299.44969862,269.44044867)(299.45469862,269.40544871)(299.44471054,269.36545931)
\curveto(299.43469864,269.32544879)(299.41969865,269.29544882)(299.39971054,269.27545931)
\curveto(299.35969871,269.19544892)(299.31969875,269.12544899)(299.27971054,269.06545931)
\curveto(299.23969883,269.00544911)(299.19469888,268.94544917)(299.14471054,268.88545931)
\lineto(298.57471054,268.10545931)
\curveto(298.39469968,267.85545026)(298.21469986,267.60045051)(298.03471054,267.34045931)
\moveto(291.17971054,263.44045931)
\curveto(291.12970694,263.46045465)(291.07970699,263.46545465)(291.02971054,263.45545931)
\curveto(290.97970709,263.44545467)(290.92970714,263.45045466)(290.87971054,263.47045931)
\curveto(290.7697073,263.49045462)(290.66470741,263.5104546)(290.56471054,263.53045931)
\curveto(290.4747076,263.56045455)(290.37970769,263.60045451)(290.27971054,263.65045931)
\curveto(289.94970812,263.79045432)(289.69470838,263.98545413)(289.51471054,264.23545931)
\curveto(289.33470874,264.49545362)(289.18970888,264.80545331)(289.07971054,265.16545931)
\curveto(289.04970902,265.24545287)(289.02970904,265.32545279)(289.01971054,265.40545931)
\curveto(289.00970906,265.49545262)(288.99470908,265.58045253)(288.97471054,265.66045931)
\curveto(288.96470911,265.7104524)(288.95970911,265.77545234)(288.95971054,265.85545931)
\curveto(288.94970912,265.88545223)(288.94470913,265.9154522)(288.94471054,265.94545931)
\curveto(288.94470913,265.98545213)(288.93970913,266.02045209)(288.92971054,266.05045931)
\lineto(288.92971054,266.20045931)
\curveto(288.91970915,266.25045186)(288.91470916,266.3104518)(288.91471054,266.38045931)
\curveto(288.91470916,266.46045165)(288.91970915,266.52545159)(288.92971054,266.57545931)
\lineto(288.92971054,266.74045931)
\curveto(288.94970912,266.79045132)(288.95470912,266.83545128)(288.94471054,266.87545931)
\curveto(288.94470913,266.92545119)(288.94970912,266.97045114)(288.95971054,267.01045931)
\curveto(288.9697091,267.05045106)(288.9747091,267.08545103)(288.97471054,267.11545931)
\curveto(288.9747091,267.15545096)(288.97970909,267.19545092)(288.98971054,267.23545931)
\curveto(289.01970905,267.34545077)(289.03970903,267.45545066)(289.04971054,267.56545931)
\curveto(289.069709,267.68545043)(289.10470897,267.80045031)(289.15471054,267.91045931)
\curveto(289.29470878,268.25044986)(289.45470862,268.52544959)(289.63471054,268.73545931)
\curveto(289.82470825,268.95544916)(290.09470798,269.13544898)(290.44471054,269.27545931)
\curveto(290.52470755,269.30544881)(290.60970746,269.32544879)(290.69971054,269.33545931)
\curveto(290.78970728,269.35544876)(290.88470719,269.37544874)(290.98471054,269.39545931)
\curveto(291.01470706,269.40544871)(291.069707,269.40544871)(291.14971054,269.39545931)
\curveto(291.22970684,269.39544872)(291.27970679,269.40544871)(291.29971054,269.42545931)
\curveto(291.85970621,269.43544868)(292.30970576,269.32544879)(292.64971054,269.09545931)
\curveto(292.99970507,268.86544925)(293.25970481,268.56044955)(293.42971054,268.18045931)
\curveto(293.4697046,268.09045002)(293.50470457,267.99545012)(293.53471054,267.89545931)
\curveto(293.56470451,267.79545032)(293.58970448,267.69545042)(293.60971054,267.59545931)
\curveto(293.62970444,267.56545055)(293.63470444,267.53545058)(293.62471054,267.50545931)
\curveto(293.62470445,267.47545064)(293.62970444,267.44545067)(293.63971054,267.41545931)
\curveto(293.6697044,267.30545081)(293.68970438,267.18045093)(293.69971054,267.04045931)
\curveto(293.70970436,266.9104512)(293.71970435,266.77545134)(293.72971054,266.63545931)
\lineto(293.72971054,266.47045931)
\curveto(293.73970433,266.4104517)(293.73970433,266.35545176)(293.72971054,266.30545931)
\curveto(293.71970435,266.25545186)(293.71470436,266.20545191)(293.71471054,266.15545931)
\lineto(293.71471054,266.02045931)
\curveto(293.70470437,265.98045213)(293.69970437,265.94045217)(293.69971054,265.90045931)
\curveto(293.70970436,265.86045225)(293.70470437,265.8154523)(293.68471054,265.76545931)
\curveto(293.66470441,265.65545246)(293.64470443,265.55045256)(293.62471054,265.45045931)
\curveto(293.61470446,265.35045276)(293.59470448,265.25045286)(293.56471054,265.15045931)
\curveto(293.43470464,264.79045332)(293.2697048,264.47545364)(293.06971054,264.20545931)
\curveto(292.8697052,263.93545418)(292.59470548,263.73045438)(292.24471054,263.59045931)
\curveto(292.16470591,263.56045455)(292.07970599,263.53545458)(291.98971054,263.51545931)
\lineto(291.71971054,263.45545931)
\curveto(291.6697064,263.44545467)(291.62470645,263.44045467)(291.58471054,263.44045931)
\curveto(291.54470653,263.45045466)(291.50470657,263.45045466)(291.46471054,263.44045931)
\curveto(291.36470671,263.42045469)(291.2697068,263.42045469)(291.17971054,263.44045931)
\moveto(290.33971054,264.83545931)
\curveto(290.37970769,264.76545335)(290.41970765,264.70045341)(290.45971054,264.64045931)
\curveto(290.49970757,264.59045352)(290.54970752,264.54045357)(290.60971054,264.49045931)
\lineto(290.75971054,264.37045931)
\curveto(290.81970725,264.34045377)(290.88470719,264.3154538)(290.95471054,264.29545931)
\curveto(290.99470708,264.27545384)(291.02970704,264.26545385)(291.05971054,264.26545931)
\curveto(291.09970697,264.27545384)(291.13970693,264.27045384)(291.17971054,264.25045931)
\curveto(291.20970686,264.25045386)(291.24970682,264.24545387)(291.29971054,264.23545931)
\curveto(291.34970672,264.23545388)(291.38970668,264.24045387)(291.41971054,264.25045931)
\lineto(291.64471054,264.29545931)
\curveto(291.89470618,264.37545374)(292.07970599,264.50045361)(292.19971054,264.67045931)
\curveto(292.27970579,264.77045334)(292.34970572,264.90045321)(292.40971054,265.06045931)
\curveto(292.48970558,265.24045287)(292.54970552,265.46545265)(292.58971054,265.73545931)
\curveto(292.62970544,266.0154521)(292.64470543,266.29545182)(292.63471054,266.57545931)
\curveto(292.62470545,266.86545125)(292.59470548,267.14045097)(292.54471054,267.40045931)
\curveto(292.49470558,267.66045045)(292.41970565,267.87045024)(292.31971054,268.03045931)
\curveto(292.19970587,268.23044988)(292.04970602,268.38044973)(291.86971054,268.48045931)
\curveto(291.78970628,268.53044958)(291.69970637,268.56044955)(291.59971054,268.57045931)
\curveto(291.49970657,268.59044952)(291.39470668,268.60044951)(291.28471054,268.60045931)
\curveto(291.26470681,268.59044952)(291.23970683,268.58544953)(291.20971054,268.58545931)
\curveto(291.18970688,268.59544952)(291.1697069,268.59544952)(291.14971054,268.58545931)
\curveto(291.09970697,268.57544954)(291.05470702,268.56544955)(291.01471054,268.55545931)
\curveto(290.9747071,268.55544956)(290.93470714,268.54544957)(290.89471054,268.52545931)
\curveto(290.71470736,268.44544967)(290.56470751,268.32544979)(290.44471054,268.16545931)
\curveto(290.33470774,268.00545011)(290.24470783,267.82545029)(290.17471054,267.62545931)
\curveto(290.11470796,267.43545068)(290.069708,267.2104509)(290.03971054,266.95045931)
\curveto(290.01970805,266.69045142)(290.01470806,266.42545169)(290.02471054,266.15545931)
\curveto(290.03470804,265.89545222)(290.06470801,265.64545247)(290.11471054,265.40545931)
\curveto(290.1747079,265.17545294)(290.24970782,264.98545313)(290.33971054,264.83545931)
\moveto(301.13971054,261.85045931)
\curveto(301.14969692,261.80045631)(301.15469692,261.7104564)(301.15471054,261.58045931)
\curveto(301.15469692,261.45045666)(301.14469693,261.36045675)(301.12471054,261.31045931)
\curveto(301.10469697,261.26045685)(301.09969697,261.20545691)(301.10971054,261.14545931)
\curveto(301.11969695,261.09545702)(301.11969695,261.04545707)(301.10971054,260.99545931)
\curveto(301.069697,260.85545726)(301.03969703,260.72045739)(301.01971054,260.59045931)
\curveto(301.00969706,260.46045765)(300.97969709,260.34045777)(300.92971054,260.23045931)
\curveto(300.78969728,259.88045823)(300.62469745,259.58545853)(300.43471054,259.34545931)
\curveto(300.24469783,259.115459)(299.9746981,258.93045918)(299.62471054,258.79045931)
\curveto(299.54469853,258.76045935)(299.45969861,258.74045937)(299.36971054,258.73045931)
\curveto(299.27969879,258.7104594)(299.19469888,258.69045942)(299.11471054,258.67045931)
\curveto(299.06469901,258.66045945)(299.01469906,258.65545946)(298.96471054,258.65545931)
\curveto(298.91469916,258.65545946)(298.86469921,258.65045946)(298.81471054,258.64045931)
\curveto(298.78469929,258.63045948)(298.73469934,258.63045948)(298.66471054,258.64045931)
\curveto(298.59469948,258.64045947)(298.54469953,258.64545947)(298.51471054,258.65545931)
\curveto(298.45469962,258.67545944)(298.39469968,258.68545943)(298.33471054,258.68545931)
\curveto(298.28469979,258.67545944)(298.23469984,258.68045943)(298.18471054,258.70045931)
\curveto(298.09469998,258.72045939)(298.00470007,258.74545937)(297.91471054,258.77545931)
\curveto(297.83470024,258.79545932)(297.75470032,258.82545929)(297.67471054,258.86545931)
\curveto(297.35470072,259.00545911)(297.10470097,259.20045891)(296.92471054,259.45045931)
\curveto(296.74470133,259.7104584)(296.59470148,260.0154581)(296.47471054,260.36545931)
\curveto(296.45470162,260.44545767)(296.43970163,260.53045758)(296.42971054,260.62045931)
\curveto(296.41970165,260.7104574)(296.40470167,260.79545732)(296.38471054,260.87545931)
\curveto(296.3747017,260.90545721)(296.3697017,260.93545718)(296.36971054,260.96545931)
\lineto(296.36971054,261.07045931)
\curveto(296.34970172,261.15045696)(296.33970173,261.23045688)(296.33971054,261.31045931)
\lineto(296.33971054,261.44545931)
\curveto(296.31970175,261.54545657)(296.31970175,261.64545647)(296.33971054,261.74545931)
\lineto(296.33971054,261.92545931)
\curveto(296.34970172,261.97545614)(296.35470172,262.02045609)(296.35471054,262.06045931)
\curveto(296.35470172,262.110456)(296.35970171,262.15545596)(296.36971054,262.19545931)
\curveto(296.37970169,262.23545588)(296.38470169,262.27045584)(296.38471054,262.30045931)
\curveto(296.38470169,262.34045577)(296.38970168,262.38045573)(296.39971054,262.42045931)
\lineto(296.45971054,262.75045931)
\curveto(296.47970159,262.87045524)(296.50970156,262.98045513)(296.54971054,263.08045931)
\curveto(296.68970138,263.4104547)(296.84970122,263.68545443)(297.02971054,263.90545931)
\curveto(297.21970085,264.13545398)(297.47970059,264.32045379)(297.80971054,264.46045931)
\curveto(297.88970018,264.50045361)(297.9747001,264.52545359)(298.06471054,264.53545931)
\lineto(298.36471054,264.59545931)
\lineto(298.49971054,264.59545931)
\curveto(298.54969952,264.60545351)(298.59969947,264.6104535)(298.64971054,264.61045931)
\curveto(299.21969885,264.63045348)(299.67969839,264.52545359)(300.02971054,264.29545931)
\curveto(300.38969768,264.07545404)(300.65469742,263.77545434)(300.82471054,263.39545931)
\curveto(300.8746972,263.29545482)(300.91469716,263.19545492)(300.94471054,263.09545931)
\curveto(300.9746971,262.99545512)(301.00469707,262.89045522)(301.03471054,262.78045931)
\curveto(301.04469703,262.74045537)(301.04969702,262.70545541)(301.04971054,262.67545931)
\curveto(301.04969702,262.65545546)(301.05469702,262.62545549)(301.06471054,262.58545931)
\curveto(301.08469699,262.5154556)(301.09469698,262.44045567)(301.09471054,262.36045931)
\curveto(301.09469698,262.28045583)(301.10469697,262.20045591)(301.12471054,262.12045931)
\curveto(301.12469695,262.07045604)(301.12469695,262.02545609)(301.12471054,261.98545931)
\curveto(301.12469695,261.94545617)(301.12969694,261.90045621)(301.13971054,261.85045931)
\moveto(300.02971054,261.41545931)
\curveto(300.03969803,261.46545665)(300.04469803,261.54045657)(300.04471054,261.64045931)
\curveto(300.05469802,261.74045637)(300.04969802,261.8154563)(300.02971054,261.86545931)
\curveto(300.00969806,261.92545619)(300.00469807,261.98045613)(300.01471054,262.03045931)
\curveto(300.03469804,262.09045602)(300.03469804,262.15045596)(300.01471054,262.21045931)
\curveto(300.00469807,262.24045587)(299.99969807,262.27545584)(299.99971054,262.31545931)
\curveto(299.99969807,262.35545576)(299.99469808,262.39545572)(299.98471054,262.43545931)
\curveto(299.96469811,262.5154556)(299.94469813,262.59045552)(299.92471054,262.66045931)
\curveto(299.91469816,262.74045537)(299.89969817,262.82045529)(299.87971054,262.90045931)
\curveto(299.84969822,262.96045515)(299.82469825,263.02045509)(299.80471054,263.08045931)
\curveto(299.78469829,263.14045497)(299.75469832,263.20045491)(299.71471054,263.26045931)
\curveto(299.61469846,263.43045468)(299.48469859,263.56545455)(299.32471054,263.66545931)
\curveto(299.24469883,263.7154544)(299.14969892,263.75045436)(299.03971054,263.77045931)
\curveto(298.92969914,263.79045432)(298.80469927,263.80045431)(298.66471054,263.80045931)
\curveto(298.64469943,263.79045432)(298.61969945,263.78545433)(298.58971054,263.78545931)
\curveto(298.55969951,263.79545432)(298.52969954,263.79545432)(298.49971054,263.78545931)
\lineto(298.34971054,263.72545931)
\curveto(298.29969977,263.7154544)(298.25469982,263.70045441)(298.21471054,263.68045931)
\curveto(298.02470005,263.57045454)(297.87970019,263.42545469)(297.77971054,263.24545931)
\curveto(297.68970038,263.06545505)(297.60970046,262.86045525)(297.53971054,262.63045931)
\curveto(297.49970057,262.50045561)(297.47970059,262.36545575)(297.47971054,262.22545931)
\curveto(297.47970059,262.09545602)(297.4697006,261.95045616)(297.44971054,261.79045931)
\curveto(297.43970063,261.74045637)(297.42970064,261.68045643)(297.41971054,261.61045931)
\curveto(297.41970065,261.54045657)(297.42970064,261.48045663)(297.44971054,261.43045931)
\lineto(297.44971054,261.26545931)
\lineto(297.44971054,261.08545931)
\curveto(297.45970061,261.03545708)(297.4697006,260.98045713)(297.47971054,260.92045931)
\curveto(297.48970058,260.87045724)(297.49470058,260.8154573)(297.49471054,260.75545931)
\curveto(297.50470057,260.69545742)(297.51970055,260.64045747)(297.53971054,260.59045931)
\curveto(297.58970048,260.40045771)(297.64970042,260.22545789)(297.71971054,260.06545931)
\curveto(297.78970028,259.90545821)(297.89470018,259.77545834)(298.03471054,259.67545931)
\curveto(298.16469991,259.57545854)(298.30469977,259.50545861)(298.45471054,259.46545931)
\curveto(298.48469959,259.45545866)(298.50969956,259.45045866)(298.52971054,259.45045931)
\curveto(298.55969951,259.46045865)(298.58969948,259.46045865)(298.61971054,259.45045931)
\curveto(298.63969943,259.45045866)(298.6696994,259.44545867)(298.70971054,259.43545931)
\curveto(298.74969932,259.43545868)(298.78469929,259.44045867)(298.81471054,259.45045931)
\curveto(298.85469922,259.46045865)(298.89469918,259.46545865)(298.93471054,259.46545931)
\curveto(298.9746991,259.46545865)(299.01469906,259.47545864)(299.05471054,259.49545931)
\curveto(299.29469878,259.57545854)(299.48969858,259.7104584)(299.63971054,259.90045931)
\curveto(299.75969831,260.08045803)(299.84969822,260.28545783)(299.90971054,260.51545931)
\curveto(299.92969814,260.58545753)(299.94469813,260.65545746)(299.95471054,260.72545931)
\curveto(299.96469811,260.80545731)(299.97969809,260.88545723)(299.99971054,260.96545931)
\curveto(299.99969807,261.02545709)(300.00469807,261.07045704)(300.01471054,261.10045931)
\curveto(300.01469806,261.12045699)(300.01469806,261.14545697)(300.01471054,261.17545931)
\curveto(300.01469806,261.2154569)(300.01969805,261.24545687)(300.02971054,261.26545931)
\lineto(300.02971054,261.41545931)
}
}
{
\newrgbcolor{curcolor}{0.80000001 0.80000001 0.80000001}
\pscustom[linestyle=none,fillstyle=solid,fillcolor=curcolor]
{
\newpath
\moveto(727.74397797,158.46993191)
\curveto(727.55926691,153.66536842)(727.13890732,148.87276262)(726.48441561,144.10940361)
\lineto(580.87528661,164.11628684)
\closepath
}
}
{
\newrgbcolor{curcolor}{0.90196079 0.90196079 0.90196079}
\pscustom[linestyle=none,fillstyle=solid,fillcolor=curcolor]
{
\newpath
\moveto(580.87528949,311.09347495)
\curveto(662.04854906,311.09347335)(727.85247631,245.28954352)(727.85247472,164.11628395)
\curveto(727.85247468,162.19965142)(727.81498434,160.28320225)(727.74002519,158.3680361)
\lineto(580.87528661,164.11628684)
\closepath
}
}
{
\newrgbcolor{curcolor}{0.7019608 0.7019608 0.7019608}
\pscustom[linestyle=none,fillstyle=solid,fillcolor=curcolor]
{
\newpath
\moveto(726.51285122,144.31745863)
\curveto(725.24067859,134.9595248)(723.0695318,125.7459883)(720.02952451,116.80464028)
\lineto(580.87528661,164.11628684)
\closepath
}
}
{
\newrgbcolor{curcolor}{0.60000002 0.60000002 0.60000002}
\pscustom[linestyle=none,fillstyle=solid,fillcolor=curcolor]
{
\newpath
\moveto(720.08261442,116.96107811)
\curveto(699.86987863,57.29087386)(643.87598866,17.13910041)(580.87529055,17.13909873)
\lineto(580.87528661,164.11628684)
\closepath
}
}
{
\newrgbcolor{curcolor}{0.50196081 0.50196081 0.50196081}
\pscustom[linestyle=none,fillstyle=solid,fillcolor=curcolor]
{
\newpath
\moveto(580.87529055,17.13909873)
\curveto(566.59124939,17.13909834)(552.38349347,19.22134082)(538.7001603,23.32014859)
\lineto(580.87528661,164.11628684)
\closepath
}
}
{
\newrgbcolor{curcolor}{0.40000001 0.40000001 0.40000001}
\pscustom[linestyle=none,fillstyle=solid,fillcolor=curcolor]
{
\newpath
\moveto(538.79940388,23.29045853)
\curveto(461.02344653,46.5283269)(416.81158994,128.41621221)(440.04945831,206.19216957)
\curveto(458.64794775,268.44036287)(515.90805638,311.09347622)(580.87528949,311.09347495)
\lineto(580.87528661,164.11628684)
\closepath
}
}
{
\newrgbcolor{curcolor}{0 0 0}
\pscustom[linestyle=none,fillstyle=solid,fillcolor=curcolor]
{
\newpath
\moveto(312.07312728,186.59297029)
\curveto(312.17312243,186.59295967)(312.26812233,186.58295968)(312.35812728,186.56297029)
\curveto(312.44812215,186.55295971)(312.51312209,186.52295974)(312.55312728,186.47297029)
\curveto(312.61312199,186.39295987)(312.64312196,186.28795998)(312.64312728,186.15797029)
\lineto(312.64312728,185.76797029)
\lineto(312.64312728,184.26797029)
\lineto(312.64312728,177.87797029)
\lineto(312.64312728,176.70797029)
\lineto(312.64312728,176.39297029)
\curveto(312.65312195,176.29296997)(312.63812196,176.21297005)(312.59812728,176.15297029)
\curveto(312.54812205,176.07297019)(312.47312213,176.02297024)(312.37312728,176.00297029)
\curveto(312.28312232,175.99297027)(312.17312243,175.98797028)(312.04312728,175.98797029)
\lineto(311.81812728,175.98797029)
\curveto(311.73812286,176.00797026)(311.66812293,176.02297024)(311.60812728,176.03297029)
\curveto(311.54812305,176.05297021)(311.4981231,176.09297017)(311.45812728,176.15297029)
\curveto(311.41812318,176.21297005)(311.3981232,176.28796998)(311.39812728,176.37797029)
\lineto(311.39812728,176.67797029)
\lineto(311.39812728,177.77297029)
\lineto(311.39812728,183.11297029)
\curveto(311.37812322,183.20296306)(311.36312324,183.27796299)(311.35312728,183.33797029)
\curveto(311.35312325,183.40796286)(311.32312328,183.4679628)(311.26312728,183.51797029)
\curveto(311.19312341,183.5679627)(311.1031235,183.59296267)(310.99312728,183.59297029)
\curveto(310.89312371,183.60296266)(310.78312382,183.60796266)(310.66312728,183.60797029)
\lineto(309.52312728,183.60797029)
\lineto(309.02812728,183.60797029)
\curveto(308.86812573,183.61796265)(308.75812584,183.67796259)(308.69812728,183.78797029)
\curveto(308.67812592,183.81796245)(308.66812593,183.84796242)(308.66812728,183.87797029)
\curveto(308.66812593,183.91796235)(308.66312594,183.9629623)(308.65312728,184.01297029)
\curveto(308.63312597,184.13296213)(308.63812596,184.24296202)(308.66812728,184.34297029)
\curveto(308.70812589,184.44296182)(308.76312584,184.51296175)(308.83312728,184.55297029)
\curveto(308.91312569,184.60296166)(309.03312557,184.62796164)(309.19312728,184.62797029)
\curveto(309.35312525,184.62796164)(309.48812511,184.64296162)(309.59812728,184.67297029)
\curveto(309.64812495,184.68296158)(309.7031249,184.68796158)(309.76312728,184.68797029)
\curveto(309.82312478,184.69796157)(309.88312472,184.71296155)(309.94312728,184.73297029)
\curveto(310.09312451,184.78296148)(310.23812436,184.83296143)(310.37812728,184.88297029)
\curveto(310.51812408,184.94296132)(310.65312395,185.01296125)(310.78312728,185.09297029)
\curveto(310.92312368,185.18296108)(311.04312356,185.28796098)(311.14312728,185.40797029)
\curveto(311.24312336,185.52796074)(311.33812326,185.65796061)(311.42812728,185.79797029)
\curveto(311.48812311,185.89796037)(311.53312307,186.00796026)(311.56312728,186.12797029)
\curveto(311.603123,186.24796002)(311.65312295,186.35295991)(311.71312728,186.44297029)
\curveto(311.76312284,186.50295976)(311.83312277,186.54295972)(311.92312728,186.56297029)
\curveto(311.94312266,186.57295969)(311.96812263,186.57795969)(311.99812728,186.57797029)
\curveto(312.02812257,186.57795969)(312.05312255,186.58295968)(312.07312728,186.59297029)
}
}
{
\newrgbcolor{curcolor}{0 0 0}
\pscustom[linestyle=none,fillstyle=solid,fillcolor=curcolor]
{
\newpath
\moveto(317.87273666,186.39797029)
\lineto(321.47273666,186.39797029)
\lineto(322.11773666,186.39797029)
\curveto(322.19773013,186.39795987)(322.27273005,186.39295987)(322.34273666,186.38297029)
\curveto(322.41272991,186.38295988)(322.47272985,186.37295989)(322.52273666,186.35297029)
\curveto(322.59272973,186.32295994)(322.64772968,186.26296)(322.68773666,186.17297029)
\curveto(322.70772962,186.14296012)(322.71772961,186.10296016)(322.71773666,186.05297029)
\lineto(322.71773666,185.91797029)
\curveto(322.7277296,185.80796046)(322.7227296,185.70296056)(322.70273666,185.60297029)
\curveto(322.69272963,185.50296076)(322.65772967,185.43296083)(322.59773666,185.39297029)
\curveto(322.50772982,185.32296094)(322.37272995,185.28796098)(322.19273666,185.28797029)
\curveto(322.01273031,185.29796097)(321.84773048,185.30296096)(321.69773666,185.30297029)
\lineto(319.70273666,185.30297029)
\lineto(319.20773666,185.30297029)
\lineto(319.07273666,185.30297029)
\curveto(319.03273329,185.30296096)(318.99273333,185.29796097)(318.95273666,185.28797029)
\lineto(318.74273666,185.28797029)
\curveto(318.63273369,185.25796101)(318.55273377,185.21796105)(318.50273666,185.16797029)
\curveto(318.45273387,185.12796114)(318.41773391,185.07296119)(318.39773666,185.00297029)
\curveto(318.37773395,184.94296132)(318.36273396,184.87296139)(318.35273666,184.79297029)
\curveto(318.34273398,184.71296155)(318.322734,184.62296164)(318.29273666,184.52297029)
\curveto(318.24273408,184.32296194)(318.20273412,184.11796215)(318.17273666,183.90797029)
\curveto(318.14273418,183.69796257)(318.10273422,183.49296277)(318.05273666,183.29297029)
\curveto(318.03273429,183.22296304)(318.0227343,183.15296311)(318.02273666,183.08297029)
\curveto(318.0227343,183.02296324)(318.01273431,182.95796331)(317.99273666,182.88797029)
\curveto(317.98273434,182.85796341)(317.97273435,182.81796345)(317.96273666,182.76797029)
\curveto(317.96273436,182.72796354)(317.96773436,182.68796358)(317.97773666,182.64797029)
\curveto(317.99773433,182.59796367)(318.0227343,182.55296371)(318.05273666,182.51297029)
\curveto(318.09273423,182.48296378)(318.15273417,182.47796379)(318.23273666,182.49797029)
\curveto(318.29273403,182.51796375)(318.35273397,182.54296372)(318.41273666,182.57297029)
\curveto(318.47273385,182.61296365)(318.53273379,182.64796362)(318.59273666,182.67797029)
\curveto(318.65273367,182.69796357)(318.70273362,182.71296355)(318.74273666,182.72297029)
\curveto(318.93273339,182.80296346)(319.13773319,182.85796341)(319.35773666,182.88797029)
\curveto(319.58773274,182.91796335)(319.81773251,182.92796334)(320.04773666,182.91797029)
\curveto(320.28773204,182.91796335)(320.51773181,182.89296337)(320.73773666,182.84297029)
\curveto(320.95773137,182.80296346)(321.15773117,182.74296352)(321.33773666,182.66297029)
\curveto(321.38773094,182.64296362)(321.43273089,182.62296364)(321.47273666,182.60297029)
\curveto(321.5227308,182.58296368)(321.57273075,182.55796371)(321.62273666,182.52797029)
\curveto(321.97273035,182.31796395)(322.25273007,182.08796418)(322.46273666,181.83797029)
\curveto(322.68272964,181.58796468)(322.87772945,181.262965)(323.04773666,180.86297029)
\curveto(323.09772923,180.75296551)(323.13272919,180.64296562)(323.15273666,180.53297029)
\curveto(323.17272915,180.42296584)(323.19772913,180.30796596)(323.22773666,180.18797029)
\curveto(323.23772909,180.15796611)(323.24272908,180.11296615)(323.24273666,180.05297029)
\curveto(323.26272906,179.99296627)(323.27272905,179.92296634)(323.27273666,179.84297029)
\curveto(323.27272905,179.77296649)(323.28272904,179.70796656)(323.30273666,179.64797029)
\lineto(323.30273666,179.48297029)
\curveto(323.31272901,179.43296683)(323.31772901,179.3629669)(323.31773666,179.27297029)
\curveto(323.31772901,179.18296708)(323.30772902,179.11296715)(323.28773666,179.06297029)
\curveto(323.26772906,179.00296726)(323.26272906,178.94296732)(323.27273666,178.88297029)
\curveto(323.28272904,178.83296743)(323.27772905,178.78296748)(323.25773666,178.73297029)
\curveto(323.21772911,178.57296769)(323.18272914,178.42296784)(323.15273666,178.28297029)
\curveto(323.1227292,178.14296812)(323.07772925,178.00796826)(323.01773666,177.87797029)
\curveto(322.85772947,177.50796876)(322.63772969,177.17296909)(322.35773666,176.87297029)
\curveto(322.07773025,176.57296969)(321.75773057,176.34296992)(321.39773666,176.18297029)
\curveto(321.2277311,176.10297016)(321.0277313,176.02797024)(320.79773666,175.95797029)
\curveto(320.68773164,175.91797035)(320.57273175,175.89297037)(320.45273666,175.88297029)
\curveto(320.33273199,175.87297039)(320.21273211,175.85297041)(320.09273666,175.82297029)
\curveto(320.04273228,175.80297046)(319.98773234,175.80297046)(319.92773666,175.82297029)
\curveto(319.86773246,175.83297043)(319.80773252,175.82797044)(319.74773666,175.80797029)
\curveto(319.64773268,175.78797048)(319.54773278,175.78797048)(319.44773666,175.80797029)
\lineto(319.31273666,175.80797029)
\curveto(319.26273306,175.82797044)(319.20273312,175.83797043)(319.13273666,175.83797029)
\curveto(319.07273325,175.82797044)(319.01773331,175.83297043)(318.96773666,175.85297029)
\curveto(318.9277334,175.8629704)(318.89273343,175.8679704)(318.86273666,175.86797029)
\curveto(318.83273349,175.8679704)(318.79773353,175.87297039)(318.75773666,175.88297029)
\lineto(318.48773666,175.94297029)
\curveto(318.39773393,175.9629703)(318.31273401,175.99297027)(318.23273666,176.03297029)
\curveto(317.89273443,176.17297009)(317.60273472,176.32796994)(317.36273666,176.49797029)
\curveto(317.1227352,176.67796959)(316.90273542,176.90796936)(316.70273666,177.18797029)
\curveto(316.55273577,177.41796885)(316.43773589,177.65796861)(316.35773666,177.90797029)
\curveto(316.33773599,177.95796831)(316.327736,178.00296826)(316.32773666,178.04297029)
\curveto(316.327736,178.09296817)(316.31773601,178.14296812)(316.29773666,178.19297029)
\curveto(316.27773605,178.25296801)(316.26273606,178.33296793)(316.25273666,178.43297029)
\curveto(316.25273607,178.53296773)(316.27273605,178.60796766)(316.31273666,178.65797029)
\curveto(316.36273596,178.73796753)(316.44273588,178.78296748)(316.55273666,178.79297029)
\curveto(316.66273566,178.80296746)(316.77773555,178.80796746)(316.89773666,178.80797029)
\lineto(317.06273666,178.80797029)
\curveto(317.1227352,178.80796746)(317.17773515,178.79796747)(317.22773666,178.77797029)
\curveto(317.31773501,178.75796751)(317.38773494,178.71796755)(317.43773666,178.65797029)
\curveto(317.50773482,178.5679677)(317.55273477,178.45796781)(317.57273666,178.32797029)
\curveto(317.60273472,178.20796806)(317.64773468,178.10296816)(317.70773666,178.01297029)
\curveto(317.89773443,177.67296859)(318.15773417,177.40296886)(318.48773666,177.20297029)
\curveto(318.58773374,177.14296912)(318.69273363,177.09296917)(318.80273666,177.05297029)
\curveto(318.9227334,177.02296924)(319.04273328,176.98796928)(319.16273666,176.94797029)
\curveto(319.33273299,176.89796937)(319.53773279,176.87796939)(319.77773666,176.88797029)
\curveto(320.0277323,176.90796936)(320.2277321,176.94296932)(320.37773666,176.99297029)
\curveto(320.74773158,177.11296915)(321.03773129,177.27296899)(321.24773666,177.47297029)
\curveto(321.46773086,177.68296858)(321.64773068,177.9629683)(321.78773666,178.31297029)
\curveto(321.83773049,178.41296785)(321.86773046,178.51796775)(321.87773666,178.62797029)
\curveto(321.89773043,178.73796753)(321.9227304,178.85296741)(321.95273666,178.97297029)
\lineto(321.95273666,179.07797029)
\curveto(321.96273036,179.11796715)(321.96773036,179.15796711)(321.96773666,179.19797029)
\curveto(321.97773035,179.22796704)(321.97773035,179.262967)(321.96773666,179.30297029)
\lineto(321.96773666,179.42297029)
\curveto(321.96773036,179.68296658)(321.93773039,179.92796634)(321.87773666,180.15797029)
\curveto(321.76773056,180.50796576)(321.61273071,180.80296546)(321.41273666,181.04297029)
\curveto(321.21273111,181.29296497)(320.95273137,181.48796478)(320.63273666,181.62797029)
\lineto(320.45273666,181.68797029)
\curveto(320.40273192,181.70796456)(320.34273198,181.72796454)(320.27273666,181.74797029)
\curveto(320.2227321,181.7679645)(320.16273216,181.77796449)(320.09273666,181.77797029)
\curveto(320.03273229,181.78796448)(319.96773236,181.80296446)(319.89773666,181.82297029)
\lineto(319.74773666,181.82297029)
\curveto(319.70773262,181.84296442)(319.65273267,181.85296441)(319.58273666,181.85297029)
\curveto(319.5227328,181.85296441)(319.46773286,181.84296442)(319.41773666,181.82297029)
\lineto(319.31273666,181.82297029)
\curveto(319.28273304,181.82296444)(319.24773308,181.81796445)(319.20773666,181.80797029)
\lineto(318.96773666,181.74797029)
\curveto(318.88773344,181.73796453)(318.80773352,181.71796455)(318.72773666,181.68797029)
\curveto(318.48773384,181.58796468)(318.25773407,181.45296481)(318.03773666,181.28297029)
\curveto(317.94773438,181.21296505)(317.86273446,181.13796513)(317.78273666,181.05797029)
\curveto(317.70273462,180.98796528)(317.60273472,180.93296533)(317.48273666,180.89297029)
\curveto(317.39273493,180.8629654)(317.25273507,180.85296541)(317.06273666,180.86297029)
\curveto(316.88273544,180.87296539)(316.76273556,180.89796537)(316.70273666,180.93797029)
\curveto(316.65273567,180.97796529)(316.61273571,181.03796523)(316.58273666,181.11797029)
\curveto(316.56273576,181.19796507)(316.56273576,181.28296498)(316.58273666,181.37297029)
\curveto(316.61273571,181.49296477)(316.63273569,181.61296465)(316.64273666,181.73297029)
\curveto(316.66273566,181.8629644)(316.68773564,181.98796428)(316.71773666,182.10797029)
\curveto(316.73773559,182.14796412)(316.74273558,182.18296408)(316.73273666,182.21297029)
\curveto(316.73273559,182.25296401)(316.74273558,182.29796397)(316.76273666,182.34797029)
\curveto(316.78273554,182.43796383)(316.79773553,182.52796374)(316.80773666,182.61797029)
\curveto(316.81773551,182.71796355)(316.83773549,182.81296345)(316.86773666,182.90297029)
\curveto(316.87773545,182.9629633)(316.88273544,183.02296324)(316.88273666,183.08297029)
\curveto(316.89273543,183.14296312)(316.90773542,183.20296306)(316.92773666,183.26297029)
\curveto(316.97773535,183.4629628)(317.01273531,183.6679626)(317.03273666,183.87797029)
\curveto(317.06273526,184.09796217)(317.10273522,184.30796196)(317.15273666,184.50797029)
\curveto(317.18273514,184.60796166)(317.20273512,184.70796156)(317.21273666,184.80797029)
\curveto(317.2227351,184.90796136)(317.23773509,185.00796126)(317.25773666,185.10797029)
\curveto(317.26773506,185.13796113)(317.27273505,185.17796109)(317.27273666,185.22797029)
\curveto(317.30273502,185.33796093)(317.322735,185.44296082)(317.33273666,185.54297029)
\curveto(317.35273497,185.65296061)(317.37773495,185.7629605)(317.40773666,185.87297029)
\curveto(317.4277349,185.95296031)(317.44273488,186.02296024)(317.45273666,186.08297029)
\curveto(317.46273486,186.15296011)(317.48773484,186.21296005)(317.52773666,186.26297029)
\curveto(317.54773478,186.29295997)(317.57773475,186.31295995)(317.61773666,186.32297029)
\curveto(317.65773467,186.34295992)(317.70273462,186.3629599)(317.75273666,186.38297029)
\curveto(317.81273451,186.38295988)(317.85273447,186.38795988)(317.87273666,186.39797029)
}
}
{
\newrgbcolor{curcolor}{0 0 0}
\pscustom[linestyle=none,fillstyle=solid,fillcolor=curcolor]
{
\newpath
\moveto(325.66734603,177.62297029)
\lineto(325.96734603,177.62297029)
\curveto(326.07734397,177.63296863)(326.18234387,177.63296863)(326.28234603,177.62297029)
\curveto(326.39234366,177.62296864)(326.49234356,177.61296865)(326.58234603,177.59297029)
\curveto(326.67234338,177.58296868)(326.74234331,177.55796871)(326.79234603,177.51797029)
\curveto(326.81234324,177.49796877)(326.82734322,177.4679688)(326.83734603,177.42797029)
\curveto(326.85734319,177.38796888)(326.87734317,177.34296892)(326.89734603,177.29297029)
\lineto(326.89734603,177.21797029)
\curveto(326.90734314,177.1679691)(326.90734314,177.11296915)(326.89734603,177.05297029)
\lineto(326.89734603,176.90297029)
\lineto(326.89734603,176.42297029)
\curveto(326.89734315,176.25297001)(326.85734319,176.13297013)(326.77734603,176.06297029)
\curveto(326.70734334,176.01297025)(326.61734343,175.98797028)(326.50734603,175.98797029)
\lineto(326.17734603,175.98797029)
\lineto(325.72734603,175.98797029)
\curveto(325.57734447,175.98797028)(325.46234459,176.01797025)(325.38234603,176.07797029)
\curveto(325.34234471,176.10797016)(325.31234474,176.15797011)(325.29234603,176.22797029)
\curveto(325.27234478,176.30796996)(325.25734479,176.39296987)(325.24734603,176.48297029)
\lineto(325.24734603,176.76797029)
\curveto(325.25734479,176.8679694)(325.26234479,176.95296931)(325.26234603,177.02297029)
\lineto(325.26234603,177.21797029)
\curveto(325.26234479,177.27796899)(325.27234478,177.33296893)(325.29234603,177.38297029)
\curveto(325.33234472,177.49296877)(325.40234465,177.5629687)(325.50234603,177.59297029)
\curveto(325.53234452,177.59296867)(325.58734446,177.60296866)(325.66734603,177.62297029)
}
}
{
\newrgbcolor{curcolor}{0 0 0}
\pscustom[linestyle=none,fillstyle=solid,fillcolor=curcolor]
{
\newpath
\moveto(332.12250228,186.59297029)
\curveto(332.81249765,186.60295966)(333.41249705,186.48295978)(333.92250228,186.23297029)
\curveto(334.44249602,185.98296028)(334.83749562,185.64796062)(335.10750228,185.22797029)
\curveto(335.1574953,185.14796112)(335.20249526,185.05796121)(335.24250228,184.95797029)
\curveto(335.28249518,184.8679614)(335.32749513,184.77296149)(335.37750228,184.67297029)
\curveto(335.41749504,184.57296169)(335.44749501,184.47296179)(335.46750228,184.37297029)
\curveto(335.48749497,184.27296199)(335.50749495,184.1679621)(335.52750228,184.05797029)
\curveto(335.54749491,184.00796226)(335.55249491,183.9629623)(335.54250228,183.92297029)
\curveto(335.53249493,183.88296238)(335.53749492,183.83796243)(335.55750228,183.78797029)
\curveto(335.56749489,183.73796253)(335.57249489,183.65296261)(335.57250228,183.53297029)
\curveto(335.57249489,183.42296284)(335.56749489,183.33796293)(335.55750228,183.27797029)
\curveto(335.53749492,183.21796305)(335.52749493,183.15796311)(335.52750228,183.09797029)
\curveto(335.53749492,183.03796323)(335.53249493,182.97796329)(335.51250228,182.91797029)
\curveto(335.47249499,182.77796349)(335.43749502,182.64296362)(335.40750228,182.51297029)
\curveto(335.37749508,182.38296388)(335.33749512,182.25796401)(335.28750228,182.13797029)
\curveto(335.22749523,181.99796427)(335.1574953,181.87296439)(335.07750228,181.76297029)
\curveto(335.00749545,181.65296461)(334.93249553,181.54296472)(334.85250228,181.43297029)
\lineto(334.79250228,181.37297029)
\curveto(334.78249568,181.35296491)(334.76749569,181.33296493)(334.74750228,181.31297029)
\curveto(334.62749583,181.15296511)(334.49249597,181.00796526)(334.34250228,180.87797029)
\curveto(334.19249627,180.74796552)(334.03249643,180.62296564)(333.86250228,180.50297029)
\curveto(333.55249691,180.28296598)(333.2574972,180.07796619)(332.97750228,179.88797029)
\curveto(332.74749771,179.74796652)(332.51749794,179.61296665)(332.28750228,179.48297029)
\curveto(332.06749839,179.35296691)(331.84749861,179.21796705)(331.62750228,179.07797029)
\curveto(331.37749908,178.90796736)(331.13749932,178.72796754)(330.90750228,178.53797029)
\curveto(330.68749977,178.34796792)(330.49749996,178.12296814)(330.33750228,177.86297029)
\curveto(330.29750016,177.80296846)(330.2625002,177.74296852)(330.23250228,177.68297029)
\curveto(330.20250026,177.63296863)(330.17250029,177.5679687)(330.14250228,177.48797029)
\curveto(330.12250034,177.41796885)(330.11750034,177.35796891)(330.12750228,177.30797029)
\curveto(330.14750031,177.23796903)(330.18250028,177.18296908)(330.23250228,177.14297029)
\curveto(330.28250018,177.11296915)(330.34250012,177.09296917)(330.41250228,177.08297029)
\lineto(330.65250228,177.08297029)
\lineto(331.40250228,177.08297029)
\lineto(334.20750228,177.08297029)
\lineto(334.86750228,177.08297029)
\curveto(334.9574955,177.08296918)(335.04249542,177.07796919)(335.12250228,177.06797029)
\curveto(335.20249526,177.0679692)(335.26749519,177.04796922)(335.31750228,177.00797029)
\curveto(335.36749509,176.9679693)(335.40749505,176.89296937)(335.43750228,176.78297029)
\curveto(335.47749498,176.68296958)(335.48749497,176.58296968)(335.46750228,176.48297029)
\lineto(335.46750228,176.34797029)
\curveto(335.44749501,176.27796999)(335.42749503,176.21797005)(335.40750228,176.16797029)
\curveto(335.38749507,176.11797015)(335.35249511,176.07797019)(335.30250228,176.04797029)
\curveto(335.25249521,176.00797026)(335.18249528,175.98797028)(335.09250228,175.98797029)
\lineto(334.82250228,175.98797029)
\lineto(333.92250228,175.98797029)
\lineto(330.41250228,175.98797029)
\lineto(329.34750228,175.98797029)
\curveto(329.26750119,175.98797028)(329.17750128,175.98297028)(329.07750228,175.97297029)
\curveto(328.97750148,175.97297029)(328.89250157,175.98297028)(328.82250228,176.00297029)
\curveto(328.61250185,176.07297019)(328.54750191,176.25297001)(328.62750228,176.54297029)
\curveto(328.63750182,176.58296968)(328.63750182,176.61796965)(328.62750228,176.64797029)
\curveto(328.62750183,176.68796958)(328.63750182,176.73296953)(328.65750228,176.78297029)
\curveto(328.67750178,176.8629694)(328.69750176,176.94796932)(328.71750228,177.03797029)
\curveto(328.73750172,177.12796914)(328.7625017,177.21296905)(328.79250228,177.29297029)
\curveto(328.95250151,177.78296848)(329.15250131,178.19796807)(329.39250228,178.53797029)
\curveto(329.57250089,178.78796748)(329.77750068,179.01296725)(330.00750228,179.21297029)
\curveto(330.23750022,179.42296684)(330.47749998,179.61796665)(330.72750228,179.79797029)
\curveto(330.98749947,179.97796629)(331.25249921,180.14796612)(331.52250228,180.30797029)
\curveto(331.80249866,180.47796579)(332.07249839,180.65296561)(332.33250228,180.83297029)
\curveto(332.44249802,180.91296535)(332.54749791,180.98796528)(332.64750228,181.05797029)
\curveto(332.7574977,181.12796514)(332.86749759,181.20296506)(332.97750228,181.28297029)
\curveto(333.01749744,181.31296495)(333.05249741,181.34296492)(333.08250228,181.37297029)
\curveto(333.12249734,181.41296485)(333.1624973,181.44296482)(333.20250228,181.46297029)
\curveto(333.34249712,181.57296469)(333.46749699,181.69796457)(333.57750228,181.83797029)
\curveto(333.59749686,181.8679644)(333.62249684,181.89296437)(333.65250228,181.91297029)
\curveto(333.68249678,181.94296432)(333.70749675,181.97296429)(333.72750228,182.00297029)
\curveto(333.80749665,182.10296416)(333.87249659,182.20296406)(333.92250228,182.30297029)
\curveto(333.98249648,182.40296386)(334.03749642,182.51296375)(334.08750228,182.63297029)
\curveto(334.11749634,182.70296356)(334.13749632,182.77796349)(334.14750228,182.85797029)
\lineto(334.20750228,183.09797029)
\lineto(334.20750228,183.18797029)
\curveto(334.21749624,183.21796305)(334.22249624,183.24796302)(334.22250228,183.27797029)
\curveto(334.24249622,183.34796292)(334.24749621,183.44296282)(334.23750228,183.56297029)
\curveto(334.23749622,183.69296257)(334.22749623,183.79296247)(334.20750228,183.86297029)
\curveto(334.18749627,183.94296232)(334.16749629,184.01796225)(334.14750228,184.08797029)
\curveto(334.13749632,184.1679621)(334.11749634,184.24796202)(334.08750228,184.32797029)
\curveto(333.97749648,184.5679617)(333.82749663,184.7679615)(333.63750228,184.92797029)
\curveto(333.457497,185.09796117)(333.23749722,185.23796103)(332.97750228,185.34797029)
\curveto(332.90749755,185.3679609)(332.83749762,185.38296088)(332.76750228,185.39297029)
\curveto(332.69749776,185.41296085)(332.62249784,185.43296083)(332.54250228,185.45297029)
\curveto(332.462498,185.47296079)(332.35249811,185.48296078)(332.21250228,185.48297029)
\curveto(332.08249838,185.48296078)(331.97749848,185.47296079)(331.89750228,185.45297029)
\curveto(331.83749862,185.44296082)(331.78249868,185.43796083)(331.73250228,185.43797029)
\curveto(331.68249878,185.43796083)(331.63249883,185.42796084)(331.58250228,185.40797029)
\curveto(331.48249898,185.3679609)(331.38749907,185.32796094)(331.29750228,185.28797029)
\curveto(331.21749924,185.24796102)(331.13749932,185.20296106)(331.05750228,185.15297029)
\curveto(331.02749943,185.13296113)(330.99749946,185.10796116)(330.96750228,185.07797029)
\curveto(330.94749951,185.04796122)(330.92249954,185.02296124)(330.89250228,185.00297029)
\lineto(330.81750228,184.92797029)
\curveto(330.78749967,184.90796136)(330.7624997,184.88796138)(330.74250228,184.86797029)
\lineto(330.59250228,184.65797029)
\curveto(330.55249991,184.59796167)(330.50749995,184.53296173)(330.45750228,184.46297029)
\curveto(330.39750006,184.37296189)(330.34750011,184.267962)(330.30750228,184.14797029)
\curveto(330.27750018,184.03796223)(330.24250022,183.92796234)(330.20250228,183.81797029)
\curveto(330.1625003,183.70796256)(330.13750032,183.5629627)(330.12750228,183.38297029)
\curveto(330.11750034,183.21296305)(330.08750037,183.08796318)(330.03750228,183.00797029)
\curveto(329.98750047,182.92796334)(329.91250055,182.88296338)(329.81250228,182.87297029)
\curveto(329.71250075,182.8629634)(329.60250086,182.85796341)(329.48250228,182.85797029)
\curveto(329.44250102,182.85796341)(329.40250106,182.85296341)(329.36250228,182.84297029)
\curveto(329.32250114,182.84296342)(329.28750117,182.84796342)(329.25750228,182.85797029)
\curveto(329.20750125,182.87796339)(329.1575013,182.88796338)(329.10750228,182.88797029)
\curveto(329.06750139,182.88796338)(329.02750143,182.89796337)(328.98750228,182.91797029)
\curveto(328.89750156,182.97796329)(328.85250161,183.11296315)(328.85250228,183.32297029)
\lineto(328.85250228,183.44297029)
\curveto(328.8625016,183.50296276)(328.86750159,183.5629627)(328.86750228,183.62297029)
\curveto(328.87750158,183.69296257)(328.88750157,183.75796251)(328.89750228,183.81797029)
\curveto(328.91750154,183.92796234)(328.93750152,184.02796224)(328.95750228,184.11797029)
\curveto(328.97750148,184.21796205)(329.00750145,184.31296195)(329.04750228,184.40297029)
\curveto(329.06750139,184.47296179)(329.08750137,184.53296173)(329.10750228,184.58297029)
\lineto(329.16750228,184.76297029)
\curveto(329.28750117,185.02296124)(329.44250102,185.267961)(329.63250228,185.49797029)
\curveto(329.83250063,185.72796054)(330.04750041,185.91296035)(330.27750228,186.05297029)
\curveto(330.38750007,186.13296013)(330.50249996,186.19796007)(330.62250228,186.24797029)
\lineto(331.01250228,186.39797029)
\curveto(331.12249934,186.44795982)(331.23749922,186.47795979)(331.35750228,186.48797029)
\curveto(331.47749898,186.50795976)(331.60249886,186.53295973)(331.73250228,186.56297029)
\curveto(331.80249866,186.5629597)(331.86749859,186.5629597)(331.92750228,186.56297029)
\curveto(331.98749847,186.57295969)(332.05249841,186.58295968)(332.12250228,186.59297029)
}
}
{
\newrgbcolor{curcolor}{0 0 0}
\pscustom[linestyle=none,fillstyle=solid,fillcolor=curcolor]
{
\newpath
\moveto(347.02711166,184.50797029)
\curveto(346.82710136,184.21796205)(346.61710157,183.93296233)(346.39711166,183.65297029)
\curveto(346.187102,183.37296289)(345.9821022,183.08796318)(345.78211166,182.79797029)
\curveto(345.182103,181.94796432)(344.57710361,181.10796516)(343.96711166,180.27797029)
\curveto(343.35710483,179.45796681)(342.75210543,178.62296764)(342.15211166,177.77297029)
\lineto(341.64211166,177.05297029)
\lineto(341.13211166,176.36297029)
\curveto(341.05210713,176.25297001)(340.97210721,176.13797013)(340.89211166,176.01797029)
\curveto(340.81210737,175.89797037)(340.71710747,175.80297046)(340.60711166,175.73297029)
\curveto(340.56710762,175.71297055)(340.50210768,175.69797057)(340.41211166,175.68797029)
\curveto(340.33210785,175.6679706)(340.24210794,175.65797061)(340.14211166,175.65797029)
\curveto(340.04210814,175.65797061)(339.94710824,175.6629706)(339.85711166,175.67297029)
\curveto(339.77710841,175.68297058)(339.71710847,175.70297056)(339.67711166,175.73297029)
\curveto(339.64710854,175.75297051)(339.62210856,175.78797048)(339.60211166,175.83797029)
\curveto(339.59210859,175.87797039)(339.59710859,175.92297034)(339.61711166,175.97297029)
\curveto(339.65710853,176.05297021)(339.70210848,176.12797014)(339.75211166,176.19797029)
\curveto(339.81210837,176.27796999)(339.86710832,176.35796991)(339.91711166,176.43797029)
\curveto(340.15710803,176.77796949)(340.40210778,177.11296915)(340.65211166,177.44297029)
\curveto(340.90210728,177.77296849)(341.14210704,178.10796816)(341.37211166,178.44797029)
\curveto(341.53210665,178.6679676)(341.69210649,178.88296738)(341.85211166,179.09297029)
\curveto(342.01210617,179.30296696)(342.17210601,179.51796675)(342.33211166,179.73797029)
\curveto(342.69210549,180.25796601)(343.05710513,180.7679655)(343.42711166,181.26797029)
\curveto(343.79710439,181.7679645)(344.16710402,182.27796399)(344.53711166,182.79797029)
\curveto(344.67710351,182.99796327)(344.81710337,183.19296307)(344.95711166,183.38297029)
\curveto(345.10710308,183.57296269)(345.25210293,183.7679625)(345.39211166,183.96797029)
\curveto(345.60210258,184.267962)(345.81710237,184.5679617)(346.03711166,184.86797029)
\lineto(346.69711166,185.76797029)
\lineto(346.87711166,186.03797029)
\lineto(347.08711166,186.30797029)
\lineto(347.20711166,186.48797029)
\curveto(347.25710093,186.54795972)(347.30710088,186.60295966)(347.35711166,186.65297029)
\curveto(347.42710076,186.70295956)(347.50210068,186.73795953)(347.58211166,186.75797029)
\curveto(347.60210058,186.7679595)(347.62710056,186.7679595)(347.65711166,186.75797029)
\curveto(347.69710049,186.75795951)(347.72710046,186.7679595)(347.74711166,186.78797029)
\curveto(347.86710032,186.78795948)(348.00210018,186.78295948)(348.15211166,186.77297029)
\curveto(348.30209988,186.77295949)(348.39209979,186.72795954)(348.42211166,186.63797029)
\curveto(348.44209974,186.60795966)(348.44709974,186.57295969)(348.43711166,186.53297029)
\curveto(348.42709976,186.49295977)(348.41209977,186.4629598)(348.39211166,186.44297029)
\curveto(348.35209983,186.3629599)(348.31209987,186.29295997)(348.27211166,186.23297029)
\curveto(348.23209995,186.17296009)(348.1871,186.11296015)(348.13711166,186.05297029)
\lineto(347.56711166,185.27297029)
\curveto(347.3871008,185.02296124)(347.20710098,184.7679615)(347.02711166,184.50797029)
\moveto(340.17211166,180.60797029)
\curveto(340.12210806,180.62796564)(340.07210811,180.63296563)(340.02211166,180.62297029)
\curveto(339.97210821,180.61296565)(339.92210826,180.61796565)(339.87211166,180.63797029)
\curveto(339.76210842,180.65796561)(339.65710853,180.67796559)(339.55711166,180.69797029)
\curveto(339.46710872,180.72796554)(339.37210881,180.7679655)(339.27211166,180.81797029)
\curveto(338.94210924,180.95796531)(338.6871095,181.15296511)(338.50711166,181.40297029)
\curveto(338.32710986,181.6629646)(338.18211,181.97296429)(338.07211166,182.33297029)
\curveto(338.04211014,182.41296385)(338.02211016,182.49296377)(338.01211166,182.57297029)
\curveto(338.00211018,182.6629636)(337.9871102,182.74796352)(337.96711166,182.82797029)
\curveto(337.95711023,182.87796339)(337.95211023,182.94296332)(337.95211166,183.02297029)
\curveto(337.94211024,183.05296321)(337.93711025,183.08296318)(337.93711166,183.11297029)
\curveto(337.93711025,183.15296311)(337.93211025,183.18796308)(337.92211166,183.21797029)
\lineto(337.92211166,183.36797029)
\curveto(337.91211027,183.41796285)(337.90711028,183.47796279)(337.90711166,183.54797029)
\curveto(337.90711028,183.62796264)(337.91211027,183.69296257)(337.92211166,183.74297029)
\lineto(337.92211166,183.90797029)
\curveto(337.94211024,183.95796231)(337.94711024,184.00296226)(337.93711166,184.04297029)
\curveto(337.93711025,184.09296217)(337.94211024,184.13796213)(337.95211166,184.17797029)
\curveto(337.96211022,184.21796205)(337.96711022,184.25296201)(337.96711166,184.28297029)
\curveto(337.96711022,184.32296194)(337.97211021,184.3629619)(337.98211166,184.40297029)
\curveto(338.01211017,184.51296175)(338.03211015,184.62296164)(338.04211166,184.73297029)
\curveto(338.06211012,184.85296141)(338.09711009,184.9679613)(338.14711166,185.07797029)
\curveto(338.2871099,185.41796085)(338.44710974,185.69296057)(338.62711166,185.90297029)
\curveto(338.81710937,186.12296014)(339.0871091,186.30295996)(339.43711166,186.44297029)
\curveto(339.51710867,186.47295979)(339.60210858,186.49295977)(339.69211166,186.50297029)
\curveto(339.7821084,186.52295974)(339.87710831,186.54295972)(339.97711166,186.56297029)
\curveto(340.00710818,186.57295969)(340.06210812,186.57295969)(340.14211166,186.56297029)
\curveto(340.22210796,186.5629597)(340.27210791,186.57295969)(340.29211166,186.59297029)
\curveto(340.85210733,186.60295966)(341.30210688,186.49295977)(341.64211166,186.26297029)
\curveto(341.99210619,186.03296023)(342.25210593,185.72796054)(342.42211166,185.34797029)
\curveto(342.46210572,185.25796101)(342.49710569,185.1629611)(342.52711166,185.06297029)
\curveto(342.55710563,184.9629613)(342.5821056,184.8629614)(342.60211166,184.76297029)
\curveto(342.62210556,184.73296153)(342.62710556,184.70296156)(342.61711166,184.67297029)
\curveto(342.61710557,184.64296162)(342.62210556,184.61296165)(342.63211166,184.58297029)
\curveto(342.66210552,184.47296179)(342.6821055,184.34796192)(342.69211166,184.20797029)
\curveto(342.70210548,184.07796219)(342.71210547,183.94296232)(342.72211166,183.80297029)
\lineto(342.72211166,183.63797029)
\curveto(342.73210545,183.57796269)(342.73210545,183.52296274)(342.72211166,183.47297029)
\curveto(342.71210547,183.42296284)(342.70710548,183.37296289)(342.70711166,183.32297029)
\lineto(342.70711166,183.18797029)
\curveto(342.69710549,183.14796312)(342.69210549,183.10796316)(342.69211166,183.06797029)
\curveto(342.70210548,183.02796324)(342.69710549,182.98296328)(342.67711166,182.93297029)
\curveto(342.65710553,182.82296344)(342.63710555,182.71796355)(342.61711166,182.61797029)
\curveto(342.60710558,182.51796375)(342.5871056,182.41796385)(342.55711166,182.31797029)
\curveto(342.42710576,181.95796431)(342.26210592,181.64296462)(342.06211166,181.37297029)
\curveto(341.86210632,181.10296516)(341.5871066,180.89796537)(341.23711166,180.75797029)
\curveto(341.15710703,180.72796554)(341.07210711,180.70296556)(340.98211166,180.68297029)
\lineto(340.71211166,180.62297029)
\curveto(340.66210752,180.61296565)(340.61710757,180.60796566)(340.57711166,180.60797029)
\curveto(340.53710765,180.61796565)(340.49710769,180.61796565)(340.45711166,180.60797029)
\curveto(340.35710783,180.58796568)(340.26210792,180.58796568)(340.17211166,180.60797029)
\moveto(339.33211166,182.00297029)
\curveto(339.37210881,181.93296433)(339.41210877,181.8679644)(339.45211166,181.80797029)
\curveto(339.49210869,181.75796451)(339.54210864,181.70796456)(339.60211166,181.65797029)
\lineto(339.75211166,181.53797029)
\curveto(339.81210837,181.50796476)(339.87710831,181.48296478)(339.94711166,181.46297029)
\curveto(339.9871082,181.44296482)(340.02210816,181.43296483)(340.05211166,181.43297029)
\curveto(340.09210809,181.44296482)(340.13210805,181.43796483)(340.17211166,181.41797029)
\curveto(340.20210798,181.41796485)(340.24210794,181.41296485)(340.29211166,181.40297029)
\curveto(340.34210784,181.40296486)(340.3821078,181.40796486)(340.41211166,181.41797029)
\lineto(340.63711166,181.46297029)
\curveto(340.8871073,181.54296472)(341.07210711,181.6679646)(341.19211166,181.83797029)
\curveto(341.27210691,181.93796433)(341.34210684,182.0679642)(341.40211166,182.22797029)
\curveto(341.4821067,182.40796386)(341.54210664,182.63296363)(341.58211166,182.90297029)
\curveto(341.62210656,183.18296308)(341.63710655,183.4629628)(341.62711166,183.74297029)
\curveto(341.61710657,184.03296223)(341.5871066,184.30796196)(341.53711166,184.56797029)
\curveto(341.4871067,184.82796144)(341.41210677,185.03796123)(341.31211166,185.19797029)
\curveto(341.19210699,185.39796087)(341.04210714,185.54796072)(340.86211166,185.64797029)
\curveto(340.7821074,185.69796057)(340.69210749,185.72796054)(340.59211166,185.73797029)
\curveto(340.49210769,185.75796051)(340.3871078,185.7679605)(340.27711166,185.76797029)
\curveto(340.25710793,185.75796051)(340.23210795,185.75296051)(340.20211166,185.75297029)
\curveto(340.182108,185.7629605)(340.16210802,185.7629605)(340.14211166,185.75297029)
\curveto(340.09210809,185.74296052)(340.04710814,185.73296053)(340.00711166,185.72297029)
\curveto(339.96710822,185.72296054)(339.92710826,185.71296055)(339.88711166,185.69297029)
\curveto(339.70710848,185.61296065)(339.55710863,185.49296077)(339.43711166,185.33297029)
\curveto(339.32710886,185.17296109)(339.23710895,184.99296127)(339.16711166,184.79297029)
\curveto(339.10710908,184.60296166)(339.06210912,184.37796189)(339.03211166,184.11797029)
\curveto(339.01210917,183.85796241)(339.00710918,183.59296267)(339.01711166,183.32297029)
\curveto(339.02710916,183.0629632)(339.05710913,182.81296345)(339.10711166,182.57297029)
\curveto(339.16710902,182.34296392)(339.24210894,182.15296411)(339.33211166,182.00297029)
\moveto(350.13211166,179.01797029)
\curveto(350.14209804,178.9679673)(350.14709804,178.87796739)(350.14711166,178.74797029)
\curveto(350.14709804,178.61796765)(350.13709805,178.52796774)(350.11711166,178.47797029)
\curveto(350.09709809,178.42796784)(350.09209809,178.37296789)(350.10211166,178.31297029)
\curveto(350.11209807,178.262968)(350.11209807,178.21296805)(350.10211166,178.16297029)
\curveto(350.06209812,178.02296824)(350.03209815,177.88796838)(350.01211166,177.75797029)
\curveto(350.00209818,177.62796864)(349.97209821,177.50796876)(349.92211166,177.39797029)
\curveto(349.7820984,177.04796922)(349.61709857,176.75296951)(349.42711166,176.51297029)
\curveto(349.23709895,176.28296998)(348.96709922,176.09797017)(348.61711166,175.95797029)
\curveto(348.53709965,175.92797034)(348.45209973,175.90797036)(348.36211166,175.89797029)
\curveto(348.27209991,175.87797039)(348.1871,175.85797041)(348.10711166,175.83797029)
\curveto(348.05710013,175.82797044)(348.00710018,175.82297044)(347.95711166,175.82297029)
\curveto(347.90710028,175.82297044)(347.85710033,175.81797045)(347.80711166,175.80797029)
\curveto(347.77710041,175.79797047)(347.72710046,175.79797047)(347.65711166,175.80797029)
\curveto(347.5871006,175.80797046)(347.53710065,175.81297045)(347.50711166,175.82297029)
\curveto(347.44710074,175.84297042)(347.3871008,175.85297041)(347.32711166,175.85297029)
\curveto(347.27710091,175.84297042)(347.22710096,175.84797042)(347.17711166,175.86797029)
\curveto(347.0871011,175.88797038)(346.99710119,175.91297035)(346.90711166,175.94297029)
\curveto(346.82710136,175.9629703)(346.74710144,175.99297027)(346.66711166,176.03297029)
\curveto(346.34710184,176.17297009)(346.09710209,176.3679699)(345.91711166,176.61797029)
\curveto(345.73710245,176.87796939)(345.5871026,177.18296908)(345.46711166,177.53297029)
\curveto(345.44710274,177.61296865)(345.43210275,177.69796857)(345.42211166,177.78797029)
\curveto(345.41210277,177.87796839)(345.39710279,177.9629683)(345.37711166,178.04297029)
\curveto(345.36710282,178.07296819)(345.36210282,178.10296816)(345.36211166,178.13297029)
\lineto(345.36211166,178.23797029)
\curveto(345.34210284,178.31796795)(345.33210285,178.39796787)(345.33211166,178.47797029)
\lineto(345.33211166,178.61297029)
\curveto(345.31210287,178.71296755)(345.31210287,178.81296745)(345.33211166,178.91297029)
\lineto(345.33211166,179.09297029)
\curveto(345.34210284,179.14296712)(345.34710284,179.18796708)(345.34711166,179.22797029)
\curveto(345.34710284,179.27796699)(345.35210283,179.32296694)(345.36211166,179.36297029)
\curveto(345.37210281,179.40296686)(345.37710281,179.43796683)(345.37711166,179.46797029)
\curveto(345.37710281,179.50796676)(345.3821028,179.54796672)(345.39211166,179.58797029)
\lineto(345.45211166,179.91797029)
\curveto(345.47210271,180.03796623)(345.50210268,180.14796612)(345.54211166,180.24797029)
\curveto(345.6821025,180.57796569)(345.84210234,180.85296541)(346.02211166,181.07297029)
\curveto(346.21210197,181.30296496)(346.47210171,181.48796478)(346.80211166,181.62797029)
\curveto(346.8821013,181.6679646)(346.96710122,181.69296457)(347.05711166,181.70297029)
\lineto(347.35711166,181.76297029)
\lineto(347.49211166,181.76297029)
\curveto(347.54210064,181.77296449)(347.59210059,181.77796449)(347.64211166,181.77797029)
\curveto(348.21209997,181.79796447)(348.67209951,181.69296457)(349.02211166,181.46297029)
\curveto(349.3820988,181.24296502)(349.64709854,180.94296532)(349.81711166,180.56297029)
\curveto(349.86709832,180.4629658)(349.90709828,180.3629659)(349.93711166,180.26297029)
\curveto(349.96709822,180.1629661)(349.99709819,180.05796621)(350.02711166,179.94797029)
\curveto(350.03709815,179.90796636)(350.04209814,179.87296639)(350.04211166,179.84297029)
\curveto(350.04209814,179.82296644)(350.04709814,179.79296647)(350.05711166,179.75297029)
\curveto(350.07709811,179.68296658)(350.0870981,179.60796666)(350.08711166,179.52797029)
\curveto(350.0870981,179.44796682)(350.09709809,179.3679669)(350.11711166,179.28797029)
\curveto(350.11709807,179.23796703)(350.11709807,179.19296707)(350.11711166,179.15297029)
\curveto(350.11709807,179.11296715)(350.12209806,179.0679672)(350.13211166,179.01797029)
\moveto(349.02211166,178.58297029)
\curveto(349.03209915,178.63296763)(349.03709915,178.70796756)(349.03711166,178.80797029)
\curveto(349.04709914,178.90796736)(349.04209914,178.98296728)(349.02211166,179.03297029)
\curveto(349.00209918,179.09296717)(348.99709919,179.14796712)(349.00711166,179.19797029)
\curveto(349.02709916,179.25796701)(349.02709916,179.31796695)(349.00711166,179.37797029)
\curveto(348.99709919,179.40796686)(348.99209919,179.44296682)(348.99211166,179.48297029)
\curveto(348.99209919,179.52296674)(348.9870992,179.5629667)(348.97711166,179.60297029)
\curveto(348.95709923,179.68296658)(348.93709925,179.75796651)(348.91711166,179.82797029)
\curveto(348.90709928,179.90796636)(348.89209929,179.98796628)(348.87211166,180.06797029)
\curveto(348.84209934,180.12796614)(348.81709937,180.18796608)(348.79711166,180.24797029)
\curveto(348.77709941,180.30796596)(348.74709944,180.3679659)(348.70711166,180.42797029)
\curveto(348.60709958,180.59796567)(348.47709971,180.73296553)(348.31711166,180.83297029)
\curveto(348.23709995,180.88296538)(348.14210004,180.91796535)(348.03211166,180.93797029)
\curveto(347.92210026,180.95796531)(347.79710039,180.9679653)(347.65711166,180.96797029)
\curveto(347.63710055,180.95796531)(347.61210057,180.95296531)(347.58211166,180.95297029)
\curveto(347.55210063,180.9629653)(347.52210066,180.9629653)(347.49211166,180.95297029)
\lineto(347.34211166,180.89297029)
\curveto(347.29210089,180.88296538)(347.24710094,180.8679654)(347.20711166,180.84797029)
\curveto(347.01710117,180.73796553)(346.87210131,180.59296567)(346.77211166,180.41297029)
\curveto(346.6821015,180.23296603)(346.60210158,180.02796624)(346.53211166,179.79797029)
\curveto(346.49210169,179.6679666)(346.47210171,179.53296673)(346.47211166,179.39297029)
\curveto(346.47210171,179.262967)(346.46210172,179.11796715)(346.44211166,178.95797029)
\curveto(346.43210175,178.90796736)(346.42210176,178.84796742)(346.41211166,178.77797029)
\curveto(346.41210177,178.70796756)(346.42210176,178.64796762)(346.44211166,178.59797029)
\lineto(346.44211166,178.43297029)
\lineto(346.44211166,178.25297029)
\curveto(346.45210173,178.20296806)(346.46210172,178.14796812)(346.47211166,178.08797029)
\curveto(346.4821017,178.03796823)(346.4871017,177.98296828)(346.48711166,177.92297029)
\curveto(346.49710169,177.8629684)(346.51210167,177.80796846)(346.53211166,177.75797029)
\curveto(346.5821016,177.5679687)(346.64210154,177.39296887)(346.71211166,177.23297029)
\curveto(346.7821014,177.07296919)(346.8871013,176.94296932)(347.02711166,176.84297029)
\curveto(347.15710103,176.74296952)(347.29710089,176.67296959)(347.44711166,176.63297029)
\curveto(347.47710071,176.62296964)(347.50210068,176.61796965)(347.52211166,176.61797029)
\curveto(347.55210063,176.62796964)(347.5821006,176.62796964)(347.61211166,176.61797029)
\curveto(347.63210055,176.61796965)(347.66210052,176.61296965)(347.70211166,176.60297029)
\curveto(347.74210044,176.60296966)(347.77710041,176.60796966)(347.80711166,176.61797029)
\curveto(347.84710034,176.62796964)(347.8871003,176.63296963)(347.92711166,176.63297029)
\curveto(347.96710022,176.63296963)(348.00710018,176.64296962)(348.04711166,176.66297029)
\curveto(348.2870999,176.74296952)(348.4820997,176.87796939)(348.63211166,177.06797029)
\curveto(348.75209943,177.24796902)(348.84209934,177.45296881)(348.90211166,177.68297029)
\curveto(348.92209926,177.75296851)(348.93709925,177.82296844)(348.94711166,177.89297029)
\curveto(348.95709923,177.97296829)(348.97209921,178.05296821)(348.99211166,178.13297029)
\curveto(348.99209919,178.19296807)(348.99709919,178.23796803)(349.00711166,178.26797029)
\curveto(349.00709918,178.28796798)(349.00709918,178.31296795)(349.00711166,178.34297029)
\curveto(349.00709918,178.38296788)(349.01209917,178.41296785)(349.02211166,178.43297029)
\lineto(349.02211166,178.58297029)
}
}
{
\newrgbcolor{curcolor}{0 0 0}
\pscustom[linestyle=none,fillstyle=solid,fillcolor=curcolor]
{
\newpath
\moveto(289.34470321,95.67921297)
\curveto(289.44469836,95.67920235)(289.53969826,95.66920236)(289.62970321,95.64921297)
\curveto(289.71969808,95.63920239)(289.78469802,95.60920242)(289.82470321,95.55921297)
\curveto(289.88469792,95.47920255)(289.91469789,95.37420265)(289.91470321,95.24421297)
\lineto(289.91470321,94.85421297)
\lineto(289.91470321,93.35421297)
\lineto(289.91470321,86.96421297)
\lineto(289.91470321,85.79421297)
\lineto(289.91470321,85.47921297)
\curveto(289.92469788,85.37921265)(289.90969789,85.29921273)(289.86970321,85.23921297)
\curveto(289.81969798,85.15921287)(289.74469806,85.10921292)(289.64470321,85.08921297)
\curveto(289.55469825,85.07921295)(289.44469836,85.07421295)(289.31470321,85.07421297)
\lineto(289.08970321,85.07421297)
\curveto(289.00969879,85.09421293)(288.93969886,85.10921292)(288.87970321,85.11921297)
\curveto(288.81969898,85.13921289)(288.76969903,85.17921285)(288.72970321,85.23921297)
\curveto(288.68969911,85.29921273)(288.66969913,85.37421265)(288.66970321,85.46421297)
\lineto(288.66970321,85.76421297)
\lineto(288.66970321,86.85921297)
\lineto(288.66970321,92.19921297)
\curveto(288.64969915,92.28920574)(288.63469917,92.36420566)(288.62470321,92.42421297)
\curveto(288.62469918,92.49420553)(288.59469921,92.55420547)(288.53470321,92.60421297)
\curveto(288.46469934,92.65420537)(288.37469943,92.67920535)(288.26470321,92.67921297)
\curveto(288.16469964,92.68920534)(288.05469975,92.69420533)(287.93470321,92.69421297)
\lineto(286.79470321,92.69421297)
\lineto(286.29970321,92.69421297)
\curveto(286.13970166,92.70420532)(286.02970177,92.76420526)(285.96970321,92.87421297)
\curveto(285.94970185,92.90420512)(285.93970186,92.93420509)(285.93970321,92.96421297)
\curveto(285.93970186,93.00420502)(285.93470187,93.04920498)(285.92470321,93.09921297)
\curveto(285.9047019,93.21920481)(285.90970189,93.3292047)(285.93970321,93.42921297)
\curveto(285.97970182,93.5292045)(286.03470177,93.59920443)(286.10470321,93.63921297)
\curveto(286.18470162,93.68920434)(286.3047015,93.71420431)(286.46470321,93.71421297)
\curveto(286.62470118,93.71420431)(286.75970104,93.7292043)(286.86970321,93.75921297)
\curveto(286.91970088,93.76920426)(286.97470083,93.77420425)(287.03470321,93.77421297)
\curveto(287.09470071,93.78420424)(287.15470065,93.79920423)(287.21470321,93.81921297)
\curveto(287.36470044,93.86920416)(287.50970029,93.91920411)(287.64970321,93.96921297)
\curveto(287.78970001,94.029204)(287.92469988,94.09920393)(288.05470321,94.17921297)
\curveto(288.19469961,94.26920376)(288.31469949,94.37420365)(288.41470321,94.49421297)
\curveto(288.51469929,94.61420341)(288.60969919,94.74420328)(288.69970321,94.88421297)
\curveto(288.75969904,94.98420304)(288.804699,95.09420293)(288.83470321,95.21421297)
\curveto(288.87469893,95.33420269)(288.92469888,95.43920259)(288.98470321,95.52921297)
\curveto(289.03469877,95.58920244)(289.1046987,95.6292024)(289.19470321,95.64921297)
\curveto(289.21469859,95.65920237)(289.23969856,95.66420236)(289.26970321,95.66421297)
\curveto(289.2996985,95.66420236)(289.32469848,95.66920236)(289.34470321,95.67921297)
}
}
{
\newrgbcolor{curcolor}{0 0 0}
\pscustom[linestyle=none,fillstyle=solid,fillcolor=curcolor]
{
\newpath
\moveto(296.88431259,95.67921297)
\curveto(297.57430795,95.68920234)(298.17430735,95.56920246)(298.68431259,95.31921297)
\curveto(299.20430632,95.06920296)(299.59930593,94.73420329)(299.86931259,94.31421297)
\curveto(299.91930561,94.23420379)(299.96430556,94.14420388)(300.00431259,94.04421297)
\curveto(300.04430548,93.95420407)(300.08930544,93.85920417)(300.13931259,93.75921297)
\curveto(300.17930535,93.65920437)(300.20930532,93.55920447)(300.22931259,93.45921297)
\curveto(300.24930528,93.35920467)(300.26930526,93.25420477)(300.28931259,93.14421297)
\curveto(300.30930522,93.09420493)(300.31430521,93.04920498)(300.30431259,93.00921297)
\curveto(300.29430523,92.96920506)(300.29930523,92.9242051)(300.31931259,92.87421297)
\curveto(300.3293052,92.8242052)(300.33430519,92.73920529)(300.33431259,92.61921297)
\curveto(300.33430519,92.50920552)(300.3293052,92.4242056)(300.31931259,92.36421297)
\curveto(300.29930523,92.30420572)(300.28930524,92.24420578)(300.28931259,92.18421297)
\curveto(300.29930523,92.1242059)(300.29430523,92.06420596)(300.27431259,92.00421297)
\curveto(300.23430529,91.86420616)(300.19930533,91.7292063)(300.16931259,91.59921297)
\curveto(300.13930539,91.46920656)(300.09930543,91.34420668)(300.04931259,91.22421297)
\curveto(299.98930554,91.08420694)(299.91930561,90.95920707)(299.83931259,90.84921297)
\curveto(299.76930576,90.73920729)(299.69430583,90.6292074)(299.61431259,90.51921297)
\lineto(299.55431259,90.45921297)
\curveto(299.54430598,90.43920759)(299.529306,90.41920761)(299.50931259,90.39921297)
\curveto(299.38930614,90.23920779)(299.25430627,90.09420793)(299.10431259,89.96421297)
\curveto(298.95430657,89.83420819)(298.79430673,89.70920832)(298.62431259,89.58921297)
\curveto(298.31430721,89.36920866)(298.01930751,89.16420886)(297.73931259,88.97421297)
\curveto(297.50930802,88.83420919)(297.27930825,88.69920933)(297.04931259,88.56921297)
\curveto(296.8293087,88.43920959)(296.60930892,88.30420972)(296.38931259,88.16421297)
\curveto(296.13930939,87.99421003)(295.89930963,87.81421021)(295.66931259,87.62421297)
\curveto(295.44931008,87.43421059)(295.25931027,87.20921082)(295.09931259,86.94921297)
\curveto(295.05931047,86.88921114)(295.0243105,86.8292112)(294.99431259,86.76921297)
\curveto(294.96431056,86.71921131)(294.93431059,86.65421137)(294.90431259,86.57421297)
\curveto(294.88431064,86.50421152)(294.87931065,86.44421158)(294.88931259,86.39421297)
\curveto(294.90931062,86.3242117)(294.94431058,86.26921176)(294.99431259,86.22921297)
\curveto(295.04431048,86.19921183)(295.10431042,86.17921185)(295.17431259,86.16921297)
\lineto(295.41431259,86.16921297)
\lineto(296.16431259,86.16921297)
\lineto(298.96931259,86.16921297)
\lineto(299.62931259,86.16921297)
\curveto(299.71930581,86.16921186)(299.80430572,86.16421186)(299.88431259,86.15421297)
\curveto(299.96430556,86.15421187)(300.0293055,86.13421189)(300.07931259,86.09421297)
\curveto(300.1293054,86.05421197)(300.16930536,85.97921205)(300.19931259,85.86921297)
\curveto(300.23930529,85.76921226)(300.24930528,85.66921236)(300.22931259,85.56921297)
\lineto(300.22931259,85.43421297)
\curveto(300.20930532,85.36421266)(300.18930534,85.30421272)(300.16931259,85.25421297)
\curveto(300.14930538,85.20421282)(300.11430541,85.16421286)(300.06431259,85.13421297)
\curveto(300.01430551,85.09421293)(299.94430558,85.07421295)(299.85431259,85.07421297)
\lineto(299.58431259,85.07421297)
\lineto(298.68431259,85.07421297)
\lineto(295.17431259,85.07421297)
\lineto(294.10931259,85.07421297)
\curveto(294.0293115,85.07421295)(293.93931159,85.06921296)(293.83931259,85.05921297)
\curveto(293.73931179,85.05921297)(293.65431187,85.06921296)(293.58431259,85.08921297)
\curveto(293.37431215,85.15921287)(293.30931222,85.33921269)(293.38931259,85.62921297)
\curveto(293.39931213,85.66921236)(293.39931213,85.70421232)(293.38931259,85.73421297)
\curveto(293.38931214,85.77421225)(293.39931213,85.81921221)(293.41931259,85.86921297)
\curveto(293.43931209,85.94921208)(293.45931207,86.03421199)(293.47931259,86.12421297)
\curveto(293.49931203,86.21421181)(293.524312,86.29921173)(293.55431259,86.37921297)
\curveto(293.71431181,86.86921116)(293.91431161,87.28421074)(294.15431259,87.62421297)
\curveto(294.33431119,87.87421015)(294.53931099,88.09920993)(294.76931259,88.29921297)
\curveto(294.99931053,88.50920952)(295.23931029,88.70420932)(295.48931259,88.88421297)
\curveto(295.74930978,89.06420896)(296.01430951,89.23420879)(296.28431259,89.39421297)
\curveto(296.56430896,89.56420846)(296.83430869,89.73920829)(297.09431259,89.91921297)
\curveto(297.20430832,89.99920803)(297.30930822,90.07420795)(297.40931259,90.14421297)
\curveto(297.51930801,90.21420781)(297.6293079,90.28920774)(297.73931259,90.36921297)
\curveto(297.77930775,90.39920763)(297.81430771,90.4292076)(297.84431259,90.45921297)
\curveto(297.88430764,90.49920753)(297.9243076,90.5292075)(297.96431259,90.54921297)
\curveto(298.10430742,90.65920737)(298.2293073,90.78420724)(298.33931259,90.92421297)
\curveto(298.35930717,90.95420707)(298.38430714,90.97920705)(298.41431259,90.99921297)
\curveto(298.44430708,91.029207)(298.46930706,91.05920697)(298.48931259,91.08921297)
\curveto(298.56930696,91.18920684)(298.63430689,91.28920674)(298.68431259,91.38921297)
\curveto(298.74430678,91.48920654)(298.79930673,91.59920643)(298.84931259,91.71921297)
\curveto(298.87930665,91.78920624)(298.89930663,91.86420616)(298.90931259,91.94421297)
\lineto(298.96931259,92.18421297)
\lineto(298.96931259,92.27421297)
\curveto(298.97930655,92.30420572)(298.98430654,92.33420569)(298.98431259,92.36421297)
\curveto(299.00430652,92.43420559)(299.00930652,92.5292055)(298.99931259,92.64921297)
\curveto(298.99930653,92.77920525)(298.98930654,92.87920515)(298.96931259,92.94921297)
\curveto(298.94930658,93.029205)(298.9293066,93.10420492)(298.90931259,93.17421297)
\curveto(298.89930663,93.25420477)(298.87930665,93.33420469)(298.84931259,93.41421297)
\curveto(298.73930679,93.65420437)(298.58930694,93.85420417)(298.39931259,94.01421297)
\curveto(298.21930731,94.18420384)(297.99930753,94.3242037)(297.73931259,94.43421297)
\curveto(297.66930786,94.45420357)(297.59930793,94.46920356)(297.52931259,94.47921297)
\curveto(297.45930807,94.49920353)(297.38430814,94.51920351)(297.30431259,94.53921297)
\curveto(297.2243083,94.55920347)(297.11430841,94.56920346)(296.97431259,94.56921297)
\curveto(296.84430868,94.56920346)(296.73930879,94.55920347)(296.65931259,94.53921297)
\curveto(296.59930893,94.5292035)(296.54430898,94.5242035)(296.49431259,94.52421297)
\curveto(296.44430908,94.5242035)(296.39430913,94.51420351)(296.34431259,94.49421297)
\curveto(296.24430928,94.45420357)(296.14930938,94.41420361)(296.05931259,94.37421297)
\curveto(295.97930955,94.33420369)(295.89930963,94.28920374)(295.81931259,94.23921297)
\curveto(295.78930974,94.21920381)(295.75930977,94.19420383)(295.72931259,94.16421297)
\curveto(295.70930982,94.13420389)(295.68430984,94.10920392)(295.65431259,94.08921297)
\lineto(295.57931259,94.01421297)
\curveto(295.54930998,93.99420403)(295.52431,93.97420405)(295.50431259,93.95421297)
\lineto(295.35431259,93.74421297)
\curveto(295.31431021,93.68420434)(295.26931026,93.61920441)(295.21931259,93.54921297)
\curveto(295.15931037,93.45920457)(295.10931042,93.35420467)(295.06931259,93.23421297)
\curveto(295.03931049,93.1242049)(295.00431052,93.01420501)(294.96431259,92.90421297)
\curveto(294.9243106,92.79420523)(294.89931063,92.64920538)(294.88931259,92.46921297)
\curveto(294.87931065,92.29920573)(294.84931068,92.17420585)(294.79931259,92.09421297)
\curveto(294.74931078,92.01420601)(294.67431085,91.96920606)(294.57431259,91.95921297)
\curveto(294.47431105,91.94920608)(294.36431116,91.94420608)(294.24431259,91.94421297)
\curveto(294.20431132,91.94420608)(294.16431136,91.93920609)(294.12431259,91.92921297)
\curveto(294.08431144,91.9292061)(294.04931148,91.93420609)(294.01931259,91.94421297)
\curveto(293.96931156,91.96420606)(293.91931161,91.97420605)(293.86931259,91.97421297)
\curveto(293.8293117,91.97420605)(293.78931174,91.98420604)(293.74931259,92.00421297)
\curveto(293.65931187,92.06420596)(293.61431191,92.19920583)(293.61431259,92.40921297)
\lineto(293.61431259,92.52921297)
\curveto(293.6243119,92.58920544)(293.6293119,92.64920538)(293.62931259,92.70921297)
\curveto(293.63931189,92.77920525)(293.64931188,92.84420518)(293.65931259,92.90421297)
\curveto(293.67931185,93.01420501)(293.69931183,93.11420491)(293.71931259,93.20421297)
\curveto(293.73931179,93.30420472)(293.76931176,93.39920463)(293.80931259,93.48921297)
\curveto(293.8293117,93.55920447)(293.84931168,93.61920441)(293.86931259,93.66921297)
\lineto(293.92931259,93.84921297)
\curveto(294.04931148,94.10920392)(294.20431132,94.35420367)(294.39431259,94.58421297)
\curveto(294.59431093,94.81420321)(294.80931072,94.99920303)(295.03931259,95.13921297)
\curveto(295.14931038,95.21920281)(295.26431026,95.28420274)(295.38431259,95.33421297)
\lineto(295.77431259,95.48421297)
\curveto(295.88430964,95.53420249)(295.99930953,95.56420246)(296.11931259,95.57421297)
\curveto(296.23930929,95.59420243)(296.36430916,95.61920241)(296.49431259,95.64921297)
\curveto(296.56430896,95.64920238)(296.6293089,95.64920238)(296.68931259,95.64921297)
\curveto(296.74930878,95.65920237)(296.81430871,95.66920236)(296.88431259,95.67921297)
}
}
{
\newrgbcolor{curcolor}{0 0 0}
\pscustom[linestyle=none,fillstyle=solid,fillcolor=curcolor]
{
\newpath
\moveto(302.93892196,86.70921297)
\lineto(303.23892196,86.70921297)
\curveto(303.3489199,86.71921131)(303.4539198,86.71921131)(303.55392196,86.70921297)
\curveto(303.66391959,86.70921132)(303.76391949,86.69921133)(303.85392196,86.67921297)
\curveto(303.94391931,86.66921136)(304.01391924,86.64421138)(304.06392196,86.60421297)
\curveto(304.08391917,86.58421144)(304.09891915,86.55421147)(304.10892196,86.51421297)
\curveto(304.12891912,86.47421155)(304.1489191,86.4292116)(304.16892196,86.37921297)
\lineto(304.16892196,86.30421297)
\curveto(304.17891907,86.25421177)(304.17891907,86.19921183)(304.16892196,86.13921297)
\lineto(304.16892196,85.98921297)
\lineto(304.16892196,85.50921297)
\curveto(304.16891908,85.33921269)(304.12891912,85.21921281)(304.04892196,85.14921297)
\curveto(303.97891927,85.09921293)(303.88891936,85.07421295)(303.77892196,85.07421297)
\lineto(303.44892196,85.07421297)
\lineto(302.99892196,85.07421297)
\curveto(302.8489204,85.07421295)(302.73392052,85.10421292)(302.65392196,85.16421297)
\curveto(302.61392064,85.19421283)(302.58392067,85.24421278)(302.56392196,85.31421297)
\curveto(302.54392071,85.39421263)(302.52892072,85.47921255)(302.51892196,85.56921297)
\lineto(302.51892196,85.85421297)
\curveto(302.52892072,85.95421207)(302.53392072,86.03921199)(302.53392196,86.10921297)
\lineto(302.53392196,86.30421297)
\curveto(302.53392072,86.36421166)(302.54392071,86.41921161)(302.56392196,86.46921297)
\curveto(302.60392065,86.57921145)(302.67392058,86.64921138)(302.77392196,86.67921297)
\curveto(302.80392045,86.67921135)(302.85892039,86.68921134)(302.93892196,86.70921297)
}
}
{
\newrgbcolor{curcolor}{0 0 0}
\pscustom[linestyle=none,fillstyle=solid,fillcolor=curcolor]
{
\newpath
\moveto(310.20407821,95.67921297)
\curveto(310.30407336,95.67920235)(310.39907326,95.66920236)(310.48907821,95.64921297)
\curveto(310.57907308,95.63920239)(310.64407302,95.60920242)(310.68407821,95.55921297)
\curveto(310.74407292,95.47920255)(310.77407289,95.37420265)(310.77407821,95.24421297)
\lineto(310.77407821,94.85421297)
\lineto(310.77407821,93.35421297)
\lineto(310.77407821,86.96421297)
\lineto(310.77407821,85.79421297)
\lineto(310.77407821,85.47921297)
\curveto(310.78407288,85.37921265)(310.76907289,85.29921273)(310.72907821,85.23921297)
\curveto(310.67907298,85.15921287)(310.60407306,85.10921292)(310.50407821,85.08921297)
\curveto(310.41407325,85.07921295)(310.30407336,85.07421295)(310.17407821,85.07421297)
\lineto(309.94907821,85.07421297)
\curveto(309.86907379,85.09421293)(309.79907386,85.10921292)(309.73907821,85.11921297)
\curveto(309.67907398,85.13921289)(309.62907403,85.17921285)(309.58907821,85.23921297)
\curveto(309.54907411,85.29921273)(309.52907413,85.37421265)(309.52907821,85.46421297)
\lineto(309.52907821,85.76421297)
\lineto(309.52907821,86.85921297)
\lineto(309.52907821,92.19921297)
\curveto(309.50907415,92.28920574)(309.49407417,92.36420566)(309.48407821,92.42421297)
\curveto(309.48407418,92.49420553)(309.45407421,92.55420547)(309.39407821,92.60421297)
\curveto(309.32407434,92.65420537)(309.23407443,92.67920535)(309.12407821,92.67921297)
\curveto(309.02407464,92.68920534)(308.91407475,92.69420533)(308.79407821,92.69421297)
\lineto(307.65407821,92.69421297)
\lineto(307.15907821,92.69421297)
\curveto(306.99907666,92.70420532)(306.88907677,92.76420526)(306.82907821,92.87421297)
\curveto(306.80907685,92.90420512)(306.79907686,92.93420509)(306.79907821,92.96421297)
\curveto(306.79907686,93.00420502)(306.79407687,93.04920498)(306.78407821,93.09921297)
\curveto(306.7640769,93.21920481)(306.76907689,93.3292047)(306.79907821,93.42921297)
\curveto(306.83907682,93.5292045)(306.89407677,93.59920443)(306.96407821,93.63921297)
\curveto(307.04407662,93.68920434)(307.1640765,93.71420431)(307.32407821,93.71421297)
\curveto(307.48407618,93.71420431)(307.61907604,93.7292043)(307.72907821,93.75921297)
\curveto(307.77907588,93.76920426)(307.83407583,93.77420425)(307.89407821,93.77421297)
\curveto(307.95407571,93.78420424)(308.01407565,93.79920423)(308.07407821,93.81921297)
\curveto(308.22407544,93.86920416)(308.36907529,93.91920411)(308.50907821,93.96921297)
\curveto(308.64907501,94.029204)(308.78407488,94.09920393)(308.91407821,94.17921297)
\curveto(309.05407461,94.26920376)(309.17407449,94.37420365)(309.27407821,94.49421297)
\curveto(309.37407429,94.61420341)(309.46907419,94.74420328)(309.55907821,94.88421297)
\curveto(309.61907404,94.98420304)(309.664074,95.09420293)(309.69407821,95.21421297)
\curveto(309.73407393,95.33420269)(309.78407388,95.43920259)(309.84407821,95.52921297)
\curveto(309.89407377,95.58920244)(309.9640737,95.6292024)(310.05407821,95.64921297)
\curveto(310.07407359,95.65920237)(310.09907356,95.66420236)(310.12907821,95.66421297)
\curveto(310.1590735,95.66420236)(310.18407348,95.66920236)(310.20407821,95.67921297)
}
}
{
\newrgbcolor{curcolor}{0 0 0}
\pscustom[linestyle=none,fillstyle=solid,fillcolor=curcolor]
{
\newpath
\moveto(324.29868759,93.59421297)
\curveto(324.09867729,93.30420472)(323.8886775,93.01920501)(323.66868759,92.73921297)
\curveto(323.45867793,92.45920557)(323.25367813,92.17420585)(323.05368759,91.88421297)
\curveto(322.45367893,91.03420699)(321.84867954,90.19420783)(321.23868759,89.36421297)
\curveto(320.62868076,88.54420948)(320.02368136,87.70921032)(319.42368759,86.85921297)
\lineto(318.91368759,86.13921297)
\lineto(318.40368759,85.44921297)
\curveto(318.32368306,85.33921269)(318.24368314,85.2242128)(318.16368759,85.10421297)
\curveto(318.0836833,84.98421304)(317.9886834,84.88921314)(317.87868759,84.81921297)
\curveto(317.83868355,84.79921323)(317.77368361,84.78421324)(317.68368759,84.77421297)
\curveto(317.60368378,84.75421327)(317.51368387,84.74421328)(317.41368759,84.74421297)
\curveto(317.31368407,84.74421328)(317.21868417,84.74921328)(317.12868759,84.75921297)
\curveto(317.04868434,84.76921326)(316.9886844,84.78921324)(316.94868759,84.81921297)
\curveto(316.91868447,84.83921319)(316.89368449,84.87421315)(316.87368759,84.92421297)
\curveto(316.86368452,84.96421306)(316.86868452,85.00921302)(316.88868759,85.05921297)
\curveto(316.92868446,85.13921289)(316.97368441,85.21421281)(317.02368759,85.28421297)
\curveto(317.0836843,85.36421266)(317.13868425,85.44421258)(317.18868759,85.52421297)
\curveto(317.42868396,85.86421216)(317.67368371,86.19921183)(317.92368759,86.52921297)
\curveto(318.17368321,86.85921117)(318.41368297,87.19421083)(318.64368759,87.53421297)
\curveto(318.80368258,87.75421027)(318.96368242,87.96921006)(319.12368759,88.17921297)
\curveto(319.2836821,88.38920964)(319.44368194,88.60420942)(319.60368759,88.82421297)
\curveto(319.96368142,89.34420868)(320.32868106,89.85420817)(320.69868759,90.35421297)
\curveto(321.06868032,90.85420717)(321.43867995,91.36420666)(321.80868759,91.88421297)
\curveto(321.94867944,92.08420594)(322.0886793,92.27920575)(322.22868759,92.46921297)
\curveto(322.37867901,92.65920537)(322.52367886,92.85420517)(322.66368759,93.05421297)
\curveto(322.87367851,93.35420467)(323.0886783,93.65420437)(323.30868759,93.95421297)
\lineto(323.96868759,94.85421297)
\lineto(324.14868759,95.12421297)
\lineto(324.35868759,95.39421297)
\lineto(324.47868759,95.57421297)
\curveto(324.52867686,95.63420239)(324.57867681,95.68920234)(324.62868759,95.73921297)
\curveto(324.69867669,95.78920224)(324.77367661,95.8242022)(324.85368759,95.84421297)
\curveto(324.87367651,95.85420217)(324.89867649,95.85420217)(324.92868759,95.84421297)
\curveto(324.96867642,95.84420218)(324.99867639,95.85420217)(325.01868759,95.87421297)
\curveto(325.13867625,95.87420215)(325.27367611,95.86920216)(325.42368759,95.85921297)
\curveto(325.57367581,95.85920217)(325.66367572,95.81420221)(325.69368759,95.72421297)
\curveto(325.71367567,95.69420233)(325.71867567,95.65920237)(325.70868759,95.61921297)
\curveto(325.69867569,95.57920245)(325.6836757,95.54920248)(325.66368759,95.52921297)
\curveto(325.62367576,95.44920258)(325.5836758,95.37920265)(325.54368759,95.31921297)
\curveto(325.50367588,95.25920277)(325.45867593,95.19920283)(325.40868759,95.13921297)
\lineto(324.83868759,94.35921297)
\curveto(324.65867673,94.10920392)(324.47867691,93.85420417)(324.29868759,93.59421297)
\moveto(317.44368759,89.69421297)
\curveto(317.39368399,89.71420831)(317.34368404,89.71920831)(317.29368759,89.70921297)
\curveto(317.24368414,89.69920833)(317.19368419,89.70420832)(317.14368759,89.72421297)
\curveto(317.03368435,89.74420828)(316.92868446,89.76420826)(316.82868759,89.78421297)
\curveto(316.73868465,89.81420821)(316.64368474,89.85420817)(316.54368759,89.90421297)
\curveto(316.21368517,90.04420798)(315.95868543,90.23920779)(315.77868759,90.48921297)
\curveto(315.59868579,90.74920728)(315.45368593,91.05920697)(315.34368759,91.41921297)
\curveto(315.31368607,91.49920653)(315.29368609,91.57920645)(315.28368759,91.65921297)
\curveto(315.27368611,91.74920628)(315.25868613,91.83420619)(315.23868759,91.91421297)
\curveto(315.22868616,91.96420606)(315.22368616,92.029206)(315.22368759,92.10921297)
\curveto(315.21368617,92.13920589)(315.20868618,92.16920586)(315.20868759,92.19921297)
\curveto(315.20868618,92.23920579)(315.20368618,92.27420575)(315.19368759,92.30421297)
\lineto(315.19368759,92.45421297)
\curveto(315.1836862,92.50420552)(315.17868621,92.56420546)(315.17868759,92.63421297)
\curveto(315.17868621,92.71420531)(315.1836862,92.77920525)(315.19368759,92.82921297)
\lineto(315.19368759,92.99421297)
\curveto(315.21368617,93.04420498)(315.21868617,93.08920494)(315.20868759,93.12921297)
\curveto(315.20868618,93.17920485)(315.21368617,93.2242048)(315.22368759,93.26421297)
\curveto(315.23368615,93.30420472)(315.23868615,93.33920469)(315.23868759,93.36921297)
\curveto(315.23868615,93.40920462)(315.24368614,93.44920458)(315.25368759,93.48921297)
\curveto(315.2836861,93.59920443)(315.30368608,93.70920432)(315.31368759,93.81921297)
\curveto(315.33368605,93.93920409)(315.36868602,94.05420397)(315.41868759,94.16421297)
\curveto(315.55868583,94.50420352)(315.71868567,94.77920325)(315.89868759,94.98921297)
\curveto(316.0886853,95.20920282)(316.35868503,95.38920264)(316.70868759,95.52921297)
\curveto(316.7886846,95.55920247)(316.87368451,95.57920245)(316.96368759,95.58921297)
\curveto(317.05368433,95.60920242)(317.14868424,95.6292024)(317.24868759,95.64921297)
\curveto(317.27868411,95.65920237)(317.33368405,95.65920237)(317.41368759,95.64921297)
\curveto(317.49368389,95.64920238)(317.54368384,95.65920237)(317.56368759,95.67921297)
\curveto(318.12368326,95.68920234)(318.57368281,95.57920245)(318.91368759,95.34921297)
\curveto(319.26368212,95.11920291)(319.52368186,94.81420321)(319.69368759,94.43421297)
\curveto(319.73368165,94.34420368)(319.76868162,94.24920378)(319.79868759,94.14921297)
\curveto(319.82868156,94.04920398)(319.85368153,93.94920408)(319.87368759,93.84921297)
\curveto(319.89368149,93.81920421)(319.89868149,93.78920424)(319.88868759,93.75921297)
\curveto(319.8886815,93.7292043)(319.89368149,93.69920433)(319.90368759,93.66921297)
\curveto(319.93368145,93.55920447)(319.95368143,93.43420459)(319.96368759,93.29421297)
\curveto(319.97368141,93.16420486)(319.9836814,93.029205)(319.99368759,92.88921297)
\lineto(319.99368759,92.72421297)
\curveto(320.00368138,92.66420536)(320.00368138,92.60920542)(319.99368759,92.55921297)
\curveto(319.9836814,92.50920552)(319.97868141,92.45920557)(319.97868759,92.40921297)
\lineto(319.97868759,92.27421297)
\curveto(319.96868142,92.23420579)(319.96368142,92.19420583)(319.96368759,92.15421297)
\curveto(319.97368141,92.11420591)(319.96868142,92.06920596)(319.94868759,92.01921297)
\curveto(319.92868146,91.90920612)(319.90868148,91.80420622)(319.88868759,91.70421297)
\curveto(319.87868151,91.60420642)(319.85868153,91.50420652)(319.82868759,91.40421297)
\curveto(319.69868169,91.04420698)(319.53368185,90.7292073)(319.33368759,90.45921297)
\curveto(319.13368225,90.18920784)(318.85868253,89.98420804)(318.50868759,89.84421297)
\curveto(318.42868296,89.81420821)(318.34368304,89.78920824)(318.25368759,89.76921297)
\lineto(317.98368759,89.70921297)
\curveto(317.93368345,89.69920833)(317.8886835,89.69420833)(317.84868759,89.69421297)
\curveto(317.80868358,89.70420832)(317.76868362,89.70420832)(317.72868759,89.69421297)
\curveto(317.62868376,89.67420835)(317.53368385,89.67420835)(317.44368759,89.69421297)
\moveto(316.60368759,91.08921297)
\curveto(316.64368474,91.01920701)(316.6836847,90.95420707)(316.72368759,90.89421297)
\curveto(316.76368462,90.84420718)(316.81368457,90.79420723)(316.87368759,90.74421297)
\lineto(317.02368759,90.62421297)
\curveto(317.0836843,90.59420743)(317.14868424,90.56920746)(317.21868759,90.54921297)
\curveto(317.25868413,90.5292075)(317.29368409,90.51920751)(317.32368759,90.51921297)
\curveto(317.36368402,90.5292075)(317.40368398,90.5242075)(317.44368759,90.50421297)
\curveto(317.47368391,90.50420752)(317.51368387,90.49920753)(317.56368759,90.48921297)
\curveto(317.61368377,90.48920754)(317.65368373,90.49420753)(317.68368759,90.50421297)
\lineto(317.90868759,90.54921297)
\curveto(318.15868323,90.6292074)(318.34368304,90.75420727)(318.46368759,90.92421297)
\curveto(318.54368284,91.024207)(318.61368277,91.15420687)(318.67368759,91.31421297)
\curveto(318.75368263,91.49420653)(318.81368257,91.71920631)(318.85368759,91.98921297)
\curveto(318.89368249,92.26920576)(318.90868248,92.54920548)(318.89868759,92.82921297)
\curveto(318.8886825,93.11920491)(318.85868253,93.39420463)(318.80868759,93.65421297)
\curveto(318.75868263,93.91420411)(318.6836827,94.1242039)(318.58368759,94.28421297)
\curveto(318.46368292,94.48420354)(318.31368307,94.63420339)(318.13368759,94.73421297)
\curveto(318.05368333,94.78420324)(317.96368342,94.81420321)(317.86368759,94.82421297)
\curveto(317.76368362,94.84420318)(317.65868373,94.85420317)(317.54868759,94.85421297)
\curveto(317.52868386,94.84420318)(317.50368388,94.83920319)(317.47368759,94.83921297)
\curveto(317.45368393,94.84920318)(317.43368395,94.84920318)(317.41368759,94.83921297)
\curveto(317.36368402,94.8292032)(317.31868407,94.81920321)(317.27868759,94.80921297)
\curveto(317.23868415,94.80920322)(317.19868419,94.79920323)(317.15868759,94.77921297)
\curveto(316.97868441,94.69920333)(316.82868456,94.57920345)(316.70868759,94.41921297)
\curveto(316.59868479,94.25920377)(316.50868488,94.07920395)(316.43868759,93.87921297)
\curveto(316.37868501,93.68920434)(316.33368505,93.46420456)(316.30368759,93.20421297)
\curveto(316.2836851,92.94420508)(316.27868511,92.67920535)(316.28868759,92.40921297)
\curveto(316.29868509,92.14920588)(316.32868506,91.89920613)(316.37868759,91.65921297)
\curveto(316.43868495,91.4292066)(316.51368487,91.23920679)(316.60368759,91.08921297)
\moveto(327.40368759,88.10421297)
\curveto(327.41367397,88.05420997)(327.41867397,87.96421006)(327.41868759,87.83421297)
\curveto(327.41867397,87.70421032)(327.40867398,87.61421041)(327.38868759,87.56421297)
\curveto(327.36867402,87.51421051)(327.36367402,87.45921057)(327.37368759,87.39921297)
\curveto(327.383674,87.34921068)(327.383674,87.29921073)(327.37368759,87.24921297)
\curveto(327.33367405,87.10921092)(327.30367408,86.97421105)(327.28368759,86.84421297)
\curveto(327.27367411,86.71421131)(327.24367414,86.59421143)(327.19368759,86.48421297)
\curveto(327.05367433,86.13421189)(326.8886745,85.83921219)(326.69868759,85.59921297)
\curveto(326.50867488,85.36921266)(326.23867515,85.18421284)(325.88868759,85.04421297)
\curveto(325.80867558,85.01421301)(325.72367566,84.99421303)(325.63368759,84.98421297)
\curveto(325.54367584,84.96421306)(325.45867593,84.94421308)(325.37868759,84.92421297)
\curveto(325.32867606,84.91421311)(325.27867611,84.90921312)(325.22868759,84.90921297)
\curveto(325.17867621,84.90921312)(325.12867626,84.90421312)(325.07868759,84.89421297)
\curveto(325.04867634,84.88421314)(324.99867639,84.88421314)(324.92868759,84.89421297)
\curveto(324.85867653,84.89421313)(324.80867658,84.89921313)(324.77868759,84.90921297)
\curveto(324.71867667,84.9292131)(324.65867673,84.93921309)(324.59868759,84.93921297)
\curveto(324.54867684,84.9292131)(324.49867689,84.93421309)(324.44868759,84.95421297)
\curveto(324.35867703,84.97421305)(324.26867712,84.99921303)(324.17868759,85.02921297)
\curveto(324.09867729,85.04921298)(324.01867737,85.07921295)(323.93868759,85.11921297)
\curveto(323.61867777,85.25921277)(323.36867802,85.45421257)(323.18868759,85.70421297)
\curveto(323.00867838,85.96421206)(322.85867853,86.26921176)(322.73868759,86.61921297)
\curveto(322.71867867,86.69921133)(322.70367868,86.78421124)(322.69368759,86.87421297)
\curveto(322.6836787,86.96421106)(322.66867872,87.04921098)(322.64868759,87.12921297)
\curveto(322.63867875,87.15921087)(322.63367875,87.18921084)(322.63368759,87.21921297)
\lineto(322.63368759,87.32421297)
\curveto(322.61367877,87.40421062)(322.60367878,87.48421054)(322.60368759,87.56421297)
\lineto(322.60368759,87.69921297)
\curveto(322.5836788,87.79921023)(322.5836788,87.89921013)(322.60368759,87.99921297)
\lineto(322.60368759,88.17921297)
\curveto(322.61367877,88.2292098)(322.61867877,88.27420975)(322.61868759,88.31421297)
\curveto(322.61867877,88.36420966)(322.62367876,88.40920962)(322.63368759,88.44921297)
\curveto(322.64367874,88.48920954)(322.64867874,88.5242095)(322.64868759,88.55421297)
\curveto(322.64867874,88.59420943)(322.65367873,88.63420939)(322.66368759,88.67421297)
\lineto(322.72368759,89.00421297)
\curveto(322.74367864,89.1242089)(322.77367861,89.23420879)(322.81368759,89.33421297)
\curveto(322.95367843,89.66420836)(323.11367827,89.93920809)(323.29368759,90.15921297)
\curveto(323.4836779,90.38920764)(323.74367764,90.57420745)(324.07368759,90.71421297)
\curveto(324.15367723,90.75420727)(324.23867715,90.77920725)(324.32868759,90.78921297)
\lineto(324.62868759,90.84921297)
\lineto(324.76368759,90.84921297)
\curveto(324.81367657,90.85920717)(324.86367652,90.86420716)(324.91368759,90.86421297)
\curveto(325.4836759,90.88420714)(325.94367544,90.77920725)(326.29368759,90.54921297)
\curveto(326.65367473,90.3292077)(326.91867447,90.029208)(327.08868759,89.64921297)
\curveto(327.13867425,89.54920848)(327.17867421,89.44920858)(327.20868759,89.34921297)
\curveto(327.23867415,89.24920878)(327.26867412,89.14420888)(327.29868759,89.03421297)
\curveto(327.30867408,88.99420903)(327.31367407,88.95920907)(327.31368759,88.92921297)
\curveto(327.31367407,88.90920912)(327.31867407,88.87920915)(327.32868759,88.83921297)
\curveto(327.34867404,88.76920926)(327.35867403,88.69420933)(327.35868759,88.61421297)
\curveto(327.35867403,88.53420949)(327.36867402,88.45420957)(327.38868759,88.37421297)
\curveto(327.388674,88.3242097)(327.388674,88.27920975)(327.38868759,88.23921297)
\curveto(327.388674,88.19920983)(327.39367399,88.15420987)(327.40368759,88.10421297)
\moveto(326.29368759,87.66921297)
\curveto(326.30367508,87.71921031)(326.30867508,87.79421023)(326.30868759,87.89421297)
\curveto(326.31867507,87.99421003)(326.31367507,88.06920996)(326.29368759,88.11921297)
\curveto(326.27367511,88.17920985)(326.26867512,88.23420979)(326.27868759,88.28421297)
\curveto(326.29867509,88.34420968)(326.29867509,88.40420962)(326.27868759,88.46421297)
\curveto(326.26867512,88.49420953)(326.26367512,88.5292095)(326.26368759,88.56921297)
\curveto(326.26367512,88.60920942)(326.25867513,88.64920938)(326.24868759,88.68921297)
\curveto(326.22867516,88.76920926)(326.20867518,88.84420918)(326.18868759,88.91421297)
\curveto(326.17867521,88.99420903)(326.16367522,89.07420895)(326.14368759,89.15421297)
\curveto(326.11367527,89.21420881)(326.0886753,89.27420875)(326.06868759,89.33421297)
\curveto(326.04867534,89.39420863)(326.01867537,89.45420857)(325.97868759,89.51421297)
\curveto(325.87867551,89.68420834)(325.74867564,89.81920821)(325.58868759,89.91921297)
\curveto(325.50867588,89.96920806)(325.41367597,90.00420802)(325.30368759,90.02421297)
\curveto(325.19367619,90.04420798)(325.06867632,90.05420797)(324.92868759,90.05421297)
\curveto(324.90867648,90.04420798)(324.8836765,90.03920799)(324.85368759,90.03921297)
\curveto(324.82367656,90.04920798)(324.79367659,90.04920798)(324.76368759,90.03921297)
\lineto(324.61368759,89.97921297)
\curveto(324.56367682,89.96920806)(324.51867687,89.95420807)(324.47868759,89.93421297)
\curveto(324.2886771,89.8242082)(324.14367724,89.67920835)(324.04368759,89.49921297)
\curveto(323.95367743,89.31920871)(323.87367751,89.11420891)(323.80368759,88.88421297)
\curveto(323.76367762,88.75420927)(323.74367764,88.61920941)(323.74368759,88.47921297)
\curveto(323.74367764,88.34920968)(323.73367765,88.20420982)(323.71368759,88.04421297)
\curveto(323.70367768,87.99421003)(323.69367769,87.93421009)(323.68368759,87.86421297)
\curveto(323.6836777,87.79421023)(323.69367769,87.73421029)(323.71368759,87.68421297)
\lineto(323.71368759,87.51921297)
\lineto(323.71368759,87.33921297)
\curveto(323.72367766,87.28921074)(323.73367765,87.23421079)(323.74368759,87.17421297)
\curveto(323.75367763,87.1242109)(323.75867763,87.06921096)(323.75868759,87.00921297)
\curveto(323.76867762,86.94921108)(323.7836776,86.89421113)(323.80368759,86.84421297)
\curveto(323.85367753,86.65421137)(323.91367747,86.47921155)(323.98368759,86.31921297)
\curveto(324.05367733,86.15921187)(324.15867723,86.029212)(324.29868759,85.92921297)
\curveto(324.42867696,85.8292122)(324.56867682,85.75921227)(324.71868759,85.71921297)
\curveto(324.74867664,85.70921232)(324.77367661,85.70421232)(324.79368759,85.70421297)
\curveto(324.82367656,85.71421231)(324.85367653,85.71421231)(324.88368759,85.70421297)
\curveto(324.90367648,85.70421232)(324.93367645,85.69921233)(324.97368759,85.68921297)
\curveto(325.01367637,85.68921234)(325.04867634,85.69421233)(325.07868759,85.70421297)
\curveto(325.11867627,85.71421231)(325.15867623,85.71921231)(325.19868759,85.71921297)
\curveto(325.23867615,85.71921231)(325.27867611,85.7292123)(325.31868759,85.74921297)
\curveto(325.55867583,85.8292122)(325.75367563,85.96421206)(325.90368759,86.15421297)
\curveto(326.02367536,86.33421169)(326.11367527,86.53921149)(326.17368759,86.76921297)
\curveto(326.19367519,86.83921119)(326.20867518,86.90921112)(326.21868759,86.97921297)
\curveto(326.22867516,87.05921097)(326.24367514,87.13921089)(326.26368759,87.21921297)
\curveto(326.26367512,87.27921075)(326.26867512,87.3242107)(326.27868759,87.35421297)
\curveto(326.27867511,87.37421065)(326.27867511,87.39921063)(326.27868759,87.42921297)
\curveto(326.27867511,87.46921056)(326.2836751,87.49921053)(326.29368759,87.51921297)
\lineto(326.29368759,87.66921297)
}
}
{
\newrgbcolor{curcolor}{0 0 0}
\pscustom[linestyle=none,fillstyle=solid,fillcolor=curcolor]
{
\newpath
\moveto(164.79102035,55.27308504)
\curveto(165.48101572,55.28307441)(166.08101512,55.16307453)(166.59102035,54.91308504)
\curveto(167.11101409,54.66307503)(167.50601369,54.32807536)(167.77602035,53.90808504)
\curveto(167.82601337,53.82807586)(167.87101333,53.73807595)(167.91102035,53.63808504)
\curveto(167.95101325,53.54807614)(167.9960132,53.45307624)(168.04602035,53.35308504)
\curveto(168.08601311,53.25307644)(168.11601308,53.15307654)(168.13602035,53.05308504)
\curveto(168.15601304,52.95307674)(168.17601302,52.84807684)(168.19602035,52.73808504)
\curveto(168.21601298,52.688077)(168.22101298,52.64307705)(168.21102035,52.60308504)
\curveto(168.201013,52.56307713)(168.20601299,52.51807717)(168.22602035,52.46808504)
\curveto(168.23601296,52.41807727)(168.24101296,52.33307736)(168.24102035,52.21308504)
\curveto(168.24101296,52.10307759)(168.23601296,52.01807767)(168.22602035,51.95808504)
\curveto(168.20601299,51.89807779)(168.196013,51.83807785)(168.19602035,51.77808504)
\curveto(168.20601299,51.71807797)(168.201013,51.65807803)(168.18102035,51.59808504)
\curveto(168.14101306,51.45807823)(168.10601309,51.32307837)(168.07602035,51.19308504)
\curveto(168.04601315,51.06307863)(168.00601319,50.93807875)(167.95602035,50.81808504)
\curveto(167.8960133,50.67807901)(167.82601337,50.55307914)(167.74602035,50.44308504)
\curveto(167.67601352,50.33307936)(167.6010136,50.22307947)(167.52102035,50.11308504)
\lineto(167.46102035,50.05308504)
\curveto(167.45101375,50.03307966)(167.43601376,50.01307968)(167.41602035,49.99308504)
\curveto(167.2960139,49.83307986)(167.16101404,49.68808)(167.01102035,49.55808504)
\curveto(166.86101434,49.42808026)(166.7010145,49.30308039)(166.53102035,49.18308504)
\curveto(166.22101498,48.96308073)(165.92601527,48.75808093)(165.64602035,48.56808504)
\curveto(165.41601578,48.42808126)(165.18601601,48.2930814)(164.95602035,48.16308504)
\curveto(164.73601646,48.03308166)(164.51601668,47.89808179)(164.29602035,47.75808504)
\curveto(164.04601715,47.5880821)(163.80601739,47.40808228)(163.57602035,47.21808504)
\curveto(163.35601784,47.02808266)(163.16601803,46.80308289)(163.00602035,46.54308504)
\curveto(162.96601823,46.48308321)(162.93101827,46.42308327)(162.90102035,46.36308504)
\curveto(162.87101833,46.31308338)(162.84101836,46.24808344)(162.81102035,46.16808504)
\curveto(162.79101841,46.09808359)(162.78601841,46.03808365)(162.79602035,45.98808504)
\curveto(162.81601838,45.91808377)(162.85101835,45.86308383)(162.90102035,45.82308504)
\curveto(162.95101825,45.7930839)(163.01101819,45.77308392)(163.08102035,45.76308504)
\lineto(163.32102035,45.76308504)
\lineto(164.07102035,45.76308504)
\lineto(166.87602035,45.76308504)
\lineto(167.53602035,45.76308504)
\curveto(167.62601357,45.76308393)(167.71101349,45.75808393)(167.79102035,45.74808504)
\curveto(167.87101333,45.74808394)(167.93601326,45.72808396)(167.98602035,45.68808504)
\curveto(168.03601316,45.64808404)(168.07601312,45.57308412)(168.10602035,45.46308504)
\curveto(168.14601305,45.36308433)(168.15601304,45.26308443)(168.13602035,45.16308504)
\lineto(168.13602035,45.02808504)
\curveto(168.11601308,44.95808473)(168.0960131,44.89808479)(168.07602035,44.84808504)
\curveto(168.05601314,44.79808489)(168.02101318,44.75808493)(167.97102035,44.72808504)
\curveto(167.92101328,44.688085)(167.85101335,44.66808502)(167.76102035,44.66808504)
\lineto(167.49102035,44.66808504)
\lineto(166.59102035,44.66808504)
\lineto(163.08102035,44.66808504)
\lineto(162.01602035,44.66808504)
\curveto(161.93601926,44.66808502)(161.84601935,44.66308503)(161.74602035,44.65308504)
\curveto(161.64601955,44.65308504)(161.56101964,44.66308503)(161.49102035,44.68308504)
\curveto(161.28101992,44.75308494)(161.21601998,44.93308476)(161.29602035,45.22308504)
\curveto(161.30601989,45.26308443)(161.30601989,45.29808439)(161.29602035,45.32808504)
\curveto(161.2960199,45.36808432)(161.30601989,45.41308428)(161.32602035,45.46308504)
\curveto(161.34601985,45.54308415)(161.36601983,45.62808406)(161.38602035,45.71808504)
\curveto(161.40601979,45.80808388)(161.43101977,45.8930838)(161.46102035,45.97308504)
\curveto(161.62101958,46.46308323)(161.82101938,46.87808281)(162.06102035,47.21808504)
\curveto(162.24101896,47.46808222)(162.44601875,47.693082)(162.67602035,47.89308504)
\curveto(162.90601829,48.10308159)(163.14601805,48.29808139)(163.39602035,48.47808504)
\curveto(163.65601754,48.65808103)(163.92101728,48.82808086)(164.19102035,48.98808504)
\curveto(164.47101673,49.15808053)(164.74101646,49.33308036)(165.00102035,49.51308504)
\curveto(165.11101609,49.5930801)(165.21601598,49.66808002)(165.31602035,49.73808504)
\curveto(165.42601577,49.80807988)(165.53601566,49.88307981)(165.64602035,49.96308504)
\curveto(165.68601551,49.9930797)(165.72101548,50.02307967)(165.75102035,50.05308504)
\curveto(165.79101541,50.0930796)(165.83101537,50.12307957)(165.87102035,50.14308504)
\curveto(166.01101519,50.25307944)(166.13601506,50.37807931)(166.24602035,50.51808504)
\curveto(166.26601493,50.54807914)(166.29101491,50.57307912)(166.32102035,50.59308504)
\curveto(166.35101485,50.62307907)(166.37601482,50.65307904)(166.39602035,50.68308504)
\curveto(166.47601472,50.78307891)(166.54101466,50.88307881)(166.59102035,50.98308504)
\curveto(166.65101455,51.08307861)(166.70601449,51.1930785)(166.75602035,51.31308504)
\curveto(166.78601441,51.38307831)(166.80601439,51.45807823)(166.81602035,51.53808504)
\lineto(166.87602035,51.77808504)
\lineto(166.87602035,51.86808504)
\curveto(166.88601431,51.89807779)(166.89101431,51.92807776)(166.89102035,51.95808504)
\curveto(166.91101429,52.02807766)(166.91601428,52.12307757)(166.90602035,52.24308504)
\curveto(166.90601429,52.37307732)(166.8960143,52.47307722)(166.87602035,52.54308504)
\curveto(166.85601434,52.62307707)(166.83601436,52.69807699)(166.81602035,52.76808504)
\curveto(166.80601439,52.84807684)(166.78601441,52.92807676)(166.75602035,53.00808504)
\curveto(166.64601455,53.24807644)(166.4960147,53.44807624)(166.30602035,53.60808504)
\curveto(166.12601507,53.77807591)(165.90601529,53.91807577)(165.64602035,54.02808504)
\curveto(165.57601562,54.04807564)(165.50601569,54.06307563)(165.43602035,54.07308504)
\curveto(165.36601583,54.0930756)(165.29101591,54.11307558)(165.21102035,54.13308504)
\curveto(165.13101607,54.15307554)(165.02101618,54.16307553)(164.88102035,54.16308504)
\curveto(164.75101645,54.16307553)(164.64601655,54.15307554)(164.56602035,54.13308504)
\curveto(164.50601669,54.12307557)(164.45101675,54.11807557)(164.40102035,54.11808504)
\curveto(164.35101685,54.11807557)(164.3010169,54.10807558)(164.25102035,54.08808504)
\curveto(164.15101705,54.04807564)(164.05601714,54.00807568)(163.96602035,53.96808504)
\curveto(163.88601731,53.92807576)(163.80601739,53.88307581)(163.72602035,53.83308504)
\curveto(163.6960175,53.81307588)(163.66601753,53.7880759)(163.63602035,53.75808504)
\curveto(163.61601758,53.72807596)(163.59101761,53.70307599)(163.56102035,53.68308504)
\lineto(163.48602035,53.60808504)
\curveto(163.45601774,53.5880761)(163.43101777,53.56807612)(163.41102035,53.54808504)
\lineto(163.26102035,53.33808504)
\curveto(163.22101798,53.27807641)(163.17601802,53.21307648)(163.12602035,53.14308504)
\curveto(163.06601813,53.05307664)(163.01601818,52.94807674)(162.97602035,52.82808504)
\curveto(162.94601825,52.71807697)(162.91101829,52.60807708)(162.87102035,52.49808504)
\curveto(162.83101837,52.3880773)(162.80601839,52.24307745)(162.79602035,52.06308504)
\curveto(162.78601841,51.8930778)(162.75601844,51.76807792)(162.70602035,51.68808504)
\curveto(162.65601854,51.60807808)(162.58101862,51.56307813)(162.48102035,51.55308504)
\curveto(162.38101882,51.54307815)(162.27101893,51.53807815)(162.15102035,51.53808504)
\curveto(162.11101909,51.53807815)(162.07101913,51.53307816)(162.03102035,51.52308504)
\curveto(161.99101921,51.52307817)(161.95601924,51.52807816)(161.92602035,51.53808504)
\curveto(161.87601932,51.55807813)(161.82601937,51.56807812)(161.77602035,51.56808504)
\curveto(161.73601946,51.56807812)(161.6960195,51.57807811)(161.65602035,51.59808504)
\curveto(161.56601963,51.65807803)(161.52101968,51.7930779)(161.52102035,52.00308504)
\lineto(161.52102035,52.12308504)
\curveto(161.53101967,52.18307751)(161.53601966,52.24307745)(161.53602035,52.30308504)
\curveto(161.54601965,52.37307732)(161.55601964,52.43807725)(161.56602035,52.49808504)
\curveto(161.58601961,52.60807708)(161.60601959,52.70807698)(161.62602035,52.79808504)
\curveto(161.64601955,52.89807679)(161.67601952,52.9930767)(161.71602035,53.08308504)
\curveto(161.73601946,53.15307654)(161.75601944,53.21307648)(161.77602035,53.26308504)
\lineto(161.83602035,53.44308504)
\curveto(161.95601924,53.70307599)(162.11101909,53.94807574)(162.30102035,54.17808504)
\curveto(162.5010187,54.40807528)(162.71601848,54.5930751)(162.94602035,54.73308504)
\curveto(163.05601814,54.81307488)(163.17101803,54.87807481)(163.29102035,54.92808504)
\lineto(163.68102035,55.07808504)
\curveto(163.79101741,55.12807456)(163.90601729,55.15807453)(164.02602035,55.16808504)
\curveto(164.14601705,55.1880745)(164.27101693,55.21307448)(164.40102035,55.24308504)
\curveto(164.47101673,55.24307445)(164.53601666,55.24307445)(164.59602035,55.24308504)
\curveto(164.65601654,55.25307444)(164.72101648,55.26307443)(164.79102035,55.27308504)
}
}
{
\newrgbcolor{curcolor}{0 0 0}
\pscustom[linestyle=none,fillstyle=solid,fillcolor=curcolor]
{
\newpath
\moveto(176.77062973,48.16308504)
\curveto(176.84062208,48.11308158)(176.88062204,48.04308165)(176.89062973,47.95308504)
\curveto(176.91062201,47.86308183)(176.920622,47.75808193)(176.92062973,47.63808504)
\curveto(176.920622,47.5880821)(176.91562201,47.53808215)(176.90562973,47.48808504)
\curveto(176.90562202,47.43808225)(176.89562203,47.3930823)(176.87562973,47.35308504)
\curveto(176.84562208,47.26308243)(176.78562214,47.20308249)(176.69562973,47.17308504)
\curveto(176.61562231,47.15308254)(176.5206224,47.14308255)(176.41062973,47.14308504)
\lineto(176.09562973,47.14308504)
\curveto(175.98562294,47.15308254)(175.88062304,47.14308255)(175.78062973,47.11308504)
\curveto(175.64062328,47.08308261)(175.55062337,47.00308269)(175.51062973,46.87308504)
\curveto(175.49062343,46.80308289)(175.48062344,46.71808297)(175.48062973,46.61808504)
\lineto(175.48062973,46.34808504)
\lineto(175.48062973,45.40308504)
\lineto(175.48062973,45.07308504)
\curveto(175.48062344,44.96308473)(175.46062346,44.87808481)(175.42062973,44.81808504)
\curveto(175.38062354,44.75808493)(175.33062359,44.71808497)(175.27062973,44.69808504)
\curveto(175.2206237,44.688085)(175.15562377,44.67308502)(175.07562973,44.65308504)
\lineto(174.88062973,44.65308504)
\curveto(174.76062416,44.65308504)(174.65562427,44.65808503)(174.56562973,44.66808504)
\curveto(174.47562445,44.688085)(174.40562452,44.73808495)(174.35562973,44.81808504)
\curveto(174.3256246,44.86808482)(174.31062461,44.93808475)(174.31062973,45.02808504)
\lineto(174.31062973,45.32808504)
\lineto(174.31062973,46.36308504)
\curveto(174.31062461,46.52308317)(174.30062462,46.66808302)(174.28062973,46.79808504)
\curveto(174.27062465,46.93808275)(174.21562471,47.03308266)(174.11562973,47.08308504)
\curveto(174.06562486,47.10308259)(173.99562493,47.11808257)(173.90562973,47.12808504)
\curveto(173.8256251,47.13808255)(173.73562519,47.14308255)(173.63562973,47.14308504)
\lineto(173.35062973,47.14308504)
\lineto(173.11062973,47.14308504)
\lineto(170.84562973,47.14308504)
\curveto(170.75562817,47.14308255)(170.65062827,47.13808255)(170.53062973,47.12808504)
\lineto(170.20062973,47.12808504)
\curveto(170.09062883,47.12808256)(169.99062893,47.13808255)(169.90062973,47.15808504)
\curveto(169.81062911,47.17808251)(169.75062917,47.21308248)(169.72062973,47.26308504)
\curveto(169.67062925,47.33308236)(169.64562928,47.42808226)(169.64562973,47.54808504)
\lineto(169.64562973,47.89308504)
\lineto(169.64562973,48.16308504)
\curveto(169.68562924,48.33308136)(169.74062918,48.47308122)(169.81062973,48.58308504)
\curveto(169.88062904,48.693081)(169.96062896,48.80808088)(170.05062973,48.92808504)
\lineto(170.41062973,49.46808504)
\curveto(170.85062807,50.09807959)(171.28562764,50.71807897)(171.71562973,51.32808504)
\lineto(173.03562973,53.18808504)
\curveto(173.19562573,53.41807627)(173.35062557,53.63807605)(173.50062973,53.84808504)
\curveto(173.65062527,54.06807562)(173.80562512,54.2930754)(173.96562973,54.52308504)
\curveto(174.01562491,54.5930751)(174.06562486,54.65807503)(174.11562973,54.71808504)
\curveto(174.16562476,54.7880749)(174.21562471,54.86307483)(174.26562973,54.94308504)
\lineto(174.32562973,55.03308504)
\curveto(174.35562457,55.07307462)(174.38562454,55.10307459)(174.41562973,55.12308504)
\curveto(174.45562447,55.15307454)(174.49562443,55.17307452)(174.53562973,55.18308504)
\curveto(174.57562435,55.20307449)(174.6206243,55.22307447)(174.67062973,55.24308504)
\curveto(174.69062423,55.24307445)(174.71062421,55.23807445)(174.73062973,55.22808504)
\curveto(174.76062416,55.22807446)(174.78562414,55.23807445)(174.80562973,55.25808504)
\curveto(174.93562399,55.25807443)(175.05562387,55.25307444)(175.16562973,55.24308504)
\curveto(175.27562365,55.23307446)(175.35562357,55.1880745)(175.40562973,55.10808504)
\curveto(175.44562348,55.05807463)(175.46562346,54.9880747)(175.46562973,54.89808504)
\curveto(175.47562345,54.80807488)(175.48062344,54.71307498)(175.48062973,54.61308504)
\lineto(175.48062973,49.15308504)
\curveto(175.48062344,49.08308061)(175.47562345,49.00808068)(175.46562973,48.92808504)
\curveto(175.46562346,48.85808083)(175.47062345,48.7880809)(175.48062973,48.71808504)
\lineto(175.48062973,48.61308504)
\curveto(175.50062342,48.56308113)(175.51562341,48.50808118)(175.52562973,48.44808504)
\curveto(175.53562339,48.39808129)(175.56062336,48.35808133)(175.60062973,48.32808504)
\curveto(175.67062325,48.27808141)(175.75562317,48.24808144)(175.85562973,48.23808504)
\lineto(176.18562973,48.23808504)
\curveto(176.29562263,48.23808145)(176.40062252,48.23308146)(176.50062973,48.22308504)
\curveto(176.61062231,48.22308147)(176.70062222,48.20308149)(176.77062973,48.16308504)
\moveto(174.20562973,48.35808504)
\curveto(174.28562464,48.46808122)(174.3206246,48.63808105)(174.31062973,48.86808504)
\lineto(174.31062973,49.48308504)
\lineto(174.31062973,51.95808504)
\lineto(174.31062973,52.27308504)
\curveto(174.3206246,52.3930773)(174.31562461,52.4930772)(174.29562973,52.57308504)
\lineto(174.29562973,52.72308504)
\curveto(174.29562463,52.81307688)(174.28062464,52.89807679)(174.25062973,52.97808504)
\curveto(174.24062468,52.99807669)(174.23062469,53.00807668)(174.22062973,53.00808504)
\lineto(174.17562973,53.05308504)
\curveto(174.15562477,53.06307663)(174.1256248,53.06807662)(174.08562973,53.06808504)
\curveto(174.06562486,53.04807664)(174.04562488,53.03307666)(174.02562973,53.02308504)
\curveto(174.01562491,53.02307667)(174.00062492,53.01807667)(173.98062973,53.00808504)
\curveto(173.920625,52.95807673)(173.86062506,52.8880768)(173.80062973,52.79808504)
\curveto(173.74062518,52.70807698)(173.68562524,52.62807706)(173.63562973,52.55808504)
\curveto(173.53562539,52.41807727)(173.44062548,52.27307742)(173.35062973,52.12308504)
\curveto(173.26062566,51.98307771)(173.16562576,51.84307785)(173.06562973,51.70308504)
\lineto(172.52562973,50.92308504)
\curveto(172.35562657,50.66307903)(172.18062674,50.40307929)(172.00062973,50.14308504)
\curveto(171.920627,50.03307966)(171.84562708,49.92807976)(171.77562973,49.82808504)
\lineto(171.56562973,49.52808504)
\curveto(171.51562741,49.44808024)(171.46562746,49.37308032)(171.41562973,49.30308504)
\curveto(171.37562755,49.23308046)(171.33062759,49.15808053)(171.28062973,49.07808504)
\curveto(171.23062769,49.01808067)(171.18062774,48.95308074)(171.13062973,48.88308504)
\curveto(171.09062783,48.82308087)(171.05062787,48.75308094)(171.01062973,48.67308504)
\curveto(170.97062795,48.61308108)(170.94562798,48.54308115)(170.93562973,48.46308504)
\curveto(170.925628,48.3930813)(170.96062796,48.33808135)(171.04062973,48.29808504)
\curveto(171.11062781,48.24808144)(171.2206277,48.22308147)(171.37062973,48.22308504)
\curveto(171.53062739,48.23308146)(171.66562726,48.23808145)(171.77562973,48.23808504)
\lineto(173.45562973,48.23808504)
\lineto(173.89062973,48.23808504)
\curveto(174.04062488,48.23808145)(174.14562478,48.27808141)(174.20562973,48.35808504)
}
}
{
\newrgbcolor{curcolor}{0 0 0}
\pscustom[linestyle=none,fillstyle=solid,fillcolor=curcolor]
{
\newpath
\moveto(179.1952391,46.30308504)
\lineto(179.4952391,46.30308504)
\curveto(179.60523704,46.31308338)(179.71023694,46.31308338)(179.8102391,46.30308504)
\curveto(179.92023673,46.30308339)(180.02023663,46.2930834)(180.1102391,46.27308504)
\curveto(180.20023645,46.26308343)(180.27023638,46.23808345)(180.3202391,46.19808504)
\curveto(180.34023631,46.17808351)(180.35523629,46.14808354)(180.3652391,46.10808504)
\curveto(180.38523626,46.06808362)(180.40523624,46.02308367)(180.4252391,45.97308504)
\lineto(180.4252391,45.89808504)
\curveto(180.43523621,45.84808384)(180.43523621,45.7930839)(180.4252391,45.73308504)
\lineto(180.4252391,45.58308504)
\lineto(180.4252391,45.10308504)
\curveto(180.42523622,44.93308476)(180.38523626,44.81308488)(180.3052391,44.74308504)
\curveto(180.23523641,44.693085)(180.1452365,44.66808502)(180.0352391,44.66808504)
\lineto(179.7052391,44.66808504)
\lineto(179.2552391,44.66808504)
\curveto(179.10523754,44.66808502)(178.99023766,44.69808499)(178.9102391,44.75808504)
\curveto(178.87023778,44.7880849)(178.84023781,44.83808485)(178.8202391,44.90808504)
\curveto(178.80023785,44.9880847)(178.78523786,45.07308462)(178.7752391,45.16308504)
\lineto(178.7752391,45.44808504)
\curveto(178.78523786,45.54808414)(178.79023786,45.63308406)(178.7902391,45.70308504)
\lineto(178.7902391,45.89808504)
\curveto(178.79023786,45.95808373)(178.80023785,46.01308368)(178.8202391,46.06308504)
\curveto(178.86023779,46.17308352)(178.93023772,46.24308345)(179.0302391,46.27308504)
\curveto(179.06023759,46.27308342)(179.11523753,46.28308341)(179.1952391,46.30308504)
}
}
{
\newrgbcolor{curcolor}{0 0 0}
\pscustom[linestyle=none,fillstyle=solid,fillcolor=curcolor]
{
\newpath
\moveto(185.65039535,55.27308504)
\curveto(186.34039072,55.28307441)(186.94039012,55.16307453)(187.45039535,54.91308504)
\curveto(187.97038909,54.66307503)(188.36538869,54.32807536)(188.63539535,53.90808504)
\curveto(188.68538837,53.82807586)(188.73038833,53.73807595)(188.77039535,53.63808504)
\curveto(188.81038825,53.54807614)(188.8553882,53.45307624)(188.90539535,53.35308504)
\curveto(188.94538811,53.25307644)(188.97538808,53.15307654)(188.99539535,53.05308504)
\curveto(189.01538804,52.95307674)(189.03538802,52.84807684)(189.05539535,52.73808504)
\curveto(189.07538798,52.688077)(189.08038798,52.64307705)(189.07039535,52.60308504)
\curveto(189.060388,52.56307713)(189.06538799,52.51807717)(189.08539535,52.46808504)
\curveto(189.09538796,52.41807727)(189.10038796,52.33307736)(189.10039535,52.21308504)
\curveto(189.10038796,52.10307759)(189.09538796,52.01807767)(189.08539535,51.95808504)
\curveto(189.06538799,51.89807779)(189.055388,51.83807785)(189.05539535,51.77808504)
\curveto(189.06538799,51.71807797)(189.060388,51.65807803)(189.04039535,51.59808504)
\curveto(189.00038806,51.45807823)(188.96538809,51.32307837)(188.93539535,51.19308504)
\curveto(188.90538815,51.06307863)(188.86538819,50.93807875)(188.81539535,50.81808504)
\curveto(188.7553883,50.67807901)(188.68538837,50.55307914)(188.60539535,50.44308504)
\curveto(188.53538852,50.33307936)(188.4603886,50.22307947)(188.38039535,50.11308504)
\lineto(188.32039535,50.05308504)
\curveto(188.31038875,50.03307966)(188.29538876,50.01307968)(188.27539535,49.99308504)
\curveto(188.1553889,49.83307986)(188.02038904,49.68808)(187.87039535,49.55808504)
\curveto(187.72038934,49.42808026)(187.5603895,49.30308039)(187.39039535,49.18308504)
\curveto(187.08038998,48.96308073)(186.78539027,48.75808093)(186.50539535,48.56808504)
\curveto(186.27539078,48.42808126)(186.04539101,48.2930814)(185.81539535,48.16308504)
\curveto(185.59539146,48.03308166)(185.37539168,47.89808179)(185.15539535,47.75808504)
\curveto(184.90539215,47.5880821)(184.66539239,47.40808228)(184.43539535,47.21808504)
\curveto(184.21539284,47.02808266)(184.02539303,46.80308289)(183.86539535,46.54308504)
\curveto(183.82539323,46.48308321)(183.79039327,46.42308327)(183.76039535,46.36308504)
\curveto(183.73039333,46.31308338)(183.70039336,46.24808344)(183.67039535,46.16808504)
\curveto(183.65039341,46.09808359)(183.64539341,46.03808365)(183.65539535,45.98808504)
\curveto(183.67539338,45.91808377)(183.71039335,45.86308383)(183.76039535,45.82308504)
\curveto(183.81039325,45.7930839)(183.87039319,45.77308392)(183.94039535,45.76308504)
\lineto(184.18039535,45.76308504)
\lineto(184.93039535,45.76308504)
\lineto(187.73539535,45.76308504)
\lineto(188.39539535,45.76308504)
\curveto(188.48538857,45.76308393)(188.57038849,45.75808393)(188.65039535,45.74808504)
\curveto(188.73038833,45.74808394)(188.79538826,45.72808396)(188.84539535,45.68808504)
\curveto(188.89538816,45.64808404)(188.93538812,45.57308412)(188.96539535,45.46308504)
\curveto(189.00538805,45.36308433)(189.01538804,45.26308443)(188.99539535,45.16308504)
\lineto(188.99539535,45.02808504)
\curveto(188.97538808,44.95808473)(188.9553881,44.89808479)(188.93539535,44.84808504)
\curveto(188.91538814,44.79808489)(188.88038818,44.75808493)(188.83039535,44.72808504)
\curveto(188.78038828,44.688085)(188.71038835,44.66808502)(188.62039535,44.66808504)
\lineto(188.35039535,44.66808504)
\lineto(187.45039535,44.66808504)
\lineto(183.94039535,44.66808504)
\lineto(182.87539535,44.66808504)
\curveto(182.79539426,44.66808502)(182.70539435,44.66308503)(182.60539535,44.65308504)
\curveto(182.50539455,44.65308504)(182.42039464,44.66308503)(182.35039535,44.68308504)
\curveto(182.14039492,44.75308494)(182.07539498,44.93308476)(182.15539535,45.22308504)
\curveto(182.16539489,45.26308443)(182.16539489,45.29808439)(182.15539535,45.32808504)
\curveto(182.1553949,45.36808432)(182.16539489,45.41308428)(182.18539535,45.46308504)
\curveto(182.20539485,45.54308415)(182.22539483,45.62808406)(182.24539535,45.71808504)
\curveto(182.26539479,45.80808388)(182.29039477,45.8930838)(182.32039535,45.97308504)
\curveto(182.48039458,46.46308323)(182.68039438,46.87808281)(182.92039535,47.21808504)
\curveto(183.10039396,47.46808222)(183.30539375,47.693082)(183.53539535,47.89308504)
\curveto(183.76539329,48.10308159)(184.00539305,48.29808139)(184.25539535,48.47808504)
\curveto(184.51539254,48.65808103)(184.78039228,48.82808086)(185.05039535,48.98808504)
\curveto(185.33039173,49.15808053)(185.60039146,49.33308036)(185.86039535,49.51308504)
\curveto(185.97039109,49.5930801)(186.07539098,49.66808002)(186.17539535,49.73808504)
\curveto(186.28539077,49.80807988)(186.39539066,49.88307981)(186.50539535,49.96308504)
\curveto(186.54539051,49.9930797)(186.58039048,50.02307967)(186.61039535,50.05308504)
\curveto(186.65039041,50.0930796)(186.69039037,50.12307957)(186.73039535,50.14308504)
\curveto(186.87039019,50.25307944)(186.99539006,50.37807931)(187.10539535,50.51808504)
\curveto(187.12538993,50.54807914)(187.15038991,50.57307912)(187.18039535,50.59308504)
\curveto(187.21038985,50.62307907)(187.23538982,50.65307904)(187.25539535,50.68308504)
\curveto(187.33538972,50.78307891)(187.40038966,50.88307881)(187.45039535,50.98308504)
\curveto(187.51038955,51.08307861)(187.56538949,51.1930785)(187.61539535,51.31308504)
\curveto(187.64538941,51.38307831)(187.66538939,51.45807823)(187.67539535,51.53808504)
\lineto(187.73539535,51.77808504)
\lineto(187.73539535,51.86808504)
\curveto(187.74538931,51.89807779)(187.75038931,51.92807776)(187.75039535,51.95808504)
\curveto(187.77038929,52.02807766)(187.77538928,52.12307757)(187.76539535,52.24308504)
\curveto(187.76538929,52.37307732)(187.7553893,52.47307722)(187.73539535,52.54308504)
\curveto(187.71538934,52.62307707)(187.69538936,52.69807699)(187.67539535,52.76808504)
\curveto(187.66538939,52.84807684)(187.64538941,52.92807676)(187.61539535,53.00808504)
\curveto(187.50538955,53.24807644)(187.3553897,53.44807624)(187.16539535,53.60808504)
\curveto(186.98539007,53.77807591)(186.76539029,53.91807577)(186.50539535,54.02808504)
\curveto(186.43539062,54.04807564)(186.36539069,54.06307563)(186.29539535,54.07308504)
\curveto(186.22539083,54.0930756)(186.15039091,54.11307558)(186.07039535,54.13308504)
\curveto(185.99039107,54.15307554)(185.88039118,54.16307553)(185.74039535,54.16308504)
\curveto(185.61039145,54.16307553)(185.50539155,54.15307554)(185.42539535,54.13308504)
\curveto(185.36539169,54.12307557)(185.31039175,54.11807557)(185.26039535,54.11808504)
\curveto(185.21039185,54.11807557)(185.1603919,54.10807558)(185.11039535,54.08808504)
\curveto(185.01039205,54.04807564)(184.91539214,54.00807568)(184.82539535,53.96808504)
\curveto(184.74539231,53.92807576)(184.66539239,53.88307581)(184.58539535,53.83308504)
\curveto(184.5553925,53.81307588)(184.52539253,53.7880759)(184.49539535,53.75808504)
\curveto(184.47539258,53.72807596)(184.45039261,53.70307599)(184.42039535,53.68308504)
\lineto(184.34539535,53.60808504)
\curveto(184.31539274,53.5880761)(184.29039277,53.56807612)(184.27039535,53.54808504)
\lineto(184.12039535,53.33808504)
\curveto(184.08039298,53.27807641)(184.03539302,53.21307648)(183.98539535,53.14308504)
\curveto(183.92539313,53.05307664)(183.87539318,52.94807674)(183.83539535,52.82808504)
\curveto(183.80539325,52.71807697)(183.77039329,52.60807708)(183.73039535,52.49808504)
\curveto(183.69039337,52.3880773)(183.66539339,52.24307745)(183.65539535,52.06308504)
\curveto(183.64539341,51.8930778)(183.61539344,51.76807792)(183.56539535,51.68808504)
\curveto(183.51539354,51.60807808)(183.44039362,51.56307813)(183.34039535,51.55308504)
\curveto(183.24039382,51.54307815)(183.13039393,51.53807815)(183.01039535,51.53808504)
\curveto(182.97039409,51.53807815)(182.93039413,51.53307816)(182.89039535,51.52308504)
\curveto(182.85039421,51.52307817)(182.81539424,51.52807816)(182.78539535,51.53808504)
\curveto(182.73539432,51.55807813)(182.68539437,51.56807812)(182.63539535,51.56808504)
\curveto(182.59539446,51.56807812)(182.5553945,51.57807811)(182.51539535,51.59808504)
\curveto(182.42539463,51.65807803)(182.38039468,51.7930779)(182.38039535,52.00308504)
\lineto(182.38039535,52.12308504)
\curveto(182.39039467,52.18307751)(182.39539466,52.24307745)(182.39539535,52.30308504)
\curveto(182.40539465,52.37307732)(182.41539464,52.43807725)(182.42539535,52.49808504)
\curveto(182.44539461,52.60807708)(182.46539459,52.70807698)(182.48539535,52.79808504)
\curveto(182.50539455,52.89807679)(182.53539452,52.9930767)(182.57539535,53.08308504)
\curveto(182.59539446,53.15307654)(182.61539444,53.21307648)(182.63539535,53.26308504)
\lineto(182.69539535,53.44308504)
\curveto(182.81539424,53.70307599)(182.97039409,53.94807574)(183.16039535,54.17808504)
\curveto(183.3603937,54.40807528)(183.57539348,54.5930751)(183.80539535,54.73308504)
\curveto(183.91539314,54.81307488)(184.03039303,54.87807481)(184.15039535,54.92808504)
\lineto(184.54039535,55.07808504)
\curveto(184.65039241,55.12807456)(184.76539229,55.15807453)(184.88539535,55.16808504)
\curveto(185.00539205,55.1880745)(185.13039193,55.21307448)(185.26039535,55.24308504)
\curveto(185.33039173,55.24307445)(185.39539166,55.24307445)(185.45539535,55.24308504)
\curveto(185.51539154,55.25307444)(185.58039148,55.26307443)(185.65039535,55.27308504)
}
}
{
\newrgbcolor{curcolor}{0 0 0}
\pscustom[linestyle=none,fillstyle=solid,fillcolor=curcolor]
{
\newpath
\moveto(200.55500473,53.18808504)
\curveto(200.35499443,52.89807679)(200.14499464,52.61307708)(199.92500473,52.33308504)
\curveto(199.71499507,52.05307764)(199.50999527,51.76807792)(199.31000473,51.47808504)
\curveto(198.70999607,50.62807906)(198.10499668,49.7880799)(197.49500473,48.95808504)
\curveto(196.8849979,48.13808155)(196.2799985,47.30308239)(195.68000473,46.45308504)
\lineto(195.17000473,45.73308504)
\lineto(194.66000473,45.04308504)
\curveto(194.5800002,44.93308476)(194.50000028,44.81808487)(194.42000473,44.69808504)
\curveto(194.34000044,44.57808511)(194.24500054,44.48308521)(194.13500473,44.41308504)
\curveto(194.09500069,44.3930853)(194.03000075,44.37808531)(193.94000473,44.36808504)
\curveto(193.86000092,44.34808534)(193.77000101,44.33808535)(193.67000473,44.33808504)
\curveto(193.57000121,44.33808535)(193.47500131,44.34308535)(193.38500473,44.35308504)
\curveto(193.30500148,44.36308533)(193.24500154,44.38308531)(193.20500473,44.41308504)
\curveto(193.17500161,44.43308526)(193.15000163,44.46808522)(193.13000473,44.51808504)
\curveto(193.12000166,44.55808513)(193.12500166,44.60308509)(193.14500473,44.65308504)
\curveto(193.1850016,44.73308496)(193.23000155,44.80808488)(193.28000473,44.87808504)
\curveto(193.34000144,44.95808473)(193.39500139,45.03808465)(193.44500473,45.11808504)
\curveto(193.6850011,45.45808423)(193.93000085,45.7930839)(194.18000473,46.12308504)
\curveto(194.43000035,46.45308324)(194.67000011,46.7880829)(194.90000473,47.12808504)
\curveto(195.05999972,47.34808234)(195.21999956,47.56308213)(195.38000473,47.77308504)
\curveto(195.53999924,47.98308171)(195.69999908,48.19808149)(195.86000473,48.41808504)
\curveto(196.21999856,48.93808075)(196.5849982,49.44808024)(196.95500473,49.94808504)
\curveto(197.32499746,50.44807924)(197.69499709,50.95807873)(198.06500473,51.47808504)
\curveto(198.20499658,51.67807801)(198.34499644,51.87307782)(198.48500473,52.06308504)
\curveto(198.63499615,52.25307744)(198.779996,52.44807724)(198.92000473,52.64808504)
\curveto(199.12999565,52.94807674)(199.34499544,53.24807644)(199.56500473,53.54808504)
\lineto(200.22500473,54.44808504)
\lineto(200.40500473,54.71808504)
\lineto(200.61500473,54.98808504)
\lineto(200.73500473,55.16808504)
\curveto(200.784994,55.22807446)(200.83499395,55.28307441)(200.88500473,55.33308504)
\curveto(200.95499383,55.38307431)(201.02999375,55.41807427)(201.11000473,55.43808504)
\curveto(201.12999365,55.44807424)(201.15499363,55.44807424)(201.18500473,55.43808504)
\curveto(201.22499356,55.43807425)(201.25499353,55.44807424)(201.27500473,55.46808504)
\curveto(201.39499339,55.46807422)(201.52999325,55.46307423)(201.68000473,55.45308504)
\curveto(201.82999295,55.45307424)(201.91999286,55.40807428)(201.95000473,55.31808504)
\curveto(201.96999281,55.2880744)(201.97499281,55.25307444)(201.96500473,55.21308504)
\curveto(201.95499283,55.17307452)(201.93999284,55.14307455)(201.92000473,55.12308504)
\curveto(201.8799929,55.04307465)(201.83999294,54.97307472)(201.80000473,54.91308504)
\curveto(201.75999302,54.85307484)(201.71499307,54.7930749)(201.66500473,54.73308504)
\lineto(201.09500473,53.95308504)
\curveto(200.91499387,53.70307599)(200.73499405,53.44807624)(200.55500473,53.18808504)
\moveto(193.70000473,49.28808504)
\curveto(193.65000113,49.30808038)(193.60000118,49.31308038)(193.55000473,49.30308504)
\curveto(193.50000128,49.2930804)(193.45000133,49.29808039)(193.40000473,49.31808504)
\curveto(193.29000149,49.33808035)(193.1850016,49.35808033)(193.08500473,49.37808504)
\curveto(192.99500179,49.40808028)(192.90000188,49.44808024)(192.80000473,49.49808504)
\curveto(192.47000231,49.63808005)(192.21500257,49.83307986)(192.03500473,50.08308504)
\curveto(191.85500293,50.34307935)(191.71000307,50.65307904)(191.60000473,51.01308504)
\curveto(191.57000321,51.0930786)(191.55000323,51.17307852)(191.54000473,51.25308504)
\curveto(191.53000325,51.34307835)(191.51500327,51.42807826)(191.49500473,51.50808504)
\curveto(191.4850033,51.55807813)(191.4800033,51.62307807)(191.48000473,51.70308504)
\curveto(191.47000331,51.73307796)(191.46500332,51.76307793)(191.46500473,51.79308504)
\curveto(191.46500332,51.83307786)(191.46000332,51.86807782)(191.45000473,51.89808504)
\lineto(191.45000473,52.04808504)
\curveto(191.44000334,52.09807759)(191.43500335,52.15807753)(191.43500473,52.22808504)
\curveto(191.43500335,52.30807738)(191.44000334,52.37307732)(191.45000473,52.42308504)
\lineto(191.45000473,52.58808504)
\curveto(191.47000331,52.63807705)(191.47500331,52.68307701)(191.46500473,52.72308504)
\curveto(191.46500332,52.77307692)(191.47000331,52.81807687)(191.48000473,52.85808504)
\curveto(191.49000329,52.89807679)(191.49500329,52.93307676)(191.49500473,52.96308504)
\curveto(191.49500329,53.00307669)(191.50000328,53.04307665)(191.51000473,53.08308504)
\curveto(191.54000324,53.1930765)(191.56000322,53.30307639)(191.57000473,53.41308504)
\curveto(191.59000319,53.53307616)(191.62500316,53.64807604)(191.67500473,53.75808504)
\curveto(191.81500297,54.09807559)(191.97500281,54.37307532)(192.15500473,54.58308504)
\curveto(192.34500244,54.80307489)(192.61500217,54.98307471)(192.96500473,55.12308504)
\curveto(193.04500174,55.15307454)(193.13000165,55.17307452)(193.22000473,55.18308504)
\curveto(193.31000147,55.20307449)(193.40500138,55.22307447)(193.50500473,55.24308504)
\curveto(193.53500125,55.25307444)(193.59000119,55.25307444)(193.67000473,55.24308504)
\curveto(193.75000103,55.24307445)(193.80000098,55.25307444)(193.82000473,55.27308504)
\curveto(194.3800004,55.28307441)(194.82999995,55.17307452)(195.17000473,54.94308504)
\curveto(195.51999926,54.71307498)(195.779999,54.40807528)(195.95000473,54.02808504)
\curveto(195.98999879,53.93807575)(196.02499876,53.84307585)(196.05500473,53.74308504)
\curveto(196.0849987,53.64307605)(196.10999867,53.54307615)(196.13000473,53.44308504)
\curveto(196.14999863,53.41307628)(196.15499863,53.38307631)(196.14500473,53.35308504)
\curveto(196.14499864,53.32307637)(196.14999863,53.2930764)(196.16000473,53.26308504)
\curveto(196.18999859,53.15307654)(196.20999857,53.02807666)(196.22000473,52.88808504)
\curveto(196.22999855,52.75807693)(196.23999854,52.62307707)(196.25000473,52.48308504)
\lineto(196.25000473,52.31808504)
\curveto(196.25999852,52.25807743)(196.25999852,52.20307749)(196.25000473,52.15308504)
\curveto(196.23999854,52.10307759)(196.23499855,52.05307764)(196.23500473,52.00308504)
\lineto(196.23500473,51.86808504)
\curveto(196.22499856,51.82807786)(196.21999856,51.7880779)(196.22000473,51.74808504)
\curveto(196.22999855,51.70807798)(196.22499856,51.66307803)(196.20500473,51.61308504)
\curveto(196.1849986,51.50307819)(196.16499862,51.39807829)(196.14500473,51.29808504)
\curveto(196.13499865,51.19807849)(196.11499867,51.09807859)(196.08500473,50.99808504)
\curveto(195.95499883,50.63807905)(195.78999899,50.32307937)(195.59000473,50.05308504)
\curveto(195.38999939,49.78307991)(195.11499967,49.57808011)(194.76500473,49.43808504)
\curveto(194.6850001,49.40808028)(194.60000018,49.38308031)(194.51000473,49.36308504)
\lineto(194.24000473,49.30308504)
\curveto(194.19000059,49.2930804)(194.14500064,49.2880804)(194.10500473,49.28808504)
\curveto(194.06500072,49.29808039)(194.02500076,49.29808039)(193.98500473,49.28808504)
\curveto(193.8850009,49.26808042)(193.79000099,49.26808042)(193.70000473,49.28808504)
\moveto(192.86000473,50.68308504)
\curveto(192.90000188,50.61307908)(192.94000184,50.54807914)(192.98000473,50.48808504)
\curveto(193.02000176,50.43807925)(193.07000171,50.3880793)(193.13000473,50.33808504)
\lineto(193.28000473,50.21808504)
\curveto(193.34000144,50.1880795)(193.40500138,50.16307953)(193.47500473,50.14308504)
\curveto(193.51500127,50.12307957)(193.55000123,50.11307958)(193.58000473,50.11308504)
\curveto(193.62000116,50.12307957)(193.66000112,50.11807957)(193.70000473,50.09808504)
\curveto(193.73000105,50.09807959)(193.77000101,50.0930796)(193.82000473,50.08308504)
\curveto(193.87000091,50.08307961)(193.91000087,50.0880796)(193.94000473,50.09808504)
\lineto(194.16500473,50.14308504)
\curveto(194.41500037,50.22307947)(194.60000018,50.34807934)(194.72000473,50.51808504)
\curveto(194.79999998,50.61807907)(194.86999991,50.74807894)(194.93000473,50.90808504)
\curveto(195.00999977,51.0880786)(195.06999971,51.31307838)(195.11000473,51.58308504)
\curveto(195.14999963,51.86307783)(195.16499962,52.14307755)(195.15500473,52.42308504)
\curveto(195.14499964,52.71307698)(195.11499967,52.9880767)(195.06500473,53.24808504)
\curveto(195.01499977,53.50807618)(194.93999984,53.71807597)(194.84000473,53.87808504)
\curveto(194.72000006,54.07807561)(194.57000021,54.22807546)(194.39000473,54.32808504)
\curveto(194.31000047,54.37807531)(194.22000056,54.40807528)(194.12000473,54.41808504)
\curveto(194.02000076,54.43807525)(193.91500087,54.44807524)(193.80500473,54.44808504)
\curveto(193.785001,54.43807525)(193.76000102,54.43307526)(193.73000473,54.43308504)
\curveto(193.71000107,54.44307525)(193.69000109,54.44307525)(193.67000473,54.43308504)
\curveto(193.62000116,54.42307527)(193.57500121,54.41307528)(193.53500473,54.40308504)
\curveto(193.49500129,54.40307529)(193.45500133,54.3930753)(193.41500473,54.37308504)
\curveto(193.23500155,54.2930754)(193.0850017,54.17307552)(192.96500473,54.01308504)
\curveto(192.85500193,53.85307584)(192.76500202,53.67307602)(192.69500473,53.47308504)
\curveto(192.63500215,53.28307641)(192.59000219,53.05807663)(192.56000473,52.79808504)
\curveto(192.54000224,52.53807715)(192.53500225,52.27307742)(192.54500473,52.00308504)
\curveto(192.55500223,51.74307795)(192.5850022,51.4930782)(192.63500473,51.25308504)
\curveto(192.69500209,51.02307867)(192.77000201,50.83307886)(192.86000473,50.68308504)
\moveto(203.66000473,47.69808504)
\curveto(203.66999111,47.64808204)(203.67499111,47.55808213)(203.67500473,47.42808504)
\curveto(203.67499111,47.29808239)(203.66499112,47.20808248)(203.64500473,47.15808504)
\curveto(203.62499116,47.10808258)(203.61999116,47.05308264)(203.63000473,46.99308504)
\curveto(203.63999114,46.94308275)(203.63999114,46.8930828)(203.63000473,46.84308504)
\curveto(203.58999119,46.70308299)(203.55999122,46.56808312)(203.54000473,46.43808504)
\curveto(203.52999125,46.30808338)(203.49999128,46.1880835)(203.45000473,46.07808504)
\curveto(203.30999147,45.72808396)(203.14499164,45.43308426)(202.95500473,45.19308504)
\curveto(202.76499202,44.96308473)(202.49499229,44.77808491)(202.14500473,44.63808504)
\curveto(202.06499272,44.60808508)(201.9799928,44.5880851)(201.89000473,44.57808504)
\curveto(201.79999298,44.55808513)(201.71499307,44.53808515)(201.63500473,44.51808504)
\curveto(201.5849932,44.50808518)(201.53499325,44.50308519)(201.48500473,44.50308504)
\curveto(201.43499335,44.50308519)(201.3849934,44.49808519)(201.33500473,44.48808504)
\curveto(201.30499348,44.47808521)(201.25499353,44.47808521)(201.18500473,44.48808504)
\curveto(201.11499367,44.4880852)(201.06499372,44.4930852)(201.03500473,44.50308504)
\curveto(200.97499381,44.52308517)(200.91499387,44.53308516)(200.85500473,44.53308504)
\curveto(200.80499398,44.52308517)(200.75499403,44.52808516)(200.70500473,44.54808504)
\curveto(200.61499417,44.56808512)(200.52499426,44.5930851)(200.43500473,44.62308504)
\curveto(200.35499443,44.64308505)(200.27499451,44.67308502)(200.19500473,44.71308504)
\curveto(199.87499491,44.85308484)(199.62499516,45.04808464)(199.44500473,45.29808504)
\curveto(199.26499552,45.55808413)(199.11499567,45.86308383)(198.99500473,46.21308504)
\curveto(198.97499581,46.2930834)(198.95999582,46.37808331)(198.95000473,46.46808504)
\curveto(198.93999584,46.55808313)(198.92499586,46.64308305)(198.90500473,46.72308504)
\curveto(198.89499589,46.75308294)(198.88999589,46.78308291)(198.89000473,46.81308504)
\lineto(198.89000473,46.91808504)
\curveto(198.86999591,46.99808269)(198.85999592,47.07808261)(198.86000473,47.15808504)
\lineto(198.86000473,47.29308504)
\curveto(198.83999594,47.3930823)(198.83999594,47.4930822)(198.86000473,47.59308504)
\lineto(198.86000473,47.77308504)
\curveto(198.86999591,47.82308187)(198.87499591,47.86808182)(198.87500473,47.90808504)
\curveto(198.87499591,47.95808173)(198.8799959,48.00308169)(198.89000473,48.04308504)
\curveto(198.89999588,48.08308161)(198.90499588,48.11808157)(198.90500473,48.14808504)
\curveto(198.90499588,48.1880815)(198.90999587,48.22808146)(198.92000473,48.26808504)
\lineto(198.98000473,48.59808504)
\curveto(198.99999578,48.71808097)(199.02999575,48.82808086)(199.07000473,48.92808504)
\curveto(199.20999557,49.25808043)(199.36999541,49.53308016)(199.55000473,49.75308504)
\curveto(199.73999504,49.98307971)(199.99999478,50.16807952)(200.33000473,50.30808504)
\curveto(200.40999437,50.34807934)(200.49499429,50.37307932)(200.58500473,50.38308504)
\lineto(200.88500473,50.44308504)
\lineto(201.02000473,50.44308504)
\curveto(201.06999371,50.45307924)(201.11999366,50.45807923)(201.17000473,50.45808504)
\curveto(201.73999304,50.47807921)(202.19999258,50.37307932)(202.55000473,50.14308504)
\curveto(202.90999187,49.92307977)(203.17499161,49.62308007)(203.34500473,49.24308504)
\curveto(203.39499139,49.14308055)(203.43499135,49.04308065)(203.46500473,48.94308504)
\curveto(203.49499129,48.84308085)(203.52499126,48.73808095)(203.55500473,48.62808504)
\curveto(203.56499122,48.5880811)(203.56999121,48.55308114)(203.57000473,48.52308504)
\curveto(203.56999121,48.50308119)(203.57499121,48.47308122)(203.58500473,48.43308504)
\curveto(203.60499118,48.36308133)(203.61499117,48.2880814)(203.61500473,48.20808504)
\curveto(203.61499117,48.12808156)(203.62499116,48.04808164)(203.64500473,47.96808504)
\curveto(203.64499114,47.91808177)(203.64499114,47.87308182)(203.64500473,47.83308504)
\curveto(203.64499114,47.7930819)(203.64999113,47.74808194)(203.66000473,47.69808504)
\moveto(202.55000473,47.26308504)
\curveto(202.55999222,47.31308238)(202.56499222,47.3880823)(202.56500473,47.48808504)
\curveto(202.57499221,47.5880821)(202.56999221,47.66308203)(202.55000473,47.71308504)
\curveto(202.52999225,47.77308192)(202.52499226,47.82808186)(202.53500473,47.87808504)
\curveto(202.55499223,47.93808175)(202.55499223,47.99808169)(202.53500473,48.05808504)
\curveto(202.52499226,48.0880816)(202.51999226,48.12308157)(202.52000473,48.16308504)
\curveto(202.51999226,48.20308149)(202.51499227,48.24308145)(202.50500473,48.28308504)
\curveto(202.4849923,48.36308133)(202.46499232,48.43808125)(202.44500473,48.50808504)
\curveto(202.43499235,48.5880811)(202.41999236,48.66808102)(202.40000473,48.74808504)
\curveto(202.36999241,48.80808088)(202.34499244,48.86808082)(202.32500473,48.92808504)
\curveto(202.30499248,48.9880807)(202.27499251,49.04808064)(202.23500473,49.10808504)
\curveto(202.13499265,49.27808041)(202.00499278,49.41308028)(201.84500473,49.51308504)
\curveto(201.76499302,49.56308013)(201.66999311,49.59808009)(201.56000473,49.61808504)
\curveto(201.44999333,49.63808005)(201.32499346,49.64808004)(201.18500473,49.64808504)
\curveto(201.16499362,49.63808005)(201.13999364,49.63308006)(201.11000473,49.63308504)
\curveto(201.0799937,49.64308005)(201.04999373,49.64308005)(201.02000473,49.63308504)
\lineto(200.87000473,49.57308504)
\curveto(200.81999396,49.56308013)(200.77499401,49.54808014)(200.73500473,49.52808504)
\curveto(200.54499424,49.41808027)(200.39999438,49.27308042)(200.30000473,49.09308504)
\curveto(200.20999457,48.91308078)(200.12999465,48.70808098)(200.06000473,48.47808504)
\curveto(200.01999476,48.34808134)(199.99999478,48.21308148)(200.00000473,48.07308504)
\curveto(199.99999478,47.94308175)(199.98999479,47.79808189)(199.97000473,47.63808504)
\curveto(199.95999482,47.5880821)(199.94999483,47.52808216)(199.94000473,47.45808504)
\curveto(199.93999484,47.3880823)(199.94999483,47.32808236)(199.97000473,47.27808504)
\lineto(199.97000473,47.11308504)
\lineto(199.97000473,46.93308504)
\curveto(199.9799948,46.88308281)(199.98999479,46.82808286)(200.00000473,46.76808504)
\curveto(200.00999477,46.71808297)(200.01499477,46.66308303)(200.01500473,46.60308504)
\curveto(200.02499476,46.54308315)(200.03999474,46.4880832)(200.06000473,46.43808504)
\curveto(200.10999467,46.24808344)(200.16999461,46.07308362)(200.24000473,45.91308504)
\curveto(200.30999447,45.75308394)(200.41499437,45.62308407)(200.55500473,45.52308504)
\curveto(200.6849941,45.42308427)(200.82499396,45.35308434)(200.97500473,45.31308504)
\curveto(201.00499378,45.30308439)(201.02999375,45.29808439)(201.05000473,45.29808504)
\curveto(201.0799937,45.30808438)(201.10999367,45.30808438)(201.14000473,45.29808504)
\curveto(201.15999362,45.29808439)(201.18999359,45.2930844)(201.23000473,45.28308504)
\curveto(201.26999351,45.28308441)(201.30499348,45.2880844)(201.33500473,45.29808504)
\curveto(201.37499341,45.30808438)(201.41499337,45.31308438)(201.45500473,45.31308504)
\curveto(201.49499329,45.31308438)(201.53499325,45.32308437)(201.57500473,45.34308504)
\curveto(201.81499297,45.42308427)(202.00999277,45.55808413)(202.16000473,45.74808504)
\curveto(202.2799925,45.92808376)(202.36999241,46.13308356)(202.43000473,46.36308504)
\curveto(202.44999233,46.43308326)(202.46499232,46.50308319)(202.47500473,46.57308504)
\curveto(202.4849923,46.65308304)(202.49999228,46.73308296)(202.52000473,46.81308504)
\curveto(202.51999226,46.87308282)(202.52499226,46.91808277)(202.53500473,46.94808504)
\curveto(202.53499225,46.96808272)(202.53499225,46.9930827)(202.53500473,47.02308504)
\curveto(202.53499225,47.06308263)(202.53999224,47.0930826)(202.55000473,47.11308504)
\lineto(202.55000473,47.26308504)
}
}
{
\newrgbcolor{curcolor}{0 0 0}
\pscustom[linestyle=none,fillstyle=solid,fillcolor=curcolor]
{
\newpath
\moveto(101.52148361,227.50413362)
\curveto(103.15147817,227.53412297)(104.20147712,226.97912353)(104.67148361,225.83913362)
\curveto(104.77147655,225.6091249)(104.83647648,225.31912519)(104.86648361,224.96913362)
\curveto(104.90647641,224.62912588)(104.88147644,224.31912619)(104.79148361,224.03913362)
\curveto(104.70147662,223.77912673)(104.58147674,223.55412695)(104.43148361,223.36413362)
\curveto(104.41147691,223.32412718)(104.38647693,223.28912722)(104.35648361,223.25913362)
\curveto(104.32647699,223.23912727)(104.30147702,223.21412729)(104.28148361,223.18413362)
\lineto(104.19148361,223.06413362)
\curveto(104.16147716,223.03412747)(104.12647719,223.0091275)(104.08648361,222.98913362)
\curveto(104.03647728,222.93912757)(103.98147734,222.89412761)(103.92148361,222.85413362)
\curveto(103.87147745,222.81412769)(103.82647749,222.76412774)(103.78648361,222.70413362)
\curveto(103.74647757,222.66412784)(103.73147759,222.61412789)(103.74148361,222.55413362)
\curveto(103.75147757,222.504128)(103.78147754,222.45912805)(103.83148361,222.41913362)
\curveto(103.88147744,222.37912813)(103.93647738,222.33912817)(103.99648361,222.29913362)
\curveto(104.06647725,222.26912824)(104.13147719,222.23912827)(104.19148361,222.20913362)
\curveto(104.25147707,222.17912833)(104.30147702,222.14912836)(104.34148361,222.11913362)
\curveto(104.66147666,221.89912861)(104.9164764,221.58912892)(105.10648361,221.18913362)
\curveto(105.14647617,221.09912941)(105.17647614,221.0041295)(105.19648361,220.90413362)
\curveto(105.22647609,220.81412969)(105.25147607,220.72412978)(105.27148361,220.63413362)
\curveto(105.28147604,220.58412992)(105.28647603,220.53412997)(105.28648361,220.48413362)
\curveto(105.29647602,220.44413006)(105.30647601,220.39913011)(105.31648361,220.34913362)
\curveto(105.32647599,220.29913021)(105.32647599,220.24913026)(105.31648361,220.19913362)
\curveto(105.30647601,220.14913036)(105.31147601,220.09913041)(105.33148361,220.04913362)
\curveto(105.34147598,219.99913051)(105.34647597,219.93913057)(105.34648361,219.86913362)
\curveto(105.34647597,219.79913071)(105.33647598,219.73913077)(105.31648361,219.68913362)
\lineto(105.31648361,219.46413362)
\lineto(105.25648361,219.22413362)
\curveto(105.24647607,219.15413135)(105.23147609,219.08413142)(105.21148361,219.01413362)
\curveto(105.18147614,218.92413158)(105.15147617,218.83913167)(105.12148361,218.75913362)
\curveto(105.10147622,218.67913183)(105.07147625,218.59913191)(105.03148361,218.51913362)
\curveto(105.01147631,218.45913205)(104.98147634,218.39913211)(104.94148361,218.33913362)
\curveto(104.91147641,218.28913222)(104.87647644,218.23913227)(104.83648361,218.18913362)
\curveto(104.63647668,217.87913263)(104.38647693,217.61913289)(104.08648361,217.40913362)
\curveto(103.78647753,217.2091333)(103.44147788,217.04413346)(103.05148361,216.91413362)
\curveto(102.93147839,216.87413363)(102.80147852,216.84913366)(102.66148361,216.83913362)
\curveto(102.53147879,216.81913369)(102.39647892,216.79413371)(102.25648361,216.76413362)
\curveto(102.18647913,216.75413375)(102.1164792,216.74913376)(102.04648361,216.74913362)
\curveto(101.98647933,216.74913376)(101.9214794,216.74413376)(101.85148361,216.73413362)
\curveto(101.81147951,216.72413378)(101.75147957,216.71913379)(101.67148361,216.71913362)
\curveto(101.60147972,216.71913379)(101.55147977,216.72413378)(101.52148361,216.73413362)
\curveto(101.47147985,216.74413376)(101.42647989,216.74913376)(101.38648361,216.74913362)
\lineto(101.26648361,216.74913362)
\curveto(101.16648015,216.76913374)(101.06648025,216.78413372)(100.96648361,216.79413362)
\curveto(100.86648045,216.8041337)(100.77148055,216.81913369)(100.68148361,216.83913362)
\curveto(100.57148075,216.86913364)(100.46148086,216.89413361)(100.35148361,216.91413362)
\curveto(100.25148107,216.94413356)(100.14648117,216.98413352)(100.03648361,217.03413362)
\curveto(99.66648165,217.19413331)(99.35148197,217.39413311)(99.09148361,217.63413362)
\curveto(98.83148249,217.88413262)(98.6214827,218.19413231)(98.46148361,218.56413362)
\curveto(98.4214829,218.65413185)(98.38648293,218.74913176)(98.35648361,218.84913362)
\curveto(98.32648299,218.94913156)(98.29648302,219.05413145)(98.26648361,219.16413362)
\curveto(98.24648307,219.21413129)(98.23648308,219.26413124)(98.23648361,219.31413362)
\curveto(98.23648308,219.37413113)(98.22648309,219.43413107)(98.20648361,219.49413362)
\curveto(98.18648313,219.55413095)(98.17648314,219.63413087)(98.17648361,219.73413362)
\curveto(98.17648314,219.83413067)(98.19148313,219.9091306)(98.22148361,219.95913362)
\curveto(98.23148309,219.98913052)(98.24648307,220.01413049)(98.26648361,220.03413362)
\lineto(98.32648361,220.09413362)
\curveto(98.36648295,220.11413039)(98.42648289,220.12913038)(98.50648361,220.13913362)
\curveto(98.59648272,220.14913036)(98.68648263,220.15413035)(98.77648361,220.15413362)
\curveto(98.86648245,220.15413035)(98.95148237,220.14913036)(99.03148361,220.13913362)
\curveto(99.1214822,220.12913038)(99.18648213,220.11913039)(99.22648361,220.10913362)
\curveto(99.24648207,220.08913042)(99.26648205,220.07413043)(99.28648361,220.06413362)
\curveto(99.30648201,220.06413044)(99.32648199,220.05413045)(99.34648361,220.03413362)
\curveto(99.4164819,219.94413056)(99.45648186,219.82913068)(99.46648361,219.68913362)
\curveto(99.48648183,219.54913096)(99.5164818,219.42413108)(99.55648361,219.31413362)
\lineto(99.70648361,218.95413362)
\curveto(99.75648156,218.84413166)(99.8214815,218.73913177)(99.90148361,218.63913362)
\curveto(99.9214814,218.6091319)(99.94148138,218.58413192)(99.96148361,218.56413362)
\curveto(99.99148133,218.54413196)(100.0164813,218.51913199)(100.03648361,218.48913362)
\curveto(100.07648124,218.42913208)(100.11148121,218.38413212)(100.14148361,218.35413362)
\curveto(100.18148114,218.32413218)(100.2164811,218.29413221)(100.24648361,218.26413362)
\curveto(100.28648103,218.23413227)(100.33148099,218.2041323)(100.38148361,218.17413362)
\curveto(100.47148085,218.11413239)(100.56648075,218.06413244)(100.66648361,218.02413362)
\lineto(100.99648361,217.90413362)
\curveto(101.14648017,217.85413265)(101.34647997,217.82413268)(101.59648361,217.81413362)
\curveto(101.84647947,217.8041327)(102.05647926,217.82413268)(102.22648361,217.87413362)
\curveto(102.30647901,217.89413261)(102.37647894,217.9091326)(102.43648361,217.91913362)
\lineto(102.64648361,217.97913362)
\curveto(102.92647839,218.09913241)(103.16647815,218.24913226)(103.36648361,218.42913362)
\curveto(103.57647774,218.6091319)(103.74147758,218.83913167)(103.86148361,219.11913362)
\curveto(103.89147743,219.18913132)(103.91147741,219.25913125)(103.92148361,219.32913362)
\lineto(103.98148361,219.56913362)
\curveto(104.0214773,219.7091308)(104.03147729,219.86913064)(104.01148361,220.04913362)
\curveto(103.99147733,220.23913027)(103.96147736,220.38913012)(103.92148361,220.49913362)
\curveto(103.79147753,220.87912963)(103.60647771,221.16912934)(103.36648361,221.36913362)
\curveto(103.13647818,221.56912894)(102.82647849,221.72912878)(102.43648361,221.84913362)
\curveto(102.32647899,221.87912863)(102.20647911,221.89912861)(102.07648361,221.90913362)
\curveto(101.95647936,221.91912859)(101.83147949,221.92412858)(101.70148361,221.92413362)
\curveto(101.54147978,221.92412858)(101.40147992,221.92912858)(101.28148361,221.93913362)
\curveto(101.16148016,221.94912856)(101.07648024,222.0091285)(101.02648361,222.11913362)
\curveto(101.00648031,222.14912836)(100.99648032,222.18412832)(100.99648361,222.22413362)
\lineto(100.99648361,222.35913362)
\curveto(100.98648033,222.45912805)(100.98648033,222.55412795)(100.99648361,222.64413362)
\curveto(101.0164803,222.73412777)(101.05648026,222.79912771)(101.11648361,222.83913362)
\curveto(101.15648016,222.86912764)(101.19648012,222.88912762)(101.23648361,222.89913362)
\curveto(101.28648003,222.9091276)(101.34147998,222.91912759)(101.40148361,222.92913362)
\curveto(101.4214799,222.93912757)(101.44647987,222.93912757)(101.47648361,222.92913362)
\curveto(101.50647981,222.92912758)(101.53147979,222.93412757)(101.55148361,222.94413362)
\lineto(101.68648361,222.94413362)
\curveto(101.79647952,222.96412754)(101.89647942,222.97412753)(101.98648361,222.97413362)
\curveto(102.08647923,222.98412752)(102.18147914,223.0041275)(102.27148361,223.03413362)
\curveto(102.59147873,223.14412736)(102.84647847,223.28912722)(103.03648361,223.46913362)
\curveto(103.22647809,223.64912686)(103.37647794,223.89912661)(103.48648361,224.21913362)
\curveto(103.5164778,224.31912619)(103.53647778,224.44412606)(103.54648361,224.59413362)
\curveto(103.56647775,224.75412575)(103.56147776,224.89912561)(103.53148361,225.02913362)
\curveto(103.51147781,225.09912541)(103.49147783,225.16412534)(103.47148361,225.22413362)
\curveto(103.46147786,225.29412521)(103.44147788,225.35912515)(103.41148361,225.41913362)
\curveto(103.31147801,225.65912485)(103.16647815,225.84912466)(102.97648361,225.98913362)
\curveto(102.78647853,226.12912438)(102.56147876,226.23912427)(102.30148361,226.31913362)
\curveto(102.24147908,226.33912417)(102.18147914,226.34912416)(102.12148361,226.34913362)
\curveto(102.06147926,226.34912416)(101.99647932,226.35912415)(101.92648361,226.37913362)
\curveto(101.84647947,226.39912411)(101.75147957,226.4091241)(101.64148361,226.40913362)
\curveto(101.53147979,226.4091241)(101.43647988,226.39912411)(101.35648361,226.37913362)
\curveto(101.30648001,226.35912415)(101.25648006,226.34912416)(101.20648361,226.34913362)
\curveto(101.16648015,226.34912416)(101.1214802,226.33912417)(101.07148361,226.31913362)
\curveto(100.89148043,226.26912424)(100.7214806,226.19412431)(100.56148361,226.09413362)
\curveto(100.41148091,226.0041245)(100.28148104,225.88912462)(100.17148361,225.74913362)
\curveto(100.08148124,225.62912488)(100.00148132,225.49912501)(99.93148361,225.35913362)
\curveto(99.86148146,225.21912529)(99.79648152,225.06412544)(99.73648361,224.89413362)
\curveto(99.70648161,224.78412572)(99.68648163,224.66412584)(99.67648361,224.53413362)
\curveto(99.66648165,224.41412609)(99.63148169,224.31412619)(99.57148361,224.23413362)
\curveto(99.55148177,224.19412631)(99.49148183,224.15412635)(99.39148361,224.11413362)
\curveto(99.35148197,224.1041264)(99.29148203,224.09412641)(99.21148361,224.08413362)
\lineto(98.95648361,224.08413362)
\curveto(98.86648245,224.09412641)(98.78148254,224.1041264)(98.70148361,224.11413362)
\curveto(98.63148269,224.12412638)(98.58148274,224.13912637)(98.55148361,224.15913362)
\curveto(98.51148281,224.18912632)(98.47648284,224.24412626)(98.44648361,224.32413362)
\curveto(98.4164829,224.4041261)(98.41148291,224.48912602)(98.43148361,224.57913362)
\curveto(98.44148288,224.62912588)(98.44648287,224.67912583)(98.44648361,224.72913362)
\lineto(98.47648361,224.90913362)
\curveto(98.50648281,225.0091255)(98.53148279,225.1091254)(98.55148361,225.20913362)
\curveto(98.58148274,225.3091252)(98.6164827,225.39912511)(98.65648361,225.47913362)
\curveto(98.70648261,225.58912492)(98.75148257,225.69412481)(98.79148361,225.79413362)
\curveto(98.83148249,225.9041246)(98.88148244,226.0091245)(98.94148361,226.10913362)
\curveto(99.27148205,226.64912386)(99.74148158,227.04412346)(100.35148361,227.29413362)
\curveto(100.47148085,227.34412316)(100.59648072,227.37912313)(100.72648361,227.39913362)
\curveto(100.86648045,227.41912309)(101.00648031,227.44412306)(101.14648361,227.47413362)
\curveto(101.20648011,227.48412302)(101.26648005,227.48912302)(101.32648361,227.48913362)
\curveto(101.39647992,227.48912302)(101.46147986,227.49412301)(101.52148361,227.50413362)
}
}
{
\newrgbcolor{curcolor}{0 0 0}
\pscustom[linestyle=none,fillstyle=solid,fillcolor=curcolor]
{
\newpath
\moveto(109.87109298,227.50413362)
\curveto(111.50108754,227.53412297)(112.55108649,226.97912353)(113.02109298,225.83913362)
\curveto(113.12108592,225.6091249)(113.18608586,225.31912519)(113.21609298,224.96913362)
\curveto(113.25608579,224.62912588)(113.23108581,224.31912619)(113.14109298,224.03913362)
\curveto(113.05108599,223.77912673)(112.93108611,223.55412695)(112.78109298,223.36413362)
\curveto(112.76108628,223.32412718)(112.73608631,223.28912722)(112.70609298,223.25913362)
\curveto(112.67608637,223.23912727)(112.65108639,223.21412729)(112.63109298,223.18413362)
\lineto(112.54109298,223.06413362)
\curveto(112.51108653,223.03412747)(112.47608657,223.0091275)(112.43609298,222.98913362)
\curveto(112.38608666,222.93912757)(112.33108671,222.89412761)(112.27109298,222.85413362)
\curveto(112.22108682,222.81412769)(112.17608687,222.76412774)(112.13609298,222.70413362)
\curveto(112.09608695,222.66412784)(112.08108696,222.61412789)(112.09109298,222.55413362)
\curveto(112.10108694,222.504128)(112.13108691,222.45912805)(112.18109298,222.41913362)
\curveto(112.23108681,222.37912813)(112.28608676,222.33912817)(112.34609298,222.29913362)
\curveto(112.41608663,222.26912824)(112.48108656,222.23912827)(112.54109298,222.20913362)
\curveto(112.60108644,222.17912833)(112.65108639,222.14912836)(112.69109298,222.11913362)
\curveto(113.01108603,221.89912861)(113.26608578,221.58912892)(113.45609298,221.18913362)
\curveto(113.49608555,221.09912941)(113.52608552,221.0041295)(113.54609298,220.90413362)
\curveto(113.57608547,220.81412969)(113.60108544,220.72412978)(113.62109298,220.63413362)
\curveto(113.63108541,220.58412992)(113.63608541,220.53412997)(113.63609298,220.48413362)
\curveto(113.6460854,220.44413006)(113.65608539,220.39913011)(113.66609298,220.34913362)
\curveto(113.67608537,220.29913021)(113.67608537,220.24913026)(113.66609298,220.19913362)
\curveto(113.65608539,220.14913036)(113.66108538,220.09913041)(113.68109298,220.04913362)
\curveto(113.69108535,219.99913051)(113.69608535,219.93913057)(113.69609298,219.86913362)
\curveto(113.69608535,219.79913071)(113.68608536,219.73913077)(113.66609298,219.68913362)
\lineto(113.66609298,219.46413362)
\lineto(113.60609298,219.22413362)
\curveto(113.59608545,219.15413135)(113.58108546,219.08413142)(113.56109298,219.01413362)
\curveto(113.53108551,218.92413158)(113.50108554,218.83913167)(113.47109298,218.75913362)
\curveto(113.45108559,218.67913183)(113.42108562,218.59913191)(113.38109298,218.51913362)
\curveto(113.36108568,218.45913205)(113.33108571,218.39913211)(113.29109298,218.33913362)
\curveto(113.26108578,218.28913222)(113.22608582,218.23913227)(113.18609298,218.18913362)
\curveto(112.98608606,217.87913263)(112.73608631,217.61913289)(112.43609298,217.40913362)
\curveto(112.13608691,217.2091333)(111.79108725,217.04413346)(111.40109298,216.91413362)
\curveto(111.28108776,216.87413363)(111.15108789,216.84913366)(111.01109298,216.83913362)
\curveto(110.88108816,216.81913369)(110.7460883,216.79413371)(110.60609298,216.76413362)
\curveto(110.53608851,216.75413375)(110.46608858,216.74913376)(110.39609298,216.74913362)
\curveto(110.33608871,216.74913376)(110.27108877,216.74413376)(110.20109298,216.73413362)
\curveto(110.16108888,216.72413378)(110.10108894,216.71913379)(110.02109298,216.71913362)
\curveto(109.95108909,216.71913379)(109.90108914,216.72413378)(109.87109298,216.73413362)
\curveto(109.82108922,216.74413376)(109.77608927,216.74913376)(109.73609298,216.74913362)
\lineto(109.61609298,216.74913362)
\curveto(109.51608953,216.76913374)(109.41608963,216.78413372)(109.31609298,216.79413362)
\curveto(109.21608983,216.8041337)(109.12108992,216.81913369)(109.03109298,216.83913362)
\curveto(108.92109012,216.86913364)(108.81109023,216.89413361)(108.70109298,216.91413362)
\curveto(108.60109044,216.94413356)(108.49609055,216.98413352)(108.38609298,217.03413362)
\curveto(108.01609103,217.19413331)(107.70109134,217.39413311)(107.44109298,217.63413362)
\curveto(107.18109186,217.88413262)(106.97109207,218.19413231)(106.81109298,218.56413362)
\curveto(106.77109227,218.65413185)(106.73609231,218.74913176)(106.70609298,218.84913362)
\curveto(106.67609237,218.94913156)(106.6460924,219.05413145)(106.61609298,219.16413362)
\curveto(106.59609245,219.21413129)(106.58609246,219.26413124)(106.58609298,219.31413362)
\curveto(106.58609246,219.37413113)(106.57609247,219.43413107)(106.55609298,219.49413362)
\curveto(106.53609251,219.55413095)(106.52609252,219.63413087)(106.52609298,219.73413362)
\curveto(106.52609252,219.83413067)(106.5410925,219.9091306)(106.57109298,219.95913362)
\curveto(106.58109246,219.98913052)(106.59609245,220.01413049)(106.61609298,220.03413362)
\lineto(106.67609298,220.09413362)
\curveto(106.71609233,220.11413039)(106.77609227,220.12913038)(106.85609298,220.13913362)
\curveto(106.9460921,220.14913036)(107.03609201,220.15413035)(107.12609298,220.15413362)
\curveto(107.21609183,220.15413035)(107.30109174,220.14913036)(107.38109298,220.13913362)
\curveto(107.47109157,220.12913038)(107.53609151,220.11913039)(107.57609298,220.10913362)
\curveto(107.59609145,220.08913042)(107.61609143,220.07413043)(107.63609298,220.06413362)
\curveto(107.65609139,220.06413044)(107.67609137,220.05413045)(107.69609298,220.03413362)
\curveto(107.76609128,219.94413056)(107.80609124,219.82913068)(107.81609298,219.68913362)
\curveto(107.83609121,219.54913096)(107.86609118,219.42413108)(107.90609298,219.31413362)
\lineto(108.05609298,218.95413362)
\curveto(108.10609094,218.84413166)(108.17109087,218.73913177)(108.25109298,218.63913362)
\curveto(108.27109077,218.6091319)(108.29109075,218.58413192)(108.31109298,218.56413362)
\curveto(108.3410907,218.54413196)(108.36609068,218.51913199)(108.38609298,218.48913362)
\curveto(108.42609062,218.42913208)(108.46109058,218.38413212)(108.49109298,218.35413362)
\curveto(108.53109051,218.32413218)(108.56609048,218.29413221)(108.59609298,218.26413362)
\curveto(108.63609041,218.23413227)(108.68109036,218.2041323)(108.73109298,218.17413362)
\curveto(108.82109022,218.11413239)(108.91609013,218.06413244)(109.01609298,218.02413362)
\lineto(109.34609298,217.90413362)
\curveto(109.49608955,217.85413265)(109.69608935,217.82413268)(109.94609298,217.81413362)
\curveto(110.19608885,217.8041327)(110.40608864,217.82413268)(110.57609298,217.87413362)
\curveto(110.65608839,217.89413261)(110.72608832,217.9091326)(110.78609298,217.91913362)
\lineto(110.99609298,217.97913362)
\curveto(111.27608777,218.09913241)(111.51608753,218.24913226)(111.71609298,218.42913362)
\curveto(111.92608712,218.6091319)(112.09108695,218.83913167)(112.21109298,219.11913362)
\curveto(112.2410868,219.18913132)(112.26108678,219.25913125)(112.27109298,219.32913362)
\lineto(112.33109298,219.56913362)
\curveto(112.37108667,219.7091308)(112.38108666,219.86913064)(112.36109298,220.04913362)
\curveto(112.3410867,220.23913027)(112.31108673,220.38913012)(112.27109298,220.49913362)
\curveto(112.1410869,220.87912963)(111.95608709,221.16912934)(111.71609298,221.36913362)
\curveto(111.48608756,221.56912894)(111.17608787,221.72912878)(110.78609298,221.84913362)
\curveto(110.67608837,221.87912863)(110.55608849,221.89912861)(110.42609298,221.90913362)
\curveto(110.30608874,221.91912859)(110.18108886,221.92412858)(110.05109298,221.92413362)
\curveto(109.89108915,221.92412858)(109.75108929,221.92912858)(109.63109298,221.93913362)
\curveto(109.51108953,221.94912856)(109.42608962,222.0091285)(109.37609298,222.11913362)
\curveto(109.35608969,222.14912836)(109.3460897,222.18412832)(109.34609298,222.22413362)
\lineto(109.34609298,222.35913362)
\curveto(109.33608971,222.45912805)(109.33608971,222.55412795)(109.34609298,222.64413362)
\curveto(109.36608968,222.73412777)(109.40608964,222.79912771)(109.46609298,222.83913362)
\curveto(109.50608954,222.86912764)(109.5460895,222.88912762)(109.58609298,222.89913362)
\curveto(109.63608941,222.9091276)(109.69108935,222.91912759)(109.75109298,222.92913362)
\curveto(109.77108927,222.93912757)(109.79608925,222.93912757)(109.82609298,222.92913362)
\curveto(109.85608919,222.92912758)(109.88108916,222.93412757)(109.90109298,222.94413362)
\lineto(110.03609298,222.94413362)
\curveto(110.1460889,222.96412754)(110.2460888,222.97412753)(110.33609298,222.97413362)
\curveto(110.43608861,222.98412752)(110.53108851,223.0041275)(110.62109298,223.03413362)
\curveto(110.9410881,223.14412736)(111.19608785,223.28912722)(111.38609298,223.46913362)
\curveto(111.57608747,223.64912686)(111.72608732,223.89912661)(111.83609298,224.21913362)
\curveto(111.86608718,224.31912619)(111.88608716,224.44412606)(111.89609298,224.59413362)
\curveto(111.91608713,224.75412575)(111.91108713,224.89912561)(111.88109298,225.02913362)
\curveto(111.86108718,225.09912541)(111.8410872,225.16412534)(111.82109298,225.22413362)
\curveto(111.81108723,225.29412521)(111.79108725,225.35912515)(111.76109298,225.41913362)
\curveto(111.66108738,225.65912485)(111.51608753,225.84912466)(111.32609298,225.98913362)
\curveto(111.13608791,226.12912438)(110.91108813,226.23912427)(110.65109298,226.31913362)
\curveto(110.59108845,226.33912417)(110.53108851,226.34912416)(110.47109298,226.34913362)
\curveto(110.41108863,226.34912416)(110.3460887,226.35912415)(110.27609298,226.37913362)
\curveto(110.19608885,226.39912411)(110.10108894,226.4091241)(109.99109298,226.40913362)
\curveto(109.88108916,226.4091241)(109.78608926,226.39912411)(109.70609298,226.37913362)
\curveto(109.65608939,226.35912415)(109.60608944,226.34912416)(109.55609298,226.34913362)
\curveto(109.51608953,226.34912416)(109.47108957,226.33912417)(109.42109298,226.31913362)
\curveto(109.2410898,226.26912424)(109.07108997,226.19412431)(108.91109298,226.09413362)
\curveto(108.76109028,226.0041245)(108.63109041,225.88912462)(108.52109298,225.74913362)
\curveto(108.43109061,225.62912488)(108.35109069,225.49912501)(108.28109298,225.35913362)
\curveto(108.21109083,225.21912529)(108.1460909,225.06412544)(108.08609298,224.89413362)
\curveto(108.05609099,224.78412572)(108.03609101,224.66412584)(108.02609298,224.53413362)
\curveto(108.01609103,224.41412609)(107.98109106,224.31412619)(107.92109298,224.23413362)
\curveto(107.90109114,224.19412631)(107.8410912,224.15412635)(107.74109298,224.11413362)
\curveto(107.70109134,224.1041264)(107.6410914,224.09412641)(107.56109298,224.08413362)
\lineto(107.30609298,224.08413362)
\curveto(107.21609183,224.09412641)(107.13109191,224.1041264)(107.05109298,224.11413362)
\curveto(106.98109206,224.12412638)(106.93109211,224.13912637)(106.90109298,224.15913362)
\curveto(106.86109218,224.18912632)(106.82609222,224.24412626)(106.79609298,224.32413362)
\curveto(106.76609228,224.4041261)(106.76109228,224.48912602)(106.78109298,224.57913362)
\curveto(106.79109225,224.62912588)(106.79609225,224.67912583)(106.79609298,224.72913362)
\lineto(106.82609298,224.90913362)
\curveto(106.85609219,225.0091255)(106.88109216,225.1091254)(106.90109298,225.20913362)
\curveto(106.93109211,225.3091252)(106.96609208,225.39912511)(107.00609298,225.47913362)
\curveto(107.05609199,225.58912492)(107.10109194,225.69412481)(107.14109298,225.79413362)
\curveto(107.18109186,225.9041246)(107.23109181,226.0091245)(107.29109298,226.10913362)
\curveto(107.62109142,226.64912386)(108.09109095,227.04412346)(108.70109298,227.29413362)
\curveto(108.82109022,227.34412316)(108.9460901,227.37912313)(109.07609298,227.39913362)
\curveto(109.21608983,227.41912309)(109.35608969,227.44412306)(109.49609298,227.47413362)
\curveto(109.55608949,227.48412302)(109.61608943,227.48912302)(109.67609298,227.48913362)
\curveto(109.7460893,227.48912302)(109.81108923,227.49412301)(109.87109298,227.50413362)
}
}
{
\newrgbcolor{curcolor}{0 0 0}
\pscustom[linestyle=none,fillstyle=solid,fillcolor=curcolor]
{
\newpath
\moveto(116.06070236,218.53413362)
\lineto(116.36070236,218.53413362)
\curveto(116.4707003,218.54413196)(116.57570019,218.54413196)(116.67570236,218.53413362)
\curveto(116.78569998,218.53413197)(116.88569988,218.52413198)(116.97570236,218.50413362)
\curveto(117.0656997,218.49413201)(117.13569963,218.46913204)(117.18570236,218.42913362)
\curveto(117.20569956,218.4091321)(117.22069955,218.37913213)(117.23070236,218.33913362)
\curveto(117.25069952,218.29913221)(117.2706995,218.25413225)(117.29070236,218.20413362)
\lineto(117.29070236,218.12913362)
\curveto(117.30069947,218.07913243)(117.30069947,218.02413248)(117.29070236,217.96413362)
\lineto(117.29070236,217.81413362)
\lineto(117.29070236,217.33413362)
\curveto(117.29069948,217.16413334)(117.25069952,217.04413346)(117.17070236,216.97413362)
\curveto(117.10069967,216.92413358)(117.01069976,216.89913361)(116.90070236,216.89913362)
\lineto(116.57070236,216.89913362)
\lineto(116.12070236,216.89913362)
\curveto(115.9707008,216.89913361)(115.85570091,216.92913358)(115.77570236,216.98913362)
\curveto(115.73570103,217.01913349)(115.70570106,217.06913344)(115.68570236,217.13913362)
\curveto(115.6657011,217.21913329)(115.65070112,217.3041332)(115.64070236,217.39413362)
\lineto(115.64070236,217.67913362)
\curveto(115.65070112,217.77913273)(115.65570111,217.86413264)(115.65570236,217.93413362)
\lineto(115.65570236,218.12913362)
\curveto(115.65570111,218.18913232)(115.6657011,218.24413226)(115.68570236,218.29413362)
\curveto(115.72570104,218.4041321)(115.79570097,218.47413203)(115.89570236,218.50413362)
\curveto(115.92570084,218.504132)(115.98070079,218.51413199)(116.06070236,218.53413362)
}
}
{
\newrgbcolor{curcolor}{0 0 0}
\pscustom[linestyle=none,fillstyle=solid,fillcolor=curcolor]
{
\newpath
\moveto(122.38085861,227.50413362)
\curveto(124.01085317,227.53412297)(125.06085212,226.97912353)(125.53085861,225.83913362)
\curveto(125.63085155,225.6091249)(125.69585148,225.31912519)(125.72585861,224.96913362)
\curveto(125.76585141,224.62912588)(125.74085144,224.31912619)(125.65085861,224.03913362)
\curveto(125.56085162,223.77912673)(125.44085174,223.55412695)(125.29085861,223.36413362)
\curveto(125.27085191,223.32412718)(125.24585193,223.28912722)(125.21585861,223.25913362)
\curveto(125.18585199,223.23912727)(125.16085202,223.21412729)(125.14085861,223.18413362)
\lineto(125.05085861,223.06413362)
\curveto(125.02085216,223.03412747)(124.98585219,223.0091275)(124.94585861,222.98913362)
\curveto(124.89585228,222.93912757)(124.84085234,222.89412761)(124.78085861,222.85413362)
\curveto(124.73085245,222.81412769)(124.68585249,222.76412774)(124.64585861,222.70413362)
\curveto(124.60585257,222.66412784)(124.59085259,222.61412789)(124.60085861,222.55413362)
\curveto(124.61085257,222.504128)(124.64085254,222.45912805)(124.69085861,222.41913362)
\curveto(124.74085244,222.37912813)(124.79585238,222.33912817)(124.85585861,222.29913362)
\curveto(124.92585225,222.26912824)(124.99085219,222.23912827)(125.05085861,222.20913362)
\curveto(125.11085207,222.17912833)(125.16085202,222.14912836)(125.20085861,222.11913362)
\curveto(125.52085166,221.89912861)(125.7758514,221.58912892)(125.96585861,221.18913362)
\curveto(126.00585117,221.09912941)(126.03585114,221.0041295)(126.05585861,220.90413362)
\curveto(126.08585109,220.81412969)(126.11085107,220.72412978)(126.13085861,220.63413362)
\curveto(126.14085104,220.58412992)(126.14585103,220.53412997)(126.14585861,220.48413362)
\curveto(126.15585102,220.44413006)(126.16585101,220.39913011)(126.17585861,220.34913362)
\curveto(126.18585099,220.29913021)(126.18585099,220.24913026)(126.17585861,220.19913362)
\curveto(126.16585101,220.14913036)(126.17085101,220.09913041)(126.19085861,220.04913362)
\curveto(126.20085098,219.99913051)(126.20585097,219.93913057)(126.20585861,219.86913362)
\curveto(126.20585097,219.79913071)(126.19585098,219.73913077)(126.17585861,219.68913362)
\lineto(126.17585861,219.46413362)
\lineto(126.11585861,219.22413362)
\curveto(126.10585107,219.15413135)(126.09085109,219.08413142)(126.07085861,219.01413362)
\curveto(126.04085114,218.92413158)(126.01085117,218.83913167)(125.98085861,218.75913362)
\curveto(125.96085122,218.67913183)(125.93085125,218.59913191)(125.89085861,218.51913362)
\curveto(125.87085131,218.45913205)(125.84085134,218.39913211)(125.80085861,218.33913362)
\curveto(125.77085141,218.28913222)(125.73585144,218.23913227)(125.69585861,218.18913362)
\curveto(125.49585168,217.87913263)(125.24585193,217.61913289)(124.94585861,217.40913362)
\curveto(124.64585253,217.2091333)(124.30085288,217.04413346)(123.91085861,216.91413362)
\curveto(123.79085339,216.87413363)(123.66085352,216.84913366)(123.52085861,216.83913362)
\curveto(123.39085379,216.81913369)(123.25585392,216.79413371)(123.11585861,216.76413362)
\curveto(123.04585413,216.75413375)(122.9758542,216.74913376)(122.90585861,216.74913362)
\curveto(122.84585433,216.74913376)(122.7808544,216.74413376)(122.71085861,216.73413362)
\curveto(122.67085451,216.72413378)(122.61085457,216.71913379)(122.53085861,216.71913362)
\curveto(122.46085472,216.71913379)(122.41085477,216.72413378)(122.38085861,216.73413362)
\curveto(122.33085485,216.74413376)(122.28585489,216.74913376)(122.24585861,216.74913362)
\lineto(122.12585861,216.74913362)
\curveto(122.02585515,216.76913374)(121.92585525,216.78413372)(121.82585861,216.79413362)
\curveto(121.72585545,216.8041337)(121.63085555,216.81913369)(121.54085861,216.83913362)
\curveto(121.43085575,216.86913364)(121.32085586,216.89413361)(121.21085861,216.91413362)
\curveto(121.11085607,216.94413356)(121.00585617,216.98413352)(120.89585861,217.03413362)
\curveto(120.52585665,217.19413331)(120.21085697,217.39413311)(119.95085861,217.63413362)
\curveto(119.69085749,217.88413262)(119.4808577,218.19413231)(119.32085861,218.56413362)
\curveto(119.2808579,218.65413185)(119.24585793,218.74913176)(119.21585861,218.84913362)
\curveto(119.18585799,218.94913156)(119.15585802,219.05413145)(119.12585861,219.16413362)
\curveto(119.10585807,219.21413129)(119.09585808,219.26413124)(119.09585861,219.31413362)
\curveto(119.09585808,219.37413113)(119.08585809,219.43413107)(119.06585861,219.49413362)
\curveto(119.04585813,219.55413095)(119.03585814,219.63413087)(119.03585861,219.73413362)
\curveto(119.03585814,219.83413067)(119.05085813,219.9091306)(119.08085861,219.95913362)
\curveto(119.09085809,219.98913052)(119.10585807,220.01413049)(119.12585861,220.03413362)
\lineto(119.18585861,220.09413362)
\curveto(119.22585795,220.11413039)(119.28585789,220.12913038)(119.36585861,220.13913362)
\curveto(119.45585772,220.14913036)(119.54585763,220.15413035)(119.63585861,220.15413362)
\curveto(119.72585745,220.15413035)(119.81085737,220.14913036)(119.89085861,220.13913362)
\curveto(119.9808572,220.12913038)(120.04585713,220.11913039)(120.08585861,220.10913362)
\curveto(120.10585707,220.08913042)(120.12585705,220.07413043)(120.14585861,220.06413362)
\curveto(120.16585701,220.06413044)(120.18585699,220.05413045)(120.20585861,220.03413362)
\curveto(120.2758569,219.94413056)(120.31585686,219.82913068)(120.32585861,219.68913362)
\curveto(120.34585683,219.54913096)(120.3758568,219.42413108)(120.41585861,219.31413362)
\lineto(120.56585861,218.95413362)
\curveto(120.61585656,218.84413166)(120.6808565,218.73913177)(120.76085861,218.63913362)
\curveto(120.7808564,218.6091319)(120.80085638,218.58413192)(120.82085861,218.56413362)
\curveto(120.85085633,218.54413196)(120.8758563,218.51913199)(120.89585861,218.48913362)
\curveto(120.93585624,218.42913208)(120.97085621,218.38413212)(121.00085861,218.35413362)
\curveto(121.04085614,218.32413218)(121.0758561,218.29413221)(121.10585861,218.26413362)
\curveto(121.14585603,218.23413227)(121.19085599,218.2041323)(121.24085861,218.17413362)
\curveto(121.33085585,218.11413239)(121.42585575,218.06413244)(121.52585861,218.02413362)
\lineto(121.85585861,217.90413362)
\curveto(122.00585517,217.85413265)(122.20585497,217.82413268)(122.45585861,217.81413362)
\curveto(122.70585447,217.8041327)(122.91585426,217.82413268)(123.08585861,217.87413362)
\curveto(123.16585401,217.89413261)(123.23585394,217.9091326)(123.29585861,217.91913362)
\lineto(123.50585861,217.97913362)
\curveto(123.78585339,218.09913241)(124.02585315,218.24913226)(124.22585861,218.42913362)
\curveto(124.43585274,218.6091319)(124.60085258,218.83913167)(124.72085861,219.11913362)
\curveto(124.75085243,219.18913132)(124.77085241,219.25913125)(124.78085861,219.32913362)
\lineto(124.84085861,219.56913362)
\curveto(124.8808523,219.7091308)(124.89085229,219.86913064)(124.87085861,220.04913362)
\curveto(124.85085233,220.23913027)(124.82085236,220.38913012)(124.78085861,220.49913362)
\curveto(124.65085253,220.87912963)(124.46585271,221.16912934)(124.22585861,221.36913362)
\curveto(123.99585318,221.56912894)(123.68585349,221.72912878)(123.29585861,221.84913362)
\curveto(123.18585399,221.87912863)(123.06585411,221.89912861)(122.93585861,221.90913362)
\curveto(122.81585436,221.91912859)(122.69085449,221.92412858)(122.56085861,221.92413362)
\curveto(122.40085478,221.92412858)(122.26085492,221.92912858)(122.14085861,221.93913362)
\curveto(122.02085516,221.94912856)(121.93585524,222.0091285)(121.88585861,222.11913362)
\curveto(121.86585531,222.14912836)(121.85585532,222.18412832)(121.85585861,222.22413362)
\lineto(121.85585861,222.35913362)
\curveto(121.84585533,222.45912805)(121.84585533,222.55412795)(121.85585861,222.64413362)
\curveto(121.8758553,222.73412777)(121.91585526,222.79912771)(121.97585861,222.83913362)
\curveto(122.01585516,222.86912764)(122.05585512,222.88912762)(122.09585861,222.89913362)
\curveto(122.14585503,222.9091276)(122.20085498,222.91912759)(122.26085861,222.92913362)
\curveto(122.2808549,222.93912757)(122.30585487,222.93912757)(122.33585861,222.92913362)
\curveto(122.36585481,222.92912758)(122.39085479,222.93412757)(122.41085861,222.94413362)
\lineto(122.54585861,222.94413362)
\curveto(122.65585452,222.96412754)(122.75585442,222.97412753)(122.84585861,222.97413362)
\curveto(122.94585423,222.98412752)(123.04085414,223.0041275)(123.13085861,223.03413362)
\curveto(123.45085373,223.14412736)(123.70585347,223.28912722)(123.89585861,223.46913362)
\curveto(124.08585309,223.64912686)(124.23585294,223.89912661)(124.34585861,224.21913362)
\curveto(124.3758528,224.31912619)(124.39585278,224.44412606)(124.40585861,224.59413362)
\curveto(124.42585275,224.75412575)(124.42085276,224.89912561)(124.39085861,225.02913362)
\curveto(124.37085281,225.09912541)(124.35085283,225.16412534)(124.33085861,225.22413362)
\curveto(124.32085286,225.29412521)(124.30085288,225.35912515)(124.27085861,225.41913362)
\curveto(124.17085301,225.65912485)(124.02585315,225.84912466)(123.83585861,225.98913362)
\curveto(123.64585353,226.12912438)(123.42085376,226.23912427)(123.16085861,226.31913362)
\curveto(123.10085408,226.33912417)(123.04085414,226.34912416)(122.98085861,226.34913362)
\curveto(122.92085426,226.34912416)(122.85585432,226.35912415)(122.78585861,226.37913362)
\curveto(122.70585447,226.39912411)(122.61085457,226.4091241)(122.50085861,226.40913362)
\curveto(122.39085479,226.4091241)(122.29585488,226.39912411)(122.21585861,226.37913362)
\curveto(122.16585501,226.35912415)(122.11585506,226.34912416)(122.06585861,226.34913362)
\curveto(122.02585515,226.34912416)(121.9808552,226.33912417)(121.93085861,226.31913362)
\curveto(121.75085543,226.26912424)(121.5808556,226.19412431)(121.42085861,226.09413362)
\curveto(121.27085591,226.0041245)(121.14085604,225.88912462)(121.03085861,225.74913362)
\curveto(120.94085624,225.62912488)(120.86085632,225.49912501)(120.79085861,225.35913362)
\curveto(120.72085646,225.21912529)(120.65585652,225.06412544)(120.59585861,224.89413362)
\curveto(120.56585661,224.78412572)(120.54585663,224.66412584)(120.53585861,224.53413362)
\curveto(120.52585665,224.41412609)(120.49085669,224.31412619)(120.43085861,224.23413362)
\curveto(120.41085677,224.19412631)(120.35085683,224.15412635)(120.25085861,224.11413362)
\curveto(120.21085697,224.1041264)(120.15085703,224.09412641)(120.07085861,224.08413362)
\lineto(119.81585861,224.08413362)
\curveto(119.72585745,224.09412641)(119.64085754,224.1041264)(119.56085861,224.11413362)
\curveto(119.49085769,224.12412638)(119.44085774,224.13912637)(119.41085861,224.15913362)
\curveto(119.37085781,224.18912632)(119.33585784,224.24412626)(119.30585861,224.32413362)
\curveto(119.2758579,224.4041261)(119.27085791,224.48912602)(119.29085861,224.57913362)
\curveto(119.30085788,224.62912588)(119.30585787,224.67912583)(119.30585861,224.72913362)
\lineto(119.33585861,224.90913362)
\curveto(119.36585781,225.0091255)(119.39085779,225.1091254)(119.41085861,225.20913362)
\curveto(119.44085774,225.3091252)(119.4758577,225.39912511)(119.51585861,225.47913362)
\curveto(119.56585761,225.58912492)(119.61085757,225.69412481)(119.65085861,225.79413362)
\curveto(119.69085749,225.9041246)(119.74085744,226.0091245)(119.80085861,226.10913362)
\curveto(120.13085705,226.64912386)(120.60085658,227.04412346)(121.21085861,227.29413362)
\curveto(121.33085585,227.34412316)(121.45585572,227.37912313)(121.58585861,227.39913362)
\curveto(121.72585545,227.41912309)(121.86585531,227.44412306)(122.00585861,227.47413362)
\curveto(122.06585511,227.48412302)(122.12585505,227.48912302)(122.18585861,227.48913362)
\curveto(122.25585492,227.48912302)(122.32085486,227.49412301)(122.38085861,227.50413362)
}
}
{
\newrgbcolor{curcolor}{0 0 0}
\pscustom[linestyle=none,fillstyle=solid,fillcolor=curcolor]
{
\newpath
\moveto(137.42046798,225.41913362)
\curveto(137.22045768,225.12912538)(137.01045789,224.84412566)(136.79046798,224.56413362)
\curveto(136.58045832,224.28412622)(136.37545853,223.99912651)(136.17546798,223.70913362)
\curveto(135.57545933,222.85912765)(134.97045993,222.01912849)(134.36046798,221.18913362)
\curveto(133.75046115,220.36913014)(133.14546176,219.53413097)(132.54546798,218.68413362)
\lineto(132.03546798,217.96413362)
\lineto(131.52546798,217.27413362)
\curveto(131.44546346,217.16413334)(131.36546354,217.04913346)(131.28546798,216.92913362)
\curveto(131.2054637,216.8091337)(131.11046379,216.71413379)(131.00046798,216.64413362)
\curveto(130.96046394,216.62413388)(130.89546401,216.6091339)(130.80546798,216.59913362)
\curveto(130.72546418,216.57913393)(130.63546427,216.56913394)(130.53546798,216.56913362)
\curveto(130.43546447,216.56913394)(130.34046456,216.57413393)(130.25046798,216.58413362)
\curveto(130.17046473,216.59413391)(130.11046479,216.61413389)(130.07046798,216.64413362)
\curveto(130.04046486,216.66413384)(130.01546489,216.69913381)(129.99546798,216.74913362)
\curveto(129.98546492,216.78913372)(129.99046491,216.83413367)(130.01046798,216.88413362)
\curveto(130.05046485,216.96413354)(130.09546481,217.03913347)(130.14546798,217.10913362)
\curveto(130.2054647,217.18913332)(130.26046464,217.26913324)(130.31046798,217.34913362)
\curveto(130.55046435,217.68913282)(130.79546411,218.02413248)(131.04546798,218.35413362)
\curveto(131.29546361,218.68413182)(131.53546337,219.01913149)(131.76546798,219.35913362)
\curveto(131.92546298,219.57913093)(132.08546282,219.79413071)(132.24546798,220.00413362)
\curveto(132.4054625,220.21413029)(132.56546234,220.42913008)(132.72546798,220.64913362)
\curveto(133.08546182,221.16912934)(133.45046145,221.67912883)(133.82046798,222.17913362)
\curveto(134.19046071,222.67912783)(134.56046034,223.18912732)(134.93046798,223.70913362)
\curveto(135.07045983,223.9091266)(135.21045969,224.1041264)(135.35046798,224.29413362)
\curveto(135.5004594,224.48412602)(135.64545926,224.67912583)(135.78546798,224.87913362)
\curveto(135.99545891,225.17912533)(136.21045869,225.47912503)(136.43046798,225.77913362)
\lineto(137.09046798,226.67913362)
\lineto(137.27046798,226.94913362)
\lineto(137.48046798,227.21913362)
\lineto(137.60046798,227.39913362)
\curveto(137.65045725,227.45912305)(137.7004572,227.51412299)(137.75046798,227.56413362)
\curveto(137.82045708,227.61412289)(137.89545701,227.64912286)(137.97546798,227.66913362)
\curveto(137.99545691,227.67912283)(138.02045688,227.67912283)(138.05046798,227.66913362)
\curveto(138.09045681,227.66912284)(138.12045678,227.67912283)(138.14046798,227.69913362)
\curveto(138.26045664,227.69912281)(138.39545651,227.69412281)(138.54546798,227.68413362)
\curveto(138.69545621,227.68412282)(138.78545612,227.63912287)(138.81546798,227.54913362)
\curveto(138.83545607,227.51912299)(138.84045606,227.48412302)(138.83046798,227.44413362)
\curveto(138.82045608,227.4041231)(138.8054561,227.37412313)(138.78546798,227.35413362)
\curveto(138.74545616,227.27412323)(138.7054562,227.2041233)(138.66546798,227.14413362)
\curveto(138.62545628,227.08412342)(138.58045632,227.02412348)(138.53046798,226.96413362)
\lineto(137.96046798,226.18413362)
\curveto(137.78045712,225.93412457)(137.6004573,225.67912483)(137.42046798,225.41913362)
\moveto(130.56546798,221.51913362)
\curveto(130.51546439,221.53912897)(130.46546444,221.54412896)(130.41546798,221.53413362)
\curveto(130.36546454,221.52412898)(130.31546459,221.52912898)(130.26546798,221.54913362)
\curveto(130.15546475,221.56912894)(130.05046485,221.58912892)(129.95046798,221.60913362)
\curveto(129.86046504,221.63912887)(129.76546514,221.67912883)(129.66546798,221.72913362)
\curveto(129.33546557,221.86912864)(129.08046582,222.06412844)(128.90046798,222.31413362)
\curveto(128.72046618,222.57412793)(128.57546633,222.88412762)(128.46546798,223.24413362)
\curveto(128.43546647,223.32412718)(128.41546649,223.4041271)(128.40546798,223.48413362)
\curveto(128.39546651,223.57412693)(128.38046652,223.65912685)(128.36046798,223.73913362)
\curveto(128.35046655,223.78912672)(128.34546656,223.85412665)(128.34546798,223.93413362)
\curveto(128.33546657,223.96412654)(128.33046657,223.99412651)(128.33046798,224.02413362)
\curveto(128.33046657,224.06412644)(128.32546658,224.09912641)(128.31546798,224.12913362)
\lineto(128.31546798,224.27913362)
\curveto(128.3054666,224.32912618)(128.3004666,224.38912612)(128.30046798,224.45913362)
\curveto(128.3004666,224.53912597)(128.3054666,224.6041259)(128.31546798,224.65413362)
\lineto(128.31546798,224.81913362)
\curveto(128.33546657,224.86912564)(128.34046656,224.91412559)(128.33046798,224.95413362)
\curveto(128.33046657,225.0041255)(128.33546657,225.04912546)(128.34546798,225.08913362)
\curveto(128.35546655,225.12912538)(128.36046654,225.16412534)(128.36046798,225.19413362)
\curveto(128.36046654,225.23412527)(128.36546654,225.27412523)(128.37546798,225.31413362)
\curveto(128.4054665,225.42412508)(128.42546648,225.53412497)(128.43546798,225.64413362)
\curveto(128.45546645,225.76412474)(128.49046641,225.87912463)(128.54046798,225.98913362)
\curveto(128.68046622,226.32912418)(128.84046606,226.6041239)(129.02046798,226.81413362)
\curveto(129.21046569,227.03412347)(129.48046542,227.21412329)(129.83046798,227.35413362)
\curveto(129.91046499,227.38412312)(129.99546491,227.4041231)(130.08546798,227.41413362)
\curveto(130.17546473,227.43412307)(130.27046463,227.45412305)(130.37046798,227.47413362)
\curveto(130.4004645,227.48412302)(130.45546445,227.48412302)(130.53546798,227.47413362)
\curveto(130.61546429,227.47412303)(130.66546424,227.48412302)(130.68546798,227.50413362)
\curveto(131.24546366,227.51412299)(131.69546321,227.4041231)(132.03546798,227.17413362)
\curveto(132.38546252,226.94412356)(132.64546226,226.63912387)(132.81546798,226.25913362)
\curveto(132.85546205,226.16912434)(132.89046201,226.07412443)(132.92046798,225.97413362)
\curveto(132.95046195,225.87412463)(132.97546193,225.77412473)(132.99546798,225.67413362)
\curveto(133.01546189,225.64412486)(133.02046188,225.61412489)(133.01046798,225.58413362)
\curveto(133.01046189,225.55412495)(133.01546189,225.52412498)(133.02546798,225.49413362)
\curveto(133.05546185,225.38412512)(133.07546183,225.25912525)(133.08546798,225.11913362)
\curveto(133.09546181,224.98912552)(133.1054618,224.85412565)(133.11546798,224.71413362)
\lineto(133.11546798,224.54913362)
\curveto(133.12546178,224.48912602)(133.12546178,224.43412607)(133.11546798,224.38413362)
\curveto(133.1054618,224.33412617)(133.1004618,224.28412622)(133.10046798,224.23413362)
\lineto(133.10046798,224.09913362)
\curveto(133.09046181,224.05912645)(133.08546182,224.01912649)(133.08546798,223.97913362)
\curveto(133.09546181,223.93912657)(133.09046181,223.89412661)(133.07046798,223.84413362)
\curveto(133.05046185,223.73412677)(133.03046187,223.62912688)(133.01046798,223.52913362)
\curveto(133.0004619,223.42912708)(132.98046192,223.32912718)(132.95046798,223.22913362)
\curveto(132.82046208,222.86912764)(132.65546225,222.55412795)(132.45546798,222.28413362)
\curveto(132.25546265,222.01412849)(131.98046292,221.8091287)(131.63046798,221.66913362)
\curveto(131.55046335,221.63912887)(131.46546344,221.61412889)(131.37546798,221.59413362)
\lineto(131.10546798,221.53413362)
\curveto(131.05546385,221.52412898)(131.01046389,221.51912899)(130.97046798,221.51913362)
\curveto(130.93046397,221.52912898)(130.89046401,221.52912898)(130.85046798,221.51913362)
\curveto(130.75046415,221.49912901)(130.65546425,221.49912901)(130.56546798,221.51913362)
\moveto(129.72546798,222.91413362)
\curveto(129.76546514,222.84412766)(129.8054651,222.77912773)(129.84546798,222.71913362)
\curveto(129.88546502,222.66912784)(129.93546497,222.61912789)(129.99546798,222.56913362)
\lineto(130.14546798,222.44913362)
\curveto(130.2054647,222.41912809)(130.27046463,222.39412811)(130.34046798,222.37413362)
\curveto(130.38046452,222.35412815)(130.41546449,222.34412816)(130.44546798,222.34413362)
\curveto(130.48546442,222.35412815)(130.52546438,222.34912816)(130.56546798,222.32913362)
\curveto(130.59546431,222.32912818)(130.63546427,222.32412818)(130.68546798,222.31413362)
\curveto(130.73546417,222.31412819)(130.77546413,222.31912819)(130.80546798,222.32913362)
\lineto(131.03046798,222.37413362)
\curveto(131.28046362,222.45412805)(131.46546344,222.57912793)(131.58546798,222.74913362)
\curveto(131.66546324,222.84912766)(131.73546317,222.97912753)(131.79546798,223.13913362)
\curveto(131.87546303,223.31912719)(131.93546297,223.54412696)(131.97546798,223.81413362)
\curveto(132.01546289,224.09412641)(132.03046287,224.37412613)(132.02046798,224.65413362)
\curveto(132.01046289,224.94412556)(131.98046292,225.21912529)(131.93046798,225.47913362)
\curveto(131.88046302,225.73912477)(131.8054631,225.94912456)(131.70546798,226.10913362)
\curveto(131.58546332,226.3091242)(131.43546347,226.45912405)(131.25546798,226.55913362)
\curveto(131.17546373,226.6091239)(131.08546382,226.63912387)(130.98546798,226.64913362)
\curveto(130.88546402,226.66912384)(130.78046412,226.67912383)(130.67046798,226.67913362)
\curveto(130.65046425,226.66912384)(130.62546428,226.66412384)(130.59546798,226.66413362)
\curveto(130.57546433,226.67412383)(130.55546435,226.67412383)(130.53546798,226.66413362)
\curveto(130.48546442,226.65412385)(130.44046446,226.64412386)(130.40046798,226.63413362)
\curveto(130.36046454,226.63412387)(130.32046458,226.62412388)(130.28046798,226.60413362)
\curveto(130.1004648,226.52412398)(129.95046495,226.4041241)(129.83046798,226.24413362)
\curveto(129.72046518,226.08412442)(129.63046527,225.9041246)(129.56046798,225.70413362)
\curveto(129.5004654,225.51412499)(129.45546545,225.28912522)(129.42546798,225.02913362)
\curveto(129.4054655,224.76912574)(129.4004655,224.504126)(129.41046798,224.23413362)
\curveto(129.42046548,223.97412653)(129.45046545,223.72412678)(129.50046798,223.48413362)
\curveto(129.56046534,223.25412725)(129.63546527,223.06412744)(129.72546798,222.91413362)
\moveto(140.52546798,219.92913362)
\curveto(140.53545437,219.87913063)(140.54045436,219.78913072)(140.54046798,219.65913362)
\curveto(140.54045436,219.52913098)(140.53045437,219.43913107)(140.51046798,219.38913362)
\curveto(140.49045441,219.33913117)(140.48545442,219.28413122)(140.49546798,219.22413362)
\curveto(140.5054544,219.17413133)(140.5054544,219.12413138)(140.49546798,219.07413362)
\curveto(140.45545445,218.93413157)(140.42545448,218.79913171)(140.40546798,218.66913362)
\curveto(140.39545451,218.53913197)(140.36545454,218.41913209)(140.31546798,218.30913362)
\curveto(140.17545473,217.95913255)(140.01045489,217.66413284)(139.82046798,217.42413362)
\curveto(139.63045527,217.19413331)(139.36045554,217.0091335)(139.01046798,216.86913362)
\curveto(138.93045597,216.83913367)(138.84545606,216.81913369)(138.75546798,216.80913362)
\curveto(138.66545624,216.78913372)(138.58045632,216.76913374)(138.50046798,216.74913362)
\curveto(138.45045645,216.73913377)(138.4004565,216.73413377)(138.35046798,216.73413362)
\curveto(138.3004566,216.73413377)(138.25045665,216.72913378)(138.20046798,216.71913362)
\curveto(138.17045673,216.7091338)(138.12045678,216.7091338)(138.05046798,216.71913362)
\curveto(137.98045692,216.71913379)(137.93045697,216.72413378)(137.90046798,216.73413362)
\curveto(137.84045706,216.75413375)(137.78045712,216.76413374)(137.72046798,216.76413362)
\curveto(137.67045723,216.75413375)(137.62045728,216.75913375)(137.57046798,216.77913362)
\curveto(137.48045742,216.79913371)(137.39045751,216.82413368)(137.30046798,216.85413362)
\curveto(137.22045768,216.87413363)(137.14045776,216.9041336)(137.06046798,216.94413362)
\curveto(136.74045816,217.08413342)(136.49045841,217.27913323)(136.31046798,217.52913362)
\curveto(136.13045877,217.78913272)(135.98045892,218.09413241)(135.86046798,218.44413362)
\curveto(135.84045906,218.52413198)(135.82545908,218.6091319)(135.81546798,218.69913362)
\curveto(135.8054591,218.78913172)(135.79045911,218.87413163)(135.77046798,218.95413362)
\curveto(135.76045914,218.98413152)(135.75545915,219.01413149)(135.75546798,219.04413362)
\lineto(135.75546798,219.14913362)
\curveto(135.73545917,219.22913128)(135.72545918,219.3091312)(135.72546798,219.38913362)
\lineto(135.72546798,219.52413362)
\curveto(135.7054592,219.62413088)(135.7054592,219.72413078)(135.72546798,219.82413362)
\lineto(135.72546798,220.00413362)
\curveto(135.73545917,220.05413045)(135.74045916,220.09913041)(135.74046798,220.13913362)
\curveto(135.74045916,220.18913032)(135.74545916,220.23413027)(135.75546798,220.27413362)
\curveto(135.76545914,220.31413019)(135.77045913,220.34913016)(135.77046798,220.37913362)
\curveto(135.77045913,220.41913009)(135.77545913,220.45913005)(135.78546798,220.49913362)
\lineto(135.84546798,220.82913362)
\curveto(135.86545904,220.94912956)(135.89545901,221.05912945)(135.93546798,221.15913362)
\curveto(136.07545883,221.48912902)(136.23545867,221.76412874)(136.41546798,221.98413362)
\curveto(136.6054583,222.21412829)(136.86545804,222.39912811)(137.19546798,222.53913362)
\curveto(137.27545763,222.57912793)(137.36045754,222.6041279)(137.45046798,222.61413362)
\lineto(137.75046798,222.67413362)
\lineto(137.88546798,222.67413362)
\curveto(137.93545697,222.68412782)(137.98545692,222.68912782)(138.03546798,222.68913362)
\curveto(138.6054563,222.7091278)(139.06545584,222.6041279)(139.41546798,222.37413362)
\curveto(139.77545513,222.15412835)(140.04045486,221.85412865)(140.21046798,221.47413362)
\curveto(140.26045464,221.37412913)(140.3004546,221.27412923)(140.33046798,221.17413362)
\curveto(140.36045454,221.07412943)(140.39045451,220.96912954)(140.42046798,220.85913362)
\curveto(140.43045447,220.81912969)(140.43545447,220.78412972)(140.43546798,220.75413362)
\curveto(140.43545447,220.73412977)(140.44045446,220.7041298)(140.45046798,220.66413362)
\curveto(140.47045443,220.59412991)(140.48045442,220.51912999)(140.48046798,220.43913362)
\curveto(140.48045442,220.35913015)(140.49045441,220.27913023)(140.51046798,220.19913362)
\curveto(140.51045439,220.14913036)(140.51045439,220.1041304)(140.51046798,220.06413362)
\curveto(140.51045439,220.02413048)(140.51545439,219.97913053)(140.52546798,219.92913362)
\moveto(139.41546798,219.49413362)
\curveto(139.42545548,219.54413096)(139.43045547,219.61913089)(139.43046798,219.71913362)
\curveto(139.44045546,219.81913069)(139.43545547,219.89413061)(139.41546798,219.94413362)
\curveto(139.39545551,220.0041305)(139.39045551,220.05913045)(139.40046798,220.10913362)
\curveto(139.42045548,220.16913034)(139.42045548,220.22913028)(139.40046798,220.28913362)
\curveto(139.39045551,220.31913019)(139.38545552,220.35413015)(139.38546798,220.39413362)
\curveto(139.38545552,220.43413007)(139.38045552,220.47413003)(139.37046798,220.51413362)
\curveto(139.35045555,220.59412991)(139.33045557,220.66912984)(139.31046798,220.73913362)
\curveto(139.3004556,220.81912969)(139.28545562,220.89912961)(139.26546798,220.97913362)
\curveto(139.23545567,221.03912947)(139.21045569,221.09912941)(139.19046798,221.15913362)
\curveto(139.17045573,221.21912929)(139.14045576,221.27912923)(139.10046798,221.33913362)
\curveto(139.0004559,221.509129)(138.87045603,221.64412886)(138.71046798,221.74413362)
\curveto(138.63045627,221.79412871)(138.53545637,221.82912868)(138.42546798,221.84913362)
\curveto(138.31545659,221.86912864)(138.19045671,221.87912863)(138.05046798,221.87913362)
\curveto(138.03045687,221.86912864)(138.0054569,221.86412864)(137.97546798,221.86413362)
\curveto(137.94545696,221.87412863)(137.91545699,221.87412863)(137.88546798,221.86413362)
\lineto(137.73546798,221.80413362)
\curveto(137.68545722,221.79412871)(137.64045726,221.77912873)(137.60046798,221.75913362)
\curveto(137.41045749,221.64912886)(137.26545764,221.504129)(137.16546798,221.32413362)
\curveto(137.07545783,221.14412936)(136.99545791,220.93912957)(136.92546798,220.70913362)
\curveto(136.88545802,220.57912993)(136.86545804,220.44413006)(136.86546798,220.30413362)
\curveto(136.86545804,220.17413033)(136.85545805,220.02913048)(136.83546798,219.86913362)
\curveto(136.82545808,219.81913069)(136.81545809,219.75913075)(136.80546798,219.68913362)
\curveto(136.8054581,219.61913089)(136.81545809,219.55913095)(136.83546798,219.50913362)
\lineto(136.83546798,219.34413362)
\lineto(136.83546798,219.16413362)
\curveto(136.84545806,219.11413139)(136.85545805,219.05913145)(136.86546798,218.99913362)
\curveto(136.87545803,218.94913156)(136.88045802,218.89413161)(136.88046798,218.83413362)
\curveto(136.89045801,218.77413173)(136.905458,218.71913179)(136.92546798,218.66913362)
\curveto(136.97545793,218.47913203)(137.03545787,218.3041322)(137.10546798,218.14413362)
\curveto(137.17545773,217.98413252)(137.28045762,217.85413265)(137.42046798,217.75413362)
\curveto(137.55045735,217.65413285)(137.69045721,217.58413292)(137.84046798,217.54413362)
\curveto(137.87045703,217.53413297)(137.89545701,217.52913298)(137.91546798,217.52913362)
\curveto(137.94545696,217.53913297)(137.97545693,217.53913297)(138.00546798,217.52913362)
\curveto(138.02545688,217.52913298)(138.05545685,217.52413298)(138.09546798,217.51413362)
\curveto(138.13545677,217.51413299)(138.17045673,217.51913299)(138.20046798,217.52913362)
\curveto(138.24045666,217.53913297)(138.28045662,217.54413296)(138.32046798,217.54413362)
\curveto(138.36045654,217.54413296)(138.4004565,217.55413295)(138.44046798,217.57413362)
\curveto(138.68045622,217.65413285)(138.87545603,217.78913272)(139.02546798,217.97913362)
\curveto(139.14545576,218.15913235)(139.23545567,218.36413214)(139.29546798,218.59413362)
\curveto(139.31545559,218.66413184)(139.33045557,218.73413177)(139.34046798,218.80413362)
\curveto(139.35045555,218.88413162)(139.36545554,218.96413154)(139.38546798,219.04413362)
\curveto(139.38545552,219.1041314)(139.39045551,219.14913136)(139.40046798,219.17913362)
\curveto(139.4004555,219.19913131)(139.4004555,219.22413128)(139.40046798,219.25413362)
\curveto(139.4004555,219.29413121)(139.4054555,219.32413118)(139.41546798,219.34413362)
\lineto(139.41546798,219.49413362)
}
}
{
\newrgbcolor{curcolor}{0 0 0}
\pscustom[linestyle=none,fillstyle=solid,fillcolor=curcolor]
{
\newpath
\moveto(455.80941757,179.48297029)
\curveto(455.87940992,179.43296683)(455.91940988,179.3629669)(455.92941757,179.27297029)
\curveto(455.94940985,179.18296708)(455.95940984,179.07796719)(455.95941757,178.95797029)
\curveto(455.95940984,178.90796736)(455.95440985,178.85796741)(455.94441757,178.80797029)
\curveto(455.94440986,178.75796751)(455.93440987,178.71296755)(455.91441757,178.67297029)
\curveto(455.88440992,178.58296768)(455.82440998,178.52296774)(455.73441757,178.49297029)
\curveto(455.65441015,178.47296779)(455.55941024,178.4629678)(455.44941757,178.46297029)
\lineto(455.13441757,178.46297029)
\curveto(455.02441078,178.47296779)(454.91941088,178.4629678)(454.81941757,178.43297029)
\curveto(454.67941112,178.40296786)(454.58941121,178.32296794)(454.54941757,178.19297029)
\curveto(454.52941127,178.12296814)(454.51941128,178.03796823)(454.51941757,177.93797029)
\lineto(454.51941757,177.66797029)
\lineto(454.51941757,176.72297029)
\lineto(454.51941757,176.39297029)
\curveto(454.51941128,176.28296998)(454.4994113,176.19797007)(454.45941757,176.13797029)
\curveto(454.41941138,176.07797019)(454.36941143,176.03797023)(454.30941757,176.01797029)
\curveto(454.25941154,176.00797026)(454.19441161,175.99297027)(454.11441757,175.97297029)
\lineto(453.91941757,175.97297029)
\curveto(453.799412,175.97297029)(453.69441211,175.97797029)(453.60441757,175.98797029)
\curveto(453.51441229,176.00797026)(453.44441236,176.05797021)(453.39441757,176.13797029)
\curveto(453.36441244,176.18797008)(453.34941245,176.25797001)(453.34941757,176.34797029)
\lineto(453.34941757,176.64797029)
\lineto(453.34941757,177.68297029)
\curveto(453.34941245,177.84296842)(453.33941246,177.98796828)(453.31941757,178.11797029)
\curveto(453.30941249,178.25796801)(453.25441255,178.35296791)(453.15441757,178.40297029)
\curveto(453.1044127,178.42296784)(453.03441277,178.43796783)(452.94441757,178.44797029)
\curveto(452.86441294,178.45796781)(452.77441303,178.4629678)(452.67441757,178.46297029)
\lineto(452.38941757,178.46297029)
\lineto(452.14941757,178.46297029)
\lineto(449.88441757,178.46297029)
\curveto(449.79441601,178.4629678)(449.68941611,178.45796781)(449.56941757,178.44797029)
\lineto(449.23941757,178.44797029)
\curveto(449.12941667,178.44796782)(449.02941677,178.45796781)(448.93941757,178.47797029)
\curveto(448.84941695,178.49796777)(448.78941701,178.53296773)(448.75941757,178.58297029)
\curveto(448.70941709,178.65296761)(448.68441712,178.74796752)(448.68441757,178.86797029)
\lineto(448.68441757,179.21297029)
\lineto(448.68441757,179.48297029)
\curveto(448.72441708,179.65296661)(448.77941702,179.79296647)(448.84941757,179.90297029)
\curveto(448.91941688,180.01296625)(448.9994168,180.12796614)(449.08941757,180.24797029)
\lineto(449.44941757,180.78797029)
\curveto(449.88941591,181.41796485)(450.32441548,182.03796423)(450.75441757,182.64797029)
\lineto(452.07441757,184.50797029)
\curveto(452.23441357,184.73796153)(452.38941341,184.95796131)(452.53941757,185.16797029)
\curveto(452.68941311,185.38796088)(452.84441296,185.61296065)(453.00441757,185.84297029)
\curveto(453.05441275,185.91296035)(453.1044127,185.97796029)(453.15441757,186.03797029)
\curveto(453.2044126,186.10796016)(453.25441255,186.18296008)(453.30441757,186.26297029)
\lineto(453.36441757,186.35297029)
\curveto(453.39441241,186.39295987)(453.42441238,186.42295984)(453.45441757,186.44297029)
\curveto(453.49441231,186.47295979)(453.53441227,186.49295977)(453.57441757,186.50297029)
\curveto(453.61441219,186.52295974)(453.65941214,186.54295972)(453.70941757,186.56297029)
\curveto(453.72941207,186.5629597)(453.74941205,186.55795971)(453.76941757,186.54797029)
\curveto(453.799412,186.54795972)(453.82441198,186.55795971)(453.84441757,186.57797029)
\curveto(453.97441183,186.57795969)(454.09441171,186.57295969)(454.20441757,186.56297029)
\curveto(454.31441149,186.55295971)(454.39441141,186.50795976)(454.44441757,186.42797029)
\curveto(454.48441132,186.37795989)(454.5044113,186.30795996)(454.50441757,186.21797029)
\curveto(454.51441129,186.12796014)(454.51941128,186.03296023)(454.51941757,185.93297029)
\lineto(454.51941757,180.47297029)
\curveto(454.51941128,180.40296586)(454.51441129,180.32796594)(454.50441757,180.24797029)
\curveto(454.5044113,180.17796609)(454.50941129,180.10796616)(454.51941757,180.03797029)
\lineto(454.51941757,179.93297029)
\curveto(454.53941126,179.88296638)(454.55441125,179.82796644)(454.56441757,179.76797029)
\curveto(454.57441123,179.71796655)(454.5994112,179.67796659)(454.63941757,179.64797029)
\curveto(454.70941109,179.59796667)(454.79441101,179.5679667)(454.89441757,179.55797029)
\lineto(455.22441757,179.55797029)
\curveto(455.33441047,179.55796671)(455.43941036,179.55296671)(455.53941757,179.54297029)
\curveto(455.64941015,179.54296672)(455.73941006,179.52296674)(455.80941757,179.48297029)
\moveto(453.24441757,179.67797029)
\curveto(453.32441248,179.78796648)(453.35941244,179.95796631)(453.34941757,180.18797029)
\lineto(453.34941757,180.80297029)
\lineto(453.34941757,183.27797029)
\lineto(453.34941757,183.59297029)
\curveto(453.35941244,183.71296255)(453.35441245,183.81296245)(453.33441757,183.89297029)
\lineto(453.33441757,184.04297029)
\curveto(453.33441247,184.13296213)(453.31941248,184.21796205)(453.28941757,184.29797029)
\curveto(453.27941252,184.31796195)(453.26941253,184.32796194)(453.25941757,184.32797029)
\lineto(453.21441757,184.37297029)
\curveto(453.19441261,184.38296188)(453.16441264,184.38796188)(453.12441757,184.38797029)
\curveto(453.1044127,184.3679619)(453.08441272,184.35296191)(453.06441757,184.34297029)
\curveto(453.05441275,184.34296192)(453.03941276,184.33796193)(453.01941757,184.32797029)
\curveto(452.95941284,184.27796199)(452.8994129,184.20796206)(452.83941757,184.11797029)
\curveto(452.77941302,184.02796224)(452.72441308,183.94796232)(452.67441757,183.87797029)
\curveto(452.57441323,183.73796253)(452.47941332,183.59296267)(452.38941757,183.44297029)
\curveto(452.2994135,183.30296296)(452.2044136,183.1629631)(452.10441757,183.02297029)
\lineto(451.56441757,182.24297029)
\curveto(451.39441441,181.98296428)(451.21941458,181.72296454)(451.03941757,181.46297029)
\curveto(450.95941484,181.35296491)(450.88441492,181.24796502)(450.81441757,181.14797029)
\lineto(450.60441757,180.84797029)
\curveto(450.55441525,180.7679655)(450.5044153,180.69296557)(450.45441757,180.62297029)
\curveto(450.41441539,180.55296571)(450.36941543,180.47796579)(450.31941757,180.39797029)
\curveto(450.26941553,180.33796593)(450.21941558,180.27296599)(450.16941757,180.20297029)
\curveto(450.12941567,180.14296612)(450.08941571,180.07296619)(450.04941757,179.99297029)
\curveto(450.00941579,179.93296633)(449.98441582,179.8629664)(449.97441757,179.78297029)
\curveto(449.96441584,179.71296655)(449.9994158,179.65796661)(450.07941757,179.61797029)
\curveto(450.14941565,179.5679667)(450.25941554,179.54296672)(450.40941757,179.54297029)
\curveto(450.56941523,179.55296671)(450.7044151,179.55796671)(450.81441757,179.55797029)
\lineto(452.49441757,179.55797029)
\lineto(452.92941757,179.55797029)
\curveto(453.07941272,179.55796671)(453.18441262,179.59796667)(453.24441757,179.67797029)
}
}
{
\newrgbcolor{curcolor}{0 0 0}
\pscustom[linestyle=none,fillstyle=solid,fillcolor=curcolor]
{
\newpath
\moveto(458.78902694,186.39797029)
\lineto(462.38902694,186.39797029)
\lineto(463.03402694,186.39797029)
\curveto(463.11402041,186.39795987)(463.18902034,186.39295987)(463.25902694,186.38297029)
\curveto(463.3290202,186.38295988)(463.38902014,186.37295989)(463.43902694,186.35297029)
\curveto(463.50902002,186.32295994)(463.56401996,186.26296)(463.60402694,186.17297029)
\curveto(463.6240199,186.14296012)(463.63401989,186.10296016)(463.63402694,186.05297029)
\lineto(463.63402694,185.91797029)
\curveto(463.64401988,185.80796046)(463.63901989,185.70296056)(463.61902694,185.60297029)
\curveto(463.60901992,185.50296076)(463.57401995,185.43296083)(463.51402694,185.39297029)
\curveto(463.4240201,185.32296094)(463.28902024,185.28796098)(463.10902694,185.28797029)
\curveto(462.9290206,185.29796097)(462.76402076,185.30296096)(462.61402694,185.30297029)
\lineto(460.61902694,185.30297029)
\lineto(460.12402694,185.30297029)
\lineto(459.98902694,185.30297029)
\curveto(459.94902358,185.30296096)(459.90902362,185.29796097)(459.86902694,185.28797029)
\lineto(459.65902694,185.28797029)
\curveto(459.54902398,185.25796101)(459.46902406,185.21796105)(459.41902694,185.16797029)
\curveto(459.36902416,185.12796114)(459.33402419,185.07296119)(459.31402694,185.00297029)
\curveto(459.29402423,184.94296132)(459.27902425,184.87296139)(459.26902694,184.79297029)
\curveto(459.25902427,184.71296155)(459.23902429,184.62296164)(459.20902694,184.52297029)
\curveto(459.15902437,184.32296194)(459.11902441,184.11796215)(459.08902694,183.90797029)
\curveto(459.05902447,183.69796257)(459.01902451,183.49296277)(458.96902694,183.29297029)
\curveto(458.94902458,183.22296304)(458.93902459,183.15296311)(458.93902694,183.08297029)
\curveto(458.93902459,183.02296324)(458.9290246,182.95796331)(458.90902694,182.88797029)
\curveto(458.89902463,182.85796341)(458.88902464,182.81796345)(458.87902694,182.76797029)
\curveto(458.87902465,182.72796354)(458.88402464,182.68796358)(458.89402694,182.64797029)
\curveto(458.91402461,182.59796367)(458.93902459,182.55296371)(458.96902694,182.51297029)
\curveto(459.00902452,182.48296378)(459.06902446,182.47796379)(459.14902694,182.49797029)
\curveto(459.20902432,182.51796375)(459.26902426,182.54296372)(459.32902694,182.57297029)
\curveto(459.38902414,182.61296365)(459.44902408,182.64796362)(459.50902694,182.67797029)
\curveto(459.56902396,182.69796357)(459.61902391,182.71296355)(459.65902694,182.72297029)
\curveto(459.84902368,182.80296346)(460.05402347,182.85796341)(460.27402694,182.88797029)
\curveto(460.50402302,182.91796335)(460.73402279,182.92796334)(460.96402694,182.91797029)
\curveto(461.20402232,182.91796335)(461.43402209,182.89296337)(461.65402694,182.84297029)
\curveto(461.87402165,182.80296346)(462.07402145,182.74296352)(462.25402694,182.66297029)
\curveto(462.30402122,182.64296362)(462.34902118,182.62296364)(462.38902694,182.60297029)
\curveto(462.43902109,182.58296368)(462.48902104,182.55796371)(462.53902694,182.52797029)
\curveto(462.88902064,182.31796395)(463.16902036,182.08796418)(463.37902694,181.83797029)
\curveto(463.59901993,181.58796468)(463.79401973,181.262965)(463.96402694,180.86297029)
\curveto(464.01401951,180.75296551)(464.04901948,180.64296562)(464.06902694,180.53297029)
\curveto(464.08901944,180.42296584)(464.11401941,180.30796596)(464.14402694,180.18797029)
\curveto(464.15401937,180.15796611)(464.15901937,180.11296615)(464.15902694,180.05297029)
\curveto(464.17901935,179.99296627)(464.18901934,179.92296634)(464.18902694,179.84297029)
\curveto(464.18901934,179.77296649)(464.19901933,179.70796656)(464.21902694,179.64797029)
\lineto(464.21902694,179.48297029)
\curveto(464.2290193,179.43296683)(464.23401929,179.3629669)(464.23402694,179.27297029)
\curveto(464.23401929,179.18296708)(464.2240193,179.11296715)(464.20402694,179.06297029)
\curveto(464.18401934,179.00296726)(464.17901935,178.94296732)(464.18902694,178.88297029)
\curveto(464.19901933,178.83296743)(464.19401933,178.78296748)(464.17402694,178.73297029)
\curveto(464.13401939,178.57296769)(464.09901943,178.42296784)(464.06902694,178.28297029)
\curveto(464.03901949,178.14296812)(463.99401953,178.00796826)(463.93402694,177.87797029)
\curveto(463.77401975,177.50796876)(463.55401997,177.17296909)(463.27402694,176.87297029)
\curveto(462.99402053,176.57296969)(462.67402085,176.34296992)(462.31402694,176.18297029)
\curveto(462.14402138,176.10297016)(461.94402158,176.02797024)(461.71402694,175.95797029)
\curveto(461.60402192,175.91797035)(461.48902204,175.89297037)(461.36902694,175.88297029)
\curveto(461.24902228,175.87297039)(461.1290224,175.85297041)(461.00902694,175.82297029)
\curveto(460.95902257,175.80297046)(460.90402262,175.80297046)(460.84402694,175.82297029)
\curveto(460.78402274,175.83297043)(460.7240228,175.82797044)(460.66402694,175.80797029)
\curveto(460.56402296,175.78797048)(460.46402306,175.78797048)(460.36402694,175.80797029)
\lineto(460.22902694,175.80797029)
\curveto(460.17902335,175.82797044)(460.11902341,175.83797043)(460.04902694,175.83797029)
\curveto(459.98902354,175.82797044)(459.93402359,175.83297043)(459.88402694,175.85297029)
\curveto(459.84402368,175.8629704)(459.80902372,175.8679704)(459.77902694,175.86797029)
\curveto(459.74902378,175.8679704)(459.71402381,175.87297039)(459.67402694,175.88297029)
\lineto(459.40402694,175.94297029)
\curveto(459.31402421,175.9629703)(459.2290243,175.99297027)(459.14902694,176.03297029)
\curveto(458.80902472,176.17297009)(458.51902501,176.32796994)(458.27902694,176.49797029)
\curveto(458.03902549,176.67796959)(457.81902571,176.90796936)(457.61902694,177.18797029)
\curveto(457.46902606,177.41796885)(457.35402617,177.65796861)(457.27402694,177.90797029)
\curveto(457.25402627,177.95796831)(457.24402628,178.00296826)(457.24402694,178.04297029)
\curveto(457.24402628,178.09296817)(457.23402629,178.14296812)(457.21402694,178.19297029)
\curveto(457.19402633,178.25296801)(457.17902635,178.33296793)(457.16902694,178.43297029)
\curveto(457.16902636,178.53296773)(457.18902634,178.60796766)(457.22902694,178.65797029)
\curveto(457.27902625,178.73796753)(457.35902617,178.78296748)(457.46902694,178.79297029)
\curveto(457.57902595,178.80296746)(457.69402583,178.80796746)(457.81402694,178.80797029)
\lineto(457.97902694,178.80797029)
\curveto(458.03902549,178.80796746)(458.09402543,178.79796747)(458.14402694,178.77797029)
\curveto(458.23402529,178.75796751)(458.30402522,178.71796755)(458.35402694,178.65797029)
\curveto(458.4240251,178.5679677)(458.46902506,178.45796781)(458.48902694,178.32797029)
\curveto(458.51902501,178.20796806)(458.56402496,178.10296816)(458.62402694,178.01297029)
\curveto(458.81402471,177.67296859)(459.07402445,177.40296886)(459.40402694,177.20297029)
\curveto(459.50402402,177.14296912)(459.60902392,177.09296917)(459.71902694,177.05297029)
\curveto(459.83902369,177.02296924)(459.95902357,176.98796928)(460.07902694,176.94797029)
\curveto(460.24902328,176.89796937)(460.45402307,176.87796939)(460.69402694,176.88797029)
\curveto(460.94402258,176.90796936)(461.14402238,176.94296932)(461.29402694,176.99297029)
\curveto(461.66402186,177.11296915)(461.95402157,177.27296899)(462.16402694,177.47297029)
\curveto(462.38402114,177.68296858)(462.56402096,177.9629683)(462.70402694,178.31297029)
\curveto(462.75402077,178.41296785)(462.78402074,178.51796775)(462.79402694,178.62797029)
\curveto(462.81402071,178.73796753)(462.83902069,178.85296741)(462.86902694,178.97297029)
\lineto(462.86902694,179.07797029)
\curveto(462.87902065,179.11796715)(462.88402064,179.15796711)(462.88402694,179.19797029)
\curveto(462.89402063,179.22796704)(462.89402063,179.262967)(462.88402694,179.30297029)
\lineto(462.88402694,179.42297029)
\curveto(462.88402064,179.68296658)(462.85402067,179.92796634)(462.79402694,180.15797029)
\curveto(462.68402084,180.50796576)(462.529021,180.80296546)(462.32902694,181.04297029)
\curveto(462.1290214,181.29296497)(461.86902166,181.48796478)(461.54902694,181.62797029)
\lineto(461.36902694,181.68797029)
\curveto(461.31902221,181.70796456)(461.25902227,181.72796454)(461.18902694,181.74797029)
\curveto(461.13902239,181.7679645)(461.07902245,181.77796449)(461.00902694,181.77797029)
\curveto(460.94902258,181.78796448)(460.88402264,181.80296446)(460.81402694,181.82297029)
\lineto(460.66402694,181.82297029)
\curveto(460.6240229,181.84296442)(460.56902296,181.85296441)(460.49902694,181.85297029)
\curveto(460.43902309,181.85296441)(460.38402314,181.84296442)(460.33402694,181.82297029)
\lineto(460.22902694,181.82297029)
\curveto(460.19902333,181.82296444)(460.16402336,181.81796445)(460.12402694,181.80797029)
\lineto(459.88402694,181.74797029)
\curveto(459.80402372,181.73796453)(459.7240238,181.71796455)(459.64402694,181.68797029)
\curveto(459.40402412,181.58796468)(459.17402435,181.45296481)(458.95402694,181.28297029)
\curveto(458.86402466,181.21296505)(458.77902475,181.13796513)(458.69902694,181.05797029)
\curveto(458.61902491,180.98796528)(458.51902501,180.93296533)(458.39902694,180.89297029)
\curveto(458.30902522,180.8629654)(458.16902536,180.85296541)(457.97902694,180.86297029)
\curveto(457.79902573,180.87296539)(457.67902585,180.89796537)(457.61902694,180.93797029)
\curveto(457.56902596,180.97796529)(457.529026,181.03796523)(457.49902694,181.11797029)
\curveto(457.47902605,181.19796507)(457.47902605,181.28296498)(457.49902694,181.37297029)
\curveto(457.529026,181.49296477)(457.54902598,181.61296465)(457.55902694,181.73297029)
\curveto(457.57902595,181.8629644)(457.60402592,181.98796428)(457.63402694,182.10797029)
\curveto(457.65402587,182.14796412)(457.65902587,182.18296408)(457.64902694,182.21297029)
\curveto(457.64902588,182.25296401)(457.65902587,182.29796397)(457.67902694,182.34797029)
\curveto(457.69902583,182.43796383)(457.71402581,182.52796374)(457.72402694,182.61797029)
\curveto(457.73402579,182.71796355)(457.75402577,182.81296345)(457.78402694,182.90297029)
\curveto(457.79402573,182.9629633)(457.79902573,183.02296324)(457.79902694,183.08297029)
\curveto(457.80902572,183.14296312)(457.8240257,183.20296306)(457.84402694,183.26297029)
\curveto(457.89402563,183.4629628)(457.9290256,183.6679626)(457.94902694,183.87797029)
\curveto(457.97902555,184.09796217)(458.01902551,184.30796196)(458.06902694,184.50797029)
\curveto(458.09902543,184.60796166)(458.11902541,184.70796156)(458.12902694,184.80797029)
\curveto(458.13902539,184.90796136)(458.15402537,185.00796126)(458.17402694,185.10797029)
\curveto(458.18402534,185.13796113)(458.18902534,185.17796109)(458.18902694,185.22797029)
\curveto(458.21902531,185.33796093)(458.23902529,185.44296082)(458.24902694,185.54297029)
\curveto(458.26902526,185.65296061)(458.29402523,185.7629605)(458.32402694,185.87297029)
\curveto(458.34402518,185.95296031)(458.35902517,186.02296024)(458.36902694,186.08297029)
\curveto(458.37902515,186.15296011)(458.40402512,186.21296005)(458.44402694,186.26297029)
\curveto(458.46402506,186.29295997)(458.49402503,186.31295995)(458.53402694,186.32297029)
\curveto(458.57402495,186.34295992)(458.61902491,186.3629599)(458.66902694,186.38297029)
\curveto(458.7290248,186.38295988)(458.76902476,186.38795988)(458.78902694,186.39797029)
}
}
{
\newrgbcolor{curcolor}{0 0 0}
\pscustom[linestyle=none,fillstyle=solid,fillcolor=curcolor]
{
\newpath
\moveto(466.58363632,177.62297029)
\lineto(466.88363632,177.62297029)
\curveto(466.99363426,177.63296863)(467.09863415,177.63296863)(467.19863632,177.62297029)
\curveto(467.30863394,177.62296864)(467.40863384,177.61296865)(467.49863632,177.59297029)
\curveto(467.58863366,177.58296868)(467.65863359,177.55796871)(467.70863632,177.51797029)
\curveto(467.72863352,177.49796877)(467.74363351,177.4679688)(467.75363632,177.42797029)
\curveto(467.77363348,177.38796888)(467.79363346,177.34296892)(467.81363632,177.29297029)
\lineto(467.81363632,177.21797029)
\curveto(467.82363343,177.1679691)(467.82363343,177.11296915)(467.81363632,177.05297029)
\lineto(467.81363632,176.90297029)
\lineto(467.81363632,176.42297029)
\curveto(467.81363344,176.25297001)(467.77363348,176.13297013)(467.69363632,176.06297029)
\curveto(467.62363363,176.01297025)(467.53363372,175.98797028)(467.42363632,175.98797029)
\lineto(467.09363632,175.98797029)
\lineto(466.64363632,175.98797029)
\curveto(466.49363476,175.98797028)(466.37863487,176.01797025)(466.29863632,176.07797029)
\curveto(466.25863499,176.10797016)(466.22863502,176.15797011)(466.20863632,176.22797029)
\curveto(466.18863506,176.30796996)(466.17363508,176.39296987)(466.16363632,176.48297029)
\lineto(466.16363632,176.76797029)
\curveto(466.17363508,176.8679694)(466.17863507,176.95296931)(466.17863632,177.02297029)
\lineto(466.17863632,177.21797029)
\curveto(466.17863507,177.27796899)(466.18863506,177.33296893)(466.20863632,177.38297029)
\curveto(466.248635,177.49296877)(466.31863493,177.5629687)(466.41863632,177.59297029)
\curveto(466.4486348,177.59296867)(466.50363475,177.60296866)(466.58363632,177.62297029)
}
}
{
\newrgbcolor{curcolor}{0 0 0}
\pscustom[linestyle=none,fillstyle=solid,fillcolor=curcolor]
{
\newpath
\moveto(471.29879257,186.39797029)
\lineto(474.89879257,186.39797029)
\lineto(475.54379257,186.39797029)
\curveto(475.62378604,186.39795987)(475.69878596,186.39295987)(475.76879257,186.38297029)
\curveto(475.83878582,186.38295988)(475.89878576,186.37295989)(475.94879257,186.35297029)
\curveto(476.01878564,186.32295994)(476.07378559,186.26296)(476.11379257,186.17297029)
\curveto(476.13378553,186.14296012)(476.14378552,186.10296016)(476.14379257,186.05297029)
\lineto(476.14379257,185.91797029)
\curveto(476.15378551,185.80796046)(476.14878551,185.70296056)(476.12879257,185.60297029)
\curveto(476.11878554,185.50296076)(476.08378558,185.43296083)(476.02379257,185.39297029)
\curveto(475.93378573,185.32296094)(475.79878586,185.28796098)(475.61879257,185.28797029)
\curveto(475.43878622,185.29796097)(475.27378639,185.30296096)(475.12379257,185.30297029)
\lineto(473.12879257,185.30297029)
\lineto(472.63379257,185.30297029)
\lineto(472.49879257,185.30297029)
\curveto(472.4587892,185.30296096)(472.41878924,185.29796097)(472.37879257,185.28797029)
\lineto(472.16879257,185.28797029)
\curveto(472.0587896,185.25796101)(471.97878968,185.21796105)(471.92879257,185.16797029)
\curveto(471.87878978,185.12796114)(471.84378982,185.07296119)(471.82379257,185.00297029)
\curveto(471.80378986,184.94296132)(471.78878987,184.87296139)(471.77879257,184.79297029)
\curveto(471.76878989,184.71296155)(471.74878991,184.62296164)(471.71879257,184.52297029)
\curveto(471.66878999,184.32296194)(471.62879003,184.11796215)(471.59879257,183.90797029)
\curveto(471.56879009,183.69796257)(471.52879013,183.49296277)(471.47879257,183.29297029)
\curveto(471.4587902,183.22296304)(471.44879021,183.15296311)(471.44879257,183.08297029)
\curveto(471.44879021,183.02296324)(471.43879022,182.95796331)(471.41879257,182.88797029)
\curveto(471.40879025,182.85796341)(471.39879026,182.81796345)(471.38879257,182.76797029)
\curveto(471.38879027,182.72796354)(471.39379027,182.68796358)(471.40379257,182.64797029)
\curveto(471.42379024,182.59796367)(471.44879021,182.55296371)(471.47879257,182.51297029)
\curveto(471.51879014,182.48296378)(471.57879008,182.47796379)(471.65879257,182.49797029)
\curveto(471.71878994,182.51796375)(471.77878988,182.54296372)(471.83879257,182.57297029)
\curveto(471.89878976,182.61296365)(471.9587897,182.64796362)(472.01879257,182.67797029)
\curveto(472.07878958,182.69796357)(472.12878953,182.71296355)(472.16879257,182.72297029)
\curveto(472.3587893,182.80296346)(472.5637891,182.85796341)(472.78379257,182.88797029)
\curveto(473.01378865,182.91796335)(473.24378842,182.92796334)(473.47379257,182.91797029)
\curveto(473.71378795,182.91796335)(473.94378772,182.89296337)(474.16379257,182.84297029)
\curveto(474.38378728,182.80296346)(474.58378708,182.74296352)(474.76379257,182.66297029)
\curveto(474.81378685,182.64296362)(474.8587868,182.62296364)(474.89879257,182.60297029)
\curveto(474.94878671,182.58296368)(474.99878666,182.55796371)(475.04879257,182.52797029)
\curveto(475.39878626,182.31796395)(475.67878598,182.08796418)(475.88879257,181.83797029)
\curveto(476.10878555,181.58796468)(476.30378536,181.262965)(476.47379257,180.86297029)
\curveto(476.52378514,180.75296551)(476.5587851,180.64296562)(476.57879257,180.53297029)
\curveto(476.59878506,180.42296584)(476.62378504,180.30796596)(476.65379257,180.18797029)
\curveto(476.663785,180.15796611)(476.66878499,180.11296615)(476.66879257,180.05297029)
\curveto(476.68878497,179.99296627)(476.69878496,179.92296634)(476.69879257,179.84297029)
\curveto(476.69878496,179.77296649)(476.70878495,179.70796656)(476.72879257,179.64797029)
\lineto(476.72879257,179.48297029)
\curveto(476.73878492,179.43296683)(476.74378492,179.3629669)(476.74379257,179.27297029)
\curveto(476.74378492,179.18296708)(476.73378493,179.11296715)(476.71379257,179.06297029)
\curveto(476.69378497,179.00296726)(476.68878497,178.94296732)(476.69879257,178.88297029)
\curveto(476.70878495,178.83296743)(476.70378496,178.78296748)(476.68379257,178.73297029)
\curveto(476.64378502,178.57296769)(476.60878505,178.42296784)(476.57879257,178.28297029)
\curveto(476.54878511,178.14296812)(476.50378516,178.00796826)(476.44379257,177.87797029)
\curveto(476.28378538,177.50796876)(476.0637856,177.17296909)(475.78379257,176.87297029)
\curveto(475.50378616,176.57296969)(475.18378648,176.34296992)(474.82379257,176.18297029)
\curveto(474.65378701,176.10297016)(474.45378721,176.02797024)(474.22379257,175.95797029)
\curveto(474.11378755,175.91797035)(473.99878766,175.89297037)(473.87879257,175.88297029)
\curveto(473.7587879,175.87297039)(473.63878802,175.85297041)(473.51879257,175.82297029)
\curveto(473.46878819,175.80297046)(473.41378825,175.80297046)(473.35379257,175.82297029)
\curveto(473.29378837,175.83297043)(473.23378843,175.82797044)(473.17379257,175.80797029)
\curveto(473.07378859,175.78797048)(472.97378869,175.78797048)(472.87379257,175.80797029)
\lineto(472.73879257,175.80797029)
\curveto(472.68878897,175.82797044)(472.62878903,175.83797043)(472.55879257,175.83797029)
\curveto(472.49878916,175.82797044)(472.44378922,175.83297043)(472.39379257,175.85297029)
\curveto(472.35378931,175.8629704)(472.31878934,175.8679704)(472.28879257,175.86797029)
\curveto(472.2587894,175.8679704)(472.22378944,175.87297039)(472.18379257,175.88297029)
\lineto(471.91379257,175.94297029)
\curveto(471.82378984,175.9629703)(471.73878992,175.99297027)(471.65879257,176.03297029)
\curveto(471.31879034,176.17297009)(471.02879063,176.32796994)(470.78879257,176.49797029)
\curveto(470.54879111,176.67796959)(470.32879133,176.90796936)(470.12879257,177.18797029)
\curveto(469.97879168,177.41796885)(469.8637918,177.65796861)(469.78379257,177.90797029)
\curveto(469.7637919,177.95796831)(469.75379191,178.00296826)(469.75379257,178.04297029)
\curveto(469.75379191,178.09296817)(469.74379192,178.14296812)(469.72379257,178.19297029)
\curveto(469.70379196,178.25296801)(469.68879197,178.33296793)(469.67879257,178.43297029)
\curveto(469.67879198,178.53296773)(469.69879196,178.60796766)(469.73879257,178.65797029)
\curveto(469.78879187,178.73796753)(469.86879179,178.78296748)(469.97879257,178.79297029)
\curveto(470.08879157,178.80296746)(470.20379146,178.80796746)(470.32379257,178.80797029)
\lineto(470.48879257,178.80797029)
\curveto(470.54879111,178.80796746)(470.60379106,178.79796747)(470.65379257,178.77797029)
\curveto(470.74379092,178.75796751)(470.81379085,178.71796755)(470.86379257,178.65797029)
\curveto(470.93379073,178.5679677)(470.97879068,178.45796781)(470.99879257,178.32797029)
\curveto(471.02879063,178.20796806)(471.07379059,178.10296816)(471.13379257,178.01297029)
\curveto(471.32379034,177.67296859)(471.58379008,177.40296886)(471.91379257,177.20297029)
\curveto(472.01378965,177.14296912)(472.11878954,177.09296917)(472.22879257,177.05297029)
\curveto(472.34878931,177.02296924)(472.46878919,176.98796928)(472.58879257,176.94797029)
\curveto(472.7587889,176.89796937)(472.9637887,176.87796939)(473.20379257,176.88797029)
\curveto(473.45378821,176.90796936)(473.65378801,176.94296932)(473.80379257,176.99297029)
\curveto(474.17378749,177.11296915)(474.4637872,177.27296899)(474.67379257,177.47297029)
\curveto(474.89378677,177.68296858)(475.07378659,177.9629683)(475.21379257,178.31297029)
\curveto(475.2637864,178.41296785)(475.29378637,178.51796775)(475.30379257,178.62797029)
\curveto(475.32378634,178.73796753)(475.34878631,178.85296741)(475.37879257,178.97297029)
\lineto(475.37879257,179.07797029)
\curveto(475.38878627,179.11796715)(475.39378627,179.15796711)(475.39379257,179.19797029)
\curveto(475.40378626,179.22796704)(475.40378626,179.262967)(475.39379257,179.30297029)
\lineto(475.39379257,179.42297029)
\curveto(475.39378627,179.68296658)(475.3637863,179.92796634)(475.30379257,180.15797029)
\curveto(475.19378647,180.50796576)(475.03878662,180.80296546)(474.83879257,181.04297029)
\curveto(474.63878702,181.29296497)(474.37878728,181.48796478)(474.05879257,181.62797029)
\lineto(473.87879257,181.68797029)
\curveto(473.82878783,181.70796456)(473.76878789,181.72796454)(473.69879257,181.74797029)
\curveto(473.64878801,181.7679645)(473.58878807,181.77796449)(473.51879257,181.77797029)
\curveto(473.4587882,181.78796448)(473.39378827,181.80296446)(473.32379257,181.82297029)
\lineto(473.17379257,181.82297029)
\curveto(473.13378853,181.84296442)(473.07878858,181.85296441)(473.00879257,181.85297029)
\curveto(472.94878871,181.85296441)(472.89378877,181.84296442)(472.84379257,181.82297029)
\lineto(472.73879257,181.82297029)
\curveto(472.70878895,181.82296444)(472.67378899,181.81796445)(472.63379257,181.80797029)
\lineto(472.39379257,181.74797029)
\curveto(472.31378935,181.73796453)(472.23378943,181.71796455)(472.15379257,181.68797029)
\curveto(471.91378975,181.58796468)(471.68378998,181.45296481)(471.46379257,181.28297029)
\curveto(471.37379029,181.21296505)(471.28879037,181.13796513)(471.20879257,181.05797029)
\curveto(471.12879053,180.98796528)(471.02879063,180.93296533)(470.90879257,180.89297029)
\curveto(470.81879084,180.8629654)(470.67879098,180.85296541)(470.48879257,180.86297029)
\curveto(470.30879135,180.87296539)(470.18879147,180.89796537)(470.12879257,180.93797029)
\curveto(470.07879158,180.97796529)(470.03879162,181.03796523)(470.00879257,181.11797029)
\curveto(469.98879167,181.19796507)(469.98879167,181.28296498)(470.00879257,181.37297029)
\curveto(470.03879162,181.49296477)(470.0587916,181.61296465)(470.06879257,181.73297029)
\curveto(470.08879157,181.8629644)(470.11379155,181.98796428)(470.14379257,182.10797029)
\curveto(470.1637915,182.14796412)(470.16879149,182.18296408)(470.15879257,182.21297029)
\curveto(470.1587915,182.25296401)(470.16879149,182.29796397)(470.18879257,182.34797029)
\curveto(470.20879145,182.43796383)(470.22379144,182.52796374)(470.23379257,182.61797029)
\curveto(470.24379142,182.71796355)(470.2637914,182.81296345)(470.29379257,182.90297029)
\curveto(470.30379136,182.9629633)(470.30879135,183.02296324)(470.30879257,183.08297029)
\curveto(470.31879134,183.14296312)(470.33379133,183.20296306)(470.35379257,183.26297029)
\curveto(470.40379126,183.4629628)(470.43879122,183.6679626)(470.45879257,183.87797029)
\curveto(470.48879117,184.09796217)(470.52879113,184.30796196)(470.57879257,184.50797029)
\curveto(470.60879105,184.60796166)(470.62879103,184.70796156)(470.63879257,184.80797029)
\curveto(470.64879101,184.90796136)(470.663791,185.00796126)(470.68379257,185.10797029)
\curveto(470.69379097,185.13796113)(470.69879096,185.17796109)(470.69879257,185.22797029)
\curveto(470.72879093,185.33796093)(470.74879091,185.44296082)(470.75879257,185.54297029)
\curveto(470.77879088,185.65296061)(470.80379086,185.7629605)(470.83379257,185.87297029)
\curveto(470.85379081,185.95296031)(470.86879079,186.02296024)(470.87879257,186.08297029)
\curveto(470.88879077,186.15296011)(470.91379075,186.21296005)(470.95379257,186.26297029)
\curveto(470.97379069,186.29295997)(471.00379066,186.31295995)(471.04379257,186.32297029)
\curveto(471.08379058,186.34295992)(471.12879053,186.3629599)(471.17879257,186.38297029)
\curveto(471.23879042,186.38295988)(471.27879038,186.38795988)(471.29879257,186.39797029)
}
}
{
\newrgbcolor{curcolor}{0 0 0}
\pscustom[linestyle=none,fillstyle=solid,fillcolor=curcolor]
{
\newpath
\moveto(487.94340194,184.50797029)
\curveto(487.74339164,184.21796205)(487.53339185,183.93296233)(487.31340194,183.65297029)
\curveto(487.10339228,183.37296289)(486.89839249,183.08796318)(486.69840194,182.79797029)
\curveto(486.09839329,181.94796432)(485.49339389,181.10796516)(484.88340194,180.27797029)
\curveto(484.27339511,179.45796681)(483.66839572,178.62296764)(483.06840194,177.77297029)
\lineto(482.55840194,177.05297029)
\lineto(482.04840194,176.36297029)
\curveto(481.96839742,176.25297001)(481.8883975,176.13797013)(481.80840194,176.01797029)
\curveto(481.72839766,175.89797037)(481.63339775,175.80297046)(481.52340194,175.73297029)
\curveto(481.4833979,175.71297055)(481.41839797,175.69797057)(481.32840194,175.68797029)
\curveto(481.24839814,175.6679706)(481.15839823,175.65797061)(481.05840194,175.65797029)
\curveto(480.95839843,175.65797061)(480.86339852,175.6629706)(480.77340194,175.67297029)
\curveto(480.69339869,175.68297058)(480.63339875,175.70297056)(480.59340194,175.73297029)
\curveto(480.56339882,175.75297051)(480.53839885,175.78797048)(480.51840194,175.83797029)
\curveto(480.50839888,175.87797039)(480.51339887,175.92297034)(480.53340194,175.97297029)
\curveto(480.57339881,176.05297021)(480.61839877,176.12797014)(480.66840194,176.19797029)
\curveto(480.72839866,176.27796999)(480.7833986,176.35796991)(480.83340194,176.43797029)
\curveto(481.07339831,176.77796949)(481.31839807,177.11296915)(481.56840194,177.44297029)
\curveto(481.81839757,177.77296849)(482.05839733,178.10796816)(482.28840194,178.44797029)
\curveto(482.44839694,178.6679676)(482.60839678,178.88296738)(482.76840194,179.09297029)
\curveto(482.92839646,179.30296696)(483.0883963,179.51796675)(483.24840194,179.73797029)
\curveto(483.60839578,180.25796601)(483.97339541,180.7679655)(484.34340194,181.26797029)
\curveto(484.71339467,181.7679645)(485.0833943,182.27796399)(485.45340194,182.79797029)
\curveto(485.59339379,182.99796327)(485.73339365,183.19296307)(485.87340194,183.38297029)
\curveto(486.02339336,183.57296269)(486.16839322,183.7679625)(486.30840194,183.96797029)
\curveto(486.51839287,184.267962)(486.73339265,184.5679617)(486.95340194,184.86797029)
\lineto(487.61340194,185.76797029)
\lineto(487.79340194,186.03797029)
\lineto(488.00340194,186.30797029)
\lineto(488.12340194,186.48797029)
\curveto(488.17339121,186.54795972)(488.22339116,186.60295966)(488.27340194,186.65297029)
\curveto(488.34339104,186.70295956)(488.41839097,186.73795953)(488.49840194,186.75797029)
\curveto(488.51839087,186.7679595)(488.54339084,186.7679595)(488.57340194,186.75797029)
\curveto(488.61339077,186.75795951)(488.64339074,186.7679595)(488.66340194,186.78797029)
\curveto(488.7833906,186.78795948)(488.91839047,186.78295948)(489.06840194,186.77297029)
\curveto(489.21839017,186.77295949)(489.30839008,186.72795954)(489.33840194,186.63797029)
\curveto(489.35839003,186.60795966)(489.36339002,186.57295969)(489.35340194,186.53297029)
\curveto(489.34339004,186.49295977)(489.32839006,186.4629598)(489.30840194,186.44297029)
\curveto(489.26839012,186.3629599)(489.22839016,186.29295997)(489.18840194,186.23297029)
\curveto(489.14839024,186.17296009)(489.10339028,186.11296015)(489.05340194,186.05297029)
\lineto(488.48340194,185.27297029)
\curveto(488.30339108,185.02296124)(488.12339126,184.7679615)(487.94340194,184.50797029)
\moveto(481.08840194,180.60797029)
\curveto(481.03839835,180.62796564)(480.9883984,180.63296563)(480.93840194,180.62297029)
\curveto(480.8883985,180.61296565)(480.83839855,180.61796565)(480.78840194,180.63797029)
\curveto(480.67839871,180.65796561)(480.57339881,180.67796559)(480.47340194,180.69797029)
\curveto(480.383399,180.72796554)(480.2883991,180.7679655)(480.18840194,180.81797029)
\curveto(479.85839953,180.95796531)(479.60339978,181.15296511)(479.42340194,181.40297029)
\curveto(479.24340014,181.6629646)(479.09840029,181.97296429)(478.98840194,182.33297029)
\curveto(478.95840043,182.41296385)(478.93840045,182.49296377)(478.92840194,182.57297029)
\curveto(478.91840047,182.6629636)(478.90340048,182.74796352)(478.88340194,182.82797029)
\curveto(478.87340051,182.87796339)(478.86840052,182.94296332)(478.86840194,183.02297029)
\curveto(478.85840053,183.05296321)(478.85340053,183.08296318)(478.85340194,183.11297029)
\curveto(478.85340053,183.15296311)(478.84840054,183.18796308)(478.83840194,183.21797029)
\lineto(478.83840194,183.36797029)
\curveto(478.82840056,183.41796285)(478.82340056,183.47796279)(478.82340194,183.54797029)
\curveto(478.82340056,183.62796264)(478.82840056,183.69296257)(478.83840194,183.74297029)
\lineto(478.83840194,183.90797029)
\curveto(478.85840053,183.95796231)(478.86340052,184.00296226)(478.85340194,184.04297029)
\curveto(478.85340053,184.09296217)(478.85840053,184.13796213)(478.86840194,184.17797029)
\curveto(478.87840051,184.21796205)(478.8834005,184.25296201)(478.88340194,184.28297029)
\curveto(478.8834005,184.32296194)(478.8884005,184.3629619)(478.89840194,184.40297029)
\curveto(478.92840046,184.51296175)(478.94840044,184.62296164)(478.95840194,184.73297029)
\curveto(478.97840041,184.85296141)(479.01340037,184.9679613)(479.06340194,185.07797029)
\curveto(479.20340018,185.41796085)(479.36340002,185.69296057)(479.54340194,185.90297029)
\curveto(479.73339965,186.12296014)(480.00339938,186.30295996)(480.35340194,186.44297029)
\curveto(480.43339895,186.47295979)(480.51839887,186.49295977)(480.60840194,186.50297029)
\curveto(480.69839869,186.52295974)(480.79339859,186.54295972)(480.89340194,186.56297029)
\curveto(480.92339846,186.57295969)(480.97839841,186.57295969)(481.05840194,186.56297029)
\curveto(481.13839825,186.5629597)(481.1883982,186.57295969)(481.20840194,186.59297029)
\curveto(481.76839762,186.60295966)(482.21839717,186.49295977)(482.55840194,186.26297029)
\curveto(482.90839648,186.03296023)(483.16839622,185.72796054)(483.33840194,185.34797029)
\curveto(483.37839601,185.25796101)(483.41339597,185.1629611)(483.44340194,185.06297029)
\curveto(483.47339591,184.9629613)(483.49839589,184.8629614)(483.51840194,184.76297029)
\curveto(483.53839585,184.73296153)(483.54339584,184.70296156)(483.53340194,184.67297029)
\curveto(483.53339585,184.64296162)(483.53839585,184.61296165)(483.54840194,184.58297029)
\curveto(483.57839581,184.47296179)(483.59839579,184.34796192)(483.60840194,184.20797029)
\curveto(483.61839577,184.07796219)(483.62839576,183.94296232)(483.63840194,183.80297029)
\lineto(483.63840194,183.63797029)
\curveto(483.64839574,183.57796269)(483.64839574,183.52296274)(483.63840194,183.47297029)
\curveto(483.62839576,183.42296284)(483.62339576,183.37296289)(483.62340194,183.32297029)
\lineto(483.62340194,183.18797029)
\curveto(483.61339577,183.14796312)(483.60839578,183.10796316)(483.60840194,183.06797029)
\curveto(483.61839577,183.02796324)(483.61339577,182.98296328)(483.59340194,182.93297029)
\curveto(483.57339581,182.82296344)(483.55339583,182.71796355)(483.53340194,182.61797029)
\curveto(483.52339586,182.51796375)(483.50339588,182.41796385)(483.47340194,182.31797029)
\curveto(483.34339604,181.95796431)(483.17839621,181.64296462)(482.97840194,181.37297029)
\curveto(482.77839661,181.10296516)(482.50339688,180.89796537)(482.15340194,180.75797029)
\curveto(482.07339731,180.72796554)(481.9883974,180.70296556)(481.89840194,180.68297029)
\lineto(481.62840194,180.62297029)
\curveto(481.57839781,180.61296565)(481.53339785,180.60796566)(481.49340194,180.60797029)
\curveto(481.45339793,180.61796565)(481.41339797,180.61796565)(481.37340194,180.60797029)
\curveto(481.27339811,180.58796568)(481.17839821,180.58796568)(481.08840194,180.60797029)
\moveto(480.24840194,182.00297029)
\curveto(480.2883991,181.93296433)(480.32839906,181.8679644)(480.36840194,181.80797029)
\curveto(480.40839898,181.75796451)(480.45839893,181.70796456)(480.51840194,181.65797029)
\lineto(480.66840194,181.53797029)
\curveto(480.72839866,181.50796476)(480.79339859,181.48296478)(480.86340194,181.46297029)
\curveto(480.90339848,181.44296482)(480.93839845,181.43296483)(480.96840194,181.43297029)
\curveto(481.00839838,181.44296482)(481.04839834,181.43796483)(481.08840194,181.41797029)
\curveto(481.11839827,181.41796485)(481.15839823,181.41296485)(481.20840194,181.40297029)
\curveto(481.25839813,181.40296486)(481.29839809,181.40796486)(481.32840194,181.41797029)
\lineto(481.55340194,181.46297029)
\curveto(481.80339758,181.54296472)(481.9883974,181.6679646)(482.10840194,181.83797029)
\curveto(482.1883972,181.93796433)(482.25839713,182.0679642)(482.31840194,182.22797029)
\curveto(482.39839699,182.40796386)(482.45839693,182.63296363)(482.49840194,182.90297029)
\curveto(482.53839685,183.18296308)(482.55339683,183.4629628)(482.54340194,183.74297029)
\curveto(482.53339685,184.03296223)(482.50339688,184.30796196)(482.45340194,184.56797029)
\curveto(482.40339698,184.82796144)(482.32839706,185.03796123)(482.22840194,185.19797029)
\curveto(482.10839728,185.39796087)(481.95839743,185.54796072)(481.77840194,185.64797029)
\curveto(481.69839769,185.69796057)(481.60839778,185.72796054)(481.50840194,185.73797029)
\curveto(481.40839798,185.75796051)(481.30339808,185.7679605)(481.19340194,185.76797029)
\curveto(481.17339821,185.75796051)(481.14839824,185.75296051)(481.11840194,185.75297029)
\curveto(481.09839829,185.7629605)(481.07839831,185.7629605)(481.05840194,185.75297029)
\curveto(481.00839838,185.74296052)(480.96339842,185.73296053)(480.92340194,185.72297029)
\curveto(480.8833985,185.72296054)(480.84339854,185.71296055)(480.80340194,185.69297029)
\curveto(480.62339876,185.61296065)(480.47339891,185.49296077)(480.35340194,185.33297029)
\curveto(480.24339914,185.17296109)(480.15339923,184.99296127)(480.08340194,184.79297029)
\curveto(480.02339936,184.60296166)(479.97839941,184.37796189)(479.94840194,184.11797029)
\curveto(479.92839946,183.85796241)(479.92339946,183.59296267)(479.93340194,183.32297029)
\curveto(479.94339944,183.0629632)(479.97339941,182.81296345)(480.02340194,182.57297029)
\curveto(480.0833993,182.34296392)(480.15839923,182.15296411)(480.24840194,182.00297029)
\moveto(491.04840194,179.01797029)
\curveto(491.05838833,178.9679673)(491.06338832,178.87796739)(491.06340194,178.74797029)
\curveto(491.06338832,178.61796765)(491.05338833,178.52796774)(491.03340194,178.47797029)
\curveto(491.01338837,178.42796784)(491.00838838,178.37296789)(491.01840194,178.31297029)
\curveto(491.02838836,178.262968)(491.02838836,178.21296805)(491.01840194,178.16297029)
\curveto(490.97838841,178.02296824)(490.94838844,177.88796838)(490.92840194,177.75797029)
\curveto(490.91838847,177.62796864)(490.8883885,177.50796876)(490.83840194,177.39797029)
\curveto(490.69838869,177.04796922)(490.53338885,176.75296951)(490.34340194,176.51297029)
\curveto(490.15338923,176.28296998)(489.8833895,176.09797017)(489.53340194,175.95797029)
\curveto(489.45338993,175.92797034)(489.36839002,175.90797036)(489.27840194,175.89797029)
\curveto(489.1883902,175.87797039)(489.10339028,175.85797041)(489.02340194,175.83797029)
\curveto(488.97339041,175.82797044)(488.92339046,175.82297044)(488.87340194,175.82297029)
\curveto(488.82339056,175.82297044)(488.77339061,175.81797045)(488.72340194,175.80797029)
\curveto(488.69339069,175.79797047)(488.64339074,175.79797047)(488.57340194,175.80797029)
\curveto(488.50339088,175.80797046)(488.45339093,175.81297045)(488.42340194,175.82297029)
\curveto(488.36339102,175.84297042)(488.30339108,175.85297041)(488.24340194,175.85297029)
\curveto(488.19339119,175.84297042)(488.14339124,175.84797042)(488.09340194,175.86797029)
\curveto(488.00339138,175.88797038)(487.91339147,175.91297035)(487.82340194,175.94297029)
\curveto(487.74339164,175.9629703)(487.66339172,175.99297027)(487.58340194,176.03297029)
\curveto(487.26339212,176.17297009)(487.01339237,176.3679699)(486.83340194,176.61797029)
\curveto(486.65339273,176.87796939)(486.50339288,177.18296908)(486.38340194,177.53297029)
\curveto(486.36339302,177.61296865)(486.34839304,177.69796857)(486.33840194,177.78797029)
\curveto(486.32839306,177.87796839)(486.31339307,177.9629683)(486.29340194,178.04297029)
\curveto(486.2833931,178.07296819)(486.27839311,178.10296816)(486.27840194,178.13297029)
\lineto(486.27840194,178.23797029)
\curveto(486.25839313,178.31796795)(486.24839314,178.39796787)(486.24840194,178.47797029)
\lineto(486.24840194,178.61297029)
\curveto(486.22839316,178.71296755)(486.22839316,178.81296745)(486.24840194,178.91297029)
\lineto(486.24840194,179.09297029)
\curveto(486.25839313,179.14296712)(486.26339312,179.18796708)(486.26340194,179.22797029)
\curveto(486.26339312,179.27796699)(486.26839312,179.32296694)(486.27840194,179.36297029)
\curveto(486.2883931,179.40296686)(486.29339309,179.43796683)(486.29340194,179.46797029)
\curveto(486.29339309,179.50796676)(486.29839309,179.54796672)(486.30840194,179.58797029)
\lineto(486.36840194,179.91797029)
\curveto(486.388393,180.03796623)(486.41839297,180.14796612)(486.45840194,180.24797029)
\curveto(486.59839279,180.57796569)(486.75839263,180.85296541)(486.93840194,181.07297029)
\curveto(487.12839226,181.30296496)(487.388392,181.48796478)(487.71840194,181.62797029)
\curveto(487.79839159,181.6679646)(487.8833915,181.69296457)(487.97340194,181.70297029)
\lineto(488.27340194,181.76297029)
\lineto(488.40840194,181.76297029)
\curveto(488.45839093,181.77296449)(488.50839088,181.77796449)(488.55840194,181.77797029)
\curveto(489.12839026,181.79796447)(489.5883898,181.69296457)(489.93840194,181.46297029)
\curveto(490.29838909,181.24296502)(490.56338882,180.94296532)(490.73340194,180.56297029)
\curveto(490.7833886,180.4629658)(490.82338856,180.3629659)(490.85340194,180.26297029)
\curveto(490.8833885,180.1629661)(490.91338847,180.05796621)(490.94340194,179.94797029)
\curveto(490.95338843,179.90796636)(490.95838843,179.87296639)(490.95840194,179.84297029)
\curveto(490.95838843,179.82296644)(490.96338842,179.79296647)(490.97340194,179.75297029)
\curveto(490.99338839,179.68296658)(491.00338838,179.60796666)(491.00340194,179.52797029)
\curveto(491.00338838,179.44796682)(491.01338837,179.3679669)(491.03340194,179.28797029)
\curveto(491.03338835,179.23796703)(491.03338835,179.19296707)(491.03340194,179.15297029)
\curveto(491.03338835,179.11296715)(491.03838835,179.0679672)(491.04840194,179.01797029)
\moveto(489.93840194,178.58297029)
\curveto(489.94838944,178.63296763)(489.95338943,178.70796756)(489.95340194,178.80797029)
\curveto(489.96338942,178.90796736)(489.95838943,178.98296728)(489.93840194,179.03297029)
\curveto(489.91838947,179.09296717)(489.91338947,179.14796712)(489.92340194,179.19797029)
\curveto(489.94338944,179.25796701)(489.94338944,179.31796695)(489.92340194,179.37797029)
\curveto(489.91338947,179.40796686)(489.90838948,179.44296682)(489.90840194,179.48297029)
\curveto(489.90838948,179.52296674)(489.90338948,179.5629667)(489.89340194,179.60297029)
\curveto(489.87338951,179.68296658)(489.85338953,179.75796651)(489.83340194,179.82797029)
\curveto(489.82338956,179.90796636)(489.80838958,179.98796628)(489.78840194,180.06797029)
\curveto(489.75838963,180.12796614)(489.73338965,180.18796608)(489.71340194,180.24797029)
\curveto(489.69338969,180.30796596)(489.66338972,180.3679659)(489.62340194,180.42797029)
\curveto(489.52338986,180.59796567)(489.39338999,180.73296553)(489.23340194,180.83297029)
\curveto(489.15339023,180.88296538)(489.05839033,180.91796535)(488.94840194,180.93797029)
\curveto(488.83839055,180.95796531)(488.71339067,180.9679653)(488.57340194,180.96797029)
\curveto(488.55339083,180.95796531)(488.52839086,180.95296531)(488.49840194,180.95297029)
\curveto(488.46839092,180.9629653)(488.43839095,180.9629653)(488.40840194,180.95297029)
\lineto(488.25840194,180.89297029)
\curveto(488.20839118,180.88296538)(488.16339122,180.8679654)(488.12340194,180.84797029)
\curveto(487.93339145,180.73796553)(487.7883916,180.59296567)(487.68840194,180.41297029)
\curveto(487.59839179,180.23296603)(487.51839187,180.02796624)(487.44840194,179.79797029)
\curveto(487.40839198,179.6679666)(487.388392,179.53296673)(487.38840194,179.39297029)
\curveto(487.388392,179.262967)(487.37839201,179.11796715)(487.35840194,178.95797029)
\curveto(487.34839204,178.90796736)(487.33839205,178.84796742)(487.32840194,178.77797029)
\curveto(487.32839206,178.70796756)(487.33839205,178.64796762)(487.35840194,178.59797029)
\lineto(487.35840194,178.43297029)
\lineto(487.35840194,178.25297029)
\curveto(487.36839202,178.20296806)(487.37839201,178.14796812)(487.38840194,178.08797029)
\curveto(487.39839199,178.03796823)(487.40339198,177.98296828)(487.40340194,177.92297029)
\curveto(487.41339197,177.8629684)(487.42839196,177.80796846)(487.44840194,177.75797029)
\curveto(487.49839189,177.5679687)(487.55839183,177.39296887)(487.62840194,177.23297029)
\curveto(487.69839169,177.07296919)(487.80339158,176.94296932)(487.94340194,176.84297029)
\curveto(488.07339131,176.74296952)(488.21339117,176.67296959)(488.36340194,176.63297029)
\curveto(488.39339099,176.62296964)(488.41839097,176.61796965)(488.43840194,176.61797029)
\curveto(488.46839092,176.62796964)(488.49839089,176.62796964)(488.52840194,176.61797029)
\curveto(488.54839084,176.61796965)(488.57839081,176.61296965)(488.61840194,176.60297029)
\curveto(488.65839073,176.60296966)(488.69339069,176.60796966)(488.72340194,176.61797029)
\curveto(488.76339062,176.62796964)(488.80339058,176.63296963)(488.84340194,176.63297029)
\curveto(488.8833905,176.63296963)(488.92339046,176.64296962)(488.96340194,176.66297029)
\curveto(489.20339018,176.74296952)(489.39838999,176.87796939)(489.54840194,177.06797029)
\curveto(489.66838972,177.24796902)(489.75838963,177.45296881)(489.81840194,177.68297029)
\curveto(489.83838955,177.75296851)(489.85338953,177.82296844)(489.86340194,177.89297029)
\curveto(489.87338951,177.97296829)(489.8883895,178.05296821)(489.90840194,178.13297029)
\curveto(489.90838948,178.19296807)(489.91338947,178.23796803)(489.92340194,178.26797029)
\curveto(489.92338946,178.28796798)(489.92338946,178.31296795)(489.92340194,178.34297029)
\curveto(489.92338946,178.38296788)(489.92838946,178.41296785)(489.93840194,178.43297029)
\lineto(489.93840194,178.58297029)
}
}
{
\newrgbcolor{curcolor}{0 0 0}
\pscustom[linestyle=none,fillstyle=solid,fillcolor=curcolor]
{
\newpath
\moveto(646.12869735,249.72749178)
\curveto(646.81869272,249.73748115)(647.41869212,249.61748127)(647.92869735,249.36749178)
\curveto(648.44869109,249.11748177)(648.84369069,248.7824821)(649.11369735,248.36249178)
\curveto(649.16369037,248.2824826)(649.20869033,248.19248269)(649.24869735,248.09249178)
\curveto(649.28869025,248.00248288)(649.3336902,247.90748298)(649.38369735,247.80749178)
\curveto(649.42369011,247.70748318)(649.45369008,247.60748328)(649.47369735,247.50749178)
\curveto(649.49369004,247.40748348)(649.51369002,247.30248358)(649.53369735,247.19249178)
\curveto(649.55368998,247.14248374)(649.55868998,247.09748379)(649.54869735,247.05749178)
\curveto(649.53869,247.01748387)(649.54368999,246.97248391)(649.56369735,246.92249178)
\curveto(649.57368996,246.87248401)(649.57868996,246.7874841)(649.57869735,246.66749178)
\curveto(649.57868996,246.55748433)(649.57368996,246.47248441)(649.56369735,246.41249178)
\curveto(649.54368999,246.35248453)(649.53369,246.29248459)(649.53369735,246.23249178)
\curveto(649.54368999,246.17248471)(649.53869,246.11248477)(649.51869735,246.05249178)
\curveto(649.47869006,245.91248497)(649.44369009,245.77748511)(649.41369735,245.64749178)
\curveto(649.38369015,245.51748537)(649.34369019,245.39248549)(649.29369735,245.27249178)
\curveto(649.2336903,245.13248575)(649.16369037,245.00748588)(649.08369735,244.89749178)
\curveto(649.01369052,244.7874861)(648.9386906,244.67748621)(648.85869735,244.56749178)
\lineto(648.79869735,244.50749178)
\curveto(648.78869075,244.4874864)(648.77369076,244.46748642)(648.75369735,244.44749178)
\curveto(648.6336909,244.2874866)(648.49869104,244.14248674)(648.34869735,244.01249178)
\curveto(648.19869134,243.882487)(648.0386915,243.75748713)(647.86869735,243.63749178)
\curveto(647.55869198,243.41748747)(647.26369227,243.21248767)(646.98369735,243.02249178)
\curveto(646.75369278,242.882488)(646.52369301,242.74748814)(646.29369735,242.61749178)
\curveto(646.07369346,242.4874884)(645.85369368,242.35248853)(645.63369735,242.21249178)
\curveto(645.38369415,242.04248884)(645.14369439,241.86248902)(644.91369735,241.67249178)
\curveto(644.69369484,241.4824894)(644.50369503,241.25748963)(644.34369735,240.99749178)
\curveto(644.30369523,240.93748995)(644.26869527,240.87749001)(644.23869735,240.81749178)
\curveto(644.20869533,240.76749012)(644.17869536,240.70249018)(644.14869735,240.62249178)
\curveto(644.12869541,240.55249033)(644.12369541,240.49249039)(644.13369735,240.44249178)
\curveto(644.15369538,240.37249051)(644.18869535,240.31749057)(644.23869735,240.27749178)
\curveto(644.28869525,240.24749064)(644.34869519,240.22749066)(644.41869735,240.21749178)
\lineto(644.65869735,240.21749178)
\lineto(645.40869735,240.21749178)
\lineto(648.21369735,240.21749178)
\lineto(648.87369735,240.21749178)
\curveto(648.96369057,240.21749067)(649.04869049,240.21249067)(649.12869735,240.20249178)
\curveto(649.20869033,240.20249068)(649.27369026,240.1824907)(649.32369735,240.14249178)
\curveto(649.37369016,240.10249078)(649.41369012,240.02749086)(649.44369735,239.91749178)
\curveto(649.48369005,239.81749107)(649.49369004,239.71749117)(649.47369735,239.61749178)
\lineto(649.47369735,239.48249178)
\curveto(649.45369008,239.41249147)(649.4336901,239.35249153)(649.41369735,239.30249178)
\curveto(649.39369014,239.25249163)(649.35869018,239.21249167)(649.30869735,239.18249178)
\curveto(649.25869028,239.14249174)(649.18869035,239.12249176)(649.09869735,239.12249178)
\lineto(648.82869735,239.12249178)
\lineto(647.92869735,239.12249178)
\lineto(644.41869735,239.12249178)
\lineto(643.35369735,239.12249178)
\curveto(643.27369626,239.12249176)(643.18369635,239.11749177)(643.08369735,239.10749178)
\curveto(642.98369655,239.10749178)(642.89869664,239.11749177)(642.82869735,239.13749178)
\curveto(642.61869692,239.20749168)(642.55369698,239.3874915)(642.63369735,239.67749178)
\curveto(642.64369689,239.71749117)(642.64369689,239.75249113)(642.63369735,239.78249178)
\curveto(642.6336969,239.82249106)(642.64369689,239.86749102)(642.66369735,239.91749178)
\curveto(642.68369685,239.99749089)(642.70369683,240.0824908)(642.72369735,240.17249178)
\curveto(642.74369679,240.26249062)(642.76869677,240.34749054)(642.79869735,240.42749178)
\curveto(642.95869658,240.91748997)(643.15869638,241.33248955)(643.39869735,241.67249178)
\curveto(643.57869596,241.92248896)(643.78369575,242.14748874)(644.01369735,242.34749178)
\curveto(644.24369529,242.55748833)(644.48369505,242.75248813)(644.73369735,242.93249178)
\curveto(644.99369454,243.11248777)(645.25869428,243.2824876)(645.52869735,243.44249178)
\curveto(645.80869373,243.61248727)(646.07869346,243.7874871)(646.33869735,243.96749178)
\curveto(646.44869309,244.04748684)(646.55369298,244.12248676)(646.65369735,244.19249178)
\curveto(646.76369277,244.26248662)(646.87369266,244.33748655)(646.98369735,244.41749178)
\curveto(647.02369251,244.44748644)(647.05869248,244.47748641)(647.08869735,244.50749178)
\curveto(647.12869241,244.54748634)(647.16869237,244.57748631)(647.20869735,244.59749178)
\curveto(647.34869219,244.70748618)(647.47369206,244.83248605)(647.58369735,244.97249178)
\curveto(647.60369193,245.00248588)(647.62869191,245.02748586)(647.65869735,245.04749178)
\curveto(647.68869185,245.07748581)(647.71369182,245.10748578)(647.73369735,245.13749178)
\curveto(647.81369172,245.23748565)(647.87869166,245.33748555)(647.92869735,245.43749178)
\curveto(647.98869155,245.53748535)(648.04369149,245.64748524)(648.09369735,245.76749178)
\curveto(648.12369141,245.83748505)(648.14369139,245.91248497)(648.15369735,245.99249178)
\lineto(648.21369735,246.23249178)
\lineto(648.21369735,246.32249178)
\curveto(648.22369131,246.35248453)(648.22869131,246.3824845)(648.22869735,246.41249178)
\curveto(648.24869129,246.4824844)(648.25369128,246.57748431)(648.24369735,246.69749178)
\curveto(648.24369129,246.82748406)(648.2336913,246.92748396)(648.21369735,246.99749178)
\curveto(648.19369134,247.07748381)(648.17369136,247.15248373)(648.15369735,247.22249178)
\curveto(648.14369139,247.30248358)(648.12369141,247.3824835)(648.09369735,247.46249178)
\curveto(647.98369155,247.70248318)(647.8336917,247.90248298)(647.64369735,248.06249178)
\curveto(647.46369207,248.23248265)(647.24369229,248.37248251)(646.98369735,248.48249178)
\curveto(646.91369262,248.50248238)(646.84369269,248.51748237)(646.77369735,248.52749178)
\curveto(646.70369283,248.54748234)(646.62869291,248.56748232)(646.54869735,248.58749178)
\curveto(646.46869307,248.60748228)(646.35869318,248.61748227)(646.21869735,248.61749178)
\curveto(646.08869345,248.61748227)(645.98369355,248.60748228)(645.90369735,248.58749178)
\curveto(645.84369369,248.57748231)(645.78869375,248.57248231)(645.73869735,248.57249178)
\curveto(645.68869385,248.57248231)(645.6386939,248.56248232)(645.58869735,248.54249178)
\curveto(645.48869405,248.50248238)(645.39369414,248.46248242)(645.30369735,248.42249178)
\curveto(645.22369431,248.3824825)(645.14369439,248.33748255)(645.06369735,248.28749178)
\curveto(645.0336945,248.26748262)(645.00369453,248.24248264)(644.97369735,248.21249178)
\curveto(644.95369458,248.1824827)(644.92869461,248.15748273)(644.89869735,248.13749178)
\lineto(644.82369735,248.06249178)
\curveto(644.79369474,248.04248284)(644.76869477,248.02248286)(644.74869735,248.00249178)
\lineto(644.59869735,247.79249178)
\curveto(644.55869498,247.73248315)(644.51369502,247.66748322)(644.46369735,247.59749178)
\curveto(644.40369513,247.50748338)(644.35369518,247.40248348)(644.31369735,247.28249178)
\curveto(644.28369525,247.17248371)(644.24869529,247.06248382)(644.20869735,246.95249178)
\curveto(644.16869537,246.84248404)(644.14369539,246.69748419)(644.13369735,246.51749178)
\curveto(644.12369541,246.34748454)(644.09369544,246.22248466)(644.04369735,246.14249178)
\curveto(643.99369554,246.06248482)(643.91869562,246.01748487)(643.81869735,246.00749178)
\curveto(643.71869582,245.99748489)(643.60869593,245.99248489)(643.48869735,245.99249178)
\curveto(643.44869609,245.99248489)(643.40869613,245.9874849)(643.36869735,245.97749178)
\curveto(643.32869621,245.97748491)(643.29369624,245.9824849)(643.26369735,245.99249178)
\curveto(643.21369632,246.01248487)(643.16369637,246.02248486)(643.11369735,246.02249178)
\curveto(643.07369646,246.02248486)(643.0336965,246.03248485)(642.99369735,246.05249178)
\curveto(642.90369663,246.11248477)(642.85869668,246.24748464)(642.85869735,246.45749178)
\lineto(642.85869735,246.57749178)
\curveto(642.86869667,246.63748425)(642.87369666,246.69748419)(642.87369735,246.75749178)
\curveto(642.88369665,246.82748406)(642.89369664,246.89248399)(642.90369735,246.95249178)
\curveto(642.92369661,247.06248382)(642.94369659,247.16248372)(642.96369735,247.25249178)
\curveto(642.98369655,247.35248353)(643.01369652,247.44748344)(643.05369735,247.53749178)
\curveto(643.07369646,247.60748328)(643.09369644,247.66748322)(643.11369735,247.71749178)
\lineto(643.17369735,247.89749178)
\curveto(643.29369624,248.15748273)(643.44869609,248.40248248)(643.63869735,248.63249178)
\curveto(643.8386957,248.86248202)(644.05369548,249.04748184)(644.28369735,249.18749178)
\curveto(644.39369514,249.26748162)(644.50869503,249.33248155)(644.62869735,249.38249178)
\lineto(645.01869735,249.53249178)
\curveto(645.12869441,249.5824813)(645.24369429,249.61248127)(645.36369735,249.62249178)
\curveto(645.48369405,249.64248124)(645.60869393,249.66748122)(645.73869735,249.69749178)
\curveto(645.80869373,249.69748119)(645.87369366,249.69748119)(645.93369735,249.69749178)
\curveto(645.99369354,249.70748118)(646.05869348,249.71748117)(646.12869735,249.72749178)
}
}
{
\newrgbcolor{curcolor}{0 0 0}
\pscustom[linestyle=none,fillstyle=solid,fillcolor=curcolor]
{
\newpath
\moveto(652.73830673,249.53249178)
\lineto(656.33830673,249.53249178)
\lineto(656.98330673,249.53249178)
\curveto(657.0633002,249.53248135)(657.13830012,249.52748136)(657.20830673,249.51749178)
\curveto(657.27829998,249.51748137)(657.33829992,249.50748138)(657.38830673,249.48749178)
\curveto(657.4582998,249.45748143)(657.51329975,249.39748149)(657.55330673,249.30749178)
\curveto(657.57329969,249.27748161)(657.58329968,249.23748165)(657.58330673,249.18749178)
\lineto(657.58330673,249.05249178)
\curveto(657.59329967,248.94248194)(657.58829967,248.83748205)(657.56830673,248.73749178)
\curveto(657.5582997,248.63748225)(657.52329974,248.56748232)(657.46330673,248.52749178)
\curveto(657.37329989,248.45748243)(657.23830002,248.42248246)(657.05830673,248.42249178)
\curveto(656.87830038,248.43248245)(656.71330055,248.43748245)(656.56330673,248.43749178)
\lineto(654.56830673,248.43749178)
\lineto(654.07330673,248.43749178)
\lineto(653.93830673,248.43749178)
\curveto(653.89830336,248.43748245)(653.8583034,248.43248245)(653.81830673,248.42249178)
\lineto(653.60830673,248.42249178)
\curveto(653.49830376,248.39248249)(653.41830384,248.35248253)(653.36830673,248.30249178)
\curveto(653.31830394,248.26248262)(653.28330398,248.20748268)(653.26330673,248.13749178)
\curveto(653.24330402,248.07748281)(653.22830403,248.00748288)(653.21830673,247.92749178)
\curveto(653.20830405,247.84748304)(653.18830407,247.75748313)(653.15830673,247.65749178)
\curveto(653.10830415,247.45748343)(653.06830419,247.25248363)(653.03830673,247.04249178)
\curveto(653.00830425,246.83248405)(652.96830429,246.62748426)(652.91830673,246.42749178)
\curveto(652.89830436,246.35748453)(652.88830437,246.2874846)(652.88830673,246.21749178)
\curveto(652.88830437,246.15748473)(652.87830438,246.09248479)(652.85830673,246.02249178)
\curveto(652.84830441,245.99248489)(652.83830442,245.95248493)(652.82830673,245.90249178)
\curveto(652.82830443,245.86248502)(652.83330443,245.82248506)(652.84330673,245.78249178)
\curveto(652.8633044,245.73248515)(652.88830437,245.6874852)(652.91830673,245.64749178)
\curveto(652.9583043,245.61748527)(653.01830424,245.61248527)(653.09830673,245.63249178)
\curveto(653.1583041,245.65248523)(653.21830404,245.67748521)(653.27830673,245.70749178)
\curveto(653.33830392,245.74748514)(653.39830386,245.7824851)(653.45830673,245.81249178)
\curveto(653.51830374,245.83248505)(653.56830369,245.84748504)(653.60830673,245.85749178)
\curveto(653.79830346,245.93748495)(654.00330326,245.99248489)(654.22330673,246.02249178)
\curveto(654.45330281,246.05248483)(654.68330258,246.06248482)(654.91330673,246.05249178)
\curveto(655.15330211,246.05248483)(655.38330188,246.02748486)(655.60330673,245.97749178)
\curveto(655.82330144,245.93748495)(656.02330124,245.87748501)(656.20330673,245.79749178)
\curveto(656.25330101,245.77748511)(656.29830096,245.75748513)(656.33830673,245.73749178)
\curveto(656.38830087,245.71748517)(656.43830082,245.69248519)(656.48830673,245.66249178)
\curveto(656.83830042,245.45248543)(657.11830014,245.22248566)(657.32830673,244.97249178)
\curveto(657.54829971,244.72248616)(657.74329952,244.39748649)(657.91330673,243.99749178)
\curveto(657.9632993,243.887487)(657.99829926,243.77748711)(658.01830673,243.66749178)
\curveto(658.03829922,243.55748733)(658.0632992,243.44248744)(658.09330673,243.32249178)
\curveto(658.10329916,243.29248759)(658.10829915,243.24748764)(658.10830673,243.18749178)
\curveto(658.12829913,243.12748776)(658.13829912,243.05748783)(658.13830673,242.97749178)
\curveto(658.13829912,242.90748798)(658.14829911,242.84248804)(658.16830673,242.78249178)
\lineto(658.16830673,242.61749178)
\curveto(658.17829908,242.56748832)(658.18329908,242.49748839)(658.18330673,242.40749178)
\curveto(658.18329908,242.31748857)(658.17329909,242.24748864)(658.15330673,242.19749178)
\curveto(658.13329913,242.13748875)(658.12829913,242.07748881)(658.13830673,242.01749178)
\curveto(658.14829911,241.96748892)(658.14329912,241.91748897)(658.12330673,241.86749178)
\curveto(658.08329918,241.70748918)(658.04829921,241.55748933)(658.01830673,241.41749178)
\curveto(657.98829927,241.27748961)(657.94329932,241.14248974)(657.88330673,241.01249178)
\curveto(657.72329954,240.64249024)(657.50329976,240.30749058)(657.22330673,240.00749178)
\curveto(656.94330032,239.70749118)(656.62330064,239.47749141)(656.26330673,239.31749178)
\curveto(656.09330117,239.23749165)(655.89330137,239.16249172)(655.66330673,239.09249178)
\curveto(655.55330171,239.05249183)(655.43830182,239.02749186)(655.31830673,239.01749178)
\curveto(655.19830206,239.00749188)(655.07830218,238.9874919)(654.95830673,238.95749178)
\curveto(654.90830235,238.93749195)(654.85330241,238.93749195)(654.79330673,238.95749178)
\curveto(654.73330253,238.96749192)(654.67330259,238.96249192)(654.61330673,238.94249178)
\curveto(654.51330275,238.92249196)(654.41330285,238.92249196)(654.31330673,238.94249178)
\lineto(654.17830673,238.94249178)
\curveto(654.12830313,238.96249192)(654.06830319,238.97249191)(653.99830673,238.97249178)
\curveto(653.93830332,238.96249192)(653.88330338,238.96749192)(653.83330673,238.98749178)
\curveto(653.79330347,238.99749189)(653.7583035,239.00249188)(653.72830673,239.00249178)
\curveto(653.69830356,239.00249188)(653.6633036,239.00749188)(653.62330673,239.01749178)
\lineto(653.35330673,239.07749178)
\curveto(653.263304,239.09749179)(653.17830408,239.12749176)(653.09830673,239.16749178)
\curveto(652.7583045,239.30749158)(652.46830479,239.46249142)(652.22830673,239.63249178)
\curveto(651.98830527,239.81249107)(651.76830549,240.04249084)(651.56830673,240.32249178)
\curveto(651.41830584,240.55249033)(651.30330596,240.79249009)(651.22330673,241.04249178)
\curveto(651.20330606,241.09248979)(651.19330607,241.13748975)(651.19330673,241.17749178)
\curveto(651.19330607,241.22748966)(651.18330608,241.27748961)(651.16330673,241.32749178)
\curveto(651.14330612,241.3874895)(651.12830613,241.46748942)(651.11830673,241.56749178)
\curveto(651.11830614,241.66748922)(651.13830612,241.74248914)(651.17830673,241.79249178)
\curveto(651.22830603,241.87248901)(651.30830595,241.91748897)(651.41830673,241.92749178)
\curveto(651.52830573,241.93748895)(651.64330562,241.94248894)(651.76330673,241.94249178)
\lineto(651.92830673,241.94249178)
\curveto(651.98830527,241.94248894)(652.04330522,241.93248895)(652.09330673,241.91249178)
\curveto(652.18330508,241.89248899)(652.25330501,241.85248903)(652.30330673,241.79249178)
\curveto(652.37330489,241.70248918)(652.41830484,241.59248929)(652.43830673,241.46249178)
\curveto(652.46830479,241.34248954)(652.51330475,241.23748965)(652.57330673,241.14749178)
\curveto(652.7633045,240.80749008)(653.02330424,240.53749035)(653.35330673,240.33749178)
\curveto(653.45330381,240.27749061)(653.5583037,240.22749066)(653.66830673,240.18749178)
\curveto(653.78830347,240.15749073)(653.90830335,240.12249076)(654.02830673,240.08249178)
\curveto(654.19830306,240.03249085)(654.40330286,240.01249087)(654.64330673,240.02249178)
\curveto(654.89330237,240.04249084)(655.09330217,240.07749081)(655.24330673,240.12749178)
\curveto(655.61330165,240.24749064)(655.90330136,240.40749048)(656.11330673,240.60749178)
\curveto(656.33330093,240.81749007)(656.51330075,241.09748979)(656.65330673,241.44749178)
\curveto(656.70330056,241.54748934)(656.73330053,241.65248923)(656.74330673,241.76249178)
\curveto(656.7633005,241.87248901)(656.78830047,241.9874889)(656.81830673,242.10749178)
\lineto(656.81830673,242.21249178)
\curveto(656.82830043,242.25248863)(656.83330043,242.29248859)(656.83330673,242.33249178)
\curveto(656.84330042,242.36248852)(656.84330042,242.39748849)(656.83330673,242.43749178)
\lineto(656.83330673,242.55749178)
\curveto(656.83330043,242.81748807)(656.80330046,243.06248782)(656.74330673,243.29249178)
\curveto(656.63330063,243.64248724)(656.47830078,243.93748695)(656.27830673,244.17749178)
\curveto(656.07830118,244.42748646)(655.81830144,244.62248626)(655.49830673,244.76249178)
\lineto(655.31830673,244.82249178)
\curveto(655.26830199,244.84248604)(655.20830205,244.86248602)(655.13830673,244.88249178)
\curveto(655.08830217,244.90248598)(655.02830223,244.91248597)(654.95830673,244.91249178)
\curveto(654.89830236,244.92248596)(654.83330243,244.93748595)(654.76330673,244.95749178)
\lineto(654.61330673,244.95749178)
\curveto(654.57330269,244.97748591)(654.51830274,244.9874859)(654.44830673,244.98749178)
\curveto(654.38830287,244.9874859)(654.33330293,244.97748591)(654.28330673,244.95749178)
\lineto(654.17830673,244.95749178)
\curveto(654.14830311,244.95748593)(654.11330315,244.95248593)(654.07330673,244.94249178)
\lineto(653.83330673,244.88249178)
\curveto(653.75330351,244.87248601)(653.67330359,244.85248603)(653.59330673,244.82249178)
\curveto(653.35330391,244.72248616)(653.12330414,244.5874863)(652.90330673,244.41749178)
\curveto(652.81330445,244.34748654)(652.72830453,244.27248661)(652.64830673,244.19249178)
\curveto(652.56830469,244.12248676)(652.46830479,244.06748682)(652.34830673,244.02749178)
\curveto(652.258305,243.99748689)(652.11830514,243.9874869)(651.92830673,243.99749178)
\curveto(651.74830551,244.00748688)(651.62830563,244.03248685)(651.56830673,244.07249178)
\curveto(651.51830574,244.11248677)(651.47830578,244.17248671)(651.44830673,244.25249178)
\curveto(651.42830583,244.33248655)(651.42830583,244.41748647)(651.44830673,244.50749178)
\curveto(651.47830578,244.62748626)(651.49830576,244.74748614)(651.50830673,244.86749178)
\curveto(651.52830573,244.99748589)(651.55330571,245.12248576)(651.58330673,245.24249178)
\curveto(651.60330566,245.2824856)(651.60830565,245.31748557)(651.59830673,245.34749178)
\curveto(651.59830566,245.3874855)(651.60830565,245.43248545)(651.62830673,245.48249178)
\curveto(651.64830561,245.57248531)(651.6633056,245.66248522)(651.67330673,245.75249178)
\curveto(651.68330558,245.85248503)(651.70330556,245.94748494)(651.73330673,246.03749178)
\curveto(651.74330552,246.09748479)(651.74830551,246.15748473)(651.74830673,246.21749178)
\curveto(651.7583055,246.27748461)(651.77330549,246.33748455)(651.79330673,246.39749178)
\curveto(651.84330542,246.59748429)(651.87830538,246.80248408)(651.89830673,247.01249178)
\curveto(651.92830533,247.23248365)(651.96830529,247.44248344)(652.01830673,247.64249178)
\curveto(652.04830521,247.74248314)(652.06830519,247.84248304)(652.07830673,247.94249178)
\curveto(652.08830517,248.04248284)(652.10330516,248.14248274)(652.12330673,248.24249178)
\curveto(652.13330513,248.27248261)(652.13830512,248.31248257)(652.13830673,248.36249178)
\curveto(652.16830509,248.47248241)(652.18830507,248.57748231)(652.19830673,248.67749178)
\curveto(652.21830504,248.7874821)(652.24330502,248.89748199)(652.27330673,249.00749178)
\curveto(652.29330497,249.0874818)(652.30830495,249.15748173)(652.31830673,249.21749178)
\curveto(652.32830493,249.2874816)(652.35330491,249.34748154)(652.39330673,249.39749178)
\curveto(652.41330485,249.42748146)(652.44330482,249.44748144)(652.48330673,249.45749178)
\curveto(652.52330474,249.47748141)(652.56830469,249.49748139)(652.61830673,249.51749178)
\curveto(652.67830458,249.51748137)(652.71830454,249.52248136)(652.73830673,249.53249178)
}
}
{
\newrgbcolor{curcolor}{0 0 0}
\pscustom[linestyle=none,fillstyle=solid,fillcolor=curcolor]
{
\newpath
\moveto(660.5329161,240.75749178)
\lineto(660.8329161,240.75749178)
\curveto(660.94291404,240.76749012)(661.04791394,240.76749012)(661.1479161,240.75749178)
\curveto(661.25791373,240.75749013)(661.35791363,240.74749014)(661.4479161,240.72749178)
\curveto(661.53791345,240.71749017)(661.60791338,240.69249019)(661.6579161,240.65249178)
\curveto(661.67791331,240.63249025)(661.69291329,240.60249028)(661.7029161,240.56249178)
\curveto(661.72291326,240.52249036)(661.74291324,240.47749041)(661.7629161,240.42749178)
\lineto(661.7629161,240.35249178)
\curveto(661.77291321,240.30249058)(661.77291321,240.24749064)(661.7629161,240.18749178)
\lineto(661.7629161,240.03749178)
\lineto(661.7629161,239.55749178)
\curveto(661.76291322,239.3874915)(661.72291326,239.26749162)(661.6429161,239.19749178)
\curveto(661.57291341,239.14749174)(661.4829135,239.12249176)(661.3729161,239.12249178)
\lineto(661.0429161,239.12249178)
\lineto(660.5929161,239.12249178)
\curveto(660.44291454,239.12249176)(660.32791466,239.15249173)(660.2479161,239.21249178)
\curveto(660.20791478,239.24249164)(660.17791481,239.29249159)(660.1579161,239.36249178)
\curveto(660.13791485,239.44249144)(660.12291486,239.52749136)(660.1129161,239.61749178)
\lineto(660.1129161,239.90249178)
\curveto(660.12291486,240.00249088)(660.12791486,240.0874908)(660.1279161,240.15749178)
\lineto(660.1279161,240.35249178)
\curveto(660.12791486,240.41249047)(660.13791485,240.46749042)(660.1579161,240.51749178)
\curveto(660.19791479,240.62749026)(660.26791472,240.69749019)(660.3679161,240.72749178)
\curveto(660.39791459,240.72749016)(660.45291453,240.73749015)(660.5329161,240.75749178)
}
}
{
\newrgbcolor{curcolor}{0 0 0}
\pscustom[linestyle=none,fillstyle=solid,fillcolor=curcolor]
{
\newpath
\moveto(670.75307235,242.19749178)
\curveto(670.76306463,242.15748873)(670.76306463,242.10748878)(670.75307235,242.04749178)
\curveto(670.75306464,241.9874889)(670.74806465,241.93748895)(670.73807235,241.89749178)
\curveto(670.73806466,241.85748903)(670.73306466,241.81748907)(670.72307235,241.77749178)
\lineto(670.72307235,241.67249178)
\curveto(670.70306469,241.59248929)(670.68806471,241.51248937)(670.67807235,241.43249178)
\curveto(670.66806473,241.35248953)(670.64806475,241.27748961)(670.61807235,241.20749178)
\curveto(670.5980648,241.12748976)(670.57806482,241.05248983)(670.55807235,240.98249178)
\curveto(670.53806486,240.91248997)(670.50806489,240.83749005)(670.46807235,240.75749178)
\curveto(670.28806511,240.33749055)(670.03306536,239.99749089)(669.70307235,239.73749178)
\curveto(669.37306602,239.47749141)(668.98306641,239.27249161)(668.53307235,239.12249178)
\curveto(668.41306698,239.0824918)(668.28806711,239.05749183)(668.15807235,239.04749178)
\curveto(668.03806736,239.02749186)(667.91306748,239.00249188)(667.78307235,238.97249178)
\curveto(667.72306767,238.96249192)(667.65806774,238.95749193)(667.58807235,238.95749178)
\curveto(667.52806787,238.95749193)(667.46306793,238.95249193)(667.39307235,238.94249178)
\lineto(667.27307235,238.94249178)
\lineto(667.07807235,238.94249178)
\curveto(667.01806838,238.93249195)(666.96306843,238.93749195)(666.91307235,238.95749178)
\curveto(666.84306855,238.97749191)(666.77806862,238.9824919)(666.71807235,238.97249178)
\curveto(666.65806874,238.96249192)(666.5980688,238.96749192)(666.53807235,238.98749178)
\curveto(666.48806891,238.99749189)(666.44306895,239.00249188)(666.40307235,239.00249178)
\curveto(666.36306903,239.00249188)(666.31806908,239.01249187)(666.26807235,239.03249178)
\curveto(666.18806921,239.05249183)(666.11306928,239.07249181)(666.04307235,239.09249178)
\curveto(665.97306942,239.10249178)(665.90306949,239.11749177)(665.83307235,239.13749178)
\curveto(665.35307004,239.30749158)(664.95307044,239.51749137)(664.63307235,239.76749178)
\curveto(664.32307107,240.02749086)(664.07307132,240.3824905)(663.88307235,240.83249178)
\curveto(663.85307154,240.89248999)(663.82807157,240.95248993)(663.80807235,241.01249178)
\curveto(663.7980716,241.0824898)(663.78307161,241.15748973)(663.76307235,241.23749178)
\curveto(663.74307165,241.29748959)(663.72807167,241.36248952)(663.71807235,241.43249178)
\curveto(663.70807169,241.50248938)(663.6930717,241.57248931)(663.67307235,241.64249178)
\curveto(663.66307173,241.69248919)(663.65807174,241.73248915)(663.65807235,241.76249178)
\lineto(663.65807235,241.88249178)
\curveto(663.64807175,241.92248896)(663.63807176,241.97248891)(663.62807235,242.03249178)
\curveto(663.62807177,242.09248879)(663.63307176,242.14248874)(663.64307235,242.18249178)
\lineto(663.64307235,242.31749178)
\curveto(663.65307174,242.36748852)(663.65807174,242.41748847)(663.65807235,242.46749178)
\curveto(663.67807172,242.56748832)(663.6930717,242.66248822)(663.70307235,242.75249178)
\curveto(663.71307168,242.85248803)(663.73307166,242.94748794)(663.76307235,243.03749178)
\curveto(663.81307158,243.1874877)(663.86807153,243.32748756)(663.92807235,243.45749178)
\curveto(663.98807141,243.5874873)(664.05807134,243.70748718)(664.13807235,243.81749178)
\curveto(664.16807123,243.86748702)(664.1980712,243.90748698)(664.22807235,243.93749178)
\curveto(664.26807113,243.96748692)(664.30307109,244.00248688)(664.33307235,244.04249178)
\curveto(664.393071,244.12248676)(664.46307093,244.19248669)(664.54307235,244.25249178)
\curveto(664.60307079,244.30248658)(664.66307073,244.34748654)(664.72307235,244.38749178)
\lineto(664.93307235,244.53749178)
\curveto(664.98307041,244.57748631)(665.03307036,244.61248627)(665.08307235,244.64249178)
\curveto(665.13307026,244.6824862)(665.16807023,244.73748615)(665.18807235,244.80749178)
\curveto(665.18807021,244.83748605)(665.17807022,244.86248602)(665.15807235,244.88249178)
\curveto(665.14807025,244.91248597)(665.13807026,244.93748595)(665.12807235,244.95749178)
\curveto(665.08807031,245.00748588)(665.03807036,245.05248583)(664.97807235,245.09249178)
\curveto(664.92807047,245.14248574)(664.87807052,245.1874857)(664.82807235,245.22749178)
\curveto(664.78807061,245.25748563)(664.73807066,245.31248557)(664.67807235,245.39249178)
\curveto(664.65807074,245.42248546)(664.62807077,245.44748544)(664.58807235,245.46749178)
\curveto(664.55807084,245.49748539)(664.53307086,245.53248535)(664.51307235,245.57249178)
\curveto(664.34307105,245.7824851)(664.21307118,246.02748486)(664.12307235,246.30749178)
\curveto(664.10307129,246.3874845)(664.08807131,246.46748442)(664.07807235,246.54749178)
\curveto(664.06807133,246.62748426)(664.05307134,246.70748418)(664.03307235,246.78749178)
\curveto(664.01307138,246.83748405)(664.00307139,246.90248398)(664.00307235,246.98249178)
\curveto(664.00307139,247.07248381)(664.01307138,247.14248374)(664.03307235,247.19249178)
\curveto(664.03307136,247.29248359)(664.03807136,247.36248352)(664.04807235,247.40249178)
\curveto(664.06807133,247.4824834)(664.08307131,247.55248333)(664.09307235,247.61249178)
\curveto(664.10307129,247.6824832)(664.11807128,247.75248313)(664.13807235,247.82249178)
\curveto(664.28807111,248.25248263)(664.50307089,248.59748229)(664.78307235,248.85749178)
\curveto(665.07307032,249.11748177)(665.42306997,249.33248155)(665.83307235,249.50249178)
\curveto(665.94306945,249.55248133)(666.05806934,249.5824813)(666.17807235,249.59249178)
\curveto(666.30806909,249.61248127)(666.43806896,249.64248124)(666.56807235,249.68249178)
\curveto(666.64806875,249.6824812)(666.71806868,249.6824812)(666.77807235,249.68249178)
\curveto(666.84806855,249.69248119)(666.92306847,249.70248118)(667.00307235,249.71249178)
\curveto(667.7930676,249.73248115)(668.44806695,249.60248128)(668.96807235,249.32249178)
\curveto(669.4980659,249.04248184)(669.87806552,248.63248225)(670.10807235,248.09249178)
\curveto(670.21806518,247.86248302)(670.28806511,247.57748331)(670.31807235,247.23749178)
\curveto(670.35806504,246.90748398)(670.32806507,246.60248428)(670.22807235,246.32249178)
\curveto(670.18806521,246.19248469)(670.13806526,246.07248481)(670.07807235,245.96249178)
\curveto(670.02806537,245.85248503)(669.96806543,245.74748514)(669.89807235,245.64749178)
\curveto(669.87806552,245.60748528)(669.84806555,245.57248531)(669.80807235,245.54249178)
\lineto(669.71807235,245.45249178)
\curveto(669.66806573,245.36248552)(669.60806579,245.29748559)(669.53807235,245.25749178)
\curveto(669.48806591,245.20748568)(669.43306596,245.15748573)(669.37307235,245.10749178)
\curveto(669.32306607,245.06748582)(669.27806612,245.02248586)(669.23807235,244.97249178)
\curveto(669.21806618,244.95248593)(669.1980662,244.92748596)(669.17807235,244.89749178)
\curveto(669.16806623,244.87748601)(669.16806623,244.85248603)(669.17807235,244.82249178)
\curveto(669.18806621,244.77248611)(669.21806618,244.72248616)(669.26807235,244.67249178)
\curveto(669.31806608,244.63248625)(669.37306602,244.59248629)(669.43307235,244.55249178)
\lineto(669.61307235,244.43249178)
\curveto(669.67306572,244.40248648)(669.72306567,244.37248651)(669.76307235,244.34249178)
\curveto(670.0930653,244.10248678)(670.34306505,243.79248709)(670.51307235,243.41249178)
\curveto(670.55306484,243.33248755)(670.58306481,243.24748764)(670.60307235,243.15749178)
\curveto(670.63306476,243.06748782)(670.65806474,242.97748791)(670.67807235,242.88749178)
\curveto(670.68806471,242.83748805)(670.6980647,242.7824881)(670.70807235,242.72249178)
\lineto(670.73807235,242.57249178)
\curveto(670.74806465,242.51248837)(670.74806465,242.44748844)(670.73807235,242.37749178)
\curveto(670.72806467,242.31748857)(670.73306466,242.25748863)(670.75307235,242.19749178)
\moveto(665.36807235,247.23749178)
\curveto(665.33807006,247.12748376)(665.33307006,246.9874839)(665.35307235,246.81749178)
\curveto(665.37307002,246.65748423)(665.39807,246.53248435)(665.42807235,246.44249178)
\curveto(665.53806986,246.12248476)(665.68806971,245.87748501)(665.87807235,245.70749178)
\curveto(666.06806933,245.54748534)(666.33306906,245.41748547)(666.67307235,245.31749178)
\curveto(666.80306859,245.2874856)(666.96806843,245.26248562)(667.16807235,245.24249178)
\curveto(667.36806803,245.23248565)(667.53806786,245.24748564)(667.67807235,245.28749178)
\curveto(667.96806743,245.36748552)(668.20806719,245.47748541)(668.39807235,245.61749178)
\curveto(668.5980668,245.76748512)(668.75306664,245.96748492)(668.86307235,246.21749178)
\curveto(668.88306651,246.26748462)(668.8930665,246.31248457)(668.89307235,246.35249178)
\curveto(668.90306649,246.39248449)(668.91806648,246.43748445)(668.93807235,246.48749178)
\curveto(668.96806643,246.59748429)(668.98806641,246.73748415)(668.99807235,246.90749178)
\curveto(669.00806639,247.07748381)(668.9980664,247.22248366)(668.96807235,247.34249178)
\curveto(668.94806645,247.43248345)(668.92306647,247.51748337)(668.89307235,247.59749178)
\curveto(668.87306652,247.67748321)(668.83806656,247.75748313)(668.78807235,247.83749178)
\curveto(668.61806678,248.10748278)(668.393067,248.30248258)(668.11307235,248.42249178)
\curveto(667.84306755,248.54248234)(667.48306791,248.60248228)(667.03307235,248.60249178)
\curveto(667.01306838,248.5824823)(666.98306841,248.57748231)(666.94307235,248.58749178)
\curveto(666.90306849,248.59748229)(666.86806853,248.59748229)(666.83807235,248.58749178)
\curveto(666.78806861,248.56748232)(666.73306866,248.55248233)(666.67307235,248.54249178)
\curveto(666.62306877,248.54248234)(666.57306882,248.53248235)(666.52307235,248.51249178)
\curveto(666.28306911,248.42248246)(666.07306932,248.30748258)(665.89307235,248.16749178)
\curveto(665.71306968,248.03748285)(665.57306982,247.85748303)(665.47307235,247.62749178)
\curveto(665.45306994,247.56748332)(665.43306996,247.50248338)(665.41307235,247.43249178)
\curveto(665.40306999,247.37248351)(665.38807001,247.30748358)(665.36807235,247.23749178)
\moveto(669.38807235,241.70249178)
\curveto(669.43806596,241.89248899)(669.44306595,242.09748879)(669.40307235,242.31749178)
\curveto(669.37306602,242.53748835)(669.32806607,242.71748817)(669.26807235,242.85749178)
\curveto(669.0980663,243.22748766)(668.83806656,243.53248735)(668.48807235,243.77249178)
\curveto(668.14806725,244.01248687)(667.71306768,244.13248675)(667.18307235,244.13249178)
\curveto(667.15306824,244.11248677)(667.11306828,244.10748678)(667.06307235,244.11749178)
\curveto(667.01306838,244.13748675)(666.97306842,244.14248674)(666.94307235,244.13249178)
\lineto(666.67307235,244.07249178)
\curveto(666.5930688,244.06248682)(666.51306888,244.04748684)(666.43307235,244.02749178)
\curveto(666.13306926,243.91748697)(665.86806953,243.77248711)(665.63807235,243.59249178)
\curveto(665.41806998,243.41248747)(665.24807015,243.1824877)(665.12807235,242.90249178)
\curveto(665.0980703,242.82248806)(665.07307032,242.74248814)(665.05307235,242.66249178)
\curveto(665.03307036,242.5824883)(665.01307038,242.49748839)(664.99307235,242.40749178)
\curveto(664.96307043,242.2874886)(664.95307044,242.13748875)(664.96307235,241.95749178)
\curveto(664.98307041,241.77748911)(665.00807039,241.63748925)(665.03807235,241.53749178)
\curveto(665.05807034,241.4874894)(665.06807033,241.44248944)(665.06807235,241.40249178)
\curveto(665.07807032,241.37248951)(665.0930703,241.33248955)(665.11307235,241.28249178)
\curveto(665.21307018,241.06248982)(665.34307005,240.86249002)(665.50307235,240.68249178)
\curveto(665.67306972,240.50249038)(665.86806953,240.36749052)(666.08807235,240.27749178)
\curveto(666.15806924,240.23749065)(666.25306914,240.20249068)(666.37307235,240.17249178)
\curveto(666.5930688,240.0824908)(666.84806855,240.03749085)(667.13807235,240.03749178)
\lineto(667.42307235,240.03749178)
\curveto(667.52306787,240.05749083)(667.61806778,240.07249081)(667.70807235,240.08249178)
\curveto(667.7980676,240.09249079)(667.88806751,240.11249077)(667.97807235,240.14249178)
\curveto(668.23806716,240.22249066)(668.47806692,240.35249053)(668.69807235,240.53249178)
\curveto(668.92806647,240.72249016)(669.0980663,240.93748995)(669.20807235,241.17749178)
\curveto(669.24806615,241.25748963)(669.27806612,241.33748955)(669.29807235,241.41749178)
\curveto(669.32806607,241.50748938)(669.35806604,241.60248928)(669.38807235,241.70249178)
}
}
{
\newrgbcolor{curcolor}{0 0 0}
\pscustom[linestyle=none,fillstyle=solid,fillcolor=curcolor]
{
\newpath
\moveto(681.89268173,247.64249178)
\curveto(681.69267143,247.35248353)(681.48267164,247.06748382)(681.26268173,246.78749178)
\curveto(681.05267207,246.50748438)(680.84767227,246.22248466)(680.64768173,245.93249178)
\curveto(680.04767307,245.0824858)(679.44267368,244.24248664)(678.83268173,243.41249178)
\curveto(678.2226749,242.59248829)(677.6176755,241.75748913)(677.01768173,240.90749178)
\lineto(676.50768173,240.18749178)
\lineto(675.99768173,239.49749178)
\curveto(675.9176772,239.3874915)(675.83767728,239.27249161)(675.75768173,239.15249178)
\curveto(675.67767744,239.03249185)(675.58267754,238.93749195)(675.47268173,238.86749178)
\curveto(675.43267769,238.84749204)(675.36767775,238.83249205)(675.27768173,238.82249178)
\curveto(675.19767792,238.80249208)(675.10767801,238.79249209)(675.00768173,238.79249178)
\curveto(674.90767821,238.79249209)(674.81267831,238.79749209)(674.72268173,238.80749178)
\curveto(674.64267848,238.81749207)(674.58267854,238.83749205)(674.54268173,238.86749178)
\curveto(674.51267861,238.887492)(674.48767863,238.92249196)(674.46768173,238.97249178)
\curveto(674.45767866,239.01249187)(674.46267866,239.05749183)(674.48268173,239.10749178)
\curveto(674.5226786,239.1874917)(674.56767855,239.26249162)(674.61768173,239.33249178)
\curveto(674.67767844,239.41249147)(674.73267839,239.49249139)(674.78268173,239.57249178)
\curveto(675.0226781,239.91249097)(675.26767785,240.24749064)(675.51768173,240.57749178)
\curveto(675.76767735,240.90748998)(676.00767711,241.24248964)(676.23768173,241.58249178)
\curveto(676.39767672,241.80248908)(676.55767656,242.01748887)(676.71768173,242.22749178)
\curveto(676.87767624,242.43748845)(677.03767608,242.65248823)(677.19768173,242.87249178)
\curveto(677.55767556,243.39248749)(677.9226752,243.90248698)(678.29268173,244.40249178)
\curveto(678.66267446,244.90248598)(679.03267409,245.41248547)(679.40268173,245.93249178)
\curveto(679.54267358,246.13248475)(679.68267344,246.32748456)(679.82268173,246.51749178)
\curveto(679.97267315,246.70748418)(680.117673,246.90248398)(680.25768173,247.10249178)
\curveto(680.46767265,247.40248348)(680.68267244,247.70248318)(680.90268173,248.00249178)
\lineto(681.56268173,248.90249178)
\lineto(681.74268173,249.17249178)
\lineto(681.95268173,249.44249178)
\lineto(682.07268173,249.62249178)
\curveto(682.122671,249.6824812)(682.17267095,249.73748115)(682.22268173,249.78749178)
\curveto(682.29267083,249.83748105)(682.36767075,249.87248101)(682.44768173,249.89249178)
\curveto(682.46767065,249.90248098)(682.49267063,249.90248098)(682.52268173,249.89249178)
\curveto(682.56267056,249.89248099)(682.59267053,249.90248098)(682.61268173,249.92249178)
\curveto(682.73267039,249.92248096)(682.86767025,249.91748097)(683.01768173,249.90749178)
\curveto(683.16766995,249.90748098)(683.25766986,249.86248102)(683.28768173,249.77249178)
\curveto(683.30766981,249.74248114)(683.31266981,249.70748118)(683.30268173,249.66749178)
\curveto(683.29266983,249.62748126)(683.27766984,249.59748129)(683.25768173,249.57749178)
\curveto(683.2176699,249.49748139)(683.17766994,249.42748146)(683.13768173,249.36749178)
\curveto(683.09767002,249.30748158)(683.05267007,249.24748164)(683.00268173,249.18749178)
\lineto(682.43268173,248.40749178)
\curveto(682.25267087,248.15748273)(682.07267105,247.90248298)(681.89268173,247.64249178)
\moveto(675.03768173,243.74249178)
\curveto(674.98767813,243.76248712)(674.93767818,243.76748712)(674.88768173,243.75749178)
\curveto(674.83767828,243.74748714)(674.78767833,243.75248713)(674.73768173,243.77249178)
\curveto(674.62767849,243.79248709)(674.5226786,243.81248707)(674.42268173,243.83249178)
\curveto(674.33267879,243.86248702)(674.23767888,243.90248698)(674.13768173,243.95249178)
\curveto(673.80767931,244.09248679)(673.55267957,244.2874866)(673.37268173,244.53749178)
\curveto(673.19267993,244.79748609)(673.04768007,245.10748578)(672.93768173,245.46749178)
\curveto(672.90768021,245.54748534)(672.88768023,245.62748526)(672.87768173,245.70749178)
\curveto(672.86768025,245.79748509)(672.85268027,245.882485)(672.83268173,245.96249178)
\curveto(672.8226803,246.01248487)(672.8176803,246.07748481)(672.81768173,246.15749178)
\curveto(672.80768031,246.1874847)(672.80268032,246.21748467)(672.80268173,246.24749178)
\curveto(672.80268032,246.2874846)(672.79768032,246.32248456)(672.78768173,246.35249178)
\lineto(672.78768173,246.50249178)
\curveto(672.77768034,246.55248433)(672.77268035,246.61248427)(672.77268173,246.68249178)
\curveto(672.77268035,246.76248412)(672.77768034,246.82748406)(672.78768173,246.87749178)
\lineto(672.78768173,247.04249178)
\curveto(672.80768031,247.09248379)(672.81268031,247.13748375)(672.80268173,247.17749178)
\curveto(672.80268032,247.22748366)(672.80768031,247.27248361)(672.81768173,247.31249178)
\curveto(672.82768029,247.35248353)(672.83268029,247.3874835)(672.83268173,247.41749178)
\curveto(672.83268029,247.45748343)(672.83768028,247.49748339)(672.84768173,247.53749178)
\curveto(672.87768024,247.64748324)(672.89768022,247.75748313)(672.90768173,247.86749178)
\curveto(672.92768019,247.9874829)(672.96268016,248.10248278)(673.01268173,248.21249178)
\curveto(673.15267997,248.55248233)(673.31267981,248.82748206)(673.49268173,249.03749178)
\curveto(673.68267944,249.25748163)(673.95267917,249.43748145)(674.30268173,249.57749178)
\curveto(674.38267874,249.60748128)(674.46767865,249.62748126)(674.55768173,249.63749178)
\curveto(674.64767847,249.65748123)(674.74267838,249.67748121)(674.84268173,249.69749178)
\curveto(674.87267825,249.70748118)(674.92767819,249.70748118)(675.00768173,249.69749178)
\curveto(675.08767803,249.69748119)(675.13767798,249.70748118)(675.15768173,249.72749178)
\curveto(675.7176774,249.73748115)(676.16767695,249.62748126)(676.50768173,249.39749178)
\curveto(676.85767626,249.16748172)(677.117676,248.86248202)(677.28768173,248.48249178)
\curveto(677.32767579,248.39248249)(677.36267576,248.29748259)(677.39268173,248.19749178)
\curveto(677.4226757,248.09748279)(677.44767567,247.99748289)(677.46768173,247.89749178)
\curveto(677.48767563,247.86748302)(677.49267563,247.83748305)(677.48268173,247.80749178)
\curveto(677.48267564,247.77748311)(677.48767563,247.74748314)(677.49768173,247.71749178)
\curveto(677.52767559,247.60748328)(677.54767557,247.4824834)(677.55768173,247.34249178)
\curveto(677.56767555,247.21248367)(677.57767554,247.07748381)(677.58768173,246.93749178)
\lineto(677.58768173,246.77249178)
\curveto(677.59767552,246.71248417)(677.59767552,246.65748423)(677.58768173,246.60749178)
\curveto(677.57767554,246.55748433)(677.57267555,246.50748438)(677.57268173,246.45749178)
\lineto(677.57268173,246.32249178)
\curveto(677.56267556,246.2824846)(677.55767556,246.24248464)(677.55768173,246.20249178)
\curveto(677.56767555,246.16248472)(677.56267556,246.11748477)(677.54268173,246.06749178)
\curveto(677.5226756,245.95748493)(677.50267562,245.85248503)(677.48268173,245.75249178)
\curveto(677.47267565,245.65248523)(677.45267567,245.55248533)(677.42268173,245.45249178)
\curveto(677.29267583,245.09248579)(677.12767599,244.77748611)(676.92768173,244.50749178)
\curveto(676.72767639,244.23748665)(676.45267667,244.03248685)(676.10268173,243.89249178)
\curveto(676.0226771,243.86248702)(675.93767718,243.83748705)(675.84768173,243.81749178)
\lineto(675.57768173,243.75749178)
\curveto(675.52767759,243.74748714)(675.48267764,243.74248714)(675.44268173,243.74249178)
\curveto(675.40267772,243.75248713)(675.36267776,243.75248713)(675.32268173,243.74249178)
\curveto(675.2226779,243.72248716)(675.12767799,243.72248716)(675.03768173,243.74249178)
\moveto(674.19768173,245.13749178)
\curveto(674.23767888,245.06748582)(674.27767884,245.00248588)(674.31768173,244.94249178)
\curveto(674.35767876,244.89248599)(674.40767871,244.84248604)(674.46768173,244.79249178)
\lineto(674.61768173,244.67249178)
\curveto(674.67767844,244.64248624)(674.74267838,244.61748627)(674.81268173,244.59749178)
\curveto(674.85267827,244.57748631)(674.88767823,244.56748632)(674.91768173,244.56749178)
\curveto(674.95767816,244.57748631)(674.99767812,244.57248631)(675.03768173,244.55249178)
\curveto(675.06767805,244.55248633)(675.10767801,244.54748634)(675.15768173,244.53749178)
\curveto(675.20767791,244.53748635)(675.24767787,244.54248634)(675.27768173,244.55249178)
\lineto(675.50268173,244.59749178)
\curveto(675.75267737,244.67748621)(675.93767718,244.80248608)(676.05768173,244.97249178)
\curveto(676.13767698,245.07248581)(676.20767691,245.20248568)(676.26768173,245.36249178)
\curveto(676.34767677,245.54248534)(676.40767671,245.76748512)(676.44768173,246.03749178)
\curveto(676.48767663,246.31748457)(676.50267662,246.59748429)(676.49268173,246.87749178)
\curveto(676.48267664,247.16748372)(676.45267667,247.44248344)(676.40268173,247.70249178)
\curveto(676.35267677,247.96248292)(676.27767684,248.17248271)(676.17768173,248.33249178)
\curveto(676.05767706,248.53248235)(675.90767721,248.6824822)(675.72768173,248.78249178)
\curveto(675.64767747,248.83248205)(675.55767756,248.86248202)(675.45768173,248.87249178)
\curveto(675.35767776,248.89248199)(675.25267787,248.90248198)(675.14268173,248.90249178)
\curveto(675.122678,248.89248199)(675.09767802,248.887482)(675.06768173,248.88749178)
\curveto(675.04767807,248.89748199)(675.02767809,248.89748199)(675.00768173,248.88749178)
\curveto(674.95767816,248.87748201)(674.91267821,248.86748202)(674.87268173,248.85749178)
\curveto(674.83267829,248.85748203)(674.79267833,248.84748204)(674.75268173,248.82749178)
\curveto(674.57267855,248.74748214)(674.4226787,248.62748226)(674.30268173,248.46749178)
\curveto(674.19267893,248.30748258)(674.10267902,248.12748276)(674.03268173,247.92749178)
\curveto(673.97267915,247.73748315)(673.92767919,247.51248337)(673.89768173,247.25249178)
\curveto(673.87767924,246.99248389)(673.87267925,246.72748416)(673.88268173,246.45749178)
\curveto(673.89267923,246.19748469)(673.9226792,245.94748494)(673.97268173,245.70749178)
\curveto(674.03267909,245.47748541)(674.10767901,245.2874856)(674.19768173,245.13749178)
\moveto(684.99768173,242.15249178)
\curveto(685.00766811,242.10248878)(685.01266811,242.01248887)(685.01268173,241.88249178)
\curveto(685.01266811,241.75248913)(685.00266812,241.66248922)(684.98268173,241.61249178)
\curveto(684.96266816,241.56248932)(684.95766816,241.50748938)(684.96768173,241.44749178)
\curveto(684.97766814,241.39748949)(684.97766814,241.34748954)(684.96768173,241.29749178)
\curveto(684.92766819,241.15748973)(684.89766822,241.02248986)(684.87768173,240.89249178)
\curveto(684.86766825,240.76249012)(684.83766828,240.64249024)(684.78768173,240.53249178)
\curveto(684.64766847,240.1824907)(684.48266864,239.887491)(684.29268173,239.64749178)
\curveto(684.10266902,239.41749147)(683.83266929,239.23249165)(683.48268173,239.09249178)
\curveto(683.40266972,239.06249182)(683.3176698,239.04249184)(683.22768173,239.03249178)
\curveto(683.13766998,239.01249187)(683.05267007,238.99249189)(682.97268173,238.97249178)
\curveto(682.9226702,238.96249192)(682.87267025,238.95749193)(682.82268173,238.95749178)
\curveto(682.77267035,238.95749193)(682.7226704,238.95249193)(682.67268173,238.94249178)
\curveto(682.64267048,238.93249195)(682.59267053,238.93249195)(682.52268173,238.94249178)
\curveto(682.45267067,238.94249194)(682.40267072,238.94749194)(682.37268173,238.95749178)
\curveto(682.31267081,238.97749191)(682.25267087,238.9874919)(682.19268173,238.98749178)
\curveto(682.14267098,238.97749191)(682.09267103,238.9824919)(682.04268173,239.00249178)
\curveto(681.95267117,239.02249186)(681.86267126,239.04749184)(681.77268173,239.07749178)
\curveto(681.69267143,239.09749179)(681.61267151,239.12749176)(681.53268173,239.16749178)
\curveto(681.21267191,239.30749158)(680.96267216,239.50249138)(680.78268173,239.75249178)
\curveto(680.60267252,240.01249087)(680.45267267,240.31749057)(680.33268173,240.66749178)
\curveto(680.31267281,240.74749014)(680.29767282,240.83249005)(680.28768173,240.92249178)
\curveto(680.27767284,241.01248987)(680.26267286,241.09748979)(680.24268173,241.17749178)
\curveto(680.23267289,241.20748968)(680.22767289,241.23748965)(680.22768173,241.26749178)
\lineto(680.22768173,241.37249178)
\curveto(680.20767291,241.45248943)(680.19767292,241.53248935)(680.19768173,241.61249178)
\lineto(680.19768173,241.74749178)
\curveto(680.17767294,241.84748904)(680.17767294,241.94748894)(680.19768173,242.04749178)
\lineto(680.19768173,242.22749178)
\curveto(680.20767291,242.27748861)(680.21267291,242.32248856)(680.21268173,242.36249178)
\curveto(680.21267291,242.41248847)(680.2176729,242.45748843)(680.22768173,242.49749178)
\curveto(680.23767288,242.53748835)(680.24267288,242.57248831)(680.24268173,242.60249178)
\curveto(680.24267288,242.64248824)(680.24767287,242.6824882)(680.25768173,242.72249178)
\lineto(680.31768173,243.05249178)
\curveto(680.33767278,243.17248771)(680.36767275,243.2824876)(680.40768173,243.38249178)
\curveto(680.54767257,243.71248717)(680.70767241,243.9874869)(680.88768173,244.20749178)
\curveto(681.07767204,244.43748645)(681.33767178,244.62248626)(681.66768173,244.76249178)
\curveto(681.74767137,244.80248608)(681.83267129,244.82748606)(681.92268173,244.83749178)
\lineto(682.22268173,244.89749178)
\lineto(682.35768173,244.89749178)
\curveto(682.40767071,244.90748598)(682.45767066,244.91248597)(682.50768173,244.91249178)
\curveto(683.07767004,244.93248595)(683.53766958,244.82748606)(683.88768173,244.59749178)
\curveto(684.24766887,244.37748651)(684.51266861,244.07748681)(684.68268173,243.69749178)
\curveto(684.73266839,243.59748729)(684.77266835,243.49748739)(684.80268173,243.39749178)
\curveto(684.83266829,243.29748759)(684.86266826,243.19248769)(684.89268173,243.08249178)
\curveto(684.90266822,243.04248784)(684.90766821,243.00748788)(684.90768173,242.97749178)
\curveto(684.90766821,242.95748793)(684.91266821,242.92748796)(684.92268173,242.88749178)
\curveto(684.94266818,242.81748807)(684.95266817,242.74248814)(684.95268173,242.66249178)
\curveto(684.95266817,242.5824883)(684.96266816,242.50248838)(684.98268173,242.42249178)
\curveto(684.98266814,242.37248851)(684.98266814,242.32748856)(684.98268173,242.28749178)
\curveto(684.98266814,242.24748864)(684.98766813,242.20248868)(684.99768173,242.15249178)
\moveto(683.88768173,241.71749178)
\curveto(683.89766922,241.76748912)(683.90266922,241.84248904)(683.90268173,241.94249178)
\curveto(683.91266921,242.04248884)(683.90766921,242.11748877)(683.88768173,242.16749178)
\curveto(683.86766925,242.22748866)(683.86266926,242.2824886)(683.87268173,242.33249178)
\curveto(683.89266923,242.39248849)(683.89266923,242.45248843)(683.87268173,242.51249178)
\curveto(683.86266926,242.54248834)(683.85766926,242.57748831)(683.85768173,242.61749178)
\curveto(683.85766926,242.65748823)(683.85266927,242.69748819)(683.84268173,242.73749178)
\curveto(683.8226693,242.81748807)(683.80266932,242.89248799)(683.78268173,242.96249178)
\curveto(683.77266935,243.04248784)(683.75766936,243.12248776)(683.73768173,243.20249178)
\curveto(683.70766941,243.26248762)(683.68266944,243.32248756)(683.66268173,243.38249178)
\curveto(683.64266948,243.44248744)(683.61266951,243.50248738)(683.57268173,243.56249178)
\curveto(683.47266965,243.73248715)(683.34266978,243.86748702)(683.18268173,243.96749178)
\curveto(683.10267002,244.01748687)(683.00767011,244.05248683)(682.89768173,244.07249178)
\curveto(682.78767033,244.09248679)(682.66267046,244.10248678)(682.52268173,244.10249178)
\curveto(682.50267062,244.09248679)(682.47767064,244.0874868)(682.44768173,244.08749178)
\curveto(682.4176707,244.09748679)(682.38767073,244.09748679)(682.35768173,244.08749178)
\lineto(682.20768173,244.02749178)
\curveto(682.15767096,244.01748687)(682.11267101,244.00248688)(682.07268173,243.98249178)
\curveto(681.88267124,243.87248701)(681.73767138,243.72748716)(681.63768173,243.54749178)
\curveto(681.54767157,243.36748752)(681.46767165,243.16248772)(681.39768173,242.93249178)
\curveto(681.35767176,242.80248808)(681.33767178,242.66748822)(681.33768173,242.52749178)
\curveto(681.33767178,242.39748849)(681.32767179,242.25248863)(681.30768173,242.09249178)
\curveto(681.29767182,242.04248884)(681.28767183,241.9824889)(681.27768173,241.91249178)
\curveto(681.27767184,241.84248904)(681.28767183,241.7824891)(681.30768173,241.73249178)
\lineto(681.30768173,241.56749178)
\lineto(681.30768173,241.38749178)
\curveto(681.3176718,241.33748955)(681.32767179,241.2824896)(681.33768173,241.22249178)
\curveto(681.34767177,241.17248971)(681.35267177,241.11748977)(681.35268173,241.05749178)
\curveto(681.36267176,240.99748989)(681.37767174,240.94248994)(681.39768173,240.89249178)
\curveto(681.44767167,240.70249018)(681.50767161,240.52749036)(681.57768173,240.36749178)
\curveto(681.64767147,240.20749068)(681.75267137,240.07749081)(681.89268173,239.97749178)
\curveto(682.0226711,239.87749101)(682.16267096,239.80749108)(682.31268173,239.76749178)
\curveto(682.34267078,239.75749113)(682.36767075,239.75249113)(682.38768173,239.75249178)
\curveto(682.4176707,239.76249112)(682.44767067,239.76249112)(682.47768173,239.75249178)
\curveto(682.49767062,239.75249113)(682.52767059,239.74749114)(682.56768173,239.73749178)
\curveto(682.60767051,239.73749115)(682.64267048,239.74249114)(682.67268173,239.75249178)
\curveto(682.71267041,239.76249112)(682.75267037,239.76749112)(682.79268173,239.76749178)
\curveto(682.83267029,239.76749112)(682.87267025,239.77749111)(682.91268173,239.79749178)
\curveto(683.15266997,239.87749101)(683.34766977,240.01249087)(683.49768173,240.20249178)
\curveto(683.6176695,240.3824905)(683.70766941,240.5874903)(683.76768173,240.81749178)
\curveto(683.78766933,240.88749)(683.80266932,240.95748993)(683.81268173,241.02749178)
\curveto(683.8226693,241.10748978)(683.83766928,241.1874897)(683.85768173,241.26749178)
\curveto(683.85766926,241.32748956)(683.86266926,241.37248951)(683.87268173,241.40249178)
\curveto(683.87266925,241.42248946)(683.87266925,241.44748944)(683.87268173,241.47749178)
\curveto(683.87266925,241.51748937)(683.87766924,241.54748934)(683.88768173,241.56749178)
\lineto(683.88768173,241.71749178)
}
}
{
\newrgbcolor{curcolor}{0 0 0}
\pscustom[linestyle=none,fillstyle=solid,fillcolor=curcolor]
{
\newpath
\moveto(631.78641464,76.48634187)
\curveto(631.88640978,76.48633125)(631.98140969,76.47633126)(632.07141464,76.45634188)
\curveto(632.16140951,76.44633129)(632.22640944,76.41633132)(632.26641464,76.36634187)
\curveto(632.32640934,76.28633145)(632.35640931,76.18133156)(632.35641464,76.05134187)
\lineto(632.35641464,75.66134187)
\lineto(632.35641464,74.16134187)
\lineto(632.35641464,67.77134187)
\lineto(632.35641464,66.60134188)
\lineto(632.35641464,66.28634187)
\curveto(632.3664093,66.18634156)(632.35140932,66.10634164)(632.31141464,66.04634187)
\curveto(632.26140941,65.96634177)(632.18640948,65.91634183)(632.08641464,65.89634187)
\curveto(631.99640967,65.88634185)(631.88640978,65.88134186)(631.75641464,65.88134187)
\lineto(631.53141464,65.88134187)
\curveto(631.45141022,65.90134184)(631.38141029,65.91634183)(631.32141464,65.92634187)
\curveto(631.26141041,65.94634179)(631.21141046,65.98634176)(631.17141464,66.04634187)
\curveto(631.13141054,66.10634164)(631.11141056,66.18134156)(631.11141464,66.27134187)
\lineto(631.11141464,66.57134188)
\lineto(631.11141464,67.66634187)
\lineto(631.11141464,73.00634187)
\curveto(631.09141058,73.09633464)(631.07641059,73.17133457)(631.06641464,73.23134187)
\curveto(631.0664106,73.30133444)(631.03641063,73.36133438)(630.97641464,73.41134187)
\curveto(630.90641076,73.46133428)(630.81641085,73.48633425)(630.70641464,73.48634187)
\curveto(630.60641106,73.49633424)(630.49641117,73.50133424)(630.37641464,73.50134187)
\lineto(629.23641464,73.50134187)
\lineto(628.74141464,73.50134187)
\curveto(628.58141309,73.51133423)(628.4714132,73.57133417)(628.41141464,73.68134188)
\curveto(628.39141328,73.71133403)(628.38141329,73.741334)(628.38141464,73.77134187)
\curveto(628.38141329,73.81133393)(628.37641329,73.85633389)(628.36641464,73.90634187)
\curveto(628.34641332,74.02633371)(628.35141332,74.13633361)(628.38141464,74.23634187)
\curveto(628.42141325,74.33633341)(628.47641319,74.40633333)(628.54641464,74.44634188)
\curveto(628.62641304,74.49633324)(628.74641292,74.52133322)(628.90641464,74.52134187)
\curveto(629.0664126,74.52133322)(629.20141247,74.53633321)(629.31141464,74.56634188)
\curveto(629.36141231,74.57633316)(629.41641225,74.58133316)(629.47641464,74.58134188)
\curveto(629.53641213,74.59133315)(629.59641207,74.60633313)(629.65641464,74.62634187)
\curveto(629.80641186,74.67633306)(629.95141172,74.72633302)(630.09141464,74.77634187)
\curveto(630.23141144,74.8363329)(630.3664113,74.90633283)(630.49641464,74.98634187)
\curveto(630.63641103,75.07633266)(630.75641091,75.18133256)(630.85641464,75.30134187)
\curveto(630.95641071,75.42133232)(631.05141062,75.55133219)(631.14141464,75.69134188)
\curveto(631.20141047,75.79133195)(631.24641042,75.90133184)(631.27641464,76.02134187)
\curveto(631.31641035,76.1413316)(631.3664103,76.24633149)(631.42641464,76.33634188)
\curveto(631.47641019,76.39633135)(631.54641012,76.4363313)(631.63641464,76.45634188)
\curveto(631.65641001,76.46633128)(631.68140999,76.47133127)(631.71141464,76.47134188)
\curveto(631.74140993,76.47133127)(631.7664099,76.47633126)(631.78641464,76.48634187)
}
}
{
\newrgbcolor{curcolor}{0 0 0}
\pscustom[linestyle=none,fillstyle=solid,fillcolor=curcolor]
{
\newpath
\moveto(643.01602401,71.47634188)
\curveto(643.01601638,71.39633635)(643.02101637,71.31633643)(643.03102401,71.23634187)
\curveto(643.04101635,71.15633658)(643.03601636,71.08133666)(643.01602401,71.01134187)
\curveto(642.9960164,70.97133677)(642.9910164,70.92633682)(643.00102401,70.87634187)
\curveto(643.01101638,70.8363369)(643.01101638,70.79633695)(643.00102401,70.75634187)
\lineto(643.00102401,70.60634187)
\curveto(642.9910164,70.51633723)(642.98601641,70.42633731)(642.98602401,70.33634188)
\curveto(642.98601641,70.25633749)(642.98101641,70.17633757)(642.97102401,70.09634188)
\lineto(642.94102401,69.85634187)
\curveto(642.93101646,69.78633796)(642.92101647,69.71133803)(642.91102401,69.63134187)
\curveto(642.90101649,69.59133815)(642.8960165,69.55133819)(642.89602401,69.51134187)
\curveto(642.8960165,69.47133827)(642.8910165,69.42633831)(642.88102401,69.37634187)
\curveto(642.84101655,69.2363385)(642.81101658,69.09633864)(642.79102401,68.95634188)
\curveto(642.78101661,68.81633892)(642.75101664,68.68133906)(642.70102401,68.55134187)
\curveto(642.65101674,68.38133936)(642.5960168,68.21633952)(642.53602401,68.05634188)
\curveto(642.48601691,67.89633984)(642.42601697,67.74134)(642.35602401,67.59134188)
\curveto(642.33601706,67.53134021)(642.30601709,67.47134027)(642.26602401,67.41134187)
\lineto(642.17602401,67.26134187)
\curveto(641.97601742,66.9413408)(641.76101763,66.67634106)(641.53102401,66.46634188)
\curveto(641.30101809,66.25634149)(641.00601839,66.07634166)(640.64602401,65.92634187)
\curveto(640.52601887,65.87634186)(640.396019,65.8413419)(640.25602401,65.82134188)
\curveto(640.12601927,65.80134194)(639.9910194,65.77634197)(639.85102401,65.74634187)
\curveto(639.7910196,65.736342)(639.73101966,65.73134201)(639.67102401,65.73134187)
\curveto(639.61101978,65.73134201)(639.54601985,65.72634202)(639.47602401,65.71634188)
\curveto(639.44601995,65.70634204)(639.39602,65.70634204)(639.32602401,65.71634188)
\lineto(639.17602401,65.71634188)
\lineto(639.02602401,65.71634188)
\curveto(638.94602045,65.736342)(638.86102053,65.75134199)(638.77102401,65.76134187)
\curveto(638.6910207,65.76134198)(638.61602078,65.77134197)(638.54602401,65.79134187)
\curveto(638.50602089,65.80134194)(638.47102092,65.80634193)(638.44102401,65.80634188)
\curveto(638.42102097,65.79634194)(638.396021,65.80134194)(638.36602401,65.82134188)
\lineto(638.09602401,65.88134187)
\curveto(638.00602139,65.91134183)(637.92102147,65.9413418)(637.84102401,65.97134188)
\curveto(637.26102213,66.21134153)(636.82602257,66.58134116)(636.53602401,67.08134188)
\curveto(636.45602294,67.21134053)(636.391023,67.34634039)(636.34102401,67.48634187)
\curveto(636.30102309,67.62634011)(636.25602314,67.77633997)(636.20602401,67.93634188)
\curveto(636.18602321,68.01633972)(636.18102321,68.09633964)(636.19102401,68.17634187)
\curveto(636.21102318,68.25633949)(636.24602315,68.31133943)(636.29602401,68.34134188)
\curveto(636.32602307,68.36133938)(636.38102301,68.37633937)(636.46102401,68.38634187)
\curveto(636.54102285,68.40633933)(636.62602277,68.41633932)(636.71602401,68.41634187)
\curveto(636.80602259,68.42633931)(636.8910225,68.42633931)(636.97102401,68.41634187)
\curveto(637.06102233,68.40633933)(637.13102226,68.39633935)(637.18102401,68.38634187)
\curveto(637.20102219,68.37633937)(637.22602217,68.36133938)(637.25602401,68.34134188)
\curveto(637.2960221,68.32133942)(637.32602207,68.30133944)(637.34602401,68.28134187)
\curveto(637.40602199,68.20133954)(637.45102194,68.10633964)(637.48102401,67.99634187)
\curveto(637.52102187,67.88633985)(637.56602183,67.78633996)(637.61602401,67.69634188)
\curveto(637.86602153,67.30634044)(638.23602116,67.03634071)(638.72602401,66.88634187)
\curveto(638.7960206,66.86634087)(638.86602053,66.85134089)(638.93602401,66.84134188)
\curveto(639.01602038,66.8413409)(639.0960203,66.83134091)(639.17602401,66.81134188)
\curveto(639.21602018,66.80134094)(639.27102012,66.79634094)(639.34102401,66.79634187)
\curveto(639.42101997,66.79634094)(639.47601992,66.80134094)(639.50602401,66.81134188)
\curveto(639.53601986,66.82134092)(639.56601983,66.82634091)(639.59602401,66.82634188)
\lineto(639.70102401,66.82634188)
\curveto(639.78101961,66.8463409)(639.85601954,66.86634087)(639.92602401,66.88634187)
\curveto(640.00601939,66.90634084)(640.08101931,66.93134081)(640.15102401,66.96134188)
\curveto(640.50101889,67.11134063)(640.77101862,67.32634042)(640.96102401,67.60634187)
\curveto(641.15101824,67.88633985)(641.30601809,68.21133953)(641.42602401,68.58134188)
\curveto(641.45601794,68.66133908)(641.47601792,68.736339)(641.48602401,68.80634188)
\curveto(641.50601789,68.87633886)(641.52601787,68.95133879)(641.54602401,69.03134187)
\curveto(641.56601783,69.12133862)(641.58101781,69.21633852)(641.59102401,69.31634188)
\curveto(641.61101778,69.42633831)(641.63101776,69.53133821)(641.65102401,69.63134187)
\curveto(641.66101773,69.68133806)(641.66601773,69.73133801)(641.66602401,69.78134187)
\curveto(641.67601772,69.8413379)(641.68101771,69.89633784)(641.68102401,69.94634188)
\curveto(641.70101769,70.00633773)(641.71101768,70.08133766)(641.71102401,70.17134187)
\curveto(641.71101768,70.27133747)(641.70101769,70.35133739)(641.68102401,70.41134187)
\curveto(641.65101774,70.50133724)(641.60101779,70.5413372)(641.53102401,70.53134187)
\curveto(641.47101792,70.52133722)(641.41601798,70.49133725)(641.36602401,70.44134188)
\curveto(641.28601811,70.39133735)(641.21601818,70.33133741)(641.15602401,70.26134187)
\curveto(641.10601829,70.19133755)(641.04101835,70.13133761)(640.96102401,70.08134188)
\curveto(640.80101859,69.97133777)(640.63601876,69.87133787)(640.46602401,69.78134187)
\curveto(640.2960191,69.70133804)(640.10101929,69.63133811)(639.88102401,69.57134188)
\curveto(639.78101961,69.5413382)(639.68101971,69.52633822)(639.58102401,69.52634187)
\curveto(639.4910199,69.52633822)(639.39102,69.51633823)(639.28102401,69.49634187)
\lineto(639.13102401,69.49634187)
\curveto(639.08102031,69.51633823)(639.03102036,69.52133822)(638.98102401,69.51134187)
\curveto(638.94102045,69.50133824)(638.90102049,69.50133824)(638.86102401,69.51134187)
\curveto(638.83102056,69.52133822)(638.78602061,69.52633822)(638.72602401,69.52634187)
\curveto(638.66602073,69.5363382)(638.60102079,69.54633819)(638.53102401,69.55634188)
\lineto(638.35102401,69.58634188)
\curveto(637.90102149,69.70633804)(637.52102187,69.87133787)(637.21102401,70.08134188)
\curveto(636.94102245,70.27133747)(636.71102268,70.50133724)(636.52102401,70.77134187)
\curveto(636.34102305,71.05133669)(636.1960232,71.36633637)(636.08602401,71.71634188)
\lineto(636.02602401,71.92634187)
\curveto(636.01602338,72.00633573)(636.00102339,72.08633565)(635.98102401,72.16634187)
\curveto(635.97102342,72.19633555)(635.96602343,72.22633551)(635.96602401,72.25634187)
\curveto(635.96602343,72.28633545)(635.96102343,72.31633543)(635.95102401,72.34634188)
\curveto(635.94102345,72.40633533)(635.93602346,72.46633528)(635.93602401,72.52634187)
\curveto(635.93602346,72.59633515)(635.92602347,72.65633509)(635.90602401,72.70634188)
\lineto(635.90602401,72.88634187)
\curveto(635.8960235,72.93633481)(635.8910235,73.00633473)(635.89102401,73.09634188)
\curveto(635.8910235,73.18633456)(635.90102349,73.25633449)(635.92102401,73.30634188)
\lineto(635.92102401,73.47134188)
\curveto(635.94102345,73.55133419)(635.95102344,73.62633411)(635.95102401,73.69634188)
\curveto(635.96102343,73.76633397)(635.97602342,73.8363339)(635.99602401,73.90634187)
\curveto(636.05602334,74.10633363)(636.11602328,74.29633344)(636.17602401,74.47634188)
\curveto(636.24602315,74.65633309)(636.33602306,74.82633291)(636.44602401,74.98634187)
\curveto(636.48602291,75.05633269)(636.52602287,75.12133262)(636.56602401,75.18134188)
\lineto(636.71602401,75.36134187)
\curveto(636.73602266,75.37133237)(636.75602264,75.38633236)(636.77602401,75.40634187)
\curveto(636.86602253,75.53633221)(636.97602242,75.64633209)(637.10602401,75.73634187)
\curveto(637.36602203,75.93633181)(637.63102176,76.09133165)(637.90102401,76.20134188)
\curveto(637.98102141,76.2413315)(638.06102133,76.27133147)(638.14102401,76.29134187)
\curveto(638.23102116,76.32133142)(638.32102107,76.34633139)(638.41102401,76.36634187)
\curveto(638.51102088,76.39633135)(638.61102078,76.41633132)(638.71102401,76.42634187)
\curveto(638.81102058,76.4363313)(638.91602048,76.45133129)(639.02602401,76.47134188)
\curveto(639.05602034,76.48133126)(639.0960203,76.48133126)(639.14602401,76.47134188)
\curveto(639.20602019,76.46133128)(639.24602015,76.46633128)(639.26602401,76.48634187)
\curveto(639.98601941,76.50633123)(640.58601881,76.39133135)(641.06602401,76.14134187)
\curveto(641.54601785,75.89133185)(641.92101747,75.55133219)(642.19102401,75.12134187)
\curveto(642.28101711,74.98133276)(642.36101703,74.8363329)(642.43102401,74.68634188)
\curveto(642.50101689,74.53633321)(642.57101682,74.37633336)(642.64102401,74.20634188)
\curveto(642.6910167,74.06633368)(642.73101666,73.91633383)(642.76102401,73.75634187)
\curveto(642.7910166,73.59633415)(642.82601657,73.4363343)(642.86602401,73.27634187)
\curveto(642.88601651,73.22633451)(642.8960165,73.17133457)(642.89602401,73.11134187)
\curveto(642.8960165,73.06133468)(642.90101649,73.01133473)(642.91102401,72.96134188)
\curveto(642.93101646,72.90133484)(642.94101645,72.8363349)(642.94102401,72.76634187)
\curveto(642.94101645,72.70633503)(642.95101644,72.65133509)(642.97102401,72.60134188)
\lineto(642.97102401,72.43634188)
\curveto(642.9910164,72.38633536)(642.9960164,72.3363354)(642.98602401,72.28634187)
\curveto(642.97601642,72.2363355)(642.98101641,72.18633556)(643.00102401,72.13634187)
\curveto(643.00101639,72.11633563)(642.9960164,72.09133565)(642.98602401,72.06134188)
\curveto(642.98601641,72.03133571)(642.9910164,72.00633573)(643.00102401,71.98634187)
\curveto(643.01101638,71.95633578)(643.01101638,71.92133582)(643.00102401,71.88134187)
\curveto(643.00101639,71.8413359)(643.00601639,71.80133594)(643.01602401,71.76134187)
\curveto(643.02601637,71.72133602)(643.02601637,71.67633606)(643.01602401,71.62634187)
\lineto(643.01602401,71.47634188)
\moveto(641.51602401,72.78134187)
\curveto(641.52601787,72.83133491)(641.53101786,72.89133485)(641.53102401,72.96134188)
\curveto(641.53101786,73.03133471)(641.52601787,73.09133465)(641.51602401,73.14134187)
\curveto(641.50601789,73.19133455)(641.50101789,73.26633448)(641.50102401,73.36634187)
\curveto(641.48101791,73.4463343)(641.46101793,73.52133422)(641.44102401,73.59134188)
\curveto(641.43101796,73.66133408)(641.41601798,73.73133401)(641.39602401,73.80134187)
\curveto(641.25601814,74.23133351)(641.06101833,74.56633317)(640.81102401,74.80634188)
\curveto(640.57101882,75.0463327)(640.22601917,75.22633251)(639.77602401,75.34634188)
\curveto(639.68601971,75.36633237)(639.58601981,75.37633236)(639.47602401,75.37634187)
\lineto(639.14602401,75.37634187)
\curveto(639.12602027,75.35633238)(639.0910203,75.34633239)(639.04102401,75.34634188)
\curveto(638.9910204,75.35633238)(638.94602045,75.35633238)(638.90602401,75.34634188)
\curveto(638.82602057,75.32633242)(638.75102064,75.30633243)(638.68102401,75.28634187)
\lineto(638.47102401,75.22634188)
\curveto(638.18102121,75.09633264)(637.95102144,74.91633283)(637.78102401,74.68634188)
\curveto(637.61102178,74.46633328)(637.47602192,74.20633353)(637.37602401,73.90634187)
\curveto(637.34602205,73.81633392)(637.32102207,73.72133402)(637.30102401,73.62134187)
\curveto(637.2910221,73.53133421)(637.27602212,73.4363343)(637.25602401,73.33634188)
\lineto(637.25602401,73.20134188)
\curveto(637.22602217,73.09133465)(637.21602218,72.95133479)(637.22602401,72.78134187)
\curveto(637.24602215,72.62133512)(637.26602213,72.49133525)(637.28602401,72.39134187)
\curveto(637.30602209,72.33133541)(637.32102207,72.27133547)(637.33102401,72.21134188)
\curveto(637.34102205,72.16133558)(637.35602204,72.11133563)(637.37602401,72.06134188)
\curveto(637.45602194,71.86133588)(637.55102184,71.67133607)(637.66102401,71.49134187)
\curveto(637.78102161,71.31133643)(637.92102147,71.16633657)(638.08102401,71.05634188)
\curveto(638.13102126,71.00633673)(638.18602121,70.96633677)(638.24602401,70.93634188)
\curveto(638.30602109,70.90633683)(638.36602103,70.87133687)(638.42602401,70.83134188)
\curveto(638.57602082,70.75133699)(638.76102063,70.68633705)(638.98102401,70.63634187)
\curveto(639.03102036,70.61633712)(639.07102032,70.61133713)(639.10102401,70.62134187)
\curveto(639.14102025,70.63133711)(639.18602021,70.62633711)(639.23602401,70.60634187)
\curveto(639.27602012,70.59633715)(639.33102006,70.59133715)(639.40102401,70.59134188)
\curveto(639.47101992,70.59133715)(639.53101986,70.59633715)(639.58102401,70.60634187)
\curveto(639.68101971,70.62633711)(639.77601962,70.6413371)(639.86602401,70.65134187)
\curveto(639.95601944,70.67133707)(640.04601935,70.70133704)(640.13602401,70.74134187)
\curveto(640.67601872,70.96133678)(641.07101832,71.35633638)(641.32102401,71.92634187)
\curveto(641.37101802,72.02633571)(641.40601799,72.12633562)(641.42602401,72.22634188)
\curveto(641.44601795,72.3363354)(641.47101792,72.4463353)(641.50102401,72.55634188)
\curveto(641.50101789,72.65633509)(641.50601789,72.73133501)(641.51602401,72.78134187)
}
}
{
\newrgbcolor{curcolor}{0 0 0}
\pscustom[linestyle=none,fillstyle=solid,fillcolor=curcolor]
{
\newpath
\moveto(645.38063339,67.51634187)
\lineto(645.68063339,67.51634187)
\curveto(645.79063133,67.52634022)(645.89563122,67.52634022)(645.99563339,67.51634187)
\curveto(646.10563101,67.51634023)(646.20563091,67.50634024)(646.29563339,67.48634187)
\curveto(646.38563073,67.47634026)(646.45563066,67.45134029)(646.50563339,67.41134187)
\curveto(646.52563059,67.39134035)(646.54063058,67.36134038)(646.55063339,67.32134188)
\curveto(646.57063055,67.28134046)(646.59063053,67.23634051)(646.61063339,67.18634188)
\lineto(646.61063339,67.11134187)
\curveto(646.6206305,67.06134068)(646.6206305,67.00634073)(646.61063339,66.94634188)
\lineto(646.61063339,66.79634187)
\lineto(646.61063339,66.31634188)
\curveto(646.61063051,66.14634159)(646.57063055,66.02634171)(646.49063339,65.95634188)
\curveto(646.4206307,65.90634184)(646.33063079,65.88134186)(646.22063339,65.88134187)
\lineto(645.89063339,65.88134187)
\lineto(645.44063339,65.88134187)
\curveto(645.29063183,65.88134186)(645.17563194,65.91134183)(645.09563339,65.97134188)
\curveto(645.05563206,66.00134174)(645.02563209,66.05134169)(645.00563339,66.12134187)
\curveto(644.98563213,66.20134154)(644.97063215,66.28634145)(644.96063339,66.37634187)
\lineto(644.96063339,66.66134187)
\curveto(644.97063215,66.76134098)(644.97563214,66.8463409)(644.97563339,66.91634187)
\lineto(644.97563339,67.11134187)
\curveto(644.97563214,67.17134057)(644.98563213,67.22634051)(645.00563339,67.27634187)
\curveto(645.04563207,67.38634036)(645.115632,67.45634029)(645.21563339,67.48634187)
\curveto(645.24563187,67.48634025)(645.30063182,67.49634024)(645.38063339,67.51634187)
}
}
{
\newrgbcolor{curcolor}{0 0 0}
\pscustom[linestyle=none,fillstyle=solid,fillcolor=curcolor]
{
\newpath
\moveto(649.04578964,76.29134187)
\lineto(653.84578964,76.29134187)
\lineto(654.85078964,76.29134187)
\curveto(654.99078254,76.29133145)(655.11078242,76.28133146)(655.21078964,76.26134187)
\curveto(655.32078221,76.25133149)(655.40078213,76.20633154)(655.45078964,76.12634187)
\curveto(655.47078206,76.08633165)(655.48078205,76.0363317)(655.48078964,75.97634188)
\curveto(655.49078204,75.91633183)(655.49578203,75.85133189)(655.49578964,75.78134187)
\lineto(655.49578964,75.51134187)
\curveto(655.49578203,75.42133232)(655.48578204,75.3413324)(655.46578964,75.27134187)
\curveto(655.4257821,75.19133255)(655.38078215,75.12133262)(655.33078964,75.06134188)
\lineto(655.18078964,74.88134187)
\curveto(655.15078238,74.83133291)(655.11578241,74.79133295)(655.07578964,74.76134187)
\curveto(655.03578249,74.73133301)(654.99578253,74.69133305)(654.95578964,74.64134187)
\curveto(654.87578265,74.53133321)(654.79078274,74.42133332)(654.70078964,74.31134188)
\curveto(654.61078292,74.21133353)(654.525783,74.10633363)(654.44578964,73.99634187)
\curveto(654.30578322,73.79633395)(654.16578336,73.58633416)(654.02578964,73.36634187)
\curveto(653.88578364,73.15633458)(653.74578378,72.9413348)(653.60578964,72.72134188)
\curveto(653.55578397,72.63133511)(653.50578402,72.53633521)(653.45578964,72.43634188)
\curveto(653.40578412,72.3363354)(653.35078418,72.2413355)(653.29078964,72.15134187)
\curveto(653.27078426,72.13133561)(653.26078427,72.10633563)(653.26078964,72.07634188)
\curveto(653.26078427,72.0463357)(653.25078428,72.02133572)(653.23078964,72.00134187)
\curveto(653.16078437,71.90133584)(653.09578443,71.78633596)(653.03578964,71.65634187)
\curveto(652.97578455,71.5363362)(652.92078461,71.42133632)(652.87078964,71.31134188)
\curveto(652.77078476,71.08133666)(652.67578485,70.8463369)(652.58578964,70.60634187)
\curveto(652.49578503,70.36633737)(652.39578513,70.12633762)(652.28578964,69.88634187)
\curveto(652.26578526,69.8363379)(652.25078528,69.79133795)(652.24078964,69.75134187)
\curveto(652.24078529,69.71133803)(652.2307853,69.66633807)(652.21078964,69.61634187)
\curveto(652.16078537,69.49633824)(652.11578541,69.37133837)(652.07578964,69.24134187)
\curveto(652.04578548,69.12133862)(652.01078552,69.00133874)(651.97078964,68.88134187)
\curveto(651.89078564,68.65133909)(651.8257857,68.41133933)(651.77578964,68.16134187)
\curveto(651.73578579,67.92133982)(651.68578584,67.68134006)(651.62578964,67.44134188)
\curveto(651.58578594,67.29134045)(651.56078597,67.1413406)(651.55078964,66.99134187)
\curveto(651.54078599,66.8413409)(651.52078601,66.69134105)(651.49078964,66.54134187)
\curveto(651.48078605,66.50134124)(651.47578605,66.4413413)(651.47578964,66.36134187)
\curveto(651.44578608,66.2413415)(651.41578611,66.1413416)(651.38578964,66.06134188)
\curveto(651.35578617,65.98134176)(651.28578624,65.92634182)(651.17578964,65.89634187)
\curveto(651.1257864,65.87634186)(651.07078646,65.86634187)(651.01078964,65.86634187)
\lineto(650.81578964,65.86634187)
\curveto(650.67578685,65.86634187)(650.53578699,65.87134187)(650.39578964,65.88134187)
\curveto(650.26578726,65.89134185)(650.17078736,65.9363418)(650.11078964,66.01634187)
\curveto(650.07078746,66.07634166)(650.05078748,66.16134158)(650.05078964,66.27134187)
\curveto(650.06078747,66.38134136)(650.07578745,66.47634126)(650.09578964,66.55634188)
\lineto(650.09578964,66.63134187)
\curveto(650.10578742,66.66134108)(650.11078742,66.69134105)(650.11078964,66.72134188)
\curveto(650.1307874,66.80134094)(650.14078739,66.87634086)(650.14078964,66.94634188)
\curveto(650.14078739,67.01634072)(650.15078738,67.08634065)(650.17078964,67.15634187)
\curveto(650.22078731,67.34634039)(650.26078727,67.53134021)(650.29078964,67.71134188)
\curveto(650.32078721,67.90133984)(650.36078717,68.08133966)(650.41078964,68.25134187)
\curveto(650.4307871,68.30133944)(650.44078709,68.3413394)(650.44078964,68.37134187)
\curveto(650.44078709,68.40133934)(650.44578708,68.4363393)(650.45578964,68.47634188)
\curveto(650.55578697,68.77633897)(650.64578688,69.07133867)(650.72578964,69.36134187)
\curveto(650.81578671,69.65133809)(650.92078661,69.93133781)(651.04078964,70.20134188)
\curveto(651.30078623,70.78133696)(651.57078596,71.33133641)(651.85078964,71.85134188)
\curveto(652.1307854,72.38133536)(652.44078509,72.88633485)(652.78078964,73.36634187)
\curveto(652.92078461,73.56633417)(653.07078446,73.75633398)(653.23078964,73.93634188)
\curveto(653.39078414,74.12633362)(653.54078399,74.31633343)(653.68078964,74.50634187)
\curveto(653.72078381,74.55633318)(653.75578377,74.60133314)(653.78578964,74.64134187)
\curveto(653.8257837,74.69133305)(653.86078367,74.741333)(653.89078964,74.79134187)
\curveto(653.90078363,74.81133293)(653.91078362,74.8363329)(653.92078964,74.86634187)
\curveto(653.94078359,74.89633284)(653.94078359,74.92633282)(653.92078964,74.95634188)
\curveto(653.90078363,75.01633272)(653.86578366,75.05133269)(653.81578964,75.06134188)
\curveto(653.76578376,75.08133266)(653.71578381,75.10133264)(653.66578964,75.12134187)
\lineto(653.56078964,75.12134187)
\curveto(653.52078401,75.13133261)(653.47078406,75.13133261)(653.41078964,75.12134187)
\lineto(653.26078964,75.12134187)
\lineto(652.66078964,75.12134187)
\lineto(650.02078964,75.12134187)
\lineto(649.28578964,75.12134187)
\lineto(649.04578964,75.12134187)
\curveto(648.97578855,75.13133261)(648.91578861,75.14633259)(648.86578964,75.16634187)
\curveto(648.77578875,75.20633254)(648.71578881,75.26633248)(648.68578964,75.34634188)
\curveto(648.63578889,75.44633229)(648.62078891,75.59133215)(648.64078964,75.78134187)
\curveto(648.66078887,75.98133176)(648.69578883,76.11633163)(648.74578964,76.18634188)
\curveto(648.76578876,76.20633154)(648.79078874,76.22133152)(648.82078964,76.23134187)
\lineto(648.94078964,76.29134187)
\curveto(648.96078857,76.29133145)(648.97578855,76.28633145)(648.98578964,76.27634187)
\curveto(649.00578852,76.27633146)(649.0257885,76.28133146)(649.04578964,76.29134187)
}
}
{
\newrgbcolor{curcolor}{0 0 0}
\pscustom[linestyle=none,fillstyle=solid,fillcolor=curcolor]
{
\newpath
\moveto(666.74039901,74.40134187)
\curveto(666.54038871,74.11133363)(666.33038892,73.82633391)(666.11039901,73.54634187)
\curveto(665.90038935,73.26633448)(665.69538956,72.98133476)(665.49539901,72.69134188)
\curveto(664.89539036,71.8413359)(664.29039096,71.00133674)(663.68039901,70.17134187)
\curveto(663.07039218,69.35133839)(662.46539279,68.51633923)(661.86539901,67.66634187)
\lineto(661.35539901,66.94634188)
\lineto(660.84539901,66.25634187)
\curveto(660.76539449,66.14634159)(660.68539457,66.03134171)(660.60539901,65.91134187)
\curveto(660.52539473,65.79134195)(660.43039482,65.69634204)(660.32039901,65.62634187)
\curveto(660.28039497,65.60634213)(660.21539504,65.59134215)(660.12539901,65.58134188)
\curveto(660.04539521,65.56134218)(659.9553953,65.55134219)(659.85539901,65.55134187)
\curveto(659.7553955,65.55134219)(659.66039559,65.55634218)(659.57039901,65.56634188)
\curveto(659.49039576,65.57634217)(659.43039582,65.59634214)(659.39039901,65.62634187)
\curveto(659.36039589,65.6463421)(659.33539592,65.68134206)(659.31539901,65.73134187)
\curveto(659.30539595,65.77134197)(659.31039594,65.81634192)(659.33039901,65.86634187)
\curveto(659.37039588,65.94634179)(659.41539584,66.02134172)(659.46539901,66.09134188)
\curveto(659.52539573,66.17134157)(659.58039567,66.25134149)(659.63039901,66.33134188)
\curveto(659.87039538,66.67134107)(660.11539514,67.00634073)(660.36539901,67.33634188)
\curveto(660.61539464,67.66634007)(660.8553944,68.00133974)(661.08539901,68.34134188)
\curveto(661.24539401,68.56133918)(661.40539385,68.77633897)(661.56539901,68.98634187)
\curveto(661.72539353,69.19633855)(661.88539337,69.41133833)(662.04539901,69.63134187)
\curveto(662.40539285,70.15133759)(662.77039248,70.66133708)(663.14039901,71.16134187)
\curveto(663.51039174,71.66133608)(663.88039137,72.17133557)(664.25039901,72.69134188)
\curveto(664.39039086,72.89133485)(664.53039072,73.08633465)(664.67039901,73.27634187)
\curveto(664.82039043,73.46633428)(664.96539029,73.66133408)(665.10539901,73.86134187)
\curveto(665.31538994,74.16133358)(665.53038972,74.46133328)(665.75039901,74.76134187)
\lineto(666.41039901,75.66134187)
\lineto(666.59039901,75.93134188)
\lineto(666.80039901,76.20134188)
\lineto(666.92039901,76.38134187)
\curveto(666.97038828,76.4413313)(667.02038823,76.49633124)(667.07039901,76.54634187)
\curveto(667.14038811,76.59633115)(667.21538804,76.63133111)(667.29539901,76.65134187)
\curveto(667.31538794,76.66133108)(667.34038791,76.66133108)(667.37039901,76.65134187)
\curveto(667.41038784,76.65133109)(667.44038781,76.66133108)(667.46039901,76.68134188)
\curveto(667.58038767,76.68133106)(667.71538754,76.67633106)(667.86539901,76.66634187)
\curveto(668.01538724,76.66633108)(668.10538715,76.62133112)(668.13539901,76.53134187)
\curveto(668.1553871,76.50133124)(668.16038709,76.46633128)(668.15039901,76.42634187)
\curveto(668.14038711,76.38633136)(668.12538713,76.35633138)(668.10539901,76.33634188)
\curveto(668.06538719,76.25633149)(668.02538723,76.18633156)(667.98539901,76.12634187)
\curveto(667.94538731,76.06633168)(667.90038735,76.00633174)(667.85039901,75.94634188)
\lineto(667.28039901,75.16634187)
\curveto(667.10038815,74.91633283)(666.92038833,74.66133308)(666.74039901,74.40134187)
\moveto(659.88539901,70.50134187)
\curveto(659.83539542,70.52133722)(659.78539547,70.52633722)(659.73539901,70.51634187)
\curveto(659.68539557,70.50633723)(659.63539562,70.51133723)(659.58539901,70.53134187)
\curveto(659.47539578,70.55133719)(659.37039588,70.57133717)(659.27039901,70.59134188)
\curveto(659.18039607,70.62133712)(659.08539617,70.66133708)(658.98539901,70.71134188)
\curveto(658.6553966,70.85133689)(658.40039685,71.0463367)(658.22039901,71.29634187)
\curveto(658.04039721,71.55633618)(657.89539736,71.86633588)(657.78539901,72.22634188)
\curveto(657.7553975,72.30633543)(657.73539752,72.38633536)(657.72539901,72.46634188)
\curveto(657.71539754,72.55633518)(657.70039755,72.6413351)(657.68039901,72.72134188)
\curveto(657.67039758,72.77133497)(657.66539759,72.8363349)(657.66539901,72.91634187)
\curveto(657.6553976,72.94633479)(657.6503976,72.97633477)(657.65039901,73.00634187)
\curveto(657.6503976,73.0463347)(657.64539761,73.08133466)(657.63539901,73.11134187)
\lineto(657.63539901,73.26134187)
\curveto(657.62539763,73.31133443)(657.62039763,73.37133437)(657.62039901,73.44134188)
\curveto(657.62039763,73.52133422)(657.62539763,73.58633416)(657.63539901,73.63634187)
\lineto(657.63539901,73.80134187)
\curveto(657.6553976,73.85133389)(657.66039759,73.89633384)(657.65039901,73.93634188)
\curveto(657.6503976,73.98633376)(657.6553976,74.03133371)(657.66539901,74.07134188)
\curveto(657.67539758,74.11133363)(657.68039757,74.14633359)(657.68039901,74.17634187)
\curveto(657.68039757,74.21633352)(657.68539757,74.25633349)(657.69539901,74.29634187)
\curveto(657.72539753,74.40633333)(657.74539751,74.51633323)(657.75539901,74.62634187)
\curveto(657.77539748,74.74633299)(657.81039744,74.86133288)(657.86039901,74.97134188)
\curveto(658.00039725,75.31133243)(658.16039709,75.58633216)(658.34039901,75.79634187)
\curveto(658.53039672,76.01633172)(658.80039645,76.19633155)(659.15039901,76.33634188)
\curveto(659.23039602,76.36633137)(659.31539594,76.38633136)(659.40539901,76.39634187)
\curveto(659.49539576,76.41633132)(659.59039566,76.4363313)(659.69039901,76.45634188)
\curveto(659.72039553,76.46633128)(659.77539548,76.46633128)(659.85539901,76.45634188)
\curveto(659.93539532,76.45633129)(659.98539527,76.46633128)(660.00539901,76.48634187)
\curveto(660.56539469,76.49633124)(661.01539424,76.38633136)(661.35539901,76.15634187)
\curveto(661.70539355,75.92633182)(661.96539329,75.62133212)(662.13539901,75.24134187)
\curveto(662.17539308,75.15133259)(662.21039304,75.05633269)(662.24039901,74.95634188)
\curveto(662.27039298,74.85633289)(662.29539296,74.75633298)(662.31539901,74.65634187)
\curveto(662.33539292,74.62633311)(662.34039291,74.59633315)(662.33039901,74.56634188)
\curveto(662.33039292,74.53633321)(662.33539292,74.50633323)(662.34539901,74.47634188)
\curveto(662.37539288,74.36633337)(662.39539286,74.2413335)(662.40539901,74.10134188)
\curveto(662.41539284,73.97133377)(662.42539283,73.8363339)(662.43539901,73.69634188)
\lineto(662.43539901,73.53134187)
\curveto(662.44539281,73.47133427)(662.44539281,73.41633432)(662.43539901,73.36634187)
\curveto(662.42539283,73.31633443)(662.42039283,73.26633448)(662.42039901,73.21634188)
\lineto(662.42039901,73.08134188)
\curveto(662.41039284,73.0413347)(662.40539285,73.00133474)(662.40539901,72.96134188)
\curveto(662.41539284,72.92133482)(662.41039284,72.87633486)(662.39039901,72.82634188)
\curveto(662.37039288,72.71633503)(662.3503929,72.61133513)(662.33039901,72.51134187)
\curveto(662.32039293,72.41133533)(662.30039295,72.31133543)(662.27039901,72.21134188)
\curveto(662.14039311,71.85133589)(661.97539328,71.5363362)(661.77539901,71.26634187)
\curveto(661.57539368,70.99633675)(661.30039395,70.79133695)(660.95039901,70.65134187)
\curveto(660.87039438,70.62133712)(660.78539447,70.59633715)(660.69539901,70.57634188)
\lineto(660.42539901,70.51634187)
\curveto(660.37539488,70.50633723)(660.33039492,70.50133724)(660.29039901,70.50134187)
\curveto(660.250395,70.51133723)(660.21039504,70.51133723)(660.17039901,70.50134187)
\curveto(660.07039518,70.48133726)(659.97539528,70.48133726)(659.88539901,70.50134187)
\moveto(659.04539901,71.89634187)
\curveto(659.08539617,71.82633591)(659.12539613,71.76133598)(659.16539901,71.70134188)
\curveto(659.20539605,71.65133609)(659.255396,71.60133614)(659.31539901,71.55134187)
\lineto(659.46539901,71.43134188)
\curveto(659.52539573,71.40133634)(659.59039566,71.37633637)(659.66039901,71.35634187)
\curveto(659.70039555,71.3363364)(659.73539552,71.32633642)(659.76539901,71.32634188)
\curveto(659.80539545,71.3363364)(659.84539541,71.33133641)(659.88539901,71.31134188)
\curveto(659.91539534,71.31133643)(659.9553953,71.30633643)(660.00539901,71.29634187)
\curveto(660.0553952,71.29633644)(660.09539516,71.30133644)(660.12539901,71.31134188)
\lineto(660.35039901,71.35634187)
\curveto(660.60039465,71.4363363)(660.78539447,71.56133618)(660.90539901,71.73134187)
\curveto(660.98539427,71.83133591)(661.0553942,71.96133578)(661.11539901,72.12134187)
\curveto(661.19539406,72.30133544)(661.255394,72.52633522)(661.29539901,72.79634187)
\curveto(661.33539392,73.07633466)(661.3503939,73.35633438)(661.34039901,73.63634187)
\curveto(661.33039392,73.92633382)(661.30039395,74.20133354)(661.25039901,74.46134188)
\curveto(661.20039405,74.72133302)(661.12539413,74.93133281)(661.02539901,75.09134188)
\curveto(660.90539435,75.29133245)(660.7553945,75.4413323)(660.57539901,75.54134187)
\curveto(660.49539476,75.59133215)(660.40539485,75.62133212)(660.30539901,75.63134187)
\curveto(660.20539505,75.65133209)(660.10039515,75.66133208)(659.99039901,75.66134187)
\curveto(659.97039528,75.65133209)(659.94539531,75.64633209)(659.91539901,75.64634187)
\curveto(659.89539536,75.65633209)(659.87539538,75.65633209)(659.85539901,75.64634187)
\curveto(659.80539545,75.6363321)(659.76039549,75.62633211)(659.72039901,75.61634187)
\curveto(659.68039557,75.61633212)(659.64039561,75.60633214)(659.60039901,75.58634188)
\curveto(659.42039583,75.50633223)(659.27039598,75.38633236)(659.15039901,75.22634188)
\curveto(659.04039621,75.06633268)(658.9503963,74.88633285)(658.88039901,74.68634188)
\curveto(658.82039643,74.49633324)(658.77539648,74.27133347)(658.74539901,74.01134187)
\curveto(658.72539653,73.75133399)(658.72039653,73.48633425)(658.73039901,73.21634188)
\curveto(658.74039651,72.95633478)(658.77039648,72.70633503)(658.82039901,72.46634188)
\curveto(658.88039637,72.2363355)(658.9553963,72.0463357)(659.04539901,71.89634187)
\moveto(669.84539901,68.91134187)
\curveto(669.8553854,68.86133888)(669.86038539,68.77133897)(669.86039901,68.64134187)
\curveto(669.86038539,68.51133923)(669.8503854,68.42133932)(669.83039901,68.37134187)
\curveto(669.81038544,68.32133942)(669.80538545,68.26633947)(669.81539901,68.20634188)
\curveto(669.82538543,68.15633958)(669.82538543,68.10633964)(669.81539901,68.05634188)
\curveto(669.77538548,67.91633983)(669.74538551,67.78133996)(669.72539901,67.65134187)
\curveto(669.71538554,67.52134022)(669.68538557,67.40134034)(669.63539901,67.29134187)
\curveto(669.49538576,66.9413408)(669.33038592,66.6463411)(669.14039901,66.40634187)
\curveto(668.9503863,66.17634157)(668.68038657,65.99134175)(668.33039901,65.85134188)
\curveto(668.250387,65.82134192)(668.16538709,65.80134194)(668.07539901,65.79134187)
\curveto(667.98538727,65.77134197)(667.90038735,65.75134199)(667.82039901,65.73134187)
\curveto(667.77038748,65.72134202)(667.72038753,65.71634203)(667.67039901,65.71634188)
\curveto(667.62038763,65.71634203)(667.57038768,65.71134203)(667.52039901,65.70134188)
\curveto(667.49038776,65.69134205)(667.44038781,65.69134205)(667.37039901,65.70134188)
\curveto(667.30038795,65.70134204)(667.250388,65.70634204)(667.22039901,65.71634188)
\curveto(667.16038809,65.736342)(667.10038815,65.74634199)(667.04039901,65.74634187)
\curveto(666.99038826,65.736342)(666.94038831,65.741342)(666.89039901,65.76134187)
\curveto(666.80038845,65.78134196)(666.71038854,65.80634193)(666.62039901,65.83634188)
\curveto(666.54038871,65.85634189)(666.46038879,65.88634185)(666.38039901,65.92634187)
\curveto(666.06038919,66.06634167)(665.81038944,66.26134148)(665.63039901,66.51134187)
\curveto(665.4503898,66.77134097)(665.30038995,67.07634066)(665.18039901,67.42634187)
\curveto(665.16039009,67.50634024)(665.14539011,67.59134015)(665.13539901,67.68134188)
\curveto(665.12539013,67.77133997)(665.11039014,67.85633989)(665.09039901,67.93634188)
\curveto(665.08039017,67.96633977)(665.07539018,67.99633975)(665.07539901,68.02634187)
\lineto(665.07539901,68.13134187)
\curveto(665.0553902,68.21133953)(665.04539021,68.29133945)(665.04539901,68.37134187)
\lineto(665.04539901,68.50634187)
\curveto(665.02539023,68.60633913)(665.02539023,68.70633904)(665.04539901,68.80634188)
\lineto(665.04539901,68.98634187)
\curveto(665.0553902,69.0363387)(665.06039019,69.08133866)(665.06039901,69.12134187)
\curveto(665.06039019,69.17133857)(665.06539019,69.21633852)(665.07539901,69.25634187)
\curveto(665.08539017,69.29633844)(665.09039016,69.33133841)(665.09039901,69.36134187)
\curveto(665.09039016,69.40133834)(665.09539016,69.4413383)(665.10539901,69.48134187)
\lineto(665.16539901,69.81134188)
\curveto(665.18539007,69.93133781)(665.21539004,70.0413377)(665.25539901,70.14134187)
\curveto(665.39538986,70.47133727)(665.5553897,70.74633699)(665.73539901,70.96634188)
\curveto(665.92538933,71.19633655)(666.18538907,71.38133636)(666.51539901,71.52134187)
\curveto(666.59538866,71.56133618)(666.68038857,71.58633616)(666.77039901,71.59634188)
\lineto(667.07039901,71.65634187)
\lineto(667.20539901,71.65634187)
\curveto(667.255388,71.66633608)(667.30538795,71.67133607)(667.35539901,71.67134187)
\curveto(667.92538733,71.69133605)(668.38538687,71.58633616)(668.73539901,71.35634187)
\curveto(669.09538616,71.1363366)(669.36038589,70.8363369)(669.53039901,70.45634188)
\curveto(669.58038567,70.35633738)(669.62038563,70.25633749)(669.65039901,70.15634187)
\curveto(669.68038557,70.05633769)(669.71038554,69.95133779)(669.74039901,69.84134188)
\curveto(669.7503855,69.80133794)(669.7553855,69.76633797)(669.75539901,69.73634187)
\curveto(669.7553855,69.71633803)(669.76038549,69.68633805)(669.77039901,69.64634187)
\curveto(669.79038546,69.57633817)(669.80038545,69.50133824)(669.80039901,69.42134187)
\curveto(669.80038545,69.3413384)(669.81038544,69.26133848)(669.83039901,69.18134188)
\curveto(669.83038542,69.13133861)(669.83038542,69.08633865)(669.83039901,69.04634187)
\curveto(669.83038542,69.00633873)(669.83538542,68.96133878)(669.84539901,68.91134187)
\moveto(668.73539901,68.47634188)
\curveto(668.74538651,68.52633922)(668.7503865,68.60133914)(668.75039901,68.70134188)
\curveto(668.76038649,68.80133894)(668.7553865,68.87633886)(668.73539901,68.92634187)
\curveto(668.71538654,68.98633876)(668.71038654,69.0413387)(668.72039901,69.09134188)
\curveto(668.74038651,69.15133859)(668.74038651,69.21133853)(668.72039901,69.27134187)
\curveto(668.71038654,69.30133844)(668.70538655,69.3363384)(668.70539901,69.37634187)
\curveto(668.70538655,69.41633832)(668.70038655,69.45633829)(668.69039901,69.49634187)
\curveto(668.67038658,69.57633817)(668.6503866,69.65133809)(668.63039901,69.72134188)
\curveto(668.62038663,69.80133794)(668.60538665,69.88133786)(668.58539901,69.96134188)
\curveto(668.5553867,70.02133772)(668.53038672,70.08133766)(668.51039901,70.14134187)
\curveto(668.49038676,70.20133754)(668.46038679,70.26133748)(668.42039901,70.32134188)
\curveto(668.32038693,70.49133725)(668.19038706,70.62633711)(668.03039901,70.72634188)
\curveto(667.9503873,70.77633697)(667.8553874,70.81133693)(667.74539901,70.83134188)
\curveto(667.63538762,70.85133689)(667.51038774,70.86133688)(667.37039901,70.86134187)
\curveto(667.3503879,70.85133689)(667.32538793,70.8463369)(667.29539901,70.84634188)
\curveto(667.26538799,70.85633689)(667.23538802,70.85633689)(667.20539901,70.84634188)
\lineto(667.05539901,70.78634187)
\curveto(667.00538825,70.77633697)(666.96038829,70.76133698)(666.92039901,70.74134187)
\curveto(666.73038852,70.63133711)(666.58538867,70.48633725)(666.48539901,70.30634188)
\curveto(666.39538886,70.12633762)(666.31538894,69.92133782)(666.24539901,69.69134188)
\curveto(666.20538905,69.56133818)(666.18538907,69.42633831)(666.18539901,69.28634187)
\curveto(666.18538907,69.15633858)(666.17538908,69.01133873)(666.15539901,68.85134188)
\curveto(666.14538911,68.80133894)(666.13538912,68.741339)(666.12539901,68.67134187)
\curveto(666.12538913,68.60133914)(666.13538912,68.5413392)(666.15539901,68.49134187)
\lineto(666.15539901,68.32634188)
\lineto(666.15539901,68.14634187)
\curveto(666.16538909,68.09633964)(666.17538908,68.0413397)(666.18539901,67.98134187)
\curveto(666.19538906,67.93133981)(666.20038905,67.87633986)(666.20039901,67.81634188)
\curveto(666.21038904,67.75633998)(666.22538903,67.70134004)(666.24539901,67.65134187)
\curveto(666.29538896,67.46134028)(666.3553889,67.28634045)(666.42539901,67.12634187)
\curveto(666.49538876,66.96634077)(666.60038865,66.83634091)(666.74039901,66.73634187)
\curveto(666.87038838,66.63634111)(667.01038824,66.56634117)(667.16039901,66.52634187)
\curveto(667.19038806,66.51634123)(667.21538804,66.51134123)(667.23539901,66.51134187)
\curveto(667.26538799,66.52134122)(667.29538796,66.52134122)(667.32539901,66.51134187)
\curveto(667.34538791,66.51134123)(667.37538788,66.50634124)(667.41539901,66.49634187)
\curveto(667.4553878,66.49634124)(667.49038776,66.50134124)(667.52039901,66.51134187)
\curveto(667.56038769,66.52134122)(667.60038765,66.52634122)(667.64039901,66.52634187)
\curveto(667.68038757,66.52634122)(667.72038753,66.5363412)(667.76039901,66.55634188)
\curveto(668.00038725,66.63634111)(668.19538706,66.77134097)(668.34539901,66.96134188)
\curveto(668.46538679,67.1413406)(668.5553867,67.34634039)(668.61539901,67.57634188)
\curveto(668.63538662,67.6463401)(668.6503866,67.71634003)(668.66039901,67.78634187)
\curveto(668.67038658,67.86633987)(668.68538657,67.94633979)(668.70539901,68.02634187)
\curveto(668.70538655,68.08633965)(668.71038654,68.13133961)(668.72039901,68.16134187)
\curveto(668.72038653,68.18133956)(668.72038653,68.20633953)(668.72039901,68.23634187)
\curveto(668.72038653,68.27633946)(668.72538653,68.30633944)(668.73539901,68.32634188)
\lineto(668.73539901,68.47634188)
}
}
\end{pspicture}

\caption{Porcentajes de los espacios virtuales y sus recursos según su tipo}
\label{espacios_pie_1}
\end{figure}

\subsection{Recursos}
Después de ver las variables que controlan a los espacios virtuales, ahora
veremos aquellas que nos muestran el uso y aprovechamiento de los recursos
generados. En la cuadro \ref{recursos_tabla_1}, se presentan los datos
recolectados acerca de los recursos del sistema consistente de las siguientes
columnas:

\begin{table}
\centering
\begin{tabular}{l|c c c c c}
$Tipo$ & $Cantidad$ & $Audiencia$ & $Comentarios$ &
$Calificadores$ & $Etiquetas$ \\
\hline
$Notas      $ & $42$ & $811$ & $17$ & $19$ & $61$ \\
$Archivos   $ & $13$ & $ 72$ & $ 9$ & $ 1$ & $13$ \\
$Eventos    $ & $ 4$ & $243$ & $ 2$ & $ 0$ & $ 5$ \\
$Enlaces    $ & $ 3$ & $ 10$ & $ 1$ & $ 0$ & $ 7$ \\
$Fotografias$ & $ 5$ & $394$ & $ 4$ & $ 2$ & $12$ \\
$Videos     $ & $ 1$ & $  1$ & $ 0$ & $ 0$ & $ 2$ \\
\hline
 & & $Aud/Can$ & $Com/Aud$ & $Cal/Aud$ & $Eti/Can$ \\
\hline
$Notas      $ & & $19.31$ & $0.021$ & $0.023$ & $1.452$ \\
$Archivos   $ & & $ 5.53$ & $0.125$ & $0.014$ & $1    $ \\
$Eventos    $ & & $60.75$ & $0.008$ & $0    $ & $1.250$ \\
$Enlaces    $ & & $ 3.33$ & $0.100$ & $0    $ & $2.333$ \\
$Fotografias$ & & $78.80$ & $0.010$ & $0.005$ & $2.400$ \\
$Videos     $ & & $ 1   $ & $0    $ & $0    $ & $2    $ \\
\end{tabular}
\caption{Clasificación de los recursos según su tipo}
\label{recursos_tabla_1}
\end{table}

\begin{description}
\item [Cantidad] Representa la cantidad total de recursos del sistema de un
tipo determinado.
\item [Audiencia] Representa la cantidad total acumulada de visualizaciones que
se obtuvieron en los recursos de un tipo establecido.
\item [Comentarios] Representa la cantidad total acumulada de comentarios que
se obtuvieron en los recursos de un tipo establecido.
\item [Calificadores] Representa la cantidad total acumulada de calificadores
que se obtuvieron en los recursos de un tipo establecido.
\item [Etiquetas] Representa la cantidad total acumulada de etiquetas utilizadas
en los recursos de un tipo establecido.
\item [Aud/Can] Representa el promedio de audiencia que un recurso de un tipo
establecido obtuvo.
\item [Com/Aud] Representa el promedio de comentarios que un recurso de un tipo
establecido obtuvo.
\item [Cal/Aud] Representa el promedio de calificadores que un recurso de un
tipo establecido obtuvo.
\item [Eti/Can] Representa el promedio de etiquetas que un recurso de un tipo
establecido a creado o utilizado.
\end{description}

En la figura \ref{recursos_bars_1}, se presenta el diagrama de barras relativo
al cuadro \ref{recursos_tabla_1}, destaca la gran cantidad de notas por sobre
los otros tipos de recursos, pudiendo esto deberse a la inmensa facilidad de
creación de estas, Aun así son las fotografías la que en proporción reciben
mejor audiencia, y son los archivos los que reciben mayor cantidad de 
comentarios
.
\begin{figure}
\centering
%LaTeX with PSTricks extensions
%%Creator: inkscape 0.48.5
%%Please note this file requires PSTricks extensions
\psset{xunit=.5pt,yunit=.5pt,runit=.5pt}
\begin{pspicture}(865,422)
{
\newrgbcolor{curcolor}{0 0 0}
\pscustom[linestyle=none,fillstyle=solid,fillcolor=curcolor]
{
\newpath
\moveto(358.52975342,24.63030273)
\lineto(363.43475342,24.63030273)
\lineto(364.72475342,24.63030273)
\curveto(364.83474339,24.63029204)(364.94474328,24.63029204)(365.05475342,24.63030273)
\curveto(365.16474306,24.64029203)(365.24974298,24.62029205)(365.30975342,24.57030273)
\curveto(365.3297429,24.55029212)(365.34474288,24.52529214)(365.35475342,24.49530273)
\curveto(365.37474285,24.4652922)(365.39474283,24.43529223)(365.41475342,24.40530273)
\curveto(365.41474281,24.33529233)(365.39974283,24.22029245)(365.36975342,24.06030273)
\curveto(365.33974289,23.91029276)(365.30474292,23.79529287)(365.26475342,23.71530273)
\curveto(365.20474302,23.57529309)(365.10474312,23.49529317)(364.96475342,23.47530273)
\curveto(364.83474339,23.4652932)(364.67974355,23.46029321)(364.49975342,23.46030273)
\lineto(362.99975342,23.46030273)
\lineto(360.47975342,23.46030273)
\lineto(359.90975342,23.46030273)
\curveto(359.69974853,23.4702932)(359.53974869,23.44529322)(359.42975342,23.38530273)
\curveto(359.31974891,23.32529334)(359.24474898,23.22029345)(359.20475342,23.07030273)
\curveto(359.17474905,22.92029375)(359.14474908,22.7652939)(359.11475342,22.60530273)
\lineto(358.79975342,21.07530273)
\curveto(358.77974945,20.9652957)(358.74974948,20.83529583)(358.70975342,20.68530273)
\curveto(358.67974955,20.53529613)(358.66474956,20.41529625)(358.66475342,20.32530273)
\curveto(358.67474955,20.20529646)(358.71974951,20.12529654)(358.79975342,20.08530273)
\curveto(358.83974939,20.0652966)(358.90474932,20.04529662)(358.99475342,20.02530273)
\lineto(359.14475342,20.02530273)
\curveto(359.18474904,20.01529665)(359.224749,20.01029666)(359.26475342,20.01030273)
\curveto(359.31474891,20.02029665)(359.36474886,20.02529664)(359.41475342,20.02530273)
\lineto(359.92475342,20.02530273)
\lineto(362.86475342,20.02530273)
\lineto(363.16475342,20.02530273)
\curveto(363.27474495,20.03529663)(363.38474484,20.03529663)(363.49475342,20.02530273)
\curveto(363.61474461,20.02529664)(363.71974451,20.01529665)(363.80975342,19.99530273)
\curveto(363.90974432,19.98529668)(363.97974425,19.9652967)(364.01975342,19.93530273)
\curveto(364.04974418,19.91529675)(364.06474416,19.8702968)(364.06475342,19.80030273)
\curveto(364.07474415,19.73029694)(364.07474415,19.65529701)(364.06475342,19.57530273)
\curveto(364.06474416,19.49529717)(364.04974418,19.41029726)(364.01975342,19.32030273)
\curveto(363.99974423,19.24029743)(363.97474425,19.1702975)(363.94475342,19.11030273)
\curveto(363.90474432,19.02029765)(363.84474438,18.95529771)(363.76475342,18.91530273)
\curveto(363.74474448,18.89529777)(363.71474451,18.88029779)(363.67475342,18.87030273)
\curveto(363.64474458,18.8702978)(363.61474461,18.8652978)(363.58475342,18.85530273)
\lineto(363.49475342,18.85530273)
\curveto(363.43474479,18.84529782)(363.37974485,18.84029783)(363.32975342,18.84030273)
\curveto(363.28974494,18.85029782)(363.24474498,18.85529781)(363.19475342,18.85530273)
\lineto(362.63975342,18.85530273)
\lineto(359.47475342,18.85530273)
\lineto(359.11475342,18.85530273)
\curveto(359.00474922,18.8652978)(358.89474933,18.86029781)(358.78475342,18.84030273)
\curveto(358.68474954,18.83029784)(358.59474963,18.80529786)(358.51475342,18.76530273)
\curveto(358.43474979,18.72529794)(358.36974986,18.65529801)(358.31975342,18.55530273)
\curveto(358.27974995,18.49529817)(358.25474997,18.42529824)(358.24475342,18.34530273)
\lineto(358.21475342,18.10530273)
\lineto(358.03475342,17.26530273)
\lineto(357.74975342,15.84030273)
\curveto(357.7297505,15.70030097)(357.70975052,15.5703011)(357.68975342,15.45030273)
\curveto(357.67975055,15.34030133)(357.70475052,15.26030141)(357.76475342,15.21030273)
\curveto(357.8247504,15.16030151)(357.89975033,15.13030154)(357.98975342,15.12030273)
\lineto(358.28975342,15.12030273)
\lineto(359.24975342,15.12030273)
\lineto(362.02475342,15.12030273)
\lineto(362.87975342,15.12030273)
\lineto(363.11975342,15.12030273)
\curveto(363.19974503,15.13030154)(363.26974496,15.12530154)(363.32975342,15.10530273)
\curveto(363.43974479,15.0653016)(363.50474472,15.01030166)(363.52475342,14.94030273)
\curveto(363.54474468,14.91030176)(363.54974468,14.86030181)(363.53975342,14.79030273)
\curveto(363.53974469,14.72030195)(363.53474469,14.64530202)(363.52475342,14.56530273)
\curveto(363.51474471,14.49530217)(363.49474473,14.42030225)(363.46475342,14.34030273)
\curveto(363.44474478,14.2703024)(363.4247448,14.21530245)(363.40475342,14.17530273)
\curveto(363.35474487,14.09530257)(363.29974493,14.04030263)(363.23975342,14.01030273)
\curveto(363.16974506,13.9703027)(363.08474514,13.95030272)(362.98475342,13.95030273)
\lineto(362.71475342,13.95030273)
\lineto(361.66475342,13.95030273)
\lineto(357.67475342,13.95030273)
\lineto(356.62475342,13.95030273)
\curveto(356.48475174,13.95030272)(356.36475186,13.95530271)(356.26475342,13.96530273)
\curveto(356.17475205,13.98530268)(356.10975212,14.03530263)(356.06975342,14.11530273)
\curveto(356.04975218,14.17530249)(356.04475218,14.25030242)(356.05475342,14.34030273)
\curveto(356.07475215,14.44030223)(356.09475213,14.53530213)(356.11475342,14.62530273)
\lineto(356.32475342,15.67530273)
\lineto(357.13475342,19.69530273)
\lineto(357.80975342,23.05530273)
\lineto(357.98975342,23.98530273)
\curveto(358.00975022,24.07529259)(358.0247502,24.1652925)(358.03475342,24.25530273)
\curveto(358.05475017,24.34529232)(358.08975014,24.41529225)(358.13975342,24.46530273)
\curveto(358.19975003,24.53529213)(358.28474994,24.58529208)(358.39475342,24.61530273)
\curveto(358.4247498,24.62529204)(358.44474978,24.62529204)(358.45475342,24.61530273)
\curveto(358.47474975,24.61529205)(358.49974973,24.62029205)(358.52975342,24.63030273)
}
}
{
\newrgbcolor{curcolor}{0 0 0}
\pscustom[linestyle=none,fillstyle=solid,fillcolor=curcolor]
{
\newpath
\moveto(368.91467529,21.85530273)
\curveto(369.63466964,21.8652948)(370.21966905,21.78029489)(370.66967529,21.60030273)
\curveto(371.12966814,21.43029524)(371.44966782,21.12529554)(371.62967529,20.68530273)
\curveto(371.67966759,20.57529609)(371.70966756,20.46029621)(371.71967529,20.34030273)
\curveto(371.73966753,20.23029644)(371.75466752,20.10529656)(371.76467529,19.96530273)
\curveto(371.7746675,19.89529677)(371.76466751,19.82029685)(371.73467529,19.74030273)
\curveto(371.71466756,19.670297)(371.68966758,19.61529705)(371.65967529,19.57530273)
\curveto(371.63966763,19.55529711)(371.60966766,19.53529713)(371.56967529,19.51530273)
\curveto(371.53966773,19.50529716)(371.51466776,19.49029718)(371.49467529,19.47030273)
\curveto(371.43466784,19.45029722)(371.37966789,19.44529722)(371.32967529,19.45530273)
\curveto(371.28966798,19.4652972)(371.24466803,19.4652972)(371.19467529,19.45530273)
\curveto(371.10466817,19.43529723)(370.99466828,19.43029724)(370.86467529,19.44030273)
\curveto(370.74466853,19.46029721)(370.65966861,19.48529718)(370.60967529,19.51530273)
\curveto(370.53966873,19.5652971)(370.49966877,19.63029704)(370.48967529,19.71030273)
\curveto(370.48966878,19.80029687)(370.4696688,19.88529678)(370.42967529,19.96530273)
\curveto(370.37966889,20.12529654)(370.28466899,20.2702964)(370.14467529,20.40030273)
\curveto(370.05466922,20.48029619)(369.94466933,20.54029613)(369.81467529,20.58030273)
\curveto(369.69466958,20.62029605)(369.56466971,20.66029601)(369.42467529,20.70030273)
\curveto(369.38466989,20.72029595)(369.33466994,20.72529594)(369.27467529,20.71530273)
\curveto(369.22467005,20.71529595)(369.17967009,20.72029595)(369.13967529,20.73030273)
\curveto(369.07967019,20.75029592)(369.00467027,20.76029591)(368.91467529,20.76030273)
\curveto(368.82467045,20.76029591)(368.74967052,20.75029592)(368.68967529,20.73030273)
\lineto(368.59967529,20.73030273)
\curveto(368.53967073,20.72029595)(368.48467079,20.71029596)(368.43467529,20.70030273)
\curveto(368.38467089,20.70029597)(368.33467094,20.69529597)(368.28467529,20.68530273)
\curveto(368.01467126,20.62529604)(367.77967149,20.54029613)(367.57967529,20.43030273)
\curveto(367.38967188,20.32029635)(367.23967203,20.13529653)(367.12967529,19.87530273)
\curveto(367.09967217,19.80529686)(367.08467219,19.73529693)(367.08467529,19.66530273)
\curveto(367.08467219,19.59529707)(367.08967218,19.53529713)(367.09967529,19.48530273)
\curveto(367.12967214,19.33529733)(367.17967209,19.22529744)(367.24967529,19.15530273)
\curveto(367.31967195,19.09529757)(367.41467186,19.02529764)(367.53467529,18.94530273)
\curveto(367.6746716,18.84529782)(367.83967143,18.7702979)(368.02967529,18.72030273)
\curveto(368.21967105,18.68029799)(368.40967086,18.63029804)(368.59967529,18.57030273)
\curveto(368.71967055,18.53029814)(368.83967043,18.50029817)(368.95967529,18.48030273)
\curveto(369.08967018,18.46029821)(369.21467006,18.43029824)(369.33467529,18.39030273)
\curveto(369.53466974,18.33029834)(369.72966954,18.2702984)(369.91967529,18.21030273)
\curveto(370.10966916,18.16029851)(370.29466898,18.09529857)(370.47467529,18.01530273)
\curveto(370.52466875,17.99529867)(370.5696687,17.97529869)(370.60967529,17.95530273)
\curveto(370.65966861,17.93529873)(370.70966856,17.91029876)(370.75967529,17.88030273)
\curveto(370.92966834,17.76029891)(371.0746682,17.62529904)(371.19467529,17.47530273)
\curveto(371.31466796,17.32529934)(371.40466787,17.13529953)(371.46467529,16.90530273)
\lineto(371.46467529,16.62030273)
\curveto(371.46466781,16.55030012)(371.45966781,16.47530019)(371.44967529,16.39530273)
\curveto(371.43966783,16.32530034)(371.42966784,16.24530042)(371.41967529,16.15530273)
\lineto(371.38967529,16.00530273)
\curveto(371.34966792,15.93530073)(371.31966795,15.8653008)(371.29967529,15.79530273)
\curveto(371.28966798,15.72530094)(371.269668,15.65530101)(371.23967529,15.58530273)
\curveto(371.18966808,15.47530119)(371.13466814,15.3703013)(371.07467529,15.27030273)
\curveto(371.01466826,15.1703015)(370.94966832,15.08030159)(370.87967529,15.00030273)
\curveto(370.6696686,14.74030193)(370.42466885,14.53030214)(370.14467529,14.37030273)
\curveto(369.86466941,14.22030245)(369.55966971,14.09030258)(369.22967529,13.98030273)
\curveto(369.12967014,13.95030272)(369.02967024,13.93030274)(368.92967529,13.92030273)
\curveto(368.82967044,13.90030277)(368.73467054,13.87530279)(368.64467529,13.84530273)
\curveto(368.53467074,13.82530284)(368.42967084,13.81530285)(368.32967529,13.81530273)
\curveto(368.22967104,13.81530285)(368.12967114,13.80530286)(368.02967529,13.78530273)
\lineto(367.87967529,13.78530273)
\curveto(367.82967144,13.77530289)(367.75967151,13.7703029)(367.66967529,13.77030273)
\curveto(367.57967169,13.7703029)(367.50967176,13.77530289)(367.45967529,13.78530273)
\lineto(367.29467529,13.78530273)
\curveto(367.23467204,13.80530286)(367.1696721,13.81530285)(367.09967529,13.81530273)
\curveto(367.02967224,13.80530286)(366.9746723,13.81030286)(366.93467529,13.83030273)
\curveto(366.88467239,13.84030283)(366.81967245,13.84530282)(366.73967529,13.84530273)
\curveto(366.65967261,13.8653028)(366.58467269,13.88530278)(366.51467529,13.90530273)
\curveto(366.44467283,13.91530275)(366.3696729,13.93530273)(366.28967529,13.96530273)
\curveto(365.99967327,14.0653026)(365.75467352,14.19030248)(365.55467529,14.34030273)
\curveto(365.35467392,14.49030218)(365.19467408,14.68530198)(365.07467529,14.92530273)
\curveto(365.01467426,15.05530161)(364.96467431,15.19030148)(364.92467529,15.33030273)
\curveto(364.89467438,15.4703012)(364.8746744,15.62530104)(364.86467529,15.79530273)
\curveto(364.85467442,15.85530081)(364.85967441,15.92530074)(364.87967529,16.00530273)
\curveto(364.89967437,16.09530057)(364.92467435,16.1653005)(364.95467529,16.21530273)
\curveto(364.99467428,16.25530041)(365.05467422,16.29530037)(365.13467529,16.33530273)
\curveto(365.18467409,16.35530031)(365.25467402,16.3653003)(365.34467529,16.36530273)
\curveto(365.44467383,16.37530029)(365.53467374,16.37530029)(365.61467529,16.36530273)
\curveto(365.70467357,16.35530031)(365.78967348,16.34030033)(365.86967529,16.32030273)
\curveto(365.95967331,16.31030036)(366.01467326,16.29530037)(366.03467529,16.27530273)
\curveto(366.09467318,16.22530044)(366.12467315,16.15030052)(366.12467529,16.05030273)
\curveto(366.13467314,15.96030071)(366.15467312,15.87530079)(366.18467529,15.79530273)
\curveto(366.23467304,15.57530109)(366.33467294,15.40530126)(366.48467529,15.28530273)
\curveto(366.58467269,15.19530147)(366.70467257,15.12530154)(366.84467529,15.07530273)
\curveto(366.98467229,15.02530164)(367.13467214,14.97530169)(367.29467529,14.92530273)
\lineto(367.60967529,14.88030273)
\lineto(367.69967529,14.88030273)
\curveto(367.75967151,14.86030181)(367.84467143,14.85030182)(367.95467529,14.85030273)
\curveto(368.0746712,14.85030182)(368.17967109,14.86030181)(368.26967529,14.88030273)
\curveto(368.33967093,14.88030179)(368.39467088,14.88530178)(368.43467529,14.89530273)
\curveto(368.49467078,14.90530176)(368.55467072,14.91030176)(368.61467529,14.91030273)
\curveto(368.6746706,14.92030175)(368.72967054,14.93030174)(368.77967529,14.94030273)
\curveto(369.08967018,15.02030165)(369.33966993,15.12530154)(369.52967529,15.25530273)
\curveto(369.72966954,15.38530128)(369.89466938,15.60530106)(370.02467529,15.91530273)
\curveto(370.05466922,15.9653007)(370.0696692,16.02030065)(370.06967529,16.08030273)
\curveto(370.07966919,16.14030053)(370.07966919,16.18530048)(370.06967529,16.21530273)
\curveto(370.05966921,16.40530026)(370.01966925,16.54530012)(369.94967529,16.63530273)
\curveto(369.87966939,16.73529993)(369.78466949,16.82529984)(369.66467529,16.90530273)
\curveto(369.58466969,16.9652997)(369.48966978,17.01529965)(369.37967529,17.05530273)
\lineto(369.07967529,17.17530273)
\curveto(369.04967022,17.18529948)(369.01967025,17.19029948)(368.98967529,17.19030273)
\curveto(368.9696703,17.19029948)(368.94967032,17.20029947)(368.92967529,17.22030273)
\curveto(368.60967066,17.33029934)(368.269671,17.41029926)(367.90967529,17.46030273)
\curveto(367.55967171,17.52029915)(367.23967203,17.61529905)(366.94967529,17.74530273)
\curveto(366.85967241,17.78529888)(366.7696725,17.82029885)(366.67967529,17.85030273)
\curveto(366.59967267,17.88029879)(366.52467275,17.92029875)(366.45467529,17.97030273)
\curveto(366.28467299,18.08029859)(366.13467314,18.20529846)(366.00467529,18.34530273)
\curveto(365.8746734,18.48529818)(365.78467349,18.66029801)(365.73467529,18.87030273)
\curveto(365.71467356,18.94029773)(365.70467357,19.01029766)(365.70467529,19.08030273)
\lineto(365.70467529,19.30530273)
\curveto(365.69467358,19.42529724)(365.70967356,19.56029711)(365.74967529,19.71030273)
\curveto(365.78967348,19.8702968)(365.82967344,20.00529666)(365.86967529,20.11530273)
\curveto(365.89967337,20.1652965)(365.91967335,20.20529646)(365.92967529,20.23530273)
\curveto(365.94967332,20.27529639)(365.9746733,20.31529635)(366.00467529,20.35530273)
\curveto(366.13467314,20.58529608)(366.29467298,20.78529588)(366.48467529,20.95530273)
\curveto(366.6746726,21.12529554)(366.88467239,21.27529539)(367.11467529,21.40530273)
\curveto(367.274672,21.49529517)(367.44967182,21.5652951)(367.63967529,21.61530273)
\curveto(367.83967143,21.67529499)(368.04467123,21.73029494)(368.25467529,21.78030273)
\curveto(368.32467095,21.79029488)(368.38967088,21.80029487)(368.44967529,21.81030273)
\curveto(368.51967075,21.82029485)(368.59467068,21.83029484)(368.67467529,21.84030273)
\curveto(368.71467056,21.85029482)(368.75467052,21.85029482)(368.79467529,21.84030273)
\curveto(368.84467043,21.83029484)(368.88467039,21.83529483)(368.91467529,21.85530273)
}
}
{
\newrgbcolor{curcolor}{0 0 0}
\pscustom[linestyle=none,fillstyle=solid,fillcolor=curcolor]
{
\newpath
\moveto(380.62967529,18.01530273)
\curveto(380.62966614,17.9652987)(380.61966615,17.90029877)(380.59967529,17.82030273)
\curveto(380.58966618,17.74029893)(380.5746662,17.67529899)(380.55467529,17.62530273)
\curveto(380.52466625,17.57529909)(380.50966626,17.52529914)(380.50967529,17.47530273)
\curveto(380.50966626,17.43529923)(380.50466627,17.39529927)(380.49467529,17.35530273)
\curveto(380.4746663,17.28529938)(380.45466632,17.23029944)(380.43467529,17.19030273)
\lineto(380.34467529,16.92030273)
\curveto(380.32466645,16.83029984)(380.29466648,16.74029993)(380.25467529,16.65030273)
\curveto(380.22466655,16.5703001)(380.18966658,16.49030018)(380.14967529,16.41030273)
\curveto(380.11966665,16.34030033)(380.07966669,16.2653004)(380.02967529,16.18530273)
\curveto(379.83966693,15.81530085)(379.60966716,15.48030119)(379.33967529,15.18030273)
\curveto(379.25966751,15.09030158)(379.1746676,15.00030167)(379.08467529,14.91030273)
\curveto(378.99466778,14.83030184)(378.90466787,14.75530191)(378.81467529,14.68530273)
\lineto(378.72467529,14.61030273)
\curveto(378.64466813,14.56030211)(378.5696682,14.51030216)(378.49967529,14.46030273)
\curveto(378.42966834,14.41030226)(378.34966842,14.36030231)(378.25967529,14.31030273)
\curveto(378.12966864,14.23030244)(377.98966878,14.16030251)(377.83967529,14.10030273)
\curveto(377.69966907,14.05030262)(377.55466922,14.00030267)(377.40467529,13.95030273)
\curveto(377.32466945,13.93030274)(377.24466953,13.91530275)(377.16467529,13.90530273)
\curveto(377.08466969,13.89530277)(377.00466977,13.88030279)(376.92467529,13.86030273)
\lineto(376.86467529,13.86030273)
\curveto(376.85466992,13.85030282)(376.83966993,13.84530282)(376.81967529,13.84530273)
\curveto(376.71967005,13.82530284)(376.57967019,13.81530285)(376.39967529,13.81530273)
\curveto(376.22967054,13.80530286)(376.09967067,13.81030286)(376.00967529,13.83030273)
\lineto(375.93467529,13.83030273)
\curveto(375.86467091,13.84030283)(375.79967097,13.85030282)(375.73967529,13.86030273)
\curveto(375.67967109,13.86030281)(375.61967115,13.8703028)(375.55967529,13.89030273)
\curveto(375.39967137,13.94030273)(375.24967152,13.98530268)(375.10967529,14.02530273)
\curveto(374.9696718,14.0653026)(374.84467193,14.12530254)(374.73467529,14.20530273)
\curveto(374.58467219,14.29530237)(374.45967231,14.39030228)(374.35967529,14.49030273)
\curveto(374.32967244,14.52030215)(374.27967249,14.56030211)(374.20967529,14.61030273)
\curveto(374.13967263,14.670302)(374.06467271,14.67530199)(373.98467529,14.62530273)
\curveto(373.94467283,14.59530207)(373.91967285,14.55530211)(373.90967529,14.50530273)
\curveto(373.89967287,14.45530221)(373.8746729,14.40030227)(373.83467529,14.34030273)
\curveto(373.82467295,14.31030236)(373.81967295,14.27530239)(373.81967529,14.23530273)
\curveto(373.81967295,14.20530246)(373.81467296,14.1703025)(373.80467529,14.13030273)
\curveto(373.76467301,14.0703026)(373.73967303,14.00530266)(373.72967529,13.93530273)
\curveto(373.71967305,13.85530281)(373.70967306,13.78530288)(373.69967529,13.72530273)
\lineto(373.33967529,11.92530273)
\curveto(373.30967346,11.78530488)(373.27967349,11.64030503)(373.24967529,11.49030273)
\curveto(373.21967355,11.34030533)(373.1696736,11.22530544)(373.09967529,11.14530273)
\curveto(373.02967374,11.07530559)(372.93967383,11.04030563)(372.82967529,11.04030273)
\curveto(372.71967405,11.03030564)(372.60967416,11.02530564)(372.49967529,11.02530273)
\lineto(372.25967529,11.02530273)
\curveto(372.19967457,11.04530562)(372.14467463,11.0653056)(372.09467529,11.08530273)
\curveto(372.05467472,11.10530556)(372.02467475,11.14030553)(372.00467529,11.19030273)
\curveto(371.9746748,11.26030541)(371.9746748,11.3703053)(372.00467529,11.52030273)
\curveto(372.04467473,11.670305)(372.0746747,11.80030487)(372.09467529,11.91030273)
\lineto(373.89467529,20.91030273)
\curveto(373.91467286,21.03029564)(373.93967283,21.15029552)(373.96967529,21.27030273)
\curveto(373.99967277,21.40029527)(374.04967272,21.50529516)(374.11967529,21.58530273)
\curveto(374.15967261,21.62529504)(374.23467254,21.65529501)(374.34467529,21.67530273)
\curveto(374.45467232,21.70529496)(374.5696722,21.71529495)(374.68967529,21.70530273)
\curveto(374.80967196,21.70529496)(374.91967185,21.69029498)(375.01967529,21.66030273)
\curveto(375.11967165,21.64029503)(375.17967159,21.61029506)(375.19967529,21.57030273)
\curveto(375.22967154,21.52029515)(375.23967153,21.46029521)(375.22967529,21.39030273)
\curveto(375.21967155,21.32029535)(375.22467155,21.25029542)(375.24467529,21.18030273)
\curveto(375.25467152,21.15029552)(375.26467151,21.12529554)(375.27467529,21.10530273)
\lineto(375.31967529,21.06030273)
\curveto(375.42967134,21.05029562)(375.52967124,21.08529558)(375.61967529,21.16530273)
\curveto(375.70967106,21.24529542)(375.79467098,21.31029536)(375.87467529,21.36030273)
\curveto(376.1746706,21.54029513)(376.51467026,21.68029499)(376.89467529,21.78030273)
\curveto(376.98466979,21.80029487)(377.0746697,21.81529485)(377.16467529,21.82530273)
\curveto(377.26466951,21.83529483)(377.3696694,21.85029482)(377.47967529,21.87030273)
\curveto(377.51966925,21.88029479)(377.5696692,21.88029479)(377.62967529,21.87030273)
\curveto(377.68966908,21.86029481)(377.72966904,21.8652948)(377.74967529,21.88530273)
\curveto(378.17966859,21.89529477)(378.54966822,21.85029482)(378.85967529,21.75030273)
\curveto(379.1696676,21.66029501)(379.43966733,21.53029514)(379.66967529,21.36030273)
\curveto(379.70966706,21.32029535)(379.74966702,21.28029539)(379.78967529,21.24030273)
\curveto(379.83966693,21.21029546)(379.88466689,21.17529549)(379.92467529,21.13530273)
\curveto(379.94466683,21.11529555)(379.95966681,21.09529557)(379.96967529,21.07530273)
\curveto(379.97966679,21.0652956)(379.99466678,21.05029562)(380.01467529,21.03030273)
\curveto(380.05466672,20.98029569)(380.09466668,20.92529574)(380.13467529,20.86530273)
\curveto(380.18466659,20.80529586)(380.22966654,20.74529592)(380.26967529,20.68530273)
\curveto(380.35966641,20.51529615)(380.44966632,20.33029634)(380.53967529,20.13030273)
\curveto(380.58966618,20.00029667)(380.62466615,19.85529681)(380.64467529,19.69530273)
\curveto(380.6746661,19.53529713)(380.69466608,19.37529729)(380.70467529,19.21530273)
\curveto(380.71466606,19.13529753)(380.71466606,19.05029762)(380.70467529,18.96030273)
\curveto(380.70466607,18.8702978)(380.70966606,18.78529788)(380.71967529,18.70530273)
\lineto(380.68967529,18.58530273)
\lineto(380.68967529,18.49530273)
\curveto(380.69966607,18.44529822)(380.69466608,18.39029828)(380.67467529,18.33030273)
\curveto(380.65466612,18.2702984)(380.64966612,18.21529845)(380.65967529,18.16530273)
\lineto(380.62967529,18.01530273)
\moveto(379.21967529,17.61030273)
\curveto(379.24966752,17.66029901)(379.26466751,17.72029895)(379.26467529,17.79030273)
\curveto(379.2746675,17.8702988)(379.28466749,17.94029873)(379.29467529,18.00030273)
\curveto(379.33466744,18.1702985)(379.35966741,18.33029834)(379.36967529,18.48030273)
\curveto(379.38966738,18.63029804)(379.38466739,18.77529789)(379.35467529,18.91530273)
\curveto(379.35466742,18.97529769)(379.34966742,19.03529763)(379.33967529,19.09530273)
\curveto(379.33966743,19.1652975)(379.32966744,19.23029744)(379.30967529,19.29030273)
\curveto(379.25966751,19.56029711)(379.13966763,19.82029685)(378.94967529,20.07030273)
\curveto(378.769668,20.32029635)(378.58466819,20.49029618)(378.39467529,20.58030273)
\curveto(378.31466846,20.62029605)(378.23466854,20.65029602)(378.15467529,20.67030273)
\curveto(378.0746687,20.69029598)(377.99466878,20.71529595)(377.91467529,20.74530273)
\curveto(377.82466895,20.7652959)(377.71966905,20.77529589)(377.59967529,20.77530273)
\lineto(377.26967529,20.77530273)
\curveto(377.24966952,20.75529591)(377.20966956,20.74529592)(377.14967529,20.74530273)
\curveto(377.09966967,20.75529591)(377.05466972,20.75529591)(377.01467529,20.74530273)
\curveto(376.91466986,20.72529594)(376.81966995,20.70529596)(376.72967529,20.68530273)
\curveto(376.64967012,20.665296)(376.56467021,20.63529603)(376.47467529,20.59530273)
\curveto(376.12467065,20.45529621)(375.81967095,20.25029642)(375.55967529,19.98030273)
\curveto(375.29967147,19.72029695)(375.07967169,19.41529725)(374.89967529,19.06530273)
\curveto(374.83967193,18.95529771)(374.78967198,18.84529782)(374.74967529,18.73530273)
\curveto(374.71967205,18.62529804)(374.68467209,18.51529815)(374.64467529,18.40530273)
\curveto(374.62467215,18.3652983)(374.60967216,18.32529834)(374.59967529,18.28530273)
\curveto(374.58967218,18.25529841)(374.57967219,18.22029845)(374.56967529,18.18030273)
\lineto(374.53967529,18.06030273)
\curveto(374.51967225,18.01029866)(374.49967227,17.93529873)(374.47967529,17.83530273)
\curveto(374.45967231,17.74529892)(374.44967232,17.67529899)(374.44967529,17.62530273)
\lineto(374.43467529,17.50530273)
\curveto(374.43467234,17.4652992)(374.42967234,17.42529924)(374.41967529,17.38530273)
\curveto(374.40967236,17.34529932)(374.40967236,17.31029936)(374.41967529,17.28030273)
\curveto(374.41967235,17.25029942)(374.41467236,17.22029945)(374.40467529,17.19030273)
\lineto(374.40467529,17.08530273)
\lineto(374.40467529,16.84530273)
\curveto(374.40467237,16.7652999)(374.40967236,16.68529998)(374.41967529,16.60530273)
\curveto(374.45967231,16.2653004)(374.54967222,15.9653007)(374.68967529,15.70530273)
\curveto(374.83967193,15.45530121)(375.05967171,15.26030141)(375.34967529,15.12030273)
\curveto(375.51967125,15.04030163)(375.69967107,14.98030169)(375.88967529,14.94030273)
\curveto(375.92967084,14.92030175)(375.9696708,14.91030176)(376.00967529,14.91030273)
\curveto(376.04967072,14.92030175)(376.08967068,14.92030175)(376.12967529,14.91030273)
\lineto(376.24967529,14.91030273)
\curveto(376.31967045,14.89030178)(376.38967038,14.89030178)(376.45967529,14.91030273)
\lineto(376.57967529,14.91030273)
\curveto(376.68967008,14.93030174)(376.79466998,14.94530172)(376.89467529,14.95530273)
\curveto(377.00466977,14.9653017)(377.11466966,14.99030168)(377.22467529,15.03030273)
\curveto(377.55466922,15.16030151)(377.83966893,15.33030134)(378.07967529,15.54030273)
\curveto(378.31966845,15.76030091)(378.53466824,16.02530064)(378.72467529,16.33530273)
\curveto(378.80466797,16.47530019)(378.8696679,16.61530005)(378.91967529,16.75530273)
\curveto(378.97966779,16.90529976)(379.04466773,17.06029961)(379.11467529,17.22030273)
\curveto(379.13466764,17.2702994)(379.14466763,17.31529935)(379.14467529,17.35530273)
\curveto(379.15466762,17.39529927)(379.1696676,17.44029923)(379.18967529,17.49030273)
\lineto(379.21967529,17.61030273)
}
}
{
\newrgbcolor{curcolor}{0 0 0}
\pscustom[linestyle=none,fillstyle=solid,fillcolor=curcolor]
{
\newpath
\moveto(388.30592529,14.50530273)
\curveto(388.29591738,14.34530232)(388.25091743,14.21030246)(388.17092529,14.10030273)
\curveto(388.09091759,14.00030267)(387.99591768,13.92530274)(387.88592529,13.87530273)
\curveto(387.83591784,13.85530281)(387.7809179,13.84530282)(387.72092529,13.84530273)
\curveto(387.67091801,13.84530282)(387.61091807,13.83530283)(387.54092529,13.81530273)
\curveto(387.31091837,13.7653029)(387.09591858,13.78030289)(386.89592529,13.86030273)
\curveto(386.69591898,13.93030274)(386.57091911,14.02030265)(386.52092529,14.13030273)
\curveto(386.4809192,14.20030247)(386.45091923,14.28030239)(386.43092529,14.37030273)
\curveto(386.41091927,14.4703022)(386.3759193,14.55030212)(386.32592529,14.61030273)
\lineto(386.26592529,14.67030273)
\curveto(386.24591943,14.69030198)(386.21591946,14.69530197)(386.17592529,14.68530273)
\curveto(386.05591962,14.65530201)(385.94091974,14.60030207)(385.83092529,14.52030273)
\curveto(385.72091996,14.44030223)(385.61592006,14.3703023)(385.51592529,14.31030273)
\curveto(385.36592031,14.23030244)(385.21092047,14.15530251)(385.05092529,14.08530273)
\curveto(384.89092079,14.02530264)(384.72092096,13.9703027)(384.54092529,13.92030273)
\curveto(384.43092125,13.89030278)(384.31592136,13.8703028)(384.19592529,13.86030273)
\curveto(384.08592159,13.85030282)(383.97092171,13.83530283)(383.85092529,13.81530273)
\curveto(383.80092188,13.80530286)(383.75592192,13.80030287)(383.71592529,13.80030273)
\lineto(383.61092529,13.80030273)
\curveto(383.50092218,13.78030289)(383.39592228,13.78030289)(383.29592529,13.80030273)
\lineto(383.16092529,13.80030273)
\curveto(383.11092257,13.81030286)(383.06092262,13.81530285)(383.01092529,13.81530273)
\curveto(382.96092272,13.81530285)(382.92092276,13.82530284)(382.89092529,13.84530273)
\curveto(382.85092283,13.85530281)(382.81592286,13.86030281)(382.78592529,13.86030273)
\curveto(382.76592291,13.85030282)(382.74092294,13.85030282)(382.71092529,13.86030273)
\lineto(382.47092529,13.92030273)
\curveto(382.40092328,13.93030274)(382.33592334,13.95030272)(382.27592529,13.98030273)
\curveto(381.99592368,14.11030256)(381.7809239,14.25530241)(381.63092529,14.41530273)
\curveto(381.4809242,14.58530208)(381.3759243,14.82030185)(381.31592529,15.12030273)
\curveto(381.26592441,15.34030133)(381.27092441,15.60530106)(381.33092529,15.91530273)
\lineto(381.40592529,16.23030273)
\curveto(381.42592425,16.28030039)(381.44092424,16.33030034)(381.45092529,16.38030273)
\lineto(381.51092529,16.56030273)
\lineto(381.69092529,16.89030273)
\curveto(381.76092392,17.00029967)(381.83092385,17.10029957)(381.90092529,17.19030273)
\curveto(382.14092354,17.48029919)(382.43092325,17.69529897)(382.77092529,17.83530273)
\curveto(383.11092257,17.97529869)(383.4759222,18.10029857)(383.86592529,18.21030273)
\curveto(384.01592166,18.25029842)(384.16592151,18.28029839)(384.31592529,18.30030273)
\curveto(384.4759212,18.32029835)(384.63092105,18.34529832)(384.78092529,18.37530273)
\curveto(384.86092082,18.39529827)(384.93092075,18.40529826)(384.99092529,18.40530273)
\curveto(385.06092062,18.40529826)(385.13592054,18.41529825)(385.21592529,18.43530273)
\curveto(385.28592039,18.45529821)(385.35592032,18.4652982)(385.42592529,18.46530273)
\curveto(385.50592017,18.47529819)(385.58592009,18.49029818)(385.66592529,18.51030273)
\curveto(385.92591975,18.5702981)(386.17091951,18.62029805)(386.40092529,18.66030273)
\curveto(386.63091905,18.71029796)(386.83091885,18.82529784)(387.00092529,19.00530273)
\curveto(387.07091861,19.08529758)(387.13591854,19.18529748)(387.19592529,19.30530273)
\curveto(387.26591841,19.43529723)(387.29591838,19.57529709)(387.28592529,19.72530273)
\curveto(387.2759184,19.9652967)(387.22591845,20.15529651)(387.13592529,20.29530273)
\curveto(387.05591862,20.43529623)(386.91591876,20.54529612)(386.71592529,20.62530273)
\curveto(386.60591907,20.67529599)(386.47091921,20.71029596)(386.31092529,20.73030273)
\curveto(386.15091953,20.75029592)(385.9809197,20.76029591)(385.80092529,20.76030273)
\curveto(385.62092006,20.76029591)(385.44092024,20.75029592)(385.26092529,20.73030273)
\curveto(385.09092059,20.71029596)(384.94092074,20.68029599)(384.81092529,20.64030273)
\curveto(384.63092105,20.58029609)(384.45092123,20.49529617)(384.27092529,20.38530273)
\curveto(384.1809215,20.32529634)(384.09092159,20.24529642)(384.00092529,20.14530273)
\curveto(383.92092176,20.05529661)(383.84592183,19.95529671)(383.77592529,19.84530273)
\curveto(383.72592195,19.7652969)(383.680922,19.68029699)(383.64092529,19.59030273)
\curveto(383.60092208,19.50029717)(383.54092214,19.43029724)(383.46092529,19.38030273)
\curveto(383.41092227,19.35029732)(383.33592234,19.32529734)(383.23592529,19.30530273)
\curveto(383.13592254,19.29529737)(383.03592264,19.29029738)(382.93592529,19.29030273)
\curveto(382.83592284,19.29029738)(382.74092294,19.29529737)(382.65092529,19.30530273)
\curveto(382.56092312,19.32529734)(382.50092318,19.35029732)(382.47092529,19.38030273)
\curveto(382.43092325,19.41029726)(382.40592327,19.46029721)(382.39592529,19.53030273)
\curveto(382.39592328,19.60029707)(382.41592326,19.67529699)(382.45592529,19.75530273)
\curveto(382.50592317,19.88529678)(382.56092312,20.00529666)(382.62092529,20.11530273)
\curveto(382.680923,20.23529643)(382.74592293,20.35029632)(382.81592529,20.46030273)
\curveto(383.0759226,20.81029586)(383.37092231,21.08029559)(383.70092529,21.27030273)
\curveto(384.03092165,21.4702952)(384.42092126,21.63029504)(384.87092529,21.75030273)
\curveto(384.9809207,21.7702949)(385.08592059,21.78529488)(385.18592529,21.79530273)
\curveto(385.29592038,21.80529486)(385.40592027,21.82029485)(385.51592529,21.84030273)
\curveto(385.56592011,21.85029482)(385.63092005,21.85029482)(385.71092529,21.84030273)
\curveto(385.80091988,21.84029483)(385.86091982,21.85029482)(385.89092529,21.87030273)
\curveto(386.59091909,21.88029479)(387.1809185,21.80029487)(387.66092529,21.63030273)
\curveto(388.15091753,21.46029521)(388.45591722,21.13529553)(388.57592529,20.65530273)
\curveto(388.62591705,20.45529621)(388.63091705,20.22029645)(388.59092529,19.95030273)
\curveto(388.55091713,19.69029698)(388.50091718,19.41529725)(388.44092529,19.12530273)
\lineto(387.78092529,15.81030273)
\curveto(387.75091793,15.670301)(387.72591795,15.53530113)(387.70592529,15.40530273)
\curveto(387.69591798,15.27530139)(387.70591797,15.1703015)(387.73592529,15.09030273)
\curveto(387.7759179,15.02030165)(387.83091785,14.9703017)(387.90092529,14.94030273)
\curveto(387.99091769,14.90030177)(388.07091761,14.8703018)(388.14092529,14.85030273)
\curveto(388.22091746,14.84030183)(388.27091741,14.79530187)(388.29092529,14.71530273)
\curveto(388.31091737,14.68530198)(388.31591736,14.65530201)(388.30592529,14.62530273)
\lineto(388.30592529,14.50530273)
\moveto(386.49092529,16.17030273)
\curveto(386.5809191,16.31030036)(386.64591903,16.4703002)(386.68592529,16.65030273)
\curveto(386.72591895,16.84029983)(386.76591891,17.03529963)(386.80592529,17.23530273)
\curveto(386.82591885,17.34529932)(386.84091884,17.44529922)(386.85092529,17.53530273)
\curveto(386.86091882,17.62529904)(386.83591884,17.69529897)(386.77592529,17.74530273)
\curveto(386.74591893,17.7652989)(386.675919,17.77529889)(386.56592529,17.77530273)
\curveto(386.54591913,17.75529891)(386.51091917,17.74529892)(386.46092529,17.74530273)
\curveto(386.41091927,17.74529892)(386.36091932,17.73529893)(386.31092529,17.71530273)
\curveto(386.23091945,17.69529897)(386.13591954,17.67529899)(386.02592529,17.65530273)
\lineto(385.72592529,17.59530273)
\curveto(385.69591998,17.59529907)(385.66092002,17.59029908)(385.62092529,17.58030273)
\lineto(385.51592529,17.58030273)
\curveto(385.35592032,17.54029913)(385.18592049,17.51529915)(385.00592529,17.50530273)
\curveto(384.83592084,17.50529916)(384.67092101,17.48529918)(384.51092529,17.44530273)
\curveto(384.42092126,17.42529924)(384.34092134,17.40529926)(384.27092529,17.38530273)
\curveto(384.21092147,17.37529929)(384.13592154,17.36029931)(384.04592529,17.34030273)
\curveto(383.8759218,17.29029938)(383.71092197,17.22529944)(383.55092529,17.14530273)
\curveto(383.40092228,17.07529959)(383.26592241,16.98529968)(383.14592529,16.87530273)
\curveto(383.02592265,16.7652999)(382.92592275,16.63030004)(382.84592529,16.47030273)
\curveto(382.76592291,16.32030035)(382.70592297,16.13530053)(382.66592529,15.91530273)
\curveto(382.64592303,15.81530085)(382.64592303,15.72030095)(382.66592529,15.63030273)
\curveto(382.68592299,15.55030112)(382.71592296,15.47530119)(382.75592529,15.40530273)
\curveto(382.80592287,15.29530137)(382.88592279,15.20030147)(382.99592529,15.12030273)
\curveto(383.11592256,15.05030162)(383.24592243,14.99030168)(383.38592529,14.94030273)
\curveto(383.43592224,14.93030174)(383.48592219,14.92530174)(383.53592529,14.92530273)
\curveto(383.58592209,14.92530174)(383.63592204,14.92030175)(383.68592529,14.91030273)
\curveto(383.75592192,14.89030178)(383.84092184,14.87530179)(383.94092529,14.86530273)
\curveto(384.04092164,14.8653018)(384.13092155,14.87530179)(384.21092529,14.89530273)
\curveto(384.27092141,14.91530175)(384.33092135,14.92030175)(384.39092529,14.91030273)
\curveto(384.45092123,14.91030176)(384.51092117,14.92030175)(384.57092529,14.94030273)
\curveto(384.66092102,14.96030171)(384.74092094,14.97530169)(384.81092529,14.98530273)
\curveto(384.89092079,14.99530167)(384.97092071,15.01530165)(385.05092529,15.04530273)
\curveto(385.36092032,15.1653015)(385.63592004,15.31030136)(385.87592529,15.48030273)
\curveto(386.11591956,15.65030102)(386.32091936,15.88030079)(386.49092529,16.17030273)
}
}
{
\newrgbcolor{curcolor}{0 0 0}
\pscustom[linestyle=none,fillstyle=solid,fillcolor=curcolor]
{
\newpath
\moveto(394.08256592,21.85530273)
\curveto(394.82255954,21.8652948)(395.41755894,21.75529491)(395.86756592,21.52530273)
\curveto(396.31755804,21.30529536)(396.64255772,20.9702957)(396.84256592,20.52030273)
\curveto(396.93255743,20.32029635)(396.99255737,20.07529659)(397.02256592,19.78530273)
\curveto(397.03255733,19.73529693)(397.03255733,19.670297)(397.02256592,19.59030273)
\curveto(397.02255734,19.51029716)(397.00755735,19.44029723)(396.97756592,19.38030273)
\curveto(396.93755742,19.33029734)(396.87755748,19.28529738)(396.79756592,19.24530273)
\curveto(396.7575576,19.22529744)(396.72255764,19.21529745)(396.69256592,19.21530273)
\curveto(396.67255769,19.22529744)(396.63755772,19.22529744)(396.58756592,19.21530273)
\curveto(396.54755781,19.20529746)(396.50755785,19.20029747)(396.46756592,19.20030273)
\curveto(396.42755793,19.21029746)(396.38755797,19.21529745)(396.34756592,19.21530273)
\lineto(396.03256592,19.21530273)
\curveto(395.94255842,19.22529744)(395.86755849,19.25529741)(395.80756592,19.30530273)
\curveto(395.73755862,19.3652973)(395.69755866,19.45029722)(395.68756592,19.56030273)
\curveto(395.67755868,19.670297)(395.6575587,19.7652969)(395.62756592,19.84530273)
\curveto(395.52755883,20.10529656)(395.37255899,20.31029636)(395.16256592,20.46030273)
\curveto(395.09255927,20.51029616)(395.01255935,20.55029612)(394.92256592,20.58030273)
\curveto(394.84255952,20.62029605)(394.7575596,20.65529601)(394.66756592,20.68530273)
\curveto(394.53755982,20.72529594)(394.35756,20.74529592)(394.12756592,20.74530273)
\curveto(393.89756046,20.75529591)(393.70256066,20.73529593)(393.54256592,20.68530273)
\curveto(393.47256089,20.665296)(393.40256096,20.65029602)(393.33256592,20.64030273)
\curveto(393.27256109,20.63029604)(393.20756115,20.61529605)(393.13756592,20.59530273)
\curveto(392.8575615,20.48529618)(392.59756176,20.33529633)(392.35756592,20.14530273)
\curveto(392.11756224,19.95529671)(391.91756244,19.73029694)(391.75756592,19.47030273)
\curveto(391.69756266,19.38029729)(391.64256272,19.28529738)(391.59256592,19.18530273)
\curveto(391.54256282,19.09529757)(391.49256287,18.99529767)(391.44256592,18.88530273)
\lineto(391.27756592,18.48030273)
\curveto(391.2575631,18.43029824)(391.24256312,18.37529829)(391.23256592,18.31530273)
\curveto(391.22256314,18.25529841)(391.20256316,18.20029847)(391.17256592,18.15030273)
\lineto(391.15756592,18.03030273)
\curveto(391.13756322,17.99029868)(391.11256325,17.92529874)(391.08256592,17.83530273)
\curveto(391.0625633,17.74529892)(391.0575633,17.68029899)(391.06756592,17.64030273)
\curveto(391.06756329,17.59029908)(391.0575633,17.54029913)(391.03756592,17.49030273)
\curveto(391.01756334,17.44029923)(391.00756335,17.39029928)(391.00756592,17.34030273)
\curveto(391.01756334,17.30029937)(391.01256335,17.23029944)(390.99256592,17.13030273)
\curveto(390.99256337,17.05029962)(390.98756337,16.9652997)(390.97756592,16.87530273)
\curveto(390.97756338,16.78529988)(390.98256338,16.70029997)(390.99256592,16.62030273)
\curveto(391.03256333,16.30030037)(391.10256326,16.02030065)(391.20256592,15.78030273)
\curveto(391.30256306,15.55030112)(391.46756289,15.35030132)(391.69756592,15.18030273)
\curveto(391.77756258,15.13030154)(391.8575625,15.08530158)(391.93756592,15.04530273)
\curveto(392.02756233,15.00530166)(392.12256224,14.9653017)(392.22256592,14.92530273)
\curveto(392.27256209,14.91530175)(392.31256205,14.91030176)(392.34256592,14.91030273)
\curveto(392.37256199,14.91030176)(392.41256195,14.90530176)(392.46256592,14.89530273)
\curveto(392.49256187,14.88530178)(392.54256182,14.88030179)(392.61256592,14.88030273)
\lineto(392.77756592,14.88030273)
\curveto(392.76756159,14.8703018)(392.78256158,14.8653018)(392.82256592,14.86530273)
\curveto(392.85256151,14.87530179)(392.87756148,14.87530179)(392.89756592,14.86530273)
\curveto(392.92756143,14.8653018)(392.9625614,14.8703018)(393.00256592,14.88030273)
\curveto(393.07256129,14.90030177)(393.13756122,14.90530176)(393.19756592,14.89530273)
\curveto(393.26756109,14.89530177)(393.33756102,14.90530176)(393.40756592,14.92530273)
\curveto(393.68756067,15.00530166)(393.93256043,15.10530156)(394.14256592,15.22530273)
\curveto(394.36256,15.35530131)(394.5575598,15.52030115)(394.72756592,15.72030273)
\curveto(394.78755957,15.80030087)(394.84755951,15.88530078)(394.90756592,15.97530273)
\lineto(395.08756592,16.24530273)
\curveto(395.11755924,16.32530034)(395.15255921,16.40030027)(395.19256592,16.47030273)
\curveto(395.23255913,16.55030012)(395.29255907,16.61530005)(395.37256592,16.66530273)
\curveto(395.41255895,16.69529997)(395.47755888,16.71529995)(395.56756592,16.72530273)
\curveto(395.66755869,16.74529992)(395.76755859,16.75529991)(395.86756592,16.75530273)
\curveto(395.97755838,16.7652999)(396.07755828,16.7652999)(396.16756592,16.75530273)
\curveto(396.2575581,16.74529992)(396.32255804,16.72529994)(396.36256592,16.69530273)
\curveto(396.41255795,16.65530001)(396.43755792,16.59530007)(396.43756592,16.51530273)
\curveto(396.43755792,16.43530023)(396.41755794,16.35030032)(396.37756592,16.26030273)
\curveto(396.29755806,16.11030056)(396.22255814,15.9653007)(396.15256592,15.82530273)
\curveto(396.08255828,15.69530097)(395.99755836,15.5653011)(395.89756592,15.43530273)
\curveto(395.68755867,15.13530153)(395.44755891,14.8703018)(395.17756592,14.64030273)
\curveto(394.90755945,14.41030226)(394.59755976,14.22530244)(394.24756592,14.08530273)
\curveto(394.1575602,14.04530262)(394.0625603,14.01030266)(393.96256592,13.98030273)
\curveto(393.87256049,13.96030271)(393.77756058,13.93530273)(393.67756592,13.90530273)
\curveto(393.5575608,13.8653028)(393.44256092,13.84530282)(393.33256592,13.84530273)
\curveto(393.22256114,13.83530283)(393.10756125,13.82030285)(392.98756592,13.80030273)
\curveto(392.94756141,13.78030289)(392.90756145,13.77530289)(392.86756592,13.78530273)
\curveto(392.82756153,13.79530287)(392.78756157,13.79530287)(392.74756592,13.78530273)
\lineto(392.61256592,13.78530273)
\lineto(392.37256592,13.78530273)
\curveto(392.30256206,13.77530289)(392.23756212,13.78030289)(392.17756592,13.80030273)
\lineto(392.10256592,13.80030273)
\lineto(391.75756592,13.84530273)
\curveto(391.63756272,13.88530278)(391.51756284,13.92030275)(391.39756592,13.95030273)
\curveto(391.28756307,13.98030269)(391.18256318,14.02030265)(391.08256592,14.07030273)
\curveto(390.75256361,14.23030244)(390.49256387,14.42030225)(390.30256592,14.64030273)
\curveto(390.11256425,14.86030181)(389.94756441,15.13030154)(389.80756592,15.45030273)
\curveto(389.77756458,15.53030114)(389.75256461,15.62030105)(389.73256592,15.72030273)
\lineto(389.67256592,16.02030273)
\curveto(389.64256472,16.13030054)(389.62756473,16.24530042)(389.62756592,16.36530273)
\curveto(389.63756472,16.48530018)(389.63756472,16.60530006)(389.62756592,16.72530273)
\curveto(389.62756473,16.7652999)(389.63256473,16.80529986)(389.64256592,16.84530273)
\curveto(389.65256471,16.88529978)(389.65256471,16.92529974)(389.64256592,16.96530273)
\curveto(389.64256472,17.02529964)(389.64756471,17.09029958)(389.65756592,17.16030273)
\curveto(389.67756468,17.23029944)(389.68756467,17.29529937)(389.68756592,17.35530273)
\lineto(389.71756592,17.50530273)
\curveto(389.71756464,17.55529911)(389.72256464,17.62529904)(389.73256592,17.71530273)
\curveto(389.75256461,17.80529886)(389.77256459,17.87529879)(389.79256592,17.92530273)
\curveto(389.81256455,17.97529869)(389.82256454,18.02029865)(389.82256592,18.06030273)
\curveto(389.83256453,18.10029857)(389.84756451,18.14029853)(389.86756592,18.18030273)
\curveto(389.89756446,18.25029842)(389.91756444,18.32029835)(389.92756592,18.39030273)
\curveto(389.93756442,18.46029821)(389.9575644,18.52529814)(389.98756592,18.58530273)
\curveto(390.0575643,18.75529791)(390.12256424,18.92529774)(390.18256592,19.09530273)
\curveto(390.25256411,19.2652974)(390.33256403,19.42529724)(390.42256592,19.57530273)
\curveto(390.74256362,20.09529657)(391.08256328,20.51529615)(391.44256592,20.83530273)
\curveto(391.80256256,21.15529551)(392.26756209,21.42029525)(392.83756592,21.63030273)
\curveto(392.9575614,21.68029499)(393.08256128,21.71529495)(393.21256592,21.73530273)
\curveto(393.34256102,21.75529491)(393.48256088,21.78029489)(393.63256592,21.81030273)
\curveto(393.70256066,21.82029485)(393.77256059,21.82529484)(393.84256592,21.82530273)
\curveto(393.91256045,21.83529483)(393.99256037,21.84529482)(394.08256592,21.85530273)
}
}
{
\newrgbcolor{curcolor}{0 0 0}
\pscustom[linestyle=none,fillstyle=solid,fillcolor=curcolor]
{
\newpath
\moveto(399.54420654,23.17530273)
\curveto(399.47420357,23.23529343)(399.45420359,23.34029333)(399.48420654,23.49030273)
\curveto(399.51420353,23.65029302)(399.5442035,23.80529286)(399.57420654,23.95530273)
\curveto(399.58420346,24.03529263)(399.59920344,24.12029255)(399.61920654,24.21030273)
\curveto(399.6392034,24.30029237)(399.66920337,24.37529229)(399.70920654,24.43530273)
\curveto(399.76920327,24.51529215)(399.85920318,24.57529209)(399.97920654,24.61530273)
\curveto(400.00920303,24.62529204)(400.03420301,24.62529204)(400.05420654,24.61530273)
\curveto(400.07420297,24.61529205)(400.09920294,24.62029205)(400.12920654,24.63030273)
\curveto(400.29920274,24.63029204)(400.45420259,24.62529204)(400.59420654,24.61530273)
\curveto(400.7442023,24.60529206)(400.83420221,24.54529212)(400.86420654,24.43530273)
\curveto(400.88420216,24.37529229)(400.88420216,24.30029237)(400.86420654,24.21030273)
\curveto(400.8442022,24.13029254)(400.82920221,24.04529262)(400.81920654,23.95530273)
\curveto(400.77920226,23.77529289)(400.7392023,23.60529306)(400.69920654,23.44530273)
\curveto(400.66920237,23.28529338)(400.58420246,23.18029349)(400.44420654,23.13030273)
\curveto(400.38420266,23.11029356)(400.32420272,23.10029357)(400.26420654,23.10030273)
\lineto(400.09920654,23.10030273)
\lineto(399.78420654,23.10030273)
\curveto(399.68420336,23.10029357)(399.60420344,23.12529354)(399.54420654,23.17530273)
\moveto(398.95920654,14.67030273)
\curveto(398.9392041,14.5703021)(398.91920412,14.4653022)(398.89920654,14.35530273)
\curveto(398.88920415,14.25530241)(398.84920419,14.17530249)(398.77920654,14.11530273)
\curveto(398.7392043,14.05530261)(398.68920435,14.01530265)(398.62920654,13.99530273)
\curveto(398.56920447,13.98530268)(398.49420455,13.9703027)(398.40420654,13.95030273)
\lineto(398.17920654,13.95030273)
\curveto(398.04920499,13.95030272)(397.9392051,13.95530271)(397.84920654,13.96530273)
\curveto(397.75920528,13.98530268)(397.69420535,14.03530263)(397.65420654,14.11530273)
\curveto(397.63420541,14.17530249)(397.62920541,14.25030242)(397.63920654,14.34030273)
\curveto(397.65920538,14.44030223)(397.67920536,14.53530213)(397.69920654,14.62530273)
\lineto(398.97420654,20.97030273)
\curveto(398.99420405,21.08029559)(399.01420403,21.18529548)(399.03420654,21.28530273)
\curveto(399.05420399,21.39529527)(399.09420395,21.48029519)(399.15420654,21.54030273)
\curveto(399.19420385,21.59029508)(399.2392038,21.62029505)(399.28920654,21.63030273)
\curveto(399.34920369,21.64029503)(399.40920363,21.65529501)(399.46920654,21.67530273)
\curveto(399.48920355,21.67529499)(399.50920353,21.670295)(399.52920654,21.66030273)
\curveto(399.55920348,21.66029501)(399.58420346,21.665295)(399.60420654,21.67530273)
\curveto(399.73420331,21.67529499)(399.86420318,21.670295)(399.99420654,21.66030273)
\curveto(400.13420291,21.66029501)(400.21920282,21.62029505)(400.24920654,21.54030273)
\curveto(400.28920275,21.48029519)(400.29920274,21.40029527)(400.27920654,21.30030273)
\curveto(400.25920278,21.21029546)(400.2392028,21.11529555)(400.21920654,21.01530273)
\lineto(398.95920654,14.67030273)
}
}
{
\newrgbcolor{curcolor}{0 0 0}
\pscustom[linestyle=none,fillstyle=solid,fillcolor=curcolor]
{
\newpath
\moveto(408.71905029,18.15030273)
\curveto(408.7290414,18.09029858)(408.71904141,17.99529867)(408.68905029,17.86530273)
\curveto(408.66904146,17.74529892)(408.64904148,17.66029901)(408.62905029,17.61030273)
\lineto(408.59905029,17.46030273)
\curveto(408.56904156,17.38029929)(408.54404159,17.30529936)(408.52405029,17.23530273)
\curveto(408.51404162,17.17529949)(408.49404164,17.10529956)(408.46405029,17.02530273)
\curveto(408.4340417,16.9652997)(408.40904172,16.90529976)(408.38905029,16.84530273)
\curveto(408.37904175,16.78529988)(408.35404178,16.72529994)(408.31405029,16.66530273)
\lineto(408.13405029,16.27530273)
\curveto(408.08404205,16.14530052)(408.01904211,16.02530064)(407.93905029,15.91530273)
\curveto(407.63904249,15.43530123)(407.27904285,15.03030164)(406.85905029,14.70030273)
\curveto(406.44904368,14.38030229)(405.96904416,14.13530253)(405.41905029,13.96530273)
\curveto(405.30904482,13.92530274)(405.18904494,13.89530277)(405.05905029,13.87530273)
\curveto(404.9290452,13.85530281)(404.79404534,13.83530283)(404.65405029,13.81530273)
\curveto(404.59404554,13.80530286)(404.5290456,13.80030287)(404.45905029,13.80030273)
\curveto(404.39904573,13.79030288)(404.33904579,13.78530288)(404.27905029,13.78530273)
\curveto(404.23904589,13.77530289)(404.17904595,13.7703029)(404.09905029,13.77030273)
\curveto(404.0290461,13.7703029)(403.97904615,13.77530289)(403.94905029,13.78530273)
\curveto(403.90904622,13.79530287)(403.86904626,13.80030287)(403.82905029,13.80030273)
\curveto(403.78904634,13.79030288)(403.75404638,13.79030288)(403.72405029,13.80030273)
\lineto(403.63405029,13.80030273)
\lineto(403.28905029,13.84530273)
\lineto(402.89905029,13.96530273)
\curveto(402.77904735,14.00530266)(402.66404747,14.05030262)(402.55405029,14.10030273)
\curveto(402.14404799,14.30030237)(401.82404831,14.56030211)(401.59405029,14.88030273)
\curveto(401.37404876,15.20030147)(401.21404892,15.59030108)(401.11405029,16.05030273)
\curveto(401.08404905,16.15030052)(401.06404907,16.25030042)(401.05405029,16.35030273)
\lineto(401.05405029,16.66530273)
\curveto(401.04404909,16.70529996)(401.04404909,16.73529993)(401.05405029,16.75530273)
\curveto(401.06404907,16.78529988)(401.06904906,16.82029985)(401.06905029,16.86030273)
\curveto(401.06904906,16.94029973)(401.07404906,17.02029965)(401.08405029,17.10030273)
\curveto(401.09404904,17.19029948)(401.09904903,17.27529939)(401.09905029,17.35530273)
\curveto(401.10904902,17.40529926)(401.11404902,17.44529922)(401.11405029,17.47530273)
\curveto(401.12404901,17.51529915)(401.129049,17.56029911)(401.12905029,17.61030273)
\curveto(401.129049,17.66029901)(401.13904899,17.74529892)(401.15905029,17.86530273)
\curveto(401.18904894,17.99529867)(401.21904891,18.09029858)(401.24905029,18.15030273)
\curveto(401.28904884,18.22029845)(401.30904882,18.29029838)(401.30905029,18.36030273)
\curveto(401.30904882,18.43029824)(401.3290488,18.50029817)(401.36905029,18.57030273)
\curveto(401.38904874,18.62029805)(401.40404873,18.66029801)(401.41405029,18.69030273)
\curveto(401.42404871,18.73029794)(401.43904869,18.77529789)(401.45905029,18.82530273)
\curveto(401.51904861,18.94529772)(401.56904856,19.0652976)(401.60905029,19.18530273)
\curveto(401.65904847,19.30529736)(401.72404841,19.42029725)(401.80405029,19.53030273)
\curveto(402.02404811,19.90029677)(402.26904786,20.23029644)(402.53905029,20.52030273)
\curveto(402.81904731,20.82029585)(403.134047,21.0702956)(403.48405029,21.27030273)
\curveto(403.61404652,21.35029532)(403.74904638,21.41529525)(403.88905029,21.46530273)
\lineto(404.33905029,21.64530273)
\curveto(404.46904566,21.69529497)(404.60404553,21.72529494)(404.74405029,21.73530273)
\curveto(404.88404525,21.75529491)(405.0290451,21.78529488)(405.17905029,21.82530273)
\lineto(405.37405029,21.82530273)
\lineto(405.58405029,21.85530273)
\curveto(406.47404366,21.8652948)(407.17404296,21.68029499)(407.68405029,21.30030273)
\curveto(408.20404193,20.92029575)(408.5290416,20.42529624)(408.65905029,19.81530273)
\curveto(408.68904144,19.71529695)(408.70904142,19.61529705)(408.71905029,19.51530273)
\curveto(408.7290414,19.41529725)(408.74404139,19.31029736)(408.76405029,19.20030273)
\curveto(408.77404136,19.09029758)(408.77404136,18.9702977)(408.76405029,18.84030273)
\lineto(408.76405029,18.46530273)
\curveto(408.76404137,18.41529825)(408.75404138,18.36029831)(408.73405029,18.30030273)
\curveto(408.72404141,18.25029842)(408.71904141,18.20029847)(408.71905029,18.15030273)
\moveto(407.21905029,17.29530273)
\curveto(407.24904288,17.3652993)(407.26904286,17.44529922)(407.27905029,17.53530273)
\curveto(407.29904283,17.62529904)(407.31404282,17.71029896)(407.32405029,17.79030273)
\curveto(407.40404273,18.18029849)(407.43904269,18.51029816)(407.42905029,18.78030273)
\curveto(407.40904272,18.86029781)(407.39404274,18.94029773)(407.38405029,19.02030273)
\curveto(407.38404275,19.10029757)(407.37904275,19.17529749)(407.36905029,19.24530273)
\curveto(407.21904291,19.89529677)(406.86404327,20.34529632)(406.30405029,20.59530273)
\curveto(406.2340439,20.62529604)(406.15904397,20.64529602)(406.07905029,20.65530273)
\curveto(406.00904412,20.67529599)(405.9340442,20.69529597)(405.85405029,20.71530273)
\curveto(405.78404435,20.73529593)(405.70404443,20.74529592)(405.61405029,20.74530273)
\lineto(405.34405029,20.74530273)
\lineto(405.05905029,20.70030273)
\curveto(404.95904517,20.68029599)(404.86404527,20.65529601)(404.77405029,20.62530273)
\curveto(404.68404545,20.60529606)(404.59404554,20.57529609)(404.50405029,20.53530273)
\curveto(404.4340457,20.51529615)(404.36404577,20.48529618)(404.29405029,20.44530273)
\curveto(404.22404591,20.40529626)(404.15904597,20.3652963)(404.09905029,20.32530273)
\curveto(403.8290463,20.15529651)(403.59404654,19.95029672)(403.39405029,19.71030273)
\curveto(403.19404694,19.4702972)(403.00904712,19.19029748)(402.83905029,18.87030273)
\curveto(402.78904734,18.7702979)(402.74904738,18.665298)(402.71905029,18.55530273)
\curveto(402.68904744,18.45529821)(402.64904748,18.35029832)(402.59905029,18.24030273)
\curveto(402.58904754,18.20029847)(402.57404756,18.13529853)(402.55405029,18.04530273)
\curveto(402.5340476,18.01529865)(402.52404761,17.98029869)(402.52405029,17.94030273)
\curveto(402.52404761,17.90029877)(402.51904761,17.85529881)(402.50905029,17.80530273)
\lineto(402.44905029,17.50530273)
\curveto(402.4290477,17.40529926)(402.41904771,17.31529935)(402.41905029,17.23530273)
\lineto(402.41905029,17.05530273)
\curveto(402.41904771,16.95529971)(402.41404772,16.85529981)(402.40405029,16.75530273)
\curveto(402.40404773,16.6653)(402.41404772,16.58030009)(402.43405029,16.50030273)
\curveto(402.48404765,16.26030041)(402.55404758,16.03530063)(402.64405029,15.82530273)
\curveto(402.74404739,15.61530105)(402.87904725,15.44030123)(403.04905029,15.30030273)
\curveto(403.09904703,15.2703014)(403.13904699,15.24530142)(403.16905029,15.22530273)
\curveto(403.20904692,15.20530146)(403.24904688,15.18030149)(403.28905029,15.15030273)
\curveto(403.35904677,15.10030157)(403.43904669,15.05530161)(403.52905029,15.01530273)
\curveto(403.61904651,14.98530168)(403.71404642,14.95530171)(403.81405029,14.92530273)
\curveto(403.86404627,14.90530176)(403.90904622,14.89530177)(403.94905029,14.89530273)
\curveto(403.99904613,14.90530176)(404.04904608,14.90530176)(404.09905029,14.89530273)
\curveto(404.129046,14.88530178)(404.18904594,14.87530179)(404.27905029,14.86530273)
\curveto(404.36904576,14.85530181)(404.44404569,14.86030181)(404.50405029,14.88030273)
\curveto(404.54404559,14.89030178)(404.58404555,14.89030178)(404.62405029,14.88030273)
\curveto(404.66404547,14.88030179)(404.70404543,14.89030178)(404.74405029,14.91030273)
\curveto(404.82404531,14.93030174)(404.90404523,14.94530172)(404.98405029,14.95530273)
\curveto(405.07404506,14.97530169)(405.15904497,15.00030167)(405.23905029,15.03030273)
\curveto(405.59904453,15.1703015)(405.90904422,15.3653013)(406.16905029,15.61530273)
\curveto(406.4290437,15.8653008)(406.66404347,16.16030051)(406.87405029,16.50030273)
\curveto(406.95404318,16.62030005)(407.01404312,16.74529992)(407.05405029,16.87530273)
\curveto(407.09404304,17.01529965)(407.14904298,17.15529951)(407.21905029,17.29530273)
}
}
{
\newrgbcolor{curcolor}{0 0 0}
\pscustom[linestyle=none,fillstyle=solid,fillcolor=curcolor]
{
\newpath
\moveto(413.38733154,21.85530273)
\curveto(414.10732589,21.8652948)(414.6923253,21.78029489)(415.14233154,21.60030273)
\curveto(415.60232439,21.43029524)(415.92232407,21.12529554)(416.10233154,20.68530273)
\curveto(416.15232384,20.57529609)(416.18232381,20.46029621)(416.19233154,20.34030273)
\curveto(416.21232378,20.23029644)(416.22732377,20.10529656)(416.23733154,19.96530273)
\curveto(416.24732375,19.89529677)(416.23732376,19.82029685)(416.20733154,19.74030273)
\curveto(416.18732381,19.670297)(416.16232383,19.61529705)(416.13233154,19.57530273)
\curveto(416.11232388,19.55529711)(416.08232391,19.53529713)(416.04233154,19.51530273)
\curveto(416.01232398,19.50529716)(415.98732401,19.49029718)(415.96733154,19.47030273)
\curveto(415.90732409,19.45029722)(415.85232414,19.44529722)(415.80233154,19.45530273)
\curveto(415.76232423,19.4652972)(415.71732428,19.4652972)(415.66733154,19.45530273)
\curveto(415.57732442,19.43529723)(415.46732453,19.43029724)(415.33733154,19.44030273)
\curveto(415.21732478,19.46029721)(415.13232486,19.48529718)(415.08233154,19.51530273)
\curveto(415.01232498,19.5652971)(414.97232502,19.63029704)(414.96233154,19.71030273)
\curveto(414.96232503,19.80029687)(414.94232505,19.88529678)(414.90233154,19.96530273)
\curveto(414.85232514,20.12529654)(414.75732524,20.2702964)(414.61733154,20.40030273)
\curveto(414.52732547,20.48029619)(414.41732558,20.54029613)(414.28733154,20.58030273)
\curveto(414.16732583,20.62029605)(414.03732596,20.66029601)(413.89733154,20.70030273)
\curveto(413.85732614,20.72029595)(413.80732619,20.72529594)(413.74733154,20.71530273)
\curveto(413.6973263,20.71529595)(413.65232634,20.72029595)(413.61233154,20.73030273)
\curveto(413.55232644,20.75029592)(413.47732652,20.76029591)(413.38733154,20.76030273)
\curveto(413.2973267,20.76029591)(413.22232677,20.75029592)(413.16233154,20.73030273)
\lineto(413.07233154,20.73030273)
\curveto(413.01232698,20.72029595)(412.95732704,20.71029596)(412.90733154,20.70030273)
\curveto(412.85732714,20.70029597)(412.80732719,20.69529597)(412.75733154,20.68530273)
\curveto(412.48732751,20.62529604)(412.25232774,20.54029613)(412.05233154,20.43030273)
\curveto(411.86232813,20.32029635)(411.71232828,20.13529653)(411.60233154,19.87530273)
\curveto(411.57232842,19.80529686)(411.55732844,19.73529693)(411.55733154,19.66530273)
\curveto(411.55732844,19.59529707)(411.56232843,19.53529713)(411.57233154,19.48530273)
\curveto(411.60232839,19.33529733)(411.65232834,19.22529744)(411.72233154,19.15530273)
\curveto(411.7923282,19.09529757)(411.88732811,19.02529764)(412.00733154,18.94530273)
\curveto(412.14732785,18.84529782)(412.31232768,18.7702979)(412.50233154,18.72030273)
\curveto(412.6923273,18.68029799)(412.88232711,18.63029804)(413.07233154,18.57030273)
\curveto(413.1923268,18.53029814)(413.31232668,18.50029817)(413.43233154,18.48030273)
\curveto(413.56232643,18.46029821)(413.68732631,18.43029824)(413.80733154,18.39030273)
\curveto(414.00732599,18.33029834)(414.20232579,18.2702984)(414.39233154,18.21030273)
\curveto(414.58232541,18.16029851)(414.76732523,18.09529857)(414.94733154,18.01530273)
\curveto(414.997325,17.99529867)(415.04232495,17.97529869)(415.08233154,17.95530273)
\curveto(415.13232486,17.93529873)(415.18232481,17.91029876)(415.23233154,17.88030273)
\curveto(415.40232459,17.76029891)(415.54732445,17.62529904)(415.66733154,17.47530273)
\curveto(415.78732421,17.32529934)(415.87732412,17.13529953)(415.93733154,16.90530273)
\lineto(415.93733154,16.62030273)
\curveto(415.93732406,16.55030012)(415.93232406,16.47530019)(415.92233154,16.39530273)
\curveto(415.91232408,16.32530034)(415.90232409,16.24530042)(415.89233154,16.15530273)
\lineto(415.86233154,16.00530273)
\curveto(415.82232417,15.93530073)(415.7923242,15.8653008)(415.77233154,15.79530273)
\curveto(415.76232423,15.72530094)(415.74232425,15.65530101)(415.71233154,15.58530273)
\curveto(415.66232433,15.47530119)(415.60732439,15.3703013)(415.54733154,15.27030273)
\curveto(415.48732451,15.1703015)(415.42232457,15.08030159)(415.35233154,15.00030273)
\curveto(415.14232485,14.74030193)(414.8973251,14.53030214)(414.61733154,14.37030273)
\curveto(414.33732566,14.22030245)(414.03232596,14.09030258)(413.70233154,13.98030273)
\curveto(413.60232639,13.95030272)(413.50232649,13.93030274)(413.40233154,13.92030273)
\curveto(413.30232669,13.90030277)(413.20732679,13.87530279)(413.11733154,13.84530273)
\curveto(413.00732699,13.82530284)(412.90232709,13.81530285)(412.80233154,13.81530273)
\curveto(412.70232729,13.81530285)(412.60232739,13.80530286)(412.50233154,13.78530273)
\lineto(412.35233154,13.78530273)
\curveto(412.30232769,13.77530289)(412.23232776,13.7703029)(412.14233154,13.77030273)
\curveto(412.05232794,13.7703029)(411.98232801,13.77530289)(411.93233154,13.78530273)
\lineto(411.76733154,13.78530273)
\curveto(411.70732829,13.80530286)(411.64232835,13.81530285)(411.57233154,13.81530273)
\curveto(411.50232849,13.80530286)(411.44732855,13.81030286)(411.40733154,13.83030273)
\curveto(411.35732864,13.84030283)(411.2923287,13.84530282)(411.21233154,13.84530273)
\curveto(411.13232886,13.8653028)(411.05732894,13.88530278)(410.98733154,13.90530273)
\curveto(410.91732908,13.91530275)(410.84232915,13.93530273)(410.76233154,13.96530273)
\curveto(410.47232952,14.0653026)(410.22732977,14.19030248)(410.02733154,14.34030273)
\curveto(409.82733017,14.49030218)(409.66733033,14.68530198)(409.54733154,14.92530273)
\curveto(409.48733051,15.05530161)(409.43733056,15.19030148)(409.39733154,15.33030273)
\curveto(409.36733063,15.4703012)(409.34733065,15.62530104)(409.33733154,15.79530273)
\curveto(409.32733067,15.85530081)(409.33233066,15.92530074)(409.35233154,16.00530273)
\curveto(409.37233062,16.09530057)(409.3973306,16.1653005)(409.42733154,16.21530273)
\curveto(409.46733053,16.25530041)(409.52733047,16.29530037)(409.60733154,16.33530273)
\curveto(409.65733034,16.35530031)(409.72733027,16.3653003)(409.81733154,16.36530273)
\curveto(409.91733008,16.37530029)(410.00732999,16.37530029)(410.08733154,16.36530273)
\curveto(410.17732982,16.35530031)(410.26232973,16.34030033)(410.34233154,16.32030273)
\curveto(410.43232956,16.31030036)(410.48732951,16.29530037)(410.50733154,16.27530273)
\curveto(410.56732943,16.22530044)(410.5973294,16.15030052)(410.59733154,16.05030273)
\curveto(410.60732939,15.96030071)(410.62732937,15.87530079)(410.65733154,15.79530273)
\curveto(410.70732929,15.57530109)(410.80732919,15.40530126)(410.95733154,15.28530273)
\curveto(411.05732894,15.19530147)(411.17732882,15.12530154)(411.31733154,15.07530273)
\curveto(411.45732854,15.02530164)(411.60732839,14.97530169)(411.76733154,14.92530273)
\lineto(412.08233154,14.88030273)
\lineto(412.17233154,14.88030273)
\curveto(412.23232776,14.86030181)(412.31732768,14.85030182)(412.42733154,14.85030273)
\curveto(412.54732745,14.85030182)(412.65232734,14.86030181)(412.74233154,14.88030273)
\curveto(412.81232718,14.88030179)(412.86732713,14.88530178)(412.90733154,14.89530273)
\curveto(412.96732703,14.90530176)(413.02732697,14.91030176)(413.08733154,14.91030273)
\curveto(413.14732685,14.92030175)(413.20232679,14.93030174)(413.25233154,14.94030273)
\curveto(413.56232643,15.02030165)(413.81232618,15.12530154)(414.00233154,15.25530273)
\curveto(414.20232579,15.38530128)(414.36732563,15.60530106)(414.49733154,15.91530273)
\curveto(414.52732547,15.9653007)(414.54232545,16.02030065)(414.54233154,16.08030273)
\curveto(414.55232544,16.14030053)(414.55232544,16.18530048)(414.54233154,16.21530273)
\curveto(414.53232546,16.40530026)(414.4923255,16.54530012)(414.42233154,16.63530273)
\curveto(414.35232564,16.73529993)(414.25732574,16.82529984)(414.13733154,16.90530273)
\curveto(414.05732594,16.9652997)(413.96232603,17.01529965)(413.85233154,17.05530273)
\lineto(413.55233154,17.17530273)
\curveto(413.52232647,17.18529948)(413.4923265,17.19029948)(413.46233154,17.19030273)
\curveto(413.44232655,17.19029948)(413.42232657,17.20029947)(413.40233154,17.22030273)
\curveto(413.08232691,17.33029934)(412.74232725,17.41029926)(412.38233154,17.46030273)
\curveto(412.03232796,17.52029915)(411.71232828,17.61529905)(411.42233154,17.74530273)
\curveto(411.33232866,17.78529888)(411.24232875,17.82029885)(411.15233154,17.85030273)
\curveto(411.07232892,17.88029879)(410.997329,17.92029875)(410.92733154,17.97030273)
\curveto(410.75732924,18.08029859)(410.60732939,18.20529846)(410.47733154,18.34530273)
\curveto(410.34732965,18.48529818)(410.25732974,18.66029801)(410.20733154,18.87030273)
\curveto(410.18732981,18.94029773)(410.17732982,19.01029766)(410.17733154,19.08030273)
\lineto(410.17733154,19.30530273)
\curveto(410.16732983,19.42529724)(410.18232981,19.56029711)(410.22233154,19.71030273)
\curveto(410.26232973,19.8702968)(410.30232969,20.00529666)(410.34233154,20.11530273)
\curveto(410.37232962,20.1652965)(410.3923296,20.20529646)(410.40233154,20.23530273)
\curveto(410.42232957,20.27529639)(410.44732955,20.31529635)(410.47733154,20.35530273)
\curveto(410.60732939,20.58529608)(410.76732923,20.78529588)(410.95733154,20.95530273)
\curveto(411.14732885,21.12529554)(411.35732864,21.27529539)(411.58733154,21.40530273)
\curveto(411.74732825,21.49529517)(411.92232807,21.5652951)(412.11233154,21.61530273)
\curveto(412.31232768,21.67529499)(412.51732748,21.73029494)(412.72733154,21.78030273)
\curveto(412.7973272,21.79029488)(412.86232713,21.80029487)(412.92233154,21.81030273)
\curveto(412.992327,21.82029485)(413.06732693,21.83029484)(413.14733154,21.84030273)
\curveto(413.18732681,21.85029482)(413.22732677,21.85029482)(413.26733154,21.84030273)
\curveto(413.31732668,21.83029484)(413.35732664,21.83529483)(413.38733154,21.85530273)
}
}
{
\newrgbcolor{curcolor}{0 0 0}
\pscustom[linestyle=none,fillstyle=solid,fillcolor=curcolor]
{
}
}
{
\newrgbcolor{curcolor}{0 0 0}
\pscustom[linestyle=none,fillstyle=solid,fillcolor=curcolor]
{
\newpath
\moveto(428.15248779,14.76030273)
\lineto(428.06248779,14.37030273)
\curveto(428.04247986,14.25030242)(428.0024799,14.15030252)(427.94248779,14.07030273)
\curveto(427.87248003,14.00030267)(427.77748013,13.96030271)(427.65748779,13.95030273)
\lineto(427.31248779,13.95030273)
\curveto(427.25248065,13.95030272)(427.19248071,13.94530272)(427.13248779,13.93530273)
\curveto(427.08248082,13.93530273)(427.03748087,13.94530272)(426.99748779,13.96530273)
\curveto(426.91748099,13.98530268)(426.86748104,14.02530264)(426.84748779,14.08530273)
\curveto(426.81748109,14.13530253)(426.8074811,14.19530247)(426.81748779,14.26530273)
\curveto(426.82748108,14.33530233)(426.82248108,14.40530226)(426.80248779,14.47530273)
\curveto(426.8024811,14.49530217)(426.79248111,14.51030216)(426.77248779,14.52030273)
\lineto(426.74248779,14.58030273)
\curveto(426.64248126,14.59030208)(426.55748135,14.5703021)(426.48748779,14.52030273)
\curveto(426.42748148,14.4703022)(426.36248154,14.42030225)(426.29248779,14.37030273)
\curveto(426.06248184,14.22030245)(425.83748207,14.10530256)(425.61748779,14.02530273)
\curveto(425.42748248,13.94530272)(425.2074827,13.88530278)(424.95748779,13.84530273)
\curveto(424.71748319,13.80530286)(424.47248343,13.78530288)(424.22248779,13.78530273)
\curveto(423.98248392,13.77530289)(423.74248416,13.79030288)(423.50248779,13.83030273)
\curveto(423.27248463,13.86030281)(423.07748483,13.91530275)(422.91748779,13.99530273)
\curveto(422.43748547,14.21530245)(422.07248583,14.51030216)(421.82248779,14.88030273)
\curveto(421.58248632,15.26030141)(421.42748648,15.73030094)(421.35748779,16.29030273)
\curveto(421.33748657,16.38030029)(421.32748658,16.4703002)(421.32748779,16.56030273)
\curveto(421.33748657,16.66030001)(421.33748657,16.76029991)(421.32748779,16.86030273)
\curveto(421.32748658,16.91029976)(421.33248657,16.96029971)(421.34248779,17.01030273)
\curveto(421.35248655,17.06029961)(421.35748655,17.11029956)(421.35748779,17.16030273)
\curveto(421.34748656,17.21029946)(421.34748656,17.26029941)(421.35748779,17.31030273)
\curveto(421.37748653,17.3702993)(421.38748652,17.42529924)(421.38748779,17.47530273)
\lineto(421.41748779,17.62530273)
\curveto(421.4074865,17.67529899)(421.4074865,17.74029893)(421.41748779,17.82030273)
\curveto(421.43748647,17.90029877)(421.46248644,17.9652987)(421.49248779,18.01530273)
\lineto(421.53748779,18.18030273)
\curveto(421.56748634,18.25029842)(421.58748632,18.32029835)(421.59748779,18.39030273)
\curveto(421.6074863,18.4702982)(421.62748628,18.54529812)(421.65748779,18.61530273)
\curveto(421.67748623,18.665298)(421.69248621,18.71029796)(421.70248779,18.75030273)
\curveto(421.71248619,18.79029788)(421.72748618,18.83529783)(421.74748779,18.88530273)
\curveto(421.79748611,18.98529768)(421.84248606,19.08029759)(421.88248779,19.17030273)
\curveto(421.92248598,19.2702974)(421.96748594,19.3652973)(422.01748779,19.45530273)
\curveto(422.21748569,19.83529683)(422.44748546,20.17529649)(422.70748779,20.47530273)
\curveto(422.97748493,20.78529588)(423.27748463,21.04029563)(423.60748779,21.24030273)
\curveto(423.8074841,21.36029531)(424.0074839,21.46029521)(424.20748779,21.54030273)
\curveto(424.4074835,21.62029505)(424.62248328,21.69029498)(424.85248779,21.75030273)
\lineto(425.06248779,21.78030273)
\curveto(425.13248277,21.79029488)(425.2024827,21.80529486)(425.27248779,21.82530273)
\lineto(425.42248779,21.82530273)
\curveto(425.51248239,21.84529482)(425.63248227,21.85529481)(425.78248779,21.85530273)
\curveto(425.94248196,21.85529481)(426.05748185,21.84529482)(426.12748779,21.82530273)
\curveto(426.16748174,21.81529485)(426.22248168,21.81029486)(426.29248779,21.81030273)
\curveto(426.39248151,21.78029489)(426.49748141,21.75529491)(426.60748779,21.73530273)
\curveto(426.71748119,21.72529494)(426.81748109,21.69529497)(426.90748779,21.64530273)
\curveto(427.04748086,21.58529508)(427.17748073,21.52029515)(427.29748779,21.45030273)
\curveto(427.41748049,21.38029529)(427.52748038,21.30029537)(427.62748779,21.21030273)
\curveto(427.67748023,21.16029551)(427.72748018,21.10529556)(427.77748779,21.04530273)
\curveto(427.83748007,20.99529567)(427.92247998,20.98029569)(428.03248779,21.00030273)
\lineto(428.10748779,21.07530273)
\curveto(428.12747978,21.09529557)(428.14247976,21.12529554)(428.15248779,21.16530273)
\curveto(428.2024797,21.25529541)(428.23747967,21.3702953)(428.25748779,21.51030273)
\curveto(428.28747962,21.65029502)(428.31247959,21.77529489)(428.33248779,21.88530273)
\lineto(428.67748779,23.61030273)
\curveto(428.7074792,23.75029292)(428.73747917,23.90529276)(428.76748779,24.07530273)
\curveto(428.8074791,24.25529241)(428.85747905,24.38529228)(428.91748779,24.46530273)
\curveto(428.97747893,24.53529213)(429.04747886,24.58029209)(429.12748779,24.60030273)
\curveto(429.14747876,24.60029207)(429.17247873,24.60029207)(429.20248779,24.60030273)
\curveto(429.23247867,24.61029206)(429.25747865,24.61529205)(429.27748779,24.61530273)
\curveto(429.42747848,24.62529204)(429.57747833,24.62529204)(429.72748779,24.61530273)
\curveto(429.87747803,24.61529205)(429.97747793,24.57529209)(430.02748779,24.49530273)
\curveto(430.05747785,24.41529225)(430.05747785,24.31529235)(430.02748779,24.19530273)
\curveto(430.0074779,24.07529259)(429.98747792,23.95029272)(429.96748779,23.82030273)
\lineto(428.15248779,14.76030273)
\moveto(427.50748779,17.59530273)
\curveto(427.53748037,17.64529902)(427.55748035,17.71029896)(427.56748779,17.79030273)
\curveto(427.58748032,17.88029879)(427.59248031,17.95029872)(427.58248779,18.00030273)
\lineto(427.62748779,18.22530273)
\curveto(427.62748028,18.31529835)(427.63248027,18.40529826)(427.64248779,18.49530273)
\curveto(427.65248025,18.59529807)(427.64748026,18.68529798)(427.62748779,18.76530273)
\lineto(427.62748779,18.99030273)
\curveto(427.62748028,19.06029761)(427.61748029,19.13029754)(427.59748779,19.20030273)
\curveto(427.53748037,19.50029717)(427.43248047,19.7652969)(427.28248779,19.99530273)
\curveto(427.14248076,20.22529644)(426.94248096,20.40529626)(426.68248779,20.53530273)
\curveto(426.59248131,20.58529608)(426.49748141,20.62029605)(426.39748779,20.64030273)
\curveto(426.29748161,20.670296)(426.18748172,20.69529597)(426.06748779,20.71530273)
\curveto(425.99748191,20.73529593)(425.91248199,20.74529592)(425.81248779,20.74530273)
\lineto(425.54248779,20.74530273)
\lineto(425.39248779,20.71530273)
\lineto(425.25748779,20.71530273)
\curveto(425.17748273,20.69529597)(425.09248281,20.67529599)(425.00248779,20.65530273)
\curveto(424.91248299,20.63529603)(424.82748308,20.61029606)(424.74748779,20.58030273)
\curveto(424.39748351,20.44029623)(424.09748381,20.23529643)(423.84748779,19.96530273)
\curveto(423.59748431,19.70529696)(423.37748453,19.40029727)(423.18748779,19.05030273)
\curveto(423.12748478,18.94029773)(423.07748483,18.82529784)(423.03748779,18.70530273)
\lineto(422.91748779,18.37530273)
\lineto(422.88748779,18.25530273)
\curveto(422.87748503,18.22529844)(422.86748504,18.19029848)(422.85748779,18.15030273)
\curveto(422.82748508,18.10029857)(422.8074851,18.04529862)(422.79748779,17.98530273)
\curveto(422.79748511,17.92529874)(422.79248511,17.8702988)(422.78248779,17.82030273)
\curveto(422.76248514,17.71029896)(422.73748517,17.60029907)(422.70748779,17.49030273)
\curveto(422.68748522,17.39029928)(422.68248522,17.29529937)(422.69248779,17.20530273)
\curveto(422.69248521,17.17529949)(422.68748522,17.12529954)(422.67748779,17.05530273)
\lineto(422.67748779,16.84530273)
\curveto(422.67748523,16.77529989)(422.68248522,16.70529996)(422.69248779,16.63530273)
\curveto(422.73248517,16.28530038)(422.82248508,15.98530068)(422.96248779,15.73530273)
\curveto(423.1024848,15.48530118)(423.3024846,15.28030139)(423.56248779,15.12030273)
\curveto(423.64248426,15.0703016)(423.72248418,15.03030164)(423.80248779,15.00030273)
\curveto(423.89248401,14.9703017)(423.98748392,14.94030173)(424.08748779,14.91030273)
\curveto(424.13748377,14.89030178)(424.18748372,14.88530178)(424.23748779,14.89530273)
\curveto(424.29748361,14.90530176)(424.35248355,14.90030177)(424.40248779,14.88030273)
\curveto(424.43248347,14.8703018)(424.46748344,14.8653018)(424.50748779,14.86530273)
\lineto(424.64248779,14.86530273)
\lineto(424.77748779,14.86530273)
\curveto(424.81748309,14.87530179)(424.87248303,14.88030179)(424.94248779,14.88030273)
\curveto(425.02248288,14.90030177)(425.1024828,14.91530175)(425.18248779,14.92530273)
\curveto(425.27248263,14.94530172)(425.35248255,14.9703017)(425.42248779,15.00030273)
\curveto(425.78248212,15.14030153)(426.08748182,15.31530135)(426.33748779,15.52530273)
\curveto(426.58748132,15.74530092)(426.81248109,16.02030065)(427.01248779,16.35030273)
\curveto(427.08248082,16.46030021)(427.13748077,16.5703001)(427.17748779,16.68030273)
\lineto(427.32748779,17.01030273)
\curveto(427.35748055,17.05029962)(427.37248053,17.08529958)(427.37248779,17.11530273)
\curveto(427.38248052,17.15529951)(427.39748051,17.19529947)(427.41748779,17.23530273)
\curveto(427.43748047,17.29529937)(427.45248045,17.35529931)(427.46248779,17.41530273)
\curveto(427.47248043,17.47529919)(427.48748042,17.53529913)(427.50748779,17.59530273)
}
}
{
\newrgbcolor{curcolor}{0 0 0}
\pscustom[linestyle=none,fillstyle=solid,fillcolor=curcolor]
{
\newpath
\moveto(437.52373779,18.12030273)
\curveto(437.52372929,18.02029865)(437.50372931,17.90529876)(437.46373779,17.77530273)
\curveto(437.42372939,17.65529901)(437.37372944,17.5702991)(437.31373779,17.52030273)
\curveto(437.25372956,17.48029919)(437.17372964,17.45029922)(437.07373779,17.43030273)
\curveto(436.97372984,17.42029925)(436.86372995,17.41529925)(436.74373779,17.41530273)
\lineto(436.38373779,17.41530273)
\curveto(436.27373054,17.42529924)(436.17373064,17.43029924)(436.08373779,17.43030273)
\lineto(432.24373779,17.43030273)
\curveto(432.16373465,17.43029924)(432.07873473,17.42529924)(431.98873779,17.41530273)
\curveto(431.9087349,17.41529925)(431.84373497,17.40029927)(431.79373779,17.37030273)
\curveto(431.74373507,17.35029932)(431.69373512,17.31029936)(431.64373779,17.25030273)
\lineto(431.55373779,17.11530273)
\curveto(431.52373529,17.0652996)(431.5137353,17.01529965)(431.52373779,16.96530273)
\curveto(431.52373529,16.91529975)(431.51873529,16.8702998)(431.50873779,16.83030273)
\lineto(431.50873779,16.71030273)
\lineto(431.50873779,16.45530273)
\curveto(431.51873529,16.37530029)(431.53373528,16.29530037)(431.55373779,16.21530273)
\curveto(431.68373513,15.67530099)(431.98873482,15.29030138)(432.46873779,15.06030273)
\curveto(432.51873429,15.03030164)(432.57873423,15.00530166)(432.64873779,14.98530273)
\curveto(432.71873409,14.9653017)(432.78373403,14.94530172)(432.84373779,14.92530273)
\curveto(432.87373394,14.91530175)(432.92373389,14.91030176)(432.99373779,14.91030273)
\curveto(433.12373369,14.8703018)(433.30373351,14.85030182)(433.53373779,14.85030273)
\curveto(433.76373305,14.85030182)(433.95373286,14.8703018)(434.10373779,14.91030273)
\curveto(434.25373256,14.95030172)(434.38873242,14.99030168)(434.50873779,15.03030273)
\curveto(434.63873217,15.08030159)(434.75873205,15.14030153)(434.86873779,15.21030273)
\curveto(434.98873182,15.28030139)(435.09873171,15.36030131)(435.19873779,15.45030273)
\curveto(435.29873151,15.55030112)(435.38873142,15.65530101)(435.46873779,15.76530273)
\curveto(435.54873126,15.8653008)(435.62373119,15.9703007)(435.69373779,16.08030273)
\curveto(435.76373105,16.19030048)(435.85873095,16.2703004)(435.97873779,16.32030273)
\curveto(436.01873079,16.34030033)(436.08373073,16.35530031)(436.17373779,16.36530273)
\curveto(436.27373054,16.37530029)(436.36373045,16.37530029)(436.44373779,16.36530273)
\curveto(436.53373028,16.3653003)(436.61873019,16.36030031)(436.69873779,16.35030273)
\curveto(436.77873003,16.34030033)(436.82872998,16.32030035)(436.84873779,16.29030273)
\curveto(436.93872987,16.22030045)(436.94372987,16.10530056)(436.86373779,15.94530273)
\curveto(436.72373009,15.67530099)(436.56873024,15.43530123)(436.39873779,15.22530273)
\curveto(436.13873067,14.90530176)(435.85873095,14.64030203)(435.55873779,14.43030273)
\curveto(435.26873154,14.23030244)(434.9137319,14.0653026)(434.49373779,13.93530273)
\curveto(434.38373243,13.89530277)(434.27873253,13.8703028)(434.17873779,13.86030273)
\curveto(434.07873273,13.84030283)(433.96873284,13.82030285)(433.84873779,13.80030273)
\curveto(433.79873301,13.79030288)(433.74873306,13.78530288)(433.69873779,13.78530273)
\curveto(433.65873315,13.78530288)(433.6137332,13.78030289)(433.56373779,13.77030273)
\lineto(433.41373779,13.77030273)
\curveto(433.36373345,13.76030291)(433.30373351,13.75530291)(433.23373779,13.75530273)
\curveto(433.17373364,13.75530291)(433.12373369,13.76030291)(433.08373779,13.77030273)
\lineto(432.94873779,13.77030273)
\curveto(432.89873391,13.78030289)(432.85373396,13.78530288)(432.81373779,13.78530273)
\curveto(432.77373404,13.78530288)(432.73373408,13.79030288)(432.69373779,13.80030273)
\curveto(432.64373417,13.81030286)(432.58873422,13.82030285)(432.52873779,13.83030273)
\curveto(432.47873433,13.83030284)(432.42873438,13.83530283)(432.37873779,13.84530273)
\curveto(432.28873452,13.8653028)(432.19873461,13.89030278)(432.10873779,13.92030273)
\curveto(432.02873478,13.94030273)(431.95373486,13.9653027)(431.88373779,13.99530273)
\curveto(431.84373497,14.01530265)(431.808735,14.02530264)(431.77873779,14.02530273)
\curveto(431.74873506,14.03530263)(431.71873509,14.05030262)(431.68873779,14.07030273)
\curveto(431.54873526,14.14030253)(431.40373541,14.22530244)(431.25373779,14.32530273)
\curveto(431.00373581,14.51530215)(430.80373601,14.74530192)(430.65373779,15.01530273)
\curveto(430.50373631,15.29530137)(430.39373642,15.60530106)(430.32373779,15.94530273)
\curveto(430.29373652,16.05530061)(430.27873653,16.1703005)(430.27873779,16.29030273)
\curveto(430.27873653,16.41030026)(430.26873654,16.53030014)(430.24873779,16.65030273)
\lineto(430.24873779,16.75530273)
\curveto(430.25873655,16.78529988)(430.26373655,16.82529984)(430.26373779,16.87530273)
\lineto(430.26373779,17.13030273)
\curveto(430.27373654,17.22029945)(430.27873653,17.31029936)(430.27873779,17.40030273)
\lineto(430.32373779,17.61030273)
\curveto(430.32373649,17.65029902)(430.32873648,17.70529896)(430.33873779,17.77530273)
\curveto(430.34873646,17.85529881)(430.36373645,17.92029875)(430.38373779,17.97030273)
\lineto(430.41373779,18.13530273)
\curveto(430.44373637,18.18529848)(430.45873635,18.23529843)(430.45873779,18.28530273)
\curveto(430.46873634,18.34529832)(430.48373633,18.40029827)(430.50373779,18.45030273)
\curveto(430.57373624,18.61029806)(430.63873617,18.7702979)(430.69873779,18.93030273)
\curveto(430.75873605,19.09029758)(430.83373598,19.24029743)(430.92373779,19.38030273)
\curveto(430.99373582,19.49029718)(431.05873575,19.60029707)(431.11873779,19.71030273)
\curveto(431.18873562,19.83029684)(431.26873554,19.94529672)(431.35873779,20.05530273)
\curveto(431.64873516,20.40529626)(431.95873485,20.70529596)(432.28873779,20.95530273)
\curveto(432.61873419,21.21529545)(433.00373381,21.43029524)(433.44373779,21.60030273)
\curveto(433.57373324,21.65029502)(433.70373311,21.68529498)(433.83373779,21.70530273)
\curveto(433.96373285,21.73529493)(434.10373271,21.7652949)(434.25373779,21.79530273)
\curveto(434.30373251,21.80529486)(434.34873246,21.81029486)(434.38873779,21.81030273)
\curveto(434.42873238,21.82029485)(434.47373234,21.82529484)(434.52373779,21.82530273)
\curveto(434.54373227,21.83529483)(434.56873224,21.83529483)(434.59873779,21.82530273)
\curveto(434.62873218,21.81529485)(434.65373216,21.82029485)(434.67373779,21.84030273)
\curveto(435.10373171,21.85029482)(435.46373135,21.80529486)(435.75373779,21.70530273)
\curveto(436.04373077,21.61529505)(436.29873051,21.49029518)(436.51873779,21.33030273)
\curveto(436.55873025,21.31029536)(436.58873022,21.28029539)(436.60873779,21.24030273)
\curveto(436.63873017,21.21029546)(436.66873014,21.18529548)(436.69873779,21.16530273)
\curveto(436.76873004,21.10529556)(436.83872997,21.03529563)(436.90873779,20.95530273)
\curveto(436.97872983,20.87529579)(437.03372978,20.79529587)(437.07373779,20.71530273)
\curveto(437.19372962,20.50529616)(437.28872952,20.30529636)(437.35873779,20.11530273)
\curveto(437.4087294,20.00529666)(437.43872937,19.88529678)(437.44873779,19.75530273)
\lineto(437.50873779,19.36530273)
\curveto(437.53872927,19.23529743)(437.54872926,19.10029757)(437.53873779,18.96030273)
\curveto(437.53872927,18.82029785)(437.54372927,18.68029799)(437.55373779,18.54030273)
\curveto(437.55372926,18.4702982)(437.54872926,18.40029827)(437.53873779,18.33030273)
\curveto(437.52872928,18.26029841)(437.52372929,18.19029848)(437.52373779,18.12030273)
\moveto(436.17373779,18.63030273)
\curveto(436.20373061,18.670298)(436.23373058,18.72029795)(436.26373779,18.78030273)
\curveto(436.30373051,18.85029782)(436.31873049,18.92029775)(436.30873779,18.99030273)
\curveto(436.29873051,19.21029746)(436.25873055,19.41529725)(436.18873779,19.60530273)
\curveto(436.08873072,19.83529683)(435.96873084,20.03029664)(435.82873779,20.19030273)
\curveto(435.69873111,20.35029632)(435.5087313,20.48529618)(435.25873779,20.59530273)
\curveto(435.18873162,20.61529605)(435.11873169,20.63029604)(435.04873779,20.64030273)
\curveto(434.98873182,20.66029601)(434.91873189,20.68029599)(434.83873779,20.70030273)
\curveto(434.76873204,20.72029595)(434.68873212,20.73029594)(434.59873779,20.73030273)
\lineto(434.34373779,20.73030273)
\curveto(434.30373251,20.71029596)(434.26373255,20.70029597)(434.22373779,20.70030273)
\curveto(434.18373263,20.71029596)(434.14873266,20.71029596)(434.11873779,20.70030273)
\lineto(433.87873779,20.64030273)
\curveto(433.79873301,20.63029604)(433.72373309,20.61529605)(433.65373779,20.59530273)
\curveto(433.33373348,20.47529619)(433.06873374,20.32529634)(432.85873779,20.14530273)
\curveto(432.64873416,19.9652967)(432.44873436,19.74029693)(432.25873779,19.47030273)
\curveto(432.21873459,19.42029725)(432.17373464,19.35529731)(432.12373779,19.27530273)
\curveto(432.08373473,19.20529746)(432.04373477,19.12529754)(432.00373779,19.03530273)
\curveto(431.96373485,18.94529772)(431.93873487,18.86029781)(431.92873779,18.78030273)
\curveto(431.92873488,18.70029797)(431.95373486,18.64029803)(432.00373779,18.60030273)
\curveto(432.07373474,18.54029813)(432.20373461,18.51029816)(432.39373779,18.51030273)
\curveto(432.59373422,18.52029815)(432.76373405,18.52529814)(432.90373779,18.52530273)
\lineto(435.18373779,18.52530273)
\curveto(435.33373148,18.52529814)(435.5137313,18.52029815)(435.72373779,18.51030273)
\curveto(435.93373088,18.51029816)(436.08373073,18.55029812)(436.17373779,18.63030273)
}
}
{
\newrgbcolor{curcolor}{0 0 0}
\pscustom[linestyle=none,fillstyle=solid,fillcolor=curcolor]
{
\newpath
\moveto(442.59537842,24.75030273)
\curveto(442.77537272,24.76029191)(442.96537253,24.76029191)(443.16537842,24.75030273)
\curveto(443.36537213,24.74029193)(443.495372,24.68029199)(443.55537842,24.57030273)
\curveto(443.58537191,24.51029216)(443.5953719,24.43529223)(443.58537842,24.34530273)
\curveto(443.57537192,24.2652924)(443.56037193,24.17529249)(443.54037842,24.07530273)
\curveto(443.52037197,23.94529272)(443.47537202,23.84029283)(443.40537842,23.76030273)
\curveto(443.35537214,23.71029296)(443.2903722,23.67529299)(443.21037842,23.65530273)
\curveto(443.13037236,23.64529302)(443.04537245,23.64029303)(442.95537842,23.64030273)
\lineto(442.68537842,23.64030273)
\curveto(442.5953729,23.65029302)(442.51037298,23.65029302)(442.43037842,23.64030273)
\curveto(442.14037335,23.56029311)(441.93537356,23.43029324)(441.81537842,23.25030273)
\curveto(441.6953738,23.08029359)(441.60037389,22.82029385)(441.53037842,22.47030273)
\curveto(441.51037398,22.40029427)(441.48537401,22.32529434)(441.45537842,22.24530273)
\curveto(441.43537406,22.17529449)(441.43037406,22.11029456)(441.44037842,22.05030273)
\curveto(441.44037405,21.90029477)(441.48537401,21.79529487)(441.57537842,21.73530273)
\curveto(441.64537385,21.70529496)(441.74037375,21.69029498)(441.86037842,21.69030273)
\lineto(442.22037842,21.69030273)
\lineto(442.44537842,21.69030273)
\curveto(442.47537302,21.670295)(442.50537299,21.665295)(442.53537842,21.67530273)
\curveto(442.56537293,21.68529498)(442.5953729,21.68029499)(442.62537842,21.66030273)
\curveto(442.71537278,21.63029504)(442.76537273,21.5702951)(442.77537842,21.48030273)
\curveto(442.7953727,21.40029527)(442.7903727,21.29529537)(442.76037842,21.16530273)
\lineto(442.73037842,21.04530273)
\lineto(442.70037842,20.92530273)
\curveto(442.64037285,20.77529589)(442.55537294,20.67529599)(442.44537842,20.62530273)
\curveto(442.30537319,20.57529609)(442.13537336,20.56029611)(441.93537842,20.58030273)
\curveto(441.73537376,20.61029606)(441.56037393,20.60529606)(441.41037842,20.56530273)
\curveto(441.33037416,20.54529612)(441.26537423,20.50529616)(441.21537842,20.44530273)
\curveto(441.16537433,20.39529627)(441.12037437,20.32529634)(441.08037842,20.23530273)
\curveto(441.05037444,20.1652965)(441.03037446,20.08529658)(441.02037842,19.99530273)
\curveto(441.01037448,19.90529676)(440.9953745,19.82029685)(440.97537842,19.74030273)
\lineto(440.78037842,18.75030273)
\lineto(440.15037842,15.57030273)
\lineto(440.00037842,14.82030273)
\curveto(439.9903755,14.76030191)(439.98037551,14.69530197)(439.97037842,14.62530273)
\curveto(439.96037553,14.55530211)(439.94037555,14.49530217)(439.91037842,14.44530273)
\lineto(439.88037842,14.32530273)
\lineto(439.82037842,14.20530273)
\curveto(439.81037568,14.1653025)(439.7903757,14.13030254)(439.76037842,14.10030273)
\curveto(439.70037579,14.03030264)(439.61537588,13.99030268)(439.50537842,13.98030273)
\curveto(439.40537609,13.9703027)(439.2953762,13.9653027)(439.17537842,13.96530273)
\lineto(438.89037842,13.96530273)
\curveto(438.85037664,13.98530268)(438.80537669,14.00030267)(438.75537842,14.01030273)
\curveto(438.71537678,14.03030264)(438.68537681,14.0653026)(438.66537842,14.11530273)
\curveto(438.65537684,14.14530252)(438.65037684,14.21030246)(438.65037842,14.31030273)
\lineto(438.66537842,14.41530273)
\curveto(438.65537684,14.4653022)(438.66037683,14.51530215)(438.68037842,14.56530273)
\curveto(438.70037679,14.62530204)(438.71537678,14.68030199)(438.72537842,14.73030273)
\lineto(438.84537842,15.33030273)
\lineto(439.65537842,19.42530273)
\curveto(439.67537582,19.53529713)(439.70037579,19.65029702)(439.73037842,19.77030273)
\curveto(439.76037573,19.89029678)(439.78037571,20.00029667)(439.79037842,20.10030273)
\curveto(439.81037568,20.21029646)(439.81037568,20.30529636)(439.79037842,20.38530273)
\curveto(439.78037571,20.4652962)(439.73537576,20.52029615)(439.65537842,20.55030273)
\curveto(439.60537589,20.58029609)(439.54037595,20.59529607)(439.46037842,20.59530273)
\lineto(439.23537842,20.59530273)
\lineto(438.99537842,20.59530273)
\curveto(438.92537657,20.59529607)(438.86037663,20.60529606)(438.80037842,20.62530273)
\curveto(438.72037677,20.665296)(438.67537682,20.75029592)(438.66537842,20.88030273)
\lineto(438.66537842,21.01530273)
\curveto(438.67537682,21.05529561)(438.68537681,21.10029557)(438.69537842,21.15030273)
\curveto(438.72537677,21.29029538)(438.76037673,21.40029527)(438.80037842,21.48030273)
\curveto(438.85037664,21.5702951)(438.93037656,21.63029504)(439.04037842,21.66030273)
\curveto(439.12037637,21.69029498)(439.20537629,21.70029497)(439.29537842,21.69030273)
\lineto(439.56537842,21.69030273)
\curveto(439.66537583,21.69029498)(439.75537574,21.70029497)(439.83537842,21.72030273)
\curveto(439.91537558,21.74029493)(439.98537551,21.78029489)(440.04537842,21.84030273)
\curveto(440.13537536,21.92029475)(440.1953753,22.04529462)(440.22537842,22.21530273)
\curveto(440.25537524,22.38529428)(440.28537521,22.54529412)(440.31537842,22.69530273)
\curveto(440.35537514,22.89529377)(440.40537509,23.08029359)(440.46537842,23.25030273)
\curveto(440.52537497,23.43029324)(440.60037489,23.59029308)(440.69037842,23.73030273)
\curveto(440.84037465,23.9702927)(441.02037447,24.1652925)(441.23037842,24.31530273)
\curveto(441.45037404,24.4652922)(441.70037379,24.58029209)(441.98037842,24.66030273)
\curveto(442.04037345,24.68029199)(442.10537339,24.69029198)(442.17537842,24.69030273)
\curveto(442.24537325,24.70029197)(442.31537318,24.71529195)(442.38537842,24.73530273)
\curveto(442.40537309,24.74529192)(442.44037305,24.74529192)(442.49037842,24.73530273)
\curveto(442.54037295,24.73529193)(442.57537292,24.74029193)(442.59537842,24.75030273)
\moveto(444.54537842,23.17530273)
\curveto(444.60537089,23.12529354)(444.68537081,23.10029357)(444.78537842,23.10030273)
\lineto(445.10037842,23.10030273)
\lineto(445.26537842,23.10030273)
\curveto(445.32537017,23.10029357)(445.38537011,23.11029356)(445.44537842,23.13030273)
\curveto(445.58536991,23.18029349)(445.67036982,23.28529338)(445.70037842,23.44530273)
\curveto(445.74036975,23.60529306)(445.78036971,23.77529289)(445.82037842,23.95530273)
\curveto(445.83036966,24.04529262)(445.84536965,24.13029254)(445.86537842,24.21030273)
\curveto(445.88536961,24.30029237)(445.88536961,24.37529229)(445.86537842,24.43530273)
\curveto(445.83536966,24.54529212)(445.74536975,24.60529206)(445.59537842,24.61530273)
\curveto(445.45537004,24.62529204)(445.30037019,24.63029204)(445.13037842,24.63030273)
\curveto(445.10037039,24.62029205)(445.07537042,24.61529205)(445.05537842,24.61530273)
\curveto(445.03537046,24.62529204)(445.01037048,24.62529204)(444.98037842,24.61530273)
\curveto(444.86037063,24.57529209)(444.77037072,24.51529215)(444.71037842,24.43530273)
\curveto(444.67037082,24.37529229)(444.64037085,24.30029237)(444.62037842,24.21030273)
\curveto(444.60037089,24.12029255)(444.58537091,24.03529263)(444.57537842,23.95530273)
\curveto(444.54537095,23.80529286)(444.51537098,23.65029302)(444.48537842,23.49030273)
\curveto(444.45537104,23.34029333)(444.47537102,23.23529343)(444.54537842,23.17530273)
\moveto(445.22037842,21.01530273)
\curveto(445.24037025,21.11529555)(445.26037023,21.21029546)(445.28037842,21.30030273)
\curveto(445.30037019,21.40029527)(445.2903702,21.48029519)(445.25037842,21.54030273)
\curveto(445.22037027,21.62029505)(445.13537036,21.66029501)(444.99537842,21.66030273)
\curveto(444.86537063,21.670295)(444.73537076,21.67529499)(444.60537842,21.67530273)
\curveto(444.58537091,21.665295)(444.56037093,21.66029501)(444.53037842,21.66030273)
\curveto(444.51037098,21.670295)(444.490371,21.67529499)(444.47037842,21.67530273)
\curveto(444.41037108,21.65529501)(444.35037114,21.64029503)(444.29037842,21.63030273)
\curveto(444.24037125,21.62029505)(444.1953713,21.59029508)(444.15537842,21.54030273)
\curveto(444.0953714,21.48029519)(444.05537144,21.39529527)(444.03537842,21.28530273)
\curveto(444.01537148,21.18529548)(443.9953715,21.08029559)(443.97537842,20.97030273)
\lineto(442.70037842,14.62530273)
\curveto(442.68037281,14.53530213)(442.66037283,14.44030223)(442.64037842,14.34030273)
\curveto(442.63037286,14.25030242)(442.63537286,14.17530249)(442.65537842,14.11530273)
\curveto(442.6953728,14.03530263)(442.76037273,13.98530268)(442.85037842,13.96530273)
\curveto(442.94037255,13.95530271)(443.05037244,13.95030272)(443.18037842,13.95030273)
\lineto(443.40537842,13.95030273)
\curveto(443.495372,13.9703027)(443.57037192,13.98530268)(443.63037842,13.99530273)
\curveto(443.6903718,14.01530265)(443.74037175,14.05530261)(443.78037842,14.11530273)
\curveto(443.85037164,14.17530249)(443.8903716,14.25530241)(443.90037842,14.35530273)
\curveto(443.92037157,14.4653022)(443.94037155,14.5703021)(443.96037842,14.67030273)
\lineto(445.22037842,21.01530273)
}
}
{
\newrgbcolor{curcolor}{0 0 0}
\pscustom[linestyle=none,fillstyle=solid,fillcolor=curcolor]
{
\newpath
\moveto(451.01905029,21.82530273)
\curveto(451.65904347,21.84529482)(452.14904298,21.76029491)(452.48905029,21.57030273)
\curveto(452.8290423,21.38029529)(453.07404206,21.09529557)(453.22405029,20.71530273)
\curveto(453.26404187,20.61529605)(453.28904184,20.50529616)(453.29905029,20.38530273)
\curveto(453.31904181,20.27529639)(453.3290418,20.16029651)(453.32905029,20.04030273)
\curveto(453.34904178,19.85029682)(453.33904179,19.64529702)(453.29905029,19.42530273)
\curveto(453.26904186,19.20529746)(453.2290419,18.98029769)(453.17905029,18.75030273)
\lineto(452.86405029,17.14530273)
\lineto(452.39905029,14.80530273)
\lineto(452.27905029,14.29530273)
\curveto(452.23904289,14.12530254)(452.14904298,14.01530265)(452.00905029,13.96530273)
\curveto(451.95904317,13.94530272)(451.90404323,13.93530273)(451.84405029,13.93530273)
\curveto(451.79404334,13.92530274)(451.73904339,13.92030275)(451.67905029,13.92030273)
\curveto(451.54904358,13.92030275)(451.42404371,13.92530274)(451.30405029,13.93530273)
\curveto(451.18404395,13.93530273)(451.10904402,13.97530269)(451.07905029,14.05530273)
\curveto(451.03904409,14.12530254)(451.0290441,14.21530245)(451.04905029,14.32530273)
\curveto(451.06904406,14.43530223)(451.09404404,14.54530212)(451.12405029,14.65530273)
\lineto(451.37905029,15.94530273)
\lineto(451.85905029,18.39030273)
\curveto(451.91904321,18.66029801)(451.96904316,18.92529774)(452.00905029,19.18530273)
\curveto(452.04904308,19.45529721)(452.04904308,19.68529698)(452.00905029,19.87530273)
\curveto(451.96904316,20.07529659)(451.87904325,20.23529643)(451.73905029,20.35530273)
\curveto(451.60904352,20.48529618)(451.44904368,20.58529608)(451.25905029,20.65530273)
\curveto(451.19904393,20.67529599)(451.134044,20.68529598)(451.06405029,20.68530273)
\curveto(451.00404413,20.69529597)(450.94904418,20.71029596)(450.89905029,20.73030273)
\curveto(450.84904428,20.74029593)(450.76904436,20.74029593)(450.65905029,20.73030273)
\curveto(450.55904457,20.73029594)(450.48404465,20.72529594)(450.43405029,20.71530273)
\curveto(450.39404474,20.69529597)(450.35904477,20.68529598)(450.32905029,20.68530273)
\curveto(450.29904483,20.69529597)(450.26404487,20.69529597)(450.22405029,20.68530273)
\curveto(450.08404505,20.65529601)(449.95404518,20.62029605)(449.83405029,20.58030273)
\curveto(449.71404542,20.55029612)(449.59904553,20.50529616)(449.48905029,20.44530273)
\curveto(449.43904569,20.42529624)(449.39904573,20.40529626)(449.36905029,20.38530273)
\curveto(449.33904579,20.3652963)(449.29904583,20.34529632)(449.24905029,20.32530273)
\curveto(448.84904628,20.07529659)(448.51904661,19.70029697)(448.25905029,19.20030273)
\curveto(448.21904691,19.12029755)(448.18404695,19.03529763)(448.15405029,18.94530273)
\lineto(448.06405029,18.70530273)
\curveto(448.0340471,18.65529801)(448.01904711,18.60529806)(448.01905029,18.55530273)
\curveto(448.01904711,18.51529815)(448.00404713,18.47529819)(447.97405029,18.43530273)
\lineto(447.91405029,18.12030273)
\curveto(447.89404724,18.09029858)(447.88404725,18.05529861)(447.88405029,18.01530273)
\curveto(447.88404725,17.97529869)(447.87904725,17.93029874)(447.86905029,17.88030273)
\lineto(447.77905029,17.43030273)
\lineto(447.47905029,15.99030273)
\lineto(447.22405029,14.67030273)
\curveto(447.20404793,14.56030211)(447.17904795,14.44530222)(447.14905029,14.32530273)
\curveto(447.129048,14.21530245)(447.08904804,14.12530254)(447.02905029,14.05530273)
\curveto(446.95904817,13.97530269)(446.85904827,13.93530273)(446.72905029,13.93530273)
\curveto(446.60904852,13.92530274)(446.48404865,13.92030275)(446.35405029,13.92030273)
\curveto(446.27404886,13.92030275)(446.19904893,13.92530274)(446.12905029,13.93530273)
\curveto(446.05904907,13.94530272)(446.00404913,13.9703027)(445.96405029,14.01030273)
\curveto(445.89404924,14.06030261)(445.87404926,14.15530251)(445.90405029,14.29530273)
\curveto(445.9340492,14.43530223)(445.95904917,14.5703021)(445.97905029,14.70030273)
\lineto(446.33905029,16.47030273)
\lineto(447.05905029,20.10030273)
\lineto(447.23905029,21.01530273)
\lineto(447.29905029,21.28530273)
\curveto(447.31904781,21.37529529)(447.35404778,21.44529522)(447.40405029,21.49530273)
\curveto(447.44404769,21.55529511)(447.49904763,21.59529507)(447.56905029,21.61530273)
\curveto(447.61904751,21.62529504)(447.67904745,21.63529503)(447.74905029,21.64530273)
\curveto(447.8290473,21.65529501)(447.90904722,21.66029501)(447.98905029,21.66030273)
\curveto(448.06904706,21.66029501)(448.14404699,21.65529501)(448.21405029,21.64530273)
\curveto(448.29404684,21.63529503)(448.34404679,21.62029505)(448.36405029,21.60030273)
\curveto(448.46404667,21.53029514)(448.49904663,21.44029523)(448.46905029,21.33030273)
\curveto(448.43904669,21.23029544)(448.4290467,21.11529555)(448.43905029,20.98530273)
\curveto(448.44904668,20.92529574)(448.47904665,20.87529579)(448.52905029,20.83530273)
\curveto(448.64904648,20.82529584)(448.75404638,20.8702958)(448.84405029,20.97030273)
\curveto(448.94404619,21.0702956)(449.03904609,21.15029552)(449.12905029,21.21030273)
\curveto(449.28904584,21.31029536)(449.44904568,21.40029527)(449.60905029,21.48030273)
\curveto(449.76904536,21.5702951)(449.95404518,21.64529502)(450.16405029,21.70530273)
\curveto(450.24404489,21.73529493)(450.3340448,21.75529491)(450.43405029,21.76530273)
\curveto(450.5340446,21.77529489)(450.6290445,21.79029488)(450.71905029,21.81030273)
\curveto(450.76904436,21.82029485)(450.81904431,21.82529484)(450.86905029,21.82530273)
\lineto(451.01905029,21.82530273)
}
}
{
\newrgbcolor{curcolor}{0 0 0}
\pscustom[linestyle=none,fillstyle=solid,fillcolor=curcolor]
{
\newpath
\moveto(456.23365967,23.17530273)
\curveto(456.16365669,23.23529343)(456.14365671,23.34029333)(456.17365967,23.49030273)
\curveto(456.20365665,23.65029302)(456.23365662,23.80529286)(456.26365967,23.95530273)
\curveto(456.27365658,24.03529263)(456.28865657,24.12029255)(456.30865967,24.21030273)
\curveto(456.32865653,24.30029237)(456.3586565,24.37529229)(456.39865967,24.43530273)
\curveto(456.4586564,24.51529215)(456.54865631,24.57529209)(456.66865967,24.61530273)
\curveto(456.69865616,24.62529204)(456.72365613,24.62529204)(456.74365967,24.61530273)
\curveto(456.76365609,24.61529205)(456.78865607,24.62029205)(456.81865967,24.63030273)
\curveto(456.98865587,24.63029204)(457.14365571,24.62529204)(457.28365967,24.61530273)
\curveto(457.43365542,24.60529206)(457.52365533,24.54529212)(457.55365967,24.43530273)
\curveto(457.57365528,24.37529229)(457.57365528,24.30029237)(457.55365967,24.21030273)
\curveto(457.53365532,24.13029254)(457.51865534,24.04529262)(457.50865967,23.95530273)
\curveto(457.46865539,23.77529289)(457.42865543,23.60529306)(457.38865967,23.44530273)
\curveto(457.3586555,23.28529338)(457.27365558,23.18029349)(457.13365967,23.13030273)
\curveto(457.07365578,23.11029356)(457.01365584,23.10029357)(456.95365967,23.10030273)
\lineto(456.78865967,23.10030273)
\lineto(456.47365967,23.10030273)
\curveto(456.37365648,23.10029357)(456.29365656,23.12529354)(456.23365967,23.17530273)
\moveto(455.64865967,14.67030273)
\curveto(455.62865723,14.5703021)(455.60865725,14.4653022)(455.58865967,14.35530273)
\curveto(455.57865728,14.25530241)(455.53865732,14.17530249)(455.46865967,14.11530273)
\curveto(455.42865743,14.05530261)(455.37865748,14.01530265)(455.31865967,13.99530273)
\curveto(455.2586576,13.98530268)(455.18365767,13.9703027)(455.09365967,13.95030273)
\lineto(454.86865967,13.95030273)
\curveto(454.73865812,13.95030272)(454.62865823,13.95530271)(454.53865967,13.96530273)
\curveto(454.44865841,13.98530268)(454.38365847,14.03530263)(454.34365967,14.11530273)
\curveto(454.32365853,14.17530249)(454.31865854,14.25030242)(454.32865967,14.34030273)
\curveto(454.34865851,14.44030223)(454.36865849,14.53530213)(454.38865967,14.62530273)
\lineto(455.66365967,20.97030273)
\curveto(455.68365717,21.08029559)(455.70365715,21.18529548)(455.72365967,21.28530273)
\curveto(455.74365711,21.39529527)(455.78365707,21.48029519)(455.84365967,21.54030273)
\curveto(455.88365697,21.59029508)(455.92865693,21.62029505)(455.97865967,21.63030273)
\curveto(456.03865682,21.64029503)(456.09865676,21.65529501)(456.15865967,21.67530273)
\curveto(456.17865668,21.67529499)(456.19865666,21.670295)(456.21865967,21.66030273)
\curveto(456.24865661,21.66029501)(456.27365658,21.665295)(456.29365967,21.67530273)
\curveto(456.42365643,21.67529499)(456.5536563,21.670295)(456.68365967,21.66030273)
\curveto(456.82365603,21.66029501)(456.90865595,21.62029505)(456.93865967,21.54030273)
\curveto(456.97865588,21.48029519)(456.98865587,21.40029527)(456.96865967,21.30030273)
\curveto(456.94865591,21.21029546)(456.92865593,21.11529555)(456.90865967,21.01530273)
\lineto(455.64865967,14.67030273)
}
}
{
\newrgbcolor{curcolor}{0 0 0}
\pscustom[linestyle=none,fillstyle=solid,fillcolor=curcolor]
{
\newpath
\moveto(464.56850342,14.76030273)
\lineto(464.47850342,14.37030273)
\curveto(464.45849549,14.25030242)(464.41849553,14.15030252)(464.35850342,14.07030273)
\curveto(464.28849566,14.00030267)(464.19349575,13.96030271)(464.07350342,13.95030273)
\lineto(463.72850342,13.95030273)
\curveto(463.66849628,13.95030272)(463.60849634,13.94530272)(463.54850342,13.93530273)
\curveto(463.49849645,13.93530273)(463.45349649,13.94530272)(463.41350342,13.96530273)
\curveto(463.33349661,13.98530268)(463.28349666,14.02530264)(463.26350342,14.08530273)
\curveto(463.23349671,14.13530253)(463.22349672,14.19530247)(463.23350342,14.26530273)
\curveto(463.2434967,14.33530233)(463.23849671,14.40530226)(463.21850342,14.47530273)
\curveto(463.21849673,14.49530217)(463.20849674,14.51030216)(463.18850342,14.52030273)
\lineto(463.15850342,14.58030273)
\curveto(463.05849689,14.59030208)(462.97349697,14.5703021)(462.90350342,14.52030273)
\curveto(462.8434971,14.4703022)(462.77849717,14.42030225)(462.70850342,14.37030273)
\curveto(462.47849747,14.22030245)(462.25349769,14.10530256)(462.03350342,14.02530273)
\curveto(461.8434981,13.94530272)(461.62349832,13.88530278)(461.37350342,13.84530273)
\curveto(461.13349881,13.80530286)(460.88849906,13.78530288)(460.63850342,13.78530273)
\curveto(460.39849955,13.77530289)(460.15849979,13.79030288)(459.91850342,13.83030273)
\curveto(459.68850026,13.86030281)(459.49350045,13.91530275)(459.33350342,13.99530273)
\curveto(458.85350109,14.21530245)(458.48850146,14.51030216)(458.23850342,14.88030273)
\curveto(457.99850195,15.26030141)(457.8435021,15.73030094)(457.77350342,16.29030273)
\curveto(457.75350219,16.38030029)(457.7435022,16.4703002)(457.74350342,16.56030273)
\curveto(457.75350219,16.66030001)(457.75350219,16.76029991)(457.74350342,16.86030273)
\curveto(457.7435022,16.91029976)(457.7485022,16.96029971)(457.75850342,17.01030273)
\curveto(457.76850218,17.06029961)(457.77350217,17.11029956)(457.77350342,17.16030273)
\curveto(457.76350218,17.21029946)(457.76350218,17.26029941)(457.77350342,17.31030273)
\curveto(457.79350215,17.3702993)(457.80350214,17.42529924)(457.80350342,17.47530273)
\lineto(457.83350342,17.62530273)
\curveto(457.82350212,17.67529899)(457.82350212,17.74029893)(457.83350342,17.82030273)
\curveto(457.85350209,17.90029877)(457.87850207,17.9652987)(457.90850342,18.01530273)
\lineto(457.95350342,18.18030273)
\curveto(457.98350196,18.25029842)(458.00350194,18.32029835)(458.01350342,18.39030273)
\curveto(458.02350192,18.4702982)(458.0435019,18.54529812)(458.07350342,18.61530273)
\curveto(458.09350185,18.665298)(458.10850184,18.71029796)(458.11850342,18.75030273)
\curveto(458.12850182,18.79029788)(458.1435018,18.83529783)(458.16350342,18.88530273)
\curveto(458.21350173,18.98529768)(458.25850169,19.08029759)(458.29850342,19.17030273)
\curveto(458.33850161,19.2702974)(458.38350156,19.3652973)(458.43350342,19.45530273)
\curveto(458.63350131,19.83529683)(458.86350108,20.17529649)(459.12350342,20.47530273)
\curveto(459.39350055,20.78529588)(459.69350025,21.04029563)(460.02350342,21.24030273)
\curveto(460.22349972,21.36029531)(460.42349952,21.46029521)(460.62350342,21.54030273)
\curveto(460.82349912,21.62029505)(461.03849891,21.69029498)(461.26850342,21.75030273)
\lineto(461.47850342,21.78030273)
\curveto(461.5484984,21.79029488)(461.61849833,21.80529486)(461.68850342,21.82530273)
\lineto(461.83850342,21.82530273)
\curveto(461.92849802,21.84529482)(462.0484979,21.85529481)(462.19850342,21.85530273)
\curveto(462.35849759,21.85529481)(462.47349747,21.84529482)(462.54350342,21.82530273)
\curveto(462.58349736,21.81529485)(462.63849731,21.81029486)(462.70850342,21.81030273)
\curveto(462.80849714,21.78029489)(462.91349703,21.75529491)(463.02350342,21.73530273)
\curveto(463.13349681,21.72529494)(463.23349671,21.69529497)(463.32350342,21.64530273)
\curveto(463.46349648,21.58529508)(463.59349635,21.52029515)(463.71350342,21.45030273)
\curveto(463.83349611,21.38029529)(463.943496,21.30029537)(464.04350342,21.21030273)
\curveto(464.09349585,21.16029551)(464.1434958,21.10529556)(464.19350342,21.04530273)
\curveto(464.25349569,20.99529567)(464.33849561,20.98029569)(464.44850342,21.00030273)
\lineto(464.52350342,21.07530273)
\curveto(464.5434954,21.09529557)(464.55849539,21.12529554)(464.56850342,21.16530273)
\curveto(464.61849533,21.25529541)(464.65349529,21.3702953)(464.67350342,21.51030273)
\curveto(464.70349524,21.65029502)(464.72849522,21.77529489)(464.74850342,21.88530273)
\lineto(465.09350342,23.61030273)
\curveto(465.12349482,23.75029292)(465.15349479,23.90529276)(465.18350342,24.07530273)
\curveto(465.22349472,24.25529241)(465.27349467,24.38529228)(465.33350342,24.46530273)
\curveto(465.39349455,24.53529213)(465.46349448,24.58029209)(465.54350342,24.60030273)
\curveto(465.56349438,24.60029207)(465.58849436,24.60029207)(465.61850342,24.60030273)
\curveto(465.6484943,24.61029206)(465.67349427,24.61529205)(465.69350342,24.61530273)
\curveto(465.8434941,24.62529204)(465.99349395,24.62529204)(466.14350342,24.61530273)
\curveto(466.29349365,24.61529205)(466.39349355,24.57529209)(466.44350342,24.49530273)
\curveto(466.47349347,24.41529225)(466.47349347,24.31529235)(466.44350342,24.19530273)
\curveto(466.42349352,24.07529259)(466.40349354,23.95029272)(466.38350342,23.82030273)
\lineto(464.56850342,14.76030273)
\moveto(463.92350342,17.59530273)
\curveto(463.95349599,17.64529902)(463.97349597,17.71029896)(463.98350342,17.79030273)
\curveto(464.00349594,17.88029879)(464.00849594,17.95029872)(463.99850342,18.00030273)
\lineto(464.04350342,18.22530273)
\curveto(464.0434959,18.31529835)(464.0484959,18.40529826)(464.05850342,18.49530273)
\curveto(464.06849588,18.59529807)(464.06349588,18.68529798)(464.04350342,18.76530273)
\lineto(464.04350342,18.99030273)
\curveto(464.0434959,19.06029761)(464.03349591,19.13029754)(464.01350342,19.20030273)
\curveto(463.95349599,19.50029717)(463.8484961,19.7652969)(463.69850342,19.99530273)
\curveto(463.55849639,20.22529644)(463.35849659,20.40529626)(463.09850342,20.53530273)
\curveto(463.00849694,20.58529608)(462.91349703,20.62029605)(462.81350342,20.64030273)
\curveto(462.71349723,20.670296)(462.60349734,20.69529597)(462.48350342,20.71530273)
\curveto(462.41349753,20.73529593)(462.32849762,20.74529592)(462.22850342,20.74530273)
\lineto(461.95850342,20.74530273)
\lineto(461.80850342,20.71530273)
\lineto(461.67350342,20.71530273)
\curveto(461.59349835,20.69529597)(461.50849844,20.67529599)(461.41850342,20.65530273)
\curveto(461.32849862,20.63529603)(461.2434987,20.61029606)(461.16350342,20.58030273)
\curveto(460.81349913,20.44029623)(460.51349943,20.23529643)(460.26350342,19.96530273)
\curveto(460.01349993,19.70529696)(459.79350015,19.40029727)(459.60350342,19.05030273)
\curveto(459.5435004,18.94029773)(459.49350045,18.82529784)(459.45350342,18.70530273)
\lineto(459.33350342,18.37530273)
\lineto(459.30350342,18.25530273)
\curveto(459.29350065,18.22529844)(459.28350066,18.19029848)(459.27350342,18.15030273)
\curveto(459.2435007,18.10029857)(459.22350072,18.04529862)(459.21350342,17.98530273)
\curveto(459.21350073,17.92529874)(459.20850074,17.8702988)(459.19850342,17.82030273)
\curveto(459.17850077,17.71029896)(459.15350079,17.60029907)(459.12350342,17.49030273)
\curveto(459.10350084,17.39029928)(459.09850085,17.29529937)(459.10850342,17.20530273)
\curveto(459.10850084,17.17529949)(459.10350084,17.12529954)(459.09350342,17.05530273)
\lineto(459.09350342,16.84530273)
\curveto(459.09350085,16.77529989)(459.09850085,16.70529996)(459.10850342,16.63530273)
\curveto(459.1485008,16.28530038)(459.23850071,15.98530068)(459.37850342,15.73530273)
\curveto(459.51850043,15.48530118)(459.71850023,15.28030139)(459.97850342,15.12030273)
\curveto(460.05849989,15.0703016)(460.13849981,15.03030164)(460.21850342,15.00030273)
\curveto(460.30849964,14.9703017)(460.40349954,14.94030173)(460.50350342,14.91030273)
\curveto(460.55349939,14.89030178)(460.60349934,14.88530178)(460.65350342,14.89530273)
\curveto(460.71349923,14.90530176)(460.76849918,14.90030177)(460.81850342,14.88030273)
\curveto(460.8484991,14.8703018)(460.88349906,14.8653018)(460.92350342,14.86530273)
\lineto(461.05850342,14.86530273)
\lineto(461.19350342,14.86530273)
\curveto(461.23349871,14.87530179)(461.28849866,14.88030179)(461.35850342,14.88030273)
\curveto(461.43849851,14.90030177)(461.51849843,14.91530175)(461.59850342,14.92530273)
\curveto(461.68849826,14.94530172)(461.76849818,14.9703017)(461.83850342,15.00030273)
\curveto(462.19849775,15.14030153)(462.50349744,15.31530135)(462.75350342,15.52530273)
\curveto(463.00349694,15.74530092)(463.22849672,16.02030065)(463.42850342,16.35030273)
\curveto(463.49849645,16.46030021)(463.55349639,16.5703001)(463.59350342,16.68030273)
\lineto(463.74350342,17.01030273)
\curveto(463.77349617,17.05029962)(463.78849616,17.08529958)(463.78850342,17.11530273)
\curveto(463.79849615,17.15529951)(463.81349613,17.19529947)(463.83350342,17.23530273)
\curveto(463.85349609,17.29529937)(463.86849608,17.35529931)(463.87850342,17.41530273)
\curveto(463.88849606,17.47529919)(463.90349604,17.53529913)(463.92350342,17.59530273)
}
}
{
\newrgbcolor{curcolor}{0 0 0}
\pscustom[linestyle=none,fillstyle=solid,fillcolor=curcolor]
{
\newpath
\moveto(474.31475342,18.15030273)
\curveto(474.32474453,18.09029858)(474.31474454,17.99529867)(474.28475342,17.86530273)
\curveto(474.26474459,17.74529892)(474.24474461,17.66029901)(474.22475342,17.61030273)
\lineto(474.19475342,17.46030273)
\curveto(474.16474469,17.38029929)(474.13974471,17.30529936)(474.11975342,17.23530273)
\curveto(474.10974474,17.17529949)(474.08974476,17.10529956)(474.05975342,17.02530273)
\curveto(474.02974482,16.9652997)(474.00474485,16.90529976)(473.98475342,16.84530273)
\curveto(473.97474488,16.78529988)(473.9497449,16.72529994)(473.90975342,16.66530273)
\lineto(473.72975342,16.27530273)
\curveto(473.67974517,16.14530052)(473.61474524,16.02530064)(473.53475342,15.91530273)
\curveto(473.23474562,15.43530123)(472.87474598,15.03030164)(472.45475342,14.70030273)
\curveto(472.04474681,14.38030229)(471.56474729,14.13530253)(471.01475342,13.96530273)
\curveto(470.90474795,13.92530274)(470.78474807,13.89530277)(470.65475342,13.87530273)
\curveto(470.52474833,13.85530281)(470.38974846,13.83530283)(470.24975342,13.81530273)
\curveto(470.18974866,13.80530286)(470.12474873,13.80030287)(470.05475342,13.80030273)
\curveto(469.99474886,13.79030288)(469.93474892,13.78530288)(469.87475342,13.78530273)
\curveto(469.83474902,13.77530289)(469.77474908,13.7703029)(469.69475342,13.77030273)
\curveto(469.62474923,13.7703029)(469.57474928,13.77530289)(469.54475342,13.78530273)
\curveto(469.50474935,13.79530287)(469.46474939,13.80030287)(469.42475342,13.80030273)
\curveto(469.38474947,13.79030288)(469.3497495,13.79030288)(469.31975342,13.80030273)
\lineto(469.22975342,13.80030273)
\lineto(468.88475342,13.84530273)
\lineto(468.49475342,13.96530273)
\curveto(468.37475048,14.00530266)(468.25975059,14.05030262)(468.14975342,14.10030273)
\curveto(467.73975111,14.30030237)(467.41975143,14.56030211)(467.18975342,14.88030273)
\curveto(466.96975188,15.20030147)(466.80975204,15.59030108)(466.70975342,16.05030273)
\curveto(466.67975217,16.15030052)(466.65975219,16.25030042)(466.64975342,16.35030273)
\lineto(466.64975342,16.66530273)
\curveto(466.63975221,16.70529996)(466.63975221,16.73529993)(466.64975342,16.75530273)
\curveto(466.65975219,16.78529988)(466.66475219,16.82029985)(466.66475342,16.86030273)
\curveto(466.66475219,16.94029973)(466.66975218,17.02029965)(466.67975342,17.10030273)
\curveto(466.68975216,17.19029948)(466.69475216,17.27529939)(466.69475342,17.35530273)
\curveto(466.70475215,17.40529926)(466.70975214,17.44529922)(466.70975342,17.47530273)
\curveto(466.71975213,17.51529915)(466.72475213,17.56029911)(466.72475342,17.61030273)
\curveto(466.72475213,17.66029901)(466.73475212,17.74529892)(466.75475342,17.86530273)
\curveto(466.78475207,17.99529867)(466.81475204,18.09029858)(466.84475342,18.15030273)
\curveto(466.88475197,18.22029845)(466.90475195,18.29029838)(466.90475342,18.36030273)
\curveto(466.90475195,18.43029824)(466.92475193,18.50029817)(466.96475342,18.57030273)
\curveto(466.98475187,18.62029805)(466.99975185,18.66029801)(467.00975342,18.69030273)
\curveto(467.01975183,18.73029794)(467.03475182,18.77529789)(467.05475342,18.82530273)
\curveto(467.11475174,18.94529772)(467.16475169,19.0652976)(467.20475342,19.18530273)
\curveto(467.2547516,19.30529736)(467.31975153,19.42029725)(467.39975342,19.53030273)
\curveto(467.61975123,19.90029677)(467.86475099,20.23029644)(468.13475342,20.52030273)
\curveto(468.41475044,20.82029585)(468.72975012,21.0702956)(469.07975342,21.27030273)
\curveto(469.20974964,21.35029532)(469.34474951,21.41529525)(469.48475342,21.46530273)
\lineto(469.93475342,21.64530273)
\curveto(470.06474879,21.69529497)(470.19974865,21.72529494)(470.33975342,21.73530273)
\curveto(470.47974837,21.75529491)(470.62474823,21.78529488)(470.77475342,21.82530273)
\lineto(470.96975342,21.82530273)
\lineto(471.17975342,21.85530273)
\curveto(472.06974678,21.8652948)(472.76974608,21.68029499)(473.27975342,21.30030273)
\curveto(473.79974505,20.92029575)(474.12474473,20.42529624)(474.25475342,19.81530273)
\curveto(474.28474457,19.71529695)(474.30474455,19.61529705)(474.31475342,19.51530273)
\curveto(474.32474453,19.41529725)(474.33974451,19.31029736)(474.35975342,19.20030273)
\curveto(474.36974448,19.09029758)(474.36974448,18.9702977)(474.35975342,18.84030273)
\lineto(474.35975342,18.46530273)
\curveto(474.35974449,18.41529825)(474.3497445,18.36029831)(474.32975342,18.30030273)
\curveto(474.31974453,18.25029842)(474.31474454,18.20029847)(474.31475342,18.15030273)
\moveto(472.81475342,17.29530273)
\curveto(472.84474601,17.3652993)(472.86474599,17.44529922)(472.87475342,17.53530273)
\curveto(472.89474596,17.62529904)(472.90974594,17.71029896)(472.91975342,17.79030273)
\curveto(472.99974585,18.18029849)(473.03474582,18.51029816)(473.02475342,18.78030273)
\curveto(473.00474585,18.86029781)(472.98974586,18.94029773)(472.97975342,19.02030273)
\curveto(472.97974587,19.10029757)(472.97474588,19.17529749)(472.96475342,19.24530273)
\curveto(472.81474604,19.89529677)(472.45974639,20.34529632)(471.89975342,20.59530273)
\curveto(471.82974702,20.62529604)(471.7547471,20.64529602)(471.67475342,20.65530273)
\curveto(471.60474725,20.67529599)(471.52974732,20.69529597)(471.44975342,20.71530273)
\curveto(471.37974747,20.73529593)(471.29974755,20.74529592)(471.20975342,20.74530273)
\lineto(470.93975342,20.74530273)
\lineto(470.65475342,20.70030273)
\curveto(470.5547483,20.68029599)(470.45974839,20.65529601)(470.36975342,20.62530273)
\curveto(470.27974857,20.60529606)(470.18974866,20.57529609)(470.09975342,20.53530273)
\curveto(470.02974882,20.51529615)(469.95974889,20.48529618)(469.88975342,20.44530273)
\curveto(469.81974903,20.40529626)(469.7547491,20.3652963)(469.69475342,20.32530273)
\curveto(469.42474943,20.15529651)(469.18974966,19.95029672)(468.98975342,19.71030273)
\curveto(468.78975006,19.4702972)(468.60475025,19.19029748)(468.43475342,18.87030273)
\curveto(468.38475047,18.7702979)(468.34475051,18.665298)(468.31475342,18.55530273)
\curveto(468.28475057,18.45529821)(468.24475061,18.35029832)(468.19475342,18.24030273)
\curveto(468.18475067,18.20029847)(468.16975068,18.13529853)(468.14975342,18.04530273)
\curveto(468.12975072,18.01529865)(468.11975073,17.98029869)(468.11975342,17.94030273)
\curveto(468.11975073,17.90029877)(468.11475074,17.85529881)(468.10475342,17.80530273)
\lineto(468.04475342,17.50530273)
\curveto(468.02475083,17.40529926)(468.01475084,17.31529935)(468.01475342,17.23530273)
\lineto(468.01475342,17.05530273)
\curveto(468.01475084,16.95529971)(468.00975084,16.85529981)(467.99975342,16.75530273)
\curveto(467.99975085,16.6653)(468.00975084,16.58030009)(468.02975342,16.50030273)
\curveto(468.07975077,16.26030041)(468.1497507,16.03530063)(468.23975342,15.82530273)
\curveto(468.33975051,15.61530105)(468.47475038,15.44030123)(468.64475342,15.30030273)
\curveto(468.69475016,15.2703014)(468.73475012,15.24530142)(468.76475342,15.22530273)
\curveto(468.80475005,15.20530146)(468.84475001,15.18030149)(468.88475342,15.15030273)
\curveto(468.9547499,15.10030157)(469.03474982,15.05530161)(469.12475342,15.01530273)
\curveto(469.21474964,14.98530168)(469.30974954,14.95530171)(469.40975342,14.92530273)
\curveto(469.45974939,14.90530176)(469.50474935,14.89530177)(469.54475342,14.89530273)
\curveto(469.59474926,14.90530176)(469.64474921,14.90530176)(469.69475342,14.89530273)
\curveto(469.72474913,14.88530178)(469.78474907,14.87530179)(469.87475342,14.86530273)
\curveto(469.96474889,14.85530181)(470.03974881,14.86030181)(470.09975342,14.88030273)
\curveto(470.13974871,14.89030178)(470.17974867,14.89030178)(470.21975342,14.88030273)
\curveto(470.25974859,14.88030179)(470.29974855,14.89030178)(470.33975342,14.91030273)
\curveto(470.41974843,14.93030174)(470.49974835,14.94530172)(470.57975342,14.95530273)
\curveto(470.66974818,14.97530169)(470.7547481,15.00030167)(470.83475342,15.03030273)
\curveto(471.19474766,15.1703015)(471.50474735,15.3653013)(471.76475342,15.61530273)
\curveto(472.02474683,15.8653008)(472.25974659,16.16030051)(472.46975342,16.50030273)
\curveto(472.5497463,16.62030005)(472.60974624,16.74529992)(472.64975342,16.87530273)
\curveto(472.68974616,17.01529965)(472.74474611,17.15529951)(472.81475342,17.29530273)
}
}
{
\newrgbcolor{curcolor}{0 0 0}
\pscustom[linestyle=none,fillstyle=solid,fillcolor=curcolor]
{
\newpath
\moveto(478.98303467,21.85530273)
\curveto(479.70302901,21.8652948)(480.28802843,21.78029489)(480.73803467,21.60030273)
\curveto(481.19802752,21.43029524)(481.5180272,21.12529554)(481.69803467,20.68530273)
\curveto(481.74802697,20.57529609)(481.77802694,20.46029621)(481.78803467,20.34030273)
\curveto(481.80802691,20.23029644)(481.82302689,20.10529656)(481.83303467,19.96530273)
\curveto(481.84302687,19.89529677)(481.83302688,19.82029685)(481.80303467,19.74030273)
\curveto(481.78302693,19.670297)(481.75802696,19.61529705)(481.72803467,19.57530273)
\curveto(481.70802701,19.55529711)(481.67802704,19.53529713)(481.63803467,19.51530273)
\curveto(481.60802711,19.50529716)(481.58302713,19.49029718)(481.56303467,19.47030273)
\curveto(481.50302721,19.45029722)(481.44802727,19.44529722)(481.39803467,19.45530273)
\curveto(481.35802736,19.4652972)(481.3130274,19.4652972)(481.26303467,19.45530273)
\curveto(481.17302754,19.43529723)(481.06302765,19.43029724)(480.93303467,19.44030273)
\curveto(480.8130279,19.46029721)(480.72802799,19.48529718)(480.67803467,19.51530273)
\curveto(480.60802811,19.5652971)(480.56802815,19.63029704)(480.55803467,19.71030273)
\curveto(480.55802816,19.80029687)(480.53802818,19.88529678)(480.49803467,19.96530273)
\curveto(480.44802827,20.12529654)(480.35302836,20.2702964)(480.21303467,20.40030273)
\curveto(480.12302859,20.48029619)(480.0130287,20.54029613)(479.88303467,20.58030273)
\curveto(479.76302895,20.62029605)(479.63302908,20.66029601)(479.49303467,20.70030273)
\curveto(479.45302926,20.72029595)(479.40302931,20.72529594)(479.34303467,20.71530273)
\curveto(479.29302942,20.71529595)(479.24802947,20.72029595)(479.20803467,20.73030273)
\curveto(479.14802957,20.75029592)(479.07302964,20.76029591)(478.98303467,20.76030273)
\curveto(478.89302982,20.76029591)(478.8180299,20.75029592)(478.75803467,20.73030273)
\lineto(478.66803467,20.73030273)
\curveto(478.60803011,20.72029595)(478.55303016,20.71029596)(478.50303467,20.70030273)
\curveto(478.45303026,20.70029597)(478.40303031,20.69529597)(478.35303467,20.68530273)
\curveto(478.08303063,20.62529604)(477.84803087,20.54029613)(477.64803467,20.43030273)
\curveto(477.45803126,20.32029635)(477.30803141,20.13529653)(477.19803467,19.87530273)
\curveto(477.16803155,19.80529686)(477.15303156,19.73529693)(477.15303467,19.66530273)
\curveto(477.15303156,19.59529707)(477.15803156,19.53529713)(477.16803467,19.48530273)
\curveto(477.19803152,19.33529733)(477.24803147,19.22529744)(477.31803467,19.15530273)
\curveto(477.38803133,19.09529757)(477.48303123,19.02529764)(477.60303467,18.94530273)
\curveto(477.74303097,18.84529782)(477.90803081,18.7702979)(478.09803467,18.72030273)
\curveto(478.28803043,18.68029799)(478.47803024,18.63029804)(478.66803467,18.57030273)
\curveto(478.78802993,18.53029814)(478.90802981,18.50029817)(479.02803467,18.48030273)
\curveto(479.15802956,18.46029821)(479.28302943,18.43029824)(479.40303467,18.39030273)
\curveto(479.60302911,18.33029834)(479.79802892,18.2702984)(479.98803467,18.21030273)
\curveto(480.17802854,18.16029851)(480.36302835,18.09529857)(480.54303467,18.01530273)
\curveto(480.59302812,17.99529867)(480.63802808,17.97529869)(480.67803467,17.95530273)
\curveto(480.72802799,17.93529873)(480.77802794,17.91029876)(480.82803467,17.88030273)
\curveto(480.99802772,17.76029891)(481.14302757,17.62529904)(481.26303467,17.47530273)
\curveto(481.38302733,17.32529934)(481.47302724,17.13529953)(481.53303467,16.90530273)
\lineto(481.53303467,16.62030273)
\curveto(481.53302718,16.55030012)(481.52802719,16.47530019)(481.51803467,16.39530273)
\curveto(481.50802721,16.32530034)(481.49802722,16.24530042)(481.48803467,16.15530273)
\lineto(481.45803467,16.00530273)
\curveto(481.4180273,15.93530073)(481.38802733,15.8653008)(481.36803467,15.79530273)
\curveto(481.35802736,15.72530094)(481.33802738,15.65530101)(481.30803467,15.58530273)
\curveto(481.25802746,15.47530119)(481.20302751,15.3703013)(481.14303467,15.27030273)
\curveto(481.08302763,15.1703015)(481.0180277,15.08030159)(480.94803467,15.00030273)
\curveto(480.73802798,14.74030193)(480.49302822,14.53030214)(480.21303467,14.37030273)
\curveto(479.93302878,14.22030245)(479.62802909,14.09030258)(479.29803467,13.98030273)
\curveto(479.19802952,13.95030272)(479.09802962,13.93030274)(478.99803467,13.92030273)
\curveto(478.89802982,13.90030277)(478.80302991,13.87530279)(478.71303467,13.84530273)
\curveto(478.60303011,13.82530284)(478.49803022,13.81530285)(478.39803467,13.81530273)
\curveto(478.29803042,13.81530285)(478.19803052,13.80530286)(478.09803467,13.78530273)
\lineto(477.94803467,13.78530273)
\curveto(477.89803082,13.77530289)(477.82803089,13.7703029)(477.73803467,13.77030273)
\curveto(477.64803107,13.7703029)(477.57803114,13.77530289)(477.52803467,13.78530273)
\lineto(477.36303467,13.78530273)
\curveto(477.30303141,13.80530286)(477.23803148,13.81530285)(477.16803467,13.81530273)
\curveto(477.09803162,13.80530286)(477.04303167,13.81030286)(477.00303467,13.83030273)
\curveto(476.95303176,13.84030283)(476.88803183,13.84530282)(476.80803467,13.84530273)
\curveto(476.72803199,13.8653028)(476.65303206,13.88530278)(476.58303467,13.90530273)
\curveto(476.5130322,13.91530275)(476.43803228,13.93530273)(476.35803467,13.96530273)
\curveto(476.06803265,14.0653026)(475.82303289,14.19030248)(475.62303467,14.34030273)
\curveto(475.42303329,14.49030218)(475.26303345,14.68530198)(475.14303467,14.92530273)
\curveto(475.08303363,15.05530161)(475.03303368,15.19030148)(474.99303467,15.33030273)
\curveto(474.96303375,15.4703012)(474.94303377,15.62530104)(474.93303467,15.79530273)
\curveto(474.92303379,15.85530081)(474.92803379,15.92530074)(474.94803467,16.00530273)
\curveto(474.96803375,16.09530057)(474.99303372,16.1653005)(475.02303467,16.21530273)
\curveto(475.06303365,16.25530041)(475.12303359,16.29530037)(475.20303467,16.33530273)
\curveto(475.25303346,16.35530031)(475.32303339,16.3653003)(475.41303467,16.36530273)
\curveto(475.5130332,16.37530029)(475.60303311,16.37530029)(475.68303467,16.36530273)
\curveto(475.77303294,16.35530031)(475.85803286,16.34030033)(475.93803467,16.32030273)
\curveto(476.02803269,16.31030036)(476.08303263,16.29530037)(476.10303467,16.27530273)
\curveto(476.16303255,16.22530044)(476.19303252,16.15030052)(476.19303467,16.05030273)
\curveto(476.20303251,15.96030071)(476.22303249,15.87530079)(476.25303467,15.79530273)
\curveto(476.30303241,15.57530109)(476.40303231,15.40530126)(476.55303467,15.28530273)
\curveto(476.65303206,15.19530147)(476.77303194,15.12530154)(476.91303467,15.07530273)
\curveto(477.05303166,15.02530164)(477.20303151,14.97530169)(477.36303467,14.92530273)
\lineto(477.67803467,14.88030273)
\lineto(477.76803467,14.88030273)
\curveto(477.82803089,14.86030181)(477.9130308,14.85030182)(478.02303467,14.85030273)
\curveto(478.14303057,14.85030182)(478.24803047,14.86030181)(478.33803467,14.88030273)
\curveto(478.40803031,14.88030179)(478.46303025,14.88530178)(478.50303467,14.89530273)
\curveto(478.56303015,14.90530176)(478.62303009,14.91030176)(478.68303467,14.91030273)
\curveto(478.74302997,14.92030175)(478.79802992,14.93030174)(478.84803467,14.94030273)
\curveto(479.15802956,15.02030165)(479.40802931,15.12530154)(479.59803467,15.25530273)
\curveto(479.79802892,15.38530128)(479.96302875,15.60530106)(480.09303467,15.91530273)
\curveto(480.12302859,15.9653007)(480.13802858,16.02030065)(480.13803467,16.08030273)
\curveto(480.14802857,16.14030053)(480.14802857,16.18530048)(480.13803467,16.21530273)
\curveto(480.12802859,16.40530026)(480.08802863,16.54530012)(480.01803467,16.63530273)
\curveto(479.94802877,16.73529993)(479.85302886,16.82529984)(479.73303467,16.90530273)
\curveto(479.65302906,16.9652997)(479.55802916,17.01529965)(479.44803467,17.05530273)
\lineto(479.14803467,17.17530273)
\curveto(479.1180296,17.18529948)(479.08802963,17.19029948)(479.05803467,17.19030273)
\curveto(479.03802968,17.19029948)(479.0180297,17.20029947)(478.99803467,17.22030273)
\curveto(478.67803004,17.33029934)(478.33803038,17.41029926)(477.97803467,17.46030273)
\curveto(477.62803109,17.52029915)(477.30803141,17.61529905)(477.01803467,17.74530273)
\curveto(476.92803179,17.78529888)(476.83803188,17.82029885)(476.74803467,17.85030273)
\curveto(476.66803205,17.88029879)(476.59303212,17.92029875)(476.52303467,17.97030273)
\curveto(476.35303236,18.08029859)(476.20303251,18.20529846)(476.07303467,18.34530273)
\curveto(475.94303277,18.48529818)(475.85303286,18.66029801)(475.80303467,18.87030273)
\curveto(475.78303293,18.94029773)(475.77303294,19.01029766)(475.77303467,19.08030273)
\lineto(475.77303467,19.30530273)
\curveto(475.76303295,19.42529724)(475.77803294,19.56029711)(475.81803467,19.71030273)
\curveto(475.85803286,19.8702968)(475.89803282,20.00529666)(475.93803467,20.11530273)
\curveto(475.96803275,20.1652965)(475.98803273,20.20529646)(475.99803467,20.23530273)
\curveto(476.0180327,20.27529639)(476.04303267,20.31529635)(476.07303467,20.35530273)
\curveto(476.20303251,20.58529608)(476.36303235,20.78529588)(476.55303467,20.95530273)
\curveto(476.74303197,21.12529554)(476.95303176,21.27529539)(477.18303467,21.40530273)
\curveto(477.34303137,21.49529517)(477.5180312,21.5652951)(477.70803467,21.61530273)
\curveto(477.90803081,21.67529499)(478.1130306,21.73029494)(478.32303467,21.78030273)
\curveto(478.39303032,21.79029488)(478.45803026,21.80029487)(478.51803467,21.81030273)
\curveto(478.58803013,21.82029485)(478.66303005,21.83029484)(478.74303467,21.84030273)
\curveto(478.78302993,21.85029482)(478.82302989,21.85029482)(478.86303467,21.84030273)
\curveto(478.9130298,21.83029484)(478.95302976,21.83529483)(478.98303467,21.85530273)
}
}
{
\newrgbcolor{curcolor}{0 0 0}
\pscustom[linewidth=1,linecolor=curcolor]
{
\newpath
\moveto(95.01786,75.52252)
\lineto(732.99776,75.52252)
}
}
{
\newrgbcolor{curcolor}{0 0 0}
\pscustom[linewidth=1,linecolor=curcolor]
{
\newpath
\moveto(95.01786,149.51623)
\lineto(732.99776,149.51623)
}
}
{
\newrgbcolor{curcolor}{0 0 0}
\pscustom[linewidth=1,linecolor=curcolor]
{
\newpath
\moveto(95.01786,223.55822)
\lineto(732.99776,223.55822)
}
}
{
\newrgbcolor{curcolor}{0 0 0}
\pscustom[linewidth=1,linecolor=curcolor]
{
\newpath
\moveto(95.01786,298.62335)
\lineto(732.99776,298.62335)
}
}
{
\newrgbcolor{curcolor}{0 0 0}
\pscustom[linewidth=1,linecolor=curcolor]
{
\newpath
\moveto(95.01786,372.589017)
\lineto(732.99776,372.589017)
}
}
{
\newrgbcolor{curcolor}{0 0 0}
\pscustom[linestyle=none,fillstyle=solid,fillcolor=curcolor]
{
\newpath
\moveto(100.64285278,409.08210297)
\curveto(101.69284611,409.102092)(102.55784524,408.92209218)(103.23785278,408.54210297)
\curveto(103.91784388,408.16209294)(104.45784334,407.65709344)(104.85785278,407.02710297)
\curveto(104.96784283,406.85709424)(105.05784274,406.68209442)(105.12785278,406.50210297)
\curveto(105.1978426,406.33209477)(105.26284254,406.14209496)(105.32285278,405.93210297)
\curveto(105.34284246,405.86209524)(105.36284244,405.78209532)(105.38285278,405.69210297)
\curveto(105.4028424,405.6020955)(105.3978424,405.51709558)(105.36785278,405.43710297)
\curveto(105.34784245,405.37709572)(105.30784249,405.33709576)(105.24785278,405.31710297)
\curveto(105.1978426,405.30709579)(105.13784266,405.29209581)(105.06785278,405.27210297)
\lineto(104.94785278,405.27210297)
\curveto(104.88784291,405.25209585)(104.81784298,405.24209586)(104.73785278,405.24210297)
\curveto(104.66784313,405.25209585)(104.5978432,405.25709584)(104.52785278,405.25710297)
\curveto(104.43784336,405.25709584)(104.32784347,405.25209585)(104.19785278,405.24210297)
\lineto(103.83785278,405.24210297)
\curveto(103.71784408,405.25209585)(103.60784419,405.26209584)(103.50785278,405.27210297)
\curveto(103.40784439,405.29209581)(103.33284447,405.31709578)(103.28285278,405.34710297)
\curveto(103.2028446,405.41709568)(103.14284466,405.51209559)(103.10285278,405.63210297)
\curveto(103.06284474,405.75209535)(103.01284479,405.85709524)(102.95285278,405.94710297)
\curveto(102.76284504,406.27709482)(102.51284529,406.53709456)(102.20285278,406.72710297)
\curveto(101.99284581,406.85709424)(101.75784604,406.96209414)(101.49785278,407.04210297)
\curveto(101.32784647,407.102094)(101.11284669,407.13209397)(100.85285278,407.13210297)
\curveto(100.6028472,407.13209397)(100.38284742,407.10709399)(100.19285278,407.05710297)
\curveto(100.11284769,407.03709406)(100.03784776,407.01709408)(99.96785278,406.99710297)
\curveto(99.90784789,406.98709411)(99.84284796,406.96709413)(99.77285278,406.93710297)
\curveto(99.09284871,406.64709445)(98.61284919,406.16709493)(98.33285278,405.49710297)
\curveto(98.28284952,405.37709572)(98.23784956,405.25209585)(98.19785278,405.12210297)
\curveto(98.15784964,404.99209611)(98.11284969,404.85709624)(98.06285278,404.71710297)
\curveto(98.05284975,404.64709645)(98.04284976,404.58209652)(98.03285278,404.52210297)
\curveto(98.02284978,404.46209664)(98.01284979,404.3970967)(98.00285278,404.32710297)
\curveto(97.98284982,404.26709683)(97.97284983,404.2020969)(97.97285278,404.13210297)
\curveto(97.98284982,404.07209703)(97.97784982,404.00709709)(97.95785278,403.93710297)
\curveto(97.93784986,403.85709724)(97.92784987,403.77209733)(97.92785278,403.68210297)
\curveto(97.93784986,403.6020975)(97.94284986,403.52209758)(97.94285278,403.44210297)
\curveto(97.94284986,403.4020977)(97.93784986,403.36209774)(97.92785278,403.32210297)
\curveto(97.92784987,403.28209782)(97.93284987,403.24209786)(97.94285278,403.20210297)
\lineto(97.94285278,403.06710297)
\curveto(97.96284984,403.01709808)(97.96784983,402.96709813)(97.95785278,402.91710297)
\curveto(97.95784984,402.86709823)(97.96784983,402.81709828)(97.98785278,402.76710297)
\curveto(97.98784981,402.70709839)(97.9978498,402.62709847)(98.01785278,402.52710297)
\curveto(98.03784976,402.41709868)(98.05784974,402.31209879)(98.07785278,402.21210297)
\curveto(98.0978497,402.12209898)(98.12284968,402.03209907)(98.15285278,401.94210297)
\curveto(98.29284951,401.52209958)(98.47284933,401.16209994)(98.69285278,400.86210297)
\curveto(98.91284889,400.57210053)(99.2028486,400.33210077)(99.56285278,400.14210297)
\curveto(99.67284813,400.09210101)(99.78784801,400.04710105)(99.90785278,400.00710297)
\curveto(100.02784777,399.97710112)(100.15784764,399.94210116)(100.29785278,399.90210297)
\curveto(100.34784745,399.89210121)(100.39284741,399.88210122)(100.43285278,399.87210297)
\lineto(100.58285278,399.87210297)
\lineto(100.70285278,399.87210297)
\curveto(100.75284705,399.85210125)(100.81784698,399.84210126)(100.89785278,399.84210297)
\curveto(100.97784682,399.85210125)(101.04284676,399.86210124)(101.09285278,399.87210297)
\curveto(101.15284665,399.87210123)(101.1978466,399.87710122)(101.22785278,399.88710297)
\curveto(101.34784645,399.90710119)(101.45784634,399.92710117)(101.55785278,399.94710297)
\curveto(101.66784613,399.96710113)(101.77284603,400.0021011)(101.87285278,400.05210297)
\curveto(102.1028457,400.15210095)(102.2978455,400.28210082)(102.45785278,400.44210297)
\curveto(102.62784517,400.61210049)(102.77784502,400.8021003)(102.90785278,401.01210297)
\curveto(102.96784483,401.11209999)(103.01784478,401.22209988)(103.05785278,401.34210297)
\curveto(103.0978447,401.46209964)(103.14284466,401.58209952)(103.19285278,401.70210297)
\curveto(103.22284458,401.81209929)(103.25284455,401.91209919)(103.28285278,402.00210297)
\curveto(103.31284449,402.09209901)(103.37784442,402.16209894)(103.47785278,402.21210297)
\curveto(103.53784426,402.23209887)(103.61284419,402.24209886)(103.70285278,402.24210297)
\lineto(103.95785278,402.24210297)
\lineto(104.87285278,402.24210297)
\lineto(105.14285278,402.24210297)
\curveto(105.24284256,402.24209886)(105.32284248,402.22209888)(105.38285278,402.18210297)
\curveto(105.45284235,402.13209897)(105.48784231,402.05209905)(105.48785278,401.94210297)
\curveto(105.4978423,401.83209927)(105.48784231,401.72709937)(105.45785278,401.62710297)
\lineto(105.36785278,401.22210297)
\curveto(105.31784248,401.07210003)(105.26784253,400.92710017)(105.21785278,400.78710297)
\curveto(105.17784262,400.64710045)(105.12284268,400.51210059)(105.05285278,400.38210297)
\curveto(105.0028428,400.3021008)(104.96284284,400.22210088)(104.93285278,400.14210297)
\curveto(104.9028429,400.07210103)(104.86284294,400.0021011)(104.81285278,399.93210297)
\curveto(104.26284354,399.07210203)(103.46284434,398.47210263)(102.41285278,398.13210297)
\curveto(102.3028455,398.09210301)(102.19284561,398.06210304)(102.08285278,398.04210297)
\lineto(101.75285278,397.98210297)
\curveto(101.7028461,397.96210314)(101.65284615,397.95710314)(101.60285278,397.96710297)
\curveto(101.56284624,397.96710313)(101.51784628,397.95710314)(101.46785278,397.93710297)
\lineto(101.25785278,397.93710297)
\curveto(101.1978466,397.92710317)(101.13284667,397.91710318)(101.06285278,397.90710297)
\lineto(100.82285278,397.90710297)
\lineto(100.55285278,397.90710297)
\curveto(100.46284734,397.90710319)(100.37784742,397.91710318)(100.29785278,397.93710297)
\curveto(100.26784753,397.94710315)(100.21784758,397.95210315)(100.14785278,397.95210297)
\lineto(99.93785278,397.98210297)
\curveto(99.86784793,397.99210311)(99.79284801,398.0021031)(99.71285278,398.01210297)
\curveto(99.61284819,398.04210306)(99.51284829,398.06710303)(99.41285278,398.08710297)
\curveto(99.32284848,398.10710299)(99.22784857,398.13210297)(99.12785278,398.16210297)
\lineto(98.85785278,398.25210297)
\curveto(98.76784903,398.29210281)(98.68284912,398.33210277)(98.60285278,398.37210297)
\curveto(98.0028498,398.63210247)(97.49285031,398.97710212)(97.07285278,399.40710297)
\curveto(96.66285114,399.84710125)(96.32785147,400.36710073)(96.06785278,400.96710297)
\curveto(96.00785179,401.11709998)(95.95785184,401.26709983)(95.91785278,401.41710297)
\curveto(95.87785192,401.56709953)(95.83285197,401.72209938)(95.78285278,401.88210297)
\curveto(95.76285204,401.93209917)(95.75285205,401.97209913)(95.75285278,402.00210297)
\curveto(95.75285205,402.04209906)(95.74785205,402.08209902)(95.73785278,402.12210297)
\curveto(95.71785208,402.21209889)(95.7028521,402.30709879)(95.69285278,402.40710297)
\curveto(95.68285212,402.50709859)(95.66785213,402.6020985)(95.64785278,402.69210297)
\curveto(95.62785217,402.74209836)(95.61785218,402.78209832)(95.61785278,402.81210297)
\curveto(95.62785217,402.85209825)(95.62785217,402.89209821)(95.61785278,402.93210297)
\lineto(95.61785278,403.21710297)
\curveto(95.5978522,403.26709783)(95.58785221,403.34209776)(95.58785278,403.44210297)
\curveto(95.58785221,403.54209756)(95.5978522,403.61709748)(95.61785278,403.66710297)
\curveto(95.62785217,403.6970974)(95.62785217,403.72709737)(95.61785278,403.75710297)
\lineto(95.61785278,403.84710297)
\lineto(95.61785278,403.98210297)
\curveto(95.63785216,404.06209704)(95.64785215,404.14709695)(95.64785278,404.23710297)
\lineto(95.67785278,404.50710297)
\curveto(95.6978521,404.58709651)(95.71285209,404.66209644)(95.72285278,404.73210297)
\curveto(95.73285207,404.81209629)(95.74785205,404.89209621)(95.76785278,404.97210297)
\curveto(95.80785199,405.11209599)(95.84285196,405.24709585)(95.87285278,405.37710297)
\curveto(95.9028519,405.51709558)(95.94285186,405.64709545)(95.99285278,405.76710297)
\lineto(96.14285278,406.15710297)
\curveto(96.2028516,406.28709481)(96.26785153,406.40709469)(96.33785278,406.51710297)
\curveto(96.4978513,406.7970943)(96.66285114,407.04709405)(96.83285278,407.26710297)
\curveto(96.88285092,407.33709376)(96.93785086,407.4020937)(96.99785278,407.46210297)
\lineto(97.17785278,407.64210297)
\curveto(97.42785037,407.89209321)(97.68785011,408.102093)(97.95785278,408.27210297)
\curveto(98.23784956,408.45209265)(98.55284925,408.61209249)(98.90285278,408.75210297)
\curveto(99.02284878,408.8020923)(99.14784865,408.84209226)(99.27785278,408.87210297)
\curveto(99.40784839,408.91209219)(99.54284826,408.94709215)(99.68285278,408.97710297)
\curveto(99.74284806,408.9970921)(99.802848,409.00709209)(99.86285278,409.00710297)
\curveto(99.92284788,409.00709209)(99.97784782,409.01709208)(100.02785278,409.03710297)
\curveto(100.10784769,409.04709205)(100.18284762,409.05209205)(100.25285278,409.05210297)
\curveto(100.33284747,409.06209204)(100.41284739,409.07209203)(100.49285278,409.08210297)
\curveto(100.51284729,409.09209201)(100.53784726,409.09209201)(100.56785278,409.08210297)
\curveto(100.5978472,409.07209203)(100.62284718,409.07209203)(100.64285278,409.08210297)
}
}
{
\newrgbcolor{curcolor}{0 0 0}
\pscustom[linestyle=none,fillstyle=solid,fillcolor=curcolor]
{
\newpath
\moveto(113.95136841,398.73210297)
\curveto(113.97136056,398.62210248)(113.98136055,398.51210259)(113.98136841,398.40210297)
\curveto(113.99136054,398.29210281)(113.94136059,398.21710288)(113.83136841,398.17710297)
\curveto(113.77136076,398.14710295)(113.70136083,398.13210297)(113.62136841,398.13210297)
\lineto(113.38136841,398.13210297)
\lineto(112.57136841,398.13210297)
\lineto(112.30136841,398.13210297)
\curveto(112.22136231,398.14210296)(112.15636237,398.16710293)(112.10636841,398.20710297)
\curveto(112.03636249,398.24710285)(111.98136255,398.3021028)(111.94136841,398.37210297)
\curveto(111.91136262,398.45210265)(111.86636266,398.51710258)(111.80636841,398.56710297)
\curveto(111.78636274,398.58710251)(111.76136277,398.6021025)(111.73136841,398.61210297)
\curveto(111.70136283,398.63210247)(111.66136287,398.63710246)(111.61136841,398.62710297)
\curveto(111.56136297,398.60710249)(111.51136302,398.58210252)(111.46136841,398.55210297)
\curveto(111.42136311,398.52210258)(111.37636315,398.4971026)(111.32636841,398.47710297)
\curveto(111.27636325,398.43710266)(111.22136331,398.4021027)(111.16136841,398.37210297)
\lineto(110.98136841,398.28210297)
\curveto(110.85136368,398.22210288)(110.71636381,398.17210293)(110.57636841,398.13210297)
\curveto(110.43636409,398.102103)(110.29136424,398.06710303)(110.14136841,398.02710297)
\curveto(110.07136446,398.00710309)(110.00136453,397.9971031)(109.93136841,397.99710297)
\curveto(109.87136466,397.98710311)(109.80636472,397.97710312)(109.73636841,397.96710297)
\lineto(109.64636841,397.96710297)
\curveto(109.61636491,397.95710314)(109.58636494,397.95210315)(109.55636841,397.95210297)
\lineto(109.39136841,397.95210297)
\curveto(109.29136524,397.93210317)(109.19136534,397.93210317)(109.09136841,397.95210297)
\lineto(108.95636841,397.95210297)
\curveto(108.88636564,397.97210313)(108.81636571,397.98210312)(108.74636841,397.98210297)
\curveto(108.68636584,397.97210313)(108.6263659,397.97710312)(108.56636841,397.99710297)
\curveto(108.46636606,398.01710308)(108.37136616,398.03710306)(108.28136841,398.05710297)
\curveto(108.19136634,398.06710303)(108.10636642,398.09210301)(108.02636841,398.13210297)
\curveto(107.73636679,398.24210286)(107.48636704,398.38210272)(107.27636841,398.55210297)
\curveto(107.07636745,398.73210237)(106.91636761,398.96710213)(106.79636841,399.25710297)
\curveto(106.76636776,399.32710177)(106.73636779,399.4021017)(106.70636841,399.48210297)
\curveto(106.68636784,399.56210154)(106.66636786,399.64710145)(106.64636841,399.73710297)
\curveto(106.6263679,399.78710131)(106.61636791,399.83710126)(106.61636841,399.88710297)
\curveto(106.6263679,399.93710116)(106.6263679,399.98710111)(106.61636841,400.03710297)
\curveto(106.60636792,400.06710103)(106.59636793,400.12710097)(106.58636841,400.21710297)
\curveto(106.58636794,400.31710078)(106.59136794,400.38710071)(106.60136841,400.42710297)
\curveto(106.62136791,400.52710057)(106.6313679,400.61210049)(106.63136841,400.68210297)
\lineto(106.72136841,401.01210297)
\curveto(106.75136778,401.13209997)(106.79136774,401.23709986)(106.84136841,401.32710297)
\curveto(107.01136752,401.61709948)(107.20636732,401.83709926)(107.42636841,401.98710297)
\curveto(107.64636688,402.13709896)(107.9263666,402.26709883)(108.26636841,402.37710297)
\curveto(108.39636613,402.42709867)(108.531366,402.46209864)(108.67136841,402.48210297)
\curveto(108.81136572,402.5020986)(108.95136558,402.52709857)(109.09136841,402.55710297)
\curveto(109.17136536,402.57709852)(109.25636527,402.58709851)(109.34636841,402.58710297)
\curveto(109.43636509,402.5970985)(109.526365,402.61209849)(109.61636841,402.63210297)
\curveto(109.68636484,402.65209845)(109.75636477,402.65709844)(109.82636841,402.64710297)
\curveto(109.89636463,402.64709845)(109.97136456,402.65709844)(110.05136841,402.67710297)
\curveto(110.12136441,402.6970984)(110.19136434,402.70709839)(110.26136841,402.70710297)
\curveto(110.3313642,402.70709839)(110.40636412,402.71709838)(110.48636841,402.73710297)
\curveto(110.69636383,402.78709831)(110.88636364,402.82709827)(111.05636841,402.85710297)
\curveto(111.23636329,402.8970982)(111.39636313,402.98709811)(111.53636841,403.12710297)
\curveto(111.6263629,403.21709788)(111.68636284,403.31709778)(111.71636841,403.42710297)
\curveto(111.7263628,403.45709764)(111.7263628,403.48209762)(111.71636841,403.50210297)
\curveto(111.71636281,403.52209758)(111.72136281,403.54209756)(111.73136841,403.56210297)
\curveto(111.74136279,403.58209752)(111.74636278,403.61209749)(111.74636841,403.65210297)
\lineto(111.74636841,403.74210297)
\lineto(111.71636841,403.86210297)
\curveto(111.71636281,403.9020972)(111.71136282,403.93709716)(111.70136841,403.96710297)
\curveto(111.60136293,404.26709683)(111.39136314,404.47209663)(111.07136841,404.58210297)
\curveto(110.98136355,404.61209649)(110.87136366,404.63209647)(110.74136841,404.64210297)
\curveto(110.62136391,404.66209644)(110.49636403,404.66709643)(110.36636841,404.65710297)
\curveto(110.23636429,404.65709644)(110.11136442,404.64709645)(109.99136841,404.62710297)
\curveto(109.87136466,404.60709649)(109.76636476,404.58209652)(109.67636841,404.55210297)
\curveto(109.61636491,404.53209657)(109.55636497,404.5020966)(109.49636841,404.46210297)
\curveto(109.44636508,404.43209667)(109.39636513,404.3970967)(109.34636841,404.35710297)
\curveto(109.29636523,404.31709678)(109.24136529,404.26209684)(109.18136841,404.19210297)
\curveto(109.1313654,404.12209698)(109.09636543,404.05709704)(109.07636841,403.99710297)
\curveto(109.0263655,403.8970972)(108.98136555,403.8020973)(108.94136841,403.71210297)
\curveto(108.91136562,403.62209748)(108.84136569,403.56209754)(108.73136841,403.53210297)
\curveto(108.65136588,403.51209759)(108.56636596,403.5020976)(108.47636841,403.50210297)
\lineto(108.20636841,403.50210297)
\lineto(107.63636841,403.50210297)
\curveto(107.58636694,403.5020976)(107.53636699,403.4970976)(107.48636841,403.48710297)
\curveto(107.43636709,403.48709761)(107.39136714,403.49209761)(107.35136841,403.50210297)
\lineto(107.21636841,403.50210297)
\curveto(107.19636733,403.51209759)(107.17136736,403.51709758)(107.14136841,403.51710297)
\curveto(107.11136742,403.51709758)(107.08636744,403.52709757)(107.06636841,403.54710297)
\curveto(106.98636754,403.56709753)(106.9313676,403.63209747)(106.90136841,403.74210297)
\curveto(106.89136764,403.79209731)(106.89136764,403.84209726)(106.90136841,403.89210297)
\curveto(106.91136762,403.94209716)(106.92136761,403.98709711)(106.93136841,404.02710297)
\curveto(106.96136757,404.13709696)(106.99136754,404.23709686)(107.02136841,404.32710297)
\curveto(107.06136747,404.42709667)(107.10636742,404.51709658)(107.15636841,404.59710297)
\lineto(107.24636841,404.74710297)
\lineto(107.33636841,404.89710297)
\curveto(107.41636711,405.00709609)(107.51636701,405.11209599)(107.63636841,405.21210297)
\curveto(107.65636687,405.22209588)(107.68636684,405.24709585)(107.72636841,405.28710297)
\curveto(107.77636675,405.32709577)(107.82136671,405.36209574)(107.86136841,405.39210297)
\curveto(107.90136663,405.42209568)(107.94636658,405.45209565)(107.99636841,405.48210297)
\curveto(108.16636636,405.59209551)(108.34636618,405.67709542)(108.53636841,405.73710297)
\curveto(108.7263658,405.80709529)(108.92136561,405.87209523)(109.12136841,405.93210297)
\curveto(109.24136529,405.96209514)(109.36636516,405.98209512)(109.49636841,405.99210297)
\curveto(109.6263649,406.0020951)(109.75636477,406.02209508)(109.88636841,406.05210297)
\curveto(109.9263646,406.06209504)(109.98636454,406.06209504)(110.06636841,406.05210297)
\curveto(110.15636437,406.04209506)(110.21136432,406.04709505)(110.23136841,406.06710297)
\curveto(110.64136389,406.07709502)(111.0313635,406.06209504)(111.40136841,406.02210297)
\curveto(111.78136275,405.98209512)(112.12136241,405.90709519)(112.42136841,405.79710297)
\curveto(112.7313618,405.68709541)(112.99636153,405.53709556)(113.21636841,405.34710297)
\curveto(113.43636109,405.16709593)(113.60636092,404.93209617)(113.72636841,404.64210297)
\curveto(113.79636073,404.47209663)(113.83636069,404.27709682)(113.84636841,404.05710297)
\curveto(113.85636067,403.83709726)(113.86136067,403.61209749)(113.86136841,403.38210297)
\lineto(113.86136841,400.03710297)
\lineto(113.86136841,399.45210297)
\curveto(113.86136067,399.26210184)(113.88136065,399.08710201)(113.92136841,398.92710297)
\curveto(113.9313606,398.8971022)(113.93636059,398.86210224)(113.93636841,398.82210297)
\curveto(113.93636059,398.79210231)(113.94136059,398.76210234)(113.95136841,398.73210297)
\moveto(111.74636841,401.04210297)
\curveto(111.75636277,401.09210001)(111.76136277,401.14709995)(111.76136841,401.20710297)
\curveto(111.76136277,401.27709982)(111.75636277,401.33709976)(111.74636841,401.38710297)
\curveto(111.7263628,401.44709965)(111.71636281,401.5020996)(111.71636841,401.55210297)
\curveto(111.71636281,401.6020995)(111.69636283,401.64209946)(111.65636841,401.67210297)
\curveto(111.60636292,401.71209939)(111.531363,401.73209937)(111.43136841,401.73210297)
\curveto(111.39136314,401.72209938)(111.35636317,401.71209939)(111.32636841,401.70210297)
\curveto(111.29636323,401.7020994)(111.26136327,401.6970994)(111.22136841,401.68710297)
\curveto(111.15136338,401.66709943)(111.07636345,401.65209945)(110.99636841,401.64210297)
\curveto(110.91636361,401.63209947)(110.83636369,401.61709948)(110.75636841,401.59710297)
\curveto(110.7263638,401.58709951)(110.68136385,401.58209952)(110.62136841,401.58210297)
\curveto(110.49136404,401.55209955)(110.36136417,401.53209957)(110.23136841,401.52210297)
\curveto(110.10136443,401.51209959)(109.97636455,401.48709961)(109.85636841,401.44710297)
\curveto(109.77636475,401.42709967)(109.70136483,401.40709969)(109.63136841,401.38710297)
\curveto(109.56136497,401.37709972)(109.49136504,401.35709974)(109.42136841,401.32710297)
\curveto(109.21136532,401.23709986)(109.0313655,401.1021)(108.88136841,400.92210297)
\curveto(108.74136579,400.74210036)(108.69136584,400.49210061)(108.73136841,400.17210297)
\curveto(108.75136578,400.0021011)(108.80636572,399.86210124)(108.89636841,399.75210297)
\curveto(108.96636556,399.64210146)(109.07136546,399.55210155)(109.21136841,399.48210297)
\curveto(109.35136518,399.42210168)(109.50136503,399.37710172)(109.66136841,399.34710297)
\curveto(109.8313647,399.31710178)(110.00636452,399.30710179)(110.18636841,399.31710297)
\curveto(110.37636415,399.33710176)(110.55136398,399.37210173)(110.71136841,399.42210297)
\curveto(110.97136356,399.5021016)(111.17636335,399.62710147)(111.32636841,399.79710297)
\curveto(111.47636305,399.97710112)(111.59136294,400.1971009)(111.67136841,400.45710297)
\curveto(111.69136284,400.52710057)(111.70136283,400.5971005)(111.70136841,400.66710297)
\curveto(111.71136282,400.74710035)(111.7263628,400.82710027)(111.74636841,400.90710297)
\lineto(111.74636841,401.04210297)
}
}
{
\newrgbcolor{curcolor}{0 0 0}
\pscustom[linestyle=none,fillstyle=solid,fillcolor=curcolor]
{
\newpath
\moveto(119.93964966,406.06710297)
\curveto(120.53964385,406.08709501)(121.03964335,406.0020951)(121.43964966,405.81210297)
\curveto(121.83964255,405.62209548)(122.15464224,405.34209576)(122.38464966,404.97210297)
\curveto(122.45464194,404.86209624)(122.50964188,404.74209636)(122.54964966,404.61210297)
\curveto(122.5896418,404.49209661)(122.62964176,404.36709673)(122.66964966,404.23710297)
\curveto(122.6896417,404.15709694)(122.69964169,404.08209702)(122.69964966,404.01210297)
\curveto(122.70964168,403.94209716)(122.72464167,403.87209723)(122.74464966,403.80210297)
\curveto(122.74464165,403.74209736)(122.74964164,403.7020974)(122.75964966,403.68210297)
\curveto(122.77964161,403.54209756)(122.7896416,403.3970977)(122.78964966,403.24710297)
\lineto(122.78964966,402.81210297)
\lineto(122.78964966,401.47710297)
\lineto(122.78964966,399.04710297)
\curveto(122.7896416,398.85710224)(122.78464161,398.67210243)(122.77464966,398.49210297)
\curveto(122.77464162,398.32210278)(122.70464169,398.21210289)(122.56464966,398.16210297)
\curveto(122.50464189,398.14210296)(122.43464196,398.13210297)(122.35464966,398.13210297)
\lineto(122.11464966,398.13210297)
\lineto(121.30464966,398.13210297)
\curveto(121.18464321,398.13210297)(121.07464332,398.13710296)(120.97464966,398.14710297)
\curveto(120.88464351,398.16710293)(120.81464358,398.21210289)(120.76464966,398.28210297)
\curveto(120.72464367,398.34210276)(120.69964369,398.41710268)(120.68964966,398.50710297)
\lineto(120.68964966,398.82210297)
\lineto(120.68964966,399.87210297)
\lineto(120.68964966,402.10710297)
\curveto(120.6896437,402.47709862)(120.67464372,402.81709828)(120.64464966,403.12710297)
\curveto(120.61464378,403.44709765)(120.52464387,403.71709738)(120.37464966,403.93710297)
\curveto(120.23464416,404.13709696)(120.02964436,404.27709682)(119.75964966,404.35710297)
\curveto(119.70964468,404.37709672)(119.65464474,404.38709671)(119.59464966,404.38710297)
\curveto(119.54464485,404.38709671)(119.4896449,404.3970967)(119.42964966,404.41710297)
\curveto(119.37964501,404.42709667)(119.31464508,404.42709667)(119.23464966,404.41710297)
\curveto(119.16464523,404.41709668)(119.10964528,404.41209669)(119.06964966,404.40210297)
\curveto(119.02964536,404.39209671)(118.9946454,404.38709671)(118.96464966,404.38710297)
\curveto(118.93464546,404.38709671)(118.90464549,404.38209672)(118.87464966,404.37210297)
\curveto(118.64464575,404.31209679)(118.45964593,404.23209687)(118.31964966,404.13210297)
\curveto(117.99964639,403.9020972)(117.80964658,403.56709753)(117.74964966,403.12710297)
\curveto(117.6896467,402.68709841)(117.65964673,402.19209891)(117.65964966,401.64210297)
\lineto(117.65964966,399.76710297)
\lineto(117.65964966,398.85210297)
\lineto(117.65964966,398.58210297)
\curveto(117.65964673,398.49210261)(117.64464675,398.41710268)(117.61464966,398.35710297)
\curveto(117.56464683,398.24710285)(117.48464691,398.18210292)(117.37464966,398.16210297)
\curveto(117.26464713,398.14210296)(117.12964726,398.13210297)(116.96964966,398.13210297)
\lineto(116.21964966,398.13210297)
\curveto(116.10964828,398.13210297)(115.99964839,398.13710296)(115.88964966,398.14710297)
\curveto(115.77964861,398.15710294)(115.69964869,398.19210291)(115.64964966,398.25210297)
\curveto(115.57964881,398.34210276)(115.54464885,398.47210263)(115.54464966,398.64210297)
\curveto(115.55464884,398.81210229)(115.55964883,398.97210213)(115.55964966,399.12210297)
\lineto(115.55964966,401.16210297)
\lineto(115.55964966,404.46210297)
\lineto(115.55964966,405.22710297)
\lineto(115.55964966,405.52710297)
\curveto(115.56964882,405.61709548)(115.59964879,405.69209541)(115.64964966,405.75210297)
\curveto(115.66964872,405.78209532)(115.69964869,405.8020953)(115.73964966,405.81210297)
\curveto(115.7896486,405.83209527)(115.83964855,405.84709525)(115.88964966,405.85710297)
\lineto(115.96464966,405.85710297)
\curveto(116.01464838,405.86709523)(116.06464833,405.87209523)(116.11464966,405.87210297)
\lineto(116.27964966,405.87210297)
\lineto(116.90964966,405.87210297)
\curveto(116.9896474,405.87209523)(117.06464733,405.86709523)(117.13464966,405.85710297)
\curveto(117.21464718,405.85709524)(117.28464711,405.84709525)(117.34464966,405.82710297)
\curveto(117.41464698,405.7970953)(117.45964693,405.75209535)(117.47964966,405.69210297)
\curveto(117.50964688,405.63209547)(117.53464686,405.56209554)(117.55464966,405.48210297)
\curveto(117.56464683,405.44209566)(117.56464683,405.40709569)(117.55464966,405.37710297)
\curveto(117.55464684,405.34709575)(117.56464683,405.31709578)(117.58464966,405.28710297)
\curveto(117.60464679,405.23709586)(117.61964677,405.20709589)(117.62964966,405.19710297)
\curveto(117.64964674,405.18709591)(117.67464672,405.17209593)(117.70464966,405.15210297)
\curveto(117.81464658,405.14209596)(117.90464649,405.17709592)(117.97464966,405.25710297)
\curveto(118.04464635,405.34709575)(118.11964627,405.41709568)(118.19964966,405.46710297)
\curveto(118.46964592,405.66709543)(118.76964562,405.82709527)(119.09964966,405.94710297)
\curveto(119.1896452,405.97709512)(119.27964511,405.9970951)(119.36964966,406.00710297)
\curveto(119.46964492,406.01709508)(119.57464482,406.03209507)(119.68464966,406.05210297)
\curveto(119.71464468,406.06209504)(119.75964463,406.06209504)(119.81964966,406.05210297)
\curveto(119.87964451,406.05209505)(119.91964447,406.05709504)(119.93964966,406.06710297)
}
}
{
\newrgbcolor{curcolor}{0 0 0}
\pscustom[linestyle=none,fillstyle=solid,fillcolor=curcolor]
{
\newpath
\moveto(125.47089966,408.18210297)
\lineto(126.47589966,408.18210297)
\curveto(126.62589667,408.18209292)(126.75589654,408.17209293)(126.86589966,408.15210297)
\curveto(126.98589631,408.14209296)(127.07089623,408.08209302)(127.12089966,407.97210297)
\curveto(127.14089616,407.92209318)(127.15089615,407.86209324)(127.15089966,407.79210297)
\lineto(127.15089966,407.58210297)
\lineto(127.15089966,406.90710297)
\curveto(127.15089615,406.85709424)(127.14589615,406.7970943)(127.13589966,406.72710297)
\curveto(127.13589616,406.66709443)(127.14089616,406.61209449)(127.15089966,406.56210297)
\lineto(127.15089966,406.39710297)
\curveto(127.15089615,406.31709478)(127.15589614,406.24209486)(127.16589966,406.17210297)
\curveto(127.17589612,406.11209499)(127.2008961,406.05709504)(127.24089966,406.00710297)
\curveto(127.31089599,405.91709518)(127.43589586,405.86709523)(127.61589966,405.85710297)
\lineto(128.15589966,405.85710297)
\lineto(128.33589966,405.85710297)
\curveto(128.3958949,405.85709524)(128.45089485,405.84709525)(128.50089966,405.82710297)
\curveto(128.61089469,405.77709532)(128.67089463,405.68709541)(128.68089966,405.55710297)
\curveto(128.7008946,405.42709567)(128.71089459,405.28209582)(128.71089966,405.12210297)
\lineto(128.71089966,404.91210297)
\curveto(128.72089458,404.84209626)(128.71589458,404.78209632)(128.69589966,404.73210297)
\curveto(128.64589465,404.57209653)(128.54089476,404.48709661)(128.38089966,404.47710297)
\curveto(128.22089508,404.46709663)(128.04089526,404.46209664)(127.84089966,404.46210297)
\lineto(127.70589966,404.46210297)
\curveto(127.66589563,404.47209663)(127.63089567,404.47209663)(127.60089966,404.46210297)
\curveto(127.56089574,404.45209665)(127.52589577,404.44709665)(127.49589966,404.44710297)
\curveto(127.46589583,404.45709664)(127.43589586,404.45209665)(127.40589966,404.43210297)
\curveto(127.32589597,404.41209669)(127.26589603,404.36709673)(127.22589966,404.29710297)
\curveto(127.1958961,404.23709686)(127.17089613,404.16209694)(127.15089966,404.07210297)
\curveto(127.14089616,404.02209708)(127.14089616,403.96709713)(127.15089966,403.90710297)
\curveto(127.16089614,403.84709725)(127.16089614,403.79209731)(127.15089966,403.74210297)
\lineto(127.15089966,402.81210297)
\lineto(127.15089966,401.05710297)
\curveto(127.15089615,400.80710029)(127.15589614,400.58710051)(127.16589966,400.39710297)
\curveto(127.18589611,400.21710088)(127.25089605,400.05710104)(127.36089966,399.91710297)
\curveto(127.41089589,399.85710124)(127.47589582,399.81210129)(127.55589966,399.78210297)
\lineto(127.82589966,399.72210297)
\curveto(127.85589544,399.71210139)(127.88589541,399.70710139)(127.91589966,399.70710297)
\curveto(127.95589534,399.71710138)(127.98589531,399.71710138)(128.00589966,399.70710297)
\lineto(128.17089966,399.70710297)
\curveto(128.28089502,399.70710139)(128.37589492,399.7021014)(128.45589966,399.69210297)
\curveto(128.53589476,399.68210142)(128.6008947,399.64210146)(128.65089966,399.57210297)
\curveto(128.69089461,399.51210159)(128.71089459,399.43210167)(128.71089966,399.33210297)
\lineto(128.71089966,399.04710297)
\curveto(128.71089459,398.83710226)(128.70589459,398.64210246)(128.69589966,398.46210297)
\curveto(128.6958946,398.29210281)(128.61589468,398.17710292)(128.45589966,398.11710297)
\curveto(128.40589489,398.097103)(128.36089494,398.09210301)(128.32089966,398.10210297)
\curveto(128.28089502,398.102103)(128.23589506,398.09210301)(128.18589966,398.07210297)
\lineto(128.03589966,398.07210297)
\curveto(128.01589528,398.07210303)(127.98589531,398.07710302)(127.94589966,398.08710297)
\curveto(127.90589539,398.08710301)(127.87089543,398.08210302)(127.84089966,398.07210297)
\curveto(127.79089551,398.06210304)(127.73589556,398.06210304)(127.67589966,398.07210297)
\lineto(127.52589966,398.07210297)
\lineto(127.37589966,398.07210297)
\curveto(127.32589597,398.06210304)(127.28089602,398.06210304)(127.24089966,398.07210297)
\lineto(127.07589966,398.07210297)
\curveto(127.02589627,398.08210302)(126.97089633,398.08710301)(126.91089966,398.08710297)
\curveto(126.85089645,398.08710301)(126.7958965,398.09210301)(126.74589966,398.10210297)
\curveto(126.67589662,398.11210299)(126.61089669,398.12210298)(126.55089966,398.13210297)
\lineto(126.37089966,398.16210297)
\curveto(126.26089704,398.19210291)(126.15589714,398.22710287)(126.05589966,398.26710297)
\curveto(125.95589734,398.30710279)(125.86089744,398.35210275)(125.77089966,398.40210297)
\lineto(125.68089966,398.46210297)
\curveto(125.65089765,398.49210261)(125.61589768,398.52210258)(125.57589966,398.55210297)
\curveto(125.55589774,398.57210253)(125.53089777,398.59210251)(125.50089966,398.61210297)
\lineto(125.42589966,398.68710297)
\curveto(125.28589801,398.87710222)(125.18089812,399.08710201)(125.11089966,399.31710297)
\curveto(125.09089821,399.35710174)(125.08089822,399.39210171)(125.08089966,399.42210297)
\curveto(125.09089821,399.46210164)(125.09089821,399.50710159)(125.08089966,399.55710297)
\curveto(125.07089823,399.57710152)(125.06589823,399.6021015)(125.06589966,399.63210297)
\curveto(125.06589823,399.66210144)(125.06089824,399.68710141)(125.05089966,399.70710297)
\lineto(125.05089966,399.85710297)
\curveto(125.04089826,399.8971012)(125.03589826,399.94210116)(125.03589966,399.99210297)
\curveto(125.04589825,400.04210106)(125.05089825,400.09210101)(125.05089966,400.14210297)
\lineto(125.05089966,400.71210297)
\lineto(125.05089966,402.94710297)
\lineto(125.05089966,403.74210297)
\lineto(125.05089966,403.95210297)
\curveto(125.06089824,404.02209708)(125.05589824,404.08709701)(125.03589966,404.14710297)
\curveto(124.9958983,404.28709681)(124.92589837,404.37709672)(124.82589966,404.41710297)
\curveto(124.71589858,404.46709663)(124.57589872,404.48209662)(124.40589966,404.46210297)
\curveto(124.23589906,404.44209666)(124.09089921,404.45709664)(123.97089966,404.50710297)
\curveto(123.89089941,404.53709656)(123.84089946,404.58209652)(123.82089966,404.64210297)
\curveto(123.8008995,404.7020964)(123.78089952,404.77709632)(123.76089966,404.86710297)
\lineto(123.76089966,405.18210297)
\curveto(123.76089954,405.36209574)(123.77089953,405.50709559)(123.79089966,405.61710297)
\curveto(123.81089949,405.72709537)(123.8958994,405.8020953)(124.04589966,405.84210297)
\curveto(124.08589921,405.86209524)(124.12589917,405.86709523)(124.16589966,405.85710297)
\lineto(124.30089966,405.85710297)
\curveto(124.45089885,405.85709524)(124.59089871,405.86209524)(124.72089966,405.87210297)
\curveto(124.85089845,405.89209521)(124.94089836,405.95209515)(124.99089966,406.05210297)
\curveto(125.02089828,406.12209498)(125.03589826,406.2020949)(125.03589966,406.29210297)
\curveto(125.04589825,406.38209472)(125.05089825,406.47209463)(125.05089966,406.56210297)
\lineto(125.05089966,407.49210297)
\lineto(125.05089966,407.74710297)
\curveto(125.05089825,407.83709326)(125.06089824,407.91209319)(125.08089966,407.97210297)
\curveto(125.13089817,408.07209303)(125.20589809,408.13709296)(125.30589966,408.16710297)
\curveto(125.32589797,408.17709292)(125.35089795,408.17709292)(125.38089966,408.16710297)
\curveto(125.42089788,408.16709293)(125.45089785,408.17209293)(125.47089966,408.18210297)
}
}
{
\newrgbcolor{curcolor}{0 0 0}
\pscustom[linestyle=none,fillstyle=solid,fillcolor=curcolor]
{
\newpath
\moveto(131.79433716,408.72210297)
\curveto(131.86433421,408.64209246)(131.89933417,408.52209258)(131.89933716,408.36210297)
\lineto(131.89933716,407.89710297)
\lineto(131.89933716,407.49210297)
\curveto(131.89933417,407.35209375)(131.86433421,407.25709384)(131.79433716,407.20710297)
\curveto(131.73433434,407.15709394)(131.65433442,407.12709397)(131.55433716,407.11710297)
\curveto(131.46433461,407.10709399)(131.36433471,407.102094)(131.25433716,407.10210297)
\lineto(130.41433716,407.10210297)
\curveto(130.30433577,407.102094)(130.20433587,407.10709399)(130.11433716,407.11710297)
\curveto(130.03433604,407.12709397)(129.96433611,407.15709394)(129.90433716,407.20710297)
\curveto(129.86433621,407.23709386)(129.83433624,407.29209381)(129.81433716,407.37210297)
\curveto(129.80433627,407.46209364)(129.79433628,407.55709354)(129.78433716,407.65710297)
\lineto(129.78433716,407.98710297)
\curveto(129.79433628,408.097093)(129.79933627,408.19209291)(129.79933716,408.27210297)
\lineto(129.79933716,408.48210297)
\curveto(129.80933626,408.55209255)(129.82933624,408.61209249)(129.85933716,408.66210297)
\curveto(129.87933619,408.7020924)(129.90433617,408.73209237)(129.93433716,408.75210297)
\lineto(130.05433716,408.81210297)
\curveto(130.074336,408.81209229)(130.09933597,408.81209229)(130.12933716,408.81210297)
\curveto(130.15933591,408.82209228)(130.18433589,408.82709227)(130.20433716,408.82710297)
\lineto(131.29933716,408.82710297)
\curveto(131.39933467,408.82709227)(131.49433458,408.82209228)(131.58433716,408.81210297)
\curveto(131.6743344,408.8020923)(131.74433433,408.77209233)(131.79433716,408.72210297)
\moveto(131.89933716,398.95710297)
\curveto(131.89933417,398.75710234)(131.89433418,398.58710251)(131.88433716,398.44710297)
\curveto(131.8743342,398.30710279)(131.78433429,398.21210289)(131.61433716,398.16210297)
\curveto(131.55433452,398.14210296)(131.48933458,398.13210297)(131.41933716,398.13210297)
\curveto(131.34933472,398.14210296)(131.2743348,398.14710295)(131.19433716,398.14710297)
\lineto(130.35433716,398.14710297)
\curveto(130.26433581,398.14710295)(130.1743359,398.15210295)(130.08433716,398.16210297)
\curveto(130.00433607,398.17210293)(129.94433613,398.2021029)(129.90433716,398.25210297)
\curveto(129.84433623,398.32210278)(129.80933626,398.40710269)(129.79933716,398.50710297)
\lineto(129.79933716,398.85210297)
\lineto(129.79933716,405.18210297)
\lineto(129.79933716,405.48210297)
\curveto(129.79933627,405.58209552)(129.81933625,405.66209544)(129.85933716,405.72210297)
\curveto(129.91933615,405.79209531)(130.00433607,405.83709526)(130.11433716,405.85710297)
\curveto(130.13433594,405.86709523)(130.15933591,405.86709523)(130.18933716,405.85710297)
\curveto(130.22933584,405.85709524)(130.25933581,405.86209524)(130.27933716,405.87210297)
\lineto(131.02933716,405.87210297)
\lineto(131.22433716,405.87210297)
\curveto(131.30433477,405.88209522)(131.3693347,405.88209522)(131.41933716,405.87210297)
\lineto(131.53933716,405.87210297)
\curveto(131.59933447,405.85209525)(131.65433442,405.83709526)(131.70433716,405.82710297)
\curveto(131.75433432,405.81709528)(131.79433428,405.78709531)(131.82433716,405.73710297)
\curveto(131.86433421,405.68709541)(131.88433419,405.61709548)(131.88433716,405.52710297)
\curveto(131.89433418,405.43709566)(131.89933417,405.34209576)(131.89933716,405.24210297)
\lineto(131.89933716,398.95710297)
}
}
{
\newrgbcolor{curcolor}{0 0 0}
\pscustom[linestyle=none,fillstyle=solid,fillcolor=curcolor]
{
\newpath
\moveto(141.15152466,398.98710297)
\lineto(141.15152466,398.56710297)
\curveto(141.15151629,398.43710266)(141.12151632,398.33210277)(141.06152466,398.25210297)
\curveto(141.01151643,398.2021029)(140.94651649,398.16710293)(140.86652466,398.14710297)
\curveto(140.78651665,398.13710296)(140.69651674,398.13210297)(140.59652466,398.13210297)
\lineto(139.77152466,398.13210297)
\lineto(139.48652466,398.13210297)
\curveto(139.40651803,398.14210296)(139.3415181,398.16710293)(139.29152466,398.20710297)
\curveto(139.22151822,398.25710284)(139.18151826,398.32210278)(139.17152466,398.40210297)
\curveto(139.16151828,398.48210262)(139.1415183,398.56210254)(139.11152466,398.64210297)
\curveto(139.09151835,398.66210244)(139.07151837,398.67710242)(139.05152466,398.68710297)
\curveto(139.0415184,398.70710239)(139.02651841,398.72710237)(139.00652466,398.74710297)
\curveto(138.89651854,398.74710235)(138.81651862,398.72210238)(138.76652466,398.67210297)
\lineto(138.61652466,398.52210297)
\curveto(138.54651889,398.47210263)(138.48151896,398.42710267)(138.42152466,398.38710297)
\curveto(138.36151908,398.35710274)(138.29651914,398.31710278)(138.22652466,398.26710297)
\curveto(138.18651925,398.24710285)(138.1415193,398.22710287)(138.09152466,398.20710297)
\curveto(138.05151939,398.18710291)(138.00651943,398.16710293)(137.95652466,398.14710297)
\curveto(137.81651962,398.097103)(137.66651977,398.05210305)(137.50652466,398.01210297)
\curveto(137.45651998,397.99210311)(137.41152003,397.98210312)(137.37152466,397.98210297)
\curveto(137.33152011,397.98210312)(137.29152015,397.97710312)(137.25152466,397.96710297)
\lineto(137.11652466,397.96710297)
\curveto(137.08652035,397.95710314)(137.04652039,397.95210315)(136.99652466,397.95210297)
\lineto(136.86152466,397.95210297)
\curveto(136.80152064,397.93210317)(136.71152073,397.92710317)(136.59152466,397.93710297)
\curveto(136.47152097,397.93710316)(136.38652105,397.94710315)(136.33652466,397.96710297)
\curveto(136.26652117,397.98710311)(136.20152124,397.9971031)(136.14152466,397.99710297)
\curveto(136.09152135,397.98710311)(136.0365214,397.99210311)(135.97652466,398.01210297)
\lineto(135.61652466,398.13210297)
\curveto(135.50652193,398.16210294)(135.39652204,398.2021029)(135.28652466,398.25210297)
\curveto(134.9365225,398.4021027)(134.62152282,398.63210247)(134.34152466,398.94210297)
\curveto(134.07152337,399.26210184)(133.85652358,399.5971015)(133.69652466,399.94710297)
\curveto(133.64652379,400.05710104)(133.60652383,400.16210094)(133.57652466,400.26210297)
\curveto(133.54652389,400.37210073)(133.51152393,400.48210062)(133.47152466,400.59210297)
\curveto(133.46152398,400.63210047)(133.45652398,400.66710043)(133.45652466,400.69710297)
\curveto(133.45652398,400.73710036)(133.44652399,400.78210032)(133.42652466,400.83210297)
\curveto(133.40652403,400.91210019)(133.38652405,400.9971001)(133.36652466,401.08710297)
\curveto(133.35652408,401.18709991)(133.3415241,401.28709981)(133.32152466,401.38710297)
\curveto(133.31152413,401.41709968)(133.30652413,401.45209965)(133.30652466,401.49210297)
\curveto(133.31652412,401.53209957)(133.31652412,401.56709953)(133.30652466,401.59710297)
\lineto(133.30652466,401.73210297)
\curveto(133.30652413,401.78209932)(133.30152414,401.83209927)(133.29152466,401.88210297)
\curveto(133.28152416,401.93209917)(133.27652416,401.98709911)(133.27652466,402.04710297)
\curveto(133.27652416,402.11709898)(133.28152416,402.17209893)(133.29152466,402.21210297)
\curveto(133.30152414,402.26209884)(133.30652413,402.30709879)(133.30652466,402.34710297)
\lineto(133.30652466,402.49710297)
\curveto(133.31652412,402.54709855)(133.31652412,402.59209851)(133.30652466,402.63210297)
\curveto(133.30652413,402.68209842)(133.31652412,402.73209837)(133.33652466,402.78210297)
\curveto(133.35652408,402.89209821)(133.37152407,402.9970981)(133.38152466,403.09710297)
\curveto(133.40152404,403.1970979)(133.42652401,403.2970978)(133.45652466,403.39710297)
\curveto(133.49652394,403.51709758)(133.53152391,403.63209747)(133.56152466,403.74210297)
\curveto(133.59152385,403.85209725)(133.63152381,403.96209714)(133.68152466,404.07210297)
\curveto(133.82152362,404.37209673)(133.99652344,404.65709644)(134.20652466,404.92710297)
\curveto(134.22652321,404.95709614)(134.25152319,404.98209612)(134.28152466,405.00210297)
\curveto(134.32152312,405.03209607)(134.35152309,405.06209604)(134.37152466,405.09210297)
\curveto(134.41152303,405.14209596)(134.45152299,405.18709591)(134.49152466,405.22710297)
\curveto(134.53152291,405.26709583)(134.57652286,405.30709579)(134.62652466,405.34710297)
\curveto(134.66652277,405.36709573)(134.70152274,405.39209571)(134.73152466,405.42210297)
\curveto(134.76152268,405.46209564)(134.79652264,405.49209561)(134.83652466,405.51210297)
\curveto(135.08652235,405.68209542)(135.37652206,405.82209528)(135.70652466,405.93210297)
\curveto(135.77652166,405.95209515)(135.84652159,405.96709513)(135.91652466,405.97710297)
\curveto(135.99652144,405.98709511)(136.07652136,406.0020951)(136.15652466,406.02210297)
\curveto(136.22652121,406.04209506)(136.31652112,406.05209505)(136.42652466,406.05210297)
\curveto(136.5365209,406.06209504)(136.64652079,406.06709503)(136.75652466,406.06710297)
\curveto(136.86652057,406.06709503)(136.97152047,406.06209504)(137.07152466,406.05210297)
\curveto(137.18152026,406.04209506)(137.27152017,406.02709507)(137.34152466,406.00710297)
\curveto(137.49151995,405.95709514)(137.6365198,405.91209519)(137.77652466,405.87210297)
\curveto(137.91651952,405.83209527)(138.04651939,405.77709532)(138.16652466,405.70710297)
\curveto(138.2365192,405.65709544)(138.30151914,405.60709549)(138.36152466,405.55710297)
\curveto(138.42151902,405.51709558)(138.48651895,405.47209563)(138.55652466,405.42210297)
\curveto(138.59651884,405.39209571)(138.65151879,405.35209575)(138.72152466,405.30210297)
\curveto(138.80151864,405.25209585)(138.87651856,405.25209585)(138.94652466,405.30210297)
\curveto(138.98651845,405.32209578)(139.00651843,405.35709574)(139.00652466,405.40710297)
\curveto(139.00651843,405.45709564)(139.01651842,405.50709559)(139.03652466,405.55710297)
\lineto(139.03652466,405.70710297)
\curveto(139.04651839,405.73709536)(139.05151839,405.77209533)(139.05152466,405.81210297)
\lineto(139.05152466,405.93210297)
\lineto(139.05152466,407.97210297)
\curveto(139.05151839,408.08209302)(139.04651839,408.2020929)(139.03652466,408.33210297)
\curveto(139.0365184,408.47209263)(139.06151838,408.57709252)(139.11152466,408.64710297)
\curveto(139.15151829,408.72709237)(139.22651821,408.77709232)(139.33652466,408.79710297)
\curveto(139.35651808,408.80709229)(139.37651806,408.80709229)(139.39652466,408.79710297)
\curveto(139.41651802,408.7970923)(139.436518,408.8020923)(139.45652466,408.81210297)
\lineto(140.52152466,408.81210297)
\curveto(140.6415168,408.81209229)(140.75151669,408.80709229)(140.85152466,408.79710297)
\curveto(140.95151649,408.78709231)(141.02651641,408.74709235)(141.07652466,408.67710297)
\curveto(141.12651631,408.5970925)(141.15151629,408.49209261)(141.15152466,408.36210297)
\lineto(141.15152466,408.00210297)
\lineto(141.15152466,398.98710297)
\moveto(139.11152466,401.92710297)
\curveto(139.12151832,401.96709913)(139.12151832,402.00709909)(139.11152466,402.04710297)
\lineto(139.11152466,402.18210297)
\curveto(139.11151833,402.28209882)(139.10651833,402.38209872)(139.09652466,402.48210297)
\curveto(139.08651835,402.58209852)(139.07151837,402.67209843)(139.05152466,402.75210297)
\curveto(139.03151841,402.86209824)(139.01151843,402.96209814)(138.99152466,403.05210297)
\curveto(138.98151846,403.14209796)(138.95651848,403.22709787)(138.91652466,403.30710297)
\curveto(138.77651866,403.66709743)(138.57151887,403.95209715)(138.30152466,404.16210297)
\curveto(138.0415194,404.37209673)(137.66151978,404.47709662)(137.16152466,404.47710297)
\curveto(137.10152034,404.47709662)(137.02152042,404.46709663)(136.92152466,404.44710297)
\curveto(136.8415206,404.42709667)(136.76652067,404.40709669)(136.69652466,404.38710297)
\curveto(136.6365208,404.37709672)(136.57652086,404.35709674)(136.51652466,404.32710297)
\curveto(136.24652119,404.21709688)(136.0365214,404.04709705)(135.88652466,403.81710297)
\curveto(135.7365217,403.58709751)(135.61652182,403.32709777)(135.52652466,403.03710297)
\curveto(135.49652194,402.93709816)(135.47652196,402.83709826)(135.46652466,402.73710297)
\curveto(135.45652198,402.63709846)(135.436522,402.53209857)(135.40652466,402.42210297)
\lineto(135.40652466,402.21210297)
\curveto(135.38652205,402.12209898)(135.38152206,401.9970991)(135.39152466,401.83710297)
\curveto(135.40152204,401.68709941)(135.41652202,401.57709952)(135.43652466,401.50710297)
\lineto(135.43652466,401.41710297)
\curveto(135.44652199,401.3970997)(135.45152199,401.37709972)(135.45152466,401.35710297)
\curveto(135.47152197,401.27709982)(135.48652195,401.2020999)(135.49652466,401.13210297)
\curveto(135.51652192,401.06210004)(135.5365219,400.98710011)(135.55652466,400.90710297)
\curveto(135.72652171,400.38710071)(136.01652142,400.0021011)(136.42652466,399.75210297)
\curveto(136.55652088,399.66210144)(136.7365207,399.59210151)(136.96652466,399.54210297)
\curveto(137.00652043,399.53210157)(137.06652037,399.52710157)(137.14652466,399.52710297)
\curveto(137.17652026,399.51710158)(137.22152022,399.50710159)(137.28152466,399.49710297)
\curveto(137.35152009,399.4971016)(137.40652003,399.5021016)(137.44652466,399.51210297)
\curveto(137.52651991,399.53210157)(137.60651983,399.54710155)(137.68652466,399.55710297)
\curveto(137.76651967,399.56710153)(137.84651959,399.58710151)(137.92652466,399.61710297)
\curveto(138.17651926,399.72710137)(138.37651906,399.86710123)(138.52652466,400.03710297)
\curveto(138.67651876,400.20710089)(138.80651863,400.42210068)(138.91652466,400.68210297)
\curveto(138.95651848,400.77210033)(138.98651845,400.86210024)(139.00652466,400.95210297)
\curveto(139.02651841,401.05210005)(139.04651839,401.15709994)(139.06652466,401.26710297)
\curveto(139.07651836,401.31709978)(139.07651836,401.36209974)(139.06652466,401.40210297)
\curveto(139.06651837,401.45209965)(139.07651836,401.5020996)(139.09652466,401.55210297)
\curveto(139.10651833,401.58209952)(139.11151833,401.61709948)(139.11152466,401.65710297)
\lineto(139.11152466,401.79210297)
\lineto(139.11152466,401.92710297)
}
}
{
\newrgbcolor{curcolor}{0 0 0}
\pscustom[linestyle=none,fillstyle=solid,fillcolor=curcolor]
{
\newpath
\moveto(149.78144653,398.73210297)
\curveto(149.80143868,398.62210248)(149.81143867,398.51210259)(149.81144653,398.40210297)
\curveto(149.82143866,398.29210281)(149.77143871,398.21710288)(149.66144653,398.17710297)
\curveto(149.60143888,398.14710295)(149.53143895,398.13210297)(149.45144653,398.13210297)
\lineto(149.21144653,398.13210297)
\lineto(148.40144653,398.13210297)
\lineto(148.13144653,398.13210297)
\curveto(148.05144043,398.14210296)(147.9864405,398.16710293)(147.93644653,398.20710297)
\curveto(147.86644062,398.24710285)(147.81144067,398.3021028)(147.77144653,398.37210297)
\curveto(147.74144074,398.45210265)(147.69644079,398.51710258)(147.63644653,398.56710297)
\curveto(147.61644087,398.58710251)(147.59144089,398.6021025)(147.56144653,398.61210297)
\curveto(147.53144095,398.63210247)(147.49144099,398.63710246)(147.44144653,398.62710297)
\curveto(147.39144109,398.60710249)(147.34144114,398.58210252)(147.29144653,398.55210297)
\curveto(147.25144123,398.52210258)(147.20644128,398.4971026)(147.15644653,398.47710297)
\curveto(147.10644138,398.43710266)(147.05144143,398.4021027)(146.99144653,398.37210297)
\lineto(146.81144653,398.28210297)
\curveto(146.6814418,398.22210288)(146.54644194,398.17210293)(146.40644653,398.13210297)
\curveto(146.26644222,398.102103)(146.12144236,398.06710303)(145.97144653,398.02710297)
\curveto(145.90144258,398.00710309)(145.83144265,397.9971031)(145.76144653,397.99710297)
\curveto(145.70144278,397.98710311)(145.63644285,397.97710312)(145.56644653,397.96710297)
\lineto(145.47644653,397.96710297)
\curveto(145.44644304,397.95710314)(145.41644307,397.95210315)(145.38644653,397.95210297)
\lineto(145.22144653,397.95210297)
\curveto(145.12144336,397.93210317)(145.02144346,397.93210317)(144.92144653,397.95210297)
\lineto(144.78644653,397.95210297)
\curveto(144.71644377,397.97210313)(144.64644384,397.98210312)(144.57644653,397.98210297)
\curveto(144.51644397,397.97210313)(144.45644403,397.97710312)(144.39644653,397.99710297)
\curveto(144.29644419,398.01710308)(144.20144428,398.03710306)(144.11144653,398.05710297)
\curveto(144.02144446,398.06710303)(143.93644455,398.09210301)(143.85644653,398.13210297)
\curveto(143.56644492,398.24210286)(143.31644517,398.38210272)(143.10644653,398.55210297)
\curveto(142.90644558,398.73210237)(142.74644574,398.96710213)(142.62644653,399.25710297)
\curveto(142.59644589,399.32710177)(142.56644592,399.4021017)(142.53644653,399.48210297)
\curveto(142.51644597,399.56210154)(142.49644599,399.64710145)(142.47644653,399.73710297)
\curveto(142.45644603,399.78710131)(142.44644604,399.83710126)(142.44644653,399.88710297)
\curveto(142.45644603,399.93710116)(142.45644603,399.98710111)(142.44644653,400.03710297)
\curveto(142.43644605,400.06710103)(142.42644606,400.12710097)(142.41644653,400.21710297)
\curveto(142.41644607,400.31710078)(142.42144606,400.38710071)(142.43144653,400.42710297)
\curveto(142.45144603,400.52710057)(142.46144602,400.61210049)(142.46144653,400.68210297)
\lineto(142.55144653,401.01210297)
\curveto(142.5814459,401.13209997)(142.62144586,401.23709986)(142.67144653,401.32710297)
\curveto(142.84144564,401.61709948)(143.03644545,401.83709926)(143.25644653,401.98710297)
\curveto(143.47644501,402.13709896)(143.75644473,402.26709883)(144.09644653,402.37710297)
\curveto(144.22644426,402.42709867)(144.36144412,402.46209864)(144.50144653,402.48210297)
\curveto(144.64144384,402.5020986)(144.7814437,402.52709857)(144.92144653,402.55710297)
\curveto(145.00144348,402.57709852)(145.0864434,402.58709851)(145.17644653,402.58710297)
\curveto(145.26644322,402.5970985)(145.35644313,402.61209849)(145.44644653,402.63210297)
\curveto(145.51644297,402.65209845)(145.5864429,402.65709844)(145.65644653,402.64710297)
\curveto(145.72644276,402.64709845)(145.80144268,402.65709844)(145.88144653,402.67710297)
\curveto(145.95144253,402.6970984)(146.02144246,402.70709839)(146.09144653,402.70710297)
\curveto(146.16144232,402.70709839)(146.23644225,402.71709838)(146.31644653,402.73710297)
\curveto(146.52644196,402.78709831)(146.71644177,402.82709827)(146.88644653,402.85710297)
\curveto(147.06644142,402.8970982)(147.22644126,402.98709811)(147.36644653,403.12710297)
\curveto(147.45644103,403.21709788)(147.51644097,403.31709778)(147.54644653,403.42710297)
\curveto(147.55644093,403.45709764)(147.55644093,403.48209762)(147.54644653,403.50210297)
\curveto(147.54644094,403.52209758)(147.55144093,403.54209756)(147.56144653,403.56210297)
\curveto(147.57144091,403.58209752)(147.57644091,403.61209749)(147.57644653,403.65210297)
\lineto(147.57644653,403.74210297)
\lineto(147.54644653,403.86210297)
\curveto(147.54644094,403.9020972)(147.54144094,403.93709716)(147.53144653,403.96710297)
\curveto(147.43144105,404.26709683)(147.22144126,404.47209663)(146.90144653,404.58210297)
\curveto(146.81144167,404.61209649)(146.70144178,404.63209647)(146.57144653,404.64210297)
\curveto(146.45144203,404.66209644)(146.32644216,404.66709643)(146.19644653,404.65710297)
\curveto(146.06644242,404.65709644)(145.94144254,404.64709645)(145.82144653,404.62710297)
\curveto(145.70144278,404.60709649)(145.59644289,404.58209652)(145.50644653,404.55210297)
\curveto(145.44644304,404.53209657)(145.3864431,404.5020966)(145.32644653,404.46210297)
\curveto(145.27644321,404.43209667)(145.22644326,404.3970967)(145.17644653,404.35710297)
\curveto(145.12644336,404.31709678)(145.07144341,404.26209684)(145.01144653,404.19210297)
\curveto(144.96144352,404.12209698)(144.92644356,404.05709704)(144.90644653,403.99710297)
\curveto(144.85644363,403.8970972)(144.81144367,403.8020973)(144.77144653,403.71210297)
\curveto(144.74144374,403.62209748)(144.67144381,403.56209754)(144.56144653,403.53210297)
\curveto(144.481444,403.51209759)(144.39644409,403.5020976)(144.30644653,403.50210297)
\lineto(144.03644653,403.50210297)
\lineto(143.46644653,403.50210297)
\curveto(143.41644507,403.5020976)(143.36644512,403.4970976)(143.31644653,403.48710297)
\curveto(143.26644522,403.48709761)(143.22144526,403.49209761)(143.18144653,403.50210297)
\lineto(143.04644653,403.50210297)
\curveto(143.02644546,403.51209759)(143.00144548,403.51709758)(142.97144653,403.51710297)
\curveto(142.94144554,403.51709758)(142.91644557,403.52709757)(142.89644653,403.54710297)
\curveto(142.81644567,403.56709753)(142.76144572,403.63209747)(142.73144653,403.74210297)
\curveto(142.72144576,403.79209731)(142.72144576,403.84209726)(142.73144653,403.89210297)
\curveto(142.74144574,403.94209716)(142.75144573,403.98709711)(142.76144653,404.02710297)
\curveto(142.79144569,404.13709696)(142.82144566,404.23709686)(142.85144653,404.32710297)
\curveto(142.89144559,404.42709667)(142.93644555,404.51709658)(142.98644653,404.59710297)
\lineto(143.07644653,404.74710297)
\lineto(143.16644653,404.89710297)
\curveto(143.24644524,405.00709609)(143.34644514,405.11209599)(143.46644653,405.21210297)
\curveto(143.486445,405.22209588)(143.51644497,405.24709585)(143.55644653,405.28710297)
\curveto(143.60644488,405.32709577)(143.65144483,405.36209574)(143.69144653,405.39210297)
\curveto(143.73144475,405.42209568)(143.77644471,405.45209565)(143.82644653,405.48210297)
\curveto(143.99644449,405.59209551)(144.17644431,405.67709542)(144.36644653,405.73710297)
\curveto(144.55644393,405.80709529)(144.75144373,405.87209523)(144.95144653,405.93210297)
\curveto(145.07144341,405.96209514)(145.19644329,405.98209512)(145.32644653,405.99210297)
\curveto(145.45644303,406.0020951)(145.5864429,406.02209508)(145.71644653,406.05210297)
\curveto(145.75644273,406.06209504)(145.81644267,406.06209504)(145.89644653,406.05210297)
\curveto(145.9864425,406.04209506)(146.04144244,406.04709505)(146.06144653,406.06710297)
\curveto(146.47144201,406.07709502)(146.86144162,406.06209504)(147.23144653,406.02210297)
\curveto(147.61144087,405.98209512)(147.95144053,405.90709519)(148.25144653,405.79710297)
\curveto(148.56143992,405.68709541)(148.82643966,405.53709556)(149.04644653,405.34710297)
\curveto(149.26643922,405.16709593)(149.43643905,404.93209617)(149.55644653,404.64210297)
\curveto(149.62643886,404.47209663)(149.66643882,404.27709682)(149.67644653,404.05710297)
\curveto(149.6864388,403.83709726)(149.69143879,403.61209749)(149.69144653,403.38210297)
\lineto(149.69144653,400.03710297)
\lineto(149.69144653,399.45210297)
\curveto(149.69143879,399.26210184)(149.71143877,399.08710201)(149.75144653,398.92710297)
\curveto(149.76143872,398.8971022)(149.76643872,398.86210224)(149.76644653,398.82210297)
\curveto(149.76643872,398.79210231)(149.77143871,398.76210234)(149.78144653,398.73210297)
\moveto(147.57644653,401.04210297)
\curveto(147.5864409,401.09210001)(147.59144089,401.14709995)(147.59144653,401.20710297)
\curveto(147.59144089,401.27709982)(147.5864409,401.33709976)(147.57644653,401.38710297)
\curveto(147.55644093,401.44709965)(147.54644094,401.5020996)(147.54644653,401.55210297)
\curveto(147.54644094,401.6020995)(147.52644096,401.64209946)(147.48644653,401.67210297)
\curveto(147.43644105,401.71209939)(147.36144112,401.73209937)(147.26144653,401.73210297)
\curveto(147.22144126,401.72209938)(147.1864413,401.71209939)(147.15644653,401.70210297)
\curveto(147.12644136,401.7020994)(147.09144139,401.6970994)(147.05144653,401.68710297)
\curveto(146.9814415,401.66709943)(146.90644158,401.65209945)(146.82644653,401.64210297)
\curveto(146.74644174,401.63209947)(146.66644182,401.61709948)(146.58644653,401.59710297)
\curveto(146.55644193,401.58709951)(146.51144197,401.58209952)(146.45144653,401.58210297)
\curveto(146.32144216,401.55209955)(146.19144229,401.53209957)(146.06144653,401.52210297)
\curveto(145.93144255,401.51209959)(145.80644268,401.48709961)(145.68644653,401.44710297)
\curveto(145.60644288,401.42709967)(145.53144295,401.40709969)(145.46144653,401.38710297)
\curveto(145.39144309,401.37709972)(145.32144316,401.35709974)(145.25144653,401.32710297)
\curveto(145.04144344,401.23709986)(144.86144362,401.1021)(144.71144653,400.92210297)
\curveto(144.57144391,400.74210036)(144.52144396,400.49210061)(144.56144653,400.17210297)
\curveto(144.5814439,400.0021011)(144.63644385,399.86210124)(144.72644653,399.75210297)
\curveto(144.79644369,399.64210146)(144.90144358,399.55210155)(145.04144653,399.48210297)
\curveto(145.1814433,399.42210168)(145.33144315,399.37710172)(145.49144653,399.34710297)
\curveto(145.66144282,399.31710178)(145.83644265,399.30710179)(146.01644653,399.31710297)
\curveto(146.20644228,399.33710176)(146.3814421,399.37210173)(146.54144653,399.42210297)
\curveto(146.80144168,399.5021016)(147.00644148,399.62710147)(147.15644653,399.79710297)
\curveto(147.30644118,399.97710112)(147.42144106,400.1971009)(147.50144653,400.45710297)
\curveto(147.52144096,400.52710057)(147.53144095,400.5971005)(147.53144653,400.66710297)
\curveto(147.54144094,400.74710035)(147.55644093,400.82710027)(147.57644653,400.90710297)
\lineto(147.57644653,401.04210297)
}
}
{
\newrgbcolor{curcolor}{0 0 0}
\pscustom[linestyle=none,fillstyle=solid,fillcolor=curcolor]
{
\newpath
\moveto(158.93472778,398.98710297)
\lineto(158.93472778,398.56710297)
\curveto(158.93471941,398.43710266)(158.90471944,398.33210277)(158.84472778,398.25210297)
\curveto(158.79471955,398.2021029)(158.72971962,398.16710293)(158.64972778,398.14710297)
\curveto(158.56971978,398.13710296)(158.47971987,398.13210297)(158.37972778,398.13210297)
\lineto(157.55472778,398.13210297)
\lineto(157.26972778,398.13210297)
\curveto(157.18972116,398.14210296)(157.12472122,398.16710293)(157.07472778,398.20710297)
\curveto(157.00472134,398.25710284)(156.96472138,398.32210278)(156.95472778,398.40210297)
\curveto(156.9447214,398.48210262)(156.92472142,398.56210254)(156.89472778,398.64210297)
\curveto(156.87472147,398.66210244)(156.85472149,398.67710242)(156.83472778,398.68710297)
\curveto(156.82472152,398.70710239)(156.80972154,398.72710237)(156.78972778,398.74710297)
\curveto(156.67972167,398.74710235)(156.59972175,398.72210238)(156.54972778,398.67210297)
\lineto(156.39972778,398.52210297)
\curveto(156.32972202,398.47210263)(156.26472208,398.42710267)(156.20472778,398.38710297)
\curveto(156.1447222,398.35710274)(156.07972227,398.31710278)(156.00972778,398.26710297)
\curveto(155.96972238,398.24710285)(155.92472242,398.22710287)(155.87472778,398.20710297)
\curveto(155.83472251,398.18710291)(155.78972256,398.16710293)(155.73972778,398.14710297)
\curveto(155.59972275,398.097103)(155.4497229,398.05210305)(155.28972778,398.01210297)
\curveto(155.23972311,397.99210311)(155.19472315,397.98210312)(155.15472778,397.98210297)
\curveto(155.11472323,397.98210312)(155.07472327,397.97710312)(155.03472778,397.96710297)
\lineto(154.89972778,397.96710297)
\curveto(154.86972348,397.95710314)(154.82972352,397.95210315)(154.77972778,397.95210297)
\lineto(154.64472778,397.95210297)
\curveto(154.58472376,397.93210317)(154.49472385,397.92710317)(154.37472778,397.93710297)
\curveto(154.25472409,397.93710316)(154.16972418,397.94710315)(154.11972778,397.96710297)
\curveto(154.0497243,397.98710311)(153.98472436,397.9971031)(153.92472778,397.99710297)
\curveto(153.87472447,397.98710311)(153.81972453,397.99210311)(153.75972778,398.01210297)
\lineto(153.39972778,398.13210297)
\curveto(153.28972506,398.16210294)(153.17972517,398.2021029)(153.06972778,398.25210297)
\curveto(152.71972563,398.4021027)(152.40472594,398.63210247)(152.12472778,398.94210297)
\curveto(151.85472649,399.26210184)(151.63972671,399.5971015)(151.47972778,399.94710297)
\curveto(151.42972692,400.05710104)(151.38972696,400.16210094)(151.35972778,400.26210297)
\curveto(151.32972702,400.37210073)(151.29472705,400.48210062)(151.25472778,400.59210297)
\curveto(151.2447271,400.63210047)(151.23972711,400.66710043)(151.23972778,400.69710297)
\curveto(151.23972711,400.73710036)(151.22972712,400.78210032)(151.20972778,400.83210297)
\curveto(151.18972716,400.91210019)(151.16972718,400.9971001)(151.14972778,401.08710297)
\curveto(151.13972721,401.18709991)(151.12472722,401.28709981)(151.10472778,401.38710297)
\curveto(151.09472725,401.41709968)(151.08972726,401.45209965)(151.08972778,401.49210297)
\curveto(151.09972725,401.53209957)(151.09972725,401.56709953)(151.08972778,401.59710297)
\lineto(151.08972778,401.73210297)
\curveto(151.08972726,401.78209932)(151.08472726,401.83209927)(151.07472778,401.88210297)
\curveto(151.06472728,401.93209917)(151.05972729,401.98709911)(151.05972778,402.04710297)
\curveto(151.05972729,402.11709898)(151.06472728,402.17209893)(151.07472778,402.21210297)
\curveto(151.08472726,402.26209884)(151.08972726,402.30709879)(151.08972778,402.34710297)
\lineto(151.08972778,402.49710297)
\curveto(151.09972725,402.54709855)(151.09972725,402.59209851)(151.08972778,402.63210297)
\curveto(151.08972726,402.68209842)(151.09972725,402.73209837)(151.11972778,402.78210297)
\curveto(151.13972721,402.89209821)(151.15472719,402.9970981)(151.16472778,403.09710297)
\curveto(151.18472716,403.1970979)(151.20972714,403.2970978)(151.23972778,403.39710297)
\curveto(151.27972707,403.51709758)(151.31472703,403.63209747)(151.34472778,403.74210297)
\curveto(151.37472697,403.85209725)(151.41472693,403.96209714)(151.46472778,404.07210297)
\curveto(151.60472674,404.37209673)(151.77972657,404.65709644)(151.98972778,404.92710297)
\curveto(152.00972634,404.95709614)(152.03472631,404.98209612)(152.06472778,405.00210297)
\curveto(152.10472624,405.03209607)(152.13472621,405.06209604)(152.15472778,405.09210297)
\curveto(152.19472615,405.14209596)(152.23472611,405.18709591)(152.27472778,405.22710297)
\curveto(152.31472603,405.26709583)(152.35972599,405.30709579)(152.40972778,405.34710297)
\curveto(152.4497259,405.36709573)(152.48472586,405.39209571)(152.51472778,405.42210297)
\curveto(152.5447258,405.46209564)(152.57972577,405.49209561)(152.61972778,405.51210297)
\curveto(152.86972548,405.68209542)(153.15972519,405.82209528)(153.48972778,405.93210297)
\curveto(153.55972479,405.95209515)(153.62972472,405.96709513)(153.69972778,405.97710297)
\curveto(153.77972457,405.98709511)(153.85972449,406.0020951)(153.93972778,406.02210297)
\curveto(154.00972434,406.04209506)(154.09972425,406.05209505)(154.20972778,406.05210297)
\curveto(154.31972403,406.06209504)(154.42972392,406.06709503)(154.53972778,406.06710297)
\curveto(154.6497237,406.06709503)(154.75472359,406.06209504)(154.85472778,406.05210297)
\curveto(154.96472338,406.04209506)(155.05472329,406.02709507)(155.12472778,406.00710297)
\curveto(155.27472307,405.95709514)(155.41972293,405.91209519)(155.55972778,405.87210297)
\curveto(155.69972265,405.83209527)(155.82972252,405.77709532)(155.94972778,405.70710297)
\curveto(156.01972233,405.65709544)(156.08472226,405.60709549)(156.14472778,405.55710297)
\curveto(156.20472214,405.51709558)(156.26972208,405.47209563)(156.33972778,405.42210297)
\curveto(156.37972197,405.39209571)(156.43472191,405.35209575)(156.50472778,405.30210297)
\curveto(156.58472176,405.25209585)(156.65972169,405.25209585)(156.72972778,405.30210297)
\curveto(156.76972158,405.32209578)(156.78972156,405.35709574)(156.78972778,405.40710297)
\curveto(156.78972156,405.45709564)(156.79972155,405.50709559)(156.81972778,405.55710297)
\lineto(156.81972778,405.70710297)
\curveto(156.82972152,405.73709536)(156.83472151,405.77209533)(156.83472778,405.81210297)
\lineto(156.83472778,405.93210297)
\lineto(156.83472778,407.97210297)
\curveto(156.83472151,408.08209302)(156.82972152,408.2020929)(156.81972778,408.33210297)
\curveto(156.81972153,408.47209263)(156.8447215,408.57709252)(156.89472778,408.64710297)
\curveto(156.93472141,408.72709237)(157.00972134,408.77709232)(157.11972778,408.79710297)
\curveto(157.13972121,408.80709229)(157.15972119,408.80709229)(157.17972778,408.79710297)
\curveto(157.19972115,408.7970923)(157.21972113,408.8020923)(157.23972778,408.81210297)
\lineto(158.30472778,408.81210297)
\curveto(158.42471992,408.81209229)(158.53471981,408.80709229)(158.63472778,408.79710297)
\curveto(158.73471961,408.78709231)(158.80971954,408.74709235)(158.85972778,408.67710297)
\curveto(158.90971944,408.5970925)(158.93471941,408.49209261)(158.93472778,408.36210297)
\lineto(158.93472778,408.00210297)
\lineto(158.93472778,398.98710297)
\moveto(156.89472778,401.92710297)
\curveto(156.90472144,401.96709913)(156.90472144,402.00709909)(156.89472778,402.04710297)
\lineto(156.89472778,402.18210297)
\curveto(156.89472145,402.28209882)(156.88972146,402.38209872)(156.87972778,402.48210297)
\curveto(156.86972148,402.58209852)(156.85472149,402.67209843)(156.83472778,402.75210297)
\curveto(156.81472153,402.86209824)(156.79472155,402.96209814)(156.77472778,403.05210297)
\curveto(156.76472158,403.14209796)(156.73972161,403.22709787)(156.69972778,403.30710297)
\curveto(156.55972179,403.66709743)(156.35472199,403.95209715)(156.08472778,404.16210297)
\curveto(155.82472252,404.37209673)(155.4447229,404.47709662)(154.94472778,404.47710297)
\curveto(154.88472346,404.47709662)(154.80472354,404.46709663)(154.70472778,404.44710297)
\curveto(154.62472372,404.42709667)(154.5497238,404.40709669)(154.47972778,404.38710297)
\curveto(154.41972393,404.37709672)(154.35972399,404.35709674)(154.29972778,404.32710297)
\curveto(154.02972432,404.21709688)(153.81972453,404.04709705)(153.66972778,403.81710297)
\curveto(153.51972483,403.58709751)(153.39972495,403.32709777)(153.30972778,403.03710297)
\curveto(153.27972507,402.93709816)(153.25972509,402.83709826)(153.24972778,402.73710297)
\curveto(153.23972511,402.63709846)(153.21972513,402.53209857)(153.18972778,402.42210297)
\lineto(153.18972778,402.21210297)
\curveto(153.16972518,402.12209898)(153.16472518,401.9970991)(153.17472778,401.83710297)
\curveto(153.18472516,401.68709941)(153.19972515,401.57709952)(153.21972778,401.50710297)
\lineto(153.21972778,401.41710297)
\curveto(153.22972512,401.3970997)(153.23472511,401.37709972)(153.23472778,401.35710297)
\curveto(153.25472509,401.27709982)(153.26972508,401.2020999)(153.27972778,401.13210297)
\curveto(153.29972505,401.06210004)(153.31972503,400.98710011)(153.33972778,400.90710297)
\curveto(153.50972484,400.38710071)(153.79972455,400.0021011)(154.20972778,399.75210297)
\curveto(154.33972401,399.66210144)(154.51972383,399.59210151)(154.74972778,399.54210297)
\curveto(154.78972356,399.53210157)(154.8497235,399.52710157)(154.92972778,399.52710297)
\curveto(154.95972339,399.51710158)(155.00472334,399.50710159)(155.06472778,399.49710297)
\curveto(155.13472321,399.4971016)(155.18972316,399.5021016)(155.22972778,399.51210297)
\curveto(155.30972304,399.53210157)(155.38972296,399.54710155)(155.46972778,399.55710297)
\curveto(155.5497228,399.56710153)(155.62972272,399.58710151)(155.70972778,399.61710297)
\curveto(155.95972239,399.72710137)(156.15972219,399.86710123)(156.30972778,400.03710297)
\curveto(156.45972189,400.20710089)(156.58972176,400.42210068)(156.69972778,400.68210297)
\curveto(156.73972161,400.77210033)(156.76972158,400.86210024)(156.78972778,400.95210297)
\curveto(156.80972154,401.05210005)(156.82972152,401.15709994)(156.84972778,401.26710297)
\curveto(156.85972149,401.31709978)(156.85972149,401.36209974)(156.84972778,401.40210297)
\curveto(156.8497215,401.45209965)(156.85972149,401.5020996)(156.87972778,401.55210297)
\curveto(156.88972146,401.58209952)(156.89472145,401.61709948)(156.89472778,401.65710297)
\lineto(156.89472778,401.79210297)
\lineto(156.89472778,401.92710297)
}
}
{
\newrgbcolor{curcolor}{0 0 0}
\pscustom[linestyle=none,fillstyle=solid,fillcolor=curcolor]
{
}
}
{
\newrgbcolor{curcolor}{0 0 0}
\pscustom[linestyle=none,fillstyle=solid,fillcolor=curcolor]
{
\newpath
\moveto(172.26480591,398.98710297)
\lineto(172.26480591,398.56710297)
\curveto(172.26479754,398.43710266)(172.23479757,398.33210277)(172.17480591,398.25210297)
\curveto(172.12479768,398.2021029)(172.05979774,398.16710293)(171.97980591,398.14710297)
\curveto(171.8997979,398.13710296)(171.80979799,398.13210297)(171.70980591,398.13210297)
\lineto(170.88480591,398.13210297)
\lineto(170.59980591,398.13210297)
\curveto(170.51979928,398.14210296)(170.45479935,398.16710293)(170.40480591,398.20710297)
\curveto(170.33479947,398.25710284)(170.29479951,398.32210278)(170.28480591,398.40210297)
\curveto(170.27479953,398.48210262)(170.25479955,398.56210254)(170.22480591,398.64210297)
\curveto(170.2047996,398.66210244)(170.18479962,398.67710242)(170.16480591,398.68710297)
\curveto(170.15479965,398.70710239)(170.13979966,398.72710237)(170.11980591,398.74710297)
\curveto(170.00979979,398.74710235)(169.92979987,398.72210238)(169.87980591,398.67210297)
\lineto(169.72980591,398.52210297)
\curveto(169.65980014,398.47210263)(169.59480021,398.42710267)(169.53480591,398.38710297)
\curveto(169.47480033,398.35710274)(169.40980039,398.31710278)(169.33980591,398.26710297)
\curveto(169.2998005,398.24710285)(169.25480055,398.22710287)(169.20480591,398.20710297)
\curveto(169.16480064,398.18710291)(169.11980068,398.16710293)(169.06980591,398.14710297)
\curveto(168.92980087,398.097103)(168.77980102,398.05210305)(168.61980591,398.01210297)
\curveto(168.56980123,397.99210311)(168.52480128,397.98210312)(168.48480591,397.98210297)
\curveto(168.44480136,397.98210312)(168.4048014,397.97710312)(168.36480591,397.96710297)
\lineto(168.22980591,397.96710297)
\curveto(168.1998016,397.95710314)(168.15980164,397.95210315)(168.10980591,397.95210297)
\lineto(167.97480591,397.95210297)
\curveto(167.91480189,397.93210317)(167.82480198,397.92710317)(167.70480591,397.93710297)
\curveto(167.58480222,397.93710316)(167.4998023,397.94710315)(167.44980591,397.96710297)
\curveto(167.37980242,397.98710311)(167.31480249,397.9971031)(167.25480591,397.99710297)
\curveto(167.2048026,397.98710311)(167.14980265,397.99210311)(167.08980591,398.01210297)
\lineto(166.72980591,398.13210297)
\curveto(166.61980318,398.16210294)(166.50980329,398.2021029)(166.39980591,398.25210297)
\curveto(166.04980375,398.4021027)(165.73480407,398.63210247)(165.45480591,398.94210297)
\curveto(165.18480462,399.26210184)(164.96980483,399.5971015)(164.80980591,399.94710297)
\curveto(164.75980504,400.05710104)(164.71980508,400.16210094)(164.68980591,400.26210297)
\curveto(164.65980514,400.37210073)(164.62480518,400.48210062)(164.58480591,400.59210297)
\curveto(164.57480523,400.63210047)(164.56980523,400.66710043)(164.56980591,400.69710297)
\curveto(164.56980523,400.73710036)(164.55980524,400.78210032)(164.53980591,400.83210297)
\curveto(164.51980528,400.91210019)(164.4998053,400.9971001)(164.47980591,401.08710297)
\curveto(164.46980533,401.18709991)(164.45480535,401.28709981)(164.43480591,401.38710297)
\curveto(164.42480538,401.41709968)(164.41980538,401.45209965)(164.41980591,401.49210297)
\curveto(164.42980537,401.53209957)(164.42980537,401.56709953)(164.41980591,401.59710297)
\lineto(164.41980591,401.73210297)
\curveto(164.41980538,401.78209932)(164.41480539,401.83209927)(164.40480591,401.88210297)
\curveto(164.39480541,401.93209917)(164.38980541,401.98709911)(164.38980591,402.04710297)
\curveto(164.38980541,402.11709898)(164.39480541,402.17209893)(164.40480591,402.21210297)
\curveto(164.41480539,402.26209884)(164.41980538,402.30709879)(164.41980591,402.34710297)
\lineto(164.41980591,402.49710297)
\curveto(164.42980537,402.54709855)(164.42980537,402.59209851)(164.41980591,402.63210297)
\curveto(164.41980538,402.68209842)(164.42980537,402.73209837)(164.44980591,402.78210297)
\curveto(164.46980533,402.89209821)(164.48480532,402.9970981)(164.49480591,403.09710297)
\curveto(164.51480529,403.1970979)(164.53980526,403.2970978)(164.56980591,403.39710297)
\curveto(164.60980519,403.51709758)(164.64480516,403.63209747)(164.67480591,403.74210297)
\curveto(164.7048051,403.85209725)(164.74480506,403.96209714)(164.79480591,404.07210297)
\curveto(164.93480487,404.37209673)(165.10980469,404.65709644)(165.31980591,404.92710297)
\curveto(165.33980446,404.95709614)(165.36480444,404.98209612)(165.39480591,405.00210297)
\curveto(165.43480437,405.03209607)(165.46480434,405.06209604)(165.48480591,405.09210297)
\curveto(165.52480428,405.14209596)(165.56480424,405.18709591)(165.60480591,405.22710297)
\curveto(165.64480416,405.26709583)(165.68980411,405.30709579)(165.73980591,405.34710297)
\curveto(165.77980402,405.36709573)(165.81480399,405.39209571)(165.84480591,405.42210297)
\curveto(165.87480393,405.46209564)(165.90980389,405.49209561)(165.94980591,405.51210297)
\curveto(166.1998036,405.68209542)(166.48980331,405.82209528)(166.81980591,405.93210297)
\curveto(166.88980291,405.95209515)(166.95980284,405.96709513)(167.02980591,405.97710297)
\curveto(167.10980269,405.98709511)(167.18980261,406.0020951)(167.26980591,406.02210297)
\curveto(167.33980246,406.04209506)(167.42980237,406.05209505)(167.53980591,406.05210297)
\curveto(167.64980215,406.06209504)(167.75980204,406.06709503)(167.86980591,406.06710297)
\curveto(167.97980182,406.06709503)(168.08480172,406.06209504)(168.18480591,406.05210297)
\curveto(168.29480151,406.04209506)(168.38480142,406.02709507)(168.45480591,406.00710297)
\curveto(168.6048012,405.95709514)(168.74980105,405.91209519)(168.88980591,405.87210297)
\curveto(169.02980077,405.83209527)(169.15980064,405.77709532)(169.27980591,405.70710297)
\curveto(169.34980045,405.65709544)(169.41480039,405.60709549)(169.47480591,405.55710297)
\curveto(169.53480027,405.51709558)(169.5998002,405.47209563)(169.66980591,405.42210297)
\curveto(169.70980009,405.39209571)(169.76480004,405.35209575)(169.83480591,405.30210297)
\curveto(169.91479989,405.25209585)(169.98979981,405.25209585)(170.05980591,405.30210297)
\curveto(170.0997997,405.32209578)(170.11979968,405.35709574)(170.11980591,405.40710297)
\curveto(170.11979968,405.45709564)(170.12979967,405.50709559)(170.14980591,405.55710297)
\lineto(170.14980591,405.70710297)
\curveto(170.15979964,405.73709536)(170.16479964,405.77209533)(170.16480591,405.81210297)
\lineto(170.16480591,405.93210297)
\lineto(170.16480591,407.97210297)
\curveto(170.16479964,408.08209302)(170.15979964,408.2020929)(170.14980591,408.33210297)
\curveto(170.14979965,408.47209263)(170.17479963,408.57709252)(170.22480591,408.64710297)
\curveto(170.26479954,408.72709237)(170.33979946,408.77709232)(170.44980591,408.79710297)
\curveto(170.46979933,408.80709229)(170.48979931,408.80709229)(170.50980591,408.79710297)
\curveto(170.52979927,408.7970923)(170.54979925,408.8020923)(170.56980591,408.81210297)
\lineto(171.63480591,408.81210297)
\curveto(171.75479805,408.81209229)(171.86479794,408.80709229)(171.96480591,408.79710297)
\curveto(172.06479774,408.78709231)(172.13979766,408.74709235)(172.18980591,408.67710297)
\curveto(172.23979756,408.5970925)(172.26479754,408.49209261)(172.26480591,408.36210297)
\lineto(172.26480591,408.00210297)
\lineto(172.26480591,398.98710297)
\moveto(170.22480591,401.92710297)
\curveto(170.23479957,401.96709913)(170.23479957,402.00709909)(170.22480591,402.04710297)
\lineto(170.22480591,402.18210297)
\curveto(170.22479958,402.28209882)(170.21979958,402.38209872)(170.20980591,402.48210297)
\curveto(170.1997996,402.58209852)(170.18479962,402.67209843)(170.16480591,402.75210297)
\curveto(170.14479966,402.86209824)(170.12479968,402.96209814)(170.10480591,403.05210297)
\curveto(170.09479971,403.14209796)(170.06979973,403.22709787)(170.02980591,403.30710297)
\curveto(169.88979991,403.66709743)(169.68480012,403.95209715)(169.41480591,404.16210297)
\curveto(169.15480065,404.37209673)(168.77480103,404.47709662)(168.27480591,404.47710297)
\curveto(168.21480159,404.47709662)(168.13480167,404.46709663)(168.03480591,404.44710297)
\curveto(167.95480185,404.42709667)(167.87980192,404.40709669)(167.80980591,404.38710297)
\curveto(167.74980205,404.37709672)(167.68980211,404.35709674)(167.62980591,404.32710297)
\curveto(167.35980244,404.21709688)(167.14980265,404.04709705)(166.99980591,403.81710297)
\curveto(166.84980295,403.58709751)(166.72980307,403.32709777)(166.63980591,403.03710297)
\curveto(166.60980319,402.93709816)(166.58980321,402.83709826)(166.57980591,402.73710297)
\curveto(166.56980323,402.63709846)(166.54980325,402.53209857)(166.51980591,402.42210297)
\lineto(166.51980591,402.21210297)
\curveto(166.4998033,402.12209898)(166.49480331,401.9970991)(166.50480591,401.83710297)
\curveto(166.51480329,401.68709941)(166.52980327,401.57709952)(166.54980591,401.50710297)
\lineto(166.54980591,401.41710297)
\curveto(166.55980324,401.3970997)(166.56480324,401.37709972)(166.56480591,401.35710297)
\curveto(166.58480322,401.27709982)(166.5998032,401.2020999)(166.60980591,401.13210297)
\curveto(166.62980317,401.06210004)(166.64980315,400.98710011)(166.66980591,400.90710297)
\curveto(166.83980296,400.38710071)(167.12980267,400.0021011)(167.53980591,399.75210297)
\curveto(167.66980213,399.66210144)(167.84980195,399.59210151)(168.07980591,399.54210297)
\curveto(168.11980168,399.53210157)(168.17980162,399.52710157)(168.25980591,399.52710297)
\curveto(168.28980151,399.51710158)(168.33480147,399.50710159)(168.39480591,399.49710297)
\curveto(168.46480134,399.4971016)(168.51980128,399.5021016)(168.55980591,399.51210297)
\curveto(168.63980116,399.53210157)(168.71980108,399.54710155)(168.79980591,399.55710297)
\curveto(168.87980092,399.56710153)(168.95980084,399.58710151)(169.03980591,399.61710297)
\curveto(169.28980051,399.72710137)(169.48980031,399.86710123)(169.63980591,400.03710297)
\curveto(169.78980001,400.20710089)(169.91979988,400.42210068)(170.02980591,400.68210297)
\curveto(170.06979973,400.77210033)(170.0997997,400.86210024)(170.11980591,400.95210297)
\curveto(170.13979966,401.05210005)(170.15979964,401.15709994)(170.17980591,401.26710297)
\curveto(170.18979961,401.31709978)(170.18979961,401.36209974)(170.17980591,401.40210297)
\curveto(170.17979962,401.45209965)(170.18979961,401.5020996)(170.20980591,401.55210297)
\curveto(170.21979958,401.58209952)(170.22479958,401.61709948)(170.22480591,401.65710297)
\lineto(170.22480591,401.79210297)
\lineto(170.22480591,401.92710297)
}
}
{
\newrgbcolor{curcolor}{0 0 0}
\pscustom[linestyle=none,fillstyle=solid,fillcolor=curcolor]
{
\newpath
\moveto(181.20972778,402.07710297)
\curveto(181.22971962,401.9970991)(181.22971962,401.90709919)(181.20972778,401.80710297)
\curveto(181.18971966,401.70709939)(181.15471969,401.64209946)(181.10472778,401.61210297)
\curveto(181.05471979,401.57209953)(180.97971987,401.54209956)(180.87972778,401.52210297)
\curveto(180.78972006,401.51209959)(180.68472016,401.5020996)(180.56472778,401.49210297)
\lineto(180.21972778,401.49210297)
\curveto(180.10972074,401.5020996)(180.00972084,401.50709959)(179.91972778,401.50710297)
\lineto(176.25972778,401.50710297)
\lineto(176.04972778,401.50710297)
\curveto(175.98972486,401.50709959)(175.93472491,401.4970996)(175.88472778,401.47710297)
\curveto(175.80472504,401.43709966)(175.75472509,401.3970997)(175.73472778,401.35710297)
\curveto(175.71472513,401.33709976)(175.69472515,401.2970998)(175.67472778,401.23710297)
\curveto(175.65472519,401.18709991)(175.6497252,401.13709996)(175.65972778,401.08710297)
\curveto(175.67972517,401.02710007)(175.68972516,400.96710013)(175.68972778,400.90710297)
\curveto(175.69972515,400.85710024)(175.71472513,400.8021003)(175.73472778,400.74210297)
\curveto(175.81472503,400.5021006)(175.90972494,400.3021008)(176.01972778,400.14210297)
\curveto(176.13972471,399.99210111)(176.29972455,399.85710124)(176.49972778,399.73710297)
\curveto(176.57972427,399.68710141)(176.65972419,399.65210145)(176.73972778,399.63210297)
\curveto(176.82972402,399.62210148)(176.91972393,399.6021015)(177.00972778,399.57210297)
\curveto(177.08972376,399.55210155)(177.19972365,399.53710156)(177.33972778,399.52710297)
\curveto(177.47972337,399.51710158)(177.59972325,399.52210158)(177.69972778,399.54210297)
\lineto(177.83472778,399.54210297)
\curveto(177.93472291,399.56210154)(178.02472282,399.58210152)(178.10472778,399.60210297)
\curveto(178.19472265,399.63210147)(178.27972257,399.66210144)(178.35972778,399.69210297)
\curveto(178.45972239,399.74210136)(178.56972228,399.80710129)(178.68972778,399.88710297)
\curveto(178.81972203,399.96710113)(178.91472193,400.04710105)(178.97472778,400.12710297)
\curveto(179.02472182,400.1971009)(179.07472177,400.26210084)(179.12472778,400.32210297)
\curveto(179.18472166,400.39210071)(179.25472159,400.44210066)(179.33472778,400.47210297)
\curveto(179.43472141,400.52210058)(179.55972129,400.54210056)(179.70972778,400.53210297)
\lineto(180.14472778,400.53210297)
\lineto(180.32472778,400.53210297)
\curveto(180.39472045,400.54210056)(180.45472039,400.53710056)(180.50472778,400.51710297)
\lineto(180.65472778,400.51710297)
\curveto(180.75472009,400.4971006)(180.82472002,400.47210063)(180.86472778,400.44210297)
\curveto(180.90471994,400.42210068)(180.92471992,400.37710072)(180.92472778,400.30710297)
\curveto(180.93471991,400.23710086)(180.92971992,400.17710092)(180.90972778,400.12710297)
\curveto(180.85971999,399.98710111)(180.80472004,399.86210124)(180.74472778,399.75210297)
\curveto(180.68472016,399.64210146)(180.61472023,399.53210157)(180.53472778,399.42210297)
\curveto(180.31472053,399.09210201)(180.06472078,398.82710227)(179.78472778,398.62710297)
\curveto(179.50472134,398.42710267)(179.15472169,398.25710284)(178.73472778,398.11710297)
\curveto(178.62472222,398.07710302)(178.51472233,398.05210305)(178.40472778,398.04210297)
\curveto(178.29472255,398.03210307)(178.17972267,398.01210309)(178.05972778,397.98210297)
\curveto(178.01972283,397.97210313)(177.97472287,397.97210313)(177.92472778,397.98210297)
\curveto(177.88472296,397.98210312)(177.844723,397.97710312)(177.80472778,397.96710297)
\lineto(177.63972778,397.96710297)
\curveto(177.58972326,397.94710315)(177.52972332,397.94210316)(177.45972778,397.95210297)
\curveto(177.39972345,397.95210315)(177.3447235,397.95710314)(177.29472778,397.96710297)
\curveto(177.21472363,397.97710312)(177.1447237,397.97710312)(177.08472778,397.96710297)
\curveto(177.02472382,397.95710314)(176.95972389,397.96210314)(176.88972778,397.98210297)
\curveto(176.83972401,398.0021031)(176.78472406,398.01210309)(176.72472778,398.01210297)
\curveto(176.66472418,398.01210309)(176.60972424,398.02210308)(176.55972778,398.04210297)
\curveto(176.4497244,398.06210304)(176.33972451,398.08710301)(176.22972778,398.11710297)
\curveto(176.11972473,398.13710296)(176.01972483,398.17210293)(175.92972778,398.22210297)
\curveto(175.81972503,398.26210284)(175.71472513,398.2971028)(175.61472778,398.32710297)
\curveto(175.52472532,398.36710273)(175.43972541,398.41210269)(175.35972778,398.46210297)
\curveto(175.03972581,398.66210244)(174.75472609,398.89210221)(174.50472778,399.15210297)
\curveto(174.25472659,399.42210168)(174.0497268,399.73210137)(173.88972778,400.08210297)
\curveto(173.83972701,400.19210091)(173.79972705,400.3021008)(173.76972778,400.41210297)
\curveto(173.73972711,400.53210057)(173.69972715,400.65210045)(173.64972778,400.77210297)
\curveto(173.63972721,400.81210029)(173.63472721,400.84710025)(173.63472778,400.87710297)
\curveto(173.63472721,400.91710018)(173.62972722,400.95710014)(173.61972778,400.99710297)
\curveto(173.57972727,401.11709998)(173.55472729,401.24709985)(173.54472778,401.38710297)
\lineto(173.51472778,401.80710297)
\curveto(173.51472733,401.85709924)(173.50972734,401.91209919)(173.49972778,401.97210297)
\curveto(173.49972735,402.03209907)(173.50472734,402.08709901)(173.51472778,402.13710297)
\lineto(173.51472778,402.31710297)
\lineto(173.55972778,402.67710297)
\curveto(173.59972725,402.84709825)(173.63472721,403.01209809)(173.66472778,403.17210297)
\curveto(173.69472715,403.33209777)(173.73972711,403.48209762)(173.79972778,403.62210297)
\curveto(174.22972662,404.66209644)(174.95972589,405.3970957)(175.98972778,405.82710297)
\curveto(176.12972472,405.88709521)(176.26972458,405.92709517)(176.40972778,405.94710297)
\curveto(176.55972429,405.97709512)(176.71472413,406.01209509)(176.87472778,406.05210297)
\curveto(176.95472389,406.06209504)(177.02972382,406.06709503)(177.09972778,406.06710297)
\curveto(177.16972368,406.06709503)(177.2447236,406.07209503)(177.32472778,406.08210297)
\curveto(177.83472301,406.09209501)(178.26972258,406.03209507)(178.62972778,405.90210297)
\curveto(178.99972185,405.78209532)(179.32972152,405.62209548)(179.61972778,405.42210297)
\curveto(179.70972114,405.36209574)(179.79972105,405.29209581)(179.88972778,405.21210297)
\curveto(179.97972087,405.14209596)(180.05972079,405.06709603)(180.12972778,404.98710297)
\curveto(180.15972069,404.93709616)(180.19972065,404.8970962)(180.24972778,404.86710297)
\curveto(180.32972052,404.75709634)(180.40472044,404.64209646)(180.47472778,404.52210297)
\curveto(180.5447203,404.41209669)(180.61972023,404.2970968)(180.69972778,404.17710297)
\curveto(180.7497201,404.08709701)(180.78972006,403.99209711)(180.81972778,403.89210297)
\curveto(180.85971999,403.8020973)(180.89971995,403.7020974)(180.93972778,403.59210297)
\curveto(180.98971986,403.46209764)(181.02971982,403.32709777)(181.05972778,403.18710297)
\curveto(181.08971976,403.04709805)(181.12471972,402.90709819)(181.16472778,402.76710297)
\curveto(181.18471966,402.68709841)(181.18971966,402.5970985)(181.17972778,402.49710297)
\curveto(181.17971967,402.40709869)(181.18971966,402.32209878)(181.20972778,402.24210297)
\lineto(181.20972778,402.07710297)
\moveto(178.95972778,402.96210297)
\curveto(179.02972182,403.06209804)(179.03472181,403.18209792)(178.97472778,403.32210297)
\curveto(178.92472192,403.47209763)(178.88472196,403.58209752)(178.85472778,403.65210297)
\curveto(178.71472213,403.92209718)(178.52972232,404.12709697)(178.29972778,404.26710297)
\curveto(178.06972278,404.41709668)(177.7497231,404.4970966)(177.33972778,404.50710297)
\curveto(177.30972354,404.48709661)(177.27472357,404.48209662)(177.23472778,404.49210297)
\curveto(177.19472365,404.5020966)(177.15972369,404.5020966)(177.12972778,404.49210297)
\curveto(177.07972377,404.47209663)(177.02472382,404.45709664)(176.96472778,404.44710297)
\curveto(176.90472394,404.44709665)(176.849724,404.43709666)(176.79972778,404.41710297)
\curveto(176.35972449,404.27709682)(176.03472481,404.0020971)(175.82472778,403.59210297)
\curveto(175.80472504,403.55209755)(175.77972507,403.4970976)(175.74972778,403.42710297)
\curveto(175.72972512,403.36709773)(175.71472513,403.3020978)(175.70472778,403.23210297)
\curveto(175.69472515,403.17209793)(175.69472515,403.11209799)(175.70472778,403.05210297)
\curveto(175.72472512,402.99209811)(175.75972509,402.94209816)(175.80972778,402.90210297)
\curveto(175.88972496,402.85209825)(175.99972485,402.82709827)(176.13972778,402.82710297)
\lineto(176.54472778,402.82710297)
\lineto(178.20972778,402.82710297)
\lineto(178.64472778,402.82710297)
\curveto(178.80472204,402.83709826)(178.90972194,402.88209822)(178.95972778,402.96210297)
}
}
{
\newrgbcolor{curcolor}{0 0 0}
\pscustom[linestyle=none,fillstyle=solid,fillcolor=curcolor]
{
}
}
{
\newrgbcolor{curcolor}{0 0 0}
\pscustom[linestyle=none,fillstyle=solid,fillcolor=curcolor]
{
\newpath
\moveto(191.04316528,406.06710297)
\curveto(191.15315997,406.06709503)(191.24815987,406.05709504)(191.32816528,406.03710297)
\curveto(191.4181597,406.01709508)(191.48815963,405.97209513)(191.53816528,405.90210297)
\curveto(191.59815952,405.82209528)(191.62815949,405.68209542)(191.62816528,405.48210297)
\lineto(191.62816528,404.97210297)
\lineto(191.62816528,404.59710297)
\curveto(191.63815948,404.45709664)(191.6231595,404.34709675)(191.58316528,404.26710297)
\curveto(191.54315958,404.1970969)(191.48315964,404.15209695)(191.40316528,404.13210297)
\curveto(191.33315979,404.11209699)(191.24815987,404.102097)(191.14816528,404.10210297)
\curveto(191.05816006,404.102097)(190.95816016,404.10709699)(190.84816528,404.11710297)
\curveto(190.74816037,404.12709697)(190.65316047,404.12209698)(190.56316528,404.10210297)
\curveto(190.49316063,404.08209702)(190.4231607,404.06709703)(190.35316528,404.05710297)
\curveto(190.28316084,404.05709704)(190.2181609,404.04709705)(190.15816528,404.02710297)
\curveto(189.99816112,403.97709712)(189.83816128,403.9020972)(189.67816528,403.80210297)
\curveto(189.5181616,403.71209739)(189.39316173,403.60709749)(189.30316528,403.48710297)
\curveto(189.25316187,403.40709769)(189.19816192,403.32209778)(189.13816528,403.23210297)
\curveto(189.08816203,403.15209795)(189.03816208,403.06709803)(188.98816528,402.97710297)
\curveto(188.95816216,402.8970982)(188.92816219,402.81209829)(188.89816528,402.72210297)
\lineto(188.83816528,402.48210297)
\curveto(188.8181623,402.41209869)(188.80816231,402.33709876)(188.80816528,402.25710297)
\curveto(188.80816231,402.18709891)(188.79816232,402.11709898)(188.77816528,402.04710297)
\curveto(188.76816235,402.00709909)(188.76316236,401.96709913)(188.76316528,401.92710297)
\curveto(188.77316235,401.8970992)(188.77316235,401.86709923)(188.76316528,401.83710297)
\lineto(188.76316528,401.59710297)
\curveto(188.74316238,401.52709957)(188.73816238,401.44709965)(188.74816528,401.35710297)
\curveto(188.75816236,401.27709982)(188.76316236,401.1970999)(188.76316528,401.11710297)
\lineto(188.76316528,400.15710297)
\lineto(188.76316528,398.88210297)
\curveto(188.76316236,398.75210235)(188.75816236,398.63210247)(188.74816528,398.52210297)
\curveto(188.73816238,398.41210269)(188.70816241,398.32210278)(188.65816528,398.25210297)
\curveto(188.63816248,398.22210288)(188.60316252,398.1971029)(188.55316528,398.17710297)
\curveto(188.51316261,398.16710293)(188.46816265,398.15710294)(188.41816528,398.14710297)
\lineto(188.34316528,398.14710297)
\curveto(188.29316283,398.13710296)(188.23816288,398.13210297)(188.17816528,398.13210297)
\lineto(188.01316528,398.13210297)
\lineto(187.36816528,398.13210297)
\curveto(187.30816381,398.14210296)(187.24316388,398.14710295)(187.17316528,398.14710297)
\lineto(186.97816528,398.14710297)
\curveto(186.92816419,398.16710293)(186.87816424,398.18210292)(186.82816528,398.19210297)
\curveto(186.77816434,398.21210289)(186.74316438,398.24710285)(186.72316528,398.29710297)
\curveto(186.68316444,398.34710275)(186.65816446,398.41710268)(186.64816528,398.50710297)
\lineto(186.64816528,398.80710297)
\lineto(186.64816528,399.82710297)
\lineto(186.64816528,404.05710297)
\lineto(186.64816528,405.16710297)
\lineto(186.64816528,405.45210297)
\curveto(186.64816447,405.55209555)(186.66816445,405.63209547)(186.70816528,405.69210297)
\curveto(186.75816436,405.77209533)(186.83316429,405.82209528)(186.93316528,405.84210297)
\curveto(187.03316409,405.86209524)(187.15316397,405.87209523)(187.29316528,405.87210297)
\lineto(188.05816528,405.87210297)
\curveto(188.17816294,405.87209523)(188.28316284,405.86209524)(188.37316528,405.84210297)
\curveto(188.46316266,405.83209527)(188.53316259,405.78709531)(188.58316528,405.70710297)
\curveto(188.61316251,405.65709544)(188.62816249,405.58709551)(188.62816528,405.49710297)
\lineto(188.65816528,405.22710297)
\curveto(188.66816245,405.14709595)(188.68316244,405.07209603)(188.70316528,405.00210297)
\curveto(188.73316239,404.93209617)(188.78316234,404.8970962)(188.85316528,404.89710297)
\curveto(188.87316225,404.91709618)(188.89316223,404.92709617)(188.91316528,404.92710297)
\curveto(188.93316219,404.92709617)(188.95316217,404.93709616)(188.97316528,404.95710297)
\curveto(189.03316209,405.00709609)(189.08316204,405.06209604)(189.12316528,405.12210297)
\curveto(189.17316195,405.19209591)(189.23316189,405.25209585)(189.30316528,405.30210297)
\curveto(189.34316178,405.33209577)(189.37816174,405.36209574)(189.40816528,405.39210297)
\curveto(189.43816168,405.43209567)(189.47316165,405.46709563)(189.51316528,405.49710297)
\lineto(189.78316528,405.67710297)
\curveto(189.88316124,405.73709536)(189.98316114,405.79209531)(190.08316528,405.84210297)
\curveto(190.18316094,405.88209522)(190.28316084,405.91709518)(190.38316528,405.94710297)
\lineto(190.71316528,406.03710297)
\curveto(190.74316038,406.04709505)(190.79816032,406.04709505)(190.87816528,406.03710297)
\curveto(190.96816015,406.03709506)(191.0231601,406.04709505)(191.04316528,406.06710297)
}
}
{
\newrgbcolor{curcolor}{0 0 0}
\pscustom[linestyle=none,fillstyle=solid,fillcolor=curcolor]
{
\newpath
\moveto(199.54957153,402.07710297)
\curveto(199.56956337,401.9970991)(199.56956337,401.90709919)(199.54957153,401.80710297)
\curveto(199.52956341,401.70709939)(199.49456344,401.64209946)(199.44457153,401.61210297)
\curveto(199.39456354,401.57209953)(199.31956362,401.54209956)(199.21957153,401.52210297)
\curveto(199.12956381,401.51209959)(199.02456391,401.5020996)(198.90457153,401.49210297)
\lineto(198.55957153,401.49210297)
\curveto(198.44956449,401.5020996)(198.34956459,401.50709959)(198.25957153,401.50710297)
\lineto(194.59957153,401.50710297)
\lineto(194.38957153,401.50710297)
\curveto(194.32956861,401.50709959)(194.27456866,401.4970996)(194.22457153,401.47710297)
\curveto(194.14456879,401.43709966)(194.09456884,401.3970997)(194.07457153,401.35710297)
\curveto(194.05456888,401.33709976)(194.0345689,401.2970998)(194.01457153,401.23710297)
\curveto(193.99456894,401.18709991)(193.98956895,401.13709996)(193.99957153,401.08710297)
\curveto(194.01956892,401.02710007)(194.02956891,400.96710013)(194.02957153,400.90710297)
\curveto(194.0395689,400.85710024)(194.05456888,400.8021003)(194.07457153,400.74210297)
\curveto(194.15456878,400.5021006)(194.24956869,400.3021008)(194.35957153,400.14210297)
\curveto(194.47956846,399.99210111)(194.6395683,399.85710124)(194.83957153,399.73710297)
\curveto(194.91956802,399.68710141)(194.99956794,399.65210145)(195.07957153,399.63210297)
\curveto(195.16956777,399.62210148)(195.25956768,399.6021015)(195.34957153,399.57210297)
\curveto(195.42956751,399.55210155)(195.5395674,399.53710156)(195.67957153,399.52710297)
\curveto(195.81956712,399.51710158)(195.939567,399.52210158)(196.03957153,399.54210297)
\lineto(196.17457153,399.54210297)
\curveto(196.27456666,399.56210154)(196.36456657,399.58210152)(196.44457153,399.60210297)
\curveto(196.5345664,399.63210147)(196.61956632,399.66210144)(196.69957153,399.69210297)
\curveto(196.79956614,399.74210136)(196.90956603,399.80710129)(197.02957153,399.88710297)
\curveto(197.15956578,399.96710113)(197.25456568,400.04710105)(197.31457153,400.12710297)
\curveto(197.36456557,400.1971009)(197.41456552,400.26210084)(197.46457153,400.32210297)
\curveto(197.52456541,400.39210071)(197.59456534,400.44210066)(197.67457153,400.47210297)
\curveto(197.77456516,400.52210058)(197.89956504,400.54210056)(198.04957153,400.53210297)
\lineto(198.48457153,400.53210297)
\lineto(198.66457153,400.53210297)
\curveto(198.7345642,400.54210056)(198.79456414,400.53710056)(198.84457153,400.51710297)
\lineto(198.99457153,400.51710297)
\curveto(199.09456384,400.4971006)(199.16456377,400.47210063)(199.20457153,400.44210297)
\curveto(199.24456369,400.42210068)(199.26456367,400.37710072)(199.26457153,400.30710297)
\curveto(199.27456366,400.23710086)(199.26956367,400.17710092)(199.24957153,400.12710297)
\curveto(199.19956374,399.98710111)(199.14456379,399.86210124)(199.08457153,399.75210297)
\curveto(199.02456391,399.64210146)(198.95456398,399.53210157)(198.87457153,399.42210297)
\curveto(198.65456428,399.09210201)(198.40456453,398.82710227)(198.12457153,398.62710297)
\curveto(197.84456509,398.42710267)(197.49456544,398.25710284)(197.07457153,398.11710297)
\curveto(196.96456597,398.07710302)(196.85456608,398.05210305)(196.74457153,398.04210297)
\curveto(196.6345663,398.03210307)(196.51956642,398.01210309)(196.39957153,397.98210297)
\curveto(196.35956658,397.97210313)(196.31456662,397.97210313)(196.26457153,397.98210297)
\curveto(196.22456671,397.98210312)(196.18456675,397.97710312)(196.14457153,397.96710297)
\lineto(195.97957153,397.96710297)
\curveto(195.92956701,397.94710315)(195.86956707,397.94210316)(195.79957153,397.95210297)
\curveto(195.7395672,397.95210315)(195.68456725,397.95710314)(195.63457153,397.96710297)
\curveto(195.55456738,397.97710312)(195.48456745,397.97710312)(195.42457153,397.96710297)
\curveto(195.36456757,397.95710314)(195.29956764,397.96210314)(195.22957153,397.98210297)
\curveto(195.17956776,398.0021031)(195.12456781,398.01210309)(195.06457153,398.01210297)
\curveto(195.00456793,398.01210309)(194.94956799,398.02210308)(194.89957153,398.04210297)
\curveto(194.78956815,398.06210304)(194.67956826,398.08710301)(194.56957153,398.11710297)
\curveto(194.45956848,398.13710296)(194.35956858,398.17210293)(194.26957153,398.22210297)
\curveto(194.15956878,398.26210284)(194.05456888,398.2971028)(193.95457153,398.32710297)
\curveto(193.86456907,398.36710273)(193.77956916,398.41210269)(193.69957153,398.46210297)
\curveto(193.37956956,398.66210244)(193.09456984,398.89210221)(192.84457153,399.15210297)
\curveto(192.59457034,399.42210168)(192.38957055,399.73210137)(192.22957153,400.08210297)
\curveto(192.17957076,400.19210091)(192.1395708,400.3021008)(192.10957153,400.41210297)
\curveto(192.07957086,400.53210057)(192.0395709,400.65210045)(191.98957153,400.77210297)
\curveto(191.97957096,400.81210029)(191.97457096,400.84710025)(191.97457153,400.87710297)
\curveto(191.97457096,400.91710018)(191.96957097,400.95710014)(191.95957153,400.99710297)
\curveto(191.91957102,401.11709998)(191.89457104,401.24709985)(191.88457153,401.38710297)
\lineto(191.85457153,401.80710297)
\curveto(191.85457108,401.85709924)(191.84957109,401.91209919)(191.83957153,401.97210297)
\curveto(191.8395711,402.03209907)(191.84457109,402.08709901)(191.85457153,402.13710297)
\lineto(191.85457153,402.31710297)
\lineto(191.89957153,402.67710297)
\curveto(191.939571,402.84709825)(191.97457096,403.01209809)(192.00457153,403.17210297)
\curveto(192.0345709,403.33209777)(192.07957086,403.48209762)(192.13957153,403.62210297)
\curveto(192.56957037,404.66209644)(193.29956964,405.3970957)(194.32957153,405.82710297)
\curveto(194.46956847,405.88709521)(194.60956833,405.92709517)(194.74957153,405.94710297)
\curveto(194.89956804,405.97709512)(195.05456788,406.01209509)(195.21457153,406.05210297)
\curveto(195.29456764,406.06209504)(195.36956757,406.06709503)(195.43957153,406.06710297)
\curveto(195.50956743,406.06709503)(195.58456735,406.07209503)(195.66457153,406.08210297)
\curveto(196.17456676,406.09209501)(196.60956633,406.03209507)(196.96957153,405.90210297)
\curveto(197.3395656,405.78209532)(197.66956527,405.62209548)(197.95957153,405.42210297)
\curveto(198.04956489,405.36209574)(198.1395648,405.29209581)(198.22957153,405.21210297)
\curveto(198.31956462,405.14209596)(198.39956454,405.06709603)(198.46957153,404.98710297)
\curveto(198.49956444,404.93709616)(198.5395644,404.8970962)(198.58957153,404.86710297)
\curveto(198.66956427,404.75709634)(198.74456419,404.64209646)(198.81457153,404.52210297)
\curveto(198.88456405,404.41209669)(198.95956398,404.2970968)(199.03957153,404.17710297)
\curveto(199.08956385,404.08709701)(199.12956381,403.99209711)(199.15957153,403.89210297)
\curveto(199.19956374,403.8020973)(199.2395637,403.7020974)(199.27957153,403.59210297)
\curveto(199.32956361,403.46209764)(199.36956357,403.32709777)(199.39957153,403.18710297)
\curveto(199.42956351,403.04709805)(199.46456347,402.90709819)(199.50457153,402.76710297)
\curveto(199.52456341,402.68709841)(199.52956341,402.5970985)(199.51957153,402.49710297)
\curveto(199.51956342,402.40709869)(199.52956341,402.32209878)(199.54957153,402.24210297)
\lineto(199.54957153,402.07710297)
\moveto(197.29957153,402.96210297)
\curveto(197.36956557,403.06209804)(197.37456556,403.18209792)(197.31457153,403.32210297)
\curveto(197.26456567,403.47209763)(197.22456571,403.58209752)(197.19457153,403.65210297)
\curveto(197.05456588,403.92209718)(196.86956607,404.12709697)(196.63957153,404.26710297)
\curveto(196.40956653,404.41709668)(196.08956685,404.4970966)(195.67957153,404.50710297)
\curveto(195.64956729,404.48709661)(195.61456732,404.48209662)(195.57457153,404.49210297)
\curveto(195.5345674,404.5020966)(195.49956744,404.5020966)(195.46957153,404.49210297)
\curveto(195.41956752,404.47209663)(195.36456757,404.45709664)(195.30457153,404.44710297)
\curveto(195.24456769,404.44709665)(195.18956775,404.43709666)(195.13957153,404.41710297)
\curveto(194.69956824,404.27709682)(194.37456856,404.0020971)(194.16457153,403.59210297)
\curveto(194.14456879,403.55209755)(194.11956882,403.4970976)(194.08957153,403.42710297)
\curveto(194.06956887,403.36709773)(194.05456888,403.3020978)(194.04457153,403.23210297)
\curveto(194.0345689,403.17209793)(194.0345689,403.11209799)(194.04457153,403.05210297)
\curveto(194.06456887,402.99209811)(194.09956884,402.94209816)(194.14957153,402.90210297)
\curveto(194.22956871,402.85209825)(194.3395686,402.82709827)(194.47957153,402.82710297)
\lineto(194.88457153,402.82710297)
\lineto(196.54957153,402.82710297)
\lineto(196.98457153,402.82710297)
\curveto(197.14456579,402.83709826)(197.24956569,402.88209822)(197.29957153,402.96210297)
}
}
{
\newrgbcolor{curcolor}{0 0 0}
\pscustom[linestyle=none,fillstyle=solid,fillcolor=curcolor]
{
\newpath
\moveto(204.36785278,406.08210297)
\curveto(205.17784762,406.102095)(205.85284695,405.98209512)(206.39285278,405.72210297)
\curveto(206.94284586,405.46209564)(207.37784542,405.09209601)(207.69785278,404.61210297)
\curveto(207.85784494,404.37209673)(207.97784482,404.097097)(208.05785278,403.78710297)
\curveto(208.07784472,403.73709736)(208.09284471,403.67209743)(208.10285278,403.59210297)
\curveto(208.12284468,403.51209759)(208.12284468,403.44209766)(208.10285278,403.38210297)
\curveto(208.06284474,403.27209783)(207.99284481,403.20709789)(207.89285278,403.18710297)
\curveto(207.79284501,403.17709792)(207.67284513,403.17209793)(207.53285278,403.17210297)
\lineto(206.75285278,403.17210297)
\lineto(206.46785278,403.17210297)
\curveto(206.37784642,403.17209793)(206.3028465,403.19209791)(206.24285278,403.23210297)
\curveto(206.16284664,403.27209783)(206.10784669,403.33209777)(206.07785278,403.41210297)
\curveto(206.04784675,403.5020976)(206.00784679,403.59209751)(205.95785278,403.68210297)
\curveto(205.8978469,403.79209731)(205.83284697,403.89209721)(205.76285278,403.98210297)
\curveto(205.69284711,404.07209703)(205.61284719,404.15209695)(205.52285278,404.22210297)
\curveto(205.38284742,404.31209679)(205.22784757,404.38209672)(205.05785278,404.43210297)
\curveto(204.9978478,404.45209665)(204.93784786,404.46209664)(204.87785278,404.46210297)
\curveto(204.81784798,404.46209664)(204.76284804,404.47209663)(204.71285278,404.49210297)
\lineto(204.56285278,404.49210297)
\curveto(204.36284844,404.49209661)(204.2028486,404.47209663)(204.08285278,404.43210297)
\curveto(203.79284901,404.34209676)(203.55784924,404.2020969)(203.37785278,404.01210297)
\curveto(203.1978496,403.83209727)(203.05284975,403.61209749)(202.94285278,403.35210297)
\curveto(202.89284991,403.24209786)(202.85284995,403.12209798)(202.82285278,402.99210297)
\curveto(202.80285,402.87209823)(202.77785002,402.74209836)(202.74785278,402.60210297)
\curveto(202.73785006,402.56209854)(202.73285007,402.52209858)(202.73285278,402.48210297)
\curveto(202.73285007,402.44209866)(202.72785007,402.4020987)(202.71785278,402.36210297)
\curveto(202.6978501,402.26209884)(202.68785011,402.12209898)(202.68785278,401.94210297)
\curveto(202.6978501,401.76209934)(202.71285009,401.62209948)(202.73285278,401.52210297)
\curveto(202.73285007,401.44209966)(202.73785006,401.38709971)(202.74785278,401.35710297)
\curveto(202.76785003,401.28709981)(202.77785002,401.21709988)(202.77785278,401.14710297)
\curveto(202.78785001,401.07710002)(202.80285,401.00710009)(202.82285278,400.93710297)
\curveto(202.9028499,400.70710039)(202.9978498,400.4971006)(203.10785278,400.30710297)
\curveto(203.21784958,400.11710098)(203.35784944,399.95710114)(203.52785278,399.82710297)
\curveto(203.56784923,399.7971013)(203.62784917,399.76210134)(203.70785278,399.72210297)
\curveto(203.81784898,399.65210145)(203.92784887,399.60710149)(204.03785278,399.58710297)
\curveto(204.15784864,399.56710153)(204.3028485,399.54710155)(204.47285278,399.52710297)
\lineto(204.56285278,399.52710297)
\curveto(204.6028482,399.52710157)(204.63284817,399.53210157)(204.65285278,399.54210297)
\lineto(204.78785278,399.54210297)
\curveto(204.85784794,399.56210154)(204.92284788,399.57710152)(204.98285278,399.58710297)
\curveto(205.05284775,399.60710149)(205.11784768,399.62710147)(205.17785278,399.64710297)
\curveto(205.47784732,399.77710132)(205.70784709,399.96710113)(205.86785278,400.21710297)
\curveto(205.90784689,400.26710083)(205.94284686,400.32210078)(205.97285278,400.38210297)
\curveto(206.0028468,400.45210065)(206.02784677,400.51210059)(206.04785278,400.56210297)
\curveto(206.08784671,400.67210043)(206.12284668,400.76710033)(206.15285278,400.84710297)
\curveto(206.18284662,400.93710016)(206.25284655,401.00710009)(206.36285278,401.05710297)
\curveto(206.45284635,401.0971)(206.5978462,401.11209999)(206.79785278,401.10210297)
\lineto(207.29285278,401.10210297)
\lineto(207.50285278,401.10210297)
\curveto(207.58284522,401.11209999)(207.64784515,401.10709999)(207.69785278,401.08710297)
\lineto(207.81785278,401.08710297)
\lineto(207.93785278,401.05710297)
\curveto(207.97784482,401.05710004)(208.00784479,401.04710005)(208.02785278,401.02710297)
\curveto(208.07784472,400.98710011)(208.10784469,400.92710017)(208.11785278,400.84710297)
\curveto(208.13784466,400.77710032)(208.13784466,400.7021004)(208.11785278,400.62210297)
\curveto(208.02784477,400.29210081)(207.91784488,399.9971011)(207.78785278,399.73710297)
\curveto(207.37784542,398.96710213)(206.72284608,398.43210267)(205.82285278,398.13210297)
\curveto(205.72284708,398.102103)(205.61784718,398.08210302)(205.50785278,398.07210297)
\curveto(205.3978474,398.05210305)(205.28784751,398.02710307)(205.17785278,397.99710297)
\curveto(205.11784768,397.98710311)(205.05784774,397.98210312)(204.99785278,397.98210297)
\curveto(204.93784786,397.98210312)(204.87784792,397.97710312)(204.81785278,397.96710297)
\lineto(204.65285278,397.96710297)
\curveto(204.6028482,397.94710315)(204.52784827,397.94210316)(204.42785278,397.95210297)
\curveto(204.32784847,397.95210315)(204.25284855,397.95710314)(204.20285278,397.96710297)
\curveto(204.12284868,397.98710311)(204.04784875,397.9971031)(203.97785278,397.99710297)
\curveto(203.91784888,397.98710311)(203.85284895,397.99210311)(203.78285278,398.01210297)
\lineto(203.63285278,398.04210297)
\curveto(203.58284922,398.04210306)(203.53284927,398.04710305)(203.48285278,398.05710297)
\curveto(203.37284943,398.08710301)(203.26784953,398.11710298)(203.16785278,398.14710297)
\curveto(203.06784973,398.17710292)(202.97284983,398.21210289)(202.88285278,398.25210297)
\curveto(202.41285039,398.45210265)(202.01785078,398.70710239)(201.69785278,399.01710297)
\curveto(201.37785142,399.33710176)(201.11785168,399.73210137)(200.91785278,400.20210297)
\curveto(200.86785193,400.29210081)(200.82785197,400.38710071)(200.79785278,400.48710297)
\lineto(200.70785278,400.81710297)
\curveto(200.6978521,400.85710024)(200.69285211,400.89210021)(200.69285278,400.92210297)
\curveto(200.69285211,400.96210014)(200.68285212,401.00710009)(200.66285278,401.05710297)
\curveto(200.64285216,401.12709997)(200.63285217,401.1970999)(200.63285278,401.26710297)
\curveto(200.63285217,401.34709975)(200.62285218,401.42209968)(200.60285278,401.49210297)
\lineto(200.60285278,401.74710297)
\curveto(200.58285222,401.7970993)(200.57285223,401.85209925)(200.57285278,401.91210297)
\curveto(200.57285223,401.98209912)(200.58285222,402.04209906)(200.60285278,402.09210297)
\curveto(200.61285219,402.14209896)(200.61285219,402.18709891)(200.60285278,402.22710297)
\curveto(200.59285221,402.26709883)(200.59285221,402.30709879)(200.60285278,402.34710297)
\curveto(200.62285218,402.41709868)(200.62785217,402.48209862)(200.61785278,402.54210297)
\curveto(200.61785218,402.6020985)(200.62785217,402.66209844)(200.64785278,402.72210297)
\curveto(200.6978521,402.9020982)(200.73785206,403.07209803)(200.76785278,403.23210297)
\curveto(200.797852,403.4020977)(200.84285196,403.56709753)(200.90285278,403.72710297)
\curveto(201.12285168,404.23709686)(201.3978514,404.66209644)(201.72785278,405.00210297)
\curveto(202.06785073,405.34209576)(202.4978503,405.61709548)(203.01785278,405.82710297)
\curveto(203.15784964,405.88709521)(203.3028495,405.92709517)(203.45285278,405.94710297)
\curveto(203.6028492,405.97709512)(203.75784904,406.01209509)(203.91785278,406.05210297)
\curveto(203.9978488,406.06209504)(204.07284873,406.06709503)(204.14285278,406.06710297)
\curveto(204.21284859,406.06709503)(204.28784851,406.07209503)(204.36785278,406.08210297)
}
}
{
\newrgbcolor{curcolor}{0 0 0}
\pscustom[linestyle=none,fillstyle=solid,fillcolor=curcolor]
{
\newpath
\moveto(209.83113403,405.85710297)
\lineto(210.95613403,405.85710297)
\curveto(211.0661316,405.85709524)(211.1661315,405.85209525)(211.25613403,405.84210297)
\curveto(211.34613132,405.83209527)(211.41113125,405.7970953)(211.45113403,405.73710297)
\curveto(211.50113116,405.67709542)(211.53113113,405.59209551)(211.54113403,405.48210297)
\curveto(211.55113111,405.38209572)(211.55613111,405.27709582)(211.55613403,405.16710297)
\lineto(211.55613403,404.11710297)
\lineto(211.55613403,401.88210297)
\curveto(211.55613111,401.52209958)(211.57113109,401.18209992)(211.60113403,400.86210297)
\curveto(211.63113103,400.54210056)(211.72113094,400.27710082)(211.87113403,400.06710297)
\curveto(212.01113065,399.85710124)(212.23613043,399.70710139)(212.54613403,399.61710297)
\curveto(212.59613007,399.60710149)(212.63613003,399.6021015)(212.66613403,399.60210297)
\curveto(212.70612996,399.6021015)(212.75112991,399.5971015)(212.80113403,399.58710297)
\curveto(212.85112981,399.57710152)(212.90612976,399.57210153)(212.96613403,399.57210297)
\curveto(213.02612964,399.57210153)(213.07112959,399.57710152)(213.10113403,399.58710297)
\curveto(213.15112951,399.60710149)(213.19112947,399.61210149)(213.22113403,399.60210297)
\curveto(213.2611294,399.59210151)(213.30112936,399.5971015)(213.34113403,399.61710297)
\curveto(213.55112911,399.66710143)(213.71612895,399.73210137)(213.83613403,399.81210297)
\curveto(214.01612865,399.92210118)(214.15612851,400.06210104)(214.25613403,400.23210297)
\curveto(214.3661283,400.41210069)(214.44112822,400.60710049)(214.48113403,400.81710297)
\curveto(214.53112813,401.03710006)(214.5611281,401.27709982)(214.57113403,401.53710297)
\curveto(214.58112808,401.80709929)(214.58612808,402.08709901)(214.58613403,402.37710297)
\lineto(214.58613403,404.19210297)
\lineto(214.58613403,405.16710297)
\lineto(214.58613403,405.43710297)
\curveto(214.58612808,405.53709556)(214.60612806,405.61709548)(214.64613403,405.67710297)
\curveto(214.69612797,405.76709533)(214.77112789,405.81709528)(214.87113403,405.82710297)
\curveto(214.97112769,405.84709525)(215.09112757,405.85709524)(215.23113403,405.85710297)
\lineto(216.02613403,405.85710297)
\lineto(216.31113403,405.85710297)
\curveto(216.40112626,405.85709524)(216.47612619,405.83709526)(216.53613403,405.79710297)
\curveto(216.61612605,405.74709535)(216.661126,405.67209543)(216.67113403,405.57210297)
\curveto(216.68112598,405.47209563)(216.68612598,405.35709574)(216.68613403,405.22710297)
\lineto(216.68613403,404.08710297)
\lineto(216.68613403,399.87210297)
\lineto(216.68613403,398.80710297)
\lineto(216.68613403,398.50710297)
\curveto(216.68612598,398.40710269)(216.666126,398.33210277)(216.62613403,398.28210297)
\curveto(216.57612609,398.2021029)(216.50112616,398.15710294)(216.40113403,398.14710297)
\curveto(216.30112636,398.13710296)(216.19612647,398.13210297)(216.08613403,398.13210297)
\lineto(215.27613403,398.13210297)
\curveto(215.1661275,398.13210297)(215.0661276,398.13710296)(214.97613403,398.14710297)
\curveto(214.89612777,398.15710294)(214.83112783,398.1971029)(214.78113403,398.26710297)
\curveto(214.7611279,398.2971028)(214.74112792,398.34210276)(214.72113403,398.40210297)
\curveto(214.71112795,398.46210264)(214.69612797,398.52210258)(214.67613403,398.58210297)
\curveto(214.666128,398.64210246)(214.65112801,398.6971024)(214.63113403,398.74710297)
\curveto(214.61112805,398.7971023)(214.58112808,398.82710227)(214.54113403,398.83710297)
\curveto(214.52112814,398.85710224)(214.49612817,398.86210224)(214.46613403,398.85210297)
\curveto(214.43612823,398.84210226)(214.41112825,398.83210227)(214.39113403,398.82210297)
\curveto(214.32112834,398.78210232)(214.2611284,398.73710236)(214.21113403,398.68710297)
\curveto(214.1611285,398.63710246)(214.10612856,398.59210251)(214.04613403,398.55210297)
\curveto(214.00612866,398.52210258)(213.9661287,398.48710261)(213.92613403,398.44710297)
\curveto(213.89612877,398.41710268)(213.85612881,398.38710271)(213.80613403,398.35710297)
\curveto(213.57612909,398.21710288)(213.30612936,398.10710299)(212.99613403,398.02710297)
\curveto(212.92612974,398.00710309)(212.85612981,397.9971031)(212.78613403,397.99710297)
\curveto(212.71612995,397.98710311)(212.64113002,397.97210313)(212.56113403,397.95210297)
\curveto(212.52113014,397.94210316)(212.47613019,397.94210316)(212.42613403,397.95210297)
\curveto(212.38613028,397.95210315)(212.34613032,397.94710315)(212.30613403,397.93710297)
\curveto(212.27613039,397.92710317)(212.21113045,397.92710317)(212.11113403,397.93710297)
\curveto(212.02113064,397.93710316)(211.9611307,397.94210316)(211.93113403,397.95210297)
\curveto(211.88113078,397.95210315)(211.83113083,397.95710314)(211.78113403,397.96710297)
\lineto(211.63113403,397.96710297)
\curveto(211.51113115,397.9971031)(211.39613127,398.02210308)(211.28613403,398.04210297)
\curveto(211.17613149,398.06210304)(211.0661316,398.09210301)(210.95613403,398.13210297)
\curveto(210.90613176,398.15210295)(210.8611318,398.16710293)(210.82113403,398.17710297)
\curveto(210.79113187,398.1971029)(210.75113191,398.21710288)(210.70113403,398.23710297)
\curveto(210.35113231,398.42710267)(210.07113259,398.69210241)(209.86113403,399.03210297)
\curveto(209.73113293,399.24210186)(209.63613303,399.49210161)(209.57613403,399.78210297)
\curveto(209.51613315,400.08210102)(209.47613319,400.3971007)(209.45613403,400.72710297)
\curveto(209.44613322,401.06710003)(209.44113322,401.41209969)(209.44113403,401.76210297)
\curveto(209.45113321,402.12209898)(209.45613321,402.47709862)(209.45613403,402.82710297)
\lineto(209.45613403,404.86710297)
\curveto(209.45613321,404.9970961)(209.45113321,405.14709595)(209.44113403,405.31710297)
\curveto(209.44113322,405.4970956)(209.4661332,405.62709547)(209.51613403,405.70710297)
\curveto(209.54613312,405.75709534)(209.60613306,405.8020953)(209.69613403,405.84210297)
\curveto(209.75613291,405.84209526)(209.80113286,405.84709525)(209.83113403,405.85710297)
}
}
{
\newrgbcolor{curcolor}{0 0 0}
\pscustom[linestyle=none,fillstyle=solid,fillcolor=curcolor]
{
\newpath
\moveto(222.74238403,406.06710297)
\curveto(222.85237872,406.06709503)(222.94737862,406.05709504)(223.02738403,406.03710297)
\curveto(223.11737845,406.01709508)(223.18737838,405.97209513)(223.23738403,405.90210297)
\curveto(223.29737827,405.82209528)(223.32737824,405.68209542)(223.32738403,405.48210297)
\lineto(223.32738403,404.97210297)
\lineto(223.32738403,404.59710297)
\curveto(223.33737823,404.45709664)(223.32237825,404.34709675)(223.28238403,404.26710297)
\curveto(223.24237833,404.1970969)(223.18237839,404.15209695)(223.10238403,404.13210297)
\curveto(223.03237854,404.11209699)(222.94737862,404.102097)(222.84738403,404.10210297)
\curveto(222.75737881,404.102097)(222.65737891,404.10709699)(222.54738403,404.11710297)
\curveto(222.44737912,404.12709697)(222.35237922,404.12209698)(222.26238403,404.10210297)
\curveto(222.19237938,404.08209702)(222.12237945,404.06709703)(222.05238403,404.05710297)
\curveto(221.98237959,404.05709704)(221.91737965,404.04709705)(221.85738403,404.02710297)
\curveto(221.69737987,403.97709712)(221.53738003,403.9020972)(221.37738403,403.80210297)
\curveto(221.21738035,403.71209739)(221.09238048,403.60709749)(221.00238403,403.48710297)
\curveto(220.95238062,403.40709769)(220.89738067,403.32209778)(220.83738403,403.23210297)
\curveto(220.78738078,403.15209795)(220.73738083,403.06709803)(220.68738403,402.97710297)
\curveto(220.65738091,402.8970982)(220.62738094,402.81209829)(220.59738403,402.72210297)
\lineto(220.53738403,402.48210297)
\curveto(220.51738105,402.41209869)(220.50738106,402.33709876)(220.50738403,402.25710297)
\curveto(220.50738106,402.18709891)(220.49738107,402.11709898)(220.47738403,402.04710297)
\curveto(220.4673811,402.00709909)(220.46238111,401.96709913)(220.46238403,401.92710297)
\curveto(220.4723811,401.8970992)(220.4723811,401.86709923)(220.46238403,401.83710297)
\lineto(220.46238403,401.59710297)
\curveto(220.44238113,401.52709957)(220.43738113,401.44709965)(220.44738403,401.35710297)
\curveto(220.45738111,401.27709982)(220.46238111,401.1970999)(220.46238403,401.11710297)
\lineto(220.46238403,400.15710297)
\lineto(220.46238403,398.88210297)
\curveto(220.46238111,398.75210235)(220.45738111,398.63210247)(220.44738403,398.52210297)
\curveto(220.43738113,398.41210269)(220.40738116,398.32210278)(220.35738403,398.25210297)
\curveto(220.33738123,398.22210288)(220.30238127,398.1971029)(220.25238403,398.17710297)
\curveto(220.21238136,398.16710293)(220.1673814,398.15710294)(220.11738403,398.14710297)
\lineto(220.04238403,398.14710297)
\curveto(219.99238158,398.13710296)(219.93738163,398.13210297)(219.87738403,398.13210297)
\lineto(219.71238403,398.13210297)
\lineto(219.06738403,398.13210297)
\curveto(219.00738256,398.14210296)(218.94238263,398.14710295)(218.87238403,398.14710297)
\lineto(218.67738403,398.14710297)
\curveto(218.62738294,398.16710293)(218.57738299,398.18210292)(218.52738403,398.19210297)
\curveto(218.47738309,398.21210289)(218.44238313,398.24710285)(218.42238403,398.29710297)
\curveto(218.38238319,398.34710275)(218.35738321,398.41710268)(218.34738403,398.50710297)
\lineto(218.34738403,398.80710297)
\lineto(218.34738403,399.82710297)
\lineto(218.34738403,404.05710297)
\lineto(218.34738403,405.16710297)
\lineto(218.34738403,405.45210297)
\curveto(218.34738322,405.55209555)(218.3673832,405.63209547)(218.40738403,405.69210297)
\curveto(218.45738311,405.77209533)(218.53238304,405.82209528)(218.63238403,405.84210297)
\curveto(218.73238284,405.86209524)(218.85238272,405.87209523)(218.99238403,405.87210297)
\lineto(219.75738403,405.87210297)
\curveto(219.87738169,405.87209523)(219.98238159,405.86209524)(220.07238403,405.84210297)
\curveto(220.16238141,405.83209527)(220.23238134,405.78709531)(220.28238403,405.70710297)
\curveto(220.31238126,405.65709544)(220.32738124,405.58709551)(220.32738403,405.49710297)
\lineto(220.35738403,405.22710297)
\curveto(220.3673812,405.14709595)(220.38238119,405.07209603)(220.40238403,405.00210297)
\curveto(220.43238114,404.93209617)(220.48238109,404.8970962)(220.55238403,404.89710297)
\curveto(220.572381,404.91709618)(220.59238098,404.92709617)(220.61238403,404.92710297)
\curveto(220.63238094,404.92709617)(220.65238092,404.93709616)(220.67238403,404.95710297)
\curveto(220.73238084,405.00709609)(220.78238079,405.06209604)(220.82238403,405.12210297)
\curveto(220.8723807,405.19209591)(220.93238064,405.25209585)(221.00238403,405.30210297)
\curveto(221.04238053,405.33209577)(221.07738049,405.36209574)(221.10738403,405.39210297)
\curveto(221.13738043,405.43209567)(221.1723804,405.46709563)(221.21238403,405.49710297)
\lineto(221.48238403,405.67710297)
\curveto(221.58237999,405.73709536)(221.68237989,405.79209531)(221.78238403,405.84210297)
\curveto(221.88237969,405.88209522)(221.98237959,405.91709518)(222.08238403,405.94710297)
\lineto(222.41238403,406.03710297)
\curveto(222.44237913,406.04709505)(222.49737907,406.04709505)(222.57738403,406.03710297)
\curveto(222.6673789,406.03709506)(222.72237885,406.04709505)(222.74238403,406.06710297)
}
}
{
\newrgbcolor{curcolor}{0 0 0}
\pscustom[linestyle=none,fillstyle=solid,fillcolor=curcolor]
{
\newpath
\moveto(227.11746216,406.08210297)
\curveto(227.86745766,406.102095)(228.51745701,406.01709508)(229.06746216,405.82710297)
\curveto(229.6274559,405.64709545)(230.05245547,405.33209577)(230.34246216,404.88210297)
\curveto(230.41245511,404.77209633)(230.47245505,404.65709644)(230.52246216,404.53710297)
\curveto(230.58245494,404.42709667)(230.63245489,404.3020968)(230.67246216,404.16210297)
\curveto(230.69245483,404.102097)(230.70245482,404.03709706)(230.70246216,403.96710297)
\curveto(230.70245482,403.8970972)(230.69245483,403.83709726)(230.67246216,403.78710297)
\curveto(230.63245489,403.72709737)(230.57745495,403.68709741)(230.50746216,403.66710297)
\curveto(230.45745507,403.64709745)(230.39745513,403.63709746)(230.32746216,403.63710297)
\lineto(230.11746216,403.63710297)
\lineto(229.45746216,403.63710297)
\curveto(229.38745614,403.63709746)(229.31745621,403.63209747)(229.24746216,403.62210297)
\curveto(229.17745635,403.62209748)(229.11245641,403.63209747)(229.05246216,403.65210297)
\curveto(228.95245657,403.67209743)(228.87745665,403.71209739)(228.82746216,403.77210297)
\curveto(228.77745675,403.83209727)(228.73245679,403.89209721)(228.69246216,403.95210297)
\lineto(228.57246216,404.16210297)
\curveto(228.54245698,404.24209686)(228.49245703,404.30709679)(228.42246216,404.35710297)
\curveto(228.3224572,404.43709666)(228.2224573,404.4970966)(228.12246216,404.53710297)
\curveto(228.03245749,404.57709652)(227.91745761,404.61209649)(227.77746216,404.64210297)
\curveto(227.70745782,404.66209644)(227.60245792,404.67709642)(227.46246216,404.68710297)
\curveto(227.33245819,404.6970964)(227.23245829,404.69209641)(227.16246216,404.67210297)
\lineto(227.05746216,404.67210297)
\lineto(226.90746216,404.64210297)
\curveto(226.86745866,404.64209646)(226.8224587,404.63709646)(226.77246216,404.62710297)
\curveto(226.60245892,404.57709652)(226.46245906,404.50709659)(226.35246216,404.41710297)
\curveto(226.25245927,404.33709676)(226.18245934,404.21209689)(226.14246216,404.04210297)
\curveto(226.1224594,403.97209713)(226.1224594,403.90709719)(226.14246216,403.84710297)
\curveto(226.16245936,403.78709731)(226.18245934,403.73709736)(226.20246216,403.69710297)
\curveto(226.27245925,403.57709752)(226.35245917,403.48209762)(226.44246216,403.41210297)
\curveto(226.54245898,403.34209776)(226.65745887,403.28209782)(226.78746216,403.23210297)
\curveto(226.97745855,403.15209795)(227.18245834,403.08209802)(227.40246216,403.02210297)
\lineto(228.09246216,402.87210297)
\curveto(228.33245719,402.83209827)(228.56245696,402.78209832)(228.78246216,402.72210297)
\curveto(229.01245651,402.67209843)(229.2274563,402.60709849)(229.42746216,402.52710297)
\curveto(229.51745601,402.48709861)(229.60245592,402.45209865)(229.68246216,402.42210297)
\curveto(229.77245575,402.4020987)(229.85745567,402.36709873)(229.93746216,402.31710297)
\curveto(230.1274554,402.1970989)(230.29745523,402.06709903)(230.44746216,401.92710297)
\curveto(230.60745492,401.78709931)(230.73245479,401.61209949)(230.82246216,401.40210297)
\curveto(230.85245467,401.33209977)(230.87745465,401.26209984)(230.89746216,401.19210297)
\curveto(230.91745461,401.12209998)(230.93745459,401.04710005)(230.95746216,400.96710297)
\curveto(230.96745456,400.90710019)(230.97245455,400.81210029)(230.97246216,400.68210297)
\curveto(230.98245454,400.56210054)(230.98245454,400.46710063)(230.97246216,400.39710297)
\lineto(230.97246216,400.32210297)
\curveto(230.95245457,400.26210084)(230.93745459,400.2021009)(230.92746216,400.14210297)
\curveto(230.9274546,400.09210101)(230.9224546,400.04210106)(230.91246216,399.99210297)
\curveto(230.84245468,399.69210141)(230.73245479,399.42710167)(230.58246216,399.19710297)
\curveto(230.4224551,398.95710214)(230.2274553,398.76210234)(229.99746216,398.61210297)
\curveto(229.76745576,398.46210264)(229.50745602,398.33210277)(229.21746216,398.22210297)
\curveto(229.10745642,398.17210293)(228.98745654,398.13710296)(228.85746216,398.11710297)
\curveto(228.73745679,398.097103)(228.61745691,398.07210303)(228.49746216,398.04210297)
\curveto(228.40745712,398.02210308)(228.31245721,398.01210309)(228.21246216,398.01210297)
\curveto(228.1224574,398.0021031)(228.03245749,397.98710311)(227.94246216,397.96710297)
\lineto(227.67246216,397.96710297)
\curveto(227.61245791,397.94710315)(227.50745802,397.93710316)(227.35746216,397.93710297)
\curveto(227.21745831,397.93710316)(227.11745841,397.94710315)(227.05746216,397.96710297)
\curveto(227.0274585,397.96710313)(226.99245853,397.97210313)(226.95246216,397.98210297)
\lineto(226.84746216,397.98210297)
\curveto(226.7274588,398.0021031)(226.60745892,398.01710308)(226.48746216,398.02710297)
\curveto(226.36745916,398.03710306)(226.25245927,398.05710304)(226.14246216,398.08710297)
\curveto(225.75245977,398.1971029)(225.40746012,398.32210278)(225.10746216,398.46210297)
\curveto(224.80746072,398.61210249)(224.55246097,398.83210227)(224.34246216,399.12210297)
\curveto(224.20246132,399.31210179)(224.08246144,399.53210157)(223.98246216,399.78210297)
\curveto(223.96246156,399.84210126)(223.94246158,399.92210118)(223.92246216,400.02210297)
\curveto(223.90246162,400.07210103)(223.88746164,400.14210096)(223.87746216,400.23210297)
\curveto(223.86746166,400.32210078)(223.87246165,400.3971007)(223.89246216,400.45710297)
\curveto(223.9224616,400.52710057)(223.97246155,400.57710052)(224.04246216,400.60710297)
\curveto(224.09246143,400.62710047)(224.15246137,400.63710046)(224.22246216,400.63710297)
\lineto(224.44746216,400.63710297)
\lineto(225.15246216,400.63710297)
\lineto(225.39246216,400.63710297)
\curveto(225.47246005,400.63710046)(225.54245998,400.62710047)(225.60246216,400.60710297)
\curveto(225.71245981,400.56710053)(225.78245974,400.5021006)(225.81246216,400.41210297)
\curveto(225.85245967,400.32210078)(225.89745963,400.22710087)(225.94746216,400.12710297)
\curveto(225.96745956,400.07710102)(226.00245952,400.01210109)(226.05246216,399.93210297)
\curveto(226.11245941,399.85210125)(226.16245936,399.8021013)(226.20246216,399.78210297)
\curveto(226.3224592,399.68210142)(226.43745909,399.6021015)(226.54746216,399.54210297)
\curveto(226.65745887,399.49210161)(226.79745873,399.44210166)(226.96746216,399.39210297)
\curveto(227.01745851,399.37210173)(227.06745846,399.36210174)(227.11746216,399.36210297)
\curveto(227.16745836,399.37210173)(227.21745831,399.37210173)(227.26746216,399.36210297)
\curveto(227.34745818,399.34210176)(227.43245809,399.33210177)(227.52246216,399.33210297)
\curveto(227.6224579,399.34210176)(227.70745782,399.35710174)(227.77746216,399.37710297)
\curveto(227.8274577,399.38710171)(227.87245765,399.39210171)(227.91246216,399.39210297)
\curveto(227.96245756,399.39210171)(228.01245751,399.4021017)(228.06246216,399.42210297)
\curveto(228.20245732,399.47210163)(228.3274572,399.53210157)(228.43746216,399.60210297)
\curveto(228.55745697,399.67210143)(228.65245687,399.76210134)(228.72246216,399.87210297)
\curveto(228.77245675,399.95210115)(228.81245671,400.07710102)(228.84246216,400.24710297)
\curveto(228.86245666,400.31710078)(228.86245666,400.38210072)(228.84246216,400.44210297)
\curveto(228.8224567,400.5021006)(228.80245672,400.55210055)(228.78246216,400.59210297)
\curveto(228.71245681,400.73210037)(228.6224569,400.83710026)(228.51246216,400.90710297)
\curveto(228.41245711,400.97710012)(228.29245723,401.04210006)(228.15246216,401.10210297)
\curveto(227.96245756,401.18209992)(227.76245776,401.24709985)(227.55246216,401.29710297)
\curveto(227.34245818,401.34709975)(227.13245839,401.4020997)(226.92246216,401.46210297)
\curveto(226.84245868,401.48209962)(226.75745877,401.4970996)(226.66746216,401.50710297)
\curveto(226.58745894,401.51709958)(226.50745902,401.53209957)(226.42746216,401.55210297)
\curveto(226.10745942,401.64209946)(225.80245972,401.72709937)(225.51246216,401.80710297)
\curveto(225.2224603,401.8970992)(224.95746057,402.02709907)(224.71746216,402.19710297)
\curveto(224.43746109,402.3970987)(224.23246129,402.66709843)(224.10246216,403.00710297)
\curveto(224.08246144,403.07709802)(224.06246146,403.17209793)(224.04246216,403.29210297)
\curveto(224.0224615,403.36209774)(224.00746152,403.44709765)(223.99746216,403.54710297)
\curveto(223.98746154,403.64709745)(223.99246153,403.73709736)(224.01246216,403.81710297)
\curveto(224.03246149,403.86709723)(224.03746149,403.90709719)(224.02746216,403.93710297)
\curveto(224.01746151,403.97709712)(224.0224615,404.02209708)(224.04246216,404.07210297)
\curveto(224.06246146,404.18209692)(224.08246144,404.28209682)(224.10246216,404.37210297)
\curveto(224.13246139,404.47209663)(224.16746136,404.56709653)(224.20746216,404.65710297)
\curveto(224.33746119,404.94709615)(224.51746101,405.18209592)(224.74746216,405.36210297)
\curveto(224.97746055,405.54209556)(225.23746029,405.68709541)(225.52746216,405.79710297)
\curveto(225.63745989,405.84709525)(225.75245977,405.88209522)(225.87246216,405.90210297)
\curveto(225.99245953,405.93209517)(226.11745941,405.96209514)(226.24746216,405.99210297)
\curveto(226.30745922,406.01209509)(226.36745916,406.02209508)(226.42746216,406.02210297)
\lineto(226.60746216,406.05210297)
\curveto(226.68745884,406.06209504)(226.77245875,406.06709503)(226.86246216,406.06710297)
\curveto(226.95245857,406.06709503)(227.03745849,406.07209503)(227.11746216,406.08210297)
}
}
{
\newrgbcolor{curcolor}{0 0 0}
\pscustom[linestyle=none,fillstyle=solid,fillcolor=curcolor]
{
\newpath
\moveto(239.97410278,402.31710297)
\curveto(239.99409421,402.25709884)(240.0040942,402.17209893)(240.00410278,402.06210297)
\curveto(240.0040942,401.95209915)(239.99409421,401.86709923)(239.97410278,401.80710297)
\lineto(239.97410278,401.65710297)
\curveto(239.95409425,401.57709952)(239.94409426,401.4970996)(239.94410278,401.41710297)
\curveto(239.95409425,401.33709976)(239.94909426,401.25709984)(239.92910278,401.17710297)
\curveto(239.9090943,401.10709999)(239.89409431,401.04210006)(239.88410278,400.98210297)
\curveto(239.87409433,400.92210018)(239.86409434,400.85710024)(239.85410278,400.78710297)
\curveto(239.81409439,400.67710042)(239.77909443,400.56210054)(239.74910278,400.44210297)
\curveto(239.71909449,400.33210077)(239.67909453,400.22710087)(239.62910278,400.12710297)
\curveto(239.41909479,399.64710145)(239.14409506,399.25710184)(238.80410278,398.95710297)
\curveto(238.46409574,398.65710244)(238.05409615,398.40710269)(237.57410278,398.20710297)
\curveto(237.45409675,398.15710294)(237.32909688,398.12210298)(237.19910278,398.10210297)
\curveto(237.07909713,398.07210303)(236.95409725,398.04210306)(236.82410278,398.01210297)
\curveto(236.77409743,397.99210311)(236.71909749,397.98210312)(236.65910278,397.98210297)
\curveto(236.59909761,397.98210312)(236.54409766,397.97710312)(236.49410278,397.96710297)
\lineto(236.38910278,397.96710297)
\curveto(236.35909785,397.95710314)(236.32909788,397.95210315)(236.29910278,397.95210297)
\curveto(236.24909796,397.94210316)(236.16909804,397.93710316)(236.05910278,397.93710297)
\curveto(235.94909826,397.92710317)(235.86409834,397.93210317)(235.80410278,397.95210297)
\lineto(235.65410278,397.95210297)
\curveto(235.6040986,397.96210314)(235.54909866,397.96710313)(235.48910278,397.96710297)
\curveto(235.43909877,397.95710314)(235.38909882,397.96210314)(235.33910278,397.98210297)
\curveto(235.29909891,397.99210311)(235.25909895,397.9971031)(235.21910278,397.99710297)
\curveto(235.18909902,397.9971031)(235.14909906,398.0021031)(235.09910278,398.01210297)
\curveto(234.99909921,398.04210306)(234.89909931,398.06710303)(234.79910278,398.08710297)
\curveto(234.69909951,398.10710299)(234.6040996,398.13710296)(234.51410278,398.17710297)
\curveto(234.39409981,398.21710288)(234.27909993,398.25710284)(234.16910278,398.29710297)
\curveto(234.06910014,398.33710276)(233.96410024,398.38710271)(233.85410278,398.44710297)
\curveto(233.5041007,398.65710244)(233.204101,398.9021022)(232.95410278,399.18210297)
\curveto(232.7041015,399.46210164)(232.49410171,399.7971013)(232.32410278,400.18710297)
\curveto(232.27410193,400.27710082)(232.23410197,400.37210073)(232.20410278,400.47210297)
\curveto(232.18410202,400.57210053)(232.15910205,400.67710042)(232.12910278,400.78710297)
\curveto(232.1091021,400.83710026)(232.09910211,400.88210022)(232.09910278,400.92210297)
\curveto(232.09910211,400.96210014)(232.08910212,401.00710009)(232.06910278,401.05710297)
\curveto(232.04910216,401.13709996)(232.03910217,401.21709988)(232.03910278,401.29710297)
\curveto(232.03910217,401.38709971)(232.02910218,401.47209963)(232.00910278,401.55210297)
\curveto(231.99910221,401.6020995)(231.99410221,401.64709945)(231.99410278,401.68710297)
\lineto(231.99410278,401.82210297)
\curveto(231.97410223,401.88209922)(231.96410224,401.96709913)(231.96410278,402.07710297)
\curveto(231.97410223,402.18709891)(231.98910222,402.27209883)(232.00910278,402.33210297)
\lineto(232.00910278,402.43710297)
\curveto(232.01910219,402.48709861)(232.01910219,402.53709856)(232.00910278,402.58710297)
\curveto(232.0091022,402.64709845)(232.01910219,402.7020984)(232.03910278,402.75210297)
\curveto(232.04910216,402.8020983)(232.05410215,402.84709825)(232.05410278,402.88710297)
\curveto(232.05410215,402.93709816)(232.06410214,402.98709811)(232.08410278,403.03710297)
\curveto(232.12410208,403.16709793)(232.15910205,403.29209781)(232.18910278,403.41210297)
\curveto(232.21910199,403.54209756)(232.25910195,403.66709743)(232.30910278,403.78710297)
\curveto(232.48910172,404.1970969)(232.7041015,404.53709656)(232.95410278,404.80710297)
\curveto(233.204101,405.08709601)(233.5091007,405.34209576)(233.86910278,405.57210297)
\curveto(233.96910024,405.62209548)(234.07410013,405.66709543)(234.18410278,405.70710297)
\curveto(234.29409991,405.74709535)(234.4040998,405.79209531)(234.51410278,405.84210297)
\curveto(234.64409956,405.89209521)(234.77909943,405.92709517)(234.91910278,405.94710297)
\curveto(235.05909915,405.96709513)(235.204099,405.9970951)(235.35410278,406.03710297)
\curveto(235.43409877,406.04709505)(235.5090987,406.05209505)(235.57910278,406.05210297)
\curveto(235.64909856,406.05209505)(235.71909849,406.05709504)(235.78910278,406.06710297)
\curveto(236.36909784,406.07709502)(236.86909734,406.01709508)(237.28910278,405.88710297)
\curveto(237.71909649,405.75709534)(238.09909611,405.57709552)(238.42910278,405.34710297)
\curveto(238.53909567,405.26709583)(238.64909556,405.17709592)(238.75910278,405.07710297)
\curveto(238.87909533,404.98709611)(238.97909523,404.88709621)(239.05910278,404.77710297)
\curveto(239.13909507,404.67709642)(239.209095,404.57709652)(239.26910278,404.47710297)
\curveto(239.33909487,404.37709672)(239.4090948,404.27209683)(239.47910278,404.16210297)
\curveto(239.54909466,404.05209705)(239.6040946,403.93209717)(239.64410278,403.80210297)
\curveto(239.68409452,403.68209742)(239.72909448,403.55209755)(239.77910278,403.41210297)
\curveto(239.8090944,403.33209777)(239.83409437,403.24709785)(239.85410278,403.15710297)
\lineto(239.91410278,402.88710297)
\curveto(239.92409428,402.84709825)(239.92909428,402.80709829)(239.92910278,402.76710297)
\curveto(239.92909428,402.72709837)(239.93409427,402.68709841)(239.94410278,402.64710297)
\curveto(239.96409424,402.5970985)(239.96909424,402.54209856)(239.95910278,402.48210297)
\curveto(239.94909426,402.42209868)(239.95409425,402.36709873)(239.97410278,402.31710297)
\moveto(237.87410278,401.77710297)
\curveto(237.88409632,401.82709927)(237.88909632,401.8970992)(237.88910278,401.98710297)
\curveto(237.88909632,402.08709901)(237.88409632,402.16209894)(237.87410278,402.21210297)
\lineto(237.87410278,402.33210297)
\curveto(237.85409635,402.38209872)(237.84409636,402.43709866)(237.84410278,402.49710297)
\curveto(237.84409636,402.55709854)(237.83909637,402.61209849)(237.82910278,402.66210297)
\curveto(237.82909638,402.7020984)(237.82409638,402.73209837)(237.81410278,402.75210297)
\lineto(237.75410278,402.99210297)
\curveto(237.74409646,403.08209802)(237.72409648,403.16709793)(237.69410278,403.24710297)
\curveto(237.58409662,403.50709759)(237.45409675,403.72709737)(237.30410278,403.90710297)
\curveto(237.15409705,404.097097)(236.95409725,404.24709685)(236.70410278,404.35710297)
\curveto(236.64409756,404.37709672)(236.58409762,404.39209671)(236.52410278,404.40210297)
\curveto(236.46409774,404.42209668)(236.39909781,404.44209666)(236.32910278,404.46210297)
\curveto(236.24909796,404.48209662)(236.16409804,404.48709661)(236.07410278,404.47710297)
\lineto(235.80410278,404.47710297)
\curveto(235.77409843,404.45709664)(235.73909847,404.44709665)(235.69910278,404.44710297)
\curveto(235.65909855,404.45709664)(235.62409858,404.45709664)(235.59410278,404.44710297)
\lineto(235.38410278,404.38710297)
\curveto(235.32409888,404.37709672)(235.26909894,404.35709674)(235.21910278,404.32710297)
\curveto(234.96909924,404.21709688)(234.76409944,404.05709704)(234.60410278,403.84710297)
\curveto(234.45409975,403.64709745)(234.33409987,403.41209769)(234.24410278,403.14210297)
\curveto(234.21409999,403.04209806)(234.18910002,402.93709816)(234.16910278,402.82710297)
\curveto(234.15910005,402.71709838)(234.14410006,402.60709849)(234.12410278,402.49710297)
\curveto(234.11410009,402.44709865)(234.1091001,402.3970987)(234.10910278,402.34710297)
\lineto(234.10910278,402.19710297)
\curveto(234.08910012,402.12709897)(234.07910013,402.02209908)(234.07910278,401.88210297)
\curveto(234.08910012,401.74209936)(234.1041001,401.63709946)(234.12410278,401.56710297)
\lineto(234.12410278,401.43210297)
\curveto(234.14410006,401.35209975)(234.15910005,401.27209983)(234.16910278,401.19210297)
\curveto(234.17910003,401.12209998)(234.19410001,401.04710005)(234.21410278,400.96710297)
\curveto(234.31409989,400.66710043)(234.41909979,400.42210068)(234.52910278,400.23210297)
\curveto(234.64909956,400.05210105)(234.83409937,399.88710121)(235.08410278,399.73710297)
\curveto(235.15409905,399.68710141)(235.22909898,399.64710145)(235.30910278,399.61710297)
\curveto(235.39909881,399.58710151)(235.48909872,399.56210154)(235.57910278,399.54210297)
\curveto(235.61909859,399.53210157)(235.65409855,399.52710157)(235.68410278,399.52710297)
\curveto(235.71409849,399.53710156)(235.74909846,399.53710156)(235.78910278,399.52710297)
\lineto(235.90910278,399.49710297)
\curveto(235.95909825,399.4971016)(236.0040982,399.5021016)(236.04410278,399.51210297)
\lineto(236.16410278,399.51210297)
\curveto(236.24409796,399.53210157)(236.32409788,399.54710155)(236.40410278,399.55710297)
\curveto(236.48409772,399.56710153)(236.55909765,399.58710151)(236.62910278,399.61710297)
\curveto(236.88909732,399.71710138)(237.09909711,399.85210125)(237.25910278,400.02210297)
\curveto(237.41909679,400.19210091)(237.55409665,400.4021007)(237.66410278,400.65210297)
\curveto(237.7040965,400.75210035)(237.73409647,400.85210025)(237.75410278,400.95210297)
\curveto(237.77409643,401.05210005)(237.79909641,401.15709994)(237.82910278,401.26710297)
\curveto(237.83909637,401.30709979)(237.84409636,401.34209976)(237.84410278,401.37210297)
\curveto(237.84409636,401.41209969)(237.84909636,401.45209965)(237.85910278,401.49210297)
\lineto(237.85910278,401.62710297)
\curveto(237.85909635,401.67709942)(237.86409634,401.72709937)(237.87410278,401.77710297)
}
}
{
\newrgbcolor{curcolor}{0 0 0}
\pscustom[linestyle=none,fillstyle=solid,fillcolor=curcolor]
{
\newpath
\moveto(244.34402466,406.08210297)
\curveto(245.09402016,406.102095)(245.74401951,406.01709508)(246.29402466,405.82710297)
\curveto(246.8540184,405.64709545)(247.27901797,405.33209577)(247.56902466,404.88210297)
\curveto(247.63901761,404.77209633)(247.69901755,404.65709644)(247.74902466,404.53710297)
\curveto(247.80901744,404.42709667)(247.85901739,404.3020968)(247.89902466,404.16210297)
\curveto(247.91901733,404.102097)(247.92901732,404.03709706)(247.92902466,403.96710297)
\curveto(247.92901732,403.8970972)(247.91901733,403.83709726)(247.89902466,403.78710297)
\curveto(247.85901739,403.72709737)(247.80401745,403.68709741)(247.73402466,403.66710297)
\curveto(247.68401757,403.64709745)(247.62401763,403.63709746)(247.55402466,403.63710297)
\lineto(247.34402466,403.63710297)
\lineto(246.68402466,403.63710297)
\curveto(246.61401864,403.63709746)(246.54401871,403.63209747)(246.47402466,403.62210297)
\curveto(246.40401885,403.62209748)(246.33901891,403.63209747)(246.27902466,403.65210297)
\curveto(246.17901907,403.67209743)(246.10401915,403.71209739)(246.05402466,403.77210297)
\curveto(246.00401925,403.83209727)(245.95901929,403.89209721)(245.91902466,403.95210297)
\lineto(245.79902466,404.16210297)
\curveto(245.76901948,404.24209686)(245.71901953,404.30709679)(245.64902466,404.35710297)
\curveto(245.5490197,404.43709666)(245.4490198,404.4970966)(245.34902466,404.53710297)
\curveto(245.25901999,404.57709652)(245.14402011,404.61209649)(245.00402466,404.64210297)
\curveto(244.93402032,404.66209644)(244.82902042,404.67709642)(244.68902466,404.68710297)
\curveto(244.55902069,404.6970964)(244.45902079,404.69209641)(244.38902466,404.67210297)
\lineto(244.28402466,404.67210297)
\lineto(244.13402466,404.64210297)
\curveto(244.09402116,404.64209646)(244.0490212,404.63709646)(243.99902466,404.62710297)
\curveto(243.82902142,404.57709652)(243.68902156,404.50709659)(243.57902466,404.41710297)
\curveto(243.47902177,404.33709676)(243.40902184,404.21209689)(243.36902466,404.04210297)
\curveto(243.3490219,403.97209713)(243.3490219,403.90709719)(243.36902466,403.84710297)
\curveto(243.38902186,403.78709731)(243.40902184,403.73709736)(243.42902466,403.69710297)
\curveto(243.49902175,403.57709752)(243.57902167,403.48209762)(243.66902466,403.41210297)
\curveto(243.76902148,403.34209776)(243.88402137,403.28209782)(244.01402466,403.23210297)
\curveto(244.20402105,403.15209795)(244.40902084,403.08209802)(244.62902466,403.02210297)
\lineto(245.31902466,402.87210297)
\curveto(245.55901969,402.83209827)(245.78901946,402.78209832)(246.00902466,402.72210297)
\curveto(246.23901901,402.67209843)(246.4540188,402.60709849)(246.65402466,402.52710297)
\curveto(246.74401851,402.48709861)(246.82901842,402.45209865)(246.90902466,402.42210297)
\curveto(246.99901825,402.4020987)(247.08401817,402.36709873)(247.16402466,402.31710297)
\curveto(247.3540179,402.1970989)(247.52401773,402.06709903)(247.67402466,401.92710297)
\curveto(247.83401742,401.78709931)(247.95901729,401.61209949)(248.04902466,401.40210297)
\curveto(248.07901717,401.33209977)(248.10401715,401.26209984)(248.12402466,401.19210297)
\curveto(248.14401711,401.12209998)(248.16401709,401.04710005)(248.18402466,400.96710297)
\curveto(248.19401706,400.90710019)(248.19901705,400.81210029)(248.19902466,400.68210297)
\curveto(248.20901704,400.56210054)(248.20901704,400.46710063)(248.19902466,400.39710297)
\lineto(248.19902466,400.32210297)
\curveto(248.17901707,400.26210084)(248.16401709,400.2021009)(248.15402466,400.14210297)
\curveto(248.1540171,400.09210101)(248.1490171,400.04210106)(248.13902466,399.99210297)
\curveto(248.06901718,399.69210141)(247.95901729,399.42710167)(247.80902466,399.19710297)
\curveto(247.6490176,398.95710214)(247.4540178,398.76210234)(247.22402466,398.61210297)
\curveto(246.99401826,398.46210264)(246.73401852,398.33210277)(246.44402466,398.22210297)
\curveto(246.33401892,398.17210293)(246.21401904,398.13710296)(246.08402466,398.11710297)
\curveto(245.96401929,398.097103)(245.84401941,398.07210303)(245.72402466,398.04210297)
\curveto(245.63401962,398.02210308)(245.53901971,398.01210309)(245.43902466,398.01210297)
\curveto(245.3490199,398.0021031)(245.25901999,397.98710311)(245.16902466,397.96710297)
\lineto(244.89902466,397.96710297)
\curveto(244.83902041,397.94710315)(244.73402052,397.93710316)(244.58402466,397.93710297)
\curveto(244.44402081,397.93710316)(244.34402091,397.94710315)(244.28402466,397.96710297)
\curveto(244.254021,397.96710313)(244.21902103,397.97210313)(244.17902466,397.98210297)
\lineto(244.07402466,397.98210297)
\curveto(243.9540213,398.0021031)(243.83402142,398.01710308)(243.71402466,398.02710297)
\curveto(243.59402166,398.03710306)(243.47902177,398.05710304)(243.36902466,398.08710297)
\curveto(242.97902227,398.1971029)(242.63402262,398.32210278)(242.33402466,398.46210297)
\curveto(242.03402322,398.61210249)(241.77902347,398.83210227)(241.56902466,399.12210297)
\curveto(241.42902382,399.31210179)(241.30902394,399.53210157)(241.20902466,399.78210297)
\curveto(241.18902406,399.84210126)(241.16902408,399.92210118)(241.14902466,400.02210297)
\curveto(241.12902412,400.07210103)(241.11402414,400.14210096)(241.10402466,400.23210297)
\curveto(241.09402416,400.32210078)(241.09902415,400.3971007)(241.11902466,400.45710297)
\curveto(241.1490241,400.52710057)(241.19902405,400.57710052)(241.26902466,400.60710297)
\curveto(241.31902393,400.62710047)(241.37902387,400.63710046)(241.44902466,400.63710297)
\lineto(241.67402466,400.63710297)
\lineto(242.37902466,400.63710297)
\lineto(242.61902466,400.63710297)
\curveto(242.69902255,400.63710046)(242.76902248,400.62710047)(242.82902466,400.60710297)
\curveto(242.93902231,400.56710053)(243.00902224,400.5021006)(243.03902466,400.41210297)
\curveto(243.07902217,400.32210078)(243.12402213,400.22710087)(243.17402466,400.12710297)
\curveto(243.19402206,400.07710102)(243.22902202,400.01210109)(243.27902466,399.93210297)
\curveto(243.33902191,399.85210125)(243.38902186,399.8021013)(243.42902466,399.78210297)
\curveto(243.5490217,399.68210142)(243.66402159,399.6021015)(243.77402466,399.54210297)
\curveto(243.88402137,399.49210161)(244.02402123,399.44210166)(244.19402466,399.39210297)
\curveto(244.24402101,399.37210173)(244.29402096,399.36210174)(244.34402466,399.36210297)
\curveto(244.39402086,399.37210173)(244.44402081,399.37210173)(244.49402466,399.36210297)
\curveto(244.57402068,399.34210176)(244.65902059,399.33210177)(244.74902466,399.33210297)
\curveto(244.8490204,399.34210176)(244.93402032,399.35710174)(245.00402466,399.37710297)
\curveto(245.0540202,399.38710171)(245.09902015,399.39210171)(245.13902466,399.39210297)
\curveto(245.18902006,399.39210171)(245.23902001,399.4021017)(245.28902466,399.42210297)
\curveto(245.42901982,399.47210163)(245.5540197,399.53210157)(245.66402466,399.60210297)
\curveto(245.78401947,399.67210143)(245.87901937,399.76210134)(245.94902466,399.87210297)
\curveto(245.99901925,399.95210115)(246.03901921,400.07710102)(246.06902466,400.24710297)
\curveto(246.08901916,400.31710078)(246.08901916,400.38210072)(246.06902466,400.44210297)
\curveto(246.0490192,400.5021006)(246.02901922,400.55210055)(246.00902466,400.59210297)
\curveto(245.93901931,400.73210037)(245.8490194,400.83710026)(245.73902466,400.90710297)
\curveto(245.63901961,400.97710012)(245.51901973,401.04210006)(245.37902466,401.10210297)
\curveto(245.18902006,401.18209992)(244.98902026,401.24709985)(244.77902466,401.29710297)
\curveto(244.56902068,401.34709975)(244.35902089,401.4020997)(244.14902466,401.46210297)
\curveto(244.06902118,401.48209962)(243.98402127,401.4970996)(243.89402466,401.50710297)
\curveto(243.81402144,401.51709958)(243.73402152,401.53209957)(243.65402466,401.55210297)
\curveto(243.33402192,401.64209946)(243.02902222,401.72709937)(242.73902466,401.80710297)
\curveto(242.4490228,401.8970992)(242.18402307,402.02709907)(241.94402466,402.19710297)
\curveto(241.66402359,402.3970987)(241.45902379,402.66709843)(241.32902466,403.00710297)
\curveto(241.30902394,403.07709802)(241.28902396,403.17209793)(241.26902466,403.29210297)
\curveto(241.249024,403.36209774)(241.23402402,403.44709765)(241.22402466,403.54710297)
\curveto(241.21402404,403.64709745)(241.21902403,403.73709736)(241.23902466,403.81710297)
\curveto(241.25902399,403.86709723)(241.26402399,403.90709719)(241.25402466,403.93710297)
\curveto(241.24402401,403.97709712)(241.249024,404.02209708)(241.26902466,404.07210297)
\curveto(241.28902396,404.18209692)(241.30902394,404.28209682)(241.32902466,404.37210297)
\curveto(241.35902389,404.47209663)(241.39402386,404.56709653)(241.43402466,404.65710297)
\curveto(241.56402369,404.94709615)(241.74402351,405.18209592)(241.97402466,405.36210297)
\curveto(242.20402305,405.54209556)(242.46402279,405.68709541)(242.75402466,405.79710297)
\curveto(242.86402239,405.84709525)(242.97902227,405.88209522)(243.09902466,405.90210297)
\curveto(243.21902203,405.93209517)(243.34402191,405.96209514)(243.47402466,405.99210297)
\curveto(243.53402172,406.01209509)(243.59402166,406.02209508)(243.65402466,406.02210297)
\lineto(243.83402466,406.05210297)
\curveto(243.91402134,406.06209504)(243.99902125,406.06709503)(244.08902466,406.06710297)
\curveto(244.17902107,406.06709503)(244.26402099,406.07209503)(244.34402466,406.08210297)
}
}
{
\newrgbcolor{curcolor}{0 0 0}
\pscustom[linestyle=none,fillstyle=solid,fillcolor=curcolor]
{
}
}
{
\newrgbcolor{curcolor}{0 0 0}
\pscustom[linestyle=none,fillstyle=solid,fillcolor=curcolor]
{
\newpath
\moveto(261.48082153,402.09210297)
\curveto(261.49081285,402.03209907)(261.49581285,401.94209916)(261.49582153,401.82210297)
\curveto(261.49581285,401.7020994)(261.48581286,401.61709948)(261.46582153,401.56710297)
\lineto(261.46582153,401.37210297)
\curveto(261.43581291,401.26209984)(261.41581293,401.15709994)(261.40582153,401.05710297)
\curveto(261.40581294,400.95710014)(261.39081295,400.85710024)(261.36082153,400.75710297)
\curveto(261.340813,400.66710043)(261.32081302,400.57210053)(261.30082153,400.47210297)
\curveto(261.28081306,400.38210072)(261.25081309,400.29210081)(261.21082153,400.20210297)
\curveto(261.1408132,400.03210107)(261.07081327,399.87210123)(261.00082153,399.72210297)
\curveto(260.93081341,399.58210152)(260.85081349,399.44210166)(260.76082153,399.30210297)
\curveto(260.70081364,399.21210189)(260.63581371,399.12710197)(260.56582153,399.04710297)
\curveto(260.50581384,398.97710212)(260.43581391,398.9021022)(260.35582153,398.82210297)
\lineto(260.25082153,398.71710297)
\curveto(260.20081414,398.66710243)(260.1458142,398.62210248)(260.08582153,398.58210297)
\lineto(259.93582153,398.46210297)
\curveto(259.85581449,398.4021027)(259.76581458,398.34710275)(259.66582153,398.29710297)
\curveto(259.57581477,398.25710284)(259.48081486,398.21210289)(259.38082153,398.16210297)
\curveto(259.28081506,398.11210299)(259.17581517,398.07710302)(259.06582153,398.05710297)
\curveto(258.96581538,398.03710306)(258.86081548,398.01710308)(258.75082153,397.99710297)
\curveto(258.69081565,397.97710312)(258.62581572,397.96710313)(258.55582153,397.96710297)
\curveto(258.49581585,397.96710313)(258.43081591,397.95710314)(258.36082153,397.93710297)
\lineto(258.22582153,397.93710297)
\curveto(258.1458162,397.91710318)(258.07081627,397.91710318)(258.00082153,397.93710297)
\lineto(257.85082153,397.93710297)
\curveto(257.79081655,397.95710314)(257.72581662,397.96710313)(257.65582153,397.96710297)
\curveto(257.59581675,397.95710314)(257.53581681,397.96210314)(257.47582153,397.98210297)
\curveto(257.31581703,398.03210307)(257.16081718,398.07710302)(257.01082153,398.11710297)
\curveto(256.87081747,398.15710294)(256.7408176,398.21710288)(256.62082153,398.29710297)
\curveto(256.55081779,398.33710276)(256.48581786,398.37710272)(256.42582153,398.41710297)
\curveto(256.36581798,398.46710263)(256.30081804,398.51710258)(256.23082153,398.56710297)
\lineto(256.05082153,398.70210297)
\curveto(255.97081837,398.76210234)(255.90081844,398.76710233)(255.84082153,398.71710297)
\curveto(255.79081855,398.68710241)(255.76581858,398.64710245)(255.76582153,398.59710297)
\curveto(255.76581858,398.55710254)(255.75581859,398.50710259)(255.73582153,398.44710297)
\curveto(255.71581863,398.34710275)(255.70581864,398.23210287)(255.70582153,398.10210297)
\curveto(255.71581863,397.97210313)(255.72081862,397.85210325)(255.72082153,397.74210297)
\lineto(255.72082153,396.21210297)
\curveto(255.72081862,396.08210502)(255.71581863,395.95710514)(255.70582153,395.83710297)
\curveto(255.70581864,395.70710539)(255.68081866,395.6021055)(255.63082153,395.52210297)
\curveto(255.60081874,395.48210562)(255.5458188,395.45210565)(255.46582153,395.43210297)
\curveto(255.38581896,395.41210569)(255.29581905,395.4021057)(255.19582153,395.40210297)
\curveto(255.09581925,395.39210571)(254.99581935,395.39210571)(254.89582153,395.40210297)
\lineto(254.64082153,395.40210297)
\lineto(254.23582153,395.40210297)
\lineto(254.13082153,395.40210297)
\curveto(254.09082025,395.4021057)(254.05582029,395.40710569)(254.02582153,395.41710297)
\lineto(253.90582153,395.41710297)
\curveto(253.73582061,395.46710563)(253.6458207,395.56710553)(253.63582153,395.71710297)
\curveto(253.62582072,395.85710524)(253.62082072,396.02710507)(253.62082153,396.22710297)
\lineto(253.62082153,405.03210297)
\curveto(253.62082072,405.14209596)(253.61582073,405.25709584)(253.60582153,405.37710297)
\curveto(253.60582074,405.50709559)(253.63082071,405.60709549)(253.68082153,405.67710297)
\curveto(253.72082062,405.74709535)(253.77582057,405.79209531)(253.84582153,405.81210297)
\curveto(253.89582045,405.83209527)(253.95582039,405.84209526)(254.02582153,405.84210297)
\lineto(254.25082153,405.84210297)
\lineto(254.97082153,405.84210297)
\lineto(255.25582153,405.84210297)
\curveto(255.345819,405.84209526)(255.42081892,405.81709528)(255.48082153,405.76710297)
\curveto(255.55081879,405.71709538)(255.58581876,405.65209545)(255.58582153,405.57210297)
\curveto(255.59581875,405.5020956)(255.62081872,405.42709567)(255.66082153,405.34710297)
\curveto(255.67081867,405.31709578)(255.68081866,405.29209581)(255.69082153,405.27210297)
\curveto(255.71081863,405.26209584)(255.73081861,405.24709585)(255.75082153,405.22710297)
\curveto(255.86081848,405.21709588)(255.95081839,405.24709585)(256.02082153,405.31710297)
\curveto(256.09081825,405.38709571)(256.16081818,405.44709565)(256.23082153,405.49710297)
\curveto(256.36081798,405.58709551)(256.49581785,405.66709543)(256.63582153,405.73710297)
\curveto(256.77581757,405.81709528)(256.93081741,405.88209522)(257.10082153,405.93210297)
\curveto(257.18081716,405.96209514)(257.26581708,405.98209512)(257.35582153,405.99210297)
\curveto(257.45581689,406.0020951)(257.55081679,406.01709508)(257.64082153,406.03710297)
\curveto(257.68081666,406.04709505)(257.72081662,406.04709505)(257.76082153,406.03710297)
\curveto(257.81081653,406.02709507)(257.85081649,406.03209507)(257.88082153,406.05210297)
\curveto(258.45081589,406.07209503)(258.93081541,405.99209511)(259.32082153,405.81210297)
\curveto(259.72081462,405.64209546)(260.06081428,405.41709568)(260.34082153,405.13710297)
\curveto(260.39081395,405.08709601)(260.43581391,405.03709606)(260.47582153,404.98710297)
\curveto(260.51581383,404.94709615)(260.55581379,404.9020962)(260.59582153,404.85210297)
\curveto(260.66581368,404.76209634)(260.72581362,404.67209643)(260.77582153,404.58210297)
\curveto(260.83581351,404.49209661)(260.89081345,404.4020967)(260.94082153,404.31210297)
\curveto(260.96081338,404.29209681)(260.97081337,404.26709683)(260.97082153,404.23710297)
\curveto(260.98081336,404.20709689)(260.99581335,404.17209693)(261.01582153,404.13210297)
\curveto(261.07581327,404.03209707)(261.13081321,403.91209719)(261.18082153,403.77210297)
\curveto(261.20081314,403.71209739)(261.22081312,403.64709745)(261.24082153,403.57710297)
\curveto(261.26081308,403.51709758)(261.28081306,403.45209765)(261.30082153,403.38210297)
\curveto(261.340813,403.26209784)(261.36581298,403.13709796)(261.37582153,403.00710297)
\curveto(261.39581295,402.87709822)(261.42081292,402.74209836)(261.45082153,402.60210297)
\lineto(261.45082153,402.43710297)
\lineto(261.48082153,402.25710297)
\lineto(261.48082153,402.09210297)
\moveto(259.36582153,401.74710297)
\curveto(259.37581497,401.7970993)(259.38081496,401.86209924)(259.38082153,401.94210297)
\curveto(259.38081496,402.03209907)(259.37581497,402.102099)(259.36582153,402.15210297)
\lineto(259.36582153,402.28710297)
\curveto(259.345815,402.34709875)(259.33581501,402.41209869)(259.33582153,402.48210297)
\curveto(259.33581501,402.55209855)(259.32581502,402.62209848)(259.30582153,402.69210297)
\curveto(259.28581506,402.79209831)(259.26581508,402.88709821)(259.24582153,402.97710297)
\curveto(259.22581512,403.07709802)(259.19581515,403.16709793)(259.15582153,403.24710297)
\curveto(259.03581531,403.56709753)(258.88081546,403.82209728)(258.69082153,404.01210297)
\curveto(258.50081584,404.2020969)(258.23081611,404.34209676)(257.88082153,404.43210297)
\curveto(257.80081654,404.45209665)(257.71081663,404.46209664)(257.61082153,404.46210297)
\lineto(257.34082153,404.46210297)
\curveto(257.30081704,404.45209665)(257.26581708,404.44709665)(257.23582153,404.44710297)
\curveto(257.20581714,404.44709665)(257.17081717,404.44209666)(257.13082153,404.43210297)
\lineto(256.92082153,404.37210297)
\curveto(256.86081748,404.36209674)(256.80081754,404.34209676)(256.74082153,404.31210297)
\curveto(256.48081786,404.2020969)(256.27581807,404.03209707)(256.12582153,403.80210297)
\curveto(255.98581836,403.57209753)(255.87081847,403.31709778)(255.78082153,403.03710297)
\curveto(255.76081858,402.95709814)(255.7458186,402.87209823)(255.73582153,402.78210297)
\curveto(255.72581862,402.7020984)(255.71081863,402.62209848)(255.69082153,402.54210297)
\curveto(255.68081866,402.5020986)(255.67581867,402.43709866)(255.67582153,402.34710297)
\curveto(255.65581869,402.30709879)(255.65081869,402.25709884)(255.66082153,402.19710297)
\curveto(255.67081867,402.14709895)(255.67081867,402.097099)(255.66082153,402.04710297)
\curveto(255.6408187,401.98709911)(255.6408187,401.93209917)(255.66082153,401.88210297)
\lineto(255.66082153,401.70210297)
\lineto(255.66082153,401.56710297)
\curveto(255.66081868,401.52709957)(255.67081867,401.48709961)(255.69082153,401.44710297)
\curveto(255.69081865,401.37709972)(255.69581865,401.32209978)(255.70582153,401.28210297)
\lineto(255.73582153,401.10210297)
\curveto(255.7458186,401.04210006)(255.76081858,400.98210012)(255.78082153,400.92210297)
\curveto(255.87081847,400.63210047)(255.97581837,400.39210071)(256.09582153,400.20210297)
\curveto(256.22581812,400.02210108)(256.40581794,399.86210124)(256.63582153,399.72210297)
\curveto(256.77581757,399.64210146)(256.9408174,399.57710152)(257.13082153,399.52710297)
\curveto(257.17081717,399.51710158)(257.20581714,399.51210159)(257.23582153,399.51210297)
\curveto(257.26581708,399.52210158)(257.30081704,399.52210158)(257.34082153,399.51210297)
\curveto(257.38081696,399.5021016)(257.4408169,399.49210161)(257.52082153,399.48210297)
\curveto(257.60081674,399.48210162)(257.66581668,399.48710161)(257.71582153,399.49710297)
\curveto(257.79581655,399.51710158)(257.87581647,399.53210157)(257.95582153,399.54210297)
\curveto(258.0458163,399.56210154)(258.13081621,399.58710151)(258.21082153,399.61710297)
\curveto(258.45081589,399.71710138)(258.6458157,399.85710124)(258.79582153,400.03710297)
\curveto(258.9458154,400.21710088)(259.07081527,400.42710067)(259.17082153,400.66710297)
\curveto(259.22081512,400.78710031)(259.25581509,400.91210019)(259.27582153,401.04210297)
\curveto(259.29581505,401.17209993)(259.32081502,401.30709979)(259.35082153,401.44710297)
\lineto(259.35082153,401.59710297)
\curveto(259.36081498,401.64709945)(259.36581498,401.6970994)(259.36582153,401.74710297)
}
}
{
\newrgbcolor{curcolor}{0 0 0}
\pscustom[linestyle=none,fillstyle=solid,fillcolor=curcolor]
{
\newpath
\moveto(263.18074341,405.85710297)
\lineto(264.30574341,405.85710297)
\curveto(264.41574097,405.85709524)(264.51574087,405.85209525)(264.60574341,405.84210297)
\curveto(264.69574069,405.83209527)(264.76074063,405.7970953)(264.80074341,405.73710297)
\curveto(264.85074054,405.67709542)(264.88074051,405.59209551)(264.89074341,405.48210297)
\curveto(264.90074049,405.38209572)(264.90574048,405.27709582)(264.90574341,405.16710297)
\lineto(264.90574341,404.11710297)
\lineto(264.90574341,401.88210297)
\curveto(264.90574048,401.52209958)(264.92074047,401.18209992)(264.95074341,400.86210297)
\curveto(264.98074041,400.54210056)(265.07074032,400.27710082)(265.22074341,400.06710297)
\curveto(265.36074003,399.85710124)(265.5857398,399.70710139)(265.89574341,399.61710297)
\curveto(265.94573944,399.60710149)(265.9857394,399.6021015)(266.01574341,399.60210297)
\curveto(266.05573933,399.6021015)(266.10073929,399.5971015)(266.15074341,399.58710297)
\curveto(266.20073919,399.57710152)(266.25573913,399.57210153)(266.31574341,399.57210297)
\curveto(266.37573901,399.57210153)(266.42073897,399.57710152)(266.45074341,399.58710297)
\curveto(266.50073889,399.60710149)(266.54073885,399.61210149)(266.57074341,399.60210297)
\curveto(266.61073878,399.59210151)(266.65073874,399.5971015)(266.69074341,399.61710297)
\curveto(266.90073849,399.66710143)(267.06573832,399.73210137)(267.18574341,399.81210297)
\curveto(267.36573802,399.92210118)(267.50573788,400.06210104)(267.60574341,400.23210297)
\curveto(267.71573767,400.41210069)(267.7907376,400.60710049)(267.83074341,400.81710297)
\curveto(267.88073751,401.03710006)(267.91073748,401.27709982)(267.92074341,401.53710297)
\curveto(267.93073746,401.80709929)(267.93573745,402.08709901)(267.93574341,402.37710297)
\lineto(267.93574341,404.19210297)
\lineto(267.93574341,405.16710297)
\lineto(267.93574341,405.43710297)
\curveto(267.93573745,405.53709556)(267.95573743,405.61709548)(267.99574341,405.67710297)
\curveto(268.04573734,405.76709533)(268.12073727,405.81709528)(268.22074341,405.82710297)
\curveto(268.32073707,405.84709525)(268.44073695,405.85709524)(268.58074341,405.85710297)
\lineto(269.37574341,405.85710297)
\lineto(269.66074341,405.85710297)
\curveto(269.75073564,405.85709524)(269.82573556,405.83709526)(269.88574341,405.79710297)
\curveto(269.96573542,405.74709535)(270.01073538,405.67209543)(270.02074341,405.57210297)
\curveto(270.03073536,405.47209563)(270.03573535,405.35709574)(270.03574341,405.22710297)
\lineto(270.03574341,404.08710297)
\lineto(270.03574341,399.87210297)
\lineto(270.03574341,398.80710297)
\lineto(270.03574341,398.50710297)
\curveto(270.03573535,398.40710269)(270.01573537,398.33210277)(269.97574341,398.28210297)
\curveto(269.92573546,398.2021029)(269.85073554,398.15710294)(269.75074341,398.14710297)
\curveto(269.65073574,398.13710296)(269.54573584,398.13210297)(269.43574341,398.13210297)
\lineto(268.62574341,398.13210297)
\curveto(268.51573687,398.13210297)(268.41573697,398.13710296)(268.32574341,398.14710297)
\curveto(268.24573714,398.15710294)(268.18073721,398.1971029)(268.13074341,398.26710297)
\curveto(268.11073728,398.2971028)(268.0907373,398.34210276)(268.07074341,398.40210297)
\curveto(268.06073733,398.46210264)(268.04573734,398.52210258)(268.02574341,398.58210297)
\curveto(268.01573737,398.64210246)(268.00073739,398.6971024)(267.98074341,398.74710297)
\curveto(267.96073743,398.7971023)(267.93073746,398.82710227)(267.89074341,398.83710297)
\curveto(267.87073752,398.85710224)(267.84573754,398.86210224)(267.81574341,398.85210297)
\curveto(267.7857376,398.84210226)(267.76073763,398.83210227)(267.74074341,398.82210297)
\curveto(267.67073772,398.78210232)(267.61073778,398.73710236)(267.56074341,398.68710297)
\curveto(267.51073788,398.63710246)(267.45573793,398.59210251)(267.39574341,398.55210297)
\curveto(267.35573803,398.52210258)(267.31573807,398.48710261)(267.27574341,398.44710297)
\curveto(267.24573814,398.41710268)(267.20573818,398.38710271)(267.15574341,398.35710297)
\curveto(266.92573846,398.21710288)(266.65573873,398.10710299)(266.34574341,398.02710297)
\curveto(266.27573911,398.00710309)(266.20573918,397.9971031)(266.13574341,397.99710297)
\curveto(266.06573932,397.98710311)(265.9907394,397.97210313)(265.91074341,397.95210297)
\curveto(265.87073952,397.94210316)(265.82573956,397.94210316)(265.77574341,397.95210297)
\curveto(265.73573965,397.95210315)(265.69573969,397.94710315)(265.65574341,397.93710297)
\curveto(265.62573976,397.92710317)(265.56073983,397.92710317)(265.46074341,397.93710297)
\curveto(265.37074002,397.93710316)(265.31074008,397.94210316)(265.28074341,397.95210297)
\curveto(265.23074016,397.95210315)(265.18074021,397.95710314)(265.13074341,397.96710297)
\lineto(264.98074341,397.96710297)
\curveto(264.86074053,397.9971031)(264.74574064,398.02210308)(264.63574341,398.04210297)
\curveto(264.52574086,398.06210304)(264.41574097,398.09210301)(264.30574341,398.13210297)
\curveto(264.25574113,398.15210295)(264.21074118,398.16710293)(264.17074341,398.17710297)
\curveto(264.14074125,398.1971029)(264.10074129,398.21710288)(264.05074341,398.23710297)
\curveto(263.70074169,398.42710267)(263.42074197,398.69210241)(263.21074341,399.03210297)
\curveto(263.08074231,399.24210186)(262.9857424,399.49210161)(262.92574341,399.78210297)
\curveto(262.86574252,400.08210102)(262.82574256,400.3971007)(262.80574341,400.72710297)
\curveto(262.79574259,401.06710003)(262.7907426,401.41209969)(262.79074341,401.76210297)
\curveto(262.80074259,402.12209898)(262.80574258,402.47709862)(262.80574341,402.82710297)
\lineto(262.80574341,404.86710297)
\curveto(262.80574258,404.9970961)(262.80074259,405.14709595)(262.79074341,405.31710297)
\curveto(262.7907426,405.4970956)(262.81574257,405.62709547)(262.86574341,405.70710297)
\curveto(262.89574249,405.75709534)(262.95574243,405.8020953)(263.04574341,405.84210297)
\curveto(263.10574228,405.84209526)(263.15074224,405.84709525)(263.18074341,405.85710297)
}
}
{
\newrgbcolor{curcolor}{0 0 0}
\pscustom[linestyle=none,fillstyle=solid,fillcolor=curcolor]
{
\newpath
\moveto(279.46699341,402.39210297)
\curveto(279.48698481,402.33209877)(279.4969848,402.22709887)(279.49699341,402.07710297)
\curveto(279.4969848,401.93709916)(279.4919848,401.83709926)(279.48199341,401.77710297)
\curveto(279.48198481,401.72709937)(279.47698482,401.68209942)(279.46699341,401.64210297)
\lineto(279.46699341,401.52210297)
\curveto(279.44698485,401.44209966)(279.43698486,401.36209974)(279.43699341,401.28210297)
\curveto(279.43698486,401.21209989)(279.42698487,401.13709996)(279.40699341,401.05710297)
\curveto(279.40698489,401.01710008)(279.3969849,400.94710015)(279.37699341,400.84710297)
\curveto(279.34698495,400.72710037)(279.31698498,400.6021005)(279.28699341,400.47210297)
\curveto(279.26698503,400.35210075)(279.23198506,400.23710086)(279.18199341,400.12710297)
\curveto(279.00198529,399.67710142)(278.77698552,399.28710181)(278.50699341,398.95710297)
\curveto(278.23698606,398.62710247)(277.88198641,398.36710273)(277.44199341,398.17710297)
\curveto(277.35198694,398.13710296)(277.25698704,398.10710299)(277.15699341,398.08710297)
\curveto(277.06698723,398.05710304)(276.96698733,398.02710307)(276.85699341,397.99710297)
\curveto(276.7969875,397.97710312)(276.73198756,397.96710313)(276.66199341,397.96710297)
\curveto(276.60198769,397.96710313)(276.54198775,397.96210314)(276.48199341,397.95210297)
\lineto(276.34699341,397.95210297)
\curveto(276.28698801,397.93210317)(276.20698809,397.92710317)(276.10699341,397.93710297)
\curveto(276.00698829,397.93710316)(275.92698837,397.94710315)(275.86699341,397.96710297)
\lineto(275.77699341,397.96710297)
\curveto(275.72698857,397.97710312)(275.67198862,397.98710311)(275.61199341,397.99710297)
\curveto(275.55198874,397.9971031)(275.4919888,398.0021031)(275.43199341,398.01210297)
\curveto(275.24198905,398.06210304)(275.06698923,398.11210299)(274.90699341,398.16210297)
\curveto(274.74698955,398.21210289)(274.5969897,398.28210282)(274.45699341,398.37210297)
\lineto(274.27699341,398.49210297)
\curveto(274.22699007,398.53210257)(274.17699012,398.57710252)(274.12699341,398.62710297)
\lineto(274.03699341,398.68710297)
\curveto(274.00699029,398.70710239)(273.97699032,398.72210238)(273.94699341,398.73210297)
\curveto(273.85699044,398.76210234)(273.80199049,398.74210236)(273.78199341,398.67210297)
\curveto(273.73199056,398.6021025)(273.6969906,398.51710258)(273.67699341,398.41710297)
\curveto(273.66699063,398.32710277)(273.63199066,398.25710284)(273.57199341,398.20710297)
\curveto(273.51199078,398.16710293)(273.44199085,398.14210296)(273.36199341,398.13210297)
\lineto(273.09199341,398.13210297)
\lineto(272.37199341,398.13210297)
\lineto(272.14699341,398.13210297)
\curveto(272.07699222,398.12210298)(272.01199228,398.12710297)(271.95199341,398.14710297)
\curveto(271.81199248,398.1971029)(271.73199256,398.28710281)(271.71199341,398.41710297)
\curveto(271.70199259,398.55710254)(271.6969926,398.71210239)(271.69699341,398.88210297)
\lineto(271.69699341,408.03210297)
\lineto(271.69699341,408.37710297)
\curveto(271.6969926,408.4970926)(271.72199257,408.59209251)(271.77199341,408.66210297)
\curveto(271.81199248,408.73209237)(271.88199241,408.77709232)(271.98199341,408.79710297)
\curveto(272.00199229,408.80709229)(272.02199227,408.80709229)(272.04199341,408.79710297)
\curveto(272.07199222,408.7970923)(272.0969922,408.8020923)(272.11699341,408.81210297)
\lineto(273.06199341,408.81210297)
\curveto(273.24199105,408.81209229)(273.3969909,408.8020923)(273.52699341,408.78210297)
\curveto(273.65699064,408.77209233)(273.74199055,408.6970924)(273.78199341,408.55710297)
\curveto(273.81199048,408.45709264)(273.82199047,408.32209278)(273.81199341,408.15210297)
\curveto(273.80199049,407.99209311)(273.7969905,407.85209325)(273.79699341,407.73210297)
\lineto(273.79699341,406.09710297)
\lineto(273.79699341,405.76710297)
\curveto(273.7969905,405.65709544)(273.80699049,405.56209554)(273.82699341,405.48210297)
\curveto(273.83699046,405.43209567)(273.84699045,405.38709571)(273.85699341,405.34710297)
\curveto(273.86699043,405.31709578)(273.8919904,405.2970958)(273.93199341,405.28710297)
\curveto(273.95199034,405.26709583)(273.97699032,405.25709584)(274.00699341,405.25710297)
\curveto(274.04699025,405.25709584)(274.07699022,405.26209584)(274.09699341,405.27210297)
\curveto(274.16699013,405.31209579)(274.23199006,405.35209575)(274.29199341,405.39210297)
\curveto(274.35198994,405.44209566)(274.41698988,405.49209561)(274.48699341,405.54210297)
\curveto(274.61698968,405.63209547)(274.75198954,405.70709539)(274.89199341,405.76710297)
\curveto(275.03198926,405.83709526)(275.18698911,405.8970952)(275.35699341,405.94710297)
\curveto(275.43698886,405.97709512)(275.51698878,405.99209511)(275.59699341,405.99210297)
\curveto(275.67698862,406.0020951)(275.75698854,406.01709508)(275.83699341,406.03710297)
\curveto(275.90698839,406.05709504)(275.98198831,406.06709503)(276.06199341,406.06710297)
\lineto(276.30199341,406.06710297)
\lineto(276.45199341,406.06710297)
\curveto(276.48198781,406.05709504)(276.51698778,406.05209505)(276.55699341,406.05210297)
\curveto(276.5969877,406.06209504)(276.63698766,406.06209504)(276.67699341,406.05210297)
\curveto(276.78698751,406.02209508)(276.88698741,405.9970951)(276.97699341,405.97710297)
\curveto(277.07698722,405.96709513)(277.17198712,405.94209516)(277.26199341,405.90210297)
\curveto(277.72198657,405.71209539)(278.0969862,405.46709563)(278.38699341,405.16710297)
\curveto(278.67698562,404.86709623)(278.92198537,404.49209661)(279.12199341,404.04210297)
\curveto(279.17198512,403.92209718)(279.21198508,403.7970973)(279.24199341,403.66710297)
\curveto(279.28198501,403.53709756)(279.32198497,403.4020977)(279.36199341,403.26210297)
\curveto(279.38198491,403.19209791)(279.3919849,403.12209798)(279.39199341,403.05210297)
\curveto(279.40198489,402.99209811)(279.41698488,402.92209818)(279.43699341,402.84210297)
\curveto(279.45698484,402.79209831)(279.46198483,402.73709836)(279.45199341,402.67710297)
\curveto(279.45198484,402.61709848)(279.45698484,402.55709854)(279.46699341,402.49710297)
\lineto(279.46699341,402.39210297)
\moveto(277.24699341,400.98210297)
\curveto(277.27698702,401.08210002)(277.30198699,401.20709989)(277.32199341,401.35710297)
\curveto(277.35198694,401.50709959)(277.36698693,401.65709944)(277.36699341,401.80710297)
\curveto(277.37698692,401.96709913)(277.37698692,402.12209898)(277.36699341,402.27210297)
\curveto(277.36698693,402.43209867)(277.35198694,402.56709853)(277.32199341,402.67710297)
\curveto(277.291987,402.77709832)(277.27198702,402.87209823)(277.26199341,402.96210297)
\curveto(277.25198704,403.05209805)(277.22698707,403.13709796)(277.18699341,403.21710297)
\curveto(277.04698725,403.56709753)(276.84698745,403.86209724)(276.58699341,404.10210297)
\curveto(276.33698796,404.35209675)(275.96698833,404.47709662)(275.47699341,404.47710297)
\curveto(275.43698886,404.47709662)(275.40198889,404.47209663)(275.37199341,404.46210297)
\lineto(275.26699341,404.46210297)
\curveto(275.1969891,404.44209666)(275.13198916,404.42209668)(275.07199341,404.40210297)
\curveto(275.01198928,404.39209671)(274.95198934,404.37709672)(274.89199341,404.35710297)
\curveto(274.60198969,404.22709687)(274.38198991,404.04209706)(274.23199341,403.80210297)
\curveto(274.08199021,403.57209753)(273.95699034,403.30709779)(273.85699341,403.00710297)
\curveto(273.82699047,402.92709817)(273.80699049,402.84209826)(273.79699341,402.75210297)
\curveto(273.7969905,402.67209843)(273.78699051,402.59209851)(273.76699341,402.51210297)
\curveto(273.75699054,402.48209862)(273.75199054,402.43209867)(273.75199341,402.36210297)
\curveto(273.74199055,402.32209878)(273.73699056,402.28209882)(273.73699341,402.24210297)
\curveto(273.74699055,402.2020989)(273.74699055,402.16209894)(273.73699341,402.12210297)
\curveto(273.71699058,402.04209906)(273.71199058,401.93209917)(273.72199341,401.79210297)
\curveto(273.73199056,401.65209945)(273.74699055,401.55209955)(273.76699341,401.49210297)
\curveto(273.78699051,401.4020997)(273.7969905,401.31709978)(273.79699341,401.23710297)
\curveto(273.80699049,401.15709994)(273.82699047,401.07710002)(273.85699341,400.99710297)
\curveto(273.94699035,400.71710038)(274.05199024,400.47210063)(274.17199341,400.26210297)
\curveto(274.30198999,400.06210104)(274.48198981,399.89210121)(274.71199341,399.75210297)
\curveto(274.87198942,399.65210145)(275.03698926,399.58210152)(275.20699341,399.54210297)
\curveto(275.22698907,399.54210156)(275.24698905,399.53710156)(275.26699341,399.52710297)
\lineto(275.35699341,399.52710297)
\curveto(275.38698891,399.51710158)(275.43698886,399.50710159)(275.50699341,399.49710297)
\curveto(275.57698872,399.4971016)(275.63698866,399.5021016)(275.68699341,399.51210297)
\curveto(275.78698851,399.53210157)(275.87698842,399.54710155)(275.95699341,399.55710297)
\curveto(276.04698825,399.57710152)(276.13198816,399.6021015)(276.21199341,399.63210297)
\curveto(276.4919878,399.76210134)(276.70698759,399.94210116)(276.85699341,400.17210297)
\curveto(277.01698728,400.4021007)(277.14698715,400.67210043)(277.24699341,400.98210297)
}
}
{
\newrgbcolor{curcolor}{0 0 0}
\pscustom[linestyle=none,fillstyle=solid,fillcolor=curcolor]
{
\newpath
\moveto(281.34691528,408.82710297)
\lineto(282.44191528,408.82710297)
\curveto(282.5419128,408.82709227)(282.6369127,408.82209228)(282.72691528,408.81210297)
\curveto(282.81691252,408.8020923)(282.88691245,408.77209233)(282.93691528,408.72210297)
\curveto(282.99691234,408.65209245)(283.02691231,408.55709254)(283.02691528,408.43710297)
\curveto(283.0369123,408.32709277)(283.0419123,408.21209289)(283.04191528,408.09210297)
\lineto(283.04191528,406.75710297)
\lineto(283.04191528,401.37210297)
\lineto(283.04191528,399.07710297)
\lineto(283.04191528,398.65710297)
\curveto(283.05191229,398.50710259)(283.03191231,398.39210271)(282.98191528,398.31210297)
\curveto(282.93191241,398.23210287)(282.8419125,398.17710292)(282.71191528,398.14710297)
\curveto(282.65191269,398.12710297)(282.58191276,398.12210298)(282.50191528,398.13210297)
\curveto(282.43191291,398.14210296)(282.36191298,398.14710295)(282.29191528,398.14710297)
\lineto(281.57191528,398.14710297)
\curveto(281.46191388,398.14710295)(281.36191398,398.15210295)(281.27191528,398.16210297)
\curveto(281.18191416,398.17210293)(281.10691423,398.2021029)(281.04691528,398.25210297)
\curveto(280.98691435,398.3021028)(280.95191439,398.37710272)(280.94191528,398.47710297)
\lineto(280.94191528,398.80710297)
\lineto(280.94191528,400.14210297)
\lineto(280.94191528,405.76710297)
\lineto(280.94191528,407.80710297)
\curveto(280.9419144,407.93709316)(280.9369144,408.09209301)(280.92691528,408.27210297)
\curveto(280.92691441,408.45209265)(280.95191439,408.58209252)(281.00191528,408.66210297)
\curveto(281.02191432,408.7020924)(281.04691429,408.73209237)(281.07691528,408.75210297)
\lineto(281.19691528,408.81210297)
\curveto(281.21691412,408.81209229)(281.2419141,408.81209229)(281.27191528,408.81210297)
\curveto(281.30191404,408.82209228)(281.32691401,408.82709227)(281.34691528,408.82710297)
}
}
{
\newrgbcolor{curcolor}{0 0 0}
\pscustom[linestyle=none,fillstyle=solid,fillcolor=curcolor]
{
\newpath
\moveto(286.80410278,408.72210297)
\curveto(286.87409983,408.64209246)(286.9090998,408.52209258)(286.90910278,408.36210297)
\lineto(286.90910278,407.89710297)
\lineto(286.90910278,407.49210297)
\curveto(286.9090998,407.35209375)(286.87409983,407.25709384)(286.80410278,407.20710297)
\curveto(286.74409996,407.15709394)(286.66410004,407.12709397)(286.56410278,407.11710297)
\curveto(286.47410023,407.10709399)(286.37410033,407.102094)(286.26410278,407.10210297)
\lineto(285.42410278,407.10210297)
\curveto(285.31410139,407.102094)(285.21410149,407.10709399)(285.12410278,407.11710297)
\curveto(285.04410166,407.12709397)(284.97410173,407.15709394)(284.91410278,407.20710297)
\curveto(284.87410183,407.23709386)(284.84410186,407.29209381)(284.82410278,407.37210297)
\curveto(284.81410189,407.46209364)(284.8041019,407.55709354)(284.79410278,407.65710297)
\lineto(284.79410278,407.98710297)
\curveto(284.8041019,408.097093)(284.8091019,408.19209291)(284.80910278,408.27210297)
\lineto(284.80910278,408.48210297)
\curveto(284.81910189,408.55209255)(284.83910187,408.61209249)(284.86910278,408.66210297)
\curveto(284.88910182,408.7020924)(284.91410179,408.73209237)(284.94410278,408.75210297)
\lineto(285.06410278,408.81210297)
\curveto(285.08410162,408.81209229)(285.1091016,408.81209229)(285.13910278,408.81210297)
\curveto(285.16910154,408.82209228)(285.19410151,408.82709227)(285.21410278,408.82710297)
\lineto(286.30910278,408.82710297)
\curveto(286.4091003,408.82709227)(286.5041002,408.82209228)(286.59410278,408.81210297)
\curveto(286.68410002,408.8020923)(286.75409995,408.77209233)(286.80410278,408.72210297)
\moveto(286.90910278,398.95710297)
\curveto(286.9090998,398.75710234)(286.9040998,398.58710251)(286.89410278,398.44710297)
\curveto(286.88409982,398.30710279)(286.79409991,398.21210289)(286.62410278,398.16210297)
\curveto(286.56410014,398.14210296)(286.49910021,398.13210297)(286.42910278,398.13210297)
\curveto(286.35910035,398.14210296)(286.28410042,398.14710295)(286.20410278,398.14710297)
\lineto(285.36410278,398.14710297)
\curveto(285.27410143,398.14710295)(285.18410152,398.15210295)(285.09410278,398.16210297)
\curveto(285.01410169,398.17210293)(284.95410175,398.2021029)(284.91410278,398.25210297)
\curveto(284.85410185,398.32210278)(284.81910189,398.40710269)(284.80910278,398.50710297)
\lineto(284.80910278,398.85210297)
\lineto(284.80910278,405.18210297)
\lineto(284.80910278,405.48210297)
\curveto(284.8091019,405.58209552)(284.82910188,405.66209544)(284.86910278,405.72210297)
\curveto(284.92910178,405.79209531)(285.01410169,405.83709526)(285.12410278,405.85710297)
\curveto(285.14410156,405.86709523)(285.16910154,405.86709523)(285.19910278,405.85710297)
\curveto(285.23910147,405.85709524)(285.26910144,405.86209524)(285.28910278,405.87210297)
\lineto(286.03910278,405.87210297)
\lineto(286.23410278,405.87210297)
\curveto(286.31410039,405.88209522)(286.37910033,405.88209522)(286.42910278,405.87210297)
\lineto(286.54910278,405.87210297)
\curveto(286.6091001,405.85209525)(286.66410004,405.83709526)(286.71410278,405.82710297)
\curveto(286.76409994,405.81709528)(286.8040999,405.78709531)(286.83410278,405.73710297)
\curveto(286.87409983,405.68709541)(286.89409981,405.61709548)(286.89410278,405.52710297)
\curveto(286.9040998,405.43709566)(286.9090998,405.34209576)(286.90910278,405.24210297)
\lineto(286.90910278,398.95710297)
}
}
{
\newrgbcolor{curcolor}{0 0 0}
\pscustom[linestyle=none,fillstyle=solid,fillcolor=curcolor]
{
\newpath
\moveto(292.14129028,406.08210297)
\curveto(292.95128512,406.102095)(293.62628445,405.98209512)(294.16629028,405.72210297)
\curveto(294.71628336,405.46209564)(295.15128292,405.09209601)(295.47129028,404.61210297)
\curveto(295.63128244,404.37209673)(295.75128232,404.097097)(295.83129028,403.78710297)
\curveto(295.85128222,403.73709736)(295.86628221,403.67209743)(295.87629028,403.59210297)
\curveto(295.89628218,403.51209759)(295.89628218,403.44209766)(295.87629028,403.38210297)
\curveto(295.83628224,403.27209783)(295.76628231,403.20709789)(295.66629028,403.18710297)
\curveto(295.56628251,403.17709792)(295.44628263,403.17209793)(295.30629028,403.17210297)
\lineto(294.52629028,403.17210297)
\lineto(294.24129028,403.17210297)
\curveto(294.15128392,403.17209793)(294.076284,403.19209791)(294.01629028,403.23210297)
\curveto(293.93628414,403.27209783)(293.88128419,403.33209777)(293.85129028,403.41210297)
\curveto(293.82128425,403.5020976)(293.78128429,403.59209751)(293.73129028,403.68210297)
\curveto(293.6712844,403.79209731)(293.60628447,403.89209721)(293.53629028,403.98210297)
\curveto(293.46628461,404.07209703)(293.38628469,404.15209695)(293.29629028,404.22210297)
\curveto(293.15628492,404.31209679)(293.00128507,404.38209672)(292.83129028,404.43210297)
\curveto(292.7712853,404.45209665)(292.71128536,404.46209664)(292.65129028,404.46210297)
\curveto(292.59128548,404.46209664)(292.53628554,404.47209663)(292.48629028,404.49210297)
\lineto(292.33629028,404.49210297)
\curveto(292.13628594,404.49209661)(291.9762861,404.47209663)(291.85629028,404.43210297)
\curveto(291.56628651,404.34209676)(291.33128674,404.2020969)(291.15129028,404.01210297)
\curveto(290.9712871,403.83209727)(290.82628725,403.61209749)(290.71629028,403.35210297)
\curveto(290.66628741,403.24209786)(290.62628745,403.12209798)(290.59629028,402.99210297)
\curveto(290.5762875,402.87209823)(290.55128752,402.74209836)(290.52129028,402.60210297)
\curveto(290.51128756,402.56209854)(290.50628757,402.52209858)(290.50629028,402.48210297)
\curveto(290.50628757,402.44209866)(290.50128757,402.4020987)(290.49129028,402.36210297)
\curveto(290.4712876,402.26209884)(290.46128761,402.12209898)(290.46129028,401.94210297)
\curveto(290.4712876,401.76209934)(290.48628759,401.62209948)(290.50629028,401.52210297)
\curveto(290.50628757,401.44209966)(290.51128756,401.38709971)(290.52129028,401.35710297)
\curveto(290.54128753,401.28709981)(290.55128752,401.21709988)(290.55129028,401.14710297)
\curveto(290.56128751,401.07710002)(290.5762875,401.00710009)(290.59629028,400.93710297)
\curveto(290.6762874,400.70710039)(290.7712873,400.4971006)(290.88129028,400.30710297)
\curveto(290.99128708,400.11710098)(291.13128694,399.95710114)(291.30129028,399.82710297)
\curveto(291.34128673,399.7971013)(291.40128667,399.76210134)(291.48129028,399.72210297)
\curveto(291.59128648,399.65210145)(291.70128637,399.60710149)(291.81129028,399.58710297)
\curveto(291.93128614,399.56710153)(292.076286,399.54710155)(292.24629028,399.52710297)
\lineto(292.33629028,399.52710297)
\curveto(292.3762857,399.52710157)(292.40628567,399.53210157)(292.42629028,399.54210297)
\lineto(292.56129028,399.54210297)
\curveto(292.63128544,399.56210154)(292.69628538,399.57710152)(292.75629028,399.58710297)
\curveto(292.82628525,399.60710149)(292.89128518,399.62710147)(292.95129028,399.64710297)
\curveto(293.25128482,399.77710132)(293.48128459,399.96710113)(293.64129028,400.21710297)
\curveto(293.68128439,400.26710083)(293.71628436,400.32210078)(293.74629028,400.38210297)
\curveto(293.7762843,400.45210065)(293.80128427,400.51210059)(293.82129028,400.56210297)
\curveto(293.86128421,400.67210043)(293.89628418,400.76710033)(293.92629028,400.84710297)
\curveto(293.95628412,400.93710016)(294.02628405,401.00710009)(294.13629028,401.05710297)
\curveto(294.22628385,401.0971)(294.3712837,401.11209999)(294.57129028,401.10210297)
\lineto(295.06629028,401.10210297)
\lineto(295.27629028,401.10210297)
\curveto(295.35628272,401.11209999)(295.42128265,401.10709999)(295.47129028,401.08710297)
\lineto(295.59129028,401.08710297)
\lineto(295.71129028,401.05710297)
\curveto(295.75128232,401.05710004)(295.78128229,401.04710005)(295.80129028,401.02710297)
\curveto(295.85128222,400.98710011)(295.88128219,400.92710017)(295.89129028,400.84710297)
\curveto(295.91128216,400.77710032)(295.91128216,400.7021004)(295.89129028,400.62210297)
\curveto(295.80128227,400.29210081)(295.69128238,399.9971011)(295.56129028,399.73710297)
\curveto(295.15128292,398.96710213)(294.49628358,398.43210267)(293.59629028,398.13210297)
\curveto(293.49628458,398.102103)(293.39128468,398.08210302)(293.28129028,398.07210297)
\curveto(293.1712849,398.05210305)(293.06128501,398.02710307)(292.95129028,397.99710297)
\curveto(292.89128518,397.98710311)(292.83128524,397.98210312)(292.77129028,397.98210297)
\curveto(292.71128536,397.98210312)(292.65128542,397.97710312)(292.59129028,397.96710297)
\lineto(292.42629028,397.96710297)
\curveto(292.3762857,397.94710315)(292.30128577,397.94210316)(292.20129028,397.95210297)
\curveto(292.10128597,397.95210315)(292.02628605,397.95710314)(291.97629028,397.96710297)
\curveto(291.89628618,397.98710311)(291.82128625,397.9971031)(291.75129028,397.99710297)
\curveto(291.69128638,397.98710311)(291.62628645,397.99210311)(291.55629028,398.01210297)
\lineto(291.40629028,398.04210297)
\curveto(291.35628672,398.04210306)(291.30628677,398.04710305)(291.25629028,398.05710297)
\curveto(291.14628693,398.08710301)(291.04128703,398.11710298)(290.94129028,398.14710297)
\curveto(290.84128723,398.17710292)(290.74628733,398.21210289)(290.65629028,398.25210297)
\curveto(290.18628789,398.45210265)(289.79128828,398.70710239)(289.47129028,399.01710297)
\curveto(289.15128892,399.33710176)(288.89128918,399.73210137)(288.69129028,400.20210297)
\curveto(288.64128943,400.29210081)(288.60128947,400.38710071)(288.57129028,400.48710297)
\lineto(288.48129028,400.81710297)
\curveto(288.4712896,400.85710024)(288.46628961,400.89210021)(288.46629028,400.92210297)
\curveto(288.46628961,400.96210014)(288.45628962,401.00710009)(288.43629028,401.05710297)
\curveto(288.41628966,401.12709997)(288.40628967,401.1970999)(288.40629028,401.26710297)
\curveto(288.40628967,401.34709975)(288.39628968,401.42209968)(288.37629028,401.49210297)
\lineto(288.37629028,401.74710297)
\curveto(288.35628972,401.7970993)(288.34628973,401.85209925)(288.34629028,401.91210297)
\curveto(288.34628973,401.98209912)(288.35628972,402.04209906)(288.37629028,402.09210297)
\curveto(288.38628969,402.14209896)(288.38628969,402.18709891)(288.37629028,402.22710297)
\curveto(288.36628971,402.26709883)(288.36628971,402.30709879)(288.37629028,402.34710297)
\curveto(288.39628968,402.41709868)(288.40128967,402.48209862)(288.39129028,402.54210297)
\curveto(288.39128968,402.6020985)(288.40128967,402.66209844)(288.42129028,402.72210297)
\curveto(288.4712896,402.9020982)(288.51128956,403.07209803)(288.54129028,403.23210297)
\curveto(288.5712895,403.4020977)(288.61628946,403.56709753)(288.67629028,403.72710297)
\curveto(288.89628918,404.23709686)(289.1712889,404.66209644)(289.50129028,405.00210297)
\curveto(289.84128823,405.34209576)(290.2712878,405.61709548)(290.79129028,405.82710297)
\curveto(290.93128714,405.88709521)(291.076287,405.92709517)(291.22629028,405.94710297)
\curveto(291.3762867,405.97709512)(291.53128654,406.01209509)(291.69129028,406.05210297)
\curveto(291.7712863,406.06209504)(291.84628623,406.06709503)(291.91629028,406.06710297)
\curveto(291.98628609,406.06709503)(292.06128601,406.07209503)(292.14129028,406.08210297)
}
}
{
\newrgbcolor{curcolor}{0 0 0}
\pscustom[linestyle=none,fillstyle=solid,fillcolor=curcolor]
{
\newpath
\moveto(304.23457153,398.73210297)
\curveto(304.25456368,398.62210248)(304.26456367,398.51210259)(304.26457153,398.40210297)
\curveto(304.27456366,398.29210281)(304.22456371,398.21710288)(304.11457153,398.17710297)
\curveto(304.05456388,398.14710295)(303.98456395,398.13210297)(303.90457153,398.13210297)
\lineto(303.66457153,398.13210297)
\lineto(302.85457153,398.13210297)
\lineto(302.58457153,398.13210297)
\curveto(302.50456543,398.14210296)(302.4395655,398.16710293)(302.38957153,398.20710297)
\curveto(302.31956562,398.24710285)(302.26456567,398.3021028)(302.22457153,398.37210297)
\curveto(302.19456574,398.45210265)(302.14956579,398.51710258)(302.08957153,398.56710297)
\curveto(302.06956587,398.58710251)(302.04456589,398.6021025)(302.01457153,398.61210297)
\curveto(301.98456595,398.63210247)(301.94456599,398.63710246)(301.89457153,398.62710297)
\curveto(301.84456609,398.60710249)(301.79456614,398.58210252)(301.74457153,398.55210297)
\curveto(301.70456623,398.52210258)(301.65956628,398.4971026)(301.60957153,398.47710297)
\curveto(301.55956638,398.43710266)(301.50456643,398.4021027)(301.44457153,398.37210297)
\lineto(301.26457153,398.28210297)
\curveto(301.1345668,398.22210288)(300.99956694,398.17210293)(300.85957153,398.13210297)
\curveto(300.71956722,398.102103)(300.57456736,398.06710303)(300.42457153,398.02710297)
\curveto(300.35456758,398.00710309)(300.28456765,397.9971031)(300.21457153,397.99710297)
\curveto(300.15456778,397.98710311)(300.08956785,397.97710312)(300.01957153,397.96710297)
\lineto(299.92957153,397.96710297)
\curveto(299.89956804,397.95710314)(299.86956807,397.95210315)(299.83957153,397.95210297)
\lineto(299.67457153,397.95210297)
\curveto(299.57456836,397.93210317)(299.47456846,397.93210317)(299.37457153,397.95210297)
\lineto(299.23957153,397.95210297)
\curveto(299.16956877,397.97210313)(299.09956884,397.98210312)(299.02957153,397.98210297)
\curveto(298.96956897,397.97210313)(298.90956903,397.97710312)(298.84957153,397.99710297)
\curveto(298.74956919,398.01710308)(298.65456928,398.03710306)(298.56457153,398.05710297)
\curveto(298.47456946,398.06710303)(298.38956955,398.09210301)(298.30957153,398.13210297)
\curveto(298.01956992,398.24210286)(297.76957017,398.38210272)(297.55957153,398.55210297)
\curveto(297.35957058,398.73210237)(297.19957074,398.96710213)(297.07957153,399.25710297)
\curveto(297.04957089,399.32710177)(297.01957092,399.4021017)(296.98957153,399.48210297)
\curveto(296.96957097,399.56210154)(296.94957099,399.64710145)(296.92957153,399.73710297)
\curveto(296.90957103,399.78710131)(296.89957104,399.83710126)(296.89957153,399.88710297)
\curveto(296.90957103,399.93710116)(296.90957103,399.98710111)(296.89957153,400.03710297)
\curveto(296.88957105,400.06710103)(296.87957106,400.12710097)(296.86957153,400.21710297)
\curveto(296.86957107,400.31710078)(296.87457106,400.38710071)(296.88457153,400.42710297)
\curveto(296.90457103,400.52710057)(296.91457102,400.61210049)(296.91457153,400.68210297)
\lineto(297.00457153,401.01210297)
\curveto(297.0345709,401.13209997)(297.07457086,401.23709986)(297.12457153,401.32710297)
\curveto(297.29457064,401.61709948)(297.48957045,401.83709926)(297.70957153,401.98710297)
\curveto(297.92957001,402.13709896)(298.20956973,402.26709883)(298.54957153,402.37710297)
\curveto(298.67956926,402.42709867)(298.81456912,402.46209864)(298.95457153,402.48210297)
\curveto(299.09456884,402.5020986)(299.2345687,402.52709857)(299.37457153,402.55710297)
\curveto(299.45456848,402.57709852)(299.5395684,402.58709851)(299.62957153,402.58710297)
\curveto(299.71956822,402.5970985)(299.80956813,402.61209849)(299.89957153,402.63210297)
\curveto(299.96956797,402.65209845)(300.0395679,402.65709844)(300.10957153,402.64710297)
\curveto(300.17956776,402.64709845)(300.25456768,402.65709844)(300.33457153,402.67710297)
\curveto(300.40456753,402.6970984)(300.47456746,402.70709839)(300.54457153,402.70710297)
\curveto(300.61456732,402.70709839)(300.68956725,402.71709838)(300.76957153,402.73710297)
\curveto(300.97956696,402.78709831)(301.16956677,402.82709827)(301.33957153,402.85710297)
\curveto(301.51956642,402.8970982)(301.67956626,402.98709811)(301.81957153,403.12710297)
\curveto(301.90956603,403.21709788)(301.96956597,403.31709778)(301.99957153,403.42710297)
\curveto(302.00956593,403.45709764)(302.00956593,403.48209762)(301.99957153,403.50210297)
\curveto(301.99956594,403.52209758)(302.00456593,403.54209756)(302.01457153,403.56210297)
\curveto(302.02456591,403.58209752)(302.02956591,403.61209749)(302.02957153,403.65210297)
\lineto(302.02957153,403.74210297)
\lineto(301.99957153,403.86210297)
\curveto(301.99956594,403.9020972)(301.99456594,403.93709716)(301.98457153,403.96710297)
\curveto(301.88456605,404.26709683)(301.67456626,404.47209663)(301.35457153,404.58210297)
\curveto(301.26456667,404.61209649)(301.15456678,404.63209647)(301.02457153,404.64210297)
\curveto(300.90456703,404.66209644)(300.77956716,404.66709643)(300.64957153,404.65710297)
\curveto(300.51956742,404.65709644)(300.39456754,404.64709645)(300.27457153,404.62710297)
\curveto(300.15456778,404.60709649)(300.04956789,404.58209652)(299.95957153,404.55210297)
\curveto(299.89956804,404.53209657)(299.8395681,404.5020966)(299.77957153,404.46210297)
\curveto(299.72956821,404.43209667)(299.67956826,404.3970967)(299.62957153,404.35710297)
\curveto(299.57956836,404.31709678)(299.52456841,404.26209684)(299.46457153,404.19210297)
\curveto(299.41456852,404.12209698)(299.37956856,404.05709704)(299.35957153,403.99710297)
\curveto(299.30956863,403.8970972)(299.26456867,403.8020973)(299.22457153,403.71210297)
\curveto(299.19456874,403.62209748)(299.12456881,403.56209754)(299.01457153,403.53210297)
\curveto(298.934569,403.51209759)(298.84956909,403.5020976)(298.75957153,403.50210297)
\lineto(298.48957153,403.50210297)
\lineto(297.91957153,403.50210297)
\curveto(297.86957007,403.5020976)(297.81957012,403.4970976)(297.76957153,403.48710297)
\curveto(297.71957022,403.48709761)(297.67457026,403.49209761)(297.63457153,403.50210297)
\lineto(297.49957153,403.50210297)
\curveto(297.47957046,403.51209759)(297.45457048,403.51709758)(297.42457153,403.51710297)
\curveto(297.39457054,403.51709758)(297.36957057,403.52709757)(297.34957153,403.54710297)
\curveto(297.26957067,403.56709753)(297.21457072,403.63209747)(297.18457153,403.74210297)
\curveto(297.17457076,403.79209731)(297.17457076,403.84209726)(297.18457153,403.89210297)
\curveto(297.19457074,403.94209716)(297.20457073,403.98709711)(297.21457153,404.02710297)
\curveto(297.24457069,404.13709696)(297.27457066,404.23709686)(297.30457153,404.32710297)
\curveto(297.34457059,404.42709667)(297.38957055,404.51709658)(297.43957153,404.59710297)
\lineto(297.52957153,404.74710297)
\lineto(297.61957153,404.89710297)
\curveto(297.69957024,405.00709609)(297.79957014,405.11209599)(297.91957153,405.21210297)
\curveto(297.93957,405.22209588)(297.96956997,405.24709585)(298.00957153,405.28710297)
\curveto(298.05956988,405.32709577)(298.10456983,405.36209574)(298.14457153,405.39210297)
\curveto(298.18456975,405.42209568)(298.22956971,405.45209565)(298.27957153,405.48210297)
\curveto(298.44956949,405.59209551)(298.62956931,405.67709542)(298.81957153,405.73710297)
\curveto(299.00956893,405.80709529)(299.20456873,405.87209523)(299.40457153,405.93210297)
\curveto(299.52456841,405.96209514)(299.64956829,405.98209512)(299.77957153,405.99210297)
\curveto(299.90956803,406.0020951)(300.0395679,406.02209508)(300.16957153,406.05210297)
\curveto(300.20956773,406.06209504)(300.26956767,406.06209504)(300.34957153,406.05210297)
\curveto(300.4395675,406.04209506)(300.49456744,406.04709505)(300.51457153,406.06710297)
\curveto(300.92456701,406.07709502)(301.31456662,406.06209504)(301.68457153,406.02210297)
\curveto(302.06456587,405.98209512)(302.40456553,405.90709519)(302.70457153,405.79710297)
\curveto(303.01456492,405.68709541)(303.27956466,405.53709556)(303.49957153,405.34710297)
\curveto(303.71956422,405.16709593)(303.88956405,404.93209617)(304.00957153,404.64210297)
\curveto(304.07956386,404.47209663)(304.11956382,404.27709682)(304.12957153,404.05710297)
\curveto(304.1395638,403.83709726)(304.14456379,403.61209749)(304.14457153,403.38210297)
\lineto(304.14457153,400.03710297)
\lineto(304.14457153,399.45210297)
\curveto(304.14456379,399.26210184)(304.16456377,399.08710201)(304.20457153,398.92710297)
\curveto(304.21456372,398.8971022)(304.21956372,398.86210224)(304.21957153,398.82210297)
\curveto(304.21956372,398.79210231)(304.22456371,398.76210234)(304.23457153,398.73210297)
\moveto(302.02957153,401.04210297)
\curveto(302.0395659,401.09210001)(302.04456589,401.14709995)(302.04457153,401.20710297)
\curveto(302.04456589,401.27709982)(302.0395659,401.33709976)(302.02957153,401.38710297)
\curveto(302.00956593,401.44709965)(301.99956594,401.5020996)(301.99957153,401.55210297)
\curveto(301.99956594,401.6020995)(301.97956596,401.64209946)(301.93957153,401.67210297)
\curveto(301.88956605,401.71209939)(301.81456612,401.73209937)(301.71457153,401.73210297)
\curveto(301.67456626,401.72209938)(301.6395663,401.71209939)(301.60957153,401.70210297)
\curveto(301.57956636,401.7020994)(301.54456639,401.6970994)(301.50457153,401.68710297)
\curveto(301.4345665,401.66709943)(301.35956658,401.65209945)(301.27957153,401.64210297)
\curveto(301.19956674,401.63209947)(301.11956682,401.61709948)(301.03957153,401.59710297)
\curveto(301.00956693,401.58709951)(300.96456697,401.58209952)(300.90457153,401.58210297)
\curveto(300.77456716,401.55209955)(300.64456729,401.53209957)(300.51457153,401.52210297)
\curveto(300.38456755,401.51209959)(300.25956768,401.48709961)(300.13957153,401.44710297)
\curveto(300.05956788,401.42709967)(299.98456795,401.40709969)(299.91457153,401.38710297)
\curveto(299.84456809,401.37709972)(299.77456816,401.35709974)(299.70457153,401.32710297)
\curveto(299.49456844,401.23709986)(299.31456862,401.1021)(299.16457153,400.92210297)
\curveto(299.02456891,400.74210036)(298.97456896,400.49210061)(299.01457153,400.17210297)
\curveto(299.0345689,400.0021011)(299.08956885,399.86210124)(299.17957153,399.75210297)
\curveto(299.24956869,399.64210146)(299.35456858,399.55210155)(299.49457153,399.48210297)
\curveto(299.6345683,399.42210168)(299.78456815,399.37710172)(299.94457153,399.34710297)
\curveto(300.11456782,399.31710178)(300.28956765,399.30710179)(300.46957153,399.31710297)
\curveto(300.65956728,399.33710176)(300.8345671,399.37210173)(300.99457153,399.42210297)
\curveto(301.25456668,399.5021016)(301.45956648,399.62710147)(301.60957153,399.79710297)
\curveto(301.75956618,399.97710112)(301.87456606,400.1971009)(301.95457153,400.45710297)
\curveto(301.97456596,400.52710057)(301.98456595,400.5971005)(301.98457153,400.66710297)
\curveto(301.99456594,400.74710035)(302.00956593,400.82710027)(302.02957153,400.90710297)
\lineto(302.02957153,401.04210297)
}
}
{
\newrgbcolor{curcolor}{0 0 0}
\pscustom[linestyle=none,fillstyle=solid,fillcolor=curcolor]
{
\newpath
\moveto(313.38785278,398.98710297)
\lineto(313.38785278,398.56710297)
\curveto(313.38784441,398.43710266)(313.35784444,398.33210277)(313.29785278,398.25210297)
\curveto(313.24784455,398.2021029)(313.18284462,398.16710293)(313.10285278,398.14710297)
\curveto(313.02284478,398.13710296)(312.93284487,398.13210297)(312.83285278,398.13210297)
\lineto(312.00785278,398.13210297)
\lineto(311.72285278,398.13210297)
\curveto(311.64284616,398.14210296)(311.57784622,398.16710293)(311.52785278,398.20710297)
\curveto(311.45784634,398.25710284)(311.41784638,398.32210278)(311.40785278,398.40210297)
\curveto(311.3978464,398.48210262)(311.37784642,398.56210254)(311.34785278,398.64210297)
\curveto(311.32784647,398.66210244)(311.30784649,398.67710242)(311.28785278,398.68710297)
\curveto(311.27784652,398.70710239)(311.26284654,398.72710237)(311.24285278,398.74710297)
\curveto(311.13284667,398.74710235)(311.05284675,398.72210238)(311.00285278,398.67210297)
\lineto(310.85285278,398.52210297)
\curveto(310.78284702,398.47210263)(310.71784708,398.42710267)(310.65785278,398.38710297)
\curveto(310.5978472,398.35710274)(310.53284727,398.31710278)(310.46285278,398.26710297)
\curveto(310.42284738,398.24710285)(310.37784742,398.22710287)(310.32785278,398.20710297)
\curveto(310.28784751,398.18710291)(310.24284756,398.16710293)(310.19285278,398.14710297)
\curveto(310.05284775,398.097103)(309.9028479,398.05210305)(309.74285278,398.01210297)
\curveto(309.69284811,397.99210311)(309.64784815,397.98210312)(309.60785278,397.98210297)
\curveto(309.56784823,397.98210312)(309.52784827,397.97710312)(309.48785278,397.96710297)
\lineto(309.35285278,397.96710297)
\curveto(309.32284848,397.95710314)(309.28284852,397.95210315)(309.23285278,397.95210297)
\lineto(309.09785278,397.95210297)
\curveto(309.03784876,397.93210317)(308.94784885,397.92710317)(308.82785278,397.93710297)
\curveto(308.70784909,397.93710316)(308.62284918,397.94710315)(308.57285278,397.96710297)
\curveto(308.5028493,397.98710311)(308.43784936,397.9971031)(308.37785278,397.99710297)
\curveto(308.32784947,397.98710311)(308.27284953,397.99210311)(308.21285278,398.01210297)
\lineto(307.85285278,398.13210297)
\curveto(307.74285006,398.16210294)(307.63285017,398.2021029)(307.52285278,398.25210297)
\curveto(307.17285063,398.4021027)(306.85785094,398.63210247)(306.57785278,398.94210297)
\curveto(306.30785149,399.26210184)(306.09285171,399.5971015)(305.93285278,399.94710297)
\curveto(305.88285192,400.05710104)(305.84285196,400.16210094)(305.81285278,400.26210297)
\curveto(305.78285202,400.37210073)(305.74785205,400.48210062)(305.70785278,400.59210297)
\curveto(305.6978521,400.63210047)(305.69285211,400.66710043)(305.69285278,400.69710297)
\curveto(305.69285211,400.73710036)(305.68285212,400.78210032)(305.66285278,400.83210297)
\curveto(305.64285216,400.91210019)(305.62285218,400.9971001)(305.60285278,401.08710297)
\curveto(305.59285221,401.18709991)(305.57785222,401.28709981)(305.55785278,401.38710297)
\curveto(305.54785225,401.41709968)(305.54285226,401.45209965)(305.54285278,401.49210297)
\curveto(305.55285225,401.53209957)(305.55285225,401.56709953)(305.54285278,401.59710297)
\lineto(305.54285278,401.73210297)
\curveto(305.54285226,401.78209932)(305.53785226,401.83209927)(305.52785278,401.88210297)
\curveto(305.51785228,401.93209917)(305.51285229,401.98709911)(305.51285278,402.04710297)
\curveto(305.51285229,402.11709898)(305.51785228,402.17209893)(305.52785278,402.21210297)
\curveto(305.53785226,402.26209884)(305.54285226,402.30709879)(305.54285278,402.34710297)
\lineto(305.54285278,402.49710297)
\curveto(305.55285225,402.54709855)(305.55285225,402.59209851)(305.54285278,402.63210297)
\curveto(305.54285226,402.68209842)(305.55285225,402.73209837)(305.57285278,402.78210297)
\curveto(305.59285221,402.89209821)(305.60785219,402.9970981)(305.61785278,403.09710297)
\curveto(305.63785216,403.1970979)(305.66285214,403.2970978)(305.69285278,403.39710297)
\curveto(305.73285207,403.51709758)(305.76785203,403.63209747)(305.79785278,403.74210297)
\curveto(305.82785197,403.85209725)(305.86785193,403.96209714)(305.91785278,404.07210297)
\curveto(306.05785174,404.37209673)(306.23285157,404.65709644)(306.44285278,404.92710297)
\curveto(306.46285134,404.95709614)(306.48785131,404.98209612)(306.51785278,405.00210297)
\curveto(306.55785124,405.03209607)(306.58785121,405.06209604)(306.60785278,405.09210297)
\curveto(306.64785115,405.14209596)(306.68785111,405.18709591)(306.72785278,405.22710297)
\curveto(306.76785103,405.26709583)(306.81285099,405.30709579)(306.86285278,405.34710297)
\curveto(306.9028509,405.36709573)(306.93785086,405.39209571)(306.96785278,405.42210297)
\curveto(306.9978508,405.46209564)(307.03285077,405.49209561)(307.07285278,405.51210297)
\curveto(307.32285048,405.68209542)(307.61285019,405.82209528)(307.94285278,405.93210297)
\curveto(308.01284979,405.95209515)(308.08284972,405.96709513)(308.15285278,405.97710297)
\curveto(308.23284957,405.98709511)(308.31284949,406.0020951)(308.39285278,406.02210297)
\curveto(308.46284934,406.04209506)(308.55284925,406.05209505)(308.66285278,406.05210297)
\curveto(308.77284903,406.06209504)(308.88284892,406.06709503)(308.99285278,406.06710297)
\curveto(309.1028487,406.06709503)(309.20784859,406.06209504)(309.30785278,406.05210297)
\curveto(309.41784838,406.04209506)(309.50784829,406.02709507)(309.57785278,406.00710297)
\curveto(309.72784807,405.95709514)(309.87284793,405.91209519)(310.01285278,405.87210297)
\curveto(310.15284765,405.83209527)(310.28284752,405.77709532)(310.40285278,405.70710297)
\curveto(310.47284733,405.65709544)(310.53784726,405.60709549)(310.59785278,405.55710297)
\curveto(310.65784714,405.51709558)(310.72284708,405.47209563)(310.79285278,405.42210297)
\curveto(310.83284697,405.39209571)(310.88784691,405.35209575)(310.95785278,405.30210297)
\curveto(311.03784676,405.25209585)(311.11284669,405.25209585)(311.18285278,405.30210297)
\curveto(311.22284658,405.32209578)(311.24284656,405.35709574)(311.24285278,405.40710297)
\curveto(311.24284656,405.45709564)(311.25284655,405.50709559)(311.27285278,405.55710297)
\lineto(311.27285278,405.70710297)
\curveto(311.28284652,405.73709536)(311.28784651,405.77209533)(311.28785278,405.81210297)
\lineto(311.28785278,405.93210297)
\lineto(311.28785278,407.97210297)
\curveto(311.28784651,408.08209302)(311.28284652,408.2020929)(311.27285278,408.33210297)
\curveto(311.27284653,408.47209263)(311.2978465,408.57709252)(311.34785278,408.64710297)
\curveto(311.38784641,408.72709237)(311.46284634,408.77709232)(311.57285278,408.79710297)
\curveto(311.59284621,408.80709229)(311.61284619,408.80709229)(311.63285278,408.79710297)
\curveto(311.65284615,408.7970923)(311.67284613,408.8020923)(311.69285278,408.81210297)
\lineto(312.75785278,408.81210297)
\curveto(312.87784492,408.81209229)(312.98784481,408.80709229)(313.08785278,408.79710297)
\curveto(313.18784461,408.78709231)(313.26284454,408.74709235)(313.31285278,408.67710297)
\curveto(313.36284444,408.5970925)(313.38784441,408.49209261)(313.38785278,408.36210297)
\lineto(313.38785278,408.00210297)
\lineto(313.38785278,398.98710297)
\moveto(311.34785278,401.92710297)
\curveto(311.35784644,401.96709913)(311.35784644,402.00709909)(311.34785278,402.04710297)
\lineto(311.34785278,402.18210297)
\curveto(311.34784645,402.28209882)(311.34284646,402.38209872)(311.33285278,402.48210297)
\curveto(311.32284648,402.58209852)(311.30784649,402.67209843)(311.28785278,402.75210297)
\curveto(311.26784653,402.86209824)(311.24784655,402.96209814)(311.22785278,403.05210297)
\curveto(311.21784658,403.14209796)(311.19284661,403.22709787)(311.15285278,403.30710297)
\curveto(311.01284679,403.66709743)(310.80784699,403.95209715)(310.53785278,404.16210297)
\curveto(310.27784752,404.37209673)(309.8978479,404.47709662)(309.39785278,404.47710297)
\curveto(309.33784846,404.47709662)(309.25784854,404.46709663)(309.15785278,404.44710297)
\curveto(309.07784872,404.42709667)(309.0028488,404.40709669)(308.93285278,404.38710297)
\curveto(308.87284893,404.37709672)(308.81284899,404.35709674)(308.75285278,404.32710297)
\curveto(308.48284932,404.21709688)(308.27284953,404.04709705)(308.12285278,403.81710297)
\curveto(307.97284983,403.58709751)(307.85284995,403.32709777)(307.76285278,403.03710297)
\curveto(307.73285007,402.93709816)(307.71285009,402.83709826)(307.70285278,402.73710297)
\curveto(307.69285011,402.63709846)(307.67285013,402.53209857)(307.64285278,402.42210297)
\lineto(307.64285278,402.21210297)
\curveto(307.62285018,402.12209898)(307.61785018,401.9970991)(307.62785278,401.83710297)
\curveto(307.63785016,401.68709941)(307.65285015,401.57709952)(307.67285278,401.50710297)
\lineto(307.67285278,401.41710297)
\curveto(307.68285012,401.3970997)(307.68785011,401.37709972)(307.68785278,401.35710297)
\curveto(307.70785009,401.27709982)(307.72285008,401.2020999)(307.73285278,401.13210297)
\curveto(307.75285005,401.06210004)(307.77285003,400.98710011)(307.79285278,400.90710297)
\curveto(307.96284984,400.38710071)(308.25284955,400.0021011)(308.66285278,399.75210297)
\curveto(308.79284901,399.66210144)(308.97284883,399.59210151)(309.20285278,399.54210297)
\curveto(309.24284856,399.53210157)(309.3028485,399.52710157)(309.38285278,399.52710297)
\curveto(309.41284839,399.51710158)(309.45784834,399.50710159)(309.51785278,399.49710297)
\curveto(309.58784821,399.4971016)(309.64284816,399.5021016)(309.68285278,399.51210297)
\curveto(309.76284804,399.53210157)(309.84284796,399.54710155)(309.92285278,399.55710297)
\curveto(310.0028478,399.56710153)(310.08284772,399.58710151)(310.16285278,399.61710297)
\curveto(310.41284739,399.72710137)(310.61284719,399.86710123)(310.76285278,400.03710297)
\curveto(310.91284689,400.20710089)(311.04284676,400.42210068)(311.15285278,400.68210297)
\curveto(311.19284661,400.77210033)(311.22284658,400.86210024)(311.24285278,400.95210297)
\curveto(311.26284654,401.05210005)(311.28284652,401.15709994)(311.30285278,401.26710297)
\curveto(311.31284649,401.31709978)(311.31284649,401.36209974)(311.30285278,401.40210297)
\curveto(311.3028465,401.45209965)(311.31284649,401.5020996)(311.33285278,401.55210297)
\curveto(311.34284646,401.58209952)(311.34784645,401.61709948)(311.34785278,401.65710297)
\lineto(311.34785278,401.79210297)
\lineto(311.34785278,401.92710297)
}
}
{
\newrgbcolor{curcolor}{0 0 0}
\pscustom[linestyle=none,fillstyle=solid,fillcolor=curcolor]
{
\newpath
\moveto(322.73777466,402.31710297)
\curveto(322.75776609,402.25709884)(322.76776608,402.17209893)(322.76777466,402.06210297)
\curveto(322.76776608,401.95209915)(322.75776609,401.86709923)(322.73777466,401.80710297)
\lineto(322.73777466,401.65710297)
\curveto(322.71776613,401.57709952)(322.70776614,401.4970996)(322.70777466,401.41710297)
\curveto(322.71776613,401.33709976)(322.71276613,401.25709984)(322.69277466,401.17710297)
\curveto(322.67276617,401.10709999)(322.65776619,401.04210006)(322.64777466,400.98210297)
\curveto(322.63776621,400.92210018)(322.62776622,400.85710024)(322.61777466,400.78710297)
\curveto(322.57776627,400.67710042)(322.5427663,400.56210054)(322.51277466,400.44210297)
\curveto(322.48276636,400.33210077)(322.4427664,400.22710087)(322.39277466,400.12710297)
\curveto(322.18276666,399.64710145)(321.90776694,399.25710184)(321.56777466,398.95710297)
\curveto(321.22776762,398.65710244)(320.81776803,398.40710269)(320.33777466,398.20710297)
\curveto(320.21776863,398.15710294)(320.09276875,398.12210298)(319.96277466,398.10210297)
\curveto(319.842769,398.07210303)(319.71776913,398.04210306)(319.58777466,398.01210297)
\curveto(319.53776931,397.99210311)(319.48276936,397.98210312)(319.42277466,397.98210297)
\curveto(319.36276948,397.98210312)(319.30776954,397.97710312)(319.25777466,397.96710297)
\lineto(319.15277466,397.96710297)
\curveto(319.12276972,397.95710314)(319.09276975,397.95210315)(319.06277466,397.95210297)
\curveto(319.01276983,397.94210316)(318.93276991,397.93710316)(318.82277466,397.93710297)
\curveto(318.71277013,397.92710317)(318.62777022,397.93210317)(318.56777466,397.95210297)
\lineto(318.41777466,397.95210297)
\curveto(318.36777048,397.96210314)(318.31277053,397.96710313)(318.25277466,397.96710297)
\curveto(318.20277064,397.95710314)(318.15277069,397.96210314)(318.10277466,397.98210297)
\curveto(318.06277078,397.99210311)(318.02277082,397.9971031)(317.98277466,397.99710297)
\curveto(317.95277089,397.9971031)(317.91277093,398.0021031)(317.86277466,398.01210297)
\curveto(317.76277108,398.04210306)(317.66277118,398.06710303)(317.56277466,398.08710297)
\curveto(317.46277138,398.10710299)(317.36777148,398.13710296)(317.27777466,398.17710297)
\curveto(317.15777169,398.21710288)(317.0427718,398.25710284)(316.93277466,398.29710297)
\curveto(316.83277201,398.33710276)(316.72777212,398.38710271)(316.61777466,398.44710297)
\curveto(316.26777258,398.65710244)(315.96777288,398.9021022)(315.71777466,399.18210297)
\curveto(315.46777338,399.46210164)(315.25777359,399.7971013)(315.08777466,400.18710297)
\curveto(315.03777381,400.27710082)(314.99777385,400.37210073)(314.96777466,400.47210297)
\curveto(314.9477739,400.57210053)(314.92277392,400.67710042)(314.89277466,400.78710297)
\curveto(314.87277397,400.83710026)(314.86277398,400.88210022)(314.86277466,400.92210297)
\curveto(314.86277398,400.96210014)(314.85277399,401.00710009)(314.83277466,401.05710297)
\curveto(314.81277403,401.13709996)(314.80277404,401.21709988)(314.80277466,401.29710297)
\curveto(314.80277404,401.38709971)(314.79277405,401.47209963)(314.77277466,401.55210297)
\curveto(314.76277408,401.6020995)(314.75777409,401.64709945)(314.75777466,401.68710297)
\lineto(314.75777466,401.82210297)
\curveto(314.73777411,401.88209922)(314.72777412,401.96709913)(314.72777466,402.07710297)
\curveto(314.73777411,402.18709891)(314.75277409,402.27209883)(314.77277466,402.33210297)
\lineto(314.77277466,402.43710297)
\curveto(314.78277406,402.48709861)(314.78277406,402.53709856)(314.77277466,402.58710297)
\curveto(314.77277407,402.64709845)(314.78277406,402.7020984)(314.80277466,402.75210297)
\curveto(314.81277403,402.8020983)(314.81777403,402.84709825)(314.81777466,402.88710297)
\curveto(314.81777403,402.93709816)(314.82777402,402.98709811)(314.84777466,403.03710297)
\curveto(314.88777396,403.16709793)(314.92277392,403.29209781)(314.95277466,403.41210297)
\curveto(314.98277386,403.54209756)(315.02277382,403.66709743)(315.07277466,403.78710297)
\curveto(315.25277359,404.1970969)(315.46777338,404.53709656)(315.71777466,404.80710297)
\curveto(315.96777288,405.08709601)(316.27277257,405.34209576)(316.63277466,405.57210297)
\curveto(316.73277211,405.62209548)(316.83777201,405.66709543)(316.94777466,405.70710297)
\curveto(317.05777179,405.74709535)(317.16777168,405.79209531)(317.27777466,405.84210297)
\curveto(317.40777144,405.89209521)(317.5427713,405.92709517)(317.68277466,405.94710297)
\curveto(317.82277102,405.96709513)(317.96777088,405.9970951)(318.11777466,406.03710297)
\curveto(318.19777065,406.04709505)(318.27277057,406.05209505)(318.34277466,406.05210297)
\curveto(318.41277043,406.05209505)(318.48277036,406.05709504)(318.55277466,406.06710297)
\curveto(319.13276971,406.07709502)(319.63276921,406.01709508)(320.05277466,405.88710297)
\curveto(320.48276836,405.75709534)(320.86276798,405.57709552)(321.19277466,405.34710297)
\curveto(321.30276754,405.26709583)(321.41276743,405.17709592)(321.52277466,405.07710297)
\curveto(321.6427672,404.98709611)(321.7427671,404.88709621)(321.82277466,404.77710297)
\curveto(321.90276694,404.67709642)(321.97276687,404.57709652)(322.03277466,404.47710297)
\curveto(322.10276674,404.37709672)(322.17276667,404.27209683)(322.24277466,404.16210297)
\curveto(322.31276653,404.05209705)(322.36776648,403.93209717)(322.40777466,403.80210297)
\curveto(322.4477664,403.68209742)(322.49276635,403.55209755)(322.54277466,403.41210297)
\curveto(322.57276627,403.33209777)(322.59776625,403.24709785)(322.61777466,403.15710297)
\lineto(322.67777466,402.88710297)
\curveto(322.68776616,402.84709825)(322.69276615,402.80709829)(322.69277466,402.76710297)
\curveto(322.69276615,402.72709837)(322.69776615,402.68709841)(322.70777466,402.64710297)
\curveto(322.72776612,402.5970985)(322.73276611,402.54209856)(322.72277466,402.48210297)
\curveto(322.71276613,402.42209868)(322.71776613,402.36709873)(322.73777466,402.31710297)
\moveto(320.63777466,401.77710297)
\curveto(320.6477682,401.82709927)(320.65276819,401.8970992)(320.65277466,401.98710297)
\curveto(320.65276819,402.08709901)(320.6477682,402.16209894)(320.63777466,402.21210297)
\lineto(320.63777466,402.33210297)
\curveto(320.61776823,402.38209872)(320.60776824,402.43709866)(320.60777466,402.49710297)
\curveto(320.60776824,402.55709854)(320.60276824,402.61209849)(320.59277466,402.66210297)
\curveto(320.59276825,402.7020984)(320.58776826,402.73209837)(320.57777466,402.75210297)
\lineto(320.51777466,402.99210297)
\curveto(320.50776834,403.08209802)(320.48776836,403.16709793)(320.45777466,403.24710297)
\curveto(320.3477685,403.50709759)(320.21776863,403.72709737)(320.06777466,403.90710297)
\curveto(319.91776893,404.097097)(319.71776913,404.24709685)(319.46777466,404.35710297)
\curveto(319.40776944,404.37709672)(319.3477695,404.39209671)(319.28777466,404.40210297)
\curveto(319.22776962,404.42209668)(319.16276968,404.44209666)(319.09277466,404.46210297)
\curveto(319.01276983,404.48209662)(318.92776992,404.48709661)(318.83777466,404.47710297)
\lineto(318.56777466,404.47710297)
\curveto(318.53777031,404.45709664)(318.50277034,404.44709665)(318.46277466,404.44710297)
\curveto(318.42277042,404.45709664)(318.38777046,404.45709664)(318.35777466,404.44710297)
\lineto(318.14777466,404.38710297)
\curveto(318.08777076,404.37709672)(318.03277081,404.35709674)(317.98277466,404.32710297)
\curveto(317.73277111,404.21709688)(317.52777132,404.05709704)(317.36777466,403.84710297)
\curveto(317.21777163,403.64709745)(317.09777175,403.41209769)(317.00777466,403.14210297)
\curveto(316.97777187,403.04209806)(316.95277189,402.93709816)(316.93277466,402.82710297)
\curveto(316.92277192,402.71709838)(316.90777194,402.60709849)(316.88777466,402.49710297)
\curveto(316.87777197,402.44709865)(316.87277197,402.3970987)(316.87277466,402.34710297)
\lineto(316.87277466,402.19710297)
\curveto(316.85277199,402.12709897)(316.842772,402.02209908)(316.84277466,401.88210297)
\curveto(316.85277199,401.74209936)(316.86777198,401.63709946)(316.88777466,401.56710297)
\lineto(316.88777466,401.43210297)
\curveto(316.90777194,401.35209975)(316.92277192,401.27209983)(316.93277466,401.19210297)
\curveto(316.9427719,401.12209998)(316.95777189,401.04710005)(316.97777466,400.96710297)
\curveto(317.07777177,400.66710043)(317.18277166,400.42210068)(317.29277466,400.23210297)
\curveto(317.41277143,400.05210105)(317.59777125,399.88710121)(317.84777466,399.73710297)
\curveto(317.91777093,399.68710141)(317.99277085,399.64710145)(318.07277466,399.61710297)
\curveto(318.16277068,399.58710151)(318.25277059,399.56210154)(318.34277466,399.54210297)
\curveto(318.38277046,399.53210157)(318.41777043,399.52710157)(318.44777466,399.52710297)
\curveto(318.47777037,399.53710156)(318.51277033,399.53710156)(318.55277466,399.52710297)
\lineto(318.67277466,399.49710297)
\curveto(318.72277012,399.4971016)(318.76777008,399.5021016)(318.80777466,399.51210297)
\lineto(318.92777466,399.51210297)
\curveto(319.00776984,399.53210157)(319.08776976,399.54710155)(319.16777466,399.55710297)
\curveto(319.2477696,399.56710153)(319.32276952,399.58710151)(319.39277466,399.61710297)
\curveto(319.65276919,399.71710138)(319.86276898,399.85210125)(320.02277466,400.02210297)
\curveto(320.18276866,400.19210091)(320.31776853,400.4021007)(320.42777466,400.65210297)
\curveto(320.46776838,400.75210035)(320.49776835,400.85210025)(320.51777466,400.95210297)
\curveto(320.53776831,401.05210005)(320.56276828,401.15709994)(320.59277466,401.26710297)
\curveto(320.60276824,401.30709979)(320.60776824,401.34209976)(320.60777466,401.37210297)
\curveto(320.60776824,401.41209969)(320.61276823,401.45209965)(320.62277466,401.49210297)
\lineto(320.62277466,401.62710297)
\curveto(320.62276822,401.67709942)(320.62776822,401.72709937)(320.63777466,401.77710297)
}
}
{
\newrgbcolor{curcolor}{0 0 0}
\pscustom[linestyle=none,fillstyle=solid,fillcolor=curcolor]
{
\newpath
\moveto(327.10769653,406.08210297)
\curveto(327.85769203,406.102095)(328.50769138,406.01709508)(329.05769653,405.82710297)
\curveto(329.61769027,405.64709545)(330.04268985,405.33209577)(330.33269653,404.88210297)
\curveto(330.40268949,404.77209633)(330.46268943,404.65709644)(330.51269653,404.53710297)
\curveto(330.57268932,404.42709667)(330.62268927,404.3020968)(330.66269653,404.16210297)
\curveto(330.68268921,404.102097)(330.6926892,404.03709706)(330.69269653,403.96710297)
\curveto(330.6926892,403.8970972)(330.68268921,403.83709726)(330.66269653,403.78710297)
\curveto(330.62268927,403.72709737)(330.56768932,403.68709741)(330.49769653,403.66710297)
\curveto(330.44768944,403.64709745)(330.3876895,403.63709746)(330.31769653,403.63710297)
\lineto(330.10769653,403.63710297)
\lineto(329.44769653,403.63710297)
\curveto(329.37769051,403.63709746)(329.30769058,403.63209747)(329.23769653,403.62210297)
\curveto(329.16769072,403.62209748)(329.10269079,403.63209747)(329.04269653,403.65210297)
\curveto(328.94269095,403.67209743)(328.86769102,403.71209739)(328.81769653,403.77210297)
\curveto(328.76769112,403.83209727)(328.72269117,403.89209721)(328.68269653,403.95210297)
\lineto(328.56269653,404.16210297)
\curveto(328.53269136,404.24209686)(328.48269141,404.30709679)(328.41269653,404.35710297)
\curveto(328.31269158,404.43709666)(328.21269168,404.4970966)(328.11269653,404.53710297)
\curveto(328.02269187,404.57709652)(327.90769198,404.61209649)(327.76769653,404.64210297)
\curveto(327.69769219,404.66209644)(327.5926923,404.67709642)(327.45269653,404.68710297)
\curveto(327.32269257,404.6970964)(327.22269267,404.69209641)(327.15269653,404.67210297)
\lineto(327.04769653,404.67210297)
\lineto(326.89769653,404.64210297)
\curveto(326.85769303,404.64209646)(326.81269308,404.63709646)(326.76269653,404.62710297)
\curveto(326.5926933,404.57709652)(326.45269344,404.50709659)(326.34269653,404.41710297)
\curveto(326.24269365,404.33709676)(326.17269372,404.21209689)(326.13269653,404.04210297)
\curveto(326.11269378,403.97209713)(326.11269378,403.90709719)(326.13269653,403.84710297)
\curveto(326.15269374,403.78709731)(326.17269372,403.73709736)(326.19269653,403.69710297)
\curveto(326.26269363,403.57709752)(326.34269355,403.48209762)(326.43269653,403.41210297)
\curveto(326.53269336,403.34209776)(326.64769324,403.28209782)(326.77769653,403.23210297)
\curveto(326.96769292,403.15209795)(327.17269272,403.08209802)(327.39269653,403.02210297)
\lineto(328.08269653,402.87210297)
\curveto(328.32269157,402.83209827)(328.55269134,402.78209832)(328.77269653,402.72210297)
\curveto(329.00269089,402.67209843)(329.21769067,402.60709849)(329.41769653,402.52710297)
\curveto(329.50769038,402.48709861)(329.5926903,402.45209865)(329.67269653,402.42210297)
\curveto(329.76269013,402.4020987)(329.84769004,402.36709873)(329.92769653,402.31710297)
\curveto(330.11768977,402.1970989)(330.2876896,402.06709903)(330.43769653,401.92710297)
\curveto(330.59768929,401.78709931)(330.72268917,401.61209949)(330.81269653,401.40210297)
\curveto(330.84268905,401.33209977)(330.86768902,401.26209984)(330.88769653,401.19210297)
\curveto(330.90768898,401.12209998)(330.92768896,401.04710005)(330.94769653,400.96710297)
\curveto(330.95768893,400.90710019)(330.96268893,400.81210029)(330.96269653,400.68210297)
\curveto(330.97268892,400.56210054)(330.97268892,400.46710063)(330.96269653,400.39710297)
\lineto(330.96269653,400.32210297)
\curveto(330.94268895,400.26210084)(330.92768896,400.2021009)(330.91769653,400.14210297)
\curveto(330.91768897,400.09210101)(330.91268898,400.04210106)(330.90269653,399.99210297)
\curveto(330.83268906,399.69210141)(330.72268917,399.42710167)(330.57269653,399.19710297)
\curveto(330.41268948,398.95710214)(330.21768967,398.76210234)(329.98769653,398.61210297)
\curveto(329.75769013,398.46210264)(329.49769039,398.33210277)(329.20769653,398.22210297)
\curveto(329.09769079,398.17210293)(328.97769091,398.13710296)(328.84769653,398.11710297)
\curveto(328.72769116,398.097103)(328.60769128,398.07210303)(328.48769653,398.04210297)
\curveto(328.39769149,398.02210308)(328.30269159,398.01210309)(328.20269653,398.01210297)
\curveto(328.11269178,398.0021031)(328.02269187,397.98710311)(327.93269653,397.96710297)
\lineto(327.66269653,397.96710297)
\curveto(327.60269229,397.94710315)(327.49769239,397.93710316)(327.34769653,397.93710297)
\curveto(327.20769268,397.93710316)(327.10769278,397.94710315)(327.04769653,397.96710297)
\curveto(327.01769287,397.96710313)(326.98269291,397.97210313)(326.94269653,397.98210297)
\lineto(326.83769653,397.98210297)
\curveto(326.71769317,398.0021031)(326.59769329,398.01710308)(326.47769653,398.02710297)
\curveto(326.35769353,398.03710306)(326.24269365,398.05710304)(326.13269653,398.08710297)
\curveto(325.74269415,398.1971029)(325.39769449,398.32210278)(325.09769653,398.46210297)
\curveto(324.79769509,398.61210249)(324.54269535,398.83210227)(324.33269653,399.12210297)
\curveto(324.1926957,399.31210179)(324.07269582,399.53210157)(323.97269653,399.78210297)
\curveto(323.95269594,399.84210126)(323.93269596,399.92210118)(323.91269653,400.02210297)
\curveto(323.892696,400.07210103)(323.87769601,400.14210096)(323.86769653,400.23210297)
\curveto(323.85769603,400.32210078)(323.86269603,400.3971007)(323.88269653,400.45710297)
\curveto(323.91269598,400.52710057)(323.96269593,400.57710052)(324.03269653,400.60710297)
\curveto(324.08269581,400.62710047)(324.14269575,400.63710046)(324.21269653,400.63710297)
\lineto(324.43769653,400.63710297)
\lineto(325.14269653,400.63710297)
\lineto(325.38269653,400.63710297)
\curveto(325.46269443,400.63710046)(325.53269436,400.62710047)(325.59269653,400.60710297)
\curveto(325.70269419,400.56710053)(325.77269412,400.5021006)(325.80269653,400.41210297)
\curveto(325.84269405,400.32210078)(325.887694,400.22710087)(325.93769653,400.12710297)
\curveto(325.95769393,400.07710102)(325.9926939,400.01210109)(326.04269653,399.93210297)
\curveto(326.10269379,399.85210125)(326.15269374,399.8021013)(326.19269653,399.78210297)
\curveto(326.31269358,399.68210142)(326.42769346,399.6021015)(326.53769653,399.54210297)
\curveto(326.64769324,399.49210161)(326.7876931,399.44210166)(326.95769653,399.39210297)
\curveto(327.00769288,399.37210173)(327.05769283,399.36210174)(327.10769653,399.36210297)
\curveto(327.15769273,399.37210173)(327.20769268,399.37210173)(327.25769653,399.36210297)
\curveto(327.33769255,399.34210176)(327.42269247,399.33210177)(327.51269653,399.33210297)
\curveto(327.61269228,399.34210176)(327.69769219,399.35710174)(327.76769653,399.37710297)
\curveto(327.81769207,399.38710171)(327.86269203,399.39210171)(327.90269653,399.39210297)
\curveto(327.95269194,399.39210171)(328.00269189,399.4021017)(328.05269653,399.42210297)
\curveto(328.1926917,399.47210163)(328.31769157,399.53210157)(328.42769653,399.60210297)
\curveto(328.54769134,399.67210143)(328.64269125,399.76210134)(328.71269653,399.87210297)
\curveto(328.76269113,399.95210115)(328.80269109,400.07710102)(328.83269653,400.24710297)
\curveto(328.85269104,400.31710078)(328.85269104,400.38210072)(328.83269653,400.44210297)
\curveto(328.81269108,400.5021006)(328.7926911,400.55210055)(328.77269653,400.59210297)
\curveto(328.70269119,400.73210037)(328.61269128,400.83710026)(328.50269653,400.90710297)
\curveto(328.40269149,400.97710012)(328.28269161,401.04210006)(328.14269653,401.10210297)
\curveto(327.95269194,401.18209992)(327.75269214,401.24709985)(327.54269653,401.29710297)
\curveto(327.33269256,401.34709975)(327.12269277,401.4020997)(326.91269653,401.46210297)
\curveto(326.83269306,401.48209962)(326.74769314,401.4970996)(326.65769653,401.50710297)
\curveto(326.57769331,401.51709958)(326.49769339,401.53209957)(326.41769653,401.55210297)
\curveto(326.09769379,401.64209946)(325.7926941,401.72709937)(325.50269653,401.80710297)
\curveto(325.21269468,401.8970992)(324.94769494,402.02709907)(324.70769653,402.19710297)
\curveto(324.42769546,402.3970987)(324.22269567,402.66709843)(324.09269653,403.00710297)
\curveto(324.07269582,403.07709802)(324.05269584,403.17209793)(324.03269653,403.29210297)
\curveto(324.01269588,403.36209774)(323.99769589,403.44709765)(323.98769653,403.54710297)
\curveto(323.97769591,403.64709745)(323.98269591,403.73709736)(324.00269653,403.81710297)
\curveto(324.02269587,403.86709723)(324.02769586,403.90709719)(324.01769653,403.93710297)
\curveto(324.00769588,403.97709712)(324.01269588,404.02209708)(324.03269653,404.07210297)
\curveto(324.05269584,404.18209692)(324.07269582,404.28209682)(324.09269653,404.37210297)
\curveto(324.12269577,404.47209663)(324.15769573,404.56709653)(324.19769653,404.65710297)
\curveto(324.32769556,404.94709615)(324.50769538,405.18209592)(324.73769653,405.36210297)
\curveto(324.96769492,405.54209556)(325.22769466,405.68709541)(325.51769653,405.79710297)
\curveto(325.62769426,405.84709525)(325.74269415,405.88209522)(325.86269653,405.90210297)
\curveto(325.98269391,405.93209517)(326.10769378,405.96209514)(326.23769653,405.99210297)
\curveto(326.29769359,406.01209509)(326.35769353,406.02209508)(326.41769653,406.02210297)
\lineto(326.59769653,406.05210297)
\curveto(326.67769321,406.06209504)(326.76269313,406.06709503)(326.85269653,406.06710297)
\curveto(326.94269295,406.06709503)(327.02769286,406.07209503)(327.10769653,406.08210297)
}
}
{
\newrgbcolor{curcolor}{0 0 0}
\pscustom[linestyle=none,fillstyle=solid,fillcolor=curcolor]
{
}
}
{
\newrgbcolor{curcolor}{0 0 0}
\pscustom[linestyle=none,fillstyle=solid,fillcolor=curcolor]
{
\newpath
\moveto(344.24449341,402.09210297)
\curveto(344.25448473,402.03209907)(344.25948472,401.94209916)(344.25949341,401.82210297)
\curveto(344.25948472,401.7020994)(344.24948473,401.61709948)(344.22949341,401.56710297)
\lineto(344.22949341,401.37210297)
\curveto(344.19948478,401.26209984)(344.1794848,401.15709994)(344.16949341,401.05710297)
\curveto(344.16948481,400.95710014)(344.15448483,400.85710024)(344.12449341,400.75710297)
\curveto(344.10448488,400.66710043)(344.0844849,400.57210053)(344.06449341,400.47210297)
\curveto(344.04448494,400.38210072)(344.01448497,400.29210081)(343.97449341,400.20210297)
\curveto(343.90448508,400.03210107)(343.83448515,399.87210123)(343.76449341,399.72210297)
\curveto(343.69448529,399.58210152)(343.61448537,399.44210166)(343.52449341,399.30210297)
\curveto(343.46448552,399.21210189)(343.39948558,399.12710197)(343.32949341,399.04710297)
\curveto(343.26948571,398.97710212)(343.19948578,398.9021022)(343.11949341,398.82210297)
\lineto(343.01449341,398.71710297)
\curveto(342.96448602,398.66710243)(342.90948607,398.62210248)(342.84949341,398.58210297)
\lineto(342.69949341,398.46210297)
\curveto(342.61948636,398.4021027)(342.52948645,398.34710275)(342.42949341,398.29710297)
\curveto(342.33948664,398.25710284)(342.24448674,398.21210289)(342.14449341,398.16210297)
\curveto(342.04448694,398.11210299)(341.93948704,398.07710302)(341.82949341,398.05710297)
\curveto(341.72948725,398.03710306)(341.62448736,398.01710308)(341.51449341,397.99710297)
\curveto(341.45448753,397.97710312)(341.38948759,397.96710313)(341.31949341,397.96710297)
\curveto(341.25948772,397.96710313)(341.19448779,397.95710314)(341.12449341,397.93710297)
\lineto(340.98949341,397.93710297)
\curveto(340.90948807,397.91710318)(340.83448815,397.91710318)(340.76449341,397.93710297)
\lineto(340.61449341,397.93710297)
\curveto(340.55448843,397.95710314)(340.48948849,397.96710313)(340.41949341,397.96710297)
\curveto(340.35948862,397.95710314)(340.29948868,397.96210314)(340.23949341,397.98210297)
\curveto(340.0794889,398.03210307)(339.92448906,398.07710302)(339.77449341,398.11710297)
\curveto(339.63448935,398.15710294)(339.50448948,398.21710288)(339.38449341,398.29710297)
\curveto(339.31448967,398.33710276)(339.24948973,398.37710272)(339.18949341,398.41710297)
\curveto(339.12948985,398.46710263)(339.06448992,398.51710258)(338.99449341,398.56710297)
\lineto(338.81449341,398.70210297)
\curveto(338.73449025,398.76210234)(338.66449032,398.76710233)(338.60449341,398.71710297)
\curveto(338.55449043,398.68710241)(338.52949045,398.64710245)(338.52949341,398.59710297)
\curveto(338.52949045,398.55710254)(338.51949046,398.50710259)(338.49949341,398.44710297)
\curveto(338.4794905,398.34710275)(338.46949051,398.23210287)(338.46949341,398.10210297)
\curveto(338.4794905,397.97210313)(338.4844905,397.85210325)(338.48449341,397.74210297)
\lineto(338.48449341,396.21210297)
\curveto(338.4844905,396.08210502)(338.4794905,395.95710514)(338.46949341,395.83710297)
\curveto(338.46949051,395.70710539)(338.44449054,395.6021055)(338.39449341,395.52210297)
\curveto(338.36449062,395.48210562)(338.30949067,395.45210565)(338.22949341,395.43210297)
\curveto(338.14949083,395.41210569)(338.05949092,395.4021057)(337.95949341,395.40210297)
\curveto(337.85949112,395.39210571)(337.75949122,395.39210571)(337.65949341,395.40210297)
\lineto(337.40449341,395.40210297)
\lineto(336.99949341,395.40210297)
\lineto(336.89449341,395.40210297)
\curveto(336.85449213,395.4021057)(336.81949216,395.40710569)(336.78949341,395.41710297)
\lineto(336.66949341,395.41710297)
\curveto(336.49949248,395.46710563)(336.40949257,395.56710553)(336.39949341,395.71710297)
\curveto(336.38949259,395.85710524)(336.3844926,396.02710507)(336.38449341,396.22710297)
\lineto(336.38449341,405.03210297)
\curveto(336.3844926,405.14209596)(336.3794926,405.25709584)(336.36949341,405.37710297)
\curveto(336.36949261,405.50709559)(336.39449259,405.60709549)(336.44449341,405.67710297)
\curveto(336.4844925,405.74709535)(336.53949244,405.79209531)(336.60949341,405.81210297)
\curveto(336.65949232,405.83209527)(336.71949226,405.84209526)(336.78949341,405.84210297)
\lineto(337.01449341,405.84210297)
\lineto(337.73449341,405.84210297)
\lineto(338.01949341,405.84210297)
\curveto(338.10949087,405.84209526)(338.1844908,405.81709528)(338.24449341,405.76710297)
\curveto(338.31449067,405.71709538)(338.34949063,405.65209545)(338.34949341,405.57210297)
\curveto(338.35949062,405.5020956)(338.3844906,405.42709567)(338.42449341,405.34710297)
\curveto(338.43449055,405.31709578)(338.44449054,405.29209581)(338.45449341,405.27210297)
\curveto(338.47449051,405.26209584)(338.49449049,405.24709585)(338.51449341,405.22710297)
\curveto(338.62449036,405.21709588)(338.71449027,405.24709585)(338.78449341,405.31710297)
\curveto(338.85449013,405.38709571)(338.92449006,405.44709565)(338.99449341,405.49710297)
\curveto(339.12448986,405.58709551)(339.25948972,405.66709543)(339.39949341,405.73710297)
\curveto(339.53948944,405.81709528)(339.69448929,405.88209522)(339.86449341,405.93210297)
\curveto(339.94448904,405.96209514)(340.02948895,405.98209512)(340.11949341,405.99210297)
\curveto(340.21948876,406.0020951)(340.31448867,406.01709508)(340.40449341,406.03710297)
\curveto(340.44448854,406.04709505)(340.4844885,406.04709505)(340.52449341,406.03710297)
\curveto(340.57448841,406.02709507)(340.61448837,406.03209507)(340.64449341,406.05210297)
\curveto(341.21448777,406.07209503)(341.69448729,405.99209511)(342.08449341,405.81210297)
\curveto(342.4844865,405.64209546)(342.82448616,405.41709568)(343.10449341,405.13710297)
\curveto(343.15448583,405.08709601)(343.19948578,405.03709606)(343.23949341,404.98710297)
\curveto(343.2794857,404.94709615)(343.31948566,404.9020962)(343.35949341,404.85210297)
\curveto(343.42948555,404.76209634)(343.48948549,404.67209643)(343.53949341,404.58210297)
\curveto(343.59948538,404.49209661)(343.65448533,404.4020967)(343.70449341,404.31210297)
\curveto(343.72448526,404.29209681)(343.73448525,404.26709683)(343.73449341,404.23710297)
\curveto(343.74448524,404.20709689)(343.75948522,404.17209693)(343.77949341,404.13210297)
\curveto(343.83948514,404.03209707)(343.89448509,403.91209719)(343.94449341,403.77210297)
\curveto(343.96448502,403.71209739)(343.984485,403.64709745)(344.00449341,403.57710297)
\curveto(344.02448496,403.51709758)(344.04448494,403.45209765)(344.06449341,403.38210297)
\curveto(344.10448488,403.26209784)(344.12948485,403.13709796)(344.13949341,403.00710297)
\curveto(344.15948482,402.87709822)(344.1844848,402.74209836)(344.21449341,402.60210297)
\lineto(344.21449341,402.43710297)
\lineto(344.24449341,402.25710297)
\lineto(344.24449341,402.09210297)
\moveto(342.12949341,401.74710297)
\curveto(342.13948684,401.7970993)(342.14448684,401.86209924)(342.14449341,401.94210297)
\curveto(342.14448684,402.03209907)(342.13948684,402.102099)(342.12949341,402.15210297)
\lineto(342.12949341,402.28710297)
\curveto(342.10948687,402.34709875)(342.09948688,402.41209869)(342.09949341,402.48210297)
\curveto(342.09948688,402.55209855)(342.08948689,402.62209848)(342.06949341,402.69210297)
\curveto(342.04948693,402.79209831)(342.02948695,402.88709821)(342.00949341,402.97710297)
\curveto(341.98948699,403.07709802)(341.95948702,403.16709793)(341.91949341,403.24710297)
\curveto(341.79948718,403.56709753)(341.64448734,403.82209728)(341.45449341,404.01210297)
\curveto(341.26448772,404.2020969)(340.99448799,404.34209676)(340.64449341,404.43210297)
\curveto(340.56448842,404.45209665)(340.47448851,404.46209664)(340.37449341,404.46210297)
\lineto(340.10449341,404.46210297)
\curveto(340.06448892,404.45209665)(340.02948895,404.44709665)(339.99949341,404.44710297)
\curveto(339.96948901,404.44709665)(339.93448905,404.44209666)(339.89449341,404.43210297)
\lineto(339.68449341,404.37210297)
\curveto(339.62448936,404.36209674)(339.56448942,404.34209676)(339.50449341,404.31210297)
\curveto(339.24448974,404.2020969)(339.03948994,404.03209707)(338.88949341,403.80210297)
\curveto(338.74949023,403.57209753)(338.63449035,403.31709778)(338.54449341,403.03710297)
\curveto(338.52449046,402.95709814)(338.50949047,402.87209823)(338.49949341,402.78210297)
\curveto(338.48949049,402.7020984)(338.47449051,402.62209848)(338.45449341,402.54210297)
\curveto(338.44449054,402.5020986)(338.43949054,402.43709866)(338.43949341,402.34710297)
\curveto(338.41949056,402.30709879)(338.41449057,402.25709884)(338.42449341,402.19710297)
\curveto(338.43449055,402.14709895)(338.43449055,402.097099)(338.42449341,402.04710297)
\curveto(338.40449058,401.98709911)(338.40449058,401.93209917)(338.42449341,401.88210297)
\lineto(338.42449341,401.70210297)
\lineto(338.42449341,401.56710297)
\curveto(338.42449056,401.52709957)(338.43449055,401.48709961)(338.45449341,401.44710297)
\curveto(338.45449053,401.37709972)(338.45949052,401.32209978)(338.46949341,401.28210297)
\lineto(338.49949341,401.10210297)
\curveto(338.50949047,401.04210006)(338.52449046,400.98210012)(338.54449341,400.92210297)
\curveto(338.63449035,400.63210047)(338.73949024,400.39210071)(338.85949341,400.20210297)
\curveto(338.98948999,400.02210108)(339.16948981,399.86210124)(339.39949341,399.72210297)
\curveto(339.53948944,399.64210146)(339.70448928,399.57710152)(339.89449341,399.52710297)
\curveto(339.93448905,399.51710158)(339.96948901,399.51210159)(339.99949341,399.51210297)
\curveto(340.02948895,399.52210158)(340.06448892,399.52210158)(340.10449341,399.51210297)
\curveto(340.14448884,399.5021016)(340.20448878,399.49210161)(340.28449341,399.48210297)
\curveto(340.36448862,399.48210162)(340.42948855,399.48710161)(340.47949341,399.49710297)
\curveto(340.55948842,399.51710158)(340.63948834,399.53210157)(340.71949341,399.54210297)
\curveto(340.80948817,399.56210154)(340.89448809,399.58710151)(340.97449341,399.61710297)
\curveto(341.21448777,399.71710138)(341.40948757,399.85710124)(341.55949341,400.03710297)
\curveto(341.70948727,400.21710088)(341.83448715,400.42710067)(341.93449341,400.66710297)
\curveto(341.984487,400.78710031)(342.01948696,400.91210019)(342.03949341,401.04210297)
\curveto(342.05948692,401.17209993)(342.0844869,401.30709979)(342.11449341,401.44710297)
\lineto(342.11449341,401.59710297)
\curveto(342.12448686,401.64709945)(342.12948685,401.6970994)(342.12949341,401.74710297)
}
}
{
\newrgbcolor{curcolor}{0 0 0}
\pscustom[linestyle=none,fillstyle=solid,fillcolor=curcolor]
{
\newpath
\moveto(353.29441528,402.31710297)
\curveto(353.31440671,402.25709884)(353.3244067,402.17209893)(353.32441528,402.06210297)
\curveto(353.3244067,401.95209915)(353.31440671,401.86709923)(353.29441528,401.80710297)
\lineto(353.29441528,401.65710297)
\curveto(353.27440675,401.57709952)(353.26440676,401.4970996)(353.26441528,401.41710297)
\curveto(353.27440675,401.33709976)(353.26940676,401.25709984)(353.24941528,401.17710297)
\curveto(353.2294068,401.10709999)(353.21440681,401.04210006)(353.20441528,400.98210297)
\curveto(353.19440683,400.92210018)(353.18440684,400.85710024)(353.17441528,400.78710297)
\curveto(353.13440689,400.67710042)(353.09940693,400.56210054)(353.06941528,400.44210297)
\curveto(353.03940699,400.33210077)(352.99940703,400.22710087)(352.94941528,400.12710297)
\curveto(352.73940729,399.64710145)(352.46440756,399.25710184)(352.12441528,398.95710297)
\curveto(351.78440824,398.65710244)(351.37440865,398.40710269)(350.89441528,398.20710297)
\curveto(350.77440925,398.15710294)(350.64940938,398.12210298)(350.51941528,398.10210297)
\curveto(350.39940963,398.07210303)(350.27440975,398.04210306)(350.14441528,398.01210297)
\curveto(350.09440993,397.99210311)(350.03940999,397.98210312)(349.97941528,397.98210297)
\curveto(349.91941011,397.98210312)(349.86441016,397.97710312)(349.81441528,397.96710297)
\lineto(349.70941528,397.96710297)
\curveto(349.67941035,397.95710314)(349.64941038,397.95210315)(349.61941528,397.95210297)
\curveto(349.56941046,397.94210316)(349.48941054,397.93710316)(349.37941528,397.93710297)
\curveto(349.26941076,397.92710317)(349.18441084,397.93210317)(349.12441528,397.95210297)
\lineto(348.97441528,397.95210297)
\curveto(348.9244111,397.96210314)(348.86941116,397.96710313)(348.80941528,397.96710297)
\curveto(348.75941127,397.95710314)(348.70941132,397.96210314)(348.65941528,397.98210297)
\curveto(348.61941141,397.99210311)(348.57941145,397.9971031)(348.53941528,397.99710297)
\curveto(348.50941152,397.9971031)(348.46941156,398.0021031)(348.41941528,398.01210297)
\curveto(348.31941171,398.04210306)(348.21941181,398.06710303)(348.11941528,398.08710297)
\curveto(348.01941201,398.10710299)(347.9244121,398.13710296)(347.83441528,398.17710297)
\curveto(347.71441231,398.21710288)(347.59941243,398.25710284)(347.48941528,398.29710297)
\curveto(347.38941264,398.33710276)(347.28441274,398.38710271)(347.17441528,398.44710297)
\curveto(346.8244132,398.65710244)(346.5244135,398.9021022)(346.27441528,399.18210297)
\curveto(346.024414,399.46210164)(345.81441421,399.7971013)(345.64441528,400.18710297)
\curveto(345.59441443,400.27710082)(345.55441447,400.37210073)(345.52441528,400.47210297)
\curveto(345.50441452,400.57210053)(345.47941455,400.67710042)(345.44941528,400.78710297)
\curveto(345.4294146,400.83710026)(345.41941461,400.88210022)(345.41941528,400.92210297)
\curveto(345.41941461,400.96210014)(345.40941462,401.00710009)(345.38941528,401.05710297)
\curveto(345.36941466,401.13709996)(345.35941467,401.21709988)(345.35941528,401.29710297)
\curveto(345.35941467,401.38709971)(345.34941468,401.47209963)(345.32941528,401.55210297)
\curveto(345.31941471,401.6020995)(345.31441471,401.64709945)(345.31441528,401.68710297)
\lineto(345.31441528,401.82210297)
\curveto(345.29441473,401.88209922)(345.28441474,401.96709913)(345.28441528,402.07710297)
\curveto(345.29441473,402.18709891)(345.30941472,402.27209883)(345.32941528,402.33210297)
\lineto(345.32941528,402.43710297)
\curveto(345.33941469,402.48709861)(345.33941469,402.53709856)(345.32941528,402.58710297)
\curveto(345.3294147,402.64709845)(345.33941469,402.7020984)(345.35941528,402.75210297)
\curveto(345.36941466,402.8020983)(345.37441465,402.84709825)(345.37441528,402.88710297)
\curveto(345.37441465,402.93709816)(345.38441464,402.98709811)(345.40441528,403.03710297)
\curveto(345.44441458,403.16709793)(345.47941455,403.29209781)(345.50941528,403.41210297)
\curveto(345.53941449,403.54209756)(345.57941445,403.66709743)(345.62941528,403.78710297)
\curveto(345.80941422,404.1970969)(346.024414,404.53709656)(346.27441528,404.80710297)
\curveto(346.5244135,405.08709601)(346.8294132,405.34209576)(347.18941528,405.57210297)
\curveto(347.28941274,405.62209548)(347.39441263,405.66709543)(347.50441528,405.70710297)
\curveto(347.61441241,405.74709535)(347.7244123,405.79209531)(347.83441528,405.84210297)
\curveto(347.96441206,405.89209521)(348.09941193,405.92709517)(348.23941528,405.94710297)
\curveto(348.37941165,405.96709513)(348.5244115,405.9970951)(348.67441528,406.03710297)
\curveto(348.75441127,406.04709505)(348.8294112,406.05209505)(348.89941528,406.05210297)
\curveto(348.96941106,406.05209505)(349.03941099,406.05709504)(349.10941528,406.06710297)
\curveto(349.68941034,406.07709502)(350.18940984,406.01709508)(350.60941528,405.88710297)
\curveto(351.03940899,405.75709534)(351.41940861,405.57709552)(351.74941528,405.34710297)
\curveto(351.85940817,405.26709583)(351.96940806,405.17709592)(352.07941528,405.07710297)
\curveto(352.19940783,404.98709611)(352.29940773,404.88709621)(352.37941528,404.77710297)
\curveto(352.45940757,404.67709642)(352.5294075,404.57709652)(352.58941528,404.47710297)
\curveto(352.65940737,404.37709672)(352.7294073,404.27209683)(352.79941528,404.16210297)
\curveto(352.86940716,404.05209705)(352.9244071,403.93209717)(352.96441528,403.80210297)
\curveto(353.00440702,403.68209742)(353.04940698,403.55209755)(353.09941528,403.41210297)
\curveto(353.1294069,403.33209777)(353.15440687,403.24709785)(353.17441528,403.15710297)
\lineto(353.23441528,402.88710297)
\curveto(353.24440678,402.84709825)(353.24940678,402.80709829)(353.24941528,402.76710297)
\curveto(353.24940678,402.72709837)(353.25440677,402.68709841)(353.26441528,402.64710297)
\curveto(353.28440674,402.5970985)(353.28940674,402.54209856)(353.27941528,402.48210297)
\curveto(353.26940676,402.42209868)(353.27440675,402.36709873)(353.29441528,402.31710297)
\moveto(351.19441528,401.77710297)
\curveto(351.20440882,401.82709927)(351.20940882,401.8970992)(351.20941528,401.98710297)
\curveto(351.20940882,402.08709901)(351.20440882,402.16209894)(351.19441528,402.21210297)
\lineto(351.19441528,402.33210297)
\curveto(351.17440885,402.38209872)(351.16440886,402.43709866)(351.16441528,402.49710297)
\curveto(351.16440886,402.55709854)(351.15940887,402.61209849)(351.14941528,402.66210297)
\curveto(351.14940888,402.7020984)(351.14440888,402.73209837)(351.13441528,402.75210297)
\lineto(351.07441528,402.99210297)
\curveto(351.06440896,403.08209802)(351.04440898,403.16709793)(351.01441528,403.24710297)
\curveto(350.90440912,403.50709759)(350.77440925,403.72709737)(350.62441528,403.90710297)
\curveto(350.47440955,404.097097)(350.27440975,404.24709685)(350.02441528,404.35710297)
\curveto(349.96441006,404.37709672)(349.90441012,404.39209671)(349.84441528,404.40210297)
\curveto(349.78441024,404.42209668)(349.71941031,404.44209666)(349.64941528,404.46210297)
\curveto(349.56941046,404.48209662)(349.48441054,404.48709661)(349.39441528,404.47710297)
\lineto(349.12441528,404.47710297)
\curveto(349.09441093,404.45709664)(349.05941097,404.44709665)(349.01941528,404.44710297)
\curveto(348.97941105,404.45709664)(348.94441108,404.45709664)(348.91441528,404.44710297)
\lineto(348.70441528,404.38710297)
\curveto(348.64441138,404.37709672)(348.58941144,404.35709674)(348.53941528,404.32710297)
\curveto(348.28941174,404.21709688)(348.08441194,404.05709704)(347.92441528,403.84710297)
\curveto(347.77441225,403.64709745)(347.65441237,403.41209769)(347.56441528,403.14210297)
\curveto(347.53441249,403.04209806)(347.50941252,402.93709816)(347.48941528,402.82710297)
\curveto(347.47941255,402.71709838)(347.46441256,402.60709849)(347.44441528,402.49710297)
\curveto(347.43441259,402.44709865)(347.4294126,402.3970987)(347.42941528,402.34710297)
\lineto(347.42941528,402.19710297)
\curveto(347.40941262,402.12709897)(347.39941263,402.02209908)(347.39941528,401.88210297)
\curveto(347.40941262,401.74209936)(347.4244126,401.63709946)(347.44441528,401.56710297)
\lineto(347.44441528,401.43210297)
\curveto(347.46441256,401.35209975)(347.47941255,401.27209983)(347.48941528,401.19210297)
\curveto(347.49941253,401.12209998)(347.51441251,401.04710005)(347.53441528,400.96710297)
\curveto(347.63441239,400.66710043)(347.73941229,400.42210068)(347.84941528,400.23210297)
\curveto(347.96941206,400.05210105)(348.15441187,399.88710121)(348.40441528,399.73710297)
\curveto(348.47441155,399.68710141)(348.54941148,399.64710145)(348.62941528,399.61710297)
\curveto(348.71941131,399.58710151)(348.80941122,399.56210154)(348.89941528,399.54210297)
\curveto(348.93941109,399.53210157)(348.97441105,399.52710157)(349.00441528,399.52710297)
\curveto(349.03441099,399.53710156)(349.06941096,399.53710156)(349.10941528,399.52710297)
\lineto(349.22941528,399.49710297)
\curveto(349.27941075,399.4971016)(349.3244107,399.5021016)(349.36441528,399.51210297)
\lineto(349.48441528,399.51210297)
\curveto(349.56441046,399.53210157)(349.64441038,399.54710155)(349.72441528,399.55710297)
\curveto(349.80441022,399.56710153)(349.87941015,399.58710151)(349.94941528,399.61710297)
\curveto(350.20940982,399.71710138)(350.41940961,399.85210125)(350.57941528,400.02210297)
\curveto(350.73940929,400.19210091)(350.87440915,400.4021007)(350.98441528,400.65210297)
\curveto(351.024409,400.75210035)(351.05440897,400.85210025)(351.07441528,400.95210297)
\curveto(351.09440893,401.05210005)(351.11940891,401.15709994)(351.14941528,401.26710297)
\curveto(351.15940887,401.30709979)(351.16440886,401.34209976)(351.16441528,401.37210297)
\curveto(351.16440886,401.41209969)(351.16940886,401.45209965)(351.17941528,401.49210297)
\lineto(351.17941528,401.62710297)
\curveto(351.17940885,401.67709942)(351.18440884,401.72709937)(351.19441528,401.77710297)
}
}
{
\newrgbcolor{curcolor}{0 0 0}
\pscustom[linestyle=none,fillstyle=solid,fillcolor=curcolor]
{
\newpath
\moveto(359.11933716,406.06710297)
\curveto(359.22933184,406.06709503)(359.32433175,406.05709504)(359.40433716,406.03710297)
\curveto(359.49433158,406.01709508)(359.56433151,405.97209513)(359.61433716,405.90210297)
\curveto(359.6743314,405.82209528)(359.70433137,405.68209542)(359.70433716,405.48210297)
\lineto(359.70433716,404.97210297)
\lineto(359.70433716,404.59710297)
\curveto(359.71433136,404.45709664)(359.69933137,404.34709675)(359.65933716,404.26710297)
\curveto(359.61933145,404.1970969)(359.55933151,404.15209695)(359.47933716,404.13210297)
\curveto(359.40933166,404.11209699)(359.32433175,404.102097)(359.22433716,404.10210297)
\curveto(359.13433194,404.102097)(359.03433204,404.10709699)(358.92433716,404.11710297)
\curveto(358.82433225,404.12709697)(358.72933234,404.12209698)(358.63933716,404.10210297)
\curveto(358.5693325,404.08209702)(358.49933257,404.06709703)(358.42933716,404.05710297)
\curveto(358.35933271,404.05709704)(358.29433278,404.04709705)(358.23433716,404.02710297)
\curveto(358.074333,403.97709712)(357.91433316,403.9020972)(357.75433716,403.80210297)
\curveto(357.59433348,403.71209739)(357.4693336,403.60709749)(357.37933716,403.48710297)
\curveto(357.32933374,403.40709769)(357.2743338,403.32209778)(357.21433716,403.23210297)
\curveto(357.16433391,403.15209795)(357.11433396,403.06709803)(357.06433716,402.97710297)
\curveto(357.03433404,402.8970982)(357.00433407,402.81209829)(356.97433716,402.72210297)
\lineto(356.91433716,402.48210297)
\curveto(356.89433418,402.41209869)(356.88433419,402.33709876)(356.88433716,402.25710297)
\curveto(356.88433419,402.18709891)(356.8743342,402.11709898)(356.85433716,402.04710297)
\curveto(356.84433423,402.00709909)(356.83933423,401.96709913)(356.83933716,401.92710297)
\curveto(356.84933422,401.8970992)(356.84933422,401.86709923)(356.83933716,401.83710297)
\lineto(356.83933716,401.59710297)
\curveto(356.81933425,401.52709957)(356.81433426,401.44709965)(356.82433716,401.35710297)
\curveto(356.83433424,401.27709982)(356.83933423,401.1970999)(356.83933716,401.11710297)
\lineto(356.83933716,400.15710297)
\lineto(356.83933716,398.88210297)
\curveto(356.83933423,398.75210235)(356.83433424,398.63210247)(356.82433716,398.52210297)
\curveto(356.81433426,398.41210269)(356.78433429,398.32210278)(356.73433716,398.25210297)
\curveto(356.71433436,398.22210288)(356.67933439,398.1971029)(356.62933716,398.17710297)
\curveto(356.58933448,398.16710293)(356.54433453,398.15710294)(356.49433716,398.14710297)
\lineto(356.41933716,398.14710297)
\curveto(356.3693347,398.13710296)(356.31433476,398.13210297)(356.25433716,398.13210297)
\lineto(356.08933716,398.13210297)
\lineto(355.44433716,398.13210297)
\curveto(355.38433569,398.14210296)(355.31933575,398.14710295)(355.24933716,398.14710297)
\lineto(355.05433716,398.14710297)
\curveto(355.00433607,398.16710293)(354.95433612,398.18210292)(354.90433716,398.19210297)
\curveto(354.85433622,398.21210289)(354.81933625,398.24710285)(354.79933716,398.29710297)
\curveto(354.75933631,398.34710275)(354.73433634,398.41710268)(354.72433716,398.50710297)
\lineto(354.72433716,398.80710297)
\lineto(354.72433716,399.82710297)
\lineto(354.72433716,404.05710297)
\lineto(354.72433716,405.16710297)
\lineto(354.72433716,405.45210297)
\curveto(354.72433635,405.55209555)(354.74433633,405.63209547)(354.78433716,405.69210297)
\curveto(354.83433624,405.77209533)(354.90933616,405.82209528)(355.00933716,405.84210297)
\curveto(355.10933596,405.86209524)(355.22933584,405.87209523)(355.36933716,405.87210297)
\lineto(356.13433716,405.87210297)
\curveto(356.25433482,405.87209523)(356.35933471,405.86209524)(356.44933716,405.84210297)
\curveto(356.53933453,405.83209527)(356.60933446,405.78709531)(356.65933716,405.70710297)
\curveto(356.68933438,405.65709544)(356.70433437,405.58709551)(356.70433716,405.49710297)
\lineto(356.73433716,405.22710297)
\curveto(356.74433433,405.14709595)(356.75933431,405.07209603)(356.77933716,405.00210297)
\curveto(356.80933426,404.93209617)(356.85933421,404.8970962)(356.92933716,404.89710297)
\curveto(356.94933412,404.91709618)(356.9693341,404.92709617)(356.98933716,404.92710297)
\curveto(357.00933406,404.92709617)(357.02933404,404.93709616)(357.04933716,404.95710297)
\curveto(357.10933396,405.00709609)(357.15933391,405.06209604)(357.19933716,405.12210297)
\curveto(357.24933382,405.19209591)(357.30933376,405.25209585)(357.37933716,405.30210297)
\curveto(357.41933365,405.33209577)(357.45433362,405.36209574)(357.48433716,405.39210297)
\curveto(357.51433356,405.43209567)(357.54933352,405.46709563)(357.58933716,405.49710297)
\lineto(357.85933716,405.67710297)
\curveto(357.95933311,405.73709536)(358.05933301,405.79209531)(358.15933716,405.84210297)
\curveto(358.25933281,405.88209522)(358.35933271,405.91709518)(358.45933716,405.94710297)
\lineto(358.78933716,406.03710297)
\curveto(358.81933225,406.04709505)(358.8743322,406.04709505)(358.95433716,406.03710297)
\curveto(359.04433203,406.03709506)(359.09933197,406.04709505)(359.11933716,406.06710297)
}
}
{
\newrgbcolor{curcolor}{0 0 0}
\pscustom[linestyle=none,fillstyle=solid,fillcolor=curcolor]
{
}
}
{
\newrgbcolor{curcolor}{0 0 0}
\pscustom[linestyle=none,fillstyle=solid,fillcolor=curcolor]
{
\newpath
\moveto(365.73457153,408.18210297)
\lineto(366.73957153,408.18210297)
\curveto(366.88956855,408.18209292)(367.01956842,408.17209293)(367.12957153,408.15210297)
\curveto(367.24956819,408.14209296)(367.3345681,408.08209302)(367.38457153,407.97210297)
\curveto(367.40456803,407.92209318)(367.41456802,407.86209324)(367.41457153,407.79210297)
\lineto(367.41457153,407.58210297)
\lineto(367.41457153,406.90710297)
\curveto(367.41456802,406.85709424)(367.40956803,406.7970943)(367.39957153,406.72710297)
\curveto(367.39956804,406.66709443)(367.40456803,406.61209449)(367.41457153,406.56210297)
\lineto(367.41457153,406.39710297)
\curveto(367.41456802,406.31709478)(367.41956802,406.24209486)(367.42957153,406.17210297)
\curveto(367.439568,406.11209499)(367.46456797,406.05709504)(367.50457153,406.00710297)
\curveto(367.57456786,405.91709518)(367.69956774,405.86709523)(367.87957153,405.85710297)
\lineto(368.41957153,405.85710297)
\lineto(368.59957153,405.85710297)
\curveto(368.65956678,405.85709524)(368.71456672,405.84709525)(368.76457153,405.82710297)
\curveto(368.87456656,405.77709532)(368.9345665,405.68709541)(368.94457153,405.55710297)
\curveto(368.96456647,405.42709567)(368.97456646,405.28209582)(368.97457153,405.12210297)
\lineto(368.97457153,404.91210297)
\curveto(368.98456645,404.84209626)(368.97956646,404.78209632)(368.95957153,404.73210297)
\curveto(368.90956653,404.57209653)(368.80456663,404.48709661)(368.64457153,404.47710297)
\curveto(368.48456695,404.46709663)(368.30456713,404.46209664)(368.10457153,404.46210297)
\lineto(367.96957153,404.46210297)
\curveto(367.92956751,404.47209663)(367.89456754,404.47209663)(367.86457153,404.46210297)
\curveto(367.82456761,404.45209665)(367.78956765,404.44709665)(367.75957153,404.44710297)
\curveto(367.72956771,404.45709664)(367.69956774,404.45209665)(367.66957153,404.43210297)
\curveto(367.58956785,404.41209669)(367.52956791,404.36709673)(367.48957153,404.29710297)
\curveto(367.45956798,404.23709686)(367.434568,404.16209694)(367.41457153,404.07210297)
\curveto(367.40456803,404.02209708)(367.40456803,403.96709713)(367.41457153,403.90710297)
\curveto(367.42456801,403.84709725)(367.42456801,403.79209731)(367.41457153,403.74210297)
\lineto(367.41457153,402.81210297)
\lineto(367.41457153,401.05710297)
\curveto(367.41456802,400.80710029)(367.41956802,400.58710051)(367.42957153,400.39710297)
\curveto(367.44956799,400.21710088)(367.51456792,400.05710104)(367.62457153,399.91710297)
\curveto(367.67456776,399.85710124)(367.7395677,399.81210129)(367.81957153,399.78210297)
\lineto(368.08957153,399.72210297)
\curveto(368.11956732,399.71210139)(368.14956729,399.70710139)(368.17957153,399.70710297)
\curveto(368.21956722,399.71710138)(368.24956719,399.71710138)(368.26957153,399.70710297)
\lineto(368.43457153,399.70710297)
\curveto(368.54456689,399.70710139)(368.6395668,399.7021014)(368.71957153,399.69210297)
\curveto(368.79956664,399.68210142)(368.86456657,399.64210146)(368.91457153,399.57210297)
\curveto(368.95456648,399.51210159)(368.97456646,399.43210167)(368.97457153,399.33210297)
\lineto(368.97457153,399.04710297)
\curveto(368.97456646,398.83710226)(368.96956647,398.64210246)(368.95957153,398.46210297)
\curveto(368.95956648,398.29210281)(368.87956656,398.17710292)(368.71957153,398.11710297)
\curveto(368.66956677,398.097103)(368.62456681,398.09210301)(368.58457153,398.10210297)
\curveto(368.54456689,398.102103)(368.49956694,398.09210301)(368.44957153,398.07210297)
\lineto(368.29957153,398.07210297)
\curveto(368.27956716,398.07210303)(368.24956719,398.07710302)(368.20957153,398.08710297)
\curveto(368.16956727,398.08710301)(368.1345673,398.08210302)(368.10457153,398.07210297)
\curveto(368.05456738,398.06210304)(367.99956744,398.06210304)(367.93957153,398.07210297)
\lineto(367.78957153,398.07210297)
\lineto(367.63957153,398.07210297)
\curveto(367.58956785,398.06210304)(367.54456789,398.06210304)(367.50457153,398.07210297)
\lineto(367.33957153,398.07210297)
\curveto(367.28956815,398.08210302)(367.2345682,398.08710301)(367.17457153,398.08710297)
\curveto(367.11456832,398.08710301)(367.05956838,398.09210301)(367.00957153,398.10210297)
\curveto(366.9395685,398.11210299)(366.87456856,398.12210298)(366.81457153,398.13210297)
\lineto(366.63457153,398.16210297)
\curveto(366.52456891,398.19210291)(366.41956902,398.22710287)(366.31957153,398.26710297)
\curveto(366.21956922,398.30710279)(366.12456931,398.35210275)(366.03457153,398.40210297)
\lineto(365.94457153,398.46210297)
\curveto(365.91456952,398.49210261)(365.87956956,398.52210258)(365.83957153,398.55210297)
\curveto(365.81956962,398.57210253)(365.79456964,398.59210251)(365.76457153,398.61210297)
\lineto(365.68957153,398.68710297)
\curveto(365.54956989,398.87710222)(365.44456999,399.08710201)(365.37457153,399.31710297)
\curveto(365.35457008,399.35710174)(365.34457009,399.39210171)(365.34457153,399.42210297)
\curveto(365.35457008,399.46210164)(365.35457008,399.50710159)(365.34457153,399.55710297)
\curveto(365.3345701,399.57710152)(365.32957011,399.6021015)(365.32957153,399.63210297)
\curveto(365.32957011,399.66210144)(365.32457011,399.68710141)(365.31457153,399.70710297)
\lineto(365.31457153,399.85710297)
\curveto(365.30457013,399.8971012)(365.29957014,399.94210116)(365.29957153,399.99210297)
\curveto(365.30957013,400.04210106)(365.31457012,400.09210101)(365.31457153,400.14210297)
\lineto(365.31457153,400.71210297)
\lineto(365.31457153,402.94710297)
\lineto(365.31457153,403.74210297)
\lineto(365.31457153,403.95210297)
\curveto(365.32457011,404.02209708)(365.31957012,404.08709701)(365.29957153,404.14710297)
\curveto(365.25957018,404.28709681)(365.18957025,404.37709672)(365.08957153,404.41710297)
\curveto(364.97957046,404.46709663)(364.8395706,404.48209662)(364.66957153,404.46210297)
\curveto(364.49957094,404.44209666)(364.35457108,404.45709664)(364.23457153,404.50710297)
\curveto(364.15457128,404.53709656)(364.10457133,404.58209652)(364.08457153,404.64210297)
\curveto(364.06457137,404.7020964)(364.04457139,404.77709632)(364.02457153,404.86710297)
\lineto(364.02457153,405.18210297)
\curveto(364.02457141,405.36209574)(364.0345714,405.50709559)(364.05457153,405.61710297)
\curveto(364.07457136,405.72709537)(364.15957128,405.8020953)(364.30957153,405.84210297)
\curveto(364.34957109,405.86209524)(364.38957105,405.86709523)(364.42957153,405.85710297)
\lineto(364.56457153,405.85710297)
\curveto(364.71457072,405.85709524)(364.85457058,405.86209524)(364.98457153,405.87210297)
\curveto(365.11457032,405.89209521)(365.20457023,405.95209515)(365.25457153,406.05210297)
\curveto(365.28457015,406.12209498)(365.29957014,406.2020949)(365.29957153,406.29210297)
\curveto(365.30957013,406.38209472)(365.31457012,406.47209463)(365.31457153,406.56210297)
\lineto(365.31457153,407.49210297)
\lineto(365.31457153,407.74710297)
\curveto(365.31457012,407.83709326)(365.32457011,407.91209319)(365.34457153,407.97210297)
\curveto(365.39457004,408.07209303)(365.46956997,408.13709296)(365.56957153,408.16710297)
\curveto(365.58956985,408.17709292)(365.61456982,408.17709292)(365.64457153,408.16710297)
\curveto(365.68456975,408.16709293)(365.71456972,408.17209293)(365.73457153,408.18210297)
}
}
{
\newrgbcolor{curcolor}{0 0 0}
\pscustom[linestyle=none,fillstyle=solid,fillcolor=curcolor]
{
\newpath
\moveto(372.05800903,408.72210297)
\curveto(372.12800608,408.64209246)(372.16300605,408.52209258)(372.16300903,408.36210297)
\lineto(372.16300903,407.89710297)
\lineto(372.16300903,407.49210297)
\curveto(372.16300605,407.35209375)(372.12800608,407.25709384)(372.05800903,407.20710297)
\curveto(371.99800621,407.15709394)(371.91800629,407.12709397)(371.81800903,407.11710297)
\curveto(371.72800648,407.10709399)(371.62800658,407.102094)(371.51800903,407.10210297)
\lineto(370.67800903,407.10210297)
\curveto(370.56800764,407.102094)(370.46800774,407.10709399)(370.37800903,407.11710297)
\curveto(370.29800791,407.12709397)(370.22800798,407.15709394)(370.16800903,407.20710297)
\curveto(370.12800808,407.23709386)(370.09800811,407.29209381)(370.07800903,407.37210297)
\curveto(370.06800814,407.46209364)(370.05800815,407.55709354)(370.04800903,407.65710297)
\lineto(370.04800903,407.98710297)
\curveto(370.05800815,408.097093)(370.06300815,408.19209291)(370.06300903,408.27210297)
\lineto(370.06300903,408.48210297)
\curveto(370.07300814,408.55209255)(370.09300812,408.61209249)(370.12300903,408.66210297)
\curveto(370.14300807,408.7020924)(370.16800804,408.73209237)(370.19800903,408.75210297)
\lineto(370.31800903,408.81210297)
\curveto(370.33800787,408.81209229)(370.36300785,408.81209229)(370.39300903,408.81210297)
\curveto(370.42300779,408.82209228)(370.44800776,408.82709227)(370.46800903,408.82710297)
\lineto(371.56300903,408.82710297)
\curveto(371.66300655,408.82709227)(371.75800645,408.82209228)(371.84800903,408.81210297)
\curveto(371.93800627,408.8020923)(372.0080062,408.77209233)(372.05800903,408.72210297)
\moveto(372.16300903,398.95710297)
\curveto(372.16300605,398.75710234)(372.15800605,398.58710251)(372.14800903,398.44710297)
\curveto(372.13800607,398.30710279)(372.04800616,398.21210289)(371.87800903,398.16210297)
\curveto(371.81800639,398.14210296)(371.75300646,398.13210297)(371.68300903,398.13210297)
\curveto(371.6130066,398.14210296)(371.53800667,398.14710295)(371.45800903,398.14710297)
\lineto(370.61800903,398.14710297)
\curveto(370.52800768,398.14710295)(370.43800777,398.15210295)(370.34800903,398.16210297)
\curveto(370.26800794,398.17210293)(370.208008,398.2021029)(370.16800903,398.25210297)
\curveto(370.1080081,398.32210278)(370.07300814,398.40710269)(370.06300903,398.50710297)
\lineto(370.06300903,398.85210297)
\lineto(370.06300903,405.18210297)
\lineto(370.06300903,405.48210297)
\curveto(370.06300815,405.58209552)(370.08300813,405.66209544)(370.12300903,405.72210297)
\curveto(370.18300803,405.79209531)(370.26800794,405.83709526)(370.37800903,405.85710297)
\curveto(370.39800781,405.86709523)(370.42300779,405.86709523)(370.45300903,405.85710297)
\curveto(370.49300772,405.85709524)(370.52300769,405.86209524)(370.54300903,405.87210297)
\lineto(371.29300903,405.87210297)
\lineto(371.48800903,405.87210297)
\curveto(371.56800664,405.88209522)(371.63300658,405.88209522)(371.68300903,405.87210297)
\lineto(371.80300903,405.87210297)
\curveto(371.86300635,405.85209525)(371.91800629,405.83709526)(371.96800903,405.82710297)
\curveto(372.01800619,405.81709528)(372.05800615,405.78709531)(372.08800903,405.73710297)
\curveto(372.12800608,405.68709541)(372.14800606,405.61709548)(372.14800903,405.52710297)
\curveto(372.15800605,405.43709566)(372.16300605,405.34209576)(372.16300903,405.24210297)
\lineto(372.16300903,398.95710297)
}
}
{
\newrgbcolor{curcolor}{0 0 0}
\pscustom[linestyle=none,fillstyle=solid,fillcolor=curcolor]
{
\newpath
\moveto(381.71519653,402.09210297)
\curveto(381.72518785,402.03209907)(381.73018785,401.94209916)(381.73019653,401.82210297)
\curveto(381.73018785,401.7020994)(381.72018786,401.61709948)(381.70019653,401.56710297)
\lineto(381.70019653,401.37210297)
\curveto(381.67018791,401.26209984)(381.65018793,401.15709994)(381.64019653,401.05710297)
\curveto(381.64018794,400.95710014)(381.62518795,400.85710024)(381.59519653,400.75710297)
\curveto(381.575188,400.66710043)(381.55518802,400.57210053)(381.53519653,400.47210297)
\curveto(381.51518806,400.38210072)(381.48518809,400.29210081)(381.44519653,400.20210297)
\curveto(381.3751882,400.03210107)(381.30518827,399.87210123)(381.23519653,399.72210297)
\curveto(381.16518841,399.58210152)(381.08518849,399.44210166)(380.99519653,399.30210297)
\curveto(380.93518864,399.21210189)(380.87018871,399.12710197)(380.80019653,399.04710297)
\curveto(380.74018884,398.97710212)(380.67018891,398.9021022)(380.59019653,398.82210297)
\lineto(380.48519653,398.71710297)
\curveto(380.43518914,398.66710243)(380.3801892,398.62210248)(380.32019653,398.58210297)
\lineto(380.17019653,398.46210297)
\curveto(380.09018949,398.4021027)(380.00018958,398.34710275)(379.90019653,398.29710297)
\curveto(379.81018977,398.25710284)(379.71518986,398.21210289)(379.61519653,398.16210297)
\curveto(379.51519006,398.11210299)(379.41019017,398.07710302)(379.30019653,398.05710297)
\curveto(379.20019038,398.03710306)(379.09519048,398.01710308)(378.98519653,397.99710297)
\curveto(378.92519065,397.97710312)(378.86019072,397.96710313)(378.79019653,397.96710297)
\curveto(378.73019085,397.96710313)(378.66519091,397.95710314)(378.59519653,397.93710297)
\lineto(378.46019653,397.93710297)
\curveto(378.3801912,397.91710318)(378.30519127,397.91710318)(378.23519653,397.93710297)
\lineto(378.08519653,397.93710297)
\curveto(378.02519155,397.95710314)(377.96019162,397.96710313)(377.89019653,397.96710297)
\curveto(377.83019175,397.95710314)(377.77019181,397.96210314)(377.71019653,397.98210297)
\curveto(377.55019203,398.03210307)(377.39519218,398.07710302)(377.24519653,398.11710297)
\curveto(377.10519247,398.15710294)(376.9751926,398.21710288)(376.85519653,398.29710297)
\curveto(376.78519279,398.33710276)(376.72019286,398.37710272)(376.66019653,398.41710297)
\curveto(376.60019298,398.46710263)(376.53519304,398.51710258)(376.46519653,398.56710297)
\lineto(376.28519653,398.70210297)
\curveto(376.20519337,398.76210234)(376.13519344,398.76710233)(376.07519653,398.71710297)
\curveto(376.02519355,398.68710241)(376.00019358,398.64710245)(376.00019653,398.59710297)
\curveto(376.00019358,398.55710254)(375.99019359,398.50710259)(375.97019653,398.44710297)
\curveto(375.95019363,398.34710275)(375.94019364,398.23210287)(375.94019653,398.10210297)
\curveto(375.95019363,397.97210313)(375.95519362,397.85210325)(375.95519653,397.74210297)
\lineto(375.95519653,396.21210297)
\curveto(375.95519362,396.08210502)(375.95019363,395.95710514)(375.94019653,395.83710297)
\curveto(375.94019364,395.70710539)(375.91519366,395.6021055)(375.86519653,395.52210297)
\curveto(375.83519374,395.48210562)(375.7801938,395.45210565)(375.70019653,395.43210297)
\curveto(375.62019396,395.41210569)(375.53019405,395.4021057)(375.43019653,395.40210297)
\curveto(375.33019425,395.39210571)(375.23019435,395.39210571)(375.13019653,395.40210297)
\lineto(374.87519653,395.40210297)
\lineto(374.47019653,395.40210297)
\lineto(374.36519653,395.40210297)
\curveto(374.32519525,395.4021057)(374.29019529,395.40710569)(374.26019653,395.41710297)
\lineto(374.14019653,395.41710297)
\curveto(373.97019561,395.46710563)(373.8801957,395.56710553)(373.87019653,395.71710297)
\curveto(373.86019572,395.85710524)(373.85519572,396.02710507)(373.85519653,396.22710297)
\lineto(373.85519653,405.03210297)
\curveto(373.85519572,405.14209596)(373.85019573,405.25709584)(373.84019653,405.37710297)
\curveto(373.84019574,405.50709559)(373.86519571,405.60709549)(373.91519653,405.67710297)
\curveto(373.95519562,405.74709535)(374.01019557,405.79209531)(374.08019653,405.81210297)
\curveto(374.13019545,405.83209527)(374.19019539,405.84209526)(374.26019653,405.84210297)
\lineto(374.48519653,405.84210297)
\lineto(375.20519653,405.84210297)
\lineto(375.49019653,405.84210297)
\curveto(375.580194,405.84209526)(375.65519392,405.81709528)(375.71519653,405.76710297)
\curveto(375.78519379,405.71709538)(375.82019376,405.65209545)(375.82019653,405.57210297)
\curveto(375.83019375,405.5020956)(375.85519372,405.42709567)(375.89519653,405.34710297)
\curveto(375.90519367,405.31709578)(375.91519366,405.29209581)(375.92519653,405.27210297)
\curveto(375.94519363,405.26209584)(375.96519361,405.24709585)(375.98519653,405.22710297)
\curveto(376.09519348,405.21709588)(376.18519339,405.24709585)(376.25519653,405.31710297)
\curveto(376.32519325,405.38709571)(376.39519318,405.44709565)(376.46519653,405.49710297)
\curveto(376.59519298,405.58709551)(376.73019285,405.66709543)(376.87019653,405.73710297)
\curveto(377.01019257,405.81709528)(377.16519241,405.88209522)(377.33519653,405.93210297)
\curveto(377.41519216,405.96209514)(377.50019208,405.98209512)(377.59019653,405.99210297)
\curveto(377.69019189,406.0020951)(377.78519179,406.01709508)(377.87519653,406.03710297)
\curveto(377.91519166,406.04709505)(377.95519162,406.04709505)(377.99519653,406.03710297)
\curveto(378.04519153,406.02709507)(378.08519149,406.03209507)(378.11519653,406.05210297)
\curveto(378.68519089,406.07209503)(379.16519041,405.99209511)(379.55519653,405.81210297)
\curveto(379.95518962,405.64209546)(380.29518928,405.41709568)(380.57519653,405.13710297)
\curveto(380.62518895,405.08709601)(380.67018891,405.03709606)(380.71019653,404.98710297)
\curveto(380.75018883,404.94709615)(380.79018879,404.9020962)(380.83019653,404.85210297)
\curveto(380.90018868,404.76209634)(380.96018862,404.67209643)(381.01019653,404.58210297)
\curveto(381.07018851,404.49209661)(381.12518845,404.4020967)(381.17519653,404.31210297)
\curveto(381.19518838,404.29209681)(381.20518837,404.26709683)(381.20519653,404.23710297)
\curveto(381.21518836,404.20709689)(381.23018835,404.17209693)(381.25019653,404.13210297)
\curveto(381.31018827,404.03209707)(381.36518821,403.91209719)(381.41519653,403.77210297)
\curveto(381.43518814,403.71209739)(381.45518812,403.64709745)(381.47519653,403.57710297)
\curveto(381.49518808,403.51709758)(381.51518806,403.45209765)(381.53519653,403.38210297)
\curveto(381.575188,403.26209784)(381.60018798,403.13709796)(381.61019653,403.00710297)
\curveto(381.63018795,402.87709822)(381.65518792,402.74209836)(381.68519653,402.60210297)
\lineto(381.68519653,402.43710297)
\lineto(381.71519653,402.25710297)
\lineto(381.71519653,402.09210297)
\moveto(379.60019653,401.74710297)
\curveto(379.61018997,401.7970993)(379.61518996,401.86209924)(379.61519653,401.94210297)
\curveto(379.61518996,402.03209907)(379.61018997,402.102099)(379.60019653,402.15210297)
\lineto(379.60019653,402.28710297)
\curveto(379.58019,402.34709875)(379.57019001,402.41209869)(379.57019653,402.48210297)
\curveto(379.57019001,402.55209855)(379.56019002,402.62209848)(379.54019653,402.69210297)
\curveto(379.52019006,402.79209831)(379.50019008,402.88709821)(379.48019653,402.97710297)
\curveto(379.46019012,403.07709802)(379.43019015,403.16709793)(379.39019653,403.24710297)
\curveto(379.27019031,403.56709753)(379.11519046,403.82209728)(378.92519653,404.01210297)
\curveto(378.73519084,404.2020969)(378.46519111,404.34209676)(378.11519653,404.43210297)
\curveto(378.03519154,404.45209665)(377.94519163,404.46209664)(377.84519653,404.46210297)
\lineto(377.57519653,404.46210297)
\curveto(377.53519204,404.45209665)(377.50019208,404.44709665)(377.47019653,404.44710297)
\curveto(377.44019214,404.44709665)(377.40519217,404.44209666)(377.36519653,404.43210297)
\lineto(377.15519653,404.37210297)
\curveto(377.09519248,404.36209674)(377.03519254,404.34209676)(376.97519653,404.31210297)
\curveto(376.71519286,404.2020969)(376.51019307,404.03209707)(376.36019653,403.80210297)
\curveto(376.22019336,403.57209753)(376.10519347,403.31709778)(376.01519653,403.03710297)
\curveto(375.99519358,402.95709814)(375.9801936,402.87209823)(375.97019653,402.78210297)
\curveto(375.96019362,402.7020984)(375.94519363,402.62209848)(375.92519653,402.54210297)
\curveto(375.91519366,402.5020986)(375.91019367,402.43709866)(375.91019653,402.34710297)
\curveto(375.89019369,402.30709879)(375.88519369,402.25709884)(375.89519653,402.19710297)
\curveto(375.90519367,402.14709895)(375.90519367,402.097099)(375.89519653,402.04710297)
\curveto(375.8751937,401.98709911)(375.8751937,401.93209917)(375.89519653,401.88210297)
\lineto(375.89519653,401.70210297)
\lineto(375.89519653,401.56710297)
\curveto(375.89519368,401.52709957)(375.90519367,401.48709961)(375.92519653,401.44710297)
\curveto(375.92519365,401.37709972)(375.93019365,401.32209978)(375.94019653,401.28210297)
\lineto(375.97019653,401.10210297)
\curveto(375.9801936,401.04210006)(375.99519358,400.98210012)(376.01519653,400.92210297)
\curveto(376.10519347,400.63210047)(376.21019337,400.39210071)(376.33019653,400.20210297)
\curveto(376.46019312,400.02210108)(376.64019294,399.86210124)(376.87019653,399.72210297)
\curveto(377.01019257,399.64210146)(377.1751924,399.57710152)(377.36519653,399.52710297)
\curveto(377.40519217,399.51710158)(377.44019214,399.51210159)(377.47019653,399.51210297)
\curveto(377.50019208,399.52210158)(377.53519204,399.52210158)(377.57519653,399.51210297)
\curveto(377.61519196,399.5021016)(377.6751919,399.49210161)(377.75519653,399.48210297)
\curveto(377.83519174,399.48210162)(377.90019168,399.48710161)(377.95019653,399.49710297)
\curveto(378.03019155,399.51710158)(378.11019147,399.53210157)(378.19019653,399.54210297)
\curveto(378.2801913,399.56210154)(378.36519121,399.58710151)(378.44519653,399.61710297)
\curveto(378.68519089,399.71710138)(378.8801907,399.85710124)(379.03019653,400.03710297)
\curveto(379.1801904,400.21710088)(379.30519027,400.42710067)(379.40519653,400.66710297)
\curveto(379.45519012,400.78710031)(379.49019009,400.91210019)(379.51019653,401.04210297)
\curveto(379.53019005,401.17209993)(379.55519002,401.30709979)(379.58519653,401.44710297)
\lineto(379.58519653,401.59710297)
\curveto(379.59518998,401.64709945)(379.60018998,401.6970994)(379.60019653,401.74710297)
}
}
{
\newrgbcolor{curcolor}{0 0 0}
\pscustom[linestyle=none,fillstyle=solid,fillcolor=curcolor]
{
\newpath
\moveto(390.76511841,402.31710297)
\curveto(390.78510984,402.25709884)(390.79510983,402.17209893)(390.79511841,402.06210297)
\curveto(390.79510983,401.95209915)(390.78510984,401.86709923)(390.76511841,401.80710297)
\lineto(390.76511841,401.65710297)
\curveto(390.74510988,401.57709952)(390.73510989,401.4970996)(390.73511841,401.41710297)
\curveto(390.74510988,401.33709976)(390.74010988,401.25709984)(390.72011841,401.17710297)
\curveto(390.70010992,401.10709999)(390.68510994,401.04210006)(390.67511841,400.98210297)
\curveto(390.66510996,400.92210018)(390.65510997,400.85710024)(390.64511841,400.78710297)
\curveto(390.60511002,400.67710042)(390.57011005,400.56210054)(390.54011841,400.44210297)
\curveto(390.51011011,400.33210077)(390.47011015,400.22710087)(390.42011841,400.12710297)
\curveto(390.21011041,399.64710145)(389.93511069,399.25710184)(389.59511841,398.95710297)
\curveto(389.25511137,398.65710244)(388.84511178,398.40710269)(388.36511841,398.20710297)
\curveto(388.24511238,398.15710294)(388.1201125,398.12210298)(387.99011841,398.10210297)
\curveto(387.87011275,398.07210303)(387.74511288,398.04210306)(387.61511841,398.01210297)
\curveto(387.56511306,397.99210311)(387.51011311,397.98210312)(387.45011841,397.98210297)
\curveto(387.39011323,397.98210312)(387.33511329,397.97710312)(387.28511841,397.96710297)
\lineto(387.18011841,397.96710297)
\curveto(387.15011347,397.95710314)(387.1201135,397.95210315)(387.09011841,397.95210297)
\curveto(387.04011358,397.94210316)(386.96011366,397.93710316)(386.85011841,397.93710297)
\curveto(386.74011388,397.92710317)(386.65511397,397.93210317)(386.59511841,397.95210297)
\lineto(386.44511841,397.95210297)
\curveto(386.39511423,397.96210314)(386.34011428,397.96710313)(386.28011841,397.96710297)
\curveto(386.23011439,397.95710314)(386.18011444,397.96210314)(386.13011841,397.98210297)
\curveto(386.09011453,397.99210311)(386.05011457,397.9971031)(386.01011841,397.99710297)
\curveto(385.98011464,397.9971031)(385.94011468,398.0021031)(385.89011841,398.01210297)
\curveto(385.79011483,398.04210306)(385.69011493,398.06710303)(385.59011841,398.08710297)
\curveto(385.49011513,398.10710299)(385.39511523,398.13710296)(385.30511841,398.17710297)
\curveto(385.18511544,398.21710288)(385.07011555,398.25710284)(384.96011841,398.29710297)
\curveto(384.86011576,398.33710276)(384.75511587,398.38710271)(384.64511841,398.44710297)
\curveto(384.29511633,398.65710244)(383.99511663,398.9021022)(383.74511841,399.18210297)
\curveto(383.49511713,399.46210164)(383.28511734,399.7971013)(383.11511841,400.18710297)
\curveto(383.06511756,400.27710082)(383.0251176,400.37210073)(382.99511841,400.47210297)
\curveto(382.97511765,400.57210053)(382.95011767,400.67710042)(382.92011841,400.78710297)
\curveto(382.90011772,400.83710026)(382.89011773,400.88210022)(382.89011841,400.92210297)
\curveto(382.89011773,400.96210014)(382.88011774,401.00710009)(382.86011841,401.05710297)
\curveto(382.84011778,401.13709996)(382.83011779,401.21709988)(382.83011841,401.29710297)
\curveto(382.83011779,401.38709971)(382.8201178,401.47209963)(382.80011841,401.55210297)
\curveto(382.79011783,401.6020995)(382.78511784,401.64709945)(382.78511841,401.68710297)
\lineto(382.78511841,401.82210297)
\curveto(382.76511786,401.88209922)(382.75511787,401.96709913)(382.75511841,402.07710297)
\curveto(382.76511786,402.18709891)(382.78011784,402.27209883)(382.80011841,402.33210297)
\lineto(382.80011841,402.43710297)
\curveto(382.81011781,402.48709861)(382.81011781,402.53709856)(382.80011841,402.58710297)
\curveto(382.80011782,402.64709845)(382.81011781,402.7020984)(382.83011841,402.75210297)
\curveto(382.84011778,402.8020983)(382.84511778,402.84709825)(382.84511841,402.88710297)
\curveto(382.84511778,402.93709816)(382.85511777,402.98709811)(382.87511841,403.03710297)
\curveto(382.91511771,403.16709793)(382.95011767,403.29209781)(382.98011841,403.41210297)
\curveto(383.01011761,403.54209756)(383.05011757,403.66709743)(383.10011841,403.78710297)
\curveto(383.28011734,404.1970969)(383.49511713,404.53709656)(383.74511841,404.80710297)
\curveto(383.99511663,405.08709601)(384.30011632,405.34209576)(384.66011841,405.57210297)
\curveto(384.76011586,405.62209548)(384.86511576,405.66709543)(384.97511841,405.70710297)
\curveto(385.08511554,405.74709535)(385.19511543,405.79209531)(385.30511841,405.84210297)
\curveto(385.43511519,405.89209521)(385.57011505,405.92709517)(385.71011841,405.94710297)
\curveto(385.85011477,405.96709513)(385.99511463,405.9970951)(386.14511841,406.03710297)
\curveto(386.2251144,406.04709505)(386.30011432,406.05209505)(386.37011841,406.05210297)
\curveto(386.44011418,406.05209505)(386.51011411,406.05709504)(386.58011841,406.06710297)
\curveto(387.16011346,406.07709502)(387.66011296,406.01709508)(388.08011841,405.88710297)
\curveto(388.51011211,405.75709534)(388.89011173,405.57709552)(389.22011841,405.34710297)
\curveto(389.33011129,405.26709583)(389.44011118,405.17709592)(389.55011841,405.07710297)
\curveto(389.67011095,404.98709611)(389.77011085,404.88709621)(389.85011841,404.77710297)
\curveto(389.93011069,404.67709642)(390.00011062,404.57709652)(390.06011841,404.47710297)
\curveto(390.13011049,404.37709672)(390.20011042,404.27209683)(390.27011841,404.16210297)
\curveto(390.34011028,404.05209705)(390.39511023,403.93209717)(390.43511841,403.80210297)
\curveto(390.47511015,403.68209742)(390.5201101,403.55209755)(390.57011841,403.41210297)
\curveto(390.60011002,403.33209777)(390.62511,403.24709785)(390.64511841,403.15710297)
\lineto(390.70511841,402.88710297)
\curveto(390.71510991,402.84709825)(390.7201099,402.80709829)(390.72011841,402.76710297)
\curveto(390.7201099,402.72709837)(390.7251099,402.68709841)(390.73511841,402.64710297)
\curveto(390.75510987,402.5970985)(390.76010986,402.54209856)(390.75011841,402.48210297)
\curveto(390.74010988,402.42209868)(390.74510988,402.36709873)(390.76511841,402.31710297)
\moveto(388.66511841,401.77710297)
\curveto(388.67511195,401.82709927)(388.68011194,401.8970992)(388.68011841,401.98710297)
\curveto(388.68011194,402.08709901)(388.67511195,402.16209894)(388.66511841,402.21210297)
\lineto(388.66511841,402.33210297)
\curveto(388.64511198,402.38209872)(388.63511199,402.43709866)(388.63511841,402.49710297)
\curveto(388.63511199,402.55709854)(388.63011199,402.61209849)(388.62011841,402.66210297)
\curveto(388.620112,402.7020984)(388.61511201,402.73209837)(388.60511841,402.75210297)
\lineto(388.54511841,402.99210297)
\curveto(388.53511209,403.08209802)(388.51511211,403.16709793)(388.48511841,403.24710297)
\curveto(388.37511225,403.50709759)(388.24511238,403.72709737)(388.09511841,403.90710297)
\curveto(387.94511268,404.097097)(387.74511288,404.24709685)(387.49511841,404.35710297)
\curveto(387.43511319,404.37709672)(387.37511325,404.39209671)(387.31511841,404.40210297)
\curveto(387.25511337,404.42209668)(387.19011343,404.44209666)(387.12011841,404.46210297)
\curveto(387.04011358,404.48209662)(386.95511367,404.48709661)(386.86511841,404.47710297)
\lineto(386.59511841,404.47710297)
\curveto(386.56511406,404.45709664)(386.53011409,404.44709665)(386.49011841,404.44710297)
\curveto(386.45011417,404.45709664)(386.41511421,404.45709664)(386.38511841,404.44710297)
\lineto(386.17511841,404.38710297)
\curveto(386.11511451,404.37709672)(386.06011456,404.35709674)(386.01011841,404.32710297)
\curveto(385.76011486,404.21709688)(385.55511507,404.05709704)(385.39511841,403.84710297)
\curveto(385.24511538,403.64709745)(385.1251155,403.41209769)(385.03511841,403.14210297)
\curveto(385.00511562,403.04209806)(384.98011564,402.93709816)(384.96011841,402.82710297)
\curveto(384.95011567,402.71709838)(384.93511569,402.60709849)(384.91511841,402.49710297)
\curveto(384.90511572,402.44709865)(384.90011572,402.3970987)(384.90011841,402.34710297)
\lineto(384.90011841,402.19710297)
\curveto(384.88011574,402.12709897)(384.87011575,402.02209908)(384.87011841,401.88210297)
\curveto(384.88011574,401.74209936)(384.89511573,401.63709946)(384.91511841,401.56710297)
\lineto(384.91511841,401.43210297)
\curveto(384.93511569,401.35209975)(384.95011567,401.27209983)(384.96011841,401.19210297)
\curveto(384.97011565,401.12209998)(384.98511564,401.04710005)(385.00511841,400.96710297)
\curveto(385.10511552,400.66710043)(385.21011541,400.42210068)(385.32011841,400.23210297)
\curveto(385.44011518,400.05210105)(385.625115,399.88710121)(385.87511841,399.73710297)
\curveto(385.94511468,399.68710141)(386.0201146,399.64710145)(386.10011841,399.61710297)
\curveto(386.19011443,399.58710151)(386.28011434,399.56210154)(386.37011841,399.54210297)
\curveto(386.41011421,399.53210157)(386.44511418,399.52710157)(386.47511841,399.52710297)
\curveto(386.50511412,399.53710156)(386.54011408,399.53710156)(386.58011841,399.52710297)
\lineto(386.70011841,399.49710297)
\curveto(386.75011387,399.4971016)(386.79511383,399.5021016)(386.83511841,399.51210297)
\lineto(386.95511841,399.51210297)
\curveto(387.03511359,399.53210157)(387.11511351,399.54710155)(387.19511841,399.55710297)
\curveto(387.27511335,399.56710153)(387.35011327,399.58710151)(387.42011841,399.61710297)
\curveto(387.68011294,399.71710138)(387.89011273,399.85210125)(388.05011841,400.02210297)
\curveto(388.21011241,400.19210091)(388.34511228,400.4021007)(388.45511841,400.65210297)
\curveto(388.49511213,400.75210035)(388.5251121,400.85210025)(388.54511841,400.95210297)
\curveto(388.56511206,401.05210005)(388.59011203,401.15709994)(388.62011841,401.26710297)
\curveto(388.63011199,401.30709979)(388.63511199,401.34209976)(388.63511841,401.37210297)
\curveto(388.63511199,401.41209969)(388.64011198,401.45209965)(388.65011841,401.49210297)
\lineto(388.65011841,401.62710297)
\curveto(388.65011197,401.67709942)(388.65511197,401.72709937)(388.66511841,401.77710297)
}
}
{
\newrgbcolor{curcolor}{0 0 0}
\pscustom[linestyle=none,fillstyle=solid,fillcolor=curcolor]
{
}
}
{
\newrgbcolor{curcolor}{0 0 0}
\pscustom[linestyle=none,fillstyle=solid,fillcolor=curcolor]
{
\newpath
\moveto(403.91519653,398.98710297)
\lineto(403.91519653,398.56710297)
\curveto(403.91518816,398.43710266)(403.88518819,398.33210277)(403.82519653,398.25210297)
\curveto(403.7751883,398.2021029)(403.71018837,398.16710293)(403.63019653,398.14710297)
\curveto(403.55018853,398.13710296)(403.46018862,398.13210297)(403.36019653,398.13210297)
\lineto(402.53519653,398.13210297)
\lineto(402.25019653,398.13210297)
\curveto(402.17018991,398.14210296)(402.10518997,398.16710293)(402.05519653,398.20710297)
\curveto(401.98519009,398.25710284)(401.94519013,398.32210278)(401.93519653,398.40210297)
\curveto(401.92519015,398.48210262)(401.90519017,398.56210254)(401.87519653,398.64210297)
\curveto(401.85519022,398.66210244)(401.83519024,398.67710242)(401.81519653,398.68710297)
\curveto(401.80519027,398.70710239)(401.79019029,398.72710237)(401.77019653,398.74710297)
\curveto(401.66019042,398.74710235)(401.5801905,398.72210238)(401.53019653,398.67210297)
\lineto(401.38019653,398.52210297)
\curveto(401.31019077,398.47210263)(401.24519083,398.42710267)(401.18519653,398.38710297)
\curveto(401.12519095,398.35710274)(401.06019102,398.31710278)(400.99019653,398.26710297)
\curveto(400.95019113,398.24710285)(400.90519117,398.22710287)(400.85519653,398.20710297)
\curveto(400.81519126,398.18710291)(400.77019131,398.16710293)(400.72019653,398.14710297)
\curveto(400.5801915,398.097103)(400.43019165,398.05210305)(400.27019653,398.01210297)
\curveto(400.22019186,397.99210311)(400.1751919,397.98210312)(400.13519653,397.98210297)
\curveto(400.09519198,397.98210312)(400.05519202,397.97710312)(400.01519653,397.96710297)
\lineto(399.88019653,397.96710297)
\curveto(399.85019223,397.95710314)(399.81019227,397.95210315)(399.76019653,397.95210297)
\lineto(399.62519653,397.95210297)
\curveto(399.56519251,397.93210317)(399.4751926,397.92710317)(399.35519653,397.93710297)
\curveto(399.23519284,397.93710316)(399.15019293,397.94710315)(399.10019653,397.96710297)
\curveto(399.03019305,397.98710311)(398.96519311,397.9971031)(398.90519653,397.99710297)
\curveto(398.85519322,397.98710311)(398.80019328,397.99210311)(398.74019653,398.01210297)
\lineto(398.38019653,398.13210297)
\curveto(398.27019381,398.16210294)(398.16019392,398.2021029)(398.05019653,398.25210297)
\curveto(397.70019438,398.4021027)(397.38519469,398.63210247)(397.10519653,398.94210297)
\curveto(396.83519524,399.26210184)(396.62019546,399.5971015)(396.46019653,399.94710297)
\curveto(396.41019567,400.05710104)(396.37019571,400.16210094)(396.34019653,400.26210297)
\curveto(396.31019577,400.37210073)(396.2751958,400.48210062)(396.23519653,400.59210297)
\curveto(396.22519585,400.63210047)(396.22019586,400.66710043)(396.22019653,400.69710297)
\curveto(396.22019586,400.73710036)(396.21019587,400.78210032)(396.19019653,400.83210297)
\curveto(396.17019591,400.91210019)(396.15019593,400.9971001)(396.13019653,401.08710297)
\curveto(396.12019596,401.18709991)(396.10519597,401.28709981)(396.08519653,401.38710297)
\curveto(396.075196,401.41709968)(396.07019601,401.45209965)(396.07019653,401.49210297)
\curveto(396.080196,401.53209957)(396.080196,401.56709953)(396.07019653,401.59710297)
\lineto(396.07019653,401.73210297)
\curveto(396.07019601,401.78209932)(396.06519601,401.83209927)(396.05519653,401.88210297)
\curveto(396.04519603,401.93209917)(396.04019604,401.98709911)(396.04019653,402.04710297)
\curveto(396.04019604,402.11709898)(396.04519603,402.17209893)(396.05519653,402.21210297)
\curveto(396.06519601,402.26209884)(396.07019601,402.30709879)(396.07019653,402.34710297)
\lineto(396.07019653,402.49710297)
\curveto(396.080196,402.54709855)(396.080196,402.59209851)(396.07019653,402.63210297)
\curveto(396.07019601,402.68209842)(396.080196,402.73209837)(396.10019653,402.78210297)
\curveto(396.12019596,402.89209821)(396.13519594,402.9970981)(396.14519653,403.09710297)
\curveto(396.16519591,403.1970979)(396.19019589,403.2970978)(396.22019653,403.39710297)
\curveto(396.26019582,403.51709758)(396.29519578,403.63209747)(396.32519653,403.74210297)
\curveto(396.35519572,403.85209725)(396.39519568,403.96209714)(396.44519653,404.07210297)
\curveto(396.58519549,404.37209673)(396.76019532,404.65709644)(396.97019653,404.92710297)
\curveto(396.99019509,404.95709614)(397.01519506,404.98209612)(397.04519653,405.00210297)
\curveto(397.08519499,405.03209607)(397.11519496,405.06209604)(397.13519653,405.09210297)
\curveto(397.1751949,405.14209596)(397.21519486,405.18709591)(397.25519653,405.22710297)
\curveto(397.29519478,405.26709583)(397.34019474,405.30709579)(397.39019653,405.34710297)
\curveto(397.43019465,405.36709573)(397.46519461,405.39209571)(397.49519653,405.42210297)
\curveto(397.52519455,405.46209564)(397.56019452,405.49209561)(397.60019653,405.51210297)
\curveto(397.85019423,405.68209542)(398.14019394,405.82209528)(398.47019653,405.93210297)
\curveto(398.54019354,405.95209515)(398.61019347,405.96709513)(398.68019653,405.97710297)
\curveto(398.76019332,405.98709511)(398.84019324,406.0020951)(398.92019653,406.02210297)
\curveto(398.99019309,406.04209506)(399.080193,406.05209505)(399.19019653,406.05210297)
\curveto(399.30019278,406.06209504)(399.41019267,406.06709503)(399.52019653,406.06710297)
\curveto(399.63019245,406.06709503)(399.73519234,406.06209504)(399.83519653,406.05210297)
\curveto(399.94519213,406.04209506)(400.03519204,406.02709507)(400.10519653,406.00710297)
\curveto(400.25519182,405.95709514)(400.40019168,405.91209519)(400.54019653,405.87210297)
\curveto(400.6801914,405.83209527)(400.81019127,405.77709532)(400.93019653,405.70710297)
\curveto(401.00019108,405.65709544)(401.06519101,405.60709549)(401.12519653,405.55710297)
\curveto(401.18519089,405.51709558)(401.25019083,405.47209563)(401.32019653,405.42210297)
\curveto(401.36019072,405.39209571)(401.41519066,405.35209575)(401.48519653,405.30210297)
\curveto(401.56519051,405.25209585)(401.64019044,405.25209585)(401.71019653,405.30210297)
\curveto(401.75019033,405.32209578)(401.77019031,405.35709574)(401.77019653,405.40710297)
\curveto(401.77019031,405.45709564)(401.7801903,405.50709559)(401.80019653,405.55710297)
\lineto(401.80019653,405.70710297)
\curveto(401.81019027,405.73709536)(401.81519026,405.77209533)(401.81519653,405.81210297)
\lineto(401.81519653,405.93210297)
\lineto(401.81519653,407.97210297)
\curveto(401.81519026,408.08209302)(401.81019027,408.2020929)(401.80019653,408.33210297)
\curveto(401.80019028,408.47209263)(401.82519025,408.57709252)(401.87519653,408.64710297)
\curveto(401.91519016,408.72709237)(401.99019009,408.77709232)(402.10019653,408.79710297)
\curveto(402.12018996,408.80709229)(402.14018994,408.80709229)(402.16019653,408.79710297)
\curveto(402.1801899,408.7970923)(402.20018988,408.8020923)(402.22019653,408.81210297)
\lineto(403.28519653,408.81210297)
\curveto(403.40518867,408.81209229)(403.51518856,408.80709229)(403.61519653,408.79710297)
\curveto(403.71518836,408.78709231)(403.79018829,408.74709235)(403.84019653,408.67710297)
\curveto(403.89018819,408.5970925)(403.91518816,408.49209261)(403.91519653,408.36210297)
\lineto(403.91519653,408.00210297)
\lineto(403.91519653,398.98710297)
\moveto(401.87519653,401.92710297)
\curveto(401.88519019,401.96709913)(401.88519019,402.00709909)(401.87519653,402.04710297)
\lineto(401.87519653,402.18210297)
\curveto(401.8751902,402.28209882)(401.87019021,402.38209872)(401.86019653,402.48210297)
\curveto(401.85019023,402.58209852)(401.83519024,402.67209843)(401.81519653,402.75210297)
\curveto(401.79519028,402.86209824)(401.7751903,402.96209814)(401.75519653,403.05210297)
\curveto(401.74519033,403.14209796)(401.72019036,403.22709787)(401.68019653,403.30710297)
\curveto(401.54019054,403.66709743)(401.33519074,403.95209715)(401.06519653,404.16210297)
\curveto(400.80519127,404.37209673)(400.42519165,404.47709662)(399.92519653,404.47710297)
\curveto(399.86519221,404.47709662)(399.78519229,404.46709663)(399.68519653,404.44710297)
\curveto(399.60519247,404.42709667)(399.53019255,404.40709669)(399.46019653,404.38710297)
\curveto(399.40019268,404.37709672)(399.34019274,404.35709674)(399.28019653,404.32710297)
\curveto(399.01019307,404.21709688)(398.80019328,404.04709705)(398.65019653,403.81710297)
\curveto(398.50019358,403.58709751)(398.3801937,403.32709777)(398.29019653,403.03710297)
\curveto(398.26019382,402.93709816)(398.24019384,402.83709826)(398.23019653,402.73710297)
\curveto(398.22019386,402.63709846)(398.20019388,402.53209857)(398.17019653,402.42210297)
\lineto(398.17019653,402.21210297)
\curveto(398.15019393,402.12209898)(398.14519393,401.9970991)(398.15519653,401.83710297)
\curveto(398.16519391,401.68709941)(398.1801939,401.57709952)(398.20019653,401.50710297)
\lineto(398.20019653,401.41710297)
\curveto(398.21019387,401.3970997)(398.21519386,401.37709972)(398.21519653,401.35710297)
\curveto(398.23519384,401.27709982)(398.25019383,401.2020999)(398.26019653,401.13210297)
\curveto(398.2801938,401.06210004)(398.30019378,400.98710011)(398.32019653,400.90710297)
\curveto(398.49019359,400.38710071)(398.7801933,400.0021011)(399.19019653,399.75210297)
\curveto(399.32019276,399.66210144)(399.50019258,399.59210151)(399.73019653,399.54210297)
\curveto(399.77019231,399.53210157)(399.83019225,399.52710157)(399.91019653,399.52710297)
\curveto(399.94019214,399.51710158)(399.98519209,399.50710159)(400.04519653,399.49710297)
\curveto(400.11519196,399.4971016)(400.17019191,399.5021016)(400.21019653,399.51210297)
\curveto(400.29019179,399.53210157)(400.37019171,399.54710155)(400.45019653,399.55710297)
\curveto(400.53019155,399.56710153)(400.61019147,399.58710151)(400.69019653,399.61710297)
\curveto(400.94019114,399.72710137)(401.14019094,399.86710123)(401.29019653,400.03710297)
\curveto(401.44019064,400.20710089)(401.57019051,400.42210068)(401.68019653,400.68210297)
\curveto(401.72019036,400.77210033)(401.75019033,400.86210024)(401.77019653,400.95210297)
\curveto(401.79019029,401.05210005)(401.81019027,401.15709994)(401.83019653,401.26710297)
\curveto(401.84019024,401.31709978)(401.84019024,401.36209974)(401.83019653,401.40210297)
\curveto(401.83019025,401.45209965)(401.84019024,401.5020996)(401.86019653,401.55210297)
\curveto(401.87019021,401.58209952)(401.8751902,401.61709948)(401.87519653,401.65710297)
\lineto(401.87519653,401.79210297)
\lineto(401.87519653,401.92710297)
}
}
{
\newrgbcolor{curcolor}{0 0 0}
\pscustom[linestyle=none,fillstyle=solid,fillcolor=curcolor]
{
\newpath
\moveto(412.86011841,402.07710297)
\curveto(412.88011024,401.9970991)(412.88011024,401.90709919)(412.86011841,401.80710297)
\curveto(412.84011028,401.70709939)(412.80511032,401.64209946)(412.75511841,401.61210297)
\curveto(412.70511042,401.57209953)(412.63011049,401.54209956)(412.53011841,401.52210297)
\curveto(412.44011068,401.51209959)(412.33511079,401.5020996)(412.21511841,401.49210297)
\lineto(411.87011841,401.49210297)
\curveto(411.76011136,401.5020996)(411.66011146,401.50709959)(411.57011841,401.50710297)
\lineto(407.91011841,401.50710297)
\lineto(407.70011841,401.50710297)
\curveto(407.64011548,401.50709959)(407.58511554,401.4970996)(407.53511841,401.47710297)
\curveto(407.45511567,401.43709966)(407.40511572,401.3970997)(407.38511841,401.35710297)
\curveto(407.36511576,401.33709976)(407.34511578,401.2970998)(407.32511841,401.23710297)
\curveto(407.30511582,401.18709991)(407.30011582,401.13709996)(407.31011841,401.08710297)
\curveto(407.33011579,401.02710007)(407.34011578,400.96710013)(407.34011841,400.90710297)
\curveto(407.35011577,400.85710024)(407.36511576,400.8021003)(407.38511841,400.74210297)
\curveto(407.46511566,400.5021006)(407.56011556,400.3021008)(407.67011841,400.14210297)
\curveto(407.79011533,399.99210111)(407.95011517,399.85710124)(408.15011841,399.73710297)
\curveto(408.23011489,399.68710141)(408.31011481,399.65210145)(408.39011841,399.63210297)
\curveto(408.48011464,399.62210148)(408.57011455,399.6021015)(408.66011841,399.57210297)
\curveto(408.74011438,399.55210155)(408.85011427,399.53710156)(408.99011841,399.52710297)
\curveto(409.13011399,399.51710158)(409.25011387,399.52210158)(409.35011841,399.54210297)
\lineto(409.48511841,399.54210297)
\curveto(409.58511354,399.56210154)(409.67511345,399.58210152)(409.75511841,399.60210297)
\curveto(409.84511328,399.63210147)(409.93011319,399.66210144)(410.01011841,399.69210297)
\curveto(410.11011301,399.74210136)(410.2201129,399.80710129)(410.34011841,399.88710297)
\curveto(410.47011265,399.96710113)(410.56511256,400.04710105)(410.62511841,400.12710297)
\curveto(410.67511245,400.1971009)(410.7251124,400.26210084)(410.77511841,400.32210297)
\curveto(410.83511229,400.39210071)(410.90511222,400.44210066)(410.98511841,400.47210297)
\curveto(411.08511204,400.52210058)(411.21011191,400.54210056)(411.36011841,400.53210297)
\lineto(411.79511841,400.53210297)
\lineto(411.97511841,400.53210297)
\curveto(412.04511108,400.54210056)(412.10511102,400.53710056)(412.15511841,400.51710297)
\lineto(412.30511841,400.51710297)
\curveto(412.40511072,400.4971006)(412.47511065,400.47210063)(412.51511841,400.44210297)
\curveto(412.55511057,400.42210068)(412.57511055,400.37710072)(412.57511841,400.30710297)
\curveto(412.58511054,400.23710086)(412.58011054,400.17710092)(412.56011841,400.12710297)
\curveto(412.51011061,399.98710111)(412.45511067,399.86210124)(412.39511841,399.75210297)
\curveto(412.33511079,399.64210146)(412.26511086,399.53210157)(412.18511841,399.42210297)
\curveto(411.96511116,399.09210201)(411.71511141,398.82710227)(411.43511841,398.62710297)
\curveto(411.15511197,398.42710267)(410.80511232,398.25710284)(410.38511841,398.11710297)
\curveto(410.27511285,398.07710302)(410.16511296,398.05210305)(410.05511841,398.04210297)
\curveto(409.94511318,398.03210307)(409.83011329,398.01210309)(409.71011841,397.98210297)
\curveto(409.67011345,397.97210313)(409.6251135,397.97210313)(409.57511841,397.98210297)
\curveto(409.53511359,397.98210312)(409.49511363,397.97710312)(409.45511841,397.96710297)
\lineto(409.29011841,397.96710297)
\curveto(409.24011388,397.94710315)(409.18011394,397.94210316)(409.11011841,397.95210297)
\curveto(409.05011407,397.95210315)(408.99511413,397.95710314)(408.94511841,397.96710297)
\curveto(408.86511426,397.97710312)(408.79511433,397.97710312)(408.73511841,397.96710297)
\curveto(408.67511445,397.95710314)(408.61011451,397.96210314)(408.54011841,397.98210297)
\curveto(408.49011463,398.0021031)(408.43511469,398.01210309)(408.37511841,398.01210297)
\curveto(408.31511481,398.01210309)(408.26011486,398.02210308)(408.21011841,398.04210297)
\curveto(408.10011502,398.06210304)(407.99011513,398.08710301)(407.88011841,398.11710297)
\curveto(407.77011535,398.13710296)(407.67011545,398.17210293)(407.58011841,398.22210297)
\curveto(407.47011565,398.26210284)(407.36511576,398.2971028)(407.26511841,398.32710297)
\curveto(407.17511595,398.36710273)(407.09011603,398.41210269)(407.01011841,398.46210297)
\curveto(406.69011643,398.66210244)(406.40511672,398.89210221)(406.15511841,399.15210297)
\curveto(405.90511722,399.42210168)(405.70011742,399.73210137)(405.54011841,400.08210297)
\curveto(405.49011763,400.19210091)(405.45011767,400.3021008)(405.42011841,400.41210297)
\curveto(405.39011773,400.53210057)(405.35011777,400.65210045)(405.30011841,400.77210297)
\curveto(405.29011783,400.81210029)(405.28511784,400.84710025)(405.28511841,400.87710297)
\curveto(405.28511784,400.91710018)(405.28011784,400.95710014)(405.27011841,400.99710297)
\curveto(405.23011789,401.11709998)(405.20511792,401.24709985)(405.19511841,401.38710297)
\lineto(405.16511841,401.80710297)
\curveto(405.16511796,401.85709924)(405.16011796,401.91209919)(405.15011841,401.97210297)
\curveto(405.15011797,402.03209907)(405.15511797,402.08709901)(405.16511841,402.13710297)
\lineto(405.16511841,402.31710297)
\lineto(405.21011841,402.67710297)
\curveto(405.25011787,402.84709825)(405.28511784,403.01209809)(405.31511841,403.17210297)
\curveto(405.34511778,403.33209777)(405.39011773,403.48209762)(405.45011841,403.62210297)
\curveto(405.88011724,404.66209644)(406.61011651,405.3970957)(407.64011841,405.82710297)
\curveto(407.78011534,405.88709521)(407.9201152,405.92709517)(408.06011841,405.94710297)
\curveto(408.21011491,405.97709512)(408.36511476,406.01209509)(408.52511841,406.05210297)
\curveto(408.60511452,406.06209504)(408.68011444,406.06709503)(408.75011841,406.06710297)
\curveto(408.8201143,406.06709503)(408.89511423,406.07209503)(408.97511841,406.08210297)
\curveto(409.48511364,406.09209501)(409.9201132,406.03209507)(410.28011841,405.90210297)
\curveto(410.65011247,405.78209532)(410.98011214,405.62209548)(411.27011841,405.42210297)
\curveto(411.36011176,405.36209574)(411.45011167,405.29209581)(411.54011841,405.21210297)
\curveto(411.63011149,405.14209596)(411.71011141,405.06709603)(411.78011841,404.98710297)
\curveto(411.81011131,404.93709616)(411.85011127,404.8970962)(411.90011841,404.86710297)
\curveto(411.98011114,404.75709634)(412.05511107,404.64209646)(412.12511841,404.52210297)
\curveto(412.19511093,404.41209669)(412.27011085,404.2970968)(412.35011841,404.17710297)
\curveto(412.40011072,404.08709701)(412.44011068,403.99209711)(412.47011841,403.89210297)
\curveto(412.51011061,403.8020973)(412.55011057,403.7020974)(412.59011841,403.59210297)
\curveto(412.64011048,403.46209764)(412.68011044,403.32709777)(412.71011841,403.18710297)
\curveto(412.74011038,403.04709805)(412.77511035,402.90709819)(412.81511841,402.76710297)
\curveto(412.83511029,402.68709841)(412.84011028,402.5970985)(412.83011841,402.49710297)
\curveto(412.83011029,402.40709869)(412.84011028,402.32209878)(412.86011841,402.24210297)
\lineto(412.86011841,402.07710297)
\moveto(410.61011841,402.96210297)
\curveto(410.68011244,403.06209804)(410.68511244,403.18209792)(410.62511841,403.32210297)
\curveto(410.57511255,403.47209763)(410.53511259,403.58209752)(410.50511841,403.65210297)
\curveto(410.36511276,403.92209718)(410.18011294,404.12709697)(409.95011841,404.26710297)
\curveto(409.7201134,404.41709668)(409.40011372,404.4970966)(408.99011841,404.50710297)
\curveto(408.96011416,404.48709661)(408.9251142,404.48209662)(408.88511841,404.49210297)
\curveto(408.84511428,404.5020966)(408.81011431,404.5020966)(408.78011841,404.49210297)
\curveto(408.73011439,404.47209663)(408.67511445,404.45709664)(408.61511841,404.44710297)
\curveto(408.55511457,404.44709665)(408.50011462,404.43709666)(408.45011841,404.41710297)
\curveto(408.01011511,404.27709682)(407.68511544,404.0020971)(407.47511841,403.59210297)
\curveto(407.45511567,403.55209755)(407.43011569,403.4970976)(407.40011841,403.42710297)
\curveto(407.38011574,403.36709773)(407.36511576,403.3020978)(407.35511841,403.23210297)
\curveto(407.34511578,403.17209793)(407.34511578,403.11209799)(407.35511841,403.05210297)
\curveto(407.37511575,402.99209811)(407.41011571,402.94209816)(407.46011841,402.90210297)
\curveto(407.54011558,402.85209825)(407.65011547,402.82709827)(407.79011841,402.82710297)
\lineto(408.19511841,402.82710297)
\lineto(409.86011841,402.82710297)
\lineto(410.29511841,402.82710297)
\curveto(410.45511267,402.83709826)(410.56011256,402.88209822)(410.61011841,402.96210297)
}
}
{
\newrgbcolor{curcolor}{0 0 0}
\pscustom[linestyle=none,fillstyle=solid,fillcolor=curcolor]
{
}
}
{
\newrgbcolor{curcolor}{0 0 0}
\pscustom[linestyle=none,fillstyle=solid,fillcolor=curcolor]
{
\newpath
\moveto(422.69355591,406.06710297)
\curveto(422.80355059,406.06709503)(422.8985505,406.05709504)(422.97855591,406.03710297)
\curveto(423.06855033,406.01709508)(423.13855026,405.97209513)(423.18855591,405.90210297)
\curveto(423.24855015,405.82209528)(423.27855012,405.68209542)(423.27855591,405.48210297)
\lineto(423.27855591,404.97210297)
\lineto(423.27855591,404.59710297)
\curveto(423.28855011,404.45709664)(423.27355012,404.34709675)(423.23355591,404.26710297)
\curveto(423.1935502,404.1970969)(423.13355026,404.15209695)(423.05355591,404.13210297)
\curveto(422.98355041,404.11209699)(422.8985505,404.102097)(422.79855591,404.10210297)
\curveto(422.70855069,404.102097)(422.60855079,404.10709699)(422.49855591,404.11710297)
\curveto(422.398551,404.12709697)(422.30355109,404.12209698)(422.21355591,404.10210297)
\curveto(422.14355125,404.08209702)(422.07355132,404.06709703)(422.00355591,404.05710297)
\curveto(421.93355146,404.05709704)(421.86855153,404.04709705)(421.80855591,404.02710297)
\curveto(421.64855175,403.97709712)(421.48855191,403.9020972)(421.32855591,403.80210297)
\curveto(421.16855223,403.71209739)(421.04355235,403.60709749)(420.95355591,403.48710297)
\curveto(420.90355249,403.40709769)(420.84855255,403.32209778)(420.78855591,403.23210297)
\curveto(420.73855266,403.15209795)(420.68855271,403.06709803)(420.63855591,402.97710297)
\curveto(420.60855279,402.8970982)(420.57855282,402.81209829)(420.54855591,402.72210297)
\lineto(420.48855591,402.48210297)
\curveto(420.46855293,402.41209869)(420.45855294,402.33709876)(420.45855591,402.25710297)
\curveto(420.45855294,402.18709891)(420.44855295,402.11709898)(420.42855591,402.04710297)
\curveto(420.41855298,402.00709909)(420.41355298,401.96709913)(420.41355591,401.92710297)
\curveto(420.42355297,401.8970992)(420.42355297,401.86709923)(420.41355591,401.83710297)
\lineto(420.41355591,401.59710297)
\curveto(420.393553,401.52709957)(420.38855301,401.44709965)(420.39855591,401.35710297)
\curveto(420.40855299,401.27709982)(420.41355298,401.1970999)(420.41355591,401.11710297)
\lineto(420.41355591,400.15710297)
\lineto(420.41355591,398.88210297)
\curveto(420.41355298,398.75210235)(420.40855299,398.63210247)(420.39855591,398.52210297)
\curveto(420.38855301,398.41210269)(420.35855304,398.32210278)(420.30855591,398.25210297)
\curveto(420.28855311,398.22210288)(420.25355314,398.1971029)(420.20355591,398.17710297)
\curveto(420.16355323,398.16710293)(420.11855328,398.15710294)(420.06855591,398.14710297)
\lineto(419.99355591,398.14710297)
\curveto(419.94355345,398.13710296)(419.88855351,398.13210297)(419.82855591,398.13210297)
\lineto(419.66355591,398.13210297)
\lineto(419.01855591,398.13210297)
\curveto(418.95855444,398.14210296)(418.8935545,398.14710295)(418.82355591,398.14710297)
\lineto(418.62855591,398.14710297)
\curveto(418.57855482,398.16710293)(418.52855487,398.18210292)(418.47855591,398.19210297)
\curveto(418.42855497,398.21210289)(418.393555,398.24710285)(418.37355591,398.29710297)
\curveto(418.33355506,398.34710275)(418.30855509,398.41710268)(418.29855591,398.50710297)
\lineto(418.29855591,398.80710297)
\lineto(418.29855591,399.82710297)
\lineto(418.29855591,404.05710297)
\lineto(418.29855591,405.16710297)
\lineto(418.29855591,405.45210297)
\curveto(418.2985551,405.55209555)(418.31855508,405.63209547)(418.35855591,405.69210297)
\curveto(418.40855499,405.77209533)(418.48355491,405.82209528)(418.58355591,405.84210297)
\curveto(418.68355471,405.86209524)(418.80355459,405.87209523)(418.94355591,405.87210297)
\lineto(419.70855591,405.87210297)
\curveto(419.82855357,405.87209523)(419.93355346,405.86209524)(420.02355591,405.84210297)
\curveto(420.11355328,405.83209527)(420.18355321,405.78709531)(420.23355591,405.70710297)
\curveto(420.26355313,405.65709544)(420.27855312,405.58709551)(420.27855591,405.49710297)
\lineto(420.30855591,405.22710297)
\curveto(420.31855308,405.14709595)(420.33355306,405.07209603)(420.35355591,405.00210297)
\curveto(420.38355301,404.93209617)(420.43355296,404.8970962)(420.50355591,404.89710297)
\curveto(420.52355287,404.91709618)(420.54355285,404.92709617)(420.56355591,404.92710297)
\curveto(420.58355281,404.92709617)(420.60355279,404.93709616)(420.62355591,404.95710297)
\curveto(420.68355271,405.00709609)(420.73355266,405.06209604)(420.77355591,405.12210297)
\curveto(420.82355257,405.19209591)(420.88355251,405.25209585)(420.95355591,405.30210297)
\curveto(420.9935524,405.33209577)(421.02855237,405.36209574)(421.05855591,405.39210297)
\curveto(421.08855231,405.43209567)(421.12355227,405.46709563)(421.16355591,405.49710297)
\lineto(421.43355591,405.67710297)
\curveto(421.53355186,405.73709536)(421.63355176,405.79209531)(421.73355591,405.84210297)
\curveto(421.83355156,405.88209522)(421.93355146,405.91709518)(422.03355591,405.94710297)
\lineto(422.36355591,406.03710297)
\curveto(422.393551,406.04709505)(422.44855095,406.04709505)(422.52855591,406.03710297)
\curveto(422.61855078,406.03709506)(422.67355072,406.04709505)(422.69355591,406.06710297)
}
}
{
\newrgbcolor{curcolor}{0 0 0}
\pscustom[linestyle=none,fillstyle=solid,fillcolor=curcolor]
{
\newpath
\moveto(431.19996216,402.07710297)
\curveto(431.21995399,401.9970991)(431.21995399,401.90709919)(431.19996216,401.80710297)
\curveto(431.17995403,401.70709939)(431.14495407,401.64209946)(431.09496216,401.61210297)
\curveto(431.04495417,401.57209953)(430.96995424,401.54209956)(430.86996216,401.52210297)
\curveto(430.77995443,401.51209959)(430.67495454,401.5020996)(430.55496216,401.49210297)
\lineto(430.20996216,401.49210297)
\curveto(430.09995511,401.5020996)(429.99995521,401.50709959)(429.90996216,401.50710297)
\lineto(426.24996216,401.50710297)
\lineto(426.03996216,401.50710297)
\curveto(425.97995923,401.50709959)(425.92495929,401.4970996)(425.87496216,401.47710297)
\curveto(425.79495942,401.43709966)(425.74495947,401.3970997)(425.72496216,401.35710297)
\curveto(425.70495951,401.33709976)(425.68495953,401.2970998)(425.66496216,401.23710297)
\curveto(425.64495957,401.18709991)(425.63995957,401.13709996)(425.64996216,401.08710297)
\curveto(425.66995954,401.02710007)(425.67995953,400.96710013)(425.67996216,400.90710297)
\curveto(425.68995952,400.85710024)(425.70495951,400.8021003)(425.72496216,400.74210297)
\curveto(425.80495941,400.5021006)(425.89995931,400.3021008)(426.00996216,400.14210297)
\curveto(426.12995908,399.99210111)(426.28995892,399.85710124)(426.48996216,399.73710297)
\curveto(426.56995864,399.68710141)(426.64995856,399.65210145)(426.72996216,399.63210297)
\curveto(426.81995839,399.62210148)(426.9099583,399.6021015)(426.99996216,399.57210297)
\curveto(427.07995813,399.55210155)(427.18995802,399.53710156)(427.32996216,399.52710297)
\curveto(427.46995774,399.51710158)(427.58995762,399.52210158)(427.68996216,399.54210297)
\lineto(427.82496216,399.54210297)
\curveto(427.92495729,399.56210154)(428.0149572,399.58210152)(428.09496216,399.60210297)
\curveto(428.18495703,399.63210147)(428.26995694,399.66210144)(428.34996216,399.69210297)
\curveto(428.44995676,399.74210136)(428.55995665,399.80710129)(428.67996216,399.88710297)
\curveto(428.8099564,399.96710113)(428.90495631,400.04710105)(428.96496216,400.12710297)
\curveto(429.0149562,400.1971009)(429.06495615,400.26210084)(429.11496216,400.32210297)
\curveto(429.17495604,400.39210071)(429.24495597,400.44210066)(429.32496216,400.47210297)
\curveto(429.42495579,400.52210058)(429.54995566,400.54210056)(429.69996216,400.53210297)
\lineto(430.13496216,400.53210297)
\lineto(430.31496216,400.53210297)
\curveto(430.38495483,400.54210056)(430.44495477,400.53710056)(430.49496216,400.51710297)
\lineto(430.64496216,400.51710297)
\curveto(430.74495447,400.4971006)(430.8149544,400.47210063)(430.85496216,400.44210297)
\curveto(430.89495432,400.42210068)(430.9149543,400.37710072)(430.91496216,400.30710297)
\curveto(430.92495429,400.23710086)(430.91995429,400.17710092)(430.89996216,400.12710297)
\curveto(430.84995436,399.98710111)(430.79495442,399.86210124)(430.73496216,399.75210297)
\curveto(430.67495454,399.64210146)(430.60495461,399.53210157)(430.52496216,399.42210297)
\curveto(430.30495491,399.09210201)(430.05495516,398.82710227)(429.77496216,398.62710297)
\curveto(429.49495572,398.42710267)(429.14495607,398.25710284)(428.72496216,398.11710297)
\curveto(428.6149566,398.07710302)(428.50495671,398.05210305)(428.39496216,398.04210297)
\curveto(428.28495693,398.03210307)(428.16995704,398.01210309)(428.04996216,397.98210297)
\curveto(428.0099572,397.97210313)(427.96495725,397.97210313)(427.91496216,397.98210297)
\curveto(427.87495734,397.98210312)(427.83495738,397.97710312)(427.79496216,397.96710297)
\lineto(427.62996216,397.96710297)
\curveto(427.57995763,397.94710315)(427.51995769,397.94210316)(427.44996216,397.95210297)
\curveto(427.38995782,397.95210315)(427.33495788,397.95710314)(427.28496216,397.96710297)
\curveto(427.20495801,397.97710312)(427.13495808,397.97710312)(427.07496216,397.96710297)
\curveto(427.0149582,397.95710314)(426.94995826,397.96210314)(426.87996216,397.98210297)
\curveto(426.82995838,398.0021031)(426.77495844,398.01210309)(426.71496216,398.01210297)
\curveto(426.65495856,398.01210309)(426.59995861,398.02210308)(426.54996216,398.04210297)
\curveto(426.43995877,398.06210304)(426.32995888,398.08710301)(426.21996216,398.11710297)
\curveto(426.1099591,398.13710296)(426.0099592,398.17210293)(425.91996216,398.22210297)
\curveto(425.8099594,398.26210284)(425.70495951,398.2971028)(425.60496216,398.32710297)
\curveto(425.5149597,398.36710273)(425.42995978,398.41210269)(425.34996216,398.46210297)
\curveto(425.02996018,398.66210244)(424.74496047,398.89210221)(424.49496216,399.15210297)
\curveto(424.24496097,399.42210168)(424.03996117,399.73210137)(423.87996216,400.08210297)
\curveto(423.82996138,400.19210091)(423.78996142,400.3021008)(423.75996216,400.41210297)
\curveto(423.72996148,400.53210057)(423.68996152,400.65210045)(423.63996216,400.77210297)
\curveto(423.62996158,400.81210029)(423.62496159,400.84710025)(423.62496216,400.87710297)
\curveto(423.62496159,400.91710018)(423.61996159,400.95710014)(423.60996216,400.99710297)
\curveto(423.56996164,401.11709998)(423.54496167,401.24709985)(423.53496216,401.38710297)
\lineto(423.50496216,401.80710297)
\curveto(423.50496171,401.85709924)(423.49996171,401.91209919)(423.48996216,401.97210297)
\curveto(423.48996172,402.03209907)(423.49496172,402.08709901)(423.50496216,402.13710297)
\lineto(423.50496216,402.31710297)
\lineto(423.54996216,402.67710297)
\curveto(423.58996162,402.84709825)(423.62496159,403.01209809)(423.65496216,403.17210297)
\curveto(423.68496153,403.33209777)(423.72996148,403.48209762)(423.78996216,403.62210297)
\curveto(424.21996099,404.66209644)(424.94996026,405.3970957)(425.97996216,405.82710297)
\curveto(426.11995909,405.88709521)(426.25995895,405.92709517)(426.39996216,405.94710297)
\curveto(426.54995866,405.97709512)(426.70495851,406.01209509)(426.86496216,406.05210297)
\curveto(426.94495827,406.06209504)(427.01995819,406.06709503)(427.08996216,406.06710297)
\curveto(427.15995805,406.06709503)(427.23495798,406.07209503)(427.31496216,406.08210297)
\curveto(427.82495739,406.09209501)(428.25995695,406.03209507)(428.61996216,405.90210297)
\curveto(428.98995622,405.78209532)(429.31995589,405.62209548)(429.60996216,405.42210297)
\curveto(429.69995551,405.36209574)(429.78995542,405.29209581)(429.87996216,405.21210297)
\curveto(429.96995524,405.14209596)(430.04995516,405.06709603)(430.11996216,404.98710297)
\curveto(430.14995506,404.93709616)(430.18995502,404.8970962)(430.23996216,404.86710297)
\curveto(430.31995489,404.75709634)(430.39495482,404.64209646)(430.46496216,404.52210297)
\curveto(430.53495468,404.41209669)(430.6099546,404.2970968)(430.68996216,404.17710297)
\curveto(430.73995447,404.08709701)(430.77995443,403.99209711)(430.80996216,403.89210297)
\curveto(430.84995436,403.8020973)(430.88995432,403.7020974)(430.92996216,403.59210297)
\curveto(430.97995423,403.46209764)(431.01995419,403.32709777)(431.04996216,403.18710297)
\curveto(431.07995413,403.04709805)(431.1149541,402.90709819)(431.15496216,402.76710297)
\curveto(431.17495404,402.68709841)(431.17995403,402.5970985)(431.16996216,402.49710297)
\curveto(431.16995404,402.40709869)(431.17995403,402.32209878)(431.19996216,402.24210297)
\lineto(431.19996216,402.07710297)
\moveto(428.94996216,402.96210297)
\curveto(429.01995619,403.06209804)(429.02495619,403.18209792)(428.96496216,403.32210297)
\curveto(428.9149563,403.47209763)(428.87495634,403.58209752)(428.84496216,403.65210297)
\curveto(428.70495651,403.92209718)(428.51995669,404.12709697)(428.28996216,404.26710297)
\curveto(428.05995715,404.41709668)(427.73995747,404.4970966)(427.32996216,404.50710297)
\curveto(427.29995791,404.48709661)(427.26495795,404.48209662)(427.22496216,404.49210297)
\curveto(427.18495803,404.5020966)(427.14995806,404.5020966)(427.11996216,404.49210297)
\curveto(427.06995814,404.47209663)(427.0149582,404.45709664)(426.95496216,404.44710297)
\curveto(426.89495832,404.44709665)(426.83995837,404.43709666)(426.78996216,404.41710297)
\curveto(426.34995886,404.27709682)(426.02495919,404.0020971)(425.81496216,403.59210297)
\curveto(425.79495942,403.55209755)(425.76995944,403.4970976)(425.73996216,403.42710297)
\curveto(425.71995949,403.36709773)(425.70495951,403.3020978)(425.69496216,403.23210297)
\curveto(425.68495953,403.17209793)(425.68495953,403.11209799)(425.69496216,403.05210297)
\curveto(425.7149595,402.99209811)(425.74995946,402.94209816)(425.79996216,402.90210297)
\curveto(425.87995933,402.85209825)(425.98995922,402.82709827)(426.12996216,402.82710297)
\lineto(426.53496216,402.82710297)
\lineto(428.19996216,402.82710297)
\lineto(428.63496216,402.82710297)
\curveto(428.79495642,402.83709826)(428.89995631,402.88209822)(428.94996216,402.96210297)
}
}
{
\newrgbcolor{curcolor}{0 0 0}
\pscustom[linestyle=none,fillstyle=solid,fillcolor=curcolor]
{
\newpath
\moveto(436.01824341,406.08210297)
\curveto(436.82823825,406.102095)(437.50323757,405.98209512)(438.04324341,405.72210297)
\curveto(438.59323648,405.46209564)(439.02823605,405.09209601)(439.34824341,404.61210297)
\curveto(439.50823557,404.37209673)(439.62823545,404.097097)(439.70824341,403.78710297)
\curveto(439.72823535,403.73709736)(439.74323533,403.67209743)(439.75324341,403.59210297)
\curveto(439.7732353,403.51209759)(439.7732353,403.44209766)(439.75324341,403.38210297)
\curveto(439.71323536,403.27209783)(439.64323543,403.20709789)(439.54324341,403.18710297)
\curveto(439.44323563,403.17709792)(439.32323575,403.17209793)(439.18324341,403.17210297)
\lineto(438.40324341,403.17210297)
\lineto(438.11824341,403.17210297)
\curveto(438.02823705,403.17209793)(437.95323712,403.19209791)(437.89324341,403.23210297)
\curveto(437.81323726,403.27209783)(437.75823732,403.33209777)(437.72824341,403.41210297)
\curveto(437.69823738,403.5020976)(437.65823742,403.59209751)(437.60824341,403.68210297)
\curveto(437.54823753,403.79209731)(437.48323759,403.89209721)(437.41324341,403.98210297)
\curveto(437.34323773,404.07209703)(437.26323781,404.15209695)(437.17324341,404.22210297)
\curveto(437.03323804,404.31209679)(436.8782382,404.38209672)(436.70824341,404.43210297)
\curveto(436.64823843,404.45209665)(436.58823849,404.46209664)(436.52824341,404.46210297)
\curveto(436.46823861,404.46209664)(436.41323866,404.47209663)(436.36324341,404.49210297)
\lineto(436.21324341,404.49210297)
\curveto(436.01323906,404.49209661)(435.85323922,404.47209663)(435.73324341,404.43210297)
\curveto(435.44323963,404.34209676)(435.20823987,404.2020969)(435.02824341,404.01210297)
\curveto(434.84824023,403.83209727)(434.70324037,403.61209749)(434.59324341,403.35210297)
\curveto(434.54324053,403.24209786)(434.50324057,403.12209798)(434.47324341,402.99210297)
\curveto(434.45324062,402.87209823)(434.42824065,402.74209836)(434.39824341,402.60210297)
\curveto(434.38824069,402.56209854)(434.38324069,402.52209858)(434.38324341,402.48210297)
\curveto(434.38324069,402.44209866)(434.3782407,402.4020987)(434.36824341,402.36210297)
\curveto(434.34824073,402.26209884)(434.33824074,402.12209898)(434.33824341,401.94210297)
\curveto(434.34824073,401.76209934)(434.36324071,401.62209948)(434.38324341,401.52210297)
\curveto(434.38324069,401.44209966)(434.38824069,401.38709971)(434.39824341,401.35710297)
\curveto(434.41824066,401.28709981)(434.42824065,401.21709988)(434.42824341,401.14710297)
\curveto(434.43824064,401.07710002)(434.45324062,401.00710009)(434.47324341,400.93710297)
\curveto(434.55324052,400.70710039)(434.64824043,400.4971006)(434.75824341,400.30710297)
\curveto(434.86824021,400.11710098)(435.00824007,399.95710114)(435.17824341,399.82710297)
\curveto(435.21823986,399.7971013)(435.2782398,399.76210134)(435.35824341,399.72210297)
\curveto(435.46823961,399.65210145)(435.5782395,399.60710149)(435.68824341,399.58710297)
\curveto(435.80823927,399.56710153)(435.95323912,399.54710155)(436.12324341,399.52710297)
\lineto(436.21324341,399.52710297)
\curveto(436.25323882,399.52710157)(436.28323879,399.53210157)(436.30324341,399.54210297)
\lineto(436.43824341,399.54210297)
\curveto(436.50823857,399.56210154)(436.5732385,399.57710152)(436.63324341,399.58710297)
\curveto(436.70323837,399.60710149)(436.76823831,399.62710147)(436.82824341,399.64710297)
\curveto(437.12823795,399.77710132)(437.35823772,399.96710113)(437.51824341,400.21710297)
\curveto(437.55823752,400.26710083)(437.59323748,400.32210078)(437.62324341,400.38210297)
\curveto(437.65323742,400.45210065)(437.6782374,400.51210059)(437.69824341,400.56210297)
\curveto(437.73823734,400.67210043)(437.7732373,400.76710033)(437.80324341,400.84710297)
\curveto(437.83323724,400.93710016)(437.90323717,401.00710009)(438.01324341,401.05710297)
\curveto(438.10323697,401.0971)(438.24823683,401.11209999)(438.44824341,401.10210297)
\lineto(438.94324341,401.10210297)
\lineto(439.15324341,401.10210297)
\curveto(439.23323584,401.11209999)(439.29823578,401.10709999)(439.34824341,401.08710297)
\lineto(439.46824341,401.08710297)
\lineto(439.58824341,401.05710297)
\curveto(439.62823545,401.05710004)(439.65823542,401.04710005)(439.67824341,401.02710297)
\curveto(439.72823535,400.98710011)(439.75823532,400.92710017)(439.76824341,400.84710297)
\curveto(439.78823529,400.77710032)(439.78823529,400.7021004)(439.76824341,400.62210297)
\curveto(439.6782354,400.29210081)(439.56823551,399.9971011)(439.43824341,399.73710297)
\curveto(439.02823605,398.96710213)(438.3732367,398.43210267)(437.47324341,398.13210297)
\curveto(437.3732377,398.102103)(437.26823781,398.08210302)(437.15824341,398.07210297)
\curveto(437.04823803,398.05210305)(436.93823814,398.02710307)(436.82824341,397.99710297)
\curveto(436.76823831,397.98710311)(436.70823837,397.98210312)(436.64824341,397.98210297)
\curveto(436.58823849,397.98210312)(436.52823855,397.97710312)(436.46824341,397.96710297)
\lineto(436.30324341,397.96710297)
\curveto(436.25323882,397.94710315)(436.1782389,397.94210316)(436.07824341,397.95210297)
\curveto(435.9782391,397.95210315)(435.90323917,397.95710314)(435.85324341,397.96710297)
\curveto(435.7732393,397.98710311)(435.69823938,397.9971031)(435.62824341,397.99710297)
\curveto(435.56823951,397.98710311)(435.50323957,397.99210311)(435.43324341,398.01210297)
\lineto(435.28324341,398.04210297)
\curveto(435.23323984,398.04210306)(435.18323989,398.04710305)(435.13324341,398.05710297)
\curveto(435.02324005,398.08710301)(434.91824016,398.11710298)(434.81824341,398.14710297)
\curveto(434.71824036,398.17710292)(434.62324045,398.21210289)(434.53324341,398.25210297)
\curveto(434.06324101,398.45210265)(433.66824141,398.70710239)(433.34824341,399.01710297)
\curveto(433.02824205,399.33710176)(432.76824231,399.73210137)(432.56824341,400.20210297)
\curveto(432.51824256,400.29210081)(432.4782426,400.38710071)(432.44824341,400.48710297)
\lineto(432.35824341,400.81710297)
\curveto(432.34824273,400.85710024)(432.34324273,400.89210021)(432.34324341,400.92210297)
\curveto(432.34324273,400.96210014)(432.33324274,401.00710009)(432.31324341,401.05710297)
\curveto(432.29324278,401.12709997)(432.28324279,401.1970999)(432.28324341,401.26710297)
\curveto(432.28324279,401.34709975)(432.2732428,401.42209968)(432.25324341,401.49210297)
\lineto(432.25324341,401.74710297)
\curveto(432.23324284,401.7970993)(432.22324285,401.85209925)(432.22324341,401.91210297)
\curveto(432.22324285,401.98209912)(432.23324284,402.04209906)(432.25324341,402.09210297)
\curveto(432.26324281,402.14209896)(432.26324281,402.18709891)(432.25324341,402.22710297)
\curveto(432.24324283,402.26709883)(432.24324283,402.30709879)(432.25324341,402.34710297)
\curveto(432.2732428,402.41709868)(432.2782428,402.48209862)(432.26824341,402.54210297)
\curveto(432.26824281,402.6020985)(432.2782428,402.66209844)(432.29824341,402.72210297)
\curveto(432.34824273,402.9020982)(432.38824269,403.07209803)(432.41824341,403.23210297)
\curveto(432.44824263,403.4020977)(432.49324258,403.56709753)(432.55324341,403.72710297)
\curveto(432.7732423,404.23709686)(433.04824203,404.66209644)(433.37824341,405.00210297)
\curveto(433.71824136,405.34209576)(434.14824093,405.61709548)(434.66824341,405.82710297)
\curveto(434.80824027,405.88709521)(434.95324012,405.92709517)(435.10324341,405.94710297)
\curveto(435.25323982,405.97709512)(435.40823967,406.01209509)(435.56824341,406.05210297)
\curveto(435.64823943,406.06209504)(435.72323935,406.06709503)(435.79324341,406.06710297)
\curveto(435.86323921,406.06709503)(435.93823914,406.07209503)(436.01824341,406.08210297)
}
}
{
\newrgbcolor{curcolor}{0 0 0}
\pscustom[linestyle=none,fillstyle=solid,fillcolor=curcolor]
{
\newpath
\moveto(441.48152466,405.85710297)
\lineto(442.60652466,405.85710297)
\curveto(442.71652222,405.85709524)(442.81652212,405.85209525)(442.90652466,405.84210297)
\curveto(442.99652194,405.83209527)(443.06152188,405.7970953)(443.10152466,405.73710297)
\curveto(443.15152179,405.67709542)(443.18152176,405.59209551)(443.19152466,405.48210297)
\curveto(443.20152174,405.38209572)(443.20652173,405.27709582)(443.20652466,405.16710297)
\lineto(443.20652466,404.11710297)
\lineto(443.20652466,401.88210297)
\curveto(443.20652173,401.52209958)(443.22152172,401.18209992)(443.25152466,400.86210297)
\curveto(443.28152166,400.54210056)(443.37152157,400.27710082)(443.52152466,400.06710297)
\curveto(443.66152128,399.85710124)(443.88652105,399.70710139)(444.19652466,399.61710297)
\curveto(444.24652069,399.60710149)(444.28652065,399.6021015)(444.31652466,399.60210297)
\curveto(444.35652058,399.6021015)(444.40152054,399.5971015)(444.45152466,399.58710297)
\curveto(444.50152044,399.57710152)(444.55652038,399.57210153)(444.61652466,399.57210297)
\curveto(444.67652026,399.57210153)(444.72152022,399.57710152)(444.75152466,399.58710297)
\curveto(444.80152014,399.60710149)(444.8415201,399.61210149)(444.87152466,399.60210297)
\curveto(444.91152003,399.59210151)(444.95151999,399.5971015)(444.99152466,399.61710297)
\curveto(445.20151974,399.66710143)(445.36651957,399.73210137)(445.48652466,399.81210297)
\curveto(445.66651927,399.92210118)(445.80651913,400.06210104)(445.90652466,400.23210297)
\curveto(446.01651892,400.41210069)(446.09151885,400.60710049)(446.13152466,400.81710297)
\curveto(446.18151876,401.03710006)(446.21151873,401.27709982)(446.22152466,401.53710297)
\curveto(446.23151871,401.80709929)(446.2365187,402.08709901)(446.23652466,402.37710297)
\lineto(446.23652466,404.19210297)
\lineto(446.23652466,405.16710297)
\lineto(446.23652466,405.43710297)
\curveto(446.2365187,405.53709556)(446.25651868,405.61709548)(446.29652466,405.67710297)
\curveto(446.34651859,405.76709533)(446.42151852,405.81709528)(446.52152466,405.82710297)
\curveto(446.62151832,405.84709525)(446.7415182,405.85709524)(446.88152466,405.85710297)
\lineto(447.67652466,405.85710297)
\lineto(447.96152466,405.85710297)
\curveto(448.05151689,405.85709524)(448.12651681,405.83709526)(448.18652466,405.79710297)
\curveto(448.26651667,405.74709535)(448.31151663,405.67209543)(448.32152466,405.57210297)
\curveto(448.33151661,405.47209563)(448.3365166,405.35709574)(448.33652466,405.22710297)
\lineto(448.33652466,404.08710297)
\lineto(448.33652466,399.87210297)
\lineto(448.33652466,398.80710297)
\lineto(448.33652466,398.50710297)
\curveto(448.3365166,398.40710269)(448.31651662,398.33210277)(448.27652466,398.28210297)
\curveto(448.22651671,398.2021029)(448.15151679,398.15710294)(448.05152466,398.14710297)
\curveto(447.95151699,398.13710296)(447.84651709,398.13210297)(447.73652466,398.13210297)
\lineto(446.92652466,398.13210297)
\curveto(446.81651812,398.13210297)(446.71651822,398.13710296)(446.62652466,398.14710297)
\curveto(446.54651839,398.15710294)(446.48151846,398.1971029)(446.43152466,398.26710297)
\curveto(446.41151853,398.2971028)(446.39151855,398.34210276)(446.37152466,398.40210297)
\curveto(446.36151858,398.46210264)(446.34651859,398.52210258)(446.32652466,398.58210297)
\curveto(446.31651862,398.64210246)(446.30151864,398.6971024)(446.28152466,398.74710297)
\curveto(446.26151868,398.7971023)(446.23151871,398.82710227)(446.19152466,398.83710297)
\curveto(446.17151877,398.85710224)(446.14651879,398.86210224)(446.11652466,398.85210297)
\curveto(446.08651885,398.84210226)(446.06151888,398.83210227)(446.04152466,398.82210297)
\curveto(445.97151897,398.78210232)(445.91151903,398.73710236)(445.86152466,398.68710297)
\curveto(445.81151913,398.63710246)(445.75651918,398.59210251)(445.69652466,398.55210297)
\curveto(445.65651928,398.52210258)(445.61651932,398.48710261)(445.57652466,398.44710297)
\curveto(445.54651939,398.41710268)(445.50651943,398.38710271)(445.45652466,398.35710297)
\curveto(445.22651971,398.21710288)(444.95651998,398.10710299)(444.64652466,398.02710297)
\curveto(444.57652036,398.00710309)(444.50652043,397.9971031)(444.43652466,397.99710297)
\curveto(444.36652057,397.98710311)(444.29152065,397.97210313)(444.21152466,397.95210297)
\curveto(444.17152077,397.94210316)(444.12652081,397.94210316)(444.07652466,397.95210297)
\curveto(444.0365209,397.95210315)(443.99652094,397.94710315)(443.95652466,397.93710297)
\curveto(443.92652101,397.92710317)(443.86152108,397.92710317)(443.76152466,397.93710297)
\curveto(443.67152127,397.93710316)(443.61152133,397.94210316)(443.58152466,397.95210297)
\curveto(443.53152141,397.95210315)(443.48152146,397.95710314)(443.43152466,397.96710297)
\lineto(443.28152466,397.96710297)
\curveto(443.16152178,397.9971031)(443.04652189,398.02210308)(442.93652466,398.04210297)
\curveto(442.82652211,398.06210304)(442.71652222,398.09210301)(442.60652466,398.13210297)
\curveto(442.55652238,398.15210295)(442.51152243,398.16710293)(442.47152466,398.17710297)
\curveto(442.4415225,398.1971029)(442.40152254,398.21710288)(442.35152466,398.23710297)
\curveto(442.00152294,398.42710267)(441.72152322,398.69210241)(441.51152466,399.03210297)
\curveto(441.38152356,399.24210186)(441.28652365,399.49210161)(441.22652466,399.78210297)
\curveto(441.16652377,400.08210102)(441.12652381,400.3971007)(441.10652466,400.72710297)
\curveto(441.09652384,401.06710003)(441.09152385,401.41209969)(441.09152466,401.76210297)
\curveto(441.10152384,402.12209898)(441.10652383,402.47709862)(441.10652466,402.82710297)
\lineto(441.10652466,404.86710297)
\curveto(441.10652383,404.9970961)(441.10152384,405.14709595)(441.09152466,405.31710297)
\curveto(441.09152385,405.4970956)(441.11652382,405.62709547)(441.16652466,405.70710297)
\curveto(441.19652374,405.75709534)(441.25652368,405.8020953)(441.34652466,405.84210297)
\curveto(441.40652353,405.84209526)(441.45152349,405.84709525)(441.48152466,405.85710297)
}
}
{
\newrgbcolor{curcolor}{0 0 0}
\pscustom[linestyle=none,fillstyle=solid,fillcolor=curcolor]
{
\newpath
\moveto(454.39277466,406.06710297)
\curveto(454.50276934,406.06709503)(454.59776925,406.05709504)(454.67777466,406.03710297)
\curveto(454.76776908,406.01709508)(454.83776901,405.97209513)(454.88777466,405.90210297)
\curveto(454.9477689,405.82209528)(454.97776887,405.68209542)(454.97777466,405.48210297)
\lineto(454.97777466,404.97210297)
\lineto(454.97777466,404.59710297)
\curveto(454.98776886,404.45709664)(454.97276887,404.34709675)(454.93277466,404.26710297)
\curveto(454.89276895,404.1970969)(454.83276901,404.15209695)(454.75277466,404.13210297)
\curveto(454.68276916,404.11209699)(454.59776925,404.102097)(454.49777466,404.10210297)
\curveto(454.40776944,404.102097)(454.30776954,404.10709699)(454.19777466,404.11710297)
\curveto(454.09776975,404.12709697)(454.00276984,404.12209698)(453.91277466,404.10210297)
\curveto(453.84277,404.08209702)(453.77277007,404.06709703)(453.70277466,404.05710297)
\curveto(453.63277021,404.05709704)(453.56777028,404.04709705)(453.50777466,404.02710297)
\curveto(453.3477705,403.97709712)(453.18777066,403.9020972)(453.02777466,403.80210297)
\curveto(452.86777098,403.71209739)(452.7427711,403.60709749)(452.65277466,403.48710297)
\curveto(452.60277124,403.40709769)(452.5477713,403.32209778)(452.48777466,403.23210297)
\curveto(452.43777141,403.15209795)(452.38777146,403.06709803)(452.33777466,402.97710297)
\curveto(452.30777154,402.8970982)(452.27777157,402.81209829)(452.24777466,402.72210297)
\lineto(452.18777466,402.48210297)
\curveto(452.16777168,402.41209869)(452.15777169,402.33709876)(452.15777466,402.25710297)
\curveto(452.15777169,402.18709891)(452.1477717,402.11709898)(452.12777466,402.04710297)
\curveto(452.11777173,402.00709909)(452.11277173,401.96709913)(452.11277466,401.92710297)
\curveto(452.12277172,401.8970992)(452.12277172,401.86709923)(452.11277466,401.83710297)
\lineto(452.11277466,401.59710297)
\curveto(452.09277175,401.52709957)(452.08777176,401.44709965)(452.09777466,401.35710297)
\curveto(452.10777174,401.27709982)(452.11277173,401.1970999)(452.11277466,401.11710297)
\lineto(452.11277466,400.15710297)
\lineto(452.11277466,398.88210297)
\curveto(452.11277173,398.75210235)(452.10777174,398.63210247)(452.09777466,398.52210297)
\curveto(452.08777176,398.41210269)(452.05777179,398.32210278)(452.00777466,398.25210297)
\curveto(451.98777186,398.22210288)(451.95277189,398.1971029)(451.90277466,398.17710297)
\curveto(451.86277198,398.16710293)(451.81777203,398.15710294)(451.76777466,398.14710297)
\lineto(451.69277466,398.14710297)
\curveto(451.6427722,398.13710296)(451.58777226,398.13210297)(451.52777466,398.13210297)
\lineto(451.36277466,398.13210297)
\lineto(450.71777466,398.13210297)
\curveto(450.65777319,398.14210296)(450.59277325,398.14710295)(450.52277466,398.14710297)
\lineto(450.32777466,398.14710297)
\curveto(450.27777357,398.16710293)(450.22777362,398.18210292)(450.17777466,398.19210297)
\curveto(450.12777372,398.21210289)(450.09277375,398.24710285)(450.07277466,398.29710297)
\curveto(450.03277381,398.34710275)(450.00777384,398.41710268)(449.99777466,398.50710297)
\lineto(449.99777466,398.80710297)
\lineto(449.99777466,399.82710297)
\lineto(449.99777466,404.05710297)
\lineto(449.99777466,405.16710297)
\lineto(449.99777466,405.45210297)
\curveto(449.99777385,405.55209555)(450.01777383,405.63209547)(450.05777466,405.69210297)
\curveto(450.10777374,405.77209533)(450.18277366,405.82209528)(450.28277466,405.84210297)
\curveto(450.38277346,405.86209524)(450.50277334,405.87209523)(450.64277466,405.87210297)
\lineto(451.40777466,405.87210297)
\curveto(451.52777232,405.87209523)(451.63277221,405.86209524)(451.72277466,405.84210297)
\curveto(451.81277203,405.83209527)(451.88277196,405.78709531)(451.93277466,405.70710297)
\curveto(451.96277188,405.65709544)(451.97777187,405.58709551)(451.97777466,405.49710297)
\lineto(452.00777466,405.22710297)
\curveto(452.01777183,405.14709595)(452.03277181,405.07209603)(452.05277466,405.00210297)
\curveto(452.08277176,404.93209617)(452.13277171,404.8970962)(452.20277466,404.89710297)
\curveto(452.22277162,404.91709618)(452.2427716,404.92709617)(452.26277466,404.92710297)
\curveto(452.28277156,404.92709617)(452.30277154,404.93709616)(452.32277466,404.95710297)
\curveto(452.38277146,405.00709609)(452.43277141,405.06209604)(452.47277466,405.12210297)
\curveto(452.52277132,405.19209591)(452.58277126,405.25209585)(452.65277466,405.30210297)
\curveto(452.69277115,405.33209577)(452.72777112,405.36209574)(452.75777466,405.39210297)
\curveto(452.78777106,405.43209567)(452.82277102,405.46709563)(452.86277466,405.49710297)
\lineto(453.13277466,405.67710297)
\curveto(453.23277061,405.73709536)(453.33277051,405.79209531)(453.43277466,405.84210297)
\curveto(453.53277031,405.88209522)(453.63277021,405.91709518)(453.73277466,405.94710297)
\lineto(454.06277466,406.03710297)
\curveto(454.09276975,406.04709505)(454.1477697,406.04709505)(454.22777466,406.03710297)
\curveto(454.31776953,406.03709506)(454.37276947,406.04709505)(454.39277466,406.06710297)
}
}
{
\newrgbcolor{curcolor}{0 0 0}
\pscustom[linestyle=none,fillstyle=solid,fillcolor=curcolor]
{
\newpath
\moveto(458.76785278,406.08210297)
\curveto(459.51784828,406.102095)(460.16784763,406.01709508)(460.71785278,405.82710297)
\curveto(461.27784652,405.64709545)(461.7028461,405.33209577)(461.99285278,404.88210297)
\curveto(462.06284574,404.77209633)(462.12284568,404.65709644)(462.17285278,404.53710297)
\curveto(462.23284557,404.42709667)(462.28284552,404.3020968)(462.32285278,404.16210297)
\curveto(462.34284546,404.102097)(462.35284545,404.03709706)(462.35285278,403.96710297)
\curveto(462.35284545,403.8970972)(462.34284546,403.83709726)(462.32285278,403.78710297)
\curveto(462.28284552,403.72709737)(462.22784557,403.68709741)(462.15785278,403.66710297)
\curveto(462.10784569,403.64709745)(462.04784575,403.63709746)(461.97785278,403.63710297)
\lineto(461.76785278,403.63710297)
\lineto(461.10785278,403.63710297)
\curveto(461.03784676,403.63709746)(460.96784683,403.63209747)(460.89785278,403.62210297)
\curveto(460.82784697,403.62209748)(460.76284704,403.63209747)(460.70285278,403.65210297)
\curveto(460.6028472,403.67209743)(460.52784727,403.71209739)(460.47785278,403.77210297)
\curveto(460.42784737,403.83209727)(460.38284742,403.89209721)(460.34285278,403.95210297)
\lineto(460.22285278,404.16210297)
\curveto(460.19284761,404.24209686)(460.14284766,404.30709679)(460.07285278,404.35710297)
\curveto(459.97284783,404.43709666)(459.87284793,404.4970966)(459.77285278,404.53710297)
\curveto(459.68284812,404.57709652)(459.56784823,404.61209649)(459.42785278,404.64210297)
\curveto(459.35784844,404.66209644)(459.25284855,404.67709642)(459.11285278,404.68710297)
\curveto(458.98284882,404.6970964)(458.88284892,404.69209641)(458.81285278,404.67210297)
\lineto(458.70785278,404.67210297)
\lineto(458.55785278,404.64210297)
\curveto(458.51784928,404.64209646)(458.47284933,404.63709646)(458.42285278,404.62710297)
\curveto(458.25284955,404.57709652)(458.11284969,404.50709659)(458.00285278,404.41710297)
\curveto(457.9028499,404.33709676)(457.83284997,404.21209689)(457.79285278,404.04210297)
\curveto(457.77285003,403.97209713)(457.77285003,403.90709719)(457.79285278,403.84710297)
\curveto(457.81284999,403.78709731)(457.83284997,403.73709736)(457.85285278,403.69710297)
\curveto(457.92284988,403.57709752)(458.0028498,403.48209762)(458.09285278,403.41210297)
\curveto(458.19284961,403.34209776)(458.30784949,403.28209782)(458.43785278,403.23210297)
\curveto(458.62784917,403.15209795)(458.83284897,403.08209802)(459.05285278,403.02210297)
\lineto(459.74285278,402.87210297)
\curveto(459.98284782,402.83209827)(460.21284759,402.78209832)(460.43285278,402.72210297)
\curveto(460.66284714,402.67209843)(460.87784692,402.60709849)(461.07785278,402.52710297)
\curveto(461.16784663,402.48709861)(461.25284655,402.45209865)(461.33285278,402.42210297)
\curveto(461.42284638,402.4020987)(461.50784629,402.36709873)(461.58785278,402.31710297)
\curveto(461.77784602,402.1970989)(461.94784585,402.06709903)(462.09785278,401.92710297)
\curveto(462.25784554,401.78709931)(462.38284542,401.61209949)(462.47285278,401.40210297)
\curveto(462.5028453,401.33209977)(462.52784527,401.26209984)(462.54785278,401.19210297)
\curveto(462.56784523,401.12209998)(462.58784521,401.04710005)(462.60785278,400.96710297)
\curveto(462.61784518,400.90710019)(462.62284518,400.81210029)(462.62285278,400.68210297)
\curveto(462.63284517,400.56210054)(462.63284517,400.46710063)(462.62285278,400.39710297)
\lineto(462.62285278,400.32210297)
\curveto(462.6028452,400.26210084)(462.58784521,400.2021009)(462.57785278,400.14210297)
\curveto(462.57784522,400.09210101)(462.57284523,400.04210106)(462.56285278,399.99210297)
\curveto(462.49284531,399.69210141)(462.38284542,399.42710167)(462.23285278,399.19710297)
\curveto(462.07284573,398.95710214)(461.87784592,398.76210234)(461.64785278,398.61210297)
\curveto(461.41784638,398.46210264)(461.15784664,398.33210277)(460.86785278,398.22210297)
\curveto(460.75784704,398.17210293)(460.63784716,398.13710296)(460.50785278,398.11710297)
\curveto(460.38784741,398.097103)(460.26784753,398.07210303)(460.14785278,398.04210297)
\curveto(460.05784774,398.02210308)(459.96284784,398.01210309)(459.86285278,398.01210297)
\curveto(459.77284803,398.0021031)(459.68284812,397.98710311)(459.59285278,397.96710297)
\lineto(459.32285278,397.96710297)
\curveto(459.26284854,397.94710315)(459.15784864,397.93710316)(459.00785278,397.93710297)
\curveto(458.86784893,397.93710316)(458.76784903,397.94710315)(458.70785278,397.96710297)
\curveto(458.67784912,397.96710313)(458.64284916,397.97210313)(458.60285278,397.98210297)
\lineto(458.49785278,397.98210297)
\curveto(458.37784942,398.0021031)(458.25784954,398.01710308)(458.13785278,398.02710297)
\curveto(458.01784978,398.03710306)(457.9028499,398.05710304)(457.79285278,398.08710297)
\curveto(457.4028504,398.1971029)(457.05785074,398.32210278)(456.75785278,398.46210297)
\curveto(456.45785134,398.61210249)(456.2028516,398.83210227)(455.99285278,399.12210297)
\curveto(455.85285195,399.31210179)(455.73285207,399.53210157)(455.63285278,399.78210297)
\curveto(455.61285219,399.84210126)(455.59285221,399.92210118)(455.57285278,400.02210297)
\curveto(455.55285225,400.07210103)(455.53785226,400.14210096)(455.52785278,400.23210297)
\curveto(455.51785228,400.32210078)(455.52285228,400.3971007)(455.54285278,400.45710297)
\curveto(455.57285223,400.52710057)(455.62285218,400.57710052)(455.69285278,400.60710297)
\curveto(455.74285206,400.62710047)(455.802852,400.63710046)(455.87285278,400.63710297)
\lineto(456.09785278,400.63710297)
\lineto(456.80285278,400.63710297)
\lineto(457.04285278,400.63710297)
\curveto(457.12285068,400.63710046)(457.19285061,400.62710047)(457.25285278,400.60710297)
\curveto(457.36285044,400.56710053)(457.43285037,400.5021006)(457.46285278,400.41210297)
\curveto(457.5028503,400.32210078)(457.54785025,400.22710087)(457.59785278,400.12710297)
\curveto(457.61785018,400.07710102)(457.65285015,400.01210109)(457.70285278,399.93210297)
\curveto(457.76285004,399.85210125)(457.81284999,399.8021013)(457.85285278,399.78210297)
\curveto(457.97284983,399.68210142)(458.08784971,399.6021015)(458.19785278,399.54210297)
\curveto(458.30784949,399.49210161)(458.44784935,399.44210166)(458.61785278,399.39210297)
\curveto(458.66784913,399.37210173)(458.71784908,399.36210174)(458.76785278,399.36210297)
\curveto(458.81784898,399.37210173)(458.86784893,399.37210173)(458.91785278,399.36210297)
\curveto(458.9978488,399.34210176)(459.08284872,399.33210177)(459.17285278,399.33210297)
\curveto(459.27284853,399.34210176)(459.35784844,399.35710174)(459.42785278,399.37710297)
\curveto(459.47784832,399.38710171)(459.52284828,399.39210171)(459.56285278,399.39210297)
\curveto(459.61284819,399.39210171)(459.66284814,399.4021017)(459.71285278,399.42210297)
\curveto(459.85284795,399.47210163)(459.97784782,399.53210157)(460.08785278,399.60210297)
\curveto(460.20784759,399.67210143)(460.3028475,399.76210134)(460.37285278,399.87210297)
\curveto(460.42284738,399.95210115)(460.46284734,400.07710102)(460.49285278,400.24710297)
\curveto(460.51284729,400.31710078)(460.51284729,400.38210072)(460.49285278,400.44210297)
\curveto(460.47284733,400.5021006)(460.45284735,400.55210055)(460.43285278,400.59210297)
\curveto(460.36284744,400.73210037)(460.27284753,400.83710026)(460.16285278,400.90710297)
\curveto(460.06284774,400.97710012)(459.94284786,401.04210006)(459.80285278,401.10210297)
\curveto(459.61284819,401.18209992)(459.41284839,401.24709985)(459.20285278,401.29710297)
\curveto(458.99284881,401.34709975)(458.78284902,401.4020997)(458.57285278,401.46210297)
\curveto(458.49284931,401.48209962)(458.40784939,401.4970996)(458.31785278,401.50710297)
\curveto(458.23784956,401.51709958)(458.15784964,401.53209957)(458.07785278,401.55210297)
\curveto(457.75785004,401.64209946)(457.45285035,401.72709937)(457.16285278,401.80710297)
\curveto(456.87285093,401.8970992)(456.60785119,402.02709907)(456.36785278,402.19710297)
\curveto(456.08785171,402.3970987)(455.88285192,402.66709843)(455.75285278,403.00710297)
\curveto(455.73285207,403.07709802)(455.71285209,403.17209793)(455.69285278,403.29210297)
\curveto(455.67285213,403.36209774)(455.65785214,403.44709765)(455.64785278,403.54710297)
\curveto(455.63785216,403.64709745)(455.64285216,403.73709736)(455.66285278,403.81710297)
\curveto(455.68285212,403.86709723)(455.68785211,403.90709719)(455.67785278,403.93710297)
\curveto(455.66785213,403.97709712)(455.67285213,404.02209708)(455.69285278,404.07210297)
\curveto(455.71285209,404.18209692)(455.73285207,404.28209682)(455.75285278,404.37210297)
\curveto(455.78285202,404.47209663)(455.81785198,404.56709653)(455.85785278,404.65710297)
\curveto(455.98785181,404.94709615)(456.16785163,405.18209592)(456.39785278,405.36210297)
\curveto(456.62785117,405.54209556)(456.88785091,405.68709541)(457.17785278,405.79710297)
\curveto(457.28785051,405.84709525)(457.4028504,405.88209522)(457.52285278,405.90210297)
\curveto(457.64285016,405.93209517)(457.76785003,405.96209514)(457.89785278,405.99210297)
\curveto(457.95784984,406.01209509)(458.01784978,406.02209508)(458.07785278,406.02210297)
\lineto(458.25785278,406.05210297)
\curveto(458.33784946,406.06209504)(458.42284938,406.06709503)(458.51285278,406.06710297)
\curveto(458.6028492,406.06709503)(458.68784911,406.07209503)(458.76785278,406.08210297)
}
}
{
\newrgbcolor{curcolor}{0 0 0}
\pscustom[linestyle=none,fillstyle=solid,fillcolor=curcolor]
{
\newpath
\moveto(471.62449341,402.31710297)
\curveto(471.64448484,402.25709884)(471.65448483,402.17209893)(471.65449341,402.06210297)
\curveto(471.65448483,401.95209915)(471.64448484,401.86709923)(471.62449341,401.80710297)
\lineto(471.62449341,401.65710297)
\curveto(471.60448488,401.57709952)(471.59448489,401.4970996)(471.59449341,401.41710297)
\curveto(471.60448488,401.33709976)(471.59948488,401.25709984)(471.57949341,401.17710297)
\curveto(471.55948492,401.10709999)(471.54448494,401.04210006)(471.53449341,400.98210297)
\curveto(471.52448496,400.92210018)(471.51448497,400.85710024)(471.50449341,400.78710297)
\curveto(471.46448502,400.67710042)(471.42948505,400.56210054)(471.39949341,400.44210297)
\curveto(471.36948511,400.33210077)(471.32948515,400.22710087)(471.27949341,400.12710297)
\curveto(471.06948541,399.64710145)(470.79448569,399.25710184)(470.45449341,398.95710297)
\curveto(470.11448637,398.65710244)(469.70448678,398.40710269)(469.22449341,398.20710297)
\curveto(469.10448738,398.15710294)(468.9794875,398.12210298)(468.84949341,398.10210297)
\curveto(468.72948775,398.07210303)(468.60448788,398.04210306)(468.47449341,398.01210297)
\curveto(468.42448806,397.99210311)(468.36948811,397.98210312)(468.30949341,397.98210297)
\curveto(468.24948823,397.98210312)(468.19448829,397.97710312)(468.14449341,397.96710297)
\lineto(468.03949341,397.96710297)
\curveto(468.00948847,397.95710314)(467.9794885,397.95210315)(467.94949341,397.95210297)
\curveto(467.89948858,397.94210316)(467.81948866,397.93710316)(467.70949341,397.93710297)
\curveto(467.59948888,397.92710317)(467.51448897,397.93210317)(467.45449341,397.95210297)
\lineto(467.30449341,397.95210297)
\curveto(467.25448923,397.96210314)(467.19948928,397.96710313)(467.13949341,397.96710297)
\curveto(467.08948939,397.95710314)(467.03948944,397.96210314)(466.98949341,397.98210297)
\curveto(466.94948953,397.99210311)(466.90948957,397.9971031)(466.86949341,397.99710297)
\curveto(466.83948964,397.9971031)(466.79948968,398.0021031)(466.74949341,398.01210297)
\curveto(466.64948983,398.04210306)(466.54948993,398.06710303)(466.44949341,398.08710297)
\curveto(466.34949013,398.10710299)(466.25449023,398.13710296)(466.16449341,398.17710297)
\curveto(466.04449044,398.21710288)(465.92949055,398.25710284)(465.81949341,398.29710297)
\curveto(465.71949076,398.33710276)(465.61449087,398.38710271)(465.50449341,398.44710297)
\curveto(465.15449133,398.65710244)(464.85449163,398.9021022)(464.60449341,399.18210297)
\curveto(464.35449213,399.46210164)(464.14449234,399.7971013)(463.97449341,400.18710297)
\curveto(463.92449256,400.27710082)(463.8844926,400.37210073)(463.85449341,400.47210297)
\curveto(463.83449265,400.57210053)(463.80949267,400.67710042)(463.77949341,400.78710297)
\curveto(463.75949272,400.83710026)(463.74949273,400.88210022)(463.74949341,400.92210297)
\curveto(463.74949273,400.96210014)(463.73949274,401.00710009)(463.71949341,401.05710297)
\curveto(463.69949278,401.13709996)(463.68949279,401.21709988)(463.68949341,401.29710297)
\curveto(463.68949279,401.38709971)(463.6794928,401.47209963)(463.65949341,401.55210297)
\curveto(463.64949283,401.6020995)(463.64449284,401.64709945)(463.64449341,401.68710297)
\lineto(463.64449341,401.82210297)
\curveto(463.62449286,401.88209922)(463.61449287,401.96709913)(463.61449341,402.07710297)
\curveto(463.62449286,402.18709891)(463.63949284,402.27209883)(463.65949341,402.33210297)
\lineto(463.65949341,402.43710297)
\curveto(463.66949281,402.48709861)(463.66949281,402.53709856)(463.65949341,402.58710297)
\curveto(463.65949282,402.64709845)(463.66949281,402.7020984)(463.68949341,402.75210297)
\curveto(463.69949278,402.8020983)(463.70449278,402.84709825)(463.70449341,402.88710297)
\curveto(463.70449278,402.93709816)(463.71449277,402.98709811)(463.73449341,403.03710297)
\curveto(463.77449271,403.16709793)(463.80949267,403.29209781)(463.83949341,403.41210297)
\curveto(463.86949261,403.54209756)(463.90949257,403.66709743)(463.95949341,403.78710297)
\curveto(464.13949234,404.1970969)(464.35449213,404.53709656)(464.60449341,404.80710297)
\curveto(464.85449163,405.08709601)(465.15949132,405.34209576)(465.51949341,405.57210297)
\curveto(465.61949086,405.62209548)(465.72449076,405.66709543)(465.83449341,405.70710297)
\curveto(465.94449054,405.74709535)(466.05449043,405.79209531)(466.16449341,405.84210297)
\curveto(466.29449019,405.89209521)(466.42949005,405.92709517)(466.56949341,405.94710297)
\curveto(466.70948977,405.96709513)(466.85448963,405.9970951)(467.00449341,406.03710297)
\curveto(467.0844894,406.04709505)(467.15948932,406.05209505)(467.22949341,406.05210297)
\curveto(467.29948918,406.05209505)(467.36948911,406.05709504)(467.43949341,406.06710297)
\curveto(468.01948846,406.07709502)(468.51948796,406.01709508)(468.93949341,405.88710297)
\curveto(469.36948711,405.75709534)(469.74948673,405.57709552)(470.07949341,405.34710297)
\curveto(470.18948629,405.26709583)(470.29948618,405.17709592)(470.40949341,405.07710297)
\curveto(470.52948595,404.98709611)(470.62948585,404.88709621)(470.70949341,404.77710297)
\curveto(470.78948569,404.67709642)(470.85948562,404.57709652)(470.91949341,404.47710297)
\curveto(470.98948549,404.37709672)(471.05948542,404.27209683)(471.12949341,404.16210297)
\curveto(471.19948528,404.05209705)(471.25448523,403.93209717)(471.29449341,403.80210297)
\curveto(471.33448515,403.68209742)(471.3794851,403.55209755)(471.42949341,403.41210297)
\curveto(471.45948502,403.33209777)(471.484485,403.24709785)(471.50449341,403.15710297)
\lineto(471.56449341,402.88710297)
\curveto(471.57448491,402.84709825)(471.5794849,402.80709829)(471.57949341,402.76710297)
\curveto(471.5794849,402.72709837)(471.5844849,402.68709841)(471.59449341,402.64710297)
\curveto(471.61448487,402.5970985)(471.61948486,402.54209856)(471.60949341,402.48210297)
\curveto(471.59948488,402.42209868)(471.60448488,402.36709873)(471.62449341,402.31710297)
\moveto(469.52449341,401.77710297)
\curveto(469.53448695,401.82709927)(469.53948694,401.8970992)(469.53949341,401.98710297)
\curveto(469.53948694,402.08709901)(469.53448695,402.16209894)(469.52449341,402.21210297)
\lineto(469.52449341,402.33210297)
\curveto(469.50448698,402.38209872)(469.49448699,402.43709866)(469.49449341,402.49710297)
\curveto(469.49448699,402.55709854)(469.48948699,402.61209849)(469.47949341,402.66210297)
\curveto(469.479487,402.7020984)(469.47448701,402.73209837)(469.46449341,402.75210297)
\lineto(469.40449341,402.99210297)
\curveto(469.39448709,403.08209802)(469.37448711,403.16709793)(469.34449341,403.24710297)
\curveto(469.23448725,403.50709759)(469.10448738,403.72709737)(468.95449341,403.90710297)
\curveto(468.80448768,404.097097)(468.60448788,404.24709685)(468.35449341,404.35710297)
\curveto(468.29448819,404.37709672)(468.23448825,404.39209671)(468.17449341,404.40210297)
\curveto(468.11448837,404.42209668)(468.04948843,404.44209666)(467.97949341,404.46210297)
\curveto(467.89948858,404.48209662)(467.81448867,404.48709661)(467.72449341,404.47710297)
\lineto(467.45449341,404.47710297)
\curveto(467.42448906,404.45709664)(467.38948909,404.44709665)(467.34949341,404.44710297)
\curveto(467.30948917,404.45709664)(467.27448921,404.45709664)(467.24449341,404.44710297)
\lineto(467.03449341,404.38710297)
\curveto(466.97448951,404.37709672)(466.91948956,404.35709674)(466.86949341,404.32710297)
\curveto(466.61948986,404.21709688)(466.41449007,404.05709704)(466.25449341,403.84710297)
\curveto(466.10449038,403.64709745)(465.9844905,403.41209769)(465.89449341,403.14210297)
\curveto(465.86449062,403.04209806)(465.83949064,402.93709816)(465.81949341,402.82710297)
\curveto(465.80949067,402.71709838)(465.79449069,402.60709849)(465.77449341,402.49710297)
\curveto(465.76449072,402.44709865)(465.75949072,402.3970987)(465.75949341,402.34710297)
\lineto(465.75949341,402.19710297)
\curveto(465.73949074,402.12709897)(465.72949075,402.02209908)(465.72949341,401.88210297)
\curveto(465.73949074,401.74209936)(465.75449073,401.63709946)(465.77449341,401.56710297)
\lineto(465.77449341,401.43210297)
\curveto(465.79449069,401.35209975)(465.80949067,401.27209983)(465.81949341,401.19210297)
\curveto(465.82949065,401.12209998)(465.84449064,401.04710005)(465.86449341,400.96710297)
\curveto(465.96449052,400.66710043)(466.06949041,400.42210068)(466.17949341,400.23210297)
\curveto(466.29949018,400.05210105)(466.48449,399.88710121)(466.73449341,399.73710297)
\curveto(466.80448968,399.68710141)(466.8794896,399.64710145)(466.95949341,399.61710297)
\curveto(467.04948943,399.58710151)(467.13948934,399.56210154)(467.22949341,399.54210297)
\curveto(467.26948921,399.53210157)(467.30448918,399.52710157)(467.33449341,399.52710297)
\curveto(467.36448912,399.53710156)(467.39948908,399.53710156)(467.43949341,399.52710297)
\lineto(467.55949341,399.49710297)
\curveto(467.60948887,399.4971016)(467.65448883,399.5021016)(467.69449341,399.51210297)
\lineto(467.81449341,399.51210297)
\curveto(467.89448859,399.53210157)(467.97448851,399.54710155)(468.05449341,399.55710297)
\curveto(468.13448835,399.56710153)(468.20948827,399.58710151)(468.27949341,399.61710297)
\curveto(468.53948794,399.71710138)(468.74948773,399.85210125)(468.90949341,400.02210297)
\curveto(469.06948741,400.19210091)(469.20448728,400.4021007)(469.31449341,400.65210297)
\curveto(469.35448713,400.75210035)(469.3844871,400.85210025)(469.40449341,400.95210297)
\curveto(469.42448706,401.05210005)(469.44948703,401.15709994)(469.47949341,401.26710297)
\curveto(469.48948699,401.30709979)(469.49448699,401.34209976)(469.49449341,401.37210297)
\curveto(469.49448699,401.41209969)(469.49948698,401.45209965)(469.50949341,401.49210297)
\lineto(469.50949341,401.62710297)
\curveto(469.50948697,401.67709942)(469.51448697,401.72709937)(469.52449341,401.77710297)
}
}
{
\newrgbcolor{curcolor}{0 0 0}
\pscustom[linestyle=none,fillstyle=solid,fillcolor=curcolor]
{
\newpath
\moveto(12.08081787,196.61524658)
\curveto(12.08082857,196.75524277)(12.08082857,196.9202426)(12.08081787,197.11024658)
\curveto(12.07082858,197.31024221)(12.10582854,197.43024209)(12.18581787,197.47024658)
\curveto(12.27582837,197.53024199)(12.41082824,197.54024198)(12.59081787,197.50024658)
\curveto(12.77082788,197.46024206)(12.94082771,197.4252421)(13.10081787,197.39524658)
\lineto(15.35081787,196.94524658)
\lineto(18.14081787,196.39024658)
\curveto(18.49082216,196.3202432)(18.83582181,196.25524327)(19.17581787,196.19524658)
\curveto(19.50582114,196.13524339)(19.80082085,196.1252434)(20.06081787,196.16524658)
\curveto(20.4508202,196.21524331)(20.77081988,196.33524319)(21.02081787,196.52524658)
\curveto(21.27081938,196.71524281)(21.47581917,196.98024254)(21.63581787,197.32024658)
\curveto(21.68581896,197.4202421)(21.72081893,197.53024199)(21.74081787,197.65024658)
\curveto(21.7508189,197.77024175)(21.77081888,197.88524164)(21.80081787,197.99524658)
\lineto(21.83081787,198.17524658)
\curveto(21.83081882,198.24524128)(21.83581881,198.30524122)(21.84581787,198.35524658)
\lineto(21.84581787,198.50524658)
\curveto(21.8458188,198.56524096)(21.8508188,198.6252409)(21.86081787,198.68524658)
\curveto(21.86081879,198.75524077)(21.8508188,198.8202407)(21.83081787,198.88024658)
\curveto(21.82081883,198.93024059)(21.82081883,198.98024054)(21.83081787,199.03024658)
\curveto(21.83081882,199.09024043)(21.82581882,199.14524038)(21.81581787,199.19524658)
\curveto(21.78581886,199.31524021)(21.76081889,199.43024009)(21.74081787,199.54024658)
\curveto(21.72081893,199.66023986)(21.68581896,199.78023974)(21.63581787,199.90024658)
\curveto(21.48581916,200.31023921)(21.29081936,200.64523888)(21.05081787,200.90524658)
\curveto(20.81081984,201.17523835)(20.49082016,201.41523811)(20.09081787,201.62524658)
\curveto(19.84082081,201.75523777)(19.56082109,201.85523767)(19.25081787,201.92524658)
\curveto(18.93082172,202.00523752)(18.60582204,202.08023744)(18.27581787,202.15024658)
\lineto(15.81581787,202.64524658)
\lineto(13.22081787,203.15524658)
\curveto(13.0508276,203.19523633)(12.86082779,203.23023629)(12.65081787,203.26024658)
\curveto(12.43082822,203.30023622)(12.27582837,203.37023615)(12.18581787,203.47024658)
\curveto(12.11582853,203.54023598)(12.08082857,203.64023588)(12.08081787,203.77024658)
\curveto(12.08082857,203.91023561)(12.08082857,204.04523548)(12.08081787,204.17524658)
\curveto(12.08082857,204.2252353)(12.08582856,204.27023525)(12.09581787,204.31024658)
\curveto(12.09582855,204.35023517)(12.09582855,204.39523513)(12.09581787,204.44524658)
\curveto(12.12582852,204.58523494)(12.17582847,204.66523486)(12.24581787,204.68524658)
\curveto(12.32582832,204.7252348)(12.44082821,204.7252348)(12.59081787,204.68524658)
\curveto(12.74082791,204.65523487)(12.87582777,204.6252349)(12.99581787,204.59524658)
\lineto(15.06581787,204.19024658)
\lineto(18.18581787,203.56024658)
\curveto(18.55582209,203.49023603)(18.92082173,203.41023611)(19.28081787,203.32024658)
\curveto(19.64082101,203.24023628)(19.96582068,203.13523639)(20.25581787,203.00524658)
\curveto(20.7458199,202.76523676)(21.16581948,202.49523703)(21.51581787,202.19524658)
\curveto(21.85581879,201.90523762)(22.1458185,201.54523798)(22.38581787,201.11524658)
\curveto(22.47581817,200.94523858)(22.55581809,200.77023875)(22.62581787,200.59024658)
\curveto(22.69581795,200.41023911)(22.76081789,200.21523931)(22.82081787,200.00524658)
\curveto(22.8508178,199.91523961)(22.87081778,199.8202397)(22.88081787,199.72024658)
\curveto(22.90081775,199.63023989)(22.92081773,199.53523999)(22.94081787,199.43524658)
\curveto(22.96081769,199.3252402)(22.97081768,199.21524031)(22.97081787,199.10524658)
\curveto(22.97081768,199.00524052)(22.98081767,198.90524062)(23.00081787,198.80524658)
\lineto(23.00081787,198.62524658)
\curveto(23.01081764,198.57524095)(23.01581763,198.49524103)(23.01581787,198.38524658)
\curveto(23.01581763,198.27524125)(23.01081764,198.19524133)(23.00081787,198.14524658)
\lineto(23.00081787,197.96524658)
\curveto(22.98081767,197.89524163)(22.97081768,197.8202417)(22.97081787,197.74024658)
\curveto(22.98081767,197.67024185)(22.97581767,197.60524192)(22.95581787,197.54524658)
\lineto(22.95581787,197.42524658)
\curveto(22.93581771,197.34524218)(22.92081773,197.26524226)(22.91081787,197.18524658)
\curveto(22.90081775,197.11524241)(22.88581776,197.04524248)(22.86581787,196.97524658)
\curveto(22.81581783,196.81524271)(22.77081788,196.65524287)(22.73081787,196.49524658)
\curveto(22.68081797,196.34524318)(22.62081803,196.20024332)(22.55081787,196.06024658)
\curveto(22.52081813,196.00024352)(22.48581816,195.94024358)(22.44581787,195.88024658)
\curveto(22.40581824,195.8202437)(22.36081829,195.76024376)(22.31081787,195.70024658)
\curveto(22.12081853,195.44024408)(21.89581875,195.23524429)(21.63581787,195.08524658)
\curveto(21.37581927,194.93524459)(21.07081958,194.8252447)(20.72081787,194.75524658)
\curveto(20.57082008,194.7252448)(20.41582023,194.71024481)(20.25581787,194.71024658)
\curveto(20.09582055,194.7202448)(19.93082072,194.7202448)(19.76081787,194.71024658)
\curveto(19.68082097,194.7202448)(19.60582104,194.73024479)(19.53581787,194.74024658)
\curveto(19.45582119,194.75024477)(19.37582127,194.75524477)(19.29581787,194.75524658)
\lineto(19.14581787,194.78524658)
\curveto(19.06582158,194.78524474)(18.98582166,194.79524473)(18.90581787,194.81524658)
\curveto(18.81582183,194.84524468)(18.73082192,194.87024465)(18.65081787,194.89024658)
\lineto(17.67581787,195.08524658)
\lineto(13.76081787,195.86524658)
\lineto(12.74081787,196.07524658)
\curveto(12.650828,196.09524343)(12.56082809,196.11024341)(12.47081787,196.12024658)
\curveto(12.38082827,196.14024338)(12.30582834,196.17524335)(12.24581787,196.22524658)
\curveto(12.17582847,196.28524324)(12.12582852,196.37024315)(12.09581787,196.48024658)
\curveto(12.09582855,196.54024298)(12.09082856,196.58524294)(12.08081787,196.61524658)
}
}
{
\newrgbcolor{curcolor}{0 0 0}
\pscustom[linestyle=none,fillstyle=solid,fillcolor=curcolor]
{
\newpath
\moveto(14.88581787,209.96009033)
\curveto(14.86582578,210.60008351)(14.9508257,211.09008302)(15.14081787,211.43009033)
\curveto(15.33082532,211.77008234)(15.61582503,212.0150821)(15.99581787,212.16509033)
\curveto(16.09582455,212.20508191)(16.20582444,212.23008188)(16.32581787,212.24009033)
\curveto(16.43582421,212.26008185)(16.5508241,212.27008184)(16.67081787,212.27009033)
\curveto(16.86082379,212.29008182)(17.06582358,212.28008183)(17.28581787,212.24009033)
\curveto(17.50582314,212.2100819)(17.73082292,212.17008194)(17.96081787,212.12009033)
\lineto(19.56581787,211.80509033)
\lineto(21.90581787,211.34009033)
\lineto(22.41581787,211.22009033)
\curveto(22.58581806,211.18008293)(22.69581795,211.09008302)(22.74581787,210.95009033)
\curveto(22.76581788,210.90008321)(22.77581787,210.84508327)(22.77581787,210.78509033)
\curveto(22.78581786,210.73508338)(22.79081786,210.68008343)(22.79081787,210.62009033)
\curveto(22.79081786,210.49008362)(22.78581786,210.36508375)(22.77581787,210.24509033)
\curveto(22.77581787,210.12508399)(22.73581791,210.05008406)(22.65581787,210.02009033)
\curveto(22.58581806,209.98008413)(22.49581815,209.97008414)(22.38581787,209.99009033)
\curveto(22.27581837,210.0100841)(22.16581848,210.03508408)(22.05581787,210.06509033)
\lineto(20.76581787,210.32009033)
\lineto(18.32081787,210.80009033)
\curveto(18.0508226,210.86008325)(17.78582286,210.9100832)(17.52581787,210.95009033)
\curveto(17.25582339,210.99008312)(17.02582362,210.99008312)(16.83581787,210.95009033)
\curveto(16.63582401,210.9100832)(16.47582417,210.82008329)(16.35581787,210.68009033)
\curveto(16.22582442,210.55008356)(16.12582452,210.39008372)(16.05581787,210.20009033)
\curveto(16.03582461,210.14008397)(16.02582462,210.07508404)(16.02581787,210.00509033)
\curveto(16.01582463,209.94508417)(16.00082465,209.89008422)(15.98081787,209.84009033)
\curveto(15.97082468,209.79008432)(15.97082468,209.7100844)(15.98081787,209.60009033)
\curveto(15.98082467,209.50008461)(15.98582466,209.42508469)(15.99581787,209.37509033)
\curveto(16.01582463,209.33508478)(16.02582462,209.30008481)(16.02581787,209.27009033)
\curveto(16.01582463,209.24008487)(16.01582463,209.20508491)(16.02581787,209.16509033)
\curveto(16.05582459,209.02508509)(16.09082456,208.89508522)(16.13081787,208.77509033)
\curveto(16.16082449,208.65508546)(16.20582444,208.54008557)(16.26581787,208.43009033)
\curveto(16.28582436,208.38008573)(16.30582434,208.34008577)(16.32581787,208.31009033)
\curveto(16.3458243,208.28008583)(16.36582428,208.24008587)(16.38581787,208.19009033)
\curveto(16.63582401,207.79008632)(17.01082364,207.46008665)(17.51081787,207.20009033)
\curveto(17.59082306,207.16008695)(17.67582297,207.12508699)(17.76581787,207.09509033)
\lineto(18.00581787,207.00509033)
\curveto(18.05582259,206.97508714)(18.10582254,206.96008715)(18.15581787,206.96009033)
\curveto(18.19582245,206.96008715)(18.23582241,206.94508717)(18.27581787,206.91509033)
\lineto(18.59081787,206.85509033)
\curveto(18.62082203,206.83508728)(18.65582199,206.82508729)(18.69581787,206.82509033)
\curveto(18.73582191,206.82508729)(18.78082187,206.82008729)(18.83081787,206.81009033)
\lineto(19.28081787,206.72009033)
\lineto(20.72081787,206.42009033)
\lineto(22.04081787,206.16509033)
\curveto(22.1508185,206.14508797)(22.26581838,206.12008799)(22.38581787,206.09009033)
\curveto(22.49581815,206.07008804)(22.58581806,206.03008808)(22.65581787,205.97009033)
\curveto(22.73581791,205.90008821)(22.77581787,205.80008831)(22.77581787,205.67009033)
\curveto(22.78581786,205.55008856)(22.79081786,205.42508869)(22.79081787,205.29509033)
\curveto(22.79081786,205.2150889)(22.78581786,205.14008897)(22.77581787,205.07009033)
\curveto(22.76581788,205.00008911)(22.74081791,204.94508917)(22.70081787,204.90509033)
\curveto(22.650818,204.83508928)(22.55581809,204.8150893)(22.41581787,204.84509033)
\curveto(22.27581837,204.87508924)(22.14081851,204.90008921)(22.01081787,204.92009033)
\lineto(20.24081787,205.28009033)
\lineto(16.61081787,206.00009033)
\lineto(15.69581787,206.18009033)
\lineto(15.42581787,206.24009033)
\curveto(15.33582531,206.26008785)(15.26582538,206.29508782)(15.21581787,206.34509033)
\curveto(15.15582549,206.38508773)(15.11582553,206.44008767)(15.09581787,206.51009033)
\curveto(15.08582556,206.56008755)(15.07582557,206.62008749)(15.06581787,206.69009033)
\curveto(15.05582559,206.77008734)(15.0508256,206.85008726)(15.05081787,206.93009033)
\curveto(15.0508256,207.0100871)(15.05582559,207.08508703)(15.06581787,207.15509033)
\curveto(15.07582557,207.23508688)(15.09082556,207.28508683)(15.11081787,207.30509033)
\curveto(15.18082547,207.40508671)(15.27082538,207.44008667)(15.38081787,207.41009033)
\curveto(15.48082517,207.38008673)(15.59582505,207.37008674)(15.72581787,207.38009033)
\curveto(15.78582486,207.39008672)(15.83582481,207.42008669)(15.87581787,207.47009033)
\curveto(15.88582476,207.59008652)(15.84082481,207.69508642)(15.74081787,207.78509033)
\curveto(15.64082501,207.88508623)(15.56082509,207.98008613)(15.50081787,208.07009033)
\curveto(15.40082525,208.23008588)(15.31082534,208.39008572)(15.23081787,208.55009033)
\curveto(15.14082551,208.7100854)(15.06582558,208.89508522)(15.00581787,209.10509033)
\curveto(14.97582567,209.18508493)(14.95582569,209.27508484)(14.94581787,209.37509033)
\curveto(14.93582571,209.47508464)(14.92082573,209.57008454)(14.90081787,209.66009033)
\curveto(14.89082576,209.7100844)(14.88582576,209.76008435)(14.88581787,209.81009033)
\lineto(14.88581787,209.96009033)
}
}
{
\newrgbcolor{curcolor}{0 0 0}
\pscustom[linestyle=none,fillstyle=solid,fillcolor=curcolor]
{
\newpath
\moveto(13.53581787,215.17469971)
\curveto(13.47582717,215.10469673)(13.37082728,215.08469675)(13.22081787,215.11469971)
\curveto(13.06082759,215.14469669)(12.90582774,215.17469666)(12.75581787,215.20469971)
\curveto(12.67582797,215.21469662)(12.59082806,215.22969661)(12.50081787,215.24969971)
\curveto(12.41082824,215.26969657)(12.33582831,215.29969654)(12.27581787,215.33969971)
\curveto(12.19582845,215.39969644)(12.13582851,215.48969635)(12.09581787,215.60969971)
\curveto(12.08582856,215.6396962)(12.08582856,215.66469617)(12.09581787,215.68469971)
\curveto(12.09582855,215.70469613)(12.09082856,215.72969611)(12.08081787,215.75969971)
\curveto(12.08082857,215.92969591)(12.08582856,216.08469575)(12.09581787,216.22469971)
\curveto(12.10582854,216.37469546)(12.16582848,216.46469537)(12.27581787,216.49469971)
\curveto(12.33582831,216.51469532)(12.41082824,216.51469532)(12.50081787,216.49469971)
\curveto(12.58082807,216.47469536)(12.66582798,216.45969538)(12.75581787,216.44969971)
\curveto(12.93582771,216.40969543)(13.10582754,216.36969547)(13.26581787,216.32969971)
\curveto(13.42582722,216.29969554)(13.53082712,216.21469562)(13.58081787,216.07469971)
\curveto(13.60082705,216.01469582)(13.61082704,215.95469588)(13.61081787,215.89469971)
\lineto(13.61081787,215.72969971)
\lineto(13.61081787,215.41469971)
\curveto(13.61082704,215.31469652)(13.58582706,215.2346966)(13.53581787,215.17469971)
\moveto(22.04081787,214.58969971)
\curveto(22.14081851,214.56969727)(22.2458184,214.54969729)(22.35581787,214.52969971)
\curveto(22.45581819,214.51969732)(22.53581811,214.47969736)(22.59581787,214.40969971)
\curveto(22.65581799,214.36969747)(22.69581795,214.31969752)(22.71581787,214.25969971)
\curveto(22.72581792,214.19969764)(22.74081791,214.12469771)(22.76081787,214.03469971)
\lineto(22.76081787,213.80969971)
\curveto(22.76081789,213.67969816)(22.75581789,213.56969827)(22.74581787,213.47969971)
\curveto(22.72581792,213.38969845)(22.67581797,213.32469851)(22.59581787,213.28469971)
\curveto(22.53581811,213.26469857)(22.46081819,213.25969858)(22.37081787,213.26969971)
\curveto(22.27081838,213.28969855)(22.17581847,213.30969853)(22.08581787,213.32969971)
\lineto(15.74081787,214.60469971)
\curveto(15.63082502,214.62469721)(15.52582512,214.64469719)(15.42581787,214.66469971)
\curveto(15.31582533,214.68469715)(15.23082542,214.72469711)(15.17081787,214.78469971)
\curveto(15.12082553,214.82469701)(15.09082556,214.86969697)(15.08081787,214.91969971)
\curveto(15.07082558,214.97969686)(15.05582559,215.0396968)(15.03581787,215.09969971)
\curveto(15.03582561,215.11969672)(15.04082561,215.1396967)(15.05081787,215.15969971)
\curveto(15.0508256,215.18969665)(15.0458256,215.21469662)(15.03581787,215.23469971)
\curveto(15.03582561,215.36469647)(15.04082561,215.49469634)(15.05081787,215.62469971)
\curveto(15.0508256,215.76469607)(15.09082556,215.84969599)(15.17081787,215.87969971)
\curveto(15.23082542,215.91969592)(15.31082534,215.92969591)(15.41081787,215.90969971)
\curveto(15.50082515,215.88969595)(15.59582505,215.86969597)(15.69581787,215.84969971)
\lineto(22.04081787,214.58969971)
}
}
{
\newrgbcolor{curcolor}{0 0 0}
\pscustom[linestyle=none,fillstyle=solid,fillcolor=curcolor]
{
\newpath
\moveto(21.95081787,223.50954346)
\lineto(22.34081787,223.41954346)
\curveto(22.46081819,223.39953553)(22.56081809,223.35953557)(22.64081787,223.29954346)
\curveto(22.71081794,223.2295357)(22.7508179,223.13453579)(22.76081787,223.01454346)
\lineto(22.76081787,222.66954346)
\curveto(22.76081789,222.60953632)(22.76581788,222.54953638)(22.77581787,222.48954346)
\curveto(22.77581787,222.43953649)(22.76581788,222.39453653)(22.74581787,222.35454346)
\curveto(22.72581792,222.27453665)(22.68581796,222.2245367)(22.62581787,222.20454346)
\curveto(22.57581807,222.17453675)(22.51581813,222.16453676)(22.44581787,222.17454346)
\curveto(22.37581827,222.18453674)(22.30581834,222.17953675)(22.23581787,222.15954346)
\curveto(22.21581843,222.15953677)(22.20081845,222.14953678)(22.19081787,222.12954346)
\lineto(22.13081787,222.09954346)
\curveto(22.12081853,221.99953693)(22.14081851,221.91453701)(22.19081787,221.84454346)
\curveto(22.24081841,221.78453714)(22.29081836,221.71953721)(22.34081787,221.64954346)
\curveto(22.49081816,221.41953751)(22.60581804,221.19453773)(22.68581787,220.97454346)
\curveto(22.76581788,220.78453814)(22.82581782,220.56453836)(22.86581787,220.31454346)
\curveto(22.90581774,220.07453885)(22.92581772,219.8295391)(22.92581787,219.57954346)
\curveto(22.93581771,219.33953959)(22.92081773,219.09953983)(22.88081787,218.85954346)
\curveto(22.8508178,218.6295403)(22.79581785,218.43454049)(22.71581787,218.27454346)
\curveto(22.49581815,217.79454113)(22.20081845,217.4295415)(21.83081787,217.17954346)
\curveto(21.4508192,216.93954199)(20.98081967,216.78454214)(20.42081787,216.71454346)
\curveto(20.33082032,216.69454223)(20.24082041,216.68454224)(20.15081787,216.68454346)
\curveto(20.0508206,216.69454223)(19.9508207,216.69454223)(19.85081787,216.68454346)
\curveto(19.80082085,216.68454224)(19.7508209,216.68954224)(19.70081787,216.69954346)
\curveto(19.650821,216.70954222)(19.60082105,216.71454221)(19.55081787,216.71454346)
\curveto(19.50082115,216.70454222)(19.4508212,216.70454222)(19.40081787,216.71454346)
\curveto(19.34082131,216.73454219)(19.28582136,216.74454218)(19.23581787,216.74454346)
\lineto(19.08581787,216.77454346)
\curveto(19.03582161,216.76454216)(18.97082168,216.76454216)(18.89081787,216.77454346)
\curveto(18.81082184,216.79454213)(18.7458219,216.81954211)(18.69581787,216.84954346)
\lineto(18.53081787,216.89454346)
\curveto(18.46082219,216.924542)(18.39082226,216.94454198)(18.32081787,216.95454346)
\curveto(18.24082241,216.96454196)(18.16582248,216.98454194)(18.09581787,217.01454346)
\curveto(18.0458226,217.03454189)(18.00082265,217.04954188)(17.96081787,217.05954346)
\curveto(17.92082273,217.06954186)(17.87582277,217.08454184)(17.82581787,217.10454346)
\curveto(17.72582292,217.15454177)(17.63082302,217.19954173)(17.54081787,217.23954346)
\curveto(17.44082321,217.27954165)(17.3458233,217.3245416)(17.25581787,217.37454346)
\curveto(16.87582377,217.57454135)(16.53582411,217.80454112)(16.23581787,218.06454346)
\curveto(15.92582472,218.33454059)(15.67082498,218.63454029)(15.47081787,218.96454346)
\curveto(15.3508253,219.16453976)(15.2508254,219.36453956)(15.17081787,219.56454346)
\curveto(15.09082556,219.76453916)(15.02082563,219.97953895)(14.96081787,220.20954346)
\lineto(14.93081787,220.41954346)
\curveto(14.92082573,220.48953844)(14.90582574,220.55953837)(14.88581787,220.62954346)
\lineto(14.88581787,220.77954346)
\curveto(14.86582578,220.86953806)(14.85582579,220.98953794)(14.85581787,221.13954346)
\curveto(14.85582579,221.29953763)(14.86582578,221.41453751)(14.88581787,221.48454346)
\curveto(14.89582575,221.5245374)(14.90082575,221.57953735)(14.90081787,221.64954346)
\curveto(14.93082572,221.74953718)(14.95582569,221.85453707)(14.97581787,221.96454346)
\curveto(14.98582566,222.07453685)(15.01582563,222.17453675)(15.06581787,222.26454346)
\curveto(15.12582552,222.40453652)(15.19082546,222.53453639)(15.26081787,222.65454346)
\curveto(15.33082532,222.77453615)(15.41082524,222.88453604)(15.50081787,222.98454346)
\curveto(15.5508251,223.03453589)(15.60582504,223.08453584)(15.66581787,223.13454346)
\curveto(15.71582493,223.19453573)(15.73082492,223.27953565)(15.71081787,223.38954346)
\lineto(15.63581787,223.46454346)
\curveto(15.61582503,223.48453544)(15.58582506,223.49953543)(15.54581787,223.50954346)
\curveto(15.45582519,223.55953537)(15.34082531,223.59453533)(15.20081787,223.61454346)
\curveto(15.06082559,223.64453528)(14.93582571,223.66953526)(14.82581787,223.68954346)
\lineto(13.10081787,224.03454346)
\curveto(12.96082769,224.06453486)(12.80582784,224.09453483)(12.63581787,224.12454346)
\curveto(12.45582819,224.16453476)(12.32582832,224.21453471)(12.24581787,224.27454346)
\curveto(12.17582847,224.33453459)(12.13082852,224.40453452)(12.11081787,224.48454346)
\curveto(12.11082854,224.50453442)(12.11082854,224.5295344)(12.11081787,224.55954346)
\curveto(12.10082855,224.58953434)(12.09582855,224.61453431)(12.09581787,224.63454346)
\curveto(12.08582856,224.78453414)(12.08582856,224.93453399)(12.09581787,225.08454346)
\curveto(12.09582855,225.23453369)(12.13582851,225.33453359)(12.21581787,225.38454346)
\curveto(12.29582835,225.41453351)(12.39582825,225.41453351)(12.51581787,225.38454346)
\curveto(12.63582801,225.36453356)(12.76082789,225.34453358)(12.89081787,225.32454346)
\lineto(21.95081787,223.50954346)
\moveto(19.11581787,222.86454346)
\curveto(19.06582158,222.89453603)(19.00082165,222.91453601)(18.92081787,222.92454346)
\curveto(18.83082182,222.94453598)(18.76082189,222.94953598)(18.71081787,222.93954346)
\lineto(18.48581787,222.98454346)
\curveto(18.39582225,222.98453594)(18.30582234,222.98953594)(18.21581787,222.99954346)
\curveto(18.11582253,223.00953592)(18.02582262,223.00453592)(17.94581787,222.98454346)
\lineto(17.72081787,222.98454346)
\curveto(17.650823,222.98453594)(17.58082307,222.97453595)(17.51081787,222.95454346)
\curveto(17.21082344,222.89453603)(16.9458237,222.78953614)(16.71581787,222.63954346)
\curveto(16.48582416,222.49953643)(16.30582434,222.29953663)(16.17581787,222.03954346)
\curveto(16.12582452,221.94953698)(16.09082456,221.85453707)(16.07081787,221.75454346)
\curveto(16.04082461,221.65453727)(16.01582463,221.54453738)(15.99581787,221.42454346)
\curveto(15.97582467,221.35453757)(15.96582468,221.26953766)(15.96581787,221.16954346)
\lineto(15.96581787,220.89954346)
\lineto(15.99581787,220.74954346)
\lineto(15.99581787,220.61454346)
\curveto(16.01582463,220.53453839)(16.03582461,220.44953848)(16.05581787,220.35954346)
\curveto(16.07582457,220.26953866)(16.10082455,220.18453874)(16.13081787,220.10454346)
\curveto(16.27082438,219.75453917)(16.47582417,219.45453947)(16.74581787,219.20454346)
\curveto(17.00582364,218.95453997)(17.31082334,218.73454019)(17.66081787,218.54454346)
\curveto(17.77082288,218.48454044)(17.88582276,218.43454049)(18.00581787,218.39454346)
\lineto(18.33581787,218.27454346)
\lineto(18.45581787,218.24454346)
\curveto(18.48582216,218.23454069)(18.52082213,218.2245407)(18.56081787,218.21454346)
\curveto(18.61082204,218.18454074)(18.66582198,218.16454076)(18.72581787,218.15454346)
\curveto(18.78582186,218.15454077)(18.84082181,218.14954078)(18.89081787,218.13954346)
\curveto(19.00082165,218.11954081)(19.11082154,218.09454083)(19.22081787,218.06454346)
\curveto(19.32082133,218.04454088)(19.41582123,218.03954089)(19.50581787,218.04954346)
\curveto(19.53582111,218.04954088)(19.58582106,218.04454088)(19.65581787,218.03454346)
\lineto(19.86581787,218.03454346)
\curveto(19.93582071,218.03454089)(20.00582064,218.03954089)(20.07581787,218.04954346)
\curveto(20.42582022,218.08954084)(20.72581992,218.17954075)(20.97581787,218.31954346)
\curveto(21.22581942,218.45954047)(21.43081922,218.65954027)(21.59081787,218.91954346)
\curveto(21.64081901,218.99953993)(21.68081897,219.07953985)(21.71081787,219.15954346)
\curveto(21.74081891,219.24953968)(21.77081888,219.34453958)(21.80081787,219.44454346)
\curveto(21.82081883,219.49453943)(21.82581882,219.54453938)(21.81581787,219.59454346)
\curveto(21.80581884,219.65453927)(21.81081884,219.70953922)(21.83081787,219.75954346)
\curveto(21.84081881,219.78953914)(21.8458188,219.8245391)(21.84581787,219.86454346)
\lineto(21.84581787,219.99954346)
\lineto(21.84581787,220.13454346)
\curveto(21.83581881,220.17453875)(21.83081882,220.2295387)(21.83081787,220.29954346)
\curveto(21.81081884,220.37953855)(21.79581885,220.45953847)(21.78581787,220.53954346)
\curveto(21.76581888,220.6295383)(21.74081891,220.70953822)(21.71081787,220.77954346)
\curveto(21.57081908,221.13953779)(21.39581925,221.44453748)(21.18581787,221.69454346)
\curveto(20.96581968,221.94453698)(20.69081996,222.16953676)(20.36081787,222.36954346)
\curveto(20.2508204,222.43953649)(20.14082051,222.49453643)(20.03081787,222.53454346)
\lineto(19.70081787,222.68454346)
\curveto(19.66082099,222.71453621)(19.62582102,222.7295362)(19.59581787,222.72954346)
\curveto(19.55582109,222.73953619)(19.51582113,222.75453617)(19.47581787,222.77454346)
\curveto(19.41582123,222.79453613)(19.35582129,222.80953612)(19.29581787,222.81954346)
\curveto(19.23582141,222.8295361)(19.17582147,222.84453608)(19.11581787,222.86454346)
}
}
{
\newrgbcolor{curcolor}{0 0 0}
\pscustom[linestyle=none,fillstyle=solid,fillcolor=curcolor]
{
\newpath
\moveto(22.20581787,232.29579346)
\curveto(22.36581828,232.28578555)(22.50081815,232.24078559)(22.61081787,232.16079346)
\curveto(22.71081794,232.08078575)(22.78581786,231.98578585)(22.83581787,231.87579346)
\curveto(22.85581779,231.82578601)(22.86581778,231.77078606)(22.86581787,231.71079346)
\curveto(22.86581778,231.66078617)(22.87581777,231.60078623)(22.89581787,231.53079346)
\curveto(22.9458177,231.30078653)(22.93081772,231.08578675)(22.85081787,230.88579346)
\curveto(22.78081787,230.68578715)(22.69081796,230.56078727)(22.58081787,230.51079346)
\curveto(22.51081814,230.47078736)(22.43081822,230.44078739)(22.34081787,230.42079346)
\curveto(22.24081841,230.40078743)(22.16081849,230.36578747)(22.10081787,230.31579346)
\lineto(22.04081787,230.25579346)
\curveto(22.02081863,230.2357876)(22.01581863,230.20578763)(22.02581787,230.16579346)
\curveto(22.05581859,230.04578779)(22.11081854,229.9307879)(22.19081787,229.82079346)
\curveto(22.27081838,229.71078812)(22.34081831,229.60578823)(22.40081787,229.50579346)
\curveto(22.48081817,229.35578848)(22.55581809,229.20078863)(22.62581787,229.04079346)
\curveto(22.68581796,228.88078895)(22.74081791,228.71078912)(22.79081787,228.53079346)
\curveto(22.82081783,228.42078941)(22.84081781,228.30578953)(22.85081787,228.18579346)
\curveto(22.86081779,228.07578976)(22.87581777,227.96078987)(22.89581787,227.84079346)
\curveto(22.90581774,227.79079004)(22.91081774,227.74579009)(22.91081787,227.70579346)
\lineto(22.91081787,227.60079346)
\curveto(22.93081772,227.49079034)(22.93081772,227.38579045)(22.91081787,227.28579346)
\lineto(22.91081787,227.15079346)
\curveto(22.90081775,227.10079073)(22.89581775,227.05079078)(22.89581787,227.00079346)
\curveto(22.89581775,226.95079088)(22.88581776,226.91079092)(22.86581787,226.88079346)
\curveto(22.85581779,226.84079099)(22.8508178,226.80579103)(22.85081787,226.77579346)
\curveto(22.86081779,226.75579108)(22.86081779,226.7307911)(22.85081787,226.70079346)
\lineto(22.79081787,226.46079346)
\curveto(22.78081787,226.39079144)(22.76081789,226.32579151)(22.73081787,226.26579346)
\curveto(22.60081805,225.98579185)(22.45581819,225.77079206)(22.29581787,225.62079346)
\curveto(22.12581852,225.47079236)(21.89081876,225.36579247)(21.59081787,225.30579346)
\curveto(21.37081928,225.25579258)(21.10581954,225.26079257)(20.79581787,225.32079346)
\lineto(20.48081787,225.39579346)
\curveto(20.43082022,225.41579242)(20.38082027,225.4307924)(20.33081787,225.44079346)
\lineto(20.15081787,225.50079346)
\lineto(19.82081787,225.68079346)
\curveto(19.71082094,225.75079208)(19.61082104,225.82079201)(19.52081787,225.89079346)
\curveto(19.23082142,226.1307917)(19.01582163,226.42079141)(18.87581787,226.76079346)
\curveto(18.73582191,227.10079073)(18.61082204,227.46579037)(18.50081787,227.85579346)
\curveto(18.46082219,228.00578983)(18.43082222,228.15578968)(18.41081787,228.30579346)
\curveto(18.39082226,228.46578937)(18.36582228,228.62078921)(18.33581787,228.77079346)
\curveto(18.31582233,228.85078898)(18.30582234,228.92078891)(18.30581787,228.98079346)
\curveto(18.30582234,229.05078878)(18.29582235,229.12578871)(18.27581787,229.20579346)
\curveto(18.25582239,229.27578856)(18.2458224,229.34578849)(18.24581787,229.41579346)
\curveto(18.23582241,229.49578834)(18.22082243,229.57578826)(18.20081787,229.65579346)
\curveto(18.14082251,229.91578792)(18.09082256,230.16078767)(18.05081787,230.39079346)
\curveto(18.00082265,230.62078721)(17.88582276,230.82078701)(17.70581787,230.99079346)
\curveto(17.62582302,231.06078677)(17.52582312,231.12578671)(17.40581787,231.18579346)
\curveto(17.27582337,231.25578658)(17.13582351,231.28578655)(16.98581787,231.27579346)
\curveto(16.7458239,231.26578657)(16.55582409,231.21578662)(16.41581787,231.12579346)
\curveto(16.27582437,231.04578679)(16.16582448,230.90578693)(16.08581787,230.70579346)
\curveto(16.03582461,230.59578724)(16.00082465,230.46078737)(15.98081787,230.30079346)
\curveto(15.96082469,230.14078769)(15.9508247,229.97078786)(15.95081787,229.79079346)
\curveto(15.9508247,229.61078822)(15.96082469,229.4307884)(15.98081787,229.25079346)
\curveto(16.00082465,229.08078875)(16.03082462,228.9307889)(16.07081787,228.80079346)
\curveto(16.13082452,228.62078921)(16.21582443,228.44078939)(16.32581787,228.26079346)
\curveto(16.38582426,228.17078966)(16.46582418,228.08078975)(16.56581787,227.99079346)
\curveto(16.65582399,227.91078992)(16.75582389,227.83579)(16.86581787,227.76579346)
\curveto(16.9458237,227.71579012)(17.03082362,227.67079016)(17.12081787,227.63079346)
\curveto(17.21082344,227.59079024)(17.28082337,227.5307903)(17.33081787,227.45079346)
\curveto(17.36082329,227.40079043)(17.38582326,227.32579051)(17.40581787,227.22579346)
\curveto(17.41582323,227.12579071)(17.42082323,227.02579081)(17.42081787,226.92579346)
\curveto(17.42082323,226.82579101)(17.41582323,226.7307911)(17.40581787,226.64079346)
\curveto(17.38582326,226.55079128)(17.36082329,226.49079134)(17.33081787,226.46079346)
\curveto(17.30082335,226.42079141)(17.2508234,226.39579144)(17.18081787,226.38579346)
\curveto(17.11082354,226.38579145)(17.03582361,226.40579143)(16.95581787,226.44579346)
\curveto(16.82582382,226.49579134)(16.70582394,226.55079128)(16.59581787,226.61079346)
\curveto(16.47582417,226.67079116)(16.36082429,226.7357911)(16.25081787,226.80579346)
\curveto(15.90082475,227.06579077)(15.63082502,227.36079047)(15.44081787,227.69079346)
\curveto(15.24082541,228.02078981)(15.08082557,228.41078942)(14.96081787,228.86079346)
\curveto(14.94082571,228.97078886)(14.92582572,229.07578876)(14.91581787,229.17579346)
\curveto(14.90582574,229.28578855)(14.89082576,229.39578844)(14.87081787,229.50579346)
\curveto(14.86082579,229.55578828)(14.86082579,229.62078821)(14.87081787,229.70079346)
\curveto(14.87082578,229.79078804)(14.86082579,229.85078798)(14.84081787,229.88079346)
\curveto(14.83082582,230.58078725)(14.91082574,231.17078666)(15.08081787,231.65079346)
\curveto(15.2508254,232.14078569)(15.57582507,232.44578539)(16.05581787,232.56579346)
\curveto(16.25582439,232.61578522)(16.49082416,232.62078521)(16.76081787,232.58079346)
\curveto(17.02082363,232.54078529)(17.29582335,232.49078534)(17.58581787,232.43079346)
\lineto(20.90081787,231.77079346)
\curveto(21.04081961,231.74078609)(21.17581947,231.71578612)(21.30581787,231.69579346)
\curveto(21.43581921,231.68578615)(21.54081911,231.69578614)(21.62081787,231.72579346)
\curveto(21.69081896,231.76578607)(21.74081891,231.82078601)(21.77081787,231.89079346)
\curveto(21.81081884,231.98078585)(21.84081881,232.06078577)(21.86081787,232.13079346)
\curveto(21.87081878,232.21078562)(21.91581873,232.26078557)(21.99581787,232.28079346)
\curveto(22.02581862,232.30078553)(22.05581859,232.30578553)(22.08581787,232.29579346)
\lineto(22.20581787,232.29579346)
\moveto(20.54081787,230.48079346)
\curveto(20.40082025,230.57078726)(20.24082041,230.6357872)(20.06081787,230.67579346)
\curveto(19.87082078,230.71578712)(19.67582097,230.75578708)(19.47581787,230.79579346)
\curveto(19.36582128,230.81578702)(19.26582138,230.830787)(19.17581787,230.84079346)
\curveto(19.08582156,230.85078698)(19.01582163,230.82578701)(18.96581787,230.76579346)
\curveto(18.9458217,230.7357871)(18.93582171,230.66578717)(18.93581787,230.55579346)
\curveto(18.95582169,230.5357873)(18.96582168,230.50078733)(18.96581787,230.45079346)
\curveto(18.96582168,230.40078743)(18.97582167,230.35078748)(18.99581787,230.30079346)
\curveto(19.01582163,230.22078761)(19.03582161,230.12578771)(19.05581787,230.01579346)
\lineto(19.11581787,229.71579346)
\curveto(19.11582153,229.68578815)(19.12082153,229.65078818)(19.13081787,229.61079346)
\lineto(19.13081787,229.50579346)
\curveto(19.17082148,229.34578849)(19.19582145,229.17578866)(19.20581787,228.99579346)
\curveto(19.20582144,228.82578901)(19.22582142,228.66078917)(19.26581787,228.50079346)
\curveto(19.28582136,228.41078942)(19.30582134,228.3307895)(19.32581787,228.26079346)
\curveto(19.33582131,228.20078963)(19.3508213,228.12578971)(19.37081787,228.03579346)
\curveto(19.42082123,227.86578997)(19.48582116,227.70079013)(19.56581787,227.54079346)
\curveto(19.63582101,227.39079044)(19.72582092,227.25579058)(19.83581787,227.13579346)
\curveto(19.9458207,227.01579082)(20.08082057,226.91579092)(20.24081787,226.83579346)
\curveto(20.39082026,226.75579108)(20.57582007,226.69579114)(20.79581787,226.65579346)
\curveto(20.89581975,226.6357912)(20.99081966,226.6357912)(21.08081787,226.65579346)
\curveto(21.16081949,226.67579116)(21.23581941,226.70579113)(21.30581787,226.74579346)
\curveto(21.41581923,226.79579104)(21.51081914,226.87579096)(21.59081787,226.98579346)
\curveto(21.66081899,227.10579073)(21.72081893,227.2357906)(21.77081787,227.37579346)
\curveto(21.78081887,227.42579041)(21.78581886,227.47579036)(21.78581787,227.52579346)
\curveto(21.78581886,227.57579026)(21.79081886,227.62579021)(21.80081787,227.67579346)
\curveto(21.82081883,227.74579009)(21.83581881,227.83079)(21.84581787,227.93079346)
\curveto(21.8458188,228.0307898)(21.83581881,228.12078971)(21.81581787,228.20079346)
\curveto(21.79581885,228.26078957)(21.79081886,228.32078951)(21.80081787,228.38079346)
\curveto(21.80081885,228.44078939)(21.79081886,228.50078933)(21.77081787,228.56079346)
\curveto(21.7508189,228.65078918)(21.73581891,228.7307891)(21.72581787,228.80079346)
\curveto(21.71581893,228.88078895)(21.69581895,228.96078887)(21.66581787,229.04079346)
\curveto(21.5458191,229.35078848)(21.40081925,229.62578821)(21.23081787,229.86579346)
\curveto(21.06081959,230.10578773)(20.83081982,230.31078752)(20.54081787,230.48079346)
}
}
{
\newrgbcolor{curcolor}{0 0 0}
\pscustom[linestyle=none,fillstyle=solid,fillcolor=curcolor]
{
\newpath
\moveto(21.95081787,240.47243408)
\lineto(22.34081787,240.38243408)
\curveto(22.46081819,240.36242615)(22.56081809,240.32242619)(22.64081787,240.26243408)
\curveto(22.71081794,240.19242632)(22.7508179,240.09742642)(22.76081787,239.97743408)
\lineto(22.76081787,239.63243408)
\curveto(22.76081789,239.57242694)(22.76581788,239.512427)(22.77581787,239.45243408)
\curveto(22.77581787,239.40242711)(22.76581788,239.35742716)(22.74581787,239.31743408)
\curveto(22.72581792,239.23742728)(22.68581796,239.18742733)(22.62581787,239.16743408)
\curveto(22.57581807,239.13742738)(22.51581813,239.12742739)(22.44581787,239.13743408)
\curveto(22.37581827,239.14742737)(22.30581834,239.14242737)(22.23581787,239.12243408)
\curveto(22.21581843,239.12242739)(22.20081845,239.1124274)(22.19081787,239.09243408)
\lineto(22.13081787,239.06243408)
\curveto(22.12081853,238.96242755)(22.14081851,238.87742764)(22.19081787,238.80743408)
\curveto(22.24081841,238.74742777)(22.29081836,238.68242783)(22.34081787,238.61243408)
\curveto(22.49081816,238.38242813)(22.60581804,238.15742836)(22.68581787,237.93743408)
\curveto(22.76581788,237.74742877)(22.82581782,237.52742899)(22.86581787,237.27743408)
\curveto(22.90581774,237.03742948)(22.92581772,236.79242972)(22.92581787,236.54243408)
\curveto(22.93581771,236.30243021)(22.92081773,236.06243045)(22.88081787,235.82243408)
\curveto(22.8508178,235.59243092)(22.79581785,235.39743112)(22.71581787,235.23743408)
\curveto(22.49581815,234.75743176)(22.20081845,234.39243212)(21.83081787,234.14243408)
\curveto(21.4508192,233.90243261)(20.98081967,233.74743277)(20.42081787,233.67743408)
\curveto(20.33082032,233.65743286)(20.24082041,233.64743287)(20.15081787,233.64743408)
\curveto(20.0508206,233.65743286)(19.9508207,233.65743286)(19.85081787,233.64743408)
\curveto(19.80082085,233.64743287)(19.7508209,233.65243286)(19.70081787,233.66243408)
\curveto(19.650821,233.67243284)(19.60082105,233.67743284)(19.55081787,233.67743408)
\curveto(19.50082115,233.66743285)(19.4508212,233.66743285)(19.40081787,233.67743408)
\curveto(19.34082131,233.69743282)(19.28582136,233.70743281)(19.23581787,233.70743408)
\lineto(19.08581787,233.73743408)
\curveto(19.03582161,233.72743279)(18.97082168,233.72743279)(18.89081787,233.73743408)
\curveto(18.81082184,233.75743276)(18.7458219,233.78243273)(18.69581787,233.81243408)
\lineto(18.53081787,233.85743408)
\curveto(18.46082219,233.88743263)(18.39082226,233.90743261)(18.32081787,233.91743408)
\curveto(18.24082241,233.92743259)(18.16582248,233.94743257)(18.09581787,233.97743408)
\curveto(18.0458226,233.99743252)(18.00082265,234.0124325)(17.96081787,234.02243408)
\curveto(17.92082273,234.03243248)(17.87582277,234.04743247)(17.82581787,234.06743408)
\curveto(17.72582292,234.1174324)(17.63082302,234.16243235)(17.54081787,234.20243408)
\curveto(17.44082321,234.24243227)(17.3458233,234.28743223)(17.25581787,234.33743408)
\curveto(16.87582377,234.53743198)(16.53582411,234.76743175)(16.23581787,235.02743408)
\curveto(15.92582472,235.29743122)(15.67082498,235.59743092)(15.47081787,235.92743408)
\curveto(15.3508253,236.12743039)(15.2508254,236.32743019)(15.17081787,236.52743408)
\curveto(15.09082556,236.72742979)(15.02082563,236.94242957)(14.96081787,237.17243408)
\lineto(14.93081787,237.38243408)
\curveto(14.92082573,237.45242906)(14.90582574,237.52242899)(14.88581787,237.59243408)
\lineto(14.88581787,237.74243408)
\curveto(14.86582578,237.83242868)(14.85582579,237.95242856)(14.85581787,238.10243408)
\curveto(14.85582579,238.26242825)(14.86582578,238.37742814)(14.88581787,238.44743408)
\curveto(14.89582575,238.48742803)(14.90082575,238.54242797)(14.90081787,238.61243408)
\curveto(14.93082572,238.7124278)(14.95582569,238.8174277)(14.97581787,238.92743408)
\curveto(14.98582566,239.03742748)(15.01582563,239.13742738)(15.06581787,239.22743408)
\curveto(15.12582552,239.36742715)(15.19082546,239.49742702)(15.26081787,239.61743408)
\curveto(15.33082532,239.73742678)(15.41082524,239.84742667)(15.50081787,239.94743408)
\curveto(15.5508251,239.99742652)(15.60582504,240.04742647)(15.66581787,240.09743408)
\curveto(15.71582493,240.15742636)(15.73082492,240.24242627)(15.71081787,240.35243408)
\lineto(15.63581787,240.42743408)
\curveto(15.61582503,240.44742607)(15.58582506,240.46242605)(15.54581787,240.47243408)
\curveto(15.45582519,240.52242599)(15.34082531,240.55742596)(15.20081787,240.57743408)
\curveto(15.06082559,240.60742591)(14.93582571,240.63242588)(14.82581787,240.65243408)
\lineto(13.10081787,240.99743408)
\curveto(12.96082769,241.02742549)(12.80582784,241.05742546)(12.63581787,241.08743408)
\curveto(12.45582819,241.12742539)(12.32582832,241.17742534)(12.24581787,241.23743408)
\curveto(12.17582847,241.29742522)(12.13082852,241.36742515)(12.11081787,241.44743408)
\curveto(12.11082854,241.46742505)(12.11082854,241.49242502)(12.11081787,241.52243408)
\curveto(12.10082855,241.55242496)(12.09582855,241.57742494)(12.09581787,241.59743408)
\curveto(12.08582856,241.74742477)(12.08582856,241.89742462)(12.09581787,242.04743408)
\curveto(12.09582855,242.19742432)(12.13582851,242.29742422)(12.21581787,242.34743408)
\curveto(12.29582835,242.37742414)(12.39582825,242.37742414)(12.51581787,242.34743408)
\curveto(12.63582801,242.32742419)(12.76082789,242.30742421)(12.89081787,242.28743408)
\lineto(21.95081787,240.47243408)
\moveto(19.11581787,239.82743408)
\curveto(19.06582158,239.85742666)(19.00082165,239.87742664)(18.92081787,239.88743408)
\curveto(18.83082182,239.90742661)(18.76082189,239.9124266)(18.71081787,239.90243408)
\lineto(18.48581787,239.94743408)
\curveto(18.39582225,239.94742657)(18.30582234,239.95242656)(18.21581787,239.96243408)
\curveto(18.11582253,239.97242654)(18.02582262,239.96742655)(17.94581787,239.94743408)
\lineto(17.72081787,239.94743408)
\curveto(17.650823,239.94742657)(17.58082307,239.93742658)(17.51081787,239.91743408)
\curveto(17.21082344,239.85742666)(16.9458237,239.75242676)(16.71581787,239.60243408)
\curveto(16.48582416,239.46242705)(16.30582434,239.26242725)(16.17581787,239.00243408)
\curveto(16.12582452,238.9124276)(16.09082456,238.8174277)(16.07081787,238.71743408)
\curveto(16.04082461,238.6174279)(16.01582463,238.50742801)(15.99581787,238.38743408)
\curveto(15.97582467,238.3174282)(15.96582468,238.23242828)(15.96581787,238.13243408)
\lineto(15.96581787,237.86243408)
\lineto(15.99581787,237.71243408)
\lineto(15.99581787,237.57743408)
\curveto(16.01582463,237.49742902)(16.03582461,237.4124291)(16.05581787,237.32243408)
\curveto(16.07582457,237.23242928)(16.10082455,237.14742937)(16.13081787,237.06743408)
\curveto(16.27082438,236.7174298)(16.47582417,236.4174301)(16.74581787,236.16743408)
\curveto(17.00582364,235.9174306)(17.31082334,235.69743082)(17.66081787,235.50743408)
\curveto(17.77082288,235.44743107)(17.88582276,235.39743112)(18.00581787,235.35743408)
\lineto(18.33581787,235.23743408)
\lineto(18.45581787,235.20743408)
\curveto(18.48582216,235.19743132)(18.52082213,235.18743133)(18.56081787,235.17743408)
\curveto(18.61082204,235.14743137)(18.66582198,235.12743139)(18.72581787,235.11743408)
\curveto(18.78582186,235.1174314)(18.84082181,235.1124314)(18.89081787,235.10243408)
\curveto(19.00082165,235.08243143)(19.11082154,235.05743146)(19.22081787,235.02743408)
\curveto(19.32082133,235.00743151)(19.41582123,235.00243151)(19.50581787,235.01243408)
\curveto(19.53582111,235.0124315)(19.58582106,235.00743151)(19.65581787,234.99743408)
\lineto(19.86581787,234.99743408)
\curveto(19.93582071,234.99743152)(20.00582064,235.00243151)(20.07581787,235.01243408)
\curveto(20.42582022,235.05243146)(20.72581992,235.14243137)(20.97581787,235.28243408)
\curveto(21.22581942,235.42243109)(21.43081922,235.62243089)(21.59081787,235.88243408)
\curveto(21.64081901,235.96243055)(21.68081897,236.04243047)(21.71081787,236.12243408)
\curveto(21.74081891,236.2124303)(21.77081888,236.30743021)(21.80081787,236.40743408)
\curveto(21.82081883,236.45743006)(21.82581882,236.50743001)(21.81581787,236.55743408)
\curveto(21.80581884,236.6174299)(21.81081884,236.67242984)(21.83081787,236.72243408)
\curveto(21.84081881,236.75242976)(21.8458188,236.78742973)(21.84581787,236.82743408)
\lineto(21.84581787,236.96243408)
\lineto(21.84581787,237.09743408)
\curveto(21.83581881,237.13742938)(21.83081882,237.19242932)(21.83081787,237.26243408)
\curveto(21.81081884,237.34242917)(21.79581885,237.42242909)(21.78581787,237.50243408)
\curveto(21.76581888,237.59242892)(21.74081891,237.67242884)(21.71081787,237.74243408)
\curveto(21.57081908,238.10242841)(21.39581925,238.40742811)(21.18581787,238.65743408)
\curveto(20.96581968,238.90742761)(20.69081996,239.13242738)(20.36081787,239.33243408)
\curveto(20.2508204,239.40242711)(20.14082051,239.45742706)(20.03081787,239.49743408)
\lineto(19.70081787,239.64743408)
\curveto(19.66082099,239.67742684)(19.62582102,239.69242682)(19.59581787,239.69243408)
\curveto(19.55582109,239.70242681)(19.51582113,239.7174268)(19.47581787,239.73743408)
\curveto(19.41582123,239.75742676)(19.35582129,239.77242674)(19.29581787,239.78243408)
\curveto(19.23582141,239.79242672)(19.17582147,239.80742671)(19.11581787,239.82743408)
}
}
{
\newrgbcolor{curcolor}{0 0 0}
\pscustom[linestyle=none,fillstyle=solid,fillcolor=curcolor]
{
\newpath
\moveto(18.59081787,249.84368408)
\curveto(18.69082196,249.84367558)(18.80582184,249.8236756)(18.93581787,249.78368408)
\curveto(19.05582159,249.74367568)(19.14082151,249.69367573)(19.19081787,249.63368408)
\curveto(19.23082142,249.57367585)(19.26082139,249.49367593)(19.28081787,249.39368408)
\curveto(19.29082136,249.29367613)(19.29582135,249.18367624)(19.29581787,249.06368408)
\lineto(19.29581787,248.70368408)
\curveto(19.28582136,248.59367683)(19.28082137,248.49367693)(19.28081787,248.40368408)
\lineto(19.28081787,244.56368408)
\curveto(19.28082137,244.48368094)(19.28582136,244.39868102)(19.29581787,244.30868408)
\curveto(19.29582135,244.22868119)(19.31082134,244.16368126)(19.34081787,244.11368408)
\curveto(19.36082129,244.06368136)(19.40082125,244.01368141)(19.46081787,243.96368408)
\lineto(19.59581787,243.87368408)
\curveto(19.645821,243.84368158)(19.69582095,243.83368159)(19.74581787,243.84368408)
\curveto(19.79582085,243.84368158)(19.84082081,243.83868158)(19.88081787,243.82868408)
\lineto(20.00081787,243.82868408)
\lineto(20.25581787,243.82868408)
\curveto(20.33582031,243.83868158)(20.41582023,243.85368157)(20.49581787,243.87368408)
\curveto(21.03581961,244.00368142)(21.42081923,244.30868111)(21.65081787,244.78868408)
\curveto(21.68081897,244.83868058)(21.70581894,244.89868052)(21.72581787,244.96868408)
\curveto(21.7458189,245.03868038)(21.76581888,245.10368032)(21.78581787,245.16368408)
\curveto(21.79581885,245.19368023)(21.80081885,245.24368018)(21.80081787,245.31368408)
\curveto(21.84081881,245.44367998)(21.86081879,245.6236798)(21.86081787,245.85368408)
\curveto(21.86081879,246.08367934)(21.84081881,246.27367915)(21.80081787,246.42368408)
\curveto(21.76081889,246.57367885)(21.72081893,246.70867871)(21.68081787,246.82868408)
\curveto(21.63081902,246.95867846)(21.57081908,247.07867834)(21.50081787,247.18868408)
\curveto(21.43081922,247.30867811)(21.3508193,247.418678)(21.26081787,247.51868408)
\curveto(21.16081949,247.6186778)(21.05581959,247.70867771)(20.94581787,247.78868408)
\curveto(20.8458198,247.86867755)(20.74081991,247.94367748)(20.63081787,248.01368408)
\curveto(20.52082013,248.08367734)(20.44082021,248.17867724)(20.39081787,248.29868408)
\curveto(20.37082028,248.33867708)(20.35582029,248.40367702)(20.34581787,248.49368408)
\curveto(20.33582031,248.59367683)(20.33582031,248.68367674)(20.34581787,248.76368408)
\curveto(20.3458203,248.85367657)(20.3508203,248.93867648)(20.36081787,249.01868408)
\curveto(20.37082028,249.09867632)(20.39082026,249.14867627)(20.42081787,249.16868408)
\curveto(20.49082016,249.25867616)(20.60582004,249.26367616)(20.76581787,249.18368408)
\curveto(21.03581961,249.04367638)(21.27581937,248.88867653)(21.48581787,248.71868408)
\curveto(21.80581884,248.45867696)(22.07081858,248.17867724)(22.28081787,247.87868408)
\curveto(22.48081817,247.58867783)(22.645818,247.23367819)(22.77581787,246.81368408)
\curveto(22.81581783,246.70367872)(22.84081781,246.59867882)(22.85081787,246.49868408)
\curveto(22.87081778,246.39867902)(22.89081776,246.28867913)(22.91081787,246.16868408)
\curveto(22.92081773,246.1186793)(22.92581772,246.06867935)(22.92581787,246.01868408)
\curveto(22.92581772,245.97867944)(22.93081772,245.93367949)(22.94081787,245.88368408)
\lineto(22.94081787,245.73368408)
\curveto(22.9508177,245.68367974)(22.95581769,245.6236798)(22.95581787,245.55368408)
\curveto(22.95581769,245.49367993)(22.9508177,245.44367998)(22.94081787,245.40368408)
\lineto(22.94081787,245.26868408)
\curveto(22.93081772,245.2186802)(22.92581772,245.17368025)(22.92581787,245.13368408)
\curveto(22.92581772,245.09368033)(22.92081773,245.05368037)(22.91081787,245.01368408)
\curveto(22.90081775,244.96368046)(22.89081776,244.90868051)(22.88081787,244.84868408)
\curveto(22.88081777,244.79868062)(22.87581777,244.74868067)(22.86581787,244.69868408)
\curveto(22.8458178,244.60868081)(22.82081783,244.5186809)(22.79081787,244.42868408)
\curveto(22.77081788,244.34868107)(22.7458179,244.27368115)(22.71581787,244.20368408)
\curveto(22.69581795,244.16368126)(22.68581796,244.12868129)(22.68581787,244.09868408)
\curveto(22.67581797,244.06868135)(22.66081799,244.03868138)(22.64081787,244.00868408)
\curveto(22.57081808,243.86868155)(22.48581816,243.7236817)(22.38581787,243.57368408)
\curveto(22.19581845,243.3236821)(21.96581868,243.1236823)(21.69581787,242.97368408)
\curveto(21.41581923,242.8236826)(21.10581954,242.71368271)(20.76581787,242.64368408)
\curveto(20.65581999,242.61368281)(20.54082011,242.59868282)(20.42081787,242.59868408)
\curveto(20.30082035,242.59868282)(20.18082047,242.58868283)(20.06081787,242.56868408)
\lineto(19.95581787,242.56868408)
\curveto(19.92582072,242.57868284)(19.88582076,242.58368284)(19.83581787,242.58368408)
\lineto(19.58081787,242.58368408)
\curveto(19.49082116,242.59368283)(19.40082125,242.59868282)(19.31081787,242.59868408)
\lineto(19.10081787,242.64368408)
\curveto(19.06082159,242.64368278)(19.00582164,242.64868277)(18.93581787,242.65868408)
\curveto(18.85582179,242.66868275)(18.79082186,242.68368274)(18.74081787,242.70368408)
\lineto(18.57581787,242.73368408)
\curveto(18.52582212,242.76368266)(18.47582217,242.77868264)(18.42581787,242.77868408)
\curveto(18.36582228,242.78868263)(18.31082234,242.80368262)(18.26081787,242.82368408)
\curveto(18.10082255,242.89368253)(17.94082271,242.95868246)(17.78081787,243.01868408)
\curveto(17.62082303,243.07868234)(17.47082318,243.15368227)(17.33081787,243.24368408)
\curveto(17.22082343,243.31368211)(17.11082354,243.37868204)(17.00081787,243.43868408)
\curveto(16.88082377,243.50868191)(16.76582388,243.58868183)(16.65581787,243.67868408)
\curveto(16.30582434,243.96868145)(16.00582464,244.27868114)(15.75581787,244.60868408)
\curveto(15.49582515,244.93868048)(15.28082537,245.3236801)(15.11081787,245.76368408)
\curveto(15.06082559,245.89367953)(15.02582562,246.0236794)(15.00581787,246.15368408)
\curveto(14.97582567,246.28367914)(14.9458257,246.423679)(14.91581787,246.57368408)
\curveto(14.90582574,246.6236788)(14.90082575,246.66867875)(14.90081787,246.70868408)
\curveto(14.89082576,246.74867867)(14.88582576,246.79367863)(14.88581787,246.84368408)
\curveto(14.87582577,246.86367856)(14.87582577,246.88867853)(14.88581787,246.91868408)
\curveto(14.89582575,246.94867847)(14.89082576,246.97367845)(14.87081787,246.99368408)
\curveto(14.86082579,247.423678)(14.90582574,247.78367764)(15.00581787,248.07368408)
\curveto(15.09582555,248.36367706)(15.22082543,248.6186768)(15.38081787,248.83868408)
\curveto(15.40082525,248.87867654)(15.43082522,248.90867651)(15.47081787,248.92868408)
\curveto(15.50082515,248.95867646)(15.52582512,248.98867643)(15.54581787,249.01868408)
\curveto(15.60582504,249.08867633)(15.67582497,249.15867626)(15.75581787,249.22868408)
\curveto(15.83582481,249.29867612)(15.91582473,249.35367607)(15.99581787,249.39368408)
\curveto(16.20582444,249.51367591)(16.40582424,249.60867581)(16.59581787,249.67868408)
\curveto(16.70582394,249.72867569)(16.82582382,249.75867566)(16.95581787,249.76868408)
\lineto(17.34581787,249.82868408)
\curveto(17.47582317,249.85867556)(17.61082304,249.86867555)(17.75081787,249.85868408)
\curveto(17.89082276,249.85867556)(18.03082262,249.86367556)(18.17081787,249.87368408)
\curveto(18.24082241,249.87367555)(18.31082234,249.86867555)(18.38081787,249.85868408)
\curveto(18.4508222,249.84867557)(18.52082213,249.84367558)(18.59081787,249.84368408)
\moveto(18.08081787,248.49368408)
\curveto(18.04082261,248.5236769)(17.99082266,248.55367687)(17.93081787,248.58368408)
\curveto(17.86082279,248.6236768)(17.79082286,248.63867678)(17.72081787,248.62868408)
\curveto(17.50082315,248.6186768)(17.29582335,248.57867684)(17.10581787,248.50868408)
\curveto(16.87582377,248.40867701)(16.68082397,248.28867713)(16.52081787,248.14868408)
\curveto(16.36082429,248.0186774)(16.22582442,247.82867759)(16.11581787,247.57868408)
\curveto(16.09582455,247.50867791)(16.08082457,247.43867798)(16.07081787,247.36868408)
\curveto(16.0508246,247.30867811)(16.03082462,247.23867818)(16.01081787,247.15868408)
\curveto(15.99082466,247.08867833)(15.98082467,247.00867841)(15.98081787,246.91868408)
\lineto(15.98081787,246.66368408)
\curveto(16.00082465,246.6236788)(16.01082464,246.58367884)(16.01081787,246.54368408)
\curveto(16.00082465,246.50367892)(16.00082465,246.46867895)(16.01081787,246.43868408)
\lineto(16.07081787,246.19868408)
\curveto(16.08082457,246.1186793)(16.09582455,246.04367938)(16.11581787,245.97368408)
\curveto(16.23582441,245.65367977)(16.38582426,245.38868003)(16.56581787,245.17868408)
\curveto(16.7458239,244.96868045)(16.97082368,244.76868065)(17.24081787,244.57868408)
\curveto(17.29082336,244.53868088)(17.35582329,244.49368093)(17.43581787,244.44368408)
\curveto(17.50582314,244.40368102)(17.58582306,244.36368106)(17.67581787,244.32368408)
\curveto(17.76582288,244.28368114)(17.8508228,244.25868116)(17.93081787,244.24868408)
\curveto(18.01082264,244.24868117)(18.07082258,244.27368115)(18.11081787,244.32368408)
\curveto(18.17082248,244.39368103)(18.20082245,244.5236809)(18.20081787,244.71368408)
\curveto(18.19082246,244.91368051)(18.18582246,245.08368034)(18.18581787,245.22368408)
\lineto(18.18581787,247.50368408)
\curveto(18.18582246,247.65367777)(18.19082246,247.83367759)(18.20081787,248.04368408)
\curveto(18.20082245,248.25367717)(18.16082249,248.40367702)(18.08081787,248.49368408)
}
}
{
\newrgbcolor{curcolor}{0 0 0}
\pscustom[linestyle=none,fillstyle=solid,fillcolor=curcolor]
{
\newpath
\moveto(14.85581787,254.33032471)
\curveto(14.8458258,255.05031905)(14.93082572,255.63531847)(15.11081787,256.08532471)
\curveto(15.28082537,256.54531756)(15.58582506,256.86531724)(16.02581787,257.04532471)
\curveto(16.13582451,257.09531701)(16.2508244,257.12531698)(16.37081787,257.13532471)
\curveto(16.48082417,257.15531695)(16.60582404,257.17031693)(16.74581787,257.18032471)
\curveto(16.81582383,257.19031691)(16.89082376,257.18031692)(16.97081787,257.15032471)
\curveto(17.04082361,257.13031697)(17.09582355,257.105317)(17.13581787,257.07532471)
\curveto(17.15582349,257.05531705)(17.17582347,257.02531708)(17.19581787,256.98532471)
\curveto(17.20582344,256.95531715)(17.22082343,256.93031717)(17.24081787,256.91032471)
\curveto(17.26082339,256.85031725)(17.26582338,256.79531731)(17.25581787,256.74532471)
\curveto(17.2458234,256.7053174)(17.2458234,256.66031744)(17.25581787,256.61032471)
\curveto(17.27582337,256.52031758)(17.28082337,256.41031769)(17.27081787,256.28032471)
\curveto(17.2508234,256.16031794)(17.22582342,256.07531803)(17.19581787,256.02532471)
\curveto(17.1458235,255.95531815)(17.08082357,255.91531819)(17.00081787,255.90532471)
\curveto(16.91082374,255.9053182)(16.82582382,255.88531822)(16.74581787,255.84532471)
\curveto(16.58582406,255.79531831)(16.44082421,255.7003184)(16.31081787,255.56032471)
\curveto(16.23082442,255.47031863)(16.17082448,255.36031874)(16.13081787,255.23032471)
\curveto(16.09082456,255.11031899)(16.0508246,254.98031912)(16.01081787,254.84032471)
\curveto(15.99082466,254.8003193)(15.98582466,254.75031935)(15.99581787,254.69032471)
\curveto(15.99582465,254.64031946)(15.99082466,254.59531951)(15.98081787,254.55532471)
\curveto(15.96082469,254.49531961)(15.9508247,254.42031968)(15.95081787,254.33032471)
\curveto(15.9508247,254.24031986)(15.96082469,254.16531994)(15.98081787,254.10532471)
\lineto(15.98081787,254.01532471)
\curveto(15.99082466,253.95532015)(16.00082465,253.9003202)(16.01081787,253.85032471)
\curveto(16.01082464,253.8003203)(16.01582463,253.75032035)(16.02581787,253.70032471)
\curveto(16.08582456,253.43032067)(16.17082448,253.19532091)(16.28081787,252.99532471)
\curveto(16.39082426,252.8053213)(16.57582407,252.65532145)(16.83581787,252.54532471)
\curveto(16.90582374,252.51532159)(16.97582367,252.5003216)(17.04581787,252.50032471)
\curveto(17.11582353,252.5003216)(17.17582347,252.5053216)(17.22581787,252.51532471)
\curveto(17.37582327,252.54532156)(17.48582316,252.59532151)(17.55581787,252.66532471)
\curveto(17.61582303,252.73532137)(17.68582296,252.83032127)(17.76581787,252.95032471)
\curveto(17.86582278,253.09032101)(17.94082271,253.25532085)(17.99081787,253.44532471)
\curveto(18.03082262,253.63532047)(18.08082257,253.82532028)(18.14081787,254.01532471)
\curveto(18.18082247,254.13531997)(18.21082244,254.25531985)(18.23081787,254.37532471)
\curveto(18.2508224,254.5053196)(18.28082237,254.63031947)(18.32081787,254.75032471)
\curveto(18.38082227,254.95031915)(18.44082221,255.14531896)(18.50081787,255.33532471)
\curveto(18.5508221,255.52531858)(18.61582203,255.71031839)(18.69581787,255.89032471)
\curveto(18.71582193,255.94031816)(18.73582191,255.98531812)(18.75581787,256.02532471)
\curveto(18.77582187,256.07531803)(18.80082185,256.12531798)(18.83081787,256.17532471)
\curveto(18.9508217,256.34531776)(19.08582156,256.49031761)(19.23581787,256.61032471)
\curveto(19.38582126,256.73031737)(19.57582107,256.82031728)(19.80581787,256.88032471)
\lineto(20.09081787,256.88032471)
\curveto(20.16082049,256.88031722)(20.23582041,256.87531723)(20.31581787,256.86532471)
\curveto(20.38582026,256.85531725)(20.46582018,256.84531726)(20.55581787,256.83532471)
\lineto(20.70581787,256.80532471)
\curveto(20.77581987,256.76531734)(20.8458198,256.73531737)(20.91581787,256.71532471)
\curveto(20.98581966,256.7053174)(21.05581959,256.68531742)(21.12581787,256.65532471)
\curveto(21.23581941,256.6053175)(21.34081931,256.55031755)(21.44081787,256.49032471)
\curveto(21.54081911,256.43031767)(21.63081902,256.36531774)(21.71081787,256.29532471)
\curveto(21.97081868,256.08531802)(22.18081847,255.84031826)(22.34081787,255.56032471)
\curveto(22.49081816,255.28031882)(22.62081803,254.97531913)(22.73081787,254.64532471)
\curveto(22.76081789,254.54531956)(22.78081787,254.44531966)(22.79081787,254.34532471)
\curveto(22.81081784,254.24531986)(22.83581781,254.15031995)(22.86581787,254.06032471)
\curveto(22.88581776,253.95032015)(22.89581775,253.84532026)(22.89581787,253.74532471)
\curveto(22.89581775,253.64532046)(22.90581774,253.54532056)(22.92581787,253.44532471)
\lineto(22.92581787,253.29532471)
\curveto(22.93581771,253.24532086)(22.94081771,253.17532093)(22.94081787,253.08532471)
\curveto(22.94081771,252.99532111)(22.93581771,252.92532118)(22.92581787,252.87532471)
\lineto(22.92581787,252.71032471)
\curveto(22.90581774,252.65032145)(22.89581775,252.58532152)(22.89581787,252.51532471)
\curveto(22.90581774,252.44532166)(22.90081775,252.39032171)(22.88081787,252.35032471)
\curveto(22.87081778,252.3003218)(22.86581778,252.23532187)(22.86581787,252.15532471)
\curveto(22.8458178,252.07532203)(22.82581782,252.0003221)(22.80581787,251.93032471)
\curveto(22.79581785,251.86032224)(22.77581787,251.78532232)(22.74581787,251.70532471)
\curveto(22.645818,251.41532269)(22.52081813,251.17032293)(22.37081787,250.97032471)
\curveto(22.22081843,250.77032333)(22.02581862,250.61032349)(21.78581787,250.49032471)
\curveto(21.65581899,250.43032367)(21.52081913,250.38032372)(21.38081787,250.34032471)
\curveto(21.24081941,250.31032379)(21.08581956,250.29032381)(20.91581787,250.28032471)
\curveto(20.85581979,250.27032383)(20.78581986,250.27532383)(20.70581787,250.29532471)
\curveto(20.61582003,250.31532379)(20.5458201,250.34032376)(20.49581787,250.37032471)
\curveto(20.45582019,250.41032369)(20.41582023,250.47032363)(20.37581787,250.55032471)
\curveto(20.35582029,250.6003235)(20.3458203,250.67032343)(20.34581787,250.76032471)
\curveto(20.33582031,250.86032324)(20.33582031,250.95032315)(20.34581787,251.03032471)
\curveto(20.35582029,251.12032298)(20.37082028,251.2053229)(20.39081787,251.28532471)
\curveto(20.40082025,251.37532273)(20.41582023,251.43032267)(20.43581787,251.45032471)
\curveto(20.48582016,251.51032259)(20.56082009,251.54032256)(20.66081787,251.54032471)
\curveto(20.7508199,251.55032255)(20.83581981,251.57032253)(20.91581787,251.60032471)
\curveto(21.13581951,251.65032245)(21.30581934,251.75032235)(21.42581787,251.90032471)
\curveto(21.51581913,252.0003221)(21.58581906,252.12032198)(21.63581787,252.26032471)
\curveto(21.68581896,252.4003217)(21.73581891,252.55032155)(21.78581787,252.71032471)
\lineto(21.83081787,253.02532471)
\lineto(21.83081787,253.11532471)
\curveto(21.8508188,253.17532093)(21.86081879,253.26032084)(21.86081787,253.37032471)
\curveto(21.86081879,253.49032061)(21.8508188,253.59532051)(21.83081787,253.68532471)
\curveto(21.83081882,253.75532035)(21.82581882,253.81032029)(21.81581787,253.85032471)
\curveto(21.80581884,253.91032019)(21.80081885,253.97032013)(21.80081787,254.03032471)
\curveto(21.79081886,254.09032001)(21.78081887,254.14531996)(21.77081787,254.19532471)
\curveto(21.69081896,254.5053196)(21.58581906,254.75531935)(21.45581787,254.94532471)
\curveto(21.32581932,255.14531896)(21.10581954,255.31031879)(20.79581787,255.44032471)
\curveto(20.7458199,255.47031863)(20.69081996,255.48531862)(20.63081787,255.48532471)
\curveto(20.57082008,255.49531861)(20.52582012,255.49531861)(20.49581787,255.48532471)
\curveto(20.30582034,255.47531863)(20.16582048,255.43531867)(20.07581787,255.36532471)
\curveto(19.97582067,255.29531881)(19.88582076,255.2003189)(19.80581787,255.08032471)
\curveto(19.7458209,255.0003191)(19.69582095,254.9053192)(19.65581787,254.79532471)
\lineto(19.53581787,254.49532471)
\curveto(19.52582112,254.46531964)(19.52082113,254.43531967)(19.52081787,254.40532471)
\curveto(19.52082113,254.38531972)(19.51082114,254.36531974)(19.49081787,254.34532471)
\curveto(19.38082127,254.02532008)(19.30082135,253.68532042)(19.25081787,253.32532471)
\curveto(19.19082146,252.97532113)(19.09582155,252.65532145)(18.96581787,252.36532471)
\curveto(18.92582172,252.27532183)(18.89082176,252.18532192)(18.86081787,252.09532471)
\curveto(18.83082182,252.01532209)(18.79082186,251.94032216)(18.74081787,251.87032471)
\curveto(18.63082202,251.7003224)(18.50582214,251.55032255)(18.36581787,251.42032471)
\curveto(18.22582242,251.29032281)(18.0508226,251.2003229)(17.84081787,251.15032471)
\curveto(17.77082288,251.13032297)(17.70082295,251.12032298)(17.63081787,251.12032471)
\lineto(17.40581787,251.12032471)
\curveto(17.28582336,251.11032299)(17.1508235,251.12532298)(17.00081787,251.16532471)
\curveto(16.84082381,251.2053229)(16.70582394,251.24532286)(16.59581787,251.28532471)
\curveto(16.5458241,251.31532279)(16.50582414,251.33532277)(16.47581787,251.34532471)
\curveto(16.43582421,251.36532274)(16.39582425,251.39032271)(16.35581787,251.42032471)
\curveto(16.12582452,251.55032255)(15.92582472,251.71032239)(15.75581787,251.90032471)
\curveto(15.58582506,252.09032201)(15.43582521,252.3003218)(15.30581787,252.53032471)
\curveto(15.21582543,252.69032141)(15.1458255,252.86532124)(15.09581787,253.05532471)
\curveto(15.03582561,253.25532085)(14.98082567,253.46032064)(14.93081787,253.67032471)
\curveto(14.92082573,253.74032036)(14.91082574,253.8053203)(14.90081787,253.86532471)
\curveto(14.89082576,253.93532017)(14.88082577,254.01032009)(14.87081787,254.09032471)
\curveto(14.86082579,254.13031997)(14.86082579,254.17031993)(14.87081787,254.21032471)
\curveto(14.88082577,254.26031984)(14.87582577,254.3003198)(14.85581787,254.33032471)
}
}
{
\newrgbcolor{curcolor}{0 0 0}
\pscustom[linestyle=none,fillstyle=solid,fillcolor=curcolor]
{
\newpath
\moveto(131.34959778,65.43651611)
\curveto(131.52959601,65.43650542)(131.72959581,65.43650542)(131.94959778,65.43651611)
\curveto(132.16959537,65.44650541)(132.3345952,65.41150544)(132.44459778,65.33151611)
\curveto(132.52459501,65.27150558)(132.59959494,65.18150567)(132.66959778,65.06151611)
\curveto(132.7395948,64.9515059)(132.80459473,64.851506)(132.86459778,64.76151611)
\curveto(132.99459454,64.56150629)(133.12459441,64.3565065)(133.25459778,64.14651611)
\curveto(133.39459414,63.94650691)(133.52959401,63.74150711)(133.65959778,63.53151611)
\lineto(133.86959778,63.20151611)
\curveto(133.94959359,63.10150775)(134.02459351,62.99650786)(134.09459778,62.88651611)
\curveto(134.39459314,62.40650845)(134.69959284,61.92650893)(135.00959778,61.44651611)
\curveto(135.31959222,60.97650988)(135.62959191,60.50151035)(135.93959778,60.02151611)
\curveto(136.01959152,59.88151097)(136.10459143,59.74651111)(136.19459778,59.61651611)
\curveto(136.29459124,59.49651136)(136.38459115,59.36651149)(136.46459778,59.22651611)
\lineto(136.97459778,58.41651611)
\curveto(137.15459038,58.1565127)(137.32959021,57.89651296)(137.49959778,57.63651611)
\curveto(137.54958999,57.5565133)(137.60958993,57.4565134)(137.67959778,57.33651611)
\curveto(137.75958978,57.22651363)(137.85458968,57.17151368)(137.96459778,57.17151611)
\curveto(138.01458952,57.19151366)(138.0395895,57.20651365)(138.03959778,57.21651611)
\curveto(138.08958945,57.27651358)(138.11458942,57.36151349)(138.11459778,57.47151611)
\lineto(138.11459778,57.78651611)
\lineto(138.11459778,58.97151611)
\lineto(138.11459778,63.56151611)
\lineto(138.11459778,64.46151611)
\curveto(138.11458942,64.53150632)(138.10958943,64.60650625)(138.09959778,64.68651611)
\curveto(138.08958945,64.76650609)(138.09458944,64.84150601)(138.11459778,64.91151611)
\lineto(138.11459778,65.07651611)
\curveto(138.1345894,65.11650574)(138.14458939,65.1565057)(138.14459778,65.19651611)
\curveto(138.15458938,65.23650562)(138.16958937,65.27150558)(138.18959778,65.30151611)
\curveto(138.24958929,65.38150547)(138.3395892,65.42150543)(138.45959778,65.42151611)
\curveto(138.57958896,65.43150542)(138.70958883,65.43650542)(138.84959778,65.43651611)
\curveto(138.90958863,65.43650542)(138.96958857,65.43650542)(139.02959778,65.43651611)
\curveto(139.09958844,65.43650542)(139.15958838,65.42650543)(139.20959778,65.40651611)
\curveto(139.32958821,65.3565055)(139.39458814,65.26650559)(139.40459778,65.13651611)
\curveto(139.42458811,65.01650584)(139.4345881,64.87150598)(139.43459778,64.70151611)
\lineto(139.43459778,63.05151611)
\lineto(139.43459778,56.76651611)
\lineto(139.43459778,55.50651611)
\lineto(139.43459778,55.17651611)
\curveto(139.44458809,55.06651579)(139.42458811,54.98151587)(139.37459778,54.92151611)
\curveto(139.3345882,54.86151599)(139.28458825,54.82151603)(139.22459778,54.80151611)
\curveto(139.17458836,54.79151606)(139.10958843,54.77651608)(139.02959778,54.75651611)
\lineto(138.63959778,54.75651611)
\lineto(138.26459778,54.75651611)
\curveto(138.14458939,54.7565161)(138.04458949,54.77651608)(137.96459778,54.81651611)
\curveto(137.88458965,54.84651601)(137.81958972,54.89651596)(137.76959778,54.96651611)
\curveto(137.72958981,55.03651582)(137.68458985,55.10651575)(137.63459778,55.17651611)
\curveto(137.55458998,55.29651556)(137.46959007,55.42151543)(137.37959778,55.55151611)
\lineto(137.13959778,55.94151611)
\curveto(136.77959076,56.48151437)(136.42459111,57.01651384)(136.07459778,57.54651611)
\curveto(135.72459181,58.07651278)(135.37459216,58.61651224)(135.02459778,59.16651611)
\curveto(134.8345927,59.46651139)(134.6395929,59.76151109)(134.43959778,60.05151611)
\curveto(134.24959329,60.34151051)(134.05959348,60.63651022)(133.86959778,60.93651611)
\curveto(133.539594,61.4565094)(133.19459434,61.98150887)(132.83459778,62.51151611)
\curveto(132.79459474,62.58150827)(132.75459478,62.64650821)(132.71459778,62.70651611)
\curveto(132.67459486,62.77650808)(132.61959492,62.83650802)(132.54959778,62.88651611)
\curveto(132.52959501,62.89650796)(132.50959503,62.91150794)(132.48959778,62.93151611)
\curveto(132.46959507,62.9515079)(132.44459509,62.9565079)(132.41459778,62.94651611)
\curveto(132.35459518,62.92650793)(132.31959522,62.88650797)(132.30959778,62.82651611)
\curveto(132.29959524,62.76650809)(132.28459525,62.70650815)(132.26459778,62.64651611)
\lineto(132.26459778,62.54151611)
\curveto(132.24459529,62.47150838)(132.2395953,62.39150846)(132.24959778,62.30151611)
\curveto(132.25959528,62.22150863)(132.26459527,62.14150871)(132.26459778,62.06151611)
\lineto(132.26459778,61.07151611)
\lineto(132.26459778,56.30151611)
\lineto(132.26459778,55.59651611)
\lineto(132.26459778,55.41651611)
\curveto(132.27459526,55.34651551)(132.26959527,55.28651557)(132.24959778,55.23651611)
\lineto(132.24959778,55.11651611)
\curveto(132.22959531,55.01651584)(132.20959533,54.94651591)(132.18959778,54.90651611)
\curveto(132.16959537,54.856516)(132.1345954,54.82151603)(132.08459778,54.80151611)
\curveto(132.0345955,54.79151606)(131.97959556,54.77651608)(131.91959778,54.75651611)
\lineto(131.61959778,54.75651611)
\curveto(131.47959606,54.7565161)(131.35459618,54.76151609)(131.24459778,54.77151611)
\curveto(131.1345964,54.78151607)(131.05459648,54.82651603)(131.00459778,54.90651611)
\curveto(130.95459658,54.96651589)(130.92959661,55.04651581)(130.92959778,55.14651611)
\lineto(130.92959778,55.47651611)
\lineto(130.92959778,56.69151611)
\lineto(130.92959778,62.96151611)
\lineto(130.92959778,64.58151611)
\lineto(130.92959778,64.95651611)
\curveto(130.92959661,65.09650576)(130.95459658,65.20650565)(131.00459778,65.28651611)
\curveto(131.0345965,65.33650552)(131.09459644,65.38150547)(131.18459778,65.42151611)
\curveto(131.20459633,65.43150542)(131.22959631,65.43150542)(131.25959778,65.42151611)
\curveto(131.29959624,65.42150543)(131.32959621,65.42650543)(131.34959778,65.43651611)
}
}
{
\newrgbcolor{curcolor}{0 0 0}
\pscustom[linestyle=none,fillstyle=solid,fillcolor=curcolor]
{
\newpath
\moveto(148.61014465,58.95651611)
\curveto(148.63013659,58.89651196)(148.64013658,58.80151205)(148.64014465,58.67151611)
\curveto(148.64013658,58.5515123)(148.63513659,58.46651239)(148.62514465,58.41651611)
\lineto(148.62514465,58.26651611)
\curveto(148.61513661,58.18651267)(148.60513662,58.11151274)(148.59514465,58.04151611)
\curveto(148.59513663,57.98151287)(148.59013663,57.91151294)(148.58014465,57.83151611)
\curveto(148.56013666,57.77151308)(148.54513668,57.71151314)(148.53514465,57.65151611)
\curveto(148.53513669,57.59151326)(148.5251367,57.53151332)(148.50514465,57.47151611)
\curveto(148.46513676,57.34151351)(148.43013679,57.21151364)(148.40014465,57.08151611)
\curveto(148.37013685,56.9515139)(148.33013689,56.83151402)(148.28014465,56.72151611)
\curveto(148.07013715,56.24151461)(147.79013743,55.83651502)(147.44014465,55.50651611)
\curveto(147.09013813,55.18651567)(146.66013856,54.94151591)(146.15014465,54.77151611)
\curveto(146.04013918,54.73151612)(145.9201393,54.70151615)(145.79014465,54.68151611)
\curveto(145.67013955,54.66151619)(145.54513968,54.64151621)(145.41514465,54.62151611)
\curveto(145.35513987,54.61151624)(145.29013993,54.60651625)(145.22014465,54.60651611)
\curveto(145.16014006,54.59651626)(145.10014012,54.59151626)(145.04014465,54.59151611)
\curveto(145.00014022,54.58151627)(144.94014028,54.57651628)(144.86014465,54.57651611)
\curveto(144.79014043,54.57651628)(144.74014048,54.58151627)(144.71014465,54.59151611)
\curveto(144.67014055,54.60151625)(144.63014059,54.60651625)(144.59014465,54.60651611)
\curveto(144.55014067,54.59651626)(144.51514071,54.59651626)(144.48514465,54.60651611)
\lineto(144.39514465,54.60651611)
\lineto(144.03514465,54.65151611)
\curveto(143.89514133,54.69151616)(143.76014146,54.73151612)(143.63014465,54.77151611)
\curveto(143.50014172,54.81151604)(143.37514185,54.856516)(143.25514465,54.90651611)
\curveto(142.80514242,55.10651575)(142.43514279,55.36651549)(142.14514465,55.68651611)
\curveto(141.85514337,56.00651485)(141.61514361,56.39651446)(141.42514465,56.85651611)
\curveto(141.37514385,56.9565139)(141.33514389,57.0565138)(141.30514465,57.15651611)
\curveto(141.28514394,57.2565136)(141.26514396,57.36151349)(141.24514465,57.47151611)
\curveto(141.225144,57.51151334)(141.21514401,57.54151331)(141.21514465,57.56151611)
\curveto(141.225144,57.59151326)(141.225144,57.62651323)(141.21514465,57.66651611)
\curveto(141.19514403,57.74651311)(141.18014404,57.82651303)(141.17014465,57.90651611)
\curveto(141.17014405,57.99651286)(141.16014406,58.08151277)(141.14014465,58.16151611)
\lineto(141.14014465,58.28151611)
\curveto(141.14014408,58.32151253)(141.13514409,58.36651249)(141.12514465,58.41651611)
\curveto(141.11514411,58.46651239)(141.11014411,58.5515123)(141.11014465,58.67151611)
\curveto(141.11014411,58.80151205)(141.1201441,58.89651196)(141.14014465,58.95651611)
\curveto(141.16014406,59.02651183)(141.16514406,59.09651176)(141.15514465,59.16651611)
\curveto(141.14514408,59.23651162)(141.15014407,59.30651155)(141.17014465,59.37651611)
\curveto(141.18014404,59.42651143)(141.18514404,59.46651139)(141.18514465,59.49651611)
\curveto(141.19514403,59.53651132)(141.20514402,59.58151127)(141.21514465,59.63151611)
\curveto(141.24514398,59.7515111)(141.27014395,59.87151098)(141.29014465,59.99151611)
\curveto(141.3201439,60.11151074)(141.36014386,60.22651063)(141.41014465,60.33651611)
\curveto(141.56014366,60.70651015)(141.74014348,61.03650982)(141.95014465,61.32651611)
\curveto(142.17014305,61.62650923)(142.43514279,61.87650898)(142.74514465,62.07651611)
\curveto(142.86514236,62.1565087)(142.99014223,62.22150863)(143.12014465,62.27151611)
\curveto(143.25014197,62.33150852)(143.38514184,62.39150846)(143.52514465,62.45151611)
\curveto(143.64514158,62.50150835)(143.77514145,62.53150832)(143.91514465,62.54151611)
\curveto(144.05514117,62.56150829)(144.19514103,62.59150826)(144.33514465,62.63151611)
\lineto(144.53014465,62.63151611)
\curveto(144.60014062,62.64150821)(144.66514056,62.6515082)(144.72514465,62.66151611)
\curveto(145.61513961,62.67150818)(146.35513887,62.48650837)(146.94514465,62.10651611)
\curveto(147.53513769,61.72650913)(147.96013726,61.23150962)(148.22014465,60.62151611)
\curveto(148.27013695,60.52151033)(148.31013691,60.42151043)(148.34014465,60.32151611)
\curveto(148.37013685,60.22151063)(148.40513682,60.11651074)(148.44514465,60.00651611)
\curveto(148.47513675,59.89651096)(148.50013672,59.77651108)(148.52014465,59.64651611)
\curveto(148.54013668,59.52651133)(148.56513666,59.40151145)(148.59514465,59.27151611)
\curveto(148.60513662,59.22151163)(148.60513662,59.16651169)(148.59514465,59.10651611)
\curveto(148.59513663,59.0565118)(148.60013662,59.00651185)(148.61014465,58.95651611)
\moveto(147.27514465,58.10151611)
\curveto(147.29513793,58.17151268)(147.30013792,58.2515126)(147.29014465,58.34151611)
\lineto(147.29014465,58.59651611)
\curveto(147.29013793,58.98651187)(147.25513797,59.31651154)(147.18514465,59.58651611)
\curveto(147.15513807,59.66651119)(147.13013809,59.74651111)(147.11014465,59.82651611)
\curveto(147.09013813,59.90651095)(147.06513816,59.98151087)(147.03514465,60.05151611)
\curveto(146.75513847,60.70151015)(146.31013891,61.1515097)(145.70014465,61.40151611)
\curveto(145.63013959,61.43150942)(145.55513967,61.4515094)(145.47514465,61.46151611)
\lineto(145.23514465,61.52151611)
\curveto(145.15514007,61.54150931)(145.07014015,61.5515093)(144.98014465,61.55151611)
\lineto(144.71014465,61.55151611)
\lineto(144.44014465,61.50651611)
\curveto(144.34014088,61.48650937)(144.24514098,61.46150939)(144.15514465,61.43151611)
\curveto(144.07514115,61.41150944)(143.99514123,61.38150947)(143.91514465,61.34151611)
\curveto(143.84514138,61.32150953)(143.78014144,61.29150956)(143.72014465,61.25151611)
\curveto(143.66014156,61.21150964)(143.60514162,61.17150968)(143.55514465,61.13151611)
\curveto(143.31514191,60.96150989)(143.1201421,60.7565101)(142.97014465,60.51651611)
\curveto(142.8201424,60.27651058)(142.69014253,59.99651086)(142.58014465,59.67651611)
\curveto(142.55014267,59.57651128)(142.53014269,59.47151138)(142.52014465,59.36151611)
\curveto(142.51014271,59.26151159)(142.49514273,59.1565117)(142.47514465,59.04651611)
\curveto(142.46514276,59.00651185)(142.46014276,58.94151191)(142.46014465,58.85151611)
\curveto(142.45014277,58.82151203)(142.44514278,58.78651207)(142.44514465,58.74651611)
\curveto(142.45514277,58.70651215)(142.46014276,58.66151219)(142.46014465,58.61151611)
\lineto(142.46014465,58.31151611)
\curveto(142.46014276,58.21151264)(142.47014275,58.12151273)(142.49014465,58.04151611)
\lineto(142.52014465,57.86151611)
\curveto(142.54014268,57.76151309)(142.55514267,57.66151319)(142.56514465,57.56151611)
\curveto(142.58514264,57.47151338)(142.61514261,57.38651347)(142.65514465,57.30651611)
\curveto(142.75514247,57.06651379)(142.87014235,56.84151401)(143.00014465,56.63151611)
\curveto(143.14014208,56.42151443)(143.31014191,56.24651461)(143.51014465,56.10651611)
\curveto(143.56014166,56.07651478)(143.60514162,56.0515148)(143.64514465,56.03151611)
\curveto(143.68514154,56.01151484)(143.73014149,55.98651487)(143.78014465,55.95651611)
\curveto(143.86014136,55.90651495)(143.94514128,55.86151499)(144.03514465,55.82151611)
\curveto(144.13514109,55.79151506)(144.24014098,55.76151509)(144.35014465,55.73151611)
\curveto(144.40014082,55.71151514)(144.44514078,55.70151515)(144.48514465,55.70151611)
\curveto(144.53514069,55.71151514)(144.58514064,55.71151514)(144.63514465,55.70151611)
\curveto(144.66514056,55.69151516)(144.7251405,55.68151517)(144.81514465,55.67151611)
\curveto(144.91514031,55.66151519)(144.99014023,55.66651519)(145.04014465,55.68651611)
\curveto(145.08014014,55.69651516)(145.1201401,55.69651516)(145.16014465,55.68651611)
\curveto(145.20014002,55.68651517)(145.24013998,55.69651516)(145.28014465,55.71651611)
\curveto(145.36013986,55.73651512)(145.44013978,55.7515151)(145.52014465,55.76151611)
\curveto(145.60013962,55.78151507)(145.67513955,55.80651505)(145.74514465,55.83651611)
\curveto(146.08513914,55.97651488)(146.36013886,56.17151468)(146.57014465,56.42151611)
\curveto(146.78013844,56.67151418)(146.95513827,56.96651389)(147.09514465,57.30651611)
\curveto(147.14513808,57.42651343)(147.17513805,57.5515133)(147.18514465,57.68151611)
\curveto(147.20513802,57.82151303)(147.23513799,57.96151289)(147.27514465,58.10151611)
}
}
{
\newrgbcolor{curcolor}{0 0 0}
\pscustom[linestyle=none,fillstyle=solid,fillcolor=curcolor]
{
\newpath
\moveto(151.0434259,64.82151611)
\curveto(151.19342389,64.82150603)(151.34342374,64.81650604)(151.4934259,64.80651611)
\curveto(151.64342344,64.80650605)(151.74842334,64.76650609)(151.8084259,64.68651611)
\curveto(151.85842323,64.62650623)(151.8834232,64.54150631)(151.8834259,64.43151611)
\curveto(151.89342319,64.33150652)(151.89842319,64.22650663)(151.8984259,64.11651611)
\lineto(151.8984259,63.24651611)
\curveto(151.89842319,63.16650769)(151.89342319,63.08150777)(151.8834259,62.99151611)
\curveto(151.8834232,62.91150794)(151.89342319,62.84150801)(151.9134259,62.78151611)
\curveto(151.95342313,62.64150821)(152.04342304,62.5515083)(152.1834259,62.51151611)
\curveto(152.23342285,62.50150835)(152.27842281,62.49650836)(152.3184259,62.49651611)
\lineto(152.4684259,62.49651611)
\lineto(152.8734259,62.49651611)
\curveto(153.03342205,62.50650835)(153.14842194,62.49650836)(153.2184259,62.46651611)
\curveto(153.30842178,62.40650845)(153.36842172,62.34650851)(153.3984259,62.28651611)
\curveto(153.41842167,62.24650861)(153.42842166,62.20150865)(153.4284259,62.15151611)
\lineto(153.4284259,62.00151611)
\curveto(153.42842166,61.89150896)(153.42342166,61.78650907)(153.4134259,61.68651611)
\curveto(153.40342168,61.59650926)(153.36842172,61.52650933)(153.3084259,61.47651611)
\curveto(153.24842184,61.42650943)(153.16342192,61.39650946)(153.0534259,61.38651611)
\lineto(152.7234259,61.38651611)
\curveto(152.61342247,61.39650946)(152.50342258,61.40150945)(152.3934259,61.40151611)
\curveto(152.2834228,61.40150945)(152.1884229,61.38650947)(152.1084259,61.35651611)
\curveto(152.03842305,61.32650953)(151.9884231,61.27650958)(151.9584259,61.20651611)
\curveto(151.92842316,61.13650972)(151.90842318,61.0515098)(151.8984259,60.95151611)
\curveto(151.8884232,60.86150999)(151.8834232,60.76151009)(151.8834259,60.65151611)
\curveto(151.89342319,60.5515103)(151.89842319,60.4515104)(151.8984259,60.35151611)
\lineto(151.8984259,57.38151611)
\curveto(151.89842319,57.16151369)(151.89342319,56.92651393)(151.8834259,56.67651611)
\curveto(151.8834232,56.43651442)(151.92842316,56.2515146)(152.0184259,56.12151611)
\curveto(152.06842302,56.04151481)(152.13342295,55.98651487)(152.2134259,55.95651611)
\curveto(152.29342279,55.92651493)(152.3884227,55.90151495)(152.4984259,55.88151611)
\curveto(152.52842256,55.87151498)(152.55842253,55.86651499)(152.5884259,55.86651611)
\curveto(152.62842246,55.87651498)(152.66342242,55.87651498)(152.6934259,55.86651611)
\lineto(152.8884259,55.86651611)
\curveto(152.9884221,55.86651499)(153.07842201,55.856515)(153.1584259,55.83651611)
\curveto(153.24842184,55.82651503)(153.31342177,55.79151506)(153.3534259,55.73151611)
\curveto(153.37342171,55.70151515)(153.3884217,55.64651521)(153.3984259,55.56651611)
\curveto(153.41842167,55.49651536)(153.42842166,55.42151543)(153.4284259,55.34151611)
\curveto(153.43842165,55.26151559)(153.43842165,55.18151567)(153.4284259,55.10151611)
\curveto(153.41842167,55.03151582)(153.39842169,54.97651588)(153.3684259,54.93651611)
\curveto(153.32842176,54.86651599)(153.25342183,54.81651604)(153.1434259,54.78651611)
\curveto(153.06342202,54.76651609)(152.97342211,54.7565161)(152.8734259,54.75651611)
\curveto(152.77342231,54.76651609)(152.6834224,54.77151608)(152.6034259,54.77151611)
\curveto(152.54342254,54.77151608)(152.4834226,54.76651609)(152.4234259,54.75651611)
\curveto(152.36342272,54.7565161)(152.30842278,54.76151609)(152.2584259,54.77151611)
\lineto(152.0784259,54.77151611)
\curveto(152.02842306,54.78151607)(151.97842311,54.78651607)(151.9284259,54.78651611)
\curveto(151.8884232,54.79651606)(151.84342324,54.80151605)(151.7934259,54.80151611)
\curveto(151.59342349,54.851516)(151.41842367,54.90651595)(151.2684259,54.96651611)
\curveto(151.12842396,55.02651583)(151.00842408,55.13151572)(150.9084259,55.28151611)
\curveto(150.76842432,55.48151537)(150.6884244,55.73151512)(150.6684259,56.03151611)
\curveto(150.64842444,56.34151451)(150.63842445,56.67151418)(150.6384259,57.02151611)
\lineto(150.6384259,60.95151611)
\curveto(150.60842448,61.08150977)(150.57842451,61.17650968)(150.5484259,61.23651611)
\curveto(150.52842456,61.29650956)(150.45842463,61.34650951)(150.3384259,61.38651611)
\curveto(150.29842479,61.39650946)(150.25842483,61.39650946)(150.2184259,61.38651611)
\curveto(150.17842491,61.37650948)(150.13842495,61.38150947)(150.0984259,61.40151611)
\lineto(149.8584259,61.40151611)
\curveto(149.72842536,61.40150945)(149.61842547,61.41150944)(149.5284259,61.43151611)
\curveto(149.44842564,61.46150939)(149.39342569,61.52150933)(149.3634259,61.61151611)
\curveto(149.34342574,61.6515092)(149.32842576,61.69650916)(149.3184259,61.74651611)
\lineto(149.3184259,61.89651611)
\curveto(149.31842577,62.03650882)(149.32842576,62.1515087)(149.3484259,62.24151611)
\curveto(149.36842572,62.34150851)(149.42842566,62.41650844)(149.5284259,62.46651611)
\curveto(149.63842545,62.50650835)(149.77842531,62.51650834)(149.9484259,62.49651611)
\curveto(150.12842496,62.47650838)(150.27842481,62.48650837)(150.3984259,62.52651611)
\curveto(150.4884246,62.57650828)(150.55842453,62.64650821)(150.6084259,62.73651611)
\curveto(150.62842446,62.79650806)(150.63842445,62.87150798)(150.6384259,62.96151611)
\lineto(150.6384259,63.21651611)
\lineto(150.6384259,64.14651611)
\lineto(150.6384259,64.38651611)
\curveto(150.63842445,64.47650638)(150.64842444,64.5515063)(150.6684259,64.61151611)
\curveto(150.70842438,64.69150616)(150.7834243,64.7565061)(150.8934259,64.80651611)
\curveto(150.92342416,64.80650605)(150.94842414,64.80650605)(150.9684259,64.80651611)
\curveto(150.99842409,64.81650604)(151.02342406,64.82150603)(151.0434259,64.82151611)
}
}
{
\newrgbcolor{curcolor}{0 0 0}
\pscustom[linestyle=none,fillstyle=solid,fillcolor=curcolor]
{
\newpath
\moveto(161.70022278,55.31151611)
\curveto(161.73021495,55.1515157)(161.71521496,55.01651584)(161.65522278,54.90651611)
\curveto(161.59521508,54.80651605)(161.51521516,54.73151612)(161.41522278,54.68151611)
\curveto(161.36521531,54.66151619)(161.31021537,54.6515162)(161.25022278,54.65151611)
\curveto(161.20021548,54.6515162)(161.14521553,54.64151621)(161.08522278,54.62151611)
\curveto(160.86521581,54.57151628)(160.64521603,54.58651627)(160.42522278,54.66651611)
\curveto(160.21521646,54.73651612)(160.07021661,54.82651603)(159.99022278,54.93651611)
\curveto(159.94021674,55.00651585)(159.89521678,55.08651577)(159.85522278,55.17651611)
\curveto(159.81521686,55.27651558)(159.76521691,55.3565155)(159.70522278,55.41651611)
\curveto(159.68521699,55.43651542)(159.66021702,55.4565154)(159.63022278,55.47651611)
\curveto(159.61021707,55.49651536)(159.5802171,55.50151535)(159.54022278,55.49151611)
\curveto(159.43021725,55.46151539)(159.32521735,55.40651545)(159.22522278,55.32651611)
\curveto(159.13521754,55.24651561)(159.04521763,55.17651568)(158.95522278,55.11651611)
\curveto(158.82521785,55.03651582)(158.68521799,54.96151589)(158.53522278,54.89151611)
\curveto(158.38521829,54.83151602)(158.22521845,54.77651608)(158.05522278,54.72651611)
\curveto(157.95521872,54.69651616)(157.84521883,54.67651618)(157.72522278,54.66651611)
\curveto(157.61521906,54.6565162)(157.50521917,54.64151621)(157.39522278,54.62151611)
\curveto(157.34521933,54.61151624)(157.30021938,54.60651625)(157.26022278,54.60651611)
\lineto(157.15522278,54.60651611)
\curveto(157.04521963,54.58651627)(156.94021974,54.58651627)(156.84022278,54.60651611)
\lineto(156.70522278,54.60651611)
\curveto(156.65522002,54.61651624)(156.60522007,54.62151623)(156.55522278,54.62151611)
\curveto(156.50522017,54.62151623)(156.46022022,54.63151622)(156.42022278,54.65151611)
\curveto(156.3802203,54.66151619)(156.34522033,54.66651619)(156.31522278,54.66651611)
\curveto(156.29522038,54.6565162)(156.27022041,54.6565162)(156.24022278,54.66651611)
\lineto(156.00022278,54.72651611)
\curveto(155.92022076,54.73651612)(155.84522083,54.7565161)(155.77522278,54.78651611)
\curveto(155.4752212,54.91651594)(155.23022145,55.06151579)(155.04022278,55.22151611)
\curveto(154.86022182,55.39151546)(154.71022197,55.62651523)(154.59022278,55.92651611)
\curveto(154.50022218,56.14651471)(154.45522222,56.41151444)(154.45522278,56.72151611)
\lineto(154.45522278,57.03651611)
\curveto(154.46522221,57.08651377)(154.47022221,57.13651372)(154.47022278,57.18651611)
\lineto(154.50022278,57.36651611)
\lineto(154.62022278,57.69651611)
\curveto(154.66022202,57.80651305)(154.71022197,57.90651295)(154.77022278,57.99651611)
\curveto(154.95022173,58.28651257)(155.19522148,58.50151235)(155.50522278,58.64151611)
\curveto(155.81522086,58.78151207)(156.15522052,58.90651195)(156.52522278,59.01651611)
\curveto(156.66522001,59.0565118)(156.81021987,59.08651177)(156.96022278,59.10651611)
\curveto(157.11021957,59.12651173)(157.26021942,59.1515117)(157.41022278,59.18151611)
\curveto(157.4802192,59.20151165)(157.54521913,59.21151164)(157.60522278,59.21151611)
\curveto(157.675219,59.21151164)(157.75021893,59.22151163)(157.83022278,59.24151611)
\curveto(157.90021878,59.26151159)(157.97021871,59.27151158)(158.04022278,59.27151611)
\curveto(158.11021857,59.28151157)(158.18521849,59.29651156)(158.26522278,59.31651611)
\curveto(158.51521816,59.37651148)(158.75021793,59.42651143)(158.97022278,59.46651611)
\curveto(159.19021749,59.51651134)(159.36521731,59.63151122)(159.49522278,59.81151611)
\curveto(159.55521712,59.89151096)(159.60521707,59.99151086)(159.64522278,60.11151611)
\curveto(159.68521699,60.24151061)(159.68521699,60.38151047)(159.64522278,60.53151611)
\curveto(159.58521709,60.77151008)(159.49521718,60.96150989)(159.37522278,61.10151611)
\curveto(159.26521741,61.24150961)(159.10521757,61.3515095)(158.89522278,61.43151611)
\curveto(158.7752179,61.48150937)(158.63021805,61.51650934)(158.46022278,61.53651611)
\curveto(158.30021838,61.5565093)(158.13021855,61.56650929)(157.95022278,61.56651611)
\curveto(157.77021891,61.56650929)(157.59521908,61.5565093)(157.42522278,61.53651611)
\curveto(157.25521942,61.51650934)(157.11021957,61.48650937)(156.99022278,61.44651611)
\curveto(156.82021986,61.38650947)(156.65522002,61.30150955)(156.49522278,61.19151611)
\curveto(156.41522026,61.13150972)(156.34022034,61.0515098)(156.27022278,60.95151611)
\curveto(156.21022047,60.86150999)(156.15522052,60.76151009)(156.10522278,60.65151611)
\curveto(156.0752206,60.57151028)(156.04522063,60.48651037)(156.01522278,60.39651611)
\curveto(155.99522068,60.30651055)(155.95022073,60.23651062)(155.88022278,60.18651611)
\curveto(155.84022084,60.1565107)(155.77022091,60.13151072)(155.67022278,60.11151611)
\curveto(155.5802211,60.10151075)(155.48522119,60.09651076)(155.38522278,60.09651611)
\curveto(155.28522139,60.09651076)(155.18522149,60.10151075)(155.08522278,60.11151611)
\curveto(154.99522168,60.13151072)(154.93022175,60.1565107)(154.89022278,60.18651611)
\curveto(154.85022183,60.21651064)(154.82022186,60.26651059)(154.80022278,60.33651611)
\curveto(154.7802219,60.40651045)(154.7802219,60.48151037)(154.80022278,60.56151611)
\curveto(154.83022185,60.69151016)(154.86022182,60.81151004)(154.89022278,60.92151611)
\curveto(154.93022175,61.04150981)(154.9752217,61.1565097)(155.02522278,61.26651611)
\curveto(155.21522146,61.61650924)(155.45522122,61.88650897)(155.74522278,62.07651611)
\curveto(156.03522064,62.27650858)(156.39522028,62.43650842)(156.82522278,62.55651611)
\curveto(156.92521975,62.57650828)(157.02521965,62.59150826)(157.12522278,62.60151611)
\curveto(157.23521944,62.61150824)(157.34521933,62.62650823)(157.45522278,62.64651611)
\curveto(157.49521918,62.6565082)(157.56021912,62.6565082)(157.65022278,62.64651611)
\curveto(157.74021894,62.64650821)(157.79521888,62.6565082)(157.81522278,62.67651611)
\curveto(158.51521816,62.68650817)(159.12521755,62.60650825)(159.64522278,62.43651611)
\curveto(160.16521651,62.26650859)(160.53021615,61.94150891)(160.74022278,61.46151611)
\curveto(160.83021585,61.26150959)(160.8802158,61.02650983)(160.89022278,60.75651611)
\curveto(160.91021577,60.49651036)(160.92021576,60.22151063)(160.92022278,59.93151611)
\lineto(160.92022278,56.61651611)
\curveto(160.92021576,56.47651438)(160.92521575,56.34151451)(160.93522278,56.21151611)
\curveto(160.94521573,56.08151477)(160.9752157,55.97651488)(161.02522278,55.89651611)
\curveto(161.0752156,55.82651503)(161.14021554,55.77651508)(161.22022278,55.74651611)
\curveto(161.31021537,55.70651515)(161.39521528,55.67651518)(161.47522278,55.65651611)
\curveto(161.55521512,55.64651521)(161.61521506,55.60151525)(161.65522278,55.52151611)
\curveto(161.675215,55.49151536)(161.68521499,55.46151539)(161.68522278,55.43151611)
\curveto(161.68521499,55.40151545)(161.69021499,55.36151549)(161.70022278,55.31151611)
\moveto(159.55522278,56.97651611)
\curveto(159.61521706,57.11651374)(159.64521703,57.27651358)(159.64522278,57.45651611)
\curveto(159.65521702,57.64651321)(159.66021702,57.84151301)(159.66022278,58.04151611)
\curveto(159.66021702,58.1515127)(159.65521702,58.2515126)(159.64522278,58.34151611)
\curveto(159.63521704,58.43151242)(159.59521708,58.50151235)(159.52522278,58.55151611)
\curveto(159.49521718,58.57151228)(159.42521725,58.58151227)(159.31522278,58.58151611)
\curveto(159.29521738,58.56151229)(159.26021742,58.5515123)(159.21022278,58.55151611)
\curveto(159.16021752,58.5515123)(159.11521756,58.54151231)(159.07522278,58.52151611)
\curveto(158.99521768,58.50151235)(158.90521777,58.48151237)(158.80522278,58.46151611)
\lineto(158.50522278,58.40151611)
\curveto(158.4752182,58.40151245)(158.44021824,58.39651246)(158.40022278,58.38651611)
\lineto(158.29522278,58.38651611)
\curveto(158.14521853,58.34651251)(157.9802187,58.32151253)(157.80022278,58.31151611)
\curveto(157.63021905,58.31151254)(157.47021921,58.29151256)(157.32022278,58.25151611)
\curveto(157.24021944,58.23151262)(157.16521951,58.21151264)(157.09522278,58.19151611)
\curveto(157.03521964,58.18151267)(156.96521971,58.16651269)(156.88522278,58.14651611)
\curveto(156.72521995,58.09651276)(156.5752201,58.03151282)(156.43522278,57.95151611)
\curveto(156.29522038,57.88151297)(156.1752205,57.79151306)(156.07522278,57.68151611)
\curveto(155.9752207,57.57151328)(155.90022078,57.43651342)(155.85022278,57.27651611)
\curveto(155.80022088,57.12651373)(155.7802209,56.94151391)(155.79022278,56.72151611)
\curveto(155.79022089,56.62151423)(155.80522087,56.52651433)(155.83522278,56.43651611)
\curveto(155.8752208,56.3565145)(155.92022076,56.28151457)(155.97022278,56.21151611)
\curveto(156.05022063,56.10151475)(156.15522052,56.00651485)(156.28522278,55.92651611)
\curveto(156.41522026,55.856515)(156.55522012,55.79651506)(156.70522278,55.74651611)
\curveto(156.75521992,55.73651512)(156.80521987,55.73151512)(156.85522278,55.73151611)
\curveto(156.90521977,55.73151512)(156.95521972,55.72651513)(157.00522278,55.71651611)
\curveto(157.0752196,55.69651516)(157.16021952,55.68151517)(157.26022278,55.67151611)
\curveto(157.37021931,55.67151518)(157.46021922,55.68151517)(157.53022278,55.70151611)
\curveto(157.59021909,55.72151513)(157.65021903,55.72651513)(157.71022278,55.71651611)
\curveto(157.77021891,55.71651514)(157.83021885,55.72651513)(157.89022278,55.74651611)
\curveto(157.97021871,55.76651509)(158.04521863,55.78151507)(158.11522278,55.79151611)
\curveto(158.19521848,55.80151505)(158.27021841,55.82151503)(158.34022278,55.85151611)
\curveto(158.63021805,55.97151488)(158.8752178,56.11651474)(159.07522278,56.28651611)
\curveto(159.28521739,56.4565144)(159.44521723,56.68651417)(159.55522278,56.97651611)
}
}
{
\newrgbcolor{curcolor}{0 0 0}
\pscustom[linestyle=none,fillstyle=solid,fillcolor=curcolor]
{
\newpath
\moveto(165.3018634,62.66151611)
\curveto(166.02185934,62.67150818)(166.62685873,62.58650827)(167.1168634,62.40651611)
\curveto(167.60685775,62.23650862)(167.98685737,61.93150892)(168.2568634,61.49151611)
\curveto(168.32685703,61.38150947)(168.38185698,61.26650959)(168.4218634,61.14651611)
\curveto(168.4618569,61.03650982)(168.50185686,60.91150994)(168.5418634,60.77151611)
\curveto(168.5618568,60.70151015)(168.56685679,60.62651023)(168.5568634,60.54651611)
\curveto(168.54685681,60.47651038)(168.53185683,60.42151043)(168.5118634,60.38151611)
\curveto(168.49185687,60.36151049)(168.46685689,60.34151051)(168.4368634,60.32151611)
\curveto(168.40685695,60.31151054)(168.38185698,60.29651056)(168.3618634,60.27651611)
\curveto(168.31185705,60.2565106)(168.2618571,60.2515106)(168.2118634,60.26151611)
\curveto(168.1618572,60.27151058)(168.11185725,60.27151058)(168.0618634,60.26151611)
\curveto(167.98185738,60.24151061)(167.87685748,60.23651062)(167.7468634,60.24651611)
\curveto(167.61685774,60.26651059)(167.52685783,60.29151056)(167.4768634,60.32151611)
\curveto(167.39685796,60.37151048)(167.34185802,60.43651042)(167.3118634,60.51651611)
\curveto(167.29185807,60.60651025)(167.2568581,60.69151016)(167.2068634,60.77151611)
\curveto(167.11685824,60.93150992)(166.99185837,61.07650978)(166.8318634,61.20651611)
\curveto(166.72185864,61.28650957)(166.60185876,61.34650951)(166.4718634,61.38651611)
\curveto(166.34185902,61.42650943)(166.20185916,61.46650939)(166.0518634,61.50651611)
\curveto(166.00185936,61.52650933)(165.95185941,61.53150932)(165.9018634,61.52151611)
\curveto(165.85185951,61.52150933)(165.80185956,61.52650933)(165.7518634,61.53651611)
\curveto(165.69185967,61.5565093)(165.61685974,61.56650929)(165.5268634,61.56651611)
\curveto(165.43685992,61.56650929)(165.36186,61.5565093)(165.3018634,61.53651611)
\lineto(165.2118634,61.53651611)
\lineto(165.0618634,61.50651611)
\curveto(165.01186035,61.50650935)(164.9618604,61.50150935)(164.9118634,61.49151611)
\curveto(164.65186071,61.43150942)(164.43686092,61.34650951)(164.2668634,61.23651611)
\curveto(164.09686126,61.12650973)(163.98186138,60.94150991)(163.9218634,60.68151611)
\curveto(163.90186146,60.61151024)(163.89686146,60.54151031)(163.9068634,60.47151611)
\curveto(163.92686143,60.40151045)(163.94686141,60.34151051)(163.9668634,60.29151611)
\curveto(164.02686133,60.14151071)(164.09686126,60.03151082)(164.1768634,59.96151611)
\curveto(164.26686109,59.90151095)(164.37686098,59.83151102)(164.5068634,59.75151611)
\curveto(164.66686069,59.6515112)(164.84686051,59.57651128)(165.0468634,59.52651611)
\curveto(165.24686011,59.48651137)(165.44685991,59.43651142)(165.6468634,59.37651611)
\curveto(165.77685958,59.33651152)(165.90685945,59.30651155)(166.0368634,59.28651611)
\curveto(166.16685919,59.26651159)(166.29685906,59.23651162)(166.4268634,59.19651611)
\curveto(166.63685872,59.13651172)(166.84185852,59.07651178)(167.0418634,59.01651611)
\curveto(167.24185812,58.96651189)(167.44185792,58.90151195)(167.6418634,58.82151611)
\lineto(167.7918634,58.76151611)
\curveto(167.84185752,58.74151211)(167.89185747,58.71651214)(167.9418634,58.68651611)
\curveto(168.14185722,58.56651229)(168.31685704,58.43151242)(168.4668634,58.28151611)
\curveto(168.61685674,58.13151272)(168.74185662,57.94151291)(168.8418634,57.71151611)
\curveto(168.8618565,57.64151321)(168.88185648,57.54651331)(168.9018634,57.42651611)
\curveto(168.92185644,57.3565135)(168.93185643,57.28151357)(168.9318634,57.20151611)
\curveto(168.94185642,57.13151372)(168.94685641,57.0515138)(168.9468634,56.96151611)
\lineto(168.9468634,56.81151611)
\curveto(168.92685643,56.74151411)(168.91685644,56.67151418)(168.9168634,56.60151611)
\curveto(168.91685644,56.53151432)(168.90685645,56.46151439)(168.8868634,56.39151611)
\curveto(168.8568565,56.28151457)(168.82185654,56.17651468)(168.7818634,56.07651611)
\curveto(168.74185662,55.97651488)(168.69685666,55.88651497)(168.6468634,55.80651611)
\curveto(168.48685687,55.54651531)(168.28185708,55.33651552)(168.0318634,55.17651611)
\curveto(167.78185758,55.02651583)(167.50185786,54.89651596)(167.1918634,54.78651611)
\curveto(167.10185826,54.7565161)(167.00685835,54.73651612)(166.9068634,54.72651611)
\curveto(166.81685854,54.70651615)(166.72685863,54.68151617)(166.6368634,54.65151611)
\curveto(166.53685882,54.63151622)(166.43685892,54.62151623)(166.3368634,54.62151611)
\curveto(166.23685912,54.62151623)(166.13685922,54.61151624)(166.0368634,54.59151611)
\lineto(165.8868634,54.59151611)
\curveto(165.83685952,54.58151627)(165.76685959,54.57651628)(165.6768634,54.57651611)
\curveto(165.58685977,54.57651628)(165.51685984,54.58151627)(165.4668634,54.59151611)
\lineto(165.3018634,54.59151611)
\curveto(165.24186012,54.61151624)(165.17686018,54.62151623)(165.1068634,54.62151611)
\curveto(165.03686032,54.61151624)(164.97686038,54.61651624)(164.9268634,54.63651611)
\curveto(164.87686048,54.64651621)(164.81186055,54.6515162)(164.7318634,54.65151611)
\lineto(164.4918634,54.71151611)
\curveto(164.42186094,54.72151613)(164.34686101,54.74151611)(164.2668634,54.77151611)
\curveto(163.9568614,54.87151598)(163.68686167,54.99651586)(163.4568634,55.14651611)
\curveto(163.22686213,55.29651556)(163.02686233,55.49151536)(162.8568634,55.73151611)
\curveto(162.76686259,55.86151499)(162.69186267,55.99651486)(162.6318634,56.13651611)
\curveto(162.57186279,56.27651458)(162.51686284,56.43151442)(162.4668634,56.60151611)
\curveto(162.44686291,56.66151419)(162.43686292,56.73151412)(162.4368634,56.81151611)
\curveto(162.44686291,56.90151395)(162.4618629,56.97151388)(162.4818634,57.02151611)
\curveto(162.51186285,57.06151379)(162.5618628,57.10151375)(162.6318634,57.14151611)
\curveto(162.68186268,57.16151369)(162.75186261,57.17151368)(162.8418634,57.17151611)
\curveto(162.93186243,57.18151367)(163.02186234,57.18151367)(163.1118634,57.17151611)
\curveto(163.20186216,57.16151369)(163.28686207,57.14651371)(163.3668634,57.12651611)
\curveto(163.4568619,57.11651374)(163.51686184,57.10151375)(163.5468634,57.08151611)
\curveto(163.61686174,57.03151382)(163.6618617,56.9565139)(163.6818634,56.85651611)
\curveto(163.71186165,56.76651409)(163.74686161,56.68151417)(163.7868634,56.60151611)
\curveto(163.88686147,56.38151447)(164.02186134,56.21151464)(164.1918634,56.09151611)
\curveto(164.31186105,56.00151485)(164.44686091,55.93151492)(164.5968634,55.88151611)
\curveto(164.74686061,55.83151502)(164.90686045,55.78151507)(165.0768634,55.73151611)
\lineto(165.3918634,55.68651611)
\lineto(165.4818634,55.68651611)
\curveto(165.55185981,55.66651519)(165.64185972,55.6565152)(165.7518634,55.65651611)
\curveto(165.87185949,55.6565152)(165.97185939,55.66651519)(166.0518634,55.68651611)
\curveto(166.12185924,55.68651517)(166.17685918,55.69151516)(166.2168634,55.70151611)
\curveto(166.27685908,55.71151514)(166.33685902,55.71651514)(166.3968634,55.71651611)
\curveto(166.4568589,55.72651513)(166.51185885,55.73651512)(166.5618634,55.74651611)
\curveto(166.85185851,55.82651503)(167.08185828,55.93151492)(167.2518634,56.06151611)
\curveto(167.42185794,56.19151466)(167.54185782,56.41151444)(167.6118634,56.72151611)
\curveto(167.63185773,56.77151408)(167.63685772,56.82651403)(167.6268634,56.88651611)
\curveto(167.61685774,56.94651391)(167.60685775,56.99151386)(167.5968634,57.02151611)
\curveto(167.54685781,57.21151364)(167.47685788,57.3515135)(167.3868634,57.44151611)
\curveto(167.29685806,57.54151331)(167.18185818,57.63151322)(167.0418634,57.71151611)
\curveto(166.95185841,57.77151308)(166.85185851,57.82151303)(166.7418634,57.86151611)
\lineto(166.4118634,57.98151611)
\curveto(166.38185898,57.99151286)(166.35185901,57.99651286)(166.3218634,57.99651611)
\curveto(166.30185906,57.99651286)(166.27685908,58.00651285)(166.2468634,58.02651611)
\curveto(165.90685945,58.13651272)(165.55185981,58.21651264)(165.1818634,58.26651611)
\curveto(164.82186054,58.32651253)(164.48186088,58.42151243)(164.1618634,58.55151611)
\curveto(164.0618613,58.59151226)(163.96686139,58.62651223)(163.8768634,58.65651611)
\curveto(163.78686157,58.68651217)(163.70186166,58.72651213)(163.6218634,58.77651611)
\curveto(163.43186193,58.88651197)(163.2568621,59.01151184)(163.0968634,59.15151611)
\curveto(162.93686242,59.29151156)(162.81186255,59.46651139)(162.7218634,59.67651611)
\curveto(162.69186267,59.74651111)(162.66686269,59.81651104)(162.6468634,59.88651611)
\curveto(162.63686272,59.9565109)(162.62186274,60.03151082)(162.6018634,60.11151611)
\curveto(162.57186279,60.23151062)(162.5618628,60.36651049)(162.5718634,60.51651611)
\curveto(162.58186278,60.67651018)(162.59686276,60.81151004)(162.6168634,60.92151611)
\curveto(162.63686272,60.97150988)(162.64686271,61.01150984)(162.6468634,61.04151611)
\curveto(162.6568627,61.08150977)(162.67186269,61.12150973)(162.6918634,61.16151611)
\curveto(162.78186258,61.39150946)(162.90186246,61.59150926)(163.0518634,61.76151611)
\curveto(163.21186215,61.93150892)(163.39186197,62.08150877)(163.5918634,62.21151611)
\curveto(163.74186162,62.30150855)(163.90686145,62.37150848)(164.0868634,62.42151611)
\curveto(164.26686109,62.48150837)(164.4568609,62.53650832)(164.6568634,62.58651611)
\curveto(164.72686063,62.59650826)(164.79186057,62.60650825)(164.8518634,62.61651611)
\curveto(164.92186044,62.62650823)(164.99686036,62.63650822)(165.0768634,62.64651611)
\curveto(165.10686025,62.6565082)(165.14686021,62.6565082)(165.1968634,62.64651611)
\curveto(165.24686011,62.63650822)(165.28186008,62.64150821)(165.3018634,62.66151611)
}
}
{
\newrgbcolor{curcolor}{0 0 0}
\pscustom[linestyle=none,fillstyle=solid,fillcolor=curcolor]
{
\newpath
\moveto(337.40385742,65.42900879)
\lineto(342.30885742,65.42900879)
\lineto(343.59885742,65.42900879)
\curveto(343.70884954,65.42899809)(343.81884943,65.42899809)(343.92885742,65.42900879)
\curveto(344.03884921,65.43899808)(344.12884912,65.4189981)(344.19885742,65.36900879)
\curveto(344.22884902,65.34899817)(344.253849,65.3239982)(344.27385742,65.29400879)
\curveto(344.29384896,65.26399826)(344.31384894,65.23399829)(344.33385742,65.20400879)
\curveto(344.3538489,65.13399839)(344.36384889,65.0189985)(344.36385742,64.85900879)
\curveto(344.36384889,64.70899881)(344.3538489,64.59399893)(344.33385742,64.51400879)
\curveto(344.29384896,64.37399915)(344.20884904,64.29399923)(344.07885742,64.27400879)
\curveto(343.9488493,64.26399926)(343.79384946,64.25899926)(343.61385742,64.25900879)
\lineto(342.11385742,64.25900879)
\lineto(339.59385742,64.25900879)
\lineto(339.02385742,64.25900879)
\curveto(338.81385444,64.26899925)(338.65885459,64.24399928)(338.55885742,64.18400879)
\curveto(338.45885479,64.1239994)(338.40385485,64.0189995)(338.39385742,63.86900879)
\lineto(338.39385742,63.40400879)
\lineto(338.39385742,61.87400879)
\curveto(338.39385486,61.76400176)(338.38885486,61.63400189)(338.37885742,61.48400879)
\curveto(338.37885487,61.33400219)(338.38885486,61.21400231)(338.40885742,61.12400879)
\curveto(338.43885481,61.00400252)(338.49885475,60.9240026)(338.58885742,60.88400879)
\curveto(338.62885462,60.86400266)(338.69885455,60.84400268)(338.79885742,60.82400879)
\lineto(338.94885742,60.82400879)
\curveto(338.98885426,60.81400271)(339.02885422,60.80900271)(339.06885742,60.80900879)
\curveto(339.11885413,60.8190027)(339.16885408,60.8240027)(339.21885742,60.82400879)
\lineto(339.72885742,60.82400879)
\lineto(342.66885742,60.82400879)
\lineto(342.96885742,60.82400879)
\curveto(343.07885017,60.83400269)(343.18885006,60.83400269)(343.29885742,60.82400879)
\curveto(343.41884983,60.8240027)(343.52384973,60.81400271)(343.61385742,60.79400879)
\curveto(343.71384954,60.78400274)(343.78884946,60.76400276)(343.83885742,60.73400879)
\curveto(343.86884938,60.71400281)(343.89384936,60.66900285)(343.91385742,60.59900879)
\curveto(343.93384932,60.52900299)(343.9488493,60.45400307)(343.95885742,60.37400879)
\curveto(343.96884928,60.29400323)(343.96884928,60.20900331)(343.95885742,60.11900879)
\curveto(343.95884929,60.03900348)(343.9488493,59.96900355)(343.92885742,59.90900879)
\curveto(343.90884934,59.8190037)(343.86384939,59.75400377)(343.79385742,59.71400879)
\curveto(343.77384948,59.69400383)(343.74384951,59.67900384)(343.70385742,59.66900879)
\curveto(343.67384958,59.66900385)(343.64384961,59.66400386)(343.61385742,59.65400879)
\lineto(343.52385742,59.65400879)
\curveto(343.47384978,59.64400388)(343.42384983,59.63900388)(343.37385742,59.63900879)
\curveto(343.32384993,59.64900387)(343.27384998,59.65400387)(343.22385742,59.65400879)
\lineto(342.66885742,59.65400879)
\lineto(339.50385742,59.65400879)
\lineto(339.14385742,59.65400879)
\curveto(339.03385422,59.66400386)(338.92885432,59.65900386)(338.82885742,59.63900879)
\curveto(338.72885452,59.62900389)(338.63885461,59.60400392)(338.55885742,59.56400879)
\curveto(338.48885476,59.524004)(338.43885481,59.45400407)(338.40885742,59.35400879)
\curveto(338.38885486,59.29400423)(338.37885487,59.2240043)(338.37885742,59.14400879)
\curveto(338.38885486,59.06400446)(338.39385486,58.98400454)(338.39385742,58.90400879)
\lineto(338.39385742,58.06400879)
\lineto(338.39385742,56.63900879)
\curveto(338.39385486,56.49900702)(338.39885485,56.36900715)(338.40885742,56.24900879)
\curveto(338.41885483,56.13900738)(338.45885479,56.05900746)(338.52885742,56.00900879)
\curveto(338.59885465,55.95900756)(338.67885457,55.92900759)(338.76885742,55.91900879)
\lineto(339.06885742,55.91900879)
\lineto(340.02885742,55.91900879)
\lineto(342.80385742,55.91900879)
\lineto(343.65885742,55.91900879)
\lineto(343.89885742,55.91900879)
\curveto(343.97884927,55.92900759)(344.0488492,55.9240076)(344.10885742,55.90400879)
\curveto(344.22884902,55.86400766)(344.30884894,55.80900771)(344.34885742,55.73900879)
\curveto(344.36884888,55.70900781)(344.38384887,55.65900786)(344.39385742,55.58900879)
\curveto(344.40384885,55.519008)(344.40884884,55.44400808)(344.40885742,55.36400879)
\curveto(344.41884883,55.29400823)(344.41884883,55.2190083)(344.40885742,55.13900879)
\curveto(344.39884885,55.06900845)(344.38884886,55.01400851)(344.37885742,54.97400879)
\curveto(344.33884891,54.89400863)(344.29384896,54.83900868)(344.24385742,54.80900879)
\curveto(344.18384907,54.76900875)(344.10384915,54.74900877)(344.00385742,54.74900879)
\lineto(343.73385742,54.74900879)
\lineto(342.68385742,54.74900879)
\lineto(338.69385742,54.74900879)
\lineto(337.64385742,54.74900879)
\curveto(337.50385575,54.74900877)(337.38385587,54.75400877)(337.28385742,54.76400879)
\curveto(337.18385607,54.78400874)(337.10885614,54.83400869)(337.05885742,54.91400879)
\curveto(337.01885623,54.97400855)(336.99885625,55.04900847)(336.99885742,55.13900879)
\lineto(336.99885742,55.42400879)
\lineto(336.99885742,56.47400879)
\lineto(336.99885742,60.49400879)
\lineto(336.99885742,63.85400879)
\lineto(336.99885742,64.78400879)
\lineto(336.99885742,65.05400879)
\curveto(336.99885625,65.14399838)(337.01885623,65.21399831)(337.05885742,65.26400879)
\curveto(337.09885615,65.33399819)(337.17385608,65.38399814)(337.28385742,65.41400879)
\curveto(337.30385595,65.4239981)(337.32385593,65.4239981)(337.34385742,65.41400879)
\curveto(337.36385589,65.41399811)(337.38385587,65.4189981)(337.40385742,65.42900879)
}
}
{
\newrgbcolor{curcolor}{0 0 0}
\pscustom[linestyle=none,fillstyle=solid,fillcolor=curcolor]
{
\newpath
\moveto(345.6887793,62.47400879)
\lineto(346.1687793,62.47400879)
\curveto(346.33877796,62.47400105)(346.46877783,62.44400108)(346.5587793,62.38400879)
\curveto(346.62877767,62.33400119)(346.67377762,62.26900125)(346.6937793,62.18900879)
\curveto(346.72377757,62.1190014)(346.75377754,62.04400148)(346.7837793,61.96400879)
\curveto(346.84377745,61.8240017)(346.8937774,61.68400184)(346.9337793,61.54400879)
\curveto(346.97377732,61.40400212)(347.01877728,61.26400226)(347.0687793,61.12400879)
\curveto(347.26877703,60.58400294)(347.45377684,60.03900348)(347.6237793,59.48900879)
\curveto(347.7937765,58.94900457)(347.97877632,58.40900511)(348.1787793,57.86900879)
\curveto(348.24877605,57.68900583)(348.30877599,57.50400602)(348.3587793,57.31400879)
\curveto(348.40877589,57.13400639)(348.47377582,56.95400657)(348.5537793,56.77400879)
\curveto(348.57377572,56.70400682)(348.5987757,56.62900689)(348.6287793,56.54900879)
\curveto(348.65877564,56.46900705)(348.70877559,56.4190071)(348.7787793,56.39900879)
\curveto(348.85877544,56.37900714)(348.91877538,56.41400711)(348.9587793,56.50400879)
\curveto(349.00877529,56.59400693)(349.04377525,56.66400686)(349.0637793,56.71400879)
\curveto(349.14377515,56.90400662)(349.20877509,57.09400643)(349.2587793,57.28400879)
\curveto(349.31877498,57.48400604)(349.38377491,57.68400584)(349.4537793,57.88400879)
\curveto(349.58377471,58.26400526)(349.70877459,58.63900488)(349.8287793,59.00900879)
\curveto(349.94877435,59.38900413)(350.07377422,59.76900375)(350.2037793,60.14900879)
\curveto(350.25377404,60.3190032)(350.30377399,60.48400304)(350.3537793,60.64400879)
\curveto(350.40377389,60.81400271)(350.46377383,60.97900254)(350.5337793,61.13900879)
\curveto(350.58377371,61.27900224)(350.62877367,61.4190021)(350.6687793,61.55900879)
\curveto(350.70877359,61.69900182)(350.75377354,61.83900168)(350.8037793,61.97900879)
\curveto(350.82377347,62.04900147)(350.84877345,62.1190014)(350.8787793,62.18900879)
\curveto(350.90877339,62.25900126)(350.94877335,62.3190012)(350.9987793,62.36900879)
\curveto(351.07877322,62.4190011)(351.16877313,62.44900107)(351.2687793,62.45900879)
\curveto(351.36877293,62.46900105)(351.48877281,62.47400105)(351.6287793,62.47400879)
\curveto(351.6987726,62.47400105)(351.76377253,62.46900105)(351.8237793,62.45900879)
\curveto(351.88377241,62.45900106)(351.93877236,62.44900107)(351.9887793,62.42900879)
\curveto(352.07877222,62.38900113)(352.12377217,62.3240012)(352.1237793,62.23400879)
\curveto(352.13377216,62.14400138)(352.11877218,62.05400147)(352.0787793,61.96400879)
\curveto(352.01877228,61.79400173)(351.95877234,61.6190019)(351.8987793,61.43900879)
\curveto(351.83877246,61.25900226)(351.76877253,61.08400244)(351.6887793,60.91400879)
\curveto(351.66877263,60.86400266)(351.65377264,60.81400271)(351.6437793,60.76400879)
\curveto(351.63377266,60.7240028)(351.61877268,60.67900284)(351.5987793,60.62900879)
\curveto(351.51877278,60.45900306)(351.45377284,60.28400324)(351.4037793,60.10400879)
\curveto(351.35377294,59.9240036)(351.28877301,59.74400378)(351.2087793,59.56400879)
\curveto(351.15877314,59.43400409)(351.10877319,59.29900422)(351.0587793,59.15900879)
\curveto(351.01877328,59.02900449)(350.96877333,58.89900462)(350.9087793,58.76900879)
\curveto(350.73877356,58.35900516)(350.58377371,57.94400558)(350.4437793,57.52400879)
\curveto(350.31377398,57.10400642)(350.16377413,56.68900683)(349.9937793,56.27900879)
\curveto(349.93377436,56.1190074)(349.87877442,55.95900756)(349.8287793,55.79900879)
\curveto(349.77877452,55.63900788)(349.71877458,55.47900804)(349.6487793,55.31900879)
\curveto(349.5987747,55.20900831)(349.55377474,55.10400842)(349.5137793,55.00400879)
\curveto(349.48377481,54.91400861)(349.41377488,54.84400868)(349.3037793,54.79400879)
\curveto(349.24377505,54.76400876)(349.17377512,54.74900877)(349.0937793,54.74900879)
\lineto(348.8687793,54.74900879)
\lineto(348.4037793,54.74900879)
\curveto(348.25377604,54.75900876)(348.14377615,54.80900871)(348.0737793,54.89900879)
\curveto(348.00377629,54.97900854)(347.95377634,55.07400845)(347.9237793,55.18400879)
\curveto(347.8937764,55.30400822)(347.85377644,55.4190081)(347.8037793,55.52900879)
\curveto(347.74377655,55.66900785)(347.68377661,55.81400771)(347.6237793,55.96400879)
\curveto(347.57377672,56.1240074)(347.52377677,56.27400725)(347.4737793,56.41400879)
\curveto(347.45377684,56.46400706)(347.43877686,56.50400702)(347.4287793,56.53400879)
\curveto(347.41877688,56.57400695)(347.40377689,56.6190069)(347.3837793,56.66900879)
\curveto(347.18377711,57.14900637)(346.9987773,57.63400589)(346.8287793,58.12400879)
\curveto(346.66877763,58.61400491)(346.48877781,59.09900442)(346.2887793,59.57900879)
\curveto(346.22877807,59.73900378)(346.16877813,59.89400363)(346.1087793,60.04400879)
\curveto(346.05877824,60.20400332)(346.00377829,60.36400316)(345.9437793,60.52400879)
\lineto(345.8837793,60.67400879)
\curveto(345.87377842,60.73400279)(345.85877844,60.78900273)(345.8387793,60.83900879)
\curveto(345.75877854,61.00900251)(345.68877861,61.17900234)(345.6287793,61.34900879)
\curveto(345.57877872,61.519002)(345.51877878,61.68900183)(345.4487793,61.85900879)
\curveto(345.42877887,61.9190016)(345.40377889,61.99900152)(345.3737793,62.09900879)
\curveto(345.34377895,62.19900132)(345.34877895,62.28400124)(345.3887793,62.35400879)
\curveto(345.43877886,62.40400112)(345.4987788,62.43900108)(345.5687793,62.45900879)
\curveto(345.63877866,62.45900106)(345.67877862,62.46400106)(345.6887793,62.47400879)
}
}
{
\newrgbcolor{curcolor}{0 0 0}
\pscustom[linestyle=none,fillstyle=solid,fillcolor=curcolor]
{
\newpath
\moveto(360.1637793,58.91900879)
\curveto(360.18377161,58.8190047)(360.18377161,58.70400482)(360.1637793,58.57400879)
\curveto(360.15377164,58.45400507)(360.12377167,58.36900515)(360.0737793,58.31900879)
\curveto(360.02377177,58.27900524)(359.94877185,58.24900527)(359.8487793,58.22900879)
\curveto(359.75877204,58.2190053)(359.65377214,58.21400531)(359.5337793,58.21400879)
\lineto(359.1737793,58.21400879)
\curveto(359.05377274,58.2240053)(358.94877285,58.22900529)(358.8587793,58.22900879)
\lineto(355.0187793,58.22900879)
\curveto(354.93877686,58.22900529)(354.85877694,58.2240053)(354.7787793,58.21400879)
\curveto(354.6987771,58.21400531)(354.63377716,58.19900532)(354.5837793,58.16900879)
\curveto(354.54377725,58.14900537)(354.50377729,58.10900541)(354.4637793,58.04900879)
\curveto(354.44377735,58.0190055)(354.42377737,57.97400555)(354.4037793,57.91400879)
\curveto(354.38377741,57.86400566)(354.38377741,57.81400571)(354.4037793,57.76400879)
\curveto(354.41377738,57.71400581)(354.41877738,57.66900585)(354.4187793,57.62900879)
\curveto(354.41877738,57.58900593)(354.42377737,57.54900597)(354.4337793,57.50900879)
\curveto(354.45377734,57.42900609)(354.47377732,57.34400618)(354.4937793,57.25400879)
\curveto(354.51377728,57.17400635)(354.54377725,57.09400643)(354.5837793,57.01400879)
\curveto(354.81377698,56.47400705)(355.1937766,56.08900743)(355.7237793,55.85900879)
\curveto(355.78377601,55.82900769)(355.84877595,55.80400772)(355.9187793,55.78400879)
\lineto(356.1287793,55.72400879)
\curveto(356.15877564,55.71400781)(356.20877559,55.70900781)(356.2787793,55.70900879)
\curveto(356.41877538,55.66900785)(356.60377519,55.64900787)(356.8337793,55.64900879)
\curveto(357.06377473,55.64900787)(357.24877455,55.66900785)(357.3887793,55.70900879)
\curveto(357.52877427,55.74900777)(357.65377414,55.78900773)(357.7637793,55.82900879)
\curveto(357.88377391,55.87900764)(357.9937738,55.93900758)(358.0937793,56.00900879)
\curveto(358.20377359,56.07900744)(358.2987735,56.15900736)(358.3787793,56.24900879)
\curveto(358.45877334,56.34900717)(358.52877327,56.45400707)(358.5887793,56.56400879)
\curveto(358.64877315,56.66400686)(358.6987731,56.76900675)(358.7387793,56.87900879)
\curveto(358.78877301,56.98900653)(358.86877293,57.06900645)(358.9787793,57.11900879)
\curveto(359.01877278,57.13900638)(359.08377271,57.15400637)(359.1737793,57.16400879)
\curveto(359.26377253,57.17400635)(359.35377244,57.17400635)(359.4437793,57.16400879)
\curveto(359.53377226,57.16400636)(359.61877218,57.15900636)(359.6987793,57.14900879)
\curveto(359.77877202,57.13900638)(359.83377196,57.1190064)(359.8637793,57.08900879)
\curveto(359.96377183,57.0190065)(359.98877181,56.90400662)(359.9387793,56.74400879)
\curveto(359.85877194,56.47400705)(359.75377204,56.23400729)(359.6237793,56.02400879)
\curveto(359.42377237,55.70400782)(359.1937726,55.43900808)(358.9337793,55.22900879)
\curveto(358.68377311,55.02900849)(358.36377343,54.86400866)(357.9737793,54.73400879)
\curveto(357.87377392,54.69400883)(357.77377402,54.66900885)(357.6737793,54.65900879)
\curveto(357.57377422,54.63900888)(357.46877433,54.6190089)(357.3587793,54.59900879)
\curveto(357.30877449,54.58900893)(357.25877454,54.58400894)(357.2087793,54.58400879)
\curveto(357.16877463,54.58400894)(357.12377467,54.57900894)(357.0737793,54.56900879)
\lineto(356.9237793,54.56900879)
\curveto(356.87377492,54.55900896)(356.81377498,54.55400897)(356.7437793,54.55400879)
\curveto(356.68377511,54.55400897)(356.63377516,54.55900896)(356.5937793,54.56900879)
\lineto(356.4587793,54.56900879)
\curveto(356.40877539,54.57900894)(356.36377543,54.58400894)(356.3237793,54.58400879)
\curveto(356.28377551,54.58400894)(356.24377555,54.58900893)(356.2037793,54.59900879)
\curveto(356.15377564,54.60900891)(356.0987757,54.6190089)(356.0387793,54.62900879)
\curveto(355.97877582,54.62900889)(355.92377587,54.63400889)(355.8737793,54.64400879)
\curveto(355.78377601,54.66400886)(355.6937761,54.68900883)(355.6037793,54.71900879)
\curveto(355.51377628,54.73900878)(355.42877637,54.76400876)(355.3487793,54.79400879)
\curveto(355.30877649,54.81400871)(355.27377652,54.8240087)(355.2437793,54.82400879)
\curveto(355.21377658,54.83400869)(355.17877662,54.84900867)(355.1387793,54.86900879)
\curveto(354.98877681,54.93900858)(354.82877697,55.0240085)(354.6587793,55.12400879)
\curveto(354.36877743,55.31400821)(354.11877768,55.54400798)(353.9087793,55.81400879)
\curveto(353.70877809,56.09400743)(353.53877826,56.40400712)(353.3987793,56.74400879)
\curveto(353.34877845,56.85400667)(353.30877849,56.96900655)(353.2787793,57.08900879)
\curveto(353.25877854,57.20900631)(353.22877857,57.32900619)(353.1887793,57.44900879)
\curveto(353.17877862,57.48900603)(353.17377862,57.524006)(353.1737793,57.55400879)
\curveto(353.17377862,57.58400594)(353.16877863,57.6240059)(353.1587793,57.67400879)
\curveto(353.13877866,57.75400577)(353.12377867,57.83900568)(353.1137793,57.92900879)
\curveto(353.10377869,58.0190055)(353.08877871,58.10900541)(353.0687793,58.19900879)
\lineto(353.0687793,58.40900879)
\curveto(353.05877874,58.44900507)(353.04877875,58.50400502)(353.0387793,58.57400879)
\curveto(353.03877876,58.65400487)(353.04377875,58.7190048)(353.0537793,58.76900879)
\lineto(353.0537793,58.93400879)
\curveto(353.07377872,58.98400454)(353.07877872,59.03400449)(353.0687793,59.08400879)
\curveto(353.06877873,59.14400438)(353.07377872,59.19900432)(353.0837793,59.24900879)
\curveto(353.12377867,59.40900411)(353.15377864,59.56900395)(353.1737793,59.72900879)
\curveto(353.20377859,59.88900363)(353.24877855,60.03900348)(353.3087793,60.17900879)
\curveto(353.35877844,60.28900323)(353.40377839,60.39900312)(353.4437793,60.50900879)
\curveto(353.4937783,60.62900289)(353.54877825,60.74400278)(353.6087793,60.85400879)
\curveto(353.82877797,61.20400232)(354.07877772,61.50400202)(354.3587793,61.75400879)
\curveto(354.63877716,62.01400151)(354.98377681,62.22900129)(355.3937793,62.39900879)
\curveto(355.51377628,62.44900107)(355.63377616,62.48400104)(355.7537793,62.50400879)
\curveto(355.88377591,62.53400099)(356.01877578,62.56400096)(356.1587793,62.59400879)
\curveto(356.20877559,62.60400092)(356.25377554,62.60900091)(356.2937793,62.60900879)
\curveto(356.33377546,62.6190009)(356.37877542,62.6240009)(356.4287793,62.62400879)
\curveto(356.44877535,62.63400089)(356.47377532,62.63400089)(356.5037793,62.62400879)
\curveto(356.53377526,62.61400091)(356.55877524,62.6190009)(356.5787793,62.63900879)
\curveto(356.9987748,62.64900087)(357.36377443,62.60400092)(357.6737793,62.50400879)
\curveto(357.98377381,62.41400111)(358.26377353,62.28900123)(358.5137793,62.12900879)
\curveto(358.56377323,62.10900141)(358.60377319,62.07900144)(358.6337793,62.03900879)
\curveto(358.66377313,62.00900151)(358.6987731,61.98400154)(358.7387793,61.96400879)
\curveto(358.81877298,61.90400162)(358.8987729,61.83400169)(358.9787793,61.75400879)
\curveto(359.06877273,61.67400185)(359.14377265,61.59400193)(359.2037793,61.51400879)
\curveto(359.36377243,61.30400222)(359.4987723,61.10400242)(359.6087793,60.91400879)
\curveto(359.67877212,60.80400272)(359.73377206,60.68400284)(359.7737793,60.55400879)
\curveto(359.81377198,60.4240031)(359.85877194,60.29400323)(359.9087793,60.16400879)
\curveto(359.95877184,60.03400349)(359.9937718,59.89900362)(360.0137793,59.75900879)
\curveto(360.04377175,59.6190039)(360.07877172,59.47900404)(360.1187793,59.33900879)
\curveto(360.12877167,59.26900425)(360.13377166,59.19900432)(360.1337793,59.12900879)
\lineto(360.1637793,58.91900879)
\moveto(358.7087793,59.42900879)
\curveto(358.73877306,59.46900405)(358.76377303,59.519004)(358.7837793,59.57900879)
\curveto(358.80377299,59.64900387)(358.80377299,59.7190038)(358.7837793,59.78900879)
\curveto(358.72377307,60.00900351)(358.63877316,60.21400331)(358.5287793,60.40400879)
\curveto(358.38877341,60.63400289)(358.23377356,60.82900269)(358.0637793,60.98900879)
\curveto(357.8937739,61.14900237)(357.67377412,61.28400224)(357.4037793,61.39400879)
\curveto(357.33377446,61.41400211)(357.26377453,61.42900209)(357.1937793,61.43900879)
\curveto(357.12377467,61.45900206)(357.04877475,61.47900204)(356.9687793,61.49900879)
\curveto(356.88877491,61.519002)(356.80377499,61.52900199)(356.7137793,61.52900879)
\lineto(356.4587793,61.52900879)
\curveto(356.42877537,61.50900201)(356.3937754,61.49900202)(356.3537793,61.49900879)
\curveto(356.31377548,61.50900201)(356.27877552,61.50900201)(356.2487793,61.49900879)
\lineto(356.0087793,61.43900879)
\curveto(355.93877586,61.42900209)(355.86877593,61.41400211)(355.7987793,61.39400879)
\curveto(355.50877629,61.27400225)(355.27377652,61.1240024)(355.0937793,60.94400879)
\curveto(354.92377687,60.76400276)(354.76877703,60.53900298)(354.6287793,60.26900879)
\curveto(354.5987772,60.2190033)(354.56877723,60.15400337)(354.5387793,60.07400879)
\curveto(354.50877729,60.00400352)(354.48377731,59.9240036)(354.4637793,59.83400879)
\curveto(354.44377735,59.74400378)(354.43877736,59.65900386)(354.4487793,59.57900879)
\curveto(354.45877734,59.49900402)(354.4937773,59.43900408)(354.5537793,59.39900879)
\curveto(354.63377716,59.33900418)(354.76877703,59.30900421)(354.9587793,59.30900879)
\curveto(355.15877664,59.3190042)(355.32877647,59.3240042)(355.4687793,59.32400879)
\lineto(357.7487793,59.32400879)
\curveto(357.8987739,59.3240042)(358.07877372,59.3190042)(358.2887793,59.30900879)
\curveto(358.4987733,59.30900421)(358.63877316,59.34900417)(358.7087793,59.42900879)
}
}
{
\newrgbcolor{curcolor}{0 0 0}
\pscustom[linestyle=none,fillstyle=solid,fillcolor=curcolor]
{
\newpath
\moveto(365.16041992,62.62400879)
\curveto(365.79041469,62.64400088)(366.29541418,62.55900096)(366.67541992,62.36900879)
\curveto(367.05541342,62.17900134)(367.36041312,61.89400163)(367.59041992,61.51400879)
\curveto(367.65041283,61.41400211)(367.69541278,61.30400222)(367.72541992,61.18400879)
\curveto(367.76541271,61.07400245)(367.80041268,60.95900256)(367.83041992,60.83900879)
\curveto(367.8804126,60.64900287)(367.91041257,60.44400308)(367.92041992,60.22400879)
\curveto(367.93041255,60.00400352)(367.93541254,59.77900374)(367.93541992,59.54900879)
\lineto(367.93541992,57.94400879)
\lineto(367.93541992,55.60400879)
\curveto(367.93541254,55.43400809)(367.93041255,55.26400826)(367.92041992,55.09400879)
\curveto(367.92041256,54.9240086)(367.85541262,54.81400871)(367.72541992,54.76400879)
\curveto(367.6754128,54.74400878)(367.62041286,54.73400879)(367.56041992,54.73400879)
\curveto(367.51041297,54.7240088)(367.45541302,54.7190088)(367.39541992,54.71900879)
\curveto(367.26541321,54.7190088)(367.14041334,54.7240088)(367.02041992,54.73400879)
\curveto(366.90041358,54.73400879)(366.81541366,54.77400875)(366.76541992,54.85400879)
\curveto(366.71541376,54.9240086)(366.69041379,55.01400851)(366.69041992,55.12400879)
\lineto(366.69041992,55.45400879)
\lineto(366.69041992,56.74400879)
\lineto(366.69041992,59.18900879)
\curveto(366.69041379,59.45900406)(366.68541379,59.7240038)(366.67541992,59.98400879)
\curveto(366.66541381,60.25400327)(366.62041386,60.48400304)(366.54041992,60.67400879)
\curveto(366.46041402,60.87400265)(366.34041414,61.03400249)(366.18041992,61.15400879)
\curveto(366.02041446,61.28400224)(365.83541464,61.38400214)(365.62541992,61.45400879)
\curveto(365.56541491,61.47400205)(365.50041498,61.48400204)(365.43041992,61.48400879)
\curveto(365.37041511,61.49400203)(365.31041517,61.50900201)(365.25041992,61.52900879)
\curveto(365.20041528,61.53900198)(365.12041536,61.53900198)(365.01041992,61.52900879)
\curveto(364.91041557,61.52900199)(364.84041564,61.524002)(364.80041992,61.51400879)
\curveto(364.76041572,61.49400203)(364.72541575,61.48400204)(364.69541992,61.48400879)
\curveto(364.66541581,61.49400203)(364.63041585,61.49400203)(364.59041992,61.48400879)
\curveto(364.46041602,61.45400207)(364.33541614,61.4190021)(364.21541992,61.37900879)
\curveto(364.10541637,61.34900217)(364.00041648,61.30400222)(363.90041992,61.24400879)
\curveto(363.86041662,61.2240023)(363.82541665,61.20400232)(363.79541992,61.18400879)
\curveto(363.76541671,61.16400236)(363.73041675,61.14400238)(363.69041992,61.12400879)
\curveto(363.34041714,60.87400265)(363.08541739,60.49900302)(362.92541992,59.99900879)
\curveto(362.89541758,59.9190036)(362.8754176,59.83400369)(362.86541992,59.74400879)
\curveto(362.85541762,59.66400386)(362.84041764,59.58400394)(362.82041992,59.50400879)
\curveto(362.80041768,59.45400407)(362.79541768,59.40400412)(362.80541992,59.35400879)
\curveto(362.81541766,59.31400421)(362.81041767,59.27400425)(362.79041992,59.23400879)
\lineto(362.79041992,58.91900879)
\curveto(362.7804177,58.88900463)(362.7754177,58.85400467)(362.77541992,58.81400879)
\curveto(362.78541769,58.77400475)(362.79041769,58.72900479)(362.79041992,58.67900879)
\lineto(362.79041992,58.22900879)
\lineto(362.79041992,56.78900879)
\lineto(362.79041992,55.46900879)
\lineto(362.79041992,55.12400879)
\curveto(362.79041769,55.01400851)(362.76541771,54.9240086)(362.71541992,54.85400879)
\curveto(362.66541781,54.77400875)(362.5754179,54.73400879)(362.44541992,54.73400879)
\curveto(362.32541815,54.7240088)(362.20041828,54.7190088)(362.07041992,54.71900879)
\curveto(361.99041849,54.7190088)(361.91541856,54.7240088)(361.84541992,54.73400879)
\curveto(361.7754187,54.74400878)(361.71541876,54.76900875)(361.66541992,54.80900879)
\curveto(361.58541889,54.85900866)(361.54541893,54.95400857)(361.54541992,55.09400879)
\lineto(361.54541992,55.49900879)
\lineto(361.54541992,57.26900879)
\lineto(361.54541992,60.89900879)
\lineto(361.54541992,61.81400879)
\lineto(361.54541992,62.08400879)
\curveto(361.54541893,62.17400135)(361.56541891,62.24400128)(361.60541992,62.29400879)
\curveto(361.63541884,62.35400117)(361.68541879,62.39400113)(361.75541992,62.41400879)
\curveto(361.79541868,62.4240011)(361.85041863,62.43400109)(361.92041992,62.44400879)
\curveto(362.00041848,62.45400107)(362.0804184,62.45900106)(362.16041992,62.45900879)
\curveto(362.24041824,62.45900106)(362.31541816,62.45400107)(362.38541992,62.44400879)
\curveto(362.46541801,62.43400109)(362.52041796,62.4190011)(362.55041992,62.39900879)
\curveto(362.66041782,62.32900119)(362.71041777,62.23900128)(362.70041992,62.12900879)
\curveto(362.69041779,62.02900149)(362.70541777,61.91400161)(362.74541992,61.78400879)
\curveto(362.76541771,61.7240018)(362.80541767,61.67400185)(362.86541992,61.63400879)
\curveto(362.98541749,61.6240019)(363.0804174,61.66900185)(363.15041992,61.76900879)
\curveto(363.23041725,61.86900165)(363.31041717,61.94900157)(363.39041992,62.00900879)
\curveto(363.53041695,62.10900141)(363.67041681,62.19900132)(363.81041992,62.27900879)
\curveto(363.96041652,62.36900115)(364.13041635,62.44400108)(364.32041992,62.50400879)
\curveto(364.40041608,62.53400099)(364.48541599,62.55400097)(364.57541992,62.56400879)
\curveto(364.6754158,62.57400095)(364.77041571,62.58900093)(364.86041992,62.60900879)
\curveto(364.91041557,62.6190009)(364.96041552,62.6240009)(365.01041992,62.62400879)
\lineto(365.16041992,62.62400879)
}
}
{
\newrgbcolor{curcolor}{0 0 0}
\pscustom[linestyle=none,fillstyle=solid,fillcolor=curcolor]
{
\newpath
\moveto(370.7650293,64.81400879)
\curveto(370.91502729,64.81399871)(371.06502714,64.80899871)(371.2150293,64.79900879)
\curveto(371.36502684,64.79899872)(371.47002673,64.75899876)(371.5300293,64.67900879)
\curveto(371.58002662,64.6189989)(371.6050266,64.53399899)(371.6050293,64.42400879)
\curveto(371.61502659,64.3239992)(371.62002658,64.2189993)(371.6200293,64.10900879)
\lineto(371.6200293,63.23900879)
\curveto(371.62002658,63.15900036)(371.61502659,63.07400045)(371.6050293,62.98400879)
\curveto(371.6050266,62.90400062)(371.61502659,62.83400069)(371.6350293,62.77400879)
\curveto(371.67502653,62.63400089)(371.76502644,62.54400098)(371.9050293,62.50400879)
\curveto(371.95502625,62.49400103)(372.0000262,62.48900103)(372.0400293,62.48900879)
\lineto(372.1900293,62.48900879)
\lineto(372.5950293,62.48900879)
\curveto(372.75502545,62.49900102)(372.87002533,62.48900103)(372.9400293,62.45900879)
\curveto(373.03002517,62.39900112)(373.09002511,62.33900118)(373.1200293,62.27900879)
\curveto(373.14002506,62.23900128)(373.15002505,62.19400133)(373.1500293,62.14400879)
\lineto(373.1500293,61.99400879)
\curveto(373.15002505,61.88400164)(373.14502506,61.77900174)(373.1350293,61.67900879)
\curveto(373.12502508,61.58900193)(373.09002511,61.519002)(373.0300293,61.46900879)
\curveto(372.97002523,61.4190021)(372.88502532,61.38900213)(372.7750293,61.37900879)
\lineto(372.4450293,61.37900879)
\curveto(372.33502587,61.38900213)(372.22502598,61.39400213)(372.1150293,61.39400879)
\curveto(372.0050262,61.39400213)(371.91002629,61.37900214)(371.8300293,61.34900879)
\curveto(371.76002644,61.3190022)(371.71002649,61.26900225)(371.6800293,61.19900879)
\curveto(371.65002655,61.12900239)(371.63002657,61.04400248)(371.6200293,60.94400879)
\curveto(371.61002659,60.85400267)(371.6050266,60.75400277)(371.6050293,60.64400879)
\curveto(371.61502659,60.54400298)(371.62002658,60.44400308)(371.6200293,60.34400879)
\lineto(371.6200293,57.37400879)
\curveto(371.62002658,57.15400637)(371.61502659,56.9190066)(371.6050293,56.66900879)
\curveto(371.6050266,56.42900709)(371.65002655,56.24400728)(371.7400293,56.11400879)
\curveto(371.79002641,56.03400749)(371.85502635,55.97900754)(371.9350293,55.94900879)
\curveto(372.01502619,55.9190076)(372.11002609,55.89400763)(372.2200293,55.87400879)
\curveto(372.25002595,55.86400766)(372.28002592,55.85900766)(372.3100293,55.85900879)
\curveto(372.35002585,55.86900765)(372.38502582,55.86900765)(372.4150293,55.85900879)
\lineto(372.6100293,55.85900879)
\curveto(372.71002549,55.85900766)(372.8000254,55.84900767)(372.8800293,55.82900879)
\curveto(372.97002523,55.8190077)(373.03502517,55.78400774)(373.0750293,55.72400879)
\curveto(373.09502511,55.69400783)(373.11002509,55.63900788)(373.1200293,55.55900879)
\curveto(373.14002506,55.48900803)(373.15002505,55.41400811)(373.1500293,55.33400879)
\curveto(373.16002504,55.25400827)(373.16002504,55.17400835)(373.1500293,55.09400879)
\curveto(373.14002506,55.0240085)(373.12002508,54.96900855)(373.0900293,54.92900879)
\curveto(373.05002515,54.85900866)(372.97502523,54.80900871)(372.8650293,54.77900879)
\curveto(372.78502542,54.75900876)(372.69502551,54.74900877)(372.5950293,54.74900879)
\curveto(372.49502571,54.75900876)(372.4050258,54.76400876)(372.3250293,54.76400879)
\curveto(372.26502594,54.76400876)(372.205026,54.75900876)(372.1450293,54.74900879)
\curveto(372.08502612,54.74900877)(372.03002617,54.75400877)(371.9800293,54.76400879)
\lineto(371.8000293,54.76400879)
\curveto(371.75002645,54.77400875)(371.7000265,54.77900874)(371.6500293,54.77900879)
\curveto(371.61002659,54.78900873)(371.56502664,54.79400873)(371.5150293,54.79400879)
\curveto(371.31502689,54.84400868)(371.14002706,54.89900862)(370.9900293,54.95900879)
\curveto(370.85002735,55.0190085)(370.73002747,55.1240084)(370.6300293,55.27400879)
\curveto(370.49002771,55.47400805)(370.41002779,55.7240078)(370.3900293,56.02400879)
\curveto(370.37002783,56.33400719)(370.36002784,56.66400686)(370.3600293,57.01400879)
\lineto(370.3600293,60.94400879)
\curveto(370.33002787,61.07400245)(370.3000279,61.16900235)(370.2700293,61.22900879)
\curveto(370.25002795,61.28900223)(370.18002802,61.33900218)(370.0600293,61.37900879)
\curveto(370.02002818,61.38900213)(369.98002822,61.38900213)(369.9400293,61.37900879)
\curveto(369.9000283,61.36900215)(369.86002834,61.37400215)(369.8200293,61.39400879)
\lineto(369.5800293,61.39400879)
\curveto(369.45002875,61.39400213)(369.34002886,61.40400212)(369.2500293,61.42400879)
\curveto(369.17002903,61.45400207)(369.11502909,61.51400201)(369.0850293,61.60400879)
\curveto(369.06502914,61.64400188)(369.05002915,61.68900183)(369.0400293,61.73900879)
\lineto(369.0400293,61.88900879)
\curveto(369.04002916,62.02900149)(369.05002915,62.14400138)(369.0700293,62.23400879)
\curveto(369.09002911,62.33400119)(369.15002905,62.40900111)(369.2500293,62.45900879)
\curveto(369.36002884,62.49900102)(369.5000287,62.50900101)(369.6700293,62.48900879)
\curveto(369.85002835,62.46900105)(370.0000282,62.47900104)(370.1200293,62.51900879)
\curveto(370.21002799,62.56900095)(370.28002792,62.63900088)(370.3300293,62.72900879)
\curveto(370.35002785,62.78900073)(370.36002784,62.86400066)(370.3600293,62.95400879)
\lineto(370.3600293,63.20900879)
\lineto(370.3600293,64.13900879)
\lineto(370.3600293,64.37900879)
\curveto(370.36002784,64.46899905)(370.37002783,64.54399898)(370.3900293,64.60400879)
\curveto(370.43002777,64.68399884)(370.5050277,64.74899877)(370.6150293,64.79900879)
\curveto(370.64502756,64.79899872)(370.67002753,64.79899872)(370.6900293,64.79900879)
\curveto(370.72002748,64.80899871)(370.74502746,64.81399871)(370.7650293,64.81400879)
}
}
{
\newrgbcolor{curcolor}{0 0 0}
\pscustom[linestyle=none,fillstyle=solid,fillcolor=curcolor]
{
\newpath
\moveto(381.66182617,58.94900879)
\curveto(381.68181811,58.88900463)(381.6918181,58.79400473)(381.69182617,58.66400879)
\curveto(381.6918181,58.54400498)(381.68681811,58.45900506)(381.67682617,58.40900879)
\lineto(381.67682617,58.25900879)
\curveto(381.66681813,58.17900534)(381.65681814,58.10400542)(381.64682617,58.03400879)
\curveto(381.64681815,57.97400555)(381.64181815,57.90400562)(381.63182617,57.82400879)
\curveto(381.61181818,57.76400576)(381.5968182,57.70400582)(381.58682617,57.64400879)
\curveto(381.58681821,57.58400594)(381.57681822,57.524006)(381.55682617,57.46400879)
\curveto(381.51681828,57.33400619)(381.48181831,57.20400632)(381.45182617,57.07400879)
\curveto(381.42181837,56.94400658)(381.38181841,56.8240067)(381.33182617,56.71400879)
\curveto(381.12181867,56.23400729)(380.84181895,55.82900769)(380.49182617,55.49900879)
\curveto(380.14181965,55.17900834)(379.71182008,54.93400859)(379.20182617,54.76400879)
\curveto(379.0918207,54.7240088)(378.97182082,54.69400883)(378.84182617,54.67400879)
\curveto(378.72182107,54.65400887)(378.5968212,54.63400889)(378.46682617,54.61400879)
\curveto(378.40682139,54.60400892)(378.34182145,54.59900892)(378.27182617,54.59900879)
\curveto(378.21182158,54.58900893)(378.15182164,54.58400894)(378.09182617,54.58400879)
\curveto(378.05182174,54.57400895)(377.9918218,54.56900895)(377.91182617,54.56900879)
\curveto(377.84182195,54.56900895)(377.791822,54.57400895)(377.76182617,54.58400879)
\curveto(377.72182207,54.59400893)(377.68182211,54.59900892)(377.64182617,54.59900879)
\curveto(377.60182219,54.58900893)(377.56682223,54.58900893)(377.53682617,54.59900879)
\lineto(377.44682617,54.59900879)
\lineto(377.08682617,54.64400879)
\curveto(376.94682285,54.68400884)(376.81182298,54.7240088)(376.68182617,54.76400879)
\curveto(376.55182324,54.80400872)(376.42682337,54.84900867)(376.30682617,54.89900879)
\curveto(375.85682394,55.09900842)(375.48682431,55.35900816)(375.19682617,55.67900879)
\curveto(374.90682489,55.99900752)(374.66682513,56.38900713)(374.47682617,56.84900879)
\curveto(374.42682537,56.94900657)(374.38682541,57.04900647)(374.35682617,57.14900879)
\curveto(374.33682546,57.24900627)(374.31682548,57.35400617)(374.29682617,57.46400879)
\curveto(374.27682552,57.50400602)(374.26682553,57.53400599)(374.26682617,57.55400879)
\curveto(374.27682552,57.58400594)(374.27682552,57.6190059)(374.26682617,57.65900879)
\curveto(374.24682555,57.73900578)(374.23182556,57.8190057)(374.22182617,57.89900879)
\curveto(374.22182557,57.98900553)(374.21182558,58.07400545)(374.19182617,58.15400879)
\lineto(374.19182617,58.27400879)
\curveto(374.1918256,58.31400521)(374.18682561,58.35900516)(374.17682617,58.40900879)
\curveto(374.16682563,58.45900506)(374.16182563,58.54400498)(374.16182617,58.66400879)
\curveto(374.16182563,58.79400473)(374.17182562,58.88900463)(374.19182617,58.94900879)
\curveto(374.21182558,59.0190045)(374.21682558,59.08900443)(374.20682617,59.15900879)
\curveto(374.1968256,59.22900429)(374.20182559,59.29900422)(374.22182617,59.36900879)
\curveto(374.23182556,59.4190041)(374.23682556,59.45900406)(374.23682617,59.48900879)
\curveto(374.24682555,59.52900399)(374.25682554,59.57400395)(374.26682617,59.62400879)
\curveto(374.2968255,59.74400378)(374.32182547,59.86400366)(374.34182617,59.98400879)
\curveto(374.37182542,60.10400342)(374.41182538,60.2190033)(374.46182617,60.32900879)
\curveto(374.61182518,60.69900282)(374.791825,61.02900249)(375.00182617,61.31900879)
\curveto(375.22182457,61.6190019)(375.48682431,61.86900165)(375.79682617,62.06900879)
\curveto(375.91682388,62.14900137)(376.04182375,62.21400131)(376.17182617,62.26400879)
\curveto(376.30182349,62.3240012)(376.43682336,62.38400114)(376.57682617,62.44400879)
\curveto(376.6968231,62.49400103)(376.82682297,62.524001)(376.96682617,62.53400879)
\curveto(377.10682269,62.55400097)(377.24682255,62.58400094)(377.38682617,62.62400879)
\lineto(377.58182617,62.62400879)
\curveto(377.65182214,62.63400089)(377.71682208,62.64400088)(377.77682617,62.65400879)
\curveto(378.66682113,62.66400086)(379.40682039,62.47900104)(379.99682617,62.09900879)
\curveto(380.58681921,61.7190018)(381.01181878,61.2240023)(381.27182617,60.61400879)
\curveto(381.32181847,60.51400301)(381.36181843,60.41400311)(381.39182617,60.31400879)
\curveto(381.42181837,60.21400331)(381.45681834,60.10900341)(381.49682617,59.99900879)
\curveto(381.52681827,59.88900363)(381.55181824,59.76900375)(381.57182617,59.63900879)
\curveto(381.5918182,59.519004)(381.61681818,59.39400413)(381.64682617,59.26400879)
\curveto(381.65681814,59.21400431)(381.65681814,59.15900436)(381.64682617,59.09900879)
\curveto(381.64681815,59.04900447)(381.65181814,58.99900452)(381.66182617,58.94900879)
\moveto(380.32682617,58.09400879)
\curveto(380.34681945,58.16400536)(380.35181944,58.24400528)(380.34182617,58.33400879)
\lineto(380.34182617,58.58900879)
\curveto(380.34181945,58.97900454)(380.30681949,59.30900421)(380.23682617,59.57900879)
\curveto(380.20681959,59.65900386)(380.18181961,59.73900378)(380.16182617,59.81900879)
\curveto(380.14181965,59.89900362)(380.11681968,59.97400355)(380.08682617,60.04400879)
\curveto(379.80681999,60.69400283)(379.36182043,61.14400238)(378.75182617,61.39400879)
\curveto(378.68182111,61.4240021)(378.60682119,61.44400208)(378.52682617,61.45400879)
\lineto(378.28682617,61.51400879)
\curveto(378.20682159,61.53400199)(378.12182167,61.54400198)(378.03182617,61.54400879)
\lineto(377.76182617,61.54400879)
\lineto(377.49182617,61.49900879)
\curveto(377.3918224,61.47900204)(377.2968225,61.45400207)(377.20682617,61.42400879)
\curveto(377.12682267,61.40400212)(377.04682275,61.37400215)(376.96682617,61.33400879)
\curveto(376.8968229,61.31400221)(376.83182296,61.28400224)(376.77182617,61.24400879)
\curveto(376.71182308,61.20400232)(376.65682314,61.16400236)(376.60682617,61.12400879)
\curveto(376.36682343,60.95400257)(376.17182362,60.74900277)(376.02182617,60.50900879)
\curveto(375.87182392,60.26900325)(375.74182405,59.98900353)(375.63182617,59.66900879)
\curveto(375.60182419,59.56900395)(375.58182421,59.46400406)(375.57182617,59.35400879)
\curveto(375.56182423,59.25400427)(375.54682425,59.14900437)(375.52682617,59.03900879)
\curveto(375.51682428,58.99900452)(375.51182428,58.93400459)(375.51182617,58.84400879)
\curveto(375.50182429,58.81400471)(375.4968243,58.77900474)(375.49682617,58.73900879)
\curveto(375.50682429,58.69900482)(375.51182428,58.65400487)(375.51182617,58.60400879)
\lineto(375.51182617,58.30400879)
\curveto(375.51182428,58.20400532)(375.52182427,58.11400541)(375.54182617,58.03400879)
\lineto(375.57182617,57.85400879)
\curveto(375.5918242,57.75400577)(375.60682419,57.65400587)(375.61682617,57.55400879)
\curveto(375.63682416,57.46400606)(375.66682413,57.37900614)(375.70682617,57.29900879)
\curveto(375.80682399,57.05900646)(375.92182387,56.83400669)(376.05182617,56.62400879)
\curveto(376.1918236,56.41400711)(376.36182343,56.23900728)(376.56182617,56.09900879)
\curveto(376.61182318,56.06900745)(376.65682314,56.04400748)(376.69682617,56.02400879)
\curveto(376.73682306,56.00400752)(376.78182301,55.97900754)(376.83182617,55.94900879)
\curveto(376.91182288,55.89900762)(376.9968228,55.85400767)(377.08682617,55.81400879)
\curveto(377.18682261,55.78400774)(377.2918225,55.75400777)(377.40182617,55.72400879)
\curveto(377.45182234,55.70400782)(377.4968223,55.69400783)(377.53682617,55.69400879)
\curveto(377.58682221,55.70400782)(377.63682216,55.70400782)(377.68682617,55.69400879)
\curveto(377.71682208,55.68400784)(377.77682202,55.67400785)(377.86682617,55.66400879)
\curveto(377.96682183,55.65400787)(378.04182175,55.65900786)(378.09182617,55.67900879)
\curveto(378.13182166,55.68900783)(378.17182162,55.68900783)(378.21182617,55.67900879)
\curveto(378.25182154,55.67900784)(378.2918215,55.68900783)(378.33182617,55.70900879)
\curveto(378.41182138,55.72900779)(378.4918213,55.74400778)(378.57182617,55.75400879)
\curveto(378.65182114,55.77400775)(378.72682107,55.79900772)(378.79682617,55.82900879)
\curveto(379.13682066,55.96900755)(379.41182038,56.16400736)(379.62182617,56.41400879)
\curveto(379.83181996,56.66400686)(380.00681979,56.95900656)(380.14682617,57.29900879)
\curveto(380.1968196,57.4190061)(380.22681957,57.54400598)(380.23682617,57.67400879)
\curveto(380.25681954,57.81400571)(380.28681951,57.95400557)(380.32682617,58.09400879)
}
}
{
\newrgbcolor{curcolor}{0 0 0}
\pscustom[linestyle=none,fillstyle=solid,fillcolor=curcolor]
{
\newpath
\moveto(385.58010742,62.65400879)
\curveto(386.30010336,62.66400086)(386.90510275,62.57900094)(387.39510742,62.39900879)
\curveto(387.88510177,62.22900129)(388.26510139,61.9240016)(388.53510742,61.48400879)
\curveto(388.60510105,61.37400215)(388.660101,61.25900226)(388.70010742,61.13900879)
\curveto(388.74010092,61.02900249)(388.78010088,60.90400262)(388.82010742,60.76400879)
\curveto(388.84010082,60.69400283)(388.84510081,60.6190029)(388.83510742,60.53900879)
\curveto(388.82510083,60.46900305)(388.81010085,60.41400311)(388.79010742,60.37400879)
\curveto(388.77010089,60.35400317)(388.74510091,60.33400319)(388.71510742,60.31400879)
\curveto(388.68510097,60.30400322)(388.660101,60.28900323)(388.64010742,60.26900879)
\curveto(388.59010107,60.24900327)(388.54010112,60.24400328)(388.49010742,60.25400879)
\curveto(388.44010122,60.26400326)(388.39010127,60.26400326)(388.34010742,60.25400879)
\curveto(388.2601014,60.23400329)(388.1551015,60.22900329)(388.02510742,60.23900879)
\curveto(387.89510176,60.25900326)(387.80510185,60.28400324)(387.75510742,60.31400879)
\curveto(387.67510198,60.36400316)(387.62010204,60.42900309)(387.59010742,60.50900879)
\curveto(387.57010209,60.59900292)(387.53510212,60.68400284)(387.48510742,60.76400879)
\curveto(387.39510226,60.9240026)(387.27010239,61.06900245)(387.11010742,61.19900879)
\curveto(387.00010266,61.27900224)(386.88010278,61.33900218)(386.75010742,61.37900879)
\curveto(386.62010304,61.4190021)(386.48010318,61.45900206)(386.33010742,61.49900879)
\curveto(386.28010338,61.519002)(386.23010343,61.524002)(386.18010742,61.51400879)
\curveto(386.13010353,61.51400201)(386.08010358,61.519002)(386.03010742,61.52900879)
\curveto(385.97010369,61.54900197)(385.89510376,61.55900196)(385.80510742,61.55900879)
\curveto(385.71510394,61.55900196)(385.64010402,61.54900197)(385.58010742,61.52900879)
\lineto(385.49010742,61.52900879)
\lineto(385.34010742,61.49900879)
\curveto(385.29010437,61.49900202)(385.24010442,61.49400203)(385.19010742,61.48400879)
\curveto(384.93010473,61.4240021)(384.71510494,61.33900218)(384.54510742,61.22900879)
\curveto(384.37510528,61.1190024)(384.2601054,60.93400259)(384.20010742,60.67400879)
\curveto(384.18010548,60.60400292)(384.17510548,60.53400299)(384.18510742,60.46400879)
\curveto(384.20510545,60.39400313)(384.22510543,60.33400319)(384.24510742,60.28400879)
\curveto(384.30510535,60.13400339)(384.37510528,60.0240035)(384.45510742,59.95400879)
\curveto(384.54510511,59.89400363)(384.655105,59.8240037)(384.78510742,59.74400879)
\curveto(384.94510471,59.64400388)(385.12510453,59.56900395)(385.32510742,59.51900879)
\curveto(385.52510413,59.47900404)(385.72510393,59.42900409)(385.92510742,59.36900879)
\curveto(386.0551036,59.32900419)(386.18510347,59.29900422)(386.31510742,59.27900879)
\curveto(386.44510321,59.25900426)(386.57510308,59.22900429)(386.70510742,59.18900879)
\curveto(386.91510274,59.12900439)(387.12010254,59.06900445)(387.32010742,59.00900879)
\curveto(387.52010214,58.95900456)(387.72010194,58.89400463)(387.92010742,58.81400879)
\lineto(388.07010742,58.75400879)
\curveto(388.12010154,58.73400479)(388.17010149,58.70900481)(388.22010742,58.67900879)
\curveto(388.42010124,58.55900496)(388.59510106,58.4240051)(388.74510742,58.27400879)
\curveto(388.89510076,58.1240054)(389.02010064,57.93400559)(389.12010742,57.70400879)
\curveto(389.14010052,57.63400589)(389.1601005,57.53900598)(389.18010742,57.41900879)
\curveto(389.20010046,57.34900617)(389.21010045,57.27400625)(389.21010742,57.19400879)
\curveto(389.22010044,57.1240064)(389.22510043,57.04400648)(389.22510742,56.95400879)
\lineto(389.22510742,56.80400879)
\curveto(389.20510045,56.73400679)(389.19510046,56.66400686)(389.19510742,56.59400879)
\curveto(389.19510046,56.524007)(389.18510047,56.45400707)(389.16510742,56.38400879)
\curveto(389.13510052,56.27400725)(389.10010056,56.16900735)(389.06010742,56.06900879)
\curveto(389.02010064,55.96900755)(388.97510068,55.87900764)(388.92510742,55.79900879)
\curveto(388.76510089,55.53900798)(388.5601011,55.32900819)(388.31010742,55.16900879)
\curveto(388.0601016,55.0190085)(387.78010188,54.88900863)(387.47010742,54.77900879)
\curveto(387.38010228,54.74900877)(387.28510237,54.72900879)(387.18510742,54.71900879)
\curveto(387.09510256,54.69900882)(387.00510265,54.67400885)(386.91510742,54.64400879)
\curveto(386.81510284,54.6240089)(386.71510294,54.61400891)(386.61510742,54.61400879)
\curveto(386.51510314,54.61400891)(386.41510324,54.60400892)(386.31510742,54.58400879)
\lineto(386.16510742,54.58400879)
\curveto(386.11510354,54.57400895)(386.04510361,54.56900895)(385.95510742,54.56900879)
\curveto(385.86510379,54.56900895)(385.79510386,54.57400895)(385.74510742,54.58400879)
\lineto(385.58010742,54.58400879)
\curveto(385.52010414,54.60400892)(385.4551042,54.61400891)(385.38510742,54.61400879)
\curveto(385.31510434,54.60400892)(385.2551044,54.60900891)(385.20510742,54.62900879)
\curveto(385.1551045,54.63900888)(385.09010457,54.64400888)(385.01010742,54.64400879)
\lineto(384.77010742,54.70400879)
\curveto(384.70010496,54.71400881)(384.62510503,54.73400879)(384.54510742,54.76400879)
\curveto(384.23510542,54.86400866)(383.96510569,54.98900853)(383.73510742,55.13900879)
\curveto(383.50510615,55.28900823)(383.30510635,55.48400804)(383.13510742,55.72400879)
\curveto(383.04510661,55.85400767)(382.97010669,55.98900753)(382.91010742,56.12900879)
\curveto(382.85010681,56.26900725)(382.79510686,56.4240071)(382.74510742,56.59400879)
\curveto(382.72510693,56.65400687)(382.71510694,56.7240068)(382.71510742,56.80400879)
\curveto(382.72510693,56.89400663)(382.74010692,56.96400656)(382.76010742,57.01400879)
\curveto(382.79010687,57.05400647)(382.84010682,57.09400643)(382.91010742,57.13400879)
\curveto(382.9601067,57.15400637)(383.03010663,57.16400636)(383.12010742,57.16400879)
\curveto(383.21010645,57.17400635)(383.30010636,57.17400635)(383.39010742,57.16400879)
\curveto(383.48010618,57.15400637)(383.56510609,57.13900638)(383.64510742,57.11900879)
\curveto(383.73510592,57.10900641)(383.79510586,57.09400643)(383.82510742,57.07400879)
\curveto(383.89510576,57.0240065)(383.94010572,56.94900657)(383.96010742,56.84900879)
\curveto(383.99010567,56.75900676)(384.02510563,56.67400685)(384.06510742,56.59400879)
\curveto(384.16510549,56.37400715)(384.30010536,56.20400732)(384.47010742,56.08400879)
\curveto(384.59010507,55.99400753)(384.72510493,55.9240076)(384.87510742,55.87400879)
\curveto(385.02510463,55.8240077)(385.18510447,55.77400775)(385.35510742,55.72400879)
\lineto(385.67010742,55.67900879)
\lineto(385.76010742,55.67900879)
\curveto(385.83010383,55.65900786)(385.92010374,55.64900787)(386.03010742,55.64900879)
\curveto(386.15010351,55.64900787)(386.25010341,55.65900786)(386.33010742,55.67900879)
\curveto(386.40010326,55.67900784)(386.4551032,55.68400784)(386.49510742,55.69400879)
\curveto(386.5551031,55.70400782)(386.61510304,55.70900781)(386.67510742,55.70900879)
\curveto(386.73510292,55.7190078)(386.79010287,55.72900779)(386.84010742,55.73900879)
\curveto(387.13010253,55.8190077)(387.3601023,55.9240076)(387.53010742,56.05400879)
\curveto(387.70010196,56.18400734)(387.82010184,56.40400712)(387.89010742,56.71400879)
\curveto(387.91010175,56.76400676)(387.91510174,56.8190067)(387.90510742,56.87900879)
\curveto(387.89510176,56.93900658)(387.88510177,56.98400654)(387.87510742,57.01400879)
\curveto(387.82510183,57.20400632)(387.7551019,57.34400618)(387.66510742,57.43400879)
\curveto(387.57510208,57.53400599)(387.4601022,57.6240059)(387.32010742,57.70400879)
\curveto(387.23010243,57.76400576)(387.13010253,57.81400571)(387.02010742,57.85400879)
\lineto(386.69010742,57.97400879)
\curveto(386.660103,57.98400554)(386.63010303,57.98900553)(386.60010742,57.98900879)
\curveto(386.58010308,57.98900553)(386.5551031,57.99900552)(386.52510742,58.01900879)
\curveto(386.18510347,58.12900539)(385.83010383,58.20900531)(385.46010742,58.25900879)
\curveto(385.10010456,58.3190052)(384.7601049,58.41400511)(384.44010742,58.54400879)
\curveto(384.34010532,58.58400494)(384.24510541,58.6190049)(384.15510742,58.64900879)
\curveto(384.06510559,58.67900484)(383.98010568,58.7190048)(383.90010742,58.76900879)
\curveto(383.71010595,58.87900464)(383.53510612,59.00400452)(383.37510742,59.14400879)
\curveto(383.21510644,59.28400424)(383.09010657,59.45900406)(383.00010742,59.66900879)
\curveto(382.97010669,59.73900378)(382.94510671,59.80900371)(382.92510742,59.87900879)
\curveto(382.91510674,59.94900357)(382.90010676,60.0240035)(382.88010742,60.10400879)
\curveto(382.85010681,60.2240033)(382.84010682,60.35900316)(382.85010742,60.50900879)
\curveto(382.8601068,60.66900285)(382.87510678,60.80400272)(382.89510742,60.91400879)
\curveto(382.91510674,60.96400256)(382.92510673,61.00400252)(382.92510742,61.03400879)
\curveto(382.93510672,61.07400245)(382.95010671,61.11400241)(382.97010742,61.15400879)
\curveto(383.0601066,61.38400214)(383.18010648,61.58400194)(383.33010742,61.75400879)
\curveto(383.49010617,61.9240016)(383.67010599,62.07400145)(383.87010742,62.20400879)
\curveto(384.02010564,62.29400123)(384.18510547,62.36400116)(384.36510742,62.41400879)
\curveto(384.54510511,62.47400105)(384.73510492,62.52900099)(384.93510742,62.57900879)
\curveto(385.00510465,62.58900093)(385.07010459,62.59900092)(385.13010742,62.60900879)
\curveto(385.20010446,62.6190009)(385.27510438,62.62900089)(385.35510742,62.63900879)
\curveto(385.38510427,62.64900087)(385.42510423,62.64900087)(385.47510742,62.63900879)
\curveto(385.52510413,62.62900089)(385.5601041,62.63400089)(385.58010742,62.65400879)
}
}
{
\newrgbcolor{curcolor}{0 0 0}
\pscustom[linestyle=none,fillstyle=solid,fillcolor=curcolor]
{
\newpath
\moveto(540.82596191,65.61150391)
\lineto(545.46096191,65.61150391)
\lineto(546.67596191,65.61150391)
\curveto(546.78595436,65.61149321)(546.89095426,65.61149321)(546.99096191,65.61150391)
\curveto(547.10095405,65.61149321)(547.18595396,65.59149323)(547.24596191,65.55150391)
\curveto(547.32595382,65.50149332)(547.37095378,65.4264934)(547.38096191,65.32650391)
\curveto(547.40095375,65.23649359)(547.41095374,65.1264937)(547.41096191,64.99650391)
\lineto(547.41096191,64.84650391)
\curveto(547.42095373,64.80649402)(547.41595373,64.76649406)(547.39596191,64.72650391)
\curveto(547.35595379,64.56649426)(547.26595388,64.47649435)(547.12596191,64.45650391)
\curveto(546.99595415,64.44649438)(546.83095432,64.44149438)(546.63096191,64.44150391)
\lineto(545.07096191,64.44150391)
\lineto(542.86596191,64.44150391)
\lineto(542.35596191,64.44150391)
\curveto(542.17595897,64.45149437)(542.04095911,64.4214944)(541.95096191,64.35150391)
\curveto(541.86095929,64.29149453)(541.81095934,64.18649464)(541.80096191,64.03650391)
\lineto(541.80096191,63.58650391)
\lineto(541.80096191,62.10150391)
\curveto(541.80095935,62.0214968)(541.79595935,61.9214969)(541.78596191,61.80150391)
\curveto(541.78595936,61.68149714)(541.79595935,61.58149724)(541.81596191,61.50150391)
\lineto(541.81596191,61.38150391)
\curveto(541.83595931,61.3214975)(541.8509593,61.26149756)(541.86096191,61.20150391)
\curveto(541.88095927,61.15149767)(541.91595923,61.11149771)(541.96596191,61.08150391)
\curveto(542.05595909,61.0214978)(542.19595895,60.99149783)(542.38596191,60.99150391)
\curveto(542.57595857,61.00149782)(542.74095841,61.00649782)(542.88096191,61.00650391)
\lineto(545.58096191,61.00650391)
\lineto(545.86596191,61.00650391)
\curveto(545.97595517,61.01649781)(546.08095507,61.01649781)(546.18096191,61.00650391)
\curveto(546.29095486,61.00649782)(546.38595476,60.99649783)(546.46596191,60.97650391)
\curveto(546.55595459,60.95649787)(546.61595453,60.9214979)(546.64596191,60.87150391)
\curveto(546.69595445,60.81149801)(546.72095443,60.73649809)(546.72096191,60.64650391)
\lineto(546.72096191,60.34650391)
\lineto(546.72096191,60.18150391)
\curveto(546.72095443,60.13149869)(546.71095444,60.08649874)(546.69096191,60.04650391)
\curveto(546.6509545,59.94649888)(546.59595455,59.89149893)(546.52596191,59.88150391)
\curveto(546.48595466,59.86149896)(546.4459547,59.85149897)(546.40596191,59.85150391)
\curveto(546.37595477,59.85149897)(546.33595481,59.84649898)(546.28596191,59.83650391)
\curveto(546.2459549,59.826499)(546.20095495,59.821499)(546.15096191,59.82150391)
\curveto(546.11095504,59.83149899)(546.07095508,59.83649899)(546.03096191,59.83650391)
\lineto(545.50596191,59.83650391)
\lineto(542.97096191,59.83650391)
\lineto(542.40096191,59.83650391)
\curveto(542.19095896,59.84649898)(542.04095911,59.81649901)(541.95096191,59.74650391)
\curveto(541.90095925,59.70649912)(541.86095929,59.64149918)(541.83096191,59.55150391)
\curveto(541.81095934,59.47149935)(541.79595935,59.37649945)(541.78596191,59.26650391)
\lineto(541.78596191,58.92150391)
\curveto(541.79595935,58.81150001)(541.80095935,58.71150011)(541.80096191,58.62150391)
\lineto(541.80096191,56.04150391)
\curveto(541.80095935,55.87150295)(541.80595934,55.68650314)(541.81596191,55.48650391)
\curveto(541.82595932,55.28650354)(541.79095936,55.13650369)(541.71096191,55.03650391)
\curveto(541.68095947,54.99650383)(541.63595951,54.97150385)(541.57596191,54.96150391)
\curveto(541.51595963,54.96150386)(541.45595969,54.95150387)(541.39596191,54.93150391)
\lineto(541.11096191,54.93150391)
\curveto(540.97096018,54.93150389)(540.84096031,54.93650389)(540.72096191,54.94650391)
\curveto(540.60096055,54.95650387)(540.51596063,55.00650382)(540.46596191,55.09650391)
\curveto(540.42596072,55.15650367)(540.40596074,55.23650359)(540.40596191,55.33650391)
\lineto(540.40596191,55.66650391)
\lineto(540.40596191,56.86650391)
\lineto(540.40596191,63.13650391)
\lineto(540.40596191,64.75650391)
\curveto(540.40596074,64.86649396)(540.40096075,64.98649384)(540.39096191,65.11650391)
\curveto(540.39096076,65.25649357)(540.41596073,65.36649346)(540.46596191,65.44650391)
\curveto(540.50596064,65.50649332)(540.58096057,65.55649327)(540.69096191,65.59650391)
\curveto(540.71096044,65.60649322)(540.73096042,65.60649322)(540.75096191,65.59650391)
\curveto(540.78096037,65.59649323)(540.80596034,65.60149322)(540.82596191,65.61150391)
}
}
{
\newrgbcolor{curcolor}{0 0 0}
\pscustom[linestyle=none,fillstyle=solid,fillcolor=curcolor]
{
\newpath
\moveto(555.88924316,59.13150391)
\curveto(555.9092351,59.07149975)(555.91923509,58.97649985)(555.91924316,58.84650391)
\curveto(555.91923509,58.7265001)(555.9142351,58.64150018)(555.90424316,58.59150391)
\lineto(555.90424316,58.44150391)
\curveto(555.89423512,58.36150046)(555.88423513,58.28650054)(555.87424316,58.21650391)
\curveto(555.87423514,58.15650067)(555.86923514,58.08650074)(555.85924316,58.00650391)
\curveto(555.83923517,57.94650088)(555.82423519,57.88650094)(555.81424316,57.82650391)
\curveto(555.8142352,57.76650106)(555.80423521,57.70650112)(555.78424316,57.64650391)
\curveto(555.74423527,57.51650131)(555.7092353,57.38650144)(555.67924316,57.25650391)
\curveto(555.64923536,57.1265017)(555.6092354,57.00650182)(555.55924316,56.89650391)
\curveto(555.34923566,56.41650241)(555.06923594,56.01150281)(554.71924316,55.68150391)
\curveto(554.36923664,55.36150346)(553.93923707,55.11650371)(553.42924316,54.94650391)
\curveto(553.31923769,54.90650392)(553.19923781,54.87650395)(553.06924316,54.85650391)
\curveto(552.94923806,54.83650399)(552.82423819,54.81650401)(552.69424316,54.79650391)
\curveto(552.63423838,54.78650404)(552.56923844,54.78150404)(552.49924316,54.78150391)
\curveto(552.43923857,54.77150405)(552.37923863,54.76650406)(552.31924316,54.76650391)
\curveto(552.27923873,54.75650407)(552.21923879,54.75150407)(552.13924316,54.75150391)
\curveto(552.06923894,54.75150407)(552.01923899,54.75650407)(551.98924316,54.76650391)
\curveto(551.94923906,54.77650405)(551.9092391,54.78150404)(551.86924316,54.78150391)
\curveto(551.82923918,54.77150405)(551.79423922,54.77150405)(551.76424316,54.78150391)
\lineto(551.67424316,54.78150391)
\lineto(551.31424316,54.82650391)
\curveto(551.17423984,54.86650396)(551.03923997,54.90650392)(550.90924316,54.94650391)
\curveto(550.77924023,54.98650384)(550.65424036,55.03150379)(550.53424316,55.08150391)
\curveto(550.08424093,55.28150354)(549.7142413,55.54150328)(549.42424316,55.86150391)
\curveto(549.13424188,56.18150264)(548.89424212,56.57150225)(548.70424316,57.03150391)
\curveto(548.65424236,57.13150169)(548.6142424,57.23150159)(548.58424316,57.33150391)
\curveto(548.56424245,57.43150139)(548.54424247,57.53650129)(548.52424316,57.64650391)
\curveto(548.50424251,57.68650114)(548.49424252,57.71650111)(548.49424316,57.73650391)
\curveto(548.50424251,57.76650106)(548.50424251,57.80150102)(548.49424316,57.84150391)
\curveto(548.47424254,57.9215009)(548.45924255,58.00150082)(548.44924316,58.08150391)
\curveto(548.44924256,58.17150065)(548.43924257,58.25650057)(548.41924316,58.33650391)
\lineto(548.41924316,58.45650391)
\curveto(548.41924259,58.49650033)(548.4142426,58.54150028)(548.40424316,58.59150391)
\curveto(548.39424262,58.64150018)(548.38924262,58.7265001)(548.38924316,58.84650391)
\curveto(548.38924262,58.97649985)(548.39924261,59.07149975)(548.41924316,59.13150391)
\curveto(548.43924257,59.20149962)(548.44424257,59.27149955)(548.43424316,59.34150391)
\curveto(548.42424259,59.41149941)(548.42924258,59.48149934)(548.44924316,59.55150391)
\curveto(548.45924255,59.60149922)(548.46424255,59.64149918)(548.46424316,59.67150391)
\curveto(548.47424254,59.71149911)(548.48424253,59.75649907)(548.49424316,59.80650391)
\curveto(548.52424249,59.9264989)(548.54924246,60.04649878)(548.56924316,60.16650391)
\curveto(548.59924241,60.28649854)(548.63924237,60.40149842)(548.68924316,60.51150391)
\curveto(548.83924217,60.88149794)(549.01924199,61.21149761)(549.22924316,61.50150391)
\curveto(549.44924156,61.80149702)(549.7142413,62.05149677)(550.02424316,62.25150391)
\curveto(550.14424087,62.33149649)(550.26924074,62.39649643)(550.39924316,62.44650391)
\curveto(550.52924048,62.50649632)(550.66424035,62.56649626)(550.80424316,62.62650391)
\curveto(550.92424009,62.67649615)(551.05423996,62.70649612)(551.19424316,62.71650391)
\curveto(551.33423968,62.73649609)(551.47423954,62.76649606)(551.61424316,62.80650391)
\lineto(551.80924316,62.80650391)
\curveto(551.87923913,62.81649601)(551.94423907,62.826496)(552.00424316,62.83650391)
\curveto(552.89423812,62.84649598)(553.63423738,62.66149616)(554.22424316,62.28150391)
\curveto(554.8142362,61.90149692)(555.23923577,61.40649742)(555.49924316,60.79650391)
\curveto(555.54923546,60.69649813)(555.58923542,60.59649823)(555.61924316,60.49650391)
\curveto(555.64923536,60.39649843)(555.68423533,60.29149853)(555.72424316,60.18150391)
\curveto(555.75423526,60.07149875)(555.77923523,59.95149887)(555.79924316,59.82150391)
\curveto(555.81923519,59.70149912)(555.84423517,59.57649925)(555.87424316,59.44650391)
\curveto(555.88423513,59.39649943)(555.88423513,59.34149948)(555.87424316,59.28150391)
\curveto(555.87423514,59.23149959)(555.87923513,59.18149964)(555.88924316,59.13150391)
\moveto(554.55424316,58.27650391)
\curveto(554.57423644,58.34650048)(554.57923643,58.4265004)(554.56924316,58.51650391)
\lineto(554.56924316,58.77150391)
\curveto(554.56923644,59.16149966)(554.53423648,59.49149933)(554.46424316,59.76150391)
\curveto(554.43423658,59.84149898)(554.4092366,59.9214989)(554.38924316,60.00150391)
\curveto(554.36923664,60.08149874)(554.34423667,60.15649867)(554.31424316,60.22650391)
\curveto(554.03423698,60.87649795)(553.58923742,61.3264975)(552.97924316,61.57650391)
\curveto(552.9092381,61.60649722)(552.83423818,61.6264972)(552.75424316,61.63650391)
\lineto(552.51424316,61.69650391)
\curveto(552.43423858,61.71649711)(552.34923866,61.7264971)(552.25924316,61.72650391)
\lineto(551.98924316,61.72650391)
\lineto(551.71924316,61.68150391)
\curveto(551.61923939,61.66149716)(551.52423949,61.63649719)(551.43424316,61.60650391)
\curveto(551.35423966,61.58649724)(551.27423974,61.55649727)(551.19424316,61.51650391)
\curveto(551.12423989,61.49649733)(551.05923995,61.46649736)(550.99924316,61.42650391)
\curveto(550.93924007,61.38649744)(550.88424013,61.34649748)(550.83424316,61.30650391)
\curveto(550.59424042,61.13649769)(550.39924061,60.93149789)(550.24924316,60.69150391)
\curveto(550.09924091,60.45149837)(549.96924104,60.17149865)(549.85924316,59.85150391)
\curveto(549.82924118,59.75149907)(549.8092412,59.64649918)(549.79924316,59.53650391)
\curveto(549.78924122,59.43649939)(549.77424124,59.33149949)(549.75424316,59.22150391)
\curveto(549.74424127,59.18149964)(549.73924127,59.11649971)(549.73924316,59.02650391)
\curveto(549.72924128,58.99649983)(549.72424129,58.96149986)(549.72424316,58.92150391)
\curveto(549.73424128,58.88149994)(549.73924127,58.83649999)(549.73924316,58.78650391)
\lineto(549.73924316,58.48650391)
\curveto(549.73924127,58.38650044)(549.74924126,58.29650053)(549.76924316,58.21650391)
\lineto(549.79924316,58.03650391)
\curveto(549.81924119,57.93650089)(549.83424118,57.83650099)(549.84424316,57.73650391)
\curveto(549.86424115,57.64650118)(549.89424112,57.56150126)(549.93424316,57.48150391)
\curveto(550.03424098,57.24150158)(550.14924086,57.01650181)(550.27924316,56.80650391)
\curveto(550.41924059,56.59650223)(550.58924042,56.4215024)(550.78924316,56.28150391)
\curveto(550.83924017,56.25150257)(550.88424013,56.2265026)(550.92424316,56.20650391)
\curveto(550.96424005,56.18650264)(551.00924,56.16150266)(551.05924316,56.13150391)
\curveto(551.13923987,56.08150274)(551.22423979,56.03650279)(551.31424316,55.99650391)
\curveto(551.4142396,55.96650286)(551.51923949,55.93650289)(551.62924316,55.90650391)
\curveto(551.67923933,55.88650294)(551.72423929,55.87650295)(551.76424316,55.87650391)
\curveto(551.8142392,55.88650294)(551.86423915,55.88650294)(551.91424316,55.87650391)
\curveto(551.94423907,55.86650296)(552.00423901,55.85650297)(552.09424316,55.84650391)
\curveto(552.19423882,55.83650299)(552.26923874,55.84150298)(552.31924316,55.86150391)
\curveto(552.35923865,55.87150295)(552.39923861,55.87150295)(552.43924316,55.86150391)
\curveto(552.47923853,55.86150296)(552.51923849,55.87150295)(552.55924316,55.89150391)
\curveto(552.63923837,55.91150291)(552.71923829,55.9265029)(552.79924316,55.93650391)
\curveto(552.87923813,55.95650287)(552.95423806,55.98150284)(553.02424316,56.01150391)
\curveto(553.36423765,56.15150267)(553.63923737,56.34650248)(553.84924316,56.59650391)
\curveto(554.05923695,56.84650198)(554.23423678,57.14150168)(554.37424316,57.48150391)
\curveto(554.42423659,57.60150122)(554.45423656,57.7265011)(554.46424316,57.85650391)
\curveto(554.48423653,57.99650083)(554.5142365,58.13650069)(554.55424316,58.27650391)
}
}
{
\newrgbcolor{curcolor}{0 0 0}
\pscustom[linestyle=none,fillstyle=solid,fillcolor=curcolor]
{
\newpath
\moveto(558.32252441,64.99650391)
\curveto(558.4725224,64.99649383)(558.62252225,64.99149383)(558.77252441,64.98150391)
\curveto(558.92252195,64.98149384)(559.02752185,64.94149388)(559.08752441,64.86150391)
\curveto(559.13752174,64.80149402)(559.16252171,64.71649411)(559.16252441,64.60650391)
\curveto(559.1725217,64.50649432)(559.1775217,64.40149442)(559.17752441,64.29150391)
\lineto(559.17752441,63.42150391)
\curveto(559.1775217,63.34149548)(559.1725217,63.25649557)(559.16252441,63.16650391)
\curveto(559.16252171,63.08649574)(559.1725217,63.01649581)(559.19252441,62.95650391)
\curveto(559.23252164,62.81649601)(559.32252155,62.7264961)(559.46252441,62.68650391)
\curveto(559.51252136,62.67649615)(559.55752132,62.67149615)(559.59752441,62.67150391)
\lineto(559.74752441,62.67150391)
\lineto(560.15252441,62.67150391)
\curveto(560.31252056,62.68149614)(560.42752045,62.67149615)(560.49752441,62.64150391)
\curveto(560.58752029,62.58149624)(560.64752023,62.5214963)(560.67752441,62.46150391)
\curveto(560.69752018,62.4214964)(560.70752017,62.37649645)(560.70752441,62.32650391)
\lineto(560.70752441,62.17650391)
\curveto(560.70752017,62.06649676)(560.70252017,61.96149686)(560.69252441,61.86150391)
\curveto(560.68252019,61.77149705)(560.64752023,61.70149712)(560.58752441,61.65150391)
\curveto(560.52752035,61.60149722)(560.44252043,61.57149725)(560.33252441,61.56150391)
\lineto(560.00252441,61.56150391)
\curveto(559.89252098,61.57149725)(559.78252109,61.57649725)(559.67252441,61.57650391)
\curveto(559.56252131,61.57649725)(559.46752141,61.56149726)(559.38752441,61.53150391)
\curveto(559.31752156,61.50149732)(559.26752161,61.45149737)(559.23752441,61.38150391)
\curveto(559.20752167,61.31149751)(559.18752169,61.2264976)(559.17752441,61.12650391)
\curveto(559.16752171,61.03649779)(559.16252171,60.93649789)(559.16252441,60.82650391)
\curveto(559.1725217,60.7264981)(559.1775217,60.6264982)(559.17752441,60.52650391)
\lineto(559.17752441,57.55650391)
\curveto(559.1775217,57.33650149)(559.1725217,57.10150172)(559.16252441,56.85150391)
\curveto(559.16252171,56.61150221)(559.20752167,56.4265024)(559.29752441,56.29650391)
\curveto(559.34752153,56.21650261)(559.41252146,56.16150266)(559.49252441,56.13150391)
\curveto(559.5725213,56.10150272)(559.66752121,56.07650275)(559.77752441,56.05650391)
\curveto(559.80752107,56.04650278)(559.83752104,56.04150278)(559.86752441,56.04150391)
\curveto(559.90752097,56.05150277)(559.94252093,56.05150277)(559.97252441,56.04150391)
\lineto(560.16752441,56.04150391)
\curveto(560.26752061,56.04150278)(560.35752052,56.03150279)(560.43752441,56.01150391)
\curveto(560.52752035,56.00150282)(560.59252028,55.96650286)(560.63252441,55.90650391)
\curveto(560.65252022,55.87650295)(560.66752021,55.821503)(560.67752441,55.74150391)
\curveto(560.69752018,55.67150315)(560.70752017,55.59650323)(560.70752441,55.51650391)
\curveto(560.71752016,55.43650339)(560.71752016,55.35650347)(560.70752441,55.27650391)
\curveto(560.69752018,55.20650362)(560.6775202,55.15150367)(560.64752441,55.11150391)
\curveto(560.60752027,55.04150378)(560.53252034,54.99150383)(560.42252441,54.96150391)
\curveto(560.34252053,54.94150388)(560.25252062,54.93150389)(560.15252441,54.93150391)
\curveto(560.05252082,54.94150388)(559.96252091,54.94650388)(559.88252441,54.94650391)
\curveto(559.82252105,54.94650388)(559.76252111,54.94150388)(559.70252441,54.93150391)
\curveto(559.64252123,54.93150389)(559.58752129,54.93650389)(559.53752441,54.94650391)
\lineto(559.35752441,54.94650391)
\curveto(559.30752157,54.95650387)(559.25752162,54.96150386)(559.20752441,54.96150391)
\curveto(559.16752171,54.97150385)(559.12252175,54.97650385)(559.07252441,54.97650391)
\curveto(558.872522,55.0265038)(558.69752218,55.08150374)(558.54752441,55.14150391)
\curveto(558.40752247,55.20150362)(558.28752259,55.30650352)(558.18752441,55.45650391)
\curveto(558.04752283,55.65650317)(557.96752291,55.90650292)(557.94752441,56.20650391)
\curveto(557.92752295,56.51650231)(557.91752296,56.84650198)(557.91752441,57.19650391)
\lineto(557.91752441,61.12650391)
\curveto(557.88752299,61.25649757)(557.85752302,61.35149747)(557.82752441,61.41150391)
\curveto(557.80752307,61.47149735)(557.73752314,61.5214973)(557.61752441,61.56150391)
\curveto(557.5775233,61.57149725)(557.53752334,61.57149725)(557.49752441,61.56150391)
\curveto(557.45752342,61.55149727)(557.41752346,61.55649727)(557.37752441,61.57650391)
\lineto(557.13752441,61.57650391)
\curveto(557.00752387,61.57649725)(556.89752398,61.58649724)(556.80752441,61.60650391)
\curveto(556.72752415,61.63649719)(556.6725242,61.69649713)(556.64252441,61.78650391)
\curveto(556.62252425,61.826497)(556.60752427,61.87149695)(556.59752441,61.92150391)
\lineto(556.59752441,62.07150391)
\curveto(556.59752428,62.21149661)(556.60752427,62.3264965)(556.62752441,62.41650391)
\curveto(556.64752423,62.51649631)(556.70752417,62.59149623)(556.80752441,62.64150391)
\curveto(556.91752396,62.68149614)(557.05752382,62.69149613)(557.22752441,62.67150391)
\curveto(557.40752347,62.65149617)(557.55752332,62.66149616)(557.67752441,62.70150391)
\curveto(557.76752311,62.75149607)(557.83752304,62.821496)(557.88752441,62.91150391)
\curveto(557.90752297,62.97149585)(557.91752296,63.04649578)(557.91752441,63.13650391)
\lineto(557.91752441,63.39150391)
\lineto(557.91752441,64.32150391)
\lineto(557.91752441,64.56150391)
\curveto(557.91752296,64.65149417)(557.92752295,64.7264941)(557.94752441,64.78650391)
\curveto(557.98752289,64.86649396)(558.06252281,64.93149389)(558.17252441,64.98150391)
\curveto(558.20252267,64.98149384)(558.22752265,64.98149384)(558.24752441,64.98150391)
\curveto(558.2775226,64.99149383)(558.30252257,64.99649383)(558.32252441,64.99650391)
}
}
{
\newrgbcolor{curcolor}{0 0 0}
\pscustom[linestyle=none,fillstyle=solid,fillcolor=curcolor]
{
\newpath
\moveto(569.21932129,59.13150391)
\curveto(569.23931323,59.07149975)(569.24931322,58.97649985)(569.24932129,58.84650391)
\curveto(569.24931322,58.7265001)(569.24431322,58.64150018)(569.23432129,58.59150391)
\lineto(569.23432129,58.44150391)
\curveto(569.22431324,58.36150046)(569.21431325,58.28650054)(569.20432129,58.21650391)
\curveto(569.20431326,58.15650067)(569.19931327,58.08650074)(569.18932129,58.00650391)
\curveto(569.1693133,57.94650088)(569.15431331,57.88650094)(569.14432129,57.82650391)
\curveto(569.14431332,57.76650106)(569.13431333,57.70650112)(569.11432129,57.64650391)
\curveto(569.07431339,57.51650131)(569.03931343,57.38650144)(569.00932129,57.25650391)
\curveto(568.97931349,57.1265017)(568.93931353,57.00650182)(568.88932129,56.89650391)
\curveto(568.67931379,56.41650241)(568.39931407,56.01150281)(568.04932129,55.68150391)
\curveto(567.69931477,55.36150346)(567.2693152,55.11650371)(566.75932129,54.94650391)
\curveto(566.64931582,54.90650392)(566.52931594,54.87650395)(566.39932129,54.85650391)
\curveto(566.27931619,54.83650399)(566.15431631,54.81650401)(566.02432129,54.79650391)
\curveto(565.9643165,54.78650404)(565.89931657,54.78150404)(565.82932129,54.78150391)
\curveto(565.7693167,54.77150405)(565.70931676,54.76650406)(565.64932129,54.76650391)
\curveto(565.60931686,54.75650407)(565.54931692,54.75150407)(565.46932129,54.75150391)
\curveto(565.39931707,54.75150407)(565.34931712,54.75650407)(565.31932129,54.76650391)
\curveto(565.27931719,54.77650405)(565.23931723,54.78150404)(565.19932129,54.78150391)
\curveto(565.15931731,54.77150405)(565.12431734,54.77150405)(565.09432129,54.78150391)
\lineto(565.00432129,54.78150391)
\lineto(564.64432129,54.82650391)
\curveto(564.50431796,54.86650396)(564.3693181,54.90650392)(564.23932129,54.94650391)
\curveto(564.10931836,54.98650384)(563.98431848,55.03150379)(563.86432129,55.08150391)
\curveto(563.41431905,55.28150354)(563.04431942,55.54150328)(562.75432129,55.86150391)
\curveto(562.46432,56.18150264)(562.22432024,56.57150225)(562.03432129,57.03150391)
\curveto(561.98432048,57.13150169)(561.94432052,57.23150159)(561.91432129,57.33150391)
\curveto(561.89432057,57.43150139)(561.87432059,57.53650129)(561.85432129,57.64650391)
\curveto(561.83432063,57.68650114)(561.82432064,57.71650111)(561.82432129,57.73650391)
\curveto(561.83432063,57.76650106)(561.83432063,57.80150102)(561.82432129,57.84150391)
\curveto(561.80432066,57.9215009)(561.78932068,58.00150082)(561.77932129,58.08150391)
\curveto(561.77932069,58.17150065)(561.7693207,58.25650057)(561.74932129,58.33650391)
\lineto(561.74932129,58.45650391)
\curveto(561.74932072,58.49650033)(561.74432072,58.54150028)(561.73432129,58.59150391)
\curveto(561.72432074,58.64150018)(561.71932075,58.7265001)(561.71932129,58.84650391)
\curveto(561.71932075,58.97649985)(561.72932074,59.07149975)(561.74932129,59.13150391)
\curveto(561.7693207,59.20149962)(561.77432069,59.27149955)(561.76432129,59.34150391)
\curveto(561.75432071,59.41149941)(561.75932071,59.48149934)(561.77932129,59.55150391)
\curveto(561.78932068,59.60149922)(561.79432067,59.64149918)(561.79432129,59.67150391)
\curveto(561.80432066,59.71149911)(561.81432065,59.75649907)(561.82432129,59.80650391)
\curveto(561.85432061,59.9264989)(561.87932059,60.04649878)(561.89932129,60.16650391)
\curveto(561.92932054,60.28649854)(561.9693205,60.40149842)(562.01932129,60.51150391)
\curveto(562.1693203,60.88149794)(562.34932012,61.21149761)(562.55932129,61.50150391)
\curveto(562.77931969,61.80149702)(563.04431942,62.05149677)(563.35432129,62.25150391)
\curveto(563.47431899,62.33149649)(563.59931887,62.39649643)(563.72932129,62.44650391)
\curveto(563.85931861,62.50649632)(563.99431847,62.56649626)(564.13432129,62.62650391)
\curveto(564.25431821,62.67649615)(564.38431808,62.70649612)(564.52432129,62.71650391)
\curveto(564.6643178,62.73649609)(564.80431766,62.76649606)(564.94432129,62.80650391)
\lineto(565.13932129,62.80650391)
\curveto(565.20931726,62.81649601)(565.27431719,62.826496)(565.33432129,62.83650391)
\curveto(566.22431624,62.84649598)(566.9643155,62.66149616)(567.55432129,62.28150391)
\curveto(568.14431432,61.90149692)(568.5693139,61.40649742)(568.82932129,60.79650391)
\curveto(568.87931359,60.69649813)(568.91931355,60.59649823)(568.94932129,60.49650391)
\curveto(568.97931349,60.39649843)(569.01431345,60.29149853)(569.05432129,60.18150391)
\curveto(569.08431338,60.07149875)(569.10931336,59.95149887)(569.12932129,59.82150391)
\curveto(569.14931332,59.70149912)(569.17431329,59.57649925)(569.20432129,59.44650391)
\curveto(569.21431325,59.39649943)(569.21431325,59.34149948)(569.20432129,59.28150391)
\curveto(569.20431326,59.23149959)(569.20931326,59.18149964)(569.21932129,59.13150391)
\moveto(567.88432129,58.27650391)
\curveto(567.90431456,58.34650048)(567.90931456,58.4265004)(567.89932129,58.51650391)
\lineto(567.89932129,58.77150391)
\curveto(567.89931457,59.16149966)(567.8643146,59.49149933)(567.79432129,59.76150391)
\curveto(567.7643147,59.84149898)(567.73931473,59.9214989)(567.71932129,60.00150391)
\curveto(567.69931477,60.08149874)(567.67431479,60.15649867)(567.64432129,60.22650391)
\curveto(567.3643151,60.87649795)(566.91931555,61.3264975)(566.30932129,61.57650391)
\curveto(566.23931623,61.60649722)(566.1643163,61.6264972)(566.08432129,61.63650391)
\lineto(565.84432129,61.69650391)
\curveto(565.7643167,61.71649711)(565.67931679,61.7264971)(565.58932129,61.72650391)
\lineto(565.31932129,61.72650391)
\lineto(565.04932129,61.68150391)
\curveto(564.94931752,61.66149716)(564.85431761,61.63649719)(564.76432129,61.60650391)
\curveto(564.68431778,61.58649724)(564.60431786,61.55649727)(564.52432129,61.51650391)
\curveto(564.45431801,61.49649733)(564.38931808,61.46649736)(564.32932129,61.42650391)
\curveto(564.2693182,61.38649744)(564.21431825,61.34649748)(564.16432129,61.30650391)
\curveto(563.92431854,61.13649769)(563.72931874,60.93149789)(563.57932129,60.69150391)
\curveto(563.42931904,60.45149837)(563.29931917,60.17149865)(563.18932129,59.85150391)
\curveto(563.15931931,59.75149907)(563.13931933,59.64649918)(563.12932129,59.53650391)
\curveto(563.11931935,59.43649939)(563.10431936,59.33149949)(563.08432129,59.22150391)
\curveto(563.07431939,59.18149964)(563.0693194,59.11649971)(563.06932129,59.02650391)
\curveto(563.05931941,58.99649983)(563.05431941,58.96149986)(563.05432129,58.92150391)
\curveto(563.0643194,58.88149994)(563.0693194,58.83649999)(563.06932129,58.78650391)
\lineto(563.06932129,58.48650391)
\curveto(563.0693194,58.38650044)(563.07931939,58.29650053)(563.09932129,58.21650391)
\lineto(563.12932129,58.03650391)
\curveto(563.14931932,57.93650089)(563.1643193,57.83650099)(563.17432129,57.73650391)
\curveto(563.19431927,57.64650118)(563.22431924,57.56150126)(563.26432129,57.48150391)
\curveto(563.3643191,57.24150158)(563.47931899,57.01650181)(563.60932129,56.80650391)
\curveto(563.74931872,56.59650223)(563.91931855,56.4215024)(564.11932129,56.28150391)
\curveto(564.1693183,56.25150257)(564.21431825,56.2265026)(564.25432129,56.20650391)
\curveto(564.29431817,56.18650264)(564.33931813,56.16150266)(564.38932129,56.13150391)
\curveto(564.469318,56.08150274)(564.55431791,56.03650279)(564.64432129,55.99650391)
\curveto(564.74431772,55.96650286)(564.84931762,55.93650289)(564.95932129,55.90650391)
\curveto(565.00931746,55.88650294)(565.05431741,55.87650295)(565.09432129,55.87650391)
\curveto(565.14431732,55.88650294)(565.19431727,55.88650294)(565.24432129,55.87650391)
\curveto(565.27431719,55.86650296)(565.33431713,55.85650297)(565.42432129,55.84650391)
\curveto(565.52431694,55.83650299)(565.59931687,55.84150298)(565.64932129,55.86150391)
\curveto(565.68931678,55.87150295)(565.72931674,55.87150295)(565.76932129,55.86150391)
\curveto(565.80931666,55.86150296)(565.84931662,55.87150295)(565.88932129,55.89150391)
\curveto(565.9693165,55.91150291)(566.04931642,55.9265029)(566.12932129,55.93650391)
\curveto(566.20931626,55.95650287)(566.28431618,55.98150284)(566.35432129,56.01150391)
\curveto(566.69431577,56.15150267)(566.9693155,56.34650248)(567.17932129,56.59650391)
\curveto(567.38931508,56.84650198)(567.5643149,57.14150168)(567.70432129,57.48150391)
\curveto(567.75431471,57.60150122)(567.78431468,57.7265011)(567.79432129,57.85650391)
\curveto(567.81431465,57.99650083)(567.84431462,58.13650069)(567.88432129,58.27650391)
}
}
{
\newrgbcolor{curcolor}{0 0 0}
\pscustom[linestyle=none,fillstyle=solid,fillcolor=curcolor]
{
\newpath
\moveto(577.32260254,62.55150391)
\curveto(577.39259494,62.50149632)(577.4275949,62.4264964)(577.42760254,62.32650391)
\curveto(577.43759489,62.2264966)(577.44259489,62.1214967)(577.44260254,62.01150391)
\lineto(577.44260254,55.74150391)
\lineto(577.44260254,55.14150391)
\curveto(577.42259491,55.09150373)(577.41759491,55.04150378)(577.42760254,54.99150391)
\curveto(577.43759489,54.95150387)(577.4325949,54.90650392)(577.41260254,54.85650391)
\curveto(577.39259494,54.75650407)(577.37759495,54.65650417)(577.36760254,54.55650391)
\curveto(577.36759496,54.44650438)(577.35259498,54.34150448)(577.32260254,54.24150391)
\curveto(577.29259504,54.13150469)(577.26259507,54.0265048)(577.23260254,53.92650391)
\curveto(577.21259512,53.826505)(577.17759515,53.7265051)(577.12760254,53.62650391)
\curveto(577.0275953,53.36650546)(576.89759543,53.13150569)(576.73760254,52.92150391)
\curveto(576.58759574,52.71150611)(576.40759592,52.53650629)(576.19760254,52.39650391)
\curveto(576.0275963,52.27650655)(575.84759648,52.18150664)(575.65760254,52.11150391)
\curveto(575.46759686,52.03150679)(575.26259707,51.95650687)(575.04260254,51.88650391)
\curveto(574.95259738,51.86650696)(574.86259747,51.85650697)(574.77260254,51.85650391)
\curveto(574.68259765,51.84650698)(574.59259774,51.83150699)(574.50260254,51.81150391)
\lineto(574.41260254,51.81150391)
\curveto(574.39259794,51.80150702)(574.37259796,51.79650703)(574.35260254,51.79650391)
\curveto(574.30259803,51.78650704)(574.25259808,51.78650704)(574.20260254,51.79650391)
\curveto(574.16259817,51.80650702)(574.11759821,51.80150702)(574.06760254,51.78150391)
\curveto(573.99759833,51.76150706)(573.88759844,51.75650707)(573.73760254,51.76650391)
\curveto(573.59759873,51.76650706)(573.49759883,51.77650705)(573.43760254,51.79650391)
\curveto(573.40759892,51.79650703)(573.37759895,51.80150702)(573.34760254,51.81150391)
\lineto(573.28760254,51.81150391)
\curveto(573.19759913,51.83150699)(573.10759922,51.84650698)(573.01760254,51.85650391)
\curveto(572.9275994,51.85650697)(572.84259949,51.86650696)(572.76260254,51.88650391)
\curveto(572.68259965,51.90650692)(572.60259973,51.93150689)(572.52260254,51.96150391)
\curveto(572.44259989,51.98150684)(572.36259997,52.00650682)(572.28260254,52.03650391)
\curveto(571.96260037,52.16650666)(571.69260064,52.31150651)(571.47260254,52.47150391)
\curveto(571.26260107,52.63150619)(571.07260126,52.85650597)(570.90260254,53.14650391)
\curveto(570.88260145,53.16650566)(570.86760146,53.19150563)(570.85760254,53.22150391)
\curveto(570.85760147,53.24150558)(570.84760148,53.26650556)(570.82760254,53.29650391)
\curveto(570.79760153,53.37650545)(570.76260157,53.49150533)(570.72260254,53.64150391)
\curveto(570.69260164,53.78150504)(570.72260161,53.88650494)(570.81260254,53.95650391)
\curveto(570.87260146,54.00650482)(570.95260138,54.03150479)(571.05260254,54.03150391)
\lineto(571.38260254,54.03150391)
\lineto(571.54760254,54.03150391)
\curveto(571.60760072,54.03150479)(571.66260067,54.0215048)(571.71260254,54.00150391)
\curveto(571.80260053,53.97150485)(571.86760046,53.9215049)(571.90760254,53.85150391)
\curveto(571.94760038,53.78150504)(571.99260034,53.70650512)(572.04260254,53.62650391)
\lineto(572.16260254,53.44650391)
\curveto(572.21260012,53.37650545)(572.26260007,53.3215055)(572.31260254,53.28150391)
\curveto(572.56259977,53.09150573)(572.86259947,52.95150587)(573.21260254,52.86150391)
\curveto(573.27259906,52.84150598)(573.332599,52.83150599)(573.39260254,52.83150391)
\curveto(573.46259887,52.821506)(573.5275988,52.80650602)(573.58760254,52.78650391)
\lineto(573.67760254,52.78650391)
\curveto(573.74759858,52.76650606)(573.8325985,52.75650607)(573.93260254,52.75650391)
\curveto(574.0325983,52.75650607)(574.12259821,52.76650606)(574.20260254,52.78650391)
\curveto(574.2325981,52.79650603)(574.27259806,52.80150602)(574.32260254,52.80150391)
\curveto(574.42259791,52.821506)(574.51759781,52.84150598)(574.60760254,52.86150391)
\curveto(574.69759763,52.87150595)(574.78259755,52.89650593)(574.86260254,52.93650391)
\curveto(575.15259718,53.05650577)(575.38759694,53.2215056)(575.56760254,53.43150391)
\curveto(575.75759657,53.63150519)(575.91259642,53.87650495)(576.03260254,54.16650391)
\curveto(576.07259626,54.25650457)(576.09759623,54.35150447)(576.10760254,54.45150391)
\curveto(576.1275962,54.55150427)(576.15259618,54.65650417)(576.18260254,54.76650391)
\curveto(576.20259613,54.81650401)(576.21259612,54.86650396)(576.21260254,54.91650391)
\curveto(576.21259612,54.96650386)(576.21759611,55.01650381)(576.22760254,55.06650391)
\curveto(576.23759609,55.09650373)(576.24259609,55.14650368)(576.24260254,55.21650391)
\curveto(576.26259607,55.29650353)(576.26259607,55.38150344)(576.24260254,55.47150391)
\curveto(576.2325961,55.5215033)(576.2275961,55.56650326)(576.22760254,55.60650391)
\curveto(576.23759609,55.64650318)(576.2325961,55.68150314)(576.21260254,55.71150391)
\curveto(576.19259614,55.73150309)(576.17759615,55.74150308)(576.16760254,55.74150391)
\lineto(576.12260254,55.78650391)
\curveto(576.02259631,55.78650304)(575.94759638,55.75650307)(575.89760254,55.69650391)
\curveto(575.85759647,55.64650318)(575.80759652,55.60150322)(575.74760254,55.56150391)
\lineto(575.50760254,55.35150391)
\curveto(575.4275969,55.29150353)(575.33759699,55.23650359)(575.23760254,55.18650391)
\curveto(575.09759723,55.09650373)(574.92259741,55.0215038)(574.71260254,54.96150391)
\curveto(574.50259783,54.91150391)(574.28259805,54.87650395)(574.05260254,54.85650391)
\curveto(573.82259851,54.83650399)(573.59259874,54.84150398)(573.36260254,54.87150391)
\curveto(573.1325992,54.89150393)(572.92259941,54.93150389)(572.73260254,54.99150391)
\curveto(571.79260054,55.30150352)(571.1326012,55.89650293)(570.75260254,56.77650391)
\curveto(570.70260163,56.87650195)(570.66260167,56.97150185)(570.63260254,57.06150391)
\curveto(570.60260173,57.16150166)(570.56760176,57.26650156)(570.52760254,57.37650391)
\curveto(570.50760182,57.4265014)(570.49760183,57.47150135)(570.49760254,57.51150391)
\curveto(570.49760183,57.55150127)(570.48760184,57.59650123)(570.46760254,57.64650391)
\curveto(570.44760188,57.71650111)(570.4326019,57.78650104)(570.42260254,57.85650391)
\curveto(570.42260191,57.93650089)(570.41260192,58.01150081)(570.39260254,58.08150391)
\curveto(570.38260195,58.1215007)(570.37760195,58.15650067)(570.37760254,58.18650391)
\curveto(570.38760194,58.2265006)(570.38760194,58.26650056)(570.37760254,58.30650391)
\curveto(570.37760195,58.34650048)(570.37260196,58.38650044)(570.36260254,58.42650391)
\lineto(570.36260254,58.54650391)
\curveto(570.34260199,58.66650016)(570.34260199,58.79150003)(570.36260254,58.92150391)
\curveto(570.37260196,58.98149984)(570.37760195,59.04149978)(570.37760254,59.10150391)
\lineto(570.37760254,59.26650391)
\curveto(570.38760194,59.31649951)(570.39260194,59.35649947)(570.39260254,59.38650391)
\curveto(570.39260194,59.4264994)(570.39760193,59.47149935)(570.40760254,59.52150391)
\curveto(570.43760189,59.63149919)(570.45760187,59.73649909)(570.46760254,59.83650391)
\curveto(570.47760185,59.94649888)(570.50260183,60.05649877)(570.54260254,60.16650391)
\curveto(570.58260175,60.28649854)(570.61760171,60.40149842)(570.64760254,60.51150391)
\curveto(570.68760164,60.63149819)(570.7326016,60.74649808)(570.78260254,60.85650391)
\curveto(570.85260148,61.01649781)(570.9326014,61.16149766)(571.02260254,61.29150391)
\curveto(571.11260122,61.43149739)(571.20760112,61.56649726)(571.30760254,61.69650391)
\curveto(571.37760095,61.80649702)(571.46760086,61.89649693)(571.57760254,61.96650391)
\lineto(571.63760254,62.02650391)
\lineto(571.69760254,62.08650391)
\lineto(571.84760254,62.20650391)
\lineto(572.02760254,62.32650391)
\curveto(572.15760017,62.40649642)(572.29260004,62.47649635)(572.43260254,62.53650391)
\curveto(572.58259975,62.59649623)(572.74259959,62.65149617)(572.91260254,62.70150391)
\curveto(573.01259932,62.73149609)(573.11259922,62.75149607)(573.21260254,62.76150391)
\curveto(573.32259901,62.77149605)(573.4325989,62.78649604)(573.54260254,62.80650391)
\curveto(573.58259875,62.81649601)(573.6325987,62.81649601)(573.69260254,62.80650391)
\curveto(573.76259857,62.79649603)(573.81259852,62.80149602)(573.84260254,62.82150391)
\curveto(574.16259817,62.83149599)(574.44759788,62.80149602)(574.69760254,62.73150391)
\curveto(574.95759737,62.66149616)(575.18759714,62.56149626)(575.38760254,62.43150391)
\curveto(575.45759687,62.39149643)(575.52259681,62.34649648)(575.58260254,62.29650391)
\lineto(575.76260254,62.14650391)
\curveto(575.81259652,62.10649672)(575.85759647,62.06149676)(575.89760254,62.01150391)
\curveto(575.94759638,61.97149685)(576.02259631,61.95149687)(576.12260254,61.95150391)
\lineto(576.16760254,61.99650391)
\curveto(576.18759614,62.01649681)(576.20759612,62.04149678)(576.22760254,62.07150391)
\curveto(576.25759607,62.15149667)(576.27259606,62.23149659)(576.27260254,62.31150391)
\curveto(576.28259605,62.39149643)(576.31259602,62.46149636)(576.36260254,62.52150391)
\curveto(576.39259594,62.56149626)(576.45259588,62.59149623)(576.54260254,62.61150391)
\curveto(576.6325957,62.64149618)(576.7275956,62.65649617)(576.82760254,62.65650391)
\curveto(576.9275954,62.65649617)(577.02259531,62.64649618)(577.11260254,62.62650391)
\curveto(577.21259512,62.60649622)(577.28259505,62.58149624)(577.32260254,62.55150391)
\moveto(576.19760254,58.77150391)
\curveto(576.20759612,58.81150001)(576.21259612,58.86149996)(576.21260254,58.92150391)
\curveto(576.21259612,58.99149983)(576.20759612,59.04649978)(576.19760254,59.08650391)
\lineto(576.19760254,59.32650391)
\curveto(576.17759615,59.41649941)(576.16259617,59.50149932)(576.15260254,59.58150391)
\curveto(576.14259619,59.67149915)(576.1275962,59.75649907)(576.10760254,59.83650391)
\curveto(576.08759624,59.91649891)(576.06759626,59.99149883)(576.04760254,60.06150391)
\curveto(576.03759629,60.14149868)(576.01759631,60.21649861)(575.98760254,60.28650391)
\curveto(575.87759645,60.56649826)(575.7325966,60.81649801)(575.55260254,61.03650391)
\curveto(575.38259695,61.25649757)(575.16259717,61.4214974)(574.89260254,61.53150391)
\curveto(574.81259752,61.57149725)(574.7275976,61.60149722)(574.63760254,61.62150391)
\curveto(574.54759778,61.65149717)(574.45259788,61.67649715)(574.35260254,61.69650391)
\curveto(574.27259806,61.71649711)(574.18259815,61.7214971)(574.08260254,61.71150391)
\lineto(573.81260254,61.71150391)
\curveto(573.76259857,61.70149712)(573.71259862,61.69649713)(573.66260254,61.69650391)
\curveto(573.62259871,61.69649713)(573.57759875,61.69149713)(573.52760254,61.68150391)
\curveto(573.33759899,61.63149719)(573.17759915,61.58149724)(573.04760254,61.53150391)
\curveto(572.70759962,61.39149743)(572.44259989,61.18149764)(572.25260254,60.90150391)
\curveto(572.06260027,60.6214982)(571.91260042,60.29649853)(571.80260254,59.92650391)
\curveto(571.78260055,59.84649898)(571.76760056,59.76649906)(571.75760254,59.68650391)
\curveto(571.75760057,59.61649921)(571.74760058,59.54149928)(571.72760254,59.46150391)
\curveto(571.70760062,59.43149939)(571.69760063,59.39649943)(571.69760254,59.35650391)
\curveto(571.70760062,59.31649951)(571.70760062,59.28149954)(571.69760254,59.25150391)
\lineto(571.69760254,58.92150391)
\lineto(571.69760254,58.57650391)
\curveto(571.69760063,58.46650036)(571.70760062,58.36150046)(571.72760254,58.26150391)
\lineto(571.72760254,58.18650391)
\curveto(571.73760059,58.15650067)(571.74260059,58.13150069)(571.74260254,58.11150391)
\curveto(571.76260057,58.0215008)(571.77760055,57.93150089)(571.78760254,57.84150391)
\curveto(571.80760052,57.75150107)(571.8326005,57.66650116)(571.86260254,57.58650391)
\curveto(571.94260039,57.3265015)(572.04260029,57.08650174)(572.16260254,56.86650391)
\curveto(572.28260005,56.64650218)(572.44259989,56.46650236)(572.64260254,56.32650391)
\lineto(572.76260254,56.23650391)
\curveto(572.80259953,56.21650261)(572.84759948,56.19650263)(572.89760254,56.17650391)
\curveto(572.97759935,56.1265027)(573.06259927,56.08650274)(573.15260254,56.05650391)
\curveto(573.24259909,56.0265028)(573.34259899,55.99650283)(573.45260254,55.96650391)
\curveto(573.50259883,55.95650287)(573.54759878,55.95150287)(573.58760254,55.95150391)
\curveto(573.63759869,55.96150286)(573.68759864,55.95650287)(573.73760254,55.93650391)
\curveto(573.76759856,55.9265029)(573.81759851,55.9215029)(573.88760254,55.92150391)
\curveto(573.95759837,55.9215029)(574.00759832,55.9265029)(574.03760254,55.93650391)
\curveto(574.06759826,55.94650288)(574.09759823,55.94650288)(574.12760254,55.93650391)
\curveto(574.16759816,55.93650289)(574.20759812,55.94150288)(574.24760254,55.95150391)
\curveto(574.33759799,55.97150285)(574.42259791,55.99150283)(574.50260254,56.01150391)
\curveto(574.58259775,56.03150279)(574.66259767,56.05650277)(574.74260254,56.08650391)
\curveto(575.08259725,56.23650259)(575.35259698,56.44650238)(575.55260254,56.71650391)
\curveto(575.75259658,56.98650184)(575.91259642,57.30150152)(576.03260254,57.66150391)
\curveto(576.06259627,57.75150107)(576.08259625,57.84150098)(576.09260254,57.93150391)
\curveto(576.11259622,58.03150079)(576.1325962,58.1265007)(576.15260254,58.21650391)
\curveto(576.16259617,58.25650057)(576.16759616,58.29150053)(576.16760254,58.32150391)
\curveto(576.16759616,58.36150046)(576.17259616,58.40150042)(576.18260254,58.44150391)
\curveto(576.20259613,58.49150033)(576.20259613,58.54150028)(576.18260254,58.59150391)
\curveto(576.17259616,58.65150017)(576.17759615,58.71150011)(576.19760254,58.77150391)
}
}
{
\newrgbcolor{curcolor}{0 0 0}
\pscustom[linestyle=none,fillstyle=solid,fillcolor=curcolor]
{
\newpath
\moveto(582.96588379,62.83650391)
\curveto(583.195879,62.83649599)(583.32587887,62.77649605)(583.35588379,62.65650391)
\curveto(583.38587881,62.54649628)(583.40087879,62.38149644)(583.40088379,62.16150391)
\lineto(583.40088379,61.87650391)
\curveto(583.40087879,61.78649704)(583.37587882,61.71149711)(583.32588379,61.65150391)
\curveto(583.26587893,61.57149725)(583.18087901,61.5264973)(583.07088379,61.51650391)
\curveto(582.96087923,61.51649731)(582.85087934,61.50149732)(582.74088379,61.47150391)
\curveto(582.60087959,61.44149738)(582.46587973,61.41149741)(582.33588379,61.38150391)
\curveto(582.21587998,61.35149747)(582.10088009,61.31149751)(581.99088379,61.26150391)
\curveto(581.70088049,61.13149769)(581.46588073,60.95149787)(581.28588379,60.72150391)
\curveto(581.10588109,60.50149832)(580.95088124,60.24649858)(580.82088379,59.95650391)
\curveto(580.78088141,59.84649898)(580.75088144,59.73149909)(580.73088379,59.61150391)
\curveto(580.71088148,59.50149932)(580.68588151,59.38649944)(580.65588379,59.26650391)
\curveto(580.64588155,59.21649961)(580.64088155,59.16649966)(580.64088379,59.11650391)
\curveto(580.65088154,59.06649976)(580.65088154,59.01649981)(580.64088379,58.96650391)
\curveto(580.61088158,58.84649998)(580.5958816,58.70650012)(580.59588379,58.54650391)
\curveto(580.60588159,58.39650043)(580.61088158,58.25150057)(580.61088379,58.11150391)
\lineto(580.61088379,56.26650391)
\lineto(580.61088379,55.92150391)
\curveto(580.61088158,55.80150302)(580.60588159,55.68650314)(580.59588379,55.57650391)
\curveto(580.58588161,55.46650336)(580.58088161,55.37150345)(580.58088379,55.29150391)
\curveto(580.5908816,55.21150361)(580.57088162,55.14150368)(580.52088379,55.08150391)
\curveto(580.47088172,55.01150381)(580.3908818,54.97150385)(580.28088379,54.96150391)
\curveto(580.18088201,54.95150387)(580.07088212,54.94650388)(579.95088379,54.94650391)
\lineto(579.68088379,54.94650391)
\curveto(579.63088256,54.96650386)(579.58088261,54.98150384)(579.53088379,54.99150391)
\curveto(579.4908827,55.01150381)(579.46088273,55.03650379)(579.44088379,55.06650391)
\curveto(579.3908828,55.13650369)(579.36088283,55.2215036)(579.35088379,55.32150391)
\lineto(579.35088379,55.65150391)
\lineto(579.35088379,56.80650391)
\lineto(579.35088379,60.96150391)
\lineto(579.35088379,61.99650391)
\lineto(579.35088379,62.29650391)
\curveto(579.36088283,62.39649643)(579.3908828,62.48149634)(579.44088379,62.55150391)
\curveto(579.47088272,62.59149623)(579.52088267,62.6214962)(579.59088379,62.64150391)
\curveto(579.67088252,62.66149616)(579.75588244,62.67149615)(579.84588379,62.67150391)
\curveto(579.93588226,62.68149614)(580.02588217,62.68149614)(580.11588379,62.67150391)
\curveto(580.20588199,62.66149616)(580.27588192,62.64649618)(580.32588379,62.62650391)
\curveto(580.40588179,62.59649623)(580.45588174,62.53649629)(580.47588379,62.44650391)
\curveto(580.50588169,62.36649646)(580.52088167,62.27649655)(580.52088379,62.17650391)
\lineto(580.52088379,61.87650391)
\curveto(580.52088167,61.77649705)(580.54088165,61.68649714)(580.58088379,61.60650391)
\curveto(580.5908816,61.58649724)(580.60088159,61.57149725)(580.61088379,61.56150391)
\lineto(580.65588379,61.51650391)
\curveto(580.76588143,61.51649731)(580.85588134,61.56149726)(580.92588379,61.65150391)
\curveto(580.9958812,61.75149707)(581.05588114,61.83149699)(581.10588379,61.89150391)
\lineto(581.19588379,61.98150391)
\curveto(581.28588091,62.09149673)(581.41088078,62.20649662)(581.57088379,62.32650391)
\curveto(581.73088046,62.44649638)(581.88088031,62.53649629)(582.02088379,62.59650391)
\curveto(582.11088008,62.64649618)(582.20587999,62.68149614)(582.30588379,62.70150391)
\curveto(582.40587979,62.73149609)(582.51087968,62.76149606)(582.62088379,62.79150391)
\curveto(582.68087951,62.80149602)(582.74087945,62.80649602)(582.80088379,62.80650391)
\curveto(582.86087933,62.81649601)(582.91587928,62.826496)(582.96588379,62.83650391)
}
}
{
\newrgbcolor{curcolor}{0 0 0}
\pscustom[linestyle=none,fillstyle=solid,fillcolor=curcolor]
{
\newpath
\moveto(591.21564941,55.48650391)
\curveto(591.24564158,55.3265035)(591.2306416,55.19150363)(591.17064941,55.08150391)
\curveto(591.11064172,54.98150384)(591.0306418,54.90650392)(590.93064941,54.85650391)
\curveto(590.88064195,54.83650399)(590.825642,54.826504)(590.76564941,54.82650391)
\curveto(590.71564211,54.826504)(590.66064217,54.81650401)(590.60064941,54.79650391)
\curveto(590.38064245,54.74650408)(590.16064267,54.76150406)(589.94064941,54.84150391)
\curveto(589.7306431,54.91150391)(589.58564324,55.00150382)(589.50564941,55.11150391)
\curveto(589.45564337,55.18150364)(589.41064342,55.26150356)(589.37064941,55.35150391)
\curveto(589.3306435,55.45150337)(589.28064355,55.53150329)(589.22064941,55.59150391)
\curveto(589.20064363,55.61150321)(589.17564365,55.63150319)(589.14564941,55.65150391)
\curveto(589.1256437,55.67150315)(589.09564373,55.67650315)(589.05564941,55.66650391)
\curveto(588.94564388,55.63650319)(588.84064399,55.58150324)(588.74064941,55.50150391)
\curveto(588.65064418,55.4215034)(588.56064427,55.35150347)(588.47064941,55.29150391)
\curveto(588.34064449,55.21150361)(588.20064463,55.13650369)(588.05064941,55.06650391)
\curveto(587.90064493,55.00650382)(587.74064509,54.95150387)(587.57064941,54.90150391)
\curveto(587.47064536,54.87150395)(587.36064547,54.85150397)(587.24064941,54.84150391)
\curveto(587.1306457,54.83150399)(587.02064581,54.81650401)(586.91064941,54.79650391)
\curveto(586.86064597,54.78650404)(586.81564601,54.78150404)(586.77564941,54.78150391)
\lineto(586.67064941,54.78150391)
\curveto(586.56064627,54.76150406)(586.45564637,54.76150406)(586.35564941,54.78150391)
\lineto(586.22064941,54.78150391)
\curveto(586.17064666,54.79150403)(586.12064671,54.79650403)(586.07064941,54.79650391)
\curveto(586.02064681,54.79650403)(585.97564685,54.80650402)(585.93564941,54.82650391)
\curveto(585.89564693,54.83650399)(585.86064697,54.84150398)(585.83064941,54.84150391)
\curveto(585.81064702,54.83150399)(585.78564704,54.83150399)(585.75564941,54.84150391)
\lineto(585.51564941,54.90150391)
\curveto(585.43564739,54.91150391)(585.36064747,54.93150389)(585.29064941,54.96150391)
\curveto(584.99064784,55.09150373)(584.74564808,55.23650359)(584.55564941,55.39650391)
\curveto(584.37564845,55.56650326)(584.2256486,55.80150302)(584.10564941,56.10150391)
\curveto(584.01564881,56.3215025)(583.97064886,56.58650224)(583.97064941,56.89650391)
\lineto(583.97064941,57.21150391)
\curveto(583.98064885,57.26150156)(583.98564884,57.31150151)(583.98564941,57.36150391)
\lineto(584.01564941,57.54150391)
\lineto(584.13564941,57.87150391)
\curveto(584.17564865,57.98150084)(584.2256486,58.08150074)(584.28564941,58.17150391)
\curveto(584.46564836,58.46150036)(584.71064812,58.67650015)(585.02064941,58.81650391)
\curveto(585.3306475,58.95649987)(585.67064716,59.08149974)(586.04064941,59.19150391)
\curveto(586.18064665,59.23149959)(586.3256465,59.26149956)(586.47564941,59.28150391)
\curveto(586.6256462,59.30149952)(586.77564605,59.3264995)(586.92564941,59.35650391)
\curveto(586.99564583,59.37649945)(587.06064577,59.38649944)(587.12064941,59.38650391)
\curveto(587.19064564,59.38649944)(587.26564556,59.39649943)(587.34564941,59.41650391)
\curveto(587.41564541,59.43649939)(587.48564534,59.44649938)(587.55564941,59.44650391)
\curveto(587.6256452,59.45649937)(587.70064513,59.47149935)(587.78064941,59.49150391)
\curveto(588.0306448,59.55149927)(588.26564456,59.60149922)(588.48564941,59.64150391)
\curveto(588.70564412,59.69149913)(588.88064395,59.80649902)(589.01064941,59.98650391)
\curveto(589.07064376,60.06649876)(589.12064371,60.16649866)(589.16064941,60.28650391)
\curveto(589.20064363,60.41649841)(589.20064363,60.55649827)(589.16064941,60.70650391)
\curveto(589.10064373,60.94649788)(589.01064382,61.13649769)(588.89064941,61.27650391)
\curveto(588.78064405,61.41649741)(588.62064421,61.5264973)(588.41064941,61.60650391)
\curveto(588.29064454,61.65649717)(588.14564468,61.69149713)(587.97564941,61.71150391)
\curveto(587.81564501,61.73149709)(587.64564518,61.74149708)(587.46564941,61.74150391)
\curveto(587.28564554,61.74149708)(587.11064572,61.73149709)(586.94064941,61.71150391)
\curveto(586.77064606,61.69149713)(586.6256462,61.66149716)(586.50564941,61.62150391)
\curveto(586.33564649,61.56149726)(586.17064666,61.47649735)(586.01064941,61.36650391)
\curveto(585.9306469,61.30649752)(585.85564697,61.2264976)(585.78564941,61.12650391)
\curveto(585.7256471,61.03649779)(585.67064716,60.93649789)(585.62064941,60.82650391)
\curveto(585.59064724,60.74649808)(585.56064727,60.66149816)(585.53064941,60.57150391)
\curveto(585.51064732,60.48149834)(585.46564736,60.41149841)(585.39564941,60.36150391)
\curveto(585.35564747,60.33149849)(585.28564754,60.30649852)(585.18564941,60.28650391)
\curveto(585.09564773,60.27649855)(585.00064783,60.27149855)(584.90064941,60.27150391)
\curveto(584.80064803,60.27149855)(584.70064813,60.27649855)(584.60064941,60.28650391)
\curveto(584.51064832,60.30649852)(584.44564838,60.33149849)(584.40564941,60.36150391)
\curveto(584.36564846,60.39149843)(584.33564849,60.44149838)(584.31564941,60.51150391)
\curveto(584.29564853,60.58149824)(584.29564853,60.65649817)(584.31564941,60.73650391)
\curveto(584.34564848,60.86649796)(584.37564845,60.98649784)(584.40564941,61.09650391)
\curveto(584.44564838,61.21649761)(584.49064834,61.33149749)(584.54064941,61.44150391)
\curveto(584.7306481,61.79149703)(584.97064786,62.06149676)(585.26064941,62.25150391)
\curveto(585.55064728,62.45149637)(585.91064692,62.61149621)(586.34064941,62.73150391)
\curveto(586.44064639,62.75149607)(586.54064629,62.76649606)(586.64064941,62.77650391)
\curveto(586.75064608,62.78649604)(586.86064597,62.80149602)(586.97064941,62.82150391)
\curveto(587.01064582,62.83149599)(587.07564575,62.83149599)(587.16564941,62.82150391)
\curveto(587.25564557,62.821496)(587.31064552,62.83149599)(587.33064941,62.85150391)
\curveto(588.0306448,62.86149596)(588.64064419,62.78149604)(589.16064941,62.61150391)
\curveto(589.68064315,62.44149638)(590.04564278,62.11649671)(590.25564941,61.63650391)
\curveto(590.34564248,61.43649739)(590.39564243,61.20149762)(590.40564941,60.93150391)
\curveto(590.4256424,60.67149815)(590.43564239,60.39649843)(590.43564941,60.10650391)
\lineto(590.43564941,56.79150391)
\curveto(590.43564239,56.65150217)(590.44064239,56.51650231)(590.45064941,56.38650391)
\curveto(590.46064237,56.25650257)(590.49064234,56.15150267)(590.54064941,56.07150391)
\curveto(590.59064224,56.00150282)(590.65564217,55.95150287)(590.73564941,55.92150391)
\curveto(590.825642,55.88150294)(590.91064192,55.85150297)(590.99064941,55.83150391)
\curveto(591.07064176,55.821503)(591.1306417,55.77650305)(591.17064941,55.69650391)
\curveto(591.19064164,55.66650316)(591.20064163,55.63650319)(591.20064941,55.60650391)
\curveto(591.20064163,55.57650325)(591.20564162,55.53650329)(591.21564941,55.48650391)
\moveto(589.07064941,57.15150391)
\curveto(589.1306437,57.29150153)(589.16064367,57.45150137)(589.16064941,57.63150391)
\curveto(589.17064366,57.821501)(589.17564365,58.01650081)(589.17564941,58.21650391)
\curveto(589.17564365,58.3265005)(589.17064366,58.4265004)(589.16064941,58.51650391)
\curveto(589.15064368,58.60650022)(589.11064372,58.67650015)(589.04064941,58.72650391)
\curveto(589.01064382,58.74650008)(588.94064389,58.75650007)(588.83064941,58.75650391)
\curveto(588.81064402,58.73650009)(588.77564405,58.7265001)(588.72564941,58.72650391)
\curveto(588.67564415,58.7265001)(588.6306442,58.71650011)(588.59064941,58.69650391)
\curveto(588.51064432,58.67650015)(588.42064441,58.65650017)(588.32064941,58.63650391)
\lineto(588.02064941,58.57650391)
\curveto(587.99064484,58.57650025)(587.95564487,58.57150025)(587.91564941,58.56150391)
\lineto(587.81064941,58.56150391)
\curveto(587.66064517,58.5215003)(587.49564533,58.49650033)(587.31564941,58.48650391)
\curveto(587.14564568,58.48650034)(586.98564584,58.46650036)(586.83564941,58.42650391)
\curveto(586.75564607,58.40650042)(586.68064615,58.38650044)(586.61064941,58.36650391)
\curveto(586.55064628,58.35650047)(586.48064635,58.34150048)(586.40064941,58.32150391)
\curveto(586.24064659,58.27150055)(586.09064674,58.20650062)(585.95064941,58.12650391)
\curveto(585.81064702,58.05650077)(585.69064714,57.96650086)(585.59064941,57.85650391)
\curveto(585.49064734,57.74650108)(585.41564741,57.61150121)(585.36564941,57.45150391)
\curveto(585.31564751,57.30150152)(585.29564753,57.11650171)(585.30564941,56.89650391)
\curveto(585.30564752,56.79650203)(585.32064751,56.70150212)(585.35064941,56.61150391)
\curveto(585.39064744,56.53150229)(585.43564739,56.45650237)(585.48564941,56.38650391)
\curveto(585.56564726,56.27650255)(585.67064716,56.18150264)(585.80064941,56.10150391)
\curveto(585.9306469,56.03150279)(586.07064676,55.97150285)(586.22064941,55.92150391)
\curveto(586.27064656,55.91150291)(586.32064651,55.90650292)(586.37064941,55.90650391)
\curveto(586.42064641,55.90650292)(586.47064636,55.90150292)(586.52064941,55.89150391)
\curveto(586.59064624,55.87150295)(586.67564615,55.85650297)(586.77564941,55.84650391)
\curveto(586.88564594,55.84650298)(586.97564585,55.85650297)(587.04564941,55.87650391)
\curveto(587.10564572,55.89650293)(587.16564566,55.90150292)(587.22564941,55.89150391)
\curveto(587.28564554,55.89150293)(587.34564548,55.90150292)(587.40564941,55.92150391)
\curveto(587.48564534,55.94150288)(587.56064527,55.95650287)(587.63064941,55.96650391)
\curveto(587.71064512,55.97650285)(587.78564504,55.99650283)(587.85564941,56.02650391)
\curveto(588.14564468,56.14650268)(588.39064444,56.29150253)(588.59064941,56.46150391)
\curveto(588.80064403,56.63150219)(588.96064387,56.86150196)(589.07064941,57.15150391)
}
}
{
\newrgbcolor{curcolor}{0 0 0}
\pscustom[linestyle=none,fillstyle=solid,fillcolor=curcolor]
{
\newpath
\moveto(594.83229004,65.73150391)
\curveto(595.0122865,65.74149308)(595.20228631,65.74149308)(595.40229004,65.73150391)
\curveto(595.60228591,65.7214931)(595.74228577,65.66149316)(595.82229004,65.55150391)
\curveto(595.86228565,65.49149333)(595.88728562,65.41649341)(595.89729004,65.32650391)
\curveto(595.9072856,65.24649358)(595.9122856,65.15649367)(595.91229004,65.05650391)
\curveto(595.9122856,64.9264939)(595.88728562,64.821494)(595.83729004,64.74150391)
\curveto(595.79728571,64.69149413)(595.73728577,64.65649417)(595.65729004,64.63650391)
\curveto(595.58728592,64.6264942)(595.507286,64.6214942)(595.41729004,64.62150391)
\lineto(595.13229004,64.62150391)
\curveto(595.04228647,64.63149419)(594.96228655,64.63149419)(594.89229004,64.62150391)
\curveto(594.6122869,64.54149428)(594.42728708,64.41149441)(594.33729004,64.23150391)
\curveto(594.25728725,64.06149476)(594.21728729,63.80149502)(594.21729004,63.45150391)
\curveto(594.21728729,63.38149544)(594.2122873,63.30649552)(594.20229004,63.22650391)
\curveto(594.19228732,63.15649567)(594.19728731,63.09149573)(594.21729004,63.03150391)
\curveto(594.24728726,62.88149594)(594.3122872,62.77649605)(594.41229004,62.71650391)
\curveto(594.49228702,62.68649614)(594.59228692,62.67149615)(594.71229004,62.67150391)
\lineto(595.07229004,62.67150391)
\lineto(595.29729004,62.67150391)
\curveto(595.32728618,62.65149617)(595.35728615,62.64649618)(595.38729004,62.65650391)
\curveto(595.41728609,62.66649616)(595.44728606,62.66149616)(595.47729004,62.64150391)
\curveto(595.57728593,62.61149621)(595.64228587,62.55149627)(595.67229004,62.46150391)
\curveto(595.70228581,62.38149644)(595.71728579,62.27649655)(595.71729004,62.14650391)
\curveto(595.7072858,62.10649672)(595.70228581,62.06649676)(595.70229004,62.02650391)
\lineto(595.70229004,61.90650391)
\curveto(595.67228584,61.75649707)(595.6072859,61.65649717)(595.50729004,61.60650391)
\curveto(595.37728613,61.55649727)(595.2072863,61.54149728)(594.99729004,61.56150391)
\curveto(594.79728671,61.59149723)(594.62728688,61.58649724)(594.48729004,61.54650391)
\curveto(594.4072871,61.5264973)(594.34728716,61.48649734)(594.30729004,61.42650391)
\curveto(594.26728724,61.37649745)(594.23728727,61.30649752)(594.21729004,61.21650391)
\curveto(594.19728731,61.14649768)(594.19228732,61.06649776)(594.20229004,60.97650391)
\curveto(594.2122873,60.88649794)(594.21728729,60.80149802)(594.21729004,60.72150391)
\lineto(594.21729004,59.73150391)
\lineto(594.21729004,56.55150391)
\lineto(594.21729004,55.80150391)
\lineto(594.21729004,55.60650391)
\curveto(594.22728728,55.53650329)(594.22228729,55.47650335)(594.20229004,55.42650391)
\lineto(594.20229004,55.30650391)
\lineto(594.17229004,55.18650391)
\curveto(594.16228735,55.14650368)(594.14728736,55.11150371)(594.12729004,55.08150391)
\curveto(594.07728743,55.01150381)(594.00228751,54.97150385)(593.90229004,54.96150391)
\curveto(593.80228771,54.95150387)(593.69228782,54.94650388)(593.57229004,54.94650391)
\lineto(593.28729004,54.94650391)
\curveto(593.23728827,54.96650386)(593.18728832,54.98150384)(593.13729004,54.99150391)
\curveto(593.09728841,55.01150381)(593.06228845,55.04650378)(593.03229004,55.09650391)
\curveto(593.0122885,55.1265037)(592.99228852,55.19150363)(592.97229004,55.29150391)
\lineto(592.97229004,55.39650391)
\curveto(592.95228856,55.44650338)(592.94228857,55.49650333)(592.94229004,55.54650391)
\curveto(592.95228856,55.60650322)(592.95728855,55.66150316)(592.95729004,55.71150391)
\lineto(592.95729004,56.31150391)
\lineto(592.95729004,60.40650391)
\lineto(592.95729004,60.75150391)
\curveto(592.96728854,60.87149795)(592.96728854,60.98149784)(592.95729004,61.08150391)
\curveto(592.95728855,61.19149763)(592.93728857,61.28649754)(592.89729004,61.36650391)
\curveto(592.86728864,61.44649738)(592.8122887,61.50149732)(592.73229004,61.53150391)
\curveto(592.67228884,61.56149726)(592.60228891,61.57649725)(592.52229004,61.57650391)
\lineto(592.29729004,61.57650391)
\lineto(592.05729004,61.57650391)
\curveto(591.98728952,61.57649725)(591.92228959,61.58649724)(591.86229004,61.60650391)
\curveto(591.77228974,61.64649718)(591.7072898,61.73149709)(591.66729004,61.86150391)
\curveto(591.65728985,61.91149691)(591.65228986,61.95649687)(591.65229004,61.99650391)
\lineto(591.65229004,62.13150391)
\curveto(591.65228986,62.27149655)(591.66728984,62.38149644)(591.69729004,62.46150391)
\curveto(591.72728978,62.55149627)(591.79228972,62.61149621)(591.89229004,62.64150391)
\curveto(591.96228955,62.67149615)(592.04228947,62.68149614)(592.13229004,62.67150391)
\lineto(592.41729004,62.67150391)
\curveto(592.51728899,62.67149615)(592.60228891,62.68149614)(592.67229004,62.70150391)
\curveto(592.75228876,62.7214961)(592.81728869,62.76149606)(592.86729004,62.82150391)
\curveto(592.93728857,62.90149592)(592.96728854,63.0264958)(592.95729004,63.19650391)
\lineto(592.95729004,63.67650391)
\curveto(592.95728855,63.87649495)(592.96728854,64.06149476)(592.98729004,64.23150391)
\curveto(593.01728849,64.41149441)(593.06228845,64.57149425)(593.12229004,64.71150391)
\curveto(593.23228828,64.95149387)(593.37728813,65.14649368)(593.55729004,65.29650391)
\curveto(593.74728776,65.44649338)(593.97228754,65.56149326)(594.23229004,65.64150391)
\curveto(594.29228722,65.66149316)(594.35228716,65.67149315)(594.41229004,65.67150391)
\curveto(594.48228703,65.68149314)(594.55228696,65.69649313)(594.62229004,65.71650391)
\curveto(594.64228687,65.7264931)(594.67728683,65.7264931)(594.72729004,65.71650391)
\curveto(594.77728673,65.71649311)(594.8122867,65.7214931)(594.83229004,65.73150391)
\moveto(597.09729004,64.15650391)
\curveto(597.16728434,64.10649472)(597.25228426,64.08149474)(597.35229004,64.08150391)
\lineto(597.66729004,64.08150391)
\lineto(597.83229004,64.08150391)
\curveto(597.89228362,64.08149474)(597.94728356,64.09149473)(597.99729004,64.11150391)
\curveto(598.12728338,64.16149466)(598.19228332,64.26649456)(598.19229004,64.42650391)
\curveto(598.20228331,64.58649424)(598.2072833,64.75649407)(598.20729004,64.93650391)
\lineto(598.20729004,65.19150391)
\curveto(598.2072833,65.28149354)(598.19228332,65.35649347)(598.16229004,65.41650391)
\curveto(598.1122834,65.5264933)(598.0122835,65.58649324)(597.86229004,65.59650391)
\curveto(597.7122838,65.60649322)(597.55228396,65.61149321)(597.38229004,65.61150391)
\curveto(597.36228415,65.60149322)(597.33728417,65.59649323)(597.30729004,65.59650391)
\curveto(597.28728422,65.60649322)(597.26728424,65.60649322)(597.24729004,65.59650391)
\curveto(597.12728438,65.55649327)(597.04728446,65.49649333)(597.00729004,65.41650391)
\curveto(596.97728453,65.35649347)(596.96228455,65.28149354)(596.96229004,65.19150391)
\lineto(596.96229004,64.93650391)
\lineto(596.96229004,64.47150391)
\curveto(596.97228454,64.3214945)(597.01728449,64.21649461)(597.09729004,64.15650391)
\moveto(598.20729004,61.99650391)
\lineto(598.20729004,62.28150391)
\curveto(598.2072833,62.38149644)(598.18228333,62.46149636)(598.13229004,62.52150391)
\curveto(598.08228343,62.60149622)(597.98728352,62.64149618)(597.84729004,62.64150391)
\curveto(597.71728379,62.65149617)(597.58728392,62.65649617)(597.45729004,62.65650391)
\curveto(597.43728407,62.64649618)(597.4122841,62.64149618)(597.38229004,62.64150391)
\curveto(597.36228415,62.65149617)(597.34228417,62.65649617)(597.32229004,62.65650391)
\curveto(597.26228425,62.63649619)(597.2072843,62.6214962)(597.15729004,62.61150391)
\curveto(597.1072844,62.60149622)(597.06728444,62.57149625)(597.03729004,62.52150391)
\curveto(596.98728452,62.46149636)(596.96228455,62.37649645)(596.96229004,62.26650391)
\lineto(596.96229004,61.95150391)
\lineto(596.96229004,55.60650391)
\lineto(596.96229004,55.32150391)
\curveto(596.96228455,55.23150359)(596.98228453,55.15650367)(597.02229004,55.09650391)
\curveto(597.07228444,55.01650381)(597.14228437,54.96650386)(597.23229004,54.94650391)
\curveto(597.33228418,54.93650389)(597.44728406,54.93150389)(597.57729004,54.93150391)
\lineto(597.80229004,54.93150391)
\curveto(597.88228363,54.95150387)(597.95228356,54.96650386)(598.01229004,54.97650391)
\curveto(598.07228344,54.99650383)(598.11728339,55.03650379)(598.14729004,55.09650391)
\curveto(598.19728331,55.15650367)(598.21728329,55.23650359)(598.20729004,55.33650391)
\lineto(598.20729004,55.65150391)
\lineto(598.20729004,61.99650391)
}
}
{
\newrgbcolor{curcolor}{0 0 0}
\pscustom[linestyle=none,fillstyle=solid,fillcolor=curcolor]
{
\newpath
\moveto(607.03596191,55.48650391)
\curveto(607.06595408,55.3265035)(607.0509541,55.19150363)(606.99096191,55.08150391)
\curveto(606.93095422,54.98150384)(606.8509543,54.90650392)(606.75096191,54.85650391)
\curveto(606.70095445,54.83650399)(606.6459545,54.826504)(606.58596191,54.82650391)
\curveto(606.53595461,54.826504)(606.48095467,54.81650401)(606.42096191,54.79650391)
\curveto(606.20095495,54.74650408)(605.98095517,54.76150406)(605.76096191,54.84150391)
\curveto(605.5509556,54.91150391)(605.40595574,55.00150382)(605.32596191,55.11150391)
\curveto(605.27595587,55.18150364)(605.23095592,55.26150356)(605.19096191,55.35150391)
\curveto(605.150956,55.45150337)(605.10095605,55.53150329)(605.04096191,55.59150391)
\curveto(605.02095613,55.61150321)(604.99595615,55.63150319)(604.96596191,55.65150391)
\curveto(604.9459562,55.67150315)(604.91595623,55.67650315)(604.87596191,55.66650391)
\curveto(604.76595638,55.63650319)(604.66095649,55.58150324)(604.56096191,55.50150391)
\curveto(604.47095668,55.4215034)(604.38095677,55.35150347)(604.29096191,55.29150391)
\curveto(604.16095699,55.21150361)(604.02095713,55.13650369)(603.87096191,55.06650391)
\curveto(603.72095743,55.00650382)(603.56095759,54.95150387)(603.39096191,54.90150391)
\curveto(603.29095786,54.87150395)(603.18095797,54.85150397)(603.06096191,54.84150391)
\curveto(602.9509582,54.83150399)(602.84095831,54.81650401)(602.73096191,54.79650391)
\curveto(602.68095847,54.78650404)(602.63595851,54.78150404)(602.59596191,54.78150391)
\lineto(602.49096191,54.78150391)
\curveto(602.38095877,54.76150406)(602.27595887,54.76150406)(602.17596191,54.78150391)
\lineto(602.04096191,54.78150391)
\curveto(601.99095916,54.79150403)(601.94095921,54.79650403)(601.89096191,54.79650391)
\curveto(601.84095931,54.79650403)(601.79595935,54.80650402)(601.75596191,54.82650391)
\curveto(601.71595943,54.83650399)(601.68095947,54.84150398)(601.65096191,54.84150391)
\curveto(601.63095952,54.83150399)(601.60595954,54.83150399)(601.57596191,54.84150391)
\lineto(601.33596191,54.90150391)
\curveto(601.25595989,54.91150391)(601.18095997,54.93150389)(601.11096191,54.96150391)
\curveto(600.81096034,55.09150373)(600.56596058,55.23650359)(600.37596191,55.39650391)
\curveto(600.19596095,55.56650326)(600.0459611,55.80150302)(599.92596191,56.10150391)
\curveto(599.83596131,56.3215025)(599.79096136,56.58650224)(599.79096191,56.89650391)
\lineto(599.79096191,57.21150391)
\curveto(599.80096135,57.26150156)(599.80596134,57.31150151)(599.80596191,57.36150391)
\lineto(599.83596191,57.54150391)
\lineto(599.95596191,57.87150391)
\curveto(599.99596115,57.98150084)(600.0459611,58.08150074)(600.10596191,58.17150391)
\curveto(600.28596086,58.46150036)(600.53096062,58.67650015)(600.84096191,58.81650391)
\curveto(601.15096,58.95649987)(601.49095966,59.08149974)(601.86096191,59.19150391)
\curveto(602.00095915,59.23149959)(602.145959,59.26149956)(602.29596191,59.28150391)
\curveto(602.4459587,59.30149952)(602.59595855,59.3264995)(602.74596191,59.35650391)
\curveto(602.81595833,59.37649945)(602.88095827,59.38649944)(602.94096191,59.38650391)
\curveto(603.01095814,59.38649944)(603.08595806,59.39649943)(603.16596191,59.41650391)
\curveto(603.23595791,59.43649939)(603.30595784,59.44649938)(603.37596191,59.44650391)
\curveto(603.4459577,59.45649937)(603.52095763,59.47149935)(603.60096191,59.49150391)
\curveto(603.8509573,59.55149927)(604.08595706,59.60149922)(604.30596191,59.64150391)
\curveto(604.52595662,59.69149913)(604.70095645,59.80649902)(604.83096191,59.98650391)
\curveto(604.89095626,60.06649876)(604.94095621,60.16649866)(604.98096191,60.28650391)
\curveto(605.02095613,60.41649841)(605.02095613,60.55649827)(604.98096191,60.70650391)
\curveto(604.92095623,60.94649788)(604.83095632,61.13649769)(604.71096191,61.27650391)
\curveto(604.60095655,61.41649741)(604.44095671,61.5264973)(604.23096191,61.60650391)
\curveto(604.11095704,61.65649717)(603.96595718,61.69149713)(603.79596191,61.71150391)
\curveto(603.63595751,61.73149709)(603.46595768,61.74149708)(603.28596191,61.74150391)
\curveto(603.10595804,61.74149708)(602.93095822,61.73149709)(602.76096191,61.71150391)
\curveto(602.59095856,61.69149713)(602.4459587,61.66149716)(602.32596191,61.62150391)
\curveto(602.15595899,61.56149726)(601.99095916,61.47649735)(601.83096191,61.36650391)
\curveto(601.7509594,61.30649752)(601.67595947,61.2264976)(601.60596191,61.12650391)
\curveto(601.5459596,61.03649779)(601.49095966,60.93649789)(601.44096191,60.82650391)
\curveto(601.41095974,60.74649808)(601.38095977,60.66149816)(601.35096191,60.57150391)
\curveto(601.33095982,60.48149834)(601.28595986,60.41149841)(601.21596191,60.36150391)
\curveto(601.17595997,60.33149849)(601.10596004,60.30649852)(601.00596191,60.28650391)
\curveto(600.91596023,60.27649855)(600.82096033,60.27149855)(600.72096191,60.27150391)
\curveto(600.62096053,60.27149855)(600.52096063,60.27649855)(600.42096191,60.28650391)
\curveto(600.33096082,60.30649852)(600.26596088,60.33149849)(600.22596191,60.36150391)
\curveto(600.18596096,60.39149843)(600.15596099,60.44149838)(600.13596191,60.51150391)
\curveto(600.11596103,60.58149824)(600.11596103,60.65649817)(600.13596191,60.73650391)
\curveto(600.16596098,60.86649796)(600.19596095,60.98649784)(600.22596191,61.09650391)
\curveto(600.26596088,61.21649761)(600.31096084,61.33149749)(600.36096191,61.44150391)
\curveto(600.5509606,61.79149703)(600.79096036,62.06149676)(601.08096191,62.25150391)
\curveto(601.37095978,62.45149637)(601.73095942,62.61149621)(602.16096191,62.73150391)
\curveto(602.26095889,62.75149607)(602.36095879,62.76649606)(602.46096191,62.77650391)
\curveto(602.57095858,62.78649604)(602.68095847,62.80149602)(602.79096191,62.82150391)
\curveto(602.83095832,62.83149599)(602.89595825,62.83149599)(602.98596191,62.82150391)
\curveto(603.07595807,62.821496)(603.13095802,62.83149599)(603.15096191,62.85150391)
\curveto(603.8509573,62.86149596)(604.46095669,62.78149604)(604.98096191,62.61150391)
\curveto(605.50095565,62.44149638)(605.86595528,62.11649671)(606.07596191,61.63650391)
\curveto(606.16595498,61.43649739)(606.21595493,61.20149762)(606.22596191,60.93150391)
\curveto(606.2459549,60.67149815)(606.25595489,60.39649843)(606.25596191,60.10650391)
\lineto(606.25596191,56.79150391)
\curveto(606.25595489,56.65150217)(606.26095489,56.51650231)(606.27096191,56.38650391)
\curveto(606.28095487,56.25650257)(606.31095484,56.15150267)(606.36096191,56.07150391)
\curveto(606.41095474,56.00150282)(606.47595467,55.95150287)(606.55596191,55.92150391)
\curveto(606.6459545,55.88150294)(606.73095442,55.85150297)(606.81096191,55.83150391)
\curveto(606.89095426,55.821503)(606.9509542,55.77650305)(606.99096191,55.69650391)
\curveto(607.01095414,55.66650316)(607.02095413,55.63650319)(607.02096191,55.60650391)
\curveto(607.02095413,55.57650325)(607.02595412,55.53650329)(607.03596191,55.48650391)
\moveto(604.89096191,57.15150391)
\curveto(604.9509562,57.29150153)(604.98095617,57.45150137)(604.98096191,57.63150391)
\curveto(604.99095616,57.821501)(604.99595615,58.01650081)(604.99596191,58.21650391)
\curveto(604.99595615,58.3265005)(604.99095616,58.4265004)(604.98096191,58.51650391)
\curveto(604.97095618,58.60650022)(604.93095622,58.67650015)(604.86096191,58.72650391)
\curveto(604.83095632,58.74650008)(604.76095639,58.75650007)(604.65096191,58.75650391)
\curveto(604.63095652,58.73650009)(604.59595655,58.7265001)(604.54596191,58.72650391)
\curveto(604.49595665,58.7265001)(604.4509567,58.71650011)(604.41096191,58.69650391)
\curveto(604.33095682,58.67650015)(604.24095691,58.65650017)(604.14096191,58.63650391)
\lineto(603.84096191,58.57650391)
\curveto(603.81095734,58.57650025)(603.77595737,58.57150025)(603.73596191,58.56150391)
\lineto(603.63096191,58.56150391)
\curveto(603.48095767,58.5215003)(603.31595783,58.49650033)(603.13596191,58.48650391)
\curveto(602.96595818,58.48650034)(602.80595834,58.46650036)(602.65596191,58.42650391)
\curveto(602.57595857,58.40650042)(602.50095865,58.38650044)(602.43096191,58.36650391)
\curveto(602.37095878,58.35650047)(602.30095885,58.34150048)(602.22096191,58.32150391)
\curveto(602.06095909,58.27150055)(601.91095924,58.20650062)(601.77096191,58.12650391)
\curveto(601.63095952,58.05650077)(601.51095964,57.96650086)(601.41096191,57.85650391)
\curveto(601.31095984,57.74650108)(601.23595991,57.61150121)(601.18596191,57.45150391)
\curveto(601.13596001,57.30150152)(601.11596003,57.11650171)(601.12596191,56.89650391)
\curveto(601.12596002,56.79650203)(601.14096001,56.70150212)(601.17096191,56.61150391)
\curveto(601.21095994,56.53150229)(601.25595989,56.45650237)(601.30596191,56.38650391)
\curveto(601.38595976,56.27650255)(601.49095966,56.18150264)(601.62096191,56.10150391)
\curveto(601.7509594,56.03150279)(601.89095926,55.97150285)(602.04096191,55.92150391)
\curveto(602.09095906,55.91150291)(602.14095901,55.90650292)(602.19096191,55.90650391)
\curveto(602.24095891,55.90650292)(602.29095886,55.90150292)(602.34096191,55.89150391)
\curveto(602.41095874,55.87150295)(602.49595865,55.85650297)(602.59596191,55.84650391)
\curveto(602.70595844,55.84650298)(602.79595835,55.85650297)(602.86596191,55.87650391)
\curveto(602.92595822,55.89650293)(602.98595816,55.90150292)(603.04596191,55.89150391)
\curveto(603.10595804,55.89150293)(603.16595798,55.90150292)(603.22596191,55.92150391)
\curveto(603.30595784,55.94150288)(603.38095777,55.95650287)(603.45096191,55.96650391)
\curveto(603.53095762,55.97650285)(603.60595754,55.99650283)(603.67596191,56.02650391)
\curveto(603.96595718,56.14650268)(604.21095694,56.29150253)(604.41096191,56.46150391)
\curveto(604.62095653,56.63150219)(604.78095637,56.86150196)(604.89096191,57.15150391)
}
}
{
\newrgbcolor{curcolor}{0 0 0}
\pscustom[linestyle=none,fillstyle=solid,fillcolor=curcolor]
{
\newpath
\moveto(610.63760254,62.83650391)
\curveto(611.35759847,62.84649598)(611.96259787,62.76149606)(612.45260254,62.58150391)
\curveto(612.94259689,62.41149641)(613.32259651,62.10649672)(613.59260254,61.66650391)
\curveto(613.66259617,61.55649727)(613.71759611,61.44149738)(613.75760254,61.32150391)
\curveto(613.79759603,61.21149761)(613.83759599,61.08649774)(613.87760254,60.94650391)
\curveto(613.89759593,60.87649795)(613.90259593,60.80149802)(613.89260254,60.72150391)
\curveto(613.88259595,60.65149817)(613.86759596,60.59649823)(613.84760254,60.55650391)
\curveto(613.827596,60.53649829)(613.80259603,60.51649831)(613.77260254,60.49650391)
\curveto(613.74259609,60.48649834)(613.71759611,60.47149835)(613.69760254,60.45150391)
\curveto(613.64759618,60.43149839)(613.59759623,60.4264984)(613.54760254,60.43650391)
\curveto(613.49759633,60.44649838)(613.44759638,60.44649838)(613.39760254,60.43650391)
\curveto(613.31759651,60.41649841)(613.21259662,60.41149841)(613.08260254,60.42150391)
\curveto(612.95259688,60.44149838)(612.86259697,60.46649836)(612.81260254,60.49650391)
\curveto(612.7325971,60.54649828)(612.67759715,60.61149821)(612.64760254,60.69150391)
\curveto(612.6275972,60.78149804)(612.59259724,60.86649796)(612.54260254,60.94650391)
\curveto(612.45259738,61.10649772)(612.3275975,61.25149757)(612.16760254,61.38150391)
\curveto(612.05759777,61.46149736)(611.93759789,61.5214973)(611.80760254,61.56150391)
\curveto(611.67759815,61.60149722)(611.53759829,61.64149718)(611.38760254,61.68150391)
\curveto(611.33759849,61.70149712)(611.28759854,61.70649712)(611.23760254,61.69650391)
\curveto(611.18759864,61.69649713)(611.13759869,61.70149712)(611.08760254,61.71150391)
\curveto(611.0275988,61.73149709)(610.95259888,61.74149708)(610.86260254,61.74150391)
\curveto(610.77259906,61.74149708)(610.69759913,61.73149709)(610.63760254,61.71150391)
\lineto(610.54760254,61.71150391)
\lineto(610.39760254,61.68150391)
\curveto(610.34759948,61.68149714)(610.29759953,61.67649715)(610.24760254,61.66650391)
\curveto(609.98759984,61.60649722)(609.77260006,61.5214973)(609.60260254,61.41150391)
\curveto(609.4326004,61.30149752)(609.31760051,61.11649771)(609.25760254,60.85650391)
\curveto(609.23760059,60.78649804)(609.2326006,60.71649811)(609.24260254,60.64650391)
\curveto(609.26260057,60.57649825)(609.28260055,60.51649831)(609.30260254,60.46650391)
\curveto(609.36260047,60.31649851)(609.4326004,60.20649862)(609.51260254,60.13650391)
\curveto(609.60260023,60.07649875)(609.71260012,60.00649882)(609.84260254,59.92650391)
\curveto(610.00259983,59.826499)(610.18259965,59.75149907)(610.38260254,59.70150391)
\curveto(610.58259925,59.66149916)(610.78259905,59.61149921)(610.98260254,59.55150391)
\curveto(611.11259872,59.51149931)(611.24259859,59.48149934)(611.37260254,59.46150391)
\curveto(611.50259833,59.44149938)(611.6325982,59.41149941)(611.76260254,59.37150391)
\curveto(611.97259786,59.31149951)(612.17759765,59.25149957)(612.37760254,59.19150391)
\curveto(612.57759725,59.14149968)(612.77759705,59.07649975)(612.97760254,58.99650391)
\lineto(613.12760254,58.93650391)
\curveto(613.17759665,58.91649991)(613.2275966,58.89149993)(613.27760254,58.86150391)
\curveto(613.47759635,58.74150008)(613.65259618,58.60650022)(613.80260254,58.45650391)
\curveto(613.95259588,58.30650052)(614.07759575,58.11650071)(614.17760254,57.88650391)
\curveto(614.19759563,57.81650101)(614.21759561,57.7215011)(614.23760254,57.60150391)
\curveto(614.25759557,57.53150129)(614.26759556,57.45650137)(614.26760254,57.37650391)
\curveto(614.27759555,57.30650152)(614.28259555,57.2265016)(614.28260254,57.13650391)
\lineto(614.28260254,56.98650391)
\curveto(614.26259557,56.91650191)(614.25259558,56.84650198)(614.25260254,56.77650391)
\curveto(614.25259558,56.70650212)(614.24259559,56.63650219)(614.22260254,56.56650391)
\curveto(614.19259564,56.45650237)(614.15759567,56.35150247)(614.11760254,56.25150391)
\curveto(614.07759575,56.15150267)(614.0325958,56.06150276)(613.98260254,55.98150391)
\curveto(613.82259601,55.7215031)(613.61759621,55.51150331)(613.36760254,55.35150391)
\curveto(613.11759671,55.20150362)(612.83759699,55.07150375)(612.52760254,54.96150391)
\curveto(612.43759739,54.93150389)(612.34259749,54.91150391)(612.24260254,54.90150391)
\curveto(612.15259768,54.88150394)(612.06259777,54.85650397)(611.97260254,54.82650391)
\curveto(611.87259796,54.80650402)(611.77259806,54.79650403)(611.67260254,54.79650391)
\curveto(611.57259826,54.79650403)(611.47259836,54.78650404)(611.37260254,54.76650391)
\lineto(611.22260254,54.76650391)
\curveto(611.17259866,54.75650407)(611.10259873,54.75150407)(611.01260254,54.75150391)
\curveto(610.92259891,54.75150407)(610.85259898,54.75650407)(610.80260254,54.76650391)
\lineto(610.63760254,54.76650391)
\curveto(610.57759925,54.78650404)(610.51259932,54.79650403)(610.44260254,54.79650391)
\curveto(610.37259946,54.78650404)(610.31259952,54.79150403)(610.26260254,54.81150391)
\curveto(610.21259962,54.821504)(610.14759968,54.826504)(610.06760254,54.82650391)
\lineto(609.82760254,54.88650391)
\curveto(609.75760007,54.89650393)(609.68260015,54.91650391)(609.60260254,54.94650391)
\curveto(609.29260054,55.04650378)(609.02260081,55.17150365)(608.79260254,55.32150391)
\curveto(608.56260127,55.47150335)(608.36260147,55.66650316)(608.19260254,55.90650391)
\curveto(608.10260173,56.03650279)(608.0276018,56.17150265)(607.96760254,56.31150391)
\curveto(607.90760192,56.45150237)(607.85260198,56.60650222)(607.80260254,56.77650391)
\curveto(607.78260205,56.83650199)(607.77260206,56.90650192)(607.77260254,56.98650391)
\curveto(607.78260205,57.07650175)(607.79760203,57.14650168)(607.81760254,57.19650391)
\curveto(607.84760198,57.23650159)(607.89760193,57.27650155)(607.96760254,57.31650391)
\curveto(608.01760181,57.33650149)(608.08760174,57.34650148)(608.17760254,57.34650391)
\curveto(608.26760156,57.35650147)(608.35760147,57.35650147)(608.44760254,57.34650391)
\curveto(608.53760129,57.33650149)(608.62260121,57.3215015)(608.70260254,57.30150391)
\curveto(608.79260104,57.29150153)(608.85260098,57.27650155)(608.88260254,57.25650391)
\curveto(608.95260088,57.20650162)(608.99760083,57.13150169)(609.01760254,57.03150391)
\curveto(609.04760078,56.94150188)(609.08260075,56.85650197)(609.12260254,56.77650391)
\curveto(609.22260061,56.55650227)(609.35760047,56.38650244)(609.52760254,56.26650391)
\curveto(609.64760018,56.17650265)(609.78260005,56.10650272)(609.93260254,56.05650391)
\curveto(610.08259975,56.00650282)(610.24259959,55.95650287)(610.41260254,55.90650391)
\lineto(610.72760254,55.86150391)
\lineto(610.81760254,55.86150391)
\curveto(610.88759894,55.84150298)(610.97759885,55.83150299)(611.08760254,55.83150391)
\curveto(611.20759862,55.83150299)(611.30759852,55.84150298)(611.38760254,55.86150391)
\curveto(611.45759837,55.86150296)(611.51259832,55.86650296)(611.55260254,55.87650391)
\curveto(611.61259822,55.88650294)(611.67259816,55.89150293)(611.73260254,55.89150391)
\curveto(611.79259804,55.90150292)(611.84759798,55.91150291)(611.89760254,55.92150391)
\curveto(612.18759764,56.00150282)(612.41759741,56.10650272)(612.58760254,56.23650391)
\curveto(612.75759707,56.36650246)(612.87759695,56.58650224)(612.94760254,56.89650391)
\curveto(612.96759686,56.94650188)(612.97259686,57.00150182)(612.96260254,57.06150391)
\curveto(612.95259688,57.1215017)(612.94259689,57.16650166)(612.93260254,57.19650391)
\curveto(612.88259695,57.38650144)(612.81259702,57.5265013)(612.72260254,57.61650391)
\curveto(612.6325972,57.71650111)(612.51759731,57.80650102)(612.37760254,57.88650391)
\curveto(612.28759754,57.94650088)(612.18759764,57.99650083)(612.07760254,58.03650391)
\lineto(611.74760254,58.15650391)
\curveto(611.71759811,58.16650066)(611.68759814,58.17150065)(611.65760254,58.17150391)
\curveto(611.63759819,58.17150065)(611.61259822,58.18150064)(611.58260254,58.20150391)
\curveto(611.24259859,58.31150051)(610.88759894,58.39150043)(610.51760254,58.44150391)
\curveto(610.15759967,58.50150032)(609.81760001,58.59650023)(609.49760254,58.72650391)
\curveto(609.39760043,58.76650006)(609.30260053,58.80150002)(609.21260254,58.83150391)
\curveto(609.12260071,58.86149996)(609.03760079,58.90149992)(608.95760254,58.95150391)
\curveto(608.76760106,59.06149976)(608.59260124,59.18649964)(608.43260254,59.32650391)
\curveto(608.27260156,59.46649936)(608.14760168,59.64149918)(608.05760254,59.85150391)
\curveto(608.0276018,59.9214989)(608.00260183,59.99149883)(607.98260254,60.06150391)
\curveto(607.97260186,60.13149869)(607.95760187,60.20649862)(607.93760254,60.28650391)
\curveto(607.90760192,60.40649842)(607.89760193,60.54149828)(607.90760254,60.69150391)
\curveto(607.91760191,60.85149797)(607.9326019,60.98649784)(607.95260254,61.09650391)
\curveto(607.97260186,61.14649768)(607.98260185,61.18649764)(607.98260254,61.21650391)
\curveto(607.99260184,61.25649757)(608.00760182,61.29649753)(608.02760254,61.33650391)
\curveto(608.11760171,61.56649726)(608.23760159,61.76649706)(608.38760254,61.93650391)
\curveto(608.54760128,62.10649672)(608.7276011,62.25649657)(608.92760254,62.38650391)
\curveto(609.07760075,62.47649635)(609.24260059,62.54649628)(609.42260254,62.59650391)
\curveto(609.60260023,62.65649617)(609.79260004,62.71149611)(609.99260254,62.76150391)
\curveto(610.06259977,62.77149605)(610.1275997,62.78149604)(610.18760254,62.79150391)
\curveto(610.25759957,62.80149602)(610.3325995,62.81149601)(610.41260254,62.82150391)
\curveto(610.44259939,62.83149599)(610.48259935,62.83149599)(610.53260254,62.82150391)
\curveto(610.58259925,62.81149601)(610.61759921,62.81649601)(610.63760254,62.83650391)
}
}
{
\newrgbcolor{curcolor}{0 0 0}
\pscustom[linestyle=none,fillstyle=solid,fillcolor=curcolor]
{
\newpath
\moveto(660.7180835,65.67151611)
\curveto(660.87808284,65.67150543)(661.05308267,65.67150543)(661.2430835,65.67151611)
\curveto(661.43308229,65.68150542)(661.57808214,65.65650545)(661.6780835,65.59651611)
\curveto(661.76808195,65.53650557)(661.82808189,65.44150566)(661.8580835,65.31151611)
\curveto(661.89808182,65.18150592)(661.93808178,65.06150604)(661.9780835,64.95151611)
\curveto(662.05808166,64.75150635)(662.12808159,64.54650656)(662.1880835,64.33651611)
\curveto(662.24808147,64.13650697)(662.3180814,63.93650717)(662.3980835,63.73651611)
\curveto(662.4180813,63.68650742)(662.43308129,63.63650747)(662.4430835,63.58651611)
\lineto(662.4730835,63.43651611)
\curveto(662.54308118,63.26650784)(662.60308112,63.08650802)(662.6530835,62.89651611)
\curveto(662.71308101,62.71650839)(662.77308095,62.53150857)(662.8330835,62.34151611)
\curveto(662.97308075,61.93150917)(663.10808061,61.52650958)(663.2380835,61.12651611)
\curveto(663.37808034,60.72651038)(663.5180802,60.32151078)(663.6580835,59.91151611)
\curveto(663.72807999,59.71151139)(663.78807993,59.5065116)(663.8380835,59.29651611)
\curveto(663.89807982,59.09651201)(663.96807975,58.89651221)(664.0480835,58.69651611)
\curveto(664.06807965,58.64651246)(664.08307964,58.59151251)(664.0930835,58.53151611)
\lineto(664.1530835,58.35151611)
\curveto(664.26307946,58.06151304)(664.36307936,57.76151334)(664.4530835,57.45151611)
\curveto(664.49307923,57.35151375)(664.52807919,57.24651386)(664.5580835,57.13651611)
\curveto(664.58807913,57.03651407)(664.63307909,56.94651416)(664.6930835,56.86651611)
\curveto(664.71307901,56.84651426)(664.74807897,56.81651429)(664.7980835,56.77651611)
\curveto(664.9180788,56.78651432)(664.99307873,56.83651427)(665.0230835,56.92651611)
\curveto(665.05307867,57.02651408)(665.08807863,57.12151398)(665.1280835,57.21151611)
\curveto(665.23807848,57.47151363)(665.32807839,57.73651337)(665.3980835,58.00651611)
\curveto(665.46807825,58.27651283)(665.55307817,58.54151256)(665.6530835,58.80151611)
\curveto(665.71307801,58.96151214)(665.76307796,59.12151198)(665.8030835,59.28151611)
\curveto(665.85307787,59.44151166)(665.90807781,59.6015115)(665.9680835,59.76151611)
\curveto(666.0180777,59.88151122)(666.05807766,60.0015111)(666.0880835,60.12151611)
\curveto(666.12807759,60.25151085)(666.17307755,60.37651073)(666.2230835,60.49651611)
\curveto(666.37307735,60.91651019)(666.51307721,61.34150976)(666.6430835,61.77151611)
\curveto(666.77307695,62.2015089)(666.9180768,62.62650848)(667.0780835,63.04651611)
\curveto(667.09807662,63.08650802)(667.10807661,63.12150798)(667.1080835,63.15151611)
\curveto(667.10807661,63.19150791)(667.1180766,63.23150787)(667.1380835,63.27151611)
\curveto(667.19807652,63.42150768)(667.25307647,63.57650753)(667.3030835,63.73651611)
\curveto(667.35307637,63.89650721)(667.40307632,64.05150705)(667.4530835,64.20151611)
\curveto(667.51307621,64.35150675)(667.56307616,64.5015066)(667.6030835,64.65151611)
\curveto(667.65307607,64.81150629)(667.70807601,64.97150613)(667.7680835,65.13151611)
\curveto(667.79807592,65.22150588)(667.82807589,65.3065058)(667.8580835,65.38651611)
\curveto(667.89807582,65.47650563)(667.95807576,65.54650556)(668.0380835,65.59651611)
\curveto(668.09807562,65.64650546)(668.17807554,65.67150543)(668.2780835,65.67151611)
\curveto(668.38807533,65.67150543)(668.49807522,65.67150543)(668.6080835,65.67151611)
\lineto(668.9380835,65.67151611)
\curveto(669.0180747,65.65150545)(669.08807463,65.63150547)(669.1480835,65.61151611)
\curveto(669.20807451,65.6015055)(669.24807447,65.55650555)(669.2680835,65.47651611)
\lineto(669.2680835,65.40151611)
\curveto(669.27807444,65.38150572)(669.27807444,65.36150574)(669.2680835,65.34151611)
\curveto(669.24807447,65.24150586)(669.2180745,65.14150596)(669.1780835,65.04151611)
\curveto(669.14807457,64.95150615)(669.1180746,64.86650624)(669.0880835,64.78651611)
\curveto(669.04807467,64.7065064)(669.01307471,64.62150648)(668.9830835,64.53151611)
\curveto(668.96307476,64.45150665)(668.93807478,64.37150673)(668.9080835,64.29151611)
\curveto(668.84807487,64.15150695)(668.79307493,64.0015071)(668.7430835,63.84151611)
\curveto(668.70307502,63.69150741)(668.65307507,63.54650756)(668.5930835,63.40651611)
\curveto(668.57307515,63.36650774)(668.56307516,63.33150777)(668.5630835,63.30151611)
\curveto(668.56307516,63.27150783)(668.55307517,63.23650787)(668.5330835,63.19651611)
\curveto(668.45307527,63.02650808)(668.38307534,62.84650826)(668.3230835,62.65651611)
\curveto(668.27307545,62.46650864)(668.20807551,62.28650882)(668.1280835,62.11651611)
\curveto(668.10807561,62.07650903)(668.09807562,62.03650907)(668.0980835,61.99651611)
\curveto(668.09807562,61.96650914)(668.08807563,61.93650917)(668.0680835,61.90651611)
\curveto(668.0180757,61.77650933)(667.96807575,61.64150946)(667.9180835,61.50151611)
\curveto(667.87807584,61.37150973)(667.83307589,61.24150986)(667.7830835,61.11151611)
\curveto(667.76307596,61.08151002)(667.74807597,61.04651006)(667.7380835,61.00651611)
\curveto(667.73807598,60.97651013)(667.72807599,60.94151016)(667.7080835,60.90151611)
\curveto(667.55807616,60.52151058)(667.4180763,60.13651097)(667.2880835,59.74651611)
\curveto(667.16807655,59.35651175)(667.03307669,58.97151213)(666.8830835,58.59151611)
\lineto(666.8380835,58.45651611)
\lineto(666.7180835,58.12651611)
\curveto(666.68807703,58.02651308)(666.65307707,57.92151318)(666.6130835,57.81151611)
\curveto(666.56307716,57.69151341)(666.5180772,57.56651354)(666.4780835,57.43651611)
\curveto(666.43807728,57.31651379)(666.39307733,57.19651391)(666.3430835,57.07651611)
\lineto(666.2830835,56.89651611)
\lineto(666.2230835,56.71651611)
\curveto(666.16307756,56.56651454)(666.10807761,56.41151469)(666.0580835,56.25151611)
\curveto(666.00807771,56.09151501)(665.95307777,55.94151516)(665.8930835,55.80151611)
\curveto(665.84307788,55.67151543)(665.79307793,55.53151557)(665.7430835,55.38151611)
\curveto(665.70307802,55.24151586)(665.63307809,55.13651597)(665.5330835,55.06651611)
\curveto(665.49307823,55.04651606)(665.44807827,55.03151607)(665.3980835,55.02151611)
\curveto(665.34807837,55.01151609)(665.29307843,55.0015161)(665.2330835,54.99151611)
\lineto(664.8130835,54.99151611)
\lineto(664.4230835,54.99151611)
\curveto(664.28307944,54.98151612)(664.17307955,55.0015161)(664.0930835,55.05151611)
\curveto(664.00307972,55.101516)(663.94307978,55.17151593)(663.9130835,55.26151611)
\curveto(663.88307984,55.36151574)(663.84307988,55.46651564)(663.7930835,55.57651611)
\curveto(663.73307999,55.72651538)(663.67808004,55.88151522)(663.6280835,56.04151611)
\curveto(663.57808014,56.21151489)(663.5180802,56.37651473)(663.4480835,56.53651611)
\curveto(663.42808029,56.57651453)(663.41308031,56.61651449)(663.4030835,56.65651611)
\curveto(663.40308032,56.69651441)(663.39308033,56.73651437)(663.3730835,56.77651611)
\curveto(663.29308043,56.97651413)(663.2180805,57.17651393)(663.1480835,57.37651611)
\curveto(663.08808063,57.58651352)(663.0180807,57.78651332)(662.9380835,57.97651611)
\curveto(662.9180808,58.02651308)(662.90308082,58.07151303)(662.8930835,58.11151611)
\curveto(662.89308083,58.15151295)(662.88308084,58.19151291)(662.8630835,58.23151611)
\curveto(662.81308091,58.37151273)(662.76308096,58.5065126)(662.7130835,58.63651611)
\lineto(662.5630835,59.05651611)
\curveto(662.54308118,59.09651201)(662.52808119,59.13651197)(662.5180835,59.17651611)
\curveto(662.5180812,59.21651189)(662.50808121,59.25651185)(662.4880835,59.29651611)
\lineto(662.3380835,59.68651611)
\curveto(662.29808142,59.82651128)(662.25308147,59.96651114)(662.2030835,60.10651611)
\curveto(662.15308157,60.21651089)(662.11308161,60.32651078)(662.0830835,60.43651611)
\curveto(662.05308167,60.55651055)(662.01308171,60.67151043)(661.9630835,60.78151611)
\curveto(661.85308187,61.06151004)(661.75308197,61.34650976)(661.6630835,61.63651611)
\curveto(661.57308215,61.93650917)(661.46808225,62.22650888)(661.3480835,62.50651611)
\curveto(661.30808241,62.59650851)(661.27308245,62.68650842)(661.2430835,62.77651611)
\curveto(661.2230825,62.87650823)(661.19808252,62.96650814)(661.1680835,63.04651611)
\curveto(661.13808258,63.106508)(661.11308261,63.16650794)(661.0930835,63.22651611)
\curveto(661.08308264,63.29650781)(661.06308266,63.36150774)(661.0330835,63.42151611)
\curveto(660.94308278,63.65150745)(660.85808286,63.88650722)(660.7780835,64.12651611)
\curveto(660.70808301,64.36650674)(660.62808309,64.6015065)(660.5380835,64.83151611)
\curveto(660.5180832,64.9015062)(660.49308323,64.97150613)(660.4630835,65.04151611)
\curveto(660.44308328,65.11150599)(660.4230833,65.18650592)(660.4030835,65.26651611)
\curveto(660.36308336,65.36650574)(660.35808336,65.45150565)(660.3880835,65.52151611)
\curveto(660.4180833,65.59150551)(660.48808323,65.63650547)(660.5980835,65.65651611)
\curveto(660.6180831,65.66650544)(660.63808308,65.66650544)(660.6580835,65.65651611)
\curveto(660.67808304,65.65650545)(660.69808302,65.66150544)(660.7180835,65.67151611)
}
}
{
\newrgbcolor{curcolor}{0 0 0}
\pscustom[linestyle=none,fillstyle=solid,fillcolor=curcolor]
{
\newpath
\moveto(670.59300537,64.23151611)
\curveto(670.51300425,64.29150681)(670.4680043,64.39650671)(670.45800537,64.54651611)
\lineto(670.45800537,65.01151611)
\lineto(670.45800537,65.26651611)
\curveto(670.45800431,65.35650575)(670.47300429,65.43150567)(670.50300537,65.49151611)
\curveto(670.54300422,65.57150553)(670.62300414,65.63150547)(670.74300537,65.67151611)
\curveto(670.763004,65.68150542)(670.78300398,65.68150542)(670.80300537,65.67151611)
\curveto(670.83300393,65.67150543)(670.85800391,65.67650543)(670.87800537,65.68651611)
\curveto(671.04800372,65.68650542)(671.20800356,65.68150542)(671.35800537,65.67151611)
\curveto(671.50800326,65.66150544)(671.60800316,65.6015055)(671.65800537,65.49151611)
\curveto(671.68800308,65.43150567)(671.70300306,65.35650575)(671.70300537,65.26651611)
\lineto(671.70300537,65.01151611)
\curveto(671.70300306,64.83150627)(671.69800307,64.66150644)(671.68800537,64.50151611)
\curveto(671.68800308,64.34150676)(671.62300314,64.23650687)(671.49300537,64.18651611)
\curveto(671.44300332,64.16650694)(671.38800338,64.15650695)(671.32800537,64.15651611)
\lineto(671.16300537,64.15651611)
\lineto(670.84800537,64.15651611)
\curveto(670.74800402,64.15650695)(670.6630041,64.18150692)(670.59300537,64.23151611)
\moveto(671.70300537,55.72651611)
\lineto(671.70300537,55.41151611)
\curveto(671.71300305,55.31151579)(671.69300307,55.23151587)(671.64300537,55.17151611)
\curveto(671.61300315,55.11151599)(671.5680032,55.07151603)(671.50800537,55.05151611)
\curveto(671.44800332,55.04151606)(671.37800339,55.02651608)(671.29800537,55.00651611)
\lineto(671.07300537,55.00651611)
\curveto(670.94300382,55.0065161)(670.82800394,55.01151609)(670.72800537,55.02151611)
\curveto(670.63800413,55.04151606)(670.5680042,55.09151601)(670.51800537,55.17151611)
\curveto(670.47800429,55.23151587)(670.45800431,55.3065158)(670.45800537,55.39651611)
\lineto(670.45800537,55.68151611)
\lineto(670.45800537,62.02651611)
\lineto(670.45800537,62.34151611)
\curveto(670.45800431,62.45150865)(670.48300428,62.53650857)(670.53300537,62.59651611)
\curveto(670.5630042,62.64650846)(670.60300416,62.67650843)(670.65300537,62.68651611)
\curveto(670.70300406,62.69650841)(670.75800401,62.71150839)(670.81800537,62.73151611)
\curveto(670.83800393,62.73150837)(670.85800391,62.72650838)(670.87800537,62.71651611)
\curveto(670.90800386,62.71650839)(670.93300383,62.72150838)(670.95300537,62.73151611)
\curveto(671.08300368,62.73150837)(671.21300355,62.72650838)(671.34300537,62.71651611)
\curveto(671.48300328,62.71650839)(671.57800319,62.67650843)(671.62800537,62.59651611)
\curveto(671.67800309,62.53650857)(671.70300306,62.45650865)(671.70300537,62.35651611)
\lineto(671.70300537,62.07151611)
\lineto(671.70300537,55.72651611)
}
}
{
\newrgbcolor{curcolor}{0 0 0}
\pscustom[linestyle=none,fillstyle=solid,fillcolor=curcolor]
{
\newpath
\moveto(680.60784912,55.81651611)
\lineto(680.60784912,55.42651611)
\curveto(680.60784125,55.3065158)(680.58284127,55.2065159)(680.53284912,55.12651611)
\curveto(680.48284137,55.05651605)(680.39784146,55.01651609)(680.27784912,55.00651611)
\lineto(679.93284912,55.00651611)
\curveto(679.87284198,55.0065161)(679.81284204,55.0015161)(679.75284912,54.99151611)
\curveto(679.70284215,54.99151611)(679.6578422,55.0015161)(679.61784912,55.02151611)
\curveto(679.52784233,55.04151606)(679.46784239,55.08151602)(679.43784912,55.14151611)
\curveto(679.39784246,55.19151591)(679.37284248,55.25151585)(679.36284912,55.32151611)
\curveto(679.36284249,55.39151571)(679.34784251,55.46151564)(679.31784912,55.53151611)
\curveto(679.30784255,55.55151555)(679.29284256,55.56651554)(679.27284912,55.57651611)
\curveto(679.26284259,55.59651551)(679.24784261,55.61651549)(679.22784912,55.63651611)
\curveto(679.12784273,55.64651546)(679.04784281,55.62651548)(678.98784912,55.57651611)
\curveto(678.93784292,55.52651558)(678.88284297,55.47651563)(678.82284912,55.42651611)
\curveto(678.62284323,55.27651583)(678.42284343,55.16151594)(678.22284912,55.08151611)
\curveto(678.04284381,55.0015161)(677.83284402,54.94151616)(677.59284912,54.90151611)
\curveto(677.36284449,54.86151624)(677.12284473,54.84151626)(676.87284912,54.84151611)
\curveto(676.63284522,54.83151627)(676.39284546,54.84651626)(676.15284912,54.88651611)
\curveto(675.91284594,54.91651619)(675.70284615,54.97151613)(675.52284912,55.05151611)
\curveto(675.00284685,55.27151583)(674.58284727,55.56651554)(674.26284912,55.93651611)
\curveto(673.94284791,56.31651479)(673.69284816,56.78651432)(673.51284912,57.34651611)
\curveto(673.47284838,57.43651367)(673.44284841,57.52651358)(673.42284912,57.61651611)
\curveto(673.41284844,57.71651339)(673.39284846,57.81651329)(673.36284912,57.91651611)
\curveto(673.3528485,57.96651314)(673.34784851,58.01651309)(673.34784912,58.06651611)
\curveto(673.34784851,58.11651299)(673.34284851,58.16651294)(673.33284912,58.21651611)
\curveto(673.31284854,58.26651284)(673.30284855,58.31651279)(673.30284912,58.36651611)
\curveto(673.31284854,58.42651268)(673.31284854,58.48151262)(673.30284912,58.53151611)
\lineto(673.30284912,58.68151611)
\curveto(673.28284857,58.73151237)(673.27284858,58.79651231)(673.27284912,58.87651611)
\curveto(673.27284858,58.95651215)(673.28284857,59.02151208)(673.30284912,59.07151611)
\lineto(673.30284912,59.23651611)
\curveto(673.32284853,59.3065118)(673.32784853,59.37651173)(673.31784912,59.44651611)
\curveto(673.31784854,59.52651158)(673.32784853,59.6015115)(673.34784912,59.67151611)
\curveto(673.3578485,59.72151138)(673.36284849,59.76651134)(673.36284912,59.80651611)
\curveto(673.36284849,59.84651126)(673.36784849,59.89151121)(673.37784912,59.94151611)
\curveto(673.40784845,60.04151106)(673.43284842,60.13651097)(673.45284912,60.22651611)
\curveto(673.47284838,60.32651078)(673.49784836,60.42151068)(673.52784912,60.51151611)
\curveto(673.6578482,60.89151021)(673.82284803,61.23150987)(674.02284912,61.53151611)
\curveto(674.23284762,61.84150926)(674.48284737,62.09650901)(674.77284912,62.29651611)
\curveto(674.94284691,62.41650869)(675.11784674,62.51650859)(675.29784912,62.59651611)
\curveto(675.48784637,62.67650843)(675.69284616,62.74650836)(675.91284912,62.80651611)
\curveto(675.98284587,62.81650829)(676.04784581,62.82650828)(676.10784912,62.83651611)
\curveto(676.17784568,62.84650826)(676.24784561,62.86150824)(676.31784912,62.88151611)
\lineto(676.46784912,62.88151611)
\curveto(676.54784531,62.9015082)(676.66284519,62.91150819)(676.81284912,62.91151611)
\curveto(676.97284488,62.91150819)(677.09284476,62.9015082)(677.17284912,62.88151611)
\curveto(677.21284464,62.87150823)(677.26784459,62.86650824)(677.33784912,62.86651611)
\curveto(677.44784441,62.83650827)(677.5578443,62.81150829)(677.66784912,62.79151611)
\curveto(677.77784408,62.78150832)(677.88284397,62.75150835)(677.98284912,62.70151611)
\curveto(678.13284372,62.64150846)(678.27284358,62.57650853)(678.40284912,62.50651611)
\curveto(678.54284331,62.43650867)(678.67284318,62.35650875)(678.79284912,62.26651611)
\curveto(678.852843,62.21650889)(678.91284294,62.16150894)(678.97284912,62.10151611)
\curveto(679.04284281,62.05150905)(679.13284272,62.03650907)(679.24284912,62.05651611)
\curveto(679.26284259,62.08650902)(679.27784258,62.11150899)(679.28784912,62.13151611)
\curveto(679.30784255,62.15150895)(679.32284253,62.18150892)(679.33284912,62.22151611)
\curveto(679.36284249,62.31150879)(679.37284248,62.42650868)(679.36284912,62.56651611)
\lineto(679.36284912,62.94151611)
\lineto(679.36284912,64.66651611)
\lineto(679.36284912,65.13151611)
\curveto(679.36284249,65.31150579)(679.38784247,65.44150566)(679.43784912,65.52151611)
\curveto(679.47784238,65.59150551)(679.53784232,65.63650547)(679.61784912,65.65651611)
\curveto(679.63784222,65.65650545)(679.66284219,65.65650545)(679.69284912,65.65651611)
\curveto(679.72284213,65.66650544)(679.74784211,65.67150543)(679.76784912,65.67151611)
\curveto(679.90784195,65.68150542)(680.0528418,65.68150542)(680.20284912,65.67151611)
\curveto(680.36284149,65.67150543)(680.47284138,65.63150547)(680.53284912,65.55151611)
\curveto(680.58284127,65.47150563)(680.60784125,65.37150573)(680.60784912,65.25151611)
\lineto(680.60784912,64.87651611)
\lineto(680.60784912,55.81651611)
\moveto(679.39284912,58.65151611)
\curveto(679.41284244,58.7015124)(679.42284243,58.76651234)(679.42284912,58.84651611)
\curveto(679.42284243,58.93651217)(679.41284244,59.0065121)(679.39284912,59.05651611)
\lineto(679.39284912,59.28151611)
\curveto(679.37284248,59.37151173)(679.3578425,59.46151164)(679.34784912,59.55151611)
\curveto(679.33784252,59.65151145)(679.31784254,59.74151136)(679.28784912,59.82151611)
\curveto(679.26784259,59.9015112)(679.24784261,59.97651113)(679.22784912,60.04651611)
\curveto(679.21784264,60.11651099)(679.19784266,60.18651092)(679.16784912,60.25651611)
\curveto(679.04784281,60.55651055)(678.89284296,60.82151028)(678.70284912,61.05151611)
\curveto(678.51284334,61.28150982)(678.27284358,61.46150964)(677.98284912,61.59151611)
\curveto(677.88284397,61.64150946)(677.77784408,61.67650943)(677.66784912,61.69651611)
\curveto(677.56784429,61.72650938)(677.4578444,61.75150935)(677.33784912,61.77151611)
\curveto(677.2578446,61.79150931)(677.16784469,61.8015093)(677.06784912,61.80151611)
\lineto(676.79784912,61.80151611)
\curveto(676.74784511,61.79150931)(676.70284515,61.78150932)(676.66284912,61.77151611)
\lineto(676.52784912,61.77151611)
\curveto(676.44784541,61.75150935)(676.36284549,61.73150937)(676.27284912,61.71151611)
\curveto(676.19284566,61.69150941)(676.11284574,61.66650944)(676.03284912,61.63651611)
\curveto(675.71284614,61.49650961)(675.4528464,61.29150981)(675.25284912,61.02151611)
\curveto(675.06284679,60.76151034)(674.90784695,60.45651065)(674.78784912,60.10651611)
\curveto(674.74784711,59.99651111)(674.71784714,59.88151122)(674.69784912,59.76151611)
\curveto(674.68784717,59.65151145)(674.67284718,59.54151156)(674.65284912,59.43151611)
\curveto(674.6528472,59.39151171)(674.64784721,59.35151175)(674.63784912,59.31151611)
\lineto(674.63784912,59.20651611)
\curveto(674.61784724,59.15651195)(674.60784725,59.101512)(674.60784912,59.04151611)
\curveto(674.61784724,58.98151212)(674.62284723,58.92651218)(674.62284912,58.87651611)
\lineto(674.62284912,58.54651611)
\curveto(674.62284723,58.44651266)(674.63284722,58.35151275)(674.65284912,58.26151611)
\curveto(674.66284719,58.23151287)(674.66784719,58.18151292)(674.66784912,58.11151611)
\curveto(674.68784717,58.04151306)(674.70284715,57.97151313)(674.71284912,57.90151611)
\lineto(674.77284912,57.69151611)
\curveto(674.88284697,57.34151376)(675.03284682,57.04151406)(675.22284912,56.79151611)
\curveto(675.41284644,56.54151456)(675.6528462,56.33651477)(675.94284912,56.17651611)
\curveto(676.03284582,56.12651498)(676.12284573,56.08651502)(676.21284912,56.05651611)
\curveto(676.30284555,56.02651508)(676.40284545,55.99651511)(676.51284912,55.96651611)
\curveto(676.56284529,55.94651516)(676.61284524,55.94151516)(676.66284912,55.95151611)
\curveto(676.72284513,55.96151514)(676.77784508,55.95651515)(676.82784912,55.93651611)
\curveto(676.86784499,55.92651518)(676.90784495,55.92151518)(676.94784912,55.92151611)
\lineto(677.08284912,55.92151611)
\lineto(677.21784912,55.92151611)
\curveto(677.24784461,55.93151517)(677.29784456,55.93651517)(677.36784912,55.93651611)
\curveto(677.44784441,55.95651515)(677.52784433,55.97151513)(677.60784912,55.98151611)
\curveto(677.68784417,56.0015151)(677.76284409,56.02651508)(677.83284912,56.05651611)
\curveto(678.16284369,56.19651491)(678.42784343,56.37151473)(678.62784912,56.58151611)
\curveto(678.83784302,56.8015143)(679.01284284,57.07651403)(679.15284912,57.40651611)
\curveto(679.20284265,57.51651359)(679.23784262,57.62651348)(679.25784912,57.73651611)
\curveto(679.27784258,57.84651326)(679.30284255,57.95651315)(679.33284912,58.06651611)
\curveto(679.3528425,58.106513)(679.36284249,58.14151296)(679.36284912,58.17151611)
\curveto(679.36284249,58.21151289)(679.36784249,58.25151285)(679.37784912,58.29151611)
\curveto(679.38784247,58.35151275)(679.38784247,58.41151269)(679.37784912,58.47151611)
\curveto(679.37784248,58.53151257)(679.38284247,58.59151251)(679.39284912,58.65151611)
}
}
{
\newrgbcolor{curcolor}{0 0 0}
\pscustom[linestyle=none,fillstyle=solid,fillcolor=curcolor]
{
\newpath
\moveto(689.30409912,59.17651611)
\curveto(689.32409144,59.07651203)(689.32409144,58.96151214)(689.30409912,58.83151611)
\curveto(689.29409147,58.71151239)(689.2640915,58.62651248)(689.21409912,58.57651611)
\curveto(689.1640916,58.53651257)(689.08909167,58.5065126)(688.98909912,58.48651611)
\curveto(688.89909186,58.47651263)(688.79409197,58.47151263)(688.67409912,58.47151611)
\lineto(688.31409912,58.47151611)
\curveto(688.19409257,58.48151262)(688.08909267,58.48651262)(687.99909912,58.48651611)
\lineto(684.15909912,58.48651611)
\curveto(684.07909668,58.48651262)(683.99909676,58.48151262)(683.91909912,58.47151611)
\curveto(683.83909692,58.47151263)(683.77409699,58.45651265)(683.72409912,58.42651611)
\curveto(683.68409708,58.4065127)(683.64409712,58.36651274)(683.60409912,58.30651611)
\curveto(683.58409718,58.27651283)(683.5640972,58.23151287)(683.54409912,58.17151611)
\curveto(683.52409724,58.12151298)(683.52409724,58.07151303)(683.54409912,58.02151611)
\curveto(683.55409721,57.97151313)(683.5590972,57.92651318)(683.55909912,57.88651611)
\curveto(683.5590972,57.84651326)(683.5640972,57.8065133)(683.57409912,57.76651611)
\curveto(683.59409717,57.68651342)(683.61409715,57.6015135)(683.63409912,57.51151611)
\curveto(683.65409711,57.43151367)(683.68409708,57.35151375)(683.72409912,57.27151611)
\curveto(683.95409681,56.73151437)(684.33409643,56.34651476)(684.86409912,56.11651611)
\curveto(684.92409584,56.08651502)(684.98909577,56.06151504)(685.05909912,56.04151611)
\lineto(685.26909912,55.98151611)
\curveto(685.29909546,55.97151513)(685.34909541,55.96651514)(685.41909912,55.96651611)
\curveto(685.5590952,55.92651518)(685.74409502,55.9065152)(685.97409912,55.90651611)
\curveto(686.20409456,55.9065152)(686.38909437,55.92651518)(686.52909912,55.96651611)
\curveto(686.66909409,56.0065151)(686.79409397,56.04651506)(686.90409912,56.08651611)
\curveto(687.02409374,56.13651497)(687.13409363,56.19651491)(687.23409912,56.26651611)
\curveto(687.34409342,56.33651477)(687.43909332,56.41651469)(687.51909912,56.50651611)
\curveto(687.59909316,56.6065145)(687.66909309,56.71151439)(687.72909912,56.82151611)
\curveto(687.78909297,56.92151418)(687.83909292,57.02651408)(687.87909912,57.13651611)
\curveto(687.92909283,57.24651386)(688.00909275,57.32651378)(688.11909912,57.37651611)
\curveto(688.1590926,57.39651371)(688.22409254,57.41151369)(688.31409912,57.42151611)
\curveto(688.40409236,57.43151367)(688.49409227,57.43151367)(688.58409912,57.42151611)
\curveto(688.67409209,57.42151368)(688.759092,57.41651369)(688.83909912,57.40651611)
\curveto(688.91909184,57.39651371)(688.97409179,57.37651373)(689.00409912,57.34651611)
\curveto(689.10409166,57.27651383)(689.12909163,57.16151394)(689.07909912,57.00151611)
\curveto(688.99909176,56.73151437)(688.89409187,56.49151461)(688.76409912,56.28151611)
\curveto(688.5640922,55.96151514)(688.33409243,55.69651541)(688.07409912,55.48651611)
\curveto(687.82409294,55.28651582)(687.50409326,55.12151598)(687.11409912,54.99151611)
\curveto(687.01409375,54.95151615)(686.91409385,54.92651618)(686.81409912,54.91651611)
\curveto(686.71409405,54.89651621)(686.60909415,54.87651623)(686.49909912,54.85651611)
\curveto(686.44909431,54.84651626)(686.39909436,54.84151626)(686.34909912,54.84151611)
\curveto(686.30909445,54.84151626)(686.2640945,54.83651627)(686.21409912,54.82651611)
\lineto(686.06409912,54.82651611)
\curveto(686.01409475,54.81651629)(685.95409481,54.81151629)(685.88409912,54.81151611)
\curveto(685.82409494,54.81151629)(685.77409499,54.81651629)(685.73409912,54.82651611)
\lineto(685.59909912,54.82651611)
\curveto(685.54909521,54.83651627)(685.50409526,54.84151626)(685.46409912,54.84151611)
\curveto(685.42409534,54.84151626)(685.38409538,54.84651626)(685.34409912,54.85651611)
\curveto(685.29409547,54.86651624)(685.23909552,54.87651623)(685.17909912,54.88651611)
\curveto(685.11909564,54.88651622)(685.0640957,54.89151621)(685.01409912,54.90151611)
\curveto(684.92409584,54.92151618)(684.83409593,54.94651616)(684.74409912,54.97651611)
\curveto(684.65409611,54.99651611)(684.56909619,55.02151608)(684.48909912,55.05151611)
\curveto(684.44909631,55.07151603)(684.41409635,55.08151602)(684.38409912,55.08151611)
\curveto(684.35409641,55.09151601)(684.31909644,55.106516)(684.27909912,55.12651611)
\curveto(684.12909663,55.19651591)(683.96909679,55.28151582)(683.79909912,55.38151611)
\curveto(683.50909725,55.57151553)(683.2590975,55.8015153)(683.04909912,56.07151611)
\curveto(682.84909791,56.35151475)(682.67909808,56.66151444)(682.53909912,57.00151611)
\curveto(682.48909827,57.11151399)(682.44909831,57.22651388)(682.41909912,57.34651611)
\curveto(682.39909836,57.46651364)(682.36909839,57.58651352)(682.32909912,57.70651611)
\curveto(682.31909844,57.74651336)(682.31409845,57.78151332)(682.31409912,57.81151611)
\curveto(682.31409845,57.84151326)(682.30909845,57.88151322)(682.29909912,57.93151611)
\curveto(682.27909848,58.01151309)(682.2640985,58.09651301)(682.25409912,58.18651611)
\curveto(682.24409852,58.27651283)(682.22909853,58.36651274)(682.20909912,58.45651611)
\lineto(682.20909912,58.66651611)
\curveto(682.19909856,58.7065124)(682.18909857,58.76151234)(682.17909912,58.83151611)
\curveto(682.17909858,58.91151219)(682.18409858,58.97651213)(682.19409912,59.02651611)
\lineto(682.19409912,59.19151611)
\curveto(682.21409855,59.24151186)(682.21909854,59.29151181)(682.20909912,59.34151611)
\curveto(682.20909855,59.4015117)(682.21409855,59.45651165)(682.22409912,59.50651611)
\curveto(682.2640985,59.66651144)(682.29409847,59.82651128)(682.31409912,59.98651611)
\curveto(682.34409842,60.14651096)(682.38909837,60.29651081)(682.44909912,60.43651611)
\curveto(682.49909826,60.54651056)(682.54409822,60.65651045)(682.58409912,60.76651611)
\curveto(682.63409813,60.88651022)(682.68909807,61.0015101)(682.74909912,61.11151611)
\curveto(682.96909779,61.46150964)(683.21909754,61.76150934)(683.49909912,62.01151611)
\curveto(683.77909698,62.27150883)(684.12409664,62.48650862)(684.53409912,62.65651611)
\curveto(684.65409611,62.7065084)(684.77409599,62.74150836)(684.89409912,62.76151611)
\curveto(685.02409574,62.79150831)(685.1590956,62.82150828)(685.29909912,62.85151611)
\curveto(685.34909541,62.86150824)(685.39409537,62.86650824)(685.43409912,62.86651611)
\curveto(685.47409529,62.87650823)(685.51909524,62.88150822)(685.56909912,62.88151611)
\curveto(685.58909517,62.89150821)(685.61409515,62.89150821)(685.64409912,62.88151611)
\curveto(685.67409509,62.87150823)(685.69909506,62.87650823)(685.71909912,62.89651611)
\curveto(686.13909462,62.9065082)(686.50409426,62.86150824)(686.81409912,62.76151611)
\curveto(687.12409364,62.67150843)(687.40409336,62.54650856)(687.65409912,62.38651611)
\curveto(687.70409306,62.36650874)(687.74409302,62.33650877)(687.77409912,62.29651611)
\curveto(687.80409296,62.26650884)(687.83909292,62.24150886)(687.87909912,62.22151611)
\curveto(687.9590928,62.16150894)(688.03909272,62.09150901)(688.11909912,62.01151611)
\curveto(688.20909255,61.93150917)(688.28409248,61.85150925)(688.34409912,61.77151611)
\curveto(688.50409226,61.56150954)(688.63909212,61.36150974)(688.74909912,61.17151611)
\curveto(688.81909194,61.06151004)(688.87409189,60.94151016)(688.91409912,60.81151611)
\curveto(688.95409181,60.68151042)(688.99909176,60.55151055)(689.04909912,60.42151611)
\curveto(689.09909166,60.29151081)(689.13409163,60.15651095)(689.15409912,60.01651611)
\curveto(689.18409158,59.87651123)(689.21909154,59.73651137)(689.25909912,59.59651611)
\curveto(689.26909149,59.52651158)(689.27409149,59.45651165)(689.27409912,59.38651611)
\lineto(689.30409912,59.17651611)
\moveto(687.84909912,59.68651611)
\curveto(687.87909288,59.72651138)(687.90409286,59.77651133)(687.92409912,59.83651611)
\curveto(687.94409282,59.9065112)(687.94409282,59.97651113)(687.92409912,60.04651611)
\curveto(687.8640929,60.26651084)(687.77909298,60.47151063)(687.66909912,60.66151611)
\curveto(687.52909323,60.89151021)(687.37409339,61.08651002)(687.20409912,61.24651611)
\curveto(687.03409373,61.4065097)(686.81409395,61.54150956)(686.54409912,61.65151611)
\curveto(686.47409429,61.67150943)(686.40409436,61.68650942)(686.33409912,61.69651611)
\curveto(686.2640945,61.71650939)(686.18909457,61.73650937)(686.10909912,61.75651611)
\curveto(686.02909473,61.77650933)(685.94409482,61.78650932)(685.85409912,61.78651611)
\lineto(685.59909912,61.78651611)
\curveto(685.56909519,61.76650934)(685.53409523,61.75650935)(685.49409912,61.75651611)
\curveto(685.45409531,61.76650934)(685.41909534,61.76650934)(685.38909912,61.75651611)
\lineto(685.14909912,61.69651611)
\curveto(685.07909568,61.68650942)(685.00909575,61.67150943)(684.93909912,61.65151611)
\curveto(684.64909611,61.53150957)(684.41409635,61.38150972)(684.23409912,61.20151611)
\curveto(684.0640967,61.02151008)(683.90909685,60.79651031)(683.76909912,60.52651611)
\curveto(683.73909702,60.47651063)(683.70909705,60.41151069)(683.67909912,60.33151611)
\curveto(683.64909711,60.26151084)(683.62409714,60.18151092)(683.60409912,60.09151611)
\curveto(683.58409718,60.0015111)(683.57909718,59.91651119)(683.58909912,59.83651611)
\curveto(683.59909716,59.75651135)(683.63409713,59.69651141)(683.69409912,59.65651611)
\curveto(683.77409699,59.59651151)(683.90909685,59.56651154)(684.09909912,59.56651611)
\curveto(684.29909646,59.57651153)(684.46909629,59.58151152)(684.60909912,59.58151611)
\lineto(686.88909912,59.58151611)
\curveto(687.03909372,59.58151152)(687.21909354,59.57651153)(687.42909912,59.56651611)
\curveto(687.63909312,59.56651154)(687.77909298,59.6065115)(687.84909912,59.68651611)
}
}
{
\newrgbcolor{curcolor}{0 0 0}
\pscustom[linestyle=none,fillstyle=solid,fillcolor=curcolor]
{
\newpath
\moveto(697.73573975,59.20651611)
\curveto(697.75573169,59.14651196)(697.76573168,59.05151205)(697.76573975,58.92151611)
\curveto(697.76573168,58.8015123)(697.76073168,58.71651239)(697.75073975,58.66651611)
\lineto(697.75073975,58.51651611)
\curveto(697.7407317,58.43651267)(697.73073171,58.36151274)(697.72073975,58.29151611)
\curveto(697.72073172,58.23151287)(697.71573173,58.16151294)(697.70573975,58.08151611)
\curveto(697.68573176,58.02151308)(697.67073177,57.96151314)(697.66073975,57.90151611)
\curveto(697.66073178,57.84151326)(697.65073179,57.78151332)(697.63073975,57.72151611)
\curveto(697.59073185,57.59151351)(697.55573189,57.46151364)(697.52573975,57.33151611)
\curveto(697.49573195,57.2015139)(697.45573199,57.08151402)(697.40573975,56.97151611)
\curveto(697.19573225,56.49151461)(696.91573253,56.08651502)(696.56573975,55.75651611)
\curveto(696.21573323,55.43651567)(695.78573366,55.19151591)(695.27573975,55.02151611)
\curveto(695.16573428,54.98151612)(695.0457344,54.95151615)(694.91573975,54.93151611)
\curveto(694.79573465,54.91151619)(694.67073477,54.89151621)(694.54073975,54.87151611)
\curveto(694.48073496,54.86151624)(694.41573503,54.85651625)(694.34573975,54.85651611)
\curveto(694.28573516,54.84651626)(694.22573522,54.84151626)(694.16573975,54.84151611)
\curveto(694.12573532,54.83151627)(694.06573538,54.82651628)(693.98573975,54.82651611)
\curveto(693.91573553,54.82651628)(693.86573558,54.83151627)(693.83573975,54.84151611)
\curveto(693.79573565,54.85151625)(693.75573569,54.85651625)(693.71573975,54.85651611)
\curveto(693.67573577,54.84651626)(693.6407358,54.84651626)(693.61073975,54.85651611)
\lineto(693.52073975,54.85651611)
\lineto(693.16073975,54.90151611)
\curveto(693.02073642,54.94151616)(692.88573656,54.98151612)(692.75573975,55.02151611)
\curveto(692.62573682,55.06151604)(692.50073694,55.106516)(692.38073975,55.15651611)
\curveto(691.93073751,55.35651575)(691.56073788,55.61651549)(691.27073975,55.93651611)
\curveto(690.98073846,56.25651485)(690.7407387,56.64651446)(690.55073975,57.10651611)
\curveto(690.50073894,57.2065139)(690.46073898,57.3065138)(690.43073975,57.40651611)
\curveto(690.41073903,57.5065136)(690.39073905,57.61151349)(690.37073975,57.72151611)
\curveto(690.35073909,57.76151334)(690.3407391,57.79151331)(690.34073975,57.81151611)
\curveto(690.35073909,57.84151326)(690.35073909,57.87651323)(690.34073975,57.91651611)
\curveto(690.32073912,57.99651311)(690.30573914,58.07651303)(690.29573975,58.15651611)
\curveto(690.29573915,58.24651286)(690.28573916,58.33151277)(690.26573975,58.41151611)
\lineto(690.26573975,58.53151611)
\curveto(690.26573918,58.57151253)(690.26073918,58.61651249)(690.25073975,58.66651611)
\curveto(690.2407392,58.71651239)(690.23573921,58.8015123)(690.23573975,58.92151611)
\curveto(690.23573921,59.05151205)(690.2457392,59.14651196)(690.26573975,59.20651611)
\curveto(690.28573916,59.27651183)(690.29073915,59.34651176)(690.28073975,59.41651611)
\curveto(690.27073917,59.48651162)(690.27573917,59.55651155)(690.29573975,59.62651611)
\curveto(690.30573914,59.67651143)(690.31073913,59.71651139)(690.31073975,59.74651611)
\curveto(690.32073912,59.78651132)(690.33073911,59.83151127)(690.34073975,59.88151611)
\curveto(690.37073907,60.0015111)(690.39573905,60.12151098)(690.41573975,60.24151611)
\curveto(690.445739,60.36151074)(690.48573896,60.47651063)(690.53573975,60.58651611)
\curveto(690.68573876,60.95651015)(690.86573858,61.28650982)(691.07573975,61.57651611)
\curveto(691.29573815,61.87650923)(691.56073788,62.12650898)(691.87073975,62.32651611)
\curveto(691.99073745,62.4065087)(692.11573733,62.47150863)(692.24573975,62.52151611)
\curveto(692.37573707,62.58150852)(692.51073693,62.64150846)(692.65073975,62.70151611)
\curveto(692.77073667,62.75150835)(692.90073654,62.78150832)(693.04073975,62.79151611)
\curveto(693.18073626,62.81150829)(693.32073612,62.84150826)(693.46073975,62.88151611)
\lineto(693.65573975,62.88151611)
\curveto(693.72573572,62.89150821)(693.79073565,62.9015082)(693.85073975,62.91151611)
\curveto(694.7407347,62.92150818)(695.48073396,62.73650837)(696.07073975,62.35651611)
\curveto(696.66073278,61.97650913)(697.08573236,61.48150962)(697.34573975,60.87151611)
\curveto(697.39573205,60.77151033)(697.43573201,60.67151043)(697.46573975,60.57151611)
\curveto(697.49573195,60.47151063)(697.53073191,60.36651074)(697.57073975,60.25651611)
\curveto(697.60073184,60.14651096)(697.62573182,60.02651108)(697.64573975,59.89651611)
\curveto(697.66573178,59.77651133)(697.69073175,59.65151145)(697.72073975,59.52151611)
\curveto(697.73073171,59.47151163)(697.73073171,59.41651169)(697.72073975,59.35651611)
\curveto(697.72073172,59.3065118)(697.72573172,59.25651185)(697.73573975,59.20651611)
\moveto(696.40073975,58.35151611)
\curveto(696.42073302,58.42151268)(696.42573302,58.5015126)(696.41573975,58.59151611)
\lineto(696.41573975,58.84651611)
\curveto(696.41573303,59.23651187)(696.38073306,59.56651154)(696.31073975,59.83651611)
\curveto(696.28073316,59.91651119)(696.25573319,59.99651111)(696.23573975,60.07651611)
\curveto(696.21573323,60.15651095)(696.19073325,60.23151087)(696.16073975,60.30151611)
\curveto(695.88073356,60.95151015)(695.43573401,61.4015097)(694.82573975,61.65151611)
\curveto(694.75573469,61.68150942)(694.68073476,61.7015094)(694.60073975,61.71151611)
\lineto(694.36073975,61.77151611)
\curveto(694.28073516,61.79150931)(694.19573525,61.8015093)(694.10573975,61.80151611)
\lineto(693.83573975,61.80151611)
\lineto(693.56573975,61.75651611)
\curveto(693.46573598,61.73650937)(693.37073607,61.71150939)(693.28073975,61.68151611)
\curveto(693.20073624,61.66150944)(693.12073632,61.63150947)(693.04073975,61.59151611)
\curveto(692.97073647,61.57150953)(692.90573654,61.54150956)(692.84573975,61.50151611)
\curveto(692.78573666,61.46150964)(692.73073671,61.42150968)(692.68073975,61.38151611)
\curveto(692.440737,61.21150989)(692.2457372,61.0065101)(692.09573975,60.76651611)
\curveto(691.9457375,60.52651058)(691.81573763,60.24651086)(691.70573975,59.92651611)
\curveto(691.67573777,59.82651128)(691.65573779,59.72151138)(691.64573975,59.61151611)
\curveto(691.63573781,59.51151159)(691.62073782,59.4065117)(691.60073975,59.29651611)
\curveto(691.59073785,59.25651185)(691.58573786,59.19151191)(691.58573975,59.10151611)
\curveto(691.57573787,59.07151203)(691.57073787,59.03651207)(691.57073975,58.99651611)
\curveto(691.58073786,58.95651215)(691.58573786,58.91151219)(691.58573975,58.86151611)
\lineto(691.58573975,58.56151611)
\curveto(691.58573786,58.46151264)(691.59573785,58.37151273)(691.61573975,58.29151611)
\lineto(691.64573975,58.11151611)
\curveto(691.66573778,58.01151309)(691.68073776,57.91151319)(691.69073975,57.81151611)
\curveto(691.71073773,57.72151338)(691.7407377,57.63651347)(691.78073975,57.55651611)
\curveto(691.88073756,57.31651379)(691.99573745,57.09151401)(692.12573975,56.88151611)
\curveto(692.26573718,56.67151443)(692.43573701,56.49651461)(692.63573975,56.35651611)
\curveto(692.68573676,56.32651478)(692.73073671,56.3015148)(692.77073975,56.28151611)
\curveto(692.81073663,56.26151484)(692.85573659,56.23651487)(692.90573975,56.20651611)
\curveto(692.98573646,56.15651495)(693.07073637,56.11151499)(693.16073975,56.07151611)
\curveto(693.26073618,56.04151506)(693.36573608,56.01151509)(693.47573975,55.98151611)
\curveto(693.52573592,55.96151514)(693.57073587,55.95151515)(693.61073975,55.95151611)
\curveto(693.66073578,55.96151514)(693.71073573,55.96151514)(693.76073975,55.95151611)
\curveto(693.79073565,55.94151516)(693.85073559,55.93151517)(693.94073975,55.92151611)
\curveto(694.0407354,55.91151519)(694.11573533,55.91651519)(694.16573975,55.93651611)
\curveto(694.20573524,55.94651516)(694.2457352,55.94651516)(694.28573975,55.93651611)
\curveto(694.32573512,55.93651517)(694.36573508,55.94651516)(694.40573975,55.96651611)
\curveto(694.48573496,55.98651512)(694.56573488,56.0015151)(694.64573975,56.01151611)
\curveto(694.72573472,56.03151507)(694.80073464,56.05651505)(694.87073975,56.08651611)
\curveto(695.21073423,56.22651488)(695.48573396,56.42151468)(695.69573975,56.67151611)
\curveto(695.90573354,56.92151418)(696.08073336,57.21651389)(696.22073975,57.55651611)
\curveto(696.27073317,57.67651343)(696.30073314,57.8015133)(696.31073975,57.93151611)
\curveto(696.33073311,58.07151303)(696.36073308,58.21151289)(696.40073975,58.35151611)
}
}
{
\newrgbcolor{curcolor}{0 0 0}
\pscustom[linestyle=none,fillstyle=solid,fillcolor=curcolor]
{
\newpath
\moveto(701.654021,62.91151611)
\curveto(702.37401693,62.92150818)(702.97901633,62.83650827)(703.469021,62.65651611)
\curveto(703.95901535,62.48650862)(704.33901497,62.18150892)(704.609021,61.74151611)
\curveto(704.67901463,61.63150947)(704.73401457,61.51650959)(704.774021,61.39651611)
\curveto(704.81401449,61.28650982)(704.85401445,61.16150994)(704.894021,61.02151611)
\curveto(704.91401439,60.95151015)(704.91901439,60.87651023)(704.909021,60.79651611)
\curveto(704.89901441,60.72651038)(704.88401442,60.67151043)(704.864021,60.63151611)
\curveto(704.84401446,60.61151049)(704.81901449,60.59151051)(704.789021,60.57151611)
\curveto(704.75901455,60.56151054)(704.73401457,60.54651056)(704.714021,60.52651611)
\curveto(704.66401464,60.5065106)(704.61401469,60.5015106)(704.564021,60.51151611)
\curveto(704.51401479,60.52151058)(704.46401484,60.52151058)(704.414021,60.51151611)
\curveto(704.33401497,60.49151061)(704.22901508,60.48651062)(704.099021,60.49651611)
\curveto(703.96901534,60.51651059)(703.87901543,60.54151056)(703.829021,60.57151611)
\curveto(703.74901556,60.62151048)(703.69401561,60.68651042)(703.664021,60.76651611)
\curveto(703.64401566,60.85651025)(703.6090157,60.94151016)(703.559021,61.02151611)
\curveto(703.46901584,61.18150992)(703.34401596,61.32650978)(703.184021,61.45651611)
\curveto(703.07401623,61.53650957)(702.95401635,61.59650951)(702.824021,61.63651611)
\curveto(702.69401661,61.67650943)(702.55401675,61.71650939)(702.404021,61.75651611)
\curveto(702.35401695,61.77650933)(702.304017,61.78150932)(702.254021,61.77151611)
\curveto(702.2040171,61.77150933)(702.15401715,61.77650933)(702.104021,61.78651611)
\curveto(702.04401726,61.8065093)(701.96901734,61.81650929)(701.879021,61.81651611)
\curveto(701.78901752,61.81650929)(701.71401759,61.8065093)(701.654021,61.78651611)
\lineto(701.564021,61.78651611)
\lineto(701.414021,61.75651611)
\curveto(701.36401794,61.75650935)(701.31401799,61.75150935)(701.264021,61.74151611)
\curveto(701.0040183,61.68150942)(700.78901852,61.59650951)(700.619021,61.48651611)
\curveto(700.44901886,61.37650973)(700.33401897,61.19150991)(700.274021,60.93151611)
\curveto(700.25401905,60.86151024)(700.24901906,60.79151031)(700.259021,60.72151611)
\curveto(700.27901903,60.65151045)(700.29901901,60.59151051)(700.319021,60.54151611)
\curveto(700.37901893,60.39151071)(700.44901886,60.28151082)(700.529021,60.21151611)
\curveto(700.61901869,60.15151095)(700.72901858,60.08151102)(700.859021,60.00151611)
\curveto(701.01901829,59.9015112)(701.19901811,59.82651128)(701.399021,59.77651611)
\curveto(701.59901771,59.73651137)(701.79901751,59.68651142)(701.999021,59.62651611)
\curveto(702.12901718,59.58651152)(702.25901705,59.55651155)(702.389021,59.53651611)
\curveto(702.51901679,59.51651159)(702.64901666,59.48651162)(702.779021,59.44651611)
\curveto(702.98901632,59.38651172)(703.19401611,59.32651178)(703.394021,59.26651611)
\curveto(703.59401571,59.21651189)(703.79401551,59.15151195)(703.994021,59.07151611)
\lineto(704.144021,59.01151611)
\curveto(704.19401511,58.99151211)(704.24401506,58.96651214)(704.294021,58.93651611)
\curveto(704.49401481,58.81651229)(704.66901464,58.68151242)(704.819021,58.53151611)
\curveto(704.96901434,58.38151272)(705.09401421,58.19151291)(705.194021,57.96151611)
\curveto(705.21401409,57.89151321)(705.23401407,57.79651331)(705.254021,57.67651611)
\curveto(705.27401403,57.6065135)(705.28401402,57.53151357)(705.284021,57.45151611)
\curveto(705.29401401,57.38151372)(705.29901401,57.3015138)(705.299021,57.21151611)
\lineto(705.299021,57.06151611)
\curveto(705.27901403,56.99151411)(705.26901404,56.92151418)(705.269021,56.85151611)
\curveto(705.26901404,56.78151432)(705.25901405,56.71151439)(705.239021,56.64151611)
\curveto(705.2090141,56.53151457)(705.17401413,56.42651468)(705.134021,56.32651611)
\curveto(705.09401421,56.22651488)(705.04901426,56.13651497)(704.999021,56.05651611)
\curveto(704.83901447,55.79651531)(704.63401467,55.58651552)(704.384021,55.42651611)
\curveto(704.13401517,55.27651583)(703.85401545,55.14651596)(703.544021,55.03651611)
\curveto(703.45401585,55.0065161)(703.35901595,54.98651612)(703.259021,54.97651611)
\curveto(703.16901614,54.95651615)(703.07901623,54.93151617)(702.989021,54.90151611)
\curveto(702.88901642,54.88151622)(702.78901652,54.87151623)(702.689021,54.87151611)
\curveto(702.58901672,54.87151623)(702.48901682,54.86151624)(702.389021,54.84151611)
\lineto(702.239021,54.84151611)
\curveto(702.18901712,54.83151627)(702.11901719,54.82651628)(702.029021,54.82651611)
\curveto(701.93901737,54.82651628)(701.86901744,54.83151627)(701.819021,54.84151611)
\lineto(701.654021,54.84151611)
\curveto(701.59401771,54.86151624)(701.52901778,54.87151623)(701.459021,54.87151611)
\curveto(701.38901792,54.86151624)(701.32901798,54.86651624)(701.279021,54.88651611)
\curveto(701.22901808,54.89651621)(701.16401814,54.9015162)(701.084021,54.90151611)
\lineto(700.844021,54.96151611)
\curveto(700.77401853,54.97151613)(700.69901861,54.99151611)(700.619021,55.02151611)
\curveto(700.309019,55.12151598)(700.03901927,55.24651586)(699.809021,55.39651611)
\curveto(699.57901973,55.54651556)(699.37901993,55.74151536)(699.209021,55.98151611)
\curveto(699.11902019,56.11151499)(699.04402026,56.24651486)(698.984021,56.38651611)
\curveto(698.92402038,56.52651458)(698.86902044,56.68151442)(698.819021,56.85151611)
\curveto(698.79902051,56.91151419)(698.78902052,56.98151412)(698.789021,57.06151611)
\curveto(698.79902051,57.15151395)(698.81402049,57.22151388)(698.834021,57.27151611)
\curveto(698.86402044,57.31151379)(698.91402039,57.35151375)(698.984021,57.39151611)
\curveto(699.03402027,57.41151369)(699.1040202,57.42151368)(699.194021,57.42151611)
\curveto(699.28402002,57.43151367)(699.37401993,57.43151367)(699.464021,57.42151611)
\curveto(699.55401975,57.41151369)(699.63901967,57.39651371)(699.719021,57.37651611)
\curveto(699.8090195,57.36651374)(699.86901944,57.35151375)(699.899021,57.33151611)
\curveto(699.96901934,57.28151382)(700.01401929,57.2065139)(700.034021,57.10651611)
\curveto(700.06401924,57.01651409)(700.09901921,56.93151417)(700.139021,56.85151611)
\curveto(700.23901907,56.63151447)(700.37401893,56.46151464)(700.544021,56.34151611)
\curveto(700.66401864,56.25151485)(700.79901851,56.18151492)(700.949021,56.13151611)
\curveto(701.09901821,56.08151502)(701.25901805,56.03151507)(701.429021,55.98151611)
\lineto(701.744021,55.93651611)
\lineto(701.834021,55.93651611)
\curveto(701.9040174,55.91651519)(701.99401731,55.9065152)(702.104021,55.90651611)
\curveto(702.22401708,55.9065152)(702.32401698,55.91651519)(702.404021,55.93651611)
\curveto(702.47401683,55.93651517)(702.52901678,55.94151516)(702.569021,55.95151611)
\curveto(702.62901668,55.96151514)(702.68901662,55.96651514)(702.749021,55.96651611)
\curveto(702.8090165,55.97651513)(702.86401644,55.98651512)(702.914021,55.99651611)
\curveto(703.2040161,56.07651503)(703.43401587,56.18151492)(703.604021,56.31151611)
\curveto(703.77401553,56.44151466)(703.89401541,56.66151444)(703.964021,56.97151611)
\curveto(703.98401532,57.02151408)(703.98901532,57.07651403)(703.979021,57.13651611)
\curveto(703.96901534,57.19651391)(703.95901535,57.24151386)(703.949021,57.27151611)
\curveto(703.89901541,57.46151364)(703.82901548,57.6015135)(703.739021,57.69151611)
\curveto(703.64901566,57.79151331)(703.53401577,57.88151322)(703.394021,57.96151611)
\curveto(703.304016,58.02151308)(703.2040161,58.07151303)(703.094021,58.11151611)
\lineto(702.764021,58.23151611)
\curveto(702.73401657,58.24151286)(702.7040166,58.24651286)(702.674021,58.24651611)
\curveto(702.65401665,58.24651286)(702.62901668,58.25651285)(702.599021,58.27651611)
\curveto(702.25901705,58.38651272)(701.9040174,58.46651264)(701.534021,58.51651611)
\curveto(701.17401813,58.57651253)(700.83401847,58.67151243)(700.514021,58.80151611)
\curveto(700.41401889,58.84151226)(700.31901899,58.87651223)(700.229021,58.90651611)
\curveto(700.13901917,58.93651217)(700.05401925,58.97651213)(699.974021,59.02651611)
\curveto(699.78401952,59.13651197)(699.6090197,59.26151184)(699.449021,59.40151611)
\curveto(699.28902002,59.54151156)(699.16402014,59.71651139)(699.074021,59.92651611)
\curveto(699.04402026,59.99651111)(699.01902029,60.06651104)(698.999021,60.13651611)
\curveto(698.98902032,60.2065109)(698.97402033,60.28151082)(698.954021,60.36151611)
\curveto(698.92402038,60.48151062)(698.91402039,60.61651049)(698.924021,60.76651611)
\curveto(698.93402037,60.92651018)(698.94902036,61.06151004)(698.969021,61.17151611)
\curveto(698.98902032,61.22150988)(698.99902031,61.26150984)(698.999021,61.29151611)
\curveto(699.0090203,61.33150977)(699.02402028,61.37150973)(699.044021,61.41151611)
\curveto(699.13402017,61.64150946)(699.25402005,61.84150926)(699.404021,62.01151611)
\curveto(699.56401974,62.18150892)(699.74401956,62.33150877)(699.944021,62.46151611)
\curveto(700.09401921,62.55150855)(700.25901905,62.62150848)(700.439021,62.67151611)
\curveto(700.61901869,62.73150837)(700.8090185,62.78650832)(701.009021,62.83651611)
\curveto(701.07901823,62.84650826)(701.14401816,62.85650825)(701.204021,62.86651611)
\curveto(701.27401803,62.87650823)(701.34901796,62.88650822)(701.429021,62.89651611)
\curveto(701.45901785,62.9065082)(701.49901781,62.9065082)(701.549021,62.89651611)
\curveto(701.59901771,62.88650822)(701.63401767,62.89150821)(701.654021,62.91151611)
}
}
{
\newrgbcolor{curcolor}{0 0 0}
\pscustom[linestyle=none,fillstyle=solid,fillcolor=curcolor]
{
\newpath
\moveto(237.42056885,55.52150146)
\curveto(237.4405593,55.47150072)(237.46555928,55.41150078)(237.49556885,55.34150146)
\curveto(237.52555922,55.27150092)(237.5455592,55.19650099)(237.55556885,55.11650146)
\curveto(237.57555917,55.04650114)(237.57555917,54.97650121)(237.55556885,54.90650146)
\curveto(237.5455592,54.84650134)(237.50555924,54.80150139)(237.43556885,54.77150146)
\curveto(237.38555936,54.75150144)(237.32555942,54.74150145)(237.25556885,54.74150146)
\lineto(237.04556885,54.74150146)
\lineto(236.59556885,54.74150146)
\curveto(236.4455603,54.74150145)(236.32556042,54.76650142)(236.23556885,54.81650146)
\curveto(236.13556061,54.87650131)(236.06056068,54.98150121)(236.01056885,55.13150146)
\curveto(235.97056077,55.28150091)(235.92556082,55.41650077)(235.87556885,55.53650146)
\curveto(235.76556098,55.79650039)(235.66556108,56.06650012)(235.57556885,56.34650146)
\curveto(235.48556126,56.62649956)(235.38556136,56.90149929)(235.27556885,57.17150146)
\curveto(235.2455615,57.26149893)(235.21556153,57.34649884)(235.18556885,57.42650146)
\curveto(235.16556158,57.50649868)(235.13556161,57.58149861)(235.09556885,57.65150146)
\curveto(235.06556168,57.72149847)(235.02056172,57.78149841)(234.96056885,57.83150146)
\curveto(234.90056184,57.88149831)(234.82056192,57.92149827)(234.72056885,57.95150146)
\curveto(234.67056207,57.97149822)(234.61056213,57.97649821)(234.54056885,57.96650146)
\lineto(234.34556885,57.96650146)
\lineto(231.51056885,57.96650146)
\lineto(231.21056885,57.96650146)
\curveto(231.10056564,57.97649821)(230.99556575,57.97649821)(230.89556885,57.96650146)
\curveto(230.79556595,57.95649823)(230.70056604,57.94149825)(230.61056885,57.92150146)
\curveto(230.53056621,57.90149829)(230.47056627,57.86149833)(230.43056885,57.80150146)
\curveto(230.35056639,57.70149849)(230.29056645,57.5864986)(230.25056885,57.45650146)
\curveto(230.22056652,57.33649885)(230.18056656,57.21149898)(230.13056885,57.08150146)
\curveto(230.03056671,56.85149934)(229.93556681,56.61149958)(229.84556885,56.36150146)
\curveto(229.76556698,56.11150008)(229.67556707,55.87150032)(229.57556885,55.64150146)
\curveto(229.55556719,55.58150061)(229.53056721,55.51150068)(229.50056885,55.43150146)
\curveto(229.48056726,55.36150083)(229.45556729,55.2865009)(229.42556885,55.20650146)
\curveto(229.39556735,55.12650106)(229.36056738,55.05150114)(229.32056885,54.98150146)
\curveto(229.29056745,54.92150127)(229.25556749,54.87650131)(229.21556885,54.84650146)
\curveto(229.13556761,54.7865014)(229.02556772,54.75150144)(228.88556885,54.74150146)
\lineto(228.46556885,54.74150146)
\lineto(228.22556885,54.74150146)
\curveto(228.15556859,54.75150144)(228.09556865,54.77650141)(228.04556885,54.81650146)
\curveto(227.99556875,54.84650134)(227.96556878,54.8915013)(227.95556885,54.95150146)
\curveto(227.95556879,55.01150118)(227.96056878,55.07150112)(227.97056885,55.13150146)
\curveto(227.99056875,55.20150099)(228.01056873,55.26650092)(228.03056885,55.32650146)
\curveto(228.06056868,55.39650079)(228.08556866,55.44650074)(228.10556885,55.47650146)
\curveto(228.2455685,55.79650039)(228.37056837,56.11150008)(228.48056885,56.42150146)
\curveto(228.59056815,56.74149945)(228.71056803,57.06149913)(228.84056885,57.38150146)
\curveto(228.93056781,57.60149859)(229.01556773,57.81649837)(229.09556885,58.02650146)
\curveto(229.17556757,58.24649794)(229.26056748,58.46649772)(229.35056885,58.68650146)
\curveto(229.65056709,59.40649678)(229.93556681,60.13149606)(230.20556885,60.86150146)
\curveto(230.47556627,61.60149459)(230.76056598,62.33649385)(231.06056885,63.06650146)
\curveto(231.17056557,63.32649286)(231.27056547,63.5914926)(231.36056885,63.86150146)
\curveto(231.46056528,64.13149206)(231.56556518,64.39649179)(231.67556885,64.65650146)
\curveto(231.72556502,64.76649142)(231.77056497,64.8864913)(231.81056885,65.01650146)
\curveto(231.86056488,65.15649103)(231.93056481,65.25649093)(232.02056885,65.31650146)
\curveto(232.06056468,65.35649083)(232.12556462,65.3864908)(232.21556885,65.40650146)
\curveto(232.23556451,65.41649077)(232.25556449,65.41649077)(232.27556885,65.40650146)
\curveto(232.30556444,65.40649078)(232.33056441,65.41149078)(232.35056885,65.42150146)
\curveto(232.53056421,65.42149077)(232.740564,65.42149077)(232.98056885,65.42150146)
\curveto(233.22056352,65.43149076)(233.39556335,65.39649079)(233.50556885,65.31650146)
\curveto(233.58556316,65.25649093)(233.6455631,65.15649103)(233.68556885,65.01650146)
\curveto(233.73556301,64.8864913)(233.78556296,64.76649142)(233.83556885,64.65650146)
\curveto(233.93556281,64.42649176)(234.02556272,64.19649199)(234.10556885,63.96650146)
\curveto(234.18556256,63.73649245)(234.27556247,63.50649268)(234.37556885,63.27650146)
\curveto(234.45556229,63.07649311)(234.53056221,62.87149332)(234.60056885,62.66150146)
\curveto(234.68056206,62.45149374)(234.76556198,62.24649394)(234.85556885,62.04650146)
\curveto(235.15556159,61.31649487)(235.4405613,60.57649561)(235.71056885,59.82650146)
\curveto(235.99056075,59.0864971)(236.28556046,58.35149784)(236.59556885,57.62150146)
\curveto(236.63556011,57.53149866)(236.66556008,57.44649874)(236.68556885,57.36650146)
\curveto(236.71556003,57.2864989)(236.74556,57.20149899)(236.77556885,57.11150146)
\curveto(236.88555986,56.85149934)(236.99055975,56.5864996)(237.09056885,56.31650146)
\curveto(237.20055954,56.04650014)(237.31055943,55.78150041)(237.42056885,55.52150146)
\moveto(234.21056885,59.16650146)
\curveto(234.30056244,59.19649699)(234.35556239,59.24649694)(234.37556885,59.31650146)
\curveto(234.40556234,59.3864968)(234.41056233,59.46149673)(234.39056885,59.54150146)
\curveto(234.38056236,59.63149656)(234.35556239,59.71649647)(234.31556885,59.79650146)
\curveto(234.28556246,59.8864963)(234.25556249,59.96149623)(234.22556885,60.02150146)
\curveto(234.20556254,60.06149613)(234.19556255,60.09649609)(234.19556885,60.12650146)
\curveto(234.19556255,60.15649603)(234.18556256,60.191496)(234.16556885,60.23150146)
\lineto(234.07556885,60.47150146)
\curveto(234.05556269,60.56149563)(234.02556272,60.65149554)(233.98556885,60.74150146)
\curveto(233.83556291,61.10149509)(233.70056304,61.46649472)(233.58056885,61.83650146)
\curveto(233.47056327,62.21649397)(233.3405634,62.5864936)(233.19056885,62.94650146)
\curveto(233.1405636,63.05649313)(233.09556365,63.16649302)(233.05556885,63.27650146)
\curveto(233.02556372,63.3864928)(232.98556376,63.4914927)(232.93556885,63.59150146)
\curveto(232.91556383,63.64149255)(232.89056385,63.6864925)(232.86056885,63.72650146)
\curveto(232.8405639,63.77649241)(232.79056395,63.80149239)(232.71056885,63.80150146)
\curveto(232.69056405,63.78149241)(232.67056407,63.76649242)(232.65056885,63.75650146)
\curveto(232.63056411,63.74649244)(232.61056413,63.73149246)(232.59056885,63.71150146)
\curveto(232.55056419,63.66149253)(232.52056422,63.60649258)(232.50056885,63.54650146)
\curveto(232.48056426,63.49649269)(232.46056428,63.44149275)(232.44056885,63.38150146)
\curveto(232.39056435,63.27149292)(232.35056439,63.16149303)(232.32056885,63.05150146)
\curveto(232.29056445,62.94149325)(232.25056449,62.83149336)(232.20056885,62.72150146)
\curveto(232.03056471,62.33149386)(231.88056486,61.93649425)(231.75056885,61.53650146)
\curveto(231.63056511,61.13649505)(231.49056525,60.74649544)(231.33056885,60.36650146)
\lineto(231.27056885,60.21650146)
\curveto(231.26056548,60.16649602)(231.2455655,60.11649607)(231.22556885,60.06650146)
\lineto(231.13556885,59.82650146)
\curveto(231.10556564,59.74649644)(231.08056566,59.66649652)(231.06056885,59.58650146)
\curveto(231.0405657,59.53649665)(231.03056571,59.48149671)(231.03056885,59.42150146)
\curveto(231.0405657,59.36149683)(231.05556569,59.31149688)(231.07556885,59.27150146)
\curveto(231.12556562,59.191497)(231.23056551,59.14649704)(231.39056885,59.13650146)
\lineto(231.84056885,59.13650146)
\lineto(233.44556885,59.13650146)
\curveto(233.55556319,59.13649705)(233.69056305,59.13149706)(233.85056885,59.12150146)
\curveto(234.01056273,59.12149707)(234.13056261,59.13649705)(234.21056885,59.16650146)
}
}
{
\newrgbcolor{curcolor}{0 0 0}
\pscustom[linestyle=none,fillstyle=solid,fillcolor=curcolor]
{
\newpath
\moveto(242.18213135,62.64650146)
\curveto(242.41212656,62.64649354)(242.54212643,62.5864936)(242.57213135,62.46650146)
\curveto(242.60212637,62.35649383)(242.61712635,62.191494)(242.61713135,61.97150146)
\lineto(242.61713135,61.68650146)
\curveto(242.61712635,61.59649459)(242.59212638,61.52149467)(242.54213135,61.46150146)
\curveto(242.48212649,61.38149481)(242.39712657,61.33649485)(242.28713135,61.32650146)
\curveto(242.17712679,61.32649486)(242.0671269,61.31149488)(241.95713135,61.28150146)
\curveto(241.81712715,61.25149494)(241.68212729,61.22149497)(241.55213135,61.19150146)
\curveto(241.43212754,61.16149503)(241.31712765,61.12149507)(241.20713135,61.07150146)
\curveto(240.91712805,60.94149525)(240.68212829,60.76149543)(240.50213135,60.53150146)
\curveto(240.32212865,60.31149588)(240.1671288,60.05649613)(240.03713135,59.76650146)
\curveto(239.99712897,59.65649653)(239.967129,59.54149665)(239.94713135,59.42150146)
\curveto(239.92712904,59.31149688)(239.90212907,59.19649699)(239.87213135,59.07650146)
\curveto(239.86212911,59.02649716)(239.85712911,58.97649721)(239.85713135,58.92650146)
\curveto(239.8671291,58.87649731)(239.8671291,58.82649736)(239.85713135,58.77650146)
\curveto(239.82712914,58.65649753)(239.81212916,58.51649767)(239.81213135,58.35650146)
\curveto(239.82212915,58.20649798)(239.82712914,58.06149813)(239.82713135,57.92150146)
\lineto(239.82713135,56.07650146)
\lineto(239.82713135,55.73150146)
\curveto(239.82712914,55.61150058)(239.82212915,55.49650069)(239.81213135,55.38650146)
\curveto(239.80212917,55.27650091)(239.79712917,55.18150101)(239.79713135,55.10150146)
\curveto(239.80712916,55.02150117)(239.78712918,54.95150124)(239.73713135,54.89150146)
\curveto(239.68712928,54.82150137)(239.60712936,54.78150141)(239.49713135,54.77150146)
\curveto(239.39712957,54.76150143)(239.28712968,54.75650143)(239.16713135,54.75650146)
\lineto(238.89713135,54.75650146)
\curveto(238.84713012,54.77650141)(238.79713017,54.7915014)(238.74713135,54.80150146)
\curveto(238.70713026,54.82150137)(238.67713029,54.84650134)(238.65713135,54.87650146)
\curveto(238.60713036,54.94650124)(238.57713039,55.03150116)(238.56713135,55.13150146)
\lineto(238.56713135,55.46150146)
\lineto(238.56713135,56.61650146)
\lineto(238.56713135,60.77150146)
\lineto(238.56713135,61.80650146)
\lineto(238.56713135,62.10650146)
\curveto(238.57713039,62.20649398)(238.60713036,62.2914939)(238.65713135,62.36150146)
\curveto(238.68713028,62.40149379)(238.73713023,62.43149376)(238.80713135,62.45150146)
\curveto(238.88713008,62.47149372)(238.97213,62.48149371)(239.06213135,62.48150146)
\curveto(239.15212982,62.4914937)(239.24212973,62.4914937)(239.33213135,62.48150146)
\curveto(239.42212955,62.47149372)(239.49212948,62.45649373)(239.54213135,62.43650146)
\curveto(239.62212935,62.40649378)(239.6721293,62.34649384)(239.69213135,62.25650146)
\curveto(239.72212925,62.17649401)(239.73712923,62.0864941)(239.73713135,61.98650146)
\lineto(239.73713135,61.68650146)
\curveto(239.73712923,61.5864946)(239.75712921,61.49649469)(239.79713135,61.41650146)
\curveto(239.80712916,61.39649479)(239.81712915,61.38149481)(239.82713135,61.37150146)
\lineto(239.87213135,61.32650146)
\curveto(239.98212899,61.32649486)(240.0721289,61.37149482)(240.14213135,61.46150146)
\curveto(240.21212876,61.56149463)(240.2721287,61.64149455)(240.32213135,61.70150146)
\lineto(240.41213135,61.79150146)
\curveto(240.50212847,61.90149429)(240.62712834,62.01649417)(240.78713135,62.13650146)
\curveto(240.94712802,62.25649393)(241.09712787,62.34649384)(241.23713135,62.40650146)
\curveto(241.32712764,62.45649373)(241.42212755,62.4914937)(241.52213135,62.51150146)
\curveto(241.62212735,62.54149365)(241.72712724,62.57149362)(241.83713135,62.60150146)
\curveto(241.89712707,62.61149358)(241.95712701,62.61649357)(242.01713135,62.61650146)
\curveto(242.07712689,62.62649356)(242.13212684,62.63649355)(242.18213135,62.64650146)
}
}
{
\newrgbcolor{curcolor}{0 0 0}
\pscustom[linestyle=none,fillstyle=solid,fillcolor=curcolor]
{
\newpath
\moveto(246.68189697,62.64650146)
\curveto(247.42189218,62.65649353)(248.03689157,62.54649364)(248.52689697,62.31650146)
\curveto(249.02689058,62.09649409)(249.42189018,61.76149443)(249.71189697,61.31150146)
\curveto(249.84188976,61.11149508)(249.95188965,60.86649532)(250.04189697,60.57650146)
\curveto(250.06188954,60.52649566)(250.07688953,60.46149573)(250.08689697,60.38150146)
\curveto(250.09688951,60.30149589)(250.09188951,60.23149596)(250.07189697,60.17150146)
\curveto(250.04188956,60.12149607)(249.99188961,60.07649611)(249.92189697,60.03650146)
\curveto(249.89188971,60.01649617)(249.86188974,60.00649618)(249.83189697,60.00650146)
\curveto(249.8018898,60.01649617)(249.76688984,60.01649617)(249.72689697,60.00650146)
\curveto(249.68688992,59.99649619)(249.64688996,59.9914962)(249.60689697,59.99150146)
\curveto(249.56689004,60.00149619)(249.52689008,60.00649618)(249.48689697,60.00650146)
\lineto(249.17189697,60.00650146)
\curveto(249.07189053,60.01649617)(248.98689062,60.04649614)(248.91689697,60.09650146)
\curveto(248.83689077,60.15649603)(248.78189082,60.24149595)(248.75189697,60.35150146)
\curveto(248.72189088,60.46149573)(248.68189092,60.55649563)(248.63189697,60.63650146)
\curveto(248.48189112,60.89649529)(248.28689132,61.10149509)(248.04689697,61.25150146)
\curveto(247.96689164,61.30149489)(247.88189172,61.34149485)(247.79189697,61.37150146)
\curveto(247.7018919,61.41149478)(247.606892,61.44649474)(247.50689697,61.47650146)
\curveto(247.36689224,61.51649467)(247.18189242,61.53649465)(246.95189697,61.53650146)
\curveto(246.72189288,61.54649464)(246.53189307,61.52649466)(246.38189697,61.47650146)
\curveto(246.31189329,61.45649473)(246.24689336,61.44149475)(246.18689697,61.43150146)
\curveto(246.12689348,61.42149477)(246.06189354,61.40649478)(245.99189697,61.38650146)
\curveto(245.73189387,61.27649491)(245.5018941,61.12649506)(245.30189697,60.93650146)
\curveto(245.1018945,60.74649544)(244.94689466,60.52149567)(244.83689697,60.26150146)
\curveto(244.79689481,60.17149602)(244.76189484,60.07649611)(244.73189697,59.97650146)
\curveto(244.7018949,59.8864963)(244.67189493,59.7864964)(244.64189697,59.67650146)
\lineto(244.55189697,59.27150146)
\curveto(244.54189506,59.22149697)(244.53689507,59.16649702)(244.53689697,59.10650146)
\curveto(244.54689506,59.04649714)(244.54189506,58.9914972)(244.52189697,58.94150146)
\lineto(244.52189697,58.82150146)
\curveto(244.51189509,58.78149741)(244.5018951,58.71649747)(244.49189697,58.62650146)
\curveto(244.49189511,58.53649765)(244.5018951,58.47149772)(244.52189697,58.43150146)
\curveto(244.53189507,58.38149781)(244.53189507,58.33149786)(244.52189697,58.28150146)
\curveto(244.51189509,58.23149796)(244.51189509,58.18149801)(244.52189697,58.13150146)
\curveto(244.53189507,58.0914981)(244.53689507,58.02149817)(244.53689697,57.92150146)
\curveto(244.55689505,57.84149835)(244.57189503,57.75649843)(244.58189697,57.66650146)
\curveto(244.601895,57.57649861)(244.62189498,57.4914987)(244.64189697,57.41150146)
\curveto(244.75189485,57.0914991)(244.87689473,56.81149938)(245.01689697,56.57150146)
\curveto(245.16689444,56.34149985)(245.37189423,56.14150005)(245.63189697,55.97150146)
\curveto(245.72189388,55.92150027)(245.81189379,55.87650031)(245.90189697,55.83650146)
\curveto(246.0018936,55.79650039)(246.1068935,55.75650043)(246.21689697,55.71650146)
\curveto(246.26689334,55.70650048)(246.3068933,55.70150049)(246.33689697,55.70150146)
\curveto(246.36689324,55.70150049)(246.4068932,55.69650049)(246.45689697,55.68650146)
\curveto(246.48689312,55.67650051)(246.53689307,55.67150052)(246.60689697,55.67150146)
\lineto(246.77189697,55.67150146)
\curveto(246.77189283,55.66150053)(246.79189281,55.65650053)(246.83189697,55.65650146)
\curveto(246.85189275,55.66650052)(246.87689273,55.66650052)(246.90689697,55.65650146)
\curveto(246.93689267,55.65650053)(246.96689264,55.66150053)(246.99689697,55.67150146)
\curveto(247.06689254,55.6915005)(247.13189247,55.69650049)(247.19189697,55.68650146)
\curveto(247.26189234,55.6865005)(247.33189227,55.69650049)(247.40189697,55.71650146)
\curveto(247.66189194,55.79650039)(247.88689172,55.89650029)(248.07689697,56.01650146)
\curveto(248.26689134,56.14650004)(248.42689118,56.31149988)(248.55689697,56.51150146)
\curveto(248.606891,56.5914996)(248.65189095,56.67649951)(248.69189697,56.76650146)
\lineto(248.81189697,57.03650146)
\curveto(248.83189077,57.11649907)(248.85189075,57.191499)(248.87189697,57.26150146)
\curveto(248.9018907,57.34149885)(248.95189065,57.40649878)(249.02189697,57.45650146)
\curveto(249.05189055,57.4864987)(249.11189049,57.50649868)(249.20189697,57.51650146)
\curveto(249.29189031,57.53649865)(249.38689022,57.54649864)(249.48689697,57.54650146)
\curveto(249.59689001,57.55649863)(249.69688991,57.55649863)(249.78689697,57.54650146)
\curveto(249.88688972,57.53649865)(249.95688965,57.51649867)(249.99689697,57.48650146)
\curveto(250.05688955,57.44649874)(250.09188951,57.3864988)(250.10189697,57.30650146)
\curveto(250.12188948,57.22649896)(250.12188948,57.14149905)(250.10189697,57.05150146)
\curveto(250.05188955,56.90149929)(250.0018896,56.75649943)(249.95189697,56.61650146)
\curveto(249.91188969,56.4864997)(249.85688975,56.35649983)(249.78689697,56.22650146)
\curveto(249.63688997,55.92650026)(249.44689016,55.66150053)(249.21689697,55.43150146)
\curveto(248.99689061,55.20150099)(248.72689088,55.01650117)(248.40689697,54.87650146)
\curveto(248.32689128,54.83650135)(248.24189136,54.80150139)(248.15189697,54.77150146)
\curveto(248.06189154,54.75150144)(247.96689164,54.72650146)(247.86689697,54.69650146)
\curveto(247.75689185,54.65650153)(247.64689196,54.63650155)(247.53689697,54.63650146)
\curveto(247.42689218,54.62650156)(247.31689229,54.61150158)(247.20689697,54.59150146)
\curveto(247.16689244,54.57150162)(247.12689248,54.56650162)(247.08689697,54.57650146)
\curveto(247.04689256,54.5865016)(247.0068926,54.5865016)(246.96689697,54.57650146)
\lineto(246.83189697,54.57650146)
\lineto(246.59189697,54.57650146)
\curveto(246.52189308,54.56650162)(246.45689315,54.57150162)(246.39689697,54.59150146)
\lineto(246.32189697,54.59150146)
\lineto(245.96189697,54.63650146)
\curveto(245.83189377,54.67650151)(245.7068939,54.71150148)(245.58689697,54.74150146)
\curveto(245.46689414,54.77150142)(245.35189425,54.81150138)(245.24189697,54.86150146)
\curveto(244.88189472,55.02150117)(244.58189502,55.21150098)(244.34189697,55.43150146)
\curveto(244.11189549,55.65150054)(243.89689571,55.92150027)(243.69689697,56.24150146)
\curveto(243.64689596,56.32149987)(243.601896,56.41149978)(243.56189697,56.51150146)
\lineto(243.44189697,56.81150146)
\curveto(243.39189621,56.92149927)(243.35689625,57.03649915)(243.33689697,57.15650146)
\curveto(243.31689629,57.27649891)(243.29189631,57.39649879)(243.26189697,57.51650146)
\curveto(243.25189635,57.55649863)(243.24689636,57.59649859)(243.24689697,57.63650146)
\curveto(243.24689636,57.67649851)(243.24189636,57.71649847)(243.23189697,57.75650146)
\curveto(243.21189639,57.81649837)(243.2018964,57.88149831)(243.20189697,57.95150146)
\curveto(243.21189639,58.02149817)(243.2068964,58.0864981)(243.18689697,58.14650146)
\lineto(243.18689697,58.29650146)
\curveto(243.17689643,58.34649784)(243.17189643,58.41649777)(243.17189697,58.50650146)
\curveto(243.17189643,58.59649759)(243.17689643,58.66649752)(243.18689697,58.71650146)
\curveto(243.19689641,58.76649742)(243.19689641,58.81149738)(243.18689697,58.85150146)
\curveto(243.18689642,58.8914973)(243.19189641,58.93149726)(243.20189697,58.97150146)
\curveto(243.22189638,59.04149715)(243.22689638,59.11149708)(243.21689697,59.18150146)
\curveto(243.21689639,59.25149694)(243.22689638,59.31649687)(243.24689697,59.37650146)
\curveto(243.28689632,59.54649664)(243.32189628,59.71649647)(243.35189697,59.88650146)
\curveto(243.38189622,60.05649613)(243.42689618,60.21649597)(243.48689697,60.36650146)
\curveto(243.69689591,60.8864953)(243.95189565,61.30649488)(244.25189697,61.62650146)
\curveto(244.55189505,61.94649424)(244.96189464,62.21149398)(245.48189697,62.42150146)
\curveto(245.59189401,62.47149372)(245.71189389,62.50649368)(245.84189697,62.52650146)
\curveto(245.97189363,62.54649364)(246.1068935,62.57149362)(246.24689697,62.60150146)
\curveto(246.31689329,62.61149358)(246.38689322,62.61649357)(246.45689697,62.61650146)
\curveto(246.52689308,62.62649356)(246.601893,62.63649355)(246.68189697,62.64650146)
}
}
{
\newrgbcolor{curcolor}{0 0 0}
\pscustom[linestyle=none,fillstyle=solid,fillcolor=curcolor]
{
\newpath
\moveto(252.0685376,65.40650146)
\curveto(252.20853608,65.40649078)(252.35353593,65.40149079)(252.5035376,65.39150146)
\curveto(252.66353562,65.3914908)(252.77353551,65.35149084)(252.8335376,65.27150146)
\curveto(252.8835354,65.20149099)(252.90853538,65.09649109)(252.9085376,64.95650146)
\lineto(252.9085376,64.56650146)
\lineto(252.9085376,62.97650146)
\lineto(252.9085376,62.52650146)
\curveto(252.90853538,62.4864937)(252.90353538,62.44649374)(252.8935376,62.40650146)
\curveto(252.89353539,62.36649382)(252.89853539,62.32649386)(252.9085376,62.28650146)
\curveto(252.91853537,62.25649393)(252.91853537,62.22149397)(252.9085376,62.18150146)
\curveto(252.90853538,62.14149405)(252.91353537,62.10649408)(252.9235376,62.07650146)
\curveto(252.94353534,61.99649419)(252.95353533,61.92149427)(252.9535376,61.85150146)
\curveto(252.96353532,61.78149441)(253.01853527,61.74649444)(253.1185376,61.74650146)
\curveto(253.13853515,61.76649442)(253.15853513,61.77649441)(253.1785376,61.77650146)
\curveto(253.20853508,61.7864944)(253.23353505,61.80149439)(253.2535376,61.82150146)
\curveto(253.31353497,61.86149433)(253.36853492,61.90149429)(253.4185376,61.94150146)
\curveto(253.46853482,61.9914942)(253.52353476,62.04149415)(253.5835376,62.09150146)
\curveto(253.69353459,62.17149402)(253.81353447,62.24149395)(253.9435376,62.30150146)
\curveto(254.0835342,62.37149382)(254.22353406,62.43149376)(254.3635376,62.48150146)
\curveto(254.44353384,62.50149369)(254.52853376,62.51649367)(254.6185376,62.52650146)
\curveto(254.70853358,62.54649364)(254.79353349,62.56649362)(254.8735376,62.58650146)
\curveto(254.91353337,62.60649358)(254.95353333,62.61149358)(254.9935376,62.60150146)
\curveto(255.03353325,62.60149359)(255.07853321,62.60649358)(255.1285376,62.61650146)
\curveto(255.17853311,62.63649355)(255.25853303,62.64649354)(255.3685376,62.64650146)
\curveto(255.4885328,62.64649354)(255.57353271,62.63649355)(255.6235376,62.61650146)
\lineto(255.7585376,62.61650146)
\curveto(255.80853248,62.61649357)(255.85853243,62.61149358)(255.9085376,62.60150146)
\curveto(255.9885323,62.58149361)(256.06853222,62.56649362)(256.1485376,62.55650146)
\curveto(256.22853206,62.54649364)(256.30853198,62.53149366)(256.3885376,62.51150146)
\curveto(256.43853185,62.4914937)(256.47853181,62.47649371)(256.5085376,62.46650146)
\curveto(256.53853175,62.46649372)(256.57853171,62.45649373)(256.6285376,62.43650146)
\curveto(256.73853155,62.3864938)(256.84353144,62.33149386)(256.9435376,62.27150146)
\curveto(257.04353124,62.22149397)(257.14353114,62.16149403)(257.2435376,62.09150146)
\curveto(257.34353094,62.00149419)(257.43853085,61.89649429)(257.5285376,61.77650146)
\curveto(257.5885307,61.6864945)(257.64353064,61.59649459)(257.6935376,61.50650146)
\curveto(257.74353054,61.41649477)(257.79353049,61.31649487)(257.8435376,61.20650146)
\curveto(257.87353041,61.13649505)(257.89353039,61.06649512)(257.9035376,60.99650146)
\curveto(257.92353036,60.92649526)(257.94353034,60.85149534)(257.9635376,60.77150146)
\curveto(257.9835303,60.72149547)(257.99353029,60.67149552)(257.9935376,60.62150146)
\curveto(257.99353029,60.57149562)(257.99853029,60.51649567)(258.0085376,60.45650146)
\curveto(258.02853026,60.40649578)(258.03353025,60.35649583)(258.0235376,60.30650146)
\curveto(258.02353026,60.25649593)(258.03353025,60.20649598)(258.0535376,60.15650146)
\lineto(258.0535376,60.00650146)
\curveto(258.06353022,59.95649623)(258.06353022,59.90149629)(258.0535376,59.84150146)
\lineto(258.0535376,59.67650146)
\lineto(258.0535376,59.03150146)
\lineto(258.0535376,55.91150146)
\lineto(258.0535376,55.61150146)
\curveto(258.06353022,55.50150069)(258.06353022,55.3915008)(258.0535376,55.28150146)
\curveto(258.05353023,55.18150101)(258.04353024,55.0865011)(258.0235376,54.99650146)
\curveto(258.00353028,54.90650128)(257.97353031,54.84650134)(257.9335376,54.81650146)
\curveto(257.86353042,54.75650143)(257.73353055,54.72650146)(257.5435376,54.72650146)
\lineto(257.1535376,54.72650146)
\curveto(257.03353125,54.72650146)(256.94353134,54.76650142)(256.8835376,54.84650146)
\curveto(256.83353145,54.91650127)(256.80853148,54.99650119)(256.8085376,55.08650146)
\lineto(256.8085376,55.40150146)
\lineto(256.8085376,56.48150146)
\lineto(256.8085376,58.88150146)
\lineto(256.8085376,59.73650146)
\curveto(256.81853147,60.04649614)(256.7885315,60.31149588)(256.7185376,60.53150146)
\curveto(256.59853169,60.87149532)(256.39853189,61.12149507)(256.1185376,61.28150146)
\curveto(256.03853225,61.33149486)(255.95353233,61.37149482)(255.8635376,61.40150146)
\curveto(255.77353251,61.44149475)(255.67353261,61.47149472)(255.5635376,61.49150146)
\curveto(255.52353276,61.50149469)(255.46353282,61.50649468)(255.3835376,61.50650146)
\curveto(255.34353294,61.51649467)(255.29353299,61.52649466)(255.2335376,61.53650146)
\curveto(255.1835331,61.54649464)(255.13353315,61.54149465)(255.0835376,61.52150146)
\curveto(255.04353324,61.51149468)(255.00353328,61.50649468)(254.9635376,61.50650146)
\curveto(254.92353336,61.51649467)(254.87853341,61.51649467)(254.8285376,61.50650146)
\curveto(254.73853355,61.4864947)(254.64353364,61.46649472)(254.5435376,61.44650146)
\curveto(254.45353383,61.43649475)(254.36853392,61.41149478)(254.2885376,61.37150146)
\curveto(253.94853434,61.23149496)(253.67853461,61.04149515)(253.4785376,60.80150146)
\curveto(253.27853501,60.56149563)(253.12353516,60.25649593)(253.0135376,59.88650146)
\curveto(252.99353529,59.81649637)(252.97853531,59.74149645)(252.9685376,59.66150146)
\curveto(252.95853533,59.58149661)(252.94353534,59.50149669)(252.9235376,59.42150146)
\curveto(252.91353537,59.3914968)(252.90853538,59.35649683)(252.9085376,59.31650146)
\curveto(252.91853537,59.2864969)(252.91853537,59.25649693)(252.9085376,59.22650146)
\curveto(252.89853539,59.17649701)(252.89853539,59.12649706)(252.9085376,59.07650146)
\curveto(252.91853537,59.02649716)(252.91853537,58.97649721)(252.9085376,58.92650146)
\lineto(252.9085376,55.91150146)
\lineto(252.9085376,55.62650146)
\curveto(252.91853537,55.52650066)(252.91853537,55.42650076)(252.9085376,55.32650146)
\curveto(252.90853538,55.22650096)(252.90353538,55.13150106)(252.8935376,55.04150146)
\curveto(252.8835354,54.95150124)(252.86353542,54.8865013)(252.8335376,54.84650146)
\curveto(252.79353549,54.79650139)(252.74353554,54.76650142)(252.6835376,54.75650146)
\curveto(252.63353565,54.75650143)(252.57353571,54.74650144)(252.5035376,54.72650146)
\lineto(252.2935376,54.72650146)
\lineto(251.9785376,54.72650146)
\curveto(251.87853641,54.73650145)(251.80353648,54.77150142)(251.7535376,54.83150146)
\curveto(251.70353658,54.91150128)(251.67353661,55.00650118)(251.6635376,55.11650146)
\lineto(251.6635376,55.49150146)
\lineto(251.6635376,56.87150146)
\lineto(251.6635376,63.12650146)
\lineto(251.6635376,64.59650146)
\curveto(251.66353662,64.70649148)(251.65853663,64.82149137)(251.6485376,64.94150146)
\curveto(251.64853664,65.07149112)(251.67353661,65.17149102)(251.7235376,65.24150146)
\curveto(251.76353652,65.31149088)(251.83853645,65.36149083)(251.9485376,65.39150146)
\curveto(251.96853632,65.40149079)(251.9885363,65.40149079)(252.0085376,65.39150146)
\curveto(252.02853626,65.3914908)(252.04853624,65.39649079)(252.0685376,65.40650146)
}
}
{
\newrgbcolor{curcolor}{0 0 0}
\pscustom[linestyle=none,fillstyle=solid,fillcolor=curcolor]
{
\newpath
\moveto(260.23814697,63.96650146)
\curveto(260.15814585,64.02649216)(260.1131459,64.13149206)(260.10314697,64.28150146)
\lineto(260.10314697,64.74650146)
\lineto(260.10314697,65.00150146)
\curveto(260.10314591,65.0914911)(260.11814589,65.16649102)(260.14814697,65.22650146)
\curveto(260.18814582,65.30649088)(260.26814574,65.36649082)(260.38814697,65.40650146)
\curveto(260.4081456,65.41649077)(260.42814558,65.41649077)(260.44814697,65.40650146)
\curveto(260.47814553,65.40649078)(260.50314551,65.41149078)(260.52314697,65.42150146)
\curveto(260.69314532,65.42149077)(260.85314516,65.41649077)(261.00314697,65.40650146)
\curveto(261.15314486,65.39649079)(261.25314476,65.33649085)(261.30314697,65.22650146)
\curveto(261.33314468,65.16649102)(261.34814466,65.0914911)(261.34814697,65.00150146)
\lineto(261.34814697,64.74650146)
\curveto(261.34814466,64.56649162)(261.34314467,64.39649179)(261.33314697,64.23650146)
\curveto(261.33314468,64.07649211)(261.26814474,63.97149222)(261.13814697,63.92150146)
\curveto(261.08814492,63.90149229)(261.03314498,63.8914923)(260.97314697,63.89150146)
\lineto(260.80814697,63.89150146)
\lineto(260.49314697,63.89150146)
\curveto(260.39314562,63.8914923)(260.3081457,63.91649227)(260.23814697,63.96650146)
\moveto(261.34814697,55.46150146)
\lineto(261.34814697,55.14650146)
\curveto(261.35814465,55.04650114)(261.33814467,54.96650122)(261.28814697,54.90650146)
\curveto(261.25814475,54.84650134)(261.2131448,54.80650138)(261.15314697,54.78650146)
\curveto(261.09314492,54.77650141)(261.02314499,54.76150143)(260.94314697,54.74150146)
\lineto(260.71814697,54.74150146)
\curveto(260.58814542,54.74150145)(260.47314554,54.74650144)(260.37314697,54.75650146)
\curveto(260.28314573,54.77650141)(260.2131458,54.82650136)(260.16314697,54.90650146)
\curveto(260.12314589,54.96650122)(260.10314591,55.04150115)(260.10314697,55.13150146)
\lineto(260.10314697,55.41650146)
\lineto(260.10314697,61.76150146)
\lineto(260.10314697,62.07650146)
\curveto(260.10314591,62.186494)(260.12814588,62.27149392)(260.17814697,62.33150146)
\curveto(260.2081458,62.38149381)(260.24814576,62.41149378)(260.29814697,62.42150146)
\curveto(260.34814566,62.43149376)(260.40314561,62.44649374)(260.46314697,62.46650146)
\curveto(260.48314553,62.46649372)(260.50314551,62.46149373)(260.52314697,62.45150146)
\curveto(260.55314546,62.45149374)(260.57814543,62.45649373)(260.59814697,62.46650146)
\curveto(260.72814528,62.46649372)(260.85814515,62.46149373)(260.98814697,62.45150146)
\curveto(261.12814488,62.45149374)(261.22314479,62.41149378)(261.27314697,62.33150146)
\curveto(261.32314469,62.27149392)(261.34814466,62.191494)(261.34814697,62.09150146)
\lineto(261.34814697,61.80650146)
\lineto(261.34814697,55.46150146)
}
}
{
\newrgbcolor{curcolor}{0 0 0}
\pscustom[linestyle=none,fillstyle=solid,fillcolor=curcolor]
{
\newpath
\moveto(263.06799072,62.46650146)
\lineto(263.54799072,62.46650146)
\curveto(263.71798938,62.46649372)(263.84798925,62.43649375)(263.93799072,62.37650146)
\curveto(264.00798909,62.32649386)(264.05298905,62.26149393)(264.07299072,62.18150146)
\curveto(264.102989,62.11149408)(264.13298897,62.03649415)(264.16299072,61.95650146)
\curveto(264.22298888,61.81649437)(264.27298883,61.67649451)(264.31299072,61.53650146)
\curveto(264.35298875,61.39649479)(264.3979887,61.25649493)(264.44799072,61.11650146)
\curveto(264.64798845,60.57649561)(264.83298827,60.03149616)(265.00299072,59.48150146)
\curveto(265.17298793,58.94149725)(265.35798774,58.40149779)(265.55799072,57.86150146)
\curveto(265.62798747,57.68149851)(265.68798741,57.49649869)(265.73799072,57.30650146)
\curveto(265.78798731,57.12649906)(265.85298725,56.94649924)(265.93299072,56.76650146)
\curveto(265.95298715,56.69649949)(265.97798712,56.62149957)(266.00799072,56.54150146)
\curveto(266.03798706,56.46149973)(266.08798701,56.41149978)(266.15799072,56.39150146)
\curveto(266.23798686,56.37149982)(266.2979868,56.40649978)(266.33799072,56.49650146)
\curveto(266.38798671,56.5864996)(266.42298668,56.65649953)(266.44299072,56.70650146)
\curveto(266.52298658,56.89649929)(266.58798651,57.0864991)(266.63799072,57.27650146)
\curveto(266.6979864,57.47649871)(266.76298634,57.67649851)(266.83299072,57.87650146)
\curveto(266.96298614,58.25649793)(267.08798601,58.63149756)(267.20799072,59.00150146)
\curveto(267.32798577,59.38149681)(267.45298565,59.76149643)(267.58299072,60.14150146)
\curveto(267.63298547,60.31149588)(267.68298542,60.47649571)(267.73299072,60.63650146)
\curveto(267.78298532,60.80649538)(267.84298526,60.97149522)(267.91299072,61.13150146)
\curveto(267.96298514,61.27149492)(268.00798509,61.41149478)(268.04799072,61.55150146)
\curveto(268.08798501,61.6914945)(268.13298497,61.83149436)(268.18299072,61.97150146)
\curveto(268.2029849,62.04149415)(268.22798487,62.11149408)(268.25799072,62.18150146)
\curveto(268.28798481,62.25149394)(268.32798477,62.31149388)(268.37799072,62.36150146)
\curveto(268.45798464,62.41149378)(268.54798455,62.44149375)(268.64799072,62.45150146)
\curveto(268.74798435,62.46149373)(268.86798423,62.46649372)(269.00799072,62.46650146)
\curveto(269.07798402,62.46649372)(269.14298396,62.46149373)(269.20299072,62.45150146)
\curveto(269.26298384,62.45149374)(269.31798378,62.44149375)(269.36799072,62.42150146)
\curveto(269.45798364,62.38149381)(269.5029836,62.31649387)(269.50299072,62.22650146)
\curveto(269.51298359,62.13649405)(269.4979836,62.04649414)(269.45799072,61.95650146)
\curveto(269.3979837,61.7864944)(269.33798376,61.61149458)(269.27799072,61.43150146)
\curveto(269.21798388,61.25149494)(269.14798395,61.07649511)(269.06799072,60.90650146)
\curveto(269.04798405,60.85649533)(269.03298407,60.80649538)(269.02299072,60.75650146)
\curveto(269.01298409,60.71649547)(268.9979841,60.67149552)(268.97799072,60.62150146)
\curveto(268.8979842,60.45149574)(268.83298427,60.27649591)(268.78299072,60.09650146)
\curveto(268.73298437,59.91649627)(268.66798443,59.73649645)(268.58799072,59.55650146)
\curveto(268.53798456,59.42649676)(268.48798461,59.2914969)(268.43799072,59.15150146)
\curveto(268.3979847,59.02149717)(268.34798475,58.8914973)(268.28799072,58.76150146)
\curveto(268.11798498,58.35149784)(267.96298514,57.93649825)(267.82299072,57.51650146)
\curveto(267.69298541,57.09649909)(267.54298556,56.68149951)(267.37299072,56.27150146)
\curveto(267.31298579,56.11150008)(267.25798584,55.95150024)(267.20799072,55.79150146)
\curveto(267.15798594,55.63150056)(267.097986,55.47150072)(267.02799072,55.31150146)
\curveto(266.97798612,55.20150099)(266.93298617,55.09650109)(266.89299072,54.99650146)
\curveto(266.86298624,54.90650128)(266.79298631,54.83650135)(266.68299072,54.78650146)
\curveto(266.62298648,54.75650143)(266.55298655,54.74150145)(266.47299072,54.74150146)
\lineto(266.24799072,54.74150146)
\lineto(265.78299072,54.74150146)
\curveto(265.63298747,54.75150144)(265.52298758,54.80150139)(265.45299072,54.89150146)
\curveto(265.38298772,54.97150122)(265.33298777,55.06650112)(265.30299072,55.17650146)
\curveto(265.27298783,55.29650089)(265.23298787,55.41150078)(265.18299072,55.52150146)
\curveto(265.12298798,55.66150053)(265.06298804,55.80650038)(265.00299072,55.95650146)
\curveto(264.95298815,56.11650007)(264.9029882,56.26649992)(264.85299072,56.40650146)
\curveto(264.83298827,56.45649973)(264.81798828,56.49649969)(264.80799072,56.52650146)
\curveto(264.7979883,56.56649962)(264.78298832,56.61149958)(264.76299072,56.66150146)
\curveto(264.56298854,57.14149905)(264.37798872,57.62649856)(264.20799072,58.11650146)
\curveto(264.04798905,58.60649758)(263.86798923,59.0914971)(263.66799072,59.57150146)
\curveto(263.60798949,59.73149646)(263.54798955,59.8864963)(263.48799072,60.03650146)
\curveto(263.43798966,60.19649599)(263.38298972,60.35649583)(263.32299072,60.51650146)
\lineto(263.26299072,60.66650146)
\curveto(263.25298985,60.72649546)(263.23798986,60.78149541)(263.21799072,60.83150146)
\curveto(263.13798996,61.00149519)(263.06799003,61.17149502)(263.00799072,61.34150146)
\curveto(262.95799014,61.51149468)(262.8979902,61.68149451)(262.82799072,61.85150146)
\curveto(262.80799029,61.91149428)(262.78299032,61.9914942)(262.75299072,62.09150146)
\curveto(262.72299038,62.191494)(262.72799037,62.27649391)(262.76799072,62.34650146)
\curveto(262.81799028,62.39649379)(262.87799022,62.43149376)(262.94799072,62.45150146)
\curveto(263.01799008,62.45149374)(263.05799004,62.45649373)(263.06799072,62.46650146)
}
}
{
\newrgbcolor{curcolor}{0 0 0}
\pscustom[linestyle=none,fillstyle=solid,fillcolor=curcolor]
{
\newpath
\moveto(277.91799072,58.94150146)
\curveto(277.93798266,58.88149731)(277.94798265,58.7864974)(277.94799072,58.65650146)
\curveto(277.94798265,58.53649765)(277.94298266,58.45149774)(277.93299072,58.40150146)
\lineto(277.93299072,58.25150146)
\curveto(277.92298268,58.17149802)(277.91298269,58.09649809)(277.90299072,58.02650146)
\curveto(277.9029827,57.96649822)(277.8979827,57.89649829)(277.88799072,57.81650146)
\curveto(277.86798273,57.75649843)(277.85298275,57.69649849)(277.84299072,57.63650146)
\curveto(277.84298276,57.57649861)(277.83298277,57.51649867)(277.81299072,57.45650146)
\curveto(277.77298283,57.32649886)(277.73798286,57.19649899)(277.70799072,57.06650146)
\curveto(277.67798292,56.93649925)(277.63798296,56.81649937)(277.58799072,56.70650146)
\curveto(277.37798322,56.22649996)(277.0979835,55.82150037)(276.74799072,55.49150146)
\curveto(276.3979842,55.17150102)(275.96798463,54.92650126)(275.45799072,54.75650146)
\curveto(275.34798525,54.71650147)(275.22798537,54.6865015)(275.09799072,54.66650146)
\curveto(274.97798562,54.64650154)(274.85298575,54.62650156)(274.72299072,54.60650146)
\curveto(274.66298594,54.59650159)(274.597986,54.5915016)(274.52799072,54.59150146)
\curveto(274.46798613,54.58150161)(274.40798619,54.57650161)(274.34799072,54.57650146)
\curveto(274.30798629,54.56650162)(274.24798635,54.56150163)(274.16799072,54.56150146)
\curveto(274.0979865,54.56150163)(274.04798655,54.56650162)(274.01799072,54.57650146)
\curveto(273.97798662,54.5865016)(273.93798666,54.5915016)(273.89799072,54.59150146)
\curveto(273.85798674,54.58150161)(273.82298678,54.58150161)(273.79299072,54.59150146)
\lineto(273.70299072,54.59150146)
\lineto(273.34299072,54.63650146)
\curveto(273.2029874,54.67650151)(273.06798753,54.71650147)(272.93799072,54.75650146)
\curveto(272.80798779,54.79650139)(272.68298792,54.84150135)(272.56299072,54.89150146)
\curveto(272.11298849,55.0915011)(271.74298886,55.35150084)(271.45299072,55.67150146)
\curveto(271.16298944,55.9915002)(270.92298968,56.38149981)(270.73299072,56.84150146)
\curveto(270.68298992,56.94149925)(270.64298996,57.04149915)(270.61299072,57.14150146)
\curveto(270.59299001,57.24149895)(270.57299003,57.34649884)(270.55299072,57.45650146)
\curveto(270.53299007,57.49649869)(270.52299008,57.52649866)(270.52299072,57.54650146)
\curveto(270.53299007,57.57649861)(270.53299007,57.61149858)(270.52299072,57.65150146)
\curveto(270.5029901,57.73149846)(270.48799011,57.81149838)(270.47799072,57.89150146)
\curveto(270.47799012,57.98149821)(270.46799013,58.06649812)(270.44799072,58.14650146)
\lineto(270.44799072,58.26650146)
\curveto(270.44799015,58.30649788)(270.44299016,58.35149784)(270.43299072,58.40150146)
\curveto(270.42299018,58.45149774)(270.41799018,58.53649765)(270.41799072,58.65650146)
\curveto(270.41799018,58.7864974)(270.42799017,58.88149731)(270.44799072,58.94150146)
\curveto(270.46799013,59.01149718)(270.47299013,59.08149711)(270.46299072,59.15150146)
\curveto(270.45299015,59.22149697)(270.45799014,59.2914969)(270.47799072,59.36150146)
\curveto(270.48799011,59.41149678)(270.49299011,59.45149674)(270.49299072,59.48150146)
\curveto(270.5029901,59.52149667)(270.51299009,59.56649662)(270.52299072,59.61650146)
\curveto(270.55299005,59.73649645)(270.57799002,59.85649633)(270.59799072,59.97650146)
\curveto(270.62798997,60.09649609)(270.66798993,60.21149598)(270.71799072,60.32150146)
\curveto(270.86798973,60.6914955)(271.04798955,61.02149517)(271.25799072,61.31150146)
\curveto(271.47798912,61.61149458)(271.74298886,61.86149433)(272.05299072,62.06150146)
\curveto(272.17298843,62.14149405)(272.2979883,62.20649398)(272.42799072,62.25650146)
\curveto(272.55798804,62.31649387)(272.69298791,62.37649381)(272.83299072,62.43650146)
\curveto(272.95298765,62.4864937)(273.08298752,62.51649367)(273.22299072,62.52650146)
\curveto(273.36298724,62.54649364)(273.5029871,62.57649361)(273.64299072,62.61650146)
\lineto(273.83799072,62.61650146)
\curveto(273.90798669,62.62649356)(273.97298663,62.63649355)(274.03299072,62.64650146)
\curveto(274.92298568,62.65649353)(275.66298494,62.47149372)(276.25299072,62.09150146)
\curveto(276.84298376,61.71149448)(277.26798333,61.21649497)(277.52799072,60.60650146)
\curveto(277.57798302,60.50649568)(277.61798298,60.40649578)(277.64799072,60.30650146)
\curveto(277.67798292,60.20649598)(277.71298289,60.10149609)(277.75299072,59.99150146)
\curveto(277.78298282,59.88149631)(277.80798279,59.76149643)(277.82799072,59.63150146)
\curveto(277.84798275,59.51149668)(277.87298273,59.3864968)(277.90299072,59.25650146)
\curveto(277.91298269,59.20649698)(277.91298269,59.15149704)(277.90299072,59.09150146)
\curveto(277.9029827,59.04149715)(277.90798269,58.9914972)(277.91799072,58.94150146)
\moveto(276.58299072,58.08650146)
\curveto(276.602984,58.15649803)(276.60798399,58.23649795)(276.59799072,58.32650146)
\lineto(276.59799072,58.58150146)
\curveto(276.597984,58.97149722)(276.56298404,59.30149689)(276.49299072,59.57150146)
\curveto(276.46298414,59.65149654)(276.43798416,59.73149646)(276.41799072,59.81150146)
\curveto(276.3979842,59.8914963)(276.37298423,59.96649622)(276.34299072,60.03650146)
\curveto(276.06298454,60.6864955)(275.61798498,61.13649505)(275.00799072,61.38650146)
\curveto(274.93798566,61.41649477)(274.86298574,61.43649475)(274.78299072,61.44650146)
\lineto(274.54299072,61.50650146)
\curveto(274.46298614,61.52649466)(274.37798622,61.53649465)(274.28799072,61.53650146)
\lineto(274.01799072,61.53650146)
\lineto(273.74799072,61.49150146)
\curveto(273.64798695,61.47149472)(273.55298705,61.44649474)(273.46299072,61.41650146)
\curveto(273.38298722,61.39649479)(273.3029873,61.36649482)(273.22299072,61.32650146)
\curveto(273.15298745,61.30649488)(273.08798751,61.27649491)(273.02799072,61.23650146)
\curveto(272.96798763,61.19649499)(272.91298769,61.15649503)(272.86299072,61.11650146)
\curveto(272.62298798,60.94649524)(272.42798817,60.74149545)(272.27799072,60.50150146)
\curveto(272.12798847,60.26149593)(271.9979886,59.98149621)(271.88799072,59.66150146)
\curveto(271.85798874,59.56149663)(271.83798876,59.45649673)(271.82799072,59.34650146)
\curveto(271.81798878,59.24649694)(271.8029888,59.14149705)(271.78299072,59.03150146)
\curveto(271.77298883,58.9914972)(271.76798883,58.92649726)(271.76799072,58.83650146)
\curveto(271.75798884,58.80649738)(271.75298885,58.77149742)(271.75299072,58.73150146)
\curveto(271.76298884,58.6914975)(271.76798883,58.64649754)(271.76799072,58.59650146)
\lineto(271.76799072,58.29650146)
\curveto(271.76798883,58.19649799)(271.77798882,58.10649808)(271.79799072,58.02650146)
\lineto(271.82799072,57.84650146)
\curveto(271.84798875,57.74649844)(271.86298874,57.64649854)(271.87299072,57.54650146)
\curveto(271.89298871,57.45649873)(271.92298868,57.37149882)(271.96299072,57.29150146)
\curveto(272.06298854,57.05149914)(272.17798842,56.82649936)(272.30799072,56.61650146)
\curveto(272.44798815,56.40649978)(272.61798798,56.23149996)(272.81799072,56.09150146)
\curveto(272.86798773,56.06150013)(272.91298769,56.03650015)(272.95299072,56.01650146)
\curveto(272.99298761,55.99650019)(273.03798756,55.97150022)(273.08799072,55.94150146)
\curveto(273.16798743,55.8915003)(273.25298735,55.84650034)(273.34299072,55.80650146)
\curveto(273.44298716,55.77650041)(273.54798705,55.74650044)(273.65799072,55.71650146)
\curveto(273.70798689,55.69650049)(273.75298685,55.6865005)(273.79299072,55.68650146)
\curveto(273.84298676,55.69650049)(273.89298671,55.69650049)(273.94299072,55.68650146)
\curveto(273.97298663,55.67650051)(274.03298657,55.66650052)(274.12299072,55.65650146)
\curveto(274.22298638,55.64650054)(274.2979863,55.65150054)(274.34799072,55.67150146)
\curveto(274.38798621,55.68150051)(274.42798617,55.68150051)(274.46799072,55.67150146)
\curveto(274.50798609,55.67150052)(274.54798605,55.68150051)(274.58799072,55.70150146)
\curveto(274.66798593,55.72150047)(274.74798585,55.73650045)(274.82799072,55.74650146)
\curveto(274.90798569,55.76650042)(274.98298562,55.7915004)(275.05299072,55.82150146)
\curveto(275.39298521,55.96150023)(275.66798493,56.15650003)(275.87799072,56.40650146)
\curveto(276.08798451,56.65649953)(276.26298434,56.95149924)(276.40299072,57.29150146)
\curveto(276.45298415,57.41149878)(276.48298412,57.53649865)(276.49299072,57.66650146)
\curveto(276.51298409,57.80649838)(276.54298406,57.94649824)(276.58299072,58.08650146)
}
}
{
\newrgbcolor{curcolor}{0 0 0}
\pscustom[linestyle=none,fillstyle=solid,fillcolor=curcolor]
{
\newpath
\moveto(281.83627197,62.64650146)
\curveto(282.55626791,62.65649353)(283.1612673,62.57149362)(283.65127197,62.39150146)
\curveto(284.14126632,62.22149397)(284.52126594,61.91649427)(284.79127197,61.47650146)
\curveto(284.8612656,61.36649482)(284.91626555,61.25149494)(284.95627197,61.13150146)
\curveto(284.99626547,61.02149517)(285.03626543,60.89649529)(285.07627197,60.75650146)
\curveto(285.09626537,60.6864955)(285.10126536,60.61149558)(285.09127197,60.53150146)
\curveto(285.08126538,60.46149573)(285.0662654,60.40649578)(285.04627197,60.36650146)
\curveto(285.02626544,60.34649584)(285.00126546,60.32649586)(284.97127197,60.30650146)
\curveto(284.94126552,60.29649589)(284.91626555,60.28149591)(284.89627197,60.26150146)
\curveto(284.84626562,60.24149595)(284.79626567,60.23649595)(284.74627197,60.24650146)
\curveto(284.69626577,60.25649593)(284.64626582,60.25649593)(284.59627197,60.24650146)
\curveto(284.51626595,60.22649596)(284.41126605,60.22149597)(284.28127197,60.23150146)
\curveto(284.15126631,60.25149594)(284.0612664,60.27649591)(284.01127197,60.30650146)
\curveto(283.93126653,60.35649583)(283.87626659,60.42149577)(283.84627197,60.50150146)
\curveto(283.82626664,60.5914956)(283.79126667,60.67649551)(283.74127197,60.75650146)
\curveto(283.65126681,60.91649527)(283.52626694,61.06149513)(283.36627197,61.19150146)
\curveto(283.25626721,61.27149492)(283.13626733,61.33149486)(283.00627197,61.37150146)
\curveto(282.87626759,61.41149478)(282.73626773,61.45149474)(282.58627197,61.49150146)
\curveto(282.53626793,61.51149468)(282.48626798,61.51649467)(282.43627197,61.50650146)
\curveto(282.38626808,61.50649468)(282.33626813,61.51149468)(282.28627197,61.52150146)
\curveto(282.22626824,61.54149465)(282.15126831,61.55149464)(282.06127197,61.55150146)
\curveto(281.97126849,61.55149464)(281.89626857,61.54149465)(281.83627197,61.52150146)
\lineto(281.74627197,61.52150146)
\lineto(281.59627197,61.49150146)
\curveto(281.54626892,61.4914947)(281.49626897,61.4864947)(281.44627197,61.47650146)
\curveto(281.18626928,61.41649477)(280.97126949,61.33149486)(280.80127197,61.22150146)
\curveto(280.63126983,61.11149508)(280.51626995,60.92649526)(280.45627197,60.66650146)
\curveto(280.43627003,60.59649559)(280.43127003,60.52649566)(280.44127197,60.45650146)
\curveto(280.46127,60.3864958)(280.48126998,60.32649586)(280.50127197,60.27650146)
\curveto(280.5612699,60.12649606)(280.63126983,60.01649617)(280.71127197,59.94650146)
\curveto(280.80126966,59.8864963)(280.91126955,59.81649637)(281.04127197,59.73650146)
\curveto(281.20126926,59.63649655)(281.38126908,59.56149663)(281.58127197,59.51150146)
\curveto(281.78126868,59.47149672)(281.98126848,59.42149677)(282.18127197,59.36150146)
\curveto(282.31126815,59.32149687)(282.44126802,59.2914969)(282.57127197,59.27150146)
\curveto(282.70126776,59.25149694)(282.83126763,59.22149697)(282.96127197,59.18150146)
\curveto(283.17126729,59.12149707)(283.37626709,59.06149713)(283.57627197,59.00150146)
\curveto(283.77626669,58.95149724)(283.97626649,58.8864973)(284.17627197,58.80650146)
\lineto(284.32627197,58.74650146)
\curveto(284.37626609,58.72649746)(284.42626604,58.70149749)(284.47627197,58.67150146)
\curveto(284.67626579,58.55149764)(284.85126561,58.41649777)(285.00127197,58.26650146)
\curveto(285.15126531,58.11649807)(285.27626519,57.92649826)(285.37627197,57.69650146)
\curveto(285.39626507,57.62649856)(285.41626505,57.53149866)(285.43627197,57.41150146)
\curveto(285.45626501,57.34149885)(285.466265,57.26649892)(285.46627197,57.18650146)
\curveto(285.47626499,57.11649907)(285.48126498,57.03649915)(285.48127197,56.94650146)
\lineto(285.48127197,56.79650146)
\curveto(285.461265,56.72649946)(285.45126501,56.65649953)(285.45127197,56.58650146)
\curveto(285.45126501,56.51649967)(285.44126502,56.44649974)(285.42127197,56.37650146)
\curveto(285.39126507,56.26649992)(285.35626511,56.16150003)(285.31627197,56.06150146)
\curveto(285.27626519,55.96150023)(285.23126523,55.87150032)(285.18127197,55.79150146)
\curveto(285.02126544,55.53150066)(284.81626565,55.32150087)(284.56627197,55.16150146)
\curveto(284.31626615,55.01150118)(284.03626643,54.88150131)(283.72627197,54.77150146)
\curveto(283.63626683,54.74150145)(283.54126692,54.72150147)(283.44127197,54.71150146)
\curveto(283.35126711,54.6915015)(283.2612672,54.66650152)(283.17127197,54.63650146)
\curveto(283.07126739,54.61650157)(282.97126749,54.60650158)(282.87127197,54.60650146)
\curveto(282.77126769,54.60650158)(282.67126779,54.59650159)(282.57127197,54.57650146)
\lineto(282.42127197,54.57650146)
\curveto(282.37126809,54.56650162)(282.30126816,54.56150163)(282.21127197,54.56150146)
\curveto(282.12126834,54.56150163)(282.05126841,54.56650162)(282.00127197,54.57650146)
\lineto(281.83627197,54.57650146)
\curveto(281.77626869,54.59650159)(281.71126875,54.60650158)(281.64127197,54.60650146)
\curveto(281.57126889,54.59650159)(281.51126895,54.60150159)(281.46127197,54.62150146)
\curveto(281.41126905,54.63150156)(281.34626912,54.63650155)(281.26627197,54.63650146)
\lineto(281.02627197,54.69650146)
\curveto(280.95626951,54.70650148)(280.88126958,54.72650146)(280.80127197,54.75650146)
\curveto(280.49126997,54.85650133)(280.22127024,54.98150121)(279.99127197,55.13150146)
\curveto(279.7612707,55.28150091)(279.5612709,55.47650071)(279.39127197,55.71650146)
\curveto(279.30127116,55.84650034)(279.22627124,55.98150021)(279.16627197,56.12150146)
\curveto(279.10627136,56.26149993)(279.05127141,56.41649977)(279.00127197,56.58650146)
\curveto(278.98127148,56.64649954)(278.97127149,56.71649947)(278.97127197,56.79650146)
\curveto(278.98127148,56.8864993)(278.99627147,56.95649923)(279.01627197,57.00650146)
\curveto(279.04627142,57.04649914)(279.09627137,57.0864991)(279.16627197,57.12650146)
\curveto(279.21627125,57.14649904)(279.28627118,57.15649903)(279.37627197,57.15650146)
\curveto(279.466271,57.16649902)(279.55627091,57.16649902)(279.64627197,57.15650146)
\curveto(279.73627073,57.14649904)(279.82127064,57.13149906)(279.90127197,57.11150146)
\curveto(279.99127047,57.10149909)(280.05127041,57.0864991)(280.08127197,57.06650146)
\curveto(280.15127031,57.01649917)(280.19627027,56.94149925)(280.21627197,56.84150146)
\curveto(280.24627022,56.75149944)(280.28127018,56.66649952)(280.32127197,56.58650146)
\curveto(280.42127004,56.36649982)(280.55626991,56.19649999)(280.72627197,56.07650146)
\curveto(280.84626962,55.9865002)(280.98126948,55.91650027)(281.13127197,55.86650146)
\curveto(281.28126918,55.81650037)(281.44126902,55.76650042)(281.61127197,55.71650146)
\lineto(281.92627197,55.67150146)
\lineto(282.01627197,55.67150146)
\curveto(282.08626838,55.65150054)(282.17626829,55.64150055)(282.28627197,55.64150146)
\curveto(282.40626806,55.64150055)(282.50626796,55.65150054)(282.58627197,55.67150146)
\curveto(282.65626781,55.67150052)(282.71126775,55.67650051)(282.75127197,55.68650146)
\curveto(282.81126765,55.69650049)(282.87126759,55.70150049)(282.93127197,55.70150146)
\curveto(282.99126747,55.71150048)(283.04626742,55.72150047)(283.09627197,55.73150146)
\curveto(283.38626708,55.81150038)(283.61626685,55.91650027)(283.78627197,56.04650146)
\curveto(283.95626651,56.17650001)(284.07626639,56.39649979)(284.14627197,56.70650146)
\curveto(284.1662663,56.75649943)(284.17126629,56.81149938)(284.16127197,56.87150146)
\curveto(284.15126631,56.93149926)(284.14126632,56.97649921)(284.13127197,57.00650146)
\curveto(284.08126638,57.19649899)(284.01126645,57.33649885)(283.92127197,57.42650146)
\curveto(283.83126663,57.52649866)(283.71626675,57.61649857)(283.57627197,57.69650146)
\curveto(283.48626698,57.75649843)(283.38626708,57.80649838)(283.27627197,57.84650146)
\lineto(282.94627197,57.96650146)
\curveto(282.91626755,57.97649821)(282.88626758,57.98149821)(282.85627197,57.98150146)
\curveto(282.83626763,57.98149821)(282.81126765,57.9914982)(282.78127197,58.01150146)
\curveto(282.44126802,58.12149807)(282.08626838,58.20149799)(281.71627197,58.25150146)
\curveto(281.35626911,58.31149788)(281.01626945,58.40649778)(280.69627197,58.53650146)
\curveto(280.59626987,58.57649761)(280.50126996,58.61149758)(280.41127197,58.64150146)
\curveto(280.32127014,58.67149752)(280.23627023,58.71149748)(280.15627197,58.76150146)
\curveto(279.9662705,58.87149732)(279.79127067,58.99649719)(279.63127197,59.13650146)
\curveto(279.47127099,59.27649691)(279.34627112,59.45149674)(279.25627197,59.66150146)
\curveto(279.22627124,59.73149646)(279.20127126,59.80149639)(279.18127197,59.87150146)
\curveto(279.17127129,59.94149625)(279.15627131,60.01649617)(279.13627197,60.09650146)
\curveto(279.10627136,60.21649597)(279.09627137,60.35149584)(279.10627197,60.50150146)
\curveto(279.11627135,60.66149553)(279.13127133,60.79649539)(279.15127197,60.90650146)
\curveto(279.17127129,60.95649523)(279.18127128,60.99649519)(279.18127197,61.02650146)
\curveto(279.19127127,61.06649512)(279.20627126,61.10649508)(279.22627197,61.14650146)
\curveto(279.31627115,61.37649481)(279.43627103,61.57649461)(279.58627197,61.74650146)
\curveto(279.74627072,61.91649427)(279.92627054,62.06649412)(280.12627197,62.19650146)
\curveto(280.27627019,62.2864939)(280.44127002,62.35649383)(280.62127197,62.40650146)
\curveto(280.80126966,62.46649372)(280.99126947,62.52149367)(281.19127197,62.57150146)
\curveto(281.2612692,62.58149361)(281.32626914,62.5914936)(281.38627197,62.60150146)
\curveto(281.45626901,62.61149358)(281.53126893,62.62149357)(281.61127197,62.63150146)
\curveto(281.64126882,62.64149355)(281.68126878,62.64149355)(281.73127197,62.63150146)
\curveto(281.78126868,62.62149357)(281.81626865,62.62649356)(281.83627197,62.64650146)
}
}
{
\newrgbcolor{curcolor}{0 0 0}
\pscustom[linestyle=none,fillstyle=solid,fillcolor=curcolor]
{
\newpath
\moveto(444.44404907,65.43651611)
\lineto(449.34904907,65.43651611)
\lineto(450.63904907,65.43651611)
\curveto(450.74904119,65.43650542)(450.85904108,65.43650542)(450.96904907,65.43651611)
\curveto(451.07904086,65.44650541)(451.16904077,65.42650543)(451.23904907,65.37651611)
\curveto(451.26904067,65.3565055)(451.29404065,65.33150552)(451.31404907,65.30151611)
\curveto(451.33404061,65.27150558)(451.35404059,65.24150561)(451.37404907,65.21151611)
\curveto(451.39404055,65.14150571)(451.40404054,65.02650583)(451.40404907,64.86651611)
\curveto(451.40404054,64.71650614)(451.39404055,64.60150625)(451.37404907,64.52151611)
\curveto(451.33404061,64.38150647)(451.24904069,64.30150655)(451.11904907,64.28151611)
\curveto(450.98904095,64.27150658)(450.83404111,64.26650659)(450.65404907,64.26651611)
\lineto(449.15404907,64.26651611)
\lineto(446.63404907,64.26651611)
\lineto(446.06404907,64.26651611)
\curveto(445.85404609,64.27650658)(445.69904624,64.2515066)(445.59904907,64.19151611)
\curveto(445.49904644,64.13150672)(445.4440465,64.02650683)(445.43404907,63.87651611)
\lineto(445.43404907,63.41151611)
\lineto(445.43404907,61.88151611)
\curveto(445.43404651,61.77150908)(445.42904651,61.64150921)(445.41904907,61.49151611)
\curveto(445.41904652,61.34150951)(445.42904651,61.22150963)(445.44904907,61.13151611)
\curveto(445.47904646,61.01150984)(445.5390464,60.93150992)(445.62904907,60.89151611)
\curveto(445.66904627,60.87150998)(445.7390462,60.85151)(445.83904907,60.83151611)
\lineto(445.98904907,60.83151611)
\curveto(446.02904591,60.82151003)(446.06904587,60.81651004)(446.10904907,60.81651611)
\curveto(446.15904578,60.82651003)(446.20904573,60.83151002)(446.25904907,60.83151611)
\lineto(446.76904907,60.83151611)
\lineto(449.70904907,60.83151611)
\lineto(450.00904907,60.83151611)
\curveto(450.11904182,60.84151001)(450.22904171,60.84151001)(450.33904907,60.83151611)
\curveto(450.45904148,60.83151002)(450.56404138,60.82151003)(450.65404907,60.80151611)
\curveto(450.75404119,60.79151006)(450.82904111,60.77151008)(450.87904907,60.74151611)
\curveto(450.90904103,60.72151013)(450.93404101,60.67651018)(450.95404907,60.60651611)
\curveto(450.97404097,60.53651032)(450.98904095,60.46151039)(450.99904907,60.38151611)
\curveto(451.00904093,60.30151055)(451.00904093,60.21651064)(450.99904907,60.12651611)
\curveto(450.99904094,60.04651081)(450.98904095,59.97651088)(450.96904907,59.91651611)
\curveto(450.94904099,59.82651103)(450.90404104,59.76151109)(450.83404907,59.72151611)
\curveto(450.81404113,59.70151115)(450.78404116,59.68651117)(450.74404907,59.67651611)
\curveto(450.71404123,59.67651118)(450.68404126,59.67151118)(450.65404907,59.66151611)
\lineto(450.56404907,59.66151611)
\curveto(450.51404143,59.6515112)(450.46404148,59.64651121)(450.41404907,59.64651611)
\curveto(450.36404158,59.6565112)(450.31404163,59.66151119)(450.26404907,59.66151611)
\lineto(449.70904907,59.66151611)
\lineto(446.54404907,59.66151611)
\lineto(446.18404907,59.66151611)
\curveto(446.07404587,59.67151118)(445.96904597,59.66651119)(445.86904907,59.64651611)
\curveto(445.76904617,59.63651122)(445.67904626,59.61151124)(445.59904907,59.57151611)
\curveto(445.52904641,59.53151132)(445.47904646,59.46151139)(445.44904907,59.36151611)
\curveto(445.42904651,59.30151155)(445.41904652,59.23151162)(445.41904907,59.15151611)
\curveto(445.42904651,59.07151178)(445.43404651,58.99151186)(445.43404907,58.91151611)
\lineto(445.43404907,58.07151611)
\lineto(445.43404907,56.64651611)
\curveto(445.43404651,56.50651435)(445.4390465,56.37651448)(445.44904907,56.25651611)
\curveto(445.45904648,56.14651471)(445.49904644,56.06651479)(445.56904907,56.01651611)
\curveto(445.6390463,55.96651489)(445.71904622,55.93651492)(445.80904907,55.92651611)
\lineto(446.10904907,55.92651611)
\lineto(447.06904907,55.92651611)
\lineto(449.84404907,55.92651611)
\lineto(450.69904907,55.92651611)
\lineto(450.93904907,55.92651611)
\curveto(451.01904092,55.93651492)(451.08904085,55.93151492)(451.14904907,55.91151611)
\curveto(451.26904067,55.87151498)(451.34904059,55.81651504)(451.38904907,55.74651611)
\curveto(451.40904053,55.71651514)(451.42404052,55.66651519)(451.43404907,55.59651611)
\curveto(451.4440405,55.52651533)(451.44904049,55.4515154)(451.44904907,55.37151611)
\curveto(451.45904048,55.30151555)(451.45904048,55.22651563)(451.44904907,55.14651611)
\curveto(451.4390405,55.07651578)(451.42904051,55.02151583)(451.41904907,54.98151611)
\curveto(451.37904056,54.90151595)(451.33404061,54.84651601)(451.28404907,54.81651611)
\curveto(451.22404072,54.77651608)(451.1440408,54.7565161)(451.04404907,54.75651611)
\lineto(450.77404907,54.75651611)
\lineto(449.72404907,54.75651611)
\lineto(445.73404907,54.75651611)
\lineto(444.68404907,54.75651611)
\curveto(444.5440474,54.7565161)(444.42404752,54.76151609)(444.32404907,54.77151611)
\curveto(444.22404772,54.79151606)(444.14904779,54.84151601)(444.09904907,54.92151611)
\curveto(444.05904788,54.98151587)(444.0390479,55.0565158)(444.03904907,55.14651611)
\lineto(444.03904907,55.43151611)
\lineto(444.03904907,56.48151611)
\lineto(444.03904907,60.50151611)
\lineto(444.03904907,63.86151611)
\lineto(444.03904907,64.79151611)
\lineto(444.03904907,65.06151611)
\curveto(444.0390479,65.1515057)(444.05904788,65.22150563)(444.09904907,65.27151611)
\curveto(444.1390478,65.34150551)(444.21404773,65.39150546)(444.32404907,65.42151611)
\curveto(444.3440476,65.43150542)(444.36404758,65.43150542)(444.38404907,65.42151611)
\curveto(444.40404754,65.42150543)(444.42404752,65.42650543)(444.44404907,65.43651611)
}
}
{
\newrgbcolor{curcolor}{0 0 0}
\pscustom[linestyle=none,fillstyle=solid,fillcolor=curcolor]
{
\newpath
\moveto(456.64397095,62.63151611)
\curveto(457.27396571,62.6515082)(457.77896521,62.56650829)(458.15897095,62.37651611)
\curveto(458.53896445,62.18650867)(458.84396414,61.90150895)(459.07397095,61.52151611)
\curveto(459.13396385,61.42150943)(459.17896381,61.31150954)(459.20897095,61.19151611)
\curveto(459.24896374,61.08150977)(459.2839637,60.96650989)(459.31397095,60.84651611)
\curveto(459.36396362,60.6565102)(459.39396359,60.4515104)(459.40397095,60.23151611)
\curveto(459.41396357,60.01151084)(459.41896357,59.78651107)(459.41897095,59.55651611)
\lineto(459.41897095,57.95151611)
\lineto(459.41897095,55.61151611)
\curveto(459.41896357,55.44151541)(459.41396357,55.27151558)(459.40397095,55.10151611)
\curveto(459.40396358,54.93151592)(459.33896365,54.82151603)(459.20897095,54.77151611)
\curveto(459.15896383,54.7515161)(459.10396388,54.74151611)(459.04397095,54.74151611)
\curveto(458.99396399,54.73151612)(458.93896405,54.72651613)(458.87897095,54.72651611)
\curveto(458.74896424,54.72651613)(458.62396436,54.73151612)(458.50397095,54.74151611)
\curveto(458.3839646,54.74151611)(458.29896469,54.78151607)(458.24897095,54.86151611)
\curveto(458.19896479,54.93151592)(458.17396481,55.02151583)(458.17397095,55.13151611)
\lineto(458.17397095,55.46151611)
\lineto(458.17397095,56.75151611)
\lineto(458.17397095,59.19651611)
\curveto(458.17396481,59.46651139)(458.16896482,59.73151112)(458.15897095,59.99151611)
\curveto(458.14896484,60.26151059)(458.10396488,60.49151036)(458.02397095,60.68151611)
\curveto(457.94396504,60.88150997)(457.82396516,61.04150981)(457.66397095,61.16151611)
\curveto(457.50396548,61.29150956)(457.31896567,61.39150946)(457.10897095,61.46151611)
\curveto(457.04896594,61.48150937)(456.983966,61.49150936)(456.91397095,61.49151611)
\curveto(456.85396613,61.50150935)(456.79396619,61.51650934)(456.73397095,61.53651611)
\curveto(456.6839663,61.54650931)(456.60396638,61.54650931)(456.49397095,61.53651611)
\curveto(456.39396659,61.53650932)(456.32396666,61.53150932)(456.28397095,61.52151611)
\curveto(456.24396674,61.50150935)(456.20896678,61.49150936)(456.17897095,61.49151611)
\curveto(456.14896684,61.50150935)(456.11396687,61.50150935)(456.07397095,61.49151611)
\curveto(455.94396704,61.46150939)(455.81896717,61.42650943)(455.69897095,61.38651611)
\curveto(455.5889674,61.3565095)(455.4839675,61.31150954)(455.38397095,61.25151611)
\curveto(455.34396764,61.23150962)(455.30896768,61.21150964)(455.27897095,61.19151611)
\curveto(455.24896774,61.17150968)(455.21396777,61.1515097)(455.17397095,61.13151611)
\curveto(454.82396816,60.88150997)(454.56896842,60.50651035)(454.40897095,60.00651611)
\curveto(454.37896861,59.92651093)(454.35896863,59.84151101)(454.34897095,59.75151611)
\curveto(454.33896865,59.67151118)(454.32396866,59.59151126)(454.30397095,59.51151611)
\curveto(454.2839687,59.46151139)(454.27896871,59.41151144)(454.28897095,59.36151611)
\curveto(454.29896869,59.32151153)(454.29396869,59.28151157)(454.27397095,59.24151611)
\lineto(454.27397095,58.92651611)
\curveto(454.26396872,58.89651196)(454.25896873,58.86151199)(454.25897095,58.82151611)
\curveto(454.26896872,58.78151207)(454.27396871,58.73651212)(454.27397095,58.68651611)
\lineto(454.27397095,58.23651611)
\lineto(454.27397095,56.79651611)
\lineto(454.27397095,55.47651611)
\lineto(454.27397095,55.13151611)
\curveto(454.27396871,55.02151583)(454.24896874,54.93151592)(454.19897095,54.86151611)
\curveto(454.14896884,54.78151607)(454.05896893,54.74151611)(453.92897095,54.74151611)
\curveto(453.80896918,54.73151612)(453.6839693,54.72651613)(453.55397095,54.72651611)
\curveto(453.47396951,54.72651613)(453.39896959,54.73151612)(453.32897095,54.74151611)
\curveto(453.25896973,54.7515161)(453.19896979,54.77651608)(453.14897095,54.81651611)
\curveto(453.06896992,54.86651599)(453.02896996,54.96151589)(453.02897095,55.10151611)
\lineto(453.02897095,55.50651611)
\lineto(453.02897095,57.27651611)
\lineto(453.02897095,60.90651611)
\lineto(453.02897095,61.82151611)
\lineto(453.02897095,62.09151611)
\curveto(453.02896996,62.18150867)(453.04896994,62.2515086)(453.08897095,62.30151611)
\curveto(453.11896987,62.36150849)(453.16896982,62.40150845)(453.23897095,62.42151611)
\curveto(453.27896971,62.43150842)(453.33396965,62.44150841)(453.40397095,62.45151611)
\curveto(453.4839695,62.46150839)(453.56396942,62.46650839)(453.64397095,62.46651611)
\curveto(453.72396926,62.46650839)(453.79896919,62.46150839)(453.86897095,62.45151611)
\curveto(453.94896904,62.44150841)(454.00396898,62.42650843)(454.03397095,62.40651611)
\curveto(454.14396884,62.33650852)(454.19396879,62.24650861)(454.18397095,62.13651611)
\curveto(454.17396881,62.03650882)(454.1889688,61.92150893)(454.22897095,61.79151611)
\curveto(454.24896874,61.73150912)(454.2889687,61.68150917)(454.34897095,61.64151611)
\curveto(454.46896852,61.63150922)(454.56396842,61.67650918)(454.63397095,61.77651611)
\curveto(454.71396827,61.87650898)(454.79396819,61.9565089)(454.87397095,62.01651611)
\curveto(455.01396797,62.11650874)(455.15396783,62.20650865)(455.29397095,62.28651611)
\curveto(455.44396754,62.37650848)(455.61396737,62.4515084)(455.80397095,62.51151611)
\curveto(455.8839671,62.54150831)(455.96896702,62.56150829)(456.05897095,62.57151611)
\curveto(456.15896683,62.58150827)(456.25396673,62.59650826)(456.34397095,62.61651611)
\curveto(456.39396659,62.62650823)(456.44396654,62.63150822)(456.49397095,62.63151611)
\lineto(456.64397095,62.63151611)
}
}
{
\newrgbcolor{curcolor}{0 0 0}
\pscustom[linestyle=none,fillstyle=solid,fillcolor=curcolor]
{
\newpath
\moveto(461.87358032,65.43651611)
\curveto(462.00357871,65.43650542)(462.13857857,65.43650542)(462.27858032,65.43651611)
\curveto(462.42857828,65.43650542)(462.53857817,65.40150545)(462.60858032,65.33151611)
\curveto(462.65857805,65.26150559)(462.68357803,65.16650569)(462.68358032,65.04651611)
\curveto(462.69357802,64.93650592)(462.69857801,64.82150603)(462.69858032,64.70151611)
\lineto(462.69858032,63.36651611)
\lineto(462.69858032,57.29151611)
\lineto(462.69858032,55.61151611)
\lineto(462.69858032,55.22151611)
\curveto(462.69857801,55.08151577)(462.67357804,54.97151588)(462.62358032,54.89151611)
\curveto(462.59357812,54.84151601)(462.54857816,54.81151604)(462.48858032,54.80151611)
\curveto(462.43857827,54.79151606)(462.37357834,54.77651608)(462.29358032,54.75651611)
\lineto(462.08358032,54.75651611)
\lineto(461.76858032,54.75651611)
\curveto(461.66857904,54.76651609)(461.59357912,54.80151605)(461.54358032,54.86151611)
\curveto(461.49357922,54.94151591)(461.46357925,55.04151581)(461.45358032,55.16151611)
\lineto(461.45358032,55.53651611)
\lineto(461.45358032,56.91651611)
\lineto(461.45358032,63.15651611)
\lineto(461.45358032,64.62651611)
\curveto(461.45357926,64.73650612)(461.44857926,64.851506)(461.43858032,64.97151611)
\curveto(461.43857927,65.10150575)(461.46357925,65.20150565)(461.51358032,65.27151611)
\curveto(461.55357916,65.33150552)(461.62857908,65.38150547)(461.73858032,65.42151611)
\curveto(461.75857895,65.43150542)(461.77857893,65.43150542)(461.79858032,65.42151611)
\curveto(461.82857888,65.42150543)(461.85357886,65.42650543)(461.87358032,65.43651611)
}
}
{
\newrgbcolor{curcolor}{0 0 0}
\pscustom[linestyle=none,fillstyle=solid,fillcolor=curcolor]
{
\newpath
\moveto(471.52842407,55.31151611)
\curveto(471.55841624,55.1515157)(471.54341626,55.01651584)(471.48342407,54.90651611)
\curveto(471.42341638,54.80651605)(471.34341646,54.73151612)(471.24342407,54.68151611)
\curveto(471.19341661,54.66151619)(471.13841666,54.6515162)(471.07842407,54.65151611)
\curveto(471.02841677,54.6515162)(470.97341683,54.64151621)(470.91342407,54.62151611)
\curveto(470.69341711,54.57151628)(470.47341733,54.58651627)(470.25342407,54.66651611)
\curveto(470.04341776,54.73651612)(469.8984179,54.82651603)(469.81842407,54.93651611)
\curveto(469.76841803,55.00651585)(469.72341808,55.08651577)(469.68342407,55.17651611)
\curveto(469.64341816,55.27651558)(469.59341821,55.3565155)(469.53342407,55.41651611)
\curveto(469.51341829,55.43651542)(469.48841831,55.4565154)(469.45842407,55.47651611)
\curveto(469.43841836,55.49651536)(469.40841839,55.50151535)(469.36842407,55.49151611)
\curveto(469.25841854,55.46151539)(469.15341865,55.40651545)(469.05342407,55.32651611)
\curveto(468.96341884,55.24651561)(468.87341893,55.17651568)(468.78342407,55.11651611)
\curveto(468.65341915,55.03651582)(468.51341929,54.96151589)(468.36342407,54.89151611)
\curveto(468.21341959,54.83151602)(468.05341975,54.77651608)(467.88342407,54.72651611)
\curveto(467.78342002,54.69651616)(467.67342013,54.67651618)(467.55342407,54.66651611)
\curveto(467.44342036,54.6565162)(467.33342047,54.64151621)(467.22342407,54.62151611)
\curveto(467.17342063,54.61151624)(467.12842067,54.60651625)(467.08842407,54.60651611)
\lineto(466.98342407,54.60651611)
\curveto(466.87342093,54.58651627)(466.76842103,54.58651627)(466.66842407,54.60651611)
\lineto(466.53342407,54.60651611)
\curveto(466.48342132,54.61651624)(466.43342137,54.62151623)(466.38342407,54.62151611)
\curveto(466.33342147,54.62151623)(466.28842151,54.63151622)(466.24842407,54.65151611)
\curveto(466.20842159,54.66151619)(466.17342163,54.66651619)(466.14342407,54.66651611)
\curveto(466.12342168,54.6565162)(466.0984217,54.6565162)(466.06842407,54.66651611)
\lineto(465.82842407,54.72651611)
\curveto(465.74842205,54.73651612)(465.67342213,54.7565161)(465.60342407,54.78651611)
\curveto(465.3034225,54.91651594)(465.05842274,55.06151579)(464.86842407,55.22151611)
\curveto(464.68842311,55.39151546)(464.53842326,55.62651523)(464.41842407,55.92651611)
\curveto(464.32842347,56.14651471)(464.28342352,56.41151444)(464.28342407,56.72151611)
\lineto(464.28342407,57.03651611)
\curveto(464.29342351,57.08651377)(464.2984235,57.13651372)(464.29842407,57.18651611)
\lineto(464.32842407,57.36651611)
\lineto(464.44842407,57.69651611)
\curveto(464.48842331,57.80651305)(464.53842326,57.90651295)(464.59842407,57.99651611)
\curveto(464.77842302,58.28651257)(465.02342278,58.50151235)(465.33342407,58.64151611)
\curveto(465.64342216,58.78151207)(465.98342182,58.90651195)(466.35342407,59.01651611)
\curveto(466.49342131,59.0565118)(466.63842116,59.08651177)(466.78842407,59.10651611)
\curveto(466.93842086,59.12651173)(467.08842071,59.1515117)(467.23842407,59.18151611)
\curveto(467.30842049,59.20151165)(467.37342043,59.21151164)(467.43342407,59.21151611)
\curveto(467.5034203,59.21151164)(467.57842022,59.22151163)(467.65842407,59.24151611)
\curveto(467.72842007,59.26151159)(467.79842,59.27151158)(467.86842407,59.27151611)
\curveto(467.93841986,59.28151157)(468.01341979,59.29651156)(468.09342407,59.31651611)
\curveto(468.34341946,59.37651148)(468.57841922,59.42651143)(468.79842407,59.46651611)
\curveto(469.01841878,59.51651134)(469.19341861,59.63151122)(469.32342407,59.81151611)
\curveto(469.38341842,59.89151096)(469.43341837,59.99151086)(469.47342407,60.11151611)
\curveto(469.51341829,60.24151061)(469.51341829,60.38151047)(469.47342407,60.53151611)
\curveto(469.41341839,60.77151008)(469.32341848,60.96150989)(469.20342407,61.10151611)
\curveto(469.09341871,61.24150961)(468.93341887,61.3515095)(468.72342407,61.43151611)
\curveto(468.6034192,61.48150937)(468.45841934,61.51650934)(468.28842407,61.53651611)
\curveto(468.12841967,61.5565093)(467.95841984,61.56650929)(467.77842407,61.56651611)
\curveto(467.5984202,61.56650929)(467.42342038,61.5565093)(467.25342407,61.53651611)
\curveto(467.08342072,61.51650934)(466.93842086,61.48650937)(466.81842407,61.44651611)
\curveto(466.64842115,61.38650947)(466.48342132,61.30150955)(466.32342407,61.19151611)
\curveto(466.24342156,61.13150972)(466.16842163,61.0515098)(466.09842407,60.95151611)
\curveto(466.03842176,60.86150999)(465.98342182,60.76151009)(465.93342407,60.65151611)
\curveto(465.9034219,60.57151028)(465.87342193,60.48651037)(465.84342407,60.39651611)
\curveto(465.82342198,60.30651055)(465.77842202,60.23651062)(465.70842407,60.18651611)
\curveto(465.66842213,60.1565107)(465.5984222,60.13151072)(465.49842407,60.11151611)
\curveto(465.40842239,60.10151075)(465.31342249,60.09651076)(465.21342407,60.09651611)
\curveto(465.11342269,60.09651076)(465.01342279,60.10151075)(464.91342407,60.11151611)
\curveto(464.82342298,60.13151072)(464.75842304,60.1565107)(464.71842407,60.18651611)
\curveto(464.67842312,60.21651064)(464.64842315,60.26651059)(464.62842407,60.33651611)
\curveto(464.60842319,60.40651045)(464.60842319,60.48151037)(464.62842407,60.56151611)
\curveto(464.65842314,60.69151016)(464.68842311,60.81151004)(464.71842407,60.92151611)
\curveto(464.75842304,61.04150981)(464.803423,61.1565097)(464.85342407,61.26651611)
\curveto(465.04342276,61.61650924)(465.28342252,61.88650897)(465.57342407,62.07651611)
\curveto(465.86342194,62.27650858)(466.22342158,62.43650842)(466.65342407,62.55651611)
\curveto(466.75342105,62.57650828)(466.85342095,62.59150826)(466.95342407,62.60151611)
\curveto(467.06342074,62.61150824)(467.17342063,62.62650823)(467.28342407,62.64651611)
\curveto(467.32342048,62.6565082)(467.38842041,62.6565082)(467.47842407,62.64651611)
\curveto(467.56842023,62.64650821)(467.62342018,62.6565082)(467.64342407,62.67651611)
\curveto(468.34341946,62.68650817)(468.95341885,62.60650825)(469.47342407,62.43651611)
\curveto(469.99341781,62.26650859)(470.35841744,61.94150891)(470.56842407,61.46151611)
\curveto(470.65841714,61.26150959)(470.70841709,61.02650983)(470.71842407,60.75651611)
\curveto(470.73841706,60.49651036)(470.74841705,60.22151063)(470.74842407,59.93151611)
\lineto(470.74842407,56.61651611)
\curveto(470.74841705,56.47651438)(470.75341705,56.34151451)(470.76342407,56.21151611)
\curveto(470.77341703,56.08151477)(470.803417,55.97651488)(470.85342407,55.89651611)
\curveto(470.9034169,55.82651503)(470.96841683,55.77651508)(471.04842407,55.74651611)
\curveto(471.13841666,55.70651515)(471.22341658,55.67651518)(471.30342407,55.65651611)
\curveto(471.38341642,55.64651521)(471.44341636,55.60151525)(471.48342407,55.52151611)
\curveto(471.5034163,55.49151536)(471.51341629,55.46151539)(471.51342407,55.43151611)
\curveto(471.51341629,55.40151545)(471.51841628,55.36151549)(471.52842407,55.31151611)
\moveto(469.38342407,56.97651611)
\curveto(469.44341836,57.11651374)(469.47341833,57.27651358)(469.47342407,57.45651611)
\curveto(469.48341832,57.64651321)(469.48841831,57.84151301)(469.48842407,58.04151611)
\curveto(469.48841831,58.1515127)(469.48341832,58.2515126)(469.47342407,58.34151611)
\curveto(469.46341834,58.43151242)(469.42341838,58.50151235)(469.35342407,58.55151611)
\curveto(469.32341848,58.57151228)(469.25341855,58.58151227)(469.14342407,58.58151611)
\curveto(469.12341868,58.56151229)(469.08841871,58.5515123)(469.03842407,58.55151611)
\curveto(468.98841881,58.5515123)(468.94341886,58.54151231)(468.90342407,58.52151611)
\curveto(468.82341898,58.50151235)(468.73341907,58.48151237)(468.63342407,58.46151611)
\lineto(468.33342407,58.40151611)
\curveto(468.3034195,58.40151245)(468.26841953,58.39651246)(468.22842407,58.38651611)
\lineto(468.12342407,58.38651611)
\curveto(467.97341983,58.34651251)(467.80841999,58.32151253)(467.62842407,58.31151611)
\curveto(467.45842034,58.31151254)(467.2984205,58.29151256)(467.14842407,58.25151611)
\curveto(467.06842073,58.23151262)(466.99342081,58.21151264)(466.92342407,58.19151611)
\curveto(466.86342094,58.18151267)(466.79342101,58.16651269)(466.71342407,58.14651611)
\curveto(466.55342125,58.09651276)(466.4034214,58.03151282)(466.26342407,57.95151611)
\curveto(466.12342168,57.88151297)(466.0034218,57.79151306)(465.90342407,57.68151611)
\curveto(465.803422,57.57151328)(465.72842207,57.43651342)(465.67842407,57.27651611)
\curveto(465.62842217,57.12651373)(465.60842219,56.94151391)(465.61842407,56.72151611)
\curveto(465.61842218,56.62151423)(465.63342217,56.52651433)(465.66342407,56.43651611)
\curveto(465.7034221,56.3565145)(465.74842205,56.28151457)(465.79842407,56.21151611)
\curveto(465.87842192,56.10151475)(465.98342182,56.00651485)(466.11342407,55.92651611)
\curveto(466.24342156,55.856515)(466.38342142,55.79651506)(466.53342407,55.74651611)
\curveto(466.58342122,55.73651512)(466.63342117,55.73151512)(466.68342407,55.73151611)
\curveto(466.73342107,55.73151512)(466.78342102,55.72651513)(466.83342407,55.71651611)
\curveto(466.9034209,55.69651516)(466.98842081,55.68151517)(467.08842407,55.67151611)
\curveto(467.1984206,55.67151518)(467.28842051,55.68151517)(467.35842407,55.70151611)
\curveto(467.41842038,55.72151513)(467.47842032,55.72651513)(467.53842407,55.71651611)
\curveto(467.5984202,55.71651514)(467.65842014,55.72651513)(467.71842407,55.74651611)
\curveto(467.79842,55.76651509)(467.87341993,55.78151507)(467.94342407,55.79151611)
\curveto(468.02341978,55.80151505)(468.0984197,55.82151503)(468.16842407,55.85151611)
\curveto(468.45841934,55.97151488)(468.7034191,56.11651474)(468.90342407,56.28651611)
\curveto(469.11341869,56.4565144)(469.27341853,56.68651417)(469.38342407,56.97651611)
}
}
{
\newrgbcolor{curcolor}{0 0 0}
\pscustom[linestyle=none,fillstyle=solid,fillcolor=curcolor]
{
\newpath
\moveto(475.8350647,62.66151611)
\curveto(476.57505991,62.67150818)(477.19005929,62.56150829)(477.6800647,62.33151611)
\curveto(478.1800583,62.11150874)(478.57505791,61.77650908)(478.8650647,61.32651611)
\curveto(478.99505749,61.12650973)(479.10505738,60.88150997)(479.1950647,60.59151611)
\curveto(479.21505727,60.54151031)(479.23005725,60.47651038)(479.2400647,60.39651611)
\curveto(479.25005723,60.31651054)(479.24505724,60.24651061)(479.2250647,60.18651611)
\curveto(479.19505729,60.13651072)(479.14505734,60.09151076)(479.0750647,60.05151611)
\curveto(479.04505744,60.03151082)(479.01505747,60.02151083)(478.9850647,60.02151611)
\curveto(478.95505753,60.03151082)(478.92005756,60.03151082)(478.8800647,60.02151611)
\curveto(478.84005764,60.01151084)(478.80005768,60.00651085)(478.7600647,60.00651611)
\curveto(478.72005776,60.01651084)(478.6800578,60.02151083)(478.6400647,60.02151611)
\lineto(478.3250647,60.02151611)
\curveto(478.22505826,60.03151082)(478.14005834,60.06151079)(478.0700647,60.11151611)
\curveto(477.99005849,60.17151068)(477.93505855,60.2565106)(477.9050647,60.36651611)
\curveto(477.87505861,60.47651038)(477.83505865,60.57151028)(477.7850647,60.65151611)
\curveto(477.63505885,60.91150994)(477.44005904,61.11650974)(477.2000647,61.26651611)
\curveto(477.12005936,61.31650954)(477.03505945,61.3565095)(476.9450647,61.38651611)
\curveto(476.85505963,61.42650943)(476.76005972,61.46150939)(476.6600647,61.49151611)
\curveto(476.52005996,61.53150932)(476.33506015,61.5515093)(476.1050647,61.55151611)
\curveto(475.87506061,61.56150929)(475.6850608,61.54150931)(475.5350647,61.49151611)
\curveto(475.46506102,61.47150938)(475.40006108,61.4565094)(475.3400647,61.44651611)
\curveto(475.2800612,61.43650942)(475.21506127,61.42150943)(475.1450647,61.40151611)
\curveto(474.8850616,61.29150956)(474.65506183,61.14150971)(474.4550647,60.95151611)
\curveto(474.25506223,60.76151009)(474.10006238,60.53651032)(473.9900647,60.27651611)
\curveto(473.95006253,60.18651067)(473.91506257,60.09151076)(473.8850647,59.99151611)
\curveto(473.85506263,59.90151095)(473.82506266,59.80151105)(473.7950647,59.69151611)
\lineto(473.7050647,59.28651611)
\curveto(473.69506279,59.23651162)(473.69006279,59.18151167)(473.6900647,59.12151611)
\curveto(473.70006278,59.06151179)(473.69506279,59.00651185)(473.6750647,58.95651611)
\lineto(473.6750647,58.83651611)
\curveto(473.66506282,58.79651206)(473.65506283,58.73151212)(473.6450647,58.64151611)
\curveto(473.64506284,58.5515123)(473.65506283,58.48651237)(473.6750647,58.44651611)
\curveto(473.6850628,58.39651246)(473.6850628,58.34651251)(473.6750647,58.29651611)
\curveto(473.66506282,58.24651261)(473.66506282,58.19651266)(473.6750647,58.14651611)
\curveto(473.6850628,58.10651275)(473.69006279,58.03651282)(473.6900647,57.93651611)
\curveto(473.71006277,57.856513)(473.72506276,57.77151308)(473.7350647,57.68151611)
\curveto(473.75506273,57.59151326)(473.77506271,57.50651335)(473.7950647,57.42651611)
\curveto(473.90506258,57.10651375)(474.03006245,56.82651403)(474.1700647,56.58651611)
\curveto(474.32006216,56.3565145)(474.52506196,56.1565147)(474.7850647,55.98651611)
\curveto(474.87506161,55.93651492)(474.96506152,55.89151496)(475.0550647,55.85151611)
\curveto(475.15506133,55.81151504)(475.26006122,55.77151508)(475.3700647,55.73151611)
\curveto(475.42006106,55.72151513)(475.46006102,55.71651514)(475.4900647,55.71651611)
\curveto(475.52006096,55.71651514)(475.56006092,55.71151514)(475.6100647,55.70151611)
\curveto(475.64006084,55.69151516)(475.69006079,55.68651517)(475.7600647,55.68651611)
\lineto(475.9250647,55.68651611)
\curveto(475.92506056,55.67651518)(475.94506054,55.67151518)(475.9850647,55.67151611)
\curveto(476.00506048,55.68151517)(476.03006045,55.68151517)(476.0600647,55.67151611)
\curveto(476.09006039,55.67151518)(476.12006036,55.67651518)(476.1500647,55.68651611)
\curveto(476.22006026,55.70651515)(476.2850602,55.71151514)(476.3450647,55.70151611)
\curveto(476.41506007,55.70151515)(476.48506,55.71151514)(476.5550647,55.73151611)
\curveto(476.81505967,55.81151504)(477.04005944,55.91151494)(477.2300647,56.03151611)
\curveto(477.42005906,56.16151469)(477.5800589,56.32651453)(477.7100647,56.52651611)
\curveto(477.76005872,56.60651425)(477.80505868,56.69151416)(477.8450647,56.78151611)
\lineto(477.9650647,57.05151611)
\curveto(477.9850585,57.13151372)(478.00505848,57.20651365)(478.0250647,57.27651611)
\curveto(478.05505843,57.3565135)(478.10505838,57.42151343)(478.1750647,57.47151611)
\curveto(478.20505828,57.50151335)(478.26505822,57.52151333)(478.3550647,57.53151611)
\curveto(478.44505804,57.5515133)(478.54005794,57.56151329)(478.6400647,57.56151611)
\curveto(478.75005773,57.57151328)(478.85005763,57.57151328)(478.9400647,57.56151611)
\curveto(479.04005744,57.5515133)(479.11005737,57.53151332)(479.1500647,57.50151611)
\curveto(479.21005727,57.46151339)(479.24505724,57.40151345)(479.2550647,57.32151611)
\curveto(479.27505721,57.24151361)(479.27505721,57.1565137)(479.2550647,57.06651611)
\curveto(479.20505728,56.91651394)(479.15505733,56.77151408)(479.1050647,56.63151611)
\curveto(479.06505742,56.50151435)(479.01005747,56.37151448)(478.9400647,56.24151611)
\curveto(478.79005769,55.94151491)(478.60005788,55.67651518)(478.3700647,55.44651611)
\curveto(478.15005833,55.21651564)(477.8800586,55.03151582)(477.5600647,54.89151611)
\curveto(477.480059,54.851516)(477.39505909,54.81651604)(477.3050647,54.78651611)
\curveto(477.21505927,54.76651609)(477.12005936,54.74151611)(477.0200647,54.71151611)
\curveto(476.91005957,54.67151618)(476.80005968,54.6515162)(476.6900647,54.65151611)
\curveto(476.5800599,54.64151621)(476.47006001,54.62651623)(476.3600647,54.60651611)
\curveto(476.32006016,54.58651627)(476.2800602,54.58151627)(476.2400647,54.59151611)
\curveto(476.20006028,54.60151625)(476.16006032,54.60151625)(476.1200647,54.59151611)
\lineto(475.9850647,54.59151611)
\lineto(475.7450647,54.59151611)
\curveto(475.67506081,54.58151627)(475.61006087,54.58651627)(475.5500647,54.60651611)
\lineto(475.4750647,54.60651611)
\lineto(475.1150647,54.65151611)
\curveto(474.9850615,54.69151616)(474.86006162,54.72651613)(474.7400647,54.75651611)
\curveto(474.62006186,54.78651607)(474.50506198,54.82651603)(474.3950647,54.87651611)
\curveto(474.03506245,55.03651582)(473.73506275,55.22651563)(473.4950647,55.44651611)
\curveto(473.26506322,55.66651519)(473.05006343,55.93651492)(472.8500647,56.25651611)
\curveto(472.80006368,56.33651452)(472.75506373,56.42651443)(472.7150647,56.52651611)
\lineto(472.5950647,56.82651611)
\curveto(472.54506394,56.93651392)(472.51006397,57.0515138)(472.4900647,57.17151611)
\curveto(472.47006401,57.29151356)(472.44506404,57.41151344)(472.4150647,57.53151611)
\curveto(472.40506408,57.57151328)(472.40006408,57.61151324)(472.4000647,57.65151611)
\curveto(472.40006408,57.69151316)(472.39506409,57.73151312)(472.3850647,57.77151611)
\curveto(472.36506412,57.83151302)(472.35506413,57.89651296)(472.3550647,57.96651611)
\curveto(472.36506412,58.03651282)(472.36006412,58.10151275)(472.3400647,58.16151611)
\lineto(472.3400647,58.31151611)
\curveto(472.33006415,58.36151249)(472.32506416,58.43151242)(472.3250647,58.52151611)
\curveto(472.32506416,58.61151224)(472.33006415,58.68151217)(472.3400647,58.73151611)
\curveto(472.35006413,58.78151207)(472.35006413,58.82651203)(472.3400647,58.86651611)
\curveto(472.34006414,58.90651195)(472.34506414,58.94651191)(472.3550647,58.98651611)
\curveto(472.37506411,59.0565118)(472.3800641,59.12651173)(472.3700647,59.19651611)
\curveto(472.37006411,59.26651159)(472.3800641,59.33151152)(472.4000647,59.39151611)
\curveto(472.44006404,59.56151129)(472.47506401,59.73151112)(472.5050647,59.90151611)
\curveto(472.53506395,60.07151078)(472.5800639,60.23151062)(472.6400647,60.38151611)
\curveto(472.85006363,60.90150995)(473.10506338,61.32150953)(473.4050647,61.64151611)
\curveto(473.70506278,61.96150889)(474.11506237,62.22650863)(474.6350647,62.43651611)
\curveto(474.74506174,62.48650837)(474.86506162,62.52150833)(474.9950647,62.54151611)
\curveto(475.12506136,62.56150829)(475.26006122,62.58650827)(475.4000647,62.61651611)
\curveto(475.47006101,62.62650823)(475.54006094,62.63150822)(475.6100647,62.63151611)
\curveto(475.6800608,62.64150821)(475.75506073,62.6515082)(475.8350647,62.66151611)
}
}
{
\newrgbcolor{curcolor}{0 0 0}
\pscustom[linestyle=none,fillstyle=solid,fillcolor=curcolor]
{
\newpath
\moveto(487.50670532,58.92651611)
\curveto(487.52669764,58.82651203)(487.52669764,58.71151214)(487.50670532,58.58151611)
\curveto(487.49669767,58.46151239)(487.4666977,58.37651248)(487.41670532,58.32651611)
\curveto(487.3666978,58.28651257)(487.29169787,58.2565126)(487.19170532,58.23651611)
\curveto(487.10169806,58.22651263)(486.99669817,58.22151263)(486.87670532,58.22151611)
\lineto(486.51670532,58.22151611)
\curveto(486.39669877,58.23151262)(486.29169887,58.23651262)(486.20170532,58.23651611)
\lineto(482.36170532,58.23651611)
\curveto(482.28170288,58.23651262)(482.20170296,58.23151262)(482.12170532,58.22151611)
\curveto(482.04170312,58.22151263)(481.97670319,58.20651265)(481.92670532,58.17651611)
\curveto(481.88670328,58.1565127)(481.84670332,58.11651274)(481.80670532,58.05651611)
\curveto(481.78670338,58.02651283)(481.7667034,57.98151287)(481.74670532,57.92151611)
\curveto(481.72670344,57.87151298)(481.72670344,57.82151303)(481.74670532,57.77151611)
\curveto(481.75670341,57.72151313)(481.7617034,57.67651318)(481.76170532,57.63651611)
\curveto(481.7617034,57.59651326)(481.7667034,57.5565133)(481.77670532,57.51651611)
\curveto(481.79670337,57.43651342)(481.81670335,57.3515135)(481.83670532,57.26151611)
\curveto(481.85670331,57.18151367)(481.88670328,57.10151375)(481.92670532,57.02151611)
\curveto(482.15670301,56.48151437)(482.53670263,56.09651476)(483.06670532,55.86651611)
\curveto(483.12670204,55.83651502)(483.19170197,55.81151504)(483.26170532,55.79151611)
\lineto(483.47170532,55.73151611)
\curveto(483.50170166,55.72151513)(483.55170161,55.71651514)(483.62170532,55.71651611)
\curveto(483.7617014,55.67651518)(483.94670122,55.6565152)(484.17670532,55.65651611)
\curveto(484.40670076,55.6565152)(484.59170057,55.67651518)(484.73170532,55.71651611)
\curveto(484.87170029,55.7565151)(484.99670017,55.79651506)(485.10670532,55.83651611)
\curveto(485.22669994,55.88651497)(485.33669983,55.94651491)(485.43670532,56.01651611)
\curveto(485.54669962,56.08651477)(485.64169952,56.16651469)(485.72170532,56.25651611)
\curveto(485.80169936,56.3565145)(485.87169929,56.46151439)(485.93170532,56.57151611)
\curveto(485.99169917,56.67151418)(486.04169912,56.77651408)(486.08170532,56.88651611)
\curveto(486.13169903,56.99651386)(486.21169895,57.07651378)(486.32170532,57.12651611)
\curveto(486.3616988,57.14651371)(486.42669874,57.16151369)(486.51670532,57.17151611)
\curveto(486.60669856,57.18151367)(486.69669847,57.18151367)(486.78670532,57.17151611)
\curveto(486.87669829,57.17151368)(486.9616982,57.16651369)(487.04170532,57.15651611)
\curveto(487.12169804,57.14651371)(487.17669799,57.12651373)(487.20670532,57.09651611)
\curveto(487.30669786,57.02651383)(487.33169783,56.91151394)(487.28170532,56.75151611)
\curveto(487.20169796,56.48151437)(487.09669807,56.24151461)(486.96670532,56.03151611)
\curveto(486.7666984,55.71151514)(486.53669863,55.44651541)(486.27670532,55.23651611)
\curveto(486.02669914,55.03651582)(485.70669946,54.87151598)(485.31670532,54.74151611)
\curveto(485.21669995,54.70151615)(485.11670005,54.67651618)(485.01670532,54.66651611)
\curveto(484.91670025,54.64651621)(484.81170035,54.62651623)(484.70170532,54.60651611)
\curveto(484.65170051,54.59651626)(484.60170056,54.59151626)(484.55170532,54.59151611)
\curveto(484.51170065,54.59151626)(484.4667007,54.58651627)(484.41670532,54.57651611)
\lineto(484.26670532,54.57651611)
\curveto(484.21670095,54.56651629)(484.15670101,54.56151629)(484.08670532,54.56151611)
\curveto(484.02670114,54.56151629)(483.97670119,54.56651629)(483.93670532,54.57651611)
\lineto(483.80170532,54.57651611)
\curveto(483.75170141,54.58651627)(483.70670146,54.59151626)(483.66670532,54.59151611)
\curveto(483.62670154,54.59151626)(483.58670158,54.59651626)(483.54670532,54.60651611)
\curveto(483.49670167,54.61651624)(483.44170172,54.62651623)(483.38170532,54.63651611)
\curveto(483.32170184,54.63651622)(483.2667019,54.64151621)(483.21670532,54.65151611)
\curveto(483.12670204,54.67151618)(483.03670213,54.69651616)(482.94670532,54.72651611)
\curveto(482.85670231,54.74651611)(482.77170239,54.77151608)(482.69170532,54.80151611)
\curveto(482.65170251,54.82151603)(482.61670255,54.83151602)(482.58670532,54.83151611)
\curveto(482.55670261,54.84151601)(482.52170264,54.856516)(482.48170532,54.87651611)
\curveto(482.33170283,54.94651591)(482.17170299,55.03151582)(482.00170532,55.13151611)
\curveto(481.71170345,55.32151553)(481.4617037,55.5515153)(481.25170532,55.82151611)
\curveto(481.05170411,56.10151475)(480.88170428,56.41151444)(480.74170532,56.75151611)
\curveto(480.69170447,56.86151399)(480.65170451,56.97651388)(480.62170532,57.09651611)
\curveto(480.60170456,57.21651364)(480.57170459,57.33651352)(480.53170532,57.45651611)
\curveto(480.52170464,57.49651336)(480.51670465,57.53151332)(480.51670532,57.56151611)
\curveto(480.51670465,57.59151326)(480.51170465,57.63151322)(480.50170532,57.68151611)
\curveto(480.48170468,57.76151309)(480.4667047,57.84651301)(480.45670532,57.93651611)
\curveto(480.44670472,58.02651283)(480.43170473,58.11651274)(480.41170532,58.20651611)
\lineto(480.41170532,58.41651611)
\curveto(480.40170476,58.4565124)(480.39170477,58.51151234)(480.38170532,58.58151611)
\curveto(480.38170478,58.66151219)(480.38670478,58.72651213)(480.39670532,58.77651611)
\lineto(480.39670532,58.94151611)
\curveto(480.41670475,58.99151186)(480.42170474,59.04151181)(480.41170532,59.09151611)
\curveto(480.41170475,59.1515117)(480.41670475,59.20651165)(480.42670532,59.25651611)
\curveto(480.4667047,59.41651144)(480.49670467,59.57651128)(480.51670532,59.73651611)
\curveto(480.54670462,59.89651096)(480.59170457,60.04651081)(480.65170532,60.18651611)
\curveto(480.70170446,60.29651056)(480.74670442,60.40651045)(480.78670532,60.51651611)
\curveto(480.83670433,60.63651022)(480.89170427,60.7515101)(480.95170532,60.86151611)
\curveto(481.17170399,61.21150964)(481.42170374,61.51150934)(481.70170532,61.76151611)
\curveto(481.98170318,62.02150883)(482.32670284,62.23650862)(482.73670532,62.40651611)
\curveto(482.85670231,62.4565084)(482.97670219,62.49150836)(483.09670532,62.51151611)
\curveto(483.22670194,62.54150831)(483.3617018,62.57150828)(483.50170532,62.60151611)
\curveto(483.55170161,62.61150824)(483.59670157,62.61650824)(483.63670532,62.61651611)
\curveto(483.67670149,62.62650823)(483.72170144,62.63150822)(483.77170532,62.63151611)
\curveto(483.79170137,62.64150821)(483.81670135,62.64150821)(483.84670532,62.63151611)
\curveto(483.87670129,62.62150823)(483.90170126,62.62650823)(483.92170532,62.64651611)
\curveto(484.34170082,62.6565082)(484.70670046,62.61150824)(485.01670532,62.51151611)
\curveto(485.32669984,62.42150843)(485.60669956,62.29650856)(485.85670532,62.13651611)
\curveto(485.90669926,62.11650874)(485.94669922,62.08650877)(485.97670532,62.04651611)
\curveto(486.00669916,62.01650884)(486.04169912,61.99150886)(486.08170532,61.97151611)
\curveto(486.161699,61.91150894)(486.24169892,61.84150901)(486.32170532,61.76151611)
\curveto(486.41169875,61.68150917)(486.48669868,61.60150925)(486.54670532,61.52151611)
\curveto(486.70669846,61.31150954)(486.84169832,61.11150974)(486.95170532,60.92151611)
\curveto(487.02169814,60.81151004)(487.07669809,60.69151016)(487.11670532,60.56151611)
\curveto(487.15669801,60.43151042)(487.20169796,60.30151055)(487.25170532,60.17151611)
\curveto(487.30169786,60.04151081)(487.33669783,59.90651095)(487.35670532,59.76651611)
\curveto(487.38669778,59.62651123)(487.42169774,59.48651137)(487.46170532,59.34651611)
\curveto(487.47169769,59.27651158)(487.47669769,59.20651165)(487.47670532,59.13651611)
\lineto(487.50670532,58.92651611)
\moveto(486.05170532,59.43651611)
\curveto(486.08169908,59.47651138)(486.10669906,59.52651133)(486.12670532,59.58651611)
\curveto(486.14669902,59.6565112)(486.14669902,59.72651113)(486.12670532,59.79651611)
\curveto(486.0666991,60.01651084)(485.98169918,60.22151063)(485.87170532,60.41151611)
\curveto(485.73169943,60.64151021)(485.57669959,60.83651002)(485.40670532,60.99651611)
\curveto(485.23669993,61.1565097)(485.01670015,61.29150956)(484.74670532,61.40151611)
\curveto(484.67670049,61.42150943)(484.60670056,61.43650942)(484.53670532,61.44651611)
\curveto(484.4667007,61.46650939)(484.39170077,61.48650937)(484.31170532,61.50651611)
\curveto(484.23170093,61.52650933)(484.14670102,61.53650932)(484.05670532,61.53651611)
\lineto(483.80170532,61.53651611)
\curveto(483.77170139,61.51650934)(483.73670143,61.50650935)(483.69670532,61.50651611)
\curveto(483.65670151,61.51650934)(483.62170154,61.51650934)(483.59170532,61.50651611)
\lineto(483.35170532,61.44651611)
\curveto(483.28170188,61.43650942)(483.21170195,61.42150943)(483.14170532,61.40151611)
\curveto(482.85170231,61.28150957)(482.61670255,61.13150972)(482.43670532,60.95151611)
\curveto(482.2667029,60.77151008)(482.11170305,60.54651031)(481.97170532,60.27651611)
\curveto(481.94170322,60.22651063)(481.91170325,60.16151069)(481.88170532,60.08151611)
\curveto(481.85170331,60.01151084)(481.82670334,59.93151092)(481.80670532,59.84151611)
\curveto(481.78670338,59.7515111)(481.78170338,59.66651119)(481.79170532,59.58651611)
\curveto(481.80170336,59.50651135)(481.83670333,59.44651141)(481.89670532,59.40651611)
\curveto(481.97670319,59.34651151)(482.11170305,59.31651154)(482.30170532,59.31651611)
\curveto(482.50170266,59.32651153)(482.67170249,59.33151152)(482.81170532,59.33151611)
\lineto(485.09170532,59.33151611)
\curveto(485.24169992,59.33151152)(485.42169974,59.32651153)(485.63170532,59.31651611)
\curveto(485.84169932,59.31651154)(485.98169918,59.3565115)(486.05170532,59.43651611)
}
}
{
\newrgbcolor{curcolor}{0 0 0}
\pscustom[linestyle=none,fillstyle=solid,fillcolor=curcolor]
{
\newpath
\moveto(491.24334595,62.66151611)
\curveto(491.96334188,62.67150818)(492.56834128,62.58650827)(493.05834595,62.40651611)
\curveto(493.5483403,62.23650862)(493.92833992,61.93150892)(494.19834595,61.49151611)
\curveto(494.26833958,61.38150947)(494.32333952,61.26650959)(494.36334595,61.14651611)
\curveto(494.40333944,61.03650982)(494.4433394,60.91150994)(494.48334595,60.77151611)
\curveto(494.50333934,60.70151015)(494.50833934,60.62651023)(494.49834595,60.54651611)
\curveto(494.48833936,60.47651038)(494.47333937,60.42151043)(494.45334595,60.38151611)
\curveto(494.43333941,60.36151049)(494.40833944,60.34151051)(494.37834595,60.32151611)
\curveto(494.3483395,60.31151054)(494.32333952,60.29651056)(494.30334595,60.27651611)
\curveto(494.25333959,60.2565106)(494.20333964,60.2515106)(494.15334595,60.26151611)
\curveto(494.10333974,60.27151058)(494.05333979,60.27151058)(494.00334595,60.26151611)
\curveto(493.92333992,60.24151061)(493.81834003,60.23651062)(493.68834595,60.24651611)
\curveto(493.55834029,60.26651059)(493.46834038,60.29151056)(493.41834595,60.32151611)
\curveto(493.33834051,60.37151048)(493.28334056,60.43651042)(493.25334595,60.51651611)
\curveto(493.23334061,60.60651025)(493.19834065,60.69151016)(493.14834595,60.77151611)
\curveto(493.05834079,60.93150992)(492.93334091,61.07650978)(492.77334595,61.20651611)
\curveto(492.66334118,61.28650957)(492.5433413,61.34650951)(492.41334595,61.38651611)
\curveto(492.28334156,61.42650943)(492.1433417,61.46650939)(491.99334595,61.50651611)
\curveto(491.9433419,61.52650933)(491.89334195,61.53150932)(491.84334595,61.52151611)
\curveto(491.79334205,61.52150933)(491.7433421,61.52650933)(491.69334595,61.53651611)
\curveto(491.63334221,61.5565093)(491.55834229,61.56650929)(491.46834595,61.56651611)
\curveto(491.37834247,61.56650929)(491.30334254,61.5565093)(491.24334595,61.53651611)
\lineto(491.15334595,61.53651611)
\lineto(491.00334595,61.50651611)
\curveto(490.95334289,61.50650935)(490.90334294,61.50150935)(490.85334595,61.49151611)
\curveto(490.59334325,61.43150942)(490.37834347,61.34650951)(490.20834595,61.23651611)
\curveto(490.03834381,61.12650973)(489.92334392,60.94150991)(489.86334595,60.68151611)
\curveto(489.843344,60.61151024)(489.83834401,60.54151031)(489.84834595,60.47151611)
\curveto(489.86834398,60.40151045)(489.88834396,60.34151051)(489.90834595,60.29151611)
\curveto(489.96834388,60.14151071)(490.03834381,60.03151082)(490.11834595,59.96151611)
\curveto(490.20834364,59.90151095)(490.31834353,59.83151102)(490.44834595,59.75151611)
\curveto(490.60834324,59.6515112)(490.78834306,59.57651128)(490.98834595,59.52651611)
\curveto(491.18834266,59.48651137)(491.38834246,59.43651142)(491.58834595,59.37651611)
\curveto(491.71834213,59.33651152)(491.848342,59.30651155)(491.97834595,59.28651611)
\curveto(492.10834174,59.26651159)(492.23834161,59.23651162)(492.36834595,59.19651611)
\curveto(492.57834127,59.13651172)(492.78334106,59.07651178)(492.98334595,59.01651611)
\curveto(493.18334066,58.96651189)(493.38334046,58.90151195)(493.58334595,58.82151611)
\lineto(493.73334595,58.76151611)
\curveto(493.78334006,58.74151211)(493.83334001,58.71651214)(493.88334595,58.68651611)
\curveto(494.08333976,58.56651229)(494.25833959,58.43151242)(494.40834595,58.28151611)
\curveto(494.55833929,58.13151272)(494.68333916,57.94151291)(494.78334595,57.71151611)
\curveto(494.80333904,57.64151321)(494.82333902,57.54651331)(494.84334595,57.42651611)
\curveto(494.86333898,57.3565135)(494.87333897,57.28151357)(494.87334595,57.20151611)
\curveto(494.88333896,57.13151372)(494.88833896,57.0515138)(494.88834595,56.96151611)
\lineto(494.88834595,56.81151611)
\curveto(494.86833898,56.74151411)(494.85833899,56.67151418)(494.85834595,56.60151611)
\curveto(494.85833899,56.53151432)(494.848339,56.46151439)(494.82834595,56.39151611)
\curveto(494.79833905,56.28151457)(494.76333908,56.17651468)(494.72334595,56.07651611)
\curveto(494.68333916,55.97651488)(494.63833921,55.88651497)(494.58834595,55.80651611)
\curveto(494.42833942,55.54651531)(494.22333962,55.33651552)(493.97334595,55.17651611)
\curveto(493.72334012,55.02651583)(493.4433404,54.89651596)(493.13334595,54.78651611)
\curveto(493.0433408,54.7565161)(492.9483409,54.73651612)(492.84834595,54.72651611)
\curveto(492.75834109,54.70651615)(492.66834118,54.68151617)(492.57834595,54.65151611)
\curveto(492.47834137,54.63151622)(492.37834147,54.62151623)(492.27834595,54.62151611)
\curveto(492.17834167,54.62151623)(492.07834177,54.61151624)(491.97834595,54.59151611)
\lineto(491.82834595,54.59151611)
\curveto(491.77834207,54.58151627)(491.70834214,54.57651628)(491.61834595,54.57651611)
\curveto(491.52834232,54.57651628)(491.45834239,54.58151627)(491.40834595,54.59151611)
\lineto(491.24334595,54.59151611)
\curveto(491.18334266,54.61151624)(491.11834273,54.62151623)(491.04834595,54.62151611)
\curveto(490.97834287,54.61151624)(490.91834293,54.61651624)(490.86834595,54.63651611)
\curveto(490.81834303,54.64651621)(490.75334309,54.6515162)(490.67334595,54.65151611)
\lineto(490.43334595,54.71151611)
\curveto(490.36334348,54.72151613)(490.28834356,54.74151611)(490.20834595,54.77151611)
\curveto(489.89834395,54.87151598)(489.62834422,54.99651586)(489.39834595,55.14651611)
\curveto(489.16834468,55.29651556)(488.96834488,55.49151536)(488.79834595,55.73151611)
\curveto(488.70834514,55.86151499)(488.63334521,55.99651486)(488.57334595,56.13651611)
\curveto(488.51334533,56.27651458)(488.45834539,56.43151442)(488.40834595,56.60151611)
\curveto(488.38834546,56.66151419)(488.37834547,56.73151412)(488.37834595,56.81151611)
\curveto(488.38834546,56.90151395)(488.40334544,56.97151388)(488.42334595,57.02151611)
\curveto(488.45334539,57.06151379)(488.50334534,57.10151375)(488.57334595,57.14151611)
\curveto(488.62334522,57.16151369)(488.69334515,57.17151368)(488.78334595,57.17151611)
\curveto(488.87334497,57.18151367)(488.96334488,57.18151367)(489.05334595,57.17151611)
\curveto(489.1433447,57.16151369)(489.22834462,57.14651371)(489.30834595,57.12651611)
\curveto(489.39834445,57.11651374)(489.45834439,57.10151375)(489.48834595,57.08151611)
\curveto(489.55834429,57.03151382)(489.60334424,56.9565139)(489.62334595,56.85651611)
\curveto(489.65334419,56.76651409)(489.68834416,56.68151417)(489.72834595,56.60151611)
\curveto(489.82834402,56.38151447)(489.96334388,56.21151464)(490.13334595,56.09151611)
\curveto(490.25334359,56.00151485)(490.38834346,55.93151492)(490.53834595,55.88151611)
\curveto(490.68834316,55.83151502)(490.848343,55.78151507)(491.01834595,55.73151611)
\lineto(491.33334595,55.68651611)
\lineto(491.42334595,55.68651611)
\curveto(491.49334235,55.66651519)(491.58334226,55.6565152)(491.69334595,55.65651611)
\curveto(491.81334203,55.6565152)(491.91334193,55.66651519)(491.99334595,55.68651611)
\curveto(492.06334178,55.68651517)(492.11834173,55.69151516)(492.15834595,55.70151611)
\curveto(492.21834163,55.71151514)(492.27834157,55.71651514)(492.33834595,55.71651611)
\curveto(492.39834145,55.72651513)(492.45334139,55.73651512)(492.50334595,55.74651611)
\curveto(492.79334105,55.82651503)(493.02334082,55.93151492)(493.19334595,56.06151611)
\curveto(493.36334048,56.19151466)(493.48334036,56.41151444)(493.55334595,56.72151611)
\curveto(493.57334027,56.77151408)(493.57834027,56.82651403)(493.56834595,56.88651611)
\curveto(493.55834029,56.94651391)(493.5483403,56.99151386)(493.53834595,57.02151611)
\curveto(493.48834036,57.21151364)(493.41834043,57.3515135)(493.32834595,57.44151611)
\curveto(493.23834061,57.54151331)(493.12334072,57.63151322)(492.98334595,57.71151611)
\curveto(492.89334095,57.77151308)(492.79334105,57.82151303)(492.68334595,57.86151611)
\lineto(492.35334595,57.98151611)
\curveto(492.32334152,57.99151286)(492.29334155,57.99651286)(492.26334595,57.99651611)
\curveto(492.2433416,57.99651286)(492.21834163,58.00651285)(492.18834595,58.02651611)
\curveto(491.848342,58.13651272)(491.49334235,58.21651264)(491.12334595,58.26651611)
\curveto(490.76334308,58.32651253)(490.42334342,58.42151243)(490.10334595,58.55151611)
\curveto(490.00334384,58.59151226)(489.90834394,58.62651223)(489.81834595,58.65651611)
\curveto(489.72834412,58.68651217)(489.6433442,58.72651213)(489.56334595,58.77651611)
\curveto(489.37334447,58.88651197)(489.19834465,59.01151184)(489.03834595,59.15151611)
\curveto(488.87834497,59.29151156)(488.75334509,59.46651139)(488.66334595,59.67651611)
\curveto(488.63334521,59.74651111)(488.60834524,59.81651104)(488.58834595,59.88651611)
\curveto(488.57834527,59.9565109)(488.56334528,60.03151082)(488.54334595,60.11151611)
\curveto(488.51334533,60.23151062)(488.50334534,60.36651049)(488.51334595,60.51651611)
\curveto(488.52334532,60.67651018)(488.53834531,60.81151004)(488.55834595,60.92151611)
\curveto(488.57834527,60.97150988)(488.58834526,61.01150984)(488.58834595,61.04151611)
\curveto(488.59834525,61.08150977)(488.61334523,61.12150973)(488.63334595,61.16151611)
\curveto(488.72334512,61.39150946)(488.843345,61.59150926)(488.99334595,61.76151611)
\curveto(489.15334469,61.93150892)(489.33334451,62.08150877)(489.53334595,62.21151611)
\curveto(489.68334416,62.30150855)(489.848344,62.37150848)(490.02834595,62.42151611)
\curveto(490.20834364,62.48150837)(490.39834345,62.53650832)(490.59834595,62.58651611)
\curveto(490.66834318,62.59650826)(490.73334311,62.60650825)(490.79334595,62.61651611)
\curveto(490.86334298,62.62650823)(490.93834291,62.63650822)(491.01834595,62.64651611)
\curveto(491.0483428,62.6565082)(491.08834276,62.6565082)(491.13834595,62.64651611)
\curveto(491.18834266,62.63650822)(491.22334262,62.64150821)(491.24334595,62.66151611)
}
}
{
\newrgbcolor{curcolor}{0 0 0}
\pscustom[linestyle=none,fillstyle=solid,fillcolor=curcolor]
{
\newpath
\moveto(79.44210083,76.16295776)
\lineto(79.44210083,75.90795776)
\curveto(79.45209313,75.827953)(79.44709313,75.75295307)(79.42710083,75.68295776)
\lineto(79.42710083,75.44295776)
\lineto(79.42710083,75.27795776)
\curveto(79.40709317,75.17795365)(79.39709318,75.07295375)(79.39710083,74.96295776)
\curveto(79.39709318,74.86295396)(79.38709319,74.76295406)(79.36710083,74.66295776)
\lineto(79.36710083,74.51295776)
\curveto(79.33709324,74.37295445)(79.31709326,74.23295459)(79.30710083,74.09295776)
\curveto(79.29709328,73.96295486)(79.27209331,73.83295499)(79.23210083,73.70295776)
\curveto(79.21209337,73.6229552)(79.19209339,73.53795529)(79.17210083,73.44795776)
\lineto(79.11210083,73.20795776)
\lineto(78.99210083,72.90795776)
\curveto(78.96209362,72.81795601)(78.92709365,72.7279561)(78.88710083,72.63795776)
\curveto(78.78709379,72.41795641)(78.65209393,72.20295662)(78.48210083,71.99295776)
\curveto(78.32209426,71.78295704)(78.14709443,71.61295721)(77.95710083,71.48295776)
\curveto(77.90709467,71.44295738)(77.84709473,71.40295742)(77.77710083,71.36295776)
\curveto(77.71709486,71.33295749)(77.65709492,71.29795753)(77.59710083,71.25795776)
\curveto(77.51709506,71.20795762)(77.42209516,71.16795766)(77.31210083,71.13795776)
\curveto(77.20209538,71.10795772)(77.09709548,71.07795775)(76.99710083,71.04795776)
\curveto(76.88709569,71.00795782)(76.7770958,70.98295784)(76.66710083,70.97295776)
\curveto(76.55709602,70.96295786)(76.44209614,70.94795788)(76.32210083,70.92795776)
\curveto(76.2820963,70.91795791)(76.23709634,70.91795791)(76.18710083,70.92795776)
\curveto(76.14709643,70.9279579)(76.10709647,70.9229579)(76.06710083,70.91295776)
\curveto(76.02709655,70.90295792)(75.97209661,70.89795793)(75.90210083,70.89795776)
\curveto(75.83209675,70.89795793)(75.7820968,70.90295792)(75.75210083,70.91295776)
\curveto(75.70209688,70.93295789)(75.65709692,70.93795789)(75.61710083,70.92795776)
\curveto(75.577097,70.91795791)(75.54209704,70.91795791)(75.51210083,70.92795776)
\lineto(75.42210083,70.92795776)
\curveto(75.36209722,70.94795788)(75.29709728,70.96295786)(75.22710083,70.97295776)
\curveto(75.16709741,70.97295785)(75.10209748,70.97795785)(75.03210083,70.98795776)
\curveto(74.86209772,71.03795779)(74.70209788,71.08795774)(74.55210083,71.13795776)
\curveto(74.40209818,71.18795764)(74.25709832,71.25295757)(74.11710083,71.33295776)
\curveto(74.06709851,71.37295745)(74.01209857,71.40295742)(73.95210083,71.42295776)
\curveto(73.90209868,71.45295737)(73.85209873,71.48795734)(73.80210083,71.52795776)
\curveto(73.56209902,71.70795712)(73.36209922,71.9279569)(73.20210083,72.18795776)
\curveto(73.04209954,72.44795638)(72.90209968,72.73295609)(72.78210083,73.04295776)
\curveto(72.72209986,73.18295564)(72.6770999,73.3229555)(72.64710083,73.46295776)
\curveto(72.61709996,73.61295521)(72.5821,73.76795506)(72.54210083,73.92795776)
\curveto(72.52210006,74.03795479)(72.50710007,74.14795468)(72.49710083,74.25795776)
\curveto(72.48710009,74.36795446)(72.47210011,74.47795435)(72.45210083,74.58795776)
\curveto(72.44210014,74.6279542)(72.43710014,74.66795416)(72.43710083,74.70795776)
\curveto(72.44710013,74.74795408)(72.44710013,74.78795404)(72.43710083,74.82795776)
\curveto(72.42710015,74.87795395)(72.42210016,74.9279539)(72.42210083,74.97795776)
\lineto(72.42210083,75.14295776)
\curveto(72.40210018,75.19295363)(72.39710018,75.24295358)(72.40710083,75.29295776)
\curveto(72.41710016,75.35295347)(72.41710016,75.40795342)(72.40710083,75.45795776)
\curveto(72.39710018,75.49795333)(72.39710018,75.54295328)(72.40710083,75.59295776)
\curveto(72.41710016,75.64295318)(72.41210017,75.69295313)(72.39210083,75.74295776)
\curveto(72.37210021,75.81295301)(72.36710021,75.88795294)(72.37710083,75.96795776)
\curveto(72.38710019,76.05795277)(72.39210019,76.14295268)(72.39210083,76.22295776)
\curveto(72.39210019,76.31295251)(72.38710019,76.41295241)(72.37710083,76.52295776)
\curveto(72.36710021,76.64295218)(72.37210021,76.74295208)(72.39210083,76.82295776)
\lineto(72.39210083,77.10795776)
\lineto(72.43710083,77.73795776)
\curveto(72.44710013,77.83795099)(72.45710012,77.93295089)(72.46710083,78.02295776)
\lineto(72.49710083,78.32295776)
\curveto(72.51710006,78.37295045)(72.52210006,78.4229504)(72.51210083,78.47295776)
\curveto(72.51210007,78.53295029)(72.52210006,78.58795024)(72.54210083,78.63795776)
\curveto(72.59209999,78.80795002)(72.63209995,78.97294985)(72.66210083,79.13295776)
\curveto(72.69209989,79.30294952)(72.74209984,79.46294936)(72.81210083,79.61295776)
\curveto(73.00209958,80.07294875)(73.22209936,80.44794838)(73.47210083,80.73795776)
\curveto(73.73209885,81.0279478)(74.09209849,81.27294755)(74.55210083,81.47295776)
\curveto(74.6820979,81.5229473)(74.81209777,81.55794727)(74.94210083,81.57795776)
\curveto(75.0820975,81.59794723)(75.22209736,81.6229472)(75.36210083,81.65295776)
\curveto(75.43209715,81.66294716)(75.49709708,81.66794716)(75.55710083,81.66795776)
\curveto(75.61709696,81.66794716)(75.6820969,81.67294715)(75.75210083,81.68295776)
\curveto(76.582096,81.70294712)(77.25209533,81.55294727)(77.76210083,81.23295776)
\curveto(78.27209431,80.9229479)(78.65209393,80.48294834)(78.90210083,79.91295776)
\curveto(78.95209363,79.79294903)(78.99709358,79.66794916)(79.03710083,79.53795776)
\curveto(79.0770935,79.40794942)(79.12209346,79.27294955)(79.17210083,79.13295776)
\curveto(79.19209339,79.05294977)(79.20709337,78.96794986)(79.21710083,78.87795776)
\lineto(79.27710083,78.63795776)
\curveto(79.30709327,78.5279503)(79.32209326,78.41795041)(79.32210083,78.30795776)
\curveto(79.33209325,78.19795063)(79.34709323,78.08795074)(79.36710083,77.97795776)
\curveto(79.38709319,77.9279509)(79.39209319,77.88295094)(79.38210083,77.84295776)
\curveto(79.3820932,77.80295102)(79.38709319,77.76295106)(79.39710083,77.72295776)
\curveto(79.40709317,77.67295115)(79.40709317,77.61795121)(79.39710083,77.55795776)
\curveto(79.39709318,77.50795132)(79.40209318,77.45795137)(79.41210083,77.40795776)
\lineto(79.41210083,77.27295776)
\curveto(79.43209315,77.21295161)(79.43209315,77.14295168)(79.41210083,77.06295776)
\curveto(79.40209318,76.99295183)(79.40709317,76.9279519)(79.42710083,76.86795776)
\curveto(79.43709314,76.83795199)(79.44209314,76.79795203)(79.44210083,76.74795776)
\lineto(79.44210083,76.62795776)
\lineto(79.44210083,76.16295776)
\moveto(77.89710083,73.83795776)
\curveto(77.99709458,74.15795467)(78.05709452,74.5229543)(78.07710083,74.93295776)
\curveto(78.09709448,75.34295348)(78.10709447,75.75295307)(78.10710083,76.16295776)
\curveto(78.10709447,76.59295223)(78.09709448,77.01295181)(78.07710083,77.42295776)
\curveto(78.05709452,77.83295099)(78.01209457,78.21795061)(77.94210083,78.57795776)
\curveto(77.87209471,78.93794989)(77.76209482,79.25794957)(77.61210083,79.53795776)
\curveto(77.47209511,79.827949)(77.2770953,80.06294876)(77.02710083,80.24295776)
\curveto(76.86709571,80.35294847)(76.68709589,80.43294839)(76.48710083,80.48295776)
\curveto(76.28709629,80.54294828)(76.04209654,80.57294825)(75.75210083,80.57295776)
\curveto(75.73209685,80.55294827)(75.69709688,80.54294828)(75.64710083,80.54295776)
\curveto(75.59709698,80.55294827)(75.55709702,80.55294827)(75.52710083,80.54295776)
\curveto(75.44709713,80.5229483)(75.37209721,80.50294832)(75.30210083,80.48295776)
\curveto(75.24209734,80.47294835)(75.1770974,80.45294837)(75.10710083,80.42295776)
\curveto(74.83709774,80.30294852)(74.61709796,80.13294869)(74.44710083,79.91295776)
\curveto(74.28709829,79.70294912)(74.15209843,79.45794937)(74.04210083,79.17795776)
\curveto(73.99209859,79.06794976)(73.95209863,78.94794988)(73.92210083,78.81795776)
\curveto(73.90209868,78.69795013)(73.8770987,78.57295025)(73.84710083,78.44295776)
\curveto(73.82709875,78.39295043)(73.81709876,78.33795049)(73.81710083,78.27795776)
\curveto(73.81709876,78.2279506)(73.81209877,78.17795065)(73.80210083,78.12795776)
\curveto(73.79209879,78.03795079)(73.7820988,77.94295088)(73.77210083,77.84295776)
\curveto(73.76209882,77.75295107)(73.75209883,77.65795117)(73.74210083,77.55795776)
\curveto(73.74209884,77.47795135)(73.73709884,77.39295143)(73.72710083,77.30295776)
\lineto(73.72710083,77.06295776)
\lineto(73.72710083,76.88295776)
\curveto(73.71709886,76.85295197)(73.71209887,76.81795201)(73.71210083,76.77795776)
\lineto(73.71210083,76.64295776)
\lineto(73.71210083,76.19295776)
\curveto(73.71209887,76.11295271)(73.70709887,76.0279528)(73.69710083,75.93795776)
\curveto(73.69709888,75.85795297)(73.70709887,75.78295304)(73.72710083,75.71295776)
\lineto(73.72710083,75.44295776)
\curveto(73.72709885,75.4229534)(73.72209886,75.39295343)(73.71210083,75.35295776)
\curveto(73.71209887,75.3229535)(73.71709886,75.29795353)(73.72710083,75.27795776)
\curveto(73.73709884,75.17795365)(73.74209884,75.07795375)(73.74210083,74.97795776)
\curveto(73.75209883,74.88795394)(73.76209882,74.78795404)(73.77210083,74.67795776)
\curveto(73.80209878,74.55795427)(73.81709876,74.43295439)(73.81710083,74.30295776)
\curveto(73.82709875,74.18295464)(73.85209873,74.06795476)(73.89210083,73.95795776)
\curveto(73.97209861,73.65795517)(74.05709852,73.39295543)(74.14710083,73.16295776)
\curveto(74.24709833,72.93295589)(74.39209819,72.71795611)(74.58210083,72.51795776)
\curveto(74.79209779,72.31795651)(75.05709752,72.16795666)(75.37710083,72.06795776)
\curveto(75.41709716,72.04795678)(75.45209713,72.03795679)(75.48210083,72.03795776)
\curveto(75.52209706,72.04795678)(75.56709701,72.04295678)(75.61710083,72.02295776)
\curveto(75.65709692,72.01295681)(75.72709685,72.00295682)(75.82710083,71.99295776)
\curveto(75.93709664,71.98295684)(76.02209656,71.98795684)(76.08210083,72.00795776)
\curveto(76.15209643,72.0279568)(76.22209636,72.03795679)(76.29210083,72.03795776)
\curveto(76.36209622,72.04795678)(76.42709615,72.06295676)(76.48710083,72.08295776)
\curveto(76.68709589,72.14295668)(76.86709571,72.2279566)(77.02710083,72.33795776)
\curveto(77.05709552,72.35795647)(77.0820955,72.37795645)(77.10210083,72.39795776)
\lineto(77.16210083,72.45795776)
\curveto(77.20209538,72.47795635)(77.25209533,72.51795631)(77.31210083,72.57795776)
\curveto(77.41209517,72.71795611)(77.49709508,72.84795598)(77.56710083,72.96795776)
\curveto(77.63709494,73.08795574)(77.70709487,73.23295559)(77.77710083,73.40295776)
\curveto(77.80709477,73.47295535)(77.82709475,73.54295528)(77.83710083,73.61295776)
\curveto(77.85709472,73.68295514)(77.8770947,73.75795507)(77.89710083,73.83795776)
}
}
{
\newrgbcolor{curcolor}{0 0 0}
\pscustom[linestyle=none,fillstyle=solid,fillcolor=curcolor]
{
\newpath
\moveto(58.99288208,155.96866333)
\curveto(59.68287745,155.9786527)(60.28287685,155.85865282)(60.79288208,155.60866333)
\curveto(61.31287582,155.35865332)(61.70787542,155.02365366)(61.97788208,154.60366333)
\curveto(62.0278751,154.52365416)(62.07287506,154.43365425)(62.11288208,154.33366333)
\curveto(62.15287498,154.24365444)(62.19787493,154.14865453)(62.24788208,154.04866333)
\curveto(62.28787484,153.94865473)(62.31787481,153.84865483)(62.33788208,153.74866333)
\curveto(62.35787477,153.64865503)(62.37787475,153.54365514)(62.39788208,153.43366333)
\curveto(62.41787471,153.3836553)(62.42287471,153.33865534)(62.41288208,153.29866333)
\curveto(62.40287473,153.25865542)(62.40787472,153.21365547)(62.42788208,153.16366333)
\curveto(62.43787469,153.11365557)(62.44287469,153.02865565)(62.44288208,152.90866333)
\curveto(62.44287469,152.79865588)(62.43787469,152.71365597)(62.42788208,152.65366333)
\curveto(62.40787472,152.59365609)(62.39787473,152.53365615)(62.39788208,152.47366333)
\curveto(62.40787472,152.41365627)(62.40287473,152.35365633)(62.38288208,152.29366333)
\curveto(62.34287479,152.15365653)(62.30787482,152.01865666)(62.27788208,151.88866333)
\curveto(62.24787488,151.75865692)(62.20787492,151.63365705)(62.15788208,151.51366333)
\curveto(62.09787503,151.37365731)(62.0278751,151.24865743)(61.94788208,151.13866333)
\curveto(61.87787525,151.02865765)(61.80287533,150.91865776)(61.72288208,150.80866333)
\lineto(61.66288208,150.74866333)
\curveto(61.65287548,150.72865795)(61.63787549,150.70865797)(61.61788208,150.68866333)
\curveto(61.49787563,150.52865815)(61.36287577,150.3836583)(61.21288208,150.25366333)
\curveto(61.06287607,150.12365856)(60.90287623,149.99865868)(60.73288208,149.87866333)
\curveto(60.42287671,149.65865902)(60.127877,149.45365923)(59.84788208,149.26366333)
\curveto(59.61787751,149.12365956)(59.38787774,148.98865969)(59.15788208,148.85866333)
\curveto(58.93787819,148.72865995)(58.71787841,148.59366009)(58.49788208,148.45366333)
\curveto(58.24787888,148.2836604)(58.00787912,148.10366058)(57.77788208,147.91366333)
\curveto(57.55787957,147.72366096)(57.36787976,147.49866118)(57.20788208,147.23866333)
\curveto(57.16787996,147.1786615)(57.13288,147.11866156)(57.10288208,147.05866333)
\curveto(57.07288006,147.00866167)(57.04288009,146.94366174)(57.01288208,146.86366333)
\curveto(56.99288014,146.79366189)(56.98788014,146.73366195)(56.99788208,146.68366333)
\curveto(57.01788011,146.61366207)(57.05288008,146.55866212)(57.10288208,146.51866333)
\curveto(57.15287998,146.48866219)(57.21287992,146.46866221)(57.28288208,146.45866333)
\lineto(57.52288208,146.45866333)
\lineto(58.27288208,146.45866333)
\lineto(61.07788208,146.45866333)
\lineto(61.73788208,146.45866333)
\curveto(61.8278753,146.45866222)(61.91287522,146.45366223)(61.99288208,146.44366333)
\curveto(62.07287506,146.44366224)(62.13787499,146.42366226)(62.18788208,146.38366333)
\curveto(62.23787489,146.34366234)(62.27787485,146.26866241)(62.30788208,146.15866333)
\curveto(62.34787478,146.05866262)(62.35787477,145.95866272)(62.33788208,145.85866333)
\lineto(62.33788208,145.72366333)
\curveto(62.31787481,145.65366303)(62.29787483,145.59366309)(62.27788208,145.54366333)
\curveto(62.25787487,145.49366319)(62.22287491,145.45366323)(62.17288208,145.42366333)
\curveto(62.12287501,145.3836633)(62.05287508,145.36366332)(61.96288208,145.36366333)
\lineto(61.69288208,145.36366333)
\lineto(60.79288208,145.36366333)
\lineto(57.28288208,145.36366333)
\lineto(56.21788208,145.36366333)
\curveto(56.13788099,145.36366332)(56.04788108,145.35866332)(55.94788208,145.34866333)
\curveto(55.84788128,145.34866333)(55.76288137,145.35866332)(55.69288208,145.37866333)
\curveto(55.48288165,145.44866323)(55.41788171,145.62866305)(55.49788208,145.91866333)
\curveto(55.50788162,145.95866272)(55.50788162,145.99366269)(55.49788208,146.02366333)
\curveto(55.49788163,146.06366262)(55.50788162,146.10866257)(55.52788208,146.15866333)
\curveto(55.54788158,146.23866244)(55.56788156,146.32366236)(55.58788208,146.41366333)
\curveto(55.60788152,146.50366218)(55.6328815,146.58866209)(55.66288208,146.66866333)
\curveto(55.82288131,147.15866152)(56.02288111,147.57366111)(56.26288208,147.91366333)
\curveto(56.44288069,148.16366052)(56.64788048,148.38866029)(56.87788208,148.58866333)
\curveto(57.10788002,148.79865988)(57.34787978,148.99365969)(57.59788208,149.17366333)
\curveto(57.85787927,149.35365933)(58.12287901,149.52365916)(58.39288208,149.68366333)
\curveto(58.67287846,149.85365883)(58.94287819,150.02865865)(59.20288208,150.20866333)
\curveto(59.31287782,150.28865839)(59.41787771,150.36365832)(59.51788208,150.43366333)
\curveto(59.6278775,150.50365818)(59.73787739,150.5786581)(59.84788208,150.65866333)
\curveto(59.88787724,150.68865799)(59.92287721,150.71865796)(59.95288208,150.74866333)
\curveto(59.99287714,150.78865789)(60.0328771,150.81865786)(60.07288208,150.83866333)
\curveto(60.21287692,150.94865773)(60.33787679,151.07365761)(60.44788208,151.21366333)
\curveto(60.46787666,151.24365744)(60.49287664,151.26865741)(60.52288208,151.28866333)
\curveto(60.55287658,151.31865736)(60.57787655,151.34865733)(60.59788208,151.37866333)
\curveto(60.67787645,151.4786572)(60.74287639,151.5786571)(60.79288208,151.67866333)
\curveto(60.85287628,151.7786569)(60.90787622,151.88865679)(60.95788208,152.00866333)
\curveto(60.98787614,152.0786566)(61.00787612,152.15365653)(61.01788208,152.23366333)
\lineto(61.07788208,152.47366333)
\lineto(61.07788208,152.56366333)
\curveto(61.08787604,152.59365609)(61.09287604,152.62365606)(61.09288208,152.65366333)
\curveto(61.11287602,152.72365596)(61.11787601,152.81865586)(61.10788208,152.93866333)
\curveto(61.10787602,153.06865561)(61.09787603,153.16865551)(61.07788208,153.23866333)
\curveto(61.05787607,153.31865536)(61.03787609,153.39365529)(61.01788208,153.46366333)
\curveto(61.00787612,153.54365514)(60.98787614,153.62365506)(60.95788208,153.70366333)
\curveto(60.84787628,153.94365474)(60.69787643,154.14365454)(60.50788208,154.30366333)
\curveto(60.3278768,154.47365421)(60.10787702,154.61365407)(59.84788208,154.72366333)
\curveto(59.77787735,154.74365394)(59.70787742,154.75865392)(59.63788208,154.76866333)
\curveto(59.56787756,154.78865389)(59.49287764,154.80865387)(59.41288208,154.82866333)
\curveto(59.3328778,154.84865383)(59.22287791,154.85865382)(59.08288208,154.85866333)
\curveto(58.95287818,154.85865382)(58.84787828,154.84865383)(58.76788208,154.82866333)
\curveto(58.70787842,154.81865386)(58.65287848,154.81365387)(58.60288208,154.81366333)
\curveto(58.55287858,154.81365387)(58.50287863,154.80365388)(58.45288208,154.78366333)
\curveto(58.35287878,154.74365394)(58.25787887,154.70365398)(58.16788208,154.66366333)
\curveto(58.08787904,154.62365406)(58.00787912,154.5786541)(57.92788208,154.52866333)
\curveto(57.89787923,154.50865417)(57.86787926,154.4836542)(57.83788208,154.45366333)
\curveto(57.81787931,154.42365426)(57.79287934,154.39865428)(57.76288208,154.37866333)
\lineto(57.68788208,154.30366333)
\curveto(57.65787947,154.2836544)(57.6328795,154.26365442)(57.61288208,154.24366333)
\lineto(57.46288208,154.03366333)
\curveto(57.42287971,153.97365471)(57.37787975,153.90865477)(57.32788208,153.83866333)
\curveto(57.26787986,153.74865493)(57.21787991,153.64365504)(57.17788208,153.52366333)
\curveto(57.14787998,153.41365527)(57.11288002,153.30365538)(57.07288208,153.19366333)
\curveto(57.0328801,153.0836556)(57.00788012,152.93865574)(56.99788208,152.75866333)
\curveto(56.98788014,152.58865609)(56.95788017,152.46365622)(56.90788208,152.38366333)
\curveto(56.85788027,152.30365638)(56.78288035,152.25865642)(56.68288208,152.24866333)
\curveto(56.58288055,152.23865644)(56.47288066,152.23365645)(56.35288208,152.23366333)
\curveto(56.31288082,152.23365645)(56.27288086,152.22865645)(56.23288208,152.21866333)
\curveto(56.19288094,152.21865646)(56.15788097,152.22365646)(56.12788208,152.23366333)
\curveto(56.07788105,152.25365643)(56.0278811,152.26365642)(55.97788208,152.26366333)
\curveto(55.93788119,152.26365642)(55.89788123,152.27365641)(55.85788208,152.29366333)
\curveto(55.76788136,152.35365633)(55.72288141,152.48865619)(55.72288208,152.69866333)
\lineto(55.72288208,152.81866333)
\curveto(55.7328814,152.8786558)(55.73788139,152.93865574)(55.73788208,152.99866333)
\curveto(55.74788138,153.06865561)(55.75788137,153.13365555)(55.76788208,153.19366333)
\curveto(55.78788134,153.30365538)(55.80788132,153.40365528)(55.82788208,153.49366333)
\curveto(55.84788128,153.59365509)(55.87788125,153.68865499)(55.91788208,153.77866333)
\curveto(55.93788119,153.84865483)(55.95788117,153.90865477)(55.97788208,153.95866333)
\lineto(56.03788208,154.13866333)
\curveto(56.15788097,154.39865428)(56.31288082,154.64365404)(56.50288208,154.87366333)
\curveto(56.70288043,155.10365358)(56.91788021,155.28865339)(57.14788208,155.42866333)
\curveto(57.25787987,155.50865317)(57.37287976,155.57365311)(57.49288208,155.62366333)
\lineto(57.88288208,155.77366333)
\curveto(57.99287914,155.82365286)(58.10787902,155.85365283)(58.22788208,155.86366333)
\curveto(58.34787878,155.8836528)(58.47287866,155.90865277)(58.60288208,155.93866333)
\curveto(58.67287846,155.93865274)(58.73787839,155.93865274)(58.79788208,155.93866333)
\curveto(58.85787827,155.94865273)(58.92287821,155.95865272)(58.99288208,155.96866333)
}
}
{
\newrgbcolor{curcolor}{0 0 0}
\pscustom[linestyle=none,fillstyle=solid,fillcolor=curcolor]
{
\newpath
\moveto(65.60249146,155.77366333)
\lineto(69.20249146,155.77366333)
\lineto(69.84749146,155.77366333)
\curveto(69.92748493,155.77365291)(70.00248485,155.76865291)(70.07249146,155.75866333)
\curveto(70.14248471,155.75865292)(70.20248465,155.74865293)(70.25249146,155.72866333)
\curveto(70.32248453,155.69865298)(70.37748448,155.63865304)(70.41749146,155.54866333)
\curveto(70.43748442,155.51865316)(70.44748441,155.4786532)(70.44749146,155.42866333)
\lineto(70.44749146,155.29366333)
\curveto(70.4574844,155.1836535)(70.4524844,155.0786536)(70.43249146,154.97866333)
\curveto(70.42248443,154.8786538)(70.38748447,154.80865387)(70.32749146,154.76866333)
\curveto(70.23748462,154.69865398)(70.10248475,154.66365402)(69.92249146,154.66366333)
\curveto(69.74248511,154.67365401)(69.57748528,154.678654)(69.42749146,154.67866333)
\lineto(67.43249146,154.67866333)
\lineto(66.93749146,154.67866333)
\lineto(66.80249146,154.67866333)
\curveto(66.76248809,154.678654)(66.72248813,154.67365401)(66.68249146,154.66366333)
\lineto(66.47249146,154.66366333)
\curveto(66.36248849,154.63365405)(66.28248857,154.59365409)(66.23249146,154.54366333)
\curveto(66.18248867,154.50365418)(66.14748871,154.44865423)(66.12749146,154.37866333)
\curveto(66.10748875,154.31865436)(66.09248876,154.24865443)(66.08249146,154.16866333)
\curveto(66.07248878,154.08865459)(66.0524888,153.99865468)(66.02249146,153.89866333)
\curveto(65.97248888,153.69865498)(65.93248892,153.49365519)(65.90249146,153.28366333)
\curveto(65.87248898,153.07365561)(65.83248902,152.86865581)(65.78249146,152.66866333)
\curveto(65.76248909,152.59865608)(65.7524891,152.52865615)(65.75249146,152.45866333)
\curveto(65.7524891,152.39865628)(65.74248911,152.33365635)(65.72249146,152.26366333)
\curveto(65.71248914,152.23365645)(65.70248915,152.19365649)(65.69249146,152.14366333)
\curveto(65.69248916,152.10365658)(65.69748916,152.06365662)(65.70749146,152.02366333)
\curveto(65.72748913,151.97365671)(65.7524891,151.92865675)(65.78249146,151.88866333)
\curveto(65.82248903,151.85865682)(65.88248897,151.85365683)(65.96249146,151.87366333)
\curveto(66.02248883,151.89365679)(66.08248877,151.91865676)(66.14249146,151.94866333)
\curveto(66.20248865,151.98865669)(66.26248859,152.02365666)(66.32249146,152.05366333)
\curveto(66.38248847,152.07365661)(66.43248842,152.08865659)(66.47249146,152.09866333)
\curveto(66.66248819,152.1786565)(66.86748799,152.23365645)(67.08749146,152.26366333)
\curveto(67.31748754,152.29365639)(67.54748731,152.30365638)(67.77749146,152.29366333)
\curveto(68.01748684,152.29365639)(68.24748661,152.26865641)(68.46749146,152.21866333)
\curveto(68.68748617,152.1786565)(68.88748597,152.11865656)(69.06749146,152.03866333)
\curveto(69.11748574,152.01865666)(69.16248569,151.99865668)(69.20249146,151.97866333)
\curveto(69.2524856,151.95865672)(69.30248555,151.93365675)(69.35249146,151.90366333)
\curveto(69.70248515,151.69365699)(69.98248487,151.46365722)(70.19249146,151.21366333)
\curveto(70.41248444,150.96365772)(70.60748425,150.63865804)(70.77749146,150.23866333)
\curveto(70.82748403,150.12865855)(70.86248399,150.01865866)(70.88249146,149.90866333)
\curveto(70.90248395,149.79865888)(70.92748393,149.683659)(70.95749146,149.56366333)
\curveto(70.96748389,149.53365915)(70.97248388,149.48865919)(70.97249146,149.42866333)
\curveto(70.99248386,149.36865931)(71.00248385,149.29865938)(71.00249146,149.21866333)
\curveto(71.00248385,149.14865953)(71.01248384,149.0836596)(71.03249146,149.02366333)
\lineto(71.03249146,148.85866333)
\curveto(71.04248381,148.80865987)(71.04748381,148.73865994)(71.04749146,148.64866333)
\curveto(71.04748381,148.55866012)(71.03748382,148.48866019)(71.01749146,148.43866333)
\curveto(70.99748386,148.3786603)(70.99248386,148.31866036)(71.00249146,148.25866333)
\curveto(71.01248384,148.20866047)(71.00748385,148.15866052)(70.98749146,148.10866333)
\curveto(70.94748391,147.94866073)(70.91248394,147.79866088)(70.88249146,147.65866333)
\curveto(70.852484,147.51866116)(70.80748405,147.3836613)(70.74749146,147.25366333)
\curveto(70.58748427,146.8836618)(70.36748449,146.54866213)(70.08749146,146.24866333)
\curveto(69.80748505,145.94866273)(69.48748537,145.71866296)(69.12749146,145.55866333)
\curveto(68.9574859,145.4786632)(68.7574861,145.40366328)(68.52749146,145.33366333)
\curveto(68.41748644,145.29366339)(68.30248655,145.26866341)(68.18249146,145.25866333)
\curveto(68.06248679,145.24866343)(67.94248691,145.22866345)(67.82249146,145.19866333)
\curveto(67.77248708,145.1786635)(67.71748714,145.1786635)(67.65749146,145.19866333)
\curveto(67.59748726,145.20866347)(67.53748732,145.20366348)(67.47749146,145.18366333)
\curveto(67.37748748,145.16366352)(67.27748758,145.16366352)(67.17749146,145.18366333)
\lineto(67.04249146,145.18366333)
\curveto(66.99248786,145.20366348)(66.93248792,145.21366347)(66.86249146,145.21366333)
\curveto(66.80248805,145.20366348)(66.74748811,145.20866347)(66.69749146,145.22866333)
\curveto(66.6574882,145.23866344)(66.62248823,145.24366344)(66.59249146,145.24366333)
\curveto(66.56248829,145.24366344)(66.52748833,145.24866343)(66.48749146,145.25866333)
\lineto(66.21749146,145.31866333)
\curveto(66.12748873,145.33866334)(66.04248881,145.36866331)(65.96249146,145.40866333)
\curveto(65.62248923,145.54866313)(65.33248952,145.70366298)(65.09249146,145.87366333)
\curveto(64.85249,146.05366263)(64.63249022,146.2836624)(64.43249146,146.56366333)
\curveto(64.28249057,146.79366189)(64.16749069,147.03366165)(64.08749146,147.28366333)
\curveto(64.06749079,147.33366135)(64.0574908,147.3786613)(64.05749146,147.41866333)
\curveto(64.0574908,147.46866121)(64.04749081,147.51866116)(64.02749146,147.56866333)
\curveto(64.00749085,147.62866105)(63.99249086,147.70866097)(63.98249146,147.80866333)
\curveto(63.98249087,147.90866077)(64.00249085,147.9836607)(64.04249146,148.03366333)
\curveto(64.09249076,148.11366057)(64.17249068,148.15866052)(64.28249146,148.16866333)
\curveto(64.39249046,148.1786605)(64.50749035,148.1836605)(64.62749146,148.18366333)
\lineto(64.79249146,148.18366333)
\curveto(64.85249,148.1836605)(64.90748995,148.17366051)(64.95749146,148.15366333)
\curveto(65.04748981,148.13366055)(65.11748974,148.09366059)(65.16749146,148.03366333)
\curveto(65.23748962,147.94366074)(65.28248957,147.83366085)(65.30249146,147.70366333)
\curveto(65.33248952,147.5836611)(65.37748948,147.4786612)(65.43749146,147.38866333)
\curveto(65.62748923,147.04866163)(65.88748897,146.7786619)(66.21749146,146.57866333)
\curveto(66.31748854,146.51866216)(66.42248843,146.46866221)(66.53249146,146.42866333)
\curveto(66.6524882,146.39866228)(66.77248808,146.36366232)(66.89249146,146.32366333)
\curveto(67.06248779,146.27366241)(67.26748759,146.25366243)(67.50749146,146.26366333)
\curveto(67.7574871,146.2836624)(67.9574869,146.31866236)(68.10749146,146.36866333)
\curveto(68.47748638,146.48866219)(68.76748609,146.64866203)(68.97749146,146.84866333)
\curveto(69.19748566,147.05866162)(69.37748548,147.33866134)(69.51749146,147.68866333)
\curveto(69.56748529,147.78866089)(69.59748526,147.89366079)(69.60749146,148.00366333)
\curveto(69.62748523,148.11366057)(69.6524852,148.22866045)(69.68249146,148.34866333)
\lineto(69.68249146,148.45366333)
\curveto(69.69248516,148.49366019)(69.69748516,148.53366015)(69.69749146,148.57366333)
\curveto(69.70748515,148.60366008)(69.70748515,148.63866004)(69.69749146,148.67866333)
\lineto(69.69749146,148.79866333)
\curveto(69.69748516,149.05865962)(69.66748519,149.30365938)(69.60749146,149.53366333)
\curveto(69.49748536,149.8836588)(69.34248551,150.1786585)(69.14249146,150.41866333)
\curveto(68.94248591,150.66865801)(68.68248617,150.86365782)(68.36249146,151.00366333)
\lineto(68.18249146,151.06366333)
\curveto(68.13248672,151.0836576)(68.07248678,151.10365758)(68.00249146,151.12366333)
\curveto(67.9524869,151.14365754)(67.89248696,151.15365753)(67.82249146,151.15366333)
\curveto(67.76248709,151.16365752)(67.69748716,151.1786575)(67.62749146,151.19866333)
\lineto(67.47749146,151.19866333)
\curveto(67.43748742,151.21865746)(67.38248747,151.22865745)(67.31249146,151.22866333)
\curveto(67.2524876,151.22865745)(67.19748766,151.21865746)(67.14749146,151.19866333)
\lineto(67.04249146,151.19866333)
\curveto(67.01248784,151.19865748)(66.97748788,151.19365749)(66.93749146,151.18366333)
\lineto(66.69749146,151.12366333)
\curveto(66.61748824,151.11365757)(66.53748832,151.09365759)(66.45749146,151.06366333)
\curveto(66.21748864,150.96365772)(65.98748887,150.82865785)(65.76749146,150.65866333)
\curveto(65.67748918,150.58865809)(65.59248926,150.51365817)(65.51249146,150.43366333)
\curveto(65.43248942,150.36365832)(65.33248952,150.30865837)(65.21249146,150.26866333)
\curveto(65.12248973,150.23865844)(64.98248987,150.22865845)(64.79249146,150.23866333)
\curveto(64.61249024,150.24865843)(64.49249036,150.27365841)(64.43249146,150.31366333)
\curveto(64.38249047,150.35365833)(64.34249051,150.41365827)(64.31249146,150.49366333)
\curveto(64.29249056,150.57365811)(64.29249056,150.65865802)(64.31249146,150.74866333)
\curveto(64.34249051,150.86865781)(64.36249049,150.98865769)(64.37249146,151.10866333)
\curveto(64.39249046,151.23865744)(64.41749044,151.36365732)(64.44749146,151.48366333)
\curveto(64.46749039,151.52365716)(64.47249038,151.55865712)(64.46249146,151.58866333)
\curveto(64.46249039,151.62865705)(64.47249038,151.67365701)(64.49249146,151.72366333)
\curveto(64.51249034,151.81365687)(64.52749033,151.90365678)(64.53749146,151.99366333)
\curveto(64.54749031,152.09365659)(64.56749029,152.18865649)(64.59749146,152.27866333)
\curveto(64.60749025,152.33865634)(64.61249024,152.39865628)(64.61249146,152.45866333)
\curveto(64.62249023,152.51865616)(64.63749022,152.5786561)(64.65749146,152.63866333)
\curveto(64.70749015,152.83865584)(64.74249011,153.04365564)(64.76249146,153.25366333)
\curveto(64.79249006,153.47365521)(64.83249002,153.683655)(64.88249146,153.88366333)
\curveto(64.91248994,153.9836547)(64.93248992,154.0836546)(64.94249146,154.18366333)
\curveto(64.9524899,154.2836544)(64.96748989,154.3836543)(64.98749146,154.48366333)
\curveto(64.99748986,154.51365417)(65.00248985,154.55365413)(65.00249146,154.60366333)
\curveto(65.03248982,154.71365397)(65.0524898,154.81865386)(65.06249146,154.91866333)
\curveto(65.08248977,155.02865365)(65.10748975,155.13865354)(65.13749146,155.24866333)
\curveto(65.1574897,155.32865335)(65.17248968,155.39865328)(65.18249146,155.45866333)
\curveto(65.19248966,155.52865315)(65.21748964,155.58865309)(65.25749146,155.63866333)
\curveto(65.27748958,155.66865301)(65.30748955,155.68865299)(65.34749146,155.69866333)
\curveto(65.38748947,155.71865296)(65.43248942,155.73865294)(65.48249146,155.75866333)
\curveto(65.54248931,155.75865292)(65.58248927,155.76365292)(65.60249146,155.77366333)
}
}
{
\newrgbcolor{curcolor}{0 0 0}
\pscustom[linestyle=none,fillstyle=solid,fillcolor=curcolor]
{
\newpath
\moveto(79.44210083,150.44866333)
\lineto(79.44210083,150.19366333)
\curveto(79.45209313,150.11365857)(79.44709313,150.03865864)(79.42710083,149.96866333)
\lineto(79.42710083,149.72866333)
\lineto(79.42710083,149.56366333)
\curveto(79.40709317,149.46365922)(79.39709318,149.35865932)(79.39710083,149.24866333)
\curveto(79.39709318,149.14865953)(79.38709319,149.04865963)(79.36710083,148.94866333)
\lineto(79.36710083,148.79866333)
\curveto(79.33709324,148.65866002)(79.31709326,148.51866016)(79.30710083,148.37866333)
\curveto(79.29709328,148.24866043)(79.27209331,148.11866056)(79.23210083,147.98866333)
\curveto(79.21209337,147.90866077)(79.19209339,147.82366086)(79.17210083,147.73366333)
\lineto(79.11210083,147.49366333)
\lineto(78.99210083,147.19366333)
\curveto(78.96209362,147.10366158)(78.92709365,147.01366167)(78.88710083,146.92366333)
\curveto(78.78709379,146.70366198)(78.65209393,146.48866219)(78.48210083,146.27866333)
\curveto(78.32209426,146.06866261)(78.14709443,145.89866278)(77.95710083,145.76866333)
\curveto(77.90709467,145.72866295)(77.84709473,145.68866299)(77.77710083,145.64866333)
\curveto(77.71709486,145.61866306)(77.65709492,145.5836631)(77.59710083,145.54366333)
\curveto(77.51709506,145.49366319)(77.42209516,145.45366323)(77.31210083,145.42366333)
\curveto(77.20209538,145.39366329)(77.09709548,145.36366332)(76.99710083,145.33366333)
\curveto(76.88709569,145.29366339)(76.7770958,145.26866341)(76.66710083,145.25866333)
\curveto(76.55709602,145.24866343)(76.44209614,145.23366345)(76.32210083,145.21366333)
\curveto(76.2820963,145.20366348)(76.23709634,145.20366348)(76.18710083,145.21366333)
\curveto(76.14709643,145.21366347)(76.10709647,145.20866347)(76.06710083,145.19866333)
\curveto(76.02709655,145.18866349)(75.97209661,145.1836635)(75.90210083,145.18366333)
\curveto(75.83209675,145.1836635)(75.7820968,145.18866349)(75.75210083,145.19866333)
\curveto(75.70209688,145.21866346)(75.65709692,145.22366346)(75.61710083,145.21366333)
\curveto(75.577097,145.20366348)(75.54209704,145.20366348)(75.51210083,145.21366333)
\lineto(75.42210083,145.21366333)
\curveto(75.36209722,145.23366345)(75.29709728,145.24866343)(75.22710083,145.25866333)
\curveto(75.16709741,145.25866342)(75.10209748,145.26366342)(75.03210083,145.27366333)
\curveto(74.86209772,145.32366336)(74.70209788,145.37366331)(74.55210083,145.42366333)
\curveto(74.40209818,145.47366321)(74.25709832,145.53866314)(74.11710083,145.61866333)
\curveto(74.06709851,145.65866302)(74.01209857,145.68866299)(73.95210083,145.70866333)
\curveto(73.90209868,145.73866294)(73.85209873,145.77366291)(73.80210083,145.81366333)
\curveto(73.56209902,145.99366269)(73.36209922,146.21366247)(73.20210083,146.47366333)
\curveto(73.04209954,146.73366195)(72.90209968,147.01866166)(72.78210083,147.32866333)
\curveto(72.72209986,147.46866121)(72.6770999,147.60866107)(72.64710083,147.74866333)
\curveto(72.61709996,147.89866078)(72.5821,148.05366063)(72.54210083,148.21366333)
\curveto(72.52210006,148.32366036)(72.50710007,148.43366025)(72.49710083,148.54366333)
\curveto(72.48710009,148.65366003)(72.47210011,148.76365992)(72.45210083,148.87366333)
\curveto(72.44210014,148.91365977)(72.43710014,148.95365973)(72.43710083,148.99366333)
\curveto(72.44710013,149.03365965)(72.44710013,149.07365961)(72.43710083,149.11366333)
\curveto(72.42710015,149.16365952)(72.42210016,149.21365947)(72.42210083,149.26366333)
\lineto(72.42210083,149.42866333)
\curveto(72.40210018,149.4786592)(72.39710018,149.52865915)(72.40710083,149.57866333)
\curveto(72.41710016,149.63865904)(72.41710016,149.69365899)(72.40710083,149.74366333)
\curveto(72.39710018,149.7836589)(72.39710018,149.82865885)(72.40710083,149.87866333)
\curveto(72.41710016,149.92865875)(72.41210017,149.9786587)(72.39210083,150.02866333)
\curveto(72.37210021,150.09865858)(72.36710021,150.17365851)(72.37710083,150.25366333)
\curveto(72.38710019,150.34365834)(72.39210019,150.42865825)(72.39210083,150.50866333)
\curveto(72.39210019,150.59865808)(72.38710019,150.69865798)(72.37710083,150.80866333)
\curveto(72.36710021,150.92865775)(72.37210021,151.02865765)(72.39210083,151.10866333)
\lineto(72.39210083,151.39366333)
\lineto(72.43710083,152.02366333)
\curveto(72.44710013,152.12365656)(72.45710012,152.21865646)(72.46710083,152.30866333)
\lineto(72.49710083,152.60866333)
\curveto(72.51710006,152.65865602)(72.52210006,152.70865597)(72.51210083,152.75866333)
\curveto(72.51210007,152.81865586)(72.52210006,152.87365581)(72.54210083,152.92366333)
\curveto(72.59209999,153.09365559)(72.63209995,153.25865542)(72.66210083,153.41866333)
\curveto(72.69209989,153.58865509)(72.74209984,153.74865493)(72.81210083,153.89866333)
\curveto(73.00209958,154.35865432)(73.22209936,154.73365395)(73.47210083,155.02366333)
\curveto(73.73209885,155.31365337)(74.09209849,155.55865312)(74.55210083,155.75866333)
\curveto(74.6820979,155.80865287)(74.81209777,155.84365284)(74.94210083,155.86366333)
\curveto(75.0820975,155.8836528)(75.22209736,155.90865277)(75.36210083,155.93866333)
\curveto(75.43209715,155.94865273)(75.49709708,155.95365273)(75.55710083,155.95366333)
\curveto(75.61709696,155.95365273)(75.6820969,155.95865272)(75.75210083,155.96866333)
\curveto(76.582096,155.98865269)(77.25209533,155.83865284)(77.76210083,155.51866333)
\curveto(78.27209431,155.20865347)(78.65209393,154.76865391)(78.90210083,154.19866333)
\curveto(78.95209363,154.0786546)(78.99709358,153.95365473)(79.03710083,153.82366333)
\curveto(79.0770935,153.69365499)(79.12209346,153.55865512)(79.17210083,153.41866333)
\curveto(79.19209339,153.33865534)(79.20709337,153.25365543)(79.21710083,153.16366333)
\lineto(79.27710083,152.92366333)
\curveto(79.30709327,152.81365587)(79.32209326,152.70365598)(79.32210083,152.59366333)
\curveto(79.33209325,152.4836562)(79.34709323,152.37365631)(79.36710083,152.26366333)
\curveto(79.38709319,152.21365647)(79.39209319,152.16865651)(79.38210083,152.12866333)
\curveto(79.3820932,152.08865659)(79.38709319,152.04865663)(79.39710083,152.00866333)
\curveto(79.40709317,151.95865672)(79.40709317,151.90365678)(79.39710083,151.84366333)
\curveto(79.39709318,151.79365689)(79.40209318,151.74365694)(79.41210083,151.69366333)
\lineto(79.41210083,151.55866333)
\curveto(79.43209315,151.49865718)(79.43209315,151.42865725)(79.41210083,151.34866333)
\curveto(79.40209318,151.2786574)(79.40709317,151.21365747)(79.42710083,151.15366333)
\curveto(79.43709314,151.12365756)(79.44209314,151.0836576)(79.44210083,151.03366333)
\lineto(79.44210083,150.91366333)
\lineto(79.44210083,150.44866333)
\moveto(77.89710083,148.12366333)
\curveto(77.99709458,148.44366024)(78.05709452,148.80865987)(78.07710083,149.21866333)
\curveto(78.09709448,149.62865905)(78.10709447,150.03865864)(78.10710083,150.44866333)
\curveto(78.10709447,150.8786578)(78.09709448,151.29865738)(78.07710083,151.70866333)
\curveto(78.05709452,152.11865656)(78.01209457,152.50365618)(77.94210083,152.86366333)
\curveto(77.87209471,153.22365546)(77.76209482,153.54365514)(77.61210083,153.82366333)
\curveto(77.47209511,154.11365457)(77.2770953,154.34865433)(77.02710083,154.52866333)
\curveto(76.86709571,154.63865404)(76.68709589,154.71865396)(76.48710083,154.76866333)
\curveto(76.28709629,154.82865385)(76.04209654,154.85865382)(75.75210083,154.85866333)
\curveto(75.73209685,154.83865384)(75.69709688,154.82865385)(75.64710083,154.82866333)
\curveto(75.59709698,154.83865384)(75.55709702,154.83865384)(75.52710083,154.82866333)
\curveto(75.44709713,154.80865387)(75.37209721,154.78865389)(75.30210083,154.76866333)
\curveto(75.24209734,154.75865392)(75.1770974,154.73865394)(75.10710083,154.70866333)
\curveto(74.83709774,154.58865409)(74.61709796,154.41865426)(74.44710083,154.19866333)
\curveto(74.28709829,153.98865469)(74.15209843,153.74365494)(74.04210083,153.46366333)
\curveto(73.99209859,153.35365533)(73.95209863,153.23365545)(73.92210083,153.10366333)
\curveto(73.90209868,152.9836557)(73.8770987,152.85865582)(73.84710083,152.72866333)
\curveto(73.82709875,152.678656)(73.81709876,152.62365606)(73.81710083,152.56366333)
\curveto(73.81709876,152.51365617)(73.81209877,152.46365622)(73.80210083,152.41366333)
\curveto(73.79209879,152.32365636)(73.7820988,152.22865645)(73.77210083,152.12866333)
\curveto(73.76209882,152.03865664)(73.75209883,151.94365674)(73.74210083,151.84366333)
\curveto(73.74209884,151.76365692)(73.73709884,151.678657)(73.72710083,151.58866333)
\lineto(73.72710083,151.34866333)
\lineto(73.72710083,151.16866333)
\curveto(73.71709886,151.13865754)(73.71209887,151.10365758)(73.71210083,151.06366333)
\lineto(73.71210083,150.92866333)
\lineto(73.71210083,150.47866333)
\curveto(73.71209887,150.39865828)(73.70709887,150.31365837)(73.69710083,150.22366333)
\curveto(73.69709888,150.14365854)(73.70709887,150.06865861)(73.72710083,149.99866333)
\lineto(73.72710083,149.72866333)
\curveto(73.72709885,149.70865897)(73.72209886,149.678659)(73.71210083,149.63866333)
\curveto(73.71209887,149.60865907)(73.71709886,149.5836591)(73.72710083,149.56366333)
\curveto(73.73709884,149.46365922)(73.74209884,149.36365932)(73.74210083,149.26366333)
\curveto(73.75209883,149.17365951)(73.76209882,149.07365961)(73.77210083,148.96366333)
\curveto(73.80209878,148.84365984)(73.81709876,148.71865996)(73.81710083,148.58866333)
\curveto(73.82709875,148.46866021)(73.85209873,148.35366033)(73.89210083,148.24366333)
\curveto(73.97209861,147.94366074)(74.05709852,147.678661)(74.14710083,147.44866333)
\curveto(74.24709833,147.21866146)(74.39209819,147.00366168)(74.58210083,146.80366333)
\curveto(74.79209779,146.60366208)(75.05709752,146.45366223)(75.37710083,146.35366333)
\curveto(75.41709716,146.33366235)(75.45209713,146.32366236)(75.48210083,146.32366333)
\curveto(75.52209706,146.33366235)(75.56709701,146.32866235)(75.61710083,146.30866333)
\curveto(75.65709692,146.29866238)(75.72709685,146.28866239)(75.82710083,146.27866333)
\curveto(75.93709664,146.26866241)(76.02209656,146.27366241)(76.08210083,146.29366333)
\curveto(76.15209643,146.31366237)(76.22209636,146.32366236)(76.29210083,146.32366333)
\curveto(76.36209622,146.33366235)(76.42709615,146.34866233)(76.48710083,146.36866333)
\curveto(76.68709589,146.42866225)(76.86709571,146.51366217)(77.02710083,146.62366333)
\curveto(77.05709552,146.64366204)(77.0820955,146.66366202)(77.10210083,146.68366333)
\lineto(77.16210083,146.74366333)
\curveto(77.20209538,146.76366192)(77.25209533,146.80366188)(77.31210083,146.86366333)
\curveto(77.41209517,147.00366168)(77.49709508,147.13366155)(77.56710083,147.25366333)
\curveto(77.63709494,147.37366131)(77.70709487,147.51866116)(77.77710083,147.68866333)
\curveto(77.80709477,147.75866092)(77.82709475,147.82866085)(77.83710083,147.89866333)
\curveto(77.85709472,147.96866071)(77.8770947,148.04366064)(77.89710083,148.12366333)
}
}
{
\newrgbcolor{curcolor}{0 0 0}
\pscustom[linestyle=none,fillstyle=solid,fillcolor=curcolor]
{
\newpath
\moveto(57.25288208,229.70223694)
\lineto(60.85288208,229.70223694)
\lineto(61.49788208,229.70223694)
\curveto(61.57787555,229.70222651)(61.65287548,229.69722652)(61.72288208,229.68723694)
\curveto(61.79287534,229.68722653)(61.85287528,229.67722654)(61.90288208,229.65723694)
\curveto(61.97287516,229.62722659)(62.0278751,229.56722665)(62.06788208,229.47723694)
\curveto(62.08787504,229.44722677)(62.09787503,229.40722681)(62.09788208,229.35723694)
\lineto(62.09788208,229.22223694)
\curveto(62.10787502,229.1122271)(62.10287503,229.00722721)(62.08288208,228.90723694)
\curveto(62.07287506,228.80722741)(62.03787509,228.73722748)(61.97788208,228.69723694)
\curveto(61.88787524,228.62722759)(61.75287538,228.59222762)(61.57288208,228.59223694)
\curveto(61.39287574,228.60222761)(61.2278759,228.60722761)(61.07788208,228.60723694)
\lineto(59.08288208,228.60723694)
\lineto(58.58788208,228.60723694)
\lineto(58.45288208,228.60723694)
\curveto(58.41287872,228.60722761)(58.37287876,228.60222761)(58.33288208,228.59223694)
\lineto(58.12288208,228.59223694)
\curveto(58.01287912,228.56222765)(57.9328792,228.52222769)(57.88288208,228.47223694)
\curveto(57.8328793,228.43222778)(57.79787933,228.37722784)(57.77788208,228.30723694)
\curveto(57.75787937,228.24722797)(57.74287939,228.17722804)(57.73288208,228.09723694)
\curveto(57.72287941,228.0172282)(57.70287943,227.92722829)(57.67288208,227.82723694)
\curveto(57.62287951,227.62722859)(57.58287955,227.42222879)(57.55288208,227.21223694)
\curveto(57.52287961,227.00222921)(57.48287965,226.79722942)(57.43288208,226.59723694)
\curveto(57.41287972,226.52722969)(57.40287973,226.45722976)(57.40288208,226.38723694)
\curveto(57.40287973,226.32722989)(57.39287974,226.26222995)(57.37288208,226.19223694)
\curveto(57.36287977,226.16223005)(57.35287978,226.12223009)(57.34288208,226.07223694)
\curveto(57.34287979,226.03223018)(57.34787978,225.99223022)(57.35788208,225.95223694)
\curveto(57.37787975,225.90223031)(57.40287973,225.85723036)(57.43288208,225.81723694)
\curveto(57.47287966,225.78723043)(57.5328796,225.78223043)(57.61288208,225.80223694)
\curveto(57.67287946,225.82223039)(57.7328794,225.84723037)(57.79288208,225.87723694)
\curveto(57.85287928,225.9172303)(57.91287922,225.95223026)(57.97288208,225.98223694)
\curveto(58.0328791,226.00223021)(58.08287905,226.0172302)(58.12288208,226.02723694)
\curveto(58.31287882,226.10723011)(58.51787861,226.16223005)(58.73788208,226.19223694)
\curveto(58.96787816,226.22222999)(59.19787793,226.23222998)(59.42788208,226.22223694)
\curveto(59.66787746,226.22222999)(59.89787723,226.19723002)(60.11788208,226.14723694)
\curveto(60.33787679,226.10723011)(60.53787659,226.04723017)(60.71788208,225.96723694)
\curveto(60.76787636,225.94723027)(60.81287632,225.92723029)(60.85288208,225.90723694)
\curveto(60.90287623,225.88723033)(60.95287618,225.86223035)(61.00288208,225.83223694)
\curveto(61.35287578,225.62223059)(61.6328755,225.39223082)(61.84288208,225.14223694)
\curveto(62.06287507,224.89223132)(62.25787487,224.56723165)(62.42788208,224.16723694)
\curveto(62.47787465,224.05723216)(62.51287462,223.94723227)(62.53288208,223.83723694)
\curveto(62.55287458,223.72723249)(62.57787455,223.6122326)(62.60788208,223.49223694)
\curveto(62.61787451,223.46223275)(62.62287451,223.4172328)(62.62288208,223.35723694)
\curveto(62.64287449,223.29723292)(62.65287448,223.22723299)(62.65288208,223.14723694)
\curveto(62.65287448,223.07723314)(62.66287447,223.0122332)(62.68288208,222.95223694)
\lineto(62.68288208,222.78723694)
\curveto(62.69287444,222.73723348)(62.69787443,222.66723355)(62.69788208,222.57723694)
\curveto(62.69787443,222.48723373)(62.68787444,222.4172338)(62.66788208,222.36723694)
\curveto(62.64787448,222.30723391)(62.64287449,222.24723397)(62.65288208,222.18723694)
\curveto(62.66287447,222.13723408)(62.65787447,222.08723413)(62.63788208,222.03723694)
\curveto(62.59787453,221.87723434)(62.56287457,221.72723449)(62.53288208,221.58723694)
\curveto(62.50287463,221.44723477)(62.45787467,221.3122349)(62.39788208,221.18223694)
\curveto(62.23787489,220.8122354)(62.01787511,220.47723574)(61.73788208,220.17723694)
\curveto(61.45787567,219.87723634)(61.13787599,219.64723657)(60.77788208,219.48723694)
\curveto(60.60787652,219.40723681)(60.40787672,219.33223688)(60.17788208,219.26223694)
\curveto(60.06787706,219.22223699)(59.95287718,219.19723702)(59.83288208,219.18723694)
\curveto(59.71287742,219.17723704)(59.59287754,219.15723706)(59.47288208,219.12723694)
\curveto(59.42287771,219.10723711)(59.36787776,219.10723711)(59.30788208,219.12723694)
\curveto(59.24787788,219.13723708)(59.18787794,219.13223708)(59.12788208,219.11223694)
\curveto(59.0278781,219.09223712)(58.9278782,219.09223712)(58.82788208,219.11223694)
\lineto(58.69288208,219.11223694)
\curveto(58.64287849,219.13223708)(58.58287855,219.14223707)(58.51288208,219.14223694)
\curveto(58.45287868,219.13223708)(58.39787873,219.13723708)(58.34788208,219.15723694)
\curveto(58.30787882,219.16723705)(58.27287886,219.17223704)(58.24288208,219.17223694)
\curveto(58.21287892,219.17223704)(58.17787895,219.17723704)(58.13788208,219.18723694)
\lineto(57.86788208,219.24723694)
\curveto(57.77787935,219.26723695)(57.69287944,219.29723692)(57.61288208,219.33723694)
\curveto(57.27287986,219.47723674)(56.98288015,219.63223658)(56.74288208,219.80223694)
\curveto(56.50288063,219.98223623)(56.28288085,220.212236)(56.08288208,220.49223694)
\curveto(55.9328812,220.72223549)(55.81788131,220.96223525)(55.73788208,221.21223694)
\curveto(55.71788141,221.26223495)(55.70788142,221.30723491)(55.70788208,221.34723694)
\curveto(55.70788142,221.39723482)(55.69788143,221.44723477)(55.67788208,221.49723694)
\curveto(55.65788147,221.55723466)(55.64288149,221.63723458)(55.63288208,221.73723694)
\curveto(55.6328815,221.83723438)(55.65288148,221.9122343)(55.69288208,221.96223694)
\curveto(55.74288139,222.04223417)(55.82288131,222.08723413)(55.93288208,222.09723694)
\curveto(56.04288109,222.10723411)(56.15788097,222.1122341)(56.27788208,222.11223694)
\lineto(56.44288208,222.11223694)
\curveto(56.50288063,222.1122341)(56.55788057,222.10223411)(56.60788208,222.08223694)
\curveto(56.69788043,222.06223415)(56.76788036,222.02223419)(56.81788208,221.96223694)
\curveto(56.88788024,221.87223434)(56.9328802,221.76223445)(56.95288208,221.63223694)
\curveto(56.98288015,221.5122347)(57.0278801,221.40723481)(57.08788208,221.31723694)
\curveto(57.27787985,220.97723524)(57.53787959,220.70723551)(57.86788208,220.50723694)
\curveto(57.96787916,220.44723577)(58.07287906,220.39723582)(58.18288208,220.35723694)
\curveto(58.30287883,220.32723589)(58.42287871,220.29223592)(58.54288208,220.25223694)
\curveto(58.71287842,220.20223601)(58.91787821,220.18223603)(59.15788208,220.19223694)
\curveto(59.40787772,220.212236)(59.60787752,220.24723597)(59.75788208,220.29723694)
\curveto(60.127877,220.4172358)(60.41787671,220.57723564)(60.62788208,220.77723694)
\curveto(60.84787628,220.98723523)(61.0278761,221.26723495)(61.16788208,221.61723694)
\curveto(61.21787591,221.7172345)(61.24787588,221.82223439)(61.25788208,221.93223694)
\curveto(61.27787585,222.04223417)(61.30287583,222.15723406)(61.33288208,222.27723694)
\lineto(61.33288208,222.38223694)
\curveto(61.34287579,222.42223379)(61.34787578,222.46223375)(61.34788208,222.50223694)
\curveto(61.35787577,222.53223368)(61.35787577,222.56723365)(61.34788208,222.60723694)
\lineto(61.34788208,222.72723694)
\curveto(61.34787578,222.98723323)(61.31787581,223.23223298)(61.25788208,223.46223694)
\curveto(61.14787598,223.8122324)(60.99287614,224.10723211)(60.79288208,224.34723694)
\curveto(60.59287654,224.59723162)(60.3328768,224.79223142)(60.01288208,224.93223694)
\lineto(59.83288208,224.99223694)
\curveto(59.78287735,225.0122312)(59.72287741,225.03223118)(59.65288208,225.05223694)
\curveto(59.60287753,225.07223114)(59.54287759,225.08223113)(59.47288208,225.08223694)
\curveto(59.41287772,225.09223112)(59.34787778,225.10723111)(59.27788208,225.12723694)
\lineto(59.12788208,225.12723694)
\curveto(59.08787804,225.14723107)(59.0328781,225.15723106)(58.96288208,225.15723694)
\curveto(58.90287823,225.15723106)(58.84787828,225.14723107)(58.79788208,225.12723694)
\lineto(58.69288208,225.12723694)
\curveto(58.66287847,225.12723109)(58.6278785,225.12223109)(58.58788208,225.11223694)
\lineto(58.34788208,225.05223694)
\curveto(58.26787886,225.04223117)(58.18787894,225.02223119)(58.10788208,224.99223694)
\curveto(57.86787926,224.89223132)(57.63787949,224.75723146)(57.41788208,224.58723694)
\curveto(57.3278798,224.5172317)(57.24287989,224.44223177)(57.16288208,224.36223694)
\curveto(57.08288005,224.29223192)(56.98288015,224.23723198)(56.86288208,224.19723694)
\curveto(56.77288036,224.16723205)(56.6328805,224.15723206)(56.44288208,224.16723694)
\curveto(56.26288087,224.17723204)(56.14288099,224.20223201)(56.08288208,224.24223694)
\curveto(56.0328811,224.28223193)(55.99288114,224.34223187)(55.96288208,224.42223694)
\curveto(55.94288119,224.50223171)(55.94288119,224.58723163)(55.96288208,224.67723694)
\curveto(55.99288114,224.79723142)(56.01288112,224.9172313)(56.02288208,225.03723694)
\curveto(56.04288109,225.16723105)(56.06788106,225.29223092)(56.09788208,225.41223694)
\curveto(56.11788101,225.45223076)(56.12288101,225.48723073)(56.11288208,225.51723694)
\curveto(56.11288102,225.55723066)(56.12288101,225.60223061)(56.14288208,225.65223694)
\curveto(56.16288097,225.74223047)(56.17788095,225.83223038)(56.18788208,225.92223694)
\curveto(56.19788093,226.02223019)(56.21788091,226.1172301)(56.24788208,226.20723694)
\curveto(56.25788087,226.26722995)(56.26288087,226.32722989)(56.26288208,226.38723694)
\curveto(56.27288086,226.44722977)(56.28788084,226.50722971)(56.30788208,226.56723694)
\curveto(56.35788077,226.76722945)(56.39288074,226.97222924)(56.41288208,227.18223694)
\curveto(56.44288069,227.40222881)(56.48288065,227.6122286)(56.53288208,227.81223694)
\curveto(56.56288057,227.9122283)(56.58288055,228.0122282)(56.59288208,228.11223694)
\curveto(56.60288053,228.212228)(56.61788051,228.3122279)(56.63788208,228.41223694)
\curveto(56.64788048,228.44222777)(56.65288048,228.48222773)(56.65288208,228.53223694)
\curveto(56.68288045,228.64222757)(56.70288043,228.74722747)(56.71288208,228.84723694)
\curveto(56.7328804,228.95722726)(56.75788037,229.06722715)(56.78788208,229.17723694)
\curveto(56.80788032,229.25722696)(56.82288031,229.32722689)(56.83288208,229.38723694)
\curveto(56.84288029,229.45722676)(56.86788026,229.5172267)(56.90788208,229.56723694)
\curveto(56.9278802,229.59722662)(56.95788017,229.6172266)(56.99788208,229.62723694)
\curveto(57.03788009,229.64722657)(57.08288005,229.66722655)(57.13288208,229.68723694)
\curveto(57.19287994,229.68722653)(57.2328799,229.69222652)(57.25288208,229.70223694)
}
}
{
\newrgbcolor{curcolor}{0 0 0}
\pscustom[linestyle=none,fillstyle=solid,fillcolor=curcolor]
{
\newpath
\moveto(71.09249146,224.37723694)
\lineto(71.09249146,224.12223694)
\curveto(71.10248375,224.04223217)(71.09748376,223.96723225)(71.07749146,223.89723694)
\lineto(71.07749146,223.65723694)
\lineto(71.07749146,223.49223694)
\curveto(71.0574838,223.39223282)(71.04748381,223.28723293)(71.04749146,223.17723694)
\curveto(71.04748381,223.07723314)(71.03748382,222.97723324)(71.01749146,222.87723694)
\lineto(71.01749146,222.72723694)
\curveto(70.98748387,222.58723363)(70.96748389,222.44723377)(70.95749146,222.30723694)
\curveto(70.94748391,222.17723404)(70.92248393,222.04723417)(70.88249146,221.91723694)
\curveto(70.86248399,221.83723438)(70.84248401,221.75223446)(70.82249146,221.66223694)
\lineto(70.76249146,221.42223694)
\lineto(70.64249146,221.12223694)
\curveto(70.61248424,221.03223518)(70.57748428,220.94223527)(70.53749146,220.85223694)
\curveto(70.43748442,220.63223558)(70.30248455,220.4172358)(70.13249146,220.20723694)
\curveto(69.97248488,219.99723622)(69.79748506,219.82723639)(69.60749146,219.69723694)
\curveto(69.5574853,219.65723656)(69.49748536,219.6172366)(69.42749146,219.57723694)
\curveto(69.36748549,219.54723667)(69.30748555,219.5122367)(69.24749146,219.47223694)
\curveto(69.16748569,219.42223679)(69.07248578,219.38223683)(68.96249146,219.35223694)
\curveto(68.852486,219.32223689)(68.74748611,219.29223692)(68.64749146,219.26223694)
\curveto(68.53748632,219.22223699)(68.42748643,219.19723702)(68.31749146,219.18723694)
\curveto(68.20748665,219.17723704)(68.09248676,219.16223705)(67.97249146,219.14223694)
\curveto(67.93248692,219.13223708)(67.88748697,219.13223708)(67.83749146,219.14223694)
\curveto(67.79748706,219.14223707)(67.7574871,219.13723708)(67.71749146,219.12723694)
\curveto(67.67748718,219.1172371)(67.62248723,219.1122371)(67.55249146,219.11223694)
\curveto(67.48248737,219.1122371)(67.43248742,219.1172371)(67.40249146,219.12723694)
\curveto(67.3524875,219.14723707)(67.30748755,219.15223706)(67.26749146,219.14223694)
\curveto(67.22748763,219.13223708)(67.19248766,219.13223708)(67.16249146,219.14223694)
\lineto(67.07249146,219.14223694)
\curveto(67.01248784,219.16223705)(66.94748791,219.17723704)(66.87749146,219.18723694)
\curveto(66.81748804,219.18723703)(66.7524881,219.19223702)(66.68249146,219.20223694)
\curveto(66.51248834,219.25223696)(66.3524885,219.30223691)(66.20249146,219.35223694)
\curveto(66.0524888,219.40223681)(65.90748895,219.46723675)(65.76749146,219.54723694)
\curveto(65.71748914,219.58723663)(65.66248919,219.6172366)(65.60249146,219.63723694)
\curveto(65.5524893,219.66723655)(65.50248935,219.70223651)(65.45249146,219.74223694)
\curveto(65.21248964,219.92223629)(65.01248984,220.14223607)(64.85249146,220.40223694)
\curveto(64.69249016,220.66223555)(64.5524903,220.94723527)(64.43249146,221.25723694)
\curveto(64.37249048,221.39723482)(64.32749053,221.53723468)(64.29749146,221.67723694)
\curveto(64.26749059,221.82723439)(64.23249062,221.98223423)(64.19249146,222.14223694)
\curveto(64.17249068,222.25223396)(64.1574907,222.36223385)(64.14749146,222.47223694)
\curveto(64.13749072,222.58223363)(64.12249073,222.69223352)(64.10249146,222.80223694)
\curveto(64.09249076,222.84223337)(64.08749077,222.88223333)(64.08749146,222.92223694)
\curveto(64.09749076,222.96223325)(64.09749076,223.00223321)(64.08749146,223.04223694)
\curveto(64.07749078,223.09223312)(64.07249078,223.14223307)(64.07249146,223.19223694)
\lineto(64.07249146,223.35723694)
\curveto(64.0524908,223.40723281)(64.04749081,223.45723276)(64.05749146,223.50723694)
\curveto(64.06749079,223.56723265)(64.06749079,223.62223259)(64.05749146,223.67223694)
\curveto(64.04749081,223.7122325)(64.04749081,223.75723246)(64.05749146,223.80723694)
\curveto(64.06749079,223.85723236)(64.06249079,223.90723231)(64.04249146,223.95723694)
\curveto(64.02249083,224.02723219)(64.01749084,224.10223211)(64.02749146,224.18223694)
\curveto(64.03749082,224.27223194)(64.04249081,224.35723186)(64.04249146,224.43723694)
\curveto(64.04249081,224.52723169)(64.03749082,224.62723159)(64.02749146,224.73723694)
\curveto(64.01749084,224.85723136)(64.02249083,224.95723126)(64.04249146,225.03723694)
\lineto(64.04249146,225.32223694)
\lineto(64.08749146,225.95223694)
\curveto(64.09749076,226.05223016)(64.10749075,226.14723007)(64.11749146,226.23723694)
\lineto(64.14749146,226.53723694)
\curveto(64.16749069,226.58722963)(64.17249068,226.63722958)(64.16249146,226.68723694)
\curveto(64.16249069,226.74722947)(64.17249068,226.80222941)(64.19249146,226.85223694)
\curveto(64.24249061,227.02222919)(64.28249057,227.18722903)(64.31249146,227.34723694)
\curveto(64.34249051,227.5172287)(64.39249046,227.67722854)(64.46249146,227.82723694)
\curveto(64.6524902,228.28722793)(64.87248998,228.66222755)(65.12249146,228.95223694)
\curveto(65.38248947,229.24222697)(65.74248911,229.48722673)(66.20249146,229.68723694)
\curveto(66.33248852,229.73722648)(66.46248839,229.77222644)(66.59249146,229.79223694)
\curveto(66.73248812,229.8122264)(66.87248798,229.83722638)(67.01249146,229.86723694)
\curveto(67.08248777,229.87722634)(67.14748771,229.88222633)(67.20749146,229.88223694)
\curveto(67.26748759,229.88222633)(67.33248752,229.88722633)(67.40249146,229.89723694)
\curveto(68.23248662,229.9172263)(68.90248595,229.76722645)(69.41249146,229.44723694)
\curveto(69.92248493,229.13722708)(70.30248455,228.69722752)(70.55249146,228.12723694)
\curveto(70.60248425,228.00722821)(70.64748421,227.88222833)(70.68749146,227.75223694)
\curveto(70.72748413,227.62222859)(70.77248408,227.48722873)(70.82249146,227.34723694)
\curveto(70.84248401,227.26722895)(70.857484,227.18222903)(70.86749146,227.09223694)
\lineto(70.92749146,226.85223694)
\curveto(70.9574839,226.74222947)(70.97248388,226.63222958)(70.97249146,226.52223694)
\curveto(70.98248387,226.4122298)(70.99748386,226.30222991)(71.01749146,226.19223694)
\curveto(71.03748382,226.14223007)(71.04248381,226.09723012)(71.03249146,226.05723694)
\curveto(71.03248382,226.0172302)(71.03748382,225.97723024)(71.04749146,225.93723694)
\curveto(71.0574838,225.88723033)(71.0574838,225.83223038)(71.04749146,225.77223694)
\curveto(71.04748381,225.72223049)(71.0524838,225.67223054)(71.06249146,225.62223694)
\lineto(71.06249146,225.48723694)
\curveto(71.08248377,225.42723079)(71.08248377,225.35723086)(71.06249146,225.27723694)
\curveto(71.0524838,225.20723101)(71.0574838,225.14223107)(71.07749146,225.08223694)
\curveto(71.08748377,225.05223116)(71.09248376,225.0122312)(71.09249146,224.96223694)
\lineto(71.09249146,224.84223694)
\lineto(71.09249146,224.37723694)
\moveto(69.54749146,222.05223694)
\curveto(69.64748521,222.37223384)(69.70748515,222.73723348)(69.72749146,223.14723694)
\curveto(69.74748511,223.55723266)(69.7574851,223.96723225)(69.75749146,224.37723694)
\curveto(69.7574851,224.80723141)(69.74748511,225.22723099)(69.72749146,225.63723694)
\curveto(69.70748515,226.04723017)(69.66248519,226.43222978)(69.59249146,226.79223694)
\curveto(69.52248533,227.15222906)(69.41248544,227.47222874)(69.26249146,227.75223694)
\curveto(69.12248573,228.04222817)(68.92748593,228.27722794)(68.67749146,228.45723694)
\curveto(68.51748634,228.56722765)(68.33748652,228.64722757)(68.13749146,228.69723694)
\curveto(67.93748692,228.75722746)(67.69248716,228.78722743)(67.40249146,228.78723694)
\curveto(67.38248747,228.76722745)(67.34748751,228.75722746)(67.29749146,228.75723694)
\curveto(67.24748761,228.76722745)(67.20748765,228.76722745)(67.17749146,228.75723694)
\curveto(67.09748776,228.73722748)(67.02248783,228.7172275)(66.95249146,228.69723694)
\curveto(66.89248796,228.68722753)(66.82748803,228.66722755)(66.75749146,228.63723694)
\curveto(66.48748837,228.5172277)(66.26748859,228.34722787)(66.09749146,228.12723694)
\curveto(65.93748892,227.9172283)(65.80248905,227.67222854)(65.69249146,227.39223694)
\curveto(65.64248921,227.28222893)(65.60248925,227.16222905)(65.57249146,227.03223694)
\curveto(65.5524893,226.9122293)(65.52748933,226.78722943)(65.49749146,226.65723694)
\curveto(65.47748938,226.60722961)(65.46748939,226.55222966)(65.46749146,226.49223694)
\curveto(65.46748939,226.44222977)(65.46248939,226.39222982)(65.45249146,226.34223694)
\curveto(65.44248941,226.25222996)(65.43248942,226.15723006)(65.42249146,226.05723694)
\curveto(65.41248944,225.96723025)(65.40248945,225.87223034)(65.39249146,225.77223694)
\curveto(65.39248946,225.69223052)(65.38748947,225.60723061)(65.37749146,225.51723694)
\lineto(65.37749146,225.27723694)
\lineto(65.37749146,225.09723694)
\curveto(65.36748949,225.06723115)(65.36248949,225.03223118)(65.36249146,224.99223694)
\lineto(65.36249146,224.85723694)
\lineto(65.36249146,224.40723694)
\curveto(65.36248949,224.32723189)(65.3574895,224.24223197)(65.34749146,224.15223694)
\curveto(65.34748951,224.07223214)(65.3574895,223.99723222)(65.37749146,223.92723694)
\lineto(65.37749146,223.65723694)
\curveto(65.37748948,223.63723258)(65.37248948,223.60723261)(65.36249146,223.56723694)
\curveto(65.36248949,223.53723268)(65.36748949,223.5122327)(65.37749146,223.49223694)
\curveto(65.38748947,223.39223282)(65.39248946,223.29223292)(65.39249146,223.19223694)
\curveto(65.40248945,223.10223311)(65.41248944,223.00223321)(65.42249146,222.89223694)
\curveto(65.4524894,222.77223344)(65.46748939,222.64723357)(65.46749146,222.51723694)
\curveto(65.47748938,222.39723382)(65.50248935,222.28223393)(65.54249146,222.17223694)
\curveto(65.62248923,221.87223434)(65.70748915,221.60723461)(65.79749146,221.37723694)
\curveto(65.89748896,221.14723507)(66.04248881,220.93223528)(66.23249146,220.73223694)
\curveto(66.44248841,220.53223568)(66.70748815,220.38223583)(67.02749146,220.28223694)
\curveto(67.06748779,220.26223595)(67.10248775,220.25223596)(67.13249146,220.25223694)
\curveto(67.17248768,220.26223595)(67.21748764,220.25723596)(67.26749146,220.23723694)
\curveto(67.30748755,220.22723599)(67.37748748,220.217236)(67.47749146,220.20723694)
\curveto(67.58748727,220.19723602)(67.67248718,220.20223601)(67.73249146,220.22223694)
\curveto(67.80248705,220.24223597)(67.87248698,220.25223596)(67.94249146,220.25223694)
\curveto(68.01248684,220.26223595)(68.07748678,220.27723594)(68.13749146,220.29723694)
\curveto(68.33748652,220.35723586)(68.51748634,220.44223577)(68.67749146,220.55223694)
\curveto(68.70748615,220.57223564)(68.73248612,220.59223562)(68.75249146,220.61223694)
\lineto(68.81249146,220.67223694)
\curveto(68.852486,220.69223552)(68.90248595,220.73223548)(68.96249146,220.79223694)
\curveto(69.06248579,220.93223528)(69.14748571,221.06223515)(69.21749146,221.18223694)
\curveto(69.28748557,221.30223491)(69.3574855,221.44723477)(69.42749146,221.61723694)
\curveto(69.4574854,221.68723453)(69.47748538,221.75723446)(69.48749146,221.82723694)
\curveto(69.50748535,221.89723432)(69.52748533,221.97223424)(69.54749146,222.05223694)
}
}
{
\newrgbcolor{curcolor}{0 0 0}
\pscustom[linestyle=none,fillstyle=solid,fillcolor=curcolor]
{
\newpath
\moveto(79.44210083,224.37723694)
\lineto(79.44210083,224.12223694)
\curveto(79.45209313,224.04223217)(79.44709313,223.96723225)(79.42710083,223.89723694)
\lineto(79.42710083,223.65723694)
\lineto(79.42710083,223.49223694)
\curveto(79.40709317,223.39223282)(79.39709318,223.28723293)(79.39710083,223.17723694)
\curveto(79.39709318,223.07723314)(79.38709319,222.97723324)(79.36710083,222.87723694)
\lineto(79.36710083,222.72723694)
\curveto(79.33709324,222.58723363)(79.31709326,222.44723377)(79.30710083,222.30723694)
\curveto(79.29709328,222.17723404)(79.27209331,222.04723417)(79.23210083,221.91723694)
\curveto(79.21209337,221.83723438)(79.19209339,221.75223446)(79.17210083,221.66223694)
\lineto(79.11210083,221.42223694)
\lineto(78.99210083,221.12223694)
\curveto(78.96209362,221.03223518)(78.92709365,220.94223527)(78.88710083,220.85223694)
\curveto(78.78709379,220.63223558)(78.65209393,220.4172358)(78.48210083,220.20723694)
\curveto(78.32209426,219.99723622)(78.14709443,219.82723639)(77.95710083,219.69723694)
\curveto(77.90709467,219.65723656)(77.84709473,219.6172366)(77.77710083,219.57723694)
\curveto(77.71709486,219.54723667)(77.65709492,219.5122367)(77.59710083,219.47223694)
\curveto(77.51709506,219.42223679)(77.42209516,219.38223683)(77.31210083,219.35223694)
\curveto(77.20209538,219.32223689)(77.09709548,219.29223692)(76.99710083,219.26223694)
\curveto(76.88709569,219.22223699)(76.7770958,219.19723702)(76.66710083,219.18723694)
\curveto(76.55709602,219.17723704)(76.44209614,219.16223705)(76.32210083,219.14223694)
\curveto(76.2820963,219.13223708)(76.23709634,219.13223708)(76.18710083,219.14223694)
\curveto(76.14709643,219.14223707)(76.10709647,219.13723708)(76.06710083,219.12723694)
\curveto(76.02709655,219.1172371)(75.97209661,219.1122371)(75.90210083,219.11223694)
\curveto(75.83209675,219.1122371)(75.7820968,219.1172371)(75.75210083,219.12723694)
\curveto(75.70209688,219.14723707)(75.65709692,219.15223706)(75.61710083,219.14223694)
\curveto(75.577097,219.13223708)(75.54209704,219.13223708)(75.51210083,219.14223694)
\lineto(75.42210083,219.14223694)
\curveto(75.36209722,219.16223705)(75.29709728,219.17723704)(75.22710083,219.18723694)
\curveto(75.16709741,219.18723703)(75.10209748,219.19223702)(75.03210083,219.20223694)
\curveto(74.86209772,219.25223696)(74.70209788,219.30223691)(74.55210083,219.35223694)
\curveto(74.40209818,219.40223681)(74.25709832,219.46723675)(74.11710083,219.54723694)
\curveto(74.06709851,219.58723663)(74.01209857,219.6172366)(73.95210083,219.63723694)
\curveto(73.90209868,219.66723655)(73.85209873,219.70223651)(73.80210083,219.74223694)
\curveto(73.56209902,219.92223629)(73.36209922,220.14223607)(73.20210083,220.40223694)
\curveto(73.04209954,220.66223555)(72.90209968,220.94723527)(72.78210083,221.25723694)
\curveto(72.72209986,221.39723482)(72.6770999,221.53723468)(72.64710083,221.67723694)
\curveto(72.61709996,221.82723439)(72.5821,221.98223423)(72.54210083,222.14223694)
\curveto(72.52210006,222.25223396)(72.50710007,222.36223385)(72.49710083,222.47223694)
\curveto(72.48710009,222.58223363)(72.47210011,222.69223352)(72.45210083,222.80223694)
\curveto(72.44210014,222.84223337)(72.43710014,222.88223333)(72.43710083,222.92223694)
\curveto(72.44710013,222.96223325)(72.44710013,223.00223321)(72.43710083,223.04223694)
\curveto(72.42710015,223.09223312)(72.42210016,223.14223307)(72.42210083,223.19223694)
\lineto(72.42210083,223.35723694)
\curveto(72.40210018,223.40723281)(72.39710018,223.45723276)(72.40710083,223.50723694)
\curveto(72.41710016,223.56723265)(72.41710016,223.62223259)(72.40710083,223.67223694)
\curveto(72.39710018,223.7122325)(72.39710018,223.75723246)(72.40710083,223.80723694)
\curveto(72.41710016,223.85723236)(72.41210017,223.90723231)(72.39210083,223.95723694)
\curveto(72.37210021,224.02723219)(72.36710021,224.10223211)(72.37710083,224.18223694)
\curveto(72.38710019,224.27223194)(72.39210019,224.35723186)(72.39210083,224.43723694)
\curveto(72.39210019,224.52723169)(72.38710019,224.62723159)(72.37710083,224.73723694)
\curveto(72.36710021,224.85723136)(72.37210021,224.95723126)(72.39210083,225.03723694)
\lineto(72.39210083,225.32223694)
\lineto(72.43710083,225.95223694)
\curveto(72.44710013,226.05223016)(72.45710012,226.14723007)(72.46710083,226.23723694)
\lineto(72.49710083,226.53723694)
\curveto(72.51710006,226.58722963)(72.52210006,226.63722958)(72.51210083,226.68723694)
\curveto(72.51210007,226.74722947)(72.52210006,226.80222941)(72.54210083,226.85223694)
\curveto(72.59209999,227.02222919)(72.63209995,227.18722903)(72.66210083,227.34723694)
\curveto(72.69209989,227.5172287)(72.74209984,227.67722854)(72.81210083,227.82723694)
\curveto(73.00209958,228.28722793)(73.22209936,228.66222755)(73.47210083,228.95223694)
\curveto(73.73209885,229.24222697)(74.09209849,229.48722673)(74.55210083,229.68723694)
\curveto(74.6820979,229.73722648)(74.81209777,229.77222644)(74.94210083,229.79223694)
\curveto(75.0820975,229.8122264)(75.22209736,229.83722638)(75.36210083,229.86723694)
\curveto(75.43209715,229.87722634)(75.49709708,229.88222633)(75.55710083,229.88223694)
\curveto(75.61709696,229.88222633)(75.6820969,229.88722633)(75.75210083,229.89723694)
\curveto(76.582096,229.9172263)(77.25209533,229.76722645)(77.76210083,229.44723694)
\curveto(78.27209431,229.13722708)(78.65209393,228.69722752)(78.90210083,228.12723694)
\curveto(78.95209363,228.00722821)(78.99709358,227.88222833)(79.03710083,227.75223694)
\curveto(79.0770935,227.62222859)(79.12209346,227.48722873)(79.17210083,227.34723694)
\curveto(79.19209339,227.26722895)(79.20709337,227.18222903)(79.21710083,227.09223694)
\lineto(79.27710083,226.85223694)
\curveto(79.30709327,226.74222947)(79.32209326,226.63222958)(79.32210083,226.52223694)
\curveto(79.33209325,226.4122298)(79.34709323,226.30222991)(79.36710083,226.19223694)
\curveto(79.38709319,226.14223007)(79.39209319,226.09723012)(79.38210083,226.05723694)
\curveto(79.3820932,226.0172302)(79.38709319,225.97723024)(79.39710083,225.93723694)
\curveto(79.40709317,225.88723033)(79.40709317,225.83223038)(79.39710083,225.77223694)
\curveto(79.39709318,225.72223049)(79.40209318,225.67223054)(79.41210083,225.62223694)
\lineto(79.41210083,225.48723694)
\curveto(79.43209315,225.42723079)(79.43209315,225.35723086)(79.41210083,225.27723694)
\curveto(79.40209318,225.20723101)(79.40709317,225.14223107)(79.42710083,225.08223694)
\curveto(79.43709314,225.05223116)(79.44209314,225.0122312)(79.44210083,224.96223694)
\lineto(79.44210083,224.84223694)
\lineto(79.44210083,224.37723694)
\moveto(77.89710083,222.05223694)
\curveto(77.99709458,222.37223384)(78.05709452,222.73723348)(78.07710083,223.14723694)
\curveto(78.09709448,223.55723266)(78.10709447,223.96723225)(78.10710083,224.37723694)
\curveto(78.10709447,224.80723141)(78.09709448,225.22723099)(78.07710083,225.63723694)
\curveto(78.05709452,226.04723017)(78.01209457,226.43222978)(77.94210083,226.79223694)
\curveto(77.87209471,227.15222906)(77.76209482,227.47222874)(77.61210083,227.75223694)
\curveto(77.47209511,228.04222817)(77.2770953,228.27722794)(77.02710083,228.45723694)
\curveto(76.86709571,228.56722765)(76.68709589,228.64722757)(76.48710083,228.69723694)
\curveto(76.28709629,228.75722746)(76.04209654,228.78722743)(75.75210083,228.78723694)
\curveto(75.73209685,228.76722745)(75.69709688,228.75722746)(75.64710083,228.75723694)
\curveto(75.59709698,228.76722745)(75.55709702,228.76722745)(75.52710083,228.75723694)
\curveto(75.44709713,228.73722748)(75.37209721,228.7172275)(75.30210083,228.69723694)
\curveto(75.24209734,228.68722753)(75.1770974,228.66722755)(75.10710083,228.63723694)
\curveto(74.83709774,228.5172277)(74.61709796,228.34722787)(74.44710083,228.12723694)
\curveto(74.28709829,227.9172283)(74.15209843,227.67222854)(74.04210083,227.39223694)
\curveto(73.99209859,227.28222893)(73.95209863,227.16222905)(73.92210083,227.03223694)
\curveto(73.90209868,226.9122293)(73.8770987,226.78722943)(73.84710083,226.65723694)
\curveto(73.82709875,226.60722961)(73.81709876,226.55222966)(73.81710083,226.49223694)
\curveto(73.81709876,226.44222977)(73.81209877,226.39222982)(73.80210083,226.34223694)
\curveto(73.79209879,226.25222996)(73.7820988,226.15723006)(73.77210083,226.05723694)
\curveto(73.76209882,225.96723025)(73.75209883,225.87223034)(73.74210083,225.77223694)
\curveto(73.74209884,225.69223052)(73.73709884,225.60723061)(73.72710083,225.51723694)
\lineto(73.72710083,225.27723694)
\lineto(73.72710083,225.09723694)
\curveto(73.71709886,225.06723115)(73.71209887,225.03223118)(73.71210083,224.99223694)
\lineto(73.71210083,224.85723694)
\lineto(73.71210083,224.40723694)
\curveto(73.71209887,224.32723189)(73.70709887,224.24223197)(73.69710083,224.15223694)
\curveto(73.69709888,224.07223214)(73.70709887,223.99723222)(73.72710083,223.92723694)
\lineto(73.72710083,223.65723694)
\curveto(73.72709885,223.63723258)(73.72209886,223.60723261)(73.71210083,223.56723694)
\curveto(73.71209887,223.53723268)(73.71709886,223.5122327)(73.72710083,223.49223694)
\curveto(73.73709884,223.39223282)(73.74209884,223.29223292)(73.74210083,223.19223694)
\curveto(73.75209883,223.10223311)(73.76209882,223.00223321)(73.77210083,222.89223694)
\curveto(73.80209878,222.77223344)(73.81709876,222.64723357)(73.81710083,222.51723694)
\curveto(73.82709875,222.39723382)(73.85209873,222.28223393)(73.89210083,222.17223694)
\curveto(73.97209861,221.87223434)(74.05709852,221.60723461)(74.14710083,221.37723694)
\curveto(74.24709833,221.14723507)(74.39209819,220.93223528)(74.58210083,220.73223694)
\curveto(74.79209779,220.53223568)(75.05709752,220.38223583)(75.37710083,220.28223694)
\curveto(75.41709716,220.26223595)(75.45209713,220.25223596)(75.48210083,220.25223694)
\curveto(75.52209706,220.26223595)(75.56709701,220.25723596)(75.61710083,220.23723694)
\curveto(75.65709692,220.22723599)(75.72709685,220.217236)(75.82710083,220.20723694)
\curveto(75.93709664,220.19723602)(76.02209656,220.20223601)(76.08210083,220.22223694)
\curveto(76.15209643,220.24223597)(76.22209636,220.25223596)(76.29210083,220.25223694)
\curveto(76.36209622,220.26223595)(76.42709615,220.27723594)(76.48710083,220.29723694)
\curveto(76.68709589,220.35723586)(76.86709571,220.44223577)(77.02710083,220.55223694)
\curveto(77.05709552,220.57223564)(77.0820955,220.59223562)(77.10210083,220.61223694)
\lineto(77.16210083,220.67223694)
\curveto(77.20209538,220.69223552)(77.25209533,220.73223548)(77.31210083,220.79223694)
\curveto(77.41209517,220.93223528)(77.49709508,221.06223515)(77.56710083,221.18223694)
\curveto(77.63709494,221.30223491)(77.70709487,221.44723477)(77.77710083,221.61723694)
\curveto(77.80709477,221.68723453)(77.82709475,221.75723446)(77.83710083,221.82723694)
\curveto(77.85709472,221.89723432)(77.8770947,221.97223424)(77.89710083,222.05223694)
}
}
{
\newrgbcolor{curcolor}{0 0 0}
\pscustom[linestyle=none,fillstyle=solid,fillcolor=curcolor]
{
\newpath
\moveto(56.20288208,304.70223694)
\lineto(61.00288208,304.70223694)
\lineto(62.00788208,304.70223694)
\curveto(62.14787498,304.70222651)(62.26787486,304.69222652)(62.36788208,304.67223694)
\curveto(62.47787465,304.66222655)(62.55787457,304.6172266)(62.60788208,304.53723694)
\curveto(62.6278745,304.49722672)(62.63787449,304.44722677)(62.63788208,304.38723694)
\curveto(62.64787448,304.32722689)(62.65287448,304.26222695)(62.65288208,304.19223694)
\lineto(62.65288208,303.92223694)
\curveto(62.65287448,303.83222738)(62.64287449,303.75222746)(62.62288208,303.68223694)
\curveto(62.58287455,303.60222761)(62.53787459,303.53222768)(62.48788208,303.47223694)
\lineto(62.33788208,303.29223694)
\curveto(62.30787482,303.24222797)(62.27287486,303.20222801)(62.23288208,303.17223694)
\curveto(62.19287494,303.14222807)(62.15287498,303.10222811)(62.11288208,303.05223694)
\curveto(62.0328751,302.94222827)(61.94787518,302.83222838)(61.85788208,302.72223694)
\curveto(61.76787536,302.62222859)(61.68287545,302.5172287)(61.60288208,302.40723694)
\curveto(61.46287567,302.20722901)(61.32287581,301.99722922)(61.18288208,301.77723694)
\curveto(61.04287609,301.56722965)(60.90287623,301.35222986)(60.76288208,301.13223694)
\curveto(60.71287642,301.04223017)(60.66287647,300.94723027)(60.61288208,300.84723694)
\curveto(60.56287657,300.74723047)(60.50787662,300.65223056)(60.44788208,300.56223694)
\curveto(60.4278767,300.54223067)(60.41787671,300.5172307)(60.41788208,300.48723694)
\curveto(60.41787671,300.45723076)(60.40787672,300.43223078)(60.38788208,300.41223694)
\curveto(60.31787681,300.3122309)(60.25287688,300.19723102)(60.19288208,300.06723694)
\curveto(60.132877,299.94723127)(60.07787705,299.83223138)(60.02788208,299.72223694)
\curveto(59.9278772,299.49223172)(59.8328773,299.25723196)(59.74288208,299.01723694)
\curveto(59.65287748,298.77723244)(59.55287758,298.53723268)(59.44288208,298.29723694)
\curveto(59.42287771,298.24723297)(59.40787772,298.20223301)(59.39788208,298.16223694)
\curveto(59.39787773,298.12223309)(59.38787774,298.07723314)(59.36788208,298.02723694)
\curveto(59.31787781,297.90723331)(59.27287786,297.78223343)(59.23288208,297.65223694)
\curveto(59.20287793,297.53223368)(59.16787796,297.4122338)(59.12788208,297.29223694)
\curveto(59.04787808,297.06223415)(58.98287815,296.82223439)(58.93288208,296.57223694)
\curveto(58.89287824,296.33223488)(58.84287829,296.09223512)(58.78288208,295.85223694)
\curveto(58.74287839,295.70223551)(58.71787841,295.55223566)(58.70788208,295.40223694)
\curveto(58.69787843,295.25223596)(58.67787845,295.10223611)(58.64788208,294.95223694)
\curveto(58.63787849,294.9122363)(58.6328785,294.85223636)(58.63288208,294.77223694)
\curveto(58.60287853,294.65223656)(58.57287856,294.55223666)(58.54288208,294.47223694)
\curveto(58.51287862,294.39223682)(58.44287869,294.33723688)(58.33288208,294.30723694)
\curveto(58.28287885,294.28723693)(58.2278789,294.27723694)(58.16788208,294.27723694)
\lineto(57.97288208,294.27723694)
\curveto(57.8328793,294.27723694)(57.69287944,294.28223693)(57.55288208,294.29223694)
\curveto(57.42287971,294.30223691)(57.3278798,294.34723687)(57.26788208,294.42723694)
\curveto(57.2278799,294.48723673)(57.20787992,294.57223664)(57.20788208,294.68223694)
\curveto(57.21787991,294.79223642)(57.2328799,294.88723633)(57.25288208,294.96723694)
\lineto(57.25288208,295.04223694)
\curveto(57.26287987,295.07223614)(57.26787986,295.10223611)(57.26788208,295.13223694)
\curveto(57.28787984,295.212236)(57.29787983,295.28723593)(57.29788208,295.35723694)
\curveto(57.29787983,295.42723579)(57.30787982,295.49723572)(57.32788208,295.56723694)
\curveto(57.37787975,295.75723546)(57.41787971,295.94223527)(57.44788208,296.12223694)
\curveto(57.47787965,296.3122349)(57.51787961,296.49223472)(57.56788208,296.66223694)
\curveto(57.58787954,296.7122345)(57.59787953,296.75223446)(57.59788208,296.78223694)
\curveto(57.59787953,296.8122344)(57.60287953,296.84723437)(57.61288208,296.88723694)
\curveto(57.71287942,297.18723403)(57.80287933,297.48223373)(57.88288208,297.77223694)
\curveto(57.97287916,298.06223315)(58.07787905,298.34223287)(58.19788208,298.61223694)
\curveto(58.45787867,299.19223202)(58.7278784,299.74223147)(59.00788208,300.26223694)
\curveto(59.28787784,300.79223042)(59.59787753,301.29722992)(59.93788208,301.77723694)
\curveto(60.07787705,301.97722924)(60.2278769,302.16722905)(60.38788208,302.34723694)
\curveto(60.54787658,302.53722868)(60.69787643,302.72722849)(60.83788208,302.91723694)
\curveto(60.87787625,302.96722825)(60.91287622,303.0122282)(60.94288208,303.05223694)
\curveto(60.98287615,303.10222811)(61.01787611,303.15222806)(61.04788208,303.20223694)
\curveto(61.05787607,303.22222799)(61.06787606,303.24722797)(61.07788208,303.27723694)
\curveto(61.09787603,303.30722791)(61.09787603,303.33722788)(61.07788208,303.36723694)
\curveto(61.05787607,303.42722779)(61.02287611,303.46222775)(60.97288208,303.47223694)
\curveto(60.92287621,303.49222772)(60.87287626,303.5122277)(60.82288208,303.53223694)
\lineto(60.71788208,303.53223694)
\curveto(60.67787645,303.54222767)(60.6278765,303.54222767)(60.56788208,303.53223694)
\lineto(60.41788208,303.53223694)
\lineto(59.81788208,303.53223694)
\lineto(57.17788208,303.53223694)
\lineto(56.44288208,303.53223694)
\lineto(56.20288208,303.53223694)
\curveto(56.132881,303.54222767)(56.07288106,303.55722766)(56.02288208,303.57723694)
\curveto(55.9328812,303.6172276)(55.87288126,303.67722754)(55.84288208,303.75723694)
\curveto(55.79288134,303.85722736)(55.77788135,304.00222721)(55.79788208,304.19223694)
\curveto(55.81788131,304.39222682)(55.85288128,304.52722669)(55.90288208,304.59723694)
\curveto(55.92288121,304.6172266)(55.94788118,304.63222658)(55.97788208,304.64223694)
\lineto(56.09788208,304.70223694)
\curveto(56.11788101,304.70222651)(56.132881,304.69722652)(56.14288208,304.68723694)
\curveto(56.16288097,304.68722653)(56.18288095,304.69222652)(56.20288208,304.70223694)
}
}
{
\newrgbcolor{curcolor}{0 0 0}
\pscustom[linestyle=none,fillstyle=solid,fillcolor=curcolor]
{
\newpath
\moveto(65.60249146,304.70223694)
\lineto(69.20249146,304.70223694)
\lineto(69.84749146,304.70223694)
\curveto(69.92748493,304.70222651)(70.00248485,304.69722652)(70.07249146,304.68723694)
\curveto(70.14248471,304.68722653)(70.20248465,304.67722654)(70.25249146,304.65723694)
\curveto(70.32248453,304.62722659)(70.37748448,304.56722665)(70.41749146,304.47723694)
\curveto(70.43748442,304.44722677)(70.44748441,304.40722681)(70.44749146,304.35723694)
\lineto(70.44749146,304.22223694)
\curveto(70.4574844,304.1122271)(70.4524844,304.00722721)(70.43249146,303.90723694)
\curveto(70.42248443,303.80722741)(70.38748447,303.73722748)(70.32749146,303.69723694)
\curveto(70.23748462,303.62722759)(70.10248475,303.59222762)(69.92249146,303.59223694)
\curveto(69.74248511,303.60222761)(69.57748528,303.60722761)(69.42749146,303.60723694)
\lineto(67.43249146,303.60723694)
\lineto(66.93749146,303.60723694)
\lineto(66.80249146,303.60723694)
\curveto(66.76248809,303.60722761)(66.72248813,303.60222761)(66.68249146,303.59223694)
\lineto(66.47249146,303.59223694)
\curveto(66.36248849,303.56222765)(66.28248857,303.52222769)(66.23249146,303.47223694)
\curveto(66.18248867,303.43222778)(66.14748871,303.37722784)(66.12749146,303.30723694)
\curveto(66.10748875,303.24722797)(66.09248876,303.17722804)(66.08249146,303.09723694)
\curveto(66.07248878,303.0172282)(66.0524888,302.92722829)(66.02249146,302.82723694)
\curveto(65.97248888,302.62722859)(65.93248892,302.42222879)(65.90249146,302.21223694)
\curveto(65.87248898,302.00222921)(65.83248902,301.79722942)(65.78249146,301.59723694)
\curveto(65.76248909,301.52722969)(65.7524891,301.45722976)(65.75249146,301.38723694)
\curveto(65.7524891,301.32722989)(65.74248911,301.26222995)(65.72249146,301.19223694)
\curveto(65.71248914,301.16223005)(65.70248915,301.12223009)(65.69249146,301.07223694)
\curveto(65.69248916,301.03223018)(65.69748916,300.99223022)(65.70749146,300.95223694)
\curveto(65.72748913,300.90223031)(65.7524891,300.85723036)(65.78249146,300.81723694)
\curveto(65.82248903,300.78723043)(65.88248897,300.78223043)(65.96249146,300.80223694)
\curveto(66.02248883,300.82223039)(66.08248877,300.84723037)(66.14249146,300.87723694)
\curveto(66.20248865,300.9172303)(66.26248859,300.95223026)(66.32249146,300.98223694)
\curveto(66.38248847,301.00223021)(66.43248842,301.0172302)(66.47249146,301.02723694)
\curveto(66.66248819,301.10723011)(66.86748799,301.16223005)(67.08749146,301.19223694)
\curveto(67.31748754,301.22222999)(67.54748731,301.23222998)(67.77749146,301.22223694)
\curveto(68.01748684,301.22222999)(68.24748661,301.19723002)(68.46749146,301.14723694)
\curveto(68.68748617,301.10723011)(68.88748597,301.04723017)(69.06749146,300.96723694)
\curveto(69.11748574,300.94723027)(69.16248569,300.92723029)(69.20249146,300.90723694)
\curveto(69.2524856,300.88723033)(69.30248555,300.86223035)(69.35249146,300.83223694)
\curveto(69.70248515,300.62223059)(69.98248487,300.39223082)(70.19249146,300.14223694)
\curveto(70.41248444,299.89223132)(70.60748425,299.56723165)(70.77749146,299.16723694)
\curveto(70.82748403,299.05723216)(70.86248399,298.94723227)(70.88249146,298.83723694)
\curveto(70.90248395,298.72723249)(70.92748393,298.6122326)(70.95749146,298.49223694)
\curveto(70.96748389,298.46223275)(70.97248388,298.4172328)(70.97249146,298.35723694)
\curveto(70.99248386,298.29723292)(71.00248385,298.22723299)(71.00249146,298.14723694)
\curveto(71.00248385,298.07723314)(71.01248384,298.0122332)(71.03249146,297.95223694)
\lineto(71.03249146,297.78723694)
\curveto(71.04248381,297.73723348)(71.04748381,297.66723355)(71.04749146,297.57723694)
\curveto(71.04748381,297.48723373)(71.03748382,297.4172338)(71.01749146,297.36723694)
\curveto(70.99748386,297.30723391)(70.99248386,297.24723397)(71.00249146,297.18723694)
\curveto(71.01248384,297.13723408)(71.00748385,297.08723413)(70.98749146,297.03723694)
\curveto(70.94748391,296.87723434)(70.91248394,296.72723449)(70.88249146,296.58723694)
\curveto(70.852484,296.44723477)(70.80748405,296.3122349)(70.74749146,296.18223694)
\curveto(70.58748427,295.8122354)(70.36748449,295.47723574)(70.08749146,295.17723694)
\curveto(69.80748505,294.87723634)(69.48748537,294.64723657)(69.12749146,294.48723694)
\curveto(68.9574859,294.40723681)(68.7574861,294.33223688)(68.52749146,294.26223694)
\curveto(68.41748644,294.22223699)(68.30248655,294.19723702)(68.18249146,294.18723694)
\curveto(68.06248679,294.17723704)(67.94248691,294.15723706)(67.82249146,294.12723694)
\curveto(67.77248708,294.10723711)(67.71748714,294.10723711)(67.65749146,294.12723694)
\curveto(67.59748726,294.13723708)(67.53748732,294.13223708)(67.47749146,294.11223694)
\curveto(67.37748748,294.09223712)(67.27748758,294.09223712)(67.17749146,294.11223694)
\lineto(67.04249146,294.11223694)
\curveto(66.99248786,294.13223708)(66.93248792,294.14223707)(66.86249146,294.14223694)
\curveto(66.80248805,294.13223708)(66.74748811,294.13723708)(66.69749146,294.15723694)
\curveto(66.6574882,294.16723705)(66.62248823,294.17223704)(66.59249146,294.17223694)
\curveto(66.56248829,294.17223704)(66.52748833,294.17723704)(66.48749146,294.18723694)
\lineto(66.21749146,294.24723694)
\curveto(66.12748873,294.26723695)(66.04248881,294.29723692)(65.96249146,294.33723694)
\curveto(65.62248923,294.47723674)(65.33248952,294.63223658)(65.09249146,294.80223694)
\curveto(64.85249,294.98223623)(64.63249022,295.212236)(64.43249146,295.49223694)
\curveto(64.28249057,295.72223549)(64.16749069,295.96223525)(64.08749146,296.21223694)
\curveto(64.06749079,296.26223495)(64.0574908,296.30723491)(64.05749146,296.34723694)
\curveto(64.0574908,296.39723482)(64.04749081,296.44723477)(64.02749146,296.49723694)
\curveto(64.00749085,296.55723466)(63.99249086,296.63723458)(63.98249146,296.73723694)
\curveto(63.98249087,296.83723438)(64.00249085,296.9122343)(64.04249146,296.96223694)
\curveto(64.09249076,297.04223417)(64.17249068,297.08723413)(64.28249146,297.09723694)
\curveto(64.39249046,297.10723411)(64.50749035,297.1122341)(64.62749146,297.11223694)
\lineto(64.79249146,297.11223694)
\curveto(64.85249,297.1122341)(64.90748995,297.10223411)(64.95749146,297.08223694)
\curveto(65.04748981,297.06223415)(65.11748974,297.02223419)(65.16749146,296.96223694)
\curveto(65.23748962,296.87223434)(65.28248957,296.76223445)(65.30249146,296.63223694)
\curveto(65.33248952,296.5122347)(65.37748948,296.40723481)(65.43749146,296.31723694)
\curveto(65.62748923,295.97723524)(65.88748897,295.70723551)(66.21749146,295.50723694)
\curveto(66.31748854,295.44723577)(66.42248843,295.39723582)(66.53249146,295.35723694)
\curveto(66.6524882,295.32723589)(66.77248808,295.29223592)(66.89249146,295.25223694)
\curveto(67.06248779,295.20223601)(67.26748759,295.18223603)(67.50749146,295.19223694)
\curveto(67.7574871,295.212236)(67.9574869,295.24723597)(68.10749146,295.29723694)
\curveto(68.47748638,295.4172358)(68.76748609,295.57723564)(68.97749146,295.77723694)
\curveto(69.19748566,295.98723523)(69.37748548,296.26723495)(69.51749146,296.61723694)
\curveto(69.56748529,296.7172345)(69.59748526,296.82223439)(69.60749146,296.93223694)
\curveto(69.62748523,297.04223417)(69.6524852,297.15723406)(69.68249146,297.27723694)
\lineto(69.68249146,297.38223694)
\curveto(69.69248516,297.42223379)(69.69748516,297.46223375)(69.69749146,297.50223694)
\curveto(69.70748515,297.53223368)(69.70748515,297.56723365)(69.69749146,297.60723694)
\lineto(69.69749146,297.72723694)
\curveto(69.69748516,297.98723323)(69.66748519,298.23223298)(69.60749146,298.46223694)
\curveto(69.49748536,298.8122324)(69.34248551,299.10723211)(69.14249146,299.34723694)
\curveto(68.94248591,299.59723162)(68.68248617,299.79223142)(68.36249146,299.93223694)
\lineto(68.18249146,299.99223694)
\curveto(68.13248672,300.0122312)(68.07248678,300.03223118)(68.00249146,300.05223694)
\curveto(67.9524869,300.07223114)(67.89248696,300.08223113)(67.82249146,300.08223694)
\curveto(67.76248709,300.09223112)(67.69748716,300.10723111)(67.62749146,300.12723694)
\lineto(67.47749146,300.12723694)
\curveto(67.43748742,300.14723107)(67.38248747,300.15723106)(67.31249146,300.15723694)
\curveto(67.2524876,300.15723106)(67.19748766,300.14723107)(67.14749146,300.12723694)
\lineto(67.04249146,300.12723694)
\curveto(67.01248784,300.12723109)(66.97748788,300.12223109)(66.93749146,300.11223694)
\lineto(66.69749146,300.05223694)
\curveto(66.61748824,300.04223117)(66.53748832,300.02223119)(66.45749146,299.99223694)
\curveto(66.21748864,299.89223132)(65.98748887,299.75723146)(65.76749146,299.58723694)
\curveto(65.67748918,299.5172317)(65.59248926,299.44223177)(65.51249146,299.36223694)
\curveto(65.43248942,299.29223192)(65.33248952,299.23723198)(65.21249146,299.19723694)
\curveto(65.12248973,299.16723205)(64.98248987,299.15723206)(64.79249146,299.16723694)
\curveto(64.61249024,299.17723204)(64.49249036,299.20223201)(64.43249146,299.24223694)
\curveto(64.38249047,299.28223193)(64.34249051,299.34223187)(64.31249146,299.42223694)
\curveto(64.29249056,299.50223171)(64.29249056,299.58723163)(64.31249146,299.67723694)
\curveto(64.34249051,299.79723142)(64.36249049,299.9172313)(64.37249146,300.03723694)
\curveto(64.39249046,300.16723105)(64.41749044,300.29223092)(64.44749146,300.41223694)
\curveto(64.46749039,300.45223076)(64.47249038,300.48723073)(64.46249146,300.51723694)
\curveto(64.46249039,300.55723066)(64.47249038,300.60223061)(64.49249146,300.65223694)
\curveto(64.51249034,300.74223047)(64.52749033,300.83223038)(64.53749146,300.92223694)
\curveto(64.54749031,301.02223019)(64.56749029,301.1172301)(64.59749146,301.20723694)
\curveto(64.60749025,301.26722995)(64.61249024,301.32722989)(64.61249146,301.38723694)
\curveto(64.62249023,301.44722977)(64.63749022,301.50722971)(64.65749146,301.56723694)
\curveto(64.70749015,301.76722945)(64.74249011,301.97222924)(64.76249146,302.18223694)
\curveto(64.79249006,302.40222881)(64.83249002,302.6122286)(64.88249146,302.81223694)
\curveto(64.91248994,302.9122283)(64.93248992,303.0122282)(64.94249146,303.11223694)
\curveto(64.9524899,303.212228)(64.96748989,303.3122279)(64.98749146,303.41223694)
\curveto(64.99748986,303.44222777)(65.00248985,303.48222773)(65.00249146,303.53223694)
\curveto(65.03248982,303.64222757)(65.0524898,303.74722747)(65.06249146,303.84723694)
\curveto(65.08248977,303.95722726)(65.10748975,304.06722715)(65.13749146,304.17723694)
\curveto(65.1574897,304.25722696)(65.17248968,304.32722689)(65.18249146,304.38723694)
\curveto(65.19248966,304.45722676)(65.21748964,304.5172267)(65.25749146,304.56723694)
\curveto(65.27748958,304.59722662)(65.30748955,304.6172266)(65.34749146,304.62723694)
\curveto(65.38748947,304.64722657)(65.43248942,304.66722655)(65.48249146,304.68723694)
\curveto(65.54248931,304.68722653)(65.58248927,304.69222652)(65.60249146,304.70223694)
}
}
{
\newrgbcolor{curcolor}{0 0 0}
\pscustom[linestyle=none,fillstyle=solid,fillcolor=curcolor]
{
\newpath
\moveto(79.44210083,299.37723694)
\lineto(79.44210083,299.12223694)
\curveto(79.45209313,299.04223217)(79.44709313,298.96723225)(79.42710083,298.89723694)
\lineto(79.42710083,298.65723694)
\lineto(79.42710083,298.49223694)
\curveto(79.40709317,298.39223282)(79.39709318,298.28723293)(79.39710083,298.17723694)
\curveto(79.39709318,298.07723314)(79.38709319,297.97723324)(79.36710083,297.87723694)
\lineto(79.36710083,297.72723694)
\curveto(79.33709324,297.58723363)(79.31709326,297.44723377)(79.30710083,297.30723694)
\curveto(79.29709328,297.17723404)(79.27209331,297.04723417)(79.23210083,296.91723694)
\curveto(79.21209337,296.83723438)(79.19209339,296.75223446)(79.17210083,296.66223694)
\lineto(79.11210083,296.42223694)
\lineto(78.99210083,296.12223694)
\curveto(78.96209362,296.03223518)(78.92709365,295.94223527)(78.88710083,295.85223694)
\curveto(78.78709379,295.63223558)(78.65209393,295.4172358)(78.48210083,295.20723694)
\curveto(78.32209426,294.99723622)(78.14709443,294.82723639)(77.95710083,294.69723694)
\curveto(77.90709467,294.65723656)(77.84709473,294.6172366)(77.77710083,294.57723694)
\curveto(77.71709486,294.54723667)(77.65709492,294.5122367)(77.59710083,294.47223694)
\curveto(77.51709506,294.42223679)(77.42209516,294.38223683)(77.31210083,294.35223694)
\curveto(77.20209538,294.32223689)(77.09709548,294.29223692)(76.99710083,294.26223694)
\curveto(76.88709569,294.22223699)(76.7770958,294.19723702)(76.66710083,294.18723694)
\curveto(76.55709602,294.17723704)(76.44209614,294.16223705)(76.32210083,294.14223694)
\curveto(76.2820963,294.13223708)(76.23709634,294.13223708)(76.18710083,294.14223694)
\curveto(76.14709643,294.14223707)(76.10709647,294.13723708)(76.06710083,294.12723694)
\curveto(76.02709655,294.1172371)(75.97209661,294.1122371)(75.90210083,294.11223694)
\curveto(75.83209675,294.1122371)(75.7820968,294.1172371)(75.75210083,294.12723694)
\curveto(75.70209688,294.14723707)(75.65709692,294.15223706)(75.61710083,294.14223694)
\curveto(75.577097,294.13223708)(75.54209704,294.13223708)(75.51210083,294.14223694)
\lineto(75.42210083,294.14223694)
\curveto(75.36209722,294.16223705)(75.29709728,294.17723704)(75.22710083,294.18723694)
\curveto(75.16709741,294.18723703)(75.10209748,294.19223702)(75.03210083,294.20223694)
\curveto(74.86209772,294.25223696)(74.70209788,294.30223691)(74.55210083,294.35223694)
\curveto(74.40209818,294.40223681)(74.25709832,294.46723675)(74.11710083,294.54723694)
\curveto(74.06709851,294.58723663)(74.01209857,294.6172366)(73.95210083,294.63723694)
\curveto(73.90209868,294.66723655)(73.85209873,294.70223651)(73.80210083,294.74223694)
\curveto(73.56209902,294.92223629)(73.36209922,295.14223607)(73.20210083,295.40223694)
\curveto(73.04209954,295.66223555)(72.90209968,295.94723527)(72.78210083,296.25723694)
\curveto(72.72209986,296.39723482)(72.6770999,296.53723468)(72.64710083,296.67723694)
\curveto(72.61709996,296.82723439)(72.5821,296.98223423)(72.54210083,297.14223694)
\curveto(72.52210006,297.25223396)(72.50710007,297.36223385)(72.49710083,297.47223694)
\curveto(72.48710009,297.58223363)(72.47210011,297.69223352)(72.45210083,297.80223694)
\curveto(72.44210014,297.84223337)(72.43710014,297.88223333)(72.43710083,297.92223694)
\curveto(72.44710013,297.96223325)(72.44710013,298.00223321)(72.43710083,298.04223694)
\curveto(72.42710015,298.09223312)(72.42210016,298.14223307)(72.42210083,298.19223694)
\lineto(72.42210083,298.35723694)
\curveto(72.40210018,298.40723281)(72.39710018,298.45723276)(72.40710083,298.50723694)
\curveto(72.41710016,298.56723265)(72.41710016,298.62223259)(72.40710083,298.67223694)
\curveto(72.39710018,298.7122325)(72.39710018,298.75723246)(72.40710083,298.80723694)
\curveto(72.41710016,298.85723236)(72.41210017,298.90723231)(72.39210083,298.95723694)
\curveto(72.37210021,299.02723219)(72.36710021,299.10223211)(72.37710083,299.18223694)
\curveto(72.38710019,299.27223194)(72.39210019,299.35723186)(72.39210083,299.43723694)
\curveto(72.39210019,299.52723169)(72.38710019,299.62723159)(72.37710083,299.73723694)
\curveto(72.36710021,299.85723136)(72.37210021,299.95723126)(72.39210083,300.03723694)
\lineto(72.39210083,300.32223694)
\lineto(72.43710083,300.95223694)
\curveto(72.44710013,301.05223016)(72.45710012,301.14723007)(72.46710083,301.23723694)
\lineto(72.49710083,301.53723694)
\curveto(72.51710006,301.58722963)(72.52210006,301.63722958)(72.51210083,301.68723694)
\curveto(72.51210007,301.74722947)(72.52210006,301.80222941)(72.54210083,301.85223694)
\curveto(72.59209999,302.02222919)(72.63209995,302.18722903)(72.66210083,302.34723694)
\curveto(72.69209989,302.5172287)(72.74209984,302.67722854)(72.81210083,302.82723694)
\curveto(73.00209958,303.28722793)(73.22209936,303.66222755)(73.47210083,303.95223694)
\curveto(73.73209885,304.24222697)(74.09209849,304.48722673)(74.55210083,304.68723694)
\curveto(74.6820979,304.73722648)(74.81209777,304.77222644)(74.94210083,304.79223694)
\curveto(75.0820975,304.8122264)(75.22209736,304.83722638)(75.36210083,304.86723694)
\curveto(75.43209715,304.87722634)(75.49709708,304.88222633)(75.55710083,304.88223694)
\curveto(75.61709696,304.88222633)(75.6820969,304.88722633)(75.75210083,304.89723694)
\curveto(76.582096,304.9172263)(77.25209533,304.76722645)(77.76210083,304.44723694)
\curveto(78.27209431,304.13722708)(78.65209393,303.69722752)(78.90210083,303.12723694)
\curveto(78.95209363,303.00722821)(78.99709358,302.88222833)(79.03710083,302.75223694)
\curveto(79.0770935,302.62222859)(79.12209346,302.48722873)(79.17210083,302.34723694)
\curveto(79.19209339,302.26722895)(79.20709337,302.18222903)(79.21710083,302.09223694)
\lineto(79.27710083,301.85223694)
\curveto(79.30709327,301.74222947)(79.32209326,301.63222958)(79.32210083,301.52223694)
\curveto(79.33209325,301.4122298)(79.34709323,301.30222991)(79.36710083,301.19223694)
\curveto(79.38709319,301.14223007)(79.39209319,301.09723012)(79.38210083,301.05723694)
\curveto(79.3820932,301.0172302)(79.38709319,300.97723024)(79.39710083,300.93723694)
\curveto(79.40709317,300.88723033)(79.40709317,300.83223038)(79.39710083,300.77223694)
\curveto(79.39709318,300.72223049)(79.40209318,300.67223054)(79.41210083,300.62223694)
\lineto(79.41210083,300.48723694)
\curveto(79.43209315,300.42723079)(79.43209315,300.35723086)(79.41210083,300.27723694)
\curveto(79.40209318,300.20723101)(79.40709317,300.14223107)(79.42710083,300.08223694)
\curveto(79.43709314,300.05223116)(79.44209314,300.0122312)(79.44210083,299.96223694)
\lineto(79.44210083,299.84223694)
\lineto(79.44210083,299.37723694)
\moveto(77.89710083,297.05223694)
\curveto(77.99709458,297.37223384)(78.05709452,297.73723348)(78.07710083,298.14723694)
\curveto(78.09709448,298.55723266)(78.10709447,298.96723225)(78.10710083,299.37723694)
\curveto(78.10709447,299.80723141)(78.09709448,300.22723099)(78.07710083,300.63723694)
\curveto(78.05709452,301.04723017)(78.01209457,301.43222978)(77.94210083,301.79223694)
\curveto(77.87209471,302.15222906)(77.76209482,302.47222874)(77.61210083,302.75223694)
\curveto(77.47209511,303.04222817)(77.2770953,303.27722794)(77.02710083,303.45723694)
\curveto(76.86709571,303.56722765)(76.68709589,303.64722757)(76.48710083,303.69723694)
\curveto(76.28709629,303.75722746)(76.04209654,303.78722743)(75.75210083,303.78723694)
\curveto(75.73209685,303.76722745)(75.69709688,303.75722746)(75.64710083,303.75723694)
\curveto(75.59709698,303.76722745)(75.55709702,303.76722745)(75.52710083,303.75723694)
\curveto(75.44709713,303.73722748)(75.37209721,303.7172275)(75.30210083,303.69723694)
\curveto(75.24209734,303.68722753)(75.1770974,303.66722755)(75.10710083,303.63723694)
\curveto(74.83709774,303.5172277)(74.61709796,303.34722787)(74.44710083,303.12723694)
\curveto(74.28709829,302.9172283)(74.15209843,302.67222854)(74.04210083,302.39223694)
\curveto(73.99209859,302.28222893)(73.95209863,302.16222905)(73.92210083,302.03223694)
\curveto(73.90209868,301.9122293)(73.8770987,301.78722943)(73.84710083,301.65723694)
\curveto(73.82709875,301.60722961)(73.81709876,301.55222966)(73.81710083,301.49223694)
\curveto(73.81709876,301.44222977)(73.81209877,301.39222982)(73.80210083,301.34223694)
\curveto(73.79209879,301.25222996)(73.7820988,301.15723006)(73.77210083,301.05723694)
\curveto(73.76209882,300.96723025)(73.75209883,300.87223034)(73.74210083,300.77223694)
\curveto(73.74209884,300.69223052)(73.73709884,300.60723061)(73.72710083,300.51723694)
\lineto(73.72710083,300.27723694)
\lineto(73.72710083,300.09723694)
\curveto(73.71709886,300.06723115)(73.71209887,300.03223118)(73.71210083,299.99223694)
\lineto(73.71210083,299.85723694)
\lineto(73.71210083,299.40723694)
\curveto(73.71209887,299.32723189)(73.70709887,299.24223197)(73.69710083,299.15223694)
\curveto(73.69709888,299.07223214)(73.70709887,298.99723222)(73.72710083,298.92723694)
\lineto(73.72710083,298.65723694)
\curveto(73.72709885,298.63723258)(73.72209886,298.60723261)(73.71210083,298.56723694)
\curveto(73.71209887,298.53723268)(73.71709886,298.5122327)(73.72710083,298.49223694)
\curveto(73.73709884,298.39223282)(73.74209884,298.29223292)(73.74210083,298.19223694)
\curveto(73.75209883,298.10223311)(73.76209882,298.00223321)(73.77210083,297.89223694)
\curveto(73.80209878,297.77223344)(73.81709876,297.64723357)(73.81710083,297.51723694)
\curveto(73.82709875,297.39723382)(73.85209873,297.28223393)(73.89210083,297.17223694)
\curveto(73.97209861,296.87223434)(74.05709852,296.60723461)(74.14710083,296.37723694)
\curveto(74.24709833,296.14723507)(74.39209819,295.93223528)(74.58210083,295.73223694)
\curveto(74.79209779,295.53223568)(75.05709752,295.38223583)(75.37710083,295.28223694)
\curveto(75.41709716,295.26223595)(75.45209713,295.25223596)(75.48210083,295.25223694)
\curveto(75.52209706,295.26223595)(75.56709701,295.25723596)(75.61710083,295.23723694)
\curveto(75.65709692,295.22723599)(75.72709685,295.217236)(75.82710083,295.20723694)
\curveto(75.93709664,295.19723602)(76.02209656,295.20223601)(76.08210083,295.22223694)
\curveto(76.15209643,295.24223597)(76.22209636,295.25223596)(76.29210083,295.25223694)
\curveto(76.36209622,295.26223595)(76.42709615,295.27723594)(76.48710083,295.29723694)
\curveto(76.68709589,295.35723586)(76.86709571,295.44223577)(77.02710083,295.55223694)
\curveto(77.05709552,295.57223564)(77.0820955,295.59223562)(77.10210083,295.61223694)
\lineto(77.16210083,295.67223694)
\curveto(77.20209538,295.69223552)(77.25209533,295.73223548)(77.31210083,295.79223694)
\curveto(77.41209517,295.93223528)(77.49709508,296.06223515)(77.56710083,296.18223694)
\curveto(77.63709494,296.30223491)(77.70709487,296.44723477)(77.77710083,296.61723694)
\curveto(77.80709477,296.68723453)(77.82709475,296.75723446)(77.83710083,296.82723694)
\curveto(77.85709472,296.89723432)(77.8770947,296.97223424)(77.89710083,297.05223694)
}
}
{
\newrgbcolor{curcolor}{0 0 0}
\pscustom[linestyle=none,fillstyle=solid,fillcolor=curcolor]
{
\newpath
\moveto(51.45327271,379.18295013)
\curveto(51.55326785,379.18293951)(51.64826776,379.17293952)(51.73827271,379.15295013)
\curveto(51.82826758,379.14293955)(51.89326751,379.11293958)(51.93327271,379.06295013)
\curveto(51.99326741,378.98293971)(52.02326738,378.87793982)(52.02327271,378.74795013)
\lineto(52.02327271,378.35795013)
\lineto(52.02327271,376.85795013)
\lineto(52.02327271,370.46795013)
\lineto(52.02327271,369.29795013)
\lineto(52.02327271,368.98295013)
\curveto(52.03326737,368.88294981)(52.01826739,368.80294989)(51.97827271,368.74295013)
\curveto(51.92826748,368.66295003)(51.85326755,368.61295008)(51.75327271,368.59295013)
\curveto(51.66326774,368.58295011)(51.55326785,368.57795012)(51.42327271,368.57795013)
\lineto(51.19827271,368.57795013)
\curveto(51.11826829,368.5979501)(51.04826836,368.61295008)(50.98827271,368.62295013)
\curveto(50.92826848,368.64295005)(50.87826853,368.68295001)(50.83827271,368.74295013)
\curveto(50.79826861,368.80294989)(50.77826863,368.87794982)(50.77827271,368.96795013)
\lineto(50.77827271,369.26795013)
\lineto(50.77827271,370.36295013)
\lineto(50.77827271,375.70295013)
\curveto(50.75826865,375.7929429)(50.74326866,375.86794283)(50.73327271,375.92795013)
\curveto(50.73326867,375.9979427)(50.7032687,376.05794264)(50.64327271,376.10795013)
\curveto(50.57326883,376.15794254)(50.48326892,376.18294251)(50.37327271,376.18295013)
\curveto(50.27326913,376.1929425)(50.16326924,376.1979425)(50.04327271,376.19795013)
\lineto(48.90327271,376.19795013)
\lineto(48.40827271,376.19795013)
\curveto(48.24827116,376.20794249)(48.13827127,376.26794243)(48.07827271,376.37795013)
\curveto(48.05827135,376.40794229)(48.04827136,376.43794226)(48.04827271,376.46795013)
\curveto(48.04827136,376.50794219)(48.04327136,376.55294214)(48.03327271,376.60295013)
\curveto(48.01327139,376.72294197)(48.01827139,376.83294186)(48.04827271,376.93295013)
\curveto(48.08827132,377.03294166)(48.14327126,377.10294159)(48.21327271,377.14295013)
\curveto(48.29327111,377.1929415)(48.41327099,377.21794148)(48.57327271,377.21795013)
\curveto(48.73327067,377.21794148)(48.86827054,377.23294146)(48.97827271,377.26295013)
\curveto(49.02827038,377.27294142)(49.08327032,377.27794142)(49.14327271,377.27795013)
\curveto(49.2032702,377.28794141)(49.26327014,377.30294139)(49.32327271,377.32295013)
\curveto(49.47326993,377.37294132)(49.61826979,377.42294127)(49.75827271,377.47295013)
\curveto(49.89826951,377.53294116)(50.03326937,377.60294109)(50.16327271,377.68295013)
\curveto(50.3032691,377.77294092)(50.42326898,377.87794082)(50.52327271,377.99795013)
\curveto(50.62326878,378.11794058)(50.71826869,378.24794045)(50.80827271,378.38795013)
\curveto(50.86826854,378.48794021)(50.91326849,378.5979401)(50.94327271,378.71795013)
\curveto(50.98326842,378.83793986)(51.03326837,378.94293975)(51.09327271,379.03295013)
\curveto(51.14326826,379.0929396)(51.21326819,379.13293956)(51.30327271,379.15295013)
\curveto(51.32326808,379.16293953)(51.34826806,379.16793953)(51.37827271,379.16795013)
\curveto(51.408268,379.16793953)(51.43326797,379.17293952)(51.45327271,379.18295013)
}
}
{
\newrgbcolor{curcolor}{0 0 0}
\pscustom[linestyle=none,fillstyle=solid,fillcolor=curcolor]
{
\newpath
\moveto(62.74288208,373.66295013)
\lineto(62.74288208,373.40795013)
\curveto(62.75287438,373.32794537)(62.74787438,373.25294544)(62.72788208,373.18295013)
\lineto(62.72788208,372.94295013)
\lineto(62.72788208,372.77795013)
\curveto(62.70787442,372.67794602)(62.69787443,372.57294612)(62.69788208,372.46295013)
\curveto(62.69787443,372.36294633)(62.68787444,372.26294643)(62.66788208,372.16295013)
\lineto(62.66788208,372.01295013)
\curveto(62.63787449,371.87294682)(62.61787451,371.73294696)(62.60788208,371.59295013)
\curveto(62.59787453,371.46294723)(62.57287456,371.33294736)(62.53288208,371.20295013)
\curveto(62.51287462,371.12294757)(62.49287464,371.03794766)(62.47288208,370.94795013)
\lineto(62.41288208,370.70795013)
\lineto(62.29288208,370.40795013)
\curveto(62.26287487,370.31794838)(62.2278749,370.22794847)(62.18788208,370.13795013)
\curveto(62.08787504,369.91794878)(61.95287518,369.70294899)(61.78288208,369.49295013)
\curveto(61.62287551,369.28294941)(61.44787568,369.11294958)(61.25788208,368.98295013)
\curveto(61.20787592,368.94294975)(61.14787598,368.90294979)(61.07788208,368.86295013)
\curveto(61.01787611,368.83294986)(60.95787617,368.7979499)(60.89788208,368.75795013)
\curveto(60.81787631,368.70794999)(60.72287641,368.66795003)(60.61288208,368.63795013)
\curveto(60.50287663,368.60795009)(60.39787673,368.57795012)(60.29788208,368.54795013)
\curveto(60.18787694,368.50795019)(60.07787705,368.48295021)(59.96788208,368.47295013)
\curveto(59.85787727,368.46295023)(59.74287739,368.44795025)(59.62288208,368.42795013)
\curveto(59.58287755,368.41795028)(59.53787759,368.41795028)(59.48788208,368.42795013)
\curveto(59.44787768,368.42795027)(59.40787772,368.42295027)(59.36788208,368.41295013)
\curveto(59.3278778,368.40295029)(59.27287786,368.3979503)(59.20288208,368.39795013)
\curveto(59.132878,368.3979503)(59.08287805,368.40295029)(59.05288208,368.41295013)
\curveto(59.00287813,368.43295026)(58.95787817,368.43795026)(58.91788208,368.42795013)
\curveto(58.87787825,368.41795028)(58.84287829,368.41795028)(58.81288208,368.42795013)
\lineto(58.72288208,368.42795013)
\curveto(58.66287847,368.44795025)(58.59787853,368.46295023)(58.52788208,368.47295013)
\curveto(58.46787866,368.47295022)(58.40287873,368.47795022)(58.33288208,368.48795013)
\curveto(58.16287897,368.53795016)(58.00287913,368.58795011)(57.85288208,368.63795013)
\curveto(57.70287943,368.68795001)(57.55787957,368.75294994)(57.41788208,368.83295013)
\curveto(57.36787976,368.87294982)(57.31287982,368.90294979)(57.25288208,368.92295013)
\curveto(57.20287993,368.95294974)(57.15287998,368.98794971)(57.10288208,369.02795013)
\curveto(56.86288027,369.20794949)(56.66288047,369.42794927)(56.50288208,369.68795013)
\curveto(56.34288079,369.94794875)(56.20288093,370.23294846)(56.08288208,370.54295013)
\curveto(56.02288111,370.68294801)(55.97788115,370.82294787)(55.94788208,370.96295013)
\curveto(55.91788121,371.11294758)(55.88288125,371.26794743)(55.84288208,371.42795013)
\curveto(55.82288131,371.53794716)(55.80788132,371.64794705)(55.79788208,371.75795013)
\curveto(55.78788134,371.86794683)(55.77288136,371.97794672)(55.75288208,372.08795013)
\curveto(55.74288139,372.12794657)(55.73788139,372.16794653)(55.73788208,372.20795013)
\curveto(55.74788138,372.24794645)(55.74788138,372.28794641)(55.73788208,372.32795013)
\curveto(55.7278814,372.37794632)(55.72288141,372.42794627)(55.72288208,372.47795013)
\lineto(55.72288208,372.64295013)
\curveto(55.70288143,372.692946)(55.69788143,372.74294595)(55.70788208,372.79295013)
\curveto(55.71788141,372.85294584)(55.71788141,372.90794579)(55.70788208,372.95795013)
\curveto(55.69788143,372.9979457)(55.69788143,373.04294565)(55.70788208,373.09295013)
\curveto(55.71788141,373.14294555)(55.71288142,373.1929455)(55.69288208,373.24295013)
\curveto(55.67288146,373.31294538)(55.66788146,373.38794531)(55.67788208,373.46795013)
\curveto(55.68788144,373.55794514)(55.69288144,373.64294505)(55.69288208,373.72295013)
\curveto(55.69288144,373.81294488)(55.68788144,373.91294478)(55.67788208,374.02295013)
\curveto(55.66788146,374.14294455)(55.67288146,374.24294445)(55.69288208,374.32295013)
\lineto(55.69288208,374.60795013)
\lineto(55.73788208,375.23795013)
\curveto(55.74788138,375.33794336)(55.75788137,375.43294326)(55.76788208,375.52295013)
\lineto(55.79788208,375.82295013)
\curveto(55.81788131,375.87294282)(55.82288131,375.92294277)(55.81288208,375.97295013)
\curveto(55.81288132,376.03294266)(55.82288131,376.08794261)(55.84288208,376.13795013)
\curveto(55.89288124,376.30794239)(55.9328812,376.47294222)(55.96288208,376.63295013)
\curveto(55.99288114,376.80294189)(56.04288109,376.96294173)(56.11288208,377.11295013)
\curveto(56.30288083,377.57294112)(56.52288061,377.94794075)(56.77288208,378.23795013)
\curveto(57.0328801,378.52794017)(57.39287974,378.77293992)(57.85288208,378.97295013)
\curveto(57.98287915,379.02293967)(58.11287902,379.05793964)(58.24288208,379.07795013)
\curveto(58.38287875,379.0979396)(58.52287861,379.12293957)(58.66288208,379.15295013)
\curveto(58.7328784,379.16293953)(58.79787833,379.16793953)(58.85788208,379.16795013)
\curveto(58.91787821,379.16793953)(58.98287815,379.17293952)(59.05288208,379.18295013)
\curveto(59.88287725,379.20293949)(60.55287658,379.05293964)(61.06288208,378.73295013)
\curveto(61.57287556,378.42294027)(61.95287518,377.98294071)(62.20288208,377.41295013)
\curveto(62.25287488,377.2929414)(62.29787483,377.16794153)(62.33788208,377.03795013)
\curveto(62.37787475,376.90794179)(62.42287471,376.77294192)(62.47288208,376.63295013)
\curveto(62.49287464,376.55294214)(62.50787462,376.46794223)(62.51788208,376.37795013)
\lineto(62.57788208,376.13795013)
\curveto(62.60787452,376.02794267)(62.62287451,375.91794278)(62.62288208,375.80795013)
\curveto(62.6328745,375.697943)(62.64787448,375.58794311)(62.66788208,375.47795013)
\curveto(62.68787444,375.42794327)(62.69287444,375.38294331)(62.68288208,375.34295013)
\curveto(62.68287445,375.30294339)(62.68787444,375.26294343)(62.69788208,375.22295013)
\curveto(62.70787442,375.17294352)(62.70787442,375.11794358)(62.69788208,375.05795013)
\curveto(62.69787443,375.00794369)(62.70287443,374.95794374)(62.71288208,374.90795013)
\lineto(62.71288208,374.77295013)
\curveto(62.7328744,374.71294398)(62.7328744,374.64294405)(62.71288208,374.56295013)
\curveto(62.70287443,374.4929442)(62.70787442,374.42794427)(62.72788208,374.36795013)
\curveto(62.73787439,374.33794436)(62.74287439,374.2979444)(62.74288208,374.24795013)
\lineto(62.74288208,374.12795013)
\lineto(62.74288208,373.66295013)
\moveto(61.19788208,371.33795013)
\curveto(61.29787583,371.65794704)(61.35787577,372.02294667)(61.37788208,372.43295013)
\curveto(61.39787573,372.84294585)(61.40787572,373.25294544)(61.40788208,373.66295013)
\curveto(61.40787572,374.0929446)(61.39787573,374.51294418)(61.37788208,374.92295013)
\curveto(61.35787577,375.33294336)(61.31287582,375.71794298)(61.24288208,376.07795013)
\curveto(61.17287596,376.43794226)(61.06287607,376.75794194)(60.91288208,377.03795013)
\curveto(60.77287636,377.32794137)(60.57787655,377.56294113)(60.32788208,377.74295013)
\curveto(60.16787696,377.85294084)(59.98787714,377.93294076)(59.78788208,377.98295013)
\curveto(59.58787754,378.04294065)(59.34287779,378.07294062)(59.05288208,378.07295013)
\curveto(59.0328781,378.05294064)(58.99787813,378.04294065)(58.94788208,378.04295013)
\curveto(58.89787823,378.05294064)(58.85787827,378.05294064)(58.82788208,378.04295013)
\curveto(58.74787838,378.02294067)(58.67287846,378.00294069)(58.60288208,377.98295013)
\curveto(58.54287859,377.97294072)(58.47787865,377.95294074)(58.40788208,377.92295013)
\curveto(58.13787899,377.80294089)(57.91787921,377.63294106)(57.74788208,377.41295013)
\curveto(57.58787954,377.20294149)(57.45287968,376.95794174)(57.34288208,376.67795013)
\curveto(57.29287984,376.56794213)(57.25287988,376.44794225)(57.22288208,376.31795013)
\curveto(57.20287993,376.1979425)(57.17787995,376.07294262)(57.14788208,375.94295013)
\curveto(57.12788,375.8929428)(57.11788001,375.83794286)(57.11788208,375.77795013)
\curveto(57.11788001,375.72794297)(57.11288002,375.67794302)(57.10288208,375.62795013)
\curveto(57.09288004,375.53794316)(57.08288005,375.44294325)(57.07288208,375.34295013)
\curveto(57.06288007,375.25294344)(57.05288008,375.15794354)(57.04288208,375.05795013)
\curveto(57.04288009,374.97794372)(57.03788009,374.8929438)(57.02788208,374.80295013)
\lineto(57.02788208,374.56295013)
\lineto(57.02788208,374.38295013)
\curveto(57.01788011,374.35294434)(57.01288012,374.31794438)(57.01288208,374.27795013)
\lineto(57.01288208,374.14295013)
\lineto(57.01288208,373.69295013)
\curveto(57.01288012,373.61294508)(57.00788012,373.52794517)(56.99788208,373.43795013)
\curveto(56.99788013,373.35794534)(57.00788012,373.28294541)(57.02788208,373.21295013)
\lineto(57.02788208,372.94295013)
\curveto(57.0278801,372.92294577)(57.02288011,372.8929458)(57.01288208,372.85295013)
\curveto(57.01288012,372.82294587)(57.01788011,372.7979459)(57.02788208,372.77795013)
\curveto(57.03788009,372.67794602)(57.04288009,372.57794612)(57.04288208,372.47795013)
\curveto(57.05288008,372.38794631)(57.06288007,372.28794641)(57.07288208,372.17795013)
\curveto(57.10288003,372.05794664)(57.11788001,371.93294676)(57.11788208,371.80295013)
\curveto(57.12788,371.68294701)(57.15287998,371.56794713)(57.19288208,371.45795013)
\curveto(57.27287986,371.15794754)(57.35787977,370.8929478)(57.44788208,370.66295013)
\curveto(57.54787958,370.43294826)(57.69287944,370.21794848)(57.88288208,370.01795013)
\curveto(58.09287904,369.81794888)(58.35787877,369.66794903)(58.67788208,369.56795013)
\curveto(58.71787841,369.54794915)(58.75287838,369.53794916)(58.78288208,369.53795013)
\curveto(58.82287831,369.54794915)(58.86787826,369.54294915)(58.91788208,369.52295013)
\curveto(58.95787817,369.51294918)(59.0278781,369.50294919)(59.12788208,369.49295013)
\curveto(59.23787789,369.48294921)(59.32287781,369.48794921)(59.38288208,369.50795013)
\curveto(59.45287768,369.52794917)(59.52287761,369.53794916)(59.59288208,369.53795013)
\curveto(59.66287747,369.54794915)(59.7278774,369.56294913)(59.78788208,369.58295013)
\curveto(59.98787714,369.64294905)(60.16787696,369.72794897)(60.32788208,369.83795013)
\curveto(60.35787677,369.85794884)(60.38287675,369.87794882)(60.40288208,369.89795013)
\lineto(60.46288208,369.95795013)
\curveto(60.50287663,369.97794872)(60.55287658,370.01794868)(60.61288208,370.07795013)
\curveto(60.71287642,370.21794848)(60.79787633,370.34794835)(60.86788208,370.46795013)
\curveto(60.93787619,370.58794811)(61.00787612,370.73294796)(61.07788208,370.90295013)
\curveto(61.10787602,370.97294772)(61.127876,371.04294765)(61.13788208,371.11295013)
\curveto(61.15787597,371.18294751)(61.17787595,371.25794744)(61.19788208,371.33795013)
}
}
{
\newrgbcolor{curcolor}{0 0 0}
\pscustom[linestyle=none,fillstyle=solid,fillcolor=curcolor]
{
\newpath
\moveto(71.09249146,373.66295013)
\lineto(71.09249146,373.40795013)
\curveto(71.10248375,373.32794537)(71.09748376,373.25294544)(71.07749146,373.18295013)
\lineto(71.07749146,372.94295013)
\lineto(71.07749146,372.77795013)
\curveto(71.0574838,372.67794602)(71.04748381,372.57294612)(71.04749146,372.46295013)
\curveto(71.04748381,372.36294633)(71.03748382,372.26294643)(71.01749146,372.16295013)
\lineto(71.01749146,372.01295013)
\curveto(70.98748387,371.87294682)(70.96748389,371.73294696)(70.95749146,371.59295013)
\curveto(70.94748391,371.46294723)(70.92248393,371.33294736)(70.88249146,371.20295013)
\curveto(70.86248399,371.12294757)(70.84248401,371.03794766)(70.82249146,370.94795013)
\lineto(70.76249146,370.70795013)
\lineto(70.64249146,370.40795013)
\curveto(70.61248424,370.31794838)(70.57748428,370.22794847)(70.53749146,370.13795013)
\curveto(70.43748442,369.91794878)(70.30248455,369.70294899)(70.13249146,369.49295013)
\curveto(69.97248488,369.28294941)(69.79748506,369.11294958)(69.60749146,368.98295013)
\curveto(69.5574853,368.94294975)(69.49748536,368.90294979)(69.42749146,368.86295013)
\curveto(69.36748549,368.83294986)(69.30748555,368.7979499)(69.24749146,368.75795013)
\curveto(69.16748569,368.70794999)(69.07248578,368.66795003)(68.96249146,368.63795013)
\curveto(68.852486,368.60795009)(68.74748611,368.57795012)(68.64749146,368.54795013)
\curveto(68.53748632,368.50795019)(68.42748643,368.48295021)(68.31749146,368.47295013)
\curveto(68.20748665,368.46295023)(68.09248676,368.44795025)(67.97249146,368.42795013)
\curveto(67.93248692,368.41795028)(67.88748697,368.41795028)(67.83749146,368.42795013)
\curveto(67.79748706,368.42795027)(67.7574871,368.42295027)(67.71749146,368.41295013)
\curveto(67.67748718,368.40295029)(67.62248723,368.3979503)(67.55249146,368.39795013)
\curveto(67.48248737,368.3979503)(67.43248742,368.40295029)(67.40249146,368.41295013)
\curveto(67.3524875,368.43295026)(67.30748755,368.43795026)(67.26749146,368.42795013)
\curveto(67.22748763,368.41795028)(67.19248766,368.41795028)(67.16249146,368.42795013)
\lineto(67.07249146,368.42795013)
\curveto(67.01248784,368.44795025)(66.94748791,368.46295023)(66.87749146,368.47295013)
\curveto(66.81748804,368.47295022)(66.7524881,368.47795022)(66.68249146,368.48795013)
\curveto(66.51248834,368.53795016)(66.3524885,368.58795011)(66.20249146,368.63795013)
\curveto(66.0524888,368.68795001)(65.90748895,368.75294994)(65.76749146,368.83295013)
\curveto(65.71748914,368.87294982)(65.66248919,368.90294979)(65.60249146,368.92295013)
\curveto(65.5524893,368.95294974)(65.50248935,368.98794971)(65.45249146,369.02795013)
\curveto(65.21248964,369.20794949)(65.01248984,369.42794927)(64.85249146,369.68795013)
\curveto(64.69249016,369.94794875)(64.5524903,370.23294846)(64.43249146,370.54295013)
\curveto(64.37249048,370.68294801)(64.32749053,370.82294787)(64.29749146,370.96295013)
\curveto(64.26749059,371.11294758)(64.23249062,371.26794743)(64.19249146,371.42795013)
\curveto(64.17249068,371.53794716)(64.1574907,371.64794705)(64.14749146,371.75795013)
\curveto(64.13749072,371.86794683)(64.12249073,371.97794672)(64.10249146,372.08795013)
\curveto(64.09249076,372.12794657)(64.08749077,372.16794653)(64.08749146,372.20795013)
\curveto(64.09749076,372.24794645)(64.09749076,372.28794641)(64.08749146,372.32795013)
\curveto(64.07749078,372.37794632)(64.07249078,372.42794627)(64.07249146,372.47795013)
\lineto(64.07249146,372.64295013)
\curveto(64.0524908,372.692946)(64.04749081,372.74294595)(64.05749146,372.79295013)
\curveto(64.06749079,372.85294584)(64.06749079,372.90794579)(64.05749146,372.95795013)
\curveto(64.04749081,372.9979457)(64.04749081,373.04294565)(64.05749146,373.09295013)
\curveto(64.06749079,373.14294555)(64.06249079,373.1929455)(64.04249146,373.24295013)
\curveto(64.02249083,373.31294538)(64.01749084,373.38794531)(64.02749146,373.46795013)
\curveto(64.03749082,373.55794514)(64.04249081,373.64294505)(64.04249146,373.72295013)
\curveto(64.04249081,373.81294488)(64.03749082,373.91294478)(64.02749146,374.02295013)
\curveto(64.01749084,374.14294455)(64.02249083,374.24294445)(64.04249146,374.32295013)
\lineto(64.04249146,374.60795013)
\lineto(64.08749146,375.23795013)
\curveto(64.09749076,375.33794336)(64.10749075,375.43294326)(64.11749146,375.52295013)
\lineto(64.14749146,375.82295013)
\curveto(64.16749069,375.87294282)(64.17249068,375.92294277)(64.16249146,375.97295013)
\curveto(64.16249069,376.03294266)(64.17249068,376.08794261)(64.19249146,376.13795013)
\curveto(64.24249061,376.30794239)(64.28249057,376.47294222)(64.31249146,376.63295013)
\curveto(64.34249051,376.80294189)(64.39249046,376.96294173)(64.46249146,377.11295013)
\curveto(64.6524902,377.57294112)(64.87248998,377.94794075)(65.12249146,378.23795013)
\curveto(65.38248947,378.52794017)(65.74248911,378.77293992)(66.20249146,378.97295013)
\curveto(66.33248852,379.02293967)(66.46248839,379.05793964)(66.59249146,379.07795013)
\curveto(66.73248812,379.0979396)(66.87248798,379.12293957)(67.01249146,379.15295013)
\curveto(67.08248777,379.16293953)(67.14748771,379.16793953)(67.20749146,379.16795013)
\curveto(67.26748759,379.16793953)(67.33248752,379.17293952)(67.40249146,379.18295013)
\curveto(68.23248662,379.20293949)(68.90248595,379.05293964)(69.41249146,378.73295013)
\curveto(69.92248493,378.42294027)(70.30248455,377.98294071)(70.55249146,377.41295013)
\curveto(70.60248425,377.2929414)(70.64748421,377.16794153)(70.68749146,377.03795013)
\curveto(70.72748413,376.90794179)(70.77248408,376.77294192)(70.82249146,376.63295013)
\curveto(70.84248401,376.55294214)(70.857484,376.46794223)(70.86749146,376.37795013)
\lineto(70.92749146,376.13795013)
\curveto(70.9574839,376.02794267)(70.97248388,375.91794278)(70.97249146,375.80795013)
\curveto(70.98248387,375.697943)(70.99748386,375.58794311)(71.01749146,375.47795013)
\curveto(71.03748382,375.42794327)(71.04248381,375.38294331)(71.03249146,375.34295013)
\curveto(71.03248382,375.30294339)(71.03748382,375.26294343)(71.04749146,375.22295013)
\curveto(71.0574838,375.17294352)(71.0574838,375.11794358)(71.04749146,375.05795013)
\curveto(71.04748381,375.00794369)(71.0524838,374.95794374)(71.06249146,374.90795013)
\lineto(71.06249146,374.77295013)
\curveto(71.08248377,374.71294398)(71.08248377,374.64294405)(71.06249146,374.56295013)
\curveto(71.0524838,374.4929442)(71.0574838,374.42794427)(71.07749146,374.36795013)
\curveto(71.08748377,374.33794436)(71.09248376,374.2979444)(71.09249146,374.24795013)
\lineto(71.09249146,374.12795013)
\lineto(71.09249146,373.66295013)
\moveto(69.54749146,371.33795013)
\curveto(69.64748521,371.65794704)(69.70748515,372.02294667)(69.72749146,372.43295013)
\curveto(69.74748511,372.84294585)(69.7574851,373.25294544)(69.75749146,373.66295013)
\curveto(69.7574851,374.0929446)(69.74748511,374.51294418)(69.72749146,374.92295013)
\curveto(69.70748515,375.33294336)(69.66248519,375.71794298)(69.59249146,376.07795013)
\curveto(69.52248533,376.43794226)(69.41248544,376.75794194)(69.26249146,377.03795013)
\curveto(69.12248573,377.32794137)(68.92748593,377.56294113)(68.67749146,377.74295013)
\curveto(68.51748634,377.85294084)(68.33748652,377.93294076)(68.13749146,377.98295013)
\curveto(67.93748692,378.04294065)(67.69248716,378.07294062)(67.40249146,378.07295013)
\curveto(67.38248747,378.05294064)(67.34748751,378.04294065)(67.29749146,378.04295013)
\curveto(67.24748761,378.05294064)(67.20748765,378.05294064)(67.17749146,378.04295013)
\curveto(67.09748776,378.02294067)(67.02248783,378.00294069)(66.95249146,377.98295013)
\curveto(66.89248796,377.97294072)(66.82748803,377.95294074)(66.75749146,377.92295013)
\curveto(66.48748837,377.80294089)(66.26748859,377.63294106)(66.09749146,377.41295013)
\curveto(65.93748892,377.20294149)(65.80248905,376.95794174)(65.69249146,376.67795013)
\curveto(65.64248921,376.56794213)(65.60248925,376.44794225)(65.57249146,376.31795013)
\curveto(65.5524893,376.1979425)(65.52748933,376.07294262)(65.49749146,375.94295013)
\curveto(65.47748938,375.8929428)(65.46748939,375.83794286)(65.46749146,375.77795013)
\curveto(65.46748939,375.72794297)(65.46248939,375.67794302)(65.45249146,375.62795013)
\curveto(65.44248941,375.53794316)(65.43248942,375.44294325)(65.42249146,375.34295013)
\curveto(65.41248944,375.25294344)(65.40248945,375.15794354)(65.39249146,375.05795013)
\curveto(65.39248946,374.97794372)(65.38748947,374.8929438)(65.37749146,374.80295013)
\lineto(65.37749146,374.56295013)
\lineto(65.37749146,374.38295013)
\curveto(65.36748949,374.35294434)(65.36248949,374.31794438)(65.36249146,374.27795013)
\lineto(65.36249146,374.14295013)
\lineto(65.36249146,373.69295013)
\curveto(65.36248949,373.61294508)(65.3574895,373.52794517)(65.34749146,373.43795013)
\curveto(65.34748951,373.35794534)(65.3574895,373.28294541)(65.37749146,373.21295013)
\lineto(65.37749146,372.94295013)
\curveto(65.37748948,372.92294577)(65.37248948,372.8929458)(65.36249146,372.85295013)
\curveto(65.36248949,372.82294587)(65.36748949,372.7979459)(65.37749146,372.77795013)
\curveto(65.38748947,372.67794602)(65.39248946,372.57794612)(65.39249146,372.47795013)
\curveto(65.40248945,372.38794631)(65.41248944,372.28794641)(65.42249146,372.17795013)
\curveto(65.4524894,372.05794664)(65.46748939,371.93294676)(65.46749146,371.80295013)
\curveto(65.47748938,371.68294701)(65.50248935,371.56794713)(65.54249146,371.45795013)
\curveto(65.62248923,371.15794754)(65.70748915,370.8929478)(65.79749146,370.66295013)
\curveto(65.89748896,370.43294826)(66.04248881,370.21794848)(66.23249146,370.01795013)
\curveto(66.44248841,369.81794888)(66.70748815,369.66794903)(67.02749146,369.56795013)
\curveto(67.06748779,369.54794915)(67.10248775,369.53794916)(67.13249146,369.53795013)
\curveto(67.17248768,369.54794915)(67.21748764,369.54294915)(67.26749146,369.52295013)
\curveto(67.30748755,369.51294918)(67.37748748,369.50294919)(67.47749146,369.49295013)
\curveto(67.58748727,369.48294921)(67.67248718,369.48794921)(67.73249146,369.50795013)
\curveto(67.80248705,369.52794917)(67.87248698,369.53794916)(67.94249146,369.53795013)
\curveto(68.01248684,369.54794915)(68.07748678,369.56294913)(68.13749146,369.58295013)
\curveto(68.33748652,369.64294905)(68.51748634,369.72794897)(68.67749146,369.83795013)
\curveto(68.70748615,369.85794884)(68.73248612,369.87794882)(68.75249146,369.89795013)
\lineto(68.81249146,369.95795013)
\curveto(68.852486,369.97794872)(68.90248595,370.01794868)(68.96249146,370.07795013)
\curveto(69.06248579,370.21794848)(69.14748571,370.34794835)(69.21749146,370.46795013)
\curveto(69.28748557,370.58794811)(69.3574855,370.73294796)(69.42749146,370.90295013)
\curveto(69.4574854,370.97294772)(69.47748538,371.04294765)(69.48749146,371.11295013)
\curveto(69.50748535,371.18294751)(69.52748533,371.25794744)(69.54749146,371.33795013)
}
}
{
\newrgbcolor{curcolor}{0 0 0}
\pscustom[linestyle=none,fillstyle=solid,fillcolor=curcolor]
{
\newpath
\moveto(79.44210083,373.66295013)
\lineto(79.44210083,373.40795013)
\curveto(79.45209313,373.32794537)(79.44709313,373.25294544)(79.42710083,373.18295013)
\lineto(79.42710083,372.94295013)
\lineto(79.42710083,372.77795013)
\curveto(79.40709317,372.67794602)(79.39709318,372.57294612)(79.39710083,372.46295013)
\curveto(79.39709318,372.36294633)(79.38709319,372.26294643)(79.36710083,372.16295013)
\lineto(79.36710083,372.01295013)
\curveto(79.33709324,371.87294682)(79.31709326,371.73294696)(79.30710083,371.59295013)
\curveto(79.29709328,371.46294723)(79.27209331,371.33294736)(79.23210083,371.20295013)
\curveto(79.21209337,371.12294757)(79.19209339,371.03794766)(79.17210083,370.94795013)
\lineto(79.11210083,370.70795013)
\lineto(78.99210083,370.40795013)
\curveto(78.96209362,370.31794838)(78.92709365,370.22794847)(78.88710083,370.13795013)
\curveto(78.78709379,369.91794878)(78.65209393,369.70294899)(78.48210083,369.49295013)
\curveto(78.32209426,369.28294941)(78.14709443,369.11294958)(77.95710083,368.98295013)
\curveto(77.90709467,368.94294975)(77.84709473,368.90294979)(77.77710083,368.86295013)
\curveto(77.71709486,368.83294986)(77.65709492,368.7979499)(77.59710083,368.75795013)
\curveto(77.51709506,368.70794999)(77.42209516,368.66795003)(77.31210083,368.63795013)
\curveto(77.20209538,368.60795009)(77.09709548,368.57795012)(76.99710083,368.54795013)
\curveto(76.88709569,368.50795019)(76.7770958,368.48295021)(76.66710083,368.47295013)
\curveto(76.55709602,368.46295023)(76.44209614,368.44795025)(76.32210083,368.42795013)
\curveto(76.2820963,368.41795028)(76.23709634,368.41795028)(76.18710083,368.42795013)
\curveto(76.14709643,368.42795027)(76.10709647,368.42295027)(76.06710083,368.41295013)
\curveto(76.02709655,368.40295029)(75.97209661,368.3979503)(75.90210083,368.39795013)
\curveto(75.83209675,368.3979503)(75.7820968,368.40295029)(75.75210083,368.41295013)
\curveto(75.70209688,368.43295026)(75.65709692,368.43795026)(75.61710083,368.42795013)
\curveto(75.577097,368.41795028)(75.54209704,368.41795028)(75.51210083,368.42795013)
\lineto(75.42210083,368.42795013)
\curveto(75.36209722,368.44795025)(75.29709728,368.46295023)(75.22710083,368.47295013)
\curveto(75.16709741,368.47295022)(75.10209748,368.47795022)(75.03210083,368.48795013)
\curveto(74.86209772,368.53795016)(74.70209788,368.58795011)(74.55210083,368.63795013)
\curveto(74.40209818,368.68795001)(74.25709832,368.75294994)(74.11710083,368.83295013)
\curveto(74.06709851,368.87294982)(74.01209857,368.90294979)(73.95210083,368.92295013)
\curveto(73.90209868,368.95294974)(73.85209873,368.98794971)(73.80210083,369.02795013)
\curveto(73.56209902,369.20794949)(73.36209922,369.42794927)(73.20210083,369.68795013)
\curveto(73.04209954,369.94794875)(72.90209968,370.23294846)(72.78210083,370.54295013)
\curveto(72.72209986,370.68294801)(72.6770999,370.82294787)(72.64710083,370.96295013)
\curveto(72.61709996,371.11294758)(72.5821,371.26794743)(72.54210083,371.42795013)
\curveto(72.52210006,371.53794716)(72.50710007,371.64794705)(72.49710083,371.75795013)
\curveto(72.48710009,371.86794683)(72.47210011,371.97794672)(72.45210083,372.08795013)
\curveto(72.44210014,372.12794657)(72.43710014,372.16794653)(72.43710083,372.20795013)
\curveto(72.44710013,372.24794645)(72.44710013,372.28794641)(72.43710083,372.32795013)
\curveto(72.42710015,372.37794632)(72.42210016,372.42794627)(72.42210083,372.47795013)
\lineto(72.42210083,372.64295013)
\curveto(72.40210018,372.692946)(72.39710018,372.74294595)(72.40710083,372.79295013)
\curveto(72.41710016,372.85294584)(72.41710016,372.90794579)(72.40710083,372.95795013)
\curveto(72.39710018,372.9979457)(72.39710018,373.04294565)(72.40710083,373.09295013)
\curveto(72.41710016,373.14294555)(72.41210017,373.1929455)(72.39210083,373.24295013)
\curveto(72.37210021,373.31294538)(72.36710021,373.38794531)(72.37710083,373.46795013)
\curveto(72.38710019,373.55794514)(72.39210019,373.64294505)(72.39210083,373.72295013)
\curveto(72.39210019,373.81294488)(72.38710019,373.91294478)(72.37710083,374.02295013)
\curveto(72.36710021,374.14294455)(72.37210021,374.24294445)(72.39210083,374.32295013)
\lineto(72.39210083,374.60795013)
\lineto(72.43710083,375.23795013)
\curveto(72.44710013,375.33794336)(72.45710012,375.43294326)(72.46710083,375.52295013)
\lineto(72.49710083,375.82295013)
\curveto(72.51710006,375.87294282)(72.52210006,375.92294277)(72.51210083,375.97295013)
\curveto(72.51210007,376.03294266)(72.52210006,376.08794261)(72.54210083,376.13795013)
\curveto(72.59209999,376.30794239)(72.63209995,376.47294222)(72.66210083,376.63295013)
\curveto(72.69209989,376.80294189)(72.74209984,376.96294173)(72.81210083,377.11295013)
\curveto(73.00209958,377.57294112)(73.22209936,377.94794075)(73.47210083,378.23795013)
\curveto(73.73209885,378.52794017)(74.09209849,378.77293992)(74.55210083,378.97295013)
\curveto(74.6820979,379.02293967)(74.81209777,379.05793964)(74.94210083,379.07795013)
\curveto(75.0820975,379.0979396)(75.22209736,379.12293957)(75.36210083,379.15295013)
\curveto(75.43209715,379.16293953)(75.49709708,379.16793953)(75.55710083,379.16795013)
\curveto(75.61709696,379.16793953)(75.6820969,379.17293952)(75.75210083,379.18295013)
\curveto(76.582096,379.20293949)(77.25209533,379.05293964)(77.76210083,378.73295013)
\curveto(78.27209431,378.42294027)(78.65209393,377.98294071)(78.90210083,377.41295013)
\curveto(78.95209363,377.2929414)(78.99709358,377.16794153)(79.03710083,377.03795013)
\curveto(79.0770935,376.90794179)(79.12209346,376.77294192)(79.17210083,376.63295013)
\curveto(79.19209339,376.55294214)(79.20709337,376.46794223)(79.21710083,376.37795013)
\lineto(79.27710083,376.13795013)
\curveto(79.30709327,376.02794267)(79.32209326,375.91794278)(79.32210083,375.80795013)
\curveto(79.33209325,375.697943)(79.34709323,375.58794311)(79.36710083,375.47795013)
\curveto(79.38709319,375.42794327)(79.39209319,375.38294331)(79.38210083,375.34295013)
\curveto(79.3820932,375.30294339)(79.38709319,375.26294343)(79.39710083,375.22295013)
\curveto(79.40709317,375.17294352)(79.40709317,375.11794358)(79.39710083,375.05795013)
\curveto(79.39709318,375.00794369)(79.40209318,374.95794374)(79.41210083,374.90795013)
\lineto(79.41210083,374.77295013)
\curveto(79.43209315,374.71294398)(79.43209315,374.64294405)(79.41210083,374.56295013)
\curveto(79.40209318,374.4929442)(79.40709317,374.42794427)(79.42710083,374.36795013)
\curveto(79.43709314,374.33794436)(79.44209314,374.2979444)(79.44210083,374.24795013)
\lineto(79.44210083,374.12795013)
\lineto(79.44210083,373.66295013)
\moveto(77.89710083,371.33795013)
\curveto(77.99709458,371.65794704)(78.05709452,372.02294667)(78.07710083,372.43295013)
\curveto(78.09709448,372.84294585)(78.10709447,373.25294544)(78.10710083,373.66295013)
\curveto(78.10709447,374.0929446)(78.09709448,374.51294418)(78.07710083,374.92295013)
\curveto(78.05709452,375.33294336)(78.01209457,375.71794298)(77.94210083,376.07795013)
\curveto(77.87209471,376.43794226)(77.76209482,376.75794194)(77.61210083,377.03795013)
\curveto(77.47209511,377.32794137)(77.2770953,377.56294113)(77.02710083,377.74295013)
\curveto(76.86709571,377.85294084)(76.68709589,377.93294076)(76.48710083,377.98295013)
\curveto(76.28709629,378.04294065)(76.04209654,378.07294062)(75.75210083,378.07295013)
\curveto(75.73209685,378.05294064)(75.69709688,378.04294065)(75.64710083,378.04295013)
\curveto(75.59709698,378.05294064)(75.55709702,378.05294064)(75.52710083,378.04295013)
\curveto(75.44709713,378.02294067)(75.37209721,378.00294069)(75.30210083,377.98295013)
\curveto(75.24209734,377.97294072)(75.1770974,377.95294074)(75.10710083,377.92295013)
\curveto(74.83709774,377.80294089)(74.61709796,377.63294106)(74.44710083,377.41295013)
\curveto(74.28709829,377.20294149)(74.15209843,376.95794174)(74.04210083,376.67795013)
\curveto(73.99209859,376.56794213)(73.95209863,376.44794225)(73.92210083,376.31795013)
\curveto(73.90209868,376.1979425)(73.8770987,376.07294262)(73.84710083,375.94295013)
\curveto(73.82709875,375.8929428)(73.81709876,375.83794286)(73.81710083,375.77795013)
\curveto(73.81709876,375.72794297)(73.81209877,375.67794302)(73.80210083,375.62795013)
\curveto(73.79209879,375.53794316)(73.7820988,375.44294325)(73.77210083,375.34295013)
\curveto(73.76209882,375.25294344)(73.75209883,375.15794354)(73.74210083,375.05795013)
\curveto(73.74209884,374.97794372)(73.73709884,374.8929438)(73.72710083,374.80295013)
\lineto(73.72710083,374.56295013)
\lineto(73.72710083,374.38295013)
\curveto(73.71709886,374.35294434)(73.71209887,374.31794438)(73.71210083,374.27795013)
\lineto(73.71210083,374.14295013)
\lineto(73.71210083,373.69295013)
\curveto(73.71209887,373.61294508)(73.70709887,373.52794517)(73.69710083,373.43795013)
\curveto(73.69709888,373.35794534)(73.70709887,373.28294541)(73.72710083,373.21295013)
\lineto(73.72710083,372.94295013)
\curveto(73.72709885,372.92294577)(73.72209886,372.8929458)(73.71210083,372.85295013)
\curveto(73.71209887,372.82294587)(73.71709886,372.7979459)(73.72710083,372.77795013)
\curveto(73.73709884,372.67794602)(73.74209884,372.57794612)(73.74210083,372.47795013)
\curveto(73.75209883,372.38794631)(73.76209882,372.28794641)(73.77210083,372.17795013)
\curveto(73.80209878,372.05794664)(73.81709876,371.93294676)(73.81710083,371.80295013)
\curveto(73.82709875,371.68294701)(73.85209873,371.56794713)(73.89210083,371.45795013)
\curveto(73.97209861,371.15794754)(74.05709852,370.8929478)(74.14710083,370.66295013)
\curveto(74.24709833,370.43294826)(74.39209819,370.21794848)(74.58210083,370.01795013)
\curveto(74.79209779,369.81794888)(75.05709752,369.66794903)(75.37710083,369.56795013)
\curveto(75.41709716,369.54794915)(75.45209713,369.53794916)(75.48210083,369.53795013)
\curveto(75.52209706,369.54794915)(75.56709701,369.54294915)(75.61710083,369.52295013)
\curveto(75.65709692,369.51294918)(75.72709685,369.50294919)(75.82710083,369.49295013)
\curveto(75.93709664,369.48294921)(76.02209656,369.48794921)(76.08210083,369.50795013)
\curveto(76.15209643,369.52794917)(76.22209636,369.53794916)(76.29210083,369.53795013)
\curveto(76.36209622,369.54794915)(76.42709615,369.56294913)(76.48710083,369.58295013)
\curveto(76.68709589,369.64294905)(76.86709571,369.72794897)(77.02710083,369.83795013)
\curveto(77.05709552,369.85794884)(77.0820955,369.87794882)(77.10210083,369.89795013)
\lineto(77.16210083,369.95795013)
\curveto(77.20209538,369.97794872)(77.25209533,370.01794868)(77.31210083,370.07795013)
\curveto(77.41209517,370.21794848)(77.49709508,370.34794835)(77.56710083,370.46795013)
\curveto(77.63709494,370.58794811)(77.70709487,370.73294796)(77.77710083,370.90295013)
\curveto(77.80709477,370.97294772)(77.82709475,371.04294765)(77.83710083,371.11295013)
\curveto(77.85709472,371.18294751)(77.8770947,371.25794744)(77.89710083,371.33795013)
}
}
{
\newrgbcolor{curcolor}{0 0 0}
\pscustom[linestyle=none,fillstyle=solid,fillcolor=curcolor]
{
\newpath
\moveto(771.90647461,372.23397308)
\curveto(772.88646811,372.25396213)(773.70646729,372.09396229)(774.36647461,371.75397308)
\curveto(775.03646596,371.42396296)(775.55646544,370.96396342)(775.92647461,370.37397308)
\curveto(776.02646497,370.21396417)(776.10646489,370.05896432)(776.16647461,369.90897308)
\curveto(776.23646476,369.76896461)(776.30146469,369.59896478)(776.36147461,369.39897308)
\curveto(776.38146461,369.34896503)(776.40146459,369.2789651)(776.42147461,369.18897308)
\curveto(776.44146455,369.10896527)(776.43646456,369.03396535)(776.40647461,368.96397308)
\curveto(776.38646461,368.90396548)(776.34646465,368.86396552)(776.28647461,368.84397308)
\curveto(776.23646476,368.83396555)(776.18146481,368.81896556)(776.12147461,368.79897308)
\lineto(775.97147461,368.79897308)
\curveto(775.94146505,368.78896559)(775.90146509,368.7839656)(775.85147461,368.78397308)
\lineto(775.73147461,368.78397308)
\curveto(775.5914654,368.7839656)(775.46146553,368.78896559)(775.34147461,368.79897308)
\curveto(775.23146576,368.81896556)(775.15146584,368.86896551)(775.10147461,368.94897308)
\curveto(775.03146596,369.04896533)(774.97646602,369.16396522)(774.93647461,369.29397308)
\curveto(774.8964661,369.42396496)(774.84146615,369.54396484)(774.77147461,369.65397308)
\curveto(774.64146635,369.87396451)(774.4914665,370.06396432)(774.32147461,370.22397308)
\curveto(774.16146683,370.383964)(773.97146702,370.53396385)(773.75147461,370.67397308)
\curveto(773.63146736,370.75396363)(773.4964675,370.81396357)(773.34647461,370.85397308)
\curveto(773.20646779,370.89396349)(773.06146793,370.93396345)(772.91147461,370.97397308)
\curveto(772.80146819,371.00396338)(772.67646832,371.02396336)(772.53647461,371.03397308)
\curveto(772.3964686,371.05396333)(772.24646875,371.06396332)(772.08647461,371.06397308)
\curveto(771.93646906,371.06396332)(771.78646921,371.05396333)(771.63647461,371.03397308)
\curveto(771.4964695,371.02396336)(771.37646962,371.00396338)(771.27647461,370.97397308)
\curveto(771.17646982,370.95396343)(771.08146991,370.93396345)(770.99147461,370.91397308)
\curveto(770.90147009,370.89396349)(770.81147018,370.86396352)(770.72147461,370.82397308)
\curveto(769.88147111,370.47396391)(769.27647172,369.87396451)(768.90647461,369.02397308)
\curveto(768.83647216,368.8839655)(768.77647222,368.73396565)(768.72647461,368.57397308)
\curveto(768.68647231,368.42396596)(768.64147235,368.26896611)(768.59147461,368.10897308)
\curveto(768.57147242,368.04896633)(768.56147243,367.9839664)(768.56147461,367.91397308)
\curveto(768.56147243,367.85396653)(768.55147244,367.79396659)(768.53147461,367.73397308)
\curveto(768.52147247,367.69396669)(768.51647248,367.65896672)(768.51647461,367.62897308)
\curveto(768.51647248,367.59896678)(768.51147248,367.56396682)(768.50147461,367.52397308)
\curveto(768.48147251,367.41396697)(768.46647253,367.29896708)(768.45647461,367.17897308)
\lineto(768.45647461,366.83397308)
\curveto(768.45647254,366.76396762)(768.45147254,366.68896769)(768.44147461,366.60897308)
\curveto(768.44147255,366.53896784)(768.44647255,366.47396791)(768.45647461,366.41397308)
\lineto(768.45647461,366.26397308)
\curveto(768.47647252,366.19396819)(768.48147251,366.12396826)(768.47147461,366.05397308)
\curveto(768.47147252,365.9839684)(768.48147251,365.91396847)(768.50147461,365.84397308)
\curveto(768.52147247,365.7839686)(768.52647247,365.72396866)(768.51647461,365.66397308)
\curveto(768.51647248,365.60396878)(768.52647247,365.54896883)(768.54647461,365.49897308)
\curveto(768.57647242,365.36896901)(768.60147239,365.23896914)(768.62147461,365.10897308)
\curveto(768.65147234,364.98896939)(768.68647231,364.86896951)(768.72647461,364.74897308)
\curveto(768.8964721,364.24897013)(769.11647188,363.81897056)(769.38647461,363.45897308)
\curveto(769.65647134,363.10897127)(770.01147098,362.81897156)(770.45147461,362.58897308)
\curveto(770.5914704,362.51897186)(770.73147026,362.46397192)(770.87147461,362.42397308)
\curveto(771.02146997,362.383972)(771.18146981,362.33897204)(771.35147461,362.28897308)
\curveto(771.42146957,362.26897211)(771.48646951,362.25897212)(771.54647461,362.25897308)
\curveto(771.60646939,362.26897211)(771.67646932,362.26397212)(771.75647461,362.24397308)
\curveto(771.80646919,362.23397215)(771.8964691,362.22397216)(772.02647461,362.21397308)
\curveto(772.15646884,362.21397217)(772.25146874,362.22397216)(772.31147461,362.24397308)
\lineto(772.41647461,362.24397308)
\curveto(772.45646854,362.25397213)(772.4964685,362.25397213)(772.53647461,362.24397308)
\curveto(772.57646842,362.24397214)(772.61646838,362.25397213)(772.65647461,362.27397308)
\curveto(772.75646824,362.29397209)(772.85146814,362.30897207)(772.94147461,362.31897308)
\curveto(773.04146795,362.33897204)(773.13646786,362.36897201)(773.22647461,362.40897308)
\curveto(774.00646699,362.72897165)(774.55646644,363.25397113)(774.87647461,363.98397308)
\curveto(774.95646604,364.16397022)(775.03146596,364.37897)(775.10147461,364.62897308)
\curveto(775.12146587,364.71896966)(775.13646586,364.80896957)(775.14647461,364.89897308)
\curveto(775.16646583,364.99896938)(775.20146579,365.08896929)(775.25147461,365.16897308)
\curveto(775.30146569,365.24896913)(775.38146561,365.29396909)(775.49147461,365.30397308)
\curveto(775.60146539,365.31396907)(775.72146527,365.31896906)(775.85147461,365.31897308)
\lineto(776.00147461,365.31897308)
\curveto(776.05146494,365.31896906)(776.0964649,365.31396907)(776.13647461,365.30397308)
\lineto(776.24147461,365.30397308)
\lineto(776.33147461,365.27397308)
\curveto(776.37146462,365.27396911)(776.40146459,365.26396912)(776.42147461,365.24397308)
\curveto(776.4914645,365.20396918)(776.53146446,365.12896925)(776.54147461,365.01897308)
\curveto(776.55146444,364.91896946)(776.54146445,364.81896956)(776.51147461,364.71897308)
\curveto(776.45146454,364.48896989)(776.3964646,364.26897011)(776.34647461,364.05897308)
\curveto(776.2964647,363.84897053)(776.22146477,363.64897073)(776.12147461,363.45897308)
\curveto(776.04146495,363.32897105)(775.96646503,363.20397118)(775.89647461,363.08397308)
\curveto(775.83646516,362.96397142)(775.76646523,362.84397154)(775.68647461,362.72397308)
\curveto(775.50646549,362.46397192)(775.28146571,362.22397216)(775.01147461,362.00397308)
\curveto(774.75146624,361.79397259)(774.46646653,361.61897276)(774.15647461,361.47897308)
\curveto(774.04646695,361.42897295)(773.93646706,361.38897299)(773.82647461,361.35897308)
\curveto(773.72646727,361.32897305)(773.62146737,361.29397309)(773.51147461,361.25397308)
\curveto(773.40146759,361.21397317)(773.28646771,361.18897319)(773.16647461,361.17897308)
\curveto(773.05646794,361.15897322)(772.94146805,361.13897324)(772.82147461,361.11897308)
\curveto(772.77146822,361.09897328)(772.72646827,361.09397329)(772.68647461,361.10397308)
\curveto(772.64646835,361.10397328)(772.60646839,361.09897328)(772.56647461,361.08897308)
\curveto(772.50646849,361.0789733)(772.44646855,361.07397331)(772.38647461,361.07397308)
\curveto(772.32646867,361.07397331)(772.26146873,361.06897331)(772.19147461,361.05897308)
\curveto(772.16146883,361.04897333)(772.0914689,361.04897333)(771.98147461,361.05897308)
\curveto(771.88146911,361.05897332)(771.81646918,361.06397332)(771.78647461,361.07397308)
\curveto(771.73646926,361.0839733)(771.68646931,361.08897329)(771.63647461,361.08897308)
\curveto(771.5964694,361.0789733)(771.55146944,361.0789733)(771.50147461,361.08897308)
\lineto(771.35147461,361.08897308)
\curveto(771.27146972,361.10897327)(771.1964698,361.12397326)(771.12647461,361.13397308)
\curveto(771.05646994,361.13397325)(770.98147001,361.14397324)(770.90147461,361.16397308)
\lineto(770.63147461,361.22397308)
\curveto(770.54147045,361.23397315)(770.45647054,361.25397313)(770.37647461,361.28397308)
\curveto(770.16647083,361.34397304)(769.97647102,361.41897296)(769.80647461,361.50897308)
\curveto(769.17647182,361.7789726)(768.66647233,362.16397222)(768.27647461,362.66397308)
\curveto(767.88647311,363.16397122)(767.57647342,363.75397063)(767.34647461,364.43397308)
\curveto(767.30647369,364.55396983)(767.27147372,364.6789697)(767.24147461,364.80897308)
\curveto(767.22147377,364.93896944)(767.1964738,365.07396931)(767.16647461,365.21397308)
\curveto(767.14647385,365.26396912)(767.13647386,365.30896907)(767.13647461,365.34897308)
\curveto(767.14647385,365.38896899)(767.14647385,365.43396895)(767.13647461,365.48397308)
\curveto(767.11647388,365.57396881)(767.10147389,365.66896871)(767.09147461,365.76897308)
\curveto(767.0914739,365.86896851)(767.08147391,365.96396842)(767.06147461,366.05397308)
\lineto(767.06147461,366.33897308)
\curveto(767.04147395,366.38896799)(767.03147396,366.47396791)(767.03147461,366.59397308)
\curveto(767.03147396,366.71396767)(767.04147395,366.79896758)(767.06147461,366.84897308)
\curveto(767.07147392,366.8789675)(767.07147392,366.90896747)(767.06147461,366.93897308)
\curveto(767.05147394,366.9789674)(767.05147394,367.00896737)(767.06147461,367.02897308)
\lineto(767.06147461,367.16397308)
\curveto(767.07147392,367.24396714)(767.07647392,367.32396706)(767.07647461,367.40397308)
\curveto(767.08647391,367.49396689)(767.10147389,367.5789668)(767.12147461,367.65897308)
\curveto(767.14147385,367.71896666)(767.15147384,367.7789666)(767.15147461,367.83897308)
\curveto(767.15147384,367.90896647)(767.16147383,367.9789664)(767.18147461,368.04897308)
\curveto(767.23147376,368.21896616)(767.27147372,368.383966)(767.30147461,368.54397308)
\curveto(767.33147366,368.70396568)(767.37647362,368.85396553)(767.43647461,368.99397308)
\lineto(767.58647461,369.38397308)
\curveto(767.64647335,369.52396486)(767.71147328,369.64896473)(767.78147461,369.75897308)
\curveto(767.93147306,370.01896436)(768.08147291,370.25396413)(768.23147461,370.46397308)
\curveto(768.26147273,370.51396387)(768.2964727,370.55396383)(768.33647461,370.58397308)
\curveto(768.38647261,370.62396376)(768.42647257,370.66896371)(768.45647461,370.71897308)
\curveto(768.51647248,370.79896358)(768.57647242,370.86896351)(768.63647461,370.92897308)
\lineto(768.84647461,371.10897308)
\curveto(768.90647209,371.15896322)(768.96147203,371.20396318)(769.01147461,371.24397308)
\curveto(769.07147192,371.29396309)(769.13647186,371.34396304)(769.20647461,371.39397308)
\curveto(769.35647164,371.50396288)(769.51147148,371.59896278)(769.67147461,371.67897308)
\curveto(769.84147115,371.75896262)(770.01647098,371.83896254)(770.19647461,371.91897308)
\curveto(770.30647069,371.96896241)(770.42147057,372.00396238)(770.54147461,372.02397308)
\curveto(770.67147032,372.05396233)(770.7964702,372.08896229)(770.91647461,372.12897308)
\curveto(770.98647001,372.13896224)(771.05146994,372.14896223)(771.11147461,372.15897308)
\lineto(771.29147461,372.18897308)
\curveto(771.37146962,372.19896218)(771.44646955,372.20396218)(771.51647461,372.20397308)
\curveto(771.5964694,372.21396217)(771.67646932,372.22396216)(771.75647461,372.23397308)
\curveto(771.77646922,372.24396214)(771.80146919,372.24396214)(771.83147461,372.23397308)
\curveto(771.86146913,372.22396216)(771.88646911,372.22396216)(771.90647461,372.23397308)
}
}
{
\newrgbcolor{curcolor}{0 0 0}
\pscustom[linestyle=none,fillstyle=solid,fillcolor=curcolor]
{
\newpath
\moveto(785.02631836,361.86897308)
\curveto(785.05631053,361.70897267)(785.04131054,361.57397281)(784.98131836,361.46397308)
\curveto(784.92131066,361.36397302)(784.84131074,361.28897309)(784.74131836,361.23897308)
\curveto(784.69131089,361.21897316)(784.63631095,361.20897317)(784.57631836,361.20897308)
\curveto(784.52631106,361.20897317)(784.47131111,361.19897318)(784.41131836,361.17897308)
\curveto(784.19131139,361.12897325)(783.97131161,361.14397324)(783.75131836,361.22397308)
\curveto(783.54131204,361.29397309)(783.39631219,361.383973)(783.31631836,361.49397308)
\curveto(783.26631232,361.56397282)(783.22131236,361.64397274)(783.18131836,361.73397308)
\curveto(783.14131244,361.83397255)(783.09131249,361.91397247)(783.03131836,361.97397308)
\curveto(783.01131257,361.99397239)(782.9863126,362.01397237)(782.95631836,362.03397308)
\curveto(782.93631265,362.05397233)(782.90631268,362.05897232)(782.86631836,362.04897308)
\curveto(782.75631283,362.01897236)(782.65131293,361.96397242)(782.55131836,361.88397308)
\curveto(782.46131312,361.80397258)(782.37131321,361.73397265)(782.28131836,361.67397308)
\curveto(782.15131343,361.59397279)(782.01131357,361.51897286)(781.86131836,361.44897308)
\curveto(781.71131387,361.38897299)(781.55131403,361.33397305)(781.38131836,361.28397308)
\curveto(781.2813143,361.25397313)(781.17131441,361.23397315)(781.05131836,361.22397308)
\curveto(780.94131464,361.21397317)(780.83131475,361.19897318)(780.72131836,361.17897308)
\curveto(780.67131491,361.16897321)(780.62631496,361.16397322)(780.58631836,361.16397308)
\lineto(780.48131836,361.16397308)
\curveto(780.37131521,361.14397324)(780.26631532,361.14397324)(780.16631836,361.16397308)
\lineto(780.03131836,361.16397308)
\curveto(779.9813156,361.17397321)(779.93131565,361.1789732)(779.88131836,361.17897308)
\curveto(779.83131575,361.1789732)(779.7863158,361.18897319)(779.74631836,361.20897308)
\curveto(779.70631588,361.21897316)(779.67131591,361.22397316)(779.64131836,361.22397308)
\curveto(779.62131596,361.21397317)(779.59631599,361.21397317)(779.56631836,361.22397308)
\lineto(779.32631836,361.28397308)
\curveto(779.24631634,361.29397309)(779.17131641,361.31397307)(779.10131836,361.34397308)
\curveto(778.80131678,361.47397291)(778.55631703,361.61897276)(778.36631836,361.77897308)
\curveto(778.1863174,361.94897243)(778.03631755,362.1839722)(777.91631836,362.48397308)
\curveto(777.82631776,362.70397168)(777.7813178,362.96897141)(777.78131836,363.27897308)
\lineto(777.78131836,363.59397308)
\curveto(777.79131779,363.64397074)(777.79631779,363.69397069)(777.79631836,363.74397308)
\lineto(777.82631836,363.92397308)
\lineto(777.94631836,364.25397308)
\curveto(777.9863176,364.36397002)(778.03631755,364.46396992)(778.09631836,364.55397308)
\curveto(778.27631731,364.84396954)(778.52131706,365.05896932)(778.83131836,365.19897308)
\curveto(779.14131644,365.33896904)(779.4813161,365.46396892)(779.85131836,365.57397308)
\curveto(779.99131559,365.61396877)(780.13631545,365.64396874)(780.28631836,365.66397308)
\curveto(780.43631515,365.6839687)(780.586315,365.70896867)(780.73631836,365.73897308)
\curveto(780.80631478,365.75896862)(780.87131471,365.76896861)(780.93131836,365.76897308)
\curveto(781.00131458,365.76896861)(781.07631451,365.7789686)(781.15631836,365.79897308)
\curveto(781.22631436,365.81896856)(781.29631429,365.82896855)(781.36631836,365.82897308)
\curveto(781.43631415,365.83896854)(781.51131407,365.85396853)(781.59131836,365.87397308)
\curveto(781.84131374,365.93396845)(782.07631351,365.9839684)(782.29631836,366.02397308)
\curveto(782.51631307,366.07396831)(782.69131289,366.18896819)(782.82131836,366.36897308)
\curveto(782.8813127,366.44896793)(782.93131265,366.54896783)(782.97131836,366.66897308)
\curveto(783.01131257,366.79896758)(783.01131257,366.93896744)(782.97131836,367.08897308)
\curveto(782.91131267,367.32896705)(782.82131276,367.51896686)(782.70131836,367.65897308)
\curveto(782.59131299,367.79896658)(782.43131315,367.90896647)(782.22131836,367.98897308)
\curveto(782.10131348,368.03896634)(781.95631363,368.07396631)(781.78631836,368.09397308)
\curveto(781.62631396,368.11396627)(781.45631413,368.12396626)(781.27631836,368.12397308)
\curveto(781.09631449,368.12396626)(780.92131466,368.11396627)(780.75131836,368.09397308)
\curveto(780.581315,368.07396631)(780.43631515,368.04396634)(780.31631836,368.00397308)
\curveto(780.14631544,367.94396644)(779.9813156,367.85896652)(779.82131836,367.74897308)
\curveto(779.74131584,367.68896669)(779.66631592,367.60896677)(779.59631836,367.50897308)
\curveto(779.53631605,367.41896696)(779.4813161,367.31896706)(779.43131836,367.20897308)
\curveto(779.40131618,367.12896725)(779.37131621,367.04396734)(779.34131836,366.95397308)
\curveto(779.32131626,366.86396752)(779.27631631,366.79396759)(779.20631836,366.74397308)
\curveto(779.16631642,366.71396767)(779.09631649,366.68896769)(778.99631836,366.66897308)
\curveto(778.90631668,366.65896772)(778.81131677,366.65396773)(778.71131836,366.65397308)
\curveto(778.61131697,366.65396773)(778.51131707,366.65896772)(778.41131836,366.66897308)
\curveto(778.32131726,366.68896769)(778.25631733,366.71396767)(778.21631836,366.74397308)
\curveto(778.17631741,366.77396761)(778.14631744,366.82396756)(778.12631836,366.89397308)
\curveto(778.10631748,366.96396742)(778.10631748,367.03896734)(778.12631836,367.11897308)
\curveto(778.15631743,367.24896713)(778.1863174,367.36896701)(778.21631836,367.47897308)
\curveto(778.25631733,367.59896678)(778.30131728,367.71396667)(778.35131836,367.82397308)
\curveto(778.54131704,368.17396621)(778.7813168,368.44396594)(779.07131836,368.63397308)
\curveto(779.36131622,368.83396555)(779.72131586,368.99396539)(780.15131836,369.11397308)
\curveto(780.25131533,369.13396525)(780.35131523,369.14896523)(780.45131836,369.15897308)
\curveto(780.56131502,369.16896521)(780.67131491,369.1839652)(780.78131836,369.20397308)
\curveto(780.82131476,369.21396517)(780.8863147,369.21396517)(780.97631836,369.20397308)
\curveto(781.06631452,369.20396518)(781.12131446,369.21396517)(781.14131836,369.23397308)
\curveto(781.84131374,369.24396514)(782.45131313,369.16396522)(782.97131836,368.99397308)
\curveto(783.49131209,368.82396556)(783.85631173,368.49896588)(784.06631836,368.01897308)
\curveto(784.15631143,367.81896656)(784.20631138,367.5839668)(784.21631836,367.31397308)
\curveto(784.23631135,367.05396733)(784.24631134,366.7789676)(784.24631836,366.48897308)
\lineto(784.24631836,363.17397308)
\curveto(784.24631134,363.03397135)(784.25131133,362.89897148)(784.26131836,362.76897308)
\curveto(784.27131131,362.63897174)(784.30131128,362.53397185)(784.35131836,362.45397308)
\curveto(784.40131118,362.383972)(784.46631112,362.33397205)(784.54631836,362.30397308)
\curveto(784.63631095,362.26397212)(784.72131086,362.23397215)(784.80131836,362.21397308)
\curveto(784.8813107,362.20397218)(784.94131064,362.15897222)(784.98131836,362.07897308)
\curveto(785.00131058,362.04897233)(785.01131057,362.01897236)(785.01131836,361.98897308)
\curveto(785.01131057,361.95897242)(785.01631057,361.91897246)(785.02631836,361.86897308)
\moveto(782.88131836,363.53397308)
\curveto(782.94131264,363.67397071)(782.97131261,363.83397055)(782.97131836,364.01397308)
\curveto(782.9813126,364.20397018)(782.9863126,364.39896998)(782.98631836,364.59897308)
\curveto(782.9863126,364.70896967)(782.9813126,364.80896957)(782.97131836,364.89897308)
\curveto(782.96131262,364.98896939)(782.92131266,365.05896932)(782.85131836,365.10897308)
\curveto(782.82131276,365.12896925)(782.75131283,365.13896924)(782.64131836,365.13897308)
\curveto(782.62131296,365.11896926)(782.586313,365.10896927)(782.53631836,365.10897308)
\curveto(782.4863131,365.10896927)(782.44131314,365.09896928)(782.40131836,365.07897308)
\curveto(782.32131326,365.05896932)(782.23131335,365.03896934)(782.13131836,365.01897308)
\lineto(781.83131836,364.95897308)
\curveto(781.80131378,364.95896942)(781.76631382,364.95396943)(781.72631836,364.94397308)
\lineto(781.62131836,364.94397308)
\curveto(781.47131411,364.90396948)(781.30631428,364.8789695)(781.12631836,364.86897308)
\curveto(780.95631463,364.86896951)(780.79631479,364.84896953)(780.64631836,364.80897308)
\curveto(780.56631502,364.78896959)(780.49131509,364.76896961)(780.42131836,364.74897308)
\curveto(780.36131522,364.73896964)(780.29131529,364.72396966)(780.21131836,364.70397308)
\curveto(780.05131553,364.65396973)(779.90131568,364.58896979)(779.76131836,364.50897308)
\curveto(779.62131596,364.43896994)(779.50131608,364.34897003)(779.40131836,364.23897308)
\curveto(779.30131628,364.12897025)(779.22631636,363.99397039)(779.17631836,363.83397308)
\curveto(779.12631646,363.6839707)(779.10631648,363.49897088)(779.11631836,363.27897308)
\curveto(779.11631647,363.1789712)(779.13131645,363.0839713)(779.16131836,362.99397308)
\curveto(779.20131638,362.91397147)(779.24631634,362.83897154)(779.29631836,362.76897308)
\curveto(779.37631621,362.65897172)(779.4813161,362.56397182)(779.61131836,362.48397308)
\curveto(779.74131584,362.41397197)(779.8813157,362.35397203)(780.03131836,362.30397308)
\curveto(780.0813155,362.29397209)(780.13131545,362.28897209)(780.18131836,362.28897308)
\curveto(780.23131535,362.28897209)(780.2813153,362.2839721)(780.33131836,362.27397308)
\curveto(780.40131518,362.25397213)(780.4863151,362.23897214)(780.58631836,362.22897308)
\curveto(780.69631489,362.22897215)(780.7863148,362.23897214)(780.85631836,362.25897308)
\curveto(780.91631467,362.2789721)(780.97631461,362.2839721)(781.03631836,362.27397308)
\curveto(781.09631449,362.27397211)(781.15631443,362.2839721)(781.21631836,362.30397308)
\curveto(781.29631429,362.32397206)(781.37131421,362.33897204)(781.44131836,362.34897308)
\curveto(781.52131406,362.35897202)(781.59631399,362.378972)(781.66631836,362.40897308)
\curveto(781.95631363,362.52897185)(782.20131338,362.67397171)(782.40131836,362.84397308)
\curveto(782.61131297,363.01397137)(782.77131281,363.24397114)(782.88131836,363.53397308)
}
}
{
\newrgbcolor{curcolor}{0 0 0}
\pscustom[linestyle=none,fillstyle=solid,fillcolor=curcolor]
{
\newpath
\moveto(789.88795898,369.18897308)
\curveto(790.51795375,369.20896517)(791.02295324,369.12396526)(791.40295898,368.93397308)
\curveto(791.78295248,368.74396564)(792.08795218,368.45896592)(792.31795898,368.07897308)
\curveto(792.37795189,367.9789664)(792.42295184,367.86896651)(792.45295898,367.74897308)
\curveto(792.49295177,367.63896674)(792.52795174,367.52396686)(792.55795898,367.40397308)
\curveto(792.60795166,367.21396717)(792.63795163,367.00896737)(792.64795898,366.78897308)
\curveto(792.65795161,366.56896781)(792.6629516,366.34396804)(792.66295898,366.11397308)
\lineto(792.66295898,364.50897308)
\lineto(792.66295898,362.16897308)
\curveto(792.6629516,361.99897238)(792.65795161,361.82897255)(792.64795898,361.65897308)
\curveto(792.64795162,361.48897289)(792.58295168,361.378973)(792.45295898,361.32897308)
\curveto(792.40295186,361.30897307)(792.34795192,361.29897308)(792.28795898,361.29897308)
\curveto(792.23795203,361.28897309)(792.18295208,361.2839731)(792.12295898,361.28397308)
\curveto(791.99295227,361.2839731)(791.8679524,361.28897309)(791.74795898,361.29897308)
\curveto(791.62795264,361.29897308)(791.54295272,361.33897304)(791.49295898,361.41897308)
\curveto(791.44295282,361.48897289)(791.41795285,361.5789728)(791.41795898,361.68897308)
\lineto(791.41795898,362.01897308)
\lineto(791.41795898,363.30897308)
\lineto(791.41795898,365.75397308)
\curveto(791.41795285,366.02396836)(791.41295285,366.28896809)(791.40295898,366.54897308)
\curveto(791.39295287,366.81896756)(791.34795292,367.04896733)(791.26795898,367.23897308)
\curveto(791.18795308,367.43896694)(791.0679532,367.59896678)(790.90795898,367.71897308)
\curveto(790.74795352,367.84896653)(790.5629537,367.94896643)(790.35295898,368.01897308)
\curveto(790.29295397,368.03896634)(790.22795404,368.04896633)(790.15795898,368.04897308)
\curveto(790.09795417,368.05896632)(790.03795423,368.07396631)(789.97795898,368.09397308)
\curveto(789.92795434,368.10396628)(789.84795442,368.10396628)(789.73795898,368.09397308)
\curveto(789.63795463,368.09396629)(789.5679547,368.08896629)(789.52795898,368.07897308)
\curveto(789.48795478,368.05896632)(789.45295481,368.04896633)(789.42295898,368.04897308)
\curveto(789.39295487,368.05896632)(789.35795491,368.05896632)(789.31795898,368.04897308)
\curveto(789.18795508,368.01896636)(789.0629552,367.9839664)(788.94295898,367.94397308)
\curveto(788.83295543,367.91396647)(788.72795554,367.86896651)(788.62795898,367.80897308)
\curveto(788.58795568,367.78896659)(788.55295571,367.76896661)(788.52295898,367.74897308)
\curveto(788.49295577,367.72896665)(788.45795581,367.70896667)(788.41795898,367.68897308)
\curveto(788.0679562,367.43896694)(787.81295645,367.06396732)(787.65295898,366.56397308)
\curveto(787.62295664,366.4839679)(787.60295666,366.39896798)(787.59295898,366.30897308)
\curveto(787.58295668,366.22896815)(787.5679567,366.14896823)(787.54795898,366.06897308)
\curveto(787.52795674,366.01896836)(787.52295674,365.96896841)(787.53295898,365.91897308)
\curveto(787.54295672,365.8789685)(787.53795673,365.83896854)(787.51795898,365.79897308)
\lineto(787.51795898,365.48397308)
\curveto(787.50795676,365.45396893)(787.50295676,365.41896896)(787.50295898,365.37897308)
\curveto(787.51295675,365.33896904)(787.51795675,365.29396909)(787.51795898,365.24397308)
\lineto(787.51795898,364.79397308)
\lineto(787.51795898,363.35397308)
\lineto(787.51795898,362.03397308)
\lineto(787.51795898,361.68897308)
\curveto(787.51795675,361.5789728)(787.49295677,361.48897289)(787.44295898,361.41897308)
\curveto(787.39295687,361.33897304)(787.30295696,361.29897308)(787.17295898,361.29897308)
\curveto(787.05295721,361.28897309)(786.92795734,361.2839731)(786.79795898,361.28397308)
\curveto(786.71795755,361.2839731)(786.64295762,361.28897309)(786.57295898,361.29897308)
\curveto(786.50295776,361.30897307)(786.44295782,361.33397305)(786.39295898,361.37397308)
\curveto(786.31295795,361.42397296)(786.27295799,361.51897286)(786.27295898,361.65897308)
\lineto(786.27295898,362.06397308)
\lineto(786.27295898,363.83397308)
\lineto(786.27295898,367.46397308)
\lineto(786.27295898,368.37897308)
\lineto(786.27295898,368.64897308)
\curveto(786.27295799,368.73896564)(786.29295797,368.80896557)(786.33295898,368.85897308)
\curveto(786.3629579,368.91896546)(786.41295785,368.95896542)(786.48295898,368.97897308)
\curveto(786.52295774,368.98896539)(786.57795769,368.99896538)(786.64795898,369.00897308)
\curveto(786.72795754,369.01896536)(786.80795746,369.02396536)(786.88795898,369.02397308)
\curveto(786.9679573,369.02396536)(787.04295722,369.01896536)(787.11295898,369.00897308)
\curveto(787.19295707,368.99896538)(787.24795702,368.9839654)(787.27795898,368.96397308)
\curveto(787.38795688,368.89396549)(787.43795683,368.80396558)(787.42795898,368.69397308)
\curveto(787.41795685,368.59396579)(787.43295683,368.4789659)(787.47295898,368.34897308)
\curveto(787.49295677,368.28896609)(787.53295673,368.23896614)(787.59295898,368.19897308)
\curveto(787.71295655,368.18896619)(787.80795646,368.23396615)(787.87795898,368.33397308)
\curveto(787.95795631,368.43396595)(788.03795623,368.51396587)(788.11795898,368.57397308)
\curveto(788.25795601,368.67396571)(788.39795587,368.76396562)(788.53795898,368.84397308)
\curveto(788.68795558,368.93396545)(788.85795541,369.00896537)(789.04795898,369.06897308)
\curveto(789.12795514,369.09896528)(789.21295505,369.11896526)(789.30295898,369.12897308)
\curveto(789.40295486,369.13896524)(789.49795477,369.15396523)(789.58795898,369.17397308)
\curveto(789.63795463,369.1839652)(789.68795458,369.18896519)(789.73795898,369.18897308)
\lineto(789.88795898,369.18897308)
}
}
{
\newrgbcolor{curcolor}{0 0 0}
\pscustom[linestyle=none,fillstyle=solid,fillcolor=curcolor]
{
\newpath
\moveto(795.49256836,371.37897308)
\curveto(795.64256635,371.378963)(795.7925662,371.37396301)(795.94256836,371.36397308)
\curveto(796.0925659,371.36396302)(796.19756579,371.32396306)(796.25756836,371.24397308)
\curveto(796.30756568,371.1839632)(796.33256566,371.09896328)(796.33256836,370.98897308)
\curveto(796.34256565,370.88896349)(796.34756564,370.7839636)(796.34756836,370.67397308)
\lineto(796.34756836,369.80397308)
\curveto(796.34756564,369.72396466)(796.34256565,369.63896474)(796.33256836,369.54897308)
\curveto(796.33256566,369.46896491)(796.34256565,369.39896498)(796.36256836,369.33897308)
\curveto(796.40256559,369.19896518)(796.4925655,369.10896527)(796.63256836,369.06897308)
\curveto(796.68256531,369.05896532)(796.72756526,369.05396533)(796.76756836,369.05397308)
\lineto(796.91756836,369.05397308)
\lineto(797.32256836,369.05397308)
\curveto(797.48256451,369.06396532)(797.59756439,369.05396533)(797.66756836,369.02397308)
\curveto(797.75756423,368.96396542)(797.81756417,368.90396548)(797.84756836,368.84397308)
\curveto(797.86756412,368.80396558)(797.87756411,368.75896562)(797.87756836,368.70897308)
\lineto(797.87756836,368.55897308)
\curveto(797.87756411,368.44896593)(797.87256412,368.34396604)(797.86256836,368.24397308)
\curveto(797.85256414,368.15396623)(797.81756417,368.0839663)(797.75756836,368.03397308)
\curveto(797.69756429,367.9839664)(797.61256438,367.95396643)(797.50256836,367.94397308)
\lineto(797.17256836,367.94397308)
\curveto(797.06256493,367.95396643)(796.95256504,367.95896642)(796.84256836,367.95897308)
\curveto(796.73256526,367.95896642)(796.63756535,367.94396644)(796.55756836,367.91397308)
\curveto(796.4875655,367.8839665)(796.43756555,367.83396655)(796.40756836,367.76397308)
\curveto(796.37756561,367.69396669)(796.35756563,367.60896677)(796.34756836,367.50897308)
\curveto(796.33756565,367.41896696)(796.33256566,367.31896706)(796.33256836,367.20897308)
\curveto(796.34256565,367.10896727)(796.34756564,367.00896737)(796.34756836,366.90897308)
\lineto(796.34756836,363.93897308)
\curveto(796.34756564,363.71897066)(796.34256565,363.4839709)(796.33256836,363.23397308)
\curveto(796.33256566,362.99397139)(796.37756561,362.80897157)(796.46756836,362.67897308)
\curveto(796.51756547,362.59897178)(796.58256541,362.54397184)(796.66256836,362.51397308)
\curveto(796.74256525,362.4839719)(796.83756515,362.45897192)(796.94756836,362.43897308)
\curveto(796.97756501,362.42897195)(797.00756498,362.42397196)(797.03756836,362.42397308)
\curveto(797.07756491,362.43397195)(797.11256488,362.43397195)(797.14256836,362.42397308)
\lineto(797.33756836,362.42397308)
\curveto(797.43756455,362.42397196)(797.52756446,362.41397197)(797.60756836,362.39397308)
\curveto(797.69756429,362.383972)(797.76256423,362.34897203)(797.80256836,362.28897308)
\curveto(797.82256417,362.25897212)(797.83756415,362.20397218)(797.84756836,362.12397308)
\curveto(797.86756412,362.05397233)(797.87756411,361.9789724)(797.87756836,361.89897308)
\curveto(797.8875641,361.81897256)(797.8875641,361.73897264)(797.87756836,361.65897308)
\curveto(797.86756412,361.58897279)(797.84756414,361.53397285)(797.81756836,361.49397308)
\curveto(797.77756421,361.42397296)(797.70256429,361.37397301)(797.59256836,361.34397308)
\curveto(797.51256448,361.32397306)(797.42256457,361.31397307)(797.32256836,361.31397308)
\curveto(797.22256477,361.32397306)(797.13256486,361.32897305)(797.05256836,361.32897308)
\curveto(796.992565,361.32897305)(796.93256506,361.32397306)(796.87256836,361.31397308)
\curveto(796.81256518,361.31397307)(796.75756523,361.31897306)(796.70756836,361.32897308)
\lineto(796.52756836,361.32897308)
\curveto(796.47756551,361.33897304)(796.42756556,361.34397304)(796.37756836,361.34397308)
\curveto(796.33756565,361.35397303)(796.2925657,361.35897302)(796.24256836,361.35897308)
\curveto(796.04256595,361.40897297)(795.86756612,361.46397292)(795.71756836,361.52397308)
\curveto(795.57756641,361.5839728)(795.45756653,361.68897269)(795.35756836,361.83897308)
\curveto(795.21756677,362.03897234)(795.13756685,362.28897209)(795.11756836,362.58897308)
\curveto(795.09756689,362.89897148)(795.0875669,363.22897115)(795.08756836,363.57897308)
\lineto(795.08756836,367.50897308)
\curveto(795.05756693,367.63896674)(795.02756696,367.73396665)(794.99756836,367.79397308)
\curveto(794.97756701,367.85396653)(794.90756708,367.90396648)(794.78756836,367.94397308)
\curveto(794.74756724,367.95396643)(794.70756728,367.95396643)(794.66756836,367.94397308)
\curveto(794.62756736,367.93396645)(794.5875674,367.93896644)(794.54756836,367.95897308)
\lineto(794.30756836,367.95897308)
\curveto(794.17756781,367.95896642)(794.06756792,367.96896641)(793.97756836,367.98897308)
\curveto(793.89756809,368.01896636)(793.84256815,368.0789663)(793.81256836,368.16897308)
\curveto(793.7925682,368.20896617)(793.77756821,368.25396613)(793.76756836,368.30397308)
\lineto(793.76756836,368.45397308)
\curveto(793.76756822,368.59396579)(793.77756821,368.70896567)(793.79756836,368.79897308)
\curveto(793.81756817,368.89896548)(793.87756811,368.97396541)(793.97756836,369.02397308)
\curveto(794.0875679,369.06396532)(794.22756776,369.07396531)(794.39756836,369.05397308)
\curveto(794.57756741,369.03396535)(794.72756726,369.04396534)(794.84756836,369.08397308)
\curveto(794.93756705,369.13396525)(795.00756698,369.20396518)(795.05756836,369.29397308)
\curveto(795.07756691,369.35396503)(795.0875669,369.42896495)(795.08756836,369.51897308)
\lineto(795.08756836,369.77397308)
\lineto(795.08756836,370.70397308)
\lineto(795.08756836,370.94397308)
\curveto(795.0875669,371.03396335)(795.09756689,371.10896327)(795.11756836,371.16897308)
\curveto(795.15756683,371.24896313)(795.23256676,371.31396307)(795.34256836,371.36397308)
\curveto(795.37256662,371.36396302)(795.39756659,371.36396302)(795.41756836,371.36397308)
\curveto(795.44756654,371.37396301)(795.47256652,371.378963)(795.49256836,371.37897308)
}
}
{
\newrgbcolor{curcolor}{0 0 0}
\pscustom[linestyle=none,fillstyle=solid,fillcolor=curcolor]
{
\newpath
\moveto(799.54936523,370.53897308)
\curveto(799.46936411,370.59896378)(799.42436416,370.70396368)(799.41436523,370.85397308)
\lineto(799.41436523,371.31897308)
\lineto(799.41436523,371.57397308)
\curveto(799.41436417,371.66396272)(799.42936415,371.73896264)(799.45936523,371.79897308)
\curveto(799.49936408,371.8789625)(799.579364,371.93896244)(799.69936523,371.97897308)
\curveto(799.71936386,371.98896239)(799.73936384,371.98896239)(799.75936523,371.97897308)
\curveto(799.78936379,371.9789624)(799.81436377,371.9839624)(799.83436523,371.99397308)
\curveto(800.00436358,371.99396239)(800.16436342,371.98896239)(800.31436523,371.97897308)
\curveto(800.46436312,371.96896241)(800.56436302,371.90896247)(800.61436523,371.79897308)
\curveto(800.64436294,371.73896264)(800.65936292,371.66396272)(800.65936523,371.57397308)
\lineto(800.65936523,371.31897308)
\curveto(800.65936292,371.13896324)(800.65436293,370.96896341)(800.64436523,370.80897308)
\curveto(800.64436294,370.64896373)(800.579363,370.54396384)(800.44936523,370.49397308)
\curveto(800.39936318,370.47396391)(800.34436324,370.46396392)(800.28436523,370.46397308)
\lineto(800.11936523,370.46397308)
\lineto(799.80436523,370.46397308)
\curveto(799.70436388,370.46396392)(799.61936396,370.48896389)(799.54936523,370.53897308)
\moveto(800.65936523,362.03397308)
\lineto(800.65936523,361.71897308)
\curveto(800.66936291,361.61897276)(800.64936293,361.53897284)(800.59936523,361.47897308)
\curveto(800.56936301,361.41897296)(800.52436306,361.378973)(800.46436523,361.35897308)
\curveto(800.40436318,361.34897303)(800.33436325,361.33397305)(800.25436523,361.31397308)
\lineto(800.02936523,361.31397308)
\curveto(799.89936368,361.31397307)(799.7843638,361.31897306)(799.68436523,361.32897308)
\curveto(799.59436399,361.34897303)(799.52436406,361.39897298)(799.47436523,361.47897308)
\curveto(799.43436415,361.53897284)(799.41436417,361.61397277)(799.41436523,361.70397308)
\lineto(799.41436523,361.98897308)
\lineto(799.41436523,368.33397308)
\lineto(799.41436523,368.64897308)
\curveto(799.41436417,368.75896562)(799.43936414,368.84396554)(799.48936523,368.90397308)
\curveto(799.51936406,368.95396543)(799.55936402,368.9839654)(799.60936523,368.99397308)
\curveto(799.65936392,369.00396538)(799.71436387,369.01896536)(799.77436523,369.03897308)
\curveto(799.79436379,369.03896534)(799.81436377,369.03396535)(799.83436523,369.02397308)
\curveto(799.86436372,369.02396536)(799.88936369,369.02896535)(799.90936523,369.03897308)
\curveto(800.03936354,369.03896534)(800.16936341,369.03396535)(800.29936523,369.02397308)
\curveto(800.43936314,369.02396536)(800.53436305,368.9839654)(800.58436523,368.90397308)
\curveto(800.63436295,368.84396554)(800.65936292,368.76396562)(800.65936523,368.66397308)
\lineto(800.65936523,368.37897308)
\lineto(800.65936523,362.03397308)
}
}
{
\newrgbcolor{curcolor}{0 0 0}
\pscustom[linestyle=none,fillstyle=solid,fillcolor=curcolor]
{
\newpath
\moveto(809.56420898,362.12397308)
\lineto(809.56420898,361.73397308)
\curveto(809.56420111,361.61397277)(809.53920113,361.51397287)(809.48920898,361.43397308)
\curveto(809.43920123,361.36397302)(809.35420132,361.32397306)(809.23420898,361.31397308)
\lineto(808.88920898,361.31397308)
\curveto(808.82920184,361.31397307)(808.7692019,361.30897307)(808.70920898,361.29897308)
\curveto(808.65920201,361.29897308)(808.61420206,361.30897307)(808.57420898,361.32897308)
\curveto(808.48420219,361.34897303)(808.42420225,361.38897299)(808.39420898,361.44897308)
\curveto(808.35420232,361.49897288)(808.32920234,361.55897282)(808.31920898,361.62897308)
\curveto(808.31920235,361.69897268)(808.30420237,361.76897261)(808.27420898,361.83897308)
\curveto(808.26420241,361.85897252)(808.24920242,361.87397251)(808.22920898,361.88397308)
\curveto(808.21920245,361.90397248)(808.20420247,361.92397246)(808.18420898,361.94397308)
\curveto(808.08420259,361.95397243)(808.00420267,361.93397245)(807.94420898,361.88397308)
\curveto(807.89420278,361.83397255)(807.83920283,361.7839726)(807.77920898,361.73397308)
\curveto(807.57920309,361.5839728)(807.37920329,361.46897291)(807.17920898,361.38897308)
\curveto(806.99920367,361.30897307)(806.78920388,361.24897313)(806.54920898,361.20897308)
\curveto(806.31920435,361.16897321)(806.07920459,361.14897323)(805.82920898,361.14897308)
\curveto(805.58920508,361.13897324)(805.34920532,361.15397323)(805.10920898,361.19397308)
\curveto(804.8692058,361.22397316)(804.65920601,361.2789731)(804.47920898,361.35897308)
\curveto(803.95920671,361.5789728)(803.53920713,361.87397251)(803.21920898,362.24397308)
\curveto(802.89920777,362.62397176)(802.64920802,363.09397129)(802.46920898,363.65397308)
\curveto(802.42920824,363.74397064)(802.39920827,363.83397055)(802.37920898,363.92397308)
\curveto(802.3692083,364.02397036)(802.34920832,364.12397026)(802.31920898,364.22397308)
\curveto(802.30920836,364.27397011)(802.30420837,364.32397006)(802.30420898,364.37397308)
\curveto(802.30420837,364.42396996)(802.29920837,364.47396991)(802.28920898,364.52397308)
\curveto(802.2692084,364.57396981)(802.25920841,364.62396976)(802.25920898,364.67397308)
\curveto(802.2692084,364.73396965)(802.2692084,364.78896959)(802.25920898,364.83897308)
\lineto(802.25920898,364.98897308)
\curveto(802.23920843,365.03896934)(802.22920844,365.10396928)(802.22920898,365.18397308)
\curveto(802.22920844,365.26396912)(802.23920843,365.32896905)(802.25920898,365.37897308)
\lineto(802.25920898,365.54397308)
\curveto(802.27920839,365.61396877)(802.28420839,365.6839687)(802.27420898,365.75397308)
\curveto(802.2742084,365.83396855)(802.28420839,365.90896847)(802.30420898,365.97897308)
\curveto(802.31420836,366.02896835)(802.31920835,366.07396831)(802.31920898,366.11397308)
\curveto(802.31920835,366.15396823)(802.32420835,366.19896818)(802.33420898,366.24897308)
\curveto(802.36420831,366.34896803)(802.38920828,366.44396794)(802.40920898,366.53397308)
\curveto(802.42920824,366.63396775)(802.45420822,366.72896765)(802.48420898,366.81897308)
\curveto(802.61420806,367.19896718)(802.77920789,367.53896684)(802.97920898,367.83897308)
\curveto(803.18920748,368.14896623)(803.43920723,368.40396598)(803.72920898,368.60397308)
\curveto(803.89920677,368.72396566)(804.0742066,368.82396556)(804.25420898,368.90397308)
\curveto(804.44420623,368.9839654)(804.64920602,369.05396533)(804.86920898,369.11397308)
\curveto(804.93920573,369.12396526)(805.00420567,369.13396525)(805.06420898,369.14397308)
\curveto(805.13420554,369.15396523)(805.20420547,369.16896521)(805.27420898,369.18897308)
\lineto(805.42420898,369.18897308)
\curveto(805.50420517,369.20896517)(805.61920505,369.21896516)(805.76920898,369.21897308)
\curveto(805.92920474,369.21896516)(806.04920462,369.20896517)(806.12920898,369.18897308)
\curveto(806.1692045,369.1789652)(806.22420445,369.17396521)(806.29420898,369.17397308)
\curveto(806.40420427,369.14396524)(806.51420416,369.11896526)(806.62420898,369.09897308)
\curveto(806.73420394,369.08896529)(806.83920383,369.05896532)(806.93920898,369.00897308)
\curveto(807.08920358,368.94896543)(807.22920344,368.8839655)(807.35920898,368.81397308)
\curveto(807.49920317,368.74396564)(807.62920304,368.66396572)(807.74920898,368.57397308)
\curveto(807.80920286,368.52396586)(807.8692028,368.46896591)(807.92920898,368.40897308)
\curveto(807.99920267,368.35896602)(808.08920258,368.34396604)(808.19920898,368.36397308)
\curveto(808.21920245,368.39396599)(808.23420244,368.41896596)(808.24420898,368.43897308)
\curveto(808.26420241,368.45896592)(808.27920239,368.48896589)(808.28920898,368.52897308)
\curveto(808.31920235,368.61896576)(808.32920234,368.73396565)(808.31920898,368.87397308)
\lineto(808.31920898,369.24897308)
\lineto(808.31920898,370.97397308)
\lineto(808.31920898,371.43897308)
\curveto(808.31920235,371.61896276)(808.34420233,371.74896263)(808.39420898,371.82897308)
\curveto(808.43420224,371.89896248)(808.49420218,371.94396244)(808.57420898,371.96397308)
\curveto(808.59420208,371.96396242)(808.61920205,371.96396242)(808.64920898,371.96397308)
\curveto(808.67920199,371.97396241)(808.70420197,371.9789624)(808.72420898,371.97897308)
\curveto(808.86420181,371.98896239)(809.00920166,371.98896239)(809.15920898,371.97897308)
\curveto(809.31920135,371.9789624)(809.42920124,371.93896244)(809.48920898,371.85897308)
\curveto(809.53920113,371.7789626)(809.56420111,371.6789627)(809.56420898,371.55897308)
\lineto(809.56420898,371.18397308)
\lineto(809.56420898,362.12397308)
\moveto(808.34920898,364.95897308)
\curveto(808.3692023,365.00896937)(808.37920229,365.07396931)(808.37920898,365.15397308)
\curveto(808.37920229,365.24396914)(808.3692023,365.31396907)(808.34920898,365.36397308)
\lineto(808.34920898,365.58897308)
\curveto(808.32920234,365.6789687)(808.31420236,365.76896861)(808.30420898,365.85897308)
\curveto(808.29420238,365.95896842)(808.2742024,366.04896833)(808.24420898,366.12897308)
\curveto(808.22420245,366.20896817)(808.20420247,366.2839681)(808.18420898,366.35397308)
\curveto(808.1742025,366.42396796)(808.15420252,366.49396789)(808.12420898,366.56397308)
\curveto(808.00420267,366.86396752)(807.84920282,367.12896725)(807.65920898,367.35897308)
\curveto(807.4692032,367.58896679)(807.22920344,367.76896661)(806.93920898,367.89897308)
\curveto(806.83920383,367.94896643)(806.73420394,367.9839664)(806.62420898,368.00397308)
\curveto(806.52420415,368.03396635)(806.41420426,368.05896632)(806.29420898,368.07897308)
\curveto(806.21420446,368.09896628)(806.12420455,368.10896627)(806.02420898,368.10897308)
\lineto(805.75420898,368.10897308)
\curveto(805.70420497,368.09896628)(805.65920501,368.08896629)(805.61920898,368.07897308)
\lineto(805.48420898,368.07897308)
\curveto(805.40420527,368.05896632)(805.31920535,368.03896634)(805.22920898,368.01897308)
\curveto(805.14920552,367.99896638)(805.0692056,367.97396641)(804.98920898,367.94397308)
\curveto(804.669206,367.80396658)(804.40920626,367.59896678)(804.20920898,367.32897308)
\curveto(804.01920665,367.06896731)(803.86420681,366.76396762)(803.74420898,366.41397308)
\curveto(803.70420697,366.30396808)(803.674207,366.18896819)(803.65420898,366.06897308)
\curveto(803.64420703,365.95896842)(803.62920704,365.84896853)(803.60920898,365.73897308)
\curveto(803.60920706,365.69896868)(803.60420707,365.65896872)(803.59420898,365.61897308)
\lineto(803.59420898,365.51397308)
\curveto(803.5742071,365.46396892)(803.56420711,365.40896897)(803.56420898,365.34897308)
\curveto(803.5742071,365.28896909)(803.57920709,365.23396915)(803.57920898,365.18397308)
\lineto(803.57920898,364.85397308)
\curveto(803.57920709,364.75396963)(803.58920708,364.65896972)(803.60920898,364.56897308)
\curveto(803.61920705,364.53896984)(803.62420705,364.48896989)(803.62420898,364.41897308)
\curveto(803.64420703,364.34897003)(803.65920701,364.2789701)(803.66920898,364.20897308)
\lineto(803.72920898,363.99897308)
\curveto(803.83920683,363.64897073)(803.98920668,363.34897103)(804.17920898,363.09897308)
\curveto(804.3692063,362.84897153)(804.60920606,362.64397174)(804.89920898,362.48397308)
\curveto(804.98920568,362.43397195)(805.07920559,362.39397199)(805.16920898,362.36397308)
\curveto(805.25920541,362.33397205)(805.35920531,362.30397208)(805.46920898,362.27397308)
\curveto(805.51920515,362.25397213)(805.5692051,362.24897213)(805.61920898,362.25897308)
\curveto(805.67920499,362.26897211)(805.73420494,362.26397212)(805.78420898,362.24397308)
\curveto(805.82420485,362.23397215)(805.86420481,362.22897215)(805.90420898,362.22897308)
\lineto(806.03920898,362.22897308)
\lineto(806.17420898,362.22897308)
\curveto(806.20420447,362.23897214)(806.25420442,362.24397214)(806.32420898,362.24397308)
\curveto(806.40420427,362.26397212)(806.48420419,362.2789721)(806.56420898,362.28897308)
\curveto(806.64420403,362.30897207)(806.71920395,362.33397205)(806.78920898,362.36397308)
\curveto(807.11920355,362.50397188)(807.38420329,362.6789717)(807.58420898,362.88897308)
\curveto(807.79420288,363.10897127)(807.9692027,363.383971)(808.10920898,363.71397308)
\curveto(808.15920251,363.82397056)(808.19420248,363.93397045)(808.21420898,364.04397308)
\curveto(808.23420244,364.15397023)(808.25920241,364.26397012)(808.28920898,364.37397308)
\curveto(808.30920236,364.41396997)(808.31920235,364.44896993)(808.31920898,364.47897308)
\curveto(808.31920235,364.51896986)(808.32420235,364.55896982)(808.33420898,364.59897308)
\curveto(808.34420233,364.65896972)(808.34420233,364.71896966)(808.33420898,364.77897308)
\curveto(808.33420234,364.83896954)(808.33920233,364.89896948)(808.34920898,364.95897308)
}
}
{
\newrgbcolor{curcolor}{0 0 0}
\pscustom[linestyle=none,fillstyle=solid,fillcolor=curcolor]
{
\newpath
\moveto(818.39545898,361.86897308)
\curveto(818.42545115,361.70897267)(818.41045117,361.57397281)(818.35045898,361.46397308)
\curveto(818.29045129,361.36397302)(818.21045137,361.28897309)(818.11045898,361.23897308)
\curveto(818.06045152,361.21897316)(818.00545157,361.20897317)(817.94545898,361.20897308)
\curveto(817.89545168,361.20897317)(817.84045174,361.19897318)(817.78045898,361.17897308)
\curveto(817.56045202,361.12897325)(817.34045224,361.14397324)(817.12045898,361.22397308)
\curveto(816.91045267,361.29397309)(816.76545281,361.383973)(816.68545898,361.49397308)
\curveto(816.63545294,361.56397282)(816.59045299,361.64397274)(816.55045898,361.73397308)
\curveto(816.51045307,361.83397255)(816.46045312,361.91397247)(816.40045898,361.97397308)
\curveto(816.3804532,361.99397239)(816.35545322,362.01397237)(816.32545898,362.03397308)
\curveto(816.30545327,362.05397233)(816.2754533,362.05897232)(816.23545898,362.04897308)
\curveto(816.12545345,362.01897236)(816.02045356,361.96397242)(815.92045898,361.88397308)
\curveto(815.83045375,361.80397258)(815.74045384,361.73397265)(815.65045898,361.67397308)
\curveto(815.52045406,361.59397279)(815.3804542,361.51897286)(815.23045898,361.44897308)
\curveto(815.0804545,361.38897299)(814.92045466,361.33397305)(814.75045898,361.28397308)
\curveto(814.65045493,361.25397313)(814.54045504,361.23397315)(814.42045898,361.22397308)
\curveto(814.31045527,361.21397317)(814.20045538,361.19897318)(814.09045898,361.17897308)
\curveto(814.04045554,361.16897321)(813.99545558,361.16397322)(813.95545898,361.16397308)
\lineto(813.85045898,361.16397308)
\curveto(813.74045584,361.14397324)(813.63545594,361.14397324)(813.53545898,361.16397308)
\lineto(813.40045898,361.16397308)
\curveto(813.35045623,361.17397321)(813.30045628,361.1789732)(813.25045898,361.17897308)
\curveto(813.20045638,361.1789732)(813.15545642,361.18897319)(813.11545898,361.20897308)
\curveto(813.0754565,361.21897316)(813.04045654,361.22397316)(813.01045898,361.22397308)
\curveto(812.99045659,361.21397317)(812.96545661,361.21397317)(812.93545898,361.22397308)
\lineto(812.69545898,361.28397308)
\curveto(812.61545696,361.29397309)(812.54045704,361.31397307)(812.47045898,361.34397308)
\curveto(812.17045741,361.47397291)(811.92545765,361.61897276)(811.73545898,361.77897308)
\curveto(811.55545802,361.94897243)(811.40545817,362.1839722)(811.28545898,362.48397308)
\curveto(811.19545838,362.70397168)(811.15045843,362.96897141)(811.15045898,363.27897308)
\lineto(811.15045898,363.59397308)
\curveto(811.16045842,363.64397074)(811.16545841,363.69397069)(811.16545898,363.74397308)
\lineto(811.19545898,363.92397308)
\lineto(811.31545898,364.25397308)
\curveto(811.35545822,364.36397002)(811.40545817,364.46396992)(811.46545898,364.55397308)
\curveto(811.64545793,364.84396954)(811.89045769,365.05896932)(812.20045898,365.19897308)
\curveto(812.51045707,365.33896904)(812.85045673,365.46396892)(813.22045898,365.57397308)
\curveto(813.36045622,365.61396877)(813.50545607,365.64396874)(813.65545898,365.66397308)
\curveto(813.80545577,365.6839687)(813.95545562,365.70896867)(814.10545898,365.73897308)
\curveto(814.1754554,365.75896862)(814.24045534,365.76896861)(814.30045898,365.76897308)
\curveto(814.37045521,365.76896861)(814.44545513,365.7789686)(814.52545898,365.79897308)
\curveto(814.59545498,365.81896856)(814.66545491,365.82896855)(814.73545898,365.82897308)
\curveto(814.80545477,365.83896854)(814.8804547,365.85396853)(814.96045898,365.87397308)
\curveto(815.21045437,365.93396845)(815.44545413,365.9839684)(815.66545898,366.02397308)
\curveto(815.88545369,366.07396831)(816.06045352,366.18896819)(816.19045898,366.36897308)
\curveto(816.25045333,366.44896793)(816.30045328,366.54896783)(816.34045898,366.66897308)
\curveto(816.3804532,366.79896758)(816.3804532,366.93896744)(816.34045898,367.08897308)
\curveto(816.2804533,367.32896705)(816.19045339,367.51896686)(816.07045898,367.65897308)
\curveto(815.96045362,367.79896658)(815.80045378,367.90896647)(815.59045898,367.98897308)
\curveto(815.47045411,368.03896634)(815.32545425,368.07396631)(815.15545898,368.09397308)
\curveto(814.99545458,368.11396627)(814.82545475,368.12396626)(814.64545898,368.12397308)
\curveto(814.46545511,368.12396626)(814.29045529,368.11396627)(814.12045898,368.09397308)
\curveto(813.95045563,368.07396631)(813.80545577,368.04396634)(813.68545898,368.00397308)
\curveto(813.51545606,367.94396644)(813.35045623,367.85896652)(813.19045898,367.74897308)
\curveto(813.11045647,367.68896669)(813.03545654,367.60896677)(812.96545898,367.50897308)
\curveto(812.90545667,367.41896696)(812.85045673,367.31896706)(812.80045898,367.20897308)
\curveto(812.77045681,367.12896725)(812.74045684,367.04396734)(812.71045898,366.95397308)
\curveto(812.69045689,366.86396752)(812.64545693,366.79396759)(812.57545898,366.74397308)
\curveto(812.53545704,366.71396767)(812.46545711,366.68896769)(812.36545898,366.66897308)
\curveto(812.2754573,366.65896772)(812.1804574,366.65396773)(812.08045898,366.65397308)
\curveto(811.9804576,366.65396773)(811.8804577,366.65896772)(811.78045898,366.66897308)
\curveto(811.69045789,366.68896769)(811.62545795,366.71396767)(811.58545898,366.74397308)
\curveto(811.54545803,366.77396761)(811.51545806,366.82396756)(811.49545898,366.89397308)
\curveto(811.4754581,366.96396742)(811.4754581,367.03896734)(811.49545898,367.11897308)
\curveto(811.52545805,367.24896713)(811.55545802,367.36896701)(811.58545898,367.47897308)
\curveto(811.62545795,367.59896678)(811.67045791,367.71396667)(811.72045898,367.82397308)
\curveto(811.91045767,368.17396621)(812.15045743,368.44396594)(812.44045898,368.63397308)
\curveto(812.73045685,368.83396555)(813.09045649,368.99396539)(813.52045898,369.11397308)
\curveto(813.62045596,369.13396525)(813.72045586,369.14896523)(813.82045898,369.15897308)
\curveto(813.93045565,369.16896521)(814.04045554,369.1839652)(814.15045898,369.20397308)
\curveto(814.19045539,369.21396517)(814.25545532,369.21396517)(814.34545898,369.20397308)
\curveto(814.43545514,369.20396518)(814.49045509,369.21396517)(814.51045898,369.23397308)
\curveto(815.21045437,369.24396514)(815.82045376,369.16396522)(816.34045898,368.99397308)
\curveto(816.86045272,368.82396556)(817.22545235,368.49896588)(817.43545898,368.01897308)
\curveto(817.52545205,367.81896656)(817.575452,367.5839668)(817.58545898,367.31397308)
\curveto(817.60545197,367.05396733)(817.61545196,366.7789676)(817.61545898,366.48897308)
\lineto(817.61545898,363.17397308)
\curveto(817.61545196,363.03397135)(817.62045196,362.89897148)(817.63045898,362.76897308)
\curveto(817.64045194,362.63897174)(817.67045191,362.53397185)(817.72045898,362.45397308)
\curveto(817.77045181,362.383972)(817.83545174,362.33397205)(817.91545898,362.30397308)
\curveto(818.00545157,362.26397212)(818.09045149,362.23397215)(818.17045898,362.21397308)
\curveto(818.25045133,362.20397218)(818.31045127,362.15897222)(818.35045898,362.07897308)
\curveto(818.37045121,362.04897233)(818.3804512,362.01897236)(818.38045898,361.98897308)
\curveto(818.3804512,361.95897242)(818.38545119,361.91897246)(818.39545898,361.86897308)
\moveto(816.25045898,363.53397308)
\curveto(816.31045327,363.67397071)(816.34045324,363.83397055)(816.34045898,364.01397308)
\curveto(816.35045323,364.20397018)(816.35545322,364.39896998)(816.35545898,364.59897308)
\curveto(816.35545322,364.70896967)(816.35045323,364.80896957)(816.34045898,364.89897308)
\curveto(816.33045325,364.98896939)(816.29045329,365.05896932)(816.22045898,365.10897308)
\curveto(816.19045339,365.12896925)(816.12045346,365.13896924)(816.01045898,365.13897308)
\curveto(815.99045359,365.11896926)(815.95545362,365.10896927)(815.90545898,365.10897308)
\curveto(815.85545372,365.10896927)(815.81045377,365.09896928)(815.77045898,365.07897308)
\curveto(815.69045389,365.05896932)(815.60045398,365.03896934)(815.50045898,365.01897308)
\lineto(815.20045898,364.95897308)
\curveto(815.17045441,364.95896942)(815.13545444,364.95396943)(815.09545898,364.94397308)
\lineto(814.99045898,364.94397308)
\curveto(814.84045474,364.90396948)(814.6754549,364.8789695)(814.49545898,364.86897308)
\curveto(814.32545525,364.86896951)(814.16545541,364.84896953)(814.01545898,364.80897308)
\curveto(813.93545564,364.78896959)(813.86045572,364.76896961)(813.79045898,364.74897308)
\curveto(813.73045585,364.73896964)(813.66045592,364.72396966)(813.58045898,364.70397308)
\curveto(813.42045616,364.65396973)(813.27045631,364.58896979)(813.13045898,364.50897308)
\curveto(812.99045659,364.43896994)(812.87045671,364.34897003)(812.77045898,364.23897308)
\curveto(812.67045691,364.12897025)(812.59545698,363.99397039)(812.54545898,363.83397308)
\curveto(812.49545708,363.6839707)(812.4754571,363.49897088)(812.48545898,363.27897308)
\curveto(812.48545709,363.1789712)(812.50045708,363.0839713)(812.53045898,362.99397308)
\curveto(812.57045701,362.91397147)(812.61545696,362.83897154)(812.66545898,362.76897308)
\curveto(812.74545683,362.65897172)(812.85045673,362.56397182)(812.98045898,362.48397308)
\curveto(813.11045647,362.41397197)(813.25045633,362.35397203)(813.40045898,362.30397308)
\curveto(813.45045613,362.29397209)(813.50045608,362.28897209)(813.55045898,362.28897308)
\curveto(813.60045598,362.28897209)(813.65045593,362.2839721)(813.70045898,362.27397308)
\curveto(813.77045581,362.25397213)(813.85545572,362.23897214)(813.95545898,362.22897308)
\curveto(814.06545551,362.22897215)(814.15545542,362.23897214)(814.22545898,362.25897308)
\curveto(814.28545529,362.2789721)(814.34545523,362.2839721)(814.40545898,362.27397308)
\curveto(814.46545511,362.27397211)(814.52545505,362.2839721)(814.58545898,362.30397308)
\curveto(814.66545491,362.32397206)(814.74045484,362.33897204)(814.81045898,362.34897308)
\curveto(814.89045469,362.35897202)(814.96545461,362.378972)(815.03545898,362.40897308)
\curveto(815.32545425,362.52897185)(815.57045401,362.67397171)(815.77045898,362.84397308)
\curveto(815.9804536,363.01397137)(816.14045344,363.24397114)(816.25045898,363.53397308)
}
}
{
\newrgbcolor{curcolor}{0 0 0}
\pscustom[linestyle=none,fillstyle=solid,fillcolor=curcolor]
{
\newpath
\moveto(826.52709961,362.12397308)
\lineto(826.52709961,361.73397308)
\curveto(826.52709173,361.61397277)(826.50209176,361.51397287)(826.45209961,361.43397308)
\curveto(826.40209186,361.36397302)(826.31709194,361.32397306)(826.19709961,361.31397308)
\lineto(825.85209961,361.31397308)
\curveto(825.79209247,361.31397307)(825.73209253,361.30897307)(825.67209961,361.29897308)
\curveto(825.62209264,361.29897308)(825.57709268,361.30897307)(825.53709961,361.32897308)
\curveto(825.44709281,361.34897303)(825.38709287,361.38897299)(825.35709961,361.44897308)
\curveto(825.31709294,361.49897288)(825.29209297,361.55897282)(825.28209961,361.62897308)
\curveto(825.28209298,361.69897268)(825.26709299,361.76897261)(825.23709961,361.83897308)
\curveto(825.22709303,361.85897252)(825.21209305,361.87397251)(825.19209961,361.88397308)
\curveto(825.18209308,361.90397248)(825.16709309,361.92397246)(825.14709961,361.94397308)
\curveto(825.04709321,361.95397243)(824.96709329,361.93397245)(824.90709961,361.88397308)
\curveto(824.8570934,361.83397255)(824.80209346,361.7839726)(824.74209961,361.73397308)
\curveto(824.54209372,361.5839728)(824.34209392,361.46897291)(824.14209961,361.38897308)
\curveto(823.9620943,361.30897307)(823.75209451,361.24897313)(823.51209961,361.20897308)
\curveto(823.28209498,361.16897321)(823.04209522,361.14897323)(822.79209961,361.14897308)
\curveto(822.55209571,361.13897324)(822.31209595,361.15397323)(822.07209961,361.19397308)
\curveto(821.83209643,361.22397316)(821.62209664,361.2789731)(821.44209961,361.35897308)
\curveto(820.92209734,361.5789728)(820.50209776,361.87397251)(820.18209961,362.24397308)
\curveto(819.8620984,362.62397176)(819.61209865,363.09397129)(819.43209961,363.65397308)
\curveto(819.39209887,363.74397064)(819.3620989,363.83397055)(819.34209961,363.92397308)
\curveto(819.33209893,364.02397036)(819.31209895,364.12397026)(819.28209961,364.22397308)
\curveto(819.27209899,364.27397011)(819.26709899,364.32397006)(819.26709961,364.37397308)
\curveto(819.26709899,364.42396996)(819.262099,364.47396991)(819.25209961,364.52397308)
\curveto(819.23209903,364.57396981)(819.22209904,364.62396976)(819.22209961,364.67397308)
\curveto(819.23209903,364.73396965)(819.23209903,364.78896959)(819.22209961,364.83897308)
\lineto(819.22209961,364.98897308)
\curveto(819.20209906,365.03896934)(819.19209907,365.10396928)(819.19209961,365.18397308)
\curveto(819.19209907,365.26396912)(819.20209906,365.32896905)(819.22209961,365.37897308)
\lineto(819.22209961,365.54397308)
\curveto(819.24209902,365.61396877)(819.24709901,365.6839687)(819.23709961,365.75397308)
\curveto(819.23709902,365.83396855)(819.24709901,365.90896847)(819.26709961,365.97897308)
\curveto(819.27709898,366.02896835)(819.28209898,366.07396831)(819.28209961,366.11397308)
\curveto(819.28209898,366.15396823)(819.28709897,366.19896818)(819.29709961,366.24897308)
\curveto(819.32709893,366.34896803)(819.35209891,366.44396794)(819.37209961,366.53397308)
\curveto(819.39209887,366.63396775)(819.41709884,366.72896765)(819.44709961,366.81897308)
\curveto(819.57709868,367.19896718)(819.74209852,367.53896684)(819.94209961,367.83897308)
\curveto(820.15209811,368.14896623)(820.40209786,368.40396598)(820.69209961,368.60397308)
\curveto(820.8620974,368.72396566)(821.03709722,368.82396556)(821.21709961,368.90397308)
\curveto(821.40709685,368.9839654)(821.61209665,369.05396533)(821.83209961,369.11397308)
\curveto(821.90209636,369.12396526)(821.96709629,369.13396525)(822.02709961,369.14397308)
\curveto(822.09709616,369.15396523)(822.16709609,369.16896521)(822.23709961,369.18897308)
\lineto(822.38709961,369.18897308)
\curveto(822.46709579,369.20896517)(822.58209568,369.21896516)(822.73209961,369.21897308)
\curveto(822.89209537,369.21896516)(823.01209525,369.20896517)(823.09209961,369.18897308)
\curveto(823.13209513,369.1789652)(823.18709507,369.17396521)(823.25709961,369.17397308)
\curveto(823.36709489,369.14396524)(823.47709478,369.11896526)(823.58709961,369.09897308)
\curveto(823.69709456,369.08896529)(823.80209446,369.05896532)(823.90209961,369.00897308)
\curveto(824.05209421,368.94896543)(824.19209407,368.8839655)(824.32209961,368.81397308)
\curveto(824.4620938,368.74396564)(824.59209367,368.66396572)(824.71209961,368.57397308)
\curveto(824.77209349,368.52396586)(824.83209343,368.46896591)(824.89209961,368.40897308)
\curveto(824.9620933,368.35896602)(825.05209321,368.34396604)(825.16209961,368.36397308)
\curveto(825.18209308,368.39396599)(825.19709306,368.41896596)(825.20709961,368.43897308)
\curveto(825.22709303,368.45896592)(825.24209302,368.48896589)(825.25209961,368.52897308)
\curveto(825.28209298,368.61896576)(825.29209297,368.73396565)(825.28209961,368.87397308)
\lineto(825.28209961,369.24897308)
\lineto(825.28209961,370.97397308)
\lineto(825.28209961,371.43897308)
\curveto(825.28209298,371.61896276)(825.30709295,371.74896263)(825.35709961,371.82897308)
\curveto(825.39709286,371.89896248)(825.4570928,371.94396244)(825.53709961,371.96397308)
\curveto(825.5570927,371.96396242)(825.58209268,371.96396242)(825.61209961,371.96397308)
\curveto(825.64209262,371.97396241)(825.66709259,371.9789624)(825.68709961,371.97897308)
\curveto(825.82709243,371.98896239)(825.97209229,371.98896239)(826.12209961,371.97897308)
\curveto(826.28209198,371.9789624)(826.39209187,371.93896244)(826.45209961,371.85897308)
\curveto(826.50209176,371.7789626)(826.52709173,371.6789627)(826.52709961,371.55897308)
\lineto(826.52709961,371.18397308)
\lineto(826.52709961,362.12397308)
\moveto(825.31209961,364.95897308)
\curveto(825.33209293,365.00896937)(825.34209292,365.07396931)(825.34209961,365.15397308)
\curveto(825.34209292,365.24396914)(825.33209293,365.31396907)(825.31209961,365.36397308)
\lineto(825.31209961,365.58897308)
\curveto(825.29209297,365.6789687)(825.27709298,365.76896861)(825.26709961,365.85897308)
\curveto(825.257093,365.95896842)(825.23709302,366.04896833)(825.20709961,366.12897308)
\curveto(825.18709307,366.20896817)(825.16709309,366.2839681)(825.14709961,366.35397308)
\curveto(825.13709312,366.42396796)(825.11709314,366.49396789)(825.08709961,366.56397308)
\curveto(824.96709329,366.86396752)(824.81209345,367.12896725)(824.62209961,367.35897308)
\curveto(824.43209383,367.58896679)(824.19209407,367.76896661)(823.90209961,367.89897308)
\curveto(823.80209446,367.94896643)(823.69709456,367.9839664)(823.58709961,368.00397308)
\curveto(823.48709477,368.03396635)(823.37709488,368.05896632)(823.25709961,368.07897308)
\curveto(823.17709508,368.09896628)(823.08709517,368.10896627)(822.98709961,368.10897308)
\lineto(822.71709961,368.10897308)
\curveto(822.66709559,368.09896628)(822.62209564,368.08896629)(822.58209961,368.07897308)
\lineto(822.44709961,368.07897308)
\curveto(822.36709589,368.05896632)(822.28209598,368.03896634)(822.19209961,368.01897308)
\curveto(822.11209615,367.99896638)(822.03209623,367.97396641)(821.95209961,367.94397308)
\curveto(821.63209663,367.80396658)(821.37209689,367.59896678)(821.17209961,367.32897308)
\curveto(820.98209728,367.06896731)(820.82709743,366.76396762)(820.70709961,366.41397308)
\curveto(820.66709759,366.30396808)(820.63709762,366.18896819)(820.61709961,366.06897308)
\curveto(820.60709765,365.95896842)(820.59209767,365.84896853)(820.57209961,365.73897308)
\curveto(820.57209769,365.69896868)(820.56709769,365.65896872)(820.55709961,365.61897308)
\lineto(820.55709961,365.51397308)
\curveto(820.53709772,365.46396892)(820.52709773,365.40896897)(820.52709961,365.34897308)
\curveto(820.53709772,365.28896909)(820.54209772,365.23396915)(820.54209961,365.18397308)
\lineto(820.54209961,364.85397308)
\curveto(820.54209772,364.75396963)(820.55209771,364.65896972)(820.57209961,364.56897308)
\curveto(820.58209768,364.53896984)(820.58709767,364.48896989)(820.58709961,364.41897308)
\curveto(820.60709765,364.34897003)(820.62209764,364.2789701)(820.63209961,364.20897308)
\lineto(820.69209961,363.99897308)
\curveto(820.80209746,363.64897073)(820.95209731,363.34897103)(821.14209961,363.09897308)
\curveto(821.33209693,362.84897153)(821.57209669,362.64397174)(821.86209961,362.48397308)
\curveto(821.95209631,362.43397195)(822.04209622,362.39397199)(822.13209961,362.36397308)
\curveto(822.22209604,362.33397205)(822.32209594,362.30397208)(822.43209961,362.27397308)
\curveto(822.48209578,362.25397213)(822.53209573,362.24897213)(822.58209961,362.25897308)
\curveto(822.64209562,362.26897211)(822.69709556,362.26397212)(822.74709961,362.24397308)
\curveto(822.78709547,362.23397215)(822.82709543,362.22897215)(822.86709961,362.22897308)
\lineto(823.00209961,362.22897308)
\lineto(823.13709961,362.22897308)
\curveto(823.16709509,362.23897214)(823.21709504,362.24397214)(823.28709961,362.24397308)
\curveto(823.36709489,362.26397212)(823.44709481,362.2789721)(823.52709961,362.28897308)
\curveto(823.60709465,362.30897207)(823.68209458,362.33397205)(823.75209961,362.36397308)
\curveto(824.08209418,362.50397188)(824.34709391,362.6789717)(824.54709961,362.88897308)
\curveto(824.7570935,363.10897127)(824.93209333,363.383971)(825.07209961,363.71397308)
\curveto(825.12209314,363.82397056)(825.1570931,363.93397045)(825.17709961,364.04397308)
\curveto(825.19709306,364.15397023)(825.22209304,364.26397012)(825.25209961,364.37397308)
\curveto(825.27209299,364.41396997)(825.28209298,364.44896993)(825.28209961,364.47897308)
\curveto(825.28209298,364.51896986)(825.28709297,364.55896982)(825.29709961,364.59897308)
\curveto(825.30709295,364.65896972)(825.30709295,364.71896966)(825.29709961,364.77897308)
\curveto(825.29709296,364.83896954)(825.30209296,364.89896948)(825.31209961,364.95897308)
}
}
{
\newrgbcolor{curcolor}{0 0 0}
\pscustom[linestyle=none,fillstyle=solid,fillcolor=curcolor]
{
\newpath
\moveto(775.88144531,338.77765076)
\curveto(775.90143577,338.72765001)(775.92643574,338.66765007)(775.95644531,338.59765076)
\curveto(775.98643568,338.52765021)(776.00643566,338.45265029)(776.01644531,338.37265076)
\curveto(776.03643563,338.30265044)(776.03643563,338.23265051)(776.01644531,338.16265076)
\curveto(776.00643566,338.10265064)(775.9664357,338.05765068)(775.89644531,338.02765076)
\curveto(775.84643582,338.00765073)(775.78643588,337.99765074)(775.71644531,337.99765076)
\lineto(775.50644531,337.99765076)
\lineto(775.05644531,337.99765076)
\curveto(774.90643676,337.99765074)(774.78643688,338.02265072)(774.69644531,338.07265076)
\curveto(774.59643707,338.13265061)(774.52143715,338.2376505)(774.47144531,338.38765076)
\curveto(774.43143724,338.5376502)(774.38643728,338.67265007)(774.33644531,338.79265076)
\curveto(774.22643744,339.05264969)(774.12643754,339.32264942)(774.03644531,339.60265076)
\curveto(773.94643772,339.88264886)(773.84643782,340.15764858)(773.73644531,340.42765076)
\curveto(773.70643796,340.51764822)(773.67643799,340.60264814)(773.64644531,340.68265076)
\curveto(773.62643804,340.76264798)(773.59643807,340.8376479)(773.55644531,340.90765076)
\curveto(773.52643814,340.97764776)(773.48143819,341.0376477)(773.42144531,341.08765076)
\curveto(773.36143831,341.1376476)(773.28143839,341.17764756)(773.18144531,341.20765076)
\curveto(773.13143854,341.22764751)(773.0714386,341.23264751)(773.00144531,341.22265076)
\lineto(772.80644531,341.22265076)
\lineto(769.97144531,341.22265076)
\lineto(769.67144531,341.22265076)
\curveto(769.56144211,341.23264751)(769.45644221,341.23264751)(769.35644531,341.22265076)
\curveto(769.25644241,341.21264753)(769.16144251,341.19764754)(769.07144531,341.17765076)
\curveto(768.99144268,341.15764758)(768.93144274,341.11764762)(768.89144531,341.05765076)
\curveto(768.81144286,340.95764778)(768.75144292,340.8426479)(768.71144531,340.71265076)
\curveto(768.68144299,340.59264815)(768.64144303,340.46764827)(768.59144531,340.33765076)
\curveto(768.49144318,340.10764863)(768.39644327,339.86764887)(768.30644531,339.61765076)
\curveto(768.22644344,339.36764937)(768.13644353,339.12764961)(768.03644531,338.89765076)
\curveto(768.01644365,338.8376499)(767.99144368,338.76764997)(767.96144531,338.68765076)
\curveto(767.94144373,338.61765012)(767.91644375,338.5426502)(767.88644531,338.46265076)
\curveto(767.85644381,338.38265036)(767.82144385,338.30765043)(767.78144531,338.23765076)
\curveto(767.75144392,338.17765056)(767.71644395,338.13265061)(767.67644531,338.10265076)
\curveto(767.59644407,338.0426507)(767.48644418,338.00765073)(767.34644531,337.99765076)
\lineto(766.92644531,337.99765076)
\lineto(766.68644531,337.99765076)
\curveto(766.61644505,338.00765073)(766.55644511,338.03265071)(766.50644531,338.07265076)
\curveto(766.45644521,338.10265064)(766.42644524,338.14765059)(766.41644531,338.20765076)
\curveto(766.41644525,338.26765047)(766.42144525,338.32765041)(766.43144531,338.38765076)
\curveto(766.45144522,338.45765028)(766.4714452,338.52265022)(766.49144531,338.58265076)
\curveto(766.52144515,338.65265009)(766.54644512,338.70265004)(766.56644531,338.73265076)
\curveto(766.70644496,339.05264969)(766.83144484,339.36764937)(766.94144531,339.67765076)
\curveto(767.05144462,339.99764874)(767.1714445,340.31764842)(767.30144531,340.63765076)
\curveto(767.39144428,340.85764788)(767.47644419,341.07264767)(767.55644531,341.28265076)
\curveto(767.63644403,341.50264724)(767.72144395,341.72264702)(767.81144531,341.94265076)
\curveto(768.11144356,342.66264608)(768.39644327,343.38764535)(768.66644531,344.11765076)
\curveto(768.93644273,344.85764388)(769.22144245,345.59264315)(769.52144531,346.32265076)
\curveto(769.63144204,346.58264216)(769.73144194,346.84764189)(769.82144531,347.11765076)
\curveto(769.92144175,347.38764135)(770.02644164,347.65264109)(770.13644531,347.91265076)
\curveto(770.18644148,348.02264072)(770.23144144,348.1426406)(770.27144531,348.27265076)
\curveto(770.32144135,348.41264033)(770.39144128,348.51264023)(770.48144531,348.57265076)
\curveto(770.52144115,348.61264013)(770.58644108,348.6426401)(770.67644531,348.66265076)
\curveto(770.69644097,348.67264007)(770.71644095,348.67264007)(770.73644531,348.66265076)
\curveto(770.7664409,348.66264008)(770.79144088,348.66764007)(770.81144531,348.67765076)
\curveto(770.99144068,348.67764006)(771.20144047,348.67764006)(771.44144531,348.67765076)
\curveto(771.68143999,348.68764005)(771.85643981,348.65264009)(771.96644531,348.57265076)
\curveto(772.04643962,348.51264023)(772.10643956,348.41264033)(772.14644531,348.27265076)
\curveto(772.19643947,348.1426406)(772.24643942,348.02264072)(772.29644531,347.91265076)
\curveto(772.39643927,347.68264106)(772.48643918,347.45264129)(772.56644531,347.22265076)
\curveto(772.64643902,346.99264175)(772.73643893,346.76264198)(772.83644531,346.53265076)
\curveto(772.91643875,346.33264241)(772.99143868,346.12764261)(773.06144531,345.91765076)
\curveto(773.14143853,345.70764303)(773.22643844,345.50264324)(773.31644531,345.30265076)
\curveto(773.61643805,344.57264417)(773.90143777,343.83264491)(774.17144531,343.08265076)
\curveto(774.45143722,342.3426464)(774.74643692,341.60764713)(775.05644531,340.87765076)
\curveto(775.09643657,340.78764795)(775.12643654,340.70264804)(775.14644531,340.62265076)
\curveto(775.17643649,340.5426482)(775.20643646,340.45764828)(775.23644531,340.36765076)
\curveto(775.34643632,340.10764863)(775.45143622,339.8426489)(775.55144531,339.57265076)
\curveto(775.66143601,339.30264944)(775.7714359,339.0376497)(775.88144531,338.77765076)
\moveto(772.67144531,342.42265076)
\curveto(772.76143891,342.45264629)(772.81643885,342.50264624)(772.83644531,342.57265076)
\curveto(772.8664388,342.6426461)(772.8714388,342.71764602)(772.85144531,342.79765076)
\curveto(772.84143883,342.88764585)(772.81643885,342.97264577)(772.77644531,343.05265076)
\curveto(772.74643892,343.1426456)(772.71643895,343.21764552)(772.68644531,343.27765076)
\curveto(772.666439,343.31764542)(772.65643901,343.35264539)(772.65644531,343.38265076)
\curveto(772.65643901,343.41264533)(772.64643902,343.44764529)(772.62644531,343.48765076)
\lineto(772.53644531,343.72765076)
\curveto(772.51643915,343.81764492)(772.48643918,343.90764483)(772.44644531,343.99765076)
\curveto(772.29643937,344.35764438)(772.16143951,344.72264402)(772.04144531,345.09265076)
\curveto(771.93143974,345.47264327)(771.80143987,345.8426429)(771.65144531,346.20265076)
\curveto(771.60144007,346.31264243)(771.55644011,346.42264232)(771.51644531,346.53265076)
\curveto(771.48644018,346.6426421)(771.44644022,346.74764199)(771.39644531,346.84765076)
\curveto(771.37644029,346.89764184)(771.35144032,346.9426418)(771.32144531,346.98265076)
\curveto(771.30144037,347.03264171)(771.25144042,347.05764168)(771.17144531,347.05765076)
\curveto(771.15144052,347.0376417)(771.13144054,347.02264172)(771.11144531,347.01265076)
\curveto(771.09144058,347.00264174)(771.0714406,346.98764175)(771.05144531,346.96765076)
\curveto(771.01144066,346.91764182)(770.98144069,346.86264188)(770.96144531,346.80265076)
\curveto(770.94144073,346.75264199)(770.92144075,346.69764204)(770.90144531,346.63765076)
\curveto(770.85144082,346.52764221)(770.81144086,346.41764232)(770.78144531,346.30765076)
\curveto(770.75144092,346.19764254)(770.71144096,346.08764265)(770.66144531,345.97765076)
\curveto(770.49144118,345.58764315)(770.34144133,345.19264355)(770.21144531,344.79265076)
\curveto(770.09144158,344.39264435)(769.95144172,344.00264474)(769.79144531,343.62265076)
\lineto(769.73144531,343.47265076)
\curveto(769.72144195,343.42264532)(769.70644196,343.37264537)(769.68644531,343.32265076)
\lineto(769.59644531,343.08265076)
\curveto(769.5664421,343.00264574)(769.54144213,342.92264582)(769.52144531,342.84265076)
\curveto(769.50144217,342.79264595)(769.49144218,342.737646)(769.49144531,342.67765076)
\curveto(769.50144217,342.61764612)(769.51644215,342.56764617)(769.53644531,342.52765076)
\curveto(769.58644208,342.44764629)(769.69144198,342.40264634)(769.85144531,342.39265076)
\lineto(770.30144531,342.39265076)
\lineto(771.90644531,342.39265076)
\curveto(772.01643965,342.39264635)(772.15143952,342.38764635)(772.31144531,342.37765076)
\curveto(772.4714392,342.37764636)(772.59143908,342.39264635)(772.67144531,342.42265076)
}
}
{
\newrgbcolor{curcolor}{0 0 0}
\pscustom[linestyle=none,fillstyle=solid,fillcolor=curcolor]
{
\newpath
\moveto(777.46300781,345.72265076)
\lineto(777.89800781,345.72265076)
\curveto(778.04800585,345.72264302)(778.15300574,345.68264306)(778.21300781,345.60265076)
\curveto(778.26300563,345.52264322)(778.28800561,345.42264332)(778.28800781,345.30265076)
\curveto(778.2980056,345.18264356)(778.30300559,345.06264368)(778.30300781,344.94265076)
\lineto(778.30300781,343.51765076)
\lineto(778.30300781,341.25265076)
\lineto(778.30300781,340.56265076)
\curveto(778.30300559,340.33264841)(778.32800557,340.13264861)(778.37800781,339.96265076)
\curveto(778.53800536,339.51264923)(778.83800506,339.19764954)(779.27800781,339.01765076)
\curveto(779.4980044,338.92764981)(779.76300413,338.89264985)(780.07300781,338.91265076)
\curveto(780.38300351,338.9426498)(780.63300326,338.99764974)(780.82300781,339.07765076)
\curveto(781.15300274,339.21764952)(781.41300248,339.39264935)(781.60300781,339.60265076)
\curveto(781.80300209,339.82264892)(781.95800194,340.10764863)(782.06800781,340.45765076)
\curveto(782.0980018,340.5376482)(782.11800178,340.61764812)(782.12800781,340.69765076)
\curveto(782.13800176,340.77764796)(782.15300174,340.86264788)(782.17300781,340.95265076)
\curveto(782.18300171,341.00264774)(782.18300171,341.04764769)(782.17300781,341.08765076)
\curveto(782.17300172,341.12764761)(782.18300171,341.17264757)(782.20300781,341.22265076)
\lineto(782.20300781,341.53765076)
\curveto(782.22300167,341.61764712)(782.22800167,341.70764703)(782.21800781,341.80765076)
\curveto(782.20800169,341.91764682)(782.20300169,342.01764672)(782.20300781,342.10765076)
\lineto(782.20300781,343.27765076)
\lineto(782.20300781,344.86765076)
\curveto(782.20300169,344.98764375)(782.1980017,345.11264363)(782.18800781,345.24265076)
\curveto(782.18800171,345.38264336)(782.21300168,345.49264325)(782.26300781,345.57265076)
\curveto(782.30300159,345.62264312)(782.34800155,345.65264309)(782.39800781,345.66265076)
\curveto(782.45800144,345.68264306)(782.52800137,345.70264304)(782.60800781,345.72265076)
\lineto(782.83300781,345.72265076)
\curveto(782.95300094,345.72264302)(783.05800084,345.71764302)(783.14800781,345.70765076)
\curveto(783.24800065,345.69764304)(783.32300057,345.65264309)(783.37300781,345.57265076)
\curveto(783.42300047,345.52264322)(783.44800045,345.44764329)(783.44800781,345.34765076)
\lineto(783.44800781,345.06265076)
\lineto(783.44800781,344.04265076)
\lineto(783.44800781,340.00765076)
\lineto(783.44800781,338.65765076)
\curveto(783.44800045,338.5376502)(783.44300045,338.42265032)(783.43300781,338.31265076)
\curveto(783.43300046,338.21265053)(783.3980005,338.1376506)(783.32800781,338.08765076)
\curveto(783.28800061,338.05765068)(783.22800067,338.03265071)(783.14800781,338.01265076)
\curveto(783.06800083,338.00265074)(782.97800092,337.99265075)(782.87800781,337.98265076)
\curveto(782.78800111,337.98265076)(782.6980012,337.98765075)(782.60800781,337.99765076)
\curveto(782.52800137,338.00765073)(782.46800143,338.02765071)(782.42800781,338.05765076)
\curveto(782.37800152,338.09765064)(782.33300156,338.16265058)(782.29300781,338.25265076)
\curveto(782.28300161,338.29265045)(782.27300162,338.34765039)(782.26300781,338.41765076)
\curveto(782.26300163,338.48765025)(782.25800164,338.55265019)(782.24800781,338.61265076)
\curveto(782.23800166,338.68265006)(782.21800168,338.73765)(782.18800781,338.77765076)
\curveto(782.15800174,338.81764992)(782.11300178,338.83264991)(782.05300781,338.82265076)
\curveto(781.97300192,338.80264994)(781.893002,338.74265)(781.81300781,338.64265076)
\curveto(781.73300216,338.55265019)(781.65800224,338.48265026)(781.58800781,338.43265076)
\curveto(781.36800253,338.27265047)(781.11800278,338.13265061)(780.83800781,338.01265076)
\curveto(780.72800317,337.96265078)(780.61300328,337.93265081)(780.49300781,337.92265076)
\curveto(780.38300351,337.90265084)(780.26800363,337.87765086)(780.14800781,337.84765076)
\curveto(780.0980038,337.8376509)(780.04300385,337.8376509)(779.98300781,337.84765076)
\curveto(779.93300396,337.85765088)(779.88300401,337.85265089)(779.83300781,337.83265076)
\curveto(779.73300416,337.81265093)(779.64300425,337.81265093)(779.56300781,337.83265076)
\lineto(779.41300781,337.83265076)
\curveto(779.36300453,337.85265089)(779.30300459,337.86265088)(779.23300781,337.86265076)
\curveto(779.17300472,337.86265088)(779.11800478,337.86765087)(779.06800781,337.87765076)
\curveto(779.02800487,337.89765084)(778.98800491,337.90765083)(778.94800781,337.90765076)
\curveto(778.91800498,337.89765084)(778.87800502,337.90265084)(778.82800781,337.92265076)
\lineto(778.58800781,337.98265076)
\curveto(778.51800538,338.00265074)(778.44300545,338.03265071)(778.36300781,338.07265076)
\curveto(778.10300579,338.18265056)(777.88300601,338.32765041)(777.70300781,338.50765076)
\curveto(777.53300636,338.69765004)(777.3930065,338.92264982)(777.28300781,339.18265076)
\curveto(777.24300665,339.27264947)(777.21300668,339.36264938)(777.19300781,339.45265076)
\lineto(777.13300781,339.75265076)
\curveto(777.11300678,339.81264893)(777.10300679,339.86764887)(777.10300781,339.91765076)
\curveto(777.11300678,339.97764876)(777.10800679,340.0426487)(777.08800781,340.11265076)
\curveto(777.07800682,340.13264861)(777.07300682,340.15764858)(777.07300781,340.18765076)
\curveto(777.07300682,340.22764851)(777.06800683,340.26264848)(777.05800781,340.29265076)
\lineto(777.05800781,340.44265076)
\curveto(777.04800685,340.48264826)(777.04300685,340.52764821)(777.04300781,340.57765076)
\curveto(777.05300684,340.6376481)(777.05800684,340.69264805)(777.05800781,340.74265076)
\lineto(777.05800781,341.34265076)
\lineto(777.05800781,344.10265076)
\lineto(777.05800781,345.06265076)
\lineto(777.05800781,345.33265076)
\curveto(777.05800684,345.42264332)(777.07800682,345.49764324)(777.11800781,345.55765076)
\curveto(777.15800674,345.62764311)(777.23300666,345.67764306)(777.34300781,345.70765076)
\curveto(777.36300653,345.71764302)(777.38300651,345.71764302)(777.40300781,345.70765076)
\curveto(777.42300647,345.70764303)(777.44300645,345.71264303)(777.46300781,345.72265076)
}
}
{
\newrgbcolor{curcolor}{0 0 0}
\pscustom[linestyle=none,fillstyle=solid,fillcolor=curcolor]
{
\newpath
\moveto(792.30761719,338.80765076)
\lineto(792.30761719,338.41765076)
\curveto(792.30760931,338.29765044)(792.28260934,338.19765054)(792.23261719,338.11765076)
\curveto(792.18260944,338.04765069)(792.09760952,338.00765073)(791.97761719,337.99765076)
\lineto(791.63261719,337.99765076)
\curveto(791.57261005,337.99765074)(791.51261011,337.99265075)(791.45261719,337.98265076)
\curveto(791.40261022,337.98265076)(791.35761026,337.99265075)(791.31761719,338.01265076)
\curveto(791.22761039,338.03265071)(791.16761045,338.07265067)(791.13761719,338.13265076)
\curveto(791.09761052,338.18265056)(791.07261055,338.2426505)(791.06261719,338.31265076)
\curveto(791.06261056,338.38265036)(791.04761057,338.45265029)(791.01761719,338.52265076)
\curveto(791.00761061,338.5426502)(790.99261063,338.55765018)(790.97261719,338.56765076)
\curveto(790.96261066,338.58765015)(790.94761067,338.60765013)(790.92761719,338.62765076)
\curveto(790.82761079,338.6376501)(790.74761087,338.61765012)(790.68761719,338.56765076)
\curveto(790.63761098,338.51765022)(790.58261104,338.46765027)(790.52261719,338.41765076)
\curveto(790.3226113,338.26765047)(790.1226115,338.15265059)(789.92261719,338.07265076)
\curveto(789.74261188,337.99265075)(789.53261209,337.93265081)(789.29261719,337.89265076)
\curveto(789.06261256,337.85265089)(788.8226128,337.83265091)(788.57261719,337.83265076)
\curveto(788.33261329,337.82265092)(788.09261353,337.8376509)(787.85261719,337.87765076)
\curveto(787.61261401,337.90765083)(787.40261422,337.96265078)(787.22261719,338.04265076)
\curveto(786.70261492,338.26265048)(786.28261534,338.55765018)(785.96261719,338.92765076)
\curveto(785.64261598,339.30764943)(785.39261623,339.77764896)(785.21261719,340.33765076)
\curveto(785.17261645,340.42764831)(785.14261648,340.51764822)(785.12261719,340.60765076)
\curveto(785.11261651,340.70764803)(785.09261653,340.80764793)(785.06261719,340.90765076)
\curveto(785.05261657,340.95764778)(785.04761657,341.00764773)(785.04761719,341.05765076)
\curveto(785.04761657,341.10764763)(785.04261658,341.15764758)(785.03261719,341.20765076)
\curveto(785.01261661,341.25764748)(785.00261662,341.30764743)(785.00261719,341.35765076)
\curveto(785.01261661,341.41764732)(785.01261661,341.47264727)(785.00261719,341.52265076)
\lineto(785.00261719,341.67265076)
\curveto(784.98261664,341.72264702)(784.97261665,341.78764695)(784.97261719,341.86765076)
\curveto(784.97261665,341.94764679)(784.98261664,342.01264673)(785.00261719,342.06265076)
\lineto(785.00261719,342.22765076)
\curveto(785.0226166,342.29764644)(785.02761659,342.36764637)(785.01761719,342.43765076)
\curveto(785.0176166,342.51764622)(785.02761659,342.59264615)(785.04761719,342.66265076)
\curveto(785.05761656,342.71264603)(785.06261656,342.75764598)(785.06261719,342.79765076)
\curveto(785.06261656,342.8376459)(785.06761655,342.88264586)(785.07761719,342.93265076)
\curveto(785.10761651,343.03264571)(785.13261649,343.12764561)(785.15261719,343.21765076)
\curveto(785.17261645,343.31764542)(785.19761642,343.41264533)(785.22761719,343.50265076)
\curveto(785.35761626,343.88264486)(785.5226161,344.22264452)(785.72261719,344.52265076)
\curveto(785.93261569,344.83264391)(786.18261544,345.08764365)(786.47261719,345.28765076)
\curveto(786.64261498,345.40764333)(786.8176148,345.50764323)(786.99761719,345.58765076)
\curveto(787.18761443,345.66764307)(787.39261423,345.737643)(787.61261719,345.79765076)
\curveto(787.68261394,345.80764293)(787.74761387,345.81764292)(787.80761719,345.82765076)
\curveto(787.87761374,345.8376429)(787.94761367,345.85264289)(788.01761719,345.87265076)
\lineto(788.16761719,345.87265076)
\curveto(788.24761337,345.89264285)(788.36261326,345.90264284)(788.51261719,345.90265076)
\curveto(788.67261295,345.90264284)(788.79261283,345.89264285)(788.87261719,345.87265076)
\curveto(788.91261271,345.86264288)(788.96761265,345.85764288)(789.03761719,345.85765076)
\curveto(789.14761247,345.82764291)(789.25761236,345.80264294)(789.36761719,345.78265076)
\curveto(789.47761214,345.77264297)(789.58261204,345.742643)(789.68261719,345.69265076)
\curveto(789.83261179,345.63264311)(789.97261165,345.56764317)(790.10261719,345.49765076)
\curveto(790.24261138,345.42764331)(790.37261125,345.34764339)(790.49261719,345.25765076)
\curveto(790.55261107,345.20764353)(790.61261101,345.15264359)(790.67261719,345.09265076)
\curveto(790.74261088,345.0426437)(790.83261079,345.02764371)(790.94261719,345.04765076)
\curveto(790.96261066,345.07764366)(790.97761064,345.10264364)(790.98761719,345.12265076)
\curveto(791.00761061,345.1426436)(791.0226106,345.17264357)(791.03261719,345.21265076)
\curveto(791.06261056,345.30264344)(791.07261055,345.41764332)(791.06261719,345.55765076)
\lineto(791.06261719,345.93265076)
\lineto(791.06261719,347.65765076)
\lineto(791.06261719,348.12265076)
\curveto(791.06261056,348.30264044)(791.08761053,348.43264031)(791.13761719,348.51265076)
\curveto(791.17761044,348.58264016)(791.23761038,348.62764011)(791.31761719,348.64765076)
\curveto(791.33761028,348.64764009)(791.36261026,348.64764009)(791.39261719,348.64765076)
\curveto(791.4226102,348.65764008)(791.44761017,348.66264008)(791.46761719,348.66265076)
\curveto(791.60761001,348.67264007)(791.75260987,348.67264007)(791.90261719,348.66265076)
\curveto(792.06260956,348.66264008)(792.17260945,348.62264012)(792.23261719,348.54265076)
\curveto(792.28260934,348.46264028)(792.30760931,348.36264038)(792.30761719,348.24265076)
\lineto(792.30761719,347.86765076)
\lineto(792.30761719,338.80765076)
\moveto(791.09261719,341.64265076)
\curveto(791.11261051,341.69264705)(791.1226105,341.75764698)(791.12261719,341.83765076)
\curveto(791.1226105,341.92764681)(791.11261051,341.99764674)(791.09261719,342.04765076)
\lineto(791.09261719,342.27265076)
\curveto(791.07261055,342.36264638)(791.05761056,342.45264629)(791.04761719,342.54265076)
\curveto(791.03761058,342.6426461)(791.0176106,342.73264601)(790.98761719,342.81265076)
\curveto(790.96761065,342.89264585)(790.94761067,342.96764577)(790.92761719,343.03765076)
\curveto(790.9176107,343.10764563)(790.89761072,343.17764556)(790.86761719,343.24765076)
\curveto(790.74761087,343.54764519)(790.59261103,343.81264493)(790.40261719,344.04265076)
\curveto(790.21261141,344.27264447)(789.97261165,344.45264429)(789.68261719,344.58265076)
\curveto(789.58261204,344.63264411)(789.47761214,344.66764407)(789.36761719,344.68765076)
\curveto(789.26761235,344.71764402)(789.15761246,344.742644)(789.03761719,344.76265076)
\curveto(788.95761266,344.78264396)(788.86761275,344.79264395)(788.76761719,344.79265076)
\lineto(788.49761719,344.79265076)
\curveto(788.44761317,344.78264396)(788.40261322,344.77264397)(788.36261719,344.76265076)
\lineto(788.22761719,344.76265076)
\curveto(788.14761347,344.742644)(788.06261356,344.72264402)(787.97261719,344.70265076)
\curveto(787.89261373,344.68264406)(787.81261381,344.65764408)(787.73261719,344.62765076)
\curveto(787.41261421,344.48764425)(787.15261447,344.28264446)(786.95261719,344.01265076)
\curveto(786.76261486,343.75264499)(786.60761501,343.44764529)(786.48761719,343.09765076)
\curveto(786.44761517,342.98764575)(786.4176152,342.87264587)(786.39761719,342.75265076)
\curveto(786.38761523,342.6426461)(786.37261525,342.53264621)(786.35261719,342.42265076)
\curveto(786.35261527,342.38264636)(786.34761527,342.3426464)(786.33761719,342.30265076)
\lineto(786.33761719,342.19765076)
\curveto(786.3176153,342.14764659)(786.30761531,342.09264665)(786.30761719,342.03265076)
\curveto(786.3176153,341.97264677)(786.3226153,341.91764682)(786.32261719,341.86765076)
\lineto(786.32261719,341.53765076)
\curveto(786.3226153,341.4376473)(786.33261529,341.3426474)(786.35261719,341.25265076)
\curveto(786.36261526,341.22264752)(786.36761525,341.17264757)(786.36761719,341.10265076)
\curveto(786.38761523,341.03264771)(786.40261522,340.96264778)(786.41261719,340.89265076)
\lineto(786.47261719,340.68265076)
\curveto(786.58261504,340.33264841)(786.73261489,340.03264871)(786.92261719,339.78265076)
\curveto(787.11261451,339.53264921)(787.35261427,339.32764941)(787.64261719,339.16765076)
\curveto(787.73261389,339.11764962)(787.8226138,339.07764966)(787.91261719,339.04765076)
\curveto(788.00261362,339.01764972)(788.10261352,338.98764975)(788.21261719,338.95765076)
\curveto(788.26261336,338.9376498)(788.31261331,338.93264981)(788.36261719,338.94265076)
\curveto(788.4226132,338.95264979)(788.47761314,338.94764979)(788.52761719,338.92765076)
\curveto(788.56761305,338.91764982)(788.60761301,338.91264983)(788.64761719,338.91265076)
\lineto(788.78261719,338.91265076)
\lineto(788.91761719,338.91265076)
\curveto(788.94761267,338.92264982)(788.99761262,338.92764981)(789.06761719,338.92765076)
\curveto(789.14761247,338.94764979)(789.22761239,338.96264978)(789.30761719,338.97265076)
\curveto(789.38761223,338.99264975)(789.46261216,339.01764972)(789.53261719,339.04765076)
\curveto(789.86261176,339.18764955)(790.12761149,339.36264938)(790.32761719,339.57265076)
\curveto(790.53761108,339.79264895)(790.71261091,340.06764867)(790.85261719,340.39765076)
\curveto(790.90261072,340.50764823)(790.93761068,340.61764812)(790.95761719,340.72765076)
\curveto(790.97761064,340.8376479)(791.00261062,340.94764779)(791.03261719,341.05765076)
\curveto(791.05261057,341.09764764)(791.06261056,341.13264761)(791.06261719,341.16265076)
\curveto(791.06261056,341.20264754)(791.06761055,341.2426475)(791.07761719,341.28265076)
\curveto(791.08761053,341.3426474)(791.08761053,341.40264734)(791.07761719,341.46265076)
\curveto(791.07761054,341.52264722)(791.08261054,341.58264716)(791.09261719,341.64265076)
}
}
{
\newrgbcolor{curcolor}{0 0 0}
\pscustom[linestyle=none,fillstyle=solid,fillcolor=curcolor]
{
\newpath
\moveto(794.53886719,347.22265076)
\curveto(794.45886607,347.28264146)(794.41386611,347.38764135)(794.40386719,347.53765076)
\lineto(794.40386719,348.00265076)
\lineto(794.40386719,348.25765076)
\curveto(794.40386612,348.34764039)(794.41886611,348.42264032)(794.44886719,348.48265076)
\curveto(794.48886604,348.56264018)(794.56886596,348.62264012)(794.68886719,348.66265076)
\curveto(794.70886582,348.67264007)(794.7288658,348.67264007)(794.74886719,348.66265076)
\curveto(794.77886575,348.66264008)(794.80386572,348.66764007)(794.82386719,348.67765076)
\curveto(794.99386553,348.67764006)(795.15386537,348.67264007)(795.30386719,348.66265076)
\curveto(795.45386507,348.65264009)(795.55386497,348.59264015)(795.60386719,348.48265076)
\curveto(795.63386489,348.42264032)(795.64886488,348.34764039)(795.64886719,348.25765076)
\lineto(795.64886719,348.00265076)
\curveto(795.64886488,347.82264092)(795.64386488,347.65264109)(795.63386719,347.49265076)
\curveto(795.63386489,347.33264141)(795.56886496,347.22764151)(795.43886719,347.17765076)
\curveto(795.38886514,347.15764158)(795.33386519,347.14764159)(795.27386719,347.14765076)
\lineto(795.10886719,347.14765076)
\lineto(794.79386719,347.14765076)
\curveto(794.69386583,347.14764159)(794.60886592,347.17264157)(794.53886719,347.22265076)
\moveto(795.64886719,338.71765076)
\lineto(795.64886719,338.40265076)
\curveto(795.65886487,338.30265044)(795.63886489,338.22265052)(795.58886719,338.16265076)
\curveto(795.55886497,338.10265064)(795.51386501,338.06265068)(795.45386719,338.04265076)
\curveto(795.39386513,338.03265071)(795.3238652,338.01765072)(795.24386719,337.99765076)
\lineto(795.01886719,337.99765076)
\curveto(794.88886564,337.99765074)(794.77386575,338.00265074)(794.67386719,338.01265076)
\curveto(794.58386594,338.03265071)(794.51386601,338.08265066)(794.46386719,338.16265076)
\curveto(794.4238661,338.22265052)(794.40386612,338.29765044)(794.40386719,338.38765076)
\lineto(794.40386719,338.67265076)
\lineto(794.40386719,345.01765076)
\lineto(794.40386719,345.33265076)
\curveto(794.40386612,345.4426433)(794.4288661,345.52764321)(794.47886719,345.58765076)
\curveto(794.50886602,345.6376431)(794.54886598,345.66764307)(794.59886719,345.67765076)
\curveto(794.64886588,345.68764305)(794.70386582,345.70264304)(794.76386719,345.72265076)
\curveto(794.78386574,345.72264302)(794.80386572,345.71764302)(794.82386719,345.70765076)
\curveto(794.85386567,345.70764303)(794.87886565,345.71264303)(794.89886719,345.72265076)
\curveto(795.0288655,345.72264302)(795.15886537,345.71764302)(795.28886719,345.70765076)
\curveto(795.4288651,345.70764303)(795.523865,345.66764307)(795.57386719,345.58765076)
\curveto(795.6238649,345.52764321)(795.64886488,345.44764329)(795.64886719,345.34765076)
\lineto(795.64886719,345.06265076)
\lineto(795.64886719,338.71765076)
}
}
{
\newrgbcolor{curcolor}{0 0 0}
\pscustom[linestyle=none,fillstyle=solid,fillcolor=curcolor]
{
\newpath
\moveto(804.34371094,342.16765076)
\curveto(804.36370325,342.06764667)(804.36370325,341.95264679)(804.34371094,341.82265076)
\curveto(804.33370328,341.70264704)(804.30370331,341.61764712)(804.25371094,341.56765076)
\curveto(804.20370341,341.52764721)(804.12870349,341.49764724)(804.02871094,341.47765076)
\curveto(803.93870368,341.46764727)(803.83370378,341.46264728)(803.71371094,341.46265076)
\lineto(803.35371094,341.46265076)
\curveto(803.23370438,341.47264727)(803.12870449,341.47764726)(803.03871094,341.47765076)
\lineto(799.19871094,341.47765076)
\curveto(799.1187085,341.47764726)(799.03870858,341.47264727)(798.95871094,341.46265076)
\curveto(798.87870874,341.46264728)(798.8137088,341.44764729)(798.76371094,341.41765076)
\curveto(798.72370889,341.39764734)(798.68370893,341.35764738)(798.64371094,341.29765076)
\curveto(798.62370899,341.26764747)(798.60370901,341.22264752)(798.58371094,341.16265076)
\curveto(798.56370905,341.11264763)(798.56370905,341.06264768)(798.58371094,341.01265076)
\curveto(798.59370902,340.96264778)(798.59870902,340.91764782)(798.59871094,340.87765076)
\curveto(798.59870902,340.8376479)(798.60370901,340.79764794)(798.61371094,340.75765076)
\curveto(798.63370898,340.67764806)(798.65370896,340.59264815)(798.67371094,340.50265076)
\curveto(798.69370892,340.42264832)(798.72370889,340.3426484)(798.76371094,340.26265076)
\curveto(798.99370862,339.72264902)(799.37370824,339.3376494)(799.90371094,339.10765076)
\curveto(799.96370765,339.07764966)(800.02870759,339.05264969)(800.09871094,339.03265076)
\lineto(800.30871094,338.97265076)
\curveto(800.33870728,338.96264978)(800.38870723,338.95764978)(800.45871094,338.95765076)
\curveto(800.59870702,338.91764982)(800.78370683,338.89764984)(801.01371094,338.89765076)
\curveto(801.24370637,338.89764984)(801.42870619,338.91764982)(801.56871094,338.95765076)
\curveto(801.70870591,338.99764974)(801.83370578,339.0376497)(801.94371094,339.07765076)
\curveto(802.06370555,339.12764961)(802.17370544,339.18764955)(802.27371094,339.25765076)
\curveto(802.38370523,339.32764941)(802.47870514,339.40764933)(802.55871094,339.49765076)
\curveto(802.63870498,339.59764914)(802.70870491,339.70264904)(802.76871094,339.81265076)
\curveto(802.82870479,339.91264883)(802.87870474,340.01764872)(802.91871094,340.12765076)
\curveto(802.96870465,340.2376485)(803.04870457,340.31764842)(803.15871094,340.36765076)
\curveto(803.19870442,340.38764835)(803.26370435,340.40264834)(803.35371094,340.41265076)
\curveto(803.44370417,340.42264832)(803.53370408,340.42264832)(803.62371094,340.41265076)
\curveto(803.7137039,340.41264833)(803.79870382,340.40764833)(803.87871094,340.39765076)
\curveto(803.95870366,340.38764835)(804.0137036,340.36764837)(804.04371094,340.33765076)
\curveto(804.14370347,340.26764847)(804.16870345,340.15264859)(804.11871094,339.99265076)
\curveto(804.03870358,339.72264902)(803.93370368,339.48264926)(803.80371094,339.27265076)
\curveto(803.60370401,338.95264979)(803.37370424,338.68765005)(803.11371094,338.47765076)
\curveto(802.86370475,338.27765046)(802.54370507,338.11265063)(802.15371094,337.98265076)
\curveto(802.05370556,337.9426508)(801.95370566,337.91765082)(801.85371094,337.90765076)
\curveto(801.75370586,337.88765085)(801.64870597,337.86765087)(801.53871094,337.84765076)
\curveto(801.48870613,337.8376509)(801.43870618,337.83265091)(801.38871094,337.83265076)
\curveto(801.34870627,337.83265091)(801.30370631,337.82765091)(801.25371094,337.81765076)
\lineto(801.10371094,337.81765076)
\curveto(801.05370656,337.80765093)(800.99370662,337.80265094)(800.92371094,337.80265076)
\curveto(800.86370675,337.80265094)(800.8137068,337.80765093)(800.77371094,337.81765076)
\lineto(800.63871094,337.81765076)
\curveto(800.58870703,337.82765091)(800.54370707,337.83265091)(800.50371094,337.83265076)
\curveto(800.46370715,337.83265091)(800.42370719,337.8376509)(800.38371094,337.84765076)
\curveto(800.33370728,337.85765088)(800.27870734,337.86765087)(800.21871094,337.87765076)
\curveto(800.15870746,337.87765086)(800.10370751,337.88265086)(800.05371094,337.89265076)
\curveto(799.96370765,337.91265083)(799.87370774,337.9376508)(799.78371094,337.96765076)
\curveto(799.69370792,337.98765075)(799.60870801,338.01265073)(799.52871094,338.04265076)
\curveto(799.48870813,338.06265068)(799.45370816,338.07265067)(799.42371094,338.07265076)
\curveto(799.39370822,338.08265066)(799.35870826,338.09765064)(799.31871094,338.11765076)
\curveto(799.16870845,338.18765055)(799.00870861,338.27265047)(798.83871094,338.37265076)
\curveto(798.54870907,338.56265018)(798.29870932,338.79264995)(798.08871094,339.06265076)
\curveto(797.88870973,339.3426494)(797.7187099,339.65264909)(797.57871094,339.99265076)
\curveto(797.52871009,340.10264864)(797.48871013,340.21764852)(797.45871094,340.33765076)
\curveto(797.43871018,340.45764828)(797.40871021,340.57764816)(797.36871094,340.69765076)
\curveto(797.35871026,340.737648)(797.35371026,340.77264797)(797.35371094,340.80265076)
\curveto(797.35371026,340.83264791)(797.34871027,340.87264787)(797.33871094,340.92265076)
\curveto(797.3187103,341.00264774)(797.30371031,341.08764765)(797.29371094,341.17765076)
\curveto(797.28371033,341.26764747)(797.26871035,341.35764738)(797.24871094,341.44765076)
\lineto(797.24871094,341.65765076)
\curveto(797.23871038,341.69764704)(797.22871039,341.75264699)(797.21871094,341.82265076)
\curveto(797.2187104,341.90264684)(797.22371039,341.96764677)(797.23371094,342.01765076)
\lineto(797.23371094,342.18265076)
\curveto(797.25371036,342.23264651)(797.25871036,342.28264646)(797.24871094,342.33265076)
\curveto(797.24871037,342.39264635)(797.25371036,342.44764629)(797.26371094,342.49765076)
\curveto(797.30371031,342.65764608)(797.33371028,342.81764592)(797.35371094,342.97765076)
\curveto(797.38371023,343.1376456)(797.42871019,343.28764545)(797.48871094,343.42765076)
\curveto(797.53871008,343.5376452)(797.58371003,343.64764509)(797.62371094,343.75765076)
\curveto(797.67370994,343.87764486)(797.72870989,343.99264475)(797.78871094,344.10265076)
\curveto(798.00870961,344.45264429)(798.25870936,344.75264399)(798.53871094,345.00265076)
\curveto(798.8187088,345.26264348)(799.16370845,345.47764326)(799.57371094,345.64765076)
\curveto(799.69370792,345.69764304)(799.8137078,345.73264301)(799.93371094,345.75265076)
\curveto(800.06370755,345.78264296)(800.19870742,345.81264293)(800.33871094,345.84265076)
\curveto(800.38870723,345.85264289)(800.43370718,345.85764288)(800.47371094,345.85765076)
\curveto(800.5137071,345.86764287)(800.55870706,345.87264287)(800.60871094,345.87265076)
\curveto(800.62870699,345.88264286)(800.65370696,345.88264286)(800.68371094,345.87265076)
\curveto(800.7137069,345.86264288)(800.73870688,345.86764287)(800.75871094,345.88765076)
\curveto(801.17870644,345.89764284)(801.54370607,345.85264289)(801.85371094,345.75265076)
\curveto(802.16370545,345.66264308)(802.44370517,345.5376432)(802.69371094,345.37765076)
\curveto(802.74370487,345.35764338)(802.78370483,345.32764341)(802.81371094,345.28765076)
\curveto(802.84370477,345.25764348)(802.87870474,345.23264351)(802.91871094,345.21265076)
\curveto(802.99870462,345.15264359)(803.07870454,345.08264366)(803.15871094,345.00265076)
\curveto(803.24870437,344.92264382)(803.32370429,344.8426439)(803.38371094,344.76265076)
\curveto(803.54370407,344.55264419)(803.67870394,344.35264439)(803.78871094,344.16265076)
\curveto(803.85870376,344.05264469)(803.9137037,343.93264481)(803.95371094,343.80265076)
\curveto(803.99370362,343.67264507)(804.03870358,343.5426452)(804.08871094,343.41265076)
\curveto(804.13870348,343.28264546)(804.17370344,343.14764559)(804.19371094,343.00765076)
\curveto(804.22370339,342.86764587)(804.25870336,342.72764601)(804.29871094,342.58765076)
\curveto(804.30870331,342.51764622)(804.3137033,342.44764629)(804.31371094,342.37765076)
\lineto(804.34371094,342.16765076)
\moveto(802.88871094,342.67765076)
\curveto(802.9187047,342.71764602)(802.94370467,342.76764597)(802.96371094,342.82765076)
\curveto(802.98370463,342.89764584)(802.98370463,342.96764577)(802.96371094,343.03765076)
\curveto(802.90370471,343.25764548)(802.8187048,343.46264528)(802.70871094,343.65265076)
\curveto(802.56870505,343.88264486)(802.4137052,344.07764466)(802.24371094,344.23765076)
\curveto(802.07370554,344.39764434)(801.85370576,344.53264421)(801.58371094,344.64265076)
\curveto(801.5137061,344.66264408)(801.44370617,344.67764406)(801.37371094,344.68765076)
\curveto(801.30370631,344.70764403)(801.22870639,344.72764401)(801.14871094,344.74765076)
\curveto(801.06870655,344.76764397)(800.98370663,344.77764396)(800.89371094,344.77765076)
\lineto(800.63871094,344.77765076)
\curveto(800.60870701,344.75764398)(800.57370704,344.74764399)(800.53371094,344.74765076)
\curveto(800.49370712,344.75764398)(800.45870716,344.75764398)(800.42871094,344.74765076)
\lineto(800.18871094,344.68765076)
\curveto(800.1187075,344.67764406)(800.04870757,344.66264408)(799.97871094,344.64265076)
\curveto(799.68870793,344.52264422)(799.45370816,344.37264437)(799.27371094,344.19265076)
\curveto(799.10370851,344.01264473)(798.94870867,343.78764495)(798.80871094,343.51765076)
\curveto(798.77870884,343.46764527)(798.74870887,343.40264534)(798.71871094,343.32265076)
\curveto(798.68870893,343.25264549)(798.66370895,343.17264557)(798.64371094,343.08265076)
\curveto(798.62370899,342.99264575)(798.618709,342.90764583)(798.62871094,342.82765076)
\curveto(798.63870898,342.74764599)(798.67370894,342.68764605)(798.73371094,342.64765076)
\curveto(798.8137088,342.58764615)(798.94870867,342.55764618)(799.13871094,342.55765076)
\curveto(799.33870828,342.56764617)(799.50870811,342.57264617)(799.64871094,342.57265076)
\lineto(801.92871094,342.57265076)
\curveto(802.07870554,342.57264617)(802.25870536,342.56764617)(802.46871094,342.55765076)
\curveto(802.67870494,342.55764618)(802.8187048,342.59764614)(802.88871094,342.67765076)
}
}
{
\newrgbcolor{curcolor}{0 0 0}
\pscustom[linestyle=none,fillstyle=solid,fillcolor=curcolor]
{
\newpath
\moveto(809.34035156,345.87265076)
\curveto(809.97034633,345.89264285)(810.47534582,345.80764293)(810.85535156,345.61765076)
\curveto(811.23534506,345.42764331)(811.54034476,345.1426436)(811.77035156,344.76265076)
\curveto(811.83034447,344.66264408)(811.87534442,344.55264419)(811.90535156,344.43265076)
\curveto(811.94534435,344.32264442)(811.98034432,344.20764453)(812.01035156,344.08765076)
\curveto(812.06034424,343.89764484)(812.09034421,343.69264505)(812.10035156,343.47265076)
\curveto(812.11034419,343.25264549)(812.11534418,343.02764571)(812.11535156,342.79765076)
\lineto(812.11535156,341.19265076)
\lineto(812.11535156,338.85265076)
\curveto(812.11534418,338.68265006)(812.11034419,338.51265023)(812.10035156,338.34265076)
\curveto(812.1003442,338.17265057)(812.03534426,338.06265068)(811.90535156,338.01265076)
\curveto(811.85534444,337.99265075)(811.8003445,337.98265076)(811.74035156,337.98265076)
\curveto(811.69034461,337.97265077)(811.63534466,337.96765077)(811.57535156,337.96765076)
\curveto(811.44534485,337.96765077)(811.32034498,337.97265077)(811.20035156,337.98265076)
\curveto(811.08034522,337.98265076)(810.9953453,338.02265072)(810.94535156,338.10265076)
\curveto(810.8953454,338.17265057)(810.87034543,338.26265048)(810.87035156,338.37265076)
\lineto(810.87035156,338.70265076)
\lineto(810.87035156,339.99265076)
\lineto(810.87035156,342.43765076)
\curveto(810.87034543,342.70764603)(810.86534543,342.97264577)(810.85535156,343.23265076)
\curveto(810.84534545,343.50264524)(810.8003455,343.73264501)(810.72035156,343.92265076)
\curveto(810.64034566,344.12264462)(810.52034578,344.28264446)(810.36035156,344.40265076)
\curveto(810.2003461,344.53264421)(810.01534628,344.63264411)(809.80535156,344.70265076)
\curveto(809.74534655,344.72264402)(809.68034662,344.73264401)(809.61035156,344.73265076)
\curveto(809.55034675,344.742644)(809.49034681,344.75764398)(809.43035156,344.77765076)
\curveto(809.38034692,344.78764395)(809.300347,344.78764395)(809.19035156,344.77765076)
\curveto(809.09034721,344.77764396)(809.02034728,344.77264397)(808.98035156,344.76265076)
\curveto(808.94034736,344.742644)(808.90534739,344.73264401)(808.87535156,344.73265076)
\curveto(808.84534745,344.742644)(808.81034749,344.742644)(808.77035156,344.73265076)
\curveto(808.64034766,344.70264404)(808.51534778,344.66764407)(808.39535156,344.62765076)
\curveto(808.28534801,344.59764414)(808.18034812,344.55264419)(808.08035156,344.49265076)
\curveto(808.04034826,344.47264427)(808.00534829,344.45264429)(807.97535156,344.43265076)
\curveto(807.94534835,344.41264433)(807.91034839,344.39264435)(807.87035156,344.37265076)
\curveto(807.52034878,344.12264462)(807.26534903,343.74764499)(807.10535156,343.24765076)
\curveto(807.07534922,343.16764557)(807.05534924,343.08264566)(807.04535156,342.99265076)
\curveto(807.03534926,342.91264583)(807.02034928,342.83264591)(807.00035156,342.75265076)
\curveto(806.98034932,342.70264604)(806.97534932,342.65264609)(806.98535156,342.60265076)
\curveto(806.9953493,342.56264618)(806.99034931,342.52264622)(806.97035156,342.48265076)
\lineto(806.97035156,342.16765076)
\curveto(806.96034934,342.1376466)(806.95534934,342.10264664)(806.95535156,342.06265076)
\curveto(806.96534933,342.02264672)(806.97034933,341.97764676)(806.97035156,341.92765076)
\lineto(806.97035156,341.47765076)
\lineto(806.97035156,340.03765076)
\lineto(806.97035156,338.71765076)
\lineto(806.97035156,338.37265076)
\curveto(806.97034933,338.26265048)(806.94534935,338.17265057)(806.89535156,338.10265076)
\curveto(806.84534945,338.02265072)(806.75534954,337.98265076)(806.62535156,337.98265076)
\curveto(806.50534979,337.97265077)(806.38034992,337.96765077)(806.25035156,337.96765076)
\curveto(806.17035013,337.96765077)(806.0953502,337.97265077)(806.02535156,337.98265076)
\curveto(805.95535034,337.99265075)(805.8953504,338.01765072)(805.84535156,338.05765076)
\curveto(805.76535053,338.10765063)(805.72535057,338.20265054)(805.72535156,338.34265076)
\lineto(805.72535156,338.74765076)
\lineto(805.72535156,340.51765076)
\lineto(805.72535156,344.14765076)
\lineto(805.72535156,345.06265076)
\lineto(805.72535156,345.33265076)
\curveto(805.72535057,345.42264332)(805.74535055,345.49264325)(805.78535156,345.54265076)
\curveto(805.81535048,345.60264314)(805.86535043,345.6426431)(805.93535156,345.66265076)
\curveto(805.97535032,345.67264307)(806.03035027,345.68264306)(806.10035156,345.69265076)
\curveto(806.18035012,345.70264304)(806.26035004,345.70764303)(806.34035156,345.70765076)
\curveto(806.42034988,345.70764303)(806.4953498,345.70264304)(806.56535156,345.69265076)
\curveto(806.64534965,345.68264306)(806.7003496,345.66764307)(806.73035156,345.64765076)
\curveto(806.84034946,345.57764316)(806.89034941,345.48764325)(806.88035156,345.37765076)
\curveto(806.87034943,345.27764346)(806.88534941,345.16264358)(806.92535156,345.03265076)
\curveto(806.94534935,344.97264377)(806.98534931,344.92264382)(807.04535156,344.88265076)
\curveto(807.16534913,344.87264387)(807.26034904,344.91764382)(807.33035156,345.01765076)
\curveto(807.41034889,345.11764362)(807.49034881,345.19764354)(807.57035156,345.25765076)
\curveto(807.71034859,345.35764338)(807.85034845,345.44764329)(807.99035156,345.52765076)
\curveto(808.14034816,345.61764312)(808.31034799,345.69264305)(808.50035156,345.75265076)
\curveto(808.58034772,345.78264296)(808.66534763,345.80264294)(808.75535156,345.81265076)
\curveto(808.85534744,345.82264292)(808.95034735,345.8376429)(809.04035156,345.85765076)
\curveto(809.09034721,345.86764287)(809.14034716,345.87264287)(809.19035156,345.87265076)
\lineto(809.34035156,345.87265076)
}
}
{
\newrgbcolor{curcolor}{0 0 0}
\pscustom[linestyle=none,fillstyle=solid,fillcolor=curcolor]
{
\newpath
\moveto(817.13496094,345.90265076)
\curveto(817.87495615,345.91264283)(818.48995553,345.80264294)(818.97996094,345.57265076)
\curveto(819.47995454,345.35264339)(819.87495415,345.01764372)(820.16496094,344.56765076)
\curveto(820.29495373,344.36764437)(820.40495362,344.12264462)(820.49496094,343.83265076)
\curveto(820.51495351,343.78264496)(820.52995349,343.71764502)(820.53996094,343.63765076)
\curveto(820.54995347,343.55764518)(820.54495348,343.48764525)(820.52496094,343.42765076)
\curveto(820.49495353,343.37764536)(820.44495358,343.33264541)(820.37496094,343.29265076)
\curveto(820.34495368,343.27264547)(820.31495371,343.26264548)(820.28496094,343.26265076)
\curveto(820.25495377,343.27264547)(820.2199538,343.27264547)(820.17996094,343.26265076)
\curveto(820.13995388,343.25264549)(820.09995392,343.24764549)(820.05996094,343.24765076)
\curveto(820.019954,343.25764548)(819.97995404,343.26264548)(819.93996094,343.26265076)
\lineto(819.62496094,343.26265076)
\curveto(819.5249545,343.27264547)(819.43995458,343.30264544)(819.36996094,343.35265076)
\curveto(819.28995473,343.41264533)(819.23495479,343.49764524)(819.20496094,343.60765076)
\curveto(819.17495485,343.71764502)(819.13495489,343.81264493)(819.08496094,343.89265076)
\curveto(818.93495509,344.15264459)(818.73995528,344.35764438)(818.49996094,344.50765076)
\curveto(818.4199556,344.55764418)(818.33495569,344.59764414)(818.24496094,344.62765076)
\curveto(818.15495587,344.66764407)(818.05995596,344.70264404)(817.95996094,344.73265076)
\curveto(817.8199562,344.77264397)(817.63495639,344.79264395)(817.40496094,344.79265076)
\curveto(817.17495685,344.80264394)(816.98495704,344.78264396)(816.83496094,344.73265076)
\curveto(816.76495726,344.71264403)(816.69995732,344.69764404)(816.63996094,344.68765076)
\curveto(816.57995744,344.67764406)(816.51495751,344.66264408)(816.44496094,344.64265076)
\curveto(816.18495784,344.53264421)(815.95495807,344.38264436)(815.75496094,344.19265076)
\curveto(815.55495847,344.00264474)(815.39995862,343.77764496)(815.28996094,343.51765076)
\curveto(815.24995877,343.42764531)(815.21495881,343.33264541)(815.18496094,343.23265076)
\curveto(815.15495887,343.1426456)(815.1249589,343.0426457)(815.09496094,342.93265076)
\lineto(815.00496094,342.52765076)
\curveto(814.99495903,342.47764626)(814.98995903,342.42264632)(814.98996094,342.36265076)
\curveto(814.99995902,342.30264644)(814.99495903,342.24764649)(814.97496094,342.19765076)
\lineto(814.97496094,342.07765076)
\curveto(814.96495906,342.0376467)(814.95495907,341.97264677)(814.94496094,341.88265076)
\curveto(814.94495908,341.79264695)(814.95495907,341.72764701)(814.97496094,341.68765076)
\curveto(814.98495904,341.6376471)(814.98495904,341.58764715)(814.97496094,341.53765076)
\curveto(814.96495906,341.48764725)(814.96495906,341.4376473)(814.97496094,341.38765076)
\curveto(814.98495904,341.34764739)(814.98995903,341.27764746)(814.98996094,341.17765076)
\curveto(815.00995901,341.09764764)(815.024959,341.01264773)(815.03496094,340.92265076)
\curveto(815.05495897,340.83264791)(815.07495895,340.74764799)(815.09496094,340.66765076)
\curveto(815.20495882,340.34764839)(815.32995869,340.06764867)(815.46996094,339.82765076)
\curveto(815.6199584,339.59764914)(815.8249582,339.39764934)(816.08496094,339.22765076)
\curveto(816.17495785,339.17764956)(816.26495776,339.13264961)(816.35496094,339.09265076)
\curveto(816.45495757,339.05264969)(816.55995746,339.01264973)(816.66996094,338.97265076)
\curveto(816.7199573,338.96264978)(816.75995726,338.95764978)(816.78996094,338.95765076)
\curveto(816.8199572,338.95764978)(816.85995716,338.95264979)(816.90996094,338.94265076)
\curveto(816.93995708,338.93264981)(816.98995703,338.92764981)(817.05996094,338.92765076)
\lineto(817.22496094,338.92765076)
\curveto(817.2249568,338.91764982)(817.24495678,338.91264983)(817.28496094,338.91265076)
\curveto(817.30495672,338.92264982)(817.32995669,338.92264982)(817.35996094,338.91265076)
\curveto(817.38995663,338.91264983)(817.4199566,338.91764982)(817.44996094,338.92765076)
\curveto(817.5199565,338.94764979)(817.58495644,338.95264979)(817.64496094,338.94265076)
\curveto(817.71495631,338.9426498)(817.78495624,338.95264979)(817.85496094,338.97265076)
\curveto(818.11495591,339.05264969)(818.33995568,339.15264959)(818.52996094,339.27265076)
\curveto(818.7199553,339.40264934)(818.87995514,339.56764917)(819.00996094,339.76765076)
\curveto(819.05995496,339.84764889)(819.10495492,339.93264881)(819.14496094,340.02265076)
\lineto(819.26496094,340.29265076)
\curveto(819.28495474,340.37264837)(819.30495472,340.44764829)(819.32496094,340.51765076)
\curveto(819.35495467,340.59764814)(819.40495462,340.66264808)(819.47496094,340.71265076)
\curveto(819.50495452,340.742648)(819.56495446,340.76264798)(819.65496094,340.77265076)
\curveto(819.74495428,340.79264795)(819.83995418,340.80264794)(819.93996094,340.80265076)
\curveto(820.04995397,340.81264793)(820.14995387,340.81264793)(820.23996094,340.80265076)
\curveto(820.33995368,340.79264795)(820.40995361,340.77264797)(820.44996094,340.74265076)
\curveto(820.50995351,340.70264804)(820.54495348,340.6426481)(820.55496094,340.56265076)
\curveto(820.57495345,340.48264826)(820.57495345,340.39764834)(820.55496094,340.30765076)
\curveto(820.50495352,340.15764858)(820.45495357,340.01264873)(820.40496094,339.87265076)
\curveto(820.36495366,339.742649)(820.30995371,339.61264913)(820.23996094,339.48265076)
\curveto(820.08995393,339.18264956)(819.89995412,338.91764982)(819.66996094,338.68765076)
\curveto(819.44995457,338.45765028)(819.17995484,338.27265047)(818.85996094,338.13265076)
\curveto(818.77995524,338.09265065)(818.69495533,338.05765068)(818.60496094,338.02765076)
\curveto(818.51495551,338.00765073)(818.4199556,337.98265076)(818.31996094,337.95265076)
\curveto(818.20995581,337.91265083)(818.09995592,337.89265085)(817.98996094,337.89265076)
\curveto(817.87995614,337.88265086)(817.76995625,337.86765087)(817.65996094,337.84765076)
\curveto(817.6199564,337.82765091)(817.57995644,337.82265092)(817.53996094,337.83265076)
\curveto(817.49995652,337.8426509)(817.45995656,337.8426509)(817.41996094,337.83265076)
\lineto(817.28496094,337.83265076)
\lineto(817.04496094,337.83265076)
\curveto(816.97495705,337.82265092)(816.90995711,337.82765091)(816.84996094,337.84765076)
\lineto(816.77496094,337.84765076)
\lineto(816.41496094,337.89265076)
\curveto(816.28495774,337.93265081)(816.15995786,337.96765077)(816.03996094,337.99765076)
\curveto(815.9199581,338.02765071)(815.80495822,338.06765067)(815.69496094,338.11765076)
\curveto(815.33495869,338.27765046)(815.03495899,338.46765027)(814.79496094,338.68765076)
\curveto(814.56495946,338.90764983)(814.34995967,339.17764956)(814.14996094,339.49765076)
\curveto(814.09995992,339.57764916)(814.05495997,339.66764907)(814.01496094,339.76765076)
\lineto(813.89496094,340.06765076)
\curveto(813.84496018,340.17764856)(813.80996021,340.29264845)(813.78996094,340.41265076)
\curveto(813.76996025,340.53264821)(813.74496028,340.65264809)(813.71496094,340.77265076)
\curveto(813.70496032,340.81264793)(813.69996032,340.85264789)(813.69996094,340.89265076)
\curveto(813.69996032,340.93264781)(813.69496033,340.97264777)(813.68496094,341.01265076)
\curveto(813.66496036,341.07264767)(813.65496037,341.1376476)(813.65496094,341.20765076)
\curveto(813.66496036,341.27764746)(813.65996036,341.3426474)(813.63996094,341.40265076)
\lineto(813.63996094,341.55265076)
\curveto(813.62996039,341.60264714)(813.6249604,341.67264707)(813.62496094,341.76265076)
\curveto(813.6249604,341.85264689)(813.62996039,341.92264682)(813.63996094,341.97265076)
\curveto(813.64996037,342.02264672)(813.64996037,342.06764667)(813.63996094,342.10765076)
\curveto(813.63996038,342.14764659)(813.64496038,342.18764655)(813.65496094,342.22765076)
\curveto(813.67496035,342.29764644)(813.67996034,342.36764637)(813.66996094,342.43765076)
\curveto(813.66996035,342.50764623)(813.67996034,342.57264617)(813.69996094,342.63265076)
\curveto(813.73996028,342.80264594)(813.77496025,342.97264577)(813.80496094,343.14265076)
\curveto(813.83496019,343.31264543)(813.87996014,343.47264527)(813.93996094,343.62265076)
\curveto(814.14995987,344.1426446)(814.40495962,344.56264418)(814.70496094,344.88265076)
\curveto(815.00495902,345.20264354)(815.41495861,345.46764327)(815.93496094,345.67765076)
\curveto(816.04495798,345.72764301)(816.16495786,345.76264298)(816.29496094,345.78265076)
\curveto(816.4249576,345.80264294)(816.55995746,345.82764291)(816.69996094,345.85765076)
\curveto(816.76995725,345.86764287)(816.83995718,345.87264287)(816.90996094,345.87265076)
\curveto(816.97995704,345.88264286)(817.05495697,345.89264285)(817.13496094,345.90265076)
}
}
{
\newrgbcolor{curcolor}{0 0 0}
\pscustom[linestyle=none,fillstyle=solid,fillcolor=curcolor]
{
\newpath
\moveto(822.34160156,347.22265076)
\curveto(822.26160044,347.28264146)(822.21660049,347.38764135)(822.20660156,347.53765076)
\lineto(822.20660156,348.00265076)
\lineto(822.20660156,348.25765076)
\curveto(822.2066005,348.34764039)(822.22160048,348.42264032)(822.25160156,348.48265076)
\curveto(822.29160041,348.56264018)(822.37160033,348.62264012)(822.49160156,348.66265076)
\curveto(822.51160019,348.67264007)(822.53160017,348.67264007)(822.55160156,348.66265076)
\curveto(822.58160012,348.66264008)(822.6066001,348.66764007)(822.62660156,348.67765076)
\curveto(822.79659991,348.67764006)(822.95659975,348.67264007)(823.10660156,348.66265076)
\curveto(823.25659945,348.65264009)(823.35659935,348.59264015)(823.40660156,348.48265076)
\curveto(823.43659927,348.42264032)(823.45159925,348.34764039)(823.45160156,348.25765076)
\lineto(823.45160156,348.00265076)
\curveto(823.45159925,347.82264092)(823.44659926,347.65264109)(823.43660156,347.49265076)
\curveto(823.43659927,347.33264141)(823.37159933,347.22764151)(823.24160156,347.17765076)
\curveto(823.19159951,347.15764158)(823.13659957,347.14764159)(823.07660156,347.14765076)
\lineto(822.91160156,347.14765076)
\lineto(822.59660156,347.14765076)
\curveto(822.49660021,347.14764159)(822.41160029,347.17264157)(822.34160156,347.22265076)
\moveto(823.45160156,338.71765076)
\lineto(823.45160156,338.40265076)
\curveto(823.46159924,338.30265044)(823.44159926,338.22265052)(823.39160156,338.16265076)
\curveto(823.36159934,338.10265064)(823.31659939,338.06265068)(823.25660156,338.04265076)
\curveto(823.19659951,338.03265071)(823.12659958,338.01765072)(823.04660156,337.99765076)
\lineto(822.82160156,337.99765076)
\curveto(822.69160001,337.99765074)(822.57660013,338.00265074)(822.47660156,338.01265076)
\curveto(822.38660032,338.03265071)(822.31660039,338.08265066)(822.26660156,338.16265076)
\curveto(822.22660048,338.22265052)(822.2066005,338.29765044)(822.20660156,338.38765076)
\lineto(822.20660156,338.67265076)
\lineto(822.20660156,345.01765076)
\lineto(822.20660156,345.33265076)
\curveto(822.2066005,345.4426433)(822.23160047,345.52764321)(822.28160156,345.58765076)
\curveto(822.31160039,345.6376431)(822.35160035,345.66764307)(822.40160156,345.67765076)
\curveto(822.45160025,345.68764305)(822.5066002,345.70264304)(822.56660156,345.72265076)
\curveto(822.58660012,345.72264302)(822.6066001,345.71764302)(822.62660156,345.70765076)
\curveto(822.65660005,345.70764303)(822.68160002,345.71264303)(822.70160156,345.72265076)
\curveto(822.83159987,345.72264302)(822.96159974,345.71764302)(823.09160156,345.70765076)
\curveto(823.23159947,345.70764303)(823.32659938,345.66764307)(823.37660156,345.58765076)
\curveto(823.42659928,345.52764321)(823.45159925,345.44764329)(823.45160156,345.34765076)
\lineto(823.45160156,345.06265076)
\lineto(823.45160156,338.71765076)
}
}
{
\newrgbcolor{curcolor}{0 0 0}
\pscustom[linestyle=none,fillstyle=solid,fillcolor=curcolor]
{
\newpath
\moveto(832.28144531,338.55265076)
\curveto(832.31143748,338.39265035)(832.2964375,338.25765048)(832.23644531,338.14765076)
\curveto(832.17643762,338.04765069)(832.0964377,337.97265077)(831.99644531,337.92265076)
\curveto(831.94643785,337.90265084)(831.8914379,337.89265085)(831.83144531,337.89265076)
\curveto(831.78143801,337.89265085)(831.72643807,337.88265086)(831.66644531,337.86265076)
\curveto(831.44643835,337.81265093)(831.22643857,337.82765091)(831.00644531,337.90765076)
\curveto(830.796439,337.97765076)(830.65143914,338.06765067)(830.57144531,338.17765076)
\curveto(830.52143927,338.24765049)(830.47643932,338.32765041)(830.43644531,338.41765076)
\curveto(830.3964394,338.51765022)(830.34643945,338.59765014)(830.28644531,338.65765076)
\curveto(830.26643953,338.67765006)(830.24143955,338.69765004)(830.21144531,338.71765076)
\curveto(830.1914396,338.73765)(830.16143963,338.74265)(830.12144531,338.73265076)
\curveto(830.01143978,338.70265004)(829.90643989,338.64765009)(829.80644531,338.56765076)
\curveto(829.71644008,338.48765025)(829.62644017,338.41765032)(829.53644531,338.35765076)
\curveto(829.40644039,338.27765046)(829.26644053,338.20265054)(829.11644531,338.13265076)
\curveto(828.96644083,338.07265067)(828.80644099,338.01765072)(828.63644531,337.96765076)
\curveto(828.53644126,337.9376508)(828.42644137,337.91765082)(828.30644531,337.90765076)
\curveto(828.1964416,337.89765084)(828.08644171,337.88265086)(827.97644531,337.86265076)
\curveto(827.92644187,337.85265089)(827.88144191,337.84765089)(827.84144531,337.84765076)
\lineto(827.73644531,337.84765076)
\curveto(827.62644217,337.82765091)(827.52144227,337.82765091)(827.42144531,337.84765076)
\lineto(827.28644531,337.84765076)
\curveto(827.23644256,337.85765088)(827.18644261,337.86265088)(827.13644531,337.86265076)
\curveto(827.08644271,337.86265088)(827.04144275,337.87265087)(827.00144531,337.89265076)
\curveto(826.96144283,337.90265084)(826.92644287,337.90765083)(826.89644531,337.90765076)
\curveto(826.87644292,337.89765084)(826.85144294,337.89765084)(826.82144531,337.90765076)
\lineto(826.58144531,337.96765076)
\curveto(826.50144329,337.97765076)(826.42644337,337.99765074)(826.35644531,338.02765076)
\curveto(826.05644374,338.15765058)(825.81144398,338.30265044)(825.62144531,338.46265076)
\curveto(825.44144435,338.63265011)(825.2914445,338.86764987)(825.17144531,339.16765076)
\curveto(825.08144471,339.38764935)(825.03644476,339.65264909)(825.03644531,339.96265076)
\lineto(825.03644531,340.27765076)
\curveto(825.04644475,340.32764841)(825.05144474,340.37764836)(825.05144531,340.42765076)
\lineto(825.08144531,340.60765076)
\lineto(825.20144531,340.93765076)
\curveto(825.24144455,341.04764769)(825.2914445,341.14764759)(825.35144531,341.23765076)
\curveto(825.53144426,341.52764721)(825.77644402,341.742647)(826.08644531,341.88265076)
\curveto(826.3964434,342.02264672)(826.73644306,342.14764659)(827.10644531,342.25765076)
\curveto(827.24644255,342.29764644)(827.3914424,342.32764641)(827.54144531,342.34765076)
\curveto(827.6914421,342.36764637)(827.84144195,342.39264635)(827.99144531,342.42265076)
\curveto(828.06144173,342.4426463)(828.12644167,342.45264629)(828.18644531,342.45265076)
\curveto(828.25644154,342.45264629)(828.33144146,342.46264628)(828.41144531,342.48265076)
\curveto(828.48144131,342.50264624)(828.55144124,342.51264623)(828.62144531,342.51265076)
\curveto(828.6914411,342.52264622)(828.76644103,342.5376462)(828.84644531,342.55765076)
\curveto(829.0964407,342.61764612)(829.33144046,342.66764607)(829.55144531,342.70765076)
\curveto(829.77144002,342.75764598)(829.94643985,342.87264587)(830.07644531,343.05265076)
\curveto(830.13643966,343.13264561)(830.18643961,343.23264551)(830.22644531,343.35265076)
\curveto(830.26643953,343.48264526)(830.26643953,343.62264512)(830.22644531,343.77265076)
\curveto(830.16643963,344.01264473)(830.07643972,344.20264454)(829.95644531,344.34265076)
\curveto(829.84643995,344.48264426)(829.68644011,344.59264415)(829.47644531,344.67265076)
\curveto(829.35644044,344.72264402)(829.21144058,344.75764398)(829.04144531,344.77765076)
\curveto(828.88144091,344.79764394)(828.71144108,344.80764393)(828.53144531,344.80765076)
\curveto(828.35144144,344.80764393)(828.17644162,344.79764394)(828.00644531,344.77765076)
\curveto(827.83644196,344.75764398)(827.6914421,344.72764401)(827.57144531,344.68765076)
\curveto(827.40144239,344.62764411)(827.23644256,344.5426442)(827.07644531,344.43265076)
\curveto(826.9964428,344.37264437)(826.92144287,344.29264445)(826.85144531,344.19265076)
\curveto(826.791443,344.10264464)(826.73644306,344.00264474)(826.68644531,343.89265076)
\curveto(826.65644314,343.81264493)(826.62644317,343.72764501)(826.59644531,343.63765076)
\curveto(826.57644322,343.54764519)(826.53144326,343.47764526)(826.46144531,343.42765076)
\curveto(826.42144337,343.39764534)(826.35144344,343.37264537)(826.25144531,343.35265076)
\curveto(826.16144363,343.3426454)(826.06644373,343.3376454)(825.96644531,343.33765076)
\curveto(825.86644393,343.3376454)(825.76644403,343.3426454)(825.66644531,343.35265076)
\curveto(825.57644422,343.37264537)(825.51144428,343.39764534)(825.47144531,343.42765076)
\curveto(825.43144436,343.45764528)(825.40144439,343.50764523)(825.38144531,343.57765076)
\curveto(825.36144443,343.64764509)(825.36144443,343.72264502)(825.38144531,343.80265076)
\curveto(825.41144438,343.93264481)(825.44144435,344.05264469)(825.47144531,344.16265076)
\curveto(825.51144428,344.28264446)(825.55644424,344.39764434)(825.60644531,344.50765076)
\curveto(825.796444,344.85764388)(826.03644376,345.12764361)(826.32644531,345.31765076)
\curveto(826.61644318,345.51764322)(826.97644282,345.67764306)(827.40644531,345.79765076)
\curveto(827.50644229,345.81764292)(827.60644219,345.83264291)(827.70644531,345.84265076)
\curveto(827.81644198,345.85264289)(827.92644187,345.86764287)(828.03644531,345.88765076)
\curveto(828.07644172,345.89764284)(828.14144165,345.89764284)(828.23144531,345.88765076)
\curveto(828.32144147,345.88764285)(828.37644142,345.89764284)(828.39644531,345.91765076)
\curveto(829.0964407,345.92764281)(829.70644009,345.84764289)(830.22644531,345.67765076)
\curveto(830.74643905,345.50764323)(831.11143868,345.18264356)(831.32144531,344.70265076)
\curveto(831.41143838,344.50264424)(831.46143833,344.26764447)(831.47144531,343.99765076)
\curveto(831.4914383,343.737645)(831.50143829,343.46264528)(831.50144531,343.17265076)
\lineto(831.50144531,339.85765076)
\curveto(831.50143829,339.71764902)(831.50643829,339.58264916)(831.51644531,339.45265076)
\curveto(831.52643827,339.32264942)(831.55643824,339.21764952)(831.60644531,339.13765076)
\curveto(831.65643814,339.06764967)(831.72143807,339.01764972)(831.80144531,338.98765076)
\curveto(831.8914379,338.94764979)(831.97643782,338.91764982)(832.05644531,338.89765076)
\curveto(832.13643766,338.88764985)(832.1964376,338.8426499)(832.23644531,338.76265076)
\curveto(832.25643754,338.73265001)(832.26643753,338.70265004)(832.26644531,338.67265076)
\curveto(832.26643753,338.6426501)(832.27143752,338.60265014)(832.28144531,338.55265076)
\moveto(830.13644531,340.21765076)
\curveto(830.1964396,340.35764838)(830.22643957,340.51764822)(830.22644531,340.69765076)
\curveto(830.23643956,340.88764785)(830.24143955,341.08264766)(830.24144531,341.28265076)
\curveto(830.24143955,341.39264735)(830.23643956,341.49264725)(830.22644531,341.58265076)
\curveto(830.21643958,341.67264707)(830.17643962,341.742647)(830.10644531,341.79265076)
\curveto(830.07643972,341.81264693)(830.00643979,341.82264692)(829.89644531,341.82265076)
\curveto(829.87643992,341.80264694)(829.84143995,341.79264695)(829.79144531,341.79265076)
\curveto(829.74144005,341.79264695)(829.6964401,341.78264696)(829.65644531,341.76265076)
\curveto(829.57644022,341.742647)(829.48644031,341.72264702)(829.38644531,341.70265076)
\lineto(829.08644531,341.64265076)
\curveto(829.05644074,341.6426471)(829.02144077,341.6376471)(828.98144531,341.62765076)
\lineto(828.87644531,341.62765076)
\curveto(828.72644107,341.58764715)(828.56144123,341.56264718)(828.38144531,341.55265076)
\curveto(828.21144158,341.55264719)(828.05144174,341.53264721)(827.90144531,341.49265076)
\curveto(827.82144197,341.47264727)(827.74644205,341.45264729)(827.67644531,341.43265076)
\curveto(827.61644218,341.42264732)(827.54644225,341.40764733)(827.46644531,341.38765076)
\curveto(827.30644249,341.3376474)(827.15644264,341.27264747)(827.01644531,341.19265076)
\curveto(826.87644292,341.12264762)(826.75644304,341.03264771)(826.65644531,340.92265076)
\curveto(826.55644324,340.81264793)(826.48144331,340.67764806)(826.43144531,340.51765076)
\curveto(826.38144341,340.36764837)(826.36144343,340.18264856)(826.37144531,339.96265076)
\curveto(826.37144342,339.86264888)(826.38644341,339.76764897)(826.41644531,339.67765076)
\curveto(826.45644334,339.59764914)(826.50144329,339.52264922)(826.55144531,339.45265076)
\curveto(826.63144316,339.3426494)(826.73644306,339.24764949)(826.86644531,339.16765076)
\curveto(826.9964428,339.09764964)(827.13644266,339.0376497)(827.28644531,338.98765076)
\curveto(827.33644246,338.97764976)(827.38644241,338.97264977)(827.43644531,338.97265076)
\curveto(827.48644231,338.97264977)(827.53644226,338.96764977)(827.58644531,338.95765076)
\curveto(827.65644214,338.9376498)(827.74144205,338.92264982)(827.84144531,338.91265076)
\curveto(827.95144184,338.91264983)(828.04144175,338.92264982)(828.11144531,338.94265076)
\curveto(828.17144162,338.96264978)(828.23144156,338.96764977)(828.29144531,338.95765076)
\curveto(828.35144144,338.95764978)(828.41144138,338.96764977)(828.47144531,338.98765076)
\curveto(828.55144124,339.00764973)(828.62644117,339.02264972)(828.69644531,339.03265076)
\curveto(828.77644102,339.0426497)(828.85144094,339.06264968)(828.92144531,339.09265076)
\curveto(829.21144058,339.21264953)(829.45644034,339.35764938)(829.65644531,339.52765076)
\curveto(829.86643993,339.69764904)(830.02643977,339.92764881)(830.13644531,340.21765076)
}
}
{
\newrgbcolor{curcolor}{0 0 0}
\pscustom[linestyle=none,fillstyle=solid,fillcolor=curcolor]
{
\newpath
\moveto(772.41642334,325.34634644)
\curveto(773.39641684,325.36633548)(774.21641602,325.20633564)(774.87642334,324.86634644)
\curveto(775.54641469,324.53633631)(776.06641417,324.07633677)(776.43642334,323.48634644)
\curveto(776.5364137,323.32633752)(776.61641362,323.17133768)(776.67642334,323.02134644)
\curveto(776.74641349,322.88133797)(776.81141342,322.71133814)(776.87142334,322.51134644)
\curveto(776.89141334,322.46133839)(776.91141332,322.39133846)(776.93142334,322.30134644)
\curveto(776.95141328,322.22133863)(776.94641329,322.1463387)(776.91642334,322.07634644)
\curveto(776.89641334,322.01633883)(776.85641338,321.97633887)(776.79642334,321.95634644)
\curveto(776.74641349,321.9463389)(776.69141354,321.93133892)(776.63142334,321.91134644)
\lineto(776.48142334,321.91134644)
\curveto(776.45141378,321.90133895)(776.41141382,321.89633895)(776.36142334,321.89634644)
\lineto(776.24142334,321.89634644)
\curveto(776.10141413,321.89633895)(775.97141426,321.90133895)(775.85142334,321.91134644)
\curveto(775.74141449,321.93133892)(775.66141457,321.98133887)(775.61142334,322.06134644)
\curveto(775.54141469,322.16133869)(775.48641475,322.27633857)(775.44642334,322.40634644)
\curveto(775.40641483,322.53633831)(775.35141488,322.65633819)(775.28142334,322.76634644)
\curveto(775.15141508,322.98633786)(775.00141523,323.17633767)(774.83142334,323.33634644)
\curveto(774.67141556,323.49633735)(774.48141575,323.6463372)(774.26142334,323.78634644)
\curveto(774.14141609,323.86633698)(774.00641623,323.92633692)(773.85642334,323.96634644)
\curveto(773.71641652,324.00633684)(773.57141666,324.0463368)(773.42142334,324.08634644)
\curveto(773.31141692,324.11633673)(773.18641705,324.13633671)(773.04642334,324.14634644)
\curveto(772.90641733,324.16633668)(772.75641748,324.17633667)(772.59642334,324.17634644)
\curveto(772.44641779,324.17633667)(772.29641794,324.16633668)(772.14642334,324.14634644)
\curveto(772.00641823,324.13633671)(771.88641835,324.11633673)(771.78642334,324.08634644)
\curveto(771.68641855,324.06633678)(771.59141864,324.0463368)(771.50142334,324.02634644)
\curveto(771.41141882,324.00633684)(771.32141891,323.97633687)(771.23142334,323.93634644)
\curveto(770.39141984,323.58633726)(769.78642045,322.98633786)(769.41642334,322.13634644)
\curveto(769.34642089,321.99633885)(769.28642095,321.846339)(769.23642334,321.68634644)
\curveto(769.19642104,321.53633931)(769.15142108,321.38133947)(769.10142334,321.22134644)
\curveto(769.08142115,321.16133969)(769.07142116,321.09633975)(769.07142334,321.02634644)
\curveto(769.07142116,320.96633988)(769.06142117,320.90633994)(769.04142334,320.84634644)
\curveto(769.0314212,320.80634004)(769.02642121,320.77134008)(769.02642334,320.74134644)
\curveto(769.02642121,320.71134014)(769.02142121,320.67634017)(769.01142334,320.63634644)
\curveto(768.99142124,320.52634032)(768.97642126,320.41134044)(768.96642334,320.29134644)
\lineto(768.96642334,319.94634644)
\curveto(768.96642127,319.87634097)(768.96142127,319.80134105)(768.95142334,319.72134644)
\curveto(768.95142128,319.6513412)(768.95642128,319.58634126)(768.96642334,319.52634644)
\lineto(768.96642334,319.37634644)
\curveto(768.98642125,319.30634154)(768.99142124,319.23634161)(768.98142334,319.16634644)
\curveto(768.98142125,319.09634175)(768.99142124,319.02634182)(769.01142334,318.95634644)
\curveto(769.0314212,318.89634195)(769.0364212,318.83634201)(769.02642334,318.77634644)
\curveto(769.02642121,318.71634213)(769.0364212,318.66134219)(769.05642334,318.61134644)
\curveto(769.08642115,318.48134237)(769.11142112,318.3513425)(769.13142334,318.22134644)
\curveto(769.16142107,318.10134275)(769.19642104,317.98134287)(769.23642334,317.86134644)
\curveto(769.40642083,317.36134349)(769.62642061,316.93134392)(769.89642334,316.57134644)
\curveto(770.16642007,316.22134463)(770.52141971,315.93134492)(770.96142334,315.70134644)
\curveto(771.10141913,315.63134522)(771.24141899,315.57634527)(771.38142334,315.53634644)
\curveto(771.5314187,315.49634535)(771.69141854,315.4513454)(771.86142334,315.40134644)
\curveto(771.9314183,315.38134547)(771.99641824,315.37134548)(772.05642334,315.37134644)
\curveto(772.11641812,315.38134547)(772.18641805,315.37634547)(772.26642334,315.35634644)
\curveto(772.31641792,315.3463455)(772.40641783,315.33634551)(772.53642334,315.32634644)
\curveto(772.66641757,315.32634552)(772.76141747,315.33634551)(772.82142334,315.35634644)
\lineto(772.92642334,315.35634644)
\curveto(772.96641727,315.36634548)(773.00641723,315.36634548)(773.04642334,315.35634644)
\curveto(773.08641715,315.35634549)(773.12641711,315.36634548)(773.16642334,315.38634644)
\curveto(773.26641697,315.40634544)(773.36141687,315.42134543)(773.45142334,315.43134644)
\curveto(773.55141668,315.4513454)(773.64641659,315.48134537)(773.73642334,315.52134644)
\curveto(774.51641572,315.84134501)(775.06641517,316.36634448)(775.38642334,317.09634644)
\curveto(775.46641477,317.27634357)(775.54141469,317.49134336)(775.61142334,317.74134644)
\curveto(775.6314146,317.83134302)(775.64641459,317.92134293)(775.65642334,318.01134644)
\curveto(775.67641456,318.11134274)(775.71141452,318.20134265)(775.76142334,318.28134644)
\curveto(775.81141442,318.36134249)(775.89141434,318.40634244)(776.00142334,318.41634644)
\curveto(776.11141412,318.42634242)(776.231414,318.43134242)(776.36142334,318.43134644)
\lineto(776.51142334,318.43134644)
\curveto(776.56141367,318.43134242)(776.60641363,318.42634242)(776.64642334,318.41634644)
\lineto(776.75142334,318.41634644)
\lineto(776.84142334,318.38634644)
\curveto(776.88141335,318.38634246)(776.91141332,318.37634247)(776.93142334,318.35634644)
\curveto(777.00141323,318.31634253)(777.04141319,318.24134261)(777.05142334,318.13134644)
\curveto(777.06141317,318.03134282)(777.05141318,317.93134292)(777.02142334,317.83134644)
\curveto(776.96141327,317.60134325)(776.90641333,317.38134347)(776.85642334,317.17134644)
\curveto(776.80641343,316.96134389)(776.7314135,316.76134409)(776.63142334,316.57134644)
\curveto(776.55141368,316.44134441)(776.47641376,316.31634453)(776.40642334,316.19634644)
\curveto(776.34641389,316.07634477)(776.27641396,315.95634489)(776.19642334,315.83634644)
\curveto(776.01641422,315.57634527)(775.79141444,315.33634551)(775.52142334,315.11634644)
\curveto(775.26141497,314.90634594)(774.97641526,314.73134612)(774.66642334,314.59134644)
\curveto(774.55641568,314.54134631)(774.44641579,314.50134635)(774.33642334,314.47134644)
\curveto(774.236416,314.44134641)(774.1314161,314.40634644)(774.02142334,314.36634644)
\curveto(773.91141632,314.32634652)(773.79641644,314.30134655)(773.67642334,314.29134644)
\curveto(773.56641667,314.27134658)(773.45141678,314.2513466)(773.33142334,314.23134644)
\curveto(773.28141695,314.21134664)(773.236417,314.20634664)(773.19642334,314.21634644)
\curveto(773.15641708,314.21634663)(773.11641712,314.21134664)(773.07642334,314.20134644)
\curveto(773.01641722,314.19134666)(772.95641728,314.18634666)(772.89642334,314.18634644)
\curveto(772.8364174,314.18634666)(772.77141746,314.18134667)(772.70142334,314.17134644)
\curveto(772.67141756,314.16134669)(772.60141763,314.16134669)(772.49142334,314.17134644)
\curveto(772.39141784,314.17134668)(772.32641791,314.17634667)(772.29642334,314.18634644)
\curveto(772.24641799,314.19634665)(772.19641804,314.20134665)(772.14642334,314.20134644)
\curveto(772.10641813,314.19134666)(772.06141817,314.19134666)(772.01142334,314.20134644)
\lineto(771.86142334,314.20134644)
\curveto(771.78141845,314.22134663)(771.70641853,314.23634661)(771.63642334,314.24634644)
\curveto(771.56641867,314.2463466)(771.49141874,314.25634659)(771.41142334,314.27634644)
\lineto(771.14142334,314.33634644)
\curveto(771.05141918,314.3463465)(770.96641927,314.36634648)(770.88642334,314.39634644)
\curveto(770.67641956,314.45634639)(770.48641975,314.53134632)(770.31642334,314.62134644)
\curveto(769.68642055,314.89134596)(769.17642106,315.27634557)(768.78642334,315.77634644)
\curveto(768.39642184,316.27634457)(768.08642215,316.86634398)(767.85642334,317.54634644)
\curveto(767.81642242,317.66634318)(767.78142245,317.79134306)(767.75142334,317.92134644)
\curveto(767.7314225,318.0513428)(767.70642253,318.18634266)(767.67642334,318.32634644)
\curveto(767.65642258,318.37634247)(767.64642259,318.42134243)(767.64642334,318.46134644)
\curveto(767.65642258,318.50134235)(767.65642258,318.5463423)(767.64642334,318.59634644)
\curveto(767.62642261,318.68634216)(767.61142262,318.78134207)(767.60142334,318.88134644)
\curveto(767.60142263,318.98134187)(767.59142264,319.07634177)(767.57142334,319.16634644)
\lineto(767.57142334,319.45134644)
\curveto(767.55142268,319.50134135)(767.54142269,319.58634126)(767.54142334,319.70634644)
\curveto(767.54142269,319.82634102)(767.55142268,319.91134094)(767.57142334,319.96134644)
\curveto(767.58142265,319.99134086)(767.58142265,320.02134083)(767.57142334,320.05134644)
\curveto(767.56142267,320.09134076)(767.56142267,320.12134073)(767.57142334,320.14134644)
\lineto(767.57142334,320.27634644)
\curveto(767.58142265,320.35634049)(767.58642265,320.43634041)(767.58642334,320.51634644)
\curveto(767.59642264,320.60634024)(767.61142262,320.69134016)(767.63142334,320.77134644)
\curveto(767.65142258,320.83134002)(767.66142257,320.89133996)(767.66142334,320.95134644)
\curveto(767.66142257,321.02133983)(767.67142256,321.09133976)(767.69142334,321.16134644)
\curveto(767.74142249,321.33133952)(767.78142245,321.49633935)(767.81142334,321.65634644)
\curveto(767.84142239,321.81633903)(767.88642235,321.96633888)(767.94642334,322.10634644)
\lineto(768.09642334,322.49634644)
\curveto(768.15642208,322.63633821)(768.22142201,322.76133809)(768.29142334,322.87134644)
\curveto(768.44142179,323.13133772)(768.59142164,323.36633748)(768.74142334,323.57634644)
\curveto(768.77142146,323.62633722)(768.80642143,323.66633718)(768.84642334,323.69634644)
\curveto(768.89642134,323.73633711)(768.9364213,323.78133707)(768.96642334,323.83134644)
\curveto(769.02642121,323.91133694)(769.08642115,323.98133687)(769.14642334,324.04134644)
\lineto(769.35642334,324.22134644)
\curveto(769.41642082,324.27133658)(769.47142076,324.31633653)(769.52142334,324.35634644)
\curveto(769.58142065,324.40633644)(769.64642059,324.45633639)(769.71642334,324.50634644)
\curveto(769.86642037,324.61633623)(770.02142021,324.71133614)(770.18142334,324.79134644)
\curveto(770.35141988,324.87133598)(770.52641971,324.9513359)(770.70642334,325.03134644)
\curveto(770.81641942,325.08133577)(770.9314193,325.11633573)(771.05142334,325.13634644)
\curveto(771.18141905,325.16633568)(771.30641893,325.20133565)(771.42642334,325.24134644)
\curveto(771.49641874,325.2513356)(771.56141867,325.26133559)(771.62142334,325.27134644)
\lineto(771.80142334,325.30134644)
\curveto(771.88141835,325.31133554)(771.95641828,325.31633553)(772.02642334,325.31634644)
\curveto(772.10641813,325.32633552)(772.18641805,325.33633551)(772.26642334,325.34634644)
\curveto(772.28641795,325.35633549)(772.31141792,325.35633549)(772.34142334,325.34634644)
\curveto(772.37141786,325.33633551)(772.39641784,325.33633551)(772.41642334,325.34634644)
}
}
{
\newrgbcolor{curcolor}{0 0 0}
\pscustom[linestyle=none,fillstyle=solid,fillcolor=curcolor]
{
\newpath
\moveto(785.77626709,318.62634644)
\curveto(785.79625903,318.56634228)(785.80625902,318.47134238)(785.80626709,318.34134644)
\curveto(785.80625902,318.22134263)(785.80125902,318.13634271)(785.79126709,318.08634644)
\lineto(785.79126709,317.93634644)
\curveto(785.78125904,317.85634299)(785.77125905,317.78134307)(785.76126709,317.71134644)
\curveto(785.76125906,317.6513432)(785.75625907,317.58134327)(785.74626709,317.50134644)
\curveto(785.7262591,317.44134341)(785.71125911,317.38134347)(785.70126709,317.32134644)
\curveto(785.70125912,317.26134359)(785.69125913,317.20134365)(785.67126709,317.14134644)
\curveto(785.63125919,317.01134384)(785.59625923,316.88134397)(785.56626709,316.75134644)
\curveto(785.53625929,316.62134423)(785.49625933,316.50134435)(785.44626709,316.39134644)
\curveto(785.23625959,315.91134494)(784.95625987,315.50634534)(784.60626709,315.17634644)
\curveto(784.25626057,314.85634599)(783.826261,314.61134624)(783.31626709,314.44134644)
\curveto(783.20626162,314.40134645)(783.08626174,314.37134648)(782.95626709,314.35134644)
\curveto(782.83626199,314.33134652)(782.71126211,314.31134654)(782.58126709,314.29134644)
\curveto(782.5212623,314.28134657)(782.45626237,314.27634657)(782.38626709,314.27634644)
\curveto(782.3262625,314.26634658)(782.26626256,314.26134659)(782.20626709,314.26134644)
\curveto(782.16626266,314.2513466)(782.10626272,314.2463466)(782.02626709,314.24634644)
\curveto(781.95626287,314.2463466)(781.90626292,314.2513466)(781.87626709,314.26134644)
\curveto(781.83626299,314.27134658)(781.79626303,314.27634657)(781.75626709,314.27634644)
\curveto(781.71626311,314.26634658)(781.68126314,314.26634658)(781.65126709,314.27634644)
\lineto(781.56126709,314.27634644)
\lineto(781.20126709,314.32134644)
\curveto(781.06126376,314.36134649)(780.9262639,314.40134645)(780.79626709,314.44134644)
\curveto(780.66626416,314.48134637)(780.54126428,314.52634632)(780.42126709,314.57634644)
\curveto(779.97126485,314.77634607)(779.60126522,315.03634581)(779.31126709,315.35634644)
\curveto(779.0212658,315.67634517)(778.78126604,316.06634478)(778.59126709,316.52634644)
\curveto(778.54126628,316.62634422)(778.50126632,316.72634412)(778.47126709,316.82634644)
\curveto(778.45126637,316.92634392)(778.43126639,317.03134382)(778.41126709,317.14134644)
\curveto(778.39126643,317.18134367)(778.38126644,317.21134364)(778.38126709,317.23134644)
\curveto(778.39126643,317.26134359)(778.39126643,317.29634355)(778.38126709,317.33634644)
\curveto(778.36126646,317.41634343)(778.34626648,317.49634335)(778.33626709,317.57634644)
\curveto(778.33626649,317.66634318)(778.3262665,317.7513431)(778.30626709,317.83134644)
\lineto(778.30626709,317.95134644)
\curveto(778.30626652,317.99134286)(778.30126652,318.03634281)(778.29126709,318.08634644)
\curveto(778.28126654,318.13634271)(778.27626655,318.22134263)(778.27626709,318.34134644)
\curveto(778.27626655,318.47134238)(778.28626654,318.56634228)(778.30626709,318.62634644)
\curveto(778.3262665,318.69634215)(778.33126649,318.76634208)(778.32126709,318.83634644)
\curveto(778.31126651,318.90634194)(778.31626651,318.97634187)(778.33626709,319.04634644)
\curveto(778.34626648,319.09634175)(778.35126647,319.13634171)(778.35126709,319.16634644)
\curveto(778.36126646,319.20634164)(778.37126645,319.2513416)(778.38126709,319.30134644)
\curveto(778.41126641,319.42134143)(778.43626639,319.54134131)(778.45626709,319.66134644)
\curveto(778.48626634,319.78134107)(778.5262663,319.89634095)(778.57626709,320.00634644)
\curveto(778.7262661,320.37634047)(778.90626592,320.70634014)(779.11626709,320.99634644)
\curveto(779.33626549,321.29633955)(779.60126522,321.5463393)(779.91126709,321.74634644)
\curveto(780.03126479,321.82633902)(780.15626467,321.89133896)(780.28626709,321.94134644)
\curveto(780.41626441,322.00133885)(780.55126427,322.06133879)(780.69126709,322.12134644)
\curveto(780.81126401,322.17133868)(780.94126388,322.20133865)(781.08126709,322.21134644)
\curveto(781.2212636,322.23133862)(781.36126346,322.26133859)(781.50126709,322.30134644)
\lineto(781.69626709,322.30134644)
\curveto(781.76626306,322.31133854)(781.83126299,322.32133853)(781.89126709,322.33134644)
\curveto(782.78126204,322.34133851)(783.5212613,322.15633869)(784.11126709,321.77634644)
\curveto(784.70126012,321.39633945)(785.1262597,320.90133995)(785.38626709,320.29134644)
\curveto(785.43625939,320.19134066)(785.47625935,320.09134076)(785.50626709,319.99134644)
\curveto(785.53625929,319.89134096)(785.57125925,319.78634106)(785.61126709,319.67634644)
\curveto(785.64125918,319.56634128)(785.66625916,319.4463414)(785.68626709,319.31634644)
\curveto(785.70625912,319.19634165)(785.73125909,319.07134178)(785.76126709,318.94134644)
\curveto(785.77125905,318.89134196)(785.77125905,318.83634201)(785.76126709,318.77634644)
\curveto(785.76125906,318.72634212)(785.76625906,318.67634217)(785.77626709,318.62634644)
\moveto(784.44126709,317.77134644)
\curveto(784.46126036,317.84134301)(784.46626036,317.92134293)(784.45626709,318.01134644)
\lineto(784.45626709,318.26634644)
\curveto(784.45626037,318.65634219)(784.4212604,318.98634186)(784.35126709,319.25634644)
\curveto(784.3212605,319.33634151)(784.29626053,319.41634143)(784.27626709,319.49634644)
\curveto(784.25626057,319.57634127)(784.23126059,319.6513412)(784.20126709,319.72134644)
\curveto(783.9212609,320.37134048)(783.47626135,320.82134003)(782.86626709,321.07134644)
\curveto(782.79626203,321.10133975)(782.7212621,321.12133973)(782.64126709,321.13134644)
\lineto(782.40126709,321.19134644)
\curveto(782.3212625,321.21133964)(782.23626259,321.22133963)(782.14626709,321.22134644)
\lineto(781.87626709,321.22134644)
\lineto(781.60626709,321.17634644)
\curveto(781.50626332,321.15633969)(781.41126341,321.13133972)(781.32126709,321.10134644)
\curveto(781.24126358,321.08133977)(781.16126366,321.0513398)(781.08126709,321.01134644)
\curveto(781.01126381,320.99133986)(780.94626388,320.96133989)(780.88626709,320.92134644)
\curveto(780.826264,320.88133997)(780.77126405,320.84134001)(780.72126709,320.80134644)
\curveto(780.48126434,320.63134022)(780.28626454,320.42634042)(780.13626709,320.18634644)
\curveto(779.98626484,319.9463409)(779.85626497,319.66634118)(779.74626709,319.34634644)
\curveto(779.71626511,319.2463416)(779.69626513,319.14134171)(779.68626709,319.03134644)
\curveto(779.67626515,318.93134192)(779.66126516,318.82634202)(779.64126709,318.71634644)
\curveto(779.63126519,318.67634217)(779.6262652,318.61134224)(779.62626709,318.52134644)
\curveto(779.61626521,318.49134236)(779.61126521,318.45634239)(779.61126709,318.41634644)
\curveto(779.6212652,318.37634247)(779.6262652,318.33134252)(779.62626709,318.28134644)
\lineto(779.62626709,317.98134644)
\curveto(779.6262652,317.88134297)(779.63626519,317.79134306)(779.65626709,317.71134644)
\lineto(779.68626709,317.53134644)
\curveto(779.70626512,317.43134342)(779.7212651,317.33134352)(779.73126709,317.23134644)
\curveto(779.75126507,317.14134371)(779.78126504,317.05634379)(779.82126709,316.97634644)
\curveto(779.9212649,316.73634411)(780.03626479,316.51134434)(780.16626709,316.30134644)
\curveto(780.30626452,316.09134476)(780.47626435,315.91634493)(780.67626709,315.77634644)
\curveto(780.7262641,315.7463451)(780.77126405,315.72134513)(780.81126709,315.70134644)
\curveto(780.85126397,315.68134517)(780.89626393,315.65634519)(780.94626709,315.62634644)
\curveto(781.0262638,315.57634527)(781.11126371,315.53134532)(781.20126709,315.49134644)
\curveto(781.30126352,315.46134539)(781.40626342,315.43134542)(781.51626709,315.40134644)
\curveto(781.56626326,315.38134547)(781.61126321,315.37134548)(781.65126709,315.37134644)
\curveto(781.70126312,315.38134547)(781.75126307,315.38134547)(781.80126709,315.37134644)
\curveto(781.83126299,315.36134549)(781.89126293,315.3513455)(781.98126709,315.34134644)
\curveto(782.08126274,315.33134552)(782.15626267,315.33634551)(782.20626709,315.35634644)
\curveto(782.24626258,315.36634548)(782.28626254,315.36634548)(782.32626709,315.35634644)
\curveto(782.36626246,315.35634549)(782.40626242,315.36634548)(782.44626709,315.38634644)
\curveto(782.5262623,315.40634544)(782.60626222,315.42134543)(782.68626709,315.43134644)
\curveto(782.76626206,315.4513454)(782.84126198,315.47634537)(782.91126709,315.50634644)
\curveto(783.25126157,315.6463452)(783.5262613,315.84134501)(783.73626709,316.09134644)
\curveto(783.94626088,316.34134451)(784.1212607,316.63634421)(784.26126709,316.97634644)
\curveto(784.31126051,317.09634375)(784.34126048,317.22134363)(784.35126709,317.35134644)
\curveto(784.37126045,317.49134336)(784.40126042,317.63134322)(784.44126709,317.77134644)
}
}
{
\newrgbcolor{curcolor}{0 0 0}
\pscustom[linestyle=none,fillstyle=solid,fillcolor=curcolor]
{
\newpath
\moveto(790.95454834,322.33134644)
\curveto(791.33454335,322.34133851)(791.65454303,322.30133855)(791.91454834,322.21134644)
\curveto(792.1845425,322.12133873)(792.42954226,321.99133886)(792.64954834,321.82134644)
\curveto(792.72954196,321.77133908)(792.79454189,321.70133915)(792.84454834,321.61134644)
\curveto(792.90454178,321.53133932)(792.96954172,321.45633939)(793.03954834,321.38634644)
\curveto(793.05954163,321.36633948)(793.0895416,321.34133951)(793.12954834,321.31134644)
\curveto(793.16954152,321.28133957)(793.21954147,321.27133958)(793.27954834,321.28134644)
\curveto(793.37954131,321.31133954)(793.46454122,321.37133948)(793.53454834,321.46134644)
\curveto(793.61454107,321.56133929)(793.69454099,321.63633921)(793.77454834,321.68634644)
\curveto(793.91454077,321.79633905)(794.05954063,321.89133896)(794.20954834,321.97134644)
\curveto(794.35954033,322.06133879)(794.52454016,322.13633871)(794.70454834,322.19634644)
\curveto(794.7845399,322.22633862)(794.86953982,322.2463386)(794.95954834,322.25634644)
\curveto(795.05953963,322.27633857)(795.15453953,322.29633855)(795.24454834,322.31634644)
\curveto(795.29453939,322.32633852)(795.33953935,322.33133852)(795.37954834,322.33134644)
\lineto(795.52954834,322.33134644)
\curveto(795.57953911,322.3513385)(795.64953904,322.35633849)(795.73954834,322.34634644)
\curveto(795.82953886,322.3463385)(795.89453879,322.34133851)(795.93454834,322.33134644)
\curveto(795.9845387,322.32133853)(796.05953863,322.31633853)(796.15954834,322.31634644)
\curveto(796.24953844,322.29633855)(796.33453835,322.27633857)(796.41454834,322.25634644)
\curveto(796.50453818,322.2463386)(796.5895381,322.22633862)(796.66954834,322.19634644)
\curveto(796.71953797,322.17633867)(796.76453792,322.16133869)(796.80454834,322.15134644)
\curveto(796.85453783,322.1513387)(796.90453778,322.14133871)(796.95454834,322.12134644)
\curveto(797.45453723,321.90133895)(797.79953689,321.56133929)(797.98954834,321.10134644)
\curveto(798.02953666,321.02133983)(798.05953663,320.93133992)(798.07954834,320.83134644)
\curveto(798.09953659,320.74134011)(798.11953657,320.64134021)(798.13954834,320.53134644)
\curveto(798.15953653,320.50134035)(798.16453652,320.46634038)(798.15454834,320.42634644)
\curveto(798.15453653,320.39634045)(798.15953653,320.36634048)(798.16954834,320.33634644)
\lineto(798.16954834,320.20134644)
\curveto(798.17953651,320.16134069)(798.17953651,320.11634073)(798.16954834,320.06634644)
\curveto(798.16953652,320.01634083)(798.16953652,319.96634088)(798.16954834,319.91634644)
\lineto(798.16954834,319.33134644)
\lineto(798.16954834,318.37134644)
\lineto(798.16954834,315.52134644)
\curveto(798.16953652,315.36134549)(798.16953652,315.17134568)(798.16954834,314.95134644)
\curveto(798.17953651,314.73134612)(798.13953655,314.58634626)(798.04954834,314.51634644)
\curveto(798.00953668,314.48634636)(797.94453674,314.46134639)(797.85454834,314.44134644)
\curveto(797.76453692,314.43134642)(797.66953702,314.42634642)(797.56954834,314.42634644)
\curveto(797.46953722,314.42634642)(797.36953732,314.43134642)(797.26954834,314.44134644)
\curveto(797.17953751,314.4513464)(797.11453757,314.47134638)(797.07454834,314.50134644)
\curveto(797.01453767,314.53134632)(796.97453771,314.59134626)(796.95454834,314.68134644)
\curveto(796.93453775,314.74134611)(796.92953776,314.80134605)(796.93954834,314.86134644)
\curveto(796.94953774,314.93134592)(796.94453774,314.99634585)(796.92454834,315.05634644)
\curveto(796.91453777,315.10634574)(796.90953778,315.16134569)(796.90954834,315.22134644)
\curveto(796.91953777,315.29134556)(796.92453776,315.35634549)(796.92454834,315.41634644)
\lineto(796.92454834,316.09134644)
\lineto(796.92454834,318.95634644)
\curveto(796.92453776,319.28634156)(796.91453777,319.59634125)(796.89454834,319.88634644)
\curveto(796.8845378,320.18634066)(796.81453787,320.43634041)(796.68454834,320.63634644)
\curveto(796.53453815,320.87633997)(796.30453838,321.0513398)(795.99454834,321.16134644)
\curveto(795.93453875,321.18133967)(795.86953882,321.19133966)(795.79954834,321.19134644)
\curveto(795.73953895,321.20133965)(795.67453901,321.21633963)(795.60454834,321.23634644)
\curveto(795.56453912,321.2463396)(795.49953919,321.2463396)(795.40954834,321.23634644)
\curveto(795.31953937,321.23633961)(795.25953943,321.23133962)(795.22954834,321.22134644)
\curveto(795.17953951,321.21133964)(795.12953956,321.20633964)(795.07954834,321.20634644)
\curveto(795.02953966,321.21633963)(794.97953971,321.21133964)(794.92954834,321.19134644)
\curveto(794.7895399,321.16133969)(794.65454003,321.12133973)(794.52454834,321.07134644)
\curveto(794.00454068,320.85134)(793.65454103,320.46634038)(793.47454834,319.91634644)
\curveto(793.42454126,319.7463411)(793.39454129,319.5513413)(793.38454834,319.33134644)
\lineto(793.38454834,318.65634644)
\lineto(793.38454834,316.69134644)
\lineto(793.38454834,315.23634644)
\lineto(793.38454834,314.86134644)
\curveto(793.3845413,314.74134611)(793.35954133,314.6463462)(793.30954834,314.57634644)
\curveto(793.25954143,314.49634635)(793.17454151,314.4513464)(793.05454834,314.44134644)
\curveto(792.93454175,314.43134642)(792.80954188,314.42634642)(792.67954834,314.42634644)
\curveto(792.50954218,314.42634642)(792.3845423,314.4463464)(792.30454834,314.48634644)
\curveto(792.21454247,314.53634631)(792.15954253,314.61634623)(792.13954834,314.72634644)
\curveto(792.12954256,314.846346)(792.12454256,314.97634587)(792.12454834,315.11634644)
\lineto(792.12454834,316.54134644)
\lineto(792.12454834,319.01634644)
\curveto(792.12454256,319.33634151)(792.11454257,319.63134122)(792.09454834,319.90134644)
\curveto(792.07454261,320.18134067)(792.00454268,320.42134043)(791.88454834,320.62134644)
\curveto(791.77454291,320.80134005)(791.64954304,320.93133992)(791.50954834,321.01134644)
\curveto(791.36954332,321.10133975)(791.17954351,321.17133968)(790.93954834,321.22134644)
\curveto(790.89954379,321.23133962)(790.85454383,321.23633961)(790.80454834,321.23634644)
\lineto(790.66954834,321.23634644)
\curveto(790.44954424,321.23633961)(790.25454443,321.21133964)(790.08454834,321.16134644)
\curveto(789.92454476,321.11133974)(789.77954491,321.0463398)(789.64954834,320.96634644)
\curveto(789.13954555,320.65634019)(788.79954589,320.19134066)(788.62954834,319.57134644)
\curveto(788.5895461,319.44134141)(788.56954612,319.29134156)(788.56954834,319.12134644)
\curveto(788.57954611,318.96134189)(788.5845461,318.80134205)(788.58454834,318.64134644)
\lineto(788.58454834,316.94634644)
\lineto(788.58454834,315.29634644)
\lineto(788.58454834,314.89134644)
\curveto(788.5845461,314.7513461)(788.55454613,314.64134621)(788.49454834,314.56134644)
\curveto(788.44454624,314.49134636)(788.36954632,314.4513464)(788.26954834,314.44134644)
\curveto(788.16954652,314.43134642)(788.06454662,314.42634642)(787.95454834,314.42634644)
\lineto(787.72954834,314.42634644)
\curveto(787.66954702,314.4463464)(787.60954708,314.46134639)(787.54954834,314.47134644)
\curveto(787.49954719,314.48134637)(787.45454723,314.51134634)(787.41454834,314.56134644)
\curveto(787.36454732,314.62134623)(787.33954735,314.69634615)(787.33954834,314.78634644)
\lineto(787.33954834,315.10134644)
\lineto(787.33954834,316.07634644)
\lineto(787.33954834,320.36634644)
\lineto(787.33954834,321.47634644)
\lineto(787.33954834,321.76134644)
\curveto(787.33954735,321.86133899)(787.35954733,321.94133891)(787.39954834,322.00134644)
\curveto(787.42954726,322.06133879)(787.47454721,322.10133875)(787.53454834,322.12134644)
\curveto(787.61454707,322.1513387)(787.73954695,322.16633868)(787.90954834,322.16634644)
\curveto(788.0895466,322.16633868)(788.21954647,322.1513387)(788.29954834,322.12134644)
\curveto(788.37954631,322.08133877)(788.43454625,322.03133882)(788.46454834,321.97134644)
\curveto(788.4845462,321.92133893)(788.49454619,321.86133899)(788.49454834,321.79134644)
\curveto(788.50454618,321.72133913)(788.51454617,321.65633919)(788.52454834,321.59634644)
\curveto(788.53454615,321.53633931)(788.55454613,321.48633936)(788.58454834,321.44634644)
\curveto(788.61454607,321.40633944)(788.66454602,321.38633946)(788.73454834,321.38634644)
\curveto(788.75454593,321.40633944)(788.77454591,321.41633943)(788.79454834,321.41634644)
\curveto(788.82454586,321.41633943)(788.84954584,321.42633942)(788.86954834,321.44634644)
\curveto(788.92954576,321.49633935)(788.9845457,321.5463393)(789.03454834,321.59634644)
\lineto(789.21454834,321.74634644)
\curveto(789.43454525,321.90633894)(789.684545,322.0463388)(789.96454834,322.16634644)
\curveto(790.06454462,322.20633864)(790.16454452,322.23133862)(790.26454834,322.24134644)
\curveto(790.36454432,322.26133859)(790.46954422,322.28633856)(790.57954834,322.31634644)
\lineto(790.75954834,322.31634644)
\curveto(790.82954386,322.32633852)(790.89454379,322.33133852)(790.95454834,322.33134644)
}
}
{
\newrgbcolor{curcolor}{0 0 0}
\pscustom[linestyle=none,fillstyle=solid,fillcolor=curcolor]
{
\newpath
\moveto(806.81728271,318.59634644)
\curveto(806.83727503,318.49634235)(806.83727503,318.38134247)(806.81728271,318.25134644)
\curveto(806.80727506,318.13134272)(806.77727509,318.0463428)(806.72728271,317.99634644)
\curveto(806.67727519,317.95634289)(806.60227526,317.92634292)(806.50228271,317.90634644)
\curveto(806.41227545,317.89634295)(806.30727556,317.89134296)(806.18728271,317.89134644)
\lineto(805.82728271,317.89134644)
\curveto(805.70727616,317.90134295)(805.60227626,317.90634294)(805.51228271,317.90634644)
\lineto(801.67228271,317.90634644)
\curveto(801.59228027,317.90634294)(801.51228035,317.90134295)(801.43228271,317.89134644)
\curveto(801.35228051,317.89134296)(801.28728058,317.87634297)(801.23728271,317.84634644)
\curveto(801.19728067,317.82634302)(801.15728071,317.78634306)(801.11728271,317.72634644)
\curveto(801.09728077,317.69634315)(801.07728079,317.6513432)(801.05728271,317.59134644)
\curveto(801.03728083,317.54134331)(801.03728083,317.49134336)(801.05728271,317.44134644)
\curveto(801.0672808,317.39134346)(801.07228079,317.3463435)(801.07228271,317.30634644)
\curveto(801.07228079,317.26634358)(801.07728079,317.22634362)(801.08728271,317.18634644)
\curveto(801.10728076,317.10634374)(801.12728074,317.02134383)(801.14728271,316.93134644)
\curveto(801.1672807,316.851344)(801.19728067,316.77134408)(801.23728271,316.69134644)
\curveto(801.4672804,316.1513447)(801.84728002,315.76634508)(802.37728271,315.53634644)
\curveto(802.43727943,315.50634534)(802.50227936,315.48134537)(802.57228271,315.46134644)
\lineto(802.78228271,315.40134644)
\curveto(802.81227905,315.39134546)(802.862279,315.38634546)(802.93228271,315.38634644)
\curveto(803.07227879,315.3463455)(803.25727861,315.32634552)(803.48728271,315.32634644)
\curveto(803.71727815,315.32634552)(803.90227796,315.3463455)(804.04228271,315.38634644)
\curveto(804.18227768,315.42634542)(804.30727756,315.46634538)(804.41728271,315.50634644)
\curveto(804.53727733,315.55634529)(804.64727722,315.61634523)(804.74728271,315.68634644)
\curveto(804.85727701,315.75634509)(804.95227691,315.83634501)(805.03228271,315.92634644)
\curveto(805.11227675,316.02634482)(805.18227668,316.13134472)(805.24228271,316.24134644)
\curveto(805.30227656,316.34134451)(805.35227651,316.4463444)(805.39228271,316.55634644)
\curveto(805.44227642,316.66634418)(805.52227634,316.7463441)(805.63228271,316.79634644)
\curveto(805.67227619,316.81634403)(805.73727613,316.83134402)(805.82728271,316.84134644)
\curveto(805.91727595,316.851344)(806.00727586,316.851344)(806.09728271,316.84134644)
\curveto(806.18727568,316.84134401)(806.27227559,316.83634401)(806.35228271,316.82634644)
\curveto(806.43227543,316.81634403)(806.48727538,316.79634405)(806.51728271,316.76634644)
\curveto(806.61727525,316.69634415)(806.64227522,316.58134427)(806.59228271,316.42134644)
\curveto(806.51227535,316.1513447)(806.40727546,315.91134494)(806.27728271,315.70134644)
\curveto(806.07727579,315.38134547)(805.84727602,315.11634573)(805.58728271,314.90634644)
\curveto(805.33727653,314.70634614)(805.01727685,314.54134631)(804.62728271,314.41134644)
\curveto(804.52727734,314.37134648)(804.42727744,314.3463465)(804.32728271,314.33634644)
\curveto(804.22727764,314.31634653)(804.12227774,314.29634655)(804.01228271,314.27634644)
\curveto(803.9622779,314.26634658)(803.91227795,314.26134659)(803.86228271,314.26134644)
\curveto(803.82227804,314.26134659)(803.77727809,314.25634659)(803.72728271,314.24634644)
\lineto(803.57728271,314.24634644)
\curveto(803.52727834,314.23634661)(803.4672784,314.23134662)(803.39728271,314.23134644)
\curveto(803.33727853,314.23134662)(803.28727858,314.23634661)(803.24728271,314.24634644)
\lineto(803.11228271,314.24634644)
\curveto(803.0622788,314.25634659)(803.01727885,314.26134659)(802.97728271,314.26134644)
\curveto(802.93727893,314.26134659)(802.89727897,314.26634658)(802.85728271,314.27634644)
\curveto(802.80727906,314.28634656)(802.75227911,314.29634655)(802.69228271,314.30634644)
\curveto(802.63227923,314.30634654)(802.57727929,314.31134654)(802.52728271,314.32134644)
\curveto(802.43727943,314.34134651)(802.34727952,314.36634648)(802.25728271,314.39634644)
\curveto(802.1672797,314.41634643)(802.08227978,314.44134641)(802.00228271,314.47134644)
\curveto(801.9622799,314.49134636)(801.92727994,314.50134635)(801.89728271,314.50134644)
\curveto(801.86728,314.51134634)(801.83228003,314.52634632)(801.79228271,314.54634644)
\curveto(801.64228022,314.61634623)(801.48228038,314.70134615)(801.31228271,314.80134644)
\curveto(801.02228084,314.99134586)(800.77228109,315.22134563)(800.56228271,315.49134644)
\curveto(800.3622815,315.77134508)(800.19228167,316.08134477)(800.05228271,316.42134644)
\curveto(800.00228186,316.53134432)(799.9622819,316.6463442)(799.93228271,316.76634644)
\curveto(799.91228195,316.88634396)(799.88228198,317.00634384)(799.84228271,317.12634644)
\curveto(799.83228203,317.16634368)(799.82728204,317.20134365)(799.82728271,317.23134644)
\curveto(799.82728204,317.26134359)(799.82228204,317.30134355)(799.81228271,317.35134644)
\curveto(799.79228207,317.43134342)(799.77728209,317.51634333)(799.76728271,317.60634644)
\curveto(799.75728211,317.69634315)(799.74228212,317.78634306)(799.72228271,317.87634644)
\lineto(799.72228271,318.08634644)
\curveto(799.71228215,318.12634272)(799.70228216,318.18134267)(799.69228271,318.25134644)
\curveto(799.69228217,318.33134252)(799.69728217,318.39634245)(799.70728271,318.44634644)
\lineto(799.70728271,318.61134644)
\curveto(799.72728214,318.66134219)(799.73228213,318.71134214)(799.72228271,318.76134644)
\curveto(799.72228214,318.82134203)(799.72728214,318.87634197)(799.73728271,318.92634644)
\curveto(799.77728209,319.08634176)(799.80728206,319.2463416)(799.82728271,319.40634644)
\curveto(799.85728201,319.56634128)(799.90228196,319.71634113)(799.96228271,319.85634644)
\curveto(800.01228185,319.96634088)(800.05728181,320.07634077)(800.09728271,320.18634644)
\curveto(800.14728172,320.30634054)(800.20228166,320.42134043)(800.26228271,320.53134644)
\curveto(800.48228138,320.88133997)(800.73228113,321.18133967)(801.01228271,321.43134644)
\curveto(801.29228057,321.69133916)(801.63728023,321.90633894)(802.04728271,322.07634644)
\curveto(802.1672797,322.12633872)(802.28727958,322.16133869)(802.40728271,322.18134644)
\curveto(802.53727933,322.21133864)(802.67227919,322.24133861)(802.81228271,322.27134644)
\curveto(802.862279,322.28133857)(802.90727896,322.28633856)(802.94728271,322.28634644)
\curveto(802.98727888,322.29633855)(803.03227883,322.30133855)(803.08228271,322.30134644)
\curveto(803.10227876,322.31133854)(803.12727874,322.31133854)(803.15728271,322.30134644)
\curveto(803.18727868,322.29133856)(803.21227865,322.29633855)(803.23228271,322.31634644)
\curveto(803.65227821,322.32633852)(804.01727785,322.28133857)(804.32728271,322.18134644)
\curveto(804.63727723,322.09133876)(804.91727695,321.96633888)(805.16728271,321.80634644)
\curveto(805.21727665,321.78633906)(805.25727661,321.75633909)(805.28728271,321.71634644)
\curveto(805.31727655,321.68633916)(805.35227651,321.66133919)(805.39228271,321.64134644)
\curveto(805.47227639,321.58133927)(805.55227631,321.51133934)(805.63228271,321.43134644)
\curveto(805.72227614,321.3513395)(805.79727607,321.27133958)(805.85728271,321.19134644)
\curveto(806.01727585,320.98133987)(806.15227571,320.78134007)(806.26228271,320.59134644)
\curveto(806.33227553,320.48134037)(806.38727548,320.36134049)(806.42728271,320.23134644)
\curveto(806.4672754,320.10134075)(806.51227535,319.97134088)(806.56228271,319.84134644)
\curveto(806.61227525,319.71134114)(806.64727522,319.57634127)(806.66728271,319.43634644)
\curveto(806.69727517,319.29634155)(806.73227513,319.15634169)(806.77228271,319.01634644)
\curveto(806.78227508,318.9463419)(806.78727508,318.87634197)(806.78728271,318.80634644)
\lineto(806.81728271,318.59634644)
\moveto(805.36228271,319.10634644)
\curveto(805.39227647,319.1463417)(805.41727645,319.19634165)(805.43728271,319.25634644)
\curveto(805.45727641,319.32634152)(805.45727641,319.39634145)(805.43728271,319.46634644)
\curveto(805.37727649,319.68634116)(805.29227657,319.89134096)(805.18228271,320.08134644)
\curveto(805.04227682,320.31134054)(804.88727698,320.50634034)(804.71728271,320.66634644)
\curveto(804.54727732,320.82634002)(804.32727754,320.96133989)(804.05728271,321.07134644)
\curveto(803.98727788,321.09133976)(803.91727795,321.10633974)(803.84728271,321.11634644)
\curveto(803.77727809,321.13633971)(803.70227816,321.15633969)(803.62228271,321.17634644)
\curveto(803.54227832,321.19633965)(803.45727841,321.20633964)(803.36728271,321.20634644)
\lineto(803.11228271,321.20634644)
\curveto(803.08227878,321.18633966)(803.04727882,321.17633967)(803.00728271,321.17634644)
\curveto(802.9672789,321.18633966)(802.93227893,321.18633966)(802.90228271,321.17634644)
\lineto(802.66228271,321.11634644)
\curveto(802.59227927,321.10633974)(802.52227934,321.09133976)(802.45228271,321.07134644)
\curveto(802.1622797,320.9513399)(801.92727994,320.80134005)(801.74728271,320.62134644)
\curveto(801.57728029,320.44134041)(801.42228044,320.21634063)(801.28228271,319.94634644)
\curveto(801.25228061,319.89634095)(801.22228064,319.83134102)(801.19228271,319.75134644)
\curveto(801.1622807,319.68134117)(801.13728073,319.60134125)(801.11728271,319.51134644)
\curveto(801.09728077,319.42134143)(801.09228077,319.33634151)(801.10228271,319.25634644)
\curveto(801.11228075,319.17634167)(801.14728072,319.11634173)(801.20728271,319.07634644)
\curveto(801.28728058,319.01634183)(801.42228044,318.98634186)(801.61228271,318.98634644)
\curveto(801.81228005,318.99634185)(801.98227988,319.00134185)(802.12228271,319.00134644)
\lineto(804.40228271,319.00134644)
\curveto(804.55227731,319.00134185)(804.73227713,318.99634185)(804.94228271,318.98634644)
\curveto(805.15227671,318.98634186)(805.29227657,319.02634182)(805.36228271,319.10634644)
}
}
{
\newrgbcolor{curcolor}{0 0 0}
\pscustom[linestyle=none,fillstyle=solid,fillcolor=curcolor]
{
\newpath
\moveto(811.81392334,322.30134644)
\curveto(812.4439181,322.32133853)(812.9489176,322.23633861)(813.32892334,322.04634644)
\curveto(813.70891684,321.85633899)(814.01391653,321.57133928)(814.24392334,321.19134644)
\curveto(814.30391624,321.09133976)(814.3489162,320.98133987)(814.37892334,320.86134644)
\curveto(814.41891613,320.7513401)(814.45391609,320.63634021)(814.48392334,320.51634644)
\curveto(814.53391601,320.32634052)(814.56391598,320.12134073)(814.57392334,319.90134644)
\curveto(814.58391596,319.68134117)(814.58891596,319.45634139)(814.58892334,319.22634644)
\lineto(814.58892334,317.62134644)
\lineto(814.58892334,315.28134644)
\curveto(814.58891596,315.11134574)(814.58391596,314.94134591)(814.57392334,314.77134644)
\curveto(814.57391597,314.60134625)(814.50891604,314.49134636)(814.37892334,314.44134644)
\curveto(814.32891622,314.42134643)(814.27391627,314.41134644)(814.21392334,314.41134644)
\curveto(814.16391638,314.40134645)(814.10891644,314.39634645)(814.04892334,314.39634644)
\curveto(813.91891663,314.39634645)(813.79391675,314.40134645)(813.67392334,314.41134644)
\curveto(813.55391699,314.41134644)(813.46891708,314.4513464)(813.41892334,314.53134644)
\curveto(813.36891718,314.60134625)(813.3439172,314.69134616)(813.34392334,314.80134644)
\lineto(813.34392334,315.13134644)
\lineto(813.34392334,316.42134644)
\lineto(813.34392334,318.86634644)
\curveto(813.3439172,319.13634171)(813.33891721,319.40134145)(813.32892334,319.66134644)
\curveto(813.31891723,319.93134092)(813.27391727,320.16134069)(813.19392334,320.35134644)
\curveto(813.11391743,320.5513403)(812.99391755,320.71134014)(812.83392334,320.83134644)
\curveto(812.67391787,320.96133989)(812.48891806,321.06133979)(812.27892334,321.13134644)
\curveto(812.21891833,321.1513397)(812.15391839,321.16133969)(812.08392334,321.16134644)
\curveto(812.02391852,321.17133968)(811.96391858,321.18633966)(811.90392334,321.20634644)
\curveto(811.85391869,321.21633963)(811.77391877,321.21633963)(811.66392334,321.20634644)
\curveto(811.56391898,321.20633964)(811.49391905,321.20133965)(811.45392334,321.19134644)
\curveto(811.41391913,321.17133968)(811.37891917,321.16133969)(811.34892334,321.16134644)
\curveto(811.31891923,321.17133968)(811.28391926,321.17133968)(811.24392334,321.16134644)
\curveto(811.11391943,321.13133972)(810.98891956,321.09633975)(810.86892334,321.05634644)
\curveto(810.75891979,321.02633982)(810.65391989,320.98133987)(810.55392334,320.92134644)
\curveto(810.51392003,320.90133995)(810.47892007,320.88133997)(810.44892334,320.86134644)
\curveto(810.41892013,320.84134001)(810.38392016,320.82134003)(810.34392334,320.80134644)
\curveto(809.99392055,320.5513403)(809.73892081,320.17634067)(809.57892334,319.67634644)
\curveto(809.548921,319.59634125)(809.52892102,319.51134134)(809.51892334,319.42134644)
\curveto(809.50892104,319.34134151)(809.49392105,319.26134159)(809.47392334,319.18134644)
\curveto(809.45392109,319.13134172)(809.4489211,319.08134177)(809.45892334,319.03134644)
\curveto(809.46892108,318.99134186)(809.46392108,318.9513419)(809.44392334,318.91134644)
\lineto(809.44392334,318.59634644)
\curveto(809.43392111,318.56634228)(809.42892112,318.53134232)(809.42892334,318.49134644)
\curveto(809.43892111,318.4513424)(809.4439211,318.40634244)(809.44392334,318.35634644)
\lineto(809.44392334,317.90634644)
\lineto(809.44392334,316.46634644)
\lineto(809.44392334,315.14634644)
\lineto(809.44392334,314.80134644)
\curveto(809.4439211,314.69134616)(809.41892113,314.60134625)(809.36892334,314.53134644)
\curveto(809.31892123,314.4513464)(809.22892132,314.41134644)(809.09892334,314.41134644)
\curveto(808.97892157,314.40134645)(808.85392169,314.39634645)(808.72392334,314.39634644)
\curveto(808.6439219,314.39634645)(808.56892198,314.40134645)(808.49892334,314.41134644)
\curveto(808.42892212,314.42134643)(808.36892218,314.4463464)(808.31892334,314.48634644)
\curveto(808.23892231,314.53634631)(808.19892235,314.63134622)(808.19892334,314.77134644)
\lineto(808.19892334,315.17634644)
\lineto(808.19892334,316.94634644)
\lineto(808.19892334,320.57634644)
\lineto(808.19892334,321.49134644)
\lineto(808.19892334,321.76134644)
\curveto(808.19892235,321.851339)(808.21892233,321.92133893)(808.25892334,321.97134644)
\curveto(808.28892226,322.03133882)(808.33892221,322.07133878)(808.40892334,322.09134644)
\curveto(808.4489221,322.10133875)(808.50392204,322.11133874)(808.57392334,322.12134644)
\curveto(808.65392189,322.13133872)(808.73392181,322.13633871)(808.81392334,322.13634644)
\curveto(808.89392165,322.13633871)(808.96892158,322.13133872)(809.03892334,322.12134644)
\curveto(809.11892143,322.11133874)(809.17392137,322.09633875)(809.20392334,322.07634644)
\curveto(809.31392123,322.00633884)(809.36392118,321.91633893)(809.35392334,321.80634644)
\curveto(809.3439212,321.70633914)(809.35892119,321.59133926)(809.39892334,321.46134644)
\curveto(809.41892113,321.40133945)(809.45892109,321.3513395)(809.51892334,321.31134644)
\curveto(809.63892091,321.30133955)(809.73392081,321.3463395)(809.80392334,321.44634644)
\curveto(809.88392066,321.5463393)(809.96392058,321.62633922)(810.04392334,321.68634644)
\curveto(810.18392036,321.78633906)(810.32392022,321.87633897)(810.46392334,321.95634644)
\curveto(810.61391993,322.0463388)(810.78391976,322.12133873)(810.97392334,322.18134644)
\curveto(811.05391949,322.21133864)(811.13891941,322.23133862)(811.22892334,322.24134644)
\curveto(811.32891922,322.2513386)(811.42391912,322.26633858)(811.51392334,322.28634644)
\curveto(811.56391898,322.29633855)(811.61391893,322.30133855)(811.66392334,322.30134644)
\lineto(811.81392334,322.30134644)
}
}
{
\newrgbcolor{curcolor}{0 0 0}
\pscustom[linestyle=none,fillstyle=solid,fillcolor=curcolor]
{
\newpath
\moveto(817.41853271,324.49134644)
\curveto(817.5685307,324.49133636)(817.71853055,324.48633636)(817.86853271,324.47634644)
\curveto(818.01853025,324.47633637)(818.12353015,324.43633641)(818.18353271,324.35634644)
\curveto(818.23353004,324.29633655)(818.25853001,324.21133664)(818.25853271,324.10134644)
\curveto(818.26853,324.00133685)(818.27353,323.89633695)(818.27353271,323.78634644)
\lineto(818.27353271,322.91634644)
\curveto(818.27353,322.83633801)(818.26853,322.7513381)(818.25853271,322.66134644)
\curveto(818.25853001,322.58133827)(818.26853,322.51133834)(818.28853271,322.45134644)
\curveto(818.32852994,322.31133854)(818.41852985,322.22133863)(818.55853271,322.18134644)
\curveto(818.60852966,322.17133868)(818.65352962,322.16633868)(818.69353271,322.16634644)
\lineto(818.84353271,322.16634644)
\lineto(819.24853271,322.16634644)
\curveto(819.40852886,322.17633867)(819.52352875,322.16633868)(819.59353271,322.13634644)
\curveto(819.68352859,322.07633877)(819.74352853,322.01633883)(819.77353271,321.95634644)
\curveto(819.79352848,321.91633893)(819.80352847,321.87133898)(819.80353271,321.82134644)
\lineto(819.80353271,321.67134644)
\curveto(819.80352847,321.56133929)(819.79852847,321.45633939)(819.78853271,321.35634644)
\curveto(819.77852849,321.26633958)(819.74352853,321.19633965)(819.68353271,321.14634644)
\curveto(819.62352865,321.09633975)(819.53852873,321.06633978)(819.42853271,321.05634644)
\lineto(819.09853271,321.05634644)
\curveto(818.98852928,321.06633978)(818.87852939,321.07133978)(818.76853271,321.07134644)
\curveto(818.65852961,321.07133978)(818.56352971,321.05633979)(818.48353271,321.02634644)
\curveto(818.41352986,320.99633985)(818.36352991,320.9463399)(818.33353271,320.87634644)
\curveto(818.30352997,320.80634004)(818.28352999,320.72134013)(818.27353271,320.62134644)
\curveto(818.26353001,320.53134032)(818.25853001,320.43134042)(818.25853271,320.32134644)
\curveto(818.26853,320.22134063)(818.27353,320.12134073)(818.27353271,320.02134644)
\lineto(818.27353271,317.05134644)
\curveto(818.27353,316.83134402)(818.26853,316.59634425)(818.25853271,316.34634644)
\curveto(818.25853001,316.10634474)(818.30352997,315.92134493)(818.39353271,315.79134644)
\curveto(818.44352983,315.71134514)(818.50852976,315.65634519)(818.58853271,315.62634644)
\curveto(818.6685296,315.59634525)(818.76352951,315.57134528)(818.87353271,315.55134644)
\curveto(818.90352937,315.54134531)(818.93352934,315.53634531)(818.96353271,315.53634644)
\curveto(819.00352927,315.5463453)(819.03852923,315.5463453)(819.06853271,315.53634644)
\lineto(819.26353271,315.53634644)
\curveto(819.36352891,315.53634531)(819.45352882,315.52634532)(819.53353271,315.50634644)
\curveto(819.62352865,315.49634535)(819.68852858,315.46134539)(819.72853271,315.40134644)
\curveto(819.74852852,315.37134548)(819.76352851,315.31634553)(819.77353271,315.23634644)
\curveto(819.79352848,315.16634568)(819.80352847,315.09134576)(819.80353271,315.01134644)
\curveto(819.81352846,314.93134592)(819.81352846,314.851346)(819.80353271,314.77134644)
\curveto(819.79352848,314.70134615)(819.7735285,314.6463462)(819.74353271,314.60634644)
\curveto(819.70352857,314.53634631)(819.62852864,314.48634636)(819.51853271,314.45634644)
\curveto(819.43852883,314.43634641)(819.34852892,314.42634642)(819.24853271,314.42634644)
\curveto(819.14852912,314.43634641)(819.05852921,314.44134641)(818.97853271,314.44134644)
\curveto(818.91852935,314.44134641)(818.85852941,314.43634641)(818.79853271,314.42634644)
\curveto(818.73852953,314.42634642)(818.68352959,314.43134642)(818.63353271,314.44134644)
\lineto(818.45353271,314.44134644)
\curveto(818.40352987,314.4513464)(818.35352992,314.45634639)(818.30353271,314.45634644)
\curveto(818.26353001,314.46634638)(818.21853005,314.47134638)(818.16853271,314.47134644)
\curveto(817.9685303,314.52134633)(817.79353048,314.57634627)(817.64353271,314.63634644)
\curveto(817.50353077,314.69634615)(817.38353089,314.80134605)(817.28353271,314.95134644)
\curveto(817.14353113,315.1513457)(817.06353121,315.40134545)(817.04353271,315.70134644)
\curveto(817.02353125,316.01134484)(817.01353126,316.34134451)(817.01353271,316.69134644)
\lineto(817.01353271,320.62134644)
\curveto(816.98353129,320.7513401)(816.95353132,320.84634)(816.92353271,320.90634644)
\curveto(816.90353137,320.96633988)(816.83353144,321.01633983)(816.71353271,321.05634644)
\curveto(816.6735316,321.06633978)(816.63353164,321.06633978)(816.59353271,321.05634644)
\curveto(816.55353172,321.0463398)(816.51353176,321.0513398)(816.47353271,321.07134644)
\lineto(816.23353271,321.07134644)
\curveto(816.10353217,321.07133978)(815.99353228,321.08133977)(815.90353271,321.10134644)
\curveto(815.82353245,321.13133972)(815.7685325,321.19133966)(815.73853271,321.28134644)
\curveto(815.71853255,321.32133953)(815.70353257,321.36633948)(815.69353271,321.41634644)
\lineto(815.69353271,321.56634644)
\curveto(815.69353258,321.70633914)(815.70353257,321.82133903)(815.72353271,321.91134644)
\curveto(815.74353253,322.01133884)(815.80353247,322.08633876)(815.90353271,322.13634644)
\curveto(816.01353226,322.17633867)(816.15353212,322.18633866)(816.32353271,322.16634644)
\curveto(816.50353177,322.1463387)(816.65353162,322.15633869)(816.77353271,322.19634644)
\curveto(816.86353141,322.2463386)(816.93353134,322.31633853)(816.98353271,322.40634644)
\curveto(817.00353127,322.46633838)(817.01353126,322.54133831)(817.01353271,322.63134644)
\lineto(817.01353271,322.88634644)
\lineto(817.01353271,323.81634644)
\lineto(817.01353271,324.05634644)
\curveto(817.01353126,324.1463367)(817.02353125,324.22133663)(817.04353271,324.28134644)
\curveto(817.08353119,324.36133649)(817.15853111,324.42633642)(817.26853271,324.47634644)
\curveto(817.29853097,324.47633637)(817.32353095,324.47633637)(817.34353271,324.47634644)
\curveto(817.3735309,324.48633636)(817.39853087,324.49133636)(817.41853271,324.49134644)
}
}
{
\newrgbcolor{curcolor}{0 0 0}
\pscustom[linestyle=none,fillstyle=solid,fillcolor=curcolor]
{
\newpath
\moveto(828.07532959,314.98134644)
\curveto(828.10532176,314.82134603)(828.09032177,314.68634616)(828.03032959,314.57634644)
\curveto(827.97032189,314.47634637)(827.89032197,314.40134645)(827.79032959,314.35134644)
\curveto(827.74032212,314.33134652)(827.68532218,314.32134653)(827.62532959,314.32134644)
\curveto(827.57532229,314.32134653)(827.52032234,314.31134654)(827.46032959,314.29134644)
\curveto(827.24032262,314.24134661)(827.02032284,314.25634659)(826.80032959,314.33634644)
\curveto(826.59032327,314.40634644)(826.44532342,314.49634635)(826.36532959,314.60634644)
\curveto(826.31532355,314.67634617)(826.27032359,314.75634609)(826.23032959,314.84634644)
\curveto(826.19032367,314.9463459)(826.14032372,315.02634582)(826.08032959,315.08634644)
\curveto(826.0603238,315.10634574)(826.03532383,315.12634572)(826.00532959,315.14634644)
\curveto(825.98532388,315.16634568)(825.95532391,315.17134568)(825.91532959,315.16134644)
\curveto(825.80532406,315.13134572)(825.70032416,315.07634577)(825.60032959,314.99634644)
\curveto(825.51032435,314.91634593)(825.42032444,314.846346)(825.33032959,314.78634644)
\curveto(825.20032466,314.70634614)(825.0603248,314.63134622)(824.91032959,314.56134644)
\curveto(824.7603251,314.50134635)(824.60032526,314.4463464)(824.43032959,314.39634644)
\curveto(824.33032553,314.36634648)(824.22032564,314.3463465)(824.10032959,314.33634644)
\curveto(823.99032587,314.32634652)(823.88032598,314.31134654)(823.77032959,314.29134644)
\curveto(823.72032614,314.28134657)(823.67532619,314.27634657)(823.63532959,314.27634644)
\lineto(823.53032959,314.27634644)
\curveto(823.42032644,314.25634659)(823.31532655,314.25634659)(823.21532959,314.27634644)
\lineto(823.08032959,314.27634644)
\curveto(823.03032683,314.28634656)(822.98032688,314.29134656)(822.93032959,314.29134644)
\curveto(822.88032698,314.29134656)(822.83532703,314.30134655)(822.79532959,314.32134644)
\curveto(822.75532711,314.33134652)(822.72032714,314.33634651)(822.69032959,314.33634644)
\curveto(822.67032719,314.32634652)(822.64532722,314.32634652)(822.61532959,314.33634644)
\lineto(822.37532959,314.39634644)
\curveto(822.29532757,314.40634644)(822.22032764,314.42634642)(822.15032959,314.45634644)
\curveto(821.85032801,314.58634626)(821.60532826,314.73134612)(821.41532959,314.89134644)
\curveto(821.23532863,315.06134579)(821.08532878,315.29634555)(820.96532959,315.59634644)
\curveto(820.87532899,315.81634503)(820.83032903,316.08134477)(820.83032959,316.39134644)
\lineto(820.83032959,316.70634644)
\curveto(820.84032902,316.75634409)(820.84532902,316.80634404)(820.84532959,316.85634644)
\lineto(820.87532959,317.03634644)
\lineto(820.99532959,317.36634644)
\curveto(821.03532883,317.47634337)(821.08532878,317.57634327)(821.14532959,317.66634644)
\curveto(821.32532854,317.95634289)(821.57032829,318.17134268)(821.88032959,318.31134644)
\curveto(822.19032767,318.4513424)(822.53032733,318.57634227)(822.90032959,318.68634644)
\curveto(823.04032682,318.72634212)(823.18532668,318.75634209)(823.33532959,318.77634644)
\curveto(823.48532638,318.79634205)(823.63532623,318.82134203)(823.78532959,318.85134644)
\curveto(823.85532601,318.87134198)(823.92032594,318.88134197)(823.98032959,318.88134644)
\curveto(824.05032581,318.88134197)(824.12532574,318.89134196)(824.20532959,318.91134644)
\curveto(824.27532559,318.93134192)(824.34532552,318.94134191)(824.41532959,318.94134644)
\curveto(824.48532538,318.9513419)(824.5603253,318.96634188)(824.64032959,318.98634644)
\curveto(824.89032497,319.0463418)(825.12532474,319.09634175)(825.34532959,319.13634644)
\curveto(825.5653243,319.18634166)(825.74032412,319.30134155)(825.87032959,319.48134644)
\curveto(825.93032393,319.56134129)(825.98032388,319.66134119)(826.02032959,319.78134644)
\curveto(826.0603238,319.91134094)(826.0603238,320.0513408)(826.02032959,320.20134644)
\curveto(825.9603239,320.44134041)(825.87032399,320.63134022)(825.75032959,320.77134644)
\curveto(825.64032422,320.91133994)(825.48032438,321.02133983)(825.27032959,321.10134644)
\curveto(825.15032471,321.1513397)(825.00532486,321.18633966)(824.83532959,321.20634644)
\curveto(824.67532519,321.22633962)(824.50532536,321.23633961)(824.32532959,321.23634644)
\curveto(824.14532572,321.23633961)(823.97032589,321.22633962)(823.80032959,321.20634644)
\curveto(823.63032623,321.18633966)(823.48532638,321.15633969)(823.36532959,321.11634644)
\curveto(823.19532667,321.05633979)(823.03032683,320.97133988)(822.87032959,320.86134644)
\curveto(822.79032707,320.80134005)(822.71532715,320.72134013)(822.64532959,320.62134644)
\curveto(822.58532728,320.53134032)(822.53032733,320.43134042)(822.48032959,320.32134644)
\curveto(822.45032741,320.24134061)(822.42032744,320.15634069)(822.39032959,320.06634644)
\curveto(822.37032749,319.97634087)(822.32532754,319.90634094)(822.25532959,319.85634644)
\curveto(822.21532765,319.82634102)(822.14532772,319.80134105)(822.04532959,319.78134644)
\curveto(821.95532791,319.77134108)(821.860328,319.76634108)(821.76032959,319.76634644)
\curveto(821.6603282,319.76634108)(821.5603283,319.77134108)(821.46032959,319.78134644)
\curveto(821.37032849,319.80134105)(821.30532856,319.82634102)(821.26532959,319.85634644)
\curveto(821.22532864,319.88634096)(821.19532867,319.93634091)(821.17532959,320.00634644)
\curveto(821.15532871,320.07634077)(821.15532871,320.1513407)(821.17532959,320.23134644)
\curveto(821.20532866,320.36134049)(821.23532863,320.48134037)(821.26532959,320.59134644)
\curveto(821.30532856,320.71134014)(821.35032851,320.82634002)(821.40032959,320.93634644)
\curveto(821.59032827,321.28633956)(821.83032803,321.55633929)(822.12032959,321.74634644)
\curveto(822.41032745,321.9463389)(822.77032709,322.10633874)(823.20032959,322.22634644)
\curveto(823.30032656,322.2463386)(823.40032646,322.26133859)(823.50032959,322.27134644)
\curveto(823.61032625,322.28133857)(823.72032614,322.29633855)(823.83032959,322.31634644)
\curveto(823.87032599,322.32633852)(823.93532593,322.32633852)(824.02532959,322.31634644)
\curveto(824.11532575,322.31633853)(824.17032569,322.32633852)(824.19032959,322.34634644)
\curveto(824.89032497,322.35633849)(825.50032436,322.27633857)(826.02032959,322.10634644)
\curveto(826.54032332,321.93633891)(826.90532296,321.61133924)(827.11532959,321.13134644)
\curveto(827.20532266,320.93133992)(827.25532261,320.69634015)(827.26532959,320.42634644)
\curveto(827.28532258,320.16634068)(827.29532257,319.89134096)(827.29532959,319.60134644)
\lineto(827.29532959,316.28634644)
\curveto(827.29532257,316.1463447)(827.30032256,316.01134484)(827.31032959,315.88134644)
\curveto(827.32032254,315.7513451)(827.35032251,315.6463452)(827.40032959,315.56634644)
\curveto(827.45032241,315.49634535)(827.51532235,315.4463454)(827.59532959,315.41634644)
\curveto(827.68532218,315.37634547)(827.77032209,315.3463455)(827.85032959,315.32634644)
\curveto(827.93032193,315.31634553)(827.99032187,315.27134558)(828.03032959,315.19134644)
\curveto(828.05032181,315.16134569)(828.0603218,315.13134572)(828.06032959,315.10134644)
\curveto(828.0603218,315.07134578)(828.0653218,315.03134582)(828.07532959,314.98134644)
\moveto(825.93032959,316.64634644)
\curveto(825.99032387,316.78634406)(826.02032384,316.9463439)(826.02032959,317.12634644)
\curveto(826.03032383,317.31634353)(826.03532383,317.51134334)(826.03532959,317.71134644)
\curveto(826.03532383,317.82134303)(826.03032383,317.92134293)(826.02032959,318.01134644)
\curveto(826.01032385,318.10134275)(825.97032389,318.17134268)(825.90032959,318.22134644)
\curveto(825.87032399,318.24134261)(825.80032406,318.2513426)(825.69032959,318.25134644)
\curveto(825.67032419,318.23134262)(825.63532423,318.22134263)(825.58532959,318.22134644)
\curveto(825.53532433,318.22134263)(825.49032437,318.21134264)(825.45032959,318.19134644)
\curveto(825.37032449,318.17134268)(825.28032458,318.1513427)(825.18032959,318.13134644)
\lineto(824.88032959,318.07134644)
\curveto(824.85032501,318.07134278)(824.81532505,318.06634278)(824.77532959,318.05634644)
\lineto(824.67032959,318.05634644)
\curveto(824.52032534,318.01634283)(824.35532551,317.99134286)(824.17532959,317.98134644)
\curveto(824.00532586,317.98134287)(823.84532602,317.96134289)(823.69532959,317.92134644)
\curveto(823.61532625,317.90134295)(823.54032632,317.88134297)(823.47032959,317.86134644)
\curveto(823.41032645,317.851343)(823.34032652,317.83634301)(823.26032959,317.81634644)
\curveto(823.10032676,317.76634308)(822.95032691,317.70134315)(822.81032959,317.62134644)
\curveto(822.67032719,317.5513433)(822.55032731,317.46134339)(822.45032959,317.35134644)
\curveto(822.35032751,317.24134361)(822.27532759,317.10634374)(822.22532959,316.94634644)
\curveto(822.17532769,316.79634405)(822.15532771,316.61134424)(822.16532959,316.39134644)
\curveto(822.1653277,316.29134456)(822.18032768,316.19634465)(822.21032959,316.10634644)
\curveto(822.25032761,316.02634482)(822.29532757,315.9513449)(822.34532959,315.88134644)
\curveto(822.42532744,315.77134508)(822.53032733,315.67634517)(822.66032959,315.59634644)
\curveto(822.79032707,315.52634532)(822.93032693,315.46634538)(823.08032959,315.41634644)
\curveto(823.13032673,315.40634544)(823.18032668,315.40134545)(823.23032959,315.40134644)
\curveto(823.28032658,315.40134545)(823.33032653,315.39634545)(823.38032959,315.38634644)
\curveto(823.45032641,315.36634548)(823.53532633,315.3513455)(823.63532959,315.34134644)
\curveto(823.74532612,315.34134551)(823.83532603,315.3513455)(823.90532959,315.37134644)
\curveto(823.9653259,315.39134546)(824.02532584,315.39634545)(824.08532959,315.38634644)
\curveto(824.14532572,315.38634546)(824.20532566,315.39634545)(824.26532959,315.41634644)
\curveto(824.34532552,315.43634541)(824.42032544,315.4513454)(824.49032959,315.46134644)
\curveto(824.57032529,315.47134538)(824.64532522,315.49134536)(824.71532959,315.52134644)
\curveto(825.00532486,315.64134521)(825.25032461,315.78634506)(825.45032959,315.95634644)
\curveto(825.6603242,316.12634472)(825.82032404,316.35634449)(825.93032959,316.64634644)
}
}
{
\newrgbcolor{curcolor}{0 0 0}
\pscustom[linestyle=none,fillstyle=solid,fillcolor=curcolor]
{
\newpath
\moveto(832.89197021,322.33134644)
\curveto(833.12196542,322.33133852)(833.25196529,322.27133858)(833.28197021,322.15134644)
\curveto(833.31196523,322.04133881)(833.32696522,321.87633897)(833.32697021,321.65634644)
\lineto(833.32697021,321.37134644)
\curveto(833.32696522,321.28133957)(833.30196524,321.20633964)(833.25197021,321.14634644)
\curveto(833.19196535,321.06633978)(833.10696544,321.02133983)(832.99697021,321.01134644)
\curveto(832.88696566,321.01133984)(832.77696577,320.99633985)(832.66697021,320.96634644)
\curveto(832.52696602,320.93633991)(832.39196615,320.90633994)(832.26197021,320.87634644)
\curveto(832.1419664,320.84634)(832.02696652,320.80634004)(831.91697021,320.75634644)
\curveto(831.62696692,320.62634022)(831.39196715,320.4463404)(831.21197021,320.21634644)
\curveto(831.03196751,319.99634085)(830.87696767,319.74134111)(830.74697021,319.45134644)
\curveto(830.70696784,319.34134151)(830.67696787,319.22634162)(830.65697021,319.10634644)
\curveto(830.63696791,318.99634185)(830.61196793,318.88134197)(830.58197021,318.76134644)
\curveto(830.57196797,318.71134214)(830.56696798,318.66134219)(830.56697021,318.61134644)
\curveto(830.57696797,318.56134229)(830.57696797,318.51134234)(830.56697021,318.46134644)
\curveto(830.53696801,318.34134251)(830.52196802,318.20134265)(830.52197021,318.04134644)
\curveto(830.53196801,317.89134296)(830.53696801,317.7463431)(830.53697021,317.60634644)
\lineto(830.53697021,315.76134644)
\lineto(830.53697021,315.41634644)
\curveto(830.53696801,315.29634555)(830.53196801,315.18134567)(830.52197021,315.07134644)
\curveto(830.51196803,314.96134589)(830.50696804,314.86634598)(830.50697021,314.78634644)
\curveto(830.51696803,314.70634614)(830.49696805,314.63634621)(830.44697021,314.57634644)
\curveto(830.39696815,314.50634634)(830.31696823,314.46634638)(830.20697021,314.45634644)
\curveto(830.10696844,314.4463464)(829.99696855,314.44134641)(829.87697021,314.44134644)
\lineto(829.60697021,314.44134644)
\curveto(829.55696899,314.46134639)(829.50696904,314.47634637)(829.45697021,314.48634644)
\curveto(829.41696913,314.50634634)(829.38696916,314.53134632)(829.36697021,314.56134644)
\curveto(829.31696923,314.63134622)(829.28696926,314.71634613)(829.27697021,314.81634644)
\lineto(829.27697021,315.14634644)
\lineto(829.27697021,316.30134644)
\lineto(829.27697021,320.45634644)
\lineto(829.27697021,321.49134644)
\lineto(829.27697021,321.79134644)
\curveto(829.28696926,321.89133896)(829.31696923,321.97633887)(829.36697021,322.04634644)
\curveto(829.39696915,322.08633876)(829.4469691,322.11633873)(829.51697021,322.13634644)
\curveto(829.59696895,322.15633869)(829.68196886,322.16633868)(829.77197021,322.16634644)
\curveto(829.86196868,322.17633867)(829.95196859,322.17633867)(830.04197021,322.16634644)
\curveto(830.13196841,322.15633869)(830.20196834,322.14133871)(830.25197021,322.12134644)
\curveto(830.33196821,322.09133876)(830.38196816,322.03133882)(830.40197021,321.94134644)
\curveto(830.43196811,321.86133899)(830.4469681,321.77133908)(830.44697021,321.67134644)
\lineto(830.44697021,321.37134644)
\curveto(830.4469681,321.27133958)(830.46696808,321.18133967)(830.50697021,321.10134644)
\curveto(830.51696803,321.08133977)(830.52696802,321.06633978)(830.53697021,321.05634644)
\lineto(830.58197021,321.01134644)
\curveto(830.69196785,321.01133984)(830.78196776,321.05633979)(830.85197021,321.14634644)
\curveto(830.92196762,321.2463396)(830.98196756,321.32633952)(831.03197021,321.38634644)
\lineto(831.12197021,321.47634644)
\curveto(831.21196733,321.58633926)(831.33696721,321.70133915)(831.49697021,321.82134644)
\curveto(831.65696689,321.94133891)(831.80696674,322.03133882)(831.94697021,322.09134644)
\curveto(832.03696651,322.14133871)(832.13196641,322.17633867)(832.23197021,322.19634644)
\curveto(832.33196621,322.22633862)(832.43696611,322.25633859)(832.54697021,322.28634644)
\curveto(832.60696594,322.29633855)(832.66696588,322.30133855)(832.72697021,322.30134644)
\curveto(832.78696576,322.31133854)(832.8419657,322.32133853)(832.89197021,322.33134644)
}
}
{
\newrgbcolor{curcolor}{0 0 0}
\pscustom[linestyle=none,fillstyle=solid,fillcolor=curcolor]
{
\newpath
\moveto(834.54173584,323.65134644)
\curveto(834.46173472,323.71133714)(834.41673476,323.81633703)(834.40673584,323.96634644)
\lineto(834.40673584,324.43134644)
\lineto(834.40673584,324.68634644)
\curveto(834.40673477,324.77633607)(834.42173476,324.851336)(834.45173584,324.91134644)
\curveto(834.49173469,324.99133586)(834.57173461,325.0513358)(834.69173584,325.09134644)
\curveto(834.71173447,325.10133575)(834.73173445,325.10133575)(834.75173584,325.09134644)
\curveto(834.7817344,325.09133576)(834.80673437,325.09633575)(834.82673584,325.10634644)
\curveto(834.99673418,325.10633574)(835.15673402,325.10133575)(835.30673584,325.09134644)
\curveto(835.45673372,325.08133577)(835.55673362,325.02133583)(835.60673584,324.91134644)
\curveto(835.63673354,324.851336)(835.65173353,324.77633607)(835.65173584,324.68634644)
\lineto(835.65173584,324.43134644)
\curveto(835.65173353,324.2513366)(835.64673353,324.08133677)(835.63673584,323.92134644)
\curveto(835.63673354,323.76133709)(835.57173361,323.65633719)(835.44173584,323.60634644)
\curveto(835.39173379,323.58633726)(835.33673384,323.57633727)(835.27673584,323.57634644)
\lineto(835.11173584,323.57634644)
\lineto(834.79673584,323.57634644)
\curveto(834.69673448,323.57633727)(834.61173457,323.60133725)(834.54173584,323.65134644)
\moveto(835.65173584,315.14634644)
\lineto(835.65173584,314.83134644)
\curveto(835.66173352,314.73134612)(835.64173354,314.6513462)(835.59173584,314.59134644)
\curveto(835.56173362,314.53134632)(835.51673366,314.49134636)(835.45673584,314.47134644)
\curveto(835.39673378,314.46134639)(835.32673385,314.4463464)(835.24673584,314.42634644)
\lineto(835.02173584,314.42634644)
\curveto(834.89173429,314.42634642)(834.7767344,314.43134642)(834.67673584,314.44134644)
\curveto(834.58673459,314.46134639)(834.51673466,314.51134634)(834.46673584,314.59134644)
\curveto(834.42673475,314.6513462)(834.40673477,314.72634612)(834.40673584,314.81634644)
\lineto(834.40673584,315.10134644)
\lineto(834.40673584,321.44634644)
\lineto(834.40673584,321.76134644)
\curveto(834.40673477,321.87133898)(834.43173475,321.95633889)(834.48173584,322.01634644)
\curveto(834.51173467,322.06633878)(834.55173463,322.09633875)(834.60173584,322.10634644)
\curveto(834.65173453,322.11633873)(834.70673447,322.13133872)(834.76673584,322.15134644)
\curveto(834.78673439,322.1513387)(834.80673437,322.1463387)(834.82673584,322.13634644)
\curveto(834.85673432,322.13633871)(834.8817343,322.14133871)(834.90173584,322.15134644)
\curveto(835.03173415,322.1513387)(835.16173402,322.1463387)(835.29173584,322.13634644)
\curveto(835.43173375,322.13633871)(835.52673365,322.09633875)(835.57673584,322.01634644)
\curveto(835.62673355,321.95633889)(835.65173353,321.87633897)(835.65173584,321.77634644)
\lineto(835.65173584,321.49134644)
\lineto(835.65173584,315.14634644)
}
}
{
\newrgbcolor{curcolor}{0 0 0}
\pscustom[linestyle=none,fillstyle=solid,fillcolor=curcolor]
{
\newpath
\moveto(844.72157959,318.62634644)
\curveto(844.74157153,318.56634228)(844.75157152,318.47134238)(844.75157959,318.34134644)
\curveto(844.75157152,318.22134263)(844.74657152,318.13634271)(844.73657959,318.08634644)
\lineto(844.73657959,317.93634644)
\curveto(844.72657154,317.85634299)(844.71657155,317.78134307)(844.70657959,317.71134644)
\curveto(844.70657156,317.6513432)(844.70157157,317.58134327)(844.69157959,317.50134644)
\curveto(844.6715716,317.44134341)(844.65657161,317.38134347)(844.64657959,317.32134644)
\curveto(844.64657162,317.26134359)(844.63657163,317.20134365)(844.61657959,317.14134644)
\curveto(844.57657169,317.01134384)(844.54157173,316.88134397)(844.51157959,316.75134644)
\curveto(844.48157179,316.62134423)(844.44157183,316.50134435)(844.39157959,316.39134644)
\curveto(844.18157209,315.91134494)(843.90157237,315.50634534)(843.55157959,315.17634644)
\curveto(843.20157307,314.85634599)(842.7715735,314.61134624)(842.26157959,314.44134644)
\curveto(842.15157412,314.40134645)(842.03157424,314.37134648)(841.90157959,314.35134644)
\curveto(841.78157449,314.33134652)(841.65657461,314.31134654)(841.52657959,314.29134644)
\curveto(841.4665748,314.28134657)(841.40157487,314.27634657)(841.33157959,314.27634644)
\curveto(841.271575,314.26634658)(841.21157506,314.26134659)(841.15157959,314.26134644)
\curveto(841.11157516,314.2513466)(841.05157522,314.2463466)(840.97157959,314.24634644)
\curveto(840.90157537,314.2463466)(840.85157542,314.2513466)(840.82157959,314.26134644)
\curveto(840.78157549,314.27134658)(840.74157553,314.27634657)(840.70157959,314.27634644)
\curveto(840.66157561,314.26634658)(840.62657564,314.26634658)(840.59657959,314.27634644)
\lineto(840.50657959,314.27634644)
\lineto(840.14657959,314.32134644)
\curveto(840.00657626,314.36134649)(839.8715764,314.40134645)(839.74157959,314.44134644)
\curveto(839.61157666,314.48134637)(839.48657678,314.52634632)(839.36657959,314.57634644)
\curveto(838.91657735,314.77634607)(838.54657772,315.03634581)(838.25657959,315.35634644)
\curveto(837.9665783,315.67634517)(837.72657854,316.06634478)(837.53657959,316.52634644)
\curveto(837.48657878,316.62634422)(837.44657882,316.72634412)(837.41657959,316.82634644)
\curveto(837.39657887,316.92634392)(837.37657889,317.03134382)(837.35657959,317.14134644)
\curveto(837.33657893,317.18134367)(837.32657894,317.21134364)(837.32657959,317.23134644)
\curveto(837.33657893,317.26134359)(837.33657893,317.29634355)(837.32657959,317.33634644)
\curveto(837.30657896,317.41634343)(837.29157898,317.49634335)(837.28157959,317.57634644)
\curveto(837.28157899,317.66634318)(837.271579,317.7513431)(837.25157959,317.83134644)
\lineto(837.25157959,317.95134644)
\curveto(837.25157902,317.99134286)(837.24657902,318.03634281)(837.23657959,318.08634644)
\curveto(837.22657904,318.13634271)(837.22157905,318.22134263)(837.22157959,318.34134644)
\curveto(837.22157905,318.47134238)(837.23157904,318.56634228)(837.25157959,318.62634644)
\curveto(837.271579,318.69634215)(837.27657899,318.76634208)(837.26657959,318.83634644)
\curveto(837.25657901,318.90634194)(837.26157901,318.97634187)(837.28157959,319.04634644)
\curveto(837.29157898,319.09634175)(837.29657897,319.13634171)(837.29657959,319.16634644)
\curveto(837.30657896,319.20634164)(837.31657895,319.2513416)(837.32657959,319.30134644)
\curveto(837.35657891,319.42134143)(837.38157889,319.54134131)(837.40157959,319.66134644)
\curveto(837.43157884,319.78134107)(837.4715788,319.89634095)(837.52157959,320.00634644)
\curveto(837.6715786,320.37634047)(837.85157842,320.70634014)(838.06157959,320.99634644)
\curveto(838.28157799,321.29633955)(838.54657772,321.5463393)(838.85657959,321.74634644)
\curveto(838.97657729,321.82633902)(839.10157717,321.89133896)(839.23157959,321.94134644)
\curveto(839.36157691,322.00133885)(839.49657677,322.06133879)(839.63657959,322.12134644)
\curveto(839.75657651,322.17133868)(839.88657638,322.20133865)(840.02657959,322.21134644)
\curveto(840.1665761,322.23133862)(840.30657596,322.26133859)(840.44657959,322.30134644)
\lineto(840.64157959,322.30134644)
\curveto(840.71157556,322.31133854)(840.77657549,322.32133853)(840.83657959,322.33134644)
\curveto(841.72657454,322.34133851)(842.4665738,322.15633869)(843.05657959,321.77634644)
\curveto(843.64657262,321.39633945)(844.0715722,320.90133995)(844.33157959,320.29134644)
\curveto(844.38157189,320.19134066)(844.42157185,320.09134076)(844.45157959,319.99134644)
\curveto(844.48157179,319.89134096)(844.51657175,319.78634106)(844.55657959,319.67634644)
\curveto(844.58657168,319.56634128)(844.61157166,319.4463414)(844.63157959,319.31634644)
\curveto(844.65157162,319.19634165)(844.67657159,319.07134178)(844.70657959,318.94134644)
\curveto(844.71657155,318.89134196)(844.71657155,318.83634201)(844.70657959,318.77634644)
\curveto(844.70657156,318.72634212)(844.71157156,318.67634217)(844.72157959,318.62634644)
\moveto(843.38657959,317.77134644)
\curveto(843.40657286,317.84134301)(843.41157286,317.92134293)(843.40157959,318.01134644)
\lineto(843.40157959,318.26634644)
\curveto(843.40157287,318.65634219)(843.3665729,318.98634186)(843.29657959,319.25634644)
\curveto(843.266573,319.33634151)(843.24157303,319.41634143)(843.22157959,319.49634644)
\curveto(843.20157307,319.57634127)(843.17657309,319.6513412)(843.14657959,319.72134644)
\curveto(842.8665734,320.37134048)(842.42157385,320.82134003)(841.81157959,321.07134644)
\curveto(841.74157453,321.10133975)(841.6665746,321.12133973)(841.58657959,321.13134644)
\lineto(841.34657959,321.19134644)
\curveto(841.266575,321.21133964)(841.18157509,321.22133963)(841.09157959,321.22134644)
\lineto(840.82157959,321.22134644)
\lineto(840.55157959,321.17634644)
\curveto(840.45157582,321.15633969)(840.35657591,321.13133972)(840.26657959,321.10134644)
\curveto(840.18657608,321.08133977)(840.10657616,321.0513398)(840.02657959,321.01134644)
\curveto(839.95657631,320.99133986)(839.89157638,320.96133989)(839.83157959,320.92134644)
\curveto(839.7715765,320.88133997)(839.71657655,320.84134001)(839.66657959,320.80134644)
\curveto(839.42657684,320.63134022)(839.23157704,320.42634042)(839.08157959,320.18634644)
\curveto(838.93157734,319.9463409)(838.80157747,319.66634118)(838.69157959,319.34634644)
\curveto(838.66157761,319.2463416)(838.64157763,319.14134171)(838.63157959,319.03134644)
\curveto(838.62157765,318.93134192)(838.60657766,318.82634202)(838.58657959,318.71634644)
\curveto(838.57657769,318.67634217)(838.5715777,318.61134224)(838.57157959,318.52134644)
\curveto(838.56157771,318.49134236)(838.55657771,318.45634239)(838.55657959,318.41634644)
\curveto(838.5665777,318.37634247)(838.5715777,318.33134252)(838.57157959,318.28134644)
\lineto(838.57157959,317.98134644)
\curveto(838.5715777,317.88134297)(838.58157769,317.79134306)(838.60157959,317.71134644)
\lineto(838.63157959,317.53134644)
\curveto(838.65157762,317.43134342)(838.6665776,317.33134352)(838.67657959,317.23134644)
\curveto(838.69657757,317.14134371)(838.72657754,317.05634379)(838.76657959,316.97634644)
\curveto(838.8665774,316.73634411)(838.98157729,316.51134434)(839.11157959,316.30134644)
\curveto(839.25157702,316.09134476)(839.42157685,315.91634493)(839.62157959,315.77634644)
\curveto(839.6715766,315.7463451)(839.71657655,315.72134513)(839.75657959,315.70134644)
\curveto(839.79657647,315.68134517)(839.84157643,315.65634519)(839.89157959,315.62634644)
\curveto(839.9715763,315.57634527)(840.05657621,315.53134532)(840.14657959,315.49134644)
\curveto(840.24657602,315.46134539)(840.35157592,315.43134542)(840.46157959,315.40134644)
\curveto(840.51157576,315.38134547)(840.55657571,315.37134548)(840.59657959,315.37134644)
\curveto(840.64657562,315.38134547)(840.69657557,315.38134547)(840.74657959,315.37134644)
\curveto(840.77657549,315.36134549)(840.83657543,315.3513455)(840.92657959,315.34134644)
\curveto(841.02657524,315.33134552)(841.10157517,315.33634551)(841.15157959,315.35634644)
\curveto(841.19157508,315.36634548)(841.23157504,315.36634548)(841.27157959,315.35634644)
\curveto(841.31157496,315.35634549)(841.35157492,315.36634548)(841.39157959,315.38634644)
\curveto(841.4715748,315.40634544)(841.55157472,315.42134543)(841.63157959,315.43134644)
\curveto(841.71157456,315.4513454)(841.78657448,315.47634537)(841.85657959,315.50634644)
\curveto(842.19657407,315.6463452)(842.4715738,315.84134501)(842.68157959,316.09134644)
\curveto(842.89157338,316.34134451)(843.0665732,316.63634421)(843.20657959,316.97634644)
\curveto(843.25657301,317.09634375)(843.28657298,317.22134363)(843.29657959,317.35134644)
\curveto(843.31657295,317.49134336)(843.34657292,317.63134322)(843.38657959,317.77134644)
}
}
{
\newrgbcolor{curcolor}{0 0 0}
\pscustom[linestyle=none,fillstyle=solid,fillcolor=curcolor]
{
\newpath
\moveto(848.63986084,322.33134644)
\curveto(849.35985677,322.34133851)(849.96485617,322.25633859)(850.45486084,322.07634644)
\curveto(850.94485519,321.90633894)(851.32485481,321.60133925)(851.59486084,321.16134644)
\curveto(851.66485447,321.0513398)(851.71985441,320.93633991)(851.75986084,320.81634644)
\curveto(851.79985433,320.70634014)(851.83985429,320.58134027)(851.87986084,320.44134644)
\curveto(851.89985423,320.37134048)(851.90485423,320.29634055)(851.89486084,320.21634644)
\curveto(851.88485425,320.1463407)(851.86985426,320.09134076)(851.84986084,320.05134644)
\curveto(851.8298543,320.03134082)(851.80485433,320.01134084)(851.77486084,319.99134644)
\curveto(851.74485439,319.98134087)(851.71985441,319.96634088)(851.69986084,319.94634644)
\curveto(851.64985448,319.92634092)(851.59985453,319.92134093)(851.54986084,319.93134644)
\curveto(851.49985463,319.94134091)(851.44985468,319.94134091)(851.39986084,319.93134644)
\curveto(851.31985481,319.91134094)(851.21485492,319.90634094)(851.08486084,319.91634644)
\curveto(850.95485518,319.93634091)(850.86485527,319.96134089)(850.81486084,319.99134644)
\curveto(850.7348554,320.04134081)(850.67985545,320.10634074)(850.64986084,320.18634644)
\curveto(850.6298555,320.27634057)(850.59485554,320.36134049)(850.54486084,320.44134644)
\curveto(850.45485568,320.60134025)(850.3298558,320.7463401)(850.16986084,320.87634644)
\curveto(850.05985607,320.95633989)(849.93985619,321.01633983)(849.80986084,321.05634644)
\curveto(849.67985645,321.09633975)(849.53985659,321.13633971)(849.38986084,321.17634644)
\curveto(849.33985679,321.19633965)(849.28985684,321.20133965)(849.23986084,321.19134644)
\curveto(849.18985694,321.19133966)(849.13985699,321.19633965)(849.08986084,321.20634644)
\curveto(849.0298571,321.22633962)(848.95485718,321.23633961)(848.86486084,321.23634644)
\curveto(848.77485736,321.23633961)(848.69985743,321.22633962)(848.63986084,321.20634644)
\lineto(848.54986084,321.20634644)
\lineto(848.39986084,321.17634644)
\curveto(848.34985778,321.17633967)(848.29985783,321.17133968)(848.24986084,321.16134644)
\curveto(847.98985814,321.10133975)(847.77485836,321.01633983)(847.60486084,320.90634644)
\curveto(847.4348587,320.79634005)(847.31985881,320.61134024)(847.25986084,320.35134644)
\curveto(847.23985889,320.28134057)(847.2348589,320.21134064)(847.24486084,320.14134644)
\curveto(847.26485887,320.07134078)(847.28485885,320.01134084)(847.30486084,319.96134644)
\curveto(847.36485877,319.81134104)(847.4348587,319.70134115)(847.51486084,319.63134644)
\curveto(847.60485853,319.57134128)(847.71485842,319.50134135)(847.84486084,319.42134644)
\curveto(848.00485813,319.32134153)(848.18485795,319.2463416)(848.38486084,319.19634644)
\curveto(848.58485755,319.15634169)(848.78485735,319.10634174)(848.98486084,319.04634644)
\curveto(849.11485702,319.00634184)(849.24485689,318.97634187)(849.37486084,318.95634644)
\curveto(849.50485663,318.93634191)(849.6348565,318.90634194)(849.76486084,318.86634644)
\curveto(849.97485616,318.80634204)(850.17985595,318.7463421)(850.37986084,318.68634644)
\curveto(850.57985555,318.63634221)(850.77985535,318.57134228)(850.97986084,318.49134644)
\lineto(851.12986084,318.43134644)
\curveto(851.17985495,318.41134244)(851.2298549,318.38634246)(851.27986084,318.35634644)
\curveto(851.47985465,318.23634261)(851.65485448,318.10134275)(851.80486084,317.95134644)
\curveto(851.95485418,317.80134305)(852.07985405,317.61134324)(852.17986084,317.38134644)
\curveto(852.19985393,317.31134354)(852.21985391,317.21634363)(852.23986084,317.09634644)
\curveto(852.25985387,317.02634382)(852.26985386,316.9513439)(852.26986084,316.87134644)
\curveto(852.27985385,316.80134405)(852.28485385,316.72134413)(852.28486084,316.63134644)
\lineto(852.28486084,316.48134644)
\curveto(852.26485387,316.41134444)(852.25485388,316.34134451)(852.25486084,316.27134644)
\curveto(852.25485388,316.20134465)(852.24485389,316.13134472)(852.22486084,316.06134644)
\curveto(852.19485394,315.9513449)(852.15985397,315.846345)(852.11986084,315.74634644)
\curveto(852.07985405,315.6463452)(852.0348541,315.55634529)(851.98486084,315.47634644)
\curveto(851.82485431,315.21634563)(851.61985451,315.00634584)(851.36986084,314.84634644)
\curveto(851.11985501,314.69634615)(850.83985529,314.56634628)(850.52986084,314.45634644)
\curveto(850.43985569,314.42634642)(850.34485579,314.40634644)(850.24486084,314.39634644)
\curveto(850.15485598,314.37634647)(850.06485607,314.3513465)(849.97486084,314.32134644)
\curveto(849.87485626,314.30134655)(849.77485636,314.29134656)(849.67486084,314.29134644)
\curveto(849.57485656,314.29134656)(849.47485666,314.28134657)(849.37486084,314.26134644)
\lineto(849.22486084,314.26134644)
\curveto(849.17485696,314.2513466)(849.10485703,314.2463466)(849.01486084,314.24634644)
\curveto(848.92485721,314.2463466)(848.85485728,314.2513466)(848.80486084,314.26134644)
\lineto(848.63986084,314.26134644)
\curveto(848.57985755,314.28134657)(848.51485762,314.29134656)(848.44486084,314.29134644)
\curveto(848.37485776,314.28134657)(848.31485782,314.28634656)(848.26486084,314.30634644)
\curveto(848.21485792,314.31634653)(848.14985798,314.32134653)(848.06986084,314.32134644)
\lineto(847.82986084,314.38134644)
\curveto(847.75985837,314.39134646)(847.68485845,314.41134644)(847.60486084,314.44134644)
\curveto(847.29485884,314.54134631)(847.02485911,314.66634618)(846.79486084,314.81634644)
\curveto(846.56485957,314.96634588)(846.36485977,315.16134569)(846.19486084,315.40134644)
\curveto(846.10486003,315.53134532)(846.0298601,315.66634518)(845.96986084,315.80634644)
\curveto(845.90986022,315.9463449)(845.85486028,316.10134475)(845.80486084,316.27134644)
\curveto(845.78486035,316.33134452)(845.77486036,316.40134445)(845.77486084,316.48134644)
\curveto(845.78486035,316.57134428)(845.79986033,316.64134421)(845.81986084,316.69134644)
\curveto(845.84986028,316.73134412)(845.89986023,316.77134408)(845.96986084,316.81134644)
\curveto(846.01986011,316.83134402)(846.08986004,316.84134401)(846.17986084,316.84134644)
\curveto(846.26985986,316.851344)(846.35985977,316.851344)(846.44986084,316.84134644)
\curveto(846.53985959,316.83134402)(846.62485951,316.81634403)(846.70486084,316.79634644)
\curveto(846.79485934,316.78634406)(846.85485928,316.77134408)(846.88486084,316.75134644)
\curveto(846.95485918,316.70134415)(846.99985913,316.62634422)(847.01986084,316.52634644)
\curveto(847.04985908,316.43634441)(847.08485905,316.3513445)(847.12486084,316.27134644)
\curveto(847.22485891,316.0513448)(847.35985877,315.88134497)(847.52986084,315.76134644)
\curveto(847.64985848,315.67134518)(847.78485835,315.60134525)(847.93486084,315.55134644)
\curveto(848.08485805,315.50134535)(848.24485789,315.4513454)(848.41486084,315.40134644)
\lineto(848.72986084,315.35634644)
\lineto(848.81986084,315.35634644)
\curveto(848.88985724,315.33634551)(848.97985715,315.32634552)(849.08986084,315.32634644)
\curveto(849.20985692,315.32634552)(849.30985682,315.33634551)(849.38986084,315.35634644)
\curveto(849.45985667,315.35634549)(849.51485662,315.36134549)(849.55486084,315.37134644)
\curveto(849.61485652,315.38134547)(849.67485646,315.38634546)(849.73486084,315.38634644)
\curveto(849.79485634,315.39634545)(849.84985628,315.40634544)(849.89986084,315.41634644)
\curveto(850.18985594,315.49634535)(850.41985571,315.60134525)(850.58986084,315.73134644)
\curveto(850.75985537,315.86134499)(850.87985525,316.08134477)(850.94986084,316.39134644)
\curveto(850.96985516,316.44134441)(850.97485516,316.49634435)(850.96486084,316.55634644)
\curveto(850.95485518,316.61634423)(850.94485519,316.66134419)(850.93486084,316.69134644)
\curveto(850.88485525,316.88134397)(850.81485532,317.02134383)(850.72486084,317.11134644)
\curveto(850.6348555,317.21134364)(850.51985561,317.30134355)(850.37986084,317.38134644)
\curveto(850.28985584,317.44134341)(850.18985594,317.49134336)(850.07986084,317.53134644)
\lineto(849.74986084,317.65134644)
\curveto(849.71985641,317.66134319)(849.68985644,317.66634318)(849.65986084,317.66634644)
\curveto(849.63985649,317.66634318)(849.61485652,317.67634317)(849.58486084,317.69634644)
\curveto(849.24485689,317.80634304)(848.88985724,317.88634296)(848.51986084,317.93634644)
\curveto(848.15985797,317.99634285)(847.81985831,318.09134276)(847.49986084,318.22134644)
\curveto(847.39985873,318.26134259)(847.30485883,318.29634255)(847.21486084,318.32634644)
\curveto(847.12485901,318.35634249)(847.03985909,318.39634245)(846.95986084,318.44634644)
\curveto(846.76985936,318.55634229)(846.59485954,318.68134217)(846.43486084,318.82134644)
\curveto(846.27485986,318.96134189)(846.14985998,319.13634171)(846.05986084,319.34634644)
\curveto(846.0298601,319.41634143)(846.00486013,319.48634136)(845.98486084,319.55634644)
\curveto(845.97486016,319.62634122)(845.95986017,319.70134115)(845.93986084,319.78134644)
\curveto(845.90986022,319.90134095)(845.89986023,320.03634081)(845.90986084,320.18634644)
\curveto(845.91986021,320.3463405)(845.9348602,320.48134037)(845.95486084,320.59134644)
\curveto(845.97486016,320.64134021)(845.98486015,320.68134017)(845.98486084,320.71134644)
\curveto(845.99486014,320.7513401)(846.00986012,320.79134006)(846.02986084,320.83134644)
\curveto(846.11986001,321.06133979)(846.23985989,321.26133959)(846.38986084,321.43134644)
\curveto(846.54985958,321.60133925)(846.7298594,321.7513391)(846.92986084,321.88134644)
\curveto(847.07985905,321.97133888)(847.24485889,322.04133881)(847.42486084,322.09134644)
\curveto(847.60485853,322.1513387)(847.79485834,322.20633864)(847.99486084,322.25634644)
\curveto(848.06485807,322.26633858)(848.129858,322.27633857)(848.18986084,322.28634644)
\curveto(848.25985787,322.29633855)(848.3348578,322.30633854)(848.41486084,322.31634644)
\curveto(848.44485769,322.32633852)(848.48485765,322.32633852)(848.53486084,322.31634644)
\curveto(848.58485755,322.30633854)(848.61985751,322.31133854)(848.63986084,322.33134644)
}
}
{
\newrgbcolor{curcolor}{0 0 0}
\pscustom[linestyle=none,fillstyle=solid,fillcolor=curcolor]
{
\newpath
\moveto(771.89145996,303.0080896)
\curveto(772.87145346,303.02807864)(773.69145264,302.8680788)(774.35145996,302.5280896)
\curveto(775.02145131,302.19807947)(775.54145079,301.73807993)(775.91145996,301.1480896)
\curveto(776.01145032,300.98808068)(776.09145024,300.83308084)(776.15145996,300.6830896)
\curveto(776.22145011,300.54308113)(776.28645005,300.3730813)(776.34645996,300.1730896)
\curveto(776.36644997,300.12308155)(776.38644995,300.05308162)(776.40645996,299.9630896)
\curveto(776.42644991,299.88308179)(776.42144991,299.80808186)(776.39145996,299.7380896)
\curveto(776.37144996,299.67808199)(776.33145,299.63808203)(776.27145996,299.6180896)
\curveto(776.22145011,299.60808206)(776.16645017,299.59308208)(776.10645996,299.5730896)
\lineto(775.95645996,299.5730896)
\curveto(775.92645041,299.56308211)(775.88645045,299.55808211)(775.83645996,299.5580896)
\lineto(775.71645996,299.5580896)
\curveto(775.57645076,299.55808211)(775.44645089,299.56308211)(775.32645996,299.5730896)
\curveto(775.21645112,299.59308208)(775.1364512,299.64308203)(775.08645996,299.7230896)
\curveto(775.01645132,299.82308185)(774.96145137,299.93808173)(774.92145996,300.0680896)
\curveto(774.88145145,300.19808147)(774.82645151,300.31808135)(774.75645996,300.4280896)
\curveto(774.62645171,300.64808102)(774.47645186,300.83808083)(774.30645996,300.9980896)
\curveto(774.14645219,301.15808051)(773.95645238,301.30808036)(773.73645996,301.4480896)
\curveto(773.61645272,301.52808014)(773.48145285,301.58808008)(773.33145996,301.6280896)
\curveto(773.19145314,301.66808)(773.04645329,301.70807996)(772.89645996,301.7480896)
\curveto(772.78645355,301.77807989)(772.66145367,301.79807987)(772.52145996,301.8080896)
\curveto(772.38145395,301.82807984)(772.2314541,301.83807983)(772.07145996,301.8380896)
\curveto(771.92145441,301.83807983)(771.77145456,301.82807984)(771.62145996,301.8080896)
\curveto(771.48145485,301.79807987)(771.36145497,301.77807989)(771.26145996,301.7480896)
\curveto(771.16145517,301.72807994)(771.06645527,301.70807996)(770.97645996,301.6880896)
\curveto(770.88645545,301.66808)(770.79645554,301.63808003)(770.70645996,301.5980896)
\curveto(769.86645647,301.24808042)(769.26145707,300.64808102)(768.89145996,299.7980896)
\curveto(768.82145751,299.65808201)(768.76145757,299.50808216)(768.71145996,299.3480896)
\curveto(768.67145766,299.19808247)(768.62645771,299.04308263)(768.57645996,298.8830896)
\curveto(768.55645778,298.82308285)(768.54645779,298.75808291)(768.54645996,298.6880896)
\curveto(768.54645779,298.62808304)(768.5364578,298.5680831)(768.51645996,298.5080896)
\curveto(768.50645783,298.4680832)(768.50145783,298.43308324)(768.50145996,298.4030896)
\curveto(768.50145783,298.3730833)(768.49645784,298.33808333)(768.48645996,298.2980896)
\curveto(768.46645787,298.18808348)(768.45145788,298.0730836)(768.44145996,297.9530896)
\lineto(768.44145996,297.6080896)
\curveto(768.44145789,297.53808413)(768.4364579,297.46308421)(768.42645996,297.3830896)
\curveto(768.42645791,297.31308436)(768.4314579,297.24808442)(768.44145996,297.1880896)
\lineto(768.44145996,297.0380896)
\curveto(768.46145787,296.9680847)(768.46645787,296.89808477)(768.45645996,296.8280896)
\curveto(768.45645788,296.75808491)(768.46645787,296.68808498)(768.48645996,296.6180896)
\curveto(768.50645783,296.55808511)(768.51145782,296.49808517)(768.50145996,296.4380896)
\curveto(768.50145783,296.37808529)(768.51145782,296.32308535)(768.53145996,296.2730896)
\curveto(768.56145777,296.14308553)(768.58645775,296.01308566)(768.60645996,295.8830896)
\curveto(768.6364577,295.76308591)(768.67145766,295.64308603)(768.71145996,295.5230896)
\curveto(768.88145745,295.02308665)(769.10145723,294.59308708)(769.37145996,294.2330896)
\curveto(769.64145669,293.88308779)(769.99645634,293.59308808)(770.43645996,293.3630896)
\curveto(770.57645576,293.29308838)(770.71645562,293.23808843)(770.85645996,293.1980896)
\curveto(771.00645533,293.15808851)(771.16645517,293.11308856)(771.33645996,293.0630896)
\curveto(771.40645493,293.04308863)(771.47145486,293.03308864)(771.53145996,293.0330896)
\curveto(771.59145474,293.04308863)(771.66145467,293.03808863)(771.74145996,293.0180896)
\curveto(771.79145454,293.00808866)(771.88145445,292.99808867)(772.01145996,292.9880896)
\curveto(772.14145419,292.98808868)(772.2364541,292.99808867)(772.29645996,293.0180896)
\lineto(772.40145996,293.0180896)
\curveto(772.44145389,293.02808864)(772.48145385,293.02808864)(772.52145996,293.0180896)
\curveto(772.56145377,293.01808865)(772.60145373,293.02808864)(772.64145996,293.0480896)
\curveto(772.74145359,293.0680886)(772.8364535,293.08308859)(772.92645996,293.0930896)
\curveto(773.02645331,293.11308856)(773.12145321,293.14308853)(773.21145996,293.1830896)
\curveto(773.99145234,293.50308817)(774.54145179,294.02808764)(774.86145996,294.7580896)
\curveto(774.94145139,294.93808673)(775.01645132,295.15308652)(775.08645996,295.4030896)
\curveto(775.10645123,295.49308618)(775.12145121,295.58308609)(775.13145996,295.6730896)
\curveto(775.15145118,295.7730859)(775.18645115,295.86308581)(775.23645996,295.9430896)
\curveto(775.28645105,296.02308565)(775.36645097,296.0680856)(775.47645996,296.0780896)
\curveto(775.58645075,296.08808558)(775.70645063,296.09308558)(775.83645996,296.0930896)
\lineto(775.98645996,296.0930896)
\curveto(776.0364503,296.09308558)(776.08145025,296.08808558)(776.12145996,296.0780896)
\lineto(776.22645996,296.0780896)
\lineto(776.31645996,296.0480896)
\curveto(776.35644998,296.04808562)(776.38644995,296.03808563)(776.40645996,296.0180896)
\curveto(776.47644986,295.97808569)(776.51644982,295.90308577)(776.52645996,295.7930896)
\curveto(776.5364498,295.69308598)(776.52644981,295.59308608)(776.49645996,295.4930896)
\curveto(776.4364499,295.26308641)(776.38144995,295.04308663)(776.33145996,294.8330896)
\curveto(776.28145005,294.62308705)(776.20645013,294.42308725)(776.10645996,294.2330896)
\curveto(776.02645031,294.10308757)(775.95145038,293.97808769)(775.88145996,293.8580896)
\curveto(775.82145051,293.73808793)(775.75145058,293.61808805)(775.67145996,293.4980896)
\curveto(775.49145084,293.23808843)(775.26645107,292.99808867)(774.99645996,292.7780896)
\curveto(774.7364516,292.5680891)(774.45145188,292.39308928)(774.14145996,292.2530896)
\curveto(774.0314523,292.20308947)(773.92145241,292.16308951)(773.81145996,292.1330896)
\curveto(773.71145262,292.10308957)(773.60645273,292.0680896)(773.49645996,292.0280896)
\curveto(773.38645295,291.98808968)(773.27145306,291.96308971)(773.15145996,291.9530896)
\curveto(773.04145329,291.93308974)(772.92645341,291.91308976)(772.80645996,291.8930896)
\curveto(772.75645358,291.8730898)(772.71145362,291.8680898)(772.67145996,291.8780896)
\curveto(772.6314537,291.87808979)(772.59145374,291.8730898)(772.55145996,291.8630896)
\curveto(772.49145384,291.85308982)(772.4314539,291.84808982)(772.37145996,291.8480896)
\curveto(772.31145402,291.84808982)(772.24645409,291.84308983)(772.17645996,291.8330896)
\curveto(772.14645419,291.82308985)(772.07645426,291.82308985)(771.96645996,291.8330896)
\curveto(771.86645447,291.83308984)(771.80145453,291.83808983)(771.77145996,291.8480896)
\curveto(771.72145461,291.85808981)(771.67145466,291.86308981)(771.62145996,291.8630896)
\curveto(771.58145475,291.85308982)(771.5364548,291.85308982)(771.48645996,291.8630896)
\lineto(771.33645996,291.8630896)
\curveto(771.25645508,291.88308979)(771.18145515,291.89808977)(771.11145996,291.9080896)
\curveto(771.04145529,291.90808976)(770.96645537,291.91808975)(770.88645996,291.9380896)
\lineto(770.61645996,291.9980896)
\curveto(770.52645581,292.00808966)(770.44145589,292.02808964)(770.36145996,292.0580896)
\curveto(770.15145618,292.11808955)(769.96145637,292.19308948)(769.79145996,292.2830896)
\curveto(769.16145717,292.55308912)(768.65145768,292.93808873)(768.26145996,293.4380896)
\curveto(767.87145846,293.93808773)(767.56145877,294.52808714)(767.33145996,295.2080896)
\curveto(767.29145904,295.32808634)(767.25645908,295.45308622)(767.22645996,295.5830896)
\curveto(767.20645913,295.71308596)(767.18145915,295.84808582)(767.15145996,295.9880896)
\curveto(767.1314592,296.03808563)(767.12145921,296.08308559)(767.12145996,296.1230896)
\curveto(767.1314592,296.16308551)(767.1314592,296.20808546)(767.12145996,296.2580896)
\curveto(767.10145923,296.34808532)(767.08645925,296.44308523)(767.07645996,296.5430896)
\curveto(767.07645926,296.64308503)(767.06645927,296.73808493)(767.04645996,296.8280896)
\lineto(767.04645996,297.1130896)
\curveto(767.02645931,297.16308451)(767.01645932,297.24808442)(767.01645996,297.3680896)
\curveto(767.01645932,297.48808418)(767.02645931,297.5730841)(767.04645996,297.6230896)
\curveto(767.05645928,297.65308402)(767.05645928,297.68308399)(767.04645996,297.7130896)
\curveto(767.0364593,297.75308392)(767.0364593,297.78308389)(767.04645996,297.8030896)
\lineto(767.04645996,297.9380896)
\curveto(767.05645928,298.01808365)(767.06145927,298.09808357)(767.06145996,298.1780896)
\curveto(767.07145926,298.2680834)(767.08645925,298.35308332)(767.10645996,298.4330896)
\curveto(767.12645921,298.49308318)(767.1364592,298.55308312)(767.13645996,298.6130896)
\curveto(767.1364592,298.68308299)(767.14645919,298.75308292)(767.16645996,298.8230896)
\curveto(767.21645912,298.99308268)(767.25645908,299.15808251)(767.28645996,299.3180896)
\curveto(767.31645902,299.47808219)(767.36145897,299.62808204)(767.42145996,299.7680896)
\lineto(767.57145996,300.1580896)
\curveto(767.6314587,300.29808137)(767.69645864,300.42308125)(767.76645996,300.5330896)
\curveto(767.91645842,300.79308088)(768.06645827,301.02808064)(768.21645996,301.2380896)
\curveto(768.24645809,301.28808038)(768.28145805,301.32808034)(768.32145996,301.3580896)
\curveto(768.37145796,301.39808027)(768.41145792,301.44308023)(768.44145996,301.4930896)
\curveto(768.50145783,301.5730801)(768.56145777,301.64308003)(768.62145996,301.7030896)
\lineto(768.83145996,301.8830896)
\curveto(768.89145744,301.93307974)(768.94645739,301.97807969)(768.99645996,302.0180896)
\curveto(769.05645728,302.0680796)(769.12145721,302.11807955)(769.19145996,302.1680896)
\curveto(769.34145699,302.27807939)(769.49645684,302.3730793)(769.65645996,302.4530896)
\curveto(769.82645651,302.53307914)(770.00145633,302.61307906)(770.18145996,302.6930896)
\curveto(770.29145604,302.74307893)(770.40645593,302.77807889)(770.52645996,302.7980896)
\curveto(770.65645568,302.82807884)(770.78145555,302.86307881)(770.90145996,302.9030896)
\curveto(770.97145536,302.91307876)(771.0364553,302.92307875)(771.09645996,302.9330896)
\lineto(771.27645996,302.9630896)
\curveto(771.35645498,302.9730787)(771.4314549,302.97807869)(771.50145996,302.9780896)
\curveto(771.58145475,302.98807868)(771.66145467,302.99807867)(771.74145996,303.0080896)
\curveto(771.76145457,303.01807865)(771.78645455,303.01807865)(771.81645996,303.0080896)
\curveto(771.84645449,302.99807867)(771.87145446,302.99807867)(771.89145996,303.0080896)
}
}
{
\newrgbcolor{curcolor}{0 0 0}
\pscustom[linestyle=none,fillstyle=solid,fillcolor=curcolor]
{
\newpath
\moveto(785.01130371,292.6430896)
\curveto(785.04129588,292.48308919)(785.0262959,292.34808932)(784.96630371,292.2380896)
\curveto(784.90629602,292.13808953)(784.8262961,292.06308961)(784.72630371,292.0130896)
\curveto(784.67629625,291.99308968)(784.6212963,291.98308969)(784.56130371,291.9830896)
\curveto(784.51129641,291.98308969)(784.45629647,291.9730897)(784.39630371,291.9530896)
\curveto(784.17629675,291.90308977)(783.95629697,291.91808975)(783.73630371,291.9980896)
\curveto(783.5262974,292.0680896)(783.38129754,292.15808951)(783.30130371,292.2680896)
\curveto(783.25129767,292.33808933)(783.20629772,292.41808925)(783.16630371,292.5080896)
\curveto(783.1262978,292.60808906)(783.07629785,292.68808898)(783.01630371,292.7480896)
\curveto(782.99629793,292.7680889)(782.97129795,292.78808888)(782.94130371,292.8080896)
\curveto(782.921298,292.82808884)(782.89129803,292.83308884)(782.85130371,292.8230896)
\curveto(782.74129818,292.79308888)(782.63629829,292.73808893)(782.53630371,292.6580896)
\curveto(782.44629848,292.57808909)(782.35629857,292.50808916)(782.26630371,292.4480896)
\curveto(782.13629879,292.3680893)(781.99629893,292.29308938)(781.84630371,292.2230896)
\curveto(781.69629923,292.16308951)(781.53629939,292.10808956)(781.36630371,292.0580896)
\curveto(781.26629966,292.02808964)(781.15629977,292.00808966)(781.03630371,291.9980896)
\curveto(780.9263,291.98808968)(780.81630011,291.9730897)(780.70630371,291.9530896)
\curveto(780.65630027,291.94308973)(780.61130031,291.93808973)(780.57130371,291.9380896)
\lineto(780.46630371,291.9380896)
\curveto(780.35630057,291.91808975)(780.25130067,291.91808975)(780.15130371,291.9380896)
\lineto(780.01630371,291.9380896)
\curveto(779.96630096,291.94808972)(779.91630101,291.95308972)(779.86630371,291.9530896)
\curveto(779.81630111,291.95308972)(779.77130115,291.96308971)(779.73130371,291.9830896)
\curveto(779.69130123,291.99308968)(779.65630127,291.99808967)(779.62630371,291.9980896)
\curveto(779.60630132,291.98808968)(779.58130134,291.98808968)(779.55130371,291.9980896)
\lineto(779.31130371,292.0580896)
\curveto(779.23130169,292.0680896)(779.15630177,292.08808958)(779.08630371,292.1180896)
\curveto(778.78630214,292.24808942)(778.54130238,292.39308928)(778.35130371,292.5530896)
\curveto(778.17130275,292.72308895)(778.0213029,292.95808871)(777.90130371,293.2580896)
\curveto(777.81130311,293.47808819)(777.76630316,293.74308793)(777.76630371,294.0530896)
\lineto(777.76630371,294.3680896)
\curveto(777.77630315,294.41808725)(777.78130314,294.4680872)(777.78130371,294.5180896)
\lineto(777.81130371,294.6980896)
\lineto(777.93130371,295.0280896)
\curveto(777.97130295,295.13808653)(778.0213029,295.23808643)(778.08130371,295.3280896)
\curveto(778.26130266,295.61808605)(778.50630242,295.83308584)(778.81630371,295.9730896)
\curveto(779.1263018,296.11308556)(779.46630146,296.23808543)(779.83630371,296.3480896)
\curveto(779.97630095,296.38808528)(780.1213008,296.41808525)(780.27130371,296.4380896)
\curveto(780.4213005,296.45808521)(780.57130035,296.48308519)(780.72130371,296.5130896)
\curveto(780.79130013,296.53308514)(780.85630007,296.54308513)(780.91630371,296.5430896)
\curveto(780.98629994,296.54308513)(781.06129986,296.55308512)(781.14130371,296.5730896)
\curveto(781.21129971,296.59308508)(781.28129964,296.60308507)(781.35130371,296.6030896)
\curveto(781.4212995,296.61308506)(781.49629943,296.62808504)(781.57630371,296.6480896)
\curveto(781.8262991,296.70808496)(782.06129886,296.75808491)(782.28130371,296.7980896)
\curveto(782.50129842,296.84808482)(782.67629825,296.96308471)(782.80630371,297.1430896)
\curveto(782.86629806,297.22308445)(782.91629801,297.32308435)(782.95630371,297.4430896)
\curveto(782.99629793,297.5730841)(782.99629793,297.71308396)(782.95630371,297.8630896)
\curveto(782.89629803,298.10308357)(782.80629812,298.29308338)(782.68630371,298.4330896)
\curveto(782.57629835,298.5730831)(782.41629851,298.68308299)(782.20630371,298.7630896)
\curveto(782.08629884,298.81308286)(781.94129898,298.84808282)(781.77130371,298.8680896)
\curveto(781.61129931,298.88808278)(781.44129948,298.89808277)(781.26130371,298.8980896)
\curveto(781.08129984,298.89808277)(780.90630002,298.88808278)(780.73630371,298.8680896)
\curveto(780.56630036,298.84808282)(780.4213005,298.81808285)(780.30130371,298.7780896)
\curveto(780.13130079,298.71808295)(779.96630096,298.63308304)(779.80630371,298.5230896)
\curveto(779.7263012,298.46308321)(779.65130127,298.38308329)(779.58130371,298.2830896)
\curveto(779.5213014,298.19308348)(779.46630146,298.09308358)(779.41630371,297.9830896)
\curveto(779.38630154,297.90308377)(779.35630157,297.81808385)(779.32630371,297.7280896)
\curveto(779.30630162,297.63808403)(779.26130166,297.5680841)(779.19130371,297.5180896)
\curveto(779.15130177,297.48808418)(779.08130184,297.46308421)(778.98130371,297.4430896)
\curveto(778.89130203,297.43308424)(778.79630213,297.42808424)(778.69630371,297.4280896)
\curveto(778.59630233,297.42808424)(778.49630243,297.43308424)(778.39630371,297.4430896)
\curveto(778.30630262,297.46308421)(778.24130268,297.48808418)(778.20130371,297.5180896)
\curveto(778.16130276,297.54808412)(778.13130279,297.59808407)(778.11130371,297.6680896)
\curveto(778.09130283,297.73808393)(778.09130283,297.81308386)(778.11130371,297.8930896)
\curveto(778.14130278,298.02308365)(778.17130275,298.14308353)(778.20130371,298.2530896)
\curveto(778.24130268,298.3730833)(778.28630264,298.48808318)(778.33630371,298.5980896)
\curveto(778.5263024,298.94808272)(778.76630216,299.21808245)(779.05630371,299.4080896)
\curveto(779.34630158,299.60808206)(779.70630122,299.7680819)(780.13630371,299.8880896)
\curveto(780.23630069,299.90808176)(780.33630059,299.92308175)(780.43630371,299.9330896)
\curveto(780.54630038,299.94308173)(780.65630027,299.95808171)(780.76630371,299.9780896)
\curveto(780.80630012,299.98808168)(780.87130005,299.98808168)(780.96130371,299.9780896)
\curveto(781.05129987,299.97808169)(781.10629982,299.98808168)(781.12630371,300.0080896)
\curveto(781.8262991,300.01808165)(782.43629849,299.93808173)(782.95630371,299.7680896)
\curveto(783.47629745,299.59808207)(783.84129708,299.2730824)(784.05130371,298.7930896)
\curveto(784.14129678,298.59308308)(784.19129673,298.35808331)(784.20130371,298.0880896)
\curveto(784.2212967,297.82808384)(784.23129669,297.55308412)(784.23130371,297.2630896)
\lineto(784.23130371,293.9480896)
\curveto(784.23129669,293.80808786)(784.23629669,293.673088)(784.24630371,293.5430896)
\curveto(784.25629667,293.41308826)(784.28629664,293.30808836)(784.33630371,293.2280896)
\curveto(784.38629654,293.15808851)(784.45129647,293.10808856)(784.53130371,293.0780896)
\curveto(784.6212963,293.03808863)(784.70629622,293.00808866)(784.78630371,292.9880896)
\curveto(784.86629606,292.97808869)(784.926296,292.93308874)(784.96630371,292.8530896)
\curveto(784.98629594,292.82308885)(784.99629593,292.79308888)(784.99630371,292.7630896)
\curveto(784.99629593,292.73308894)(785.00129592,292.69308898)(785.01130371,292.6430896)
\moveto(782.86630371,294.3080896)
\curveto(782.926298,294.44808722)(782.95629797,294.60808706)(782.95630371,294.7880896)
\curveto(782.96629796,294.97808669)(782.97129795,295.1730865)(782.97130371,295.3730896)
\curveto(782.97129795,295.48308619)(782.96629796,295.58308609)(782.95630371,295.6730896)
\curveto(782.94629798,295.76308591)(782.90629802,295.83308584)(782.83630371,295.8830896)
\curveto(782.80629812,295.90308577)(782.73629819,295.91308576)(782.62630371,295.9130896)
\curveto(782.60629832,295.89308578)(782.57129835,295.88308579)(782.52130371,295.8830896)
\curveto(782.47129845,295.88308579)(782.4262985,295.8730858)(782.38630371,295.8530896)
\curveto(782.30629862,295.83308584)(782.21629871,295.81308586)(782.11630371,295.7930896)
\lineto(781.81630371,295.7330896)
\curveto(781.78629914,295.73308594)(781.75129917,295.72808594)(781.71130371,295.7180896)
\lineto(781.60630371,295.7180896)
\curveto(781.45629947,295.67808599)(781.29129963,295.65308602)(781.11130371,295.6430896)
\curveto(780.94129998,295.64308603)(780.78130014,295.62308605)(780.63130371,295.5830896)
\curveto(780.55130037,295.56308611)(780.47630045,295.54308613)(780.40630371,295.5230896)
\curveto(780.34630058,295.51308616)(780.27630065,295.49808617)(780.19630371,295.4780896)
\curveto(780.03630089,295.42808624)(779.88630104,295.36308631)(779.74630371,295.2830896)
\curveto(779.60630132,295.21308646)(779.48630144,295.12308655)(779.38630371,295.0130896)
\curveto(779.28630164,294.90308677)(779.21130171,294.7680869)(779.16130371,294.6080896)
\curveto(779.11130181,294.45808721)(779.09130183,294.2730874)(779.10130371,294.0530896)
\curveto(779.10130182,293.95308772)(779.11630181,293.85808781)(779.14630371,293.7680896)
\curveto(779.18630174,293.68808798)(779.23130169,293.61308806)(779.28130371,293.5430896)
\curveto(779.36130156,293.43308824)(779.46630146,293.33808833)(779.59630371,293.2580896)
\curveto(779.7263012,293.18808848)(779.86630106,293.12808854)(780.01630371,293.0780896)
\curveto(780.06630086,293.0680886)(780.11630081,293.06308861)(780.16630371,293.0630896)
\curveto(780.21630071,293.06308861)(780.26630066,293.05808861)(780.31630371,293.0480896)
\curveto(780.38630054,293.02808864)(780.47130045,293.01308866)(780.57130371,293.0030896)
\curveto(780.68130024,293.00308867)(780.77130015,293.01308866)(780.84130371,293.0330896)
\curveto(780.90130002,293.05308862)(780.96129996,293.05808861)(781.02130371,293.0480896)
\curveto(781.08129984,293.04808862)(781.14129978,293.05808861)(781.20130371,293.0780896)
\curveto(781.28129964,293.09808857)(781.35629957,293.11308856)(781.42630371,293.1230896)
\curveto(781.50629942,293.13308854)(781.58129934,293.15308852)(781.65130371,293.1830896)
\curveto(781.94129898,293.30308837)(782.18629874,293.44808822)(782.38630371,293.6180896)
\curveto(782.59629833,293.78808788)(782.75629817,294.01808765)(782.86630371,294.3080896)
}
}
{
\newrgbcolor{curcolor}{0 0 0}
\pscustom[linestyle=none,fillstyle=solid,fillcolor=curcolor]
{
\newpath
\moveto(786.75294434,302.7680896)
\curveto(786.88294272,302.7680789)(787.01794259,302.7680789)(787.15794434,302.7680896)
\curveto(787.3079423,302.7680789)(787.41794219,302.73307894)(787.48794434,302.6630896)
\curveto(787.53794207,302.59307908)(787.56294204,302.49807917)(787.56294434,302.3780896)
\curveto(787.57294203,302.2680794)(787.57794203,302.15307952)(787.57794434,302.0330896)
\lineto(787.57794434,300.6980896)
\lineto(787.57794434,294.6230896)
\lineto(787.57794434,292.9430896)
\lineto(787.57794434,292.5530896)
\curveto(787.57794203,292.41308926)(787.55294205,292.30308937)(787.50294434,292.2230896)
\curveto(787.47294213,292.1730895)(787.42794218,292.14308953)(787.36794434,292.1330896)
\curveto(787.31794229,292.12308955)(787.25294235,292.10808956)(787.17294434,292.0880896)
\lineto(786.96294434,292.0880896)
\lineto(786.64794434,292.0880896)
\curveto(786.54794306,292.09808957)(786.47294313,292.13308954)(786.42294434,292.1930896)
\curveto(786.37294323,292.2730894)(786.34294326,292.3730893)(786.33294434,292.4930896)
\lineto(786.33294434,292.8680896)
\lineto(786.33294434,294.2480896)
\lineto(786.33294434,300.4880896)
\lineto(786.33294434,301.9580896)
\curveto(786.33294327,302.0680796)(786.32794328,302.18307949)(786.31794434,302.3030896)
\curveto(786.31794329,302.43307924)(786.34294326,302.53307914)(786.39294434,302.6030896)
\curveto(786.43294317,302.66307901)(786.5079431,302.71307896)(786.61794434,302.7530896)
\curveto(786.63794297,302.76307891)(786.65794295,302.76307891)(786.67794434,302.7530896)
\curveto(786.7079429,302.75307892)(786.73294287,302.75807891)(786.75294434,302.7680896)
}
}
{
\newrgbcolor{curcolor}{0 0 0}
\pscustom[linestyle=none,fillstyle=solid,fillcolor=curcolor]
{
\newpath
\moveto(789.80778809,301.3130896)
\curveto(789.72778697,301.3730803)(789.68278701,301.47808019)(789.67278809,301.6280896)
\lineto(789.67278809,302.0930896)
\lineto(789.67278809,302.3480896)
\curveto(789.67278702,302.43807923)(789.68778701,302.51307916)(789.71778809,302.5730896)
\curveto(789.75778694,302.65307902)(789.83778686,302.71307896)(789.95778809,302.7530896)
\curveto(789.97778672,302.76307891)(789.9977867,302.76307891)(790.01778809,302.7530896)
\curveto(790.04778665,302.75307892)(790.07278662,302.75807891)(790.09278809,302.7680896)
\curveto(790.26278643,302.7680789)(790.42278627,302.76307891)(790.57278809,302.7530896)
\curveto(790.72278597,302.74307893)(790.82278587,302.68307899)(790.87278809,302.5730896)
\curveto(790.90278579,302.51307916)(790.91778578,302.43807923)(790.91778809,302.3480896)
\lineto(790.91778809,302.0930896)
\curveto(790.91778578,301.91307976)(790.91278578,301.74307993)(790.90278809,301.5830896)
\curveto(790.90278579,301.42308025)(790.83778586,301.31808035)(790.70778809,301.2680896)
\curveto(790.65778604,301.24808042)(790.60278609,301.23808043)(790.54278809,301.2380896)
\lineto(790.37778809,301.2380896)
\lineto(790.06278809,301.2380896)
\curveto(789.96278673,301.23808043)(789.87778682,301.26308041)(789.80778809,301.3130896)
\moveto(790.91778809,292.8080896)
\lineto(790.91778809,292.4930896)
\curveto(790.92778577,292.39308928)(790.90778579,292.31308936)(790.85778809,292.2530896)
\curveto(790.82778587,292.19308948)(790.78278591,292.15308952)(790.72278809,292.1330896)
\curveto(790.66278603,292.12308955)(790.5927861,292.10808956)(790.51278809,292.0880896)
\lineto(790.28778809,292.0880896)
\curveto(790.15778654,292.08808958)(790.04278665,292.09308958)(789.94278809,292.1030896)
\curveto(789.85278684,292.12308955)(789.78278691,292.1730895)(789.73278809,292.2530896)
\curveto(789.692787,292.31308936)(789.67278702,292.38808928)(789.67278809,292.4780896)
\lineto(789.67278809,292.7630896)
\lineto(789.67278809,299.1080896)
\lineto(789.67278809,299.4230896)
\curveto(789.67278702,299.53308214)(789.697787,299.61808205)(789.74778809,299.6780896)
\curveto(789.77778692,299.72808194)(789.81778688,299.75808191)(789.86778809,299.7680896)
\curveto(789.91778678,299.77808189)(789.97278672,299.79308188)(790.03278809,299.8130896)
\curveto(790.05278664,299.81308186)(790.07278662,299.80808186)(790.09278809,299.7980896)
\curveto(790.12278657,299.79808187)(790.14778655,299.80308187)(790.16778809,299.8130896)
\curveto(790.2977864,299.81308186)(790.42778627,299.80808186)(790.55778809,299.7980896)
\curveto(790.697786,299.79808187)(790.7927859,299.75808191)(790.84278809,299.6780896)
\curveto(790.8927858,299.61808205)(790.91778578,299.53808213)(790.91778809,299.4380896)
\lineto(790.91778809,299.1530896)
\lineto(790.91778809,292.8080896)
}
}
{
\newrgbcolor{curcolor}{0 0 0}
\pscustom[linestyle=none,fillstyle=solid,fillcolor=curcolor]
{
\newpath
\moveto(795.30763184,302.8880896)
\curveto(795.4876283,302.89807877)(795.67762811,302.89807877)(795.87763184,302.8880896)
\curveto(796.07762771,302.87807879)(796.21762757,302.81807885)(796.29763184,302.7080896)
\curveto(796.33762745,302.64807902)(796.36262742,302.5730791)(796.37263184,302.4830896)
\curveto(796.3826274,302.40307927)(796.3876274,302.31307936)(796.38763184,302.2130896)
\curveto(796.3876274,302.08307959)(796.36262742,301.97807969)(796.31263184,301.8980896)
\curveto(796.27262751,301.84807982)(796.21262757,301.81307986)(796.13263184,301.7930896)
\curveto(796.06262772,301.78307989)(795.9826278,301.77807989)(795.89263184,301.7780896)
\lineto(795.60763184,301.7780896)
\curveto(795.51762827,301.78807988)(795.43762835,301.78807988)(795.36763184,301.7780896)
\curveto(795.0876287,301.69807997)(794.90262888,301.5680801)(794.81263184,301.3880896)
\curveto(794.73262905,301.21808045)(794.69262909,300.95808071)(794.69263184,300.6080896)
\curveto(794.69262909,300.53808113)(794.6876291,300.46308121)(794.67763184,300.3830896)
\curveto(794.66762912,300.31308136)(794.67262911,300.24808142)(794.69263184,300.1880896)
\curveto(794.72262906,300.03808163)(794.787629,299.93308174)(794.88763184,299.8730896)
\curveto(794.96762882,299.84308183)(795.06762872,299.82808184)(795.18763184,299.8280896)
\lineto(795.54763184,299.8280896)
\lineto(795.77263184,299.8280896)
\curveto(795.80262798,299.80808186)(795.83262795,299.80308187)(795.86263184,299.8130896)
\curveto(795.89262789,299.82308185)(795.92262786,299.81808185)(795.95263184,299.7980896)
\curveto(796.05262773,299.7680819)(796.11762767,299.70808196)(796.14763184,299.6180896)
\curveto(796.17762761,299.53808213)(796.19262759,299.43308224)(796.19263184,299.3030896)
\curveto(796.1826276,299.26308241)(796.17762761,299.22308245)(796.17763184,299.1830896)
\lineto(796.17763184,299.0630896)
\curveto(796.14762764,298.91308276)(796.0826277,298.81308286)(795.98263184,298.7630896)
\curveto(795.85262793,298.71308296)(795.6826281,298.69808297)(795.47263184,298.7180896)
\curveto(795.27262851,298.74808292)(795.10262868,298.74308293)(794.96263184,298.7030896)
\curveto(794.8826289,298.68308299)(794.82262896,298.64308303)(794.78263184,298.5830896)
\curveto(794.74262904,298.53308314)(794.71262907,298.46308321)(794.69263184,298.3730896)
\curveto(794.67262911,298.30308337)(794.66762912,298.22308345)(794.67763184,298.1330896)
\curveto(794.6876291,298.04308363)(794.69262909,297.95808371)(794.69263184,297.8780896)
\lineto(794.69263184,296.8880896)
\lineto(794.69263184,293.7080896)
\lineto(794.69263184,292.9580896)
\lineto(794.69263184,292.7630896)
\curveto(794.70262908,292.69308898)(794.69762909,292.63308904)(794.67763184,292.5830896)
\lineto(794.67763184,292.4630896)
\lineto(794.64763184,292.3430896)
\curveto(794.63762915,292.30308937)(794.62262916,292.2680894)(794.60263184,292.2380896)
\curveto(794.55262923,292.1680895)(794.47762931,292.12808954)(794.37763184,292.1180896)
\curveto(794.27762951,292.10808956)(794.16762962,292.10308957)(794.04763184,292.1030896)
\lineto(793.76263184,292.1030896)
\curveto(793.71263007,292.12308955)(793.66263012,292.13808953)(793.61263184,292.1480896)
\curveto(793.57263021,292.1680895)(793.53763025,292.20308947)(793.50763184,292.2530896)
\curveto(793.4876303,292.28308939)(793.46763032,292.34808932)(793.44763184,292.4480896)
\lineto(793.44763184,292.5530896)
\curveto(793.42763036,292.60308907)(793.41763037,292.65308902)(793.41763184,292.7030896)
\curveto(793.42763036,292.76308891)(793.43263035,292.81808885)(793.43263184,292.8680896)
\lineto(793.43263184,293.4680896)
\lineto(793.43263184,297.5630896)
\lineto(793.43263184,297.9080896)
\curveto(793.44263034,298.02808364)(793.44263034,298.13808353)(793.43263184,298.2380896)
\curveto(793.43263035,298.34808332)(793.41263037,298.44308323)(793.37263184,298.5230896)
\curveto(793.34263044,298.60308307)(793.2876305,298.65808301)(793.20763184,298.6880896)
\curveto(793.14763064,298.71808295)(793.07763071,298.73308294)(792.99763184,298.7330896)
\lineto(792.77263184,298.7330896)
\lineto(792.53263184,298.7330896)
\curveto(792.46263132,298.73308294)(792.39763139,298.74308293)(792.33763184,298.7630896)
\curveto(792.24763154,298.80308287)(792.1826316,298.88808278)(792.14263184,299.0180896)
\curveto(792.13263165,299.0680826)(792.12763166,299.11308256)(792.12763184,299.1530896)
\lineto(792.12763184,299.2880896)
\curveto(792.12763166,299.42808224)(792.14263164,299.53808213)(792.17263184,299.6180896)
\curveto(792.20263158,299.70808196)(792.26763152,299.7680819)(792.36763184,299.7980896)
\curveto(792.43763135,299.82808184)(792.51763127,299.83808183)(792.60763184,299.8280896)
\lineto(792.89263184,299.8280896)
\curveto(792.99263079,299.82808184)(793.07763071,299.83808183)(793.14763184,299.8580896)
\curveto(793.22763056,299.87808179)(793.29263049,299.91808175)(793.34263184,299.9780896)
\curveto(793.41263037,300.05808161)(793.44263034,300.18308149)(793.43263184,300.3530896)
\lineto(793.43263184,300.8330896)
\curveto(793.43263035,301.03308064)(793.44263034,301.21808045)(793.46263184,301.3880896)
\curveto(793.49263029,301.5680801)(793.53763025,301.72807994)(793.59763184,301.8680896)
\curveto(793.70763008,302.10807956)(793.85262993,302.30307937)(794.03263184,302.4530896)
\curveto(794.22262956,302.60307907)(794.44762934,302.71807895)(794.70763184,302.7980896)
\curveto(794.76762902,302.81807885)(794.82762896,302.82807884)(794.88763184,302.8280896)
\curveto(794.95762883,302.83807883)(795.02762876,302.85307882)(795.09763184,302.8730896)
\curveto(795.11762867,302.88307879)(795.15262863,302.88307879)(795.20263184,302.8730896)
\curveto(795.25262853,302.8730788)(795.2876285,302.87807879)(795.30763184,302.8880896)
\moveto(797.57263184,301.3130896)
\curveto(797.64262614,301.26308041)(797.72762606,301.23808043)(797.82763184,301.2380896)
\lineto(798.14263184,301.2380896)
\lineto(798.30763184,301.2380896)
\curveto(798.36762542,301.23808043)(798.42262536,301.24808042)(798.47263184,301.2680896)
\curveto(798.60262518,301.31808035)(798.66762512,301.42308025)(798.66763184,301.5830896)
\curveto(798.67762511,301.74307993)(798.6826251,301.91307976)(798.68263184,302.0930896)
\lineto(798.68263184,302.3480896)
\curveto(798.6826251,302.43807923)(798.66762512,302.51307916)(798.63763184,302.5730896)
\curveto(798.5876252,302.68307899)(798.4876253,302.74307893)(798.33763184,302.7530896)
\curveto(798.1876256,302.76307891)(798.02762576,302.7680789)(797.85763184,302.7680896)
\curveto(797.83762595,302.75807891)(797.81262597,302.75307892)(797.78263184,302.7530896)
\curveto(797.76262602,302.76307891)(797.74262604,302.76307891)(797.72263184,302.7530896)
\curveto(797.60262618,302.71307896)(797.52262626,302.65307902)(797.48263184,302.5730896)
\curveto(797.45262633,302.51307916)(797.43762635,302.43807923)(797.43763184,302.3480896)
\lineto(797.43763184,302.0930896)
\lineto(797.43763184,301.6280896)
\curveto(797.44762634,301.47808019)(797.49262629,301.3730803)(797.57263184,301.3130896)
\moveto(798.68263184,299.1530896)
\lineto(798.68263184,299.4380896)
\curveto(798.6826251,299.53808213)(798.65762513,299.61808205)(798.60763184,299.6780896)
\curveto(798.55762523,299.75808191)(798.46262532,299.79808187)(798.32263184,299.7980896)
\curveto(798.19262559,299.80808186)(798.06262572,299.81308186)(797.93263184,299.8130896)
\curveto(797.91262587,299.80308187)(797.8876259,299.79808187)(797.85763184,299.7980896)
\curveto(797.83762595,299.80808186)(797.81762597,299.81308186)(797.79763184,299.8130896)
\curveto(797.73762605,299.79308188)(797.6826261,299.77808189)(797.63263184,299.7680896)
\curveto(797.5826262,299.75808191)(797.54262624,299.72808194)(797.51263184,299.6780896)
\curveto(797.46262632,299.61808205)(797.43762635,299.53308214)(797.43763184,299.4230896)
\lineto(797.43763184,299.1080896)
\lineto(797.43763184,292.7630896)
\lineto(797.43763184,292.4780896)
\curveto(797.43762635,292.38808928)(797.45762633,292.31308936)(797.49763184,292.2530896)
\curveto(797.54762624,292.1730895)(797.61762617,292.12308955)(797.70763184,292.1030896)
\curveto(797.80762598,292.09308958)(797.92262586,292.08808958)(798.05263184,292.0880896)
\lineto(798.27763184,292.0880896)
\curveto(798.35762543,292.10808956)(798.42762536,292.12308955)(798.48763184,292.1330896)
\curveto(798.54762524,292.15308952)(798.59262519,292.19308948)(798.62263184,292.2530896)
\curveto(798.67262511,292.31308936)(798.69262509,292.39308928)(798.68263184,292.4930896)
\lineto(798.68263184,292.8080896)
\lineto(798.68263184,299.1530896)
}
}
{
\newrgbcolor{curcolor}{0 0 0}
\pscustom[linestyle=none,fillstyle=solid,fillcolor=curcolor]
{
\newpath
\moveto(803.76130371,299.9930896)
\curveto(804.50129892,300.00308167)(805.11629831,299.89308178)(805.60630371,299.6630896)
\curveto(806.10629732,299.44308223)(806.50129692,299.10808256)(806.79130371,298.6580896)
\curveto(806.9212965,298.45808321)(807.03129639,298.21308346)(807.12130371,297.9230896)
\curveto(807.14129628,297.8730838)(807.15629627,297.80808386)(807.16630371,297.7280896)
\curveto(807.17629625,297.64808402)(807.17129625,297.57808409)(807.15130371,297.5180896)
\curveto(807.1212963,297.4680842)(807.07129635,297.42308425)(807.00130371,297.3830896)
\curveto(806.97129645,297.36308431)(806.94129648,297.35308432)(806.91130371,297.3530896)
\curveto(806.88129654,297.36308431)(806.84629658,297.36308431)(806.80630371,297.3530896)
\curveto(806.76629666,297.34308433)(806.7262967,297.33808433)(806.68630371,297.3380896)
\curveto(806.64629678,297.34808432)(806.60629682,297.35308432)(806.56630371,297.3530896)
\lineto(806.25130371,297.3530896)
\curveto(806.15129727,297.36308431)(806.06629736,297.39308428)(805.99630371,297.4430896)
\curveto(805.91629751,297.50308417)(805.86129756,297.58808408)(805.83130371,297.6980896)
\curveto(805.80129762,297.80808386)(805.76129766,297.90308377)(805.71130371,297.9830896)
\curveto(805.56129786,298.24308343)(805.36629806,298.44808322)(805.12630371,298.5980896)
\curveto(805.04629838,298.64808302)(804.96129846,298.68808298)(804.87130371,298.7180896)
\curveto(804.78129864,298.75808291)(804.68629874,298.79308288)(804.58630371,298.8230896)
\curveto(804.44629898,298.86308281)(804.26129916,298.88308279)(804.03130371,298.8830896)
\curveto(803.80129962,298.89308278)(803.61129981,298.8730828)(803.46130371,298.8230896)
\curveto(803.39130003,298.80308287)(803.3263001,298.78808288)(803.26630371,298.7780896)
\curveto(803.20630022,298.7680829)(803.14130028,298.75308292)(803.07130371,298.7330896)
\curveto(802.81130061,298.62308305)(802.58130084,298.4730832)(802.38130371,298.2830896)
\curveto(802.18130124,298.09308358)(802.0263014,297.8680838)(801.91630371,297.6080896)
\curveto(801.87630155,297.51808415)(801.84130158,297.42308425)(801.81130371,297.3230896)
\curveto(801.78130164,297.23308444)(801.75130167,297.13308454)(801.72130371,297.0230896)
\lineto(801.63130371,296.6180896)
\curveto(801.6213018,296.5680851)(801.61630181,296.51308516)(801.61630371,296.4530896)
\curveto(801.6263018,296.39308528)(801.6213018,296.33808533)(801.60130371,296.2880896)
\lineto(801.60130371,296.1680896)
\curveto(801.59130183,296.12808554)(801.58130184,296.06308561)(801.57130371,295.9730896)
\curveto(801.57130185,295.88308579)(801.58130184,295.81808585)(801.60130371,295.7780896)
\curveto(801.61130181,295.72808594)(801.61130181,295.67808599)(801.60130371,295.6280896)
\curveto(801.59130183,295.57808609)(801.59130183,295.52808614)(801.60130371,295.4780896)
\curveto(801.61130181,295.43808623)(801.61630181,295.3680863)(801.61630371,295.2680896)
\curveto(801.63630179,295.18808648)(801.65130177,295.10308657)(801.66130371,295.0130896)
\curveto(801.68130174,294.92308675)(801.70130172,294.83808683)(801.72130371,294.7580896)
\curveto(801.83130159,294.43808723)(801.95630147,294.15808751)(802.09630371,293.9180896)
\curveto(802.24630118,293.68808798)(802.45130097,293.48808818)(802.71130371,293.3180896)
\curveto(802.80130062,293.2680884)(802.89130053,293.22308845)(802.98130371,293.1830896)
\curveto(803.08130034,293.14308853)(803.18630024,293.10308857)(803.29630371,293.0630896)
\curveto(803.34630008,293.05308862)(803.38630004,293.04808862)(803.41630371,293.0480896)
\curveto(803.44629998,293.04808862)(803.48629994,293.04308863)(803.53630371,293.0330896)
\curveto(803.56629986,293.02308865)(803.61629981,293.01808865)(803.68630371,293.0180896)
\lineto(803.85130371,293.0180896)
\curveto(803.85129957,293.00808866)(803.87129955,293.00308867)(803.91130371,293.0030896)
\curveto(803.93129949,293.01308866)(803.95629947,293.01308866)(803.98630371,293.0030896)
\curveto(804.01629941,293.00308867)(804.04629938,293.00808866)(804.07630371,293.0180896)
\curveto(804.14629928,293.03808863)(804.21129921,293.04308863)(804.27130371,293.0330896)
\curveto(804.34129908,293.03308864)(804.41129901,293.04308863)(804.48130371,293.0630896)
\curveto(804.74129868,293.14308853)(804.96629846,293.24308843)(805.15630371,293.3630896)
\curveto(805.34629808,293.49308818)(805.50629792,293.65808801)(805.63630371,293.8580896)
\curveto(805.68629774,293.93808773)(805.73129769,294.02308765)(805.77130371,294.1130896)
\lineto(805.89130371,294.3830896)
\curveto(805.91129751,294.46308721)(805.93129749,294.53808713)(805.95130371,294.6080896)
\curveto(805.98129744,294.68808698)(806.03129739,294.75308692)(806.10130371,294.8030896)
\curveto(806.13129729,294.83308684)(806.19129723,294.85308682)(806.28130371,294.8630896)
\curveto(806.37129705,294.88308679)(806.46629696,294.89308678)(806.56630371,294.8930896)
\curveto(806.67629675,294.90308677)(806.77629665,294.90308677)(806.86630371,294.8930896)
\curveto(806.96629646,294.88308679)(807.03629639,294.86308681)(807.07630371,294.8330896)
\curveto(807.13629629,294.79308688)(807.17129625,294.73308694)(807.18130371,294.6530896)
\curveto(807.20129622,294.5730871)(807.20129622,294.48808718)(807.18130371,294.3980896)
\curveto(807.13129629,294.24808742)(807.08129634,294.10308757)(807.03130371,293.9630896)
\curveto(806.99129643,293.83308784)(806.93629649,293.70308797)(806.86630371,293.5730896)
\curveto(806.71629671,293.2730884)(806.5262969,293.00808866)(806.29630371,292.7780896)
\curveto(806.07629735,292.54808912)(805.80629762,292.36308931)(805.48630371,292.2230896)
\curveto(805.40629802,292.18308949)(805.3212981,292.14808952)(805.23130371,292.1180896)
\curveto(805.14129828,292.09808957)(805.04629838,292.0730896)(804.94630371,292.0430896)
\curveto(804.83629859,292.00308967)(804.7262987,291.98308969)(804.61630371,291.9830896)
\curveto(804.50629892,291.9730897)(804.39629903,291.95808971)(804.28630371,291.9380896)
\curveto(804.24629918,291.91808975)(804.20629922,291.91308976)(804.16630371,291.9230896)
\curveto(804.1262993,291.93308974)(804.08629934,291.93308974)(804.04630371,291.9230896)
\lineto(803.91130371,291.9230896)
\lineto(803.67130371,291.9230896)
\curveto(803.60129982,291.91308976)(803.53629989,291.91808975)(803.47630371,291.9380896)
\lineto(803.40130371,291.9380896)
\lineto(803.04130371,291.9830896)
\curveto(802.91130051,292.02308965)(802.78630064,292.05808961)(802.66630371,292.0880896)
\curveto(802.54630088,292.11808955)(802.43130099,292.15808951)(802.32130371,292.2080896)
\curveto(801.96130146,292.3680893)(801.66130176,292.55808911)(801.42130371,292.7780896)
\curveto(801.19130223,292.99808867)(800.97630245,293.2680884)(800.77630371,293.5880896)
\curveto(800.7263027,293.668088)(800.68130274,293.75808791)(800.64130371,293.8580896)
\lineto(800.52130371,294.1580896)
\curveto(800.47130295,294.2680874)(800.43630299,294.38308729)(800.41630371,294.5030896)
\curveto(800.39630303,294.62308705)(800.37130305,294.74308693)(800.34130371,294.8630896)
\curveto(800.33130309,294.90308677)(800.3263031,294.94308673)(800.32630371,294.9830896)
\curveto(800.3263031,295.02308665)(800.3213031,295.06308661)(800.31130371,295.1030896)
\curveto(800.29130313,295.16308651)(800.28130314,295.22808644)(800.28130371,295.2980896)
\curveto(800.29130313,295.3680863)(800.28630314,295.43308624)(800.26630371,295.4930896)
\lineto(800.26630371,295.6430896)
\curveto(800.25630317,295.69308598)(800.25130317,295.76308591)(800.25130371,295.8530896)
\curveto(800.25130317,295.94308573)(800.25630317,296.01308566)(800.26630371,296.0630896)
\curveto(800.27630315,296.11308556)(800.27630315,296.15808551)(800.26630371,296.1980896)
\curveto(800.26630316,296.23808543)(800.27130315,296.27808539)(800.28130371,296.3180896)
\curveto(800.30130312,296.38808528)(800.30630312,296.45808521)(800.29630371,296.5280896)
\curveto(800.29630313,296.59808507)(800.30630312,296.66308501)(800.32630371,296.7230896)
\curveto(800.36630306,296.89308478)(800.40130302,297.06308461)(800.43130371,297.2330896)
\curveto(800.46130296,297.40308427)(800.50630292,297.56308411)(800.56630371,297.7130896)
\curveto(800.77630265,298.23308344)(801.03130239,298.65308302)(801.33130371,298.9730896)
\curveto(801.63130179,299.29308238)(802.04130138,299.55808211)(802.56130371,299.7680896)
\curveto(802.67130075,299.81808185)(802.79130063,299.85308182)(802.92130371,299.8730896)
\curveto(803.05130037,299.89308178)(803.18630024,299.91808175)(803.32630371,299.9480896)
\curveto(803.39630003,299.95808171)(803.46629996,299.96308171)(803.53630371,299.9630896)
\curveto(803.60629982,299.9730817)(803.68129974,299.98308169)(803.76130371,299.9930896)
}
}
{
\newrgbcolor{curcolor}{0 0 0}
\pscustom[linestyle=none,fillstyle=solid,fillcolor=curcolor]
{
\newpath
\moveto(815.56794434,292.6430896)
\curveto(815.59793651,292.48308919)(815.58293652,292.34808932)(815.52294434,292.2380896)
\curveto(815.46293664,292.13808953)(815.38293672,292.06308961)(815.28294434,292.0130896)
\curveto(815.23293687,291.99308968)(815.17793693,291.98308969)(815.11794434,291.9830896)
\curveto(815.06793704,291.98308969)(815.01293709,291.9730897)(814.95294434,291.9530896)
\curveto(814.73293737,291.90308977)(814.51293759,291.91808975)(814.29294434,291.9980896)
\curveto(814.08293802,292.0680896)(813.93793817,292.15808951)(813.85794434,292.2680896)
\curveto(813.8079383,292.33808933)(813.76293834,292.41808925)(813.72294434,292.5080896)
\curveto(813.68293842,292.60808906)(813.63293847,292.68808898)(813.57294434,292.7480896)
\curveto(813.55293855,292.7680889)(813.52793858,292.78808888)(813.49794434,292.8080896)
\curveto(813.47793863,292.82808884)(813.44793866,292.83308884)(813.40794434,292.8230896)
\curveto(813.29793881,292.79308888)(813.19293891,292.73808893)(813.09294434,292.6580896)
\curveto(813.0029391,292.57808909)(812.91293919,292.50808916)(812.82294434,292.4480896)
\curveto(812.69293941,292.3680893)(812.55293955,292.29308938)(812.40294434,292.2230896)
\curveto(812.25293985,292.16308951)(812.09294001,292.10808956)(811.92294434,292.0580896)
\curveto(811.82294028,292.02808964)(811.71294039,292.00808966)(811.59294434,291.9980896)
\curveto(811.48294062,291.98808968)(811.37294073,291.9730897)(811.26294434,291.9530896)
\curveto(811.21294089,291.94308973)(811.16794094,291.93808973)(811.12794434,291.9380896)
\lineto(811.02294434,291.9380896)
\curveto(810.91294119,291.91808975)(810.8079413,291.91808975)(810.70794434,291.9380896)
\lineto(810.57294434,291.9380896)
\curveto(810.52294158,291.94808972)(810.47294163,291.95308972)(810.42294434,291.9530896)
\curveto(810.37294173,291.95308972)(810.32794178,291.96308971)(810.28794434,291.9830896)
\curveto(810.24794186,291.99308968)(810.21294189,291.99808967)(810.18294434,291.9980896)
\curveto(810.16294194,291.98808968)(810.13794197,291.98808968)(810.10794434,291.9980896)
\lineto(809.86794434,292.0580896)
\curveto(809.78794232,292.0680896)(809.71294239,292.08808958)(809.64294434,292.1180896)
\curveto(809.34294276,292.24808942)(809.09794301,292.39308928)(808.90794434,292.5530896)
\curveto(808.72794338,292.72308895)(808.57794353,292.95808871)(808.45794434,293.2580896)
\curveto(808.36794374,293.47808819)(808.32294378,293.74308793)(808.32294434,294.0530896)
\lineto(808.32294434,294.3680896)
\curveto(808.33294377,294.41808725)(808.33794377,294.4680872)(808.33794434,294.5180896)
\lineto(808.36794434,294.6980896)
\lineto(808.48794434,295.0280896)
\curveto(808.52794358,295.13808653)(808.57794353,295.23808643)(808.63794434,295.3280896)
\curveto(808.81794329,295.61808605)(809.06294304,295.83308584)(809.37294434,295.9730896)
\curveto(809.68294242,296.11308556)(810.02294208,296.23808543)(810.39294434,296.3480896)
\curveto(810.53294157,296.38808528)(810.67794143,296.41808525)(810.82794434,296.4380896)
\curveto(810.97794113,296.45808521)(811.12794098,296.48308519)(811.27794434,296.5130896)
\curveto(811.34794076,296.53308514)(811.41294069,296.54308513)(811.47294434,296.5430896)
\curveto(811.54294056,296.54308513)(811.61794049,296.55308512)(811.69794434,296.5730896)
\curveto(811.76794034,296.59308508)(811.83794027,296.60308507)(811.90794434,296.6030896)
\curveto(811.97794013,296.61308506)(812.05294005,296.62808504)(812.13294434,296.6480896)
\curveto(812.38293972,296.70808496)(812.61793949,296.75808491)(812.83794434,296.7980896)
\curveto(813.05793905,296.84808482)(813.23293887,296.96308471)(813.36294434,297.1430896)
\curveto(813.42293868,297.22308445)(813.47293863,297.32308435)(813.51294434,297.4430896)
\curveto(813.55293855,297.5730841)(813.55293855,297.71308396)(813.51294434,297.8630896)
\curveto(813.45293865,298.10308357)(813.36293874,298.29308338)(813.24294434,298.4330896)
\curveto(813.13293897,298.5730831)(812.97293913,298.68308299)(812.76294434,298.7630896)
\curveto(812.64293946,298.81308286)(812.49793961,298.84808282)(812.32794434,298.8680896)
\curveto(812.16793994,298.88808278)(811.99794011,298.89808277)(811.81794434,298.8980896)
\curveto(811.63794047,298.89808277)(811.46294064,298.88808278)(811.29294434,298.8680896)
\curveto(811.12294098,298.84808282)(810.97794113,298.81808285)(810.85794434,298.7780896)
\curveto(810.68794142,298.71808295)(810.52294158,298.63308304)(810.36294434,298.5230896)
\curveto(810.28294182,298.46308321)(810.2079419,298.38308329)(810.13794434,298.2830896)
\curveto(810.07794203,298.19308348)(810.02294208,298.09308358)(809.97294434,297.9830896)
\curveto(809.94294216,297.90308377)(809.91294219,297.81808385)(809.88294434,297.7280896)
\curveto(809.86294224,297.63808403)(809.81794229,297.5680841)(809.74794434,297.5180896)
\curveto(809.7079424,297.48808418)(809.63794247,297.46308421)(809.53794434,297.4430896)
\curveto(809.44794266,297.43308424)(809.35294275,297.42808424)(809.25294434,297.4280896)
\curveto(809.15294295,297.42808424)(809.05294305,297.43308424)(808.95294434,297.4430896)
\curveto(808.86294324,297.46308421)(808.79794331,297.48808418)(808.75794434,297.5180896)
\curveto(808.71794339,297.54808412)(808.68794342,297.59808407)(808.66794434,297.6680896)
\curveto(808.64794346,297.73808393)(808.64794346,297.81308386)(808.66794434,297.8930896)
\curveto(808.69794341,298.02308365)(808.72794338,298.14308353)(808.75794434,298.2530896)
\curveto(808.79794331,298.3730833)(808.84294326,298.48808318)(808.89294434,298.5980896)
\curveto(809.08294302,298.94808272)(809.32294278,299.21808245)(809.61294434,299.4080896)
\curveto(809.9029422,299.60808206)(810.26294184,299.7680819)(810.69294434,299.8880896)
\curveto(810.79294131,299.90808176)(810.89294121,299.92308175)(810.99294434,299.9330896)
\curveto(811.102941,299.94308173)(811.21294089,299.95808171)(811.32294434,299.9780896)
\curveto(811.36294074,299.98808168)(811.42794068,299.98808168)(811.51794434,299.9780896)
\curveto(811.6079405,299.97808169)(811.66294044,299.98808168)(811.68294434,300.0080896)
\curveto(812.38293972,300.01808165)(812.99293911,299.93808173)(813.51294434,299.7680896)
\curveto(814.03293807,299.59808207)(814.39793771,299.2730824)(814.60794434,298.7930896)
\curveto(814.69793741,298.59308308)(814.74793736,298.35808331)(814.75794434,298.0880896)
\curveto(814.77793733,297.82808384)(814.78793732,297.55308412)(814.78794434,297.2630896)
\lineto(814.78794434,293.9480896)
\curveto(814.78793732,293.80808786)(814.79293731,293.673088)(814.80294434,293.5430896)
\curveto(814.81293729,293.41308826)(814.84293726,293.30808836)(814.89294434,293.2280896)
\curveto(814.94293716,293.15808851)(815.0079371,293.10808856)(815.08794434,293.0780896)
\curveto(815.17793693,293.03808863)(815.26293684,293.00808866)(815.34294434,292.9880896)
\curveto(815.42293668,292.97808869)(815.48293662,292.93308874)(815.52294434,292.8530896)
\curveto(815.54293656,292.82308885)(815.55293655,292.79308888)(815.55294434,292.7630896)
\curveto(815.55293655,292.73308894)(815.55793655,292.69308898)(815.56794434,292.6430896)
\moveto(813.42294434,294.3080896)
\curveto(813.48293862,294.44808722)(813.51293859,294.60808706)(813.51294434,294.7880896)
\curveto(813.52293858,294.97808669)(813.52793858,295.1730865)(813.52794434,295.3730896)
\curveto(813.52793858,295.48308619)(813.52293858,295.58308609)(813.51294434,295.6730896)
\curveto(813.5029386,295.76308591)(813.46293864,295.83308584)(813.39294434,295.8830896)
\curveto(813.36293874,295.90308577)(813.29293881,295.91308576)(813.18294434,295.9130896)
\curveto(813.16293894,295.89308578)(813.12793898,295.88308579)(813.07794434,295.8830896)
\curveto(813.02793908,295.88308579)(812.98293912,295.8730858)(812.94294434,295.8530896)
\curveto(812.86293924,295.83308584)(812.77293933,295.81308586)(812.67294434,295.7930896)
\lineto(812.37294434,295.7330896)
\curveto(812.34293976,295.73308594)(812.3079398,295.72808594)(812.26794434,295.7180896)
\lineto(812.16294434,295.7180896)
\curveto(812.01294009,295.67808599)(811.84794026,295.65308602)(811.66794434,295.6430896)
\curveto(811.49794061,295.64308603)(811.33794077,295.62308605)(811.18794434,295.5830896)
\curveto(811.107941,295.56308611)(811.03294107,295.54308613)(810.96294434,295.5230896)
\curveto(810.9029412,295.51308616)(810.83294127,295.49808617)(810.75294434,295.4780896)
\curveto(810.59294151,295.42808624)(810.44294166,295.36308631)(810.30294434,295.2830896)
\curveto(810.16294194,295.21308646)(810.04294206,295.12308655)(809.94294434,295.0130896)
\curveto(809.84294226,294.90308677)(809.76794234,294.7680869)(809.71794434,294.6080896)
\curveto(809.66794244,294.45808721)(809.64794246,294.2730874)(809.65794434,294.0530896)
\curveto(809.65794245,293.95308772)(809.67294243,293.85808781)(809.70294434,293.7680896)
\curveto(809.74294236,293.68808798)(809.78794232,293.61308806)(809.83794434,293.5430896)
\curveto(809.91794219,293.43308824)(810.02294208,293.33808833)(810.15294434,293.2580896)
\curveto(810.28294182,293.18808848)(810.42294168,293.12808854)(810.57294434,293.0780896)
\curveto(810.62294148,293.0680886)(810.67294143,293.06308861)(810.72294434,293.0630896)
\curveto(810.77294133,293.06308861)(810.82294128,293.05808861)(810.87294434,293.0480896)
\curveto(810.94294116,293.02808864)(811.02794108,293.01308866)(811.12794434,293.0030896)
\curveto(811.23794087,293.00308867)(811.32794078,293.01308866)(811.39794434,293.0330896)
\curveto(811.45794065,293.05308862)(811.51794059,293.05808861)(811.57794434,293.0480896)
\curveto(811.63794047,293.04808862)(811.69794041,293.05808861)(811.75794434,293.0780896)
\curveto(811.83794027,293.09808857)(811.91294019,293.11308856)(811.98294434,293.1230896)
\curveto(812.06294004,293.13308854)(812.13793997,293.15308852)(812.20794434,293.1830896)
\curveto(812.49793961,293.30308837)(812.74293936,293.44808822)(812.94294434,293.6180896)
\curveto(813.15293895,293.78808788)(813.31293879,294.01808765)(813.42294434,294.3080896)
}
}
{
\newrgbcolor{curcolor}{0 0 0}
\pscustom[linestyle=none,fillstyle=solid,fillcolor=curcolor]
{
\newpath
\moveto(823.69958496,292.8980896)
\lineto(823.69958496,292.5080896)
\curveto(823.69957709,292.38808928)(823.67457711,292.28808938)(823.62458496,292.2080896)
\curveto(823.57457721,292.13808953)(823.4895773,292.09808957)(823.36958496,292.0880896)
\lineto(823.02458496,292.0880896)
\curveto(822.96457782,292.08808958)(822.90457788,292.08308959)(822.84458496,292.0730896)
\curveto(822.79457799,292.0730896)(822.74957804,292.08308959)(822.70958496,292.1030896)
\curveto(822.61957817,292.12308955)(822.55957823,292.16308951)(822.52958496,292.2230896)
\curveto(822.4895783,292.2730894)(822.46457832,292.33308934)(822.45458496,292.4030896)
\curveto(822.45457833,292.4730892)(822.43957835,292.54308913)(822.40958496,292.6130896)
\curveto(822.39957839,292.63308904)(822.3845784,292.64808902)(822.36458496,292.6580896)
\curveto(822.35457843,292.67808899)(822.33957845,292.69808897)(822.31958496,292.7180896)
\curveto(822.21957857,292.72808894)(822.13957865,292.70808896)(822.07958496,292.6580896)
\curveto(822.02957876,292.60808906)(821.97457881,292.55808911)(821.91458496,292.5080896)
\curveto(821.71457907,292.35808931)(821.51457927,292.24308943)(821.31458496,292.1630896)
\curveto(821.13457965,292.08308959)(820.92457986,292.02308965)(820.68458496,291.9830896)
\curveto(820.45458033,291.94308973)(820.21458057,291.92308975)(819.96458496,291.9230896)
\curveto(819.72458106,291.91308976)(819.4845813,291.92808974)(819.24458496,291.9680896)
\curveto(819.00458178,291.99808967)(818.79458199,292.05308962)(818.61458496,292.1330896)
\curveto(818.09458269,292.35308932)(817.67458311,292.64808902)(817.35458496,293.0180896)
\curveto(817.03458375,293.39808827)(816.784584,293.8680878)(816.60458496,294.4280896)
\curveto(816.56458422,294.51808715)(816.53458425,294.60808706)(816.51458496,294.6980896)
\curveto(816.50458428,294.79808687)(816.4845843,294.89808677)(816.45458496,294.9980896)
\curveto(816.44458434,295.04808662)(816.43958435,295.09808657)(816.43958496,295.1480896)
\curveto(816.43958435,295.19808647)(816.43458435,295.24808642)(816.42458496,295.2980896)
\curveto(816.40458438,295.34808632)(816.39458439,295.39808627)(816.39458496,295.4480896)
\curveto(816.40458438,295.50808616)(816.40458438,295.56308611)(816.39458496,295.6130896)
\lineto(816.39458496,295.7630896)
\curveto(816.37458441,295.81308586)(816.36458442,295.87808579)(816.36458496,295.9580896)
\curveto(816.36458442,296.03808563)(816.37458441,296.10308557)(816.39458496,296.1530896)
\lineto(816.39458496,296.3180896)
\curveto(816.41458437,296.38808528)(816.41958437,296.45808521)(816.40958496,296.5280896)
\curveto(816.40958438,296.60808506)(816.41958437,296.68308499)(816.43958496,296.7530896)
\curveto(816.44958434,296.80308487)(816.45458433,296.84808482)(816.45458496,296.8880896)
\curveto(816.45458433,296.92808474)(816.45958433,296.9730847)(816.46958496,297.0230896)
\curveto(816.49958429,297.12308455)(816.52458426,297.21808445)(816.54458496,297.3080896)
\curveto(816.56458422,297.40808426)(816.5895842,297.50308417)(816.61958496,297.5930896)
\curveto(816.74958404,297.9730837)(816.91458387,298.31308336)(817.11458496,298.6130896)
\curveto(817.32458346,298.92308275)(817.57458321,299.17808249)(817.86458496,299.3780896)
\curveto(818.03458275,299.49808217)(818.20958258,299.59808207)(818.38958496,299.6780896)
\curveto(818.57958221,299.75808191)(818.784582,299.82808184)(819.00458496,299.8880896)
\curveto(819.07458171,299.89808177)(819.13958165,299.90808176)(819.19958496,299.9180896)
\curveto(819.26958152,299.92808174)(819.33958145,299.94308173)(819.40958496,299.9630896)
\lineto(819.55958496,299.9630896)
\curveto(819.63958115,299.98308169)(819.75458103,299.99308168)(819.90458496,299.9930896)
\curveto(820.06458072,299.99308168)(820.1845806,299.98308169)(820.26458496,299.9630896)
\curveto(820.30458048,299.95308172)(820.35958043,299.94808172)(820.42958496,299.9480896)
\curveto(820.53958025,299.91808175)(820.64958014,299.89308178)(820.75958496,299.8730896)
\curveto(820.86957992,299.86308181)(820.97457981,299.83308184)(821.07458496,299.7830896)
\curveto(821.22457956,299.72308195)(821.36457942,299.65808201)(821.49458496,299.5880896)
\curveto(821.63457915,299.51808215)(821.76457902,299.43808223)(821.88458496,299.3480896)
\curveto(821.94457884,299.29808237)(822.00457878,299.24308243)(822.06458496,299.1830896)
\curveto(822.13457865,299.13308254)(822.22457856,299.11808255)(822.33458496,299.1380896)
\curveto(822.35457843,299.1680825)(822.36957842,299.19308248)(822.37958496,299.2130896)
\curveto(822.39957839,299.23308244)(822.41457837,299.26308241)(822.42458496,299.3030896)
\curveto(822.45457833,299.39308228)(822.46457832,299.50808216)(822.45458496,299.6480896)
\lineto(822.45458496,300.0230896)
\lineto(822.45458496,301.7480896)
\lineto(822.45458496,302.2130896)
\curveto(822.45457833,302.39307928)(822.47957831,302.52307915)(822.52958496,302.6030896)
\curveto(822.56957822,302.673079)(822.62957816,302.71807895)(822.70958496,302.7380896)
\curveto(822.72957806,302.73807893)(822.75457803,302.73807893)(822.78458496,302.7380896)
\curveto(822.81457797,302.74807892)(822.83957795,302.75307892)(822.85958496,302.7530896)
\curveto(822.99957779,302.76307891)(823.14457764,302.76307891)(823.29458496,302.7530896)
\curveto(823.45457733,302.75307892)(823.56457722,302.71307896)(823.62458496,302.6330896)
\curveto(823.67457711,302.55307912)(823.69957709,302.45307922)(823.69958496,302.3330896)
\lineto(823.69958496,301.9580896)
\lineto(823.69958496,292.8980896)
\moveto(822.48458496,295.7330896)
\curveto(822.50457828,295.78308589)(822.51457827,295.84808582)(822.51458496,295.9280896)
\curveto(822.51457827,296.01808565)(822.50457828,296.08808558)(822.48458496,296.1380896)
\lineto(822.48458496,296.3630896)
\curveto(822.46457832,296.45308522)(822.44957834,296.54308513)(822.43958496,296.6330896)
\curveto(822.42957836,296.73308494)(822.40957838,296.82308485)(822.37958496,296.9030896)
\curveto(822.35957843,296.98308469)(822.33957845,297.05808461)(822.31958496,297.1280896)
\curveto(822.30957848,297.19808447)(822.2895785,297.2680844)(822.25958496,297.3380896)
\curveto(822.13957865,297.63808403)(821.9845788,297.90308377)(821.79458496,298.1330896)
\curveto(821.60457918,298.36308331)(821.36457942,298.54308313)(821.07458496,298.6730896)
\curveto(820.97457981,298.72308295)(820.86957992,298.75808291)(820.75958496,298.7780896)
\curveto(820.65958013,298.80808286)(820.54958024,298.83308284)(820.42958496,298.8530896)
\curveto(820.34958044,298.8730828)(820.25958053,298.88308279)(820.15958496,298.8830896)
\lineto(819.88958496,298.8830896)
\curveto(819.83958095,298.8730828)(819.79458099,298.86308281)(819.75458496,298.8530896)
\lineto(819.61958496,298.8530896)
\curveto(819.53958125,298.83308284)(819.45458133,298.81308286)(819.36458496,298.7930896)
\curveto(819.2845815,298.7730829)(819.20458158,298.74808292)(819.12458496,298.7180896)
\curveto(818.80458198,298.57808309)(818.54458224,298.3730833)(818.34458496,298.1030896)
\curveto(818.15458263,297.84308383)(817.99958279,297.53808413)(817.87958496,297.1880896)
\curveto(817.83958295,297.07808459)(817.80958298,296.96308471)(817.78958496,296.8430896)
\curveto(817.77958301,296.73308494)(817.76458302,296.62308505)(817.74458496,296.5130896)
\curveto(817.74458304,296.4730852)(817.73958305,296.43308524)(817.72958496,296.3930896)
\lineto(817.72958496,296.2880896)
\curveto(817.70958308,296.23808543)(817.69958309,296.18308549)(817.69958496,296.1230896)
\curveto(817.70958308,296.06308561)(817.71458307,296.00808566)(817.71458496,295.9580896)
\lineto(817.71458496,295.6280896)
\curveto(817.71458307,295.52808614)(817.72458306,295.43308624)(817.74458496,295.3430896)
\curveto(817.75458303,295.31308636)(817.75958303,295.26308641)(817.75958496,295.1930896)
\curveto(817.77958301,295.12308655)(817.79458299,295.05308662)(817.80458496,294.9830896)
\lineto(817.86458496,294.7730896)
\curveto(817.97458281,294.42308725)(818.12458266,294.12308755)(818.31458496,293.8730896)
\curveto(818.50458228,293.62308805)(818.74458204,293.41808825)(819.03458496,293.2580896)
\curveto(819.12458166,293.20808846)(819.21458157,293.1680885)(819.30458496,293.1380896)
\curveto(819.39458139,293.10808856)(819.49458129,293.07808859)(819.60458496,293.0480896)
\curveto(819.65458113,293.02808864)(819.70458108,293.02308865)(819.75458496,293.0330896)
\curveto(819.81458097,293.04308863)(819.86958092,293.03808863)(819.91958496,293.0180896)
\curveto(819.95958083,293.00808866)(819.99958079,293.00308867)(820.03958496,293.0030896)
\lineto(820.17458496,293.0030896)
\lineto(820.30958496,293.0030896)
\curveto(820.33958045,293.01308866)(820.3895804,293.01808865)(820.45958496,293.0180896)
\curveto(820.53958025,293.03808863)(820.61958017,293.05308862)(820.69958496,293.0630896)
\curveto(820.77958001,293.08308859)(820.85457993,293.10808856)(820.92458496,293.1380896)
\curveto(821.25457953,293.27808839)(821.51957927,293.45308822)(821.71958496,293.6630896)
\curveto(821.92957886,293.88308779)(822.10457868,294.15808751)(822.24458496,294.4880896)
\curveto(822.29457849,294.59808707)(822.32957846,294.70808696)(822.34958496,294.8180896)
\curveto(822.36957842,294.92808674)(822.39457839,295.03808663)(822.42458496,295.1480896)
\curveto(822.44457834,295.18808648)(822.45457833,295.22308645)(822.45458496,295.2530896)
\curveto(822.45457833,295.29308638)(822.45957833,295.33308634)(822.46958496,295.3730896)
\curveto(822.47957831,295.43308624)(822.47957831,295.49308618)(822.46958496,295.5530896)
\curveto(822.46957832,295.61308606)(822.47457831,295.673086)(822.48458496,295.7330896)
}
}
{
\newrgbcolor{curcolor}{0 0 0}
\pscustom[linestyle=none,fillstyle=solid,fillcolor=curcolor]
{
\newpath
\moveto(832.77083496,296.2880896)
\curveto(832.7908269,296.22808544)(832.80082689,296.13308554)(832.80083496,296.0030896)
\curveto(832.80082689,295.88308579)(832.7958269,295.79808587)(832.78583496,295.7480896)
\lineto(832.78583496,295.5980896)
\curveto(832.77582692,295.51808615)(832.76582693,295.44308623)(832.75583496,295.3730896)
\curveto(832.75582694,295.31308636)(832.75082694,295.24308643)(832.74083496,295.1630896)
\curveto(832.72082697,295.10308657)(832.70582699,295.04308663)(832.69583496,294.9830896)
\curveto(832.695827,294.92308675)(832.68582701,294.86308681)(832.66583496,294.8030896)
\curveto(832.62582707,294.673087)(832.5908271,294.54308713)(832.56083496,294.4130896)
\curveto(832.53082716,294.28308739)(832.4908272,294.16308751)(832.44083496,294.0530896)
\curveto(832.23082746,293.5730881)(831.95082774,293.1680885)(831.60083496,292.8380896)
\curveto(831.25082844,292.51808915)(830.82082887,292.2730894)(830.31083496,292.1030896)
\curveto(830.20082949,292.06308961)(830.08082961,292.03308964)(829.95083496,292.0130896)
\curveto(829.83082986,291.99308968)(829.70582999,291.9730897)(829.57583496,291.9530896)
\curveto(829.51583018,291.94308973)(829.45083024,291.93808973)(829.38083496,291.9380896)
\curveto(829.32083037,291.92808974)(829.26083043,291.92308975)(829.20083496,291.9230896)
\curveto(829.16083053,291.91308976)(829.10083059,291.90808976)(829.02083496,291.9080896)
\curveto(828.95083074,291.90808976)(828.90083079,291.91308976)(828.87083496,291.9230896)
\curveto(828.83083086,291.93308974)(828.7908309,291.93808973)(828.75083496,291.9380896)
\curveto(828.71083098,291.92808974)(828.67583102,291.92808974)(828.64583496,291.9380896)
\lineto(828.55583496,291.9380896)
\lineto(828.19583496,291.9830896)
\curveto(828.05583164,292.02308965)(827.92083177,292.06308961)(827.79083496,292.1030896)
\curveto(827.66083203,292.14308953)(827.53583216,292.18808948)(827.41583496,292.2380896)
\curveto(826.96583273,292.43808923)(826.5958331,292.69808897)(826.30583496,293.0180896)
\curveto(826.01583368,293.33808833)(825.77583392,293.72808794)(825.58583496,294.1880896)
\curveto(825.53583416,294.28808738)(825.4958342,294.38808728)(825.46583496,294.4880896)
\curveto(825.44583425,294.58808708)(825.42583427,294.69308698)(825.40583496,294.8030896)
\curveto(825.38583431,294.84308683)(825.37583432,294.8730868)(825.37583496,294.8930896)
\curveto(825.38583431,294.92308675)(825.38583431,294.95808671)(825.37583496,294.9980896)
\curveto(825.35583434,295.07808659)(825.34083435,295.15808651)(825.33083496,295.2380896)
\curveto(825.33083436,295.32808634)(825.32083437,295.41308626)(825.30083496,295.4930896)
\lineto(825.30083496,295.6130896)
\curveto(825.30083439,295.65308602)(825.2958344,295.69808597)(825.28583496,295.7480896)
\curveto(825.27583442,295.79808587)(825.27083442,295.88308579)(825.27083496,296.0030896)
\curveto(825.27083442,296.13308554)(825.28083441,296.22808544)(825.30083496,296.2880896)
\curveto(825.32083437,296.35808531)(825.32583437,296.42808524)(825.31583496,296.4980896)
\curveto(825.30583439,296.5680851)(825.31083438,296.63808503)(825.33083496,296.7080896)
\curveto(825.34083435,296.75808491)(825.34583435,296.79808487)(825.34583496,296.8280896)
\curveto(825.35583434,296.8680848)(825.36583433,296.91308476)(825.37583496,296.9630896)
\curveto(825.40583429,297.08308459)(825.43083426,297.20308447)(825.45083496,297.3230896)
\curveto(825.48083421,297.44308423)(825.52083417,297.55808411)(825.57083496,297.6680896)
\curveto(825.72083397,298.03808363)(825.90083379,298.3680833)(826.11083496,298.6580896)
\curveto(826.33083336,298.95808271)(826.5958331,299.20808246)(826.90583496,299.4080896)
\curveto(827.02583267,299.48808218)(827.15083254,299.55308212)(827.28083496,299.6030896)
\curveto(827.41083228,299.66308201)(827.54583215,299.72308195)(827.68583496,299.7830896)
\curveto(827.80583189,299.83308184)(827.93583176,299.86308181)(828.07583496,299.8730896)
\curveto(828.21583148,299.89308178)(828.35583134,299.92308175)(828.49583496,299.9630896)
\lineto(828.69083496,299.9630896)
\curveto(828.76083093,299.9730817)(828.82583087,299.98308169)(828.88583496,299.9930896)
\curveto(829.77582992,300.00308167)(830.51582918,299.81808185)(831.10583496,299.4380896)
\curveto(831.695828,299.05808261)(832.12082757,298.56308311)(832.38083496,297.9530896)
\curveto(832.43082726,297.85308382)(832.47082722,297.75308392)(832.50083496,297.6530896)
\curveto(832.53082716,297.55308412)(832.56582713,297.44808422)(832.60583496,297.3380896)
\curveto(832.63582706,297.22808444)(832.66082703,297.10808456)(832.68083496,296.9780896)
\curveto(832.70082699,296.85808481)(832.72582697,296.73308494)(832.75583496,296.6030896)
\curveto(832.76582693,296.55308512)(832.76582693,296.49808517)(832.75583496,296.4380896)
\curveto(832.75582694,296.38808528)(832.76082693,296.33808533)(832.77083496,296.2880896)
\moveto(831.43583496,295.4330896)
\curveto(831.45582824,295.50308617)(831.46082823,295.58308609)(831.45083496,295.6730896)
\lineto(831.45083496,295.9280896)
\curveto(831.45082824,296.31808535)(831.41582828,296.64808502)(831.34583496,296.9180896)
\curveto(831.31582838,296.99808467)(831.2908284,297.07808459)(831.27083496,297.1580896)
\curveto(831.25082844,297.23808443)(831.22582847,297.31308436)(831.19583496,297.3830896)
\curveto(830.91582878,298.03308364)(830.47082922,298.48308319)(829.86083496,298.7330896)
\curveto(829.7908299,298.76308291)(829.71582998,298.78308289)(829.63583496,298.7930896)
\lineto(829.39583496,298.8530896)
\curveto(829.31583038,298.8730828)(829.23083046,298.88308279)(829.14083496,298.8830896)
\lineto(828.87083496,298.8830896)
\lineto(828.60083496,298.8380896)
\curveto(828.50083119,298.81808285)(828.40583129,298.79308288)(828.31583496,298.7630896)
\curveto(828.23583146,298.74308293)(828.15583154,298.71308296)(828.07583496,298.6730896)
\curveto(828.00583169,298.65308302)(827.94083175,298.62308305)(827.88083496,298.5830896)
\curveto(827.82083187,298.54308313)(827.76583193,298.50308317)(827.71583496,298.4630896)
\curveto(827.47583222,298.29308338)(827.28083241,298.08808358)(827.13083496,297.8480896)
\curveto(826.98083271,297.60808406)(826.85083284,297.32808434)(826.74083496,297.0080896)
\curveto(826.71083298,296.90808476)(826.690833,296.80308487)(826.68083496,296.6930896)
\curveto(826.67083302,296.59308508)(826.65583304,296.48808518)(826.63583496,296.3780896)
\curveto(826.62583307,296.33808533)(826.62083307,296.2730854)(826.62083496,296.1830896)
\curveto(826.61083308,296.15308552)(826.60583309,296.11808555)(826.60583496,296.0780896)
\curveto(826.61583308,296.03808563)(826.62083307,295.99308568)(826.62083496,295.9430896)
\lineto(826.62083496,295.6430896)
\curveto(826.62083307,295.54308613)(826.63083306,295.45308622)(826.65083496,295.3730896)
\lineto(826.68083496,295.1930896)
\curveto(826.70083299,295.09308658)(826.71583298,294.99308668)(826.72583496,294.8930896)
\curveto(826.74583295,294.80308687)(826.77583292,294.71808695)(826.81583496,294.6380896)
\curveto(826.91583278,294.39808727)(827.03083266,294.1730875)(827.16083496,293.9630896)
\curveto(827.30083239,293.75308792)(827.47083222,293.57808809)(827.67083496,293.4380896)
\curveto(827.72083197,293.40808826)(827.76583193,293.38308829)(827.80583496,293.3630896)
\curveto(827.84583185,293.34308833)(827.8908318,293.31808835)(827.94083496,293.2880896)
\curveto(828.02083167,293.23808843)(828.10583159,293.19308848)(828.19583496,293.1530896)
\curveto(828.2958314,293.12308855)(828.40083129,293.09308858)(828.51083496,293.0630896)
\curveto(828.56083113,293.04308863)(828.60583109,293.03308864)(828.64583496,293.0330896)
\curveto(828.695831,293.04308863)(828.74583095,293.04308863)(828.79583496,293.0330896)
\curveto(828.82583087,293.02308865)(828.88583081,293.01308866)(828.97583496,293.0030896)
\curveto(829.07583062,292.99308868)(829.15083054,292.99808867)(829.20083496,293.0180896)
\curveto(829.24083045,293.02808864)(829.28083041,293.02808864)(829.32083496,293.0180896)
\curveto(829.36083033,293.01808865)(829.40083029,293.02808864)(829.44083496,293.0480896)
\curveto(829.52083017,293.0680886)(829.60083009,293.08308859)(829.68083496,293.0930896)
\curveto(829.76082993,293.11308856)(829.83582986,293.13808853)(829.90583496,293.1680896)
\curveto(830.24582945,293.30808836)(830.52082917,293.50308817)(830.73083496,293.7530896)
\curveto(830.94082875,294.00308767)(831.11582858,294.29808737)(831.25583496,294.6380896)
\curveto(831.30582839,294.75808691)(831.33582836,294.88308679)(831.34583496,295.0130896)
\curveto(831.36582833,295.15308652)(831.3958283,295.29308638)(831.43583496,295.4330896)
}
}
{
\newrgbcolor{curcolor}{0 0 0}
\pscustom[linestyle=none,fillstyle=solid,fillcolor=curcolor]
{
\newpath
\moveto(837.90411621,299.9930896)
\curveto(838.13411142,299.99308168)(838.26411129,299.93308174)(838.29411621,299.8130896)
\curveto(838.32411123,299.70308197)(838.33911122,299.53808213)(838.33911621,299.3180896)
\lineto(838.33911621,299.0330896)
\curveto(838.33911122,298.94308273)(838.31411124,298.8680828)(838.26411621,298.8080896)
\curveto(838.20411135,298.72808294)(838.11911144,298.68308299)(838.00911621,298.6730896)
\curveto(837.89911166,298.673083)(837.78911177,298.65808301)(837.67911621,298.6280896)
\curveto(837.53911202,298.59808307)(837.40411215,298.5680831)(837.27411621,298.5380896)
\curveto(837.1541124,298.50808316)(837.03911252,298.4680832)(836.92911621,298.4180896)
\curveto(836.63911292,298.28808338)(836.40411315,298.10808356)(836.22411621,297.8780896)
\curveto(836.04411351,297.65808401)(835.88911367,297.40308427)(835.75911621,297.1130896)
\curveto(835.71911384,297.00308467)(835.68911387,296.88808478)(835.66911621,296.7680896)
\curveto(835.64911391,296.65808501)(835.62411393,296.54308513)(835.59411621,296.4230896)
\curveto(835.58411397,296.3730853)(835.57911398,296.32308535)(835.57911621,296.2730896)
\curveto(835.58911397,296.22308545)(835.58911397,296.1730855)(835.57911621,296.1230896)
\curveto(835.54911401,296.00308567)(835.53411402,295.86308581)(835.53411621,295.7030896)
\curveto(835.54411401,295.55308612)(835.54911401,295.40808626)(835.54911621,295.2680896)
\lineto(835.54911621,293.4230896)
\lineto(835.54911621,293.0780896)
\curveto(835.54911401,292.95808871)(835.54411401,292.84308883)(835.53411621,292.7330896)
\curveto(835.52411403,292.62308905)(835.51911404,292.52808914)(835.51911621,292.4480896)
\curveto(835.52911403,292.3680893)(835.50911405,292.29808937)(835.45911621,292.2380896)
\curveto(835.40911415,292.1680895)(835.32911423,292.12808954)(835.21911621,292.1180896)
\curveto(835.11911444,292.10808956)(835.00911455,292.10308957)(834.88911621,292.1030896)
\lineto(834.61911621,292.1030896)
\curveto(834.56911499,292.12308955)(834.51911504,292.13808953)(834.46911621,292.1480896)
\curveto(834.42911513,292.1680895)(834.39911516,292.19308948)(834.37911621,292.2230896)
\curveto(834.32911523,292.29308938)(834.29911526,292.37808929)(834.28911621,292.4780896)
\lineto(834.28911621,292.8080896)
\lineto(834.28911621,293.9630896)
\lineto(834.28911621,298.1180896)
\lineto(834.28911621,299.1530896)
\lineto(834.28911621,299.4530896)
\curveto(834.29911526,299.55308212)(834.32911523,299.63808203)(834.37911621,299.7080896)
\curveto(834.40911515,299.74808192)(834.4591151,299.77808189)(834.52911621,299.7980896)
\curveto(834.60911495,299.81808185)(834.69411486,299.82808184)(834.78411621,299.8280896)
\curveto(834.87411468,299.83808183)(834.96411459,299.83808183)(835.05411621,299.8280896)
\curveto(835.14411441,299.81808185)(835.21411434,299.80308187)(835.26411621,299.7830896)
\curveto(835.34411421,299.75308192)(835.39411416,299.69308198)(835.41411621,299.6030896)
\curveto(835.44411411,299.52308215)(835.4591141,299.43308224)(835.45911621,299.3330896)
\lineto(835.45911621,299.0330896)
\curveto(835.4591141,298.93308274)(835.47911408,298.84308283)(835.51911621,298.7630896)
\curveto(835.52911403,298.74308293)(835.53911402,298.72808294)(835.54911621,298.7180896)
\lineto(835.59411621,298.6730896)
\curveto(835.70411385,298.673083)(835.79411376,298.71808295)(835.86411621,298.8080896)
\curveto(835.93411362,298.90808276)(835.99411356,298.98808268)(836.04411621,299.0480896)
\lineto(836.13411621,299.1380896)
\curveto(836.22411333,299.24808242)(836.34911321,299.36308231)(836.50911621,299.4830896)
\curveto(836.66911289,299.60308207)(836.81911274,299.69308198)(836.95911621,299.7530896)
\curveto(837.04911251,299.80308187)(837.14411241,299.83808183)(837.24411621,299.8580896)
\curveto(837.34411221,299.88808178)(837.44911211,299.91808175)(837.55911621,299.9480896)
\curveto(837.61911194,299.95808171)(837.67911188,299.96308171)(837.73911621,299.9630896)
\curveto(837.79911176,299.9730817)(837.8541117,299.98308169)(837.90411621,299.9930896)
}
}
{
\newrgbcolor{curcolor}{0 0 0}
\pscustom[linestyle=none,fillstyle=solid,fillcolor=curcolor]
{
\newpath
\moveto(846.01888184,296.2580896)
\curveto(846.03887415,296.15808551)(846.03887415,296.04308563)(846.01888184,295.9130896)
\curveto(846.00887418,295.79308588)(845.97887421,295.70808596)(845.92888184,295.6580896)
\curveto(845.87887431,295.61808605)(845.80387439,295.58808608)(845.70388184,295.5680896)
\curveto(845.61387458,295.55808611)(845.50887468,295.55308612)(845.38888184,295.5530896)
\lineto(845.02888184,295.5530896)
\curveto(844.90887528,295.56308611)(844.80387539,295.5680861)(844.71388184,295.5680896)
\lineto(840.87388184,295.5680896)
\curveto(840.7938794,295.5680861)(840.71387948,295.56308611)(840.63388184,295.5530896)
\curveto(840.55387964,295.55308612)(840.4888797,295.53808613)(840.43888184,295.5080896)
\curveto(840.39887979,295.48808618)(840.35887983,295.44808622)(840.31888184,295.3880896)
\curveto(840.29887989,295.35808631)(840.27887991,295.31308636)(840.25888184,295.2530896)
\curveto(840.23887995,295.20308647)(840.23887995,295.15308652)(840.25888184,295.1030896)
\curveto(840.26887992,295.05308662)(840.27387992,295.00808666)(840.27388184,294.9680896)
\curveto(840.27387992,294.92808674)(840.27887991,294.88808678)(840.28888184,294.8480896)
\curveto(840.30887988,294.7680869)(840.32887986,294.68308699)(840.34888184,294.5930896)
\curveto(840.36887982,294.51308716)(840.39887979,294.43308724)(840.43888184,294.3530896)
\curveto(840.66887952,293.81308786)(841.04887914,293.42808824)(841.57888184,293.1980896)
\curveto(841.63887855,293.1680885)(841.70387849,293.14308853)(841.77388184,293.1230896)
\lineto(841.98388184,293.0630896)
\curveto(842.01387818,293.05308862)(842.06387813,293.04808862)(842.13388184,293.0480896)
\curveto(842.27387792,293.00808866)(842.45887773,292.98808868)(842.68888184,292.9880896)
\curveto(842.91887727,292.98808868)(843.10387709,293.00808866)(843.24388184,293.0480896)
\curveto(843.38387681,293.08808858)(843.50887668,293.12808854)(843.61888184,293.1680896)
\curveto(843.73887645,293.21808845)(843.84887634,293.27808839)(843.94888184,293.3480896)
\curveto(844.05887613,293.41808825)(844.15387604,293.49808817)(844.23388184,293.5880896)
\curveto(844.31387588,293.68808798)(844.38387581,293.79308788)(844.44388184,293.9030896)
\curveto(844.50387569,294.00308767)(844.55387564,294.10808756)(844.59388184,294.2180896)
\curveto(844.64387555,294.32808734)(844.72387547,294.40808726)(844.83388184,294.4580896)
\curveto(844.87387532,294.47808719)(844.93887525,294.49308718)(845.02888184,294.5030896)
\curveto(845.11887507,294.51308716)(845.20887498,294.51308716)(845.29888184,294.5030896)
\curveto(845.3888748,294.50308717)(845.47387472,294.49808717)(845.55388184,294.4880896)
\curveto(845.63387456,294.47808719)(845.6888745,294.45808721)(845.71888184,294.4280896)
\curveto(845.81887437,294.35808731)(845.84387435,294.24308743)(845.79388184,294.0830896)
\curveto(845.71387448,293.81308786)(845.60887458,293.5730881)(845.47888184,293.3630896)
\curveto(845.27887491,293.04308863)(845.04887514,292.77808889)(844.78888184,292.5680896)
\curveto(844.53887565,292.3680893)(844.21887597,292.20308947)(843.82888184,292.0730896)
\curveto(843.72887646,292.03308964)(843.62887656,292.00808966)(843.52888184,291.9980896)
\curveto(843.42887676,291.97808969)(843.32387687,291.95808971)(843.21388184,291.9380896)
\curveto(843.16387703,291.92808974)(843.11387708,291.92308975)(843.06388184,291.9230896)
\curveto(843.02387717,291.92308975)(842.97887721,291.91808975)(842.92888184,291.9080896)
\lineto(842.77888184,291.9080896)
\curveto(842.72887746,291.89808977)(842.66887752,291.89308978)(842.59888184,291.8930896)
\curveto(842.53887765,291.89308978)(842.4888777,291.89808977)(842.44888184,291.9080896)
\lineto(842.31388184,291.9080896)
\curveto(842.26387793,291.91808975)(842.21887797,291.92308975)(842.17888184,291.9230896)
\curveto(842.13887805,291.92308975)(842.09887809,291.92808974)(842.05888184,291.9380896)
\curveto(842.00887818,291.94808972)(841.95387824,291.95808971)(841.89388184,291.9680896)
\curveto(841.83387836,291.9680897)(841.77887841,291.9730897)(841.72888184,291.9830896)
\curveto(841.63887855,292.00308967)(841.54887864,292.02808964)(841.45888184,292.0580896)
\curveto(841.36887882,292.07808959)(841.28387891,292.10308957)(841.20388184,292.1330896)
\curveto(841.16387903,292.15308952)(841.12887906,292.16308951)(841.09888184,292.1630896)
\curveto(841.06887912,292.1730895)(841.03387916,292.18808948)(840.99388184,292.2080896)
\curveto(840.84387935,292.27808939)(840.68387951,292.36308931)(840.51388184,292.4630896)
\curveto(840.22387997,292.65308902)(839.97388022,292.88308879)(839.76388184,293.1530896)
\curveto(839.56388063,293.43308824)(839.3938808,293.74308793)(839.25388184,294.0830896)
\curveto(839.20388099,294.19308748)(839.16388103,294.30808736)(839.13388184,294.4280896)
\curveto(839.11388108,294.54808712)(839.08388111,294.668087)(839.04388184,294.7880896)
\curveto(839.03388116,294.82808684)(839.02888116,294.86308681)(839.02888184,294.8930896)
\curveto(839.02888116,294.92308675)(839.02388117,294.96308671)(839.01388184,295.0130896)
\curveto(838.9938812,295.09308658)(838.97888121,295.17808649)(838.96888184,295.2680896)
\curveto(838.95888123,295.35808631)(838.94388125,295.44808622)(838.92388184,295.5380896)
\lineto(838.92388184,295.7480896)
\curveto(838.91388128,295.78808588)(838.90388129,295.84308583)(838.89388184,295.9130896)
\curveto(838.8938813,295.99308568)(838.89888129,296.05808561)(838.90888184,296.1080896)
\lineto(838.90888184,296.2730896)
\curveto(838.92888126,296.32308535)(838.93388126,296.3730853)(838.92388184,296.4230896)
\curveto(838.92388127,296.48308519)(838.92888126,296.53808513)(838.93888184,296.5880896)
\curveto(838.97888121,296.74808492)(839.00888118,296.90808476)(839.02888184,297.0680896)
\curveto(839.05888113,297.22808444)(839.10388109,297.37808429)(839.16388184,297.5180896)
\curveto(839.21388098,297.62808404)(839.25888093,297.73808393)(839.29888184,297.8480896)
\curveto(839.34888084,297.9680837)(839.40388079,298.08308359)(839.46388184,298.1930896)
\curveto(839.68388051,298.54308313)(839.93388026,298.84308283)(840.21388184,299.0930896)
\curveto(840.4938797,299.35308232)(840.83887935,299.5680821)(841.24888184,299.7380896)
\curveto(841.36887882,299.78808188)(841.4888787,299.82308185)(841.60888184,299.8430896)
\curveto(841.73887845,299.8730818)(841.87387832,299.90308177)(842.01388184,299.9330896)
\curveto(842.06387813,299.94308173)(842.10887808,299.94808172)(842.14888184,299.9480896)
\curveto(842.188878,299.95808171)(842.23387796,299.96308171)(842.28388184,299.9630896)
\curveto(842.30387789,299.9730817)(842.32887786,299.9730817)(842.35888184,299.9630896)
\curveto(842.3888778,299.95308172)(842.41387778,299.95808171)(842.43388184,299.9780896)
\curveto(842.85387734,299.98808168)(843.21887697,299.94308173)(843.52888184,299.8430896)
\curveto(843.83887635,299.75308192)(844.11887607,299.62808204)(844.36888184,299.4680896)
\curveto(844.41887577,299.44808222)(844.45887573,299.41808225)(844.48888184,299.3780896)
\curveto(844.51887567,299.34808232)(844.55387564,299.32308235)(844.59388184,299.3030896)
\curveto(844.67387552,299.24308243)(844.75387544,299.1730825)(844.83388184,299.0930896)
\curveto(844.92387527,299.01308266)(844.99887519,298.93308274)(845.05888184,298.8530896)
\curveto(845.21887497,298.64308303)(845.35387484,298.44308323)(845.46388184,298.2530896)
\curveto(845.53387466,298.14308353)(845.5888746,298.02308365)(845.62888184,297.8930896)
\curveto(845.66887452,297.76308391)(845.71387448,297.63308404)(845.76388184,297.5030896)
\curveto(845.81387438,297.3730843)(845.84887434,297.23808443)(845.86888184,297.0980896)
\curveto(845.89887429,296.95808471)(845.93387426,296.81808485)(845.97388184,296.6780896)
\curveto(845.98387421,296.60808506)(845.9888742,296.53808513)(845.98888184,296.4680896)
\lineto(846.01888184,296.2580896)
\moveto(844.56388184,296.7680896)
\curveto(844.5938756,296.80808486)(844.61887557,296.85808481)(844.63888184,296.9180896)
\curveto(844.65887553,296.98808468)(844.65887553,297.05808461)(844.63888184,297.1280896)
\curveto(844.57887561,297.34808432)(844.4938757,297.55308412)(844.38388184,297.7430896)
\curveto(844.24387595,297.9730837)(844.0888761,298.1680835)(843.91888184,298.3280896)
\curveto(843.74887644,298.48808318)(843.52887666,298.62308305)(843.25888184,298.7330896)
\curveto(843.188877,298.75308292)(843.11887707,298.7680829)(843.04888184,298.7780896)
\curveto(842.97887721,298.79808287)(842.90387729,298.81808285)(842.82388184,298.8380896)
\curveto(842.74387745,298.85808281)(842.65887753,298.8680828)(842.56888184,298.8680896)
\lineto(842.31388184,298.8680896)
\curveto(842.28387791,298.84808282)(842.24887794,298.83808283)(842.20888184,298.8380896)
\curveto(842.16887802,298.84808282)(842.13387806,298.84808282)(842.10388184,298.8380896)
\lineto(841.86388184,298.7780896)
\curveto(841.7938784,298.7680829)(841.72387847,298.75308292)(841.65388184,298.7330896)
\curveto(841.36387883,298.61308306)(841.12887906,298.46308321)(840.94888184,298.2830896)
\curveto(840.77887941,298.10308357)(840.62387957,297.87808379)(840.48388184,297.6080896)
\curveto(840.45387974,297.55808411)(840.42387977,297.49308418)(840.39388184,297.4130896)
\curveto(840.36387983,297.34308433)(840.33887985,297.26308441)(840.31888184,297.1730896)
\curveto(840.29887989,297.08308459)(840.2938799,296.99808467)(840.30388184,296.9180896)
\curveto(840.31387988,296.83808483)(840.34887984,296.77808489)(840.40888184,296.7380896)
\curveto(840.4888797,296.67808499)(840.62387957,296.64808502)(840.81388184,296.6480896)
\curveto(841.01387918,296.65808501)(841.18387901,296.66308501)(841.32388184,296.6630896)
\lineto(843.60388184,296.6630896)
\curveto(843.75387644,296.66308501)(843.93387626,296.65808501)(844.14388184,296.6480896)
\curveto(844.35387584,296.64808502)(844.4938757,296.68808498)(844.56388184,296.7680896)
}
}
{
\newrgbcolor{curcolor}{0 0 0}
\pscustom[linestyle=none,fillstyle=solid,fillcolor=curcolor]
{
\newpath
\moveto(849.75552246,299.9930896)
\curveto(850.4755184,300.00308167)(851.08051779,299.91808175)(851.57052246,299.7380896)
\curveto(852.06051681,299.5680821)(852.44051643,299.26308241)(852.71052246,298.8230896)
\curveto(852.78051609,298.71308296)(852.83551604,298.59808307)(852.87552246,298.4780896)
\curveto(852.91551596,298.3680833)(852.95551592,298.24308343)(852.99552246,298.1030896)
\curveto(853.01551586,298.03308364)(853.02051585,297.95808371)(853.01052246,297.8780896)
\curveto(853.00051587,297.80808386)(852.98551589,297.75308392)(852.96552246,297.7130896)
\curveto(852.94551593,297.69308398)(852.92051595,297.673084)(852.89052246,297.6530896)
\curveto(852.86051601,297.64308403)(852.83551604,297.62808404)(852.81552246,297.6080896)
\curveto(852.76551611,297.58808408)(852.71551616,297.58308409)(852.66552246,297.5930896)
\curveto(852.61551626,297.60308407)(852.56551631,297.60308407)(852.51552246,297.5930896)
\curveto(852.43551644,297.5730841)(852.33051654,297.5680841)(852.20052246,297.5780896)
\curveto(852.0705168,297.59808407)(851.98051689,297.62308405)(851.93052246,297.6530896)
\curveto(851.85051702,297.70308397)(851.79551708,297.7680839)(851.76552246,297.8480896)
\curveto(851.74551713,297.93808373)(851.71051716,298.02308365)(851.66052246,298.1030896)
\curveto(851.5705173,298.26308341)(851.44551743,298.40808326)(851.28552246,298.5380896)
\curveto(851.1755177,298.61808305)(851.05551782,298.67808299)(850.92552246,298.7180896)
\curveto(850.79551808,298.75808291)(850.65551822,298.79808287)(850.50552246,298.8380896)
\curveto(850.45551842,298.85808281)(850.40551847,298.86308281)(850.35552246,298.8530896)
\curveto(850.30551857,298.85308282)(850.25551862,298.85808281)(850.20552246,298.8680896)
\curveto(850.14551873,298.88808278)(850.0705188,298.89808277)(849.98052246,298.8980896)
\curveto(849.89051898,298.89808277)(849.81551906,298.88808278)(849.75552246,298.8680896)
\lineto(849.66552246,298.8680896)
\lineto(849.51552246,298.8380896)
\curveto(849.46551941,298.83808283)(849.41551946,298.83308284)(849.36552246,298.8230896)
\curveto(849.10551977,298.76308291)(848.89051998,298.67808299)(848.72052246,298.5680896)
\curveto(848.55052032,298.45808321)(848.43552044,298.2730834)(848.37552246,298.0130896)
\curveto(848.35552052,297.94308373)(848.35052052,297.8730838)(848.36052246,297.8030896)
\curveto(848.38052049,297.73308394)(848.40052047,297.673084)(848.42052246,297.6230896)
\curveto(848.48052039,297.4730842)(848.55052032,297.36308431)(848.63052246,297.2930896)
\curveto(848.72052015,297.23308444)(848.83052004,297.16308451)(848.96052246,297.0830896)
\curveto(849.12051975,296.98308469)(849.30051957,296.90808476)(849.50052246,296.8580896)
\curveto(849.70051917,296.81808485)(849.90051897,296.7680849)(850.10052246,296.7080896)
\curveto(850.23051864,296.668085)(850.36051851,296.63808503)(850.49052246,296.6180896)
\curveto(850.62051825,296.59808507)(850.75051812,296.5680851)(850.88052246,296.5280896)
\curveto(851.09051778,296.4680852)(851.29551758,296.40808526)(851.49552246,296.3480896)
\curveto(851.69551718,296.29808537)(851.89551698,296.23308544)(852.09552246,296.1530896)
\lineto(852.24552246,296.0930896)
\curveto(852.29551658,296.0730856)(852.34551653,296.04808562)(852.39552246,296.0180896)
\curveto(852.59551628,295.89808577)(852.7705161,295.76308591)(852.92052246,295.6130896)
\curveto(853.0705158,295.46308621)(853.19551568,295.2730864)(853.29552246,295.0430896)
\curveto(853.31551556,294.9730867)(853.33551554,294.87808679)(853.35552246,294.7580896)
\curveto(853.3755155,294.68808698)(853.38551549,294.61308706)(853.38552246,294.5330896)
\curveto(853.39551548,294.46308721)(853.40051547,294.38308729)(853.40052246,294.2930896)
\lineto(853.40052246,294.1430896)
\curveto(853.38051549,294.0730876)(853.3705155,294.00308767)(853.37052246,293.9330896)
\curveto(853.3705155,293.86308781)(853.36051551,293.79308788)(853.34052246,293.7230896)
\curveto(853.31051556,293.61308806)(853.2755156,293.50808816)(853.23552246,293.4080896)
\curveto(853.19551568,293.30808836)(853.15051572,293.21808845)(853.10052246,293.1380896)
\curveto(852.94051593,292.87808879)(852.73551614,292.668089)(852.48552246,292.5080896)
\curveto(852.23551664,292.35808931)(851.95551692,292.22808944)(851.64552246,292.1180896)
\curveto(851.55551732,292.08808958)(851.46051741,292.0680896)(851.36052246,292.0580896)
\curveto(851.2705176,292.03808963)(851.18051769,292.01308966)(851.09052246,291.9830896)
\curveto(850.99051788,291.96308971)(850.89051798,291.95308972)(850.79052246,291.9530896)
\curveto(850.69051818,291.95308972)(850.59051828,291.94308973)(850.49052246,291.9230896)
\lineto(850.34052246,291.9230896)
\curveto(850.29051858,291.91308976)(850.22051865,291.90808976)(850.13052246,291.9080896)
\curveto(850.04051883,291.90808976)(849.9705189,291.91308976)(849.92052246,291.9230896)
\lineto(849.75552246,291.9230896)
\curveto(849.69551918,291.94308973)(849.63051924,291.95308972)(849.56052246,291.9530896)
\curveto(849.49051938,291.94308973)(849.43051944,291.94808972)(849.38052246,291.9680896)
\curveto(849.33051954,291.97808969)(849.26551961,291.98308969)(849.18552246,291.9830896)
\lineto(848.94552246,292.0430896)
\curveto(848.87552,292.05308962)(848.80052007,292.0730896)(848.72052246,292.1030896)
\curveto(848.41052046,292.20308947)(848.14052073,292.32808934)(847.91052246,292.4780896)
\curveto(847.68052119,292.62808904)(847.48052139,292.82308885)(847.31052246,293.0630896)
\curveto(847.22052165,293.19308848)(847.14552173,293.32808834)(847.08552246,293.4680896)
\curveto(847.02552185,293.60808806)(846.9705219,293.76308791)(846.92052246,293.9330896)
\curveto(846.90052197,293.99308768)(846.89052198,294.06308761)(846.89052246,294.1430896)
\curveto(846.90052197,294.23308744)(846.91552196,294.30308737)(846.93552246,294.3530896)
\curveto(846.96552191,294.39308728)(847.01552186,294.43308724)(847.08552246,294.4730896)
\curveto(847.13552174,294.49308718)(847.20552167,294.50308717)(847.29552246,294.5030896)
\curveto(847.38552149,294.51308716)(847.4755214,294.51308716)(847.56552246,294.5030896)
\curveto(847.65552122,294.49308718)(847.74052113,294.47808719)(847.82052246,294.4580896)
\curveto(847.91052096,294.44808722)(847.9705209,294.43308724)(848.00052246,294.4130896)
\curveto(848.0705208,294.36308731)(848.11552076,294.28808738)(848.13552246,294.1880896)
\curveto(848.16552071,294.09808757)(848.20052067,294.01308766)(848.24052246,293.9330896)
\curveto(848.34052053,293.71308796)(848.4755204,293.54308813)(848.64552246,293.4230896)
\curveto(848.76552011,293.33308834)(848.90051997,293.26308841)(849.05052246,293.2130896)
\curveto(849.20051967,293.16308851)(849.36051951,293.11308856)(849.53052246,293.0630896)
\lineto(849.84552246,293.0180896)
\lineto(849.93552246,293.0180896)
\curveto(850.00551887,292.99808867)(850.09551878,292.98808868)(850.20552246,292.9880896)
\curveto(850.32551855,292.98808868)(850.42551845,292.99808867)(850.50552246,293.0180896)
\curveto(850.5755183,293.01808865)(850.63051824,293.02308865)(850.67052246,293.0330896)
\curveto(850.73051814,293.04308863)(850.79051808,293.04808862)(850.85052246,293.0480896)
\curveto(850.91051796,293.05808861)(850.96551791,293.0680886)(851.01552246,293.0780896)
\curveto(851.30551757,293.15808851)(851.53551734,293.26308841)(851.70552246,293.3930896)
\curveto(851.875517,293.52308815)(851.99551688,293.74308793)(852.06552246,294.0530896)
\curveto(852.08551679,294.10308757)(852.09051678,294.15808751)(852.08052246,294.2180896)
\curveto(852.0705168,294.27808739)(852.06051681,294.32308735)(852.05052246,294.3530896)
\curveto(852.00051687,294.54308713)(851.93051694,294.68308699)(851.84052246,294.7730896)
\curveto(851.75051712,294.8730868)(851.63551724,294.96308671)(851.49552246,295.0430896)
\curveto(851.40551747,295.10308657)(851.30551757,295.15308652)(851.19552246,295.1930896)
\lineto(850.86552246,295.3130896)
\curveto(850.83551804,295.32308635)(850.80551807,295.32808634)(850.77552246,295.3280896)
\curveto(850.75551812,295.32808634)(850.73051814,295.33808633)(850.70052246,295.3580896)
\curveto(850.36051851,295.4680862)(850.00551887,295.54808612)(849.63552246,295.5980896)
\curveto(849.2755196,295.65808601)(848.93551994,295.75308592)(848.61552246,295.8830896)
\curveto(848.51552036,295.92308575)(848.42052045,295.95808571)(848.33052246,295.9880896)
\curveto(848.24052063,296.01808565)(848.15552072,296.05808561)(848.07552246,296.1080896)
\curveto(847.88552099,296.21808545)(847.71052116,296.34308533)(847.55052246,296.4830896)
\curveto(847.39052148,296.62308505)(847.26552161,296.79808487)(847.17552246,297.0080896)
\curveto(847.14552173,297.07808459)(847.12052175,297.14808452)(847.10052246,297.2180896)
\curveto(847.09052178,297.28808438)(847.0755218,297.36308431)(847.05552246,297.4430896)
\curveto(847.02552185,297.56308411)(847.01552186,297.69808397)(847.02552246,297.8480896)
\curveto(847.03552184,298.00808366)(847.05052182,298.14308353)(847.07052246,298.2530896)
\curveto(847.09052178,298.30308337)(847.10052177,298.34308333)(847.10052246,298.3730896)
\curveto(847.11052176,298.41308326)(847.12552175,298.45308322)(847.14552246,298.4930896)
\curveto(847.23552164,298.72308295)(847.35552152,298.92308275)(847.50552246,299.0930896)
\curveto(847.66552121,299.26308241)(847.84552103,299.41308226)(848.04552246,299.5430896)
\curveto(848.19552068,299.63308204)(848.36052051,299.70308197)(848.54052246,299.7530896)
\curveto(848.72052015,299.81308186)(848.91051996,299.8680818)(849.11052246,299.9180896)
\curveto(849.18051969,299.92808174)(849.24551963,299.93808173)(849.30552246,299.9480896)
\curveto(849.3755195,299.95808171)(849.45051942,299.9680817)(849.53052246,299.9780896)
\curveto(849.56051931,299.98808168)(849.60051927,299.98808168)(849.65052246,299.9780896)
\curveto(849.70051917,299.9680817)(849.73551914,299.9730817)(849.75552246,299.9930896)
}
}
{
\newrgbcolor{curcolor}{0.80000001 0.80000001 0.80000001}
\pscustom[linestyle=none,fillstyle=solid,fillcolor=curcolor]
{
\newpath
\moveto(747.9732666,372.02397156)
\lineto(762.9732666,372.02397156)
\lineto(762.9732666,357.02397156)
\lineto(747.9732666,357.02397156)
\closepath
}
}
{
\newrgbcolor{curcolor}{0.7019608 0.7019608 0.7019608}
\pscustom[linestyle=none,fillstyle=solid,fillcolor=curcolor]
{
\newpath
\moveto(747.9732666,348.69264984)
\lineto(762.9732666,348.69264984)
\lineto(762.9732666,333.69264984)
\lineto(747.9732666,333.69264984)
\closepath
}
}
{
\newrgbcolor{curcolor}{0.60000002 0.60000002 0.60000002}
\pscustom[linestyle=none,fillstyle=solid,fillcolor=curcolor]
{
\newpath
\moveto(747.9732666,325.37634277)
\lineto(762.9732666,325.37634277)
\lineto(762.9732666,310.37634277)
\lineto(747.9732666,310.37634277)
\closepath
}
}
{
\newrgbcolor{curcolor}{0.50196081 0.50196081 0.50196081}
\pscustom[linestyle=none,fillstyle=solid,fillcolor=curcolor]
{
\newpath
\moveto(747.9732666,302.78309631)
\lineto(762.9732666,302.78309631)
\lineto(762.9732666,287.78309631)
\lineto(747.9732666,287.78309631)
\closepath
}
}
{
\newrgbcolor{curcolor}{0.80000001 0.80000001 0.80000001}
\pscustom[linestyle=none,fillstyle=solid,fillcolor=curcolor]
{
\newpath
\moveto(541.98950195,77.19110107)
\lineto(555.03574848,77.19110107)
\lineto(555.03574848,76.08277607)
\lineto(541.98950195,76.08277607)
\closepath
}
}
{
\newrgbcolor{curcolor}{0 0 0}
\pscustom[linestyle=none,fillstyle=solid,fillcolor=curcolor]
{
\newpath
\moveto(767.94645996,280.0180896)
\lineto(772.85145996,280.0180896)
\lineto(774.14145996,280.0180896)
\curveto(774.25145208,280.0180789)(774.36145197,280.0180789)(774.47145996,280.0180896)
\curveto(774.58145175,280.02807889)(774.67145166,280.00807891)(774.74145996,279.9580896)
\curveto(774.77145156,279.93807898)(774.79645154,279.91307901)(774.81645996,279.8830896)
\curveto(774.8364515,279.85307907)(774.85645148,279.8230791)(774.87645996,279.7930896)
\curveto(774.89645144,279.7230792)(774.90645143,279.60807931)(774.90645996,279.4480896)
\curveto(774.90645143,279.29807962)(774.89645144,279.18307974)(774.87645996,279.1030896)
\curveto(774.8364515,278.96307996)(774.75145158,278.88308004)(774.62145996,278.8630896)
\curveto(774.49145184,278.85308007)(774.336452,278.84808007)(774.15645996,278.8480896)
\lineto(772.65645996,278.8480896)
\lineto(770.13645996,278.8480896)
\lineto(769.56645996,278.8480896)
\curveto(769.35645698,278.85808006)(769.20145713,278.83308009)(769.10145996,278.7730896)
\curveto(769.00145733,278.71308021)(768.94645739,278.60808031)(768.93645996,278.4580896)
\lineto(768.93645996,277.9930896)
\lineto(768.93645996,276.4630896)
\curveto(768.9364574,276.35308257)(768.9314574,276.2230827)(768.92145996,276.0730896)
\curveto(768.92145741,275.923083)(768.9314574,275.80308312)(768.95145996,275.7130896)
\curveto(768.98145735,275.59308333)(769.04145729,275.51308341)(769.13145996,275.4730896)
\curveto(769.17145716,275.45308347)(769.24145709,275.43308349)(769.34145996,275.4130896)
\lineto(769.49145996,275.4130896)
\curveto(769.5314568,275.40308352)(769.57145676,275.39808352)(769.61145996,275.3980896)
\curveto(769.66145667,275.40808351)(769.71145662,275.41308351)(769.76145996,275.4130896)
\lineto(770.27145996,275.4130896)
\lineto(773.21145996,275.4130896)
\lineto(773.51145996,275.4130896)
\curveto(773.62145271,275.4230835)(773.7314526,275.4230835)(773.84145996,275.4130896)
\curveto(773.96145237,275.41308351)(774.06645227,275.40308352)(774.15645996,275.3830896)
\curveto(774.25645208,275.37308355)(774.331452,275.35308357)(774.38145996,275.3230896)
\curveto(774.41145192,275.30308362)(774.4364519,275.25808366)(774.45645996,275.1880896)
\curveto(774.47645186,275.1180838)(774.49145184,275.04308388)(774.50145996,274.9630896)
\curveto(774.51145182,274.88308404)(774.51145182,274.79808412)(774.50145996,274.7080896)
\curveto(774.50145183,274.62808429)(774.49145184,274.55808436)(774.47145996,274.4980896)
\curveto(774.45145188,274.40808451)(774.40645193,274.34308458)(774.33645996,274.3030896)
\curveto(774.31645202,274.28308464)(774.28645205,274.26808465)(774.24645996,274.2580896)
\curveto(774.21645212,274.25808466)(774.18645215,274.25308467)(774.15645996,274.2430896)
\lineto(774.06645996,274.2430896)
\curveto(774.01645232,274.23308469)(773.96645237,274.22808469)(773.91645996,274.2280896)
\curveto(773.86645247,274.23808468)(773.81645252,274.24308468)(773.76645996,274.2430896)
\lineto(773.21145996,274.2430896)
\lineto(770.04645996,274.2430896)
\lineto(769.68645996,274.2430896)
\curveto(769.57645676,274.25308467)(769.47145686,274.24808467)(769.37145996,274.2280896)
\curveto(769.27145706,274.2180847)(769.18145715,274.19308473)(769.10145996,274.1530896)
\curveto(769.0314573,274.11308481)(768.98145735,274.04308488)(768.95145996,273.9430896)
\curveto(768.9314574,273.88308504)(768.92145741,273.81308511)(768.92145996,273.7330896)
\curveto(768.9314574,273.65308527)(768.9364574,273.57308535)(768.93645996,273.4930896)
\lineto(768.93645996,272.6530896)
\lineto(768.93645996,271.2280896)
\curveto(768.9364574,271.08808783)(768.94145739,270.95808796)(768.95145996,270.8380896)
\curveto(768.96145737,270.72808819)(769.00145733,270.64808827)(769.07145996,270.5980896)
\curveto(769.14145719,270.54808837)(769.22145711,270.5180884)(769.31145996,270.5080896)
\lineto(769.61145996,270.5080896)
\lineto(770.57145996,270.5080896)
\lineto(773.34645996,270.5080896)
\lineto(774.20145996,270.5080896)
\lineto(774.44145996,270.5080896)
\curveto(774.52145181,270.5180884)(774.59145174,270.51308841)(774.65145996,270.4930896)
\curveto(774.77145156,270.45308847)(774.85145148,270.39808852)(774.89145996,270.3280896)
\curveto(774.91145142,270.29808862)(774.92645141,270.24808867)(774.93645996,270.1780896)
\curveto(774.94645139,270.10808881)(774.95145138,270.03308889)(774.95145996,269.9530896)
\curveto(774.96145137,269.88308904)(774.96145137,269.80808911)(774.95145996,269.7280896)
\curveto(774.94145139,269.65808926)(774.9314514,269.60308932)(774.92145996,269.5630896)
\curveto(774.88145145,269.48308944)(774.8364515,269.42808949)(774.78645996,269.3980896)
\curveto(774.72645161,269.35808956)(774.64645169,269.33808958)(774.54645996,269.3380896)
\lineto(774.27645996,269.3380896)
\lineto(773.22645996,269.3380896)
\lineto(769.23645996,269.3380896)
\lineto(768.18645996,269.3380896)
\curveto(768.04645829,269.33808958)(767.92645841,269.34308958)(767.82645996,269.3530896)
\curveto(767.72645861,269.37308955)(767.65145868,269.4230895)(767.60145996,269.5030896)
\curveto(767.56145877,269.56308936)(767.54145879,269.63808928)(767.54145996,269.7280896)
\lineto(767.54145996,270.0130896)
\lineto(767.54145996,271.0630896)
\lineto(767.54145996,275.0830896)
\lineto(767.54145996,278.4430896)
\lineto(767.54145996,279.3730896)
\lineto(767.54145996,279.6430896)
\curveto(767.54145879,279.73307919)(767.56145877,279.80307912)(767.60145996,279.8530896)
\curveto(767.64145869,279.923079)(767.71645862,279.97307895)(767.82645996,280.0030896)
\curveto(767.84645849,280.01307891)(767.86645847,280.01307891)(767.88645996,280.0030896)
\curveto(767.90645843,280.00307892)(767.92645841,280.00807891)(767.94645996,280.0180896)
}
}
{
\newrgbcolor{curcolor}{0 0 0}
\pscustom[linestyle=none,fillstyle=solid,fillcolor=curcolor]
{
\newpath
\moveto(777.40138184,279.4030896)
\curveto(777.55137983,279.40307952)(777.70137968,279.39807952)(777.85138184,279.3880896)
\curveto(778.00137938,279.38807953)(778.10637927,279.34807957)(778.16638184,279.2680896)
\curveto(778.21637916,279.20807971)(778.24137914,279.1230798)(778.24138184,279.0130896)
\curveto(778.25137913,278.91308001)(778.25637912,278.80808011)(778.25638184,278.6980896)
\lineto(778.25638184,277.8280896)
\curveto(778.25637912,277.74808117)(778.25137913,277.66308126)(778.24138184,277.5730896)
\curveto(778.24137914,277.49308143)(778.25137913,277.4230815)(778.27138184,277.3630896)
\curveto(778.31137907,277.2230817)(778.40137898,277.13308179)(778.54138184,277.0930896)
\curveto(778.59137879,277.08308184)(778.63637874,277.07808184)(778.67638184,277.0780896)
\lineto(778.82638184,277.0780896)
\lineto(779.23138184,277.0780896)
\curveto(779.39137799,277.08808183)(779.50637787,277.07808184)(779.57638184,277.0480896)
\curveto(779.66637771,276.98808193)(779.72637765,276.92808199)(779.75638184,276.8680896)
\curveto(779.7763776,276.82808209)(779.78637759,276.78308214)(779.78638184,276.7330896)
\lineto(779.78638184,276.5830896)
\curveto(779.78637759,276.47308245)(779.7813776,276.36808255)(779.77138184,276.2680896)
\curveto(779.76137762,276.17808274)(779.72637765,276.10808281)(779.66638184,276.0580896)
\curveto(779.60637777,276.00808291)(779.52137786,275.97808294)(779.41138184,275.9680896)
\lineto(779.08138184,275.9680896)
\curveto(778.97137841,275.97808294)(778.86137852,275.98308294)(778.75138184,275.9830896)
\curveto(778.64137874,275.98308294)(778.54637883,275.96808295)(778.46638184,275.9380896)
\curveto(778.39637898,275.90808301)(778.34637903,275.85808306)(778.31638184,275.7880896)
\curveto(778.28637909,275.7180832)(778.26637911,275.63308329)(778.25638184,275.5330896)
\curveto(778.24637913,275.44308348)(778.24137914,275.34308358)(778.24138184,275.2330896)
\curveto(778.25137913,275.13308379)(778.25637912,275.03308389)(778.25638184,274.9330896)
\lineto(778.25638184,271.9630896)
\curveto(778.25637912,271.74308718)(778.25137913,271.50808741)(778.24138184,271.2580896)
\curveto(778.24137914,271.0180879)(778.28637909,270.83308809)(778.37638184,270.7030896)
\curveto(778.42637895,270.6230883)(778.49137889,270.56808835)(778.57138184,270.5380896)
\curveto(778.65137873,270.50808841)(778.74637863,270.48308844)(778.85638184,270.4630896)
\curveto(778.88637849,270.45308847)(778.91637846,270.44808847)(778.94638184,270.4480896)
\curveto(778.98637839,270.45808846)(779.02137836,270.45808846)(779.05138184,270.4480896)
\lineto(779.24638184,270.4480896)
\curveto(779.34637803,270.44808847)(779.43637794,270.43808848)(779.51638184,270.4180896)
\curveto(779.60637777,270.40808851)(779.67137771,270.37308855)(779.71138184,270.3130896)
\curveto(779.73137765,270.28308864)(779.74637763,270.22808869)(779.75638184,270.1480896)
\curveto(779.7763776,270.07808884)(779.78637759,270.00308892)(779.78638184,269.9230896)
\curveto(779.79637758,269.84308908)(779.79637758,269.76308916)(779.78638184,269.6830896)
\curveto(779.7763776,269.61308931)(779.75637762,269.55808936)(779.72638184,269.5180896)
\curveto(779.68637769,269.44808947)(779.61137777,269.39808952)(779.50138184,269.3680896)
\curveto(779.42137796,269.34808957)(779.33137805,269.33808958)(779.23138184,269.3380896)
\curveto(779.13137825,269.34808957)(779.04137834,269.35308957)(778.96138184,269.3530896)
\curveto(778.90137848,269.35308957)(778.84137854,269.34808957)(778.78138184,269.3380896)
\curveto(778.72137866,269.33808958)(778.66637871,269.34308958)(778.61638184,269.3530896)
\lineto(778.43638184,269.3530896)
\curveto(778.38637899,269.36308956)(778.33637904,269.36808955)(778.28638184,269.3680896)
\curveto(778.24637913,269.37808954)(778.20137918,269.38308954)(778.15138184,269.3830896)
\curveto(777.95137943,269.43308949)(777.7763796,269.48808943)(777.62638184,269.5480896)
\curveto(777.48637989,269.60808931)(777.36638001,269.71308921)(777.26638184,269.8630896)
\curveto(777.12638025,270.06308886)(777.04638033,270.31308861)(777.02638184,270.6130896)
\curveto(777.00638037,270.923088)(776.99638038,271.25308767)(776.99638184,271.6030896)
\lineto(776.99638184,275.5330896)
\curveto(776.96638041,275.66308326)(776.93638044,275.75808316)(776.90638184,275.8180896)
\curveto(776.88638049,275.87808304)(776.81638056,275.92808299)(776.69638184,275.9680896)
\curveto(776.65638072,275.97808294)(776.61638076,275.97808294)(776.57638184,275.9680896)
\curveto(776.53638084,275.95808296)(776.49638088,275.96308296)(776.45638184,275.9830896)
\lineto(776.21638184,275.9830896)
\curveto(776.08638129,275.98308294)(775.9763814,275.99308293)(775.88638184,276.0130896)
\curveto(775.80638157,276.04308288)(775.75138163,276.10308282)(775.72138184,276.1930896)
\curveto(775.70138168,276.23308269)(775.68638169,276.27808264)(775.67638184,276.3280896)
\lineto(775.67638184,276.4780896)
\curveto(775.6763817,276.6180823)(775.68638169,276.73308219)(775.70638184,276.8230896)
\curveto(775.72638165,276.923082)(775.78638159,276.99808192)(775.88638184,277.0480896)
\curveto(775.99638138,277.08808183)(776.13638124,277.09808182)(776.30638184,277.0780896)
\curveto(776.48638089,277.05808186)(776.63638074,277.06808185)(776.75638184,277.1080896)
\curveto(776.84638053,277.15808176)(776.91638046,277.22808169)(776.96638184,277.3180896)
\curveto(776.98638039,277.37808154)(776.99638038,277.45308147)(776.99638184,277.5430896)
\lineto(776.99638184,277.7980896)
\lineto(776.99638184,278.7280896)
\lineto(776.99638184,278.9680896)
\curveto(776.99638038,279.05807986)(777.00638037,279.13307979)(777.02638184,279.1930896)
\curveto(777.06638031,279.27307965)(777.14138024,279.33807958)(777.25138184,279.3880896)
\curveto(777.2813801,279.38807953)(777.30638007,279.38807953)(777.32638184,279.3880896)
\curveto(777.35638002,279.39807952)(777.38138,279.40307952)(777.40138184,279.4030896)
}
}
{
\newrgbcolor{curcolor}{0 0 0}
\pscustom[linestyle=none,fillstyle=solid,fillcolor=curcolor]
{
\newpath
\moveto(781.45817871,278.5630896)
\curveto(781.37817759,278.6230803)(781.33317764,278.72808019)(781.32317871,278.8780896)
\lineto(781.32317871,279.3430896)
\lineto(781.32317871,279.5980896)
\curveto(781.32317765,279.68807923)(781.33817763,279.76307916)(781.36817871,279.8230896)
\curveto(781.40817756,279.90307902)(781.48817748,279.96307896)(781.60817871,280.0030896)
\curveto(781.62817734,280.01307891)(781.64817732,280.01307891)(781.66817871,280.0030896)
\curveto(781.69817727,280.00307892)(781.72317725,280.00807891)(781.74317871,280.0180896)
\curveto(781.91317706,280.0180789)(782.0731769,280.01307891)(782.22317871,280.0030896)
\curveto(782.3731766,279.99307893)(782.4731765,279.93307899)(782.52317871,279.8230896)
\curveto(782.55317642,279.76307916)(782.5681764,279.68807923)(782.56817871,279.5980896)
\lineto(782.56817871,279.3430896)
\curveto(782.5681764,279.16307976)(782.56317641,278.99307993)(782.55317871,278.8330896)
\curveto(782.55317642,278.67308025)(782.48817648,278.56808035)(782.35817871,278.5180896)
\curveto(782.30817666,278.49808042)(782.25317672,278.48808043)(782.19317871,278.4880896)
\lineto(782.02817871,278.4880896)
\lineto(781.71317871,278.4880896)
\curveto(781.61317736,278.48808043)(781.52817744,278.51308041)(781.45817871,278.5630896)
\moveto(782.56817871,270.0580896)
\lineto(782.56817871,269.7430896)
\curveto(782.57817639,269.64308928)(782.55817641,269.56308936)(782.50817871,269.5030896)
\curveto(782.47817649,269.44308948)(782.43317654,269.40308952)(782.37317871,269.3830896)
\curveto(782.31317666,269.37308955)(782.24317673,269.35808956)(782.16317871,269.3380896)
\lineto(781.93817871,269.3380896)
\curveto(781.80817716,269.33808958)(781.69317728,269.34308958)(781.59317871,269.3530896)
\curveto(781.50317747,269.37308955)(781.43317754,269.4230895)(781.38317871,269.5030896)
\curveto(781.34317763,269.56308936)(781.32317765,269.63808928)(781.32317871,269.7280896)
\lineto(781.32317871,270.0130896)
\lineto(781.32317871,276.3580896)
\lineto(781.32317871,276.6730896)
\curveto(781.32317765,276.78308214)(781.34817762,276.86808205)(781.39817871,276.9280896)
\curveto(781.42817754,276.97808194)(781.4681775,277.00808191)(781.51817871,277.0180896)
\curveto(781.5681774,277.02808189)(781.62317735,277.04308188)(781.68317871,277.0630896)
\curveto(781.70317727,277.06308186)(781.72317725,277.05808186)(781.74317871,277.0480896)
\curveto(781.7731772,277.04808187)(781.79817717,277.05308187)(781.81817871,277.0630896)
\curveto(781.94817702,277.06308186)(782.07817689,277.05808186)(782.20817871,277.0480896)
\curveto(782.34817662,277.04808187)(782.44317653,277.00808191)(782.49317871,276.9280896)
\curveto(782.54317643,276.86808205)(782.5681764,276.78808213)(782.56817871,276.6880896)
\lineto(782.56817871,276.4030896)
\lineto(782.56817871,270.0580896)
}
}
{
\newrgbcolor{curcolor}{0 0 0}
\pscustom[linestyle=none,fillstyle=solid,fillcolor=curcolor]
{
\newpath
\moveto(791.47302246,267.1330896)
\lineto(791.47302246,266.8030896)
\curveto(791.48301458,266.69309223)(791.4630146,266.60809231)(791.41302246,266.5480896)
\curveto(791.39301467,266.5180924)(791.36801469,266.49309243)(791.33802246,266.4730896)
\lineto(791.24802246,266.4130896)
\curveto(791.21801484,266.40309252)(791.16801489,266.39809252)(791.09802246,266.3980896)
\curveto(791.02801503,266.38809253)(790.95301511,266.38309254)(790.87302246,266.3830896)
\curveto(790.80301526,266.38309254)(790.73301533,266.38809253)(790.66302246,266.3980896)
\curveto(790.59301547,266.39809252)(790.54301552,266.40309252)(790.51302246,266.4130896)
\curveto(790.41301565,266.43309249)(790.34301572,266.47809244)(790.30302246,266.5480896)
\curveto(790.25301581,266.62809229)(790.22801583,266.75309217)(790.22802246,266.9230896)
\lineto(790.22802246,267.3430896)
\lineto(790.22802246,269.1880896)
\lineto(790.22802246,269.5480896)
\curveto(790.23801582,269.68808923)(790.22301584,269.80308912)(790.18302246,269.8930896)
\curveto(790.1630159,269.91308901)(790.14301592,269.92808899)(790.12302246,269.9380896)
\curveto(790.11301595,269.95808896)(790.09801596,269.97808894)(790.07802246,269.9980896)
\curveto(789.97801608,269.99808892)(789.89801616,269.97308895)(789.83802246,269.9230896)
\lineto(789.68802246,269.7730896)
\curveto(789.60801645,269.71308921)(789.52301654,269.65308927)(789.43302246,269.5930896)
\curveto(789.34301672,269.54308938)(789.24301682,269.49308943)(789.13302246,269.4430896)
\curveto(788.98301708,269.38308954)(788.80801725,269.33308959)(788.60802246,269.2930896)
\curveto(788.41801764,269.24308968)(788.21301785,269.21308971)(787.99302246,269.2030896)
\curveto(787.78301828,269.18308974)(787.56801849,269.18308974)(787.34802246,269.2030896)
\curveto(787.13801892,269.21308971)(786.94301912,269.24308968)(786.76302246,269.2930896)
\curveto(786.71301935,269.31308961)(786.6630194,269.32808959)(786.61302246,269.3380896)
\curveto(786.5630195,269.34808957)(786.51301955,269.36308956)(786.46302246,269.3830896)
\curveto(786.37301969,269.4230895)(786.28301978,269.45808946)(786.19302246,269.4880896)
\curveto(786.10301996,269.52808939)(786.01802004,269.57308935)(785.93802246,269.6230896)
\curveto(785.58802047,269.84308908)(785.29302077,270.09308883)(785.05302246,270.3730896)
\curveto(784.82302124,270.66308826)(784.62302144,271.0180879)(784.45302246,271.4380896)
\curveto(784.41302165,271.53808738)(784.37802168,271.64308728)(784.34802246,271.7530896)
\curveto(784.32802173,271.86308706)(784.30302176,271.97308695)(784.27302246,272.0830896)
\curveto(784.2630218,272.10308682)(784.2580218,272.1230868)(784.25802246,272.1430896)
\curveto(784.2580218,272.17308675)(784.25302181,272.20308672)(784.24302246,272.2330896)
\curveto(784.22302184,272.31308661)(784.20802185,272.40308652)(784.19802246,272.5030896)
\curveto(784.19802186,272.60308632)(784.18802187,272.69808622)(784.16802246,272.7880896)
\lineto(784.16802246,273.0430896)
\curveto(784.14802191,273.09308583)(784.13802192,273.15808576)(784.13802246,273.2380896)
\curveto(784.13802192,273.3180856)(784.14802191,273.38308554)(784.16802246,273.4330896)
\lineto(784.16802246,273.5980896)
\curveto(784.16802189,273.65808526)(784.17302189,273.7180852)(784.18302246,273.7780896)
\curveto(784.19302187,273.8180851)(784.19302187,273.85808506)(784.18302246,273.8980896)
\curveto(784.18302188,273.93808498)(784.18802187,273.98308494)(784.19802246,274.0330896)
\curveto(784.22802183,274.14308478)(784.24802181,274.24808467)(784.25802246,274.3480896)
\curveto(784.27802178,274.45808446)(784.30302176,274.56308436)(784.33302246,274.6630896)
\curveto(784.37302169,274.79308413)(784.41302165,274.91308401)(784.45302246,275.0230896)
\curveto(784.49302157,275.14308378)(784.53802152,275.25808366)(784.58802246,275.3680896)
\curveto(784.6580214,275.50808341)(784.73302133,275.63808328)(784.81302246,275.7580896)
\curveto(784.90302116,275.88808303)(784.99302107,276.01308291)(785.08302246,276.1330896)
\curveto(785.09302097,276.13308279)(785.10802095,276.14308278)(785.12802246,276.1630896)
\curveto(785.17802088,276.24308268)(785.25302081,276.3230826)(785.35302246,276.4030896)
\curveto(785.3630207,276.41308251)(785.36802069,276.4230825)(785.36802246,276.4330896)
\curveto(785.37802068,276.44308248)(785.39302067,276.45308247)(785.41302246,276.4630896)
\curveto(785.45302061,276.49308243)(785.48802057,276.5230824)(785.51802246,276.5530896)
\curveto(785.5580205,276.59308233)(785.60302046,276.62808229)(785.65302246,276.6580896)
\curveto(785.79302027,276.76808215)(785.94802011,276.85808206)(786.11802246,276.9280896)
\curveto(786.28801977,276.99808192)(786.46801959,277.06308186)(786.65802246,277.1230896)
\curveto(786.7580193,277.16308176)(786.8630192,277.18808173)(786.97302246,277.1980896)
\curveto(787.08301898,277.20808171)(787.19301887,277.2230817)(787.30302246,277.2430896)
\curveto(787.34301872,277.25308167)(787.39801866,277.25308167)(787.46802246,277.2430896)
\curveto(787.54801851,277.23308169)(787.59801846,277.23808168)(787.61802246,277.2580896)
\curveto(787.94801811,277.25808166)(788.26801779,277.2180817)(788.57802246,277.1380896)
\curveto(788.88801717,277.05808186)(789.14301692,276.95808196)(789.34302246,276.8380896)
\lineto(789.52302246,276.7180896)
\curveto(789.58301648,276.67808224)(789.64301642,276.63308229)(789.70302246,276.5830896)
\lineto(789.85302246,276.4630896)
\curveto(789.90301616,276.4230825)(789.97801608,276.40308252)(790.07802246,276.4030896)
\curveto(790.09801596,276.4230825)(790.11801594,276.43808248)(790.13802246,276.4480896)
\curveto(790.1580159,276.46808245)(790.17301589,276.49308243)(790.18302246,276.5230896)
\curveto(790.21301585,276.59308233)(790.22801583,276.66808225)(790.22802246,276.7480896)
\curveto(790.23801582,276.82808209)(790.26801579,276.89308203)(790.31802246,276.9430896)
\curveto(790.34801571,276.98308194)(790.40801565,277.01308191)(790.49802246,277.0330896)
\curveto(790.59801546,277.06308186)(790.70301536,277.07808184)(790.81302246,277.0780896)
\curveto(790.92301514,277.08808183)(791.02801503,277.08308184)(791.12802246,277.0630896)
\curveto(791.22801483,277.04308188)(791.30301476,277.0180819)(791.35302246,276.9880896)
\curveto(791.42301464,276.93808198)(791.4580146,276.85308207)(791.45802246,276.7330896)
\curveto(791.46801459,276.61308231)(791.47301459,276.49308243)(791.47302246,276.3730896)
\lineto(791.47302246,267.1330896)
\moveto(790.25802246,272.9980896)
\curveto(790.26801579,273.04808587)(790.27301579,273.1180858)(790.27302246,273.2080896)
\curveto(790.28301578,273.29808562)(790.27801578,273.36808555)(790.25802246,273.4180896)
\lineto(790.25802246,273.6280896)
\lineto(790.19802246,273.9280896)
\curveto(790.18801587,274.02808489)(790.17301589,274.1180848)(790.15302246,274.1980896)
\curveto(790.13301593,274.27808464)(790.11301595,274.34808457)(790.09302246,274.4080896)
\curveto(790.08301598,274.47808444)(790.063016,274.54808437)(790.03302246,274.6180896)
\curveto(789.92301614,274.88808403)(789.74801631,275.15308377)(789.50802246,275.4130896)
\curveto(789.26801679,275.67308325)(789.03801702,275.85308307)(788.81802246,275.9530896)
\curveto(788.73801732,275.99308293)(788.65301741,276.0230829)(788.56302246,276.0430896)
\curveto(788.48301758,276.06308286)(788.39801766,276.08808283)(788.30802246,276.1180896)
\curveto(788.20801785,276.13808278)(788.09801796,276.14808277)(787.97802246,276.1480896)
\lineto(787.63302246,276.1480896)
\lineto(787.48302246,276.1180896)
\lineto(787.34802246,276.1180896)
\lineto(787.10802246,276.0580896)
\curveto(787.02801903,276.03808288)(786.95301911,276.00808291)(786.88302246,275.9680896)
\curveto(786.5630195,275.82808309)(786.30301976,275.62808329)(786.10302246,275.3680896)
\curveto(785.91302015,275.10808381)(785.7630203,274.80308412)(785.65302246,274.4530896)
\curveto(785.61302045,274.34308458)(785.58302048,274.2230847)(785.56302246,274.0930896)
\curveto(785.55302051,273.97308495)(785.53302053,273.84808507)(785.50302246,273.7180896)
\lineto(785.50302246,273.5830896)
\curveto(785.50302056,273.54308538)(785.49802056,273.49808542)(785.48802246,273.4480896)
\curveto(785.47802058,273.40808551)(785.47302059,273.36308556)(785.47302246,273.3130896)
\curveto(785.48302058,273.26308566)(785.48802057,273.21308571)(785.48802246,273.1630896)
\lineto(785.48802246,272.8630896)
\curveto(785.48802057,272.77308615)(785.49802056,272.68808623)(785.51802246,272.6080896)
\curveto(785.52802053,272.57808634)(785.53302053,272.53308639)(785.53302246,272.4730896)
\curveto(785.55302051,272.40308652)(785.56802049,272.33308659)(785.57802246,272.2630896)
\lineto(785.63802246,272.0530896)
\curveto(785.72802033,271.76308716)(785.84802021,271.49808742)(785.99802246,271.2580896)
\curveto(786.14801991,271.02808789)(786.33801972,270.83308809)(786.56802246,270.6730896)
\lineto(786.65802246,270.6130896)
\curveto(786.69801936,270.59308833)(786.73301933,270.57308835)(786.76302246,270.5530896)
\curveto(786.8630192,270.49308843)(786.96801909,270.44308848)(787.07802246,270.4030896)
\lineto(787.43802246,270.3130896)
\curveto(787.48801857,270.29308863)(787.52801853,270.28308864)(787.55802246,270.2830896)
\curveto(787.58801847,270.29308863)(787.62801843,270.29308863)(787.67802246,270.2830896)
\curveto(787.71801834,270.27308865)(787.76801829,270.26308866)(787.82802246,270.2530896)
\curveto(787.88801817,270.25308867)(787.94301812,270.26308866)(787.99302246,270.2830896)
\lineto(788.11302246,270.2830896)
\curveto(788.14301792,270.29308863)(788.17301789,270.29308863)(788.20302246,270.2830896)
\curveto(788.23301783,270.28308864)(788.2630178,270.28808863)(788.29302246,270.2980896)
\curveto(788.37301769,270.3180886)(788.45301761,270.33308859)(788.53302246,270.3430896)
\curveto(788.61301745,270.36308856)(788.68801737,270.38808853)(788.75802246,270.4180896)
\curveto(789.06801699,270.54808837)(789.32301674,270.7230882)(789.52302246,270.9430896)
\curveto(789.72301634,271.17308775)(789.88801617,271.43808748)(790.01802246,271.7380896)
\curveto(790.06801599,271.84808707)(790.10301596,271.95808696)(790.12302246,272.0680896)
\curveto(790.14301592,272.17808674)(790.16801589,272.29308663)(790.19802246,272.4130896)
\curveto(790.21801584,272.45308647)(790.22801583,272.49308643)(790.22802246,272.5330896)
\curveto(790.22801583,272.57308635)(790.23301583,272.61308631)(790.24302246,272.6530896)
\curveto(790.25301581,272.70308622)(790.25301581,272.75808616)(790.24302246,272.8180896)
\curveto(790.24301582,272.87808604)(790.24801581,272.93808598)(790.25802246,272.9980896)
}
}
{
\newrgbcolor{curcolor}{0 0 0}
\pscustom[linestyle=none,fillstyle=solid,fillcolor=curcolor]
{
\newpath
\moveto(793.88427246,277.0630896)
\lineto(794.31927246,277.0630896)
\curveto(794.4692705,277.06308186)(794.57427039,277.0230819)(794.63427246,276.9430896)
\curveto(794.68427028,276.86308206)(794.70927026,276.76308216)(794.70927246,276.6430896)
\curveto(794.71927025,276.5230824)(794.72427024,276.40308252)(794.72427246,276.2830896)
\lineto(794.72427246,274.8580896)
\lineto(794.72427246,272.5930896)
\lineto(794.72427246,271.9030896)
\curveto(794.72427024,271.67308725)(794.74927022,271.47308745)(794.79927246,271.3030896)
\curveto(794.95927001,270.85308807)(795.25926971,270.53808838)(795.69927246,270.3580896)
\curveto(795.91926905,270.26808865)(796.18426878,270.23308869)(796.49427246,270.2530896)
\curveto(796.80426816,270.28308864)(797.05426791,270.33808858)(797.24427246,270.4180896)
\curveto(797.57426739,270.55808836)(797.83426713,270.73308819)(798.02427246,270.9430896)
\curveto(798.22426674,271.16308776)(798.37926659,271.44808747)(798.48927246,271.7980896)
\curveto(798.51926645,271.87808704)(798.53926643,271.95808696)(798.54927246,272.0380896)
\curveto(798.55926641,272.1180868)(798.57426639,272.20308672)(798.59427246,272.2930896)
\curveto(798.60426636,272.34308658)(798.60426636,272.38808653)(798.59427246,272.4280896)
\curveto(798.59426637,272.46808645)(798.60426636,272.51308641)(798.62427246,272.5630896)
\lineto(798.62427246,272.8780896)
\curveto(798.64426632,272.95808596)(798.64926632,273.04808587)(798.63927246,273.1480896)
\curveto(798.62926634,273.25808566)(798.62426634,273.35808556)(798.62427246,273.4480896)
\lineto(798.62427246,274.6180896)
\lineto(798.62427246,276.2080896)
\curveto(798.62426634,276.32808259)(798.61926635,276.45308247)(798.60927246,276.5830896)
\curveto(798.60926636,276.7230822)(798.63426633,276.83308209)(798.68427246,276.9130896)
\curveto(798.72426624,276.96308196)(798.7692662,276.99308193)(798.81927246,277.0030896)
\curveto(798.87926609,277.0230819)(798.94926602,277.04308188)(799.02927246,277.0630896)
\lineto(799.25427246,277.0630896)
\curveto(799.37426559,277.06308186)(799.47926549,277.05808186)(799.56927246,277.0480896)
\curveto(799.6692653,277.03808188)(799.74426522,276.99308193)(799.79427246,276.9130896)
\curveto(799.84426512,276.86308206)(799.8692651,276.78808213)(799.86927246,276.6880896)
\lineto(799.86927246,276.4030896)
\lineto(799.86927246,275.3830896)
\lineto(799.86927246,271.3480896)
\lineto(799.86927246,269.9980896)
\curveto(799.8692651,269.87808904)(799.8642651,269.76308916)(799.85427246,269.6530896)
\curveto(799.85426511,269.55308937)(799.81926515,269.47808944)(799.74927246,269.4280896)
\curveto(799.70926526,269.39808952)(799.64926532,269.37308955)(799.56927246,269.3530896)
\curveto(799.48926548,269.34308958)(799.39926557,269.33308959)(799.29927246,269.3230896)
\curveto(799.20926576,269.3230896)(799.11926585,269.32808959)(799.02927246,269.3380896)
\curveto(798.94926602,269.34808957)(798.88926608,269.36808955)(798.84927246,269.3980896)
\curveto(798.79926617,269.43808948)(798.75426621,269.50308942)(798.71427246,269.5930896)
\curveto(798.70426626,269.63308929)(798.69426627,269.68808923)(798.68427246,269.7580896)
\curveto(798.68426628,269.82808909)(798.67926629,269.89308903)(798.66927246,269.9530896)
\curveto(798.65926631,270.0230889)(798.63926633,270.07808884)(798.60927246,270.1180896)
\curveto(798.57926639,270.15808876)(798.53426643,270.17308875)(798.47427246,270.1630896)
\curveto(798.39426657,270.14308878)(798.31426665,270.08308884)(798.23427246,269.9830896)
\curveto(798.15426681,269.89308903)(798.07926689,269.8230891)(798.00927246,269.7730896)
\curveto(797.78926718,269.61308931)(797.53926743,269.47308945)(797.25927246,269.3530896)
\curveto(797.14926782,269.30308962)(797.03426793,269.27308965)(796.91427246,269.2630896)
\curveto(796.80426816,269.24308968)(796.68926828,269.2180897)(796.56927246,269.1880896)
\curveto(796.51926845,269.17808974)(796.4642685,269.17808974)(796.40427246,269.1880896)
\curveto(796.35426861,269.19808972)(796.30426866,269.19308973)(796.25427246,269.1730896)
\curveto(796.15426881,269.15308977)(796.0642689,269.15308977)(795.98427246,269.1730896)
\lineto(795.83427246,269.1730896)
\curveto(795.78426918,269.19308973)(795.72426924,269.20308972)(795.65427246,269.2030896)
\curveto(795.59426937,269.20308972)(795.53926943,269.20808971)(795.48927246,269.2180896)
\curveto(795.44926952,269.23808968)(795.40926956,269.24808967)(795.36927246,269.2480896)
\curveto(795.33926963,269.23808968)(795.29926967,269.24308968)(795.24927246,269.2630896)
\lineto(795.00927246,269.3230896)
\curveto(794.93927003,269.34308958)(794.8642701,269.37308955)(794.78427246,269.4130896)
\curveto(794.52427044,269.5230894)(794.30427066,269.66808925)(794.12427246,269.8480896)
\curveto(793.95427101,270.03808888)(793.81427115,270.26308866)(793.70427246,270.5230896)
\curveto(793.6642713,270.61308831)(793.63427133,270.70308822)(793.61427246,270.7930896)
\lineto(793.55427246,271.0930896)
\curveto(793.53427143,271.15308777)(793.52427144,271.20808771)(793.52427246,271.2580896)
\curveto(793.53427143,271.3180876)(793.52927144,271.38308754)(793.50927246,271.4530896)
\curveto(793.49927147,271.47308745)(793.49427147,271.49808742)(793.49427246,271.5280896)
\curveto(793.49427147,271.56808735)(793.48927148,271.60308732)(793.47927246,271.6330896)
\lineto(793.47927246,271.7830896)
\curveto(793.4692715,271.8230871)(793.4642715,271.86808705)(793.46427246,271.9180896)
\curveto(793.47427149,271.97808694)(793.47927149,272.03308689)(793.47927246,272.0830896)
\lineto(793.47927246,272.6830896)
\lineto(793.47927246,275.4430896)
\lineto(793.47927246,276.4030896)
\lineto(793.47927246,276.6730896)
\curveto(793.47927149,276.76308216)(793.49927147,276.83808208)(793.53927246,276.8980896)
\curveto(793.57927139,276.96808195)(793.65427131,277.0180819)(793.76427246,277.0480896)
\curveto(793.78427118,277.05808186)(793.80427116,277.05808186)(793.82427246,277.0480896)
\curveto(793.84427112,277.04808187)(793.8642711,277.05308187)(793.88427246,277.0630896)
}
}
{
\newrgbcolor{curcolor}{0 0 0}
\pscustom[linestyle=none,fillstyle=solid,fillcolor=curcolor]
{
\newpath
\moveto(808.51888184,273.5080896)
\curveto(808.53887415,273.40808551)(808.53887415,273.29308563)(808.51888184,273.1630896)
\curveto(808.50887418,273.04308588)(808.47887421,272.95808596)(808.42888184,272.9080896)
\curveto(808.37887431,272.86808605)(808.30387439,272.83808608)(808.20388184,272.8180896)
\curveto(808.11387458,272.80808611)(808.00887468,272.80308612)(807.88888184,272.8030896)
\lineto(807.52888184,272.8030896)
\curveto(807.40887528,272.81308611)(807.30387539,272.8180861)(807.21388184,272.8180896)
\lineto(803.37388184,272.8180896)
\curveto(803.2938794,272.8180861)(803.21387948,272.81308611)(803.13388184,272.8030896)
\curveto(803.05387964,272.80308612)(802.9888797,272.78808613)(802.93888184,272.7580896)
\curveto(802.89887979,272.73808618)(802.85887983,272.69808622)(802.81888184,272.6380896)
\curveto(802.79887989,272.60808631)(802.77887991,272.56308636)(802.75888184,272.5030896)
\curveto(802.73887995,272.45308647)(802.73887995,272.40308652)(802.75888184,272.3530896)
\curveto(802.76887992,272.30308662)(802.77387992,272.25808666)(802.77388184,272.2180896)
\curveto(802.77387992,272.17808674)(802.77887991,272.13808678)(802.78888184,272.0980896)
\curveto(802.80887988,272.0180869)(802.82887986,271.93308699)(802.84888184,271.8430896)
\curveto(802.86887982,271.76308716)(802.89887979,271.68308724)(802.93888184,271.6030896)
\curveto(803.16887952,271.06308786)(803.54887914,270.67808824)(804.07888184,270.4480896)
\curveto(804.13887855,270.4180885)(804.20387849,270.39308853)(804.27388184,270.3730896)
\lineto(804.48388184,270.3130896)
\curveto(804.51387818,270.30308862)(804.56387813,270.29808862)(804.63388184,270.2980896)
\curveto(804.77387792,270.25808866)(804.95887773,270.23808868)(805.18888184,270.2380896)
\curveto(805.41887727,270.23808868)(805.60387709,270.25808866)(805.74388184,270.2980896)
\curveto(805.88387681,270.33808858)(806.00887668,270.37808854)(806.11888184,270.4180896)
\curveto(806.23887645,270.46808845)(806.34887634,270.52808839)(806.44888184,270.5980896)
\curveto(806.55887613,270.66808825)(806.65387604,270.74808817)(806.73388184,270.8380896)
\curveto(806.81387588,270.93808798)(806.88387581,271.04308788)(806.94388184,271.1530896)
\curveto(807.00387569,271.25308767)(807.05387564,271.35808756)(807.09388184,271.4680896)
\curveto(807.14387555,271.57808734)(807.22387547,271.65808726)(807.33388184,271.7080896)
\curveto(807.37387532,271.72808719)(807.43887525,271.74308718)(807.52888184,271.7530896)
\curveto(807.61887507,271.76308716)(807.70887498,271.76308716)(807.79888184,271.7530896)
\curveto(807.8888748,271.75308717)(807.97387472,271.74808717)(808.05388184,271.7380896)
\curveto(808.13387456,271.72808719)(808.1888745,271.70808721)(808.21888184,271.6780896)
\curveto(808.31887437,271.60808731)(808.34387435,271.49308743)(808.29388184,271.3330896)
\curveto(808.21387448,271.06308786)(808.10887458,270.8230881)(807.97888184,270.6130896)
\curveto(807.77887491,270.29308863)(807.54887514,270.02808889)(807.28888184,269.8180896)
\curveto(807.03887565,269.6180893)(806.71887597,269.45308947)(806.32888184,269.3230896)
\curveto(806.22887646,269.28308964)(806.12887656,269.25808966)(806.02888184,269.2480896)
\curveto(805.92887676,269.22808969)(805.82387687,269.20808971)(805.71388184,269.1880896)
\curveto(805.66387703,269.17808974)(805.61387708,269.17308975)(805.56388184,269.1730896)
\curveto(805.52387717,269.17308975)(805.47887721,269.16808975)(805.42888184,269.1580896)
\lineto(805.27888184,269.1580896)
\curveto(805.22887746,269.14808977)(805.16887752,269.14308978)(805.09888184,269.1430896)
\curveto(805.03887765,269.14308978)(804.9888777,269.14808977)(804.94888184,269.1580896)
\lineto(804.81388184,269.1580896)
\curveto(804.76387793,269.16808975)(804.71887797,269.17308975)(804.67888184,269.1730896)
\curveto(804.63887805,269.17308975)(804.59887809,269.17808974)(804.55888184,269.1880896)
\curveto(804.50887818,269.19808972)(804.45387824,269.20808971)(804.39388184,269.2180896)
\curveto(804.33387836,269.2180897)(804.27887841,269.2230897)(804.22888184,269.2330896)
\curveto(804.13887855,269.25308967)(804.04887864,269.27808964)(803.95888184,269.3080896)
\curveto(803.86887882,269.32808959)(803.78387891,269.35308957)(803.70388184,269.3830896)
\curveto(803.66387903,269.40308952)(803.62887906,269.41308951)(803.59888184,269.4130896)
\curveto(803.56887912,269.4230895)(803.53387916,269.43808948)(803.49388184,269.4580896)
\curveto(803.34387935,269.52808939)(803.18387951,269.61308931)(803.01388184,269.7130896)
\curveto(802.72387997,269.90308902)(802.47388022,270.13308879)(802.26388184,270.4030896)
\curveto(802.06388063,270.68308824)(801.8938808,270.99308793)(801.75388184,271.3330896)
\curveto(801.70388099,271.44308748)(801.66388103,271.55808736)(801.63388184,271.6780896)
\curveto(801.61388108,271.79808712)(801.58388111,271.918087)(801.54388184,272.0380896)
\curveto(801.53388116,272.07808684)(801.52888116,272.11308681)(801.52888184,272.1430896)
\curveto(801.52888116,272.17308675)(801.52388117,272.21308671)(801.51388184,272.2630896)
\curveto(801.4938812,272.34308658)(801.47888121,272.42808649)(801.46888184,272.5180896)
\curveto(801.45888123,272.60808631)(801.44388125,272.69808622)(801.42388184,272.7880896)
\lineto(801.42388184,272.9980896)
\curveto(801.41388128,273.03808588)(801.40388129,273.09308583)(801.39388184,273.1630896)
\curveto(801.3938813,273.24308568)(801.39888129,273.30808561)(801.40888184,273.3580896)
\lineto(801.40888184,273.5230896)
\curveto(801.42888126,273.57308535)(801.43388126,273.6230853)(801.42388184,273.6730896)
\curveto(801.42388127,273.73308519)(801.42888126,273.78808513)(801.43888184,273.8380896)
\curveto(801.47888121,273.99808492)(801.50888118,274.15808476)(801.52888184,274.3180896)
\curveto(801.55888113,274.47808444)(801.60388109,274.62808429)(801.66388184,274.7680896)
\curveto(801.71388098,274.87808404)(801.75888093,274.98808393)(801.79888184,275.0980896)
\curveto(801.84888084,275.2180837)(801.90388079,275.33308359)(801.96388184,275.4430896)
\curveto(802.18388051,275.79308313)(802.43388026,276.09308283)(802.71388184,276.3430896)
\curveto(802.9938797,276.60308232)(803.33887935,276.8180821)(803.74888184,276.9880896)
\curveto(803.86887882,277.03808188)(803.9888787,277.07308185)(804.10888184,277.0930896)
\curveto(804.23887845,277.1230818)(804.37387832,277.15308177)(804.51388184,277.1830896)
\curveto(804.56387813,277.19308173)(804.60887808,277.19808172)(804.64888184,277.1980896)
\curveto(804.688878,277.20808171)(804.73387796,277.21308171)(804.78388184,277.2130896)
\curveto(804.80387789,277.2230817)(804.82887786,277.2230817)(804.85888184,277.2130896)
\curveto(804.8888778,277.20308172)(804.91387778,277.20808171)(804.93388184,277.2280896)
\curveto(805.35387734,277.23808168)(805.71887697,277.19308173)(806.02888184,277.0930896)
\curveto(806.33887635,277.00308192)(806.61887607,276.87808204)(806.86888184,276.7180896)
\curveto(806.91887577,276.69808222)(806.95887573,276.66808225)(806.98888184,276.6280896)
\curveto(807.01887567,276.59808232)(807.05387564,276.57308235)(807.09388184,276.5530896)
\curveto(807.17387552,276.49308243)(807.25387544,276.4230825)(807.33388184,276.3430896)
\curveto(807.42387527,276.26308266)(807.49887519,276.18308274)(807.55888184,276.1030896)
\curveto(807.71887497,275.89308303)(807.85387484,275.69308323)(807.96388184,275.5030896)
\curveto(808.03387466,275.39308353)(808.0888746,275.27308365)(808.12888184,275.1430896)
\curveto(808.16887452,275.01308391)(808.21387448,274.88308404)(808.26388184,274.7530896)
\curveto(808.31387438,274.6230843)(808.34887434,274.48808443)(808.36888184,274.3480896)
\curveto(808.39887429,274.20808471)(808.43387426,274.06808485)(808.47388184,273.9280896)
\curveto(808.48387421,273.85808506)(808.4888742,273.78808513)(808.48888184,273.7180896)
\lineto(808.51888184,273.5080896)
\moveto(807.06388184,274.0180896)
\curveto(807.0938756,274.05808486)(807.11887557,274.10808481)(807.13888184,274.1680896)
\curveto(807.15887553,274.23808468)(807.15887553,274.30808461)(807.13888184,274.3780896)
\curveto(807.07887561,274.59808432)(806.9938757,274.80308412)(806.88388184,274.9930896)
\curveto(806.74387595,275.2230837)(806.5888761,275.4180835)(806.41888184,275.5780896)
\curveto(806.24887644,275.73808318)(806.02887666,275.87308305)(805.75888184,275.9830896)
\curveto(805.688877,276.00308292)(805.61887707,276.0180829)(805.54888184,276.0280896)
\curveto(805.47887721,276.04808287)(805.40387729,276.06808285)(805.32388184,276.0880896)
\curveto(805.24387745,276.10808281)(805.15887753,276.1180828)(805.06888184,276.1180896)
\lineto(804.81388184,276.1180896)
\curveto(804.78387791,276.09808282)(804.74887794,276.08808283)(804.70888184,276.0880896)
\curveto(804.66887802,276.09808282)(804.63387806,276.09808282)(804.60388184,276.0880896)
\lineto(804.36388184,276.0280896)
\curveto(804.2938784,276.0180829)(804.22387847,276.00308292)(804.15388184,275.9830896)
\curveto(803.86387883,275.86308306)(803.62887906,275.71308321)(803.44888184,275.5330896)
\curveto(803.27887941,275.35308357)(803.12387957,275.12808379)(802.98388184,274.8580896)
\curveto(802.95387974,274.80808411)(802.92387977,274.74308418)(802.89388184,274.6630896)
\curveto(802.86387983,274.59308433)(802.83887985,274.51308441)(802.81888184,274.4230896)
\curveto(802.79887989,274.33308459)(802.7938799,274.24808467)(802.80388184,274.1680896)
\curveto(802.81387988,274.08808483)(802.84887984,274.02808489)(802.90888184,273.9880896)
\curveto(802.9888797,273.92808499)(803.12387957,273.89808502)(803.31388184,273.8980896)
\curveto(803.51387918,273.90808501)(803.68387901,273.91308501)(803.82388184,273.9130896)
\lineto(806.10388184,273.9130896)
\curveto(806.25387644,273.91308501)(806.43387626,273.90808501)(806.64388184,273.8980896)
\curveto(806.85387584,273.89808502)(806.9938757,273.93808498)(807.06388184,274.0180896)
}
}
{
\newrgbcolor{curcolor}{0 0 0}
\pscustom[linestyle=none,fillstyle=solid,fillcolor=curcolor]
{
\newpath
\moveto(810.77052246,279.4030896)
\curveto(810.92052045,279.40307952)(811.0705203,279.39807952)(811.22052246,279.3880896)
\curveto(811.37052,279.38807953)(811.4755199,279.34807957)(811.53552246,279.2680896)
\curveto(811.58551979,279.20807971)(811.61051976,279.1230798)(811.61052246,279.0130896)
\curveto(811.62051975,278.91308001)(811.62551975,278.80808011)(811.62552246,278.6980896)
\lineto(811.62552246,277.8280896)
\curveto(811.62551975,277.74808117)(811.62051975,277.66308126)(811.61052246,277.5730896)
\curveto(811.61051976,277.49308143)(811.62051975,277.4230815)(811.64052246,277.3630896)
\curveto(811.68051969,277.2230817)(811.7705196,277.13308179)(811.91052246,277.0930896)
\curveto(811.96051941,277.08308184)(812.00551937,277.07808184)(812.04552246,277.0780896)
\lineto(812.19552246,277.0780896)
\lineto(812.60052246,277.0780896)
\curveto(812.76051861,277.08808183)(812.8755185,277.07808184)(812.94552246,277.0480896)
\curveto(813.03551834,276.98808193)(813.09551828,276.92808199)(813.12552246,276.8680896)
\curveto(813.14551823,276.82808209)(813.15551822,276.78308214)(813.15552246,276.7330896)
\lineto(813.15552246,276.5830896)
\curveto(813.15551822,276.47308245)(813.15051822,276.36808255)(813.14052246,276.2680896)
\curveto(813.13051824,276.17808274)(813.09551828,276.10808281)(813.03552246,276.0580896)
\curveto(812.9755184,276.00808291)(812.89051848,275.97808294)(812.78052246,275.9680896)
\lineto(812.45052246,275.9680896)
\curveto(812.34051903,275.97808294)(812.23051914,275.98308294)(812.12052246,275.9830896)
\curveto(812.01051936,275.98308294)(811.91551946,275.96808295)(811.83552246,275.9380896)
\curveto(811.76551961,275.90808301)(811.71551966,275.85808306)(811.68552246,275.7880896)
\curveto(811.65551972,275.7180832)(811.63551974,275.63308329)(811.62552246,275.5330896)
\curveto(811.61551976,275.44308348)(811.61051976,275.34308358)(811.61052246,275.2330896)
\curveto(811.62051975,275.13308379)(811.62551975,275.03308389)(811.62552246,274.9330896)
\lineto(811.62552246,271.9630896)
\curveto(811.62551975,271.74308718)(811.62051975,271.50808741)(811.61052246,271.2580896)
\curveto(811.61051976,271.0180879)(811.65551972,270.83308809)(811.74552246,270.7030896)
\curveto(811.79551958,270.6230883)(811.86051951,270.56808835)(811.94052246,270.5380896)
\curveto(812.02051935,270.50808841)(812.11551926,270.48308844)(812.22552246,270.4630896)
\curveto(812.25551912,270.45308847)(812.28551909,270.44808847)(812.31552246,270.4480896)
\curveto(812.35551902,270.45808846)(812.39051898,270.45808846)(812.42052246,270.4480896)
\lineto(812.61552246,270.4480896)
\curveto(812.71551866,270.44808847)(812.80551857,270.43808848)(812.88552246,270.4180896)
\curveto(812.9755184,270.40808851)(813.04051833,270.37308855)(813.08052246,270.3130896)
\curveto(813.10051827,270.28308864)(813.11551826,270.22808869)(813.12552246,270.1480896)
\curveto(813.14551823,270.07808884)(813.15551822,270.00308892)(813.15552246,269.9230896)
\curveto(813.16551821,269.84308908)(813.16551821,269.76308916)(813.15552246,269.6830896)
\curveto(813.14551823,269.61308931)(813.12551825,269.55808936)(813.09552246,269.5180896)
\curveto(813.05551832,269.44808947)(812.98051839,269.39808952)(812.87052246,269.3680896)
\curveto(812.79051858,269.34808957)(812.70051867,269.33808958)(812.60052246,269.3380896)
\curveto(812.50051887,269.34808957)(812.41051896,269.35308957)(812.33052246,269.3530896)
\curveto(812.2705191,269.35308957)(812.21051916,269.34808957)(812.15052246,269.3380896)
\curveto(812.09051928,269.33808958)(812.03551934,269.34308958)(811.98552246,269.3530896)
\lineto(811.80552246,269.3530896)
\curveto(811.75551962,269.36308956)(811.70551967,269.36808955)(811.65552246,269.3680896)
\curveto(811.61551976,269.37808954)(811.5705198,269.38308954)(811.52052246,269.3830896)
\curveto(811.32052005,269.43308949)(811.14552023,269.48808943)(810.99552246,269.5480896)
\curveto(810.85552052,269.60808931)(810.73552064,269.71308921)(810.63552246,269.8630896)
\curveto(810.49552088,270.06308886)(810.41552096,270.31308861)(810.39552246,270.6130896)
\curveto(810.375521,270.923088)(810.36552101,271.25308767)(810.36552246,271.6030896)
\lineto(810.36552246,275.5330896)
\curveto(810.33552104,275.66308326)(810.30552107,275.75808316)(810.27552246,275.8180896)
\curveto(810.25552112,275.87808304)(810.18552119,275.92808299)(810.06552246,275.9680896)
\curveto(810.02552135,275.97808294)(809.98552139,275.97808294)(809.94552246,275.9680896)
\curveto(809.90552147,275.95808296)(809.86552151,275.96308296)(809.82552246,275.9830896)
\lineto(809.58552246,275.9830896)
\curveto(809.45552192,275.98308294)(809.34552203,275.99308293)(809.25552246,276.0130896)
\curveto(809.1755222,276.04308288)(809.12052225,276.10308282)(809.09052246,276.1930896)
\curveto(809.0705223,276.23308269)(809.05552232,276.27808264)(809.04552246,276.3280896)
\lineto(809.04552246,276.4780896)
\curveto(809.04552233,276.6180823)(809.05552232,276.73308219)(809.07552246,276.8230896)
\curveto(809.09552228,276.923082)(809.15552222,276.99808192)(809.25552246,277.0480896)
\curveto(809.36552201,277.08808183)(809.50552187,277.09808182)(809.67552246,277.0780896)
\curveto(809.85552152,277.05808186)(810.00552137,277.06808185)(810.12552246,277.1080896)
\curveto(810.21552116,277.15808176)(810.28552109,277.22808169)(810.33552246,277.3180896)
\curveto(810.35552102,277.37808154)(810.36552101,277.45308147)(810.36552246,277.5430896)
\lineto(810.36552246,277.7980896)
\lineto(810.36552246,278.7280896)
\lineto(810.36552246,278.9680896)
\curveto(810.36552101,279.05807986)(810.375521,279.13307979)(810.39552246,279.1930896)
\curveto(810.43552094,279.27307965)(810.51052086,279.33807958)(810.62052246,279.3880896)
\curveto(810.65052072,279.38807953)(810.6755207,279.38807953)(810.69552246,279.3880896)
\curveto(810.72552065,279.39807952)(810.75052062,279.40307952)(810.77052246,279.4030896)
}
}
{
\newrgbcolor{curcolor}{0 0 0}
\pscustom[linestyle=none,fillstyle=solid,fillcolor=curcolor]
{
\newpath
\moveto(821.42731934,269.8930896)
\curveto(821.45731151,269.73308919)(821.44231152,269.59808932)(821.38231934,269.4880896)
\curveto(821.32231164,269.38808953)(821.24231172,269.31308961)(821.14231934,269.2630896)
\curveto(821.09231187,269.24308968)(821.03731193,269.23308969)(820.97731934,269.2330896)
\curveto(820.92731204,269.23308969)(820.87231209,269.2230897)(820.81231934,269.2030896)
\curveto(820.59231237,269.15308977)(820.37231259,269.16808975)(820.15231934,269.2480896)
\curveto(819.94231302,269.3180896)(819.79731317,269.40808951)(819.71731934,269.5180896)
\curveto(819.6673133,269.58808933)(819.62231334,269.66808925)(819.58231934,269.7580896)
\curveto(819.54231342,269.85808906)(819.49231347,269.93808898)(819.43231934,269.9980896)
\curveto(819.41231355,270.0180889)(819.38731358,270.03808888)(819.35731934,270.0580896)
\curveto(819.33731363,270.07808884)(819.30731366,270.08308884)(819.26731934,270.0730896)
\curveto(819.15731381,270.04308888)(819.05231391,269.98808893)(818.95231934,269.9080896)
\curveto(818.8623141,269.82808909)(818.77231419,269.75808916)(818.68231934,269.6980896)
\curveto(818.55231441,269.6180893)(818.41231455,269.54308938)(818.26231934,269.4730896)
\curveto(818.11231485,269.41308951)(817.95231501,269.35808956)(817.78231934,269.3080896)
\curveto(817.68231528,269.27808964)(817.57231539,269.25808966)(817.45231934,269.2480896)
\curveto(817.34231562,269.23808968)(817.23231573,269.2230897)(817.12231934,269.2030896)
\curveto(817.07231589,269.19308973)(817.02731594,269.18808973)(816.98731934,269.1880896)
\lineto(816.88231934,269.1880896)
\curveto(816.77231619,269.16808975)(816.6673163,269.16808975)(816.56731934,269.1880896)
\lineto(816.43231934,269.1880896)
\curveto(816.38231658,269.19808972)(816.33231663,269.20308972)(816.28231934,269.2030896)
\curveto(816.23231673,269.20308972)(816.18731678,269.21308971)(816.14731934,269.2330896)
\curveto(816.10731686,269.24308968)(816.07231689,269.24808967)(816.04231934,269.2480896)
\curveto(816.02231694,269.23808968)(815.99731697,269.23808968)(815.96731934,269.2480896)
\lineto(815.72731934,269.3080896)
\curveto(815.64731732,269.3180896)(815.57231739,269.33808958)(815.50231934,269.3680896)
\curveto(815.20231776,269.49808942)(814.95731801,269.64308928)(814.76731934,269.8030896)
\curveto(814.58731838,269.97308895)(814.43731853,270.20808871)(814.31731934,270.5080896)
\curveto(814.22731874,270.72808819)(814.18231878,270.99308793)(814.18231934,271.3030896)
\lineto(814.18231934,271.6180896)
\curveto(814.19231877,271.66808725)(814.19731877,271.7180872)(814.19731934,271.7680896)
\lineto(814.22731934,271.9480896)
\lineto(814.34731934,272.2780896)
\curveto(814.38731858,272.38808653)(814.43731853,272.48808643)(814.49731934,272.5780896)
\curveto(814.67731829,272.86808605)(814.92231804,273.08308584)(815.23231934,273.2230896)
\curveto(815.54231742,273.36308556)(815.88231708,273.48808543)(816.25231934,273.5980896)
\curveto(816.39231657,273.63808528)(816.53731643,273.66808525)(816.68731934,273.6880896)
\curveto(816.83731613,273.70808521)(816.98731598,273.73308519)(817.13731934,273.7630896)
\curveto(817.20731576,273.78308514)(817.27231569,273.79308513)(817.33231934,273.7930896)
\curveto(817.40231556,273.79308513)(817.47731549,273.80308512)(817.55731934,273.8230896)
\curveto(817.62731534,273.84308508)(817.69731527,273.85308507)(817.76731934,273.8530896)
\curveto(817.83731513,273.86308506)(817.91231505,273.87808504)(817.99231934,273.8980896)
\curveto(818.24231472,273.95808496)(818.47731449,274.00808491)(818.69731934,274.0480896)
\curveto(818.91731405,274.09808482)(819.09231387,274.21308471)(819.22231934,274.3930896)
\curveto(819.28231368,274.47308445)(819.33231363,274.57308435)(819.37231934,274.6930896)
\curveto(819.41231355,274.8230841)(819.41231355,274.96308396)(819.37231934,275.1130896)
\curveto(819.31231365,275.35308357)(819.22231374,275.54308338)(819.10231934,275.6830896)
\curveto(818.99231397,275.8230831)(818.83231413,275.93308299)(818.62231934,276.0130896)
\curveto(818.50231446,276.06308286)(818.35731461,276.09808282)(818.18731934,276.1180896)
\curveto(818.02731494,276.13808278)(817.85731511,276.14808277)(817.67731934,276.1480896)
\curveto(817.49731547,276.14808277)(817.32231564,276.13808278)(817.15231934,276.1180896)
\curveto(816.98231598,276.09808282)(816.83731613,276.06808285)(816.71731934,276.0280896)
\curveto(816.54731642,275.96808295)(816.38231658,275.88308304)(816.22231934,275.7730896)
\curveto(816.14231682,275.71308321)(816.0673169,275.63308329)(815.99731934,275.5330896)
\curveto(815.93731703,275.44308348)(815.88231708,275.34308358)(815.83231934,275.2330896)
\curveto(815.80231716,275.15308377)(815.77231719,275.06808385)(815.74231934,274.9780896)
\curveto(815.72231724,274.88808403)(815.67731729,274.8180841)(815.60731934,274.7680896)
\curveto(815.5673174,274.73808418)(815.49731747,274.71308421)(815.39731934,274.6930896)
\curveto(815.30731766,274.68308424)(815.21231775,274.67808424)(815.11231934,274.6780896)
\curveto(815.01231795,274.67808424)(814.91231805,274.68308424)(814.81231934,274.6930896)
\curveto(814.72231824,274.71308421)(814.65731831,274.73808418)(814.61731934,274.7680896)
\curveto(814.57731839,274.79808412)(814.54731842,274.84808407)(814.52731934,274.9180896)
\curveto(814.50731846,274.98808393)(814.50731846,275.06308386)(814.52731934,275.1430896)
\curveto(814.55731841,275.27308365)(814.58731838,275.39308353)(814.61731934,275.5030896)
\curveto(814.65731831,275.6230833)(814.70231826,275.73808318)(814.75231934,275.8480896)
\curveto(814.94231802,276.19808272)(815.18231778,276.46808245)(815.47231934,276.6580896)
\curveto(815.7623172,276.85808206)(816.12231684,277.0180819)(816.55231934,277.1380896)
\curveto(816.65231631,277.15808176)(816.75231621,277.17308175)(816.85231934,277.1830896)
\curveto(816.962316,277.19308173)(817.07231589,277.20808171)(817.18231934,277.2280896)
\curveto(817.22231574,277.23808168)(817.28731568,277.23808168)(817.37731934,277.2280896)
\curveto(817.4673155,277.22808169)(817.52231544,277.23808168)(817.54231934,277.2580896)
\curveto(818.24231472,277.26808165)(818.85231411,277.18808173)(819.37231934,277.0180896)
\curveto(819.89231307,276.84808207)(820.25731271,276.5230824)(820.46731934,276.0430896)
\curveto(820.55731241,275.84308308)(820.60731236,275.60808331)(820.61731934,275.3380896)
\curveto(820.63731233,275.07808384)(820.64731232,274.80308412)(820.64731934,274.5130896)
\lineto(820.64731934,271.1980896)
\curveto(820.64731232,271.05808786)(820.65231231,270.923088)(820.66231934,270.7930896)
\curveto(820.67231229,270.66308826)(820.70231226,270.55808836)(820.75231934,270.4780896)
\curveto(820.80231216,270.40808851)(820.8673121,270.35808856)(820.94731934,270.3280896)
\curveto(821.03731193,270.28808863)(821.12231184,270.25808866)(821.20231934,270.2380896)
\curveto(821.28231168,270.22808869)(821.34231162,270.18308874)(821.38231934,270.1030896)
\curveto(821.40231156,270.07308885)(821.41231155,270.04308888)(821.41231934,270.0130896)
\curveto(821.41231155,269.98308894)(821.41731155,269.94308898)(821.42731934,269.8930896)
\moveto(819.28231934,271.5580896)
\curveto(819.34231362,271.69808722)(819.37231359,271.85808706)(819.37231934,272.0380896)
\curveto(819.38231358,272.22808669)(819.38731358,272.4230865)(819.38731934,272.6230896)
\curveto(819.38731358,272.73308619)(819.38231358,272.83308609)(819.37231934,272.9230896)
\curveto(819.3623136,273.01308591)(819.32231364,273.08308584)(819.25231934,273.1330896)
\curveto(819.22231374,273.15308577)(819.15231381,273.16308576)(819.04231934,273.1630896)
\curveto(819.02231394,273.14308578)(818.98731398,273.13308579)(818.93731934,273.1330896)
\curveto(818.88731408,273.13308579)(818.84231412,273.1230858)(818.80231934,273.1030896)
\curveto(818.72231424,273.08308584)(818.63231433,273.06308586)(818.53231934,273.0430896)
\lineto(818.23231934,272.9830896)
\curveto(818.20231476,272.98308594)(818.1673148,272.97808594)(818.12731934,272.9680896)
\lineto(818.02231934,272.9680896)
\curveto(817.87231509,272.92808599)(817.70731526,272.90308602)(817.52731934,272.8930896)
\curveto(817.35731561,272.89308603)(817.19731577,272.87308605)(817.04731934,272.8330896)
\curveto(816.967316,272.81308611)(816.89231607,272.79308613)(816.82231934,272.7730896)
\curveto(816.7623162,272.76308616)(816.69231627,272.74808617)(816.61231934,272.7280896)
\curveto(816.45231651,272.67808624)(816.30231666,272.61308631)(816.16231934,272.5330896)
\curveto(816.02231694,272.46308646)(815.90231706,272.37308655)(815.80231934,272.2630896)
\curveto(815.70231726,272.15308677)(815.62731734,272.0180869)(815.57731934,271.8580896)
\curveto(815.52731744,271.70808721)(815.50731746,271.5230874)(815.51731934,271.3030896)
\curveto(815.51731745,271.20308772)(815.53231743,271.10808781)(815.56231934,271.0180896)
\curveto(815.60231736,270.93808798)(815.64731732,270.86308806)(815.69731934,270.7930896)
\curveto(815.77731719,270.68308824)(815.88231708,270.58808833)(816.01231934,270.5080896)
\curveto(816.14231682,270.43808848)(816.28231668,270.37808854)(816.43231934,270.3280896)
\curveto(816.48231648,270.3180886)(816.53231643,270.31308861)(816.58231934,270.3130896)
\curveto(816.63231633,270.31308861)(816.68231628,270.30808861)(816.73231934,270.2980896)
\curveto(816.80231616,270.27808864)(816.88731608,270.26308866)(816.98731934,270.2530896)
\curveto(817.09731587,270.25308867)(817.18731578,270.26308866)(817.25731934,270.2830896)
\curveto(817.31731565,270.30308862)(817.37731559,270.30808861)(817.43731934,270.2980896)
\curveto(817.49731547,270.29808862)(817.55731541,270.30808861)(817.61731934,270.3280896)
\curveto(817.69731527,270.34808857)(817.77231519,270.36308856)(817.84231934,270.3730896)
\curveto(817.92231504,270.38308854)(817.99731497,270.40308852)(818.06731934,270.4330896)
\curveto(818.35731461,270.55308837)(818.60231436,270.69808822)(818.80231934,270.8680896)
\curveto(819.01231395,271.03808788)(819.17231379,271.26808765)(819.28231934,271.5580896)
}
}
{
\newrgbcolor{curcolor}{0 0 0}
\pscustom[linestyle=none,fillstyle=solid,fillcolor=curcolor]
{
\newpath
\moveto(825.02895996,277.2430896)
\curveto(825.7489559,277.25308167)(826.35395529,277.16808175)(826.84395996,276.9880896)
\curveto(827.33395431,276.8180821)(827.71395393,276.51308241)(827.98395996,276.0730896)
\curveto(828.05395359,275.96308296)(828.10895354,275.84808307)(828.14895996,275.7280896)
\curveto(828.18895346,275.6180833)(828.22895342,275.49308343)(828.26895996,275.3530896)
\curveto(828.28895336,275.28308364)(828.29395335,275.20808371)(828.28395996,275.1280896)
\curveto(828.27395337,275.05808386)(828.25895339,275.00308392)(828.23895996,274.9630896)
\curveto(828.21895343,274.94308398)(828.19395345,274.923084)(828.16395996,274.9030896)
\curveto(828.13395351,274.89308403)(828.10895354,274.87808404)(828.08895996,274.8580896)
\curveto(828.03895361,274.83808408)(827.98895366,274.83308409)(827.93895996,274.8430896)
\curveto(827.88895376,274.85308407)(827.83895381,274.85308407)(827.78895996,274.8430896)
\curveto(827.70895394,274.8230841)(827.60395404,274.8180841)(827.47395996,274.8280896)
\curveto(827.3439543,274.84808407)(827.25395439,274.87308405)(827.20395996,274.9030896)
\curveto(827.12395452,274.95308397)(827.06895458,275.0180839)(827.03895996,275.0980896)
\curveto(827.01895463,275.18808373)(826.98395466,275.27308365)(826.93395996,275.3530896)
\curveto(826.8439548,275.51308341)(826.71895493,275.65808326)(826.55895996,275.7880896)
\curveto(826.4489552,275.86808305)(826.32895532,275.92808299)(826.19895996,275.9680896)
\curveto(826.06895558,276.00808291)(825.92895572,276.04808287)(825.77895996,276.0880896)
\curveto(825.72895592,276.10808281)(825.67895597,276.11308281)(825.62895996,276.1030896)
\curveto(825.57895607,276.10308282)(825.52895612,276.10808281)(825.47895996,276.1180896)
\curveto(825.41895623,276.13808278)(825.3439563,276.14808277)(825.25395996,276.1480896)
\curveto(825.16395648,276.14808277)(825.08895656,276.13808278)(825.02895996,276.1180896)
\lineto(824.93895996,276.1180896)
\lineto(824.78895996,276.0880896)
\curveto(824.73895691,276.08808283)(824.68895696,276.08308284)(824.63895996,276.0730896)
\curveto(824.37895727,276.01308291)(824.16395748,275.92808299)(823.99395996,275.8180896)
\curveto(823.82395782,275.70808321)(823.70895794,275.5230834)(823.64895996,275.2630896)
\curveto(823.62895802,275.19308373)(823.62395802,275.1230838)(823.63395996,275.0530896)
\curveto(823.65395799,274.98308394)(823.67395797,274.923084)(823.69395996,274.8730896)
\curveto(823.75395789,274.7230842)(823.82395782,274.61308431)(823.90395996,274.5430896)
\curveto(823.99395765,274.48308444)(824.10395754,274.41308451)(824.23395996,274.3330896)
\curveto(824.39395725,274.23308469)(824.57395707,274.15808476)(824.77395996,274.1080896)
\curveto(824.97395667,274.06808485)(825.17395647,274.0180849)(825.37395996,273.9580896)
\curveto(825.50395614,273.918085)(825.63395601,273.88808503)(825.76395996,273.8680896)
\curveto(825.89395575,273.84808507)(826.02395562,273.8180851)(826.15395996,273.7780896)
\curveto(826.36395528,273.7180852)(826.56895508,273.65808526)(826.76895996,273.5980896)
\curveto(826.96895468,273.54808537)(827.16895448,273.48308544)(827.36895996,273.4030896)
\lineto(827.51895996,273.3430896)
\curveto(827.56895408,273.3230856)(827.61895403,273.29808562)(827.66895996,273.2680896)
\curveto(827.86895378,273.14808577)(828.0439536,273.01308591)(828.19395996,272.8630896)
\curveto(828.3439533,272.71308621)(828.46895318,272.5230864)(828.56895996,272.2930896)
\curveto(828.58895306,272.2230867)(828.60895304,272.12808679)(828.62895996,272.0080896)
\curveto(828.648953,271.93808698)(828.65895299,271.86308706)(828.65895996,271.7830896)
\curveto(828.66895298,271.71308721)(828.67395297,271.63308729)(828.67395996,271.5430896)
\lineto(828.67395996,271.3930896)
\curveto(828.65395299,271.3230876)(828.643953,271.25308767)(828.64395996,271.1830896)
\curveto(828.643953,271.11308781)(828.63395301,271.04308788)(828.61395996,270.9730896)
\curveto(828.58395306,270.86308806)(828.5489531,270.75808816)(828.50895996,270.6580896)
\curveto(828.46895318,270.55808836)(828.42395322,270.46808845)(828.37395996,270.3880896)
\curveto(828.21395343,270.12808879)(828.00895364,269.918089)(827.75895996,269.7580896)
\curveto(827.50895414,269.60808931)(827.22895442,269.47808944)(826.91895996,269.3680896)
\curveto(826.82895482,269.33808958)(826.73395491,269.3180896)(826.63395996,269.3080896)
\curveto(826.5439551,269.28808963)(826.45395519,269.26308966)(826.36395996,269.2330896)
\curveto(826.26395538,269.21308971)(826.16395548,269.20308972)(826.06395996,269.2030896)
\curveto(825.96395568,269.20308972)(825.86395578,269.19308973)(825.76395996,269.1730896)
\lineto(825.61395996,269.1730896)
\curveto(825.56395608,269.16308976)(825.49395615,269.15808976)(825.40395996,269.1580896)
\curveto(825.31395633,269.15808976)(825.2439564,269.16308976)(825.19395996,269.1730896)
\lineto(825.02895996,269.1730896)
\curveto(824.96895668,269.19308973)(824.90395674,269.20308972)(824.83395996,269.2030896)
\curveto(824.76395688,269.19308973)(824.70395694,269.19808972)(824.65395996,269.2180896)
\curveto(824.60395704,269.22808969)(824.53895711,269.23308969)(824.45895996,269.2330896)
\lineto(824.21895996,269.2930896)
\curveto(824.1489575,269.30308962)(824.07395757,269.3230896)(823.99395996,269.3530896)
\curveto(823.68395796,269.45308947)(823.41395823,269.57808934)(823.18395996,269.7280896)
\curveto(822.95395869,269.87808904)(822.75395889,270.07308885)(822.58395996,270.3130896)
\curveto(822.49395915,270.44308848)(822.41895923,270.57808834)(822.35895996,270.7180896)
\curveto(822.29895935,270.85808806)(822.2439594,271.01308791)(822.19395996,271.1830896)
\curveto(822.17395947,271.24308768)(822.16395948,271.31308761)(822.16395996,271.3930896)
\curveto(822.17395947,271.48308744)(822.18895946,271.55308737)(822.20895996,271.6030896)
\curveto(822.23895941,271.64308728)(822.28895936,271.68308724)(822.35895996,271.7230896)
\curveto(822.40895924,271.74308718)(822.47895917,271.75308717)(822.56895996,271.7530896)
\curveto(822.65895899,271.76308716)(822.7489589,271.76308716)(822.83895996,271.7530896)
\curveto(822.92895872,271.74308718)(823.01395863,271.72808719)(823.09395996,271.7080896)
\curveto(823.18395846,271.69808722)(823.2439584,271.68308724)(823.27395996,271.6630896)
\curveto(823.3439583,271.61308731)(823.38895826,271.53808738)(823.40895996,271.4380896)
\curveto(823.43895821,271.34808757)(823.47395817,271.26308766)(823.51395996,271.1830896)
\curveto(823.61395803,270.96308796)(823.7489579,270.79308813)(823.91895996,270.6730896)
\curveto(824.03895761,270.58308834)(824.17395747,270.51308841)(824.32395996,270.4630896)
\curveto(824.47395717,270.41308851)(824.63395701,270.36308856)(824.80395996,270.3130896)
\lineto(825.11895996,270.2680896)
\lineto(825.20895996,270.2680896)
\curveto(825.27895637,270.24808867)(825.36895628,270.23808868)(825.47895996,270.2380896)
\curveto(825.59895605,270.23808868)(825.69895595,270.24808867)(825.77895996,270.2680896)
\curveto(825.8489558,270.26808865)(825.90395574,270.27308865)(825.94395996,270.2830896)
\curveto(826.00395564,270.29308863)(826.06395558,270.29808862)(826.12395996,270.2980896)
\curveto(826.18395546,270.30808861)(826.23895541,270.3180886)(826.28895996,270.3280896)
\curveto(826.57895507,270.40808851)(826.80895484,270.51308841)(826.97895996,270.6430896)
\curveto(827.1489545,270.77308815)(827.26895438,270.99308793)(827.33895996,271.3030896)
\curveto(827.35895429,271.35308757)(827.36395428,271.40808751)(827.35395996,271.4680896)
\curveto(827.3439543,271.52808739)(827.33395431,271.57308735)(827.32395996,271.6030896)
\curveto(827.27395437,271.79308713)(827.20395444,271.93308699)(827.11395996,272.0230896)
\curveto(827.02395462,272.1230868)(826.90895474,272.21308671)(826.76895996,272.2930896)
\curveto(826.67895497,272.35308657)(826.57895507,272.40308652)(826.46895996,272.4430896)
\lineto(826.13895996,272.5630896)
\curveto(826.10895554,272.57308635)(826.07895557,272.57808634)(826.04895996,272.5780896)
\curveto(826.02895562,272.57808634)(826.00395564,272.58808633)(825.97395996,272.6080896)
\curveto(825.63395601,272.7180862)(825.27895637,272.79808612)(824.90895996,272.8480896)
\curveto(824.5489571,272.90808601)(824.20895744,273.00308592)(823.88895996,273.1330896)
\curveto(823.78895786,273.17308575)(823.69395795,273.20808571)(823.60395996,273.2380896)
\curveto(823.51395813,273.26808565)(823.42895822,273.30808561)(823.34895996,273.3580896)
\curveto(823.15895849,273.46808545)(822.98395866,273.59308533)(822.82395996,273.7330896)
\curveto(822.66395898,273.87308505)(822.53895911,274.04808487)(822.44895996,274.2580896)
\curveto(822.41895923,274.32808459)(822.39395925,274.39808452)(822.37395996,274.4680896)
\curveto(822.36395928,274.53808438)(822.3489593,274.61308431)(822.32895996,274.6930896)
\curveto(822.29895935,274.81308411)(822.28895936,274.94808397)(822.29895996,275.0980896)
\curveto(822.30895934,275.25808366)(822.32395932,275.39308353)(822.34395996,275.5030896)
\curveto(822.36395928,275.55308337)(822.37395927,275.59308333)(822.37395996,275.6230896)
\curveto(822.38395926,275.66308326)(822.39895925,275.70308322)(822.41895996,275.7430896)
\curveto(822.50895914,275.97308295)(822.62895902,276.17308275)(822.77895996,276.3430896)
\curveto(822.93895871,276.51308241)(823.11895853,276.66308226)(823.31895996,276.7930896)
\curveto(823.46895818,276.88308204)(823.63395801,276.95308197)(823.81395996,277.0030896)
\curveto(823.99395765,277.06308186)(824.18395746,277.1180818)(824.38395996,277.1680896)
\curveto(824.45395719,277.17808174)(824.51895713,277.18808173)(824.57895996,277.1980896)
\curveto(824.648957,277.20808171)(824.72395692,277.2180817)(824.80395996,277.2280896)
\curveto(824.83395681,277.23808168)(824.87395677,277.23808168)(824.92395996,277.2280896)
\curveto(824.97395667,277.2180817)(825.00895664,277.2230817)(825.02895996,277.2430896)
}
}
{
\newrgbcolor{curcolor}{0.40000001 0.40000001 0.40000001}
\pscustom[linestyle=none,fillstyle=solid,fillcolor=curcolor]
{
\newpath
\moveto(747.9732666,280.03309631)
\lineto(762.9732666,280.03309631)
\lineto(762.9732666,265.03309631)
\lineto(747.9732666,265.03309631)
\closepath
}
}
{
\newrgbcolor{curcolor}{0.7019608 0.7019608 0.7019608}
\pscustom[linestyle=none,fillstyle=solid,fillcolor=curcolor]
{
\newpath
\moveto(554.98260498,192.00756836)
\lineto(568.02885151,192.00756836)
\lineto(568.02885151,76.08277893)
\lineto(554.98260498,76.08277893)
\closepath
}
}
{
\newrgbcolor{curcolor}{0.60000002 0.60000002 0.60000002}
\pscustom[linestyle=none,fillstyle=solid,fillcolor=curcolor]
{
\newpath
\moveto(567.97570801,77.18908691)
\lineto(581.02195454,77.18908691)
\lineto(581.02195454,76.08278656)
\lineto(567.97570801,76.08278656)
\closepath
}
}
{
\newrgbcolor{curcolor}{0.50196081 0.50196081 0.50196081}
\pscustom[linestyle=none,fillstyle=solid,fillcolor=curcolor]
{
\newpath
\moveto(580.96881104,76.50158691)
\lineto(594.01505756,76.50158691)
\lineto(594.01505756,76.08278662)
\lineto(580.96881104,76.08278662)
\closepath
}
}
{
\newrgbcolor{curcolor}{0.40000001 0.40000001 0.40000001}
\pscustom[linestyle=none,fillstyle=solid,fillcolor=curcolor]
{
\newpath
\moveto(593.96191406,78.93908691)
\lineto(607.00816059,78.93908691)
\lineto(607.00816059,76.08278656)
\lineto(593.96191406,76.08278656)
\closepath
}
}
{
\newrgbcolor{curcolor}{0.80000001 0.80000001 0.80000001}
\pscustom[linestyle=none,fillstyle=solid,fillcolor=curcolor]
{
\newpath
\moveto(114.89689636,87.00222778)
\lineto(127.94314289,87.00222778)
\lineto(127.94314289,76.0827961)
\lineto(114.89689636,76.0827961)
\closepath
}
}
{
\newrgbcolor{curcolor}{0.7019608 0.7019608 0.7019608}
\pscustom[linestyle=none,fillstyle=solid,fillcolor=curcolor]
{
\newpath
\moveto(127.88999939,315.98626709)
\lineto(140.93624592,315.98626709)
\lineto(140.93624592,76.08279419)
\lineto(127.88999939,76.08279419)
\closepath
}
}
{
\newrgbcolor{curcolor}{0.60000002 0.60000002 0.60000002}
\pscustom[linestyle=none,fillstyle=solid,fillcolor=curcolor]
{
\newpath
\moveto(140.88310242,80.00158691)
\lineto(153.92934895,80.00158691)
\lineto(153.92934895,76.08278656)
\lineto(140.88310242,76.08278656)
\closepath
}
}
{
\newrgbcolor{curcolor}{0.50196081 0.50196081 0.50196081}
\pscustom[linestyle=none,fillstyle=solid,fillcolor=curcolor]
{
\newpath
\moveto(153.87620544,80.93908691)
\lineto(166.92245197,80.93908691)
\lineto(166.92245197,76.08278656)
\lineto(153.87620544,76.08278656)
\closepath
}
}
{
\newrgbcolor{curcolor}{0.40000001 0.40000001 0.40000001}
\pscustom[linestyle=none,fillstyle=solid,fillcolor=curcolor]
{
\newpath
\moveto(166.86930847,93.00158691)
\lineto(179.915555,93.00158691)
\lineto(179.915555,76.08278656)
\lineto(166.86930847,76.08278656)
\closepath
}
}
{
\newrgbcolor{curcolor}{0.80000001 0.80000001 0.80000001}
\pscustom[linestyle=none,fillstyle=solid,fillcolor=curcolor]
{
\newpath
\moveto(222.05667114,79.00308228)
\lineto(235.10291767,79.00308228)
\lineto(235.10291767,76.0827961)
\lineto(222.05667114,76.0827961)
\closepath
}
}
{
\newrgbcolor{curcolor}{0.7019608 0.7019608 0.7019608}
\pscustom[linestyle=none,fillstyle=solid,fillcolor=curcolor]
{
\newpath
\moveto(235.04977417,96.01782227)
\lineto(248.0960207,96.01782227)
\lineto(248.0960207,76.08277893)
\lineto(235.04977417,76.08277893)
\closepath
}
}
{
\newrgbcolor{curcolor}{0.60000002 0.60000002 0.60000002}
\pscustom[linestyle=none,fillstyle=solid,fillcolor=curcolor]
{
\newpath
\moveto(248.0428772,78.03079224)
\lineto(261.08912373,78.03079224)
\lineto(261.08912373,76.08277786)
\lineto(248.0428772,76.08277786)
\closepath
}
}
{
\newrgbcolor{curcolor}{0.50196081 0.50196081 0.50196081}
\pscustom[linestyle=none,fillstyle=solid,fillcolor=curcolor]
{
\newpath
\moveto(261.03598022,76.8817749)
\lineto(274.08222675,76.8817749)
\lineto(274.08222675,76.08280909)
\lineto(261.03598022,76.08280909)
\closepath
}
}
{
\newrgbcolor{curcolor}{0.40000001 0.40000001 0.40000001}
\pscustom[linestyle=none,fillstyle=solid,fillcolor=curcolor]
{
\newpath
\moveto(274.02908325,79.04727173)
\lineto(287.07532978,79.04727173)
\lineto(287.07532978,76.08279133)
\lineto(274.02908325,76.08279133)
\closepath
}
}
{
\newrgbcolor{curcolor}{0.80000001 0.80000001 0.80000001}
\pscustom[linestyle=none,fillstyle=solid,fillcolor=curcolor]
{
\newpath
\moveto(328.92700195,77.00158691)
\lineto(341.97324848,77.00158691)
\lineto(341.97324848,76.08278662)
\lineto(328.92700195,76.08278662)
\closepath
}
}
{
\newrgbcolor{curcolor}{0.7019608 0.7019608 0.7019608}
\pscustom[linestyle=none,fillstyle=solid,fillcolor=curcolor]
{
\newpath
\moveto(341.92010498,147.06210327)
\lineto(354.96635151,147.06210327)
\lineto(354.96635151,76.08279419)
\lineto(341.92010498,76.08279419)
\closepath
}
}
{
\newrgbcolor{curcolor}{0.60000002 0.60000002 0.60000002}
\pscustom[linestyle=none,fillstyle=solid,fillcolor=curcolor]
{
\newpath
\moveto(354.91320801,77.01434326)
\lineto(367.95945454,77.01434326)
\lineto(367.95945454,76.0827949)
\lineto(354.91320801,76.0827949)
\closepath
}
}
{
\newrgbcolor{curcolor}{0.40000001 0.40000001 0.40000001}
\pscustom[linestyle=none,fillstyle=solid,fillcolor=curcolor]
{
\newpath
\moveto(380.89941406,77.10272217)
\lineto(393.94566059,77.10272217)
\lineto(393.94566059,76.08278549)
\lineto(380.89941406,76.08278549)
\closepath
}
}
{
\newrgbcolor{curcolor}{0.80000001 0.80000001 0.80000001}
\pscustom[linestyle=none,fillstyle=solid,fillcolor=curcolor]
{
\newpath
\moveto(434.98950195,77.00158691)
\lineto(448.03574848,77.00158691)
\lineto(448.03574848,76.08278662)
\lineto(434.98950195,76.08278662)
\closepath
}
}
{
\newrgbcolor{curcolor}{0.7019608 0.7019608 0.7019608}
\pscustom[linestyle=none,fillstyle=solid,fillcolor=curcolor]
{
\newpath
\moveto(447.98260498,78.06408691)
\lineto(461.02885151,78.06408691)
\lineto(461.02885151,76.08278656)
\lineto(447.98260498,76.08278656)
\closepath
}
}
{
\newrgbcolor{curcolor}{0.60000002 0.60000002 0.60000002}
\pscustom[linestyle=none,fillstyle=solid,fillcolor=curcolor]
{
\newpath
\moveto(460.97570801,76.87658691)
\lineto(474.02195454,76.87658691)
\lineto(474.02195454,76.08278662)
\lineto(460.97570801,76.08278662)
\closepath
}
}
{
\newrgbcolor{curcolor}{0.40000001 0.40000001 0.40000001}
\pscustom[linestyle=none,fillstyle=solid,fillcolor=curcolor]
{
\newpath
\moveto(486.96191406,78.00158691)
\lineto(500.00816059,78.00158691)
\lineto(500.00816059,76.08278656)
\lineto(486.96191406,76.08278656)
\closepath
}
}
{
\newrgbcolor{curcolor}{0.80000001 0.80000001 0.80000001}
\pscustom[linestyle=none,fillstyle=solid,fillcolor=curcolor]
{
\newpath
\moveto(648.92700195,77.06408691)
\lineto(661.97324848,77.06408691)
\lineto(661.97324848,76.08278662)
\lineto(648.92700195,76.08278662)
\closepath
}
}
{
\newrgbcolor{curcolor}{0.7019608 0.7019608 0.7019608}
\pscustom[linestyle=none,fillstyle=solid,fillcolor=curcolor]
{
\newpath
\moveto(661.92010498,77.06408691)
\lineto(674.96635151,77.06408691)
\lineto(674.96635151,76.08278662)
\lineto(661.92010498,76.08278662)
\closepath
}
}
{
\newrgbcolor{curcolor}{0.40000001 0.40000001 0.40000001}
\pscustom[linestyle=none,fillstyle=solid,fillcolor=curcolor]
{
\newpath
\moveto(700.89941406,77.06408691)
\lineto(713.94566059,77.06408691)
\lineto(713.94566059,76.08278662)
\lineto(700.89941406,76.08278662)
\closepath
}
}
\end{pspicture}

\caption{Diagrama de barras de los recursos y sus niveles de repercusión}
\label{recursos_bars_1}
\end{figure}

Puede verse en la figura \ref{recursos_pie_1}, las gráficas circulares relativas
a la distribución de los recursos sobre las diferentes variables a tomadas en
cuenta, es notorio como el uso de calificadores aún no satisface las respectivas
que se tienen.

\begin{figure}
\centering
%LaTeX with PSTricks extensions
%%Creator: inkscape 0.48.5
%%Please note this file requires PSTricks extensions
\psset{xunit=.5pt,yunit=.5pt,runit=.5pt}
\begin{pspicture}(923,727)
{
\newrgbcolor{curcolor}{0 0 0}
\pscustom[linestyle=none,fillstyle=solid,fillcolor=curcolor]
{
\newpath
\moveto(28.32111654,703.11942034)
\curveto(28.32110606,703.08941467)(28.32110606,703.04941471)(28.32111654,702.99942034)
\curveto(28.33110605,702.94941481)(28.33610604,702.89441487)(28.33611654,702.83442034)
\curveto(28.33610604,702.77441499)(28.33110605,702.71941504)(28.32111654,702.66942034)
\curveto(28.32110606,702.61941514)(28.32110606,702.58441518)(28.32111654,702.56442034)
\curveto(28.32110606,702.49441527)(28.31610606,702.42441534)(28.30611654,702.35442034)
\curveto(28.30610607,702.29441547)(28.30610607,702.23441553)(28.30611654,702.17442034)
\curveto(28.28610609,702.12441564)(28.2761061,702.07441569)(28.27611654,702.02442034)
\curveto(28.28610609,701.97441579)(28.28610609,701.92441584)(28.27611654,701.87442034)
\curveto(28.25610612,701.764416)(28.24110614,701.65441611)(28.23111654,701.54442034)
\curveto(28.22110616,701.43441633)(28.20110618,701.32441644)(28.17111654,701.21442034)
\curveto(28.12110626,701.04441672)(28.0761063,700.87941688)(28.03611654,700.71942034)
\curveto(27.99610638,700.56941719)(27.94610643,700.41941734)(27.88611654,700.26942034)
\curveto(27.71610666,699.84941791)(27.50610687,699.46941829)(27.25611654,699.12942034)
\curveto(27.00610737,698.78941897)(26.70610767,698.49941926)(26.35611654,698.25942034)
\curveto(26.15610822,698.11941964)(25.94610843,697.99941976)(25.72611654,697.89942034)
\curveto(25.51610886,697.79941996)(25.28610909,697.70942005)(25.03611654,697.62942034)
\curveto(24.93610944,697.59942016)(24.83110955,697.57442019)(24.72111654,697.55442034)
\curveto(24.62110976,697.54442022)(24.51610986,697.52442024)(24.40611654,697.49442034)
\curveto(24.35611002,697.48442028)(24.30611007,697.47942028)(24.25611654,697.47942034)
\curveto(24.21611016,697.47942028)(24.17111021,697.47442029)(24.12111654,697.46442034)
\curveto(24.0811103,697.45442031)(24.04111034,697.44942031)(24.00111654,697.44942034)
\curveto(23.96111042,697.4594203)(23.91611046,697.4594203)(23.86611654,697.44942034)
\curveto(23.84611053,697.43942032)(23.81611056,697.43442033)(23.77611654,697.43442034)
\curveto(23.73611064,697.44442032)(23.70611067,697.44442032)(23.68611654,697.43442034)
\curveto(23.60611077,697.41442035)(23.50611087,697.40942035)(23.38611654,697.41942034)
\curveto(23.26611111,697.42942033)(23.16111122,697.43442033)(23.07111654,697.43442034)
\lineto(19.57611654,697.43442034)
\curveto(19.40611497,697.43442033)(19.26111512,697.43942032)(19.14111654,697.44942034)
\curveto(19.03111535,697.46942029)(18.95111543,697.53942022)(18.90111654,697.65942034)
\curveto(18.87111551,697.73942002)(18.85611552,697.8594199)(18.85611654,698.01942034)
\curveto(18.86611551,698.18941957)(18.87111551,698.32941943)(18.87111654,698.43942034)
\lineto(18.87111654,707.24442034)
\curveto(18.87111551,707.3644104)(18.86611551,707.48941027)(18.85611654,707.61942034)
\curveto(18.85611552,707.75941)(18.8811155,707.86940989)(18.93111654,707.94942034)
\curveto(18.97111541,708.00940975)(19.04611533,708.0594097)(19.15611654,708.09942034)
\curveto(19.1761152,708.10940965)(19.19611518,708.10940965)(19.21611654,708.09942034)
\curveto(19.23611514,708.09940966)(19.25611512,708.10440966)(19.27611654,708.11442034)
\lineto(23.31111654,708.11442034)
\curveto(23.37111101,708.11440965)(23.43111095,708.11440965)(23.49111654,708.11442034)
\curveto(23.56111082,708.12440964)(23.62111076,708.12440964)(23.67111654,708.11442034)
\lineto(23.85111654,708.11442034)
\curveto(23.90111048,708.09440967)(23.95611042,708.08440968)(24.01611654,708.08442034)
\curveto(24.0761103,708.09440967)(24.13111025,708.08940967)(24.18111654,708.06942034)
\curveto(24.24111014,708.04940971)(24.29611008,708.03940972)(24.34611654,708.03942034)
\curveto(24.40610997,708.04940971)(24.46610991,708.04440972)(24.52611654,708.02442034)
\curveto(24.66610971,707.99440977)(24.80110958,707.9644098)(24.93111654,707.93442034)
\curveto(25.06110932,707.91440985)(25.18610919,707.87940988)(25.30611654,707.82942034)
\curveto(25.41610896,707.77940998)(25.52610885,707.73441003)(25.63611654,707.69442034)
\curveto(25.74610863,707.65441011)(25.85110853,707.60441016)(25.95111654,707.54442034)
\curveto(26.20110818,707.38441038)(26.43110795,707.22941053)(26.64111654,707.07942034)
\lineto(26.73111654,706.98942034)
\curveto(26.83110755,706.90941085)(26.92110746,706.81941094)(27.00111654,706.71942034)
\lineto(27.13611654,706.59942034)
\curveto(27.18610719,706.51941124)(27.24110714,706.43941132)(27.30111654,706.35942034)
\curveto(27.37110701,706.28941147)(27.43110695,706.21441155)(27.48111654,706.13442034)
\curveto(27.61110677,705.92441184)(27.72610665,705.69941206)(27.82611654,705.45942034)
\curveto(27.92610645,705.22941253)(28.01610636,704.98441278)(28.09611654,704.72442034)
\curveto(28.14610623,704.59441317)(28.1761062,704.4594133)(28.18611654,704.31942034)
\curveto(28.20610617,704.17941358)(28.23110615,704.03941372)(28.26111654,703.89942034)
\curveto(28.26110612,703.84941391)(28.26110612,703.80441396)(28.26111654,703.76442034)
\curveto(28.27110611,703.73441403)(28.2761061,703.69941406)(28.27611654,703.65942034)
\curveto(28.29610608,703.59941416)(28.30110608,703.53441423)(28.29111654,703.46442034)
\curveto(28.29110609,703.39441437)(28.30110608,703.33441443)(28.32111654,703.28442034)
\lineto(28.32111654,703.11942034)
\moveto(25.98111654,702.39942034)
\curveto(26.00110838,702.44941531)(26.01110837,702.52941523)(26.01111654,702.63942034)
\curveto(26.01110837,702.74941501)(26.00110838,702.82941493)(25.98111654,702.87942034)
\lineto(25.98111654,703.16442034)
\curveto(25.96110842,703.25441451)(25.94610843,703.34941441)(25.93611654,703.44942034)
\curveto(25.93610844,703.54941421)(25.92610845,703.63941412)(25.90611654,703.71942034)
\curveto(25.88610849,703.76941399)(25.8761085,703.81441395)(25.87611654,703.85442034)
\curveto(25.88610849,703.90441386)(25.8811085,703.95441381)(25.86111654,704.00442034)
\curveto(25.81110857,704.1644136)(25.76110862,704.31441345)(25.71111654,704.45442034)
\curveto(25.67110871,704.60441316)(25.61110877,704.74441302)(25.53111654,704.87442034)
\curveto(25.381109,705.11441265)(25.20610917,705.31941244)(25.00611654,705.48942034)
\curveto(24.81610956,705.66941209)(24.5811098,705.81941194)(24.30111654,705.93942034)
\curveto(24.21111017,705.96941179)(24.12111026,705.99441177)(24.03111654,706.01442034)
\curveto(23.94111044,706.04441172)(23.85111053,706.06941169)(23.76111654,706.08942034)
\curveto(23.6811107,706.09941166)(23.60611077,706.10441166)(23.53611654,706.10442034)
\curveto(23.4761109,706.11441165)(23.40611097,706.12941163)(23.32611654,706.14942034)
\curveto(23.28611109,706.1594116)(23.24611113,706.1594116)(23.20611654,706.14942034)
\curveto(23.16611121,706.14941161)(23.13111125,706.15441161)(23.10111654,706.16442034)
\lineto(22.77111654,706.16442034)
\curveto(22.72111166,706.17441159)(22.66611171,706.17441159)(22.60611654,706.16442034)
\lineto(22.42611654,706.16442034)
\lineto(21.75111654,706.16442034)
\curveto(21.73111265,706.14441162)(21.69611268,706.13941162)(21.64611654,706.14942034)
\curveto(21.60611277,706.1594116)(21.57111281,706.1594116)(21.54111654,706.14942034)
\lineto(21.39111654,706.08942034)
\curveto(21.34111304,706.07941168)(21.30111308,706.04941171)(21.27111654,705.99942034)
\curveto(21.23111315,705.94941181)(21.21111317,705.87941188)(21.21111654,705.78942034)
\lineto(21.21111654,705.48942034)
\curveto(21.21111317,705.3594124)(21.20611317,705.22441254)(21.19611654,705.08442034)
\lineto(21.19611654,704.66442034)
\lineto(21.19611654,700.47942034)
\curveto(21.19611318,700.41941734)(21.19111319,700.35441741)(21.18111654,700.28442034)
\curveto(21.1811132,700.21441755)(21.19111319,700.15441761)(21.21111654,700.10442034)
\lineto(21.21111654,699.95442034)
\lineto(21.21111654,699.74442034)
\curveto(21.22111316,699.68441808)(21.23611314,699.62941813)(21.25611654,699.57942034)
\curveto(21.31611306,699.4594183)(21.43111295,699.39441837)(21.60111654,699.38442034)
\lineto(22.12611654,699.38442034)
\lineto(23.31111654,699.38442034)
\curveto(23.71111067,699.39441837)(24.05111033,699.45441831)(24.33111654,699.56442034)
\curveto(24.70110968,699.71441805)(24.99110939,699.91441785)(25.20111654,700.16442034)
\curveto(25.42110896,700.41441735)(25.60610877,700.72441704)(25.75611654,701.09442034)
\curveto(25.79610858,701.17441659)(25.82610855,701.2644165)(25.84611654,701.36442034)
\curveto(25.86610851,701.4644163)(25.89110849,701.5644162)(25.92111654,701.66442034)
\lineto(25.92111654,701.78442034)
\curveto(25.94110844,701.85441591)(25.95110843,701.92941583)(25.95111654,702.00942034)
\curveto(25.95110843,702.08941567)(25.96110842,702.16941559)(25.98111654,702.24942034)
\lineto(25.98111654,702.39942034)
}
}
{
\newrgbcolor{curcolor}{0 0 0}
\pscustom[linestyle=none,fillstyle=solid,fillcolor=curcolor]
{
\newpath
\moveto(31.81963217,708.00942034)
\curveto(31.88962922,707.92940983)(31.92462918,707.80940995)(31.92463217,707.64942034)
\lineto(31.92463217,707.18442034)
\lineto(31.92463217,706.77942034)
\curveto(31.92462918,706.63941112)(31.88962922,706.54441122)(31.81963217,706.49442034)
\curveto(31.75962935,706.44441132)(31.67962943,706.41441135)(31.57963217,706.40442034)
\curveto(31.48962962,706.39441137)(31.38962972,706.38941137)(31.27963217,706.38942034)
\lineto(30.43963217,706.38942034)
\curveto(30.32963078,706.38941137)(30.22963088,706.39441137)(30.13963217,706.40442034)
\curveto(30.05963105,706.41441135)(29.98963112,706.44441132)(29.92963217,706.49442034)
\curveto(29.88963122,706.52441124)(29.85963125,706.57941118)(29.83963217,706.65942034)
\curveto(29.82963128,706.74941101)(29.81963129,706.84441092)(29.80963217,706.94442034)
\lineto(29.80963217,707.27442034)
\curveto(29.81963129,707.38441038)(29.82463128,707.47941028)(29.82463217,707.55942034)
\lineto(29.82463217,707.76942034)
\curveto(29.83463127,707.83940992)(29.85463125,707.89940986)(29.88463217,707.94942034)
\curveto(29.9046312,707.98940977)(29.92963118,708.01940974)(29.95963217,708.03942034)
\lineto(30.07963217,708.09942034)
\curveto(30.09963101,708.09940966)(30.12463098,708.09940966)(30.15463217,708.09942034)
\curveto(30.18463092,708.10940965)(30.2096309,708.11440965)(30.22963217,708.11442034)
\lineto(31.32463217,708.11442034)
\curveto(31.42462968,708.11440965)(31.51962959,708.10940965)(31.60963217,708.09942034)
\curveto(31.69962941,708.08940967)(31.76962934,708.0594097)(31.81963217,708.00942034)
\moveto(31.92463217,698.24442034)
\curveto(31.92462918,698.04441972)(31.91962919,697.87441989)(31.90963217,697.73442034)
\curveto(31.89962921,697.59442017)(31.8096293,697.49942026)(31.63963217,697.44942034)
\curveto(31.57962953,697.42942033)(31.51462959,697.41942034)(31.44463217,697.41942034)
\curveto(31.37462973,697.42942033)(31.29962981,697.43442033)(31.21963217,697.43442034)
\lineto(30.37963217,697.43442034)
\curveto(30.28963082,697.43442033)(30.19963091,697.43942032)(30.10963217,697.44942034)
\curveto(30.02963108,697.4594203)(29.96963114,697.48942027)(29.92963217,697.53942034)
\curveto(29.86963124,697.60942015)(29.83463127,697.69442007)(29.82463217,697.79442034)
\lineto(29.82463217,698.13942034)
\lineto(29.82463217,704.46942034)
\lineto(29.82463217,704.76942034)
\curveto(29.82463128,704.86941289)(29.84463126,704.94941281)(29.88463217,705.00942034)
\curveto(29.94463116,705.07941268)(30.02963108,705.12441264)(30.13963217,705.14442034)
\curveto(30.15963095,705.15441261)(30.18463092,705.15441261)(30.21463217,705.14442034)
\curveto(30.25463085,705.14441262)(30.28463082,705.14941261)(30.30463217,705.15942034)
\lineto(31.05463217,705.15942034)
\lineto(31.24963217,705.15942034)
\curveto(31.32962978,705.16941259)(31.39462971,705.16941259)(31.44463217,705.15942034)
\lineto(31.56463217,705.15942034)
\curveto(31.62462948,705.13941262)(31.67962943,705.12441264)(31.72963217,705.11442034)
\curveto(31.77962933,705.10441266)(31.81962929,705.07441269)(31.84963217,705.02442034)
\curveto(31.88962922,704.97441279)(31.9096292,704.90441286)(31.90963217,704.81442034)
\curveto(31.91962919,704.72441304)(31.92462918,704.62941313)(31.92463217,704.52942034)
\lineto(31.92463217,698.24442034)
}
}
{
\newrgbcolor{curcolor}{0 0 0}
\pscustom[linestyle=none,fillstyle=solid,fillcolor=curcolor]
{
\newpath
\moveto(36.55681967,705.36942034)
\curveto(37.30681517,705.38941237)(37.95681452,705.30441246)(38.50681967,705.11442034)
\curveto(39.06681341,704.93441283)(39.49181298,704.61941314)(39.78181967,704.16942034)
\curveto(39.85181262,704.0594137)(39.91181256,703.94441382)(39.96181967,703.82442034)
\curveto(40.02181245,703.71441405)(40.0718124,703.58941417)(40.11181967,703.44942034)
\curveto(40.13181234,703.38941437)(40.14181233,703.32441444)(40.14181967,703.25442034)
\curveto(40.14181233,703.18441458)(40.13181234,703.12441464)(40.11181967,703.07442034)
\curveto(40.0718124,703.01441475)(40.01681246,702.97441479)(39.94681967,702.95442034)
\curveto(39.89681258,702.93441483)(39.83681264,702.92441484)(39.76681967,702.92442034)
\lineto(39.55681967,702.92442034)
\lineto(38.89681967,702.92442034)
\curveto(38.82681365,702.92441484)(38.75681372,702.91941484)(38.68681967,702.90942034)
\curveto(38.61681386,702.90941485)(38.55181392,702.91941484)(38.49181967,702.93942034)
\curveto(38.39181408,702.9594148)(38.31681416,702.99941476)(38.26681967,703.05942034)
\curveto(38.21681426,703.11941464)(38.1718143,703.17941458)(38.13181967,703.23942034)
\lineto(38.01181967,703.44942034)
\curveto(37.98181449,703.52941423)(37.93181454,703.59441417)(37.86181967,703.64442034)
\curveto(37.76181471,703.72441404)(37.66181481,703.78441398)(37.56181967,703.82442034)
\curveto(37.471815,703.8644139)(37.35681512,703.89941386)(37.21681967,703.92942034)
\curveto(37.14681533,703.94941381)(37.04181543,703.9644138)(36.90181967,703.97442034)
\curveto(36.7718157,703.98441378)(36.6718158,703.97941378)(36.60181967,703.95942034)
\lineto(36.49681967,703.95942034)
\lineto(36.34681967,703.92942034)
\curveto(36.30681617,703.92941383)(36.26181621,703.92441384)(36.21181967,703.91442034)
\curveto(36.04181643,703.8644139)(35.90181657,703.79441397)(35.79181967,703.70442034)
\curveto(35.69181678,703.62441414)(35.62181685,703.49941426)(35.58181967,703.32942034)
\curveto(35.56181691,703.2594145)(35.56181691,703.19441457)(35.58181967,703.13442034)
\curveto(35.60181687,703.07441469)(35.62181685,703.02441474)(35.64181967,702.98442034)
\curveto(35.71181676,702.8644149)(35.79181668,702.76941499)(35.88181967,702.69942034)
\curveto(35.98181649,702.62941513)(36.09681638,702.56941519)(36.22681967,702.51942034)
\curveto(36.41681606,702.43941532)(36.62181585,702.36941539)(36.84181967,702.30942034)
\lineto(37.53181967,702.15942034)
\curveto(37.7718147,702.11941564)(38.00181447,702.06941569)(38.22181967,702.00942034)
\curveto(38.45181402,701.9594158)(38.66681381,701.89441587)(38.86681967,701.81442034)
\curveto(38.95681352,701.77441599)(39.04181343,701.73941602)(39.12181967,701.70942034)
\curveto(39.21181326,701.68941607)(39.29681318,701.65441611)(39.37681967,701.60442034)
\curveto(39.56681291,701.48441628)(39.73681274,701.35441641)(39.88681967,701.21442034)
\curveto(40.04681243,701.07441669)(40.1718123,700.89941686)(40.26181967,700.68942034)
\curveto(40.29181218,700.61941714)(40.31681216,700.54941721)(40.33681967,700.47942034)
\curveto(40.35681212,700.40941735)(40.3768121,700.33441743)(40.39681967,700.25442034)
\curveto(40.40681207,700.19441757)(40.41181206,700.09941766)(40.41181967,699.96942034)
\curveto(40.42181205,699.84941791)(40.42181205,699.75441801)(40.41181967,699.68442034)
\lineto(40.41181967,699.60942034)
\curveto(40.39181208,699.54941821)(40.3768121,699.48941827)(40.36681967,699.42942034)
\curveto(40.36681211,699.37941838)(40.36181211,699.32941843)(40.35181967,699.27942034)
\curveto(40.28181219,698.97941878)(40.1718123,698.71441905)(40.02181967,698.48442034)
\curveto(39.86181261,698.24441952)(39.66681281,698.04941971)(39.43681967,697.89942034)
\curveto(39.20681327,697.74942001)(38.94681353,697.61942014)(38.65681967,697.50942034)
\curveto(38.54681393,697.4594203)(38.42681405,697.42442034)(38.29681967,697.40442034)
\curveto(38.1768143,697.38442038)(38.05681442,697.3594204)(37.93681967,697.32942034)
\curveto(37.84681463,697.30942045)(37.75181472,697.29942046)(37.65181967,697.29942034)
\curveto(37.56181491,697.28942047)(37.471815,697.27442049)(37.38181967,697.25442034)
\lineto(37.11181967,697.25442034)
\curveto(37.05181542,697.23442053)(36.94681553,697.22442054)(36.79681967,697.22442034)
\curveto(36.65681582,697.22442054)(36.55681592,697.23442053)(36.49681967,697.25442034)
\curveto(36.46681601,697.25442051)(36.43181604,697.2594205)(36.39181967,697.26942034)
\lineto(36.28681967,697.26942034)
\curveto(36.16681631,697.28942047)(36.04681643,697.30442046)(35.92681967,697.31442034)
\curveto(35.80681667,697.32442044)(35.69181678,697.34442042)(35.58181967,697.37442034)
\curveto(35.19181728,697.48442028)(34.84681763,697.60942015)(34.54681967,697.74942034)
\curveto(34.24681823,697.89941986)(33.99181848,698.11941964)(33.78181967,698.40942034)
\curveto(33.64181883,698.59941916)(33.52181895,698.81941894)(33.42181967,699.06942034)
\curveto(33.40181907,699.12941863)(33.38181909,699.20941855)(33.36181967,699.30942034)
\curveto(33.34181913,699.3594184)(33.32681915,699.42941833)(33.31681967,699.51942034)
\curveto(33.30681917,699.60941815)(33.31181916,699.68441808)(33.33181967,699.74442034)
\curveto(33.36181911,699.81441795)(33.41181906,699.8644179)(33.48181967,699.89442034)
\curveto(33.53181894,699.91441785)(33.59181888,699.92441784)(33.66181967,699.92442034)
\lineto(33.88681967,699.92442034)
\lineto(34.59181967,699.92442034)
\lineto(34.83181967,699.92442034)
\curveto(34.91181756,699.92441784)(34.98181749,699.91441785)(35.04181967,699.89442034)
\curveto(35.15181732,699.85441791)(35.22181725,699.78941797)(35.25181967,699.69942034)
\curveto(35.29181718,699.60941815)(35.33681714,699.51441825)(35.38681967,699.41442034)
\curveto(35.40681707,699.3644184)(35.44181703,699.29941846)(35.49181967,699.21942034)
\curveto(35.55181692,699.13941862)(35.60181687,699.08941867)(35.64181967,699.06942034)
\curveto(35.76181671,698.96941879)(35.8768166,698.88941887)(35.98681967,698.82942034)
\curveto(36.09681638,698.77941898)(36.23681624,698.72941903)(36.40681967,698.67942034)
\curveto(36.45681602,698.6594191)(36.50681597,698.64941911)(36.55681967,698.64942034)
\curveto(36.60681587,698.6594191)(36.65681582,698.6594191)(36.70681967,698.64942034)
\curveto(36.78681569,698.62941913)(36.8718156,698.61941914)(36.96181967,698.61942034)
\curveto(37.06181541,698.62941913)(37.14681533,698.64441912)(37.21681967,698.66442034)
\curveto(37.26681521,698.67441909)(37.31181516,698.67941908)(37.35181967,698.67942034)
\curveto(37.40181507,698.67941908)(37.45181502,698.68941907)(37.50181967,698.70942034)
\curveto(37.64181483,698.759419)(37.76681471,698.81941894)(37.87681967,698.88942034)
\curveto(37.99681448,698.9594188)(38.09181438,699.04941871)(38.16181967,699.15942034)
\curveto(38.21181426,699.23941852)(38.25181422,699.3644184)(38.28181967,699.53442034)
\curveto(38.30181417,699.60441816)(38.30181417,699.66941809)(38.28181967,699.72942034)
\curveto(38.26181421,699.78941797)(38.24181423,699.83941792)(38.22181967,699.87942034)
\curveto(38.15181432,700.01941774)(38.06181441,700.12441764)(37.95181967,700.19442034)
\curveto(37.85181462,700.2644175)(37.73181474,700.32941743)(37.59181967,700.38942034)
\curveto(37.40181507,700.46941729)(37.20181527,700.53441723)(36.99181967,700.58442034)
\curveto(36.78181569,700.63441713)(36.5718159,700.68941707)(36.36181967,700.74942034)
\curveto(36.28181619,700.76941699)(36.19681628,700.78441698)(36.10681967,700.79442034)
\curveto(36.02681645,700.80441696)(35.94681653,700.81941694)(35.86681967,700.83942034)
\curveto(35.54681693,700.92941683)(35.24181723,701.01441675)(34.95181967,701.09442034)
\curveto(34.66181781,701.18441658)(34.39681808,701.31441645)(34.15681967,701.48442034)
\curveto(33.8768186,701.68441608)(33.6718188,701.95441581)(33.54181967,702.29442034)
\curveto(33.52181895,702.3644154)(33.50181897,702.4594153)(33.48181967,702.57942034)
\curveto(33.46181901,702.64941511)(33.44681903,702.73441503)(33.43681967,702.83442034)
\curveto(33.42681905,702.93441483)(33.43181904,703.02441474)(33.45181967,703.10442034)
\curveto(33.471819,703.15441461)(33.476819,703.19441457)(33.46681967,703.22442034)
\curveto(33.45681902,703.2644145)(33.46181901,703.30941445)(33.48181967,703.35942034)
\curveto(33.50181897,703.46941429)(33.52181895,703.56941419)(33.54181967,703.65942034)
\curveto(33.5718189,703.759414)(33.60681887,703.85441391)(33.64681967,703.94442034)
\curveto(33.7768187,704.23441353)(33.95681852,704.46941329)(34.18681967,704.64942034)
\curveto(34.41681806,704.82941293)(34.6768178,704.97441279)(34.96681967,705.08442034)
\curveto(35.0768174,705.13441263)(35.19181728,705.16941259)(35.31181967,705.18942034)
\curveto(35.43181704,705.21941254)(35.55681692,705.24941251)(35.68681967,705.27942034)
\curveto(35.74681673,705.29941246)(35.80681667,705.30941245)(35.86681967,705.30942034)
\lineto(36.04681967,705.33942034)
\curveto(36.12681635,705.34941241)(36.21181626,705.35441241)(36.30181967,705.35442034)
\curveto(36.39181608,705.35441241)(36.476816,705.3594124)(36.55681967,705.36942034)
}
}
{
\newrgbcolor{curcolor}{0 0 0}
\pscustom[linestyle=none,fillstyle=solid,fillcolor=curcolor]
{
\newpath
\moveto(42.69346029,707.46942034)
\lineto(43.69846029,707.46942034)
\curveto(43.84845731,707.46941029)(43.97845718,707.4594103)(44.08846029,707.43942034)
\curveto(44.20845695,707.42941033)(44.29345686,707.36941039)(44.34346029,707.25942034)
\curveto(44.36345679,707.20941055)(44.37345678,707.14941061)(44.37346029,707.07942034)
\lineto(44.37346029,706.86942034)
\lineto(44.37346029,706.19442034)
\curveto(44.37345678,706.14441162)(44.36845679,706.08441168)(44.35846029,706.01442034)
\curveto(44.3584568,705.95441181)(44.36345679,705.89941186)(44.37346029,705.84942034)
\lineto(44.37346029,705.68442034)
\curveto(44.37345678,705.60441216)(44.37845678,705.52941223)(44.38846029,705.45942034)
\curveto(44.39845676,705.39941236)(44.42345673,705.34441242)(44.46346029,705.29442034)
\curveto(44.53345662,705.20441256)(44.6584565,705.15441261)(44.83846029,705.14442034)
\lineto(45.37846029,705.14442034)
\lineto(45.55846029,705.14442034)
\curveto(45.61845554,705.14441262)(45.67345548,705.13441263)(45.72346029,705.11442034)
\curveto(45.83345532,705.0644127)(45.89345526,704.97441279)(45.90346029,704.84442034)
\curveto(45.92345523,704.71441305)(45.93345522,704.56941319)(45.93346029,704.40942034)
\lineto(45.93346029,704.19942034)
\curveto(45.94345521,704.12941363)(45.93845522,704.06941369)(45.91846029,704.01942034)
\curveto(45.86845529,703.8594139)(45.76345539,703.77441399)(45.60346029,703.76442034)
\curveto(45.44345571,703.75441401)(45.26345589,703.74941401)(45.06346029,703.74942034)
\lineto(44.92846029,703.74942034)
\curveto(44.88845627,703.759414)(44.8534563,703.759414)(44.82346029,703.74942034)
\curveto(44.78345637,703.73941402)(44.74845641,703.73441403)(44.71846029,703.73442034)
\curveto(44.68845647,703.74441402)(44.6584565,703.73941402)(44.62846029,703.71942034)
\curveto(44.54845661,703.69941406)(44.48845667,703.65441411)(44.44846029,703.58442034)
\curveto(44.41845674,703.52441424)(44.39345676,703.44941431)(44.37346029,703.35942034)
\curveto(44.36345679,703.30941445)(44.36345679,703.25441451)(44.37346029,703.19442034)
\curveto(44.38345677,703.13441463)(44.38345677,703.07941468)(44.37346029,703.02942034)
\lineto(44.37346029,702.09942034)
\lineto(44.37346029,700.34442034)
\curveto(44.37345678,700.09441767)(44.37845678,699.87441789)(44.38846029,699.68442034)
\curveto(44.40845675,699.50441826)(44.47345668,699.34441842)(44.58346029,699.20442034)
\curveto(44.63345652,699.14441862)(44.69845646,699.09941866)(44.77846029,699.06942034)
\lineto(45.04846029,699.00942034)
\curveto(45.07845608,698.99941876)(45.10845605,698.99441877)(45.13846029,698.99442034)
\curveto(45.17845598,699.00441876)(45.20845595,699.00441876)(45.22846029,698.99442034)
\lineto(45.39346029,698.99442034)
\curveto(45.50345565,698.99441877)(45.59845556,698.98941877)(45.67846029,698.97942034)
\curveto(45.7584554,698.96941879)(45.82345533,698.92941883)(45.87346029,698.85942034)
\curveto(45.91345524,698.79941896)(45.93345522,698.71941904)(45.93346029,698.61942034)
\lineto(45.93346029,698.33442034)
\curveto(45.93345522,698.12441964)(45.92845523,697.92941983)(45.91846029,697.74942034)
\curveto(45.91845524,697.57942018)(45.83845532,697.4644203)(45.67846029,697.40442034)
\curveto(45.62845553,697.38442038)(45.58345557,697.37942038)(45.54346029,697.38942034)
\curveto(45.50345565,697.38942037)(45.4584557,697.37942038)(45.40846029,697.35942034)
\lineto(45.25846029,697.35942034)
\curveto(45.23845592,697.3594204)(45.20845595,697.3644204)(45.16846029,697.37442034)
\curveto(45.12845603,697.37442039)(45.09345606,697.36942039)(45.06346029,697.35942034)
\curveto(45.01345614,697.34942041)(44.9584562,697.34942041)(44.89846029,697.35942034)
\lineto(44.74846029,697.35942034)
\lineto(44.59846029,697.35942034)
\curveto(44.54845661,697.34942041)(44.50345665,697.34942041)(44.46346029,697.35942034)
\lineto(44.29846029,697.35942034)
\curveto(44.24845691,697.36942039)(44.19345696,697.37442039)(44.13346029,697.37442034)
\curveto(44.07345708,697.37442039)(44.01845714,697.37942038)(43.96846029,697.38942034)
\curveto(43.89845726,697.39942036)(43.83345732,697.40942035)(43.77346029,697.41942034)
\lineto(43.59346029,697.44942034)
\curveto(43.48345767,697.47942028)(43.37845778,697.51442025)(43.27846029,697.55442034)
\curveto(43.17845798,697.59442017)(43.08345807,697.63942012)(42.99346029,697.68942034)
\lineto(42.90346029,697.74942034)
\curveto(42.87345828,697.77941998)(42.83845832,697.80941995)(42.79846029,697.83942034)
\curveto(42.77845838,697.8594199)(42.7534584,697.87941988)(42.72346029,697.89942034)
\lineto(42.64846029,697.97442034)
\curveto(42.50845865,698.1644196)(42.40345875,698.37441939)(42.33346029,698.60442034)
\curveto(42.31345884,698.64441912)(42.30345885,698.67941908)(42.30346029,698.70942034)
\curveto(42.31345884,698.74941901)(42.31345884,698.79441897)(42.30346029,698.84442034)
\curveto(42.29345886,698.8644189)(42.28845887,698.88941887)(42.28846029,698.91942034)
\curveto(42.28845887,698.94941881)(42.28345887,698.97441879)(42.27346029,698.99442034)
\lineto(42.27346029,699.14442034)
\curveto(42.26345889,699.18441858)(42.2584589,699.22941853)(42.25846029,699.27942034)
\curveto(42.26845889,699.32941843)(42.27345888,699.37941838)(42.27346029,699.42942034)
\lineto(42.27346029,699.99942034)
\lineto(42.27346029,702.23442034)
\lineto(42.27346029,703.02942034)
\lineto(42.27346029,703.23942034)
\curveto(42.28345887,703.30941445)(42.27845888,703.37441439)(42.25846029,703.43442034)
\curveto(42.21845894,703.57441419)(42.14845901,703.6644141)(42.04846029,703.70442034)
\curveto(41.93845922,703.75441401)(41.79845936,703.76941399)(41.62846029,703.74942034)
\curveto(41.4584597,703.72941403)(41.31345984,703.74441402)(41.19346029,703.79442034)
\curveto(41.11346004,703.82441394)(41.06346009,703.86941389)(41.04346029,703.92942034)
\curveto(41.02346013,703.98941377)(41.00346015,704.0644137)(40.98346029,704.15442034)
\lineto(40.98346029,704.46942034)
\curveto(40.98346017,704.64941311)(40.99346016,704.79441297)(41.01346029,704.90442034)
\curveto(41.03346012,705.01441275)(41.11846004,705.08941267)(41.26846029,705.12942034)
\curveto(41.30845985,705.14941261)(41.34845981,705.15441261)(41.38846029,705.14442034)
\lineto(41.52346029,705.14442034)
\curveto(41.67345948,705.14441262)(41.81345934,705.14941261)(41.94346029,705.15942034)
\curveto(42.07345908,705.17941258)(42.16345899,705.23941252)(42.21346029,705.33942034)
\curveto(42.24345891,705.40941235)(42.2584589,705.48941227)(42.25846029,705.57942034)
\curveto(42.26845889,705.66941209)(42.27345888,705.759412)(42.27346029,705.84942034)
\lineto(42.27346029,706.77942034)
\lineto(42.27346029,707.03442034)
\curveto(42.27345888,707.12441064)(42.28345887,707.19941056)(42.30346029,707.25942034)
\curveto(42.3534588,707.3594104)(42.42845873,707.42441034)(42.52846029,707.45442034)
\curveto(42.54845861,707.4644103)(42.57345858,707.4644103)(42.60346029,707.45442034)
\curveto(42.64345851,707.45441031)(42.67345848,707.4594103)(42.69346029,707.46942034)
}
}
{
\newrgbcolor{curcolor}{0 0 0}
\pscustom[linestyle=none,fillstyle=solid,fillcolor=curcolor]
{
\newpath
\moveto(51.34189779,705.35442034)
\curveto(51.45189248,705.35441241)(51.54689238,705.34441242)(51.62689779,705.32442034)
\curveto(51.71689221,705.30441246)(51.78689214,705.2594125)(51.83689779,705.18942034)
\curveto(51.89689203,705.10941265)(51.926892,704.96941279)(51.92689779,704.76942034)
\lineto(51.92689779,704.25942034)
\lineto(51.92689779,703.88442034)
\curveto(51.93689199,703.74441402)(51.92189201,703.63441413)(51.88189779,703.55442034)
\curveto(51.84189209,703.48441428)(51.78189215,703.43941432)(51.70189779,703.41942034)
\curveto(51.6318923,703.39941436)(51.54689238,703.38941437)(51.44689779,703.38942034)
\curveto(51.35689257,703.38941437)(51.25689267,703.39441437)(51.14689779,703.40442034)
\curveto(51.04689288,703.41441435)(50.95189298,703.40941435)(50.86189779,703.38942034)
\curveto(50.79189314,703.36941439)(50.72189321,703.35441441)(50.65189779,703.34442034)
\curveto(50.58189335,703.34441442)(50.51689341,703.33441443)(50.45689779,703.31442034)
\curveto(50.29689363,703.2644145)(50.13689379,703.18941457)(49.97689779,703.08942034)
\curveto(49.81689411,702.99941476)(49.69189424,702.89441487)(49.60189779,702.77442034)
\curveto(49.55189438,702.69441507)(49.49689443,702.60941515)(49.43689779,702.51942034)
\curveto(49.38689454,702.43941532)(49.33689459,702.35441541)(49.28689779,702.26442034)
\curveto(49.25689467,702.18441558)(49.2268947,702.09941566)(49.19689779,702.00942034)
\lineto(49.13689779,701.76942034)
\curveto(49.11689481,701.69941606)(49.10689482,701.62441614)(49.10689779,701.54442034)
\curveto(49.10689482,701.47441629)(49.09689483,701.40441636)(49.07689779,701.33442034)
\curveto(49.06689486,701.29441647)(49.06189487,701.25441651)(49.06189779,701.21442034)
\curveto(49.07189486,701.18441658)(49.07189486,701.15441661)(49.06189779,701.12442034)
\lineto(49.06189779,700.88442034)
\curveto(49.04189489,700.81441695)(49.03689489,700.73441703)(49.04689779,700.64442034)
\curveto(49.05689487,700.5644172)(49.06189487,700.48441728)(49.06189779,700.40442034)
\lineto(49.06189779,699.44442034)
\lineto(49.06189779,698.16942034)
\curveto(49.06189487,698.03941972)(49.05689487,697.91941984)(49.04689779,697.80942034)
\curveto(49.03689489,697.69942006)(49.00689492,697.60942015)(48.95689779,697.53942034)
\curveto(48.93689499,697.50942025)(48.90189503,697.48442028)(48.85189779,697.46442034)
\curveto(48.81189512,697.45442031)(48.76689516,697.44442032)(48.71689779,697.43442034)
\lineto(48.64189779,697.43442034)
\curveto(48.59189534,697.42442034)(48.53689539,697.41942034)(48.47689779,697.41942034)
\lineto(48.31189779,697.41942034)
\lineto(47.66689779,697.41942034)
\curveto(47.60689632,697.42942033)(47.54189639,697.43442033)(47.47189779,697.43442034)
\lineto(47.27689779,697.43442034)
\curveto(47.2268967,697.45442031)(47.17689675,697.46942029)(47.12689779,697.47942034)
\curveto(47.07689685,697.49942026)(47.04189689,697.53442023)(47.02189779,697.58442034)
\curveto(46.98189695,697.63442013)(46.95689697,697.70442006)(46.94689779,697.79442034)
\lineto(46.94689779,698.09442034)
\lineto(46.94689779,699.11442034)
\lineto(46.94689779,703.34442034)
\lineto(46.94689779,704.45442034)
\lineto(46.94689779,704.73942034)
\curveto(46.94689698,704.83941292)(46.96689696,704.91941284)(47.00689779,704.97942034)
\curveto(47.05689687,705.0594127)(47.1318968,705.10941265)(47.23189779,705.12942034)
\curveto(47.3318966,705.14941261)(47.45189648,705.1594126)(47.59189779,705.15942034)
\lineto(48.35689779,705.15942034)
\curveto(48.47689545,705.1594126)(48.58189535,705.14941261)(48.67189779,705.12942034)
\curveto(48.76189517,705.11941264)(48.8318951,705.07441269)(48.88189779,704.99442034)
\curveto(48.91189502,704.94441282)(48.926895,704.87441289)(48.92689779,704.78442034)
\lineto(48.95689779,704.51442034)
\curveto(48.96689496,704.43441333)(48.98189495,704.3594134)(49.00189779,704.28942034)
\curveto(49.0318949,704.21941354)(49.08189485,704.18441358)(49.15189779,704.18442034)
\curveto(49.17189476,704.20441356)(49.19189474,704.21441355)(49.21189779,704.21442034)
\curveto(49.2318947,704.21441355)(49.25189468,704.22441354)(49.27189779,704.24442034)
\curveto(49.3318946,704.29441347)(49.38189455,704.34941341)(49.42189779,704.40942034)
\curveto(49.47189446,704.47941328)(49.5318944,704.53941322)(49.60189779,704.58942034)
\curveto(49.64189429,704.61941314)(49.67689425,704.64941311)(49.70689779,704.67942034)
\curveto(49.73689419,704.71941304)(49.77189416,704.75441301)(49.81189779,704.78442034)
\lineto(50.08189779,704.96442034)
\curveto(50.18189375,705.02441274)(50.28189365,705.07941268)(50.38189779,705.12942034)
\curveto(50.48189345,705.16941259)(50.58189335,705.20441256)(50.68189779,705.23442034)
\lineto(51.01189779,705.32442034)
\curveto(51.04189289,705.33441243)(51.09689283,705.33441243)(51.17689779,705.32442034)
\curveto(51.26689266,705.32441244)(51.32189261,705.33441243)(51.34189779,705.35442034)
}
}
{
\newrgbcolor{curcolor}{0 0 0}
\pscustom[linestyle=none,fillstyle=solid,fillcolor=curcolor]
{
\newpath
\moveto(54.84697592,708.00942034)
\curveto(54.91697297,707.92940983)(54.95197293,707.80940995)(54.95197592,707.64942034)
\lineto(54.95197592,707.18442034)
\lineto(54.95197592,706.77942034)
\curveto(54.95197293,706.63941112)(54.91697297,706.54441122)(54.84697592,706.49442034)
\curveto(54.7869731,706.44441132)(54.70697318,706.41441135)(54.60697592,706.40442034)
\curveto(54.51697337,706.39441137)(54.41697347,706.38941137)(54.30697592,706.38942034)
\lineto(53.46697592,706.38942034)
\curveto(53.35697453,706.38941137)(53.25697463,706.39441137)(53.16697592,706.40442034)
\curveto(53.0869748,706.41441135)(53.01697487,706.44441132)(52.95697592,706.49442034)
\curveto(52.91697497,706.52441124)(52.886975,706.57941118)(52.86697592,706.65942034)
\curveto(52.85697503,706.74941101)(52.84697504,706.84441092)(52.83697592,706.94442034)
\lineto(52.83697592,707.27442034)
\curveto(52.84697504,707.38441038)(52.85197503,707.47941028)(52.85197592,707.55942034)
\lineto(52.85197592,707.76942034)
\curveto(52.86197502,707.83940992)(52.881975,707.89940986)(52.91197592,707.94942034)
\curveto(52.93197495,707.98940977)(52.95697493,708.01940974)(52.98697592,708.03942034)
\lineto(53.10697592,708.09942034)
\curveto(53.12697476,708.09940966)(53.15197473,708.09940966)(53.18197592,708.09942034)
\curveto(53.21197467,708.10940965)(53.23697465,708.11440965)(53.25697592,708.11442034)
\lineto(54.35197592,708.11442034)
\curveto(54.45197343,708.11440965)(54.54697334,708.10940965)(54.63697592,708.09942034)
\curveto(54.72697316,708.08940967)(54.79697309,708.0594097)(54.84697592,708.00942034)
\moveto(54.95197592,698.24442034)
\curveto(54.95197293,698.04441972)(54.94697294,697.87441989)(54.93697592,697.73442034)
\curveto(54.92697296,697.59442017)(54.83697305,697.49942026)(54.66697592,697.44942034)
\curveto(54.60697328,697.42942033)(54.54197334,697.41942034)(54.47197592,697.41942034)
\curveto(54.40197348,697.42942033)(54.32697356,697.43442033)(54.24697592,697.43442034)
\lineto(53.40697592,697.43442034)
\curveto(53.31697457,697.43442033)(53.22697466,697.43942032)(53.13697592,697.44942034)
\curveto(53.05697483,697.4594203)(52.99697489,697.48942027)(52.95697592,697.53942034)
\curveto(52.89697499,697.60942015)(52.86197502,697.69442007)(52.85197592,697.79442034)
\lineto(52.85197592,698.13942034)
\lineto(52.85197592,704.46942034)
\lineto(52.85197592,704.76942034)
\curveto(52.85197503,704.86941289)(52.87197501,704.94941281)(52.91197592,705.00942034)
\curveto(52.97197491,705.07941268)(53.05697483,705.12441264)(53.16697592,705.14442034)
\curveto(53.1869747,705.15441261)(53.21197467,705.15441261)(53.24197592,705.14442034)
\curveto(53.2819746,705.14441262)(53.31197457,705.14941261)(53.33197592,705.15942034)
\lineto(54.08197592,705.15942034)
\lineto(54.27697592,705.15942034)
\curveto(54.35697353,705.16941259)(54.42197346,705.16941259)(54.47197592,705.15942034)
\lineto(54.59197592,705.15942034)
\curveto(54.65197323,705.13941262)(54.70697318,705.12441264)(54.75697592,705.11442034)
\curveto(54.80697308,705.10441266)(54.84697304,705.07441269)(54.87697592,705.02442034)
\curveto(54.91697297,704.97441279)(54.93697295,704.90441286)(54.93697592,704.81442034)
\curveto(54.94697294,704.72441304)(54.95197293,704.62941313)(54.95197592,704.52942034)
\lineto(54.95197592,698.24442034)
}
}
{
\newrgbcolor{curcolor}{0 0 0}
\pscustom[linestyle=none,fillstyle=solid,fillcolor=curcolor]
{
\newpath
\moveto(64.41416342,701.67942034)
\curveto(64.43415482,701.61941614)(64.44415481,701.51441625)(64.44416342,701.36442034)
\curveto(64.44415481,701.22441654)(64.43915481,701.12441664)(64.42916342,701.06442034)
\curveto(64.42915482,701.01441675)(64.42415483,700.96941679)(64.41416342,700.92942034)
\lineto(64.41416342,700.80942034)
\curveto(64.39415486,700.72941703)(64.38415487,700.64941711)(64.38416342,700.56942034)
\curveto(64.38415487,700.49941726)(64.37415488,700.42441734)(64.35416342,700.34442034)
\curveto(64.3541549,700.30441746)(64.34415491,700.23441753)(64.32416342,700.13442034)
\curveto(64.29415496,700.01441775)(64.26415499,699.88941787)(64.23416342,699.75942034)
\curveto(64.21415504,699.63941812)(64.17915507,699.52441824)(64.12916342,699.41442034)
\curveto(63.9491553,698.9644188)(63.72415553,698.57441919)(63.45416342,698.24442034)
\curveto(63.18415607,697.91441985)(62.82915642,697.65442011)(62.38916342,697.46442034)
\curveto(62.29915695,697.42442034)(62.20415705,697.39442037)(62.10416342,697.37442034)
\curveto(62.01415724,697.34442042)(61.91415734,697.31442045)(61.80416342,697.28442034)
\curveto(61.74415751,697.2644205)(61.67915757,697.25442051)(61.60916342,697.25442034)
\curveto(61.5491577,697.25442051)(61.48915776,697.24942051)(61.42916342,697.23942034)
\lineto(61.29416342,697.23942034)
\curveto(61.23415802,697.21942054)(61.1541581,697.21442055)(61.05416342,697.22442034)
\curveto(60.9541583,697.22442054)(60.87415838,697.23442053)(60.81416342,697.25442034)
\lineto(60.72416342,697.25442034)
\curveto(60.67415858,697.2644205)(60.61915863,697.27442049)(60.55916342,697.28442034)
\curveto(60.49915875,697.28442048)(60.43915881,697.28942047)(60.37916342,697.29942034)
\curveto(60.18915906,697.34942041)(60.01415924,697.39942036)(59.85416342,697.44942034)
\curveto(59.69415956,697.49942026)(59.54415971,697.56942019)(59.40416342,697.65942034)
\lineto(59.22416342,697.77942034)
\curveto(59.17416008,697.81941994)(59.12416013,697.8644199)(59.07416342,697.91442034)
\lineto(58.98416342,697.97442034)
\curveto(58.9541603,697.99441977)(58.92416033,698.00941975)(58.89416342,698.01942034)
\curveto(58.80416045,698.04941971)(58.7491605,698.02941973)(58.72916342,697.95942034)
\curveto(58.67916057,697.88941987)(58.64416061,697.80441996)(58.62416342,697.70442034)
\curveto(58.61416064,697.61442015)(58.57916067,697.54442022)(58.51916342,697.49442034)
\curveto(58.45916079,697.45442031)(58.38916086,697.42942033)(58.30916342,697.41942034)
\lineto(58.03916342,697.41942034)
\lineto(57.31916342,697.41942034)
\lineto(57.09416342,697.41942034)
\curveto(57.02416223,697.40942035)(56.95916229,697.41442035)(56.89916342,697.43442034)
\curveto(56.75916249,697.48442028)(56.67916257,697.57442019)(56.65916342,697.70442034)
\curveto(56.6491626,697.84441992)(56.64416261,697.99941976)(56.64416342,698.16942034)
\lineto(56.64416342,707.31942034)
\lineto(56.64416342,707.66442034)
\curveto(56.64416261,707.78440998)(56.66916258,707.87940988)(56.71916342,707.94942034)
\curveto(56.75916249,708.01940974)(56.82916242,708.0644097)(56.92916342,708.08442034)
\curveto(56.9491623,708.09440967)(56.96916228,708.09440967)(56.98916342,708.08442034)
\curveto(57.01916223,708.08440968)(57.04416221,708.08940967)(57.06416342,708.09942034)
\lineto(58.00916342,708.09942034)
\curveto(58.18916106,708.09940966)(58.34416091,708.08940967)(58.47416342,708.06942034)
\curveto(58.60416065,708.0594097)(58.68916056,707.98440978)(58.72916342,707.84442034)
\curveto(58.75916049,707.74441002)(58.76916048,707.60941015)(58.75916342,707.43942034)
\curveto(58.7491605,707.27941048)(58.74416051,707.13941062)(58.74416342,707.01942034)
\lineto(58.74416342,705.38442034)
\lineto(58.74416342,705.05442034)
\curveto(58.74416051,704.94441282)(58.7541605,704.84941291)(58.77416342,704.76942034)
\curveto(58.78416047,704.71941304)(58.79416046,704.67441309)(58.80416342,704.63442034)
\curveto(58.81416044,704.60441316)(58.83916041,704.58441318)(58.87916342,704.57442034)
\curveto(58.89916035,704.55441321)(58.92416033,704.54441322)(58.95416342,704.54442034)
\curveto(58.99416026,704.54441322)(59.02416023,704.54941321)(59.04416342,704.55942034)
\curveto(59.11416014,704.59941316)(59.17916007,704.63941312)(59.23916342,704.67942034)
\curveto(59.29915995,704.72941303)(59.36415989,704.77941298)(59.43416342,704.82942034)
\curveto(59.56415969,704.91941284)(59.69915955,704.99441277)(59.83916342,705.05442034)
\curveto(59.97915927,705.12441264)(60.13415912,705.18441258)(60.30416342,705.23442034)
\curveto(60.38415887,705.2644125)(60.46415879,705.27941248)(60.54416342,705.27942034)
\curveto(60.62415863,705.28941247)(60.70415855,705.30441246)(60.78416342,705.32442034)
\curveto(60.8541584,705.34441242)(60.92915832,705.35441241)(61.00916342,705.35442034)
\lineto(61.24916342,705.35442034)
\lineto(61.39916342,705.35442034)
\curveto(61.42915782,705.34441242)(61.46415779,705.33941242)(61.50416342,705.33942034)
\curveto(61.54415771,705.34941241)(61.58415767,705.34941241)(61.62416342,705.33942034)
\curveto(61.73415752,705.30941245)(61.83415742,705.28441248)(61.92416342,705.26442034)
\curveto(62.02415723,705.25441251)(62.11915713,705.22941253)(62.20916342,705.18942034)
\curveto(62.66915658,704.99941276)(63.04415621,704.75441301)(63.33416342,704.45442034)
\curveto(63.62415563,704.15441361)(63.86915538,703.77941398)(64.06916342,703.32942034)
\curveto(64.11915513,703.20941455)(64.15915509,703.08441468)(64.18916342,702.95442034)
\curveto(64.22915502,702.82441494)(64.26915498,702.68941507)(64.30916342,702.54942034)
\curveto(64.32915492,702.47941528)(64.33915491,702.40941535)(64.33916342,702.33942034)
\curveto(64.3491549,702.27941548)(64.36415489,702.20941555)(64.38416342,702.12942034)
\curveto(64.40415485,702.07941568)(64.40915484,702.02441574)(64.39916342,701.96442034)
\curveto(64.39915485,701.90441586)(64.40415485,701.84441592)(64.41416342,701.78442034)
\lineto(64.41416342,701.67942034)
\moveto(62.19416342,700.26942034)
\curveto(62.22415703,700.36941739)(62.249157,700.49441727)(62.26916342,700.64442034)
\curveto(62.29915695,700.79441697)(62.31415694,700.94441682)(62.31416342,701.09442034)
\curveto(62.32415693,701.25441651)(62.32415693,701.40941635)(62.31416342,701.55942034)
\curveto(62.31415694,701.71941604)(62.29915695,701.85441591)(62.26916342,701.96442034)
\curveto(62.23915701,702.0644157)(62.21915703,702.1594156)(62.20916342,702.24942034)
\curveto(62.19915705,702.33941542)(62.17415708,702.42441534)(62.13416342,702.50442034)
\curveto(61.99415726,702.85441491)(61.79415746,703.14941461)(61.53416342,703.38942034)
\curveto(61.28415797,703.63941412)(60.91415834,703.764414)(60.42416342,703.76442034)
\curveto(60.38415887,703.764414)(60.3491589,703.759414)(60.31916342,703.74942034)
\lineto(60.21416342,703.74942034)
\curveto(60.14415911,703.72941403)(60.07915917,703.70941405)(60.01916342,703.68942034)
\curveto(59.95915929,703.67941408)(59.89915935,703.6644141)(59.83916342,703.64442034)
\curveto(59.5491597,703.51441425)(59.32915992,703.32941443)(59.17916342,703.08942034)
\curveto(59.02916022,702.8594149)(58.90416035,702.59441517)(58.80416342,702.29442034)
\curveto(58.77416048,702.21441555)(58.7541605,702.12941563)(58.74416342,702.03942034)
\curveto(58.74416051,701.9594158)(58.73416052,701.87941588)(58.71416342,701.79942034)
\curveto(58.70416055,701.76941599)(58.69916055,701.71941604)(58.69916342,701.64942034)
\curveto(58.68916056,701.60941615)(58.68416057,701.56941619)(58.68416342,701.52942034)
\curveto(58.69416056,701.48941627)(58.69416056,701.44941631)(58.68416342,701.40942034)
\curveto(58.66416059,701.32941643)(58.65916059,701.21941654)(58.66916342,701.07942034)
\curveto(58.67916057,700.93941682)(58.69416056,700.83941692)(58.71416342,700.77942034)
\curveto(58.73416052,700.68941707)(58.74416051,700.60441716)(58.74416342,700.52442034)
\curveto(58.7541605,700.44441732)(58.77416048,700.3644174)(58.80416342,700.28442034)
\curveto(58.89416036,700.00441776)(58.99916025,699.759418)(59.11916342,699.54942034)
\curveto(59.24916,699.34941841)(59.42915982,699.17941858)(59.65916342,699.03942034)
\curveto(59.81915943,698.93941882)(59.98415927,698.86941889)(60.15416342,698.82942034)
\curveto(60.17415908,698.82941893)(60.19415906,698.82441894)(60.21416342,698.81442034)
\lineto(60.30416342,698.81442034)
\curveto(60.33415892,698.80441896)(60.38415887,698.79441897)(60.45416342,698.78442034)
\curveto(60.52415873,698.78441898)(60.58415867,698.78941897)(60.63416342,698.79942034)
\curveto(60.73415852,698.81941894)(60.82415843,698.83441893)(60.90416342,698.84442034)
\curveto(60.99415826,698.8644189)(61.07915817,698.88941887)(61.15916342,698.91942034)
\curveto(61.43915781,699.04941871)(61.6541576,699.22941853)(61.80416342,699.45942034)
\curveto(61.96415729,699.68941807)(62.09415716,699.9594178)(62.19416342,700.26942034)
}
}
{
\newrgbcolor{curcolor}{0 0 0}
\pscustom[linestyle=none,fillstyle=solid,fillcolor=curcolor]
{
\newpath
\moveto(66.20408529,705.14442034)
\lineto(67.32908529,705.14442034)
\curveto(67.43908286,705.14441262)(67.53908276,705.13941262)(67.62908529,705.12942034)
\curveto(67.71908258,705.11941264)(67.78408251,705.08441268)(67.82408529,705.02442034)
\curveto(67.87408242,704.9644128)(67.90408239,704.87941288)(67.91408529,704.76942034)
\curveto(67.92408237,704.66941309)(67.92908237,704.5644132)(67.92908529,704.45442034)
\lineto(67.92908529,703.40442034)
\lineto(67.92908529,701.16942034)
\curveto(67.92908237,700.80941695)(67.94408235,700.46941729)(67.97408529,700.14942034)
\curveto(68.00408229,699.82941793)(68.0940822,699.5644182)(68.24408529,699.35442034)
\curveto(68.38408191,699.14441862)(68.60908169,698.99441877)(68.91908529,698.90442034)
\curveto(68.96908133,698.89441887)(69.00908129,698.88941887)(69.03908529,698.88942034)
\curveto(69.07908122,698.88941887)(69.12408117,698.88441888)(69.17408529,698.87442034)
\curveto(69.22408107,698.8644189)(69.27908102,698.8594189)(69.33908529,698.85942034)
\curveto(69.3990809,698.8594189)(69.44408085,698.8644189)(69.47408529,698.87442034)
\curveto(69.52408077,698.89441887)(69.56408073,698.89941886)(69.59408529,698.88942034)
\curveto(69.63408066,698.87941888)(69.67408062,698.88441888)(69.71408529,698.90442034)
\curveto(69.92408037,698.95441881)(70.08908021,699.01941874)(70.20908529,699.09942034)
\curveto(70.38907991,699.20941855)(70.52907977,699.34941841)(70.62908529,699.51942034)
\curveto(70.73907956,699.69941806)(70.81407948,699.89441787)(70.85408529,700.10442034)
\curveto(70.90407939,700.32441744)(70.93407936,700.5644172)(70.94408529,700.82442034)
\curveto(70.95407934,701.09441667)(70.95907934,701.37441639)(70.95908529,701.66442034)
\lineto(70.95908529,703.47942034)
\lineto(70.95908529,704.45442034)
\lineto(70.95908529,704.72442034)
\curveto(70.95907934,704.82441294)(70.97907932,704.90441286)(71.01908529,704.96442034)
\curveto(71.06907923,705.05441271)(71.14407915,705.10441266)(71.24408529,705.11442034)
\curveto(71.34407895,705.13441263)(71.46407883,705.14441262)(71.60408529,705.14442034)
\lineto(72.39908529,705.14442034)
\lineto(72.68408529,705.14442034)
\curveto(72.77407752,705.14441262)(72.84907745,705.12441264)(72.90908529,705.08442034)
\curveto(72.98907731,705.03441273)(73.03407726,704.9594128)(73.04408529,704.85942034)
\curveto(73.05407724,704.759413)(73.05907724,704.64441312)(73.05908529,704.51442034)
\lineto(73.05908529,703.37442034)
\lineto(73.05908529,699.15942034)
\lineto(73.05908529,698.09442034)
\lineto(73.05908529,697.79442034)
\curveto(73.05907724,697.69442007)(73.03907726,697.61942014)(72.99908529,697.56942034)
\curveto(72.94907735,697.48942027)(72.87407742,697.44442032)(72.77408529,697.43442034)
\curveto(72.67407762,697.42442034)(72.56907773,697.41942034)(72.45908529,697.41942034)
\lineto(71.64908529,697.41942034)
\curveto(71.53907876,697.41942034)(71.43907886,697.42442034)(71.34908529,697.43442034)
\curveto(71.26907903,697.44442032)(71.20407909,697.48442028)(71.15408529,697.55442034)
\curveto(71.13407916,697.58442018)(71.11407918,697.62942013)(71.09408529,697.68942034)
\curveto(71.08407921,697.74942001)(71.06907923,697.80941995)(71.04908529,697.86942034)
\curveto(71.03907926,697.92941983)(71.02407927,697.98441978)(71.00408529,698.03442034)
\curveto(70.98407931,698.08441968)(70.95407934,698.11441965)(70.91408529,698.12442034)
\curveto(70.8940794,698.14441962)(70.86907943,698.14941961)(70.83908529,698.13942034)
\curveto(70.80907949,698.12941963)(70.78407951,698.11941964)(70.76408529,698.10942034)
\curveto(70.6940796,698.06941969)(70.63407966,698.02441974)(70.58408529,697.97442034)
\curveto(70.53407976,697.92441984)(70.47907982,697.87941988)(70.41908529,697.83942034)
\curveto(70.37907992,697.80941995)(70.33907996,697.77441999)(70.29908529,697.73442034)
\curveto(70.26908003,697.70442006)(70.22908007,697.67442009)(70.17908529,697.64442034)
\curveto(69.94908035,697.50442026)(69.67908062,697.39442037)(69.36908529,697.31442034)
\curveto(69.299081,697.29442047)(69.22908107,697.28442048)(69.15908529,697.28442034)
\curveto(69.08908121,697.27442049)(69.01408128,697.2594205)(68.93408529,697.23942034)
\curveto(68.8940814,697.22942053)(68.84908145,697.22942053)(68.79908529,697.23942034)
\curveto(68.75908154,697.23942052)(68.71908158,697.23442053)(68.67908529,697.22442034)
\curveto(68.64908165,697.21442055)(68.58408171,697.21442055)(68.48408529,697.22442034)
\curveto(68.3940819,697.22442054)(68.33408196,697.22942053)(68.30408529,697.23942034)
\curveto(68.25408204,697.23942052)(68.20408209,697.24442052)(68.15408529,697.25442034)
\lineto(68.00408529,697.25442034)
\curveto(67.88408241,697.28442048)(67.76908253,697.30942045)(67.65908529,697.32942034)
\curveto(67.54908275,697.34942041)(67.43908286,697.37942038)(67.32908529,697.41942034)
\curveto(67.27908302,697.43942032)(67.23408306,697.45442031)(67.19408529,697.46442034)
\curveto(67.16408313,697.48442028)(67.12408317,697.50442026)(67.07408529,697.52442034)
\curveto(66.72408357,697.71442005)(66.44408385,697.97941978)(66.23408529,698.31942034)
\curveto(66.10408419,698.52941923)(66.00908429,698.77941898)(65.94908529,699.06942034)
\curveto(65.88908441,699.36941839)(65.84908445,699.68441808)(65.82908529,700.01442034)
\curveto(65.81908448,700.35441741)(65.81408448,700.69941706)(65.81408529,701.04942034)
\curveto(65.82408447,701.40941635)(65.82908447,701.764416)(65.82908529,702.11442034)
\lineto(65.82908529,704.15442034)
\curveto(65.82908447,704.28441348)(65.82408447,704.43441333)(65.81408529,704.60442034)
\curveto(65.81408448,704.78441298)(65.83908446,704.91441285)(65.88908529,704.99442034)
\curveto(65.91908438,705.04441272)(65.97908432,705.08941267)(66.06908529,705.12942034)
\curveto(66.12908417,705.12941263)(66.17408412,705.13441263)(66.20408529,705.14442034)
}
}
{
\newrgbcolor{curcolor}{0 0 0}
\pscustom[linestyle=none,fillstyle=solid,fillcolor=curcolor]
{
\newpath
\moveto(78.26033529,705.36942034)
\curveto(79.07033013,705.38941237)(79.74532946,705.26941249)(80.28533529,705.00942034)
\curveto(80.83532837,704.74941301)(81.27032793,704.37941338)(81.59033529,703.89942034)
\curveto(81.75032745,703.6594141)(81.87032733,703.38441438)(81.95033529,703.07442034)
\curveto(81.97032723,703.02441474)(81.98532722,702.9594148)(81.99533529,702.87942034)
\curveto(82.01532719,702.79941496)(82.01532719,702.72941503)(81.99533529,702.66942034)
\curveto(81.95532725,702.5594152)(81.88532732,702.49441527)(81.78533529,702.47442034)
\curveto(81.68532752,702.4644153)(81.56532764,702.4594153)(81.42533529,702.45942034)
\lineto(80.64533529,702.45942034)
\lineto(80.36033529,702.45942034)
\curveto(80.27032893,702.4594153)(80.19532901,702.47941528)(80.13533529,702.51942034)
\curveto(80.05532915,702.5594152)(80.0003292,702.61941514)(79.97033529,702.69942034)
\curveto(79.94032926,702.78941497)(79.9003293,702.87941488)(79.85033529,702.96942034)
\curveto(79.79032941,703.07941468)(79.72532948,703.17941458)(79.65533529,703.26942034)
\curveto(79.58532962,703.3594144)(79.5053297,703.43941432)(79.41533529,703.50942034)
\curveto(79.27532993,703.59941416)(79.12033008,703.66941409)(78.95033529,703.71942034)
\curveto(78.89033031,703.73941402)(78.83033037,703.74941401)(78.77033529,703.74942034)
\curveto(78.71033049,703.74941401)(78.65533055,703.759414)(78.60533529,703.77942034)
\lineto(78.45533529,703.77942034)
\curveto(78.25533095,703.77941398)(78.09533111,703.759414)(77.97533529,703.71942034)
\curveto(77.68533152,703.62941413)(77.45033175,703.48941427)(77.27033529,703.29942034)
\curveto(77.09033211,703.11941464)(76.94533226,702.89941486)(76.83533529,702.63942034)
\curveto(76.78533242,702.52941523)(76.74533246,702.40941535)(76.71533529,702.27942034)
\curveto(76.69533251,702.1594156)(76.67033253,702.02941573)(76.64033529,701.88942034)
\curveto(76.63033257,701.84941591)(76.62533258,701.80941595)(76.62533529,701.76942034)
\curveto(76.62533258,701.72941603)(76.62033258,701.68941607)(76.61033529,701.64942034)
\curveto(76.59033261,701.54941621)(76.58033262,701.40941635)(76.58033529,701.22942034)
\curveto(76.59033261,701.04941671)(76.6053326,700.90941685)(76.62533529,700.80942034)
\curveto(76.62533258,700.72941703)(76.63033257,700.67441709)(76.64033529,700.64442034)
\curveto(76.66033254,700.57441719)(76.67033253,700.50441726)(76.67033529,700.43442034)
\curveto(76.68033252,700.3644174)(76.69533251,700.29441747)(76.71533529,700.22442034)
\curveto(76.79533241,699.99441777)(76.89033231,699.78441798)(77.00033529,699.59442034)
\curveto(77.11033209,699.40441836)(77.25033195,699.24441852)(77.42033529,699.11442034)
\curveto(77.46033174,699.08441868)(77.52033168,699.04941871)(77.60033529,699.00942034)
\curveto(77.71033149,698.93941882)(77.82033138,698.89441887)(77.93033529,698.87442034)
\curveto(78.05033115,698.85441891)(78.19533101,698.83441893)(78.36533529,698.81442034)
\lineto(78.45533529,698.81442034)
\curveto(78.49533071,698.81441895)(78.52533068,698.81941894)(78.54533529,698.82942034)
\lineto(78.68033529,698.82942034)
\curveto(78.75033045,698.84941891)(78.81533039,698.8644189)(78.87533529,698.87442034)
\curveto(78.94533026,698.89441887)(79.01033019,698.91441885)(79.07033529,698.93442034)
\curveto(79.37032983,699.0644187)(79.6003296,699.25441851)(79.76033529,699.50442034)
\curveto(79.8003294,699.55441821)(79.83532937,699.60941815)(79.86533529,699.66942034)
\curveto(79.89532931,699.73941802)(79.92032928,699.79941796)(79.94033529,699.84942034)
\curveto(79.98032922,699.9594178)(80.01532919,700.05441771)(80.04533529,700.13442034)
\curveto(80.07532913,700.22441754)(80.14532906,700.29441747)(80.25533529,700.34442034)
\curveto(80.34532886,700.38441738)(80.49032871,700.39941736)(80.69033529,700.38942034)
\lineto(81.18533529,700.38942034)
\lineto(81.39533529,700.38942034)
\curveto(81.47532773,700.39941736)(81.54032766,700.39441737)(81.59033529,700.37442034)
\lineto(81.71033529,700.37442034)
\lineto(81.83033529,700.34442034)
\curveto(81.87032733,700.34441742)(81.9003273,700.33441743)(81.92033529,700.31442034)
\curveto(81.97032723,700.27441749)(82.0003272,700.21441755)(82.01033529,700.13442034)
\curveto(82.03032717,700.0644177)(82.03032717,699.98941777)(82.01033529,699.90942034)
\curveto(81.92032728,699.57941818)(81.81032739,699.28441848)(81.68033529,699.02442034)
\curveto(81.27032793,698.25441951)(80.61532859,697.71942004)(79.71533529,697.41942034)
\curveto(79.61532959,697.38942037)(79.51032969,697.36942039)(79.40033529,697.35942034)
\curveto(79.29032991,697.33942042)(79.18033002,697.31442045)(79.07033529,697.28442034)
\curveto(79.01033019,697.27442049)(78.95033025,697.26942049)(78.89033529,697.26942034)
\curveto(78.83033037,697.26942049)(78.77033043,697.2644205)(78.71033529,697.25442034)
\lineto(78.54533529,697.25442034)
\curveto(78.49533071,697.23442053)(78.42033078,697.22942053)(78.32033529,697.23942034)
\curveto(78.22033098,697.23942052)(78.14533106,697.24442052)(78.09533529,697.25442034)
\curveto(78.01533119,697.27442049)(77.94033126,697.28442048)(77.87033529,697.28442034)
\curveto(77.81033139,697.27442049)(77.74533146,697.27942048)(77.67533529,697.29942034)
\lineto(77.52533529,697.32942034)
\curveto(77.47533173,697.32942043)(77.42533178,697.33442043)(77.37533529,697.34442034)
\curveto(77.26533194,697.37442039)(77.16033204,697.40442036)(77.06033529,697.43442034)
\curveto(76.96033224,697.4644203)(76.86533234,697.49942026)(76.77533529,697.53942034)
\curveto(76.3053329,697.73942002)(75.91033329,697.99441977)(75.59033529,698.30442034)
\curveto(75.27033393,698.62441914)(75.01033419,699.01941874)(74.81033529,699.48942034)
\curveto(74.76033444,699.57941818)(74.72033448,699.67441809)(74.69033529,699.77442034)
\lineto(74.60033529,700.10442034)
\curveto(74.59033461,700.14441762)(74.58533462,700.17941758)(74.58533529,700.20942034)
\curveto(74.58533462,700.24941751)(74.57533463,700.29441747)(74.55533529,700.34442034)
\curveto(74.53533467,700.41441735)(74.52533468,700.48441728)(74.52533529,700.55442034)
\curveto(74.52533468,700.63441713)(74.51533469,700.70941705)(74.49533529,700.77942034)
\lineto(74.49533529,701.03442034)
\curveto(74.47533473,701.08441668)(74.46533474,701.13941662)(74.46533529,701.19942034)
\curveto(74.46533474,701.26941649)(74.47533473,701.32941643)(74.49533529,701.37942034)
\curveto(74.5053347,701.42941633)(74.5053347,701.47441629)(74.49533529,701.51442034)
\curveto(74.48533472,701.55441621)(74.48533472,701.59441617)(74.49533529,701.63442034)
\curveto(74.51533469,701.70441606)(74.52033468,701.76941599)(74.51033529,701.82942034)
\curveto(74.51033469,701.88941587)(74.52033468,701.94941581)(74.54033529,702.00942034)
\curveto(74.59033461,702.18941557)(74.63033457,702.3594154)(74.66033529,702.51942034)
\curveto(74.69033451,702.68941507)(74.73533447,702.85441491)(74.79533529,703.01442034)
\curveto(75.01533419,703.52441424)(75.29033391,703.94941381)(75.62033529,704.28942034)
\curveto(75.96033324,704.62941313)(76.39033281,704.90441286)(76.91033529,705.11442034)
\curveto(77.05033215,705.17441259)(77.19533201,705.21441255)(77.34533529,705.23442034)
\curveto(77.49533171,705.2644125)(77.65033155,705.29941246)(77.81033529,705.33942034)
\curveto(77.89033131,705.34941241)(77.96533124,705.35441241)(78.03533529,705.35442034)
\curveto(78.1053311,705.35441241)(78.18033102,705.3594124)(78.26033529,705.36942034)
}
}
{
\newrgbcolor{curcolor}{0 0 0}
\pscustom[linestyle=none,fillstyle=solid,fillcolor=curcolor]
{
\newpath
\moveto(85.40361654,708.00942034)
\curveto(85.47361359,707.92940983)(85.50861356,707.80940995)(85.50861654,707.64942034)
\lineto(85.50861654,707.18442034)
\lineto(85.50861654,706.77942034)
\curveto(85.50861356,706.63941112)(85.47361359,706.54441122)(85.40361654,706.49442034)
\curveto(85.34361372,706.44441132)(85.2636138,706.41441135)(85.16361654,706.40442034)
\curveto(85.07361399,706.39441137)(84.97361409,706.38941137)(84.86361654,706.38942034)
\lineto(84.02361654,706.38942034)
\curveto(83.91361515,706.38941137)(83.81361525,706.39441137)(83.72361654,706.40442034)
\curveto(83.64361542,706.41441135)(83.57361549,706.44441132)(83.51361654,706.49442034)
\curveto(83.47361559,706.52441124)(83.44361562,706.57941118)(83.42361654,706.65942034)
\curveto(83.41361565,706.74941101)(83.40361566,706.84441092)(83.39361654,706.94442034)
\lineto(83.39361654,707.27442034)
\curveto(83.40361566,707.38441038)(83.40861566,707.47941028)(83.40861654,707.55942034)
\lineto(83.40861654,707.76942034)
\curveto(83.41861565,707.83940992)(83.43861563,707.89940986)(83.46861654,707.94942034)
\curveto(83.48861558,707.98940977)(83.51361555,708.01940974)(83.54361654,708.03942034)
\lineto(83.66361654,708.09942034)
\curveto(83.68361538,708.09940966)(83.70861536,708.09940966)(83.73861654,708.09942034)
\curveto(83.7686153,708.10940965)(83.79361527,708.11440965)(83.81361654,708.11442034)
\lineto(84.90861654,708.11442034)
\curveto(85.00861406,708.11440965)(85.10361396,708.10940965)(85.19361654,708.09942034)
\curveto(85.28361378,708.08940967)(85.35361371,708.0594097)(85.40361654,708.00942034)
\moveto(85.50861654,698.24442034)
\curveto(85.50861356,698.04441972)(85.50361356,697.87441989)(85.49361654,697.73442034)
\curveto(85.48361358,697.59442017)(85.39361367,697.49942026)(85.22361654,697.44942034)
\curveto(85.1636139,697.42942033)(85.09861397,697.41942034)(85.02861654,697.41942034)
\curveto(84.95861411,697.42942033)(84.88361418,697.43442033)(84.80361654,697.43442034)
\lineto(83.96361654,697.43442034)
\curveto(83.87361519,697.43442033)(83.78361528,697.43942032)(83.69361654,697.44942034)
\curveto(83.61361545,697.4594203)(83.55361551,697.48942027)(83.51361654,697.53942034)
\curveto(83.45361561,697.60942015)(83.41861565,697.69442007)(83.40861654,697.79442034)
\lineto(83.40861654,698.13942034)
\lineto(83.40861654,704.46942034)
\lineto(83.40861654,704.76942034)
\curveto(83.40861566,704.86941289)(83.42861564,704.94941281)(83.46861654,705.00942034)
\curveto(83.52861554,705.07941268)(83.61361545,705.12441264)(83.72361654,705.14442034)
\curveto(83.74361532,705.15441261)(83.7686153,705.15441261)(83.79861654,705.14442034)
\curveto(83.83861523,705.14441262)(83.8686152,705.14941261)(83.88861654,705.15942034)
\lineto(84.63861654,705.15942034)
\lineto(84.83361654,705.15942034)
\curveto(84.91361415,705.16941259)(84.97861409,705.16941259)(85.02861654,705.15942034)
\lineto(85.14861654,705.15942034)
\curveto(85.20861386,705.13941262)(85.2636138,705.12441264)(85.31361654,705.11442034)
\curveto(85.3636137,705.10441266)(85.40361366,705.07441269)(85.43361654,705.02442034)
\curveto(85.47361359,704.97441279)(85.49361357,704.90441286)(85.49361654,704.81442034)
\curveto(85.50361356,704.72441304)(85.50861356,704.62941313)(85.50861654,704.52942034)
\lineto(85.50861654,698.24442034)
}
}
{
\newrgbcolor{curcolor}{0 0 0}
\pscustom[linestyle=none,fillstyle=solid,fillcolor=curcolor]
{
\newpath
\moveto(94.94080404,701.60442034)
\curveto(94.92079551,701.65441611)(94.91579552,701.70941605)(94.92580404,701.76942034)
\curveto(94.9357955,701.82941593)(94.9307955,701.88441588)(94.91080404,701.93442034)
\curveto(94.90079553,701.97441579)(94.89579554,702.01441575)(94.89580404,702.05442034)
\curveto(94.89579554,702.09441567)(94.89079554,702.13441563)(94.88080404,702.17442034)
\lineto(94.82080404,702.44442034)
\curveto(94.80079563,702.53441523)(94.77579566,702.61941514)(94.74580404,702.69942034)
\curveto(94.69579574,702.83941492)(94.65079578,702.96941479)(94.61080404,703.08942034)
\curveto(94.57079586,703.21941454)(94.51579592,703.33941442)(94.44580404,703.44942034)
\curveto(94.37579606,703.5594142)(94.30579613,703.6644141)(94.23580404,703.76442034)
\curveto(94.17579626,703.8644139)(94.10579633,703.9644138)(94.02580404,704.06442034)
\curveto(93.94579649,704.17441359)(93.84579659,704.27441349)(93.72580404,704.36442034)
\curveto(93.61579682,704.4644133)(93.50579693,704.55441321)(93.39580404,704.63442034)
\curveto(93.06579737,704.8644129)(92.68579775,705.04441272)(92.25580404,705.17442034)
\curveto(91.8357986,705.30441246)(91.3357991,705.3644124)(90.75580404,705.35442034)
\curveto(90.68579975,705.34441242)(90.61579982,705.33941242)(90.54580404,705.33942034)
\curveto(90.47579996,705.33941242)(90.40080003,705.33441243)(90.32080404,705.32442034)
\curveto(90.17080026,705.28441248)(90.02580041,705.25441251)(89.88580404,705.23442034)
\curveto(89.74580069,705.21441255)(89.61080082,705.17941258)(89.48080404,705.12942034)
\curveto(89.37080106,705.07941268)(89.26080117,705.03441273)(89.15080404,704.99442034)
\curveto(89.04080139,704.95441281)(88.9358015,704.90941285)(88.83580404,704.85942034)
\curveto(88.47580196,704.62941313)(88.17080226,704.37441339)(87.92080404,704.09442034)
\curveto(87.67080276,703.82441394)(87.45580298,703.48441428)(87.27580404,703.07442034)
\curveto(87.22580321,702.95441481)(87.18580325,702.82941493)(87.15580404,702.69942034)
\curveto(87.12580331,702.57941518)(87.09080334,702.45441531)(87.05080404,702.32442034)
\curveto(87.0308034,702.27441549)(87.02080341,702.22441554)(87.02080404,702.17442034)
\curveto(87.02080341,702.13441563)(87.01580342,702.08941567)(87.00580404,702.03942034)
\curveto(86.98580345,701.98941577)(86.97580346,701.93441583)(86.97580404,701.87442034)
\curveto(86.98580345,701.82441594)(86.98580345,701.77441599)(86.97580404,701.72442034)
\lineto(86.97580404,701.61942034)
\curveto(86.95580348,701.5594162)(86.94080349,701.47441629)(86.93080404,701.36442034)
\curveto(86.9308035,701.25441651)(86.94080349,701.16941659)(86.96080404,701.10942034)
\lineto(86.96080404,700.97442034)
\curveto(86.96080347,700.93441683)(86.96580347,700.88941687)(86.97580404,700.83942034)
\curveto(86.99580344,700.759417)(87.00580343,700.67441709)(87.00580404,700.58442034)
\curveto(87.00580343,700.50441726)(87.01580342,700.42441734)(87.03580404,700.34442034)
\curveto(87.05580338,700.29441747)(87.06580337,700.24941751)(87.06580404,700.20942034)
\curveto(87.06580337,700.16941759)(87.07580336,700.12441764)(87.09580404,700.07442034)
\curveto(87.12580331,699.9644178)(87.15080328,699.8594179)(87.17080404,699.75942034)
\curveto(87.20080323,699.6594181)(87.24080319,699.5644182)(87.29080404,699.47442034)
\curveto(87.46080297,699.08441868)(87.67080276,698.74941901)(87.92080404,698.46942034)
\curveto(88.17080226,698.18941957)(88.47080196,697.94441982)(88.82080404,697.73442034)
\curveto(88.9308015,697.67442009)(89.0358014,697.62442014)(89.13580404,697.58442034)
\curveto(89.24580119,697.54442022)(89.36080107,697.50442026)(89.48080404,697.46442034)
\curveto(89.57080086,697.42442034)(89.66580077,697.39442037)(89.76580404,697.37442034)
\curveto(89.86580057,697.35442041)(89.96580047,697.32942043)(90.06580404,697.29942034)
\curveto(90.11580032,697.28942047)(90.15580028,697.28442048)(90.18580404,697.28442034)
\curveto(90.22580021,697.28442048)(90.26580017,697.27942048)(90.30580404,697.26942034)
\curveto(90.35580008,697.24942051)(90.40580003,697.24442052)(90.45580404,697.25442034)
\curveto(90.51579992,697.25442051)(90.57079986,697.24942051)(90.62080404,697.23942034)
\lineto(90.77080404,697.23942034)
\curveto(90.8307996,697.21942054)(90.91579952,697.21442055)(91.02580404,697.22442034)
\curveto(91.1357993,697.22442054)(91.21579922,697.22942053)(91.26580404,697.23942034)
\curveto(91.29579914,697.23942052)(91.32579911,697.24442052)(91.35580404,697.25442034)
\lineto(91.46080404,697.25442034)
\curveto(91.51079892,697.2644205)(91.56579887,697.26942049)(91.62580404,697.26942034)
\curveto(91.68579875,697.26942049)(91.74079869,697.27942048)(91.79080404,697.29942034)
\curveto(91.92079851,697.32942043)(92.04579839,697.3594204)(92.16580404,697.38942034)
\curveto(92.29579814,697.40942035)(92.42079801,697.44442032)(92.54080404,697.49442034)
\curveto(93.02079741,697.69442007)(93.430797,697.94441982)(93.77080404,698.24442034)
\curveto(94.11079632,698.54441922)(94.38579605,698.93441883)(94.59580404,699.41442034)
\curveto(94.64579579,699.51441825)(94.68579575,699.61941814)(94.71580404,699.72942034)
\curveto(94.74579569,699.84941791)(94.78079565,699.9644178)(94.82080404,700.07442034)
\curveto(94.8307956,700.14441762)(94.84079559,700.20941755)(94.85080404,700.26942034)
\curveto(94.86079557,700.32941743)(94.87579556,700.39441737)(94.89580404,700.46442034)
\curveto(94.91579552,700.54441722)(94.92079551,700.62441714)(94.91080404,700.70442034)
\curveto(94.91079552,700.78441698)(94.92079551,700.8644169)(94.94080404,700.94442034)
\lineto(94.94080404,701.09442034)
\curveto(94.96079547,701.15441661)(94.97079546,701.23941652)(94.97080404,701.34942034)
\curveto(94.97079546,701.4594163)(94.96079547,701.54441622)(94.94080404,701.60442034)
\moveto(92.84080404,701.06442034)
\curveto(92.8307976,701.01441675)(92.82579761,700.9644168)(92.82580404,700.91442034)
\lineto(92.82580404,700.77942034)
\curveto(92.81579762,700.73941702)(92.81079762,700.69941706)(92.81080404,700.65942034)
\curveto(92.81079762,700.62941713)(92.80579763,700.59441717)(92.79580404,700.55442034)
\curveto(92.76579767,700.44441732)(92.74079769,700.33941742)(92.72080404,700.23942034)
\curveto(92.70079773,700.13941762)(92.67079776,700.03941772)(92.63080404,699.93942034)
\curveto(92.52079791,699.68941807)(92.38579805,699.47941828)(92.22580404,699.30942034)
\curveto(92.06579837,699.13941862)(91.85579858,699.00441876)(91.59580404,698.90442034)
\curveto(91.52579891,698.87441889)(91.45079898,698.85441891)(91.37080404,698.84442034)
\curveto(91.29079914,698.83441893)(91.21079922,698.81941894)(91.13080404,698.79942034)
\lineto(91.01080404,698.79942034)
\curveto(90.97079946,698.78941897)(90.92579951,698.78441898)(90.87580404,698.78442034)
\lineto(90.75580404,698.81442034)
\curveto(90.71579972,698.82441894)(90.68079975,698.82441894)(90.65080404,698.81442034)
\curveto(90.62079981,698.81441895)(90.58579985,698.81941894)(90.54580404,698.82942034)
\curveto(90.45579998,698.84941891)(90.36580007,698.87441889)(90.27580404,698.90442034)
\curveto(90.19580024,698.93441883)(90.12080031,698.97441879)(90.05080404,699.02442034)
\curveto(89.80080063,699.17441859)(89.61580082,699.33941842)(89.49580404,699.51942034)
\curveto(89.38580105,699.70941805)(89.28080115,699.95441781)(89.18080404,700.25442034)
\curveto(89.16080127,700.33441743)(89.14580129,700.40941735)(89.13580404,700.47942034)
\curveto(89.12580131,700.5594172)(89.11080132,700.63941712)(89.09080404,700.71942034)
\lineto(89.09080404,700.85442034)
\curveto(89.07080136,700.92441684)(89.05580138,701.02941673)(89.04580404,701.16942034)
\curveto(89.04580139,701.30941645)(89.05580138,701.41441635)(89.07580404,701.48442034)
\lineto(89.07580404,701.63442034)
\curveto(89.07580136,701.68441608)(89.08080135,701.73441603)(89.09080404,701.78442034)
\curveto(89.11080132,701.89441587)(89.12580131,702.00441576)(89.13580404,702.11442034)
\curveto(89.15580128,702.22441554)(89.18080125,702.32941543)(89.21080404,702.42942034)
\curveto(89.30080113,702.69941506)(89.42080101,702.93441483)(89.57080404,703.13442034)
\curveto(89.7308007,703.34441442)(89.9358005,703.50441426)(90.18580404,703.61442034)
\curveto(90.2358002,703.64441412)(90.29080014,703.6644141)(90.35080404,703.67442034)
\lineto(90.56080404,703.73442034)
\curveto(90.59079984,703.74441402)(90.62579981,703.74441402)(90.66580404,703.73442034)
\curveto(90.70579973,703.73441403)(90.74079969,703.74441402)(90.77080404,703.76442034)
\lineto(91.04080404,703.76442034)
\curveto(91.1307993,703.77441399)(91.21579922,703.76941399)(91.29580404,703.74942034)
\curveto(91.36579907,703.72941403)(91.430799,703.70941405)(91.49080404,703.68942034)
\curveto(91.55079888,703.67941408)(91.61079882,703.6644141)(91.67080404,703.64442034)
\curveto(91.92079851,703.53441423)(92.12079831,703.38441438)(92.27080404,703.19442034)
\curveto(92.42079801,703.01441475)(92.55079788,702.79441497)(92.66080404,702.53442034)
\curveto(92.69079774,702.45441531)(92.71079772,702.36941539)(92.72080404,702.27942034)
\lineto(92.78080404,702.03942034)
\curveto(92.79079764,702.01941574)(92.79579764,701.98941577)(92.79580404,701.94942034)
\curveto(92.80579763,701.89941586)(92.81079762,701.84441592)(92.81080404,701.78442034)
\curveto(92.81079762,701.72441604)(92.82079761,701.66941609)(92.84080404,701.61942034)
\lineto(92.84080404,701.49942034)
\curveto(92.85079758,701.44941631)(92.85579758,701.37441639)(92.85580404,701.27442034)
\curveto(92.85579758,701.18441658)(92.85079758,701.11441665)(92.84080404,701.06442034)
\moveto(91.61080404,708.23442034)
\lineto(92.67580404,708.23442034)
\curveto(92.75579768,708.23440953)(92.85079758,708.23440953)(92.96080404,708.23442034)
\curveto(93.07079736,708.23440953)(93.15079728,708.21940954)(93.20080404,708.18942034)
\curveto(93.22079721,708.17940958)(93.2307972,708.1644096)(93.23080404,708.14442034)
\curveto(93.24079719,708.13440963)(93.25579718,708.12440964)(93.27580404,708.11442034)
\curveto(93.28579715,707.99440977)(93.2357972,707.88940987)(93.12580404,707.79942034)
\curveto(93.02579741,707.70941005)(92.94079749,707.62941013)(92.87080404,707.55942034)
\curveto(92.79079764,707.48941027)(92.71079772,707.41441035)(92.63080404,707.33442034)
\curveto(92.56079787,707.2644105)(92.48579795,707.19941056)(92.40580404,707.13942034)
\curveto(92.36579807,707.10941065)(92.3307981,707.07441069)(92.30080404,707.03442034)
\curveto(92.28079815,707.00441076)(92.25079818,706.97941078)(92.21080404,706.95942034)
\curveto(92.19079824,706.92941083)(92.16579827,706.90441086)(92.13580404,706.88442034)
\lineto(91.98580404,706.73442034)
\lineto(91.83580404,706.61442034)
\lineto(91.79080404,706.56942034)
\curveto(91.79079864,706.5594112)(91.78079865,706.54441122)(91.76080404,706.52442034)
\curveto(91.68079875,706.4644113)(91.60079883,706.39941136)(91.52080404,706.32942034)
\curveto(91.45079898,706.2594115)(91.36079907,706.20441156)(91.25080404,706.16442034)
\curveto(91.21079922,706.15441161)(91.17079926,706.14941161)(91.13080404,706.14942034)
\curveto(91.10079933,706.14941161)(91.06079937,706.14441162)(91.01080404,706.13442034)
\curveto(90.98079945,706.12441164)(90.94079949,706.11941164)(90.89080404,706.11942034)
\curveto(90.84079959,706.12941163)(90.79579964,706.13441163)(90.75580404,706.13442034)
\lineto(90.41080404,706.13442034)
\curveto(90.29080014,706.13441163)(90.20080023,706.1594116)(90.14080404,706.20942034)
\curveto(90.08080035,706.24941151)(90.06580037,706.31941144)(90.09580404,706.41942034)
\curveto(90.11580032,706.49941126)(90.15080028,706.56941119)(90.20080404,706.62942034)
\curveto(90.25080018,706.69941106)(90.29580014,706.76941099)(90.33580404,706.83942034)
\curveto(90.4358,706.97941078)(90.5307999,707.11441065)(90.62080404,707.24442034)
\curveto(90.71079972,707.37441039)(90.80079963,707.50941025)(90.89080404,707.64942034)
\curveto(90.94079949,707.72941003)(90.99079944,707.81440995)(91.04080404,707.90442034)
\curveto(91.10079933,707.99440977)(91.16579927,708.0644097)(91.23580404,708.11442034)
\curveto(91.27579916,708.14440962)(91.34579909,708.17940958)(91.44580404,708.21942034)
\curveto(91.46579897,708.22940953)(91.49079894,708.22940953)(91.52080404,708.21942034)
\curveto(91.56079887,708.21940954)(91.59079884,708.22440954)(91.61080404,708.23442034)
}
}
{
\newrgbcolor{curcolor}{0 0 0}
\pscustom[linestyle=none,fillstyle=solid,fillcolor=curcolor]
{
\newpath
\moveto(100.76572592,705.35442034)
\curveto(101.36572011,705.37441239)(101.86571961,705.28941247)(102.26572592,705.09942034)
\curveto(102.66571881,704.90941285)(102.9807185,704.62941313)(103.21072592,704.25942034)
\curveto(103.2807182,704.14941361)(103.33571814,704.02941373)(103.37572592,703.89942034)
\curveto(103.41571806,703.77941398)(103.45571802,703.65441411)(103.49572592,703.52442034)
\curveto(103.51571796,703.44441432)(103.52571795,703.36941439)(103.52572592,703.29942034)
\curveto(103.53571794,703.22941453)(103.55071793,703.1594146)(103.57072592,703.08942034)
\curveto(103.57071791,703.02941473)(103.5757179,702.98941477)(103.58572592,702.96942034)
\curveto(103.60571787,702.82941493)(103.61571786,702.68441508)(103.61572592,702.53442034)
\lineto(103.61572592,702.09942034)
\lineto(103.61572592,700.76442034)
\lineto(103.61572592,698.33442034)
\curveto(103.61571786,698.14441962)(103.61071787,697.9594198)(103.60072592,697.77942034)
\curveto(103.60071788,697.60942015)(103.53071795,697.49942026)(103.39072592,697.44942034)
\curveto(103.33071815,697.42942033)(103.26071822,697.41942034)(103.18072592,697.41942034)
\lineto(102.94072592,697.41942034)
\lineto(102.13072592,697.41942034)
\curveto(102.01071947,697.41942034)(101.90071958,697.42442034)(101.80072592,697.43442034)
\curveto(101.71071977,697.45442031)(101.64071984,697.49942026)(101.59072592,697.56942034)
\curveto(101.55071993,697.62942013)(101.52571995,697.70442006)(101.51572592,697.79442034)
\lineto(101.51572592,698.10942034)
\lineto(101.51572592,699.15942034)
\lineto(101.51572592,701.39442034)
\curveto(101.51571996,701.764416)(101.50071998,702.10441566)(101.47072592,702.41442034)
\curveto(101.44072004,702.73441503)(101.35072013,703.00441476)(101.20072592,703.22442034)
\curveto(101.06072042,703.42441434)(100.85572062,703.5644142)(100.58572592,703.64442034)
\curveto(100.53572094,703.6644141)(100.480721,703.67441409)(100.42072592,703.67442034)
\curveto(100.37072111,703.67441409)(100.31572116,703.68441408)(100.25572592,703.70442034)
\curveto(100.20572127,703.71441405)(100.14072134,703.71441405)(100.06072592,703.70442034)
\curveto(99.99072149,703.70441406)(99.93572154,703.69941406)(99.89572592,703.68942034)
\curveto(99.85572162,703.67941408)(99.82072166,703.67441409)(99.79072592,703.67442034)
\curveto(99.76072172,703.67441409)(99.73072175,703.66941409)(99.70072592,703.65942034)
\curveto(99.47072201,703.59941416)(99.28572219,703.51941424)(99.14572592,703.41942034)
\curveto(98.82572265,703.18941457)(98.63572284,702.85441491)(98.57572592,702.41442034)
\curveto(98.51572296,701.97441579)(98.48572299,701.47941628)(98.48572592,700.92942034)
\lineto(98.48572592,699.05442034)
\lineto(98.48572592,698.13942034)
\lineto(98.48572592,697.86942034)
\curveto(98.48572299,697.77941998)(98.47072301,697.70442006)(98.44072592,697.64442034)
\curveto(98.39072309,697.53442023)(98.31072317,697.46942029)(98.20072592,697.44942034)
\curveto(98.09072339,697.42942033)(97.95572352,697.41942034)(97.79572592,697.41942034)
\lineto(97.04572592,697.41942034)
\curveto(96.93572454,697.41942034)(96.82572465,697.42442034)(96.71572592,697.43442034)
\curveto(96.60572487,697.44442032)(96.52572495,697.47942028)(96.47572592,697.53942034)
\curveto(96.40572507,697.62942013)(96.37072511,697.75942)(96.37072592,697.92942034)
\curveto(96.3807251,698.09941966)(96.38572509,698.2594195)(96.38572592,698.40942034)
\lineto(96.38572592,700.44942034)
\lineto(96.38572592,703.74942034)
\lineto(96.38572592,704.51442034)
\lineto(96.38572592,704.81442034)
\curveto(96.39572508,704.90441286)(96.42572505,704.97941278)(96.47572592,705.03942034)
\curveto(96.49572498,705.06941269)(96.52572495,705.08941267)(96.56572592,705.09942034)
\curveto(96.61572486,705.11941264)(96.66572481,705.13441263)(96.71572592,705.14442034)
\lineto(96.79072592,705.14442034)
\curveto(96.84072464,705.15441261)(96.89072459,705.1594126)(96.94072592,705.15942034)
\lineto(97.10572592,705.15942034)
\lineto(97.73572592,705.15942034)
\curveto(97.81572366,705.1594126)(97.89072359,705.15441261)(97.96072592,705.14442034)
\curveto(98.04072344,705.14441262)(98.11072337,705.13441263)(98.17072592,705.11442034)
\curveto(98.24072324,705.08441268)(98.28572319,705.03941272)(98.30572592,704.97942034)
\curveto(98.33572314,704.91941284)(98.36072312,704.84941291)(98.38072592,704.76942034)
\curveto(98.39072309,704.72941303)(98.39072309,704.69441307)(98.38072592,704.66442034)
\curveto(98.3807231,704.63441313)(98.39072309,704.60441316)(98.41072592,704.57442034)
\curveto(98.43072305,704.52441324)(98.44572303,704.49441327)(98.45572592,704.48442034)
\curveto(98.475723,704.47441329)(98.50072298,704.4594133)(98.53072592,704.43942034)
\curveto(98.64072284,704.42941333)(98.73072275,704.4644133)(98.80072592,704.54442034)
\curveto(98.87072261,704.63441313)(98.94572253,704.70441306)(99.02572592,704.75442034)
\curveto(99.29572218,704.95441281)(99.59572188,705.11441265)(99.92572592,705.23442034)
\curveto(100.01572146,705.2644125)(100.10572137,705.28441248)(100.19572592,705.29442034)
\curveto(100.29572118,705.30441246)(100.40072108,705.31941244)(100.51072592,705.33942034)
\curveto(100.54072094,705.34941241)(100.58572089,705.34941241)(100.64572592,705.33942034)
\curveto(100.70572077,705.33941242)(100.74572073,705.34441242)(100.76572592,705.35442034)
}
}
{
\newrgbcolor{curcolor}{0 0 0}
\pscustom[linestyle=none,fillstyle=solid,fillcolor=curcolor]
{
}
}
{
\newrgbcolor{curcolor}{0 0 0}
\pscustom[linestyle=none,fillstyle=solid,fillcolor=curcolor]
{
\newpath
\moveto(116.99713217,698.27442034)
\lineto(116.99713217,697.85442034)
\curveto(116.9971238,697.72442004)(116.96712383,697.61942014)(116.90713217,697.53942034)
\curveto(116.85712394,697.48942027)(116.792124,697.45442031)(116.71213217,697.43442034)
\curveto(116.63212416,697.42442034)(116.54212425,697.41942034)(116.44213217,697.41942034)
\lineto(115.61713217,697.41942034)
\lineto(115.33213217,697.41942034)
\curveto(115.25212554,697.42942033)(115.18712561,697.45442031)(115.13713217,697.49442034)
\curveto(115.06712573,697.54442022)(115.02712577,697.60942015)(115.01713217,697.68942034)
\curveto(115.00712579,697.76941999)(114.98712581,697.84941991)(114.95713217,697.92942034)
\curveto(114.93712586,697.94941981)(114.91712588,697.9644198)(114.89713217,697.97442034)
\curveto(114.88712591,697.99441977)(114.87212592,698.01441975)(114.85213217,698.03442034)
\curveto(114.74212605,698.03441973)(114.66212613,698.00941975)(114.61213217,697.95942034)
\lineto(114.46213217,697.80942034)
\curveto(114.3921264,697.75942)(114.32712647,697.71442005)(114.26713217,697.67442034)
\curveto(114.20712659,697.64442012)(114.14212665,697.60442016)(114.07213217,697.55442034)
\curveto(114.03212676,697.53442023)(113.98712681,697.51442025)(113.93713217,697.49442034)
\curveto(113.8971269,697.47442029)(113.85212694,697.45442031)(113.80213217,697.43442034)
\curveto(113.66212713,697.38442038)(113.51212728,697.33942042)(113.35213217,697.29942034)
\curveto(113.30212749,697.27942048)(113.25712754,697.26942049)(113.21713217,697.26942034)
\curveto(113.17712762,697.26942049)(113.13712766,697.2644205)(113.09713217,697.25442034)
\lineto(112.96213217,697.25442034)
\curveto(112.93212786,697.24442052)(112.8921279,697.23942052)(112.84213217,697.23942034)
\lineto(112.70713217,697.23942034)
\curveto(112.64712815,697.21942054)(112.55712824,697.21442055)(112.43713217,697.22442034)
\curveto(112.31712848,697.22442054)(112.23212856,697.23442053)(112.18213217,697.25442034)
\curveto(112.11212868,697.27442049)(112.04712875,697.28442048)(111.98713217,697.28442034)
\curveto(111.93712886,697.27442049)(111.88212891,697.27942048)(111.82213217,697.29942034)
\lineto(111.46213217,697.41942034)
\curveto(111.35212944,697.44942031)(111.24212955,697.48942027)(111.13213217,697.53942034)
\curveto(110.78213001,697.68942007)(110.46713033,697.91941984)(110.18713217,698.22942034)
\curveto(109.91713088,698.54941921)(109.70213109,698.88441888)(109.54213217,699.23442034)
\curveto(109.4921313,699.34441842)(109.45213134,699.44941831)(109.42213217,699.54942034)
\curveto(109.3921314,699.6594181)(109.35713144,699.76941799)(109.31713217,699.87942034)
\curveto(109.30713149,699.91941784)(109.30213149,699.95441781)(109.30213217,699.98442034)
\curveto(109.30213149,700.02441774)(109.2921315,700.06941769)(109.27213217,700.11942034)
\curveto(109.25213154,700.19941756)(109.23213156,700.28441748)(109.21213217,700.37442034)
\curveto(109.20213159,700.47441729)(109.18713161,700.57441719)(109.16713217,700.67442034)
\curveto(109.15713164,700.70441706)(109.15213164,700.73941702)(109.15213217,700.77942034)
\curveto(109.16213163,700.81941694)(109.16213163,700.85441691)(109.15213217,700.88442034)
\lineto(109.15213217,701.01942034)
\curveto(109.15213164,701.06941669)(109.14713165,701.11941664)(109.13713217,701.16942034)
\curveto(109.12713167,701.21941654)(109.12213167,701.27441649)(109.12213217,701.33442034)
\curveto(109.12213167,701.40441636)(109.12713167,701.4594163)(109.13713217,701.49942034)
\curveto(109.14713165,701.54941621)(109.15213164,701.59441617)(109.15213217,701.63442034)
\lineto(109.15213217,701.78442034)
\curveto(109.16213163,701.83441593)(109.16213163,701.87941588)(109.15213217,701.91942034)
\curveto(109.15213164,701.96941579)(109.16213163,702.01941574)(109.18213217,702.06942034)
\curveto(109.20213159,702.17941558)(109.21713158,702.28441548)(109.22713217,702.38442034)
\curveto(109.24713155,702.48441528)(109.27213152,702.58441518)(109.30213217,702.68442034)
\curveto(109.34213145,702.80441496)(109.37713142,702.91941484)(109.40713217,703.02942034)
\curveto(109.43713136,703.13941462)(109.47713132,703.24941451)(109.52713217,703.35942034)
\curveto(109.66713113,703.6594141)(109.84213095,703.94441382)(110.05213217,704.21442034)
\curveto(110.07213072,704.24441352)(110.0971307,704.26941349)(110.12713217,704.28942034)
\curveto(110.16713063,704.31941344)(110.1971306,704.34941341)(110.21713217,704.37942034)
\curveto(110.25713054,704.42941333)(110.2971305,704.47441329)(110.33713217,704.51442034)
\curveto(110.37713042,704.55441321)(110.42213037,704.59441317)(110.47213217,704.63442034)
\curveto(110.51213028,704.65441311)(110.54713025,704.67941308)(110.57713217,704.70942034)
\curveto(110.60713019,704.74941301)(110.64213015,704.77941298)(110.68213217,704.79942034)
\curveto(110.93212986,704.96941279)(111.22212957,705.10941265)(111.55213217,705.21942034)
\curveto(111.62212917,705.23941252)(111.6921291,705.25441251)(111.76213217,705.26442034)
\curveto(111.84212895,705.27441249)(111.92212887,705.28941247)(112.00213217,705.30942034)
\curveto(112.07212872,705.32941243)(112.16212863,705.33941242)(112.27213217,705.33942034)
\curveto(112.38212841,705.34941241)(112.4921283,705.35441241)(112.60213217,705.35442034)
\curveto(112.71212808,705.35441241)(112.81712798,705.34941241)(112.91713217,705.33942034)
\curveto(113.02712777,705.32941243)(113.11712768,705.31441245)(113.18713217,705.29442034)
\curveto(113.33712746,705.24441252)(113.48212731,705.19941256)(113.62213217,705.15942034)
\curveto(113.76212703,705.11941264)(113.8921269,705.0644127)(114.01213217,704.99442034)
\curveto(114.08212671,704.94441282)(114.14712665,704.89441287)(114.20713217,704.84442034)
\curveto(114.26712653,704.80441296)(114.33212646,704.759413)(114.40213217,704.70942034)
\curveto(114.44212635,704.67941308)(114.4971263,704.63941312)(114.56713217,704.58942034)
\curveto(114.64712615,704.53941322)(114.72212607,704.53941322)(114.79213217,704.58942034)
\curveto(114.83212596,704.60941315)(114.85212594,704.64441312)(114.85213217,704.69442034)
\curveto(114.85212594,704.74441302)(114.86212593,704.79441297)(114.88213217,704.84442034)
\lineto(114.88213217,704.99442034)
\curveto(114.8921259,705.02441274)(114.8971259,705.0594127)(114.89713217,705.09942034)
\lineto(114.89713217,705.21942034)
\lineto(114.89713217,707.25942034)
\curveto(114.8971259,707.36941039)(114.8921259,707.48941027)(114.88213217,707.61942034)
\curveto(114.88212591,707.75941)(114.90712589,707.8644099)(114.95713217,707.93442034)
\curveto(114.9971258,708.01440975)(115.07212572,708.0644097)(115.18213217,708.08442034)
\curveto(115.20212559,708.09440967)(115.22212557,708.09440967)(115.24213217,708.08442034)
\curveto(115.26212553,708.08440968)(115.28212551,708.08940967)(115.30213217,708.09942034)
\lineto(116.36713217,708.09942034)
\curveto(116.48712431,708.09940966)(116.5971242,708.09440967)(116.69713217,708.08442034)
\curveto(116.797124,708.07440969)(116.87212392,708.03440973)(116.92213217,707.96442034)
\curveto(116.97212382,707.88440988)(116.9971238,707.77940998)(116.99713217,707.64942034)
\lineto(116.99713217,707.28942034)
\lineto(116.99713217,698.27442034)
\moveto(114.95713217,701.21442034)
\curveto(114.96712583,701.25441651)(114.96712583,701.29441647)(114.95713217,701.33442034)
\lineto(114.95713217,701.46942034)
\curveto(114.95712584,701.56941619)(114.95212584,701.66941609)(114.94213217,701.76942034)
\curveto(114.93212586,701.86941589)(114.91712588,701.9594158)(114.89713217,702.03942034)
\curveto(114.87712592,702.14941561)(114.85712594,702.24941551)(114.83713217,702.33942034)
\curveto(114.82712597,702.42941533)(114.80212599,702.51441525)(114.76213217,702.59442034)
\curveto(114.62212617,702.95441481)(114.41712638,703.23941452)(114.14713217,703.44942034)
\curveto(113.88712691,703.6594141)(113.50712729,703.764414)(113.00713217,703.76442034)
\curveto(112.94712785,703.764414)(112.86712793,703.75441401)(112.76713217,703.73442034)
\curveto(112.68712811,703.71441405)(112.61212818,703.69441407)(112.54213217,703.67442034)
\curveto(112.48212831,703.6644141)(112.42212837,703.64441412)(112.36213217,703.61442034)
\curveto(112.0921287,703.50441426)(111.88212891,703.33441443)(111.73213217,703.10442034)
\curveto(111.58212921,702.87441489)(111.46212933,702.61441515)(111.37213217,702.32442034)
\curveto(111.34212945,702.22441554)(111.32212947,702.12441564)(111.31213217,702.02442034)
\curveto(111.30212949,701.92441584)(111.28212951,701.81941594)(111.25213217,701.70942034)
\lineto(111.25213217,701.49942034)
\curveto(111.23212956,701.40941635)(111.22712957,701.28441648)(111.23713217,701.12442034)
\curveto(111.24712955,700.97441679)(111.26212953,700.8644169)(111.28213217,700.79442034)
\lineto(111.28213217,700.70442034)
\curveto(111.2921295,700.68441708)(111.2971295,700.6644171)(111.29713217,700.64442034)
\curveto(111.31712948,700.5644172)(111.33212946,700.48941727)(111.34213217,700.41942034)
\curveto(111.36212943,700.34941741)(111.38212941,700.27441749)(111.40213217,700.19442034)
\curveto(111.57212922,699.67441809)(111.86212893,699.28941847)(112.27213217,699.03942034)
\curveto(112.40212839,698.94941881)(112.58212821,698.87941888)(112.81213217,698.82942034)
\curveto(112.85212794,698.81941894)(112.91212788,698.81441895)(112.99213217,698.81442034)
\curveto(113.02212777,698.80441896)(113.06712773,698.79441897)(113.12713217,698.78442034)
\curveto(113.1971276,698.78441898)(113.25212754,698.78941897)(113.29213217,698.79942034)
\curveto(113.37212742,698.81941894)(113.45212734,698.83441893)(113.53213217,698.84442034)
\curveto(113.61212718,698.85441891)(113.6921271,698.87441889)(113.77213217,698.90442034)
\curveto(114.02212677,699.01441875)(114.22212657,699.15441861)(114.37213217,699.32442034)
\curveto(114.52212627,699.49441827)(114.65212614,699.70941805)(114.76213217,699.96942034)
\curveto(114.80212599,700.0594177)(114.83212596,700.14941761)(114.85213217,700.23942034)
\curveto(114.87212592,700.33941742)(114.8921259,700.44441732)(114.91213217,700.55442034)
\curveto(114.92212587,700.60441716)(114.92212587,700.64941711)(114.91213217,700.68942034)
\curveto(114.91212588,700.73941702)(114.92212587,700.78941697)(114.94213217,700.83942034)
\curveto(114.95212584,700.86941689)(114.95712584,700.90441686)(114.95713217,700.94442034)
\lineto(114.95713217,701.07942034)
\lineto(114.95713217,701.21442034)
}
}
{
\newrgbcolor{curcolor}{0 0 0}
\pscustom[linestyle=none,fillstyle=solid,fillcolor=curcolor]
{
\newpath
\moveto(125.94205404,701.36442034)
\curveto(125.96204588,701.28441648)(125.96204588,701.19441657)(125.94205404,701.09442034)
\curveto(125.92204592,700.99441677)(125.88704595,700.92941683)(125.83705404,700.89942034)
\curveto(125.78704605,700.8594169)(125.71204613,700.82941693)(125.61205404,700.80942034)
\curveto(125.52204632,700.79941696)(125.41704642,700.78941697)(125.29705404,700.77942034)
\lineto(124.95205404,700.77942034)
\curveto(124.842047,700.78941697)(124.7420471,700.79441697)(124.65205404,700.79442034)
\lineto(120.99205404,700.79442034)
\lineto(120.78205404,700.79442034)
\curveto(120.72205112,700.79441697)(120.66705117,700.78441698)(120.61705404,700.76442034)
\curveto(120.5370513,700.72441704)(120.48705135,700.68441708)(120.46705404,700.64442034)
\curveto(120.44705139,700.62441714)(120.42705141,700.58441718)(120.40705404,700.52442034)
\curveto(120.38705145,700.47441729)(120.38205146,700.42441734)(120.39205404,700.37442034)
\curveto(120.41205143,700.31441745)(120.42205142,700.25441751)(120.42205404,700.19442034)
\curveto(120.43205141,700.14441762)(120.44705139,700.08941767)(120.46705404,700.02942034)
\curveto(120.54705129,699.78941797)(120.6420512,699.58941817)(120.75205404,699.42942034)
\curveto(120.87205097,699.27941848)(121.03205081,699.14441862)(121.23205404,699.02442034)
\curveto(121.31205053,698.97441879)(121.39205045,698.93941882)(121.47205404,698.91942034)
\curveto(121.56205028,698.90941885)(121.65205019,698.88941887)(121.74205404,698.85942034)
\curveto(121.82205002,698.83941892)(121.93204991,698.82441894)(122.07205404,698.81442034)
\curveto(122.21204963,698.80441896)(122.33204951,698.80941895)(122.43205404,698.82942034)
\lineto(122.56705404,698.82942034)
\curveto(122.66704917,698.84941891)(122.75704908,698.86941889)(122.83705404,698.88942034)
\curveto(122.92704891,698.91941884)(123.01204883,698.94941881)(123.09205404,698.97942034)
\curveto(123.19204865,699.02941873)(123.30204854,699.09441867)(123.42205404,699.17442034)
\curveto(123.55204829,699.25441851)(123.64704819,699.33441843)(123.70705404,699.41442034)
\curveto(123.75704808,699.48441828)(123.80704803,699.54941821)(123.85705404,699.60942034)
\curveto(123.91704792,699.67941808)(123.98704785,699.72941803)(124.06705404,699.75942034)
\curveto(124.16704767,699.80941795)(124.29204755,699.82941793)(124.44205404,699.81942034)
\lineto(124.87705404,699.81942034)
\lineto(125.05705404,699.81942034)
\curveto(125.12704671,699.82941793)(125.18704665,699.82441794)(125.23705404,699.80442034)
\lineto(125.38705404,699.80442034)
\curveto(125.48704635,699.78441798)(125.55704628,699.759418)(125.59705404,699.72942034)
\curveto(125.6370462,699.70941805)(125.65704618,699.6644181)(125.65705404,699.59442034)
\curveto(125.66704617,699.52441824)(125.66204618,699.4644183)(125.64205404,699.41442034)
\curveto(125.59204625,699.27441849)(125.5370463,699.14941861)(125.47705404,699.03942034)
\curveto(125.41704642,698.92941883)(125.34704649,698.81941894)(125.26705404,698.70942034)
\curveto(125.04704679,698.37941938)(124.79704704,698.11441965)(124.51705404,697.91442034)
\curveto(124.2370476,697.71442005)(123.88704795,697.54442022)(123.46705404,697.40442034)
\curveto(123.35704848,697.3644204)(123.24704859,697.33942042)(123.13705404,697.32942034)
\curveto(123.02704881,697.31942044)(122.91204893,697.29942046)(122.79205404,697.26942034)
\curveto(122.75204909,697.2594205)(122.70704913,697.2594205)(122.65705404,697.26942034)
\curveto(122.61704922,697.26942049)(122.57704926,697.2644205)(122.53705404,697.25442034)
\lineto(122.37205404,697.25442034)
\curveto(122.32204952,697.23442053)(122.26204958,697.22942053)(122.19205404,697.23942034)
\curveto(122.13204971,697.23942052)(122.07704976,697.24442052)(122.02705404,697.25442034)
\curveto(121.94704989,697.2644205)(121.87704996,697.2644205)(121.81705404,697.25442034)
\curveto(121.75705008,697.24442052)(121.69205015,697.24942051)(121.62205404,697.26942034)
\curveto(121.57205027,697.28942047)(121.51705032,697.29942046)(121.45705404,697.29942034)
\curveto(121.39705044,697.29942046)(121.3420505,697.30942045)(121.29205404,697.32942034)
\curveto(121.18205066,697.34942041)(121.07205077,697.37442039)(120.96205404,697.40442034)
\curveto(120.85205099,697.42442034)(120.75205109,697.4594203)(120.66205404,697.50942034)
\curveto(120.55205129,697.54942021)(120.44705139,697.58442018)(120.34705404,697.61442034)
\curveto(120.25705158,697.65442011)(120.17205167,697.69942006)(120.09205404,697.74942034)
\curveto(119.77205207,697.94941981)(119.48705235,698.17941958)(119.23705404,698.43942034)
\curveto(118.98705285,698.70941905)(118.78205306,699.01941874)(118.62205404,699.36942034)
\curveto(118.57205327,699.47941828)(118.53205331,699.58941817)(118.50205404,699.69942034)
\curveto(118.47205337,699.81941794)(118.43205341,699.93941782)(118.38205404,700.05942034)
\curveto(118.37205347,700.09941766)(118.36705347,700.13441763)(118.36705404,700.16442034)
\curveto(118.36705347,700.20441756)(118.36205348,700.24441752)(118.35205404,700.28442034)
\curveto(118.31205353,700.40441736)(118.28705355,700.53441723)(118.27705404,700.67442034)
\lineto(118.24705404,701.09442034)
\curveto(118.24705359,701.14441662)(118.2420536,701.19941656)(118.23205404,701.25942034)
\curveto(118.23205361,701.31941644)(118.2370536,701.37441639)(118.24705404,701.42442034)
\lineto(118.24705404,701.60442034)
\lineto(118.29205404,701.96442034)
\curveto(118.33205351,702.13441563)(118.36705347,702.29941546)(118.39705404,702.45942034)
\curveto(118.42705341,702.61941514)(118.47205337,702.76941499)(118.53205404,702.90942034)
\curveto(118.96205288,703.94941381)(119.69205215,704.68441308)(120.72205404,705.11442034)
\curveto(120.86205098,705.17441259)(121.00205084,705.21441255)(121.14205404,705.23442034)
\curveto(121.29205055,705.2644125)(121.44705039,705.29941246)(121.60705404,705.33942034)
\curveto(121.68705015,705.34941241)(121.76205008,705.35441241)(121.83205404,705.35442034)
\curveto(121.90204994,705.35441241)(121.97704986,705.3594124)(122.05705404,705.36942034)
\curveto(122.56704927,705.37941238)(123.00204884,705.31941244)(123.36205404,705.18942034)
\curveto(123.73204811,705.06941269)(124.06204778,704.90941285)(124.35205404,704.70942034)
\curveto(124.4420474,704.64941311)(124.53204731,704.57941318)(124.62205404,704.49942034)
\curveto(124.71204713,704.42941333)(124.79204705,704.35441341)(124.86205404,704.27442034)
\curveto(124.89204695,704.22441354)(124.93204691,704.18441358)(124.98205404,704.15442034)
\curveto(125.06204678,704.04441372)(125.1370467,703.92941383)(125.20705404,703.80942034)
\curveto(125.27704656,703.69941406)(125.35204649,703.58441418)(125.43205404,703.46442034)
\curveto(125.48204636,703.37441439)(125.52204632,703.27941448)(125.55205404,703.17942034)
\curveto(125.59204625,703.08941467)(125.63204621,702.98941477)(125.67205404,702.87942034)
\curveto(125.72204612,702.74941501)(125.76204608,702.61441515)(125.79205404,702.47442034)
\curveto(125.82204602,702.33441543)(125.85704598,702.19441557)(125.89705404,702.05442034)
\curveto(125.91704592,701.97441579)(125.92204592,701.88441588)(125.91205404,701.78442034)
\curveto(125.91204593,701.69441607)(125.92204592,701.60941615)(125.94205404,701.52942034)
\lineto(125.94205404,701.36442034)
\moveto(123.69205404,702.24942034)
\curveto(123.76204808,702.34941541)(123.76704807,702.46941529)(123.70705404,702.60942034)
\curveto(123.65704818,702.759415)(123.61704822,702.86941489)(123.58705404,702.93942034)
\curveto(123.44704839,703.20941455)(123.26204858,703.41441435)(123.03205404,703.55442034)
\curveto(122.80204904,703.70441406)(122.48204936,703.78441398)(122.07205404,703.79442034)
\curveto(122.0420498,703.77441399)(122.00704983,703.76941399)(121.96705404,703.77942034)
\curveto(121.92704991,703.78941397)(121.89204995,703.78941397)(121.86205404,703.77942034)
\curveto(121.81205003,703.759414)(121.75705008,703.74441402)(121.69705404,703.73442034)
\curveto(121.6370502,703.73441403)(121.58205026,703.72441404)(121.53205404,703.70442034)
\curveto(121.09205075,703.5644142)(120.76705107,703.28941447)(120.55705404,702.87942034)
\curveto(120.5370513,702.83941492)(120.51205133,702.78441498)(120.48205404,702.71442034)
\curveto(120.46205138,702.65441511)(120.44705139,702.58941517)(120.43705404,702.51942034)
\curveto(120.42705141,702.4594153)(120.42705141,702.39941536)(120.43705404,702.33942034)
\curveto(120.45705138,702.27941548)(120.49205135,702.22941553)(120.54205404,702.18942034)
\curveto(120.62205122,702.13941562)(120.73205111,702.11441565)(120.87205404,702.11442034)
\lineto(121.27705404,702.11442034)
\lineto(122.94205404,702.11442034)
\lineto(123.37705404,702.11442034)
\curveto(123.5370483,702.12441564)(123.6420482,702.16941559)(123.69205404,702.24942034)
}
}
{
\newrgbcolor{curcolor}{0 0 0}
\pscustom[linestyle=none,fillstyle=solid,fillcolor=curcolor]
{
}
}
{
\newrgbcolor{curcolor}{0 0 0}
\pscustom[linestyle=none,fillstyle=solid,fillcolor=curcolor]
{
\newpath
\moveto(131.86049154,708.11442034)
\lineto(132.95549154,708.11442034)
\curveto(133.05548906,708.11440965)(133.15048896,708.10940965)(133.24049154,708.09942034)
\curveto(133.33048878,708.08940967)(133.40048871,708.0594097)(133.45049154,708.00942034)
\curveto(133.5104886,707.93940982)(133.54048857,707.84440992)(133.54049154,707.72442034)
\curveto(133.55048856,707.61441015)(133.55548856,707.49941026)(133.55549154,707.37942034)
\lineto(133.55549154,706.04442034)
\lineto(133.55549154,700.65942034)
\lineto(133.55549154,698.36442034)
\lineto(133.55549154,697.94442034)
\curveto(133.56548855,697.79441997)(133.54548857,697.67942008)(133.49549154,697.59942034)
\curveto(133.44548867,697.51942024)(133.35548876,697.4644203)(133.22549154,697.43442034)
\curveto(133.16548895,697.41442035)(133.09548902,697.40942035)(133.01549154,697.41942034)
\curveto(132.94548917,697.42942033)(132.87548924,697.43442033)(132.80549154,697.43442034)
\lineto(132.08549154,697.43442034)
\curveto(131.97549014,697.43442033)(131.87549024,697.43942032)(131.78549154,697.44942034)
\curveto(131.69549042,697.4594203)(131.62049049,697.48942027)(131.56049154,697.53942034)
\curveto(131.50049061,697.58942017)(131.46549065,697.6644201)(131.45549154,697.76442034)
\lineto(131.45549154,698.09442034)
\lineto(131.45549154,699.42942034)
\lineto(131.45549154,705.05442034)
\lineto(131.45549154,707.09442034)
\curveto(131.45549066,707.22441054)(131.45049066,707.37941038)(131.44049154,707.55942034)
\curveto(131.44049067,707.73941002)(131.46549065,707.86940989)(131.51549154,707.94942034)
\curveto(131.53549058,707.98940977)(131.56049055,708.01940974)(131.59049154,708.03942034)
\lineto(131.71049154,708.09942034)
\curveto(131.73049038,708.09940966)(131.75549036,708.09940966)(131.78549154,708.09942034)
\curveto(131.8154903,708.10940965)(131.84049027,708.11440965)(131.86049154,708.11442034)
}
}
{
\newrgbcolor{curcolor}{0 0 0}
\pscustom[linestyle=none,fillstyle=solid,fillcolor=curcolor]
{
\newpath
\moveto(142.98767904,701.60442034)
\curveto(143.00767047,701.54441622)(143.01767046,701.4594163)(143.01767904,701.34942034)
\curveto(143.01767046,701.23941652)(143.00767047,701.15441661)(142.98767904,701.09442034)
\lineto(142.98767904,700.94442034)
\curveto(142.96767051,700.8644169)(142.95767052,700.78441698)(142.95767904,700.70442034)
\curveto(142.96767051,700.62441714)(142.96267052,700.54441722)(142.94267904,700.46442034)
\curveto(142.92267056,700.39441737)(142.90767057,700.32941743)(142.89767904,700.26942034)
\curveto(142.88767059,700.20941755)(142.8776706,700.14441762)(142.86767904,700.07442034)
\curveto(142.82767065,699.9644178)(142.79267069,699.84941791)(142.76267904,699.72942034)
\curveto(142.73267075,699.61941814)(142.69267079,699.51441825)(142.64267904,699.41442034)
\curveto(142.43267105,698.93441883)(142.15767132,698.54441922)(141.81767904,698.24442034)
\curveto(141.477672,697.94441982)(141.06767241,697.69442007)(140.58767904,697.49442034)
\curveto(140.46767301,697.44442032)(140.34267314,697.40942035)(140.21267904,697.38942034)
\curveto(140.09267339,697.3594204)(139.96767351,697.32942043)(139.83767904,697.29942034)
\curveto(139.78767369,697.27942048)(139.73267375,697.26942049)(139.67267904,697.26942034)
\curveto(139.61267387,697.26942049)(139.55767392,697.2644205)(139.50767904,697.25442034)
\lineto(139.40267904,697.25442034)
\curveto(139.37267411,697.24442052)(139.34267414,697.23942052)(139.31267904,697.23942034)
\curveto(139.26267422,697.22942053)(139.1826743,697.22442054)(139.07267904,697.22442034)
\curveto(138.96267452,697.21442055)(138.8776746,697.21942054)(138.81767904,697.23942034)
\lineto(138.66767904,697.23942034)
\curveto(138.61767486,697.24942051)(138.56267492,697.25442051)(138.50267904,697.25442034)
\curveto(138.45267503,697.24442052)(138.40267508,697.24942051)(138.35267904,697.26942034)
\curveto(138.31267517,697.27942048)(138.27267521,697.28442048)(138.23267904,697.28442034)
\curveto(138.20267528,697.28442048)(138.16267532,697.28942047)(138.11267904,697.29942034)
\curveto(138.01267547,697.32942043)(137.91267557,697.35442041)(137.81267904,697.37442034)
\curveto(137.71267577,697.39442037)(137.61767586,697.42442034)(137.52767904,697.46442034)
\curveto(137.40767607,697.50442026)(137.29267619,697.54442022)(137.18267904,697.58442034)
\curveto(137.0826764,697.62442014)(136.9776765,697.67442009)(136.86767904,697.73442034)
\curveto(136.51767696,697.94441982)(136.21767726,698.18941957)(135.96767904,698.46942034)
\curveto(135.71767776,698.74941901)(135.50767797,699.08441868)(135.33767904,699.47442034)
\curveto(135.28767819,699.5644182)(135.24767823,699.6594181)(135.21767904,699.75942034)
\curveto(135.19767828,699.8594179)(135.17267831,699.9644178)(135.14267904,700.07442034)
\curveto(135.12267836,700.12441764)(135.11267837,700.16941759)(135.11267904,700.20942034)
\curveto(135.11267837,700.24941751)(135.10267838,700.29441747)(135.08267904,700.34442034)
\curveto(135.06267842,700.42441734)(135.05267843,700.50441726)(135.05267904,700.58442034)
\curveto(135.05267843,700.67441709)(135.04267844,700.759417)(135.02267904,700.83942034)
\curveto(135.01267847,700.88941687)(135.00767847,700.93441683)(135.00767904,700.97442034)
\lineto(135.00767904,701.10942034)
\curveto(134.98767849,701.16941659)(134.9776785,701.25441651)(134.97767904,701.36442034)
\curveto(134.98767849,701.47441629)(135.00267848,701.5594162)(135.02267904,701.61942034)
\lineto(135.02267904,701.72442034)
\curveto(135.03267845,701.77441599)(135.03267845,701.82441594)(135.02267904,701.87442034)
\curveto(135.02267846,701.93441583)(135.03267845,701.98941577)(135.05267904,702.03942034)
\curveto(135.06267842,702.08941567)(135.06767841,702.13441563)(135.06767904,702.17442034)
\curveto(135.06767841,702.22441554)(135.0776784,702.27441549)(135.09767904,702.32442034)
\curveto(135.13767834,702.45441531)(135.17267831,702.57941518)(135.20267904,702.69942034)
\curveto(135.23267825,702.82941493)(135.27267821,702.95441481)(135.32267904,703.07442034)
\curveto(135.50267798,703.48441428)(135.71767776,703.82441394)(135.96767904,704.09442034)
\curveto(136.21767726,704.37441339)(136.52267696,704.62941313)(136.88267904,704.85942034)
\curveto(136.9826765,704.90941285)(137.08767639,704.95441281)(137.19767904,704.99442034)
\curveto(137.30767617,705.03441273)(137.41767606,705.07941268)(137.52767904,705.12942034)
\curveto(137.65767582,705.17941258)(137.79267569,705.21441255)(137.93267904,705.23442034)
\curveto(138.07267541,705.25441251)(138.21767526,705.28441248)(138.36767904,705.32442034)
\curveto(138.44767503,705.33441243)(138.52267496,705.33941242)(138.59267904,705.33942034)
\curveto(138.66267482,705.33941242)(138.73267475,705.34441242)(138.80267904,705.35442034)
\curveto(139.3826741,705.3644124)(139.8826736,705.30441246)(140.30267904,705.17442034)
\curveto(140.73267275,705.04441272)(141.11267237,704.8644129)(141.44267904,704.63442034)
\curveto(141.55267193,704.55441321)(141.66267182,704.4644133)(141.77267904,704.36442034)
\curveto(141.89267159,704.27441349)(141.99267149,704.17441359)(142.07267904,704.06442034)
\curveto(142.15267133,703.9644138)(142.22267126,703.8644139)(142.28267904,703.76442034)
\curveto(142.35267113,703.6644141)(142.42267106,703.5594142)(142.49267904,703.44942034)
\curveto(142.56267092,703.33941442)(142.61767086,703.21941454)(142.65767904,703.08942034)
\curveto(142.69767078,702.96941479)(142.74267074,702.83941492)(142.79267904,702.69942034)
\curveto(142.82267066,702.61941514)(142.84767063,702.53441523)(142.86767904,702.44442034)
\lineto(142.92767904,702.17442034)
\curveto(142.93767054,702.13441563)(142.94267054,702.09441567)(142.94267904,702.05442034)
\curveto(142.94267054,702.01441575)(142.94767053,701.97441579)(142.95767904,701.93442034)
\curveto(142.9776705,701.88441588)(142.9826705,701.82941593)(142.97267904,701.76942034)
\curveto(142.96267052,701.70941605)(142.96767051,701.65441611)(142.98767904,701.60442034)
\moveto(140.88767904,701.06442034)
\curveto(140.89767258,701.11441665)(140.90267258,701.18441658)(140.90267904,701.27442034)
\curveto(140.90267258,701.37441639)(140.89767258,701.44941631)(140.88767904,701.49942034)
\lineto(140.88767904,701.61942034)
\curveto(140.86767261,701.66941609)(140.85767262,701.72441604)(140.85767904,701.78442034)
\curveto(140.85767262,701.84441592)(140.85267263,701.89941586)(140.84267904,701.94942034)
\curveto(140.84267264,701.98941577)(140.83767264,702.01941574)(140.82767904,702.03942034)
\lineto(140.76767904,702.27942034)
\curveto(140.75767272,702.36941539)(140.73767274,702.45441531)(140.70767904,702.53442034)
\curveto(140.59767288,702.79441497)(140.46767301,703.01441475)(140.31767904,703.19442034)
\curveto(140.16767331,703.38441438)(139.96767351,703.53441423)(139.71767904,703.64442034)
\curveto(139.65767382,703.6644141)(139.59767388,703.67941408)(139.53767904,703.68942034)
\curveto(139.477674,703.70941405)(139.41267407,703.72941403)(139.34267904,703.74942034)
\curveto(139.26267422,703.76941399)(139.1776743,703.77441399)(139.08767904,703.76442034)
\lineto(138.81767904,703.76442034)
\curveto(138.78767469,703.74441402)(138.75267473,703.73441403)(138.71267904,703.73442034)
\curveto(138.67267481,703.74441402)(138.63767484,703.74441402)(138.60767904,703.73442034)
\lineto(138.39767904,703.67442034)
\curveto(138.33767514,703.6644141)(138.2826752,703.64441412)(138.23267904,703.61442034)
\curveto(137.9826755,703.50441426)(137.7776757,703.34441442)(137.61767904,703.13442034)
\curveto(137.46767601,702.93441483)(137.34767613,702.69941506)(137.25767904,702.42942034)
\curveto(137.22767625,702.32941543)(137.20267628,702.22441554)(137.18267904,702.11442034)
\curveto(137.17267631,702.00441576)(137.15767632,701.89441587)(137.13767904,701.78442034)
\curveto(137.12767635,701.73441603)(137.12267636,701.68441608)(137.12267904,701.63442034)
\lineto(137.12267904,701.48442034)
\curveto(137.10267638,701.41441635)(137.09267639,701.30941645)(137.09267904,701.16942034)
\curveto(137.10267638,701.02941673)(137.11767636,700.92441684)(137.13767904,700.85442034)
\lineto(137.13767904,700.71942034)
\curveto(137.15767632,700.63941712)(137.17267631,700.5594172)(137.18267904,700.47942034)
\curveto(137.19267629,700.40941735)(137.20767627,700.33441743)(137.22767904,700.25442034)
\curveto(137.32767615,699.95441781)(137.43267605,699.70941805)(137.54267904,699.51942034)
\curveto(137.66267582,699.33941842)(137.84767563,699.17441859)(138.09767904,699.02442034)
\curveto(138.16767531,698.97441879)(138.24267524,698.93441883)(138.32267904,698.90442034)
\curveto(138.41267507,698.87441889)(138.50267498,698.84941891)(138.59267904,698.82942034)
\curveto(138.63267485,698.81941894)(138.66767481,698.81441895)(138.69767904,698.81442034)
\curveto(138.72767475,698.82441894)(138.76267472,698.82441894)(138.80267904,698.81442034)
\lineto(138.92267904,698.78442034)
\curveto(138.97267451,698.78441898)(139.01767446,698.78941897)(139.05767904,698.79942034)
\lineto(139.17767904,698.79942034)
\curveto(139.25767422,698.81941894)(139.33767414,698.83441893)(139.41767904,698.84442034)
\curveto(139.49767398,698.85441891)(139.57267391,698.87441889)(139.64267904,698.90442034)
\curveto(139.90267358,699.00441876)(140.11267337,699.13941862)(140.27267904,699.30942034)
\curveto(140.43267305,699.47941828)(140.56767291,699.68941807)(140.67767904,699.93942034)
\curveto(140.71767276,700.03941772)(140.74767273,700.13941762)(140.76767904,700.23942034)
\curveto(140.78767269,700.33941742)(140.81267267,700.44441732)(140.84267904,700.55442034)
\curveto(140.85267263,700.59441717)(140.85767262,700.62941713)(140.85767904,700.65942034)
\curveto(140.85767262,700.69941706)(140.86267262,700.73941702)(140.87267904,700.77942034)
\lineto(140.87267904,700.91442034)
\curveto(140.87267261,700.9644168)(140.8776726,701.01441675)(140.88767904,701.06442034)
}
}
{
\newrgbcolor{curcolor}{0 0 0}
\pscustom[linestyle=none,fillstyle=solid,fillcolor=curcolor]
{
\newpath
\moveto(147.35760092,705.36942034)
\curveto(148.10759642,705.38941237)(148.75759577,705.30441246)(149.30760092,705.11442034)
\curveto(149.86759466,704.93441283)(150.29259423,704.61941314)(150.58260092,704.16942034)
\curveto(150.65259387,704.0594137)(150.71259381,703.94441382)(150.76260092,703.82442034)
\curveto(150.8225937,703.71441405)(150.87259365,703.58941417)(150.91260092,703.44942034)
\curveto(150.93259359,703.38941437)(150.94259358,703.32441444)(150.94260092,703.25442034)
\curveto(150.94259358,703.18441458)(150.93259359,703.12441464)(150.91260092,703.07442034)
\curveto(150.87259365,703.01441475)(150.81759371,702.97441479)(150.74760092,702.95442034)
\curveto(150.69759383,702.93441483)(150.63759389,702.92441484)(150.56760092,702.92442034)
\lineto(150.35760092,702.92442034)
\lineto(149.69760092,702.92442034)
\curveto(149.6275949,702.92441484)(149.55759497,702.91941484)(149.48760092,702.90942034)
\curveto(149.41759511,702.90941485)(149.35259517,702.91941484)(149.29260092,702.93942034)
\curveto(149.19259533,702.9594148)(149.11759541,702.99941476)(149.06760092,703.05942034)
\curveto(149.01759551,703.11941464)(148.97259555,703.17941458)(148.93260092,703.23942034)
\lineto(148.81260092,703.44942034)
\curveto(148.78259574,703.52941423)(148.73259579,703.59441417)(148.66260092,703.64442034)
\curveto(148.56259596,703.72441404)(148.46259606,703.78441398)(148.36260092,703.82442034)
\curveto(148.27259625,703.8644139)(148.15759637,703.89941386)(148.01760092,703.92942034)
\curveto(147.94759658,703.94941381)(147.84259668,703.9644138)(147.70260092,703.97442034)
\curveto(147.57259695,703.98441378)(147.47259705,703.97941378)(147.40260092,703.95942034)
\lineto(147.29760092,703.95942034)
\lineto(147.14760092,703.92942034)
\curveto(147.10759742,703.92941383)(147.06259746,703.92441384)(147.01260092,703.91442034)
\curveto(146.84259768,703.8644139)(146.70259782,703.79441397)(146.59260092,703.70442034)
\curveto(146.49259803,703.62441414)(146.4225981,703.49941426)(146.38260092,703.32942034)
\curveto(146.36259816,703.2594145)(146.36259816,703.19441457)(146.38260092,703.13442034)
\curveto(146.40259812,703.07441469)(146.4225981,703.02441474)(146.44260092,702.98442034)
\curveto(146.51259801,702.8644149)(146.59259793,702.76941499)(146.68260092,702.69942034)
\curveto(146.78259774,702.62941513)(146.89759763,702.56941519)(147.02760092,702.51942034)
\curveto(147.21759731,702.43941532)(147.4225971,702.36941539)(147.64260092,702.30942034)
\lineto(148.33260092,702.15942034)
\curveto(148.57259595,702.11941564)(148.80259572,702.06941569)(149.02260092,702.00942034)
\curveto(149.25259527,701.9594158)(149.46759506,701.89441587)(149.66760092,701.81442034)
\curveto(149.75759477,701.77441599)(149.84259468,701.73941602)(149.92260092,701.70942034)
\curveto(150.01259451,701.68941607)(150.09759443,701.65441611)(150.17760092,701.60442034)
\curveto(150.36759416,701.48441628)(150.53759399,701.35441641)(150.68760092,701.21442034)
\curveto(150.84759368,701.07441669)(150.97259355,700.89941686)(151.06260092,700.68942034)
\curveto(151.09259343,700.61941714)(151.11759341,700.54941721)(151.13760092,700.47942034)
\curveto(151.15759337,700.40941735)(151.17759335,700.33441743)(151.19760092,700.25442034)
\curveto(151.20759332,700.19441757)(151.21259331,700.09941766)(151.21260092,699.96942034)
\curveto(151.2225933,699.84941791)(151.2225933,699.75441801)(151.21260092,699.68442034)
\lineto(151.21260092,699.60942034)
\curveto(151.19259333,699.54941821)(151.17759335,699.48941827)(151.16760092,699.42942034)
\curveto(151.16759336,699.37941838)(151.16259336,699.32941843)(151.15260092,699.27942034)
\curveto(151.08259344,698.97941878)(150.97259355,698.71441905)(150.82260092,698.48442034)
\curveto(150.66259386,698.24441952)(150.46759406,698.04941971)(150.23760092,697.89942034)
\curveto(150.00759452,697.74942001)(149.74759478,697.61942014)(149.45760092,697.50942034)
\curveto(149.34759518,697.4594203)(149.2275953,697.42442034)(149.09760092,697.40442034)
\curveto(148.97759555,697.38442038)(148.85759567,697.3594204)(148.73760092,697.32942034)
\curveto(148.64759588,697.30942045)(148.55259597,697.29942046)(148.45260092,697.29942034)
\curveto(148.36259616,697.28942047)(148.27259625,697.27442049)(148.18260092,697.25442034)
\lineto(147.91260092,697.25442034)
\curveto(147.85259667,697.23442053)(147.74759678,697.22442054)(147.59760092,697.22442034)
\curveto(147.45759707,697.22442054)(147.35759717,697.23442053)(147.29760092,697.25442034)
\curveto(147.26759726,697.25442051)(147.23259729,697.2594205)(147.19260092,697.26942034)
\lineto(147.08760092,697.26942034)
\curveto(146.96759756,697.28942047)(146.84759768,697.30442046)(146.72760092,697.31442034)
\curveto(146.60759792,697.32442044)(146.49259803,697.34442042)(146.38260092,697.37442034)
\curveto(145.99259853,697.48442028)(145.64759888,697.60942015)(145.34760092,697.74942034)
\curveto(145.04759948,697.89941986)(144.79259973,698.11941964)(144.58260092,698.40942034)
\curveto(144.44260008,698.59941916)(144.3226002,698.81941894)(144.22260092,699.06942034)
\curveto(144.20260032,699.12941863)(144.18260034,699.20941855)(144.16260092,699.30942034)
\curveto(144.14260038,699.3594184)(144.1276004,699.42941833)(144.11760092,699.51942034)
\curveto(144.10760042,699.60941815)(144.11260041,699.68441808)(144.13260092,699.74442034)
\curveto(144.16260036,699.81441795)(144.21260031,699.8644179)(144.28260092,699.89442034)
\curveto(144.33260019,699.91441785)(144.39260013,699.92441784)(144.46260092,699.92442034)
\lineto(144.68760092,699.92442034)
\lineto(145.39260092,699.92442034)
\lineto(145.63260092,699.92442034)
\curveto(145.71259881,699.92441784)(145.78259874,699.91441785)(145.84260092,699.89442034)
\curveto(145.95259857,699.85441791)(146.0225985,699.78941797)(146.05260092,699.69942034)
\curveto(146.09259843,699.60941815)(146.13759839,699.51441825)(146.18760092,699.41442034)
\curveto(146.20759832,699.3644184)(146.24259828,699.29941846)(146.29260092,699.21942034)
\curveto(146.35259817,699.13941862)(146.40259812,699.08941867)(146.44260092,699.06942034)
\curveto(146.56259796,698.96941879)(146.67759785,698.88941887)(146.78760092,698.82942034)
\curveto(146.89759763,698.77941898)(147.03759749,698.72941903)(147.20760092,698.67942034)
\curveto(147.25759727,698.6594191)(147.30759722,698.64941911)(147.35760092,698.64942034)
\curveto(147.40759712,698.6594191)(147.45759707,698.6594191)(147.50760092,698.64942034)
\curveto(147.58759694,698.62941913)(147.67259685,698.61941914)(147.76260092,698.61942034)
\curveto(147.86259666,698.62941913)(147.94759658,698.64441912)(148.01760092,698.66442034)
\curveto(148.06759646,698.67441909)(148.11259641,698.67941908)(148.15260092,698.67942034)
\curveto(148.20259632,698.67941908)(148.25259627,698.68941907)(148.30260092,698.70942034)
\curveto(148.44259608,698.759419)(148.56759596,698.81941894)(148.67760092,698.88942034)
\curveto(148.79759573,698.9594188)(148.89259563,699.04941871)(148.96260092,699.15942034)
\curveto(149.01259551,699.23941852)(149.05259547,699.3644184)(149.08260092,699.53442034)
\curveto(149.10259542,699.60441816)(149.10259542,699.66941809)(149.08260092,699.72942034)
\curveto(149.06259546,699.78941797)(149.04259548,699.83941792)(149.02260092,699.87942034)
\curveto(148.95259557,700.01941774)(148.86259566,700.12441764)(148.75260092,700.19442034)
\curveto(148.65259587,700.2644175)(148.53259599,700.32941743)(148.39260092,700.38942034)
\curveto(148.20259632,700.46941729)(148.00259652,700.53441723)(147.79260092,700.58442034)
\curveto(147.58259694,700.63441713)(147.37259715,700.68941707)(147.16260092,700.74942034)
\curveto(147.08259744,700.76941699)(146.99759753,700.78441698)(146.90760092,700.79442034)
\curveto(146.8275977,700.80441696)(146.74759778,700.81941694)(146.66760092,700.83942034)
\curveto(146.34759818,700.92941683)(146.04259848,701.01441675)(145.75260092,701.09442034)
\curveto(145.46259906,701.18441658)(145.19759933,701.31441645)(144.95760092,701.48442034)
\curveto(144.67759985,701.68441608)(144.47260005,701.95441581)(144.34260092,702.29442034)
\curveto(144.3226002,702.3644154)(144.30260022,702.4594153)(144.28260092,702.57942034)
\curveto(144.26260026,702.64941511)(144.24760028,702.73441503)(144.23760092,702.83442034)
\curveto(144.2276003,702.93441483)(144.23260029,703.02441474)(144.25260092,703.10442034)
\curveto(144.27260025,703.15441461)(144.27760025,703.19441457)(144.26760092,703.22442034)
\curveto(144.25760027,703.2644145)(144.26260026,703.30941445)(144.28260092,703.35942034)
\curveto(144.30260022,703.46941429)(144.3226002,703.56941419)(144.34260092,703.65942034)
\curveto(144.37260015,703.759414)(144.40760012,703.85441391)(144.44760092,703.94442034)
\curveto(144.57759995,704.23441353)(144.75759977,704.46941329)(144.98760092,704.64942034)
\curveto(145.21759931,704.82941293)(145.47759905,704.97441279)(145.76760092,705.08442034)
\curveto(145.87759865,705.13441263)(145.99259853,705.16941259)(146.11260092,705.18942034)
\curveto(146.23259829,705.21941254)(146.35759817,705.24941251)(146.48760092,705.27942034)
\curveto(146.54759798,705.29941246)(146.60759792,705.30941245)(146.66760092,705.30942034)
\lineto(146.84760092,705.33942034)
\curveto(146.9275976,705.34941241)(147.01259751,705.35441241)(147.10260092,705.35442034)
\curveto(147.19259733,705.35441241)(147.27759725,705.3594124)(147.35760092,705.36942034)
}
}
{
\newrgbcolor{curcolor}{0 0 0}
\pscustom[linestyle=none,fillstyle=solid,fillcolor=curcolor]
{
}
}
{
\newrgbcolor{curcolor}{0 0 0}
\pscustom[linestyle=none,fillstyle=solid,fillcolor=curcolor]
{
\newpath
\moveto(161.02939779,705.35442034)
\curveto(161.13939248,705.35441241)(161.23439238,705.34441242)(161.31439779,705.32442034)
\curveto(161.40439221,705.30441246)(161.47439214,705.2594125)(161.52439779,705.18942034)
\curveto(161.58439203,705.10941265)(161.614392,704.96941279)(161.61439779,704.76942034)
\lineto(161.61439779,704.25942034)
\lineto(161.61439779,703.88442034)
\curveto(161.62439199,703.74441402)(161.60939201,703.63441413)(161.56939779,703.55442034)
\curveto(161.52939209,703.48441428)(161.46939215,703.43941432)(161.38939779,703.41942034)
\curveto(161.3193923,703.39941436)(161.23439238,703.38941437)(161.13439779,703.38942034)
\curveto(161.04439257,703.38941437)(160.94439267,703.39441437)(160.83439779,703.40442034)
\curveto(160.73439288,703.41441435)(160.63939298,703.40941435)(160.54939779,703.38942034)
\curveto(160.47939314,703.36941439)(160.40939321,703.35441441)(160.33939779,703.34442034)
\curveto(160.26939335,703.34441442)(160.20439341,703.33441443)(160.14439779,703.31442034)
\curveto(159.98439363,703.2644145)(159.82439379,703.18941457)(159.66439779,703.08942034)
\curveto(159.50439411,702.99941476)(159.37939424,702.89441487)(159.28939779,702.77442034)
\curveto(159.23939438,702.69441507)(159.18439443,702.60941515)(159.12439779,702.51942034)
\curveto(159.07439454,702.43941532)(159.02439459,702.35441541)(158.97439779,702.26442034)
\curveto(158.94439467,702.18441558)(158.9143947,702.09941566)(158.88439779,702.00942034)
\lineto(158.82439779,701.76942034)
\curveto(158.80439481,701.69941606)(158.79439482,701.62441614)(158.79439779,701.54442034)
\curveto(158.79439482,701.47441629)(158.78439483,701.40441636)(158.76439779,701.33442034)
\curveto(158.75439486,701.29441647)(158.74939487,701.25441651)(158.74939779,701.21442034)
\curveto(158.75939486,701.18441658)(158.75939486,701.15441661)(158.74939779,701.12442034)
\lineto(158.74939779,700.88442034)
\curveto(158.72939489,700.81441695)(158.72439489,700.73441703)(158.73439779,700.64442034)
\curveto(158.74439487,700.5644172)(158.74939487,700.48441728)(158.74939779,700.40442034)
\lineto(158.74939779,699.44442034)
\lineto(158.74939779,698.16942034)
\curveto(158.74939487,698.03941972)(158.74439487,697.91941984)(158.73439779,697.80942034)
\curveto(158.72439489,697.69942006)(158.69439492,697.60942015)(158.64439779,697.53942034)
\curveto(158.62439499,697.50942025)(158.58939503,697.48442028)(158.53939779,697.46442034)
\curveto(158.49939512,697.45442031)(158.45439516,697.44442032)(158.40439779,697.43442034)
\lineto(158.32939779,697.43442034)
\curveto(158.27939534,697.42442034)(158.22439539,697.41942034)(158.16439779,697.41942034)
\lineto(157.99939779,697.41942034)
\lineto(157.35439779,697.41942034)
\curveto(157.29439632,697.42942033)(157.22939639,697.43442033)(157.15939779,697.43442034)
\lineto(156.96439779,697.43442034)
\curveto(156.9143967,697.45442031)(156.86439675,697.46942029)(156.81439779,697.47942034)
\curveto(156.76439685,697.49942026)(156.72939689,697.53442023)(156.70939779,697.58442034)
\curveto(156.66939695,697.63442013)(156.64439697,697.70442006)(156.63439779,697.79442034)
\lineto(156.63439779,698.09442034)
\lineto(156.63439779,699.11442034)
\lineto(156.63439779,703.34442034)
\lineto(156.63439779,704.45442034)
\lineto(156.63439779,704.73942034)
\curveto(156.63439698,704.83941292)(156.65439696,704.91941284)(156.69439779,704.97942034)
\curveto(156.74439687,705.0594127)(156.8193968,705.10941265)(156.91939779,705.12942034)
\curveto(157.0193966,705.14941261)(157.13939648,705.1594126)(157.27939779,705.15942034)
\lineto(158.04439779,705.15942034)
\curveto(158.16439545,705.1594126)(158.26939535,705.14941261)(158.35939779,705.12942034)
\curveto(158.44939517,705.11941264)(158.5193951,705.07441269)(158.56939779,704.99442034)
\curveto(158.59939502,704.94441282)(158.614395,704.87441289)(158.61439779,704.78442034)
\lineto(158.64439779,704.51442034)
\curveto(158.65439496,704.43441333)(158.66939495,704.3594134)(158.68939779,704.28942034)
\curveto(158.7193949,704.21941354)(158.76939485,704.18441358)(158.83939779,704.18442034)
\curveto(158.85939476,704.20441356)(158.87939474,704.21441355)(158.89939779,704.21442034)
\curveto(158.9193947,704.21441355)(158.93939468,704.22441354)(158.95939779,704.24442034)
\curveto(159.0193946,704.29441347)(159.06939455,704.34941341)(159.10939779,704.40942034)
\curveto(159.15939446,704.47941328)(159.2193944,704.53941322)(159.28939779,704.58942034)
\curveto(159.32939429,704.61941314)(159.36439425,704.64941311)(159.39439779,704.67942034)
\curveto(159.42439419,704.71941304)(159.45939416,704.75441301)(159.49939779,704.78442034)
\lineto(159.76939779,704.96442034)
\curveto(159.86939375,705.02441274)(159.96939365,705.07941268)(160.06939779,705.12942034)
\curveto(160.16939345,705.16941259)(160.26939335,705.20441256)(160.36939779,705.23442034)
\lineto(160.69939779,705.32442034)
\curveto(160.72939289,705.33441243)(160.78439283,705.33441243)(160.86439779,705.32442034)
\curveto(160.95439266,705.32441244)(161.00939261,705.33441243)(161.02939779,705.35442034)
}
}
{
\newrgbcolor{curcolor}{0 0 0}
\pscustom[linestyle=none,fillstyle=solid,fillcolor=curcolor]
{
\newpath
\moveto(169.53580404,701.36442034)
\curveto(169.55579588,701.28441648)(169.55579588,701.19441657)(169.53580404,701.09442034)
\curveto(169.51579592,700.99441677)(169.48079595,700.92941683)(169.43080404,700.89942034)
\curveto(169.38079605,700.8594169)(169.30579613,700.82941693)(169.20580404,700.80942034)
\curveto(169.11579632,700.79941696)(169.01079642,700.78941697)(168.89080404,700.77942034)
\lineto(168.54580404,700.77942034)
\curveto(168.435797,700.78941697)(168.3357971,700.79441697)(168.24580404,700.79442034)
\lineto(164.58580404,700.79442034)
\lineto(164.37580404,700.79442034)
\curveto(164.31580112,700.79441697)(164.26080117,700.78441698)(164.21080404,700.76442034)
\curveto(164.1308013,700.72441704)(164.08080135,700.68441708)(164.06080404,700.64442034)
\curveto(164.04080139,700.62441714)(164.02080141,700.58441718)(164.00080404,700.52442034)
\curveto(163.98080145,700.47441729)(163.97580146,700.42441734)(163.98580404,700.37442034)
\curveto(164.00580143,700.31441745)(164.01580142,700.25441751)(164.01580404,700.19442034)
\curveto(164.02580141,700.14441762)(164.04080139,700.08941767)(164.06080404,700.02942034)
\curveto(164.14080129,699.78941797)(164.2358012,699.58941817)(164.34580404,699.42942034)
\curveto(164.46580097,699.27941848)(164.62580081,699.14441862)(164.82580404,699.02442034)
\curveto(164.90580053,698.97441879)(164.98580045,698.93941882)(165.06580404,698.91942034)
\curveto(165.15580028,698.90941885)(165.24580019,698.88941887)(165.33580404,698.85942034)
\curveto(165.41580002,698.83941892)(165.52579991,698.82441894)(165.66580404,698.81442034)
\curveto(165.80579963,698.80441896)(165.92579951,698.80941895)(166.02580404,698.82942034)
\lineto(166.16080404,698.82942034)
\curveto(166.26079917,698.84941891)(166.35079908,698.86941889)(166.43080404,698.88942034)
\curveto(166.52079891,698.91941884)(166.60579883,698.94941881)(166.68580404,698.97942034)
\curveto(166.78579865,699.02941873)(166.89579854,699.09441867)(167.01580404,699.17442034)
\curveto(167.14579829,699.25441851)(167.24079819,699.33441843)(167.30080404,699.41442034)
\curveto(167.35079808,699.48441828)(167.40079803,699.54941821)(167.45080404,699.60942034)
\curveto(167.51079792,699.67941808)(167.58079785,699.72941803)(167.66080404,699.75942034)
\curveto(167.76079767,699.80941795)(167.88579755,699.82941793)(168.03580404,699.81942034)
\lineto(168.47080404,699.81942034)
\lineto(168.65080404,699.81942034)
\curveto(168.72079671,699.82941793)(168.78079665,699.82441794)(168.83080404,699.80442034)
\lineto(168.98080404,699.80442034)
\curveto(169.08079635,699.78441798)(169.15079628,699.759418)(169.19080404,699.72942034)
\curveto(169.2307962,699.70941805)(169.25079618,699.6644181)(169.25080404,699.59442034)
\curveto(169.26079617,699.52441824)(169.25579618,699.4644183)(169.23580404,699.41442034)
\curveto(169.18579625,699.27441849)(169.1307963,699.14941861)(169.07080404,699.03942034)
\curveto(169.01079642,698.92941883)(168.94079649,698.81941894)(168.86080404,698.70942034)
\curveto(168.64079679,698.37941938)(168.39079704,698.11441965)(168.11080404,697.91442034)
\curveto(167.8307976,697.71442005)(167.48079795,697.54442022)(167.06080404,697.40442034)
\curveto(166.95079848,697.3644204)(166.84079859,697.33942042)(166.73080404,697.32942034)
\curveto(166.62079881,697.31942044)(166.50579893,697.29942046)(166.38580404,697.26942034)
\curveto(166.34579909,697.2594205)(166.30079913,697.2594205)(166.25080404,697.26942034)
\curveto(166.21079922,697.26942049)(166.17079926,697.2644205)(166.13080404,697.25442034)
\lineto(165.96580404,697.25442034)
\curveto(165.91579952,697.23442053)(165.85579958,697.22942053)(165.78580404,697.23942034)
\curveto(165.72579971,697.23942052)(165.67079976,697.24442052)(165.62080404,697.25442034)
\curveto(165.54079989,697.2644205)(165.47079996,697.2644205)(165.41080404,697.25442034)
\curveto(165.35080008,697.24442052)(165.28580015,697.24942051)(165.21580404,697.26942034)
\curveto(165.16580027,697.28942047)(165.11080032,697.29942046)(165.05080404,697.29942034)
\curveto(164.99080044,697.29942046)(164.9358005,697.30942045)(164.88580404,697.32942034)
\curveto(164.77580066,697.34942041)(164.66580077,697.37442039)(164.55580404,697.40442034)
\curveto(164.44580099,697.42442034)(164.34580109,697.4594203)(164.25580404,697.50942034)
\curveto(164.14580129,697.54942021)(164.04080139,697.58442018)(163.94080404,697.61442034)
\curveto(163.85080158,697.65442011)(163.76580167,697.69942006)(163.68580404,697.74942034)
\curveto(163.36580207,697.94941981)(163.08080235,698.17941958)(162.83080404,698.43942034)
\curveto(162.58080285,698.70941905)(162.37580306,699.01941874)(162.21580404,699.36942034)
\curveto(162.16580327,699.47941828)(162.12580331,699.58941817)(162.09580404,699.69942034)
\curveto(162.06580337,699.81941794)(162.02580341,699.93941782)(161.97580404,700.05942034)
\curveto(161.96580347,700.09941766)(161.96080347,700.13441763)(161.96080404,700.16442034)
\curveto(161.96080347,700.20441756)(161.95580348,700.24441752)(161.94580404,700.28442034)
\curveto(161.90580353,700.40441736)(161.88080355,700.53441723)(161.87080404,700.67442034)
\lineto(161.84080404,701.09442034)
\curveto(161.84080359,701.14441662)(161.8358036,701.19941656)(161.82580404,701.25942034)
\curveto(161.82580361,701.31941644)(161.8308036,701.37441639)(161.84080404,701.42442034)
\lineto(161.84080404,701.60442034)
\lineto(161.88580404,701.96442034)
\curveto(161.92580351,702.13441563)(161.96080347,702.29941546)(161.99080404,702.45942034)
\curveto(162.02080341,702.61941514)(162.06580337,702.76941499)(162.12580404,702.90942034)
\curveto(162.55580288,703.94941381)(163.28580215,704.68441308)(164.31580404,705.11442034)
\curveto(164.45580098,705.17441259)(164.59580084,705.21441255)(164.73580404,705.23442034)
\curveto(164.88580055,705.2644125)(165.04080039,705.29941246)(165.20080404,705.33942034)
\curveto(165.28080015,705.34941241)(165.35580008,705.35441241)(165.42580404,705.35442034)
\curveto(165.49579994,705.35441241)(165.57079986,705.3594124)(165.65080404,705.36942034)
\curveto(166.16079927,705.37941238)(166.59579884,705.31941244)(166.95580404,705.18942034)
\curveto(167.32579811,705.06941269)(167.65579778,704.90941285)(167.94580404,704.70942034)
\curveto(168.0357974,704.64941311)(168.12579731,704.57941318)(168.21580404,704.49942034)
\curveto(168.30579713,704.42941333)(168.38579705,704.35441341)(168.45580404,704.27442034)
\curveto(168.48579695,704.22441354)(168.52579691,704.18441358)(168.57580404,704.15442034)
\curveto(168.65579678,704.04441372)(168.7307967,703.92941383)(168.80080404,703.80942034)
\curveto(168.87079656,703.69941406)(168.94579649,703.58441418)(169.02580404,703.46442034)
\curveto(169.07579636,703.37441439)(169.11579632,703.27941448)(169.14580404,703.17942034)
\curveto(169.18579625,703.08941467)(169.22579621,702.98941477)(169.26580404,702.87942034)
\curveto(169.31579612,702.74941501)(169.35579608,702.61441515)(169.38580404,702.47442034)
\curveto(169.41579602,702.33441543)(169.45079598,702.19441557)(169.49080404,702.05442034)
\curveto(169.51079592,701.97441579)(169.51579592,701.88441588)(169.50580404,701.78442034)
\curveto(169.50579593,701.69441607)(169.51579592,701.60941615)(169.53580404,701.52942034)
\lineto(169.53580404,701.36442034)
\moveto(167.28580404,702.24942034)
\curveto(167.35579808,702.34941541)(167.36079807,702.46941529)(167.30080404,702.60942034)
\curveto(167.25079818,702.759415)(167.21079822,702.86941489)(167.18080404,702.93942034)
\curveto(167.04079839,703.20941455)(166.85579858,703.41441435)(166.62580404,703.55442034)
\curveto(166.39579904,703.70441406)(166.07579936,703.78441398)(165.66580404,703.79442034)
\curveto(165.6357998,703.77441399)(165.60079983,703.76941399)(165.56080404,703.77942034)
\curveto(165.52079991,703.78941397)(165.48579995,703.78941397)(165.45580404,703.77942034)
\curveto(165.40580003,703.759414)(165.35080008,703.74441402)(165.29080404,703.73442034)
\curveto(165.2308002,703.73441403)(165.17580026,703.72441404)(165.12580404,703.70442034)
\curveto(164.68580075,703.5644142)(164.36080107,703.28941447)(164.15080404,702.87942034)
\curveto(164.1308013,702.83941492)(164.10580133,702.78441498)(164.07580404,702.71442034)
\curveto(164.05580138,702.65441511)(164.04080139,702.58941517)(164.03080404,702.51942034)
\curveto(164.02080141,702.4594153)(164.02080141,702.39941536)(164.03080404,702.33942034)
\curveto(164.05080138,702.27941548)(164.08580135,702.22941553)(164.13580404,702.18942034)
\curveto(164.21580122,702.13941562)(164.32580111,702.11441565)(164.46580404,702.11442034)
\lineto(164.87080404,702.11442034)
\lineto(166.53580404,702.11442034)
\lineto(166.97080404,702.11442034)
\curveto(167.1307983,702.12441564)(167.2357982,702.16941559)(167.28580404,702.24942034)
}
}
{
\newrgbcolor{curcolor}{0 0 0}
\pscustom[linestyle=none,fillstyle=solid,fillcolor=curcolor]
{
\newpath
\moveto(174.35408529,705.36942034)
\curveto(175.16408013,705.38941237)(175.83907946,705.26941249)(176.37908529,705.00942034)
\curveto(176.92907837,704.74941301)(177.36407793,704.37941338)(177.68408529,703.89942034)
\curveto(177.84407745,703.6594141)(177.96407733,703.38441438)(178.04408529,703.07442034)
\curveto(178.06407723,703.02441474)(178.07907722,702.9594148)(178.08908529,702.87942034)
\curveto(178.10907719,702.79941496)(178.10907719,702.72941503)(178.08908529,702.66942034)
\curveto(178.04907725,702.5594152)(177.97907732,702.49441527)(177.87908529,702.47442034)
\curveto(177.77907752,702.4644153)(177.65907764,702.4594153)(177.51908529,702.45942034)
\lineto(176.73908529,702.45942034)
\lineto(176.45408529,702.45942034)
\curveto(176.36407893,702.4594153)(176.28907901,702.47941528)(176.22908529,702.51942034)
\curveto(176.14907915,702.5594152)(176.0940792,702.61941514)(176.06408529,702.69942034)
\curveto(176.03407926,702.78941497)(175.9940793,702.87941488)(175.94408529,702.96942034)
\curveto(175.88407941,703.07941468)(175.81907948,703.17941458)(175.74908529,703.26942034)
\curveto(175.67907962,703.3594144)(175.5990797,703.43941432)(175.50908529,703.50942034)
\curveto(175.36907993,703.59941416)(175.21408008,703.66941409)(175.04408529,703.71942034)
\curveto(174.98408031,703.73941402)(174.92408037,703.74941401)(174.86408529,703.74942034)
\curveto(174.80408049,703.74941401)(174.74908055,703.759414)(174.69908529,703.77942034)
\lineto(174.54908529,703.77942034)
\curveto(174.34908095,703.77941398)(174.18908111,703.759414)(174.06908529,703.71942034)
\curveto(173.77908152,703.62941413)(173.54408175,703.48941427)(173.36408529,703.29942034)
\curveto(173.18408211,703.11941464)(173.03908226,702.89941486)(172.92908529,702.63942034)
\curveto(172.87908242,702.52941523)(172.83908246,702.40941535)(172.80908529,702.27942034)
\curveto(172.78908251,702.1594156)(172.76408253,702.02941573)(172.73408529,701.88942034)
\curveto(172.72408257,701.84941591)(172.71908258,701.80941595)(172.71908529,701.76942034)
\curveto(172.71908258,701.72941603)(172.71408258,701.68941607)(172.70408529,701.64942034)
\curveto(172.68408261,701.54941621)(172.67408262,701.40941635)(172.67408529,701.22942034)
\curveto(172.68408261,701.04941671)(172.6990826,700.90941685)(172.71908529,700.80942034)
\curveto(172.71908258,700.72941703)(172.72408257,700.67441709)(172.73408529,700.64442034)
\curveto(172.75408254,700.57441719)(172.76408253,700.50441726)(172.76408529,700.43442034)
\curveto(172.77408252,700.3644174)(172.78908251,700.29441747)(172.80908529,700.22442034)
\curveto(172.88908241,699.99441777)(172.98408231,699.78441798)(173.09408529,699.59442034)
\curveto(173.20408209,699.40441836)(173.34408195,699.24441852)(173.51408529,699.11442034)
\curveto(173.55408174,699.08441868)(173.61408168,699.04941871)(173.69408529,699.00942034)
\curveto(173.80408149,698.93941882)(173.91408138,698.89441887)(174.02408529,698.87442034)
\curveto(174.14408115,698.85441891)(174.28908101,698.83441893)(174.45908529,698.81442034)
\lineto(174.54908529,698.81442034)
\curveto(174.58908071,698.81441895)(174.61908068,698.81941894)(174.63908529,698.82942034)
\lineto(174.77408529,698.82942034)
\curveto(174.84408045,698.84941891)(174.90908039,698.8644189)(174.96908529,698.87442034)
\curveto(175.03908026,698.89441887)(175.10408019,698.91441885)(175.16408529,698.93442034)
\curveto(175.46407983,699.0644187)(175.6940796,699.25441851)(175.85408529,699.50442034)
\curveto(175.8940794,699.55441821)(175.92907937,699.60941815)(175.95908529,699.66942034)
\curveto(175.98907931,699.73941802)(176.01407928,699.79941796)(176.03408529,699.84942034)
\curveto(176.07407922,699.9594178)(176.10907919,700.05441771)(176.13908529,700.13442034)
\curveto(176.16907913,700.22441754)(176.23907906,700.29441747)(176.34908529,700.34442034)
\curveto(176.43907886,700.38441738)(176.58407871,700.39941736)(176.78408529,700.38942034)
\lineto(177.27908529,700.38942034)
\lineto(177.48908529,700.38942034)
\curveto(177.56907773,700.39941736)(177.63407766,700.39441737)(177.68408529,700.37442034)
\lineto(177.80408529,700.37442034)
\lineto(177.92408529,700.34442034)
\curveto(177.96407733,700.34441742)(177.9940773,700.33441743)(178.01408529,700.31442034)
\curveto(178.06407723,700.27441749)(178.0940772,700.21441755)(178.10408529,700.13442034)
\curveto(178.12407717,700.0644177)(178.12407717,699.98941777)(178.10408529,699.90942034)
\curveto(178.01407728,699.57941818)(177.90407739,699.28441848)(177.77408529,699.02442034)
\curveto(177.36407793,698.25441951)(176.70907859,697.71942004)(175.80908529,697.41942034)
\curveto(175.70907959,697.38942037)(175.60407969,697.36942039)(175.49408529,697.35942034)
\curveto(175.38407991,697.33942042)(175.27408002,697.31442045)(175.16408529,697.28442034)
\curveto(175.10408019,697.27442049)(175.04408025,697.26942049)(174.98408529,697.26942034)
\curveto(174.92408037,697.26942049)(174.86408043,697.2644205)(174.80408529,697.25442034)
\lineto(174.63908529,697.25442034)
\curveto(174.58908071,697.23442053)(174.51408078,697.22942053)(174.41408529,697.23942034)
\curveto(174.31408098,697.23942052)(174.23908106,697.24442052)(174.18908529,697.25442034)
\curveto(174.10908119,697.27442049)(174.03408126,697.28442048)(173.96408529,697.28442034)
\curveto(173.90408139,697.27442049)(173.83908146,697.27942048)(173.76908529,697.29942034)
\lineto(173.61908529,697.32942034)
\curveto(173.56908173,697.32942043)(173.51908178,697.33442043)(173.46908529,697.34442034)
\curveto(173.35908194,697.37442039)(173.25408204,697.40442036)(173.15408529,697.43442034)
\curveto(173.05408224,697.4644203)(172.95908234,697.49942026)(172.86908529,697.53942034)
\curveto(172.3990829,697.73942002)(172.00408329,697.99441977)(171.68408529,698.30442034)
\curveto(171.36408393,698.62441914)(171.10408419,699.01941874)(170.90408529,699.48942034)
\curveto(170.85408444,699.57941818)(170.81408448,699.67441809)(170.78408529,699.77442034)
\lineto(170.69408529,700.10442034)
\curveto(170.68408461,700.14441762)(170.67908462,700.17941758)(170.67908529,700.20942034)
\curveto(170.67908462,700.24941751)(170.66908463,700.29441747)(170.64908529,700.34442034)
\curveto(170.62908467,700.41441735)(170.61908468,700.48441728)(170.61908529,700.55442034)
\curveto(170.61908468,700.63441713)(170.60908469,700.70941705)(170.58908529,700.77942034)
\lineto(170.58908529,701.03442034)
\curveto(170.56908473,701.08441668)(170.55908474,701.13941662)(170.55908529,701.19942034)
\curveto(170.55908474,701.26941649)(170.56908473,701.32941643)(170.58908529,701.37942034)
\curveto(170.5990847,701.42941633)(170.5990847,701.47441629)(170.58908529,701.51442034)
\curveto(170.57908472,701.55441621)(170.57908472,701.59441617)(170.58908529,701.63442034)
\curveto(170.60908469,701.70441606)(170.61408468,701.76941599)(170.60408529,701.82942034)
\curveto(170.60408469,701.88941587)(170.61408468,701.94941581)(170.63408529,702.00942034)
\curveto(170.68408461,702.18941557)(170.72408457,702.3594154)(170.75408529,702.51942034)
\curveto(170.78408451,702.68941507)(170.82908447,702.85441491)(170.88908529,703.01442034)
\curveto(171.10908419,703.52441424)(171.38408391,703.94941381)(171.71408529,704.28942034)
\curveto(172.05408324,704.62941313)(172.48408281,704.90441286)(173.00408529,705.11442034)
\curveto(173.14408215,705.17441259)(173.28908201,705.21441255)(173.43908529,705.23442034)
\curveto(173.58908171,705.2644125)(173.74408155,705.29941246)(173.90408529,705.33942034)
\curveto(173.98408131,705.34941241)(174.05908124,705.35441241)(174.12908529,705.35442034)
\curveto(174.1990811,705.35441241)(174.27408102,705.3594124)(174.35408529,705.36942034)
}
}
{
\newrgbcolor{curcolor}{0 0 0}
\pscustom[linestyle=none,fillstyle=solid,fillcolor=curcolor]
{
\newpath
\moveto(179.81736654,705.14442034)
\lineto(180.94236654,705.14442034)
\curveto(181.05236411,705.14441262)(181.15236401,705.13941262)(181.24236654,705.12942034)
\curveto(181.33236383,705.11941264)(181.39736376,705.08441268)(181.43736654,705.02442034)
\curveto(181.48736367,704.9644128)(181.51736364,704.87941288)(181.52736654,704.76942034)
\curveto(181.53736362,704.66941309)(181.54236362,704.5644132)(181.54236654,704.45442034)
\lineto(181.54236654,703.40442034)
\lineto(181.54236654,701.16942034)
\curveto(181.54236362,700.80941695)(181.5573636,700.46941729)(181.58736654,700.14942034)
\curveto(181.61736354,699.82941793)(181.70736345,699.5644182)(181.85736654,699.35442034)
\curveto(181.99736316,699.14441862)(182.22236294,698.99441877)(182.53236654,698.90442034)
\curveto(182.58236258,698.89441887)(182.62236254,698.88941887)(182.65236654,698.88942034)
\curveto(182.69236247,698.88941887)(182.73736242,698.88441888)(182.78736654,698.87442034)
\curveto(182.83736232,698.8644189)(182.89236227,698.8594189)(182.95236654,698.85942034)
\curveto(183.01236215,698.8594189)(183.0573621,698.8644189)(183.08736654,698.87442034)
\curveto(183.13736202,698.89441887)(183.17736198,698.89941886)(183.20736654,698.88942034)
\curveto(183.24736191,698.87941888)(183.28736187,698.88441888)(183.32736654,698.90442034)
\curveto(183.53736162,698.95441881)(183.70236146,699.01941874)(183.82236654,699.09942034)
\curveto(184.00236116,699.20941855)(184.14236102,699.34941841)(184.24236654,699.51942034)
\curveto(184.35236081,699.69941806)(184.42736073,699.89441787)(184.46736654,700.10442034)
\curveto(184.51736064,700.32441744)(184.54736061,700.5644172)(184.55736654,700.82442034)
\curveto(184.56736059,701.09441667)(184.57236059,701.37441639)(184.57236654,701.66442034)
\lineto(184.57236654,703.47942034)
\lineto(184.57236654,704.45442034)
\lineto(184.57236654,704.72442034)
\curveto(184.57236059,704.82441294)(184.59236057,704.90441286)(184.63236654,704.96442034)
\curveto(184.68236048,705.05441271)(184.7573604,705.10441266)(184.85736654,705.11442034)
\curveto(184.9573602,705.13441263)(185.07736008,705.14441262)(185.21736654,705.14442034)
\lineto(186.01236654,705.14442034)
\lineto(186.29736654,705.14442034)
\curveto(186.38735877,705.14441262)(186.4623587,705.12441264)(186.52236654,705.08442034)
\curveto(186.60235856,705.03441273)(186.64735851,704.9594128)(186.65736654,704.85942034)
\curveto(186.66735849,704.759413)(186.67235849,704.64441312)(186.67236654,704.51442034)
\lineto(186.67236654,703.37442034)
\lineto(186.67236654,699.15942034)
\lineto(186.67236654,698.09442034)
\lineto(186.67236654,697.79442034)
\curveto(186.67235849,697.69442007)(186.65235851,697.61942014)(186.61236654,697.56942034)
\curveto(186.5623586,697.48942027)(186.48735867,697.44442032)(186.38736654,697.43442034)
\curveto(186.28735887,697.42442034)(186.18235898,697.41942034)(186.07236654,697.41942034)
\lineto(185.26236654,697.41942034)
\curveto(185.15236001,697.41942034)(185.05236011,697.42442034)(184.96236654,697.43442034)
\curveto(184.88236028,697.44442032)(184.81736034,697.48442028)(184.76736654,697.55442034)
\curveto(184.74736041,697.58442018)(184.72736043,697.62942013)(184.70736654,697.68942034)
\curveto(184.69736046,697.74942001)(184.68236048,697.80941995)(184.66236654,697.86942034)
\curveto(184.65236051,697.92941983)(184.63736052,697.98441978)(184.61736654,698.03442034)
\curveto(184.59736056,698.08441968)(184.56736059,698.11441965)(184.52736654,698.12442034)
\curveto(184.50736065,698.14441962)(184.48236068,698.14941961)(184.45236654,698.13942034)
\curveto(184.42236074,698.12941963)(184.39736076,698.11941964)(184.37736654,698.10942034)
\curveto(184.30736085,698.06941969)(184.24736091,698.02441974)(184.19736654,697.97442034)
\curveto(184.14736101,697.92441984)(184.09236107,697.87941988)(184.03236654,697.83942034)
\curveto(183.99236117,697.80941995)(183.95236121,697.77441999)(183.91236654,697.73442034)
\curveto(183.88236128,697.70442006)(183.84236132,697.67442009)(183.79236654,697.64442034)
\curveto(183.5623616,697.50442026)(183.29236187,697.39442037)(182.98236654,697.31442034)
\curveto(182.91236225,697.29442047)(182.84236232,697.28442048)(182.77236654,697.28442034)
\curveto(182.70236246,697.27442049)(182.62736253,697.2594205)(182.54736654,697.23942034)
\curveto(182.50736265,697.22942053)(182.4623627,697.22942053)(182.41236654,697.23942034)
\curveto(182.37236279,697.23942052)(182.33236283,697.23442053)(182.29236654,697.22442034)
\curveto(182.2623629,697.21442055)(182.19736296,697.21442055)(182.09736654,697.22442034)
\curveto(182.00736315,697.22442054)(181.94736321,697.22942053)(181.91736654,697.23942034)
\curveto(181.86736329,697.23942052)(181.81736334,697.24442052)(181.76736654,697.25442034)
\lineto(181.61736654,697.25442034)
\curveto(181.49736366,697.28442048)(181.38236378,697.30942045)(181.27236654,697.32942034)
\curveto(181.162364,697.34942041)(181.05236411,697.37942038)(180.94236654,697.41942034)
\curveto(180.89236427,697.43942032)(180.84736431,697.45442031)(180.80736654,697.46442034)
\curveto(180.77736438,697.48442028)(180.73736442,697.50442026)(180.68736654,697.52442034)
\curveto(180.33736482,697.71442005)(180.0573651,697.97941978)(179.84736654,698.31942034)
\curveto(179.71736544,698.52941923)(179.62236554,698.77941898)(179.56236654,699.06942034)
\curveto(179.50236566,699.36941839)(179.4623657,699.68441808)(179.44236654,700.01442034)
\curveto(179.43236573,700.35441741)(179.42736573,700.69941706)(179.42736654,701.04942034)
\curveto(179.43736572,701.40941635)(179.44236572,701.764416)(179.44236654,702.11442034)
\lineto(179.44236654,704.15442034)
\curveto(179.44236572,704.28441348)(179.43736572,704.43441333)(179.42736654,704.60442034)
\curveto(179.42736573,704.78441298)(179.45236571,704.91441285)(179.50236654,704.99442034)
\curveto(179.53236563,705.04441272)(179.59236557,705.08941267)(179.68236654,705.12942034)
\curveto(179.74236542,705.12941263)(179.78736537,705.13441263)(179.81736654,705.14442034)
}
}
{
\newrgbcolor{curcolor}{0 0 0}
\pscustom[linestyle=none,fillstyle=solid,fillcolor=curcolor]
{
\newpath
\moveto(192.72861654,705.35442034)
\curveto(192.83861123,705.35441241)(192.93361113,705.34441242)(193.01361654,705.32442034)
\curveto(193.10361096,705.30441246)(193.17361089,705.2594125)(193.22361654,705.18942034)
\curveto(193.28361078,705.10941265)(193.31361075,704.96941279)(193.31361654,704.76942034)
\lineto(193.31361654,704.25942034)
\lineto(193.31361654,703.88442034)
\curveto(193.32361074,703.74441402)(193.30861076,703.63441413)(193.26861654,703.55442034)
\curveto(193.22861084,703.48441428)(193.1686109,703.43941432)(193.08861654,703.41942034)
\curveto(193.01861105,703.39941436)(192.93361113,703.38941437)(192.83361654,703.38942034)
\curveto(192.74361132,703.38941437)(192.64361142,703.39441437)(192.53361654,703.40442034)
\curveto(192.43361163,703.41441435)(192.33861173,703.40941435)(192.24861654,703.38942034)
\curveto(192.17861189,703.36941439)(192.10861196,703.35441441)(192.03861654,703.34442034)
\curveto(191.9686121,703.34441442)(191.90361216,703.33441443)(191.84361654,703.31442034)
\curveto(191.68361238,703.2644145)(191.52361254,703.18941457)(191.36361654,703.08942034)
\curveto(191.20361286,702.99941476)(191.07861299,702.89441487)(190.98861654,702.77442034)
\curveto(190.93861313,702.69441507)(190.88361318,702.60941515)(190.82361654,702.51942034)
\curveto(190.77361329,702.43941532)(190.72361334,702.35441541)(190.67361654,702.26442034)
\curveto(190.64361342,702.18441558)(190.61361345,702.09941566)(190.58361654,702.00942034)
\lineto(190.52361654,701.76942034)
\curveto(190.50361356,701.69941606)(190.49361357,701.62441614)(190.49361654,701.54442034)
\curveto(190.49361357,701.47441629)(190.48361358,701.40441636)(190.46361654,701.33442034)
\curveto(190.45361361,701.29441647)(190.44861362,701.25441651)(190.44861654,701.21442034)
\curveto(190.45861361,701.18441658)(190.45861361,701.15441661)(190.44861654,701.12442034)
\lineto(190.44861654,700.88442034)
\curveto(190.42861364,700.81441695)(190.42361364,700.73441703)(190.43361654,700.64442034)
\curveto(190.44361362,700.5644172)(190.44861362,700.48441728)(190.44861654,700.40442034)
\lineto(190.44861654,699.44442034)
\lineto(190.44861654,698.16942034)
\curveto(190.44861362,698.03941972)(190.44361362,697.91941984)(190.43361654,697.80942034)
\curveto(190.42361364,697.69942006)(190.39361367,697.60942015)(190.34361654,697.53942034)
\curveto(190.32361374,697.50942025)(190.28861378,697.48442028)(190.23861654,697.46442034)
\curveto(190.19861387,697.45442031)(190.15361391,697.44442032)(190.10361654,697.43442034)
\lineto(190.02861654,697.43442034)
\curveto(189.97861409,697.42442034)(189.92361414,697.41942034)(189.86361654,697.41942034)
\lineto(189.69861654,697.41942034)
\lineto(189.05361654,697.41942034)
\curveto(188.99361507,697.42942033)(188.92861514,697.43442033)(188.85861654,697.43442034)
\lineto(188.66361654,697.43442034)
\curveto(188.61361545,697.45442031)(188.5636155,697.46942029)(188.51361654,697.47942034)
\curveto(188.4636156,697.49942026)(188.42861564,697.53442023)(188.40861654,697.58442034)
\curveto(188.3686157,697.63442013)(188.34361572,697.70442006)(188.33361654,697.79442034)
\lineto(188.33361654,698.09442034)
\lineto(188.33361654,699.11442034)
\lineto(188.33361654,703.34442034)
\lineto(188.33361654,704.45442034)
\lineto(188.33361654,704.73942034)
\curveto(188.33361573,704.83941292)(188.35361571,704.91941284)(188.39361654,704.97942034)
\curveto(188.44361562,705.0594127)(188.51861555,705.10941265)(188.61861654,705.12942034)
\curveto(188.71861535,705.14941261)(188.83861523,705.1594126)(188.97861654,705.15942034)
\lineto(189.74361654,705.15942034)
\curveto(189.8636142,705.1594126)(189.9686141,705.14941261)(190.05861654,705.12942034)
\curveto(190.14861392,705.11941264)(190.21861385,705.07441269)(190.26861654,704.99442034)
\curveto(190.29861377,704.94441282)(190.31361375,704.87441289)(190.31361654,704.78442034)
\lineto(190.34361654,704.51442034)
\curveto(190.35361371,704.43441333)(190.3686137,704.3594134)(190.38861654,704.28942034)
\curveto(190.41861365,704.21941354)(190.4686136,704.18441358)(190.53861654,704.18442034)
\curveto(190.55861351,704.20441356)(190.57861349,704.21441355)(190.59861654,704.21442034)
\curveto(190.61861345,704.21441355)(190.63861343,704.22441354)(190.65861654,704.24442034)
\curveto(190.71861335,704.29441347)(190.7686133,704.34941341)(190.80861654,704.40942034)
\curveto(190.85861321,704.47941328)(190.91861315,704.53941322)(190.98861654,704.58942034)
\curveto(191.02861304,704.61941314)(191.063613,704.64941311)(191.09361654,704.67942034)
\curveto(191.12361294,704.71941304)(191.15861291,704.75441301)(191.19861654,704.78442034)
\lineto(191.46861654,704.96442034)
\curveto(191.5686125,705.02441274)(191.6686124,705.07941268)(191.76861654,705.12942034)
\curveto(191.8686122,705.16941259)(191.9686121,705.20441256)(192.06861654,705.23442034)
\lineto(192.39861654,705.32442034)
\curveto(192.42861164,705.33441243)(192.48361158,705.33441243)(192.56361654,705.32442034)
\curveto(192.65361141,705.32441244)(192.70861136,705.33441243)(192.72861654,705.35442034)
}
}
{
\newrgbcolor{curcolor}{0 0 0}
\pscustom[linestyle=none,fillstyle=solid,fillcolor=curcolor]
{
\newpath
\moveto(197.10369467,705.36942034)
\curveto(197.85369017,705.38941237)(198.50368952,705.30441246)(199.05369467,705.11442034)
\curveto(199.61368841,704.93441283)(200.03868798,704.61941314)(200.32869467,704.16942034)
\curveto(200.39868762,704.0594137)(200.45868756,703.94441382)(200.50869467,703.82442034)
\curveto(200.56868745,703.71441405)(200.6186874,703.58941417)(200.65869467,703.44942034)
\curveto(200.67868734,703.38941437)(200.68868733,703.32441444)(200.68869467,703.25442034)
\curveto(200.68868733,703.18441458)(200.67868734,703.12441464)(200.65869467,703.07442034)
\curveto(200.6186874,703.01441475)(200.56368746,702.97441479)(200.49369467,702.95442034)
\curveto(200.44368758,702.93441483)(200.38368764,702.92441484)(200.31369467,702.92442034)
\lineto(200.10369467,702.92442034)
\lineto(199.44369467,702.92442034)
\curveto(199.37368865,702.92441484)(199.30368872,702.91941484)(199.23369467,702.90942034)
\curveto(199.16368886,702.90941485)(199.09868892,702.91941484)(199.03869467,702.93942034)
\curveto(198.93868908,702.9594148)(198.86368916,702.99941476)(198.81369467,703.05942034)
\curveto(198.76368926,703.11941464)(198.7186893,703.17941458)(198.67869467,703.23942034)
\lineto(198.55869467,703.44942034)
\curveto(198.52868949,703.52941423)(198.47868954,703.59441417)(198.40869467,703.64442034)
\curveto(198.30868971,703.72441404)(198.20868981,703.78441398)(198.10869467,703.82442034)
\curveto(198.01869,703.8644139)(197.90369012,703.89941386)(197.76369467,703.92942034)
\curveto(197.69369033,703.94941381)(197.58869043,703.9644138)(197.44869467,703.97442034)
\curveto(197.3186907,703.98441378)(197.2186908,703.97941378)(197.14869467,703.95942034)
\lineto(197.04369467,703.95942034)
\lineto(196.89369467,703.92942034)
\curveto(196.85369117,703.92941383)(196.80869121,703.92441384)(196.75869467,703.91442034)
\curveto(196.58869143,703.8644139)(196.44869157,703.79441397)(196.33869467,703.70442034)
\curveto(196.23869178,703.62441414)(196.16869185,703.49941426)(196.12869467,703.32942034)
\curveto(196.10869191,703.2594145)(196.10869191,703.19441457)(196.12869467,703.13442034)
\curveto(196.14869187,703.07441469)(196.16869185,703.02441474)(196.18869467,702.98442034)
\curveto(196.25869176,702.8644149)(196.33869168,702.76941499)(196.42869467,702.69942034)
\curveto(196.52869149,702.62941513)(196.64369138,702.56941519)(196.77369467,702.51942034)
\curveto(196.96369106,702.43941532)(197.16869085,702.36941539)(197.38869467,702.30942034)
\lineto(198.07869467,702.15942034)
\curveto(198.3186897,702.11941564)(198.54868947,702.06941569)(198.76869467,702.00942034)
\curveto(198.99868902,701.9594158)(199.21368881,701.89441587)(199.41369467,701.81442034)
\curveto(199.50368852,701.77441599)(199.58868843,701.73941602)(199.66869467,701.70942034)
\curveto(199.75868826,701.68941607)(199.84368818,701.65441611)(199.92369467,701.60442034)
\curveto(200.11368791,701.48441628)(200.28368774,701.35441641)(200.43369467,701.21442034)
\curveto(200.59368743,701.07441669)(200.7186873,700.89941686)(200.80869467,700.68942034)
\curveto(200.83868718,700.61941714)(200.86368716,700.54941721)(200.88369467,700.47942034)
\curveto(200.90368712,700.40941735)(200.9236871,700.33441743)(200.94369467,700.25442034)
\curveto(200.95368707,700.19441757)(200.95868706,700.09941766)(200.95869467,699.96942034)
\curveto(200.96868705,699.84941791)(200.96868705,699.75441801)(200.95869467,699.68442034)
\lineto(200.95869467,699.60942034)
\curveto(200.93868708,699.54941821)(200.9236871,699.48941827)(200.91369467,699.42942034)
\curveto(200.91368711,699.37941838)(200.90868711,699.32941843)(200.89869467,699.27942034)
\curveto(200.82868719,698.97941878)(200.7186873,698.71441905)(200.56869467,698.48442034)
\curveto(200.40868761,698.24441952)(200.21368781,698.04941971)(199.98369467,697.89942034)
\curveto(199.75368827,697.74942001)(199.49368853,697.61942014)(199.20369467,697.50942034)
\curveto(199.09368893,697.4594203)(198.97368905,697.42442034)(198.84369467,697.40442034)
\curveto(198.7236893,697.38442038)(198.60368942,697.3594204)(198.48369467,697.32942034)
\curveto(198.39368963,697.30942045)(198.29868972,697.29942046)(198.19869467,697.29942034)
\curveto(198.10868991,697.28942047)(198.01869,697.27442049)(197.92869467,697.25442034)
\lineto(197.65869467,697.25442034)
\curveto(197.59869042,697.23442053)(197.49369053,697.22442054)(197.34369467,697.22442034)
\curveto(197.20369082,697.22442054)(197.10369092,697.23442053)(197.04369467,697.25442034)
\curveto(197.01369101,697.25442051)(196.97869104,697.2594205)(196.93869467,697.26942034)
\lineto(196.83369467,697.26942034)
\curveto(196.71369131,697.28942047)(196.59369143,697.30442046)(196.47369467,697.31442034)
\curveto(196.35369167,697.32442044)(196.23869178,697.34442042)(196.12869467,697.37442034)
\curveto(195.73869228,697.48442028)(195.39369263,697.60942015)(195.09369467,697.74942034)
\curveto(194.79369323,697.89941986)(194.53869348,698.11941964)(194.32869467,698.40942034)
\curveto(194.18869383,698.59941916)(194.06869395,698.81941894)(193.96869467,699.06942034)
\curveto(193.94869407,699.12941863)(193.92869409,699.20941855)(193.90869467,699.30942034)
\curveto(193.88869413,699.3594184)(193.87369415,699.42941833)(193.86369467,699.51942034)
\curveto(193.85369417,699.60941815)(193.85869416,699.68441808)(193.87869467,699.74442034)
\curveto(193.90869411,699.81441795)(193.95869406,699.8644179)(194.02869467,699.89442034)
\curveto(194.07869394,699.91441785)(194.13869388,699.92441784)(194.20869467,699.92442034)
\lineto(194.43369467,699.92442034)
\lineto(195.13869467,699.92442034)
\lineto(195.37869467,699.92442034)
\curveto(195.45869256,699.92441784)(195.52869249,699.91441785)(195.58869467,699.89442034)
\curveto(195.69869232,699.85441791)(195.76869225,699.78941797)(195.79869467,699.69942034)
\curveto(195.83869218,699.60941815)(195.88369214,699.51441825)(195.93369467,699.41442034)
\curveto(195.95369207,699.3644184)(195.98869203,699.29941846)(196.03869467,699.21942034)
\curveto(196.09869192,699.13941862)(196.14869187,699.08941867)(196.18869467,699.06942034)
\curveto(196.30869171,698.96941879)(196.4236916,698.88941887)(196.53369467,698.82942034)
\curveto(196.64369138,698.77941898)(196.78369124,698.72941903)(196.95369467,698.67942034)
\curveto(197.00369102,698.6594191)(197.05369097,698.64941911)(197.10369467,698.64942034)
\curveto(197.15369087,698.6594191)(197.20369082,698.6594191)(197.25369467,698.64942034)
\curveto(197.33369069,698.62941913)(197.4186906,698.61941914)(197.50869467,698.61942034)
\curveto(197.60869041,698.62941913)(197.69369033,698.64441912)(197.76369467,698.66442034)
\curveto(197.81369021,698.67441909)(197.85869016,698.67941908)(197.89869467,698.67942034)
\curveto(197.94869007,698.67941908)(197.99869002,698.68941907)(198.04869467,698.70942034)
\curveto(198.18868983,698.759419)(198.31368971,698.81941894)(198.42369467,698.88942034)
\curveto(198.54368948,698.9594188)(198.63868938,699.04941871)(198.70869467,699.15942034)
\curveto(198.75868926,699.23941852)(198.79868922,699.3644184)(198.82869467,699.53442034)
\curveto(198.84868917,699.60441816)(198.84868917,699.66941809)(198.82869467,699.72942034)
\curveto(198.80868921,699.78941797)(198.78868923,699.83941792)(198.76869467,699.87942034)
\curveto(198.69868932,700.01941774)(198.60868941,700.12441764)(198.49869467,700.19442034)
\curveto(198.39868962,700.2644175)(198.27868974,700.32941743)(198.13869467,700.38942034)
\curveto(197.94869007,700.46941729)(197.74869027,700.53441723)(197.53869467,700.58442034)
\curveto(197.32869069,700.63441713)(197.1186909,700.68941707)(196.90869467,700.74942034)
\curveto(196.82869119,700.76941699)(196.74369128,700.78441698)(196.65369467,700.79442034)
\curveto(196.57369145,700.80441696)(196.49369153,700.81941694)(196.41369467,700.83942034)
\curveto(196.09369193,700.92941683)(195.78869223,701.01441675)(195.49869467,701.09442034)
\curveto(195.20869281,701.18441658)(194.94369308,701.31441645)(194.70369467,701.48442034)
\curveto(194.4236936,701.68441608)(194.2186938,701.95441581)(194.08869467,702.29442034)
\curveto(194.06869395,702.3644154)(194.04869397,702.4594153)(194.02869467,702.57942034)
\curveto(194.00869401,702.64941511)(193.99369403,702.73441503)(193.98369467,702.83442034)
\curveto(193.97369405,702.93441483)(193.97869404,703.02441474)(193.99869467,703.10442034)
\curveto(194.018694,703.15441461)(194.023694,703.19441457)(194.01369467,703.22442034)
\curveto(194.00369402,703.2644145)(194.00869401,703.30941445)(194.02869467,703.35942034)
\curveto(194.04869397,703.46941429)(194.06869395,703.56941419)(194.08869467,703.65942034)
\curveto(194.1186939,703.759414)(194.15369387,703.85441391)(194.19369467,703.94442034)
\curveto(194.3236937,704.23441353)(194.50369352,704.46941329)(194.73369467,704.64942034)
\curveto(194.96369306,704.82941293)(195.2236928,704.97441279)(195.51369467,705.08442034)
\curveto(195.6236924,705.13441263)(195.73869228,705.16941259)(195.85869467,705.18942034)
\curveto(195.97869204,705.21941254)(196.10369192,705.24941251)(196.23369467,705.27942034)
\curveto(196.29369173,705.29941246)(196.35369167,705.30941245)(196.41369467,705.30942034)
\lineto(196.59369467,705.33942034)
\curveto(196.67369135,705.34941241)(196.75869126,705.35441241)(196.84869467,705.35442034)
\curveto(196.93869108,705.35441241)(197.023691,705.3594124)(197.10369467,705.36942034)
}
}
{
\newrgbcolor{curcolor}{0 0 0}
\pscustom[linestyle=none,fillstyle=solid,fillcolor=curcolor]
{
\newpath
\moveto(209.96033529,701.60442034)
\curveto(209.98032672,701.54441622)(209.99032671,701.4594163)(209.99033529,701.34942034)
\curveto(209.99032671,701.23941652)(209.98032672,701.15441661)(209.96033529,701.09442034)
\lineto(209.96033529,700.94442034)
\curveto(209.94032676,700.8644169)(209.93032677,700.78441698)(209.93033529,700.70442034)
\curveto(209.94032676,700.62441714)(209.93532677,700.54441722)(209.91533529,700.46442034)
\curveto(209.89532681,700.39441737)(209.88032682,700.32941743)(209.87033529,700.26942034)
\curveto(209.86032684,700.20941755)(209.85032685,700.14441762)(209.84033529,700.07442034)
\curveto(209.8003269,699.9644178)(209.76532694,699.84941791)(209.73533529,699.72942034)
\curveto(209.705327,699.61941814)(209.66532704,699.51441825)(209.61533529,699.41442034)
\curveto(209.4053273,698.93441883)(209.13032757,698.54441922)(208.79033529,698.24442034)
\curveto(208.45032825,697.94441982)(208.04032866,697.69442007)(207.56033529,697.49442034)
\curveto(207.44032926,697.44442032)(207.31532939,697.40942035)(207.18533529,697.38942034)
\curveto(207.06532964,697.3594204)(206.94032976,697.32942043)(206.81033529,697.29942034)
\curveto(206.76032994,697.27942048)(206.70533,697.26942049)(206.64533529,697.26942034)
\curveto(206.58533012,697.26942049)(206.53033017,697.2644205)(206.48033529,697.25442034)
\lineto(206.37533529,697.25442034)
\curveto(206.34533036,697.24442052)(206.31533039,697.23942052)(206.28533529,697.23942034)
\curveto(206.23533047,697.22942053)(206.15533055,697.22442054)(206.04533529,697.22442034)
\curveto(205.93533077,697.21442055)(205.85033085,697.21942054)(205.79033529,697.23942034)
\lineto(205.64033529,697.23942034)
\curveto(205.59033111,697.24942051)(205.53533117,697.25442051)(205.47533529,697.25442034)
\curveto(205.42533128,697.24442052)(205.37533133,697.24942051)(205.32533529,697.26942034)
\curveto(205.28533142,697.27942048)(205.24533146,697.28442048)(205.20533529,697.28442034)
\curveto(205.17533153,697.28442048)(205.13533157,697.28942047)(205.08533529,697.29942034)
\curveto(204.98533172,697.32942043)(204.88533182,697.35442041)(204.78533529,697.37442034)
\curveto(204.68533202,697.39442037)(204.59033211,697.42442034)(204.50033529,697.46442034)
\curveto(204.38033232,697.50442026)(204.26533244,697.54442022)(204.15533529,697.58442034)
\curveto(204.05533265,697.62442014)(203.95033275,697.67442009)(203.84033529,697.73442034)
\curveto(203.49033321,697.94441982)(203.19033351,698.18941957)(202.94033529,698.46942034)
\curveto(202.69033401,698.74941901)(202.48033422,699.08441868)(202.31033529,699.47442034)
\curveto(202.26033444,699.5644182)(202.22033448,699.6594181)(202.19033529,699.75942034)
\curveto(202.17033453,699.8594179)(202.14533456,699.9644178)(202.11533529,700.07442034)
\curveto(202.09533461,700.12441764)(202.08533462,700.16941759)(202.08533529,700.20942034)
\curveto(202.08533462,700.24941751)(202.07533463,700.29441747)(202.05533529,700.34442034)
\curveto(202.03533467,700.42441734)(202.02533468,700.50441726)(202.02533529,700.58442034)
\curveto(202.02533468,700.67441709)(202.01533469,700.759417)(201.99533529,700.83942034)
\curveto(201.98533472,700.88941687)(201.98033472,700.93441683)(201.98033529,700.97442034)
\lineto(201.98033529,701.10942034)
\curveto(201.96033474,701.16941659)(201.95033475,701.25441651)(201.95033529,701.36442034)
\curveto(201.96033474,701.47441629)(201.97533473,701.5594162)(201.99533529,701.61942034)
\lineto(201.99533529,701.72442034)
\curveto(202.0053347,701.77441599)(202.0053347,701.82441594)(201.99533529,701.87442034)
\curveto(201.99533471,701.93441583)(202.0053347,701.98941577)(202.02533529,702.03942034)
\curveto(202.03533467,702.08941567)(202.04033466,702.13441563)(202.04033529,702.17442034)
\curveto(202.04033466,702.22441554)(202.05033465,702.27441549)(202.07033529,702.32442034)
\curveto(202.11033459,702.45441531)(202.14533456,702.57941518)(202.17533529,702.69942034)
\curveto(202.2053345,702.82941493)(202.24533446,702.95441481)(202.29533529,703.07442034)
\curveto(202.47533423,703.48441428)(202.69033401,703.82441394)(202.94033529,704.09442034)
\curveto(203.19033351,704.37441339)(203.49533321,704.62941313)(203.85533529,704.85942034)
\curveto(203.95533275,704.90941285)(204.06033264,704.95441281)(204.17033529,704.99442034)
\curveto(204.28033242,705.03441273)(204.39033231,705.07941268)(204.50033529,705.12942034)
\curveto(204.63033207,705.17941258)(204.76533194,705.21441255)(204.90533529,705.23442034)
\curveto(205.04533166,705.25441251)(205.19033151,705.28441248)(205.34033529,705.32442034)
\curveto(205.42033128,705.33441243)(205.49533121,705.33941242)(205.56533529,705.33942034)
\curveto(205.63533107,705.33941242)(205.705331,705.34441242)(205.77533529,705.35442034)
\curveto(206.35533035,705.3644124)(206.85532985,705.30441246)(207.27533529,705.17442034)
\curveto(207.705329,705.04441272)(208.08532862,704.8644129)(208.41533529,704.63442034)
\curveto(208.52532818,704.55441321)(208.63532807,704.4644133)(208.74533529,704.36442034)
\curveto(208.86532784,704.27441349)(208.96532774,704.17441359)(209.04533529,704.06442034)
\curveto(209.12532758,703.9644138)(209.19532751,703.8644139)(209.25533529,703.76442034)
\curveto(209.32532738,703.6644141)(209.39532731,703.5594142)(209.46533529,703.44942034)
\curveto(209.53532717,703.33941442)(209.59032711,703.21941454)(209.63033529,703.08942034)
\curveto(209.67032703,702.96941479)(209.71532699,702.83941492)(209.76533529,702.69942034)
\curveto(209.79532691,702.61941514)(209.82032688,702.53441523)(209.84033529,702.44442034)
\lineto(209.90033529,702.17442034)
\curveto(209.91032679,702.13441563)(209.91532679,702.09441567)(209.91533529,702.05442034)
\curveto(209.91532679,702.01441575)(209.92032678,701.97441579)(209.93033529,701.93442034)
\curveto(209.95032675,701.88441588)(209.95532675,701.82941593)(209.94533529,701.76942034)
\curveto(209.93532677,701.70941605)(209.94032676,701.65441611)(209.96033529,701.60442034)
\moveto(207.86033529,701.06442034)
\curveto(207.87032883,701.11441665)(207.87532883,701.18441658)(207.87533529,701.27442034)
\curveto(207.87532883,701.37441639)(207.87032883,701.44941631)(207.86033529,701.49942034)
\lineto(207.86033529,701.61942034)
\curveto(207.84032886,701.66941609)(207.83032887,701.72441604)(207.83033529,701.78442034)
\curveto(207.83032887,701.84441592)(207.82532888,701.89941586)(207.81533529,701.94942034)
\curveto(207.81532889,701.98941577)(207.81032889,702.01941574)(207.80033529,702.03942034)
\lineto(207.74033529,702.27942034)
\curveto(207.73032897,702.36941539)(207.71032899,702.45441531)(207.68033529,702.53442034)
\curveto(207.57032913,702.79441497)(207.44032926,703.01441475)(207.29033529,703.19442034)
\curveto(207.14032956,703.38441438)(206.94032976,703.53441423)(206.69033529,703.64442034)
\curveto(206.63033007,703.6644141)(206.57033013,703.67941408)(206.51033529,703.68942034)
\curveto(206.45033025,703.70941405)(206.38533032,703.72941403)(206.31533529,703.74942034)
\curveto(206.23533047,703.76941399)(206.15033055,703.77441399)(206.06033529,703.76442034)
\lineto(205.79033529,703.76442034)
\curveto(205.76033094,703.74441402)(205.72533098,703.73441403)(205.68533529,703.73442034)
\curveto(205.64533106,703.74441402)(205.61033109,703.74441402)(205.58033529,703.73442034)
\lineto(205.37033529,703.67442034)
\curveto(205.31033139,703.6644141)(205.25533145,703.64441412)(205.20533529,703.61442034)
\curveto(204.95533175,703.50441426)(204.75033195,703.34441442)(204.59033529,703.13442034)
\curveto(204.44033226,702.93441483)(204.32033238,702.69941506)(204.23033529,702.42942034)
\curveto(204.2003325,702.32941543)(204.17533253,702.22441554)(204.15533529,702.11442034)
\curveto(204.14533256,702.00441576)(204.13033257,701.89441587)(204.11033529,701.78442034)
\curveto(204.1003326,701.73441603)(204.09533261,701.68441608)(204.09533529,701.63442034)
\lineto(204.09533529,701.48442034)
\curveto(204.07533263,701.41441635)(204.06533264,701.30941645)(204.06533529,701.16942034)
\curveto(204.07533263,701.02941673)(204.09033261,700.92441684)(204.11033529,700.85442034)
\lineto(204.11033529,700.71942034)
\curveto(204.13033257,700.63941712)(204.14533256,700.5594172)(204.15533529,700.47942034)
\curveto(204.16533254,700.40941735)(204.18033252,700.33441743)(204.20033529,700.25442034)
\curveto(204.3003324,699.95441781)(204.4053323,699.70941805)(204.51533529,699.51942034)
\curveto(204.63533207,699.33941842)(204.82033188,699.17441859)(205.07033529,699.02442034)
\curveto(205.14033156,698.97441879)(205.21533149,698.93441883)(205.29533529,698.90442034)
\curveto(205.38533132,698.87441889)(205.47533123,698.84941891)(205.56533529,698.82942034)
\curveto(205.6053311,698.81941894)(205.64033106,698.81441895)(205.67033529,698.81442034)
\curveto(205.700331,698.82441894)(205.73533097,698.82441894)(205.77533529,698.81442034)
\lineto(205.89533529,698.78442034)
\curveto(205.94533076,698.78441898)(205.99033071,698.78941897)(206.03033529,698.79942034)
\lineto(206.15033529,698.79942034)
\curveto(206.23033047,698.81941894)(206.31033039,698.83441893)(206.39033529,698.84442034)
\curveto(206.47033023,698.85441891)(206.54533016,698.87441889)(206.61533529,698.90442034)
\curveto(206.87532983,699.00441876)(207.08532962,699.13941862)(207.24533529,699.30942034)
\curveto(207.4053293,699.47941828)(207.54032916,699.68941807)(207.65033529,699.93942034)
\curveto(207.69032901,700.03941772)(207.72032898,700.13941762)(207.74033529,700.23942034)
\curveto(207.76032894,700.33941742)(207.78532892,700.44441732)(207.81533529,700.55442034)
\curveto(207.82532888,700.59441717)(207.83032887,700.62941713)(207.83033529,700.65942034)
\curveto(207.83032887,700.69941706)(207.83532887,700.73941702)(207.84533529,700.77942034)
\lineto(207.84533529,700.91442034)
\curveto(207.84532886,700.9644168)(207.85032885,701.01441675)(207.86033529,701.06442034)
}
}
{
\newrgbcolor{curcolor}{0 0 0}
\pscustom[linestyle=none,fillstyle=solid,fillcolor=curcolor]
{
\newpath
\moveto(214.33025717,705.36942034)
\curveto(215.08025267,705.38941237)(215.73025202,705.30441246)(216.28025717,705.11442034)
\curveto(216.84025091,704.93441283)(217.26525048,704.61941314)(217.55525717,704.16942034)
\curveto(217.62525012,704.0594137)(217.68525006,703.94441382)(217.73525717,703.82442034)
\curveto(217.79524995,703.71441405)(217.8452499,703.58941417)(217.88525717,703.44942034)
\curveto(217.90524984,703.38941437)(217.91524983,703.32441444)(217.91525717,703.25442034)
\curveto(217.91524983,703.18441458)(217.90524984,703.12441464)(217.88525717,703.07442034)
\curveto(217.8452499,703.01441475)(217.79024996,702.97441479)(217.72025717,702.95442034)
\curveto(217.67025008,702.93441483)(217.61025014,702.92441484)(217.54025717,702.92442034)
\lineto(217.33025717,702.92442034)
\lineto(216.67025717,702.92442034)
\curveto(216.60025115,702.92441484)(216.53025122,702.91941484)(216.46025717,702.90942034)
\curveto(216.39025136,702.90941485)(216.32525142,702.91941484)(216.26525717,702.93942034)
\curveto(216.16525158,702.9594148)(216.09025166,702.99941476)(216.04025717,703.05942034)
\curveto(215.99025176,703.11941464)(215.9452518,703.17941458)(215.90525717,703.23942034)
\lineto(215.78525717,703.44942034)
\curveto(215.75525199,703.52941423)(215.70525204,703.59441417)(215.63525717,703.64442034)
\curveto(215.53525221,703.72441404)(215.43525231,703.78441398)(215.33525717,703.82442034)
\curveto(215.2452525,703.8644139)(215.13025262,703.89941386)(214.99025717,703.92942034)
\curveto(214.92025283,703.94941381)(214.81525293,703.9644138)(214.67525717,703.97442034)
\curveto(214.5452532,703.98441378)(214.4452533,703.97941378)(214.37525717,703.95942034)
\lineto(214.27025717,703.95942034)
\lineto(214.12025717,703.92942034)
\curveto(214.08025367,703.92941383)(214.03525371,703.92441384)(213.98525717,703.91442034)
\curveto(213.81525393,703.8644139)(213.67525407,703.79441397)(213.56525717,703.70442034)
\curveto(213.46525428,703.62441414)(213.39525435,703.49941426)(213.35525717,703.32942034)
\curveto(213.33525441,703.2594145)(213.33525441,703.19441457)(213.35525717,703.13442034)
\curveto(213.37525437,703.07441469)(213.39525435,703.02441474)(213.41525717,702.98442034)
\curveto(213.48525426,702.8644149)(213.56525418,702.76941499)(213.65525717,702.69942034)
\curveto(213.75525399,702.62941513)(213.87025388,702.56941519)(214.00025717,702.51942034)
\curveto(214.19025356,702.43941532)(214.39525335,702.36941539)(214.61525717,702.30942034)
\lineto(215.30525717,702.15942034)
\curveto(215.5452522,702.11941564)(215.77525197,702.06941569)(215.99525717,702.00942034)
\curveto(216.22525152,701.9594158)(216.44025131,701.89441587)(216.64025717,701.81442034)
\curveto(216.73025102,701.77441599)(216.81525093,701.73941602)(216.89525717,701.70942034)
\curveto(216.98525076,701.68941607)(217.07025068,701.65441611)(217.15025717,701.60442034)
\curveto(217.34025041,701.48441628)(217.51025024,701.35441641)(217.66025717,701.21442034)
\curveto(217.82024993,701.07441669)(217.9452498,700.89941686)(218.03525717,700.68942034)
\curveto(218.06524968,700.61941714)(218.09024966,700.54941721)(218.11025717,700.47942034)
\curveto(218.13024962,700.40941735)(218.1502496,700.33441743)(218.17025717,700.25442034)
\curveto(218.18024957,700.19441757)(218.18524956,700.09941766)(218.18525717,699.96942034)
\curveto(218.19524955,699.84941791)(218.19524955,699.75441801)(218.18525717,699.68442034)
\lineto(218.18525717,699.60942034)
\curveto(218.16524958,699.54941821)(218.1502496,699.48941827)(218.14025717,699.42942034)
\curveto(218.14024961,699.37941838)(218.13524961,699.32941843)(218.12525717,699.27942034)
\curveto(218.05524969,698.97941878)(217.9452498,698.71441905)(217.79525717,698.48442034)
\curveto(217.63525011,698.24441952)(217.44025031,698.04941971)(217.21025717,697.89942034)
\curveto(216.98025077,697.74942001)(216.72025103,697.61942014)(216.43025717,697.50942034)
\curveto(216.32025143,697.4594203)(216.20025155,697.42442034)(216.07025717,697.40442034)
\curveto(215.9502518,697.38442038)(215.83025192,697.3594204)(215.71025717,697.32942034)
\curveto(215.62025213,697.30942045)(215.52525222,697.29942046)(215.42525717,697.29942034)
\curveto(215.33525241,697.28942047)(215.2452525,697.27442049)(215.15525717,697.25442034)
\lineto(214.88525717,697.25442034)
\curveto(214.82525292,697.23442053)(214.72025303,697.22442054)(214.57025717,697.22442034)
\curveto(214.43025332,697.22442054)(214.33025342,697.23442053)(214.27025717,697.25442034)
\curveto(214.24025351,697.25442051)(214.20525354,697.2594205)(214.16525717,697.26942034)
\lineto(214.06025717,697.26942034)
\curveto(213.94025381,697.28942047)(213.82025393,697.30442046)(213.70025717,697.31442034)
\curveto(213.58025417,697.32442044)(213.46525428,697.34442042)(213.35525717,697.37442034)
\curveto(212.96525478,697.48442028)(212.62025513,697.60942015)(212.32025717,697.74942034)
\curveto(212.02025573,697.89941986)(211.76525598,698.11941964)(211.55525717,698.40942034)
\curveto(211.41525633,698.59941916)(211.29525645,698.81941894)(211.19525717,699.06942034)
\curveto(211.17525657,699.12941863)(211.15525659,699.20941855)(211.13525717,699.30942034)
\curveto(211.11525663,699.3594184)(211.10025665,699.42941833)(211.09025717,699.51942034)
\curveto(211.08025667,699.60941815)(211.08525666,699.68441808)(211.10525717,699.74442034)
\curveto(211.13525661,699.81441795)(211.18525656,699.8644179)(211.25525717,699.89442034)
\curveto(211.30525644,699.91441785)(211.36525638,699.92441784)(211.43525717,699.92442034)
\lineto(211.66025717,699.92442034)
\lineto(212.36525717,699.92442034)
\lineto(212.60525717,699.92442034)
\curveto(212.68525506,699.92441784)(212.75525499,699.91441785)(212.81525717,699.89442034)
\curveto(212.92525482,699.85441791)(212.99525475,699.78941797)(213.02525717,699.69942034)
\curveto(213.06525468,699.60941815)(213.11025464,699.51441825)(213.16025717,699.41442034)
\curveto(213.18025457,699.3644184)(213.21525453,699.29941846)(213.26525717,699.21942034)
\curveto(213.32525442,699.13941862)(213.37525437,699.08941867)(213.41525717,699.06942034)
\curveto(213.53525421,698.96941879)(213.6502541,698.88941887)(213.76025717,698.82942034)
\curveto(213.87025388,698.77941898)(214.01025374,698.72941903)(214.18025717,698.67942034)
\curveto(214.23025352,698.6594191)(214.28025347,698.64941911)(214.33025717,698.64942034)
\curveto(214.38025337,698.6594191)(214.43025332,698.6594191)(214.48025717,698.64942034)
\curveto(214.56025319,698.62941913)(214.6452531,698.61941914)(214.73525717,698.61942034)
\curveto(214.83525291,698.62941913)(214.92025283,698.64441912)(214.99025717,698.66442034)
\curveto(215.04025271,698.67441909)(215.08525266,698.67941908)(215.12525717,698.67942034)
\curveto(215.17525257,698.67941908)(215.22525252,698.68941907)(215.27525717,698.70942034)
\curveto(215.41525233,698.759419)(215.54025221,698.81941894)(215.65025717,698.88942034)
\curveto(215.77025198,698.9594188)(215.86525188,699.04941871)(215.93525717,699.15942034)
\curveto(215.98525176,699.23941852)(216.02525172,699.3644184)(216.05525717,699.53442034)
\curveto(216.07525167,699.60441816)(216.07525167,699.66941809)(216.05525717,699.72942034)
\curveto(216.03525171,699.78941797)(216.01525173,699.83941792)(215.99525717,699.87942034)
\curveto(215.92525182,700.01941774)(215.83525191,700.12441764)(215.72525717,700.19442034)
\curveto(215.62525212,700.2644175)(215.50525224,700.32941743)(215.36525717,700.38942034)
\curveto(215.17525257,700.46941729)(214.97525277,700.53441723)(214.76525717,700.58442034)
\curveto(214.55525319,700.63441713)(214.3452534,700.68941707)(214.13525717,700.74942034)
\curveto(214.05525369,700.76941699)(213.97025378,700.78441698)(213.88025717,700.79442034)
\curveto(213.80025395,700.80441696)(213.72025403,700.81941694)(213.64025717,700.83942034)
\curveto(213.32025443,700.92941683)(213.01525473,701.01441675)(212.72525717,701.09442034)
\curveto(212.43525531,701.18441658)(212.17025558,701.31441645)(211.93025717,701.48442034)
\curveto(211.6502561,701.68441608)(211.4452563,701.95441581)(211.31525717,702.29442034)
\curveto(211.29525645,702.3644154)(211.27525647,702.4594153)(211.25525717,702.57942034)
\curveto(211.23525651,702.64941511)(211.22025653,702.73441503)(211.21025717,702.83442034)
\curveto(211.20025655,702.93441483)(211.20525654,703.02441474)(211.22525717,703.10442034)
\curveto(211.2452565,703.15441461)(211.2502565,703.19441457)(211.24025717,703.22442034)
\curveto(211.23025652,703.2644145)(211.23525651,703.30941445)(211.25525717,703.35942034)
\curveto(211.27525647,703.46941429)(211.29525645,703.56941419)(211.31525717,703.65942034)
\curveto(211.3452564,703.759414)(211.38025637,703.85441391)(211.42025717,703.94442034)
\curveto(211.5502562,704.23441353)(211.73025602,704.46941329)(211.96025717,704.64942034)
\curveto(212.19025556,704.82941293)(212.4502553,704.97441279)(212.74025717,705.08442034)
\curveto(212.8502549,705.13441263)(212.96525478,705.16941259)(213.08525717,705.18942034)
\curveto(213.20525454,705.21941254)(213.33025442,705.24941251)(213.46025717,705.27942034)
\curveto(213.52025423,705.29941246)(213.58025417,705.30941245)(213.64025717,705.30942034)
\lineto(213.82025717,705.33942034)
\curveto(213.90025385,705.34941241)(213.98525376,705.35441241)(214.07525717,705.35442034)
\curveto(214.16525358,705.35441241)(214.2502535,705.3594124)(214.33025717,705.36942034)
}
}
{
\newrgbcolor{curcolor}{0 0 0}
\pscustom[linestyle=none,fillstyle=solid,fillcolor=curcolor]
{
\newpath
\moveto(21.58611654,686.61942034)
\curveto(22.33611204,686.63941237)(22.98611139,686.55441246)(23.53611654,686.36442034)
\curveto(24.09611028,686.18441283)(24.52110986,685.86941314)(24.81111654,685.41942034)
\curveto(24.8811095,685.3094137)(24.94110944,685.19441382)(24.99111654,685.07442034)
\curveto(25.05110933,684.96441405)(25.10110928,684.83941417)(25.14111654,684.69942034)
\curveto(25.16110922,684.63941437)(25.17110921,684.57441444)(25.17111654,684.50442034)
\curveto(25.17110921,684.43441458)(25.16110922,684.37441464)(25.14111654,684.32442034)
\curveto(25.10110928,684.26441475)(25.04610933,684.22441479)(24.97611654,684.20442034)
\curveto(24.92610945,684.18441483)(24.86610951,684.17441484)(24.79611654,684.17442034)
\lineto(24.58611654,684.17442034)
\lineto(23.92611654,684.17442034)
\curveto(23.85611052,684.17441484)(23.78611059,684.16941484)(23.71611654,684.15942034)
\curveto(23.64611073,684.15941485)(23.5811108,684.16941484)(23.52111654,684.18942034)
\curveto(23.42111096,684.2094148)(23.34611103,684.24941476)(23.29611654,684.30942034)
\curveto(23.24611113,684.36941464)(23.20111118,684.42941458)(23.16111654,684.48942034)
\lineto(23.04111654,684.69942034)
\curveto(23.01111137,684.77941423)(22.96111142,684.84441417)(22.89111654,684.89442034)
\curveto(22.79111159,684.97441404)(22.69111169,685.03441398)(22.59111654,685.07442034)
\curveto(22.50111188,685.1144139)(22.38611199,685.14941386)(22.24611654,685.17942034)
\curveto(22.1761122,685.19941381)(22.07111231,685.2144138)(21.93111654,685.22442034)
\curveto(21.80111258,685.23441378)(21.70111268,685.22941378)(21.63111654,685.20942034)
\lineto(21.52611654,685.20942034)
\lineto(21.37611654,685.17942034)
\curveto(21.33611304,685.17941383)(21.29111309,685.17441384)(21.24111654,685.16442034)
\curveto(21.07111331,685.1144139)(20.93111345,685.04441397)(20.82111654,684.95442034)
\curveto(20.72111366,684.87441414)(20.65111373,684.74941426)(20.61111654,684.57942034)
\curveto(20.59111379,684.5094145)(20.59111379,684.44441457)(20.61111654,684.38442034)
\curveto(20.63111375,684.32441469)(20.65111373,684.27441474)(20.67111654,684.23442034)
\curveto(20.74111364,684.1144149)(20.82111356,684.01941499)(20.91111654,683.94942034)
\curveto(21.01111337,683.87941513)(21.12611325,683.81941519)(21.25611654,683.76942034)
\curveto(21.44611293,683.68941532)(21.65111273,683.61941539)(21.87111654,683.55942034)
\lineto(22.56111654,683.40942034)
\curveto(22.80111158,683.36941564)(23.03111135,683.31941569)(23.25111654,683.25942034)
\curveto(23.4811109,683.2094158)(23.69611068,683.14441587)(23.89611654,683.06442034)
\curveto(23.98611039,683.02441599)(24.07111031,682.98941602)(24.15111654,682.95942034)
\curveto(24.24111014,682.93941607)(24.32611005,682.90441611)(24.40611654,682.85442034)
\curveto(24.59610978,682.73441628)(24.76610961,682.60441641)(24.91611654,682.46442034)
\curveto(25.0761093,682.32441669)(25.20110918,682.14941686)(25.29111654,681.93942034)
\curveto(25.32110906,681.86941714)(25.34610903,681.79941721)(25.36611654,681.72942034)
\curveto(25.38610899,681.65941735)(25.40610897,681.58441743)(25.42611654,681.50442034)
\curveto(25.43610894,681.44441757)(25.44110894,681.34941766)(25.44111654,681.21942034)
\curveto(25.45110893,681.09941791)(25.45110893,681.00441801)(25.44111654,680.93442034)
\lineto(25.44111654,680.85942034)
\curveto(25.42110896,680.79941821)(25.40610897,680.73941827)(25.39611654,680.67942034)
\curveto(25.39610898,680.62941838)(25.39110899,680.57941843)(25.38111654,680.52942034)
\curveto(25.31110907,680.22941878)(25.20110918,679.96441905)(25.05111654,679.73442034)
\curveto(24.89110949,679.49441952)(24.69610968,679.29941971)(24.46611654,679.14942034)
\curveto(24.23611014,678.99942001)(23.9761104,678.86942014)(23.68611654,678.75942034)
\curveto(23.5761108,678.7094203)(23.45611092,678.67442034)(23.32611654,678.65442034)
\curveto(23.20611117,678.63442038)(23.08611129,678.6094204)(22.96611654,678.57942034)
\curveto(22.8761115,678.55942045)(22.7811116,678.54942046)(22.68111654,678.54942034)
\curveto(22.59111179,678.53942047)(22.50111188,678.52442049)(22.41111654,678.50442034)
\lineto(22.14111654,678.50442034)
\curveto(22.0811123,678.48442053)(21.9761124,678.47442054)(21.82611654,678.47442034)
\curveto(21.68611269,678.47442054)(21.58611279,678.48442053)(21.52611654,678.50442034)
\curveto(21.49611288,678.50442051)(21.46111292,678.5094205)(21.42111654,678.51942034)
\lineto(21.31611654,678.51942034)
\curveto(21.19611318,678.53942047)(21.0761133,678.55442046)(20.95611654,678.56442034)
\curveto(20.83611354,678.57442044)(20.72111366,678.59442042)(20.61111654,678.62442034)
\curveto(20.22111416,678.73442028)(19.8761145,678.85942015)(19.57611654,678.99942034)
\curveto(19.2761151,679.14941986)(19.02111536,679.36941964)(18.81111654,679.65942034)
\curveto(18.67111571,679.84941916)(18.55111583,680.06941894)(18.45111654,680.31942034)
\curveto(18.43111595,680.37941863)(18.41111597,680.45941855)(18.39111654,680.55942034)
\curveto(18.37111601,680.6094184)(18.35611602,680.67941833)(18.34611654,680.76942034)
\curveto(18.33611604,680.85941815)(18.34111604,680.93441808)(18.36111654,680.99442034)
\curveto(18.39111599,681.06441795)(18.44111594,681.1144179)(18.51111654,681.14442034)
\curveto(18.56111582,681.16441785)(18.62111576,681.17441784)(18.69111654,681.17442034)
\lineto(18.91611654,681.17442034)
\lineto(19.62111654,681.17442034)
\lineto(19.86111654,681.17442034)
\curveto(19.94111444,681.17441784)(20.01111437,681.16441785)(20.07111654,681.14442034)
\curveto(20.1811142,681.10441791)(20.25111413,681.03941797)(20.28111654,680.94942034)
\curveto(20.32111406,680.85941815)(20.36611401,680.76441825)(20.41611654,680.66442034)
\curveto(20.43611394,680.6144184)(20.47111391,680.54941846)(20.52111654,680.46942034)
\curveto(20.5811138,680.38941862)(20.63111375,680.33941867)(20.67111654,680.31942034)
\curveto(20.79111359,680.21941879)(20.90611347,680.13941887)(21.01611654,680.07942034)
\curveto(21.12611325,680.02941898)(21.26611311,679.97941903)(21.43611654,679.92942034)
\curveto(21.48611289,679.9094191)(21.53611284,679.89941911)(21.58611654,679.89942034)
\curveto(21.63611274,679.9094191)(21.68611269,679.9094191)(21.73611654,679.89942034)
\curveto(21.81611256,679.87941913)(21.90111248,679.86941914)(21.99111654,679.86942034)
\curveto(22.09111229,679.87941913)(22.1761122,679.89441912)(22.24611654,679.91442034)
\curveto(22.29611208,679.92441909)(22.34111204,679.92941908)(22.38111654,679.92942034)
\curveto(22.43111195,679.92941908)(22.4811119,679.93941907)(22.53111654,679.95942034)
\curveto(22.67111171,680.009419)(22.79611158,680.06941894)(22.90611654,680.13942034)
\curveto(23.02611135,680.2094188)(23.12111126,680.29941871)(23.19111654,680.40942034)
\curveto(23.24111114,680.48941852)(23.2811111,680.6144184)(23.31111654,680.78442034)
\curveto(23.33111105,680.85441816)(23.33111105,680.91941809)(23.31111654,680.97942034)
\curveto(23.29111109,681.03941797)(23.27111111,681.08941792)(23.25111654,681.12942034)
\curveto(23.1811112,681.26941774)(23.09111129,681.37441764)(22.98111654,681.44442034)
\curveto(22.8811115,681.5144175)(22.76111162,681.57941743)(22.62111654,681.63942034)
\curveto(22.43111195,681.71941729)(22.23111215,681.78441723)(22.02111654,681.83442034)
\curveto(21.81111257,681.88441713)(21.60111278,681.93941707)(21.39111654,681.99942034)
\curveto(21.31111307,682.01941699)(21.22611315,682.03441698)(21.13611654,682.04442034)
\curveto(21.05611332,682.05441696)(20.9761134,682.06941694)(20.89611654,682.08942034)
\curveto(20.5761138,682.17941683)(20.27111411,682.26441675)(19.98111654,682.34442034)
\curveto(19.69111469,682.43441658)(19.42611495,682.56441645)(19.18611654,682.73442034)
\curveto(18.90611547,682.93441608)(18.70111568,683.20441581)(18.57111654,683.54442034)
\curveto(18.55111583,683.6144154)(18.53111585,683.7094153)(18.51111654,683.82942034)
\curveto(18.49111589,683.89941511)(18.4761159,683.98441503)(18.46611654,684.08442034)
\curveto(18.45611592,684.18441483)(18.46111592,684.27441474)(18.48111654,684.35442034)
\curveto(18.50111588,684.40441461)(18.50611587,684.44441457)(18.49611654,684.47442034)
\curveto(18.48611589,684.5144145)(18.49111589,684.55941445)(18.51111654,684.60942034)
\curveto(18.53111585,684.71941429)(18.55111583,684.81941419)(18.57111654,684.90942034)
\curveto(18.60111578,685.009414)(18.63611574,685.10441391)(18.67611654,685.19442034)
\curveto(18.80611557,685.48441353)(18.98611539,685.71941329)(19.21611654,685.89942034)
\curveto(19.44611493,686.07941293)(19.70611467,686.22441279)(19.99611654,686.33442034)
\curveto(20.10611427,686.38441263)(20.22111416,686.41941259)(20.34111654,686.43942034)
\curveto(20.46111392,686.46941254)(20.58611379,686.49941251)(20.71611654,686.52942034)
\curveto(20.7761136,686.54941246)(20.83611354,686.55941245)(20.89611654,686.55942034)
\lineto(21.07611654,686.58942034)
\curveto(21.15611322,686.59941241)(21.24111314,686.60441241)(21.33111654,686.60442034)
\curveto(21.42111296,686.60441241)(21.50611287,686.6094124)(21.58611654,686.61942034)
}
}
{
\newrgbcolor{curcolor}{0 0 0}
\pscustom[linestyle=none,fillstyle=solid,fillcolor=curcolor]
{
\newpath
\moveto(34.03775717,682.61442034)
\curveto(34.057749,682.53441648)(34.057749,682.44441657)(34.03775717,682.34442034)
\curveto(34.01774904,682.24441677)(33.98274908,682.17941683)(33.93275717,682.14942034)
\curveto(33.88274918,682.1094169)(33.80774925,682.07941693)(33.70775717,682.05942034)
\curveto(33.61774944,682.04941696)(33.51274955,682.03941697)(33.39275717,682.02942034)
\lineto(33.04775717,682.02942034)
\curveto(32.93775012,682.03941697)(32.83775022,682.04441697)(32.74775717,682.04442034)
\lineto(29.08775717,682.04442034)
\lineto(28.87775717,682.04442034)
\curveto(28.81775424,682.04441697)(28.7627543,682.03441698)(28.71275717,682.01442034)
\curveto(28.63275443,681.97441704)(28.58275448,681.93441708)(28.56275717,681.89442034)
\curveto(28.54275452,681.87441714)(28.52275454,681.83441718)(28.50275717,681.77442034)
\curveto(28.48275458,681.72441729)(28.47775458,681.67441734)(28.48775717,681.62442034)
\curveto(28.50775455,681.56441745)(28.51775454,681.50441751)(28.51775717,681.44442034)
\curveto(28.52775453,681.39441762)(28.54275452,681.33941767)(28.56275717,681.27942034)
\curveto(28.64275442,681.03941797)(28.73775432,680.83941817)(28.84775717,680.67942034)
\curveto(28.96775409,680.52941848)(29.12775393,680.39441862)(29.32775717,680.27442034)
\curveto(29.40775365,680.22441879)(29.48775357,680.18941882)(29.56775717,680.16942034)
\curveto(29.6577534,680.15941885)(29.74775331,680.13941887)(29.83775717,680.10942034)
\curveto(29.91775314,680.08941892)(30.02775303,680.07441894)(30.16775717,680.06442034)
\curveto(30.30775275,680.05441896)(30.42775263,680.05941895)(30.52775717,680.07942034)
\lineto(30.66275717,680.07942034)
\curveto(30.7627523,680.09941891)(30.85275221,680.11941889)(30.93275717,680.13942034)
\curveto(31.02275204,680.16941884)(31.10775195,680.19941881)(31.18775717,680.22942034)
\curveto(31.28775177,680.27941873)(31.39775166,680.34441867)(31.51775717,680.42442034)
\curveto(31.64775141,680.50441851)(31.74275132,680.58441843)(31.80275717,680.66442034)
\curveto(31.85275121,680.73441828)(31.90275116,680.79941821)(31.95275717,680.85942034)
\curveto(32.01275105,680.92941808)(32.08275098,680.97941803)(32.16275717,681.00942034)
\curveto(32.2627508,681.05941795)(32.38775067,681.07941793)(32.53775717,681.06942034)
\lineto(32.97275717,681.06942034)
\lineto(33.15275717,681.06942034)
\curveto(33.22274984,681.07941793)(33.28274978,681.07441794)(33.33275717,681.05442034)
\lineto(33.48275717,681.05442034)
\curveto(33.58274948,681.03441798)(33.65274941,681.009418)(33.69275717,680.97942034)
\curveto(33.73274933,680.95941805)(33.75274931,680.9144181)(33.75275717,680.84442034)
\curveto(33.7627493,680.77441824)(33.7577493,680.7144183)(33.73775717,680.66442034)
\curveto(33.68774937,680.52441849)(33.63274943,680.39941861)(33.57275717,680.28942034)
\curveto(33.51274955,680.17941883)(33.44274962,680.06941894)(33.36275717,679.95942034)
\curveto(33.14274992,679.62941938)(32.89275017,679.36441965)(32.61275717,679.16442034)
\curveto(32.33275073,678.96442005)(31.98275108,678.79442022)(31.56275717,678.65442034)
\curveto(31.45275161,678.6144204)(31.34275172,678.58942042)(31.23275717,678.57942034)
\curveto(31.12275194,678.56942044)(31.00775205,678.54942046)(30.88775717,678.51942034)
\curveto(30.84775221,678.5094205)(30.80275226,678.5094205)(30.75275717,678.51942034)
\curveto(30.71275235,678.51942049)(30.67275239,678.5144205)(30.63275717,678.50442034)
\lineto(30.46775717,678.50442034)
\curveto(30.41775264,678.48442053)(30.3577527,678.47942053)(30.28775717,678.48942034)
\curveto(30.22775283,678.48942052)(30.17275289,678.49442052)(30.12275717,678.50442034)
\curveto(30.04275302,678.5144205)(29.97275309,678.5144205)(29.91275717,678.50442034)
\curveto(29.85275321,678.49442052)(29.78775327,678.49942051)(29.71775717,678.51942034)
\curveto(29.66775339,678.53942047)(29.61275345,678.54942046)(29.55275717,678.54942034)
\curveto(29.49275357,678.54942046)(29.43775362,678.55942045)(29.38775717,678.57942034)
\curveto(29.27775378,678.59942041)(29.16775389,678.62442039)(29.05775717,678.65442034)
\curveto(28.94775411,678.67442034)(28.84775421,678.7094203)(28.75775717,678.75942034)
\curveto(28.64775441,678.79942021)(28.54275452,678.83442018)(28.44275717,678.86442034)
\curveto(28.35275471,678.90442011)(28.26775479,678.94942006)(28.18775717,678.99942034)
\curveto(27.86775519,679.19941981)(27.58275548,679.42941958)(27.33275717,679.68942034)
\curveto(27.08275598,679.95941905)(26.87775618,680.26941874)(26.71775717,680.61942034)
\curveto(26.66775639,680.72941828)(26.62775643,680.83941817)(26.59775717,680.94942034)
\curveto(26.56775649,681.06941794)(26.52775653,681.18941782)(26.47775717,681.30942034)
\curveto(26.46775659,681.34941766)(26.4627566,681.38441763)(26.46275717,681.41442034)
\curveto(26.4627566,681.45441756)(26.4577566,681.49441752)(26.44775717,681.53442034)
\curveto(26.40775665,681.65441736)(26.38275668,681.78441723)(26.37275717,681.92442034)
\lineto(26.34275717,682.34442034)
\curveto(26.34275672,682.39441662)(26.33775672,682.44941656)(26.32775717,682.50942034)
\curveto(26.32775673,682.56941644)(26.33275673,682.62441639)(26.34275717,682.67442034)
\lineto(26.34275717,682.85442034)
\lineto(26.38775717,683.21442034)
\curveto(26.42775663,683.38441563)(26.4627566,683.54941546)(26.49275717,683.70942034)
\curveto(26.52275654,683.86941514)(26.56775649,684.01941499)(26.62775717,684.15942034)
\curveto(27.057756,685.19941381)(27.78775527,685.93441308)(28.81775717,686.36442034)
\curveto(28.9577541,686.42441259)(29.09775396,686.46441255)(29.23775717,686.48442034)
\curveto(29.38775367,686.5144125)(29.54275352,686.54941246)(29.70275717,686.58942034)
\curveto(29.78275328,686.59941241)(29.8577532,686.60441241)(29.92775717,686.60442034)
\curveto(29.99775306,686.60441241)(30.07275299,686.6094124)(30.15275717,686.61942034)
\curveto(30.6627524,686.62941238)(31.09775196,686.56941244)(31.45775717,686.43942034)
\curveto(31.82775123,686.31941269)(32.1577509,686.15941285)(32.44775717,685.95942034)
\curveto(32.53775052,685.89941311)(32.62775043,685.82941318)(32.71775717,685.74942034)
\curveto(32.80775025,685.67941333)(32.88775017,685.60441341)(32.95775717,685.52442034)
\curveto(32.98775007,685.47441354)(33.02775003,685.43441358)(33.07775717,685.40442034)
\curveto(33.1577499,685.29441372)(33.23274983,685.17941383)(33.30275717,685.05942034)
\curveto(33.37274969,684.94941406)(33.44774961,684.83441418)(33.52775717,684.71442034)
\curveto(33.57774948,684.62441439)(33.61774944,684.52941448)(33.64775717,684.42942034)
\curveto(33.68774937,684.33941467)(33.72774933,684.23941477)(33.76775717,684.12942034)
\curveto(33.81774924,683.99941501)(33.8577492,683.86441515)(33.88775717,683.72442034)
\curveto(33.91774914,683.58441543)(33.95274911,683.44441557)(33.99275717,683.30442034)
\curveto(34.01274905,683.22441579)(34.01774904,683.13441588)(34.00775717,683.03442034)
\curveto(34.00774905,682.94441607)(34.01774904,682.85941615)(34.03775717,682.77942034)
\lineto(34.03775717,682.61442034)
\moveto(31.78775717,683.49942034)
\curveto(31.8577512,683.59941541)(31.8627512,683.71941529)(31.80275717,683.85942034)
\curveto(31.75275131,684.009415)(31.71275135,684.11941489)(31.68275717,684.18942034)
\curveto(31.54275152,684.45941455)(31.3577517,684.66441435)(31.12775717,684.80442034)
\curveto(30.89775216,684.95441406)(30.57775248,685.03441398)(30.16775717,685.04442034)
\curveto(30.13775292,685.02441399)(30.10275296,685.01941399)(30.06275717,685.02942034)
\curveto(30.02275304,685.03941397)(29.98775307,685.03941397)(29.95775717,685.02942034)
\curveto(29.90775315,685.009414)(29.85275321,684.99441402)(29.79275717,684.98442034)
\curveto(29.73275333,684.98441403)(29.67775338,684.97441404)(29.62775717,684.95442034)
\curveto(29.18775387,684.8144142)(28.8627542,684.53941447)(28.65275717,684.12942034)
\curveto(28.63275443,684.08941492)(28.60775445,684.03441498)(28.57775717,683.96442034)
\curveto(28.5577545,683.90441511)(28.54275452,683.83941517)(28.53275717,683.76942034)
\curveto(28.52275454,683.7094153)(28.52275454,683.64941536)(28.53275717,683.58942034)
\curveto(28.55275451,683.52941548)(28.58775447,683.47941553)(28.63775717,683.43942034)
\curveto(28.71775434,683.38941562)(28.82775423,683.36441565)(28.96775717,683.36442034)
\lineto(29.37275717,683.36442034)
\lineto(31.03775717,683.36442034)
\lineto(31.47275717,683.36442034)
\curveto(31.63275143,683.37441564)(31.73775132,683.41941559)(31.78775717,683.49942034)
}
}
{
\newrgbcolor{curcolor}{0 0 0}
\pscustom[linestyle=none,fillstyle=solid,fillcolor=curcolor]
{
\newpath
\moveto(42.63603842,686.31942034)
\curveto(42.70603022,686.26941274)(42.74103018,686.18441283)(42.74103842,686.06442034)
\curveto(42.75103017,685.95441306)(42.75603017,685.83941317)(42.75603842,685.71942034)
\lineto(42.75603842,679.31442034)
\curveto(42.75603017,679.23441978)(42.75103017,679.15441986)(42.74103842,679.07442034)
\lineto(42.74103842,678.84942034)
\curveto(42.73103019,678.76942024)(42.7210302,678.69942031)(42.71103842,678.63942034)
\curveto(42.71103021,678.56942044)(42.70603022,678.49442052)(42.69603842,678.41442034)
\curveto(42.65603027,678.27442074)(42.6210303,678.14442087)(42.59103842,678.02442034)
\curveto(42.57103035,677.89442112)(42.53603039,677.77442124)(42.48603842,677.66442034)
\curveto(42.31603061,677.28442173)(42.09603083,676.96942204)(41.82603842,676.71942034)
\curveto(41.56603136,676.46942254)(41.24603168,676.26442275)(40.86603842,676.10442034)
\curveto(40.75603217,676.05442296)(40.64603228,676.014423)(40.53603842,675.98442034)
\curveto(40.4260325,675.95442306)(40.31103261,675.92442309)(40.19103842,675.89442034)
\curveto(40.08103284,675.86442315)(39.97103295,675.84442317)(39.86103842,675.83442034)
\curveto(39.75103317,675.82442319)(39.64103328,675.8094232)(39.53103842,675.78942034)
\lineto(39.41103842,675.78942034)
\curveto(39.37103355,675.77942323)(39.3260336,675.77442324)(39.27603842,675.77442034)
\curveto(39.23603369,675.76442325)(39.19103373,675.76442325)(39.14103842,675.77442034)
\curveto(39.09103383,675.77442324)(39.04103388,675.76942324)(38.99103842,675.75942034)
\curveto(38.94103398,675.74942326)(38.87603405,675.74442327)(38.79603842,675.74442034)
\curveto(38.71603421,675.74442327)(38.65103427,675.74942326)(38.60103842,675.75942034)
\lineto(38.46603842,675.75942034)
\curveto(38.4260345,675.75942325)(38.38603454,675.76442325)(38.34603842,675.77442034)
\curveto(38.26603466,675.79442322)(38.18103474,675.80442321)(38.09103842,675.80442034)
\curveto(38.01103491,675.80442321)(37.93603499,675.8144232)(37.86603842,675.83442034)
\curveto(37.84603508,675.84442317)(37.8210351,675.84942316)(37.79103842,675.84942034)
\curveto(37.76103516,675.84942316)(37.73603519,675.85442316)(37.71603842,675.86442034)
\curveto(37.61603531,675.88442313)(37.51603541,675.9094231)(37.41603842,675.93942034)
\curveto(37.3260356,675.95942305)(37.23603569,675.98942302)(37.14603842,676.02942034)
\curveto(36.76603616,676.18942282)(36.4260365,676.39442262)(36.12603842,676.64442034)
\curveto(35.8260371,676.88442213)(35.60603732,677.2094218)(35.46603842,677.61942034)
\curveto(35.44603748,677.64942136)(35.43603749,677.67942133)(35.43603842,677.70942034)
\curveto(35.43603749,677.73942127)(35.43103749,677.76442125)(35.42103842,677.78442034)
\curveto(35.39103753,677.9144211)(35.40103752,678.014421)(35.45103842,678.08442034)
\curveto(35.51103741,678.14442087)(35.59103733,678.18442083)(35.69103842,678.20442034)
\curveto(35.79103713,678.22442079)(35.90103702,678.23442078)(36.02103842,678.23442034)
\curveto(36.15103677,678.22442079)(36.27103665,678.21942079)(36.38103842,678.21942034)
\lineto(36.89103842,678.21942034)
\lineto(37.01103842,678.21942034)
\curveto(37.05103587,678.2094208)(37.09603583,678.20442081)(37.14603842,678.20442034)
\curveto(37.30603562,678.16442085)(37.40603552,678.1144209)(37.44603842,678.05442034)
\curveto(37.48603544,677.98442103)(37.54603538,677.89442112)(37.62603842,677.78442034)
\curveto(37.65603527,677.74442127)(37.70103522,677.69442132)(37.76103842,677.63442034)
\curveto(37.77103515,677.6144214)(37.78103514,677.59942141)(37.79103842,677.58942034)
\curveto(37.80103512,677.57942143)(37.81103511,677.56442145)(37.82103842,677.54442034)
\curveto(37.90103502,677.48442153)(37.98603494,677.42942158)(38.07603842,677.37942034)
\curveto(38.16603476,677.32942168)(38.26603466,677.28442173)(38.37603842,677.24442034)
\curveto(38.44603448,677.22442179)(38.51603441,677.2144218)(38.58603842,677.21442034)
\curveto(38.65603427,677.20442181)(38.73103419,677.18942182)(38.81103842,677.16942034)
\lineto(38.97603842,677.16942034)
\curveto(39.04603388,677.14942186)(39.13603379,677.14942186)(39.24603842,677.16942034)
\curveto(39.35603357,677.17942183)(39.44103348,677.19442182)(39.50103842,677.21442034)
\curveto(39.55103337,677.23442178)(39.59103333,677.24442177)(39.62103842,677.24442034)
\curveto(39.66103326,677.24442177)(39.70103322,677.25442176)(39.74103842,677.27442034)
\curveto(39.95103297,677.36442165)(40.1260328,677.48442153)(40.26603842,677.63442034)
\curveto(40.40603252,677.78442123)(40.5210324,677.95942105)(40.61103842,678.15942034)
\curveto(40.63103229,678.21942079)(40.64603228,678.27942073)(40.65603842,678.33942034)
\curveto(40.66603226,678.39942061)(40.68103224,678.46442055)(40.70103842,678.53442034)
\curveto(40.7210322,678.62442039)(40.73103219,678.71942029)(40.73103842,678.81942034)
\curveto(40.74103218,678.92942008)(40.74603218,679.03941997)(40.74603842,679.14942034)
\lineto(40.74603842,679.26942034)
\curveto(40.75603217,679.3094197)(40.75603217,679.34441967)(40.74603842,679.37442034)
\curveto(40.7260322,679.42441959)(40.71603221,679.46941954)(40.71603842,679.50942034)
\curveto(40.7260322,679.54941946)(40.7210322,679.58941942)(40.70103842,679.62942034)
\curveto(40.69103223,679.64941936)(40.67603225,679.66441935)(40.65603842,679.67442034)
\lineto(40.61103842,679.71942034)
\curveto(40.5210324,679.72941928)(40.44603248,679.7094193)(40.38603842,679.65942034)
\curveto(40.33603259,679.6094194)(40.28603264,679.56441945)(40.23603842,679.52442034)
\curveto(40.14603278,679.45441956)(40.05603287,679.38941962)(39.96603842,679.32942034)
\curveto(39.87603305,679.26941974)(39.77603315,679.2144198)(39.66603842,679.16442034)
\curveto(39.55603337,679.1144199)(39.44603348,679.07441994)(39.33603842,679.04442034)
\curveto(39.2260337,679.01442)(39.11103381,678.98442003)(38.99103842,678.95442034)
\lineto(38.81103842,678.92442034)
\curveto(38.76103416,678.92442009)(38.71103421,678.91942009)(38.66103842,678.90942034)
\curveto(38.61103431,678.89942011)(38.53103439,678.89442012)(38.42103842,678.89442034)
\curveto(38.31103461,678.89442012)(38.23103469,678.89942011)(38.18103842,678.90942034)
\lineto(38.06103842,678.90942034)
\curveto(38.03103489,678.91942009)(37.99603493,678.92442009)(37.95603842,678.92442034)
\curveto(37.926035,678.92442009)(37.89103503,678.92942008)(37.85103842,678.93942034)
\curveto(37.71103521,678.96942004)(37.57603535,678.99442002)(37.44603842,679.01442034)
\curveto(37.31603561,679.04441997)(37.19603573,679.08441993)(37.08603842,679.13442034)
\curveto(36.65603627,679.30441971)(36.30603662,679.53941947)(36.03603842,679.83942034)
\curveto(35.77603715,680.14941886)(35.55603737,680.51941849)(35.37603842,680.94942034)
\curveto(35.3260376,681.05941795)(35.29103763,681.17441784)(35.27103842,681.29442034)
\curveto(35.25103767,681.4144176)(35.2210377,681.53441748)(35.18103842,681.65442034)
\curveto(35.18103774,681.70441731)(35.17603775,681.74441727)(35.16603842,681.77442034)
\curveto(35.14603778,681.85441716)(35.13603779,681.93941707)(35.13603842,682.02942034)
\curveto(35.13603779,682.12941688)(35.1260378,682.21941679)(35.10603842,682.29942034)
\curveto(35.09603783,682.34941666)(35.09103783,682.39441662)(35.09103842,682.43442034)
\lineto(35.09103842,682.58442034)
\curveto(35.08103784,682.63441638)(35.07603785,682.69441632)(35.07603842,682.76442034)
\curveto(35.07603785,682.84441617)(35.08103784,682.9094161)(35.09103842,682.95942034)
\lineto(35.09103842,683.10942034)
\curveto(35.10103782,683.14941586)(35.10103782,683.18941582)(35.09103842,683.22942034)
\curveto(35.09103783,683.26941574)(35.10103782,683.3094157)(35.12103842,683.34942034)
\curveto(35.14103778,683.44941556)(35.15603777,683.54441547)(35.16603842,683.63442034)
\curveto(35.17603775,683.73441528)(35.19103773,683.83441518)(35.21103842,683.93442034)
\curveto(35.27103765,684.13441488)(35.33103759,684.32441469)(35.39103842,684.50442034)
\curveto(35.46103746,684.68441433)(35.54603738,684.85441416)(35.64603842,685.01442034)
\curveto(35.69603723,685.1144139)(35.75103717,685.20441381)(35.81103842,685.28442034)
\lineto(36.02103842,685.55442034)
\curveto(36.05103687,685.60441341)(36.09103683,685.65441336)(36.14103842,685.70442034)
\curveto(36.20103672,685.75441326)(36.25603667,685.79941321)(36.30603842,685.83942034)
\lineto(36.39603842,685.92942034)
\curveto(36.44603648,685.96941304)(36.49603643,686.00441301)(36.54603842,686.03442034)
\curveto(36.59603633,686.07441294)(36.64603628,686.1094129)(36.69603842,686.13942034)
\curveto(36.8260361,686.21941279)(36.96103596,686.28941272)(37.10103842,686.34942034)
\curveto(37.24103568,686.4094126)(37.39603553,686.46441255)(37.56603842,686.51442034)
\curveto(37.64603528,686.54441247)(37.7260352,686.55941245)(37.80603842,686.55942034)
\curveto(37.89603503,686.56941244)(37.98103494,686.58441243)(38.06103842,686.60442034)
\curveto(38.10103482,686.6144124)(38.15603477,686.6144124)(38.22603842,686.60442034)
\curveto(38.29603463,686.59441242)(38.34103458,686.59941241)(38.36103842,686.61942034)
\curveto(38.68103424,686.62941238)(38.96603396,686.59941241)(39.21603842,686.52942034)
\curveto(39.47603345,686.45941255)(39.70603322,686.35941265)(39.90603842,686.22942034)
\curveto(39.93603299,686.2094128)(39.96603296,686.18441283)(39.99603842,686.15442034)
\curveto(40.0260329,686.13441288)(40.06103286,686.1094129)(40.10103842,686.07942034)
\curveto(40.16103276,686.02941298)(40.21603271,685.97941303)(40.26603842,685.92942034)
\curveto(40.31603261,685.87941313)(40.37603255,685.83441318)(40.44603842,685.79442034)
\curveto(40.46603246,685.78441323)(40.49103243,685.77441324)(40.52103842,685.76442034)
\curveto(40.56103236,685.75441326)(40.59103233,685.75941325)(40.61103842,685.77942034)
\curveto(40.66103226,685.79941321)(40.69103223,685.83441318)(40.70103842,685.88442034)
\curveto(40.71103221,685.93441308)(40.7260322,685.98441303)(40.74603842,686.03442034)
\curveto(40.76603216,686.08441293)(40.78103214,686.13441288)(40.79103842,686.18442034)
\curveto(40.81103211,686.24441277)(40.84103208,686.29441272)(40.88103842,686.33442034)
\curveto(40.94103198,686.37441264)(41.01103191,686.39441262)(41.09103842,686.39442034)
\curveto(41.18103174,686.40441261)(41.27103165,686.4094126)(41.36103842,686.40942034)
\lineto(42.12603842,686.40942034)
\curveto(42.23603069,686.4094126)(42.33103059,686.40441261)(42.41103842,686.39442034)
\curveto(42.50103042,686.39441262)(42.57603035,686.36941264)(42.63603842,686.31942034)
\moveto(40.58103842,681.68442034)
\curveto(40.6210323,681.77441724)(40.65603227,681.88941712)(40.68603842,682.02942034)
\curveto(40.71603221,682.16941684)(40.73603219,682.3144167)(40.74603842,682.46442034)
\curveto(40.75603217,682.62441639)(40.75603217,682.77941623)(40.74603842,682.92942034)
\curveto(40.74603218,683.07941593)(40.73103219,683.2144158)(40.70103842,683.33442034)
\curveto(40.68103224,683.37441564)(40.67103225,683.40441561)(40.67103842,683.42442034)
\curveto(40.68103224,683.45441556)(40.68103224,683.48941552)(40.67103842,683.52942034)
\lineto(40.61103842,683.73942034)
\curveto(40.59103233,683.8094152)(40.56603236,683.87441514)(40.53603842,683.93442034)
\curveto(40.39603253,684.28441473)(40.19603273,684.55441446)(39.93603842,684.74442034)
\curveto(39.67603325,684.93441408)(39.29603363,685.02941398)(38.79603842,685.02942034)
\curveto(38.77603415,685.009414)(38.74603418,684.99941401)(38.70603842,684.99942034)
\curveto(38.67603425,685.009414)(38.64603428,685.009414)(38.61603842,684.99942034)
\curveto(38.54603438,684.97941403)(38.48103444,684.95941405)(38.42103842,684.93942034)
\curveto(38.36103456,684.92941408)(38.30103462,684.9144141)(38.24103842,684.89442034)
\curveto(37.98103494,684.78441423)(37.78103514,684.61941439)(37.64103842,684.39942034)
\curveto(37.50103542,684.17941483)(37.38603554,683.93441508)(37.29603842,683.66442034)
\curveto(37.27603565,683.6144154)(37.26603566,683.57441544)(37.26603842,683.54442034)
\curveto(37.26603566,683.5144155)(37.26103566,683.47441554)(37.25103842,683.42442034)
\curveto(37.2210357,683.3144157)(37.20103572,683.15441586)(37.19103842,682.94442034)
\curveto(37.18103574,682.73441628)(37.19103573,682.56441645)(37.22103842,682.43442034)
\lineto(37.22103842,682.28442034)
\curveto(37.24103568,682.20441681)(37.25603567,682.12441689)(37.26603842,682.04442034)
\curveto(37.27603565,681.97441704)(37.29103563,681.89941711)(37.31103842,681.81942034)
\curveto(37.40103552,681.55941745)(37.51103541,681.32941768)(37.64103842,681.12942034)
\curveto(37.77103515,680.93941807)(37.95103497,680.78441823)(38.18103842,680.66442034)
\curveto(38.28103464,680.6144184)(38.4210345,680.56441845)(38.60103842,680.51442034)
\curveto(38.67103425,680.5144185)(38.7260342,680.5094185)(38.76603842,680.49942034)
\curveto(38.78603414,680.49941851)(38.81603411,680.49441852)(38.85603842,680.48442034)
\curveto(38.89603403,680.48441853)(38.926034,680.48941852)(38.94603842,680.49942034)
\lineto(39.09603842,680.49942034)
\curveto(39.18603374,680.51941849)(39.27103365,680.53441848)(39.35103842,680.54442034)
\curveto(39.43103349,680.55441846)(39.51103341,680.57941843)(39.59103842,680.61942034)
\curveto(39.84103308,680.71941829)(40.04103288,680.85941815)(40.19103842,681.03942034)
\curveto(40.35103257,681.21941779)(40.48103244,681.43441758)(40.58103842,681.68442034)
}
}
{
\newrgbcolor{curcolor}{0 0 0}
\pscustom[linestyle=none,fillstyle=solid,fillcolor=curcolor]
{
\newpath
\moveto(44.87596029,686.39442034)
\lineto(46.00096029,686.39442034)
\curveto(46.11095786,686.39441262)(46.21095776,686.38941262)(46.30096029,686.37942034)
\curveto(46.39095758,686.36941264)(46.45595751,686.33441268)(46.49596029,686.27442034)
\curveto(46.54595742,686.2144128)(46.57595739,686.12941288)(46.58596029,686.01942034)
\curveto(46.59595737,685.91941309)(46.60095737,685.8144132)(46.60096029,685.70442034)
\lineto(46.60096029,684.65442034)
\lineto(46.60096029,682.41942034)
\curveto(46.60095737,682.05941695)(46.61595735,681.71941729)(46.64596029,681.39942034)
\curveto(46.67595729,681.07941793)(46.7659572,680.8144182)(46.91596029,680.60442034)
\curveto(47.05595691,680.39441862)(47.28095669,680.24441877)(47.59096029,680.15442034)
\curveto(47.64095633,680.14441887)(47.68095629,680.13941887)(47.71096029,680.13942034)
\curveto(47.75095622,680.13941887)(47.79595617,680.13441888)(47.84596029,680.12442034)
\curveto(47.89595607,680.1144189)(47.95095602,680.1094189)(48.01096029,680.10942034)
\curveto(48.0709559,680.1094189)(48.11595585,680.1144189)(48.14596029,680.12442034)
\curveto(48.19595577,680.14441887)(48.23595573,680.14941886)(48.26596029,680.13942034)
\curveto(48.30595566,680.12941888)(48.34595562,680.13441888)(48.38596029,680.15442034)
\curveto(48.59595537,680.20441881)(48.76095521,680.26941874)(48.88096029,680.34942034)
\curveto(49.06095491,680.45941855)(49.20095477,680.59941841)(49.30096029,680.76942034)
\curveto(49.41095456,680.94941806)(49.48595448,681.14441787)(49.52596029,681.35442034)
\curveto(49.57595439,681.57441744)(49.60595436,681.8144172)(49.61596029,682.07442034)
\curveto(49.62595434,682.34441667)(49.63095434,682.62441639)(49.63096029,682.91442034)
\lineto(49.63096029,684.72942034)
\lineto(49.63096029,685.70442034)
\lineto(49.63096029,685.97442034)
\curveto(49.63095434,686.07441294)(49.65095432,686.15441286)(49.69096029,686.21442034)
\curveto(49.74095423,686.30441271)(49.81595415,686.35441266)(49.91596029,686.36442034)
\curveto(50.01595395,686.38441263)(50.13595383,686.39441262)(50.27596029,686.39442034)
\lineto(51.07096029,686.39442034)
\lineto(51.35596029,686.39442034)
\curveto(51.44595252,686.39441262)(51.52095245,686.37441264)(51.58096029,686.33442034)
\curveto(51.66095231,686.28441273)(51.70595226,686.2094128)(51.71596029,686.10942034)
\curveto(51.72595224,686.009413)(51.73095224,685.89441312)(51.73096029,685.76442034)
\lineto(51.73096029,684.62442034)
\lineto(51.73096029,680.40942034)
\lineto(51.73096029,679.34442034)
\lineto(51.73096029,679.04442034)
\curveto(51.73095224,678.94442007)(51.71095226,678.86942014)(51.67096029,678.81942034)
\curveto(51.62095235,678.73942027)(51.54595242,678.69442032)(51.44596029,678.68442034)
\curveto(51.34595262,678.67442034)(51.24095273,678.66942034)(51.13096029,678.66942034)
\lineto(50.32096029,678.66942034)
\curveto(50.21095376,678.66942034)(50.11095386,678.67442034)(50.02096029,678.68442034)
\curveto(49.94095403,678.69442032)(49.87595409,678.73442028)(49.82596029,678.80442034)
\curveto(49.80595416,678.83442018)(49.78595418,678.87942013)(49.76596029,678.93942034)
\curveto(49.75595421,678.99942001)(49.74095423,679.05941995)(49.72096029,679.11942034)
\curveto(49.71095426,679.17941983)(49.69595427,679.23441978)(49.67596029,679.28442034)
\curveto(49.65595431,679.33441968)(49.62595434,679.36441965)(49.58596029,679.37442034)
\curveto(49.5659544,679.39441962)(49.54095443,679.39941961)(49.51096029,679.38942034)
\curveto(49.48095449,679.37941963)(49.45595451,679.36941964)(49.43596029,679.35942034)
\curveto(49.3659546,679.31941969)(49.30595466,679.27441974)(49.25596029,679.22442034)
\curveto(49.20595476,679.17441984)(49.15095482,679.12941988)(49.09096029,679.08942034)
\curveto(49.05095492,679.05941995)(49.01095496,679.02441999)(48.97096029,678.98442034)
\curveto(48.94095503,678.95442006)(48.90095507,678.92442009)(48.85096029,678.89442034)
\curveto(48.62095535,678.75442026)(48.35095562,678.64442037)(48.04096029,678.56442034)
\curveto(47.970956,678.54442047)(47.90095607,678.53442048)(47.83096029,678.53442034)
\curveto(47.76095621,678.52442049)(47.68595628,678.5094205)(47.60596029,678.48942034)
\curveto(47.5659564,678.47942053)(47.52095645,678.47942053)(47.47096029,678.48942034)
\curveto(47.43095654,678.48942052)(47.39095658,678.48442053)(47.35096029,678.47442034)
\curveto(47.32095665,678.46442055)(47.25595671,678.46442055)(47.15596029,678.47442034)
\curveto(47.0659569,678.47442054)(47.00595696,678.47942053)(46.97596029,678.48942034)
\curveto(46.92595704,678.48942052)(46.87595709,678.49442052)(46.82596029,678.50442034)
\lineto(46.67596029,678.50442034)
\curveto(46.55595741,678.53442048)(46.44095753,678.55942045)(46.33096029,678.57942034)
\curveto(46.22095775,678.59942041)(46.11095786,678.62942038)(46.00096029,678.66942034)
\curveto(45.95095802,678.68942032)(45.90595806,678.70442031)(45.86596029,678.71442034)
\curveto(45.83595813,678.73442028)(45.79595817,678.75442026)(45.74596029,678.77442034)
\curveto(45.39595857,678.96442005)(45.11595885,679.22941978)(44.90596029,679.56942034)
\curveto(44.77595919,679.77941923)(44.68095929,680.02941898)(44.62096029,680.31942034)
\curveto(44.56095941,680.61941839)(44.52095945,680.93441808)(44.50096029,681.26442034)
\curveto(44.49095948,681.60441741)(44.48595948,681.94941706)(44.48596029,682.29942034)
\curveto(44.49595947,682.65941635)(44.50095947,683.014416)(44.50096029,683.36442034)
\lineto(44.50096029,685.40442034)
\curveto(44.50095947,685.53441348)(44.49595947,685.68441333)(44.48596029,685.85442034)
\curveto(44.48595948,686.03441298)(44.51095946,686.16441285)(44.56096029,686.24442034)
\curveto(44.59095938,686.29441272)(44.65095932,686.33941267)(44.74096029,686.37942034)
\curveto(44.80095917,686.37941263)(44.84595912,686.38441263)(44.87596029,686.39442034)
\moveto(48.74596029,689.48442034)
\lineto(49.81096029,689.48442034)
\curveto(49.89095408,689.48440953)(49.98595398,689.48440953)(50.09596029,689.48442034)
\curveto(50.20595376,689.48440953)(50.28595368,689.46940954)(50.33596029,689.43942034)
\curveto(50.35595361,689.42940958)(50.3659536,689.4144096)(50.36596029,689.39442034)
\curveto(50.37595359,689.38440963)(50.39095358,689.37440964)(50.41096029,689.36442034)
\curveto(50.42095355,689.24440977)(50.3709536,689.13940987)(50.26096029,689.04942034)
\curveto(50.16095381,688.95941005)(50.07595389,688.87941013)(50.00596029,688.80942034)
\curveto(49.92595404,688.73941027)(49.84595412,688.66441035)(49.76596029,688.58442034)
\curveto(49.69595427,688.5144105)(49.62095435,688.44941056)(49.54096029,688.38942034)
\curveto(49.50095447,688.35941065)(49.4659545,688.32441069)(49.43596029,688.28442034)
\curveto(49.41595455,688.25441076)(49.38595458,688.22941078)(49.34596029,688.20942034)
\curveto(49.32595464,688.17941083)(49.30095467,688.15441086)(49.27096029,688.13442034)
\lineto(49.12096029,687.98442034)
\lineto(48.97096029,687.86442034)
\lineto(48.92596029,687.81942034)
\curveto(48.92595504,687.8094112)(48.91595505,687.79441122)(48.89596029,687.77442034)
\curveto(48.81595515,687.7144113)(48.73595523,687.64941136)(48.65596029,687.57942034)
\curveto(48.58595538,687.5094115)(48.49595547,687.45441156)(48.38596029,687.41442034)
\curveto(48.34595562,687.40441161)(48.30595566,687.39941161)(48.26596029,687.39942034)
\curveto(48.23595573,687.39941161)(48.19595577,687.39441162)(48.14596029,687.38442034)
\curveto(48.11595585,687.37441164)(48.07595589,687.36941164)(48.02596029,687.36942034)
\curveto(47.97595599,687.37941163)(47.93095604,687.38441163)(47.89096029,687.38442034)
\lineto(47.54596029,687.38442034)
\curveto(47.42595654,687.38441163)(47.33595663,687.4094116)(47.27596029,687.45942034)
\curveto(47.21595675,687.49941151)(47.20095677,687.56941144)(47.23096029,687.66942034)
\curveto(47.25095672,687.74941126)(47.28595668,687.81941119)(47.33596029,687.87942034)
\curveto(47.38595658,687.94941106)(47.43095654,688.01941099)(47.47096029,688.08942034)
\curveto(47.5709564,688.22941078)(47.6659563,688.36441065)(47.75596029,688.49442034)
\curveto(47.84595612,688.62441039)(47.93595603,688.75941025)(48.02596029,688.89942034)
\curveto(48.07595589,688.97941003)(48.12595584,689.06440995)(48.17596029,689.15442034)
\curveto(48.23595573,689.24440977)(48.30095567,689.3144097)(48.37096029,689.36442034)
\curveto(48.41095556,689.39440962)(48.48095549,689.42940958)(48.58096029,689.46942034)
\curveto(48.60095537,689.47940953)(48.62595534,689.47940953)(48.65596029,689.46942034)
\curveto(48.69595527,689.46940954)(48.72595524,689.47440954)(48.74596029,689.48442034)
}
}
{
\newrgbcolor{curcolor}{0 0 0}
\pscustom[linestyle=none,fillstyle=solid,fillcolor=curcolor]
{
\newpath
\moveto(57.78721029,686.60442034)
\curveto(58.38720449,686.62441239)(58.88720399,686.53941247)(59.28721029,686.34942034)
\curveto(59.68720319,686.15941285)(60.00220287,685.87941313)(60.23221029,685.50942034)
\curveto(60.30220257,685.39941361)(60.35720252,685.27941373)(60.39721029,685.14942034)
\curveto(60.43720244,685.02941398)(60.4772024,684.90441411)(60.51721029,684.77442034)
\curveto(60.53720234,684.69441432)(60.54720233,684.61941439)(60.54721029,684.54942034)
\curveto(60.55720232,684.47941453)(60.5722023,684.4094146)(60.59221029,684.33942034)
\curveto(60.59220228,684.27941473)(60.59720228,684.23941477)(60.60721029,684.21942034)
\curveto(60.62720225,684.07941493)(60.63720224,683.93441508)(60.63721029,683.78442034)
\lineto(60.63721029,683.34942034)
\lineto(60.63721029,682.01442034)
\lineto(60.63721029,679.58442034)
\curveto(60.63720224,679.39441962)(60.63220224,679.2094198)(60.62221029,679.02942034)
\curveto(60.62220225,678.85942015)(60.55220232,678.74942026)(60.41221029,678.69942034)
\curveto(60.35220252,678.67942033)(60.28220259,678.66942034)(60.20221029,678.66942034)
\lineto(59.96221029,678.66942034)
\lineto(59.15221029,678.66942034)
\curveto(59.03220384,678.66942034)(58.92220395,678.67442034)(58.82221029,678.68442034)
\curveto(58.73220414,678.70442031)(58.66220421,678.74942026)(58.61221029,678.81942034)
\curveto(58.5722043,678.87942013)(58.54720433,678.95442006)(58.53721029,679.04442034)
\lineto(58.53721029,679.35942034)
\lineto(58.53721029,680.40942034)
\lineto(58.53721029,682.64442034)
\curveto(58.53720434,683.014416)(58.52220435,683.35441566)(58.49221029,683.66442034)
\curveto(58.46220441,683.98441503)(58.3722045,684.25441476)(58.22221029,684.47442034)
\curveto(58.08220479,684.67441434)(57.877205,684.8144142)(57.60721029,684.89442034)
\curveto(57.55720532,684.9144141)(57.50220537,684.92441409)(57.44221029,684.92442034)
\curveto(57.39220548,684.92441409)(57.33720554,684.93441408)(57.27721029,684.95442034)
\curveto(57.22720565,684.96441405)(57.16220571,684.96441405)(57.08221029,684.95442034)
\curveto(57.01220586,684.95441406)(56.95720592,684.94941406)(56.91721029,684.93942034)
\curveto(56.877206,684.92941408)(56.84220603,684.92441409)(56.81221029,684.92442034)
\curveto(56.78220609,684.92441409)(56.75220612,684.91941409)(56.72221029,684.90942034)
\curveto(56.49220638,684.84941416)(56.30720657,684.76941424)(56.16721029,684.66942034)
\curveto(55.84720703,684.43941457)(55.65720722,684.10441491)(55.59721029,683.66442034)
\curveto(55.53720734,683.22441579)(55.50720737,682.72941628)(55.50721029,682.17942034)
\lineto(55.50721029,680.30442034)
\lineto(55.50721029,679.38942034)
\lineto(55.50721029,679.11942034)
\curveto(55.50720737,679.02941998)(55.49220738,678.95442006)(55.46221029,678.89442034)
\curveto(55.41220746,678.78442023)(55.33220754,678.71942029)(55.22221029,678.69942034)
\curveto(55.11220776,678.67942033)(54.9772079,678.66942034)(54.81721029,678.66942034)
\lineto(54.06721029,678.66942034)
\curveto(53.95720892,678.66942034)(53.84720903,678.67442034)(53.73721029,678.68442034)
\curveto(53.62720925,678.69442032)(53.54720933,678.72942028)(53.49721029,678.78942034)
\curveto(53.42720945,678.87942013)(53.39220948,679.00942)(53.39221029,679.17942034)
\curveto(53.40220947,679.34941966)(53.40720947,679.5094195)(53.40721029,679.65942034)
\lineto(53.40721029,681.69942034)
\lineto(53.40721029,684.99942034)
\lineto(53.40721029,685.76442034)
\lineto(53.40721029,686.06442034)
\curveto(53.41720946,686.15441286)(53.44720943,686.22941278)(53.49721029,686.28942034)
\curveto(53.51720936,686.31941269)(53.54720933,686.33941267)(53.58721029,686.34942034)
\curveto(53.63720924,686.36941264)(53.68720919,686.38441263)(53.73721029,686.39442034)
\lineto(53.81221029,686.39442034)
\curveto(53.86220901,686.40441261)(53.91220896,686.4094126)(53.96221029,686.40942034)
\lineto(54.12721029,686.40942034)
\lineto(54.75721029,686.40942034)
\curveto(54.83720804,686.4094126)(54.91220796,686.40441261)(54.98221029,686.39442034)
\curveto(55.06220781,686.39441262)(55.13220774,686.38441263)(55.19221029,686.36442034)
\curveto(55.26220761,686.33441268)(55.30720757,686.28941272)(55.32721029,686.22942034)
\curveto(55.35720752,686.16941284)(55.38220749,686.09941291)(55.40221029,686.01942034)
\curveto(55.41220746,685.97941303)(55.41220746,685.94441307)(55.40221029,685.91442034)
\curveto(55.40220747,685.88441313)(55.41220746,685.85441316)(55.43221029,685.82442034)
\curveto(55.45220742,685.77441324)(55.46720741,685.74441327)(55.47721029,685.73442034)
\curveto(55.49720738,685.72441329)(55.52220735,685.7094133)(55.55221029,685.68942034)
\curveto(55.66220721,685.67941333)(55.75220712,685.7144133)(55.82221029,685.79442034)
\curveto(55.89220698,685.88441313)(55.96720691,685.95441306)(56.04721029,686.00442034)
\curveto(56.31720656,686.20441281)(56.61720626,686.36441265)(56.94721029,686.48442034)
\curveto(57.03720584,686.5144125)(57.12720575,686.53441248)(57.21721029,686.54442034)
\curveto(57.31720556,686.55441246)(57.42220545,686.56941244)(57.53221029,686.58942034)
\curveto(57.56220531,686.59941241)(57.60720527,686.59941241)(57.66721029,686.58942034)
\curveto(57.72720515,686.58941242)(57.76720511,686.59441242)(57.78721029,686.60442034)
}
}
{
\newrgbcolor{curcolor}{0 0 0}
\pscustom[linestyle=none,fillstyle=solid,fillcolor=curcolor]
{
}
}
{
\newrgbcolor{curcolor}{0 0 0}
\pscustom[linestyle=none,fillstyle=solid,fillcolor=curcolor]
{
\newpath
\moveto(69.39861654,686.61942034)
\curveto(70.14861204,686.63941237)(70.79861139,686.55441246)(71.34861654,686.36442034)
\curveto(71.90861028,686.18441283)(72.33360986,685.86941314)(72.62361654,685.41942034)
\curveto(72.6936095,685.3094137)(72.75360944,685.19441382)(72.80361654,685.07442034)
\curveto(72.86360933,684.96441405)(72.91360928,684.83941417)(72.95361654,684.69942034)
\curveto(72.97360922,684.63941437)(72.98360921,684.57441444)(72.98361654,684.50442034)
\curveto(72.98360921,684.43441458)(72.97360922,684.37441464)(72.95361654,684.32442034)
\curveto(72.91360928,684.26441475)(72.85860933,684.22441479)(72.78861654,684.20442034)
\curveto(72.73860945,684.18441483)(72.67860951,684.17441484)(72.60861654,684.17442034)
\lineto(72.39861654,684.17442034)
\lineto(71.73861654,684.17442034)
\curveto(71.66861052,684.17441484)(71.59861059,684.16941484)(71.52861654,684.15942034)
\curveto(71.45861073,684.15941485)(71.3936108,684.16941484)(71.33361654,684.18942034)
\curveto(71.23361096,684.2094148)(71.15861103,684.24941476)(71.10861654,684.30942034)
\curveto(71.05861113,684.36941464)(71.01361118,684.42941458)(70.97361654,684.48942034)
\lineto(70.85361654,684.69942034)
\curveto(70.82361137,684.77941423)(70.77361142,684.84441417)(70.70361654,684.89442034)
\curveto(70.60361159,684.97441404)(70.50361169,685.03441398)(70.40361654,685.07442034)
\curveto(70.31361188,685.1144139)(70.19861199,685.14941386)(70.05861654,685.17942034)
\curveto(69.9886122,685.19941381)(69.88361231,685.2144138)(69.74361654,685.22442034)
\curveto(69.61361258,685.23441378)(69.51361268,685.22941378)(69.44361654,685.20942034)
\lineto(69.33861654,685.20942034)
\lineto(69.18861654,685.17942034)
\curveto(69.14861304,685.17941383)(69.10361309,685.17441384)(69.05361654,685.16442034)
\curveto(68.88361331,685.1144139)(68.74361345,685.04441397)(68.63361654,684.95442034)
\curveto(68.53361366,684.87441414)(68.46361373,684.74941426)(68.42361654,684.57942034)
\curveto(68.40361379,684.5094145)(68.40361379,684.44441457)(68.42361654,684.38442034)
\curveto(68.44361375,684.32441469)(68.46361373,684.27441474)(68.48361654,684.23442034)
\curveto(68.55361364,684.1144149)(68.63361356,684.01941499)(68.72361654,683.94942034)
\curveto(68.82361337,683.87941513)(68.93861325,683.81941519)(69.06861654,683.76942034)
\curveto(69.25861293,683.68941532)(69.46361273,683.61941539)(69.68361654,683.55942034)
\lineto(70.37361654,683.40942034)
\curveto(70.61361158,683.36941564)(70.84361135,683.31941569)(71.06361654,683.25942034)
\curveto(71.2936109,683.2094158)(71.50861068,683.14441587)(71.70861654,683.06442034)
\curveto(71.79861039,683.02441599)(71.88361031,682.98941602)(71.96361654,682.95942034)
\curveto(72.05361014,682.93941607)(72.13861005,682.90441611)(72.21861654,682.85442034)
\curveto(72.40860978,682.73441628)(72.57860961,682.60441641)(72.72861654,682.46442034)
\curveto(72.8886093,682.32441669)(73.01360918,682.14941686)(73.10361654,681.93942034)
\curveto(73.13360906,681.86941714)(73.15860903,681.79941721)(73.17861654,681.72942034)
\curveto(73.19860899,681.65941735)(73.21860897,681.58441743)(73.23861654,681.50442034)
\curveto(73.24860894,681.44441757)(73.25360894,681.34941766)(73.25361654,681.21942034)
\curveto(73.26360893,681.09941791)(73.26360893,681.00441801)(73.25361654,680.93442034)
\lineto(73.25361654,680.85942034)
\curveto(73.23360896,680.79941821)(73.21860897,680.73941827)(73.20861654,680.67942034)
\curveto(73.20860898,680.62941838)(73.20360899,680.57941843)(73.19361654,680.52942034)
\curveto(73.12360907,680.22941878)(73.01360918,679.96441905)(72.86361654,679.73442034)
\curveto(72.70360949,679.49441952)(72.50860968,679.29941971)(72.27861654,679.14942034)
\curveto(72.04861014,678.99942001)(71.7886104,678.86942014)(71.49861654,678.75942034)
\curveto(71.3886108,678.7094203)(71.26861092,678.67442034)(71.13861654,678.65442034)
\curveto(71.01861117,678.63442038)(70.89861129,678.6094204)(70.77861654,678.57942034)
\curveto(70.6886115,678.55942045)(70.5936116,678.54942046)(70.49361654,678.54942034)
\curveto(70.40361179,678.53942047)(70.31361188,678.52442049)(70.22361654,678.50442034)
\lineto(69.95361654,678.50442034)
\curveto(69.8936123,678.48442053)(69.7886124,678.47442054)(69.63861654,678.47442034)
\curveto(69.49861269,678.47442054)(69.39861279,678.48442053)(69.33861654,678.50442034)
\curveto(69.30861288,678.50442051)(69.27361292,678.5094205)(69.23361654,678.51942034)
\lineto(69.12861654,678.51942034)
\curveto(69.00861318,678.53942047)(68.8886133,678.55442046)(68.76861654,678.56442034)
\curveto(68.64861354,678.57442044)(68.53361366,678.59442042)(68.42361654,678.62442034)
\curveto(68.03361416,678.73442028)(67.6886145,678.85942015)(67.38861654,678.99942034)
\curveto(67.0886151,679.14941986)(66.83361536,679.36941964)(66.62361654,679.65942034)
\curveto(66.48361571,679.84941916)(66.36361583,680.06941894)(66.26361654,680.31942034)
\curveto(66.24361595,680.37941863)(66.22361597,680.45941855)(66.20361654,680.55942034)
\curveto(66.18361601,680.6094184)(66.16861602,680.67941833)(66.15861654,680.76942034)
\curveto(66.14861604,680.85941815)(66.15361604,680.93441808)(66.17361654,680.99442034)
\curveto(66.20361599,681.06441795)(66.25361594,681.1144179)(66.32361654,681.14442034)
\curveto(66.37361582,681.16441785)(66.43361576,681.17441784)(66.50361654,681.17442034)
\lineto(66.72861654,681.17442034)
\lineto(67.43361654,681.17442034)
\lineto(67.67361654,681.17442034)
\curveto(67.75361444,681.17441784)(67.82361437,681.16441785)(67.88361654,681.14442034)
\curveto(67.9936142,681.10441791)(68.06361413,681.03941797)(68.09361654,680.94942034)
\curveto(68.13361406,680.85941815)(68.17861401,680.76441825)(68.22861654,680.66442034)
\curveto(68.24861394,680.6144184)(68.28361391,680.54941846)(68.33361654,680.46942034)
\curveto(68.3936138,680.38941862)(68.44361375,680.33941867)(68.48361654,680.31942034)
\curveto(68.60361359,680.21941879)(68.71861347,680.13941887)(68.82861654,680.07942034)
\curveto(68.93861325,680.02941898)(69.07861311,679.97941903)(69.24861654,679.92942034)
\curveto(69.29861289,679.9094191)(69.34861284,679.89941911)(69.39861654,679.89942034)
\curveto(69.44861274,679.9094191)(69.49861269,679.9094191)(69.54861654,679.89942034)
\curveto(69.62861256,679.87941913)(69.71361248,679.86941914)(69.80361654,679.86942034)
\curveto(69.90361229,679.87941913)(69.9886122,679.89441912)(70.05861654,679.91442034)
\curveto(70.10861208,679.92441909)(70.15361204,679.92941908)(70.19361654,679.92942034)
\curveto(70.24361195,679.92941908)(70.2936119,679.93941907)(70.34361654,679.95942034)
\curveto(70.48361171,680.009419)(70.60861158,680.06941894)(70.71861654,680.13942034)
\curveto(70.83861135,680.2094188)(70.93361126,680.29941871)(71.00361654,680.40942034)
\curveto(71.05361114,680.48941852)(71.0936111,680.6144184)(71.12361654,680.78442034)
\curveto(71.14361105,680.85441816)(71.14361105,680.91941809)(71.12361654,680.97942034)
\curveto(71.10361109,681.03941797)(71.08361111,681.08941792)(71.06361654,681.12942034)
\curveto(70.9936112,681.26941774)(70.90361129,681.37441764)(70.79361654,681.44442034)
\curveto(70.6936115,681.5144175)(70.57361162,681.57941743)(70.43361654,681.63942034)
\curveto(70.24361195,681.71941729)(70.04361215,681.78441723)(69.83361654,681.83442034)
\curveto(69.62361257,681.88441713)(69.41361278,681.93941707)(69.20361654,681.99942034)
\curveto(69.12361307,682.01941699)(69.03861315,682.03441698)(68.94861654,682.04442034)
\curveto(68.86861332,682.05441696)(68.7886134,682.06941694)(68.70861654,682.08942034)
\curveto(68.3886138,682.17941683)(68.08361411,682.26441675)(67.79361654,682.34442034)
\curveto(67.50361469,682.43441658)(67.23861495,682.56441645)(66.99861654,682.73442034)
\curveto(66.71861547,682.93441608)(66.51361568,683.20441581)(66.38361654,683.54442034)
\curveto(66.36361583,683.6144154)(66.34361585,683.7094153)(66.32361654,683.82942034)
\curveto(66.30361589,683.89941511)(66.2886159,683.98441503)(66.27861654,684.08442034)
\curveto(66.26861592,684.18441483)(66.27361592,684.27441474)(66.29361654,684.35442034)
\curveto(66.31361588,684.40441461)(66.31861587,684.44441457)(66.30861654,684.47442034)
\curveto(66.29861589,684.5144145)(66.30361589,684.55941445)(66.32361654,684.60942034)
\curveto(66.34361585,684.71941429)(66.36361583,684.81941419)(66.38361654,684.90942034)
\curveto(66.41361578,685.009414)(66.44861574,685.10441391)(66.48861654,685.19442034)
\curveto(66.61861557,685.48441353)(66.79861539,685.71941329)(67.02861654,685.89942034)
\curveto(67.25861493,686.07941293)(67.51861467,686.22441279)(67.80861654,686.33442034)
\curveto(67.91861427,686.38441263)(68.03361416,686.41941259)(68.15361654,686.43942034)
\curveto(68.27361392,686.46941254)(68.39861379,686.49941251)(68.52861654,686.52942034)
\curveto(68.5886136,686.54941246)(68.64861354,686.55941245)(68.70861654,686.55942034)
\lineto(68.88861654,686.58942034)
\curveto(68.96861322,686.59941241)(69.05361314,686.60441241)(69.14361654,686.60442034)
\curveto(69.23361296,686.60441241)(69.31861287,686.6094124)(69.39861654,686.61942034)
}
}
{
\newrgbcolor{curcolor}{0 0 0}
\pscustom[linestyle=none,fillstyle=solid,fillcolor=curcolor]
{
\newpath
\moveto(74.90525717,686.39442034)
\lineto(76.03025717,686.39442034)
\curveto(76.14025473,686.39441262)(76.24025463,686.38941262)(76.33025717,686.37942034)
\curveto(76.42025445,686.36941264)(76.48525439,686.33441268)(76.52525717,686.27442034)
\curveto(76.5752543,686.2144128)(76.60525427,686.12941288)(76.61525717,686.01942034)
\curveto(76.62525425,685.91941309)(76.63025424,685.8144132)(76.63025717,685.70442034)
\lineto(76.63025717,684.65442034)
\lineto(76.63025717,682.41942034)
\curveto(76.63025424,682.05941695)(76.64525423,681.71941729)(76.67525717,681.39942034)
\curveto(76.70525417,681.07941793)(76.79525408,680.8144182)(76.94525717,680.60442034)
\curveto(77.08525379,680.39441862)(77.31025356,680.24441877)(77.62025717,680.15442034)
\curveto(77.6702532,680.14441887)(77.71025316,680.13941887)(77.74025717,680.13942034)
\curveto(77.78025309,680.13941887)(77.82525305,680.13441888)(77.87525717,680.12442034)
\curveto(77.92525295,680.1144189)(77.98025289,680.1094189)(78.04025717,680.10942034)
\curveto(78.10025277,680.1094189)(78.14525273,680.1144189)(78.17525717,680.12442034)
\curveto(78.22525265,680.14441887)(78.26525261,680.14941886)(78.29525717,680.13942034)
\curveto(78.33525254,680.12941888)(78.3752525,680.13441888)(78.41525717,680.15442034)
\curveto(78.62525225,680.20441881)(78.79025208,680.26941874)(78.91025717,680.34942034)
\curveto(79.09025178,680.45941855)(79.23025164,680.59941841)(79.33025717,680.76942034)
\curveto(79.44025143,680.94941806)(79.51525136,681.14441787)(79.55525717,681.35442034)
\curveto(79.60525127,681.57441744)(79.63525124,681.8144172)(79.64525717,682.07442034)
\curveto(79.65525122,682.34441667)(79.66025121,682.62441639)(79.66025717,682.91442034)
\lineto(79.66025717,684.72942034)
\lineto(79.66025717,685.70442034)
\lineto(79.66025717,685.97442034)
\curveto(79.66025121,686.07441294)(79.68025119,686.15441286)(79.72025717,686.21442034)
\curveto(79.7702511,686.30441271)(79.84525103,686.35441266)(79.94525717,686.36442034)
\curveto(80.04525083,686.38441263)(80.16525071,686.39441262)(80.30525717,686.39442034)
\lineto(81.10025717,686.39442034)
\lineto(81.38525717,686.39442034)
\curveto(81.4752494,686.39441262)(81.55024932,686.37441264)(81.61025717,686.33442034)
\curveto(81.69024918,686.28441273)(81.73524914,686.2094128)(81.74525717,686.10942034)
\curveto(81.75524912,686.009413)(81.76024911,685.89441312)(81.76025717,685.76442034)
\lineto(81.76025717,684.62442034)
\lineto(81.76025717,680.40942034)
\lineto(81.76025717,679.34442034)
\lineto(81.76025717,679.04442034)
\curveto(81.76024911,678.94442007)(81.74024913,678.86942014)(81.70025717,678.81942034)
\curveto(81.65024922,678.73942027)(81.5752493,678.69442032)(81.47525717,678.68442034)
\curveto(81.3752495,678.67442034)(81.2702496,678.66942034)(81.16025717,678.66942034)
\lineto(80.35025717,678.66942034)
\curveto(80.24025063,678.66942034)(80.14025073,678.67442034)(80.05025717,678.68442034)
\curveto(79.9702509,678.69442032)(79.90525097,678.73442028)(79.85525717,678.80442034)
\curveto(79.83525104,678.83442018)(79.81525106,678.87942013)(79.79525717,678.93942034)
\curveto(79.78525109,678.99942001)(79.7702511,679.05941995)(79.75025717,679.11942034)
\curveto(79.74025113,679.17941983)(79.72525115,679.23441978)(79.70525717,679.28442034)
\curveto(79.68525119,679.33441968)(79.65525122,679.36441965)(79.61525717,679.37442034)
\curveto(79.59525128,679.39441962)(79.5702513,679.39941961)(79.54025717,679.38942034)
\curveto(79.51025136,679.37941963)(79.48525139,679.36941964)(79.46525717,679.35942034)
\curveto(79.39525148,679.31941969)(79.33525154,679.27441974)(79.28525717,679.22442034)
\curveto(79.23525164,679.17441984)(79.18025169,679.12941988)(79.12025717,679.08942034)
\curveto(79.08025179,679.05941995)(79.04025183,679.02441999)(79.00025717,678.98442034)
\curveto(78.9702519,678.95442006)(78.93025194,678.92442009)(78.88025717,678.89442034)
\curveto(78.65025222,678.75442026)(78.38025249,678.64442037)(78.07025717,678.56442034)
\curveto(78.00025287,678.54442047)(77.93025294,678.53442048)(77.86025717,678.53442034)
\curveto(77.79025308,678.52442049)(77.71525316,678.5094205)(77.63525717,678.48942034)
\curveto(77.59525328,678.47942053)(77.55025332,678.47942053)(77.50025717,678.48942034)
\curveto(77.46025341,678.48942052)(77.42025345,678.48442053)(77.38025717,678.47442034)
\curveto(77.35025352,678.46442055)(77.28525359,678.46442055)(77.18525717,678.47442034)
\curveto(77.09525378,678.47442054)(77.03525384,678.47942053)(77.00525717,678.48942034)
\curveto(76.95525392,678.48942052)(76.90525397,678.49442052)(76.85525717,678.50442034)
\lineto(76.70525717,678.50442034)
\curveto(76.58525429,678.53442048)(76.4702544,678.55942045)(76.36025717,678.57942034)
\curveto(76.25025462,678.59942041)(76.14025473,678.62942038)(76.03025717,678.66942034)
\curveto(75.98025489,678.68942032)(75.93525494,678.70442031)(75.89525717,678.71442034)
\curveto(75.86525501,678.73442028)(75.82525505,678.75442026)(75.77525717,678.77442034)
\curveto(75.42525545,678.96442005)(75.14525573,679.22941978)(74.93525717,679.56942034)
\curveto(74.80525607,679.77941923)(74.71025616,680.02941898)(74.65025717,680.31942034)
\curveto(74.59025628,680.61941839)(74.55025632,680.93441808)(74.53025717,681.26442034)
\curveto(74.52025635,681.60441741)(74.51525636,681.94941706)(74.51525717,682.29942034)
\curveto(74.52525635,682.65941635)(74.53025634,683.014416)(74.53025717,683.36442034)
\lineto(74.53025717,685.40442034)
\curveto(74.53025634,685.53441348)(74.52525635,685.68441333)(74.51525717,685.85442034)
\curveto(74.51525636,686.03441298)(74.54025633,686.16441285)(74.59025717,686.24442034)
\curveto(74.62025625,686.29441272)(74.68025619,686.33941267)(74.77025717,686.37942034)
\curveto(74.83025604,686.37941263)(74.875256,686.38441263)(74.90525717,686.39442034)
}
}
{
\newrgbcolor{curcolor}{0 0 0}
\pscustom[linestyle=none,fillstyle=solid,fillcolor=curcolor]
{
}
}
{
\newrgbcolor{curcolor}{0 0 0}
\pscustom[linestyle=none,fillstyle=solid,fillcolor=curcolor]
{
\newpath
\moveto(88.60166342,688.71942034)
\lineto(89.60666342,688.71942034)
\curveto(89.75666043,688.71941029)(89.8866603,688.7094103)(89.99666342,688.68942034)
\curveto(90.11666007,688.67941033)(90.20165999,688.61941039)(90.25166342,688.50942034)
\curveto(90.27165992,688.45941055)(90.28165991,688.39941061)(90.28166342,688.32942034)
\lineto(90.28166342,688.11942034)
\lineto(90.28166342,687.44442034)
\curveto(90.28165991,687.39441162)(90.27665991,687.33441168)(90.26666342,687.26442034)
\curveto(90.26665992,687.20441181)(90.27165992,687.14941186)(90.28166342,687.09942034)
\lineto(90.28166342,686.93442034)
\curveto(90.28165991,686.85441216)(90.2866599,686.77941223)(90.29666342,686.70942034)
\curveto(90.30665988,686.64941236)(90.33165986,686.59441242)(90.37166342,686.54442034)
\curveto(90.44165975,686.45441256)(90.56665962,686.40441261)(90.74666342,686.39442034)
\lineto(91.28666342,686.39442034)
\lineto(91.46666342,686.39442034)
\curveto(91.52665866,686.39441262)(91.58165861,686.38441263)(91.63166342,686.36442034)
\curveto(91.74165845,686.3144127)(91.80165839,686.22441279)(91.81166342,686.09442034)
\curveto(91.83165836,685.96441305)(91.84165835,685.81941319)(91.84166342,685.65942034)
\lineto(91.84166342,685.44942034)
\curveto(91.85165834,685.37941363)(91.84665834,685.31941369)(91.82666342,685.26942034)
\curveto(91.77665841,685.1094139)(91.67165852,685.02441399)(91.51166342,685.01442034)
\curveto(91.35165884,685.00441401)(91.17165902,684.99941401)(90.97166342,684.99942034)
\lineto(90.83666342,684.99942034)
\curveto(90.79665939,685.009414)(90.76165943,685.009414)(90.73166342,684.99942034)
\curveto(90.6916595,684.98941402)(90.65665953,684.98441403)(90.62666342,684.98442034)
\curveto(90.59665959,684.99441402)(90.56665962,684.98941402)(90.53666342,684.96942034)
\curveto(90.45665973,684.94941406)(90.39665979,684.90441411)(90.35666342,684.83442034)
\curveto(90.32665986,684.77441424)(90.30165989,684.69941431)(90.28166342,684.60942034)
\curveto(90.27165992,684.55941445)(90.27165992,684.50441451)(90.28166342,684.44442034)
\curveto(90.2916599,684.38441463)(90.2916599,684.32941468)(90.28166342,684.27942034)
\lineto(90.28166342,683.34942034)
\lineto(90.28166342,681.59442034)
\curveto(90.28165991,681.34441767)(90.2866599,681.12441789)(90.29666342,680.93442034)
\curveto(90.31665987,680.75441826)(90.38165981,680.59441842)(90.49166342,680.45442034)
\curveto(90.54165965,680.39441862)(90.60665958,680.34941866)(90.68666342,680.31942034)
\lineto(90.95666342,680.25942034)
\curveto(90.9866592,680.24941876)(91.01665917,680.24441877)(91.04666342,680.24442034)
\curveto(91.0866591,680.25441876)(91.11665907,680.25441876)(91.13666342,680.24442034)
\lineto(91.30166342,680.24442034)
\curveto(91.41165878,680.24441877)(91.50665868,680.23941877)(91.58666342,680.22942034)
\curveto(91.66665852,680.21941879)(91.73165846,680.17941883)(91.78166342,680.10942034)
\curveto(91.82165837,680.04941896)(91.84165835,679.96941904)(91.84166342,679.86942034)
\lineto(91.84166342,679.58442034)
\curveto(91.84165835,679.37441964)(91.83665835,679.17941983)(91.82666342,678.99942034)
\curveto(91.82665836,678.82942018)(91.74665844,678.7144203)(91.58666342,678.65442034)
\curveto(91.53665865,678.63442038)(91.4916587,678.62942038)(91.45166342,678.63942034)
\curveto(91.41165878,678.63942037)(91.36665882,678.62942038)(91.31666342,678.60942034)
\lineto(91.16666342,678.60942034)
\curveto(91.14665904,678.6094204)(91.11665907,678.6144204)(91.07666342,678.62442034)
\curveto(91.03665915,678.62442039)(91.00165919,678.61942039)(90.97166342,678.60942034)
\curveto(90.92165927,678.59942041)(90.86665932,678.59942041)(90.80666342,678.60942034)
\lineto(90.65666342,678.60942034)
\lineto(90.50666342,678.60942034)
\curveto(90.45665973,678.59942041)(90.41165978,678.59942041)(90.37166342,678.60942034)
\lineto(90.20666342,678.60942034)
\curveto(90.15666003,678.61942039)(90.10166009,678.62442039)(90.04166342,678.62442034)
\curveto(89.98166021,678.62442039)(89.92666026,678.62942038)(89.87666342,678.63942034)
\curveto(89.80666038,678.64942036)(89.74166045,678.65942035)(89.68166342,678.66942034)
\lineto(89.50166342,678.69942034)
\curveto(89.3916608,678.72942028)(89.2866609,678.76442025)(89.18666342,678.80442034)
\curveto(89.0866611,678.84442017)(88.9916612,678.88942012)(88.90166342,678.93942034)
\lineto(88.81166342,678.99942034)
\curveto(88.78166141,679.02941998)(88.74666144,679.05941995)(88.70666342,679.08942034)
\curveto(88.6866615,679.1094199)(88.66166153,679.12941988)(88.63166342,679.14942034)
\lineto(88.55666342,679.22442034)
\curveto(88.41666177,679.4144196)(88.31166188,679.62441939)(88.24166342,679.85442034)
\curveto(88.22166197,679.89441912)(88.21166198,679.92941908)(88.21166342,679.95942034)
\curveto(88.22166197,679.99941901)(88.22166197,680.04441897)(88.21166342,680.09442034)
\curveto(88.20166199,680.1144189)(88.19666199,680.13941887)(88.19666342,680.16942034)
\curveto(88.19666199,680.19941881)(88.191662,680.22441879)(88.18166342,680.24442034)
\lineto(88.18166342,680.39442034)
\curveto(88.17166202,680.43441858)(88.16666202,680.47941853)(88.16666342,680.52942034)
\curveto(88.17666201,680.57941843)(88.18166201,680.62941838)(88.18166342,680.67942034)
\lineto(88.18166342,681.24942034)
\lineto(88.18166342,683.48442034)
\lineto(88.18166342,684.27942034)
\lineto(88.18166342,684.48942034)
\curveto(88.191662,684.55941445)(88.186662,684.62441439)(88.16666342,684.68442034)
\curveto(88.12666206,684.82441419)(88.05666213,684.9144141)(87.95666342,684.95442034)
\curveto(87.84666234,685.00441401)(87.70666248,685.01941399)(87.53666342,684.99942034)
\curveto(87.36666282,684.97941403)(87.22166297,684.99441402)(87.10166342,685.04442034)
\curveto(87.02166317,685.07441394)(86.97166322,685.11941389)(86.95166342,685.17942034)
\curveto(86.93166326,685.23941377)(86.91166328,685.3144137)(86.89166342,685.40442034)
\lineto(86.89166342,685.71942034)
\curveto(86.8916633,685.89941311)(86.90166329,686.04441297)(86.92166342,686.15442034)
\curveto(86.94166325,686.26441275)(87.02666316,686.33941267)(87.17666342,686.37942034)
\curveto(87.21666297,686.39941261)(87.25666293,686.40441261)(87.29666342,686.39442034)
\lineto(87.43166342,686.39442034)
\curveto(87.58166261,686.39441262)(87.72166247,686.39941261)(87.85166342,686.40942034)
\curveto(87.98166221,686.42941258)(88.07166212,686.48941252)(88.12166342,686.58942034)
\curveto(88.15166204,686.65941235)(88.16666202,686.73941227)(88.16666342,686.82942034)
\curveto(88.17666201,686.91941209)(88.18166201,687.009412)(88.18166342,687.09942034)
\lineto(88.18166342,688.02942034)
\lineto(88.18166342,688.28442034)
\curveto(88.18166201,688.37441064)(88.191662,688.44941056)(88.21166342,688.50942034)
\curveto(88.26166193,688.6094104)(88.33666185,688.67441034)(88.43666342,688.70442034)
\curveto(88.45666173,688.7144103)(88.48166171,688.7144103)(88.51166342,688.70442034)
\curveto(88.55166164,688.70441031)(88.58166161,688.7094103)(88.60166342,688.71942034)
}
}
{
\newrgbcolor{curcolor}{0 0 0}
\pscustom[linestyle=none,fillstyle=solid,fillcolor=curcolor]
{
\newpath
\moveto(94.92510092,689.25942034)
\curveto(94.99509797,689.17940983)(95.03009793,689.05940995)(95.03010092,688.89942034)
\lineto(95.03010092,688.43442034)
\lineto(95.03010092,688.02942034)
\curveto(95.03009793,687.88941112)(94.99509797,687.79441122)(94.92510092,687.74442034)
\curveto(94.8650981,687.69441132)(94.78509818,687.66441135)(94.68510092,687.65442034)
\curveto(94.59509837,687.64441137)(94.49509847,687.63941137)(94.38510092,687.63942034)
\lineto(93.54510092,687.63942034)
\curveto(93.43509953,687.63941137)(93.33509963,687.64441137)(93.24510092,687.65442034)
\curveto(93.1650998,687.66441135)(93.09509987,687.69441132)(93.03510092,687.74442034)
\curveto(92.99509997,687.77441124)(92.9651,687.82941118)(92.94510092,687.90942034)
\curveto(92.93510003,687.99941101)(92.92510004,688.09441092)(92.91510092,688.19442034)
\lineto(92.91510092,688.52442034)
\curveto(92.92510004,688.63441038)(92.93010003,688.72941028)(92.93010092,688.80942034)
\lineto(92.93010092,689.01942034)
\curveto(92.94010002,689.08940992)(92.9601,689.14940986)(92.99010092,689.19942034)
\curveto(93.01009995,689.23940977)(93.03509993,689.26940974)(93.06510092,689.28942034)
\lineto(93.18510092,689.34942034)
\curveto(93.20509976,689.34940966)(93.23009973,689.34940966)(93.26010092,689.34942034)
\curveto(93.29009967,689.35940965)(93.31509965,689.36440965)(93.33510092,689.36442034)
\lineto(94.43010092,689.36442034)
\curveto(94.53009843,689.36440965)(94.62509834,689.35940965)(94.71510092,689.34942034)
\curveto(94.80509816,689.33940967)(94.87509809,689.3094097)(94.92510092,689.25942034)
\moveto(95.03010092,679.49442034)
\curveto(95.03009793,679.29441972)(95.02509794,679.12441989)(95.01510092,678.98442034)
\curveto(95.00509796,678.84442017)(94.91509805,678.74942026)(94.74510092,678.69942034)
\curveto(94.68509828,678.67942033)(94.62009834,678.66942034)(94.55010092,678.66942034)
\curveto(94.48009848,678.67942033)(94.40509856,678.68442033)(94.32510092,678.68442034)
\lineto(93.48510092,678.68442034)
\curveto(93.39509957,678.68442033)(93.30509966,678.68942032)(93.21510092,678.69942034)
\curveto(93.13509983,678.7094203)(93.07509989,678.73942027)(93.03510092,678.78942034)
\curveto(92.97509999,678.85942015)(92.94010002,678.94442007)(92.93010092,679.04442034)
\lineto(92.93010092,679.38942034)
\lineto(92.93010092,685.71942034)
\lineto(92.93010092,686.01942034)
\curveto(92.93010003,686.11941289)(92.95010001,686.19941281)(92.99010092,686.25942034)
\curveto(93.05009991,686.32941268)(93.13509983,686.37441264)(93.24510092,686.39442034)
\curveto(93.2650997,686.40441261)(93.29009967,686.40441261)(93.32010092,686.39442034)
\curveto(93.3600996,686.39441262)(93.39009957,686.39941261)(93.41010092,686.40942034)
\lineto(94.16010092,686.40942034)
\lineto(94.35510092,686.40942034)
\curveto(94.43509853,686.41941259)(94.50009846,686.41941259)(94.55010092,686.40942034)
\lineto(94.67010092,686.40942034)
\curveto(94.73009823,686.38941262)(94.78509818,686.37441264)(94.83510092,686.36442034)
\curveto(94.88509808,686.35441266)(94.92509804,686.32441269)(94.95510092,686.27442034)
\curveto(94.99509797,686.22441279)(95.01509795,686.15441286)(95.01510092,686.06442034)
\curveto(95.02509794,685.97441304)(95.03009793,685.87941313)(95.03010092,685.77942034)
\lineto(95.03010092,679.49442034)
}
}
{
\newrgbcolor{curcolor}{0 0 0}
\pscustom[linestyle=none,fillstyle=solid,fillcolor=curcolor]
{
\newpath
\moveto(104.58228842,682.62942034)
\curveto(104.59227974,682.56941644)(104.59727973,682.47941653)(104.59728842,682.35942034)
\curveto(104.59727973,682.23941677)(104.58727974,682.15441686)(104.56728842,682.10442034)
\lineto(104.56728842,681.90942034)
\curveto(104.53727979,681.79941721)(104.51727981,681.69441732)(104.50728842,681.59442034)
\curveto(104.50727982,681.49441752)(104.49227984,681.39441762)(104.46228842,681.29442034)
\curveto(104.44227989,681.20441781)(104.42227991,681.1094179)(104.40228842,681.00942034)
\curveto(104.38227995,680.91941809)(104.35227998,680.82941818)(104.31228842,680.73942034)
\curveto(104.24228009,680.56941844)(104.17228016,680.4094186)(104.10228842,680.25942034)
\curveto(104.0322803,680.11941889)(103.95228038,679.97941903)(103.86228842,679.83942034)
\curveto(103.80228053,679.74941926)(103.73728059,679.66441935)(103.66728842,679.58442034)
\curveto(103.60728072,679.5144195)(103.53728079,679.43941957)(103.45728842,679.35942034)
\lineto(103.35228842,679.25442034)
\curveto(103.30228103,679.20441981)(103.24728108,679.15941985)(103.18728842,679.11942034)
\lineto(103.03728842,678.99942034)
\curveto(102.95728137,678.93942007)(102.86728146,678.88442013)(102.76728842,678.83442034)
\curveto(102.67728165,678.79442022)(102.58228175,678.74942026)(102.48228842,678.69942034)
\curveto(102.38228195,678.64942036)(102.27728205,678.6144204)(102.16728842,678.59442034)
\curveto(102.06728226,678.57442044)(101.96228237,678.55442046)(101.85228842,678.53442034)
\curveto(101.79228254,678.5144205)(101.7272826,678.50442051)(101.65728842,678.50442034)
\curveto(101.59728273,678.50442051)(101.5322828,678.49442052)(101.46228842,678.47442034)
\lineto(101.32728842,678.47442034)
\curveto(101.24728308,678.45442056)(101.17228316,678.45442056)(101.10228842,678.47442034)
\lineto(100.95228842,678.47442034)
\curveto(100.89228344,678.49442052)(100.8272835,678.50442051)(100.75728842,678.50442034)
\curveto(100.69728363,678.49442052)(100.63728369,678.49942051)(100.57728842,678.51942034)
\curveto(100.41728391,678.56942044)(100.26228407,678.6144204)(100.11228842,678.65442034)
\curveto(99.97228436,678.69442032)(99.84228449,678.75442026)(99.72228842,678.83442034)
\curveto(99.65228468,678.87442014)(99.58728474,678.9144201)(99.52728842,678.95442034)
\curveto(99.46728486,679.00442001)(99.40228493,679.05441996)(99.33228842,679.10442034)
\lineto(99.15228842,679.23942034)
\curveto(99.07228526,679.29941971)(99.00228533,679.30441971)(98.94228842,679.25442034)
\curveto(98.89228544,679.22441979)(98.86728546,679.18441983)(98.86728842,679.13442034)
\curveto(98.86728546,679.09441992)(98.85728547,679.04441997)(98.83728842,678.98442034)
\curveto(98.81728551,678.88442013)(98.80728552,678.76942024)(98.80728842,678.63942034)
\curveto(98.81728551,678.5094205)(98.82228551,678.38942062)(98.82228842,678.27942034)
\lineto(98.82228842,676.74942034)
\curveto(98.82228551,676.61942239)(98.81728551,676.49442252)(98.80728842,676.37442034)
\curveto(98.80728552,676.24442277)(98.78228555,676.13942287)(98.73228842,676.05942034)
\curveto(98.70228563,676.01942299)(98.64728568,675.98942302)(98.56728842,675.96942034)
\curveto(98.48728584,675.94942306)(98.39728593,675.93942307)(98.29728842,675.93942034)
\curveto(98.19728613,675.92942308)(98.09728623,675.92942308)(97.99728842,675.93942034)
\lineto(97.74228842,675.93942034)
\lineto(97.33728842,675.93942034)
\lineto(97.23228842,675.93942034)
\curveto(97.19228714,675.93942307)(97.15728717,675.94442307)(97.12728842,675.95442034)
\lineto(97.00728842,675.95442034)
\curveto(96.83728749,676.00442301)(96.74728758,676.10442291)(96.73728842,676.25442034)
\curveto(96.7272876,676.39442262)(96.72228761,676.56442245)(96.72228842,676.76442034)
\lineto(96.72228842,685.56942034)
\curveto(96.72228761,685.67941333)(96.71728761,685.79441322)(96.70728842,685.91442034)
\curveto(96.70728762,686.04441297)(96.7322876,686.14441287)(96.78228842,686.21442034)
\curveto(96.82228751,686.28441273)(96.87728745,686.32941268)(96.94728842,686.34942034)
\curveto(96.99728733,686.36941264)(97.05728727,686.37941263)(97.12728842,686.37942034)
\lineto(97.35228842,686.37942034)
\lineto(98.07228842,686.37942034)
\lineto(98.35728842,686.37942034)
\curveto(98.44728588,686.37941263)(98.52228581,686.35441266)(98.58228842,686.30442034)
\curveto(98.65228568,686.25441276)(98.68728564,686.18941282)(98.68728842,686.10942034)
\curveto(98.69728563,686.03941297)(98.72228561,685.96441305)(98.76228842,685.88442034)
\curveto(98.77228556,685.85441316)(98.78228555,685.82941318)(98.79228842,685.80942034)
\curveto(98.81228552,685.79941321)(98.8322855,685.78441323)(98.85228842,685.76442034)
\curveto(98.96228537,685.75441326)(99.05228528,685.78441323)(99.12228842,685.85442034)
\curveto(99.19228514,685.92441309)(99.26228507,685.98441303)(99.33228842,686.03442034)
\curveto(99.46228487,686.12441289)(99.59728473,686.20441281)(99.73728842,686.27442034)
\curveto(99.87728445,686.35441266)(100.0322843,686.41941259)(100.20228842,686.46942034)
\curveto(100.28228405,686.49941251)(100.36728396,686.51941249)(100.45728842,686.52942034)
\curveto(100.55728377,686.53941247)(100.65228368,686.55441246)(100.74228842,686.57442034)
\curveto(100.78228355,686.58441243)(100.82228351,686.58441243)(100.86228842,686.57442034)
\curveto(100.91228342,686.56441245)(100.95228338,686.56941244)(100.98228842,686.58942034)
\curveto(101.55228278,686.6094124)(102.0322823,686.52941248)(102.42228842,686.34942034)
\curveto(102.82228151,686.17941283)(103.16228117,685.95441306)(103.44228842,685.67442034)
\curveto(103.49228084,685.62441339)(103.53728079,685.57441344)(103.57728842,685.52442034)
\curveto(103.61728071,685.48441353)(103.65728067,685.43941357)(103.69728842,685.38942034)
\curveto(103.76728056,685.29941371)(103.8272805,685.2094138)(103.87728842,685.11942034)
\curveto(103.93728039,685.02941398)(103.99228034,684.93941407)(104.04228842,684.84942034)
\curveto(104.06228027,684.82941418)(104.07228026,684.80441421)(104.07228842,684.77442034)
\curveto(104.08228025,684.74441427)(104.09728023,684.7094143)(104.11728842,684.66942034)
\curveto(104.17728015,684.56941444)(104.2322801,684.44941456)(104.28228842,684.30942034)
\curveto(104.30228003,684.24941476)(104.32228001,684.18441483)(104.34228842,684.11442034)
\curveto(104.36227997,684.05441496)(104.38227995,683.98941502)(104.40228842,683.91942034)
\curveto(104.44227989,683.79941521)(104.46727986,683.67441534)(104.47728842,683.54442034)
\curveto(104.49727983,683.4144156)(104.52227981,683.27941573)(104.55228842,683.13942034)
\lineto(104.55228842,682.97442034)
\lineto(104.58228842,682.79442034)
\lineto(104.58228842,682.62942034)
\moveto(102.46728842,682.28442034)
\curveto(102.47728185,682.33441668)(102.48228185,682.39941661)(102.48228842,682.47942034)
\curveto(102.48228185,682.56941644)(102.47728185,682.63941637)(102.46728842,682.68942034)
\lineto(102.46728842,682.82442034)
\curveto(102.44728188,682.88441613)(102.43728189,682.94941606)(102.43728842,683.01942034)
\curveto(102.43728189,683.08941592)(102.4272819,683.15941585)(102.40728842,683.22942034)
\curveto(102.38728194,683.32941568)(102.36728196,683.42441559)(102.34728842,683.51442034)
\curveto(102.327282,683.6144154)(102.29728203,683.70441531)(102.25728842,683.78442034)
\curveto(102.13728219,684.10441491)(101.98228235,684.35941465)(101.79228842,684.54942034)
\curveto(101.60228273,684.73941427)(101.332283,684.87941413)(100.98228842,684.96942034)
\curveto(100.90228343,684.98941402)(100.81228352,684.99941401)(100.71228842,684.99942034)
\lineto(100.44228842,684.99942034)
\curveto(100.40228393,684.98941402)(100.36728396,684.98441403)(100.33728842,684.98442034)
\curveto(100.30728402,684.98441403)(100.27228406,684.97941403)(100.23228842,684.96942034)
\lineto(100.02228842,684.90942034)
\curveto(99.96228437,684.89941411)(99.90228443,684.87941413)(99.84228842,684.84942034)
\curveto(99.58228475,684.73941427)(99.37728495,684.56941444)(99.22728842,684.33942034)
\curveto(99.08728524,684.1094149)(98.97228536,683.85441516)(98.88228842,683.57442034)
\curveto(98.86228547,683.49441552)(98.84728548,683.4094156)(98.83728842,683.31942034)
\curveto(98.8272855,683.23941577)(98.81228552,683.15941585)(98.79228842,683.07942034)
\curveto(98.78228555,683.03941597)(98.77728555,682.97441604)(98.77728842,682.88442034)
\curveto(98.75728557,682.84441617)(98.75228558,682.79441622)(98.76228842,682.73442034)
\curveto(98.77228556,682.68441633)(98.77228556,682.63441638)(98.76228842,682.58442034)
\curveto(98.74228559,682.52441649)(98.74228559,682.46941654)(98.76228842,682.41942034)
\lineto(98.76228842,682.23942034)
\lineto(98.76228842,682.10442034)
\curveto(98.76228557,682.06441695)(98.77228556,682.02441699)(98.79228842,681.98442034)
\curveto(98.79228554,681.9144171)(98.79728553,681.85941715)(98.80728842,681.81942034)
\lineto(98.83728842,681.63942034)
\curveto(98.84728548,681.57941743)(98.86228547,681.51941749)(98.88228842,681.45942034)
\curveto(98.97228536,681.16941784)(99.07728525,680.92941808)(99.19728842,680.73942034)
\curveto(99.327285,680.55941845)(99.50728482,680.39941861)(99.73728842,680.25942034)
\curveto(99.87728445,680.17941883)(100.04228429,680.1144189)(100.23228842,680.06442034)
\curveto(100.27228406,680.05441896)(100.30728402,680.04941896)(100.33728842,680.04942034)
\curveto(100.36728396,680.05941895)(100.40228393,680.05941895)(100.44228842,680.04942034)
\curveto(100.48228385,680.03941897)(100.54228379,680.02941898)(100.62228842,680.01942034)
\curveto(100.70228363,680.01941899)(100.76728356,680.02441899)(100.81728842,680.03442034)
\curveto(100.89728343,680.05441896)(100.97728335,680.06941894)(101.05728842,680.07942034)
\curveto(101.14728318,680.09941891)(101.2322831,680.12441889)(101.31228842,680.15442034)
\curveto(101.55228278,680.25441876)(101.74728258,680.39441862)(101.89728842,680.57442034)
\curveto(102.04728228,680.75441826)(102.17228216,680.96441805)(102.27228842,681.20442034)
\curveto(102.32228201,681.32441769)(102.35728197,681.44941756)(102.37728842,681.57942034)
\curveto(102.39728193,681.7094173)(102.42228191,681.84441717)(102.45228842,681.98442034)
\lineto(102.45228842,682.13442034)
\curveto(102.46228187,682.18441683)(102.46728186,682.23441678)(102.46728842,682.28442034)
}
}
{
\newrgbcolor{curcolor}{0 0 0}
\pscustom[linestyle=none,fillstyle=solid,fillcolor=curcolor]
{
\newpath
\moveto(113.63221029,682.85442034)
\curveto(113.65220172,682.79441622)(113.66220171,682.7094163)(113.66221029,682.59942034)
\curveto(113.66220171,682.48941652)(113.65220172,682.40441661)(113.63221029,682.34442034)
\lineto(113.63221029,682.19442034)
\curveto(113.61220176,682.1144169)(113.60220177,682.03441698)(113.60221029,681.95442034)
\curveto(113.61220176,681.87441714)(113.60720177,681.79441722)(113.58721029,681.71442034)
\curveto(113.56720181,681.64441737)(113.55220182,681.57941743)(113.54221029,681.51942034)
\curveto(113.53220184,681.45941755)(113.52220185,681.39441762)(113.51221029,681.32442034)
\curveto(113.4722019,681.2144178)(113.43720194,681.09941791)(113.40721029,680.97942034)
\curveto(113.377202,680.86941814)(113.33720204,680.76441825)(113.28721029,680.66442034)
\curveto(113.0772023,680.18441883)(112.80220257,679.79441922)(112.46221029,679.49442034)
\curveto(112.12220325,679.19441982)(111.71220366,678.94442007)(111.23221029,678.74442034)
\curveto(111.11220426,678.69442032)(110.98720439,678.65942035)(110.85721029,678.63942034)
\curveto(110.73720464,678.6094204)(110.61220476,678.57942043)(110.48221029,678.54942034)
\curveto(110.43220494,678.52942048)(110.377205,678.51942049)(110.31721029,678.51942034)
\curveto(110.25720512,678.51942049)(110.20220517,678.5144205)(110.15221029,678.50442034)
\lineto(110.04721029,678.50442034)
\curveto(110.01720536,678.49442052)(109.98720539,678.48942052)(109.95721029,678.48942034)
\curveto(109.90720547,678.47942053)(109.82720555,678.47442054)(109.71721029,678.47442034)
\curveto(109.60720577,678.46442055)(109.52220585,678.46942054)(109.46221029,678.48942034)
\lineto(109.31221029,678.48942034)
\curveto(109.26220611,678.49942051)(109.20720617,678.50442051)(109.14721029,678.50442034)
\curveto(109.09720628,678.49442052)(109.04720633,678.49942051)(108.99721029,678.51942034)
\curveto(108.95720642,678.52942048)(108.91720646,678.53442048)(108.87721029,678.53442034)
\curveto(108.84720653,678.53442048)(108.80720657,678.53942047)(108.75721029,678.54942034)
\curveto(108.65720672,678.57942043)(108.55720682,678.60442041)(108.45721029,678.62442034)
\curveto(108.35720702,678.64442037)(108.26220711,678.67442034)(108.17221029,678.71442034)
\curveto(108.05220732,678.75442026)(107.93720744,678.79442022)(107.82721029,678.83442034)
\curveto(107.72720765,678.87442014)(107.62220775,678.92442009)(107.51221029,678.98442034)
\curveto(107.16220821,679.19441982)(106.86220851,679.43941957)(106.61221029,679.71942034)
\curveto(106.36220901,679.99941901)(106.15220922,680.33441868)(105.98221029,680.72442034)
\curveto(105.93220944,680.8144182)(105.89220948,680.9094181)(105.86221029,681.00942034)
\curveto(105.84220953,681.1094179)(105.81720956,681.2144178)(105.78721029,681.32442034)
\curveto(105.76720961,681.37441764)(105.75720962,681.41941759)(105.75721029,681.45942034)
\curveto(105.75720962,681.49941751)(105.74720963,681.54441747)(105.72721029,681.59442034)
\curveto(105.70720967,681.67441734)(105.69720968,681.75441726)(105.69721029,681.83442034)
\curveto(105.69720968,681.92441709)(105.68720969,682.009417)(105.66721029,682.08942034)
\curveto(105.65720972,682.13941687)(105.65220972,682.18441683)(105.65221029,682.22442034)
\lineto(105.65221029,682.35942034)
\curveto(105.63220974,682.41941659)(105.62220975,682.50441651)(105.62221029,682.61442034)
\curveto(105.63220974,682.72441629)(105.64720973,682.8094162)(105.66721029,682.86942034)
\lineto(105.66721029,682.97442034)
\curveto(105.6772097,683.02441599)(105.6772097,683.07441594)(105.66721029,683.12442034)
\curveto(105.66720971,683.18441583)(105.6772097,683.23941577)(105.69721029,683.28942034)
\curveto(105.70720967,683.33941567)(105.71220966,683.38441563)(105.71221029,683.42442034)
\curveto(105.71220966,683.47441554)(105.72220965,683.52441549)(105.74221029,683.57442034)
\curveto(105.78220959,683.70441531)(105.81720956,683.82941518)(105.84721029,683.94942034)
\curveto(105.8772095,684.07941493)(105.91720946,684.20441481)(105.96721029,684.32442034)
\curveto(106.14720923,684.73441428)(106.36220901,685.07441394)(106.61221029,685.34442034)
\curveto(106.86220851,685.62441339)(107.16720821,685.87941313)(107.52721029,686.10942034)
\curveto(107.62720775,686.15941285)(107.73220764,686.20441281)(107.84221029,686.24442034)
\curveto(107.95220742,686.28441273)(108.06220731,686.32941268)(108.17221029,686.37942034)
\curveto(108.30220707,686.42941258)(108.43720694,686.46441255)(108.57721029,686.48442034)
\curveto(108.71720666,686.50441251)(108.86220651,686.53441248)(109.01221029,686.57442034)
\curveto(109.09220628,686.58441243)(109.16720621,686.58941242)(109.23721029,686.58942034)
\curveto(109.30720607,686.58941242)(109.377206,686.59441242)(109.44721029,686.60442034)
\curveto(110.02720535,686.6144124)(110.52720485,686.55441246)(110.94721029,686.42442034)
\curveto(111.377204,686.29441272)(111.75720362,686.1144129)(112.08721029,685.88442034)
\curveto(112.19720318,685.80441321)(112.30720307,685.7144133)(112.41721029,685.61442034)
\curveto(112.53720284,685.52441349)(112.63720274,685.42441359)(112.71721029,685.31442034)
\curveto(112.79720258,685.2144138)(112.86720251,685.1144139)(112.92721029,685.01442034)
\curveto(112.99720238,684.9144141)(113.06720231,684.8094142)(113.13721029,684.69942034)
\curveto(113.20720217,684.58941442)(113.26220211,684.46941454)(113.30221029,684.33942034)
\curveto(113.34220203,684.21941479)(113.38720199,684.08941492)(113.43721029,683.94942034)
\curveto(113.46720191,683.86941514)(113.49220188,683.78441523)(113.51221029,683.69442034)
\lineto(113.57221029,683.42442034)
\curveto(113.58220179,683.38441563)(113.58720179,683.34441567)(113.58721029,683.30442034)
\curveto(113.58720179,683.26441575)(113.59220178,683.22441579)(113.60221029,683.18442034)
\curveto(113.62220175,683.13441588)(113.62720175,683.07941593)(113.61721029,683.01942034)
\curveto(113.60720177,682.95941605)(113.61220176,682.90441611)(113.63221029,682.85442034)
\moveto(111.53221029,682.31442034)
\curveto(111.54220383,682.36441665)(111.54720383,682.43441658)(111.54721029,682.52442034)
\curveto(111.54720383,682.62441639)(111.54220383,682.69941631)(111.53221029,682.74942034)
\lineto(111.53221029,682.86942034)
\curveto(111.51220386,682.91941609)(111.50220387,682.97441604)(111.50221029,683.03442034)
\curveto(111.50220387,683.09441592)(111.49720388,683.14941586)(111.48721029,683.19942034)
\curveto(111.48720389,683.23941577)(111.48220389,683.26941574)(111.47221029,683.28942034)
\lineto(111.41221029,683.52942034)
\curveto(111.40220397,683.61941539)(111.38220399,683.70441531)(111.35221029,683.78442034)
\curveto(111.24220413,684.04441497)(111.11220426,684.26441475)(110.96221029,684.44442034)
\curveto(110.81220456,684.63441438)(110.61220476,684.78441423)(110.36221029,684.89442034)
\curveto(110.30220507,684.9144141)(110.24220513,684.92941408)(110.18221029,684.93942034)
\curveto(110.12220525,684.95941405)(110.05720532,684.97941403)(109.98721029,684.99942034)
\curveto(109.90720547,685.01941399)(109.82220555,685.02441399)(109.73221029,685.01442034)
\lineto(109.46221029,685.01442034)
\curveto(109.43220594,684.99441402)(109.39720598,684.98441403)(109.35721029,684.98442034)
\curveto(109.31720606,684.99441402)(109.28220609,684.99441402)(109.25221029,684.98442034)
\lineto(109.04221029,684.92442034)
\curveto(108.98220639,684.9144141)(108.92720645,684.89441412)(108.87721029,684.86442034)
\curveto(108.62720675,684.75441426)(108.42220695,684.59441442)(108.26221029,684.38442034)
\curveto(108.11220726,684.18441483)(107.99220738,683.94941506)(107.90221029,683.67942034)
\curveto(107.8722075,683.57941543)(107.84720753,683.47441554)(107.82721029,683.36442034)
\curveto(107.81720756,683.25441576)(107.80220757,683.14441587)(107.78221029,683.03442034)
\curveto(107.7722076,682.98441603)(107.76720761,682.93441608)(107.76721029,682.88442034)
\lineto(107.76721029,682.73442034)
\curveto(107.74720763,682.66441635)(107.73720764,682.55941645)(107.73721029,682.41942034)
\curveto(107.74720763,682.27941673)(107.76220761,682.17441684)(107.78221029,682.10442034)
\lineto(107.78221029,681.96942034)
\curveto(107.80220757,681.88941712)(107.81720756,681.8094172)(107.82721029,681.72942034)
\curveto(107.83720754,681.65941735)(107.85220752,681.58441743)(107.87221029,681.50442034)
\curveto(107.9722074,681.20441781)(108.0772073,680.95941805)(108.18721029,680.76942034)
\curveto(108.30720707,680.58941842)(108.49220688,680.42441859)(108.74221029,680.27442034)
\curveto(108.81220656,680.22441879)(108.88720649,680.18441883)(108.96721029,680.15442034)
\curveto(109.05720632,680.12441889)(109.14720623,680.09941891)(109.23721029,680.07942034)
\curveto(109.2772061,680.06941894)(109.31220606,680.06441895)(109.34221029,680.06442034)
\curveto(109.372206,680.07441894)(109.40720597,680.07441894)(109.44721029,680.06442034)
\lineto(109.56721029,680.03442034)
\curveto(109.61720576,680.03441898)(109.66220571,680.03941897)(109.70221029,680.04942034)
\lineto(109.82221029,680.04942034)
\curveto(109.90220547,680.06941894)(109.98220539,680.08441893)(110.06221029,680.09442034)
\curveto(110.14220523,680.10441891)(110.21720516,680.12441889)(110.28721029,680.15442034)
\curveto(110.54720483,680.25441876)(110.75720462,680.38941862)(110.91721029,680.55942034)
\curveto(111.0772043,680.72941828)(111.21220416,680.93941807)(111.32221029,681.18942034)
\curveto(111.36220401,681.28941772)(111.39220398,681.38941762)(111.41221029,681.48942034)
\curveto(111.43220394,681.58941742)(111.45720392,681.69441732)(111.48721029,681.80442034)
\curveto(111.49720388,681.84441717)(111.50220387,681.87941713)(111.50221029,681.90942034)
\curveto(111.50220387,681.94941706)(111.50720387,681.98941702)(111.51721029,682.02942034)
\lineto(111.51721029,682.16442034)
\curveto(111.51720386,682.2144168)(111.52220385,682.26441675)(111.53221029,682.31442034)
}
}
{
\newrgbcolor{curcolor}{0 0 0}
\pscustom[linestyle=none,fillstyle=solid,fillcolor=curcolor]
{
\newpath
\moveto(832.8075815,693.36292498)
\curveto(832.98757973,693.36291429)(833.18757953,693.36291429)(833.4075815,693.36292498)
\curveto(833.62757909,693.37291428)(833.79257892,693.33791431)(833.9025815,693.25792498)
\curveto(833.98257873,693.19791445)(834.05757866,693.10791454)(834.1275815,692.98792498)
\curveto(834.19757852,692.87791477)(834.26257845,692.77791487)(834.3225815,692.68792498)
\curveto(834.45257826,692.48791516)(834.58257813,692.28291537)(834.7125815,692.07292498)
\curveto(834.85257786,691.87291578)(834.98757773,691.66791598)(835.1175815,691.45792498)
\lineto(835.3275815,691.12792498)
\curveto(835.40757731,691.02791662)(835.48257723,690.92291673)(835.5525815,690.81292498)
\curveto(835.85257686,690.33291732)(836.15757656,689.8529178)(836.4675815,689.37292498)
\curveto(836.77757594,688.90291875)(837.08757563,688.42791922)(837.3975815,687.94792498)
\curveto(837.47757524,687.80791984)(837.56257515,687.67291998)(837.6525815,687.54292498)
\curveto(837.75257496,687.42292023)(837.84257487,687.29292036)(837.9225815,687.15292498)
\lineto(838.4325815,686.34292498)
\curveto(838.6125741,686.08292157)(838.78757393,685.82292183)(838.9575815,685.56292498)
\curveto(839.00757371,685.48292217)(839.06757365,685.38292227)(839.1375815,685.26292498)
\curveto(839.2175735,685.1529225)(839.3125734,685.09792255)(839.4225815,685.09792498)
\curveto(839.47257324,685.11792253)(839.49757322,685.13292252)(839.4975815,685.14292498)
\curveto(839.54757317,685.20292245)(839.57257314,685.28792236)(839.5725815,685.39792498)
\lineto(839.5725815,685.71292498)
\lineto(839.5725815,686.89792498)
\lineto(839.5725815,691.48792498)
\lineto(839.5725815,692.38792498)
\curveto(839.57257314,692.45791519)(839.56757315,692.53291512)(839.5575815,692.61292498)
\curveto(839.54757317,692.69291496)(839.55257316,692.76791488)(839.5725815,692.83792498)
\lineto(839.5725815,693.00292498)
\curveto(839.59257312,693.04291461)(839.60257311,693.08291457)(839.6025815,693.12292498)
\curveto(839.6125731,693.16291449)(839.62757309,693.19791445)(839.6475815,693.22792498)
\curveto(839.70757301,693.30791434)(839.79757292,693.3479143)(839.9175815,693.34792498)
\curveto(840.03757268,693.35791429)(840.16757255,693.36291429)(840.3075815,693.36292498)
\curveto(840.36757235,693.36291429)(840.42757229,693.36291429)(840.4875815,693.36292498)
\curveto(840.55757216,693.36291429)(840.6175721,693.3529143)(840.6675815,693.33292498)
\curveto(840.78757193,693.28291437)(840.85257186,693.19291446)(840.8625815,693.06292498)
\curveto(840.88257183,692.94291471)(840.89257182,692.79791485)(840.8925815,692.62792498)
\lineto(840.8925815,690.97792498)
\lineto(840.8925815,684.69292498)
\lineto(840.8925815,683.43292498)
\lineto(840.8925815,683.10292498)
\curveto(840.90257181,682.99292466)(840.88257183,682.90792474)(840.8325815,682.84792498)
\curveto(840.79257192,682.78792486)(840.74257197,682.7479249)(840.6825815,682.72792498)
\curveto(840.63257208,682.71792493)(840.56757215,682.70292495)(840.4875815,682.68292498)
\lineto(840.0975815,682.68292498)
\lineto(839.7225815,682.68292498)
\curveto(839.60257311,682.68292497)(839.50257321,682.70292495)(839.4225815,682.74292498)
\curveto(839.34257337,682.77292488)(839.27757344,682.82292483)(839.2275815,682.89292498)
\curveto(839.18757353,682.96292469)(839.14257357,683.03292462)(839.0925815,683.10292498)
\curveto(839.0125737,683.22292443)(838.92757379,683.3479243)(838.8375815,683.47792498)
\lineto(838.5975815,683.86792498)
\curveto(838.23757448,684.40792324)(837.88257483,684.94292271)(837.5325815,685.47292498)
\curveto(837.18257553,686.00292165)(836.83257588,686.54292111)(836.4825815,687.09292498)
\curveto(836.29257642,687.39292026)(836.09757662,687.68791996)(835.8975815,687.97792498)
\curveto(835.70757701,688.26791938)(835.5175772,688.56291909)(835.3275815,688.86292498)
\curveto(834.99757772,689.38291827)(834.65257806,689.90791774)(834.2925815,690.43792498)
\curveto(834.25257846,690.50791714)(834.2125785,690.57291708)(834.1725815,690.63292498)
\curveto(834.13257858,690.70291695)(834.07757864,690.76291689)(834.0075815,690.81292498)
\curveto(833.98757873,690.82291683)(833.96757875,690.83791681)(833.9475815,690.85792498)
\curveto(833.92757879,690.87791677)(833.90257881,690.88291677)(833.8725815,690.87292498)
\curveto(833.8125789,690.8529168)(833.77757894,690.81291684)(833.7675815,690.75292498)
\curveto(833.75757896,690.69291696)(833.74257897,690.63291702)(833.7225815,690.57292498)
\lineto(833.7225815,690.46792498)
\curveto(833.70257901,690.39791725)(833.69757902,690.31791733)(833.7075815,690.22792498)
\curveto(833.717579,690.1479175)(833.72257899,690.06791758)(833.7225815,689.98792498)
\lineto(833.7225815,688.99792498)
\lineto(833.7225815,684.22792498)
\lineto(833.7225815,683.52292498)
\lineto(833.7225815,683.34292498)
\curveto(833.73257898,683.27292438)(833.72757899,683.21292444)(833.7075815,683.16292498)
\lineto(833.7075815,683.04292498)
\curveto(833.68757903,682.94292471)(833.66757905,682.87292478)(833.6475815,682.83292498)
\curveto(833.62757909,682.78292487)(833.59257912,682.7479249)(833.5425815,682.72792498)
\curveto(833.49257922,682.71792493)(833.43757928,682.70292495)(833.3775815,682.68292498)
\lineto(833.0775815,682.68292498)
\curveto(832.93757978,682.68292497)(832.8125799,682.68792496)(832.7025815,682.69792498)
\curveto(832.59258012,682.70792494)(832.5125802,682.7529249)(832.4625815,682.83292498)
\curveto(832.4125803,682.89292476)(832.38758033,682.97292468)(832.3875815,683.07292498)
\lineto(832.3875815,683.40292498)
\lineto(832.3875815,684.61792498)
\lineto(832.3875815,690.88792498)
\lineto(832.3875815,692.50792498)
\lineto(832.3875815,692.88292498)
\curveto(832.38758033,693.02291463)(832.4125803,693.13291452)(832.4625815,693.21292498)
\curveto(832.49258022,693.26291439)(832.55258016,693.30791434)(832.6425815,693.34792498)
\curveto(832.66258005,693.35791429)(832.68758003,693.35791429)(832.7175815,693.34792498)
\curveto(832.75757996,693.3479143)(832.78757993,693.3529143)(832.8075815,693.36292498)
}
}
{
\newrgbcolor{curcolor}{0 0 0}
\pscustom[linestyle=none,fillstyle=solid,fillcolor=curcolor]
{
\newpath
\moveto(850.06812837,686.88292498)
\curveto(850.08812031,686.82292083)(850.0981203,686.72792092)(850.09812837,686.59792498)
\curveto(850.0981203,686.47792117)(850.09312031,686.39292126)(850.08312837,686.34292498)
\lineto(850.08312837,686.19292498)
\curveto(850.07312033,686.11292154)(850.06312034,686.03792161)(850.05312837,685.96792498)
\curveto(850.05312035,685.90792174)(850.04812035,685.83792181)(850.03812837,685.75792498)
\curveto(850.01812038,685.69792195)(850.0031204,685.63792201)(849.99312837,685.57792498)
\curveto(849.99312041,685.51792213)(849.98312042,685.45792219)(849.96312837,685.39792498)
\curveto(849.92312048,685.26792238)(849.88812051,685.13792251)(849.85812837,685.00792498)
\curveto(849.82812057,684.87792277)(849.78812061,684.75792289)(849.73812837,684.64792498)
\curveto(849.52812087,684.16792348)(849.24812115,683.76292389)(848.89812837,683.43292498)
\curveto(848.54812185,683.11292454)(848.11812228,682.86792478)(847.60812837,682.69792498)
\curveto(847.4981229,682.65792499)(847.37812302,682.62792502)(847.24812837,682.60792498)
\curveto(847.12812327,682.58792506)(847.0031234,682.56792508)(846.87312837,682.54792498)
\curveto(846.81312359,682.53792511)(846.74812365,682.53292512)(846.67812837,682.53292498)
\curveto(846.61812378,682.52292513)(846.55812384,682.51792513)(846.49812837,682.51792498)
\curveto(846.45812394,682.50792514)(846.398124,682.50292515)(846.31812837,682.50292498)
\curveto(846.24812415,682.50292515)(846.1981242,682.50792514)(846.16812837,682.51792498)
\curveto(846.12812427,682.52792512)(846.08812431,682.53292512)(846.04812837,682.53292498)
\curveto(846.00812439,682.52292513)(845.97312443,682.52292513)(845.94312837,682.53292498)
\lineto(845.85312837,682.53292498)
\lineto(845.49312837,682.57792498)
\curveto(845.35312505,682.61792503)(845.21812518,682.65792499)(845.08812837,682.69792498)
\curveto(844.95812544,682.73792491)(844.83312557,682.78292487)(844.71312837,682.83292498)
\curveto(844.26312614,683.03292462)(843.89312651,683.29292436)(843.60312837,683.61292498)
\curveto(843.31312709,683.93292372)(843.07312733,684.32292333)(842.88312837,684.78292498)
\curveto(842.83312757,684.88292277)(842.79312761,684.98292267)(842.76312837,685.08292498)
\curveto(842.74312766,685.18292247)(842.72312768,685.28792236)(842.70312837,685.39792498)
\curveto(842.68312772,685.43792221)(842.67312773,685.46792218)(842.67312837,685.48792498)
\curveto(842.68312772,685.51792213)(842.68312772,685.5529221)(842.67312837,685.59292498)
\curveto(842.65312775,685.67292198)(842.63812776,685.7529219)(842.62812837,685.83292498)
\curveto(842.62812777,685.92292173)(842.61812778,686.00792164)(842.59812837,686.08792498)
\lineto(842.59812837,686.20792498)
\curveto(842.5981278,686.2479214)(842.59312781,686.29292136)(842.58312837,686.34292498)
\curveto(842.57312783,686.39292126)(842.56812783,686.47792117)(842.56812837,686.59792498)
\curveto(842.56812783,686.72792092)(842.57812782,686.82292083)(842.59812837,686.88292498)
\curveto(842.61812778,686.9529207)(842.62312778,687.02292063)(842.61312837,687.09292498)
\curveto(842.6031278,687.16292049)(842.60812779,687.23292042)(842.62812837,687.30292498)
\curveto(842.63812776,687.3529203)(842.64312776,687.39292026)(842.64312837,687.42292498)
\curveto(842.65312775,687.46292019)(842.66312774,687.50792014)(842.67312837,687.55792498)
\curveto(842.7031277,687.67791997)(842.72812767,687.79791985)(842.74812837,687.91792498)
\curveto(842.77812762,688.03791961)(842.81812758,688.1529195)(842.86812837,688.26292498)
\curveto(843.01812738,688.63291902)(843.1981272,688.96291869)(843.40812837,689.25292498)
\curveto(843.62812677,689.5529181)(843.89312651,689.80291785)(844.20312837,690.00292498)
\curveto(844.32312608,690.08291757)(844.44812595,690.1479175)(844.57812837,690.19792498)
\curveto(844.70812569,690.25791739)(844.84312556,690.31791733)(844.98312837,690.37792498)
\curveto(845.1031253,690.42791722)(845.23312517,690.45791719)(845.37312837,690.46792498)
\curveto(845.51312489,690.48791716)(845.65312475,690.51791713)(845.79312837,690.55792498)
\lineto(845.98812837,690.55792498)
\curveto(846.05812434,690.56791708)(846.12312428,690.57791707)(846.18312837,690.58792498)
\curveto(847.07312333,690.59791705)(847.81312259,690.41291724)(848.40312837,690.03292498)
\curveto(848.99312141,689.652918)(849.41812098,689.15791849)(849.67812837,688.54792498)
\curveto(849.72812067,688.4479192)(849.76812063,688.3479193)(849.79812837,688.24792498)
\curveto(849.82812057,688.1479195)(849.86312054,688.04291961)(849.90312837,687.93292498)
\curveto(849.93312047,687.82291983)(849.95812044,687.70291995)(849.97812837,687.57292498)
\curveto(849.9981204,687.4529202)(850.02312038,687.32792032)(850.05312837,687.19792498)
\curveto(850.06312034,687.1479205)(850.06312034,687.09292056)(850.05312837,687.03292498)
\curveto(850.05312035,686.98292067)(850.05812034,686.93292072)(850.06812837,686.88292498)
\moveto(848.73312837,686.02792498)
\curveto(848.75312165,686.09792155)(848.75812164,686.17792147)(848.74812837,686.26792498)
\lineto(848.74812837,686.52292498)
\curveto(848.74812165,686.91292074)(848.71312169,687.24292041)(848.64312837,687.51292498)
\curveto(848.61312179,687.59292006)(848.58812181,687.67291998)(848.56812837,687.75292498)
\curveto(848.54812185,687.83291982)(848.52312188,687.90791974)(848.49312837,687.97792498)
\curveto(848.21312219,688.62791902)(847.76812263,689.07791857)(847.15812837,689.32792498)
\curveto(847.08812331,689.35791829)(847.01312339,689.37791827)(846.93312837,689.38792498)
\lineto(846.69312837,689.44792498)
\curveto(846.61312379,689.46791818)(846.52812387,689.47791817)(846.43812837,689.47792498)
\lineto(846.16812837,689.47792498)
\lineto(845.89812837,689.43292498)
\curveto(845.7981246,689.41291824)(845.7031247,689.38791826)(845.61312837,689.35792498)
\curveto(845.53312487,689.33791831)(845.45312495,689.30791834)(845.37312837,689.26792498)
\curveto(845.3031251,689.2479184)(845.23812516,689.21791843)(845.17812837,689.17792498)
\curveto(845.11812528,689.13791851)(845.06312534,689.09791855)(845.01312837,689.05792498)
\curveto(844.77312563,688.88791876)(844.57812582,688.68291897)(844.42812837,688.44292498)
\curveto(844.27812612,688.20291945)(844.14812625,687.92291973)(844.03812837,687.60292498)
\curveto(844.00812639,687.50292015)(843.98812641,687.39792025)(843.97812837,687.28792498)
\curveto(843.96812643,687.18792046)(843.95312645,687.08292057)(843.93312837,686.97292498)
\curveto(843.92312648,686.93292072)(843.91812648,686.86792078)(843.91812837,686.77792498)
\curveto(843.90812649,686.7479209)(843.9031265,686.71292094)(843.90312837,686.67292498)
\curveto(843.91312649,686.63292102)(843.91812648,686.58792106)(843.91812837,686.53792498)
\lineto(843.91812837,686.23792498)
\curveto(843.91812648,686.13792151)(843.92812647,686.0479216)(843.94812837,685.96792498)
\lineto(843.97812837,685.78792498)
\curveto(843.9981264,685.68792196)(844.01312639,685.58792206)(844.02312837,685.48792498)
\curveto(844.04312636,685.39792225)(844.07312633,685.31292234)(844.11312837,685.23292498)
\curveto(844.21312619,684.99292266)(844.32812607,684.76792288)(844.45812837,684.55792498)
\curveto(844.5981258,684.3479233)(844.76812563,684.17292348)(844.96812837,684.03292498)
\curveto(845.01812538,684.00292365)(845.06312534,683.97792367)(845.10312837,683.95792498)
\curveto(845.14312526,683.93792371)(845.18812521,683.91292374)(845.23812837,683.88292498)
\curveto(845.31812508,683.83292382)(845.403125,683.78792386)(845.49312837,683.74792498)
\curveto(845.59312481,683.71792393)(845.6981247,683.68792396)(845.80812837,683.65792498)
\curveto(845.85812454,683.63792401)(845.9031245,683.62792402)(845.94312837,683.62792498)
\curveto(845.99312441,683.63792401)(846.04312436,683.63792401)(846.09312837,683.62792498)
\curveto(846.12312428,683.61792403)(846.18312422,683.60792404)(846.27312837,683.59792498)
\curveto(846.37312403,683.58792406)(846.44812395,683.59292406)(846.49812837,683.61292498)
\curveto(846.53812386,683.62292403)(846.57812382,683.62292403)(846.61812837,683.61292498)
\curveto(846.65812374,683.61292404)(846.6981237,683.62292403)(846.73812837,683.64292498)
\curveto(846.81812358,683.66292399)(846.8981235,683.67792397)(846.97812837,683.68792498)
\curveto(847.05812334,683.70792394)(847.13312327,683.73292392)(847.20312837,683.76292498)
\curveto(847.54312286,683.90292375)(847.81812258,684.09792355)(848.02812837,684.34792498)
\curveto(848.23812216,684.59792305)(848.41312199,684.89292276)(848.55312837,685.23292498)
\curveto(848.6031218,685.3529223)(848.63312177,685.47792217)(848.64312837,685.60792498)
\curveto(848.66312174,685.7479219)(848.69312171,685.88792176)(848.73312837,686.02792498)
}
}
{
\newrgbcolor{curcolor}{0 0 0}
\pscustom[linestyle=none,fillstyle=solid,fillcolor=curcolor]
{
\newpath
\moveto(852.50140962,692.74792498)
\curveto(852.65140761,692.7479149)(852.80140746,692.74291491)(852.95140962,692.73292498)
\curveto(853.10140716,692.73291492)(853.20640706,692.69291496)(853.26640962,692.61292498)
\curveto(853.31640695,692.5529151)(853.34140692,692.46791518)(853.34140962,692.35792498)
\curveto(853.35140691,692.25791539)(853.35640691,692.1529155)(853.35640962,692.04292498)
\lineto(853.35640962,691.17292498)
\curveto(853.35640691,691.09291656)(853.35140691,691.00791664)(853.34140962,690.91792498)
\curveto(853.34140692,690.83791681)(853.35140691,690.76791688)(853.37140962,690.70792498)
\curveto(853.41140685,690.56791708)(853.50140676,690.47791717)(853.64140962,690.43792498)
\curveto(853.69140657,690.42791722)(853.73640653,690.42291723)(853.77640962,690.42292498)
\lineto(853.92640962,690.42292498)
\lineto(854.33140962,690.42292498)
\curveto(854.49140577,690.43291722)(854.60640566,690.42291723)(854.67640962,690.39292498)
\curveto(854.7664055,690.33291732)(854.82640544,690.27291738)(854.85640962,690.21292498)
\curveto(854.87640539,690.17291748)(854.88640538,690.12791752)(854.88640962,690.07792498)
\lineto(854.88640962,689.92792498)
\curveto(854.88640538,689.81791783)(854.88140538,689.71291794)(854.87140962,689.61292498)
\curveto(854.8614054,689.52291813)(854.82640544,689.4529182)(854.76640962,689.40292498)
\curveto(854.70640556,689.3529183)(854.62140564,689.32291833)(854.51140962,689.31292498)
\lineto(854.18140962,689.31292498)
\curveto(854.07140619,689.32291833)(853.9614063,689.32791832)(853.85140962,689.32792498)
\curveto(853.74140652,689.32791832)(853.64640662,689.31291834)(853.56640962,689.28292498)
\curveto(853.49640677,689.2529184)(853.44640682,689.20291845)(853.41640962,689.13292498)
\curveto(853.38640688,689.06291859)(853.3664069,688.97791867)(853.35640962,688.87792498)
\curveto(853.34640692,688.78791886)(853.34140692,688.68791896)(853.34140962,688.57792498)
\curveto(853.35140691,688.47791917)(853.35640691,688.37791927)(853.35640962,688.27792498)
\lineto(853.35640962,685.30792498)
\curveto(853.35640691,685.08792256)(853.35140691,684.8529228)(853.34140962,684.60292498)
\curveto(853.34140692,684.36292329)(853.38640688,684.17792347)(853.47640962,684.04792498)
\curveto(853.52640674,683.96792368)(853.59140667,683.91292374)(853.67140962,683.88292498)
\curveto(853.75140651,683.8529238)(853.84640642,683.82792382)(853.95640962,683.80792498)
\curveto(853.98640628,683.79792385)(854.01640625,683.79292386)(854.04640962,683.79292498)
\curveto(854.08640618,683.80292385)(854.12140614,683.80292385)(854.15140962,683.79292498)
\lineto(854.34640962,683.79292498)
\curveto(854.44640582,683.79292386)(854.53640573,683.78292387)(854.61640962,683.76292498)
\curveto(854.70640556,683.7529239)(854.77140549,683.71792393)(854.81140962,683.65792498)
\curveto(854.83140543,683.62792402)(854.84640542,683.57292408)(854.85640962,683.49292498)
\curveto(854.87640539,683.42292423)(854.88640538,683.3479243)(854.88640962,683.26792498)
\curveto(854.89640537,683.18792446)(854.89640537,683.10792454)(854.88640962,683.02792498)
\curveto(854.87640539,682.95792469)(854.85640541,682.90292475)(854.82640962,682.86292498)
\curveto(854.78640548,682.79292486)(854.71140555,682.74292491)(854.60140962,682.71292498)
\curveto(854.52140574,682.69292496)(854.43140583,682.68292497)(854.33140962,682.68292498)
\curveto(854.23140603,682.69292496)(854.14140612,682.69792495)(854.06140962,682.69792498)
\curveto(854.00140626,682.69792495)(853.94140632,682.69292496)(853.88140962,682.68292498)
\curveto(853.82140644,682.68292497)(853.7664065,682.68792496)(853.71640962,682.69792498)
\lineto(853.53640962,682.69792498)
\curveto(853.48640678,682.70792494)(853.43640683,682.71292494)(853.38640962,682.71292498)
\curveto(853.34640692,682.72292493)(853.30140696,682.72792492)(853.25140962,682.72792498)
\curveto(853.05140721,682.77792487)(852.87640739,682.83292482)(852.72640962,682.89292498)
\curveto(852.58640768,682.9529247)(852.4664078,683.05792459)(852.36640962,683.20792498)
\curveto(852.22640804,683.40792424)(852.14640812,683.65792399)(852.12640962,683.95792498)
\curveto(852.10640816,684.26792338)(852.09640817,684.59792305)(852.09640962,684.94792498)
\lineto(852.09640962,688.87792498)
\curveto(852.0664082,689.00791864)(852.03640823,689.10291855)(852.00640962,689.16292498)
\curveto(851.98640828,689.22291843)(851.91640835,689.27291838)(851.79640962,689.31292498)
\curveto(851.75640851,689.32291833)(851.71640855,689.32291833)(851.67640962,689.31292498)
\curveto(851.63640863,689.30291835)(851.59640867,689.30791834)(851.55640962,689.32792498)
\lineto(851.31640962,689.32792498)
\curveto(851.18640908,689.32791832)(851.07640919,689.33791831)(850.98640962,689.35792498)
\curveto(850.90640936,689.38791826)(850.85140941,689.4479182)(850.82140962,689.53792498)
\curveto(850.80140946,689.57791807)(850.78640948,689.62291803)(850.77640962,689.67292498)
\lineto(850.77640962,689.82292498)
\curveto(850.77640949,689.96291769)(850.78640948,690.07791757)(850.80640962,690.16792498)
\curveto(850.82640944,690.26791738)(850.88640938,690.34291731)(850.98640962,690.39292498)
\curveto(851.09640917,690.43291722)(851.23640903,690.44291721)(851.40640962,690.42292498)
\curveto(851.58640868,690.40291725)(851.73640853,690.41291724)(851.85640962,690.45292498)
\curveto(851.94640832,690.50291715)(852.01640825,690.57291708)(852.06640962,690.66292498)
\curveto(852.08640818,690.72291693)(852.09640817,690.79791685)(852.09640962,690.88792498)
\lineto(852.09640962,691.14292498)
\lineto(852.09640962,692.07292498)
\lineto(852.09640962,692.31292498)
\curveto(852.09640817,692.40291525)(852.10640816,692.47791517)(852.12640962,692.53792498)
\curveto(852.1664081,692.61791503)(852.24140802,692.68291497)(852.35140962,692.73292498)
\curveto(852.38140788,692.73291492)(852.40640786,692.73291492)(852.42640962,692.73292498)
\curveto(852.45640781,692.74291491)(852.48140778,692.7479149)(852.50140962,692.74792498)
}
}
{
\newrgbcolor{curcolor}{0 0 0}
\pscustom[linestyle=none,fillstyle=solid,fillcolor=curcolor]
{
\newpath
\moveto(863.1582065,683.23792498)
\curveto(863.18819867,683.07792457)(863.17319868,682.94292471)(863.1132065,682.83292498)
\curveto(863.0531988,682.73292492)(862.97319888,682.65792499)(862.8732065,682.60792498)
\curveto(862.82319903,682.58792506)(862.76819909,682.57792507)(862.7082065,682.57792498)
\curveto(862.6581992,682.57792507)(862.60319925,682.56792508)(862.5432065,682.54792498)
\curveto(862.32319953,682.49792515)(862.10319975,682.51292514)(861.8832065,682.59292498)
\curveto(861.67320018,682.66292499)(861.52820033,682.7529249)(861.4482065,682.86292498)
\curveto(861.39820046,682.93292472)(861.3532005,683.01292464)(861.3132065,683.10292498)
\curveto(861.27320058,683.20292445)(861.22320063,683.28292437)(861.1632065,683.34292498)
\curveto(861.14320071,683.36292429)(861.11820074,683.38292427)(861.0882065,683.40292498)
\curveto(861.06820079,683.42292423)(861.03820082,683.42792422)(860.9982065,683.41792498)
\curveto(860.88820097,683.38792426)(860.78320107,683.33292432)(860.6832065,683.25292498)
\curveto(860.59320126,683.17292448)(860.50320135,683.10292455)(860.4132065,683.04292498)
\curveto(860.28320157,682.96292469)(860.14320171,682.88792476)(859.9932065,682.81792498)
\curveto(859.84320201,682.75792489)(859.68320217,682.70292495)(859.5132065,682.65292498)
\curveto(859.41320244,682.62292503)(859.30320255,682.60292505)(859.1832065,682.59292498)
\curveto(859.07320278,682.58292507)(858.96320289,682.56792508)(858.8532065,682.54792498)
\curveto(858.80320305,682.53792511)(858.7582031,682.53292512)(858.7182065,682.53292498)
\lineto(858.6132065,682.53292498)
\curveto(858.50320335,682.51292514)(858.39820346,682.51292514)(858.2982065,682.53292498)
\lineto(858.1632065,682.53292498)
\curveto(858.11320374,682.54292511)(858.06320379,682.5479251)(858.0132065,682.54792498)
\curveto(857.96320389,682.5479251)(857.91820394,682.55792509)(857.8782065,682.57792498)
\curveto(857.83820402,682.58792506)(857.80320405,682.59292506)(857.7732065,682.59292498)
\curveto(857.7532041,682.58292507)(857.72820413,682.58292507)(857.6982065,682.59292498)
\lineto(857.4582065,682.65292498)
\curveto(857.37820448,682.66292499)(857.30320455,682.68292497)(857.2332065,682.71292498)
\curveto(856.93320492,682.84292481)(856.68820517,682.98792466)(856.4982065,683.14792498)
\curveto(856.31820554,683.31792433)(856.16820569,683.5529241)(856.0482065,683.85292498)
\curveto(855.9582059,684.07292358)(855.91320594,684.33792331)(855.9132065,684.64792498)
\lineto(855.9132065,684.96292498)
\curveto(855.92320593,685.01292264)(855.92820593,685.06292259)(855.9282065,685.11292498)
\lineto(855.9582065,685.29292498)
\lineto(856.0782065,685.62292498)
\curveto(856.11820574,685.73292192)(856.16820569,685.83292182)(856.2282065,685.92292498)
\curveto(856.40820545,686.21292144)(856.6532052,686.42792122)(856.9632065,686.56792498)
\curveto(857.27320458,686.70792094)(857.61320424,686.83292082)(857.9832065,686.94292498)
\curveto(858.12320373,686.98292067)(858.26820359,687.01292064)(858.4182065,687.03292498)
\curveto(858.56820329,687.0529206)(858.71820314,687.07792057)(858.8682065,687.10792498)
\curveto(858.93820292,687.12792052)(859.00320285,687.13792051)(859.0632065,687.13792498)
\curveto(859.13320272,687.13792051)(859.20820265,687.1479205)(859.2882065,687.16792498)
\curveto(859.3582025,687.18792046)(859.42820243,687.19792045)(859.4982065,687.19792498)
\curveto(859.56820229,687.20792044)(859.64320221,687.22292043)(859.7232065,687.24292498)
\curveto(859.97320188,687.30292035)(860.20820165,687.3529203)(860.4282065,687.39292498)
\curveto(860.64820121,687.44292021)(860.82320103,687.55792009)(860.9532065,687.73792498)
\curveto(861.01320084,687.81791983)(861.06320079,687.91791973)(861.1032065,688.03792498)
\curveto(861.14320071,688.16791948)(861.14320071,688.30791934)(861.1032065,688.45792498)
\curveto(861.04320081,688.69791895)(860.9532009,688.88791876)(860.8332065,689.02792498)
\curveto(860.72320113,689.16791848)(860.56320129,689.27791837)(860.3532065,689.35792498)
\curveto(860.23320162,689.40791824)(860.08820177,689.44291821)(859.9182065,689.46292498)
\curveto(859.7582021,689.48291817)(859.58820227,689.49291816)(859.4082065,689.49292498)
\curveto(859.22820263,689.49291816)(859.0532028,689.48291817)(858.8832065,689.46292498)
\curveto(858.71320314,689.44291821)(858.56820329,689.41291824)(858.4482065,689.37292498)
\curveto(858.27820358,689.31291834)(858.11320374,689.22791842)(857.9532065,689.11792498)
\curveto(857.87320398,689.05791859)(857.79820406,688.97791867)(857.7282065,688.87792498)
\curveto(857.66820419,688.78791886)(857.61320424,688.68791896)(857.5632065,688.57792498)
\curveto(857.53320432,688.49791915)(857.50320435,688.41291924)(857.4732065,688.32292498)
\curveto(857.4532044,688.23291942)(857.40820445,688.16291949)(857.3382065,688.11292498)
\curveto(857.29820456,688.08291957)(857.22820463,688.05791959)(857.1282065,688.03792498)
\curveto(857.03820482,688.02791962)(856.94320491,688.02291963)(856.8432065,688.02292498)
\curveto(856.74320511,688.02291963)(856.64320521,688.02791962)(856.5432065,688.03792498)
\curveto(856.4532054,688.05791959)(856.38820547,688.08291957)(856.3482065,688.11292498)
\curveto(856.30820555,688.14291951)(856.27820558,688.19291946)(856.2582065,688.26292498)
\curveto(856.23820562,688.33291932)(856.23820562,688.40791924)(856.2582065,688.48792498)
\curveto(856.28820557,688.61791903)(856.31820554,688.73791891)(856.3482065,688.84792498)
\curveto(856.38820547,688.96791868)(856.43320542,689.08291857)(856.4832065,689.19292498)
\curveto(856.67320518,689.54291811)(856.91320494,689.81291784)(857.2032065,690.00292498)
\curveto(857.49320436,690.20291745)(857.853204,690.36291729)(858.2832065,690.48292498)
\curveto(858.38320347,690.50291715)(858.48320337,690.51791713)(858.5832065,690.52792498)
\curveto(858.69320316,690.53791711)(858.80320305,690.5529171)(858.9132065,690.57292498)
\curveto(858.9532029,690.58291707)(859.01820284,690.58291707)(859.1082065,690.57292498)
\curveto(859.19820266,690.57291708)(859.2532026,690.58291707)(859.2732065,690.60292498)
\curveto(859.97320188,690.61291704)(860.58320127,690.53291712)(861.1032065,690.36292498)
\curveto(861.62320023,690.19291746)(861.98819987,689.86791778)(862.1982065,689.38792498)
\curveto(862.28819957,689.18791846)(862.33819952,688.9529187)(862.3482065,688.68292498)
\curveto(862.36819949,688.42291923)(862.37819948,688.1479195)(862.3782065,687.85792498)
\lineto(862.3782065,684.54292498)
\curveto(862.37819948,684.40292325)(862.38319947,684.26792338)(862.3932065,684.13792498)
\curveto(862.40319945,684.00792364)(862.43319942,683.90292375)(862.4832065,683.82292498)
\curveto(862.53319932,683.7529239)(862.59819926,683.70292395)(862.6782065,683.67292498)
\curveto(862.76819909,683.63292402)(862.853199,683.60292405)(862.9332065,683.58292498)
\curveto(863.01319884,683.57292408)(863.07319878,683.52792412)(863.1132065,683.44792498)
\curveto(863.13319872,683.41792423)(863.14319871,683.38792426)(863.1432065,683.35792498)
\curveto(863.14319871,683.32792432)(863.14819871,683.28792436)(863.1582065,683.23792498)
\moveto(861.0132065,684.90292498)
\curveto(861.07320078,685.04292261)(861.10320075,685.20292245)(861.1032065,685.38292498)
\curveto(861.11320074,685.57292208)(861.11820074,685.76792188)(861.1182065,685.96792498)
\curveto(861.11820074,686.07792157)(861.11320074,686.17792147)(861.1032065,686.26792498)
\curveto(861.09320076,686.35792129)(861.0532008,686.42792122)(860.9832065,686.47792498)
\curveto(860.9532009,686.49792115)(860.88320097,686.50792114)(860.7732065,686.50792498)
\curveto(860.7532011,686.48792116)(860.71820114,686.47792117)(860.6682065,686.47792498)
\curveto(860.61820124,686.47792117)(860.57320128,686.46792118)(860.5332065,686.44792498)
\curveto(860.4532014,686.42792122)(860.36320149,686.40792124)(860.2632065,686.38792498)
\lineto(859.9632065,686.32792498)
\curveto(859.93320192,686.32792132)(859.89820196,686.32292133)(859.8582065,686.31292498)
\lineto(859.7532065,686.31292498)
\curveto(859.60320225,686.27292138)(859.43820242,686.2479214)(859.2582065,686.23792498)
\curveto(859.08820277,686.23792141)(858.92820293,686.21792143)(858.7782065,686.17792498)
\curveto(858.69820316,686.15792149)(858.62320323,686.13792151)(858.5532065,686.11792498)
\curveto(858.49320336,686.10792154)(858.42320343,686.09292156)(858.3432065,686.07292498)
\curveto(858.18320367,686.02292163)(858.03320382,685.95792169)(857.8932065,685.87792498)
\curveto(857.7532041,685.80792184)(857.63320422,685.71792193)(857.5332065,685.60792498)
\curveto(857.43320442,685.49792215)(857.3582045,685.36292229)(857.3082065,685.20292498)
\curveto(857.2582046,685.0529226)(857.23820462,684.86792278)(857.2482065,684.64792498)
\curveto(857.24820461,684.5479231)(857.26320459,684.4529232)(857.2932065,684.36292498)
\curveto(857.33320452,684.28292337)(857.37820448,684.20792344)(857.4282065,684.13792498)
\curveto(857.50820435,684.02792362)(857.61320424,683.93292372)(857.7432065,683.85292498)
\curveto(857.87320398,683.78292387)(858.01320384,683.72292393)(858.1632065,683.67292498)
\curveto(858.21320364,683.66292399)(858.26320359,683.65792399)(858.3132065,683.65792498)
\curveto(858.36320349,683.65792399)(858.41320344,683.652924)(858.4632065,683.64292498)
\curveto(858.53320332,683.62292403)(858.61820324,683.60792404)(858.7182065,683.59792498)
\curveto(858.82820303,683.59792405)(858.91820294,683.60792404)(858.9882065,683.62792498)
\curveto(859.04820281,683.647924)(859.10820275,683.652924)(859.1682065,683.64292498)
\curveto(859.22820263,683.64292401)(859.28820257,683.652924)(859.3482065,683.67292498)
\curveto(859.42820243,683.69292396)(859.50320235,683.70792394)(859.5732065,683.71792498)
\curveto(859.6532022,683.72792392)(859.72820213,683.7479239)(859.7982065,683.77792498)
\curveto(860.08820177,683.89792375)(860.33320152,684.04292361)(860.5332065,684.21292498)
\curveto(860.74320111,684.38292327)(860.90320095,684.61292304)(861.0132065,684.90292498)
}
}
{
\newrgbcolor{curcolor}{0 0 0}
\pscustom[linestyle=none,fillstyle=solid,fillcolor=curcolor]
{
\newpath
\moveto(866.75984712,690.58792498)
\curveto(867.47984306,690.59791705)(868.08484245,690.51291714)(868.57484712,690.33292498)
\curveto(869.06484147,690.16291749)(869.44484109,689.85791779)(869.71484712,689.41792498)
\curveto(869.78484075,689.30791834)(869.8398407,689.19291846)(869.87984712,689.07292498)
\curveto(869.91984062,688.96291869)(869.95984058,688.83791881)(869.99984712,688.69792498)
\curveto(870.01984052,688.62791902)(870.02484051,688.5529191)(870.01484712,688.47292498)
\curveto(870.00484053,688.40291925)(869.98984055,688.3479193)(869.96984712,688.30792498)
\curveto(869.94984059,688.28791936)(869.92484061,688.26791938)(869.89484712,688.24792498)
\curveto(869.86484067,688.23791941)(869.8398407,688.22291943)(869.81984712,688.20292498)
\curveto(869.76984077,688.18291947)(869.71984082,688.17791947)(869.66984712,688.18792498)
\curveto(869.61984092,688.19791945)(869.56984097,688.19791945)(869.51984712,688.18792498)
\curveto(869.4398411,688.16791948)(869.3348412,688.16291949)(869.20484712,688.17292498)
\curveto(869.07484146,688.19291946)(868.98484155,688.21791943)(868.93484712,688.24792498)
\curveto(868.85484168,688.29791935)(868.79984174,688.36291929)(868.76984712,688.44292498)
\curveto(868.74984179,688.53291912)(868.71484182,688.61791903)(868.66484712,688.69792498)
\curveto(868.57484196,688.85791879)(868.44984209,689.00291865)(868.28984712,689.13292498)
\curveto(868.17984236,689.21291844)(868.05984248,689.27291838)(867.92984712,689.31292498)
\curveto(867.79984274,689.3529183)(867.65984288,689.39291826)(867.50984712,689.43292498)
\curveto(867.45984308,689.4529182)(867.40984313,689.45791819)(867.35984712,689.44792498)
\curveto(867.30984323,689.4479182)(867.25984328,689.4529182)(867.20984712,689.46292498)
\curveto(867.14984339,689.48291817)(867.07484346,689.49291816)(866.98484712,689.49292498)
\curveto(866.89484364,689.49291816)(866.81984372,689.48291817)(866.75984712,689.46292498)
\lineto(866.66984712,689.46292498)
\lineto(866.51984712,689.43292498)
\curveto(866.46984407,689.43291822)(866.41984412,689.42791822)(866.36984712,689.41792498)
\curveto(866.10984443,689.35791829)(865.89484464,689.27291838)(865.72484712,689.16292498)
\curveto(865.55484498,689.0529186)(865.4398451,688.86791878)(865.37984712,688.60792498)
\curveto(865.35984518,688.53791911)(865.35484518,688.46791918)(865.36484712,688.39792498)
\curveto(865.38484515,688.32791932)(865.40484513,688.26791938)(865.42484712,688.21792498)
\curveto(865.48484505,688.06791958)(865.55484498,687.95791969)(865.63484712,687.88792498)
\curveto(865.72484481,687.82791982)(865.8348447,687.75791989)(865.96484712,687.67792498)
\curveto(866.12484441,687.57792007)(866.30484423,687.50292015)(866.50484712,687.45292498)
\curveto(866.70484383,687.41292024)(866.90484363,687.36292029)(867.10484712,687.30292498)
\curveto(867.2348433,687.26292039)(867.36484317,687.23292042)(867.49484712,687.21292498)
\curveto(867.62484291,687.19292046)(867.75484278,687.16292049)(867.88484712,687.12292498)
\curveto(868.09484244,687.06292059)(868.29984224,687.00292065)(868.49984712,686.94292498)
\curveto(868.69984184,686.89292076)(868.89984164,686.82792082)(869.09984712,686.74792498)
\lineto(869.24984712,686.68792498)
\curveto(869.29984124,686.66792098)(869.34984119,686.64292101)(869.39984712,686.61292498)
\curveto(869.59984094,686.49292116)(869.77484076,686.35792129)(869.92484712,686.20792498)
\curveto(870.07484046,686.05792159)(870.19984034,685.86792178)(870.29984712,685.63792498)
\curveto(870.31984022,685.56792208)(870.3398402,685.47292218)(870.35984712,685.35292498)
\curveto(870.37984016,685.28292237)(870.38984015,685.20792244)(870.38984712,685.12792498)
\curveto(870.39984014,685.05792259)(870.40484013,684.97792267)(870.40484712,684.88792498)
\lineto(870.40484712,684.73792498)
\curveto(870.38484015,684.66792298)(870.37484016,684.59792305)(870.37484712,684.52792498)
\curveto(870.37484016,684.45792319)(870.36484017,684.38792326)(870.34484712,684.31792498)
\curveto(870.31484022,684.20792344)(870.27984026,684.10292355)(870.23984712,684.00292498)
\curveto(870.19984034,683.90292375)(870.15484038,683.81292384)(870.10484712,683.73292498)
\curveto(869.94484059,683.47292418)(869.7398408,683.26292439)(869.48984712,683.10292498)
\curveto(869.2398413,682.9529247)(868.95984158,682.82292483)(868.64984712,682.71292498)
\curveto(868.55984198,682.68292497)(868.46484207,682.66292499)(868.36484712,682.65292498)
\curveto(868.27484226,682.63292502)(868.18484235,682.60792504)(868.09484712,682.57792498)
\curveto(867.99484254,682.55792509)(867.89484264,682.5479251)(867.79484712,682.54792498)
\curveto(867.69484284,682.5479251)(867.59484294,682.53792511)(867.49484712,682.51792498)
\lineto(867.34484712,682.51792498)
\curveto(867.29484324,682.50792514)(867.22484331,682.50292515)(867.13484712,682.50292498)
\curveto(867.04484349,682.50292515)(866.97484356,682.50792514)(866.92484712,682.51792498)
\lineto(866.75984712,682.51792498)
\curveto(866.69984384,682.53792511)(866.6348439,682.5479251)(866.56484712,682.54792498)
\curveto(866.49484404,682.53792511)(866.4348441,682.54292511)(866.38484712,682.56292498)
\curveto(866.3348442,682.57292508)(866.26984427,682.57792507)(866.18984712,682.57792498)
\lineto(865.94984712,682.63792498)
\curveto(865.87984466,682.647925)(865.80484473,682.66792498)(865.72484712,682.69792498)
\curveto(865.41484512,682.79792485)(865.14484539,682.92292473)(864.91484712,683.07292498)
\curveto(864.68484585,683.22292443)(864.48484605,683.41792423)(864.31484712,683.65792498)
\curveto(864.22484631,683.78792386)(864.14984639,683.92292373)(864.08984712,684.06292498)
\curveto(864.02984651,684.20292345)(863.97484656,684.35792329)(863.92484712,684.52792498)
\curveto(863.90484663,684.58792306)(863.89484664,684.65792299)(863.89484712,684.73792498)
\curveto(863.90484663,684.82792282)(863.91984662,684.89792275)(863.93984712,684.94792498)
\curveto(863.96984657,684.98792266)(864.01984652,685.02792262)(864.08984712,685.06792498)
\curveto(864.1398464,685.08792256)(864.20984633,685.09792255)(864.29984712,685.09792498)
\curveto(864.38984615,685.10792254)(864.47984606,685.10792254)(864.56984712,685.09792498)
\curveto(864.65984588,685.08792256)(864.74484579,685.07292258)(864.82484712,685.05292498)
\curveto(864.91484562,685.04292261)(864.97484556,685.02792262)(865.00484712,685.00792498)
\curveto(865.07484546,684.95792269)(865.11984542,684.88292277)(865.13984712,684.78292498)
\curveto(865.16984537,684.69292296)(865.20484533,684.60792304)(865.24484712,684.52792498)
\curveto(865.34484519,684.30792334)(865.47984506,684.13792351)(865.64984712,684.01792498)
\curveto(865.76984477,683.92792372)(865.90484463,683.85792379)(866.05484712,683.80792498)
\curveto(866.20484433,683.75792389)(866.36484417,683.70792394)(866.53484712,683.65792498)
\lineto(866.84984712,683.61292498)
\lineto(866.93984712,683.61292498)
\curveto(867.00984353,683.59292406)(867.09984344,683.58292407)(867.20984712,683.58292498)
\curveto(867.32984321,683.58292407)(867.42984311,683.59292406)(867.50984712,683.61292498)
\curveto(867.57984296,683.61292404)(867.6348429,683.61792403)(867.67484712,683.62792498)
\curveto(867.7348428,683.63792401)(867.79484274,683.64292401)(867.85484712,683.64292498)
\curveto(867.91484262,683.652924)(867.96984257,683.66292399)(868.01984712,683.67292498)
\curveto(868.30984223,683.7529239)(868.539842,683.85792379)(868.70984712,683.98792498)
\curveto(868.87984166,684.11792353)(868.99984154,684.33792331)(869.06984712,684.64792498)
\curveto(869.08984145,684.69792295)(869.09484144,684.7529229)(869.08484712,684.81292498)
\curveto(869.07484146,684.87292278)(869.06484147,684.91792273)(869.05484712,684.94792498)
\curveto(869.00484153,685.13792251)(868.9348416,685.27792237)(868.84484712,685.36792498)
\curveto(868.75484178,685.46792218)(868.6398419,685.55792209)(868.49984712,685.63792498)
\curveto(868.40984213,685.69792195)(868.30984223,685.7479219)(868.19984712,685.78792498)
\lineto(867.86984712,685.90792498)
\curveto(867.8398427,685.91792173)(867.80984273,685.92292173)(867.77984712,685.92292498)
\curveto(867.75984278,685.92292173)(867.7348428,685.93292172)(867.70484712,685.95292498)
\curveto(867.36484317,686.06292159)(867.00984353,686.14292151)(866.63984712,686.19292498)
\curveto(866.27984426,686.2529214)(865.9398446,686.3479213)(865.61984712,686.47792498)
\curveto(865.51984502,686.51792113)(865.42484511,686.5529211)(865.33484712,686.58292498)
\curveto(865.24484529,686.61292104)(865.15984538,686.652921)(865.07984712,686.70292498)
\curveto(864.88984565,686.81292084)(864.71484582,686.93792071)(864.55484712,687.07792498)
\curveto(864.39484614,687.21792043)(864.26984627,687.39292026)(864.17984712,687.60292498)
\curveto(864.14984639,687.67291998)(864.12484641,687.74291991)(864.10484712,687.81292498)
\curveto(864.09484644,687.88291977)(864.07984646,687.95791969)(864.05984712,688.03792498)
\curveto(864.02984651,688.15791949)(864.01984652,688.29291936)(864.02984712,688.44292498)
\curveto(864.0398465,688.60291905)(864.05484648,688.73791891)(864.07484712,688.84792498)
\curveto(864.09484644,688.89791875)(864.10484643,688.93791871)(864.10484712,688.96792498)
\curveto(864.11484642,689.00791864)(864.12984641,689.0479186)(864.14984712,689.08792498)
\curveto(864.2398463,689.31791833)(864.35984618,689.51791813)(864.50984712,689.68792498)
\curveto(864.66984587,689.85791779)(864.84984569,690.00791764)(865.04984712,690.13792498)
\curveto(865.19984534,690.22791742)(865.36484517,690.29791735)(865.54484712,690.34792498)
\curveto(865.72484481,690.40791724)(865.91484462,690.46291719)(866.11484712,690.51292498)
\curveto(866.18484435,690.52291713)(866.24984429,690.53291712)(866.30984712,690.54292498)
\curveto(866.37984416,690.5529171)(866.45484408,690.56291709)(866.53484712,690.57292498)
\curveto(866.56484397,690.58291707)(866.60484393,690.58291707)(866.65484712,690.57292498)
\curveto(866.70484383,690.56291709)(866.7398438,690.56791708)(866.75984712,690.58792498)
}
}
{
\newrgbcolor{curcolor}{0.90196079 0.90196079 0.90196079}
\pscustom[linestyle=none,fillstyle=solid,fillcolor=curcolor]
{
\newpath
\moveto(812.80437349,693.3929616)
\lineto(827.80437349,693.3929616)
\lineto(827.80437349,678.3929616)
\lineto(812.80437349,678.3929616)
\closepath
}
}
{
\newrgbcolor{curcolor}{0 0 0}
\pscustom[linestyle=none,fillstyle=solid,fillcolor=curcolor]
{
\newpath
\moveto(840.7425815,660.42721942)
\curveto(840.76257195,660.37721867)(840.78757193,660.31721873)(840.8175815,660.24721942)
\curveto(840.84757187,660.17721887)(840.86757185,660.10221895)(840.8775815,660.02221942)
\curveto(840.89757182,659.9522191)(840.89757182,659.88221917)(840.8775815,659.81221942)
\curveto(840.86757185,659.7522193)(840.82757189,659.70721934)(840.7575815,659.67721942)
\curveto(840.70757201,659.65721939)(840.64757207,659.6472194)(840.5775815,659.64721942)
\lineto(840.3675815,659.64721942)
\lineto(839.9175815,659.64721942)
\curveto(839.76757295,659.6472194)(839.64757307,659.67221938)(839.5575815,659.72221942)
\curveto(839.45757326,659.78221927)(839.38257333,659.88721916)(839.3325815,660.03721942)
\curveto(839.29257342,660.18721886)(839.24757347,660.32221873)(839.1975815,660.44221942)
\curveto(839.08757363,660.70221835)(838.98757373,660.97221808)(838.8975815,661.25221942)
\curveto(838.80757391,661.53221752)(838.70757401,661.80721724)(838.5975815,662.07721942)
\curveto(838.56757415,662.16721688)(838.53757418,662.2522168)(838.5075815,662.33221942)
\curveto(838.48757423,662.41221664)(838.45757426,662.48721656)(838.4175815,662.55721942)
\curveto(838.38757433,662.62721642)(838.34257437,662.68721636)(838.2825815,662.73721942)
\curveto(838.22257449,662.78721626)(838.14257457,662.82721622)(838.0425815,662.85721942)
\curveto(837.99257472,662.87721617)(837.93257478,662.88221617)(837.8625815,662.87221942)
\lineto(837.6675815,662.87221942)
\lineto(834.8325815,662.87221942)
\lineto(834.5325815,662.87221942)
\curveto(834.42257829,662.88221617)(834.3175784,662.88221617)(834.2175815,662.87221942)
\curveto(834.1175786,662.86221619)(834.02257869,662.8472162)(833.9325815,662.82721942)
\curveto(833.85257886,662.80721624)(833.79257892,662.76721628)(833.7525815,662.70721942)
\curveto(833.67257904,662.60721644)(833.6125791,662.49221656)(833.5725815,662.36221942)
\curveto(833.54257917,662.24221681)(833.50257921,662.11721693)(833.4525815,661.98721942)
\curveto(833.35257936,661.75721729)(833.25757946,661.51721753)(833.1675815,661.26721942)
\curveto(833.08757963,661.01721803)(832.99757972,660.77721827)(832.8975815,660.54721942)
\curveto(832.87757984,660.48721856)(832.85257986,660.41721863)(832.8225815,660.33721942)
\curveto(832.80257991,660.26721878)(832.77757994,660.19221886)(832.7475815,660.11221942)
\curveto(832.71758,660.03221902)(832.68258003,659.95721909)(832.6425815,659.88721942)
\curveto(832.6125801,659.82721922)(832.57758014,659.78221927)(832.5375815,659.75221942)
\curveto(832.45758026,659.69221936)(832.34758037,659.65721939)(832.2075815,659.64721942)
\lineto(831.7875815,659.64721942)
\lineto(831.5475815,659.64721942)
\curveto(831.47758124,659.65721939)(831.4175813,659.68221937)(831.3675815,659.72221942)
\curveto(831.3175814,659.7522193)(831.28758143,659.79721925)(831.2775815,659.85721942)
\curveto(831.27758144,659.91721913)(831.28258143,659.97721907)(831.2925815,660.03721942)
\curveto(831.3125814,660.10721894)(831.33258138,660.17221888)(831.3525815,660.23221942)
\curveto(831.38258133,660.30221875)(831.40758131,660.3522187)(831.4275815,660.38221942)
\curveto(831.56758115,660.70221835)(831.69258102,661.01721803)(831.8025815,661.32721942)
\curveto(831.9125808,661.6472174)(832.03258068,661.96721708)(832.1625815,662.28721942)
\curveto(832.25258046,662.50721654)(832.33758038,662.72221633)(832.4175815,662.93221942)
\curveto(832.49758022,663.1522159)(832.58258013,663.37221568)(832.6725815,663.59221942)
\curveto(832.97257974,664.31221474)(833.25757946,665.03721401)(833.5275815,665.76721942)
\curveto(833.79757892,666.50721254)(834.08257863,667.24221181)(834.3825815,667.97221942)
\curveto(834.49257822,668.23221082)(834.59257812,668.49721055)(834.6825815,668.76721942)
\curveto(834.78257793,669.03721001)(834.88757783,669.30220975)(834.9975815,669.56221942)
\curveto(835.04757767,669.67220938)(835.09257762,669.79220926)(835.1325815,669.92221942)
\curveto(835.18257753,670.06220899)(835.25257746,670.16220889)(835.3425815,670.22221942)
\curveto(835.38257733,670.26220879)(835.44757727,670.29220876)(835.5375815,670.31221942)
\curveto(835.55757716,670.32220873)(835.57757714,670.32220873)(835.5975815,670.31221942)
\curveto(835.62757709,670.31220874)(835.65257706,670.31720873)(835.6725815,670.32721942)
\curveto(835.85257686,670.32720872)(836.06257665,670.32720872)(836.3025815,670.32721942)
\curveto(836.54257617,670.33720871)(836.717576,670.30220875)(836.8275815,670.22221942)
\curveto(836.90757581,670.16220889)(836.96757575,670.06220899)(837.0075815,669.92221942)
\curveto(837.05757566,669.79220926)(837.10757561,669.67220938)(837.1575815,669.56221942)
\curveto(837.25757546,669.33220972)(837.34757537,669.10220995)(837.4275815,668.87221942)
\curveto(837.50757521,668.64221041)(837.59757512,668.41221064)(837.6975815,668.18221942)
\curveto(837.77757494,667.98221107)(837.85257486,667.77721127)(837.9225815,667.56721942)
\curveto(838.00257471,667.35721169)(838.08757463,667.1522119)(838.1775815,666.95221942)
\curveto(838.47757424,666.22221283)(838.76257395,665.48221357)(839.0325815,664.73221942)
\curveto(839.3125734,663.99221506)(839.60757311,663.25721579)(839.9175815,662.52721942)
\curveto(839.95757276,662.43721661)(839.98757273,662.3522167)(840.0075815,662.27221942)
\curveto(840.03757268,662.19221686)(840.06757265,662.10721694)(840.0975815,662.01721942)
\curveto(840.20757251,661.75721729)(840.3125724,661.49221756)(840.4125815,661.22221942)
\curveto(840.52257219,660.9522181)(840.63257208,660.68721836)(840.7425815,660.42721942)
\moveto(837.5325815,664.07221942)
\curveto(837.62257509,664.10221495)(837.67757504,664.1522149)(837.6975815,664.22221942)
\curveto(837.72757499,664.29221476)(837.73257498,664.36721468)(837.7125815,664.44721942)
\curveto(837.70257501,664.53721451)(837.67757504,664.62221443)(837.6375815,664.70221942)
\curveto(837.60757511,664.79221426)(837.57757514,664.86721418)(837.5475815,664.92721942)
\curveto(837.52757519,664.96721408)(837.5175752,665.00221405)(837.5175815,665.03221942)
\curveto(837.5175752,665.06221399)(837.50757521,665.09721395)(837.4875815,665.13721942)
\lineto(837.3975815,665.37721942)
\curveto(837.37757534,665.46721358)(837.34757537,665.55721349)(837.3075815,665.64721942)
\curveto(837.15757556,666.00721304)(837.02257569,666.37221268)(836.9025815,666.74221942)
\curveto(836.79257592,667.12221193)(836.66257605,667.49221156)(836.5125815,667.85221942)
\curveto(836.46257625,667.96221109)(836.4175763,668.07221098)(836.3775815,668.18221942)
\curveto(836.34757637,668.29221076)(836.30757641,668.39721065)(836.2575815,668.49721942)
\curveto(836.23757648,668.5472105)(836.2125765,668.59221046)(836.1825815,668.63221942)
\curveto(836.16257655,668.68221037)(836.1125766,668.70721034)(836.0325815,668.70721942)
\curveto(836.0125767,668.68721036)(835.99257672,668.67221038)(835.9725815,668.66221942)
\curveto(835.95257676,668.6522104)(835.93257678,668.63721041)(835.9125815,668.61721942)
\curveto(835.87257684,668.56721048)(835.84257687,668.51221054)(835.8225815,668.45221942)
\curveto(835.80257691,668.40221065)(835.78257693,668.3472107)(835.7625815,668.28721942)
\curveto(835.712577,668.17721087)(835.67257704,668.06721098)(835.6425815,667.95721942)
\curveto(835.6125771,667.8472112)(835.57257714,667.73721131)(835.5225815,667.62721942)
\curveto(835.35257736,667.23721181)(835.20257751,666.84221221)(835.0725815,666.44221942)
\curveto(834.95257776,666.04221301)(834.8125779,665.6522134)(834.6525815,665.27221942)
\lineto(834.5925815,665.12221942)
\curveto(834.58257813,665.07221398)(834.56757815,665.02221403)(834.5475815,664.97221942)
\lineto(834.4575815,664.73221942)
\curveto(834.42757829,664.6522144)(834.40257831,664.57221448)(834.3825815,664.49221942)
\curveto(834.36257835,664.44221461)(834.35257836,664.38721466)(834.3525815,664.32721942)
\curveto(834.36257835,664.26721478)(834.37757834,664.21721483)(834.3975815,664.17721942)
\curveto(834.44757827,664.09721495)(834.55257816,664.052215)(834.7125815,664.04221942)
\lineto(835.1625815,664.04221942)
\lineto(836.7675815,664.04221942)
\curveto(836.87757584,664.04221501)(837.0125757,664.03721501)(837.1725815,664.02721942)
\curveto(837.33257538,664.02721502)(837.45257526,664.04221501)(837.5325815,664.07221942)
}
}
{
\newrgbcolor{curcolor}{0 0 0}
\pscustom[linestyle=none,fillstyle=solid,fillcolor=curcolor]
{
\newpath
\moveto(845.504144,667.55221942)
\curveto(845.73413921,667.5522115)(845.86413908,667.49221156)(845.894144,667.37221942)
\curveto(845.92413902,667.26221179)(845.939139,667.09721195)(845.939144,666.87721942)
\lineto(845.939144,666.59221942)
\curveto(845.939139,666.50221255)(845.91413903,666.42721262)(845.864144,666.36721942)
\curveto(845.80413914,666.28721276)(845.71913922,666.24221281)(845.609144,666.23221942)
\curveto(845.49913944,666.23221282)(845.38913955,666.21721283)(845.279144,666.18721942)
\curveto(845.1391398,666.15721289)(845.00413994,666.12721292)(844.874144,666.09721942)
\curveto(844.75414019,666.06721298)(844.6391403,666.02721302)(844.529144,665.97721942)
\curveto(844.2391407,665.8472132)(844.00414094,665.66721338)(843.824144,665.43721942)
\curveto(843.6441413,665.21721383)(843.48914145,664.96221409)(843.359144,664.67221942)
\curveto(843.31914162,664.56221449)(843.28914165,664.4472146)(843.269144,664.32721942)
\curveto(843.24914169,664.21721483)(843.22414172,664.10221495)(843.194144,663.98221942)
\curveto(843.18414176,663.93221512)(843.17914176,663.88221517)(843.179144,663.83221942)
\curveto(843.18914175,663.78221527)(843.18914175,663.73221532)(843.179144,663.68221942)
\curveto(843.14914179,663.56221549)(843.13414181,663.42221563)(843.134144,663.26221942)
\curveto(843.1441418,663.11221594)(843.14914179,662.96721608)(843.149144,662.82721942)
\lineto(843.149144,660.98221942)
\lineto(843.149144,660.63721942)
\curveto(843.14914179,660.51721853)(843.1441418,660.40221865)(843.134144,660.29221942)
\curveto(843.12414182,660.18221887)(843.11914182,660.08721896)(843.119144,660.00721942)
\curveto(843.12914181,659.92721912)(843.10914183,659.85721919)(843.059144,659.79721942)
\curveto(843.00914193,659.72721932)(842.92914201,659.68721936)(842.819144,659.67721942)
\curveto(842.71914222,659.66721938)(842.60914233,659.66221939)(842.489144,659.66221942)
\lineto(842.219144,659.66221942)
\curveto(842.16914277,659.68221937)(842.11914282,659.69721935)(842.069144,659.70721942)
\curveto(842.02914291,659.72721932)(841.99914294,659.7522193)(841.979144,659.78221942)
\curveto(841.92914301,659.8522192)(841.89914304,659.93721911)(841.889144,660.03721942)
\lineto(841.889144,660.36721942)
\lineto(841.889144,661.52221942)
\lineto(841.889144,665.67721942)
\lineto(841.889144,666.71221942)
\lineto(841.889144,667.01221942)
\curveto(841.89914304,667.11221194)(841.92914301,667.19721185)(841.979144,667.26721942)
\curveto(842.00914293,667.30721174)(842.05914288,667.33721171)(842.129144,667.35721942)
\curveto(842.20914273,667.37721167)(842.29414265,667.38721166)(842.384144,667.38721942)
\curveto(842.47414247,667.39721165)(842.56414238,667.39721165)(842.654144,667.38721942)
\curveto(842.7441422,667.37721167)(842.81414213,667.36221169)(842.864144,667.34221942)
\curveto(842.944142,667.31221174)(842.99414195,667.2522118)(843.014144,667.16221942)
\curveto(843.0441419,667.08221197)(843.05914188,666.99221206)(843.059144,666.89221942)
\lineto(843.059144,666.59221942)
\curveto(843.05914188,666.49221256)(843.07914186,666.40221265)(843.119144,666.32221942)
\curveto(843.12914181,666.30221275)(843.1391418,666.28721276)(843.149144,666.27721942)
\lineto(843.194144,666.23221942)
\curveto(843.30414164,666.23221282)(843.39414155,666.27721277)(843.464144,666.36721942)
\curveto(843.53414141,666.46721258)(843.59414135,666.5472125)(843.644144,666.60721942)
\lineto(843.734144,666.69721942)
\curveto(843.82414112,666.80721224)(843.94914099,666.92221213)(844.109144,667.04221942)
\curveto(844.26914067,667.16221189)(844.41914052,667.2522118)(844.559144,667.31221942)
\curveto(844.64914029,667.36221169)(844.7441402,667.39721165)(844.844144,667.41721942)
\curveto(844.94414,667.4472116)(845.04913989,667.47721157)(845.159144,667.50721942)
\curveto(845.21913972,667.51721153)(845.27913966,667.52221153)(845.339144,667.52221942)
\curveto(845.39913954,667.53221152)(845.45413949,667.54221151)(845.504144,667.55221942)
}
}
{
\newrgbcolor{curcolor}{0 0 0}
\pscustom[linestyle=none,fillstyle=solid,fillcolor=curcolor]
{
\newpath
\moveto(850.00390962,667.55221942)
\curveto(850.74390483,667.56221149)(851.35890422,667.4522116)(851.84890962,667.22221942)
\curveto(852.34890323,667.00221205)(852.74390283,666.66721238)(853.03390962,666.21721942)
\curveto(853.16390241,666.01721303)(853.2739023,665.77221328)(853.36390962,665.48221942)
\curveto(853.38390219,665.43221362)(853.39890218,665.36721368)(853.40890962,665.28721942)
\curveto(853.41890216,665.20721384)(853.41390216,665.13721391)(853.39390962,665.07721942)
\curveto(853.36390221,665.02721402)(853.31390226,664.98221407)(853.24390962,664.94221942)
\curveto(853.21390236,664.92221413)(853.18390239,664.91221414)(853.15390962,664.91221942)
\curveto(853.12390245,664.92221413)(853.08890249,664.92221413)(853.04890962,664.91221942)
\curveto(853.00890257,664.90221415)(852.96890261,664.89721415)(852.92890962,664.89721942)
\curveto(852.88890269,664.90721414)(852.84890273,664.91221414)(852.80890962,664.91221942)
\lineto(852.49390962,664.91221942)
\curveto(852.39390318,664.92221413)(852.30890327,664.9522141)(852.23890962,665.00221942)
\curveto(852.15890342,665.06221399)(852.10390347,665.1472139)(852.07390962,665.25721942)
\curveto(852.04390353,665.36721368)(852.00390357,665.46221359)(851.95390962,665.54221942)
\curveto(851.80390377,665.80221325)(851.60890397,666.00721304)(851.36890962,666.15721942)
\curveto(851.28890429,666.20721284)(851.20390437,666.2472128)(851.11390962,666.27721942)
\curveto(851.02390455,666.31721273)(850.92890465,666.3522127)(850.82890962,666.38221942)
\curveto(850.68890489,666.42221263)(850.50390507,666.44221261)(850.27390962,666.44221942)
\curveto(850.04390553,666.4522126)(849.85390572,666.43221262)(849.70390962,666.38221942)
\curveto(849.63390594,666.36221269)(849.56890601,666.3472127)(849.50890962,666.33721942)
\curveto(849.44890613,666.32721272)(849.38390619,666.31221274)(849.31390962,666.29221942)
\curveto(849.05390652,666.18221287)(848.82390675,666.03221302)(848.62390962,665.84221942)
\curveto(848.42390715,665.6522134)(848.26890731,665.42721362)(848.15890962,665.16721942)
\curveto(848.11890746,665.07721397)(848.08390749,664.98221407)(848.05390962,664.88221942)
\curveto(848.02390755,664.79221426)(847.99390758,664.69221436)(847.96390962,664.58221942)
\lineto(847.87390962,664.17721942)
\curveto(847.86390771,664.12721492)(847.85890772,664.07221498)(847.85890962,664.01221942)
\curveto(847.86890771,663.9522151)(847.86390771,663.89721515)(847.84390962,663.84721942)
\lineto(847.84390962,663.72721942)
\curveto(847.83390774,663.68721536)(847.82390775,663.62221543)(847.81390962,663.53221942)
\curveto(847.81390776,663.44221561)(847.82390775,663.37721567)(847.84390962,663.33721942)
\curveto(847.85390772,663.28721576)(847.85390772,663.23721581)(847.84390962,663.18721942)
\curveto(847.83390774,663.13721591)(847.83390774,663.08721596)(847.84390962,663.03721942)
\curveto(847.85390772,662.99721605)(847.85890772,662.92721612)(847.85890962,662.82721942)
\curveto(847.8789077,662.7472163)(847.89390768,662.66221639)(847.90390962,662.57221942)
\curveto(847.92390765,662.48221657)(847.94390763,662.39721665)(847.96390962,662.31721942)
\curveto(848.0739075,661.99721705)(848.19890738,661.71721733)(848.33890962,661.47721942)
\curveto(848.48890709,661.2472178)(848.69390688,661.047218)(848.95390962,660.87721942)
\curveto(849.04390653,660.82721822)(849.13390644,660.78221827)(849.22390962,660.74221942)
\curveto(849.32390625,660.70221835)(849.42890615,660.66221839)(849.53890962,660.62221942)
\curveto(849.58890599,660.61221844)(849.62890595,660.60721844)(849.65890962,660.60721942)
\curveto(849.68890589,660.60721844)(849.72890585,660.60221845)(849.77890962,660.59221942)
\curveto(849.80890577,660.58221847)(849.85890572,660.57721847)(849.92890962,660.57721942)
\lineto(850.09390962,660.57721942)
\curveto(850.09390548,660.56721848)(850.11390546,660.56221849)(850.15390962,660.56221942)
\curveto(850.1739054,660.57221848)(850.19890538,660.57221848)(850.22890962,660.56221942)
\curveto(850.25890532,660.56221849)(850.28890529,660.56721848)(850.31890962,660.57721942)
\curveto(850.38890519,660.59721845)(850.45390512,660.60221845)(850.51390962,660.59221942)
\curveto(850.58390499,660.59221846)(850.65390492,660.60221845)(850.72390962,660.62221942)
\curveto(850.98390459,660.70221835)(851.20890437,660.80221825)(851.39890962,660.92221942)
\curveto(851.58890399,661.052218)(851.74890383,661.21721783)(851.87890962,661.41721942)
\curveto(851.92890365,661.49721755)(851.9739036,661.58221747)(852.01390962,661.67221942)
\lineto(852.13390962,661.94221942)
\curveto(852.15390342,662.02221703)(852.1739034,662.09721695)(852.19390962,662.16721942)
\curveto(852.22390335,662.2472168)(852.2739033,662.31221674)(852.34390962,662.36221942)
\curveto(852.3739032,662.39221666)(852.43390314,662.41221664)(852.52390962,662.42221942)
\curveto(852.61390296,662.44221661)(852.70890287,662.4522166)(852.80890962,662.45221942)
\curveto(852.91890266,662.46221659)(853.01890256,662.46221659)(853.10890962,662.45221942)
\curveto(853.20890237,662.44221661)(853.2789023,662.42221663)(853.31890962,662.39221942)
\curveto(853.3789022,662.3522167)(853.41390216,662.29221676)(853.42390962,662.21221942)
\curveto(853.44390213,662.13221692)(853.44390213,662.047217)(853.42390962,661.95721942)
\curveto(853.3739022,661.80721724)(853.32390225,661.66221739)(853.27390962,661.52221942)
\curveto(853.23390234,661.39221766)(853.1789024,661.26221779)(853.10890962,661.13221942)
\curveto(852.95890262,660.83221822)(852.76890281,660.56721848)(852.53890962,660.33721942)
\curveto(852.31890326,660.10721894)(852.04890353,659.92221913)(851.72890962,659.78221942)
\curveto(851.64890393,659.74221931)(851.56390401,659.70721934)(851.47390962,659.67721942)
\curveto(851.38390419,659.65721939)(851.28890429,659.63221942)(851.18890962,659.60221942)
\curveto(851.0789045,659.56221949)(850.96890461,659.54221951)(850.85890962,659.54221942)
\curveto(850.74890483,659.53221952)(850.63890494,659.51721953)(850.52890962,659.49721942)
\curveto(850.48890509,659.47721957)(850.44890513,659.47221958)(850.40890962,659.48221942)
\curveto(850.36890521,659.49221956)(850.32890525,659.49221956)(850.28890962,659.48221942)
\lineto(850.15390962,659.48221942)
\lineto(849.91390962,659.48221942)
\curveto(849.84390573,659.47221958)(849.7789058,659.47721957)(849.71890962,659.49721942)
\lineto(849.64390962,659.49721942)
\lineto(849.28390962,659.54221942)
\curveto(849.15390642,659.58221947)(849.02890655,659.61721943)(848.90890962,659.64721942)
\curveto(848.78890679,659.67721937)(848.6739069,659.71721933)(848.56390962,659.76721942)
\curveto(848.20390737,659.92721912)(847.90390767,660.11721893)(847.66390962,660.33721942)
\curveto(847.43390814,660.55721849)(847.21890836,660.82721822)(847.01890962,661.14721942)
\curveto(846.96890861,661.22721782)(846.92390865,661.31721773)(846.88390962,661.41721942)
\lineto(846.76390962,661.71721942)
\curveto(846.71390886,661.82721722)(846.6789089,661.94221711)(846.65890962,662.06221942)
\curveto(846.63890894,662.18221687)(846.61390896,662.30221675)(846.58390962,662.42221942)
\curveto(846.573909,662.46221659)(846.56890901,662.50221655)(846.56890962,662.54221942)
\curveto(846.56890901,662.58221647)(846.56390901,662.62221643)(846.55390962,662.66221942)
\curveto(846.53390904,662.72221633)(846.52390905,662.78721626)(846.52390962,662.85721942)
\curveto(846.53390904,662.92721612)(846.52890905,662.99221606)(846.50890962,663.05221942)
\lineto(846.50890962,663.20221942)
\curveto(846.49890908,663.2522158)(846.49390908,663.32221573)(846.49390962,663.41221942)
\curveto(846.49390908,663.50221555)(846.49890908,663.57221548)(846.50890962,663.62221942)
\curveto(846.51890906,663.67221538)(846.51890906,663.71721533)(846.50890962,663.75721942)
\curveto(846.50890907,663.79721525)(846.51390906,663.83721521)(846.52390962,663.87721942)
\curveto(846.54390903,663.9472151)(846.54890903,664.01721503)(846.53890962,664.08721942)
\curveto(846.53890904,664.15721489)(846.54890903,664.22221483)(846.56890962,664.28221942)
\curveto(846.60890897,664.4522146)(846.64390893,664.62221443)(846.67390962,664.79221942)
\curveto(846.70390887,664.96221409)(846.74890883,665.12221393)(846.80890962,665.27221942)
\curveto(847.01890856,665.79221326)(847.2739083,666.21221284)(847.57390962,666.53221942)
\curveto(847.8739077,666.8522122)(848.28390729,667.11721193)(848.80390962,667.32721942)
\curveto(848.91390666,667.37721167)(849.03390654,667.41221164)(849.16390962,667.43221942)
\curveto(849.29390628,667.4522116)(849.42890615,667.47721157)(849.56890962,667.50721942)
\curveto(849.63890594,667.51721153)(849.70890587,667.52221153)(849.77890962,667.52221942)
\curveto(849.84890573,667.53221152)(849.92390565,667.54221151)(850.00390962,667.55221942)
}
}
{
\newrgbcolor{curcolor}{0 0 0}
\pscustom[linestyle=none,fillstyle=solid,fillcolor=curcolor]
{
\newpath
\moveto(855.39055025,670.31221942)
\curveto(855.53054873,670.31220874)(855.67554858,670.30720874)(855.82555025,670.29721942)
\curveto(855.98554827,670.29720875)(856.09554816,670.25720879)(856.15555025,670.17721942)
\curveto(856.20554805,670.10720894)(856.23054803,670.00220905)(856.23055025,669.86221942)
\lineto(856.23055025,669.47221942)
\lineto(856.23055025,667.88221942)
\lineto(856.23055025,667.43221942)
\curveto(856.23054803,667.39221166)(856.22554803,667.3522117)(856.21555025,667.31221942)
\curveto(856.21554804,667.27221178)(856.22054804,667.23221182)(856.23055025,667.19221942)
\curveto(856.24054802,667.16221189)(856.24054802,667.12721192)(856.23055025,667.08721942)
\curveto(856.23054803,667.047212)(856.23554802,667.01221204)(856.24555025,666.98221942)
\curveto(856.26554799,666.90221215)(856.27554798,666.82721222)(856.27555025,666.75721942)
\curveto(856.28554797,666.68721236)(856.34054792,666.6522124)(856.44055025,666.65221942)
\curveto(856.4605478,666.67221238)(856.48054778,666.68221237)(856.50055025,666.68221942)
\curveto(856.53054773,666.69221236)(856.5555477,666.70721234)(856.57555025,666.72721942)
\curveto(856.63554762,666.76721228)(856.69054757,666.80721224)(856.74055025,666.84721942)
\curveto(856.79054747,666.89721215)(856.84554741,666.9472121)(856.90555025,666.99721942)
\curveto(857.01554724,667.07721197)(857.13554712,667.1472119)(857.26555025,667.20721942)
\curveto(857.40554685,667.27721177)(857.54554671,667.33721171)(857.68555025,667.38721942)
\curveto(857.76554649,667.40721164)(857.85054641,667.42221163)(857.94055025,667.43221942)
\curveto(858.03054623,667.4522116)(858.11554614,667.47221158)(858.19555025,667.49221942)
\curveto(858.23554602,667.51221154)(858.27554598,667.51721153)(858.31555025,667.50721942)
\curveto(858.3555459,667.50721154)(858.40054586,667.51221154)(858.45055025,667.52221942)
\curveto(858.50054576,667.54221151)(858.58054568,667.5522115)(858.69055025,667.55221942)
\curveto(858.81054545,667.5522115)(858.89554536,667.54221151)(858.94555025,667.52221942)
\lineto(859.08055025,667.52221942)
\curveto(859.13054513,667.52221153)(859.18054508,667.51721153)(859.23055025,667.50721942)
\curveto(859.31054495,667.48721156)(859.39054487,667.47221158)(859.47055025,667.46221942)
\curveto(859.55054471,667.4522116)(859.63054463,667.43721161)(859.71055025,667.41721942)
\curveto(859.7605445,667.39721165)(859.80054446,667.38221167)(859.83055025,667.37221942)
\curveto(859.8605444,667.37221168)(859.90054436,667.36221169)(859.95055025,667.34221942)
\curveto(860.0605442,667.29221176)(860.16554409,667.23721181)(860.26555025,667.17721942)
\curveto(860.36554389,667.12721192)(860.46554379,667.06721198)(860.56555025,666.99721942)
\curveto(860.66554359,666.90721214)(860.7605435,666.80221225)(860.85055025,666.68221942)
\curveto(860.91054335,666.59221246)(860.96554329,666.50221255)(861.01555025,666.41221942)
\curveto(861.06554319,666.32221273)(861.11554314,666.22221283)(861.16555025,666.11221942)
\curveto(861.19554306,666.04221301)(861.21554304,665.97221308)(861.22555025,665.90221942)
\curveto(861.24554301,665.83221322)(861.26554299,665.75721329)(861.28555025,665.67721942)
\curveto(861.30554295,665.62721342)(861.31554294,665.57721347)(861.31555025,665.52721942)
\curveto(861.31554294,665.47721357)(861.32054294,665.42221363)(861.33055025,665.36221942)
\curveto(861.35054291,665.31221374)(861.3555429,665.26221379)(861.34555025,665.21221942)
\curveto(861.34554291,665.16221389)(861.3555429,665.11221394)(861.37555025,665.06221942)
\lineto(861.37555025,664.91221942)
\curveto(861.38554287,664.86221419)(861.38554287,664.80721424)(861.37555025,664.74721942)
\lineto(861.37555025,664.58221942)
\lineto(861.37555025,663.93721942)
\lineto(861.37555025,660.81721942)
\lineto(861.37555025,660.51721942)
\curveto(861.38554287,660.40721864)(861.38554287,660.29721875)(861.37555025,660.18721942)
\curveto(861.37554288,660.08721896)(861.36554289,659.99221906)(861.34555025,659.90221942)
\curveto(861.32554293,659.81221924)(861.29554296,659.7522193)(861.25555025,659.72221942)
\curveto(861.18554307,659.66221939)(861.0555432,659.63221942)(860.86555025,659.63221942)
\lineto(860.47555025,659.63221942)
\curveto(860.3555439,659.63221942)(860.26554399,659.67221938)(860.20555025,659.75221942)
\curveto(860.1555441,659.82221923)(860.13054413,659.90221915)(860.13055025,659.99221942)
\lineto(860.13055025,660.30721942)
\lineto(860.13055025,661.38721942)
\lineto(860.13055025,663.78721942)
\lineto(860.13055025,664.64221942)
\curveto(860.14054412,664.9522141)(860.11054415,665.21721383)(860.04055025,665.43721942)
\curveto(859.92054434,665.77721327)(859.72054454,666.02721302)(859.44055025,666.18721942)
\curveto(859.3605449,666.23721281)(859.27554498,666.27721277)(859.18555025,666.30721942)
\curveto(859.09554516,666.3472127)(858.99554526,666.37721267)(858.88555025,666.39721942)
\curveto(858.84554541,666.40721264)(858.78554547,666.41221264)(858.70555025,666.41221942)
\curveto(858.66554559,666.42221263)(858.61554564,666.43221262)(858.55555025,666.44221942)
\curveto(858.50554575,666.4522126)(858.4555458,666.4472126)(858.40555025,666.42721942)
\curveto(858.36554589,666.41721263)(858.32554593,666.41221264)(858.28555025,666.41221942)
\curveto(858.24554601,666.42221263)(858.20054606,666.42221263)(858.15055025,666.41221942)
\curveto(858.0605462,666.39221266)(857.96554629,666.37221268)(857.86555025,666.35221942)
\curveto(857.77554648,666.34221271)(857.69054657,666.31721273)(857.61055025,666.27721942)
\curveto(857.27054699,666.13721291)(857.00054726,665.9472131)(856.80055025,665.70721942)
\curveto(856.60054766,665.46721358)(856.44554781,665.16221389)(856.33555025,664.79221942)
\curveto(856.31554794,664.72221433)(856.30054796,664.6472144)(856.29055025,664.56721942)
\curveto(856.28054798,664.48721456)(856.26554799,664.40721464)(856.24555025,664.32721942)
\curveto(856.23554802,664.29721475)(856.23054803,664.26221479)(856.23055025,664.22221942)
\curveto(856.24054802,664.19221486)(856.24054802,664.16221489)(856.23055025,664.13221942)
\curveto(856.22054804,664.08221497)(856.22054804,664.03221502)(856.23055025,663.98221942)
\curveto(856.24054802,663.93221512)(856.24054802,663.88221517)(856.23055025,663.83221942)
\lineto(856.23055025,660.81721942)
\lineto(856.23055025,660.53221942)
\curveto(856.24054802,660.43221862)(856.24054802,660.33221872)(856.23055025,660.23221942)
\curveto(856.23054803,660.13221892)(856.22554803,660.03721901)(856.21555025,659.94721942)
\curveto(856.20554805,659.85721919)(856.18554807,659.79221926)(856.15555025,659.75221942)
\curveto(856.11554814,659.70221935)(856.06554819,659.67221938)(856.00555025,659.66221942)
\curveto(855.9555483,659.66221939)(855.89554836,659.6522194)(855.82555025,659.63221942)
\lineto(855.61555025,659.63221942)
\lineto(855.30055025,659.63221942)
\curveto(855.20054906,659.64221941)(855.12554913,659.67721937)(855.07555025,659.73721942)
\curveto(855.02554923,659.81721923)(854.99554926,659.91221914)(854.98555025,660.02221942)
\lineto(854.98555025,660.39721942)
\lineto(854.98555025,661.77721942)
\lineto(854.98555025,668.03221942)
\lineto(854.98555025,669.50221942)
\curveto(854.98554927,669.61220944)(854.98054928,669.72720932)(854.97055025,669.84721942)
\curveto(854.97054929,669.97720907)(854.99554926,670.07720897)(855.04555025,670.14721942)
\curveto(855.08554917,670.21720883)(855.1605491,670.26720878)(855.27055025,670.29721942)
\curveto(855.29054897,670.30720874)(855.31054895,670.30720874)(855.33055025,670.29721942)
\curveto(855.35054891,670.29720875)(855.37054889,670.30220875)(855.39055025,670.31221942)
}
}
{
\newrgbcolor{curcolor}{0 0 0}
\pscustom[linestyle=none,fillstyle=solid,fillcolor=curcolor]
{
\newpath
\moveto(863.56015962,668.87221942)
\curveto(863.4801585,668.93221012)(863.43515855,669.03721001)(863.42515962,669.18721942)
\lineto(863.42515962,669.65221942)
\lineto(863.42515962,669.90721942)
\curveto(863.42515856,669.99720905)(863.44015854,670.07220898)(863.47015962,670.13221942)
\curveto(863.51015847,670.21220884)(863.59015839,670.27220878)(863.71015962,670.31221942)
\curveto(863.73015825,670.32220873)(863.75015823,670.32220873)(863.77015962,670.31221942)
\curveto(863.80015818,670.31220874)(863.82515816,670.31720873)(863.84515962,670.32721942)
\curveto(864.01515797,670.32720872)(864.17515781,670.32220873)(864.32515962,670.31221942)
\curveto(864.47515751,670.30220875)(864.57515741,670.24220881)(864.62515962,670.13221942)
\curveto(864.65515733,670.07220898)(864.67015731,669.99720905)(864.67015962,669.90721942)
\lineto(864.67015962,669.65221942)
\curveto(864.67015731,669.47220958)(864.66515732,669.30220975)(864.65515962,669.14221942)
\curveto(864.65515733,668.98221007)(864.59015739,668.87721017)(864.46015962,668.82721942)
\curveto(864.41015757,668.80721024)(864.35515763,668.79721025)(864.29515962,668.79721942)
\lineto(864.13015962,668.79721942)
\lineto(863.81515962,668.79721942)
\curveto(863.71515827,668.79721025)(863.63015835,668.82221023)(863.56015962,668.87221942)
\moveto(864.67015962,660.36721942)
\lineto(864.67015962,660.05221942)
\curveto(864.6801573,659.9522191)(864.66015732,659.87221918)(864.61015962,659.81221942)
\curveto(864.5801574,659.7522193)(864.53515745,659.71221934)(864.47515962,659.69221942)
\curveto(864.41515757,659.68221937)(864.34515764,659.66721938)(864.26515962,659.64721942)
\lineto(864.04015962,659.64721942)
\curveto(863.91015807,659.6472194)(863.79515819,659.6522194)(863.69515962,659.66221942)
\curveto(863.60515838,659.68221937)(863.53515845,659.73221932)(863.48515962,659.81221942)
\curveto(863.44515854,659.87221918)(863.42515856,659.9472191)(863.42515962,660.03721942)
\lineto(863.42515962,660.32221942)
\lineto(863.42515962,666.66721942)
\lineto(863.42515962,666.98221942)
\curveto(863.42515856,667.09221196)(863.45015853,667.17721187)(863.50015962,667.23721942)
\curveto(863.53015845,667.28721176)(863.57015841,667.31721173)(863.62015962,667.32721942)
\curveto(863.67015831,667.33721171)(863.72515826,667.3522117)(863.78515962,667.37221942)
\curveto(863.80515818,667.37221168)(863.82515816,667.36721168)(863.84515962,667.35721942)
\curveto(863.87515811,667.35721169)(863.90015808,667.36221169)(863.92015962,667.37221942)
\curveto(864.05015793,667.37221168)(864.1801578,667.36721168)(864.31015962,667.35721942)
\curveto(864.45015753,667.35721169)(864.54515744,667.31721173)(864.59515962,667.23721942)
\curveto(864.64515734,667.17721187)(864.67015731,667.09721195)(864.67015962,666.99721942)
\lineto(864.67015962,666.71221942)
\lineto(864.67015962,660.36721942)
}
}
{
\newrgbcolor{curcolor}{0 0 0}
\pscustom[linestyle=none,fillstyle=solid,fillcolor=curcolor]
{
\newpath
\moveto(866.39000337,667.37221942)
\lineto(866.87000337,667.37221942)
\curveto(867.04000203,667.37221168)(867.1700019,667.34221171)(867.26000337,667.28221942)
\curveto(867.33000174,667.23221182)(867.3750017,667.16721188)(867.39500337,667.08721942)
\curveto(867.42500165,667.01721203)(867.45500162,666.94221211)(867.48500337,666.86221942)
\curveto(867.54500153,666.72221233)(867.59500148,666.58221247)(867.63500337,666.44221942)
\curveto(867.6750014,666.30221275)(867.72000135,666.16221289)(867.77000337,666.02221942)
\curveto(867.9700011,665.48221357)(868.15500092,664.93721411)(868.32500337,664.38721942)
\curveto(868.49500058,663.8472152)(868.68000039,663.30721574)(868.88000337,662.76721942)
\curveto(868.95000012,662.58721646)(869.01000006,662.40221665)(869.06000337,662.21221942)
\curveto(869.10999996,662.03221702)(869.1749999,661.8522172)(869.25500337,661.67221942)
\curveto(869.2749998,661.60221745)(869.29999977,661.52721752)(869.33000337,661.44721942)
\curveto(869.35999971,661.36721768)(869.40999966,661.31721773)(869.48000337,661.29721942)
\curveto(869.55999951,661.27721777)(869.61999945,661.31221774)(869.66000337,661.40221942)
\curveto(869.70999936,661.49221756)(869.74499933,661.56221749)(869.76500337,661.61221942)
\curveto(869.84499923,661.80221725)(869.90999916,661.99221706)(869.96000337,662.18221942)
\curveto(870.01999905,662.38221667)(870.08499899,662.58221647)(870.15500337,662.78221942)
\curveto(870.28499879,663.16221589)(870.40999866,663.53721551)(870.53000337,663.90721942)
\curveto(870.64999842,664.28721476)(870.7749983,664.66721438)(870.90500337,665.04721942)
\curveto(870.95499812,665.21721383)(871.00499807,665.38221367)(871.05500337,665.54221942)
\curveto(871.10499797,665.71221334)(871.16499791,665.87721317)(871.23500337,666.03721942)
\curveto(871.28499779,666.17721287)(871.32999774,666.31721273)(871.37000337,666.45721942)
\curveto(871.40999766,666.59721245)(871.45499762,666.73721231)(871.50500337,666.87721942)
\curveto(871.52499755,666.9472121)(871.54999752,667.01721203)(871.58000337,667.08721942)
\curveto(871.60999746,667.15721189)(871.64999742,667.21721183)(871.70000337,667.26721942)
\curveto(871.77999729,667.31721173)(871.8699972,667.3472117)(871.97000337,667.35721942)
\curveto(872.069997,667.36721168)(872.18999688,667.37221168)(872.33000337,667.37221942)
\curveto(872.39999667,667.37221168)(872.46499661,667.36721168)(872.52500337,667.35721942)
\curveto(872.58499649,667.35721169)(872.63999643,667.3472117)(872.69000337,667.32721942)
\curveto(872.77999629,667.28721176)(872.82499625,667.22221183)(872.82500337,667.13221942)
\curveto(872.83499624,667.04221201)(872.81999625,666.9522121)(872.78000337,666.86221942)
\curveto(872.71999635,666.69221236)(872.65999641,666.51721253)(872.60000337,666.33721942)
\curveto(872.53999653,666.15721289)(872.4699966,665.98221307)(872.39000337,665.81221942)
\curveto(872.3699967,665.76221329)(872.35499672,665.71221334)(872.34500337,665.66221942)
\curveto(872.33499674,665.62221343)(872.31999675,665.57721347)(872.30000337,665.52721942)
\curveto(872.21999685,665.35721369)(872.15499692,665.18221387)(872.10500337,665.00221942)
\curveto(872.05499702,664.82221423)(871.98999708,664.64221441)(871.91000337,664.46221942)
\curveto(871.85999721,664.33221472)(871.80999726,664.19721485)(871.76000337,664.05721942)
\curveto(871.71999735,663.92721512)(871.6699974,663.79721525)(871.61000337,663.66721942)
\curveto(871.43999763,663.25721579)(871.28499779,662.84221621)(871.14500337,662.42221942)
\curveto(871.01499806,662.00221705)(870.86499821,661.58721746)(870.69500337,661.17721942)
\curveto(870.63499844,661.01721803)(870.57999849,660.85721819)(870.53000337,660.69721942)
\curveto(870.47999859,660.53721851)(870.41999865,660.37721867)(870.35000337,660.21721942)
\curveto(870.29999877,660.10721894)(870.25499882,660.00221905)(870.21500337,659.90221942)
\curveto(870.18499889,659.81221924)(870.11499896,659.74221931)(870.00500337,659.69221942)
\curveto(869.94499913,659.66221939)(869.8749992,659.6472194)(869.79500337,659.64721942)
\lineto(869.57000337,659.64721942)
\lineto(869.10500337,659.64721942)
\curveto(868.95500012,659.65721939)(868.84500023,659.70721934)(868.77500337,659.79721942)
\curveto(868.70500037,659.87721917)(868.65500042,659.97221908)(868.62500337,660.08221942)
\curveto(868.59500048,660.20221885)(868.55500052,660.31721873)(868.50500337,660.42721942)
\curveto(868.44500063,660.56721848)(868.38500069,660.71221834)(868.32500337,660.86221942)
\curveto(868.2750008,661.02221803)(868.22500085,661.17221788)(868.17500337,661.31221942)
\curveto(868.15500092,661.36221769)(868.14000093,661.40221765)(868.13000337,661.43221942)
\curveto(868.12000095,661.47221758)(868.10500097,661.51721753)(868.08500337,661.56721942)
\curveto(867.88500119,662.047217)(867.70000137,662.53221652)(867.53000337,663.02221942)
\curveto(867.3700017,663.51221554)(867.19000188,663.99721505)(866.99000337,664.47721942)
\curveto(866.93000214,664.63721441)(866.8700022,664.79221426)(866.81000337,664.94221942)
\curveto(866.76000231,665.10221395)(866.70500237,665.26221379)(866.64500337,665.42221942)
\lineto(866.58500337,665.57221942)
\curveto(866.5750025,665.63221342)(866.56000251,665.68721336)(866.54000337,665.73721942)
\curveto(866.46000261,665.90721314)(866.39000268,666.07721297)(866.33000337,666.24721942)
\curveto(866.28000279,666.41721263)(866.22000285,666.58721246)(866.15000337,666.75721942)
\curveto(866.13000294,666.81721223)(866.10500297,666.89721215)(866.07500337,666.99721942)
\curveto(866.04500303,667.09721195)(866.05000302,667.18221187)(866.09000337,667.25221942)
\curveto(866.14000293,667.30221175)(866.20000287,667.33721171)(866.27000337,667.35721942)
\curveto(866.34000273,667.35721169)(866.38000269,667.36221169)(866.39000337,667.37221942)
}
}
{
\newrgbcolor{curcolor}{0 0 0}
\pscustom[linestyle=none,fillstyle=solid,fillcolor=curcolor]
{
\newpath
\moveto(881.24000337,663.84721942)
\curveto(881.25999531,663.78721526)(881.2699953,663.69221536)(881.27000337,663.56221942)
\curveto(881.2699953,663.44221561)(881.26499531,663.35721569)(881.25500337,663.30721942)
\lineto(881.25500337,663.15721942)
\curveto(881.24499533,663.07721597)(881.23499534,663.00221605)(881.22500337,662.93221942)
\curveto(881.22499535,662.87221618)(881.21999535,662.80221625)(881.21000337,662.72221942)
\curveto(881.18999538,662.66221639)(881.1749954,662.60221645)(881.16500337,662.54221942)
\curveto(881.16499541,662.48221657)(881.15499542,662.42221663)(881.13500337,662.36221942)
\curveto(881.09499548,662.23221682)(881.05999551,662.10221695)(881.03000337,661.97221942)
\curveto(880.99999557,661.84221721)(880.95999561,661.72221733)(880.91000337,661.61221942)
\curveto(880.69999587,661.13221792)(880.41999615,660.72721832)(880.07000337,660.39721942)
\curveto(879.71999685,660.07721897)(879.28999728,659.83221922)(878.78000337,659.66221942)
\curveto(878.6699979,659.62221943)(878.54999802,659.59221946)(878.42000337,659.57221942)
\curveto(878.29999827,659.5522195)(878.1749984,659.53221952)(878.04500337,659.51221942)
\curveto(877.98499859,659.50221955)(877.91999865,659.49721955)(877.85000337,659.49721942)
\curveto(877.78999878,659.48721956)(877.72999884,659.48221957)(877.67000337,659.48221942)
\curveto(877.62999894,659.47221958)(877.569999,659.46721958)(877.49000337,659.46721942)
\curveto(877.41999915,659.46721958)(877.3699992,659.47221958)(877.34000337,659.48221942)
\curveto(877.29999927,659.49221956)(877.25999931,659.49721955)(877.22000337,659.49721942)
\curveto(877.17999939,659.48721956)(877.14499943,659.48721956)(877.11500337,659.49721942)
\lineto(877.02500337,659.49721942)
\lineto(876.66500337,659.54221942)
\curveto(876.52500005,659.58221947)(876.39000018,659.62221943)(876.26000337,659.66221942)
\curveto(876.13000044,659.70221935)(876.00500057,659.7472193)(875.88500337,659.79721942)
\curveto(875.43500114,659.99721905)(875.06500151,660.25721879)(874.77500337,660.57721942)
\curveto(874.48500209,660.89721815)(874.24500233,661.28721776)(874.05500337,661.74721942)
\curveto(874.00500257,661.8472172)(873.96500261,661.9472171)(873.93500337,662.04721942)
\curveto(873.91500266,662.1472169)(873.89500268,662.2522168)(873.87500337,662.36221942)
\curveto(873.85500272,662.40221665)(873.84500273,662.43221662)(873.84500337,662.45221942)
\curveto(873.85500272,662.48221657)(873.85500272,662.51721653)(873.84500337,662.55721942)
\curveto(873.82500275,662.63721641)(873.81000276,662.71721633)(873.80000337,662.79721942)
\curveto(873.80000277,662.88721616)(873.79000278,662.97221608)(873.77000337,663.05221942)
\lineto(873.77000337,663.17221942)
\curveto(873.7700028,663.21221584)(873.76500281,663.25721579)(873.75500337,663.30721942)
\curveto(873.74500283,663.35721569)(873.74000283,663.44221561)(873.74000337,663.56221942)
\curveto(873.74000283,663.69221536)(873.75000282,663.78721526)(873.77000337,663.84721942)
\curveto(873.79000278,663.91721513)(873.79500278,663.98721506)(873.78500337,664.05721942)
\curveto(873.7750028,664.12721492)(873.78000279,664.19721485)(873.80000337,664.26721942)
\curveto(873.81000276,664.31721473)(873.81500276,664.35721469)(873.81500337,664.38721942)
\curveto(873.82500275,664.42721462)(873.83500274,664.47221458)(873.84500337,664.52221942)
\curveto(873.8750027,664.64221441)(873.90000267,664.76221429)(873.92000337,664.88221942)
\curveto(873.95000262,665.00221405)(873.99000258,665.11721393)(874.04000337,665.22721942)
\curveto(874.19000238,665.59721345)(874.3700022,665.92721312)(874.58000337,666.21721942)
\curveto(874.80000177,666.51721253)(875.06500151,666.76721228)(875.37500337,666.96721942)
\curveto(875.49500108,667.047212)(875.62000095,667.11221194)(875.75000337,667.16221942)
\curveto(875.88000069,667.22221183)(876.01500056,667.28221177)(876.15500337,667.34221942)
\curveto(876.2750003,667.39221166)(876.40500017,667.42221163)(876.54500337,667.43221942)
\curveto(876.68499989,667.4522116)(876.82499975,667.48221157)(876.96500337,667.52221942)
\lineto(877.16000337,667.52221942)
\curveto(877.22999934,667.53221152)(877.29499928,667.54221151)(877.35500337,667.55221942)
\curveto(878.24499833,667.56221149)(878.98499759,667.37721167)(879.57500337,666.99721942)
\curveto(880.16499641,666.61721243)(880.58999598,666.12221293)(880.85000337,665.51221942)
\curveto(880.89999567,665.41221364)(880.93999563,665.31221374)(880.97000337,665.21221942)
\curveto(880.99999557,665.11221394)(881.03499554,665.00721404)(881.07500337,664.89721942)
\curveto(881.10499547,664.78721426)(881.12999544,664.66721438)(881.15000337,664.53721942)
\curveto(881.1699954,664.41721463)(881.19499538,664.29221476)(881.22500337,664.16221942)
\curveto(881.23499534,664.11221494)(881.23499534,664.05721499)(881.22500337,663.99721942)
\curveto(881.22499535,663.9472151)(881.22999534,663.89721515)(881.24000337,663.84721942)
\moveto(879.90500337,662.99221942)
\curveto(879.92499665,663.06221599)(879.92999664,663.14221591)(879.92000337,663.23221942)
\lineto(879.92000337,663.48721942)
\curveto(879.91999665,663.87721517)(879.88499669,664.20721484)(879.81500337,664.47721942)
\curveto(879.78499679,664.55721449)(879.75999681,664.63721441)(879.74000337,664.71721942)
\curveto(879.71999685,664.79721425)(879.69499688,664.87221418)(879.66500337,664.94221942)
\curveto(879.38499719,665.59221346)(878.93999763,666.04221301)(878.33000337,666.29221942)
\curveto(878.25999831,666.32221273)(878.18499839,666.34221271)(878.10500337,666.35221942)
\lineto(877.86500337,666.41221942)
\curveto(877.78499879,666.43221262)(877.69999887,666.44221261)(877.61000337,666.44221942)
\lineto(877.34000337,666.44221942)
\lineto(877.07000337,666.39721942)
\curveto(876.9699996,666.37721267)(876.8749997,666.3522127)(876.78500337,666.32221942)
\curveto(876.70499987,666.30221275)(876.62499995,666.27221278)(876.54500337,666.23221942)
\curveto(876.4750001,666.21221284)(876.41000016,666.18221287)(876.35000337,666.14221942)
\curveto(876.29000028,666.10221295)(876.23500034,666.06221299)(876.18500337,666.02221942)
\curveto(875.94500063,665.8522132)(875.75000082,665.6472134)(875.60000337,665.40721942)
\curveto(875.45000112,665.16721388)(875.32000125,664.88721416)(875.21000337,664.56721942)
\curveto(875.18000139,664.46721458)(875.16000141,664.36221469)(875.15000337,664.25221942)
\curveto(875.14000143,664.1522149)(875.12500145,664.047215)(875.10500337,663.93721942)
\curveto(875.09500148,663.89721515)(875.09000148,663.83221522)(875.09000337,663.74221942)
\curveto(875.08000149,663.71221534)(875.0750015,663.67721537)(875.07500337,663.63721942)
\curveto(875.08500149,663.59721545)(875.09000148,663.5522155)(875.09000337,663.50221942)
\lineto(875.09000337,663.20221942)
\curveto(875.09000148,663.10221595)(875.10000147,663.01221604)(875.12000337,662.93221942)
\lineto(875.15000337,662.75221942)
\curveto(875.1700014,662.6522164)(875.18500139,662.5522165)(875.19500337,662.45221942)
\curveto(875.21500136,662.36221669)(875.24500133,662.27721677)(875.28500337,662.19721942)
\curveto(875.38500119,661.95721709)(875.50000107,661.73221732)(875.63000337,661.52221942)
\curveto(875.7700008,661.31221774)(875.94000063,661.13721791)(876.14000337,660.99721942)
\curveto(876.19000038,660.96721808)(876.23500034,660.94221811)(876.27500337,660.92221942)
\curveto(876.31500026,660.90221815)(876.36000021,660.87721817)(876.41000337,660.84721942)
\curveto(876.49000008,660.79721825)(876.575,660.7522183)(876.66500337,660.71221942)
\curveto(876.76499981,660.68221837)(876.8699997,660.6522184)(876.98000337,660.62221942)
\curveto(877.02999954,660.60221845)(877.0749995,660.59221846)(877.11500337,660.59221942)
\curveto(877.16499941,660.60221845)(877.21499936,660.60221845)(877.26500337,660.59221942)
\curveto(877.29499928,660.58221847)(877.35499922,660.57221848)(877.44500337,660.56221942)
\curveto(877.54499903,660.5522185)(877.61999895,660.55721849)(877.67000337,660.57721942)
\curveto(877.70999886,660.58721846)(877.74999882,660.58721846)(877.79000337,660.57721942)
\curveto(877.82999874,660.57721847)(877.8699987,660.58721846)(877.91000337,660.60721942)
\curveto(877.98999858,660.62721842)(878.0699985,660.64221841)(878.15000337,660.65221942)
\curveto(878.22999834,660.67221838)(878.30499827,660.69721835)(878.37500337,660.72721942)
\curveto(878.71499786,660.86721818)(878.98999758,661.06221799)(879.20000337,661.31221942)
\curveto(879.40999716,661.56221749)(879.58499699,661.85721719)(879.72500337,662.19721942)
\curveto(879.7749968,662.31721673)(879.80499677,662.44221661)(879.81500337,662.57221942)
\curveto(879.83499674,662.71221634)(879.86499671,662.8522162)(879.90500337,662.99221942)
}
}
{
\newrgbcolor{curcolor}{0 0 0}
\pscustom[linestyle=none,fillstyle=solid,fillcolor=curcolor]
{
\newpath
\moveto(885.15828462,667.55221942)
\curveto(885.87828056,667.56221149)(886.48327995,667.47721157)(886.97328462,667.29721942)
\curveto(887.46327897,667.12721192)(887.84327859,666.82221223)(888.11328462,666.38221942)
\curveto(888.18327825,666.27221278)(888.2382782,666.15721289)(888.27828462,666.03721942)
\curveto(888.31827812,665.92721312)(888.35827808,665.80221325)(888.39828462,665.66221942)
\curveto(888.41827802,665.59221346)(888.42327801,665.51721353)(888.41328462,665.43721942)
\curveto(888.40327803,665.36721368)(888.38827805,665.31221374)(888.36828462,665.27221942)
\curveto(888.34827809,665.2522138)(888.32327811,665.23221382)(888.29328462,665.21221942)
\curveto(888.26327817,665.20221385)(888.2382782,665.18721386)(888.21828462,665.16721942)
\curveto(888.16827827,665.1472139)(888.11827832,665.14221391)(888.06828462,665.15221942)
\curveto(888.01827842,665.16221389)(887.96827847,665.16221389)(887.91828462,665.15221942)
\curveto(887.8382786,665.13221392)(887.7332787,665.12721392)(887.60328462,665.13721942)
\curveto(887.47327896,665.15721389)(887.38327905,665.18221387)(887.33328462,665.21221942)
\curveto(887.25327918,665.26221379)(887.19827924,665.32721372)(887.16828462,665.40721942)
\curveto(887.14827929,665.49721355)(887.11327932,665.58221347)(887.06328462,665.66221942)
\curveto(886.97327946,665.82221323)(886.84827959,665.96721308)(886.68828462,666.09721942)
\curveto(886.57827986,666.17721287)(886.45827998,666.23721281)(886.32828462,666.27721942)
\curveto(886.19828024,666.31721273)(886.05828038,666.35721269)(885.90828462,666.39721942)
\curveto(885.85828058,666.41721263)(885.80828063,666.42221263)(885.75828462,666.41221942)
\curveto(885.70828073,666.41221264)(885.65828078,666.41721263)(885.60828462,666.42721942)
\curveto(885.54828089,666.4472126)(885.47328096,666.45721259)(885.38328462,666.45721942)
\curveto(885.29328114,666.45721259)(885.21828122,666.4472126)(885.15828462,666.42721942)
\lineto(885.06828462,666.42721942)
\lineto(884.91828462,666.39721942)
\curveto(884.86828157,666.39721265)(884.81828162,666.39221266)(884.76828462,666.38221942)
\curveto(884.50828193,666.32221273)(884.29328214,666.23721281)(884.12328462,666.12721942)
\curveto(883.95328248,666.01721303)(883.8382826,665.83221322)(883.77828462,665.57221942)
\curveto(883.75828268,665.50221355)(883.75328268,665.43221362)(883.76328462,665.36221942)
\curveto(883.78328265,665.29221376)(883.80328263,665.23221382)(883.82328462,665.18221942)
\curveto(883.88328255,665.03221402)(883.95328248,664.92221413)(884.03328462,664.85221942)
\curveto(884.12328231,664.79221426)(884.2332822,664.72221433)(884.36328462,664.64221942)
\curveto(884.52328191,664.54221451)(884.70328173,664.46721458)(884.90328462,664.41721942)
\curveto(885.10328133,664.37721467)(885.30328113,664.32721472)(885.50328462,664.26721942)
\curveto(885.6332808,664.22721482)(885.76328067,664.19721485)(885.89328462,664.17721942)
\curveto(886.02328041,664.15721489)(886.15328028,664.12721492)(886.28328462,664.08721942)
\curveto(886.49327994,664.02721502)(886.69827974,663.96721508)(886.89828462,663.90721942)
\curveto(887.09827934,663.85721519)(887.29827914,663.79221526)(887.49828462,663.71221942)
\lineto(887.64828462,663.65221942)
\curveto(887.69827874,663.63221542)(887.74827869,663.60721544)(887.79828462,663.57721942)
\curveto(887.99827844,663.45721559)(888.17327826,663.32221573)(888.32328462,663.17221942)
\curveto(888.47327796,663.02221603)(888.59827784,662.83221622)(888.69828462,662.60221942)
\curveto(888.71827772,662.53221652)(888.7382777,662.43721661)(888.75828462,662.31721942)
\curveto(888.77827766,662.2472168)(888.78827765,662.17221688)(888.78828462,662.09221942)
\curveto(888.79827764,662.02221703)(888.80327763,661.94221711)(888.80328462,661.85221942)
\lineto(888.80328462,661.70221942)
\curveto(888.78327765,661.63221742)(888.77327766,661.56221749)(888.77328462,661.49221942)
\curveto(888.77327766,661.42221763)(888.76327767,661.3522177)(888.74328462,661.28221942)
\curveto(888.71327772,661.17221788)(888.67827776,661.06721798)(888.63828462,660.96721942)
\curveto(888.59827784,660.86721818)(888.55327788,660.77721827)(888.50328462,660.69721942)
\curveto(888.34327809,660.43721861)(888.1382783,660.22721882)(887.88828462,660.06721942)
\curveto(887.6382788,659.91721913)(887.35827908,659.78721926)(887.04828462,659.67721942)
\curveto(886.95827948,659.6472194)(886.86327957,659.62721942)(886.76328462,659.61721942)
\curveto(886.67327976,659.59721945)(886.58327985,659.57221948)(886.49328462,659.54221942)
\curveto(886.39328004,659.52221953)(886.29328014,659.51221954)(886.19328462,659.51221942)
\curveto(886.09328034,659.51221954)(885.99328044,659.50221955)(885.89328462,659.48221942)
\lineto(885.74328462,659.48221942)
\curveto(885.69328074,659.47221958)(885.62328081,659.46721958)(885.53328462,659.46721942)
\curveto(885.44328099,659.46721958)(885.37328106,659.47221958)(885.32328462,659.48221942)
\lineto(885.15828462,659.48221942)
\curveto(885.09828134,659.50221955)(885.0332814,659.51221954)(884.96328462,659.51221942)
\curveto(884.89328154,659.50221955)(884.8332816,659.50721954)(884.78328462,659.52721942)
\curveto(884.7332817,659.53721951)(884.66828177,659.54221951)(884.58828462,659.54221942)
\lineto(884.34828462,659.60221942)
\curveto(884.27828216,659.61221944)(884.20328223,659.63221942)(884.12328462,659.66221942)
\curveto(883.81328262,659.76221929)(883.54328289,659.88721916)(883.31328462,660.03721942)
\curveto(883.08328335,660.18721886)(882.88328355,660.38221867)(882.71328462,660.62221942)
\curveto(882.62328381,660.7522183)(882.54828389,660.88721816)(882.48828462,661.02721942)
\curveto(882.42828401,661.16721788)(882.37328406,661.32221773)(882.32328462,661.49221942)
\curveto(882.30328413,661.5522175)(882.29328414,661.62221743)(882.29328462,661.70221942)
\curveto(882.30328413,661.79221726)(882.31828412,661.86221719)(882.33828462,661.91221942)
\curveto(882.36828407,661.9522171)(882.41828402,661.99221706)(882.48828462,662.03221942)
\curveto(882.5382839,662.052217)(882.60828383,662.06221699)(882.69828462,662.06221942)
\curveto(882.78828365,662.07221698)(882.87828356,662.07221698)(882.96828462,662.06221942)
\curveto(883.05828338,662.052217)(883.14328329,662.03721701)(883.22328462,662.01721942)
\curveto(883.31328312,662.00721704)(883.37328306,661.99221706)(883.40328462,661.97221942)
\curveto(883.47328296,661.92221713)(883.51828292,661.8472172)(883.53828462,661.74721942)
\curveto(883.56828287,661.65721739)(883.60328283,661.57221748)(883.64328462,661.49221942)
\curveto(883.74328269,661.27221778)(883.87828256,661.10221795)(884.04828462,660.98221942)
\curveto(884.16828227,660.89221816)(884.30328213,660.82221823)(884.45328462,660.77221942)
\curveto(884.60328183,660.72221833)(884.76328167,660.67221838)(884.93328462,660.62221942)
\lineto(885.24828462,660.57721942)
\lineto(885.33828462,660.57721942)
\curveto(885.40828103,660.55721849)(885.49828094,660.5472185)(885.60828462,660.54721942)
\curveto(885.72828071,660.5472185)(885.82828061,660.55721849)(885.90828462,660.57721942)
\curveto(885.97828046,660.57721847)(886.0332804,660.58221847)(886.07328462,660.59221942)
\curveto(886.1332803,660.60221845)(886.19328024,660.60721844)(886.25328462,660.60721942)
\curveto(886.31328012,660.61721843)(886.36828007,660.62721842)(886.41828462,660.63721942)
\curveto(886.70827973,660.71721833)(886.9382795,660.82221823)(887.10828462,660.95221942)
\curveto(887.27827916,661.08221797)(887.39827904,661.30221775)(887.46828462,661.61221942)
\curveto(887.48827895,661.66221739)(887.49327894,661.71721733)(887.48328462,661.77721942)
\curveto(887.47327896,661.83721721)(887.46327897,661.88221717)(887.45328462,661.91221942)
\curveto(887.40327903,662.10221695)(887.3332791,662.24221681)(887.24328462,662.33221942)
\curveto(887.15327928,662.43221662)(887.0382794,662.52221653)(886.89828462,662.60221942)
\curveto(886.80827963,662.66221639)(886.70827973,662.71221634)(886.59828462,662.75221942)
\lineto(886.26828462,662.87221942)
\curveto(886.2382802,662.88221617)(886.20828023,662.88721616)(886.17828462,662.88721942)
\curveto(886.15828028,662.88721616)(886.1332803,662.89721615)(886.10328462,662.91721942)
\curveto(885.76328067,663.02721602)(885.40828103,663.10721594)(885.03828462,663.15721942)
\curveto(884.67828176,663.21721583)(884.3382821,663.31221574)(884.01828462,663.44221942)
\curveto(883.91828252,663.48221557)(883.82328261,663.51721553)(883.73328462,663.54721942)
\curveto(883.64328279,663.57721547)(883.55828288,663.61721543)(883.47828462,663.66721942)
\curveto(883.28828315,663.77721527)(883.11328332,663.90221515)(882.95328462,664.04221942)
\curveto(882.79328364,664.18221487)(882.66828377,664.35721469)(882.57828462,664.56721942)
\curveto(882.54828389,664.63721441)(882.52328391,664.70721434)(882.50328462,664.77721942)
\curveto(882.49328394,664.8472142)(882.47828396,664.92221413)(882.45828462,665.00221942)
\curveto(882.42828401,665.12221393)(882.41828402,665.25721379)(882.42828462,665.40721942)
\curveto(882.438284,665.56721348)(882.45328398,665.70221335)(882.47328462,665.81221942)
\curveto(882.49328394,665.86221319)(882.50328393,665.90221315)(882.50328462,665.93221942)
\curveto(882.51328392,665.97221308)(882.52828391,666.01221304)(882.54828462,666.05221942)
\curveto(882.6382838,666.28221277)(882.75828368,666.48221257)(882.90828462,666.65221942)
\curveto(883.06828337,666.82221223)(883.24828319,666.97221208)(883.44828462,667.10221942)
\curveto(883.59828284,667.19221186)(883.76328267,667.26221179)(883.94328462,667.31221942)
\curveto(884.12328231,667.37221168)(884.31328212,667.42721162)(884.51328462,667.47721942)
\curveto(884.58328185,667.48721156)(884.64828179,667.49721155)(884.70828462,667.50721942)
\curveto(884.77828166,667.51721153)(884.85328158,667.52721152)(884.93328462,667.53721942)
\curveto(884.96328147,667.5472115)(885.00328143,667.5472115)(885.05328462,667.53721942)
\curveto(885.10328133,667.52721152)(885.1382813,667.53221152)(885.15828462,667.55221942)
}
}
{
\newrgbcolor{curcolor}{0.80000001 0.80000001 0.80000001}
\pscustom[linestyle=none,fillstyle=solid,fillcolor=curcolor]
{
\newpath
\moveto(812.80437349,670.35725604)
\lineto(827.80437349,670.35725604)
\lineto(827.80437349,655.35725604)
\lineto(812.80437349,655.35725604)
\closepath
}
}
{
\newrgbcolor{curcolor}{0 0 0}
\pscustom[linestyle=none,fillstyle=solid,fillcolor=curcolor]
{
\newpath
\moveto(832.7925815,647.47004168)
\lineto(837.6975815,647.47004168)
\lineto(838.9875815,647.47004168)
\curveto(839.09757362,647.47003099)(839.20757351,647.47003099)(839.3175815,647.47004168)
\curveto(839.42757329,647.48003098)(839.5175732,647.460031)(839.5875815,647.41004168)
\curveto(839.6175731,647.39003107)(839.64257307,647.36503109)(839.6625815,647.33504168)
\curveto(839.68257303,647.30503115)(839.70257301,647.27503118)(839.7225815,647.24504168)
\curveto(839.74257297,647.17503128)(839.75257296,647.0600314)(839.7525815,646.90004168)
\curveto(839.75257296,646.75003171)(839.74257297,646.63503182)(839.7225815,646.55504168)
\curveto(839.68257303,646.41503204)(839.59757312,646.33503212)(839.4675815,646.31504168)
\curveto(839.33757338,646.30503215)(839.18257353,646.30003216)(839.0025815,646.30004168)
\lineto(837.5025815,646.30004168)
\lineto(834.9825815,646.30004168)
\lineto(834.4125815,646.30004168)
\curveto(834.20257851,646.31003215)(834.04757867,646.28503217)(833.9475815,646.22504168)
\curveto(833.84757887,646.16503229)(833.79257892,646.0600324)(833.7825815,645.91004168)
\lineto(833.7825815,645.44504168)
\lineto(833.7825815,643.91504168)
\curveto(833.78257893,643.80503465)(833.77757894,643.67503478)(833.7675815,643.52504168)
\curveto(833.76757895,643.37503508)(833.77757894,643.2550352)(833.7975815,643.16504168)
\curveto(833.82757889,643.04503541)(833.88757883,642.96503549)(833.9775815,642.92504168)
\curveto(834.0175787,642.90503555)(834.08757863,642.88503557)(834.1875815,642.86504168)
\lineto(834.3375815,642.86504168)
\curveto(834.37757834,642.8550356)(834.4175783,642.85003561)(834.4575815,642.85004168)
\curveto(834.50757821,642.8600356)(834.55757816,642.86503559)(834.6075815,642.86504168)
\lineto(835.1175815,642.86504168)
\lineto(838.0575815,642.86504168)
\lineto(838.3575815,642.86504168)
\curveto(838.46757425,642.87503558)(838.57757414,642.87503558)(838.6875815,642.86504168)
\curveto(838.80757391,642.86503559)(838.9125738,642.8550356)(839.0025815,642.83504168)
\curveto(839.10257361,642.82503563)(839.17757354,642.80503565)(839.2275815,642.77504168)
\curveto(839.25757346,642.7550357)(839.28257343,642.71003575)(839.3025815,642.64004168)
\curveto(839.32257339,642.57003589)(839.33757338,642.49503596)(839.3475815,642.41504168)
\curveto(839.35757336,642.33503612)(839.35757336,642.25003621)(839.3475815,642.16004168)
\curveto(839.34757337,642.08003638)(839.33757338,642.01003645)(839.3175815,641.95004168)
\curveto(839.29757342,641.8600366)(839.25257346,641.79503666)(839.1825815,641.75504168)
\curveto(839.16257355,641.73503672)(839.13257358,641.72003674)(839.0925815,641.71004168)
\curveto(839.06257365,641.71003675)(839.03257368,641.70503675)(839.0025815,641.69504168)
\lineto(838.9125815,641.69504168)
\curveto(838.86257385,641.68503677)(838.8125739,641.68003678)(838.7625815,641.68004168)
\curveto(838.712574,641.69003677)(838.66257405,641.69503676)(838.6125815,641.69504168)
\lineto(838.0575815,641.69504168)
\lineto(834.8925815,641.69504168)
\lineto(834.5325815,641.69504168)
\curveto(834.42257829,641.70503675)(834.3175784,641.70003676)(834.2175815,641.68004168)
\curveto(834.1175786,641.67003679)(834.02757869,641.64503681)(833.9475815,641.60504168)
\curveto(833.87757884,641.56503689)(833.82757889,641.49503696)(833.7975815,641.39504168)
\curveto(833.77757894,641.33503712)(833.76757895,641.26503719)(833.7675815,641.18504168)
\curveto(833.77757894,641.10503735)(833.78257893,641.02503743)(833.7825815,640.94504168)
\lineto(833.7825815,640.10504168)
\lineto(833.7825815,638.68004168)
\curveto(833.78257893,638.54003992)(833.78757893,638.41004005)(833.7975815,638.29004168)
\curveto(833.80757891,638.18004028)(833.84757887,638.10004036)(833.9175815,638.05004168)
\curveto(833.98757873,638.00004046)(834.06757865,637.97004049)(834.1575815,637.96004168)
\lineto(834.4575815,637.96004168)
\lineto(835.4175815,637.96004168)
\lineto(838.1925815,637.96004168)
\lineto(839.0475815,637.96004168)
\lineto(839.2875815,637.96004168)
\curveto(839.36757335,637.97004049)(839.43757328,637.96504049)(839.4975815,637.94504168)
\curveto(839.6175731,637.90504055)(839.69757302,637.85004061)(839.7375815,637.78004168)
\curveto(839.75757296,637.75004071)(839.77257294,637.70004076)(839.7825815,637.63004168)
\curveto(839.79257292,637.5600409)(839.79757292,637.48504097)(839.7975815,637.40504168)
\curveto(839.80757291,637.33504112)(839.80757291,637.2600412)(839.7975815,637.18004168)
\curveto(839.78757293,637.11004135)(839.77757294,637.0550414)(839.7675815,637.01504168)
\curveto(839.72757299,636.93504152)(839.68257303,636.88004158)(839.6325815,636.85004168)
\curveto(839.57257314,636.81004165)(839.49257322,636.79004167)(839.3925815,636.79004168)
\lineto(839.1225815,636.79004168)
\lineto(838.0725815,636.79004168)
\lineto(834.0825815,636.79004168)
\lineto(833.0325815,636.79004168)
\curveto(832.89257982,636.79004167)(832.77257994,636.79504166)(832.6725815,636.80504168)
\curveto(832.57258014,636.82504163)(832.49758022,636.87504158)(832.4475815,636.95504168)
\curveto(832.40758031,637.01504144)(832.38758033,637.09004137)(832.3875815,637.18004168)
\lineto(832.3875815,637.46504168)
\lineto(832.3875815,638.51504168)
\lineto(832.3875815,642.53504168)
\lineto(832.3875815,645.89504168)
\lineto(832.3875815,646.82504168)
\lineto(832.3875815,647.09504168)
\curveto(832.38758033,647.18503127)(832.40758031,647.2550312)(832.4475815,647.30504168)
\curveto(832.48758023,647.37503108)(832.56258015,647.42503103)(832.6725815,647.45504168)
\curveto(832.69258002,647.46503099)(832.71258,647.46503099)(832.7325815,647.45504168)
\curveto(832.75257996,647.455031)(832.77257994,647.460031)(832.7925815,647.47004168)
}
}
{
\newrgbcolor{curcolor}{0 0 0}
\pscustom[linestyle=none,fillstyle=solid,fillcolor=curcolor]
{
\newpath
\moveto(841.07750337,644.51504168)
\lineto(841.55750337,644.51504168)
\curveto(841.72750203,644.51503394)(841.8575019,644.48503397)(841.94750337,644.42504168)
\curveto(842.01750174,644.37503408)(842.0625017,644.31003415)(842.08250337,644.23004168)
\curveto(842.11250165,644.1600343)(842.14250162,644.08503437)(842.17250337,644.00504168)
\curveto(842.23250153,643.86503459)(842.28250148,643.72503473)(842.32250337,643.58504168)
\curveto(842.3625014,643.44503501)(842.40750135,643.30503515)(842.45750337,643.16504168)
\curveto(842.6575011,642.62503583)(842.84250092,642.08003638)(843.01250337,641.53004168)
\curveto(843.18250058,640.99003747)(843.36750039,640.45003801)(843.56750337,639.91004168)
\curveto(843.63750012,639.73003873)(843.69750006,639.54503891)(843.74750337,639.35504168)
\curveto(843.79749996,639.17503928)(843.8624999,638.99503946)(843.94250337,638.81504168)
\curveto(843.9624998,638.74503971)(843.98749977,638.67003979)(844.01750337,638.59004168)
\curveto(844.04749971,638.51003995)(844.09749966,638.46004)(844.16750337,638.44004168)
\curveto(844.24749951,638.42004004)(844.30749945,638.45504)(844.34750337,638.54504168)
\curveto(844.39749936,638.63503982)(844.43249933,638.70503975)(844.45250337,638.75504168)
\curveto(844.53249923,638.94503951)(844.59749916,639.13503932)(844.64750337,639.32504168)
\curveto(844.70749905,639.52503893)(844.77249899,639.72503873)(844.84250337,639.92504168)
\curveto(844.97249879,640.30503815)(845.09749866,640.68003778)(845.21750337,641.05004168)
\curveto(845.33749842,641.43003703)(845.4624983,641.81003665)(845.59250337,642.19004168)
\curveto(845.64249812,642.3600361)(845.69249807,642.52503593)(845.74250337,642.68504168)
\curveto(845.79249797,642.8550356)(845.85249791,643.02003544)(845.92250337,643.18004168)
\curveto(845.97249779,643.32003514)(846.01749774,643.460035)(846.05750337,643.60004168)
\curveto(846.09749766,643.74003472)(846.14249762,643.88003458)(846.19250337,644.02004168)
\curveto(846.21249755,644.09003437)(846.23749752,644.1600343)(846.26750337,644.23004168)
\curveto(846.29749746,644.30003416)(846.33749742,644.3600341)(846.38750337,644.41004168)
\curveto(846.46749729,644.460034)(846.5574972,644.49003397)(846.65750337,644.50004168)
\curveto(846.757497,644.51003395)(846.87749688,644.51503394)(847.01750337,644.51504168)
\curveto(847.08749667,644.51503394)(847.15249661,644.51003395)(847.21250337,644.50004168)
\curveto(847.27249649,644.50003396)(847.32749643,644.49003397)(847.37750337,644.47004168)
\curveto(847.46749629,644.43003403)(847.51249625,644.36503409)(847.51250337,644.27504168)
\curveto(847.52249624,644.18503427)(847.50749625,644.09503436)(847.46750337,644.00504168)
\curveto(847.40749635,643.83503462)(847.34749641,643.6600348)(847.28750337,643.48004168)
\curveto(847.22749653,643.30003516)(847.1574966,643.12503533)(847.07750337,642.95504168)
\curveto(847.0574967,642.90503555)(847.04249672,642.8550356)(847.03250337,642.80504168)
\curveto(847.02249674,642.76503569)(847.00749675,642.72003574)(846.98750337,642.67004168)
\curveto(846.90749685,642.50003596)(846.84249692,642.32503613)(846.79250337,642.14504168)
\curveto(846.74249702,641.96503649)(846.67749708,641.78503667)(846.59750337,641.60504168)
\curveto(846.54749721,641.47503698)(846.49749726,641.34003712)(846.44750337,641.20004168)
\curveto(846.40749735,641.07003739)(846.3574974,640.94003752)(846.29750337,640.81004168)
\curveto(846.12749763,640.40003806)(845.97249779,639.98503847)(845.83250337,639.56504168)
\curveto(845.70249806,639.14503931)(845.55249821,638.73003973)(845.38250337,638.32004168)
\curveto(845.32249844,638.1600403)(845.26749849,638.00004046)(845.21750337,637.84004168)
\curveto(845.16749859,637.68004078)(845.10749865,637.52004094)(845.03750337,637.36004168)
\curveto(844.98749877,637.25004121)(844.94249882,637.14504131)(844.90250337,637.04504168)
\curveto(844.87249889,636.9550415)(844.80249896,636.88504157)(844.69250337,636.83504168)
\curveto(844.63249913,636.80504165)(844.5624992,636.79004167)(844.48250337,636.79004168)
\lineto(844.25750337,636.79004168)
\lineto(843.79250337,636.79004168)
\curveto(843.64250012,636.80004166)(843.53250023,636.85004161)(843.46250337,636.94004168)
\curveto(843.39250037,637.02004144)(843.34250042,637.11504134)(843.31250337,637.22504168)
\curveto(843.28250048,637.34504111)(843.24250052,637.460041)(843.19250337,637.57004168)
\curveto(843.13250063,637.71004075)(843.07250069,637.8550406)(843.01250337,638.00504168)
\curveto(842.9625008,638.16504029)(842.91250085,638.31504014)(842.86250337,638.45504168)
\curveto(842.84250092,638.50503995)(842.82750093,638.54503991)(842.81750337,638.57504168)
\curveto(842.80750095,638.61503984)(842.79250097,638.6600398)(842.77250337,638.71004168)
\curveto(842.57250119,639.19003927)(842.38750137,639.67503878)(842.21750337,640.16504168)
\curveto(842.0575017,640.6550378)(841.87750188,641.14003732)(841.67750337,641.62004168)
\curveto(841.61750214,641.78003668)(841.5575022,641.93503652)(841.49750337,642.08504168)
\curveto(841.44750231,642.24503621)(841.39250237,642.40503605)(841.33250337,642.56504168)
\lineto(841.27250337,642.71504168)
\curveto(841.2625025,642.77503568)(841.24750251,642.83003563)(841.22750337,642.88004168)
\curveto(841.14750261,643.05003541)(841.07750268,643.22003524)(841.01750337,643.39004168)
\curveto(840.96750279,643.5600349)(840.90750285,643.73003473)(840.83750337,643.90004168)
\curveto(840.81750294,643.9600345)(840.79250297,644.04003442)(840.76250337,644.14004168)
\curveto(840.73250303,644.24003422)(840.73750302,644.32503413)(840.77750337,644.39504168)
\curveto(840.82750293,644.44503401)(840.88750287,644.48003398)(840.95750337,644.50004168)
\curveto(841.02750273,644.50003396)(841.06750269,644.50503395)(841.07750337,644.51504168)
}
}
{
\newrgbcolor{curcolor}{0 0 0}
\pscustom[linestyle=none,fillstyle=solid,fillcolor=curcolor]
{
\newpath
\moveto(855.55250337,640.96004168)
\curveto(855.57249569,640.8600376)(855.57249569,640.74503771)(855.55250337,640.61504168)
\curveto(855.54249572,640.49503796)(855.51249575,640.41003805)(855.46250337,640.36004168)
\curveto(855.41249585,640.32003814)(855.33749592,640.29003817)(855.23750337,640.27004168)
\curveto(855.14749611,640.2600382)(855.04249622,640.2550382)(854.92250337,640.25504168)
\lineto(854.56250337,640.25504168)
\curveto(854.44249682,640.26503819)(854.33749692,640.27003819)(854.24750337,640.27004168)
\lineto(850.40750337,640.27004168)
\curveto(850.32750093,640.27003819)(850.24750101,640.26503819)(850.16750337,640.25504168)
\curveto(850.08750117,640.2550382)(850.02250124,640.24003822)(849.97250337,640.21004168)
\curveto(849.93250133,640.19003827)(849.89250137,640.15003831)(849.85250337,640.09004168)
\curveto(849.83250143,640.0600384)(849.81250145,640.01503844)(849.79250337,639.95504168)
\curveto(849.77250149,639.90503855)(849.77250149,639.8550386)(849.79250337,639.80504168)
\curveto(849.80250146,639.7550387)(849.80750145,639.71003875)(849.80750337,639.67004168)
\curveto(849.80750145,639.63003883)(849.81250145,639.59003887)(849.82250337,639.55004168)
\curveto(849.84250142,639.47003899)(849.8625014,639.38503907)(849.88250337,639.29504168)
\curveto(849.90250136,639.21503924)(849.93250133,639.13503932)(849.97250337,639.05504168)
\curveto(850.20250106,638.51503994)(850.58250068,638.13004033)(851.11250337,637.90004168)
\curveto(851.17250009,637.87004059)(851.23750002,637.84504061)(851.30750337,637.82504168)
\lineto(851.51750337,637.76504168)
\curveto(851.54749971,637.7550407)(851.59749966,637.75004071)(851.66750337,637.75004168)
\curveto(851.80749945,637.71004075)(851.99249927,637.69004077)(852.22250337,637.69004168)
\curveto(852.45249881,637.69004077)(852.63749862,637.71004075)(852.77750337,637.75004168)
\curveto(852.91749834,637.79004067)(853.04249822,637.83004063)(853.15250337,637.87004168)
\curveto(853.27249799,637.92004054)(853.38249788,637.98004048)(853.48250337,638.05004168)
\curveto(853.59249767,638.12004034)(853.68749757,638.20004026)(853.76750337,638.29004168)
\curveto(853.84749741,638.39004007)(853.91749734,638.49503996)(853.97750337,638.60504168)
\curveto(854.03749722,638.70503975)(854.08749717,638.81003965)(854.12750337,638.92004168)
\curveto(854.17749708,639.03003943)(854.257497,639.11003935)(854.36750337,639.16004168)
\curveto(854.40749685,639.18003928)(854.47249679,639.19503926)(854.56250337,639.20504168)
\curveto(854.65249661,639.21503924)(854.74249652,639.21503924)(854.83250337,639.20504168)
\curveto(854.92249634,639.20503925)(855.00749625,639.20003926)(855.08750337,639.19004168)
\curveto(855.16749609,639.18003928)(855.22249604,639.1600393)(855.25250337,639.13004168)
\curveto(855.35249591,639.0600394)(855.37749588,638.94503951)(855.32750337,638.78504168)
\curveto(855.24749601,638.51503994)(855.14249612,638.27504018)(855.01250337,638.06504168)
\curveto(854.81249645,637.74504071)(854.58249668,637.48004098)(854.32250337,637.27004168)
\curveto(854.07249719,637.07004139)(853.75249751,636.90504155)(853.36250337,636.77504168)
\curveto(853.262498,636.73504172)(853.1624981,636.71004175)(853.06250337,636.70004168)
\curveto(852.9624983,636.68004178)(852.8574984,636.6600418)(852.74750337,636.64004168)
\curveto(852.69749856,636.63004183)(852.64749861,636.62504183)(852.59750337,636.62504168)
\curveto(852.5574987,636.62504183)(852.51249875,636.62004184)(852.46250337,636.61004168)
\lineto(852.31250337,636.61004168)
\curveto(852.262499,636.60004186)(852.20249906,636.59504186)(852.13250337,636.59504168)
\curveto(852.07249919,636.59504186)(852.02249924,636.60004186)(851.98250337,636.61004168)
\lineto(851.84750337,636.61004168)
\curveto(851.79749946,636.62004184)(851.75249951,636.62504183)(851.71250337,636.62504168)
\curveto(851.67249959,636.62504183)(851.63249963,636.63004183)(851.59250337,636.64004168)
\curveto(851.54249972,636.65004181)(851.48749977,636.6600418)(851.42750337,636.67004168)
\curveto(851.36749989,636.67004179)(851.31249995,636.67504178)(851.26250337,636.68504168)
\curveto(851.17250009,636.70504175)(851.08250018,636.73004173)(850.99250337,636.76004168)
\curveto(850.90250036,636.78004168)(850.81750044,636.80504165)(850.73750337,636.83504168)
\curveto(850.69750056,636.8550416)(850.6625006,636.86504159)(850.63250337,636.86504168)
\curveto(850.60250066,636.87504158)(850.56750069,636.89004157)(850.52750337,636.91004168)
\curveto(850.37750088,636.98004148)(850.21750104,637.06504139)(850.04750337,637.16504168)
\curveto(849.7575015,637.3550411)(849.50750175,637.58504087)(849.29750337,637.85504168)
\curveto(849.09750216,638.13504032)(848.92750233,638.44504001)(848.78750337,638.78504168)
\curveto(848.73750252,638.89503956)(848.69750256,639.01003945)(848.66750337,639.13004168)
\curveto(848.64750261,639.25003921)(848.61750264,639.37003909)(848.57750337,639.49004168)
\curveto(848.56750269,639.53003893)(848.5625027,639.56503889)(848.56250337,639.59504168)
\curveto(848.5625027,639.62503883)(848.5575027,639.66503879)(848.54750337,639.71504168)
\curveto(848.52750273,639.79503866)(848.51250275,639.88003858)(848.50250337,639.97004168)
\curveto(848.49250277,640.0600384)(848.47750278,640.15003831)(848.45750337,640.24004168)
\lineto(848.45750337,640.45004168)
\curveto(848.44750281,640.49003797)(848.43750282,640.54503791)(848.42750337,640.61504168)
\curveto(848.42750283,640.69503776)(848.43250283,640.7600377)(848.44250337,640.81004168)
\lineto(848.44250337,640.97504168)
\curveto(848.4625028,641.02503743)(848.46750279,641.07503738)(848.45750337,641.12504168)
\curveto(848.4575028,641.18503727)(848.4625028,641.24003722)(848.47250337,641.29004168)
\curveto(848.51250275,641.45003701)(848.54250272,641.61003685)(848.56250337,641.77004168)
\curveto(848.59250267,641.93003653)(848.63750262,642.08003638)(848.69750337,642.22004168)
\curveto(848.74750251,642.33003613)(848.79250247,642.44003602)(848.83250337,642.55004168)
\curveto(848.88250238,642.67003579)(848.93750232,642.78503567)(848.99750337,642.89504168)
\curveto(849.21750204,643.24503521)(849.46750179,643.54503491)(849.74750337,643.79504168)
\curveto(850.02750123,644.0550344)(850.37250089,644.27003419)(850.78250337,644.44004168)
\curveto(850.90250036,644.49003397)(851.02250024,644.52503393)(851.14250337,644.54504168)
\curveto(851.27249999,644.57503388)(851.40749985,644.60503385)(851.54750337,644.63504168)
\curveto(851.59749966,644.64503381)(851.64249962,644.65003381)(851.68250337,644.65004168)
\curveto(851.72249954,644.6600338)(851.76749949,644.66503379)(851.81750337,644.66504168)
\curveto(851.83749942,644.67503378)(851.8624994,644.67503378)(851.89250337,644.66504168)
\curveto(851.92249934,644.6550338)(851.94749931,644.6600338)(851.96750337,644.68004168)
\curveto(852.38749887,644.69003377)(852.75249851,644.64503381)(853.06250337,644.54504168)
\curveto(853.37249789,644.455034)(853.65249761,644.33003413)(853.90250337,644.17004168)
\curveto(853.95249731,644.15003431)(853.99249727,644.12003434)(854.02250337,644.08004168)
\curveto(854.05249721,644.05003441)(854.08749717,644.02503443)(854.12750337,644.00504168)
\curveto(854.20749705,643.94503451)(854.28749697,643.87503458)(854.36750337,643.79504168)
\curveto(854.4574968,643.71503474)(854.53249673,643.63503482)(854.59250337,643.55504168)
\curveto(854.75249651,643.34503511)(854.88749637,643.14503531)(854.99750337,642.95504168)
\curveto(855.06749619,642.84503561)(855.12249614,642.72503573)(855.16250337,642.59504168)
\curveto(855.20249606,642.46503599)(855.24749601,642.33503612)(855.29750337,642.20504168)
\curveto(855.34749591,642.07503638)(855.38249588,641.94003652)(855.40250337,641.80004168)
\curveto(855.43249583,641.6600368)(855.46749579,641.52003694)(855.50750337,641.38004168)
\curveto(855.51749574,641.31003715)(855.52249574,641.24003722)(855.52250337,641.17004168)
\lineto(855.55250337,640.96004168)
\moveto(854.09750337,641.47004168)
\curveto(854.12749713,641.51003695)(854.15249711,641.5600369)(854.17250337,641.62004168)
\curveto(854.19249707,641.69003677)(854.19249707,641.7600367)(854.17250337,641.83004168)
\curveto(854.11249715,642.05003641)(854.02749723,642.2550362)(853.91750337,642.44504168)
\curveto(853.77749748,642.67503578)(853.62249764,642.87003559)(853.45250337,643.03004168)
\curveto(853.28249798,643.19003527)(853.0624982,643.32503513)(852.79250337,643.43504168)
\curveto(852.72249854,643.455035)(852.65249861,643.47003499)(852.58250337,643.48004168)
\curveto(852.51249875,643.50003496)(852.43749882,643.52003494)(852.35750337,643.54004168)
\curveto(852.27749898,643.5600349)(852.19249907,643.57003489)(852.10250337,643.57004168)
\lineto(851.84750337,643.57004168)
\curveto(851.81749944,643.55003491)(851.78249948,643.54003492)(851.74250337,643.54004168)
\curveto(851.70249956,643.55003491)(851.66749959,643.55003491)(851.63750337,643.54004168)
\lineto(851.39750337,643.48004168)
\curveto(851.32749993,643.47003499)(851.2575,643.455035)(851.18750337,643.43504168)
\curveto(850.89750036,643.31503514)(850.6625006,643.16503529)(850.48250337,642.98504168)
\curveto(850.31250095,642.80503565)(850.1575011,642.58003588)(850.01750337,642.31004168)
\curveto(849.98750127,642.2600362)(849.9575013,642.19503626)(849.92750337,642.11504168)
\curveto(849.89750136,642.04503641)(849.87250139,641.96503649)(849.85250337,641.87504168)
\curveto(849.83250143,641.78503667)(849.82750143,641.70003676)(849.83750337,641.62004168)
\curveto(849.84750141,641.54003692)(849.88250138,641.48003698)(849.94250337,641.44004168)
\curveto(850.02250124,641.38003708)(850.1575011,641.35003711)(850.34750337,641.35004168)
\curveto(850.54750071,641.3600371)(850.71750054,641.36503709)(850.85750337,641.36504168)
\lineto(853.13750337,641.36504168)
\curveto(853.28749797,641.36503709)(853.46749779,641.3600371)(853.67750337,641.35004168)
\curveto(853.88749737,641.35003711)(854.02749723,641.39003707)(854.09750337,641.47004168)
}
}
{
\newrgbcolor{curcolor}{0 0 0}
\pscustom[linestyle=none,fillstyle=solid,fillcolor=curcolor]
{
\newpath
\moveto(860.549144,644.66504168)
\curveto(861.17913876,644.68503377)(861.68413826,644.60003386)(862.064144,644.41004168)
\curveto(862.4441375,644.22003424)(862.74913719,643.93503452)(862.979144,643.55504168)
\curveto(863.0391369,643.455035)(863.08413686,643.34503511)(863.114144,643.22504168)
\curveto(863.15413679,643.11503534)(863.18913675,643.00003546)(863.219144,642.88004168)
\curveto(863.26913667,642.69003577)(863.29913664,642.48503597)(863.309144,642.26504168)
\curveto(863.31913662,642.04503641)(863.32413662,641.82003664)(863.324144,641.59004168)
\lineto(863.324144,639.98504168)
\lineto(863.324144,637.64504168)
\curveto(863.32413662,637.47504098)(863.31913662,637.30504115)(863.309144,637.13504168)
\curveto(863.30913663,636.96504149)(863.2441367,636.8550416)(863.114144,636.80504168)
\curveto(863.06413688,636.78504167)(863.00913693,636.77504168)(862.949144,636.77504168)
\curveto(862.89913704,636.76504169)(862.8441371,636.7600417)(862.784144,636.76004168)
\curveto(862.65413729,636.7600417)(862.52913741,636.76504169)(862.409144,636.77504168)
\curveto(862.28913765,636.77504168)(862.20413774,636.81504164)(862.154144,636.89504168)
\curveto(862.10413784,636.96504149)(862.07913786,637.0550414)(862.079144,637.16504168)
\lineto(862.079144,637.49504168)
\lineto(862.079144,638.78504168)
\lineto(862.079144,641.23004168)
\curveto(862.07913786,641.50003696)(862.07413787,641.76503669)(862.064144,642.02504168)
\curveto(862.05413789,642.29503616)(862.00913793,642.52503593)(861.929144,642.71504168)
\curveto(861.84913809,642.91503554)(861.72913821,643.07503538)(861.569144,643.19504168)
\curveto(861.40913853,643.32503513)(861.22413872,643.42503503)(861.014144,643.49504168)
\curveto(860.95413899,643.51503494)(860.88913905,643.52503493)(860.819144,643.52504168)
\curveto(860.75913918,643.53503492)(860.69913924,643.55003491)(860.639144,643.57004168)
\curveto(860.58913935,643.58003488)(860.50913943,643.58003488)(860.399144,643.57004168)
\curveto(860.29913964,643.57003489)(860.22913971,643.56503489)(860.189144,643.55504168)
\curveto(860.14913979,643.53503492)(860.11413983,643.52503493)(860.084144,643.52504168)
\curveto(860.05413989,643.53503492)(860.01913992,643.53503492)(859.979144,643.52504168)
\curveto(859.84914009,643.49503496)(859.72414022,643.460035)(859.604144,643.42004168)
\curveto(859.49414045,643.39003507)(859.38914055,643.34503511)(859.289144,643.28504168)
\curveto(859.24914069,643.26503519)(859.21414073,643.24503521)(859.184144,643.22504168)
\curveto(859.15414079,643.20503525)(859.11914082,643.18503527)(859.079144,643.16504168)
\curveto(858.72914121,642.91503554)(858.47414147,642.54003592)(858.314144,642.04004168)
\curveto(858.28414166,641.9600365)(858.26414168,641.87503658)(858.254144,641.78504168)
\curveto(858.2441417,641.70503675)(858.22914171,641.62503683)(858.209144,641.54504168)
\curveto(858.18914175,641.49503696)(858.18414176,641.44503701)(858.194144,641.39504168)
\curveto(858.20414174,641.3550371)(858.19914174,641.31503714)(858.179144,641.27504168)
\lineto(858.179144,640.96004168)
\curveto(858.16914177,640.93003753)(858.16414178,640.89503756)(858.164144,640.85504168)
\curveto(858.17414177,640.81503764)(858.17914176,640.77003769)(858.179144,640.72004168)
\lineto(858.179144,640.27004168)
\lineto(858.179144,638.83004168)
\lineto(858.179144,637.51004168)
\lineto(858.179144,637.16504168)
\curveto(858.17914176,637.0550414)(858.15414179,636.96504149)(858.104144,636.89504168)
\curveto(858.05414189,636.81504164)(857.96414198,636.77504168)(857.834144,636.77504168)
\curveto(857.71414223,636.76504169)(857.58914235,636.7600417)(857.459144,636.76004168)
\curveto(857.37914256,636.7600417)(857.30414264,636.76504169)(857.234144,636.77504168)
\curveto(857.16414278,636.78504167)(857.10414284,636.81004165)(857.054144,636.85004168)
\curveto(856.97414297,636.90004156)(856.93414301,636.99504146)(856.934144,637.13504168)
\lineto(856.934144,637.54004168)
\lineto(856.934144,639.31004168)
\lineto(856.934144,642.94004168)
\lineto(856.934144,643.85504168)
\lineto(856.934144,644.12504168)
\curveto(856.93414301,644.21503424)(856.95414299,644.28503417)(856.994144,644.33504168)
\curveto(857.02414292,644.39503406)(857.07414287,644.43503402)(857.144144,644.45504168)
\curveto(857.18414276,644.46503399)(857.2391427,644.47503398)(857.309144,644.48504168)
\curveto(857.38914255,644.49503396)(857.46914247,644.50003396)(857.549144,644.50004168)
\curveto(857.62914231,644.50003396)(857.70414224,644.49503396)(857.774144,644.48504168)
\curveto(857.85414209,644.47503398)(857.90914203,644.460034)(857.939144,644.44004168)
\curveto(858.04914189,644.37003409)(858.09914184,644.28003418)(858.089144,644.17004168)
\curveto(858.07914186,644.07003439)(858.09414185,643.9550345)(858.134144,643.82504168)
\curveto(858.15414179,643.76503469)(858.19414175,643.71503474)(858.254144,643.67504168)
\curveto(858.37414157,643.66503479)(858.46914147,643.71003475)(858.539144,643.81004168)
\curveto(858.61914132,643.91003455)(858.69914124,643.99003447)(858.779144,644.05004168)
\curveto(858.91914102,644.15003431)(859.05914088,644.24003422)(859.199144,644.32004168)
\curveto(859.34914059,644.41003405)(859.51914042,644.48503397)(859.709144,644.54504168)
\curveto(859.78914015,644.57503388)(859.87414007,644.59503386)(859.964144,644.60504168)
\curveto(860.06413988,644.61503384)(860.15913978,644.63003383)(860.249144,644.65004168)
\curveto(860.29913964,644.6600338)(860.34913959,644.66503379)(860.399144,644.66504168)
\lineto(860.549144,644.66504168)
}
}
{
\newrgbcolor{curcolor}{0 0 0}
\pscustom[linestyle=none,fillstyle=solid,fillcolor=curcolor]
{
\newpath
\moveto(866.15375337,646.85504168)
\curveto(866.30375136,646.8550316)(866.45375121,646.85003161)(866.60375337,646.84004168)
\curveto(866.75375091,646.84003162)(866.85875081,646.80003166)(866.91875337,646.72004168)
\curveto(866.9687507,646.6600318)(866.99375067,646.57503188)(866.99375337,646.46504168)
\curveto(867.00375066,646.36503209)(867.00875066,646.2600322)(867.00875337,646.15004168)
\lineto(867.00875337,645.28004168)
\curveto(867.00875066,645.20003326)(867.00375066,645.11503334)(866.99375337,645.02504168)
\curveto(866.99375067,644.94503351)(867.00375066,644.87503358)(867.02375337,644.81504168)
\curveto(867.0637506,644.67503378)(867.15375051,644.58503387)(867.29375337,644.54504168)
\curveto(867.34375032,644.53503392)(867.38875028,644.53003393)(867.42875337,644.53004168)
\lineto(867.57875337,644.53004168)
\lineto(867.98375337,644.53004168)
\curveto(868.14374952,644.54003392)(868.25874941,644.53003393)(868.32875337,644.50004168)
\curveto(868.41874925,644.44003402)(868.47874919,644.38003408)(868.50875337,644.32004168)
\curveto(868.52874914,644.28003418)(868.53874913,644.23503422)(868.53875337,644.18504168)
\lineto(868.53875337,644.03504168)
\curveto(868.53874913,643.92503453)(868.53374913,643.82003464)(868.52375337,643.72004168)
\curveto(868.51374915,643.63003483)(868.47874919,643.5600349)(868.41875337,643.51004168)
\curveto(868.35874931,643.460035)(868.27374939,643.43003503)(868.16375337,643.42004168)
\lineto(867.83375337,643.42004168)
\curveto(867.72374994,643.43003503)(867.61375005,643.43503502)(867.50375337,643.43504168)
\curveto(867.39375027,643.43503502)(867.29875037,643.42003504)(867.21875337,643.39004168)
\curveto(867.14875052,643.3600351)(867.09875057,643.31003515)(867.06875337,643.24004168)
\curveto(867.03875063,643.17003529)(867.01875065,643.08503537)(867.00875337,642.98504168)
\curveto(866.99875067,642.89503556)(866.99375067,642.79503566)(866.99375337,642.68504168)
\curveto(867.00375066,642.58503587)(867.00875066,642.48503597)(867.00875337,642.38504168)
\lineto(867.00875337,639.41504168)
\curveto(867.00875066,639.19503926)(867.00375066,638.9600395)(866.99375337,638.71004168)
\curveto(866.99375067,638.47003999)(867.03875063,638.28504017)(867.12875337,638.15504168)
\curveto(867.17875049,638.07504038)(867.24375042,638.02004044)(867.32375337,637.99004168)
\curveto(867.40375026,637.9600405)(867.49875017,637.93504052)(867.60875337,637.91504168)
\curveto(867.63875003,637.90504055)(867.66875,637.90004056)(867.69875337,637.90004168)
\curveto(867.73874993,637.91004055)(867.77374989,637.91004055)(867.80375337,637.90004168)
\lineto(867.99875337,637.90004168)
\curveto(868.09874957,637.90004056)(868.18874948,637.89004057)(868.26875337,637.87004168)
\curveto(868.35874931,637.8600406)(868.42374924,637.82504063)(868.46375337,637.76504168)
\curveto(868.48374918,637.73504072)(868.49874917,637.68004078)(868.50875337,637.60004168)
\curveto(868.52874914,637.53004093)(868.53874913,637.455041)(868.53875337,637.37504168)
\curveto(868.54874912,637.29504116)(868.54874912,637.21504124)(868.53875337,637.13504168)
\curveto(868.52874914,637.06504139)(868.50874916,637.01004145)(868.47875337,636.97004168)
\curveto(868.43874923,636.90004156)(868.3637493,636.85004161)(868.25375337,636.82004168)
\curveto(868.17374949,636.80004166)(868.08374958,636.79004167)(867.98375337,636.79004168)
\curveto(867.88374978,636.80004166)(867.79374987,636.80504165)(867.71375337,636.80504168)
\curveto(867.65375001,636.80504165)(867.59375007,636.80004166)(867.53375337,636.79004168)
\curveto(867.47375019,636.79004167)(867.41875025,636.79504166)(867.36875337,636.80504168)
\lineto(867.18875337,636.80504168)
\curveto(867.13875053,636.81504164)(867.08875058,636.82004164)(867.03875337,636.82004168)
\curveto(866.99875067,636.83004163)(866.95375071,636.83504162)(866.90375337,636.83504168)
\curveto(866.70375096,636.88504157)(866.52875114,636.94004152)(866.37875337,637.00004168)
\curveto(866.23875143,637.0600414)(866.11875155,637.16504129)(866.01875337,637.31504168)
\curveto(865.87875179,637.51504094)(865.79875187,637.76504069)(865.77875337,638.06504168)
\curveto(865.75875191,638.37504008)(865.74875192,638.70503975)(865.74875337,639.05504168)
\lineto(865.74875337,642.98504168)
\curveto(865.71875195,643.11503534)(865.68875198,643.21003525)(865.65875337,643.27004168)
\curveto(865.63875203,643.33003513)(865.5687521,643.38003508)(865.44875337,643.42004168)
\curveto(865.40875226,643.43003503)(865.3687523,643.43003503)(865.32875337,643.42004168)
\curveto(865.28875238,643.41003505)(865.24875242,643.41503504)(865.20875337,643.43504168)
\lineto(864.96875337,643.43504168)
\curveto(864.83875283,643.43503502)(864.72875294,643.44503501)(864.63875337,643.46504168)
\curveto(864.55875311,643.49503496)(864.50375316,643.5550349)(864.47375337,643.64504168)
\curveto(864.45375321,643.68503477)(864.43875323,643.73003473)(864.42875337,643.78004168)
\lineto(864.42875337,643.93004168)
\curveto(864.42875324,644.07003439)(864.43875323,644.18503427)(864.45875337,644.27504168)
\curveto(864.47875319,644.37503408)(864.53875313,644.45003401)(864.63875337,644.50004168)
\curveto(864.74875292,644.54003392)(864.88875278,644.55003391)(865.05875337,644.53004168)
\curveto(865.23875243,644.51003395)(865.38875228,644.52003394)(865.50875337,644.56004168)
\curveto(865.59875207,644.61003385)(865.668752,644.68003378)(865.71875337,644.77004168)
\curveto(865.73875193,644.83003363)(865.74875192,644.90503355)(865.74875337,644.99504168)
\lineto(865.74875337,645.25004168)
\lineto(865.74875337,646.18004168)
\lineto(865.74875337,646.42004168)
\curveto(865.74875192,646.51003195)(865.75875191,646.58503187)(865.77875337,646.64504168)
\curveto(865.81875185,646.72503173)(865.89375177,646.79003167)(866.00375337,646.84004168)
\curveto(866.03375163,646.84003162)(866.05875161,646.84003162)(866.07875337,646.84004168)
\curveto(866.10875156,646.85003161)(866.13375153,646.8550316)(866.15375337,646.85504168)
}
}
{
\newrgbcolor{curcolor}{0 0 0}
\pscustom[linestyle=none,fillstyle=solid,fillcolor=curcolor]
{
\newpath
\moveto(877.05055025,640.99004168)
\curveto(877.07054219,640.93003753)(877.08054218,640.83503762)(877.08055025,640.70504168)
\curveto(877.08054218,640.58503787)(877.07554218,640.50003796)(877.06555025,640.45004168)
\lineto(877.06555025,640.30004168)
\curveto(877.0555422,640.22003824)(877.04554221,640.14503831)(877.03555025,640.07504168)
\curveto(877.03554222,640.01503844)(877.03054223,639.94503851)(877.02055025,639.86504168)
\curveto(877.00054226,639.80503865)(876.98554227,639.74503871)(876.97555025,639.68504168)
\curveto(876.97554228,639.62503883)(876.96554229,639.56503889)(876.94555025,639.50504168)
\curveto(876.90554235,639.37503908)(876.87054239,639.24503921)(876.84055025,639.11504168)
\curveto(876.81054245,638.98503947)(876.77054249,638.86503959)(876.72055025,638.75504168)
\curveto(876.51054275,638.27504018)(876.23054303,637.87004059)(875.88055025,637.54004168)
\curveto(875.53054373,637.22004124)(875.10054416,636.97504148)(874.59055025,636.80504168)
\curveto(874.48054478,636.76504169)(874.3605449,636.73504172)(874.23055025,636.71504168)
\curveto(874.11054515,636.69504176)(873.98554527,636.67504178)(873.85555025,636.65504168)
\curveto(873.79554546,636.64504181)(873.73054553,636.64004182)(873.66055025,636.64004168)
\curveto(873.60054566,636.63004183)(873.54054572,636.62504183)(873.48055025,636.62504168)
\curveto(873.44054582,636.61504184)(873.38054588,636.61004185)(873.30055025,636.61004168)
\curveto(873.23054603,636.61004185)(873.18054608,636.61504184)(873.15055025,636.62504168)
\curveto(873.11054615,636.63504182)(873.07054619,636.64004182)(873.03055025,636.64004168)
\curveto(872.99054627,636.63004183)(872.9555463,636.63004183)(872.92555025,636.64004168)
\lineto(872.83555025,636.64004168)
\lineto(872.47555025,636.68504168)
\curveto(872.33554692,636.72504173)(872.20054706,636.76504169)(872.07055025,636.80504168)
\curveto(871.94054732,636.84504161)(871.81554744,636.89004157)(871.69555025,636.94004168)
\curveto(871.24554801,637.14004132)(870.87554838,637.40004106)(870.58555025,637.72004168)
\curveto(870.29554896,638.04004042)(870.0555492,638.43004003)(869.86555025,638.89004168)
\curveto(869.81554944,638.99003947)(869.77554948,639.09003937)(869.74555025,639.19004168)
\curveto(869.72554953,639.29003917)(869.70554955,639.39503906)(869.68555025,639.50504168)
\curveto(869.66554959,639.54503891)(869.6555496,639.57503888)(869.65555025,639.59504168)
\curveto(869.66554959,639.62503883)(869.66554959,639.6600388)(869.65555025,639.70004168)
\curveto(869.63554962,639.78003868)(869.62054964,639.8600386)(869.61055025,639.94004168)
\curveto(869.61054965,640.03003843)(869.60054966,640.11503834)(869.58055025,640.19504168)
\lineto(869.58055025,640.31504168)
\curveto(869.58054968,640.3550381)(869.57554968,640.40003806)(869.56555025,640.45004168)
\curveto(869.5555497,640.50003796)(869.55054971,640.58503787)(869.55055025,640.70504168)
\curveto(869.55054971,640.83503762)(869.5605497,640.93003753)(869.58055025,640.99004168)
\curveto(869.60054966,641.0600374)(869.60554965,641.13003733)(869.59555025,641.20004168)
\curveto(869.58554967,641.27003719)(869.59054967,641.34003712)(869.61055025,641.41004168)
\curveto(869.62054964,641.460037)(869.62554963,641.50003696)(869.62555025,641.53004168)
\curveto(869.63554962,641.57003689)(869.64554961,641.61503684)(869.65555025,641.66504168)
\curveto(869.68554957,641.78503667)(869.71054955,641.90503655)(869.73055025,642.02504168)
\curveto(869.7605495,642.14503631)(869.80054946,642.2600362)(869.85055025,642.37004168)
\curveto(870.00054926,642.74003572)(870.18054908,643.07003539)(870.39055025,643.36004168)
\curveto(870.61054865,643.6600348)(870.87554838,643.91003455)(871.18555025,644.11004168)
\curveto(871.30554795,644.19003427)(871.43054783,644.2550342)(871.56055025,644.30504168)
\curveto(871.69054757,644.36503409)(871.82554743,644.42503403)(871.96555025,644.48504168)
\curveto(872.08554717,644.53503392)(872.21554704,644.56503389)(872.35555025,644.57504168)
\curveto(872.49554676,644.59503386)(872.63554662,644.62503383)(872.77555025,644.66504168)
\lineto(872.97055025,644.66504168)
\curveto(873.04054622,644.67503378)(873.10554615,644.68503377)(873.16555025,644.69504168)
\curveto(874.0555452,644.70503375)(874.79554446,644.52003394)(875.38555025,644.14004168)
\curveto(875.97554328,643.7600347)(876.40054286,643.26503519)(876.66055025,642.65504168)
\curveto(876.71054255,642.5550359)(876.75054251,642.455036)(876.78055025,642.35504168)
\curveto(876.81054245,642.2550362)(876.84554241,642.15003631)(876.88555025,642.04004168)
\curveto(876.91554234,641.93003653)(876.94054232,641.81003665)(876.96055025,641.68004168)
\curveto(876.98054228,641.5600369)(877.00554225,641.43503702)(877.03555025,641.30504168)
\curveto(877.04554221,641.2550372)(877.04554221,641.20003726)(877.03555025,641.14004168)
\curveto(877.03554222,641.09003737)(877.04054222,641.04003742)(877.05055025,640.99004168)
\moveto(875.71555025,640.13504168)
\curveto(875.73554352,640.20503825)(875.74054352,640.28503817)(875.73055025,640.37504168)
\lineto(875.73055025,640.63004168)
\curveto(875.73054353,641.02003744)(875.69554356,641.35003711)(875.62555025,641.62004168)
\curveto(875.59554366,641.70003676)(875.57054369,641.78003668)(875.55055025,641.86004168)
\curveto(875.53054373,641.94003652)(875.50554375,642.01503644)(875.47555025,642.08504168)
\curveto(875.19554406,642.73503572)(874.75054451,643.18503527)(874.14055025,643.43504168)
\curveto(874.07054519,643.46503499)(873.99554526,643.48503497)(873.91555025,643.49504168)
\lineto(873.67555025,643.55504168)
\curveto(873.59554566,643.57503488)(873.51054575,643.58503487)(873.42055025,643.58504168)
\lineto(873.15055025,643.58504168)
\lineto(872.88055025,643.54004168)
\curveto(872.78054648,643.52003494)(872.68554657,643.49503496)(872.59555025,643.46504168)
\curveto(872.51554674,643.44503501)(872.43554682,643.41503504)(872.35555025,643.37504168)
\curveto(872.28554697,643.3550351)(872.22054704,643.32503513)(872.16055025,643.28504168)
\curveto(872.10054716,643.24503521)(872.04554721,643.20503525)(871.99555025,643.16504168)
\curveto(871.7555475,642.99503546)(871.5605477,642.79003567)(871.41055025,642.55004168)
\curveto(871.260548,642.31003615)(871.13054813,642.03003643)(871.02055025,641.71004168)
\curveto(870.99054827,641.61003685)(870.97054829,641.50503695)(870.96055025,641.39504168)
\curveto(870.95054831,641.29503716)(870.93554832,641.19003727)(870.91555025,641.08004168)
\curveto(870.90554835,641.04003742)(870.90054836,640.97503748)(870.90055025,640.88504168)
\curveto(870.89054837,640.8550376)(870.88554837,640.82003764)(870.88555025,640.78004168)
\curveto(870.89554836,640.74003772)(870.90054836,640.69503776)(870.90055025,640.64504168)
\lineto(870.90055025,640.34504168)
\curveto(870.90054836,640.24503821)(870.91054835,640.1550383)(870.93055025,640.07504168)
\lineto(870.96055025,639.89504168)
\curveto(870.98054828,639.79503866)(870.99554826,639.69503876)(871.00555025,639.59504168)
\curveto(871.02554823,639.50503895)(871.0555482,639.42003904)(871.09555025,639.34004168)
\curveto(871.19554806,639.10003936)(871.31054795,638.87503958)(871.44055025,638.66504168)
\curveto(871.58054768,638.45504)(871.75054751,638.28004018)(871.95055025,638.14004168)
\curveto(872.00054726,638.11004035)(872.04554721,638.08504037)(872.08555025,638.06504168)
\curveto(872.12554713,638.04504041)(872.17054709,638.02004044)(872.22055025,637.99004168)
\curveto(872.30054696,637.94004052)(872.38554687,637.89504056)(872.47555025,637.85504168)
\curveto(872.57554668,637.82504063)(872.68054658,637.79504066)(872.79055025,637.76504168)
\curveto(872.84054642,637.74504071)(872.88554637,637.73504072)(872.92555025,637.73504168)
\curveto(872.97554628,637.74504071)(873.02554623,637.74504071)(873.07555025,637.73504168)
\curveto(873.10554615,637.72504073)(873.16554609,637.71504074)(873.25555025,637.70504168)
\curveto(873.3555459,637.69504076)(873.43054583,637.70004076)(873.48055025,637.72004168)
\curveto(873.52054574,637.73004073)(873.5605457,637.73004073)(873.60055025,637.72004168)
\curveto(873.64054562,637.72004074)(873.68054558,637.73004073)(873.72055025,637.75004168)
\curveto(873.80054546,637.77004069)(873.88054538,637.78504067)(873.96055025,637.79504168)
\curveto(874.04054522,637.81504064)(874.11554514,637.84004062)(874.18555025,637.87004168)
\curveto(874.52554473,638.01004045)(874.80054446,638.20504025)(875.01055025,638.45504168)
\curveto(875.22054404,638.70503975)(875.39554386,639.00003946)(875.53555025,639.34004168)
\curveto(875.58554367,639.460039)(875.61554364,639.58503887)(875.62555025,639.71504168)
\curveto(875.64554361,639.8550386)(875.67554358,639.99503846)(875.71555025,640.13504168)
}
}
{
\newrgbcolor{curcolor}{0 0 0}
\pscustom[linestyle=none,fillstyle=solid,fillcolor=curcolor]
{
\newpath
\moveto(880.9688315,644.69504168)
\curveto(881.68882743,644.70503375)(882.29382683,644.62003384)(882.7838315,644.44004168)
\curveto(883.27382585,644.27003419)(883.65382547,643.96503449)(883.9238315,643.52504168)
\curveto(883.99382513,643.41503504)(884.04882507,643.30003516)(884.0888315,643.18004168)
\curveto(884.12882499,643.07003539)(884.16882495,642.94503551)(884.2088315,642.80504168)
\curveto(884.22882489,642.73503572)(884.23382489,642.6600358)(884.2238315,642.58004168)
\curveto(884.21382491,642.51003595)(884.19882492,642.455036)(884.1788315,642.41504168)
\curveto(884.15882496,642.39503606)(884.13382499,642.37503608)(884.1038315,642.35504168)
\curveto(884.07382505,642.34503611)(884.04882507,642.33003613)(884.0288315,642.31004168)
\curveto(883.97882514,642.29003617)(883.92882519,642.28503617)(883.8788315,642.29504168)
\curveto(883.82882529,642.30503615)(883.77882534,642.30503615)(883.7288315,642.29504168)
\curveto(883.64882547,642.27503618)(883.54382558,642.27003619)(883.4138315,642.28004168)
\curveto(883.28382584,642.30003616)(883.19382593,642.32503613)(883.1438315,642.35504168)
\curveto(883.06382606,642.40503605)(883.00882611,642.47003599)(882.9788315,642.55004168)
\curveto(882.95882616,642.64003582)(882.9238262,642.72503573)(882.8738315,642.80504168)
\curveto(882.78382634,642.96503549)(882.65882646,643.11003535)(882.4988315,643.24004168)
\curveto(882.38882673,643.32003514)(882.26882685,643.38003508)(882.1388315,643.42004168)
\curveto(882.00882711,643.460035)(881.86882725,643.50003496)(881.7188315,643.54004168)
\curveto(881.66882745,643.5600349)(881.6188275,643.56503489)(881.5688315,643.55504168)
\curveto(881.5188276,643.5550349)(881.46882765,643.5600349)(881.4188315,643.57004168)
\curveto(881.35882776,643.59003487)(881.28382784,643.60003486)(881.1938315,643.60004168)
\curveto(881.10382802,643.60003486)(881.02882809,643.59003487)(880.9688315,643.57004168)
\lineto(880.8788315,643.57004168)
\lineto(880.7288315,643.54004168)
\curveto(880.67882844,643.54003492)(880.62882849,643.53503492)(880.5788315,643.52504168)
\curveto(880.3188288,643.46503499)(880.10382902,643.38003508)(879.9338315,643.27004168)
\curveto(879.76382936,643.1600353)(879.64882947,642.97503548)(879.5888315,642.71504168)
\curveto(879.56882955,642.64503581)(879.56382956,642.57503588)(879.5738315,642.50504168)
\curveto(879.59382953,642.43503602)(879.61382951,642.37503608)(879.6338315,642.32504168)
\curveto(879.69382943,642.17503628)(879.76382936,642.06503639)(879.8438315,641.99504168)
\curveto(879.93382919,641.93503652)(880.04382908,641.86503659)(880.1738315,641.78504168)
\curveto(880.33382879,641.68503677)(880.51382861,641.61003685)(880.7138315,641.56004168)
\curveto(880.91382821,641.52003694)(881.11382801,641.47003699)(881.3138315,641.41004168)
\curveto(881.44382768,641.37003709)(881.57382755,641.34003712)(881.7038315,641.32004168)
\curveto(881.83382729,641.30003716)(881.96382716,641.27003719)(882.0938315,641.23004168)
\curveto(882.30382682,641.17003729)(882.50882661,641.11003735)(882.7088315,641.05004168)
\curveto(882.90882621,641.00003746)(883.10882601,640.93503752)(883.3088315,640.85504168)
\lineto(883.4588315,640.79504168)
\curveto(883.50882561,640.77503768)(883.55882556,640.75003771)(883.6088315,640.72004168)
\curveto(883.80882531,640.60003786)(883.98382514,640.46503799)(884.1338315,640.31504168)
\curveto(884.28382484,640.16503829)(884.40882471,639.97503848)(884.5088315,639.74504168)
\curveto(884.52882459,639.67503878)(884.54882457,639.58003888)(884.5688315,639.46004168)
\curveto(884.58882453,639.39003907)(884.59882452,639.31503914)(884.5988315,639.23504168)
\curveto(884.60882451,639.16503929)(884.61382451,639.08503937)(884.6138315,638.99504168)
\lineto(884.6138315,638.84504168)
\curveto(884.59382453,638.77503968)(884.58382454,638.70503975)(884.5838315,638.63504168)
\curveto(884.58382454,638.56503989)(884.57382455,638.49503996)(884.5538315,638.42504168)
\curveto(884.5238246,638.31504014)(884.48882463,638.21004025)(884.4488315,638.11004168)
\curveto(884.40882471,638.01004045)(884.36382476,637.92004054)(884.3138315,637.84004168)
\curveto(884.15382497,637.58004088)(883.94882517,637.37004109)(883.6988315,637.21004168)
\curveto(883.44882567,637.0600414)(883.16882595,636.93004153)(882.8588315,636.82004168)
\curveto(882.76882635,636.79004167)(882.67382645,636.77004169)(882.5738315,636.76004168)
\curveto(882.48382664,636.74004172)(882.39382673,636.71504174)(882.3038315,636.68504168)
\curveto(882.20382692,636.66504179)(882.10382702,636.6550418)(882.0038315,636.65504168)
\curveto(881.90382722,636.6550418)(881.80382732,636.64504181)(881.7038315,636.62504168)
\lineto(881.5538315,636.62504168)
\curveto(881.50382762,636.61504184)(881.43382769,636.61004185)(881.3438315,636.61004168)
\curveto(881.25382787,636.61004185)(881.18382794,636.61504184)(881.1338315,636.62504168)
\lineto(880.9688315,636.62504168)
\curveto(880.90882821,636.64504181)(880.84382828,636.6550418)(880.7738315,636.65504168)
\curveto(880.70382842,636.64504181)(880.64382848,636.65004181)(880.5938315,636.67004168)
\curveto(880.54382858,636.68004178)(880.47882864,636.68504177)(880.3988315,636.68504168)
\lineto(880.1588315,636.74504168)
\curveto(880.08882903,636.7550417)(880.01382911,636.77504168)(879.9338315,636.80504168)
\curveto(879.6238295,636.90504155)(879.35382977,637.03004143)(879.1238315,637.18004168)
\curveto(878.89383023,637.33004113)(878.69383043,637.52504093)(878.5238315,637.76504168)
\curveto(878.43383069,637.89504056)(878.35883076,638.03004043)(878.2988315,638.17004168)
\curveto(878.23883088,638.31004015)(878.18383094,638.46503999)(878.1338315,638.63504168)
\curveto(878.11383101,638.69503976)(878.10383102,638.76503969)(878.1038315,638.84504168)
\curveto(878.11383101,638.93503952)(878.12883099,639.00503945)(878.1488315,639.05504168)
\curveto(878.17883094,639.09503936)(878.22883089,639.13503932)(878.2988315,639.17504168)
\curveto(878.34883077,639.19503926)(878.4188307,639.20503925)(878.5088315,639.20504168)
\curveto(878.59883052,639.21503924)(878.68883043,639.21503924)(878.7788315,639.20504168)
\curveto(878.86883025,639.19503926)(878.95383017,639.18003928)(879.0338315,639.16004168)
\curveto(879.12383,639.15003931)(879.18382994,639.13503932)(879.2138315,639.11504168)
\curveto(879.28382984,639.06503939)(879.32882979,638.99003947)(879.3488315,638.89004168)
\curveto(879.37882974,638.80003966)(879.41382971,638.71503974)(879.4538315,638.63504168)
\curveto(879.55382957,638.41504004)(879.68882943,638.24504021)(879.8588315,638.12504168)
\curveto(879.97882914,638.03504042)(880.11382901,637.96504049)(880.2638315,637.91504168)
\curveto(880.41382871,637.86504059)(880.57382855,637.81504064)(880.7438315,637.76504168)
\lineto(881.0588315,637.72004168)
\lineto(881.1488315,637.72004168)
\curveto(881.2188279,637.70004076)(881.30882781,637.69004077)(881.4188315,637.69004168)
\curveto(881.53882758,637.69004077)(881.63882748,637.70004076)(881.7188315,637.72004168)
\curveto(881.78882733,637.72004074)(881.84382728,637.72504073)(881.8838315,637.73504168)
\curveto(881.94382718,637.74504071)(882.00382712,637.75004071)(882.0638315,637.75004168)
\curveto(882.123827,637.7600407)(882.17882694,637.77004069)(882.2288315,637.78004168)
\curveto(882.5188266,637.8600406)(882.74882637,637.96504049)(882.9188315,638.09504168)
\curveto(883.08882603,638.22504023)(883.20882591,638.44504001)(883.2788315,638.75504168)
\curveto(883.29882582,638.80503965)(883.30382582,638.8600396)(883.2938315,638.92004168)
\curveto(883.28382584,638.98003948)(883.27382585,639.02503943)(883.2638315,639.05504168)
\curveto(883.21382591,639.24503921)(883.14382598,639.38503907)(883.0538315,639.47504168)
\curveto(882.96382616,639.57503888)(882.84882627,639.66503879)(882.7088315,639.74504168)
\curveto(882.6188265,639.80503865)(882.5188266,639.8550386)(882.4088315,639.89504168)
\lineto(882.0788315,640.01504168)
\curveto(882.04882707,640.02503843)(882.0188271,640.03003843)(881.9888315,640.03004168)
\curveto(881.96882715,640.03003843)(881.94382718,640.04003842)(881.9138315,640.06004168)
\curveto(881.57382755,640.17003829)(881.2188279,640.25003821)(880.8488315,640.30004168)
\curveto(880.48882863,640.3600381)(880.14882897,640.455038)(879.8288315,640.58504168)
\curveto(879.72882939,640.62503783)(879.63382949,640.6600378)(879.5438315,640.69004168)
\curveto(879.45382967,640.72003774)(879.36882975,640.7600377)(879.2888315,640.81004168)
\curveto(879.09883002,640.92003754)(878.9238302,641.04503741)(878.7638315,641.18504168)
\curveto(878.60383052,641.32503713)(878.47883064,641.50003696)(878.3888315,641.71004168)
\curveto(878.35883076,641.78003668)(878.33383079,641.85003661)(878.3138315,641.92004168)
\curveto(878.30383082,641.99003647)(878.28883083,642.06503639)(878.2688315,642.14504168)
\curveto(878.23883088,642.26503619)(878.22883089,642.40003606)(878.2388315,642.55004168)
\curveto(878.24883087,642.71003575)(878.26383086,642.84503561)(878.2838315,642.95504168)
\curveto(878.30383082,643.00503545)(878.31383081,643.04503541)(878.3138315,643.07504168)
\curveto(878.3238308,643.11503534)(878.33883078,643.1550353)(878.3588315,643.19504168)
\curveto(878.44883067,643.42503503)(878.56883055,643.62503483)(878.7188315,643.79504168)
\curveto(878.87883024,643.96503449)(879.05883006,644.11503434)(879.2588315,644.24504168)
\curveto(879.40882971,644.33503412)(879.57382955,644.40503405)(879.7538315,644.45504168)
\curveto(879.93382919,644.51503394)(880.123829,644.57003389)(880.3238315,644.62004168)
\curveto(880.39382873,644.63003383)(880.45882866,644.64003382)(880.5188315,644.65004168)
\curveto(880.58882853,644.6600338)(880.66382846,644.67003379)(880.7438315,644.68004168)
\curveto(880.77382835,644.69003377)(880.81382831,644.69003377)(880.8638315,644.68004168)
\curveto(880.91382821,644.67003379)(880.94882817,644.67503378)(880.9688315,644.69504168)
}
}
{
\newrgbcolor{curcolor}{0.7019608 0.7019608 0.7019608}
\pscustom[linestyle=none,fillstyle=solid,fillcolor=curcolor]
{
\newpath
\moveto(812.80437349,647.5000783)
\lineto(827.80437349,647.5000783)
\lineto(827.80437349,632.5000783)
\lineto(812.80437349,632.5000783)
\closepath
}
}
{
\newrgbcolor{curcolor}{0 0 0}
\pscustom[linestyle=none,fillstyle=solid,fillcolor=curcolor]
{
\newpath
\moveto(832.7925815,624.43427508)
\lineto(837.6975815,624.43427508)
\lineto(838.9875815,624.43427508)
\curveto(839.09757362,624.43426438)(839.20757351,624.43426438)(839.3175815,624.43427508)
\curveto(839.42757329,624.44426437)(839.5175732,624.42426439)(839.5875815,624.37427508)
\curveto(839.6175731,624.35426446)(839.64257307,624.32926449)(839.6625815,624.29927508)
\curveto(839.68257303,624.26926455)(839.70257301,624.23926458)(839.7225815,624.20927508)
\curveto(839.74257297,624.13926468)(839.75257296,624.02426479)(839.7525815,623.86427508)
\curveto(839.75257296,623.7142651)(839.74257297,623.59926522)(839.7225815,623.51927508)
\curveto(839.68257303,623.37926544)(839.59757312,623.29926552)(839.4675815,623.27927508)
\curveto(839.33757338,623.26926555)(839.18257353,623.26426555)(839.0025815,623.26427508)
\lineto(837.5025815,623.26427508)
\lineto(834.9825815,623.26427508)
\lineto(834.4125815,623.26427508)
\curveto(834.20257851,623.27426554)(834.04757867,623.24926557)(833.9475815,623.18927508)
\curveto(833.84757887,623.12926569)(833.79257892,623.02426579)(833.7825815,622.87427508)
\lineto(833.7825815,622.40927508)
\lineto(833.7825815,620.87927508)
\curveto(833.78257893,620.76926805)(833.77757894,620.63926818)(833.7675815,620.48927508)
\curveto(833.76757895,620.33926848)(833.77757894,620.2192686)(833.7975815,620.12927508)
\curveto(833.82757889,620.00926881)(833.88757883,619.92926889)(833.9775815,619.88927508)
\curveto(834.0175787,619.86926895)(834.08757863,619.84926897)(834.1875815,619.82927508)
\lineto(834.3375815,619.82927508)
\curveto(834.37757834,619.819269)(834.4175783,619.814269)(834.4575815,619.81427508)
\curveto(834.50757821,619.82426899)(834.55757816,619.82926899)(834.6075815,619.82927508)
\lineto(835.1175815,619.82927508)
\lineto(838.0575815,619.82927508)
\lineto(838.3575815,619.82927508)
\curveto(838.46757425,619.83926898)(838.57757414,619.83926898)(838.6875815,619.82927508)
\curveto(838.80757391,619.82926899)(838.9125738,619.819269)(839.0025815,619.79927508)
\curveto(839.10257361,619.78926903)(839.17757354,619.76926905)(839.2275815,619.73927508)
\curveto(839.25757346,619.7192691)(839.28257343,619.67426914)(839.3025815,619.60427508)
\curveto(839.32257339,619.53426928)(839.33757338,619.45926936)(839.3475815,619.37927508)
\curveto(839.35757336,619.29926952)(839.35757336,619.2142696)(839.3475815,619.12427508)
\curveto(839.34757337,619.04426977)(839.33757338,618.97426984)(839.3175815,618.91427508)
\curveto(839.29757342,618.82426999)(839.25257346,618.75927006)(839.1825815,618.71927508)
\curveto(839.16257355,618.69927012)(839.13257358,618.68427013)(839.0925815,618.67427508)
\curveto(839.06257365,618.67427014)(839.03257368,618.66927015)(839.0025815,618.65927508)
\lineto(838.9125815,618.65927508)
\curveto(838.86257385,618.64927017)(838.8125739,618.64427017)(838.7625815,618.64427508)
\curveto(838.712574,618.65427016)(838.66257405,618.65927016)(838.6125815,618.65927508)
\lineto(838.0575815,618.65927508)
\lineto(834.8925815,618.65927508)
\lineto(834.5325815,618.65927508)
\curveto(834.42257829,618.66927015)(834.3175784,618.66427015)(834.2175815,618.64427508)
\curveto(834.1175786,618.63427018)(834.02757869,618.60927021)(833.9475815,618.56927508)
\curveto(833.87757884,618.52927029)(833.82757889,618.45927036)(833.7975815,618.35927508)
\curveto(833.77757894,618.29927052)(833.76757895,618.22927059)(833.7675815,618.14927508)
\curveto(833.77757894,618.06927075)(833.78257893,617.98927083)(833.7825815,617.90927508)
\lineto(833.7825815,617.06927508)
\lineto(833.7825815,615.64427508)
\curveto(833.78257893,615.50427331)(833.78757893,615.37427344)(833.7975815,615.25427508)
\curveto(833.80757891,615.14427367)(833.84757887,615.06427375)(833.9175815,615.01427508)
\curveto(833.98757873,614.96427385)(834.06757865,614.93427388)(834.1575815,614.92427508)
\lineto(834.4575815,614.92427508)
\lineto(835.4175815,614.92427508)
\lineto(838.1925815,614.92427508)
\lineto(839.0475815,614.92427508)
\lineto(839.2875815,614.92427508)
\curveto(839.36757335,614.93427388)(839.43757328,614.92927389)(839.4975815,614.90927508)
\curveto(839.6175731,614.86927395)(839.69757302,614.814274)(839.7375815,614.74427508)
\curveto(839.75757296,614.7142741)(839.77257294,614.66427415)(839.7825815,614.59427508)
\curveto(839.79257292,614.52427429)(839.79757292,614.44927437)(839.7975815,614.36927508)
\curveto(839.80757291,614.29927452)(839.80757291,614.22427459)(839.7975815,614.14427508)
\curveto(839.78757293,614.07427474)(839.77757294,614.0192748)(839.7675815,613.97927508)
\curveto(839.72757299,613.89927492)(839.68257303,613.84427497)(839.6325815,613.81427508)
\curveto(839.57257314,613.77427504)(839.49257322,613.75427506)(839.3925815,613.75427508)
\lineto(839.1225815,613.75427508)
\lineto(838.0725815,613.75427508)
\lineto(834.0825815,613.75427508)
\lineto(833.0325815,613.75427508)
\curveto(832.89257982,613.75427506)(832.77257994,613.75927506)(832.6725815,613.76927508)
\curveto(832.57258014,613.78927503)(832.49758022,613.83927498)(832.4475815,613.91927508)
\curveto(832.40758031,613.97927484)(832.38758033,614.05427476)(832.3875815,614.14427508)
\lineto(832.3875815,614.42927508)
\lineto(832.3875815,615.47927508)
\lineto(832.3875815,619.49927508)
\lineto(832.3875815,622.85927508)
\lineto(832.3875815,623.78927508)
\lineto(832.3875815,624.05927508)
\curveto(832.38758033,624.14926467)(832.40758031,624.2192646)(832.4475815,624.26927508)
\curveto(832.48758023,624.33926448)(832.56258015,624.38926443)(832.6725815,624.41927508)
\curveto(832.69258002,624.42926439)(832.71258,624.42926439)(832.7325815,624.41927508)
\curveto(832.75257996,624.4192644)(832.77257994,624.42426439)(832.7925815,624.43427508)
}
}
{
\newrgbcolor{curcolor}{0 0 0}
\pscustom[linestyle=none,fillstyle=solid,fillcolor=curcolor]
{
\newpath
\moveto(844.99250337,621.62927508)
\curveto(845.62249814,621.64926717)(846.12749763,621.56426725)(846.50750337,621.37427508)
\curveto(846.88749687,621.18426763)(847.19249657,620.89926792)(847.42250337,620.51927508)
\curveto(847.48249628,620.4192684)(847.52749623,620.30926851)(847.55750337,620.18927508)
\curveto(847.59749616,620.07926874)(847.63249613,619.96426885)(847.66250337,619.84427508)
\curveto(847.71249605,619.65426916)(847.74249602,619.44926937)(847.75250337,619.22927508)
\curveto(847.762496,619.00926981)(847.76749599,618.78427003)(847.76750337,618.55427508)
\lineto(847.76750337,616.94927508)
\lineto(847.76750337,614.60927508)
\curveto(847.76749599,614.43927438)(847.762496,614.26927455)(847.75250337,614.09927508)
\curveto(847.75249601,613.92927489)(847.68749607,613.819275)(847.55750337,613.76927508)
\curveto(847.50749625,613.74927507)(847.45249631,613.73927508)(847.39250337,613.73927508)
\curveto(847.34249642,613.72927509)(847.28749647,613.72427509)(847.22750337,613.72427508)
\curveto(847.09749666,613.72427509)(846.97249679,613.72927509)(846.85250337,613.73927508)
\curveto(846.73249703,613.73927508)(846.64749711,613.77927504)(846.59750337,613.85927508)
\curveto(846.54749721,613.92927489)(846.52249724,614.0192748)(846.52250337,614.12927508)
\lineto(846.52250337,614.45927508)
\lineto(846.52250337,615.74927508)
\lineto(846.52250337,618.19427508)
\curveto(846.52249724,618.46427035)(846.51749724,618.72927009)(846.50750337,618.98927508)
\curveto(846.49749726,619.25926956)(846.45249731,619.48926933)(846.37250337,619.67927508)
\curveto(846.29249747,619.87926894)(846.17249759,620.03926878)(846.01250337,620.15927508)
\curveto(845.85249791,620.28926853)(845.66749809,620.38926843)(845.45750337,620.45927508)
\curveto(845.39749836,620.47926834)(845.33249843,620.48926833)(845.26250337,620.48927508)
\curveto(845.20249856,620.49926832)(845.14249862,620.5142683)(845.08250337,620.53427508)
\curveto(845.03249873,620.54426827)(844.95249881,620.54426827)(844.84250337,620.53427508)
\curveto(844.74249902,620.53426828)(844.67249909,620.52926829)(844.63250337,620.51927508)
\curveto(844.59249917,620.49926832)(844.5574992,620.48926833)(844.52750337,620.48927508)
\curveto(844.49749926,620.49926832)(844.4624993,620.49926832)(844.42250337,620.48927508)
\curveto(844.29249947,620.45926836)(844.16749959,620.42426839)(844.04750337,620.38427508)
\curveto(843.93749982,620.35426846)(843.83249993,620.30926851)(843.73250337,620.24927508)
\curveto(843.69250007,620.22926859)(843.6575001,620.20926861)(843.62750337,620.18927508)
\curveto(843.59750016,620.16926865)(843.5625002,620.14926867)(843.52250337,620.12927508)
\curveto(843.17250059,619.87926894)(842.91750084,619.50426931)(842.75750337,619.00427508)
\curveto(842.72750103,618.92426989)(842.70750105,618.83926998)(842.69750337,618.74927508)
\curveto(842.68750107,618.66927015)(842.67250109,618.58927023)(842.65250337,618.50927508)
\curveto(842.63250113,618.45927036)(842.62750113,618.40927041)(842.63750337,618.35927508)
\curveto(842.64750111,618.3192705)(842.64250112,618.27927054)(842.62250337,618.23927508)
\lineto(842.62250337,617.92427508)
\curveto(842.61250115,617.89427092)(842.60750115,617.85927096)(842.60750337,617.81927508)
\curveto(842.61750114,617.77927104)(842.62250114,617.73427108)(842.62250337,617.68427508)
\lineto(842.62250337,617.23427508)
\lineto(842.62250337,615.79427508)
\lineto(842.62250337,614.47427508)
\lineto(842.62250337,614.12927508)
\curveto(842.62250114,614.0192748)(842.59750116,613.92927489)(842.54750337,613.85927508)
\curveto(842.49750126,613.77927504)(842.40750135,613.73927508)(842.27750337,613.73927508)
\curveto(842.1575016,613.72927509)(842.03250173,613.72427509)(841.90250337,613.72427508)
\curveto(841.82250194,613.72427509)(841.74750201,613.72927509)(841.67750337,613.73927508)
\curveto(841.60750215,613.74927507)(841.54750221,613.77427504)(841.49750337,613.81427508)
\curveto(841.41750234,613.86427495)(841.37750238,613.95927486)(841.37750337,614.09927508)
\lineto(841.37750337,614.50427508)
\lineto(841.37750337,616.27427508)
\lineto(841.37750337,619.90427508)
\lineto(841.37750337,620.81927508)
\lineto(841.37750337,621.08927508)
\curveto(841.37750238,621.17926764)(841.39750236,621.24926757)(841.43750337,621.29927508)
\curveto(841.46750229,621.35926746)(841.51750224,621.39926742)(841.58750337,621.41927508)
\curveto(841.62750213,621.42926739)(841.68250208,621.43926738)(841.75250337,621.44927508)
\curveto(841.83250193,621.45926736)(841.91250185,621.46426735)(841.99250337,621.46427508)
\curveto(842.07250169,621.46426735)(842.14750161,621.45926736)(842.21750337,621.44927508)
\curveto(842.29750146,621.43926738)(842.35250141,621.42426739)(842.38250337,621.40427508)
\curveto(842.49250127,621.33426748)(842.54250122,621.24426757)(842.53250337,621.13427508)
\curveto(842.52250124,621.03426778)(842.53750122,620.9192679)(842.57750337,620.78927508)
\curveto(842.59750116,620.72926809)(842.63750112,620.67926814)(842.69750337,620.63927508)
\curveto(842.81750094,620.62926819)(842.91250085,620.67426814)(842.98250337,620.77427508)
\curveto(843.0625007,620.87426794)(843.14250062,620.95426786)(843.22250337,621.01427508)
\curveto(843.3625004,621.1142677)(843.50250026,621.20426761)(843.64250337,621.28427508)
\curveto(843.79249997,621.37426744)(843.9624998,621.44926737)(844.15250337,621.50927508)
\curveto(844.23249953,621.53926728)(844.31749944,621.55926726)(844.40750337,621.56927508)
\curveto(844.50749925,621.57926724)(844.60249916,621.59426722)(844.69250337,621.61427508)
\curveto(844.74249902,621.62426719)(844.79249897,621.62926719)(844.84250337,621.62927508)
\lineto(844.99250337,621.62927508)
}
}
{
\newrgbcolor{curcolor}{0 0 0}
\pscustom[linestyle=none,fillstyle=solid,fillcolor=curcolor]
{
\newpath
\moveto(850.22211275,624.43427508)
\curveto(850.35211113,624.43426438)(850.487111,624.43426438)(850.62711275,624.43427508)
\curveto(850.77711071,624.43426438)(850.8871106,624.39926442)(850.95711275,624.32927508)
\curveto(851.00711048,624.25926456)(851.03211045,624.16426465)(851.03211275,624.04427508)
\curveto(851.04211044,623.93426488)(851.04711044,623.819265)(851.04711275,623.69927508)
\lineto(851.04711275,622.36427508)
\lineto(851.04711275,616.28927508)
\lineto(851.04711275,614.60927508)
\lineto(851.04711275,614.21927508)
\curveto(851.04711044,614.07927474)(851.02211046,613.96927485)(850.97211275,613.88927508)
\curveto(850.94211054,613.83927498)(850.89711059,613.80927501)(850.83711275,613.79927508)
\curveto(850.7871107,613.78927503)(850.72211076,613.77427504)(850.64211275,613.75427508)
\lineto(850.43211275,613.75427508)
\lineto(850.11711275,613.75427508)
\curveto(850.01711147,613.76427505)(849.94211154,613.79927502)(849.89211275,613.85927508)
\curveto(849.84211164,613.93927488)(849.81211167,614.03927478)(849.80211275,614.15927508)
\lineto(849.80211275,614.53427508)
\lineto(849.80211275,615.91427508)
\lineto(849.80211275,622.15427508)
\lineto(849.80211275,623.62427508)
\curveto(849.80211168,623.73426508)(849.79711169,623.84926497)(849.78711275,623.96927508)
\curveto(849.7871117,624.09926472)(849.81211167,624.19926462)(849.86211275,624.26927508)
\curveto(849.90211158,624.32926449)(849.97711151,624.37926444)(850.08711275,624.41927508)
\curveto(850.10711138,624.42926439)(850.12711136,624.42926439)(850.14711275,624.41927508)
\curveto(850.17711131,624.4192644)(850.20211128,624.42426439)(850.22211275,624.43427508)
}
}
{
\newrgbcolor{curcolor}{0 0 0}
\pscustom[linestyle=none,fillstyle=solid,fillcolor=curcolor]
{
\newpath
\moveto(859.8769565,614.30927508)
\curveto(859.90694867,614.14927467)(859.89194868,614.0142748)(859.8319565,613.90427508)
\curveto(859.7719488,613.80427501)(859.69194888,613.72927509)(859.5919565,613.67927508)
\curveto(859.54194903,613.65927516)(859.48694909,613.64927517)(859.4269565,613.64927508)
\curveto(859.3769492,613.64927517)(859.32194925,613.63927518)(859.2619565,613.61927508)
\curveto(859.04194953,613.56927525)(858.82194975,613.58427523)(858.6019565,613.66427508)
\curveto(858.39195018,613.73427508)(858.24695033,613.82427499)(858.1669565,613.93427508)
\curveto(858.11695046,614.00427481)(858.0719505,614.08427473)(858.0319565,614.17427508)
\curveto(857.99195058,614.27427454)(857.94195063,614.35427446)(857.8819565,614.41427508)
\curveto(857.86195071,614.43427438)(857.83695074,614.45427436)(857.8069565,614.47427508)
\curveto(857.78695079,614.49427432)(857.75695082,614.49927432)(857.7169565,614.48927508)
\curveto(857.60695097,614.45927436)(857.50195107,614.40427441)(857.4019565,614.32427508)
\curveto(857.31195126,614.24427457)(857.22195135,614.17427464)(857.1319565,614.11427508)
\curveto(857.00195157,614.03427478)(856.86195171,613.95927486)(856.7119565,613.88927508)
\curveto(856.56195201,613.82927499)(856.40195217,613.77427504)(856.2319565,613.72427508)
\curveto(856.13195244,613.69427512)(856.02195255,613.67427514)(855.9019565,613.66427508)
\curveto(855.79195278,613.65427516)(855.68195289,613.63927518)(855.5719565,613.61927508)
\curveto(855.52195305,613.60927521)(855.4769531,613.60427521)(855.4369565,613.60427508)
\lineto(855.3319565,613.60427508)
\curveto(855.22195335,613.58427523)(855.11695346,613.58427523)(855.0169565,613.60427508)
\lineto(854.8819565,613.60427508)
\curveto(854.83195374,613.6142752)(854.78195379,613.6192752)(854.7319565,613.61927508)
\curveto(854.68195389,613.6192752)(854.63695394,613.62927519)(854.5969565,613.64927508)
\curveto(854.55695402,613.65927516)(854.52195405,613.66427515)(854.4919565,613.66427508)
\curveto(854.4719541,613.65427516)(854.44695413,613.65427516)(854.4169565,613.66427508)
\lineto(854.1769565,613.72427508)
\curveto(854.09695448,613.73427508)(854.02195455,613.75427506)(853.9519565,613.78427508)
\curveto(853.65195492,613.9142749)(853.40695517,614.05927476)(853.2169565,614.21927508)
\curveto(853.03695554,614.38927443)(852.88695569,614.62427419)(852.7669565,614.92427508)
\curveto(852.6769559,615.14427367)(852.63195594,615.40927341)(852.6319565,615.71927508)
\lineto(852.6319565,616.03427508)
\curveto(852.64195593,616.08427273)(852.64695593,616.13427268)(852.6469565,616.18427508)
\lineto(852.6769565,616.36427508)
\lineto(852.7969565,616.69427508)
\curveto(852.83695574,616.80427201)(852.88695569,616.90427191)(852.9469565,616.99427508)
\curveto(853.12695545,617.28427153)(853.3719552,617.49927132)(853.6819565,617.63927508)
\curveto(853.99195458,617.77927104)(854.33195424,617.90427091)(854.7019565,618.01427508)
\curveto(854.84195373,618.05427076)(854.98695359,618.08427073)(855.1369565,618.10427508)
\curveto(855.28695329,618.12427069)(855.43695314,618.14927067)(855.5869565,618.17927508)
\curveto(855.65695292,618.19927062)(855.72195285,618.20927061)(855.7819565,618.20927508)
\curveto(855.85195272,618.20927061)(855.92695265,618.2192706)(856.0069565,618.23927508)
\curveto(856.0769525,618.25927056)(856.14695243,618.26927055)(856.2169565,618.26927508)
\curveto(856.28695229,618.27927054)(856.36195221,618.29427052)(856.4419565,618.31427508)
\curveto(856.69195188,618.37427044)(856.92695165,618.42427039)(857.1469565,618.46427508)
\curveto(857.36695121,618.5142703)(857.54195103,618.62927019)(857.6719565,618.80927508)
\curveto(857.73195084,618.88926993)(857.78195079,618.98926983)(857.8219565,619.10927508)
\curveto(857.86195071,619.23926958)(857.86195071,619.37926944)(857.8219565,619.52927508)
\curveto(857.76195081,619.76926905)(857.6719509,619.95926886)(857.5519565,620.09927508)
\curveto(857.44195113,620.23926858)(857.28195129,620.34926847)(857.0719565,620.42927508)
\curveto(856.95195162,620.47926834)(856.80695177,620.5142683)(856.6369565,620.53427508)
\curveto(856.4769521,620.55426826)(856.30695227,620.56426825)(856.1269565,620.56427508)
\curveto(855.94695263,620.56426825)(855.7719528,620.55426826)(855.6019565,620.53427508)
\curveto(855.43195314,620.5142683)(855.28695329,620.48426833)(855.1669565,620.44427508)
\curveto(854.99695358,620.38426843)(854.83195374,620.29926852)(854.6719565,620.18927508)
\curveto(854.59195398,620.12926869)(854.51695406,620.04926877)(854.4469565,619.94927508)
\curveto(854.38695419,619.85926896)(854.33195424,619.75926906)(854.2819565,619.64927508)
\curveto(854.25195432,619.56926925)(854.22195435,619.48426933)(854.1919565,619.39427508)
\curveto(854.1719544,619.30426951)(854.12695445,619.23426958)(854.0569565,619.18427508)
\curveto(854.01695456,619.15426966)(853.94695463,619.12926969)(853.8469565,619.10927508)
\curveto(853.75695482,619.09926972)(853.66195491,619.09426972)(853.5619565,619.09427508)
\curveto(853.46195511,619.09426972)(853.36195521,619.09926972)(853.2619565,619.10927508)
\curveto(853.1719554,619.12926969)(853.10695547,619.15426966)(853.0669565,619.18427508)
\curveto(853.02695555,619.2142696)(852.99695558,619.26426955)(852.9769565,619.33427508)
\curveto(852.95695562,619.40426941)(852.95695562,619.47926934)(852.9769565,619.55927508)
\curveto(853.00695557,619.68926913)(853.03695554,619.80926901)(853.0669565,619.91927508)
\curveto(853.10695547,620.03926878)(853.15195542,620.15426866)(853.2019565,620.26427508)
\curveto(853.39195518,620.6142682)(853.63195494,620.88426793)(853.9219565,621.07427508)
\curveto(854.21195436,621.27426754)(854.571954,621.43426738)(855.0019565,621.55427508)
\curveto(855.10195347,621.57426724)(855.20195337,621.58926723)(855.3019565,621.59927508)
\curveto(855.41195316,621.60926721)(855.52195305,621.62426719)(855.6319565,621.64427508)
\curveto(855.6719529,621.65426716)(855.73695284,621.65426716)(855.8269565,621.64427508)
\curveto(855.91695266,621.64426717)(855.9719526,621.65426716)(855.9919565,621.67427508)
\curveto(856.69195188,621.68426713)(857.30195127,621.60426721)(857.8219565,621.43427508)
\curveto(858.34195023,621.26426755)(858.70694987,620.93926788)(858.9169565,620.45927508)
\curveto(859.00694957,620.25926856)(859.05694952,620.02426879)(859.0669565,619.75427508)
\curveto(859.08694949,619.49426932)(859.09694948,619.2192696)(859.0969565,618.92927508)
\lineto(859.0969565,615.61427508)
\curveto(859.09694948,615.47427334)(859.10194947,615.33927348)(859.1119565,615.20927508)
\curveto(859.12194945,615.07927374)(859.15194942,614.97427384)(859.2019565,614.89427508)
\curveto(859.25194932,614.82427399)(859.31694926,614.77427404)(859.3969565,614.74427508)
\curveto(859.48694909,614.70427411)(859.571949,614.67427414)(859.6519565,614.65427508)
\curveto(859.73194884,614.64427417)(859.79194878,614.59927422)(859.8319565,614.51927508)
\curveto(859.85194872,614.48927433)(859.86194871,614.45927436)(859.8619565,614.42927508)
\curveto(859.86194871,614.39927442)(859.86694871,614.35927446)(859.8769565,614.30927508)
\moveto(857.7319565,615.97427508)
\curveto(857.79195078,616.1142727)(857.82195075,616.27427254)(857.8219565,616.45427508)
\curveto(857.83195074,616.64427217)(857.83695074,616.83927198)(857.8369565,617.03927508)
\curveto(857.83695074,617.14927167)(857.83195074,617.24927157)(857.8219565,617.33927508)
\curveto(857.81195076,617.42927139)(857.7719508,617.49927132)(857.7019565,617.54927508)
\curveto(857.6719509,617.56927125)(857.60195097,617.57927124)(857.4919565,617.57927508)
\curveto(857.4719511,617.55927126)(857.43695114,617.54927127)(857.3869565,617.54927508)
\curveto(857.33695124,617.54927127)(857.29195128,617.53927128)(857.2519565,617.51927508)
\curveto(857.1719514,617.49927132)(857.08195149,617.47927134)(856.9819565,617.45927508)
\lineto(856.6819565,617.39927508)
\curveto(856.65195192,617.39927142)(856.61695196,617.39427142)(856.5769565,617.38427508)
\lineto(856.4719565,617.38427508)
\curveto(856.32195225,617.34427147)(856.15695242,617.3192715)(855.9769565,617.30927508)
\curveto(855.80695277,617.30927151)(855.64695293,617.28927153)(855.4969565,617.24927508)
\curveto(855.41695316,617.22927159)(855.34195323,617.20927161)(855.2719565,617.18927508)
\curveto(855.21195336,617.17927164)(855.14195343,617.16427165)(855.0619565,617.14427508)
\curveto(854.90195367,617.09427172)(854.75195382,617.02927179)(854.6119565,616.94927508)
\curveto(854.4719541,616.87927194)(854.35195422,616.78927203)(854.2519565,616.67927508)
\curveto(854.15195442,616.56927225)(854.0769545,616.43427238)(854.0269565,616.27427508)
\curveto(853.9769546,616.12427269)(853.95695462,615.93927288)(853.9669565,615.71927508)
\curveto(853.96695461,615.6192732)(853.98195459,615.52427329)(854.0119565,615.43427508)
\curveto(854.05195452,615.35427346)(854.09695448,615.27927354)(854.1469565,615.20927508)
\curveto(854.22695435,615.09927372)(854.33195424,615.00427381)(854.4619565,614.92427508)
\curveto(854.59195398,614.85427396)(854.73195384,614.79427402)(854.8819565,614.74427508)
\curveto(854.93195364,614.73427408)(854.98195359,614.72927409)(855.0319565,614.72927508)
\curveto(855.08195349,614.72927409)(855.13195344,614.72427409)(855.1819565,614.71427508)
\curveto(855.25195332,614.69427412)(855.33695324,614.67927414)(855.4369565,614.66927508)
\curveto(855.54695303,614.66927415)(855.63695294,614.67927414)(855.7069565,614.69927508)
\curveto(855.76695281,614.7192741)(855.82695275,614.72427409)(855.8869565,614.71427508)
\curveto(855.94695263,614.7142741)(856.00695257,614.72427409)(856.0669565,614.74427508)
\curveto(856.14695243,614.76427405)(856.22195235,614.77927404)(856.2919565,614.78927508)
\curveto(856.3719522,614.79927402)(856.44695213,614.819274)(856.5169565,614.84927508)
\curveto(856.80695177,614.96927385)(857.05195152,615.1142737)(857.2519565,615.28427508)
\curveto(857.46195111,615.45427336)(857.62195095,615.68427313)(857.7319565,615.97427508)
}
}
{
\newrgbcolor{curcolor}{0 0 0}
\pscustom[linestyle=none,fillstyle=solid,fillcolor=curcolor]
{
\newpath
\moveto(864.18359712,621.65927508)
\curveto(864.92359233,621.66926715)(865.53859172,621.55926726)(866.02859712,621.32927508)
\curveto(866.52859073,621.10926771)(866.92359033,620.77426804)(867.21359712,620.32427508)
\curveto(867.34358991,620.12426869)(867.4535898,619.87926894)(867.54359712,619.58927508)
\curveto(867.56358969,619.53926928)(867.57858968,619.47426934)(867.58859712,619.39427508)
\curveto(867.59858966,619.3142695)(867.59358966,619.24426957)(867.57359712,619.18427508)
\curveto(867.54358971,619.13426968)(867.49358976,619.08926973)(867.42359712,619.04927508)
\curveto(867.39358986,619.02926979)(867.36358989,619.0192698)(867.33359712,619.01927508)
\curveto(867.30358995,619.02926979)(867.26858999,619.02926979)(867.22859712,619.01927508)
\curveto(867.18859007,619.00926981)(867.14859011,619.00426981)(867.10859712,619.00427508)
\curveto(867.06859019,619.0142698)(867.02859023,619.0192698)(866.98859712,619.01927508)
\lineto(866.67359712,619.01927508)
\curveto(866.57359068,619.02926979)(866.48859077,619.05926976)(866.41859712,619.10927508)
\curveto(866.33859092,619.16926965)(866.28359097,619.25426956)(866.25359712,619.36427508)
\curveto(866.22359103,619.47426934)(866.18359107,619.56926925)(866.13359712,619.64927508)
\curveto(865.98359127,619.90926891)(865.78859147,620.1142687)(865.54859712,620.26427508)
\curveto(865.46859179,620.3142685)(865.38359187,620.35426846)(865.29359712,620.38427508)
\curveto(865.20359205,620.42426839)(865.10859215,620.45926836)(865.00859712,620.48927508)
\curveto(864.86859239,620.52926829)(864.68359257,620.54926827)(864.45359712,620.54927508)
\curveto(864.22359303,620.55926826)(864.03359322,620.53926828)(863.88359712,620.48927508)
\curveto(863.81359344,620.46926835)(863.74859351,620.45426836)(863.68859712,620.44427508)
\curveto(863.62859363,620.43426838)(863.56359369,620.4192684)(863.49359712,620.39927508)
\curveto(863.23359402,620.28926853)(863.00359425,620.13926868)(862.80359712,619.94927508)
\curveto(862.60359465,619.75926906)(862.44859481,619.53426928)(862.33859712,619.27427508)
\curveto(862.29859496,619.18426963)(862.26359499,619.08926973)(862.23359712,618.98927508)
\curveto(862.20359505,618.89926992)(862.17359508,618.79927002)(862.14359712,618.68927508)
\lineto(862.05359712,618.28427508)
\curveto(862.04359521,618.23427058)(862.03859522,618.17927064)(862.03859712,618.11927508)
\curveto(862.04859521,618.05927076)(862.04359521,618.00427081)(862.02359712,617.95427508)
\lineto(862.02359712,617.83427508)
\curveto(862.01359524,617.79427102)(862.00359525,617.72927109)(861.99359712,617.63927508)
\curveto(861.99359526,617.54927127)(862.00359525,617.48427133)(862.02359712,617.44427508)
\curveto(862.03359522,617.39427142)(862.03359522,617.34427147)(862.02359712,617.29427508)
\curveto(862.01359524,617.24427157)(862.01359524,617.19427162)(862.02359712,617.14427508)
\curveto(862.03359522,617.10427171)(862.03859522,617.03427178)(862.03859712,616.93427508)
\curveto(862.0585952,616.85427196)(862.07359518,616.76927205)(862.08359712,616.67927508)
\curveto(862.10359515,616.58927223)(862.12359513,616.50427231)(862.14359712,616.42427508)
\curveto(862.253595,616.10427271)(862.37859488,615.82427299)(862.51859712,615.58427508)
\curveto(862.66859459,615.35427346)(862.87359438,615.15427366)(863.13359712,614.98427508)
\curveto(863.22359403,614.93427388)(863.31359394,614.88927393)(863.40359712,614.84927508)
\curveto(863.50359375,614.80927401)(863.60859365,614.76927405)(863.71859712,614.72927508)
\curveto(863.76859349,614.7192741)(863.80859345,614.7142741)(863.83859712,614.71427508)
\curveto(863.86859339,614.7142741)(863.90859335,614.70927411)(863.95859712,614.69927508)
\curveto(863.98859327,614.68927413)(864.03859322,614.68427413)(864.10859712,614.68427508)
\lineto(864.27359712,614.68427508)
\curveto(864.27359298,614.67427414)(864.29359296,614.66927415)(864.33359712,614.66927508)
\curveto(864.3535929,614.67927414)(864.37859288,614.67927414)(864.40859712,614.66927508)
\curveto(864.43859282,614.66927415)(864.46859279,614.67427414)(864.49859712,614.68427508)
\curveto(864.56859269,614.70427411)(864.63359262,614.70927411)(864.69359712,614.69927508)
\curveto(864.76359249,614.69927412)(864.83359242,614.70927411)(864.90359712,614.72927508)
\curveto(865.16359209,614.80927401)(865.38859187,614.90927391)(865.57859712,615.02927508)
\curveto(865.76859149,615.15927366)(865.92859133,615.32427349)(866.05859712,615.52427508)
\curveto(866.10859115,615.60427321)(866.1535911,615.68927313)(866.19359712,615.77927508)
\lineto(866.31359712,616.04927508)
\curveto(866.33359092,616.12927269)(866.3535909,616.20427261)(866.37359712,616.27427508)
\curveto(866.40359085,616.35427246)(866.4535908,616.4192724)(866.52359712,616.46927508)
\curveto(866.5535907,616.49927232)(866.61359064,616.5192723)(866.70359712,616.52927508)
\curveto(866.79359046,616.54927227)(866.88859037,616.55927226)(866.98859712,616.55927508)
\curveto(867.09859016,616.56927225)(867.19859006,616.56927225)(867.28859712,616.55927508)
\curveto(867.38858987,616.54927227)(867.4585898,616.52927229)(867.49859712,616.49927508)
\curveto(867.5585897,616.45927236)(867.59358966,616.39927242)(867.60359712,616.31927508)
\curveto(867.62358963,616.23927258)(867.62358963,616.15427266)(867.60359712,616.06427508)
\curveto(867.5535897,615.9142729)(867.50358975,615.76927305)(867.45359712,615.62927508)
\curveto(867.41358984,615.49927332)(867.3585899,615.36927345)(867.28859712,615.23927508)
\curveto(867.13859012,614.93927388)(866.94859031,614.67427414)(866.71859712,614.44427508)
\curveto(866.49859076,614.2142746)(866.22859103,614.02927479)(865.90859712,613.88927508)
\curveto(865.82859143,613.84927497)(865.74359151,613.814275)(865.65359712,613.78427508)
\curveto(865.56359169,613.76427505)(865.46859179,613.73927508)(865.36859712,613.70927508)
\curveto(865.258592,613.66927515)(865.14859211,613.64927517)(865.03859712,613.64927508)
\curveto(864.92859233,613.63927518)(864.81859244,613.62427519)(864.70859712,613.60427508)
\curveto(864.66859259,613.58427523)(864.62859263,613.57927524)(864.58859712,613.58927508)
\curveto(864.54859271,613.59927522)(864.50859275,613.59927522)(864.46859712,613.58927508)
\lineto(864.33359712,613.58927508)
\lineto(864.09359712,613.58927508)
\curveto(864.02359323,613.57927524)(863.9585933,613.58427523)(863.89859712,613.60427508)
\lineto(863.82359712,613.60427508)
\lineto(863.46359712,613.64927508)
\curveto(863.33359392,613.68927513)(863.20859405,613.72427509)(863.08859712,613.75427508)
\curveto(862.96859429,613.78427503)(862.8535944,613.82427499)(862.74359712,613.87427508)
\curveto(862.38359487,614.03427478)(862.08359517,614.22427459)(861.84359712,614.44427508)
\curveto(861.61359564,614.66427415)(861.39859586,614.93427388)(861.19859712,615.25427508)
\curveto(861.14859611,615.33427348)(861.10359615,615.42427339)(861.06359712,615.52427508)
\lineto(860.94359712,615.82427508)
\curveto(860.89359636,615.93427288)(860.8585964,616.04927277)(860.83859712,616.16927508)
\curveto(860.81859644,616.28927253)(860.79359646,616.40927241)(860.76359712,616.52927508)
\curveto(860.7535965,616.56927225)(860.74859651,616.60927221)(860.74859712,616.64927508)
\curveto(860.74859651,616.68927213)(860.74359651,616.72927209)(860.73359712,616.76927508)
\curveto(860.71359654,616.82927199)(860.70359655,616.89427192)(860.70359712,616.96427508)
\curveto(860.71359654,617.03427178)(860.70859655,617.09927172)(860.68859712,617.15927508)
\lineto(860.68859712,617.30927508)
\curveto(860.67859658,617.35927146)(860.67359658,617.42927139)(860.67359712,617.51927508)
\curveto(860.67359658,617.60927121)(860.67859658,617.67927114)(860.68859712,617.72927508)
\curveto(860.69859656,617.77927104)(860.69859656,617.82427099)(860.68859712,617.86427508)
\curveto(860.68859657,617.90427091)(860.69359656,617.94427087)(860.70359712,617.98427508)
\curveto(860.72359653,618.05427076)(860.72859653,618.12427069)(860.71859712,618.19427508)
\curveto(860.71859654,618.26427055)(860.72859653,618.32927049)(860.74859712,618.38927508)
\curveto(860.78859647,618.55927026)(860.82359643,618.72927009)(860.85359712,618.89927508)
\curveto(860.88359637,619.06926975)(860.92859633,619.22926959)(860.98859712,619.37927508)
\curveto(861.19859606,619.89926892)(861.4535958,620.3192685)(861.75359712,620.63927508)
\curveto(862.0535952,620.95926786)(862.46359479,621.22426759)(862.98359712,621.43427508)
\curveto(863.09359416,621.48426733)(863.21359404,621.5192673)(863.34359712,621.53927508)
\curveto(863.47359378,621.55926726)(863.60859365,621.58426723)(863.74859712,621.61427508)
\curveto(863.81859344,621.62426719)(863.88859337,621.62926719)(863.95859712,621.62927508)
\curveto(864.02859323,621.63926718)(864.10359315,621.64926717)(864.18359712,621.65927508)
}
}
{
\newrgbcolor{curcolor}{0 0 0}
\pscustom[linestyle=none,fillstyle=solid,fillcolor=curcolor]
{
\newpath
\moveto(875.85523775,617.92427508)
\curveto(875.87523006,617.82427099)(875.87523006,617.70927111)(875.85523775,617.57927508)
\curveto(875.84523009,617.45927136)(875.81523012,617.37427144)(875.76523775,617.32427508)
\curveto(875.71523022,617.28427153)(875.6402303,617.25427156)(875.54023775,617.23427508)
\curveto(875.45023049,617.22427159)(875.34523059,617.2192716)(875.22523775,617.21927508)
\lineto(874.86523775,617.21927508)
\curveto(874.74523119,617.22927159)(874.6402313,617.23427158)(874.55023775,617.23427508)
\lineto(870.71023775,617.23427508)
\curveto(870.63023531,617.23427158)(870.55023539,617.22927159)(870.47023775,617.21927508)
\curveto(870.39023555,617.2192716)(870.32523561,617.20427161)(870.27523775,617.17427508)
\curveto(870.2352357,617.15427166)(870.19523574,617.1142717)(870.15523775,617.05427508)
\curveto(870.1352358,617.02427179)(870.11523582,616.97927184)(870.09523775,616.91927508)
\curveto(870.07523586,616.86927195)(870.07523586,616.819272)(870.09523775,616.76927508)
\curveto(870.10523583,616.7192721)(870.11023583,616.67427214)(870.11023775,616.63427508)
\curveto(870.11023583,616.59427222)(870.11523582,616.55427226)(870.12523775,616.51427508)
\curveto(870.14523579,616.43427238)(870.16523577,616.34927247)(870.18523775,616.25927508)
\curveto(870.20523573,616.17927264)(870.2352357,616.09927272)(870.27523775,616.01927508)
\curveto(870.50523543,615.47927334)(870.88523505,615.09427372)(871.41523775,614.86427508)
\curveto(871.47523446,614.83427398)(871.5402344,614.80927401)(871.61023775,614.78927508)
\lineto(871.82023775,614.72927508)
\curveto(871.85023409,614.7192741)(871.90023404,614.7142741)(871.97023775,614.71427508)
\curveto(872.11023383,614.67427414)(872.29523364,614.65427416)(872.52523775,614.65427508)
\curveto(872.75523318,614.65427416)(872.940233,614.67427414)(873.08023775,614.71427508)
\curveto(873.22023272,614.75427406)(873.34523259,614.79427402)(873.45523775,614.83427508)
\curveto(873.57523236,614.88427393)(873.68523225,614.94427387)(873.78523775,615.01427508)
\curveto(873.89523204,615.08427373)(873.99023195,615.16427365)(874.07023775,615.25427508)
\curveto(874.15023179,615.35427346)(874.22023172,615.45927336)(874.28023775,615.56927508)
\curveto(874.3402316,615.66927315)(874.39023155,615.77427304)(874.43023775,615.88427508)
\curveto(874.48023146,615.99427282)(874.56023138,616.07427274)(874.67023775,616.12427508)
\curveto(874.71023123,616.14427267)(874.77523116,616.15927266)(874.86523775,616.16927508)
\curveto(874.95523098,616.17927264)(875.04523089,616.17927264)(875.13523775,616.16927508)
\curveto(875.22523071,616.16927265)(875.31023063,616.16427265)(875.39023775,616.15427508)
\curveto(875.47023047,616.14427267)(875.52523041,616.12427269)(875.55523775,616.09427508)
\curveto(875.65523028,616.02427279)(875.68023026,615.90927291)(875.63023775,615.74927508)
\curveto(875.55023039,615.47927334)(875.44523049,615.23927358)(875.31523775,615.02927508)
\curveto(875.11523082,614.70927411)(874.88523105,614.44427437)(874.62523775,614.23427508)
\curveto(874.37523156,614.03427478)(874.05523188,613.86927495)(873.66523775,613.73927508)
\curveto(873.56523237,613.69927512)(873.46523247,613.67427514)(873.36523775,613.66427508)
\curveto(873.26523267,613.64427517)(873.16023278,613.62427519)(873.05023775,613.60427508)
\curveto(873.00023294,613.59427522)(872.95023299,613.58927523)(872.90023775,613.58927508)
\curveto(872.86023308,613.58927523)(872.81523312,613.58427523)(872.76523775,613.57427508)
\lineto(872.61523775,613.57427508)
\curveto(872.56523337,613.56427525)(872.50523343,613.55927526)(872.43523775,613.55927508)
\curveto(872.37523356,613.55927526)(872.32523361,613.56427525)(872.28523775,613.57427508)
\lineto(872.15023775,613.57427508)
\curveto(872.10023384,613.58427523)(872.05523388,613.58927523)(872.01523775,613.58927508)
\curveto(871.97523396,613.58927523)(871.935234,613.59427522)(871.89523775,613.60427508)
\curveto(871.84523409,613.6142752)(871.79023415,613.62427519)(871.73023775,613.63427508)
\curveto(871.67023427,613.63427518)(871.61523432,613.63927518)(871.56523775,613.64927508)
\curveto(871.47523446,613.66927515)(871.38523455,613.69427512)(871.29523775,613.72427508)
\curveto(871.20523473,613.74427507)(871.12023482,613.76927505)(871.04023775,613.79927508)
\curveto(871.00023494,613.819275)(870.96523497,613.82927499)(870.93523775,613.82927508)
\curveto(870.90523503,613.83927498)(870.87023507,613.85427496)(870.83023775,613.87427508)
\curveto(870.68023526,613.94427487)(870.52023542,614.02927479)(870.35023775,614.12927508)
\curveto(870.06023588,614.3192745)(869.81023613,614.54927427)(869.60023775,614.81927508)
\curveto(869.40023654,615.09927372)(869.23023671,615.40927341)(869.09023775,615.74927508)
\curveto(869.0402369,615.85927296)(869.00023694,615.97427284)(868.97023775,616.09427508)
\curveto(868.95023699,616.2142726)(868.92023702,616.33427248)(868.88023775,616.45427508)
\curveto(868.87023707,616.49427232)(868.86523707,616.52927229)(868.86523775,616.55927508)
\curveto(868.86523707,616.58927223)(868.86023708,616.62927219)(868.85023775,616.67927508)
\curveto(868.83023711,616.75927206)(868.81523712,616.84427197)(868.80523775,616.93427508)
\curveto(868.79523714,617.02427179)(868.78023716,617.1142717)(868.76023775,617.20427508)
\lineto(868.76023775,617.41427508)
\curveto(868.75023719,617.45427136)(868.7402372,617.50927131)(868.73023775,617.57927508)
\curveto(868.73023721,617.65927116)(868.7352372,617.72427109)(868.74523775,617.77427508)
\lineto(868.74523775,617.93927508)
\curveto(868.76523717,617.98927083)(868.77023717,618.03927078)(868.76023775,618.08927508)
\curveto(868.76023718,618.14927067)(868.76523717,618.20427061)(868.77523775,618.25427508)
\curveto(868.81523712,618.4142704)(868.84523709,618.57427024)(868.86523775,618.73427508)
\curveto(868.89523704,618.89426992)(868.940237,619.04426977)(869.00023775,619.18427508)
\curveto(869.05023689,619.29426952)(869.09523684,619.40426941)(869.13523775,619.51427508)
\curveto(869.18523675,619.63426918)(869.2402367,619.74926907)(869.30023775,619.85927508)
\curveto(869.52023642,620.20926861)(869.77023617,620.50926831)(870.05023775,620.75927508)
\curveto(870.33023561,621.0192678)(870.67523526,621.23426758)(871.08523775,621.40427508)
\curveto(871.20523473,621.45426736)(871.32523461,621.48926733)(871.44523775,621.50927508)
\curveto(871.57523436,621.53926728)(871.71023423,621.56926725)(871.85023775,621.59927508)
\curveto(871.90023404,621.60926721)(871.94523399,621.6142672)(871.98523775,621.61427508)
\curveto(872.02523391,621.62426719)(872.07023387,621.62926719)(872.12023775,621.62927508)
\curveto(872.1402338,621.63926718)(872.16523377,621.63926718)(872.19523775,621.62927508)
\curveto(872.22523371,621.6192672)(872.25023369,621.62426719)(872.27023775,621.64427508)
\curveto(872.69023325,621.65426716)(873.05523288,621.60926721)(873.36523775,621.50927508)
\curveto(873.67523226,621.4192674)(873.95523198,621.29426752)(874.20523775,621.13427508)
\curveto(874.25523168,621.1142677)(874.29523164,621.08426773)(874.32523775,621.04427508)
\curveto(874.35523158,621.0142678)(874.39023155,620.98926783)(874.43023775,620.96927508)
\curveto(874.51023143,620.90926791)(874.59023135,620.83926798)(874.67023775,620.75927508)
\curveto(874.76023118,620.67926814)(874.8352311,620.59926822)(874.89523775,620.51927508)
\curveto(875.05523088,620.30926851)(875.19023075,620.10926871)(875.30023775,619.91927508)
\curveto(875.37023057,619.80926901)(875.42523051,619.68926913)(875.46523775,619.55927508)
\curveto(875.50523043,619.42926939)(875.55023039,619.29926952)(875.60023775,619.16927508)
\curveto(875.65023029,619.03926978)(875.68523025,618.90426991)(875.70523775,618.76427508)
\curveto(875.7352302,618.62427019)(875.77023017,618.48427033)(875.81023775,618.34427508)
\curveto(875.82023012,618.27427054)(875.82523011,618.20427061)(875.82523775,618.13427508)
\lineto(875.85523775,617.92427508)
\moveto(874.40023775,618.43427508)
\curveto(874.43023151,618.47427034)(874.45523148,618.52427029)(874.47523775,618.58427508)
\curveto(874.49523144,618.65427016)(874.49523144,618.72427009)(874.47523775,618.79427508)
\curveto(874.41523152,619.0142698)(874.33023161,619.2192696)(874.22023775,619.40927508)
\curveto(874.08023186,619.63926918)(873.92523201,619.83426898)(873.75523775,619.99427508)
\curveto(873.58523235,620.15426866)(873.36523257,620.28926853)(873.09523775,620.39927508)
\curveto(873.02523291,620.4192684)(872.95523298,620.43426838)(872.88523775,620.44427508)
\curveto(872.81523312,620.46426835)(872.7402332,620.48426833)(872.66023775,620.50427508)
\curveto(872.58023336,620.52426829)(872.49523344,620.53426828)(872.40523775,620.53427508)
\lineto(872.15023775,620.53427508)
\curveto(872.12023382,620.5142683)(872.08523385,620.50426831)(872.04523775,620.50427508)
\curveto(872.00523393,620.5142683)(871.97023397,620.5142683)(871.94023775,620.50427508)
\lineto(871.70023775,620.44427508)
\curveto(871.63023431,620.43426838)(871.56023438,620.4192684)(871.49023775,620.39927508)
\curveto(871.20023474,620.27926854)(870.96523497,620.12926869)(870.78523775,619.94927508)
\curveto(870.61523532,619.76926905)(870.46023548,619.54426927)(870.32023775,619.27427508)
\curveto(870.29023565,619.22426959)(870.26023568,619.15926966)(870.23023775,619.07927508)
\curveto(870.20023574,619.00926981)(870.17523576,618.92926989)(870.15523775,618.83927508)
\curveto(870.1352358,618.74927007)(870.13023581,618.66427015)(870.14023775,618.58427508)
\curveto(870.15023579,618.50427031)(870.18523575,618.44427037)(870.24523775,618.40427508)
\curveto(870.32523561,618.34427047)(870.46023548,618.3142705)(870.65023775,618.31427508)
\curveto(870.85023509,618.32427049)(871.02023492,618.32927049)(871.16023775,618.32927508)
\lineto(873.44023775,618.32927508)
\curveto(873.59023235,618.32927049)(873.77023217,618.32427049)(873.98023775,618.31427508)
\curveto(874.19023175,618.3142705)(874.33023161,618.35427046)(874.40023775,618.43427508)
}
}
{
\newrgbcolor{curcolor}{0 0 0}
\pscustom[linestyle=none,fillstyle=solid,fillcolor=curcolor]
{
\newpath
\moveto(879.59187837,621.65927508)
\curveto(880.31187431,621.66926715)(880.9168737,621.58426723)(881.40687837,621.40427508)
\curveto(881.89687272,621.23426758)(882.27687234,620.92926789)(882.54687837,620.48927508)
\curveto(882.616872,620.37926844)(882.67187195,620.26426855)(882.71187837,620.14427508)
\curveto(882.75187187,620.03426878)(882.79187183,619.90926891)(882.83187837,619.76927508)
\curveto(882.85187177,619.69926912)(882.85687176,619.62426919)(882.84687837,619.54427508)
\curveto(882.83687178,619.47426934)(882.8218718,619.4192694)(882.80187837,619.37927508)
\curveto(882.78187184,619.35926946)(882.75687186,619.33926948)(882.72687837,619.31927508)
\curveto(882.69687192,619.30926951)(882.67187195,619.29426952)(882.65187837,619.27427508)
\curveto(882.60187202,619.25426956)(882.55187207,619.24926957)(882.50187837,619.25927508)
\curveto(882.45187217,619.26926955)(882.40187222,619.26926955)(882.35187837,619.25927508)
\curveto(882.27187235,619.23926958)(882.16687245,619.23426958)(882.03687837,619.24427508)
\curveto(881.90687271,619.26426955)(881.8168728,619.28926953)(881.76687837,619.31927508)
\curveto(881.68687293,619.36926945)(881.63187299,619.43426938)(881.60187837,619.51427508)
\curveto(881.58187304,619.60426921)(881.54687307,619.68926913)(881.49687837,619.76927508)
\curveto(881.40687321,619.92926889)(881.28187334,620.07426874)(881.12187837,620.20427508)
\curveto(881.01187361,620.28426853)(880.89187373,620.34426847)(880.76187837,620.38427508)
\curveto(880.63187399,620.42426839)(880.49187413,620.46426835)(880.34187837,620.50427508)
\curveto(880.29187433,620.52426829)(880.24187438,620.52926829)(880.19187837,620.51927508)
\curveto(880.14187448,620.5192683)(880.09187453,620.52426829)(880.04187837,620.53427508)
\curveto(879.98187464,620.55426826)(879.90687471,620.56426825)(879.81687837,620.56427508)
\curveto(879.72687489,620.56426825)(879.65187497,620.55426826)(879.59187837,620.53427508)
\lineto(879.50187837,620.53427508)
\lineto(879.35187837,620.50427508)
\curveto(879.30187532,620.50426831)(879.25187537,620.49926832)(879.20187837,620.48927508)
\curveto(878.94187568,620.42926839)(878.72687589,620.34426847)(878.55687837,620.23427508)
\curveto(878.38687623,620.12426869)(878.27187635,619.93926888)(878.21187837,619.67927508)
\curveto(878.19187643,619.60926921)(878.18687643,619.53926928)(878.19687837,619.46927508)
\curveto(878.2168764,619.39926942)(878.23687638,619.33926948)(878.25687837,619.28927508)
\curveto(878.3168763,619.13926968)(878.38687623,619.02926979)(878.46687837,618.95927508)
\curveto(878.55687606,618.89926992)(878.66687595,618.82926999)(878.79687837,618.74927508)
\curveto(878.95687566,618.64927017)(879.13687548,618.57427024)(879.33687837,618.52427508)
\curveto(879.53687508,618.48427033)(879.73687488,618.43427038)(879.93687837,618.37427508)
\curveto(880.06687455,618.33427048)(880.19687442,618.30427051)(880.32687837,618.28427508)
\curveto(880.45687416,618.26427055)(880.58687403,618.23427058)(880.71687837,618.19427508)
\curveto(880.92687369,618.13427068)(881.13187349,618.07427074)(881.33187837,618.01427508)
\curveto(881.53187309,617.96427085)(881.73187289,617.89927092)(881.93187837,617.81927508)
\lineto(882.08187837,617.75927508)
\curveto(882.13187249,617.73927108)(882.18187244,617.7142711)(882.23187837,617.68427508)
\curveto(882.43187219,617.56427125)(882.60687201,617.42927139)(882.75687837,617.27927508)
\curveto(882.90687171,617.12927169)(883.03187159,616.93927188)(883.13187837,616.70927508)
\curveto(883.15187147,616.63927218)(883.17187145,616.54427227)(883.19187837,616.42427508)
\curveto(883.21187141,616.35427246)(883.2218714,616.27927254)(883.22187837,616.19927508)
\curveto(883.23187139,616.12927269)(883.23687138,616.04927277)(883.23687837,615.95927508)
\lineto(883.23687837,615.80927508)
\curveto(883.2168714,615.73927308)(883.20687141,615.66927315)(883.20687837,615.59927508)
\curveto(883.20687141,615.52927329)(883.19687142,615.45927336)(883.17687837,615.38927508)
\curveto(883.14687147,615.27927354)(883.11187151,615.17427364)(883.07187837,615.07427508)
\curveto(883.03187159,614.97427384)(882.98687163,614.88427393)(882.93687837,614.80427508)
\curveto(882.77687184,614.54427427)(882.57187205,614.33427448)(882.32187837,614.17427508)
\curveto(882.07187255,614.02427479)(881.79187283,613.89427492)(881.48187837,613.78427508)
\curveto(881.39187323,613.75427506)(881.29687332,613.73427508)(881.19687837,613.72427508)
\curveto(881.10687351,613.70427511)(881.0168736,613.67927514)(880.92687837,613.64927508)
\curveto(880.82687379,613.62927519)(880.72687389,613.6192752)(880.62687837,613.61927508)
\curveto(880.52687409,613.6192752)(880.42687419,613.60927521)(880.32687837,613.58927508)
\lineto(880.17687837,613.58927508)
\curveto(880.12687449,613.57927524)(880.05687456,613.57427524)(879.96687837,613.57427508)
\curveto(879.87687474,613.57427524)(879.80687481,613.57927524)(879.75687837,613.58927508)
\lineto(879.59187837,613.58927508)
\curveto(879.53187509,613.60927521)(879.46687515,613.6192752)(879.39687837,613.61927508)
\curveto(879.32687529,613.60927521)(879.26687535,613.6142752)(879.21687837,613.63427508)
\curveto(879.16687545,613.64427517)(879.10187552,613.64927517)(879.02187837,613.64927508)
\lineto(878.78187837,613.70927508)
\curveto(878.71187591,613.7192751)(878.63687598,613.73927508)(878.55687837,613.76927508)
\curveto(878.24687637,613.86927495)(877.97687664,613.99427482)(877.74687837,614.14427508)
\curveto(877.5168771,614.29427452)(877.3168773,614.48927433)(877.14687837,614.72927508)
\curveto(877.05687756,614.85927396)(876.98187764,614.99427382)(876.92187837,615.13427508)
\curveto(876.86187776,615.27427354)(876.80687781,615.42927339)(876.75687837,615.59927508)
\curveto(876.73687788,615.65927316)(876.72687789,615.72927309)(876.72687837,615.80927508)
\curveto(876.73687788,615.89927292)(876.75187787,615.96927285)(876.77187837,616.01927508)
\curveto(876.80187782,616.05927276)(876.85187777,616.09927272)(876.92187837,616.13927508)
\curveto(876.97187765,616.15927266)(877.04187758,616.16927265)(877.13187837,616.16927508)
\curveto(877.2218774,616.17927264)(877.31187731,616.17927264)(877.40187837,616.16927508)
\curveto(877.49187713,616.15927266)(877.57687704,616.14427267)(877.65687837,616.12427508)
\curveto(877.74687687,616.1142727)(877.80687681,616.09927272)(877.83687837,616.07927508)
\curveto(877.90687671,616.02927279)(877.95187667,615.95427286)(877.97187837,615.85427508)
\curveto(878.00187662,615.76427305)(878.03687658,615.67927314)(878.07687837,615.59927508)
\curveto(878.17687644,615.37927344)(878.31187631,615.20927361)(878.48187837,615.08927508)
\curveto(878.60187602,614.99927382)(878.73687588,614.92927389)(878.88687837,614.87927508)
\curveto(879.03687558,614.82927399)(879.19687542,614.77927404)(879.36687837,614.72927508)
\lineto(879.68187837,614.68427508)
\lineto(879.77187837,614.68427508)
\curveto(879.84187478,614.66427415)(879.93187469,614.65427416)(880.04187837,614.65427508)
\curveto(880.16187446,614.65427416)(880.26187436,614.66427415)(880.34187837,614.68427508)
\curveto(880.41187421,614.68427413)(880.46687415,614.68927413)(880.50687837,614.69927508)
\curveto(880.56687405,614.70927411)(880.62687399,614.7142741)(880.68687837,614.71427508)
\curveto(880.74687387,614.72427409)(880.80187382,614.73427408)(880.85187837,614.74427508)
\curveto(881.14187348,614.82427399)(881.37187325,614.92927389)(881.54187837,615.05927508)
\curveto(881.71187291,615.18927363)(881.83187279,615.40927341)(881.90187837,615.71927508)
\curveto(881.9218727,615.76927305)(881.92687269,615.82427299)(881.91687837,615.88427508)
\curveto(881.90687271,615.94427287)(881.89687272,615.98927283)(881.88687837,616.01927508)
\curveto(881.83687278,616.20927261)(881.76687285,616.34927247)(881.67687837,616.43927508)
\curveto(881.58687303,616.53927228)(881.47187315,616.62927219)(881.33187837,616.70927508)
\curveto(881.24187338,616.76927205)(881.14187348,616.819272)(881.03187837,616.85927508)
\lineto(880.70187837,616.97927508)
\curveto(880.67187395,616.98927183)(880.64187398,616.99427182)(880.61187837,616.99427508)
\curveto(880.59187403,616.99427182)(880.56687405,617.00427181)(880.53687837,617.02427508)
\curveto(880.19687442,617.13427168)(879.84187478,617.2142716)(879.47187837,617.26427508)
\curveto(879.11187551,617.32427149)(878.77187585,617.4192714)(878.45187837,617.54927508)
\curveto(878.35187627,617.58927123)(878.25687636,617.62427119)(878.16687837,617.65427508)
\curveto(878.07687654,617.68427113)(877.99187663,617.72427109)(877.91187837,617.77427508)
\curveto(877.7218769,617.88427093)(877.54687707,618.00927081)(877.38687837,618.14927508)
\curveto(877.22687739,618.28927053)(877.10187752,618.46427035)(877.01187837,618.67427508)
\curveto(876.98187764,618.74427007)(876.95687766,618.81427)(876.93687837,618.88427508)
\curveto(876.92687769,618.95426986)(876.91187771,619.02926979)(876.89187837,619.10927508)
\curveto(876.86187776,619.22926959)(876.85187777,619.36426945)(876.86187837,619.51427508)
\curveto(876.87187775,619.67426914)(876.88687773,619.80926901)(876.90687837,619.91927508)
\curveto(876.92687769,619.96926885)(876.93687768,620.00926881)(876.93687837,620.03927508)
\curveto(876.94687767,620.07926874)(876.96187766,620.1192687)(876.98187837,620.15927508)
\curveto(877.07187755,620.38926843)(877.19187743,620.58926823)(877.34187837,620.75927508)
\curveto(877.50187712,620.92926789)(877.68187694,621.07926774)(877.88187837,621.20927508)
\curveto(878.03187659,621.29926752)(878.19687642,621.36926745)(878.37687837,621.41927508)
\curveto(878.55687606,621.47926734)(878.74687587,621.53426728)(878.94687837,621.58427508)
\curveto(879.0168756,621.59426722)(879.08187554,621.60426721)(879.14187837,621.61427508)
\curveto(879.21187541,621.62426719)(879.28687533,621.63426718)(879.36687837,621.64427508)
\curveto(879.39687522,621.65426716)(879.43687518,621.65426716)(879.48687837,621.64427508)
\curveto(879.53687508,621.63426718)(879.57187505,621.63926718)(879.59187837,621.65927508)
}
}
{
\newrgbcolor{curcolor}{0.60000002 0.60000002 0.60000002}
\pscustom[linestyle=none,fillstyle=solid,fillcolor=curcolor]
{
\newpath
\moveto(812.80437349,624.4643117)
\lineto(827.80437349,624.4643117)
\lineto(827.80437349,609.4643117)
\lineto(812.80437349,609.4643117)
\closepath
}
}
{
\newrgbcolor{curcolor}{0 0 0}
\pscustom[linestyle=none,fillstyle=solid,fillcolor=curcolor]
{
\newpath
\moveto(832.8075815,601.39856951)
\lineto(837.4425815,601.39856951)
\lineto(838.6575815,601.39856951)
\curveto(838.76757395,601.39855882)(838.87257384,601.39855882)(838.9725815,601.39856951)
\curveto(839.08257363,601.39855882)(839.16757355,601.37855884)(839.2275815,601.33856951)
\curveto(839.30757341,601.28855893)(839.35257336,601.213559)(839.3625815,601.11356951)
\curveto(839.38257333,601.02355919)(839.39257332,600.9135593)(839.3925815,600.78356951)
\lineto(839.3925815,600.63356951)
\curveto(839.40257331,600.59355962)(839.39757332,600.55355966)(839.3775815,600.51356951)
\curveto(839.33757338,600.35355986)(839.24757347,600.26355995)(839.1075815,600.24356951)
\curveto(838.97757374,600.23355998)(838.8125739,600.22855999)(838.6125815,600.22856951)
\lineto(837.0525815,600.22856951)
\lineto(834.8475815,600.22856951)
\lineto(834.3375815,600.22856951)
\curveto(834.15757856,600.23855998)(834.02257869,600.20856001)(833.9325815,600.13856951)
\curveto(833.84257887,600.07856014)(833.79257892,599.97356024)(833.7825815,599.82356951)
\lineto(833.7825815,599.37356951)
\lineto(833.7825815,597.88856951)
\curveto(833.78257893,597.80856241)(833.77757894,597.70856251)(833.7675815,597.58856951)
\curveto(833.76757895,597.46856275)(833.77757894,597.36856285)(833.7975815,597.28856951)
\lineto(833.7975815,597.16856951)
\curveto(833.8175789,597.10856311)(833.83257888,597.04856317)(833.8425815,596.98856951)
\curveto(833.86257885,596.93856328)(833.89757882,596.89856332)(833.9475815,596.86856951)
\curveto(834.03757868,596.80856341)(834.17757854,596.77856344)(834.3675815,596.77856951)
\curveto(834.55757816,596.78856343)(834.72257799,596.79356342)(834.8625815,596.79356951)
\lineto(837.5625815,596.79356951)
\lineto(837.8475815,596.79356951)
\curveto(837.95757476,596.80356341)(838.06257465,596.80356341)(838.1625815,596.79356951)
\curveto(838.27257444,596.79356342)(838.36757435,596.78356343)(838.4475815,596.76356951)
\curveto(838.53757418,596.74356347)(838.59757412,596.70856351)(838.6275815,596.65856951)
\curveto(838.67757404,596.59856362)(838.70257401,596.52356369)(838.7025815,596.43356951)
\lineto(838.7025815,596.13356951)
\lineto(838.7025815,595.96856951)
\curveto(838.70257401,595.9185643)(838.69257402,595.87356434)(838.6725815,595.83356951)
\curveto(838.63257408,595.73356448)(838.57757414,595.67856454)(838.5075815,595.66856951)
\curveto(838.46757425,595.64856457)(838.42757429,595.63856458)(838.3875815,595.63856951)
\curveto(838.35757436,595.63856458)(838.3175744,595.63356458)(838.2675815,595.62356951)
\curveto(838.22757449,595.6135646)(838.18257453,595.60856461)(838.1325815,595.60856951)
\curveto(838.09257462,595.6185646)(838.05257466,595.62356459)(838.0125815,595.62356951)
\lineto(837.4875815,595.62356951)
\lineto(834.9525815,595.62356951)
\lineto(834.3825815,595.62356951)
\curveto(834.17257854,595.63356458)(834.02257869,595.60356461)(833.9325815,595.53356951)
\curveto(833.88257883,595.49356472)(833.84257887,595.42856479)(833.8125815,595.33856951)
\curveto(833.79257892,595.25856496)(833.77757894,595.16356505)(833.7675815,595.05356951)
\lineto(833.7675815,594.70856951)
\curveto(833.77757894,594.59856562)(833.78257893,594.49856572)(833.7825815,594.40856951)
\lineto(833.7825815,591.82856951)
\curveto(833.78257893,591.65856856)(833.78757893,591.47356874)(833.7975815,591.27356951)
\curveto(833.80757891,591.07356914)(833.77257894,590.92356929)(833.6925815,590.82356951)
\curveto(833.66257905,590.78356943)(833.6175791,590.75856946)(833.5575815,590.74856951)
\curveto(833.49757922,590.74856947)(833.43757928,590.73856948)(833.3775815,590.71856951)
\lineto(833.0925815,590.71856951)
\curveto(832.95257976,590.7185695)(832.82257989,590.72356949)(832.7025815,590.73356951)
\curveto(832.58258013,590.74356947)(832.49758022,590.79356942)(832.4475815,590.88356951)
\curveto(832.40758031,590.94356927)(832.38758033,591.02356919)(832.3875815,591.12356951)
\lineto(832.3875815,591.45356951)
\lineto(832.3875815,592.65356951)
\lineto(832.3875815,598.92356951)
\lineto(832.3875815,600.54356951)
\curveto(832.38758033,600.65355956)(832.38258033,600.77355944)(832.3725815,600.90356951)
\curveto(832.37258034,601.04355917)(832.39758032,601.15355906)(832.4475815,601.23356951)
\curveto(832.48758023,601.29355892)(832.56258015,601.34355887)(832.6725815,601.38356951)
\curveto(832.69258002,601.39355882)(832.71258,601.39355882)(832.7325815,601.38356951)
\curveto(832.76257995,601.38355883)(832.78757993,601.38855883)(832.8075815,601.39856951)
}
}
{
\newrgbcolor{curcolor}{0 0 0}
\pscustom[linestyle=none,fillstyle=solid,fillcolor=curcolor]
{
\newpath
\moveto(847.87086275,594.91856951)
\curveto(847.89085469,594.85856536)(847.90085468,594.76356545)(847.90086275,594.63356951)
\curveto(847.90085468,594.5135657)(847.89585468,594.42856579)(847.88586275,594.37856951)
\lineto(847.88586275,594.22856951)
\curveto(847.8758547,594.14856607)(847.86585471,594.07356614)(847.85586275,594.00356951)
\curveto(847.85585472,593.94356627)(847.85085473,593.87356634)(847.84086275,593.79356951)
\curveto(847.82085476,593.73356648)(847.80585477,593.67356654)(847.79586275,593.61356951)
\curveto(847.79585478,593.55356666)(847.78585479,593.49356672)(847.76586275,593.43356951)
\curveto(847.72585485,593.30356691)(847.69085489,593.17356704)(847.66086275,593.04356951)
\curveto(847.63085495,592.9135673)(847.59085499,592.79356742)(847.54086275,592.68356951)
\curveto(847.33085525,592.20356801)(847.05085553,591.79856842)(846.70086275,591.46856951)
\curveto(846.35085623,591.14856907)(845.92085666,590.90356931)(845.41086275,590.73356951)
\curveto(845.30085728,590.69356952)(845.1808574,590.66356955)(845.05086275,590.64356951)
\curveto(844.93085765,590.62356959)(844.80585777,590.60356961)(844.67586275,590.58356951)
\curveto(844.61585796,590.57356964)(844.55085803,590.56856965)(844.48086275,590.56856951)
\curveto(844.42085816,590.55856966)(844.36085822,590.55356966)(844.30086275,590.55356951)
\curveto(844.26085832,590.54356967)(844.20085838,590.53856968)(844.12086275,590.53856951)
\curveto(844.05085853,590.53856968)(844.00085858,590.54356967)(843.97086275,590.55356951)
\curveto(843.93085865,590.56356965)(843.89085869,590.56856965)(843.85086275,590.56856951)
\curveto(843.81085877,590.55856966)(843.7758588,590.55856966)(843.74586275,590.56856951)
\lineto(843.65586275,590.56856951)
\lineto(843.29586275,590.61356951)
\curveto(843.15585942,590.65356956)(843.02085956,590.69356952)(842.89086275,590.73356951)
\curveto(842.76085982,590.77356944)(842.63585994,590.8185694)(842.51586275,590.86856951)
\curveto(842.06586051,591.06856915)(841.69586088,591.32856889)(841.40586275,591.64856951)
\curveto(841.11586146,591.96856825)(840.8758617,592.35856786)(840.68586275,592.81856951)
\curveto(840.63586194,592.9185673)(840.59586198,593.0185672)(840.56586275,593.11856951)
\curveto(840.54586203,593.218567)(840.52586205,593.32356689)(840.50586275,593.43356951)
\curveto(840.48586209,593.47356674)(840.4758621,593.50356671)(840.47586275,593.52356951)
\curveto(840.48586209,593.55356666)(840.48586209,593.58856663)(840.47586275,593.62856951)
\curveto(840.45586212,593.70856651)(840.44086214,593.78856643)(840.43086275,593.86856951)
\curveto(840.43086215,593.95856626)(840.42086216,594.04356617)(840.40086275,594.12356951)
\lineto(840.40086275,594.24356951)
\curveto(840.40086218,594.28356593)(840.39586218,594.32856589)(840.38586275,594.37856951)
\curveto(840.3758622,594.42856579)(840.37086221,594.5135657)(840.37086275,594.63356951)
\curveto(840.37086221,594.76356545)(840.3808622,594.85856536)(840.40086275,594.91856951)
\curveto(840.42086216,594.98856523)(840.42586215,595.05856516)(840.41586275,595.12856951)
\curveto(840.40586217,595.19856502)(840.41086217,595.26856495)(840.43086275,595.33856951)
\curveto(840.44086214,595.38856483)(840.44586213,595.42856479)(840.44586275,595.45856951)
\curveto(840.45586212,595.49856472)(840.46586211,595.54356467)(840.47586275,595.59356951)
\curveto(840.50586207,595.7135645)(840.53086205,595.83356438)(840.55086275,595.95356951)
\curveto(840.580862,596.07356414)(840.62086196,596.18856403)(840.67086275,596.29856951)
\curveto(840.82086176,596.66856355)(841.00086158,596.99856322)(841.21086275,597.28856951)
\curveto(841.43086115,597.58856263)(841.69586088,597.83856238)(842.00586275,598.03856951)
\curveto(842.12586045,598.1185621)(842.25086033,598.18356203)(842.38086275,598.23356951)
\curveto(842.51086007,598.29356192)(842.64585993,598.35356186)(842.78586275,598.41356951)
\curveto(842.90585967,598.46356175)(843.03585954,598.49356172)(843.17586275,598.50356951)
\curveto(843.31585926,598.52356169)(843.45585912,598.55356166)(843.59586275,598.59356951)
\lineto(843.79086275,598.59356951)
\curveto(843.86085872,598.60356161)(843.92585865,598.6135616)(843.98586275,598.62356951)
\curveto(844.8758577,598.63356158)(845.61585696,598.44856177)(846.20586275,598.06856951)
\curveto(846.79585578,597.68856253)(847.22085536,597.19356302)(847.48086275,596.58356951)
\curveto(847.53085505,596.48356373)(847.57085501,596.38356383)(847.60086275,596.28356951)
\curveto(847.63085495,596.18356403)(847.66585491,596.07856414)(847.70586275,595.96856951)
\curveto(847.73585484,595.85856436)(847.76085482,595.73856448)(847.78086275,595.60856951)
\curveto(847.80085478,595.48856473)(847.82585475,595.36356485)(847.85586275,595.23356951)
\curveto(847.86585471,595.18356503)(847.86585471,595.12856509)(847.85586275,595.06856951)
\curveto(847.85585472,595.0185652)(847.86085472,594.96856525)(847.87086275,594.91856951)
\moveto(846.53586275,594.06356951)
\curveto(846.55585602,594.13356608)(846.56085602,594.213566)(846.55086275,594.30356951)
\lineto(846.55086275,594.55856951)
\curveto(846.55085603,594.94856527)(846.51585606,595.27856494)(846.44586275,595.54856951)
\curveto(846.41585616,595.62856459)(846.39085619,595.70856451)(846.37086275,595.78856951)
\curveto(846.35085623,595.86856435)(846.32585625,595.94356427)(846.29586275,596.01356951)
\curveto(846.01585656,596.66356355)(845.57085701,597.1135631)(844.96086275,597.36356951)
\curveto(844.89085769,597.39356282)(844.81585776,597.4135628)(844.73586275,597.42356951)
\lineto(844.49586275,597.48356951)
\curveto(844.41585816,597.50356271)(844.33085825,597.5135627)(844.24086275,597.51356951)
\lineto(843.97086275,597.51356951)
\lineto(843.70086275,597.46856951)
\curveto(843.60085898,597.44856277)(843.50585907,597.42356279)(843.41586275,597.39356951)
\curveto(843.33585924,597.37356284)(843.25585932,597.34356287)(843.17586275,597.30356951)
\curveto(843.10585947,597.28356293)(843.04085954,597.25356296)(842.98086275,597.21356951)
\curveto(842.92085966,597.17356304)(842.86585971,597.13356308)(842.81586275,597.09356951)
\curveto(842.57586,596.92356329)(842.3808602,596.7185635)(842.23086275,596.47856951)
\curveto(842.0808605,596.23856398)(841.95086063,595.95856426)(841.84086275,595.63856951)
\curveto(841.81086077,595.53856468)(841.79086079,595.43356478)(841.78086275,595.32356951)
\curveto(841.77086081,595.22356499)(841.75586082,595.1185651)(841.73586275,595.00856951)
\curveto(841.72586085,594.96856525)(841.72086086,594.90356531)(841.72086275,594.81356951)
\curveto(841.71086087,594.78356543)(841.70586087,594.74856547)(841.70586275,594.70856951)
\curveto(841.71586086,594.66856555)(841.72086086,594.62356559)(841.72086275,594.57356951)
\lineto(841.72086275,594.27356951)
\curveto(841.72086086,594.17356604)(841.73086085,594.08356613)(841.75086275,594.00356951)
\lineto(841.78086275,593.82356951)
\curveto(841.80086078,593.72356649)(841.81586076,593.62356659)(841.82586275,593.52356951)
\curveto(841.84586073,593.43356678)(841.8758607,593.34856687)(841.91586275,593.26856951)
\curveto(842.01586056,593.02856719)(842.13086045,592.80356741)(842.26086275,592.59356951)
\curveto(842.40086018,592.38356783)(842.57086001,592.20856801)(842.77086275,592.06856951)
\curveto(842.82085976,592.03856818)(842.86585971,592.0135682)(842.90586275,591.99356951)
\curveto(842.94585963,591.97356824)(842.99085959,591.94856827)(843.04086275,591.91856951)
\curveto(843.12085946,591.86856835)(843.20585937,591.82356839)(843.29586275,591.78356951)
\curveto(843.39585918,591.75356846)(843.50085908,591.72356849)(843.61086275,591.69356951)
\curveto(843.66085892,591.67356854)(843.70585887,591.66356855)(843.74586275,591.66356951)
\curveto(843.79585878,591.67356854)(843.84585873,591.67356854)(843.89586275,591.66356951)
\curveto(843.92585865,591.65356856)(843.98585859,591.64356857)(844.07586275,591.63356951)
\curveto(844.1758584,591.62356859)(844.25085833,591.62856859)(844.30086275,591.64856951)
\curveto(844.34085824,591.65856856)(844.3808582,591.65856856)(844.42086275,591.64856951)
\curveto(844.46085812,591.64856857)(844.50085808,591.65856856)(844.54086275,591.67856951)
\curveto(844.62085796,591.69856852)(844.70085788,591.7135685)(844.78086275,591.72356951)
\curveto(844.86085772,591.74356847)(844.93585764,591.76856845)(845.00586275,591.79856951)
\curveto(845.34585723,591.93856828)(845.62085696,592.13356808)(845.83086275,592.38356951)
\curveto(846.04085654,592.63356758)(846.21585636,592.92856729)(846.35586275,593.26856951)
\curveto(846.40585617,593.38856683)(846.43585614,593.5135667)(846.44586275,593.64356951)
\curveto(846.46585611,593.78356643)(846.49585608,593.92356629)(846.53586275,594.06356951)
}
}
{
\newrgbcolor{curcolor}{0 0 0}
\pscustom[linestyle=none,fillstyle=solid,fillcolor=curcolor]
{
\newpath
\moveto(850.304144,600.78356951)
\curveto(850.45414199,600.78355943)(850.60414184,600.77855944)(850.754144,600.76856951)
\curveto(850.90414154,600.76855945)(851.00914143,600.72855949)(851.069144,600.64856951)
\curveto(851.11914132,600.58855963)(851.1441413,600.50355971)(851.144144,600.39356951)
\curveto(851.15414129,600.29355992)(851.15914128,600.18856003)(851.159144,600.07856951)
\lineto(851.159144,599.20856951)
\curveto(851.15914128,599.12856109)(851.15414129,599.04356117)(851.144144,598.95356951)
\curveto(851.1441413,598.87356134)(851.15414129,598.80356141)(851.174144,598.74356951)
\curveto(851.21414123,598.60356161)(851.30414114,598.5135617)(851.444144,598.47356951)
\curveto(851.49414095,598.46356175)(851.5391409,598.45856176)(851.579144,598.45856951)
\lineto(851.729144,598.45856951)
\lineto(852.134144,598.45856951)
\curveto(852.29414015,598.46856175)(852.40914003,598.45856176)(852.479144,598.42856951)
\curveto(852.56913987,598.36856185)(852.62913981,598.30856191)(852.659144,598.24856951)
\curveto(852.67913976,598.20856201)(852.68913975,598.16356205)(852.689144,598.11356951)
\lineto(852.689144,597.96356951)
\curveto(852.68913975,597.85356236)(852.68413976,597.74856247)(852.674144,597.64856951)
\curveto(852.66413978,597.55856266)(852.62913981,597.48856273)(852.569144,597.43856951)
\curveto(852.50913993,597.38856283)(852.42414002,597.35856286)(852.314144,597.34856951)
\lineto(851.984144,597.34856951)
\curveto(851.87414057,597.35856286)(851.76414068,597.36356285)(851.654144,597.36356951)
\curveto(851.5441409,597.36356285)(851.44914099,597.34856287)(851.369144,597.31856951)
\curveto(851.29914114,597.28856293)(851.24914119,597.23856298)(851.219144,597.16856951)
\curveto(851.18914125,597.09856312)(851.16914127,597.0135632)(851.159144,596.91356951)
\curveto(851.14914129,596.82356339)(851.1441413,596.72356349)(851.144144,596.61356951)
\curveto(851.15414129,596.5135637)(851.15914128,596.4135638)(851.159144,596.31356951)
\lineto(851.159144,593.34356951)
\curveto(851.15914128,593.12356709)(851.15414129,592.88856733)(851.144144,592.63856951)
\curveto(851.1441413,592.39856782)(851.18914125,592.213568)(851.279144,592.08356951)
\curveto(851.32914111,592.00356821)(851.39414105,591.94856827)(851.474144,591.91856951)
\curveto(851.55414089,591.88856833)(851.64914079,591.86356835)(851.759144,591.84356951)
\curveto(851.78914065,591.83356838)(851.81914062,591.82856839)(851.849144,591.82856951)
\curveto(851.88914055,591.83856838)(851.92414052,591.83856838)(851.954144,591.82856951)
\lineto(852.149144,591.82856951)
\curveto(852.24914019,591.82856839)(852.3391401,591.8185684)(852.419144,591.79856951)
\curveto(852.50913993,591.78856843)(852.57413987,591.75356846)(852.614144,591.69356951)
\curveto(852.63413981,591.66356855)(852.64913979,591.60856861)(852.659144,591.52856951)
\curveto(852.67913976,591.45856876)(852.68913975,591.38356883)(852.689144,591.30356951)
\curveto(852.69913974,591.22356899)(852.69913974,591.14356907)(852.689144,591.06356951)
\curveto(852.67913976,590.99356922)(852.65913978,590.93856928)(852.629144,590.89856951)
\curveto(852.58913985,590.82856939)(852.51413993,590.77856944)(852.404144,590.74856951)
\curveto(852.32414012,590.72856949)(852.23414021,590.7185695)(852.134144,590.71856951)
\curveto(852.03414041,590.72856949)(851.9441405,590.73356948)(851.864144,590.73356951)
\curveto(851.80414064,590.73356948)(851.7441407,590.72856949)(851.684144,590.71856951)
\curveto(851.62414082,590.7185695)(851.56914087,590.72356949)(851.519144,590.73356951)
\lineto(851.339144,590.73356951)
\curveto(851.28914115,590.74356947)(851.2391412,590.74856947)(851.189144,590.74856951)
\curveto(851.14914129,590.75856946)(851.10414134,590.76356945)(851.054144,590.76356951)
\curveto(850.85414159,590.8135694)(850.67914176,590.86856935)(850.529144,590.92856951)
\curveto(850.38914205,590.98856923)(850.26914217,591.09356912)(850.169144,591.24356951)
\curveto(850.02914241,591.44356877)(849.94914249,591.69356852)(849.929144,591.99356951)
\curveto(849.90914253,592.30356791)(849.89914254,592.63356758)(849.899144,592.98356951)
\lineto(849.899144,596.91356951)
\curveto(849.86914257,597.04356317)(849.8391426,597.13856308)(849.809144,597.19856951)
\curveto(849.78914265,597.25856296)(849.71914272,597.30856291)(849.599144,597.34856951)
\curveto(849.55914288,597.35856286)(849.51914292,597.35856286)(849.479144,597.34856951)
\curveto(849.439143,597.33856288)(849.39914304,597.34356287)(849.359144,597.36356951)
\lineto(849.119144,597.36356951)
\curveto(848.98914345,597.36356285)(848.87914356,597.37356284)(848.789144,597.39356951)
\curveto(848.70914373,597.42356279)(848.65414379,597.48356273)(848.624144,597.57356951)
\curveto(848.60414384,597.6135626)(848.58914385,597.65856256)(848.579144,597.70856951)
\lineto(848.579144,597.85856951)
\curveto(848.57914386,597.99856222)(848.58914385,598.1135621)(848.609144,598.20356951)
\curveto(848.62914381,598.30356191)(848.68914375,598.37856184)(848.789144,598.42856951)
\curveto(848.89914354,598.46856175)(849.0391434,598.47856174)(849.209144,598.45856951)
\curveto(849.38914305,598.43856178)(849.5391429,598.44856177)(849.659144,598.48856951)
\curveto(849.74914269,598.53856168)(849.81914262,598.60856161)(849.869144,598.69856951)
\curveto(849.88914255,598.75856146)(849.89914254,598.83356138)(849.899144,598.92356951)
\lineto(849.899144,599.17856951)
\lineto(849.899144,600.10856951)
\lineto(849.899144,600.34856951)
\curveto(849.89914254,600.43855978)(849.90914253,600.5135597)(849.929144,600.57356951)
\curveto(849.96914247,600.65355956)(850.0441424,600.7185595)(850.154144,600.76856951)
\curveto(850.18414226,600.76855945)(850.20914223,600.76855945)(850.229144,600.76856951)
\curveto(850.25914218,600.77855944)(850.28414216,600.78355943)(850.304144,600.78356951)
}
}
{
\newrgbcolor{curcolor}{0 0 0}
\pscustom[linestyle=none,fillstyle=solid,fillcolor=curcolor]
{
\newpath
\moveto(861.20094087,594.91856951)
\curveto(861.22093281,594.85856536)(861.2309328,594.76356545)(861.23094087,594.63356951)
\curveto(861.2309328,594.5135657)(861.22593281,594.42856579)(861.21594087,594.37856951)
\lineto(861.21594087,594.22856951)
\curveto(861.20593283,594.14856607)(861.19593284,594.07356614)(861.18594087,594.00356951)
\curveto(861.18593285,593.94356627)(861.18093285,593.87356634)(861.17094087,593.79356951)
\curveto(861.15093288,593.73356648)(861.1359329,593.67356654)(861.12594087,593.61356951)
\curveto(861.12593291,593.55356666)(861.11593292,593.49356672)(861.09594087,593.43356951)
\curveto(861.05593298,593.30356691)(861.02093301,593.17356704)(860.99094087,593.04356951)
\curveto(860.96093307,592.9135673)(860.92093311,592.79356742)(860.87094087,592.68356951)
\curveto(860.66093337,592.20356801)(860.38093365,591.79856842)(860.03094087,591.46856951)
\curveto(859.68093435,591.14856907)(859.25093478,590.90356931)(858.74094087,590.73356951)
\curveto(858.6309354,590.69356952)(858.51093552,590.66356955)(858.38094087,590.64356951)
\curveto(858.26093577,590.62356959)(858.1359359,590.60356961)(858.00594087,590.58356951)
\curveto(857.94593609,590.57356964)(857.88093615,590.56856965)(857.81094087,590.56856951)
\curveto(857.75093628,590.55856966)(857.69093634,590.55356966)(857.63094087,590.55356951)
\curveto(857.59093644,590.54356967)(857.5309365,590.53856968)(857.45094087,590.53856951)
\curveto(857.38093665,590.53856968)(857.3309367,590.54356967)(857.30094087,590.55356951)
\curveto(857.26093677,590.56356965)(857.22093681,590.56856965)(857.18094087,590.56856951)
\curveto(857.14093689,590.55856966)(857.10593693,590.55856966)(857.07594087,590.56856951)
\lineto(856.98594087,590.56856951)
\lineto(856.62594087,590.61356951)
\curveto(856.48593755,590.65356956)(856.35093768,590.69356952)(856.22094087,590.73356951)
\curveto(856.09093794,590.77356944)(855.96593807,590.8185694)(855.84594087,590.86856951)
\curveto(855.39593864,591.06856915)(855.02593901,591.32856889)(854.73594087,591.64856951)
\curveto(854.44593959,591.96856825)(854.20593983,592.35856786)(854.01594087,592.81856951)
\curveto(853.96594007,592.9185673)(853.92594011,593.0185672)(853.89594087,593.11856951)
\curveto(853.87594016,593.218567)(853.85594018,593.32356689)(853.83594087,593.43356951)
\curveto(853.81594022,593.47356674)(853.80594023,593.50356671)(853.80594087,593.52356951)
\curveto(853.81594022,593.55356666)(853.81594022,593.58856663)(853.80594087,593.62856951)
\curveto(853.78594025,593.70856651)(853.77094026,593.78856643)(853.76094087,593.86856951)
\curveto(853.76094027,593.95856626)(853.75094028,594.04356617)(853.73094087,594.12356951)
\lineto(853.73094087,594.24356951)
\curveto(853.7309403,594.28356593)(853.72594031,594.32856589)(853.71594087,594.37856951)
\curveto(853.70594033,594.42856579)(853.70094033,594.5135657)(853.70094087,594.63356951)
\curveto(853.70094033,594.76356545)(853.71094032,594.85856536)(853.73094087,594.91856951)
\curveto(853.75094028,594.98856523)(853.75594028,595.05856516)(853.74594087,595.12856951)
\curveto(853.7359403,595.19856502)(853.74094029,595.26856495)(853.76094087,595.33856951)
\curveto(853.77094026,595.38856483)(853.77594026,595.42856479)(853.77594087,595.45856951)
\curveto(853.78594025,595.49856472)(853.79594024,595.54356467)(853.80594087,595.59356951)
\curveto(853.8359402,595.7135645)(853.86094017,595.83356438)(853.88094087,595.95356951)
\curveto(853.91094012,596.07356414)(853.95094008,596.18856403)(854.00094087,596.29856951)
\curveto(854.15093988,596.66856355)(854.3309397,596.99856322)(854.54094087,597.28856951)
\curveto(854.76093927,597.58856263)(855.02593901,597.83856238)(855.33594087,598.03856951)
\curveto(855.45593858,598.1185621)(855.58093845,598.18356203)(855.71094087,598.23356951)
\curveto(855.84093819,598.29356192)(855.97593806,598.35356186)(856.11594087,598.41356951)
\curveto(856.2359378,598.46356175)(856.36593767,598.49356172)(856.50594087,598.50356951)
\curveto(856.64593739,598.52356169)(856.78593725,598.55356166)(856.92594087,598.59356951)
\lineto(857.12094087,598.59356951)
\curveto(857.19093684,598.60356161)(857.25593678,598.6135616)(857.31594087,598.62356951)
\curveto(858.20593583,598.63356158)(858.94593509,598.44856177)(859.53594087,598.06856951)
\curveto(860.12593391,597.68856253)(860.55093348,597.19356302)(860.81094087,596.58356951)
\curveto(860.86093317,596.48356373)(860.90093313,596.38356383)(860.93094087,596.28356951)
\curveto(860.96093307,596.18356403)(860.99593304,596.07856414)(861.03594087,595.96856951)
\curveto(861.06593297,595.85856436)(861.09093294,595.73856448)(861.11094087,595.60856951)
\curveto(861.1309329,595.48856473)(861.15593288,595.36356485)(861.18594087,595.23356951)
\curveto(861.19593284,595.18356503)(861.19593284,595.12856509)(861.18594087,595.06856951)
\curveto(861.18593285,595.0185652)(861.19093284,594.96856525)(861.20094087,594.91856951)
\moveto(859.86594087,594.06356951)
\curveto(859.88593415,594.13356608)(859.89093414,594.213566)(859.88094087,594.30356951)
\lineto(859.88094087,594.55856951)
\curveto(859.88093415,594.94856527)(859.84593419,595.27856494)(859.77594087,595.54856951)
\curveto(859.74593429,595.62856459)(859.72093431,595.70856451)(859.70094087,595.78856951)
\curveto(859.68093435,595.86856435)(859.65593438,595.94356427)(859.62594087,596.01356951)
\curveto(859.34593469,596.66356355)(858.90093513,597.1135631)(858.29094087,597.36356951)
\curveto(858.22093581,597.39356282)(858.14593589,597.4135628)(858.06594087,597.42356951)
\lineto(857.82594087,597.48356951)
\curveto(857.74593629,597.50356271)(857.66093637,597.5135627)(857.57094087,597.51356951)
\lineto(857.30094087,597.51356951)
\lineto(857.03094087,597.46856951)
\curveto(856.9309371,597.44856277)(856.8359372,597.42356279)(856.74594087,597.39356951)
\curveto(856.66593737,597.37356284)(856.58593745,597.34356287)(856.50594087,597.30356951)
\curveto(856.4359376,597.28356293)(856.37093766,597.25356296)(856.31094087,597.21356951)
\curveto(856.25093778,597.17356304)(856.19593784,597.13356308)(856.14594087,597.09356951)
\curveto(855.90593813,596.92356329)(855.71093832,596.7185635)(855.56094087,596.47856951)
\curveto(855.41093862,596.23856398)(855.28093875,595.95856426)(855.17094087,595.63856951)
\curveto(855.14093889,595.53856468)(855.12093891,595.43356478)(855.11094087,595.32356951)
\curveto(855.10093893,595.22356499)(855.08593895,595.1185651)(855.06594087,595.00856951)
\curveto(855.05593898,594.96856525)(855.05093898,594.90356531)(855.05094087,594.81356951)
\curveto(855.04093899,594.78356543)(855.035939,594.74856547)(855.03594087,594.70856951)
\curveto(855.04593899,594.66856555)(855.05093898,594.62356559)(855.05094087,594.57356951)
\lineto(855.05094087,594.27356951)
\curveto(855.05093898,594.17356604)(855.06093897,594.08356613)(855.08094087,594.00356951)
\lineto(855.11094087,593.82356951)
\curveto(855.1309389,593.72356649)(855.14593889,593.62356659)(855.15594087,593.52356951)
\curveto(855.17593886,593.43356678)(855.20593883,593.34856687)(855.24594087,593.26856951)
\curveto(855.34593869,593.02856719)(855.46093857,592.80356741)(855.59094087,592.59356951)
\curveto(855.7309383,592.38356783)(855.90093813,592.20856801)(856.10094087,592.06856951)
\curveto(856.15093788,592.03856818)(856.19593784,592.0135682)(856.23594087,591.99356951)
\curveto(856.27593776,591.97356824)(856.32093771,591.94856827)(856.37094087,591.91856951)
\curveto(856.45093758,591.86856835)(856.5359375,591.82356839)(856.62594087,591.78356951)
\curveto(856.72593731,591.75356846)(856.8309372,591.72356849)(856.94094087,591.69356951)
\curveto(856.99093704,591.67356854)(857.035937,591.66356855)(857.07594087,591.66356951)
\curveto(857.12593691,591.67356854)(857.17593686,591.67356854)(857.22594087,591.66356951)
\curveto(857.25593678,591.65356856)(857.31593672,591.64356857)(857.40594087,591.63356951)
\curveto(857.50593653,591.62356859)(857.58093645,591.62856859)(857.63094087,591.64856951)
\curveto(857.67093636,591.65856856)(857.71093632,591.65856856)(857.75094087,591.64856951)
\curveto(857.79093624,591.64856857)(857.8309362,591.65856856)(857.87094087,591.67856951)
\curveto(857.95093608,591.69856852)(858.030936,591.7135685)(858.11094087,591.72356951)
\curveto(858.19093584,591.74356847)(858.26593577,591.76856845)(858.33594087,591.79856951)
\curveto(858.67593536,591.93856828)(858.95093508,592.13356808)(859.16094087,592.38356951)
\curveto(859.37093466,592.63356758)(859.54593449,592.92856729)(859.68594087,593.26856951)
\curveto(859.7359343,593.38856683)(859.76593427,593.5135667)(859.77594087,593.64356951)
\curveto(859.79593424,593.78356643)(859.82593421,593.92356629)(859.86594087,594.06356951)
}
}
{
\newrgbcolor{curcolor}{0 0 0}
\pscustom[linestyle=none,fillstyle=solid,fillcolor=curcolor]
{
\newpath
\moveto(869.30422212,598.33856951)
\curveto(869.37421452,598.28856193)(869.40921449,598.213562)(869.40922212,598.11356951)
\curveto(869.41921448,598.0135622)(869.42421447,597.90856231)(869.42422212,597.79856951)
\lineto(869.42422212,591.52856951)
\lineto(869.42422212,590.92856951)
\curveto(869.40421449,590.87856934)(869.3992145,590.82856939)(869.40922212,590.77856951)
\curveto(869.41921448,590.73856948)(869.41421448,590.69356952)(869.39422212,590.64356951)
\curveto(869.37421452,590.54356967)(869.35921454,590.44356977)(869.34922212,590.34356951)
\curveto(869.34921455,590.23356998)(869.33421456,590.12857009)(869.30422212,590.02856951)
\curveto(869.27421462,589.9185703)(869.24421465,589.8135704)(869.21422212,589.71356951)
\curveto(869.1942147,589.6135706)(869.15921474,589.5135707)(869.10922212,589.41356951)
\curveto(869.00921489,589.15357106)(868.87921502,588.9185713)(868.71922212,588.70856951)
\curveto(868.56921533,588.49857172)(868.38921551,588.32357189)(868.17922212,588.18356951)
\curveto(868.00921589,588.06357215)(867.82921607,587.96857225)(867.63922212,587.89856951)
\curveto(867.44921645,587.8185724)(867.24421665,587.74357247)(867.02422212,587.67356951)
\curveto(866.93421696,587.65357256)(866.84421705,587.64357257)(866.75422212,587.64356951)
\curveto(866.66421723,587.63357258)(866.57421732,587.6185726)(866.48422212,587.59856951)
\lineto(866.39422212,587.59856951)
\curveto(866.37421752,587.58857263)(866.35421754,587.58357263)(866.33422212,587.58356951)
\curveto(866.28421761,587.57357264)(866.23421766,587.57357264)(866.18422212,587.58356951)
\curveto(866.14421775,587.59357262)(866.0992178,587.58857263)(866.04922212,587.56856951)
\curveto(865.97921792,587.54857267)(865.86921803,587.54357267)(865.71922212,587.55356951)
\curveto(865.57921832,587.55357266)(865.47921842,587.56357265)(865.41922212,587.58356951)
\curveto(865.38921851,587.58357263)(865.35921854,587.58857263)(865.32922212,587.59856951)
\lineto(865.26922212,587.59856951)
\curveto(865.17921872,587.6185726)(865.08921881,587.63357258)(864.99922212,587.64356951)
\curveto(864.90921899,587.64357257)(864.82421907,587.65357256)(864.74422212,587.67356951)
\curveto(864.66421923,587.69357252)(864.58421931,587.7185725)(864.50422212,587.74856951)
\curveto(864.42421947,587.76857245)(864.34421955,587.79357242)(864.26422212,587.82356951)
\curveto(863.94421995,587.95357226)(863.67422022,588.09857212)(863.45422212,588.25856951)
\curveto(863.24422065,588.4185718)(863.05422084,588.64357157)(862.88422212,588.93356951)
\curveto(862.86422103,588.95357126)(862.84922105,588.97857124)(862.83922212,589.00856951)
\curveto(862.83922106,589.02857119)(862.82922107,589.05357116)(862.80922212,589.08356951)
\curveto(862.77922112,589.16357105)(862.74422115,589.27857094)(862.70422212,589.42856951)
\curveto(862.67422122,589.56857065)(862.70422119,589.67357054)(862.79422212,589.74356951)
\curveto(862.85422104,589.79357042)(862.93422096,589.8185704)(863.03422212,589.81856951)
\lineto(863.36422212,589.81856951)
\lineto(863.52922212,589.81856951)
\curveto(863.58922031,589.8185704)(863.64422025,589.80857041)(863.69422212,589.78856951)
\curveto(863.78422011,589.75857046)(863.84922005,589.70857051)(863.88922212,589.63856951)
\curveto(863.92921997,589.56857065)(863.97421992,589.49357072)(864.02422212,589.41356951)
\lineto(864.14422212,589.23356951)
\curveto(864.1942197,589.16357105)(864.24421965,589.10857111)(864.29422212,589.06856951)
\curveto(864.54421935,588.87857134)(864.84421905,588.73857148)(865.19422212,588.64856951)
\curveto(865.25421864,588.62857159)(865.31421858,588.6185716)(865.37422212,588.61856951)
\curveto(865.44421845,588.60857161)(865.50921839,588.59357162)(865.56922212,588.57356951)
\lineto(865.65922212,588.57356951)
\curveto(865.72921817,588.55357166)(865.81421808,588.54357167)(865.91422212,588.54356951)
\curveto(866.01421788,588.54357167)(866.10421779,588.55357166)(866.18422212,588.57356951)
\curveto(866.21421768,588.58357163)(866.25421764,588.58857163)(866.30422212,588.58856951)
\curveto(866.40421749,588.60857161)(866.4992174,588.62857159)(866.58922212,588.64856951)
\curveto(866.67921722,588.65857156)(866.76421713,588.68357153)(866.84422212,588.72356951)
\curveto(867.13421676,588.84357137)(867.36921653,589.00857121)(867.54922212,589.21856951)
\curveto(867.73921616,589.4185708)(867.894216,589.66357055)(868.01422212,589.95356951)
\curveto(868.05421584,590.04357017)(868.07921582,590.13857008)(868.08922212,590.23856951)
\curveto(868.10921579,590.33856988)(868.13421576,590.44356977)(868.16422212,590.55356951)
\curveto(868.18421571,590.60356961)(868.1942157,590.65356956)(868.19422212,590.70356951)
\curveto(868.1942157,590.75356946)(868.1992157,590.80356941)(868.20922212,590.85356951)
\curveto(868.21921568,590.88356933)(868.22421567,590.93356928)(868.22422212,591.00356951)
\curveto(868.24421565,591.08356913)(868.24421565,591.16856905)(868.22422212,591.25856951)
\curveto(868.21421568,591.30856891)(868.20921569,591.35356886)(868.20922212,591.39356951)
\curveto(868.21921568,591.43356878)(868.21421568,591.46856875)(868.19422212,591.49856951)
\curveto(868.17421572,591.5185687)(868.15921574,591.52856869)(868.14922212,591.52856951)
\lineto(868.10422212,591.57356951)
\curveto(868.00421589,591.57356864)(867.92921597,591.54356867)(867.87922212,591.48356951)
\curveto(867.83921606,591.43356878)(867.78921611,591.38856883)(867.72922212,591.34856951)
\lineto(867.48922212,591.13856951)
\curveto(867.40921649,591.07856914)(867.31921658,591.02356919)(867.21922212,590.97356951)
\curveto(867.07921682,590.88356933)(866.90421699,590.80856941)(866.69422212,590.74856951)
\curveto(866.48421741,590.69856952)(866.26421763,590.66356955)(866.03422212,590.64356951)
\curveto(865.80421809,590.62356959)(865.57421832,590.62856959)(865.34422212,590.65856951)
\curveto(865.11421878,590.67856954)(864.90421899,590.7185695)(864.71422212,590.77856951)
\curveto(863.77422012,591.08856913)(863.11422078,591.68356853)(862.73422212,592.56356951)
\curveto(862.68422121,592.66356755)(862.64422125,592.75856746)(862.61422212,592.84856951)
\curveto(862.58422131,592.94856727)(862.54922135,593.05356716)(862.50922212,593.16356951)
\curveto(862.48922141,593.213567)(862.47922142,593.25856696)(862.47922212,593.29856951)
\curveto(862.47922142,593.33856688)(862.46922143,593.38356683)(862.44922212,593.43356951)
\curveto(862.42922147,593.50356671)(862.41422148,593.57356664)(862.40422212,593.64356951)
\curveto(862.40422149,593.72356649)(862.3942215,593.79856642)(862.37422212,593.86856951)
\curveto(862.36422153,593.90856631)(862.35922154,593.94356627)(862.35922212,593.97356951)
\curveto(862.36922153,594.0135662)(862.36922153,594.05356616)(862.35922212,594.09356951)
\curveto(862.35922154,594.13356608)(862.35422154,594.17356604)(862.34422212,594.21356951)
\lineto(862.34422212,594.33356951)
\curveto(862.32422157,594.45356576)(862.32422157,594.57856564)(862.34422212,594.70856951)
\curveto(862.35422154,594.76856545)(862.35922154,594.82856539)(862.35922212,594.88856951)
\lineto(862.35922212,595.05356951)
\curveto(862.36922153,595.10356511)(862.37422152,595.14356507)(862.37422212,595.17356951)
\curveto(862.37422152,595.213565)(862.37922152,595.25856496)(862.38922212,595.30856951)
\curveto(862.41922148,595.4185648)(862.43922146,595.52356469)(862.44922212,595.62356951)
\curveto(862.45922144,595.73356448)(862.48422141,595.84356437)(862.52422212,595.95356951)
\curveto(862.56422133,596.07356414)(862.5992213,596.18856403)(862.62922212,596.29856951)
\curveto(862.66922123,596.4185638)(862.71422118,596.53356368)(862.76422212,596.64356951)
\curveto(862.83422106,596.80356341)(862.91422098,596.94856327)(863.00422212,597.07856951)
\curveto(863.0942208,597.218563)(863.18922071,597.35356286)(863.28922212,597.48356951)
\curveto(863.35922054,597.59356262)(863.44922045,597.68356253)(863.55922212,597.75356951)
\lineto(863.61922212,597.81356951)
\lineto(863.67922212,597.87356951)
\lineto(863.82922212,597.99356951)
\lineto(864.00922212,598.11356951)
\curveto(864.13921976,598.19356202)(864.27421962,598.26356195)(864.41422212,598.32356951)
\curveto(864.56421933,598.38356183)(864.72421917,598.43856178)(864.89422212,598.48856951)
\curveto(864.9942189,598.5185617)(865.0942188,598.53856168)(865.19422212,598.54856951)
\curveto(865.30421859,598.55856166)(865.41421848,598.57356164)(865.52422212,598.59356951)
\curveto(865.56421833,598.60356161)(865.61421828,598.60356161)(865.67422212,598.59356951)
\curveto(865.74421815,598.58356163)(865.7942181,598.58856163)(865.82422212,598.60856951)
\curveto(866.14421775,598.6185616)(866.42921747,598.58856163)(866.67922212,598.51856951)
\curveto(866.93921696,598.44856177)(867.16921673,598.34856187)(867.36922212,598.21856951)
\curveto(867.43921646,598.17856204)(867.50421639,598.13356208)(867.56422212,598.08356951)
\lineto(867.74422212,597.93356951)
\curveto(867.7942161,597.89356232)(867.83921606,597.84856237)(867.87922212,597.79856951)
\curveto(867.92921597,597.75856246)(868.00421589,597.73856248)(868.10422212,597.73856951)
\lineto(868.14922212,597.78356951)
\curveto(868.16921573,597.80356241)(868.18921571,597.82856239)(868.20922212,597.85856951)
\curveto(868.23921566,597.93856228)(868.25421564,598.0185622)(868.25422212,598.09856951)
\curveto(868.26421563,598.17856204)(868.2942156,598.24856197)(868.34422212,598.30856951)
\curveto(868.37421552,598.34856187)(868.43421546,598.37856184)(868.52422212,598.39856951)
\curveto(868.61421528,598.42856179)(868.70921519,598.44356177)(868.80922212,598.44356951)
\curveto(868.90921499,598.44356177)(869.00421489,598.43356178)(869.09422212,598.41356951)
\curveto(869.1942147,598.39356182)(869.26421463,598.36856185)(869.30422212,598.33856951)
\moveto(868.17922212,594.55856951)
\curveto(868.18921571,594.59856562)(868.1942157,594.64856557)(868.19422212,594.70856951)
\curveto(868.1942157,594.77856544)(868.18921571,594.83356538)(868.17922212,594.87356951)
\lineto(868.17922212,595.11356951)
\curveto(868.15921574,595.20356501)(868.14421575,595.28856493)(868.13422212,595.36856951)
\curveto(868.12421577,595.45856476)(868.10921579,595.54356467)(868.08922212,595.62356951)
\curveto(868.06921583,595.70356451)(868.04921585,595.77856444)(868.02922212,595.84856951)
\curveto(868.01921588,595.92856429)(867.9992159,596.00356421)(867.96922212,596.07356951)
\curveto(867.85921604,596.35356386)(867.71421618,596.60356361)(867.53422212,596.82356951)
\curveto(867.36421653,597.04356317)(867.14421675,597.20856301)(866.87422212,597.31856951)
\curveto(866.7942171,597.35856286)(866.70921719,597.38856283)(866.61922212,597.40856951)
\curveto(866.52921737,597.43856278)(866.43421746,597.46356275)(866.33422212,597.48356951)
\curveto(866.25421764,597.50356271)(866.16421773,597.50856271)(866.06422212,597.49856951)
\lineto(865.79422212,597.49856951)
\curveto(865.74421815,597.48856273)(865.6942182,597.48356273)(865.64422212,597.48356951)
\curveto(865.60421829,597.48356273)(865.55921834,597.47856274)(865.50922212,597.46856951)
\curveto(865.31921858,597.4185628)(865.15921874,597.36856285)(865.02922212,597.31856951)
\curveto(864.68921921,597.17856304)(864.42421947,596.96856325)(864.23422212,596.68856951)
\curveto(864.04421985,596.40856381)(863.89422,596.08356413)(863.78422212,595.71356951)
\curveto(863.76422013,595.63356458)(863.74922015,595.55356466)(863.73922212,595.47356951)
\curveto(863.73922016,595.40356481)(863.72922017,595.32856489)(863.70922212,595.24856951)
\curveto(863.68922021,595.218565)(863.67922022,595.18356503)(863.67922212,595.14356951)
\curveto(863.68922021,595.10356511)(863.68922021,595.06856515)(863.67922212,595.03856951)
\lineto(863.67922212,594.70856951)
\lineto(863.67922212,594.36356951)
\curveto(863.67922022,594.25356596)(863.68922021,594.14856607)(863.70922212,594.04856951)
\lineto(863.70922212,593.97356951)
\curveto(863.71922018,593.94356627)(863.72422017,593.9185663)(863.72422212,593.89856951)
\curveto(863.74422015,593.80856641)(863.75922014,593.7185665)(863.76922212,593.62856951)
\curveto(863.78922011,593.53856668)(863.81422008,593.45356676)(863.84422212,593.37356951)
\curveto(863.92421997,593.1135671)(864.02421987,592.87356734)(864.14422212,592.65356951)
\curveto(864.26421963,592.43356778)(864.42421947,592.25356796)(864.62422212,592.11356951)
\lineto(864.74422212,592.02356951)
\curveto(864.78421911,592.00356821)(864.82921907,591.98356823)(864.87922212,591.96356951)
\curveto(864.95921894,591.9135683)(865.04421885,591.87356834)(865.13422212,591.84356951)
\curveto(865.22421867,591.8135684)(865.32421857,591.78356843)(865.43422212,591.75356951)
\curveto(865.48421841,591.74356847)(865.52921837,591.73856848)(865.56922212,591.73856951)
\curveto(865.61921828,591.74856847)(865.66921823,591.74356847)(865.71922212,591.72356951)
\curveto(865.74921815,591.7135685)(865.7992181,591.70856851)(865.86922212,591.70856951)
\curveto(865.93921796,591.70856851)(865.98921791,591.7135685)(866.01922212,591.72356951)
\curveto(866.04921785,591.73356848)(866.07921782,591.73356848)(866.10922212,591.72356951)
\curveto(866.14921775,591.72356849)(866.18921771,591.72856849)(866.22922212,591.73856951)
\curveto(866.31921758,591.75856846)(866.40421749,591.77856844)(866.48422212,591.79856951)
\curveto(866.56421733,591.8185684)(866.64421725,591.84356837)(866.72422212,591.87356951)
\curveto(867.06421683,592.02356819)(867.33421656,592.23356798)(867.53422212,592.50356951)
\curveto(867.73421616,592.77356744)(867.894216,593.08856713)(868.01422212,593.44856951)
\curveto(868.04421585,593.53856668)(868.06421583,593.62856659)(868.07422212,593.71856951)
\curveto(868.0942158,593.8185664)(868.11421578,593.9135663)(868.13422212,594.00356951)
\curveto(868.14421575,594.04356617)(868.14921575,594.07856614)(868.14922212,594.10856951)
\curveto(868.14921575,594.14856607)(868.15421574,594.18856603)(868.16422212,594.22856951)
\curveto(868.18421571,594.27856594)(868.18421571,594.32856589)(868.16422212,594.37856951)
\curveto(868.15421574,594.43856578)(868.15921574,594.49856572)(868.17922212,594.55856951)
}
}
{
\newrgbcolor{curcolor}{0 0 0}
\pscustom[linestyle=none,fillstyle=solid,fillcolor=curcolor]
{
\newpath
\moveto(874.94750337,598.62356951)
\curveto(875.17749858,598.62356159)(875.30749845,598.56356165)(875.33750337,598.44356951)
\curveto(875.36749839,598.33356188)(875.38249838,598.16856205)(875.38250337,597.94856951)
\lineto(875.38250337,597.66356951)
\curveto(875.38249838,597.57356264)(875.3574984,597.49856272)(875.30750337,597.43856951)
\curveto(875.24749851,597.35856286)(875.1624986,597.3135629)(875.05250337,597.30356951)
\curveto(874.94249882,597.30356291)(874.83249893,597.28856293)(874.72250337,597.25856951)
\curveto(874.58249918,597.22856299)(874.44749931,597.19856302)(874.31750337,597.16856951)
\curveto(874.19749956,597.13856308)(874.08249968,597.09856312)(873.97250337,597.04856951)
\curveto(873.68250008,596.9185633)(873.44750031,596.73856348)(873.26750337,596.50856951)
\curveto(873.08750067,596.28856393)(872.93250083,596.03356418)(872.80250337,595.74356951)
\curveto(872.762501,595.63356458)(872.73250103,595.5185647)(872.71250337,595.39856951)
\curveto(872.69250107,595.28856493)(872.66750109,595.17356504)(872.63750337,595.05356951)
\curveto(872.62750113,595.00356521)(872.62250114,594.95356526)(872.62250337,594.90356951)
\curveto(872.63250113,594.85356536)(872.63250113,594.80356541)(872.62250337,594.75356951)
\curveto(872.59250117,594.63356558)(872.57750118,594.49356572)(872.57750337,594.33356951)
\curveto(872.58750117,594.18356603)(872.59250117,594.03856618)(872.59250337,593.89856951)
\lineto(872.59250337,592.05356951)
\lineto(872.59250337,591.70856951)
\curveto(872.59250117,591.58856863)(872.58750117,591.47356874)(872.57750337,591.36356951)
\curveto(872.56750119,591.25356896)(872.5625012,591.15856906)(872.56250337,591.07856951)
\curveto(872.57250119,590.99856922)(872.55250121,590.92856929)(872.50250337,590.86856951)
\curveto(872.45250131,590.79856942)(872.37250139,590.75856946)(872.26250337,590.74856951)
\curveto(872.1625016,590.73856948)(872.05250171,590.73356948)(871.93250337,590.73356951)
\lineto(871.66250337,590.73356951)
\curveto(871.61250215,590.75356946)(871.5625022,590.76856945)(871.51250337,590.77856951)
\curveto(871.47250229,590.79856942)(871.44250232,590.82356939)(871.42250337,590.85356951)
\curveto(871.37250239,590.92356929)(871.34250242,591.00856921)(871.33250337,591.10856951)
\lineto(871.33250337,591.43856951)
\lineto(871.33250337,592.59356951)
\lineto(871.33250337,596.74856951)
\lineto(871.33250337,597.78356951)
\lineto(871.33250337,598.08356951)
\curveto(871.34250242,598.18356203)(871.37250239,598.26856195)(871.42250337,598.33856951)
\curveto(871.45250231,598.37856184)(871.50250226,598.40856181)(871.57250337,598.42856951)
\curveto(871.65250211,598.44856177)(871.73750202,598.45856176)(871.82750337,598.45856951)
\curveto(871.91750184,598.46856175)(872.00750175,598.46856175)(872.09750337,598.45856951)
\curveto(872.18750157,598.44856177)(872.2575015,598.43356178)(872.30750337,598.41356951)
\curveto(872.38750137,598.38356183)(872.43750132,598.32356189)(872.45750337,598.23356951)
\curveto(872.48750127,598.15356206)(872.50250126,598.06356215)(872.50250337,597.96356951)
\lineto(872.50250337,597.66356951)
\curveto(872.50250126,597.56356265)(872.52250124,597.47356274)(872.56250337,597.39356951)
\curveto(872.57250119,597.37356284)(872.58250118,597.35856286)(872.59250337,597.34856951)
\lineto(872.63750337,597.30356951)
\curveto(872.74750101,597.30356291)(872.83750092,597.34856287)(872.90750337,597.43856951)
\curveto(872.97750078,597.53856268)(873.03750072,597.6185626)(873.08750337,597.67856951)
\lineto(873.17750337,597.76856951)
\curveto(873.26750049,597.87856234)(873.39250037,597.99356222)(873.55250337,598.11356951)
\curveto(873.71250005,598.23356198)(873.8624999,598.32356189)(874.00250337,598.38356951)
\curveto(874.09249967,598.43356178)(874.18749957,598.46856175)(874.28750337,598.48856951)
\curveto(874.38749937,598.5185617)(874.49249927,598.54856167)(874.60250337,598.57856951)
\curveto(874.6624991,598.58856163)(874.72249904,598.59356162)(874.78250337,598.59356951)
\curveto(874.84249892,598.60356161)(874.89749886,598.6135616)(874.94750337,598.62356951)
}
}
{
\newrgbcolor{curcolor}{0 0 0}
\pscustom[linestyle=none,fillstyle=solid,fillcolor=curcolor]
{
\newpath
\moveto(883.197269,591.27356951)
\curveto(883.22726117,591.1135691)(883.21226118,590.97856924)(883.152269,590.86856951)
\curveto(883.0922613,590.76856945)(883.01226138,590.69356952)(882.912269,590.64356951)
\curveto(882.86226153,590.62356959)(882.80726159,590.6135696)(882.747269,590.61356951)
\curveto(882.6972617,590.6135696)(882.64226175,590.60356961)(882.582269,590.58356951)
\curveto(882.36226203,590.53356968)(882.14226225,590.54856967)(881.922269,590.62856951)
\curveto(881.71226268,590.69856952)(881.56726283,590.78856943)(881.487269,590.89856951)
\curveto(881.43726296,590.96856925)(881.392263,591.04856917)(881.352269,591.13856951)
\curveto(881.31226308,591.23856898)(881.26226313,591.3185689)(881.202269,591.37856951)
\curveto(881.18226321,591.39856882)(881.15726324,591.4185688)(881.127269,591.43856951)
\curveto(881.10726329,591.45856876)(881.07726332,591.46356875)(881.037269,591.45356951)
\curveto(880.92726347,591.42356879)(880.82226357,591.36856885)(880.722269,591.28856951)
\curveto(880.63226376,591.20856901)(880.54226385,591.13856908)(880.452269,591.07856951)
\curveto(880.32226407,590.99856922)(880.18226421,590.92356929)(880.032269,590.85356951)
\curveto(879.88226451,590.79356942)(879.72226467,590.73856948)(879.552269,590.68856951)
\curveto(879.45226494,590.65856956)(879.34226505,590.63856958)(879.222269,590.62856951)
\curveto(879.11226528,590.6185696)(879.00226539,590.60356961)(878.892269,590.58356951)
\curveto(878.84226555,590.57356964)(878.7972656,590.56856965)(878.757269,590.56856951)
\lineto(878.652269,590.56856951)
\curveto(878.54226585,590.54856967)(878.43726596,590.54856967)(878.337269,590.56856951)
\lineto(878.202269,590.56856951)
\curveto(878.15226624,590.57856964)(878.10226629,590.58356963)(878.052269,590.58356951)
\curveto(878.00226639,590.58356963)(877.95726644,590.59356962)(877.917269,590.61356951)
\curveto(877.87726652,590.62356959)(877.84226655,590.62856959)(877.812269,590.62856951)
\curveto(877.7922666,590.6185696)(877.76726663,590.6185696)(877.737269,590.62856951)
\lineto(877.497269,590.68856951)
\curveto(877.41726698,590.69856952)(877.34226705,590.7185695)(877.272269,590.74856951)
\curveto(876.97226742,590.87856934)(876.72726767,591.02356919)(876.537269,591.18356951)
\curveto(876.35726804,591.35356886)(876.20726819,591.58856863)(876.087269,591.88856951)
\curveto(875.9972684,592.10856811)(875.95226844,592.37356784)(875.952269,592.68356951)
\lineto(875.952269,592.99856951)
\curveto(875.96226843,593.04856717)(875.96726843,593.09856712)(875.967269,593.14856951)
\lineto(875.997269,593.32856951)
\lineto(876.117269,593.65856951)
\curveto(876.15726824,593.76856645)(876.20726819,593.86856635)(876.267269,593.95856951)
\curveto(876.44726795,594.24856597)(876.6922677,594.46356575)(877.002269,594.60356951)
\curveto(877.31226708,594.74356547)(877.65226674,594.86856535)(878.022269,594.97856951)
\curveto(878.16226623,595.0185652)(878.30726609,595.04856517)(878.457269,595.06856951)
\curveto(878.60726579,595.08856513)(878.75726564,595.1135651)(878.907269,595.14356951)
\curveto(878.97726542,595.16356505)(879.04226535,595.17356504)(879.102269,595.17356951)
\curveto(879.17226522,595.17356504)(879.24726515,595.18356503)(879.327269,595.20356951)
\curveto(879.397265,595.22356499)(879.46726493,595.23356498)(879.537269,595.23356951)
\curveto(879.60726479,595.24356497)(879.68226471,595.25856496)(879.762269,595.27856951)
\curveto(880.01226438,595.33856488)(880.24726415,595.38856483)(880.467269,595.42856951)
\curveto(880.68726371,595.47856474)(880.86226353,595.59356462)(880.992269,595.77356951)
\curveto(881.05226334,595.85356436)(881.10226329,595.95356426)(881.142269,596.07356951)
\curveto(881.18226321,596.20356401)(881.18226321,596.34356387)(881.142269,596.49356951)
\curveto(881.08226331,596.73356348)(880.9922634,596.92356329)(880.872269,597.06356951)
\curveto(880.76226363,597.20356301)(880.60226379,597.3135629)(880.392269,597.39356951)
\curveto(880.27226412,597.44356277)(880.12726427,597.47856274)(879.957269,597.49856951)
\curveto(879.7972646,597.5185627)(879.62726477,597.52856269)(879.447269,597.52856951)
\curveto(879.26726513,597.52856269)(879.0922653,597.5185627)(878.922269,597.49856951)
\curveto(878.75226564,597.47856274)(878.60726579,597.44856277)(878.487269,597.40856951)
\curveto(878.31726608,597.34856287)(878.15226624,597.26356295)(877.992269,597.15356951)
\curveto(877.91226648,597.09356312)(877.83726656,597.0135632)(877.767269,596.91356951)
\curveto(877.70726669,596.82356339)(877.65226674,596.72356349)(877.602269,596.61356951)
\curveto(877.57226682,596.53356368)(877.54226685,596.44856377)(877.512269,596.35856951)
\curveto(877.4922669,596.26856395)(877.44726695,596.19856402)(877.377269,596.14856951)
\curveto(877.33726706,596.1185641)(877.26726713,596.09356412)(877.167269,596.07356951)
\curveto(877.07726732,596.06356415)(876.98226741,596.05856416)(876.882269,596.05856951)
\curveto(876.78226761,596.05856416)(876.68226771,596.06356415)(876.582269,596.07356951)
\curveto(876.4922679,596.09356412)(876.42726797,596.1185641)(876.387269,596.14856951)
\curveto(876.34726805,596.17856404)(876.31726808,596.22856399)(876.297269,596.29856951)
\curveto(876.27726812,596.36856385)(876.27726812,596.44356377)(876.297269,596.52356951)
\curveto(876.32726807,596.65356356)(876.35726804,596.77356344)(876.387269,596.88356951)
\curveto(876.42726797,597.00356321)(876.47226792,597.1185631)(876.522269,597.22856951)
\curveto(876.71226768,597.57856264)(876.95226744,597.84856237)(877.242269,598.03856951)
\curveto(877.53226686,598.23856198)(877.8922665,598.39856182)(878.322269,598.51856951)
\curveto(878.42226597,598.53856168)(878.52226587,598.55356166)(878.622269,598.56356951)
\curveto(878.73226566,598.57356164)(878.84226555,598.58856163)(878.952269,598.60856951)
\curveto(878.9922654,598.6185616)(879.05726534,598.6185616)(879.147269,598.60856951)
\curveto(879.23726516,598.60856161)(879.2922651,598.6185616)(879.312269,598.63856951)
\curveto(880.01226438,598.64856157)(880.62226377,598.56856165)(881.142269,598.39856951)
\curveto(881.66226273,598.22856199)(882.02726237,597.90356231)(882.237269,597.42356951)
\curveto(882.32726207,597.22356299)(882.37726202,596.98856323)(882.387269,596.71856951)
\curveto(882.40726199,596.45856376)(882.41726198,596.18356403)(882.417269,595.89356951)
\lineto(882.417269,592.57856951)
\curveto(882.41726198,592.43856778)(882.42226197,592.30356791)(882.432269,592.17356951)
\curveto(882.44226195,592.04356817)(882.47226192,591.93856828)(882.522269,591.85856951)
\curveto(882.57226182,591.78856843)(882.63726176,591.73856848)(882.717269,591.70856951)
\curveto(882.80726159,591.66856855)(882.8922615,591.63856858)(882.972269,591.61856951)
\curveto(883.05226134,591.60856861)(883.11226128,591.56356865)(883.152269,591.48356951)
\curveto(883.17226122,591.45356876)(883.18226121,591.42356879)(883.182269,591.39356951)
\curveto(883.18226121,591.36356885)(883.18726121,591.32356889)(883.197269,591.27356951)
\moveto(881.052269,592.93856951)
\curveto(881.11226328,593.07856714)(881.14226325,593.23856698)(881.142269,593.41856951)
\curveto(881.15226324,593.60856661)(881.15726324,593.80356641)(881.157269,594.00356951)
\curveto(881.15726324,594.1135661)(881.15226324,594.213566)(881.142269,594.30356951)
\curveto(881.13226326,594.39356582)(881.0922633,594.46356575)(881.022269,594.51356951)
\curveto(880.9922634,594.53356568)(880.92226347,594.54356567)(880.812269,594.54356951)
\curveto(880.7922636,594.52356569)(880.75726364,594.5135657)(880.707269,594.51356951)
\curveto(880.65726374,594.5135657)(880.61226378,594.50356571)(880.572269,594.48356951)
\curveto(880.4922639,594.46356575)(880.40226399,594.44356577)(880.302269,594.42356951)
\lineto(880.002269,594.36356951)
\curveto(879.97226442,594.36356585)(879.93726446,594.35856586)(879.897269,594.34856951)
\lineto(879.792269,594.34856951)
\curveto(879.64226475,594.30856591)(879.47726492,594.28356593)(879.297269,594.27356951)
\curveto(879.12726527,594.27356594)(878.96726543,594.25356596)(878.817269,594.21356951)
\curveto(878.73726566,594.19356602)(878.66226573,594.17356604)(878.592269,594.15356951)
\curveto(878.53226586,594.14356607)(878.46226593,594.12856609)(878.382269,594.10856951)
\curveto(878.22226617,594.05856616)(878.07226632,593.99356622)(877.932269,593.91356951)
\curveto(877.7922666,593.84356637)(877.67226672,593.75356646)(877.572269,593.64356951)
\curveto(877.47226692,593.53356668)(877.397267,593.39856682)(877.347269,593.23856951)
\curveto(877.2972671,593.08856713)(877.27726712,592.90356731)(877.287269,592.68356951)
\curveto(877.28726711,592.58356763)(877.30226709,592.48856773)(877.332269,592.39856951)
\curveto(877.37226702,592.3185679)(877.41726698,592.24356797)(877.467269,592.17356951)
\curveto(877.54726685,592.06356815)(877.65226674,591.96856825)(877.782269,591.88856951)
\curveto(877.91226648,591.8185684)(878.05226634,591.75856846)(878.202269,591.70856951)
\curveto(878.25226614,591.69856852)(878.30226609,591.69356852)(878.352269,591.69356951)
\curveto(878.40226599,591.69356852)(878.45226594,591.68856853)(878.502269,591.67856951)
\curveto(878.57226582,591.65856856)(878.65726574,591.64356857)(878.757269,591.63356951)
\curveto(878.86726553,591.63356858)(878.95726544,591.64356857)(879.027269,591.66356951)
\curveto(879.08726531,591.68356853)(879.14726525,591.68856853)(879.207269,591.67856951)
\curveto(879.26726513,591.67856854)(879.32726507,591.68856853)(879.387269,591.70856951)
\curveto(879.46726493,591.72856849)(879.54226485,591.74356847)(879.612269,591.75356951)
\curveto(879.6922647,591.76356845)(879.76726463,591.78356843)(879.837269,591.81356951)
\curveto(880.12726427,591.93356828)(880.37226402,592.07856814)(880.572269,592.24856951)
\curveto(880.78226361,592.4185678)(880.94226345,592.64856757)(881.052269,592.93856951)
}
}
{
\newrgbcolor{curcolor}{0 0 0}
\pscustom[linestyle=none,fillstyle=solid,fillcolor=curcolor]
{
\newpath
\moveto(886.81390962,601.51856951)
\curveto(886.99390608,601.52855869)(887.18390589,601.52855869)(887.38390962,601.51856951)
\curveto(887.58390549,601.50855871)(887.72390535,601.44855877)(887.80390962,601.33856951)
\curveto(887.84390523,601.27855894)(887.86890521,601.20355901)(887.87890962,601.11356951)
\curveto(887.88890519,601.03355918)(887.89390518,600.94355927)(887.89390962,600.84356951)
\curveto(887.89390518,600.7135595)(887.86890521,600.60855961)(887.81890962,600.52856951)
\curveto(887.7789053,600.47855974)(887.71890536,600.44355977)(887.63890962,600.42356951)
\curveto(887.56890551,600.4135598)(887.48890559,600.40855981)(887.39890962,600.40856951)
\lineto(887.11390962,600.40856951)
\curveto(887.02390605,600.4185598)(886.94390613,600.4185598)(886.87390962,600.40856951)
\curveto(886.59390648,600.32855989)(886.40890667,600.19856002)(886.31890962,600.01856951)
\curveto(886.23890684,599.84856037)(886.19890688,599.58856063)(886.19890962,599.23856951)
\curveto(886.19890688,599.16856105)(886.19390688,599.09356112)(886.18390962,599.01356951)
\curveto(886.1739069,598.94356127)(886.1789069,598.87856134)(886.19890962,598.81856951)
\curveto(886.22890685,598.66856155)(886.29390678,598.56356165)(886.39390962,598.50356951)
\curveto(886.4739066,598.47356174)(886.5739065,598.45856176)(886.69390962,598.45856951)
\lineto(887.05390962,598.45856951)
\lineto(887.27890962,598.45856951)
\curveto(887.30890577,598.43856178)(887.33890574,598.43356178)(887.36890962,598.44356951)
\curveto(887.39890568,598.45356176)(887.42890565,598.44856177)(887.45890962,598.42856951)
\curveto(887.55890552,598.39856182)(887.62390545,598.33856188)(887.65390962,598.24856951)
\curveto(887.68390539,598.16856205)(887.69890538,598.06356215)(887.69890962,597.93356951)
\curveto(887.68890539,597.89356232)(887.68390539,597.85356236)(887.68390962,597.81356951)
\lineto(887.68390962,597.69356951)
\curveto(887.65390542,597.54356267)(887.58890549,597.44356277)(887.48890962,597.39356951)
\curveto(887.35890572,597.34356287)(887.18890589,597.32856289)(886.97890962,597.34856951)
\curveto(886.7789063,597.37856284)(886.60890647,597.37356284)(886.46890962,597.33356951)
\curveto(886.38890669,597.3135629)(886.32890675,597.27356294)(886.28890962,597.21356951)
\curveto(886.24890683,597.16356305)(886.21890686,597.09356312)(886.19890962,597.00356951)
\curveto(886.1789069,596.93356328)(886.1739069,596.85356336)(886.18390962,596.76356951)
\curveto(886.19390688,596.67356354)(886.19890688,596.58856363)(886.19890962,596.50856951)
\lineto(886.19890962,595.51856951)
\lineto(886.19890962,592.33856951)
\lineto(886.19890962,591.58856951)
\lineto(886.19890962,591.39356951)
\curveto(886.20890687,591.32356889)(886.20390687,591.26356895)(886.18390962,591.21356951)
\lineto(886.18390962,591.09356951)
\lineto(886.15390962,590.97356951)
\curveto(886.14390693,590.93356928)(886.12890695,590.89856932)(886.10890962,590.86856951)
\curveto(886.05890702,590.79856942)(885.98390709,590.75856946)(885.88390962,590.74856951)
\curveto(885.78390729,590.73856948)(885.6739074,590.73356948)(885.55390962,590.73356951)
\lineto(885.26890962,590.73356951)
\curveto(885.21890786,590.75356946)(885.16890791,590.76856945)(885.11890962,590.77856951)
\curveto(885.078908,590.79856942)(885.04390803,590.83356938)(885.01390962,590.88356951)
\curveto(884.99390808,590.9135693)(884.9739081,590.97856924)(884.95390962,591.07856951)
\lineto(884.95390962,591.18356951)
\curveto(884.93390814,591.23356898)(884.92390815,591.28356893)(884.92390962,591.33356951)
\curveto(884.93390814,591.39356882)(884.93890814,591.44856877)(884.93890962,591.49856951)
\lineto(884.93890962,592.09856951)
\lineto(884.93890962,596.19356951)
\lineto(884.93890962,596.53856951)
\curveto(884.94890813,596.65856356)(884.94890813,596.76856345)(884.93890962,596.86856951)
\curveto(884.93890814,596.97856324)(884.91890816,597.07356314)(884.87890962,597.15356951)
\curveto(884.84890823,597.23356298)(884.79390828,597.28856293)(884.71390962,597.31856951)
\curveto(884.65390842,597.34856287)(884.58390849,597.36356285)(884.50390962,597.36356951)
\lineto(884.27890962,597.36356951)
\lineto(884.03890962,597.36356951)
\curveto(883.96890911,597.36356285)(883.90390917,597.37356284)(883.84390962,597.39356951)
\curveto(883.75390932,597.43356278)(883.68890939,597.5185627)(883.64890962,597.64856951)
\curveto(883.63890944,597.69856252)(883.63390944,597.74356247)(883.63390962,597.78356951)
\lineto(883.63390962,597.91856951)
\curveto(883.63390944,598.05856216)(883.64890943,598.16856205)(883.67890962,598.24856951)
\curveto(883.70890937,598.33856188)(883.7739093,598.39856182)(883.87390962,598.42856951)
\curveto(883.94390913,598.45856176)(884.02390905,598.46856175)(884.11390962,598.45856951)
\lineto(884.39890962,598.45856951)
\curveto(884.49890858,598.45856176)(884.58390849,598.46856175)(884.65390962,598.48856951)
\curveto(884.73390834,598.50856171)(884.79890828,598.54856167)(884.84890962,598.60856951)
\curveto(884.91890816,598.68856153)(884.94890813,598.8135614)(884.93890962,598.98356951)
\lineto(884.93890962,599.46356951)
\curveto(884.93890814,599.66356055)(884.94890813,599.84856037)(884.96890962,600.01856951)
\curveto(884.99890808,600.19856002)(885.04390803,600.35855986)(885.10390962,600.49856951)
\curveto(885.21390786,600.73855948)(885.35890772,600.93355928)(885.53890962,601.08356951)
\curveto(885.72890735,601.23355898)(885.95390712,601.34855887)(886.21390962,601.42856951)
\curveto(886.2739068,601.44855877)(886.33390674,601.45855876)(886.39390962,601.45856951)
\curveto(886.46390661,601.46855875)(886.53390654,601.48355873)(886.60390962,601.50356951)
\curveto(886.62390645,601.5135587)(886.65890642,601.5135587)(886.70890962,601.50356951)
\curveto(886.75890632,601.50355871)(886.79390628,601.50855871)(886.81390962,601.51856951)
\moveto(889.07890962,599.94356951)
\curveto(889.14890393,599.89356032)(889.23390384,599.86856035)(889.33390962,599.86856951)
\lineto(889.64890962,599.86856951)
\lineto(889.81390962,599.86856951)
\curveto(889.8739032,599.86856035)(889.92890315,599.87856034)(889.97890962,599.89856951)
\curveto(890.10890297,599.94856027)(890.1739029,600.05356016)(890.17390962,600.21356951)
\curveto(890.18390289,600.37355984)(890.18890289,600.54355967)(890.18890962,600.72356951)
\lineto(890.18890962,600.97856951)
\curveto(890.18890289,601.06855915)(890.1739029,601.14355907)(890.14390962,601.20356951)
\curveto(890.09390298,601.3135589)(889.99390308,601.37355884)(889.84390962,601.38356951)
\curveto(889.69390338,601.39355882)(889.53390354,601.39855882)(889.36390962,601.39856951)
\curveto(889.34390373,601.38855883)(889.31890376,601.38355883)(889.28890962,601.38356951)
\curveto(889.26890381,601.39355882)(889.24890383,601.39355882)(889.22890962,601.38356951)
\curveto(889.10890397,601.34355887)(889.02890405,601.28355893)(888.98890962,601.20356951)
\curveto(888.95890412,601.14355907)(888.94390413,601.06855915)(888.94390962,600.97856951)
\lineto(888.94390962,600.72356951)
\lineto(888.94390962,600.25856951)
\curveto(888.95390412,600.10856011)(888.99890408,600.00356021)(889.07890962,599.94356951)
\moveto(890.18890962,597.78356951)
\lineto(890.18890962,598.06856951)
\curveto(890.18890289,598.16856205)(890.16390291,598.24856197)(890.11390962,598.30856951)
\curveto(890.06390301,598.38856183)(889.96890311,598.42856179)(889.82890962,598.42856951)
\curveto(889.69890338,598.43856178)(889.56890351,598.44356177)(889.43890962,598.44356951)
\curveto(889.41890366,598.43356178)(889.39390368,598.42856179)(889.36390962,598.42856951)
\curveto(889.34390373,598.43856178)(889.32390375,598.44356177)(889.30390962,598.44356951)
\curveto(889.24390383,598.42356179)(889.18890389,598.40856181)(889.13890962,598.39856951)
\curveto(889.08890399,598.38856183)(889.04890403,598.35856186)(889.01890962,598.30856951)
\curveto(888.96890411,598.24856197)(888.94390413,598.16356205)(888.94390962,598.05356951)
\lineto(888.94390962,597.73856951)
\lineto(888.94390962,591.39356951)
\lineto(888.94390962,591.10856951)
\curveto(888.94390413,591.0185692)(888.96390411,590.94356927)(889.00390962,590.88356951)
\curveto(889.05390402,590.80356941)(889.12390395,590.75356946)(889.21390962,590.73356951)
\curveto(889.31390376,590.72356949)(889.42890365,590.7185695)(889.55890962,590.71856951)
\lineto(889.78390962,590.71856951)
\curveto(889.86390321,590.73856948)(889.93390314,590.75356946)(889.99390962,590.76356951)
\curveto(890.05390302,590.78356943)(890.09890298,590.82356939)(890.12890962,590.88356951)
\curveto(890.1789029,590.94356927)(890.19890288,591.02356919)(890.18890962,591.12356951)
\lineto(890.18890962,591.43856951)
\lineto(890.18890962,597.78356951)
}
}
{
\newrgbcolor{curcolor}{0 0 0}
\pscustom[linestyle=none,fillstyle=solid,fillcolor=curcolor]
{
\newpath
\moveto(899.0175815,591.27356951)
\curveto(899.04757367,591.1135691)(899.03257368,590.97856924)(898.9725815,590.86856951)
\curveto(898.9125738,590.76856945)(898.83257388,590.69356952)(898.7325815,590.64356951)
\curveto(898.68257403,590.62356959)(898.62757409,590.6135696)(898.5675815,590.61356951)
\curveto(898.5175742,590.6135696)(898.46257425,590.60356961)(898.4025815,590.58356951)
\curveto(898.18257453,590.53356968)(897.96257475,590.54856967)(897.7425815,590.62856951)
\curveto(897.53257518,590.69856952)(897.38757533,590.78856943)(897.3075815,590.89856951)
\curveto(897.25757546,590.96856925)(897.2125755,591.04856917)(897.1725815,591.13856951)
\curveto(897.13257558,591.23856898)(897.08257563,591.3185689)(897.0225815,591.37856951)
\curveto(897.00257571,591.39856882)(896.97757574,591.4185688)(896.9475815,591.43856951)
\curveto(896.92757579,591.45856876)(896.89757582,591.46356875)(896.8575815,591.45356951)
\curveto(896.74757597,591.42356879)(896.64257607,591.36856885)(896.5425815,591.28856951)
\curveto(896.45257626,591.20856901)(896.36257635,591.13856908)(896.2725815,591.07856951)
\curveto(896.14257657,590.99856922)(896.00257671,590.92356929)(895.8525815,590.85356951)
\curveto(895.70257701,590.79356942)(895.54257717,590.73856948)(895.3725815,590.68856951)
\curveto(895.27257744,590.65856956)(895.16257755,590.63856958)(895.0425815,590.62856951)
\curveto(894.93257778,590.6185696)(894.82257789,590.60356961)(894.7125815,590.58356951)
\curveto(894.66257805,590.57356964)(894.6175781,590.56856965)(894.5775815,590.56856951)
\lineto(894.4725815,590.56856951)
\curveto(894.36257835,590.54856967)(894.25757846,590.54856967)(894.1575815,590.56856951)
\lineto(894.0225815,590.56856951)
\curveto(893.97257874,590.57856964)(893.92257879,590.58356963)(893.8725815,590.58356951)
\curveto(893.82257889,590.58356963)(893.77757894,590.59356962)(893.7375815,590.61356951)
\curveto(893.69757902,590.62356959)(893.66257905,590.62856959)(893.6325815,590.62856951)
\curveto(893.6125791,590.6185696)(893.58757913,590.6185696)(893.5575815,590.62856951)
\lineto(893.3175815,590.68856951)
\curveto(893.23757948,590.69856952)(893.16257955,590.7185695)(893.0925815,590.74856951)
\curveto(892.79257992,590.87856934)(892.54758017,591.02356919)(892.3575815,591.18356951)
\curveto(892.17758054,591.35356886)(892.02758069,591.58856863)(891.9075815,591.88856951)
\curveto(891.8175809,592.10856811)(891.77258094,592.37356784)(891.7725815,592.68356951)
\lineto(891.7725815,592.99856951)
\curveto(891.78258093,593.04856717)(891.78758093,593.09856712)(891.7875815,593.14856951)
\lineto(891.8175815,593.32856951)
\lineto(891.9375815,593.65856951)
\curveto(891.97758074,593.76856645)(892.02758069,593.86856635)(892.0875815,593.95856951)
\curveto(892.26758045,594.24856597)(892.5125802,594.46356575)(892.8225815,594.60356951)
\curveto(893.13257958,594.74356547)(893.47257924,594.86856535)(893.8425815,594.97856951)
\curveto(893.98257873,595.0185652)(894.12757859,595.04856517)(894.2775815,595.06856951)
\curveto(894.42757829,595.08856513)(894.57757814,595.1135651)(894.7275815,595.14356951)
\curveto(894.79757792,595.16356505)(894.86257785,595.17356504)(894.9225815,595.17356951)
\curveto(894.99257772,595.17356504)(895.06757765,595.18356503)(895.1475815,595.20356951)
\curveto(895.2175775,595.22356499)(895.28757743,595.23356498)(895.3575815,595.23356951)
\curveto(895.42757729,595.24356497)(895.50257721,595.25856496)(895.5825815,595.27856951)
\curveto(895.83257688,595.33856488)(896.06757665,595.38856483)(896.2875815,595.42856951)
\curveto(896.50757621,595.47856474)(896.68257603,595.59356462)(896.8125815,595.77356951)
\curveto(896.87257584,595.85356436)(896.92257579,595.95356426)(896.9625815,596.07356951)
\curveto(897.00257571,596.20356401)(897.00257571,596.34356387)(896.9625815,596.49356951)
\curveto(896.90257581,596.73356348)(896.8125759,596.92356329)(896.6925815,597.06356951)
\curveto(896.58257613,597.20356301)(896.42257629,597.3135629)(896.2125815,597.39356951)
\curveto(896.09257662,597.44356277)(895.94757677,597.47856274)(895.7775815,597.49856951)
\curveto(895.6175771,597.5185627)(895.44757727,597.52856269)(895.2675815,597.52856951)
\curveto(895.08757763,597.52856269)(894.9125778,597.5185627)(894.7425815,597.49856951)
\curveto(894.57257814,597.47856274)(894.42757829,597.44856277)(894.3075815,597.40856951)
\curveto(894.13757858,597.34856287)(893.97257874,597.26356295)(893.8125815,597.15356951)
\curveto(893.73257898,597.09356312)(893.65757906,597.0135632)(893.5875815,596.91356951)
\curveto(893.52757919,596.82356339)(893.47257924,596.72356349)(893.4225815,596.61356951)
\curveto(893.39257932,596.53356368)(893.36257935,596.44856377)(893.3325815,596.35856951)
\curveto(893.3125794,596.26856395)(893.26757945,596.19856402)(893.1975815,596.14856951)
\curveto(893.15757956,596.1185641)(893.08757963,596.09356412)(892.9875815,596.07356951)
\curveto(892.89757982,596.06356415)(892.80257991,596.05856416)(892.7025815,596.05856951)
\curveto(892.60258011,596.05856416)(892.50258021,596.06356415)(892.4025815,596.07356951)
\curveto(892.3125804,596.09356412)(892.24758047,596.1185641)(892.2075815,596.14856951)
\curveto(892.16758055,596.17856404)(892.13758058,596.22856399)(892.1175815,596.29856951)
\curveto(892.09758062,596.36856385)(892.09758062,596.44356377)(892.1175815,596.52356951)
\curveto(892.14758057,596.65356356)(892.17758054,596.77356344)(892.2075815,596.88356951)
\curveto(892.24758047,597.00356321)(892.29258042,597.1185631)(892.3425815,597.22856951)
\curveto(892.53258018,597.57856264)(892.77257994,597.84856237)(893.0625815,598.03856951)
\curveto(893.35257936,598.23856198)(893.712579,598.39856182)(894.1425815,598.51856951)
\curveto(894.24257847,598.53856168)(894.34257837,598.55356166)(894.4425815,598.56356951)
\curveto(894.55257816,598.57356164)(894.66257805,598.58856163)(894.7725815,598.60856951)
\curveto(894.8125779,598.6185616)(894.87757784,598.6185616)(894.9675815,598.60856951)
\curveto(895.05757766,598.60856161)(895.1125776,598.6185616)(895.1325815,598.63856951)
\curveto(895.83257688,598.64856157)(896.44257627,598.56856165)(896.9625815,598.39856951)
\curveto(897.48257523,598.22856199)(897.84757487,597.90356231)(898.0575815,597.42356951)
\curveto(898.14757457,597.22356299)(898.19757452,596.98856323)(898.2075815,596.71856951)
\curveto(898.22757449,596.45856376)(898.23757448,596.18356403)(898.2375815,595.89356951)
\lineto(898.2375815,592.57856951)
\curveto(898.23757448,592.43856778)(898.24257447,592.30356791)(898.2525815,592.17356951)
\curveto(898.26257445,592.04356817)(898.29257442,591.93856828)(898.3425815,591.85856951)
\curveto(898.39257432,591.78856843)(898.45757426,591.73856848)(898.5375815,591.70856951)
\curveto(898.62757409,591.66856855)(898.712574,591.63856858)(898.7925815,591.61856951)
\curveto(898.87257384,591.60856861)(898.93257378,591.56356865)(898.9725815,591.48356951)
\curveto(898.99257372,591.45356876)(899.00257371,591.42356879)(899.0025815,591.39356951)
\curveto(899.00257371,591.36356885)(899.00757371,591.32356889)(899.0175815,591.27356951)
\moveto(896.8725815,592.93856951)
\curveto(896.93257578,593.07856714)(896.96257575,593.23856698)(896.9625815,593.41856951)
\curveto(896.97257574,593.60856661)(896.97757574,593.80356641)(896.9775815,594.00356951)
\curveto(896.97757574,594.1135661)(896.97257574,594.213566)(896.9625815,594.30356951)
\curveto(896.95257576,594.39356582)(896.9125758,594.46356575)(896.8425815,594.51356951)
\curveto(896.8125759,594.53356568)(896.74257597,594.54356567)(896.6325815,594.54356951)
\curveto(896.6125761,594.52356569)(896.57757614,594.5135657)(896.5275815,594.51356951)
\curveto(896.47757624,594.5135657)(896.43257628,594.50356571)(896.3925815,594.48356951)
\curveto(896.3125764,594.46356575)(896.22257649,594.44356577)(896.1225815,594.42356951)
\lineto(895.8225815,594.36356951)
\curveto(895.79257692,594.36356585)(895.75757696,594.35856586)(895.7175815,594.34856951)
\lineto(895.6125815,594.34856951)
\curveto(895.46257725,594.30856591)(895.29757742,594.28356593)(895.1175815,594.27356951)
\curveto(894.94757777,594.27356594)(894.78757793,594.25356596)(894.6375815,594.21356951)
\curveto(894.55757816,594.19356602)(894.48257823,594.17356604)(894.4125815,594.15356951)
\curveto(894.35257836,594.14356607)(894.28257843,594.12856609)(894.2025815,594.10856951)
\curveto(894.04257867,594.05856616)(893.89257882,593.99356622)(893.7525815,593.91356951)
\curveto(893.6125791,593.84356637)(893.49257922,593.75356646)(893.3925815,593.64356951)
\curveto(893.29257942,593.53356668)(893.2175795,593.39856682)(893.1675815,593.23856951)
\curveto(893.1175796,593.08856713)(893.09757962,592.90356731)(893.1075815,592.68356951)
\curveto(893.10757961,592.58356763)(893.12257959,592.48856773)(893.1525815,592.39856951)
\curveto(893.19257952,592.3185679)(893.23757948,592.24356797)(893.2875815,592.17356951)
\curveto(893.36757935,592.06356815)(893.47257924,591.96856825)(893.6025815,591.88856951)
\curveto(893.73257898,591.8185684)(893.87257884,591.75856846)(894.0225815,591.70856951)
\curveto(894.07257864,591.69856852)(894.12257859,591.69356852)(894.1725815,591.69356951)
\curveto(894.22257849,591.69356852)(894.27257844,591.68856853)(894.3225815,591.67856951)
\curveto(894.39257832,591.65856856)(894.47757824,591.64356857)(894.5775815,591.63356951)
\curveto(894.68757803,591.63356858)(894.77757794,591.64356857)(894.8475815,591.66356951)
\curveto(894.90757781,591.68356853)(894.96757775,591.68856853)(895.0275815,591.67856951)
\curveto(895.08757763,591.67856854)(895.14757757,591.68856853)(895.2075815,591.70856951)
\curveto(895.28757743,591.72856849)(895.36257735,591.74356847)(895.4325815,591.75356951)
\curveto(895.5125772,591.76356845)(895.58757713,591.78356843)(895.6575815,591.81356951)
\curveto(895.94757677,591.93356828)(896.19257652,592.07856814)(896.3925815,592.24856951)
\curveto(896.60257611,592.4185678)(896.76257595,592.64856757)(896.8725815,592.93856951)
}
}
{
\newrgbcolor{curcolor}{0 0 0}
\pscustom[linestyle=none,fillstyle=solid,fillcolor=curcolor]
{
\newpath
\moveto(902.61922212,598.62356951)
\curveto(903.33921806,598.63356158)(903.94421745,598.54856167)(904.43422212,598.36856951)
\curveto(904.92421647,598.19856202)(905.30421609,597.89356232)(905.57422212,597.45356951)
\curveto(905.64421575,597.34356287)(905.6992157,597.22856299)(905.73922212,597.10856951)
\curveto(905.77921562,596.99856322)(905.81921558,596.87356334)(905.85922212,596.73356951)
\curveto(905.87921552,596.66356355)(905.88421551,596.58856363)(905.87422212,596.50856951)
\curveto(905.86421553,596.43856378)(905.84921555,596.38356383)(905.82922212,596.34356951)
\curveto(905.80921559,596.32356389)(905.78421561,596.30356391)(905.75422212,596.28356951)
\curveto(905.72421567,596.27356394)(905.6992157,596.25856396)(905.67922212,596.23856951)
\curveto(905.62921577,596.218564)(905.57921582,596.213564)(905.52922212,596.22356951)
\curveto(905.47921592,596.23356398)(905.42921597,596.23356398)(905.37922212,596.22356951)
\curveto(905.2992161,596.20356401)(905.1942162,596.19856402)(905.06422212,596.20856951)
\curveto(904.93421646,596.22856399)(904.84421655,596.25356396)(904.79422212,596.28356951)
\curveto(904.71421668,596.33356388)(904.65921674,596.39856382)(904.62922212,596.47856951)
\curveto(904.60921679,596.56856365)(904.57421682,596.65356356)(904.52422212,596.73356951)
\curveto(904.43421696,596.89356332)(904.30921709,597.03856318)(904.14922212,597.16856951)
\curveto(904.03921736,597.24856297)(903.91921748,597.30856291)(903.78922212,597.34856951)
\curveto(903.65921774,597.38856283)(903.51921788,597.42856279)(903.36922212,597.46856951)
\curveto(903.31921808,597.48856273)(903.26921813,597.49356272)(903.21922212,597.48356951)
\curveto(903.16921823,597.48356273)(903.11921828,597.48856273)(903.06922212,597.49856951)
\curveto(903.00921839,597.5185627)(902.93421846,597.52856269)(902.84422212,597.52856951)
\curveto(902.75421864,597.52856269)(902.67921872,597.5185627)(902.61922212,597.49856951)
\lineto(902.52922212,597.49856951)
\lineto(902.37922212,597.46856951)
\curveto(902.32921907,597.46856275)(902.27921912,597.46356275)(902.22922212,597.45356951)
\curveto(901.96921943,597.39356282)(901.75421964,597.30856291)(901.58422212,597.19856951)
\curveto(901.41421998,597.08856313)(901.2992201,596.90356331)(901.23922212,596.64356951)
\curveto(901.21922018,596.57356364)(901.21422018,596.50356371)(901.22422212,596.43356951)
\curveto(901.24422015,596.36356385)(901.26422013,596.30356391)(901.28422212,596.25356951)
\curveto(901.34422005,596.10356411)(901.41421998,595.99356422)(901.49422212,595.92356951)
\curveto(901.58421981,595.86356435)(901.6942197,595.79356442)(901.82422212,595.71356951)
\curveto(901.98421941,595.6135646)(902.16421923,595.53856468)(902.36422212,595.48856951)
\curveto(902.56421883,595.44856477)(902.76421863,595.39856482)(902.96422212,595.33856951)
\curveto(903.0942183,595.29856492)(903.22421817,595.26856495)(903.35422212,595.24856951)
\curveto(903.48421791,595.22856499)(903.61421778,595.19856502)(903.74422212,595.15856951)
\curveto(903.95421744,595.09856512)(904.15921724,595.03856518)(904.35922212,594.97856951)
\curveto(904.55921684,594.92856529)(904.75921664,594.86356535)(904.95922212,594.78356951)
\lineto(905.10922212,594.72356951)
\curveto(905.15921624,594.70356551)(905.20921619,594.67856554)(905.25922212,594.64856951)
\curveto(905.45921594,594.52856569)(905.63421576,594.39356582)(905.78422212,594.24356951)
\curveto(905.93421546,594.09356612)(906.05921534,593.90356631)(906.15922212,593.67356951)
\curveto(906.17921522,593.60356661)(906.1992152,593.50856671)(906.21922212,593.38856951)
\curveto(906.23921516,593.3185669)(906.24921515,593.24356697)(906.24922212,593.16356951)
\curveto(906.25921514,593.09356712)(906.26421513,593.0135672)(906.26422212,592.92356951)
\lineto(906.26422212,592.77356951)
\curveto(906.24421515,592.70356751)(906.23421516,592.63356758)(906.23422212,592.56356951)
\curveto(906.23421516,592.49356772)(906.22421517,592.42356779)(906.20422212,592.35356951)
\curveto(906.17421522,592.24356797)(906.13921526,592.13856808)(906.09922212,592.03856951)
\curveto(906.05921534,591.93856828)(906.01421538,591.84856837)(905.96422212,591.76856951)
\curveto(905.80421559,591.50856871)(905.5992158,591.29856892)(905.34922212,591.13856951)
\curveto(905.0992163,590.98856923)(904.81921658,590.85856936)(904.50922212,590.74856951)
\curveto(904.41921698,590.7185695)(904.32421707,590.69856952)(904.22422212,590.68856951)
\curveto(904.13421726,590.66856955)(904.04421735,590.64356957)(903.95422212,590.61356951)
\curveto(903.85421754,590.59356962)(903.75421764,590.58356963)(903.65422212,590.58356951)
\curveto(903.55421784,590.58356963)(903.45421794,590.57356964)(903.35422212,590.55356951)
\lineto(903.20422212,590.55356951)
\curveto(903.15421824,590.54356967)(903.08421831,590.53856968)(902.99422212,590.53856951)
\curveto(902.90421849,590.53856968)(902.83421856,590.54356967)(902.78422212,590.55356951)
\lineto(902.61922212,590.55356951)
\curveto(902.55921884,590.57356964)(902.4942189,590.58356963)(902.42422212,590.58356951)
\curveto(902.35421904,590.57356964)(902.2942191,590.57856964)(902.24422212,590.59856951)
\curveto(902.1942192,590.60856961)(902.12921927,590.6135696)(902.04922212,590.61356951)
\lineto(901.80922212,590.67356951)
\curveto(901.73921966,590.68356953)(901.66421973,590.70356951)(901.58422212,590.73356951)
\curveto(901.27422012,590.83356938)(901.00422039,590.95856926)(900.77422212,591.10856951)
\curveto(900.54422085,591.25856896)(900.34422105,591.45356876)(900.17422212,591.69356951)
\curveto(900.08422131,591.82356839)(900.00922139,591.95856826)(899.94922212,592.09856951)
\curveto(899.88922151,592.23856798)(899.83422156,592.39356782)(899.78422212,592.56356951)
\curveto(899.76422163,592.62356759)(899.75422164,592.69356752)(899.75422212,592.77356951)
\curveto(899.76422163,592.86356735)(899.77922162,592.93356728)(899.79922212,592.98356951)
\curveto(899.82922157,593.02356719)(899.87922152,593.06356715)(899.94922212,593.10356951)
\curveto(899.9992214,593.12356709)(900.06922133,593.13356708)(900.15922212,593.13356951)
\curveto(900.24922115,593.14356707)(900.33922106,593.14356707)(900.42922212,593.13356951)
\curveto(900.51922088,593.12356709)(900.60422079,593.10856711)(900.68422212,593.08856951)
\curveto(900.77422062,593.07856714)(900.83422056,593.06356715)(900.86422212,593.04356951)
\curveto(900.93422046,592.99356722)(900.97922042,592.9185673)(900.99922212,592.81856951)
\curveto(901.02922037,592.72856749)(901.06422033,592.64356757)(901.10422212,592.56356951)
\curveto(901.20422019,592.34356787)(901.33922006,592.17356804)(901.50922212,592.05356951)
\curveto(901.62921977,591.96356825)(901.76421963,591.89356832)(901.91422212,591.84356951)
\curveto(902.06421933,591.79356842)(902.22421917,591.74356847)(902.39422212,591.69356951)
\lineto(902.70922212,591.64856951)
\lineto(902.79922212,591.64856951)
\curveto(902.86921853,591.62856859)(902.95921844,591.6185686)(903.06922212,591.61856951)
\curveto(903.18921821,591.6185686)(903.28921811,591.62856859)(903.36922212,591.64856951)
\curveto(903.43921796,591.64856857)(903.4942179,591.65356856)(903.53422212,591.66356951)
\curveto(903.5942178,591.67356854)(903.65421774,591.67856854)(903.71422212,591.67856951)
\curveto(903.77421762,591.68856853)(903.82921757,591.69856852)(903.87922212,591.70856951)
\curveto(904.16921723,591.78856843)(904.399217,591.89356832)(904.56922212,592.02356951)
\curveto(904.73921666,592.15356806)(904.85921654,592.37356784)(904.92922212,592.68356951)
\curveto(904.94921645,592.73356748)(904.95421644,592.78856743)(904.94422212,592.84856951)
\curveto(904.93421646,592.90856731)(904.92421647,592.95356726)(904.91422212,592.98356951)
\curveto(904.86421653,593.17356704)(904.7942166,593.3135669)(904.70422212,593.40356951)
\curveto(904.61421678,593.50356671)(904.4992169,593.59356662)(904.35922212,593.67356951)
\curveto(904.26921713,593.73356648)(904.16921723,593.78356643)(904.05922212,593.82356951)
\lineto(903.72922212,593.94356951)
\curveto(903.6992177,593.95356626)(903.66921773,593.95856626)(903.63922212,593.95856951)
\curveto(903.61921778,593.95856626)(903.5942178,593.96856625)(903.56422212,593.98856951)
\curveto(903.22421817,594.09856612)(902.86921853,594.17856604)(902.49922212,594.22856951)
\curveto(902.13921926,594.28856593)(901.7992196,594.38356583)(901.47922212,594.51356951)
\curveto(901.37922002,594.55356566)(901.28422011,594.58856563)(901.19422212,594.61856951)
\curveto(901.10422029,594.64856557)(901.01922038,594.68856553)(900.93922212,594.73856951)
\curveto(900.74922065,594.84856537)(900.57422082,594.97356524)(900.41422212,595.11356951)
\curveto(900.25422114,595.25356496)(900.12922127,595.42856479)(900.03922212,595.63856951)
\curveto(900.00922139,595.70856451)(899.98422141,595.77856444)(899.96422212,595.84856951)
\curveto(899.95422144,595.9185643)(899.93922146,595.99356422)(899.91922212,596.07356951)
\curveto(899.88922151,596.19356402)(899.87922152,596.32856389)(899.88922212,596.47856951)
\curveto(899.8992215,596.63856358)(899.91422148,596.77356344)(899.93422212,596.88356951)
\curveto(899.95422144,596.93356328)(899.96422143,596.97356324)(899.96422212,597.00356951)
\curveto(899.97422142,597.04356317)(899.98922141,597.08356313)(900.00922212,597.12356951)
\curveto(900.0992213,597.35356286)(900.21922118,597.55356266)(900.36922212,597.72356951)
\curveto(900.52922087,597.89356232)(900.70922069,598.04356217)(900.90922212,598.17356951)
\curveto(901.05922034,598.26356195)(901.22422017,598.33356188)(901.40422212,598.38356951)
\curveto(901.58421981,598.44356177)(901.77421962,598.49856172)(901.97422212,598.54856951)
\curveto(902.04421935,598.55856166)(902.10921929,598.56856165)(902.16922212,598.57856951)
\curveto(902.23921916,598.58856163)(902.31421908,598.59856162)(902.39422212,598.60856951)
\curveto(902.42421897,598.6185616)(902.46421893,598.6185616)(902.51422212,598.60856951)
\curveto(902.56421883,598.59856162)(902.5992188,598.60356161)(902.61922212,598.62356951)
}
}
{
\newrgbcolor{curcolor}{0.50196081 0.50196081 0.50196081}
\pscustom[linestyle=none,fillstyle=solid,fillcolor=curcolor]
{
\newpath
\moveto(812.80437349,601.42860613)
\lineto(827.80437349,601.42860613)
\lineto(827.80437349,586.42860613)
\lineto(812.80437349,586.42860613)
\closepath
}
}
{
\newrgbcolor{curcolor}{0 0 0}
\pscustom[linestyle=none,fillstyle=solid,fillcolor=curcolor]
{
\newpath
\moveto(831.7125815,578.52639178)
\curveto(831.87258084,578.5263811)(832.04758067,578.5263811)(832.2375815,578.52639178)
\curveto(832.42758029,578.53638109)(832.57258014,578.51138111)(832.6725815,578.45139178)
\curveto(832.76257995,578.39138123)(832.82257989,578.29638133)(832.8525815,578.16639178)
\curveto(832.89257982,578.03638159)(832.93257978,577.91638171)(832.9725815,577.80639178)
\curveto(833.05257966,577.60638202)(833.12257959,577.40138222)(833.1825815,577.19139178)
\curveto(833.24257947,576.99138263)(833.3125794,576.79138283)(833.3925815,576.59139178)
\curveto(833.4125793,576.54138308)(833.42757929,576.49138313)(833.4375815,576.44139178)
\lineto(833.4675815,576.29139178)
\curveto(833.53757918,576.1213835)(833.59757912,575.94138368)(833.6475815,575.75139178)
\curveto(833.70757901,575.57138405)(833.76757895,575.38638424)(833.8275815,575.19639178)
\curveto(833.96757875,574.78638484)(834.10257861,574.38138524)(834.2325815,573.98139178)
\curveto(834.37257834,573.58138604)(834.5125782,573.17638645)(834.6525815,572.76639178)
\curveto(834.72257799,572.56638706)(834.78257793,572.36138726)(834.8325815,572.15139178)
\curveto(834.89257782,571.95138767)(834.96257775,571.75138787)(835.0425815,571.55139178)
\curveto(835.06257765,571.50138812)(835.07757764,571.44638818)(835.0875815,571.38639178)
\lineto(835.1475815,571.20639178)
\curveto(835.25757746,570.91638871)(835.35757736,570.61638901)(835.4475815,570.30639178)
\curveto(835.48757723,570.20638942)(835.52257719,570.10138952)(835.5525815,569.99139178)
\curveto(835.58257713,569.89138973)(835.62757709,569.80138982)(835.6875815,569.72139178)
\curveto(835.70757701,569.70138992)(835.74257697,569.67138995)(835.7925815,569.63139178)
\curveto(835.9125768,569.64138998)(835.98757673,569.69138993)(836.0175815,569.78139178)
\curveto(836.04757667,569.88138974)(836.08257663,569.97638965)(836.1225815,570.06639178)
\curveto(836.23257648,570.3263893)(836.32257639,570.59138903)(836.3925815,570.86139178)
\curveto(836.46257625,571.13138849)(836.54757617,571.39638823)(836.6475815,571.65639178)
\curveto(836.70757601,571.81638781)(836.75757596,571.97638765)(836.7975815,572.13639178)
\curveto(836.84757587,572.29638733)(836.90257581,572.45638717)(836.9625815,572.61639178)
\curveto(837.0125757,572.73638689)(837.05257566,572.85638677)(837.0825815,572.97639178)
\curveto(837.12257559,573.10638652)(837.16757555,573.23138639)(837.2175815,573.35139178)
\curveto(837.36757535,573.77138585)(837.50757521,574.19638543)(837.6375815,574.62639178)
\curveto(837.76757495,575.05638457)(837.9125748,575.48138414)(838.0725815,575.90139178)
\curveto(838.09257462,575.94138368)(838.10257461,575.97638365)(838.1025815,576.00639178)
\curveto(838.10257461,576.04638358)(838.1125746,576.08638354)(838.1325815,576.12639178)
\curveto(838.19257452,576.27638335)(838.24757447,576.43138319)(838.2975815,576.59139178)
\curveto(838.34757437,576.75138287)(838.39757432,576.90638272)(838.4475815,577.05639178)
\curveto(838.50757421,577.20638242)(838.55757416,577.35638227)(838.5975815,577.50639178)
\curveto(838.64757407,577.66638196)(838.70257401,577.8263818)(838.7625815,577.98639178)
\curveto(838.79257392,578.07638155)(838.82257389,578.16138146)(838.8525815,578.24139178)
\curveto(838.89257382,578.33138129)(838.95257376,578.40138122)(839.0325815,578.45139178)
\curveto(839.09257362,578.50138112)(839.17257354,578.5263811)(839.2725815,578.52639178)
\curveto(839.38257333,578.5263811)(839.49257322,578.5263811)(839.6025815,578.52639178)
\lineto(839.9325815,578.52639178)
\curveto(840.0125727,578.50638112)(840.08257263,578.48638114)(840.1425815,578.46639178)
\curveto(840.20257251,578.45638117)(840.24257247,578.41138121)(840.2625815,578.33139178)
\lineto(840.2625815,578.25639178)
\curveto(840.27257244,578.23638139)(840.27257244,578.21638141)(840.2625815,578.19639178)
\curveto(840.24257247,578.09638153)(840.2125725,577.99638163)(840.1725815,577.89639178)
\curveto(840.14257257,577.80638182)(840.1125726,577.7213819)(840.0825815,577.64139178)
\curveto(840.04257267,577.56138206)(840.00757271,577.47638215)(839.9775815,577.38639178)
\curveto(839.95757276,577.30638232)(839.93257278,577.2263824)(839.9025815,577.14639178)
\curveto(839.84257287,577.00638262)(839.78757293,576.85638277)(839.7375815,576.69639178)
\curveto(839.69757302,576.54638308)(839.64757307,576.40138322)(839.5875815,576.26139178)
\curveto(839.56757315,576.2213834)(839.55757316,576.18638344)(839.5575815,576.15639178)
\curveto(839.55757316,576.1263835)(839.54757317,576.09138353)(839.5275815,576.05139178)
\curveto(839.44757327,575.88138374)(839.37757334,575.70138392)(839.3175815,575.51139178)
\curveto(839.26757345,575.3213843)(839.20257351,575.14138448)(839.1225815,574.97139178)
\curveto(839.10257361,574.93138469)(839.09257362,574.89138473)(839.0925815,574.85139178)
\curveto(839.09257362,574.8213848)(839.08257363,574.79138483)(839.0625815,574.76139178)
\curveto(839.0125737,574.63138499)(838.96257375,574.49638513)(838.9125815,574.35639178)
\curveto(838.87257384,574.2263854)(838.82757389,574.09638553)(838.7775815,573.96639178)
\curveto(838.75757396,573.93638569)(838.74257397,573.90138572)(838.7325815,573.86139178)
\curveto(838.73257398,573.83138579)(838.72257399,573.79638583)(838.7025815,573.75639178)
\curveto(838.55257416,573.37638625)(838.4125743,572.99138663)(838.2825815,572.60139178)
\curveto(838.16257455,572.21138741)(838.02757469,571.8263878)(837.8775815,571.44639178)
\lineto(837.8325815,571.31139178)
\lineto(837.7125815,570.98139178)
\curveto(837.68257503,570.88138874)(837.64757507,570.77638885)(837.6075815,570.66639178)
\curveto(837.55757516,570.54638908)(837.5125752,570.4213892)(837.4725815,570.29139178)
\curveto(837.43257528,570.17138945)(837.38757533,570.05138957)(837.3375815,569.93139178)
\lineto(837.2775815,569.75139178)
\lineto(837.2175815,569.57139178)
\curveto(837.15757556,569.4213902)(837.10257561,569.26639036)(837.0525815,569.10639178)
\curveto(837.00257571,568.94639068)(836.94757577,568.79639083)(836.8875815,568.65639178)
\curveto(836.83757588,568.5263911)(836.78757593,568.38639124)(836.7375815,568.23639178)
\curveto(836.69757602,568.09639153)(836.62757609,567.99139163)(836.5275815,567.92139178)
\curveto(836.48757623,567.90139172)(836.44257627,567.88639174)(836.3925815,567.87639178)
\curveto(836.34257637,567.86639176)(836.28757643,567.85639177)(836.2275815,567.84639178)
\lineto(835.8075815,567.84639178)
\lineto(835.4175815,567.84639178)
\curveto(835.27757744,567.83639179)(835.16757755,567.85639177)(835.0875815,567.90639178)
\curveto(834.99757772,567.95639167)(834.93757778,568.0263916)(834.9075815,568.11639178)
\curveto(834.87757784,568.21639141)(834.83757788,568.3213913)(834.7875815,568.43139178)
\curveto(834.72757799,568.58139104)(834.67257804,568.73639089)(834.6225815,568.89639178)
\curveto(834.57257814,569.06639056)(834.5125782,569.23139039)(834.4425815,569.39139178)
\curveto(834.42257829,569.43139019)(834.40757831,569.47139015)(834.3975815,569.51139178)
\curveto(834.39757832,569.55139007)(834.38757833,569.59139003)(834.3675815,569.63139178)
\curveto(834.28757843,569.83138979)(834.2125785,570.03138959)(834.1425815,570.23139178)
\curveto(834.08257863,570.44138918)(834.0125787,570.64138898)(833.9325815,570.83139178)
\curveto(833.9125788,570.88138874)(833.89757882,570.9263887)(833.8875815,570.96639178)
\curveto(833.88757883,571.00638862)(833.87757884,571.04638858)(833.8575815,571.08639178)
\curveto(833.80757891,571.2263884)(833.75757896,571.36138826)(833.7075815,571.49139178)
\lineto(833.5575815,571.91139178)
\curveto(833.53757918,571.95138767)(833.52257919,571.99138763)(833.5125815,572.03139178)
\curveto(833.5125792,572.07138755)(833.50257921,572.11138751)(833.4825815,572.15139178)
\lineto(833.3325815,572.54139178)
\curveto(833.29257942,572.68138694)(833.24757947,572.8213868)(833.1975815,572.96139178)
\curveto(833.14757957,573.07138655)(833.10757961,573.18138644)(833.0775815,573.29139178)
\curveto(833.04757967,573.41138621)(833.00757971,573.5263861)(832.9575815,573.63639178)
\curveto(832.84757987,573.91638571)(832.74757997,574.20138542)(832.6575815,574.49139178)
\curveto(832.56758015,574.79138483)(832.46258025,575.08138454)(832.3425815,575.36139178)
\curveto(832.30258041,575.45138417)(832.26758045,575.54138408)(832.2375815,575.63139178)
\curveto(832.2175805,575.73138389)(832.19258052,575.8213838)(832.1625815,575.90139178)
\curveto(832.13258058,575.96138366)(832.10758061,576.0213836)(832.0875815,576.08139178)
\curveto(832.07758064,576.15138347)(832.05758066,576.21638341)(832.0275815,576.27639178)
\curveto(831.93758078,576.50638312)(831.85258086,576.74138288)(831.7725815,576.98139178)
\curveto(831.70258101,577.2213824)(831.62258109,577.45638217)(831.5325815,577.68639178)
\curveto(831.5125812,577.75638187)(831.48758123,577.8263818)(831.4575815,577.89639178)
\curveto(831.43758128,577.96638166)(831.4175813,578.04138158)(831.3975815,578.12139178)
\curveto(831.35758136,578.2213814)(831.35258136,578.30638132)(831.3825815,578.37639178)
\curveto(831.4125813,578.44638118)(831.48258123,578.49138113)(831.5925815,578.51139178)
\curveto(831.6125811,578.5213811)(831.63258108,578.5213811)(831.6525815,578.51139178)
\curveto(831.67258104,578.51138111)(831.69258102,578.51638111)(831.7125815,578.52639178)
}
}
{
\newrgbcolor{curcolor}{0 0 0}
\pscustom[linestyle=none,fillstyle=solid,fillcolor=curcolor]
{
\newpath
\moveto(841.58750337,577.08639178)
\curveto(841.50750225,577.14638248)(841.4625023,577.25138237)(841.45250337,577.40139178)
\lineto(841.45250337,577.86639178)
\lineto(841.45250337,578.12139178)
\curveto(841.45250231,578.21138141)(841.46750229,578.28638134)(841.49750337,578.34639178)
\curveto(841.53750222,578.4263812)(841.61750214,578.48638114)(841.73750337,578.52639178)
\curveto(841.757502,578.53638109)(841.77750198,578.53638109)(841.79750337,578.52639178)
\curveto(841.82750193,578.5263811)(841.85250191,578.53138109)(841.87250337,578.54139178)
\curveto(842.04250172,578.54138108)(842.20250156,578.53638109)(842.35250337,578.52639178)
\curveto(842.50250126,578.51638111)(842.60250116,578.45638117)(842.65250337,578.34639178)
\curveto(842.68250108,578.28638134)(842.69750106,578.21138141)(842.69750337,578.12139178)
\lineto(842.69750337,577.86639178)
\curveto(842.69750106,577.68638194)(842.69250107,577.51638211)(842.68250337,577.35639178)
\curveto(842.68250108,577.19638243)(842.61750114,577.09138253)(842.48750337,577.04139178)
\curveto(842.43750132,577.0213826)(842.38250138,577.01138261)(842.32250337,577.01139178)
\lineto(842.15750337,577.01139178)
\lineto(841.84250337,577.01139178)
\curveto(841.74250202,577.01138261)(841.6575021,577.03638259)(841.58750337,577.08639178)
\moveto(842.69750337,568.58139178)
\lineto(842.69750337,568.26639178)
\curveto(842.70750105,568.16639146)(842.68750107,568.08639154)(842.63750337,568.02639178)
\curveto(842.60750115,567.96639166)(842.5625012,567.9263917)(842.50250337,567.90639178)
\curveto(842.44250132,567.89639173)(842.37250139,567.88139174)(842.29250337,567.86139178)
\lineto(842.06750337,567.86139178)
\curveto(841.93750182,567.86139176)(841.82250194,567.86639176)(841.72250337,567.87639178)
\curveto(841.63250213,567.89639173)(841.5625022,567.94639168)(841.51250337,568.02639178)
\curveto(841.47250229,568.08639154)(841.45250231,568.16139146)(841.45250337,568.25139178)
\lineto(841.45250337,568.53639178)
\lineto(841.45250337,574.88139178)
\lineto(841.45250337,575.19639178)
\curveto(841.45250231,575.30638432)(841.47750228,575.39138423)(841.52750337,575.45139178)
\curveto(841.5575022,575.50138412)(841.59750216,575.53138409)(841.64750337,575.54139178)
\curveto(841.69750206,575.55138407)(841.75250201,575.56638406)(841.81250337,575.58639178)
\curveto(841.83250193,575.58638404)(841.85250191,575.58138404)(841.87250337,575.57139178)
\curveto(841.90250186,575.57138405)(841.92750183,575.57638405)(841.94750337,575.58639178)
\curveto(842.07750168,575.58638404)(842.20750155,575.58138404)(842.33750337,575.57139178)
\curveto(842.47750128,575.57138405)(842.57250119,575.53138409)(842.62250337,575.45139178)
\curveto(842.67250109,575.39138423)(842.69750106,575.31138431)(842.69750337,575.21139178)
\lineto(842.69750337,574.92639178)
\lineto(842.69750337,568.58139178)
}
}
{
\newrgbcolor{curcolor}{0 0 0}
\pscustom[linestyle=none,fillstyle=solid,fillcolor=curcolor]
{
\newpath
\moveto(851.60234712,568.67139178)
\lineto(851.60234712,568.28139178)
\curveto(851.60233925,568.16139146)(851.57733927,568.06139156)(851.52734712,567.98139178)
\curveto(851.47733937,567.91139171)(851.39233946,567.87139175)(851.27234712,567.86139178)
\lineto(850.92734712,567.86139178)
\curveto(850.86733998,567.86139176)(850.80734004,567.85639177)(850.74734712,567.84639178)
\curveto(850.69734015,567.84639178)(850.6523402,567.85639177)(850.61234712,567.87639178)
\curveto(850.52234033,567.89639173)(850.46234039,567.93639169)(850.43234712,567.99639178)
\curveto(850.39234046,568.04639158)(850.36734048,568.10639152)(850.35734712,568.17639178)
\curveto(850.35734049,568.24639138)(850.34234051,568.31639131)(850.31234712,568.38639178)
\curveto(850.30234055,568.40639122)(850.28734056,568.4213912)(850.26734712,568.43139178)
\curveto(850.25734059,568.45139117)(850.24234061,568.47139115)(850.22234712,568.49139178)
\curveto(850.12234073,568.50139112)(850.04234081,568.48139114)(849.98234712,568.43139178)
\curveto(849.93234092,568.38139124)(849.87734097,568.33139129)(849.81734712,568.28139178)
\curveto(849.61734123,568.13139149)(849.41734143,568.01639161)(849.21734712,567.93639178)
\curveto(849.03734181,567.85639177)(848.82734202,567.79639183)(848.58734712,567.75639178)
\curveto(848.35734249,567.71639191)(848.11734273,567.69639193)(847.86734712,567.69639178)
\curveto(847.62734322,567.68639194)(847.38734346,567.70139192)(847.14734712,567.74139178)
\curveto(846.90734394,567.77139185)(846.69734415,567.8263918)(846.51734712,567.90639178)
\curveto(845.99734485,568.1263915)(845.57734527,568.4213912)(845.25734712,568.79139178)
\curveto(844.93734591,569.17139045)(844.68734616,569.64138998)(844.50734712,570.20139178)
\curveto(844.46734638,570.29138933)(844.43734641,570.38138924)(844.41734712,570.47139178)
\curveto(844.40734644,570.57138905)(844.38734646,570.67138895)(844.35734712,570.77139178)
\curveto(844.3473465,570.8213888)(844.34234651,570.87138875)(844.34234712,570.92139178)
\curveto(844.34234651,570.97138865)(844.33734651,571.0213886)(844.32734712,571.07139178)
\curveto(844.30734654,571.1213885)(844.29734655,571.17138845)(844.29734712,571.22139178)
\curveto(844.30734654,571.28138834)(844.30734654,571.33638829)(844.29734712,571.38639178)
\lineto(844.29734712,571.53639178)
\curveto(844.27734657,571.58638804)(844.26734658,571.65138797)(844.26734712,571.73139178)
\curveto(844.26734658,571.81138781)(844.27734657,571.87638775)(844.29734712,571.92639178)
\lineto(844.29734712,572.09139178)
\curveto(844.31734653,572.16138746)(844.32234653,572.23138739)(844.31234712,572.30139178)
\curveto(844.31234654,572.38138724)(844.32234653,572.45638717)(844.34234712,572.52639178)
\curveto(844.3523465,572.57638705)(844.35734649,572.621387)(844.35734712,572.66139178)
\curveto(844.35734649,572.70138692)(844.36234649,572.74638688)(844.37234712,572.79639178)
\curveto(844.40234645,572.89638673)(844.42734642,572.99138663)(844.44734712,573.08139178)
\curveto(844.46734638,573.18138644)(844.49234636,573.27638635)(844.52234712,573.36639178)
\curveto(844.6523462,573.74638588)(844.81734603,574.08638554)(845.01734712,574.38639178)
\curveto(845.22734562,574.69638493)(845.47734537,574.95138467)(845.76734712,575.15139178)
\curveto(845.93734491,575.27138435)(846.11234474,575.37138425)(846.29234712,575.45139178)
\curveto(846.48234437,575.53138409)(846.68734416,575.60138402)(846.90734712,575.66139178)
\curveto(846.97734387,575.67138395)(847.04234381,575.68138394)(847.10234712,575.69139178)
\curveto(847.17234368,575.70138392)(847.24234361,575.71638391)(847.31234712,575.73639178)
\lineto(847.46234712,575.73639178)
\curveto(847.54234331,575.75638387)(847.65734319,575.76638386)(847.80734712,575.76639178)
\curveto(847.96734288,575.76638386)(848.08734276,575.75638387)(848.16734712,575.73639178)
\curveto(848.20734264,575.7263839)(848.26234259,575.7213839)(848.33234712,575.72139178)
\curveto(848.44234241,575.69138393)(848.5523423,575.66638396)(848.66234712,575.64639178)
\curveto(848.77234208,575.63638399)(848.87734197,575.60638402)(848.97734712,575.55639178)
\curveto(849.12734172,575.49638413)(849.26734158,575.43138419)(849.39734712,575.36139178)
\curveto(849.53734131,575.29138433)(849.66734118,575.21138441)(849.78734712,575.12139178)
\curveto(849.847341,575.07138455)(849.90734094,575.01638461)(849.96734712,574.95639178)
\curveto(850.03734081,574.90638472)(850.12734072,574.89138473)(850.23734712,574.91139178)
\curveto(850.25734059,574.94138468)(850.27234058,574.96638466)(850.28234712,574.98639178)
\curveto(850.30234055,575.00638462)(850.31734053,575.03638459)(850.32734712,575.07639178)
\curveto(850.35734049,575.16638446)(850.36734048,575.28138434)(850.35734712,575.42139178)
\lineto(850.35734712,575.79639178)
\lineto(850.35734712,577.52139178)
\lineto(850.35734712,577.98639178)
\curveto(850.35734049,578.16638146)(850.38234047,578.29638133)(850.43234712,578.37639178)
\curveto(850.47234038,578.44638118)(850.53234032,578.49138113)(850.61234712,578.51139178)
\curveto(850.63234022,578.51138111)(850.65734019,578.51138111)(850.68734712,578.51139178)
\curveto(850.71734013,578.5213811)(850.74234011,578.5263811)(850.76234712,578.52639178)
\curveto(850.90233995,578.53638109)(851.0473398,578.53638109)(851.19734712,578.52639178)
\curveto(851.35733949,578.5263811)(851.46733938,578.48638114)(851.52734712,578.40639178)
\curveto(851.57733927,578.3263813)(851.60233925,578.2263814)(851.60234712,578.10639178)
\lineto(851.60234712,577.73139178)
\lineto(851.60234712,568.67139178)
\moveto(850.38734712,571.50639178)
\curveto(850.40734044,571.55638807)(850.41734043,571.621388)(850.41734712,571.70139178)
\curveto(850.41734043,571.79138783)(850.40734044,571.86138776)(850.38734712,571.91139178)
\lineto(850.38734712,572.13639178)
\curveto(850.36734048,572.2263874)(850.3523405,572.31638731)(850.34234712,572.40639178)
\curveto(850.33234052,572.50638712)(850.31234054,572.59638703)(850.28234712,572.67639178)
\curveto(850.26234059,572.75638687)(850.24234061,572.83138679)(850.22234712,572.90139178)
\curveto(850.21234064,572.97138665)(850.19234066,573.04138658)(850.16234712,573.11139178)
\curveto(850.04234081,573.41138621)(849.88734096,573.67638595)(849.69734712,573.90639178)
\curveto(849.50734134,574.13638549)(849.26734158,574.31638531)(848.97734712,574.44639178)
\curveto(848.87734197,574.49638513)(848.77234208,574.53138509)(848.66234712,574.55139178)
\curveto(848.56234229,574.58138504)(848.4523424,574.60638502)(848.33234712,574.62639178)
\curveto(848.2523426,574.64638498)(848.16234269,574.65638497)(848.06234712,574.65639178)
\lineto(847.79234712,574.65639178)
\curveto(847.74234311,574.64638498)(847.69734315,574.63638499)(847.65734712,574.62639178)
\lineto(847.52234712,574.62639178)
\curveto(847.44234341,574.60638502)(847.35734349,574.58638504)(847.26734712,574.56639178)
\curveto(847.18734366,574.54638508)(847.10734374,574.5213851)(847.02734712,574.49139178)
\curveto(846.70734414,574.35138527)(846.4473444,574.14638548)(846.24734712,573.87639178)
\curveto(846.05734479,573.61638601)(845.90234495,573.31138631)(845.78234712,572.96139178)
\curveto(845.74234511,572.85138677)(845.71234514,572.73638689)(845.69234712,572.61639178)
\curveto(845.68234517,572.50638712)(845.66734518,572.39638723)(845.64734712,572.28639178)
\curveto(845.6473452,572.24638738)(845.64234521,572.20638742)(845.63234712,572.16639178)
\lineto(845.63234712,572.06139178)
\curveto(845.61234524,572.01138761)(845.60234525,571.95638767)(845.60234712,571.89639178)
\curveto(845.61234524,571.83638779)(845.61734523,571.78138784)(845.61734712,571.73139178)
\lineto(845.61734712,571.40139178)
\curveto(845.61734523,571.30138832)(845.62734522,571.20638842)(845.64734712,571.11639178)
\curveto(845.65734519,571.08638854)(845.66234519,571.03638859)(845.66234712,570.96639178)
\curveto(845.68234517,570.89638873)(845.69734515,570.8263888)(845.70734712,570.75639178)
\lineto(845.76734712,570.54639178)
\curveto(845.87734497,570.19638943)(846.02734482,569.89638973)(846.21734712,569.64639178)
\curveto(846.40734444,569.39639023)(846.6473442,569.19139043)(846.93734712,569.03139178)
\curveto(847.02734382,568.98139064)(847.11734373,568.94139068)(847.20734712,568.91139178)
\curveto(847.29734355,568.88139074)(847.39734345,568.85139077)(847.50734712,568.82139178)
\curveto(847.55734329,568.80139082)(847.60734324,568.79639083)(847.65734712,568.80639178)
\curveto(847.71734313,568.81639081)(847.77234308,568.81139081)(847.82234712,568.79139178)
\curveto(847.86234299,568.78139084)(847.90234295,568.77639085)(847.94234712,568.77639178)
\lineto(848.07734712,568.77639178)
\lineto(848.21234712,568.77639178)
\curveto(848.24234261,568.78639084)(848.29234256,568.79139083)(848.36234712,568.79139178)
\curveto(848.44234241,568.81139081)(848.52234233,568.8263908)(848.60234712,568.83639178)
\curveto(848.68234217,568.85639077)(848.75734209,568.88139074)(848.82734712,568.91139178)
\curveto(849.15734169,569.05139057)(849.42234143,569.2263904)(849.62234712,569.43639178)
\curveto(849.83234102,569.65638997)(850.00734084,569.93138969)(850.14734712,570.26139178)
\curveto(850.19734065,570.37138925)(850.23234062,570.48138914)(850.25234712,570.59139178)
\curveto(850.27234058,570.70138892)(850.29734055,570.81138881)(850.32734712,570.92139178)
\curveto(850.3473405,570.96138866)(850.35734049,570.99638863)(850.35734712,571.02639178)
\curveto(850.35734049,571.06638856)(850.36234049,571.10638852)(850.37234712,571.14639178)
\curveto(850.38234047,571.20638842)(850.38234047,571.26638836)(850.37234712,571.32639178)
\curveto(850.37234048,571.38638824)(850.37734047,571.44638818)(850.38734712,571.50639178)
}
}
{
\newrgbcolor{curcolor}{0 0 0}
\pscustom[linestyle=none,fillstyle=solid,fillcolor=curcolor]
{
\newpath
\moveto(860.29859712,572.03139178)
\curveto(860.31858944,571.93138769)(860.31858944,571.81638781)(860.29859712,571.68639178)
\curveto(860.28858947,571.56638806)(860.2585895,571.48138814)(860.20859712,571.43139178)
\curveto(860.1585896,571.39138823)(860.08358967,571.36138826)(859.98359712,571.34139178)
\curveto(859.89358986,571.33138829)(859.78858997,571.3263883)(859.66859712,571.32639178)
\lineto(859.30859712,571.32639178)
\curveto(859.18859057,571.33638829)(859.08359067,571.34138828)(858.99359712,571.34139178)
\lineto(855.15359712,571.34139178)
\curveto(855.07359468,571.34138828)(854.99359476,571.33638829)(854.91359712,571.32639178)
\curveto(854.83359492,571.3263883)(854.76859499,571.31138831)(854.71859712,571.28139178)
\curveto(854.67859508,571.26138836)(854.63859512,571.2213884)(854.59859712,571.16139178)
\curveto(854.57859518,571.13138849)(854.5585952,571.08638854)(854.53859712,571.02639178)
\curveto(854.51859524,570.97638865)(854.51859524,570.9263887)(854.53859712,570.87639178)
\curveto(854.54859521,570.8263888)(854.5535952,570.78138884)(854.55359712,570.74139178)
\curveto(854.5535952,570.70138892)(854.5585952,570.66138896)(854.56859712,570.62139178)
\curveto(854.58859517,570.54138908)(854.60859515,570.45638917)(854.62859712,570.36639178)
\curveto(854.64859511,570.28638934)(854.67859508,570.20638942)(854.71859712,570.12639178)
\curveto(854.94859481,569.58639004)(855.32859443,569.20139042)(855.85859712,568.97139178)
\curveto(855.91859384,568.94139068)(855.98359377,568.91639071)(856.05359712,568.89639178)
\lineto(856.26359712,568.83639178)
\curveto(856.29359346,568.8263908)(856.34359341,568.8213908)(856.41359712,568.82139178)
\curveto(856.5535932,568.78139084)(856.73859302,568.76139086)(856.96859712,568.76139178)
\curveto(857.19859256,568.76139086)(857.38359237,568.78139084)(857.52359712,568.82139178)
\curveto(857.66359209,568.86139076)(857.78859197,568.90139072)(857.89859712,568.94139178)
\curveto(858.01859174,568.99139063)(858.12859163,569.05139057)(858.22859712,569.12139178)
\curveto(858.33859142,569.19139043)(858.43359132,569.27139035)(858.51359712,569.36139178)
\curveto(858.59359116,569.46139016)(858.66359109,569.56639006)(858.72359712,569.67639178)
\curveto(858.78359097,569.77638985)(858.83359092,569.88138974)(858.87359712,569.99139178)
\curveto(858.92359083,570.10138952)(859.00359075,570.18138944)(859.11359712,570.23139178)
\curveto(859.1535906,570.25138937)(859.21859054,570.26638936)(859.30859712,570.27639178)
\curveto(859.39859036,570.28638934)(859.48859027,570.28638934)(859.57859712,570.27639178)
\curveto(859.66859009,570.27638935)(859.75359,570.27138935)(859.83359712,570.26139178)
\curveto(859.91358984,570.25138937)(859.96858979,570.23138939)(859.99859712,570.20139178)
\curveto(860.09858966,570.13138949)(860.12358963,570.01638961)(860.07359712,569.85639178)
\curveto(859.99358976,569.58639004)(859.88858987,569.34639028)(859.75859712,569.13639178)
\curveto(859.5585902,568.81639081)(859.32859043,568.55139107)(859.06859712,568.34139178)
\curveto(858.81859094,568.14139148)(858.49859126,567.97639165)(858.10859712,567.84639178)
\curveto(858.00859175,567.80639182)(857.90859185,567.78139184)(857.80859712,567.77139178)
\curveto(857.70859205,567.75139187)(857.60359215,567.73139189)(857.49359712,567.71139178)
\curveto(857.44359231,567.70139192)(857.39359236,567.69639193)(857.34359712,567.69639178)
\curveto(857.30359245,567.69639193)(857.2585925,567.69139193)(857.20859712,567.68139178)
\lineto(857.05859712,567.68139178)
\curveto(857.00859275,567.67139195)(856.94859281,567.66639196)(856.87859712,567.66639178)
\curveto(856.81859294,567.66639196)(856.76859299,567.67139195)(856.72859712,567.68139178)
\lineto(856.59359712,567.68139178)
\curveto(856.54359321,567.69139193)(856.49859326,567.69639193)(856.45859712,567.69639178)
\curveto(856.41859334,567.69639193)(856.37859338,567.70139192)(856.33859712,567.71139178)
\curveto(856.28859347,567.7213919)(856.23359352,567.73139189)(856.17359712,567.74139178)
\curveto(856.11359364,567.74139188)(856.0585937,567.74639188)(856.00859712,567.75639178)
\curveto(855.91859384,567.77639185)(855.82859393,567.80139182)(855.73859712,567.83139178)
\curveto(855.64859411,567.85139177)(855.56359419,567.87639175)(855.48359712,567.90639178)
\curveto(855.44359431,567.9263917)(855.40859435,567.93639169)(855.37859712,567.93639178)
\curveto(855.34859441,567.94639168)(855.31359444,567.96139166)(855.27359712,567.98139178)
\curveto(855.12359463,568.05139157)(854.96359479,568.13639149)(854.79359712,568.23639178)
\curveto(854.50359525,568.4263912)(854.2535955,568.65639097)(854.04359712,568.92639178)
\curveto(853.84359591,569.20639042)(853.67359608,569.51639011)(853.53359712,569.85639178)
\curveto(853.48359627,569.96638966)(853.44359631,570.08138954)(853.41359712,570.20139178)
\curveto(853.39359636,570.3213893)(853.36359639,570.44138918)(853.32359712,570.56139178)
\curveto(853.31359644,570.60138902)(853.30859645,570.63638899)(853.30859712,570.66639178)
\curveto(853.30859645,570.69638893)(853.30359645,570.73638889)(853.29359712,570.78639178)
\curveto(853.27359648,570.86638876)(853.2585965,570.95138867)(853.24859712,571.04139178)
\curveto(853.23859652,571.13138849)(853.22359653,571.2213884)(853.20359712,571.31139178)
\lineto(853.20359712,571.52139178)
\curveto(853.19359656,571.56138806)(853.18359657,571.61638801)(853.17359712,571.68639178)
\curveto(853.17359658,571.76638786)(853.17859658,571.83138779)(853.18859712,571.88139178)
\lineto(853.18859712,572.04639178)
\curveto(853.20859655,572.09638753)(853.21359654,572.14638748)(853.20359712,572.19639178)
\curveto(853.20359655,572.25638737)(853.20859655,572.31138731)(853.21859712,572.36139178)
\curveto(853.2585965,572.5213871)(853.28859647,572.68138694)(853.30859712,572.84139178)
\curveto(853.33859642,573.00138662)(853.38359637,573.15138647)(853.44359712,573.29139178)
\curveto(853.49359626,573.40138622)(853.53859622,573.51138611)(853.57859712,573.62139178)
\curveto(853.62859613,573.74138588)(853.68359607,573.85638577)(853.74359712,573.96639178)
\curveto(853.96359579,574.31638531)(854.21359554,574.61638501)(854.49359712,574.86639178)
\curveto(854.77359498,575.1263845)(855.11859464,575.34138428)(855.52859712,575.51139178)
\curveto(855.64859411,575.56138406)(855.76859399,575.59638403)(855.88859712,575.61639178)
\curveto(856.01859374,575.64638398)(856.1535936,575.67638395)(856.29359712,575.70639178)
\curveto(856.34359341,575.71638391)(856.38859337,575.7213839)(856.42859712,575.72139178)
\curveto(856.46859329,575.73138389)(856.51359324,575.73638389)(856.56359712,575.73639178)
\curveto(856.58359317,575.74638388)(856.60859315,575.74638388)(856.63859712,575.73639178)
\curveto(856.66859309,575.7263839)(856.69359306,575.73138389)(856.71359712,575.75139178)
\curveto(857.13359262,575.76138386)(857.49859226,575.71638391)(857.80859712,575.61639178)
\curveto(858.11859164,575.5263841)(858.39859136,575.40138422)(858.64859712,575.24139178)
\curveto(858.69859106,575.2213844)(858.73859102,575.19138443)(858.76859712,575.15139178)
\curveto(858.79859096,575.1213845)(858.83359092,575.09638453)(858.87359712,575.07639178)
\curveto(858.9535908,575.01638461)(859.03359072,574.94638468)(859.11359712,574.86639178)
\curveto(859.20359055,574.78638484)(859.27859048,574.70638492)(859.33859712,574.62639178)
\curveto(859.49859026,574.41638521)(859.63359012,574.21638541)(859.74359712,574.02639178)
\curveto(859.81358994,573.91638571)(859.86858989,573.79638583)(859.90859712,573.66639178)
\curveto(859.94858981,573.53638609)(859.99358976,573.40638622)(860.04359712,573.27639178)
\curveto(860.09358966,573.14638648)(860.12858963,573.01138661)(860.14859712,572.87139178)
\curveto(860.17858958,572.73138689)(860.21358954,572.59138703)(860.25359712,572.45139178)
\curveto(860.26358949,572.38138724)(860.26858949,572.31138731)(860.26859712,572.24139178)
\lineto(860.29859712,572.03139178)
\moveto(858.84359712,572.54139178)
\curveto(858.87359088,572.58138704)(858.89859086,572.63138699)(858.91859712,572.69139178)
\curveto(858.93859082,572.76138686)(858.93859082,572.83138679)(858.91859712,572.90139178)
\curveto(858.8585909,573.1213865)(858.77359098,573.3263863)(858.66359712,573.51639178)
\curveto(858.52359123,573.74638588)(858.36859139,573.94138568)(858.19859712,574.10139178)
\curveto(858.02859173,574.26138536)(857.80859195,574.39638523)(857.53859712,574.50639178)
\curveto(857.46859229,574.5263851)(857.39859236,574.54138508)(857.32859712,574.55139178)
\curveto(857.2585925,574.57138505)(857.18359257,574.59138503)(857.10359712,574.61139178)
\curveto(857.02359273,574.63138499)(856.93859282,574.64138498)(856.84859712,574.64139178)
\lineto(856.59359712,574.64139178)
\curveto(856.56359319,574.621385)(856.52859323,574.61138501)(856.48859712,574.61139178)
\curveto(856.44859331,574.621385)(856.41359334,574.621385)(856.38359712,574.61139178)
\lineto(856.14359712,574.55139178)
\curveto(856.07359368,574.54138508)(856.00359375,574.5263851)(855.93359712,574.50639178)
\curveto(855.64359411,574.38638524)(855.40859435,574.23638539)(855.22859712,574.05639178)
\curveto(855.0585947,573.87638575)(854.90359485,573.65138597)(854.76359712,573.38139178)
\curveto(854.73359502,573.33138629)(854.70359505,573.26638636)(854.67359712,573.18639178)
\curveto(854.64359511,573.11638651)(854.61859514,573.03638659)(854.59859712,572.94639178)
\curveto(854.57859518,572.85638677)(854.57359518,572.77138685)(854.58359712,572.69139178)
\curveto(854.59359516,572.61138701)(854.62859513,572.55138707)(854.68859712,572.51139178)
\curveto(854.76859499,572.45138717)(854.90359485,572.4213872)(855.09359712,572.42139178)
\curveto(855.29359446,572.43138719)(855.46359429,572.43638719)(855.60359712,572.43639178)
\lineto(857.88359712,572.43639178)
\curveto(858.03359172,572.43638719)(858.21359154,572.43138719)(858.42359712,572.42139178)
\curveto(858.63359112,572.4213872)(858.77359098,572.46138716)(858.84359712,572.54139178)
}
}
{
\newrgbcolor{curcolor}{0 0 0}
\pscustom[linestyle=none,fillstyle=solid,fillcolor=curcolor]
{
\newpath
\moveto(868.73023775,572.06139178)
\curveto(868.75022969,572.00138762)(868.76022968,571.90638772)(868.76023775,571.77639178)
\curveto(868.76022968,571.65638797)(868.75522968,571.57138805)(868.74523775,571.52139178)
\lineto(868.74523775,571.37139178)
\curveto(868.7352297,571.29138833)(868.72522971,571.21638841)(868.71523775,571.14639178)
\curveto(868.71522972,571.08638854)(868.71022973,571.01638861)(868.70023775,570.93639178)
\curveto(868.68022976,570.87638875)(868.66522977,570.81638881)(868.65523775,570.75639178)
\curveto(868.65522978,570.69638893)(868.64522979,570.63638899)(868.62523775,570.57639178)
\curveto(868.58522985,570.44638918)(868.55022989,570.31638931)(868.52023775,570.18639178)
\curveto(868.49022995,570.05638957)(868.45022999,569.93638969)(868.40023775,569.82639178)
\curveto(868.19023025,569.34639028)(867.91023053,568.94139068)(867.56023775,568.61139178)
\curveto(867.21023123,568.29139133)(866.78023166,568.04639158)(866.27023775,567.87639178)
\curveto(866.16023228,567.83639179)(866.0402324,567.80639182)(865.91023775,567.78639178)
\curveto(865.79023265,567.76639186)(865.66523277,567.74639188)(865.53523775,567.72639178)
\curveto(865.47523296,567.71639191)(865.41023303,567.71139191)(865.34023775,567.71139178)
\curveto(865.28023316,567.70139192)(865.22023322,567.69639193)(865.16023775,567.69639178)
\curveto(865.12023332,567.68639194)(865.06023338,567.68139194)(864.98023775,567.68139178)
\curveto(864.91023353,567.68139194)(864.86023358,567.68639194)(864.83023775,567.69639178)
\curveto(864.79023365,567.70639192)(864.75023369,567.71139191)(864.71023775,567.71139178)
\curveto(864.67023377,567.70139192)(864.6352338,567.70139192)(864.60523775,567.71139178)
\lineto(864.51523775,567.71139178)
\lineto(864.15523775,567.75639178)
\curveto(864.01523442,567.79639183)(863.88023456,567.83639179)(863.75023775,567.87639178)
\curveto(863.62023482,567.91639171)(863.49523494,567.96139166)(863.37523775,568.01139178)
\curveto(862.92523551,568.21139141)(862.55523588,568.47139115)(862.26523775,568.79139178)
\curveto(861.97523646,569.11139051)(861.7352367,569.50139012)(861.54523775,569.96139178)
\curveto(861.49523694,570.06138956)(861.45523698,570.16138946)(861.42523775,570.26139178)
\curveto(861.40523703,570.36138926)(861.38523705,570.46638916)(861.36523775,570.57639178)
\curveto(861.34523709,570.61638901)(861.3352371,570.64638898)(861.33523775,570.66639178)
\curveto(861.34523709,570.69638893)(861.34523709,570.73138889)(861.33523775,570.77139178)
\curveto(861.31523712,570.85138877)(861.30023714,570.93138869)(861.29023775,571.01139178)
\curveto(861.29023715,571.10138852)(861.28023716,571.18638844)(861.26023775,571.26639178)
\lineto(861.26023775,571.38639178)
\curveto(861.26023718,571.4263882)(861.25523718,571.47138815)(861.24523775,571.52139178)
\curveto(861.2352372,571.57138805)(861.23023721,571.65638797)(861.23023775,571.77639178)
\curveto(861.23023721,571.90638772)(861.2402372,572.00138762)(861.26023775,572.06139178)
\curveto(861.28023716,572.13138749)(861.28523715,572.20138742)(861.27523775,572.27139178)
\curveto(861.26523717,572.34138728)(861.27023717,572.41138721)(861.29023775,572.48139178)
\curveto(861.30023714,572.53138709)(861.30523713,572.57138705)(861.30523775,572.60139178)
\curveto(861.31523712,572.64138698)(861.32523711,572.68638694)(861.33523775,572.73639178)
\curveto(861.36523707,572.85638677)(861.39023705,572.97638665)(861.41023775,573.09639178)
\curveto(861.440237,573.21638641)(861.48023696,573.33138629)(861.53023775,573.44139178)
\curveto(861.68023676,573.81138581)(861.86023658,574.14138548)(862.07023775,574.43139178)
\curveto(862.29023615,574.73138489)(862.55523588,574.98138464)(862.86523775,575.18139178)
\curveto(862.98523545,575.26138436)(863.11023533,575.3263843)(863.24023775,575.37639178)
\curveto(863.37023507,575.43638419)(863.50523493,575.49638413)(863.64523775,575.55639178)
\curveto(863.76523467,575.60638402)(863.89523454,575.63638399)(864.03523775,575.64639178)
\curveto(864.17523426,575.66638396)(864.31523412,575.69638393)(864.45523775,575.73639178)
\lineto(864.65023775,575.73639178)
\curveto(864.72023372,575.74638388)(864.78523365,575.75638387)(864.84523775,575.76639178)
\curveto(865.7352327,575.77638385)(866.47523196,575.59138403)(867.06523775,575.21139178)
\curveto(867.65523078,574.83138479)(868.08023036,574.33638529)(868.34023775,573.72639178)
\curveto(868.39023005,573.626386)(868.43023001,573.5263861)(868.46023775,573.42639178)
\curveto(868.49022995,573.3263863)(868.52522991,573.2213864)(868.56523775,573.11139178)
\curveto(868.59522984,573.00138662)(868.62022982,572.88138674)(868.64023775,572.75139178)
\curveto(868.66022978,572.63138699)(868.68522975,572.50638712)(868.71523775,572.37639178)
\curveto(868.72522971,572.3263873)(868.72522971,572.27138735)(868.71523775,572.21139178)
\curveto(868.71522972,572.16138746)(868.72022972,572.11138751)(868.73023775,572.06139178)
\moveto(867.39523775,571.20639178)
\curveto(867.41523102,571.27638835)(867.42023102,571.35638827)(867.41023775,571.44639178)
\lineto(867.41023775,571.70139178)
\curveto(867.41023103,572.09138753)(867.37523106,572.4213872)(867.30523775,572.69139178)
\curveto(867.27523116,572.77138685)(867.25023119,572.85138677)(867.23023775,572.93139178)
\curveto(867.21023123,573.01138661)(867.18523125,573.08638654)(867.15523775,573.15639178)
\curveto(866.87523156,573.80638582)(866.43023201,574.25638537)(865.82023775,574.50639178)
\curveto(865.75023269,574.53638509)(865.67523276,574.55638507)(865.59523775,574.56639178)
\lineto(865.35523775,574.62639178)
\curveto(865.27523316,574.64638498)(865.19023325,574.65638497)(865.10023775,574.65639178)
\lineto(864.83023775,574.65639178)
\lineto(864.56023775,574.61139178)
\curveto(864.46023398,574.59138503)(864.36523407,574.56638506)(864.27523775,574.53639178)
\curveto(864.19523424,574.51638511)(864.11523432,574.48638514)(864.03523775,574.44639178)
\curveto(863.96523447,574.4263852)(863.90023454,574.39638523)(863.84023775,574.35639178)
\curveto(863.78023466,574.31638531)(863.72523471,574.27638535)(863.67523775,574.23639178)
\curveto(863.435235,574.06638556)(863.2402352,573.86138576)(863.09023775,573.62139178)
\curveto(862.9402355,573.38138624)(862.81023563,573.10138652)(862.70023775,572.78139178)
\curveto(862.67023577,572.68138694)(862.65023579,572.57638705)(862.64023775,572.46639178)
\curveto(862.63023581,572.36638726)(862.61523582,572.26138736)(862.59523775,572.15139178)
\curveto(862.58523585,572.11138751)(862.58023586,572.04638758)(862.58023775,571.95639178)
\curveto(862.57023587,571.9263877)(862.56523587,571.89138773)(862.56523775,571.85139178)
\curveto(862.57523586,571.81138781)(862.58023586,571.76638786)(862.58023775,571.71639178)
\lineto(862.58023775,571.41639178)
\curveto(862.58023586,571.31638831)(862.59023585,571.2263884)(862.61023775,571.14639178)
\lineto(862.64023775,570.96639178)
\curveto(862.66023578,570.86638876)(862.67523576,570.76638886)(862.68523775,570.66639178)
\curveto(862.70523573,570.57638905)(862.7352357,570.49138913)(862.77523775,570.41139178)
\curveto(862.87523556,570.17138945)(862.99023545,569.94638968)(863.12023775,569.73639178)
\curveto(863.26023518,569.5263901)(863.43023501,569.35139027)(863.63023775,569.21139178)
\curveto(863.68023476,569.18139044)(863.72523471,569.15639047)(863.76523775,569.13639178)
\curveto(863.80523463,569.11639051)(863.85023459,569.09139053)(863.90023775,569.06139178)
\curveto(863.98023446,569.01139061)(864.06523437,568.96639066)(864.15523775,568.92639178)
\curveto(864.25523418,568.89639073)(864.36023408,568.86639076)(864.47023775,568.83639178)
\curveto(864.52023392,568.81639081)(864.56523387,568.80639082)(864.60523775,568.80639178)
\curveto(864.65523378,568.81639081)(864.70523373,568.81639081)(864.75523775,568.80639178)
\curveto(864.78523365,568.79639083)(864.84523359,568.78639084)(864.93523775,568.77639178)
\curveto(865.0352334,568.76639086)(865.11023333,568.77139085)(865.16023775,568.79139178)
\curveto(865.20023324,568.80139082)(865.2402332,568.80139082)(865.28023775,568.79139178)
\curveto(865.32023312,568.79139083)(865.36023308,568.80139082)(865.40023775,568.82139178)
\curveto(865.48023296,568.84139078)(865.56023288,568.85639077)(865.64023775,568.86639178)
\curveto(865.72023272,568.88639074)(865.79523264,568.91139071)(865.86523775,568.94139178)
\curveto(866.20523223,569.08139054)(866.48023196,569.27639035)(866.69023775,569.52639178)
\curveto(866.90023154,569.77638985)(867.07523136,570.07138955)(867.21523775,570.41139178)
\curveto(867.26523117,570.53138909)(867.29523114,570.65638897)(867.30523775,570.78639178)
\curveto(867.32523111,570.9263887)(867.35523108,571.06638856)(867.39523775,571.20639178)
}
}
{
\newrgbcolor{curcolor}{0 0 0}
\pscustom[linestyle=none,fillstyle=solid,fillcolor=curcolor]
{
\newpath
\moveto(872.648519,575.76639178)
\curveto(873.36851493,575.77638385)(873.97351433,575.69138393)(874.463519,575.51139178)
\curveto(874.95351335,575.34138428)(875.33351297,575.03638459)(875.603519,574.59639178)
\curveto(875.67351263,574.48638514)(875.72851257,574.37138525)(875.768519,574.25139178)
\curveto(875.80851249,574.14138548)(875.84851245,574.01638561)(875.888519,573.87639178)
\curveto(875.90851239,573.80638582)(875.91351239,573.73138589)(875.903519,573.65139178)
\curveto(875.89351241,573.58138604)(875.87851242,573.5263861)(875.858519,573.48639178)
\curveto(875.83851246,573.46638616)(875.81351249,573.44638618)(875.783519,573.42639178)
\curveto(875.75351255,573.41638621)(875.72851257,573.40138622)(875.708519,573.38139178)
\curveto(875.65851264,573.36138626)(875.60851269,573.35638627)(875.558519,573.36639178)
\curveto(875.50851279,573.37638625)(875.45851284,573.37638625)(875.408519,573.36639178)
\curveto(875.32851297,573.34638628)(875.22351308,573.34138628)(875.093519,573.35139178)
\curveto(874.96351334,573.37138625)(874.87351343,573.39638623)(874.823519,573.42639178)
\curveto(874.74351356,573.47638615)(874.68851361,573.54138608)(874.658519,573.62139178)
\curveto(874.63851366,573.71138591)(874.6035137,573.79638583)(874.553519,573.87639178)
\curveto(874.46351384,574.03638559)(874.33851396,574.18138544)(874.178519,574.31139178)
\curveto(874.06851423,574.39138523)(873.94851435,574.45138517)(873.818519,574.49139178)
\curveto(873.68851461,574.53138509)(873.54851475,574.57138505)(873.398519,574.61139178)
\curveto(873.34851495,574.63138499)(873.298515,574.63638499)(873.248519,574.62639178)
\curveto(873.1985151,574.626385)(873.14851515,574.63138499)(873.098519,574.64139178)
\curveto(873.03851526,574.66138496)(872.96351534,574.67138495)(872.873519,574.67139178)
\curveto(872.78351552,574.67138495)(872.70851559,574.66138496)(872.648519,574.64139178)
\lineto(872.558519,574.64139178)
\lineto(872.408519,574.61139178)
\curveto(872.35851594,574.61138501)(872.30851599,574.60638502)(872.258519,574.59639178)
\curveto(871.9985163,574.53638509)(871.78351652,574.45138517)(871.613519,574.34139178)
\curveto(871.44351686,574.23138539)(871.32851697,574.04638558)(871.268519,573.78639178)
\curveto(871.24851705,573.71638591)(871.24351706,573.64638598)(871.253519,573.57639178)
\curveto(871.27351703,573.50638612)(871.29351701,573.44638618)(871.313519,573.39639178)
\curveto(871.37351693,573.24638638)(871.44351686,573.13638649)(871.523519,573.06639178)
\curveto(871.61351669,573.00638662)(871.72351658,572.93638669)(871.853519,572.85639178)
\curveto(872.01351629,572.75638687)(872.19351611,572.68138694)(872.393519,572.63139178)
\curveto(872.59351571,572.59138703)(872.79351551,572.54138708)(872.993519,572.48139178)
\curveto(873.12351518,572.44138718)(873.25351505,572.41138721)(873.383519,572.39139178)
\curveto(873.51351479,572.37138725)(873.64351466,572.34138728)(873.773519,572.30139178)
\curveto(873.98351432,572.24138738)(874.18851411,572.18138744)(874.388519,572.12139178)
\curveto(874.58851371,572.07138755)(874.78851351,572.00638762)(874.988519,571.92639178)
\lineto(875.138519,571.86639178)
\curveto(875.18851311,571.84638778)(875.23851306,571.8213878)(875.288519,571.79139178)
\curveto(875.48851281,571.67138795)(875.66351264,571.53638809)(875.813519,571.38639178)
\curveto(875.96351234,571.23638839)(876.08851221,571.04638858)(876.188519,570.81639178)
\curveto(876.20851209,570.74638888)(876.22851207,570.65138897)(876.248519,570.53139178)
\curveto(876.26851203,570.46138916)(876.27851202,570.38638924)(876.278519,570.30639178)
\curveto(876.28851201,570.23638939)(876.29351201,570.15638947)(876.293519,570.06639178)
\lineto(876.293519,569.91639178)
\curveto(876.27351203,569.84638978)(876.26351204,569.77638985)(876.263519,569.70639178)
\curveto(876.26351204,569.63638999)(876.25351205,569.56639006)(876.233519,569.49639178)
\curveto(876.2035121,569.38639024)(876.16851213,569.28139034)(876.128519,569.18139178)
\curveto(876.08851221,569.08139054)(876.04351226,568.99139063)(875.993519,568.91139178)
\curveto(875.83351247,568.65139097)(875.62851267,568.44139118)(875.378519,568.28139178)
\curveto(875.12851317,568.13139149)(874.84851345,568.00139162)(874.538519,567.89139178)
\curveto(874.44851385,567.86139176)(874.35351395,567.84139178)(874.253519,567.83139178)
\curveto(874.16351414,567.81139181)(874.07351423,567.78639184)(873.983519,567.75639178)
\curveto(873.88351442,567.73639189)(873.78351452,567.7263919)(873.683519,567.72639178)
\curveto(873.58351472,567.7263919)(873.48351482,567.71639191)(873.383519,567.69639178)
\lineto(873.233519,567.69639178)
\curveto(873.18351512,567.68639194)(873.11351519,567.68139194)(873.023519,567.68139178)
\curveto(872.93351537,567.68139194)(872.86351544,567.68639194)(872.813519,567.69639178)
\lineto(872.648519,567.69639178)
\curveto(872.58851571,567.71639191)(872.52351578,567.7263919)(872.453519,567.72639178)
\curveto(872.38351592,567.71639191)(872.32351598,567.7213919)(872.273519,567.74139178)
\curveto(872.22351608,567.75139187)(872.15851614,567.75639187)(872.078519,567.75639178)
\lineto(871.838519,567.81639178)
\curveto(871.76851653,567.8263918)(871.69351661,567.84639178)(871.613519,567.87639178)
\curveto(871.303517,567.97639165)(871.03351727,568.10139152)(870.803519,568.25139178)
\curveto(870.57351773,568.40139122)(870.37351793,568.59639103)(870.203519,568.83639178)
\curveto(870.11351819,568.96639066)(870.03851826,569.10139052)(869.978519,569.24139178)
\curveto(869.91851838,569.38139024)(869.86351844,569.53639009)(869.813519,569.70639178)
\curveto(869.79351851,569.76638986)(869.78351852,569.83638979)(869.783519,569.91639178)
\curveto(869.79351851,570.00638962)(869.80851849,570.07638955)(869.828519,570.12639178)
\curveto(869.85851844,570.16638946)(869.90851839,570.20638942)(869.978519,570.24639178)
\curveto(870.02851827,570.26638936)(870.0985182,570.27638935)(870.188519,570.27639178)
\curveto(870.27851802,570.28638934)(870.36851793,570.28638934)(870.458519,570.27639178)
\curveto(870.54851775,570.26638936)(870.63351767,570.25138937)(870.713519,570.23139178)
\curveto(870.8035175,570.2213894)(870.86351744,570.20638942)(870.893519,570.18639178)
\curveto(870.96351734,570.13638949)(871.00851729,570.06138956)(871.028519,569.96139178)
\curveto(871.05851724,569.87138975)(871.09351721,569.78638984)(871.133519,569.70639178)
\curveto(871.23351707,569.48639014)(871.36851693,569.31639031)(871.538519,569.19639178)
\curveto(871.65851664,569.10639052)(871.79351651,569.03639059)(871.943519,568.98639178)
\curveto(872.09351621,568.93639069)(872.25351605,568.88639074)(872.423519,568.83639178)
\lineto(872.738519,568.79139178)
\lineto(872.828519,568.79139178)
\curveto(872.8985154,568.77139085)(872.98851531,568.76139086)(873.098519,568.76139178)
\curveto(873.21851508,568.76139086)(873.31851498,568.77139085)(873.398519,568.79139178)
\curveto(873.46851483,568.79139083)(873.52351478,568.79639083)(873.563519,568.80639178)
\curveto(873.62351468,568.81639081)(873.68351462,568.8213908)(873.743519,568.82139178)
\curveto(873.8035145,568.83139079)(873.85851444,568.84139078)(873.908519,568.85139178)
\curveto(874.1985141,568.93139069)(874.42851387,569.03639059)(874.598519,569.16639178)
\curveto(874.76851353,569.29639033)(874.88851341,569.51639011)(874.958519,569.82639178)
\curveto(874.97851332,569.87638975)(874.98351332,569.93138969)(874.973519,569.99139178)
\curveto(874.96351334,570.05138957)(874.95351335,570.09638953)(874.943519,570.12639178)
\curveto(874.89351341,570.31638931)(874.82351348,570.45638917)(874.733519,570.54639178)
\curveto(874.64351366,570.64638898)(874.52851377,570.73638889)(874.388519,570.81639178)
\curveto(874.298514,570.87638875)(874.1985141,570.9263887)(874.088519,570.96639178)
\lineto(873.758519,571.08639178)
\curveto(873.72851457,571.09638853)(873.6985146,571.10138852)(873.668519,571.10139178)
\curveto(873.64851465,571.10138852)(873.62351468,571.11138851)(873.593519,571.13139178)
\curveto(873.25351505,571.24138838)(872.8985154,571.3213883)(872.528519,571.37139178)
\curveto(872.16851613,571.43138819)(871.82851647,571.5263881)(871.508519,571.65639178)
\curveto(871.40851689,571.69638793)(871.31351699,571.73138789)(871.223519,571.76139178)
\curveto(871.13351717,571.79138783)(871.04851725,571.83138779)(870.968519,571.88139178)
\curveto(870.77851752,571.99138763)(870.6035177,572.11638751)(870.443519,572.25639178)
\curveto(870.28351802,572.39638723)(870.15851814,572.57138705)(870.068519,572.78139178)
\curveto(870.03851826,572.85138677)(870.01351829,572.9213867)(869.993519,572.99139178)
\curveto(869.98351832,573.06138656)(869.96851833,573.13638649)(869.948519,573.21639178)
\curveto(869.91851838,573.33638629)(869.90851839,573.47138615)(869.918519,573.62139178)
\curveto(869.92851837,573.78138584)(869.94351836,573.91638571)(869.963519,574.02639178)
\curveto(869.98351832,574.07638555)(869.99351831,574.11638551)(869.993519,574.14639178)
\curveto(870.0035183,574.18638544)(870.01851828,574.2263854)(870.038519,574.26639178)
\curveto(870.12851817,574.49638513)(870.24851805,574.69638493)(870.398519,574.86639178)
\curveto(870.55851774,575.03638459)(870.73851756,575.18638444)(870.938519,575.31639178)
\curveto(871.08851721,575.40638422)(871.25351705,575.47638415)(871.433519,575.52639178)
\curveto(871.61351669,575.58638404)(871.8035165,575.64138398)(872.003519,575.69139178)
\curveto(872.07351623,575.70138392)(872.13851616,575.71138391)(872.198519,575.72139178)
\curveto(872.26851603,575.73138389)(872.34351596,575.74138388)(872.423519,575.75139178)
\curveto(872.45351585,575.76138386)(872.49351581,575.76138386)(872.543519,575.75139178)
\curveto(872.59351571,575.74138388)(872.62851567,575.74638388)(872.648519,575.76639178)
}
}
{
\newrgbcolor{curcolor}{0.40000001 0.40000001 0.40000001}
\pscustom[linestyle=none,fillstyle=solid,fillcolor=curcolor]
{
\newpath
\moveto(812.80437349,578.5714284)
\lineto(827.80437349,578.5714284)
\lineto(827.80437349,563.5714284)
\lineto(812.80437349,563.5714284)
\closepath
}
}
{
\newrgbcolor{curcolor}{0 0 0}
\pscustom[linestyle=none,fillstyle=solid,fillcolor=curcolor]
{
\newpath
\moveto(415.20953951,703.11942034)
\curveto(415.20952902,703.08941467)(415.20952902,703.04941471)(415.20953951,702.99942034)
\curveto(415.21952901,702.94941481)(415.22452901,702.89441487)(415.22453951,702.83442034)
\curveto(415.22452901,702.77441499)(415.21952901,702.71941504)(415.20953951,702.66942034)
\curveto(415.20952902,702.61941514)(415.20952902,702.58441518)(415.20953951,702.56442034)
\curveto(415.20952902,702.49441527)(415.20452903,702.42441534)(415.19453951,702.35442034)
\curveto(415.19452904,702.29441547)(415.19452904,702.23441553)(415.19453951,702.17442034)
\curveto(415.17452906,702.12441564)(415.16452907,702.07441569)(415.16453951,702.02442034)
\curveto(415.17452906,701.97441579)(415.17452906,701.92441584)(415.16453951,701.87442034)
\curveto(415.14452909,701.764416)(415.1295291,701.65441611)(415.11953951,701.54442034)
\curveto(415.10952912,701.43441633)(415.08952914,701.32441644)(415.05953951,701.21442034)
\curveto(415.00952922,701.04441672)(414.96452927,700.87941688)(414.92453951,700.71942034)
\curveto(414.88452935,700.56941719)(414.8345294,700.41941734)(414.77453951,700.26942034)
\curveto(414.60452963,699.84941791)(414.39452984,699.46941829)(414.14453951,699.12942034)
\curveto(413.89453034,698.78941897)(413.59453064,698.49941926)(413.24453951,698.25942034)
\curveto(413.04453119,698.11941964)(412.8345314,697.99941976)(412.61453951,697.89942034)
\curveto(412.40453183,697.79941996)(412.17453206,697.70942005)(411.92453951,697.62942034)
\curveto(411.82453241,697.59942016)(411.71953251,697.57442019)(411.60953951,697.55442034)
\curveto(411.50953272,697.54442022)(411.40453283,697.52442024)(411.29453951,697.49442034)
\curveto(411.24453299,697.48442028)(411.19453304,697.47942028)(411.14453951,697.47942034)
\curveto(411.10453313,697.47942028)(411.05953317,697.47442029)(411.00953951,697.46442034)
\curveto(410.96953326,697.45442031)(410.9295333,697.44942031)(410.88953951,697.44942034)
\curveto(410.84953338,697.4594203)(410.80453343,697.4594203)(410.75453951,697.44942034)
\curveto(410.7345335,697.43942032)(410.70453353,697.43442033)(410.66453951,697.43442034)
\curveto(410.62453361,697.44442032)(410.59453364,697.44442032)(410.57453951,697.43442034)
\curveto(410.49453374,697.41442035)(410.39453384,697.40942035)(410.27453951,697.41942034)
\curveto(410.15453408,697.42942033)(410.04953418,697.43442033)(409.95953951,697.43442034)
\lineto(406.46453951,697.43442034)
\curveto(406.29453794,697.43442033)(406.14953808,697.43942032)(406.02953951,697.44942034)
\curveto(405.91953831,697.46942029)(405.83953839,697.53942022)(405.78953951,697.65942034)
\curveto(405.75953847,697.73942002)(405.74453849,697.8594199)(405.74453951,698.01942034)
\curveto(405.75453848,698.18941957)(405.75953847,698.32941943)(405.75953951,698.43942034)
\lineto(405.75953951,707.24442034)
\curveto(405.75953847,707.3644104)(405.75453848,707.48941027)(405.74453951,707.61942034)
\curveto(405.74453849,707.75941)(405.76953846,707.86940989)(405.81953951,707.94942034)
\curveto(405.85953837,708.00940975)(405.9345383,708.0594097)(406.04453951,708.09942034)
\curveto(406.06453817,708.10940965)(406.08453815,708.10940965)(406.10453951,708.09942034)
\curveto(406.12453811,708.09940966)(406.14453809,708.10440966)(406.16453951,708.11442034)
\lineto(410.19953951,708.11442034)
\curveto(410.25953397,708.11440965)(410.31953391,708.11440965)(410.37953951,708.11442034)
\curveto(410.44953378,708.12440964)(410.50953372,708.12440964)(410.55953951,708.11442034)
\lineto(410.73953951,708.11442034)
\curveto(410.78953344,708.09440967)(410.84453339,708.08440968)(410.90453951,708.08442034)
\curveto(410.96453327,708.09440967)(411.01953321,708.08940967)(411.06953951,708.06942034)
\curveto(411.1295331,708.04940971)(411.18453305,708.03940972)(411.23453951,708.03942034)
\curveto(411.29453294,708.04940971)(411.35453288,708.04440972)(411.41453951,708.02442034)
\curveto(411.55453268,707.99440977)(411.68953254,707.9644098)(411.81953951,707.93442034)
\curveto(411.94953228,707.91440985)(412.07453216,707.87940988)(412.19453951,707.82942034)
\curveto(412.30453193,707.77940998)(412.41453182,707.73441003)(412.52453951,707.69442034)
\curveto(412.6345316,707.65441011)(412.73953149,707.60441016)(412.83953951,707.54442034)
\curveto(413.08953114,707.38441038)(413.31953091,707.22941053)(413.52953951,707.07942034)
\lineto(413.61953951,706.98942034)
\curveto(413.71953051,706.90941085)(413.80953042,706.81941094)(413.88953951,706.71942034)
\lineto(414.02453951,706.59942034)
\curveto(414.07453016,706.51941124)(414.1295301,706.43941132)(414.18953951,706.35942034)
\curveto(414.25952997,706.28941147)(414.31952991,706.21441155)(414.36953951,706.13442034)
\curveto(414.49952973,705.92441184)(414.61452962,705.69941206)(414.71453951,705.45942034)
\curveto(414.81452942,705.22941253)(414.90452933,704.98441278)(414.98453951,704.72442034)
\curveto(415.0345292,704.59441317)(415.06452917,704.4594133)(415.07453951,704.31942034)
\curveto(415.09452914,704.17941358)(415.11952911,704.03941372)(415.14953951,703.89942034)
\curveto(415.14952908,703.84941391)(415.14952908,703.80441396)(415.14953951,703.76442034)
\curveto(415.15952907,703.73441403)(415.16452907,703.69941406)(415.16453951,703.65942034)
\curveto(415.18452905,703.59941416)(415.18952904,703.53441423)(415.17953951,703.46442034)
\curveto(415.17952905,703.39441437)(415.18952904,703.33441443)(415.20953951,703.28442034)
\lineto(415.20953951,703.11942034)
\moveto(412.86953951,702.39942034)
\curveto(412.88953134,702.44941531)(412.89953133,702.52941523)(412.89953951,702.63942034)
\curveto(412.89953133,702.74941501)(412.88953134,702.82941493)(412.86953951,702.87942034)
\lineto(412.86953951,703.16442034)
\curveto(412.84953138,703.25441451)(412.8345314,703.34941441)(412.82453951,703.44942034)
\curveto(412.82453141,703.54941421)(412.81453142,703.63941412)(412.79453951,703.71942034)
\curveto(412.77453146,703.76941399)(412.76453147,703.81441395)(412.76453951,703.85442034)
\curveto(412.77453146,703.90441386)(412.76953146,703.95441381)(412.74953951,704.00442034)
\curveto(412.69953153,704.1644136)(412.64953158,704.31441345)(412.59953951,704.45442034)
\curveto(412.55953167,704.60441316)(412.49953173,704.74441302)(412.41953951,704.87442034)
\curveto(412.26953196,705.11441265)(412.09453214,705.31941244)(411.89453951,705.48942034)
\curveto(411.70453253,705.66941209)(411.46953276,705.81941194)(411.18953951,705.93942034)
\curveto(411.09953313,705.96941179)(411.00953322,705.99441177)(410.91953951,706.01442034)
\curveto(410.8295334,706.04441172)(410.73953349,706.06941169)(410.64953951,706.08942034)
\curveto(410.56953366,706.09941166)(410.49453374,706.10441166)(410.42453951,706.10442034)
\curveto(410.36453387,706.11441165)(410.29453394,706.12941163)(410.21453951,706.14942034)
\curveto(410.17453406,706.1594116)(410.1345341,706.1594116)(410.09453951,706.14942034)
\curveto(410.05453418,706.14941161)(410.01953421,706.15441161)(409.98953951,706.16442034)
\lineto(409.65953951,706.16442034)
\curveto(409.60953462,706.17441159)(409.55453468,706.17441159)(409.49453951,706.16442034)
\lineto(409.31453951,706.16442034)
\lineto(408.63953951,706.16442034)
\curveto(408.61953561,706.14441162)(408.58453565,706.13941162)(408.53453951,706.14942034)
\curveto(408.49453574,706.1594116)(408.45953577,706.1594116)(408.42953951,706.14942034)
\lineto(408.27953951,706.08942034)
\curveto(408.229536,706.07941168)(408.18953604,706.04941171)(408.15953951,705.99942034)
\curveto(408.11953611,705.94941181)(408.09953613,705.87941188)(408.09953951,705.78942034)
\lineto(408.09953951,705.48942034)
\curveto(408.09953613,705.3594124)(408.09453614,705.22441254)(408.08453951,705.08442034)
\lineto(408.08453951,704.66442034)
\lineto(408.08453951,700.47942034)
\curveto(408.08453615,700.41941734)(408.07953615,700.35441741)(408.06953951,700.28442034)
\curveto(408.06953616,700.21441755)(408.07953615,700.15441761)(408.09953951,700.10442034)
\lineto(408.09953951,699.95442034)
\lineto(408.09953951,699.74442034)
\curveto(408.10953612,699.68441808)(408.12453611,699.62941813)(408.14453951,699.57942034)
\curveto(408.20453603,699.4594183)(408.31953591,699.39441837)(408.48953951,699.38442034)
\lineto(409.01453951,699.38442034)
\lineto(410.19953951,699.38442034)
\curveto(410.59953363,699.39441837)(410.93953329,699.45441831)(411.21953951,699.56442034)
\curveto(411.58953264,699.71441805)(411.87953235,699.91441785)(412.08953951,700.16442034)
\curveto(412.30953192,700.41441735)(412.49453174,700.72441704)(412.64453951,701.09442034)
\curveto(412.68453155,701.17441659)(412.71453152,701.2644165)(412.73453951,701.36442034)
\curveto(412.75453148,701.4644163)(412.77953145,701.5644162)(412.80953951,701.66442034)
\lineto(412.80953951,701.78442034)
\curveto(412.8295314,701.85441591)(412.83953139,701.92941583)(412.83953951,702.00942034)
\curveto(412.83953139,702.08941567)(412.84953138,702.16941559)(412.86953951,702.24942034)
\lineto(412.86953951,702.39942034)
}
}
{
\newrgbcolor{curcolor}{0 0 0}
\pscustom[linestyle=none,fillstyle=solid,fillcolor=curcolor]
{
\newpath
\moveto(418.70805513,708.00942034)
\curveto(418.77805218,707.92940983)(418.81305215,707.80940995)(418.81305513,707.64942034)
\lineto(418.81305513,707.18442034)
\lineto(418.81305513,706.77942034)
\curveto(418.81305215,706.63941112)(418.77805218,706.54441122)(418.70805513,706.49442034)
\curveto(418.64805231,706.44441132)(418.56805239,706.41441135)(418.46805513,706.40442034)
\curveto(418.37805258,706.39441137)(418.27805268,706.38941137)(418.16805513,706.38942034)
\lineto(417.32805513,706.38942034)
\curveto(417.21805374,706.38941137)(417.11805384,706.39441137)(417.02805513,706.40442034)
\curveto(416.94805401,706.41441135)(416.87805408,706.44441132)(416.81805513,706.49442034)
\curveto(416.77805418,706.52441124)(416.74805421,706.57941118)(416.72805513,706.65942034)
\curveto(416.71805424,706.74941101)(416.70805425,706.84441092)(416.69805513,706.94442034)
\lineto(416.69805513,707.27442034)
\curveto(416.70805425,707.38441038)(416.71305425,707.47941028)(416.71305513,707.55942034)
\lineto(416.71305513,707.76942034)
\curveto(416.72305424,707.83940992)(416.74305422,707.89940986)(416.77305513,707.94942034)
\curveto(416.79305417,707.98940977)(416.81805414,708.01940974)(416.84805513,708.03942034)
\lineto(416.96805513,708.09942034)
\curveto(416.98805397,708.09940966)(417.01305395,708.09940966)(417.04305513,708.09942034)
\curveto(417.07305389,708.10940965)(417.09805386,708.11440965)(417.11805513,708.11442034)
\lineto(418.21305513,708.11442034)
\curveto(418.31305265,708.11440965)(418.40805255,708.10940965)(418.49805513,708.09942034)
\curveto(418.58805237,708.08940967)(418.6580523,708.0594097)(418.70805513,708.00942034)
\moveto(418.81305513,698.24442034)
\curveto(418.81305215,698.04441972)(418.80805215,697.87441989)(418.79805513,697.73442034)
\curveto(418.78805217,697.59442017)(418.69805226,697.49942026)(418.52805513,697.44942034)
\curveto(418.46805249,697.42942033)(418.40305256,697.41942034)(418.33305513,697.41942034)
\curveto(418.2630527,697.42942033)(418.18805277,697.43442033)(418.10805513,697.43442034)
\lineto(417.26805513,697.43442034)
\curveto(417.17805378,697.43442033)(417.08805387,697.43942032)(416.99805513,697.44942034)
\curveto(416.91805404,697.4594203)(416.8580541,697.48942027)(416.81805513,697.53942034)
\curveto(416.7580542,697.60942015)(416.72305424,697.69442007)(416.71305513,697.79442034)
\lineto(416.71305513,698.13942034)
\lineto(416.71305513,704.46942034)
\lineto(416.71305513,704.76942034)
\curveto(416.71305425,704.86941289)(416.73305423,704.94941281)(416.77305513,705.00942034)
\curveto(416.83305413,705.07941268)(416.91805404,705.12441264)(417.02805513,705.14442034)
\curveto(417.04805391,705.15441261)(417.07305389,705.15441261)(417.10305513,705.14442034)
\curveto(417.14305382,705.14441262)(417.17305379,705.14941261)(417.19305513,705.15942034)
\lineto(417.94305513,705.15942034)
\lineto(418.13805513,705.15942034)
\curveto(418.21805274,705.16941259)(418.28305268,705.16941259)(418.33305513,705.15942034)
\lineto(418.45305513,705.15942034)
\curveto(418.51305245,705.13941262)(418.56805239,705.12441264)(418.61805513,705.11442034)
\curveto(418.66805229,705.10441266)(418.70805225,705.07441269)(418.73805513,705.02442034)
\curveto(418.77805218,704.97441279)(418.79805216,704.90441286)(418.79805513,704.81442034)
\curveto(418.80805215,704.72441304)(418.81305215,704.62941313)(418.81305513,704.52942034)
\lineto(418.81305513,698.24442034)
}
}
{
\newrgbcolor{curcolor}{0 0 0}
\pscustom[linestyle=none,fillstyle=solid,fillcolor=curcolor]
{
\newpath
\moveto(423.44524263,705.36942034)
\curveto(424.19523813,705.38941237)(424.84523748,705.30441246)(425.39524263,705.11442034)
\curveto(425.95523637,704.93441283)(426.38023595,704.61941314)(426.67024263,704.16942034)
\curveto(426.74023559,704.0594137)(426.80023553,703.94441382)(426.85024263,703.82442034)
\curveto(426.91023542,703.71441405)(426.96023537,703.58941417)(427.00024263,703.44942034)
\curveto(427.02023531,703.38941437)(427.0302353,703.32441444)(427.03024263,703.25442034)
\curveto(427.0302353,703.18441458)(427.02023531,703.12441464)(427.00024263,703.07442034)
\curveto(426.96023537,703.01441475)(426.90523542,702.97441479)(426.83524263,702.95442034)
\curveto(426.78523554,702.93441483)(426.7252356,702.92441484)(426.65524263,702.92442034)
\lineto(426.44524263,702.92442034)
\lineto(425.78524263,702.92442034)
\curveto(425.71523661,702.92441484)(425.64523668,702.91941484)(425.57524263,702.90942034)
\curveto(425.50523682,702.90941485)(425.44023689,702.91941484)(425.38024263,702.93942034)
\curveto(425.28023705,702.9594148)(425.20523712,702.99941476)(425.15524263,703.05942034)
\curveto(425.10523722,703.11941464)(425.06023727,703.17941458)(425.02024263,703.23942034)
\lineto(424.90024263,703.44942034)
\curveto(424.87023746,703.52941423)(424.82023751,703.59441417)(424.75024263,703.64442034)
\curveto(424.65023768,703.72441404)(424.55023778,703.78441398)(424.45024263,703.82442034)
\curveto(424.36023797,703.8644139)(424.24523808,703.89941386)(424.10524263,703.92942034)
\curveto(424.03523829,703.94941381)(423.9302384,703.9644138)(423.79024263,703.97442034)
\curveto(423.66023867,703.98441378)(423.56023877,703.97941378)(423.49024263,703.95942034)
\lineto(423.38524263,703.95942034)
\lineto(423.23524263,703.92942034)
\curveto(423.19523913,703.92941383)(423.15023918,703.92441384)(423.10024263,703.91442034)
\curveto(422.9302394,703.8644139)(422.79023954,703.79441397)(422.68024263,703.70442034)
\curveto(422.58023975,703.62441414)(422.51023982,703.49941426)(422.47024263,703.32942034)
\curveto(422.45023988,703.2594145)(422.45023988,703.19441457)(422.47024263,703.13442034)
\curveto(422.49023984,703.07441469)(422.51023982,703.02441474)(422.53024263,702.98442034)
\curveto(422.60023973,702.8644149)(422.68023965,702.76941499)(422.77024263,702.69942034)
\curveto(422.87023946,702.62941513)(422.98523934,702.56941519)(423.11524263,702.51942034)
\curveto(423.30523902,702.43941532)(423.51023882,702.36941539)(423.73024263,702.30942034)
\lineto(424.42024263,702.15942034)
\curveto(424.66023767,702.11941564)(424.89023744,702.06941569)(425.11024263,702.00942034)
\curveto(425.34023699,701.9594158)(425.55523677,701.89441587)(425.75524263,701.81442034)
\curveto(425.84523648,701.77441599)(425.9302364,701.73941602)(426.01024263,701.70942034)
\curveto(426.10023623,701.68941607)(426.18523614,701.65441611)(426.26524263,701.60442034)
\curveto(426.45523587,701.48441628)(426.6252357,701.35441641)(426.77524263,701.21442034)
\curveto(426.93523539,701.07441669)(427.06023527,700.89941686)(427.15024263,700.68942034)
\curveto(427.18023515,700.61941714)(427.20523512,700.54941721)(427.22524263,700.47942034)
\curveto(427.24523508,700.40941735)(427.26523506,700.33441743)(427.28524263,700.25442034)
\curveto(427.29523503,700.19441757)(427.30023503,700.09941766)(427.30024263,699.96942034)
\curveto(427.31023502,699.84941791)(427.31023502,699.75441801)(427.30024263,699.68442034)
\lineto(427.30024263,699.60942034)
\curveto(427.28023505,699.54941821)(427.26523506,699.48941827)(427.25524263,699.42942034)
\curveto(427.25523507,699.37941838)(427.25023508,699.32941843)(427.24024263,699.27942034)
\curveto(427.17023516,698.97941878)(427.06023527,698.71441905)(426.91024263,698.48442034)
\curveto(426.75023558,698.24441952)(426.55523577,698.04941971)(426.32524263,697.89942034)
\curveto(426.09523623,697.74942001)(425.83523649,697.61942014)(425.54524263,697.50942034)
\curveto(425.43523689,697.4594203)(425.31523701,697.42442034)(425.18524263,697.40442034)
\curveto(425.06523726,697.38442038)(424.94523738,697.3594204)(424.82524263,697.32942034)
\curveto(424.73523759,697.30942045)(424.64023769,697.29942046)(424.54024263,697.29942034)
\curveto(424.45023788,697.28942047)(424.36023797,697.27442049)(424.27024263,697.25442034)
\lineto(424.00024263,697.25442034)
\curveto(423.94023839,697.23442053)(423.83523849,697.22442054)(423.68524263,697.22442034)
\curveto(423.54523878,697.22442054)(423.44523888,697.23442053)(423.38524263,697.25442034)
\curveto(423.35523897,697.25442051)(423.32023901,697.2594205)(423.28024263,697.26942034)
\lineto(423.17524263,697.26942034)
\curveto(423.05523927,697.28942047)(422.93523939,697.30442046)(422.81524263,697.31442034)
\curveto(422.69523963,697.32442044)(422.58023975,697.34442042)(422.47024263,697.37442034)
\curveto(422.08024025,697.48442028)(421.73524059,697.60942015)(421.43524263,697.74942034)
\curveto(421.13524119,697.89941986)(420.88024145,698.11941964)(420.67024263,698.40942034)
\curveto(420.5302418,698.59941916)(420.41024192,698.81941894)(420.31024263,699.06942034)
\curveto(420.29024204,699.12941863)(420.27024206,699.20941855)(420.25024263,699.30942034)
\curveto(420.2302421,699.3594184)(420.21524211,699.42941833)(420.20524263,699.51942034)
\curveto(420.19524213,699.60941815)(420.20024213,699.68441808)(420.22024263,699.74442034)
\curveto(420.25024208,699.81441795)(420.30024203,699.8644179)(420.37024263,699.89442034)
\curveto(420.42024191,699.91441785)(420.48024185,699.92441784)(420.55024263,699.92442034)
\lineto(420.77524263,699.92442034)
\lineto(421.48024263,699.92442034)
\lineto(421.72024263,699.92442034)
\curveto(421.80024053,699.92441784)(421.87024046,699.91441785)(421.93024263,699.89442034)
\curveto(422.04024029,699.85441791)(422.11024022,699.78941797)(422.14024263,699.69942034)
\curveto(422.18024015,699.60941815)(422.2252401,699.51441825)(422.27524263,699.41442034)
\curveto(422.29524003,699.3644184)(422.33024,699.29941846)(422.38024263,699.21942034)
\curveto(422.44023989,699.13941862)(422.49023984,699.08941867)(422.53024263,699.06942034)
\curveto(422.65023968,698.96941879)(422.76523956,698.88941887)(422.87524263,698.82942034)
\curveto(422.98523934,698.77941898)(423.1252392,698.72941903)(423.29524263,698.67942034)
\curveto(423.34523898,698.6594191)(423.39523893,698.64941911)(423.44524263,698.64942034)
\curveto(423.49523883,698.6594191)(423.54523878,698.6594191)(423.59524263,698.64942034)
\curveto(423.67523865,698.62941913)(423.76023857,698.61941914)(423.85024263,698.61942034)
\curveto(423.95023838,698.62941913)(424.03523829,698.64441912)(424.10524263,698.66442034)
\curveto(424.15523817,698.67441909)(424.20023813,698.67941908)(424.24024263,698.67942034)
\curveto(424.29023804,698.67941908)(424.34023799,698.68941907)(424.39024263,698.70942034)
\curveto(424.5302378,698.759419)(424.65523767,698.81941894)(424.76524263,698.88942034)
\curveto(424.88523744,698.9594188)(424.98023735,699.04941871)(425.05024263,699.15942034)
\curveto(425.10023723,699.23941852)(425.14023719,699.3644184)(425.17024263,699.53442034)
\curveto(425.19023714,699.60441816)(425.19023714,699.66941809)(425.17024263,699.72942034)
\curveto(425.15023718,699.78941797)(425.1302372,699.83941792)(425.11024263,699.87942034)
\curveto(425.04023729,700.01941774)(424.95023738,700.12441764)(424.84024263,700.19442034)
\curveto(424.74023759,700.2644175)(424.62023771,700.32941743)(424.48024263,700.38942034)
\curveto(424.29023804,700.46941729)(424.09023824,700.53441723)(423.88024263,700.58442034)
\curveto(423.67023866,700.63441713)(423.46023887,700.68941707)(423.25024263,700.74942034)
\curveto(423.17023916,700.76941699)(423.08523924,700.78441698)(422.99524263,700.79442034)
\curveto(422.91523941,700.80441696)(422.83523949,700.81941694)(422.75524263,700.83942034)
\curveto(422.43523989,700.92941683)(422.1302402,701.01441675)(421.84024263,701.09442034)
\curveto(421.55024078,701.18441658)(421.28524104,701.31441645)(421.04524263,701.48442034)
\curveto(420.76524156,701.68441608)(420.56024177,701.95441581)(420.43024263,702.29442034)
\curveto(420.41024192,702.3644154)(420.39024194,702.4594153)(420.37024263,702.57942034)
\curveto(420.35024198,702.64941511)(420.33524199,702.73441503)(420.32524263,702.83442034)
\curveto(420.31524201,702.93441483)(420.32024201,703.02441474)(420.34024263,703.10442034)
\curveto(420.36024197,703.15441461)(420.36524196,703.19441457)(420.35524263,703.22442034)
\curveto(420.34524198,703.2644145)(420.35024198,703.30941445)(420.37024263,703.35942034)
\curveto(420.39024194,703.46941429)(420.41024192,703.56941419)(420.43024263,703.65942034)
\curveto(420.46024187,703.759414)(420.49524183,703.85441391)(420.53524263,703.94442034)
\curveto(420.66524166,704.23441353)(420.84524148,704.46941329)(421.07524263,704.64942034)
\curveto(421.30524102,704.82941293)(421.56524076,704.97441279)(421.85524263,705.08442034)
\curveto(421.96524036,705.13441263)(422.08024025,705.16941259)(422.20024263,705.18942034)
\curveto(422.32024001,705.21941254)(422.44523988,705.24941251)(422.57524263,705.27942034)
\curveto(422.63523969,705.29941246)(422.69523963,705.30941245)(422.75524263,705.30942034)
\lineto(422.93524263,705.33942034)
\curveto(423.01523931,705.34941241)(423.10023923,705.35441241)(423.19024263,705.35442034)
\curveto(423.28023905,705.35441241)(423.36523896,705.3594124)(423.44524263,705.36942034)
}
}
{
\newrgbcolor{curcolor}{0 0 0}
\pscustom[linestyle=none,fillstyle=solid,fillcolor=curcolor]
{
\newpath
\moveto(429.58188326,707.46942034)
\lineto(430.58688326,707.46942034)
\curveto(430.73688027,707.46941029)(430.86688014,707.4594103)(430.97688326,707.43942034)
\curveto(431.09687991,707.42941033)(431.18187983,707.36941039)(431.23188326,707.25942034)
\curveto(431.25187976,707.20941055)(431.26187975,707.14941061)(431.26188326,707.07942034)
\lineto(431.26188326,706.86942034)
\lineto(431.26188326,706.19442034)
\curveto(431.26187975,706.14441162)(431.25687975,706.08441168)(431.24688326,706.01442034)
\curveto(431.24687976,705.95441181)(431.25187976,705.89941186)(431.26188326,705.84942034)
\lineto(431.26188326,705.68442034)
\curveto(431.26187975,705.60441216)(431.26687974,705.52941223)(431.27688326,705.45942034)
\curveto(431.28687972,705.39941236)(431.3118797,705.34441242)(431.35188326,705.29442034)
\curveto(431.42187959,705.20441256)(431.54687946,705.15441261)(431.72688326,705.14442034)
\lineto(432.26688326,705.14442034)
\lineto(432.44688326,705.14442034)
\curveto(432.5068785,705.14441262)(432.56187845,705.13441263)(432.61188326,705.11442034)
\curveto(432.72187829,705.0644127)(432.78187823,704.97441279)(432.79188326,704.84442034)
\curveto(432.8118782,704.71441305)(432.82187819,704.56941319)(432.82188326,704.40942034)
\lineto(432.82188326,704.19942034)
\curveto(432.83187818,704.12941363)(432.82687818,704.06941369)(432.80688326,704.01942034)
\curveto(432.75687825,703.8594139)(432.65187836,703.77441399)(432.49188326,703.76442034)
\curveto(432.33187868,703.75441401)(432.15187886,703.74941401)(431.95188326,703.74942034)
\lineto(431.81688326,703.74942034)
\curveto(431.77687923,703.759414)(431.74187927,703.759414)(431.71188326,703.74942034)
\curveto(431.67187934,703.73941402)(431.63687937,703.73441403)(431.60688326,703.73442034)
\curveto(431.57687943,703.74441402)(431.54687946,703.73941402)(431.51688326,703.71942034)
\curveto(431.43687957,703.69941406)(431.37687963,703.65441411)(431.33688326,703.58442034)
\curveto(431.3068797,703.52441424)(431.28187973,703.44941431)(431.26188326,703.35942034)
\curveto(431.25187976,703.30941445)(431.25187976,703.25441451)(431.26188326,703.19442034)
\curveto(431.27187974,703.13441463)(431.27187974,703.07941468)(431.26188326,703.02942034)
\lineto(431.26188326,702.09942034)
\lineto(431.26188326,700.34442034)
\curveto(431.26187975,700.09441767)(431.26687974,699.87441789)(431.27688326,699.68442034)
\curveto(431.29687971,699.50441826)(431.36187965,699.34441842)(431.47188326,699.20442034)
\curveto(431.52187949,699.14441862)(431.58687942,699.09941866)(431.66688326,699.06942034)
\lineto(431.93688326,699.00942034)
\curveto(431.96687904,698.99941876)(431.99687901,698.99441877)(432.02688326,698.99442034)
\curveto(432.06687894,699.00441876)(432.09687891,699.00441876)(432.11688326,698.99442034)
\lineto(432.28188326,698.99442034)
\curveto(432.39187862,698.99441877)(432.48687852,698.98941877)(432.56688326,698.97942034)
\curveto(432.64687836,698.96941879)(432.7118783,698.92941883)(432.76188326,698.85942034)
\curveto(432.80187821,698.79941896)(432.82187819,698.71941904)(432.82188326,698.61942034)
\lineto(432.82188326,698.33442034)
\curveto(432.82187819,698.12441964)(432.81687819,697.92941983)(432.80688326,697.74942034)
\curveto(432.8068782,697.57942018)(432.72687828,697.4644203)(432.56688326,697.40442034)
\curveto(432.51687849,697.38442038)(432.47187854,697.37942038)(432.43188326,697.38942034)
\curveto(432.39187862,697.38942037)(432.34687866,697.37942038)(432.29688326,697.35942034)
\lineto(432.14688326,697.35942034)
\curveto(432.12687888,697.3594204)(432.09687891,697.3644204)(432.05688326,697.37442034)
\curveto(432.01687899,697.37442039)(431.98187903,697.36942039)(431.95188326,697.35942034)
\curveto(431.90187911,697.34942041)(431.84687916,697.34942041)(431.78688326,697.35942034)
\lineto(431.63688326,697.35942034)
\lineto(431.48688326,697.35942034)
\curveto(431.43687957,697.34942041)(431.39187962,697.34942041)(431.35188326,697.35942034)
\lineto(431.18688326,697.35942034)
\curveto(431.13687987,697.36942039)(431.08187993,697.37442039)(431.02188326,697.37442034)
\curveto(430.96188005,697.37442039)(430.9068801,697.37942038)(430.85688326,697.38942034)
\curveto(430.78688022,697.39942036)(430.72188029,697.40942035)(430.66188326,697.41942034)
\lineto(430.48188326,697.44942034)
\curveto(430.37188064,697.47942028)(430.26688074,697.51442025)(430.16688326,697.55442034)
\curveto(430.06688094,697.59442017)(429.97188104,697.63942012)(429.88188326,697.68942034)
\lineto(429.79188326,697.74942034)
\curveto(429.76188125,697.77941998)(429.72688128,697.80941995)(429.68688326,697.83942034)
\curveto(429.66688134,697.8594199)(429.64188137,697.87941988)(429.61188326,697.89942034)
\lineto(429.53688326,697.97442034)
\curveto(429.39688161,698.1644196)(429.29188172,698.37441939)(429.22188326,698.60442034)
\curveto(429.20188181,698.64441912)(429.19188182,698.67941908)(429.19188326,698.70942034)
\curveto(429.20188181,698.74941901)(429.20188181,698.79441897)(429.19188326,698.84442034)
\curveto(429.18188183,698.8644189)(429.17688183,698.88941887)(429.17688326,698.91942034)
\curveto(429.17688183,698.94941881)(429.17188184,698.97441879)(429.16188326,698.99442034)
\lineto(429.16188326,699.14442034)
\curveto(429.15188186,699.18441858)(429.14688186,699.22941853)(429.14688326,699.27942034)
\curveto(429.15688185,699.32941843)(429.16188185,699.37941838)(429.16188326,699.42942034)
\lineto(429.16188326,699.99942034)
\lineto(429.16188326,702.23442034)
\lineto(429.16188326,703.02942034)
\lineto(429.16188326,703.23942034)
\curveto(429.17188184,703.30941445)(429.16688184,703.37441439)(429.14688326,703.43442034)
\curveto(429.1068819,703.57441419)(429.03688197,703.6644141)(428.93688326,703.70442034)
\curveto(428.82688218,703.75441401)(428.68688232,703.76941399)(428.51688326,703.74942034)
\curveto(428.34688266,703.72941403)(428.20188281,703.74441402)(428.08188326,703.79442034)
\curveto(428.00188301,703.82441394)(427.95188306,703.86941389)(427.93188326,703.92942034)
\curveto(427.9118831,703.98941377)(427.89188312,704.0644137)(427.87188326,704.15442034)
\lineto(427.87188326,704.46942034)
\curveto(427.87188314,704.64941311)(427.88188313,704.79441297)(427.90188326,704.90442034)
\curveto(427.92188309,705.01441275)(428.006883,705.08941267)(428.15688326,705.12942034)
\curveto(428.19688281,705.14941261)(428.23688277,705.15441261)(428.27688326,705.14442034)
\lineto(428.41188326,705.14442034)
\curveto(428.56188245,705.14441262)(428.70188231,705.14941261)(428.83188326,705.15942034)
\curveto(428.96188205,705.17941258)(429.05188196,705.23941252)(429.10188326,705.33942034)
\curveto(429.13188188,705.40941235)(429.14688186,705.48941227)(429.14688326,705.57942034)
\curveto(429.15688185,705.66941209)(429.16188185,705.759412)(429.16188326,705.84942034)
\lineto(429.16188326,706.77942034)
\lineto(429.16188326,707.03442034)
\curveto(429.16188185,707.12441064)(429.17188184,707.19941056)(429.19188326,707.25942034)
\curveto(429.24188177,707.3594104)(429.31688169,707.42441034)(429.41688326,707.45442034)
\curveto(429.43688157,707.4644103)(429.46188155,707.4644103)(429.49188326,707.45442034)
\curveto(429.53188148,707.45441031)(429.56188145,707.4594103)(429.58188326,707.46942034)
}
}
{
\newrgbcolor{curcolor}{0 0 0}
\pscustom[linestyle=none,fillstyle=solid,fillcolor=curcolor]
{
\newpath
\moveto(438.23032076,705.35442034)
\curveto(438.34031544,705.35441241)(438.43531535,705.34441242)(438.51532076,705.32442034)
\curveto(438.60531518,705.30441246)(438.67531511,705.2594125)(438.72532076,705.18942034)
\curveto(438.785315,705.10941265)(438.81531497,704.96941279)(438.81532076,704.76942034)
\lineto(438.81532076,704.25942034)
\lineto(438.81532076,703.88442034)
\curveto(438.82531496,703.74441402)(438.81031497,703.63441413)(438.77032076,703.55442034)
\curveto(438.73031505,703.48441428)(438.67031511,703.43941432)(438.59032076,703.41942034)
\curveto(438.52031526,703.39941436)(438.43531535,703.38941437)(438.33532076,703.38942034)
\curveto(438.24531554,703.38941437)(438.14531564,703.39441437)(438.03532076,703.40442034)
\curveto(437.93531585,703.41441435)(437.84031594,703.40941435)(437.75032076,703.38942034)
\curveto(437.6803161,703.36941439)(437.61031617,703.35441441)(437.54032076,703.34442034)
\curveto(437.47031631,703.34441442)(437.40531638,703.33441443)(437.34532076,703.31442034)
\curveto(437.1853166,703.2644145)(437.02531676,703.18941457)(436.86532076,703.08942034)
\curveto(436.70531708,702.99941476)(436.5803172,702.89441487)(436.49032076,702.77442034)
\curveto(436.44031734,702.69441507)(436.3853174,702.60941515)(436.32532076,702.51942034)
\curveto(436.27531751,702.43941532)(436.22531756,702.35441541)(436.17532076,702.26442034)
\curveto(436.14531764,702.18441558)(436.11531767,702.09941566)(436.08532076,702.00942034)
\lineto(436.02532076,701.76942034)
\curveto(436.00531778,701.69941606)(435.99531779,701.62441614)(435.99532076,701.54442034)
\curveto(435.99531779,701.47441629)(435.9853178,701.40441636)(435.96532076,701.33442034)
\curveto(435.95531783,701.29441647)(435.95031783,701.25441651)(435.95032076,701.21442034)
\curveto(435.96031782,701.18441658)(435.96031782,701.15441661)(435.95032076,701.12442034)
\lineto(435.95032076,700.88442034)
\curveto(435.93031785,700.81441695)(435.92531786,700.73441703)(435.93532076,700.64442034)
\curveto(435.94531784,700.5644172)(435.95031783,700.48441728)(435.95032076,700.40442034)
\lineto(435.95032076,699.44442034)
\lineto(435.95032076,698.16942034)
\curveto(435.95031783,698.03941972)(435.94531784,697.91941984)(435.93532076,697.80942034)
\curveto(435.92531786,697.69942006)(435.89531789,697.60942015)(435.84532076,697.53942034)
\curveto(435.82531796,697.50942025)(435.79031799,697.48442028)(435.74032076,697.46442034)
\curveto(435.70031808,697.45442031)(435.65531813,697.44442032)(435.60532076,697.43442034)
\lineto(435.53032076,697.43442034)
\curveto(435.4803183,697.42442034)(435.42531836,697.41942034)(435.36532076,697.41942034)
\lineto(435.20032076,697.41942034)
\lineto(434.55532076,697.41942034)
\curveto(434.49531929,697.42942033)(434.43031935,697.43442033)(434.36032076,697.43442034)
\lineto(434.16532076,697.43442034)
\curveto(434.11531967,697.45442031)(434.06531972,697.46942029)(434.01532076,697.47942034)
\curveto(433.96531982,697.49942026)(433.93031985,697.53442023)(433.91032076,697.58442034)
\curveto(433.87031991,697.63442013)(433.84531994,697.70442006)(433.83532076,697.79442034)
\lineto(433.83532076,698.09442034)
\lineto(433.83532076,699.11442034)
\lineto(433.83532076,703.34442034)
\lineto(433.83532076,704.45442034)
\lineto(433.83532076,704.73942034)
\curveto(433.83531995,704.83941292)(433.85531993,704.91941284)(433.89532076,704.97942034)
\curveto(433.94531984,705.0594127)(434.02031976,705.10941265)(434.12032076,705.12942034)
\curveto(434.22031956,705.14941261)(434.34031944,705.1594126)(434.48032076,705.15942034)
\lineto(435.24532076,705.15942034)
\curveto(435.36531842,705.1594126)(435.47031831,705.14941261)(435.56032076,705.12942034)
\curveto(435.65031813,705.11941264)(435.72031806,705.07441269)(435.77032076,704.99442034)
\curveto(435.80031798,704.94441282)(435.81531797,704.87441289)(435.81532076,704.78442034)
\lineto(435.84532076,704.51442034)
\curveto(435.85531793,704.43441333)(435.87031791,704.3594134)(435.89032076,704.28942034)
\curveto(435.92031786,704.21941354)(435.97031781,704.18441358)(436.04032076,704.18442034)
\curveto(436.06031772,704.20441356)(436.0803177,704.21441355)(436.10032076,704.21442034)
\curveto(436.12031766,704.21441355)(436.14031764,704.22441354)(436.16032076,704.24442034)
\curveto(436.22031756,704.29441347)(436.27031751,704.34941341)(436.31032076,704.40942034)
\curveto(436.36031742,704.47941328)(436.42031736,704.53941322)(436.49032076,704.58942034)
\curveto(436.53031725,704.61941314)(436.56531722,704.64941311)(436.59532076,704.67942034)
\curveto(436.62531716,704.71941304)(436.66031712,704.75441301)(436.70032076,704.78442034)
\lineto(436.97032076,704.96442034)
\curveto(437.07031671,705.02441274)(437.17031661,705.07941268)(437.27032076,705.12942034)
\curveto(437.37031641,705.16941259)(437.47031631,705.20441256)(437.57032076,705.23442034)
\lineto(437.90032076,705.32442034)
\curveto(437.93031585,705.33441243)(437.9853158,705.33441243)(438.06532076,705.32442034)
\curveto(438.15531563,705.32441244)(438.21031557,705.33441243)(438.23032076,705.35442034)
}
}
{
\newrgbcolor{curcolor}{0 0 0}
\pscustom[linestyle=none,fillstyle=solid,fillcolor=curcolor]
{
\newpath
\moveto(441.73539888,708.00942034)
\curveto(441.80539593,707.92940983)(441.8403959,707.80940995)(441.84039888,707.64942034)
\lineto(441.84039888,707.18442034)
\lineto(441.84039888,706.77942034)
\curveto(441.8403959,706.63941112)(441.80539593,706.54441122)(441.73539888,706.49442034)
\curveto(441.67539606,706.44441132)(441.59539614,706.41441135)(441.49539888,706.40442034)
\curveto(441.40539633,706.39441137)(441.30539643,706.38941137)(441.19539888,706.38942034)
\lineto(440.35539888,706.38942034)
\curveto(440.24539749,706.38941137)(440.14539759,706.39441137)(440.05539888,706.40442034)
\curveto(439.97539776,706.41441135)(439.90539783,706.44441132)(439.84539888,706.49442034)
\curveto(439.80539793,706.52441124)(439.77539796,706.57941118)(439.75539888,706.65942034)
\curveto(439.74539799,706.74941101)(439.735398,706.84441092)(439.72539888,706.94442034)
\lineto(439.72539888,707.27442034)
\curveto(439.735398,707.38441038)(439.740398,707.47941028)(439.74039888,707.55942034)
\lineto(439.74039888,707.76942034)
\curveto(439.75039799,707.83940992)(439.77039797,707.89940986)(439.80039888,707.94942034)
\curveto(439.82039792,707.98940977)(439.84539789,708.01940974)(439.87539888,708.03942034)
\lineto(439.99539888,708.09942034)
\curveto(440.01539772,708.09940966)(440.0403977,708.09940966)(440.07039888,708.09942034)
\curveto(440.10039764,708.10940965)(440.12539761,708.11440965)(440.14539888,708.11442034)
\lineto(441.24039888,708.11442034)
\curveto(441.3403964,708.11440965)(441.4353963,708.10940965)(441.52539888,708.09942034)
\curveto(441.61539612,708.08940967)(441.68539605,708.0594097)(441.73539888,708.00942034)
\moveto(441.84039888,698.24442034)
\curveto(441.8403959,698.04441972)(441.8353959,697.87441989)(441.82539888,697.73442034)
\curveto(441.81539592,697.59442017)(441.72539601,697.49942026)(441.55539888,697.44942034)
\curveto(441.49539624,697.42942033)(441.43039631,697.41942034)(441.36039888,697.41942034)
\curveto(441.29039645,697.42942033)(441.21539652,697.43442033)(441.13539888,697.43442034)
\lineto(440.29539888,697.43442034)
\curveto(440.20539753,697.43442033)(440.11539762,697.43942032)(440.02539888,697.44942034)
\curveto(439.94539779,697.4594203)(439.88539785,697.48942027)(439.84539888,697.53942034)
\curveto(439.78539795,697.60942015)(439.75039799,697.69442007)(439.74039888,697.79442034)
\lineto(439.74039888,698.13942034)
\lineto(439.74039888,704.46942034)
\lineto(439.74039888,704.76942034)
\curveto(439.740398,704.86941289)(439.76039798,704.94941281)(439.80039888,705.00942034)
\curveto(439.86039788,705.07941268)(439.94539779,705.12441264)(440.05539888,705.14442034)
\curveto(440.07539766,705.15441261)(440.10039764,705.15441261)(440.13039888,705.14442034)
\curveto(440.17039757,705.14441262)(440.20039754,705.14941261)(440.22039888,705.15942034)
\lineto(440.97039888,705.15942034)
\lineto(441.16539888,705.15942034)
\curveto(441.24539649,705.16941259)(441.31039643,705.16941259)(441.36039888,705.15942034)
\lineto(441.48039888,705.15942034)
\curveto(441.5403962,705.13941262)(441.59539614,705.12441264)(441.64539888,705.11442034)
\curveto(441.69539604,705.10441266)(441.735396,705.07441269)(441.76539888,705.02442034)
\curveto(441.80539593,704.97441279)(441.82539591,704.90441286)(441.82539888,704.81442034)
\curveto(441.8353959,704.72441304)(441.8403959,704.62941313)(441.84039888,704.52942034)
\lineto(441.84039888,698.24442034)
}
}
{
\newrgbcolor{curcolor}{0 0 0}
\pscustom[linestyle=none,fillstyle=solid,fillcolor=curcolor]
{
\newpath
\moveto(451.30258638,701.67942034)
\curveto(451.32257778,701.61941614)(451.33257777,701.51441625)(451.33258638,701.36442034)
\curveto(451.33257777,701.22441654)(451.32757778,701.12441664)(451.31758638,701.06442034)
\curveto(451.31757779,701.01441675)(451.31257779,700.96941679)(451.30258638,700.92942034)
\lineto(451.30258638,700.80942034)
\curveto(451.28257782,700.72941703)(451.27257783,700.64941711)(451.27258638,700.56942034)
\curveto(451.27257783,700.49941726)(451.26257784,700.42441734)(451.24258638,700.34442034)
\curveto(451.24257786,700.30441746)(451.23257787,700.23441753)(451.21258638,700.13442034)
\curveto(451.18257792,700.01441775)(451.15257795,699.88941787)(451.12258638,699.75942034)
\curveto(451.102578,699.63941812)(451.06757804,699.52441824)(451.01758638,699.41442034)
\curveto(450.83757827,698.9644188)(450.61257849,698.57441919)(450.34258638,698.24442034)
\curveto(450.07257903,697.91441985)(449.71757939,697.65442011)(449.27758638,697.46442034)
\curveto(449.18757992,697.42442034)(449.09258001,697.39442037)(448.99258638,697.37442034)
\curveto(448.9025802,697.34442042)(448.8025803,697.31442045)(448.69258638,697.28442034)
\curveto(448.63258047,697.2644205)(448.56758054,697.25442051)(448.49758638,697.25442034)
\curveto(448.43758067,697.25442051)(448.37758073,697.24942051)(448.31758638,697.23942034)
\lineto(448.18258638,697.23942034)
\curveto(448.12258098,697.21942054)(448.04258106,697.21442055)(447.94258638,697.22442034)
\curveto(447.84258126,697.22442054)(447.76258134,697.23442053)(447.70258638,697.25442034)
\lineto(447.61258638,697.25442034)
\curveto(447.56258154,697.2644205)(447.5075816,697.27442049)(447.44758638,697.28442034)
\curveto(447.38758172,697.28442048)(447.32758178,697.28942047)(447.26758638,697.29942034)
\curveto(447.07758203,697.34942041)(446.9025822,697.39942036)(446.74258638,697.44942034)
\curveto(446.58258252,697.49942026)(446.43258267,697.56942019)(446.29258638,697.65942034)
\lineto(446.11258638,697.77942034)
\curveto(446.06258304,697.81941994)(446.01258309,697.8644199)(445.96258638,697.91442034)
\lineto(445.87258638,697.97442034)
\curveto(445.84258326,697.99441977)(445.81258329,698.00941975)(445.78258638,698.01942034)
\curveto(445.69258341,698.04941971)(445.63758347,698.02941973)(445.61758638,697.95942034)
\curveto(445.56758354,697.88941987)(445.53258357,697.80441996)(445.51258638,697.70442034)
\curveto(445.5025836,697.61442015)(445.46758364,697.54442022)(445.40758638,697.49442034)
\curveto(445.34758376,697.45442031)(445.27758383,697.42942033)(445.19758638,697.41942034)
\lineto(444.92758638,697.41942034)
\lineto(444.20758638,697.41942034)
\lineto(443.98258638,697.41942034)
\curveto(443.91258519,697.40942035)(443.84758526,697.41442035)(443.78758638,697.43442034)
\curveto(443.64758546,697.48442028)(443.56758554,697.57442019)(443.54758638,697.70442034)
\curveto(443.53758557,697.84441992)(443.53258557,697.99941976)(443.53258638,698.16942034)
\lineto(443.53258638,707.31942034)
\lineto(443.53258638,707.66442034)
\curveto(443.53258557,707.78440998)(443.55758555,707.87940988)(443.60758638,707.94942034)
\curveto(443.64758546,708.01940974)(443.71758539,708.0644097)(443.81758638,708.08442034)
\curveto(443.83758527,708.09440967)(443.85758525,708.09440967)(443.87758638,708.08442034)
\curveto(443.9075852,708.08440968)(443.93258517,708.08940967)(443.95258638,708.09942034)
\lineto(444.89758638,708.09942034)
\curveto(445.07758403,708.09940966)(445.23258387,708.08940967)(445.36258638,708.06942034)
\curveto(445.49258361,708.0594097)(445.57758353,707.98440978)(445.61758638,707.84442034)
\curveto(445.64758346,707.74441002)(445.65758345,707.60941015)(445.64758638,707.43942034)
\curveto(445.63758347,707.27941048)(445.63258347,707.13941062)(445.63258638,707.01942034)
\lineto(445.63258638,705.38442034)
\lineto(445.63258638,705.05442034)
\curveto(445.63258347,704.94441282)(445.64258346,704.84941291)(445.66258638,704.76942034)
\curveto(445.67258343,704.71941304)(445.68258342,704.67441309)(445.69258638,704.63442034)
\curveto(445.7025834,704.60441316)(445.72758338,704.58441318)(445.76758638,704.57442034)
\curveto(445.78758332,704.55441321)(445.81258329,704.54441322)(445.84258638,704.54442034)
\curveto(445.88258322,704.54441322)(445.91258319,704.54941321)(445.93258638,704.55942034)
\curveto(446.0025831,704.59941316)(446.06758304,704.63941312)(446.12758638,704.67942034)
\curveto(446.18758292,704.72941303)(446.25258285,704.77941298)(446.32258638,704.82942034)
\curveto(446.45258265,704.91941284)(446.58758252,704.99441277)(446.72758638,705.05442034)
\curveto(446.86758224,705.12441264)(447.02258208,705.18441258)(447.19258638,705.23442034)
\curveto(447.27258183,705.2644125)(447.35258175,705.27941248)(447.43258638,705.27942034)
\curveto(447.51258159,705.28941247)(447.59258151,705.30441246)(447.67258638,705.32442034)
\curveto(447.74258136,705.34441242)(447.81758129,705.35441241)(447.89758638,705.35442034)
\lineto(448.13758638,705.35442034)
\lineto(448.28758638,705.35442034)
\curveto(448.31758079,705.34441242)(448.35258075,705.33941242)(448.39258638,705.33942034)
\curveto(448.43258067,705.34941241)(448.47258063,705.34941241)(448.51258638,705.33942034)
\curveto(448.62258048,705.30941245)(448.72258038,705.28441248)(448.81258638,705.26442034)
\curveto(448.91258019,705.25441251)(449.0075801,705.22941253)(449.09758638,705.18942034)
\curveto(449.55757955,704.99941276)(449.93257917,704.75441301)(450.22258638,704.45442034)
\curveto(450.51257859,704.15441361)(450.75757835,703.77941398)(450.95758638,703.32942034)
\curveto(451.0075781,703.20941455)(451.04757806,703.08441468)(451.07758638,702.95442034)
\curveto(451.11757799,702.82441494)(451.15757795,702.68941507)(451.19758638,702.54942034)
\curveto(451.21757789,702.47941528)(451.22757788,702.40941535)(451.22758638,702.33942034)
\curveto(451.23757787,702.27941548)(451.25257785,702.20941555)(451.27258638,702.12942034)
\curveto(451.29257781,702.07941568)(451.29757781,702.02441574)(451.28758638,701.96442034)
\curveto(451.28757782,701.90441586)(451.29257781,701.84441592)(451.30258638,701.78442034)
\lineto(451.30258638,701.67942034)
\moveto(449.08258638,700.26942034)
\curveto(449.11257999,700.36941739)(449.13757997,700.49441727)(449.15758638,700.64442034)
\curveto(449.18757992,700.79441697)(449.2025799,700.94441682)(449.20258638,701.09442034)
\curveto(449.21257989,701.25441651)(449.21257989,701.40941635)(449.20258638,701.55942034)
\curveto(449.2025799,701.71941604)(449.18757992,701.85441591)(449.15758638,701.96442034)
\curveto(449.12757998,702.0644157)(449.10758,702.1594156)(449.09758638,702.24942034)
\curveto(449.08758002,702.33941542)(449.06258004,702.42441534)(449.02258638,702.50442034)
\curveto(448.88258022,702.85441491)(448.68258042,703.14941461)(448.42258638,703.38942034)
\curveto(448.17258093,703.63941412)(447.8025813,703.764414)(447.31258638,703.76442034)
\curveto(447.27258183,703.764414)(447.23758187,703.759414)(447.20758638,703.74942034)
\lineto(447.10258638,703.74942034)
\curveto(447.03258207,703.72941403)(446.96758214,703.70941405)(446.90758638,703.68942034)
\curveto(446.84758226,703.67941408)(446.78758232,703.6644141)(446.72758638,703.64442034)
\curveto(446.43758267,703.51441425)(446.21758289,703.32941443)(446.06758638,703.08942034)
\curveto(445.91758319,702.8594149)(445.79258331,702.59441517)(445.69258638,702.29442034)
\curveto(445.66258344,702.21441555)(445.64258346,702.12941563)(445.63258638,702.03942034)
\curveto(445.63258347,701.9594158)(445.62258348,701.87941588)(445.60258638,701.79942034)
\curveto(445.59258351,701.76941599)(445.58758352,701.71941604)(445.58758638,701.64942034)
\curveto(445.57758353,701.60941615)(445.57258353,701.56941619)(445.57258638,701.52942034)
\curveto(445.58258352,701.48941627)(445.58258352,701.44941631)(445.57258638,701.40942034)
\curveto(445.55258355,701.32941643)(445.54758356,701.21941654)(445.55758638,701.07942034)
\curveto(445.56758354,700.93941682)(445.58258352,700.83941692)(445.60258638,700.77942034)
\curveto(445.62258348,700.68941707)(445.63258347,700.60441716)(445.63258638,700.52442034)
\curveto(445.64258346,700.44441732)(445.66258344,700.3644174)(445.69258638,700.28442034)
\curveto(445.78258332,700.00441776)(445.88758322,699.759418)(446.00758638,699.54942034)
\curveto(446.13758297,699.34941841)(446.31758279,699.17941858)(446.54758638,699.03942034)
\curveto(446.7075824,698.93941882)(446.87258223,698.86941889)(447.04258638,698.82942034)
\curveto(447.06258204,698.82941893)(447.08258202,698.82441894)(447.10258638,698.81442034)
\lineto(447.19258638,698.81442034)
\curveto(447.22258188,698.80441896)(447.27258183,698.79441897)(447.34258638,698.78442034)
\curveto(447.41258169,698.78441898)(447.47258163,698.78941897)(447.52258638,698.79942034)
\curveto(447.62258148,698.81941894)(447.71258139,698.83441893)(447.79258638,698.84442034)
\curveto(447.88258122,698.8644189)(447.96758114,698.88941887)(448.04758638,698.91942034)
\curveto(448.32758078,699.04941871)(448.54258056,699.22941853)(448.69258638,699.45942034)
\curveto(448.85258025,699.68941807)(448.98258012,699.9594178)(449.08258638,700.26942034)
}
}
{
\newrgbcolor{curcolor}{0 0 0}
\pscustom[linestyle=none,fillstyle=solid,fillcolor=curcolor]
{
\newpath
\moveto(453.09250826,705.14442034)
\lineto(454.21750826,705.14442034)
\curveto(454.32750582,705.14441262)(454.42750572,705.13941262)(454.51750826,705.12942034)
\curveto(454.60750554,705.11941264)(454.67250548,705.08441268)(454.71250826,705.02442034)
\curveto(454.76250539,704.9644128)(454.79250536,704.87941288)(454.80250826,704.76942034)
\curveto(454.81250534,704.66941309)(454.81750533,704.5644132)(454.81750826,704.45442034)
\lineto(454.81750826,703.40442034)
\lineto(454.81750826,701.16942034)
\curveto(454.81750533,700.80941695)(454.83250532,700.46941729)(454.86250826,700.14942034)
\curveto(454.89250526,699.82941793)(454.98250517,699.5644182)(455.13250826,699.35442034)
\curveto(455.27250488,699.14441862)(455.49750465,698.99441877)(455.80750826,698.90442034)
\curveto(455.85750429,698.89441887)(455.89750425,698.88941887)(455.92750826,698.88942034)
\curveto(455.96750418,698.88941887)(456.01250414,698.88441888)(456.06250826,698.87442034)
\curveto(456.11250404,698.8644189)(456.16750398,698.8594189)(456.22750826,698.85942034)
\curveto(456.28750386,698.8594189)(456.33250382,698.8644189)(456.36250826,698.87442034)
\curveto(456.41250374,698.89441887)(456.4525037,698.89941886)(456.48250826,698.88942034)
\curveto(456.52250363,698.87941888)(456.56250359,698.88441888)(456.60250826,698.90442034)
\curveto(456.81250334,698.95441881)(456.97750317,699.01941874)(457.09750826,699.09942034)
\curveto(457.27750287,699.20941855)(457.41750273,699.34941841)(457.51750826,699.51942034)
\curveto(457.62750252,699.69941806)(457.70250245,699.89441787)(457.74250826,700.10442034)
\curveto(457.79250236,700.32441744)(457.82250233,700.5644172)(457.83250826,700.82442034)
\curveto(457.84250231,701.09441667)(457.8475023,701.37441639)(457.84750826,701.66442034)
\lineto(457.84750826,703.47942034)
\lineto(457.84750826,704.45442034)
\lineto(457.84750826,704.72442034)
\curveto(457.8475023,704.82441294)(457.86750228,704.90441286)(457.90750826,704.96442034)
\curveto(457.95750219,705.05441271)(458.03250212,705.10441266)(458.13250826,705.11442034)
\curveto(458.23250192,705.13441263)(458.3525018,705.14441262)(458.49250826,705.14442034)
\lineto(459.28750826,705.14442034)
\lineto(459.57250826,705.14442034)
\curveto(459.66250049,705.14441262)(459.73750041,705.12441264)(459.79750826,705.08442034)
\curveto(459.87750027,705.03441273)(459.92250023,704.9594128)(459.93250826,704.85942034)
\curveto(459.94250021,704.759413)(459.9475002,704.64441312)(459.94750826,704.51442034)
\lineto(459.94750826,703.37442034)
\lineto(459.94750826,699.15942034)
\lineto(459.94750826,698.09442034)
\lineto(459.94750826,697.79442034)
\curveto(459.9475002,697.69442007)(459.92750022,697.61942014)(459.88750826,697.56942034)
\curveto(459.83750031,697.48942027)(459.76250039,697.44442032)(459.66250826,697.43442034)
\curveto(459.56250059,697.42442034)(459.45750069,697.41942034)(459.34750826,697.41942034)
\lineto(458.53750826,697.41942034)
\curveto(458.42750172,697.41942034)(458.32750182,697.42442034)(458.23750826,697.43442034)
\curveto(458.15750199,697.44442032)(458.09250206,697.48442028)(458.04250826,697.55442034)
\curveto(458.02250213,697.58442018)(458.00250215,697.62942013)(457.98250826,697.68942034)
\curveto(457.97250218,697.74942001)(457.95750219,697.80941995)(457.93750826,697.86942034)
\curveto(457.92750222,697.92941983)(457.91250224,697.98441978)(457.89250826,698.03442034)
\curveto(457.87250228,698.08441968)(457.84250231,698.11441965)(457.80250826,698.12442034)
\curveto(457.78250237,698.14441962)(457.75750239,698.14941961)(457.72750826,698.13942034)
\curveto(457.69750245,698.12941963)(457.67250248,698.11941964)(457.65250826,698.10942034)
\curveto(457.58250257,698.06941969)(457.52250263,698.02441974)(457.47250826,697.97442034)
\curveto(457.42250273,697.92441984)(457.36750278,697.87941988)(457.30750826,697.83942034)
\curveto(457.26750288,697.80941995)(457.22750292,697.77441999)(457.18750826,697.73442034)
\curveto(457.15750299,697.70442006)(457.11750303,697.67442009)(457.06750826,697.64442034)
\curveto(456.83750331,697.50442026)(456.56750358,697.39442037)(456.25750826,697.31442034)
\curveto(456.18750396,697.29442047)(456.11750403,697.28442048)(456.04750826,697.28442034)
\curveto(455.97750417,697.27442049)(455.90250425,697.2594205)(455.82250826,697.23942034)
\curveto(455.78250437,697.22942053)(455.73750441,697.22942053)(455.68750826,697.23942034)
\curveto(455.6475045,697.23942052)(455.60750454,697.23442053)(455.56750826,697.22442034)
\curveto(455.53750461,697.21442055)(455.47250468,697.21442055)(455.37250826,697.22442034)
\curveto(455.28250487,697.22442054)(455.22250493,697.22942053)(455.19250826,697.23942034)
\curveto(455.14250501,697.23942052)(455.09250506,697.24442052)(455.04250826,697.25442034)
\lineto(454.89250826,697.25442034)
\curveto(454.77250538,697.28442048)(454.65750549,697.30942045)(454.54750826,697.32942034)
\curveto(454.43750571,697.34942041)(454.32750582,697.37942038)(454.21750826,697.41942034)
\curveto(454.16750598,697.43942032)(454.12250603,697.45442031)(454.08250826,697.46442034)
\curveto(454.0525061,697.48442028)(454.01250614,697.50442026)(453.96250826,697.52442034)
\curveto(453.61250654,697.71442005)(453.33250682,697.97941978)(453.12250826,698.31942034)
\curveto(452.99250716,698.52941923)(452.89750725,698.77941898)(452.83750826,699.06942034)
\curveto(452.77750737,699.36941839)(452.73750741,699.68441808)(452.71750826,700.01442034)
\curveto(452.70750744,700.35441741)(452.70250745,700.69941706)(452.70250826,701.04942034)
\curveto(452.71250744,701.40941635)(452.71750743,701.764416)(452.71750826,702.11442034)
\lineto(452.71750826,704.15442034)
\curveto(452.71750743,704.28441348)(452.71250744,704.43441333)(452.70250826,704.60442034)
\curveto(452.70250745,704.78441298)(452.72750742,704.91441285)(452.77750826,704.99442034)
\curveto(452.80750734,705.04441272)(452.86750728,705.08941267)(452.95750826,705.12942034)
\curveto(453.01750713,705.12941263)(453.06250709,705.13441263)(453.09250826,705.14442034)
}
}
{
\newrgbcolor{curcolor}{0 0 0}
\pscustom[linestyle=none,fillstyle=solid,fillcolor=curcolor]
{
\newpath
\moveto(465.14875826,705.36942034)
\curveto(465.9587531,705.38941237)(466.63375242,705.26941249)(467.17375826,705.00942034)
\curveto(467.72375133,704.74941301)(468.1587509,704.37941338)(468.47875826,703.89942034)
\curveto(468.63875042,703.6594141)(468.7587503,703.38441438)(468.83875826,703.07442034)
\curveto(468.8587502,703.02441474)(468.87375018,702.9594148)(468.88375826,702.87942034)
\curveto(468.90375015,702.79941496)(468.90375015,702.72941503)(468.88375826,702.66942034)
\curveto(468.84375021,702.5594152)(468.77375028,702.49441527)(468.67375826,702.47442034)
\curveto(468.57375048,702.4644153)(468.4537506,702.4594153)(468.31375826,702.45942034)
\lineto(467.53375826,702.45942034)
\lineto(467.24875826,702.45942034)
\curveto(467.1587519,702.4594153)(467.08375197,702.47941528)(467.02375826,702.51942034)
\curveto(466.94375211,702.5594152)(466.88875217,702.61941514)(466.85875826,702.69942034)
\curveto(466.82875223,702.78941497)(466.78875227,702.87941488)(466.73875826,702.96942034)
\curveto(466.67875238,703.07941468)(466.61375244,703.17941458)(466.54375826,703.26942034)
\curveto(466.47375258,703.3594144)(466.39375266,703.43941432)(466.30375826,703.50942034)
\curveto(466.16375289,703.59941416)(466.00875305,703.66941409)(465.83875826,703.71942034)
\curveto(465.77875328,703.73941402)(465.71875334,703.74941401)(465.65875826,703.74942034)
\curveto(465.59875346,703.74941401)(465.54375351,703.759414)(465.49375826,703.77942034)
\lineto(465.34375826,703.77942034)
\curveto(465.14375391,703.77941398)(464.98375407,703.759414)(464.86375826,703.71942034)
\curveto(464.57375448,703.62941413)(464.33875472,703.48941427)(464.15875826,703.29942034)
\curveto(463.97875508,703.11941464)(463.83375522,702.89941486)(463.72375826,702.63942034)
\curveto(463.67375538,702.52941523)(463.63375542,702.40941535)(463.60375826,702.27942034)
\curveto(463.58375547,702.1594156)(463.5587555,702.02941573)(463.52875826,701.88942034)
\curveto(463.51875554,701.84941591)(463.51375554,701.80941595)(463.51375826,701.76942034)
\curveto(463.51375554,701.72941603)(463.50875555,701.68941607)(463.49875826,701.64942034)
\curveto(463.47875558,701.54941621)(463.46875559,701.40941635)(463.46875826,701.22942034)
\curveto(463.47875558,701.04941671)(463.49375556,700.90941685)(463.51375826,700.80942034)
\curveto(463.51375554,700.72941703)(463.51875554,700.67441709)(463.52875826,700.64442034)
\curveto(463.54875551,700.57441719)(463.5587555,700.50441726)(463.55875826,700.43442034)
\curveto(463.56875549,700.3644174)(463.58375547,700.29441747)(463.60375826,700.22442034)
\curveto(463.68375537,699.99441777)(463.77875528,699.78441798)(463.88875826,699.59442034)
\curveto(463.99875506,699.40441836)(464.13875492,699.24441852)(464.30875826,699.11442034)
\curveto(464.34875471,699.08441868)(464.40875465,699.04941871)(464.48875826,699.00942034)
\curveto(464.59875446,698.93941882)(464.70875435,698.89441887)(464.81875826,698.87442034)
\curveto(464.93875412,698.85441891)(465.08375397,698.83441893)(465.25375826,698.81442034)
\lineto(465.34375826,698.81442034)
\curveto(465.38375367,698.81441895)(465.41375364,698.81941894)(465.43375826,698.82942034)
\lineto(465.56875826,698.82942034)
\curveto(465.63875342,698.84941891)(465.70375335,698.8644189)(465.76375826,698.87442034)
\curveto(465.83375322,698.89441887)(465.89875316,698.91441885)(465.95875826,698.93442034)
\curveto(466.2587528,699.0644187)(466.48875257,699.25441851)(466.64875826,699.50442034)
\curveto(466.68875237,699.55441821)(466.72375233,699.60941815)(466.75375826,699.66942034)
\curveto(466.78375227,699.73941802)(466.80875225,699.79941796)(466.82875826,699.84942034)
\curveto(466.86875219,699.9594178)(466.90375215,700.05441771)(466.93375826,700.13442034)
\curveto(466.96375209,700.22441754)(467.03375202,700.29441747)(467.14375826,700.34442034)
\curveto(467.23375182,700.38441738)(467.37875168,700.39941736)(467.57875826,700.38942034)
\lineto(468.07375826,700.38942034)
\lineto(468.28375826,700.38942034)
\curveto(468.36375069,700.39941736)(468.42875063,700.39441737)(468.47875826,700.37442034)
\lineto(468.59875826,700.37442034)
\lineto(468.71875826,700.34442034)
\curveto(468.7587503,700.34441742)(468.78875027,700.33441743)(468.80875826,700.31442034)
\curveto(468.8587502,700.27441749)(468.88875017,700.21441755)(468.89875826,700.13442034)
\curveto(468.91875014,700.0644177)(468.91875014,699.98941777)(468.89875826,699.90942034)
\curveto(468.80875025,699.57941818)(468.69875036,699.28441848)(468.56875826,699.02442034)
\curveto(468.1587509,698.25441951)(467.50375155,697.71942004)(466.60375826,697.41942034)
\curveto(466.50375255,697.38942037)(466.39875266,697.36942039)(466.28875826,697.35942034)
\curveto(466.17875288,697.33942042)(466.06875299,697.31442045)(465.95875826,697.28442034)
\curveto(465.89875316,697.27442049)(465.83875322,697.26942049)(465.77875826,697.26942034)
\curveto(465.71875334,697.26942049)(465.6587534,697.2644205)(465.59875826,697.25442034)
\lineto(465.43375826,697.25442034)
\curveto(465.38375367,697.23442053)(465.30875375,697.22942053)(465.20875826,697.23942034)
\curveto(465.10875395,697.23942052)(465.03375402,697.24442052)(464.98375826,697.25442034)
\curveto(464.90375415,697.27442049)(464.82875423,697.28442048)(464.75875826,697.28442034)
\curveto(464.69875436,697.27442049)(464.63375442,697.27942048)(464.56375826,697.29942034)
\lineto(464.41375826,697.32942034)
\curveto(464.36375469,697.32942043)(464.31375474,697.33442043)(464.26375826,697.34442034)
\curveto(464.1537549,697.37442039)(464.04875501,697.40442036)(463.94875826,697.43442034)
\curveto(463.84875521,697.4644203)(463.7537553,697.49942026)(463.66375826,697.53942034)
\curveto(463.19375586,697.73942002)(462.79875626,697.99441977)(462.47875826,698.30442034)
\curveto(462.1587569,698.62441914)(461.89875716,699.01941874)(461.69875826,699.48942034)
\curveto(461.64875741,699.57941818)(461.60875745,699.67441809)(461.57875826,699.77442034)
\lineto(461.48875826,700.10442034)
\curveto(461.47875758,700.14441762)(461.47375758,700.17941758)(461.47375826,700.20942034)
\curveto(461.47375758,700.24941751)(461.46375759,700.29441747)(461.44375826,700.34442034)
\curveto(461.42375763,700.41441735)(461.41375764,700.48441728)(461.41375826,700.55442034)
\curveto(461.41375764,700.63441713)(461.40375765,700.70941705)(461.38375826,700.77942034)
\lineto(461.38375826,701.03442034)
\curveto(461.36375769,701.08441668)(461.3537577,701.13941662)(461.35375826,701.19942034)
\curveto(461.3537577,701.26941649)(461.36375769,701.32941643)(461.38375826,701.37942034)
\curveto(461.39375766,701.42941633)(461.39375766,701.47441629)(461.38375826,701.51442034)
\curveto(461.37375768,701.55441621)(461.37375768,701.59441617)(461.38375826,701.63442034)
\curveto(461.40375765,701.70441606)(461.40875765,701.76941599)(461.39875826,701.82942034)
\curveto(461.39875766,701.88941587)(461.40875765,701.94941581)(461.42875826,702.00942034)
\curveto(461.47875758,702.18941557)(461.51875754,702.3594154)(461.54875826,702.51942034)
\curveto(461.57875748,702.68941507)(461.62375743,702.85441491)(461.68375826,703.01442034)
\curveto(461.90375715,703.52441424)(462.17875688,703.94941381)(462.50875826,704.28942034)
\curveto(462.84875621,704.62941313)(463.27875578,704.90441286)(463.79875826,705.11442034)
\curveto(463.93875512,705.17441259)(464.08375497,705.21441255)(464.23375826,705.23442034)
\curveto(464.38375467,705.2644125)(464.53875452,705.29941246)(464.69875826,705.33942034)
\curveto(464.77875428,705.34941241)(464.8537542,705.35441241)(464.92375826,705.35442034)
\curveto(464.99375406,705.35441241)(465.06875399,705.3594124)(465.14875826,705.36942034)
}
}
{
\newrgbcolor{curcolor}{0 0 0}
\pscustom[linestyle=none,fillstyle=solid,fillcolor=curcolor]
{
\newpath
\moveto(472.29203951,708.00942034)
\curveto(472.36203656,707.92940983)(472.39703652,707.80940995)(472.39703951,707.64942034)
\lineto(472.39703951,707.18442034)
\lineto(472.39703951,706.77942034)
\curveto(472.39703652,706.63941112)(472.36203656,706.54441122)(472.29203951,706.49442034)
\curveto(472.23203669,706.44441132)(472.15203677,706.41441135)(472.05203951,706.40442034)
\curveto(471.96203696,706.39441137)(471.86203706,706.38941137)(471.75203951,706.38942034)
\lineto(470.91203951,706.38942034)
\curveto(470.80203812,706.38941137)(470.70203822,706.39441137)(470.61203951,706.40442034)
\curveto(470.53203839,706.41441135)(470.46203846,706.44441132)(470.40203951,706.49442034)
\curveto(470.36203856,706.52441124)(470.33203859,706.57941118)(470.31203951,706.65942034)
\curveto(470.30203862,706.74941101)(470.29203863,706.84441092)(470.28203951,706.94442034)
\lineto(470.28203951,707.27442034)
\curveto(470.29203863,707.38441038)(470.29703862,707.47941028)(470.29703951,707.55942034)
\lineto(470.29703951,707.76942034)
\curveto(470.30703861,707.83940992)(470.32703859,707.89940986)(470.35703951,707.94942034)
\curveto(470.37703854,707.98940977)(470.40203852,708.01940974)(470.43203951,708.03942034)
\lineto(470.55203951,708.09942034)
\curveto(470.57203835,708.09940966)(470.59703832,708.09940966)(470.62703951,708.09942034)
\curveto(470.65703826,708.10940965)(470.68203824,708.11440965)(470.70203951,708.11442034)
\lineto(471.79703951,708.11442034)
\curveto(471.89703702,708.11440965)(471.99203693,708.10940965)(472.08203951,708.09942034)
\curveto(472.17203675,708.08940967)(472.24203668,708.0594097)(472.29203951,708.00942034)
\moveto(472.39703951,698.24442034)
\curveto(472.39703652,698.04441972)(472.39203653,697.87441989)(472.38203951,697.73442034)
\curveto(472.37203655,697.59442017)(472.28203664,697.49942026)(472.11203951,697.44942034)
\curveto(472.05203687,697.42942033)(471.98703693,697.41942034)(471.91703951,697.41942034)
\curveto(471.84703707,697.42942033)(471.77203715,697.43442033)(471.69203951,697.43442034)
\lineto(470.85203951,697.43442034)
\curveto(470.76203816,697.43442033)(470.67203825,697.43942032)(470.58203951,697.44942034)
\curveto(470.50203842,697.4594203)(470.44203848,697.48942027)(470.40203951,697.53942034)
\curveto(470.34203858,697.60942015)(470.30703861,697.69442007)(470.29703951,697.79442034)
\lineto(470.29703951,698.13942034)
\lineto(470.29703951,704.46942034)
\lineto(470.29703951,704.76942034)
\curveto(470.29703862,704.86941289)(470.3170386,704.94941281)(470.35703951,705.00942034)
\curveto(470.4170385,705.07941268)(470.50203842,705.12441264)(470.61203951,705.14442034)
\curveto(470.63203829,705.15441261)(470.65703826,705.15441261)(470.68703951,705.14442034)
\curveto(470.72703819,705.14441262)(470.75703816,705.14941261)(470.77703951,705.15942034)
\lineto(471.52703951,705.15942034)
\lineto(471.72203951,705.15942034)
\curveto(471.80203712,705.16941259)(471.86703705,705.16941259)(471.91703951,705.15942034)
\lineto(472.03703951,705.15942034)
\curveto(472.09703682,705.13941262)(472.15203677,705.12441264)(472.20203951,705.11442034)
\curveto(472.25203667,705.10441266)(472.29203663,705.07441269)(472.32203951,705.02442034)
\curveto(472.36203656,704.97441279)(472.38203654,704.90441286)(472.38203951,704.81442034)
\curveto(472.39203653,704.72441304)(472.39703652,704.62941313)(472.39703951,704.52942034)
\lineto(472.39703951,698.24442034)
}
}
{
\newrgbcolor{curcolor}{0 0 0}
\pscustom[linestyle=none,fillstyle=solid,fillcolor=curcolor]
{
\newpath
\moveto(481.82922701,701.60442034)
\curveto(481.80921848,701.65441611)(481.80421848,701.70941605)(481.81422701,701.76942034)
\curveto(481.82421846,701.82941593)(481.81921847,701.88441588)(481.79922701,701.93442034)
\curveto(481.7892185,701.97441579)(481.7842185,702.01441575)(481.78422701,702.05442034)
\curveto(481.7842185,702.09441567)(481.77921851,702.13441563)(481.76922701,702.17442034)
\lineto(481.70922701,702.44442034)
\curveto(481.6892186,702.53441523)(481.66421862,702.61941514)(481.63422701,702.69942034)
\curveto(481.5842187,702.83941492)(481.53921875,702.96941479)(481.49922701,703.08942034)
\curveto(481.45921883,703.21941454)(481.40421888,703.33941442)(481.33422701,703.44942034)
\curveto(481.26421902,703.5594142)(481.19421909,703.6644141)(481.12422701,703.76442034)
\curveto(481.06421922,703.8644139)(480.99421929,703.9644138)(480.91422701,704.06442034)
\curveto(480.83421945,704.17441359)(480.73421955,704.27441349)(480.61422701,704.36442034)
\curveto(480.50421978,704.4644133)(480.39421989,704.55441321)(480.28422701,704.63442034)
\curveto(479.95422033,704.8644129)(479.57422071,705.04441272)(479.14422701,705.17442034)
\curveto(478.72422156,705.30441246)(478.22422206,705.3644124)(477.64422701,705.35442034)
\curveto(477.57422271,705.34441242)(477.50422278,705.33941242)(477.43422701,705.33942034)
\curveto(477.36422292,705.33941242)(477.289223,705.33441243)(477.20922701,705.32442034)
\curveto(477.05922323,705.28441248)(476.91422337,705.25441251)(476.77422701,705.23442034)
\curveto(476.63422365,705.21441255)(476.49922379,705.17941258)(476.36922701,705.12942034)
\curveto(476.25922403,705.07941268)(476.14922414,705.03441273)(476.03922701,704.99442034)
\curveto(475.92922436,704.95441281)(475.82422446,704.90941285)(475.72422701,704.85942034)
\curveto(475.36422492,704.62941313)(475.05922523,704.37441339)(474.80922701,704.09442034)
\curveto(474.55922573,703.82441394)(474.34422594,703.48441428)(474.16422701,703.07442034)
\curveto(474.11422617,702.95441481)(474.07422621,702.82941493)(474.04422701,702.69942034)
\curveto(474.01422627,702.57941518)(473.97922631,702.45441531)(473.93922701,702.32442034)
\curveto(473.91922637,702.27441549)(473.90922638,702.22441554)(473.90922701,702.17442034)
\curveto(473.90922638,702.13441563)(473.90422638,702.08941567)(473.89422701,702.03942034)
\curveto(473.87422641,701.98941577)(473.86422642,701.93441583)(473.86422701,701.87442034)
\curveto(473.87422641,701.82441594)(473.87422641,701.77441599)(473.86422701,701.72442034)
\lineto(473.86422701,701.61942034)
\curveto(473.84422644,701.5594162)(473.82922646,701.47441629)(473.81922701,701.36442034)
\curveto(473.81922647,701.25441651)(473.82922646,701.16941659)(473.84922701,701.10942034)
\lineto(473.84922701,700.97442034)
\curveto(473.84922644,700.93441683)(473.85422643,700.88941687)(473.86422701,700.83942034)
\curveto(473.8842264,700.759417)(473.89422639,700.67441709)(473.89422701,700.58442034)
\curveto(473.89422639,700.50441726)(473.90422638,700.42441734)(473.92422701,700.34442034)
\curveto(473.94422634,700.29441747)(473.95422633,700.24941751)(473.95422701,700.20942034)
\curveto(473.95422633,700.16941759)(473.96422632,700.12441764)(473.98422701,700.07442034)
\curveto(474.01422627,699.9644178)(474.03922625,699.8594179)(474.05922701,699.75942034)
\curveto(474.0892262,699.6594181)(474.12922616,699.5644182)(474.17922701,699.47442034)
\curveto(474.34922594,699.08441868)(474.55922573,698.74941901)(474.80922701,698.46942034)
\curveto(475.05922523,698.18941957)(475.35922493,697.94441982)(475.70922701,697.73442034)
\curveto(475.81922447,697.67442009)(475.92422436,697.62442014)(476.02422701,697.58442034)
\curveto(476.13422415,697.54442022)(476.24922404,697.50442026)(476.36922701,697.46442034)
\curveto(476.45922383,697.42442034)(476.55422373,697.39442037)(476.65422701,697.37442034)
\curveto(476.75422353,697.35442041)(476.85422343,697.32942043)(476.95422701,697.29942034)
\curveto(477.00422328,697.28942047)(477.04422324,697.28442048)(477.07422701,697.28442034)
\curveto(477.11422317,697.28442048)(477.15422313,697.27942048)(477.19422701,697.26942034)
\curveto(477.24422304,697.24942051)(477.29422299,697.24442052)(477.34422701,697.25442034)
\curveto(477.40422288,697.25442051)(477.45922283,697.24942051)(477.50922701,697.23942034)
\lineto(477.65922701,697.23942034)
\curveto(477.71922257,697.21942054)(477.80422248,697.21442055)(477.91422701,697.22442034)
\curveto(478.02422226,697.22442054)(478.10422218,697.22942053)(478.15422701,697.23942034)
\curveto(478.1842221,697.23942052)(478.21422207,697.24442052)(478.24422701,697.25442034)
\lineto(478.34922701,697.25442034)
\curveto(478.39922189,697.2644205)(478.45422183,697.26942049)(478.51422701,697.26942034)
\curveto(478.57422171,697.26942049)(478.62922166,697.27942048)(478.67922701,697.29942034)
\curveto(478.80922148,697.32942043)(478.93422135,697.3594204)(479.05422701,697.38942034)
\curveto(479.1842211,697.40942035)(479.30922098,697.44442032)(479.42922701,697.49442034)
\curveto(479.90922038,697.69442007)(480.31921997,697.94441982)(480.65922701,698.24442034)
\curveto(480.99921929,698.54441922)(481.27421901,698.93441883)(481.48422701,699.41442034)
\curveto(481.53421875,699.51441825)(481.57421871,699.61941814)(481.60422701,699.72942034)
\curveto(481.63421865,699.84941791)(481.66921862,699.9644178)(481.70922701,700.07442034)
\curveto(481.71921857,700.14441762)(481.72921856,700.20941755)(481.73922701,700.26942034)
\curveto(481.74921854,700.32941743)(481.76421852,700.39441737)(481.78422701,700.46442034)
\curveto(481.80421848,700.54441722)(481.80921848,700.62441714)(481.79922701,700.70442034)
\curveto(481.79921849,700.78441698)(481.80921848,700.8644169)(481.82922701,700.94442034)
\lineto(481.82922701,701.09442034)
\curveto(481.84921844,701.15441661)(481.85921843,701.23941652)(481.85922701,701.34942034)
\curveto(481.85921843,701.4594163)(481.84921844,701.54441622)(481.82922701,701.60442034)
\moveto(479.72922701,701.06442034)
\curveto(479.71922057,701.01441675)(479.71422057,700.9644168)(479.71422701,700.91442034)
\lineto(479.71422701,700.77942034)
\curveto(479.70422058,700.73941702)(479.69922059,700.69941706)(479.69922701,700.65942034)
\curveto(479.69922059,700.62941713)(479.69422059,700.59441717)(479.68422701,700.55442034)
\curveto(479.65422063,700.44441732)(479.62922066,700.33941742)(479.60922701,700.23942034)
\curveto(479.5892207,700.13941762)(479.55922073,700.03941772)(479.51922701,699.93942034)
\curveto(479.40922088,699.68941807)(479.27422101,699.47941828)(479.11422701,699.30942034)
\curveto(478.95422133,699.13941862)(478.74422154,699.00441876)(478.48422701,698.90442034)
\curveto(478.41422187,698.87441889)(478.33922195,698.85441891)(478.25922701,698.84442034)
\curveto(478.17922211,698.83441893)(478.09922219,698.81941894)(478.01922701,698.79942034)
\lineto(477.89922701,698.79942034)
\curveto(477.85922243,698.78941897)(477.81422247,698.78441898)(477.76422701,698.78442034)
\lineto(477.64422701,698.81442034)
\curveto(477.60422268,698.82441894)(477.56922272,698.82441894)(477.53922701,698.81442034)
\curveto(477.50922278,698.81441895)(477.47422281,698.81941894)(477.43422701,698.82942034)
\curveto(477.34422294,698.84941891)(477.25422303,698.87441889)(477.16422701,698.90442034)
\curveto(477.0842232,698.93441883)(477.00922328,698.97441879)(476.93922701,699.02442034)
\curveto(476.6892236,699.17441859)(476.50422378,699.33941842)(476.38422701,699.51942034)
\curveto(476.27422401,699.70941805)(476.16922412,699.95441781)(476.06922701,700.25442034)
\curveto(476.04922424,700.33441743)(476.03422425,700.40941735)(476.02422701,700.47942034)
\curveto(476.01422427,700.5594172)(475.99922429,700.63941712)(475.97922701,700.71942034)
\lineto(475.97922701,700.85442034)
\curveto(475.95922433,700.92441684)(475.94422434,701.02941673)(475.93422701,701.16942034)
\curveto(475.93422435,701.30941645)(475.94422434,701.41441635)(475.96422701,701.48442034)
\lineto(475.96422701,701.63442034)
\curveto(475.96422432,701.68441608)(475.96922432,701.73441603)(475.97922701,701.78442034)
\curveto(475.99922429,701.89441587)(476.01422427,702.00441576)(476.02422701,702.11442034)
\curveto(476.04422424,702.22441554)(476.06922422,702.32941543)(476.09922701,702.42942034)
\curveto(476.1892241,702.69941506)(476.30922398,702.93441483)(476.45922701,703.13442034)
\curveto(476.61922367,703.34441442)(476.82422346,703.50441426)(477.07422701,703.61442034)
\curveto(477.12422316,703.64441412)(477.17922311,703.6644141)(477.23922701,703.67442034)
\lineto(477.44922701,703.73442034)
\curveto(477.47922281,703.74441402)(477.51422277,703.74441402)(477.55422701,703.73442034)
\curveto(477.59422269,703.73441403)(477.62922266,703.74441402)(477.65922701,703.76442034)
\lineto(477.92922701,703.76442034)
\curveto(478.01922227,703.77441399)(478.10422218,703.76941399)(478.18422701,703.74942034)
\curveto(478.25422203,703.72941403)(478.31922197,703.70941405)(478.37922701,703.68942034)
\curveto(478.43922185,703.67941408)(478.49922179,703.6644141)(478.55922701,703.64442034)
\curveto(478.80922148,703.53441423)(479.00922128,703.38441438)(479.15922701,703.19442034)
\curveto(479.30922098,703.01441475)(479.43922085,702.79441497)(479.54922701,702.53442034)
\curveto(479.57922071,702.45441531)(479.59922069,702.36941539)(479.60922701,702.27942034)
\lineto(479.66922701,702.03942034)
\curveto(479.67922061,702.01941574)(479.6842206,701.98941577)(479.68422701,701.94942034)
\curveto(479.69422059,701.89941586)(479.69922059,701.84441592)(479.69922701,701.78442034)
\curveto(479.69922059,701.72441604)(479.70922058,701.66941609)(479.72922701,701.61942034)
\lineto(479.72922701,701.49942034)
\curveto(479.73922055,701.44941631)(479.74422054,701.37441639)(479.74422701,701.27442034)
\curveto(479.74422054,701.18441658)(479.73922055,701.11441665)(479.72922701,701.06442034)
\moveto(478.49922701,708.23442034)
\lineto(479.56422701,708.23442034)
\curveto(479.64422064,708.23440953)(479.73922055,708.23440953)(479.84922701,708.23442034)
\curveto(479.95922033,708.23440953)(480.03922025,708.21940954)(480.08922701,708.18942034)
\curveto(480.10922018,708.17940958)(480.11922017,708.1644096)(480.11922701,708.14442034)
\curveto(480.12922016,708.13440963)(480.14422014,708.12440964)(480.16422701,708.11442034)
\curveto(480.17422011,707.99440977)(480.12422016,707.88940987)(480.01422701,707.79942034)
\curveto(479.91422037,707.70941005)(479.82922046,707.62941013)(479.75922701,707.55942034)
\curveto(479.67922061,707.48941027)(479.59922069,707.41441035)(479.51922701,707.33442034)
\curveto(479.44922084,707.2644105)(479.37422091,707.19941056)(479.29422701,707.13942034)
\curveto(479.25422103,707.10941065)(479.21922107,707.07441069)(479.18922701,707.03442034)
\curveto(479.16922112,707.00441076)(479.13922115,706.97941078)(479.09922701,706.95942034)
\curveto(479.07922121,706.92941083)(479.05422123,706.90441086)(479.02422701,706.88442034)
\lineto(478.87422701,706.73442034)
\lineto(478.72422701,706.61442034)
\lineto(478.67922701,706.56942034)
\curveto(478.67922161,706.5594112)(478.66922162,706.54441122)(478.64922701,706.52442034)
\curveto(478.56922172,706.4644113)(478.4892218,706.39941136)(478.40922701,706.32942034)
\curveto(478.33922195,706.2594115)(478.24922204,706.20441156)(478.13922701,706.16442034)
\curveto(478.09922219,706.15441161)(478.05922223,706.14941161)(478.01922701,706.14942034)
\curveto(477.9892223,706.14941161)(477.94922234,706.14441162)(477.89922701,706.13442034)
\curveto(477.86922242,706.12441164)(477.82922246,706.11941164)(477.77922701,706.11942034)
\curveto(477.72922256,706.12941163)(477.6842226,706.13441163)(477.64422701,706.13442034)
\lineto(477.29922701,706.13442034)
\curveto(477.17922311,706.13441163)(477.0892232,706.1594116)(477.02922701,706.20942034)
\curveto(476.96922332,706.24941151)(476.95422333,706.31941144)(476.98422701,706.41942034)
\curveto(477.00422328,706.49941126)(477.03922325,706.56941119)(477.08922701,706.62942034)
\curveto(477.13922315,706.69941106)(477.1842231,706.76941099)(477.22422701,706.83942034)
\curveto(477.32422296,706.97941078)(477.41922287,707.11441065)(477.50922701,707.24442034)
\curveto(477.59922269,707.37441039)(477.6892226,707.50941025)(477.77922701,707.64942034)
\curveto(477.82922246,707.72941003)(477.87922241,707.81440995)(477.92922701,707.90442034)
\curveto(477.9892223,707.99440977)(478.05422223,708.0644097)(478.12422701,708.11442034)
\curveto(478.16422212,708.14440962)(478.23422205,708.17940958)(478.33422701,708.21942034)
\curveto(478.35422193,708.22940953)(478.37922191,708.22940953)(478.40922701,708.21942034)
\curveto(478.44922184,708.21940954)(478.47922181,708.22440954)(478.49922701,708.23442034)
}
}
{
\newrgbcolor{curcolor}{0 0 0}
\pscustom[linestyle=none,fillstyle=solid,fillcolor=curcolor]
{
\newpath
\moveto(487.65414888,705.35442034)
\curveto(488.25414308,705.37441239)(488.75414258,705.28941247)(489.15414888,705.09942034)
\curveto(489.55414178,704.90941285)(489.86914146,704.62941313)(490.09914888,704.25942034)
\curveto(490.16914116,704.14941361)(490.22414111,704.02941373)(490.26414888,703.89942034)
\curveto(490.30414103,703.77941398)(490.34414099,703.65441411)(490.38414888,703.52442034)
\curveto(490.40414093,703.44441432)(490.41414092,703.36941439)(490.41414888,703.29942034)
\curveto(490.42414091,703.22941453)(490.43914089,703.1594146)(490.45914888,703.08942034)
\curveto(490.45914087,703.02941473)(490.46414087,702.98941477)(490.47414888,702.96942034)
\curveto(490.49414084,702.82941493)(490.50414083,702.68441508)(490.50414888,702.53442034)
\lineto(490.50414888,702.09942034)
\lineto(490.50414888,700.76442034)
\lineto(490.50414888,698.33442034)
\curveto(490.50414083,698.14441962)(490.49914083,697.9594198)(490.48914888,697.77942034)
\curveto(490.48914084,697.60942015)(490.41914091,697.49942026)(490.27914888,697.44942034)
\curveto(490.21914111,697.42942033)(490.14914118,697.41942034)(490.06914888,697.41942034)
\lineto(489.82914888,697.41942034)
\lineto(489.01914888,697.41942034)
\curveto(488.89914243,697.41942034)(488.78914254,697.42442034)(488.68914888,697.43442034)
\curveto(488.59914273,697.45442031)(488.5291428,697.49942026)(488.47914888,697.56942034)
\curveto(488.43914289,697.62942013)(488.41414292,697.70442006)(488.40414888,697.79442034)
\lineto(488.40414888,698.10942034)
\lineto(488.40414888,699.15942034)
\lineto(488.40414888,701.39442034)
\curveto(488.40414293,701.764416)(488.38914294,702.10441566)(488.35914888,702.41442034)
\curveto(488.329143,702.73441503)(488.23914309,703.00441476)(488.08914888,703.22442034)
\curveto(487.94914338,703.42441434)(487.74414359,703.5644142)(487.47414888,703.64442034)
\curveto(487.42414391,703.6644141)(487.36914396,703.67441409)(487.30914888,703.67442034)
\curveto(487.25914407,703.67441409)(487.20414413,703.68441408)(487.14414888,703.70442034)
\curveto(487.09414424,703.71441405)(487.0291443,703.71441405)(486.94914888,703.70442034)
\curveto(486.87914445,703.70441406)(486.82414451,703.69941406)(486.78414888,703.68942034)
\curveto(486.74414459,703.67941408)(486.70914462,703.67441409)(486.67914888,703.67442034)
\curveto(486.64914468,703.67441409)(486.61914471,703.66941409)(486.58914888,703.65942034)
\curveto(486.35914497,703.59941416)(486.17414516,703.51941424)(486.03414888,703.41942034)
\curveto(485.71414562,703.18941457)(485.52414581,702.85441491)(485.46414888,702.41442034)
\curveto(485.40414593,701.97441579)(485.37414596,701.47941628)(485.37414888,700.92942034)
\lineto(485.37414888,699.05442034)
\lineto(485.37414888,698.13942034)
\lineto(485.37414888,697.86942034)
\curveto(485.37414596,697.77941998)(485.35914597,697.70442006)(485.32914888,697.64442034)
\curveto(485.27914605,697.53442023)(485.19914613,697.46942029)(485.08914888,697.44942034)
\curveto(484.97914635,697.42942033)(484.84414649,697.41942034)(484.68414888,697.41942034)
\lineto(483.93414888,697.41942034)
\curveto(483.82414751,697.41942034)(483.71414762,697.42442034)(483.60414888,697.43442034)
\curveto(483.49414784,697.44442032)(483.41414792,697.47942028)(483.36414888,697.53942034)
\curveto(483.29414804,697.62942013)(483.25914807,697.75942)(483.25914888,697.92942034)
\curveto(483.26914806,698.09941966)(483.27414806,698.2594195)(483.27414888,698.40942034)
\lineto(483.27414888,700.44942034)
\lineto(483.27414888,703.74942034)
\lineto(483.27414888,704.51442034)
\lineto(483.27414888,704.81442034)
\curveto(483.28414805,704.90441286)(483.31414802,704.97941278)(483.36414888,705.03942034)
\curveto(483.38414795,705.06941269)(483.41414792,705.08941267)(483.45414888,705.09942034)
\curveto(483.50414783,705.11941264)(483.55414778,705.13441263)(483.60414888,705.14442034)
\lineto(483.67914888,705.14442034)
\curveto(483.7291476,705.15441261)(483.77914755,705.1594126)(483.82914888,705.15942034)
\lineto(483.99414888,705.15942034)
\lineto(484.62414888,705.15942034)
\curveto(484.70414663,705.1594126)(484.77914655,705.15441261)(484.84914888,705.14442034)
\curveto(484.9291464,705.14441262)(484.99914633,705.13441263)(485.05914888,705.11442034)
\curveto(485.1291462,705.08441268)(485.17414616,705.03941272)(485.19414888,704.97942034)
\curveto(485.22414611,704.91941284)(485.24914608,704.84941291)(485.26914888,704.76942034)
\curveto(485.27914605,704.72941303)(485.27914605,704.69441307)(485.26914888,704.66442034)
\curveto(485.26914606,704.63441313)(485.27914605,704.60441316)(485.29914888,704.57442034)
\curveto(485.31914601,704.52441324)(485.334146,704.49441327)(485.34414888,704.48442034)
\curveto(485.36414597,704.47441329)(485.38914594,704.4594133)(485.41914888,704.43942034)
\curveto(485.5291458,704.42941333)(485.61914571,704.4644133)(485.68914888,704.54442034)
\curveto(485.75914557,704.63441313)(485.8341455,704.70441306)(485.91414888,704.75442034)
\curveto(486.18414515,704.95441281)(486.48414485,705.11441265)(486.81414888,705.23442034)
\curveto(486.90414443,705.2644125)(486.99414434,705.28441248)(487.08414888,705.29442034)
\curveto(487.18414415,705.30441246)(487.28914404,705.31941244)(487.39914888,705.33942034)
\curveto(487.4291439,705.34941241)(487.47414386,705.34941241)(487.53414888,705.33942034)
\curveto(487.59414374,705.33941242)(487.6341437,705.34441242)(487.65414888,705.35442034)
}
}
{
\newrgbcolor{curcolor}{0 0 0}
\pscustom[linestyle=none,fillstyle=solid,fillcolor=curcolor]
{
}
}
{
\newrgbcolor{curcolor}{0 0 0}
\pscustom[linestyle=none,fillstyle=solid,fillcolor=curcolor]
{
\newpath
\moveto(503.88555513,698.27442034)
\lineto(503.88555513,697.85442034)
\curveto(503.88554676,697.72442004)(503.85554679,697.61942014)(503.79555513,697.53942034)
\curveto(503.7455469,697.48942027)(503.68054697,697.45442031)(503.60055513,697.43442034)
\curveto(503.52054713,697.42442034)(503.43054722,697.41942034)(503.33055513,697.41942034)
\lineto(502.50555513,697.41942034)
\lineto(502.22055513,697.41942034)
\curveto(502.14054851,697.42942033)(502.07554857,697.45442031)(502.02555513,697.49442034)
\curveto(501.95554869,697.54442022)(501.91554873,697.60942015)(501.90555513,697.68942034)
\curveto(501.89554875,697.76941999)(501.87554877,697.84941991)(501.84555513,697.92942034)
\curveto(501.82554882,697.94941981)(501.80554884,697.9644198)(501.78555513,697.97442034)
\curveto(501.77554887,697.99441977)(501.76054889,698.01441975)(501.74055513,698.03442034)
\curveto(501.63054902,698.03441973)(501.5505491,698.00941975)(501.50055513,697.95942034)
\lineto(501.35055513,697.80942034)
\curveto(501.28054937,697.75942)(501.21554943,697.71442005)(501.15555513,697.67442034)
\curveto(501.09554955,697.64442012)(501.03054962,697.60442016)(500.96055513,697.55442034)
\curveto(500.92054973,697.53442023)(500.87554977,697.51442025)(500.82555513,697.49442034)
\curveto(500.78554986,697.47442029)(500.74054991,697.45442031)(500.69055513,697.43442034)
\curveto(500.5505501,697.38442038)(500.40055025,697.33942042)(500.24055513,697.29942034)
\curveto(500.19055046,697.27942048)(500.1455505,697.26942049)(500.10555513,697.26942034)
\curveto(500.06555058,697.26942049)(500.02555062,697.2644205)(499.98555513,697.25442034)
\lineto(499.85055513,697.25442034)
\curveto(499.82055083,697.24442052)(499.78055087,697.23942052)(499.73055513,697.23942034)
\lineto(499.59555513,697.23942034)
\curveto(499.53555111,697.21942054)(499.4455512,697.21442055)(499.32555513,697.22442034)
\curveto(499.20555144,697.22442054)(499.12055153,697.23442053)(499.07055513,697.25442034)
\curveto(499.00055165,697.27442049)(498.93555171,697.28442048)(498.87555513,697.28442034)
\curveto(498.82555182,697.27442049)(498.77055188,697.27942048)(498.71055513,697.29942034)
\lineto(498.35055513,697.41942034)
\curveto(498.24055241,697.44942031)(498.13055252,697.48942027)(498.02055513,697.53942034)
\curveto(497.67055298,697.68942007)(497.35555329,697.91941984)(497.07555513,698.22942034)
\curveto(496.80555384,698.54941921)(496.59055406,698.88441888)(496.43055513,699.23442034)
\curveto(496.38055427,699.34441842)(496.34055431,699.44941831)(496.31055513,699.54942034)
\curveto(496.28055437,699.6594181)(496.2455544,699.76941799)(496.20555513,699.87942034)
\curveto(496.19555445,699.91941784)(496.19055446,699.95441781)(496.19055513,699.98442034)
\curveto(496.19055446,700.02441774)(496.18055447,700.06941769)(496.16055513,700.11942034)
\curveto(496.14055451,700.19941756)(496.12055453,700.28441748)(496.10055513,700.37442034)
\curveto(496.09055456,700.47441729)(496.07555457,700.57441719)(496.05555513,700.67442034)
\curveto(496.0455546,700.70441706)(496.04055461,700.73941702)(496.04055513,700.77942034)
\curveto(496.0505546,700.81941694)(496.0505546,700.85441691)(496.04055513,700.88442034)
\lineto(496.04055513,701.01942034)
\curveto(496.04055461,701.06941669)(496.03555461,701.11941664)(496.02555513,701.16942034)
\curveto(496.01555463,701.21941654)(496.01055464,701.27441649)(496.01055513,701.33442034)
\curveto(496.01055464,701.40441636)(496.01555463,701.4594163)(496.02555513,701.49942034)
\curveto(496.03555461,701.54941621)(496.04055461,701.59441617)(496.04055513,701.63442034)
\lineto(496.04055513,701.78442034)
\curveto(496.0505546,701.83441593)(496.0505546,701.87941588)(496.04055513,701.91942034)
\curveto(496.04055461,701.96941579)(496.0505546,702.01941574)(496.07055513,702.06942034)
\curveto(496.09055456,702.17941558)(496.10555454,702.28441548)(496.11555513,702.38442034)
\curveto(496.13555451,702.48441528)(496.16055449,702.58441518)(496.19055513,702.68442034)
\curveto(496.23055442,702.80441496)(496.26555438,702.91941484)(496.29555513,703.02942034)
\curveto(496.32555432,703.13941462)(496.36555428,703.24941451)(496.41555513,703.35942034)
\curveto(496.55555409,703.6594141)(496.73055392,703.94441382)(496.94055513,704.21442034)
\curveto(496.96055369,704.24441352)(496.98555366,704.26941349)(497.01555513,704.28942034)
\curveto(497.05555359,704.31941344)(497.08555356,704.34941341)(497.10555513,704.37942034)
\curveto(497.1455535,704.42941333)(497.18555346,704.47441329)(497.22555513,704.51442034)
\curveto(497.26555338,704.55441321)(497.31055334,704.59441317)(497.36055513,704.63442034)
\curveto(497.40055325,704.65441311)(497.43555321,704.67941308)(497.46555513,704.70942034)
\curveto(497.49555315,704.74941301)(497.53055312,704.77941298)(497.57055513,704.79942034)
\curveto(497.82055283,704.96941279)(498.11055254,705.10941265)(498.44055513,705.21942034)
\curveto(498.51055214,705.23941252)(498.58055207,705.25441251)(498.65055513,705.26442034)
\curveto(498.73055192,705.27441249)(498.81055184,705.28941247)(498.89055513,705.30942034)
\curveto(498.96055169,705.32941243)(499.0505516,705.33941242)(499.16055513,705.33942034)
\curveto(499.27055138,705.34941241)(499.38055127,705.35441241)(499.49055513,705.35442034)
\curveto(499.60055105,705.35441241)(499.70555094,705.34941241)(499.80555513,705.33942034)
\curveto(499.91555073,705.32941243)(500.00555064,705.31441245)(500.07555513,705.29442034)
\curveto(500.22555042,705.24441252)(500.37055028,705.19941256)(500.51055513,705.15942034)
\curveto(500.65055,705.11941264)(500.78054987,705.0644127)(500.90055513,704.99442034)
\curveto(500.97054968,704.94441282)(501.03554961,704.89441287)(501.09555513,704.84442034)
\curveto(501.15554949,704.80441296)(501.22054943,704.759413)(501.29055513,704.70942034)
\curveto(501.33054932,704.67941308)(501.38554926,704.63941312)(501.45555513,704.58942034)
\curveto(501.53554911,704.53941322)(501.61054904,704.53941322)(501.68055513,704.58942034)
\curveto(501.72054893,704.60941315)(501.74054891,704.64441312)(501.74055513,704.69442034)
\curveto(501.74054891,704.74441302)(501.7505489,704.79441297)(501.77055513,704.84442034)
\lineto(501.77055513,704.99442034)
\curveto(501.78054887,705.02441274)(501.78554886,705.0594127)(501.78555513,705.09942034)
\lineto(501.78555513,705.21942034)
\lineto(501.78555513,707.25942034)
\curveto(501.78554886,707.36941039)(501.78054887,707.48941027)(501.77055513,707.61942034)
\curveto(501.77054888,707.75941)(501.79554885,707.8644099)(501.84555513,707.93442034)
\curveto(501.88554876,708.01440975)(501.96054869,708.0644097)(502.07055513,708.08442034)
\curveto(502.09054856,708.09440967)(502.11054854,708.09440967)(502.13055513,708.08442034)
\curveto(502.1505485,708.08440968)(502.17054848,708.08940967)(502.19055513,708.09942034)
\lineto(503.25555513,708.09942034)
\curveto(503.37554727,708.09940966)(503.48554716,708.09440967)(503.58555513,708.08442034)
\curveto(503.68554696,708.07440969)(503.76054689,708.03440973)(503.81055513,707.96442034)
\curveto(503.86054679,707.88440988)(503.88554676,707.77940998)(503.88555513,707.64942034)
\lineto(503.88555513,707.28942034)
\lineto(503.88555513,698.27442034)
\moveto(501.84555513,701.21442034)
\curveto(501.85554879,701.25441651)(501.85554879,701.29441647)(501.84555513,701.33442034)
\lineto(501.84555513,701.46942034)
\curveto(501.8455488,701.56941619)(501.84054881,701.66941609)(501.83055513,701.76942034)
\curveto(501.82054883,701.86941589)(501.80554884,701.9594158)(501.78555513,702.03942034)
\curveto(501.76554888,702.14941561)(501.7455489,702.24941551)(501.72555513,702.33942034)
\curveto(501.71554893,702.42941533)(501.69054896,702.51441525)(501.65055513,702.59442034)
\curveto(501.51054914,702.95441481)(501.30554934,703.23941452)(501.03555513,703.44942034)
\curveto(500.77554987,703.6594141)(500.39555025,703.764414)(499.89555513,703.76442034)
\curveto(499.83555081,703.764414)(499.75555089,703.75441401)(499.65555513,703.73442034)
\curveto(499.57555107,703.71441405)(499.50055115,703.69441407)(499.43055513,703.67442034)
\curveto(499.37055128,703.6644141)(499.31055134,703.64441412)(499.25055513,703.61442034)
\curveto(498.98055167,703.50441426)(498.77055188,703.33441443)(498.62055513,703.10442034)
\curveto(498.47055218,702.87441489)(498.3505523,702.61441515)(498.26055513,702.32442034)
\curveto(498.23055242,702.22441554)(498.21055244,702.12441564)(498.20055513,702.02442034)
\curveto(498.19055246,701.92441584)(498.17055248,701.81941594)(498.14055513,701.70942034)
\lineto(498.14055513,701.49942034)
\curveto(498.12055253,701.40941635)(498.11555253,701.28441648)(498.12555513,701.12442034)
\curveto(498.13555251,700.97441679)(498.1505525,700.8644169)(498.17055513,700.79442034)
\lineto(498.17055513,700.70442034)
\curveto(498.18055247,700.68441708)(498.18555246,700.6644171)(498.18555513,700.64442034)
\curveto(498.20555244,700.5644172)(498.22055243,700.48941727)(498.23055513,700.41942034)
\curveto(498.2505524,700.34941741)(498.27055238,700.27441749)(498.29055513,700.19442034)
\curveto(498.46055219,699.67441809)(498.7505519,699.28941847)(499.16055513,699.03942034)
\curveto(499.29055136,698.94941881)(499.47055118,698.87941888)(499.70055513,698.82942034)
\curveto(499.74055091,698.81941894)(499.80055085,698.81441895)(499.88055513,698.81442034)
\curveto(499.91055074,698.80441896)(499.95555069,698.79441897)(500.01555513,698.78442034)
\curveto(500.08555056,698.78441898)(500.14055051,698.78941897)(500.18055513,698.79942034)
\curveto(500.26055039,698.81941894)(500.34055031,698.83441893)(500.42055513,698.84442034)
\curveto(500.50055015,698.85441891)(500.58055007,698.87441889)(500.66055513,698.90442034)
\curveto(500.91054974,699.01441875)(501.11054954,699.15441861)(501.26055513,699.32442034)
\curveto(501.41054924,699.49441827)(501.54054911,699.70941805)(501.65055513,699.96942034)
\curveto(501.69054896,700.0594177)(501.72054893,700.14941761)(501.74055513,700.23942034)
\curveto(501.76054889,700.33941742)(501.78054887,700.44441732)(501.80055513,700.55442034)
\curveto(501.81054884,700.60441716)(501.81054884,700.64941711)(501.80055513,700.68942034)
\curveto(501.80054885,700.73941702)(501.81054884,700.78941697)(501.83055513,700.83942034)
\curveto(501.84054881,700.86941689)(501.8455488,700.90441686)(501.84555513,700.94442034)
\lineto(501.84555513,701.07942034)
\lineto(501.84555513,701.21442034)
}
}
{
\newrgbcolor{curcolor}{0 0 0}
\pscustom[linestyle=none,fillstyle=solid,fillcolor=curcolor]
{
\newpath
\moveto(512.83047701,701.36442034)
\curveto(512.85046884,701.28441648)(512.85046884,701.19441657)(512.83047701,701.09442034)
\curveto(512.81046888,700.99441677)(512.77546892,700.92941683)(512.72547701,700.89942034)
\curveto(512.67546902,700.8594169)(512.60046909,700.82941693)(512.50047701,700.80942034)
\curveto(512.41046928,700.79941696)(512.30546939,700.78941697)(512.18547701,700.77942034)
\lineto(511.84047701,700.77942034)
\curveto(511.73046996,700.78941697)(511.63047006,700.79441697)(511.54047701,700.79442034)
\lineto(507.88047701,700.79442034)
\lineto(507.67047701,700.79442034)
\curveto(507.61047408,700.79441697)(507.55547414,700.78441698)(507.50547701,700.76442034)
\curveto(507.42547427,700.72441704)(507.37547432,700.68441708)(507.35547701,700.64442034)
\curveto(507.33547436,700.62441714)(507.31547438,700.58441718)(507.29547701,700.52442034)
\curveto(507.27547442,700.47441729)(507.27047442,700.42441734)(507.28047701,700.37442034)
\curveto(507.30047439,700.31441745)(507.31047438,700.25441751)(507.31047701,700.19442034)
\curveto(507.32047437,700.14441762)(507.33547436,700.08941767)(507.35547701,700.02942034)
\curveto(507.43547426,699.78941797)(507.53047416,699.58941817)(507.64047701,699.42942034)
\curveto(507.76047393,699.27941848)(507.92047377,699.14441862)(508.12047701,699.02442034)
\curveto(508.20047349,698.97441879)(508.28047341,698.93941882)(508.36047701,698.91942034)
\curveto(508.45047324,698.90941885)(508.54047315,698.88941887)(508.63047701,698.85942034)
\curveto(508.71047298,698.83941892)(508.82047287,698.82441894)(508.96047701,698.81442034)
\curveto(509.10047259,698.80441896)(509.22047247,698.80941895)(509.32047701,698.82942034)
\lineto(509.45547701,698.82942034)
\curveto(509.55547214,698.84941891)(509.64547205,698.86941889)(509.72547701,698.88942034)
\curveto(509.81547188,698.91941884)(509.90047179,698.94941881)(509.98047701,698.97942034)
\curveto(510.08047161,699.02941873)(510.1904715,699.09441867)(510.31047701,699.17442034)
\curveto(510.44047125,699.25441851)(510.53547116,699.33441843)(510.59547701,699.41442034)
\curveto(510.64547105,699.48441828)(510.695471,699.54941821)(510.74547701,699.60942034)
\curveto(510.80547089,699.67941808)(510.87547082,699.72941803)(510.95547701,699.75942034)
\curveto(511.05547064,699.80941795)(511.18047051,699.82941793)(511.33047701,699.81942034)
\lineto(511.76547701,699.81942034)
\lineto(511.94547701,699.81942034)
\curveto(512.01546968,699.82941793)(512.07546962,699.82441794)(512.12547701,699.80442034)
\lineto(512.27547701,699.80442034)
\curveto(512.37546932,699.78441798)(512.44546925,699.759418)(512.48547701,699.72942034)
\curveto(512.52546917,699.70941805)(512.54546915,699.6644181)(512.54547701,699.59442034)
\curveto(512.55546914,699.52441824)(512.55046914,699.4644183)(512.53047701,699.41442034)
\curveto(512.48046921,699.27441849)(512.42546927,699.14941861)(512.36547701,699.03942034)
\curveto(512.30546939,698.92941883)(512.23546946,698.81941894)(512.15547701,698.70942034)
\curveto(511.93546976,698.37941938)(511.68547001,698.11441965)(511.40547701,697.91442034)
\curveto(511.12547057,697.71442005)(510.77547092,697.54442022)(510.35547701,697.40442034)
\curveto(510.24547145,697.3644204)(510.13547156,697.33942042)(510.02547701,697.32942034)
\curveto(509.91547178,697.31942044)(509.80047189,697.29942046)(509.68047701,697.26942034)
\curveto(509.64047205,697.2594205)(509.5954721,697.2594205)(509.54547701,697.26942034)
\curveto(509.50547219,697.26942049)(509.46547223,697.2644205)(509.42547701,697.25442034)
\lineto(509.26047701,697.25442034)
\curveto(509.21047248,697.23442053)(509.15047254,697.22942053)(509.08047701,697.23942034)
\curveto(509.02047267,697.23942052)(508.96547273,697.24442052)(508.91547701,697.25442034)
\curveto(508.83547286,697.2644205)(508.76547293,697.2644205)(508.70547701,697.25442034)
\curveto(508.64547305,697.24442052)(508.58047311,697.24942051)(508.51047701,697.26942034)
\curveto(508.46047323,697.28942047)(508.40547329,697.29942046)(508.34547701,697.29942034)
\curveto(508.28547341,697.29942046)(508.23047346,697.30942045)(508.18047701,697.32942034)
\curveto(508.07047362,697.34942041)(507.96047373,697.37442039)(507.85047701,697.40442034)
\curveto(507.74047395,697.42442034)(507.64047405,697.4594203)(507.55047701,697.50942034)
\curveto(507.44047425,697.54942021)(507.33547436,697.58442018)(507.23547701,697.61442034)
\curveto(507.14547455,697.65442011)(507.06047463,697.69942006)(506.98047701,697.74942034)
\curveto(506.66047503,697.94941981)(506.37547532,698.17941958)(506.12547701,698.43942034)
\curveto(505.87547582,698.70941905)(505.67047602,699.01941874)(505.51047701,699.36942034)
\curveto(505.46047623,699.47941828)(505.42047627,699.58941817)(505.39047701,699.69942034)
\curveto(505.36047633,699.81941794)(505.32047637,699.93941782)(505.27047701,700.05942034)
\curveto(505.26047643,700.09941766)(505.25547644,700.13441763)(505.25547701,700.16442034)
\curveto(505.25547644,700.20441756)(505.25047644,700.24441752)(505.24047701,700.28442034)
\curveto(505.20047649,700.40441736)(505.17547652,700.53441723)(505.16547701,700.67442034)
\lineto(505.13547701,701.09442034)
\curveto(505.13547656,701.14441662)(505.13047656,701.19941656)(505.12047701,701.25942034)
\curveto(505.12047657,701.31941644)(505.12547657,701.37441639)(505.13547701,701.42442034)
\lineto(505.13547701,701.60442034)
\lineto(505.18047701,701.96442034)
\curveto(505.22047647,702.13441563)(505.25547644,702.29941546)(505.28547701,702.45942034)
\curveto(505.31547638,702.61941514)(505.36047633,702.76941499)(505.42047701,702.90942034)
\curveto(505.85047584,703.94941381)(506.58047511,704.68441308)(507.61047701,705.11442034)
\curveto(507.75047394,705.17441259)(507.8904738,705.21441255)(508.03047701,705.23442034)
\curveto(508.18047351,705.2644125)(508.33547336,705.29941246)(508.49547701,705.33942034)
\curveto(508.57547312,705.34941241)(508.65047304,705.35441241)(508.72047701,705.35442034)
\curveto(508.7904729,705.35441241)(508.86547283,705.3594124)(508.94547701,705.36942034)
\curveto(509.45547224,705.37941238)(509.8904718,705.31941244)(510.25047701,705.18942034)
\curveto(510.62047107,705.06941269)(510.95047074,704.90941285)(511.24047701,704.70942034)
\curveto(511.33047036,704.64941311)(511.42047027,704.57941318)(511.51047701,704.49942034)
\curveto(511.60047009,704.42941333)(511.68047001,704.35441341)(511.75047701,704.27442034)
\curveto(511.78046991,704.22441354)(511.82046987,704.18441358)(511.87047701,704.15442034)
\curveto(511.95046974,704.04441372)(512.02546967,703.92941383)(512.09547701,703.80942034)
\curveto(512.16546953,703.69941406)(512.24046945,703.58441418)(512.32047701,703.46442034)
\curveto(512.37046932,703.37441439)(512.41046928,703.27941448)(512.44047701,703.17942034)
\curveto(512.48046921,703.08941467)(512.52046917,702.98941477)(512.56047701,702.87942034)
\curveto(512.61046908,702.74941501)(512.65046904,702.61441515)(512.68047701,702.47442034)
\curveto(512.71046898,702.33441543)(512.74546895,702.19441557)(512.78547701,702.05442034)
\curveto(512.80546889,701.97441579)(512.81046888,701.88441588)(512.80047701,701.78442034)
\curveto(512.80046889,701.69441607)(512.81046888,701.60941615)(512.83047701,701.52942034)
\lineto(512.83047701,701.36442034)
\moveto(510.58047701,702.24942034)
\curveto(510.65047104,702.34941541)(510.65547104,702.46941529)(510.59547701,702.60942034)
\curveto(510.54547115,702.759415)(510.50547119,702.86941489)(510.47547701,702.93942034)
\curveto(510.33547136,703.20941455)(510.15047154,703.41441435)(509.92047701,703.55442034)
\curveto(509.690472,703.70441406)(509.37047232,703.78441398)(508.96047701,703.79442034)
\curveto(508.93047276,703.77441399)(508.8954728,703.76941399)(508.85547701,703.77942034)
\curveto(508.81547288,703.78941397)(508.78047291,703.78941397)(508.75047701,703.77942034)
\curveto(508.70047299,703.759414)(508.64547305,703.74441402)(508.58547701,703.73442034)
\curveto(508.52547317,703.73441403)(508.47047322,703.72441404)(508.42047701,703.70442034)
\curveto(507.98047371,703.5644142)(507.65547404,703.28941447)(507.44547701,702.87942034)
\curveto(507.42547427,702.83941492)(507.40047429,702.78441498)(507.37047701,702.71442034)
\curveto(507.35047434,702.65441511)(507.33547436,702.58941517)(507.32547701,702.51942034)
\curveto(507.31547438,702.4594153)(507.31547438,702.39941536)(507.32547701,702.33942034)
\curveto(507.34547435,702.27941548)(507.38047431,702.22941553)(507.43047701,702.18942034)
\curveto(507.51047418,702.13941562)(507.62047407,702.11441565)(507.76047701,702.11442034)
\lineto(508.16547701,702.11442034)
\lineto(509.83047701,702.11442034)
\lineto(510.26547701,702.11442034)
\curveto(510.42547127,702.12441564)(510.53047116,702.16941559)(510.58047701,702.24942034)
}
}
{
\newrgbcolor{curcolor}{0 0 0}
\pscustom[linestyle=none,fillstyle=solid,fillcolor=curcolor]
{
}
}
{
\newrgbcolor{curcolor}{0 0 0}
\pscustom[linestyle=none,fillstyle=solid,fillcolor=curcolor]
{
\newpath
\moveto(518.74891451,708.11442034)
\lineto(519.84391451,708.11442034)
\curveto(519.94391202,708.11440965)(520.03891193,708.10940965)(520.12891451,708.09942034)
\curveto(520.21891175,708.08940967)(520.28891168,708.0594097)(520.33891451,708.00942034)
\curveto(520.39891157,707.93940982)(520.42891154,707.84440992)(520.42891451,707.72442034)
\curveto(520.43891153,707.61441015)(520.44391152,707.49941026)(520.44391451,707.37942034)
\lineto(520.44391451,706.04442034)
\lineto(520.44391451,700.65942034)
\lineto(520.44391451,698.36442034)
\lineto(520.44391451,697.94442034)
\curveto(520.45391151,697.79441997)(520.43391153,697.67942008)(520.38391451,697.59942034)
\curveto(520.33391163,697.51942024)(520.24391172,697.4644203)(520.11391451,697.43442034)
\curveto(520.05391191,697.41442035)(519.98391198,697.40942035)(519.90391451,697.41942034)
\curveto(519.83391213,697.42942033)(519.7639122,697.43442033)(519.69391451,697.43442034)
\lineto(518.97391451,697.43442034)
\curveto(518.8639131,697.43442033)(518.7639132,697.43942032)(518.67391451,697.44942034)
\curveto(518.58391338,697.4594203)(518.50891346,697.48942027)(518.44891451,697.53942034)
\curveto(518.38891358,697.58942017)(518.35391361,697.6644201)(518.34391451,697.76442034)
\lineto(518.34391451,698.09442034)
\lineto(518.34391451,699.42942034)
\lineto(518.34391451,705.05442034)
\lineto(518.34391451,707.09442034)
\curveto(518.34391362,707.22441054)(518.33891363,707.37941038)(518.32891451,707.55942034)
\curveto(518.32891364,707.73941002)(518.35391361,707.86940989)(518.40391451,707.94942034)
\curveto(518.42391354,707.98940977)(518.44891352,708.01940974)(518.47891451,708.03942034)
\lineto(518.59891451,708.09942034)
\curveto(518.61891335,708.09940966)(518.64391332,708.09940966)(518.67391451,708.09942034)
\curveto(518.70391326,708.10940965)(518.72891324,708.11440965)(518.74891451,708.11442034)
}
}
{
\newrgbcolor{curcolor}{0 0 0}
\pscustom[linestyle=none,fillstyle=solid,fillcolor=curcolor]
{
\newpath
\moveto(529.15610201,698.01942034)
\curveto(529.17609416,697.90941985)(529.18609415,697.79941996)(529.18610201,697.68942034)
\curveto(529.19609414,697.57942018)(529.14609419,697.50442026)(529.03610201,697.46442034)
\curveto(528.97609436,697.43442033)(528.90609443,697.41942034)(528.82610201,697.41942034)
\lineto(528.58610201,697.41942034)
\lineto(527.77610201,697.41942034)
\lineto(527.50610201,697.41942034)
\curveto(527.42609591,697.42942033)(527.36109597,697.45442031)(527.31110201,697.49442034)
\curveto(527.24109609,697.53442023)(527.18609615,697.58942017)(527.14610201,697.65942034)
\curveto(527.11609622,697.73942002)(527.07109626,697.80441996)(527.01110201,697.85442034)
\curveto(526.99109634,697.87441989)(526.96609637,697.88941987)(526.93610201,697.89942034)
\curveto(526.90609643,697.91941984)(526.86609647,697.92441984)(526.81610201,697.91442034)
\curveto(526.76609657,697.89441987)(526.71609662,697.86941989)(526.66610201,697.83942034)
\curveto(526.62609671,697.80941995)(526.58109675,697.78441998)(526.53110201,697.76442034)
\curveto(526.48109685,697.72442004)(526.42609691,697.68942007)(526.36610201,697.65942034)
\lineto(526.18610201,697.56942034)
\curveto(526.05609728,697.50942025)(525.92109741,697.4594203)(525.78110201,697.41942034)
\curveto(525.64109769,697.38942037)(525.49609784,697.35442041)(525.34610201,697.31442034)
\curveto(525.27609806,697.29442047)(525.20609813,697.28442048)(525.13610201,697.28442034)
\curveto(525.07609826,697.27442049)(525.01109832,697.2644205)(524.94110201,697.25442034)
\lineto(524.85110201,697.25442034)
\curveto(524.82109851,697.24442052)(524.79109854,697.23942052)(524.76110201,697.23942034)
\lineto(524.59610201,697.23942034)
\curveto(524.49609884,697.21942054)(524.39609894,697.21942054)(524.29610201,697.23942034)
\lineto(524.16110201,697.23942034)
\curveto(524.09109924,697.2594205)(524.02109931,697.26942049)(523.95110201,697.26942034)
\curveto(523.89109944,697.2594205)(523.8310995,697.2644205)(523.77110201,697.28442034)
\curveto(523.67109966,697.30442046)(523.57609976,697.32442044)(523.48610201,697.34442034)
\curveto(523.39609994,697.35442041)(523.31110002,697.37942038)(523.23110201,697.41942034)
\curveto(522.94110039,697.52942023)(522.69110064,697.66942009)(522.48110201,697.83942034)
\curveto(522.28110105,698.01941974)(522.12110121,698.25441951)(522.00110201,698.54442034)
\curveto(521.97110136,698.61441915)(521.94110139,698.68941907)(521.91110201,698.76942034)
\curveto(521.89110144,698.84941891)(521.87110146,698.93441883)(521.85110201,699.02442034)
\curveto(521.8311015,699.07441869)(521.82110151,699.12441864)(521.82110201,699.17442034)
\curveto(521.8311015,699.22441854)(521.8311015,699.27441849)(521.82110201,699.32442034)
\curveto(521.81110152,699.35441841)(521.80110153,699.41441835)(521.79110201,699.50442034)
\curveto(521.79110154,699.60441816)(521.79610154,699.67441809)(521.80610201,699.71442034)
\curveto(521.82610151,699.81441795)(521.8361015,699.89941786)(521.83610201,699.96942034)
\lineto(521.92610201,700.29942034)
\curveto(521.95610138,700.41941734)(521.99610134,700.52441724)(522.04610201,700.61442034)
\curveto(522.21610112,700.90441686)(522.41110092,701.12441664)(522.63110201,701.27442034)
\curveto(522.85110048,701.42441634)(523.1311002,701.55441621)(523.47110201,701.66442034)
\curveto(523.60109973,701.71441605)(523.7360996,701.74941601)(523.87610201,701.76942034)
\curveto(524.01609932,701.78941597)(524.15609918,701.81441595)(524.29610201,701.84442034)
\curveto(524.37609896,701.8644159)(524.46109887,701.87441589)(524.55110201,701.87442034)
\curveto(524.64109869,701.88441588)(524.7310986,701.89941586)(524.82110201,701.91942034)
\curveto(524.89109844,701.93941582)(524.96109837,701.94441582)(525.03110201,701.93442034)
\curveto(525.10109823,701.93441583)(525.17609816,701.94441582)(525.25610201,701.96442034)
\curveto(525.32609801,701.98441578)(525.39609794,701.99441577)(525.46610201,701.99442034)
\curveto(525.5360978,701.99441577)(525.61109772,702.00441576)(525.69110201,702.02442034)
\curveto(525.90109743,702.07441569)(526.09109724,702.11441565)(526.26110201,702.14442034)
\curveto(526.44109689,702.18441558)(526.60109673,702.27441549)(526.74110201,702.41442034)
\curveto(526.8310965,702.50441526)(526.89109644,702.60441516)(526.92110201,702.71442034)
\curveto(526.9310964,702.74441502)(526.9310964,702.76941499)(526.92110201,702.78942034)
\curveto(526.92109641,702.80941495)(526.92609641,702.82941493)(526.93610201,702.84942034)
\curveto(526.94609639,702.86941489)(526.95109638,702.89941486)(526.95110201,702.93942034)
\lineto(526.95110201,703.02942034)
\lineto(526.92110201,703.14942034)
\curveto(526.92109641,703.18941457)(526.91609642,703.22441454)(526.90610201,703.25442034)
\curveto(526.80609653,703.55441421)(526.59609674,703.759414)(526.27610201,703.86942034)
\curveto(526.18609715,703.89941386)(526.07609726,703.91941384)(525.94610201,703.92942034)
\curveto(525.82609751,703.94941381)(525.70109763,703.95441381)(525.57110201,703.94442034)
\curveto(525.44109789,703.94441382)(525.31609802,703.93441383)(525.19610201,703.91442034)
\curveto(525.07609826,703.89441387)(524.97109836,703.86941389)(524.88110201,703.83942034)
\curveto(524.82109851,703.81941394)(524.76109857,703.78941397)(524.70110201,703.74942034)
\curveto(524.65109868,703.71941404)(524.60109873,703.68441408)(524.55110201,703.64442034)
\curveto(524.50109883,703.60441416)(524.44609889,703.54941421)(524.38610201,703.47942034)
\curveto(524.336099,703.40941435)(524.30109903,703.34441442)(524.28110201,703.28442034)
\curveto(524.2310991,703.18441458)(524.18609915,703.08941467)(524.14610201,702.99942034)
\curveto(524.11609922,702.90941485)(524.04609929,702.84941491)(523.93610201,702.81942034)
\curveto(523.85609948,702.79941496)(523.77109956,702.78941497)(523.68110201,702.78942034)
\lineto(523.41110201,702.78942034)
\lineto(522.84110201,702.78942034)
\curveto(522.79110054,702.78941497)(522.74110059,702.78441498)(522.69110201,702.77442034)
\curveto(522.64110069,702.77441499)(522.59610074,702.77941498)(522.55610201,702.78942034)
\lineto(522.42110201,702.78942034)
\curveto(522.40110093,702.79941496)(522.37610096,702.80441496)(522.34610201,702.80442034)
\curveto(522.31610102,702.80441496)(522.29110104,702.81441495)(522.27110201,702.83442034)
\curveto(522.19110114,702.85441491)(522.1361012,702.91941484)(522.10610201,703.02942034)
\curveto(522.09610124,703.07941468)(522.09610124,703.12941463)(522.10610201,703.17942034)
\curveto(522.11610122,703.22941453)(522.12610121,703.27441449)(522.13610201,703.31442034)
\curveto(522.16610117,703.42441434)(522.19610114,703.52441424)(522.22610201,703.61442034)
\curveto(522.26610107,703.71441405)(522.31110102,703.80441396)(522.36110201,703.88442034)
\lineto(522.45110201,704.03442034)
\lineto(522.54110201,704.18442034)
\curveto(522.62110071,704.29441347)(522.72110061,704.39941336)(522.84110201,704.49942034)
\curveto(522.86110047,704.50941325)(522.89110044,704.53441323)(522.93110201,704.57442034)
\curveto(522.98110035,704.61441315)(523.02610031,704.64941311)(523.06610201,704.67942034)
\curveto(523.10610023,704.70941305)(523.15110018,704.73941302)(523.20110201,704.76942034)
\curveto(523.37109996,704.87941288)(523.55109978,704.9644128)(523.74110201,705.02442034)
\curveto(523.9310994,705.09441267)(524.12609921,705.1594126)(524.32610201,705.21942034)
\curveto(524.44609889,705.24941251)(524.57109876,705.26941249)(524.70110201,705.27942034)
\curveto(524.8310985,705.28941247)(524.96109837,705.30941245)(525.09110201,705.33942034)
\curveto(525.1310982,705.34941241)(525.19109814,705.34941241)(525.27110201,705.33942034)
\curveto(525.36109797,705.32941243)(525.41609792,705.33441243)(525.43610201,705.35442034)
\curveto(525.84609749,705.3644124)(526.2360971,705.34941241)(526.60610201,705.30942034)
\curveto(526.98609635,705.26941249)(527.32609601,705.19441257)(527.62610201,705.08442034)
\curveto(527.9360954,704.97441279)(528.20109513,704.82441294)(528.42110201,704.63442034)
\curveto(528.64109469,704.45441331)(528.81109452,704.21941354)(528.93110201,703.92942034)
\curveto(529.00109433,703.759414)(529.04109429,703.5644142)(529.05110201,703.34442034)
\curveto(529.06109427,703.12441464)(529.06609427,702.89941486)(529.06610201,702.66942034)
\lineto(529.06610201,699.32442034)
\lineto(529.06610201,698.73942034)
\curveto(529.06609427,698.54941921)(529.08609425,698.37441939)(529.12610201,698.21442034)
\curveto(529.1360942,698.18441958)(529.14109419,698.14941961)(529.14110201,698.10942034)
\curveto(529.14109419,698.07941968)(529.14609419,698.04941971)(529.15610201,698.01942034)
\moveto(526.95110201,700.32942034)
\curveto(526.96109637,700.37941738)(526.96609637,700.43441733)(526.96610201,700.49442034)
\curveto(526.96609637,700.5644172)(526.96109637,700.62441714)(526.95110201,700.67442034)
\curveto(526.9310964,700.73441703)(526.92109641,700.78941697)(526.92110201,700.83942034)
\curveto(526.92109641,700.88941687)(526.90109643,700.92941683)(526.86110201,700.95942034)
\curveto(526.81109652,700.99941676)(526.7360966,701.01941674)(526.63610201,701.01942034)
\curveto(526.59609674,701.00941675)(526.56109677,700.99941676)(526.53110201,700.98942034)
\curveto(526.50109683,700.98941677)(526.46609687,700.98441678)(526.42610201,700.97442034)
\curveto(526.35609698,700.95441681)(526.28109705,700.93941682)(526.20110201,700.92942034)
\curveto(526.12109721,700.91941684)(526.04109729,700.90441686)(525.96110201,700.88442034)
\curveto(525.9310974,700.87441689)(525.88609745,700.86941689)(525.82610201,700.86942034)
\curveto(525.69609764,700.83941692)(525.56609777,700.81941694)(525.43610201,700.80942034)
\curveto(525.30609803,700.79941696)(525.18109815,700.77441699)(525.06110201,700.73442034)
\curveto(524.98109835,700.71441705)(524.90609843,700.69441707)(524.83610201,700.67442034)
\curveto(524.76609857,700.6644171)(524.69609864,700.64441712)(524.62610201,700.61442034)
\curveto(524.41609892,700.52441724)(524.2360991,700.38941737)(524.08610201,700.20942034)
\curveto(523.94609939,700.02941773)(523.89609944,699.77941798)(523.93610201,699.45942034)
\curveto(523.95609938,699.28941847)(524.01109932,699.14941861)(524.10110201,699.03942034)
\curveto(524.17109916,698.92941883)(524.27609906,698.83941892)(524.41610201,698.76942034)
\curveto(524.55609878,698.70941905)(524.70609863,698.6644191)(524.86610201,698.63442034)
\curveto(525.0360983,698.60441916)(525.21109812,698.59441917)(525.39110201,698.60442034)
\curveto(525.58109775,698.62441914)(525.75609758,698.6594191)(525.91610201,698.70942034)
\curveto(526.17609716,698.78941897)(526.38109695,698.91441885)(526.53110201,699.08442034)
\curveto(526.68109665,699.2644185)(526.79609654,699.48441828)(526.87610201,699.74442034)
\curveto(526.89609644,699.81441795)(526.90609643,699.88441788)(526.90610201,699.95442034)
\curveto(526.91609642,700.03441773)(526.9310964,700.11441765)(526.95110201,700.19442034)
\lineto(526.95110201,700.32942034)
}
}
{
\newrgbcolor{curcolor}{0 0 0}
\pscustom[linestyle=none,fillstyle=solid,fillcolor=curcolor]
{
}
}
{
\newrgbcolor{curcolor}{0 0 0}
\pscustom[linestyle=none,fillstyle=solid,fillcolor=curcolor]
{
\newpath
\moveto(541.92953951,698.01942034)
\curveto(541.94953166,697.90941985)(541.95953165,697.79941996)(541.95953951,697.68942034)
\curveto(541.96953164,697.57942018)(541.91953169,697.50442026)(541.80953951,697.46442034)
\curveto(541.74953186,697.43442033)(541.67953193,697.41942034)(541.59953951,697.41942034)
\lineto(541.35953951,697.41942034)
\lineto(540.54953951,697.41942034)
\lineto(540.27953951,697.41942034)
\curveto(540.19953341,697.42942033)(540.13453347,697.45442031)(540.08453951,697.49442034)
\curveto(540.01453359,697.53442023)(539.95953365,697.58942017)(539.91953951,697.65942034)
\curveto(539.88953372,697.73942002)(539.84453376,697.80441996)(539.78453951,697.85442034)
\curveto(539.76453384,697.87441989)(539.73953387,697.88941987)(539.70953951,697.89942034)
\curveto(539.67953393,697.91941984)(539.63953397,697.92441984)(539.58953951,697.91442034)
\curveto(539.53953407,697.89441987)(539.48953412,697.86941989)(539.43953951,697.83942034)
\curveto(539.39953421,697.80941995)(539.35453425,697.78441998)(539.30453951,697.76442034)
\curveto(539.25453435,697.72442004)(539.19953441,697.68942007)(539.13953951,697.65942034)
\lineto(538.95953951,697.56942034)
\curveto(538.82953478,697.50942025)(538.69453491,697.4594203)(538.55453951,697.41942034)
\curveto(538.41453519,697.38942037)(538.26953534,697.35442041)(538.11953951,697.31442034)
\curveto(538.04953556,697.29442047)(537.97953563,697.28442048)(537.90953951,697.28442034)
\curveto(537.84953576,697.27442049)(537.78453582,697.2644205)(537.71453951,697.25442034)
\lineto(537.62453951,697.25442034)
\curveto(537.59453601,697.24442052)(537.56453604,697.23942052)(537.53453951,697.23942034)
\lineto(537.36953951,697.23942034)
\curveto(537.26953634,697.21942054)(537.16953644,697.21942054)(537.06953951,697.23942034)
\lineto(536.93453951,697.23942034)
\curveto(536.86453674,697.2594205)(536.79453681,697.26942049)(536.72453951,697.26942034)
\curveto(536.66453694,697.2594205)(536.604537,697.2644205)(536.54453951,697.28442034)
\curveto(536.44453716,697.30442046)(536.34953726,697.32442044)(536.25953951,697.34442034)
\curveto(536.16953744,697.35442041)(536.08453752,697.37942038)(536.00453951,697.41942034)
\curveto(535.71453789,697.52942023)(535.46453814,697.66942009)(535.25453951,697.83942034)
\curveto(535.05453855,698.01941974)(534.89453871,698.25441951)(534.77453951,698.54442034)
\curveto(534.74453886,698.61441915)(534.71453889,698.68941907)(534.68453951,698.76942034)
\curveto(534.66453894,698.84941891)(534.64453896,698.93441883)(534.62453951,699.02442034)
\curveto(534.604539,699.07441869)(534.59453901,699.12441864)(534.59453951,699.17442034)
\curveto(534.604539,699.22441854)(534.604539,699.27441849)(534.59453951,699.32442034)
\curveto(534.58453902,699.35441841)(534.57453903,699.41441835)(534.56453951,699.50442034)
\curveto(534.56453904,699.60441816)(534.56953904,699.67441809)(534.57953951,699.71442034)
\curveto(534.59953901,699.81441795)(534.609539,699.89941786)(534.60953951,699.96942034)
\lineto(534.69953951,700.29942034)
\curveto(534.72953888,700.41941734)(534.76953884,700.52441724)(534.81953951,700.61442034)
\curveto(534.98953862,700.90441686)(535.18453842,701.12441664)(535.40453951,701.27442034)
\curveto(535.62453798,701.42441634)(535.9045377,701.55441621)(536.24453951,701.66442034)
\curveto(536.37453723,701.71441605)(536.5095371,701.74941601)(536.64953951,701.76942034)
\curveto(536.78953682,701.78941597)(536.92953668,701.81441595)(537.06953951,701.84442034)
\curveto(537.14953646,701.8644159)(537.23453637,701.87441589)(537.32453951,701.87442034)
\curveto(537.41453619,701.88441588)(537.5045361,701.89941586)(537.59453951,701.91942034)
\curveto(537.66453594,701.93941582)(537.73453587,701.94441582)(537.80453951,701.93442034)
\curveto(537.87453573,701.93441583)(537.94953566,701.94441582)(538.02953951,701.96442034)
\curveto(538.09953551,701.98441578)(538.16953544,701.99441577)(538.23953951,701.99442034)
\curveto(538.3095353,701.99441577)(538.38453522,702.00441576)(538.46453951,702.02442034)
\curveto(538.67453493,702.07441569)(538.86453474,702.11441565)(539.03453951,702.14442034)
\curveto(539.21453439,702.18441558)(539.37453423,702.27441549)(539.51453951,702.41442034)
\curveto(539.604534,702.50441526)(539.66453394,702.60441516)(539.69453951,702.71442034)
\curveto(539.7045339,702.74441502)(539.7045339,702.76941499)(539.69453951,702.78942034)
\curveto(539.69453391,702.80941495)(539.69953391,702.82941493)(539.70953951,702.84942034)
\curveto(539.71953389,702.86941489)(539.72453388,702.89941486)(539.72453951,702.93942034)
\lineto(539.72453951,703.02942034)
\lineto(539.69453951,703.14942034)
\curveto(539.69453391,703.18941457)(539.68953392,703.22441454)(539.67953951,703.25442034)
\curveto(539.57953403,703.55441421)(539.36953424,703.759414)(539.04953951,703.86942034)
\curveto(538.95953465,703.89941386)(538.84953476,703.91941384)(538.71953951,703.92942034)
\curveto(538.59953501,703.94941381)(538.47453513,703.95441381)(538.34453951,703.94442034)
\curveto(538.21453539,703.94441382)(538.08953552,703.93441383)(537.96953951,703.91442034)
\curveto(537.84953576,703.89441387)(537.74453586,703.86941389)(537.65453951,703.83942034)
\curveto(537.59453601,703.81941394)(537.53453607,703.78941397)(537.47453951,703.74942034)
\curveto(537.42453618,703.71941404)(537.37453623,703.68441408)(537.32453951,703.64442034)
\curveto(537.27453633,703.60441416)(537.21953639,703.54941421)(537.15953951,703.47942034)
\curveto(537.1095365,703.40941435)(537.07453653,703.34441442)(537.05453951,703.28442034)
\curveto(537.0045366,703.18441458)(536.95953665,703.08941467)(536.91953951,702.99942034)
\curveto(536.88953672,702.90941485)(536.81953679,702.84941491)(536.70953951,702.81942034)
\curveto(536.62953698,702.79941496)(536.54453706,702.78941497)(536.45453951,702.78942034)
\lineto(536.18453951,702.78942034)
\lineto(535.61453951,702.78942034)
\curveto(535.56453804,702.78941497)(535.51453809,702.78441498)(535.46453951,702.77442034)
\curveto(535.41453819,702.77441499)(535.36953824,702.77941498)(535.32953951,702.78942034)
\lineto(535.19453951,702.78942034)
\curveto(535.17453843,702.79941496)(535.14953846,702.80441496)(535.11953951,702.80442034)
\curveto(535.08953852,702.80441496)(535.06453854,702.81441495)(535.04453951,702.83442034)
\curveto(534.96453864,702.85441491)(534.9095387,702.91941484)(534.87953951,703.02942034)
\curveto(534.86953874,703.07941468)(534.86953874,703.12941463)(534.87953951,703.17942034)
\curveto(534.88953872,703.22941453)(534.89953871,703.27441449)(534.90953951,703.31442034)
\curveto(534.93953867,703.42441434)(534.96953864,703.52441424)(534.99953951,703.61442034)
\curveto(535.03953857,703.71441405)(535.08453852,703.80441396)(535.13453951,703.88442034)
\lineto(535.22453951,704.03442034)
\lineto(535.31453951,704.18442034)
\curveto(535.39453821,704.29441347)(535.49453811,704.39941336)(535.61453951,704.49942034)
\curveto(535.63453797,704.50941325)(535.66453794,704.53441323)(535.70453951,704.57442034)
\curveto(535.75453785,704.61441315)(535.79953781,704.64941311)(535.83953951,704.67942034)
\curveto(535.87953773,704.70941305)(535.92453768,704.73941302)(535.97453951,704.76942034)
\curveto(536.14453746,704.87941288)(536.32453728,704.9644128)(536.51453951,705.02442034)
\curveto(536.7045369,705.09441267)(536.89953671,705.1594126)(537.09953951,705.21942034)
\curveto(537.21953639,705.24941251)(537.34453626,705.26941249)(537.47453951,705.27942034)
\curveto(537.604536,705.28941247)(537.73453587,705.30941245)(537.86453951,705.33942034)
\curveto(537.9045357,705.34941241)(537.96453564,705.34941241)(538.04453951,705.33942034)
\curveto(538.13453547,705.32941243)(538.18953542,705.33441243)(538.20953951,705.35442034)
\curveto(538.61953499,705.3644124)(539.0095346,705.34941241)(539.37953951,705.30942034)
\curveto(539.75953385,705.26941249)(540.09953351,705.19441257)(540.39953951,705.08442034)
\curveto(540.7095329,704.97441279)(540.97453263,704.82441294)(541.19453951,704.63442034)
\curveto(541.41453219,704.45441331)(541.58453202,704.21941354)(541.70453951,703.92942034)
\curveto(541.77453183,703.759414)(541.81453179,703.5644142)(541.82453951,703.34442034)
\curveto(541.83453177,703.12441464)(541.83953177,702.89941486)(541.83953951,702.66942034)
\lineto(541.83953951,699.32442034)
\lineto(541.83953951,698.73942034)
\curveto(541.83953177,698.54941921)(541.85953175,698.37441939)(541.89953951,698.21442034)
\curveto(541.9095317,698.18441958)(541.91453169,698.14941961)(541.91453951,698.10942034)
\curveto(541.91453169,698.07941968)(541.91953169,698.04941971)(541.92953951,698.01942034)
\moveto(539.72453951,700.32942034)
\curveto(539.73453387,700.37941738)(539.73953387,700.43441733)(539.73953951,700.49442034)
\curveto(539.73953387,700.5644172)(539.73453387,700.62441714)(539.72453951,700.67442034)
\curveto(539.7045339,700.73441703)(539.69453391,700.78941697)(539.69453951,700.83942034)
\curveto(539.69453391,700.88941687)(539.67453393,700.92941683)(539.63453951,700.95942034)
\curveto(539.58453402,700.99941676)(539.5095341,701.01941674)(539.40953951,701.01942034)
\curveto(539.36953424,701.00941675)(539.33453427,700.99941676)(539.30453951,700.98942034)
\curveto(539.27453433,700.98941677)(539.23953437,700.98441678)(539.19953951,700.97442034)
\curveto(539.12953448,700.95441681)(539.05453455,700.93941682)(538.97453951,700.92942034)
\curveto(538.89453471,700.91941684)(538.81453479,700.90441686)(538.73453951,700.88442034)
\curveto(538.7045349,700.87441689)(538.65953495,700.86941689)(538.59953951,700.86942034)
\curveto(538.46953514,700.83941692)(538.33953527,700.81941694)(538.20953951,700.80942034)
\curveto(538.07953553,700.79941696)(537.95453565,700.77441699)(537.83453951,700.73442034)
\curveto(537.75453585,700.71441705)(537.67953593,700.69441707)(537.60953951,700.67442034)
\curveto(537.53953607,700.6644171)(537.46953614,700.64441712)(537.39953951,700.61442034)
\curveto(537.18953642,700.52441724)(537.0095366,700.38941737)(536.85953951,700.20942034)
\curveto(536.71953689,700.02941773)(536.66953694,699.77941798)(536.70953951,699.45942034)
\curveto(536.72953688,699.28941847)(536.78453682,699.14941861)(536.87453951,699.03942034)
\curveto(536.94453666,698.92941883)(537.04953656,698.83941892)(537.18953951,698.76942034)
\curveto(537.32953628,698.70941905)(537.47953613,698.6644191)(537.63953951,698.63442034)
\curveto(537.8095358,698.60441916)(537.98453562,698.59441917)(538.16453951,698.60442034)
\curveto(538.35453525,698.62441914)(538.52953508,698.6594191)(538.68953951,698.70942034)
\curveto(538.94953466,698.78941897)(539.15453445,698.91441885)(539.30453951,699.08442034)
\curveto(539.45453415,699.2644185)(539.56953404,699.48441828)(539.64953951,699.74442034)
\curveto(539.66953394,699.81441795)(539.67953393,699.88441788)(539.67953951,699.95442034)
\curveto(539.68953392,700.03441773)(539.7045339,700.11441765)(539.72453951,700.19442034)
\lineto(539.72453951,700.32942034)
}
}
{
\newrgbcolor{curcolor}{0 0 0}
\pscustom[linestyle=none,fillstyle=solid,fillcolor=curcolor]
{
\newpath
\moveto(543.91282076,705.14442034)
\lineto(545.03782076,705.14442034)
\curveto(545.14781832,705.14441262)(545.24781822,705.13941262)(545.33782076,705.12942034)
\curveto(545.42781804,705.11941264)(545.49281798,705.08441268)(545.53282076,705.02442034)
\curveto(545.58281789,704.9644128)(545.61281786,704.87941288)(545.62282076,704.76942034)
\curveto(545.63281784,704.66941309)(545.63781783,704.5644132)(545.63782076,704.45442034)
\lineto(545.63782076,703.40442034)
\lineto(545.63782076,701.16942034)
\curveto(545.63781783,700.80941695)(545.65281782,700.46941729)(545.68282076,700.14942034)
\curveto(545.71281776,699.82941793)(545.80281767,699.5644182)(545.95282076,699.35442034)
\curveto(546.09281738,699.14441862)(546.31781715,698.99441877)(546.62782076,698.90442034)
\curveto(546.67781679,698.89441887)(546.71781675,698.88941887)(546.74782076,698.88942034)
\curveto(546.78781668,698.88941887)(546.83281664,698.88441888)(546.88282076,698.87442034)
\curveto(546.93281654,698.8644189)(546.98781648,698.8594189)(547.04782076,698.85942034)
\curveto(547.10781636,698.8594189)(547.15281632,698.8644189)(547.18282076,698.87442034)
\curveto(547.23281624,698.89441887)(547.2728162,698.89941886)(547.30282076,698.88942034)
\curveto(547.34281613,698.87941888)(547.38281609,698.88441888)(547.42282076,698.90442034)
\curveto(547.63281584,698.95441881)(547.79781567,699.01941874)(547.91782076,699.09942034)
\curveto(548.09781537,699.20941855)(548.23781523,699.34941841)(548.33782076,699.51942034)
\curveto(548.44781502,699.69941806)(548.52281495,699.89441787)(548.56282076,700.10442034)
\curveto(548.61281486,700.32441744)(548.64281483,700.5644172)(548.65282076,700.82442034)
\curveto(548.66281481,701.09441667)(548.6678148,701.37441639)(548.66782076,701.66442034)
\lineto(548.66782076,703.47942034)
\lineto(548.66782076,704.45442034)
\lineto(548.66782076,704.72442034)
\curveto(548.6678148,704.82441294)(548.68781478,704.90441286)(548.72782076,704.96442034)
\curveto(548.77781469,705.05441271)(548.85281462,705.10441266)(548.95282076,705.11442034)
\curveto(549.05281442,705.13441263)(549.1728143,705.14441262)(549.31282076,705.14442034)
\lineto(550.10782076,705.14442034)
\lineto(550.39282076,705.14442034)
\curveto(550.48281299,705.14441262)(550.55781291,705.12441264)(550.61782076,705.08442034)
\curveto(550.69781277,705.03441273)(550.74281273,704.9594128)(550.75282076,704.85942034)
\curveto(550.76281271,704.759413)(550.7678127,704.64441312)(550.76782076,704.51442034)
\lineto(550.76782076,703.37442034)
\lineto(550.76782076,699.15942034)
\lineto(550.76782076,698.09442034)
\lineto(550.76782076,697.79442034)
\curveto(550.7678127,697.69442007)(550.74781272,697.61942014)(550.70782076,697.56942034)
\curveto(550.65781281,697.48942027)(550.58281289,697.44442032)(550.48282076,697.43442034)
\curveto(550.38281309,697.42442034)(550.27781319,697.41942034)(550.16782076,697.41942034)
\lineto(549.35782076,697.41942034)
\curveto(549.24781422,697.41942034)(549.14781432,697.42442034)(549.05782076,697.43442034)
\curveto(548.97781449,697.44442032)(548.91281456,697.48442028)(548.86282076,697.55442034)
\curveto(548.84281463,697.58442018)(548.82281465,697.62942013)(548.80282076,697.68942034)
\curveto(548.79281468,697.74942001)(548.77781469,697.80941995)(548.75782076,697.86942034)
\curveto(548.74781472,697.92941983)(548.73281474,697.98441978)(548.71282076,698.03442034)
\curveto(548.69281478,698.08441968)(548.66281481,698.11441965)(548.62282076,698.12442034)
\curveto(548.60281487,698.14441962)(548.57781489,698.14941961)(548.54782076,698.13942034)
\curveto(548.51781495,698.12941963)(548.49281498,698.11941964)(548.47282076,698.10942034)
\curveto(548.40281507,698.06941969)(548.34281513,698.02441974)(548.29282076,697.97442034)
\curveto(548.24281523,697.92441984)(548.18781528,697.87941988)(548.12782076,697.83942034)
\curveto(548.08781538,697.80941995)(548.04781542,697.77441999)(548.00782076,697.73442034)
\curveto(547.97781549,697.70442006)(547.93781553,697.67442009)(547.88782076,697.64442034)
\curveto(547.65781581,697.50442026)(547.38781608,697.39442037)(547.07782076,697.31442034)
\curveto(547.00781646,697.29442047)(546.93781653,697.28442048)(546.86782076,697.28442034)
\curveto(546.79781667,697.27442049)(546.72281675,697.2594205)(546.64282076,697.23942034)
\curveto(546.60281687,697.22942053)(546.55781691,697.22942053)(546.50782076,697.23942034)
\curveto(546.467817,697.23942052)(546.42781704,697.23442053)(546.38782076,697.22442034)
\curveto(546.35781711,697.21442055)(546.29281718,697.21442055)(546.19282076,697.22442034)
\curveto(546.10281737,697.22442054)(546.04281743,697.22942053)(546.01282076,697.23942034)
\curveto(545.96281751,697.23942052)(545.91281756,697.24442052)(545.86282076,697.25442034)
\lineto(545.71282076,697.25442034)
\curveto(545.59281788,697.28442048)(545.47781799,697.30942045)(545.36782076,697.32942034)
\curveto(545.25781821,697.34942041)(545.14781832,697.37942038)(545.03782076,697.41942034)
\curveto(544.98781848,697.43942032)(544.94281853,697.45442031)(544.90282076,697.46442034)
\curveto(544.8728186,697.48442028)(544.83281864,697.50442026)(544.78282076,697.52442034)
\curveto(544.43281904,697.71442005)(544.15281932,697.97941978)(543.94282076,698.31942034)
\curveto(543.81281966,698.52941923)(543.71781975,698.77941898)(543.65782076,699.06942034)
\curveto(543.59781987,699.36941839)(543.55781991,699.68441808)(543.53782076,700.01442034)
\curveto(543.52781994,700.35441741)(543.52281995,700.69941706)(543.52282076,701.04942034)
\curveto(543.53281994,701.40941635)(543.53781993,701.764416)(543.53782076,702.11442034)
\lineto(543.53782076,704.15442034)
\curveto(543.53781993,704.28441348)(543.53281994,704.43441333)(543.52282076,704.60442034)
\curveto(543.52281995,704.78441298)(543.54781992,704.91441285)(543.59782076,704.99442034)
\curveto(543.62781984,705.04441272)(543.68781978,705.08941267)(543.77782076,705.12942034)
\curveto(543.83781963,705.12941263)(543.88281959,705.13441263)(543.91282076,705.14442034)
}
}
{
\newrgbcolor{curcolor}{0 0 0}
\pscustom[linestyle=none,fillstyle=solid,fillcolor=curcolor]
{
\newpath
\moveto(559.98907076,698.27442034)
\lineto(559.98907076,697.85442034)
\curveto(559.98906239,697.72442004)(559.95906242,697.61942014)(559.89907076,697.53942034)
\curveto(559.84906253,697.48942027)(559.78406259,697.45442031)(559.70407076,697.43442034)
\curveto(559.62406275,697.42442034)(559.53406284,697.41942034)(559.43407076,697.41942034)
\lineto(558.60907076,697.41942034)
\lineto(558.32407076,697.41942034)
\curveto(558.24406413,697.42942033)(558.1790642,697.45442031)(558.12907076,697.49442034)
\curveto(558.05906432,697.54442022)(558.01906436,697.60942015)(558.00907076,697.68942034)
\curveto(557.99906438,697.76941999)(557.9790644,697.84941991)(557.94907076,697.92942034)
\curveto(557.92906445,697.94941981)(557.90906447,697.9644198)(557.88907076,697.97442034)
\curveto(557.8790645,697.99441977)(557.86406451,698.01441975)(557.84407076,698.03442034)
\curveto(557.73406464,698.03441973)(557.65406472,698.00941975)(557.60407076,697.95942034)
\lineto(557.45407076,697.80942034)
\curveto(557.38406499,697.75942)(557.31906506,697.71442005)(557.25907076,697.67442034)
\curveto(557.19906518,697.64442012)(557.13406524,697.60442016)(557.06407076,697.55442034)
\curveto(557.02406535,697.53442023)(556.9790654,697.51442025)(556.92907076,697.49442034)
\curveto(556.88906549,697.47442029)(556.84406553,697.45442031)(556.79407076,697.43442034)
\curveto(556.65406572,697.38442038)(556.50406587,697.33942042)(556.34407076,697.29942034)
\curveto(556.29406608,697.27942048)(556.24906613,697.26942049)(556.20907076,697.26942034)
\curveto(556.16906621,697.26942049)(556.12906625,697.2644205)(556.08907076,697.25442034)
\lineto(555.95407076,697.25442034)
\curveto(555.92406645,697.24442052)(555.88406649,697.23942052)(555.83407076,697.23942034)
\lineto(555.69907076,697.23942034)
\curveto(555.63906674,697.21942054)(555.54906683,697.21442055)(555.42907076,697.22442034)
\curveto(555.30906707,697.22442054)(555.22406715,697.23442053)(555.17407076,697.25442034)
\curveto(555.10406727,697.27442049)(555.03906734,697.28442048)(554.97907076,697.28442034)
\curveto(554.92906745,697.27442049)(554.8740675,697.27942048)(554.81407076,697.29942034)
\lineto(554.45407076,697.41942034)
\curveto(554.34406803,697.44942031)(554.23406814,697.48942027)(554.12407076,697.53942034)
\curveto(553.7740686,697.68942007)(553.45906892,697.91941984)(553.17907076,698.22942034)
\curveto(552.90906947,698.54941921)(552.69406968,698.88441888)(552.53407076,699.23442034)
\curveto(552.48406989,699.34441842)(552.44406993,699.44941831)(552.41407076,699.54942034)
\curveto(552.38406999,699.6594181)(552.34907003,699.76941799)(552.30907076,699.87942034)
\curveto(552.29907008,699.91941784)(552.29407008,699.95441781)(552.29407076,699.98442034)
\curveto(552.29407008,700.02441774)(552.28407009,700.06941769)(552.26407076,700.11942034)
\curveto(552.24407013,700.19941756)(552.22407015,700.28441748)(552.20407076,700.37442034)
\curveto(552.19407018,700.47441729)(552.1790702,700.57441719)(552.15907076,700.67442034)
\curveto(552.14907023,700.70441706)(552.14407023,700.73941702)(552.14407076,700.77942034)
\curveto(552.15407022,700.81941694)(552.15407022,700.85441691)(552.14407076,700.88442034)
\lineto(552.14407076,701.01942034)
\curveto(552.14407023,701.06941669)(552.13907024,701.11941664)(552.12907076,701.16942034)
\curveto(552.11907026,701.21941654)(552.11407026,701.27441649)(552.11407076,701.33442034)
\curveto(552.11407026,701.40441636)(552.11907026,701.4594163)(552.12907076,701.49942034)
\curveto(552.13907024,701.54941621)(552.14407023,701.59441617)(552.14407076,701.63442034)
\lineto(552.14407076,701.78442034)
\curveto(552.15407022,701.83441593)(552.15407022,701.87941588)(552.14407076,701.91942034)
\curveto(552.14407023,701.96941579)(552.15407022,702.01941574)(552.17407076,702.06942034)
\curveto(552.19407018,702.17941558)(552.20907017,702.28441548)(552.21907076,702.38442034)
\curveto(552.23907014,702.48441528)(552.26407011,702.58441518)(552.29407076,702.68442034)
\curveto(552.33407004,702.80441496)(552.36907001,702.91941484)(552.39907076,703.02942034)
\curveto(552.42906995,703.13941462)(552.46906991,703.24941451)(552.51907076,703.35942034)
\curveto(552.65906972,703.6594141)(552.83406954,703.94441382)(553.04407076,704.21442034)
\curveto(553.06406931,704.24441352)(553.08906929,704.26941349)(553.11907076,704.28942034)
\curveto(553.15906922,704.31941344)(553.18906919,704.34941341)(553.20907076,704.37942034)
\curveto(553.24906913,704.42941333)(553.28906909,704.47441329)(553.32907076,704.51442034)
\curveto(553.36906901,704.55441321)(553.41406896,704.59441317)(553.46407076,704.63442034)
\curveto(553.50406887,704.65441311)(553.53906884,704.67941308)(553.56907076,704.70942034)
\curveto(553.59906878,704.74941301)(553.63406874,704.77941298)(553.67407076,704.79942034)
\curveto(553.92406845,704.96941279)(554.21406816,705.10941265)(554.54407076,705.21942034)
\curveto(554.61406776,705.23941252)(554.68406769,705.25441251)(554.75407076,705.26442034)
\curveto(554.83406754,705.27441249)(554.91406746,705.28941247)(554.99407076,705.30942034)
\curveto(555.06406731,705.32941243)(555.15406722,705.33941242)(555.26407076,705.33942034)
\curveto(555.374067,705.34941241)(555.48406689,705.35441241)(555.59407076,705.35442034)
\curveto(555.70406667,705.35441241)(555.80906657,705.34941241)(555.90907076,705.33942034)
\curveto(556.01906636,705.32941243)(556.10906627,705.31441245)(556.17907076,705.29442034)
\curveto(556.32906605,705.24441252)(556.4740659,705.19941256)(556.61407076,705.15942034)
\curveto(556.75406562,705.11941264)(556.88406549,705.0644127)(557.00407076,704.99442034)
\curveto(557.0740653,704.94441282)(557.13906524,704.89441287)(557.19907076,704.84442034)
\curveto(557.25906512,704.80441296)(557.32406505,704.759413)(557.39407076,704.70942034)
\curveto(557.43406494,704.67941308)(557.48906489,704.63941312)(557.55907076,704.58942034)
\curveto(557.63906474,704.53941322)(557.71406466,704.53941322)(557.78407076,704.58942034)
\curveto(557.82406455,704.60941315)(557.84406453,704.64441312)(557.84407076,704.69442034)
\curveto(557.84406453,704.74441302)(557.85406452,704.79441297)(557.87407076,704.84442034)
\lineto(557.87407076,704.99442034)
\curveto(557.88406449,705.02441274)(557.88906449,705.0594127)(557.88907076,705.09942034)
\lineto(557.88907076,705.21942034)
\lineto(557.88907076,707.25942034)
\curveto(557.88906449,707.36941039)(557.88406449,707.48941027)(557.87407076,707.61942034)
\curveto(557.8740645,707.75941)(557.89906448,707.8644099)(557.94907076,707.93442034)
\curveto(557.98906439,708.01440975)(558.06406431,708.0644097)(558.17407076,708.08442034)
\curveto(558.19406418,708.09440967)(558.21406416,708.09440967)(558.23407076,708.08442034)
\curveto(558.25406412,708.08440968)(558.2740641,708.08940967)(558.29407076,708.09942034)
\lineto(559.35907076,708.09942034)
\curveto(559.4790629,708.09940966)(559.58906279,708.09440967)(559.68907076,708.08442034)
\curveto(559.78906259,708.07440969)(559.86406251,708.03440973)(559.91407076,707.96442034)
\curveto(559.96406241,707.88440988)(559.98906239,707.77940998)(559.98907076,707.64942034)
\lineto(559.98907076,707.28942034)
\lineto(559.98907076,698.27442034)
\moveto(557.94907076,701.21442034)
\curveto(557.95906442,701.25441651)(557.95906442,701.29441647)(557.94907076,701.33442034)
\lineto(557.94907076,701.46942034)
\curveto(557.94906443,701.56941619)(557.94406443,701.66941609)(557.93407076,701.76942034)
\curveto(557.92406445,701.86941589)(557.90906447,701.9594158)(557.88907076,702.03942034)
\curveto(557.86906451,702.14941561)(557.84906453,702.24941551)(557.82907076,702.33942034)
\curveto(557.81906456,702.42941533)(557.79406458,702.51441525)(557.75407076,702.59442034)
\curveto(557.61406476,702.95441481)(557.40906497,703.23941452)(557.13907076,703.44942034)
\curveto(556.8790655,703.6594141)(556.49906588,703.764414)(555.99907076,703.76442034)
\curveto(555.93906644,703.764414)(555.85906652,703.75441401)(555.75907076,703.73442034)
\curveto(555.6790667,703.71441405)(555.60406677,703.69441407)(555.53407076,703.67442034)
\curveto(555.4740669,703.6644141)(555.41406696,703.64441412)(555.35407076,703.61442034)
\curveto(555.08406729,703.50441426)(554.8740675,703.33441443)(554.72407076,703.10442034)
\curveto(554.5740678,702.87441489)(554.45406792,702.61441515)(554.36407076,702.32442034)
\curveto(554.33406804,702.22441554)(554.31406806,702.12441564)(554.30407076,702.02442034)
\curveto(554.29406808,701.92441584)(554.2740681,701.81941594)(554.24407076,701.70942034)
\lineto(554.24407076,701.49942034)
\curveto(554.22406815,701.40941635)(554.21906816,701.28441648)(554.22907076,701.12442034)
\curveto(554.23906814,700.97441679)(554.25406812,700.8644169)(554.27407076,700.79442034)
\lineto(554.27407076,700.70442034)
\curveto(554.28406809,700.68441708)(554.28906809,700.6644171)(554.28907076,700.64442034)
\curveto(554.30906807,700.5644172)(554.32406805,700.48941727)(554.33407076,700.41942034)
\curveto(554.35406802,700.34941741)(554.374068,700.27441749)(554.39407076,700.19442034)
\curveto(554.56406781,699.67441809)(554.85406752,699.28941847)(555.26407076,699.03942034)
\curveto(555.39406698,698.94941881)(555.5740668,698.87941888)(555.80407076,698.82942034)
\curveto(555.84406653,698.81941894)(555.90406647,698.81441895)(555.98407076,698.81442034)
\curveto(556.01406636,698.80441896)(556.05906632,698.79441897)(556.11907076,698.78442034)
\curveto(556.18906619,698.78441898)(556.24406613,698.78941897)(556.28407076,698.79942034)
\curveto(556.36406601,698.81941894)(556.44406593,698.83441893)(556.52407076,698.84442034)
\curveto(556.60406577,698.85441891)(556.68406569,698.87441889)(556.76407076,698.90442034)
\curveto(557.01406536,699.01441875)(557.21406516,699.15441861)(557.36407076,699.32442034)
\curveto(557.51406486,699.49441827)(557.64406473,699.70941805)(557.75407076,699.96942034)
\curveto(557.79406458,700.0594177)(557.82406455,700.14941761)(557.84407076,700.23942034)
\curveto(557.86406451,700.33941742)(557.88406449,700.44441732)(557.90407076,700.55442034)
\curveto(557.91406446,700.60441716)(557.91406446,700.64941711)(557.90407076,700.68942034)
\curveto(557.90406447,700.73941702)(557.91406446,700.78941697)(557.93407076,700.83942034)
\curveto(557.94406443,700.86941689)(557.94906443,700.90441686)(557.94907076,700.94442034)
\lineto(557.94907076,701.07942034)
\lineto(557.94907076,701.21442034)
}
}
{
\newrgbcolor{curcolor}{0 0 0}
\pscustom[linestyle=none,fillstyle=solid,fillcolor=curcolor]
{
\newpath
\moveto(563.66899263,708.00942034)
\curveto(563.73898968,707.92940983)(563.77398965,707.80940995)(563.77399263,707.64942034)
\lineto(563.77399263,707.18442034)
\lineto(563.77399263,706.77942034)
\curveto(563.77398965,706.63941112)(563.73898968,706.54441122)(563.66899263,706.49442034)
\curveto(563.60898981,706.44441132)(563.52898989,706.41441135)(563.42899263,706.40442034)
\curveto(563.33899008,706.39441137)(563.23899018,706.38941137)(563.12899263,706.38942034)
\lineto(562.28899263,706.38942034)
\curveto(562.17899124,706.38941137)(562.07899134,706.39441137)(561.98899263,706.40442034)
\curveto(561.90899151,706.41441135)(561.83899158,706.44441132)(561.77899263,706.49442034)
\curveto(561.73899168,706.52441124)(561.70899171,706.57941118)(561.68899263,706.65942034)
\curveto(561.67899174,706.74941101)(561.66899175,706.84441092)(561.65899263,706.94442034)
\lineto(561.65899263,707.27442034)
\curveto(561.66899175,707.38441038)(561.67399175,707.47941028)(561.67399263,707.55942034)
\lineto(561.67399263,707.76942034)
\curveto(561.68399174,707.83940992)(561.70399172,707.89940986)(561.73399263,707.94942034)
\curveto(561.75399167,707.98940977)(561.77899164,708.01940974)(561.80899263,708.03942034)
\lineto(561.92899263,708.09942034)
\curveto(561.94899147,708.09940966)(561.97399145,708.09940966)(562.00399263,708.09942034)
\curveto(562.03399139,708.10940965)(562.05899136,708.11440965)(562.07899263,708.11442034)
\lineto(563.17399263,708.11442034)
\curveto(563.27399015,708.11440965)(563.36899005,708.10940965)(563.45899263,708.09942034)
\curveto(563.54898987,708.08940967)(563.6189898,708.0594097)(563.66899263,708.00942034)
\moveto(563.77399263,698.24442034)
\curveto(563.77398965,698.04441972)(563.76898965,697.87441989)(563.75899263,697.73442034)
\curveto(563.74898967,697.59442017)(563.65898976,697.49942026)(563.48899263,697.44942034)
\curveto(563.42898999,697.42942033)(563.36399006,697.41942034)(563.29399263,697.41942034)
\curveto(563.2239902,697.42942033)(563.14899027,697.43442033)(563.06899263,697.43442034)
\lineto(562.22899263,697.43442034)
\curveto(562.13899128,697.43442033)(562.04899137,697.43942032)(561.95899263,697.44942034)
\curveto(561.87899154,697.4594203)(561.8189916,697.48942027)(561.77899263,697.53942034)
\curveto(561.7189917,697.60942015)(561.68399174,697.69442007)(561.67399263,697.79442034)
\lineto(561.67399263,698.13942034)
\lineto(561.67399263,704.46942034)
\lineto(561.67399263,704.76942034)
\curveto(561.67399175,704.86941289)(561.69399173,704.94941281)(561.73399263,705.00942034)
\curveto(561.79399163,705.07941268)(561.87899154,705.12441264)(561.98899263,705.14442034)
\curveto(562.00899141,705.15441261)(562.03399139,705.15441261)(562.06399263,705.14442034)
\curveto(562.10399132,705.14441262)(562.13399129,705.14941261)(562.15399263,705.15942034)
\lineto(562.90399263,705.15942034)
\lineto(563.09899263,705.15942034)
\curveto(563.17899024,705.16941259)(563.24399018,705.16941259)(563.29399263,705.15942034)
\lineto(563.41399263,705.15942034)
\curveto(563.47398995,705.13941262)(563.52898989,705.12441264)(563.57899263,705.11442034)
\curveto(563.62898979,705.10441266)(563.66898975,705.07441269)(563.69899263,705.02442034)
\curveto(563.73898968,704.97441279)(563.75898966,704.90441286)(563.75899263,704.81442034)
\curveto(563.76898965,704.72441304)(563.77398965,704.62941313)(563.77399263,704.52942034)
\lineto(563.77399263,698.24442034)
}
}
{
\newrgbcolor{curcolor}{0 0 0}
\pscustom[linestyle=none,fillstyle=solid,fillcolor=curcolor]
{
\newpath
\moveto(572.80118013,701.36442034)
\curveto(572.82117197,701.28441648)(572.82117197,701.19441657)(572.80118013,701.09442034)
\curveto(572.78117201,700.99441677)(572.74617204,700.92941683)(572.69618013,700.89942034)
\curveto(572.64617214,700.8594169)(572.57117222,700.82941693)(572.47118013,700.80942034)
\curveto(572.38117241,700.79941696)(572.27617251,700.78941697)(572.15618013,700.77942034)
\lineto(571.81118013,700.77942034)
\curveto(571.70117309,700.78941697)(571.60117319,700.79441697)(571.51118013,700.79442034)
\lineto(567.85118013,700.79442034)
\lineto(567.64118013,700.79442034)
\curveto(567.58117721,700.79441697)(567.52617726,700.78441698)(567.47618013,700.76442034)
\curveto(567.39617739,700.72441704)(567.34617744,700.68441708)(567.32618013,700.64442034)
\curveto(567.30617748,700.62441714)(567.2861775,700.58441718)(567.26618013,700.52442034)
\curveto(567.24617754,700.47441729)(567.24117755,700.42441734)(567.25118013,700.37442034)
\curveto(567.27117752,700.31441745)(567.28117751,700.25441751)(567.28118013,700.19442034)
\curveto(567.2911775,700.14441762)(567.30617748,700.08941767)(567.32618013,700.02942034)
\curveto(567.40617738,699.78941797)(567.50117729,699.58941817)(567.61118013,699.42942034)
\curveto(567.73117706,699.27941848)(567.8911769,699.14441862)(568.09118013,699.02442034)
\curveto(568.17117662,698.97441879)(568.25117654,698.93941882)(568.33118013,698.91942034)
\curveto(568.42117637,698.90941885)(568.51117628,698.88941887)(568.60118013,698.85942034)
\curveto(568.68117611,698.83941892)(568.791176,698.82441894)(568.93118013,698.81442034)
\curveto(569.07117572,698.80441896)(569.1911756,698.80941895)(569.29118013,698.82942034)
\lineto(569.42618013,698.82942034)
\curveto(569.52617526,698.84941891)(569.61617517,698.86941889)(569.69618013,698.88942034)
\curveto(569.786175,698.91941884)(569.87117492,698.94941881)(569.95118013,698.97942034)
\curveto(570.05117474,699.02941873)(570.16117463,699.09441867)(570.28118013,699.17442034)
\curveto(570.41117438,699.25441851)(570.50617428,699.33441843)(570.56618013,699.41442034)
\curveto(570.61617417,699.48441828)(570.66617412,699.54941821)(570.71618013,699.60942034)
\curveto(570.77617401,699.67941808)(570.84617394,699.72941803)(570.92618013,699.75942034)
\curveto(571.02617376,699.80941795)(571.15117364,699.82941793)(571.30118013,699.81942034)
\lineto(571.73618013,699.81942034)
\lineto(571.91618013,699.81942034)
\curveto(571.9861728,699.82941793)(572.04617274,699.82441794)(572.09618013,699.80442034)
\lineto(572.24618013,699.80442034)
\curveto(572.34617244,699.78441798)(572.41617237,699.759418)(572.45618013,699.72942034)
\curveto(572.49617229,699.70941805)(572.51617227,699.6644181)(572.51618013,699.59442034)
\curveto(572.52617226,699.52441824)(572.52117227,699.4644183)(572.50118013,699.41442034)
\curveto(572.45117234,699.27441849)(572.39617239,699.14941861)(572.33618013,699.03942034)
\curveto(572.27617251,698.92941883)(572.20617258,698.81941894)(572.12618013,698.70942034)
\curveto(571.90617288,698.37941938)(571.65617313,698.11441965)(571.37618013,697.91442034)
\curveto(571.09617369,697.71442005)(570.74617404,697.54442022)(570.32618013,697.40442034)
\curveto(570.21617457,697.3644204)(570.10617468,697.33942042)(569.99618013,697.32942034)
\curveto(569.8861749,697.31942044)(569.77117502,697.29942046)(569.65118013,697.26942034)
\curveto(569.61117518,697.2594205)(569.56617522,697.2594205)(569.51618013,697.26942034)
\curveto(569.47617531,697.26942049)(569.43617535,697.2644205)(569.39618013,697.25442034)
\lineto(569.23118013,697.25442034)
\curveto(569.18117561,697.23442053)(569.12117567,697.22942053)(569.05118013,697.23942034)
\curveto(568.9911758,697.23942052)(568.93617585,697.24442052)(568.88618013,697.25442034)
\curveto(568.80617598,697.2644205)(568.73617605,697.2644205)(568.67618013,697.25442034)
\curveto(568.61617617,697.24442052)(568.55117624,697.24942051)(568.48118013,697.26942034)
\curveto(568.43117636,697.28942047)(568.37617641,697.29942046)(568.31618013,697.29942034)
\curveto(568.25617653,697.29942046)(568.20117659,697.30942045)(568.15118013,697.32942034)
\curveto(568.04117675,697.34942041)(567.93117686,697.37442039)(567.82118013,697.40442034)
\curveto(567.71117708,697.42442034)(567.61117718,697.4594203)(567.52118013,697.50942034)
\curveto(567.41117738,697.54942021)(567.30617748,697.58442018)(567.20618013,697.61442034)
\curveto(567.11617767,697.65442011)(567.03117776,697.69942006)(566.95118013,697.74942034)
\curveto(566.63117816,697.94941981)(566.34617844,698.17941958)(566.09618013,698.43942034)
\curveto(565.84617894,698.70941905)(565.64117915,699.01941874)(565.48118013,699.36942034)
\curveto(565.43117936,699.47941828)(565.3911794,699.58941817)(565.36118013,699.69942034)
\curveto(565.33117946,699.81941794)(565.2911795,699.93941782)(565.24118013,700.05942034)
\curveto(565.23117956,700.09941766)(565.22617956,700.13441763)(565.22618013,700.16442034)
\curveto(565.22617956,700.20441756)(565.22117957,700.24441752)(565.21118013,700.28442034)
\curveto(565.17117962,700.40441736)(565.14617964,700.53441723)(565.13618013,700.67442034)
\lineto(565.10618013,701.09442034)
\curveto(565.10617968,701.14441662)(565.10117969,701.19941656)(565.09118013,701.25942034)
\curveto(565.0911797,701.31941644)(565.09617969,701.37441639)(565.10618013,701.42442034)
\lineto(565.10618013,701.60442034)
\lineto(565.15118013,701.96442034)
\curveto(565.1911796,702.13441563)(565.22617956,702.29941546)(565.25618013,702.45942034)
\curveto(565.2861795,702.61941514)(565.33117946,702.76941499)(565.39118013,702.90942034)
\curveto(565.82117897,703.94941381)(566.55117824,704.68441308)(567.58118013,705.11442034)
\curveto(567.72117707,705.17441259)(567.86117693,705.21441255)(568.00118013,705.23442034)
\curveto(568.15117664,705.2644125)(568.30617648,705.29941246)(568.46618013,705.33942034)
\curveto(568.54617624,705.34941241)(568.62117617,705.35441241)(568.69118013,705.35442034)
\curveto(568.76117603,705.35441241)(568.83617595,705.3594124)(568.91618013,705.36942034)
\curveto(569.42617536,705.37941238)(569.86117493,705.31941244)(570.22118013,705.18942034)
\curveto(570.5911742,705.06941269)(570.92117387,704.90941285)(571.21118013,704.70942034)
\curveto(571.30117349,704.64941311)(571.3911734,704.57941318)(571.48118013,704.49942034)
\curveto(571.57117322,704.42941333)(571.65117314,704.35441341)(571.72118013,704.27442034)
\curveto(571.75117304,704.22441354)(571.791173,704.18441358)(571.84118013,704.15442034)
\curveto(571.92117287,704.04441372)(571.99617279,703.92941383)(572.06618013,703.80942034)
\curveto(572.13617265,703.69941406)(572.21117258,703.58441418)(572.29118013,703.46442034)
\curveto(572.34117245,703.37441439)(572.38117241,703.27941448)(572.41118013,703.17942034)
\curveto(572.45117234,703.08941467)(572.4911723,702.98941477)(572.53118013,702.87942034)
\curveto(572.58117221,702.74941501)(572.62117217,702.61441515)(572.65118013,702.47442034)
\curveto(572.68117211,702.33441543)(572.71617207,702.19441557)(572.75618013,702.05442034)
\curveto(572.77617201,701.97441579)(572.78117201,701.88441588)(572.77118013,701.78442034)
\curveto(572.77117202,701.69441607)(572.78117201,701.60941615)(572.80118013,701.52942034)
\lineto(572.80118013,701.36442034)
\moveto(570.55118013,702.24942034)
\curveto(570.62117417,702.34941541)(570.62617416,702.46941529)(570.56618013,702.60942034)
\curveto(570.51617427,702.759415)(570.47617431,702.86941489)(570.44618013,702.93942034)
\curveto(570.30617448,703.20941455)(570.12117467,703.41441435)(569.89118013,703.55442034)
\curveto(569.66117513,703.70441406)(569.34117545,703.78441398)(568.93118013,703.79442034)
\curveto(568.90117589,703.77441399)(568.86617592,703.76941399)(568.82618013,703.77942034)
\curveto(568.786176,703.78941397)(568.75117604,703.78941397)(568.72118013,703.77942034)
\curveto(568.67117612,703.759414)(568.61617617,703.74441402)(568.55618013,703.73442034)
\curveto(568.49617629,703.73441403)(568.44117635,703.72441404)(568.39118013,703.70442034)
\curveto(567.95117684,703.5644142)(567.62617716,703.28941447)(567.41618013,702.87942034)
\curveto(567.39617739,702.83941492)(567.37117742,702.78441498)(567.34118013,702.71442034)
\curveto(567.32117747,702.65441511)(567.30617748,702.58941517)(567.29618013,702.51942034)
\curveto(567.2861775,702.4594153)(567.2861775,702.39941536)(567.29618013,702.33942034)
\curveto(567.31617747,702.27941548)(567.35117744,702.22941553)(567.40118013,702.18942034)
\curveto(567.48117731,702.13941562)(567.5911772,702.11441565)(567.73118013,702.11442034)
\lineto(568.13618013,702.11442034)
\lineto(569.80118013,702.11442034)
\lineto(570.23618013,702.11442034)
\curveto(570.39617439,702.12441564)(570.50117429,702.16941559)(570.55118013,702.24942034)
}
}
{
\newrgbcolor{curcolor}{0 0 0}
\pscustom[linestyle=none,fillstyle=solid,fillcolor=curcolor]
{
\newpath
\moveto(578.47446138,705.35442034)
\curveto(579.07445558,705.37441239)(579.57445508,705.28941247)(579.97446138,705.09942034)
\curveto(580.37445428,704.90941285)(580.68945396,704.62941313)(580.91946138,704.25942034)
\curveto(580.98945366,704.14941361)(581.04445361,704.02941373)(581.08446138,703.89942034)
\curveto(581.12445353,703.77941398)(581.16445349,703.65441411)(581.20446138,703.52442034)
\curveto(581.22445343,703.44441432)(581.23445342,703.36941439)(581.23446138,703.29942034)
\curveto(581.24445341,703.22941453)(581.25945339,703.1594146)(581.27946138,703.08942034)
\curveto(581.27945337,703.02941473)(581.28445337,702.98941477)(581.29446138,702.96942034)
\curveto(581.31445334,702.82941493)(581.32445333,702.68441508)(581.32446138,702.53442034)
\lineto(581.32446138,702.09942034)
\lineto(581.32446138,700.76442034)
\lineto(581.32446138,698.33442034)
\curveto(581.32445333,698.14441962)(581.31945333,697.9594198)(581.30946138,697.77942034)
\curveto(581.30945334,697.60942015)(581.23945341,697.49942026)(581.09946138,697.44942034)
\curveto(581.03945361,697.42942033)(580.96945368,697.41942034)(580.88946138,697.41942034)
\lineto(580.64946138,697.41942034)
\lineto(579.83946138,697.41942034)
\curveto(579.71945493,697.41942034)(579.60945504,697.42442034)(579.50946138,697.43442034)
\curveto(579.41945523,697.45442031)(579.3494553,697.49942026)(579.29946138,697.56942034)
\curveto(579.25945539,697.62942013)(579.23445542,697.70442006)(579.22446138,697.79442034)
\lineto(579.22446138,698.10942034)
\lineto(579.22446138,699.15942034)
\lineto(579.22446138,701.39442034)
\curveto(579.22445543,701.764416)(579.20945544,702.10441566)(579.17946138,702.41442034)
\curveto(579.1494555,702.73441503)(579.05945559,703.00441476)(578.90946138,703.22442034)
\curveto(578.76945588,703.42441434)(578.56445609,703.5644142)(578.29446138,703.64442034)
\curveto(578.24445641,703.6644141)(578.18945646,703.67441409)(578.12946138,703.67442034)
\curveto(578.07945657,703.67441409)(578.02445663,703.68441408)(577.96446138,703.70442034)
\curveto(577.91445674,703.71441405)(577.8494568,703.71441405)(577.76946138,703.70442034)
\curveto(577.69945695,703.70441406)(577.64445701,703.69941406)(577.60446138,703.68942034)
\curveto(577.56445709,703.67941408)(577.52945712,703.67441409)(577.49946138,703.67442034)
\curveto(577.46945718,703.67441409)(577.43945721,703.66941409)(577.40946138,703.65942034)
\curveto(577.17945747,703.59941416)(576.99445766,703.51941424)(576.85446138,703.41942034)
\curveto(576.53445812,703.18941457)(576.34445831,702.85441491)(576.28446138,702.41442034)
\curveto(576.22445843,701.97441579)(576.19445846,701.47941628)(576.19446138,700.92942034)
\lineto(576.19446138,699.05442034)
\lineto(576.19446138,698.13942034)
\lineto(576.19446138,697.86942034)
\curveto(576.19445846,697.77941998)(576.17945847,697.70442006)(576.14946138,697.64442034)
\curveto(576.09945855,697.53442023)(576.01945863,697.46942029)(575.90946138,697.44942034)
\curveto(575.79945885,697.42942033)(575.66445899,697.41942034)(575.50446138,697.41942034)
\lineto(574.75446138,697.41942034)
\curveto(574.64446001,697.41942034)(574.53446012,697.42442034)(574.42446138,697.43442034)
\curveto(574.31446034,697.44442032)(574.23446042,697.47942028)(574.18446138,697.53942034)
\curveto(574.11446054,697.62942013)(574.07946057,697.75942)(574.07946138,697.92942034)
\curveto(574.08946056,698.09941966)(574.09446056,698.2594195)(574.09446138,698.40942034)
\lineto(574.09446138,700.44942034)
\lineto(574.09446138,703.74942034)
\lineto(574.09446138,704.51442034)
\lineto(574.09446138,704.81442034)
\curveto(574.10446055,704.90441286)(574.13446052,704.97941278)(574.18446138,705.03942034)
\curveto(574.20446045,705.06941269)(574.23446042,705.08941267)(574.27446138,705.09942034)
\curveto(574.32446033,705.11941264)(574.37446028,705.13441263)(574.42446138,705.14442034)
\lineto(574.49946138,705.14442034)
\curveto(574.5494601,705.15441261)(574.59946005,705.1594126)(574.64946138,705.15942034)
\lineto(574.81446138,705.15942034)
\lineto(575.44446138,705.15942034)
\curveto(575.52445913,705.1594126)(575.59945905,705.15441261)(575.66946138,705.14442034)
\curveto(575.7494589,705.14441262)(575.81945883,705.13441263)(575.87946138,705.11442034)
\curveto(575.9494587,705.08441268)(575.99445866,705.03941272)(576.01446138,704.97942034)
\curveto(576.04445861,704.91941284)(576.06945858,704.84941291)(576.08946138,704.76942034)
\curveto(576.09945855,704.72941303)(576.09945855,704.69441307)(576.08946138,704.66442034)
\curveto(576.08945856,704.63441313)(576.09945855,704.60441316)(576.11946138,704.57442034)
\curveto(576.13945851,704.52441324)(576.1544585,704.49441327)(576.16446138,704.48442034)
\curveto(576.18445847,704.47441329)(576.20945844,704.4594133)(576.23946138,704.43942034)
\curveto(576.3494583,704.42941333)(576.43945821,704.4644133)(576.50946138,704.54442034)
\curveto(576.57945807,704.63441313)(576.654458,704.70441306)(576.73446138,704.75442034)
\curveto(577.00445765,704.95441281)(577.30445735,705.11441265)(577.63446138,705.23442034)
\curveto(577.72445693,705.2644125)(577.81445684,705.28441248)(577.90446138,705.29442034)
\curveto(578.00445665,705.30441246)(578.10945654,705.31941244)(578.21946138,705.33942034)
\curveto(578.2494564,705.34941241)(578.29445636,705.34941241)(578.35446138,705.33942034)
\curveto(578.41445624,705.33941242)(578.4544562,705.34441242)(578.47446138,705.35442034)
}
}
{
\newrgbcolor{curcolor}{0 0 0}
\pscustom[linestyle=none,fillstyle=solid,fillcolor=curcolor]
{
\newpath
\moveto(586.52571138,705.36942034)
\curveto(587.33570622,705.38941237)(588.01070555,705.26941249)(588.55071138,705.00942034)
\curveto(589.10070446,704.74941301)(589.53570402,704.37941338)(589.85571138,703.89942034)
\curveto(590.01570354,703.6594141)(590.13570342,703.38441438)(590.21571138,703.07442034)
\curveto(590.23570332,703.02441474)(590.25070331,702.9594148)(590.26071138,702.87942034)
\curveto(590.28070328,702.79941496)(590.28070328,702.72941503)(590.26071138,702.66942034)
\curveto(590.22070334,702.5594152)(590.15070341,702.49441527)(590.05071138,702.47442034)
\curveto(589.95070361,702.4644153)(589.83070373,702.4594153)(589.69071138,702.45942034)
\lineto(588.91071138,702.45942034)
\lineto(588.62571138,702.45942034)
\curveto(588.53570502,702.4594153)(588.4607051,702.47941528)(588.40071138,702.51942034)
\curveto(588.32070524,702.5594152)(588.26570529,702.61941514)(588.23571138,702.69942034)
\curveto(588.20570535,702.78941497)(588.16570539,702.87941488)(588.11571138,702.96942034)
\curveto(588.0557055,703.07941468)(587.99070557,703.17941458)(587.92071138,703.26942034)
\curveto(587.85070571,703.3594144)(587.77070579,703.43941432)(587.68071138,703.50942034)
\curveto(587.54070602,703.59941416)(587.38570617,703.66941409)(587.21571138,703.71942034)
\curveto(587.1557064,703.73941402)(587.09570646,703.74941401)(587.03571138,703.74942034)
\curveto(586.97570658,703.74941401)(586.92070664,703.759414)(586.87071138,703.77942034)
\lineto(586.72071138,703.77942034)
\curveto(586.52070704,703.77941398)(586.3607072,703.759414)(586.24071138,703.71942034)
\curveto(585.95070761,703.62941413)(585.71570784,703.48941427)(585.53571138,703.29942034)
\curveto(585.3557082,703.11941464)(585.21070835,702.89941486)(585.10071138,702.63942034)
\curveto(585.05070851,702.52941523)(585.01070855,702.40941535)(584.98071138,702.27942034)
\curveto(584.9607086,702.1594156)(584.93570862,702.02941573)(584.90571138,701.88942034)
\curveto(584.89570866,701.84941591)(584.89070867,701.80941595)(584.89071138,701.76942034)
\curveto(584.89070867,701.72941603)(584.88570867,701.68941607)(584.87571138,701.64942034)
\curveto(584.8557087,701.54941621)(584.84570871,701.40941635)(584.84571138,701.22942034)
\curveto(584.8557087,701.04941671)(584.87070869,700.90941685)(584.89071138,700.80942034)
\curveto(584.89070867,700.72941703)(584.89570866,700.67441709)(584.90571138,700.64442034)
\curveto(584.92570863,700.57441719)(584.93570862,700.50441726)(584.93571138,700.43442034)
\curveto(584.94570861,700.3644174)(584.9607086,700.29441747)(584.98071138,700.22442034)
\curveto(585.0607085,699.99441777)(585.1557084,699.78441798)(585.26571138,699.59442034)
\curveto(585.37570818,699.40441836)(585.51570804,699.24441852)(585.68571138,699.11442034)
\curveto(585.72570783,699.08441868)(585.78570777,699.04941871)(585.86571138,699.00942034)
\curveto(585.97570758,698.93941882)(586.08570747,698.89441887)(586.19571138,698.87442034)
\curveto(586.31570724,698.85441891)(586.4607071,698.83441893)(586.63071138,698.81442034)
\lineto(586.72071138,698.81442034)
\curveto(586.7607068,698.81441895)(586.79070677,698.81941894)(586.81071138,698.82942034)
\lineto(586.94571138,698.82942034)
\curveto(587.01570654,698.84941891)(587.08070648,698.8644189)(587.14071138,698.87442034)
\curveto(587.21070635,698.89441887)(587.27570628,698.91441885)(587.33571138,698.93442034)
\curveto(587.63570592,699.0644187)(587.86570569,699.25441851)(588.02571138,699.50442034)
\curveto(588.06570549,699.55441821)(588.10070546,699.60941815)(588.13071138,699.66942034)
\curveto(588.1607054,699.73941802)(588.18570537,699.79941796)(588.20571138,699.84942034)
\curveto(588.24570531,699.9594178)(588.28070528,700.05441771)(588.31071138,700.13442034)
\curveto(588.34070522,700.22441754)(588.41070515,700.29441747)(588.52071138,700.34442034)
\curveto(588.61070495,700.38441738)(588.7557048,700.39941736)(588.95571138,700.38942034)
\lineto(589.45071138,700.38942034)
\lineto(589.66071138,700.38942034)
\curveto(589.74070382,700.39941736)(589.80570375,700.39441737)(589.85571138,700.37442034)
\lineto(589.97571138,700.37442034)
\lineto(590.09571138,700.34442034)
\curveto(590.13570342,700.34441742)(590.16570339,700.33441743)(590.18571138,700.31442034)
\curveto(590.23570332,700.27441749)(590.26570329,700.21441755)(590.27571138,700.13442034)
\curveto(590.29570326,700.0644177)(590.29570326,699.98941777)(590.27571138,699.90942034)
\curveto(590.18570337,699.57941818)(590.07570348,699.28441848)(589.94571138,699.02442034)
\curveto(589.53570402,698.25441951)(588.88070468,697.71942004)(587.98071138,697.41942034)
\curveto(587.88070568,697.38942037)(587.77570578,697.36942039)(587.66571138,697.35942034)
\curveto(587.555706,697.33942042)(587.44570611,697.31442045)(587.33571138,697.28442034)
\curveto(587.27570628,697.27442049)(587.21570634,697.26942049)(587.15571138,697.26942034)
\curveto(587.09570646,697.26942049)(587.03570652,697.2644205)(586.97571138,697.25442034)
\lineto(586.81071138,697.25442034)
\curveto(586.7607068,697.23442053)(586.68570687,697.22942053)(586.58571138,697.23942034)
\curveto(586.48570707,697.23942052)(586.41070715,697.24442052)(586.36071138,697.25442034)
\curveto(586.28070728,697.27442049)(586.20570735,697.28442048)(586.13571138,697.28442034)
\curveto(586.07570748,697.27442049)(586.01070755,697.27942048)(585.94071138,697.29942034)
\lineto(585.79071138,697.32942034)
\curveto(585.74070782,697.32942043)(585.69070787,697.33442043)(585.64071138,697.34442034)
\curveto(585.53070803,697.37442039)(585.42570813,697.40442036)(585.32571138,697.43442034)
\curveto(585.22570833,697.4644203)(585.13070843,697.49942026)(585.04071138,697.53942034)
\curveto(584.57070899,697.73942002)(584.17570938,697.99441977)(583.85571138,698.30442034)
\curveto(583.53571002,698.62441914)(583.27571028,699.01941874)(583.07571138,699.48942034)
\curveto(583.02571053,699.57941818)(582.98571057,699.67441809)(582.95571138,699.77442034)
\lineto(582.86571138,700.10442034)
\curveto(582.8557107,700.14441762)(582.85071071,700.17941758)(582.85071138,700.20942034)
\curveto(582.85071071,700.24941751)(582.84071072,700.29441747)(582.82071138,700.34442034)
\curveto(582.80071076,700.41441735)(582.79071077,700.48441728)(582.79071138,700.55442034)
\curveto(582.79071077,700.63441713)(582.78071078,700.70941705)(582.76071138,700.77942034)
\lineto(582.76071138,701.03442034)
\curveto(582.74071082,701.08441668)(582.73071083,701.13941662)(582.73071138,701.19942034)
\curveto(582.73071083,701.26941649)(582.74071082,701.32941643)(582.76071138,701.37942034)
\curveto(582.77071079,701.42941633)(582.77071079,701.47441629)(582.76071138,701.51442034)
\curveto(582.75071081,701.55441621)(582.75071081,701.59441617)(582.76071138,701.63442034)
\curveto(582.78071078,701.70441606)(582.78571077,701.76941599)(582.77571138,701.82942034)
\curveto(582.77571078,701.88941587)(582.78571077,701.94941581)(582.80571138,702.00942034)
\curveto(582.8557107,702.18941557)(582.89571066,702.3594154)(582.92571138,702.51942034)
\curveto(582.9557106,702.68941507)(583.00071056,702.85441491)(583.06071138,703.01442034)
\curveto(583.28071028,703.52441424)(583.55571,703.94941381)(583.88571138,704.28942034)
\curveto(584.22570933,704.62941313)(584.6557089,704.90441286)(585.17571138,705.11442034)
\curveto(585.31570824,705.17441259)(585.4607081,705.21441255)(585.61071138,705.23442034)
\curveto(585.7607078,705.2644125)(585.91570764,705.29941246)(586.07571138,705.33942034)
\curveto(586.1557074,705.34941241)(586.23070733,705.35441241)(586.30071138,705.35442034)
\curveto(586.37070719,705.35441241)(586.44570711,705.3594124)(586.52571138,705.36942034)
}
}
{
\newrgbcolor{curcolor}{0 0 0}
\pscustom[linestyle=none,fillstyle=solid,fillcolor=curcolor]
{
\newpath
\moveto(593.66899263,708.00942034)
\curveto(593.73898968,707.92940983)(593.77398965,707.80940995)(593.77399263,707.64942034)
\lineto(593.77399263,707.18442034)
\lineto(593.77399263,706.77942034)
\curveto(593.77398965,706.63941112)(593.73898968,706.54441122)(593.66899263,706.49442034)
\curveto(593.60898981,706.44441132)(593.52898989,706.41441135)(593.42899263,706.40442034)
\curveto(593.33899008,706.39441137)(593.23899018,706.38941137)(593.12899263,706.38942034)
\lineto(592.28899263,706.38942034)
\curveto(592.17899124,706.38941137)(592.07899134,706.39441137)(591.98899263,706.40442034)
\curveto(591.90899151,706.41441135)(591.83899158,706.44441132)(591.77899263,706.49442034)
\curveto(591.73899168,706.52441124)(591.70899171,706.57941118)(591.68899263,706.65942034)
\curveto(591.67899174,706.74941101)(591.66899175,706.84441092)(591.65899263,706.94442034)
\lineto(591.65899263,707.27442034)
\curveto(591.66899175,707.38441038)(591.67399175,707.47941028)(591.67399263,707.55942034)
\lineto(591.67399263,707.76942034)
\curveto(591.68399174,707.83940992)(591.70399172,707.89940986)(591.73399263,707.94942034)
\curveto(591.75399167,707.98940977)(591.77899164,708.01940974)(591.80899263,708.03942034)
\lineto(591.92899263,708.09942034)
\curveto(591.94899147,708.09940966)(591.97399145,708.09940966)(592.00399263,708.09942034)
\curveto(592.03399139,708.10940965)(592.05899136,708.11440965)(592.07899263,708.11442034)
\lineto(593.17399263,708.11442034)
\curveto(593.27399015,708.11440965)(593.36899005,708.10940965)(593.45899263,708.09942034)
\curveto(593.54898987,708.08940967)(593.6189898,708.0594097)(593.66899263,708.00942034)
\moveto(593.77399263,698.24442034)
\curveto(593.77398965,698.04441972)(593.76898965,697.87441989)(593.75899263,697.73442034)
\curveto(593.74898967,697.59442017)(593.65898976,697.49942026)(593.48899263,697.44942034)
\curveto(593.42898999,697.42942033)(593.36399006,697.41942034)(593.29399263,697.41942034)
\curveto(593.2239902,697.42942033)(593.14899027,697.43442033)(593.06899263,697.43442034)
\lineto(592.22899263,697.43442034)
\curveto(592.13899128,697.43442033)(592.04899137,697.43942032)(591.95899263,697.44942034)
\curveto(591.87899154,697.4594203)(591.8189916,697.48942027)(591.77899263,697.53942034)
\curveto(591.7189917,697.60942015)(591.68399174,697.69442007)(591.67399263,697.79442034)
\lineto(591.67399263,698.13942034)
\lineto(591.67399263,704.46942034)
\lineto(591.67399263,704.76942034)
\curveto(591.67399175,704.86941289)(591.69399173,704.94941281)(591.73399263,705.00942034)
\curveto(591.79399163,705.07941268)(591.87899154,705.12441264)(591.98899263,705.14442034)
\curveto(592.00899141,705.15441261)(592.03399139,705.15441261)(592.06399263,705.14442034)
\curveto(592.10399132,705.14441262)(592.13399129,705.14941261)(592.15399263,705.15942034)
\lineto(592.90399263,705.15942034)
\lineto(593.09899263,705.15942034)
\curveto(593.17899024,705.16941259)(593.24399018,705.16941259)(593.29399263,705.15942034)
\lineto(593.41399263,705.15942034)
\curveto(593.47398995,705.13941262)(593.52898989,705.12441264)(593.57899263,705.11442034)
\curveto(593.62898979,705.10441266)(593.66898975,705.07441269)(593.69899263,705.02442034)
\curveto(593.73898968,704.97441279)(593.75898966,704.90441286)(593.75899263,704.81442034)
\curveto(593.76898965,704.72441304)(593.77398965,704.62941313)(593.77399263,704.52942034)
\lineto(593.77399263,698.24442034)
}
}
{
\newrgbcolor{curcolor}{0 0 0}
\pscustom[linestyle=none,fillstyle=solid,fillcolor=curcolor]
{
\newpath
\moveto(602.48618013,698.01942034)
\curveto(602.50617228,697.90941985)(602.51617227,697.79941996)(602.51618013,697.68942034)
\curveto(602.52617226,697.57942018)(602.47617231,697.50442026)(602.36618013,697.46442034)
\curveto(602.30617248,697.43442033)(602.23617255,697.41942034)(602.15618013,697.41942034)
\lineto(601.91618013,697.41942034)
\lineto(601.10618013,697.41942034)
\lineto(600.83618013,697.41942034)
\curveto(600.75617403,697.42942033)(600.6911741,697.45442031)(600.64118013,697.49442034)
\curveto(600.57117422,697.53442023)(600.51617427,697.58942017)(600.47618013,697.65942034)
\curveto(600.44617434,697.73942002)(600.40117439,697.80441996)(600.34118013,697.85442034)
\curveto(600.32117447,697.87441989)(600.29617449,697.88941987)(600.26618013,697.89942034)
\curveto(600.23617455,697.91941984)(600.19617459,697.92441984)(600.14618013,697.91442034)
\curveto(600.09617469,697.89441987)(600.04617474,697.86941989)(599.99618013,697.83942034)
\curveto(599.95617483,697.80941995)(599.91117488,697.78441998)(599.86118013,697.76442034)
\curveto(599.81117498,697.72442004)(599.75617503,697.68942007)(599.69618013,697.65942034)
\lineto(599.51618013,697.56942034)
\curveto(599.3861754,697.50942025)(599.25117554,697.4594203)(599.11118013,697.41942034)
\curveto(598.97117582,697.38942037)(598.82617596,697.35442041)(598.67618013,697.31442034)
\curveto(598.60617618,697.29442047)(598.53617625,697.28442048)(598.46618013,697.28442034)
\curveto(598.40617638,697.27442049)(598.34117645,697.2644205)(598.27118013,697.25442034)
\lineto(598.18118013,697.25442034)
\curveto(598.15117664,697.24442052)(598.12117667,697.23942052)(598.09118013,697.23942034)
\lineto(597.92618013,697.23942034)
\curveto(597.82617696,697.21942054)(597.72617706,697.21942054)(597.62618013,697.23942034)
\lineto(597.49118013,697.23942034)
\curveto(597.42117737,697.2594205)(597.35117744,697.26942049)(597.28118013,697.26942034)
\curveto(597.22117757,697.2594205)(597.16117763,697.2644205)(597.10118013,697.28442034)
\curveto(597.00117779,697.30442046)(596.90617788,697.32442044)(596.81618013,697.34442034)
\curveto(596.72617806,697.35442041)(596.64117815,697.37942038)(596.56118013,697.41942034)
\curveto(596.27117852,697.52942023)(596.02117877,697.66942009)(595.81118013,697.83942034)
\curveto(595.61117918,698.01941974)(595.45117934,698.25441951)(595.33118013,698.54442034)
\curveto(595.30117949,698.61441915)(595.27117952,698.68941907)(595.24118013,698.76942034)
\curveto(595.22117957,698.84941891)(595.20117959,698.93441883)(595.18118013,699.02442034)
\curveto(595.16117963,699.07441869)(595.15117964,699.12441864)(595.15118013,699.17442034)
\curveto(595.16117963,699.22441854)(595.16117963,699.27441849)(595.15118013,699.32442034)
\curveto(595.14117965,699.35441841)(595.13117966,699.41441835)(595.12118013,699.50442034)
\curveto(595.12117967,699.60441816)(595.12617966,699.67441809)(595.13618013,699.71442034)
\curveto(595.15617963,699.81441795)(595.16617962,699.89941786)(595.16618013,699.96942034)
\lineto(595.25618013,700.29942034)
\curveto(595.2861795,700.41941734)(595.32617946,700.52441724)(595.37618013,700.61442034)
\curveto(595.54617924,700.90441686)(595.74117905,701.12441664)(595.96118013,701.27442034)
\curveto(596.18117861,701.42441634)(596.46117833,701.55441621)(596.80118013,701.66442034)
\curveto(596.93117786,701.71441605)(597.06617772,701.74941601)(597.20618013,701.76942034)
\curveto(597.34617744,701.78941597)(597.4861773,701.81441595)(597.62618013,701.84442034)
\curveto(597.70617708,701.8644159)(597.791177,701.87441589)(597.88118013,701.87442034)
\curveto(597.97117682,701.88441588)(598.06117673,701.89941586)(598.15118013,701.91942034)
\curveto(598.22117657,701.93941582)(598.2911765,701.94441582)(598.36118013,701.93442034)
\curveto(598.43117636,701.93441583)(598.50617628,701.94441582)(598.58618013,701.96442034)
\curveto(598.65617613,701.98441578)(598.72617606,701.99441577)(598.79618013,701.99442034)
\curveto(598.86617592,701.99441577)(598.94117585,702.00441576)(599.02118013,702.02442034)
\curveto(599.23117556,702.07441569)(599.42117537,702.11441565)(599.59118013,702.14442034)
\curveto(599.77117502,702.18441558)(599.93117486,702.27441549)(600.07118013,702.41442034)
\curveto(600.16117463,702.50441526)(600.22117457,702.60441516)(600.25118013,702.71442034)
\curveto(600.26117453,702.74441502)(600.26117453,702.76941499)(600.25118013,702.78942034)
\curveto(600.25117454,702.80941495)(600.25617453,702.82941493)(600.26618013,702.84942034)
\curveto(600.27617451,702.86941489)(600.28117451,702.89941486)(600.28118013,702.93942034)
\lineto(600.28118013,703.02942034)
\lineto(600.25118013,703.14942034)
\curveto(600.25117454,703.18941457)(600.24617454,703.22441454)(600.23618013,703.25442034)
\curveto(600.13617465,703.55441421)(599.92617486,703.759414)(599.60618013,703.86942034)
\curveto(599.51617527,703.89941386)(599.40617538,703.91941384)(599.27618013,703.92942034)
\curveto(599.15617563,703.94941381)(599.03117576,703.95441381)(598.90118013,703.94442034)
\curveto(598.77117602,703.94441382)(598.64617614,703.93441383)(598.52618013,703.91442034)
\curveto(598.40617638,703.89441387)(598.30117649,703.86941389)(598.21118013,703.83942034)
\curveto(598.15117664,703.81941394)(598.0911767,703.78941397)(598.03118013,703.74942034)
\curveto(597.98117681,703.71941404)(597.93117686,703.68441408)(597.88118013,703.64442034)
\curveto(597.83117696,703.60441416)(597.77617701,703.54941421)(597.71618013,703.47942034)
\curveto(597.66617712,703.40941435)(597.63117716,703.34441442)(597.61118013,703.28442034)
\curveto(597.56117723,703.18441458)(597.51617727,703.08941467)(597.47618013,702.99942034)
\curveto(597.44617734,702.90941485)(597.37617741,702.84941491)(597.26618013,702.81942034)
\curveto(597.1861776,702.79941496)(597.10117769,702.78941497)(597.01118013,702.78942034)
\lineto(596.74118013,702.78942034)
\lineto(596.17118013,702.78942034)
\curveto(596.12117867,702.78941497)(596.07117872,702.78441498)(596.02118013,702.77442034)
\curveto(595.97117882,702.77441499)(595.92617886,702.77941498)(595.88618013,702.78942034)
\lineto(595.75118013,702.78942034)
\curveto(595.73117906,702.79941496)(595.70617908,702.80441496)(595.67618013,702.80442034)
\curveto(595.64617914,702.80441496)(595.62117917,702.81441495)(595.60118013,702.83442034)
\curveto(595.52117927,702.85441491)(595.46617932,702.91941484)(595.43618013,703.02942034)
\curveto(595.42617936,703.07941468)(595.42617936,703.12941463)(595.43618013,703.17942034)
\curveto(595.44617934,703.22941453)(595.45617933,703.27441449)(595.46618013,703.31442034)
\curveto(595.49617929,703.42441434)(595.52617926,703.52441424)(595.55618013,703.61442034)
\curveto(595.59617919,703.71441405)(595.64117915,703.80441396)(595.69118013,703.88442034)
\lineto(595.78118013,704.03442034)
\lineto(595.87118013,704.18442034)
\curveto(595.95117884,704.29441347)(596.05117874,704.39941336)(596.17118013,704.49942034)
\curveto(596.1911786,704.50941325)(596.22117857,704.53441323)(596.26118013,704.57442034)
\curveto(596.31117848,704.61441315)(596.35617843,704.64941311)(596.39618013,704.67942034)
\curveto(596.43617835,704.70941305)(596.48117831,704.73941302)(596.53118013,704.76942034)
\curveto(596.70117809,704.87941288)(596.88117791,704.9644128)(597.07118013,705.02442034)
\curveto(597.26117753,705.09441267)(597.45617733,705.1594126)(597.65618013,705.21942034)
\curveto(597.77617701,705.24941251)(597.90117689,705.26941249)(598.03118013,705.27942034)
\curveto(598.16117663,705.28941247)(598.2911765,705.30941245)(598.42118013,705.33942034)
\curveto(598.46117633,705.34941241)(598.52117627,705.34941241)(598.60118013,705.33942034)
\curveto(598.6911761,705.32941243)(598.74617604,705.33441243)(598.76618013,705.35442034)
\curveto(599.17617561,705.3644124)(599.56617522,705.34941241)(599.93618013,705.30942034)
\curveto(600.31617447,705.26941249)(600.65617413,705.19441257)(600.95618013,705.08442034)
\curveto(601.26617352,704.97441279)(601.53117326,704.82441294)(601.75118013,704.63442034)
\curveto(601.97117282,704.45441331)(602.14117265,704.21941354)(602.26118013,703.92942034)
\curveto(602.33117246,703.759414)(602.37117242,703.5644142)(602.38118013,703.34442034)
\curveto(602.3911724,703.12441464)(602.39617239,702.89941486)(602.39618013,702.66942034)
\lineto(602.39618013,699.32442034)
\lineto(602.39618013,698.73942034)
\curveto(602.39617239,698.54941921)(602.41617237,698.37441939)(602.45618013,698.21442034)
\curveto(602.46617232,698.18441958)(602.47117232,698.14941961)(602.47118013,698.10942034)
\curveto(602.47117232,698.07941968)(602.47617231,698.04941971)(602.48618013,698.01942034)
\moveto(600.28118013,700.32942034)
\curveto(600.2911745,700.37941738)(600.29617449,700.43441733)(600.29618013,700.49442034)
\curveto(600.29617449,700.5644172)(600.2911745,700.62441714)(600.28118013,700.67442034)
\curveto(600.26117453,700.73441703)(600.25117454,700.78941697)(600.25118013,700.83942034)
\curveto(600.25117454,700.88941687)(600.23117456,700.92941683)(600.19118013,700.95942034)
\curveto(600.14117465,700.99941676)(600.06617472,701.01941674)(599.96618013,701.01942034)
\curveto(599.92617486,701.00941675)(599.8911749,700.99941676)(599.86118013,700.98942034)
\curveto(599.83117496,700.98941677)(599.79617499,700.98441678)(599.75618013,700.97442034)
\curveto(599.6861751,700.95441681)(599.61117518,700.93941682)(599.53118013,700.92942034)
\curveto(599.45117534,700.91941684)(599.37117542,700.90441686)(599.29118013,700.88442034)
\curveto(599.26117553,700.87441689)(599.21617557,700.86941689)(599.15618013,700.86942034)
\curveto(599.02617576,700.83941692)(598.89617589,700.81941694)(598.76618013,700.80942034)
\curveto(598.63617615,700.79941696)(598.51117628,700.77441699)(598.39118013,700.73442034)
\curveto(598.31117648,700.71441705)(598.23617655,700.69441707)(598.16618013,700.67442034)
\curveto(598.09617669,700.6644171)(598.02617676,700.64441712)(597.95618013,700.61442034)
\curveto(597.74617704,700.52441724)(597.56617722,700.38941737)(597.41618013,700.20942034)
\curveto(597.27617751,700.02941773)(597.22617756,699.77941798)(597.26618013,699.45942034)
\curveto(597.2861775,699.28941847)(597.34117745,699.14941861)(597.43118013,699.03942034)
\curveto(597.50117729,698.92941883)(597.60617718,698.83941892)(597.74618013,698.76942034)
\curveto(597.8861769,698.70941905)(598.03617675,698.6644191)(598.19618013,698.63442034)
\curveto(598.36617642,698.60441916)(598.54117625,698.59441917)(598.72118013,698.60442034)
\curveto(598.91117588,698.62441914)(599.0861757,698.6594191)(599.24618013,698.70942034)
\curveto(599.50617528,698.78941897)(599.71117508,698.91441885)(599.86118013,699.08442034)
\curveto(600.01117478,699.2644185)(600.12617466,699.48441828)(600.20618013,699.74442034)
\curveto(600.22617456,699.81441795)(600.23617455,699.88441788)(600.23618013,699.95442034)
\curveto(600.24617454,700.03441773)(600.26117453,700.11441765)(600.28118013,700.19442034)
\lineto(600.28118013,700.32942034)
}
}
{
\newrgbcolor{curcolor}{0 0 0}
\pscustom[linestyle=none,fillstyle=solid,fillcolor=curcolor]
{
\newpath
\moveto(413.39453951,682.62942034)
\curveto(413.40453083,682.56941644)(413.40953082,682.47941653)(413.40953951,682.35942034)
\curveto(413.40953082,682.23941677)(413.39953083,682.15441686)(413.37953951,682.10442034)
\lineto(413.37953951,681.90942034)
\curveto(413.34953088,681.79941721)(413.3295309,681.69441732)(413.31953951,681.59442034)
\curveto(413.31953091,681.49441752)(413.30453093,681.39441762)(413.27453951,681.29442034)
\curveto(413.25453098,681.20441781)(413.234531,681.1094179)(413.21453951,681.00942034)
\curveto(413.19453104,680.91941809)(413.16453107,680.82941818)(413.12453951,680.73942034)
\curveto(413.05453118,680.56941844)(412.98453125,680.4094186)(412.91453951,680.25942034)
\curveto(412.84453139,680.11941889)(412.76453147,679.97941903)(412.67453951,679.83942034)
\curveto(412.61453162,679.74941926)(412.54953168,679.66441935)(412.47953951,679.58442034)
\curveto(412.41953181,679.5144195)(412.34953188,679.43941957)(412.26953951,679.35942034)
\lineto(412.16453951,679.25442034)
\curveto(412.11453212,679.20441981)(412.05953217,679.15941985)(411.99953951,679.11942034)
\lineto(411.84953951,678.99942034)
\curveto(411.76953246,678.93942007)(411.67953255,678.88442013)(411.57953951,678.83442034)
\curveto(411.48953274,678.79442022)(411.39453284,678.74942026)(411.29453951,678.69942034)
\curveto(411.19453304,678.64942036)(411.08953314,678.6144204)(410.97953951,678.59442034)
\curveto(410.87953335,678.57442044)(410.77453346,678.55442046)(410.66453951,678.53442034)
\curveto(410.60453363,678.5144205)(410.53953369,678.50442051)(410.46953951,678.50442034)
\curveto(410.40953382,678.50442051)(410.34453389,678.49442052)(410.27453951,678.47442034)
\lineto(410.13953951,678.47442034)
\curveto(410.05953417,678.45442056)(409.98453425,678.45442056)(409.91453951,678.47442034)
\lineto(409.76453951,678.47442034)
\curveto(409.70453453,678.49442052)(409.63953459,678.50442051)(409.56953951,678.50442034)
\curveto(409.50953472,678.49442052)(409.44953478,678.49942051)(409.38953951,678.51942034)
\curveto(409.229535,678.56942044)(409.07453516,678.6144204)(408.92453951,678.65442034)
\curveto(408.78453545,678.69442032)(408.65453558,678.75442026)(408.53453951,678.83442034)
\curveto(408.46453577,678.87442014)(408.39953583,678.9144201)(408.33953951,678.95442034)
\curveto(408.27953595,679.00442001)(408.21453602,679.05441996)(408.14453951,679.10442034)
\lineto(407.96453951,679.23942034)
\curveto(407.88453635,679.29941971)(407.81453642,679.30441971)(407.75453951,679.25442034)
\curveto(407.70453653,679.22441979)(407.67953655,679.18441983)(407.67953951,679.13442034)
\curveto(407.67953655,679.09441992)(407.66953656,679.04441997)(407.64953951,678.98442034)
\curveto(407.6295366,678.88442013)(407.61953661,678.76942024)(407.61953951,678.63942034)
\curveto(407.6295366,678.5094205)(407.6345366,678.38942062)(407.63453951,678.27942034)
\lineto(407.63453951,676.74942034)
\curveto(407.6345366,676.61942239)(407.6295366,676.49442252)(407.61953951,676.37442034)
\curveto(407.61953661,676.24442277)(407.59453664,676.13942287)(407.54453951,676.05942034)
\curveto(407.51453672,676.01942299)(407.45953677,675.98942302)(407.37953951,675.96942034)
\curveto(407.29953693,675.94942306)(407.20953702,675.93942307)(407.10953951,675.93942034)
\curveto(407.00953722,675.92942308)(406.90953732,675.92942308)(406.80953951,675.93942034)
\lineto(406.55453951,675.93942034)
\lineto(406.14953951,675.93942034)
\lineto(406.04453951,675.93942034)
\curveto(406.00453823,675.93942307)(405.96953826,675.94442307)(405.93953951,675.95442034)
\lineto(405.81953951,675.95442034)
\curveto(405.64953858,676.00442301)(405.55953867,676.10442291)(405.54953951,676.25442034)
\curveto(405.53953869,676.39442262)(405.5345387,676.56442245)(405.53453951,676.76442034)
\lineto(405.53453951,685.56942034)
\curveto(405.5345387,685.67941333)(405.5295387,685.79441322)(405.51953951,685.91442034)
\curveto(405.51953871,686.04441297)(405.54453869,686.14441287)(405.59453951,686.21442034)
\curveto(405.6345386,686.28441273)(405.68953854,686.32941268)(405.75953951,686.34942034)
\curveto(405.80953842,686.36941264)(405.86953836,686.37941263)(405.93953951,686.37942034)
\lineto(406.16453951,686.37942034)
\lineto(406.88453951,686.37942034)
\lineto(407.16953951,686.37942034)
\curveto(407.25953697,686.37941263)(407.3345369,686.35441266)(407.39453951,686.30442034)
\curveto(407.46453677,686.25441276)(407.49953673,686.18941282)(407.49953951,686.10942034)
\curveto(407.50953672,686.03941297)(407.5345367,685.96441305)(407.57453951,685.88442034)
\curveto(407.58453665,685.85441316)(407.59453664,685.82941318)(407.60453951,685.80942034)
\curveto(407.62453661,685.79941321)(407.64453659,685.78441323)(407.66453951,685.76442034)
\curveto(407.77453646,685.75441326)(407.86453637,685.78441323)(407.93453951,685.85442034)
\curveto(408.00453623,685.92441309)(408.07453616,685.98441303)(408.14453951,686.03442034)
\curveto(408.27453596,686.12441289)(408.40953582,686.20441281)(408.54953951,686.27442034)
\curveto(408.68953554,686.35441266)(408.84453539,686.41941259)(409.01453951,686.46942034)
\curveto(409.09453514,686.49941251)(409.17953505,686.51941249)(409.26953951,686.52942034)
\curveto(409.36953486,686.53941247)(409.46453477,686.55441246)(409.55453951,686.57442034)
\curveto(409.59453464,686.58441243)(409.6345346,686.58441243)(409.67453951,686.57442034)
\curveto(409.72453451,686.56441245)(409.76453447,686.56941244)(409.79453951,686.58942034)
\curveto(410.36453387,686.6094124)(410.84453339,686.52941248)(411.23453951,686.34942034)
\curveto(411.6345326,686.17941283)(411.97453226,685.95441306)(412.25453951,685.67442034)
\curveto(412.30453193,685.62441339)(412.34953188,685.57441344)(412.38953951,685.52442034)
\curveto(412.4295318,685.48441353)(412.46953176,685.43941357)(412.50953951,685.38942034)
\curveto(412.57953165,685.29941371)(412.63953159,685.2094138)(412.68953951,685.11942034)
\curveto(412.74953148,685.02941398)(412.80453143,684.93941407)(412.85453951,684.84942034)
\curveto(412.87453136,684.82941418)(412.88453135,684.80441421)(412.88453951,684.77442034)
\curveto(412.89453134,684.74441427)(412.90953132,684.7094143)(412.92953951,684.66942034)
\curveto(412.98953124,684.56941444)(413.04453119,684.44941456)(413.09453951,684.30942034)
\curveto(413.11453112,684.24941476)(413.1345311,684.18441483)(413.15453951,684.11442034)
\curveto(413.17453106,684.05441496)(413.19453104,683.98941502)(413.21453951,683.91942034)
\curveto(413.25453098,683.79941521)(413.27953095,683.67441534)(413.28953951,683.54442034)
\curveto(413.30953092,683.4144156)(413.3345309,683.27941573)(413.36453951,683.13942034)
\lineto(413.36453951,682.97442034)
\lineto(413.39453951,682.79442034)
\lineto(413.39453951,682.62942034)
\moveto(411.27953951,682.28442034)
\curveto(411.28953294,682.33441668)(411.29453294,682.39941661)(411.29453951,682.47942034)
\curveto(411.29453294,682.56941644)(411.28953294,682.63941637)(411.27953951,682.68942034)
\lineto(411.27953951,682.82442034)
\curveto(411.25953297,682.88441613)(411.24953298,682.94941606)(411.24953951,683.01942034)
\curveto(411.24953298,683.08941592)(411.23953299,683.15941585)(411.21953951,683.22942034)
\curveto(411.19953303,683.32941568)(411.17953305,683.42441559)(411.15953951,683.51442034)
\curveto(411.13953309,683.6144154)(411.10953312,683.70441531)(411.06953951,683.78442034)
\curveto(410.94953328,684.10441491)(410.79453344,684.35941465)(410.60453951,684.54942034)
\curveto(410.41453382,684.73941427)(410.14453409,684.87941413)(409.79453951,684.96942034)
\curveto(409.71453452,684.98941402)(409.62453461,684.99941401)(409.52453951,684.99942034)
\lineto(409.25453951,684.99942034)
\curveto(409.21453502,684.98941402)(409.17953505,684.98441403)(409.14953951,684.98442034)
\curveto(409.11953511,684.98441403)(409.08453515,684.97941403)(409.04453951,684.96942034)
\lineto(408.83453951,684.90942034)
\curveto(408.77453546,684.89941411)(408.71453552,684.87941413)(408.65453951,684.84942034)
\curveto(408.39453584,684.73941427)(408.18953604,684.56941444)(408.03953951,684.33942034)
\curveto(407.89953633,684.1094149)(407.78453645,683.85441516)(407.69453951,683.57442034)
\curveto(407.67453656,683.49441552)(407.65953657,683.4094156)(407.64953951,683.31942034)
\curveto(407.63953659,683.23941577)(407.62453661,683.15941585)(407.60453951,683.07942034)
\curveto(407.59453664,683.03941597)(407.58953664,682.97441604)(407.58953951,682.88442034)
\curveto(407.56953666,682.84441617)(407.56453667,682.79441622)(407.57453951,682.73442034)
\curveto(407.58453665,682.68441633)(407.58453665,682.63441638)(407.57453951,682.58442034)
\curveto(407.55453668,682.52441649)(407.55453668,682.46941654)(407.57453951,682.41942034)
\lineto(407.57453951,682.23942034)
\lineto(407.57453951,682.10442034)
\curveto(407.57453666,682.06441695)(407.58453665,682.02441699)(407.60453951,681.98442034)
\curveto(407.60453663,681.9144171)(407.60953662,681.85941715)(407.61953951,681.81942034)
\lineto(407.64953951,681.63942034)
\curveto(407.65953657,681.57941743)(407.67453656,681.51941749)(407.69453951,681.45942034)
\curveto(407.78453645,681.16941784)(407.88953634,680.92941808)(408.00953951,680.73942034)
\curveto(408.13953609,680.55941845)(408.31953591,680.39941861)(408.54953951,680.25942034)
\curveto(408.68953554,680.17941883)(408.85453538,680.1144189)(409.04453951,680.06442034)
\curveto(409.08453515,680.05441896)(409.11953511,680.04941896)(409.14953951,680.04942034)
\curveto(409.17953505,680.05941895)(409.21453502,680.05941895)(409.25453951,680.04942034)
\curveto(409.29453494,680.03941897)(409.35453488,680.02941898)(409.43453951,680.01942034)
\curveto(409.51453472,680.01941899)(409.57953465,680.02441899)(409.62953951,680.03442034)
\curveto(409.70953452,680.05441896)(409.78953444,680.06941894)(409.86953951,680.07942034)
\curveto(409.95953427,680.09941891)(410.04453419,680.12441889)(410.12453951,680.15442034)
\curveto(410.36453387,680.25441876)(410.55953367,680.39441862)(410.70953951,680.57442034)
\curveto(410.85953337,680.75441826)(410.98453325,680.96441805)(411.08453951,681.20442034)
\curveto(411.1345331,681.32441769)(411.16953306,681.44941756)(411.18953951,681.57942034)
\curveto(411.20953302,681.7094173)(411.234533,681.84441717)(411.26453951,681.98442034)
\lineto(411.26453951,682.13442034)
\curveto(411.27453296,682.18441683)(411.27953295,682.23441678)(411.27953951,682.28442034)
}
}
{
\newrgbcolor{curcolor}{0 0 0}
\pscustom[linestyle=none,fillstyle=solid,fillcolor=curcolor]
{
\newpath
\moveto(422.44446138,682.85442034)
\curveto(422.46445281,682.79441622)(422.4744528,682.7094163)(422.47446138,682.59942034)
\curveto(422.4744528,682.48941652)(422.46445281,682.40441661)(422.44446138,682.34442034)
\lineto(422.44446138,682.19442034)
\curveto(422.42445285,682.1144169)(422.41445286,682.03441698)(422.41446138,681.95442034)
\curveto(422.42445285,681.87441714)(422.41945286,681.79441722)(422.39946138,681.71442034)
\curveto(422.3794529,681.64441737)(422.36445291,681.57941743)(422.35446138,681.51942034)
\curveto(422.34445293,681.45941755)(422.33445294,681.39441762)(422.32446138,681.32442034)
\curveto(422.28445299,681.2144178)(422.24945303,681.09941791)(422.21946138,680.97942034)
\curveto(422.18945309,680.86941814)(422.14945313,680.76441825)(422.09946138,680.66442034)
\curveto(421.88945339,680.18441883)(421.61445366,679.79441922)(421.27446138,679.49442034)
\curveto(420.93445434,679.19441982)(420.52445475,678.94442007)(420.04446138,678.74442034)
\curveto(419.92445535,678.69442032)(419.79945548,678.65942035)(419.66946138,678.63942034)
\curveto(419.54945573,678.6094204)(419.42445585,678.57942043)(419.29446138,678.54942034)
\curveto(419.24445603,678.52942048)(419.18945609,678.51942049)(419.12946138,678.51942034)
\curveto(419.06945621,678.51942049)(419.01445626,678.5144205)(418.96446138,678.50442034)
\lineto(418.85946138,678.50442034)
\curveto(418.82945645,678.49442052)(418.79945648,678.48942052)(418.76946138,678.48942034)
\curveto(418.71945656,678.47942053)(418.63945664,678.47442054)(418.52946138,678.47442034)
\curveto(418.41945686,678.46442055)(418.33445694,678.46942054)(418.27446138,678.48942034)
\lineto(418.12446138,678.48942034)
\curveto(418.0744572,678.49942051)(418.01945726,678.50442051)(417.95946138,678.50442034)
\curveto(417.90945737,678.49442052)(417.85945742,678.49942051)(417.80946138,678.51942034)
\curveto(417.76945751,678.52942048)(417.72945755,678.53442048)(417.68946138,678.53442034)
\curveto(417.65945762,678.53442048)(417.61945766,678.53942047)(417.56946138,678.54942034)
\curveto(417.46945781,678.57942043)(417.36945791,678.60442041)(417.26946138,678.62442034)
\curveto(417.16945811,678.64442037)(417.0744582,678.67442034)(416.98446138,678.71442034)
\curveto(416.86445841,678.75442026)(416.74945853,678.79442022)(416.63946138,678.83442034)
\curveto(416.53945874,678.87442014)(416.43445884,678.92442009)(416.32446138,678.98442034)
\curveto(415.9744593,679.19441982)(415.6744596,679.43941957)(415.42446138,679.71942034)
\curveto(415.1744601,679.99941901)(414.96446031,680.33441868)(414.79446138,680.72442034)
\curveto(414.74446053,680.8144182)(414.70446057,680.9094181)(414.67446138,681.00942034)
\curveto(414.65446062,681.1094179)(414.62946065,681.2144178)(414.59946138,681.32442034)
\curveto(414.5794607,681.37441764)(414.56946071,681.41941759)(414.56946138,681.45942034)
\curveto(414.56946071,681.49941751)(414.55946072,681.54441747)(414.53946138,681.59442034)
\curveto(414.51946076,681.67441734)(414.50946077,681.75441726)(414.50946138,681.83442034)
\curveto(414.50946077,681.92441709)(414.49946078,682.009417)(414.47946138,682.08942034)
\curveto(414.46946081,682.13941687)(414.46446081,682.18441683)(414.46446138,682.22442034)
\lineto(414.46446138,682.35942034)
\curveto(414.44446083,682.41941659)(414.43446084,682.50441651)(414.43446138,682.61442034)
\curveto(414.44446083,682.72441629)(414.45946082,682.8094162)(414.47946138,682.86942034)
\lineto(414.47946138,682.97442034)
\curveto(414.48946079,683.02441599)(414.48946079,683.07441594)(414.47946138,683.12442034)
\curveto(414.4794608,683.18441583)(414.48946079,683.23941577)(414.50946138,683.28942034)
\curveto(414.51946076,683.33941567)(414.52446075,683.38441563)(414.52446138,683.42442034)
\curveto(414.52446075,683.47441554)(414.53446074,683.52441549)(414.55446138,683.57442034)
\curveto(414.59446068,683.70441531)(414.62946065,683.82941518)(414.65946138,683.94942034)
\curveto(414.68946059,684.07941493)(414.72946055,684.20441481)(414.77946138,684.32442034)
\curveto(414.95946032,684.73441428)(415.1744601,685.07441394)(415.42446138,685.34442034)
\curveto(415.6744596,685.62441339)(415.9794593,685.87941313)(416.33946138,686.10942034)
\curveto(416.43945884,686.15941285)(416.54445873,686.20441281)(416.65446138,686.24442034)
\curveto(416.76445851,686.28441273)(416.8744584,686.32941268)(416.98446138,686.37942034)
\curveto(417.11445816,686.42941258)(417.24945803,686.46441255)(417.38946138,686.48442034)
\curveto(417.52945775,686.50441251)(417.6744576,686.53441248)(417.82446138,686.57442034)
\curveto(417.90445737,686.58441243)(417.9794573,686.58941242)(418.04946138,686.58942034)
\curveto(418.11945716,686.58941242)(418.18945709,686.59441242)(418.25946138,686.60442034)
\curveto(418.83945644,686.6144124)(419.33945594,686.55441246)(419.75946138,686.42442034)
\curveto(420.18945509,686.29441272)(420.56945471,686.1144129)(420.89946138,685.88442034)
\curveto(421.00945427,685.80441321)(421.11945416,685.7144133)(421.22946138,685.61442034)
\curveto(421.34945393,685.52441349)(421.44945383,685.42441359)(421.52946138,685.31442034)
\curveto(421.60945367,685.2144138)(421.6794536,685.1144139)(421.73946138,685.01442034)
\curveto(421.80945347,684.9144141)(421.8794534,684.8094142)(421.94946138,684.69942034)
\curveto(422.01945326,684.58941442)(422.0744532,684.46941454)(422.11446138,684.33942034)
\curveto(422.15445312,684.21941479)(422.19945308,684.08941492)(422.24946138,683.94942034)
\curveto(422.279453,683.86941514)(422.30445297,683.78441523)(422.32446138,683.69442034)
\lineto(422.38446138,683.42442034)
\curveto(422.39445288,683.38441563)(422.39945288,683.34441567)(422.39946138,683.30442034)
\curveto(422.39945288,683.26441575)(422.40445287,683.22441579)(422.41446138,683.18442034)
\curveto(422.43445284,683.13441588)(422.43945284,683.07941593)(422.42946138,683.01942034)
\curveto(422.41945286,682.95941605)(422.42445285,682.90441611)(422.44446138,682.85442034)
\moveto(420.34446138,682.31442034)
\curveto(420.35445492,682.36441665)(420.35945492,682.43441658)(420.35946138,682.52442034)
\curveto(420.35945492,682.62441639)(420.35445492,682.69941631)(420.34446138,682.74942034)
\lineto(420.34446138,682.86942034)
\curveto(420.32445495,682.91941609)(420.31445496,682.97441604)(420.31446138,683.03442034)
\curveto(420.31445496,683.09441592)(420.30945497,683.14941586)(420.29946138,683.19942034)
\curveto(420.29945498,683.23941577)(420.29445498,683.26941574)(420.28446138,683.28942034)
\lineto(420.22446138,683.52942034)
\curveto(420.21445506,683.61941539)(420.19445508,683.70441531)(420.16446138,683.78442034)
\curveto(420.05445522,684.04441497)(419.92445535,684.26441475)(419.77446138,684.44442034)
\curveto(419.62445565,684.63441438)(419.42445585,684.78441423)(419.17446138,684.89442034)
\curveto(419.11445616,684.9144141)(419.05445622,684.92941408)(418.99446138,684.93942034)
\curveto(418.93445634,684.95941405)(418.86945641,684.97941403)(418.79946138,684.99942034)
\curveto(418.71945656,685.01941399)(418.63445664,685.02441399)(418.54446138,685.01442034)
\lineto(418.27446138,685.01442034)
\curveto(418.24445703,684.99441402)(418.20945707,684.98441403)(418.16946138,684.98442034)
\curveto(418.12945715,684.99441402)(418.09445718,684.99441402)(418.06446138,684.98442034)
\lineto(417.85446138,684.92442034)
\curveto(417.79445748,684.9144141)(417.73945754,684.89441412)(417.68946138,684.86442034)
\curveto(417.43945784,684.75441426)(417.23445804,684.59441442)(417.07446138,684.38442034)
\curveto(416.92445835,684.18441483)(416.80445847,683.94941506)(416.71446138,683.67942034)
\curveto(416.68445859,683.57941543)(416.65945862,683.47441554)(416.63946138,683.36442034)
\curveto(416.62945865,683.25441576)(416.61445866,683.14441587)(416.59446138,683.03442034)
\curveto(416.58445869,682.98441603)(416.5794587,682.93441608)(416.57946138,682.88442034)
\lineto(416.57946138,682.73442034)
\curveto(416.55945872,682.66441635)(416.54945873,682.55941645)(416.54946138,682.41942034)
\curveto(416.55945872,682.27941673)(416.5744587,682.17441684)(416.59446138,682.10442034)
\lineto(416.59446138,681.96942034)
\curveto(416.61445866,681.88941712)(416.62945865,681.8094172)(416.63946138,681.72942034)
\curveto(416.64945863,681.65941735)(416.66445861,681.58441743)(416.68446138,681.50442034)
\curveto(416.78445849,681.20441781)(416.88945839,680.95941805)(416.99946138,680.76942034)
\curveto(417.11945816,680.58941842)(417.30445797,680.42441859)(417.55446138,680.27442034)
\curveto(417.62445765,680.22441879)(417.69945758,680.18441883)(417.77946138,680.15442034)
\curveto(417.86945741,680.12441889)(417.95945732,680.09941891)(418.04946138,680.07942034)
\curveto(418.08945719,680.06941894)(418.12445715,680.06441895)(418.15446138,680.06442034)
\curveto(418.18445709,680.07441894)(418.21945706,680.07441894)(418.25946138,680.06442034)
\lineto(418.37946138,680.03442034)
\curveto(418.42945685,680.03441898)(418.4744568,680.03941897)(418.51446138,680.04942034)
\lineto(418.63446138,680.04942034)
\curveto(418.71445656,680.06941894)(418.79445648,680.08441893)(418.87446138,680.09442034)
\curveto(418.95445632,680.10441891)(419.02945625,680.12441889)(419.09946138,680.15442034)
\curveto(419.35945592,680.25441876)(419.56945571,680.38941862)(419.72946138,680.55942034)
\curveto(419.88945539,680.72941828)(420.02445525,680.93941807)(420.13446138,681.18942034)
\curveto(420.1744551,681.28941772)(420.20445507,681.38941762)(420.22446138,681.48942034)
\curveto(420.24445503,681.58941742)(420.26945501,681.69441732)(420.29946138,681.80442034)
\curveto(420.30945497,681.84441717)(420.31445496,681.87941713)(420.31446138,681.90942034)
\curveto(420.31445496,681.94941706)(420.31945496,681.98941702)(420.32946138,682.02942034)
\lineto(420.32946138,682.16442034)
\curveto(420.32945495,682.2144168)(420.33445494,682.26441675)(420.34446138,682.31442034)
}
}
{
\newrgbcolor{curcolor}{0 0 0}
\pscustom[linestyle=none,fillstyle=solid,fillcolor=curcolor]
{
\newpath
\moveto(428.26938326,686.60442034)
\curveto(428.37937794,686.60441241)(428.47437785,686.59441242)(428.55438326,686.57442034)
\curveto(428.64437768,686.55441246)(428.71437761,686.5094125)(428.76438326,686.43942034)
\curveto(428.8243775,686.35941265)(428.85437747,686.21941279)(428.85438326,686.01942034)
\lineto(428.85438326,685.50942034)
\lineto(428.85438326,685.13442034)
\curveto(428.86437746,684.99441402)(428.84937747,684.88441413)(428.80938326,684.80442034)
\curveto(428.76937755,684.73441428)(428.70937761,684.68941432)(428.62938326,684.66942034)
\curveto(428.55937776,684.64941436)(428.47437785,684.63941437)(428.37438326,684.63942034)
\curveto(428.28437804,684.63941437)(428.18437814,684.64441437)(428.07438326,684.65442034)
\curveto(427.97437835,684.66441435)(427.87937844,684.65941435)(427.78938326,684.63942034)
\curveto(427.7193786,684.61941439)(427.64937867,684.60441441)(427.57938326,684.59442034)
\curveto(427.50937881,684.59441442)(427.44437888,684.58441443)(427.38438326,684.56442034)
\curveto(427.2243791,684.5144145)(427.06437926,684.43941457)(426.90438326,684.33942034)
\curveto(426.74437958,684.24941476)(426.6193797,684.14441487)(426.52938326,684.02442034)
\curveto(426.47937984,683.94441507)(426.4243799,683.85941515)(426.36438326,683.76942034)
\curveto(426.31438001,683.68941532)(426.26438006,683.60441541)(426.21438326,683.51442034)
\curveto(426.18438014,683.43441558)(426.15438017,683.34941566)(426.12438326,683.25942034)
\lineto(426.06438326,683.01942034)
\curveto(426.04438028,682.94941606)(426.03438029,682.87441614)(426.03438326,682.79442034)
\curveto(426.03438029,682.72441629)(426.0243803,682.65441636)(426.00438326,682.58442034)
\curveto(425.99438033,682.54441647)(425.98938033,682.50441651)(425.98938326,682.46442034)
\curveto(425.99938032,682.43441658)(425.99938032,682.40441661)(425.98938326,682.37442034)
\lineto(425.98938326,682.13442034)
\curveto(425.96938035,682.06441695)(425.96438036,681.98441703)(425.97438326,681.89442034)
\curveto(425.98438034,681.8144172)(425.98938033,681.73441728)(425.98938326,681.65442034)
\lineto(425.98938326,680.69442034)
\lineto(425.98938326,679.41942034)
\curveto(425.98938033,679.28941972)(425.98438034,679.16941984)(425.97438326,679.05942034)
\curveto(425.96438036,678.94942006)(425.93438039,678.85942015)(425.88438326,678.78942034)
\curveto(425.86438046,678.75942025)(425.82938049,678.73442028)(425.77938326,678.71442034)
\curveto(425.73938058,678.70442031)(425.69438063,678.69442032)(425.64438326,678.68442034)
\lineto(425.56938326,678.68442034)
\curveto(425.5193808,678.67442034)(425.46438086,678.66942034)(425.40438326,678.66942034)
\lineto(425.23938326,678.66942034)
\lineto(424.59438326,678.66942034)
\curveto(424.53438179,678.67942033)(424.46938185,678.68442033)(424.39938326,678.68442034)
\lineto(424.20438326,678.68442034)
\curveto(424.15438217,678.70442031)(424.10438222,678.71942029)(424.05438326,678.72942034)
\curveto(424.00438232,678.74942026)(423.96938235,678.78442023)(423.94938326,678.83442034)
\curveto(423.90938241,678.88442013)(423.88438244,678.95442006)(423.87438326,679.04442034)
\lineto(423.87438326,679.34442034)
\lineto(423.87438326,680.36442034)
\lineto(423.87438326,684.59442034)
\lineto(423.87438326,685.70442034)
\lineto(423.87438326,685.98942034)
\curveto(423.87438245,686.08941292)(423.89438243,686.16941284)(423.93438326,686.22942034)
\curveto(423.98438234,686.3094127)(424.05938226,686.35941265)(424.15938326,686.37942034)
\curveto(424.25938206,686.39941261)(424.37938194,686.4094126)(424.51938326,686.40942034)
\lineto(425.28438326,686.40942034)
\curveto(425.40438092,686.4094126)(425.50938081,686.39941261)(425.59938326,686.37942034)
\curveto(425.68938063,686.36941264)(425.75938056,686.32441269)(425.80938326,686.24442034)
\curveto(425.83938048,686.19441282)(425.85438047,686.12441289)(425.85438326,686.03442034)
\lineto(425.88438326,685.76442034)
\curveto(425.89438043,685.68441333)(425.90938041,685.6094134)(425.92938326,685.53942034)
\curveto(425.95938036,685.46941354)(426.00938031,685.43441358)(426.07938326,685.43442034)
\curveto(426.09938022,685.45441356)(426.1193802,685.46441355)(426.13938326,685.46442034)
\curveto(426.15938016,685.46441355)(426.17938014,685.47441354)(426.19938326,685.49442034)
\curveto(426.25938006,685.54441347)(426.30938001,685.59941341)(426.34938326,685.65942034)
\curveto(426.39937992,685.72941328)(426.45937986,685.78941322)(426.52938326,685.83942034)
\curveto(426.56937975,685.86941314)(426.60437972,685.89941311)(426.63438326,685.92942034)
\curveto(426.66437966,685.96941304)(426.69937962,686.00441301)(426.73938326,686.03442034)
\lineto(427.00938326,686.21442034)
\curveto(427.10937921,686.27441274)(427.20937911,686.32941268)(427.30938326,686.37942034)
\curveto(427.40937891,686.41941259)(427.50937881,686.45441256)(427.60938326,686.48442034)
\lineto(427.93938326,686.57442034)
\curveto(427.96937835,686.58441243)(428.0243783,686.58441243)(428.10438326,686.57442034)
\curveto(428.19437813,686.57441244)(428.24937807,686.58441243)(428.26938326,686.60442034)
}
}
{
\newrgbcolor{curcolor}{0 0 0}
\pscustom[linestyle=none,fillstyle=solid,fillcolor=curcolor]
{
}
}
{
\newrgbcolor{curcolor}{0 0 0}
\pscustom[linestyle=none,fillstyle=solid,fillcolor=curcolor]
{
\newpath
\moveto(434.88461763,688.71942034)
\lineto(435.88961763,688.71942034)
\curveto(436.03961465,688.71941029)(436.16961452,688.7094103)(436.27961763,688.68942034)
\curveto(436.39961429,688.67941033)(436.4846142,688.61941039)(436.53461763,688.50942034)
\curveto(436.55461413,688.45941055)(436.56461412,688.39941061)(436.56461763,688.32942034)
\lineto(436.56461763,688.11942034)
\lineto(436.56461763,687.44442034)
\curveto(436.56461412,687.39441162)(436.55961413,687.33441168)(436.54961763,687.26442034)
\curveto(436.54961414,687.20441181)(436.55461413,687.14941186)(436.56461763,687.09942034)
\lineto(436.56461763,686.93442034)
\curveto(436.56461412,686.85441216)(436.56961412,686.77941223)(436.57961763,686.70942034)
\curveto(436.5896141,686.64941236)(436.61461407,686.59441242)(436.65461763,686.54442034)
\curveto(436.72461396,686.45441256)(436.84961384,686.40441261)(437.02961763,686.39442034)
\lineto(437.56961763,686.39442034)
\lineto(437.74961763,686.39442034)
\curveto(437.80961288,686.39441262)(437.86461282,686.38441263)(437.91461763,686.36442034)
\curveto(438.02461266,686.3144127)(438.0846126,686.22441279)(438.09461763,686.09442034)
\curveto(438.11461257,685.96441305)(438.12461256,685.81941319)(438.12461763,685.65942034)
\lineto(438.12461763,685.44942034)
\curveto(438.13461255,685.37941363)(438.12961256,685.31941369)(438.10961763,685.26942034)
\curveto(438.05961263,685.1094139)(437.95461273,685.02441399)(437.79461763,685.01442034)
\curveto(437.63461305,685.00441401)(437.45461323,684.99941401)(437.25461763,684.99942034)
\lineto(437.11961763,684.99942034)
\curveto(437.07961361,685.009414)(437.04461364,685.009414)(437.01461763,684.99942034)
\curveto(436.97461371,684.98941402)(436.93961375,684.98441403)(436.90961763,684.98442034)
\curveto(436.87961381,684.99441402)(436.84961384,684.98941402)(436.81961763,684.96942034)
\curveto(436.73961395,684.94941406)(436.67961401,684.90441411)(436.63961763,684.83442034)
\curveto(436.60961408,684.77441424)(436.5846141,684.69941431)(436.56461763,684.60942034)
\curveto(436.55461413,684.55941445)(436.55461413,684.50441451)(436.56461763,684.44442034)
\curveto(436.57461411,684.38441463)(436.57461411,684.32941468)(436.56461763,684.27942034)
\lineto(436.56461763,683.34942034)
\lineto(436.56461763,681.59442034)
\curveto(436.56461412,681.34441767)(436.56961412,681.12441789)(436.57961763,680.93442034)
\curveto(436.59961409,680.75441826)(436.66461402,680.59441842)(436.77461763,680.45442034)
\curveto(436.82461386,680.39441862)(436.8896138,680.34941866)(436.96961763,680.31942034)
\lineto(437.23961763,680.25942034)
\curveto(437.26961342,680.24941876)(437.29961339,680.24441877)(437.32961763,680.24442034)
\curveto(437.36961332,680.25441876)(437.39961329,680.25441876)(437.41961763,680.24442034)
\lineto(437.58461763,680.24442034)
\curveto(437.69461299,680.24441877)(437.7896129,680.23941877)(437.86961763,680.22942034)
\curveto(437.94961274,680.21941879)(438.01461267,680.17941883)(438.06461763,680.10942034)
\curveto(438.10461258,680.04941896)(438.12461256,679.96941904)(438.12461763,679.86942034)
\lineto(438.12461763,679.58442034)
\curveto(438.12461256,679.37441964)(438.11961257,679.17941983)(438.10961763,678.99942034)
\curveto(438.10961258,678.82942018)(438.02961266,678.7144203)(437.86961763,678.65442034)
\curveto(437.81961287,678.63442038)(437.77461291,678.62942038)(437.73461763,678.63942034)
\curveto(437.69461299,678.63942037)(437.64961304,678.62942038)(437.59961763,678.60942034)
\lineto(437.44961763,678.60942034)
\curveto(437.42961326,678.6094204)(437.39961329,678.6144204)(437.35961763,678.62442034)
\curveto(437.31961337,678.62442039)(437.2846134,678.61942039)(437.25461763,678.60942034)
\curveto(437.20461348,678.59942041)(437.14961354,678.59942041)(437.08961763,678.60942034)
\lineto(436.93961763,678.60942034)
\lineto(436.78961763,678.60942034)
\curveto(436.73961395,678.59942041)(436.69461399,678.59942041)(436.65461763,678.60942034)
\lineto(436.48961763,678.60942034)
\curveto(436.43961425,678.61942039)(436.3846143,678.62442039)(436.32461763,678.62442034)
\curveto(436.26461442,678.62442039)(436.20961448,678.62942038)(436.15961763,678.63942034)
\curveto(436.0896146,678.64942036)(436.02461466,678.65942035)(435.96461763,678.66942034)
\lineto(435.78461763,678.69942034)
\curveto(435.67461501,678.72942028)(435.56961512,678.76442025)(435.46961763,678.80442034)
\curveto(435.36961532,678.84442017)(435.27461541,678.88942012)(435.18461763,678.93942034)
\lineto(435.09461763,678.99942034)
\curveto(435.06461562,679.02941998)(435.02961566,679.05941995)(434.98961763,679.08942034)
\curveto(434.96961572,679.1094199)(434.94461574,679.12941988)(434.91461763,679.14942034)
\lineto(434.83961763,679.22442034)
\curveto(434.69961599,679.4144196)(434.59461609,679.62441939)(434.52461763,679.85442034)
\curveto(434.50461618,679.89441912)(434.49461619,679.92941908)(434.49461763,679.95942034)
\curveto(434.50461618,679.99941901)(434.50461618,680.04441897)(434.49461763,680.09442034)
\curveto(434.4846162,680.1144189)(434.47961621,680.13941887)(434.47961763,680.16942034)
\curveto(434.47961621,680.19941881)(434.47461621,680.22441879)(434.46461763,680.24442034)
\lineto(434.46461763,680.39442034)
\curveto(434.45461623,680.43441858)(434.44961624,680.47941853)(434.44961763,680.52942034)
\curveto(434.45961623,680.57941843)(434.46461622,680.62941838)(434.46461763,680.67942034)
\lineto(434.46461763,681.24942034)
\lineto(434.46461763,683.48442034)
\lineto(434.46461763,684.27942034)
\lineto(434.46461763,684.48942034)
\curveto(434.47461621,684.55941445)(434.46961622,684.62441439)(434.44961763,684.68442034)
\curveto(434.40961628,684.82441419)(434.33961635,684.9144141)(434.23961763,684.95442034)
\curveto(434.12961656,685.00441401)(433.9896167,685.01941399)(433.81961763,684.99942034)
\curveto(433.64961704,684.97941403)(433.50461718,684.99441402)(433.38461763,685.04442034)
\curveto(433.30461738,685.07441394)(433.25461743,685.11941389)(433.23461763,685.17942034)
\curveto(433.21461747,685.23941377)(433.19461749,685.3144137)(433.17461763,685.40442034)
\lineto(433.17461763,685.71942034)
\curveto(433.17461751,685.89941311)(433.1846175,686.04441297)(433.20461763,686.15442034)
\curveto(433.22461746,686.26441275)(433.30961738,686.33941267)(433.45961763,686.37942034)
\curveto(433.49961719,686.39941261)(433.53961715,686.40441261)(433.57961763,686.39442034)
\lineto(433.71461763,686.39442034)
\curveto(433.86461682,686.39441262)(434.00461668,686.39941261)(434.13461763,686.40942034)
\curveto(434.26461642,686.42941258)(434.35461633,686.48941252)(434.40461763,686.58942034)
\curveto(434.43461625,686.65941235)(434.44961624,686.73941227)(434.44961763,686.82942034)
\curveto(434.45961623,686.91941209)(434.46461622,687.009412)(434.46461763,687.09942034)
\lineto(434.46461763,688.02942034)
\lineto(434.46461763,688.28442034)
\curveto(434.46461622,688.37441064)(434.47461621,688.44941056)(434.49461763,688.50942034)
\curveto(434.54461614,688.6094104)(434.61961607,688.67441034)(434.71961763,688.70442034)
\curveto(434.73961595,688.7144103)(434.76461592,688.7144103)(434.79461763,688.70442034)
\curveto(434.83461585,688.70441031)(434.86461582,688.7094103)(434.88461763,688.71942034)
}
}
{
\newrgbcolor{curcolor}{0 0 0}
\pscustom[linestyle=none,fillstyle=solid,fillcolor=curcolor]
{
\newpath
\moveto(441.20805513,689.25942034)
\curveto(441.27805218,689.17940983)(441.31305215,689.05940995)(441.31305513,688.89942034)
\lineto(441.31305513,688.43442034)
\lineto(441.31305513,688.02942034)
\curveto(441.31305215,687.88941112)(441.27805218,687.79441122)(441.20805513,687.74442034)
\curveto(441.14805231,687.69441132)(441.06805239,687.66441135)(440.96805513,687.65442034)
\curveto(440.87805258,687.64441137)(440.77805268,687.63941137)(440.66805513,687.63942034)
\lineto(439.82805513,687.63942034)
\curveto(439.71805374,687.63941137)(439.61805384,687.64441137)(439.52805513,687.65442034)
\curveto(439.44805401,687.66441135)(439.37805408,687.69441132)(439.31805513,687.74442034)
\curveto(439.27805418,687.77441124)(439.24805421,687.82941118)(439.22805513,687.90942034)
\curveto(439.21805424,687.99941101)(439.20805425,688.09441092)(439.19805513,688.19442034)
\lineto(439.19805513,688.52442034)
\curveto(439.20805425,688.63441038)(439.21305425,688.72941028)(439.21305513,688.80942034)
\lineto(439.21305513,689.01942034)
\curveto(439.22305424,689.08940992)(439.24305422,689.14940986)(439.27305513,689.19942034)
\curveto(439.29305417,689.23940977)(439.31805414,689.26940974)(439.34805513,689.28942034)
\lineto(439.46805513,689.34942034)
\curveto(439.48805397,689.34940966)(439.51305395,689.34940966)(439.54305513,689.34942034)
\curveto(439.57305389,689.35940965)(439.59805386,689.36440965)(439.61805513,689.36442034)
\lineto(440.71305513,689.36442034)
\curveto(440.81305265,689.36440965)(440.90805255,689.35940965)(440.99805513,689.34942034)
\curveto(441.08805237,689.33940967)(441.1580523,689.3094097)(441.20805513,689.25942034)
\moveto(441.31305513,679.49442034)
\curveto(441.31305215,679.29441972)(441.30805215,679.12441989)(441.29805513,678.98442034)
\curveto(441.28805217,678.84442017)(441.19805226,678.74942026)(441.02805513,678.69942034)
\curveto(440.96805249,678.67942033)(440.90305256,678.66942034)(440.83305513,678.66942034)
\curveto(440.7630527,678.67942033)(440.68805277,678.68442033)(440.60805513,678.68442034)
\lineto(439.76805513,678.68442034)
\curveto(439.67805378,678.68442033)(439.58805387,678.68942032)(439.49805513,678.69942034)
\curveto(439.41805404,678.7094203)(439.3580541,678.73942027)(439.31805513,678.78942034)
\curveto(439.2580542,678.85942015)(439.22305424,678.94442007)(439.21305513,679.04442034)
\lineto(439.21305513,679.38942034)
\lineto(439.21305513,685.71942034)
\lineto(439.21305513,686.01942034)
\curveto(439.21305425,686.11941289)(439.23305423,686.19941281)(439.27305513,686.25942034)
\curveto(439.33305413,686.32941268)(439.41805404,686.37441264)(439.52805513,686.39442034)
\curveto(439.54805391,686.40441261)(439.57305389,686.40441261)(439.60305513,686.39442034)
\curveto(439.64305382,686.39441262)(439.67305379,686.39941261)(439.69305513,686.40942034)
\lineto(440.44305513,686.40942034)
\lineto(440.63805513,686.40942034)
\curveto(440.71805274,686.41941259)(440.78305268,686.41941259)(440.83305513,686.40942034)
\lineto(440.95305513,686.40942034)
\curveto(441.01305245,686.38941262)(441.06805239,686.37441264)(441.11805513,686.36442034)
\curveto(441.16805229,686.35441266)(441.20805225,686.32441269)(441.23805513,686.27442034)
\curveto(441.27805218,686.22441279)(441.29805216,686.15441286)(441.29805513,686.06442034)
\curveto(441.30805215,685.97441304)(441.31305215,685.87941313)(441.31305513,685.77942034)
\lineto(441.31305513,679.49442034)
}
}
{
\newrgbcolor{curcolor}{0 0 0}
\pscustom[linestyle=none,fillstyle=solid,fillcolor=curcolor]
{
\newpath
\moveto(450.86524263,682.62942034)
\curveto(450.87523395,682.56941644)(450.88023395,682.47941653)(450.88024263,682.35942034)
\curveto(450.88023395,682.23941677)(450.87023396,682.15441686)(450.85024263,682.10442034)
\lineto(450.85024263,681.90942034)
\curveto(450.82023401,681.79941721)(450.80023403,681.69441732)(450.79024263,681.59442034)
\curveto(450.79023404,681.49441752)(450.77523405,681.39441762)(450.74524263,681.29442034)
\curveto(450.7252341,681.20441781)(450.70523412,681.1094179)(450.68524263,681.00942034)
\curveto(450.66523416,680.91941809)(450.63523419,680.82941818)(450.59524263,680.73942034)
\curveto(450.5252343,680.56941844)(450.45523437,680.4094186)(450.38524263,680.25942034)
\curveto(450.31523451,680.11941889)(450.23523459,679.97941903)(450.14524263,679.83942034)
\curveto(450.08523474,679.74941926)(450.02023481,679.66441935)(449.95024263,679.58442034)
\curveto(449.89023494,679.5144195)(449.82023501,679.43941957)(449.74024263,679.35942034)
\lineto(449.63524263,679.25442034)
\curveto(449.58523524,679.20441981)(449.5302353,679.15941985)(449.47024263,679.11942034)
\lineto(449.32024263,678.99942034)
\curveto(449.24023559,678.93942007)(449.15023568,678.88442013)(449.05024263,678.83442034)
\curveto(448.96023587,678.79442022)(448.86523596,678.74942026)(448.76524263,678.69942034)
\curveto(448.66523616,678.64942036)(448.56023627,678.6144204)(448.45024263,678.59442034)
\curveto(448.35023648,678.57442044)(448.24523658,678.55442046)(448.13524263,678.53442034)
\curveto(448.07523675,678.5144205)(448.01023682,678.50442051)(447.94024263,678.50442034)
\curveto(447.88023695,678.50442051)(447.81523701,678.49442052)(447.74524263,678.47442034)
\lineto(447.61024263,678.47442034)
\curveto(447.5302373,678.45442056)(447.45523737,678.45442056)(447.38524263,678.47442034)
\lineto(447.23524263,678.47442034)
\curveto(447.17523765,678.49442052)(447.11023772,678.50442051)(447.04024263,678.50442034)
\curveto(446.98023785,678.49442052)(446.92023791,678.49942051)(446.86024263,678.51942034)
\curveto(446.70023813,678.56942044)(446.54523828,678.6144204)(446.39524263,678.65442034)
\curveto(446.25523857,678.69442032)(446.1252387,678.75442026)(446.00524263,678.83442034)
\curveto(445.93523889,678.87442014)(445.87023896,678.9144201)(445.81024263,678.95442034)
\curveto(445.75023908,679.00442001)(445.68523914,679.05441996)(445.61524263,679.10442034)
\lineto(445.43524263,679.23942034)
\curveto(445.35523947,679.29941971)(445.28523954,679.30441971)(445.22524263,679.25442034)
\curveto(445.17523965,679.22441979)(445.15023968,679.18441983)(445.15024263,679.13442034)
\curveto(445.15023968,679.09441992)(445.14023969,679.04441997)(445.12024263,678.98442034)
\curveto(445.10023973,678.88442013)(445.09023974,678.76942024)(445.09024263,678.63942034)
\curveto(445.10023973,678.5094205)(445.10523972,678.38942062)(445.10524263,678.27942034)
\lineto(445.10524263,676.74942034)
\curveto(445.10523972,676.61942239)(445.10023973,676.49442252)(445.09024263,676.37442034)
\curveto(445.09023974,676.24442277)(445.06523976,676.13942287)(445.01524263,676.05942034)
\curveto(444.98523984,676.01942299)(444.9302399,675.98942302)(444.85024263,675.96942034)
\curveto(444.77024006,675.94942306)(444.68024015,675.93942307)(444.58024263,675.93942034)
\curveto(444.48024035,675.92942308)(444.38024045,675.92942308)(444.28024263,675.93942034)
\lineto(444.02524263,675.93942034)
\lineto(443.62024263,675.93942034)
\lineto(443.51524263,675.93942034)
\curveto(443.47524135,675.93942307)(443.44024139,675.94442307)(443.41024263,675.95442034)
\lineto(443.29024263,675.95442034)
\curveto(443.12024171,676.00442301)(443.0302418,676.10442291)(443.02024263,676.25442034)
\curveto(443.01024182,676.39442262)(443.00524182,676.56442245)(443.00524263,676.76442034)
\lineto(443.00524263,685.56942034)
\curveto(443.00524182,685.67941333)(443.00024183,685.79441322)(442.99024263,685.91442034)
\curveto(442.99024184,686.04441297)(443.01524181,686.14441287)(443.06524263,686.21442034)
\curveto(443.10524172,686.28441273)(443.16024167,686.32941268)(443.23024263,686.34942034)
\curveto(443.28024155,686.36941264)(443.34024149,686.37941263)(443.41024263,686.37942034)
\lineto(443.63524263,686.37942034)
\lineto(444.35524263,686.37942034)
\lineto(444.64024263,686.37942034)
\curveto(444.7302401,686.37941263)(444.80524002,686.35441266)(444.86524263,686.30442034)
\curveto(444.93523989,686.25441276)(444.97023986,686.18941282)(444.97024263,686.10942034)
\curveto(444.98023985,686.03941297)(445.00523982,685.96441305)(445.04524263,685.88442034)
\curveto(445.05523977,685.85441316)(445.06523976,685.82941318)(445.07524263,685.80942034)
\curveto(445.09523973,685.79941321)(445.11523971,685.78441323)(445.13524263,685.76442034)
\curveto(445.24523958,685.75441326)(445.33523949,685.78441323)(445.40524263,685.85442034)
\curveto(445.47523935,685.92441309)(445.54523928,685.98441303)(445.61524263,686.03442034)
\curveto(445.74523908,686.12441289)(445.88023895,686.20441281)(446.02024263,686.27442034)
\curveto(446.16023867,686.35441266)(446.31523851,686.41941259)(446.48524263,686.46942034)
\curveto(446.56523826,686.49941251)(446.65023818,686.51941249)(446.74024263,686.52942034)
\curveto(446.84023799,686.53941247)(446.93523789,686.55441246)(447.02524263,686.57442034)
\curveto(447.06523776,686.58441243)(447.10523772,686.58441243)(447.14524263,686.57442034)
\curveto(447.19523763,686.56441245)(447.23523759,686.56941244)(447.26524263,686.58942034)
\curveto(447.83523699,686.6094124)(448.31523651,686.52941248)(448.70524263,686.34942034)
\curveto(449.10523572,686.17941283)(449.44523538,685.95441306)(449.72524263,685.67442034)
\curveto(449.77523505,685.62441339)(449.82023501,685.57441344)(449.86024263,685.52442034)
\curveto(449.90023493,685.48441353)(449.94023489,685.43941357)(449.98024263,685.38942034)
\curveto(450.05023478,685.29941371)(450.11023472,685.2094138)(450.16024263,685.11942034)
\curveto(450.22023461,685.02941398)(450.27523455,684.93941407)(450.32524263,684.84942034)
\curveto(450.34523448,684.82941418)(450.35523447,684.80441421)(450.35524263,684.77442034)
\curveto(450.36523446,684.74441427)(450.38023445,684.7094143)(450.40024263,684.66942034)
\curveto(450.46023437,684.56941444)(450.51523431,684.44941456)(450.56524263,684.30942034)
\curveto(450.58523424,684.24941476)(450.60523422,684.18441483)(450.62524263,684.11442034)
\curveto(450.64523418,684.05441496)(450.66523416,683.98941502)(450.68524263,683.91942034)
\curveto(450.7252341,683.79941521)(450.75023408,683.67441534)(450.76024263,683.54442034)
\curveto(450.78023405,683.4144156)(450.80523402,683.27941573)(450.83524263,683.13942034)
\lineto(450.83524263,682.97442034)
\lineto(450.86524263,682.79442034)
\lineto(450.86524263,682.62942034)
\moveto(448.75024263,682.28442034)
\curveto(448.76023607,682.33441668)(448.76523606,682.39941661)(448.76524263,682.47942034)
\curveto(448.76523606,682.56941644)(448.76023607,682.63941637)(448.75024263,682.68942034)
\lineto(448.75024263,682.82442034)
\curveto(448.7302361,682.88441613)(448.72023611,682.94941606)(448.72024263,683.01942034)
\curveto(448.72023611,683.08941592)(448.71023612,683.15941585)(448.69024263,683.22942034)
\curveto(448.67023616,683.32941568)(448.65023618,683.42441559)(448.63024263,683.51442034)
\curveto(448.61023622,683.6144154)(448.58023625,683.70441531)(448.54024263,683.78442034)
\curveto(448.42023641,684.10441491)(448.26523656,684.35941465)(448.07524263,684.54942034)
\curveto(447.88523694,684.73941427)(447.61523721,684.87941413)(447.26524263,684.96942034)
\curveto(447.18523764,684.98941402)(447.09523773,684.99941401)(446.99524263,684.99942034)
\lineto(446.72524263,684.99942034)
\curveto(446.68523814,684.98941402)(446.65023818,684.98441403)(446.62024263,684.98442034)
\curveto(446.59023824,684.98441403)(446.55523827,684.97941403)(446.51524263,684.96942034)
\lineto(446.30524263,684.90942034)
\curveto(446.24523858,684.89941411)(446.18523864,684.87941413)(446.12524263,684.84942034)
\curveto(445.86523896,684.73941427)(445.66023917,684.56941444)(445.51024263,684.33942034)
\curveto(445.37023946,684.1094149)(445.25523957,683.85441516)(445.16524263,683.57442034)
\curveto(445.14523968,683.49441552)(445.1302397,683.4094156)(445.12024263,683.31942034)
\curveto(445.11023972,683.23941577)(445.09523973,683.15941585)(445.07524263,683.07942034)
\curveto(445.06523976,683.03941597)(445.06023977,682.97441604)(445.06024263,682.88442034)
\curveto(445.04023979,682.84441617)(445.03523979,682.79441622)(445.04524263,682.73442034)
\curveto(445.05523977,682.68441633)(445.05523977,682.63441638)(445.04524263,682.58442034)
\curveto(445.0252398,682.52441649)(445.0252398,682.46941654)(445.04524263,682.41942034)
\lineto(445.04524263,682.23942034)
\lineto(445.04524263,682.10442034)
\curveto(445.04523978,682.06441695)(445.05523977,682.02441699)(445.07524263,681.98442034)
\curveto(445.07523975,681.9144171)(445.08023975,681.85941715)(445.09024263,681.81942034)
\lineto(445.12024263,681.63942034)
\curveto(445.1302397,681.57941743)(445.14523968,681.51941749)(445.16524263,681.45942034)
\curveto(445.25523957,681.16941784)(445.36023947,680.92941808)(445.48024263,680.73942034)
\curveto(445.61023922,680.55941845)(445.79023904,680.39941861)(446.02024263,680.25942034)
\curveto(446.16023867,680.17941883)(446.3252385,680.1144189)(446.51524263,680.06442034)
\curveto(446.55523827,680.05441896)(446.59023824,680.04941896)(446.62024263,680.04942034)
\curveto(446.65023818,680.05941895)(446.68523814,680.05941895)(446.72524263,680.04942034)
\curveto(446.76523806,680.03941897)(446.825238,680.02941898)(446.90524263,680.01942034)
\curveto(446.98523784,680.01941899)(447.05023778,680.02441899)(447.10024263,680.03442034)
\curveto(447.18023765,680.05441896)(447.26023757,680.06941894)(447.34024263,680.07942034)
\curveto(447.4302374,680.09941891)(447.51523731,680.12441889)(447.59524263,680.15442034)
\curveto(447.83523699,680.25441876)(448.0302368,680.39441862)(448.18024263,680.57442034)
\curveto(448.3302365,680.75441826)(448.45523637,680.96441805)(448.55524263,681.20442034)
\curveto(448.60523622,681.32441769)(448.64023619,681.44941756)(448.66024263,681.57942034)
\curveto(448.68023615,681.7094173)(448.70523612,681.84441717)(448.73524263,681.98442034)
\lineto(448.73524263,682.13442034)
\curveto(448.74523608,682.18441683)(448.75023608,682.23441678)(448.75024263,682.28442034)
}
}
{
\newrgbcolor{curcolor}{0 0 0}
\pscustom[linestyle=none,fillstyle=solid,fillcolor=curcolor]
{
\newpath
\moveto(459.91516451,682.85442034)
\curveto(459.93515594,682.79441622)(459.94515593,682.7094163)(459.94516451,682.59942034)
\curveto(459.94515593,682.48941652)(459.93515594,682.40441661)(459.91516451,682.34442034)
\lineto(459.91516451,682.19442034)
\curveto(459.89515598,682.1144169)(459.88515599,682.03441698)(459.88516451,681.95442034)
\curveto(459.89515598,681.87441714)(459.89015598,681.79441722)(459.87016451,681.71442034)
\curveto(459.85015602,681.64441737)(459.83515604,681.57941743)(459.82516451,681.51942034)
\curveto(459.81515606,681.45941755)(459.80515607,681.39441762)(459.79516451,681.32442034)
\curveto(459.75515612,681.2144178)(459.72015615,681.09941791)(459.69016451,680.97942034)
\curveto(459.66015621,680.86941814)(459.62015625,680.76441825)(459.57016451,680.66442034)
\curveto(459.36015651,680.18441883)(459.08515679,679.79441922)(458.74516451,679.49442034)
\curveto(458.40515747,679.19441982)(457.99515788,678.94442007)(457.51516451,678.74442034)
\curveto(457.39515848,678.69442032)(457.2701586,678.65942035)(457.14016451,678.63942034)
\curveto(457.02015885,678.6094204)(456.89515898,678.57942043)(456.76516451,678.54942034)
\curveto(456.71515916,678.52942048)(456.66015921,678.51942049)(456.60016451,678.51942034)
\curveto(456.54015933,678.51942049)(456.48515939,678.5144205)(456.43516451,678.50442034)
\lineto(456.33016451,678.50442034)
\curveto(456.30015957,678.49442052)(456.2701596,678.48942052)(456.24016451,678.48942034)
\curveto(456.19015968,678.47942053)(456.11015976,678.47442054)(456.00016451,678.47442034)
\curveto(455.89015998,678.46442055)(455.80516007,678.46942054)(455.74516451,678.48942034)
\lineto(455.59516451,678.48942034)
\curveto(455.54516033,678.49942051)(455.49016038,678.50442051)(455.43016451,678.50442034)
\curveto(455.38016049,678.49442052)(455.33016054,678.49942051)(455.28016451,678.51942034)
\curveto(455.24016063,678.52942048)(455.20016067,678.53442048)(455.16016451,678.53442034)
\curveto(455.13016074,678.53442048)(455.09016078,678.53942047)(455.04016451,678.54942034)
\curveto(454.94016093,678.57942043)(454.84016103,678.60442041)(454.74016451,678.62442034)
\curveto(454.64016123,678.64442037)(454.54516133,678.67442034)(454.45516451,678.71442034)
\curveto(454.33516154,678.75442026)(454.22016165,678.79442022)(454.11016451,678.83442034)
\curveto(454.01016186,678.87442014)(453.90516197,678.92442009)(453.79516451,678.98442034)
\curveto(453.44516243,679.19441982)(453.14516273,679.43941957)(452.89516451,679.71942034)
\curveto(452.64516323,679.99941901)(452.43516344,680.33441868)(452.26516451,680.72442034)
\curveto(452.21516366,680.8144182)(452.1751637,680.9094181)(452.14516451,681.00942034)
\curveto(452.12516375,681.1094179)(452.10016377,681.2144178)(452.07016451,681.32442034)
\curveto(452.05016382,681.37441764)(452.04016383,681.41941759)(452.04016451,681.45942034)
\curveto(452.04016383,681.49941751)(452.03016384,681.54441747)(452.01016451,681.59442034)
\curveto(451.99016388,681.67441734)(451.98016389,681.75441726)(451.98016451,681.83442034)
\curveto(451.98016389,681.92441709)(451.9701639,682.009417)(451.95016451,682.08942034)
\curveto(451.94016393,682.13941687)(451.93516394,682.18441683)(451.93516451,682.22442034)
\lineto(451.93516451,682.35942034)
\curveto(451.91516396,682.41941659)(451.90516397,682.50441651)(451.90516451,682.61442034)
\curveto(451.91516396,682.72441629)(451.93016394,682.8094162)(451.95016451,682.86942034)
\lineto(451.95016451,682.97442034)
\curveto(451.96016391,683.02441599)(451.96016391,683.07441594)(451.95016451,683.12442034)
\curveto(451.95016392,683.18441583)(451.96016391,683.23941577)(451.98016451,683.28942034)
\curveto(451.99016388,683.33941567)(451.99516388,683.38441563)(451.99516451,683.42442034)
\curveto(451.99516388,683.47441554)(452.00516387,683.52441549)(452.02516451,683.57442034)
\curveto(452.06516381,683.70441531)(452.10016377,683.82941518)(452.13016451,683.94942034)
\curveto(452.16016371,684.07941493)(452.20016367,684.20441481)(452.25016451,684.32442034)
\curveto(452.43016344,684.73441428)(452.64516323,685.07441394)(452.89516451,685.34442034)
\curveto(453.14516273,685.62441339)(453.45016242,685.87941313)(453.81016451,686.10942034)
\curveto(453.91016196,686.15941285)(454.01516186,686.20441281)(454.12516451,686.24442034)
\curveto(454.23516164,686.28441273)(454.34516153,686.32941268)(454.45516451,686.37942034)
\curveto(454.58516129,686.42941258)(454.72016115,686.46441255)(454.86016451,686.48442034)
\curveto(455.00016087,686.50441251)(455.14516073,686.53441248)(455.29516451,686.57442034)
\curveto(455.3751605,686.58441243)(455.45016042,686.58941242)(455.52016451,686.58942034)
\curveto(455.59016028,686.58941242)(455.66016021,686.59441242)(455.73016451,686.60442034)
\curveto(456.31015956,686.6144124)(456.81015906,686.55441246)(457.23016451,686.42442034)
\curveto(457.66015821,686.29441272)(458.04015783,686.1144129)(458.37016451,685.88442034)
\curveto(458.48015739,685.80441321)(458.59015728,685.7144133)(458.70016451,685.61442034)
\curveto(458.82015705,685.52441349)(458.92015695,685.42441359)(459.00016451,685.31442034)
\curveto(459.08015679,685.2144138)(459.15015672,685.1144139)(459.21016451,685.01442034)
\curveto(459.28015659,684.9144141)(459.35015652,684.8094142)(459.42016451,684.69942034)
\curveto(459.49015638,684.58941442)(459.54515633,684.46941454)(459.58516451,684.33942034)
\curveto(459.62515625,684.21941479)(459.6701562,684.08941492)(459.72016451,683.94942034)
\curveto(459.75015612,683.86941514)(459.7751561,683.78441523)(459.79516451,683.69442034)
\lineto(459.85516451,683.42442034)
\curveto(459.86515601,683.38441563)(459.870156,683.34441567)(459.87016451,683.30442034)
\curveto(459.870156,683.26441575)(459.875156,683.22441579)(459.88516451,683.18442034)
\curveto(459.90515597,683.13441588)(459.91015596,683.07941593)(459.90016451,683.01942034)
\curveto(459.89015598,682.95941605)(459.89515598,682.90441611)(459.91516451,682.85442034)
\moveto(457.81516451,682.31442034)
\curveto(457.82515805,682.36441665)(457.83015804,682.43441658)(457.83016451,682.52442034)
\curveto(457.83015804,682.62441639)(457.82515805,682.69941631)(457.81516451,682.74942034)
\lineto(457.81516451,682.86942034)
\curveto(457.79515808,682.91941609)(457.78515809,682.97441604)(457.78516451,683.03442034)
\curveto(457.78515809,683.09441592)(457.78015809,683.14941586)(457.77016451,683.19942034)
\curveto(457.7701581,683.23941577)(457.76515811,683.26941574)(457.75516451,683.28942034)
\lineto(457.69516451,683.52942034)
\curveto(457.68515819,683.61941539)(457.66515821,683.70441531)(457.63516451,683.78442034)
\curveto(457.52515835,684.04441497)(457.39515848,684.26441475)(457.24516451,684.44442034)
\curveto(457.09515878,684.63441438)(456.89515898,684.78441423)(456.64516451,684.89442034)
\curveto(456.58515929,684.9144141)(456.52515935,684.92941408)(456.46516451,684.93942034)
\curveto(456.40515947,684.95941405)(456.34015953,684.97941403)(456.27016451,684.99942034)
\curveto(456.19015968,685.01941399)(456.10515977,685.02441399)(456.01516451,685.01442034)
\lineto(455.74516451,685.01442034)
\curveto(455.71516016,684.99441402)(455.68016019,684.98441403)(455.64016451,684.98442034)
\curveto(455.60016027,684.99441402)(455.56516031,684.99441402)(455.53516451,684.98442034)
\lineto(455.32516451,684.92442034)
\curveto(455.26516061,684.9144141)(455.21016066,684.89441412)(455.16016451,684.86442034)
\curveto(454.91016096,684.75441426)(454.70516117,684.59441442)(454.54516451,684.38442034)
\curveto(454.39516148,684.18441483)(454.2751616,683.94941506)(454.18516451,683.67942034)
\curveto(454.15516172,683.57941543)(454.13016174,683.47441554)(454.11016451,683.36442034)
\curveto(454.10016177,683.25441576)(454.08516179,683.14441587)(454.06516451,683.03442034)
\curveto(454.05516182,682.98441603)(454.05016182,682.93441608)(454.05016451,682.88442034)
\lineto(454.05016451,682.73442034)
\curveto(454.03016184,682.66441635)(454.02016185,682.55941645)(454.02016451,682.41942034)
\curveto(454.03016184,682.27941673)(454.04516183,682.17441684)(454.06516451,682.10442034)
\lineto(454.06516451,681.96942034)
\curveto(454.08516179,681.88941712)(454.10016177,681.8094172)(454.11016451,681.72942034)
\curveto(454.12016175,681.65941735)(454.13516174,681.58441743)(454.15516451,681.50442034)
\curveto(454.25516162,681.20441781)(454.36016151,680.95941805)(454.47016451,680.76942034)
\curveto(454.59016128,680.58941842)(454.7751611,680.42441859)(455.02516451,680.27442034)
\curveto(455.09516078,680.22441879)(455.1701607,680.18441883)(455.25016451,680.15442034)
\curveto(455.34016053,680.12441889)(455.43016044,680.09941891)(455.52016451,680.07942034)
\curveto(455.56016031,680.06941894)(455.59516028,680.06441895)(455.62516451,680.06442034)
\curveto(455.65516022,680.07441894)(455.69016018,680.07441894)(455.73016451,680.06442034)
\lineto(455.85016451,680.03442034)
\curveto(455.90015997,680.03441898)(455.94515993,680.03941897)(455.98516451,680.04942034)
\lineto(456.10516451,680.04942034)
\curveto(456.18515969,680.06941894)(456.26515961,680.08441893)(456.34516451,680.09442034)
\curveto(456.42515945,680.10441891)(456.50015937,680.12441889)(456.57016451,680.15442034)
\curveto(456.83015904,680.25441876)(457.04015883,680.38941862)(457.20016451,680.55942034)
\curveto(457.36015851,680.72941828)(457.49515838,680.93941807)(457.60516451,681.18942034)
\curveto(457.64515823,681.28941772)(457.6751582,681.38941762)(457.69516451,681.48942034)
\curveto(457.71515816,681.58941742)(457.74015813,681.69441732)(457.77016451,681.80442034)
\curveto(457.78015809,681.84441717)(457.78515809,681.87941713)(457.78516451,681.90942034)
\curveto(457.78515809,681.94941706)(457.79015808,681.98941702)(457.80016451,682.02942034)
\lineto(457.80016451,682.16442034)
\curveto(457.80015807,682.2144168)(457.80515807,682.26441675)(457.81516451,682.31442034)
}
}
{
\newrgbcolor{curcolor}{0 0 0}
\pscustom[linestyle=none,fillstyle=solid,fillcolor=curcolor]
{
}
}
{
\newrgbcolor{curcolor}{0 0 0}
\pscustom[linestyle=none,fillstyle=solid,fillcolor=curcolor]
{
\newpath
\moveto(473.06524263,679.52442034)
\lineto(473.06524263,679.10442034)
\curveto(473.06523426,678.97442004)(473.03523429,678.86942014)(472.97524263,678.78942034)
\curveto(472.9252344,678.73942027)(472.86023447,678.70442031)(472.78024263,678.68442034)
\curveto(472.70023463,678.67442034)(472.61023472,678.66942034)(472.51024263,678.66942034)
\lineto(471.68524263,678.66942034)
\lineto(471.40024263,678.66942034)
\curveto(471.32023601,678.67942033)(471.25523607,678.70442031)(471.20524263,678.74442034)
\curveto(471.13523619,678.79442022)(471.09523623,678.85942015)(471.08524263,678.93942034)
\curveto(471.07523625,679.01941999)(471.05523627,679.09941991)(471.02524263,679.17942034)
\curveto(471.00523632,679.19941981)(470.98523634,679.2144198)(470.96524263,679.22442034)
\curveto(470.95523637,679.24441977)(470.94023639,679.26441975)(470.92024263,679.28442034)
\curveto(470.81023652,679.28441973)(470.7302366,679.25941975)(470.68024263,679.20942034)
\lineto(470.53024263,679.05942034)
\curveto(470.46023687,679.00942)(470.39523693,678.96442005)(470.33524263,678.92442034)
\curveto(470.27523705,678.89442012)(470.21023712,678.85442016)(470.14024263,678.80442034)
\curveto(470.10023723,678.78442023)(470.05523727,678.76442025)(470.00524263,678.74442034)
\curveto(469.96523736,678.72442029)(469.92023741,678.70442031)(469.87024263,678.68442034)
\curveto(469.7302376,678.63442038)(469.58023775,678.58942042)(469.42024263,678.54942034)
\curveto(469.37023796,678.52942048)(469.325238,678.51942049)(469.28524263,678.51942034)
\curveto(469.24523808,678.51942049)(469.20523812,678.5144205)(469.16524263,678.50442034)
\lineto(469.03024263,678.50442034)
\curveto(469.00023833,678.49442052)(468.96023837,678.48942052)(468.91024263,678.48942034)
\lineto(468.77524263,678.48942034)
\curveto(468.71523861,678.46942054)(468.6252387,678.46442055)(468.50524263,678.47442034)
\curveto(468.38523894,678.47442054)(468.30023903,678.48442053)(468.25024263,678.50442034)
\curveto(468.18023915,678.52442049)(468.11523921,678.53442048)(468.05524263,678.53442034)
\curveto(468.00523932,678.52442049)(467.95023938,678.52942048)(467.89024263,678.54942034)
\lineto(467.53024263,678.66942034)
\curveto(467.42023991,678.69942031)(467.31024002,678.73942027)(467.20024263,678.78942034)
\curveto(466.85024048,678.93942007)(466.53524079,679.16941984)(466.25524263,679.47942034)
\curveto(465.98524134,679.79941921)(465.77024156,680.13441888)(465.61024263,680.48442034)
\curveto(465.56024177,680.59441842)(465.52024181,680.69941831)(465.49024263,680.79942034)
\curveto(465.46024187,680.9094181)(465.4252419,681.01941799)(465.38524263,681.12942034)
\curveto(465.37524195,681.16941784)(465.37024196,681.20441781)(465.37024263,681.23442034)
\curveto(465.37024196,681.27441774)(465.36024197,681.31941769)(465.34024263,681.36942034)
\curveto(465.32024201,681.44941756)(465.30024203,681.53441748)(465.28024263,681.62442034)
\curveto(465.27024206,681.72441729)(465.25524207,681.82441719)(465.23524263,681.92442034)
\curveto(465.2252421,681.95441706)(465.22024211,681.98941702)(465.22024263,682.02942034)
\curveto(465.2302421,682.06941694)(465.2302421,682.10441691)(465.22024263,682.13442034)
\lineto(465.22024263,682.26942034)
\curveto(465.22024211,682.31941669)(465.21524211,682.36941664)(465.20524263,682.41942034)
\curveto(465.19524213,682.46941654)(465.19024214,682.52441649)(465.19024263,682.58442034)
\curveto(465.19024214,682.65441636)(465.19524213,682.7094163)(465.20524263,682.74942034)
\curveto(465.21524211,682.79941621)(465.22024211,682.84441617)(465.22024263,682.88442034)
\lineto(465.22024263,683.03442034)
\curveto(465.2302421,683.08441593)(465.2302421,683.12941588)(465.22024263,683.16942034)
\curveto(465.22024211,683.21941579)(465.2302421,683.26941574)(465.25024263,683.31942034)
\curveto(465.27024206,683.42941558)(465.28524204,683.53441548)(465.29524263,683.63442034)
\curveto(465.31524201,683.73441528)(465.34024199,683.83441518)(465.37024263,683.93442034)
\curveto(465.41024192,684.05441496)(465.44524188,684.16941484)(465.47524263,684.27942034)
\curveto(465.50524182,684.38941462)(465.54524178,684.49941451)(465.59524263,684.60942034)
\curveto(465.73524159,684.9094141)(465.91024142,685.19441382)(466.12024263,685.46442034)
\curveto(466.14024119,685.49441352)(466.16524116,685.51941349)(466.19524263,685.53942034)
\curveto(466.23524109,685.56941344)(466.26524106,685.59941341)(466.28524263,685.62942034)
\curveto(466.325241,685.67941333)(466.36524096,685.72441329)(466.40524263,685.76442034)
\curveto(466.44524088,685.80441321)(466.49024084,685.84441317)(466.54024263,685.88442034)
\curveto(466.58024075,685.90441311)(466.61524071,685.92941308)(466.64524263,685.95942034)
\curveto(466.67524065,685.99941301)(466.71024062,686.02941298)(466.75024263,686.04942034)
\curveto(467.00024033,686.21941279)(467.29024004,686.35941265)(467.62024263,686.46942034)
\curveto(467.69023964,686.48941252)(467.76023957,686.50441251)(467.83024263,686.51442034)
\curveto(467.91023942,686.52441249)(467.99023934,686.53941247)(468.07024263,686.55942034)
\curveto(468.14023919,686.57941243)(468.2302391,686.58941242)(468.34024263,686.58942034)
\curveto(468.45023888,686.59941241)(468.56023877,686.60441241)(468.67024263,686.60442034)
\curveto(468.78023855,686.60441241)(468.88523844,686.59941241)(468.98524263,686.58942034)
\curveto(469.09523823,686.57941243)(469.18523814,686.56441245)(469.25524263,686.54442034)
\curveto(469.40523792,686.49441252)(469.55023778,686.44941256)(469.69024263,686.40942034)
\curveto(469.8302375,686.36941264)(469.96023737,686.3144127)(470.08024263,686.24442034)
\curveto(470.15023718,686.19441282)(470.21523711,686.14441287)(470.27524263,686.09442034)
\curveto(470.33523699,686.05441296)(470.40023693,686.009413)(470.47024263,685.95942034)
\curveto(470.51023682,685.92941308)(470.56523676,685.88941312)(470.63524263,685.83942034)
\curveto(470.71523661,685.78941322)(470.79023654,685.78941322)(470.86024263,685.83942034)
\curveto(470.90023643,685.85941315)(470.92023641,685.89441312)(470.92024263,685.94442034)
\curveto(470.92023641,685.99441302)(470.9302364,686.04441297)(470.95024263,686.09442034)
\lineto(470.95024263,686.24442034)
\curveto(470.96023637,686.27441274)(470.96523636,686.3094127)(470.96524263,686.34942034)
\lineto(470.96524263,686.46942034)
\lineto(470.96524263,688.50942034)
\curveto(470.96523636,688.61941039)(470.96023637,688.73941027)(470.95024263,688.86942034)
\curveto(470.95023638,689.00941)(470.97523635,689.1144099)(471.02524263,689.18442034)
\curveto(471.06523626,689.26440975)(471.14023619,689.3144097)(471.25024263,689.33442034)
\curveto(471.27023606,689.34440967)(471.29023604,689.34440967)(471.31024263,689.33442034)
\curveto(471.330236,689.33440968)(471.35023598,689.33940967)(471.37024263,689.34942034)
\lineto(472.43524263,689.34942034)
\curveto(472.55523477,689.34940966)(472.66523466,689.34440967)(472.76524263,689.33442034)
\curveto(472.86523446,689.32440969)(472.94023439,689.28440973)(472.99024263,689.21442034)
\curveto(473.04023429,689.13440988)(473.06523426,689.02940998)(473.06524263,688.89942034)
\lineto(473.06524263,688.53942034)
\lineto(473.06524263,679.52442034)
\moveto(471.02524263,682.46442034)
\curveto(471.03523629,682.50441651)(471.03523629,682.54441647)(471.02524263,682.58442034)
\lineto(471.02524263,682.71942034)
\curveto(471.0252363,682.81941619)(471.02023631,682.91941609)(471.01024263,683.01942034)
\curveto(471.00023633,683.11941589)(470.98523634,683.2094158)(470.96524263,683.28942034)
\curveto(470.94523638,683.39941561)(470.9252364,683.49941551)(470.90524263,683.58942034)
\curveto(470.89523643,683.67941533)(470.87023646,683.76441525)(470.83024263,683.84442034)
\curveto(470.69023664,684.20441481)(470.48523684,684.48941452)(470.21524263,684.69942034)
\curveto(469.95523737,684.9094141)(469.57523775,685.014414)(469.07524263,685.01442034)
\curveto(469.01523831,685.014414)(468.93523839,685.00441401)(468.83524263,684.98442034)
\curveto(468.75523857,684.96441405)(468.68023865,684.94441407)(468.61024263,684.92442034)
\curveto(468.55023878,684.9144141)(468.49023884,684.89441412)(468.43024263,684.86442034)
\curveto(468.16023917,684.75441426)(467.95023938,684.58441443)(467.80024263,684.35442034)
\curveto(467.65023968,684.12441489)(467.5302398,683.86441515)(467.44024263,683.57442034)
\curveto(467.41023992,683.47441554)(467.39023994,683.37441564)(467.38024263,683.27442034)
\curveto(467.37023996,683.17441584)(467.35023998,683.06941594)(467.32024263,682.95942034)
\lineto(467.32024263,682.74942034)
\curveto(467.30024003,682.65941635)(467.29524003,682.53441648)(467.30524263,682.37442034)
\curveto(467.31524001,682.22441679)(467.33024,682.1144169)(467.35024263,682.04442034)
\lineto(467.35024263,681.95442034)
\curveto(467.36023997,681.93441708)(467.36523996,681.9144171)(467.36524263,681.89442034)
\curveto(467.38523994,681.8144172)(467.40023993,681.73941727)(467.41024263,681.66942034)
\curveto(467.4302399,681.59941741)(467.45023988,681.52441749)(467.47024263,681.44442034)
\curveto(467.64023969,680.92441809)(467.9302394,680.53941847)(468.34024263,680.28942034)
\curveto(468.47023886,680.19941881)(468.65023868,680.12941888)(468.88024263,680.07942034)
\curveto(468.92023841,680.06941894)(468.98023835,680.06441895)(469.06024263,680.06442034)
\curveto(469.09023824,680.05441896)(469.13523819,680.04441897)(469.19524263,680.03442034)
\curveto(469.26523806,680.03441898)(469.32023801,680.03941897)(469.36024263,680.04942034)
\curveto(469.44023789,680.06941894)(469.52023781,680.08441893)(469.60024263,680.09442034)
\curveto(469.68023765,680.10441891)(469.76023757,680.12441889)(469.84024263,680.15442034)
\curveto(470.09023724,680.26441875)(470.29023704,680.40441861)(470.44024263,680.57442034)
\curveto(470.59023674,680.74441827)(470.72023661,680.95941805)(470.83024263,681.21942034)
\curveto(470.87023646,681.3094177)(470.90023643,681.39941761)(470.92024263,681.48942034)
\curveto(470.94023639,681.58941742)(470.96023637,681.69441732)(470.98024263,681.80442034)
\curveto(470.99023634,681.85441716)(470.99023634,681.89941711)(470.98024263,681.93942034)
\curveto(470.98023635,681.98941702)(470.99023634,682.03941697)(471.01024263,682.08942034)
\curveto(471.02023631,682.11941689)(471.0252363,682.15441686)(471.02524263,682.19442034)
\lineto(471.02524263,682.32942034)
\lineto(471.02524263,682.46442034)
}
}
{
\newrgbcolor{curcolor}{0 0 0}
\pscustom[linestyle=none,fillstyle=solid,fillcolor=curcolor]
{
\newpath
\moveto(482.01016451,682.61442034)
\curveto(482.03015634,682.53441648)(482.03015634,682.44441657)(482.01016451,682.34442034)
\curveto(481.99015638,682.24441677)(481.95515642,682.17941683)(481.90516451,682.14942034)
\curveto(481.85515652,682.1094169)(481.78015659,682.07941693)(481.68016451,682.05942034)
\curveto(481.59015678,682.04941696)(481.48515689,682.03941697)(481.36516451,682.02942034)
\lineto(481.02016451,682.02942034)
\curveto(480.91015746,682.03941697)(480.81015756,682.04441697)(480.72016451,682.04442034)
\lineto(477.06016451,682.04442034)
\lineto(476.85016451,682.04442034)
\curveto(476.79016158,682.04441697)(476.73516164,682.03441698)(476.68516451,682.01442034)
\curveto(476.60516177,681.97441704)(476.55516182,681.93441708)(476.53516451,681.89442034)
\curveto(476.51516186,681.87441714)(476.49516188,681.83441718)(476.47516451,681.77442034)
\curveto(476.45516192,681.72441729)(476.45016192,681.67441734)(476.46016451,681.62442034)
\curveto(476.48016189,681.56441745)(476.49016188,681.50441751)(476.49016451,681.44442034)
\curveto(476.50016187,681.39441762)(476.51516186,681.33941767)(476.53516451,681.27942034)
\curveto(476.61516176,681.03941797)(476.71016166,680.83941817)(476.82016451,680.67942034)
\curveto(476.94016143,680.52941848)(477.10016127,680.39441862)(477.30016451,680.27442034)
\curveto(477.38016099,680.22441879)(477.46016091,680.18941882)(477.54016451,680.16942034)
\curveto(477.63016074,680.15941885)(477.72016065,680.13941887)(477.81016451,680.10942034)
\curveto(477.89016048,680.08941892)(478.00016037,680.07441894)(478.14016451,680.06442034)
\curveto(478.28016009,680.05441896)(478.40015997,680.05941895)(478.50016451,680.07942034)
\lineto(478.63516451,680.07942034)
\curveto(478.73515964,680.09941891)(478.82515955,680.11941889)(478.90516451,680.13942034)
\curveto(478.99515938,680.16941884)(479.08015929,680.19941881)(479.16016451,680.22942034)
\curveto(479.26015911,680.27941873)(479.370159,680.34441867)(479.49016451,680.42442034)
\curveto(479.62015875,680.50441851)(479.71515866,680.58441843)(479.77516451,680.66442034)
\curveto(479.82515855,680.73441828)(479.8751585,680.79941821)(479.92516451,680.85942034)
\curveto(479.98515839,680.92941808)(480.05515832,680.97941803)(480.13516451,681.00942034)
\curveto(480.23515814,681.05941795)(480.36015801,681.07941793)(480.51016451,681.06942034)
\lineto(480.94516451,681.06942034)
\lineto(481.12516451,681.06942034)
\curveto(481.19515718,681.07941793)(481.25515712,681.07441794)(481.30516451,681.05442034)
\lineto(481.45516451,681.05442034)
\curveto(481.55515682,681.03441798)(481.62515675,681.009418)(481.66516451,680.97942034)
\curveto(481.70515667,680.95941805)(481.72515665,680.9144181)(481.72516451,680.84442034)
\curveto(481.73515664,680.77441824)(481.73015664,680.7144183)(481.71016451,680.66442034)
\curveto(481.66015671,680.52441849)(481.60515677,680.39941861)(481.54516451,680.28942034)
\curveto(481.48515689,680.17941883)(481.41515696,680.06941894)(481.33516451,679.95942034)
\curveto(481.11515726,679.62941938)(480.86515751,679.36441965)(480.58516451,679.16442034)
\curveto(480.30515807,678.96442005)(479.95515842,678.79442022)(479.53516451,678.65442034)
\curveto(479.42515895,678.6144204)(479.31515906,678.58942042)(479.20516451,678.57942034)
\curveto(479.09515928,678.56942044)(478.98015939,678.54942046)(478.86016451,678.51942034)
\curveto(478.82015955,678.5094205)(478.7751596,678.5094205)(478.72516451,678.51942034)
\curveto(478.68515969,678.51942049)(478.64515973,678.5144205)(478.60516451,678.50442034)
\lineto(478.44016451,678.50442034)
\curveto(478.39015998,678.48442053)(478.33016004,678.47942053)(478.26016451,678.48942034)
\curveto(478.20016017,678.48942052)(478.14516023,678.49442052)(478.09516451,678.50442034)
\curveto(478.01516036,678.5144205)(477.94516043,678.5144205)(477.88516451,678.50442034)
\curveto(477.82516055,678.49442052)(477.76016061,678.49942051)(477.69016451,678.51942034)
\curveto(477.64016073,678.53942047)(477.58516079,678.54942046)(477.52516451,678.54942034)
\curveto(477.46516091,678.54942046)(477.41016096,678.55942045)(477.36016451,678.57942034)
\curveto(477.25016112,678.59942041)(477.14016123,678.62442039)(477.03016451,678.65442034)
\curveto(476.92016145,678.67442034)(476.82016155,678.7094203)(476.73016451,678.75942034)
\curveto(476.62016175,678.79942021)(476.51516186,678.83442018)(476.41516451,678.86442034)
\curveto(476.32516205,678.90442011)(476.24016213,678.94942006)(476.16016451,678.99942034)
\curveto(475.84016253,679.19941981)(475.55516282,679.42941958)(475.30516451,679.68942034)
\curveto(475.05516332,679.95941905)(474.85016352,680.26941874)(474.69016451,680.61942034)
\curveto(474.64016373,680.72941828)(474.60016377,680.83941817)(474.57016451,680.94942034)
\curveto(474.54016383,681.06941794)(474.50016387,681.18941782)(474.45016451,681.30942034)
\curveto(474.44016393,681.34941766)(474.43516394,681.38441763)(474.43516451,681.41442034)
\curveto(474.43516394,681.45441756)(474.43016394,681.49441752)(474.42016451,681.53442034)
\curveto(474.38016399,681.65441736)(474.35516402,681.78441723)(474.34516451,681.92442034)
\lineto(474.31516451,682.34442034)
\curveto(474.31516406,682.39441662)(474.31016406,682.44941656)(474.30016451,682.50942034)
\curveto(474.30016407,682.56941644)(474.30516407,682.62441639)(474.31516451,682.67442034)
\lineto(474.31516451,682.85442034)
\lineto(474.36016451,683.21442034)
\curveto(474.40016397,683.38441563)(474.43516394,683.54941546)(474.46516451,683.70942034)
\curveto(474.49516388,683.86941514)(474.54016383,684.01941499)(474.60016451,684.15942034)
\curveto(475.03016334,685.19941381)(475.76016261,685.93441308)(476.79016451,686.36442034)
\curveto(476.93016144,686.42441259)(477.0701613,686.46441255)(477.21016451,686.48442034)
\curveto(477.36016101,686.5144125)(477.51516086,686.54941246)(477.67516451,686.58942034)
\curveto(477.75516062,686.59941241)(477.83016054,686.60441241)(477.90016451,686.60442034)
\curveto(477.9701604,686.60441241)(478.04516033,686.6094124)(478.12516451,686.61942034)
\curveto(478.63515974,686.62941238)(479.0701593,686.56941244)(479.43016451,686.43942034)
\curveto(479.80015857,686.31941269)(480.13015824,686.15941285)(480.42016451,685.95942034)
\curveto(480.51015786,685.89941311)(480.60015777,685.82941318)(480.69016451,685.74942034)
\curveto(480.78015759,685.67941333)(480.86015751,685.60441341)(480.93016451,685.52442034)
\curveto(480.96015741,685.47441354)(481.00015737,685.43441358)(481.05016451,685.40442034)
\curveto(481.13015724,685.29441372)(481.20515717,685.17941383)(481.27516451,685.05942034)
\curveto(481.34515703,684.94941406)(481.42015695,684.83441418)(481.50016451,684.71442034)
\curveto(481.55015682,684.62441439)(481.59015678,684.52941448)(481.62016451,684.42942034)
\curveto(481.66015671,684.33941467)(481.70015667,684.23941477)(481.74016451,684.12942034)
\curveto(481.79015658,683.99941501)(481.83015654,683.86441515)(481.86016451,683.72442034)
\curveto(481.89015648,683.58441543)(481.92515645,683.44441557)(481.96516451,683.30442034)
\curveto(481.98515639,683.22441579)(481.99015638,683.13441588)(481.98016451,683.03442034)
\curveto(481.98015639,682.94441607)(481.99015638,682.85941615)(482.01016451,682.77942034)
\lineto(482.01016451,682.61442034)
\moveto(479.76016451,683.49942034)
\curveto(479.83015854,683.59941541)(479.83515854,683.71941529)(479.77516451,683.85942034)
\curveto(479.72515865,684.009415)(479.68515869,684.11941489)(479.65516451,684.18942034)
\curveto(479.51515886,684.45941455)(479.33015904,684.66441435)(479.10016451,684.80442034)
\curveto(478.8701595,684.95441406)(478.55015982,685.03441398)(478.14016451,685.04442034)
\curveto(478.11016026,685.02441399)(478.0751603,685.01941399)(478.03516451,685.02942034)
\curveto(477.99516038,685.03941397)(477.96016041,685.03941397)(477.93016451,685.02942034)
\curveto(477.88016049,685.009414)(477.82516055,684.99441402)(477.76516451,684.98442034)
\curveto(477.70516067,684.98441403)(477.65016072,684.97441404)(477.60016451,684.95442034)
\curveto(477.16016121,684.8144142)(476.83516154,684.53941447)(476.62516451,684.12942034)
\curveto(476.60516177,684.08941492)(476.58016179,684.03441498)(476.55016451,683.96442034)
\curveto(476.53016184,683.90441511)(476.51516186,683.83941517)(476.50516451,683.76942034)
\curveto(476.49516188,683.7094153)(476.49516188,683.64941536)(476.50516451,683.58942034)
\curveto(476.52516185,683.52941548)(476.56016181,683.47941553)(476.61016451,683.43942034)
\curveto(476.69016168,683.38941562)(476.80016157,683.36441565)(476.94016451,683.36442034)
\lineto(477.34516451,683.36442034)
\lineto(479.01016451,683.36442034)
\lineto(479.44516451,683.36442034)
\curveto(479.60515877,683.37441564)(479.71015866,683.41941559)(479.76016451,683.49942034)
}
}
{
\newrgbcolor{curcolor}{0 0 0}
\pscustom[linestyle=none,fillstyle=solid,fillcolor=curcolor]
{
}
}
{
\newrgbcolor{curcolor}{0 0 0}
\pscustom[linestyle=none,fillstyle=solid,fillcolor=curcolor]
{
\newpath
\moveto(491.84360201,686.60442034)
\curveto(491.95359669,686.60441241)(492.0485966,686.59441242)(492.12860201,686.57442034)
\curveto(492.21859643,686.55441246)(492.28859636,686.5094125)(492.33860201,686.43942034)
\curveto(492.39859625,686.35941265)(492.42859622,686.21941279)(492.42860201,686.01942034)
\lineto(492.42860201,685.50942034)
\lineto(492.42860201,685.13442034)
\curveto(492.43859621,684.99441402)(492.42359622,684.88441413)(492.38360201,684.80442034)
\curveto(492.3435963,684.73441428)(492.28359636,684.68941432)(492.20360201,684.66942034)
\curveto(492.13359651,684.64941436)(492.0485966,684.63941437)(491.94860201,684.63942034)
\curveto(491.85859679,684.63941437)(491.75859689,684.64441437)(491.64860201,684.65442034)
\curveto(491.5485971,684.66441435)(491.45359719,684.65941435)(491.36360201,684.63942034)
\curveto(491.29359735,684.61941439)(491.22359742,684.60441441)(491.15360201,684.59442034)
\curveto(491.08359756,684.59441442)(491.01859763,684.58441443)(490.95860201,684.56442034)
\curveto(490.79859785,684.5144145)(490.63859801,684.43941457)(490.47860201,684.33942034)
\curveto(490.31859833,684.24941476)(490.19359845,684.14441487)(490.10360201,684.02442034)
\curveto(490.05359859,683.94441507)(489.99859865,683.85941515)(489.93860201,683.76942034)
\curveto(489.88859876,683.68941532)(489.83859881,683.60441541)(489.78860201,683.51442034)
\curveto(489.75859889,683.43441558)(489.72859892,683.34941566)(489.69860201,683.25942034)
\lineto(489.63860201,683.01942034)
\curveto(489.61859903,682.94941606)(489.60859904,682.87441614)(489.60860201,682.79442034)
\curveto(489.60859904,682.72441629)(489.59859905,682.65441636)(489.57860201,682.58442034)
\curveto(489.56859908,682.54441647)(489.56359908,682.50441651)(489.56360201,682.46442034)
\curveto(489.57359907,682.43441658)(489.57359907,682.40441661)(489.56360201,682.37442034)
\lineto(489.56360201,682.13442034)
\curveto(489.5435991,682.06441695)(489.53859911,681.98441703)(489.54860201,681.89442034)
\curveto(489.55859909,681.8144172)(489.56359908,681.73441728)(489.56360201,681.65442034)
\lineto(489.56360201,680.69442034)
\lineto(489.56360201,679.41942034)
\curveto(489.56359908,679.28941972)(489.55859909,679.16941984)(489.54860201,679.05942034)
\curveto(489.53859911,678.94942006)(489.50859914,678.85942015)(489.45860201,678.78942034)
\curveto(489.43859921,678.75942025)(489.40359924,678.73442028)(489.35360201,678.71442034)
\curveto(489.31359933,678.70442031)(489.26859938,678.69442032)(489.21860201,678.68442034)
\lineto(489.14360201,678.68442034)
\curveto(489.09359955,678.67442034)(489.03859961,678.66942034)(488.97860201,678.66942034)
\lineto(488.81360201,678.66942034)
\lineto(488.16860201,678.66942034)
\curveto(488.10860054,678.67942033)(488.0436006,678.68442033)(487.97360201,678.68442034)
\lineto(487.77860201,678.68442034)
\curveto(487.72860092,678.70442031)(487.67860097,678.71942029)(487.62860201,678.72942034)
\curveto(487.57860107,678.74942026)(487.5436011,678.78442023)(487.52360201,678.83442034)
\curveto(487.48360116,678.88442013)(487.45860119,678.95442006)(487.44860201,679.04442034)
\lineto(487.44860201,679.34442034)
\lineto(487.44860201,680.36442034)
\lineto(487.44860201,684.59442034)
\lineto(487.44860201,685.70442034)
\lineto(487.44860201,685.98942034)
\curveto(487.4486012,686.08941292)(487.46860118,686.16941284)(487.50860201,686.22942034)
\curveto(487.55860109,686.3094127)(487.63360101,686.35941265)(487.73360201,686.37942034)
\curveto(487.83360081,686.39941261)(487.95360069,686.4094126)(488.09360201,686.40942034)
\lineto(488.85860201,686.40942034)
\curveto(488.97859967,686.4094126)(489.08359956,686.39941261)(489.17360201,686.37942034)
\curveto(489.26359938,686.36941264)(489.33359931,686.32441269)(489.38360201,686.24442034)
\curveto(489.41359923,686.19441282)(489.42859922,686.12441289)(489.42860201,686.03442034)
\lineto(489.45860201,685.76442034)
\curveto(489.46859918,685.68441333)(489.48359916,685.6094134)(489.50360201,685.53942034)
\curveto(489.53359911,685.46941354)(489.58359906,685.43441358)(489.65360201,685.43442034)
\curveto(489.67359897,685.45441356)(489.69359895,685.46441355)(489.71360201,685.46442034)
\curveto(489.73359891,685.46441355)(489.75359889,685.47441354)(489.77360201,685.49442034)
\curveto(489.83359881,685.54441347)(489.88359876,685.59941341)(489.92360201,685.65942034)
\curveto(489.97359867,685.72941328)(490.03359861,685.78941322)(490.10360201,685.83942034)
\curveto(490.1435985,685.86941314)(490.17859847,685.89941311)(490.20860201,685.92942034)
\curveto(490.23859841,685.96941304)(490.27359837,686.00441301)(490.31360201,686.03442034)
\lineto(490.58360201,686.21442034)
\curveto(490.68359796,686.27441274)(490.78359786,686.32941268)(490.88360201,686.37942034)
\curveto(490.98359766,686.41941259)(491.08359756,686.45441256)(491.18360201,686.48442034)
\lineto(491.51360201,686.57442034)
\curveto(491.5435971,686.58441243)(491.59859705,686.58441243)(491.67860201,686.57442034)
\curveto(491.76859688,686.57441244)(491.82359682,686.58441243)(491.84360201,686.60442034)
}
}
{
\newrgbcolor{curcolor}{0 0 0}
\pscustom[linestyle=none,fillstyle=solid,fillcolor=curcolor]
{
\newpath
\moveto(500.35000826,682.61442034)
\curveto(500.37000009,682.53441648)(500.37000009,682.44441657)(500.35000826,682.34442034)
\curveto(500.33000013,682.24441677)(500.29500017,682.17941683)(500.24500826,682.14942034)
\curveto(500.19500027,682.1094169)(500.12000034,682.07941693)(500.02000826,682.05942034)
\curveto(499.93000053,682.04941696)(499.82500064,682.03941697)(499.70500826,682.02942034)
\lineto(499.36000826,682.02942034)
\curveto(499.25000121,682.03941697)(499.15000131,682.04441697)(499.06000826,682.04442034)
\lineto(495.40000826,682.04442034)
\lineto(495.19000826,682.04442034)
\curveto(495.13000533,682.04441697)(495.07500539,682.03441698)(495.02500826,682.01442034)
\curveto(494.94500552,681.97441704)(494.89500557,681.93441708)(494.87500826,681.89442034)
\curveto(494.85500561,681.87441714)(494.83500563,681.83441718)(494.81500826,681.77442034)
\curveto(494.79500567,681.72441729)(494.79000567,681.67441734)(494.80000826,681.62442034)
\curveto(494.82000564,681.56441745)(494.83000563,681.50441751)(494.83000826,681.44442034)
\curveto(494.84000562,681.39441762)(494.85500561,681.33941767)(494.87500826,681.27942034)
\curveto(494.95500551,681.03941797)(495.05000541,680.83941817)(495.16000826,680.67942034)
\curveto(495.28000518,680.52941848)(495.44000502,680.39441862)(495.64000826,680.27442034)
\curveto(495.72000474,680.22441879)(495.80000466,680.18941882)(495.88000826,680.16942034)
\curveto(495.97000449,680.15941885)(496.0600044,680.13941887)(496.15000826,680.10942034)
\curveto(496.23000423,680.08941892)(496.34000412,680.07441894)(496.48000826,680.06442034)
\curveto(496.62000384,680.05441896)(496.74000372,680.05941895)(496.84000826,680.07942034)
\lineto(496.97500826,680.07942034)
\curveto(497.07500339,680.09941891)(497.1650033,680.11941889)(497.24500826,680.13942034)
\curveto(497.33500313,680.16941884)(497.42000304,680.19941881)(497.50000826,680.22942034)
\curveto(497.60000286,680.27941873)(497.71000275,680.34441867)(497.83000826,680.42442034)
\curveto(497.9600025,680.50441851)(498.05500241,680.58441843)(498.11500826,680.66442034)
\curveto(498.1650023,680.73441828)(498.21500225,680.79941821)(498.26500826,680.85942034)
\curveto(498.32500214,680.92941808)(498.39500207,680.97941803)(498.47500826,681.00942034)
\curveto(498.57500189,681.05941795)(498.70000176,681.07941793)(498.85000826,681.06942034)
\lineto(499.28500826,681.06942034)
\lineto(499.46500826,681.06942034)
\curveto(499.53500093,681.07941793)(499.59500087,681.07441794)(499.64500826,681.05442034)
\lineto(499.79500826,681.05442034)
\curveto(499.89500057,681.03441798)(499.9650005,681.009418)(500.00500826,680.97942034)
\curveto(500.04500042,680.95941805)(500.0650004,680.9144181)(500.06500826,680.84442034)
\curveto(500.07500039,680.77441824)(500.07000039,680.7144183)(500.05000826,680.66442034)
\curveto(500.00000046,680.52441849)(499.94500052,680.39941861)(499.88500826,680.28942034)
\curveto(499.82500064,680.17941883)(499.75500071,680.06941894)(499.67500826,679.95942034)
\curveto(499.45500101,679.62941938)(499.20500126,679.36441965)(498.92500826,679.16442034)
\curveto(498.64500182,678.96442005)(498.29500217,678.79442022)(497.87500826,678.65442034)
\curveto(497.7650027,678.6144204)(497.65500281,678.58942042)(497.54500826,678.57942034)
\curveto(497.43500303,678.56942044)(497.32000314,678.54942046)(497.20000826,678.51942034)
\curveto(497.1600033,678.5094205)(497.11500335,678.5094205)(497.06500826,678.51942034)
\curveto(497.02500344,678.51942049)(496.98500348,678.5144205)(496.94500826,678.50442034)
\lineto(496.78000826,678.50442034)
\curveto(496.73000373,678.48442053)(496.67000379,678.47942053)(496.60000826,678.48942034)
\curveto(496.54000392,678.48942052)(496.48500398,678.49442052)(496.43500826,678.50442034)
\curveto(496.35500411,678.5144205)(496.28500418,678.5144205)(496.22500826,678.50442034)
\curveto(496.1650043,678.49442052)(496.10000436,678.49942051)(496.03000826,678.51942034)
\curveto(495.98000448,678.53942047)(495.92500454,678.54942046)(495.86500826,678.54942034)
\curveto(495.80500466,678.54942046)(495.75000471,678.55942045)(495.70000826,678.57942034)
\curveto(495.59000487,678.59942041)(495.48000498,678.62442039)(495.37000826,678.65442034)
\curveto(495.2600052,678.67442034)(495.1600053,678.7094203)(495.07000826,678.75942034)
\curveto(494.9600055,678.79942021)(494.85500561,678.83442018)(494.75500826,678.86442034)
\curveto(494.6650058,678.90442011)(494.58000588,678.94942006)(494.50000826,678.99942034)
\curveto(494.18000628,679.19941981)(493.89500657,679.42941958)(493.64500826,679.68942034)
\curveto(493.39500707,679.95941905)(493.19000727,680.26941874)(493.03000826,680.61942034)
\curveto(492.98000748,680.72941828)(492.94000752,680.83941817)(492.91000826,680.94942034)
\curveto(492.88000758,681.06941794)(492.84000762,681.18941782)(492.79000826,681.30942034)
\curveto(492.78000768,681.34941766)(492.77500769,681.38441763)(492.77500826,681.41442034)
\curveto(492.77500769,681.45441756)(492.77000769,681.49441752)(492.76000826,681.53442034)
\curveto(492.72000774,681.65441736)(492.69500777,681.78441723)(492.68500826,681.92442034)
\lineto(492.65500826,682.34442034)
\curveto(492.65500781,682.39441662)(492.65000781,682.44941656)(492.64000826,682.50942034)
\curveto(492.64000782,682.56941644)(492.64500782,682.62441639)(492.65500826,682.67442034)
\lineto(492.65500826,682.85442034)
\lineto(492.70000826,683.21442034)
\curveto(492.74000772,683.38441563)(492.77500769,683.54941546)(492.80500826,683.70942034)
\curveto(492.83500763,683.86941514)(492.88000758,684.01941499)(492.94000826,684.15942034)
\curveto(493.37000709,685.19941381)(494.10000636,685.93441308)(495.13000826,686.36442034)
\curveto(495.27000519,686.42441259)(495.41000505,686.46441255)(495.55000826,686.48442034)
\curveto(495.70000476,686.5144125)(495.85500461,686.54941246)(496.01500826,686.58942034)
\curveto(496.09500437,686.59941241)(496.17000429,686.60441241)(496.24000826,686.60442034)
\curveto(496.31000415,686.60441241)(496.38500408,686.6094124)(496.46500826,686.61942034)
\curveto(496.97500349,686.62941238)(497.41000305,686.56941244)(497.77000826,686.43942034)
\curveto(498.14000232,686.31941269)(498.47000199,686.15941285)(498.76000826,685.95942034)
\curveto(498.85000161,685.89941311)(498.94000152,685.82941318)(499.03000826,685.74942034)
\curveto(499.12000134,685.67941333)(499.20000126,685.60441341)(499.27000826,685.52442034)
\curveto(499.30000116,685.47441354)(499.34000112,685.43441358)(499.39000826,685.40442034)
\curveto(499.47000099,685.29441372)(499.54500092,685.17941383)(499.61500826,685.05942034)
\curveto(499.68500078,684.94941406)(499.7600007,684.83441418)(499.84000826,684.71442034)
\curveto(499.89000057,684.62441439)(499.93000053,684.52941448)(499.96000826,684.42942034)
\curveto(500.00000046,684.33941467)(500.04000042,684.23941477)(500.08000826,684.12942034)
\curveto(500.13000033,683.99941501)(500.17000029,683.86441515)(500.20000826,683.72442034)
\curveto(500.23000023,683.58441543)(500.2650002,683.44441557)(500.30500826,683.30442034)
\curveto(500.32500014,683.22441579)(500.33000013,683.13441588)(500.32000826,683.03442034)
\curveto(500.32000014,682.94441607)(500.33000013,682.85941615)(500.35000826,682.77942034)
\lineto(500.35000826,682.61442034)
\moveto(498.10000826,683.49942034)
\curveto(498.17000229,683.59941541)(498.17500229,683.71941529)(498.11500826,683.85942034)
\curveto(498.0650024,684.009415)(498.02500244,684.11941489)(497.99500826,684.18942034)
\curveto(497.85500261,684.45941455)(497.67000279,684.66441435)(497.44000826,684.80442034)
\curveto(497.21000325,684.95441406)(496.89000357,685.03441398)(496.48000826,685.04442034)
\curveto(496.45000401,685.02441399)(496.41500405,685.01941399)(496.37500826,685.02942034)
\curveto(496.33500413,685.03941397)(496.30000416,685.03941397)(496.27000826,685.02942034)
\curveto(496.22000424,685.009414)(496.1650043,684.99441402)(496.10500826,684.98442034)
\curveto(496.04500442,684.98441403)(495.99000447,684.97441404)(495.94000826,684.95442034)
\curveto(495.50000496,684.8144142)(495.17500529,684.53941447)(494.96500826,684.12942034)
\curveto(494.94500552,684.08941492)(494.92000554,684.03441498)(494.89000826,683.96442034)
\curveto(494.87000559,683.90441511)(494.85500561,683.83941517)(494.84500826,683.76942034)
\curveto(494.83500563,683.7094153)(494.83500563,683.64941536)(494.84500826,683.58942034)
\curveto(494.8650056,683.52941548)(494.90000556,683.47941553)(494.95000826,683.43942034)
\curveto(495.03000543,683.38941562)(495.14000532,683.36441565)(495.28000826,683.36442034)
\lineto(495.68500826,683.36442034)
\lineto(497.35000826,683.36442034)
\lineto(497.78500826,683.36442034)
\curveto(497.94500252,683.37441564)(498.05000241,683.41941559)(498.10000826,683.49942034)
}
}
{
\newrgbcolor{curcolor}{0 0 0}
\pscustom[linestyle=none,fillstyle=solid,fillcolor=curcolor]
{
\newpath
\moveto(505.16828951,686.61942034)
\curveto(505.97828435,686.63941237)(506.65328367,686.51941249)(507.19328951,686.25942034)
\curveto(507.74328258,685.99941301)(508.17828215,685.62941338)(508.49828951,685.14942034)
\curveto(508.65828167,684.9094141)(508.77828155,684.63441438)(508.85828951,684.32442034)
\curveto(508.87828145,684.27441474)(508.89328143,684.2094148)(508.90328951,684.12942034)
\curveto(508.9232814,684.04941496)(508.9232814,683.97941503)(508.90328951,683.91942034)
\curveto(508.86328146,683.8094152)(508.79328153,683.74441527)(508.69328951,683.72442034)
\curveto(508.59328173,683.7144153)(508.47328185,683.7094153)(508.33328951,683.70942034)
\lineto(507.55328951,683.70942034)
\lineto(507.26828951,683.70942034)
\curveto(507.17828315,683.7094153)(507.10328322,683.72941528)(507.04328951,683.76942034)
\curveto(506.96328336,683.8094152)(506.90828342,683.86941514)(506.87828951,683.94942034)
\curveto(506.84828348,684.03941497)(506.80828352,684.12941488)(506.75828951,684.21942034)
\curveto(506.69828363,684.32941468)(506.63328369,684.42941458)(506.56328951,684.51942034)
\curveto(506.49328383,684.6094144)(506.41328391,684.68941432)(506.32328951,684.75942034)
\curveto(506.18328414,684.84941416)(506.0282843,684.91941409)(505.85828951,684.96942034)
\curveto(505.79828453,684.98941402)(505.73828459,684.99941401)(505.67828951,684.99942034)
\curveto(505.61828471,684.99941401)(505.56328476,685.009414)(505.51328951,685.02942034)
\lineto(505.36328951,685.02942034)
\curveto(505.16328516,685.02941398)(505.00328532,685.009414)(504.88328951,684.96942034)
\curveto(504.59328573,684.87941413)(504.35828597,684.73941427)(504.17828951,684.54942034)
\curveto(503.99828633,684.36941464)(503.85328647,684.14941486)(503.74328951,683.88942034)
\curveto(503.69328663,683.77941523)(503.65328667,683.65941535)(503.62328951,683.52942034)
\curveto(503.60328672,683.4094156)(503.57828675,683.27941573)(503.54828951,683.13942034)
\curveto(503.53828679,683.09941591)(503.53328679,683.05941595)(503.53328951,683.01942034)
\curveto(503.53328679,682.97941603)(503.5282868,682.93941607)(503.51828951,682.89942034)
\curveto(503.49828683,682.79941621)(503.48828684,682.65941635)(503.48828951,682.47942034)
\curveto(503.49828683,682.29941671)(503.51328681,682.15941685)(503.53328951,682.05942034)
\curveto(503.53328679,681.97941703)(503.53828679,681.92441709)(503.54828951,681.89442034)
\curveto(503.56828676,681.82441719)(503.57828675,681.75441726)(503.57828951,681.68442034)
\curveto(503.58828674,681.6144174)(503.60328672,681.54441747)(503.62328951,681.47442034)
\curveto(503.70328662,681.24441777)(503.79828653,681.03441798)(503.90828951,680.84442034)
\curveto(504.01828631,680.65441836)(504.15828617,680.49441852)(504.32828951,680.36442034)
\curveto(504.36828596,680.33441868)(504.4282859,680.29941871)(504.50828951,680.25942034)
\curveto(504.61828571,680.18941882)(504.7282856,680.14441887)(504.83828951,680.12442034)
\curveto(504.95828537,680.10441891)(505.10328522,680.08441893)(505.27328951,680.06442034)
\lineto(505.36328951,680.06442034)
\curveto(505.40328492,680.06441895)(505.43328489,680.06941894)(505.45328951,680.07942034)
\lineto(505.58828951,680.07942034)
\curveto(505.65828467,680.09941891)(505.7232846,680.1144189)(505.78328951,680.12442034)
\curveto(505.85328447,680.14441887)(505.91828441,680.16441885)(505.97828951,680.18442034)
\curveto(506.27828405,680.3144187)(506.50828382,680.50441851)(506.66828951,680.75442034)
\curveto(506.70828362,680.80441821)(506.74328358,680.85941815)(506.77328951,680.91942034)
\curveto(506.80328352,680.98941802)(506.8282835,681.04941796)(506.84828951,681.09942034)
\curveto(506.88828344,681.2094178)(506.9232834,681.30441771)(506.95328951,681.38442034)
\curveto(506.98328334,681.47441754)(507.05328327,681.54441747)(507.16328951,681.59442034)
\curveto(507.25328307,681.63441738)(507.39828293,681.64941736)(507.59828951,681.63942034)
\lineto(508.09328951,681.63942034)
\lineto(508.30328951,681.63942034)
\curveto(508.38328194,681.64941736)(508.44828188,681.64441737)(508.49828951,681.62442034)
\lineto(508.61828951,681.62442034)
\lineto(508.73828951,681.59442034)
\curveto(508.77828155,681.59441742)(508.80828152,681.58441743)(508.82828951,681.56442034)
\curveto(508.87828145,681.52441749)(508.90828142,681.46441755)(508.91828951,681.38442034)
\curveto(508.93828139,681.3144177)(508.93828139,681.23941777)(508.91828951,681.15942034)
\curveto(508.8282815,680.82941818)(508.71828161,680.53441848)(508.58828951,680.27442034)
\curveto(508.17828215,679.50441951)(507.5232828,678.96942004)(506.62328951,678.66942034)
\curveto(506.5232838,678.63942037)(506.41828391,678.61942039)(506.30828951,678.60942034)
\curveto(506.19828413,678.58942042)(506.08828424,678.56442045)(505.97828951,678.53442034)
\curveto(505.91828441,678.52442049)(505.85828447,678.51942049)(505.79828951,678.51942034)
\curveto(505.73828459,678.51942049)(505.67828465,678.5144205)(505.61828951,678.50442034)
\lineto(505.45328951,678.50442034)
\curveto(505.40328492,678.48442053)(505.328285,678.47942053)(505.22828951,678.48942034)
\curveto(505.1282852,678.48942052)(505.05328527,678.49442052)(505.00328951,678.50442034)
\curveto(504.9232854,678.52442049)(504.84828548,678.53442048)(504.77828951,678.53442034)
\curveto(504.71828561,678.52442049)(504.65328567,678.52942048)(504.58328951,678.54942034)
\lineto(504.43328951,678.57942034)
\curveto(504.38328594,678.57942043)(504.33328599,678.58442043)(504.28328951,678.59442034)
\curveto(504.17328615,678.62442039)(504.06828626,678.65442036)(503.96828951,678.68442034)
\curveto(503.86828646,678.7144203)(503.77328655,678.74942026)(503.68328951,678.78942034)
\curveto(503.21328711,678.98942002)(502.81828751,679.24441977)(502.49828951,679.55442034)
\curveto(502.17828815,679.87441914)(501.91828841,680.26941874)(501.71828951,680.73942034)
\curveto(501.66828866,680.82941818)(501.6282887,680.92441809)(501.59828951,681.02442034)
\lineto(501.50828951,681.35442034)
\curveto(501.49828883,681.39441762)(501.49328883,681.42941758)(501.49328951,681.45942034)
\curveto(501.49328883,681.49941751)(501.48328884,681.54441747)(501.46328951,681.59442034)
\curveto(501.44328888,681.66441735)(501.43328889,681.73441728)(501.43328951,681.80442034)
\curveto(501.43328889,681.88441713)(501.4232889,681.95941705)(501.40328951,682.02942034)
\lineto(501.40328951,682.28442034)
\curveto(501.38328894,682.33441668)(501.37328895,682.38941662)(501.37328951,682.44942034)
\curveto(501.37328895,682.51941649)(501.38328894,682.57941643)(501.40328951,682.62942034)
\curveto(501.41328891,682.67941633)(501.41328891,682.72441629)(501.40328951,682.76442034)
\curveto(501.39328893,682.80441621)(501.39328893,682.84441617)(501.40328951,682.88442034)
\curveto(501.4232889,682.95441606)(501.4282889,683.01941599)(501.41828951,683.07942034)
\curveto(501.41828891,683.13941587)(501.4282889,683.19941581)(501.44828951,683.25942034)
\curveto(501.49828883,683.43941557)(501.53828879,683.6094154)(501.56828951,683.76942034)
\curveto(501.59828873,683.93941507)(501.64328868,684.10441491)(501.70328951,684.26442034)
\curveto(501.9232884,684.77441424)(502.19828813,685.19941381)(502.52828951,685.53942034)
\curveto(502.86828746,685.87941313)(503.29828703,686.15441286)(503.81828951,686.36442034)
\curveto(503.95828637,686.42441259)(504.10328622,686.46441255)(504.25328951,686.48442034)
\curveto(504.40328592,686.5144125)(504.55828577,686.54941246)(504.71828951,686.58942034)
\curveto(504.79828553,686.59941241)(504.87328545,686.60441241)(504.94328951,686.60442034)
\curveto(505.01328531,686.60441241)(505.08828524,686.6094124)(505.16828951,686.61942034)
}
}
{
\newrgbcolor{curcolor}{0 0 0}
\pscustom[linestyle=none,fillstyle=solid,fillcolor=curcolor]
{
\newpath
\moveto(510.63157076,686.39442034)
\lineto(511.75657076,686.39442034)
\curveto(511.86656832,686.39441262)(511.96656822,686.38941262)(512.05657076,686.37942034)
\curveto(512.14656804,686.36941264)(512.21156798,686.33441268)(512.25157076,686.27442034)
\curveto(512.30156789,686.2144128)(512.33156786,686.12941288)(512.34157076,686.01942034)
\curveto(512.35156784,685.91941309)(512.35656783,685.8144132)(512.35657076,685.70442034)
\lineto(512.35657076,684.65442034)
\lineto(512.35657076,682.41942034)
\curveto(512.35656783,682.05941695)(512.37156782,681.71941729)(512.40157076,681.39942034)
\curveto(512.43156776,681.07941793)(512.52156767,680.8144182)(512.67157076,680.60442034)
\curveto(512.81156738,680.39441862)(513.03656715,680.24441877)(513.34657076,680.15442034)
\curveto(513.39656679,680.14441887)(513.43656675,680.13941887)(513.46657076,680.13942034)
\curveto(513.50656668,680.13941887)(513.55156664,680.13441888)(513.60157076,680.12442034)
\curveto(513.65156654,680.1144189)(513.70656648,680.1094189)(513.76657076,680.10942034)
\curveto(513.82656636,680.1094189)(513.87156632,680.1144189)(513.90157076,680.12442034)
\curveto(513.95156624,680.14441887)(513.9915662,680.14941886)(514.02157076,680.13942034)
\curveto(514.06156613,680.12941888)(514.10156609,680.13441888)(514.14157076,680.15442034)
\curveto(514.35156584,680.20441881)(514.51656567,680.26941874)(514.63657076,680.34942034)
\curveto(514.81656537,680.45941855)(514.95656523,680.59941841)(515.05657076,680.76942034)
\curveto(515.16656502,680.94941806)(515.24156495,681.14441787)(515.28157076,681.35442034)
\curveto(515.33156486,681.57441744)(515.36156483,681.8144172)(515.37157076,682.07442034)
\curveto(515.38156481,682.34441667)(515.3865648,682.62441639)(515.38657076,682.91442034)
\lineto(515.38657076,684.72942034)
\lineto(515.38657076,685.70442034)
\lineto(515.38657076,685.97442034)
\curveto(515.3865648,686.07441294)(515.40656478,686.15441286)(515.44657076,686.21442034)
\curveto(515.49656469,686.30441271)(515.57156462,686.35441266)(515.67157076,686.36442034)
\curveto(515.77156442,686.38441263)(515.8915643,686.39441262)(516.03157076,686.39442034)
\lineto(516.82657076,686.39442034)
\lineto(517.11157076,686.39442034)
\curveto(517.20156299,686.39441262)(517.27656291,686.37441264)(517.33657076,686.33442034)
\curveto(517.41656277,686.28441273)(517.46156273,686.2094128)(517.47157076,686.10942034)
\curveto(517.48156271,686.009413)(517.4865627,685.89441312)(517.48657076,685.76442034)
\lineto(517.48657076,684.62442034)
\lineto(517.48657076,680.40942034)
\lineto(517.48657076,679.34442034)
\lineto(517.48657076,679.04442034)
\curveto(517.4865627,678.94442007)(517.46656272,678.86942014)(517.42657076,678.81942034)
\curveto(517.37656281,678.73942027)(517.30156289,678.69442032)(517.20157076,678.68442034)
\curveto(517.10156309,678.67442034)(516.99656319,678.66942034)(516.88657076,678.66942034)
\lineto(516.07657076,678.66942034)
\curveto(515.96656422,678.66942034)(515.86656432,678.67442034)(515.77657076,678.68442034)
\curveto(515.69656449,678.69442032)(515.63156456,678.73442028)(515.58157076,678.80442034)
\curveto(515.56156463,678.83442018)(515.54156465,678.87942013)(515.52157076,678.93942034)
\curveto(515.51156468,678.99942001)(515.49656469,679.05941995)(515.47657076,679.11942034)
\curveto(515.46656472,679.17941983)(515.45156474,679.23441978)(515.43157076,679.28442034)
\curveto(515.41156478,679.33441968)(515.38156481,679.36441965)(515.34157076,679.37442034)
\curveto(515.32156487,679.39441962)(515.29656489,679.39941961)(515.26657076,679.38942034)
\curveto(515.23656495,679.37941963)(515.21156498,679.36941964)(515.19157076,679.35942034)
\curveto(515.12156507,679.31941969)(515.06156513,679.27441974)(515.01157076,679.22442034)
\curveto(514.96156523,679.17441984)(514.90656528,679.12941988)(514.84657076,679.08942034)
\curveto(514.80656538,679.05941995)(514.76656542,679.02441999)(514.72657076,678.98442034)
\curveto(514.69656549,678.95442006)(514.65656553,678.92442009)(514.60657076,678.89442034)
\curveto(514.37656581,678.75442026)(514.10656608,678.64442037)(513.79657076,678.56442034)
\curveto(513.72656646,678.54442047)(513.65656653,678.53442048)(513.58657076,678.53442034)
\curveto(513.51656667,678.52442049)(513.44156675,678.5094205)(513.36157076,678.48942034)
\curveto(513.32156687,678.47942053)(513.27656691,678.47942053)(513.22657076,678.48942034)
\curveto(513.186567,678.48942052)(513.14656704,678.48442053)(513.10657076,678.47442034)
\curveto(513.07656711,678.46442055)(513.01156718,678.46442055)(512.91157076,678.47442034)
\curveto(512.82156737,678.47442054)(512.76156743,678.47942053)(512.73157076,678.48942034)
\curveto(512.68156751,678.48942052)(512.63156756,678.49442052)(512.58157076,678.50442034)
\lineto(512.43157076,678.50442034)
\curveto(512.31156788,678.53442048)(512.19656799,678.55942045)(512.08657076,678.57942034)
\curveto(511.97656821,678.59942041)(511.86656832,678.62942038)(511.75657076,678.66942034)
\curveto(511.70656848,678.68942032)(511.66156853,678.70442031)(511.62157076,678.71442034)
\curveto(511.5915686,678.73442028)(511.55156864,678.75442026)(511.50157076,678.77442034)
\curveto(511.15156904,678.96442005)(510.87156932,679.22941978)(510.66157076,679.56942034)
\curveto(510.53156966,679.77941923)(510.43656975,680.02941898)(510.37657076,680.31942034)
\curveto(510.31656987,680.61941839)(510.27656991,680.93441808)(510.25657076,681.26442034)
\curveto(510.24656994,681.60441741)(510.24156995,681.94941706)(510.24157076,682.29942034)
\curveto(510.25156994,682.65941635)(510.25656993,683.014416)(510.25657076,683.36442034)
\lineto(510.25657076,685.40442034)
\curveto(510.25656993,685.53441348)(510.25156994,685.68441333)(510.24157076,685.85442034)
\curveto(510.24156995,686.03441298)(510.26656992,686.16441285)(510.31657076,686.24442034)
\curveto(510.34656984,686.29441272)(510.40656978,686.33941267)(510.49657076,686.37942034)
\curveto(510.55656963,686.37941263)(510.60156959,686.38441263)(510.63157076,686.39442034)
}
}
{
\newrgbcolor{curcolor}{0 0 0}
\pscustom[linestyle=none,fillstyle=solid,fillcolor=curcolor]
{
\newpath
\moveto(523.54282076,686.60442034)
\curveto(523.65281544,686.60441241)(523.74781535,686.59441242)(523.82782076,686.57442034)
\curveto(523.91781518,686.55441246)(523.98781511,686.5094125)(524.03782076,686.43942034)
\curveto(524.097815,686.35941265)(524.12781497,686.21941279)(524.12782076,686.01942034)
\lineto(524.12782076,685.50942034)
\lineto(524.12782076,685.13442034)
\curveto(524.13781496,684.99441402)(524.12281497,684.88441413)(524.08282076,684.80442034)
\curveto(524.04281505,684.73441428)(523.98281511,684.68941432)(523.90282076,684.66942034)
\curveto(523.83281526,684.64941436)(523.74781535,684.63941437)(523.64782076,684.63942034)
\curveto(523.55781554,684.63941437)(523.45781564,684.64441437)(523.34782076,684.65442034)
\curveto(523.24781585,684.66441435)(523.15281594,684.65941435)(523.06282076,684.63942034)
\curveto(522.9928161,684.61941439)(522.92281617,684.60441441)(522.85282076,684.59442034)
\curveto(522.78281631,684.59441442)(522.71781638,684.58441443)(522.65782076,684.56442034)
\curveto(522.4978166,684.5144145)(522.33781676,684.43941457)(522.17782076,684.33942034)
\curveto(522.01781708,684.24941476)(521.8928172,684.14441487)(521.80282076,684.02442034)
\curveto(521.75281734,683.94441507)(521.6978174,683.85941515)(521.63782076,683.76942034)
\curveto(521.58781751,683.68941532)(521.53781756,683.60441541)(521.48782076,683.51442034)
\curveto(521.45781764,683.43441558)(521.42781767,683.34941566)(521.39782076,683.25942034)
\lineto(521.33782076,683.01942034)
\curveto(521.31781778,682.94941606)(521.30781779,682.87441614)(521.30782076,682.79442034)
\curveto(521.30781779,682.72441629)(521.2978178,682.65441636)(521.27782076,682.58442034)
\curveto(521.26781783,682.54441647)(521.26281783,682.50441651)(521.26282076,682.46442034)
\curveto(521.27281782,682.43441658)(521.27281782,682.40441661)(521.26282076,682.37442034)
\lineto(521.26282076,682.13442034)
\curveto(521.24281785,682.06441695)(521.23781786,681.98441703)(521.24782076,681.89442034)
\curveto(521.25781784,681.8144172)(521.26281783,681.73441728)(521.26282076,681.65442034)
\lineto(521.26282076,680.69442034)
\lineto(521.26282076,679.41942034)
\curveto(521.26281783,679.28941972)(521.25781784,679.16941984)(521.24782076,679.05942034)
\curveto(521.23781786,678.94942006)(521.20781789,678.85942015)(521.15782076,678.78942034)
\curveto(521.13781796,678.75942025)(521.10281799,678.73442028)(521.05282076,678.71442034)
\curveto(521.01281808,678.70442031)(520.96781813,678.69442032)(520.91782076,678.68442034)
\lineto(520.84282076,678.68442034)
\curveto(520.7928183,678.67442034)(520.73781836,678.66942034)(520.67782076,678.66942034)
\lineto(520.51282076,678.66942034)
\lineto(519.86782076,678.66942034)
\curveto(519.80781929,678.67942033)(519.74281935,678.68442033)(519.67282076,678.68442034)
\lineto(519.47782076,678.68442034)
\curveto(519.42781967,678.70442031)(519.37781972,678.71942029)(519.32782076,678.72942034)
\curveto(519.27781982,678.74942026)(519.24281985,678.78442023)(519.22282076,678.83442034)
\curveto(519.18281991,678.88442013)(519.15781994,678.95442006)(519.14782076,679.04442034)
\lineto(519.14782076,679.34442034)
\lineto(519.14782076,680.36442034)
\lineto(519.14782076,684.59442034)
\lineto(519.14782076,685.70442034)
\lineto(519.14782076,685.98942034)
\curveto(519.14781995,686.08941292)(519.16781993,686.16941284)(519.20782076,686.22942034)
\curveto(519.25781984,686.3094127)(519.33281976,686.35941265)(519.43282076,686.37942034)
\curveto(519.53281956,686.39941261)(519.65281944,686.4094126)(519.79282076,686.40942034)
\lineto(520.55782076,686.40942034)
\curveto(520.67781842,686.4094126)(520.78281831,686.39941261)(520.87282076,686.37942034)
\curveto(520.96281813,686.36941264)(521.03281806,686.32441269)(521.08282076,686.24442034)
\curveto(521.11281798,686.19441282)(521.12781797,686.12441289)(521.12782076,686.03442034)
\lineto(521.15782076,685.76442034)
\curveto(521.16781793,685.68441333)(521.18281791,685.6094134)(521.20282076,685.53942034)
\curveto(521.23281786,685.46941354)(521.28281781,685.43441358)(521.35282076,685.43442034)
\curveto(521.37281772,685.45441356)(521.3928177,685.46441355)(521.41282076,685.46442034)
\curveto(521.43281766,685.46441355)(521.45281764,685.47441354)(521.47282076,685.49442034)
\curveto(521.53281756,685.54441347)(521.58281751,685.59941341)(521.62282076,685.65942034)
\curveto(521.67281742,685.72941328)(521.73281736,685.78941322)(521.80282076,685.83942034)
\curveto(521.84281725,685.86941314)(521.87781722,685.89941311)(521.90782076,685.92942034)
\curveto(521.93781716,685.96941304)(521.97281712,686.00441301)(522.01282076,686.03442034)
\lineto(522.28282076,686.21442034)
\curveto(522.38281671,686.27441274)(522.48281661,686.32941268)(522.58282076,686.37942034)
\curveto(522.68281641,686.41941259)(522.78281631,686.45441256)(522.88282076,686.48442034)
\lineto(523.21282076,686.57442034)
\curveto(523.24281585,686.58441243)(523.2978158,686.58441243)(523.37782076,686.57442034)
\curveto(523.46781563,686.57441244)(523.52281557,686.58441243)(523.54282076,686.60442034)
}
}
{
\newrgbcolor{curcolor}{0 0 0}
\pscustom[linestyle=none,fillstyle=solid,fillcolor=curcolor]
{
\newpath
\moveto(527.91789888,686.61942034)
\curveto(528.66789438,686.63941237)(529.31789373,686.55441246)(529.86789888,686.36442034)
\curveto(530.42789262,686.18441283)(530.8528922,685.86941314)(531.14289888,685.41942034)
\curveto(531.21289184,685.3094137)(531.27289178,685.19441382)(531.32289888,685.07442034)
\curveto(531.38289167,684.96441405)(531.43289162,684.83941417)(531.47289888,684.69942034)
\curveto(531.49289156,684.63941437)(531.50289155,684.57441444)(531.50289888,684.50442034)
\curveto(531.50289155,684.43441458)(531.49289156,684.37441464)(531.47289888,684.32442034)
\curveto(531.43289162,684.26441475)(531.37789167,684.22441479)(531.30789888,684.20442034)
\curveto(531.25789179,684.18441483)(531.19789185,684.17441484)(531.12789888,684.17442034)
\lineto(530.91789888,684.17442034)
\lineto(530.25789888,684.17442034)
\curveto(530.18789286,684.17441484)(530.11789293,684.16941484)(530.04789888,684.15942034)
\curveto(529.97789307,684.15941485)(529.91289314,684.16941484)(529.85289888,684.18942034)
\curveto(529.7528933,684.2094148)(529.67789337,684.24941476)(529.62789888,684.30942034)
\curveto(529.57789347,684.36941464)(529.53289352,684.42941458)(529.49289888,684.48942034)
\lineto(529.37289888,684.69942034)
\curveto(529.34289371,684.77941423)(529.29289376,684.84441417)(529.22289888,684.89442034)
\curveto(529.12289393,684.97441404)(529.02289403,685.03441398)(528.92289888,685.07442034)
\curveto(528.83289422,685.1144139)(528.71789433,685.14941386)(528.57789888,685.17942034)
\curveto(528.50789454,685.19941381)(528.40289465,685.2144138)(528.26289888,685.22442034)
\curveto(528.13289492,685.23441378)(528.03289502,685.22941378)(527.96289888,685.20942034)
\lineto(527.85789888,685.20942034)
\lineto(527.70789888,685.17942034)
\curveto(527.66789538,685.17941383)(527.62289543,685.17441384)(527.57289888,685.16442034)
\curveto(527.40289565,685.1144139)(527.26289579,685.04441397)(527.15289888,684.95442034)
\curveto(527.052896,684.87441414)(526.98289607,684.74941426)(526.94289888,684.57942034)
\curveto(526.92289613,684.5094145)(526.92289613,684.44441457)(526.94289888,684.38442034)
\curveto(526.96289609,684.32441469)(526.98289607,684.27441474)(527.00289888,684.23442034)
\curveto(527.07289598,684.1144149)(527.1528959,684.01941499)(527.24289888,683.94942034)
\curveto(527.34289571,683.87941513)(527.45789559,683.81941519)(527.58789888,683.76942034)
\curveto(527.77789527,683.68941532)(527.98289507,683.61941539)(528.20289888,683.55942034)
\lineto(528.89289888,683.40942034)
\curveto(529.13289392,683.36941564)(529.36289369,683.31941569)(529.58289888,683.25942034)
\curveto(529.81289324,683.2094158)(530.02789302,683.14441587)(530.22789888,683.06442034)
\curveto(530.31789273,683.02441599)(530.40289265,682.98941602)(530.48289888,682.95942034)
\curveto(530.57289248,682.93941607)(530.65789239,682.90441611)(530.73789888,682.85442034)
\curveto(530.92789212,682.73441628)(531.09789195,682.60441641)(531.24789888,682.46442034)
\curveto(531.40789164,682.32441669)(531.53289152,682.14941686)(531.62289888,681.93942034)
\curveto(531.6528914,681.86941714)(531.67789137,681.79941721)(531.69789888,681.72942034)
\curveto(531.71789133,681.65941735)(531.73789131,681.58441743)(531.75789888,681.50442034)
\curveto(531.76789128,681.44441757)(531.77289128,681.34941766)(531.77289888,681.21942034)
\curveto(531.78289127,681.09941791)(531.78289127,681.00441801)(531.77289888,680.93442034)
\lineto(531.77289888,680.85942034)
\curveto(531.7528913,680.79941821)(531.73789131,680.73941827)(531.72789888,680.67942034)
\curveto(531.72789132,680.62941838)(531.72289133,680.57941843)(531.71289888,680.52942034)
\curveto(531.64289141,680.22941878)(531.53289152,679.96441905)(531.38289888,679.73442034)
\curveto(531.22289183,679.49441952)(531.02789202,679.29941971)(530.79789888,679.14942034)
\curveto(530.56789248,678.99942001)(530.30789274,678.86942014)(530.01789888,678.75942034)
\curveto(529.90789314,678.7094203)(529.78789326,678.67442034)(529.65789888,678.65442034)
\curveto(529.53789351,678.63442038)(529.41789363,678.6094204)(529.29789888,678.57942034)
\curveto(529.20789384,678.55942045)(529.11289394,678.54942046)(529.01289888,678.54942034)
\curveto(528.92289413,678.53942047)(528.83289422,678.52442049)(528.74289888,678.50442034)
\lineto(528.47289888,678.50442034)
\curveto(528.41289464,678.48442053)(528.30789474,678.47442054)(528.15789888,678.47442034)
\curveto(528.01789503,678.47442054)(527.91789513,678.48442053)(527.85789888,678.50442034)
\curveto(527.82789522,678.50442051)(527.79289526,678.5094205)(527.75289888,678.51942034)
\lineto(527.64789888,678.51942034)
\curveto(527.52789552,678.53942047)(527.40789564,678.55442046)(527.28789888,678.56442034)
\curveto(527.16789588,678.57442044)(527.052896,678.59442042)(526.94289888,678.62442034)
\curveto(526.5528965,678.73442028)(526.20789684,678.85942015)(525.90789888,678.99942034)
\curveto(525.60789744,679.14941986)(525.3528977,679.36941964)(525.14289888,679.65942034)
\curveto(525.00289805,679.84941916)(524.88289817,680.06941894)(524.78289888,680.31942034)
\curveto(524.76289829,680.37941863)(524.74289831,680.45941855)(524.72289888,680.55942034)
\curveto(524.70289835,680.6094184)(524.68789836,680.67941833)(524.67789888,680.76942034)
\curveto(524.66789838,680.85941815)(524.67289838,680.93441808)(524.69289888,680.99442034)
\curveto(524.72289833,681.06441795)(524.77289828,681.1144179)(524.84289888,681.14442034)
\curveto(524.89289816,681.16441785)(524.9528981,681.17441784)(525.02289888,681.17442034)
\lineto(525.24789888,681.17442034)
\lineto(525.95289888,681.17442034)
\lineto(526.19289888,681.17442034)
\curveto(526.27289678,681.17441784)(526.34289671,681.16441785)(526.40289888,681.14442034)
\curveto(526.51289654,681.10441791)(526.58289647,681.03941797)(526.61289888,680.94942034)
\curveto(526.6528964,680.85941815)(526.69789635,680.76441825)(526.74789888,680.66442034)
\curveto(526.76789628,680.6144184)(526.80289625,680.54941846)(526.85289888,680.46942034)
\curveto(526.91289614,680.38941862)(526.96289609,680.33941867)(527.00289888,680.31942034)
\curveto(527.12289593,680.21941879)(527.23789581,680.13941887)(527.34789888,680.07942034)
\curveto(527.45789559,680.02941898)(527.59789545,679.97941903)(527.76789888,679.92942034)
\curveto(527.81789523,679.9094191)(527.86789518,679.89941911)(527.91789888,679.89942034)
\curveto(527.96789508,679.9094191)(528.01789503,679.9094191)(528.06789888,679.89942034)
\curveto(528.1478949,679.87941913)(528.23289482,679.86941914)(528.32289888,679.86942034)
\curveto(528.42289463,679.87941913)(528.50789454,679.89441912)(528.57789888,679.91442034)
\curveto(528.62789442,679.92441909)(528.67289438,679.92941908)(528.71289888,679.92942034)
\curveto(528.76289429,679.92941908)(528.81289424,679.93941907)(528.86289888,679.95942034)
\curveto(529.00289405,680.009419)(529.12789392,680.06941894)(529.23789888,680.13942034)
\curveto(529.35789369,680.2094188)(529.4528936,680.29941871)(529.52289888,680.40942034)
\curveto(529.57289348,680.48941852)(529.61289344,680.6144184)(529.64289888,680.78442034)
\curveto(529.66289339,680.85441816)(529.66289339,680.91941809)(529.64289888,680.97942034)
\curveto(529.62289343,681.03941797)(529.60289345,681.08941792)(529.58289888,681.12942034)
\curveto(529.51289354,681.26941774)(529.42289363,681.37441764)(529.31289888,681.44442034)
\curveto(529.21289384,681.5144175)(529.09289396,681.57941743)(528.95289888,681.63942034)
\curveto(528.76289429,681.71941729)(528.56289449,681.78441723)(528.35289888,681.83442034)
\curveto(528.14289491,681.88441713)(527.93289512,681.93941707)(527.72289888,681.99942034)
\curveto(527.64289541,682.01941699)(527.55789549,682.03441698)(527.46789888,682.04442034)
\curveto(527.38789566,682.05441696)(527.30789574,682.06941694)(527.22789888,682.08942034)
\curveto(526.90789614,682.17941683)(526.60289645,682.26441675)(526.31289888,682.34442034)
\curveto(526.02289703,682.43441658)(525.75789729,682.56441645)(525.51789888,682.73442034)
\curveto(525.23789781,682.93441608)(525.03289802,683.20441581)(524.90289888,683.54442034)
\curveto(524.88289817,683.6144154)(524.86289819,683.7094153)(524.84289888,683.82942034)
\curveto(524.82289823,683.89941511)(524.80789824,683.98441503)(524.79789888,684.08442034)
\curveto(524.78789826,684.18441483)(524.79289826,684.27441474)(524.81289888,684.35442034)
\curveto(524.83289822,684.40441461)(524.83789821,684.44441457)(524.82789888,684.47442034)
\curveto(524.81789823,684.5144145)(524.82289823,684.55941445)(524.84289888,684.60942034)
\curveto(524.86289819,684.71941429)(524.88289817,684.81941419)(524.90289888,684.90942034)
\curveto(524.93289812,685.009414)(524.96789808,685.10441391)(525.00789888,685.19442034)
\curveto(525.13789791,685.48441353)(525.31789773,685.71941329)(525.54789888,685.89942034)
\curveto(525.77789727,686.07941293)(526.03789701,686.22441279)(526.32789888,686.33442034)
\curveto(526.43789661,686.38441263)(526.5528965,686.41941259)(526.67289888,686.43942034)
\curveto(526.79289626,686.46941254)(526.91789613,686.49941251)(527.04789888,686.52942034)
\curveto(527.10789594,686.54941246)(527.16789588,686.55941245)(527.22789888,686.55942034)
\lineto(527.40789888,686.58942034)
\curveto(527.48789556,686.59941241)(527.57289548,686.60441241)(527.66289888,686.60442034)
\curveto(527.7528953,686.60441241)(527.83789521,686.6094124)(527.91789888,686.61942034)
}
}
{
\newrgbcolor{curcolor}{0 0 0}
\pscustom[linestyle=none,fillstyle=solid,fillcolor=curcolor]
{
\newpath
\moveto(540.77453951,682.85442034)
\curveto(540.79453094,682.79441622)(540.80453093,682.7094163)(540.80453951,682.59942034)
\curveto(540.80453093,682.48941652)(540.79453094,682.40441661)(540.77453951,682.34442034)
\lineto(540.77453951,682.19442034)
\curveto(540.75453098,682.1144169)(540.74453099,682.03441698)(540.74453951,681.95442034)
\curveto(540.75453098,681.87441714)(540.74953098,681.79441722)(540.72953951,681.71442034)
\curveto(540.70953102,681.64441737)(540.69453104,681.57941743)(540.68453951,681.51942034)
\curveto(540.67453106,681.45941755)(540.66453107,681.39441762)(540.65453951,681.32442034)
\curveto(540.61453112,681.2144178)(540.57953115,681.09941791)(540.54953951,680.97942034)
\curveto(540.51953121,680.86941814)(540.47953125,680.76441825)(540.42953951,680.66442034)
\curveto(540.21953151,680.18441883)(539.94453179,679.79441922)(539.60453951,679.49442034)
\curveto(539.26453247,679.19441982)(538.85453288,678.94442007)(538.37453951,678.74442034)
\curveto(538.25453348,678.69442032)(538.1295336,678.65942035)(537.99953951,678.63942034)
\curveto(537.87953385,678.6094204)(537.75453398,678.57942043)(537.62453951,678.54942034)
\curveto(537.57453416,678.52942048)(537.51953421,678.51942049)(537.45953951,678.51942034)
\curveto(537.39953433,678.51942049)(537.34453439,678.5144205)(537.29453951,678.50442034)
\lineto(537.18953951,678.50442034)
\curveto(537.15953457,678.49442052)(537.1295346,678.48942052)(537.09953951,678.48942034)
\curveto(537.04953468,678.47942053)(536.96953476,678.47442054)(536.85953951,678.47442034)
\curveto(536.74953498,678.46442055)(536.66453507,678.46942054)(536.60453951,678.48942034)
\lineto(536.45453951,678.48942034)
\curveto(536.40453533,678.49942051)(536.34953538,678.50442051)(536.28953951,678.50442034)
\curveto(536.23953549,678.49442052)(536.18953554,678.49942051)(536.13953951,678.51942034)
\curveto(536.09953563,678.52942048)(536.05953567,678.53442048)(536.01953951,678.53442034)
\curveto(535.98953574,678.53442048)(535.94953578,678.53942047)(535.89953951,678.54942034)
\curveto(535.79953593,678.57942043)(535.69953603,678.60442041)(535.59953951,678.62442034)
\curveto(535.49953623,678.64442037)(535.40453633,678.67442034)(535.31453951,678.71442034)
\curveto(535.19453654,678.75442026)(535.07953665,678.79442022)(534.96953951,678.83442034)
\curveto(534.86953686,678.87442014)(534.76453697,678.92442009)(534.65453951,678.98442034)
\curveto(534.30453743,679.19441982)(534.00453773,679.43941957)(533.75453951,679.71942034)
\curveto(533.50453823,679.99941901)(533.29453844,680.33441868)(533.12453951,680.72442034)
\curveto(533.07453866,680.8144182)(533.0345387,680.9094181)(533.00453951,681.00942034)
\curveto(532.98453875,681.1094179)(532.95953877,681.2144178)(532.92953951,681.32442034)
\curveto(532.90953882,681.37441764)(532.89953883,681.41941759)(532.89953951,681.45942034)
\curveto(532.89953883,681.49941751)(532.88953884,681.54441747)(532.86953951,681.59442034)
\curveto(532.84953888,681.67441734)(532.83953889,681.75441726)(532.83953951,681.83442034)
\curveto(532.83953889,681.92441709)(532.8295389,682.009417)(532.80953951,682.08942034)
\curveto(532.79953893,682.13941687)(532.79453894,682.18441683)(532.79453951,682.22442034)
\lineto(532.79453951,682.35942034)
\curveto(532.77453896,682.41941659)(532.76453897,682.50441651)(532.76453951,682.61442034)
\curveto(532.77453896,682.72441629)(532.78953894,682.8094162)(532.80953951,682.86942034)
\lineto(532.80953951,682.97442034)
\curveto(532.81953891,683.02441599)(532.81953891,683.07441594)(532.80953951,683.12442034)
\curveto(532.80953892,683.18441583)(532.81953891,683.23941577)(532.83953951,683.28942034)
\curveto(532.84953888,683.33941567)(532.85453888,683.38441563)(532.85453951,683.42442034)
\curveto(532.85453888,683.47441554)(532.86453887,683.52441549)(532.88453951,683.57442034)
\curveto(532.92453881,683.70441531)(532.95953877,683.82941518)(532.98953951,683.94942034)
\curveto(533.01953871,684.07941493)(533.05953867,684.20441481)(533.10953951,684.32442034)
\curveto(533.28953844,684.73441428)(533.50453823,685.07441394)(533.75453951,685.34442034)
\curveto(534.00453773,685.62441339)(534.30953742,685.87941313)(534.66953951,686.10942034)
\curveto(534.76953696,686.15941285)(534.87453686,686.20441281)(534.98453951,686.24442034)
\curveto(535.09453664,686.28441273)(535.20453653,686.32941268)(535.31453951,686.37942034)
\curveto(535.44453629,686.42941258)(535.57953615,686.46441255)(535.71953951,686.48442034)
\curveto(535.85953587,686.50441251)(536.00453573,686.53441248)(536.15453951,686.57442034)
\curveto(536.2345355,686.58441243)(536.30953542,686.58941242)(536.37953951,686.58942034)
\curveto(536.44953528,686.58941242)(536.51953521,686.59441242)(536.58953951,686.60442034)
\curveto(537.16953456,686.6144124)(537.66953406,686.55441246)(538.08953951,686.42442034)
\curveto(538.51953321,686.29441272)(538.89953283,686.1144129)(539.22953951,685.88442034)
\curveto(539.33953239,685.80441321)(539.44953228,685.7144133)(539.55953951,685.61442034)
\curveto(539.67953205,685.52441349)(539.77953195,685.42441359)(539.85953951,685.31442034)
\curveto(539.93953179,685.2144138)(540.00953172,685.1144139)(540.06953951,685.01442034)
\curveto(540.13953159,684.9144141)(540.20953152,684.8094142)(540.27953951,684.69942034)
\curveto(540.34953138,684.58941442)(540.40453133,684.46941454)(540.44453951,684.33942034)
\curveto(540.48453125,684.21941479)(540.5295312,684.08941492)(540.57953951,683.94942034)
\curveto(540.60953112,683.86941514)(540.6345311,683.78441523)(540.65453951,683.69442034)
\lineto(540.71453951,683.42442034)
\curveto(540.72453101,683.38441563)(540.729531,683.34441567)(540.72953951,683.30442034)
\curveto(540.729531,683.26441575)(540.734531,683.22441579)(540.74453951,683.18442034)
\curveto(540.76453097,683.13441588)(540.76953096,683.07941593)(540.75953951,683.01942034)
\curveto(540.74953098,682.95941605)(540.75453098,682.90441611)(540.77453951,682.85442034)
\moveto(538.67453951,682.31442034)
\curveto(538.68453305,682.36441665)(538.68953304,682.43441658)(538.68953951,682.52442034)
\curveto(538.68953304,682.62441639)(538.68453305,682.69941631)(538.67453951,682.74942034)
\lineto(538.67453951,682.86942034)
\curveto(538.65453308,682.91941609)(538.64453309,682.97441604)(538.64453951,683.03442034)
\curveto(538.64453309,683.09441592)(538.63953309,683.14941586)(538.62953951,683.19942034)
\curveto(538.6295331,683.23941577)(538.62453311,683.26941574)(538.61453951,683.28942034)
\lineto(538.55453951,683.52942034)
\curveto(538.54453319,683.61941539)(538.52453321,683.70441531)(538.49453951,683.78442034)
\curveto(538.38453335,684.04441497)(538.25453348,684.26441475)(538.10453951,684.44442034)
\curveto(537.95453378,684.63441438)(537.75453398,684.78441423)(537.50453951,684.89442034)
\curveto(537.44453429,684.9144141)(537.38453435,684.92941408)(537.32453951,684.93942034)
\curveto(537.26453447,684.95941405)(537.19953453,684.97941403)(537.12953951,684.99942034)
\curveto(537.04953468,685.01941399)(536.96453477,685.02441399)(536.87453951,685.01442034)
\lineto(536.60453951,685.01442034)
\curveto(536.57453516,684.99441402)(536.53953519,684.98441403)(536.49953951,684.98442034)
\curveto(536.45953527,684.99441402)(536.42453531,684.99441402)(536.39453951,684.98442034)
\lineto(536.18453951,684.92442034)
\curveto(536.12453561,684.9144141)(536.06953566,684.89441412)(536.01953951,684.86442034)
\curveto(535.76953596,684.75441426)(535.56453617,684.59441442)(535.40453951,684.38442034)
\curveto(535.25453648,684.18441483)(535.1345366,683.94941506)(535.04453951,683.67942034)
\curveto(535.01453672,683.57941543)(534.98953674,683.47441554)(534.96953951,683.36442034)
\curveto(534.95953677,683.25441576)(534.94453679,683.14441587)(534.92453951,683.03442034)
\curveto(534.91453682,682.98441603)(534.90953682,682.93441608)(534.90953951,682.88442034)
\lineto(534.90953951,682.73442034)
\curveto(534.88953684,682.66441635)(534.87953685,682.55941645)(534.87953951,682.41942034)
\curveto(534.88953684,682.27941673)(534.90453683,682.17441684)(534.92453951,682.10442034)
\lineto(534.92453951,681.96942034)
\curveto(534.94453679,681.88941712)(534.95953677,681.8094172)(534.96953951,681.72942034)
\curveto(534.97953675,681.65941735)(534.99453674,681.58441743)(535.01453951,681.50442034)
\curveto(535.11453662,681.20441781)(535.21953651,680.95941805)(535.32953951,680.76942034)
\curveto(535.44953628,680.58941842)(535.6345361,680.42441859)(535.88453951,680.27442034)
\curveto(535.95453578,680.22441879)(536.0295357,680.18441883)(536.10953951,680.15442034)
\curveto(536.19953553,680.12441889)(536.28953544,680.09941891)(536.37953951,680.07942034)
\curveto(536.41953531,680.06941894)(536.45453528,680.06441895)(536.48453951,680.06442034)
\curveto(536.51453522,680.07441894)(536.54953518,680.07441894)(536.58953951,680.06442034)
\lineto(536.70953951,680.03442034)
\curveto(536.75953497,680.03441898)(536.80453493,680.03941897)(536.84453951,680.04942034)
\lineto(536.96453951,680.04942034)
\curveto(537.04453469,680.06941894)(537.12453461,680.08441893)(537.20453951,680.09442034)
\curveto(537.28453445,680.10441891)(537.35953437,680.12441889)(537.42953951,680.15442034)
\curveto(537.68953404,680.25441876)(537.89953383,680.38941862)(538.05953951,680.55942034)
\curveto(538.21953351,680.72941828)(538.35453338,680.93941807)(538.46453951,681.18942034)
\curveto(538.50453323,681.28941772)(538.5345332,681.38941762)(538.55453951,681.48942034)
\curveto(538.57453316,681.58941742)(538.59953313,681.69441732)(538.62953951,681.80442034)
\curveto(538.63953309,681.84441717)(538.64453309,681.87941713)(538.64453951,681.90942034)
\curveto(538.64453309,681.94941706)(538.64953308,681.98941702)(538.65953951,682.02942034)
\lineto(538.65953951,682.16442034)
\curveto(538.65953307,682.2144168)(538.66453307,682.26441675)(538.67453951,682.31442034)
}
}
{
\newrgbcolor{curcolor}{0 0 0}
\pscustom[linestyle=none,fillstyle=solid,fillcolor=curcolor]
{
\newpath
\moveto(28.32111654,343.25226336)
\curveto(28.32110606,343.22225769)(28.32110606,343.18225773)(28.32111654,343.13226336)
\curveto(28.33110605,343.08225783)(28.33610604,343.02725789)(28.33611654,342.96726336)
\curveto(28.33610604,342.90725801)(28.33110605,342.85225806)(28.32111654,342.80226336)
\curveto(28.32110606,342.75225816)(28.32110606,342.7172582)(28.32111654,342.69726336)
\curveto(28.32110606,342.62725829)(28.31610606,342.55725836)(28.30611654,342.48726336)
\curveto(28.30610607,342.42725849)(28.30610607,342.36725855)(28.30611654,342.30726336)
\curveto(28.28610609,342.25725866)(28.2761061,342.20725871)(28.27611654,342.15726336)
\curveto(28.28610609,342.10725881)(28.28610609,342.05725886)(28.27611654,342.00726336)
\curveto(28.25610612,341.89725902)(28.24110614,341.78725913)(28.23111654,341.67726336)
\curveto(28.22110616,341.56725935)(28.20110618,341.45725946)(28.17111654,341.34726336)
\curveto(28.12110626,341.17725974)(28.0761063,341.0122599)(28.03611654,340.85226336)
\curveto(27.99610638,340.70226021)(27.94610643,340.55226036)(27.88611654,340.40226336)
\curveto(27.71610666,339.98226093)(27.50610687,339.60226131)(27.25611654,339.26226336)
\curveto(27.00610737,338.92226199)(26.70610767,338.63226228)(26.35611654,338.39226336)
\curveto(26.15610822,338.25226266)(25.94610843,338.13226278)(25.72611654,338.03226336)
\curveto(25.51610886,337.93226298)(25.28610909,337.84226307)(25.03611654,337.76226336)
\curveto(24.93610944,337.73226318)(24.83110955,337.70726321)(24.72111654,337.68726336)
\curveto(24.62110976,337.67726324)(24.51610986,337.65726326)(24.40611654,337.62726336)
\curveto(24.35611002,337.6172633)(24.30611007,337.6122633)(24.25611654,337.61226336)
\curveto(24.21611016,337.6122633)(24.17111021,337.60726331)(24.12111654,337.59726336)
\curveto(24.0811103,337.58726333)(24.04111034,337.58226333)(24.00111654,337.58226336)
\curveto(23.96111042,337.59226332)(23.91611046,337.59226332)(23.86611654,337.58226336)
\curveto(23.84611053,337.57226334)(23.81611056,337.56726335)(23.77611654,337.56726336)
\curveto(23.73611064,337.57726334)(23.70611067,337.57726334)(23.68611654,337.56726336)
\curveto(23.60611077,337.54726337)(23.50611087,337.54226337)(23.38611654,337.55226336)
\curveto(23.26611111,337.56226335)(23.16111122,337.56726335)(23.07111654,337.56726336)
\lineto(19.57611654,337.56726336)
\curveto(19.40611497,337.56726335)(19.26111512,337.57226334)(19.14111654,337.58226336)
\curveto(19.03111535,337.60226331)(18.95111543,337.67226324)(18.90111654,337.79226336)
\curveto(18.87111551,337.87226304)(18.85611552,337.99226292)(18.85611654,338.15226336)
\curveto(18.86611551,338.32226259)(18.87111551,338.46226245)(18.87111654,338.57226336)
\lineto(18.87111654,347.37726336)
\curveto(18.87111551,347.49725342)(18.86611551,347.62225329)(18.85611654,347.75226336)
\curveto(18.85611552,347.89225302)(18.8811155,348.00225291)(18.93111654,348.08226336)
\curveto(18.97111541,348.14225277)(19.04611533,348.19225272)(19.15611654,348.23226336)
\curveto(19.1761152,348.24225267)(19.19611518,348.24225267)(19.21611654,348.23226336)
\curveto(19.23611514,348.23225268)(19.25611512,348.23725268)(19.27611654,348.24726336)
\lineto(23.31111654,348.24726336)
\curveto(23.37111101,348.24725267)(23.43111095,348.24725267)(23.49111654,348.24726336)
\curveto(23.56111082,348.25725266)(23.62111076,348.25725266)(23.67111654,348.24726336)
\lineto(23.85111654,348.24726336)
\curveto(23.90111048,348.22725269)(23.95611042,348.2172527)(24.01611654,348.21726336)
\curveto(24.0761103,348.22725269)(24.13111025,348.22225269)(24.18111654,348.20226336)
\curveto(24.24111014,348.18225273)(24.29611008,348.17225274)(24.34611654,348.17226336)
\curveto(24.40610997,348.18225273)(24.46610991,348.17725274)(24.52611654,348.15726336)
\curveto(24.66610971,348.12725279)(24.80110958,348.09725282)(24.93111654,348.06726336)
\curveto(25.06110932,348.04725287)(25.18610919,348.0122529)(25.30611654,347.96226336)
\curveto(25.41610896,347.912253)(25.52610885,347.86725305)(25.63611654,347.82726336)
\curveto(25.74610863,347.78725313)(25.85110853,347.73725318)(25.95111654,347.67726336)
\curveto(26.20110818,347.5172534)(26.43110795,347.36225355)(26.64111654,347.21226336)
\lineto(26.73111654,347.12226336)
\curveto(26.83110755,347.04225387)(26.92110746,346.95225396)(27.00111654,346.85226336)
\lineto(27.13611654,346.73226336)
\curveto(27.18610719,346.65225426)(27.24110714,346.57225434)(27.30111654,346.49226336)
\curveto(27.37110701,346.42225449)(27.43110695,346.34725457)(27.48111654,346.26726336)
\curveto(27.61110677,346.05725486)(27.72610665,345.83225508)(27.82611654,345.59226336)
\curveto(27.92610645,345.36225555)(28.01610636,345.1172558)(28.09611654,344.85726336)
\curveto(28.14610623,344.72725619)(28.1761062,344.59225632)(28.18611654,344.45226336)
\curveto(28.20610617,344.3122566)(28.23110615,344.17225674)(28.26111654,344.03226336)
\curveto(28.26110612,343.98225693)(28.26110612,343.93725698)(28.26111654,343.89726336)
\curveto(28.27110611,343.86725705)(28.2761061,343.83225708)(28.27611654,343.79226336)
\curveto(28.29610608,343.73225718)(28.30110608,343.66725725)(28.29111654,343.59726336)
\curveto(28.29110609,343.52725739)(28.30110608,343.46725745)(28.32111654,343.41726336)
\lineto(28.32111654,343.25226336)
\moveto(25.98111654,342.53226336)
\curveto(26.00110838,342.58225833)(26.01110837,342.66225825)(26.01111654,342.77226336)
\curveto(26.01110837,342.88225803)(26.00110838,342.96225795)(25.98111654,343.01226336)
\lineto(25.98111654,343.29726336)
\curveto(25.96110842,343.38725753)(25.94610843,343.48225743)(25.93611654,343.58226336)
\curveto(25.93610844,343.68225723)(25.92610845,343.77225714)(25.90611654,343.85226336)
\curveto(25.88610849,343.90225701)(25.8761085,343.94725697)(25.87611654,343.98726336)
\curveto(25.88610849,344.03725688)(25.8811085,344.08725683)(25.86111654,344.13726336)
\curveto(25.81110857,344.29725662)(25.76110862,344.44725647)(25.71111654,344.58726336)
\curveto(25.67110871,344.73725618)(25.61110877,344.87725604)(25.53111654,345.00726336)
\curveto(25.381109,345.24725567)(25.20610917,345.45225546)(25.00611654,345.62226336)
\curveto(24.81610956,345.80225511)(24.5811098,345.95225496)(24.30111654,346.07226336)
\curveto(24.21111017,346.10225481)(24.12111026,346.12725479)(24.03111654,346.14726336)
\curveto(23.94111044,346.17725474)(23.85111053,346.20225471)(23.76111654,346.22226336)
\curveto(23.6811107,346.23225468)(23.60611077,346.23725468)(23.53611654,346.23726336)
\curveto(23.4761109,346.24725467)(23.40611097,346.26225465)(23.32611654,346.28226336)
\curveto(23.28611109,346.29225462)(23.24611113,346.29225462)(23.20611654,346.28226336)
\curveto(23.16611121,346.28225463)(23.13111125,346.28725463)(23.10111654,346.29726336)
\lineto(22.77111654,346.29726336)
\curveto(22.72111166,346.30725461)(22.66611171,346.30725461)(22.60611654,346.29726336)
\lineto(22.42611654,346.29726336)
\lineto(21.75111654,346.29726336)
\curveto(21.73111265,346.27725464)(21.69611268,346.27225464)(21.64611654,346.28226336)
\curveto(21.60611277,346.29225462)(21.57111281,346.29225462)(21.54111654,346.28226336)
\lineto(21.39111654,346.22226336)
\curveto(21.34111304,346.2122547)(21.30111308,346.18225473)(21.27111654,346.13226336)
\curveto(21.23111315,346.08225483)(21.21111317,346.0122549)(21.21111654,345.92226336)
\lineto(21.21111654,345.62226336)
\curveto(21.21111317,345.49225542)(21.20611317,345.35725556)(21.19611654,345.21726336)
\lineto(21.19611654,344.79726336)
\lineto(21.19611654,340.61226336)
\curveto(21.19611318,340.55226036)(21.19111319,340.48726043)(21.18111654,340.41726336)
\curveto(21.1811132,340.34726057)(21.19111319,340.28726063)(21.21111654,340.23726336)
\lineto(21.21111654,340.08726336)
\lineto(21.21111654,339.87726336)
\curveto(21.22111316,339.8172611)(21.23611314,339.76226115)(21.25611654,339.71226336)
\curveto(21.31611306,339.59226132)(21.43111295,339.52726139)(21.60111654,339.51726336)
\lineto(22.12611654,339.51726336)
\lineto(23.31111654,339.51726336)
\curveto(23.71111067,339.52726139)(24.05111033,339.58726133)(24.33111654,339.69726336)
\curveto(24.70110968,339.84726107)(24.99110939,340.04726087)(25.20111654,340.29726336)
\curveto(25.42110896,340.54726037)(25.60610877,340.85726006)(25.75611654,341.22726336)
\curveto(25.79610858,341.30725961)(25.82610855,341.39725952)(25.84611654,341.49726336)
\curveto(25.86610851,341.59725932)(25.89110849,341.69725922)(25.92111654,341.79726336)
\lineto(25.92111654,341.91726336)
\curveto(25.94110844,341.98725893)(25.95110843,342.06225885)(25.95111654,342.14226336)
\curveto(25.95110843,342.22225869)(25.96110842,342.30225861)(25.98111654,342.38226336)
\lineto(25.98111654,342.53226336)
}
}
{
\newrgbcolor{curcolor}{0 0 0}
\pscustom[linestyle=none,fillstyle=solid,fillcolor=curcolor]
{
\newpath
\moveto(31.81963217,348.14226336)
\curveto(31.88962922,348.06225285)(31.92462918,347.94225297)(31.92463217,347.78226336)
\lineto(31.92463217,347.31726336)
\lineto(31.92463217,346.91226336)
\curveto(31.92462918,346.77225414)(31.88962922,346.67725424)(31.81963217,346.62726336)
\curveto(31.75962935,346.57725434)(31.67962943,346.54725437)(31.57963217,346.53726336)
\curveto(31.48962962,346.52725439)(31.38962972,346.52225439)(31.27963217,346.52226336)
\lineto(30.43963217,346.52226336)
\curveto(30.32963078,346.52225439)(30.22963088,346.52725439)(30.13963217,346.53726336)
\curveto(30.05963105,346.54725437)(29.98963112,346.57725434)(29.92963217,346.62726336)
\curveto(29.88963122,346.65725426)(29.85963125,346.7122542)(29.83963217,346.79226336)
\curveto(29.82963128,346.88225403)(29.81963129,346.97725394)(29.80963217,347.07726336)
\lineto(29.80963217,347.40726336)
\curveto(29.81963129,347.5172534)(29.82463128,347.6122533)(29.82463217,347.69226336)
\lineto(29.82463217,347.90226336)
\curveto(29.83463127,347.97225294)(29.85463125,348.03225288)(29.88463217,348.08226336)
\curveto(29.9046312,348.12225279)(29.92963118,348.15225276)(29.95963217,348.17226336)
\lineto(30.07963217,348.23226336)
\curveto(30.09963101,348.23225268)(30.12463098,348.23225268)(30.15463217,348.23226336)
\curveto(30.18463092,348.24225267)(30.2096309,348.24725267)(30.22963217,348.24726336)
\lineto(31.32463217,348.24726336)
\curveto(31.42462968,348.24725267)(31.51962959,348.24225267)(31.60963217,348.23226336)
\curveto(31.69962941,348.22225269)(31.76962934,348.19225272)(31.81963217,348.14226336)
\moveto(31.92463217,338.37726336)
\curveto(31.92462918,338.17726274)(31.91962919,338.00726291)(31.90963217,337.86726336)
\curveto(31.89962921,337.72726319)(31.8096293,337.63226328)(31.63963217,337.58226336)
\curveto(31.57962953,337.56226335)(31.51462959,337.55226336)(31.44463217,337.55226336)
\curveto(31.37462973,337.56226335)(31.29962981,337.56726335)(31.21963217,337.56726336)
\lineto(30.37963217,337.56726336)
\curveto(30.28963082,337.56726335)(30.19963091,337.57226334)(30.10963217,337.58226336)
\curveto(30.02963108,337.59226332)(29.96963114,337.62226329)(29.92963217,337.67226336)
\curveto(29.86963124,337.74226317)(29.83463127,337.82726309)(29.82463217,337.92726336)
\lineto(29.82463217,338.27226336)
\lineto(29.82463217,344.60226336)
\lineto(29.82463217,344.90226336)
\curveto(29.82463128,345.00225591)(29.84463126,345.08225583)(29.88463217,345.14226336)
\curveto(29.94463116,345.2122557)(30.02963108,345.25725566)(30.13963217,345.27726336)
\curveto(30.15963095,345.28725563)(30.18463092,345.28725563)(30.21463217,345.27726336)
\curveto(30.25463085,345.27725564)(30.28463082,345.28225563)(30.30463217,345.29226336)
\lineto(31.05463217,345.29226336)
\lineto(31.24963217,345.29226336)
\curveto(31.32962978,345.30225561)(31.39462971,345.30225561)(31.44463217,345.29226336)
\lineto(31.56463217,345.29226336)
\curveto(31.62462948,345.27225564)(31.67962943,345.25725566)(31.72963217,345.24726336)
\curveto(31.77962933,345.23725568)(31.81962929,345.20725571)(31.84963217,345.15726336)
\curveto(31.88962922,345.10725581)(31.9096292,345.03725588)(31.90963217,344.94726336)
\curveto(31.91962919,344.85725606)(31.92462918,344.76225615)(31.92463217,344.66226336)
\lineto(31.92463217,338.37726336)
}
}
{
\newrgbcolor{curcolor}{0 0 0}
\pscustom[linestyle=none,fillstyle=solid,fillcolor=curcolor]
{
\newpath
\moveto(36.55681967,345.50226336)
\curveto(37.30681517,345.52225539)(37.95681452,345.43725548)(38.50681967,345.24726336)
\curveto(39.06681341,345.06725585)(39.49181298,344.75225616)(39.78181967,344.30226336)
\curveto(39.85181262,344.19225672)(39.91181256,344.07725684)(39.96181967,343.95726336)
\curveto(40.02181245,343.84725707)(40.0718124,343.72225719)(40.11181967,343.58226336)
\curveto(40.13181234,343.52225739)(40.14181233,343.45725746)(40.14181967,343.38726336)
\curveto(40.14181233,343.3172576)(40.13181234,343.25725766)(40.11181967,343.20726336)
\curveto(40.0718124,343.14725777)(40.01681246,343.10725781)(39.94681967,343.08726336)
\curveto(39.89681258,343.06725785)(39.83681264,343.05725786)(39.76681967,343.05726336)
\lineto(39.55681967,343.05726336)
\lineto(38.89681967,343.05726336)
\curveto(38.82681365,343.05725786)(38.75681372,343.05225786)(38.68681967,343.04226336)
\curveto(38.61681386,343.04225787)(38.55181392,343.05225786)(38.49181967,343.07226336)
\curveto(38.39181408,343.09225782)(38.31681416,343.13225778)(38.26681967,343.19226336)
\curveto(38.21681426,343.25225766)(38.1718143,343.3122576)(38.13181967,343.37226336)
\lineto(38.01181967,343.58226336)
\curveto(37.98181449,343.66225725)(37.93181454,343.72725719)(37.86181967,343.77726336)
\curveto(37.76181471,343.85725706)(37.66181481,343.917257)(37.56181967,343.95726336)
\curveto(37.471815,343.99725692)(37.35681512,344.03225688)(37.21681967,344.06226336)
\curveto(37.14681533,344.08225683)(37.04181543,344.09725682)(36.90181967,344.10726336)
\curveto(36.7718157,344.1172568)(36.6718158,344.1122568)(36.60181967,344.09226336)
\lineto(36.49681967,344.09226336)
\lineto(36.34681967,344.06226336)
\curveto(36.30681617,344.06225685)(36.26181621,344.05725686)(36.21181967,344.04726336)
\curveto(36.04181643,343.99725692)(35.90181657,343.92725699)(35.79181967,343.83726336)
\curveto(35.69181678,343.75725716)(35.62181685,343.63225728)(35.58181967,343.46226336)
\curveto(35.56181691,343.39225752)(35.56181691,343.32725759)(35.58181967,343.26726336)
\curveto(35.60181687,343.20725771)(35.62181685,343.15725776)(35.64181967,343.11726336)
\curveto(35.71181676,342.99725792)(35.79181668,342.90225801)(35.88181967,342.83226336)
\curveto(35.98181649,342.76225815)(36.09681638,342.70225821)(36.22681967,342.65226336)
\curveto(36.41681606,342.57225834)(36.62181585,342.50225841)(36.84181967,342.44226336)
\lineto(37.53181967,342.29226336)
\curveto(37.7718147,342.25225866)(38.00181447,342.20225871)(38.22181967,342.14226336)
\curveto(38.45181402,342.09225882)(38.66681381,342.02725889)(38.86681967,341.94726336)
\curveto(38.95681352,341.90725901)(39.04181343,341.87225904)(39.12181967,341.84226336)
\curveto(39.21181326,341.82225909)(39.29681318,341.78725913)(39.37681967,341.73726336)
\curveto(39.56681291,341.6172593)(39.73681274,341.48725943)(39.88681967,341.34726336)
\curveto(40.04681243,341.20725971)(40.1718123,341.03225988)(40.26181967,340.82226336)
\curveto(40.29181218,340.75226016)(40.31681216,340.68226023)(40.33681967,340.61226336)
\curveto(40.35681212,340.54226037)(40.3768121,340.46726045)(40.39681967,340.38726336)
\curveto(40.40681207,340.32726059)(40.41181206,340.23226068)(40.41181967,340.10226336)
\curveto(40.42181205,339.98226093)(40.42181205,339.88726103)(40.41181967,339.81726336)
\lineto(40.41181967,339.74226336)
\curveto(40.39181208,339.68226123)(40.3768121,339.62226129)(40.36681967,339.56226336)
\curveto(40.36681211,339.5122614)(40.36181211,339.46226145)(40.35181967,339.41226336)
\curveto(40.28181219,339.1122618)(40.1718123,338.84726207)(40.02181967,338.61726336)
\curveto(39.86181261,338.37726254)(39.66681281,338.18226273)(39.43681967,338.03226336)
\curveto(39.20681327,337.88226303)(38.94681353,337.75226316)(38.65681967,337.64226336)
\curveto(38.54681393,337.59226332)(38.42681405,337.55726336)(38.29681967,337.53726336)
\curveto(38.1768143,337.5172634)(38.05681442,337.49226342)(37.93681967,337.46226336)
\curveto(37.84681463,337.44226347)(37.75181472,337.43226348)(37.65181967,337.43226336)
\curveto(37.56181491,337.42226349)(37.471815,337.40726351)(37.38181967,337.38726336)
\lineto(37.11181967,337.38726336)
\curveto(37.05181542,337.36726355)(36.94681553,337.35726356)(36.79681967,337.35726336)
\curveto(36.65681582,337.35726356)(36.55681592,337.36726355)(36.49681967,337.38726336)
\curveto(36.46681601,337.38726353)(36.43181604,337.39226352)(36.39181967,337.40226336)
\lineto(36.28681967,337.40226336)
\curveto(36.16681631,337.42226349)(36.04681643,337.43726348)(35.92681967,337.44726336)
\curveto(35.80681667,337.45726346)(35.69181678,337.47726344)(35.58181967,337.50726336)
\curveto(35.19181728,337.6172633)(34.84681763,337.74226317)(34.54681967,337.88226336)
\curveto(34.24681823,338.03226288)(33.99181848,338.25226266)(33.78181967,338.54226336)
\curveto(33.64181883,338.73226218)(33.52181895,338.95226196)(33.42181967,339.20226336)
\curveto(33.40181907,339.26226165)(33.38181909,339.34226157)(33.36181967,339.44226336)
\curveto(33.34181913,339.49226142)(33.32681915,339.56226135)(33.31681967,339.65226336)
\curveto(33.30681917,339.74226117)(33.31181916,339.8172611)(33.33181967,339.87726336)
\curveto(33.36181911,339.94726097)(33.41181906,339.99726092)(33.48181967,340.02726336)
\curveto(33.53181894,340.04726087)(33.59181888,340.05726086)(33.66181967,340.05726336)
\lineto(33.88681967,340.05726336)
\lineto(34.59181967,340.05726336)
\lineto(34.83181967,340.05726336)
\curveto(34.91181756,340.05726086)(34.98181749,340.04726087)(35.04181967,340.02726336)
\curveto(35.15181732,339.98726093)(35.22181725,339.92226099)(35.25181967,339.83226336)
\curveto(35.29181718,339.74226117)(35.33681714,339.64726127)(35.38681967,339.54726336)
\curveto(35.40681707,339.49726142)(35.44181703,339.43226148)(35.49181967,339.35226336)
\curveto(35.55181692,339.27226164)(35.60181687,339.22226169)(35.64181967,339.20226336)
\curveto(35.76181671,339.10226181)(35.8768166,339.02226189)(35.98681967,338.96226336)
\curveto(36.09681638,338.912262)(36.23681624,338.86226205)(36.40681967,338.81226336)
\curveto(36.45681602,338.79226212)(36.50681597,338.78226213)(36.55681967,338.78226336)
\curveto(36.60681587,338.79226212)(36.65681582,338.79226212)(36.70681967,338.78226336)
\curveto(36.78681569,338.76226215)(36.8718156,338.75226216)(36.96181967,338.75226336)
\curveto(37.06181541,338.76226215)(37.14681533,338.77726214)(37.21681967,338.79726336)
\curveto(37.26681521,338.80726211)(37.31181516,338.8122621)(37.35181967,338.81226336)
\curveto(37.40181507,338.8122621)(37.45181502,338.82226209)(37.50181967,338.84226336)
\curveto(37.64181483,338.89226202)(37.76681471,338.95226196)(37.87681967,339.02226336)
\curveto(37.99681448,339.09226182)(38.09181438,339.18226173)(38.16181967,339.29226336)
\curveto(38.21181426,339.37226154)(38.25181422,339.49726142)(38.28181967,339.66726336)
\curveto(38.30181417,339.73726118)(38.30181417,339.80226111)(38.28181967,339.86226336)
\curveto(38.26181421,339.92226099)(38.24181423,339.97226094)(38.22181967,340.01226336)
\curveto(38.15181432,340.15226076)(38.06181441,340.25726066)(37.95181967,340.32726336)
\curveto(37.85181462,340.39726052)(37.73181474,340.46226045)(37.59181967,340.52226336)
\curveto(37.40181507,340.60226031)(37.20181527,340.66726025)(36.99181967,340.71726336)
\curveto(36.78181569,340.76726015)(36.5718159,340.82226009)(36.36181967,340.88226336)
\curveto(36.28181619,340.90226001)(36.19681628,340.91726)(36.10681967,340.92726336)
\curveto(36.02681645,340.93725998)(35.94681653,340.95225996)(35.86681967,340.97226336)
\curveto(35.54681693,341.06225985)(35.24181723,341.14725977)(34.95181967,341.22726336)
\curveto(34.66181781,341.3172596)(34.39681808,341.44725947)(34.15681967,341.61726336)
\curveto(33.8768186,341.8172591)(33.6718188,342.08725883)(33.54181967,342.42726336)
\curveto(33.52181895,342.49725842)(33.50181897,342.59225832)(33.48181967,342.71226336)
\curveto(33.46181901,342.78225813)(33.44681903,342.86725805)(33.43681967,342.96726336)
\curveto(33.42681905,343.06725785)(33.43181904,343.15725776)(33.45181967,343.23726336)
\curveto(33.471819,343.28725763)(33.476819,343.32725759)(33.46681967,343.35726336)
\curveto(33.45681902,343.39725752)(33.46181901,343.44225747)(33.48181967,343.49226336)
\curveto(33.50181897,343.60225731)(33.52181895,343.70225721)(33.54181967,343.79226336)
\curveto(33.5718189,343.89225702)(33.60681887,343.98725693)(33.64681967,344.07726336)
\curveto(33.7768187,344.36725655)(33.95681852,344.60225631)(34.18681967,344.78226336)
\curveto(34.41681806,344.96225595)(34.6768178,345.10725581)(34.96681967,345.21726336)
\curveto(35.0768174,345.26725565)(35.19181728,345.30225561)(35.31181967,345.32226336)
\curveto(35.43181704,345.35225556)(35.55681692,345.38225553)(35.68681967,345.41226336)
\curveto(35.74681673,345.43225548)(35.80681667,345.44225547)(35.86681967,345.44226336)
\lineto(36.04681967,345.47226336)
\curveto(36.12681635,345.48225543)(36.21181626,345.48725543)(36.30181967,345.48726336)
\curveto(36.39181608,345.48725543)(36.476816,345.49225542)(36.55681967,345.50226336)
}
}
{
\newrgbcolor{curcolor}{0 0 0}
\pscustom[linestyle=none,fillstyle=solid,fillcolor=curcolor]
{
\newpath
\moveto(42.69346029,347.60226336)
\lineto(43.69846029,347.60226336)
\curveto(43.84845731,347.60225331)(43.97845718,347.59225332)(44.08846029,347.57226336)
\curveto(44.20845695,347.56225335)(44.29345686,347.50225341)(44.34346029,347.39226336)
\curveto(44.36345679,347.34225357)(44.37345678,347.28225363)(44.37346029,347.21226336)
\lineto(44.37346029,347.00226336)
\lineto(44.37346029,346.32726336)
\curveto(44.37345678,346.27725464)(44.36845679,346.2172547)(44.35846029,346.14726336)
\curveto(44.3584568,346.08725483)(44.36345679,346.03225488)(44.37346029,345.98226336)
\lineto(44.37346029,345.81726336)
\curveto(44.37345678,345.73725518)(44.37845678,345.66225525)(44.38846029,345.59226336)
\curveto(44.39845676,345.53225538)(44.42345673,345.47725544)(44.46346029,345.42726336)
\curveto(44.53345662,345.33725558)(44.6584565,345.28725563)(44.83846029,345.27726336)
\lineto(45.37846029,345.27726336)
\lineto(45.55846029,345.27726336)
\curveto(45.61845554,345.27725564)(45.67345548,345.26725565)(45.72346029,345.24726336)
\curveto(45.83345532,345.19725572)(45.89345526,345.10725581)(45.90346029,344.97726336)
\curveto(45.92345523,344.84725607)(45.93345522,344.70225621)(45.93346029,344.54226336)
\lineto(45.93346029,344.33226336)
\curveto(45.94345521,344.26225665)(45.93845522,344.20225671)(45.91846029,344.15226336)
\curveto(45.86845529,343.99225692)(45.76345539,343.90725701)(45.60346029,343.89726336)
\curveto(45.44345571,343.88725703)(45.26345589,343.88225703)(45.06346029,343.88226336)
\lineto(44.92846029,343.88226336)
\curveto(44.88845627,343.89225702)(44.8534563,343.89225702)(44.82346029,343.88226336)
\curveto(44.78345637,343.87225704)(44.74845641,343.86725705)(44.71846029,343.86726336)
\curveto(44.68845647,343.87725704)(44.6584565,343.87225704)(44.62846029,343.85226336)
\curveto(44.54845661,343.83225708)(44.48845667,343.78725713)(44.44846029,343.71726336)
\curveto(44.41845674,343.65725726)(44.39345676,343.58225733)(44.37346029,343.49226336)
\curveto(44.36345679,343.44225747)(44.36345679,343.38725753)(44.37346029,343.32726336)
\curveto(44.38345677,343.26725765)(44.38345677,343.2122577)(44.37346029,343.16226336)
\lineto(44.37346029,342.23226336)
\lineto(44.37346029,340.47726336)
\curveto(44.37345678,340.22726069)(44.37845678,340.00726091)(44.38846029,339.81726336)
\curveto(44.40845675,339.63726128)(44.47345668,339.47726144)(44.58346029,339.33726336)
\curveto(44.63345652,339.27726164)(44.69845646,339.23226168)(44.77846029,339.20226336)
\lineto(45.04846029,339.14226336)
\curveto(45.07845608,339.13226178)(45.10845605,339.12726179)(45.13846029,339.12726336)
\curveto(45.17845598,339.13726178)(45.20845595,339.13726178)(45.22846029,339.12726336)
\lineto(45.39346029,339.12726336)
\curveto(45.50345565,339.12726179)(45.59845556,339.12226179)(45.67846029,339.11226336)
\curveto(45.7584554,339.10226181)(45.82345533,339.06226185)(45.87346029,338.99226336)
\curveto(45.91345524,338.93226198)(45.93345522,338.85226206)(45.93346029,338.75226336)
\lineto(45.93346029,338.46726336)
\curveto(45.93345522,338.25726266)(45.92845523,338.06226285)(45.91846029,337.88226336)
\curveto(45.91845524,337.7122632)(45.83845532,337.59726332)(45.67846029,337.53726336)
\curveto(45.62845553,337.5172634)(45.58345557,337.5122634)(45.54346029,337.52226336)
\curveto(45.50345565,337.52226339)(45.4584557,337.5122634)(45.40846029,337.49226336)
\lineto(45.25846029,337.49226336)
\curveto(45.23845592,337.49226342)(45.20845595,337.49726342)(45.16846029,337.50726336)
\curveto(45.12845603,337.50726341)(45.09345606,337.50226341)(45.06346029,337.49226336)
\curveto(45.01345614,337.48226343)(44.9584562,337.48226343)(44.89846029,337.49226336)
\lineto(44.74846029,337.49226336)
\lineto(44.59846029,337.49226336)
\curveto(44.54845661,337.48226343)(44.50345665,337.48226343)(44.46346029,337.49226336)
\lineto(44.29846029,337.49226336)
\curveto(44.24845691,337.50226341)(44.19345696,337.50726341)(44.13346029,337.50726336)
\curveto(44.07345708,337.50726341)(44.01845714,337.5122634)(43.96846029,337.52226336)
\curveto(43.89845726,337.53226338)(43.83345732,337.54226337)(43.77346029,337.55226336)
\lineto(43.59346029,337.58226336)
\curveto(43.48345767,337.6122633)(43.37845778,337.64726327)(43.27846029,337.68726336)
\curveto(43.17845798,337.72726319)(43.08345807,337.77226314)(42.99346029,337.82226336)
\lineto(42.90346029,337.88226336)
\curveto(42.87345828,337.912263)(42.83845832,337.94226297)(42.79846029,337.97226336)
\curveto(42.77845838,337.99226292)(42.7534584,338.0122629)(42.72346029,338.03226336)
\lineto(42.64846029,338.10726336)
\curveto(42.50845865,338.29726262)(42.40345875,338.50726241)(42.33346029,338.73726336)
\curveto(42.31345884,338.77726214)(42.30345885,338.8122621)(42.30346029,338.84226336)
\curveto(42.31345884,338.88226203)(42.31345884,338.92726199)(42.30346029,338.97726336)
\curveto(42.29345886,338.99726192)(42.28845887,339.02226189)(42.28846029,339.05226336)
\curveto(42.28845887,339.08226183)(42.28345887,339.10726181)(42.27346029,339.12726336)
\lineto(42.27346029,339.27726336)
\curveto(42.26345889,339.3172616)(42.2584589,339.36226155)(42.25846029,339.41226336)
\curveto(42.26845889,339.46226145)(42.27345888,339.5122614)(42.27346029,339.56226336)
\lineto(42.27346029,340.13226336)
\lineto(42.27346029,342.36726336)
\lineto(42.27346029,343.16226336)
\lineto(42.27346029,343.37226336)
\curveto(42.28345887,343.44225747)(42.27845888,343.50725741)(42.25846029,343.56726336)
\curveto(42.21845894,343.70725721)(42.14845901,343.79725712)(42.04846029,343.83726336)
\curveto(41.93845922,343.88725703)(41.79845936,343.90225701)(41.62846029,343.88226336)
\curveto(41.4584597,343.86225705)(41.31345984,343.87725704)(41.19346029,343.92726336)
\curveto(41.11346004,343.95725696)(41.06346009,344.00225691)(41.04346029,344.06226336)
\curveto(41.02346013,344.12225679)(41.00346015,344.19725672)(40.98346029,344.28726336)
\lineto(40.98346029,344.60226336)
\curveto(40.98346017,344.78225613)(40.99346016,344.92725599)(41.01346029,345.03726336)
\curveto(41.03346012,345.14725577)(41.11846004,345.22225569)(41.26846029,345.26226336)
\curveto(41.30845985,345.28225563)(41.34845981,345.28725563)(41.38846029,345.27726336)
\lineto(41.52346029,345.27726336)
\curveto(41.67345948,345.27725564)(41.81345934,345.28225563)(41.94346029,345.29226336)
\curveto(42.07345908,345.3122556)(42.16345899,345.37225554)(42.21346029,345.47226336)
\curveto(42.24345891,345.54225537)(42.2584589,345.62225529)(42.25846029,345.71226336)
\curveto(42.26845889,345.80225511)(42.27345888,345.89225502)(42.27346029,345.98226336)
\lineto(42.27346029,346.91226336)
\lineto(42.27346029,347.16726336)
\curveto(42.27345888,347.25725366)(42.28345887,347.33225358)(42.30346029,347.39226336)
\curveto(42.3534588,347.49225342)(42.42845873,347.55725336)(42.52846029,347.58726336)
\curveto(42.54845861,347.59725332)(42.57345858,347.59725332)(42.60346029,347.58726336)
\curveto(42.64345851,347.58725333)(42.67345848,347.59225332)(42.69346029,347.60226336)
}
}
{
\newrgbcolor{curcolor}{0 0 0}
\pscustom[linestyle=none,fillstyle=solid,fillcolor=curcolor]
{
\newpath
\moveto(51.34189779,345.48726336)
\curveto(51.45189248,345.48725543)(51.54689238,345.47725544)(51.62689779,345.45726336)
\curveto(51.71689221,345.43725548)(51.78689214,345.39225552)(51.83689779,345.32226336)
\curveto(51.89689203,345.24225567)(51.926892,345.10225581)(51.92689779,344.90226336)
\lineto(51.92689779,344.39226336)
\lineto(51.92689779,344.01726336)
\curveto(51.93689199,343.87725704)(51.92189201,343.76725715)(51.88189779,343.68726336)
\curveto(51.84189209,343.6172573)(51.78189215,343.57225734)(51.70189779,343.55226336)
\curveto(51.6318923,343.53225738)(51.54689238,343.52225739)(51.44689779,343.52226336)
\curveto(51.35689257,343.52225739)(51.25689267,343.52725739)(51.14689779,343.53726336)
\curveto(51.04689288,343.54725737)(50.95189298,343.54225737)(50.86189779,343.52226336)
\curveto(50.79189314,343.50225741)(50.72189321,343.48725743)(50.65189779,343.47726336)
\curveto(50.58189335,343.47725744)(50.51689341,343.46725745)(50.45689779,343.44726336)
\curveto(50.29689363,343.39725752)(50.13689379,343.32225759)(49.97689779,343.22226336)
\curveto(49.81689411,343.13225778)(49.69189424,343.02725789)(49.60189779,342.90726336)
\curveto(49.55189438,342.82725809)(49.49689443,342.74225817)(49.43689779,342.65226336)
\curveto(49.38689454,342.57225834)(49.33689459,342.48725843)(49.28689779,342.39726336)
\curveto(49.25689467,342.3172586)(49.2268947,342.23225868)(49.19689779,342.14226336)
\lineto(49.13689779,341.90226336)
\curveto(49.11689481,341.83225908)(49.10689482,341.75725916)(49.10689779,341.67726336)
\curveto(49.10689482,341.60725931)(49.09689483,341.53725938)(49.07689779,341.46726336)
\curveto(49.06689486,341.42725949)(49.06189487,341.38725953)(49.06189779,341.34726336)
\curveto(49.07189486,341.3172596)(49.07189486,341.28725963)(49.06189779,341.25726336)
\lineto(49.06189779,341.01726336)
\curveto(49.04189489,340.94725997)(49.03689489,340.86726005)(49.04689779,340.77726336)
\curveto(49.05689487,340.69726022)(49.06189487,340.6172603)(49.06189779,340.53726336)
\lineto(49.06189779,339.57726336)
\lineto(49.06189779,338.30226336)
\curveto(49.06189487,338.17226274)(49.05689487,338.05226286)(49.04689779,337.94226336)
\curveto(49.03689489,337.83226308)(49.00689492,337.74226317)(48.95689779,337.67226336)
\curveto(48.93689499,337.64226327)(48.90189503,337.6172633)(48.85189779,337.59726336)
\curveto(48.81189512,337.58726333)(48.76689516,337.57726334)(48.71689779,337.56726336)
\lineto(48.64189779,337.56726336)
\curveto(48.59189534,337.55726336)(48.53689539,337.55226336)(48.47689779,337.55226336)
\lineto(48.31189779,337.55226336)
\lineto(47.66689779,337.55226336)
\curveto(47.60689632,337.56226335)(47.54189639,337.56726335)(47.47189779,337.56726336)
\lineto(47.27689779,337.56726336)
\curveto(47.2268967,337.58726333)(47.17689675,337.60226331)(47.12689779,337.61226336)
\curveto(47.07689685,337.63226328)(47.04189689,337.66726325)(47.02189779,337.71726336)
\curveto(46.98189695,337.76726315)(46.95689697,337.83726308)(46.94689779,337.92726336)
\lineto(46.94689779,338.22726336)
\lineto(46.94689779,339.24726336)
\lineto(46.94689779,343.47726336)
\lineto(46.94689779,344.58726336)
\lineto(46.94689779,344.87226336)
\curveto(46.94689698,344.97225594)(46.96689696,345.05225586)(47.00689779,345.11226336)
\curveto(47.05689687,345.19225572)(47.1318968,345.24225567)(47.23189779,345.26226336)
\curveto(47.3318966,345.28225563)(47.45189648,345.29225562)(47.59189779,345.29226336)
\lineto(48.35689779,345.29226336)
\curveto(48.47689545,345.29225562)(48.58189535,345.28225563)(48.67189779,345.26226336)
\curveto(48.76189517,345.25225566)(48.8318951,345.20725571)(48.88189779,345.12726336)
\curveto(48.91189502,345.07725584)(48.926895,345.00725591)(48.92689779,344.91726336)
\lineto(48.95689779,344.64726336)
\curveto(48.96689496,344.56725635)(48.98189495,344.49225642)(49.00189779,344.42226336)
\curveto(49.0318949,344.35225656)(49.08189485,344.3172566)(49.15189779,344.31726336)
\curveto(49.17189476,344.33725658)(49.19189474,344.34725657)(49.21189779,344.34726336)
\curveto(49.2318947,344.34725657)(49.25189468,344.35725656)(49.27189779,344.37726336)
\curveto(49.3318946,344.42725649)(49.38189455,344.48225643)(49.42189779,344.54226336)
\curveto(49.47189446,344.6122563)(49.5318944,344.67225624)(49.60189779,344.72226336)
\curveto(49.64189429,344.75225616)(49.67689425,344.78225613)(49.70689779,344.81226336)
\curveto(49.73689419,344.85225606)(49.77189416,344.88725603)(49.81189779,344.91726336)
\lineto(50.08189779,345.09726336)
\curveto(50.18189375,345.15725576)(50.28189365,345.2122557)(50.38189779,345.26226336)
\curveto(50.48189345,345.30225561)(50.58189335,345.33725558)(50.68189779,345.36726336)
\lineto(51.01189779,345.45726336)
\curveto(51.04189289,345.46725545)(51.09689283,345.46725545)(51.17689779,345.45726336)
\curveto(51.26689266,345.45725546)(51.32189261,345.46725545)(51.34189779,345.48726336)
}
}
{
\newrgbcolor{curcolor}{0 0 0}
\pscustom[linestyle=none,fillstyle=solid,fillcolor=curcolor]
{
\newpath
\moveto(54.84697592,348.14226336)
\curveto(54.91697297,348.06225285)(54.95197293,347.94225297)(54.95197592,347.78226336)
\lineto(54.95197592,347.31726336)
\lineto(54.95197592,346.91226336)
\curveto(54.95197293,346.77225414)(54.91697297,346.67725424)(54.84697592,346.62726336)
\curveto(54.7869731,346.57725434)(54.70697318,346.54725437)(54.60697592,346.53726336)
\curveto(54.51697337,346.52725439)(54.41697347,346.52225439)(54.30697592,346.52226336)
\lineto(53.46697592,346.52226336)
\curveto(53.35697453,346.52225439)(53.25697463,346.52725439)(53.16697592,346.53726336)
\curveto(53.0869748,346.54725437)(53.01697487,346.57725434)(52.95697592,346.62726336)
\curveto(52.91697497,346.65725426)(52.886975,346.7122542)(52.86697592,346.79226336)
\curveto(52.85697503,346.88225403)(52.84697504,346.97725394)(52.83697592,347.07726336)
\lineto(52.83697592,347.40726336)
\curveto(52.84697504,347.5172534)(52.85197503,347.6122533)(52.85197592,347.69226336)
\lineto(52.85197592,347.90226336)
\curveto(52.86197502,347.97225294)(52.881975,348.03225288)(52.91197592,348.08226336)
\curveto(52.93197495,348.12225279)(52.95697493,348.15225276)(52.98697592,348.17226336)
\lineto(53.10697592,348.23226336)
\curveto(53.12697476,348.23225268)(53.15197473,348.23225268)(53.18197592,348.23226336)
\curveto(53.21197467,348.24225267)(53.23697465,348.24725267)(53.25697592,348.24726336)
\lineto(54.35197592,348.24726336)
\curveto(54.45197343,348.24725267)(54.54697334,348.24225267)(54.63697592,348.23226336)
\curveto(54.72697316,348.22225269)(54.79697309,348.19225272)(54.84697592,348.14226336)
\moveto(54.95197592,338.37726336)
\curveto(54.95197293,338.17726274)(54.94697294,338.00726291)(54.93697592,337.86726336)
\curveto(54.92697296,337.72726319)(54.83697305,337.63226328)(54.66697592,337.58226336)
\curveto(54.60697328,337.56226335)(54.54197334,337.55226336)(54.47197592,337.55226336)
\curveto(54.40197348,337.56226335)(54.32697356,337.56726335)(54.24697592,337.56726336)
\lineto(53.40697592,337.56726336)
\curveto(53.31697457,337.56726335)(53.22697466,337.57226334)(53.13697592,337.58226336)
\curveto(53.05697483,337.59226332)(52.99697489,337.62226329)(52.95697592,337.67226336)
\curveto(52.89697499,337.74226317)(52.86197502,337.82726309)(52.85197592,337.92726336)
\lineto(52.85197592,338.27226336)
\lineto(52.85197592,344.60226336)
\lineto(52.85197592,344.90226336)
\curveto(52.85197503,345.00225591)(52.87197501,345.08225583)(52.91197592,345.14226336)
\curveto(52.97197491,345.2122557)(53.05697483,345.25725566)(53.16697592,345.27726336)
\curveto(53.1869747,345.28725563)(53.21197467,345.28725563)(53.24197592,345.27726336)
\curveto(53.2819746,345.27725564)(53.31197457,345.28225563)(53.33197592,345.29226336)
\lineto(54.08197592,345.29226336)
\lineto(54.27697592,345.29226336)
\curveto(54.35697353,345.30225561)(54.42197346,345.30225561)(54.47197592,345.29226336)
\lineto(54.59197592,345.29226336)
\curveto(54.65197323,345.27225564)(54.70697318,345.25725566)(54.75697592,345.24726336)
\curveto(54.80697308,345.23725568)(54.84697304,345.20725571)(54.87697592,345.15726336)
\curveto(54.91697297,345.10725581)(54.93697295,345.03725588)(54.93697592,344.94726336)
\curveto(54.94697294,344.85725606)(54.95197293,344.76225615)(54.95197592,344.66226336)
\lineto(54.95197592,338.37726336)
}
}
{
\newrgbcolor{curcolor}{0 0 0}
\pscustom[linestyle=none,fillstyle=solid,fillcolor=curcolor]
{
\newpath
\moveto(64.41416342,341.81226336)
\curveto(64.43415482,341.75225916)(64.44415481,341.64725927)(64.44416342,341.49726336)
\curveto(64.44415481,341.35725956)(64.43915481,341.25725966)(64.42916342,341.19726336)
\curveto(64.42915482,341.14725977)(64.42415483,341.10225981)(64.41416342,341.06226336)
\lineto(64.41416342,340.94226336)
\curveto(64.39415486,340.86226005)(64.38415487,340.78226013)(64.38416342,340.70226336)
\curveto(64.38415487,340.63226028)(64.37415488,340.55726036)(64.35416342,340.47726336)
\curveto(64.3541549,340.43726048)(64.34415491,340.36726055)(64.32416342,340.26726336)
\curveto(64.29415496,340.14726077)(64.26415499,340.02226089)(64.23416342,339.89226336)
\curveto(64.21415504,339.77226114)(64.17915507,339.65726126)(64.12916342,339.54726336)
\curveto(63.9491553,339.09726182)(63.72415553,338.70726221)(63.45416342,338.37726336)
\curveto(63.18415607,338.04726287)(62.82915642,337.78726313)(62.38916342,337.59726336)
\curveto(62.29915695,337.55726336)(62.20415705,337.52726339)(62.10416342,337.50726336)
\curveto(62.01415724,337.47726344)(61.91415734,337.44726347)(61.80416342,337.41726336)
\curveto(61.74415751,337.39726352)(61.67915757,337.38726353)(61.60916342,337.38726336)
\curveto(61.5491577,337.38726353)(61.48915776,337.38226353)(61.42916342,337.37226336)
\lineto(61.29416342,337.37226336)
\curveto(61.23415802,337.35226356)(61.1541581,337.34726357)(61.05416342,337.35726336)
\curveto(60.9541583,337.35726356)(60.87415838,337.36726355)(60.81416342,337.38726336)
\lineto(60.72416342,337.38726336)
\curveto(60.67415858,337.39726352)(60.61915863,337.40726351)(60.55916342,337.41726336)
\curveto(60.49915875,337.4172635)(60.43915881,337.42226349)(60.37916342,337.43226336)
\curveto(60.18915906,337.48226343)(60.01415924,337.53226338)(59.85416342,337.58226336)
\curveto(59.69415956,337.63226328)(59.54415971,337.70226321)(59.40416342,337.79226336)
\lineto(59.22416342,337.91226336)
\curveto(59.17416008,337.95226296)(59.12416013,337.99726292)(59.07416342,338.04726336)
\lineto(58.98416342,338.10726336)
\curveto(58.9541603,338.12726279)(58.92416033,338.14226277)(58.89416342,338.15226336)
\curveto(58.80416045,338.18226273)(58.7491605,338.16226275)(58.72916342,338.09226336)
\curveto(58.67916057,338.02226289)(58.64416061,337.93726298)(58.62416342,337.83726336)
\curveto(58.61416064,337.74726317)(58.57916067,337.67726324)(58.51916342,337.62726336)
\curveto(58.45916079,337.58726333)(58.38916086,337.56226335)(58.30916342,337.55226336)
\lineto(58.03916342,337.55226336)
\lineto(57.31916342,337.55226336)
\lineto(57.09416342,337.55226336)
\curveto(57.02416223,337.54226337)(56.95916229,337.54726337)(56.89916342,337.56726336)
\curveto(56.75916249,337.6172633)(56.67916257,337.70726321)(56.65916342,337.83726336)
\curveto(56.6491626,337.97726294)(56.64416261,338.13226278)(56.64416342,338.30226336)
\lineto(56.64416342,347.45226336)
\lineto(56.64416342,347.79726336)
\curveto(56.64416261,347.917253)(56.66916258,348.0122529)(56.71916342,348.08226336)
\curveto(56.75916249,348.15225276)(56.82916242,348.19725272)(56.92916342,348.21726336)
\curveto(56.9491623,348.22725269)(56.96916228,348.22725269)(56.98916342,348.21726336)
\curveto(57.01916223,348.2172527)(57.04416221,348.22225269)(57.06416342,348.23226336)
\lineto(58.00916342,348.23226336)
\curveto(58.18916106,348.23225268)(58.34416091,348.22225269)(58.47416342,348.20226336)
\curveto(58.60416065,348.19225272)(58.68916056,348.1172528)(58.72916342,347.97726336)
\curveto(58.75916049,347.87725304)(58.76916048,347.74225317)(58.75916342,347.57226336)
\curveto(58.7491605,347.4122535)(58.74416051,347.27225364)(58.74416342,347.15226336)
\lineto(58.74416342,345.51726336)
\lineto(58.74416342,345.18726336)
\curveto(58.74416051,345.07725584)(58.7541605,344.98225593)(58.77416342,344.90226336)
\curveto(58.78416047,344.85225606)(58.79416046,344.80725611)(58.80416342,344.76726336)
\curveto(58.81416044,344.73725618)(58.83916041,344.7172562)(58.87916342,344.70726336)
\curveto(58.89916035,344.68725623)(58.92416033,344.67725624)(58.95416342,344.67726336)
\curveto(58.99416026,344.67725624)(59.02416023,344.68225623)(59.04416342,344.69226336)
\curveto(59.11416014,344.73225618)(59.17916007,344.77225614)(59.23916342,344.81226336)
\curveto(59.29915995,344.86225605)(59.36415989,344.912256)(59.43416342,344.96226336)
\curveto(59.56415969,345.05225586)(59.69915955,345.12725579)(59.83916342,345.18726336)
\curveto(59.97915927,345.25725566)(60.13415912,345.3172556)(60.30416342,345.36726336)
\curveto(60.38415887,345.39725552)(60.46415879,345.4122555)(60.54416342,345.41226336)
\curveto(60.62415863,345.42225549)(60.70415855,345.43725548)(60.78416342,345.45726336)
\curveto(60.8541584,345.47725544)(60.92915832,345.48725543)(61.00916342,345.48726336)
\lineto(61.24916342,345.48726336)
\lineto(61.39916342,345.48726336)
\curveto(61.42915782,345.47725544)(61.46415779,345.47225544)(61.50416342,345.47226336)
\curveto(61.54415771,345.48225543)(61.58415767,345.48225543)(61.62416342,345.47226336)
\curveto(61.73415752,345.44225547)(61.83415742,345.4172555)(61.92416342,345.39726336)
\curveto(62.02415723,345.38725553)(62.11915713,345.36225555)(62.20916342,345.32226336)
\curveto(62.66915658,345.13225578)(63.04415621,344.88725603)(63.33416342,344.58726336)
\curveto(63.62415563,344.28725663)(63.86915538,343.912257)(64.06916342,343.46226336)
\curveto(64.11915513,343.34225757)(64.15915509,343.2172577)(64.18916342,343.08726336)
\curveto(64.22915502,342.95725796)(64.26915498,342.82225809)(64.30916342,342.68226336)
\curveto(64.32915492,342.6122583)(64.33915491,342.54225837)(64.33916342,342.47226336)
\curveto(64.3491549,342.4122585)(64.36415489,342.34225857)(64.38416342,342.26226336)
\curveto(64.40415485,342.2122587)(64.40915484,342.15725876)(64.39916342,342.09726336)
\curveto(64.39915485,342.03725888)(64.40415485,341.97725894)(64.41416342,341.91726336)
\lineto(64.41416342,341.81226336)
\moveto(62.19416342,340.40226336)
\curveto(62.22415703,340.50226041)(62.249157,340.62726029)(62.26916342,340.77726336)
\curveto(62.29915695,340.92725999)(62.31415694,341.07725984)(62.31416342,341.22726336)
\curveto(62.32415693,341.38725953)(62.32415693,341.54225937)(62.31416342,341.69226336)
\curveto(62.31415694,341.85225906)(62.29915695,341.98725893)(62.26916342,342.09726336)
\curveto(62.23915701,342.19725872)(62.21915703,342.29225862)(62.20916342,342.38226336)
\curveto(62.19915705,342.47225844)(62.17415708,342.55725836)(62.13416342,342.63726336)
\curveto(61.99415726,342.98725793)(61.79415746,343.28225763)(61.53416342,343.52226336)
\curveto(61.28415797,343.77225714)(60.91415834,343.89725702)(60.42416342,343.89726336)
\curveto(60.38415887,343.89725702)(60.3491589,343.89225702)(60.31916342,343.88226336)
\lineto(60.21416342,343.88226336)
\curveto(60.14415911,343.86225705)(60.07915917,343.84225707)(60.01916342,343.82226336)
\curveto(59.95915929,343.8122571)(59.89915935,343.79725712)(59.83916342,343.77726336)
\curveto(59.5491597,343.64725727)(59.32915992,343.46225745)(59.17916342,343.22226336)
\curveto(59.02916022,342.99225792)(58.90416035,342.72725819)(58.80416342,342.42726336)
\curveto(58.77416048,342.34725857)(58.7541605,342.26225865)(58.74416342,342.17226336)
\curveto(58.74416051,342.09225882)(58.73416052,342.0122589)(58.71416342,341.93226336)
\curveto(58.70416055,341.90225901)(58.69916055,341.85225906)(58.69916342,341.78226336)
\curveto(58.68916056,341.74225917)(58.68416057,341.70225921)(58.68416342,341.66226336)
\curveto(58.69416056,341.62225929)(58.69416056,341.58225933)(58.68416342,341.54226336)
\curveto(58.66416059,341.46225945)(58.65916059,341.35225956)(58.66916342,341.21226336)
\curveto(58.67916057,341.07225984)(58.69416056,340.97225994)(58.71416342,340.91226336)
\curveto(58.73416052,340.82226009)(58.74416051,340.73726018)(58.74416342,340.65726336)
\curveto(58.7541605,340.57726034)(58.77416048,340.49726042)(58.80416342,340.41726336)
\curveto(58.89416036,340.13726078)(58.99916025,339.89226102)(59.11916342,339.68226336)
\curveto(59.24916,339.48226143)(59.42915982,339.3122616)(59.65916342,339.17226336)
\curveto(59.81915943,339.07226184)(59.98415927,339.00226191)(60.15416342,338.96226336)
\curveto(60.17415908,338.96226195)(60.19415906,338.95726196)(60.21416342,338.94726336)
\lineto(60.30416342,338.94726336)
\curveto(60.33415892,338.93726198)(60.38415887,338.92726199)(60.45416342,338.91726336)
\curveto(60.52415873,338.917262)(60.58415867,338.92226199)(60.63416342,338.93226336)
\curveto(60.73415852,338.95226196)(60.82415843,338.96726195)(60.90416342,338.97726336)
\curveto(60.99415826,338.99726192)(61.07915817,339.02226189)(61.15916342,339.05226336)
\curveto(61.43915781,339.18226173)(61.6541576,339.36226155)(61.80416342,339.59226336)
\curveto(61.96415729,339.82226109)(62.09415716,340.09226082)(62.19416342,340.40226336)
}
}
{
\newrgbcolor{curcolor}{0 0 0}
\pscustom[linestyle=none,fillstyle=solid,fillcolor=curcolor]
{
\newpath
\moveto(66.20408529,345.27726336)
\lineto(67.32908529,345.27726336)
\curveto(67.43908286,345.27725564)(67.53908276,345.27225564)(67.62908529,345.26226336)
\curveto(67.71908258,345.25225566)(67.78408251,345.2172557)(67.82408529,345.15726336)
\curveto(67.87408242,345.09725582)(67.90408239,345.0122559)(67.91408529,344.90226336)
\curveto(67.92408237,344.80225611)(67.92908237,344.69725622)(67.92908529,344.58726336)
\lineto(67.92908529,343.53726336)
\lineto(67.92908529,341.30226336)
\curveto(67.92908237,340.94225997)(67.94408235,340.60226031)(67.97408529,340.28226336)
\curveto(68.00408229,339.96226095)(68.0940822,339.69726122)(68.24408529,339.48726336)
\curveto(68.38408191,339.27726164)(68.60908169,339.12726179)(68.91908529,339.03726336)
\curveto(68.96908133,339.02726189)(69.00908129,339.02226189)(69.03908529,339.02226336)
\curveto(69.07908122,339.02226189)(69.12408117,339.0172619)(69.17408529,339.00726336)
\curveto(69.22408107,338.99726192)(69.27908102,338.99226192)(69.33908529,338.99226336)
\curveto(69.3990809,338.99226192)(69.44408085,338.99726192)(69.47408529,339.00726336)
\curveto(69.52408077,339.02726189)(69.56408073,339.03226188)(69.59408529,339.02226336)
\curveto(69.63408066,339.0122619)(69.67408062,339.0172619)(69.71408529,339.03726336)
\curveto(69.92408037,339.08726183)(70.08908021,339.15226176)(70.20908529,339.23226336)
\curveto(70.38907991,339.34226157)(70.52907977,339.48226143)(70.62908529,339.65226336)
\curveto(70.73907956,339.83226108)(70.81407948,340.02726089)(70.85408529,340.23726336)
\curveto(70.90407939,340.45726046)(70.93407936,340.69726022)(70.94408529,340.95726336)
\curveto(70.95407934,341.22725969)(70.95907934,341.50725941)(70.95908529,341.79726336)
\lineto(70.95908529,343.61226336)
\lineto(70.95908529,344.58726336)
\lineto(70.95908529,344.85726336)
\curveto(70.95907934,344.95725596)(70.97907932,345.03725588)(71.01908529,345.09726336)
\curveto(71.06907923,345.18725573)(71.14407915,345.23725568)(71.24408529,345.24726336)
\curveto(71.34407895,345.26725565)(71.46407883,345.27725564)(71.60408529,345.27726336)
\lineto(72.39908529,345.27726336)
\lineto(72.68408529,345.27726336)
\curveto(72.77407752,345.27725564)(72.84907745,345.25725566)(72.90908529,345.21726336)
\curveto(72.98907731,345.16725575)(73.03407726,345.09225582)(73.04408529,344.99226336)
\curveto(73.05407724,344.89225602)(73.05907724,344.77725614)(73.05908529,344.64726336)
\lineto(73.05908529,343.50726336)
\lineto(73.05908529,339.29226336)
\lineto(73.05908529,338.22726336)
\lineto(73.05908529,337.92726336)
\curveto(73.05907724,337.82726309)(73.03907726,337.75226316)(72.99908529,337.70226336)
\curveto(72.94907735,337.62226329)(72.87407742,337.57726334)(72.77408529,337.56726336)
\curveto(72.67407762,337.55726336)(72.56907773,337.55226336)(72.45908529,337.55226336)
\lineto(71.64908529,337.55226336)
\curveto(71.53907876,337.55226336)(71.43907886,337.55726336)(71.34908529,337.56726336)
\curveto(71.26907903,337.57726334)(71.20407909,337.6172633)(71.15408529,337.68726336)
\curveto(71.13407916,337.7172632)(71.11407918,337.76226315)(71.09408529,337.82226336)
\curveto(71.08407921,337.88226303)(71.06907923,337.94226297)(71.04908529,338.00226336)
\curveto(71.03907926,338.06226285)(71.02407927,338.1172628)(71.00408529,338.16726336)
\curveto(70.98407931,338.2172627)(70.95407934,338.24726267)(70.91408529,338.25726336)
\curveto(70.8940794,338.27726264)(70.86907943,338.28226263)(70.83908529,338.27226336)
\curveto(70.80907949,338.26226265)(70.78407951,338.25226266)(70.76408529,338.24226336)
\curveto(70.6940796,338.20226271)(70.63407966,338.15726276)(70.58408529,338.10726336)
\curveto(70.53407976,338.05726286)(70.47907982,338.0122629)(70.41908529,337.97226336)
\curveto(70.37907992,337.94226297)(70.33907996,337.90726301)(70.29908529,337.86726336)
\curveto(70.26908003,337.83726308)(70.22908007,337.80726311)(70.17908529,337.77726336)
\curveto(69.94908035,337.63726328)(69.67908062,337.52726339)(69.36908529,337.44726336)
\curveto(69.299081,337.42726349)(69.22908107,337.4172635)(69.15908529,337.41726336)
\curveto(69.08908121,337.40726351)(69.01408128,337.39226352)(68.93408529,337.37226336)
\curveto(68.8940814,337.36226355)(68.84908145,337.36226355)(68.79908529,337.37226336)
\curveto(68.75908154,337.37226354)(68.71908158,337.36726355)(68.67908529,337.35726336)
\curveto(68.64908165,337.34726357)(68.58408171,337.34726357)(68.48408529,337.35726336)
\curveto(68.3940819,337.35726356)(68.33408196,337.36226355)(68.30408529,337.37226336)
\curveto(68.25408204,337.37226354)(68.20408209,337.37726354)(68.15408529,337.38726336)
\lineto(68.00408529,337.38726336)
\curveto(67.88408241,337.4172635)(67.76908253,337.44226347)(67.65908529,337.46226336)
\curveto(67.54908275,337.48226343)(67.43908286,337.5122634)(67.32908529,337.55226336)
\curveto(67.27908302,337.57226334)(67.23408306,337.58726333)(67.19408529,337.59726336)
\curveto(67.16408313,337.6172633)(67.12408317,337.63726328)(67.07408529,337.65726336)
\curveto(66.72408357,337.84726307)(66.44408385,338.1122628)(66.23408529,338.45226336)
\curveto(66.10408419,338.66226225)(66.00908429,338.912262)(65.94908529,339.20226336)
\curveto(65.88908441,339.50226141)(65.84908445,339.8172611)(65.82908529,340.14726336)
\curveto(65.81908448,340.48726043)(65.81408448,340.83226008)(65.81408529,341.18226336)
\curveto(65.82408447,341.54225937)(65.82908447,341.89725902)(65.82908529,342.24726336)
\lineto(65.82908529,344.28726336)
\curveto(65.82908447,344.4172565)(65.82408447,344.56725635)(65.81408529,344.73726336)
\curveto(65.81408448,344.917256)(65.83908446,345.04725587)(65.88908529,345.12726336)
\curveto(65.91908438,345.17725574)(65.97908432,345.22225569)(66.06908529,345.26226336)
\curveto(66.12908417,345.26225565)(66.17408412,345.26725565)(66.20408529,345.27726336)
}
}
{
\newrgbcolor{curcolor}{0 0 0}
\pscustom[linestyle=none,fillstyle=solid,fillcolor=curcolor]
{
\newpath
\moveto(78.26033529,345.50226336)
\curveto(79.07033013,345.52225539)(79.74532946,345.40225551)(80.28533529,345.14226336)
\curveto(80.83532837,344.88225603)(81.27032793,344.5122564)(81.59033529,344.03226336)
\curveto(81.75032745,343.79225712)(81.87032733,343.5172574)(81.95033529,343.20726336)
\curveto(81.97032723,343.15725776)(81.98532722,343.09225782)(81.99533529,343.01226336)
\curveto(82.01532719,342.93225798)(82.01532719,342.86225805)(81.99533529,342.80226336)
\curveto(81.95532725,342.69225822)(81.88532732,342.62725829)(81.78533529,342.60726336)
\curveto(81.68532752,342.59725832)(81.56532764,342.59225832)(81.42533529,342.59226336)
\lineto(80.64533529,342.59226336)
\lineto(80.36033529,342.59226336)
\curveto(80.27032893,342.59225832)(80.19532901,342.6122583)(80.13533529,342.65226336)
\curveto(80.05532915,342.69225822)(80.0003292,342.75225816)(79.97033529,342.83226336)
\curveto(79.94032926,342.92225799)(79.9003293,343.0122579)(79.85033529,343.10226336)
\curveto(79.79032941,343.2122577)(79.72532948,343.3122576)(79.65533529,343.40226336)
\curveto(79.58532962,343.49225742)(79.5053297,343.57225734)(79.41533529,343.64226336)
\curveto(79.27532993,343.73225718)(79.12033008,343.80225711)(78.95033529,343.85226336)
\curveto(78.89033031,343.87225704)(78.83033037,343.88225703)(78.77033529,343.88226336)
\curveto(78.71033049,343.88225703)(78.65533055,343.89225702)(78.60533529,343.91226336)
\lineto(78.45533529,343.91226336)
\curveto(78.25533095,343.912257)(78.09533111,343.89225702)(77.97533529,343.85226336)
\curveto(77.68533152,343.76225715)(77.45033175,343.62225729)(77.27033529,343.43226336)
\curveto(77.09033211,343.25225766)(76.94533226,343.03225788)(76.83533529,342.77226336)
\curveto(76.78533242,342.66225825)(76.74533246,342.54225837)(76.71533529,342.41226336)
\curveto(76.69533251,342.29225862)(76.67033253,342.16225875)(76.64033529,342.02226336)
\curveto(76.63033257,341.98225893)(76.62533258,341.94225897)(76.62533529,341.90226336)
\curveto(76.62533258,341.86225905)(76.62033258,341.82225909)(76.61033529,341.78226336)
\curveto(76.59033261,341.68225923)(76.58033262,341.54225937)(76.58033529,341.36226336)
\curveto(76.59033261,341.18225973)(76.6053326,341.04225987)(76.62533529,340.94226336)
\curveto(76.62533258,340.86226005)(76.63033257,340.80726011)(76.64033529,340.77726336)
\curveto(76.66033254,340.70726021)(76.67033253,340.63726028)(76.67033529,340.56726336)
\curveto(76.68033252,340.49726042)(76.69533251,340.42726049)(76.71533529,340.35726336)
\curveto(76.79533241,340.12726079)(76.89033231,339.917261)(77.00033529,339.72726336)
\curveto(77.11033209,339.53726138)(77.25033195,339.37726154)(77.42033529,339.24726336)
\curveto(77.46033174,339.2172617)(77.52033168,339.18226173)(77.60033529,339.14226336)
\curveto(77.71033149,339.07226184)(77.82033138,339.02726189)(77.93033529,339.00726336)
\curveto(78.05033115,338.98726193)(78.19533101,338.96726195)(78.36533529,338.94726336)
\lineto(78.45533529,338.94726336)
\curveto(78.49533071,338.94726197)(78.52533068,338.95226196)(78.54533529,338.96226336)
\lineto(78.68033529,338.96226336)
\curveto(78.75033045,338.98226193)(78.81533039,338.99726192)(78.87533529,339.00726336)
\curveto(78.94533026,339.02726189)(79.01033019,339.04726187)(79.07033529,339.06726336)
\curveto(79.37032983,339.19726172)(79.6003296,339.38726153)(79.76033529,339.63726336)
\curveto(79.8003294,339.68726123)(79.83532937,339.74226117)(79.86533529,339.80226336)
\curveto(79.89532931,339.87226104)(79.92032928,339.93226098)(79.94033529,339.98226336)
\curveto(79.98032922,340.09226082)(80.01532919,340.18726073)(80.04533529,340.26726336)
\curveto(80.07532913,340.35726056)(80.14532906,340.42726049)(80.25533529,340.47726336)
\curveto(80.34532886,340.5172604)(80.49032871,340.53226038)(80.69033529,340.52226336)
\lineto(81.18533529,340.52226336)
\lineto(81.39533529,340.52226336)
\curveto(81.47532773,340.53226038)(81.54032766,340.52726039)(81.59033529,340.50726336)
\lineto(81.71033529,340.50726336)
\lineto(81.83033529,340.47726336)
\curveto(81.87032733,340.47726044)(81.9003273,340.46726045)(81.92033529,340.44726336)
\curveto(81.97032723,340.40726051)(82.0003272,340.34726057)(82.01033529,340.26726336)
\curveto(82.03032717,340.19726072)(82.03032717,340.12226079)(82.01033529,340.04226336)
\curveto(81.92032728,339.7122612)(81.81032739,339.4172615)(81.68033529,339.15726336)
\curveto(81.27032793,338.38726253)(80.61532859,337.85226306)(79.71533529,337.55226336)
\curveto(79.61532959,337.52226339)(79.51032969,337.50226341)(79.40033529,337.49226336)
\curveto(79.29032991,337.47226344)(79.18033002,337.44726347)(79.07033529,337.41726336)
\curveto(79.01033019,337.40726351)(78.95033025,337.40226351)(78.89033529,337.40226336)
\curveto(78.83033037,337.40226351)(78.77033043,337.39726352)(78.71033529,337.38726336)
\lineto(78.54533529,337.38726336)
\curveto(78.49533071,337.36726355)(78.42033078,337.36226355)(78.32033529,337.37226336)
\curveto(78.22033098,337.37226354)(78.14533106,337.37726354)(78.09533529,337.38726336)
\curveto(78.01533119,337.40726351)(77.94033126,337.4172635)(77.87033529,337.41726336)
\curveto(77.81033139,337.40726351)(77.74533146,337.4122635)(77.67533529,337.43226336)
\lineto(77.52533529,337.46226336)
\curveto(77.47533173,337.46226345)(77.42533178,337.46726345)(77.37533529,337.47726336)
\curveto(77.26533194,337.50726341)(77.16033204,337.53726338)(77.06033529,337.56726336)
\curveto(76.96033224,337.59726332)(76.86533234,337.63226328)(76.77533529,337.67226336)
\curveto(76.3053329,337.87226304)(75.91033329,338.12726279)(75.59033529,338.43726336)
\curveto(75.27033393,338.75726216)(75.01033419,339.15226176)(74.81033529,339.62226336)
\curveto(74.76033444,339.7122612)(74.72033448,339.80726111)(74.69033529,339.90726336)
\lineto(74.60033529,340.23726336)
\curveto(74.59033461,340.27726064)(74.58533462,340.3122606)(74.58533529,340.34226336)
\curveto(74.58533462,340.38226053)(74.57533463,340.42726049)(74.55533529,340.47726336)
\curveto(74.53533467,340.54726037)(74.52533468,340.6172603)(74.52533529,340.68726336)
\curveto(74.52533468,340.76726015)(74.51533469,340.84226007)(74.49533529,340.91226336)
\lineto(74.49533529,341.16726336)
\curveto(74.47533473,341.2172597)(74.46533474,341.27225964)(74.46533529,341.33226336)
\curveto(74.46533474,341.40225951)(74.47533473,341.46225945)(74.49533529,341.51226336)
\curveto(74.5053347,341.56225935)(74.5053347,341.60725931)(74.49533529,341.64726336)
\curveto(74.48533472,341.68725923)(74.48533472,341.72725919)(74.49533529,341.76726336)
\curveto(74.51533469,341.83725908)(74.52033468,341.90225901)(74.51033529,341.96226336)
\curveto(74.51033469,342.02225889)(74.52033468,342.08225883)(74.54033529,342.14226336)
\curveto(74.59033461,342.32225859)(74.63033457,342.49225842)(74.66033529,342.65226336)
\curveto(74.69033451,342.82225809)(74.73533447,342.98725793)(74.79533529,343.14726336)
\curveto(75.01533419,343.65725726)(75.29033391,344.08225683)(75.62033529,344.42226336)
\curveto(75.96033324,344.76225615)(76.39033281,345.03725588)(76.91033529,345.24726336)
\curveto(77.05033215,345.30725561)(77.19533201,345.34725557)(77.34533529,345.36726336)
\curveto(77.49533171,345.39725552)(77.65033155,345.43225548)(77.81033529,345.47226336)
\curveto(77.89033131,345.48225543)(77.96533124,345.48725543)(78.03533529,345.48726336)
\curveto(78.1053311,345.48725543)(78.18033102,345.49225542)(78.26033529,345.50226336)
}
}
{
\newrgbcolor{curcolor}{0 0 0}
\pscustom[linestyle=none,fillstyle=solid,fillcolor=curcolor]
{
\newpath
\moveto(85.40361654,348.14226336)
\curveto(85.47361359,348.06225285)(85.50861356,347.94225297)(85.50861654,347.78226336)
\lineto(85.50861654,347.31726336)
\lineto(85.50861654,346.91226336)
\curveto(85.50861356,346.77225414)(85.47361359,346.67725424)(85.40361654,346.62726336)
\curveto(85.34361372,346.57725434)(85.2636138,346.54725437)(85.16361654,346.53726336)
\curveto(85.07361399,346.52725439)(84.97361409,346.52225439)(84.86361654,346.52226336)
\lineto(84.02361654,346.52226336)
\curveto(83.91361515,346.52225439)(83.81361525,346.52725439)(83.72361654,346.53726336)
\curveto(83.64361542,346.54725437)(83.57361549,346.57725434)(83.51361654,346.62726336)
\curveto(83.47361559,346.65725426)(83.44361562,346.7122542)(83.42361654,346.79226336)
\curveto(83.41361565,346.88225403)(83.40361566,346.97725394)(83.39361654,347.07726336)
\lineto(83.39361654,347.40726336)
\curveto(83.40361566,347.5172534)(83.40861566,347.6122533)(83.40861654,347.69226336)
\lineto(83.40861654,347.90226336)
\curveto(83.41861565,347.97225294)(83.43861563,348.03225288)(83.46861654,348.08226336)
\curveto(83.48861558,348.12225279)(83.51361555,348.15225276)(83.54361654,348.17226336)
\lineto(83.66361654,348.23226336)
\curveto(83.68361538,348.23225268)(83.70861536,348.23225268)(83.73861654,348.23226336)
\curveto(83.7686153,348.24225267)(83.79361527,348.24725267)(83.81361654,348.24726336)
\lineto(84.90861654,348.24726336)
\curveto(85.00861406,348.24725267)(85.10361396,348.24225267)(85.19361654,348.23226336)
\curveto(85.28361378,348.22225269)(85.35361371,348.19225272)(85.40361654,348.14226336)
\moveto(85.50861654,338.37726336)
\curveto(85.50861356,338.17726274)(85.50361356,338.00726291)(85.49361654,337.86726336)
\curveto(85.48361358,337.72726319)(85.39361367,337.63226328)(85.22361654,337.58226336)
\curveto(85.1636139,337.56226335)(85.09861397,337.55226336)(85.02861654,337.55226336)
\curveto(84.95861411,337.56226335)(84.88361418,337.56726335)(84.80361654,337.56726336)
\lineto(83.96361654,337.56726336)
\curveto(83.87361519,337.56726335)(83.78361528,337.57226334)(83.69361654,337.58226336)
\curveto(83.61361545,337.59226332)(83.55361551,337.62226329)(83.51361654,337.67226336)
\curveto(83.45361561,337.74226317)(83.41861565,337.82726309)(83.40861654,337.92726336)
\lineto(83.40861654,338.27226336)
\lineto(83.40861654,344.60226336)
\lineto(83.40861654,344.90226336)
\curveto(83.40861566,345.00225591)(83.42861564,345.08225583)(83.46861654,345.14226336)
\curveto(83.52861554,345.2122557)(83.61361545,345.25725566)(83.72361654,345.27726336)
\curveto(83.74361532,345.28725563)(83.7686153,345.28725563)(83.79861654,345.27726336)
\curveto(83.83861523,345.27725564)(83.8686152,345.28225563)(83.88861654,345.29226336)
\lineto(84.63861654,345.29226336)
\lineto(84.83361654,345.29226336)
\curveto(84.91361415,345.30225561)(84.97861409,345.30225561)(85.02861654,345.29226336)
\lineto(85.14861654,345.29226336)
\curveto(85.20861386,345.27225564)(85.2636138,345.25725566)(85.31361654,345.24726336)
\curveto(85.3636137,345.23725568)(85.40361366,345.20725571)(85.43361654,345.15726336)
\curveto(85.47361359,345.10725581)(85.49361357,345.03725588)(85.49361654,344.94726336)
\curveto(85.50361356,344.85725606)(85.50861356,344.76225615)(85.50861654,344.66226336)
\lineto(85.50861654,338.37726336)
}
}
{
\newrgbcolor{curcolor}{0 0 0}
\pscustom[linestyle=none,fillstyle=solid,fillcolor=curcolor]
{
\newpath
\moveto(94.94080404,341.73726336)
\curveto(94.92079551,341.78725913)(94.91579552,341.84225907)(94.92580404,341.90226336)
\curveto(94.9357955,341.96225895)(94.9307955,342.0172589)(94.91080404,342.06726336)
\curveto(94.90079553,342.10725881)(94.89579554,342.14725877)(94.89580404,342.18726336)
\curveto(94.89579554,342.22725869)(94.89079554,342.26725865)(94.88080404,342.30726336)
\lineto(94.82080404,342.57726336)
\curveto(94.80079563,342.66725825)(94.77579566,342.75225816)(94.74580404,342.83226336)
\curveto(94.69579574,342.97225794)(94.65079578,343.10225781)(94.61080404,343.22226336)
\curveto(94.57079586,343.35225756)(94.51579592,343.47225744)(94.44580404,343.58226336)
\curveto(94.37579606,343.69225722)(94.30579613,343.79725712)(94.23580404,343.89726336)
\curveto(94.17579626,343.99725692)(94.10579633,344.09725682)(94.02580404,344.19726336)
\curveto(93.94579649,344.30725661)(93.84579659,344.40725651)(93.72580404,344.49726336)
\curveto(93.61579682,344.59725632)(93.50579693,344.68725623)(93.39580404,344.76726336)
\curveto(93.06579737,344.99725592)(92.68579775,345.17725574)(92.25580404,345.30726336)
\curveto(91.8357986,345.43725548)(91.3357991,345.49725542)(90.75580404,345.48726336)
\curveto(90.68579975,345.47725544)(90.61579982,345.47225544)(90.54580404,345.47226336)
\curveto(90.47579996,345.47225544)(90.40080003,345.46725545)(90.32080404,345.45726336)
\curveto(90.17080026,345.4172555)(90.02580041,345.38725553)(89.88580404,345.36726336)
\curveto(89.74580069,345.34725557)(89.61080082,345.3122556)(89.48080404,345.26226336)
\curveto(89.37080106,345.2122557)(89.26080117,345.16725575)(89.15080404,345.12726336)
\curveto(89.04080139,345.08725583)(88.9358015,345.04225587)(88.83580404,344.99226336)
\curveto(88.47580196,344.76225615)(88.17080226,344.50725641)(87.92080404,344.22726336)
\curveto(87.67080276,343.95725696)(87.45580298,343.6172573)(87.27580404,343.20726336)
\curveto(87.22580321,343.08725783)(87.18580325,342.96225795)(87.15580404,342.83226336)
\curveto(87.12580331,342.7122582)(87.09080334,342.58725833)(87.05080404,342.45726336)
\curveto(87.0308034,342.40725851)(87.02080341,342.35725856)(87.02080404,342.30726336)
\curveto(87.02080341,342.26725865)(87.01580342,342.22225869)(87.00580404,342.17226336)
\curveto(86.98580345,342.12225879)(86.97580346,342.06725885)(86.97580404,342.00726336)
\curveto(86.98580345,341.95725896)(86.98580345,341.90725901)(86.97580404,341.85726336)
\lineto(86.97580404,341.75226336)
\curveto(86.95580348,341.69225922)(86.94080349,341.60725931)(86.93080404,341.49726336)
\curveto(86.9308035,341.38725953)(86.94080349,341.30225961)(86.96080404,341.24226336)
\lineto(86.96080404,341.10726336)
\curveto(86.96080347,341.06725985)(86.96580347,341.02225989)(86.97580404,340.97226336)
\curveto(86.99580344,340.89226002)(87.00580343,340.80726011)(87.00580404,340.71726336)
\curveto(87.00580343,340.63726028)(87.01580342,340.55726036)(87.03580404,340.47726336)
\curveto(87.05580338,340.42726049)(87.06580337,340.38226053)(87.06580404,340.34226336)
\curveto(87.06580337,340.30226061)(87.07580336,340.25726066)(87.09580404,340.20726336)
\curveto(87.12580331,340.09726082)(87.15080328,339.99226092)(87.17080404,339.89226336)
\curveto(87.20080323,339.79226112)(87.24080319,339.69726122)(87.29080404,339.60726336)
\curveto(87.46080297,339.2172617)(87.67080276,338.88226203)(87.92080404,338.60226336)
\curveto(88.17080226,338.32226259)(88.47080196,338.07726284)(88.82080404,337.86726336)
\curveto(88.9308015,337.80726311)(89.0358014,337.75726316)(89.13580404,337.71726336)
\curveto(89.24580119,337.67726324)(89.36080107,337.63726328)(89.48080404,337.59726336)
\curveto(89.57080086,337.55726336)(89.66580077,337.52726339)(89.76580404,337.50726336)
\curveto(89.86580057,337.48726343)(89.96580047,337.46226345)(90.06580404,337.43226336)
\curveto(90.11580032,337.42226349)(90.15580028,337.4172635)(90.18580404,337.41726336)
\curveto(90.22580021,337.4172635)(90.26580017,337.4122635)(90.30580404,337.40226336)
\curveto(90.35580008,337.38226353)(90.40580003,337.37726354)(90.45580404,337.38726336)
\curveto(90.51579992,337.38726353)(90.57079986,337.38226353)(90.62080404,337.37226336)
\lineto(90.77080404,337.37226336)
\curveto(90.8307996,337.35226356)(90.91579952,337.34726357)(91.02580404,337.35726336)
\curveto(91.1357993,337.35726356)(91.21579922,337.36226355)(91.26580404,337.37226336)
\curveto(91.29579914,337.37226354)(91.32579911,337.37726354)(91.35580404,337.38726336)
\lineto(91.46080404,337.38726336)
\curveto(91.51079892,337.39726352)(91.56579887,337.40226351)(91.62580404,337.40226336)
\curveto(91.68579875,337.40226351)(91.74079869,337.4122635)(91.79080404,337.43226336)
\curveto(91.92079851,337.46226345)(92.04579839,337.49226342)(92.16580404,337.52226336)
\curveto(92.29579814,337.54226337)(92.42079801,337.57726334)(92.54080404,337.62726336)
\curveto(93.02079741,337.82726309)(93.430797,338.07726284)(93.77080404,338.37726336)
\curveto(94.11079632,338.67726224)(94.38579605,339.06726185)(94.59580404,339.54726336)
\curveto(94.64579579,339.64726127)(94.68579575,339.75226116)(94.71580404,339.86226336)
\curveto(94.74579569,339.98226093)(94.78079565,340.09726082)(94.82080404,340.20726336)
\curveto(94.8307956,340.27726064)(94.84079559,340.34226057)(94.85080404,340.40226336)
\curveto(94.86079557,340.46226045)(94.87579556,340.52726039)(94.89580404,340.59726336)
\curveto(94.91579552,340.67726024)(94.92079551,340.75726016)(94.91080404,340.83726336)
\curveto(94.91079552,340.91726)(94.92079551,340.99725992)(94.94080404,341.07726336)
\lineto(94.94080404,341.22726336)
\curveto(94.96079547,341.28725963)(94.97079546,341.37225954)(94.97080404,341.48226336)
\curveto(94.97079546,341.59225932)(94.96079547,341.67725924)(94.94080404,341.73726336)
\moveto(92.84080404,341.19726336)
\curveto(92.8307976,341.14725977)(92.82579761,341.09725982)(92.82580404,341.04726336)
\lineto(92.82580404,340.91226336)
\curveto(92.81579762,340.87226004)(92.81079762,340.83226008)(92.81080404,340.79226336)
\curveto(92.81079762,340.76226015)(92.80579763,340.72726019)(92.79580404,340.68726336)
\curveto(92.76579767,340.57726034)(92.74079769,340.47226044)(92.72080404,340.37226336)
\curveto(92.70079773,340.27226064)(92.67079776,340.17226074)(92.63080404,340.07226336)
\curveto(92.52079791,339.82226109)(92.38579805,339.6122613)(92.22580404,339.44226336)
\curveto(92.06579837,339.27226164)(91.85579858,339.13726178)(91.59580404,339.03726336)
\curveto(91.52579891,339.00726191)(91.45079898,338.98726193)(91.37080404,338.97726336)
\curveto(91.29079914,338.96726195)(91.21079922,338.95226196)(91.13080404,338.93226336)
\lineto(91.01080404,338.93226336)
\curveto(90.97079946,338.92226199)(90.92579951,338.917262)(90.87580404,338.91726336)
\lineto(90.75580404,338.94726336)
\curveto(90.71579972,338.95726196)(90.68079975,338.95726196)(90.65080404,338.94726336)
\curveto(90.62079981,338.94726197)(90.58579985,338.95226196)(90.54580404,338.96226336)
\curveto(90.45579998,338.98226193)(90.36580007,339.00726191)(90.27580404,339.03726336)
\curveto(90.19580024,339.06726185)(90.12080031,339.10726181)(90.05080404,339.15726336)
\curveto(89.80080063,339.30726161)(89.61580082,339.47226144)(89.49580404,339.65226336)
\curveto(89.38580105,339.84226107)(89.28080115,340.08726083)(89.18080404,340.38726336)
\curveto(89.16080127,340.46726045)(89.14580129,340.54226037)(89.13580404,340.61226336)
\curveto(89.12580131,340.69226022)(89.11080132,340.77226014)(89.09080404,340.85226336)
\lineto(89.09080404,340.98726336)
\curveto(89.07080136,341.05725986)(89.05580138,341.16225975)(89.04580404,341.30226336)
\curveto(89.04580139,341.44225947)(89.05580138,341.54725937)(89.07580404,341.61726336)
\lineto(89.07580404,341.76726336)
\curveto(89.07580136,341.8172591)(89.08080135,341.86725905)(89.09080404,341.91726336)
\curveto(89.11080132,342.02725889)(89.12580131,342.13725878)(89.13580404,342.24726336)
\curveto(89.15580128,342.35725856)(89.18080125,342.46225845)(89.21080404,342.56226336)
\curveto(89.30080113,342.83225808)(89.42080101,343.06725785)(89.57080404,343.26726336)
\curveto(89.7308007,343.47725744)(89.9358005,343.63725728)(90.18580404,343.74726336)
\curveto(90.2358002,343.77725714)(90.29080014,343.79725712)(90.35080404,343.80726336)
\lineto(90.56080404,343.86726336)
\curveto(90.59079984,343.87725704)(90.62579981,343.87725704)(90.66580404,343.86726336)
\curveto(90.70579973,343.86725705)(90.74079969,343.87725704)(90.77080404,343.89726336)
\lineto(91.04080404,343.89726336)
\curveto(91.1307993,343.90725701)(91.21579922,343.90225701)(91.29580404,343.88226336)
\curveto(91.36579907,343.86225705)(91.430799,343.84225707)(91.49080404,343.82226336)
\curveto(91.55079888,343.8122571)(91.61079882,343.79725712)(91.67080404,343.77726336)
\curveto(91.92079851,343.66725725)(92.12079831,343.5172574)(92.27080404,343.32726336)
\curveto(92.42079801,343.14725777)(92.55079788,342.92725799)(92.66080404,342.66726336)
\curveto(92.69079774,342.58725833)(92.71079772,342.50225841)(92.72080404,342.41226336)
\lineto(92.78080404,342.17226336)
\curveto(92.79079764,342.15225876)(92.79579764,342.12225879)(92.79580404,342.08226336)
\curveto(92.80579763,342.03225888)(92.81079762,341.97725894)(92.81080404,341.91726336)
\curveto(92.81079762,341.85725906)(92.82079761,341.80225911)(92.84080404,341.75226336)
\lineto(92.84080404,341.63226336)
\curveto(92.85079758,341.58225933)(92.85579758,341.50725941)(92.85580404,341.40726336)
\curveto(92.85579758,341.3172596)(92.85079758,341.24725967)(92.84080404,341.19726336)
\moveto(91.61080404,348.36726336)
\lineto(92.67580404,348.36726336)
\curveto(92.75579768,348.36725255)(92.85079758,348.36725255)(92.96080404,348.36726336)
\curveto(93.07079736,348.36725255)(93.15079728,348.35225256)(93.20080404,348.32226336)
\curveto(93.22079721,348.3122526)(93.2307972,348.29725262)(93.23080404,348.27726336)
\curveto(93.24079719,348.26725265)(93.25579718,348.25725266)(93.27580404,348.24726336)
\curveto(93.28579715,348.12725279)(93.2357972,348.02225289)(93.12580404,347.93226336)
\curveto(93.02579741,347.84225307)(92.94079749,347.76225315)(92.87080404,347.69226336)
\curveto(92.79079764,347.62225329)(92.71079772,347.54725337)(92.63080404,347.46726336)
\curveto(92.56079787,347.39725352)(92.48579795,347.33225358)(92.40580404,347.27226336)
\curveto(92.36579807,347.24225367)(92.3307981,347.20725371)(92.30080404,347.16726336)
\curveto(92.28079815,347.13725378)(92.25079818,347.1122538)(92.21080404,347.09226336)
\curveto(92.19079824,347.06225385)(92.16579827,347.03725388)(92.13580404,347.01726336)
\lineto(91.98580404,346.86726336)
\lineto(91.83580404,346.74726336)
\lineto(91.79080404,346.70226336)
\curveto(91.79079864,346.69225422)(91.78079865,346.67725424)(91.76080404,346.65726336)
\curveto(91.68079875,346.59725432)(91.60079883,346.53225438)(91.52080404,346.46226336)
\curveto(91.45079898,346.39225452)(91.36079907,346.33725458)(91.25080404,346.29726336)
\curveto(91.21079922,346.28725463)(91.17079926,346.28225463)(91.13080404,346.28226336)
\curveto(91.10079933,346.28225463)(91.06079937,346.27725464)(91.01080404,346.26726336)
\curveto(90.98079945,346.25725466)(90.94079949,346.25225466)(90.89080404,346.25226336)
\curveto(90.84079959,346.26225465)(90.79579964,346.26725465)(90.75580404,346.26726336)
\lineto(90.41080404,346.26726336)
\curveto(90.29080014,346.26725465)(90.20080023,346.29225462)(90.14080404,346.34226336)
\curveto(90.08080035,346.38225453)(90.06580037,346.45225446)(90.09580404,346.55226336)
\curveto(90.11580032,346.63225428)(90.15080028,346.70225421)(90.20080404,346.76226336)
\curveto(90.25080018,346.83225408)(90.29580014,346.90225401)(90.33580404,346.97226336)
\curveto(90.4358,347.1122538)(90.5307999,347.24725367)(90.62080404,347.37726336)
\curveto(90.71079972,347.50725341)(90.80079963,347.64225327)(90.89080404,347.78226336)
\curveto(90.94079949,347.86225305)(90.99079944,347.94725297)(91.04080404,348.03726336)
\curveto(91.10079933,348.12725279)(91.16579927,348.19725272)(91.23580404,348.24726336)
\curveto(91.27579916,348.27725264)(91.34579909,348.3122526)(91.44580404,348.35226336)
\curveto(91.46579897,348.36225255)(91.49079894,348.36225255)(91.52080404,348.35226336)
\curveto(91.56079887,348.35225256)(91.59079884,348.35725256)(91.61080404,348.36726336)
}
}
{
\newrgbcolor{curcolor}{0 0 0}
\pscustom[linestyle=none,fillstyle=solid,fillcolor=curcolor]
{
\newpath
\moveto(100.76572592,345.48726336)
\curveto(101.36572011,345.50725541)(101.86571961,345.42225549)(102.26572592,345.23226336)
\curveto(102.66571881,345.04225587)(102.9807185,344.76225615)(103.21072592,344.39226336)
\curveto(103.2807182,344.28225663)(103.33571814,344.16225675)(103.37572592,344.03226336)
\curveto(103.41571806,343.912257)(103.45571802,343.78725713)(103.49572592,343.65726336)
\curveto(103.51571796,343.57725734)(103.52571795,343.50225741)(103.52572592,343.43226336)
\curveto(103.53571794,343.36225755)(103.55071793,343.29225762)(103.57072592,343.22226336)
\curveto(103.57071791,343.16225775)(103.5757179,343.12225779)(103.58572592,343.10226336)
\curveto(103.60571787,342.96225795)(103.61571786,342.8172581)(103.61572592,342.66726336)
\lineto(103.61572592,342.23226336)
\lineto(103.61572592,340.89726336)
\lineto(103.61572592,338.46726336)
\curveto(103.61571786,338.27726264)(103.61071787,338.09226282)(103.60072592,337.91226336)
\curveto(103.60071788,337.74226317)(103.53071795,337.63226328)(103.39072592,337.58226336)
\curveto(103.33071815,337.56226335)(103.26071822,337.55226336)(103.18072592,337.55226336)
\lineto(102.94072592,337.55226336)
\lineto(102.13072592,337.55226336)
\curveto(102.01071947,337.55226336)(101.90071958,337.55726336)(101.80072592,337.56726336)
\curveto(101.71071977,337.58726333)(101.64071984,337.63226328)(101.59072592,337.70226336)
\curveto(101.55071993,337.76226315)(101.52571995,337.83726308)(101.51572592,337.92726336)
\lineto(101.51572592,338.24226336)
\lineto(101.51572592,339.29226336)
\lineto(101.51572592,341.52726336)
\curveto(101.51571996,341.89725902)(101.50071998,342.23725868)(101.47072592,342.54726336)
\curveto(101.44072004,342.86725805)(101.35072013,343.13725778)(101.20072592,343.35726336)
\curveto(101.06072042,343.55725736)(100.85572062,343.69725722)(100.58572592,343.77726336)
\curveto(100.53572094,343.79725712)(100.480721,343.80725711)(100.42072592,343.80726336)
\curveto(100.37072111,343.80725711)(100.31572116,343.8172571)(100.25572592,343.83726336)
\curveto(100.20572127,343.84725707)(100.14072134,343.84725707)(100.06072592,343.83726336)
\curveto(99.99072149,343.83725708)(99.93572154,343.83225708)(99.89572592,343.82226336)
\curveto(99.85572162,343.8122571)(99.82072166,343.80725711)(99.79072592,343.80726336)
\curveto(99.76072172,343.80725711)(99.73072175,343.80225711)(99.70072592,343.79226336)
\curveto(99.47072201,343.73225718)(99.28572219,343.65225726)(99.14572592,343.55226336)
\curveto(98.82572265,343.32225759)(98.63572284,342.98725793)(98.57572592,342.54726336)
\curveto(98.51572296,342.10725881)(98.48572299,341.6122593)(98.48572592,341.06226336)
\lineto(98.48572592,339.18726336)
\lineto(98.48572592,338.27226336)
\lineto(98.48572592,338.00226336)
\curveto(98.48572299,337.912263)(98.47072301,337.83726308)(98.44072592,337.77726336)
\curveto(98.39072309,337.66726325)(98.31072317,337.60226331)(98.20072592,337.58226336)
\curveto(98.09072339,337.56226335)(97.95572352,337.55226336)(97.79572592,337.55226336)
\lineto(97.04572592,337.55226336)
\curveto(96.93572454,337.55226336)(96.82572465,337.55726336)(96.71572592,337.56726336)
\curveto(96.60572487,337.57726334)(96.52572495,337.6122633)(96.47572592,337.67226336)
\curveto(96.40572507,337.76226315)(96.37072511,337.89226302)(96.37072592,338.06226336)
\curveto(96.3807251,338.23226268)(96.38572509,338.39226252)(96.38572592,338.54226336)
\lineto(96.38572592,340.58226336)
\lineto(96.38572592,343.88226336)
\lineto(96.38572592,344.64726336)
\lineto(96.38572592,344.94726336)
\curveto(96.39572508,345.03725588)(96.42572505,345.1122558)(96.47572592,345.17226336)
\curveto(96.49572498,345.20225571)(96.52572495,345.22225569)(96.56572592,345.23226336)
\curveto(96.61572486,345.25225566)(96.66572481,345.26725565)(96.71572592,345.27726336)
\lineto(96.79072592,345.27726336)
\curveto(96.84072464,345.28725563)(96.89072459,345.29225562)(96.94072592,345.29226336)
\lineto(97.10572592,345.29226336)
\lineto(97.73572592,345.29226336)
\curveto(97.81572366,345.29225562)(97.89072359,345.28725563)(97.96072592,345.27726336)
\curveto(98.04072344,345.27725564)(98.11072337,345.26725565)(98.17072592,345.24726336)
\curveto(98.24072324,345.2172557)(98.28572319,345.17225574)(98.30572592,345.11226336)
\curveto(98.33572314,345.05225586)(98.36072312,344.98225593)(98.38072592,344.90226336)
\curveto(98.39072309,344.86225605)(98.39072309,344.82725609)(98.38072592,344.79726336)
\curveto(98.3807231,344.76725615)(98.39072309,344.73725618)(98.41072592,344.70726336)
\curveto(98.43072305,344.65725626)(98.44572303,344.62725629)(98.45572592,344.61726336)
\curveto(98.475723,344.60725631)(98.50072298,344.59225632)(98.53072592,344.57226336)
\curveto(98.64072284,344.56225635)(98.73072275,344.59725632)(98.80072592,344.67726336)
\curveto(98.87072261,344.76725615)(98.94572253,344.83725608)(99.02572592,344.88726336)
\curveto(99.29572218,345.08725583)(99.59572188,345.24725567)(99.92572592,345.36726336)
\curveto(100.01572146,345.39725552)(100.10572137,345.4172555)(100.19572592,345.42726336)
\curveto(100.29572118,345.43725548)(100.40072108,345.45225546)(100.51072592,345.47226336)
\curveto(100.54072094,345.48225543)(100.58572089,345.48225543)(100.64572592,345.47226336)
\curveto(100.70572077,345.47225544)(100.74572073,345.47725544)(100.76572592,345.48726336)
}
}
{
\newrgbcolor{curcolor}{0 0 0}
\pscustom[linestyle=none,fillstyle=solid,fillcolor=curcolor]
{
}
}
{
\newrgbcolor{curcolor}{0 0 0}
\pscustom[linestyle=none,fillstyle=solid,fillcolor=curcolor]
{
\newpath
\moveto(116.99713217,338.40726336)
\lineto(116.99713217,337.98726336)
\curveto(116.9971238,337.85726306)(116.96712383,337.75226316)(116.90713217,337.67226336)
\curveto(116.85712394,337.62226329)(116.792124,337.58726333)(116.71213217,337.56726336)
\curveto(116.63212416,337.55726336)(116.54212425,337.55226336)(116.44213217,337.55226336)
\lineto(115.61713217,337.55226336)
\lineto(115.33213217,337.55226336)
\curveto(115.25212554,337.56226335)(115.18712561,337.58726333)(115.13713217,337.62726336)
\curveto(115.06712573,337.67726324)(115.02712577,337.74226317)(115.01713217,337.82226336)
\curveto(115.00712579,337.90226301)(114.98712581,337.98226293)(114.95713217,338.06226336)
\curveto(114.93712586,338.08226283)(114.91712588,338.09726282)(114.89713217,338.10726336)
\curveto(114.88712591,338.12726279)(114.87212592,338.14726277)(114.85213217,338.16726336)
\curveto(114.74212605,338.16726275)(114.66212613,338.14226277)(114.61213217,338.09226336)
\lineto(114.46213217,337.94226336)
\curveto(114.3921264,337.89226302)(114.32712647,337.84726307)(114.26713217,337.80726336)
\curveto(114.20712659,337.77726314)(114.14212665,337.73726318)(114.07213217,337.68726336)
\curveto(114.03212676,337.66726325)(113.98712681,337.64726327)(113.93713217,337.62726336)
\curveto(113.8971269,337.60726331)(113.85212694,337.58726333)(113.80213217,337.56726336)
\curveto(113.66212713,337.5172634)(113.51212728,337.47226344)(113.35213217,337.43226336)
\curveto(113.30212749,337.4122635)(113.25712754,337.40226351)(113.21713217,337.40226336)
\curveto(113.17712762,337.40226351)(113.13712766,337.39726352)(113.09713217,337.38726336)
\lineto(112.96213217,337.38726336)
\curveto(112.93212786,337.37726354)(112.8921279,337.37226354)(112.84213217,337.37226336)
\lineto(112.70713217,337.37226336)
\curveto(112.64712815,337.35226356)(112.55712824,337.34726357)(112.43713217,337.35726336)
\curveto(112.31712848,337.35726356)(112.23212856,337.36726355)(112.18213217,337.38726336)
\curveto(112.11212868,337.40726351)(112.04712875,337.4172635)(111.98713217,337.41726336)
\curveto(111.93712886,337.40726351)(111.88212891,337.4122635)(111.82213217,337.43226336)
\lineto(111.46213217,337.55226336)
\curveto(111.35212944,337.58226333)(111.24212955,337.62226329)(111.13213217,337.67226336)
\curveto(110.78213001,337.82226309)(110.46713033,338.05226286)(110.18713217,338.36226336)
\curveto(109.91713088,338.68226223)(109.70213109,339.0172619)(109.54213217,339.36726336)
\curveto(109.4921313,339.47726144)(109.45213134,339.58226133)(109.42213217,339.68226336)
\curveto(109.3921314,339.79226112)(109.35713144,339.90226101)(109.31713217,340.01226336)
\curveto(109.30713149,340.05226086)(109.30213149,340.08726083)(109.30213217,340.11726336)
\curveto(109.30213149,340.15726076)(109.2921315,340.20226071)(109.27213217,340.25226336)
\curveto(109.25213154,340.33226058)(109.23213156,340.4172605)(109.21213217,340.50726336)
\curveto(109.20213159,340.60726031)(109.18713161,340.70726021)(109.16713217,340.80726336)
\curveto(109.15713164,340.83726008)(109.15213164,340.87226004)(109.15213217,340.91226336)
\curveto(109.16213163,340.95225996)(109.16213163,340.98725993)(109.15213217,341.01726336)
\lineto(109.15213217,341.15226336)
\curveto(109.15213164,341.20225971)(109.14713165,341.25225966)(109.13713217,341.30226336)
\curveto(109.12713167,341.35225956)(109.12213167,341.40725951)(109.12213217,341.46726336)
\curveto(109.12213167,341.53725938)(109.12713167,341.59225932)(109.13713217,341.63226336)
\curveto(109.14713165,341.68225923)(109.15213164,341.72725919)(109.15213217,341.76726336)
\lineto(109.15213217,341.91726336)
\curveto(109.16213163,341.96725895)(109.16213163,342.0122589)(109.15213217,342.05226336)
\curveto(109.15213164,342.10225881)(109.16213163,342.15225876)(109.18213217,342.20226336)
\curveto(109.20213159,342.3122586)(109.21713158,342.4172585)(109.22713217,342.51726336)
\curveto(109.24713155,342.6172583)(109.27213152,342.7172582)(109.30213217,342.81726336)
\curveto(109.34213145,342.93725798)(109.37713142,343.05225786)(109.40713217,343.16226336)
\curveto(109.43713136,343.27225764)(109.47713132,343.38225753)(109.52713217,343.49226336)
\curveto(109.66713113,343.79225712)(109.84213095,344.07725684)(110.05213217,344.34726336)
\curveto(110.07213072,344.37725654)(110.0971307,344.40225651)(110.12713217,344.42226336)
\curveto(110.16713063,344.45225646)(110.1971306,344.48225643)(110.21713217,344.51226336)
\curveto(110.25713054,344.56225635)(110.2971305,344.60725631)(110.33713217,344.64726336)
\curveto(110.37713042,344.68725623)(110.42213037,344.72725619)(110.47213217,344.76726336)
\curveto(110.51213028,344.78725613)(110.54713025,344.8122561)(110.57713217,344.84226336)
\curveto(110.60713019,344.88225603)(110.64213015,344.912256)(110.68213217,344.93226336)
\curveto(110.93212986,345.10225581)(111.22212957,345.24225567)(111.55213217,345.35226336)
\curveto(111.62212917,345.37225554)(111.6921291,345.38725553)(111.76213217,345.39726336)
\curveto(111.84212895,345.40725551)(111.92212887,345.42225549)(112.00213217,345.44226336)
\curveto(112.07212872,345.46225545)(112.16212863,345.47225544)(112.27213217,345.47226336)
\curveto(112.38212841,345.48225543)(112.4921283,345.48725543)(112.60213217,345.48726336)
\curveto(112.71212808,345.48725543)(112.81712798,345.48225543)(112.91713217,345.47226336)
\curveto(113.02712777,345.46225545)(113.11712768,345.44725547)(113.18713217,345.42726336)
\curveto(113.33712746,345.37725554)(113.48212731,345.33225558)(113.62213217,345.29226336)
\curveto(113.76212703,345.25225566)(113.8921269,345.19725572)(114.01213217,345.12726336)
\curveto(114.08212671,345.07725584)(114.14712665,345.02725589)(114.20713217,344.97726336)
\curveto(114.26712653,344.93725598)(114.33212646,344.89225602)(114.40213217,344.84226336)
\curveto(114.44212635,344.8122561)(114.4971263,344.77225614)(114.56713217,344.72226336)
\curveto(114.64712615,344.67225624)(114.72212607,344.67225624)(114.79213217,344.72226336)
\curveto(114.83212596,344.74225617)(114.85212594,344.77725614)(114.85213217,344.82726336)
\curveto(114.85212594,344.87725604)(114.86212593,344.92725599)(114.88213217,344.97726336)
\lineto(114.88213217,345.12726336)
\curveto(114.8921259,345.15725576)(114.8971259,345.19225572)(114.89713217,345.23226336)
\lineto(114.89713217,345.35226336)
\lineto(114.89713217,347.39226336)
\curveto(114.8971259,347.50225341)(114.8921259,347.62225329)(114.88213217,347.75226336)
\curveto(114.88212591,347.89225302)(114.90712589,347.99725292)(114.95713217,348.06726336)
\curveto(114.9971258,348.14725277)(115.07212572,348.19725272)(115.18213217,348.21726336)
\curveto(115.20212559,348.22725269)(115.22212557,348.22725269)(115.24213217,348.21726336)
\curveto(115.26212553,348.2172527)(115.28212551,348.22225269)(115.30213217,348.23226336)
\lineto(116.36713217,348.23226336)
\curveto(116.48712431,348.23225268)(116.5971242,348.22725269)(116.69713217,348.21726336)
\curveto(116.797124,348.20725271)(116.87212392,348.16725275)(116.92213217,348.09726336)
\curveto(116.97212382,348.0172529)(116.9971238,347.912253)(116.99713217,347.78226336)
\lineto(116.99713217,347.42226336)
\lineto(116.99713217,338.40726336)
\moveto(114.95713217,341.34726336)
\curveto(114.96712583,341.38725953)(114.96712583,341.42725949)(114.95713217,341.46726336)
\lineto(114.95713217,341.60226336)
\curveto(114.95712584,341.70225921)(114.95212584,341.80225911)(114.94213217,341.90226336)
\curveto(114.93212586,342.00225891)(114.91712588,342.09225882)(114.89713217,342.17226336)
\curveto(114.87712592,342.28225863)(114.85712594,342.38225853)(114.83713217,342.47226336)
\curveto(114.82712597,342.56225835)(114.80212599,342.64725827)(114.76213217,342.72726336)
\curveto(114.62212617,343.08725783)(114.41712638,343.37225754)(114.14713217,343.58226336)
\curveto(113.88712691,343.79225712)(113.50712729,343.89725702)(113.00713217,343.89726336)
\curveto(112.94712785,343.89725702)(112.86712793,343.88725703)(112.76713217,343.86726336)
\curveto(112.68712811,343.84725707)(112.61212818,343.82725709)(112.54213217,343.80726336)
\curveto(112.48212831,343.79725712)(112.42212837,343.77725714)(112.36213217,343.74726336)
\curveto(112.0921287,343.63725728)(111.88212891,343.46725745)(111.73213217,343.23726336)
\curveto(111.58212921,343.00725791)(111.46212933,342.74725817)(111.37213217,342.45726336)
\curveto(111.34212945,342.35725856)(111.32212947,342.25725866)(111.31213217,342.15726336)
\curveto(111.30212949,342.05725886)(111.28212951,341.95225896)(111.25213217,341.84226336)
\lineto(111.25213217,341.63226336)
\curveto(111.23212956,341.54225937)(111.22712957,341.4172595)(111.23713217,341.25726336)
\curveto(111.24712955,341.10725981)(111.26212953,340.99725992)(111.28213217,340.92726336)
\lineto(111.28213217,340.83726336)
\curveto(111.2921295,340.8172601)(111.2971295,340.79726012)(111.29713217,340.77726336)
\curveto(111.31712948,340.69726022)(111.33212946,340.62226029)(111.34213217,340.55226336)
\curveto(111.36212943,340.48226043)(111.38212941,340.40726051)(111.40213217,340.32726336)
\curveto(111.57212922,339.80726111)(111.86212893,339.42226149)(112.27213217,339.17226336)
\curveto(112.40212839,339.08226183)(112.58212821,339.0122619)(112.81213217,338.96226336)
\curveto(112.85212794,338.95226196)(112.91212788,338.94726197)(112.99213217,338.94726336)
\curveto(113.02212777,338.93726198)(113.06712773,338.92726199)(113.12713217,338.91726336)
\curveto(113.1971276,338.917262)(113.25212754,338.92226199)(113.29213217,338.93226336)
\curveto(113.37212742,338.95226196)(113.45212734,338.96726195)(113.53213217,338.97726336)
\curveto(113.61212718,338.98726193)(113.6921271,339.00726191)(113.77213217,339.03726336)
\curveto(114.02212677,339.14726177)(114.22212657,339.28726163)(114.37213217,339.45726336)
\curveto(114.52212627,339.62726129)(114.65212614,339.84226107)(114.76213217,340.10226336)
\curveto(114.80212599,340.19226072)(114.83212596,340.28226063)(114.85213217,340.37226336)
\curveto(114.87212592,340.47226044)(114.8921259,340.57726034)(114.91213217,340.68726336)
\curveto(114.92212587,340.73726018)(114.92212587,340.78226013)(114.91213217,340.82226336)
\curveto(114.91212588,340.87226004)(114.92212587,340.92225999)(114.94213217,340.97226336)
\curveto(114.95212584,341.00225991)(114.95712584,341.03725988)(114.95713217,341.07726336)
\lineto(114.95713217,341.21226336)
\lineto(114.95713217,341.34726336)
}
}
{
\newrgbcolor{curcolor}{0 0 0}
\pscustom[linestyle=none,fillstyle=solid,fillcolor=curcolor]
{
\newpath
\moveto(125.94205404,341.49726336)
\curveto(125.96204588,341.4172595)(125.96204588,341.32725959)(125.94205404,341.22726336)
\curveto(125.92204592,341.12725979)(125.88704595,341.06225985)(125.83705404,341.03226336)
\curveto(125.78704605,340.99225992)(125.71204613,340.96225995)(125.61205404,340.94226336)
\curveto(125.52204632,340.93225998)(125.41704642,340.92225999)(125.29705404,340.91226336)
\lineto(124.95205404,340.91226336)
\curveto(124.842047,340.92225999)(124.7420471,340.92725999)(124.65205404,340.92726336)
\lineto(120.99205404,340.92726336)
\lineto(120.78205404,340.92726336)
\curveto(120.72205112,340.92725999)(120.66705117,340.91726)(120.61705404,340.89726336)
\curveto(120.5370513,340.85726006)(120.48705135,340.8172601)(120.46705404,340.77726336)
\curveto(120.44705139,340.75726016)(120.42705141,340.7172602)(120.40705404,340.65726336)
\curveto(120.38705145,340.60726031)(120.38205146,340.55726036)(120.39205404,340.50726336)
\curveto(120.41205143,340.44726047)(120.42205142,340.38726053)(120.42205404,340.32726336)
\curveto(120.43205141,340.27726064)(120.44705139,340.22226069)(120.46705404,340.16226336)
\curveto(120.54705129,339.92226099)(120.6420512,339.72226119)(120.75205404,339.56226336)
\curveto(120.87205097,339.4122615)(121.03205081,339.27726164)(121.23205404,339.15726336)
\curveto(121.31205053,339.10726181)(121.39205045,339.07226184)(121.47205404,339.05226336)
\curveto(121.56205028,339.04226187)(121.65205019,339.02226189)(121.74205404,338.99226336)
\curveto(121.82205002,338.97226194)(121.93204991,338.95726196)(122.07205404,338.94726336)
\curveto(122.21204963,338.93726198)(122.33204951,338.94226197)(122.43205404,338.96226336)
\lineto(122.56705404,338.96226336)
\curveto(122.66704917,338.98226193)(122.75704908,339.00226191)(122.83705404,339.02226336)
\curveto(122.92704891,339.05226186)(123.01204883,339.08226183)(123.09205404,339.11226336)
\curveto(123.19204865,339.16226175)(123.30204854,339.22726169)(123.42205404,339.30726336)
\curveto(123.55204829,339.38726153)(123.64704819,339.46726145)(123.70705404,339.54726336)
\curveto(123.75704808,339.6172613)(123.80704803,339.68226123)(123.85705404,339.74226336)
\curveto(123.91704792,339.8122611)(123.98704785,339.86226105)(124.06705404,339.89226336)
\curveto(124.16704767,339.94226097)(124.29204755,339.96226095)(124.44205404,339.95226336)
\lineto(124.87705404,339.95226336)
\lineto(125.05705404,339.95226336)
\curveto(125.12704671,339.96226095)(125.18704665,339.95726096)(125.23705404,339.93726336)
\lineto(125.38705404,339.93726336)
\curveto(125.48704635,339.917261)(125.55704628,339.89226102)(125.59705404,339.86226336)
\curveto(125.6370462,339.84226107)(125.65704618,339.79726112)(125.65705404,339.72726336)
\curveto(125.66704617,339.65726126)(125.66204618,339.59726132)(125.64205404,339.54726336)
\curveto(125.59204625,339.40726151)(125.5370463,339.28226163)(125.47705404,339.17226336)
\curveto(125.41704642,339.06226185)(125.34704649,338.95226196)(125.26705404,338.84226336)
\curveto(125.04704679,338.5122624)(124.79704704,338.24726267)(124.51705404,338.04726336)
\curveto(124.2370476,337.84726307)(123.88704795,337.67726324)(123.46705404,337.53726336)
\curveto(123.35704848,337.49726342)(123.24704859,337.47226344)(123.13705404,337.46226336)
\curveto(123.02704881,337.45226346)(122.91204893,337.43226348)(122.79205404,337.40226336)
\curveto(122.75204909,337.39226352)(122.70704913,337.39226352)(122.65705404,337.40226336)
\curveto(122.61704922,337.40226351)(122.57704926,337.39726352)(122.53705404,337.38726336)
\lineto(122.37205404,337.38726336)
\curveto(122.32204952,337.36726355)(122.26204958,337.36226355)(122.19205404,337.37226336)
\curveto(122.13204971,337.37226354)(122.07704976,337.37726354)(122.02705404,337.38726336)
\curveto(121.94704989,337.39726352)(121.87704996,337.39726352)(121.81705404,337.38726336)
\curveto(121.75705008,337.37726354)(121.69205015,337.38226353)(121.62205404,337.40226336)
\curveto(121.57205027,337.42226349)(121.51705032,337.43226348)(121.45705404,337.43226336)
\curveto(121.39705044,337.43226348)(121.3420505,337.44226347)(121.29205404,337.46226336)
\curveto(121.18205066,337.48226343)(121.07205077,337.50726341)(120.96205404,337.53726336)
\curveto(120.85205099,337.55726336)(120.75205109,337.59226332)(120.66205404,337.64226336)
\curveto(120.55205129,337.68226323)(120.44705139,337.7172632)(120.34705404,337.74726336)
\curveto(120.25705158,337.78726313)(120.17205167,337.83226308)(120.09205404,337.88226336)
\curveto(119.77205207,338.08226283)(119.48705235,338.3122626)(119.23705404,338.57226336)
\curveto(118.98705285,338.84226207)(118.78205306,339.15226176)(118.62205404,339.50226336)
\curveto(118.57205327,339.6122613)(118.53205331,339.72226119)(118.50205404,339.83226336)
\curveto(118.47205337,339.95226096)(118.43205341,340.07226084)(118.38205404,340.19226336)
\curveto(118.37205347,340.23226068)(118.36705347,340.26726065)(118.36705404,340.29726336)
\curveto(118.36705347,340.33726058)(118.36205348,340.37726054)(118.35205404,340.41726336)
\curveto(118.31205353,340.53726038)(118.28705355,340.66726025)(118.27705404,340.80726336)
\lineto(118.24705404,341.22726336)
\curveto(118.24705359,341.27725964)(118.2420536,341.33225958)(118.23205404,341.39226336)
\curveto(118.23205361,341.45225946)(118.2370536,341.50725941)(118.24705404,341.55726336)
\lineto(118.24705404,341.73726336)
\lineto(118.29205404,342.09726336)
\curveto(118.33205351,342.26725865)(118.36705347,342.43225848)(118.39705404,342.59226336)
\curveto(118.42705341,342.75225816)(118.47205337,342.90225801)(118.53205404,343.04226336)
\curveto(118.96205288,344.08225683)(119.69205215,344.8172561)(120.72205404,345.24726336)
\curveto(120.86205098,345.30725561)(121.00205084,345.34725557)(121.14205404,345.36726336)
\curveto(121.29205055,345.39725552)(121.44705039,345.43225548)(121.60705404,345.47226336)
\curveto(121.68705015,345.48225543)(121.76205008,345.48725543)(121.83205404,345.48726336)
\curveto(121.90204994,345.48725543)(121.97704986,345.49225542)(122.05705404,345.50226336)
\curveto(122.56704927,345.5122554)(123.00204884,345.45225546)(123.36205404,345.32226336)
\curveto(123.73204811,345.20225571)(124.06204778,345.04225587)(124.35205404,344.84226336)
\curveto(124.4420474,344.78225613)(124.53204731,344.7122562)(124.62205404,344.63226336)
\curveto(124.71204713,344.56225635)(124.79204705,344.48725643)(124.86205404,344.40726336)
\curveto(124.89204695,344.35725656)(124.93204691,344.3172566)(124.98205404,344.28726336)
\curveto(125.06204678,344.17725674)(125.1370467,344.06225685)(125.20705404,343.94226336)
\curveto(125.27704656,343.83225708)(125.35204649,343.7172572)(125.43205404,343.59726336)
\curveto(125.48204636,343.50725741)(125.52204632,343.4122575)(125.55205404,343.31226336)
\curveto(125.59204625,343.22225769)(125.63204621,343.12225779)(125.67205404,343.01226336)
\curveto(125.72204612,342.88225803)(125.76204608,342.74725817)(125.79205404,342.60726336)
\curveto(125.82204602,342.46725845)(125.85704598,342.32725859)(125.89705404,342.18726336)
\curveto(125.91704592,342.10725881)(125.92204592,342.0172589)(125.91205404,341.91726336)
\curveto(125.91204593,341.82725909)(125.92204592,341.74225917)(125.94205404,341.66226336)
\lineto(125.94205404,341.49726336)
\moveto(123.69205404,342.38226336)
\curveto(123.76204808,342.48225843)(123.76704807,342.60225831)(123.70705404,342.74226336)
\curveto(123.65704818,342.89225802)(123.61704822,343.00225791)(123.58705404,343.07226336)
\curveto(123.44704839,343.34225757)(123.26204858,343.54725737)(123.03205404,343.68726336)
\curveto(122.80204904,343.83725708)(122.48204936,343.917257)(122.07205404,343.92726336)
\curveto(122.0420498,343.90725701)(122.00704983,343.90225701)(121.96705404,343.91226336)
\curveto(121.92704991,343.92225699)(121.89204995,343.92225699)(121.86205404,343.91226336)
\curveto(121.81205003,343.89225702)(121.75705008,343.87725704)(121.69705404,343.86726336)
\curveto(121.6370502,343.86725705)(121.58205026,343.85725706)(121.53205404,343.83726336)
\curveto(121.09205075,343.69725722)(120.76705107,343.42225749)(120.55705404,343.01226336)
\curveto(120.5370513,342.97225794)(120.51205133,342.917258)(120.48205404,342.84726336)
\curveto(120.46205138,342.78725813)(120.44705139,342.72225819)(120.43705404,342.65226336)
\curveto(120.42705141,342.59225832)(120.42705141,342.53225838)(120.43705404,342.47226336)
\curveto(120.45705138,342.4122585)(120.49205135,342.36225855)(120.54205404,342.32226336)
\curveto(120.62205122,342.27225864)(120.73205111,342.24725867)(120.87205404,342.24726336)
\lineto(121.27705404,342.24726336)
\lineto(122.94205404,342.24726336)
\lineto(123.37705404,342.24726336)
\curveto(123.5370483,342.25725866)(123.6420482,342.30225861)(123.69205404,342.38226336)
}
}
{
\newrgbcolor{curcolor}{0 0 0}
\pscustom[linestyle=none,fillstyle=solid,fillcolor=curcolor]
{
}
}
{
\newrgbcolor{curcolor}{0 0 0}
\pscustom[linestyle=none,fillstyle=solid,fillcolor=curcolor]
{
\newpath
\moveto(131.86049154,348.24726336)
\lineto(132.95549154,348.24726336)
\curveto(133.05548906,348.24725267)(133.15048896,348.24225267)(133.24049154,348.23226336)
\curveto(133.33048878,348.22225269)(133.40048871,348.19225272)(133.45049154,348.14226336)
\curveto(133.5104886,348.07225284)(133.54048857,347.97725294)(133.54049154,347.85726336)
\curveto(133.55048856,347.74725317)(133.55548856,347.63225328)(133.55549154,347.51226336)
\lineto(133.55549154,346.17726336)
\lineto(133.55549154,340.79226336)
\lineto(133.55549154,338.49726336)
\lineto(133.55549154,338.07726336)
\curveto(133.56548855,337.92726299)(133.54548857,337.8122631)(133.49549154,337.73226336)
\curveto(133.44548867,337.65226326)(133.35548876,337.59726332)(133.22549154,337.56726336)
\curveto(133.16548895,337.54726337)(133.09548902,337.54226337)(133.01549154,337.55226336)
\curveto(132.94548917,337.56226335)(132.87548924,337.56726335)(132.80549154,337.56726336)
\lineto(132.08549154,337.56726336)
\curveto(131.97549014,337.56726335)(131.87549024,337.57226334)(131.78549154,337.58226336)
\curveto(131.69549042,337.59226332)(131.62049049,337.62226329)(131.56049154,337.67226336)
\curveto(131.50049061,337.72226319)(131.46549065,337.79726312)(131.45549154,337.89726336)
\lineto(131.45549154,338.22726336)
\lineto(131.45549154,339.56226336)
\lineto(131.45549154,345.18726336)
\lineto(131.45549154,347.22726336)
\curveto(131.45549066,347.35725356)(131.45049066,347.5122534)(131.44049154,347.69226336)
\curveto(131.44049067,347.87225304)(131.46549065,348.00225291)(131.51549154,348.08226336)
\curveto(131.53549058,348.12225279)(131.56049055,348.15225276)(131.59049154,348.17226336)
\lineto(131.71049154,348.23226336)
\curveto(131.73049038,348.23225268)(131.75549036,348.23225268)(131.78549154,348.23226336)
\curveto(131.8154903,348.24225267)(131.84049027,348.24725267)(131.86049154,348.24726336)
}
}
{
\newrgbcolor{curcolor}{0 0 0}
\pscustom[linestyle=none,fillstyle=solid,fillcolor=curcolor]
{
\newpath
\moveto(142.98767904,341.73726336)
\curveto(143.00767047,341.67725924)(143.01767046,341.59225932)(143.01767904,341.48226336)
\curveto(143.01767046,341.37225954)(143.00767047,341.28725963)(142.98767904,341.22726336)
\lineto(142.98767904,341.07726336)
\curveto(142.96767051,340.99725992)(142.95767052,340.91726)(142.95767904,340.83726336)
\curveto(142.96767051,340.75726016)(142.96267052,340.67726024)(142.94267904,340.59726336)
\curveto(142.92267056,340.52726039)(142.90767057,340.46226045)(142.89767904,340.40226336)
\curveto(142.88767059,340.34226057)(142.8776706,340.27726064)(142.86767904,340.20726336)
\curveto(142.82767065,340.09726082)(142.79267069,339.98226093)(142.76267904,339.86226336)
\curveto(142.73267075,339.75226116)(142.69267079,339.64726127)(142.64267904,339.54726336)
\curveto(142.43267105,339.06726185)(142.15767132,338.67726224)(141.81767904,338.37726336)
\curveto(141.477672,338.07726284)(141.06767241,337.82726309)(140.58767904,337.62726336)
\curveto(140.46767301,337.57726334)(140.34267314,337.54226337)(140.21267904,337.52226336)
\curveto(140.09267339,337.49226342)(139.96767351,337.46226345)(139.83767904,337.43226336)
\curveto(139.78767369,337.4122635)(139.73267375,337.40226351)(139.67267904,337.40226336)
\curveto(139.61267387,337.40226351)(139.55767392,337.39726352)(139.50767904,337.38726336)
\lineto(139.40267904,337.38726336)
\curveto(139.37267411,337.37726354)(139.34267414,337.37226354)(139.31267904,337.37226336)
\curveto(139.26267422,337.36226355)(139.1826743,337.35726356)(139.07267904,337.35726336)
\curveto(138.96267452,337.34726357)(138.8776746,337.35226356)(138.81767904,337.37226336)
\lineto(138.66767904,337.37226336)
\curveto(138.61767486,337.38226353)(138.56267492,337.38726353)(138.50267904,337.38726336)
\curveto(138.45267503,337.37726354)(138.40267508,337.38226353)(138.35267904,337.40226336)
\curveto(138.31267517,337.4122635)(138.27267521,337.4172635)(138.23267904,337.41726336)
\curveto(138.20267528,337.4172635)(138.16267532,337.42226349)(138.11267904,337.43226336)
\curveto(138.01267547,337.46226345)(137.91267557,337.48726343)(137.81267904,337.50726336)
\curveto(137.71267577,337.52726339)(137.61767586,337.55726336)(137.52767904,337.59726336)
\curveto(137.40767607,337.63726328)(137.29267619,337.67726324)(137.18267904,337.71726336)
\curveto(137.0826764,337.75726316)(136.9776765,337.80726311)(136.86767904,337.86726336)
\curveto(136.51767696,338.07726284)(136.21767726,338.32226259)(135.96767904,338.60226336)
\curveto(135.71767776,338.88226203)(135.50767797,339.2172617)(135.33767904,339.60726336)
\curveto(135.28767819,339.69726122)(135.24767823,339.79226112)(135.21767904,339.89226336)
\curveto(135.19767828,339.99226092)(135.17267831,340.09726082)(135.14267904,340.20726336)
\curveto(135.12267836,340.25726066)(135.11267837,340.30226061)(135.11267904,340.34226336)
\curveto(135.11267837,340.38226053)(135.10267838,340.42726049)(135.08267904,340.47726336)
\curveto(135.06267842,340.55726036)(135.05267843,340.63726028)(135.05267904,340.71726336)
\curveto(135.05267843,340.80726011)(135.04267844,340.89226002)(135.02267904,340.97226336)
\curveto(135.01267847,341.02225989)(135.00767847,341.06725985)(135.00767904,341.10726336)
\lineto(135.00767904,341.24226336)
\curveto(134.98767849,341.30225961)(134.9776785,341.38725953)(134.97767904,341.49726336)
\curveto(134.98767849,341.60725931)(135.00267848,341.69225922)(135.02267904,341.75226336)
\lineto(135.02267904,341.85726336)
\curveto(135.03267845,341.90725901)(135.03267845,341.95725896)(135.02267904,342.00726336)
\curveto(135.02267846,342.06725885)(135.03267845,342.12225879)(135.05267904,342.17226336)
\curveto(135.06267842,342.22225869)(135.06767841,342.26725865)(135.06767904,342.30726336)
\curveto(135.06767841,342.35725856)(135.0776784,342.40725851)(135.09767904,342.45726336)
\curveto(135.13767834,342.58725833)(135.17267831,342.7122582)(135.20267904,342.83226336)
\curveto(135.23267825,342.96225795)(135.27267821,343.08725783)(135.32267904,343.20726336)
\curveto(135.50267798,343.6172573)(135.71767776,343.95725696)(135.96767904,344.22726336)
\curveto(136.21767726,344.50725641)(136.52267696,344.76225615)(136.88267904,344.99226336)
\curveto(136.9826765,345.04225587)(137.08767639,345.08725583)(137.19767904,345.12726336)
\curveto(137.30767617,345.16725575)(137.41767606,345.2122557)(137.52767904,345.26226336)
\curveto(137.65767582,345.3122556)(137.79267569,345.34725557)(137.93267904,345.36726336)
\curveto(138.07267541,345.38725553)(138.21767526,345.4172555)(138.36767904,345.45726336)
\curveto(138.44767503,345.46725545)(138.52267496,345.47225544)(138.59267904,345.47226336)
\curveto(138.66267482,345.47225544)(138.73267475,345.47725544)(138.80267904,345.48726336)
\curveto(139.3826741,345.49725542)(139.8826736,345.43725548)(140.30267904,345.30726336)
\curveto(140.73267275,345.17725574)(141.11267237,344.99725592)(141.44267904,344.76726336)
\curveto(141.55267193,344.68725623)(141.66267182,344.59725632)(141.77267904,344.49726336)
\curveto(141.89267159,344.40725651)(141.99267149,344.30725661)(142.07267904,344.19726336)
\curveto(142.15267133,344.09725682)(142.22267126,343.99725692)(142.28267904,343.89726336)
\curveto(142.35267113,343.79725712)(142.42267106,343.69225722)(142.49267904,343.58226336)
\curveto(142.56267092,343.47225744)(142.61767086,343.35225756)(142.65767904,343.22226336)
\curveto(142.69767078,343.10225781)(142.74267074,342.97225794)(142.79267904,342.83226336)
\curveto(142.82267066,342.75225816)(142.84767063,342.66725825)(142.86767904,342.57726336)
\lineto(142.92767904,342.30726336)
\curveto(142.93767054,342.26725865)(142.94267054,342.22725869)(142.94267904,342.18726336)
\curveto(142.94267054,342.14725877)(142.94767053,342.10725881)(142.95767904,342.06726336)
\curveto(142.9776705,342.0172589)(142.9826705,341.96225895)(142.97267904,341.90226336)
\curveto(142.96267052,341.84225907)(142.96767051,341.78725913)(142.98767904,341.73726336)
\moveto(140.88767904,341.19726336)
\curveto(140.89767258,341.24725967)(140.90267258,341.3172596)(140.90267904,341.40726336)
\curveto(140.90267258,341.50725941)(140.89767258,341.58225933)(140.88767904,341.63226336)
\lineto(140.88767904,341.75226336)
\curveto(140.86767261,341.80225911)(140.85767262,341.85725906)(140.85767904,341.91726336)
\curveto(140.85767262,341.97725894)(140.85267263,342.03225888)(140.84267904,342.08226336)
\curveto(140.84267264,342.12225879)(140.83767264,342.15225876)(140.82767904,342.17226336)
\lineto(140.76767904,342.41226336)
\curveto(140.75767272,342.50225841)(140.73767274,342.58725833)(140.70767904,342.66726336)
\curveto(140.59767288,342.92725799)(140.46767301,343.14725777)(140.31767904,343.32726336)
\curveto(140.16767331,343.5172574)(139.96767351,343.66725725)(139.71767904,343.77726336)
\curveto(139.65767382,343.79725712)(139.59767388,343.8122571)(139.53767904,343.82226336)
\curveto(139.477674,343.84225707)(139.41267407,343.86225705)(139.34267904,343.88226336)
\curveto(139.26267422,343.90225701)(139.1776743,343.90725701)(139.08767904,343.89726336)
\lineto(138.81767904,343.89726336)
\curveto(138.78767469,343.87725704)(138.75267473,343.86725705)(138.71267904,343.86726336)
\curveto(138.67267481,343.87725704)(138.63767484,343.87725704)(138.60767904,343.86726336)
\lineto(138.39767904,343.80726336)
\curveto(138.33767514,343.79725712)(138.2826752,343.77725714)(138.23267904,343.74726336)
\curveto(137.9826755,343.63725728)(137.7776757,343.47725744)(137.61767904,343.26726336)
\curveto(137.46767601,343.06725785)(137.34767613,342.83225808)(137.25767904,342.56226336)
\curveto(137.22767625,342.46225845)(137.20267628,342.35725856)(137.18267904,342.24726336)
\curveto(137.17267631,342.13725878)(137.15767632,342.02725889)(137.13767904,341.91726336)
\curveto(137.12767635,341.86725905)(137.12267636,341.8172591)(137.12267904,341.76726336)
\lineto(137.12267904,341.61726336)
\curveto(137.10267638,341.54725937)(137.09267639,341.44225947)(137.09267904,341.30226336)
\curveto(137.10267638,341.16225975)(137.11767636,341.05725986)(137.13767904,340.98726336)
\lineto(137.13767904,340.85226336)
\curveto(137.15767632,340.77226014)(137.17267631,340.69226022)(137.18267904,340.61226336)
\curveto(137.19267629,340.54226037)(137.20767627,340.46726045)(137.22767904,340.38726336)
\curveto(137.32767615,340.08726083)(137.43267605,339.84226107)(137.54267904,339.65226336)
\curveto(137.66267582,339.47226144)(137.84767563,339.30726161)(138.09767904,339.15726336)
\curveto(138.16767531,339.10726181)(138.24267524,339.06726185)(138.32267904,339.03726336)
\curveto(138.41267507,339.00726191)(138.50267498,338.98226193)(138.59267904,338.96226336)
\curveto(138.63267485,338.95226196)(138.66767481,338.94726197)(138.69767904,338.94726336)
\curveto(138.72767475,338.95726196)(138.76267472,338.95726196)(138.80267904,338.94726336)
\lineto(138.92267904,338.91726336)
\curveto(138.97267451,338.917262)(139.01767446,338.92226199)(139.05767904,338.93226336)
\lineto(139.17767904,338.93226336)
\curveto(139.25767422,338.95226196)(139.33767414,338.96726195)(139.41767904,338.97726336)
\curveto(139.49767398,338.98726193)(139.57267391,339.00726191)(139.64267904,339.03726336)
\curveto(139.90267358,339.13726178)(140.11267337,339.27226164)(140.27267904,339.44226336)
\curveto(140.43267305,339.6122613)(140.56767291,339.82226109)(140.67767904,340.07226336)
\curveto(140.71767276,340.17226074)(140.74767273,340.27226064)(140.76767904,340.37226336)
\curveto(140.78767269,340.47226044)(140.81267267,340.57726034)(140.84267904,340.68726336)
\curveto(140.85267263,340.72726019)(140.85767262,340.76226015)(140.85767904,340.79226336)
\curveto(140.85767262,340.83226008)(140.86267262,340.87226004)(140.87267904,340.91226336)
\lineto(140.87267904,341.04726336)
\curveto(140.87267261,341.09725982)(140.8776726,341.14725977)(140.88767904,341.19726336)
}
}
{
\newrgbcolor{curcolor}{0 0 0}
\pscustom[linestyle=none,fillstyle=solid,fillcolor=curcolor]
{
\newpath
\moveto(147.35760092,345.50226336)
\curveto(148.10759642,345.52225539)(148.75759577,345.43725548)(149.30760092,345.24726336)
\curveto(149.86759466,345.06725585)(150.29259423,344.75225616)(150.58260092,344.30226336)
\curveto(150.65259387,344.19225672)(150.71259381,344.07725684)(150.76260092,343.95726336)
\curveto(150.8225937,343.84725707)(150.87259365,343.72225719)(150.91260092,343.58226336)
\curveto(150.93259359,343.52225739)(150.94259358,343.45725746)(150.94260092,343.38726336)
\curveto(150.94259358,343.3172576)(150.93259359,343.25725766)(150.91260092,343.20726336)
\curveto(150.87259365,343.14725777)(150.81759371,343.10725781)(150.74760092,343.08726336)
\curveto(150.69759383,343.06725785)(150.63759389,343.05725786)(150.56760092,343.05726336)
\lineto(150.35760092,343.05726336)
\lineto(149.69760092,343.05726336)
\curveto(149.6275949,343.05725786)(149.55759497,343.05225786)(149.48760092,343.04226336)
\curveto(149.41759511,343.04225787)(149.35259517,343.05225786)(149.29260092,343.07226336)
\curveto(149.19259533,343.09225782)(149.11759541,343.13225778)(149.06760092,343.19226336)
\curveto(149.01759551,343.25225766)(148.97259555,343.3122576)(148.93260092,343.37226336)
\lineto(148.81260092,343.58226336)
\curveto(148.78259574,343.66225725)(148.73259579,343.72725719)(148.66260092,343.77726336)
\curveto(148.56259596,343.85725706)(148.46259606,343.917257)(148.36260092,343.95726336)
\curveto(148.27259625,343.99725692)(148.15759637,344.03225688)(148.01760092,344.06226336)
\curveto(147.94759658,344.08225683)(147.84259668,344.09725682)(147.70260092,344.10726336)
\curveto(147.57259695,344.1172568)(147.47259705,344.1122568)(147.40260092,344.09226336)
\lineto(147.29760092,344.09226336)
\lineto(147.14760092,344.06226336)
\curveto(147.10759742,344.06225685)(147.06259746,344.05725686)(147.01260092,344.04726336)
\curveto(146.84259768,343.99725692)(146.70259782,343.92725699)(146.59260092,343.83726336)
\curveto(146.49259803,343.75725716)(146.4225981,343.63225728)(146.38260092,343.46226336)
\curveto(146.36259816,343.39225752)(146.36259816,343.32725759)(146.38260092,343.26726336)
\curveto(146.40259812,343.20725771)(146.4225981,343.15725776)(146.44260092,343.11726336)
\curveto(146.51259801,342.99725792)(146.59259793,342.90225801)(146.68260092,342.83226336)
\curveto(146.78259774,342.76225815)(146.89759763,342.70225821)(147.02760092,342.65226336)
\curveto(147.21759731,342.57225834)(147.4225971,342.50225841)(147.64260092,342.44226336)
\lineto(148.33260092,342.29226336)
\curveto(148.57259595,342.25225866)(148.80259572,342.20225871)(149.02260092,342.14226336)
\curveto(149.25259527,342.09225882)(149.46759506,342.02725889)(149.66760092,341.94726336)
\curveto(149.75759477,341.90725901)(149.84259468,341.87225904)(149.92260092,341.84226336)
\curveto(150.01259451,341.82225909)(150.09759443,341.78725913)(150.17760092,341.73726336)
\curveto(150.36759416,341.6172593)(150.53759399,341.48725943)(150.68760092,341.34726336)
\curveto(150.84759368,341.20725971)(150.97259355,341.03225988)(151.06260092,340.82226336)
\curveto(151.09259343,340.75226016)(151.11759341,340.68226023)(151.13760092,340.61226336)
\curveto(151.15759337,340.54226037)(151.17759335,340.46726045)(151.19760092,340.38726336)
\curveto(151.20759332,340.32726059)(151.21259331,340.23226068)(151.21260092,340.10226336)
\curveto(151.2225933,339.98226093)(151.2225933,339.88726103)(151.21260092,339.81726336)
\lineto(151.21260092,339.74226336)
\curveto(151.19259333,339.68226123)(151.17759335,339.62226129)(151.16760092,339.56226336)
\curveto(151.16759336,339.5122614)(151.16259336,339.46226145)(151.15260092,339.41226336)
\curveto(151.08259344,339.1122618)(150.97259355,338.84726207)(150.82260092,338.61726336)
\curveto(150.66259386,338.37726254)(150.46759406,338.18226273)(150.23760092,338.03226336)
\curveto(150.00759452,337.88226303)(149.74759478,337.75226316)(149.45760092,337.64226336)
\curveto(149.34759518,337.59226332)(149.2275953,337.55726336)(149.09760092,337.53726336)
\curveto(148.97759555,337.5172634)(148.85759567,337.49226342)(148.73760092,337.46226336)
\curveto(148.64759588,337.44226347)(148.55259597,337.43226348)(148.45260092,337.43226336)
\curveto(148.36259616,337.42226349)(148.27259625,337.40726351)(148.18260092,337.38726336)
\lineto(147.91260092,337.38726336)
\curveto(147.85259667,337.36726355)(147.74759678,337.35726356)(147.59760092,337.35726336)
\curveto(147.45759707,337.35726356)(147.35759717,337.36726355)(147.29760092,337.38726336)
\curveto(147.26759726,337.38726353)(147.23259729,337.39226352)(147.19260092,337.40226336)
\lineto(147.08760092,337.40226336)
\curveto(146.96759756,337.42226349)(146.84759768,337.43726348)(146.72760092,337.44726336)
\curveto(146.60759792,337.45726346)(146.49259803,337.47726344)(146.38260092,337.50726336)
\curveto(145.99259853,337.6172633)(145.64759888,337.74226317)(145.34760092,337.88226336)
\curveto(145.04759948,338.03226288)(144.79259973,338.25226266)(144.58260092,338.54226336)
\curveto(144.44260008,338.73226218)(144.3226002,338.95226196)(144.22260092,339.20226336)
\curveto(144.20260032,339.26226165)(144.18260034,339.34226157)(144.16260092,339.44226336)
\curveto(144.14260038,339.49226142)(144.1276004,339.56226135)(144.11760092,339.65226336)
\curveto(144.10760042,339.74226117)(144.11260041,339.8172611)(144.13260092,339.87726336)
\curveto(144.16260036,339.94726097)(144.21260031,339.99726092)(144.28260092,340.02726336)
\curveto(144.33260019,340.04726087)(144.39260013,340.05726086)(144.46260092,340.05726336)
\lineto(144.68760092,340.05726336)
\lineto(145.39260092,340.05726336)
\lineto(145.63260092,340.05726336)
\curveto(145.71259881,340.05726086)(145.78259874,340.04726087)(145.84260092,340.02726336)
\curveto(145.95259857,339.98726093)(146.0225985,339.92226099)(146.05260092,339.83226336)
\curveto(146.09259843,339.74226117)(146.13759839,339.64726127)(146.18760092,339.54726336)
\curveto(146.20759832,339.49726142)(146.24259828,339.43226148)(146.29260092,339.35226336)
\curveto(146.35259817,339.27226164)(146.40259812,339.22226169)(146.44260092,339.20226336)
\curveto(146.56259796,339.10226181)(146.67759785,339.02226189)(146.78760092,338.96226336)
\curveto(146.89759763,338.912262)(147.03759749,338.86226205)(147.20760092,338.81226336)
\curveto(147.25759727,338.79226212)(147.30759722,338.78226213)(147.35760092,338.78226336)
\curveto(147.40759712,338.79226212)(147.45759707,338.79226212)(147.50760092,338.78226336)
\curveto(147.58759694,338.76226215)(147.67259685,338.75226216)(147.76260092,338.75226336)
\curveto(147.86259666,338.76226215)(147.94759658,338.77726214)(148.01760092,338.79726336)
\curveto(148.06759646,338.80726211)(148.11259641,338.8122621)(148.15260092,338.81226336)
\curveto(148.20259632,338.8122621)(148.25259627,338.82226209)(148.30260092,338.84226336)
\curveto(148.44259608,338.89226202)(148.56759596,338.95226196)(148.67760092,339.02226336)
\curveto(148.79759573,339.09226182)(148.89259563,339.18226173)(148.96260092,339.29226336)
\curveto(149.01259551,339.37226154)(149.05259547,339.49726142)(149.08260092,339.66726336)
\curveto(149.10259542,339.73726118)(149.10259542,339.80226111)(149.08260092,339.86226336)
\curveto(149.06259546,339.92226099)(149.04259548,339.97226094)(149.02260092,340.01226336)
\curveto(148.95259557,340.15226076)(148.86259566,340.25726066)(148.75260092,340.32726336)
\curveto(148.65259587,340.39726052)(148.53259599,340.46226045)(148.39260092,340.52226336)
\curveto(148.20259632,340.60226031)(148.00259652,340.66726025)(147.79260092,340.71726336)
\curveto(147.58259694,340.76726015)(147.37259715,340.82226009)(147.16260092,340.88226336)
\curveto(147.08259744,340.90226001)(146.99759753,340.91726)(146.90760092,340.92726336)
\curveto(146.8275977,340.93725998)(146.74759778,340.95225996)(146.66760092,340.97226336)
\curveto(146.34759818,341.06225985)(146.04259848,341.14725977)(145.75260092,341.22726336)
\curveto(145.46259906,341.3172596)(145.19759933,341.44725947)(144.95760092,341.61726336)
\curveto(144.67759985,341.8172591)(144.47260005,342.08725883)(144.34260092,342.42726336)
\curveto(144.3226002,342.49725842)(144.30260022,342.59225832)(144.28260092,342.71226336)
\curveto(144.26260026,342.78225813)(144.24760028,342.86725805)(144.23760092,342.96726336)
\curveto(144.2276003,343.06725785)(144.23260029,343.15725776)(144.25260092,343.23726336)
\curveto(144.27260025,343.28725763)(144.27760025,343.32725759)(144.26760092,343.35726336)
\curveto(144.25760027,343.39725752)(144.26260026,343.44225747)(144.28260092,343.49226336)
\curveto(144.30260022,343.60225731)(144.3226002,343.70225721)(144.34260092,343.79226336)
\curveto(144.37260015,343.89225702)(144.40760012,343.98725693)(144.44760092,344.07726336)
\curveto(144.57759995,344.36725655)(144.75759977,344.60225631)(144.98760092,344.78226336)
\curveto(145.21759931,344.96225595)(145.47759905,345.10725581)(145.76760092,345.21726336)
\curveto(145.87759865,345.26725565)(145.99259853,345.30225561)(146.11260092,345.32226336)
\curveto(146.23259829,345.35225556)(146.35759817,345.38225553)(146.48760092,345.41226336)
\curveto(146.54759798,345.43225548)(146.60759792,345.44225547)(146.66760092,345.44226336)
\lineto(146.84760092,345.47226336)
\curveto(146.9275976,345.48225543)(147.01259751,345.48725543)(147.10260092,345.48726336)
\curveto(147.19259733,345.48725543)(147.27759725,345.49225542)(147.35760092,345.50226336)
}
}
{
\newrgbcolor{curcolor}{0 0 0}
\pscustom[linestyle=none,fillstyle=solid,fillcolor=curcolor]
{
}
}
{
\newrgbcolor{curcolor}{0 0 0}
\pscustom[linestyle=none,fillstyle=solid,fillcolor=curcolor]
{
\newpath
\moveto(160.17439779,345.50226336)
\curveto(160.98439263,345.52225539)(161.65939196,345.40225551)(162.19939779,345.14226336)
\curveto(162.74939087,344.88225603)(163.18439043,344.5122564)(163.50439779,344.03226336)
\curveto(163.66438995,343.79225712)(163.78438983,343.5172574)(163.86439779,343.20726336)
\curveto(163.88438973,343.15725776)(163.89938972,343.09225782)(163.90939779,343.01226336)
\curveto(163.92938969,342.93225798)(163.92938969,342.86225805)(163.90939779,342.80226336)
\curveto(163.86938975,342.69225822)(163.79938982,342.62725829)(163.69939779,342.60726336)
\curveto(163.59939002,342.59725832)(163.47939014,342.59225832)(163.33939779,342.59226336)
\lineto(162.55939779,342.59226336)
\lineto(162.27439779,342.59226336)
\curveto(162.18439143,342.59225832)(162.10939151,342.6122583)(162.04939779,342.65226336)
\curveto(161.96939165,342.69225822)(161.9143917,342.75225816)(161.88439779,342.83226336)
\curveto(161.85439176,342.92225799)(161.8143918,343.0122579)(161.76439779,343.10226336)
\curveto(161.70439191,343.2122577)(161.63939198,343.3122576)(161.56939779,343.40226336)
\curveto(161.49939212,343.49225742)(161.4193922,343.57225734)(161.32939779,343.64226336)
\curveto(161.18939243,343.73225718)(161.03439258,343.80225711)(160.86439779,343.85226336)
\curveto(160.80439281,343.87225704)(160.74439287,343.88225703)(160.68439779,343.88226336)
\curveto(160.62439299,343.88225703)(160.56939305,343.89225702)(160.51939779,343.91226336)
\lineto(160.36939779,343.91226336)
\curveto(160.16939345,343.912257)(160.00939361,343.89225702)(159.88939779,343.85226336)
\curveto(159.59939402,343.76225715)(159.36439425,343.62225729)(159.18439779,343.43226336)
\curveto(159.00439461,343.25225766)(158.85939476,343.03225788)(158.74939779,342.77226336)
\curveto(158.69939492,342.66225825)(158.65939496,342.54225837)(158.62939779,342.41226336)
\curveto(158.60939501,342.29225862)(158.58439503,342.16225875)(158.55439779,342.02226336)
\curveto(158.54439507,341.98225893)(158.53939508,341.94225897)(158.53939779,341.90226336)
\curveto(158.53939508,341.86225905)(158.53439508,341.82225909)(158.52439779,341.78226336)
\curveto(158.50439511,341.68225923)(158.49439512,341.54225937)(158.49439779,341.36226336)
\curveto(158.50439511,341.18225973)(158.5193951,341.04225987)(158.53939779,340.94226336)
\curveto(158.53939508,340.86226005)(158.54439507,340.80726011)(158.55439779,340.77726336)
\curveto(158.57439504,340.70726021)(158.58439503,340.63726028)(158.58439779,340.56726336)
\curveto(158.59439502,340.49726042)(158.60939501,340.42726049)(158.62939779,340.35726336)
\curveto(158.70939491,340.12726079)(158.80439481,339.917261)(158.91439779,339.72726336)
\curveto(159.02439459,339.53726138)(159.16439445,339.37726154)(159.33439779,339.24726336)
\curveto(159.37439424,339.2172617)(159.43439418,339.18226173)(159.51439779,339.14226336)
\curveto(159.62439399,339.07226184)(159.73439388,339.02726189)(159.84439779,339.00726336)
\curveto(159.96439365,338.98726193)(160.10939351,338.96726195)(160.27939779,338.94726336)
\lineto(160.36939779,338.94726336)
\curveto(160.40939321,338.94726197)(160.43939318,338.95226196)(160.45939779,338.96226336)
\lineto(160.59439779,338.96226336)
\curveto(160.66439295,338.98226193)(160.72939289,338.99726192)(160.78939779,339.00726336)
\curveto(160.85939276,339.02726189)(160.92439269,339.04726187)(160.98439779,339.06726336)
\curveto(161.28439233,339.19726172)(161.5143921,339.38726153)(161.67439779,339.63726336)
\curveto(161.7143919,339.68726123)(161.74939187,339.74226117)(161.77939779,339.80226336)
\curveto(161.80939181,339.87226104)(161.83439178,339.93226098)(161.85439779,339.98226336)
\curveto(161.89439172,340.09226082)(161.92939169,340.18726073)(161.95939779,340.26726336)
\curveto(161.98939163,340.35726056)(162.05939156,340.42726049)(162.16939779,340.47726336)
\curveto(162.25939136,340.5172604)(162.40439121,340.53226038)(162.60439779,340.52226336)
\lineto(163.09939779,340.52226336)
\lineto(163.30939779,340.52226336)
\curveto(163.38939023,340.53226038)(163.45439016,340.52726039)(163.50439779,340.50726336)
\lineto(163.62439779,340.50726336)
\lineto(163.74439779,340.47726336)
\curveto(163.78438983,340.47726044)(163.8143898,340.46726045)(163.83439779,340.44726336)
\curveto(163.88438973,340.40726051)(163.9143897,340.34726057)(163.92439779,340.26726336)
\curveto(163.94438967,340.19726072)(163.94438967,340.12226079)(163.92439779,340.04226336)
\curveto(163.83438978,339.7122612)(163.72438989,339.4172615)(163.59439779,339.15726336)
\curveto(163.18439043,338.38726253)(162.52939109,337.85226306)(161.62939779,337.55226336)
\curveto(161.52939209,337.52226339)(161.42439219,337.50226341)(161.31439779,337.49226336)
\curveto(161.20439241,337.47226344)(161.09439252,337.44726347)(160.98439779,337.41726336)
\curveto(160.92439269,337.40726351)(160.86439275,337.40226351)(160.80439779,337.40226336)
\curveto(160.74439287,337.40226351)(160.68439293,337.39726352)(160.62439779,337.38726336)
\lineto(160.45939779,337.38726336)
\curveto(160.40939321,337.36726355)(160.33439328,337.36226355)(160.23439779,337.37226336)
\curveto(160.13439348,337.37226354)(160.05939356,337.37726354)(160.00939779,337.38726336)
\curveto(159.92939369,337.40726351)(159.85439376,337.4172635)(159.78439779,337.41726336)
\curveto(159.72439389,337.40726351)(159.65939396,337.4122635)(159.58939779,337.43226336)
\lineto(159.43939779,337.46226336)
\curveto(159.38939423,337.46226345)(159.33939428,337.46726345)(159.28939779,337.47726336)
\curveto(159.17939444,337.50726341)(159.07439454,337.53726338)(158.97439779,337.56726336)
\curveto(158.87439474,337.59726332)(158.77939484,337.63226328)(158.68939779,337.67226336)
\curveto(158.2193954,337.87226304)(157.82439579,338.12726279)(157.50439779,338.43726336)
\curveto(157.18439643,338.75726216)(156.92439669,339.15226176)(156.72439779,339.62226336)
\curveto(156.67439694,339.7122612)(156.63439698,339.80726111)(156.60439779,339.90726336)
\lineto(156.51439779,340.23726336)
\curveto(156.50439711,340.27726064)(156.49939712,340.3122606)(156.49939779,340.34226336)
\curveto(156.49939712,340.38226053)(156.48939713,340.42726049)(156.46939779,340.47726336)
\curveto(156.44939717,340.54726037)(156.43939718,340.6172603)(156.43939779,340.68726336)
\curveto(156.43939718,340.76726015)(156.42939719,340.84226007)(156.40939779,340.91226336)
\lineto(156.40939779,341.16726336)
\curveto(156.38939723,341.2172597)(156.37939724,341.27225964)(156.37939779,341.33226336)
\curveto(156.37939724,341.40225951)(156.38939723,341.46225945)(156.40939779,341.51226336)
\curveto(156.4193972,341.56225935)(156.4193972,341.60725931)(156.40939779,341.64726336)
\curveto(156.39939722,341.68725923)(156.39939722,341.72725919)(156.40939779,341.76726336)
\curveto(156.42939719,341.83725908)(156.43439718,341.90225901)(156.42439779,341.96226336)
\curveto(156.42439719,342.02225889)(156.43439718,342.08225883)(156.45439779,342.14226336)
\curveto(156.50439711,342.32225859)(156.54439707,342.49225842)(156.57439779,342.65226336)
\curveto(156.60439701,342.82225809)(156.64939697,342.98725793)(156.70939779,343.14726336)
\curveto(156.92939669,343.65725726)(157.20439641,344.08225683)(157.53439779,344.42226336)
\curveto(157.87439574,344.76225615)(158.30439531,345.03725588)(158.82439779,345.24726336)
\curveto(158.96439465,345.30725561)(159.10939451,345.34725557)(159.25939779,345.36726336)
\curveto(159.40939421,345.39725552)(159.56439405,345.43225548)(159.72439779,345.47226336)
\curveto(159.80439381,345.48225543)(159.87939374,345.48725543)(159.94939779,345.48726336)
\curveto(160.0193936,345.48725543)(160.09439352,345.49225542)(160.17439779,345.50226336)
}
}
{
\newrgbcolor{curcolor}{0 0 0}
\pscustom[linestyle=none,fillstyle=solid,fillcolor=curcolor]
{
\newpath
\moveto(172.98767904,341.73726336)
\curveto(173.00767047,341.67725924)(173.01767046,341.59225932)(173.01767904,341.48226336)
\curveto(173.01767046,341.37225954)(173.00767047,341.28725963)(172.98767904,341.22726336)
\lineto(172.98767904,341.07726336)
\curveto(172.96767051,340.99725992)(172.95767052,340.91726)(172.95767904,340.83726336)
\curveto(172.96767051,340.75726016)(172.96267052,340.67726024)(172.94267904,340.59726336)
\curveto(172.92267056,340.52726039)(172.90767057,340.46226045)(172.89767904,340.40226336)
\curveto(172.88767059,340.34226057)(172.8776706,340.27726064)(172.86767904,340.20726336)
\curveto(172.82767065,340.09726082)(172.79267069,339.98226093)(172.76267904,339.86226336)
\curveto(172.73267075,339.75226116)(172.69267079,339.64726127)(172.64267904,339.54726336)
\curveto(172.43267105,339.06726185)(172.15767132,338.67726224)(171.81767904,338.37726336)
\curveto(171.477672,338.07726284)(171.06767241,337.82726309)(170.58767904,337.62726336)
\curveto(170.46767301,337.57726334)(170.34267314,337.54226337)(170.21267904,337.52226336)
\curveto(170.09267339,337.49226342)(169.96767351,337.46226345)(169.83767904,337.43226336)
\curveto(169.78767369,337.4122635)(169.73267375,337.40226351)(169.67267904,337.40226336)
\curveto(169.61267387,337.40226351)(169.55767392,337.39726352)(169.50767904,337.38726336)
\lineto(169.40267904,337.38726336)
\curveto(169.37267411,337.37726354)(169.34267414,337.37226354)(169.31267904,337.37226336)
\curveto(169.26267422,337.36226355)(169.1826743,337.35726356)(169.07267904,337.35726336)
\curveto(168.96267452,337.34726357)(168.8776746,337.35226356)(168.81767904,337.37226336)
\lineto(168.66767904,337.37226336)
\curveto(168.61767486,337.38226353)(168.56267492,337.38726353)(168.50267904,337.38726336)
\curveto(168.45267503,337.37726354)(168.40267508,337.38226353)(168.35267904,337.40226336)
\curveto(168.31267517,337.4122635)(168.27267521,337.4172635)(168.23267904,337.41726336)
\curveto(168.20267528,337.4172635)(168.16267532,337.42226349)(168.11267904,337.43226336)
\curveto(168.01267547,337.46226345)(167.91267557,337.48726343)(167.81267904,337.50726336)
\curveto(167.71267577,337.52726339)(167.61767586,337.55726336)(167.52767904,337.59726336)
\curveto(167.40767607,337.63726328)(167.29267619,337.67726324)(167.18267904,337.71726336)
\curveto(167.0826764,337.75726316)(166.9776765,337.80726311)(166.86767904,337.86726336)
\curveto(166.51767696,338.07726284)(166.21767726,338.32226259)(165.96767904,338.60226336)
\curveto(165.71767776,338.88226203)(165.50767797,339.2172617)(165.33767904,339.60726336)
\curveto(165.28767819,339.69726122)(165.24767823,339.79226112)(165.21767904,339.89226336)
\curveto(165.19767828,339.99226092)(165.17267831,340.09726082)(165.14267904,340.20726336)
\curveto(165.12267836,340.25726066)(165.11267837,340.30226061)(165.11267904,340.34226336)
\curveto(165.11267837,340.38226053)(165.10267838,340.42726049)(165.08267904,340.47726336)
\curveto(165.06267842,340.55726036)(165.05267843,340.63726028)(165.05267904,340.71726336)
\curveto(165.05267843,340.80726011)(165.04267844,340.89226002)(165.02267904,340.97226336)
\curveto(165.01267847,341.02225989)(165.00767847,341.06725985)(165.00767904,341.10726336)
\lineto(165.00767904,341.24226336)
\curveto(164.98767849,341.30225961)(164.9776785,341.38725953)(164.97767904,341.49726336)
\curveto(164.98767849,341.60725931)(165.00267848,341.69225922)(165.02267904,341.75226336)
\lineto(165.02267904,341.85726336)
\curveto(165.03267845,341.90725901)(165.03267845,341.95725896)(165.02267904,342.00726336)
\curveto(165.02267846,342.06725885)(165.03267845,342.12225879)(165.05267904,342.17226336)
\curveto(165.06267842,342.22225869)(165.06767841,342.26725865)(165.06767904,342.30726336)
\curveto(165.06767841,342.35725856)(165.0776784,342.40725851)(165.09767904,342.45726336)
\curveto(165.13767834,342.58725833)(165.17267831,342.7122582)(165.20267904,342.83226336)
\curveto(165.23267825,342.96225795)(165.27267821,343.08725783)(165.32267904,343.20726336)
\curveto(165.50267798,343.6172573)(165.71767776,343.95725696)(165.96767904,344.22726336)
\curveto(166.21767726,344.50725641)(166.52267696,344.76225615)(166.88267904,344.99226336)
\curveto(166.9826765,345.04225587)(167.08767639,345.08725583)(167.19767904,345.12726336)
\curveto(167.30767617,345.16725575)(167.41767606,345.2122557)(167.52767904,345.26226336)
\curveto(167.65767582,345.3122556)(167.79267569,345.34725557)(167.93267904,345.36726336)
\curveto(168.07267541,345.38725553)(168.21767526,345.4172555)(168.36767904,345.45726336)
\curveto(168.44767503,345.46725545)(168.52267496,345.47225544)(168.59267904,345.47226336)
\curveto(168.66267482,345.47225544)(168.73267475,345.47725544)(168.80267904,345.48726336)
\curveto(169.3826741,345.49725542)(169.8826736,345.43725548)(170.30267904,345.30726336)
\curveto(170.73267275,345.17725574)(171.11267237,344.99725592)(171.44267904,344.76726336)
\curveto(171.55267193,344.68725623)(171.66267182,344.59725632)(171.77267904,344.49726336)
\curveto(171.89267159,344.40725651)(171.99267149,344.30725661)(172.07267904,344.19726336)
\curveto(172.15267133,344.09725682)(172.22267126,343.99725692)(172.28267904,343.89726336)
\curveto(172.35267113,343.79725712)(172.42267106,343.69225722)(172.49267904,343.58226336)
\curveto(172.56267092,343.47225744)(172.61767086,343.35225756)(172.65767904,343.22226336)
\curveto(172.69767078,343.10225781)(172.74267074,342.97225794)(172.79267904,342.83226336)
\curveto(172.82267066,342.75225816)(172.84767063,342.66725825)(172.86767904,342.57726336)
\lineto(172.92767904,342.30726336)
\curveto(172.93767054,342.26725865)(172.94267054,342.22725869)(172.94267904,342.18726336)
\curveto(172.94267054,342.14725877)(172.94767053,342.10725881)(172.95767904,342.06726336)
\curveto(172.9776705,342.0172589)(172.9826705,341.96225895)(172.97267904,341.90226336)
\curveto(172.96267052,341.84225907)(172.96767051,341.78725913)(172.98767904,341.73726336)
\moveto(170.88767904,341.19726336)
\curveto(170.89767258,341.24725967)(170.90267258,341.3172596)(170.90267904,341.40726336)
\curveto(170.90267258,341.50725941)(170.89767258,341.58225933)(170.88767904,341.63226336)
\lineto(170.88767904,341.75226336)
\curveto(170.86767261,341.80225911)(170.85767262,341.85725906)(170.85767904,341.91726336)
\curveto(170.85767262,341.97725894)(170.85267263,342.03225888)(170.84267904,342.08226336)
\curveto(170.84267264,342.12225879)(170.83767264,342.15225876)(170.82767904,342.17226336)
\lineto(170.76767904,342.41226336)
\curveto(170.75767272,342.50225841)(170.73767274,342.58725833)(170.70767904,342.66726336)
\curveto(170.59767288,342.92725799)(170.46767301,343.14725777)(170.31767904,343.32726336)
\curveto(170.16767331,343.5172574)(169.96767351,343.66725725)(169.71767904,343.77726336)
\curveto(169.65767382,343.79725712)(169.59767388,343.8122571)(169.53767904,343.82226336)
\curveto(169.477674,343.84225707)(169.41267407,343.86225705)(169.34267904,343.88226336)
\curveto(169.26267422,343.90225701)(169.1776743,343.90725701)(169.08767904,343.89726336)
\lineto(168.81767904,343.89726336)
\curveto(168.78767469,343.87725704)(168.75267473,343.86725705)(168.71267904,343.86726336)
\curveto(168.67267481,343.87725704)(168.63767484,343.87725704)(168.60767904,343.86726336)
\lineto(168.39767904,343.80726336)
\curveto(168.33767514,343.79725712)(168.2826752,343.77725714)(168.23267904,343.74726336)
\curveto(167.9826755,343.63725728)(167.7776757,343.47725744)(167.61767904,343.26726336)
\curveto(167.46767601,343.06725785)(167.34767613,342.83225808)(167.25767904,342.56226336)
\curveto(167.22767625,342.46225845)(167.20267628,342.35725856)(167.18267904,342.24726336)
\curveto(167.17267631,342.13725878)(167.15767632,342.02725889)(167.13767904,341.91726336)
\curveto(167.12767635,341.86725905)(167.12267636,341.8172591)(167.12267904,341.76726336)
\lineto(167.12267904,341.61726336)
\curveto(167.10267638,341.54725937)(167.09267639,341.44225947)(167.09267904,341.30226336)
\curveto(167.10267638,341.16225975)(167.11767636,341.05725986)(167.13767904,340.98726336)
\lineto(167.13767904,340.85226336)
\curveto(167.15767632,340.77226014)(167.17267631,340.69226022)(167.18267904,340.61226336)
\curveto(167.19267629,340.54226037)(167.20767627,340.46726045)(167.22767904,340.38726336)
\curveto(167.32767615,340.08726083)(167.43267605,339.84226107)(167.54267904,339.65226336)
\curveto(167.66267582,339.47226144)(167.84767563,339.30726161)(168.09767904,339.15726336)
\curveto(168.16767531,339.10726181)(168.24267524,339.06726185)(168.32267904,339.03726336)
\curveto(168.41267507,339.00726191)(168.50267498,338.98226193)(168.59267904,338.96226336)
\curveto(168.63267485,338.95226196)(168.66767481,338.94726197)(168.69767904,338.94726336)
\curveto(168.72767475,338.95726196)(168.76267472,338.95726196)(168.80267904,338.94726336)
\lineto(168.92267904,338.91726336)
\curveto(168.97267451,338.917262)(169.01767446,338.92226199)(169.05767904,338.93226336)
\lineto(169.17767904,338.93226336)
\curveto(169.25767422,338.95226196)(169.33767414,338.96726195)(169.41767904,338.97726336)
\curveto(169.49767398,338.98726193)(169.57267391,339.00726191)(169.64267904,339.03726336)
\curveto(169.90267358,339.13726178)(170.11267337,339.27226164)(170.27267904,339.44226336)
\curveto(170.43267305,339.6122613)(170.56767291,339.82226109)(170.67767904,340.07226336)
\curveto(170.71767276,340.17226074)(170.74767273,340.27226064)(170.76767904,340.37226336)
\curveto(170.78767269,340.47226044)(170.81267267,340.57726034)(170.84267904,340.68726336)
\curveto(170.85267263,340.72726019)(170.85767262,340.76226015)(170.85767904,340.79226336)
\curveto(170.85767262,340.83226008)(170.86267262,340.87226004)(170.87267904,340.91226336)
\lineto(170.87267904,341.04726336)
\curveto(170.87267261,341.09725982)(170.8776726,341.14725977)(170.88767904,341.19726336)
}
}
{
\newrgbcolor{curcolor}{0 0 0}
\pscustom[linestyle=none,fillstyle=solid,fillcolor=curcolor]
{
\newpath
\moveto(178.84260092,345.48726336)
\curveto(179.21259531,345.49725542)(179.53759499,345.45725546)(179.81760092,345.36726336)
\curveto(180.09759443,345.27725564)(180.34259418,345.15225576)(180.55260092,344.99226336)
\curveto(180.63259389,344.93225598)(180.70259382,344.86225605)(180.76260092,344.78226336)
\curveto(180.83259369,344.70225621)(180.90759362,344.62225629)(180.98760092,344.54226336)
\curveto(181.00759352,344.52225639)(181.03759349,344.49225642)(181.07760092,344.45226336)
\curveto(181.1275934,344.42225649)(181.17759335,344.4172565)(181.22760092,344.43726336)
\curveto(181.33759319,344.46725645)(181.44259308,344.53725638)(181.54260092,344.64726336)
\curveto(181.64259288,344.76725615)(181.73759279,344.85725606)(181.82760092,344.91726336)
\curveto(181.96759256,345.02725589)(182.11759241,345.1172558)(182.27760092,345.18726336)
\curveto(182.43759209,345.26725565)(182.61759191,345.34225557)(182.81760092,345.41226336)
\curveto(182.89759163,345.43225548)(182.99259153,345.44725547)(183.10260092,345.45726336)
\curveto(183.2225913,345.47725544)(183.34259118,345.48725543)(183.46260092,345.48726336)
\curveto(183.59259093,345.49725542)(183.71259081,345.49725542)(183.82260092,345.48726336)
\curveto(183.94259058,345.47725544)(184.04759048,345.46225545)(184.13760092,345.44226336)
\curveto(184.18759034,345.43225548)(184.23259029,345.42725549)(184.27260092,345.42726336)
\curveto(184.31259021,345.42725549)(184.35759017,345.4172555)(184.40760092,345.39726336)
\curveto(184.54758998,345.35725556)(184.68258984,345.3172556)(184.81260092,345.27726336)
\curveto(184.94258958,345.23725568)(185.06258946,345.18225573)(185.17260092,345.11226336)
\curveto(185.59258893,344.85225606)(185.90758862,344.47225644)(186.11760092,343.97226336)
\curveto(186.15758837,343.88225703)(186.18758834,343.78725713)(186.20760092,343.68726336)
\curveto(186.2275883,343.59725732)(186.24758828,343.50725741)(186.26760092,343.41726336)
\curveto(186.27758825,343.34725757)(186.28258824,343.28225763)(186.28260092,343.22226336)
\curveto(186.29258823,343.16225775)(186.30258822,343.10225781)(186.31260092,343.04226336)
\lineto(186.31260092,342.89226336)
\curveto(186.3225882,342.83225808)(186.3225882,342.76225815)(186.31260092,342.68226336)
\curveto(186.31258821,342.60225831)(186.31258821,342.52725839)(186.31260092,342.45726336)
\lineto(186.31260092,341.58726336)
\lineto(186.31260092,338.66226336)
\curveto(186.31258821,338.58226233)(186.31258821,338.48726243)(186.31260092,338.37726336)
\curveto(186.3225882,338.27726264)(186.3225882,338.17726274)(186.31260092,338.07726336)
\curveto(186.31258821,337.98726293)(186.30258822,337.89726302)(186.28260092,337.80726336)
\curveto(186.26258826,337.72726319)(186.23258829,337.67226324)(186.19260092,337.64226336)
\curveto(186.13258839,337.59226332)(186.05258847,337.56226335)(185.95260092,337.55226336)
\lineto(185.65260092,337.55226336)
\lineto(184.85760092,337.55226336)
\curveto(184.71758981,337.55226336)(184.59258993,337.56226335)(184.48260092,337.58226336)
\curveto(184.37259015,337.60226331)(184.29759023,337.65726326)(184.25760092,337.74726336)
\curveto(184.2275903,337.8172631)(184.21259031,337.89226302)(184.21260092,337.97226336)
\curveto(184.21259031,338.06226285)(184.21259031,338.14726277)(184.21260092,338.22726336)
\lineto(184.21260092,339.06726336)
\lineto(184.21260092,341.09226336)
\lineto(184.21260092,341.72226336)
\curveto(184.21259031,341.77225914)(184.21259031,341.82725909)(184.21260092,341.88726336)
\curveto(184.2225903,341.94725897)(184.21759031,342.00225891)(184.19760092,342.05226336)
\lineto(184.19760092,342.17226336)
\curveto(184.19759033,342.23225868)(184.19759033,342.29225862)(184.19760092,342.35226336)
\curveto(184.19759033,342.4122585)(184.19259033,342.47225844)(184.18260092,342.53226336)
\curveto(184.17259035,342.57225834)(184.16759036,342.6122583)(184.16760092,342.65226336)
\curveto(184.16759036,342.70225821)(184.16259036,342.74725817)(184.15260092,342.78726336)
\curveto(184.11259041,342.93725798)(184.06759046,343.06725785)(184.01760092,343.17726336)
\curveto(183.97759055,343.29725762)(183.91259061,343.40225751)(183.82260092,343.49226336)
\curveto(183.68259084,343.63225728)(183.51259101,343.73225718)(183.31260092,343.79226336)
\curveto(183.27259125,343.80225711)(183.23759129,343.80225711)(183.20760092,343.79226336)
\curveto(183.17759135,343.79225712)(183.14259138,343.80225711)(183.10260092,343.82226336)
\curveto(183.06259146,343.83225708)(183.01259151,343.83725708)(182.95260092,343.83726336)
\curveto(182.90259162,343.84725707)(182.85259167,343.84725707)(182.80260092,343.83726336)
\curveto(182.74259178,343.8172571)(182.68259184,343.80725711)(182.62260092,343.80726336)
\curveto(182.56259196,343.80725711)(182.50259202,343.79725712)(182.44260092,343.77726336)
\curveto(182.15259237,343.67725724)(181.94259258,343.52725739)(181.81260092,343.32726336)
\curveto(181.64259288,343.09725782)(181.53759299,342.80725811)(181.49760092,342.45726336)
\curveto(181.46759306,342.1172588)(181.45259307,341.74225917)(181.45260092,341.33226336)
\lineto(181.45260092,339.35226336)
\lineto(181.45260092,338.24226336)
\lineto(181.45260092,337.94226336)
\curveto(181.45259307,337.84226307)(181.4275931,337.76226315)(181.37760092,337.70226336)
\curveto(181.3275932,337.63226328)(181.25259327,337.58726333)(181.15260092,337.56726336)
\curveto(181.06259346,337.55726336)(180.95759357,337.55226336)(180.83760092,337.55226336)
\lineto(180.02760092,337.55226336)
\lineto(179.75760092,337.55226336)
\curveto(179.67759485,337.56226335)(179.60759492,337.57726334)(179.54760092,337.59726336)
\curveto(179.44759508,337.64726327)(179.38759514,337.72726319)(179.36760092,337.83726336)
\curveto(179.35759517,337.94726297)(179.35259517,338.07226284)(179.35260092,338.21226336)
\lineto(179.35260092,339.48726336)
\lineto(179.35260092,341.84226336)
\curveto(179.35259517,342.13225878)(179.34259518,342.40725851)(179.32260092,342.66726336)
\curveto(179.30259522,342.92725799)(179.23759529,343.14225777)(179.12760092,343.31226336)
\curveto(179.04759548,343.45225746)(178.94259558,343.55725736)(178.81260092,343.62726336)
\curveto(178.69259583,343.69725722)(178.54259598,343.75725716)(178.36260092,343.80726336)
\curveto(178.3225962,343.8172571)(178.28259624,343.8172571)(178.24260092,343.80726336)
\curveto(178.20259632,343.80725711)(178.15759637,343.8122571)(178.10760092,343.82226336)
\curveto(177.99759653,343.84225707)(177.89259663,343.83225708)(177.79260092,343.79226336)
\curveto(177.77259675,343.79225712)(177.75259677,343.78725713)(177.73260092,343.77726336)
\lineto(177.67260092,343.77726336)
\curveto(177.51259701,343.72725719)(177.35759717,343.64225727)(177.20760092,343.52226336)
\curveto(177.04759748,343.40225751)(176.9225976,343.26225765)(176.83260092,343.10226336)
\curveto(176.75259777,342.95225796)(176.69259783,342.77725814)(176.65260092,342.57726336)
\curveto(176.6225979,342.38725853)(176.60259792,342.17725874)(176.59260092,341.94726336)
\lineto(176.59260092,341.19726336)
\lineto(176.59260092,339.17226336)
\lineto(176.59260092,338.25726336)
\lineto(176.59260092,337.98726336)
\curveto(176.59259793,337.89726302)(176.57759795,337.8172631)(176.54760092,337.74726336)
\curveto(176.50759802,337.65726326)(176.43259809,337.60226331)(176.32260092,337.58226336)
\curveto(176.21259831,337.56226335)(176.08759844,337.55226336)(175.94760092,337.55226336)
\lineto(175.16760092,337.55226336)
\lineto(174.86760092,337.55226336)
\curveto(174.77759975,337.56226335)(174.70259982,337.58726333)(174.64260092,337.62726336)
\curveto(174.55259997,337.67726324)(174.50260002,337.76726315)(174.49260092,337.89726336)
\lineto(174.49260092,338.33226336)
\lineto(174.49260092,340.08726336)
\lineto(174.49260092,343.74726336)
\lineto(174.49260092,344.64726336)
\lineto(174.49260092,344.93226336)
\curveto(174.50260002,345.02225589)(174.5276,345.09725582)(174.56760092,345.15726336)
\curveto(174.61759991,345.2172557)(174.69759983,345.25725566)(174.80760092,345.27726336)
\lineto(174.89760092,345.27726336)
\curveto(174.94759958,345.28725563)(174.99759953,345.29225562)(175.04760092,345.29226336)
\lineto(175.21260092,345.29226336)
\lineto(175.82760092,345.29226336)
\curveto(175.90759862,345.29225562)(175.98259854,345.28725563)(176.05260092,345.27726336)
\curveto(176.13259839,345.27725564)(176.20259832,345.26725565)(176.26260092,345.24726336)
\curveto(176.34259818,345.2172557)(176.39259813,345.16725575)(176.41260092,345.09726336)
\curveto(176.44259808,345.02725589)(176.46759806,344.94725597)(176.48760092,344.85726336)
\curveto(176.49759803,344.82725609)(176.49759803,344.79725612)(176.48760092,344.76726336)
\curveto(176.48759804,344.74725617)(176.49759803,344.72725619)(176.51760092,344.70726336)
\curveto(176.527598,344.67725624)(176.53759799,344.65225626)(176.54760092,344.63226336)
\curveto(176.56759796,344.62225629)(176.58759794,344.60725631)(176.60760092,344.58726336)
\curveto(176.7275978,344.57725634)(176.8275977,344.6122563)(176.90760092,344.69226336)
\curveto(176.98759754,344.78225613)(177.06259746,344.85225606)(177.13260092,344.90226336)
\curveto(177.27259725,345.00225591)(177.41259711,345.09225582)(177.55260092,345.17226336)
\curveto(177.70259682,345.25225566)(177.86259666,345.3172556)(178.03260092,345.36726336)
\curveto(178.1225964,345.39725552)(178.21259631,345.4172555)(178.30260092,345.42726336)
\curveto(178.39259613,345.43725548)(178.48759604,345.45225546)(178.58760092,345.47226336)
\curveto(178.61759591,345.48225543)(178.66259586,345.48225543)(178.72260092,345.47226336)
\curveto(178.78259574,345.47225544)(178.8225957,345.47725544)(178.84260092,345.48726336)
}
}
{
\newrgbcolor{curcolor}{0 0 0}
\pscustom[linestyle=none,fillstyle=solid,fillcolor=curcolor]
{
\newpath
\moveto(195.34635092,341.49726336)
\curveto(195.36634275,341.4172595)(195.36634275,341.32725959)(195.34635092,341.22726336)
\curveto(195.32634279,341.12725979)(195.29134283,341.06225985)(195.24135092,341.03226336)
\curveto(195.19134293,340.99225992)(195.116343,340.96225995)(195.01635092,340.94226336)
\curveto(194.92634319,340.93225998)(194.8213433,340.92225999)(194.70135092,340.91226336)
\lineto(194.35635092,340.91226336)
\curveto(194.24634387,340.92225999)(194.14634397,340.92725999)(194.05635092,340.92726336)
\lineto(190.39635092,340.92726336)
\lineto(190.18635092,340.92726336)
\curveto(190.12634799,340.92725999)(190.07134805,340.91726)(190.02135092,340.89726336)
\curveto(189.94134818,340.85726006)(189.89134823,340.8172601)(189.87135092,340.77726336)
\curveto(189.85134827,340.75726016)(189.83134829,340.7172602)(189.81135092,340.65726336)
\curveto(189.79134833,340.60726031)(189.78634833,340.55726036)(189.79635092,340.50726336)
\curveto(189.8163483,340.44726047)(189.82634829,340.38726053)(189.82635092,340.32726336)
\curveto(189.83634828,340.27726064)(189.85134827,340.22226069)(189.87135092,340.16226336)
\curveto(189.95134817,339.92226099)(190.04634807,339.72226119)(190.15635092,339.56226336)
\curveto(190.27634784,339.4122615)(190.43634768,339.27726164)(190.63635092,339.15726336)
\curveto(190.7163474,339.10726181)(190.79634732,339.07226184)(190.87635092,339.05226336)
\curveto(190.96634715,339.04226187)(191.05634706,339.02226189)(191.14635092,338.99226336)
\curveto(191.22634689,338.97226194)(191.33634678,338.95726196)(191.47635092,338.94726336)
\curveto(191.6163465,338.93726198)(191.73634638,338.94226197)(191.83635092,338.96226336)
\lineto(191.97135092,338.96226336)
\curveto(192.07134605,338.98226193)(192.16134596,339.00226191)(192.24135092,339.02226336)
\curveto(192.33134579,339.05226186)(192.4163457,339.08226183)(192.49635092,339.11226336)
\curveto(192.59634552,339.16226175)(192.70634541,339.22726169)(192.82635092,339.30726336)
\curveto(192.95634516,339.38726153)(193.05134507,339.46726145)(193.11135092,339.54726336)
\curveto(193.16134496,339.6172613)(193.21134491,339.68226123)(193.26135092,339.74226336)
\curveto(193.3213448,339.8122611)(193.39134473,339.86226105)(193.47135092,339.89226336)
\curveto(193.57134455,339.94226097)(193.69634442,339.96226095)(193.84635092,339.95226336)
\lineto(194.28135092,339.95226336)
\lineto(194.46135092,339.95226336)
\curveto(194.53134359,339.96226095)(194.59134353,339.95726096)(194.64135092,339.93726336)
\lineto(194.79135092,339.93726336)
\curveto(194.89134323,339.917261)(194.96134316,339.89226102)(195.00135092,339.86226336)
\curveto(195.04134308,339.84226107)(195.06134306,339.79726112)(195.06135092,339.72726336)
\curveto(195.07134305,339.65726126)(195.06634305,339.59726132)(195.04635092,339.54726336)
\curveto(194.99634312,339.40726151)(194.94134318,339.28226163)(194.88135092,339.17226336)
\curveto(194.8213433,339.06226185)(194.75134337,338.95226196)(194.67135092,338.84226336)
\curveto(194.45134367,338.5122624)(194.20134392,338.24726267)(193.92135092,338.04726336)
\curveto(193.64134448,337.84726307)(193.29134483,337.67726324)(192.87135092,337.53726336)
\curveto(192.76134536,337.49726342)(192.65134547,337.47226344)(192.54135092,337.46226336)
\curveto(192.43134569,337.45226346)(192.3163458,337.43226348)(192.19635092,337.40226336)
\curveto(192.15634596,337.39226352)(192.11134601,337.39226352)(192.06135092,337.40226336)
\curveto(192.0213461,337.40226351)(191.98134614,337.39726352)(191.94135092,337.38726336)
\lineto(191.77635092,337.38726336)
\curveto(191.72634639,337.36726355)(191.66634645,337.36226355)(191.59635092,337.37226336)
\curveto(191.53634658,337.37226354)(191.48134664,337.37726354)(191.43135092,337.38726336)
\curveto(191.35134677,337.39726352)(191.28134684,337.39726352)(191.22135092,337.38726336)
\curveto(191.16134696,337.37726354)(191.09634702,337.38226353)(191.02635092,337.40226336)
\curveto(190.97634714,337.42226349)(190.9213472,337.43226348)(190.86135092,337.43226336)
\curveto(190.80134732,337.43226348)(190.74634737,337.44226347)(190.69635092,337.46226336)
\curveto(190.58634753,337.48226343)(190.47634764,337.50726341)(190.36635092,337.53726336)
\curveto(190.25634786,337.55726336)(190.15634796,337.59226332)(190.06635092,337.64226336)
\curveto(189.95634816,337.68226323)(189.85134827,337.7172632)(189.75135092,337.74726336)
\curveto(189.66134846,337.78726313)(189.57634854,337.83226308)(189.49635092,337.88226336)
\curveto(189.17634894,338.08226283)(188.89134923,338.3122626)(188.64135092,338.57226336)
\curveto(188.39134973,338.84226207)(188.18634993,339.15226176)(188.02635092,339.50226336)
\curveto(187.97635014,339.6122613)(187.93635018,339.72226119)(187.90635092,339.83226336)
\curveto(187.87635024,339.95226096)(187.83635028,340.07226084)(187.78635092,340.19226336)
\curveto(187.77635034,340.23226068)(187.77135035,340.26726065)(187.77135092,340.29726336)
\curveto(187.77135035,340.33726058)(187.76635035,340.37726054)(187.75635092,340.41726336)
\curveto(187.7163504,340.53726038)(187.69135043,340.66726025)(187.68135092,340.80726336)
\lineto(187.65135092,341.22726336)
\curveto(187.65135047,341.27725964)(187.64635047,341.33225958)(187.63635092,341.39226336)
\curveto(187.63635048,341.45225946)(187.64135048,341.50725941)(187.65135092,341.55726336)
\lineto(187.65135092,341.73726336)
\lineto(187.69635092,342.09726336)
\curveto(187.73635038,342.26725865)(187.77135035,342.43225848)(187.80135092,342.59226336)
\curveto(187.83135029,342.75225816)(187.87635024,342.90225801)(187.93635092,343.04226336)
\curveto(188.36634975,344.08225683)(189.09634902,344.8172561)(190.12635092,345.24726336)
\curveto(190.26634785,345.30725561)(190.40634771,345.34725557)(190.54635092,345.36726336)
\curveto(190.69634742,345.39725552)(190.85134727,345.43225548)(191.01135092,345.47226336)
\curveto(191.09134703,345.48225543)(191.16634695,345.48725543)(191.23635092,345.48726336)
\curveto(191.30634681,345.48725543)(191.38134674,345.49225542)(191.46135092,345.50226336)
\curveto(191.97134615,345.5122554)(192.40634571,345.45225546)(192.76635092,345.32226336)
\curveto(193.13634498,345.20225571)(193.46634465,345.04225587)(193.75635092,344.84226336)
\curveto(193.84634427,344.78225613)(193.93634418,344.7122562)(194.02635092,344.63226336)
\curveto(194.116344,344.56225635)(194.19634392,344.48725643)(194.26635092,344.40726336)
\curveto(194.29634382,344.35725656)(194.33634378,344.3172566)(194.38635092,344.28726336)
\curveto(194.46634365,344.17725674)(194.54134358,344.06225685)(194.61135092,343.94226336)
\curveto(194.68134344,343.83225708)(194.75634336,343.7172572)(194.83635092,343.59726336)
\curveto(194.88634323,343.50725741)(194.92634319,343.4122575)(194.95635092,343.31226336)
\curveto(194.99634312,343.22225769)(195.03634308,343.12225779)(195.07635092,343.01226336)
\curveto(195.12634299,342.88225803)(195.16634295,342.74725817)(195.19635092,342.60726336)
\curveto(195.22634289,342.46725845)(195.26134286,342.32725859)(195.30135092,342.18726336)
\curveto(195.3213428,342.10725881)(195.32634279,342.0172589)(195.31635092,341.91726336)
\curveto(195.3163428,341.82725909)(195.32634279,341.74225917)(195.34635092,341.66226336)
\lineto(195.34635092,341.49726336)
\moveto(193.09635092,342.38226336)
\curveto(193.16634495,342.48225843)(193.17134495,342.60225831)(193.11135092,342.74226336)
\curveto(193.06134506,342.89225802)(193.0213451,343.00225791)(192.99135092,343.07226336)
\curveto(192.85134527,343.34225757)(192.66634545,343.54725737)(192.43635092,343.68726336)
\curveto(192.20634591,343.83725708)(191.88634623,343.917257)(191.47635092,343.92726336)
\curveto(191.44634667,343.90725701)(191.41134671,343.90225701)(191.37135092,343.91226336)
\curveto(191.33134679,343.92225699)(191.29634682,343.92225699)(191.26635092,343.91226336)
\curveto(191.2163469,343.89225702)(191.16134696,343.87725704)(191.10135092,343.86726336)
\curveto(191.04134708,343.86725705)(190.98634713,343.85725706)(190.93635092,343.83726336)
\curveto(190.49634762,343.69725722)(190.17134795,343.42225749)(189.96135092,343.01226336)
\curveto(189.94134818,342.97225794)(189.9163482,342.917258)(189.88635092,342.84726336)
\curveto(189.86634825,342.78725813)(189.85134827,342.72225819)(189.84135092,342.65226336)
\curveto(189.83134829,342.59225832)(189.83134829,342.53225838)(189.84135092,342.47226336)
\curveto(189.86134826,342.4122585)(189.89634822,342.36225855)(189.94635092,342.32226336)
\curveto(190.02634809,342.27225864)(190.13634798,342.24725867)(190.27635092,342.24726336)
\lineto(190.68135092,342.24726336)
\lineto(192.34635092,342.24726336)
\lineto(192.78135092,342.24726336)
\curveto(192.94134518,342.25725866)(193.04634507,342.30225861)(193.09635092,342.38226336)
}
}
{
\newrgbcolor{curcolor}{0 0 0}
\pscustom[linestyle=none,fillstyle=solid,fillcolor=curcolor]
{
\newpath
\moveto(201.01963217,345.48726336)
\curveto(201.61962636,345.50725541)(202.11962586,345.42225549)(202.51963217,345.23226336)
\curveto(202.91962506,345.04225587)(203.23462475,344.76225615)(203.46463217,344.39226336)
\curveto(203.53462445,344.28225663)(203.58962439,344.16225675)(203.62963217,344.03226336)
\curveto(203.66962431,343.912257)(203.70962427,343.78725713)(203.74963217,343.65726336)
\curveto(203.76962421,343.57725734)(203.7796242,343.50225741)(203.77963217,343.43226336)
\curveto(203.78962419,343.36225755)(203.80462418,343.29225762)(203.82463217,343.22226336)
\curveto(203.82462416,343.16225775)(203.82962415,343.12225779)(203.83963217,343.10226336)
\curveto(203.85962412,342.96225795)(203.86962411,342.8172581)(203.86963217,342.66726336)
\lineto(203.86963217,342.23226336)
\lineto(203.86963217,340.89726336)
\lineto(203.86963217,338.46726336)
\curveto(203.86962411,338.27726264)(203.86462412,338.09226282)(203.85463217,337.91226336)
\curveto(203.85462413,337.74226317)(203.7846242,337.63226328)(203.64463217,337.58226336)
\curveto(203.5846244,337.56226335)(203.51462447,337.55226336)(203.43463217,337.55226336)
\lineto(203.19463217,337.55226336)
\lineto(202.38463217,337.55226336)
\curveto(202.26462572,337.55226336)(202.15462583,337.55726336)(202.05463217,337.56726336)
\curveto(201.96462602,337.58726333)(201.89462609,337.63226328)(201.84463217,337.70226336)
\curveto(201.80462618,337.76226315)(201.7796262,337.83726308)(201.76963217,337.92726336)
\lineto(201.76963217,338.24226336)
\lineto(201.76963217,339.29226336)
\lineto(201.76963217,341.52726336)
\curveto(201.76962621,341.89725902)(201.75462623,342.23725868)(201.72463217,342.54726336)
\curveto(201.69462629,342.86725805)(201.60462638,343.13725778)(201.45463217,343.35726336)
\curveto(201.31462667,343.55725736)(201.10962687,343.69725722)(200.83963217,343.77726336)
\curveto(200.78962719,343.79725712)(200.73462725,343.80725711)(200.67463217,343.80726336)
\curveto(200.62462736,343.80725711)(200.56962741,343.8172571)(200.50963217,343.83726336)
\curveto(200.45962752,343.84725707)(200.39462759,343.84725707)(200.31463217,343.83726336)
\curveto(200.24462774,343.83725708)(200.18962779,343.83225708)(200.14963217,343.82226336)
\curveto(200.10962787,343.8122571)(200.07462791,343.80725711)(200.04463217,343.80726336)
\curveto(200.01462797,343.80725711)(199.984628,343.80225711)(199.95463217,343.79226336)
\curveto(199.72462826,343.73225718)(199.53962844,343.65225726)(199.39963217,343.55226336)
\curveto(199.0796289,343.32225759)(198.88962909,342.98725793)(198.82963217,342.54726336)
\curveto(198.76962921,342.10725881)(198.73962924,341.6122593)(198.73963217,341.06226336)
\lineto(198.73963217,339.18726336)
\lineto(198.73963217,338.27226336)
\lineto(198.73963217,338.00226336)
\curveto(198.73962924,337.912263)(198.72462926,337.83726308)(198.69463217,337.77726336)
\curveto(198.64462934,337.66726325)(198.56462942,337.60226331)(198.45463217,337.58226336)
\curveto(198.34462964,337.56226335)(198.20962977,337.55226336)(198.04963217,337.55226336)
\lineto(197.29963217,337.55226336)
\curveto(197.18963079,337.55226336)(197.0796309,337.55726336)(196.96963217,337.56726336)
\curveto(196.85963112,337.57726334)(196.7796312,337.6122633)(196.72963217,337.67226336)
\curveto(196.65963132,337.76226315)(196.62463136,337.89226302)(196.62463217,338.06226336)
\curveto(196.63463135,338.23226268)(196.63963134,338.39226252)(196.63963217,338.54226336)
\lineto(196.63963217,340.58226336)
\lineto(196.63963217,343.88226336)
\lineto(196.63963217,344.64726336)
\lineto(196.63963217,344.94726336)
\curveto(196.64963133,345.03725588)(196.6796313,345.1122558)(196.72963217,345.17226336)
\curveto(196.74963123,345.20225571)(196.7796312,345.22225569)(196.81963217,345.23226336)
\curveto(196.86963111,345.25225566)(196.91963106,345.26725565)(196.96963217,345.27726336)
\lineto(197.04463217,345.27726336)
\curveto(197.09463089,345.28725563)(197.14463084,345.29225562)(197.19463217,345.29226336)
\lineto(197.35963217,345.29226336)
\lineto(197.98963217,345.29226336)
\curveto(198.06962991,345.29225562)(198.14462984,345.28725563)(198.21463217,345.27726336)
\curveto(198.29462969,345.27725564)(198.36462962,345.26725565)(198.42463217,345.24726336)
\curveto(198.49462949,345.2172557)(198.53962944,345.17225574)(198.55963217,345.11226336)
\curveto(198.58962939,345.05225586)(198.61462937,344.98225593)(198.63463217,344.90226336)
\curveto(198.64462934,344.86225605)(198.64462934,344.82725609)(198.63463217,344.79726336)
\curveto(198.63462935,344.76725615)(198.64462934,344.73725618)(198.66463217,344.70726336)
\curveto(198.6846293,344.65725626)(198.69962928,344.62725629)(198.70963217,344.61726336)
\curveto(198.72962925,344.60725631)(198.75462923,344.59225632)(198.78463217,344.57226336)
\curveto(198.89462909,344.56225635)(198.984629,344.59725632)(199.05463217,344.67726336)
\curveto(199.12462886,344.76725615)(199.19962878,344.83725608)(199.27963217,344.88726336)
\curveto(199.54962843,345.08725583)(199.84962813,345.24725567)(200.17963217,345.36726336)
\curveto(200.26962771,345.39725552)(200.35962762,345.4172555)(200.44963217,345.42726336)
\curveto(200.54962743,345.43725548)(200.65462733,345.45225546)(200.76463217,345.47226336)
\curveto(200.79462719,345.48225543)(200.83962714,345.48225543)(200.89963217,345.47226336)
\curveto(200.95962702,345.47225544)(200.99962698,345.47725544)(201.01963217,345.48726336)
}
}
{
\newrgbcolor{curcolor}{0 0 0}
\pscustom[linestyle=none,fillstyle=solid,fillcolor=curcolor]
{
\newpath
\moveto(206.55088217,347.60226336)
\lineto(207.55588217,347.60226336)
\curveto(207.70587918,347.60225331)(207.83587905,347.59225332)(207.94588217,347.57226336)
\curveto(208.06587882,347.56225335)(208.15087874,347.50225341)(208.20088217,347.39226336)
\curveto(208.22087867,347.34225357)(208.23087866,347.28225363)(208.23088217,347.21226336)
\lineto(208.23088217,347.00226336)
\lineto(208.23088217,346.32726336)
\curveto(208.23087866,346.27725464)(208.22587866,346.2172547)(208.21588217,346.14726336)
\curveto(208.21587867,346.08725483)(208.22087867,346.03225488)(208.23088217,345.98226336)
\lineto(208.23088217,345.81726336)
\curveto(208.23087866,345.73725518)(208.23587865,345.66225525)(208.24588217,345.59226336)
\curveto(208.25587863,345.53225538)(208.28087861,345.47725544)(208.32088217,345.42726336)
\curveto(208.3908785,345.33725558)(208.51587837,345.28725563)(208.69588217,345.27726336)
\lineto(209.23588217,345.27726336)
\lineto(209.41588217,345.27726336)
\curveto(209.47587741,345.27725564)(209.53087736,345.26725565)(209.58088217,345.24726336)
\curveto(209.6908772,345.19725572)(209.75087714,345.10725581)(209.76088217,344.97726336)
\curveto(209.78087711,344.84725607)(209.7908771,344.70225621)(209.79088217,344.54226336)
\lineto(209.79088217,344.33226336)
\curveto(209.80087709,344.26225665)(209.79587709,344.20225671)(209.77588217,344.15226336)
\curveto(209.72587716,343.99225692)(209.62087727,343.90725701)(209.46088217,343.89726336)
\curveto(209.30087759,343.88725703)(209.12087777,343.88225703)(208.92088217,343.88226336)
\lineto(208.78588217,343.88226336)
\curveto(208.74587814,343.89225702)(208.71087818,343.89225702)(208.68088217,343.88226336)
\curveto(208.64087825,343.87225704)(208.60587828,343.86725705)(208.57588217,343.86726336)
\curveto(208.54587834,343.87725704)(208.51587837,343.87225704)(208.48588217,343.85226336)
\curveto(208.40587848,343.83225708)(208.34587854,343.78725713)(208.30588217,343.71726336)
\curveto(208.27587861,343.65725726)(208.25087864,343.58225733)(208.23088217,343.49226336)
\curveto(208.22087867,343.44225747)(208.22087867,343.38725753)(208.23088217,343.32726336)
\curveto(208.24087865,343.26725765)(208.24087865,343.2122577)(208.23088217,343.16226336)
\lineto(208.23088217,342.23226336)
\lineto(208.23088217,340.47726336)
\curveto(208.23087866,340.22726069)(208.23587865,340.00726091)(208.24588217,339.81726336)
\curveto(208.26587862,339.63726128)(208.33087856,339.47726144)(208.44088217,339.33726336)
\curveto(208.4908784,339.27726164)(208.55587833,339.23226168)(208.63588217,339.20226336)
\lineto(208.90588217,339.14226336)
\curveto(208.93587795,339.13226178)(208.96587792,339.12726179)(208.99588217,339.12726336)
\curveto(209.03587785,339.13726178)(209.06587782,339.13726178)(209.08588217,339.12726336)
\lineto(209.25088217,339.12726336)
\curveto(209.36087753,339.12726179)(209.45587743,339.12226179)(209.53588217,339.11226336)
\curveto(209.61587727,339.10226181)(209.68087721,339.06226185)(209.73088217,338.99226336)
\curveto(209.77087712,338.93226198)(209.7908771,338.85226206)(209.79088217,338.75226336)
\lineto(209.79088217,338.46726336)
\curveto(209.7908771,338.25726266)(209.7858771,338.06226285)(209.77588217,337.88226336)
\curveto(209.77587711,337.7122632)(209.69587719,337.59726332)(209.53588217,337.53726336)
\curveto(209.4858774,337.5172634)(209.44087745,337.5122634)(209.40088217,337.52226336)
\curveto(209.36087753,337.52226339)(209.31587757,337.5122634)(209.26588217,337.49226336)
\lineto(209.11588217,337.49226336)
\curveto(209.09587779,337.49226342)(209.06587782,337.49726342)(209.02588217,337.50726336)
\curveto(208.9858779,337.50726341)(208.95087794,337.50226341)(208.92088217,337.49226336)
\curveto(208.87087802,337.48226343)(208.81587807,337.48226343)(208.75588217,337.49226336)
\lineto(208.60588217,337.49226336)
\lineto(208.45588217,337.49226336)
\curveto(208.40587848,337.48226343)(208.36087853,337.48226343)(208.32088217,337.49226336)
\lineto(208.15588217,337.49226336)
\curveto(208.10587878,337.50226341)(208.05087884,337.50726341)(207.99088217,337.50726336)
\curveto(207.93087896,337.50726341)(207.87587901,337.5122634)(207.82588217,337.52226336)
\curveto(207.75587913,337.53226338)(207.6908792,337.54226337)(207.63088217,337.55226336)
\lineto(207.45088217,337.58226336)
\curveto(207.34087955,337.6122633)(207.23587965,337.64726327)(207.13588217,337.68726336)
\curveto(207.03587985,337.72726319)(206.94087995,337.77226314)(206.85088217,337.82226336)
\lineto(206.76088217,337.88226336)
\curveto(206.73088016,337.912263)(206.69588019,337.94226297)(206.65588217,337.97226336)
\curveto(206.63588025,337.99226292)(206.61088028,338.0122629)(206.58088217,338.03226336)
\lineto(206.50588217,338.10726336)
\curveto(206.36588052,338.29726262)(206.26088063,338.50726241)(206.19088217,338.73726336)
\curveto(206.17088072,338.77726214)(206.16088073,338.8122621)(206.16088217,338.84226336)
\curveto(206.17088072,338.88226203)(206.17088072,338.92726199)(206.16088217,338.97726336)
\curveto(206.15088074,338.99726192)(206.14588074,339.02226189)(206.14588217,339.05226336)
\curveto(206.14588074,339.08226183)(206.14088075,339.10726181)(206.13088217,339.12726336)
\lineto(206.13088217,339.27726336)
\curveto(206.12088077,339.3172616)(206.11588077,339.36226155)(206.11588217,339.41226336)
\curveto(206.12588076,339.46226145)(206.13088076,339.5122614)(206.13088217,339.56226336)
\lineto(206.13088217,340.13226336)
\lineto(206.13088217,342.36726336)
\lineto(206.13088217,343.16226336)
\lineto(206.13088217,343.37226336)
\curveto(206.14088075,343.44225747)(206.13588075,343.50725741)(206.11588217,343.56726336)
\curveto(206.07588081,343.70725721)(206.00588088,343.79725712)(205.90588217,343.83726336)
\curveto(205.79588109,343.88725703)(205.65588123,343.90225701)(205.48588217,343.88226336)
\curveto(205.31588157,343.86225705)(205.17088172,343.87725704)(205.05088217,343.92726336)
\curveto(204.97088192,343.95725696)(204.92088197,344.00225691)(204.90088217,344.06226336)
\curveto(204.88088201,344.12225679)(204.86088203,344.19725672)(204.84088217,344.28726336)
\lineto(204.84088217,344.60226336)
\curveto(204.84088205,344.78225613)(204.85088204,344.92725599)(204.87088217,345.03726336)
\curveto(204.890882,345.14725577)(204.97588191,345.22225569)(205.12588217,345.26226336)
\curveto(205.16588172,345.28225563)(205.20588168,345.28725563)(205.24588217,345.27726336)
\lineto(205.38088217,345.27726336)
\curveto(205.53088136,345.27725564)(205.67088122,345.28225563)(205.80088217,345.29226336)
\curveto(205.93088096,345.3122556)(206.02088087,345.37225554)(206.07088217,345.47226336)
\curveto(206.10088079,345.54225537)(206.11588077,345.62225529)(206.11588217,345.71226336)
\curveto(206.12588076,345.80225511)(206.13088076,345.89225502)(206.13088217,345.98226336)
\lineto(206.13088217,346.91226336)
\lineto(206.13088217,347.16726336)
\curveto(206.13088076,347.25725366)(206.14088075,347.33225358)(206.16088217,347.39226336)
\curveto(206.21088068,347.49225342)(206.2858806,347.55725336)(206.38588217,347.58726336)
\curveto(206.40588048,347.59725332)(206.43088046,347.59725332)(206.46088217,347.58726336)
\curveto(206.50088039,347.58725333)(206.53088036,347.59225332)(206.55088217,347.60226336)
}
}
{
\newrgbcolor{curcolor}{0 0 0}
\pscustom[linestyle=none,fillstyle=solid,fillcolor=curcolor]
{
\newpath
\moveto(217.82431967,338.15226336)
\curveto(217.84431182,338.04226287)(217.85431181,337.93226298)(217.85431967,337.82226336)
\curveto(217.8643118,337.7122632)(217.81431185,337.63726328)(217.70431967,337.59726336)
\curveto(217.64431202,337.56726335)(217.57431209,337.55226336)(217.49431967,337.55226336)
\lineto(217.25431967,337.55226336)
\lineto(216.44431967,337.55226336)
\lineto(216.17431967,337.55226336)
\curveto(216.09431357,337.56226335)(216.02931363,337.58726333)(215.97931967,337.62726336)
\curveto(215.90931375,337.66726325)(215.85431381,337.72226319)(215.81431967,337.79226336)
\curveto(215.78431388,337.87226304)(215.73931392,337.93726298)(215.67931967,337.98726336)
\curveto(215.659314,338.00726291)(215.63431403,338.02226289)(215.60431967,338.03226336)
\curveto(215.57431409,338.05226286)(215.53431413,338.05726286)(215.48431967,338.04726336)
\curveto(215.43431423,338.02726289)(215.38431428,338.00226291)(215.33431967,337.97226336)
\curveto(215.29431437,337.94226297)(215.24931441,337.917263)(215.19931967,337.89726336)
\curveto(215.14931451,337.85726306)(215.09431457,337.82226309)(215.03431967,337.79226336)
\lineto(214.85431967,337.70226336)
\curveto(214.72431494,337.64226327)(214.58931507,337.59226332)(214.44931967,337.55226336)
\curveto(214.30931535,337.52226339)(214.1643155,337.48726343)(214.01431967,337.44726336)
\curveto(213.94431572,337.42726349)(213.87431579,337.4172635)(213.80431967,337.41726336)
\curveto(213.74431592,337.40726351)(213.67931598,337.39726352)(213.60931967,337.38726336)
\lineto(213.51931967,337.38726336)
\curveto(213.48931617,337.37726354)(213.4593162,337.37226354)(213.42931967,337.37226336)
\lineto(213.26431967,337.37226336)
\curveto(213.1643165,337.35226356)(213.0643166,337.35226356)(212.96431967,337.37226336)
\lineto(212.82931967,337.37226336)
\curveto(212.7593169,337.39226352)(212.68931697,337.40226351)(212.61931967,337.40226336)
\curveto(212.5593171,337.39226352)(212.49931716,337.39726352)(212.43931967,337.41726336)
\curveto(212.33931732,337.43726348)(212.24431742,337.45726346)(212.15431967,337.47726336)
\curveto(212.0643176,337.48726343)(211.97931768,337.5122634)(211.89931967,337.55226336)
\curveto(211.60931805,337.66226325)(211.3593183,337.80226311)(211.14931967,337.97226336)
\curveto(210.94931871,338.15226276)(210.78931887,338.38726253)(210.66931967,338.67726336)
\curveto(210.63931902,338.74726217)(210.60931905,338.82226209)(210.57931967,338.90226336)
\curveto(210.5593191,338.98226193)(210.53931912,339.06726185)(210.51931967,339.15726336)
\curveto(210.49931916,339.20726171)(210.48931917,339.25726166)(210.48931967,339.30726336)
\curveto(210.49931916,339.35726156)(210.49931916,339.40726151)(210.48931967,339.45726336)
\curveto(210.47931918,339.48726143)(210.46931919,339.54726137)(210.45931967,339.63726336)
\curveto(210.4593192,339.73726118)(210.4643192,339.80726111)(210.47431967,339.84726336)
\curveto(210.49431917,339.94726097)(210.50431916,340.03226088)(210.50431967,340.10226336)
\lineto(210.59431967,340.43226336)
\curveto(210.62431904,340.55226036)(210.664319,340.65726026)(210.71431967,340.74726336)
\curveto(210.88431878,341.03725988)(211.07931858,341.25725966)(211.29931967,341.40726336)
\curveto(211.51931814,341.55725936)(211.79931786,341.68725923)(212.13931967,341.79726336)
\curveto(212.26931739,341.84725907)(212.40431726,341.88225903)(212.54431967,341.90226336)
\curveto(212.68431698,341.92225899)(212.82431684,341.94725897)(212.96431967,341.97726336)
\curveto(213.04431662,341.99725892)(213.12931653,342.00725891)(213.21931967,342.00726336)
\curveto(213.30931635,342.0172589)(213.39931626,342.03225888)(213.48931967,342.05226336)
\curveto(213.5593161,342.07225884)(213.62931603,342.07725884)(213.69931967,342.06726336)
\curveto(213.76931589,342.06725885)(213.84431582,342.07725884)(213.92431967,342.09726336)
\curveto(213.99431567,342.1172588)(214.0643156,342.12725879)(214.13431967,342.12726336)
\curveto(214.20431546,342.12725879)(214.27931538,342.13725878)(214.35931967,342.15726336)
\curveto(214.56931509,342.20725871)(214.7593149,342.24725867)(214.92931967,342.27726336)
\curveto(215.10931455,342.3172586)(215.26931439,342.40725851)(215.40931967,342.54726336)
\curveto(215.49931416,342.63725828)(215.5593141,342.73725818)(215.58931967,342.84726336)
\curveto(215.59931406,342.87725804)(215.59931406,342.90225801)(215.58931967,342.92226336)
\curveto(215.58931407,342.94225797)(215.59431407,342.96225795)(215.60431967,342.98226336)
\curveto(215.61431405,343.00225791)(215.61931404,343.03225788)(215.61931967,343.07226336)
\lineto(215.61931967,343.16226336)
\lineto(215.58931967,343.28226336)
\curveto(215.58931407,343.32225759)(215.58431408,343.35725756)(215.57431967,343.38726336)
\curveto(215.47431419,343.68725723)(215.2643144,343.89225702)(214.94431967,344.00226336)
\curveto(214.85431481,344.03225688)(214.74431492,344.05225686)(214.61431967,344.06226336)
\curveto(214.49431517,344.08225683)(214.36931529,344.08725683)(214.23931967,344.07726336)
\curveto(214.10931555,344.07725684)(213.98431568,344.06725685)(213.86431967,344.04726336)
\curveto(213.74431592,344.02725689)(213.63931602,344.00225691)(213.54931967,343.97226336)
\curveto(213.48931617,343.95225696)(213.42931623,343.92225699)(213.36931967,343.88226336)
\curveto(213.31931634,343.85225706)(213.26931639,343.8172571)(213.21931967,343.77726336)
\curveto(213.16931649,343.73725718)(213.11431655,343.68225723)(213.05431967,343.61226336)
\curveto(213.00431666,343.54225737)(212.96931669,343.47725744)(212.94931967,343.41726336)
\curveto(212.89931676,343.3172576)(212.85431681,343.22225769)(212.81431967,343.13226336)
\curveto(212.78431688,343.04225787)(212.71431695,342.98225793)(212.60431967,342.95226336)
\curveto(212.52431714,342.93225798)(212.43931722,342.92225799)(212.34931967,342.92226336)
\lineto(212.07931967,342.92226336)
\lineto(211.50931967,342.92226336)
\curveto(211.4593182,342.92225799)(211.40931825,342.917258)(211.35931967,342.90726336)
\curveto(211.30931835,342.90725801)(211.2643184,342.912258)(211.22431967,342.92226336)
\lineto(211.08931967,342.92226336)
\curveto(211.06931859,342.93225798)(211.04431862,342.93725798)(211.01431967,342.93726336)
\curveto(210.98431868,342.93725798)(210.9593187,342.94725797)(210.93931967,342.96726336)
\curveto(210.8593188,342.98725793)(210.80431886,343.05225786)(210.77431967,343.16226336)
\curveto(210.7643189,343.2122577)(210.7643189,343.26225765)(210.77431967,343.31226336)
\curveto(210.78431888,343.36225755)(210.79431887,343.40725751)(210.80431967,343.44726336)
\curveto(210.83431883,343.55725736)(210.8643188,343.65725726)(210.89431967,343.74726336)
\curveto(210.93431873,343.84725707)(210.97931868,343.93725698)(211.02931967,344.01726336)
\lineto(211.11931967,344.16726336)
\lineto(211.20931967,344.31726336)
\curveto(211.28931837,344.42725649)(211.38931827,344.53225638)(211.50931967,344.63226336)
\curveto(211.52931813,344.64225627)(211.5593181,344.66725625)(211.59931967,344.70726336)
\curveto(211.64931801,344.74725617)(211.69431797,344.78225613)(211.73431967,344.81226336)
\curveto(211.77431789,344.84225607)(211.81931784,344.87225604)(211.86931967,344.90226336)
\curveto(212.03931762,345.0122559)(212.21931744,345.09725582)(212.40931967,345.15726336)
\curveto(212.59931706,345.22725569)(212.79431687,345.29225562)(212.99431967,345.35226336)
\curveto(213.11431655,345.38225553)(213.23931642,345.40225551)(213.36931967,345.41226336)
\curveto(213.49931616,345.42225549)(213.62931603,345.44225547)(213.75931967,345.47226336)
\curveto(213.79931586,345.48225543)(213.8593158,345.48225543)(213.93931967,345.47226336)
\curveto(214.02931563,345.46225545)(214.08431558,345.46725545)(214.10431967,345.48726336)
\curveto(214.51431515,345.49725542)(214.90431476,345.48225543)(215.27431967,345.44226336)
\curveto(215.65431401,345.40225551)(215.99431367,345.32725559)(216.29431967,345.21726336)
\curveto(216.60431306,345.10725581)(216.86931279,344.95725596)(217.08931967,344.76726336)
\curveto(217.30931235,344.58725633)(217.47931218,344.35225656)(217.59931967,344.06226336)
\curveto(217.66931199,343.89225702)(217.70931195,343.69725722)(217.71931967,343.47726336)
\curveto(217.72931193,343.25725766)(217.73431193,343.03225788)(217.73431967,342.80226336)
\lineto(217.73431967,339.45726336)
\lineto(217.73431967,338.87226336)
\curveto(217.73431193,338.68226223)(217.75431191,338.50726241)(217.79431967,338.34726336)
\curveto(217.80431186,338.3172626)(217.80931185,338.28226263)(217.80931967,338.24226336)
\curveto(217.80931185,338.2122627)(217.81431185,338.18226273)(217.82431967,338.15226336)
\moveto(215.61931967,340.46226336)
\curveto(215.62931403,340.5122604)(215.63431403,340.56726035)(215.63431967,340.62726336)
\curveto(215.63431403,340.69726022)(215.62931403,340.75726016)(215.61931967,340.80726336)
\curveto(215.59931406,340.86726005)(215.58931407,340.92225999)(215.58931967,340.97226336)
\curveto(215.58931407,341.02225989)(215.56931409,341.06225985)(215.52931967,341.09226336)
\curveto(215.47931418,341.13225978)(215.40431426,341.15225976)(215.30431967,341.15226336)
\curveto(215.2643144,341.14225977)(215.22931443,341.13225978)(215.19931967,341.12226336)
\curveto(215.16931449,341.12225979)(215.13431453,341.1172598)(215.09431967,341.10726336)
\curveto(215.02431464,341.08725983)(214.94931471,341.07225984)(214.86931967,341.06226336)
\curveto(214.78931487,341.05225986)(214.70931495,341.03725988)(214.62931967,341.01726336)
\curveto(214.59931506,341.00725991)(214.55431511,341.00225991)(214.49431967,341.00226336)
\curveto(214.3643153,340.97225994)(214.23431543,340.95225996)(214.10431967,340.94226336)
\curveto(213.97431569,340.93225998)(213.84931581,340.90726001)(213.72931967,340.86726336)
\curveto(213.64931601,340.84726007)(213.57431609,340.82726009)(213.50431967,340.80726336)
\curveto(213.43431623,340.79726012)(213.3643163,340.77726014)(213.29431967,340.74726336)
\curveto(213.08431658,340.65726026)(212.90431676,340.52226039)(212.75431967,340.34226336)
\curveto(212.61431705,340.16226075)(212.5643171,339.912261)(212.60431967,339.59226336)
\curveto(212.62431704,339.42226149)(212.67931698,339.28226163)(212.76931967,339.17226336)
\curveto(212.83931682,339.06226185)(212.94431672,338.97226194)(213.08431967,338.90226336)
\curveto(213.22431644,338.84226207)(213.37431629,338.79726212)(213.53431967,338.76726336)
\curveto(213.70431596,338.73726218)(213.87931578,338.72726219)(214.05931967,338.73726336)
\curveto(214.24931541,338.75726216)(214.42431524,338.79226212)(214.58431967,338.84226336)
\curveto(214.84431482,338.92226199)(215.04931461,339.04726187)(215.19931967,339.21726336)
\curveto(215.34931431,339.39726152)(215.4643142,339.6172613)(215.54431967,339.87726336)
\curveto(215.5643141,339.94726097)(215.57431409,340.0172609)(215.57431967,340.08726336)
\curveto(215.58431408,340.16726075)(215.59931406,340.24726067)(215.61931967,340.32726336)
\lineto(215.61931967,340.46226336)
}
}
{
\newrgbcolor{curcolor}{0 0 0}
\pscustom[linestyle=none,fillstyle=solid,fillcolor=curcolor]
{
\newpath
\moveto(223.81260092,345.48726336)
\curveto(223.9225956,345.48725543)(224.01759551,345.47725544)(224.09760092,345.45726336)
\curveto(224.18759534,345.43725548)(224.25759527,345.39225552)(224.30760092,345.32226336)
\curveto(224.36759516,345.24225567)(224.39759513,345.10225581)(224.39760092,344.90226336)
\lineto(224.39760092,344.39226336)
\lineto(224.39760092,344.01726336)
\curveto(224.40759512,343.87725704)(224.39259513,343.76725715)(224.35260092,343.68726336)
\curveto(224.31259521,343.6172573)(224.25259527,343.57225734)(224.17260092,343.55226336)
\curveto(224.10259542,343.53225738)(224.01759551,343.52225739)(223.91760092,343.52226336)
\curveto(223.8275957,343.52225739)(223.7275958,343.52725739)(223.61760092,343.53726336)
\curveto(223.51759601,343.54725737)(223.4225961,343.54225737)(223.33260092,343.52226336)
\curveto(223.26259626,343.50225741)(223.19259633,343.48725743)(223.12260092,343.47726336)
\curveto(223.05259647,343.47725744)(222.98759654,343.46725745)(222.92760092,343.44726336)
\curveto(222.76759676,343.39725752)(222.60759692,343.32225759)(222.44760092,343.22226336)
\curveto(222.28759724,343.13225778)(222.16259736,343.02725789)(222.07260092,342.90726336)
\curveto(222.0225975,342.82725809)(221.96759756,342.74225817)(221.90760092,342.65226336)
\curveto(221.85759767,342.57225834)(221.80759772,342.48725843)(221.75760092,342.39726336)
\curveto(221.7275978,342.3172586)(221.69759783,342.23225868)(221.66760092,342.14226336)
\lineto(221.60760092,341.90226336)
\curveto(221.58759794,341.83225908)(221.57759795,341.75725916)(221.57760092,341.67726336)
\curveto(221.57759795,341.60725931)(221.56759796,341.53725938)(221.54760092,341.46726336)
\curveto(221.53759799,341.42725949)(221.53259799,341.38725953)(221.53260092,341.34726336)
\curveto(221.54259798,341.3172596)(221.54259798,341.28725963)(221.53260092,341.25726336)
\lineto(221.53260092,341.01726336)
\curveto(221.51259801,340.94725997)(221.50759802,340.86726005)(221.51760092,340.77726336)
\curveto(221.527598,340.69726022)(221.53259799,340.6172603)(221.53260092,340.53726336)
\lineto(221.53260092,339.57726336)
\lineto(221.53260092,338.30226336)
\curveto(221.53259799,338.17226274)(221.527598,338.05226286)(221.51760092,337.94226336)
\curveto(221.50759802,337.83226308)(221.47759805,337.74226317)(221.42760092,337.67226336)
\curveto(221.40759812,337.64226327)(221.37259815,337.6172633)(221.32260092,337.59726336)
\curveto(221.28259824,337.58726333)(221.23759829,337.57726334)(221.18760092,337.56726336)
\lineto(221.11260092,337.56726336)
\curveto(221.06259846,337.55726336)(221.00759852,337.55226336)(220.94760092,337.55226336)
\lineto(220.78260092,337.55226336)
\lineto(220.13760092,337.55226336)
\curveto(220.07759945,337.56226335)(220.01259951,337.56726335)(219.94260092,337.56726336)
\lineto(219.74760092,337.56726336)
\curveto(219.69759983,337.58726333)(219.64759988,337.60226331)(219.59760092,337.61226336)
\curveto(219.54759998,337.63226328)(219.51260001,337.66726325)(219.49260092,337.71726336)
\curveto(219.45260007,337.76726315)(219.4276001,337.83726308)(219.41760092,337.92726336)
\lineto(219.41760092,338.22726336)
\lineto(219.41760092,339.24726336)
\lineto(219.41760092,343.47726336)
\lineto(219.41760092,344.58726336)
\lineto(219.41760092,344.87226336)
\curveto(219.41760011,344.97225594)(219.43760009,345.05225586)(219.47760092,345.11226336)
\curveto(219.5276,345.19225572)(219.60259992,345.24225567)(219.70260092,345.26226336)
\curveto(219.80259972,345.28225563)(219.9225996,345.29225562)(220.06260092,345.29226336)
\lineto(220.82760092,345.29226336)
\curveto(220.94759858,345.29225562)(221.05259847,345.28225563)(221.14260092,345.26226336)
\curveto(221.23259829,345.25225566)(221.30259822,345.20725571)(221.35260092,345.12726336)
\curveto(221.38259814,345.07725584)(221.39759813,345.00725591)(221.39760092,344.91726336)
\lineto(221.42760092,344.64726336)
\curveto(221.43759809,344.56725635)(221.45259807,344.49225642)(221.47260092,344.42226336)
\curveto(221.50259802,344.35225656)(221.55259797,344.3172566)(221.62260092,344.31726336)
\curveto(221.64259788,344.33725658)(221.66259786,344.34725657)(221.68260092,344.34726336)
\curveto(221.70259782,344.34725657)(221.7225978,344.35725656)(221.74260092,344.37726336)
\curveto(221.80259772,344.42725649)(221.85259767,344.48225643)(221.89260092,344.54226336)
\curveto(221.94259758,344.6122563)(222.00259752,344.67225624)(222.07260092,344.72226336)
\curveto(222.11259741,344.75225616)(222.14759738,344.78225613)(222.17760092,344.81226336)
\curveto(222.20759732,344.85225606)(222.24259728,344.88725603)(222.28260092,344.91726336)
\lineto(222.55260092,345.09726336)
\curveto(222.65259687,345.15725576)(222.75259677,345.2122557)(222.85260092,345.26226336)
\curveto(222.95259657,345.30225561)(223.05259647,345.33725558)(223.15260092,345.36726336)
\lineto(223.48260092,345.45726336)
\curveto(223.51259601,345.46725545)(223.56759596,345.46725545)(223.64760092,345.45726336)
\curveto(223.73759579,345.45725546)(223.79259573,345.46725545)(223.81260092,345.48726336)
}
}
{
\newrgbcolor{curcolor}{0 0 0}
\pscustom[linestyle=none,fillstyle=solid,fillcolor=curcolor]
{
\newpath
\moveto(227.31767904,348.14226336)
\curveto(227.38767609,348.06225285)(227.42267606,347.94225297)(227.42267904,347.78226336)
\lineto(227.42267904,347.31726336)
\lineto(227.42267904,346.91226336)
\curveto(227.42267606,346.77225414)(227.38767609,346.67725424)(227.31767904,346.62726336)
\curveto(227.25767622,346.57725434)(227.1776763,346.54725437)(227.07767904,346.53726336)
\curveto(226.98767649,346.52725439)(226.88767659,346.52225439)(226.77767904,346.52226336)
\lineto(225.93767904,346.52226336)
\curveto(225.82767765,346.52225439)(225.72767775,346.52725439)(225.63767904,346.53726336)
\curveto(225.55767792,346.54725437)(225.48767799,346.57725434)(225.42767904,346.62726336)
\curveto(225.38767809,346.65725426)(225.35767812,346.7122542)(225.33767904,346.79226336)
\curveto(225.32767815,346.88225403)(225.31767816,346.97725394)(225.30767904,347.07726336)
\lineto(225.30767904,347.40726336)
\curveto(225.31767816,347.5172534)(225.32267816,347.6122533)(225.32267904,347.69226336)
\lineto(225.32267904,347.90226336)
\curveto(225.33267815,347.97225294)(225.35267813,348.03225288)(225.38267904,348.08226336)
\curveto(225.40267808,348.12225279)(225.42767805,348.15225276)(225.45767904,348.17226336)
\lineto(225.57767904,348.23226336)
\curveto(225.59767788,348.23225268)(225.62267786,348.23225268)(225.65267904,348.23226336)
\curveto(225.6826778,348.24225267)(225.70767777,348.24725267)(225.72767904,348.24726336)
\lineto(226.82267904,348.24726336)
\curveto(226.92267656,348.24725267)(227.01767646,348.24225267)(227.10767904,348.23226336)
\curveto(227.19767628,348.22225269)(227.26767621,348.19225272)(227.31767904,348.14226336)
\moveto(227.42267904,338.37726336)
\curveto(227.42267606,338.17726274)(227.41767606,338.00726291)(227.40767904,337.86726336)
\curveto(227.39767608,337.72726319)(227.30767617,337.63226328)(227.13767904,337.58226336)
\curveto(227.0776764,337.56226335)(227.01267647,337.55226336)(226.94267904,337.55226336)
\curveto(226.87267661,337.56226335)(226.79767668,337.56726335)(226.71767904,337.56726336)
\lineto(225.87767904,337.56726336)
\curveto(225.78767769,337.56726335)(225.69767778,337.57226334)(225.60767904,337.58226336)
\curveto(225.52767795,337.59226332)(225.46767801,337.62226329)(225.42767904,337.67226336)
\curveto(225.36767811,337.74226317)(225.33267815,337.82726309)(225.32267904,337.92726336)
\lineto(225.32267904,338.27226336)
\lineto(225.32267904,344.60226336)
\lineto(225.32267904,344.90226336)
\curveto(225.32267816,345.00225591)(225.34267814,345.08225583)(225.38267904,345.14226336)
\curveto(225.44267804,345.2122557)(225.52767795,345.25725566)(225.63767904,345.27726336)
\curveto(225.65767782,345.28725563)(225.6826778,345.28725563)(225.71267904,345.27726336)
\curveto(225.75267773,345.27725564)(225.7826777,345.28225563)(225.80267904,345.29226336)
\lineto(226.55267904,345.29226336)
\lineto(226.74767904,345.29226336)
\curveto(226.82767665,345.30225561)(226.89267659,345.30225561)(226.94267904,345.29226336)
\lineto(227.06267904,345.29226336)
\curveto(227.12267636,345.27225564)(227.1776763,345.25725566)(227.22767904,345.24726336)
\curveto(227.2776762,345.23725568)(227.31767616,345.20725571)(227.34767904,345.15726336)
\curveto(227.38767609,345.10725581)(227.40767607,345.03725588)(227.40767904,344.94726336)
\curveto(227.41767606,344.85725606)(227.42267606,344.76225615)(227.42267904,344.66226336)
\lineto(227.42267904,338.37726336)
}
}
{
\newrgbcolor{curcolor}{0 0 0}
\pscustom[linestyle=none,fillstyle=solid,fillcolor=curcolor]
{
\newpath
\moveto(236.85486654,341.73726336)
\curveto(236.87485797,341.67725924)(236.88485796,341.59225932)(236.88486654,341.48226336)
\curveto(236.88485796,341.37225954)(236.87485797,341.28725963)(236.85486654,341.22726336)
\lineto(236.85486654,341.07726336)
\curveto(236.83485801,340.99725992)(236.82485802,340.91726)(236.82486654,340.83726336)
\curveto(236.83485801,340.75726016)(236.82985802,340.67726024)(236.80986654,340.59726336)
\curveto(236.78985806,340.52726039)(236.77485807,340.46226045)(236.76486654,340.40226336)
\curveto(236.75485809,340.34226057)(236.7448581,340.27726064)(236.73486654,340.20726336)
\curveto(236.69485815,340.09726082)(236.65985819,339.98226093)(236.62986654,339.86226336)
\curveto(236.59985825,339.75226116)(236.55985829,339.64726127)(236.50986654,339.54726336)
\curveto(236.29985855,339.06726185)(236.02485882,338.67726224)(235.68486654,338.37726336)
\curveto(235.3448595,338.07726284)(234.93485991,337.82726309)(234.45486654,337.62726336)
\curveto(234.33486051,337.57726334)(234.20986064,337.54226337)(234.07986654,337.52226336)
\curveto(233.95986089,337.49226342)(233.83486101,337.46226345)(233.70486654,337.43226336)
\curveto(233.65486119,337.4122635)(233.59986125,337.40226351)(233.53986654,337.40226336)
\curveto(233.47986137,337.40226351)(233.42486142,337.39726352)(233.37486654,337.38726336)
\lineto(233.26986654,337.38726336)
\curveto(233.23986161,337.37726354)(233.20986164,337.37226354)(233.17986654,337.37226336)
\curveto(233.12986172,337.36226355)(233.0498618,337.35726356)(232.93986654,337.35726336)
\curveto(232.82986202,337.34726357)(232.7448621,337.35226356)(232.68486654,337.37226336)
\lineto(232.53486654,337.37226336)
\curveto(232.48486236,337.38226353)(232.42986242,337.38726353)(232.36986654,337.38726336)
\curveto(232.31986253,337.37726354)(232.26986258,337.38226353)(232.21986654,337.40226336)
\curveto(232.17986267,337.4122635)(232.13986271,337.4172635)(232.09986654,337.41726336)
\curveto(232.06986278,337.4172635)(232.02986282,337.42226349)(231.97986654,337.43226336)
\curveto(231.87986297,337.46226345)(231.77986307,337.48726343)(231.67986654,337.50726336)
\curveto(231.57986327,337.52726339)(231.48486336,337.55726336)(231.39486654,337.59726336)
\curveto(231.27486357,337.63726328)(231.15986369,337.67726324)(231.04986654,337.71726336)
\curveto(230.9498639,337.75726316)(230.844864,337.80726311)(230.73486654,337.86726336)
\curveto(230.38486446,338.07726284)(230.08486476,338.32226259)(229.83486654,338.60226336)
\curveto(229.58486526,338.88226203)(229.37486547,339.2172617)(229.20486654,339.60726336)
\curveto(229.15486569,339.69726122)(229.11486573,339.79226112)(229.08486654,339.89226336)
\curveto(229.06486578,339.99226092)(229.03986581,340.09726082)(229.00986654,340.20726336)
\curveto(228.98986586,340.25726066)(228.97986587,340.30226061)(228.97986654,340.34226336)
\curveto(228.97986587,340.38226053)(228.96986588,340.42726049)(228.94986654,340.47726336)
\curveto(228.92986592,340.55726036)(228.91986593,340.63726028)(228.91986654,340.71726336)
\curveto(228.91986593,340.80726011)(228.90986594,340.89226002)(228.88986654,340.97226336)
\curveto(228.87986597,341.02225989)(228.87486597,341.06725985)(228.87486654,341.10726336)
\lineto(228.87486654,341.24226336)
\curveto(228.85486599,341.30225961)(228.844866,341.38725953)(228.84486654,341.49726336)
\curveto(228.85486599,341.60725931)(228.86986598,341.69225922)(228.88986654,341.75226336)
\lineto(228.88986654,341.85726336)
\curveto(228.89986595,341.90725901)(228.89986595,341.95725896)(228.88986654,342.00726336)
\curveto(228.88986596,342.06725885)(228.89986595,342.12225879)(228.91986654,342.17226336)
\curveto(228.92986592,342.22225869)(228.93486591,342.26725865)(228.93486654,342.30726336)
\curveto(228.93486591,342.35725856)(228.9448659,342.40725851)(228.96486654,342.45726336)
\curveto(229.00486584,342.58725833)(229.03986581,342.7122582)(229.06986654,342.83226336)
\curveto(229.09986575,342.96225795)(229.13986571,343.08725783)(229.18986654,343.20726336)
\curveto(229.36986548,343.6172573)(229.58486526,343.95725696)(229.83486654,344.22726336)
\curveto(230.08486476,344.50725641)(230.38986446,344.76225615)(230.74986654,344.99226336)
\curveto(230.849864,345.04225587)(230.95486389,345.08725583)(231.06486654,345.12726336)
\curveto(231.17486367,345.16725575)(231.28486356,345.2122557)(231.39486654,345.26226336)
\curveto(231.52486332,345.3122556)(231.65986319,345.34725557)(231.79986654,345.36726336)
\curveto(231.93986291,345.38725553)(232.08486276,345.4172555)(232.23486654,345.45726336)
\curveto(232.31486253,345.46725545)(232.38986246,345.47225544)(232.45986654,345.47226336)
\curveto(232.52986232,345.47225544)(232.59986225,345.47725544)(232.66986654,345.48726336)
\curveto(233.2498616,345.49725542)(233.7498611,345.43725548)(234.16986654,345.30726336)
\curveto(234.59986025,345.17725574)(234.97985987,344.99725592)(235.30986654,344.76726336)
\curveto(235.41985943,344.68725623)(235.52985932,344.59725632)(235.63986654,344.49726336)
\curveto(235.75985909,344.40725651)(235.85985899,344.30725661)(235.93986654,344.19726336)
\curveto(236.01985883,344.09725682)(236.08985876,343.99725692)(236.14986654,343.89726336)
\curveto(236.21985863,343.79725712)(236.28985856,343.69225722)(236.35986654,343.58226336)
\curveto(236.42985842,343.47225744)(236.48485836,343.35225756)(236.52486654,343.22226336)
\curveto(236.56485828,343.10225781)(236.60985824,342.97225794)(236.65986654,342.83226336)
\curveto(236.68985816,342.75225816)(236.71485813,342.66725825)(236.73486654,342.57726336)
\lineto(236.79486654,342.30726336)
\curveto(236.80485804,342.26725865)(236.80985804,342.22725869)(236.80986654,342.18726336)
\curveto(236.80985804,342.14725877)(236.81485803,342.10725881)(236.82486654,342.06726336)
\curveto(236.844858,342.0172589)(236.849858,341.96225895)(236.83986654,341.90226336)
\curveto(236.82985802,341.84225907)(236.83485801,341.78725913)(236.85486654,341.73726336)
\moveto(234.75486654,341.19726336)
\curveto(234.76486008,341.24725967)(234.76986008,341.3172596)(234.76986654,341.40726336)
\curveto(234.76986008,341.50725941)(234.76486008,341.58225933)(234.75486654,341.63226336)
\lineto(234.75486654,341.75226336)
\curveto(234.73486011,341.80225911)(234.72486012,341.85725906)(234.72486654,341.91726336)
\curveto(234.72486012,341.97725894)(234.71986013,342.03225888)(234.70986654,342.08226336)
\curveto(234.70986014,342.12225879)(234.70486014,342.15225876)(234.69486654,342.17226336)
\lineto(234.63486654,342.41226336)
\curveto(234.62486022,342.50225841)(234.60486024,342.58725833)(234.57486654,342.66726336)
\curveto(234.46486038,342.92725799)(234.33486051,343.14725777)(234.18486654,343.32726336)
\curveto(234.03486081,343.5172574)(233.83486101,343.66725725)(233.58486654,343.77726336)
\curveto(233.52486132,343.79725712)(233.46486138,343.8122571)(233.40486654,343.82226336)
\curveto(233.3448615,343.84225707)(233.27986157,343.86225705)(233.20986654,343.88226336)
\curveto(233.12986172,343.90225701)(233.0448618,343.90725701)(232.95486654,343.89726336)
\lineto(232.68486654,343.89726336)
\curveto(232.65486219,343.87725704)(232.61986223,343.86725705)(232.57986654,343.86726336)
\curveto(232.53986231,343.87725704)(232.50486234,343.87725704)(232.47486654,343.86726336)
\lineto(232.26486654,343.80726336)
\curveto(232.20486264,343.79725712)(232.1498627,343.77725714)(232.09986654,343.74726336)
\curveto(231.849863,343.63725728)(231.6448632,343.47725744)(231.48486654,343.26726336)
\curveto(231.33486351,343.06725785)(231.21486363,342.83225808)(231.12486654,342.56226336)
\curveto(231.09486375,342.46225845)(231.06986378,342.35725856)(231.04986654,342.24726336)
\curveto(231.03986381,342.13725878)(231.02486382,342.02725889)(231.00486654,341.91726336)
\curveto(230.99486385,341.86725905)(230.98986386,341.8172591)(230.98986654,341.76726336)
\lineto(230.98986654,341.61726336)
\curveto(230.96986388,341.54725937)(230.95986389,341.44225947)(230.95986654,341.30226336)
\curveto(230.96986388,341.16225975)(230.98486386,341.05725986)(231.00486654,340.98726336)
\lineto(231.00486654,340.85226336)
\curveto(231.02486382,340.77226014)(231.03986381,340.69226022)(231.04986654,340.61226336)
\curveto(231.05986379,340.54226037)(231.07486377,340.46726045)(231.09486654,340.38726336)
\curveto(231.19486365,340.08726083)(231.29986355,339.84226107)(231.40986654,339.65226336)
\curveto(231.52986332,339.47226144)(231.71486313,339.30726161)(231.96486654,339.15726336)
\curveto(232.03486281,339.10726181)(232.10986274,339.06726185)(232.18986654,339.03726336)
\curveto(232.27986257,339.00726191)(232.36986248,338.98226193)(232.45986654,338.96226336)
\curveto(232.49986235,338.95226196)(232.53486231,338.94726197)(232.56486654,338.94726336)
\curveto(232.59486225,338.95726196)(232.62986222,338.95726196)(232.66986654,338.94726336)
\lineto(232.78986654,338.91726336)
\curveto(232.83986201,338.917262)(232.88486196,338.92226199)(232.92486654,338.93226336)
\lineto(233.04486654,338.93226336)
\curveto(233.12486172,338.95226196)(233.20486164,338.96726195)(233.28486654,338.97726336)
\curveto(233.36486148,338.98726193)(233.43986141,339.00726191)(233.50986654,339.03726336)
\curveto(233.76986108,339.13726178)(233.97986087,339.27226164)(234.13986654,339.44226336)
\curveto(234.29986055,339.6122613)(234.43486041,339.82226109)(234.54486654,340.07226336)
\curveto(234.58486026,340.17226074)(234.61486023,340.27226064)(234.63486654,340.37226336)
\curveto(234.65486019,340.47226044)(234.67986017,340.57726034)(234.70986654,340.68726336)
\curveto(234.71986013,340.72726019)(234.72486012,340.76226015)(234.72486654,340.79226336)
\curveto(234.72486012,340.83226008)(234.72986012,340.87226004)(234.73986654,340.91226336)
\lineto(234.73986654,341.04726336)
\curveto(234.73986011,341.09725982)(234.7448601,341.14725977)(234.75486654,341.19726336)
}
}
{
\newrgbcolor{curcolor}{0 0 0}
\pscustom[linestyle=none,fillstyle=solid,fillcolor=curcolor]
{
\newpath
\moveto(241.22478842,345.50226336)
\curveto(241.97478392,345.52225539)(242.62478327,345.43725548)(243.17478842,345.24726336)
\curveto(243.73478216,345.06725585)(244.15978173,344.75225616)(244.44978842,344.30226336)
\curveto(244.51978137,344.19225672)(244.57978131,344.07725684)(244.62978842,343.95726336)
\curveto(244.6897812,343.84725707)(244.73978115,343.72225719)(244.77978842,343.58226336)
\curveto(244.79978109,343.52225739)(244.80978108,343.45725746)(244.80978842,343.38726336)
\curveto(244.80978108,343.3172576)(244.79978109,343.25725766)(244.77978842,343.20726336)
\curveto(244.73978115,343.14725777)(244.68478121,343.10725781)(244.61478842,343.08726336)
\curveto(244.56478133,343.06725785)(244.50478139,343.05725786)(244.43478842,343.05726336)
\lineto(244.22478842,343.05726336)
\lineto(243.56478842,343.05726336)
\curveto(243.4947824,343.05725786)(243.42478247,343.05225786)(243.35478842,343.04226336)
\curveto(243.28478261,343.04225787)(243.21978267,343.05225786)(243.15978842,343.07226336)
\curveto(243.05978283,343.09225782)(242.98478291,343.13225778)(242.93478842,343.19226336)
\curveto(242.88478301,343.25225766)(242.83978305,343.3122576)(242.79978842,343.37226336)
\lineto(242.67978842,343.58226336)
\curveto(242.64978324,343.66225725)(242.59978329,343.72725719)(242.52978842,343.77726336)
\curveto(242.42978346,343.85725706)(242.32978356,343.917257)(242.22978842,343.95726336)
\curveto(242.13978375,343.99725692)(242.02478387,344.03225688)(241.88478842,344.06226336)
\curveto(241.81478408,344.08225683)(241.70978418,344.09725682)(241.56978842,344.10726336)
\curveto(241.43978445,344.1172568)(241.33978455,344.1122568)(241.26978842,344.09226336)
\lineto(241.16478842,344.09226336)
\lineto(241.01478842,344.06226336)
\curveto(240.97478492,344.06225685)(240.92978496,344.05725686)(240.87978842,344.04726336)
\curveto(240.70978518,343.99725692)(240.56978532,343.92725699)(240.45978842,343.83726336)
\curveto(240.35978553,343.75725716)(240.2897856,343.63225728)(240.24978842,343.46226336)
\curveto(240.22978566,343.39225752)(240.22978566,343.32725759)(240.24978842,343.26726336)
\curveto(240.26978562,343.20725771)(240.2897856,343.15725776)(240.30978842,343.11726336)
\curveto(240.37978551,342.99725792)(240.45978543,342.90225801)(240.54978842,342.83226336)
\curveto(240.64978524,342.76225815)(240.76478513,342.70225821)(240.89478842,342.65226336)
\curveto(241.08478481,342.57225834)(241.2897846,342.50225841)(241.50978842,342.44226336)
\lineto(242.19978842,342.29226336)
\curveto(242.43978345,342.25225866)(242.66978322,342.20225871)(242.88978842,342.14226336)
\curveto(243.11978277,342.09225882)(243.33478256,342.02725889)(243.53478842,341.94726336)
\curveto(243.62478227,341.90725901)(243.70978218,341.87225904)(243.78978842,341.84226336)
\curveto(243.87978201,341.82225909)(243.96478193,341.78725913)(244.04478842,341.73726336)
\curveto(244.23478166,341.6172593)(244.40478149,341.48725943)(244.55478842,341.34726336)
\curveto(244.71478118,341.20725971)(244.83978105,341.03225988)(244.92978842,340.82226336)
\curveto(244.95978093,340.75226016)(244.98478091,340.68226023)(245.00478842,340.61226336)
\curveto(245.02478087,340.54226037)(245.04478085,340.46726045)(245.06478842,340.38726336)
\curveto(245.07478082,340.32726059)(245.07978081,340.23226068)(245.07978842,340.10226336)
\curveto(245.0897808,339.98226093)(245.0897808,339.88726103)(245.07978842,339.81726336)
\lineto(245.07978842,339.74226336)
\curveto(245.05978083,339.68226123)(245.04478085,339.62226129)(245.03478842,339.56226336)
\curveto(245.03478086,339.5122614)(245.02978086,339.46226145)(245.01978842,339.41226336)
\curveto(244.94978094,339.1122618)(244.83978105,338.84726207)(244.68978842,338.61726336)
\curveto(244.52978136,338.37726254)(244.33478156,338.18226273)(244.10478842,338.03226336)
\curveto(243.87478202,337.88226303)(243.61478228,337.75226316)(243.32478842,337.64226336)
\curveto(243.21478268,337.59226332)(243.0947828,337.55726336)(242.96478842,337.53726336)
\curveto(242.84478305,337.5172634)(242.72478317,337.49226342)(242.60478842,337.46226336)
\curveto(242.51478338,337.44226347)(242.41978347,337.43226348)(242.31978842,337.43226336)
\curveto(242.22978366,337.42226349)(242.13978375,337.40726351)(242.04978842,337.38726336)
\lineto(241.77978842,337.38726336)
\curveto(241.71978417,337.36726355)(241.61478428,337.35726356)(241.46478842,337.35726336)
\curveto(241.32478457,337.35726356)(241.22478467,337.36726355)(241.16478842,337.38726336)
\curveto(241.13478476,337.38726353)(241.09978479,337.39226352)(241.05978842,337.40226336)
\lineto(240.95478842,337.40226336)
\curveto(240.83478506,337.42226349)(240.71478518,337.43726348)(240.59478842,337.44726336)
\curveto(240.47478542,337.45726346)(240.35978553,337.47726344)(240.24978842,337.50726336)
\curveto(239.85978603,337.6172633)(239.51478638,337.74226317)(239.21478842,337.88226336)
\curveto(238.91478698,338.03226288)(238.65978723,338.25226266)(238.44978842,338.54226336)
\curveto(238.30978758,338.73226218)(238.1897877,338.95226196)(238.08978842,339.20226336)
\curveto(238.06978782,339.26226165)(238.04978784,339.34226157)(238.02978842,339.44226336)
\curveto(238.00978788,339.49226142)(237.9947879,339.56226135)(237.98478842,339.65226336)
\curveto(237.97478792,339.74226117)(237.97978791,339.8172611)(237.99978842,339.87726336)
\curveto(238.02978786,339.94726097)(238.07978781,339.99726092)(238.14978842,340.02726336)
\curveto(238.19978769,340.04726087)(238.25978763,340.05726086)(238.32978842,340.05726336)
\lineto(238.55478842,340.05726336)
\lineto(239.25978842,340.05726336)
\lineto(239.49978842,340.05726336)
\curveto(239.57978631,340.05726086)(239.64978624,340.04726087)(239.70978842,340.02726336)
\curveto(239.81978607,339.98726093)(239.889786,339.92226099)(239.91978842,339.83226336)
\curveto(239.95978593,339.74226117)(240.00478589,339.64726127)(240.05478842,339.54726336)
\curveto(240.07478582,339.49726142)(240.10978578,339.43226148)(240.15978842,339.35226336)
\curveto(240.21978567,339.27226164)(240.26978562,339.22226169)(240.30978842,339.20226336)
\curveto(240.42978546,339.10226181)(240.54478535,339.02226189)(240.65478842,338.96226336)
\curveto(240.76478513,338.912262)(240.90478499,338.86226205)(241.07478842,338.81226336)
\curveto(241.12478477,338.79226212)(241.17478472,338.78226213)(241.22478842,338.78226336)
\curveto(241.27478462,338.79226212)(241.32478457,338.79226212)(241.37478842,338.78226336)
\curveto(241.45478444,338.76226215)(241.53978435,338.75226216)(241.62978842,338.75226336)
\curveto(241.72978416,338.76226215)(241.81478408,338.77726214)(241.88478842,338.79726336)
\curveto(241.93478396,338.80726211)(241.97978391,338.8122621)(242.01978842,338.81226336)
\curveto(242.06978382,338.8122621)(242.11978377,338.82226209)(242.16978842,338.84226336)
\curveto(242.30978358,338.89226202)(242.43478346,338.95226196)(242.54478842,339.02226336)
\curveto(242.66478323,339.09226182)(242.75978313,339.18226173)(242.82978842,339.29226336)
\curveto(242.87978301,339.37226154)(242.91978297,339.49726142)(242.94978842,339.66726336)
\curveto(242.96978292,339.73726118)(242.96978292,339.80226111)(242.94978842,339.86226336)
\curveto(242.92978296,339.92226099)(242.90978298,339.97226094)(242.88978842,340.01226336)
\curveto(242.81978307,340.15226076)(242.72978316,340.25726066)(242.61978842,340.32726336)
\curveto(242.51978337,340.39726052)(242.39978349,340.46226045)(242.25978842,340.52226336)
\curveto(242.06978382,340.60226031)(241.86978402,340.66726025)(241.65978842,340.71726336)
\curveto(241.44978444,340.76726015)(241.23978465,340.82226009)(241.02978842,340.88226336)
\curveto(240.94978494,340.90226001)(240.86478503,340.91726)(240.77478842,340.92726336)
\curveto(240.6947852,340.93725998)(240.61478528,340.95225996)(240.53478842,340.97226336)
\curveto(240.21478568,341.06225985)(239.90978598,341.14725977)(239.61978842,341.22726336)
\curveto(239.32978656,341.3172596)(239.06478683,341.44725947)(238.82478842,341.61726336)
\curveto(238.54478735,341.8172591)(238.33978755,342.08725883)(238.20978842,342.42726336)
\curveto(238.1897877,342.49725842)(238.16978772,342.59225832)(238.14978842,342.71226336)
\curveto(238.12978776,342.78225813)(238.11478778,342.86725805)(238.10478842,342.96726336)
\curveto(238.0947878,343.06725785)(238.09978779,343.15725776)(238.11978842,343.23726336)
\curveto(238.13978775,343.28725763)(238.14478775,343.32725759)(238.13478842,343.35726336)
\curveto(238.12478777,343.39725752)(238.12978776,343.44225747)(238.14978842,343.49226336)
\curveto(238.16978772,343.60225731)(238.1897877,343.70225721)(238.20978842,343.79226336)
\curveto(238.23978765,343.89225702)(238.27478762,343.98725693)(238.31478842,344.07726336)
\curveto(238.44478745,344.36725655)(238.62478727,344.60225631)(238.85478842,344.78226336)
\curveto(239.08478681,344.96225595)(239.34478655,345.10725581)(239.63478842,345.21726336)
\curveto(239.74478615,345.26725565)(239.85978603,345.30225561)(239.97978842,345.32226336)
\curveto(240.09978579,345.35225556)(240.22478567,345.38225553)(240.35478842,345.41226336)
\curveto(240.41478548,345.43225548)(240.47478542,345.44225547)(240.53478842,345.44226336)
\lineto(240.71478842,345.47226336)
\curveto(240.7947851,345.48225543)(240.87978501,345.48725543)(240.96978842,345.48726336)
\curveto(241.05978483,345.48725543)(241.14478475,345.49225542)(241.22478842,345.50226336)
}
}
{
\newrgbcolor{curcolor}{0 0 0}
\pscustom[linestyle=none,fillstyle=solid,fillcolor=curcolor]
{
\newpath
\moveto(26.50611654,322.76226336)
\curveto(26.51610786,322.70225946)(26.52110786,322.61225955)(26.52111654,322.49226336)
\curveto(26.52110786,322.37225979)(26.51110787,322.28725988)(26.49111654,322.23726336)
\lineto(26.49111654,322.04226336)
\curveto(26.46110792,321.93226023)(26.44110794,321.82726034)(26.43111654,321.72726336)
\curveto(26.43110795,321.62726054)(26.41610796,321.52726064)(26.38611654,321.42726336)
\curveto(26.36610801,321.33726083)(26.34610803,321.24226092)(26.32611654,321.14226336)
\curveto(26.30610807,321.05226111)(26.2761081,320.9622612)(26.23611654,320.87226336)
\curveto(26.16610821,320.70226146)(26.09610828,320.54226162)(26.02611654,320.39226336)
\curveto(25.95610842,320.25226191)(25.8761085,320.11226205)(25.78611654,319.97226336)
\curveto(25.72610865,319.88226228)(25.66110872,319.79726237)(25.59111654,319.71726336)
\curveto(25.53110885,319.64726252)(25.46110892,319.57226259)(25.38111654,319.49226336)
\lineto(25.27611654,319.38726336)
\curveto(25.22610915,319.33726283)(25.17110921,319.29226287)(25.11111654,319.25226336)
\lineto(24.96111654,319.13226336)
\curveto(24.8811095,319.07226309)(24.79110959,319.01726315)(24.69111654,318.96726336)
\curveto(24.60110978,318.92726324)(24.50610987,318.88226328)(24.40611654,318.83226336)
\curveto(24.30611007,318.78226338)(24.20111018,318.74726342)(24.09111654,318.72726336)
\curveto(23.99111039,318.70726346)(23.88611049,318.68726348)(23.77611654,318.66726336)
\curveto(23.71611066,318.64726352)(23.65111073,318.63726353)(23.58111654,318.63726336)
\curveto(23.52111086,318.63726353)(23.45611092,318.62726354)(23.38611654,318.60726336)
\lineto(23.25111654,318.60726336)
\curveto(23.17111121,318.58726358)(23.09611128,318.58726358)(23.02611654,318.60726336)
\lineto(22.87611654,318.60726336)
\curveto(22.81611156,318.62726354)(22.75111163,318.63726353)(22.68111654,318.63726336)
\curveto(22.62111176,318.62726354)(22.56111182,318.63226353)(22.50111654,318.65226336)
\curveto(22.34111204,318.70226346)(22.18611219,318.74726342)(22.03611654,318.78726336)
\curveto(21.89611248,318.82726334)(21.76611261,318.88726328)(21.64611654,318.96726336)
\curveto(21.5761128,319.00726316)(21.51111287,319.04726312)(21.45111654,319.08726336)
\curveto(21.39111299,319.13726303)(21.32611305,319.18726298)(21.25611654,319.23726336)
\lineto(21.07611654,319.37226336)
\curveto(20.99611338,319.43226273)(20.92611345,319.43726273)(20.86611654,319.38726336)
\curveto(20.81611356,319.35726281)(20.79111359,319.31726285)(20.79111654,319.26726336)
\curveto(20.79111359,319.22726294)(20.7811136,319.17726299)(20.76111654,319.11726336)
\curveto(20.74111364,319.01726315)(20.73111365,318.90226326)(20.73111654,318.77226336)
\curveto(20.74111364,318.64226352)(20.74611363,318.52226364)(20.74611654,318.41226336)
\lineto(20.74611654,316.88226336)
\curveto(20.74611363,316.75226541)(20.74111364,316.62726554)(20.73111654,316.50726336)
\curveto(20.73111365,316.37726579)(20.70611367,316.27226589)(20.65611654,316.19226336)
\curveto(20.62611375,316.15226601)(20.57111381,316.12226604)(20.49111654,316.10226336)
\curveto(20.41111397,316.08226608)(20.32111406,316.07226609)(20.22111654,316.07226336)
\curveto(20.12111426,316.0622661)(20.02111436,316.0622661)(19.92111654,316.07226336)
\lineto(19.66611654,316.07226336)
\lineto(19.26111654,316.07226336)
\lineto(19.15611654,316.07226336)
\curveto(19.11611526,316.07226609)(19.0811153,316.07726609)(19.05111654,316.08726336)
\lineto(18.93111654,316.08726336)
\curveto(18.76111562,316.13726603)(18.67111571,316.23726593)(18.66111654,316.38726336)
\curveto(18.65111573,316.52726564)(18.64611573,316.69726547)(18.64611654,316.89726336)
\lineto(18.64611654,325.70226336)
\curveto(18.64611573,325.81225635)(18.64111574,325.92725624)(18.63111654,326.04726336)
\curveto(18.63111575,326.17725599)(18.65611572,326.27725589)(18.70611654,326.34726336)
\curveto(18.74611563,326.41725575)(18.80111558,326.4622557)(18.87111654,326.48226336)
\curveto(18.92111546,326.50225566)(18.9811154,326.51225565)(19.05111654,326.51226336)
\lineto(19.27611654,326.51226336)
\lineto(19.99611654,326.51226336)
\lineto(20.28111654,326.51226336)
\curveto(20.37111401,326.51225565)(20.44611393,326.48725568)(20.50611654,326.43726336)
\curveto(20.5761138,326.38725578)(20.61111377,326.32225584)(20.61111654,326.24226336)
\curveto(20.62111376,326.17225599)(20.64611373,326.09725607)(20.68611654,326.01726336)
\curveto(20.69611368,325.98725618)(20.70611367,325.9622562)(20.71611654,325.94226336)
\curveto(20.73611364,325.93225623)(20.75611362,325.91725625)(20.77611654,325.89726336)
\curveto(20.88611349,325.88725628)(20.9761134,325.91725625)(21.04611654,325.98726336)
\curveto(21.11611326,326.05725611)(21.18611319,326.11725605)(21.25611654,326.16726336)
\curveto(21.38611299,326.25725591)(21.52111286,326.33725583)(21.66111654,326.40726336)
\curveto(21.80111258,326.48725568)(21.95611242,326.55225561)(22.12611654,326.60226336)
\curveto(22.20611217,326.63225553)(22.29111209,326.65225551)(22.38111654,326.66226336)
\curveto(22.4811119,326.67225549)(22.5761118,326.68725548)(22.66611654,326.70726336)
\curveto(22.70611167,326.71725545)(22.74611163,326.71725545)(22.78611654,326.70726336)
\curveto(22.83611154,326.69725547)(22.8761115,326.70225546)(22.90611654,326.72226336)
\curveto(23.4761109,326.74225542)(23.95611042,326.6622555)(24.34611654,326.48226336)
\curveto(24.74610963,326.31225585)(25.08610929,326.08725608)(25.36611654,325.80726336)
\curveto(25.41610896,325.75725641)(25.46110892,325.70725646)(25.50111654,325.65726336)
\curveto(25.54110884,325.61725655)(25.5811088,325.57225659)(25.62111654,325.52226336)
\curveto(25.69110869,325.43225673)(25.75110863,325.34225682)(25.80111654,325.25226336)
\curveto(25.86110852,325.162257)(25.91610846,325.07225709)(25.96611654,324.98226336)
\curveto(25.98610839,324.9622572)(25.99610838,324.93725723)(25.99611654,324.90726336)
\curveto(26.00610837,324.87725729)(26.02110836,324.84225732)(26.04111654,324.80226336)
\curveto(26.10110828,324.70225746)(26.15610822,324.58225758)(26.20611654,324.44226336)
\curveto(26.22610815,324.38225778)(26.24610813,324.31725785)(26.26611654,324.24726336)
\curveto(26.28610809,324.18725798)(26.30610807,324.12225804)(26.32611654,324.05226336)
\curveto(26.36610801,323.93225823)(26.39110799,323.80725836)(26.40111654,323.67726336)
\curveto(26.42110796,323.54725862)(26.44610793,323.41225875)(26.47611654,323.27226336)
\lineto(26.47611654,323.10726336)
\lineto(26.50611654,322.92726336)
\lineto(26.50611654,322.76226336)
\moveto(24.39111654,322.41726336)
\curveto(24.40110998,322.4672597)(24.40610997,322.53225963)(24.40611654,322.61226336)
\curveto(24.40610997,322.70225946)(24.40110998,322.77225939)(24.39111654,322.82226336)
\lineto(24.39111654,322.95726336)
\curveto(24.37111001,323.01725915)(24.36111002,323.08225908)(24.36111654,323.15226336)
\curveto(24.36111002,323.22225894)(24.35111003,323.29225887)(24.33111654,323.36226336)
\curveto(24.31111007,323.4622587)(24.29111009,323.55725861)(24.27111654,323.64726336)
\curveto(24.25111013,323.74725842)(24.22111016,323.83725833)(24.18111654,323.91726336)
\curveto(24.06111032,324.23725793)(23.90611047,324.49225767)(23.71611654,324.68226336)
\curveto(23.52611085,324.87225729)(23.25611112,325.01225715)(22.90611654,325.10226336)
\curveto(22.82611155,325.12225704)(22.73611164,325.13225703)(22.63611654,325.13226336)
\lineto(22.36611654,325.13226336)
\curveto(22.32611205,325.12225704)(22.29111209,325.11725705)(22.26111654,325.11726336)
\curveto(22.23111215,325.11725705)(22.19611218,325.11225705)(22.15611654,325.10226336)
\lineto(21.94611654,325.04226336)
\curveto(21.88611249,325.03225713)(21.82611255,325.01225715)(21.76611654,324.98226336)
\curveto(21.50611287,324.87225729)(21.30111308,324.70225746)(21.15111654,324.47226336)
\curveto(21.01111337,324.24225792)(20.89611348,323.98725818)(20.80611654,323.70726336)
\curveto(20.78611359,323.62725854)(20.77111361,323.54225862)(20.76111654,323.45226336)
\curveto(20.75111363,323.37225879)(20.73611364,323.29225887)(20.71611654,323.21226336)
\curveto(20.70611367,323.17225899)(20.70111368,323.10725906)(20.70111654,323.01726336)
\curveto(20.6811137,322.97725919)(20.6761137,322.92725924)(20.68611654,322.86726336)
\curveto(20.69611368,322.81725935)(20.69611368,322.7672594)(20.68611654,322.71726336)
\curveto(20.66611371,322.65725951)(20.66611371,322.60225956)(20.68611654,322.55226336)
\lineto(20.68611654,322.37226336)
\lineto(20.68611654,322.23726336)
\curveto(20.68611369,322.19725997)(20.69611368,322.15726001)(20.71611654,322.11726336)
\curveto(20.71611366,322.04726012)(20.72111366,321.99226017)(20.73111654,321.95226336)
\lineto(20.76111654,321.77226336)
\curveto(20.77111361,321.71226045)(20.78611359,321.65226051)(20.80611654,321.59226336)
\curveto(20.89611348,321.30226086)(21.00111338,321.0622611)(21.12111654,320.87226336)
\curveto(21.25111313,320.69226147)(21.43111295,320.53226163)(21.66111654,320.39226336)
\curveto(21.80111258,320.31226185)(21.96611241,320.24726192)(22.15611654,320.19726336)
\curveto(22.19611218,320.18726198)(22.23111215,320.18226198)(22.26111654,320.18226336)
\curveto(22.29111209,320.19226197)(22.32611205,320.19226197)(22.36611654,320.18226336)
\curveto(22.40611197,320.17226199)(22.46611191,320.162262)(22.54611654,320.15226336)
\curveto(22.62611175,320.15226201)(22.69111169,320.15726201)(22.74111654,320.16726336)
\curveto(22.82111156,320.18726198)(22.90111148,320.20226196)(22.98111654,320.21226336)
\curveto(23.07111131,320.23226193)(23.15611122,320.25726191)(23.23611654,320.28726336)
\curveto(23.4761109,320.38726178)(23.67111071,320.52726164)(23.82111654,320.70726336)
\curveto(23.97111041,320.88726128)(24.09611028,321.09726107)(24.19611654,321.33726336)
\curveto(24.24611013,321.45726071)(24.2811101,321.58226058)(24.30111654,321.71226336)
\curveto(24.32111006,321.84226032)(24.34611003,321.97726019)(24.37611654,322.11726336)
\lineto(24.37611654,322.26726336)
\curveto(24.38610999,322.31725985)(24.39110999,322.3672598)(24.39111654,322.41726336)
}
}
{
\newrgbcolor{curcolor}{0 0 0}
\pscustom[linestyle=none,fillstyle=solid,fillcolor=curcolor]
{
\newpath
\moveto(35.55603842,322.98726336)
\curveto(35.57602985,322.92725924)(35.58602984,322.84225932)(35.58603842,322.73226336)
\curveto(35.58602984,322.62225954)(35.57602985,322.53725963)(35.55603842,322.47726336)
\lineto(35.55603842,322.32726336)
\curveto(35.53602989,322.24725992)(35.5260299,322.16726)(35.52603842,322.08726336)
\curveto(35.53602989,322.00726016)(35.53102989,321.92726024)(35.51103842,321.84726336)
\curveto(35.49102993,321.77726039)(35.47602995,321.71226045)(35.46603842,321.65226336)
\curveto(35.45602997,321.59226057)(35.44602998,321.52726064)(35.43603842,321.45726336)
\curveto(35.39603003,321.34726082)(35.36103006,321.23226093)(35.33103842,321.11226336)
\curveto(35.30103012,321.00226116)(35.26103016,320.89726127)(35.21103842,320.79726336)
\curveto(35.00103042,320.31726185)(34.7260307,319.92726224)(34.38603842,319.62726336)
\curveto(34.04603138,319.32726284)(33.63603179,319.07726309)(33.15603842,318.87726336)
\curveto(33.03603239,318.82726334)(32.91103251,318.79226337)(32.78103842,318.77226336)
\curveto(32.66103276,318.74226342)(32.53603289,318.71226345)(32.40603842,318.68226336)
\curveto(32.35603307,318.6622635)(32.30103312,318.65226351)(32.24103842,318.65226336)
\curveto(32.18103324,318.65226351)(32.1260333,318.64726352)(32.07603842,318.63726336)
\lineto(31.97103842,318.63726336)
\curveto(31.94103348,318.62726354)(31.91103351,318.62226354)(31.88103842,318.62226336)
\curveto(31.83103359,318.61226355)(31.75103367,318.60726356)(31.64103842,318.60726336)
\curveto(31.53103389,318.59726357)(31.44603398,318.60226356)(31.38603842,318.62226336)
\lineto(31.23603842,318.62226336)
\curveto(31.18603424,318.63226353)(31.13103429,318.63726353)(31.07103842,318.63726336)
\curveto(31.0210344,318.62726354)(30.97103445,318.63226353)(30.92103842,318.65226336)
\curveto(30.88103454,318.6622635)(30.84103458,318.6672635)(30.80103842,318.66726336)
\curveto(30.77103465,318.6672635)(30.73103469,318.67226349)(30.68103842,318.68226336)
\curveto(30.58103484,318.71226345)(30.48103494,318.73726343)(30.38103842,318.75726336)
\curveto(30.28103514,318.77726339)(30.18603524,318.80726336)(30.09603842,318.84726336)
\curveto(29.97603545,318.88726328)(29.86103556,318.92726324)(29.75103842,318.96726336)
\curveto(29.65103577,319.00726316)(29.54603588,319.05726311)(29.43603842,319.11726336)
\curveto(29.08603634,319.32726284)(28.78603664,319.57226259)(28.53603842,319.85226336)
\curveto(28.28603714,320.13226203)(28.07603735,320.4672617)(27.90603842,320.85726336)
\curveto(27.85603757,320.94726122)(27.81603761,321.04226112)(27.78603842,321.14226336)
\curveto(27.76603766,321.24226092)(27.74103768,321.34726082)(27.71103842,321.45726336)
\curveto(27.69103773,321.50726066)(27.68103774,321.55226061)(27.68103842,321.59226336)
\curveto(27.68103774,321.63226053)(27.67103775,321.67726049)(27.65103842,321.72726336)
\curveto(27.63103779,321.80726036)(27.6210378,321.88726028)(27.62103842,321.96726336)
\curveto(27.6210378,322.05726011)(27.61103781,322.14226002)(27.59103842,322.22226336)
\curveto(27.58103784,322.27225989)(27.57603785,322.31725985)(27.57603842,322.35726336)
\lineto(27.57603842,322.49226336)
\curveto(27.55603787,322.55225961)(27.54603788,322.63725953)(27.54603842,322.74726336)
\curveto(27.55603787,322.85725931)(27.57103785,322.94225922)(27.59103842,323.00226336)
\lineto(27.59103842,323.10726336)
\curveto(27.60103782,323.15725901)(27.60103782,323.20725896)(27.59103842,323.25726336)
\curveto(27.59103783,323.31725885)(27.60103782,323.37225879)(27.62103842,323.42226336)
\curveto(27.63103779,323.47225869)(27.63603779,323.51725865)(27.63603842,323.55726336)
\curveto(27.63603779,323.60725856)(27.64603778,323.65725851)(27.66603842,323.70726336)
\curveto(27.70603772,323.83725833)(27.74103768,323.9622582)(27.77103842,324.08226336)
\curveto(27.80103762,324.21225795)(27.84103758,324.33725783)(27.89103842,324.45726336)
\curveto(28.07103735,324.8672573)(28.28603714,325.20725696)(28.53603842,325.47726336)
\curveto(28.78603664,325.75725641)(29.09103633,326.01225615)(29.45103842,326.24226336)
\curveto(29.55103587,326.29225587)(29.65603577,326.33725583)(29.76603842,326.37726336)
\curveto(29.87603555,326.41725575)(29.98603544,326.4622557)(30.09603842,326.51226336)
\curveto(30.2260352,326.5622556)(30.36103506,326.59725557)(30.50103842,326.61726336)
\curveto(30.64103478,326.63725553)(30.78603464,326.6672555)(30.93603842,326.70726336)
\curveto(31.01603441,326.71725545)(31.09103433,326.72225544)(31.16103842,326.72226336)
\curveto(31.23103419,326.72225544)(31.30103412,326.72725544)(31.37103842,326.73726336)
\curveto(31.95103347,326.74725542)(32.45103297,326.68725548)(32.87103842,326.55726336)
\curveto(33.30103212,326.42725574)(33.68103174,326.24725592)(34.01103842,326.01726336)
\curveto(34.1210313,325.93725623)(34.23103119,325.84725632)(34.34103842,325.74726336)
\curveto(34.46103096,325.65725651)(34.56103086,325.55725661)(34.64103842,325.44726336)
\curveto(34.7210307,325.34725682)(34.79103063,325.24725692)(34.85103842,325.14726336)
\curveto(34.9210305,325.04725712)(34.99103043,324.94225722)(35.06103842,324.83226336)
\curveto(35.13103029,324.72225744)(35.18603024,324.60225756)(35.22603842,324.47226336)
\curveto(35.26603016,324.35225781)(35.31103011,324.22225794)(35.36103842,324.08226336)
\curveto(35.39103003,324.00225816)(35.41603001,323.91725825)(35.43603842,323.82726336)
\lineto(35.49603842,323.55726336)
\curveto(35.50602992,323.51725865)(35.51102991,323.47725869)(35.51103842,323.43726336)
\curveto(35.51102991,323.39725877)(35.51602991,323.35725881)(35.52603842,323.31726336)
\curveto(35.54602988,323.2672589)(35.55102987,323.21225895)(35.54103842,323.15226336)
\curveto(35.53102989,323.09225907)(35.53602989,323.03725913)(35.55603842,322.98726336)
\moveto(33.45603842,322.44726336)
\curveto(33.46603196,322.49725967)(33.47103195,322.5672596)(33.47103842,322.65726336)
\curveto(33.47103195,322.75725941)(33.46603196,322.83225933)(33.45603842,322.88226336)
\lineto(33.45603842,323.00226336)
\curveto(33.43603199,323.05225911)(33.426032,323.10725906)(33.42603842,323.16726336)
\curveto(33.426032,323.22725894)(33.421032,323.28225888)(33.41103842,323.33226336)
\curveto(33.41103201,323.37225879)(33.40603202,323.40225876)(33.39603842,323.42226336)
\lineto(33.33603842,323.66226336)
\curveto(33.3260321,323.75225841)(33.30603212,323.83725833)(33.27603842,323.91726336)
\curveto(33.16603226,324.17725799)(33.03603239,324.39725777)(32.88603842,324.57726336)
\curveto(32.73603269,324.7672574)(32.53603289,324.91725725)(32.28603842,325.02726336)
\curveto(32.2260332,325.04725712)(32.16603326,325.0622571)(32.10603842,325.07226336)
\curveto(32.04603338,325.09225707)(31.98103344,325.11225705)(31.91103842,325.13226336)
\curveto(31.83103359,325.15225701)(31.74603368,325.15725701)(31.65603842,325.14726336)
\lineto(31.38603842,325.14726336)
\curveto(31.35603407,325.12725704)(31.3210341,325.11725705)(31.28103842,325.11726336)
\curveto(31.24103418,325.12725704)(31.20603422,325.12725704)(31.17603842,325.11726336)
\lineto(30.96603842,325.05726336)
\curveto(30.90603452,325.04725712)(30.85103457,325.02725714)(30.80103842,324.99726336)
\curveto(30.55103487,324.88725728)(30.34603508,324.72725744)(30.18603842,324.51726336)
\curveto(30.03603539,324.31725785)(29.91603551,324.08225808)(29.82603842,323.81226336)
\curveto(29.79603563,323.71225845)(29.77103565,323.60725856)(29.75103842,323.49726336)
\curveto(29.74103568,323.38725878)(29.7260357,323.27725889)(29.70603842,323.16726336)
\curveto(29.69603573,323.11725905)(29.69103573,323.0672591)(29.69103842,323.01726336)
\lineto(29.69103842,322.86726336)
\curveto(29.67103575,322.79725937)(29.66103576,322.69225947)(29.66103842,322.55226336)
\curveto(29.67103575,322.41225975)(29.68603574,322.30725986)(29.70603842,322.23726336)
\lineto(29.70603842,322.10226336)
\curveto(29.7260357,322.02226014)(29.74103568,321.94226022)(29.75103842,321.86226336)
\curveto(29.76103566,321.79226037)(29.77603565,321.71726045)(29.79603842,321.63726336)
\curveto(29.89603553,321.33726083)(30.00103542,321.09226107)(30.11103842,320.90226336)
\curveto(30.23103519,320.72226144)(30.41603501,320.55726161)(30.66603842,320.40726336)
\curveto(30.73603469,320.35726181)(30.81103461,320.31726185)(30.89103842,320.28726336)
\curveto(30.98103444,320.25726191)(31.07103435,320.23226193)(31.16103842,320.21226336)
\curveto(31.20103422,320.20226196)(31.23603419,320.19726197)(31.26603842,320.19726336)
\curveto(31.29603413,320.20726196)(31.33103409,320.20726196)(31.37103842,320.19726336)
\lineto(31.49103842,320.16726336)
\curveto(31.54103388,320.167262)(31.58603384,320.17226199)(31.62603842,320.18226336)
\lineto(31.74603842,320.18226336)
\curveto(31.8260336,320.20226196)(31.90603352,320.21726195)(31.98603842,320.22726336)
\curveto(32.06603336,320.23726193)(32.14103328,320.25726191)(32.21103842,320.28726336)
\curveto(32.47103295,320.38726178)(32.68103274,320.52226164)(32.84103842,320.69226336)
\curveto(33.00103242,320.8622613)(33.13603229,321.07226109)(33.24603842,321.32226336)
\curveto(33.28603214,321.42226074)(33.31603211,321.52226064)(33.33603842,321.62226336)
\curveto(33.35603207,321.72226044)(33.38103204,321.82726034)(33.41103842,321.93726336)
\curveto(33.421032,321.97726019)(33.426032,322.01226015)(33.42603842,322.04226336)
\curveto(33.426032,322.08226008)(33.43103199,322.12226004)(33.44103842,322.16226336)
\lineto(33.44103842,322.29726336)
\curveto(33.44103198,322.34725982)(33.44603198,322.39725977)(33.45603842,322.44726336)
}
}
{
\newrgbcolor{curcolor}{0 0 0}
\pscustom[linestyle=none,fillstyle=solid,fillcolor=curcolor]
{
\newpath
\moveto(41.38096029,326.73726336)
\curveto(41.49095498,326.73725543)(41.58595488,326.72725544)(41.66596029,326.70726336)
\curveto(41.75595471,326.68725548)(41.82595464,326.64225552)(41.87596029,326.57226336)
\curveto(41.93595453,326.49225567)(41.9659545,326.35225581)(41.96596029,326.15226336)
\lineto(41.96596029,325.64226336)
\lineto(41.96596029,325.26726336)
\curveto(41.97595449,325.12725704)(41.96095451,325.01725715)(41.92096029,324.93726336)
\curveto(41.88095459,324.8672573)(41.82095465,324.82225734)(41.74096029,324.80226336)
\curveto(41.6709548,324.78225738)(41.58595488,324.77225739)(41.48596029,324.77226336)
\curveto(41.39595507,324.77225739)(41.29595517,324.77725739)(41.18596029,324.78726336)
\curveto(41.08595538,324.79725737)(40.99095548,324.79225737)(40.90096029,324.77226336)
\curveto(40.83095564,324.75225741)(40.76095571,324.73725743)(40.69096029,324.72726336)
\curveto(40.62095585,324.72725744)(40.55595591,324.71725745)(40.49596029,324.69726336)
\curveto(40.33595613,324.64725752)(40.17595629,324.57225759)(40.01596029,324.47226336)
\curveto(39.85595661,324.38225778)(39.73095674,324.27725789)(39.64096029,324.15726336)
\curveto(39.59095688,324.07725809)(39.53595693,323.99225817)(39.47596029,323.90226336)
\curveto(39.42595704,323.82225834)(39.37595709,323.73725843)(39.32596029,323.64726336)
\curveto(39.29595717,323.5672586)(39.2659572,323.48225868)(39.23596029,323.39226336)
\lineto(39.17596029,323.15226336)
\curveto(39.15595731,323.08225908)(39.14595732,323.00725916)(39.14596029,322.92726336)
\curveto(39.14595732,322.85725931)(39.13595733,322.78725938)(39.11596029,322.71726336)
\curveto(39.10595736,322.67725949)(39.10095737,322.63725953)(39.10096029,322.59726336)
\curveto(39.11095736,322.5672596)(39.11095736,322.53725963)(39.10096029,322.50726336)
\lineto(39.10096029,322.26726336)
\curveto(39.08095739,322.19725997)(39.07595739,322.11726005)(39.08596029,322.02726336)
\curveto(39.09595737,321.94726022)(39.10095737,321.8672603)(39.10096029,321.78726336)
\lineto(39.10096029,320.82726336)
\lineto(39.10096029,319.55226336)
\curveto(39.10095737,319.42226274)(39.09595737,319.30226286)(39.08596029,319.19226336)
\curveto(39.07595739,319.08226308)(39.04595742,318.99226317)(38.99596029,318.92226336)
\curveto(38.97595749,318.89226327)(38.94095753,318.8672633)(38.89096029,318.84726336)
\curveto(38.85095762,318.83726333)(38.80595766,318.82726334)(38.75596029,318.81726336)
\lineto(38.68096029,318.81726336)
\curveto(38.63095784,318.80726336)(38.57595789,318.80226336)(38.51596029,318.80226336)
\lineto(38.35096029,318.80226336)
\lineto(37.70596029,318.80226336)
\curveto(37.64595882,318.81226335)(37.58095889,318.81726335)(37.51096029,318.81726336)
\lineto(37.31596029,318.81726336)
\curveto(37.2659592,318.83726333)(37.21595925,318.85226331)(37.16596029,318.86226336)
\curveto(37.11595935,318.88226328)(37.08095939,318.91726325)(37.06096029,318.96726336)
\curveto(37.02095945,319.01726315)(36.99595947,319.08726308)(36.98596029,319.17726336)
\lineto(36.98596029,319.47726336)
\lineto(36.98596029,320.49726336)
\lineto(36.98596029,324.72726336)
\lineto(36.98596029,325.83726336)
\lineto(36.98596029,326.12226336)
\curveto(36.98595948,326.22225594)(37.00595946,326.30225586)(37.04596029,326.36226336)
\curveto(37.09595937,326.44225572)(37.1709593,326.49225567)(37.27096029,326.51226336)
\curveto(37.3709591,326.53225563)(37.49095898,326.54225562)(37.63096029,326.54226336)
\lineto(38.39596029,326.54226336)
\curveto(38.51595795,326.54225562)(38.62095785,326.53225563)(38.71096029,326.51226336)
\curveto(38.80095767,326.50225566)(38.8709576,326.45725571)(38.92096029,326.37726336)
\curveto(38.95095752,326.32725584)(38.9659575,326.25725591)(38.96596029,326.16726336)
\lineto(38.99596029,325.89726336)
\curveto(39.00595746,325.81725635)(39.02095745,325.74225642)(39.04096029,325.67226336)
\curveto(39.0709574,325.60225656)(39.12095735,325.5672566)(39.19096029,325.56726336)
\curveto(39.21095726,325.58725658)(39.23095724,325.59725657)(39.25096029,325.59726336)
\curveto(39.2709572,325.59725657)(39.29095718,325.60725656)(39.31096029,325.62726336)
\curveto(39.3709571,325.67725649)(39.42095705,325.73225643)(39.46096029,325.79226336)
\curveto(39.51095696,325.8622563)(39.5709569,325.92225624)(39.64096029,325.97226336)
\curveto(39.68095679,326.00225616)(39.71595675,326.03225613)(39.74596029,326.06226336)
\curveto(39.77595669,326.10225606)(39.81095666,326.13725603)(39.85096029,326.16726336)
\lineto(40.12096029,326.34726336)
\curveto(40.22095625,326.40725576)(40.32095615,326.4622557)(40.42096029,326.51226336)
\curveto(40.52095595,326.55225561)(40.62095585,326.58725558)(40.72096029,326.61726336)
\lineto(41.05096029,326.70726336)
\curveto(41.08095539,326.71725545)(41.13595533,326.71725545)(41.21596029,326.70726336)
\curveto(41.30595516,326.70725546)(41.36095511,326.71725545)(41.38096029,326.73726336)
}
}
{
\newrgbcolor{curcolor}{0 0 0}
\pscustom[linestyle=none,fillstyle=solid,fillcolor=curcolor]
{
}
}
{
\newrgbcolor{curcolor}{0 0 0}
\pscustom[linestyle=none,fillstyle=solid,fillcolor=curcolor]
{
\newpath
\moveto(47.99619467,328.85226336)
\lineto(49.00119467,328.85226336)
\curveto(49.15119168,328.85225331)(49.28119155,328.84225332)(49.39119467,328.82226336)
\curveto(49.51119132,328.81225335)(49.59619124,328.75225341)(49.64619467,328.64226336)
\curveto(49.66619117,328.59225357)(49.67619116,328.53225363)(49.67619467,328.46226336)
\lineto(49.67619467,328.25226336)
\lineto(49.67619467,327.57726336)
\curveto(49.67619116,327.52725464)(49.67119116,327.4672547)(49.66119467,327.39726336)
\curveto(49.66119117,327.33725483)(49.66619117,327.28225488)(49.67619467,327.23226336)
\lineto(49.67619467,327.06726336)
\curveto(49.67619116,326.98725518)(49.68119115,326.91225525)(49.69119467,326.84226336)
\curveto(49.70119113,326.78225538)(49.72619111,326.72725544)(49.76619467,326.67726336)
\curveto(49.836191,326.58725558)(49.96119087,326.53725563)(50.14119467,326.52726336)
\lineto(50.68119467,326.52726336)
\lineto(50.86119467,326.52726336)
\curveto(50.92118991,326.52725564)(50.97618986,326.51725565)(51.02619467,326.49726336)
\curveto(51.1361897,326.44725572)(51.19618964,326.35725581)(51.20619467,326.22726336)
\curveto(51.22618961,326.09725607)(51.2361896,325.95225621)(51.23619467,325.79226336)
\lineto(51.23619467,325.58226336)
\curveto(51.24618959,325.51225665)(51.24118959,325.45225671)(51.22119467,325.40226336)
\curveto(51.17118966,325.24225692)(51.06618977,325.15725701)(50.90619467,325.14726336)
\curveto(50.74619009,325.13725703)(50.56619027,325.13225703)(50.36619467,325.13226336)
\lineto(50.23119467,325.13226336)
\curveto(50.19119064,325.14225702)(50.15619068,325.14225702)(50.12619467,325.13226336)
\curveto(50.08619075,325.12225704)(50.05119078,325.11725705)(50.02119467,325.11726336)
\curveto(49.99119084,325.12725704)(49.96119087,325.12225704)(49.93119467,325.10226336)
\curveto(49.85119098,325.08225708)(49.79119104,325.03725713)(49.75119467,324.96726336)
\curveto(49.72119111,324.90725726)(49.69619114,324.83225733)(49.67619467,324.74226336)
\curveto(49.66619117,324.69225747)(49.66619117,324.63725753)(49.67619467,324.57726336)
\curveto(49.68619115,324.51725765)(49.68619115,324.4622577)(49.67619467,324.41226336)
\lineto(49.67619467,323.48226336)
\lineto(49.67619467,321.72726336)
\curveto(49.67619116,321.47726069)(49.68119115,321.25726091)(49.69119467,321.06726336)
\curveto(49.71119112,320.88726128)(49.77619106,320.72726144)(49.88619467,320.58726336)
\curveto(49.9361909,320.52726164)(50.00119083,320.48226168)(50.08119467,320.45226336)
\lineto(50.35119467,320.39226336)
\curveto(50.38119045,320.38226178)(50.41119042,320.37726179)(50.44119467,320.37726336)
\curveto(50.48119035,320.38726178)(50.51119032,320.38726178)(50.53119467,320.37726336)
\lineto(50.69619467,320.37726336)
\curveto(50.80619003,320.37726179)(50.90118993,320.37226179)(50.98119467,320.36226336)
\curveto(51.06118977,320.35226181)(51.12618971,320.31226185)(51.17619467,320.24226336)
\curveto(51.21618962,320.18226198)(51.2361896,320.10226206)(51.23619467,320.00226336)
\lineto(51.23619467,319.71726336)
\curveto(51.2361896,319.50726266)(51.2311896,319.31226285)(51.22119467,319.13226336)
\curveto(51.22118961,318.9622632)(51.14118969,318.84726332)(50.98119467,318.78726336)
\curveto(50.9311899,318.7672634)(50.88618995,318.7622634)(50.84619467,318.77226336)
\curveto(50.80619003,318.77226339)(50.76119007,318.7622634)(50.71119467,318.74226336)
\lineto(50.56119467,318.74226336)
\curveto(50.54119029,318.74226342)(50.51119032,318.74726342)(50.47119467,318.75726336)
\curveto(50.4311904,318.75726341)(50.39619044,318.75226341)(50.36619467,318.74226336)
\curveto(50.31619052,318.73226343)(50.26119057,318.73226343)(50.20119467,318.74226336)
\lineto(50.05119467,318.74226336)
\lineto(49.90119467,318.74226336)
\curveto(49.85119098,318.73226343)(49.80619103,318.73226343)(49.76619467,318.74226336)
\lineto(49.60119467,318.74226336)
\curveto(49.55119128,318.75226341)(49.49619134,318.75726341)(49.43619467,318.75726336)
\curveto(49.37619146,318.75726341)(49.32119151,318.7622634)(49.27119467,318.77226336)
\curveto(49.20119163,318.78226338)(49.1361917,318.79226337)(49.07619467,318.80226336)
\lineto(48.89619467,318.83226336)
\curveto(48.78619205,318.8622633)(48.68119215,318.89726327)(48.58119467,318.93726336)
\curveto(48.48119235,318.97726319)(48.38619245,319.02226314)(48.29619467,319.07226336)
\lineto(48.20619467,319.13226336)
\curveto(48.17619266,319.162263)(48.14119269,319.19226297)(48.10119467,319.22226336)
\curveto(48.08119275,319.24226292)(48.05619278,319.2622629)(48.02619467,319.28226336)
\lineto(47.95119467,319.35726336)
\curveto(47.81119302,319.54726262)(47.70619313,319.75726241)(47.63619467,319.98726336)
\curveto(47.61619322,320.02726214)(47.60619323,320.0622621)(47.60619467,320.09226336)
\curveto(47.61619322,320.13226203)(47.61619322,320.17726199)(47.60619467,320.22726336)
\curveto(47.59619324,320.24726192)(47.59119324,320.27226189)(47.59119467,320.30226336)
\curveto(47.59119324,320.33226183)(47.58619325,320.35726181)(47.57619467,320.37726336)
\lineto(47.57619467,320.52726336)
\curveto(47.56619327,320.5672616)(47.56119327,320.61226155)(47.56119467,320.66226336)
\curveto(47.57119326,320.71226145)(47.57619326,320.7622614)(47.57619467,320.81226336)
\lineto(47.57619467,321.38226336)
\lineto(47.57619467,323.61726336)
\lineto(47.57619467,324.41226336)
\lineto(47.57619467,324.62226336)
\curveto(47.58619325,324.69225747)(47.58119325,324.75725741)(47.56119467,324.81726336)
\curveto(47.52119331,324.95725721)(47.45119338,325.04725712)(47.35119467,325.08726336)
\curveto(47.24119359,325.13725703)(47.10119373,325.15225701)(46.93119467,325.13226336)
\curveto(46.76119407,325.11225705)(46.61619422,325.12725704)(46.49619467,325.17726336)
\curveto(46.41619442,325.20725696)(46.36619447,325.25225691)(46.34619467,325.31226336)
\curveto(46.32619451,325.37225679)(46.30619453,325.44725672)(46.28619467,325.53726336)
\lineto(46.28619467,325.85226336)
\curveto(46.28619455,326.03225613)(46.29619454,326.17725599)(46.31619467,326.28726336)
\curveto(46.3361945,326.39725577)(46.42119441,326.47225569)(46.57119467,326.51226336)
\curveto(46.61119422,326.53225563)(46.65119418,326.53725563)(46.69119467,326.52726336)
\lineto(46.82619467,326.52726336)
\curveto(46.97619386,326.52725564)(47.11619372,326.53225563)(47.24619467,326.54226336)
\curveto(47.37619346,326.5622556)(47.46619337,326.62225554)(47.51619467,326.72226336)
\curveto(47.54619329,326.79225537)(47.56119327,326.87225529)(47.56119467,326.96226336)
\curveto(47.57119326,327.05225511)(47.57619326,327.14225502)(47.57619467,327.23226336)
\lineto(47.57619467,328.16226336)
\lineto(47.57619467,328.41726336)
\curveto(47.57619326,328.50725366)(47.58619325,328.58225358)(47.60619467,328.64226336)
\curveto(47.65619318,328.74225342)(47.7311931,328.80725336)(47.83119467,328.83726336)
\curveto(47.85119298,328.84725332)(47.87619296,328.84725332)(47.90619467,328.83726336)
\curveto(47.94619289,328.83725333)(47.97619286,328.84225332)(47.99619467,328.85226336)
}
}
{
\newrgbcolor{curcolor}{0 0 0}
\pscustom[linestyle=none,fillstyle=solid,fillcolor=curcolor]
{
\newpath
\moveto(54.31963217,329.39226336)
\curveto(54.38962922,329.31225285)(54.42462918,329.19225297)(54.42463217,329.03226336)
\lineto(54.42463217,328.56726336)
\lineto(54.42463217,328.16226336)
\curveto(54.42462918,328.02225414)(54.38962922,327.92725424)(54.31963217,327.87726336)
\curveto(54.25962935,327.82725434)(54.17962943,327.79725437)(54.07963217,327.78726336)
\curveto(53.98962962,327.77725439)(53.88962972,327.77225439)(53.77963217,327.77226336)
\lineto(52.93963217,327.77226336)
\curveto(52.82963078,327.77225439)(52.72963088,327.77725439)(52.63963217,327.78726336)
\curveto(52.55963105,327.79725437)(52.48963112,327.82725434)(52.42963217,327.87726336)
\curveto(52.38963122,327.90725426)(52.35963125,327.9622542)(52.33963217,328.04226336)
\curveto(52.32963128,328.13225403)(52.31963129,328.22725394)(52.30963217,328.32726336)
\lineto(52.30963217,328.65726336)
\curveto(52.31963129,328.7672534)(52.32463128,328.8622533)(52.32463217,328.94226336)
\lineto(52.32463217,329.15226336)
\curveto(52.33463127,329.22225294)(52.35463125,329.28225288)(52.38463217,329.33226336)
\curveto(52.4046312,329.37225279)(52.42963118,329.40225276)(52.45963217,329.42226336)
\lineto(52.57963217,329.48226336)
\curveto(52.59963101,329.48225268)(52.62463098,329.48225268)(52.65463217,329.48226336)
\curveto(52.68463092,329.49225267)(52.7096309,329.49725267)(52.72963217,329.49726336)
\lineto(53.82463217,329.49726336)
\curveto(53.92462968,329.49725267)(54.01962959,329.49225267)(54.10963217,329.48226336)
\curveto(54.19962941,329.47225269)(54.26962934,329.44225272)(54.31963217,329.39226336)
\moveto(54.42463217,319.62726336)
\curveto(54.42462918,319.42726274)(54.41962919,319.25726291)(54.40963217,319.11726336)
\curveto(54.39962921,318.97726319)(54.3096293,318.88226328)(54.13963217,318.83226336)
\curveto(54.07962953,318.81226335)(54.01462959,318.80226336)(53.94463217,318.80226336)
\curveto(53.87462973,318.81226335)(53.79962981,318.81726335)(53.71963217,318.81726336)
\lineto(52.87963217,318.81726336)
\curveto(52.78963082,318.81726335)(52.69963091,318.82226334)(52.60963217,318.83226336)
\curveto(52.52963108,318.84226332)(52.46963114,318.87226329)(52.42963217,318.92226336)
\curveto(52.36963124,318.99226317)(52.33463127,319.07726309)(52.32463217,319.17726336)
\lineto(52.32463217,319.52226336)
\lineto(52.32463217,325.85226336)
\lineto(52.32463217,326.15226336)
\curveto(52.32463128,326.25225591)(52.34463126,326.33225583)(52.38463217,326.39226336)
\curveto(52.44463116,326.4622557)(52.52963108,326.50725566)(52.63963217,326.52726336)
\curveto(52.65963095,326.53725563)(52.68463092,326.53725563)(52.71463217,326.52726336)
\curveto(52.75463085,326.52725564)(52.78463082,326.53225563)(52.80463217,326.54226336)
\lineto(53.55463217,326.54226336)
\lineto(53.74963217,326.54226336)
\curveto(53.82962978,326.55225561)(53.89462971,326.55225561)(53.94463217,326.54226336)
\lineto(54.06463217,326.54226336)
\curveto(54.12462948,326.52225564)(54.17962943,326.50725566)(54.22963217,326.49726336)
\curveto(54.27962933,326.48725568)(54.31962929,326.45725571)(54.34963217,326.40726336)
\curveto(54.38962922,326.35725581)(54.4096292,326.28725588)(54.40963217,326.19726336)
\curveto(54.41962919,326.10725606)(54.42462918,326.01225615)(54.42463217,325.91226336)
\lineto(54.42463217,319.62726336)
}
}
{
\newrgbcolor{curcolor}{0 0 0}
\pscustom[linestyle=none,fillstyle=solid,fillcolor=curcolor]
{
\newpath
\moveto(63.97681967,322.76226336)
\curveto(63.98681099,322.70225946)(63.99181098,322.61225955)(63.99181967,322.49226336)
\curveto(63.99181098,322.37225979)(63.98181099,322.28725988)(63.96181967,322.23726336)
\lineto(63.96181967,322.04226336)
\curveto(63.93181104,321.93226023)(63.91181106,321.82726034)(63.90181967,321.72726336)
\curveto(63.90181107,321.62726054)(63.88681109,321.52726064)(63.85681967,321.42726336)
\curveto(63.83681114,321.33726083)(63.81681116,321.24226092)(63.79681967,321.14226336)
\curveto(63.7768112,321.05226111)(63.74681123,320.9622612)(63.70681967,320.87226336)
\curveto(63.63681134,320.70226146)(63.56681141,320.54226162)(63.49681967,320.39226336)
\curveto(63.42681155,320.25226191)(63.34681163,320.11226205)(63.25681967,319.97226336)
\curveto(63.19681178,319.88226228)(63.13181184,319.79726237)(63.06181967,319.71726336)
\curveto(63.00181197,319.64726252)(62.93181204,319.57226259)(62.85181967,319.49226336)
\lineto(62.74681967,319.38726336)
\curveto(62.69681228,319.33726283)(62.64181233,319.29226287)(62.58181967,319.25226336)
\lineto(62.43181967,319.13226336)
\curveto(62.35181262,319.07226309)(62.26181271,319.01726315)(62.16181967,318.96726336)
\curveto(62.0718129,318.92726324)(61.976813,318.88226328)(61.87681967,318.83226336)
\curveto(61.7768132,318.78226338)(61.6718133,318.74726342)(61.56181967,318.72726336)
\curveto(61.46181351,318.70726346)(61.35681362,318.68726348)(61.24681967,318.66726336)
\curveto(61.18681379,318.64726352)(61.12181385,318.63726353)(61.05181967,318.63726336)
\curveto(60.99181398,318.63726353)(60.92681405,318.62726354)(60.85681967,318.60726336)
\lineto(60.72181967,318.60726336)
\curveto(60.64181433,318.58726358)(60.56681441,318.58726358)(60.49681967,318.60726336)
\lineto(60.34681967,318.60726336)
\curveto(60.28681469,318.62726354)(60.22181475,318.63726353)(60.15181967,318.63726336)
\curveto(60.09181488,318.62726354)(60.03181494,318.63226353)(59.97181967,318.65226336)
\curveto(59.81181516,318.70226346)(59.65681532,318.74726342)(59.50681967,318.78726336)
\curveto(59.36681561,318.82726334)(59.23681574,318.88726328)(59.11681967,318.96726336)
\curveto(59.04681593,319.00726316)(58.98181599,319.04726312)(58.92181967,319.08726336)
\curveto(58.86181611,319.13726303)(58.79681618,319.18726298)(58.72681967,319.23726336)
\lineto(58.54681967,319.37226336)
\curveto(58.46681651,319.43226273)(58.39681658,319.43726273)(58.33681967,319.38726336)
\curveto(58.28681669,319.35726281)(58.26181671,319.31726285)(58.26181967,319.26726336)
\curveto(58.26181671,319.22726294)(58.25181672,319.17726299)(58.23181967,319.11726336)
\curveto(58.21181676,319.01726315)(58.20181677,318.90226326)(58.20181967,318.77226336)
\curveto(58.21181676,318.64226352)(58.21681676,318.52226364)(58.21681967,318.41226336)
\lineto(58.21681967,316.88226336)
\curveto(58.21681676,316.75226541)(58.21181676,316.62726554)(58.20181967,316.50726336)
\curveto(58.20181677,316.37726579)(58.1768168,316.27226589)(58.12681967,316.19226336)
\curveto(58.09681688,316.15226601)(58.04181693,316.12226604)(57.96181967,316.10226336)
\curveto(57.88181709,316.08226608)(57.79181718,316.07226609)(57.69181967,316.07226336)
\curveto(57.59181738,316.0622661)(57.49181748,316.0622661)(57.39181967,316.07226336)
\lineto(57.13681967,316.07226336)
\lineto(56.73181967,316.07226336)
\lineto(56.62681967,316.07226336)
\curveto(56.58681839,316.07226609)(56.55181842,316.07726609)(56.52181967,316.08726336)
\lineto(56.40181967,316.08726336)
\curveto(56.23181874,316.13726603)(56.14181883,316.23726593)(56.13181967,316.38726336)
\curveto(56.12181885,316.52726564)(56.11681886,316.69726547)(56.11681967,316.89726336)
\lineto(56.11681967,325.70226336)
\curveto(56.11681886,325.81225635)(56.11181886,325.92725624)(56.10181967,326.04726336)
\curveto(56.10181887,326.17725599)(56.12681885,326.27725589)(56.17681967,326.34726336)
\curveto(56.21681876,326.41725575)(56.2718187,326.4622557)(56.34181967,326.48226336)
\curveto(56.39181858,326.50225566)(56.45181852,326.51225565)(56.52181967,326.51226336)
\lineto(56.74681967,326.51226336)
\lineto(57.46681967,326.51226336)
\lineto(57.75181967,326.51226336)
\curveto(57.84181713,326.51225565)(57.91681706,326.48725568)(57.97681967,326.43726336)
\curveto(58.04681693,326.38725578)(58.08181689,326.32225584)(58.08181967,326.24226336)
\curveto(58.09181688,326.17225599)(58.11681686,326.09725607)(58.15681967,326.01726336)
\curveto(58.16681681,325.98725618)(58.1768168,325.9622562)(58.18681967,325.94226336)
\curveto(58.20681677,325.93225623)(58.22681675,325.91725625)(58.24681967,325.89726336)
\curveto(58.35681662,325.88725628)(58.44681653,325.91725625)(58.51681967,325.98726336)
\curveto(58.58681639,326.05725611)(58.65681632,326.11725605)(58.72681967,326.16726336)
\curveto(58.85681612,326.25725591)(58.99181598,326.33725583)(59.13181967,326.40726336)
\curveto(59.2718157,326.48725568)(59.42681555,326.55225561)(59.59681967,326.60226336)
\curveto(59.6768153,326.63225553)(59.76181521,326.65225551)(59.85181967,326.66226336)
\curveto(59.95181502,326.67225549)(60.04681493,326.68725548)(60.13681967,326.70726336)
\curveto(60.1768148,326.71725545)(60.21681476,326.71725545)(60.25681967,326.70726336)
\curveto(60.30681467,326.69725547)(60.34681463,326.70225546)(60.37681967,326.72226336)
\curveto(60.94681403,326.74225542)(61.42681355,326.6622555)(61.81681967,326.48226336)
\curveto(62.21681276,326.31225585)(62.55681242,326.08725608)(62.83681967,325.80726336)
\curveto(62.88681209,325.75725641)(62.93181204,325.70725646)(62.97181967,325.65726336)
\curveto(63.01181196,325.61725655)(63.05181192,325.57225659)(63.09181967,325.52226336)
\curveto(63.16181181,325.43225673)(63.22181175,325.34225682)(63.27181967,325.25226336)
\curveto(63.33181164,325.162257)(63.38681159,325.07225709)(63.43681967,324.98226336)
\curveto(63.45681152,324.9622572)(63.46681151,324.93725723)(63.46681967,324.90726336)
\curveto(63.4768115,324.87725729)(63.49181148,324.84225732)(63.51181967,324.80226336)
\curveto(63.5718114,324.70225746)(63.62681135,324.58225758)(63.67681967,324.44226336)
\curveto(63.69681128,324.38225778)(63.71681126,324.31725785)(63.73681967,324.24726336)
\curveto(63.75681122,324.18725798)(63.7768112,324.12225804)(63.79681967,324.05226336)
\curveto(63.83681114,323.93225823)(63.86181111,323.80725836)(63.87181967,323.67726336)
\curveto(63.89181108,323.54725862)(63.91681106,323.41225875)(63.94681967,323.27226336)
\lineto(63.94681967,323.10726336)
\lineto(63.97681967,322.92726336)
\lineto(63.97681967,322.76226336)
\moveto(61.86181967,322.41726336)
\curveto(61.8718131,322.4672597)(61.8768131,322.53225963)(61.87681967,322.61226336)
\curveto(61.8768131,322.70225946)(61.8718131,322.77225939)(61.86181967,322.82226336)
\lineto(61.86181967,322.95726336)
\curveto(61.84181313,323.01725915)(61.83181314,323.08225908)(61.83181967,323.15226336)
\curveto(61.83181314,323.22225894)(61.82181315,323.29225887)(61.80181967,323.36226336)
\curveto(61.78181319,323.4622587)(61.76181321,323.55725861)(61.74181967,323.64726336)
\curveto(61.72181325,323.74725842)(61.69181328,323.83725833)(61.65181967,323.91726336)
\curveto(61.53181344,324.23725793)(61.3768136,324.49225767)(61.18681967,324.68226336)
\curveto(60.99681398,324.87225729)(60.72681425,325.01225715)(60.37681967,325.10226336)
\curveto(60.29681468,325.12225704)(60.20681477,325.13225703)(60.10681967,325.13226336)
\lineto(59.83681967,325.13226336)
\curveto(59.79681518,325.12225704)(59.76181521,325.11725705)(59.73181967,325.11726336)
\curveto(59.70181527,325.11725705)(59.66681531,325.11225705)(59.62681967,325.10226336)
\lineto(59.41681967,325.04226336)
\curveto(59.35681562,325.03225713)(59.29681568,325.01225715)(59.23681967,324.98226336)
\curveto(58.976816,324.87225729)(58.7718162,324.70225746)(58.62181967,324.47226336)
\curveto(58.48181649,324.24225792)(58.36681661,323.98725818)(58.27681967,323.70726336)
\curveto(58.25681672,323.62725854)(58.24181673,323.54225862)(58.23181967,323.45226336)
\curveto(58.22181675,323.37225879)(58.20681677,323.29225887)(58.18681967,323.21226336)
\curveto(58.1768168,323.17225899)(58.1718168,323.10725906)(58.17181967,323.01726336)
\curveto(58.15181682,322.97725919)(58.14681683,322.92725924)(58.15681967,322.86726336)
\curveto(58.16681681,322.81725935)(58.16681681,322.7672594)(58.15681967,322.71726336)
\curveto(58.13681684,322.65725951)(58.13681684,322.60225956)(58.15681967,322.55226336)
\lineto(58.15681967,322.37226336)
\lineto(58.15681967,322.23726336)
\curveto(58.15681682,322.19725997)(58.16681681,322.15726001)(58.18681967,322.11726336)
\curveto(58.18681679,322.04726012)(58.19181678,321.99226017)(58.20181967,321.95226336)
\lineto(58.23181967,321.77226336)
\curveto(58.24181673,321.71226045)(58.25681672,321.65226051)(58.27681967,321.59226336)
\curveto(58.36681661,321.30226086)(58.4718165,321.0622611)(58.59181967,320.87226336)
\curveto(58.72181625,320.69226147)(58.90181607,320.53226163)(59.13181967,320.39226336)
\curveto(59.2718157,320.31226185)(59.43681554,320.24726192)(59.62681967,320.19726336)
\curveto(59.66681531,320.18726198)(59.70181527,320.18226198)(59.73181967,320.18226336)
\curveto(59.76181521,320.19226197)(59.79681518,320.19226197)(59.83681967,320.18226336)
\curveto(59.8768151,320.17226199)(59.93681504,320.162262)(60.01681967,320.15226336)
\curveto(60.09681488,320.15226201)(60.16181481,320.15726201)(60.21181967,320.16726336)
\curveto(60.29181468,320.18726198)(60.3718146,320.20226196)(60.45181967,320.21226336)
\curveto(60.54181443,320.23226193)(60.62681435,320.25726191)(60.70681967,320.28726336)
\curveto(60.94681403,320.38726178)(61.14181383,320.52726164)(61.29181967,320.70726336)
\curveto(61.44181353,320.88726128)(61.56681341,321.09726107)(61.66681967,321.33726336)
\curveto(61.71681326,321.45726071)(61.75181322,321.58226058)(61.77181967,321.71226336)
\curveto(61.79181318,321.84226032)(61.81681316,321.97726019)(61.84681967,322.11726336)
\lineto(61.84681967,322.26726336)
\curveto(61.85681312,322.31725985)(61.86181311,322.3672598)(61.86181967,322.41726336)
}
}
{
\newrgbcolor{curcolor}{0 0 0}
\pscustom[linestyle=none,fillstyle=solid,fillcolor=curcolor]
{
\newpath
\moveto(73.02674154,322.98726336)
\curveto(73.04673297,322.92725924)(73.05673296,322.84225932)(73.05674154,322.73226336)
\curveto(73.05673296,322.62225954)(73.04673297,322.53725963)(73.02674154,322.47726336)
\lineto(73.02674154,322.32726336)
\curveto(73.00673301,322.24725992)(72.99673302,322.16726)(72.99674154,322.08726336)
\curveto(73.00673301,322.00726016)(73.00173302,321.92726024)(72.98174154,321.84726336)
\curveto(72.96173306,321.77726039)(72.94673307,321.71226045)(72.93674154,321.65226336)
\curveto(72.92673309,321.59226057)(72.9167331,321.52726064)(72.90674154,321.45726336)
\curveto(72.86673315,321.34726082)(72.83173319,321.23226093)(72.80174154,321.11226336)
\curveto(72.77173325,321.00226116)(72.73173329,320.89726127)(72.68174154,320.79726336)
\curveto(72.47173355,320.31726185)(72.19673382,319.92726224)(71.85674154,319.62726336)
\curveto(71.5167345,319.32726284)(71.10673491,319.07726309)(70.62674154,318.87726336)
\curveto(70.50673551,318.82726334)(70.38173564,318.79226337)(70.25174154,318.77226336)
\curveto(70.13173589,318.74226342)(70.00673601,318.71226345)(69.87674154,318.68226336)
\curveto(69.82673619,318.6622635)(69.77173625,318.65226351)(69.71174154,318.65226336)
\curveto(69.65173637,318.65226351)(69.59673642,318.64726352)(69.54674154,318.63726336)
\lineto(69.44174154,318.63726336)
\curveto(69.41173661,318.62726354)(69.38173664,318.62226354)(69.35174154,318.62226336)
\curveto(69.30173672,318.61226355)(69.2217368,318.60726356)(69.11174154,318.60726336)
\curveto(69.00173702,318.59726357)(68.9167371,318.60226356)(68.85674154,318.62226336)
\lineto(68.70674154,318.62226336)
\curveto(68.65673736,318.63226353)(68.60173742,318.63726353)(68.54174154,318.63726336)
\curveto(68.49173753,318.62726354)(68.44173758,318.63226353)(68.39174154,318.65226336)
\curveto(68.35173767,318.6622635)(68.31173771,318.6672635)(68.27174154,318.66726336)
\curveto(68.24173778,318.6672635)(68.20173782,318.67226349)(68.15174154,318.68226336)
\curveto(68.05173797,318.71226345)(67.95173807,318.73726343)(67.85174154,318.75726336)
\curveto(67.75173827,318.77726339)(67.65673836,318.80726336)(67.56674154,318.84726336)
\curveto(67.44673857,318.88726328)(67.33173869,318.92726324)(67.22174154,318.96726336)
\curveto(67.1217389,319.00726316)(67.016739,319.05726311)(66.90674154,319.11726336)
\curveto(66.55673946,319.32726284)(66.25673976,319.57226259)(66.00674154,319.85226336)
\curveto(65.75674026,320.13226203)(65.54674047,320.4672617)(65.37674154,320.85726336)
\curveto(65.32674069,320.94726122)(65.28674073,321.04226112)(65.25674154,321.14226336)
\curveto(65.23674078,321.24226092)(65.21174081,321.34726082)(65.18174154,321.45726336)
\curveto(65.16174086,321.50726066)(65.15174087,321.55226061)(65.15174154,321.59226336)
\curveto(65.15174087,321.63226053)(65.14174088,321.67726049)(65.12174154,321.72726336)
\curveto(65.10174092,321.80726036)(65.09174093,321.88726028)(65.09174154,321.96726336)
\curveto(65.09174093,322.05726011)(65.08174094,322.14226002)(65.06174154,322.22226336)
\curveto(65.05174097,322.27225989)(65.04674097,322.31725985)(65.04674154,322.35726336)
\lineto(65.04674154,322.49226336)
\curveto(65.02674099,322.55225961)(65.016741,322.63725953)(65.01674154,322.74726336)
\curveto(65.02674099,322.85725931)(65.04174098,322.94225922)(65.06174154,323.00226336)
\lineto(65.06174154,323.10726336)
\curveto(65.07174095,323.15725901)(65.07174095,323.20725896)(65.06174154,323.25726336)
\curveto(65.06174096,323.31725885)(65.07174095,323.37225879)(65.09174154,323.42226336)
\curveto(65.10174092,323.47225869)(65.10674091,323.51725865)(65.10674154,323.55726336)
\curveto(65.10674091,323.60725856)(65.1167409,323.65725851)(65.13674154,323.70726336)
\curveto(65.17674084,323.83725833)(65.21174081,323.9622582)(65.24174154,324.08226336)
\curveto(65.27174075,324.21225795)(65.31174071,324.33725783)(65.36174154,324.45726336)
\curveto(65.54174048,324.8672573)(65.75674026,325.20725696)(66.00674154,325.47726336)
\curveto(66.25673976,325.75725641)(66.56173946,326.01225615)(66.92174154,326.24226336)
\curveto(67.021739,326.29225587)(67.12673889,326.33725583)(67.23674154,326.37726336)
\curveto(67.34673867,326.41725575)(67.45673856,326.4622557)(67.56674154,326.51226336)
\curveto(67.69673832,326.5622556)(67.83173819,326.59725557)(67.97174154,326.61726336)
\curveto(68.11173791,326.63725553)(68.25673776,326.6672555)(68.40674154,326.70726336)
\curveto(68.48673753,326.71725545)(68.56173746,326.72225544)(68.63174154,326.72226336)
\curveto(68.70173732,326.72225544)(68.77173725,326.72725544)(68.84174154,326.73726336)
\curveto(69.4217366,326.74725542)(69.9217361,326.68725548)(70.34174154,326.55726336)
\curveto(70.77173525,326.42725574)(71.15173487,326.24725592)(71.48174154,326.01726336)
\curveto(71.59173443,325.93725623)(71.70173432,325.84725632)(71.81174154,325.74726336)
\curveto(71.93173409,325.65725651)(72.03173399,325.55725661)(72.11174154,325.44726336)
\curveto(72.19173383,325.34725682)(72.26173376,325.24725692)(72.32174154,325.14726336)
\curveto(72.39173363,325.04725712)(72.46173356,324.94225722)(72.53174154,324.83226336)
\curveto(72.60173342,324.72225744)(72.65673336,324.60225756)(72.69674154,324.47226336)
\curveto(72.73673328,324.35225781)(72.78173324,324.22225794)(72.83174154,324.08226336)
\curveto(72.86173316,324.00225816)(72.88673313,323.91725825)(72.90674154,323.82726336)
\lineto(72.96674154,323.55726336)
\curveto(72.97673304,323.51725865)(72.98173304,323.47725869)(72.98174154,323.43726336)
\curveto(72.98173304,323.39725877)(72.98673303,323.35725881)(72.99674154,323.31726336)
\curveto(73.016733,323.2672589)(73.021733,323.21225895)(73.01174154,323.15226336)
\curveto(73.00173302,323.09225907)(73.00673301,323.03725913)(73.02674154,322.98726336)
\moveto(70.92674154,322.44726336)
\curveto(70.93673508,322.49725967)(70.94173508,322.5672596)(70.94174154,322.65726336)
\curveto(70.94173508,322.75725941)(70.93673508,322.83225933)(70.92674154,322.88226336)
\lineto(70.92674154,323.00226336)
\curveto(70.90673511,323.05225911)(70.89673512,323.10725906)(70.89674154,323.16726336)
\curveto(70.89673512,323.22725894)(70.89173513,323.28225888)(70.88174154,323.33226336)
\curveto(70.88173514,323.37225879)(70.87673514,323.40225876)(70.86674154,323.42226336)
\lineto(70.80674154,323.66226336)
\curveto(70.79673522,323.75225841)(70.77673524,323.83725833)(70.74674154,323.91726336)
\curveto(70.63673538,324.17725799)(70.50673551,324.39725777)(70.35674154,324.57726336)
\curveto(70.20673581,324.7672574)(70.00673601,324.91725725)(69.75674154,325.02726336)
\curveto(69.69673632,325.04725712)(69.63673638,325.0622571)(69.57674154,325.07226336)
\curveto(69.5167365,325.09225707)(69.45173657,325.11225705)(69.38174154,325.13226336)
\curveto(69.30173672,325.15225701)(69.2167368,325.15725701)(69.12674154,325.14726336)
\lineto(68.85674154,325.14726336)
\curveto(68.82673719,325.12725704)(68.79173723,325.11725705)(68.75174154,325.11726336)
\curveto(68.71173731,325.12725704)(68.67673734,325.12725704)(68.64674154,325.11726336)
\lineto(68.43674154,325.05726336)
\curveto(68.37673764,325.04725712)(68.3217377,325.02725714)(68.27174154,324.99726336)
\curveto(68.021738,324.88725728)(67.8167382,324.72725744)(67.65674154,324.51726336)
\curveto(67.50673851,324.31725785)(67.38673863,324.08225808)(67.29674154,323.81226336)
\curveto(67.26673875,323.71225845)(67.24173878,323.60725856)(67.22174154,323.49726336)
\curveto(67.21173881,323.38725878)(67.19673882,323.27725889)(67.17674154,323.16726336)
\curveto(67.16673885,323.11725905)(67.16173886,323.0672591)(67.16174154,323.01726336)
\lineto(67.16174154,322.86726336)
\curveto(67.14173888,322.79725937)(67.13173889,322.69225947)(67.13174154,322.55226336)
\curveto(67.14173888,322.41225975)(67.15673886,322.30725986)(67.17674154,322.23726336)
\lineto(67.17674154,322.10226336)
\curveto(67.19673882,322.02226014)(67.21173881,321.94226022)(67.22174154,321.86226336)
\curveto(67.23173879,321.79226037)(67.24673877,321.71726045)(67.26674154,321.63726336)
\curveto(67.36673865,321.33726083)(67.47173855,321.09226107)(67.58174154,320.90226336)
\curveto(67.70173832,320.72226144)(67.88673813,320.55726161)(68.13674154,320.40726336)
\curveto(68.20673781,320.35726181)(68.28173774,320.31726185)(68.36174154,320.28726336)
\curveto(68.45173757,320.25726191)(68.54173748,320.23226193)(68.63174154,320.21226336)
\curveto(68.67173735,320.20226196)(68.70673731,320.19726197)(68.73674154,320.19726336)
\curveto(68.76673725,320.20726196)(68.80173722,320.20726196)(68.84174154,320.19726336)
\lineto(68.96174154,320.16726336)
\curveto(69.01173701,320.167262)(69.05673696,320.17226199)(69.09674154,320.18226336)
\lineto(69.21674154,320.18226336)
\curveto(69.29673672,320.20226196)(69.37673664,320.21726195)(69.45674154,320.22726336)
\curveto(69.53673648,320.23726193)(69.61173641,320.25726191)(69.68174154,320.28726336)
\curveto(69.94173608,320.38726178)(70.15173587,320.52226164)(70.31174154,320.69226336)
\curveto(70.47173555,320.8622613)(70.60673541,321.07226109)(70.71674154,321.32226336)
\curveto(70.75673526,321.42226074)(70.78673523,321.52226064)(70.80674154,321.62226336)
\curveto(70.82673519,321.72226044)(70.85173517,321.82726034)(70.88174154,321.93726336)
\curveto(70.89173513,321.97726019)(70.89673512,322.01226015)(70.89674154,322.04226336)
\curveto(70.89673512,322.08226008)(70.90173512,322.12226004)(70.91174154,322.16226336)
\lineto(70.91174154,322.29726336)
\curveto(70.91173511,322.34725982)(70.9167351,322.39725977)(70.92674154,322.44726336)
}
}
{
\newrgbcolor{curcolor}{0 0 0}
\pscustom[linestyle=none,fillstyle=solid,fillcolor=curcolor]
{
}
}
{
\newrgbcolor{curcolor}{0 0 0}
\pscustom[linestyle=none,fillstyle=solid,fillcolor=curcolor]
{
\newpath
\moveto(86.17681967,319.65726336)
\lineto(86.17681967,319.23726336)
\curveto(86.1768113,319.10726306)(86.14681133,319.00226316)(86.08681967,318.92226336)
\curveto(86.03681144,318.87226329)(85.9718115,318.83726333)(85.89181967,318.81726336)
\curveto(85.81181166,318.80726336)(85.72181175,318.80226336)(85.62181967,318.80226336)
\lineto(84.79681967,318.80226336)
\lineto(84.51181967,318.80226336)
\curveto(84.43181304,318.81226335)(84.36681311,318.83726333)(84.31681967,318.87726336)
\curveto(84.24681323,318.92726324)(84.20681327,318.99226317)(84.19681967,319.07226336)
\curveto(84.18681329,319.15226301)(84.16681331,319.23226293)(84.13681967,319.31226336)
\curveto(84.11681336,319.33226283)(84.09681338,319.34726282)(84.07681967,319.35726336)
\curveto(84.06681341,319.37726279)(84.05181342,319.39726277)(84.03181967,319.41726336)
\curveto(83.92181355,319.41726275)(83.84181363,319.39226277)(83.79181967,319.34226336)
\lineto(83.64181967,319.19226336)
\curveto(83.5718139,319.14226302)(83.50681397,319.09726307)(83.44681967,319.05726336)
\curveto(83.38681409,319.02726314)(83.32181415,318.98726318)(83.25181967,318.93726336)
\curveto(83.21181426,318.91726325)(83.16681431,318.89726327)(83.11681967,318.87726336)
\curveto(83.0768144,318.85726331)(83.03181444,318.83726333)(82.98181967,318.81726336)
\curveto(82.84181463,318.7672634)(82.69181478,318.72226344)(82.53181967,318.68226336)
\curveto(82.48181499,318.6622635)(82.43681504,318.65226351)(82.39681967,318.65226336)
\curveto(82.35681512,318.65226351)(82.31681516,318.64726352)(82.27681967,318.63726336)
\lineto(82.14181967,318.63726336)
\curveto(82.11181536,318.62726354)(82.0718154,318.62226354)(82.02181967,318.62226336)
\lineto(81.88681967,318.62226336)
\curveto(81.82681565,318.60226356)(81.73681574,318.59726357)(81.61681967,318.60726336)
\curveto(81.49681598,318.60726356)(81.41181606,318.61726355)(81.36181967,318.63726336)
\curveto(81.29181618,318.65726351)(81.22681625,318.6672635)(81.16681967,318.66726336)
\curveto(81.11681636,318.65726351)(81.06181641,318.6622635)(81.00181967,318.68226336)
\lineto(80.64181967,318.80226336)
\curveto(80.53181694,318.83226333)(80.42181705,318.87226329)(80.31181967,318.92226336)
\curveto(79.96181751,319.07226309)(79.64681783,319.30226286)(79.36681967,319.61226336)
\curveto(79.09681838,319.93226223)(78.88181859,320.2672619)(78.72181967,320.61726336)
\curveto(78.6718188,320.72726144)(78.63181884,320.83226133)(78.60181967,320.93226336)
\curveto(78.5718189,321.04226112)(78.53681894,321.15226101)(78.49681967,321.26226336)
\curveto(78.48681899,321.30226086)(78.48181899,321.33726083)(78.48181967,321.36726336)
\curveto(78.48181899,321.40726076)(78.471819,321.45226071)(78.45181967,321.50226336)
\curveto(78.43181904,321.58226058)(78.41181906,321.6672605)(78.39181967,321.75726336)
\curveto(78.38181909,321.85726031)(78.36681911,321.95726021)(78.34681967,322.05726336)
\curveto(78.33681914,322.08726008)(78.33181914,322.12226004)(78.33181967,322.16226336)
\curveto(78.34181913,322.20225996)(78.34181913,322.23725993)(78.33181967,322.26726336)
\lineto(78.33181967,322.40226336)
\curveto(78.33181914,322.45225971)(78.32681915,322.50225966)(78.31681967,322.55226336)
\curveto(78.30681917,322.60225956)(78.30181917,322.65725951)(78.30181967,322.71726336)
\curveto(78.30181917,322.78725938)(78.30681917,322.84225932)(78.31681967,322.88226336)
\curveto(78.32681915,322.93225923)(78.33181914,322.97725919)(78.33181967,323.01726336)
\lineto(78.33181967,323.16726336)
\curveto(78.34181913,323.21725895)(78.34181913,323.2622589)(78.33181967,323.30226336)
\curveto(78.33181914,323.35225881)(78.34181913,323.40225876)(78.36181967,323.45226336)
\curveto(78.38181909,323.5622586)(78.39681908,323.6672585)(78.40681967,323.76726336)
\curveto(78.42681905,323.8672583)(78.45181902,323.9672582)(78.48181967,324.06726336)
\curveto(78.52181895,324.18725798)(78.55681892,324.30225786)(78.58681967,324.41226336)
\curveto(78.61681886,324.52225764)(78.65681882,324.63225753)(78.70681967,324.74226336)
\curveto(78.84681863,325.04225712)(79.02181845,325.32725684)(79.23181967,325.59726336)
\curveto(79.25181822,325.62725654)(79.2768182,325.65225651)(79.30681967,325.67226336)
\curveto(79.34681813,325.70225646)(79.3768181,325.73225643)(79.39681967,325.76226336)
\curveto(79.43681804,325.81225635)(79.476818,325.85725631)(79.51681967,325.89726336)
\curveto(79.55681792,325.93725623)(79.60181787,325.97725619)(79.65181967,326.01726336)
\curveto(79.69181778,326.03725613)(79.72681775,326.0622561)(79.75681967,326.09226336)
\curveto(79.78681769,326.13225603)(79.82181765,326.162256)(79.86181967,326.18226336)
\curveto(80.11181736,326.35225581)(80.40181707,326.49225567)(80.73181967,326.60226336)
\curveto(80.80181667,326.62225554)(80.8718166,326.63725553)(80.94181967,326.64726336)
\curveto(81.02181645,326.65725551)(81.10181637,326.67225549)(81.18181967,326.69226336)
\curveto(81.25181622,326.71225545)(81.34181613,326.72225544)(81.45181967,326.72226336)
\curveto(81.56181591,326.73225543)(81.6718158,326.73725543)(81.78181967,326.73726336)
\curveto(81.89181558,326.73725543)(81.99681548,326.73225543)(82.09681967,326.72226336)
\curveto(82.20681527,326.71225545)(82.29681518,326.69725547)(82.36681967,326.67726336)
\curveto(82.51681496,326.62725554)(82.66181481,326.58225558)(82.80181967,326.54226336)
\curveto(82.94181453,326.50225566)(83.0718144,326.44725572)(83.19181967,326.37726336)
\curveto(83.26181421,326.32725584)(83.32681415,326.27725589)(83.38681967,326.22726336)
\curveto(83.44681403,326.18725598)(83.51181396,326.14225602)(83.58181967,326.09226336)
\curveto(83.62181385,326.0622561)(83.6768138,326.02225614)(83.74681967,325.97226336)
\curveto(83.82681365,325.92225624)(83.90181357,325.92225624)(83.97181967,325.97226336)
\curveto(84.01181346,325.99225617)(84.03181344,326.02725614)(84.03181967,326.07726336)
\curveto(84.03181344,326.12725604)(84.04181343,326.17725599)(84.06181967,326.22726336)
\lineto(84.06181967,326.37726336)
\curveto(84.0718134,326.40725576)(84.0768134,326.44225572)(84.07681967,326.48226336)
\lineto(84.07681967,326.60226336)
\lineto(84.07681967,328.64226336)
\curveto(84.0768134,328.75225341)(84.0718134,328.87225329)(84.06181967,329.00226336)
\curveto(84.06181341,329.14225302)(84.08681339,329.24725292)(84.13681967,329.31726336)
\curveto(84.1768133,329.39725277)(84.25181322,329.44725272)(84.36181967,329.46726336)
\curveto(84.38181309,329.47725269)(84.40181307,329.47725269)(84.42181967,329.46726336)
\curveto(84.44181303,329.4672527)(84.46181301,329.47225269)(84.48181967,329.48226336)
\lineto(85.54681967,329.48226336)
\curveto(85.66681181,329.48225268)(85.7768117,329.47725269)(85.87681967,329.46726336)
\curveto(85.9768115,329.45725271)(86.05181142,329.41725275)(86.10181967,329.34726336)
\curveto(86.15181132,329.2672529)(86.1768113,329.162253)(86.17681967,329.03226336)
\lineto(86.17681967,328.67226336)
\lineto(86.17681967,319.65726336)
\moveto(84.13681967,322.59726336)
\curveto(84.14681333,322.63725953)(84.14681333,322.67725949)(84.13681967,322.71726336)
\lineto(84.13681967,322.85226336)
\curveto(84.13681334,322.95225921)(84.13181334,323.05225911)(84.12181967,323.15226336)
\curveto(84.11181336,323.25225891)(84.09681338,323.34225882)(84.07681967,323.42226336)
\curveto(84.05681342,323.53225863)(84.03681344,323.63225853)(84.01681967,323.72226336)
\curveto(84.00681347,323.81225835)(83.98181349,323.89725827)(83.94181967,323.97726336)
\curveto(83.80181367,324.33725783)(83.59681388,324.62225754)(83.32681967,324.83226336)
\curveto(83.06681441,325.04225712)(82.68681479,325.14725702)(82.18681967,325.14726336)
\curveto(82.12681535,325.14725702)(82.04681543,325.13725703)(81.94681967,325.11726336)
\curveto(81.86681561,325.09725707)(81.79181568,325.07725709)(81.72181967,325.05726336)
\curveto(81.66181581,325.04725712)(81.60181587,325.02725714)(81.54181967,324.99726336)
\curveto(81.2718162,324.88725728)(81.06181641,324.71725745)(80.91181967,324.48726336)
\curveto(80.76181671,324.25725791)(80.64181683,323.99725817)(80.55181967,323.70726336)
\curveto(80.52181695,323.60725856)(80.50181697,323.50725866)(80.49181967,323.40726336)
\curveto(80.48181699,323.30725886)(80.46181701,323.20225896)(80.43181967,323.09226336)
\lineto(80.43181967,322.88226336)
\curveto(80.41181706,322.79225937)(80.40681707,322.6672595)(80.41681967,322.50726336)
\curveto(80.42681705,322.35725981)(80.44181703,322.24725992)(80.46181967,322.17726336)
\lineto(80.46181967,322.08726336)
\curveto(80.471817,322.0672601)(80.476817,322.04726012)(80.47681967,322.02726336)
\curveto(80.49681698,321.94726022)(80.51181696,321.87226029)(80.52181967,321.80226336)
\curveto(80.54181693,321.73226043)(80.56181691,321.65726051)(80.58181967,321.57726336)
\curveto(80.75181672,321.05726111)(81.04181643,320.67226149)(81.45181967,320.42226336)
\curveto(81.58181589,320.33226183)(81.76181571,320.2622619)(81.99181967,320.21226336)
\curveto(82.03181544,320.20226196)(82.09181538,320.19726197)(82.17181967,320.19726336)
\curveto(82.20181527,320.18726198)(82.24681523,320.17726199)(82.30681967,320.16726336)
\curveto(82.3768151,320.167262)(82.43181504,320.17226199)(82.47181967,320.18226336)
\curveto(82.55181492,320.20226196)(82.63181484,320.21726195)(82.71181967,320.22726336)
\curveto(82.79181468,320.23726193)(82.8718146,320.25726191)(82.95181967,320.28726336)
\curveto(83.20181427,320.39726177)(83.40181407,320.53726163)(83.55181967,320.70726336)
\curveto(83.70181377,320.87726129)(83.83181364,321.09226107)(83.94181967,321.35226336)
\curveto(83.98181349,321.44226072)(84.01181346,321.53226063)(84.03181967,321.62226336)
\curveto(84.05181342,321.72226044)(84.0718134,321.82726034)(84.09181967,321.93726336)
\curveto(84.10181337,321.98726018)(84.10181337,322.03226013)(84.09181967,322.07226336)
\curveto(84.09181338,322.12226004)(84.10181337,322.17225999)(84.12181967,322.22226336)
\curveto(84.13181334,322.25225991)(84.13681334,322.28725988)(84.13681967,322.32726336)
\lineto(84.13681967,322.46226336)
\lineto(84.13681967,322.59726336)
}
}
{
\newrgbcolor{curcolor}{0 0 0}
\pscustom[linestyle=none,fillstyle=solid,fillcolor=curcolor]
{
\newpath
\moveto(95.12174154,322.74726336)
\curveto(95.14173338,322.6672595)(95.14173338,322.57725959)(95.12174154,322.47726336)
\curveto(95.10173342,322.37725979)(95.06673345,322.31225985)(95.01674154,322.28226336)
\curveto(94.96673355,322.24225992)(94.89173363,322.21225995)(94.79174154,322.19226336)
\curveto(94.70173382,322.18225998)(94.59673392,322.17225999)(94.47674154,322.16226336)
\lineto(94.13174154,322.16226336)
\curveto(94.0217345,322.17225999)(93.9217346,322.17725999)(93.83174154,322.17726336)
\lineto(90.17174154,322.17726336)
\lineto(89.96174154,322.17726336)
\curveto(89.90173862,322.17725999)(89.84673867,322.16726)(89.79674154,322.14726336)
\curveto(89.7167388,322.10726006)(89.66673885,322.0672601)(89.64674154,322.02726336)
\curveto(89.62673889,322.00726016)(89.60673891,321.9672602)(89.58674154,321.90726336)
\curveto(89.56673895,321.85726031)(89.56173896,321.80726036)(89.57174154,321.75726336)
\curveto(89.59173893,321.69726047)(89.60173892,321.63726053)(89.60174154,321.57726336)
\curveto(89.61173891,321.52726064)(89.62673889,321.47226069)(89.64674154,321.41226336)
\curveto(89.72673879,321.17226099)(89.8217387,320.97226119)(89.93174154,320.81226336)
\curveto(90.05173847,320.6622615)(90.21173831,320.52726164)(90.41174154,320.40726336)
\curveto(90.49173803,320.35726181)(90.57173795,320.32226184)(90.65174154,320.30226336)
\curveto(90.74173778,320.29226187)(90.83173769,320.27226189)(90.92174154,320.24226336)
\curveto(91.00173752,320.22226194)(91.11173741,320.20726196)(91.25174154,320.19726336)
\curveto(91.39173713,320.18726198)(91.51173701,320.19226197)(91.61174154,320.21226336)
\lineto(91.74674154,320.21226336)
\curveto(91.84673667,320.23226193)(91.93673658,320.25226191)(92.01674154,320.27226336)
\curveto(92.10673641,320.30226186)(92.19173633,320.33226183)(92.27174154,320.36226336)
\curveto(92.37173615,320.41226175)(92.48173604,320.47726169)(92.60174154,320.55726336)
\curveto(92.73173579,320.63726153)(92.82673569,320.71726145)(92.88674154,320.79726336)
\curveto(92.93673558,320.8672613)(92.98673553,320.93226123)(93.03674154,320.99226336)
\curveto(93.09673542,321.0622611)(93.16673535,321.11226105)(93.24674154,321.14226336)
\curveto(93.34673517,321.19226097)(93.47173505,321.21226095)(93.62174154,321.20226336)
\lineto(94.05674154,321.20226336)
\lineto(94.23674154,321.20226336)
\curveto(94.30673421,321.21226095)(94.36673415,321.20726096)(94.41674154,321.18726336)
\lineto(94.56674154,321.18726336)
\curveto(94.66673385,321.167261)(94.73673378,321.14226102)(94.77674154,321.11226336)
\curveto(94.8167337,321.09226107)(94.83673368,321.04726112)(94.83674154,320.97726336)
\curveto(94.84673367,320.90726126)(94.84173368,320.84726132)(94.82174154,320.79726336)
\curveto(94.77173375,320.65726151)(94.7167338,320.53226163)(94.65674154,320.42226336)
\curveto(94.59673392,320.31226185)(94.52673399,320.20226196)(94.44674154,320.09226336)
\curveto(94.22673429,319.7622624)(93.97673454,319.49726267)(93.69674154,319.29726336)
\curveto(93.4167351,319.09726307)(93.06673545,318.92726324)(92.64674154,318.78726336)
\curveto(92.53673598,318.74726342)(92.42673609,318.72226344)(92.31674154,318.71226336)
\curveto(92.20673631,318.70226346)(92.09173643,318.68226348)(91.97174154,318.65226336)
\curveto(91.93173659,318.64226352)(91.88673663,318.64226352)(91.83674154,318.65226336)
\curveto(91.79673672,318.65226351)(91.75673676,318.64726352)(91.71674154,318.63726336)
\lineto(91.55174154,318.63726336)
\curveto(91.50173702,318.61726355)(91.44173708,318.61226355)(91.37174154,318.62226336)
\curveto(91.31173721,318.62226354)(91.25673726,318.62726354)(91.20674154,318.63726336)
\curveto(91.12673739,318.64726352)(91.05673746,318.64726352)(90.99674154,318.63726336)
\curveto(90.93673758,318.62726354)(90.87173765,318.63226353)(90.80174154,318.65226336)
\curveto(90.75173777,318.67226349)(90.69673782,318.68226348)(90.63674154,318.68226336)
\curveto(90.57673794,318.68226348)(90.521738,318.69226347)(90.47174154,318.71226336)
\curveto(90.36173816,318.73226343)(90.25173827,318.75726341)(90.14174154,318.78726336)
\curveto(90.03173849,318.80726336)(89.93173859,318.84226332)(89.84174154,318.89226336)
\curveto(89.73173879,318.93226323)(89.62673889,318.9672632)(89.52674154,318.99726336)
\curveto(89.43673908,319.03726313)(89.35173917,319.08226308)(89.27174154,319.13226336)
\curveto(88.95173957,319.33226283)(88.66673985,319.5622626)(88.41674154,319.82226336)
\curveto(88.16674035,320.09226207)(87.96174056,320.40226176)(87.80174154,320.75226336)
\curveto(87.75174077,320.8622613)(87.71174081,320.97226119)(87.68174154,321.08226336)
\curveto(87.65174087,321.20226096)(87.61174091,321.32226084)(87.56174154,321.44226336)
\curveto(87.55174097,321.48226068)(87.54674097,321.51726065)(87.54674154,321.54726336)
\curveto(87.54674097,321.58726058)(87.54174098,321.62726054)(87.53174154,321.66726336)
\curveto(87.49174103,321.78726038)(87.46674105,321.91726025)(87.45674154,322.05726336)
\lineto(87.42674154,322.47726336)
\curveto(87.42674109,322.52725964)(87.4217411,322.58225958)(87.41174154,322.64226336)
\curveto(87.41174111,322.70225946)(87.4167411,322.75725941)(87.42674154,322.80726336)
\lineto(87.42674154,322.98726336)
\lineto(87.47174154,323.34726336)
\curveto(87.51174101,323.51725865)(87.54674097,323.68225848)(87.57674154,323.84226336)
\curveto(87.60674091,324.00225816)(87.65174087,324.15225801)(87.71174154,324.29226336)
\curveto(88.14174038,325.33225683)(88.87173965,326.0672561)(89.90174154,326.49726336)
\curveto(90.04173848,326.55725561)(90.18173834,326.59725557)(90.32174154,326.61726336)
\curveto(90.47173805,326.64725552)(90.62673789,326.68225548)(90.78674154,326.72226336)
\curveto(90.86673765,326.73225543)(90.94173758,326.73725543)(91.01174154,326.73726336)
\curveto(91.08173744,326.73725543)(91.15673736,326.74225542)(91.23674154,326.75226336)
\curveto(91.74673677,326.7622554)(92.18173634,326.70225546)(92.54174154,326.57226336)
\curveto(92.91173561,326.45225571)(93.24173528,326.29225587)(93.53174154,326.09226336)
\curveto(93.6217349,326.03225613)(93.71173481,325.9622562)(93.80174154,325.88226336)
\curveto(93.89173463,325.81225635)(93.97173455,325.73725643)(94.04174154,325.65726336)
\curveto(94.07173445,325.60725656)(94.11173441,325.5672566)(94.16174154,325.53726336)
\curveto(94.24173428,325.42725674)(94.3167342,325.31225685)(94.38674154,325.19226336)
\curveto(94.45673406,325.08225708)(94.53173399,324.9672572)(94.61174154,324.84726336)
\curveto(94.66173386,324.75725741)(94.70173382,324.6622575)(94.73174154,324.56226336)
\curveto(94.77173375,324.47225769)(94.81173371,324.37225779)(94.85174154,324.26226336)
\curveto(94.90173362,324.13225803)(94.94173358,323.99725817)(94.97174154,323.85726336)
\curveto(95.00173352,323.71725845)(95.03673348,323.57725859)(95.07674154,323.43726336)
\curveto(95.09673342,323.35725881)(95.10173342,323.2672589)(95.09174154,323.16726336)
\curveto(95.09173343,323.07725909)(95.10173342,322.99225917)(95.12174154,322.91226336)
\lineto(95.12174154,322.74726336)
\moveto(92.87174154,323.63226336)
\curveto(92.94173558,323.73225843)(92.94673557,323.85225831)(92.88674154,323.99226336)
\curveto(92.83673568,324.14225802)(92.79673572,324.25225791)(92.76674154,324.32226336)
\curveto(92.62673589,324.59225757)(92.44173608,324.79725737)(92.21174154,324.93726336)
\curveto(91.98173654,325.08725708)(91.66173686,325.167257)(91.25174154,325.17726336)
\curveto(91.2217373,325.15725701)(91.18673733,325.15225701)(91.14674154,325.16226336)
\curveto(91.10673741,325.17225699)(91.07173745,325.17225699)(91.04174154,325.16226336)
\curveto(90.99173753,325.14225702)(90.93673758,325.12725704)(90.87674154,325.11726336)
\curveto(90.8167377,325.11725705)(90.76173776,325.10725706)(90.71174154,325.08726336)
\curveto(90.27173825,324.94725722)(89.94673857,324.67225749)(89.73674154,324.26226336)
\curveto(89.7167388,324.22225794)(89.69173883,324.167258)(89.66174154,324.09726336)
\curveto(89.64173888,324.03725813)(89.62673889,323.97225819)(89.61674154,323.90226336)
\curveto(89.60673891,323.84225832)(89.60673891,323.78225838)(89.61674154,323.72226336)
\curveto(89.63673888,323.6622585)(89.67173885,323.61225855)(89.72174154,323.57226336)
\curveto(89.80173872,323.52225864)(89.91173861,323.49725867)(90.05174154,323.49726336)
\lineto(90.45674154,323.49726336)
\lineto(92.12174154,323.49726336)
\lineto(92.55674154,323.49726336)
\curveto(92.7167358,323.50725866)(92.8217357,323.55225861)(92.87174154,323.63226336)
}
}
{
\newrgbcolor{curcolor}{0 0 0}
\pscustom[linestyle=none,fillstyle=solid,fillcolor=curcolor]
{
}
}
{
\newrgbcolor{curcolor}{0 0 0}
\pscustom[linestyle=none,fillstyle=solid,fillcolor=curcolor]
{
\newpath
\moveto(104.95517904,326.73726336)
\curveto(105.06517373,326.73725543)(105.16017363,326.72725544)(105.24017904,326.70726336)
\curveto(105.33017346,326.68725548)(105.40017339,326.64225552)(105.45017904,326.57226336)
\curveto(105.51017328,326.49225567)(105.54017325,326.35225581)(105.54017904,326.15226336)
\lineto(105.54017904,325.64226336)
\lineto(105.54017904,325.26726336)
\curveto(105.55017324,325.12725704)(105.53517326,325.01725715)(105.49517904,324.93726336)
\curveto(105.45517334,324.8672573)(105.3951734,324.82225734)(105.31517904,324.80226336)
\curveto(105.24517355,324.78225738)(105.16017363,324.77225739)(105.06017904,324.77226336)
\curveto(104.97017382,324.77225739)(104.87017392,324.77725739)(104.76017904,324.78726336)
\curveto(104.66017413,324.79725737)(104.56517423,324.79225737)(104.47517904,324.77226336)
\curveto(104.40517439,324.75225741)(104.33517446,324.73725743)(104.26517904,324.72726336)
\curveto(104.1951746,324.72725744)(104.13017466,324.71725745)(104.07017904,324.69726336)
\curveto(103.91017488,324.64725752)(103.75017504,324.57225759)(103.59017904,324.47226336)
\curveto(103.43017536,324.38225778)(103.30517549,324.27725789)(103.21517904,324.15726336)
\curveto(103.16517563,324.07725809)(103.11017568,323.99225817)(103.05017904,323.90226336)
\curveto(103.00017579,323.82225834)(102.95017584,323.73725843)(102.90017904,323.64726336)
\curveto(102.87017592,323.5672586)(102.84017595,323.48225868)(102.81017904,323.39226336)
\lineto(102.75017904,323.15226336)
\curveto(102.73017606,323.08225908)(102.72017607,323.00725916)(102.72017904,322.92726336)
\curveto(102.72017607,322.85725931)(102.71017608,322.78725938)(102.69017904,322.71726336)
\curveto(102.68017611,322.67725949)(102.67517612,322.63725953)(102.67517904,322.59726336)
\curveto(102.68517611,322.5672596)(102.68517611,322.53725963)(102.67517904,322.50726336)
\lineto(102.67517904,322.26726336)
\curveto(102.65517614,322.19725997)(102.65017614,322.11726005)(102.66017904,322.02726336)
\curveto(102.67017612,321.94726022)(102.67517612,321.8672603)(102.67517904,321.78726336)
\lineto(102.67517904,320.82726336)
\lineto(102.67517904,319.55226336)
\curveto(102.67517612,319.42226274)(102.67017612,319.30226286)(102.66017904,319.19226336)
\curveto(102.65017614,319.08226308)(102.62017617,318.99226317)(102.57017904,318.92226336)
\curveto(102.55017624,318.89226327)(102.51517628,318.8672633)(102.46517904,318.84726336)
\curveto(102.42517637,318.83726333)(102.38017641,318.82726334)(102.33017904,318.81726336)
\lineto(102.25517904,318.81726336)
\curveto(102.20517659,318.80726336)(102.15017664,318.80226336)(102.09017904,318.80226336)
\lineto(101.92517904,318.80226336)
\lineto(101.28017904,318.80226336)
\curveto(101.22017757,318.81226335)(101.15517764,318.81726335)(101.08517904,318.81726336)
\lineto(100.89017904,318.81726336)
\curveto(100.84017795,318.83726333)(100.790178,318.85226331)(100.74017904,318.86226336)
\curveto(100.6901781,318.88226328)(100.65517814,318.91726325)(100.63517904,318.96726336)
\curveto(100.5951782,319.01726315)(100.57017822,319.08726308)(100.56017904,319.17726336)
\lineto(100.56017904,319.47726336)
\lineto(100.56017904,320.49726336)
\lineto(100.56017904,324.72726336)
\lineto(100.56017904,325.83726336)
\lineto(100.56017904,326.12226336)
\curveto(100.56017823,326.22225594)(100.58017821,326.30225586)(100.62017904,326.36226336)
\curveto(100.67017812,326.44225572)(100.74517805,326.49225567)(100.84517904,326.51226336)
\curveto(100.94517785,326.53225563)(101.06517773,326.54225562)(101.20517904,326.54226336)
\lineto(101.97017904,326.54226336)
\curveto(102.0901767,326.54225562)(102.1951766,326.53225563)(102.28517904,326.51226336)
\curveto(102.37517642,326.50225566)(102.44517635,326.45725571)(102.49517904,326.37726336)
\curveto(102.52517627,326.32725584)(102.54017625,326.25725591)(102.54017904,326.16726336)
\lineto(102.57017904,325.89726336)
\curveto(102.58017621,325.81725635)(102.5951762,325.74225642)(102.61517904,325.67226336)
\curveto(102.64517615,325.60225656)(102.6951761,325.5672566)(102.76517904,325.56726336)
\curveto(102.78517601,325.58725658)(102.80517599,325.59725657)(102.82517904,325.59726336)
\curveto(102.84517595,325.59725657)(102.86517593,325.60725656)(102.88517904,325.62726336)
\curveto(102.94517585,325.67725649)(102.9951758,325.73225643)(103.03517904,325.79226336)
\curveto(103.08517571,325.8622563)(103.14517565,325.92225624)(103.21517904,325.97226336)
\curveto(103.25517554,326.00225616)(103.2901755,326.03225613)(103.32017904,326.06226336)
\curveto(103.35017544,326.10225606)(103.38517541,326.13725603)(103.42517904,326.16726336)
\lineto(103.69517904,326.34726336)
\curveto(103.795175,326.40725576)(103.8951749,326.4622557)(103.99517904,326.51226336)
\curveto(104.0951747,326.55225561)(104.1951746,326.58725558)(104.29517904,326.61726336)
\lineto(104.62517904,326.70726336)
\curveto(104.65517414,326.71725545)(104.71017408,326.71725545)(104.79017904,326.70726336)
\curveto(104.88017391,326.70725546)(104.93517386,326.71725545)(104.95517904,326.73726336)
}
}
{
\newrgbcolor{curcolor}{0 0 0}
\pscustom[linestyle=none,fillstyle=solid,fillcolor=curcolor]
{
\newpath
\moveto(113.46158529,322.74726336)
\curveto(113.48157713,322.6672595)(113.48157713,322.57725959)(113.46158529,322.47726336)
\curveto(113.44157717,322.37725979)(113.4065772,322.31225985)(113.35658529,322.28226336)
\curveto(113.3065773,322.24225992)(113.23157738,322.21225995)(113.13158529,322.19226336)
\curveto(113.04157757,322.18225998)(112.93657767,322.17225999)(112.81658529,322.16226336)
\lineto(112.47158529,322.16226336)
\curveto(112.36157825,322.17225999)(112.26157835,322.17725999)(112.17158529,322.17726336)
\lineto(108.51158529,322.17726336)
\lineto(108.30158529,322.17726336)
\curveto(108.24158237,322.17725999)(108.18658242,322.16726)(108.13658529,322.14726336)
\curveto(108.05658255,322.10726006)(108.0065826,322.0672601)(107.98658529,322.02726336)
\curveto(107.96658264,322.00726016)(107.94658266,321.9672602)(107.92658529,321.90726336)
\curveto(107.9065827,321.85726031)(107.90158271,321.80726036)(107.91158529,321.75726336)
\curveto(107.93158268,321.69726047)(107.94158267,321.63726053)(107.94158529,321.57726336)
\curveto(107.95158266,321.52726064)(107.96658264,321.47226069)(107.98658529,321.41226336)
\curveto(108.06658254,321.17226099)(108.16158245,320.97226119)(108.27158529,320.81226336)
\curveto(108.39158222,320.6622615)(108.55158206,320.52726164)(108.75158529,320.40726336)
\curveto(108.83158178,320.35726181)(108.9115817,320.32226184)(108.99158529,320.30226336)
\curveto(109.08158153,320.29226187)(109.17158144,320.27226189)(109.26158529,320.24226336)
\curveto(109.34158127,320.22226194)(109.45158116,320.20726196)(109.59158529,320.19726336)
\curveto(109.73158088,320.18726198)(109.85158076,320.19226197)(109.95158529,320.21226336)
\lineto(110.08658529,320.21226336)
\curveto(110.18658042,320.23226193)(110.27658033,320.25226191)(110.35658529,320.27226336)
\curveto(110.44658016,320.30226186)(110.53158008,320.33226183)(110.61158529,320.36226336)
\curveto(110.7115799,320.41226175)(110.82157979,320.47726169)(110.94158529,320.55726336)
\curveto(111.07157954,320.63726153)(111.16657944,320.71726145)(111.22658529,320.79726336)
\curveto(111.27657933,320.8672613)(111.32657928,320.93226123)(111.37658529,320.99226336)
\curveto(111.43657917,321.0622611)(111.5065791,321.11226105)(111.58658529,321.14226336)
\curveto(111.68657892,321.19226097)(111.8115788,321.21226095)(111.96158529,321.20226336)
\lineto(112.39658529,321.20226336)
\lineto(112.57658529,321.20226336)
\curveto(112.64657796,321.21226095)(112.7065779,321.20726096)(112.75658529,321.18726336)
\lineto(112.90658529,321.18726336)
\curveto(113.0065776,321.167261)(113.07657753,321.14226102)(113.11658529,321.11226336)
\curveto(113.15657745,321.09226107)(113.17657743,321.04726112)(113.17658529,320.97726336)
\curveto(113.18657742,320.90726126)(113.18157743,320.84726132)(113.16158529,320.79726336)
\curveto(113.1115775,320.65726151)(113.05657755,320.53226163)(112.99658529,320.42226336)
\curveto(112.93657767,320.31226185)(112.86657774,320.20226196)(112.78658529,320.09226336)
\curveto(112.56657804,319.7622624)(112.31657829,319.49726267)(112.03658529,319.29726336)
\curveto(111.75657885,319.09726307)(111.4065792,318.92726324)(110.98658529,318.78726336)
\curveto(110.87657973,318.74726342)(110.76657984,318.72226344)(110.65658529,318.71226336)
\curveto(110.54658006,318.70226346)(110.43158018,318.68226348)(110.31158529,318.65226336)
\curveto(110.27158034,318.64226352)(110.22658038,318.64226352)(110.17658529,318.65226336)
\curveto(110.13658047,318.65226351)(110.09658051,318.64726352)(110.05658529,318.63726336)
\lineto(109.89158529,318.63726336)
\curveto(109.84158077,318.61726355)(109.78158083,318.61226355)(109.71158529,318.62226336)
\curveto(109.65158096,318.62226354)(109.59658101,318.62726354)(109.54658529,318.63726336)
\curveto(109.46658114,318.64726352)(109.39658121,318.64726352)(109.33658529,318.63726336)
\curveto(109.27658133,318.62726354)(109.2115814,318.63226353)(109.14158529,318.65226336)
\curveto(109.09158152,318.67226349)(109.03658157,318.68226348)(108.97658529,318.68226336)
\curveto(108.91658169,318.68226348)(108.86158175,318.69226347)(108.81158529,318.71226336)
\curveto(108.70158191,318.73226343)(108.59158202,318.75726341)(108.48158529,318.78726336)
\curveto(108.37158224,318.80726336)(108.27158234,318.84226332)(108.18158529,318.89226336)
\curveto(108.07158254,318.93226323)(107.96658264,318.9672632)(107.86658529,318.99726336)
\curveto(107.77658283,319.03726313)(107.69158292,319.08226308)(107.61158529,319.13226336)
\curveto(107.29158332,319.33226283)(107.0065836,319.5622626)(106.75658529,319.82226336)
\curveto(106.5065841,320.09226207)(106.30158431,320.40226176)(106.14158529,320.75226336)
\curveto(106.09158452,320.8622613)(106.05158456,320.97226119)(106.02158529,321.08226336)
\curveto(105.99158462,321.20226096)(105.95158466,321.32226084)(105.90158529,321.44226336)
\curveto(105.89158472,321.48226068)(105.88658472,321.51726065)(105.88658529,321.54726336)
\curveto(105.88658472,321.58726058)(105.88158473,321.62726054)(105.87158529,321.66726336)
\curveto(105.83158478,321.78726038)(105.8065848,321.91726025)(105.79658529,322.05726336)
\lineto(105.76658529,322.47726336)
\curveto(105.76658484,322.52725964)(105.76158485,322.58225958)(105.75158529,322.64226336)
\curveto(105.75158486,322.70225946)(105.75658485,322.75725941)(105.76658529,322.80726336)
\lineto(105.76658529,322.98726336)
\lineto(105.81158529,323.34726336)
\curveto(105.85158476,323.51725865)(105.88658472,323.68225848)(105.91658529,323.84226336)
\curveto(105.94658466,324.00225816)(105.99158462,324.15225801)(106.05158529,324.29226336)
\curveto(106.48158413,325.33225683)(107.2115834,326.0672561)(108.24158529,326.49726336)
\curveto(108.38158223,326.55725561)(108.52158209,326.59725557)(108.66158529,326.61726336)
\curveto(108.8115818,326.64725552)(108.96658164,326.68225548)(109.12658529,326.72226336)
\curveto(109.2065814,326.73225543)(109.28158133,326.73725543)(109.35158529,326.73726336)
\curveto(109.42158119,326.73725543)(109.49658111,326.74225542)(109.57658529,326.75226336)
\curveto(110.08658052,326.7622554)(110.52158009,326.70225546)(110.88158529,326.57226336)
\curveto(111.25157936,326.45225571)(111.58157903,326.29225587)(111.87158529,326.09226336)
\curveto(111.96157865,326.03225613)(112.05157856,325.9622562)(112.14158529,325.88226336)
\curveto(112.23157838,325.81225635)(112.3115783,325.73725643)(112.38158529,325.65726336)
\curveto(112.4115782,325.60725656)(112.45157816,325.5672566)(112.50158529,325.53726336)
\curveto(112.58157803,325.42725674)(112.65657795,325.31225685)(112.72658529,325.19226336)
\curveto(112.79657781,325.08225708)(112.87157774,324.9672572)(112.95158529,324.84726336)
\curveto(113.00157761,324.75725741)(113.04157757,324.6622575)(113.07158529,324.56226336)
\curveto(113.1115775,324.47225769)(113.15157746,324.37225779)(113.19158529,324.26226336)
\curveto(113.24157737,324.13225803)(113.28157733,323.99725817)(113.31158529,323.85726336)
\curveto(113.34157727,323.71725845)(113.37657723,323.57725859)(113.41658529,323.43726336)
\curveto(113.43657717,323.35725881)(113.44157717,323.2672589)(113.43158529,323.16726336)
\curveto(113.43157718,323.07725909)(113.44157717,322.99225917)(113.46158529,322.91226336)
\lineto(113.46158529,322.74726336)
\moveto(111.21158529,323.63226336)
\curveto(111.28157933,323.73225843)(111.28657932,323.85225831)(111.22658529,323.99226336)
\curveto(111.17657943,324.14225802)(111.13657947,324.25225791)(111.10658529,324.32226336)
\curveto(110.96657964,324.59225757)(110.78157983,324.79725737)(110.55158529,324.93726336)
\curveto(110.32158029,325.08725708)(110.00158061,325.167257)(109.59158529,325.17726336)
\curveto(109.56158105,325.15725701)(109.52658108,325.15225701)(109.48658529,325.16226336)
\curveto(109.44658116,325.17225699)(109.4115812,325.17225699)(109.38158529,325.16226336)
\curveto(109.33158128,325.14225702)(109.27658133,325.12725704)(109.21658529,325.11726336)
\curveto(109.15658145,325.11725705)(109.10158151,325.10725706)(109.05158529,325.08726336)
\curveto(108.611582,324.94725722)(108.28658232,324.67225749)(108.07658529,324.26226336)
\curveto(108.05658255,324.22225794)(108.03158258,324.167258)(108.00158529,324.09726336)
\curveto(107.98158263,324.03725813)(107.96658264,323.97225819)(107.95658529,323.90226336)
\curveto(107.94658266,323.84225832)(107.94658266,323.78225838)(107.95658529,323.72226336)
\curveto(107.97658263,323.6622585)(108.0115826,323.61225855)(108.06158529,323.57226336)
\curveto(108.14158247,323.52225864)(108.25158236,323.49725867)(108.39158529,323.49726336)
\lineto(108.79658529,323.49726336)
\lineto(110.46158529,323.49726336)
\lineto(110.89658529,323.49726336)
\curveto(111.05657955,323.50725866)(111.16157945,323.55225861)(111.21158529,323.63226336)
}
}
{
\newrgbcolor{curcolor}{0 0 0}
\pscustom[linestyle=none,fillstyle=solid,fillcolor=curcolor]
{
\newpath
\moveto(118.27986654,326.75226336)
\curveto(119.08986138,326.77225539)(119.76486071,326.65225551)(120.30486654,326.39226336)
\curveto(120.85485962,326.13225603)(121.28985918,325.7622564)(121.60986654,325.28226336)
\curveto(121.7698587,325.04225712)(121.88985858,324.7672574)(121.96986654,324.45726336)
\curveto(121.98985848,324.40725776)(122.00485847,324.34225782)(122.01486654,324.26226336)
\curveto(122.03485844,324.18225798)(122.03485844,324.11225805)(122.01486654,324.05226336)
\curveto(121.9748585,323.94225822)(121.90485857,323.87725829)(121.80486654,323.85726336)
\curveto(121.70485877,323.84725832)(121.58485889,323.84225832)(121.44486654,323.84226336)
\lineto(120.66486654,323.84226336)
\lineto(120.37986654,323.84226336)
\curveto(120.28986018,323.84225832)(120.21486026,323.8622583)(120.15486654,323.90226336)
\curveto(120.0748604,323.94225822)(120.01986045,324.00225816)(119.98986654,324.08226336)
\curveto(119.95986051,324.17225799)(119.91986055,324.2622579)(119.86986654,324.35226336)
\curveto(119.80986066,324.4622577)(119.74486073,324.5622576)(119.67486654,324.65226336)
\curveto(119.60486087,324.74225742)(119.52486095,324.82225734)(119.43486654,324.89226336)
\curveto(119.29486118,324.98225718)(119.13986133,325.05225711)(118.96986654,325.10226336)
\curveto(118.90986156,325.12225704)(118.84986162,325.13225703)(118.78986654,325.13226336)
\curveto(118.72986174,325.13225703)(118.6748618,325.14225702)(118.62486654,325.16226336)
\lineto(118.47486654,325.16226336)
\curveto(118.2748622,325.162257)(118.11486236,325.14225702)(117.99486654,325.10226336)
\curveto(117.70486277,325.01225715)(117.469863,324.87225729)(117.28986654,324.68226336)
\curveto(117.10986336,324.50225766)(116.96486351,324.28225788)(116.85486654,324.02226336)
\curveto(116.80486367,323.91225825)(116.76486371,323.79225837)(116.73486654,323.66226336)
\curveto(116.71486376,323.54225862)(116.68986378,323.41225875)(116.65986654,323.27226336)
\curveto(116.64986382,323.23225893)(116.64486383,323.19225897)(116.64486654,323.15226336)
\curveto(116.64486383,323.11225905)(116.63986383,323.07225909)(116.62986654,323.03226336)
\curveto(116.60986386,322.93225923)(116.59986387,322.79225937)(116.59986654,322.61226336)
\curveto(116.60986386,322.43225973)(116.62486385,322.29225987)(116.64486654,322.19226336)
\curveto(116.64486383,322.11226005)(116.64986382,322.05726011)(116.65986654,322.02726336)
\curveto(116.67986379,321.95726021)(116.68986378,321.88726028)(116.68986654,321.81726336)
\curveto(116.69986377,321.74726042)(116.71486376,321.67726049)(116.73486654,321.60726336)
\curveto(116.81486366,321.37726079)(116.90986356,321.167261)(117.01986654,320.97726336)
\curveto(117.12986334,320.78726138)(117.2698632,320.62726154)(117.43986654,320.49726336)
\curveto(117.47986299,320.4672617)(117.53986293,320.43226173)(117.61986654,320.39226336)
\curveto(117.72986274,320.32226184)(117.83986263,320.27726189)(117.94986654,320.25726336)
\curveto(118.0698624,320.23726193)(118.21486226,320.21726195)(118.38486654,320.19726336)
\lineto(118.47486654,320.19726336)
\curveto(118.51486196,320.19726197)(118.54486193,320.20226196)(118.56486654,320.21226336)
\lineto(118.69986654,320.21226336)
\curveto(118.7698617,320.23226193)(118.83486164,320.24726192)(118.89486654,320.25726336)
\curveto(118.96486151,320.27726189)(119.02986144,320.29726187)(119.08986654,320.31726336)
\curveto(119.38986108,320.44726172)(119.61986085,320.63726153)(119.77986654,320.88726336)
\curveto(119.81986065,320.93726123)(119.85486062,320.99226117)(119.88486654,321.05226336)
\curveto(119.91486056,321.12226104)(119.93986053,321.18226098)(119.95986654,321.23226336)
\curveto(119.99986047,321.34226082)(120.03486044,321.43726073)(120.06486654,321.51726336)
\curveto(120.09486038,321.60726056)(120.16486031,321.67726049)(120.27486654,321.72726336)
\curveto(120.36486011,321.7672604)(120.50985996,321.78226038)(120.70986654,321.77226336)
\lineto(121.20486654,321.77226336)
\lineto(121.41486654,321.77226336)
\curveto(121.49485898,321.78226038)(121.55985891,321.77726039)(121.60986654,321.75726336)
\lineto(121.72986654,321.75726336)
\lineto(121.84986654,321.72726336)
\curveto(121.88985858,321.72726044)(121.91985855,321.71726045)(121.93986654,321.69726336)
\curveto(121.98985848,321.65726051)(122.01985845,321.59726057)(122.02986654,321.51726336)
\curveto(122.04985842,321.44726072)(122.04985842,321.37226079)(122.02986654,321.29226336)
\curveto(121.93985853,320.9622612)(121.82985864,320.6672615)(121.69986654,320.40726336)
\curveto(121.28985918,319.63726253)(120.63485984,319.10226306)(119.73486654,318.80226336)
\curveto(119.63486084,318.77226339)(119.52986094,318.75226341)(119.41986654,318.74226336)
\curveto(119.30986116,318.72226344)(119.19986127,318.69726347)(119.08986654,318.66726336)
\curveto(119.02986144,318.65726351)(118.9698615,318.65226351)(118.90986654,318.65226336)
\curveto(118.84986162,318.65226351)(118.78986168,318.64726352)(118.72986654,318.63726336)
\lineto(118.56486654,318.63726336)
\curveto(118.51486196,318.61726355)(118.43986203,318.61226355)(118.33986654,318.62226336)
\curveto(118.23986223,318.62226354)(118.16486231,318.62726354)(118.11486654,318.63726336)
\curveto(118.03486244,318.65726351)(117.95986251,318.6672635)(117.88986654,318.66726336)
\curveto(117.82986264,318.65726351)(117.76486271,318.6622635)(117.69486654,318.68226336)
\lineto(117.54486654,318.71226336)
\curveto(117.49486298,318.71226345)(117.44486303,318.71726345)(117.39486654,318.72726336)
\curveto(117.28486319,318.75726341)(117.17986329,318.78726338)(117.07986654,318.81726336)
\curveto(116.97986349,318.84726332)(116.88486359,318.88226328)(116.79486654,318.92226336)
\curveto(116.32486415,319.12226304)(115.92986454,319.37726279)(115.60986654,319.68726336)
\curveto(115.28986518,320.00726216)(115.02986544,320.40226176)(114.82986654,320.87226336)
\curveto(114.77986569,320.9622612)(114.73986573,321.05726111)(114.70986654,321.15726336)
\lineto(114.61986654,321.48726336)
\curveto(114.60986586,321.52726064)(114.60486587,321.5622606)(114.60486654,321.59226336)
\curveto(114.60486587,321.63226053)(114.59486588,321.67726049)(114.57486654,321.72726336)
\curveto(114.55486592,321.79726037)(114.54486593,321.8672603)(114.54486654,321.93726336)
\curveto(114.54486593,322.01726015)(114.53486594,322.09226007)(114.51486654,322.16226336)
\lineto(114.51486654,322.41726336)
\curveto(114.49486598,322.4672597)(114.48486599,322.52225964)(114.48486654,322.58226336)
\curveto(114.48486599,322.65225951)(114.49486598,322.71225945)(114.51486654,322.76226336)
\curveto(114.52486595,322.81225935)(114.52486595,322.85725931)(114.51486654,322.89726336)
\curveto(114.50486597,322.93725923)(114.50486597,322.97725919)(114.51486654,323.01726336)
\curveto(114.53486594,323.08725908)(114.53986593,323.15225901)(114.52986654,323.21226336)
\curveto(114.52986594,323.27225889)(114.53986593,323.33225883)(114.55986654,323.39226336)
\curveto(114.60986586,323.57225859)(114.64986582,323.74225842)(114.67986654,323.90226336)
\curveto(114.70986576,324.07225809)(114.75486572,324.23725793)(114.81486654,324.39726336)
\curveto(115.03486544,324.90725726)(115.30986516,325.33225683)(115.63986654,325.67226336)
\curveto(115.97986449,326.01225615)(116.40986406,326.28725588)(116.92986654,326.49726336)
\curveto(117.0698634,326.55725561)(117.21486326,326.59725557)(117.36486654,326.61726336)
\curveto(117.51486296,326.64725552)(117.6698628,326.68225548)(117.82986654,326.72226336)
\curveto(117.90986256,326.73225543)(117.98486249,326.73725543)(118.05486654,326.73726336)
\curveto(118.12486235,326.73725543)(118.19986227,326.74225542)(118.27986654,326.75226336)
}
}
{
\newrgbcolor{curcolor}{0 0 0}
\pscustom[linestyle=none,fillstyle=solid,fillcolor=curcolor]
{
\newpath
\moveto(123.74314779,326.52726336)
\lineto(124.86814779,326.52726336)
\curveto(124.97814536,326.52725564)(125.07814526,326.52225564)(125.16814779,326.51226336)
\curveto(125.25814508,326.50225566)(125.32314501,326.4672557)(125.36314779,326.40726336)
\curveto(125.41314492,326.34725582)(125.44314489,326.2622559)(125.45314779,326.15226336)
\curveto(125.46314487,326.05225611)(125.46814487,325.94725622)(125.46814779,325.83726336)
\lineto(125.46814779,324.78726336)
\lineto(125.46814779,322.55226336)
\curveto(125.46814487,322.19225997)(125.48314485,321.85226031)(125.51314779,321.53226336)
\curveto(125.54314479,321.21226095)(125.6331447,320.94726122)(125.78314779,320.73726336)
\curveto(125.92314441,320.52726164)(126.14814419,320.37726179)(126.45814779,320.28726336)
\curveto(126.50814383,320.27726189)(126.54814379,320.27226189)(126.57814779,320.27226336)
\curveto(126.61814372,320.27226189)(126.66314367,320.2672619)(126.71314779,320.25726336)
\curveto(126.76314357,320.24726192)(126.81814352,320.24226192)(126.87814779,320.24226336)
\curveto(126.9381434,320.24226192)(126.98314335,320.24726192)(127.01314779,320.25726336)
\curveto(127.06314327,320.27726189)(127.10314323,320.28226188)(127.13314779,320.27226336)
\curveto(127.17314316,320.2622619)(127.21314312,320.2672619)(127.25314779,320.28726336)
\curveto(127.46314287,320.33726183)(127.62814271,320.40226176)(127.74814779,320.48226336)
\curveto(127.92814241,320.59226157)(128.06814227,320.73226143)(128.16814779,320.90226336)
\curveto(128.27814206,321.08226108)(128.35314198,321.27726089)(128.39314779,321.48726336)
\curveto(128.44314189,321.70726046)(128.47314186,321.94726022)(128.48314779,322.20726336)
\curveto(128.49314184,322.47725969)(128.49814184,322.75725941)(128.49814779,323.04726336)
\lineto(128.49814779,324.86226336)
\lineto(128.49814779,325.83726336)
\lineto(128.49814779,326.10726336)
\curveto(128.49814184,326.20725596)(128.51814182,326.28725588)(128.55814779,326.34726336)
\curveto(128.60814173,326.43725573)(128.68314165,326.48725568)(128.78314779,326.49726336)
\curveto(128.88314145,326.51725565)(129.00314133,326.52725564)(129.14314779,326.52726336)
\lineto(129.93814779,326.52726336)
\lineto(130.22314779,326.52726336)
\curveto(130.31314002,326.52725564)(130.38813995,326.50725566)(130.44814779,326.46726336)
\curveto(130.52813981,326.41725575)(130.57313976,326.34225582)(130.58314779,326.24226336)
\curveto(130.59313974,326.14225602)(130.59813974,326.02725614)(130.59814779,325.89726336)
\lineto(130.59814779,324.75726336)
\lineto(130.59814779,320.54226336)
\lineto(130.59814779,319.47726336)
\lineto(130.59814779,319.17726336)
\curveto(130.59813974,319.07726309)(130.57813976,319.00226316)(130.53814779,318.95226336)
\curveto(130.48813985,318.87226329)(130.41313992,318.82726334)(130.31314779,318.81726336)
\curveto(130.21314012,318.80726336)(130.10814023,318.80226336)(129.99814779,318.80226336)
\lineto(129.18814779,318.80226336)
\curveto(129.07814126,318.80226336)(128.97814136,318.80726336)(128.88814779,318.81726336)
\curveto(128.80814153,318.82726334)(128.74314159,318.8672633)(128.69314779,318.93726336)
\curveto(128.67314166,318.9672632)(128.65314168,319.01226315)(128.63314779,319.07226336)
\curveto(128.62314171,319.13226303)(128.60814173,319.19226297)(128.58814779,319.25226336)
\curveto(128.57814176,319.31226285)(128.56314177,319.3672628)(128.54314779,319.41726336)
\curveto(128.52314181,319.4672627)(128.49314184,319.49726267)(128.45314779,319.50726336)
\curveto(128.4331419,319.52726264)(128.40814193,319.53226263)(128.37814779,319.52226336)
\curveto(128.34814199,319.51226265)(128.32314201,319.50226266)(128.30314779,319.49226336)
\curveto(128.2331421,319.45226271)(128.17314216,319.40726276)(128.12314779,319.35726336)
\curveto(128.07314226,319.30726286)(128.01814232,319.2622629)(127.95814779,319.22226336)
\curveto(127.91814242,319.19226297)(127.87814246,319.15726301)(127.83814779,319.11726336)
\curveto(127.80814253,319.08726308)(127.76814257,319.05726311)(127.71814779,319.02726336)
\curveto(127.48814285,318.88726328)(127.21814312,318.77726339)(126.90814779,318.69726336)
\curveto(126.8381435,318.67726349)(126.76814357,318.6672635)(126.69814779,318.66726336)
\curveto(126.62814371,318.65726351)(126.55314378,318.64226352)(126.47314779,318.62226336)
\curveto(126.4331439,318.61226355)(126.38814395,318.61226355)(126.33814779,318.62226336)
\curveto(126.29814404,318.62226354)(126.25814408,318.61726355)(126.21814779,318.60726336)
\curveto(126.18814415,318.59726357)(126.12314421,318.59726357)(126.02314779,318.60726336)
\curveto(125.9331444,318.60726356)(125.87314446,318.61226355)(125.84314779,318.62226336)
\curveto(125.79314454,318.62226354)(125.74314459,318.62726354)(125.69314779,318.63726336)
\lineto(125.54314779,318.63726336)
\curveto(125.42314491,318.6672635)(125.30814503,318.69226347)(125.19814779,318.71226336)
\curveto(125.08814525,318.73226343)(124.97814536,318.7622634)(124.86814779,318.80226336)
\curveto(124.81814552,318.82226334)(124.77314556,318.83726333)(124.73314779,318.84726336)
\curveto(124.70314563,318.8672633)(124.66314567,318.88726328)(124.61314779,318.90726336)
\curveto(124.26314607,319.09726307)(123.98314635,319.3622628)(123.77314779,319.70226336)
\curveto(123.64314669,319.91226225)(123.54814679,320.162262)(123.48814779,320.45226336)
\curveto(123.42814691,320.75226141)(123.38814695,321.0672611)(123.36814779,321.39726336)
\curveto(123.35814698,321.73726043)(123.35314698,322.08226008)(123.35314779,322.43226336)
\curveto(123.36314697,322.79225937)(123.36814697,323.14725902)(123.36814779,323.49726336)
\lineto(123.36814779,325.53726336)
\curveto(123.36814697,325.6672565)(123.36314697,325.81725635)(123.35314779,325.98726336)
\curveto(123.35314698,326.167256)(123.37814696,326.29725587)(123.42814779,326.37726336)
\curveto(123.45814688,326.42725574)(123.51814682,326.47225569)(123.60814779,326.51226336)
\curveto(123.66814667,326.51225565)(123.71314662,326.51725565)(123.74314779,326.52726336)
}
}
{
\newrgbcolor{curcolor}{0 0 0}
\pscustom[linestyle=none,fillstyle=solid,fillcolor=curcolor]
{
\newpath
\moveto(136.65439779,326.73726336)
\curveto(136.76439248,326.73725543)(136.85939238,326.72725544)(136.93939779,326.70726336)
\curveto(137.02939221,326.68725548)(137.09939214,326.64225552)(137.14939779,326.57226336)
\curveto(137.20939203,326.49225567)(137.239392,326.35225581)(137.23939779,326.15226336)
\lineto(137.23939779,325.64226336)
\lineto(137.23939779,325.26726336)
\curveto(137.24939199,325.12725704)(137.23439201,325.01725715)(137.19439779,324.93726336)
\curveto(137.15439209,324.8672573)(137.09439215,324.82225734)(137.01439779,324.80226336)
\curveto(136.9443923,324.78225738)(136.85939238,324.77225739)(136.75939779,324.77226336)
\curveto(136.66939257,324.77225739)(136.56939267,324.77725739)(136.45939779,324.78726336)
\curveto(136.35939288,324.79725737)(136.26439298,324.79225737)(136.17439779,324.77226336)
\curveto(136.10439314,324.75225741)(136.03439321,324.73725743)(135.96439779,324.72726336)
\curveto(135.89439335,324.72725744)(135.82939341,324.71725745)(135.76939779,324.69726336)
\curveto(135.60939363,324.64725752)(135.44939379,324.57225759)(135.28939779,324.47226336)
\curveto(135.12939411,324.38225778)(135.00439424,324.27725789)(134.91439779,324.15726336)
\curveto(134.86439438,324.07725809)(134.80939443,323.99225817)(134.74939779,323.90226336)
\curveto(134.69939454,323.82225834)(134.64939459,323.73725843)(134.59939779,323.64726336)
\curveto(134.56939467,323.5672586)(134.5393947,323.48225868)(134.50939779,323.39226336)
\lineto(134.44939779,323.15226336)
\curveto(134.42939481,323.08225908)(134.41939482,323.00725916)(134.41939779,322.92726336)
\curveto(134.41939482,322.85725931)(134.40939483,322.78725938)(134.38939779,322.71726336)
\curveto(134.37939486,322.67725949)(134.37439487,322.63725953)(134.37439779,322.59726336)
\curveto(134.38439486,322.5672596)(134.38439486,322.53725963)(134.37439779,322.50726336)
\lineto(134.37439779,322.26726336)
\curveto(134.35439489,322.19725997)(134.34939489,322.11726005)(134.35939779,322.02726336)
\curveto(134.36939487,321.94726022)(134.37439487,321.8672603)(134.37439779,321.78726336)
\lineto(134.37439779,320.82726336)
\lineto(134.37439779,319.55226336)
\curveto(134.37439487,319.42226274)(134.36939487,319.30226286)(134.35939779,319.19226336)
\curveto(134.34939489,319.08226308)(134.31939492,318.99226317)(134.26939779,318.92226336)
\curveto(134.24939499,318.89226327)(134.21439503,318.8672633)(134.16439779,318.84726336)
\curveto(134.12439512,318.83726333)(134.07939516,318.82726334)(134.02939779,318.81726336)
\lineto(133.95439779,318.81726336)
\curveto(133.90439534,318.80726336)(133.84939539,318.80226336)(133.78939779,318.80226336)
\lineto(133.62439779,318.80226336)
\lineto(132.97939779,318.80226336)
\curveto(132.91939632,318.81226335)(132.85439639,318.81726335)(132.78439779,318.81726336)
\lineto(132.58939779,318.81726336)
\curveto(132.5393967,318.83726333)(132.48939675,318.85226331)(132.43939779,318.86226336)
\curveto(132.38939685,318.88226328)(132.35439689,318.91726325)(132.33439779,318.96726336)
\curveto(132.29439695,319.01726315)(132.26939697,319.08726308)(132.25939779,319.17726336)
\lineto(132.25939779,319.47726336)
\lineto(132.25939779,320.49726336)
\lineto(132.25939779,324.72726336)
\lineto(132.25939779,325.83726336)
\lineto(132.25939779,326.12226336)
\curveto(132.25939698,326.22225594)(132.27939696,326.30225586)(132.31939779,326.36226336)
\curveto(132.36939687,326.44225572)(132.4443968,326.49225567)(132.54439779,326.51226336)
\curveto(132.6443966,326.53225563)(132.76439648,326.54225562)(132.90439779,326.54226336)
\lineto(133.66939779,326.54226336)
\curveto(133.78939545,326.54225562)(133.89439535,326.53225563)(133.98439779,326.51226336)
\curveto(134.07439517,326.50225566)(134.1443951,326.45725571)(134.19439779,326.37726336)
\curveto(134.22439502,326.32725584)(134.239395,326.25725591)(134.23939779,326.16726336)
\lineto(134.26939779,325.89726336)
\curveto(134.27939496,325.81725635)(134.29439495,325.74225642)(134.31439779,325.67226336)
\curveto(134.3443949,325.60225656)(134.39439485,325.5672566)(134.46439779,325.56726336)
\curveto(134.48439476,325.58725658)(134.50439474,325.59725657)(134.52439779,325.59726336)
\curveto(134.5443947,325.59725657)(134.56439468,325.60725656)(134.58439779,325.62726336)
\curveto(134.6443946,325.67725649)(134.69439455,325.73225643)(134.73439779,325.79226336)
\curveto(134.78439446,325.8622563)(134.8443944,325.92225624)(134.91439779,325.97226336)
\curveto(134.95439429,326.00225616)(134.98939425,326.03225613)(135.01939779,326.06226336)
\curveto(135.04939419,326.10225606)(135.08439416,326.13725603)(135.12439779,326.16726336)
\lineto(135.39439779,326.34726336)
\curveto(135.49439375,326.40725576)(135.59439365,326.4622557)(135.69439779,326.51226336)
\curveto(135.79439345,326.55225561)(135.89439335,326.58725558)(135.99439779,326.61726336)
\lineto(136.32439779,326.70726336)
\curveto(136.35439289,326.71725545)(136.40939283,326.71725545)(136.48939779,326.70726336)
\curveto(136.57939266,326.70725546)(136.63439261,326.71725545)(136.65439779,326.73726336)
}
}
{
\newrgbcolor{curcolor}{0 0 0}
\pscustom[linestyle=none,fillstyle=solid,fillcolor=curcolor]
{
\newpath
\moveto(141.02947592,326.75226336)
\curveto(141.77947142,326.77225539)(142.42947077,326.68725548)(142.97947592,326.49726336)
\curveto(143.53946966,326.31725585)(143.96446923,326.00225616)(144.25447592,325.55226336)
\curveto(144.32446887,325.44225672)(144.38446881,325.32725684)(144.43447592,325.20726336)
\curveto(144.4944687,325.09725707)(144.54446865,324.97225719)(144.58447592,324.83226336)
\curveto(144.60446859,324.77225739)(144.61446858,324.70725746)(144.61447592,324.63726336)
\curveto(144.61446858,324.5672576)(144.60446859,324.50725766)(144.58447592,324.45726336)
\curveto(144.54446865,324.39725777)(144.48946871,324.35725781)(144.41947592,324.33726336)
\curveto(144.36946883,324.31725785)(144.30946889,324.30725786)(144.23947592,324.30726336)
\lineto(144.02947592,324.30726336)
\lineto(143.36947592,324.30726336)
\curveto(143.2994699,324.30725786)(143.22946997,324.30225786)(143.15947592,324.29226336)
\curveto(143.08947011,324.29225787)(143.02447017,324.30225786)(142.96447592,324.32226336)
\curveto(142.86447033,324.34225782)(142.78947041,324.38225778)(142.73947592,324.44226336)
\curveto(142.68947051,324.50225766)(142.64447055,324.5622576)(142.60447592,324.62226336)
\lineto(142.48447592,324.83226336)
\curveto(142.45447074,324.91225725)(142.40447079,324.97725719)(142.33447592,325.02726336)
\curveto(142.23447096,325.10725706)(142.13447106,325.167257)(142.03447592,325.20726336)
\curveto(141.94447125,325.24725692)(141.82947137,325.28225688)(141.68947592,325.31226336)
\curveto(141.61947158,325.33225683)(141.51447168,325.34725682)(141.37447592,325.35726336)
\curveto(141.24447195,325.3672568)(141.14447205,325.3622568)(141.07447592,325.34226336)
\lineto(140.96947592,325.34226336)
\lineto(140.81947592,325.31226336)
\curveto(140.77947242,325.31225685)(140.73447246,325.30725686)(140.68447592,325.29726336)
\curveto(140.51447268,325.24725692)(140.37447282,325.17725699)(140.26447592,325.08726336)
\curveto(140.16447303,325.00725716)(140.0944731,324.88225728)(140.05447592,324.71226336)
\curveto(140.03447316,324.64225752)(140.03447316,324.57725759)(140.05447592,324.51726336)
\curveto(140.07447312,324.45725771)(140.0944731,324.40725776)(140.11447592,324.36726336)
\curveto(140.18447301,324.24725792)(140.26447293,324.15225801)(140.35447592,324.08226336)
\curveto(140.45447274,324.01225815)(140.56947263,323.95225821)(140.69947592,323.90226336)
\curveto(140.88947231,323.82225834)(141.0944721,323.75225841)(141.31447592,323.69226336)
\lineto(142.00447592,323.54226336)
\curveto(142.24447095,323.50225866)(142.47447072,323.45225871)(142.69447592,323.39226336)
\curveto(142.92447027,323.34225882)(143.13947006,323.27725889)(143.33947592,323.19726336)
\curveto(143.42946977,323.15725901)(143.51446968,323.12225904)(143.59447592,323.09226336)
\curveto(143.68446951,323.07225909)(143.76946943,323.03725913)(143.84947592,322.98726336)
\curveto(144.03946916,322.8672593)(144.20946899,322.73725943)(144.35947592,322.59726336)
\curveto(144.51946868,322.45725971)(144.64446855,322.28225988)(144.73447592,322.07226336)
\curveto(144.76446843,322.00226016)(144.78946841,321.93226023)(144.80947592,321.86226336)
\curveto(144.82946837,321.79226037)(144.84946835,321.71726045)(144.86947592,321.63726336)
\curveto(144.87946832,321.57726059)(144.88446831,321.48226068)(144.88447592,321.35226336)
\curveto(144.8944683,321.23226093)(144.8944683,321.13726103)(144.88447592,321.06726336)
\lineto(144.88447592,320.99226336)
\curveto(144.86446833,320.93226123)(144.84946835,320.87226129)(144.83947592,320.81226336)
\curveto(144.83946836,320.7622614)(144.83446836,320.71226145)(144.82447592,320.66226336)
\curveto(144.75446844,320.3622618)(144.64446855,320.09726207)(144.49447592,319.86726336)
\curveto(144.33446886,319.62726254)(144.13946906,319.43226273)(143.90947592,319.28226336)
\curveto(143.67946952,319.13226303)(143.41946978,319.00226316)(143.12947592,318.89226336)
\curveto(143.01947018,318.84226332)(142.8994703,318.80726336)(142.76947592,318.78726336)
\curveto(142.64947055,318.7672634)(142.52947067,318.74226342)(142.40947592,318.71226336)
\curveto(142.31947088,318.69226347)(142.22447097,318.68226348)(142.12447592,318.68226336)
\curveto(142.03447116,318.67226349)(141.94447125,318.65726351)(141.85447592,318.63726336)
\lineto(141.58447592,318.63726336)
\curveto(141.52447167,318.61726355)(141.41947178,318.60726356)(141.26947592,318.60726336)
\curveto(141.12947207,318.60726356)(141.02947217,318.61726355)(140.96947592,318.63726336)
\curveto(140.93947226,318.63726353)(140.90447229,318.64226352)(140.86447592,318.65226336)
\lineto(140.75947592,318.65226336)
\curveto(140.63947256,318.67226349)(140.51947268,318.68726348)(140.39947592,318.69726336)
\curveto(140.27947292,318.70726346)(140.16447303,318.72726344)(140.05447592,318.75726336)
\curveto(139.66447353,318.8672633)(139.31947388,318.99226317)(139.01947592,319.13226336)
\curveto(138.71947448,319.28226288)(138.46447473,319.50226266)(138.25447592,319.79226336)
\curveto(138.11447508,319.98226218)(137.9944752,320.20226196)(137.89447592,320.45226336)
\curveto(137.87447532,320.51226165)(137.85447534,320.59226157)(137.83447592,320.69226336)
\curveto(137.81447538,320.74226142)(137.7994754,320.81226135)(137.78947592,320.90226336)
\curveto(137.77947542,320.99226117)(137.78447541,321.0672611)(137.80447592,321.12726336)
\curveto(137.83447536,321.19726097)(137.88447531,321.24726092)(137.95447592,321.27726336)
\curveto(138.00447519,321.29726087)(138.06447513,321.30726086)(138.13447592,321.30726336)
\lineto(138.35947592,321.30726336)
\lineto(139.06447592,321.30726336)
\lineto(139.30447592,321.30726336)
\curveto(139.38447381,321.30726086)(139.45447374,321.29726087)(139.51447592,321.27726336)
\curveto(139.62447357,321.23726093)(139.6944735,321.17226099)(139.72447592,321.08226336)
\curveto(139.76447343,320.99226117)(139.80947339,320.89726127)(139.85947592,320.79726336)
\curveto(139.87947332,320.74726142)(139.91447328,320.68226148)(139.96447592,320.60226336)
\curveto(140.02447317,320.52226164)(140.07447312,320.47226169)(140.11447592,320.45226336)
\curveto(140.23447296,320.35226181)(140.34947285,320.27226189)(140.45947592,320.21226336)
\curveto(140.56947263,320.162262)(140.70947249,320.11226205)(140.87947592,320.06226336)
\curveto(140.92947227,320.04226212)(140.97947222,320.03226213)(141.02947592,320.03226336)
\curveto(141.07947212,320.04226212)(141.12947207,320.04226212)(141.17947592,320.03226336)
\curveto(141.25947194,320.01226215)(141.34447185,320.00226216)(141.43447592,320.00226336)
\curveto(141.53447166,320.01226215)(141.61947158,320.02726214)(141.68947592,320.04726336)
\curveto(141.73947146,320.05726211)(141.78447141,320.0622621)(141.82447592,320.06226336)
\curveto(141.87447132,320.0622621)(141.92447127,320.07226209)(141.97447592,320.09226336)
\curveto(142.11447108,320.14226202)(142.23947096,320.20226196)(142.34947592,320.27226336)
\curveto(142.46947073,320.34226182)(142.56447063,320.43226173)(142.63447592,320.54226336)
\curveto(142.68447051,320.62226154)(142.72447047,320.74726142)(142.75447592,320.91726336)
\curveto(142.77447042,320.98726118)(142.77447042,321.05226111)(142.75447592,321.11226336)
\curveto(142.73447046,321.17226099)(142.71447048,321.22226094)(142.69447592,321.26226336)
\curveto(142.62447057,321.40226076)(142.53447066,321.50726066)(142.42447592,321.57726336)
\curveto(142.32447087,321.64726052)(142.20447099,321.71226045)(142.06447592,321.77226336)
\curveto(141.87447132,321.85226031)(141.67447152,321.91726025)(141.46447592,321.96726336)
\curveto(141.25447194,322.01726015)(141.04447215,322.07226009)(140.83447592,322.13226336)
\curveto(140.75447244,322.15226001)(140.66947253,322.16726)(140.57947592,322.17726336)
\curveto(140.4994727,322.18725998)(140.41947278,322.20225996)(140.33947592,322.22226336)
\curveto(140.01947318,322.31225985)(139.71447348,322.39725977)(139.42447592,322.47726336)
\curveto(139.13447406,322.5672596)(138.86947433,322.69725947)(138.62947592,322.86726336)
\curveto(138.34947485,323.0672591)(138.14447505,323.33725883)(138.01447592,323.67726336)
\curveto(137.9944752,323.74725842)(137.97447522,323.84225832)(137.95447592,323.96226336)
\curveto(137.93447526,324.03225813)(137.91947528,324.11725805)(137.90947592,324.21726336)
\curveto(137.8994753,324.31725785)(137.90447529,324.40725776)(137.92447592,324.48726336)
\curveto(137.94447525,324.53725763)(137.94947525,324.57725759)(137.93947592,324.60726336)
\curveto(137.92947527,324.64725752)(137.93447526,324.69225747)(137.95447592,324.74226336)
\curveto(137.97447522,324.85225731)(137.9944752,324.95225721)(138.01447592,325.04226336)
\curveto(138.04447515,325.14225702)(138.07947512,325.23725693)(138.11947592,325.32726336)
\curveto(138.24947495,325.61725655)(138.42947477,325.85225631)(138.65947592,326.03226336)
\curveto(138.88947431,326.21225595)(139.14947405,326.35725581)(139.43947592,326.46726336)
\curveto(139.54947365,326.51725565)(139.66447353,326.55225561)(139.78447592,326.57226336)
\curveto(139.90447329,326.60225556)(140.02947317,326.63225553)(140.15947592,326.66226336)
\curveto(140.21947298,326.68225548)(140.27947292,326.69225547)(140.33947592,326.69226336)
\lineto(140.51947592,326.72226336)
\curveto(140.5994726,326.73225543)(140.68447251,326.73725543)(140.77447592,326.73726336)
\curveto(140.86447233,326.73725543)(140.94947225,326.74225542)(141.02947592,326.75226336)
}
}
{
\newrgbcolor{curcolor}{0 0 0}
\pscustom[linestyle=none,fillstyle=solid,fillcolor=curcolor]
{
\newpath
\moveto(153.88611654,322.98726336)
\curveto(153.90610797,322.92725924)(153.91610796,322.84225932)(153.91611654,322.73226336)
\curveto(153.91610796,322.62225954)(153.90610797,322.53725963)(153.88611654,322.47726336)
\lineto(153.88611654,322.32726336)
\curveto(153.86610801,322.24725992)(153.85610802,322.16726)(153.85611654,322.08726336)
\curveto(153.86610801,322.00726016)(153.86110802,321.92726024)(153.84111654,321.84726336)
\curveto(153.82110806,321.77726039)(153.80610807,321.71226045)(153.79611654,321.65226336)
\curveto(153.78610809,321.59226057)(153.7761081,321.52726064)(153.76611654,321.45726336)
\curveto(153.72610815,321.34726082)(153.69110819,321.23226093)(153.66111654,321.11226336)
\curveto(153.63110825,321.00226116)(153.59110829,320.89726127)(153.54111654,320.79726336)
\curveto(153.33110855,320.31726185)(153.05610882,319.92726224)(152.71611654,319.62726336)
\curveto(152.3761095,319.32726284)(151.96610991,319.07726309)(151.48611654,318.87726336)
\curveto(151.36611051,318.82726334)(151.24111064,318.79226337)(151.11111654,318.77226336)
\curveto(150.99111089,318.74226342)(150.86611101,318.71226345)(150.73611654,318.68226336)
\curveto(150.68611119,318.6622635)(150.63111125,318.65226351)(150.57111654,318.65226336)
\curveto(150.51111137,318.65226351)(150.45611142,318.64726352)(150.40611654,318.63726336)
\lineto(150.30111654,318.63726336)
\curveto(150.27111161,318.62726354)(150.24111164,318.62226354)(150.21111654,318.62226336)
\curveto(150.16111172,318.61226355)(150.0811118,318.60726356)(149.97111654,318.60726336)
\curveto(149.86111202,318.59726357)(149.7761121,318.60226356)(149.71611654,318.62226336)
\lineto(149.56611654,318.62226336)
\curveto(149.51611236,318.63226353)(149.46111242,318.63726353)(149.40111654,318.63726336)
\curveto(149.35111253,318.62726354)(149.30111258,318.63226353)(149.25111654,318.65226336)
\curveto(149.21111267,318.6622635)(149.17111271,318.6672635)(149.13111654,318.66726336)
\curveto(149.10111278,318.6672635)(149.06111282,318.67226349)(149.01111654,318.68226336)
\curveto(148.91111297,318.71226345)(148.81111307,318.73726343)(148.71111654,318.75726336)
\curveto(148.61111327,318.77726339)(148.51611336,318.80726336)(148.42611654,318.84726336)
\curveto(148.30611357,318.88726328)(148.19111369,318.92726324)(148.08111654,318.96726336)
\curveto(147.9811139,319.00726316)(147.876114,319.05726311)(147.76611654,319.11726336)
\curveto(147.41611446,319.32726284)(147.11611476,319.57226259)(146.86611654,319.85226336)
\curveto(146.61611526,320.13226203)(146.40611547,320.4672617)(146.23611654,320.85726336)
\curveto(146.18611569,320.94726122)(146.14611573,321.04226112)(146.11611654,321.14226336)
\curveto(146.09611578,321.24226092)(146.07111581,321.34726082)(146.04111654,321.45726336)
\curveto(146.02111586,321.50726066)(146.01111587,321.55226061)(146.01111654,321.59226336)
\curveto(146.01111587,321.63226053)(146.00111588,321.67726049)(145.98111654,321.72726336)
\curveto(145.96111592,321.80726036)(145.95111593,321.88726028)(145.95111654,321.96726336)
\curveto(145.95111593,322.05726011)(145.94111594,322.14226002)(145.92111654,322.22226336)
\curveto(145.91111597,322.27225989)(145.90611597,322.31725985)(145.90611654,322.35726336)
\lineto(145.90611654,322.49226336)
\curveto(145.88611599,322.55225961)(145.876116,322.63725953)(145.87611654,322.74726336)
\curveto(145.88611599,322.85725931)(145.90111598,322.94225922)(145.92111654,323.00226336)
\lineto(145.92111654,323.10726336)
\curveto(145.93111595,323.15725901)(145.93111595,323.20725896)(145.92111654,323.25726336)
\curveto(145.92111596,323.31725885)(145.93111595,323.37225879)(145.95111654,323.42226336)
\curveto(145.96111592,323.47225869)(145.96611591,323.51725865)(145.96611654,323.55726336)
\curveto(145.96611591,323.60725856)(145.9761159,323.65725851)(145.99611654,323.70726336)
\curveto(146.03611584,323.83725833)(146.07111581,323.9622582)(146.10111654,324.08226336)
\curveto(146.13111575,324.21225795)(146.17111571,324.33725783)(146.22111654,324.45726336)
\curveto(146.40111548,324.8672573)(146.61611526,325.20725696)(146.86611654,325.47726336)
\curveto(147.11611476,325.75725641)(147.42111446,326.01225615)(147.78111654,326.24226336)
\curveto(147.881114,326.29225587)(147.98611389,326.33725583)(148.09611654,326.37726336)
\curveto(148.20611367,326.41725575)(148.31611356,326.4622557)(148.42611654,326.51226336)
\curveto(148.55611332,326.5622556)(148.69111319,326.59725557)(148.83111654,326.61726336)
\curveto(148.97111291,326.63725553)(149.11611276,326.6672555)(149.26611654,326.70726336)
\curveto(149.34611253,326.71725545)(149.42111246,326.72225544)(149.49111654,326.72226336)
\curveto(149.56111232,326.72225544)(149.63111225,326.72725544)(149.70111654,326.73726336)
\curveto(150.2811116,326.74725542)(150.7811111,326.68725548)(151.20111654,326.55726336)
\curveto(151.63111025,326.42725574)(152.01110987,326.24725592)(152.34111654,326.01726336)
\curveto(152.45110943,325.93725623)(152.56110932,325.84725632)(152.67111654,325.74726336)
\curveto(152.79110909,325.65725651)(152.89110899,325.55725661)(152.97111654,325.44726336)
\curveto(153.05110883,325.34725682)(153.12110876,325.24725692)(153.18111654,325.14726336)
\curveto(153.25110863,325.04725712)(153.32110856,324.94225722)(153.39111654,324.83226336)
\curveto(153.46110842,324.72225744)(153.51610836,324.60225756)(153.55611654,324.47226336)
\curveto(153.59610828,324.35225781)(153.64110824,324.22225794)(153.69111654,324.08226336)
\curveto(153.72110816,324.00225816)(153.74610813,323.91725825)(153.76611654,323.82726336)
\lineto(153.82611654,323.55726336)
\curveto(153.83610804,323.51725865)(153.84110804,323.47725869)(153.84111654,323.43726336)
\curveto(153.84110804,323.39725877)(153.84610803,323.35725881)(153.85611654,323.31726336)
\curveto(153.876108,323.2672589)(153.881108,323.21225895)(153.87111654,323.15226336)
\curveto(153.86110802,323.09225907)(153.86610801,323.03725913)(153.88611654,322.98726336)
\moveto(151.78611654,322.44726336)
\curveto(151.79611008,322.49725967)(151.80111008,322.5672596)(151.80111654,322.65726336)
\curveto(151.80111008,322.75725941)(151.79611008,322.83225933)(151.78611654,322.88226336)
\lineto(151.78611654,323.00226336)
\curveto(151.76611011,323.05225911)(151.75611012,323.10725906)(151.75611654,323.16726336)
\curveto(151.75611012,323.22725894)(151.75111013,323.28225888)(151.74111654,323.33226336)
\curveto(151.74111014,323.37225879)(151.73611014,323.40225876)(151.72611654,323.42226336)
\lineto(151.66611654,323.66226336)
\curveto(151.65611022,323.75225841)(151.63611024,323.83725833)(151.60611654,323.91726336)
\curveto(151.49611038,324.17725799)(151.36611051,324.39725777)(151.21611654,324.57726336)
\curveto(151.06611081,324.7672574)(150.86611101,324.91725725)(150.61611654,325.02726336)
\curveto(150.55611132,325.04725712)(150.49611138,325.0622571)(150.43611654,325.07226336)
\curveto(150.3761115,325.09225707)(150.31111157,325.11225705)(150.24111654,325.13226336)
\curveto(150.16111172,325.15225701)(150.0761118,325.15725701)(149.98611654,325.14726336)
\lineto(149.71611654,325.14726336)
\curveto(149.68611219,325.12725704)(149.65111223,325.11725705)(149.61111654,325.11726336)
\curveto(149.57111231,325.12725704)(149.53611234,325.12725704)(149.50611654,325.11726336)
\lineto(149.29611654,325.05726336)
\curveto(149.23611264,325.04725712)(149.1811127,325.02725714)(149.13111654,324.99726336)
\curveto(148.881113,324.88725728)(148.6761132,324.72725744)(148.51611654,324.51726336)
\curveto(148.36611351,324.31725785)(148.24611363,324.08225808)(148.15611654,323.81226336)
\curveto(148.12611375,323.71225845)(148.10111378,323.60725856)(148.08111654,323.49726336)
\curveto(148.07111381,323.38725878)(148.05611382,323.27725889)(148.03611654,323.16726336)
\curveto(148.02611385,323.11725905)(148.02111386,323.0672591)(148.02111654,323.01726336)
\lineto(148.02111654,322.86726336)
\curveto(148.00111388,322.79725937)(147.99111389,322.69225947)(147.99111654,322.55226336)
\curveto(148.00111388,322.41225975)(148.01611386,322.30725986)(148.03611654,322.23726336)
\lineto(148.03611654,322.10226336)
\curveto(148.05611382,322.02226014)(148.07111381,321.94226022)(148.08111654,321.86226336)
\curveto(148.09111379,321.79226037)(148.10611377,321.71726045)(148.12611654,321.63726336)
\curveto(148.22611365,321.33726083)(148.33111355,321.09226107)(148.44111654,320.90226336)
\curveto(148.56111332,320.72226144)(148.74611313,320.55726161)(148.99611654,320.40726336)
\curveto(149.06611281,320.35726181)(149.14111274,320.31726185)(149.22111654,320.28726336)
\curveto(149.31111257,320.25726191)(149.40111248,320.23226193)(149.49111654,320.21226336)
\curveto(149.53111235,320.20226196)(149.56611231,320.19726197)(149.59611654,320.19726336)
\curveto(149.62611225,320.20726196)(149.66111222,320.20726196)(149.70111654,320.19726336)
\lineto(149.82111654,320.16726336)
\curveto(149.87111201,320.167262)(149.91611196,320.17226199)(149.95611654,320.18226336)
\lineto(150.07611654,320.18226336)
\curveto(150.15611172,320.20226196)(150.23611164,320.21726195)(150.31611654,320.22726336)
\curveto(150.39611148,320.23726193)(150.47111141,320.25726191)(150.54111654,320.28726336)
\curveto(150.80111108,320.38726178)(151.01111087,320.52226164)(151.17111654,320.69226336)
\curveto(151.33111055,320.8622613)(151.46611041,321.07226109)(151.57611654,321.32226336)
\curveto(151.61611026,321.42226074)(151.64611023,321.52226064)(151.66611654,321.62226336)
\curveto(151.68611019,321.72226044)(151.71111017,321.82726034)(151.74111654,321.93726336)
\curveto(151.75111013,321.97726019)(151.75611012,322.01226015)(151.75611654,322.04226336)
\curveto(151.75611012,322.08226008)(151.76111012,322.12226004)(151.77111654,322.16226336)
\lineto(151.77111654,322.29726336)
\curveto(151.77111011,322.34725982)(151.7761101,322.39725977)(151.78611654,322.44726336)
}
}
{
\newrgbcolor{curcolor}{0 0 0}
\pscustom[linestyle=none,fillstyle=solid,fillcolor=curcolor]
{
\newpath
\moveto(415.20953951,343.25226336)
\curveto(415.20952902,343.22225769)(415.20952902,343.18225773)(415.20953951,343.13226336)
\curveto(415.21952901,343.08225783)(415.22452901,343.02725789)(415.22453951,342.96726336)
\curveto(415.22452901,342.90725801)(415.21952901,342.85225806)(415.20953951,342.80226336)
\curveto(415.20952902,342.75225816)(415.20952902,342.7172582)(415.20953951,342.69726336)
\curveto(415.20952902,342.62725829)(415.20452903,342.55725836)(415.19453951,342.48726336)
\curveto(415.19452904,342.42725849)(415.19452904,342.36725855)(415.19453951,342.30726336)
\curveto(415.17452906,342.25725866)(415.16452907,342.20725871)(415.16453951,342.15726336)
\curveto(415.17452906,342.10725881)(415.17452906,342.05725886)(415.16453951,342.00726336)
\curveto(415.14452909,341.89725902)(415.1295291,341.78725913)(415.11953951,341.67726336)
\curveto(415.10952912,341.56725935)(415.08952914,341.45725946)(415.05953951,341.34726336)
\curveto(415.00952922,341.17725974)(414.96452927,341.0122599)(414.92453951,340.85226336)
\curveto(414.88452935,340.70226021)(414.8345294,340.55226036)(414.77453951,340.40226336)
\curveto(414.60452963,339.98226093)(414.39452984,339.60226131)(414.14453951,339.26226336)
\curveto(413.89453034,338.92226199)(413.59453064,338.63226228)(413.24453951,338.39226336)
\curveto(413.04453119,338.25226266)(412.8345314,338.13226278)(412.61453951,338.03226336)
\curveto(412.40453183,337.93226298)(412.17453206,337.84226307)(411.92453951,337.76226336)
\curveto(411.82453241,337.73226318)(411.71953251,337.70726321)(411.60953951,337.68726336)
\curveto(411.50953272,337.67726324)(411.40453283,337.65726326)(411.29453951,337.62726336)
\curveto(411.24453299,337.6172633)(411.19453304,337.6122633)(411.14453951,337.61226336)
\curveto(411.10453313,337.6122633)(411.05953317,337.60726331)(411.00953951,337.59726336)
\curveto(410.96953326,337.58726333)(410.9295333,337.58226333)(410.88953951,337.58226336)
\curveto(410.84953338,337.59226332)(410.80453343,337.59226332)(410.75453951,337.58226336)
\curveto(410.7345335,337.57226334)(410.70453353,337.56726335)(410.66453951,337.56726336)
\curveto(410.62453361,337.57726334)(410.59453364,337.57726334)(410.57453951,337.56726336)
\curveto(410.49453374,337.54726337)(410.39453384,337.54226337)(410.27453951,337.55226336)
\curveto(410.15453408,337.56226335)(410.04953418,337.56726335)(409.95953951,337.56726336)
\lineto(406.46453951,337.56726336)
\curveto(406.29453794,337.56726335)(406.14953808,337.57226334)(406.02953951,337.58226336)
\curveto(405.91953831,337.60226331)(405.83953839,337.67226324)(405.78953951,337.79226336)
\curveto(405.75953847,337.87226304)(405.74453849,337.99226292)(405.74453951,338.15226336)
\curveto(405.75453848,338.32226259)(405.75953847,338.46226245)(405.75953951,338.57226336)
\lineto(405.75953951,347.37726336)
\curveto(405.75953847,347.49725342)(405.75453848,347.62225329)(405.74453951,347.75226336)
\curveto(405.74453849,347.89225302)(405.76953846,348.00225291)(405.81953951,348.08226336)
\curveto(405.85953837,348.14225277)(405.9345383,348.19225272)(406.04453951,348.23226336)
\curveto(406.06453817,348.24225267)(406.08453815,348.24225267)(406.10453951,348.23226336)
\curveto(406.12453811,348.23225268)(406.14453809,348.23725268)(406.16453951,348.24726336)
\lineto(410.19953951,348.24726336)
\curveto(410.25953397,348.24725267)(410.31953391,348.24725267)(410.37953951,348.24726336)
\curveto(410.44953378,348.25725266)(410.50953372,348.25725266)(410.55953951,348.24726336)
\lineto(410.73953951,348.24726336)
\curveto(410.78953344,348.22725269)(410.84453339,348.2172527)(410.90453951,348.21726336)
\curveto(410.96453327,348.22725269)(411.01953321,348.22225269)(411.06953951,348.20226336)
\curveto(411.1295331,348.18225273)(411.18453305,348.17225274)(411.23453951,348.17226336)
\curveto(411.29453294,348.18225273)(411.35453288,348.17725274)(411.41453951,348.15726336)
\curveto(411.55453268,348.12725279)(411.68953254,348.09725282)(411.81953951,348.06726336)
\curveto(411.94953228,348.04725287)(412.07453216,348.0122529)(412.19453951,347.96226336)
\curveto(412.30453193,347.912253)(412.41453182,347.86725305)(412.52453951,347.82726336)
\curveto(412.6345316,347.78725313)(412.73953149,347.73725318)(412.83953951,347.67726336)
\curveto(413.08953114,347.5172534)(413.31953091,347.36225355)(413.52953951,347.21226336)
\lineto(413.61953951,347.12226336)
\curveto(413.71953051,347.04225387)(413.80953042,346.95225396)(413.88953951,346.85226336)
\lineto(414.02453951,346.73226336)
\curveto(414.07453016,346.65225426)(414.1295301,346.57225434)(414.18953951,346.49226336)
\curveto(414.25952997,346.42225449)(414.31952991,346.34725457)(414.36953951,346.26726336)
\curveto(414.49952973,346.05725486)(414.61452962,345.83225508)(414.71453951,345.59226336)
\curveto(414.81452942,345.36225555)(414.90452933,345.1172558)(414.98453951,344.85726336)
\curveto(415.0345292,344.72725619)(415.06452917,344.59225632)(415.07453951,344.45226336)
\curveto(415.09452914,344.3122566)(415.11952911,344.17225674)(415.14953951,344.03226336)
\curveto(415.14952908,343.98225693)(415.14952908,343.93725698)(415.14953951,343.89726336)
\curveto(415.15952907,343.86725705)(415.16452907,343.83225708)(415.16453951,343.79226336)
\curveto(415.18452905,343.73225718)(415.18952904,343.66725725)(415.17953951,343.59726336)
\curveto(415.17952905,343.52725739)(415.18952904,343.46725745)(415.20953951,343.41726336)
\lineto(415.20953951,343.25226336)
\moveto(412.86953951,342.53226336)
\curveto(412.88953134,342.58225833)(412.89953133,342.66225825)(412.89953951,342.77226336)
\curveto(412.89953133,342.88225803)(412.88953134,342.96225795)(412.86953951,343.01226336)
\lineto(412.86953951,343.29726336)
\curveto(412.84953138,343.38725753)(412.8345314,343.48225743)(412.82453951,343.58226336)
\curveto(412.82453141,343.68225723)(412.81453142,343.77225714)(412.79453951,343.85226336)
\curveto(412.77453146,343.90225701)(412.76453147,343.94725697)(412.76453951,343.98726336)
\curveto(412.77453146,344.03725688)(412.76953146,344.08725683)(412.74953951,344.13726336)
\curveto(412.69953153,344.29725662)(412.64953158,344.44725647)(412.59953951,344.58726336)
\curveto(412.55953167,344.73725618)(412.49953173,344.87725604)(412.41953951,345.00726336)
\curveto(412.26953196,345.24725567)(412.09453214,345.45225546)(411.89453951,345.62226336)
\curveto(411.70453253,345.80225511)(411.46953276,345.95225496)(411.18953951,346.07226336)
\curveto(411.09953313,346.10225481)(411.00953322,346.12725479)(410.91953951,346.14726336)
\curveto(410.8295334,346.17725474)(410.73953349,346.20225471)(410.64953951,346.22226336)
\curveto(410.56953366,346.23225468)(410.49453374,346.23725468)(410.42453951,346.23726336)
\curveto(410.36453387,346.24725467)(410.29453394,346.26225465)(410.21453951,346.28226336)
\curveto(410.17453406,346.29225462)(410.1345341,346.29225462)(410.09453951,346.28226336)
\curveto(410.05453418,346.28225463)(410.01953421,346.28725463)(409.98953951,346.29726336)
\lineto(409.65953951,346.29726336)
\curveto(409.60953462,346.30725461)(409.55453468,346.30725461)(409.49453951,346.29726336)
\lineto(409.31453951,346.29726336)
\lineto(408.63953951,346.29726336)
\curveto(408.61953561,346.27725464)(408.58453565,346.27225464)(408.53453951,346.28226336)
\curveto(408.49453574,346.29225462)(408.45953577,346.29225462)(408.42953951,346.28226336)
\lineto(408.27953951,346.22226336)
\curveto(408.229536,346.2122547)(408.18953604,346.18225473)(408.15953951,346.13226336)
\curveto(408.11953611,346.08225483)(408.09953613,346.0122549)(408.09953951,345.92226336)
\lineto(408.09953951,345.62226336)
\curveto(408.09953613,345.49225542)(408.09453614,345.35725556)(408.08453951,345.21726336)
\lineto(408.08453951,344.79726336)
\lineto(408.08453951,340.61226336)
\curveto(408.08453615,340.55226036)(408.07953615,340.48726043)(408.06953951,340.41726336)
\curveto(408.06953616,340.34726057)(408.07953615,340.28726063)(408.09953951,340.23726336)
\lineto(408.09953951,340.08726336)
\lineto(408.09953951,339.87726336)
\curveto(408.10953612,339.8172611)(408.12453611,339.76226115)(408.14453951,339.71226336)
\curveto(408.20453603,339.59226132)(408.31953591,339.52726139)(408.48953951,339.51726336)
\lineto(409.01453951,339.51726336)
\lineto(410.19953951,339.51726336)
\curveto(410.59953363,339.52726139)(410.93953329,339.58726133)(411.21953951,339.69726336)
\curveto(411.58953264,339.84726107)(411.87953235,340.04726087)(412.08953951,340.29726336)
\curveto(412.30953192,340.54726037)(412.49453174,340.85726006)(412.64453951,341.22726336)
\curveto(412.68453155,341.30725961)(412.71453152,341.39725952)(412.73453951,341.49726336)
\curveto(412.75453148,341.59725932)(412.77953145,341.69725922)(412.80953951,341.79726336)
\lineto(412.80953951,341.91726336)
\curveto(412.8295314,341.98725893)(412.83953139,342.06225885)(412.83953951,342.14226336)
\curveto(412.83953139,342.22225869)(412.84953138,342.30225861)(412.86953951,342.38226336)
\lineto(412.86953951,342.53226336)
}
}
{
\newrgbcolor{curcolor}{0 0 0}
\pscustom[linestyle=none,fillstyle=solid,fillcolor=curcolor]
{
\newpath
\moveto(418.70805513,348.14226336)
\curveto(418.77805218,348.06225285)(418.81305215,347.94225297)(418.81305513,347.78226336)
\lineto(418.81305513,347.31726336)
\lineto(418.81305513,346.91226336)
\curveto(418.81305215,346.77225414)(418.77805218,346.67725424)(418.70805513,346.62726336)
\curveto(418.64805231,346.57725434)(418.56805239,346.54725437)(418.46805513,346.53726336)
\curveto(418.37805258,346.52725439)(418.27805268,346.52225439)(418.16805513,346.52226336)
\lineto(417.32805513,346.52226336)
\curveto(417.21805374,346.52225439)(417.11805384,346.52725439)(417.02805513,346.53726336)
\curveto(416.94805401,346.54725437)(416.87805408,346.57725434)(416.81805513,346.62726336)
\curveto(416.77805418,346.65725426)(416.74805421,346.7122542)(416.72805513,346.79226336)
\curveto(416.71805424,346.88225403)(416.70805425,346.97725394)(416.69805513,347.07726336)
\lineto(416.69805513,347.40726336)
\curveto(416.70805425,347.5172534)(416.71305425,347.6122533)(416.71305513,347.69226336)
\lineto(416.71305513,347.90226336)
\curveto(416.72305424,347.97225294)(416.74305422,348.03225288)(416.77305513,348.08226336)
\curveto(416.79305417,348.12225279)(416.81805414,348.15225276)(416.84805513,348.17226336)
\lineto(416.96805513,348.23226336)
\curveto(416.98805397,348.23225268)(417.01305395,348.23225268)(417.04305513,348.23226336)
\curveto(417.07305389,348.24225267)(417.09805386,348.24725267)(417.11805513,348.24726336)
\lineto(418.21305513,348.24726336)
\curveto(418.31305265,348.24725267)(418.40805255,348.24225267)(418.49805513,348.23226336)
\curveto(418.58805237,348.22225269)(418.6580523,348.19225272)(418.70805513,348.14226336)
\moveto(418.81305513,338.37726336)
\curveto(418.81305215,338.17726274)(418.80805215,338.00726291)(418.79805513,337.86726336)
\curveto(418.78805217,337.72726319)(418.69805226,337.63226328)(418.52805513,337.58226336)
\curveto(418.46805249,337.56226335)(418.40305256,337.55226336)(418.33305513,337.55226336)
\curveto(418.2630527,337.56226335)(418.18805277,337.56726335)(418.10805513,337.56726336)
\lineto(417.26805513,337.56726336)
\curveto(417.17805378,337.56726335)(417.08805387,337.57226334)(416.99805513,337.58226336)
\curveto(416.91805404,337.59226332)(416.8580541,337.62226329)(416.81805513,337.67226336)
\curveto(416.7580542,337.74226317)(416.72305424,337.82726309)(416.71305513,337.92726336)
\lineto(416.71305513,338.27226336)
\lineto(416.71305513,344.60226336)
\lineto(416.71305513,344.90226336)
\curveto(416.71305425,345.00225591)(416.73305423,345.08225583)(416.77305513,345.14226336)
\curveto(416.83305413,345.2122557)(416.91805404,345.25725566)(417.02805513,345.27726336)
\curveto(417.04805391,345.28725563)(417.07305389,345.28725563)(417.10305513,345.27726336)
\curveto(417.14305382,345.27725564)(417.17305379,345.28225563)(417.19305513,345.29226336)
\lineto(417.94305513,345.29226336)
\lineto(418.13805513,345.29226336)
\curveto(418.21805274,345.30225561)(418.28305268,345.30225561)(418.33305513,345.29226336)
\lineto(418.45305513,345.29226336)
\curveto(418.51305245,345.27225564)(418.56805239,345.25725566)(418.61805513,345.24726336)
\curveto(418.66805229,345.23725568)(418.70805225,345.20725571)(418.73805513,345.15726336)
\curveto(418.77805218,345.10725581)(418.79805216,345.03725588)(418.79805513,344.94726336)
\curveto(418.80805215,344.85725606)(418.81305215,344.76225615)(418.81305513,344.66226336)
\lineto(418.81305513,338.37726336)
}
}
{
\newrgbcolor{curcolor}{0 0 0}
\pscustom[linestyle=none,fillstyle=solid,fillcolor=curcolor]
{
\newpath
\moveto(423.44524263,345.50226336)
\curveto(424.19523813,345.52225539)(424.84523748,345.43725548)(425.39524263,345.24726336)
\curveto(425.95523637,345.06725585)(426.38023595,344.75225616)(426.67024263,344.30226336)
\curveto(426.74023559,344.19225672)(426.80023553,344.07725684)(426.85024263,343.95726336)
\curveto(426.91023542,343.84725707)(426.96023537,343.72225719)(427.00024263,343.58226336)
\curveto(427.02023531,343.52225739)(427.0302353,343.45725746)(427.03024263,343.38726336)
\curveto(427.0302353,343.3172576)(427.02023531,343.25725766)(427.00024263,343.20726336)
\curveto(426.96023537,343.14725777)(426.90523542,343.10725781)(426.83524263,343.08726336)
\curveto(426.78523554,343.06725785)(426.7252356,343.05725786)(426.65524263,343.05726336)
\lineto(426.44524263,343.05726336)
\lineto(425.78524263,343.05726336)
\curveto(425.71523661,343.05725786)(425.64523668,343.05225786)(425.57524263,343.04226336)
\curveto(425.50523682,343.04225787)(425.44023689,343.05225786)(425.38024263,343.07226336)
\curveto(425.28023705,343.09225782)(425.20523712,343.13225778)(425.15524263,343.19226336)
\curveto(425.10523722,343.25225766)(425.06023727,343.3122576)(425.02024263,343.37226336)
\lineto(424.90024263,343.58226336)
\curveto(424.87023746,343.66225725)(424.82023751,343.72725719)(424.75024263,343.77726336)
\curveto(424.65023768,343.85725706)(424.55023778,343.917257)(424.45024263,343.95726336)
\curveto(424.36023797,343.99725692)(424.24523808,344.03225688)(424.10524263,344.06226336)
\curveto(424.03523829,344.08225683)(423.9302384,344.09725682)(423.79024263,344.10726336)
\curveto(423.66023867,344.1172568)(423.56023877,344.1122568)(423.49024263,344.09226336)
\lineto(423.38524263,344.09226336)
\lineto(423.23524263,344.06226336)
\curveto(423.19523913,344.06225685)(423.15023918,344.05725686)(423.10024263,344.04726336)
\curveto(422.9302394,343.99725692)(422.79023954,343.92725699)(422.68024263,343.83726336)
\curveto(422.58023975,343.75725716)(422.51023982,343.63225728)(422.47024263,343.46226336)
\curveto(422.45023988,343.39225752)(422.45023988,343.32725759)(422.47024263,343.26726336)
\curveto(422.49023984,343.20725771)(422.51023982,343.15725776)(422.53024263,343.11726336)
\curveto(422.60023973,342.99725792)(422.68023965,342.90225801)(422.77024263,342.83226336)
\curveto(422.87023946,342.76225815)(422.98523934,342.70225821)(423.11524263,342.65226336)
\curveto(423.30523902,342.57225834)(423.51023882,342.50225841)(423.73024263,342.44226336)
\lineto(424.42024263,342.29226336)
\curveto(424.66023767,342.25225866)(424.89023744,342.20225871)(425.11024263,342.14226336)
\curveto(425.34023699,342.09225882)(425.55523677,342.02725889)(425.75524263,341.94726336)
\curveto(425.84523648,341.90725901)(425.9302364,341.87225904)(426.01024263,341.84226336)
\curveto(426.10023623,341.82225909)(426.18523614,341.78725913)(426.26524263,341.73726336)
\curveto(426.45523587,341.6172593)(426.6252357,341.48725943)(426.77524263,341.34726336)
\curveto(426.93523539,341.20725971)(427.06023527,341.03225988)(427.15024263,340.82226336)
\curveto(427.18023515,340.75226016)(427.20523512,340.68226023)(427.22524263,340.61226336)
\curveto(427.24523508,340.54226037)(427.26523506,340.46726045)(427.28524263,340.38726336)
\curveto(427.29523503,340.32726059)(427.30023503,340.23226068)(427.30024263,340.10226336)
\curveto(427.31023502,339.98226093)(427.31023502,339.88726103)(427.30024263,339.81726336)
\lineto(427.30024263,339.74226336)
\curveto(427.28023505,339.68226123)(427.26523506,339.62226129)(427.25524263,339.56226336)
\curveto(427.25523507,339.5122614)(427.25023508,339.46226145)(427.24024263,339.41226336)
\curveto(427.17023516,339.1122618)(427.06023527,338.84726207)(426.91024263,338.61726336)
\curveto(426.75023558,338.37726254)(426.55523577,338.18226273)(426.32524263,338.03226336)
\curveto(426.09523623,337.88226303)(425.83523649,337.75226316)(425.54524263,337.64226336)
\curveto(425.43523689,337.59226332)(425.31523701,337.55726336)(425.18524263,337.53726336)
\curveto(425.06523726,337.5172634)(424.94523738,337.49226342)(424.82524263,337.46226336)
\curveto(424.73523759,337.44226347)(424.64023769,337.43226348)(424.54024263,337.43226336)
\curveto(424.45023788,337.42226349)(424.36023797,337.40726351)(424.27024263,337.38726336)
\lineto(424.00024263,337.38726336)
\curveto(423.94023839,337.36726355)(423.83523849,337.35726356)(423.68524263,337.35726336)
\curveto(423.54523878,337.35726356)(423.44523888,337.36726355)(423.38524263,337.38726336)
\curveto(423.35523897,337.38726353)(423.32023901,337.39226352)(423.28024263,337.40226336)
\lineto(423.17524263,337.40226336)
\curveto(423.05523927,337.42226349)(422.93523939,337.43726348)(422.81524263,337.44726336)
\curveto(422.69523963,337.45726346)(422.58023975,337.47726344)(422.47024263,337.50726336)
\curveto(422.08024025,337.6172633)(421.73524059,337.74226317)(421.43524263,337.88226336)
\curveto(421.13524119,338.03226288)(420.88024145,338.25226266)(420.67024263,338.54226336)
\curveto(420.5302418,338.73226218)(420.41024192,338.95226196)(420.31024263,339.20226336)
\curveto(420.29024204,339.26226165)(420.27024206,339.34226157)(420.25024263,339.44226336)
\curveto(420.2302421,339.49226142)(420.21524211,339.56226135)(420.20524263,339.65226336)
\curveto(420.19524213,339.74226117)(420.20024213,339.8172611)(420.22024263,339.87726336)
\curveto(420.25024208,339.94726097)(420.30024203,339.99726092)(420.37024263,340.02726336)
\curveto(420.42024191,340.04726087)(420.48024185,340.05726086)(420.55024263,340.05726336)
\lineto(420.77524263,340.05726336)
\lineto(421.48024263,340.05726336)
\lineto(421.72024263,340.05726336)
\curveto(421.80024053,340.05726086)(421.87024046,340.04726087)(421.93024263,340.02726336)
\curveto(422.04024029,339.98726093)(422.11024022,339.92226099)(422.14024263,339.83226336)
\curveto(422.18024015,339.74226117)(422.2252401,339.64726127)(422.27524263,339.54726336)
\curveto(422.29524003,339.49726142)(422.33024,339.43226148)(422.38024263,339.35226336)
\curveto(422.44023989,339.27226164)(422.49023984,339.22226169)(422.53024263,339.20226336)
\curveto(422.65023968,339.10226181)(422.76523956,339.02226189)(422.87524263,338.96226336)
\curveto(422.98523934,338.912262)(423.1252392,338.86226205)(423.29524263,338.81226336)
\curveto(423.34523898,338.79226212)(423.39523893,338.78226213)(423.44524263,338.78226336)
\curveto(423.49523883,338.79226212)(423.54523878,338.79226212)(423.59524263,338.78226336)
\curveto(423.67523865,338.76226215)(423.76023857,338.75226216)(423.85024263,338.75226336)
\curveto(423.95023838,338.76226215)(424.03523829,338.77726214)(424.10524263,338.79726336)
\curveto(424.15523817,338.80726211)(424.20023813,338.8122621)(424.24024263,338.81226336)
\curveto(424.29023804,338.8122621)(424.34023799,338.82226209)(424.39024263,338.84226336)
\curveto(424.5302378,338.89226202)(424.65523767,338.95226196)(424.76524263,339.02226336)
\curveto(424.88523744,339.09226182)(424.98023735,339.18226173)(425.05024263,339.29226336)
\curveto(425.10023723,339.37226154)(425.14023719,339.49726142)(425.17024263,339.66726336)
\curveto(425.19023714,339.73726118)(425.19023714,339.80226111)(425.17024263,339.86226336)
\curveto(425.15023718,339.92226099)(425.1302372,339.97226094)(425.11024263,340.01226336)
\curveto(425.04023729,340.15226076)(424.95023738,340.25726066)(424.84024263,340.32726336)
\curveto(424.74023759,340.39726052)(424.62023771,340.46226045)(424.48024263,340.52226336)
\curveto(424.29023804,340.60226031)(424.09023824,340.66726025)(423.88024263,340.71726336)
\curveto(423.67023866,340.76726015)(423.46023887,340.82226009)(423.25024263,340.88226336)
\curveto(423.17023916,340.90226001)(423.08523924,340.91726)(422.99524263,340.92726336)
\curveto(422.91523941,340.93725998)(422.83523949,340.95225996)(422.75524263,340.97226336)
\curveto(422.43523989,341.06225985)(422.1302402,341.14725977)(421.84024263,341.22726336)
\curveto(421.55024078,341.3172596)(421.28524104,341.44725947)(421.04524263,341.61726336)
\curveto(420.76524156,341.8172591)(420.56024177,342.08725883)(420.43024263,342.42726336)
\curveto(420.41024192,342.49725842)(420.39024194,342.59225832)(420.37024263,342.71226336)
\curveto(420.35024198,342.78225813)(420.33524199,342.86725805)(420.32524263,342.96726336)
\curveto(420.31524201,343.06725785)(420.32024201,343.15725776)(420.34024263,343.23726336)
\curveto(420.36024197,343.28725763)(420.36524196,343.32725759)(420.35524263,343.35726336)
\curveto(420.34524198,343.39725752)(420.35024198,343.44225747)(420.37024263,343.49226336)
\curveto(420.39024194,343.60225731)(420.41024192,343.70225721)(420.43024263,343.79226336)
\curveto(420.46024187,343.89225702)(420.49524183,343.98725693)(420.53524263,344.07726336)
\curveto(420.66524166,344.36725655)(420.84524148,344.60225631)(421.07524263,344.78226336)
\curveto(421.30524102,344.96225595)(421.56524076,345.10725581)(421.85524263,345.21726336)
\curveto(421.96524036,345.26725565)(422.08024025,345.30225561)(422.20024263,345.32226336)
\curveto(422.32024001,345.35225556)(422.44523988,345.38225553)(422.57524263,345.41226336)
\curveto(422.63523969,345.43225548)(422.69523963,345.44225547)(422.75524263,345.44226336)
\lineto(422.93524263,345.47226336)
\curveto(423.01523931,345.48225543)(423.10023923,345.48725543)(423.19024263,345.48726336)
\curveto(423.28023905,345.48725543)(423.36523896,345.49225542)(423.44524263,345.50226336)
}
}
{
\newrgbcolor{curcolor}{0 0 0}
\pscustom[linestyle=none,fillstyle=solid,fillcolor=curcolor]
{
\newpath
\moveto(429.58188326,347.60226336)
\lineto(430.58688326,347.60226336)
\curveto(430.73688027,347.60225331)(430.86688014,347.59225332)(430.97688326,347.57226336)
\curveto(431.09687991,347.56225335)(431.18187983,347.50225341)(431.23188326,347.39226336)
\curveto(431.25187976,347.34225357)(431.26187975,347.28225363)(431.26188326,347.21226336)
\lineto(431.26188326,347.00226336)
\lineto(431.26188326,346.32726336)
\curveto(431.26187975,346.27725464)(431.25687975,346.2172547)(431.24688326,346.14726336)
\curveto(431.24687976,346.08725483)(431.25187976,346.03225488)(431.26188326,345.98226336)
\lineto(431.26188326,345.81726336)
\curveto(431.26187975,345.73725518)(431.26687974,345.66225525)(431.27688326,345.59226336)
\curveto(431.28687972,345.53225538)(431.3118797,345.47725544)(431.35188326,345.42726336)
\curveto(431.42187959,345.33725558)(431.54687946,345.28725563)(431.72688326,345.27726336)
\lineto(432.26688326,345.27726336)
\lineto(432.44688326,345.27726336)
\curveto(432.5068785,345.27725564)(432.56187845,345.26725565)(432.61188326,345.24726336)
\curveto(432.72187829,345.19725572)(432.78187823,345.10725581)(432.79188326,344.97726336)
\curveto(432.8118782,344.84725607)(432.82187819,344.70225621)(432.82188326,344.54226336)
\lineto(432.82188326,344.33226336)
\curveto(432.83187818,344.26225665)(432.82687818,344.20225671)(432.80688326,344.15226336)
\curveto(432.75687825,343.99225692)(432.65187836,343.90725701)(432.49188326,343.89726336)
\curveto(432.33187868,343.88725703)(432.15187886,343.88225703)(431.95188326,343.88226336)
\lineto(431.81688326,343.88226336)
\curveto(431.77687923,343.89225702)(431.74187927,343.89225702)(431.71188326,343.88226336)
\curveto(431.67187934,343.87225704)(431.63687937,343.86725705)(431.60688326,343.86726336)
\curveto(431.57687943,343.87725704)(431.54687946,343.87225704)(431.51688326,343.85226336)
\curveto(431.43687957,343.83225708)(431.37687963,343.78725713)(431.33688326,343.71726336)
\curveto(431.3068797,343.65725726)(431.28187973,343.58225733)(431.26188326,343.49226336)
\curveto(431.25187976,343.44225747)(431.25187976,343.38725753)(431.26188326,343.32726336)
\curveto(431.27187974,343.26725765)(431.27187974,343.2122577)(431.26188326,343.16226336)
\lineto(431.26188326,342.23226336)
\lineto(431.26188326,340.47726336)
\curveto(431.26187975,340.22726069)(431.26687974,340.00726091)(431.27688326,339.81726336)
\curveto(431.29687971,339.63726128)(431.36187965,339.47726144)(431.47188326,339.33726336)
\curveto(431.52187949,339.27726164)(431.58687942,339.23226168)(431.66688326,339.20226336)
\lineto(431.93688326,339.14226336)
\curveto(431.96687904,339.13226178)(431.99687901,339.12726179)(432.02688326,339.12726336)
\curveto(432.06687894,339.13726178)(432.09687891,339.13726178)(432.11688326,339.12726336)
\lineto(432.28188326,339.12726336)
\curveto(432.39187862,339.12726179)(432.48687852,339.12226179)(432.56688326,339.11226336)
\curveto(432.64687836,339.10226181)(432.7118783,339.06226185)(432.76188326,338.99226336)
\curveto(432.80187821,338.93226198)(432.82187819,338.85226206)(432.82188326,338.75226336)
\lineto(432.82188326,338.46726336)
\curveto(432.82187819,338.25726266)(432.81687819,338.06226285)(432.80688326,337.88226336)
\curveto(432.8068782,337.7122632)(432.72687828,337.59726332)(432.56688326,337.53726336)
\curveto(432.51687849,337.5172634)(432.47187854,337.5122634)(432.43188326,337.52226336)
\curveto(432.39187862,337.52226339)(432.34687866,337.5122634)(432.29688326,337.49226336)
\lineto(432.14688326,337.49226336)
\curveto(432.12687888,337.49226342)(432.09687891,337.49726342)(432.05688326,337.50726336)
\curveto(432.01687899,337.50726341)(431.98187903,337.50226341)(431.95188326,337.49226336)
\curveto(431.90187911,337.48226343)(431.84687916,337.48226343)(431.78688326,337.49226336)
\lineto(431.63688326,337.49226336)
\lineto(431.48688326,337.49226336)
\curveto(431.43687957,337.48226343)(431.39187962,337.48226343)(431.35188326,337.49226336)
\lineto(431.18688326,337.49226336)
\curveto(431.13687987,337.50226341)(431.08187993,337.50726341)(431.02188326,337.50726336)
\curveto(430.96188005,337.50726341)(430.9068801,337.5122634)(430.85688326,337.52226336)
\curveto(430.78688022,337.53226338)(430.72188029,337.54226337)(430.66188326,337.55226336)
\lineto(430.48188326,337.58226336)
\curveto(430.37188064,337.6122633)(430.26688074,337.64726327)(430.16688326,337.68726336)
\curveto(430.06688094,337.72726319)(429.97188104,337.77226314)(429.88188326,337.82226336)
\lineto(429.79188326,337.88226336)
\curveto(429.76188125,337.912263)(429.72688128,337.94226297)(429.68688326,337.97226336)
\curveto(429.66688134,337.99226292)(429.64188137,338.0122629)(429.61188326,338.03226336)
\lineto(429.53688326,338.10726336)
\curveto(429.39688161,338.29726262)(429.29188172,338.50726241)(429.22188326,338.73726336)
\curveto(429.20188181,338.77726214)(429.19188182,338.8122621)(429.19188326,338.84226336)
\curveto(429.20188181,338.88226203)(429.20188181,338.92726199)(429.19188326,338.97726336)
\curveto(429.18188183,338.99726192)(429.17688183,339.02226189)(429.17688326,339.05226336)
\curveto(429.17688183,339.08226183)(429.17188184,339.10726181)(429.16188326,339.12726336)
\lineto(429.16188326,339.27726336)
\curveto(429.15188186,339.3172616)(429.14688186,339.36226155)(429.14688326,339.41226336)
\curveto(429.15688185,339.46226145)(429.16188185,339.5122614)(429.16188326,339.56226336)
\lineto(429.16188326,340.13226336)
\lineto(429.16188326,342.36726336)
\lineto(429.16188326,343.16226336)
\lineto(429.16188326,343.37226336)
\curveto(429.17188184,343.44225747)(429.16688184,343.50725741)(429.14688326,343.56726336)
\curveto(429.1068819,343.70725721)(429.03688197,343.79725712)(428.93688326,343.83726336)
\curveto(428.82688218,343.88725703)(428.68688232,343.90225701)(428.51688326,343.88226336)
\curveto(428.34688266,343.86225705)(428.20188281,343.87725704)(428.08188326,343.92726336)
\curveto(428.00188301,343.95725696)(427.95188306,344.00225691)(427.93188326,344.06226336)
\curveto(427.9118831,344.12225679)(427.89188312,344.19725672)(427.87188326,344.28726336)
\lineto(427.87188326,344.60226336)
\curveto(427.87188314,344.78225613)(427.88188313,344.92725599)(427.90188326,345.03726336)
\curveto(427.92188309,345.14725577)(428.006883,345.22225569)(428.15688326,345.26226336)
\curveto(428.19688281,345.28225563)(428.23688277,345.28725563)(428.27688326,345.27726336)
\lineto(428.41188326,345.27726336)
\curveto(428.56188245,345.27725564)(428.70188231,345.28225563)(428.83188326,345.29226336)
\curveto(428.96188205,345.3122556)(429.05188196,345.37225554)(429.10188326,345.47226336)
\curveto(429.13188188,345.54225537)(429.14688186,345.62225529)(429.14688326,345.71226336)
\curveto(429.15688185,345.80225511)(429.16188185,345.89225502)(429.16188326,345.98226336)
\lineto(429.16188326,346.91226336)
\lineto(429.16188326,347.16726336)
\curveto(429.16188185,347.25725366)(429.17188184,347.33225358)(429.19188326,347.39226336)
\curveto(429.24188177,347.49225342)(429.31688169,347.55725336)(429.41688326,347.58726336)
\curveto(429.43688157,347.59725332)(429.46188155,347.59725332)(429.49188326,347.58726336)
\curveto(429.53188148,347.58725333)(429.56188145,347.59225332)(429.58188326,347.60226336)
}
}
{
\newrgbcolor{curcolor}{0 0 0}
\pscustom[linestyle=none,fillstyle=solid,fillcolor=curcolor]
{
\newpath
\moveto(438.23032076,345.48726336)
\curveto(438.34031544,345.48725543)(438.43531535,345.47725544)(438.51532076,345.45726336)
\curveto(438.60531518,345.43725548)(438.67531511,345.39225552)(438.72532076,345.32226336)
\curveto(438.785315,345.24225567)(438.81531497,345.10225581)(438.81532076,344.90226336)
\lineto(438.81532076,344.39226336)
\lineto(438.81532076,344.01726336)
\curveto(438.82531496,343.87725704)(438.81031497,343.76725715)(438.77032076,343.68726336)
\curveto(438.73031505,343.6172573)(438.67031511,343.57225734)(438.59032076,343.55226336)
\curveto(438.52031526,343.53225738)(438.43531535,343.52225739)(438.33532076,343.52226336)
\curveto(438.24531554,343.52225739)(438.14531564,343.52725739)(438.03532076,343.53726336)
\curveto(437.93531585,343.54725737)(437.84031594,343.54225737)(437.75032076,343.52226336)
\curveto(437.6803161,343.50225741)(437.61031617,343.48725743)(437.54032076,343.47726336)
\curveto(437.47031631,343.47725744)(437.40531638,343.46725745)(437.34532076,343.44726336)
\curveto(437.1853166,343.39725752)(437.02531676,343.32225759)(436.86532076,343.22226336)
\curveto(436.70531708,343.13225778)(436.5803172,343.02725789)(436.49032076,342.90726336)
\curveto(436.44031734,342.82725809)(436.3853174,342.74225817)(436.32532076,342.65226336)
\curveto(436.27531751,342.57225834)(436.22531756,342.48725843)(436.17532076,342.39726336)
\curveto(436.14531764,342.3172586)(436.11531767,342.23225868)(436.08532076,342.14226336)
\lineto(436.02532076,341.90226336)
\curveto(436.00531778,341.83225908)(435.99531779,341.75725916)(435.99532076,341.67726336)
\curveto(435.99531779,341.60725931)(435.9853178,341.53725938)(435.96532076,341.46726336)
\curveto(435.95531783,341.42725949)(435.95031783,341.38725953)(435.95032076,341.34726336)
\curveto(435.96031782,341.3172596)(435.96031782,341.28725963)(435.95032076,341.25726336)
\lineto(435.95032076,341.01726336)
\curveto(435.93031785,340.94725997)(435.92531786,340.86726005)(435.93532076,340.77726336)
\curveto(435.94531784,340.69726022)(435.95031783,340.6172603)(435.95032076,340.53726336)
\lineto(435.95032076,339.57726336)
\lineto(435.95032076,338.30226336)
\curveto(435.95031783,338.17226274)(435.94531784,338.05226286)(435.93532076,337.94226336)
\curveto(435.92531786,337.83226308)(435.89531789,337.74226317)(435.84532076,337.67226336)
\curveto(435.82531796,337.64226327)(435.79031799,337.6172633)(435.74032076,337.59726336)
\curveto(435.70031808,337.58726333)(435.65531813,337.57726334)(435.60532076,337.56726336)
\lineto(435.53032076,337.56726336)
\curveto(435.4803183,337.55726336)(435.42531836,337.55226336)(435.36532076,337.55226336)
\lineto(435.20032076,337.55226336)
\lineto(434.55532076,337.55226336)
\curveto(434.49531929,337.56226335)(434.43031935,337.56726335)(434.36032076,337.56726336)
\lineto(434.16532076,337.56726336)
\curveto(434.11531967,337.58726333)(434.06531972,337.60226331)(434.01532076,337.61226336)
\curveto(433.96531982,337.63226328)(433.93031985,337.66726325)(433.91032076,337.71726336)
\curveto(433.87031991,337.76726315)(433.84531994,337.83726308)(433.83532076,337.92726336)
\lineto(433.83532076,338.22726336)
\lineto(433.83532076,339.24726336)
\lineto(433.83532076,343.47726336)
\lineto(433.83532076,344.58726336)
\lineto(433.83532076,344.87226336)
\curveto(433.83531995,344.97225594)(433.85531993,345.05225586)(433.89532076,345.11226336)
\curveto(433.94531984,345.19225572)(434.02031976,345.24225567)(434.12032076,345.26226336)
\curveto(434.22031956,345.28225563)(434.34031944,345.29225562)(434.48032076,345.29226336)
\lineto(435.24532076,345.29226336)
\curveto(435.36531842,345.29225562)(435.47031831,345.28225563)(435.56032076,345.26226336)
\curveto(435.65031813,345.25225566)(435.72031806,345.20725571)(435.77032076,345.12726336)
\curveto(435.80031798,345.07725584)(435.81531797,345.00725591)(435.81532076,344.91726336)
\lineto(435.84532076,344.64726336)
\curveto(435.85531793,344.56725635)(435.87031791,344.49225642)(435.89032076,344.42226336)
\curveto(435.92031786,344.35225656)(435.97031781,344.3172566)(436.04032076,344.31726336)
\curveto(436.06031772,344.33725658)(436.0803177,344.34725657)(436.10032076,344.34726336)
\curveto(436.12031766,344.34725657)(436.14031764,344.35725656)(436.16032076,344.37726336)
\curveto(436.22031756,344.42725649)(436.27031751,344.48225643)(436.31032076,344.54226336)
\curveto(436.36031742,344.6122563)(436.42031736,344.67225624)(436.49032076,344.72226336)
\curveto(436.53031725,344.75225616)(436.56531722,344.78225613)(436.59532076,344.81226336)
\curveto(436.62531716,344.85225606)(436.66031712,344.88725603)(436.70032076,344.91726336)
\lineto(436.97032076,345.09726336)
\curveto(437.07031671,345.15725576)(437.17031661,345.2122557)(437.27032076,345.26226336)
\curveto(437.37031641,345.30225561)(437.47031631,345.33725558)(437.57032076,345.36726336)
\lineto(437.90032076,345.45726336)
\curveto(437.93031585,345.46725545)(437.9853158,345.46725545)(438.06532076,345.45726336)
\curveto(438.15531563,345.45725546)(438.21031557,345.46725545)(438.23032076,345.48726336)
}
}
{
\newrgbcolor{curcolor}{0 0 0}
\pscustom[linestyle=none,fillstyle=solid,fillcolor=curcolor]
{
\newpath
\moveto(441.73539888,348.14226336)
\curveto(441.80539593,348.06225285)(441.8403959,347.94225297)(441.84039888,347.78226336)
\lineto(441.84039888,347.31726336)
\lineto(441.84039888,346.91226336)
\curveto(441.8403959,346.77225414)(441.80539593,346.67725424)(441.73539888,346.62726336)
\curveto(441.67539606,346.57725434)(441.59539614,346.54725437)(441.49539888,346.53726336)
\curveto(441.40539633,346.52725439)(441.30539643,346.52225439)(441.19539888,346.52226336)
\lineto(440.35539888,346.52226336)
\curveto(440.24539749,346.52225439)(440.14539759,346.52725439)(440.05539888,346.53726336)
\curveto(439.97539776,346.54725437)(439.90539783,346.57725434)(439.84539888,346.62726336)
\curveto(439.80539793,346.65725426)(439.77539796,346.7122542)(439.75539888,346.79226336)
\curveto(439.74539799,346.88225403)(439.735398,346.97725394)(439.72539888,347.07726336)
\lineto(439.72539888,347.40726336)
\curveto(439.735398,347.5172534)(439.740398,347.6122533)(439.74039888,347.69226336)
\lineto(439.74039888,347.90226336)
\curveto(439.75039799,347.97225294)(439.77039797,348.03225288)(439.80039888,348.08226336)
\curveto(439.82039792,348.12225279)(439.84539789,348.15225276)(439.87539888,348.17226336)
\lineto(439.99539888,348.23226336)
\curveto(440.01539772,348.23225268)(440.0403977,348.23225268)(440.07039888,348.23226336)
\curveto(440.10039764,348.24225267)(440.12539761,348.24725267)(440.14539888,348.24726336)
\lineto(441.24039888,348.24726336)
\curveto(441.3403964,348.24725267)(441.4353963,348.24225267)(441.52539888,348.23226336)
\curveto(441.61539612,348.22225269)(441.68539605,348.19225272)(441.73539888,348.14226336)
\moveto(441.84039888,338.37726336)
\curveto(441.8403959,338.17726274)(441.8353959,338.00726291)(441.82539888,337.86726336)
\curveto(441.81539592,337.72726319)(441.72539601,337.63226328)(441.55539888,337.58226336)
\curveto(441.49539624,337.56226335)(441.43039631,337.55226336)(441.36039888,337.55226336)
\curveto(441.29039645,337.56226335)(441.21539652,337.56726335)(441.13539888,337.56726336)
\lineto(440.29539888,337.56726336)
\curveto(440.20539753,337.56726335)(440.11539762,337.57226334)(440.02539888,337.58226336)
\curveto(439.94539779,337.59226332)(439.88539785,337.62226329)(439.84539888,337.67226336)
\curveto(439.78539795,337.74226317)(439.75039799,337.82726309)(439.74039888,337.92726336)
\lineto(439.74039888,338.27226336)
\lineto(439.74039888,344.60226336)
\lineto(439.74039888,344.90226336)
\curveto(439.740398,345.00225591)(439.76039798,345.08225583)(439.80039888,345.14226336)
\curveto(439.86039788,345.2122557)(439.94539779,345.25725566)(440.05539888,345.27726336)
\curveto(440.07539766,345.28725563)(440.10039764,345.28725563)(440.13039888,345.27726336)
\curveto(440.17039757,345.27725564)(440.20039754,345.28225563)(440.22039888,345.29226336)
\lineto(440.97039888,345.29226336)
\lineto(441.16539888,345.29226336)
\curveto(441.24539649,345.30225561)(441.31039643,345.30225561)(441.36039888,345.29226336)
\lineto(441.48039888,345.29226336)
\curveto(441.5403962,345.27225564)(441.59539614,345.25725566)(441.64539888,345.24726336)
\curveto(441.69539604,345.23725568)(441.735396,345.20725571)(441.76539888,345.15726336)
\curveto(441.80539593,345.10725581)(441.82539591,345.03725588)(441.82539888,344.94726336)
\curveto(441.8353959,344.85725606)(441.8403959,344.76225615)(441.84039888,344.66226336)
\lineto(441.84039888,338.37726336)
}
}
{
\newrgbcolor{curcolor}{0 0 0}
\pscustom[linestyle=none,fillstyle=solid,fillcolor=curcolor]
{
\newpath
\moveto(451.30258638,341.81226336)
\curveto(451.32257778,341.75225916)(451.33257777,341.64725927)(451.33258638,341.49726336)
\curveto(451.33257777,341.35725956)(451.32757778,341.25725966)(451.31758638,341.19726336)
\curveto(451.31757779,341.14725977)(451.31257779,341.10225981)(451.30258638,341.06226336)
\lineto(451.30258638,340.94226336)
\curveto(451.28257782,340.86226005)(451.27257783,340.78226013)(451.27258638,340.70226336)
\curveto(451.27257783,340.63226028)(451.26257784,340.55726036)(451.24258638,340.47726336)
\curveto(451.24257786,340.43726048)(451.23257787,340.36726055)(451.21258638,340.26726336)
\curveto(451.18257792,340.14726077)(451.15257795,340.02226089)(451.12258638,339.89226336)
\curveto(451.102578,339.77226114)(451.06757804,339.65726126)(451.01758638,339.54726336)
\curveto(450.83757827,339.09726182)(450.61257849,338.70726221)(450.34258638,338.37726336)
\curveto(450.07257903,338.04726287)(449.71757939,337.78726313)(449.27758638,337.59726336)
\curveto(449.18757992,337.55726336)(449.09258001,337.52726339)(448.99258638,337.50726336)
\curveto(448.9025802,337.47726344)(448.8025803,337.44726347)(448.69258638,337.41726336)
\curveto(448.63258047,337.39726352)(448.56758054,337.38726353)(448.49758638,337.38726336)
\curveto(448.43758067,337.38726353)(448.37758073,337.38226353)(448.31758638,337.37226336)
\lineto(448.18258638,337.37226336)
\curveto(448.12258098,337.35226356)(448.04258106,337.34726357)(447.94258638,337.35726336)
\curveto(447.84258126,337.35726356)(447.76258134,337.36726355)(447.70258638,337.38726336)
\lineto(447.61258638,337.38726336)
\curveto(447.56258154,337.39726352)(447.5075816,337.40726351)(447.44758638,337.41726336)
\curveto(447.38758172,337.4172635)(447.32758178,337.42226349)(447.26758638,337.43226336)
\curveto(447.07758203,337.48226343)(446.9025822,337.53226338)(446.74258638,337.58226336)
\curveto(446.58258252,337.63226328)(446.43258267,337.70226321)(446.29258638,337.79226336)
\lineto(446.11258638,337.91226336)
\curveto(446.06258304,337.95226296)(446.01258309,337.99726292)(445.96258638,338.04726336)
\lineto(445.87258638,338.10726336)
\curveto(445.84258326,338.12726279)(445.81258329,338.14226277)(445.78258638,338.15226336)
\curveto(445.69258341,338.18226273)(445.63758347,338.16226275)(445.61758638,338.09226336)
\curveto(445.56758354,338.02226289)(445.53258357,337.93726298)(445.51258638,337.83726336)
\curveto(445.5025836,337.74726317)(445.46758364,337.67726324)(445.40758638,337.62726336)
\curveto(445.34758376,337.58726333)(445.27758383,337.56226335)(445.19758638,337.55226336)
\lineto(444.92758638,337.55226336)
\lineto(444.20758638,337.55226336)
\lineto(443.98258638,337.55226336)
\curveto(443.91258519,337.54226337)(443.84758526,337.54726337)(443.78758638,337.56726336)
\curveto(443.64758546,337.6172633)(443.56758554,337.70726321)(443.54758638,337.83726336)
\curveto(443.53758557,337.97726294)(443.53258557,338.13226278)(443.53258638,338.30226336)
\lineto(443.53258638,347.45226336)
\lineto(443.53258638,347.79726336)
\curveto(443.53258557,347.917253)(443.55758555,348.0122529)(443.60758638,348.08226336)
\curveto(443.64758546,348.15225276)(443.71758539,348.19725272)(443.81758638,348.21726336)
\curveto(443.83758527,348.22725269)(443.85758525,348.22725269)(443.87758638,348.21726336)
\curveto(443.9075852,348.2172527)(443.93258517,348.22225269)(443.95258638,348.23226336)
\lineto(444.89758638,348.23226336)
\curveto(445.07758403,348.23225268)(445.23258387,348.22225269)(445.36258638,348.20226336)
\curveto(445.49258361,348.19225272)(445.57758353,348.1172528)(445.61758638,347.97726336)
\curveto(445.64758346,347.87725304)(445.65758345,347.74225317)(445.64758638,347.57226336)
\curveto(445.63758347,347.4122535)(445.63258347,347.27225364)(445.63258638,347.15226336)
\lineto(445.63258638,345.51726336)
\lineto(445.63258638,345.18726336)
\curveto(445.63258347,345.07725584)(445.64258346,344.98225593)(445.66258638,344.90226336)
\curveto(445.67258343,344.85225606)(445.68258342,344.80725611)(445.69258638,344.76726336)
\curveto(445.7025834,344.73725618)(445.72758338,344.7172562)(445.76758638,344.70726336)
\curveto(445.78758332,344.68725623)(445.81258329,344.67725624)(445.84258638,344.67726336)
\curveto(445.88258322,344.67725624)(445.91258319,344.68225623)(445.93258638,344.69226336)
\curveto(446.0025831,344.73225618)(446.06758304,344.77225614)(446.12758638,344.81226336)
\curveto(446.18758292,344.86225605)(446.25258285,344.912256)(446.32258638,344.96226336)
\curveto(446.45258265,345.05225586)(446.58758252,345.12725579)(446.72758638,345.18726336)
\curveto(446.86758224,345.25725566)(447.02258208,345.3172556)(447.19258638,345.36726336)
\curveto(447.27258183,345.39725552)(447.35258175,345.4122555)(447.43258638,345.41226336)
\curveto(447.51258159,345.42225549)(447.59258151,345.43725548)(447.67258638,345.45726336)
\curveto(447.74258136,345.47725544)(447.81758129,345.48725543)(447.89758638,345.48726336)
\lineto(448.13758638,345.48726336)
\lineto(448.28758638,345.48726336)
\curveto(448.31758079,345.47725544)(448.35258075,345.47225544)(448.39258638,345.47226336)
\curveto(448.43258067,345.48225543)(448.47258063,345.48225543)(448.51258638,345.47226336)
\curveto(448.62258048,345.44225547)(448.72258038,345.4172555)(448.81258638,345.39726336)
\curveto(448.91258019,345.38725553)(449.0075801,345.36225555)(449.09758638,345.32226336)
\curveto(449.55757955,345.13225578)(449.93257917,344.88725603)(450.22258638,344.58726336)
\curveto(450.51257859,344.28725663)(450.75757835,343.912257)(450.95758638,343.46226336)
\curveto(451.0075781,343.34225757)(451.04757806,343.2172577)(451.07758638,343.08726336)
\curveto(451.11757799,342.95725796)(451.15757795,342.82225809)(451.19758638,342.68226336)
\curveto(451.21757789,342.6122583)(451.22757788,342.54225837)(451.22758638,342.47226336)
\curveto(451.23757787,342.4122585)(451.25257785,342.34225857)(451.27258638,342.26226336)
\curveto(451.29257781,342.2122587)(451.29757781,342.15725876)(451.28758638,342.09726336)
\curveto(451.28757782,342.03725888)(451.29257781,341.97725894)(451.30258638,341.91726336)
\lineto(451.30258638,341.81226336)
\moveto(449.08258638,340.40226336)
\curveto(449.11257999,340.50226041)(449.13757997,340.62726029)(449.15758638,340.77726336)
\curveto(449.18757992,340.92725999)(449.2025799,341.07725984)(449.20258638,341.22726336)
\curveto(449.21257989,341.38725953)(449.21257989,341.54225937)(449.20258638,341.69226336)
\curveto(449.2025799,341.85225906)(449.18757992,341.98725893)(449.15758638,342.09726336)
\curveto(449.12757998,342.19725872)(449.10758,342.29225862)(449.09758638,342.38226336)
\curveto(449.08758002,342.47225844)(449.06258004,342.55725836)(449.02258638,342.63726336)
\curveto(448.88258022,342.98725793)(448.68258042,343.28225763)(448.42258638,343.52226336)
\curveto(448.17258093,343.77225714)(447.8025813,343.89725702)(447.31258638,343.89726336)
\curveto(447.27258183,343.89725702)(447.23758187,343.89225702)(447.20758638,343.88226336)
\lineto(447.10258638,343.88226336)
\curveto(447.03258207,343.86225705)(446.96758214,343.84225707)(446.90758638,343.82226336)
\curveto(446.84758226,343.8122571)(446.78758232,343.79725712)(446.72758638,343.77726336)
\curveto(446.43758267,343.64725727)(446.21758289,343.46225745)(446.06758638,343.22226336)
\curveto(445.91758319,342.99225792)(445.79258331,342.72725819)(445.69258638,342.42726336)
\curveto(445.66258344,342.34725857)(445.64258346,342.26225865)(445.63258638,342.17226336)
\curveto(445.63258347,342.09225882)(445.62258348,342.0122589)(445.60258638,341.93226336)
\curveto(445.59258351,341.90225901)(445.58758352,341.85225906)(445.58758638,341.78226336)
\curveto(445.57758353,341.74225917)(445.57258353,341.70225921)(445.57258638,341.66226336)
\curveto(445.58258352,341.62225929)(445.58258352,341.58225933)(445.57258638,341.54226336)
\curveto(445.55258355,341.46225945)(445.54758356,341.35225956)(445.55758638,341.21226336)
\curveto(445.56758354,341.07225984)(445.58258352,340.97225994)(445.60258638,340.91226336)
\curveto(445.62258348,340.82226009)(445.63258347,340.73726018)(445.63258638,340.65726336)
\curveto(445.64258346,340.57726034)(445.66258344,340.49726042)(445.69258638,340.41726336)
\curveto(445.78258332,340.13726078)(445.88758322,339.89226102)(446.00758638,339.68226336)
\curveto(446.13758297,339.48226143)(446.31758279,339.3122616)(446.54758638,339.17226336)
\curveto(446.7075824,339.07226184)(446.87258223,339.00226191)(447.04258638,338.96226336)
\curveto(447.06258204,338.96226195)(447.08258202,338.95726196)(447.10258638,338.94726336)
\lineto(447.19258638,338.94726336)
\curveto(447.22258188,338.93726198)(447.27258183,338.92726199)(447.34258638,338.91726336)
\curveto(447.41258169,338.917262)(447.47258163,338.92226199)(447.52258638,338.93226336)
\curveto(447.62258148,338.95226196)(447.71258139,338.96726195)(447.79258638,338.97726336)
\curveto(447.88258122,338.99726192)(447.96758114,339.02226189)(448.04758638,339.05226336)
\curveto(448.32758078,339.18226173)(448.54258056,339.36226155)(448.69258638,339.59226336)
\curveto(448.85258025,339.82226109)(448.98258012,340.09226082)(449.08258638,340.40226336)
}
}
{
\newrgbcolor{curcolor}{0 0 0}
\pscustom[linestyle=none,fillstyle=solid,fillcolor=curcolor]
{
\newpath
\moveto(453.09250826,345.27726336)
\lineto(454.21750826,345.27726336)
\curveto(454.32750582,345.27725564)(454.42750572,345.27225564)(454.51750826,345.26226336)
\curveto(454.60750554,345.25225566)(454.67250548,345.2172557)(454.71250826,345.15726336)
\curveto(454.76250539,345.09725582)(454.79250536,345.0122559)(454.80250826,344.90226336)
\curveto(454.81250534,344.80225611)(454.81750533,344.69725622)(454.81750826,344.58726336)
\lineto(454.81750826,343.53726336)
\lineto(454.81750826,341.30226336)
\curveto(454.81750533,340.94225997)(454.83250532,340.60226031)(454.86250826,340.28226336)
\curveto(454.89250526,339.96226095)(454.98250517,339.69726122)(455.13250826,339.48726336)
\curveto(455.27250488,339.27726164)(455.49750465,339.12726179)(455.80750826,339.03726336)
\curveto(455.85750429,339.02726189)(455.89750425,339.02226189)(455.92750826,339.02226336)
\curveto(455.96750418,339.02226189)(456.01250414,339.0172619)(456.06250826,339.00726336)
\curveto(456.11250404,338.99726192)(456.16750398,338.99226192)(456.22750826,338.99226336)
\curveto(456.28750386,338.99226192)(456.33250382,338.99726192)(456.36250826,339.00726336)
\curveto(456.41250374,339.02726189)(456.4525037,339.03226188)(456.48250826,339.02226336)
\curveto(456.52250363,339.0122619)(456.56250359,339.0172619)(456.60250826,339.03726336)
\curveto(456.81250334,339.08726183)(456.97750317,339.15226176)(457.09750826,339.23226336)
\curveto(457.27750287,339.34226157)(457.41750273,339.48226143)(457.51750826,339.65226336)
\curveto(457.62750252,339.83226108)(457.70250245,340.02726089)(457.74250826,340.23726336)
\curveto(457.79250236,340.45726046)(457.82250233,340.69726022)(457.83250826,340.95726336)
\curveto(457.84250231,341.22725969)(457.8475023,341.50725941)(457.84750826,341.79726336)
\lineto(457.84750826,343.61226336)
\lineto(457.84750826,344.58726336)
\lineto(457.84750826,344.85726336)
\curveto(457.8475023,344.95725596)(457.86750228,345.03725588)(457.90750826,345.09726336)
\curveto(457.95750219,345.18725573)(458.03250212,345.23725568)(458.13250826,345.24726336)
\curveto(458.23250192,345.26725565)(458.3525018,345.27725564)(458.49250826,345.27726336)
\lineto(459.28750826,345.27726336)
\lineto(459.57250826,345.27726336)
\curveto(459.66250049,345.27725564)(459.73750041,345.25725566)(459.79750826,345.21726336)
\curveto(459.87750027,345.16725575)(459.92250023,345.09225582)(459.93250826,344.99226336)
\curveto(459.94250021,344.89225602)(459.9475002,344.77725614)(459.94750826,344.64726336)
\lineto(459.94750826,343.50726336)
\lineto(459.94750826,339.29226336)
\lineto(459.94750826,338.22726336)
\lineto(459.94750826,337.92726336)
\curveto(459.9475002,337.82726309)(459.92750022,337.75226316)(459.88750826,337.70226336)
\curveto(459.83750031,337.62226329)(459.76250039,337.57726334)(459.66250826,337.56726336)
\curveto(459.56250059,337.55726336)(459.45750069,337.55226336)(459.34750826,337.55226336)
\lineto(458.53750826,337.55226336)
\curveto(458.42750172,337.55226336)(458.32750182,337.55726336)(458.23750826,337.56726336)
\curveto(458.15750199,337.57726334)(458.09250206,337.6172633)(458.04250826,337.68726336)
\curveto(458.02250213,337.7172632)(458.00250215,337.76226315)(457.98250826,337.82226336)
\curveto(457.97250218,337.88226303)(457.95750219,337.94226297)(457.93750826,338.00226336)
\curveto(457.92750222,338.06226285)(457.91250224,338.1172628)(457.89250826,338.16726336)
\curveto(457.87250228,338.2172627)(457.84250231,338.24726267)(457.80250826,338.25726336)
\curveto(457.78250237,338.27726264)(457.75750239,338.28226263)(457.72750826,338.27226336)
\curveto(457.69750245,338.26226265)(457.67250248,338.25226266)(457.65250826,338.24226336)
\curveto(457.58250257,338.20226271)(457.52250263,338.15726276)(457.47250826,338.10726336)
\curveto(457.42250273,338.05726286)(457.36750278,338.0122629)(457.30750826,337.97226336)
\curveto(457.26750288,337.94226297)(457.22750292,337.90726301)(457.18750826,337.86726336)
\curveto(457.15750299,337.83726308)(457.11750303,337.80726311)(457.06750826,337.77726336)
\curveto(456.83750331,337.63726328)(456.56750358,337.52726339)(456.25750826,337.44726336)
\curveto(456.18750396,337.42726349)(456.11750403,337.4172635)(456.04750826,337.41726336)
\curveto(455.97750417,337.40726351)(455.90250425,337.39226352)(455.82250826,337.37226336)
\curveto(455.78250437,337.36226355)(455.73750441,337.36226355)(455.68750826,337.37226336)
\curveto(455.6475045,337.37226354)(455.60750454,337.36726355)(455.56750826,337.35726336)
\curveto(455.53750461,337.34726357)(455.47250468,337.34726357)(455.37250826,337.35726336)
\curveto(455.28250487,337.35726356)(455.22250493,337.36226355)(455.19250826,337.37226336)
\curveto(455.14250501,337.37226354)(455.09250506,337.37726354)(455.04250826,337.38726336)
\lineto(454.89250826,337.38726336)
\curveto(454.77250538,337.4172635)(454.65750549,337.44226347)(454.54750826,337.46226336)
\curveto(454.43750571,337.48226343)(454.32750582,337.5122634)(454.21750826,337.55226336)
\curveto(454.16750598,337.57226334)(454.12250603,337.58726333)(454.08250826,337.59726336)
\curveto(454.0525061,337.6172633)(454.01250614,337.63726328)(453.96250826,337.65726336)
\curveto(453.61250654,337.84726307)(453.33250682,338.1122628)(453.12250826,338.45226336)
\curveto(452.99250716,338.66226225)(452.89750725,338.912262)(452.83750826,339.20226336)
\curveto(452.77750737,339.50226141)(452.73750741,339.8172611)(452.71750826,340.14726336)
\curveto(452.70750744,340.48726043)(452.70250745,340.83226008)(452.70250826,341.18226336)
\curveto(452.71250744,341.54225937)(452.71750743,341.89725902)(452.71750826,342.24726336)
\lineto(452.71750826,344.28726336)
\curveto(452.71750743,344.4172565)(452.71250744,344.56725635)(452.70250826,344.73726336)
\curveto(452.70250745,344.917256)(452.72750742,345.04725587)(452.77750826,345.12726336)
\curveto(452.80750734,345.17725574)(452.86750728,345.22225569)(452.95750826,345.26226336)
\curveto(453.01750713,345.26225565)(453.06250709,345.26725565)(453.09250826,345.27726336)
}
}
{
\newrgbcolor{curcolor}{0 0 0}
\pscustom[linestyle=none,fillstyle=solid,fillcolor=curcolor]
{
\newpath
\moveto(465.14875826,345.50226336)
\curveto(465.9587531,345.52225539)(466.63375242,345.40225551)(467.17375826,345.14226336)
\curveto(467.72375133,344.88225603)(468.1587509,344.5122564)(468.47875826,344.03226336)
\curveto(468.63875042,343.79225712)(468.7587503,343.5172574)(468.83875826,343.20726336)
\curveto(468.8587502,343.15725776)(468.87375018,343.09225782)(468.88375826,343.01226336)
\curveto(468.90375015,342.93225798)(468.90375015,342.86225805)(468.88375826,342.80226336)
\curveto(468.84375021,342.69225822)(468.77375028,342.62725829)(468.67375826,342.60726336)
\curveto(468.57375048,342.59725832)(468.4537506,342.59225832)(468.31375826,342.59226336)
\lineto(467.53375826,342.59226336)
\lineto(467.24875826,342.59226336)
\curveto(467.1587519,342.59225832)(467.08375197,342.6122583)(467.02375826,342.65226336)
\curveto(466.94375211,342.69225822)(466.88875217,342.75225816)(466.85875826,342.83226336)
\curveto(466.82875223,342.92225799)(466.78875227,343.0122579)(466.73875826,343.10226336)
\curveto(466.67875238,343.2122577)(466.61375244,343.3122576)(466.54375826,343.40226336)
\curveto(466.47375258,343.49225742)(466.39375266,343.57225734)(466.30375826,343.64226336)
\curveto(466.16375289,343.73225718)(466.00875305,343.80225711)(465.83875826,343.85226336)
\curveto(465.77875328,343.87225704)(465.71875334,343.88225703)(465.65875826,343.88226336)
\curveto(465.59875346,343.88225703)(465.54375351,343.89225702)(465.49375826,343.91226336)
\lineto(465.34375826,343.91226336)
\curveto(465.14375391,343.912257)(464.98375407,343.89225702)(464.86375826,343.85226336)
\curveto(464.57375448,343.76225715)(464.33875472,343.62225729)(464.15875826,343.43226336)
\curveto(463.97875508,343.25225766)(463.83375522,343.03225788)(463.72375826,342.77226336)
\curveto(463.67375538,342.66225825)(463.63375542,342.54225837)(463.60375826,342.41226336)
\curveto(463.58375547,342.29225862)(463.5587555,342.16225875)(463.52875826,342.02226336)
\curveto(463.51875554,341.98225893)(463.51375554,341.94225897)(463.51375826,341.90226336)
\curveto(463.51375554,341.86225905)(463.50875555,341.82225909)(463.49875826,341.78226336)
\curveto(463.47875558,341.68225923)(463.46875559,341.54225937)(463.46875826,341.36226336)
\curveto(463.47875558,341.18225973)(463.49375556,341.04225987)(463.51375826,340.94226336)
\curveto(463.51375554,340.86226005)(463.51875554,340.80726011)(463.52875826,340.77726336)
\curveto(463.54875551,340.70726021)(463.5587555,340.63726028)(463.55875826,340.56726336)
\curveto(463.56875549,340.49726042)(463.58375547,340.42726049)(463.60375826,340.35726336)
\curveto(463.68375537,340.12726079)(463.77875528,339.917261)(463.88875826,339.72726336)
\curveto(463.99875506,339.53726138)(464.13875492,339.37726154)(464.30875826,339.24726336)
\curveto(464.34875471,339.2172617)(464.40875465,339.18226173)(464.48875826,339.14226336)
\curveto(464.59875446,339.07226184)(464.70875435,339.02726189)(464.81875826,339.00726336)
\curveto(464.93875412,338.98726193)(465.08375397,338.96726195)(465.25375826,338.94726336)
\lineto(465.34375826,338.94726336)
\curveto(465.38375367,338.94726197)(465.41375364,338.95226196)(465.43375826,338.96226336)
\lineto(465.56875826,338.96226336)
\curveto(465.63875342,338.98226193)(465.70375335,338.99726192)(465.76375826,339.00726336)
\curveto(465.83375322,339.02726189)(465.89875316,339.04726187)(465.95875826,339.06726336)
\curveto(466.2587528,339.19726172)(466.48875257,339.38726153)(466.64875826,339.63726336)
\curveto(466.68875237,339.68726123)(466.72375233,339.74226117)(466.75375826,339.80226336)
\curveto(466.78375227,339.87226104)(466.80875225,339.93226098)(466.82875826,339.98226336)
\curveto(466.86875219,340.09226082)(466.90375215,340.18726073)(466.93375826,340.26726336)
\curveto(466.96375209,340.35726056)(467.03375202,340.42726049)(467.14375826,340.47726336)
\curveto(467.23375182,340.5172604)(467.37875168,340.53226038)(467.57875826,340.52226336)
\lineto(468.07375826,340.52226336)
\lineto(468.28375826,340.52226336)
\curveto(468.36375069,340.53226038)(468.42875063,340.52726039)(468.47875826,340.50726336)
\lineto(468.59875826,340.50726336)
\lineto(468.71875826,340.47726336)
\curveto(468.7587503,340.47726044)(468.78875027,340.46726045)(468.80875826,340.44726336)
\curveto(468.8587502,340.40726051)(468.88875017,340.34726057)(468.89875826,340.26726336)
\curveto(468.91875014,340.19726072)(468.91875014,340.12226079)(468.89875826,340.04226336)
\curveto(468.80875025,339.7122612)(468.69875036,339.4172615)(468.56875826,339.15726336)
\curveto(468.1587509,338.38726253)(467.50375155,337.85226306)(466.60375826,337.55226336)
\curveto(466.50375255,337.52226339)(466.39875266,337.50226341)(466.28875826,337.49226336)
\curveto(466.17875288,337.47226344)(466.06875299,337.44726347)(465.95875826,337.41726336)
\curveto(465.89875316,337.40726351)(465.83875322,337.40226351)(465.77875826,337.40226336)
\curveto(465.71875334,337.40226351)(465.6587534,337.39726352)(465.59875826,337.38726336)
\lineto(465.43375826,337.38726336)
\curveto(465.38375367,337.36726355)(465.30875375,337.36226355)(465.20875826,337.37226336)
\curveto(465.10875395,337.37226354)(465.03375402,337.37726354)(464.98375826,337.38726336)
\curveto(464.90375415,337.40726351)(464.82875423,337.4172635)(464.75875826,337.41726336)
\curveto(464.69875436,337.40726351)(464.63375442,337.4122635)(464.56375826,337.43226336)
\lineto(464.41375826,337.46226336)
\curveto(464.36375469,337.46226345)(464.31375474,337.46726345)(464.26375826,337.47726336)
\curveto(464.1537549,337.50726341)(464.04875501,337.53726338)(463.94875826,337.56726336)
\curveto(463.84875521,337.59726332)(463.7537553,337.63226328)(463.66375826,337.67226336)
\curveto(463.19375586,337.87226304)(462.79875626,338.12726279)(462.47875826,338.43726336)
\curveto(462.1587569,338.75726216)(461.89875716,339.15226176)(461.69875826,339.62226336)
\curveto(461.64875741,339.7122612)(461.60875745,339.80726111)(461.57875826,339.90726336)
\lineto(461.48875826,340.23726336)
\curveto(461.47875758,340.27726064)(461.47375758,340.3122606)(461.47375826,340.34226336)
\curveto(461.47375758,340.38226053)(461.46375759,340.42726049)(461.44375826,340.47726336)
\curveto(461.42375763,340.54726037)(461.41375764,340.6172603)(461.41375826,340.68726336)
\curveto(461.41375764,340.76726015)(461.40375765,340.84226007)(461.38375826,340.91226336)
\lineto(461.38375826,341.16726336)
\curveto(461.36375769,341.2172597)(461.3537577,341.27225964)(461.35375826,341.33226336)
\curveto(461.3537577,341.40225951)(461.36375769,341.46225945)(461.38375826,341.51226336)
\curveto(461.39375766,341.56225935)(461.39375766,341.60725931)(461.38375826,341.64726336)
\curveto(461.37375768,341.68725923)(461.37375768,341.72725919)(461.38375826,341.76726336)
\curveto(461.40375765,341.83725908)(461.40875765,341.90225901)(461.39875826,341.96226336)
\curveto(461.39875766,342.02225889)(461.40875765,342.08225883)(461.42875826,342.14226336)
\curveto(461.47875758,342.32225859)(461.51875754,342.49225842)(461.54875826,342.65226336)
\curveto(461.57875748,342.82225809)(461.62375743,342.98725793)(461.68375826,343.14726336)
\curveto(461.90375715,343.65725726)(462.17875688,344.08225683)(462.50875826,344.42226336)
\curveto(462.84875621,344.76225615)(463.27875578,345.03725588)(463.79875826,345.24726336)
\curveto(463.93875512,345.30725561)(464.08375497,345.34725557)(464.23375826,345.36726336)
\curveto(464.38375467,345.39725552)(464.53875452,345.43225548)(464.69875826,345.47226336)
\curveto(464.77875428,345.48225543)(464.8537542,345.48725543)(464.92375826,345.48726336)
\curveto(464.99375406,345.48725543)(465.06875399,345.49225542)(465.14875826,345.50226336)
}
}
{
\newrgbcolor{curcolor}{0 0 0}
\pscustom[linestyle=none,fillstyle=solid,fillcolor=curcolor]
{
\newpath
\moveto(472.29203951,348.14226336)
\curveto(472.36203656,348.06225285)(472.39703652,347.94225297)(472.39703951,347.78226336)
\lineto(472.39703951,347.31726336)
\lineto(472.39703951,346.91226336)
\curveto(472.39703652,346.77225414)(472.36203656,346.67725424)(472.29203951,346.62726336)
\curveto(472.23203669,346.57725434)(472.15203677,346.54725437)(472.05203951,346.53726336)
\curveto(471.96203696,346.52725439)(471.86203706,346.52225439)(471.75203951,346.52226336)
\lineto(470.91203951,346.52226336)
\curveto(470.80203812,346.52225439)(470.70203822,346.52725439)(470.61203951,346.53726336)
\curveto(470.53203839,346.54725437)(470.46203846,346.57725434)(470.40203951,346.62726336)
\curveto(470.36203856,346.65725426)(470.33203859,346.7122542)(470.31203951,346.79226336)
\curveto(470.30203862,346.88225403)(470.29203863,346.97725394)(470.28203951,347.07726336)
\lineto(470.28203951,347.40726336)
\curveto(470.29203863,347.5172534)(470.29703862,347.6122533)(470.29703951,347.69226336)
\lineto(470.29703951,347.90226336)
\curveto(470.30703861,347.97225294)(470.32703859,348.03225288)(470.35703951,348.08226336)
\curveto(470.37703854,348.12225279)(470.40203852,348.15225276)(470.43203951,348.17226336)
\lineto(470.55203951,348.23226336)
\curveto(470.57203835,348.23225268)(470.59703832,348.23225268)(470.62703951,348.23226336)
\curveto(470.65703826,348.24225267)(470.68203824,348.24725267)(470.70203951,348.24726336)
\lineto(471.79703951,348.24726336)
\curveto(471.89703702,348.24725267)(471.99203693,348.24225267)(472.08203951,348.23226336)
\curveto(472.17203675,348.22225269)(472.24203668,348.19225272)(472.29203951,348.14226336)
\moveto(472.39703951,338.37726336)
\curveto(472.39703652,338.17726274)(472.39203653,338.00726291)(472.38203951,337.86726336)
\curveto(472.37203655,337.72726319)(472.28203664,337.63226328)(472.11203951,337.58226336)
\curveto(472.05203687,337.56226335)(471.98703693,337.55226336)(471.91703951,337.55226336)
\curveto(471.84703707,337.56226335)(471.77203715,337.56726335)(471.69203951,337.56726336)
\lineto(470.85203951,337.56726336)
\curveto(470.76203816,337.56726335)(470.67203825,337.57226334)(470.58203951,337.58226336)
\curveto(470.50203842,337.59226332)(470.44203848,337.62226329)(470.40203951,337.67226336)
\curveto(470.34203858,337.74226317)(470.30703861,337.82726309)(470.29703951,337.92726336)
\lineto(470.29703951,338.27226336)
\lineto(470.29703951,344.60226336)
\lineto(470.29703951,344.90226336)
\curveto(470.29703862,345.00225591)(470.3170386,345.08225583)(470.35703951,345.14226336)
\curveto(470.4170385,345.2122557)(470.50203842,345.25725566)(470.61203951,345.27726336)
\curveto(470.63203829,345.28725563)(470.65703826,345.28725563)(470.68703951,345.27726336)
\curveto(470.72703819,345.27725564)(470.75703816,345.28225563)(470.77703951,345.29226336)
\lineto(471.52703951,345.29226336)
\lineto(471.72203951,345.29226336)
\curveto(471.80203712,345.30225561)(471.86703705,345.30225561)(471.91703951,345.29226336)
\lineto(472.03703951,345.29226336)
\curveto(472.09703682,345.27225564)(472.15203677,345.25725566)(472.20203951,345.24726336)
\curveto(472.25203667,345.23725568)(472.29203663,345.20725571)(472.32203951,345.15726336)
\curveto(472.36203656,345.10725581)(472.38203654,345.03725588)(472.38203951,344.94726336)
\curveto(472.39203653,344.85725606)(472.39703652,344.76225615)(472.39703951,344.66226336)
\lineto(472.39703951,338.37726336)
}
}
{
\newrgbcolor{curcolor}{0 0 0}
\pscustom[linestyle=none,fillstyle=solid,fillcolor=curcolor]
{
\newpath
\moveto(481.82922701,341.73726336)
\curveto(481.80921848,341.78725913)(481.80421848,341.84225907)(481.81422701,341.90226336)
\curveto(481.82421846,341.96225895)(481.81921847,342.0172589)(481.79922701,342.06726336)
\curveto(481.7892185,342.10725881)(481.7842185,342.14725877)(481.78422701,342.18726336)
\curveto(481.7842185,342.22725869)(481.77921851,342.26725865)(481.76922701,342.30726336)
\lineto(481.70922701,342.57726336)
\curveto(481.6892186,342.66725825)(481.66421862,342.75225816)(481.63422701,342.83226336)
\curveto(481.5842187,342.97225794)(481.53921875,343.10225781)(481.49922701,343.22226336)
\curveto(481.45921883,343.35225756)(481.40421888,343.47225744)(481.33422701,343.58226336)
\curveto(481.26421902,343.69225722)(481.19421909,343.79725712)(481.12422701,343.89726336)
\curveto(481.06421922,343.99725692)(480.99421929,344.09725682)(480.91422701,344.19726336)
\curveto(480.83421945,344.30725661)(480.73421955,344.40725651)(480.61422701,344.49726336)
\curveto(480.50421978,344.59725632)(480.39421989,344.68725623)(480.28422701,344.76726336)
\curveto(479.95422033,344.99725592)(479.57422071,345.17725574)(479.14422701,345.30726336)
\curveto(478.72422156,345.43725548)(478.22422206,345.49725542)(477.64422701,345.48726336)
\curveto(477.57422271,345.47725544)(477.50422278,345.47225544)(477.43422701,345.47226336)
\curveto(477.36422292,345.47225544)(477.289223,345.46725545)(477.20922701,345.45726336)
\curveto(477.05922323,345.4172555)(476.91422337,345.38725553)(476.77422701,345.36726336)
\curveto(476.63422365,345.34725557)(476.49922379,345.3122556)(476.36922701,345.26226336)
\curveto(476.25922403,345.2122557)(476.14922414,345.16725575)(476.03922701,345.12726336)
\curveto(475.92922436,345.08725583)(475.82422446,345.04225587)(475.72422701,344.99226336)
\curveto(475.36422492,344.76225615)(475.05922523,344.50725641)(474.80922701,344.22726336)
\curveto(474.55922573,343.95725696)(474.34422594,343.6172573)(474.16422701,343.20726336)
\curveto(474.11422617,343.08725783)(474.07422621,342.96225795)(474.04422701,342.83226336)
\curveto(474.01422627,342.7122582)(473.97922631,342.58725833)(473.93922701,342.45726336)
\curveto(473.91922637,342.40725851)(473.90922638,342.35725856)(473.90922701,342.30726336)
\curveto(473.90922638,342.26725865)(473.90422638,342.22225869)(473.89422701,342.17226336)
\curveto(473.87422641,342.12225879)(473.86422642,342.06725885)(473.86422701,342.00726336)
\curveto(473.87422641,341.95725896)(473.87422641,341.90725901)(473.86422701,341.85726336)
\lineto(473.86422701,341.75226336)
\curveto(473.84422644,341.69225922)(473.82922646,341.60725931)(473.81922701,341.49726336)
\curveto(473.81922647,341.38725953)(473.82922646,341.30225961)(473.84922701,341.24226336)
\lineto(473.84922701,341.10726336)
\curveto(473.84922644,341.06725985)(473.85422643,341.02225989)(473.86422701,340.97226336)
\curveto(473.8842264,340.89226002)(473.89422639,340.80726011)(473.89422701,340.71726336)
\curveto(473.89422639,340.63726028)(473.90422638,340.55726036)(473.92422701,340.47726336)
\curveto(473.94422634,340.42726049)(473.95422633,340.38226053)(473.95422701,340.34226336)
\curveto(473.95422633,340.30226061)(473.96422632,340.25726066)(473.98422701,340.20726336)
\curveto(474.01422627,340.09726082)(474.03922625,339.99226092)(474.05922701,339.89226336)
\curveto(474.0892262,339.79226112)(474.12922616,339.69726122)(474.17922701,339.60726336)
\curveto(474.34922594,339.2172617)(474.55922573,338.88226203)(474.80922701,338.60226336)
\curveto(475.05922523,338.32226259)(475.35922493,338.07726284)(475.70922701,337.86726336)
\curveto(475.81922447,337.80726311)(475.92422436,337.75726316)(476.02422701,337.71726336)
\curveto(476.13422415,337.67726324)(476.24922404,337.63726328)(476.36922701,337.59726336)
\curveto(476.45922383,337.55726336)(476.55422373,337.52726339)(476.65422701,337.50726336)
\curveto(476.75422353,337.48726343)(476.85422343,337.46226345)(476.95422701,337.43226336)
\curveto(477.00422328,337.42226349)(477.04422324,337.4172635)(477.07422701,337.41726336)
\curveto(477.11422317,337.4172635)(477.15422313,337.4122635)(477.19422701,337.40226336)
\curveto(477.24422304,337.38226353)(477.29422299,337.37726354)(477.34422701,337.38726336)
\curveto(477.40422288,337.38726353)(477.45922283,337.38226353)(477.50922701,337.37226336)
\lineto(477.65922701,337.37226336)
\curveto(477.71922257,337.35226356)(477.80422248,337.34726357)(477.91422701,337.35726336)
\curveto(478.02422226,337.35726356)(478.10422218,337.36226355)(478.15422701,337.37226336)
\curveto(478.1842221,337.37226354)(478.21422207,337.37726354)(478.24422701,337.38726336)
\lineto(478.34922701,337.38726336)
\curveto(478.39922189,337.39726352)(478.45422183,337.40226351)(478.51422701,337.40226336)
\curveto(478.57422171,337.40226351)(478.62922166,337.4122635)(478.67922701,337.43226336)
\curveto(478.80922148,337.46226345)(478.93422135,337.49226342)(479.05422701,337.52226336)
\curveto(479.1842211,337.54226337)(479.30922098,337.57726334)(479.42922701,337.62726336)
\curveto(479.90922038,337.82726309)(480.31921997,338.07726284)(480.65922701,338.37726336)
\curveto(480.99921929,338.67726224)(481.27421901,339.06726185)(481.48422701,339.54726336)
\curveto(481.53421875,339.64726127)(481.57421871,339.75226116)(481.60422701,339.86226336)
\curveto(481.63421865,339.98226093)(481.66921862,340.09726082)(481.70922701,340.20726336)
\curveto(481.71921857,340.27726064)(481.72921856,340.34226057)(481.73922701,340.40226336)
\curveto(481.74921854,340.46226045)(481.76421852,340.52726039)(481.78422701,340.59726336)
\curveto(481.80421848,340.67726024)(481.80921848,340.75726016)(481.79922701,340.83726336)
\curveto(481.79921849,340.91726)(481.80921848,340.99725992)(481.82922701,341.07726336)
\lineto(481.82922701,341.22726336)
\curveto(481.84921844,341.28725963)(481.85921843,341.37225954)(481.85922701,341.48226336)
\curveto(481.85921843,341.59225932)(481.84921844,341.67725924)(481.82922701,341.73726336)
\moveto(479.72922701,341.19726336)
\curveto(479.71922057,341.14725977)(479.71422057,341.09725982)(479.71422701,341.04726336)
\lineto(479.71422701,340.91226336)
\curveto(479.70422058,340.87226004)(479.69922059,340.83226008)(479.69922701,340.79226336)
\curveto(479.69922059,340.76226015)(479.69422059,340.72726019)(479.68422701,340.68726336)
\curveto(479.65422063,340.57726034)(479.62922066,340.47226044)(479.60922701,340.37226336)
\curveto(479.5892207,340.27226064)(479.55922073,340.17226074)(479.51922701,340.07226336)
\curveto(479.40922088,339.82226109)(479.27422101,339.6122613)(479.11422701,339.44226336)
\curveto(478.95422133,339.27226164)(478.74422154,339.13726178)(478.48422701,339.03726336)
\curveto(478.41422187,339.00726191)(478.33922195,338.98726193)(478.25922701,338.97726336)
\curveto(478.17922211,338.96726195)(478.09922219,338.95226196)(478.01922701,338.93226336)
\lineto(477.89922701,338.93226336)
\curveto(477.85922243,338.92226199)(477.81422247,338.917262)(477.76422701,338.91726336)
\lineto(477.64422701,338.94726336)
\curveto(477.60422268,338.95726196)(477.56922272,338.95726196)(477.53922701,338.94726336)
\curveto(477.50922278,338.94726197)(477.47422281,338.95226196)(477.43422701,338.96226336)
\curveto(477.34422294,338.98226193)(477.25422303,339.00726191)(477.16422701,339.03726336)
\curveto(477.0842232,339.06726185)(477.00922328,339.10726181)(476.93922701,339.15726336)
\curveto(476.6892236,339.30726161)(476.50422378,339.47226144)(476.38422701,339.65226336)
\curveto(476.27422401,339.84226107)(476.16922412,340.08726083)(476.06922701,340.38726336)
\curveto(476.04922424,340.46726045)(476.03422425,340.54226037)(476.02422701,340.61226336)
\curveto(476.01422427,340.69226022)(475.99922429,340.77226014)(475.97922701,340.85226336)
\lineto(475.97922701,340.98726336)
\curveto(475.95922433,341.05725986)(475.94422434,341.16225975)(475.93422701,341.30226336)
\curveto(475.93422435,341.44225947)(475.94422434,341.54725937)(475.96422701,341.61726336)
\lineto(475.96422701,341.76726336)
\curveto(475.96422432,341.8172591)(475.96922432,341.86725905)(475.97922701,341.91726336)
\curveto(475.99922429,342.02725889)(476.01422427,342.13725878)(476.02422701,342.24726336)
\curveto(476.04422424,342.35725856)(476.06922422,342.46225845)(476.09922701,342.56226336)
\curveto(476.1892241,342.83225808)(476.30922398,343.06725785)(476.45922701,343.26726336)
\curveto(476.61922367,343.47725744)(476.82422346,343.63725728)(477.07422701,343.74726336)
\curveto(477.12422316,343.77725714)(477.17922311,343.79725712)(477.23922701,343.80726336)
\lineto(477.44922701,343.86726336)
\curveto(477.47922281,343.87725704)(477.51422277,343.87725704)(477.55422701,343.86726336)
\curveto(477.59422269,343.86725705)(477.62922266,343.87725704)(477.65922701,343.89726336)
\lineto(477.92922701,343.89726336)
\curveto(478.01922227,343.90725701)(478.10422218,343.90225701)(478.18422701,343.88226336)
\curveto(478.25422203,343.86225705)(478.31922197,343.84225707)(478.37922701,343.82226336)
\curveto(478.43922185,343.8122571)(478.49922179,343.79725712)(478.55922701,343.77726336)
\curveto(478.80922148,343.66725725)(479.00922128,343.5172574)(479.15922701,343.32726336)
\curveto(479.30922098,343.14725777)(479.43922085,342.92725799)(479.54922701,342.66726336)
\curveto(479.57922071,342.58725833)(479.59922069,342.50225841)(479.60922701,342.41226336)
\lineto(479.66922701,342.17226336)
\curveto(479.67922061,342.15225876)(479.6842206,342.12225879)(479.68422701,342.08226336)
\curveto(479.69422059,342.03225888)(479.69922059,341.97725894)(479.69922701,341.91726336)
\curveto(479.69922059,341.85725906)(479.70922058,341.80225911)(479.72922701,341.75226336)
\lineto(479.72922701,341.63226336)
\curveto(479.73922055,341.58225933)(479.74422054,341.50725941)(479.74422701,341.40726336)
\curveto(479.74422054,341.3172596)(479.73922055,341.24725967)(479.72922701,341.19726336)
\moveto(478.49922701,348.36726336)
\lineto(479.56422701,348.36726336)
\curveto(479.64422064,348.36725255)(479.73922055,348.36725255)(479.84922701,348.36726336)
\curveto(479.95922033,348.36725255)(480.03922025,348.35225256)(480.08922701,348.32226336)
\curveto(480.10922018,348.3122526)(480.11922017,348.29725262)(480.11922701,348.27726336)
\curveto(480.12922016,348.26725265)(480.14422014,348.25725266)(480.16422701,348.24726336)
\curveto(480.17422011,348.12725279)(480.12422016,348.02225289)(480.01422701,347.93226336)
\curveto(479.91422037,347.84225307)(479.82922046,347.76225315)(479.75922701,347.69226336)
\curveto(479.67922061,347.62225329)(479.59922069,347.54725337)(479.51922701,347.46726336)
\curveto(479.44922084,347.39725352)(479.37422091,347.33225358)(479.29422701,347.27226336)
\curveto(479.25422103,347.24225367)(479.21922107,347.20725371)(479.18922701,347.16726336)
\curveto(479.16922112,347.13725378)(479.13922115,347.1122538)(479.09922701,347.09226336)
\curveto(479.07922121,347.06225385)(479.05422123,347.03725388)(479.02422701,347.01726336)
\lineto(478.87422701,346.86726336)
\lineto(478.72422701,346.74726336)
\lineto(478.67922701,346.70226336)
\curveto(478.67922161,346.69225422)(478.66922162,346.67725424)(478.64922701,346.65726336)
\curveto(478.56922172,346.59725432)(478.4892218,346.53225438)(478.40922701,346.46226336)
\curveto(478.33922195,346.39225452)(478.24922204,346.33725458)(478.13922701,346.29726336)
\curveto(478.09922219,346.28725463)(478.05922223,346.28225463)(478.01922701,346.28226336)
\curveto(477.9892223,346.28225463)(477.94922234,346.27725464)(477.89922701,346.26726336)
\curveto(477.86922242,346.25725466)(477.82922246,346.25225466)(477.77922701,346.25226336)
\curveto(477.72922256,346.26225465)(477.6842226,346.26725465)(477.64422701,346.26726336)
\lineto(477.29922701,346.26726336)
\curveto(477.17922311,346.26725465)(477.0892232,346.29225462)(477.02922701,346.34226336)
\curveto(476.96922332,346.38225453)(476.95422333,346.45225446)(476.98422701,346.55226336)
\curveto(477.00422328,346.63225428)(477.03922325,346.70225421)(477.08922701,346.76226336)
\curveto(477.13922315,346.83225408)(477.1842231,346.90225401)(477.22422701,346.97226336)
\curveto(477.32422296,347.1122538)(477.41922287,347.24725367)(477.50922701,347.37726336)
\curveto(477.59922269,347.50725341)(477.6892226,347.64225327)(477.77922701,347.78226336)
\curveto(477.82922246,347.86225305)(477.87922241,347.94725297)(477.92922701,348.03726336)
\curveto(477.9892223,348.12725279)(478.05422223,348.19725272)(478.12422701,348.24726336)
\curveto(478.16422212,348.27725264)(478.23422205,348.3122526)(478.33422701,348.35226336)
\curveto(478.35422193,348.36225255)(478.37922191,348.36225255)(478.40922701,348.35226336)
\curveto(478.44922184,348.35225256)(478.47922181,348.35725256)(478.49922701,348.36726336)
}
}
{
\newrgbcolor{curcolor}{0 0 0}
\pscustom[linestyle=none,fillstyle=solid,fillcolor=curcolor]
{
\newpath
\moveto(487.65414888,345.48726336)
\curveto(488.25414308,345.50725541)(488.75414258,345.42225549)(489.15414888,345.23226336)
\curveto(489.55414178,345.04225587)(489.86914146,344.76225615)(490.09914888,344.39226336)
\curveto(490.16914116,344.28225663)(490.22414111,344.16225675)(490.26414888,344.03226336)
\curveto(490.30414103,343.912257)(490.34414099,343.78725713)(490.38414888,343.65726336)
\curveto(490.40414093,343.57725734)(490.41414092,343.50225741)(490.41414888,343.43226336)
\curveto(490.42414091,343.36225755)(490.43914089,343.29225762)(490.45914888,343.22226336)
\curveto(490.45914087,343.16225775)(490.46414087,343.12225779)(490.47414888,343.10226336)
\curveto(490.49414084,342.96225795)(490.50414083,342.8172581)(490.50414888,342.66726336)
\lineto(490.50414888,342.23226336)
\lineto(490.50414888,340.89726336)
\lineto(490.50414888,338.46726336)
\curveto(490.50414083,338.27726264)(490.49914083,338.09226282)(490.48914888,337.91226336)
\curveto(490.48914084,337.74226317)(490.41914091,337.63226328)(490.27914888,337.58226336)
\curveto(490.21914111,337.56226335)(490.14914118,337.55226336)(490.06914888,337.55226336)
\lineto(489.82914888,337.55226336)
\lineto(489.01914888,337.55226336)
\curveto(488.89914243,337.55226336)(488.78914254,337.55726336)(488.68914888,337.56726336)
\curveto(488.59914273,337.58726333)(488.5291428,337.63226328)(488.47914888,337.70226336)
\curveto(488.43914289,337.76226315)(488.41414292,337.83726308)(488.40414888,337.92726336)
\lineto(488.40414888,338.24226336)
\lineto(488.40414888,339.29226336)
\lineto(488.40414888,341.52726336)
\curveto(488.40414293,341.89725902)(488.38914294,342.23725868)(488.35914888,342.54726336)
\curveto(488.329143,342.86725805)(488.23914309,343.13725778)(488.08914888,343.35726336)
\curveto(487.94914338,343.55725736)(487.74414359,343.69725722)(487.47414888,343.77726336)
\curveto(487.42414391,343.79725712)(487.36914396,343.80725711)(487.30914888,343.80726336)
\curveto(487.25914407,343.80725711)(487.20414413,343.8172571)(487.14414888,343.83726336)
\curveto(487.09414424,343.84725707)(487.0291443,343.84725707)(486.94914888,343.83726336)
\curveto(486.87914445,343.83725708)(486.82414451,343.83225708)(486.78414888,343.82226336)
\curveto(486.74414459,343.8122571)(486.70914462,343.80725711)(486.67914888,343.80726336)
\curveto(486.64914468,343.80725711)(486.61914471,343.80225711)(486.58914888,343.79226336)
\curveto(486.35914497,343.73225718)(486.17414516,343.65225726)(486.03414888,343.55226336)
\curveto(485.71414562,343.32225759)(485.52414581,342.98725793)(485.46414888,342.54726336)
\curveto(485.40414593,342.10725881)(485.37414596,341.6122593)(485.37414888,341.06226336)
\lineto(485.37414888,339.18726336)
\lineto(485.37414888,338.27226336)
\lineto(485.37414888,338.00226336)
\curveto(485.37414596,337.912263)(485.35914597,337.83726308)(485.32914888,337.77726336)
\curveto(485.27914605,337.66726325)(485.19914613,337.60226331)(485.08914888,337.58226336)
\curveto(484.97914635,337.56226335)(484.84414649,337.55226336)(484.68414888,337.55226336)
\lineto(483.93414888,337.55226336)
\curveto(483.82414751,337.55226336)(483.71414762,337.55726336)(483.60414888,337.56726336)
\curveto(483.49414784,337.57726334)(483.41414792,337.6122633)(483.36414888,337.67226336)
\curveto(483.29414804,337.76226315)(483.25914807,337.89226302)(483.25914888,338.06226336)
\curveto(483.26914806,338.23226268)(483.27414806,338.39226252)(483.27414888,338.54226336)
\lineto(483.27414888,340.58226336)
\lineto(483.27414888,343.88226336)
\lineto(483.27414888,344.64726336)
\lineto(483.27414888,344.94726336)
\curveto(483.28414805,345.03725588)(483.31414802,345.1122558)(483.36414888,345.17226336)
\curveto(483.38414795,345.20225571)(483.41414792,345.22225569)(483.45414888,345.23226336)
\curveto(483.50414783,345.25225566)(483.55414778,345.26725565)(483.60414888,345.27726336)
\lineto(483.67914888,345.27726336)
\curveto(483.7291476,345.28725563)(483.77914755,345.29225562)(483.82914888,345.29226336)
\lineto(483.99414888,345.29226336)
\lineto(484.62414888,345.29226336)
\curveto(484.70414663,345.29225562)(484.77914655,345.28725563)(484.84914888,345.27726336)
\curveto(484.9291464,345.27725564)(484.99914633,345.26725565)(485.05914888,345.24726336)
\curveto(485.1291462,345.2172557)(485.17414616,345.17225574)(485.19414888,345.11226336)
\curveto(485.22414611,345.05225586)(485.24914608,344.98225593)(485.26914888,344.90226336)
\curveto(485.27914605,344.86225605)(485.27914605,344.82725609)(485.26914888,344.79726336)
\curveto(485.26914606,344.76725615)(485.27914605,344.73725618)(485.29914888,344.70726336)
\curveto(485.31914601,344.65725626)(485.334146,344.62725629)(485.34414888,344.61726336)
\curveto(485.36414597,344.60725631)(485.38914594,344.59225632)(485.41914888,344.57226336)
\curveto(485.5291458,344.56225635)(485.61914571,344.59725632)(485.68914888,344.67726336)
\curveto(485.75914557,344.76725615)(485.8341455,344.83725608)(485.91414888,344.88726336)
\curveto(486.18414515,345.08725583)(486.48414485,345.24725567)(486.81414888,345.36726336)
\curveto(486.90414443,345.39725552)(486.99414434,345.4172555)(487.08414888,345.42726336)
\curveto(487.18414415,345.43725548)(487.28914404,345.45225546)(487.39914888,345.47226336)
\curveto(487.4291439,345.48225543)(487.47414386,345.48225543)(487.53414888,345.47226336)
\curveto(487.59414374,345.47225544)(487.6341437,345.47725544)(487.65414888,345.48726336)
}
}
{
\newrgbcolor{curcolor}{0 0 0}
\pscustom[linestyle=none,fillstyle=solid,fillcolor=curcolor]
{
}
}
{
\newrgbcolor{curcolor}{0 0 0}
\pscustom[linestyle=none,fillstyle=solid,fillcolor=curcolor]
{
\newpath
\moveto(503.88555513,338.40726336)
\lineto(503.88555513,337.98726336)
\curveto(503.88554676,337.85726306)(503.85554679,337.75226316)(503.79555513,337.67226336)
\curveto(503.7455469,337.62226329)(503.68054697,337.58726333)(503.60055513,337.56726336)
\curveto(503.52054713,337.55726336)(503.43054722,337.55226336)(503.33055513,337.55226336)
\lineto(502.50555513,337.55226336)
\lineto(502.22055513,337.55226336)
\curveto(502.14054851,337.56226335)(502.07554857,337.58726333)(502.02555513,337.62726336)
\curveto(501.95554869,337.67726324)(501.91554873,337.74226317)(501.90555513,337.82226336)
\curveto(501.89554875,337.90226301)(501.87554877,337.98226293)(501.84555513,338.06226336)
\curveto(501.82554882,338.08226283)(501.80554884,338.09726282)(501.78555513,338.10726336)
\curveto(501.77554887,338.12726279)(501.76054889,338.14726277)(501.74055513,338.16726336)
\curveto(501.63054902,338.16726275)(501.5505491,338.14226277)(501.50055513,338.09226336)
\lineto(501.35055513,337.94226336)
\curveto(501.28054937,337.89226302)(501.21554943,337.84726307)(501.15555513,337.80726336)
\curveto(501.09554955,337.77726314)(501.03054962,337.73726318)(500.96055513,337.68726336)
\curveto(500.92054973,337.66726325)(500.87554977,337.64726327)(500.82555513,337.62726336)
\curveto(500.78554986,337.60726331)(500.74054991,337.58726333)(500.69055513,337.56726336)
\curveto(500.5505501,337.5172634)(500.40055025,337.47226344)(500.24055513,337.43226336)
\curveto(500.19055046,337.4122635)(500.1455505,337.40226351)(500.10555513,337.40226336)
\curveto(500.06555058,337.40226351)(500.02555062,337.39726352)(499.98555513,337.38726336)
\lineto(499.85055513,337.38726336)
\curveto(499.82055083,337.37726354)(499.78055087,337.37226354)(499.73055513,337.37226336)
\lineto(499.59555513,337.37226336)
\curveto(499.53555111,337.35226356)(499.4455512,337.34726357)(499.32555513,337.35726336)
\curveto(499.20555144,337.35726356)(499.12055153,337.36726355)(499.07055513,337.38726336)
\curveto(499.00055165,337.40726351)(498.93555171,337.4172635)(498.87555513,337.41726336)
\curveto(498.82555182,337.40726351)(498.77055188,337.4122635)(498.71055513,337.43226336)
\lineto(498.35055513,337.55226336)
\curveto(498.24055241,337.58226333)(498.13055252,337.62226329)(498.02055513,337.67226336)
\curveto(497.67055298,337.82226309)(497.35555329,338.05226286)(497.07555513,338.36226336)
\curveto(496.80555384,338.68226223)(496.59055406,339.0172619)(496.43055513,339.36726336)
\curveto(496.38055427,339.47726144)(496.34055431,339.58226133)(496.31055513,339.68226336)
\curveto(496.28055437,339.79226112)(496.2455544,339.90226101)(496.20555513,340.01226336)
\curveto(496.19555445,340.05226086)(496.19055446,340.08726083)(496.19055513,340.11726336)
\curveto(496.19055446,340.15726076)(496.18055447,340.20226071)(496.16055513,340.25226336)
\curveto(496.14055451,340.33226058)(496.12055453,340.4172605)(496.10055513,340.50726336)
\curveto(496.09055456,340.60726031)(496.07555457,340.70726021)(496.05555513,340.80726336)
\curveto(496.0455546,340.83726008)(496.04055461,340.87226004)(496.04055513,340.91226336)
\curveto(496.0505546,340.95225996)(496.0505546,340.98725993)(496.04055513,341.01726336)
\lineto(496.04055513,341.15226336)
\curveto(496.04055461,341.20225971)(496.03555461,341.25225966)(496.02555513,341.30226336)
\curveto(496.01555463,341.35225956)(496.01055464,341.40725951)(496.01055513,341.46726336)
\curveto(496.01055464,341.53725938)(496.01555463,341.59225932)(496.02555513,341.63226336)
\curveto(496.03555461,341.68225923)(496.04055461,341.72725919)(496.04055513,341.76726336)
\lineto(496.04055513,341.91726336)
\curveto(496.0505546,341.96725895)(496.0505546,342.0122589)(496.04055513,342.05226336)
\curveto(496.04055461,342.10225881)(496.0505546,342.15225876)(496.07055513,342.20226336)
\curveto(496.09055456,342.3122586)(496.10555454,342.4172585)(496.11555513,342.51726336)
\curveto(496.13555451,342.6172583)(496.16055449,342.7172582)(496.19055513,342.81726336)
\curveto(496.23055442,342.93725798)(496.26555438,343.05225786)(496.29555513,343.16226336)
\curveto(496.32555432,343.27225764)(496.36555428,343.38225753)(496.41555513,343.49226336)
\curveto(496.55555409,343.79225712)(496.73055392,344.07725684)(496.94055513,344.34726336)
\curveto(496.96055369,344.37725654)(496.98555366,344.40225651)(497.01555513,344.42226336)
\curveto(497.05555359,344.45225646)(497.08555356,344.48225643)(497.10555513,344.51226336)
\curveto(497.1455535,344.56225635)(497.18555346,344.60725631)(497.22555513,344.64726336)
\curveto(497.26555338,344.68725623)(497.31055334,344.72725619)(497.36055513,344.76726336)
\curveto(497.40055325,344.78725613)(497.43555321,344.8122561)(497.46555513,344.84226336)
\curveto(497.49555315,344.88225603)(497.53055312,344.912256)(497.57055513,344.93226336)
\curveto(497.82055283,345.10225581)(498.11055254,345.24225567)(498.44055513,345.35226336)
\curveto(498.51055214,345.37225554)(498.58055207,345.38725553)(498.65055513,345.39726336)
\curveto(498.73055192,345.40725551)(498.81055184,345.42225549)(498.89055513,345.44226336)
\curveto(498.96055169,345.46225545)(499.0505516,345.47225544)(499.16055513,345.47226336)
\curveto(499.27055138,345.48225543)(499.38055127,345.48725543)(499.49055513,345.48726336)
\curveto(499.60055105,345.48725543)(499.70555094,345.48225543)(499.80555513,345.47226336)
\curveto(499.91555073,345.46225545)(500.00555064,345.44725547)(500.07555513,345.42726336)
\curveto(500.22555042,345.37725554)(500.37055028,345.33225558)(500.51055513,345.29226336)
\curveto(500.65055,345.25225566)(500.78054987,345.19725572)(500.90055513,345.12726336)
\curveto(500.97054968,345.07725584)(501.03554961,345.02725589)(501.09555513,344.97726336)
\curveto(501.15554949,344.93725598)(501.22054943,344.89225602)(501.29055513,344.84226336)
\curveto(501.33054932,344.8122561)(501.38554926,344.77225614)(501.45555513,344.72226336)
\curveto(501.53554911,344.67225624)(501.61054904,344.67225624)(501.68055513,344.72226336)
\curveto(501.72054893,344.74225617)(501.74054891,344.77725614)(501.74055513,344.82726336)
\curveto(501.74054891,344.87725604)(501.7505489,344.92725599)(501.77055513,344.97726336)
\lineto(501.77055513,345.12726336)
\curveto(501.78054887,345.15725576)(501.78554886,345.19225572)(501.78555513,345.23226336)
\lineto(501.78555513,345.35226336)
\lineto(501.78555513,347.39226336)
\curveto(501.78554886,347.50225341)(501.78054887,347.62225329)(501.77055513,347.75226336)
\curveto(501.77054888,347.89225302)(501.79554885,347.99725292)(501.84555513,348.06726336)
\curveto(501.88554876,348.14725277)(501.96054869,348.19725272)(502.07055513,348.21726336)
\curveto(502.09054856,348.22725269)(502.11054854,348.22725269)(502.13055513,348.21726336)
\curveto(502.1505485,348.2172527)(502.17054848,348.22225269)(502.19055513,348.23226336)
\lineto(503.25555513,348.23226336)
\curveto(503.37554727,348.23225268)(503.48554716,348.22725269)(503.58555513,348.21726336)
\curveto(503.68554696,348.20725271)(503.76054689,348.16725275)(503.81055513,348.09726336)
\curveto(503.86054679,348.0172529)(503.88554676,347.912253)(503.88555513,347.78226336)
\lineto(503.88555513,347.42226336)
\lineto(503.88555513,338.40726336)
\moveto(501.84555513,341.34726336)
\curveto(501.85554879,341.38725953)(501.85554879,341.42725949)(501.84555513,341.46726336)
\lineto(501.84555513,341.60226336)
\curveto(501.8455488,341.70225921)(501.84054881,341.80225911)(501.83055513,341.90226336)
\curveto(501.82054883,342.00225891)(501.80554884,342.09225882)(501.78555513,342.17226336)
\curveto(501.76554888,342.28225863)(501.7455489,342.38225853)(501.72555513,342.47226336)
\curveto(501.71554893,342.56225835)(501.69054896,342.64725827)(501.65055513,342.72726336)
\curveto(501.51054914,343.08725783)(501.30554934,343.37225754)(501.03555513,343.58226336)
\curveto(500.77554987,343.79225712)(500.39555025,343.89725702)(499.89555513,343.89726336)
\curveto(499.83555081,343.89725702)(499.75555089,343.88725703)(499.65555513,343.86726336)
\curveto(499.57555107,343.84725707)(499.50055115,343.82725709)(499.43055513,343.80726336)
\curveto(499.37055128,343.79725712)(499.31055134,343.77725714)(499.25055513,343.74726336)
\curveto(498.98055167,343.63725728)(498.77055188,343.46725745)(498.62055513,343.23726336)
\curveto(498.47055218,343.00725791)(498.3505523,342.74725817)(498.26055513,342.45726336)
\curveto(498.23055242,342.35725856)(498.21055244,342.25725866)(498.20055513,342.15726336)
\curveto(498.19055246,342.05725886)(498.17055248,341.95225896)(498.14055513,341.84226336)
\lineto(498.14055513,341.63226336)
\curveto(498.12055253,341.54225937)(498.11555253,341.4172595)(498.12555513,341.25726336)
\curveto(498.13555251,341.10725981)(498.1505525,340.99725992)(498.17055513,340.92726336)
\lineto(498.17055513,340.83726336)
\curveto(498.18055247,340.8172601)(498.18555246,340.79726012)(498.18555513,340.77726336)
\curveto(498.20555244,340.69726022)(498.22055243,340.62226029)(498.23055513,340.55226336)
\curveto(498.2505524,340.48226043)(498.27055238,340.40726051)(498.29055513,340.32726336)
\curveto(498.46055219,339.80726111)(498.7505519,339.42226149)(499.16055513,339.17226336)
\curveto(499.29055136,339.08226183)(499.47055118,339.0122619)(499.70055513,338.96226336)
\curveto(499.74055091,338.95226196)(499.80055085,338.94726197)(499.88055513,338.94726336)
\curveto(499.91055074,338.93726198)(499.95555069,338.92726199)(500.01555513,338.91726336)
\curveto(500.08555056,338.917262)(500.14055051,338.92226199)(500.18055513,338.93226336)
\curveto(500.26055039,338.95226196)(500.34055031,338.96726195)(500.42055513,338.97726336)
\curveto(500.50055015,338.98726193)(500.58055007,339.00726191)(500.66055513,339.03726336)
\curveto(500.91054974,339.14726177)(501.11054954,339.28726163)(501.26055513,339.45726336)
\curveto(501.41054924,339.62726129)(501.54054911,339.84226107)(501.65055513,340.10226336)
\curveto(501.69054896,340.19226072)(501.72054893,340.28226063)(501.74055513,340.37226336)
\curveto(501.76054889,340.47226044)(501.78054887,340.57726034)(501.80055513,340.68726336)
\curveto(501.81054884,340.73726018)(501.81054884,340.78226013)(501.80055513,340.82226336)
\curveto(501.80054885,340.87226004)(501.81054884,340.92225999)(501.83055513,340.97226336)
\curveto(501.84054881,341.00225991)(501.8455488,341.03725988)(501.84555513,341.07726336)
\lineto(501.84555513,341.21226336)
\lineto(501.84555513,341.34726336)
}
}
{
\newrgbcolor{curcolor}{0 0 0}
\pscustom[linestyle=none,fillstyle=solid,fillcolor=curcolor]
{
\newpath
\moveto(512.83047701,341.49726336)
\curveto(512.85046884,341.4172595)(512.85046884,341.32725959)(512.83047701,341.22726336)
\curveto(512.81046888,341.12725979)(512.77546892,341.06225985)(512.72547701,341.03226336)
\curveto(512.67546902,340.99225992)(512.60046909,340.96225995)(512.50047701,340.94226336)
\curveto(512.41046928,340.93225998)(512.30546939,340.92225999)(512.18547701,340.91226336)
\lineto(511.84047701,340.91226336)
\curveto(511.73046996,340.92225999)(511.63047006,340.92725999)(511.54047701,340.92726336)
\lineto(507.88047701,340.92726336)
\lineto(507.67047701,340.92726336)
\curveto(507.61047408,340.92725999)(507.55547414,340.91726)(507.50547701,340.89726336)
\curveto(507.42547427,340.85726006)(507.37547432,340.8172601)(507.35547701,340.77726336)
\curveto(507.33547436,340.75726016)(507.31547438,340.7172602)(507.29547701,340.65726336)
\curveto(507.27547442,340.60726031)(507.27047442,340.55726036)(507.28047701,340.50726336)
\curveto(507.30047439,340.44726047)(507.31047438,340.38726053)(507.31047701,340.32726336)
\curveto(507.32047437,340.27726064)(507.33547436,340.22226069)(507.35547701,340.16226336)
\curveto(507.43547426,339.92226099)(507.53047416,339.72226119)(507.64047701,339.56226336)
\curveto(507.76047393,339.4122615)(507.92047377,339.27726164)(508.12047701,339.15726336)
\curveto(508.20047349,339.10726181)(508.28047341,339.07226184)(508.36047701,339.05226336)
\curveto(508.45047324,339.04226187)(508.54047315,339.02226189)(508.63047701,338.99226336)
\curveto(508.71047298,338.97226194)(508.82047287,338.95726196)(508.96047701,338.94726336)
\curveto(509.10047259,338.93726198)(509.22047247,338.94226197)(509.32047701,338.96226336)
\lineto(509.45547701,338.96226336)
\curveto(509.55547214,338.98226193)(509.64547205,339.00226191)(509.72547701,339.02226336)
\curveto(509.81547188,339.05226186)(509.90047179,339.08226183)(509.98047701,339.11226336)
\curveto(510.08047161,339.16226175)(510.1904715,339.22726169)(510.31047701,339.30726336)
\curveto(510.44047125,339.38726153)(510.53547116,339.46726145)(510.59547701,339.54726336)
\curveto(510.64547105,339.6172613)(510.695471,339.68226123)(510.74547701,339.74226336)
\curveto(510.80547089,339.8122611)(510.87547082,339.86226105)(510.95547701,339.89226336)
\curveto(511.05547064,339.94226097)(511.18047051,339.96226095)(511.33047701,339.95226336)
\lineto(511.76547701,339.95226336)
\lineto(511.94547701,339.95226336)
\curveto(512.01546968,339.96226095)(512.07546962,339.95726096)(512.12547701,339.93726336)
\lineto(512.27547701,339.93726336)
\curveto(512.37546932,339.917261)(512.44546925,339.89226102)(512.48547701,339.86226336)
\curveto(512.52546917,339.84226107)(512.54546915,339.79726112)(512.54547701,339.72726336)
\curveto(512.55546914,339.65726126)(512.55046914,339.59726132)(512.53047701,339.54726336)
\curveto(512.48046921,339.40726151)(512.42546927,339.28226163)(512.36547701,339.17226336)
\curveto(512.30546939,339.06226185)(512.23546946,338.95226196)(512.15547701,338.84226336)
\curveto(511.93546976,338.5122624)(511.68547001,338.24726267)(511.40547701,338.04726336)
\curveto(511.12547057,337.84726307)(510.77547092,337.67726324)(510.35547701,337.53726336)
\curveto(510.24547145,337.49726342)(510.13547156,337.47226344)(510.02547701,337.46226336)
\curveto(509.91547178,337.45226346)(509.80047189,337.43226348)(509.68047701,337.40226336)
\curveto(509.64047205,337.39226352)(509.5954721,337.39226352)(509.54547701,337.40226336)
\curveto(509.50547219,337.40226351)(509.46547223,337.39726352)(509.42547701,337.38726336)
\lineto(509.26047701,337.38726336)
\curveto(509.21047248,337.36726355)(509.15047254,337.36226355)(509.08047701,337.37226336)
\curveto(509.02047267,337.37226354)(508.96547273,337.37726354)(508.91547701,337.38726336)
\curveto(508.83547286,337.39726352)(508.76547293,337.39726352)(508.70547701,337.38726336)
\curveto(508.64547305,337.37726354)(508.58047311,337.38226353)(508.51047701,337.40226336)
\curveto(508.46047323,337.42226349)(508.40547329,337.43226348)(508.34547701,337.43226336)
\curveto(508.28547341,337.43226348)(508.23047346,337.44226347)(508.18047701,337.46226336)
\curveto(508.07047362,337.48226343)(507.96047373,337.50726341)(507.85047701,337.53726336)
\curveto(507.74047395,337.55726336)(507.64047405,337.59226332)(507.55047701,337.64226336)
\curveto(507.44047425,337.68226323)(507.33547436,337.7172632)(507.23547701,337.74726336)
\curveto(507.14547455,337.78726313)(507.06047463,337.83226308)(506.98047701,337.88226336)
\curveto(506.66047503,338.08226283)(506.37547532,338.3122626)(506.12547701,338.57226336)
\curveto(505.87547582,338.84226207)(505.67047602,339.15226176)(505.51047701,339.50226336)
\curveto(505.46047623,339.6122613)(505.42047627,339.72226119)(505.39047701,339.83226336)
\curveto(505.36047633,339.95226096)(505.32047637,340.07226084)(505.27047701,340.19226336)
\curveto(505.26047643,340.23226068)(505.25547644,340.26726065)(505.25547701,340.29726336)
\curveto(505.25547644,340.33726058)(505.25047644,340.37726054)(505.24047701,340.41726336)
\curveto(505.20047649,340.53726038)(505.17547652,340.66726025)(505.16547701,340.80726336)
\lineto(505.13547701,341.22726336)
\curveto(505.13547656,341.27725964)(505.13047656,341.33225958)(505.12047701,341.39226336)
\curveto(505.12047657,341.45225946)(505.12547657,341.50725941)(505.13547701,341.55726336)
\lineto(505.13547701,341.73726336)
\lineto(505.18047701,342.09726336)
\curveto(505.22047647,342.26725865)(505.25547644,342.43225848)(505.28547701,342.59226336)
\curveto(505.31547638,342.75225816)(505.36047633,342.90225801)(505.42047701,343.04226336)
\curveto(505.85047584,344.08225683)(506.58047511,344.8172561)(507.61047701,345.24726336)
\curveto(507.75047394,345.30725561)(507.8904738,345.34725557)(508.03047701,345.36726336)
\curveto(508.18047351,345.39725552)(508.33547336,345.43225548)(508.49547701,345.47226336)
\curveto(508.57547312,345.48225543)(508.65047304,345.48725543)(508.72047701,345.48726336)
\curveto(508.7904729,345.48725543)(508.86547283,345.49225542)(508.94547701,345.50226336)
\curveto(509.45547224,345.5122554)(509.8904718,345.45225546)(510.25047701,345.32226336)
\curveto(510.62047107,345.20225571)(510.95047074,345.04225587)(511.24047701,344.84226336)
\curveto(511.33047036,344.78225613)(511.42047027,344.7122562)(511.51047701,344.63226336)
\curveto(511.60047009,344.56225635)(511.68047001,344.48725643)(511.75047701,344.40726336)
\curveto(511.78046991,344.35725656)(511.82046987,344.3172566)(511.87047701,344.28726336)
\curveto(511.95046974,344.17725674)(512.02546967,344.06225685)(512.09547701,343.94226336)
\curveto(512.16546953,343.83225708)(512.24046945,343.7172572)(512.32047701,343.59726336)
\curveto(512.37046932,343.50725741)(512.41046928,343.4122575)(512.44047701,343.31226336)
\curveto(512.48046921,343.22225769)(512.52046917,343.12225779)(512.56047701,343.01226336)
\curveto(512.61046908,342.88225803)(512.65046904,342.74725817)(512.68047701,342.60726336)
\curveto(512.71046898,342.46725845)(512.74546895,342.32725859)(512.78547701,342.18726336)
\curveto(512.80546889,342.10725881)(512.81046888,342.0172589)(512.80047701,341.91726336)
\curveto(512.80046889,341.82725909)(512.81046888,341.74225917)(512.83047701,341.66226336)
\lineto(512.83047701,341.49726336)
\moveto(510.58047701,342.38226336)
\curveto(510.65047104,342.48225843)(510.65547104,342.60225831)(510.59547701,342.74226336)
\curveto(510.54547115,342.89225802)(510.50547119,343.00225791)(510.47547701,343.07226336)
\curveto(510.33547136,343.34225757)(510.15047154,343.54725737)(509.92047701,343.68726336)
\curveto(509.690472,343.83725708)(509.37047232,343.917257)(508.96047701,343.92726336)
\curveto(508.93047276,343.90725701)(508.8954728,343.90225701)(508.85547701,343.91226336)
\curveto(508.81547288,343.92225699)(508.78047291,343.92225699)(508.75047701,343.91226336)
\curveto(508.70047299,343.89225702)(508.64547305,343.87725704)(508.58547701,343.86726336)
\curveto(508.52547317,343.86725705)(508.47047322,343.85725706)(508.42047701,343.83726336)
\curveto(507.98047371,343.69725722)(507.65547404,343.42225749)(507.44547701,343.01226336)
\curveto(507.42547427,342.97225794)(507.40047429,342.917258)(507.37047701,342.84726336)
\curveto(507.35047434,342.78725813)(507.33547436,342.72225819)(507.32547701,342.65226336)
\curveto(507.31547438,342.59225832)(507.31547438,342.53225838)(507.32547701,342.47226336)
\curveto(507.34547435,342.4122585)(507.38047431,342.36225855)(507.43047701,342.32226336)
\curveto(507.51047418,342.27225864)(507.62047407,342.24725867)(507.76047701,342.24726336)
\lineto(508.16547701,342.24726336)
\lineto(509.83047701,342.24726336)
\lineto(510.26547701,342.24726336)
\curveto(510.42547127,342.25725866)(510.53047116,342.30225861)(510.58047701,342.38226336)
}
}
{
\newrgbcolor{curcolor}{0 0 0}
\pscustom[linestyle=none,fillstyle=solid,fillcolor=curcolor]
{
}
}
{
\newrgbcolor{curcolor}{0 0 0}
\pscustom[linestyle=none,fillstyle=solid,fillcolor=curcolor]
{
\newpath
\moveto(518.74891451,348.24726336)
\lineto(519.84391451,348.24726336)
\curveto(519.94391202,348.24725267)(520.03891193,348.24225267)(520.12891451,348.23226336)
\curveto(520.21891175,348.22225269)(520.28891168,348.19225272)(520.33891451,348.14226336)
\curveto(520.39891157,348.07225284)(520.42891154,347.97725294)(520.42891451,347.85726336)
\curveto(520.43891153,347.74725317)(520.44391152,347.63225328)(520.44391451,347.51226336)
\lineto(520.44391451,346.17726336)
\lineto(520.44391451,340.79226336)
\lineto(520.44391451,338.49726336)
\lineto(520.44391451,338.07726336)
\curveto(520.45391151,337.92726299)(520.43391153,337.8122631)(520.38391451,337.73226336)
\curveto(520.33391163,337.65226326)(520.24391172,337.59726332)(520.11391451,337.56726336)
\curveto(520.05391191,337.54726337)(519.98391198,337.54226337)(519.90391451,337.55226336)
\curveto(519.83391213,337.56226335)(519.7639122,337.56726335)(519.69391451,337.56726336)
\lineto(518.97391451,337.56726336)
\curveto(518.8639131,337.56726335)(518.7639132,337.57226334)(518.67391451,337.58226336)
\curveto(518.58391338,337.59226332)(518.50891346,337.62226329)(518.44891451,337.67226336)
\curveto(518.38891358,337.72226319)(518.35391361,337.79726312)(518.34391451,337.89726336)
\lineto(518.34391451,338.22726336)
\lineto(518.34391451,339.56226336)
\lineto(518.34391451,345.18726336)
\lineto(518.34391451,347.22726336)
\curveto(518.34391362,347.35725356)(518.33891363,347.5122534)(518.32891451,347.69226336)
\curveto(518.32891364,347.87225304)(518.35391361,348.00225291)(518.40391451,348.08226336)
\curveto(518.42391354,348.12225279)(518.44891352,348.15225276)(518.47891451,348.17226336)
\lineto(518.59891451,348.23226336)
\curveto(518.61891335,348.23225268)(518.64391332,348.23225268)(518.67391451,348.23226336)
\curveto(518.70391326,348.24225267)(518.72891324,348.24725267)(518.74891451,348.24726336)
}
}
{
\newrgbcolor{curcolor}{0 0 0}
\pscustom[linestyle=none,fillstyle=solid,fillcolor=curcolor]
{
\newpath
\moveto(529.87610201,341.73726336)
\curveto(529.89609344,341.67725924)(529.90609343,341.59225932)(529.90610201,341.48226336)
\curveto(529.90609343,341.37225954)(529.89609344,341.28725963)(529.87610201,341.22726336)
\lineto(529.87610201,341.07726336)
\curveto(529.85609348,340.99725992)(529.84609349,340.91726)(529.84610201,340.83726336)
\curveto(529.85609348,340.75726016)(529.85109348,340.67726024)(529.83110201,340.59726336)
\curveto(529.81109352,340.52726039)(529.79609354,340.46226045)(529.78610201,340.40226336)
\curveto(529.77609356,340.34226057)(529.76609357,340.27726064)(529.75610201,340.20726336)
\curveto(529.71609362,340.09726082)(529.68109365,339.98226093)(529.65110201,339.86226336)
\curveto(529.62109371,339.75226116)(529.58109375,339.64726127)(529.53110201,339.54726336)
\curveto(529.32109401,339.06726185)(529.04609429,338.67726224)(528.70610201,338.37726336)
\curveto(528.36609497,338.07726284)(527.95609538,337.82726309)(527.47610201,337.62726336)
\curveto(527.35609598,337.57726334)(527.2310961,337.54226337)(527.10110201,337.52226336)
\curveto(526.98109635,337.49226342)(526.85609648,337.46226345)(526.72610201,337.43226336)
\curveto(526.67609666,337.4122635)(526.62109671,337.40226351)(526.56110201,337.40226336)
\curveto(526.50109683,337.40226351)(526.44609689,337.39726352)(526.39610201,337.38726336)
\lineto(526.29110201,337.38726336)
\curveto(526.26109707,337.37726354)(526.2310971,337.37226354)(526.20110201,337.37226336)
\curveto(526.15109718,337.36226355)(526.07109726,337.35726356)(525.96110201,337.35726336)
\curveto(525.85109748,337.34726357)(525.76609757,337.35226356)(525.70610201,337.37226336)
\lineto(525.55610201,337.37226336)
\curveto(525.50609783,337.38226353)(525.45109788,337.38726353)(525.39110201,337.38726336)
\curveto(525.34109799,337.37726354)(525.29109804,337.38226353)(525.24110201,337.40226336)
\curveto(525.20109813,337.4122635)(525.16109817,337.4172635)(525.12110201,337.41726336)
\curveto(525.09109824,337.4172635)(525.05109828,337.42226349)(525.00110201,337.43226336)
\curveto(524.90109843,337.46226345)(524.80109853,337.48726343)(524.70110201,337.50726336)
\curveto(524.60109873,337.52726339)(524.50609883,337.55726336)(524.41610201,337.59726336)
\curveto(524.29609904,337.63726328)(524.18109915,337.67726324)(524.07110201,337.71726336)
\curveto(523.97109936,337.75726316)(523.86609947,337.80726311)(523.75610201,337.86726336)
\curveto(523.40609993,338.07726284)(523.10610023,338.32226259)(522.85610201,338.60226336)
\curveto(522.60610073,338.88226203)(522.39610094,339.2172617)(522.22610201,339.60726336)
\curveto(522.17610116,339.69726122)(522.1361012,339.79226112)(522.10610201,339.89226336)
\curveto(522.08610125,339.99226092)(522.06110127,340.09726082)(522.03110201,340.20726336)
\curveto(522.01110132,340.25726066)(522.00110133,340.30226061)(522.00110201,340.34226336)
\curveto(522.00110133,340.38226053)(521.99110134,340.42726049)(521.97110201,340.47726336)
\curveto(521.95110138,340.55726036)(521.94110139,340.63726028)(521.94110201,340.71726336)
\curveto(521.94110139,340.80726011)(521.9311014,340.89226002)(521.91110201,340.97226336)
\curveto(521.90110143,341.02225989)(521.89610144,341.06725985)(521.89610201,341.10726336)
\lineto(521.89610201,341.24226336)
\curveto(521.87610146,341.30225961)(521.86610147,341.38725953)(521.86610201,341.49726336)
\curveto(521.87610146,341.60725931)(521.89110144,341.69225922)(521.91110201,341.75226336)
\lineto(521.91110201,341.85726336)
\curveto(521.92110141,341.90725901)(521.92110141,341.95725896)(521.91110201,342.00726336)
\curveto(521.91110142,342.06725885)(521.92110141,342.12225879)(521.94110201,342.17226336)
\curveto(521.95110138,342.22225869)(521.95610138,342.26725865)(521.95610201,342.30726336)
\curveto(521.95610138,342.35725856)(521.96610137,342.40725851)(521.98610201,342.45726336)
\curveto(522.02610131,342.58725833)(522.06110127,342.7122582)(522.09110201,342.83226336)
\curveto(522.12110121,342.96225795)(522.16110117,343.08725783)(522.21110201,343.20726336)
\curveto(522.39110094,343.6172573)(522.60610073,343.95725696)(522.85610201,344.22726336)
\curveto(523.10610023,344.50725641)(523.41109992,344.76225615)(523.77110201,344.99226336)
\curveto(523.87109946,345.04225587)(523.97609936,345.08725583)(524.08610201,345.12726336)
\curveto(524.19609914,345.16725575)(524.30609903,345.2122557)(524.41610201,345.26226336)
\curveto(524.54609879,345.3122556)(524.68109865,345.34725557)(524.82110201,345.36726336)
\curveto(524.96109837,345.38725553)(525.10609823,345.4172555)(525.25610201,345.45726336)
\curveto(525.336098,345.46725545)(525.41109792,345.47225544)(525.48110201,345.47226336)
\curveto(525.55109778,345.47225544)(525.62109771,345.47725544)(525.69110201,345.48726336)
\curveto(526.27109706,345.49725542)(526.77109656,345.43725548)(527.19110201,345.30726336)
\curveto(527.62109571,345.17725574)(528.00109533,344.99725592)(528.33110201,344.76726336)
\curveto(528.44109489,344.68725623)(528.55109478,344.59725632)(528.66110201,344.49726336)
\curveto(528.78109455,344.40725651)(528.88109445,344.30725661)(528.96110201,344.19726336)
\curveto(529.04109429,344.09725682)(529.11109422,343.99725692)(529.17110201,343.89726336)
\curveto(529.24109409,343.79725712)(529.31109402,343.69225722)(529.38110201,343.58226336)
\curveto(529.45109388,343.47225744)(529.50609383,343.35225756)(529.54610201,343.22226336)
\curveto(529.58609375,343.10225781)(529.6310937,342.97225794)(529.68110201,342.83226336)
\curveto(529.71109362,342.75225816)(529.7360936,342.66725825)(529.75610201,342.57726336)
\lineto(529.81610201,342.30726336)
\curveto(529.82609351,342.26725865)(529.8310935,342.22725869)(529.83110201,342.18726336)
\curveto(529.8310935,342.14725877)(529.8360935,342.10725881)(529.84610201,342.06726336)
\curveto(529.86609347,342.0172589)(529.87109346,341.96225895)(529.86110201,341.90226336)
\curveto(529.85109348,341.84225907)(529.85609348,341.78725913)(529.87610201,341.73726336)
\moveto(527.77610201,341.19726336)
\curveto(527.78609555,341.24725967)(527.79109554,341.3172596)(527.79110201,341.40726336)
\curveto(527.79109554,341.50725941)(527.78609555,341.58225933)(527.77610201,341.63226336)
\lineto(527.77610201,341.75226336)
\curveto(527.75609558,341.80225911)(527.74609559,341.85725906)(527.74610201,341.91726336)
\curveto(527.74609559,341.97725894)(527.74109559,342.03225888)(527.73110201,342.08226336)
\curveto(527.7310956,342.12225879)(527.72609561,342.15225876)(527.71610201,342.17226336)
\lineto(527.65610201,342.41226336)
\curveto(527.64609569,342.50225841)(527.62609571,342.58725833)(527.59610201,342.66726336)
\curveto(527.48609585,342.92725799)(527.35609598,343.14725777)(527.20610201,343.32726336)
\curveto(527.05609628,343.5172574)(526.85609648,343.66725725)(526.60610201,343.77726336)
\curveto(526.54609679,343.79725712)(526.48609685,343.8122571)(526.42610201,343.82226336)
\curveto(526.36609697,343.84225707)(526.30109703,343.86225705)(526.23110201,343.88226336)
\curveto(526.15109718,343.90225701)(526.06609727,343.90725701)(525.97610201,343.89726336)
\lineto(525.70610201,343.89726336)
\curveto(525.67609766,343.87725704)(525.64109769,343.86725705)(525.60110201,343.86726336)
\curveto(525.56109777,343.87725704)(525.52609781,343.87725704)(525.49610201,343.86726336)
\lineto(525.28610201,343.80726336)
\curveto(525.22609811,343.79725712)(525.17109816,343.77725714)(525.12110201,343.74726336)
\curveto(524.87109846,343.63725728)(524.66609867,343.47725744)(524.50610201,343.26726336)
\curveto(524.35609898,343.06725785)(524.2360991,342.83225808)(524.14610201,342.56226336)
\curveto(524.11609922,342.46225845)(524.09109924,342.35725856)(524.07110201,342.24726336)
\curveto(524.06109927,342.13725878)(524.04609929,342.02725889)(524.02610201,341.91726336)
\curveto(524.01609932,341.86725905)(524.01109932,341.8172591)(524.01110201,341.76726336)
\lineto(524.01110201,341.61726336)
\curveto(523.99109934,341.54725937)(523.98109935,341.44225947)(523.98110201,341.30226336)
\curveto(523.99109934,341.16225975)(524.00609933,341.05725986)(524.02610201,340.98726336)
\lineto(524.02610201,340.85226336)
\curveto(524.04609929,340.77226014)(524.06109927,340.69226022)(524.07110201,340.61226336)
\curveto(524.08109925,340.54226037)(524.09609924,340.46726045)(524.11610201,340.38726336)
\curveto(524.21609912,340.08726083)(524.32109901,339.84226107)(524.43110201,339.65226336)
\curveto(524.55109878,339.47226144)(524.7360986,339.30726161)(524.98610201,339.15726336)
\curveto(525.05609828,339.10726181)(525.1310982,339.06726185)(525.21110201,339.03726336)
\curveto(525.30109803,339.00726191)(525.39109794,338.98226193)(525.48110201,338.96226336)
\curveto(525.52109781,338.95226196)(525.55609778,338.94726197)(525.58610201,338.94726336)
\curveto(525.61609772,338.95726196)(525.65109768,338.95726196)(525.69110201,338.94726336)
\lineto(525.81110201,338.91726336)
\curveto(525.86109747,338.917262)(525.90609743,338.92226199)(525.94610201,338.93226336)
\lineto(526.06610201,338.93226336)
\curveto(526.14609719,338.95226196)(526.22609711,338.96726195)(526.30610201,338.97726336)
\curveto(526.38609695,338.98726193)(526.46109687,339.00726191)(526.53110201,339.03726336)
\curveto(526.79109654,339.13726178)(527.00109633,339.27226164)(527.16110201,339.44226336)
\curveto(527.32109601,339.6122613)(527.45609588,339.82226109)(527.56610201,340.07226336)
\curveto(527.60609573,340.17226074)(527.6360957,340.27226064)(527.65610201,340.37226336)
\curveto(527.67609566,340.47226044)(527.70109563,340.57726034)(527.73110201,340.68726336)
\curveto(527.74109559,340.72726019)(527.74609559,340.76226015)(527.74610201,340.79226336)
\curveto(527.74609559,340.83226008)(527.75109558,340.87226004)(527.76110201,340.91226336)
\lineto(527.76110201,341.04726336)
\curveto(527.76109557,341.09725982)(527.76609557,341.14725977)(527.77610201,341.19726336)
}
}
{
\newrgbcolor{curcolor}{0 0 0}
\pscustom[linestyle=none,fillstyle=solid,fillcolor=curcolor]
{
\newpath
\moveto(534.24602388,345.50226336)
\curveto(534.99601938,345.52225539)(535.64601873,345.43725548)(536.19602388,345.24726336)
\curveto(536.75601762,345.06725585)(537.1810172,344.75225616)(537.47102388,344.30226336)
\curveto(537.54101684,344.19225672)(537.60101678,344.07725684)(537.65102388,343.95726336)
\curveto(537.71101667,343.84725707)(537.76101662,343.72225719)(537.80102388,343.58226336)
\curveto(537.82101656,343.52225739)(537.83101655,343.45725746)(537.83102388,343.38726336)
\curveto(537.83101655,343.3172576)(537.82101656,343.25725766)(537.80102388,343.20726336)
\curveto(537.76101662,343.14725777)(537.70601667,343.10725781)(537.63602388,343.08726336)
\curveto(537.58601679,343.06725785)(537.52601685,343.05725786)(537.45602388,343.05726336)
\lineto(537.24602388,343.05726336)
\lineto(536.58602388,343.05726336)
\curveto(536.51601786,343.05725786)(536.44601793,343.05225786)(536.37602388,343.04226336)
\curveto(536.30601807,343.04225787)(536.24101814,343.05225786)(536.18102388,343.07226336)
\curveto(536.0810183,343.09225782)(536.00601837,343.13225778)(535.95602388,343.19226336)
\curveto(535.90601847,343.25225766)(535.86101852,343.3122576)(535.82102388,343.37226336)
\lineto(535.70102388,343.58226336)
\curveto(535.67101871,343.66225725)(535.62101876,343.72725719)(535.55102388,343.77726336)
\curveto(535.45101893,343.85725706)(535.35101903,343.917257)(535.25102388,343.95726336)
\curveto(535.16101922,343.99725692)(535.04601933,344.03225688)(534.90602388,344.06226336)
\curveto(534.83601954,344.08225683)(534.73101965,344.09725682)(534.59102388,344.10726336)
\curveto(534.46101992,344.1172568)(534.36102002,344.1122568)(534.29102388,344.09226336)
\lineto(534.18602388,344.09226336)
\lineto(534.03602388,344.06226336)
\curveto(533.99602038,344.06225685)(533.95102043,344.05725686)(533.90102388,344.04726336)
\curveto(533.73102065,343.99725692)(533.59102079,343.92725699)(533.48102388,343.83726336)
\curveto(533.381021,343.75725716)(533.31102107,343.63225728)(533.27102388,343.46226336)
\curveto(533.25102113,343.39225752)(533.25102113,343.32725759)(533.27102388,343.26726336)
\curveto(533.29102109,343.20725771)(533.31102107,343.15725776)(533.33102388,343.11726336)
\curveto(533.40102098,342.99725792)(533.4810209,342.90225801)(533.57102388,342.83226336)
\curveto(533.67102071,342.76225815)(533.78602059,342.70225821)(533.91602388,342.65226336)
\curveto(534.10602027,342.57225834)(534.31102007,342.50225841)(534.53102388,342.44226336)
\lineto(535.22102388,342.29226336)
\curveto(535.46101892,342.25225866)(535.69101869,342.20225871)(535.91102388,342.14226336)
\curveto(536.14101824,342.09225882)(536.35601802,342.02725889)(536.55602388,341.94726336)
\curveto(536.64601773,341.90725901)(536.73101765,341.87225904)(536.81102388,341.84226336)
\curveto(536.90101748,341.82225909)(536.98601739,341.78725913)(537.06602388,341.73726336)
\curveto(537.25601712,341.6172593)(537.42601695,341.48725943)(537.57602388,341.34726336)
\curveto(537.73601664,341.20725971)(537.86101652,341.03225988)(537.95102388,340.82226336)
\curveto(537.9810164,340.75226016)(538.00601637,340.68226023)(538.02602388,340.61226336)
\curveto(538.04601633,340.54226037)(538.06601631,340.46726045)(538.08602388,340.38726336)
\curveto(538.09601628,340.32726059)(538.10101628,340.23226068)(538.10102388,340.10226336)
\curveto(538.11101627,339.98226093)(538.11101627,339.88726103)(538.10102388,339.81726336)
\lineto(538.10102388,339.74226336)
\curveto(538.0810163,339.68226123)(538.06601631,339.62226129)(538.05602388,339.56226336)
\curveto(538.05601632,339.5122614)(538.05101633,339.46226145)(538.04102388,339.41226336)
\curveto(537.97101641,339.1122618)(537.86101652,338.84726207)(537.71102388,338.61726336)
\curveto(537.55101683,338.37726254)(537.35601702,338.18226273)(537.12602388,338.03226336)
\curveto(536.89601748,337.88226303)(536.63601774,337.75226316)(536.34602388,337.64226336)
\curveto(536.23601814,337.59226332)(536.11601826,337.55726336)(535.98602388,337.53726336)
\curveto(535.86601851,337.5172634)(535.74601863,337.49226342)(535.62602388,337.46226336)
\curveto(535.53601884,337.44226347)(535.44101894,337.43226348)(535.34102388,337.43226336)
\curveto(535.25101913,337.42226349)(535.16101922,337.40726351)(535.07102388,337.38726336)
\lineto(534.80102388,337.38726336)
\curveto(534.74101964,337.36726355)(534.63601974,337.35726356)(534.48602388,337.35726336)
\curveto(534.34602003,337.35726356)(534.24602013,337.36726355)(534.18602388,337.38726336)
\curveto(534.15602022,337.38726353)(534.12102026,337.39226352)(534.08102388,337.40226336)
\lineto(533.97602388,337.40226336)
\curveto(533.85602052,337.42226349)(533.73602064,337.43726348)(533.61602388,337.44726336)
\curveto(533.49602088,337.45726346)(533.381021,337.47726344)(533.27102388,337.50726336)
\curveto(532.8810215,337.6172633)(532.53602184,337.74226317)(532.23602388,337.88226336)
\curveto(531.93602244,338.03226288)(531.6810227,338.25226266)(531.47102388,338.54226336)
\curveto(531.33102305,338.73226218)(531.21102317,338.95226196)(531.11102388,339.20226336)
\curveto(531.09102329,339.26226165)(531.07102331,339.34226157)(531.05102388,339.44226336)
\curveto(531.03102335,339.49226142)(531.01602336,339.56226135)(531.00602388,339.65226336)
\curveto(530.99602338,339.74226117)(531.00102338,339.8172611)(531.02102388,339.87726336)
\curveto(531.05102333,339.94726097)(531.10102328,339.99726092)(531.17102388,340.02726336)
\curveto(531.22102316,340.04726087)(531.2810231,340.05726086)(531.35102388,340.05726336)
\lineto(531.57602388,340.05726336)
\lineto(532.28102388,340.05726336)
\lineto(532.52102388,340.05726336)
\curveto(532.60102178,340.05726086)(532.67102171,340.04726087)(532.73102388,340.02726336)
\curveto(532.84102154,339.98726093)(532.91102147,339.92226099)(532.94102388,339.83226336)
\curveto(532.9810214,339.74226117)(533.02602135,339.64726127)(533.07602388,339.54726336)
\curveto(533.09602128,339.49726142)(533.13102125,339.43226148)(533.18102388,339.35226336)
\curveto(533.24102114,339.27226164)(533.29102109,339.22226169)(533.33102388,339.20226336)
\curveto(533.45102093,339.10226181)(533.56602081,339.02226189)(533.67602388,338.96226336)
\curveto(533.78602059,338.912262)(533.92602045,338.86226205)(534.09602388,338.81226336)
\curveto(534.14602023,338.79226212)(534.19602018,338.78226213)(534.24602388,338.78226336)
\curveto(534.29602008,338.79226212)(534.34602003,338.79226212)(534.39602388,338.78226336)
\curveto(534.4760199,338.76226215)(534.56101982,338.75226216)(534.65102388,338.75226336)
\curveto(534.75101963,338.76226215)(534.83601954,338.77726214)(534.90602388,338.79726336)
\curveto(534.95601942,338.80726211)(535.00101938,338.8122621)(535.04102388,338.81226336)
\curveto(535.09101929,338.8122621)(535.14101924,338.82226209)(535.19102388,338.84226336)
\curveto(535.33101905,338.89226202)(535.45601892,338.95226196)(535.56602388,339.02226336)
\curveto(535.68601869,339.09226182)(535.7810186,339.18226173)(535.85102388,339.29226336)
\curveto(535.90101848,339.37226154)(535.94101844,339.49726142)(535.97102388,339.66726336)
\curveto(535.99101839,339.73726118)(535.99101839,339.80226111)(535.97102388,339.86226336)
\curveto(535.95101843,339.92226099)(535.93101845,339.97226094)(535.91102388,340.01226336)
\curveto(535.84101854,340.15226076)(535.75101863,340.25726066)(535.64102388,340.32726336)
\curveto(535.54101884,340.39726052)(535.42101896,340.46226045)(535.28102388,340.52226336)
\curveto(535.09101929,340.60226031)(534.89101949,340.66726025)(534.68102388,340.71726336)
\curveto(534.47101991,340.76726015)(534.26102012,340.82226009)(534.05102388,340.88226336)
\curveto(533.97102041,340.90226001)(533.88602049,340.91726)(533.79602388,340.92726336)
\curveto(533.71602066,340.93725998)(533.63602074,340.95225996)(533.55602388,340.97226336)
\curveto(533.23602114,341.06225985)(532.93102145,341.14725977)(532.64102388,341.22726336)
\curveto(532.35102203,341.3172596)(532.08602229,341.44725947)(531.84602388,341.61726336)
\curveto(531.56602281,341.8172591)(531.36102302,342.08725883)(531.23102388,342.42726336)
\curveto(531.21102317,342.49725842)(531.19102319,342.59225832)(531.17102388,342.71226336)
\curveto(531.15102323,342.78225813)(531.13602324,342.86725805)(531.12602388,342.96726336)
\curveto(531.11602326,343.06725785)(531.12102326,343.15725776)(531.14102388,343.23726336)
\curveto(531.16102322,343.28725763)(531.16602321,343.32725759)(531.15602388,343.35726336)
\curveto(531.14602323,343.39725752)(531.15102323,343.44225747)(531.17102388,343.49226336)
\curveto(531.19102319,343.60225731)(531.21102317,343.70225721)(531.23102388,343.79226336)
\curveto(531.26102312,343.89225702)(531.29602308,343.98725693)(531.33602388,344.07726336)
\curveto(531.46602291,344.36725655)(531.64602273,344.60225631)(531.87602388,344.78226336)
\curveto(532.10602227,344.96225595)(532.36602201,345.10725581)(532.65602388,345.21726336)
\curveto(532.76602161,345.26725565)(532.8810215,345.30225561)(533.00102388,345.32226336)
\curveto(533.12102126,345.35225556)(533.24602113,345.38225553)(533.37602388,345.41226336)
\curveto(533.43602094,345.43225548)(533.49602088,345.44225547)(533.55602388,345.44226336)
\lineto(533.73602388,345.47226336)
\curveto(533.81602056,345.48225543)(533.90102048,345.48725543)(533.99102388,345.48726336)
\curveto(534.0810203,345.48725543)(534.16602021,345.49225542)(534.24602388,345.50226336)
}
}
{
\newrgbcolor{curcolor}{0 0 0}
\pscustom[linestyle=none,fillstyle=solid,fillcolor=curcolor]
{
}
}
{
\newrgbcolor{curcolor}{0 0 0}
\pscustom[linestyle=none,fillstyle=solid,fillcolor=curcolor]
{
\newpath
\moveto(547.06282076,345.50226336)
\curveto(547.8728156,345.52225539)(548.54781492,345.40225551)(549.08782076,345.14226336)
\curveto(549.63781383,344.88225603)(550.0728134,344.5122564)(550.39282076,344.03226336)
\curveto(550.55281292,343.79225712)(550.6728128,343.5172574)(550.75282076,343.20726336)
\curveto(550.7728127,343.15725776)(550.78781268,343.09225782)(550.79782076,343.01226336)
\curveto(550.81781265,342.93225798)(550.81781265,342.86225805)(550.79782076,342.80226336)
\curveto(550.75781271,342.69225822)(550.68781278,342.62725829)(550.58782076,342.60726336)
\curveto(550.48781298,342.59725832)(550.3678131,342.59225832)(550.22782076,342.59226336)
\lineto(549.44782076,342.59226336)
\lineto(549.16282076,342.59226336)
\curveto(549.0728144,342.59225832)(548.99781447,342.6122583)(548.93782076,342.65226336)
\curveto(548.85781461,342.69225822)(548.80281467,342.75225816)(548.77282076,342.83226336)
\curveto(548.74281473,342.92225799)(548.70281477,343.0122579)(548.65282076,343.10226336)
\curveto(548.59281488,343.2122577)(548.52781494,343.3122576)(548.45782076,343.40226336)
\curveto(548.38781508,343.49225742)(548.30781516,343.57225734)(548.21782076,343.64226336)
\curveto(548.07781539,343.73225718)(547.92281555,343.80225711)(547.75282076,343.85226336)
\curveto(547.69281578,343.87225704)(547.63281584,343.88225703)(547.57282076,343.88226336)
\curveto(547.51281596,343.88225703)(547.45781601,343.89225702)(547.40782076,343.91226336)
\lineto(547.25782076,343.91226336)
\curveto(547.05781641,343.912257)(546.89781657,343.89225702)(546.77782076,343.85226336)
\curveto(546.48781698,343.76225715)(546.25281722,343.62225729)(546.07282076,343.43226336)
\curveto(545.89281758,343.25225766)(545.74781772,343.03225788)(545.63782076,342.77226336)
\curveto(545.58781788,342.66225825)(545.54781792,342.54225837)(545.51782076,342.41226336)
\curveto(545.49781797,342.29225862)(545.472818,342.16225875)(545.44282076,342.02226336)
\curveto(545.43281804,341.98225893)(545.42781804,341.94225897)(545.42782076,341.90226336)
\curveto(545.42781804,341.86225905)(545.42281805,341.82225909)(545.41282076,341.78226336)
\curveto(545.39281808,341.68225923)(545.38281809,341.54225937)(545.38282076,341.36226336)
\curveto(545.39281808,341.18225973)(545.40781806,341.04225987)(545.42782076,340.94226336)
\curveto(545.42781804,340.86226005)(545.43281804,340.80726011)(545.44282076,340.77726336)
\curveto(545.46281801,340.70726021)(545.472818,340.63726028)(545.47282076,340.56726336)
\curveto(545.48281799,340.49726042)(545.49781797,340.42726049)(545.51782076,340.35726336)
\curveto(545.59781787,340.12726079)(545.69281778,339.917261)(545.80282076,339.72726336)
\curveto(545.91281756,339.53726138)(546.05281742,339.37726154)(546.22282076,339.24726336)
\curveto(546.26281721,339.2172617)(546.32281715,339.18226173)(546.40282076,339.14226336)
\curveto(546.51281696,339.07226184)(546.62281685,339.02726189)(546.73282076,339.00726336)
\curveto(546.85281662,338.98726193)(546.99781647,338.96726195)(547.16782076,338.94726336)
\lineto(547.25782076,338.94726336)
\curveto(547.29781617,338.94726197)(547.32781614,338.95226196)(547.34782076,338.96226336)
\lineto(547.48282076,338.96226336)
\curveto(547.55281592,338.98226193)(547.61781585,338.99726192)(547.67782076,339.00726336)
\curveto(547.74781572,339.02726189)(547.81281566,339.04726187)(547.87282076,339.06726336)
\curveto(548.1728153,339.19726172)(548.40281507,339.38726153)(548.56282076,339.63726336)
\curveto(548.60281487,339.68726123)(548.63781483,339.74226117)(548.66782076,339.80226336)
\curveto(548.69781477,339.87226104)(548.72281475,339.93226098)(548.74282076,339.98226336)
\curveto(548.78281469,340.09226082)(548.81781465,340.18726073)(548.84782076,340.26726336)
\curveto(548.87781459,340.35726056)(548.94781452,340.42726049)(549.05782076,340.47726336)
\curveto(549.14781432,340.5172604)(549.29281418,340.53226038)(549.49282076,340.52226336)
\lineto(549.98782076,340.52226336)
\lineto(550.19782076,340.52226336)
\curveto(550.27781319,340.53226038)(550.34281313,340.52726039)(550.39282076,340.50726336)
\lineto(550.51282076,340.50726336)
\lineto(550.63282076,340.47726336)
\curveto(550.6728128,340.47726044)(550.70281277,340.46726045)(550.72282076,340.44726336)
\curveto(550.7728127,340.40726051)(550.80281267,340.34726057)(550.81282076,340.26726336)
\curveto(550.83281264,340.19726072)(550.83281264,340.12226079)(550.81282076,340.04226336)
\curveto(550.72281275,339.7122612)(550.61281286,339.4172615)(550.48282076,339.15726336)
\curveto(550.0728134,338.38726253)(549.41781405,337.85226306)(548.51782076,337.55226336)
\curveto(548.41781505,337.52226339)(548.31281516,337.50226341)(548.20282076,337.49226336)
\curveto(548.09281538,337.47226344)(547.98281549,337.44726347)(547.87282076,337.41726336)
\curveto(547.81281566,337.40726351)(547.75281572,337.40226351)(547.69282076,337.40226336)
\curveto(547.63281584,337.40226351)(547.5728159,337.39726352)(547.51282076,337.38726336)
\lineto(547.34782076,337.38726336)
\curveto(547.29781617,337.36726355)(547.22281625,337.36226355)(547.12282076,337.37226336)
\curveto(547.02281645,337.37226354)(546.94781652,337.37726354)(546.89782076,337.38726336)
\curveto(546.81781665,337.40726351)(546.74281673,337.4172635)(546.67282076,337.41726336)
\curveto(546.61281686,337.40726351)(546.54781692,337.4122635)(546.47782076,337.43226336)
\lineto(546.32782076,337.46226336)
\curveto(546.27781719,337.46226345)(546.22781724,337.46726345)(546.17782076,337.47726336)
\curveto(546.0678174,337.50726341)(545.96281751,337.53726338)(545.86282076,337.56726336)
\curveto(545.76281771,337.59726332)(545.6678178,337.63226328)(545.57782076,337.67226336)
\curveto(545.10781836,337.87226304)(544.71281876,338.12726279)(544.39282076,338.43726336)
\curveto(544.0728194,338.75726216)(543.81281966,339.15226176)(543.61282076,339.62226336)
\curveto(543.56281991,339.7122612)(543.52281995,339.80726111)(543.49282076,339.90726336)
\lineto(543.40282076,340.23726336)
\curveto(543.39282008,340.27726064)(543.38782008,340.3122606)(543.38782076,340.34226336)
\curveto(543.38782008,340.38226053)(543.37782009,340.42726049)(543.35782076,340.47726336)
\curveto(543.33782013,340.54726037)(543.32782014,340.6172603)(543.32782076,340.68726336)
\curveto(543.32782014,340.76726015)(543.31782015,340.84226007)(543.29782076,340.91226336)
\lineto(543.29782076,341.16726336)
\curveto(543.27782019,341.2172597)(543.2678202,341.27225964)(543.26782076,341.33226336)
\curveto(543.2678202,341.40225951)(543.27782019,341.46225945)(543.29782076,341.51226336)
\curveto(543.30782016,341.56225935)(543.30782016,341.60725931)(543.29782076,341.64726336)
\curveto(543.28782018,341.68725923)(543.28782018,341.72725919)(543.29782076,341.76726336)
\curveto(543.31782015,341.83725908)(543.32282015,341.90225901)(543.31282076,341.96226336)
\curveto(543.31282016,342.02225889)(543.32282015,342.08225883)(543.34282076,342.14226336)
\curveto(543.39282008,342.32225859)(543.43282004,342.49225842)(543.46282076,342.65226336)
\curveto(543.49281998,342.82225809)(543.53781993,342.98725793)(543.59782076,343.14726336)
\curveto(543.81781965,343.65725726)(544.09281938,344.08225683)(544.42282076,344.42226336)
\curveto(544.76281871,344.76225615)(545.19281828,345.03725588)(545.71282076,345.24726336)
\curveto(545.85281762,345.30725561)(545.99781747,345.34725557)(546.14782076,345.36726336)
\curveto(546.29781717,345.39725552)(546.45281702,345.43225548)(546.61282076,345.47226336)
\curveto(546.69281678,345.48225543)(546.7678167,345.48725543)(546.83782076,345.48726336)
\curveto(546.90781656,345.48725543)(546.98281649,345.49225542)(547.06282076,345.50226336)
}
}
{
\newrgbcolor{curcolor}{0 0 0}
\pscustom[linestyle=none,fillstyle=solid,fillcolor=curcolor]
{
\newpath
\moveto(559.15610201,338.15226336)
\curveto(559.17609416,338.04226287)(559.18609415,337.93226298)(559.18610201,337.82226336)
\curveto(559.19609414,337.7122632)(559.14609419,337.63726328)(559.03610201,337.59726336)
\curveto(558.97609436,337.56726335)(558.90609443,337.55226336)(558.82610201,337.55226336)
\lineto(558.58610201,337.55226336)
\lineto(557.77610201,337.55226336)
\lineto(557.50610201,337.55226336)
\curveto(557.42609591,337.56226335)(557.36109597,337.58726333)(557.31110201,337.62726336)
\curveto(557.24109609,337.66726325)(557.18609615,337.72226319)(557.14610201,337.79226336)
\curveto(557.11609622,337.87226304)(557.07109626,337.93726298)(557.01110201,337.98726336)
\curveto(556.99109634,338.00726291)(556.96609637,338.02226289)(556.93610201,338.03226336)
\curveto(556.90609643,338.05226286)(556.86609647,338.05726286)(556.81610201,338.04726336)
\curveto(556.76609657,338.02726289)(556.71609662,338.00226291)(556.66610201,337.97226336)
\curveto(556.62609671,337.94226297)(556.58109675,337.917263)(556.53110201,337.89726336)
\curveto(556.48109685,337.85726306)(556.42609691,337.82226309)(556.36610201,337.79226336)
\lineto(556.18610201,337.70226336)
\curveto(556.05609728,337.64226327)(555.92109741,337.59226332)(555.78110201,337.55226336)
\curveto(555.64109769,337.52226339)(555.49609784,337.48726343)(555.34610201,337.44726336)
\curveto(555.27609806,337.42726349)(555.20609813,337.4172635)(555.13610201,337.41726336)
\curveto(555.07609826,337.40726351)(555.01109832,337.39726352)(554.94110201,337.38726336)
\lineto(554.85110201,337.38726336)
\curveto(554.82109851,337.37726354)(554.79109854,337.37226354)(554.76110201,337.37226336)
\lineto(554.59610201,337.37226336)
\curveto(554.49609884,337.35226356)(554.39609894,337.35226356)(554.29610201,337.37226336)
\lineto(554.16110201,337.37226336)
\curveto(554.09109924,337.39226352)(554.02109931,337.40226351)(553.95110201,337.40226336)
\curveto(553.89109944,337.39226352)(553.8310995,337.39726352)(553.77110201,337.41726336)
\curveto(553.67109966,337.43726348)(553.57609976,337.45726346)(553.48610201,337.47726336)
\curveto(553.39609994,337.48726343)(553.31110002,337.5122634)(553.23110201,337.55226336)
\curveto(552.94110039,337.66226325)(552.69110064,337.80226311)(552.48110201,337.97226336)
\curveto(552.28110105,338.15226276)(552.12110121,338.38726253)(552.00110201,338.67726336)
\curveto(551.97110136,338.74726217)(551.94110139,338.82226209)(551.91110201,338.90226336)
\curveto(551.89110144,338.98226193)(551.87110146,339.06726185)(551.85110201,339.15726336)
\curveto(551.8311015,339.20726171)(551.82110151,339.25726166)(551.82110201,339.30726336)
\curveto(551.8311015,339.35726156)(551.8311015,339.40726151)(551.82110201,339.45726336)
\curveto(551.81110152,339.48726143)(551.80110153,339.54726137)(551.79110201,339.63726336)
\curveto(551.79110154,339.73726118)(551.79610154,339.80726111)(551.80610201,339.84726336)
\curveto(551.82610151,339.94726097)(551.8361015,340.03226088)(551.83610201,340.10226336)
\lineto(551.92610201,340.43226336)
\curveto(551.95610138,340.55226036)(551.99610134,340.65726026)(552.04610201,340.74726336)
\curveto(552.21610112,341.03725988)(552.41110092,341.25725966)(552.63110201,341.40726336)
\curveto(552.85110048,341.55725936)(553.1311002,341.68725923)(553.47110201,341.79726336)
\curveto(553.60109973,341.84725907)(553.7360996,341.88225903)(553.87610201,341.90226336)
\curveto(554.01609932,341.92225899)(554.15609918,341.94725897)(554.29610201,341.97726336)
\curveto(554.37609896,341.99725892)(554.46109887,342.00725891)(554.55110201,342.00726336)
\curveto(554.64109869,342.0172589)(554.7310986,342.03225888)(554.82110201,342.05226336)
\curveto(554.89109844,342.07225884)(554.96109837,342.07725884)(555.03110201,342.06726336)
\curveto(555.10109823,342.06725885)(555.17609816,342.07725884)(555.25610201,342.09726336)
\curveto(555.32609801,342.1172588)(555.39609794,342.12725879)(555.46610201,342.12726336)
\curveto(555.5360978,342.12725879)(555.61109772,342.13725878)(555.69110201,342.15726336)
\curveto(555.90109743,342.20725871)(556.09109724,342.24725867)(556.26110201,342.27726336)
\curveto(556.44109689,342.3172586)(556.60109673,342.40725851)(556.74110201,342.54726336)
\curveto(556.8310965,342.63725828)(556.89109644,342.73725818)(556.92110201,342.84726336)
\curveto(556.9310964,342.87725804)(556.9310964,342.90225801)(556.92110201,342.92226336)
\curveto(556.92109641,342.94225797)(556.92609641,342.96225795)(556.93610201,342.98226336)
\curveto(556.94609639,343.00225791)(556.95109638,343.03225788)(556.95110201,343.07226336)
\lineto(556.95110201,343.16226336)
\lineto(556.92110201,343.28226336)
\curveto(556.92109641,343.32225759)(556.91609642,343.35725756)(556.90610201,343.38726336)
\curveto(556.80609653,343.68725723)(556.59609674,343.89225702)(556.27610201,344.00226336)
\curveto(556.18609715,344.03225688)(556.07609726,344.05225686)(555.94610201,344.06226336)
\curveto(555.82609751,344.08225683)(555.70109763,344.08725683)(555.57110201,344.07726336)
\curveto(555.44109789,344.07725684)(555.31609802,344.06725685)(555.19610201,344.04726336)
\curveto(555.07609826,344.02725689)(554.97109836,344.00225691)(554.88110201,343.97226336)
\curveto(554.82109851,343.95225696)(554.76109857,343.92225699)(554.70110201,343.88226336)
\curveto(554.65109868,343.85225706)(554.60109873,343.8172571)(554.55110201,343.77726336)
\curveto(554.50109883,343.73725718)(554.44609889,343.68225723)(554.38610201,343.61226336)
\curveto(554.336099,343.54225737)(554.30109903,343.47725744)(554.28110201,343.41726336)
\curveto(554.2310991,343.3172576)(554.18609915,343.22225769)(554.14610201,343.13226336)
\curveto(554.11609922,343.04225787)(554.04609929,342.98225793)(553.93610201,342.95226336)
\curveto(553.85609948,342.93225798)(553.77109956,342.92225799)(553.68110201,342.92226336)
\lineto(553.41110201,342.92226336)
\lineto(552.84110201,342.92226336)
\curveto(552.79110054,342.92225799)(552.74110059,342.917258)(552.69110201,342.90726336)
\curveto(552.64110069,342.90725801)(552.59610074,342.912258)(552.55610201,342.92226336)
\lineto(552.42110201,342.92226336)
\curveto(552.40110093,342.93225798)(552.37610096,342.93725798)(552.34610201,342.93726336)
\curveto(552.31610102,342.93725798)(552.29110104,342.94725797)(552.27110201,342.96726336)
\curveto(552.19110114,342.98725793)(552.1361012,343.05225786)(552.10610201,343.16226336)
\curveto(552.09610124,343.2122577)(552.09610124,343.26225765)(552.10610201,343.31226336)
\curveto(552.11610122,343.36225755)(552.12610121,343.40725751)(552.13610201,343.44726336)
\curveto(552.16610117,343.55725736)(552.19610114,343.65725726)(552.22610201,343.74726336)
\curveto(552.26610107,343.84725707)(552.31110102,343.93725698)(552.36110201,344.01726336)
\lineto(552.45110201,344.16726336)
\lineto(552.54110201,344.31726336)
\curveto(552.62110071,344.42725649)(552.72110061,344.53225638)(552.84110201,344.63226336)
\curveto(552.86110047,344.64225627)(552.89110044,344.66725625)(552.93110201,344.70726336)
\curveto(552.98110035,344.74725617)(553.02610031,344.78225613)(553.06610201,344.81226336)
\curveto(553.10610023,344.84225607)(553.15110018,344.87225604)(553.20110201,344.90226336)
\curveto(553.37109996,345.0122559)(553.55109978,345.09725582)(553.74110201,345.15726336)
\curveto(553.9310994,345.22725569)(554.12609921,345.29225562)(554.32610201,345.35226336)
\curveto(554.44609889,345.38225553)(554.57109876,345.40225551)(554.70110201,345.41226336)
\curveto(554.8310985,345.42225549)(554.96109837,345.44225547)(555.09110201,345.47226336)
\curveto(555.1310982,345.48225543)(555.19109814,345.48225543)(555.27110201,345.47226336)
\curveto(555.36109797,345.46225545)(555.41609792,345.46725545)(555.43610201,345.48726336)
\curveto(555.84609749,345.49725542)(556.2360971,345.48225543)(556.60610201,345.44226336)
\curveto(556.98609635,345.40225551)(557.32609601,345.32725559)(557.62610201,345.21726336)
\curveto(557.9360954,345.10725581)(558.20109513,344.95725596)(558.42110201,344.76726336)
\curveto(558.64109469,344.58725633)(558.81109452,344.35225656)(558.93110201,344.06226336)
\curveto(559.00109433,343.89225702)(559.04109429,343.69725722)(559.05110201,343.47726336)
\curveto(559.06109427,343.25725766)(559.06609427,343.03225788)(559.06610201,342.80226336)
\lineto(559.06610201,339.45726336)
\lineto(559.06610201,338.87226336)
\curveto(559.06609427,338.68226223)(559.08609425,338.50726241)(559.12610201,338.34726336)
\curveto(559.1360942,338.3172626)(559.14109419,338.28226263)(559.14110201,338.24226336)
\curveto(559.14109419,338.2122627)(559.14609419,338.18226273)(559.15610201,338.15226336)
\moveto(556.95110201,340.46226336)
\curveto(556.96109637,340.5122604)(556.96609637,340.56726035)(556.96610201,340.62726336)
\curveto(556.96609637,340.69726022)(556.96109637,340.75726016)(556.95110201,340.80726336)
\curveto(556.9310964,340.86726005)(556.92109641,340.92225999)(556.92110201,340.97226336)
\curveto(556.92109641,341.02225989)(556.90109643,341.06225985)(556.86110201,341.09226336)
\curveto(556.81109652,341.13225978)(556.7360966,341.15225976)(556.63610201,341.15226336)
\curveto(556.59609674,341.14225977)(556.56109677,341.13225978)(556.53110201,341.12226336)
\curveto(556.50109683,341.12225979)(556.46609687,341.1172598)(556.42610201,341.10726336)
\curveto(556.35609698,341.08725983)(556.28109705,341.07225984)(556.20110201,341.06226336)
\curveto(556.12109721,341.05225986)(556.04109729,341.03725988)(555.96110201,341.01726336)
\curveto(555.9310974,341.00725991)(555.88609745,341.00225991)(555.82610201,341.00226336)
\curveto(555.69609764,340.97225994)(555.56609777,340.95225996)(555.43610201,340.94226336)
\curveto(555.30609803,340.93225998)(555.18109815,340.90726001)(555.06110201,340.86726336)
\curveto(554.98109835,340.84726007)(554.90609843,340.82726009)(554.83610201,340.80726336)
\curveto(554.76609857,340.79726012)(554.69609864,340.77726014)(554.62610201,340.74726336)
\curveto(554.41609892,340.65726026)(554.2360991,340.52226039)(554.08610201,340.34226336)
\curveto(553.94609939,340.16226075)(553.89609944,339.912261)(553.93610201,339.59226336)
\curveto(553.95609938,339.42226149)(554.01109932,339.28226163)(554.10110201,339.17226336)
\curveto(554.17109916,339.06226185)(554.27609906,338.97226194)(554.41610201,338.90226336)
\curveto(554.55609878,338.84226207)(554.70609863,338.79726212)(554.86610201,338.76726336)
\curveto(555.0360983,338.73726218)(555.21109812,338.72726219)(555.39110201,338.73726336)
\curveto(555.58109775,338.75726216)(555.75609758,338.79226212)(555.91610201,338.84226336)
\curveto(556.17609716,338.92226199)(556.38109695,339.04726187)(556.53110201,339.21726336)
\curveto(556.68109665,339.39726152)(556.79609654,339.6172613)(556.87610201,339.87726336)
\curveto(556.89609644,339.94726097)(556.90609643,340.0172609)(556.90610201,340.08726336)
\curveto(556.91609642,340.16726075)(556.9310964,340.24726067)(556.95110201,340.32726336)
\lineto(556.95110201,340.46226336)
}
}
{
\newrgbcolor{curcolor}{0 0 0}
\pscustom[linestyle=none,fillstyle=solid,fillcolor=curcolor]
{
\newpath
\moveto(561.22938326,348.24726336)
\lineto(562.32438326,348.24726336)
\curveto(562.42438077,348.24725267)(562.51938068,348.24225267)(562.60938326,348.23226336)
\curveto(562.6993805,348.22225269)(562.76938043,348.19225272)(562.81938326,348.14226336)
\curveto(562.87938032,348.07225284)(562.90938029,347.97725294)(562.90938326,347.85726336)
\curveto(562.91938028,347.74725317)(562.92438027,347.63225328)(562.92438326,347.51226336)
\lineto(562.92438326,346.17726336)
\lineto(562.92438326,340.79226336)
\lineto(562.92438326,338.49726336)
\lineto(562.92438326,338.07726336)
\curveto(562.93438026,337.92726299)(562.91438028,337.8122631)(562.86438326,337.73226336)
\curveto(562.81438038,337.65226326)(562.72438047,337.59726332)(562.59438326,337.56726336)
\curveto(562.53438066,337.54726337)(562.46438073,337.54226337)(562.38438326,337.55226336)
\curveto(562.31438088,337.56226335)(562.24438095,337.56726335)(562.17438326,337.56726336)
\lineto(561.45438326,337.56726336)
\curveto(561.34438185,337.56726335)(561.24438195,337.57226334)(561.15438326,337.58226336)
\curveto(561.06438213,337.59226332)(560.98938221,337.62226329)(560.92938326,337.67226336)
\curveto(560.86938233,337.72226319)(560.83438236,337.79726312)(560.82438326,337.89726336)
\lineto(560.82438326,338.22726336)
\lineto(560.82438326,339.56226336)
\lineto(560.82438326,345.18726336)
\lineto(560.82438326,347.22726336)
\curveto(560.82438237,347.35725356)(560.81938238,347.5122534)(560.80938326,347.69226336)
\curveto(560.80938239,347.87225304)(560.83438236,348.00225291)(560.88438326,348.08226336)
\curveto(560.90438229,348.12225279)(560.92938227,348.15225276)(560.95938326,348.17226336)
\lineto(561.07938326,348.23226336)
\curveto(561.0993821,348.23225268)(561.12438207,348.23225268)(561.15438326,348.23226336)
\curveto(561.18438201,348.24225267)(561.20938199,348.24725267)(561.22938326,348.24726336)
}
}
{
\newrgbcolor{curcolor}{0 0 0}
\pscustom[linestyle=none,fillstyle=solid,fillcolor=curcolor]
{
\newpath
\moveto(566.68657076,348.14226336)
\curveto(566.75656781,348.06225285)(566.79156777,347.94225297)(566.79157076,347.78226336)
\lineto(566.79157076,347.31726336)
\lineto(566.79157076,346.91226336)
\curveto(566.79156777,346.77225414)(566.75656781,346.67725424)(566.68657076,346.62726336)
\curveto(566.62656794,346.57725434)(566.54656802,346.54725437)(566.44657076,346.53726336)
\curveto(566.35656821,346.52725439)(566.25656831,346.52225439)(566.14657076,346.52226336)
\lineto(565.30657076,346.52226336)
\curveto(565.19656937,346.52225439)(565.09656947,346.52725439)(565.00657076,346.53726336)
\curveto(564.92656964,346.54725437)(564.85656971,346.57725434)(564.79657076,346.62726336)
\curveto(564.75656981,346.65725426)(564.72656984,346.7122542)(564.70657076,346.79226336)
\curveto(564.69656987,346.88225403)(564.68656988,346.97725394)(564.67657076,347.07726336)
\lineto(564.67657076,347.40726336)
\curveto(564.68656988,347.5172534)(564.69156987,347.6122533)(564.69157076,347.69226336)
\lineto(564.69157076,347.90226336)
\curveto(564.70156986,347.97225294)(564.72156984,348.03225288)(564.75157076,348.08226336)
\curveto(564.77156979,348.12225279)(564.79656977,348.15225276)(564.82657076,348.17226336)
\lineto(564.94657076,348.23226336)
\curveto(564.9665696,348.23225268)(564.99156957,348.23225268)(565.02157076,348.23226336)
\curveto(565.05156951,348.24225267)(565.07656949,348.24725267)(565.09657076,348.24726336)
\lineto(566.19157076,348.24726336)
\curveto(566.29156827,348.24725267)(566.38656818,348.24225267)(566.47657076,348.23226336)
\curveto(566.566568,348.22225269)(566.63656793,348.19225272)(566.68657076,348.14226336)
\moveto(566.79157076,338.37726336)
\curveto(566.79156777,338.17726274)(566.78656778,338.00726291)(566.77657076,337.86726336)
\curveto(566.7665678,337.72726319)(566.67656789,337.63226328)(566.50657076,337.58226336)
\curveto(566.44656812,337.56226335)(566.38156818,337.55226336)(566.31157076,337.55226336)
\curveto(566.24156832,337.56226335)(566.1665684,337.56726335)(566.08657076,337.56726336)
\lineto(565.24657076,337.56726336)
\curveto(565.15656941,337.56726335)(565.0665695,337.57226334)(564.97657076,337.58226336)
\curveto(564.89656967,337.59226332)(564.83656973,337.62226329)(564.79657076,337.67226336)
\curveto(564.73656983,337.74226317)(564.70156986,337.82726309)(564.69157076,337.92726336)
\lineto(564.69157076,338.27226336)
\lineto(564.69157076,344.60226336)
\lineto(564.69157076,344.90226336)
\curveto(564.69156987,345.00225591)(564.71156985,345.08225583)(564.75157076,345.14226336)
\curveto(564.81156975,345.2122557)(564.89656967,345.25725566)(565.00657076,345.27726336)
\curveto(565.02656954,345.28725563)(565.05156951,345.28725563)(565.08157076,345.27726336)
\curveto(565.12156944,345.27725564)(565.15156941,345.28225563)(565.17157076,345.29226336)
\lineto(565.92157076,345.29226336)
\lineto(566.11657076,345.29226336)
\curveto(566.19656837,345.30225561)(566.2615683,345.30225561)(566.31157076,345.29226336)
\lineto(566.43157076,345.29226336)
\curveto(566.49156807,345.27225564)(566.54656802,345.25725566)(566.59657076,345.24726336)
\curveto(566.64656792,345.23725568)(566.68656788,345.20725571)(566.71657076,345.15726336)
\curveto(566.75656781,345.10725581)(566.77656779,345.03725588)(566.77657076,344.94726336)
\curveto(566.78656778,344.85725606)(566.79156777,344.76225615)(566.79157076,344.66226336)
\lineto(566.79157076,338.37726336)
}
}
{
\newrgbcolor{curcolor}{0 0 0}
\pscustom[linestyle=none,fillstyle=solid,fillcolor=curcolor]
{
\newpath
\moveto(571.42375826,348.24726336)
\curveto(571.51375442,348.24725267)(571.61375432,348.24725267)(571.72375826,348.24726336)
\curveto(571.84375409,348.24725267)(571.95875397,348.24225267)(572.06875826,348.23226336)
\curveto(572.18875374,348.22225269)(572.29375364,348.20225271)(572.38375826,348.17226336)
\curveto(572.47375346,348.15225276)(572.5337534,348.1172528)(572.56375826,348.06726336)
\curveto(572.62375331,347.98725293)(572.65375328,347.87225304)(572.65375826,347.72226336)
\lineto(572.65375826,347.31726336)
\curveto(572.65375328,347.2172537)(572.64875328,347.1172538)(572.63875826,347.01726336)
\curveto(572.63875329,346.917254)(572.61875331,346.84225407)(572.57875826,346.79226336)
\curveto(572.53875339,346.73225418)(572.48875344,346.69225422)(572.42875826,346.67226336)
\curveto(572.36875356,346.66225425)(572.29875363,346.65725426)(572.21875826,346.65726336)
\lineto(571.99375826,346.65726336)
\curveto(571.92375401,346.66725425)(571.85375408,346.66725425)(571.78375826,346.65726336)
\curveto(571.60375433,346.6172543)(571.46375447,346.56725435)(571.36375826,346.50726336)
\curveto(571.26375467,346.45725446)(571.18375475,346.34725457)(571.12375826,346.17726336)
\curveto(571.10375483,346.14725477)(571.09375484,346.1172548)(571.09375826,346.08726336)
\curveto(571.10375483,346.06725485)(571.10375483,346.04225487)(571.09375826,346.01226336)
\curveto(571.08375485,345.97225494)(571.07375486,345.912255)(571.06375826,345.83226336)
\curveto(571.05375488,345.75225516)(571.05375488,345.68725523)(571.06375826,345.63726336)
\curveto(571.08375485,345.56725535)(571.10875482,345.50725541)(571.13875826,345.45726336)
\curveto(571.16875476,345.40725551)(571.21375472,345.36725555)(571.27375826,345.33726336)
\curveto(571.37375456,345.28725563)(571.49375444,345.27225564)(571.63375826,345.29226336)
\curveto(571.77375416,345.3122556)(571.90375403,345.3122556)(572.02375826,345.29226336)
\curveto(572.07375386,345.28225563)(572.11375382,345.27725564)(572.14375826,345.27726336)
\curveto(572.18375375,345.28725563)(572.22375371,345.28725563)(572.26375826,345.27726336)
\curveto(572.35375358,345.23725568)(572.41875351,345.19225572)(572.45875826,345.14226336)
\curveto(572.47875345,345.1122558)(572.49375344,345.06225585)(572.50375826,344.99226336)
\curveto(572.51375342,344.93225598)(572.52375341,344.86225605)(572.53375826,344.78226336)
\curveto(572.54375339,344.7122562)(572.54375339,344.63725628)(572.53375826,344.55726336)
\curveto(572.5337534,344.48725643)(572.5287534,344.43225648)(572.51875826,344.39226336)
\curveto(572.50875342,344.35225656)(572.50875342,344.3122566)(572.51875826,344.27226336)
\curveto(572.5287534,344.24225667)(572.52375341,344.20725671)(572.50375826,344.16726336)
\curveto(572.48375345,344.04725687)(572.42375351,343.97225694)(572.32375826,343.94226336)
\curveto(572.24375369,343.90225701)(572.14875378,343.88225703)(572.03875826,343.88226336)
\curveto(571.928754,343.89225702)(571.81875411,343.89725702)(571.70875826,343.89726336)
\lineto(571.60375826,343.89726336)
\curveto(571.56375437,343.89725702)(571.5287544,343.89225702)(571.49875826,343.88226336)
\lineto(571.37875826,343.88226336)
\curveto(571.20875472,343.84225707)(571.10375483,343.73225718)(571.06375826,343.55226336)
\curveto(571.04375489,343.49225742)(571.03875489,343.42225749)(571.04875826,343.34226336)
\curveto(571.05875487,343.26225765)(571.06375487,343.18225773)(571.06375826,343.10226336)
\lineto(571.06375826,342.18726336)
\lineto(571.06375826,339.26226336)
\lineto(571.06375826,338.55726336)
\lineto(571.06375826,338.36226336)
\curveto(571.07375486,338.30226261)(571.06875486,338.24726267)(571.04875826,338.19726336)
\lineto(571.04875826,338.03226336)
\curveto(571.04875488,337.87226304)(571.02375491,337.75726316)(570.97375826,337.68726336)
\curveto(570.95375498,337.65726326)(570.91875501,337.63226328)(570.86875826,337.61226336)
\curveto(570.81875511,337.60226331)(570.76875516,337.58726333)(570.71875826,337.56726336)
\lineto(570.64375826,337.56726336)
\curveto(570.59375534,337.55726336)(570.53875539,337.55226336)(570.47875826,337.55226336)
\curveto(570.41875551,337.56226335)(570.36375557,337.56726335)(570.31375826,337.56726336)
\lineto(569.65375826,337.56726336)
\curveto(569.58375635,337.56726335)(569.50875642,337.56226335)(569.42875826,337.55226336)
\curveto(569.35875657,337.55226336)(569.29875663,337.56226335)(569.24875826,337.58226336)
\curveto(569.1287568,337.6122633)(569.04875688,337.66226325)(569.00875826,337.73226336)
\curveto(568.97875695,337.78226313)(568.95875697,337.84726307)(568.94875826,337.92726336)
\lineto(568.94875826,338.16726336)
\lineto(568.94875826,338.94726336)
\lineto(568.94875826,343.14726336)
\curveto(568.94875698,343.3172576)(568.93875699,343.46225745)(568.91875826,343.58226336)
\curveto(568.89875703,343.7122572)(568.8287571,343.80225711)(568.70875826,343.85226336)
\curveto(568.59875733,343.90225701)(568.46375747,343.912257)(568.30375826,343.88226336)
\curveto(568.14375779,343.86225705)(568.00875792,343.87725704)(567.89875826,343.92726336)
\curveto(567.78875814,343.97725694)(567.71875821,344.06225685)(567.68875826,344.18226336)
\curveto(567.66875826,344.23225668)(567.66375827,344.29225662)(567.67375826,344.36226336)
\lineto(567.67375826,344.57226336)
\curveto(567.67375826,344.75225616)(567.68375825,344.90225601)(567.70375826,345.02226336)
\curveto(567.72375821,345.14225577)(567.80875812,345.22725569)(567.95875826,345.27726336)
\curveto(568.03875789,345.29725562)(568.12375781,345.30725561)(568.21375826,345.30726336)
\lineto(568.46875826,345.30726336)
\curveto(568.55875737,345.30725561)(568.63875729,345.3122556)(568.70875826,345.32226336)
\curveto(568.77875715,345.34225557)(568.8337571,345.38225553)(568.87375826,345.44226336)
\curveto(568.94375699,345.54225537)(568.96875696,345.66725525)(568.94875826,345.81726336)
\curveto(568.93875699,345.97725494)(568.94875698,346.12725479)(568.97875826,346.26726336)
\curveto(568.98875694,346.30725461)(568.99375694,346.34725457)(568.99375826,346.38726336)
\curveto(569.00375693,346.42725449)(569.01375692,346.47225444)(569.02375826,346.52226336)
\curveto(569.06375687,346.66225425)(569.10375683,346.78725413)(569.14375826,346.89726336)
\curveto(569.18375675,347.0172539)(569.23875669,347.12725379)(569.30875826,347.22726336)
\curveto(569.44875648,347.46725345)(569.6337563,347.65725326)(569.86375826,347.79726336)
\curveto(570.09375584,347.94725297)(570.35375558,348.06225285)(570.64375826,348.14226336)
\curveto(570.72375521,348.17225274)(570.80875512,348.18725273)(570.89875826,348.18726336)
\curveto(570.98875494,348.19725272)(571.07875485,348.2122527)(571.16875826,348.23226336)
\curveto(571.19875473,348.24225267)(571.24375469,348.24225267)(571.30375826,348.23226336)
\curveto(571.36375457,348.22225269)(571.40375453,348.22725269)(571.42375826,348.24726336)
\moveto(575.84875826,348.14226336)
\curveto(575.79875013,348.19225272)(575.7287502,348.22225269)(575.63875826,348.23226336)
\curveto(575.54875038,348.24225267)(575.45375048,348.24725267)(575.35375826,348.24726336)
\lineto(574.25875826,348.24726336)
\curveto(574.23875169,348.24725267)(574.21375172,348.24225267)(574.18375826,348.23226336)
\curveto(574.15375178,348.23225268)(574.1287518,348.23225268)(574.10875826,348.23226336)
\lineto(573.98875826,348.17226336)
\curveto(573.95875197,348.15225276)(573.933752,348.12225279)(573.91375826,348.08226336)
\curveto(573.88375205,348.03225288)(573.86375207,347.97225294)(573.85375826,347.90226336)
\lineto(573.85375826,347.69226336)
\curveto(573.85375208,347.6122533)(573.84875208,347.5172534)(573.83875826,347.40726336)
\lineto(573.83875826,347.07726336)
\curveto(573.84875208,346.97725394)(573.85875207,346.88225403)(573.86875826,346.79226336)
\curveto(573.88875204,346.7122542)(573.91875201,346.65725426)(573.95875826,346.62726336)
\curveto(574.01875191,346.57725434)(574.08875184,346.54725437)(574.16875826,346.53726336)
\curveto(574.25875167,346.52725439)(574.35875157,346.52225439)(574.46875826,346.52226336)
\lineto(575.30875826,346.52226336)
\curveto(575.41875051,346.52225439)(575.51875041,346.52725439)(575.60875826,346.53726336)
\curveto(575.70875022,346.54725437)(575.78875014,346.57725434)(575.84875826,346.62726336)
\curveto(575.91875001,346.67725424)(575.95374998,346.77225414)(575.95375826,346.91226336)
\lineto(575.95375826,347.31726336)
\lineto(575.95375826,347.78226336)
\curveto(575.95374998,347.94225297)(575.91875001,348.06225285)(575.84875826,348.14226336)
\moveto(575.95375826,344.66226336)
\curveto(575.95374998,344.76225615)(575.94874998,344.85725606)(575.93875826,344.94726336)
\curveto(575.93874999,345.03725588)(575.91875001,345.10725581)(575.87875826,345.15726336)
\curveto(575.84875008,345.20725571)(575.80875012,345.23725568)(575.75875826,345.24726336)
\curveto(575.70875022,345.25725566)(575.65375028,345.27225564)(575.59375826,345.29226336)
\lineto(575.47375826,345.29226336)
\curveto(575.42375051,345.30225561)(575.35875057,345.30225561)(575.27875826,345.29226336)
\lineto(575.08375826,345.29226336)
\lineto(574.33375826,345.29226336)
\curveto(574.31375162,345.28225563)(574.28375165,345.27725564)(574.24375826,345.27726336)
\curveto(574.21375172,345.28725563)(574.18875174,345.28725563)(574.16875826,345.27726336)
\curveto(574.05875187,345.25725566)(573.97375196,345.2122557)(573.91375826,345.14226336)
\curveto(573.87375206,345.08225583)(573.85375208,345.00225591)(573.85375826,344.90226336)
\lineto(573.85375826,344.60226336)
\lineto(573.85375826,338.27226336)
\lineto(573.85375826,337.92726336)
\curveto(573.86375207,337.82726309)(573.89875203,337.74226317)(573.95875826,337.67226336)
\curveto(573.99875193,337.62226329)(574.05875187,337.59226332)(574.13875826,337.58226336)
\curveto(574.2287517,337.57226334)(574.31875161,337.56726335)(574.40875826,337.56726336)
\lineto(575.24875826,337.56726336)
\curveto(575.3287506,337.56726335)(575.40375053,337.56226335)(575.47375826,337.55226336)
\curveto(575.54375039,337.55226336)(575.60875032,337.56226335)(575.66875826,337.58226336)
\curveto(575.83875009,337.63226328)(575.92875,337.72726319)(575.93875826,337.86726336)
\curveto(575.94874998,338.00726291)(575.95374998,338.17726274)(575.95375826,338.37726336)
\lineto(575.95375826,344.66226336)
}
}
{
\newrgbcolor{curcolor}{0 0 0}
\pscustom[linestyle=none,fillstyle=solid,fillcolor=curcolor]
{
\newpath
\moveto(581.19368013,345.50226336)
\curveto(582.00367497,345.52225539)(582.6786743,345.40225551)(583.21868013,345.14226336)
\curveto(583.76867321,344.88225603)(584.20367277,344.5122564)(584.52368013,344.03226336)
\curveto(584.68367229,343.79225712)(584.80367217,343.5172574)(584.88368013,343.20726336)
\curveto(584.90367207,343.15725776)(584.91867206,343.09225782)(584.92868013,343.01226336)
\curveto(584.94867203,342.93225798)(584.94867203,342.86225805)(584.92868013,342.80226336)
\curveto(584.88867209,342.69225822)(584.81867216,342.62725829)(584.71868013,342.60726336)
\curveto(584.61867236,342.59725832)(584.49867248,342.59225832)(584.35868013,342.59226336)
\lineto(583.57868013,342.59226336)
\lineto(583.29368013,342.59226336)
\curveto(583.20367377,342.59225832)(583.12867385,342.6122583)(583.06868013,342.65226336)
\curveto(582.98867399,342.69225822)(582.93367404,342.75225816)(582.90368013,342.83226336)
\curveto(582.8736741,342.92225799)(582.83367414,343.0122579)(582.78368013,343.10226336)
\curveto(582.72367425,343.2122577)(582.65867432,343.3122576)(582.58868013,343.40226336)
\curveto(582.51867446,343.49225742)(582.43867454,343.57225734)(582.34868013,343.64226336)
\curveto(582.20867477,343.73225718)(582.05367492,343.80225711)(581.88368013,343.85226336)
\curveto(581.82367515,343.87225704)(581.76367521,343.88225703)(581.70368013,343.88226336)
\curveto(581.64367533,343.88225703)(581.58867539,343.89225702)(581.53868013,343.91226336)
\lineto(581.38868013,343.91226336)
\curveto(581.18867579,343.912257)(581.02867595,343.89225702)(580.90868013,343.85226336)
\curveto(580.61867636,343.76225715)(580.38367659,343.62225729)(580.20368013,343.43226336)
\curveto(580.02367695,343.25225766)(579.8786771,343.03225788)(579.76868013,342.77226336)
\curveto(579.71867726,342.66225825)(579.6786773,342.54225837)(579.64868013,342.41226336)
\curveto(579.62867735,342.29225862)(579.60367737,342.16225875)(579.57368013,342.02226336)
\curveto(579.56367741,341.98225893)(579.55867742,341.94225897)(579.55868013,341.90226336)
\curveto(579.55867742,341.86225905)(579.55367742,341.82225909)(579.54368013,341.78226336)
\curveto(579.52367745,341.68225923)(579.51367746,341.54225937)(579.51368013,341.36226336)
\curveto(579.52367745,341.18225973)(579.53867744,341.04225987)(579.55868013,340.94226336)
\curveto(579.55867742,340.86226005)(579.56367741,340.80726011)(579.57368013,340.77726336)
\curveto(579.59367738,340.70726021)(579.60367737,340.63726028)(579.60368013,340.56726336)
\curveto(579.61367736,340.49726042)(579.62867735,340.42726049)(579.64868013,340.35726336)
\curveto(579.72867725,340.12726079)(579.82367715,339.917261)(579.93368013,339.72726336)
\curveto(580.04367693,339.53726138)(580.18367679,339.37726154)(580.35368013,339.24726336)
\curveto(580.39367658,339.2172617)(580.45367652,339.18226173)(580.53368013,339.14226336)
\curveto(580.64367633,339.07226184)(580.75367622,339.02726189)(580.86368013,339.00726336)
\curveto(580.98367599,338.98726193)(581.12867585,338.96726195)(581.29868013,338.94726336)
\lineto(581.38868013,338.94726336)
\curveto(581.42867555,338.94726197)(581.45867552,338.95226196)(581.47868013,338.96226336)
\lineto(581.61368013,338.96226336)
\curveto(581.68367529,338.98226193)(581.74867523,338.99726192)(581.80868013,339.00726336)
\curveto(581.8786751,339.02726189)(581.94367503,339.04726187)(582.00368013,339.06726336)
\curveto(582.30367467,339.19726172)(582.53367444,339.38726153)(582.69368013,339.63726336)
\curveto(582.73367424,339.68726123)(582.76867421,339.74226117)(582.79868013,339.80226336)
\curveto(582.82867415,339.87226104)(582.85367412,339.93226098)(582.87368013,339.98226336)
\curveto(582.91367406,340.09226082)(582.94867403,340.18726073)(582.97868013,340.26726336)
\curveto(583.00867397,340.35726056)(583.0786739,340.42726049)(583.18868013,340.47726336)
\curveto(583.2786737,340.5172604)(583.42367355,340.53226038)(583.62368013,340.52226336)
\lineto(584.11868013,340.52226336)
\lineto(584.32868013,340.52226336)
\curveto(584.40867257,340.53226038)(584.4736725,340.52726039)(584.52368013,340.50726336)
\lineto(584.64368013,340.50726336)
\lineto(584.76368013,340.47726336)
\curveto(584.80367217,340.47726044)(584.83367214,340.46726045)(584.85368013,340.44726336)
\curveto(584.90367207,340.40726051)(584.93367204,340.34726057)(584.94368013,340.26726336)
\curveto(584.96367201,340.19726072)(584.96367201,340.12226079)(584.94368013,340.04226336)
\curveto(584.85367212,339.7122612)(584.74367223,339.4172615)(584.61368013,339.15726336)
\curveto(584.20367277,338.38726253)(583.54867343,337.85226306)(582.64868013,337.55226336)
\curveto(582.54867443,337.52226339)(582.44367453,337.50226341)(582.33368013,337.49226336)
\curveto(582.22367475,337.47226344)(582.11367486,337.44726347)(582.00368013,337.41726336)
\curveto(581.94367503,337.40726351)(581.88367509,337.40226351)(581.82368013,337.40226336)
\curveto(581.76367521,337.40226351)(581.70367527,337.39726352)(581.64368013,337.38726336)
\lineto(581.47868013,337.38726336)
\curveto(581.42867555,337.36726355)(581.35367562,337.36226355)(581.25368013,337.37226336)
\curveto(581.15367582,337.37226354)(581.0786759,337.37726354)(581.02868013,337.38726336)
\curveto(580.94867603,337.40726351)(580.8736761,337.4172635)(580.80368013,337.41726336)
\curveto(580.74367623,337.40726351)(580.6786763,337.4122635)(580.60868013,337.43226336)
\lineto(580.45868013,337.46226336)
\curveto(580.40867657,337.46226345)(580.35867662,337.46726345)(580.30868013,337.47726336)
\curveto(580.19867678,337.50726341)(580.09367688,337.53726338)(579.99368013,337.56726336)
\curveto(579.89367708,337.59726332)(579.79867718,337.63226328)(579.70868013,337.67226336)
\curveto(579.23867774,337.87226304)(578.84367813,338.12726279)(578.52368013,338.43726336)
\curveto(578.20367877,338.75726216)(577.94367903,339.15226176)(577.74368013,339.62226336)
\curveto(577.69367928,339.7122612)(577.65367932,339.80726111)(577.62368013,339.90726336)
\lineto(577.53368013,340.23726336)
\curveto(577.52367945,340.27726064)(577.51867946,340.3122606)(577.51868013,340.34226336)
\curveto(577.51867946,340.38226053)(577.50867947,340.42726049)(577.48868013,340.47726336)
\curveto(577.46867951,340.54726037)(577.45867952,340.6172603)(577.45868013,340.68726336)
\curveto(577.45867952,340.76726015)(577.44867953,340.84226007)(577.42868013,340.91226336)
\lineto(577.42868013,341.16726336)
\curveto(577.40867957,341.2172597)(577.39867958,341.27225964)(577.39868013,341.33226336)
\curveto(577.39867958,341.40225951)(577.40867957,341.46225945)(577.42868013,341.51226336)
\curveto(577.43867954,341.56225935)(577.43867954,341.60725931)(577.42868013,341.64726336)
\curveto(577.41867956,341.68725923)(577.41867956,341.72725919)(577.42868013,341.76726336)
\curveto(577.44867953,341.83725908)(577.45367952,341.90225901)(577.44368013,341.96226336)
\curveto(577.44367953,342.02225889)(577.45367952,342.08225883)(577.47368013,342.14226336)
\curveto(577.52367945,342.32225859)(577.56367941,342.49225842)(577.59368013,342.65226336)
\curveto(577.62367935,342.82225809)(577.66867931,342.98725793)(577.72868013,343.14726336)
\curveto(577.94867903,343.65725726)(578.22367875,344.08225683)(578.55368013,344.42226336)
\curveto(578.89367808,344.76225615)(579.32367765,345.03725588)(579.84368013,345.24726336)
\curveto(579.98367699,345.30725561)(580.12867685,345.34725557)(580.27868013,345.36726336)
\curveto(580.42867655,345.39725552)(580.58367639,345.43225548)(580.74368013,345.47226336)
\curveto(580.82367615,345.48225543)(580.89867608,345.48725543)(580.96868013,345.48726336)
\curveto(581.03867594,345.48725543)(581.11367586,345.49225542)(581.19368013,345.50226336)
}
}
{
\newrgbcolor{curcolor}{0 0 0}
\pscustom[linestyle=none,fillstyle=solid,fillcolor=curcolor]
{
\newpath
\moveto(593.28696138,338.15226336)
\curveto(593.30695353,338.04226287)(593.31695352,337.93226298)(593.31696138,337.82226336)
\curveto(593.32695351,337.7122632)(593.27695356,337.63726328)(593.16696138,337.59726336)
\curveto(593.10695373,337.56726335)(593.0369538,337.55226336)(592.95696138,337.55226336)
\lineto(592.71696138,337.55226336)
\lineto(591.90696138,337.55226336)
\lineto(591.63696138,337.55226336)
\curveto(591.55695528,337.56226335)(591.49195535,337.58726333)(591.44196138,337.62726336)
\curveto(591.37195547,337.66726325)(591.31695552,337.72226319)(591.27696138,337.79226336)
\curveto(591.24695559,337.87226304)(591.20195564,337.93726298)(591.14196138,337.98726336)
\curveto(591.12195572,338.00726291)(591.09695574,338.02226289)(591.06696138,338.03226336)
\curveto(591.0369558,338.05226286)(590.99695584,338.05726286)(590.94696138,338.04726336)
\curveto(590.89695594,338.02726289)(590.84695599,338.00226291)(590.79696138,337.97226336)
\curveto(590.75695608,337.94226297)(590.71195613,337.917263)(590.66196138,337.89726336)
\curveto(590.61195623,337.85726306)(590.55695628,337.82226309)(590.49696138,337.79226336)
\lineto(590.31696138,337.70226336)
\curveto(590.18695665,337.64226327)(590.05195679,337.59226332)(589.91196138,337.55226336)
\curveto(589.77195707,337.52226339)(589.62695721,337.48726343)(589.47696138,337.44726336)
\curveto(589.40695743,337.42726349)(589.3369575,337.4172635)(589.26696138,337.41726336)
\curveto(589.20695763,337.40726351)(589.1419577,337.39726352)(589.07196138,337.38726336)
\lineto(588.98196138,337.38726336)
\curveto(588.95195789,337.37726354)(588.92195792,337.37226354)(588.89196138,337.37226336)
\lineto(588.72696138,337.37226336)
\curveto(588.62695821,337.35226356)(588.52695831,337.35226356)(588.42696138,337.37226336)
\lineto(588.29196138,337.37226336)
\curveto(588.22195862,337.39226352)(588.15195869,337.40226351)(588.08196138,337.40226336)
\curveto(588.02195882,337.39226352)(587.96195888,337.39726352)(587.90196138,337.41726336)
\curveto(587.80195904,337.43726348)(587.70695913,337.45726346)(587.61696138,337.47726336)
\curveto(587.52695931,337.48726343)(587.4419594,337.5122634)(587.36196138,337.55226336)
\curveto(587.07195977,337.66226325)(586.82196002,337.80226311)(586.61196138,337.97226336)
\curveto(586.41196043,338.15226276)(586.25196059,338.38726253)(586.13196138,338.67726336)
\curveto(586.10196074,338.74726217)(586.07196077,338.82226209)(586.04196138,338.90226336)
\curveto(586.02196082,338.98226193)(586.00196084,339.06726185)(585.98196138,339.15726336)
\curveto(585.96196088,339.20726171)(585.95196089,339.25726166)(585.95196138,339.30726336)
\curveto(585.96196088,339.35726156)(585.96196088,339.40726151)(585.95196138,339.45726336)
\curveto(585.9419609,339.48726143)(585.93196091,339.54726137)(585.92196138,339.63726336)
\curveto(585.92196092,339.73726118)(585.92696091,339.80726111)(585.93696138,339.84726336)
\curveto(585.95696088,339.94726097)(585.96696087,340.03226088)(585.96696138,340.10226336)
\lineto(586.05696138,340.43226336)
\curveto(586.08696075,340.55226036)(586.12696071,340.65726026)(586.17696138,340.74726336)
\curveto(586.34696049,341.03725988)(586.5419603,341.25725966)(586.76196138,341.40726336)
\curveto(586.98195986,341.55725936)(587.26195958,341.68725923)(587.60196138,341.79726336)
\curveto(587.73195911,341.84725907)(587.86695897,341.88225903)(588.00696138,341.90226336)
\curveto(588.14695869,341.92225899)(588.28695855,341.94725897)(588.42696138,341.97726336)
\curveto(588.50695833,341.99725892)(588.59195825,342.00725891)(588.68196138,342.00726336)
\curveto(588.77195807,342.0172589)(588.86195798,342.03225888)(588.95196138,342.05226336)
\curveto(589.02195782,342.07225884)(589.09195775,342.07725884)(589.16196138,342.06726336)
\curveto(589.23195761,342.06725885)(589.30695753,342.07725884)(589.38696138,342.09726336)
\curveto(589.45695738,342.1172588)(589.52695731,342.12725879)(589.59696138,342.12726336)
\curveto(589.66695717,342.12725879)(589.7419571,342.13725878)(589.82196138,342.15726336)
\curveto(590.03195681,342.20725871)(590.22195662,342.24725867)(590.39196138,342.27726336)
\curveto(590.57195627,342.3172586)(590.73195611,342.40725851)(590.87196138,342.54726336)
\curveto(590.96195588,342.63725828)(591.02195582,342.73725818)(591.05196138,342.84726336)
\curveto(591.06195578,342.87725804)(591.06195578,342.90225801)(591.05196138,342.92226336)
\curveto(591.05195579,342.94225797)(591.05695578,342.96225795)(591.06696138,342.98226336)
\curveto(591.07695576,343.00225791)(591.08195576,343.03225788)(591.08196138,343.07226336)
\lineto(591.08196138,343.16226336)
\lineto(591.05196138,343.28226336)
\curveto(591.05195579,343.32225759)(591.04695579,343.35725756)(591.03696138,343.38726336)
\curveto(590.9369559,343.68725723)(590.72695611,343.89225702)(590.40696138,344.00226336)
\curveto(590.31695652,344.03225688)(590.20695663,344.05225686)(590.07696138,344.06226336)
\curveto(589.95695688,344.08225683)(589.83195701,344.08725683)(589.70196138,344.07726336)
\curveto(589.57195727,344.07725684)(589.44695739,344.06725685)(589.32696138,344.04726336)
\curveto(589.20695763,344.02725689)(589.10195774,344.00225691)(589.01196138,343.97226336)
\curveto(588.95195789,343.95225696)(588.89195795,343.92225699)(588.83196138,343.88226336)
\curveto(588.78195806,343.85225706)(588.73195811,343.8172571)(588.68196138,343.77726336)
\curveto(588.63195821,343.73725718)(588.57695826,343.68225723)(588.51696138,343.61226336)
\curveto(588.46695837,343.54225737)(588.43195841,343.47725744)(588.41196138,343.41726336)
\curveto(588.36195848,343.3172576)(588.31695852,343.22225769)(588.27696138,343.13226336)
\curveto(588.24695859,343.04225787)(588.17695866,342.98225793)(588.06696138,342.95226336)
\curveto(587.98695885,342.93225798)(587.90195894,342.92225799)(587.81196138,342.92226336)
\lineto(587.54196138,342.92226336)
\lineto(586.97196138,342.92226336)
\curveto(586.92195992,342.92225799)(586.87195997,342.917258)(586.82196138,342.90726336)
\curveto(586.77196007,342.90725801)(586.72696011,342.912258)(586.68696138,342.92226336)
\lineto(586.55196138,342.92226336)
\curveto(586.53196031,342.93225798)(586.50696033,342.93725798)(586.47696138,342.93726336)
\curveto(586.44696039,342.93725798)(586.42196042,342.94725797)(586.40196138,342.96726336)
\curveto(586.32196052,342.98725793)(586.26696057,343.05225786)(586.23696138,343.16226336)
\curveto(586.22696061,343.2122577)(586.22696061,343.26225765)(586.23696138,343.31226336)
\curveto(586.24696059,343.36225755)(586.25696058,343.40725751)(586.26696138,343.44726336)
\curveto(586.29696054,343.55725736)(586.32696051,343.65725726)(586.35696138,343.74726336)
\curveto(586.39696044,343.84725707)(586.4419604,343.93725698)(586.49196138,344.01726336)
\lineto(586.58196138,344.16726336)
\lineto(586.67196138,344.31726336)
\curveto(586.75196009,344.42725649)(586.85195999,344.53225638)(586.97196138,344.63226336)
\curveto(586.99195985,344.64225627)(587.02195982,344.66725625)(587.06196138,344.70726336)
\curveto(587.11195973,344.74725617)(587.15695968,344.78225613)(587.19696138,344.81226336)
\curveto(587.2369596,344.84225607)(587.28195956,344.87225604)(587.33196138,344.90226336)
\curveto(587.50195934,345.0122559)(587.68195916,345.09725582)(587.87196138,345.15726336)
\curveto(588.06195878,345.22725569)(588.25695858,345.29225562)(588.45696138,345.35226336)
\curveto(588.57695826,345.38225553)(588.70195814,345.40225551)(588.83196138,345.41226336)
\curveto(588.96195788,345.42225549)(589.09195775,345.44225547)(589.22196138,345.47226336)
\curveto(589.26195758,345.48225543)(589.32195752,345.48225543)(589.40196138,345.47226336)
\curveto(589.49195735,345.46225545)(589.54695729,345.46725545)(589.56696138,345.48726336)
\curveto(589.97695686,345.49725542)(590.36695647,345.48225543)(590.73696138,345.44226336)
\curveto(591.11695572,345.40225551)(591.45695538,345.32725559)(591.75696138,345.21726336)
\curveto(592.06695477,345.10725581)(592.33195451,344.95725596)(592.55196138,344.76726336)
\curveto(592.77195407,344.58725633)(592.9419539,344.35225656)(593.06196138,344.06226336)
\curveto(593.13195371,343.89225702)(593.17195367,343.69725722)(593.18196138,343.47726336)
\curveto(593.19195365,343.25725766)(593.19695364,343.03225788)(593.19696138,342.80226336)
\lineto(593.19696138,339.45726336)
\lineto(593.19696138,338.87226336)
\curveto(593.19695364,338.68226223)(593.21695362,338.50726241)(593.25696138,338.34726336)
\curveto(593.26695357,338.3172626)(593.27195357,338.28226263)(593.27196138,338.24226336)
\curveto(593.27195357,338.2122627)(593.27695356,338.18226273)(593.28696138,338.15226336)
\moveto(591.08196138,340.46226336)
\curveto(591.09195575,340.5122604)(591.09695574,340.56726035)(591.09696138,340.62726336)
\curveto(591.09695574,340.69726022)(591.09195575,340.75726016)(591.08196138,340.80726336)
\curveto(591.06195578,340.86726005)(591.05195579,340.92225999)(591.05196138,340.97226336)
\curveto(591.05195579,341.02225989)(591.03195581,341.06225985)(590.99196138,341.09226336)
\curveto(590.9419559,341.13225978)(590.86695597,341.15225976)(590.76696138,341.15226336)
\curveto(590.72695611,341.14225977)(590.69195615,341.13225978)(590.66196138,341.12226336)
\curveto(590.63195621,341.12225979)(590.59695624,341.1172598)(590.55696138,341.10726336)
\curveto(590.48695635,341.08725983)(590.41195643,341.07225984)(590.33196138,341.06226336)
\curveto(590.25195659,341.05225986)(590.17195667,341.03725988)(590.09196138,341.01726336)
\curveto(590.06195678,341.00725991)(590.01695682,341.00225991)(589.95696138,341.00226336)
\curveto(589.82695701,340.97225994)(589.69695714,340.95225996)(589.56696138,340.94226336)
\curveto(589.4369574,340.93225998)(589.31195753,340.90726001)(589.19196138,340.86726336)
\curveto(589.11195773,340.84726007)(589.0369578,340.82726009)(588.96696138,340.80726336)
\curveto(588.89695794,340.79726012)(588.82695801,340.77726014)(588.75696138,340.74726336)
\curveto(588.54695829,340.65726026)(588.36695847,340.52226039)(588.21696138,340.34226336)
\curveto(588.07695876,340.16226075)(588.02695881,339.912261)(588.06696138,339.59226336)
\curveto(588.08695875,339.42226149)(588.1419587,339.28226163)(588.23196138,339.17226336)
\curveto(588.30195854,339.06226185)(588.40695843,338.97226194)(588.54696138,338.90226336)
\curveto(588.68695815,338.84226207)(588.836958,338.79726212)(588.99696138,338.76726336)
\curveto(589.16695767,338.73726218)(589.3419575,338.72726219)(589.52196138,338.73726336)
\curveto(589.71195713,338.75726216)(589.88695695,338.79226212)(590.04696138,338.84226336)
\curveto(590.30695653,338.92226199)(590.51195633,339.04726187)(590.66196138,339.21726336)
\curveto(590.81195603,339.39726152)(590.92695591,339.6172613)(591.00696138,339.87726336)
\curveto(591.02695581,339.94726097)(591.0369558,340.0172609)(591.03696138,340.08726336)
\curveto(591.04695579,340.16726075)(591.06195578,340.24726067)(591.08196138,340.32726336)
\lineto(591.08196138,340.46226336)
}
}
{
\newrgbcolor{curcolor}{0 0 0}
\pscustom[linestyle=none,fillstyle=solid,fillcolor=curcolor]
{
\newpath
\moveto(602.44024263,338.40726336)
\lineto(602.44024263,337.98726336)
\curveto(602.44023426,337.85726306)(602.41023429,337.75226316)(602.35024263,337.67226336)
\curveto(602.3002344,337.62226329)(602.23523447,337.58726333)(602.15524263,337.56726336)
\curveto(602.07523463,337.55726336)(601.98523472,337.55226336)(601.88524263,337.55226336)
\lineto(601.06024263,337.55226336)
\lineto(600.77524263,337.55226336)
\curveto(600.69523601,337.56226335)(600.63023607,337.58726333)(600.58024263,337.62726336)
\curveto(600.51023619,337.67726324)(600.47023623,337.74226317)(600.46024263,337.82226336)
\curveto(600.45023625,337.90226301)(600.43023627,337.98226293)(600.40024263,338.06226336)
\curveto(600.38023632,338.08226283)(600.36023634,338.09726282)(600.34024263,338.10726336)
\curveto(600.33023637,338.12726279)(600.31523639,338.14726277)(600.29524263,338.16726336)
\curveto(600.18523652,338.16726275)(600.1052366,338.14226277)(600.05524263,338.09226336)
\lineto(599.90524263,337.94226336)
\curveto(599.83523687,337.89226302)(599.77023693,337.84726307)(599.71024263,337.80726336)
\curveto(599.65023705,337.77726314)(599.58523712,337.73726318)(599.51524263,337.68726336)
\curveto(599.47523723,337.66726325)(599.43023727,337.64726327)(599.38024263,337.62726336)
\curveto(599.34023736,337.60726331)(599.29523741,337.58726333)(599.24524263,337.56726336)
\curveto(599.1052376,337.5172634)(598.95523775,337.47226344)(598.79524263,337.43226336)
\curveto(598.74523796,337.4122635)(598.700238,337.40226351)(598.66024263,337.40226336)
\curveto(598.62023808,337.40226351)(598.58023812,337.39726352)(598.54024263,337.38726336)
\lineto(598.40524263,337.38726336)
\curveto(598.37523833,337.37726354)(598.33523837,337.37226354)(598.28524263,337.37226336)
\lineto(598.15024263,337.37226336)
\curveto(598.09023861,337.35226356)(598.0002387,337.34726357)(597.88024263,337.35726336)
\curveto(597.76023894,337.35726356)(597.67523903,337.36726355)(597.62524263,337.38726336)
\curveto(597.55523915,337.40726351)(597.49023921,337.4172635)(597.43024263,337.41726336)
\curveto(597.38023932,337.40726351)(597.32523938,337.4122635)(597.26524263,337.43226336)
\lineto(596.90524263,337.55226336)
\curveto(596.79523991,337.58226333)(596.68524002,337.62226329)(596.57524263,337.67226336)
\curveto(596.22524048,337.82226309)(595.91024079,338.05226286)(595.63024263,338.36226336)
\curveto(595.36024134,338.68226223)(595.14524156,339.0172619)(594.98524263,339.36726336)
\curveto(594.93524177,339.47726144)(594.89524181,339.58226133)(594.86524263,339.68226336)
\curveto(594.83524187,339.79226112)(594.8002419,339.90226101)(594.76024263,340.01226336)
\curveto(594.75024195,340.05226086)(594.74524196,340.08726083)(594.74524263,340.11726336)
\curveto(594.74524196,340.15726076)(594.73524197,340.20226071)(594.71524263,340.25226336)
\curveto(594.69524201,340.33226058)(594.67524203,340.4172605)(594.65524263,340.50726336)
\curveto(594.64524206,340.60726031)(594.63024207,340.70726021)(594.61024263,340.80726336)
\curveto(594.6002421,340.83726008)(594.59524211,340.87226004)(594.59524263,340.91226336)
\curveto(594.6052421,340.95225996)(594.6052421,340.98725993)(594.59524263,341.01726336)
\lineto(594.59524263,341.15226336)
\curveto(594.59524211,341.20225971)(594.59024211,341.25225966)(594.58024263,341.30226336)
\curveto(594.57024213,341.35225956)(594.56524214,341.40725951)(594.56524263,341.46726336)
\curveto(594.56524214,341.53725938)(594.57024213,341.59225932)(594.58024263,341.63226336)
\curveto(594.59024211,341.68225923)(594.59524211,341.72725919)(594.59524263,341.76726336)
\lineto(594.59524263,341.91726336)
\curveto(594.6052421,341.96725895)(594.6052421,342.0122589)(594.59524263,342.05226336)
\curveto(594.59524211,342.10225881)(594.6052421,342.15225876)(594.62524263,342.20226336)
\curveto(594.64524206,342.3122586)(594.66024204,342.4172585)(594.67024263,342.51726336)
\curveto(594.69024201,342.6172583)(594.71524199,342.7172582)(594.74524263,342.81726336)
\curveto(594.78524192,342.93725798)(594.82024188,343.05225786)(594.85024263,343.16226336)
\curveto(594.88024182,343.27225764)(594.92024178,343.38225753)(594.97024263,343.49226336)
\curveto(595.11024159,343.79225712)(595.28524142,344.07725684)(595.49524263,344.34726336)
\curveto(595.51524119,344.37725654)(595.54024116,344.40225651)(595.57024263,344.42226336)
\curveto(595.61024109,344.45225646)(595.64024106,344.48225643)(595.66024263,344.51226336)
\curveto(595.700241,344.56225635)(595.74024096,344.60725631)(595.78024263,344.64726336)
\curveto(595.82024088,344.68725623)(595.86524084,344.72725619)(595.91524263,344.76726336)
\curveto(595.95524075,344.78725613)(595.99024071,344.8122561)(596.02024263,344.84226336)
\curveto(596.05024065,344.88225603)(596.08524062,344.912256)(596.12524263,344.93226336)
\curveto(596.37524033,345.10225581)(596.66524004,345.24225567)(596.99524263,345.35226336)
\curveto(597.06523964,345.37225554)(597.13523957,345.38725553)(597.20524263,345.39726336)
\curveto(597.28523942,345.40725551)(597.36523934,345.42225549)(597.44524263,345.44226336)
\curveto(597.51523919,345.46225545)(597.6052391,345.47225544)(597.71524263,345.47226336)
\curveto(597.82523888,345.48225543)(597.93523877,345.48725543)(598.04524263,345.48726336)
\curveto(598.15523855,345.48725543)(598.26023844,345.48225543)(598.36024263,345.47226336)
\curveto(598.47023823,345.46225545)(598.56023814,345.44725547)(598.63024263,345.42726336)
\curveto(598.78023792,345.37725554)(598.92523778,345.33225558)(599.06524263,345.29226336)
\curveto(599.2052375,345.25225566)(599.33523737,345.19725572)(599.45524263,345.12726336)
\curveto(599.52523718,345.07725584)(599.59023711,345.02725589)(599.65024263,344.97726336)
\curveto(599.71023699,344.93725598)(599.77523693,344.89225602)(599.84524263,344.84226336)
\curveto(599.88523682,344.8122561)(599.94023676,344.77225614)(600.01024263,344.72226336)
\curveto(600.09023661,344.67225624)(600.16523654,344.67225624)(600.23524263,344.72226336)
\curveto(600.27523643,344.74225617)(600.29523641,344.77725614)(600.29524263,344.82726336)
\curveto(600.29523641,344.87725604)(600.3052364,344.92725599)(600.32524263,344.97726336)
\lineto(600.32524263,345.12726336)
\curveto(600.33523637,345.15725576)(600.34023636,345.19225572)(600.34024263,345.23226336)
\lineto(600.34024263,345.35226336)
\lineto(600.34024263,347.39226336)
\curveto(600.34023636,347.50225341)(600.33523637,347.62225329)(600.32524263,347.75226336)
\curveto(600.32523638,347.89225302)(600.35023635,347.99725292)(600.40024263,348.06726336)
\curveto(600.44023626,348.14725277)(600.51523619,348.19725272)(600.62524263,348.21726336)
\curveto(600.64523606,348.22725269)(600.66523604,348.22725269)(600.68524263,348.21726336)
\curveto(600.705236,348.2172527)(600.72523598,348.22225269)(600.74524263,348.23226336)
\lineto(601.81024263,348.23226336)
\curveto(601.93023477,348.23225268)(602.04023466,348.22725269)(602.14024263,348.21726336)
\curveto(602.24023446,348.20725271)(602.31523439,348.16725275)(602.36524263,348.09726336)
\curveto(602.41523429,348.0172529)(602.44023426,347.912253)(602.44024263,347.78226336)
\lineto(602.44024263,347.42226336)
\lineto(602.44024263,338.40726336)
\moveto(600.40024263,341.34726336)
\curveto(600.41023629,341.38725953)(600.41023629,341.42725949)(600.40024263,341.46726336)
\lineto(600.40024263,341.60226336)
\curveto(600.4002363,341.70225921)(600.39523631,341.80225911)(600.38524263,341.90226336)
\curveto(600.37523633,342.00225891)(600.36023634,342.09225882)(600.34024263,342.17226336)
\curveto(600.32023638,342.28225863)(600.3002364,342.38225853)(600.28024263,342.47226336)
\curveto(600.27023643,342.56225835)(600.24523646,342.64725827)(600.20524263,342.72726336)
\curveto(600.06523664,343.08725783)(599.86023684,343.37225754)(599.59024263,343.58226336)
\curveto(599.33023737,343.79225712)(598.95023775,343.89725702)(598.45024263,343.89726336)
\curveto(598.39023831,343.89725702)(598.31023839,343.88725703)(598.21024263,343.86726336)
\curveto(598.13023857,343.84725707)(598.05523865,343.82725709)(597.98524263,343.80726336)
\curveto(597.92523878,343.79725712)(597.86523884,343.77725714)(597.80524263,343.74726336)
\curveto(597.53523917,343.63725728)(597.32523938,343.46725745)(597.17524263,343.23726336)
\curveto(597.02523968,343.00725791)(596.9052398,342.74725817)(596.81524263,342.45726336)
\curveto(596.78523992,342.35725856)(596.76523994,342.25725866)(596.75524263,342.15726336)
\curveto(596.74523996,342.05725886)(596.72523998,341.95225896)(596.69524263,341.84226336)
\lineto(596.69524263,341.63226336)
\curveto(596.67524003,341.54225937)(596.67024003,341.4172595)(596.68024263,341.25726336)
\curveto(596.69024001,341.10725981)(596.70524,340.99725992)(596.72524263,340.92726336)
\lineto(596.72524263,340.83726336)
\curveto(596.73523997,340.8172601)(596.74023996,340.79726012)(596.74024263,340.77726336)
\curveto(596.76023994,340.69726022)(596.77523993,340.62226029)(596.78524263,340.55226336)
\curveto(596.8052399,340.48226043)(596.82523988,340.40726051)(596.84524263,340.32726336)
\curveto(597.01523969,339.80726111)(597.3052394,339.42226149)(597.71524263,339.17226336)
\curveto(597.84523886,339.08226183)(598.02523868,339.0122619)(598.25524263,338.96226336)
\curveto(598.29523841,338.95226196)(598.35523835,338.94726197)(598.43524263,338.94726336)
\curveto(598.46523824,338.93726198)(598.51023819,338.92726199)(598.57024263,338.91726336)
\curveto(598.64023806,338.917262)(598.69523801,338.92226199)(598.73524263,338.93226336)
\curveto(598.81523789,338.95226196)(598.89523781,338.96726195)(598.97524263,338.97726336)
\curveto(599.05523765,338.98726193)(599.13523757,339.00726191)(599.21524263,339.03726336)
\curveto(599.46523724,339.14726177)(599.66523704,339.28726163)(599.81524263,339.45726336)
\curveto(599.96523674,339.62726129)(600.09523661,339.84226107)(600.20524263,340.10226336)
\curveto(600.24523646,340.19226072)(600.27523643,340.28226063)(600.29524263,340.37226336)
\curveto(600.31523639,340.47226044)(600.33523637,340.57726034)(600.35524263,340.68726336)
\curveto(600.36523634,340.73726018)(600.36523634,340.78226013)(600.35524263,340.82226336)
\curveto(600.35523635,340.87226004)(600.36523634,340.92225999)(600.38524263,340.97226336)
\curveto(600.39523631,341.00225991)(600.4002363,341.03725988)(600.40024263,341.07726336)
\lineto(600.40024263,341.21226336)
\lineto(600.40024263,341.34726336)
}
}
{
\newrgbcolor{curcolor}{0 0 0}
\pscustom[linestyle=none,fillstyle=solid,fillcolor=curcolor]
{
\newpath
\moveto(611.79016451,341.73726336)
\curveto(611.81015594,341.67725924)(611.82015593,341.59225932)(611.82016451,341.48226336)
\curveto(611.82015593,341.37225954)(611.81015594,341.28725963)(611.79016451,341.22726336)
\lineto(611.79016451,341.07726336)
\curveto(611.77015598,340.99725992)(611.76015599,340.91726)(611.76016451,340.83726336)
\curveto(611.77015598,340.75726016)(611.76515598,340.67726024)(611.74516451,340.59726336)
\curveto(611.72515602,340.52726039)(611.71015604,340.46226045)(611.70016451,340.40226336)
\curveto(611.69015606,340.34226057)(611.68015607,340.27726064)(611.67016451,340.20726336)
\curveto(611.63015612,340.09726082)(611.59515615,339.98226093)(611.56516451,339.86226336)
\curveto(611.53515621,339.75226116)(611.49515625,339.64726127)(611.44516451,339.54726336)
\curveto(611.23515651,339.06726185)(610.96015679,338.67726224)(610.62016451,338.37726336)
\curveto(610.28015747,338.07726284)(609.87015788,337.82726309)(609.39016451,337.62726336)
\curveto(609.27015848,337.57726334)(609.1451586,337.54226337)(609.01516451,337.52226336)
\curveto(608.89515885,337.49226342)(608.77015898,337.46226345)(608.64016451,337.43226336)
\curveto(608.59015916,337.4122635)(608.53515921,337.40226351)(608.47516451,337.40226336)
\curveto(608.41515933,337.40226351)(608.36015939,337.39726352)(608.31016451,337.38726336)
\lineto(608.20516451,337.38726336)
\curveto(608.17515957,337.37726354)(608.1451596,337.37226354)(608.11516451,337.37226336)
\curveto(608.06515968,337.36226355)(607.98515976,337.35726356)(607.87516451,337.35726336)
\curveto(607.76515998,337.34726357)(607.68016007,337.35226356)(607.62016451,337.37226336)
\lineto(607.47016451,337.37226336)
\curveto(607.42016033,337.38226353)(607.36516038,337.38726353)(607.30516451,337.38726336)
\curveto(607.25516049,337.37726354)(607.20516054,337.38226353)(607.15516451,337.40226336)
\curveto(607.11516063,337.4122635)(607.07516067,337.4172635)(607.03516451,337.41726336)
\curveto(607.00516074,337.4172635)(606.96516078,337.42226349)(606.91516451,337.43226336)
\curveto(606.81516093,337.46226345)(606.71516103,337.48726343)(606.61516451,337.50726336)
\curveto(606.51516123,337.52726339)(606.42016133,337.55726336)(606.33016451,337.59726336)
\curveto(606.21016154,337.63726328)(606.09516165,337.67726324)(605.98516451,337.71726336)
\curveto(605.88516186,337.75726316)(605.78016197,337.80726311)(605.67016451,337.86726336)
\curveto(605.32016243,338.07726284)(605.02016273,338.32226259)(604.77016451,338.60226336)
\curveto(604.52016323,338.88226203)(604.31016344,339.2172617)(604.14016451,339.60726336)
\curveto(604.09016366,339.69726122)(604.0501637,339.79226112)(604.02016451,339.89226336)
\curveto(604.00016375,339.99226092)(603.97516377,340.09726082)(603.94516451,340.20726336)
\curveto(603.92516382,340.25726066)(603.91516383,340.30226061)(603.91516451,340.34226336)
\curveto(603.91516383,340.38226053)(603.90516384,340.42726049)(603.88516451,340.47726336)
\curveto(603.86516388,340.55726036)(603.85516389,340.63726028)(603.85516451,340.71726336)
\curveto(603.85516389,340.80726011)(603.8451639,340.89226002)(603.82516451,340.97226336)
\curveto(603.81516393,341.02225989)(603.81016394,341.06725985)(603.81016451,341.10726336)
\lineto(603.81016451,341.24226336)
\curveto(603.79016396,341.30225961)(603.78016397,341.38725953)(603.78016451,341.49726336)
\curveto(603.79016396,341.60725931)(603.80516394,341.69225922)(603.82516451,341.75226336)
\lineto(603.82516451,341.85726336)
\curveto(603.83516391,341.90725901)(603.83516391,341.95725896)(603.82516451,342.00726336)
\curveto(603.82516392,342.06725885)(603.83516391,342.12225879)(603.85516451,342.17226336)
\curveto(603.86516388,342.22225869)(603.87016388,342.26725865)(603.87016451,342.30726336)
\curveto(603.87016388,342.35725856)(603.88016387,342.40725851)(603.90016451,342.45726336)
\curveto(603.94016381,342.58725833)(603.97516377,342.7122582)(604.00516451,342.83226336)
\curveto(604.03516371,342.96225795)(604.07516367,343.08725783)(604.12516451,343.20726336)
\curveto(604.30516344,343.6172573)(604.52016323,343.95725696)(604.77016451,344.22726336)
\curveto(605.02016273,344.50725641)(605.32516242,344.76225615)(605.68516451,344.99226336)
\curveto(605.78516196,345.04225587)(605.89016186,345.08725583)(606.00016451,345.12726336)
\curveto(606.11016164,345.16725575)(606.22016153,345.2122557)(606.33016451,345.26226336)
\curveto(606.46016129,345.3122556)(606.59516115,345.34725557)(606.73516451,345.36726336)
\curveto(606.87516087,345.38725553)(607.02016073,345.4172555)(607.17016451,345.45726336)
\curveto(607.2501605,345.46725545)(607.32516042,345.47225544)(607.39516451,345.47226336)
\curveto(607.46516028,345.47225544)(607.53516021,345.47725544)(607.60516451,345.48726336)
\curveto(608.18515956,345.49725542)(608.68515906,345.43725548)(609.10516451,345.30726336)
\curveto(609.53515821,345.17725574)(609.91515783,344.99725592)(610.24516451,344.76726336)
\curveto(610.35515739,344.68725623)(610.46515728,344.59725632)(610.57516451,344.49726336)
\curveto(610.69515705,344.40725651)(610.79515695,344.30725661)(610.87516451,344.19726336)
\curveto(610.95515679,344.09725682)(611.02515672,343.99725692)(611.08516451,343.89726336)
\curveto(611.15515659,343.79725712)(611.22515652,343.69225722)(611.29516451,343.58226336)
\curveto(611.36515638,343.47225744)(611.42015633,343.35225756)(611.46016451,343.22226336)
\curveto(611.50015625,343.10225781)(611.5451562,342.97225794)(611.59516451,342.83226336)
\curveto(611.62515612,342.75225816)(611.6501561,342.66725825)(611.67016451,342.57726336)
\lineto(611.73016451,342.30726336)
\curveto(611.74015601,342.26725865)(611.745156,342.22725869)(611.74516451,342.18726336)
\curveto(611.745156,342.14725877)(611.750156,342.10725881)(611.76016451,342.06726336)
\curveto(611.78015597,342.0172589)(611.78515596,341.96225895)(611.77516451,341.90226336)
\curveto(611.76515598,341.84225907)(611.77015598,341.78725913)(611.79016451,341.73726336)
\moveto(609.69016451,341.19726336)
\curveto(609.70015805,341.24725967)(609.70515804,341.3172596)(609.70516451,341.40726336)
\curveto(609.70515804,341.50725941)(609.70015805,341.58225933)(609.69016451,341.63226336)
\lineto(609.69016451,341.75226336)
\curveto(609.67015808,341.80225911)(609.66015809,341.85725906)(609.66016451,341.91726336)
\curveto(609.66015809,341.97725894)(609.65515809,342.03225888)(609.64516451,342.08226336)
\curveto(609.6451581,342.12225879)(609.64015811,342.15225876)(609.63016451,342.17226336)
\lineto(609.57016451,342.41226336)
\curveto(609.56015819,342.50225841)(609.54015821,342.58725833)(609.51016451,342.66726336)
\curveto(609.40015835,342.92725799)(609.27015848,343.14725777)(609.12016451,343.32726336)
\curveto(608.97015878,343.5172574)(608.77015898,343.66725725)(608.52016451,343.77726336)
\curveto(608.46015929,343.79725712)(608.40015935,343.8122571)(608.34016451,343.82226336)
\curveto(608.28015947,343.84225707)(608.21515953,343.86225705)(608.14516451,343.88226336)
\curveto(608.06515968,343.90225701)(607.98015977,343.90725701)(607.89016451,343.89726336)
\lineto(607.62016451,343.89726336)
\curveto(607.59016016,343.87725704)(607.55516019,343.86725705)(607.51516451,343.86726336)
\curveto(607.47516027,343.87725704)(607.44016031,343.87725704)(607.41016451,343.86726336)
\lineto(607.20016451,343.80726336)
\curveto(607.14016061,343.79725712)(607.08516066,343.77725714)(607.03516451,343.74726336)
\curveto(606.78516096,343.63725728)(606.58016117,343.47725744)(606.42016451,343.26726336)
\curveto(606.27016148,343.06725785)(606.1501616,342.83225808)(606.06016451,342.56226336)
\curveto(606.03016172,342.46225845)(606.00516174,342.35725856)(605.98516451,342.24726336)
\curveto(605.97516177,342.13725878)(605.96016179,342.02725889)(605.94016451,341.91726336)
\curveto(605.93016182,341.86725905)(605.92516182,341.8172591)(605.92516451,341.76726336)
\lineto(605.92516451,341.61726336)
\curveto(605.90516184,341.54725937)(605.89516185,341.44225947)(605.89516451,341.30226336)
\curveto(605.90516184,341.16225975)(605.92016183,341.05725986)(605.94016451,340.98726336)
\lineto(605.94016451,340.85226336)
\curveto(605.96016179,340.77226014)(605.97516177,340.69226022)(605.98516451,340.61226336)
\curveto(605.99516175,340.54226037)(606.01016174,340.46726045)(606.03016451,340.38726336)
\curveto(606.13016162,340.08726083)(606.23516151,339.84226107)(606.34516451,339.65226336)
\curveto(606.46516128,339.47226144)(606.6501611,339.30726161)(606.90016451,339.15726336)
\curveto(606.97016078,339.10726181)(607.0451607,339.06726185)(607.12516451,339.03726336)
\curveto(607.21516053,339.00726191)(607.30516044,338.98226193)(607.39516451,338.96226336)
\curveto(607.43516031,338.95226196)(607.47016028,338.94726197)(607.50016451,338.94726336)
\curveto(607.53016022,338.95726196)(607.56516018,338.95726196)(607.60516451,338.94726336)
\lineto(607.72516451,338.91726336)
\curveto(607.77515997,338.917262)(607.82015993,338.92226199)(607.86016451,338.93226336)
\lineto(607.98016451,338.93226336)
\curveto(608.06015969,338.95226196)(608.14015961,338.96726195)(608.22016451,338.97726336)
\curveto(608.30015945,338.98726193)(608.37515937,339.00726191)(608.44516451,339.03726336)
\curveto(608.70515904,339.13726178)(608.91515883,339.27226164)(609.07516451,339.44226336)
\curveto(609.23515851,339.6122613)(609.37015838,339.82226109)(609.48016451,340.07226336)
\curveto(609.52015823,340.17226074)(609.5501582,340.27226064)(609.57016451,340.37226336)
\curveto(609.59015816,340.47226044)(609.61515813,340.57726034)(609.64516451,340.68726336)
\curveto(609.65515809,340.72726019)(609.66015809,340.76226015)(609.66016451,340.79226336)
\curveto(609.66015809,340.83226008)(609.66515808,340.87226004)(609.67516451,340.91226336)
\lineto(609.67516451,341.04726336)
\curveto(609.67515807,341.09725982)(609.68015807,341.14725977)(609.69016451,341.19726336)
}
}
{
\newrgbcolor{curcolor}{0 0 0}
\pscustom[linestyle=none,fillstyle=solid,fillcolor=curcolor]
{
\newpath
\moveto(617.61508638,345.48726336)
\curveto(617.72508107,345.48725543)(617.82008097,345.47725544)(617.90008638,345.45726336)
\curveto(617.9900808,345.43725548)(618.06008073,345.39225552)(618.11008638,345.32226336)
\curveto(618.17008062,345.24225567)(618.20008059,345.10225581)(618.20008638,344.90226336)
\lineto(618.20008638,344.39226336)
\lineto(618.20008638,344.01726336)
\curveto(618.21008058,343.87725704)(618.1950806,343.76725715)(618.15508638,343.68726336)
\curveto(618.11508068,343.6172573)(618.05508074,343.57225734)(617.97508638,343.55226336)
\curveto(617.90508089,343.53225738)(617.82008097,343.52225739)(617.72008638,343.52226336)
\curveto(617.63008116,343.52225739)(617.53008126,343.52725739)(617.42008638,343.53726336)
\curveto(617.32008147,343.54725737)(617.22508157,343.54225737)(617.13508638,343.52226336)
\curveto(617.06508173,343.50225741)(616.9950818,343.48725743)(616.92508638,343.47726336)
\curveto(616.85508194,343.47725744)(616.790082,343.46725745)(616.73008638,343.44726336)
\curveto(616.57008222,343.39725752)(616.41008238,343.32225759)(616.25008638,343.22226336)
\curveto(616.0900827,343.13225778)(615.96508283,343.02725789)(615.87508638,342.90726336)
\curveto(615.82508297,342.82725809)(615.77008302,342.74225817)(615.71008638,342.65226336)
\curveto(615.66008313,342.57225834)(615.61008318,342.48725843)(615.56008638,342.39726336)
\curveto(615.53008326,342.3172586)(615.50008329,342.23225868)(615.47008638,342.14226336)
\lineto(615.41008638,341.90226336)
\curveto(615.3900834,341.83225908)(615.38008341,341.75725916)(615.38008638,341.67726336)
\curveto(615.38008341,341.60725931)(615.37008342,341.53725938)(615.35008638,341.46726336)
\curveto(615.34008345,341.42725949)(615.33508346,341.38725953)(615.33508638,341.34726336)
\curveto(615.34508345,341.3172596)(615.34508345,341.28725963)(615.33508638,341.25726336)
\lineto(615.33508638,341.01726336)
\curveto(615.31508348,340.94725997)(615.31008348,340.86726005)(615.32008638,340.77726336)
\curveto(615.33008346,340.69726022)(615.33508346,340.6172603)(615.33508638,340.53726336)
\lineto(615.33508638,339.57726336)
\lineto(615.33508638,338.30226336)
\curveto(615.33508346,338.17226274)(615.33008346,338.05226286)(615.32008638,337.94226336)
\curveto(615.31008348,337.83226308)(615.28008351,337.74226317)(615.23008638,337.67226336)
\curveto(615.21008358,337.64226327)(615.17508362,337.6172633)(615.12508638,337.59726336)
\curveto(615.08508371,337.58726333)(615.04008375,337.57726334)(614.99008638,337.56726336)
\lineto(614.91508638,337.56726336)
\curveto(614.86508393,337.55726336)(614.81008398,337.55226336)(614.75008638,337.55226336)
\lineto(614.58508638,337.55226336)
\lineto(613.94008638,337.55226336)
\curveto(613.88008491,337.56226335)(613.81508498,337.56726335)(613.74508638,337.56726336)
\lineto(613.55008638,337.56726336)
\curveto(613.50008529,337.58726333)(613.45008534,337.60226331)(613.40008638,337.61226336)
\curveto(613.35008544,337.63226328)(613.31508548,337.66726325)(613.29508638,337.71726336)
\curveto(613.25508554,337.76726315)(613.23008556,337.83726308)(613.22008638,337.92726336)
\lineto(613.22008638,338.22726336)
\lineto(613.22008638,339.24726336)
\lineto(613.22008638,343.47726336)
\lineto(613.22008638,344.58726336)
\lineto(613.22008638,344.87226336)
\curveto(613.22008557,344.97225594)(613.24008555,345.05225586)(613.28008638,345.11226336)
\curveto(613.33008546,345.19225572)(613.40508539,345.24225567)(613.50508638,345.26226336)
\curveto(613.60508519,345.28225563)(613.72508507,345.29225562)(613.86508638,345.29226336)
\lineto(614.63008638,345.29226336)
\curveto(614.75008404,345.29225562)(614.85508394,345.28225563)(614.94508638,345.26226336)
\curveto(615.03508376,345.25225566)(615.10508369,345.20725571)(615.15508638,345.12726336)
\curveto(615.18508361,345.07725584)(615.20008359,345.00725591)(615.20008638,344.91726336)
\lineto(615.23008638,344.64726336)
\curveto(615.24008355,344.56725635)(615.25508354,344.49225642)(615.27508638,344.42226336)
\curveto(615.30508349,344.35225656)(615.35508344,344.3172566)(615.42508638,344.31726336)
\curveto(615.44508335,344.33725658)(615.46508333,344.34725657)(615.48508638,344.34726336)
\curveto(615.50508329,344.34725657)(615.52508327,344.35725656)(615.54508638,344.37726336)
\curveto(615.60508319,344.42725649)(615.65508314,344.48225643)(615.69508638,344.54226336)
\curveto(615.74508305,344.6122563)(615.80508299,344.67225624)(615.87508638,344.72226336)
\curveto(615.91508288,344.75225616)(615.95008284,344.78225613)(615.98008638,344.81226336)
\curveto(616.01008278,344.85225606)(616.04508275,344.88725603)(616.08508638,344.91726336)
\lineto(616.35508638,345.09726336)
\curveto(616.45508234,345.15725576)(616.55508224,345.2122557)(616.65508638,345.26226336)
\curveto(616.75508204,345.30225561)(616.85508194,345.33725558)(616.95508638,345.36726336)
\lineto(617.28508638,345.45726336)
\curveto(617.31508148,345.46725545)(617.37008142,345.46725545)(617.45008638,345.45726336)
\curveto(617.54008125,345.45725546)(617.5950812,345.46725545)(617.61508638,345.48726336)
}
}
{
\newrgbcolor{curcolor}{0 0 0}
\pscustom[linestyle=none,fillstyle=solid,fillcolor=curcolor]
{
\newpath
\moveto(626.12149263,341.49726336)
\curveto(626.14148447,341.4172595)(626.14148447,341.32725959)(626.12149263,341.22726336)
\curveto(626.10148451,341.12725979)(626.06648454,341.06225985)(626.01649263,341.03226336)
\curveto(625.96648464,340.99225992)(625.89148472,340.96225995)(625.79149263,340.94226336)
\curveto(625.70148491,340.93225998)(625.59648501,340.92225999)(625.47649263,340.91226336)
\lineto(625.13149263,340.91226336)
\curveto(625.02148559,340.92225999)(624.92148569,340.92725999)(624.83149263,340.92726336)
\lineto(621.17149263,340.92726336)
\lineto(620.96149263,340.92726336)
\curveto(620.90148971,340.92725999)(620.84648976,340.91726)(620.79649263,340.89726336)
\curveto(620.71648989,340.85726006)(620.66648994,340.8172601)(620.64649263,340.77726336)
\curveto(620.62648998,340.75726016)(620.60649,340.7172602)(620.58649263,340.65726336)
\curveto(620.56649004,340.60726031)(620.56149005,340.55726036)(620.57149263,340.50726336)
\curveto(620.59149002,340.44726047)(620.60149001,340.38726053)(620.60149263,340.32726336)
\curveto(620.61149,340.27726064)(620.62648998,340.22226069)(620.64649263,340.16226336)
\curveto(620.72648988,339.92226099)(620.82148979,339.72226119)(620.93149263,339.56226336)
\curveto(621.05148956,339.4122615)(621.2114894,339.27726164)(621.41149263,339.15726336)
\curveto(621.49148912,339.10726181)(621.57148904,339.07226184)(621.65149263,339.05226336)
\curveto(621.74148887,339.04226187)(621.83148878,339.02226189)(621.92149263,338.99226336)
\curveto(622.00148861,338.97226194)(622.1114885,338.95726196)(622.25149263,338.94726336)
\curveto(622.39148822,338.93726198)(622.5114881,338.94226197)(622.61149263,338.96226336)
\lineto(622.74649263,338.96226336)
\curveto(622.84648776,338.98226193)(622.93648767,339.00226191)(623.01649263,339.02226336)
\curveto(623.1064875,339.05226186)(623.19148742,339.08226183)(623.27149263,339.11226336)
\curveto(623.37148724,339.16226175)(623.48148713,339.22726169)(623.60149263,339.30726336)
\curveto(623.73148688,339.38726153)(623.82648678,339.46726145)(623.88649263,339.54726336)
\curveto(623.93648667,339.6172613)(623.98648662,339.68226123)(624.03649263,339.74226336)
\curveto(624.09648651,339.8122611)(624.16648644,339.86226105)(624.24649263,339.89226336)
\curveto(624.34648626,339.94226097)(624.47148614,339.96226095)(624.62149263,339.95226336)
\lineto(625.05649263,339.95226336)
\lineto(625.23649263,339.95226336)
\curveto(625.3064853,339.96226095)(625.36648524,339.95726096)(625.41649263,339.93726336)
\lineto(625.56649263,339.93726336)
\curveto(625.66648494,339.917261)(625.73648487,339.89226102)(625.77649263,339.86226336)
\curveto(625.81648479,339.84226107)(625.83648477,339.79726112)(625.83649263,339.72726336)
\curveto(625.84648476,339.65726126)(625.84148477,339.59726132)(625.82149263,339.54726336)
\curveto(625.77148484,339.40726151)(625.71648489,339.28226163)(625.65649263,339.17226336)
\curveto(625.59648501,339.06226185)(625.52648508,338.95226196)(625.44649263,338.84226336)
\curveto(625.22648538,338.5122624)(624.97648563,338.24726267)(624.69649263,338.04726336)
\curveto(624.41648619,337.84726307)(624.06648654,337.67726324)(623.64649263,337.53726336)
\curveto(623.53648707,337.49726342)(623.42648718,337.47226344)(623.31649263,337.46226336)
\curveto(623.2064874,337.45226346)(623.09148752,337.43226348)(622.97149263,337.40226336)
\curveto(622.93148768,337.39226352)(622.88648772,337.39226352)(622.83649263,337.40226336)
\curveto(622.79648781,337.40226351)(622.75648785,337.39726352)(622.71649263,337.38726336)
\lineto(622.55149263,337.38726336)
\curveto(622.50148811,337.36726355)(622.44148817,337.36226355)(622.37149263,337.37226336)
\curveto(622.3114883,337.37226354)(622.25648835,337.37726354)(622.20649263,337.38726336)
\curveto(622.12648848,337.39726352)(622.05648855,337.39726352)(621.99649263,337.38726336)
\curveto(621.93648867,337.37726354)(621.87148874,337.38226353)(621.80149263,337.40226336)
\curveto(621.75148886,337.42226349)(621.69648891,337.43226348)(621.63649263,337.43226336)
\curveto(621.57648903,337.43226348)(621.52148909,337.44226347)(621.47149263,337.46226336)
\curveto(621.36148925,337.48226343)(621.25148936,337.50726341)(621.14149263,337.53726336)
\curveto(621.03148958,337.55726336)(620.93148968,337.59226332)(620.84149263,337.64226336)
\curveto(620.73148988,337.68226323)(620.62648998,337.7172632)(620.52649263,337.74726336)
\curveto(620.43649017,337.78726313)(620.35149026,337.83226308)(620.27149263,337.88226336)
\curveto(619.95149066,338.08226283)(619.66649094,338.3122626)(619.41649263,338.57226336)
\curveto(619.16649144,338.84226207)(618.96149165,339.15226176)(618.80149263,339.50226336)
\curveto(618.75149186,339.6122613)(618.7114919,339.72226119)(618.68149263,339.83226336)
\curveto(618.65149196,339.95226096)(618.611492,340.07226084)(618.56149263,340.19226336)
\curveto(618.55149206,340.23226068)(618.54649206,340.26726065)(618.54649263,340.29726336)
\curveto(618.54649206,340.33726058)(618.54149207,340.37726054)(618.53149263,340.41726336)
\curveto(618.49149212,340.53726038)(618.46649214,340.66726025)(618.45649263,340.80726336)
\lineto(618.42649263,341.22726336)
\curveto(618.42649218,341.27725964)(618.42149219,341.33225958)(618.41149263,341.39226336)
\curveto(618.4114922,341.45225946)(618.41649219,341.50725941)(618.42649263,341.55726336)
\lineto(618.42649263,341.73726336)
\lineto(618.47149263,342.09726336)
\curveto(618.5114921,342.26725865)(618.54649206,342.43225848)(618.57649263,342.59226336)
\curveto(618.606492,342.75225816)(618.65149196,342.90225801)(618.71149263,343.04226336)
\curveto(619.14149147,344.08225683)(619.87149074,344.8172561)(620.90149263,345.24726336)
\curveto(621.04148957,345.30725561)(621.18148943,345.34725557)(621.32149263,345.36726336)
\curveto(621.47148914,345.39725552)(621.62648898,345.43225548)(621.78649263,345.47226336)
\curveto(621.86648874,345.48225543)(621.94148867,345.48725543)(622.01149263,345.48726336)
\curveto(622.08148853,345.48725543)(622.15648845,345.49225542)(622.23649263,345.50226336)
\curveto(622.74648786,345.5122554)(623.18148743,345.45225546)(623.54149263,345.32226336)
\curveto(623.9114867,345.20225571)(624.24148637,345.04225587)(624.53149263,344.84226336)
\curveto(624.62148599,344.78225613)(624.7114859,344.7122562)(624.80149263,344.63226336)
\curveto(624.89148572,344.56225635)(624.97148564,344.48725643)(625.04149263,344.40726336)
\curveto(625.07148554,344.35725656)(625.1114855,344.3172566)(625.16149263,344.28726336)
\curveto(625.24148537,344.17725674)(625.31648529,344.06225685)(625.38649263,343.94226336)
\curveto(625.45648515,343.83225708)(625.53148508,343.7172572)(625.61149263,343.59726336)
\curveto(625.66148495,343.50725741)(625.70148491,343.4122575)(625.73149263,343.31226336)
\curveto(625.77148484,343.22225769)(625.8114848,343.12225779)(625.85149263,343.01226336)
\curveto(625.90148471,342.88225803)(625.94148467,342.74725817)(625.97149263,342.60726336)
\curveto(626.00148461,342.46725845)(626.03648457,342.32725859)(626.07649263,342.18726336)
\curveto(626.09648451,342.10725881)(626.10148451,342.0172589)(626.09149263,341.91726336)
\curveto(626.09148452,341.82725909)(626.10148451,341.74225917)(626.12149263,341.66226336)
\lineto(626.12149263,341.49726336)
\moveto(623.87149263,342.38226336)
\curveto(623.94148667,342.48225843)(623.94648666,342.60225831)(623.88649263,342.74226336)
\curveto(623.83648677,342.89225802)(623.79648681,343.00225791)(623.76649263,343.07226336)
\curveto(623.62648698,343.34225757)(623.44148717,343.54725737)(623.21149263,343.68726336)
\curveto(622.98148763,343.83725708)(622.66148795,343.917257)(622.25149263,343.92726336)
\curveto(622.22148839,343.90725701)(622.18648842,343.90225701)(622.14649263,343.91226336)
\curveto(622.1064885,343.92225699)(622.07148854,343.92225699)(622.04149263,343.91226336)
\curveto(621.99148862,343.89225702)(621.93648867,343.87725704)(621.87649263,343.86726336)
\curveto(621.81648879,343.86725705)(621.76148885,343.85725706)(621.71149263,343.83726336)
\curveto(621.27148934,343.69725722)(620.94648966,343.42225749)(620.73649263,343.01226336)
\curveto(620.71648989,342.97225794)(620.69148992,342.917258)(620.66149263,342.84726336)
\curveto(620.64148997,342.78725813)(620.62648998,342.72225819)(620.61649263,342.65226336)
\curveto(620.60649,342.59225832)(620.60649,342.53225838)(620.61649263,342.47226336)
\curveto(620.63648997,342.4122585)(620.67148994,342.36225855)(620.72149263,342.32226336)
\curveto(620.80148981,342.27225864)(620.9114897,342.24725867)(621.05149263,342.24726336)
\lineto(621.45649263,342.24726336)
\lineto(623.12149263,342.24726336)
\lineto(623.55649263,342.24726336)
\curveto(623.71648689,342.25725866)(623.82148679,342.30225861)(623.87149263,342.38226336)
}
}
{
\newrgbcolor{curcolor}{0 0 0}
\pscustom[linestyle=none,fillstyle=solid,fillcolor=curcolor]
{
\newpath
\moveto(630.33977388,345.50226336)
\curveto(631.08976938,345.52225539)(631.73976873,345.43725548)(632.28977388,345.24726336)
\curveto(632.84976762,345.06725585)(633.2747672,344.75225616)(633.56477388,344.30226336)
\curveto(633.63476684,344.19225672)(633.69476678,344.07725684)(633.74477388,343.95726336)
\curveto(633.80476667,343.84725707)(633.85476662,343.72225719)(633.89477388,343.58226336)
\curveto(633.91476656,343.52225739)(633.92476655,343.45725746)(633.92477388,343.38726336)
\curveto(633.92476655,343.3172576)(633.91476656,343.25725766)(633.89477388,343.20726336)
\curveto(633.85476662,343.14725777)(633.79976667,343.10725781)(633.72977388,343.08726336)
\curveto(633.67976679,343.06725785)(633.61976685,343.05725786)(633.54977388,343.05726336)
\lineto(633.33977388,343.05726336)
\lineto(632.67977388,343.05726336)
\curveto(632.60976786,343.05725786)(632.53976793,343.05225786)(632.46977388,343.04226336)
\curveto(632.39976807,343.04225787)(632.33476814,343.05225786)(632.27477388,343.07226336)
\curveto(632.1747683,343.09225782)(632.09976837,343.13225778)(632.04977388,343.19226336)
\curveto(631.99976847,343.25225766)(631.95476852,343.3122576)(631.91477388,343.37226336)
\lineto(631.79477388,343.58226336)
\curveto(631.76476871,343.66225725)(631.71476876,343.72725719)(631.64477388,343.77726336)
\curveto(631.54476893,343.85725706)(631.44476903,343.917257)(631.34477388,343.95726336)
\curveto(631.25476922,343.99725692)(631.13976933,344.03225688)(630.99977388,344.06226336)
\curveto(630.92976954,344.08225683)(630.82476965,344.09725682)(630.68477388,344.10726336)
\curveto(630.55476992,344.1172568)(630.45477002,344.1122568)(630.38477388,344.09226336)
\lineto(630.27977388,344.09226336)
\lineto(630.12977388,344.06226336)
\curveto(630.08977038,344.06225685)(630.04477043,344.05725686)(629.99477388,344.04726336)
\curveto(629.82477065,343.99725692)(629.68477079,343.92725699)(629.57477388,343.83726336)
\curveto(629.474771,343.75725716)(629.40477107,343.63225728)(629.36477388,343.46226336)
\curveto(629.34477113,343.39225752)(629.34477113,343.32725759)(629.36477388,343.26726336)
\curveto(629.38477109,343.20725771)(629.40477107,343.15725776)(629.42477388,343.11726336)
\curveto(629.49477098,342.99725792)(629.5747709,342.90225801)(629.66477388,342.83226336)
\curveto(629.76477071,342.76225815)(629.87977059,342.70225821)(630.00977388,342.65226336)
\curveto(630.19977027,342.57225834)(630.40477007,342.50225841)(630.62477388,342.44226336)
\lineto(631.31477388,342.29226336)
\curveto(631.55476892,342.25225866)(631.78476869,342.20225871)(632.00477388,342.14226336)
\curveto(632.23476824,342.09225882)(632.44976802,342.02725889)(632.64977388,341.94726336)
\curveto(632.73976773,341.90725901)(632.82476765,341.87225904)(632.90477388,341.84226336)
\curveto(632.99476748,341.82225909)(633.07976739,341.78725913)(633.15977388,341.73726336)
\curveto(633.34976712,341.6172593)(633.51976695,341.48725943)(633.66977388,341.34726336)
\curveto(633.82976664,341.20725971)(633.95476652,341.03225988)(634.04477388,340.82226336)
\curveto(634.0747664,340.75226016)(634.09976637,340.68226023)(634.11977388,340.61226336)
\curveto(634.13976633,340.54226037)(634.15976631,340.46726045)(634.17977388,340.38726336)
\curveto(634.18976628,340.32726059)(634.19476628,340.23226068)(634.19477388,340.10226336)
\curveto(634.20476627,339.98226093)(634.20476627,339.88726103)(634.19477388,339.81726336)
\lineto(634.19477388,339.74226336)
\curveto(634.1747663,339.68226123)(634.15976631,339.62226129)(634.14977388,339.56226336)
\curveto(634.14976632,339.5122614)(634.14476633,339.46226145)(634.13477388,339.41226336)
\curveto(634.06476641,339.1122618)(633.95476652,338.84726207)(633.80477388,338.61726336)
\curveto(633.64476683,338.37726254)(633.44976702,338.18226273)(633.21977388,338.03226336)
\curveto(632.98976748,337.88226303)(632.72976774,337.75226316)(632.43977388,337.64226336)
\curveto(632.32976814,337.59226332)(632.20976826,337.55726336)(632.07977388,337.53726336)
\curveto(631.95976851,337.5172634)(631.83976863,337.49226342)(631.71977388,337.46226336)
\curveto(631.62976884,337.44226347)(631.53476894,337.43226348)(631.43477388,337.43226336)
\curveto(631.34476913,337.42226349)(631.25476922,337.40726351)(631.16477388,337.38726336)
\lineto(630.89477388,337.38726336)
\curveto(630.83476964,337.36726355)(630.72976974,337.35726356)(630.57977388,337.35726336)
\curveto(630.43977003,337.35726356)(630.33977013,337.36726355)(630.27977388,337.38726336)
\curveto(630.24977022,337.38726353)(630.21477026,337.39226352)(630.17477388,337.40226336)
\lineto(630.06977388,337.40226336)
\curveto(629.94977052,337.42226349)(629.82977064,337.43726348)(629.70977388,337.44726336)
\curveto(629.58977088,337.45726346)(629.474771,337.47726344)(629.36477388,337.50726336)
\curveto(628.9747715,337.6172633)(628.62977184,337.74226317)(628.32977388,337.88226336)
\curveto(628.02977244,338.03226288)(627.7747727,338.25226266)(627.56477388,338.54226336)
\curveto(627.42477305,338.73226218)(627.30477317,338.95226196)(627.20477388,339.20226336)
\curveto(627.18477329,339.26226165)(627.16477331,339.34226157)(627.14477388,339.44226336)
\curveto(627.12477335,339.49226142)(627.10977336,339.56226135)(627.09977388,339.65226336)
\curveto(627.08977338,339.74226117)(627.09477338,339.8172611)(627.11477388,339.87726336)
\curveto(627.14477333,339.94726097)(627.19477328,339.99726092)(627.26477388,340.02726336)
\curveto(627.31477316,340.04726087)(627.3747731,340.05726086)(627.44477388,340.05726336)
\lineto(627.66977388,340.05726336)
\lineto(628.37477388,340.05726336)
\lineto(628.61477388,340.05726336)
\curveto(628.69477178,340.05726086)(628.76477171,340.04726087)(628.82477388,340.02726336)
\curveto(628.93477154,339.98726093)(629.00477147,339.92226099)(629.03477388,339.83226336)
\curveto(629.0747714,339.74226117)(629.11977135,339.64726127)(629.16977388,339.54726336)
\curveto(629.18977128,339.49726142)(629.22477125,339.43226148)(629.27477388,339.35226336)
\curveto(629.33477114,339.27226164)(629.38477109,339.22226169)(629.42477388,339.20226336)
\curveto(629.54477093,339.10226181)(629.65977081,339.02226189)(629.76977388,338.96226336)
\curveto(629.87977059,338.912262)(630.01977045,338.86226205)(630.18977388,338.81226336)
\curveto(630.23977023,338.79226212)(630.28977018,338.78226213)(630.33977388,338.78226336)
\curveto(630.38977008,338.79226212)(630.43977003,338.79226212)(630.48977388,338.78226336)
\curveto(630.5697699,338.76226215)(630.65476982,338.75226216)(630.74477388,338.75226336)
\curveto(630.84476963,338.76226215)(630.92976954,338.77726214)(630.99977388,338.79726336)
\curveto(631.04976942,338.80726211)(631.09476938,338.8122621)(631.13477388,338.81226336)
\curveto(631.18476929,338.8122621)(631.23476924,338.82226209)(631.28477388,338.84226336)
\curveto(631.42476905,338.89226202)(631.54976892,338.95226196)(631.65977388,339.02226336)
\curveto(631.77976869,339.09226182)(631.8747686,339.18226173)(631.94477388,339.29226336)
\curveto(631.99476848,339.37226154)(632.03476844,339.49726142)(632.06477388,339.66726336)
\curveto(632.08476839,339.73726118)(632.08476839,339.80226111)(632.06477388,339.86226336)
\curveto(632.04476843,339.92226099)(632.02476845,339.97226094)(632.00477388,340.01226336)
\curveto(631.93476854,340.15226076)(631.84476863,340.25726066)(631.73477388,340.32726336)
\curveto(631.63476884,340.39726052)(631.51476896,340.46226045)(631.37477388,340.52226336)
\curveto(631.18476929,340.60226031)(630.98476949,340.66726025)(630.77477388,340.71726336)
\curveto(630.56476991,340.76726015)(630.35477012,340.82226009)(630.14477388,340.88226336)
\curveto(630.06477041,340.90226001)(629.97977049,340.91726)(629.88977388,340.92726336)
\curveto(629.80977066,340.93725998)(629.72977074,340.95225996)(629.64977388,340.97226336)
\curveto(629.32977114,341.06225985)(629.02477145,341.14725977)(628.73477388,341.22726336)
\curveto(628.44477203,341.3172596)(628.17977229,341.44725947)(627.93977388,341.61726336)
\curveto(627.65977281,341.8172591)(627.45477302,342.08725883)(627.32477388,342.42726336)
\curveto(627.30477317,342.49725842)(627.28477319,342.59225832)(627.26477388,342.71226336)
\curveto(627.24477323,342.78225813)(627.22977324,342.86725805)(627.21977388,342.96726336)
\curveto(627.20977326,343.06725785)(627.21477326,343.15725776)(627.23477388,343.23726336)
\curveto(627.25477322,343.28725763)(627.25977321,343.32725759)(627.24977388,343.35726336)
\curveto(627.23977323,343.39725752)(627.24477323,343.44225747)(627.26477388,343.49226336)
\curveto(627.28477319,343.60225731)(627.30477317,343.70225721)(627.32477388,343.79226336)
\curveto(627.35477312,343.89225702)(627.38977308,343.98725693)(627.42977388,344.07726336)
\curveto(627.55977291,344.36725655)(627.73977273,344.60225631)(627.96977388,344.78226336)
\curveto(628.19977227,344.96225595)(628.45977201,345.10725581)(628.74977388,345.21726336)
\curveto(628.85977161,345.26725565)(628.9747715,345.30225561)(629.09477388,345.32226336)
\curveto(629.21477126,345.35225556)(629.33977113,345.38225553)(629.46977388,345.41226336)
\curveto(629.52977094,345.43225548)(629.58977088,345.44225547)(629.64977388,345.44226336)
\lineto(629.82977388,345.47226336)
\curveto(629.90977056,345.48225543)(629.99477048,345.48725543)(630.08477388,345.48726336)
\curveto(630.1747703,345.48725543)(630.25977021,345.49225542)(630.33977388,345.50226336)
}
}
{
\newrgbcolor{curcolor}{0 0 0}
\pscustom[linestyle=none,fillstyle=solid,fillcolor=curcolor]
{
\newpath
\moveto(413.39453951,322.76226336)
\curveto(413.40453083,322.70225946)(413.40953082,322.61225955)(413.40953951,322.49226336)
\curveto(413.40953082,322.37225979)(413.39953083,322.28725988)(413.37953951,322.23726336)
\lineto(413.37953951,322.04226336)
\curveto(413.34953088,321.93226023)(413.3295309,321.82726034)(413.31953951,321.72726336)
\curveto(413.31953091,321.62726054)(413.30453093,321.52726064)(413.27453951,321.42726336)
\curveto(413.25453098,321.33726083)(413.234531,321.24226092)(413.21453951,321.14226336)
\curveto(413.19453104,321.05226111)(413.16453107,320.9622612)(413.12453951,320.87226336)
\curveto(413.05453118,320.70226146)(412.98453125,320.54226162)(412.91453951,320.39226336)
\curveto(412.84453139,320.25226191)(412.76453147,320.11226205)(412.67453951,319.97226336)
\curveto(412.61453162,319.88226228)(412.54953168,319.79726237)(412.47953951,319.71726336)
\curveto(412.41953181,319.64726252)(412.34953188,319.57226259)(412.26953951,319.49226336)
\lineto(412.16453951,319.38726336)
\curveto(412.11453212,319.33726283)(412.05953217,319.29226287)(411.99953951,319.25226336)
\lineto(411.84953951,319.13226336)
\curveto(411.76953246,319.07226309)(411.67953255,319.01726315)(411.57953951,318.96726336)
\curveto(411.48953274,318.92726324)(411.39453284,318.88226328)(411.29453951,318.83226336)
\curveto(411.19453304,318.78226338)(411.08953314,318.74726342)(410.97953951,318.72726336)
\curveto(410.87953335,318.70726346)(410.77453346,318.68726348)(410.66453951,318.66726336)
\curveto(410.60453363,318.64726352)(410.53953369,318.63726353)(410.46953951,318.63726336)
\curveto(410.40953382,318.63726353)(410.34453389,318.62726354)(410.27453951,318.60726336)
\lineto(410.13953951,318.60726336)
\curveto(410.05953417,318.58726358)(409.98453425,318.58726358)(409.91453951,318.60726336)
\lineto(409.76453951,318.60726336)
\curveto(409.70453453,318.62726354)(409.63953459,318.63726353)(409.56953951,318.63726336)
\curveto(409.50953472,318.62726354)(409.44953478,318.63226353)(409.38953951,318.65226336)
\curveto(409.229535,318.70226346)(409.07453516,318.74726342)(408.92453951,318.78726336)
\curveto(408.78453545,318.82726334)(408.65453558,318.88726328)(408.53453951,318.96726336)
\curveto(408.46453577,319.00726316)(408.39953583,319.04726312)(408.33953951,319.08726336)
\curveto(408.27953595,319.13726303)(408.21453602,319.18726298)(408.14453951,319.23726336)
\lineto(407.96453951,319.37226336)
\curveto(407.88453635,319.43226273)(407.81453642,319.43726273)(407.75453951,319.38726336)
\curveto(407.70453653,319.35726281)(407.67953655,319.31726285)(407.67953951,319.26726336)
\curveto(407.67953655,319.22726294)(407.66953656,319.17726299)(407.64953951,319.11726336)
\curveto(407.6295366,319.01726315)(407.61953661,318.90226326)(407.61953951,318.77226336)
\curveto(407.6295366,318.64226352)(407.6345366,318.52226364)(407.63453951,318.41226336)
\lineto(407.63453951,316.88226336)
\curveto(407.6345366,316.75226541)(407.6295366,316.62726554)(407.61953951,316.50726336)
\curveto(407.61953661,316.37726579)(407.59453664,316.27226589)(407.54453951,316.19226336)
\curveto(407.51453672,316.15226601)(407.45953677,316.12226604)(407.37953951,316.10226336)
\curveto(407.29953693,316.08226608)(407.20953702,316.07226609)(407.10953951,316.07226336)
\curveto(407.00953722,316.0622661)(406.90953732,316.0622661)(406.80953951,316.07226336)
\lineto(406.55453951,316.07226336)
\lineto(406.14953951,316.07226336)
\lineto(406.04453951,316.07226336)
\curveto(406.00453823,316.07226609)(405.96953826,316.07726609)(405.93953951,316.08726336)
\lineto(405.81953951,316.08726336)
\curveto(405.64953858,316.13726603)(405.55953867,316.23726593)(405.54953951,316.38726336)
\curveto(405.53953869,316.52726564)(405.5345387,316.69726547)(405.53453951,316.89726336)
\lineto(405.53453951,325.70226336)
\curveto(405.5345387,325.81225635)(405.5295387,325.92725624)(405.51953951,326.04726336)
\curveto(405.51953871,326.17725599)(405.54453869,326.27725589)(405.59453951,326.34726336)
\curveto(405.6345386,326.41725575)(405.68953854,326.4622557)(405.75953951,326.48226336)
\curveto(405.80953842,326.50225566)(405.86953836,326.51225565)(405.93953951,326.51226336)
\lineto(406.16453951,326.51226336)
\lineto(406.88453951,326.51226336)
\lineto(407.16953951,326.51226336)
\curveto(407.25953697,326.51225565)(407.3345369,326.48725568)(407.39453951,326.43726336)
\curveto(407.46453677,326.38725578)(407.49953673,326.32225584)(407.49953951,326.24226336)
\curveto(407.50953672,326.17225599)(407.5345367,326.09725607)(407.57453951,326.01726336)
\curveto(407.58453665,325.98725618)(407.59453664,325.9622562)(407.60453951,325.94226336)
\curveto(407.62453661,325.93225623)(407.64453659,325.91725625)(407.66453951,325.89726336)
\curveto(407.77453646,325.88725628)(407.86453637,325.91725625)(407.93453951,325.98726336)
\curveto(408.00453623,326.05725611)(408.07453616,326.11725605)(408.14453951,326.16726336)
\curveto(408.27453596,326.25725591)(408.40953582,326.33725583)(408.54953951,326.40726336)
\curveto(408.68953554,326.48725568)(408.84453539,326.55225561)(409.01453951,326.60226336)
\curveto(409.09453514,326.63225553)(409.17953505,326.65225551)(409.26953951,326.66226336)
\curveto(409.36953486,326.67225549)(409.46453477,326.68725548)(409.55453951,326.70726336)
\curveto(409.59453464,326.71725545)(409.6345346,326.71725545)(409.67453951,326.70726336)
\curveto(409.72453451,326.69725547)(409.76453447,326.70225546)(409.79453951,326.72226336)
\curveto(410.36453387,326.74225542)(410.84453339,326.6622555)(411.23453951,326.48226336)
\curveto(411.6345326,326.31225585)(411.97453226,326.08725608)(412.25453951,325.80726336)
\curveto(412.30453193,325.75725641)(412.34953188,325.70725646)(412.38953951,325.65726336)
\curveto(412.4295318,325.61725655)(412.46953176,325.57225659)(412.50953951,325.52226336)
\curveto(412.57953165,325.43225673)(412.63953159,325.34225682)(412.68953951,325.25226336)
\curveto(412.74953148,325.162257)(412.80453143,325.07225709)(412.85453951,324.98226336)
\curveto(412.87453136,324.9622572)(412.88453135,324.93725723)(412.88453951,324.90726336)
\curveto(412.89453134,324.87725729)(412.90953132,324.84225732)(412.92953951,324.80226336)
\curveto(412.98953124,324.70225746)(413.04453119,324.58225758)(413.09453951,324.44226336)
\curveto(413.11453112,324.38225778)(413.1345311,324.31725785)(413.15453951,324.24726336)
\curveto(413.17453106,324.18725798)(413.19453104,324.12225804)(413.21453951,324.05226336)
\curveto(413.25453098,323.93225823)(413.27953095,323.80725836)(413.28953951,323.67726336)
\curveto(413.30953092,323.54725862)(413.3345309,323.41225875)(413.36453951,323.27226336)
\lineto(413.36453951,323.10726336)
\lineto(413.39453951,322.92726336)
\lineto(413.39453951,322.76226336)
\moveto(411.27953951,322.41726336)
\curveto(411.28953294,322.4672597)(411.29453294,322.53225963)(411.29453951,322.61226336)
\curveto(411.29453294,322.70225946)(411.28953294,322.77225939)(411.27953951,322.82226336)
\lineto(411.27953951,322.95726336)
\curveto(411.25953297,323.01725915)(411.24953298,323.08225908)(411.24953951,323.15226336)
\curveto(411.24953298,323.22225894)(411.23953299,323.29225887)(411.21953951,323.36226336)
\curveto(411.19953303,323.4622587)(411.17953305,323.55725861)(411.15953951,323.64726336)
\curveto(411.13953309,323.74725842)(411.10953312,323.83725833)(411.06953951,323.91726336)
\curveto(410.94953328,324.23725793)(410.79453344,324.49225767)(410.60453951,324.68226336)
\curveto(410.41453382,324.87225729)(410.14453409,325.01225715)(409.79453951,325.10226336)
\curveto(409.71453452,325.12225704)(409.62453461,325.13225703)(409.52453951,325.13226336)
\lineto(409.25453951,325.13226336)
\curveto(409.21453502,325.12225704)(409.17953505,325.11725705)(409.14953951,325.11726336)
\curveto(409.11953511,325.11725705)(409.08453515,325.11225705)(409.04453951,325.10226336)
\lineto(408.83453951,325.04226336)
\curveto(408.77453546,325.03225713)(408.71453552,325.01225715)(408.65453951,324.98226336)
\curveto(408.39453584,324.87225729)(408.18953604,324.70225746)(408.03953951,324.47226336)
\curveto(407.89953633,324.24225792)(407.78453645,323.98725818)(407.69453951,323.70726336)
\curveto(407.67453656,323.62725854)(407.65953657,323.54225862)(407.64953951,323.45226336)
\curveto(407.63953659,323.37225879)(407.62453661,323.29225887)(407.60453951,323.21226336)
\curveto(407.59453664,323.17225899)(407.58953664,323.10725906)(407.58953951,323.01726336)
\curveto(407.56953666,322.97725919)(407.56453667,322.92725924)(407.57453951,322.86726336)
\curveto(407.58453665,322.81725935)(407.58453665,322.7672594)(407.57453951,322.71726336)
\curveto(407.55453668,322.65725951)(407.55453668,322.60225956)(407.57453951,322.55226336)
\lineto(407.57453951,322.37226336)
\lineto(407.57453951,322.23726336)
\curveto(407.57453666,322.19725997)(407.58453665,322.15726001)(407.60453951,322.11726336)
\curveto(407.60453663,322.04726012)(407.60953662,321.99226017)(407.61953951,321.95226336)
\lineto(407.64953951,321.77226336)
\curveto(407.65953657,321.71226045)(407.67453656,321.65226051)(407.69453951,321.59226336)
\curveto(407.78453645,321.30226086)(407.88953634,321.0622611)(408.00953951,320.87226336)
\curveto(408.13953609,320.69226147)(408.31953591,320.53226163)(408.54953951,320.39226336)
\curveto(408.68953554,320.31226185)(408.85453538,320.24726192)(409.04453951,320.19726336)
\curveto(409.08453515,320.18726198)(409.11953511,320.18226198)(409.14953951,320.18226336)
\curveto(409.17953505,320.19226197)(409.21453502,320.19226197)(409.25453951,320.18226336)
\curveto(409.29453494,320.17226199)(409.35453488,320.162262)(409.43453951,320.15226336)
\curveto(409.51453472,320.15226201)(409.57953465,320.15726201)(409.62953951,320.16726336)
\curveto(409.70953452,320.18726198)(409.78953444,320.20226196)(409.86953951,320.21226336)
\curveto(409.95953427,320.23226193)(410.04453419,320.25726191)(410.12453951,320.28726336)
\curveto(410.36453387,320.38726178)(410.55953367,320.52726164)(410.70953951,320.70726336)
\curveto(410.85953337,320.88726128)(410.98453325,321.09726107)(411.08453951,321.33726336)
\curveto(411.1345331,321.45726071)(411.16953306,321.58226058)(411.18953951,321.71226336)
\curveto(411.20953302,321.84226032)(411.234533,321.97726019)(411.26453951,322.11726336)
\lineto(411.26453951,322.26726336)
\curveto(411.27453296,322.31725985)(411.27953295,322.3672598)(411.27953951,322.41726336)
}
}
{
\newrgbcolor{curcolor}{0 0 0}
\pscustom[linestyle=none,fillstyle=solid,fillcolor=curcolor]
{
\newpath
\moveto(422.44446138,322.98726336)
\curveto(422.46445281,322.92725924)(422.4744528,322.84225932)(422.47446138,322.73226336)
\curveto(422.4744528,322.62225954)(422.46445281,322.53725963)(422.44446138,322.47726336)
\lineto(422.44446138,322.32726336)
\curveto(422.42445285,322.24725992)(422.41445286,322.16726)(422.41446138,322.08726336)
\curveto(422.42445285,322.00726016)(422.41945286,321.92726024)(422.39946138,321.84726336)
\curveto(422.3794529,321.77726039)(422.36445291,321.71226045)(422.35446138,321.65226336)
\curveto(422.34445293,321.59226057)(422.33445294,321.52726064)(422.32446138,321.45726336)
\curveto(422.28445299,321.34726082)(422.24945303,321.23226093)(422.21946138,321.11226336)
\curveto(422.18945309,321.00226116)(422.14945313,320.89726127)(422.09946138,320.79726336)
\curveto(421.88945339,320.31726185)(421.61445366,319.92726224)(421.27446138,319.62726336)
\curveto(420.93445434,319.32726284)(420.52445475,319.07726309)(420.04446138,318.87726336)
\curveto(419.92445535,318.82726334)(419.79945548,318.79226337)(419.66946138,318.77226336)
\curveto(419.54945573,318.74226342)(419.42445585,318.71226345)(419.29446138,318.68226336)
\curveto(419.24445603,318.6622635)(419.18945609,318.65226351)(419.12946138,318.65226336)
\curveto(419.06945621,318.65226351)(419.01445626,318.64726352)(418.96446138,318.63726336)
\lineto(418.85946138,318.63726336)
\curveto(418.82945645,318.62726354)(418.79945648,318.62226354)(418.76946138,318.62226336)
\curveto(418.71945656,318.61226355)(418.63945664,318.60726356)(418.52946138,318.60726336)
\curveto(418.41945686,318.59726357)(418.33445694,318.60226356)(418.27446138,318.62226336)
\lineto(418.12446138,318.62226336)
\curveto(418.0744572,318.63226353)(418.01945726,318.63726353)(417.95946138,318.63726336)
\curveto(417.90945737,318.62726354)(417.85945742,318.63226353)(417.80946138,318.65226336)
\curveto(417.76945751,318.6622635)(417.72945755,318.6672635)(417.68946138,318.66726336)
\curveto(417.65945762,318.6672635)(417.61945766,318.67226349)(417.56946138,318.68226336)
\curveto(417.46945781,318.71226345)(417.36945791,318.73726343)(417.26946138,318.75726336)
\curveto(417.16945811,318.77726339)(417.0744582,318.80726336)(416.98446138,318.84726336)
\curveto(416.86445841,318.88726328)(416.74945853,318.92726324)(416.63946138,318.96726336)
\curveto(416.53945874,319.00726316)(416.43445884,319.05726311)(416.32446138,319.11726336)
\curveto(415.9744593,319.32726284)(415.6744596,319.57226259)(415.42446138,319.85226336)
\curveto(415.1744601,320.13226203)(414.96446031,320.4672617)(414.79446138,320.85726336)
\curveto(414.74446053,320.94726122)(414.70446057,321.04226112)(414.67446138,321.14226336)
\curveto(414.65446062,321.24226092)(414.62946065,321.34726082)(414.59946138,321.45726336)
\curveto(414.5794607,321.50726066)(414.56946071,321.55226061)(414.56946138,321.59226336)
\curveto(414.56946071,321.63226053)(414.55946072,321.67726049)(414.53946138,321.72726336)
\curveto(414.51946076,321.80726036)(414.50946077,321.88726028)(414.50946138,321.96726336)
\curveto(414.50946077,322.05726011)(414.49946078,322.14226002)(414.47946138,322.22226336)
\curveto(414.46946081,322.27225989)(414.46446081,322.31725985)(414.46446138,322.35726336)
\lineto(414.46446138,322.49226336)
\curveto(414.44446083,322.55225961)(414.43446084,322.63725953)(414.43446138,322.74726336)
\curveto(414.44446083,322.85725931)(414.45946082,322.94225922)(414.47946138,323.00226336)
\lineto(414.47946138,323.10726336)
\curveto(414.48946079,323.15725901)(414.48946079,323.20725896)(414.47946138,323.25726336)
\curveto(414.4794608,323.31725885)(414.48946079,323.37225879)(414.50946138,323.42226336)
\curveto(414.51946076,323.47225869)(414.52446075,323.51725865)(414.52446138,323.55726336)
\curveto(414.52446075,323.60725856)(414.53446074,323.65725851)(414.55446138,323.70726336)
\curveto(414.59446068,323.83725833)(414.62946065,323.9622582)(414.65946138,324.08226336)
\curveto(414.68946059,324.21225795)(414.72946055,324.33725783)(414.77946138,324.45726336)
\curveto(414.95946032,324.8672573)(415.1744601,325.20725696)(415.42446138,325.47726336)
\curveto(415.6744596,325.75725641)(415.9794593,326.01225615)(416.33946138,326.24226336)
\curveto(416.43945884,326.29225587)(416.54445873,326.33725583)(416.65446138,326.37726336)
\curveto(416.76445851,326.41725575)(416.8744584,326.4622557)(416.98446138,326.51226336)
\curveto(417.11445816,326.5622556)(417.24945803,326.59725557)(417.38946138,326.61726336)
\curveto(417.52945775,326.63725553)(417.6744576,326.6672555)(417.82446138,326.70726336)
\curveto(417.90445737,326.71725545)(417.9794573,326.72225544)(418.04946138,326.72226336)
\curveto(418.11945716,326.72225544)(418.18945709,326.72725544)(418.25946138,326.73726336)
\curveto(418.83945644,326.74725542)(419.33945594,326.68725548)(419.75946138,326.55726336)
\curveto(420.18945509,326.42725574)(420.56945471,326.24725592)(420.89946138,326.01726336)
\curveto(421.00945427,325.93725623)(421.11945416,325.84725632)(421.22946138,325.74726336)
\curveto(421.34945393,325.65725651)(421.44945383,325.55725661)(421.52946138,325.44726336)
\curveto(421.60945367,325.34725682)(421.6794536,325.24725692)(421.73946138,325.14726336)
\curveto(421.80945347,325.04725712)(421.8794534,324.94225722)(421.94946138,324.83226336)
\curveto(422.01945326,324.72225744)(422.0744532,324.60225756)(422.11446138,324.47226336)
\curveto(422.15445312,324.35225781)(422.19945308,324.22225794)(422.24946138,324.08226336)
\curveto(422.279453,324.00225816)(422.30445297,323.91725825)(422.32446138,323.82726336)
\lineto(422.38446138,323.55726336)
\curveto(422.39445288,323.51725865)(422.39945288,323.47725869)(422.39946138,323.43726336)
\curveto(422.39945288,323.39725877)(422.40445287,323.35725881)(422.41446138,323.31726336)
\curveto(422.43445284,323.2672589)(422.43945284,323.21225895)(422.42946138,323.15226336)
\curveto(422.41945286,323.09225907)(422.42445285,323.03725913)(422.44446138,322.98726336)
\moveto(420.34446138,322.44726336)
\curveto(420.35445492,322.49725967)(420.35945492,322.5672596)(420.35946138,322.65726336)
\curveto(420.35945492,322.75725941)(420.35445492,322.83225933)(420.34446138,322.88226336)
\lineto(420.34446138,323.00226336)
\curveto(420.32445495,323.05225911)(420.31445496,323.10725906)(420.31446138,323.16726336)
\curveto(420.31445496,323.22725894)(420.30945497,323.28225888)(420.29946138,323.33226336)
\curveto(420.29945498,323.37225879)(420.29445498,323.40225876)(420.28446138,323.42226336)
\lineto(420.22446138,323.66226336)
\curveto(420.21445506,323.75225841)(420.19445508,323.83725833)(420.16446138,323.91726336)
\curveto(420.05445522,324.17725799)(419.92445535,324.39725777)(419.77446138,324.57726336)
\curveto(419.62445565,324.7672574)(419.42445585,324.91725725)(419.17446138,325.02726336)
\curveto(419.11445616,325.04725712)(419.05445622,325.0622571)(418.99446138,325.07226336)
\curveto(418.93445634,325.09225707)(418.86945641,325.11225705)(418.79946138,325.13226336)
\curveto(418.71945656,325.15225701)(418.63445664,325.15725701)(418.54446138,325.14726336)
\lineto(418.27446138,325.14726336)
\curveto(418.24445703,325.12725704)(418.20945707,325.11725705)(418.16946138,325.11726336)
\curveto(418.12945715,325.12725704)(418.09445718,325.12725704)(418.06446138,325.11726336)
\lineto(417.85446138,325.05726336)
\curveto(417.79445748,325.04725712)(417.73945754,325.02725714)(417.68946138,324.99726336)
\curveto(417.43945784,324.88725728)(417.23445804,324.72725744)(417.07446138,324.51726336)
\curveto(416.92445835,324.31725785)(416.80445847,324.08225808)(416.71446138,323.81226336)
\curveto(416.68445859,323.71225845)(416.65945862,323.60725856)(416.63946138,323.49726336)
\curveto(416.62945865,323.38725878)(416.61445866,323.27725889)(416.59446138,323.16726336)
\curveto(416.58445869,323.11725905)(416.5794587,323.0672591)(416.57946138,323.01726336)
\lineto(416.57946138,322.86726336)
\curveto(416.55945872,322.79725937)(416.54945873,322.69225947)(416.54946138,322.55226336)
\curveto(416.55945872,322.41225975)(416.5744587,322.30725986)(416.59446138,322.23726336)
\lineto(416.59446138,322.10226336)
\curveto(416.61445866,322.02226014)(416.62945865,321.94226022)(416.63946138,321.86226336)
\curveto(416.64945863,321.79226037)(416.66445861,321.71726045)(416.68446138,321.63726336)
\curveto(416.78445849,321.33726083)(416.88945839,321.09226107)(416.99946138,320.90226336)
\curveto(417.11945816,320.72226144)(417.30445797,320.55726161)(417.55446138,320.40726336)
\curveto(417.62445765,320.35726181)(417.69945758,320.31726185)(417.77946138,320.28726336)
\curveto(417.86945741,320.25726191)(417.95945732,320.23226193)(418.04946138,320.21226336)
\curveto(418.08945719,320.20226196)(418.12445715,320.19726197)(418.15446138,320.19726336)
\curveto(418.18445709,320.20726196)(418.21945706,320.20726196)(418.25946138,320.19726336)
\lineto(418.37946138,320.16726336)
\curveto(418.42945685,320.167262)(418.4744568,320.17226199)(418.51446138,320.18226336)
\lineto(418.63446138,320.18226336)
\curveto(418.71445656,320.20226196)(418.79445648,320.21726195)(418.87446138,320.22726336)
\curveto(418.95445632,320.23726193)(419.02945625,320.25726191)(419.09946138,320.28726336)
\curveto(419.35945592,320.38726178)(419.56945571,320.52226164)(419.72946138,320.69226336)
\curveto(419.88945539,320.8622613)(420.02445525,321.07226109)(420.13446138,321.32226336)
\curveto(420.1744551,321.42226074)(420.20445507,321.52226064)(420.22446138,321.62226336)
\curveto(420.24445503,321.72226044)(420.26945501,321.82726034)(420.29946138,321.93726336)
\curveto(420.30945497,321.97726019)(420.31445496,322.01226015)(420.31446138,322.04226336)
\curveto(420.31445496,322.08226008)(420.31945496,322.12226004)(420.32946138,322.16226336)
\lineto(420.32946138,322.29726336)
\curveto(420.32945495,322.34725982)(420.33445494,322.39725977)(420.34446138,322.44726336)
}
}
{
\newrgbcolor{curcolor}{0 0 0}
\pscustom[linestyle=none,fillstyle=solid,fillcolor=curcolor]
{
\newpath
\moveto(428.26938326,326.73726336)
\curveto(428.37937794,326.73725543)(428.47437785,326.72725544)(428.55438326,326.70726336)
\curveto(428.64437768,326.68725548)(428.71437761,326.64225552)(428.76438326,326.57226336)
\curveto(428.8243775,326.49225567)(428.85437747,326.35225581)(428.85438326,326.15226336)
\lineto(428.85438326,325.64226336)
\lineto(428.85438326,325.26726336)
\curveto(428.86437746,325.12725704)(428.84937747,325.01725715)(428.80938326,324.93726336)
\curveto(428.76937755,324.8672573)(428.70937761,324.82225734)(428.62938326,324.80226336)
\curveto(428.55937776,324.78225738)(428.47437785,324.77225739)(428.37438326,324.77226336)
\curveto(428.28437804,324.77225739)(428.18437814,324.77725739)(428.07438326,324.78726336)
\curveto(427.97437835,324.79725737)(427.87937844,324.79225737)(427.78938326,324.77226336)
\curveto(427.7193786,324.75225741)(427.64937867,324.73725743)(427.57938326,324.72726336)
\curveto(427.50937881,324.72725744)(427.44437888,324.71725745)(427.38438326,324.69726336)
\curveto(427.2243791,324.64725752)(427.06437926,324.57225759)(426.90438326,324.47226336)
\curveto(426.74437958,324.38225778)(426.6193797,324.27725789)(426.52938326,324.15726336)
\curveto(426.47937984,324.07725809)(426.4243799,323.99225817)(426.36438326,323.90226336)
\curveto(426.31438001,323.82225834)(426.26438006,323.73725843)(426.21438326,323.64726336)
\curveto(426.18438014,323.5672586)(426.15438017,323.48225868)(426.12438326,323.39226336)
\lineto(426.06438326,323.15226336)
\curveto(426.04438028,323.08225908)(426.03438029,323.00725916)(426.03438326,322.92726336)
\curveto(426.03438029,322.85725931)(426.0243803,322.78725938)(426.00438326,322.71726336)
\curveto(425.99438033,322.67725949)(425.98938033,322.63725953)(425.98938326,322.59726336)
\curveto(425.99938032,322.5672596)(425.99938032,322.53725963)(425.98938326,322.50726336)
\lineto(425.98938326,322.26726336)
\curveto(425.96938035,322.19725997)(425.96438036,322.11726005)(425.97438326,322.02726336)
\curveto(425.98438034,321.94726022)(425.98938033,321.8672603)(425.98938326,321.78726336)
\lineto(425.98938326,320.82726336)
\lineto(425.98938326,319.55226336)
\curveto(425.98938033,319.42226274)(425.98438034,319.30226286)(425.97438326,319.19226336)
\curveto(425.96438036,319.08226308)(425.93438039,318.99226317)(425.88438326,318.92226336)
\curveto(425.86438046,318.89226327)(425.82938049,318.8672633)(425.77938326,318.84726336)
\curveto(425.73938058,318.83726333)(425.69438063,318.82726334)(425.64438326,318.81726336)
\lineto(425.56938326,318.81726336)
\curveto(425.5193808,318.80726336)(425.46438086,318.80226336)(425.40438326,318.80226336)
\lineto(425.23938326,318.80226336)
\lineto(424.59438326,318.80226336)
\curveto(424.53438179,318.81226335)(424.46938185,318.81726335)(424.39938326,318.81726336)
\lineto(424.20438326,318.81726336)
\curveto(424.15438217,318.83726333)(424.10438222,318.85226331)(424.05438326,318.86226336)
\curveto(424.00438232,318.88226328)(423.96938235,318.91726325)(423.94938326,318.96726336)
\curveto(423.90938241,319.01726315)(423.88438244,319.08726308)(423.87438326,319.17726336)
\lineto(423.87438326,319.47726336)
\lineto(423.87438326,320.49726336)
\lineto(423.87438326,324.72726336)
\lineto(423.87438326,325.83726336)
\lineto(423.87438326,326.12226336)
\curveto(423.87438245,326.22225594)(423.89438243,326.30225586)(423.93438326,326.36226336)
\curveto(423.98438234,326.44225572)(424.05938226,326.49225567)(424.15938326,326.51226336)
\curveto(424.25938206,326.53225563)(424.37938194,326.54225562)(424.51938326,326.54226336)
\lineto(425.28438326,326.54226336)
\curveto(425.40438092,326.54225562)(425.50938081,326.53225563)(425.59938326,326.51226336)
\curveto(425.68938063,326.50225566)(425.75938056,326.45725571)(425.80938326,326.37726336)
\curveto(425.83938048,326.32725584)(425.85438047,326.25725591)(425.85438326,326.16726336)
\lineto(425.88438326,325.89726336)
\curveto(425.89438043,325.81725635)(425.90938041,325.74225642)(425.92938326,325.67226336)
\curveto(425.95938036,325.60225656)(426.00938031,325.5672566)(426.07938326,325.56726336)
\curveto(426.09938022,325.58725658)(426.1193802,325.59725657)(426.13938326,325.59726336)
\curveto(426.15938016,325.59725657)(426.17938014,325.60725656)(426.19938326,325.62726336)
\curveto(426.25938006,325.67725649)(426.30938001,325.73225643)(426.34938326,325.79226336)
\curveto(426.39937992,325.8622563)(426.45937986,325.92225624)(426.52938326,325.97226336)
\curveto(426.56937975,326.00225616)(426.60437972,326.03225613)(426.63438326,326.06226336)
\curveto(426.66437966,326.10225606)(426.69937962,326.13725603)(426.73938326,326.16726336)
\lineto(427.00938326,326.34726336)
\curveto(427.10937921,326.40725576)(427.20937911,326.4622557)(427.30938326,326.51226336)
\curveto(427.40937891,326.55225561)(427.50937881,326.58725558)(427.60938326,326.61726336)
\lineto(427.93938326,326.70726336)
\curveto(427.96937835,326.71725545)(428.0243783,326.71725545)(428.10438326,326.70726336)
\curveto(428.19437813,326.70725546)(428.24937807,326.71725545)(428.26938326,326.73726336)
}
}
{
\newrgbcolor{curcolor}{0 0 0}
\pscustom[linestyle=none,fillstyle=solid,fillcolor=curcolor]
{
}
}
{
\newrgbcolor{curcolor}{0 0 0}
\pscustom[linestyle=none,fillstyle=solid,fillcolor=curcolor]
{
\newpath
\moveto(434.88461763,328.85226336)
\lineto(435.88961763,328.85226336)
\curveto(436.03961465,328.85225331)(436.16961452,328.84225332)(436.27961763,328.82226336)
\curveto(436.39961429,328.81225335)(436.4846142,328.75225341)(436.53461763,328.64226336)
\curveto(436.55461413,328.59225357)(436.56461412,328.53225363)(436.56461763,328.46226336)
\lineto(436.56461763,328.25226336)
\lineto(436.56461763,327.57726336)
\curveto(436.56461412,327.52725464)(436.55961413,327.4672547)(436.54961763,327.39726336)
\curveto(436.54961414,327.33725483)(436.55461413,327.28225488)(436.56461763,327.23226336)
\lineto(436.56461763,327.06726336)
\curveto(436.56461412,326.98725518)(436.56961412,326.91225525)(436.57961763,326.84226336)
\curveto(436.5896141,326.78225538)(436.61461407,326.72725544)(436.65461763,326.67726336)
\curveto(436.72461396,326.58725558)(436.84961384,326.53725563)(437.02961763,326.52726336)
\lineto(437.56961763,326.52726336)
\lineto(437.74961763,326.52726336)
\curveto(437.80961288,326.52725564)(437.86461282,326.51725565)(437.91461763,326.49726336)
\curveto(438.02461266,326.44725572)(438.0846126,326.35725581)(438.09461763,326.22726336)
\curveto(438.11461257,326.09725607)(438.12461256,325.95225621)(438.12461763,325.79226336)
\lineto(438.12461763,325.58226336)
\curveto(438.13461255,325.51225665)(438.12961256,325.45225671)(438.10961763,325.40226336)
\curveto(438.05961263,325.24225692)(437.95461273,325.15725701)(437.79461763,325.14726336)
\curveto(437.63461305,325.13725703)(437.45461323,325.13225703)(437.25461763,325.13226336)
\lineto(437.11961763,325.13226336)
\curveto(437.07961361,325.14225702)(437.04461364,325.14225702)(437.01461763,325.13226336)
\curveto(436.97461371,325.12225704)(436.93961375,325.11725705)(436.90961763,325.11726336)
\curveto(436.87961381,325.12725704)(436.84961384,325.12225704)(436.81961763,325.10226336)
\curveto(436.73961395,325.08225708)(436.67961401,325.03725713)(436.63961763,324.96726336)
\curveto(436.60961408,324.90725726)(436.5846141,324.83225733)(436.56461763,324.74226336)
\curveto(436.55461413,324.69225747)(436.55461413,324.63725753)(436.56461763,324.57726336)
\curveto(436.57461411,324.51725765)(436.57461411,324.4622577)(436.56461763,324.41226336)
\lineto(436.56461763,323.48226336)
\lineto(436.56461763,321.72726336)
\curveto(436.56461412,321.47726069)(436.56961412,321.25726091)(436.57961763,321.06726336)
\curveto(436.59961409,320.88726128)(436.66461402,320.72726144)(436.77461763,320.58726336)
\curveto(436.82461386,320.52726164)(436.8896138,320.48226168)(436.96961763,320.45226336)
\lineto(437.23961763,320.39226336)
\curveto(437.26961342,320.38226178)(437.29961339,320.37726179)(437.32961763,320.37726336)
\curveto(437.36961332,320.38726178)(437.39961329,320.38726178)(437.41961763,320.37726336)
\lineto(437.58461763,320.37726336)
\curveto(437.69461299,320.37726179)(437.7896129,320.37226179)(437.86961763,320.36226336)
\curveto(437.94961274,320.35226181)(438.01461267,320.31226185)(438.06461763,320.24226336)
\curveto(438.10461258,320.18226198)(438.12461256,320.10226206)(438.12461763,320.00226336)
\lineto(438.12461763,319.71726336)
\curveto(438.12461256,319.50726266)(438.11961257,319.31226285)(438.10961763,319.13226336)
\curveto(438.10961258,318.9622632)(438.02961266,318.84726332)(437.86961763,318.78726336)
\curveto(437.81961287,318.7672634)(437.77461291,318.7622634)(437.73461763,318.77226336)
\curveto(437.69461299,318.77226339)(437.64961304,318.7622634)(437.59961763,318.74226336)
\lineto(437.44961763,318.74226336)
\curveto(437.42961326,318.74226342)(437.39961329,318.74726342)(437.35961763,318.75726336)
\curveto(437.31961337,318.75726341)(437.2846134,318.75226341)(437.25461763,318.74226336)
\curveto(437.20461348,318.73226343)(437.14961354,318.73226343)(437.08961763,318.74226336)
\lineto(436.93961763,318.74226336)
\lineto(436.78961763,318.74226336)
\curveto(436.73961395,318.73226343)(436.69461399,318.73226343)(436.65461763,318.74226336)
\lineto(436.48961763,318.74226336)
\curveto(436.43961425,318.75226341)(436.3846143,318.75726341)(436.32461763,318.75726336)
\curveto(436.26461442,318.75726341)(436.20961448,318.7622634)(436.15961763,318.77226336)
\curveto(436.0896146,318.78226338)(436.02461466,318.79226337)(435.96461763,318.80226336)
\lineto(435.78461763,318.83226336)
\curveto(435.67461501,318.8622633)(435.56961512,318.89726327)(435.46961763,318.93726336)
\curveto(435.36961532,318.97726319)(435.27461541,319.02226314)(435.18461763,319.07226336)
\lineto(435.09461763,319.13226336)
\curveto(435.06461562,319.162263)(435.02961566,319.19226297)(434.98961763,319.22226336)
\curveto(434.96961572,319.24226292)(434.94461574,319.2622629)(434.91461763,319.28226336)
\lineto(434.83961763,319.35726336)
\curveto(434.69961599,319.54726262)(434.59461609,319.75726241)(434.52461763,319.98726336)
\curveto(434.50461618,320.02726214)(434.49461619,320.0622621)(434.49461763,320.09226336)
\curveto(434.50461618,320.13226203)(434.50461618,320.17726199)(434.49461763,320.22726336)
\curveto(434.4846162,320.24726192)(434.47961621,320.27226189)(434.47961763,320.30226336)
\curveto(434.47961621,320.33226183)(434.47461621,320.35726181)(434.46461763,320.37726336)
\lineto(434.46461763,320.52726336)
\curveto(434.45461623,320.5672616)(434.44961624,320.61226155)(434.44961763,320.66226336)
\curveto(434.45961623,320.71226145)(434.46461622,320.7622614)(434.46461763,320.81226336)
\lineto(434.46461763,321.38226336)
\lineto(434.46461763,323.61726336)
\lineto(434.46461763,324.41226336)
\lineto(434.46461763,324.62226336)
\curveto(434.47461621,324.69225747)(434.46961622,324.75725741)(434.44961763,324.81726336)
\curveto(434.40961628,324.95725721)(434.33961635,325.04725712)(434.23961763,325.08726336)
\curveto(434.12961656,325.13725703)(433.9896167,325.15225701)(433.81961763,325.13226336)
\curveto(433.64961704,325.11225705)(433.50461718,325.12725704)(433.38461763,325.17726336)
\curveto(433.30461738,325.20725696)(433.25461743,325.25225691)(433.23461763,325.31226336)
\curveto(433.21461747,325.37225679)(433.19461749,325.44725672)(433.17461763,325.53726336)
\lineto(433.17461763,325.85226336)
\curveto(433.17461751,326.03225613)(433.1846175,326.17725599)(433.20461763,326.28726336)
\curveto(433.22461746,326.39725577)(433.30961738,326.47225569)(433.45961763,326.51226336)
\curveto(433.49961719,326.53225563)(433.53961715,326.53725563)(433.57961763,326.52726336)
\lineto(433.71461763,326.52726336)
\curveto(433.86461682,326.52725564)(434.00461668,326.53225563)(434.13461763,326.54226336)
\curveto(434.26461642,326.5622556)(434.35461633,326.62225554)(434.40461763,326.72226336)
\curveto(434.43461625,326.79225537)(434.44961624,326.87225529)(434.44961763,326.96226336)
\curveto(434.45961623,327.05225511)(434.46461622,327.14225502)(434.46461763,327.23226336)
\lineto(434.46461763,328.16226336)
\lineto(434.46461763,328.41726336)
\curveto(434.46461622,328.50725366)(434.47461621,328.58225358)(434.49461763,328.64226336)
\curveto(434.54461614,328.74225342)(434.61961607,328.80725336)(434.71961763,328.83726336)
\curveto(434.73961595,328.84725332)(434.76461592,328.84725332)(434.79461763,328.83726336)
\curveto(434.83461585,328.83725333)(434.86461582,328.84225332)(434.88461763,328.85226336)
}
}
{
\newrgbcolor{curcolor}{0 0 0}
\pscustom[linestyle=none,fillstyle=solid,fillcolor=curcolor]
{
\newpath
\moveto(441.20805513,329.39226336)
\curveto(441.27805218,329.31225285)(441.31305215,329.19225297)(441.31305513,329.03226336)
\lineto(441.31305513,328.56726336)
\lineto(441.31305513,328.16226336)
\curveto(441.31305215,328.02225414)(441.27805218,327.92725424)(441.20805513,327.87726336)
\curveto(441.14805231,327.82725434)(441.06805239,327.79725437)(440.96805513,327.78726336)
\curveto(440.87805258,327.77725439)(440.77805268,327.77225439)(440.66805513,327.77226336)
\lineto(439.82805513,327.77226336)
\curveto(439.71805374,327.77225439)(439.61805384,327.77725439)(439.52805513,327.78726336)
\curveto(439.44805401,327.79725437)(439.37805408,327.82725434)(439.31805513,327.87726336)
\curveto(439.27805418,327.90725426)(439.24805421,327.9622542)(439.22805513,328.04226336)
\curveto(439.21805424,328.13225403)(439.20805425,328.22725394)(439.19805513,328.32726336)
\lineto(439.19805513,328.65726336)
\curveto(439.20805425,328.7672534)(439.21305425,328.8622533)(439.21305513,328.94226336)
\lineto(439.21305513,329.15226336)
\curveto(439.22305424,329.22225294)(439.24305422,329.28225288)(439.27305513,329.33226336)
\curveto(439.29305417,329.37225279)(439.31805414,329.40225276)(439.34805513,329.42226336)
\lineto(439.46805513,329.48226336)
\curveto(439.48805397,329.48225268)(439.51305395,329.48225268)(439.54305513,329.48226336)
\curveto(439.57305389,329.49225267)(439.59805386,329.49725267)(439.61805513,329.49726336)
\lineto(440.71305513,329.49726336)
\curveto(440.81305265,329.49725267)(440.90805255,329.49225267)(440.99805513,329.48226336)
\curveto(441.08805237,329.47225269)(441.1580523,329.44225272)(441.20805513,329.39226336)
\moveto(441.31305513,319.62726336)
\curveto(441.31305215,319.42726274)(441.30805215,319.25726291)(441.29805513,319.11726336)
\curveto(441.28805217,318.97726319)(441.19805226,318.88226328)(441.02805513,318.83226336)
\curveto(440.96805249,318.81226335)(440.90305256,318.80226336)(440.83305513,318.80226336)
\curveto(440.7630527,318.81226335)(440.68805277,318.81726335)(440.60805513,318.81726336)
\lineto(439.76805513,318.81726336)
\curveto(439.67805378,318.81726335)(439.58805387,318.82226334)(439.49805513,318.83226336)
\curveto(439.41805404,318.84226332)(439.3580541,318.87226329)(439.31805513,318.92226336)
\curveto(439.2580542,318.99226317)(439.22305424,319.07726309)(439.21305513,319.17726336)
\lineto(439.21305513,319.52226336)
\lineto(439.21305513,325.85226336)
\lineto(439.21305513,326.15226336)
\curveto(439.21305425,326.25225591)(439.23305423,326.33225583)(439.27305513,326.39226336)
\curveto(439.33305413,326.4622557)(439.41805404,326.50725566)(439.52805513,326.52726336)
\curveto(439.54805391,326.53725563)(439.57305389,326.53725563)(439.60305513,326.52726336)
\curveto(439.64305382,326.52725564)(439.67305379,326.53225563)(439.69305513,326.54226336)
\lineto(440.44305513,326.54226336)
\lineto(440.63805513,326.54226336)
\curveto(440.71805274,326.55225561)(440.78305268,326.55225561)(440.83305513,326.54226336)
\lineto(440.95305513,326.54226336)
\curveto(441.01305245,326.52225564)(441.06805239,326.50725566)(441.11805513,326.49726336)
\curveto(441.16805229,326.48725568)(441.20805225,326.45725571)(441.23805513,326.40726336)
\curveto(441.27805218,326.35725581)(441.29805216,326.28725588)(441.29805513,326.19726336)
\curveto(441.30805215,326.10725606)(441.31305215,326.01225615)(441.31305513,325.91226336)
\lineto(441.31305513,319.62726336)
}
}
{
\newrgbcolor{curcolor}{0 0 0}
\pscustom[linestyle=none,fillstyle=solid,fillcolor=curcolor]
{
\newpath
\moveto(450.86524263,322.76226336)
\curveto(450.87523395,322.70225946)(450.88023395,322.61225955)(450.88024263,322.49226336)
\curveto(450.88023395,322.37225979)(450.87023396,322.28725988)(450.85024263,322.23726336)
\lineto(450.85024263,322.04226336)
\curveto(450.82023401,321.93226023)(450.80023403,321.82726034)(450.79024263,321.72726336)
\curveto(450.79023404,321.62726054)(450.77523405,321.52726064)(450.74524263,321.42726336)
\curveto(450.7252341,321.33726083)(450.70523412,321.24226092)(450.68524263,321.14226336)
\curveto(450.66523416,321.05226111)(450.63523419,320.9622612)(450.59524263,320.87226336)
\curveto(450.5252343,320.70226146)(450.45523437,320.54226162)(450.38524263,320.39226336)
\curveto(450.31523451,320.25226191)(450.23523459,320.11226205)(450.14524263,319.97226336)
\curveto(450.08523474,319.88226228)(450.02023481,319.79726237)(449.95024263,319.71726336)
\curveto(449.89023494,319.64726252)(449.82023501,319.57226259)(449.74024263,319.49226336)
\lineto(449.63524263,319.38726336)
\curveto(449.58523524,319.33726283)(449.5302353,319.29226287)(449.47024263,319.25226336)
\lineto(449.32024263,319.13226336)
\curveto(449.24023559,319.07226309)(449.15023568,319.01726315)(449.05024263,318.96726336)
\curveto(448.96023587,318.92726324)(448.86523596,318.88226328)(448.76524263,318.83226336)
\curveto(448.66523616,318.78226338)(448.56023627,318.74726342)(448.45024263,318.72726336)
\curveto(448.35023648,318.70726346)(448.24523658,318.68726348)(448.13524263,318.66726336)
\curveto(448.07523675,318.64726352)(448.01023682,318.63726353)(447.94024263,318.63726336)
\curveto(447.88023695,318.63726353)(447.81523701,318.62726354)(447.74524263,318.60726336)
\lineto(447.61024263,318.60726336)
\curveto(447.5302373,318.58726358)(447.45523737,318.58726358)(447.38524263,318.60726336)
\lineto(447.23524263,318.60726336)
\curveto(447.17523765,318.62726354)(447.11023772,318.63726353)(447.04024263,318.63726336)
\curveto(446.98023785,318.62726354)(446.92023791,318.63226353)(446.86024263,318.65226336)
\curveto(446.70023813,318.70226346)(446.54523828,318.74726342)(446.39524263,318.78726336)
\curveto(446.25523857,318.82726334)(446.1252387,318.88726328)(446.00524263,318.96726336)
\curveto(445.93523889,319.00726316)(445.87023896,319.04726312)(445.81024263,319.08726336)
\curveto(445.75023908,319.13726303)(445.68523914,319.18726298)(445.61524263,319.23726336)
\lineto(445.43524263,319.37226336)
\curveto(445.35523947,319.43226273)(445.28523954,319.43726273)(445.22524263,319.38726336)
\curveto(445.17523965,319.35726281)(445.15023968,319.31726285)(445.15024263,319.26726336)
\curveto(445.15023968,319.22726294)(445.14023969,319.17726299)(445.12024263,319.11726336)
\curveto(445.10023973,319.01726315)(445.09023974,318.90226326)(445.09024263,318.77226336)
\curveto(445.10023973,318.64226352)(445.10523972,318.52226364)(445.10524263,318.41226336)
\lineto(445.10524263,316.88226336)
\curveto(445.10523972,316.75226541)(445.10023973,316.62726554)(445.09024263,316.50726336)
\curveto(445.09023974,316.37726579)(445.06523976,316.27226589)(445.01524263,316.19226336)
\curveto(444.98523984,316.15226601)(444.9302399,316.12226604)(444.85024263,316.10226336)
\curveto(444.77024006,316.08226608)(444.68024015,316.07226609)(444.58024263,316.07226336)
\curveto(444.48024035,316.0622661)(444.38024045,316.0622661)(444.28024263,316.07226336)
\lineto(444.02524263,316.07226336)
\lineto(443.62024263,316.07226336)
\lineto(443.51524263,316.07226336)
\curveto(443.47524135,316.07226609)(443.44024139,316.07726609)(443.41024263,316.08726336)
\lineto(443.29024263,316.08726336)
\curveto(443.12024171,316.13726603)(443.0302418,316.23726593)(443.02024263,316.38726336)
\curveto(443.01024182,316.52726564)(443.00524182,316.69726547)(443.00524263,316.89726336)
\lineto(443.00524263,325.70226336)
\curveto(443.00524182,325.81225635)(443.00024183,325.92725624)(442.99024263,326.04726336)
\curveto(442.99024184,326.17725599)(443.01524181,326.27725589)(443.06524263,326.34726336)
\curveto(443.10524172,326.41725575)(443.16024167,326.4622557)(443.23024263,326.48226336)
\curveto(443.28024155,326.50225566)(443.34024149,326.51225565)(443.41024263,326.51226336)
\lineto(443.63524263,326.51226336)
\lineto(444.35524263,326.51226336)
\lineto(444.64024263,326.51226336)
\curveto(444.7302401,326.51225565)(444.80524002,326.48725568)(444.86524263,326.43726336)
\curveto(444.93523989,326.38725578)(444.97023986,326.32225584)(444.97024263,326.24226336)
\curveto(444.98023985,326.17225599)(445.00523982,326.09725607)(445.04524263,326.01726336)
\curveto(445.05523977,325.98725618)(445.06523976,325.9622562)(445.07524263,325.94226336)
\curveto(445.09523973,325.93225623)(445.11523971,325.91725625)(445.13524263,325.89726336)
\curveto(445.24523958,325.88725628)(445.33523949,325.91725625)(445.40524263,325.98726336)
\curveto(445.47523935,326.05725611)(445.54523928,326.11725605)(445.61524263,326.16726336)
\curveto(445.74523908,326.25725591)(445.88023895,326.33725583)(446.02024263,326.40726336)
\curveto(446.16023867,326.48725568)(446.31523851,326.55225561)(446.48524263,326.60226336)
\curveto(446.56523826,326.63225553)(446.65023818,326.65225551)(446.74024263,326.66226336)
\curveto(446.84023799,326.67225549)(446.93523789,326.68725548)(447.02524263,326.70726336)
\curveto(447.06523776,326.71725545)(447.10523772,326.71725545)(447.14524263,326.70726336)
\curveto(447.19523763,326.69725547)(447.23523759,326.70225546)(447.26524263,326.72226336)
\curveto(447.83523699,326.74225542)(448.31523651,326.6622555)(448.70524263,326.48226336)
\curveto(449.10523572,326.31225585)(449.44523538,326.08725608)(449.72524263,325.80726336)
\curveto(449.77523505,325.75725641)(449.82023501,325.70725646)(449.86024263,325.65726336)
\curveto(449.90023493,325.61725655)(449.94023489,325.57225659)(449.98024263,325.52226336)
\curveto(450.05023478,325.43225673)(450.11023472,325.34225682)(450.16024263,325.25226336)
\curveto(450.22023461,325.162257)(450.27523455,325.07225709)(450.32524263,324.98226336)
\curveto(450.34523448,324.9622572)(450.35523447,324.93725723)(450.35524263,324.90726336)
\curveto(450.36523446,324.87725729)(450.38023445,324.84225732)(450.40024263,324.80226336)
\curveto(450.46023437,324.70225746)(450.51523431,324.58225758)(450.56524263,324.44226336)
\curveto(450.58523424,324.38225778)(450.60523422,324.31725785)(450.62524263,324.24726336)
\curveto(450.64523418,324.18725798)(450.66523416,324.12225804)(450.68524263,324.05226336)
\curveto(450.7252341,323.93225823)(450.75023408,323.80725836)(450.76024263,323.67726336)
\curveto(450.78023405,323.54725862)(450.80523402,323.41225875)(450.83524263,323.27226336)
\lineto(450.83524263,323.10726336)
\lineto(450.86524263,322.92726336)
\lineto(450.86524263,322.76226336)
\moveto(448.75024263,322.41726336)
\curveto(448.76023607,322.4672597)(448.76523606,322.53225963)(448.76524263,322.61226336)
\curveto(448.76523606,322.70225946)(448.76023607,322.77225939)(448.75024263,322.82226336)
\lineto(448.75024263,322.95726336)
\curveto(448.7302361,323.01725915)(448.72023611,323.08225908)(448.72024263,323.15226336)
\curveto(448.72023611,323.22225894)(448.71023612,323.29225887)(448.69024263,323.36226336)
\curveto(448.67023616,323.4622587)(448.65023618,323.55725861)(448.63024263,323.64726336)
\curveto(448.61023622,323.74725842)(448.58023625,323.83725833)(448.54024263,323.91726336)
\curveto(448.42023641,324.23725793)(448.26523656,324.49225767)(448.07524263,324.68226336)
\curveto(447.88523694,324.87225729)(447.61523721,325.01225715)(447.26524263,325.10226336)
\curveto(447.18523764,325.12225704)(447.09523773,325.13225703)(446.99524263,325.13226336)
\lineto(446.72524263,325.13226336)
\curveto(446.68523814,325.12225704)(446.65023818,325.11725705)(446.62024263,325.11726336)
\curveto(446.59023824,325.11725705)(446.55523827,325.11225705)(446.51524263,325.10226336)
\lineto(446.30524263,325.04226336)
\curveto(446.24523858,325.03225713)(446.18523864,325.01225715)(446.12524263,324.98226336)
\curveto(445.86523896,324.87225729)(445.66023917,324.70225746)(445.51024263,324.47226336)
\curveto(445.37023946,324.24225792)(445.25523957,323.98725818)(445.16524263,323.70726336)
\curveto(445.14523968,323.62725854)(445.1302397,323.54225862)(445.12024263,323.45226336)
\curveto(445.11023972,323.37225879)(445.09523973,323.29225887)(445.07524263,323.21226336)
\curveto(445.06523976,323.17225899)(445.06023977,323.10725906)(445.06024263,323.01726336)
\curveto(445.04023979,322.97725919)(445.03523979,322.92725924)(445.04524263,322.86726336)
\curveto(445.05523977,322.81725935)(445.05523977,322.7672594)(445.04524263,322.71726336)
\curveto(445.0252398,322.65725951)(445.0252398,322.60225956)(445.04524263,322.55226336)
\lineto(445.04524263,322.37226336)
\lineto(445.04524263,322.23726336)
\curveto(445.04523978,322.19725997)(445.05523977,322.15726001)(445.07524263,322.11726336)
\curveto(445.07523975,322.04726012)(445.08023975,321.99226017)(445.09024263,321.95226336)
\lineto(445.12024263,321.77226336)
\curveto(445.1302397,321.71226045)(445.14523968,321.65226051)(445.16524263,321.59226336)
\curveto(445.25523957,321.30226086)(445.36023947,321.0622611)(445.48024263,320.87226336)
\curveto(445.61023922,320.69226147)(445.79023904,320.53226163)(446.02024263,320.39226336)
\curveto(446.16023867,320.31226185)(446.3252385,320.24726192)(446.51524263,320.19726336)
\curveto(446.55523827,320.18726198)(446.59023824,320.18226198)(446.62024263,320.18226336)
\curveto(446.65023818,320.19226197)(446.68523814,320.19226197)(446.72524263,320.18226336)
\curveto(446.76523806,320.17226199)(446.825238,320.162262)(446.90524263,320.15226336)
\curveto(446.98523784,320.15226201)(447.05023778,320.15726201)(447.10024263,320.16726336)
\curveto(447.18023765,320.18726198)(447.26023757,320.20226196)(447.34024263,320.21226336)
\curveto(447.4302374,320.23226193)(447.51523731,320.25726191)(447.59524263,320.28726336)
\curveto(447.83523699,320.38726178)(448.0302368,320.52726164)(448.18024263,320.70726336)
\curveto(448.3302365,320.88726128)(448.45523637,321.09726107)(448.55524263,321.33726336)
\curveto(448.60523622,321.45726071)(448.64023619,321.58226058)(448.66024263,321.71226336)
\curveto(448.68023615,321.84226032)(448.70523612,321.97726019)(448.73524263,322.11726336)
\lineto(448.73524263,322.26726336)
\curveto(448.74523608,322.31725985)(448.75023608,322.3672598)(448.75024263,322.41726336)
}
}
{
\newrgbcolor{curcolor}{0 0 0}
\pscustom[linestyle=none,fillstyle=solid,fillcolor=curcolor]
{
\newpath
\moveto(459.91516451,322.98726336)
\curveto(459.93515594,322.92725924)(459.94515593,322.84225932)(459.94516451,322.73226336)
\curveto(459.94515593,322.62225954)(459.93515594,322.53725963)(459.91516451,322.47726336)
\lineto(459.91516451,322.32726336)
\curveto(459.89515598,322.24725992)(459.88515599,322.16726)(459.88516451,322.08726336)
\curveto(459.89515598,322.00726016)(459.89015598,321.92726024)(459.87016451,321.84726336)
\curveto(459.85015602,321.77726039)(459.83515604,321.71226045)(459.82516451,321.65226336)
\curveto(459.81515606,321.59226057)(459.80515607,321.52726064)(459.79516451,321.45726336)
\curveto(459.75515612,321.34726082)(459.72015615,321.23226093)(459.69016451,321.11226336)
\curveto(459.66015621,321.00226116)(459.62015625,320.89726127)(459.57016451,320.79726336)
\curveto(459.36015651,320.31726185)(459.08515679,319.92726224)(458.74516451,319.62726336)
\curveto(458.40515747,319.32726284)(457.99515788,319.07726309)(457.51516451,318.87726336)
\curveto(457.39515848,318.82726334)(457.2701586,318.79226337)(457.14016451,318.77226336)
\curveto(457.02015885,318.74226342)(456.89515898,318.71226345)(456.76516451,318.68226336)
\curveto(456.71515916,318.6622635)(456.66015921,318.65226351)(456.60016451,318.65226336)
\curveto(456.54015933,318.65226351)(456.48515939,318.64726352)(456.43516451,318.63726336)
\lineto(456.33016451,318.63726336)
\curveto(456.30015957,318.62726354)(456.2701596,318.62226354)(456.24016451,318.62226336)
\curveto(456.19015968,318.61226355)(456.11015976,318.60726356)(456.00016451,318.60726336)
\curveto(455.89015998,318.59726357)(455.80516007,318.60226356)(455.74516451,318.62226336)
\lineto(455.59516451,318.62226336)
\curveto(455.54516033,318.63226353)(455.49016038,318.63726353)(455.43016451,318.63726336)
\curveto(455.38016049,318.62726354)(455.33016054,318.63226353)(455.28016451,318.65226336)
\curveto(455.24016063,318.6622635)(455.20016067,318.6672635)(455.16016451,318.66726336)
\curveto(455.13016074,318.6672635)(455.09016078,318.67226349)(455.04016451,318.68226336)
\curveto(454.94016093,318.71226345)(454.84016103,318.73726343)(454.74016451,318.75726336)
\curveto(454.64016123,318.77726339)(454.54516133,318.80726336)(454.45516451,318.84726336)
\curveto(454.33516154,318.88726328)(454.22016165,318.92726324)(454.11016451,318.96726336)
\curveto(454.01016186,319.00726316)(453.90516197,319.05726311)(453.79516451,319.11726336)
\curveto(453.44516243,319.32726284)(453.14516273,319.57226259)(452.89516451,319.85226336)
\curveto(452.64516323,320.13226203)(452.43516344,320.4672617)(452.26516451,320.85726336)
\curveto(452.21516366,320.94726122)(452.1751637,321.04226112)(452.14516451,321.14226336)
\curveto(452.12516375,321.24226092)(452.10016377,321.34726082)(452.07016451,321.45726336)
\curveto(452.05016382,321.50726066)(452.04016383,321.55226061)(452.04016451,321.59226336)
\curveto(452.04016383,321.63226053)(452.03016384,321.67726049)(452.01016451,321.72726336)
\curveto(451.99016388,321.80726036)(451.98016389,321.88726028)(451.98016451,321.96726336)
\curveto(451.98016389,322.05726011)(451.9701639,322.14226002)(451.95016451,322.22226336)
\curveto(451.94016393,322.27225989)(451.93516394,322.31725985)(451.93516451,322.35726336)
\lineto(451.93516451,322.49226336)
\curveto(451.91516396,322.55225961)(451.90516397,322.63725953)(451.90516451,322.74726336)
\curveto(451.91516396,322.85725931)(451.93016394,322.94225922)(451.95016451,323.00226336)
\lineto(451.95016451,323.10726336)
\curveto(451.96016391,323.15725901)(451.96016391,323.20725896)(451.95016451,323.25726336)
\curveto(451.95016392,323.31725885)(451.96016391,323.37225879)(451.98016451,323.42226336)
\curveto(451.99016388,323.47225869)(451.99516388,323.51725865)(451.99516451,323.55726336)
\curveto(451.99516388,323.60725856)(452.00516387,323.65725851)(452.02516451,323.70726336)
\curveto(452.06516381,323.83725833)(452.10016377,323.9622582)(452.13016451,324.08226336)
\curveto(452.16016371,324.21225795)(452.20016367,324.33725783)(452.25016451,324.45726336)
\curveto(452.43016344,324.8672573)(452.64516323,325.20725696)(452.89516451,325.47726336)
\curveto(453.14516273,325.75725641)(453.45016242,326.01225615)(453.81016451,326.24226336)
\curveto(453.91016196,326.29225587)(454.01516186,326.33725583)(454.12516451,326.37726336)
\curveto(454.23516164,326.41725575)(454.34516153,326.4622557)(454.45516451,326.51226336)
\curveto(454.58516129,326.5622556)(454.72016115,326.59725557)(454.86016451,326.61726336)
\curveto(455.00016087,326.63725553)(455.14516073,326.6672555)(455.29516451,326.70726336)
\curveto(455.3751605,326.71725545)(455.45016042,326.72225544)(455.52016451,326.72226336)
\curveto(455.59016028,326.72225544)(455.66016021,326.72725544)(455.73016451,326.73726336)
\curveto(456.31015956,326.74725542)(456.81015906,326.68725548)(457.23016451,326.55726336)
\curveto(457.66015821,326.42725574)(458.04015783,326.24725592)(458.37016451,326.01726336)
\curveto(458.48015739,325.93725623)(458.59015728,325.84725632)(458.70016451,325.74726336)
\curveto(458.82015705,325.65725651)(458.92015695,325.55725661)(459.00016451,325.44726336)
\curveto(459.08015679,325.34725682)(459.15015672,325.24725692)(459.21016451,325.14726336)
\curveto(459.28015659,325.04725712)(459.35015652,324.94225722)(459.42016451,324.83226336)
\curveto(459.49015638,324.72225744)(459.54515633,324.60225756)(459.58516451,324.47226336)
\curveto(459.62515625,324.35225781)(459.6701562,324.22225794)(459.72016451,324.08226336)
\curveto(459.75015612,324.00225816)(459.7751561,323.91725825)(459.79516451,323.82726336)
\lineto(459.85516451,323.55726336)
\curveto(459.86515601,323.51725865)(459.870156,323.47725869)(459.87016451,323.43726336)
\curveto(459.870156,323.39725877)(459.875156,323.35725881)(459.88516451,323.31726336)
\curveto(459.90515597,323.2672589)(459.91015596,323.21225895)(459.90016451,323.15226336)
\curveto(459.89015598,323.09225907)(459.89515598,323.03725913)(459.91516451,322.98726336)
\moveto(457.81516451,322.44726336)
\curveto(457.82515805,322.49725967)(457.83015804,322.5672596)(457.83016451,322.65726336)
\curveto(457.83015804,322.75725941)(457.82515805,322.83225933)(457.81516451,322.88226336)
\lineto(457.81516451,323.00226336)
\curveto(457.79515808,323.05225911)(457.78515809,323.10725906)(457.78516451,323.16726336)
\curveto(457.78515809,323.22725894)(457.78015809,323.28225888)(457.77016451,323.33226336)
\curveto(457.7701581,323.37225879)(457.76515811,323.40225876)(457.75516451,323.42226336)
\lineto(457.69516451,323.66226336)
\curveto(457.68515819,323.75225841)(457.66515821,323.83725833)(457.63516451,323.91726336)
\curveto(457.52515835,324.17725799)(457.39515848,324.39725777)(457.24516451,324.57726336)
\curveto(457.09515878,324.7672574)(456.89515898,324.91725725)(456.64516451,325.02726336)
\curveto(456.58515929,325.04725712)(456.52515935,325.0622571)(456.46516451,325.07226336)
\curveto(456.40515947,325.09225707)(456.34015953,325.11225705)(456.27016451,325.13226336)
\curveto(456.19015968,325.15225701)(456.10515977,325.15725701)(456.01516451,325.14726336)
\lineto(455.74516451,325.14726336)
\curveto(455.71516016,325.12725704)(455.68016019,325.11725705)(455.64016451,325.11726336)
\curveto(455.60016027,325.12725704)(455.56516031,325.12725704)(455.53516451,325.11726336)
\lineto(455.32516451,325.05726336)
\curveto(455.26516061,325.04725712)(455.21016066,325.02725714)(455.16016451,324.99726336)
\curveto(454.91016096,324.88725728)(454.70516117,324.72725744)(454.54516451,324.51726336)
\curveto(454.39516148,324.31725785)(454.2751616,324.08225808)(454.18516451,323.81226336)
\curveto(454.15516172,323.71225845)(454.13016174,323.60725856)(454.11016451,323.49726336)
\curveto(454.10016177,323.38725878)(454.08516179,323.27725889)(454.06516451,323.16726336)
\curveto(454.05516182,323.11725905)(454.05016182,323.0672591)(454.05016451,323.01726336)
\lineto(454.05016451,322.86726336)
\curveto(454.03016184,322.79725937)(454.02016185,322.69225947)(454.02016451,322.55226336)
\curveto(454.03016184,322.41225975)(454.04516183,322.30725986)(454.06516451,322.23726336)
\lineto(454.06516451,322.10226336)
\curveto(454.08516179,322.02226014)(454.10016177,321.94226022)(454.11016451,321.86226336)
\curveto(454.12016175,321.79226037)(454.13516174,321.71726045)(454.15516451,321.63726336)
\curveto(454.25516162,321.33726083)(454.36016151,321.09226107)(454.47016451,320.90226336)
\curveto(454.59016128,320.72226144)(454.7751611,320.55726161)(455.02516451,320.40726336)
\curveto(455.09516078,320.35726181)(455.1701607,320.31726185)(455.25016451,320.28726336)
\curveto(455.34016053,320.25726191)(455.43016044,320.23226193)(455.52016451,320.21226336)
\curveto(455.56016031,320.20226196)(455.59516028,320.19726197)(455.62516451,320.19726336)
\curveto(455.65516022,320.20726196)(455.69016018,320.20726196)(455.73016451,320.19726336)
\lineto(455.85016451,320.16726336)
\curveto(455.90015997,320.167262)(455.94515993,320.17226199)(455.98516451,320.18226336)
\lineto(456.10516451,320.18226336)
\curveto(456.18515969,320.20226196)(456.26515961,320.21726195)(456.34516451,320.22726336)
\curveto(456.42515945,320.23726193)(456.50015937,320.25726191)(456.57016451,320.28726336)
\curveto(456.83015904,320.38726178)(457.04015883,320.52226164)(457.20016451,320.69226336)
\curveto(457.36015851,320.8622613)(457.49515838,321.07226109)(457.60516451,321.32226336)
\curveto(457.64515823,321.42226074)(457.6751582,321.52226064)(457.69516451,321.62226336)
\curveto(457.71515816,321.72226044)(457.74015813,321.82726034)(457.77016451,321.93726336)
\curveto(457.78015809,321.97726019)(457.78515809,322.01226015)(457.78516451,322.04226336)
\curveto(457.78515809,322.08226008)(457.79015808,322.12226004)(457.80016451,322.16226336)
\lineto(457.80016451,322.29726336)
\curveto(457.80015807,322.34725982)(457.80515807,322.39725977)(457.81516451,322.44726336)
}
}
{
\newrgbcolor{curcolor}{0 0 0}
\pscustom[linestyle=none,fillstyle=solid,fillcolor=curcolor]
{
}
}
{
\newrgbcolor{curcolor}{0 0 0}
\pscustom[linestyle=none,fillstyle=solid,fillcolor=curcolor]
{
\newpath
\moveto(473.06524263,319.65726336)
\lineto(473.06524263,319.23726336)
\curveto(473.06523426,319.10726306)(473.03523429,319.00226316)(472.97524263,318.92226336)
\curveto(472.9252344,318.87226329)(472.86023447,318.83726333)(472.78024263,318.81726336)
\curveto(472.70023463,318.80726336)(472.61023472,318.80226336)(472.51024263,318.80226336)
\lineto(471.68524263,318.80226336)
\lineto(471.40024263,318.80226336)
\curveto(471.32023601,318.81226335)(471.25523607,318.83726333)(471.20524263,318.87726336)
\curveto(471.13523619,318.92726324)(471.09523623,318.99226317)(471.08524263,319.07226336)
\curveto(471.07523625,319.15226301)(471.05523627,319.23226293)(471.02524263,319.31226336)
\curveto(471.00523632,319.33226283)(470.98523634,319.34726282)(470.96524263,319.35726336)
\curveto(470.95523637,319.37726279)(470.94023639,319.39726277)(470.92024263,319.41726336)
\curveto(470.81023652,319.41726275)(470.7302366,319.39226277)(470.68024263,319.34226336)
\lineto(470.53024263,319.19226336)
\curveto(470.46023687,319.14226302)(470.39523693,319.09726307)(470.33524263,319.05726336)
\curveto(470.27523705,319.02726314)(470.21023712,318.98726318)(470.14024263,318.93726336)
\curveto(470.10023723,318.91726325)(470.05523727,318.89726327)(470.00524263,318.87726336)
\curveto(469.96523736,318.85726331)(469.92023741,318.83726333)(469.87024263,318.81726336)
\curveto(469.7302376,318.7672634)(469.58023775,318.72226344)(469.42024263,318.68226336)
\curveto(469.37023796,318.6622635)(469.325238,318.65226351)(469.28524263,318.65226336)
\curveto(469.24523808,318.65226351)(469.20523812,318.64726352)(469.16524263,318.63726336)
\lineto(469.03024263,318.63726336)
\curveto(469.00023833,318.62726354)(468.96023837,318.62226354)(468.91024263,318.62226336)
\lineto(468.77524263,318.62226336)
\curveto(468.71523861,318.60226356)(468.6252387,318.59726357)(468.50524263,318.60726336)
\curveto(468.38523894,318.60726356)(468.30023903,318.61726355)(468.25024263,318.63726336)
\curveto(468.18023915,318.65726351)(468.11523921,318.6672635)(468.05524263,318.66726336)
\curveto(468.00523932,318.65726351)(467.95023938,318.6622635)(467.89024263,318.68226336)
\lineto(467.53024263,318.80226336)
\curveto(467.42023991,318.83226333)(467.31024002,318.87226329)(467.20024263,318.92226336)
\curveto(466.85024048,319.07226309)(466.53524079,319.30226286)(466.25524263,319.61226336)
\curveto(465.98524134,319.93226223)(465.77024156,320.2672619)(465.61024263,320.61726336)
\curveto(465.56024177,320.72726144)(465.52024181,320.83226133)(465.49024263,320.93226336)
\curveto(465.46024187,321.04226112)(465.4252419,321.15226101)(465.38524263,321.26226336)
\curveto(465.37524195,321.30226086)(465.37024196,321.33726083)(465.37024263,321.36726336)
\curveto(465.37024196,321.40726076)(465.36024197,321.45226071)(465.34024263,321.50226336)
\curveto(465.32024201,321.58226058)(465.30024203,321.6672605)(465.28024263,321.75726336)
\curveto(465.27024206,321.85726031)(465.25524207,321.95726021)(465.23524263,322.05726336)
\curveto(465.2252421,322.08726008)(465.22024211,322.12226004)(465.22024263,322.16226336)
\curveto(465.2302421,322.20225996)(465.2302421,322.23725993)(465.22024263,322.26726336)
\lineto(465.22024263,322.40226336)
\curveto(465.22024211,322.45225971)(465.21524211,322.50225966)(465.20524263,322.55226336)
\curveto(465.19524213,322.60225956)(465.19024214,322.65725951)(465.19024263,322.71726336)
\curveto(465.19024214,322.78725938)(465.19524213,322.84225932)(465.20524263,322.88226336)
\curveto(465.21524211,322.93225923)(465.22024211,322.97725919)(465.22024263,323.01726336)
\lineto(465.22024263,323.16726336)
\curveto(465.2302421,323.21725895)(465.2302421,323.2622589)(465.22024263,323.30226336)
\curveto(465.22024211,323.35225881)(465.2302421,323.40225876)(465.25024263,323.45226336)
\curveto(465.27024206,323.5622586)(465.28524204,323.6672585)(465.29524263,323.76726336)
\curveto(465.31524201,323.8672583)(465.34024199,323.9672582)(465.37024263,324.06726336)
\curveto(465.41024192,324.18725798)(465.44524188,324.30225786)(465.47524263,324.41226336)
\curveto(465.50524182,324.52225764)(465.54524178,324.63225753)(465.59524263,324.74226336)
\curveto(465.73524159,325.04225712)(465.91024142,325.32725684)(466.12024263,325.59726336)
\curveto(466.14024119,325.62725654)(466.16524116,325.65225651)(466.19524263,325.67226336)
\curveto(466.23524109,325.70225646)(466.26524106,325.73225643)(466.28524263,325.76226336)
\curveto(466.325241,325.81225635)(466.36524096,325.85725631)(466.40524263,325.89726336)
\curveto(466.44524088,325.93725623)(466.49024084,325.97725619)(466.54024263,326.01726336)
\curveto(466.58024075,326.03725613)(466.61524071,326.0622561)(466.64524263,326.09226336)
\curveto(466.67524065,326.13225603)(466.71024062,326.162256)(466.75024263,326.18226336)
\curveto(467.00024033,326.35225581)(467.29024004,326.49225567)(467.62024263,326.60226336)
\curveto(467.69023964,326.62225554)(467.76023957,326.63725553)(467.83024263,326.64726336)
\curveto(467.91023942,326.65725551)(467.99023934,326.67225549)(468.07024263,326.69226336)
\curveto(468.14023919,326.71225545)(468.2302391,326.72225544)(468.34024263,326.72226336)
\curveto(468.45023888,326.73225543)(468.56023877,326.73725543)(468.67024263,326.73726336)
\curveto(468.78023855,326.73725543)(468.88523844,326.73225543)(468.98524263,326.72226336)
\curveto(469.09523823,326.71225545)(469.18523814,326.69725547)(469.25524263,326.67726336)
\curveto(469.40523792,326.62725554)(469.55023778,326.58225558)(469.69024263,326.54226336)
\curveto(469.8302375,326.50225566)(469.96023737,326.44725572)(470.08024263,326.37726336)
\curveto(470.15023718,326.32725584)(470.21523711,326.27725589)(470.27524263,326.22726336)
\curveto(470.33523699,326.18725598)(470.40023693,326.14225602)(470.47024263,326.09226336)
\curveto(470.51023682,326.0622561)(470.56523676,326.02225614)(470.63524263,325.97226336)
\curveto(470.71523661,325.92225624)(470.79023654,325.92225624)(470.86024263,325.97226336)
\curveto(470.90023643,325.99225617)(470.92023641,326.02725614)(470.92024263,326.07726336)
\curveto(470.92023641,326.12725604)(470.9302364,326.17725599)(470.95024263,326.22726336)
\lineto(470.95024263,326.37726336)
\curveto(470.96023637,326.40725576)(470.96523636,326.44225572)(470.96524263,326.48226336)
\lineto(470.96524263,326.60226336)
\lineto(470.96524263,328.64226336)
\curveto(470.96523636,328.75225341)(470.96023637,328.87225329)(470.95024263,329.00226336)
\curveto(470.95023638,329.14225302)(470.97523635,329.24725292)(471.02524263,329.31726336)
\curveto(471.06523626,329.39725277)(471.14023619,329.44725272)(471.25024263,329.46726336)
\curveto(471.27023606,329.47725269)(471.29023604,329.47725269)(471.31024263,329.46726336)
\curveto(471.330236,329.4672527)(471.35023598,329.47225269)(471.37024263,329.48226336)
\lineto(472.43524263,329.48226336)
\curveto(472.55523477,329.48225268)(472.66523466,329.47725269)(472.76524263,329.46726336)
\curveto(472.86523446,329.45725271)(472.94023439,329.41725275)(472.99024263,329.34726336)
\curveto(473.04023429,329.2672529)(473.06523426,329.162253)(473.06524263,329.03226336)
\lineto(473.06524263,328.67226336)
\lineto(473.06524263,319.65726336)
\moveto(471.02524263,322.59726336)
\curveto(471.03523629,322.63725953)(471.03523629,322.67725949)(471.02524263,322.71726336)
\lineto(471.02524263,322.85226336)
\curveto(471.0252363,322.95225921)(471.02023631,323.05225911)(471.01024263,323.15226336)
\curveto(471.00023633,323.25225891)(470.98523634,323.34225882)(470.96524263,323.42226336)
\curveto(470.94523638,323.53225863)(470.9252364,323.63225853)(470.90524263,323.72226336)
\curveto(470.89523643,323.81225835)(470.87023646,323.89725827)(470.83024263,323.97726336)
\curveto(470.69023664,324.33725783)(470.48523684,324.62225754)(470.21524263,324.83226336)
\curveto(469.95523737,325.04225712)(469.57523775,325.14725702)(469.07524263,325.14726336)
\curveto(469.01523831,325.14725702)(468.93523839,325.13725703)(468.83524263,325.11726336)
\curveto(468.75523857,325.09725707)(468.68023865,325.07725709)(468.61024263,325.05726336)
\curveto(468.55023878,325.04725712)(468.49023884,325.02725714)(468.43024263,324.99726336)
\curveto(468.16023917,324.88725728)(467.95023938,324.71725745)(467.80024263,324.48726336)
\curveto(467.65023968,324.25725791)(467.5302398,323.99725817)(467.44024263,323.70726336)
\curveto(467.41023992,323.60725856)(467.39023994,323.50725866)(467.38024263,323.40726336)
\curveto(467.37023996,323.30725886)(467.35023998,323.20225896)(467.32024263,323.09226336)
\lineto(467.32024263,322.88226336)
\curveto(467.30024003,322.79225937)(467.29524003,322.6672595)(467.30524263,322.50726336)
\curveto(467.31524001,322.35725981)(467.33024,322.24725992)(467.35024263,322.17726336)
\lineto(467.35024263,322.08726336)
\curveto(467.36023997,322.0672601)(467.36523996,322.04726012)(467.36524263,322.02726336)
\curveto(467.38523994,321.94726022)(467.40023993,321.87226029)(467.41024263,321.80226336)
\curveto(467.4302399,321.73226043)(467.45023988,321.65726051)(467.47024263,321.57726336)
\curveto(467.64023969,321.05726111)(467.9302394,320.67226149)(468.34024263,320.42226336)
\curveto(468.47023886,320.33226183)(468.65023868,320.2622619)(468.88024263,320.21226336)
\curveto(468.92023841,320.20226196)(468.98023835,320.19726197)(469.06024263,320.19726336)
\curveto(469.09023824,320.18726198)(469.13523819,320.17726199)(469.19524263,320.16726336)
\curveto(469.26523806,320.167262)(469.32023801,320.17226199)(469.36024263,320.18226336)
\curveto(469.44023789,320.20226196)(469.52023781,320.21726195)(469.60024263,320.22726336)
\curveto(469.68023765,320.23726193)(469.76023757,320.25726191)(469.84024263,320.28726336)
\curveto(470.09023724,320.39726177)(470.29023704,320.53726163)(470.44024263,320.70726336)
\curveto(470.59023674,320.87726129)(470.72023661,321.09226107)(470.83024263,321.35226336)
\curveto(470.87023646,321.44226072)(470.90023643,321.53226063)(470.92024263,321.62226336)
\curveto(470.94023639,321.72226044)(470.96023637,321.82726034)(470.98024263,321.93726336)
\curveto(470.99023634,321.98726018)(470.99023634,322.03226013)(470.98024263,322.07226336)
\curveto(470.98023635,322.12226004)(470.99023634,322.17225999)(471.01024263,322.22226336)
\curveto(471.02023631,322.25225991)(471.0252363,322.28725988)(471.02524263,322.32726336)
\lineto(471.02524263,322.46226336)
\lineto(471.02524263,322.59726336)
}
}
{
\newrgbcolor{curcolor}{0 0 0}
\pscustom[linestyle=none,fillstyle=solid,fillcolor=curcolor]
{
\newpath
\moveto(482.01016451,322.74726336)
\curveto(482.03015634,322.6672595)(482.03015634,322.57725959)(482.01016451,322.47726336)
\curveto(481.99015638,322.37725979)(481.95515642,322.31225985)(481.90516451,322.28226336)
\curveto(481.85515652,322.24225992)(481.78015659,322.21225995)(481.68016451,322.19226336)
\curveto(481.59015678,322.18225998)(481.48515689,322.17225999)(481.36516451,322.16226336)
\lineto(481.02016451,322.16226336)
\curveto(480.91015746,322.17225999)(480.81015756,322.17725999)(480.72016451,322.17726336)
\lineto(477.06016451,322.17726336)
\lineto(476.85016451,322.17726336)
\curveto(476.79016158,322.17725999)(476.73516164,322.16726)(476.68516451,322.14726336)
\curveto(476.60516177,322.10726006)(476.55516182,322.0672601)(476.53516451,322.02726336)
\curveto(476.51516186,322.00726016)(476.49516188,321.9672602)(476.47516451,321.90726336)
\curveto(476.45516192,321.85726031)(476.45016192,321.80726036)(476.46016451,321.75726336)
\curveto(476.48016189,321.69726047)(476.49016188,321.63726053)(476.49016451,321.57726336)
\curveto(476.50016187,321.52726064)(476.51516186,321.47226069)(476.53516451,321.41226336)
\curveto(476.61516176,321.17226099)(476.71016166,320.97226119)(476.82016451,320.81226336)
\curveto(476.94016143,320.6622615)(477.10016127,320.52726164)(477.30016451,320.40726336)
\curveto(477.38016099,320.35726181)(477.46016091,320.32226184)(477.54016451,320.30226336)
\curveto(477.63016074,320.29226187)(477.72016065,320.27226189)(477.81016451,320.24226336)
\curveto(477.89016048,320.22226194)(478.00016037,320.20726196)(478.14016451,320.19726336)
\curveto(478.28016009,320.18726198)(478.40015997,320.19226197)(478.50016451,320.21226336)
\lineto(478.63516451,320.21226336)
\curveto(478.73515964,320.23226193)(478.82515955,320.25226191)(478.90516451,320.27226336)
\curveto(478.99515938,320.30226186)(479.08015929,320.33226183)(479.16016451,320.36226336)
\curveto(479.26015911,320.41226175)(479.370159,320.47726169)(479.49016451,320.55726336)
\curveto(479.62015875,320.63726153)(479.71515866,320.71726145)(479.77516451,320.79726336)
\curveto(479.82515855,320.8672613)(479.8751585,320.93226123)(479.92516451,320.99226336)
\curveto(479.98515839,321.0622611)(480.05515832,321.11226105)(480.13516451,321.14226336)
\curveto(480.23515814,321.19226097)(480.36015801,321.21226095)(480.51016451,321.20226336)
\lineto(480.94516451,321.20226336)
\lineto(481.12516451,321.20226336)
\curveto(481.19515718,321.21226095)(481.25515712,321.20726096)(481.30516451,321.18726336)
\lineto(481.45516451,321.18726336)
\curveto(481.55515682,321.167261)(481.62515675,321.14226102)(481.66516451,321.11226336)
\curveto(481.70515667,321.09226107)(481.72515665,321.04726112)(481.72516451,320.97726336)
\curveto(481.73515664,320.90726126)(481.73015664,320.84726132)(481.71016451,320.79726336)
\curveto(481.66015671,320.65726151)(481.60515677,320.53226163)(481.54516451,320.42226336)
\curveto(481.48515689,320.31226185)(481.41515696,320.20226196)(481.33516451,320.09226336)
\curveto(481.11515726,319.7622624)(480.86515751,319.49726267)(480.58516451,319.29726336)
\curveto(480.30515807,319.09726307)(479.95515842,318.92726324)(479.53516451,318.78726336)
\curveto(479.42515895,318.74726342)(479.31515906,318.72226344)(479.20516451,318.71226336)
\curveto(479.09515928,318.70226346)(478.98015939,318.68226348)(478.86016451,318.65226336)
\curveto(478.82015955,318.64226352)(478.7751596,318.64226352)(478.72516451,318.65226336)
\curveto(478.68515969,318.65226351)(478.64515973,318.64726352)(478.60516451,318.63726336)
\lineto(478.44016451,318.63726336)
\curveto(478.39015998,318.61726355)(478.33016004,318.61226355)(478.26016451,318.62226336)
\curveto(478.20016017,318.62226354)(478.14516023,318.62726354)(478.09516451,318.63726336)
\curveto(478.01516036,318.64726352)(477.94516043,318.64726352)(477.88516451,318.63726336)
\curveto(477.82516055,318.62726354)(477.76016061,318.63226353)(477.69016451,318.65226336)
\curveto(477.64016073,318.67226349)(477.58516079,318.68226348)(477.52516451,318.68226336)
\curveto(477.46516091,318.68226348)(477.41016096,318.69226347)(477.36016451,318.71226336)
\curveto(477.25016112,318.73226343)(477.14016123,318.75726341)(477.03016451,318.78726336)
\curveto(476.92016145,318.80726336)(476.82016155,318.84226332)(476.73016451,318.89226336)
\curveto(476.62016175,318.93226323)(476.51516186,318.9672632)(476.41516451,318.99726336)
\curveto(476.32516205,319.03726313)(476.24016213,319.08226308)(476.16016451,319.13226336)
\curveto(475.84016253,319.33226283)(475.55516282,319.5622626)(475.30516451,319.82226336)
\curveto(475.05516332,320.09226207)(474.85016352,320.40226176)(474.69016451,320.75226336)
\curveto(474.64016373,320.8622613)(474.60016377,320.97226119)(474.57016451,321.08226336)
\curveto(474.54016383,321.20226096)(474.50016387,321.32226084)(474.45016451,321.44226336)
\curveto(474.44016393,321.48226068)(474.43516394,321.51726065)(474.43516451,321.54726336)
\curveto(474.43516394,321.58726058)(474.43016394,321.62726054)(474.42016451,321.66726336)
\curveto(474.38016399,321.78726038)(474.35516402,321.91726025)(474.34516451,322.05726336)
\lineto(474.31516451,322.47726336)
\curveto(474.31516406,322.52725964)(474.31016406,322.58225958)(474.30016451,322.64226336)
\curveto(474.30016407,322.70225946)(474.30516407,322.75725941)(474.31516451,322.80726336)
\lineto(474.31516451,322.98726336)
\lineto(474.36016451,323.34726336)
\curveto(474.40016397,323.51725865)(474.43516394,323.68225848)(474.46516451,323.84226336)
\curveto(474.49516388,324.00225816)(474.54016383,324.15225801)(474.60016451,324.29226336)
\curveto(475.03016334,325.33225683)(475.76016261,326.0672561)(476.79016451,326.49726336)
\curveto(476.93016144,326.55725561)(477.0701613,326.59725557)(477.21016451,326.61726336)
\curveto(477.36016101,326.64725552)(477.51516086,326.68225548)(477.67516451,326.72226336)
\curveto(477.75516062,326.73225543)(477.83016054,326.73725543)(477.90016451,326.73726336)
\curveto(477.9701604,326.73725543)(478.04516033,326.74225542)(478.12516451,326.75226336)
\curveto(478.63515974,326.7622554)(479.0701593,326.70225546)(479.43016451,326.57226336)
\curveto(479.80015857,326.45225571)(480.13015824,326.29225587)(480.42016451,326.09226336)
\curveto(480.51015786,326.03225613)(480.60015777,325.9622562)(480.69016451,325.88226336)
\curveto(480.78015759,325.81225635)(480.86015751,325.73725643)(480.93016451,325.65726336)
\curveto(480.96015741,325.60725656)(481.00015737,325.5672566)(481.05016451,325.53726336)
\curveto(481.13015724,325.42725674)(481.20515717,325.31225685)(481.27516451,325.19226336)
\curveto(481.34515703,325.08225708)(481.42015695,324.9672572)(481.50016451,324.84726336)
\curveto(481.55015682,324.75725741)(481.59015678,324.6622575)(481.62016451,324.56226336)
\curveto(481.66015671,324.47225769)(481.70015667,324.37225779)(481.74016451,324.26226336)
\curveto(481.79015658,324.13225803)(481.83015654,323.99725817)(481.86016451,323.85726336)
\curveto(481.89015648,323.71725845)(481.92515645,323.57725859)(481.96516451,323.43726336)
\curveto(481.98515639,323.35725881)(481.99015638,323.2672589)(481.98016451,323.16726336)
\curveto(481.98015639,323.07725909)(481.99015638,322.99225917)(482.01016451,322.91226336)
\lineto(482.01016451,322.74726336)
\moveto(479.76016451,323.63226336)
\curveto(479.83015854,323.73225843)(479.83515854,323.85225831)(479.77516451,323.99226336)
\curveto(479.72515865,324.14225802)(479.68515869,324.25225791)(479.65516451,324.32226336)
\curveto(479.51515886,324.59225757)(479.33015904,324.79725737)(479.10016451,324.93726336)
\curveto(478.8701595,325.08725708)(478.55015982,325.167257)(478.14016451,325.17726336)
\curveto(478.11016026,325.15725701)(478.0751603,325.15225701)(478.03516451,325.16226336)
\curveto(477.99516038,325.17225699)(477.96016041,325.17225699)(477.93016451,325.16226336)
\curveto(477.88016049,325.14225702)(477.82516055,325.12725704)(477.76516451,325.11726336)
\curveto(477.70516067,325.11725705)(477.65016072,325.10725706)(477.60016451,325.08726336)
\curveto(477.16016121,324.94725722)(476.83516154,324.67225749)(476.62516451,324.26226336)
\curveto(476.60516177,324.22225794)(476.58016179,324.167258)(476.55016451,324.09726336)
\curveto(476.53016184,324.03725813)(476.51516186,323.97225819)(476.50516451,323.90226336)
\curveto(476.49516188,323.84225832)(476.49516188,323.78225838)(476.50516451,323.72226336)
\curveto(476.52516185,323.6622585)(476.56016181,323.61225855)(476.61016451,323.57226336)
\curveto(476.69016168,323.52225864)(476.80016157,323.49725867)(476.94016451,323.49726336)
\lineto(477.34516451,323.49726336)
\lineto(479.01016451,323.49726336)
\lineto(479.44516451,323.49726336)
\curveto(479.60515877,323.50725866)(479.71015866,323.55225861)(479.76016451,323.63226336)
}
}
{
\newrgbcolor{curcolor}{0 0 0}
\pscustom[linestyle=none,fillstyle=solid,fillcolor=curcolor]
{
}
}
{
\newrgbcolor{curcolor}{0 0 0}
\pscustom[linestyle=none,fillstyle=solid,fillcolor=curcolor]
{
\newpath
\moveto(491.84360201,326.73726336)
\curveto(491.95359669,326.73725543)(492.0485966,326.72725544)(492.12860201,326.70726336)
\curveto(492.21859643,326.68725548)(492.28859636,326.64225552)(492.33860201,326.57226336)
\curveto(492.39859625,326.49225567)(492.42859622,326.35225581)(492.42860201,326.15226336)
\lineto(492.42860201,325.64226336)
\lineto(492.42860201,325.26726336)
\curveto(492.43859621,325.12725704)(492.42359622,325.01725715)(492.38360201,324.93726336)
\curveto(492.3435963,324.8672573)(492.28359636,324.82225734)(492.20360201,324.80226336)
\curveto(492.13359651,324.78225738)(492.0485966,324.77225739)(491.94860201,324.77226336)
\curveto(491.85859679,324.77225739)(491.75859689,324.77725739)(491.64860201,324.78726336)
\curveto(491.5485971,324.79725737)(491.45359719,324.79225737)(491.36360201,324.77226336)
\curveto(491.29359735,324.75225741)(491.22359742,324.73725743)(491.15360201,324.72726336)
\curveto(491.08359756,324.72725744)(491.01859763,324.71725745)(490.95860201,324.69726336)
\curveto(490.79859785,324.64725752)(490.63859801,324.57225759)(490.47860201,324.47226336)
\curveto(490.31859833,324.38225778)(490.19359845,324.27725789)(490.10360201,324.15726336)
\curveto(490.05359859,324.07725809)(489.99859865,323.99225817)(489.93860201,323.90226336)
\curveto(489.88859876,323.82225834)(489.83859881,323.73725843)(489.78860201,323.64726336)
\curveto(489.75859889,323.5672586)(489.72859892,323.48225868)(489.69860201,323.39226336)
\lineto(489.63860201,323.15226336)
\curveto(489.61859903,323.08225908)(489.60859904,323.00725916)(489.60860201,322.92726336)
\curveto(489.60859904,322.85725931)(489.59859905,322.78725938)(489.57860201,322.71726336)
\curveto(489.56859908,322.67725949)(489.56359908,322.63725953)(489.56360201,322.59726336)
\curveto(489.57359907,322.5672596)(489.57359907,322.53725963)(489.56360201,322.50726336)
\lineto(489.56360201,322.26726336)
\curveto(489.5435991,322.19725997)(489.53859911,322.11726005)(489.54860201,322.02726336)
\curveto(489.55859909,321.94726022)(489.56359908,321.8672603)(489.56360201,321.78726336)
\lineto(489.56360201,320.82726336)
\lineto(489.56360201,319.55226336)
\curveto(489.56359908,319.42226274)(489.55859909,319.30226286)(489.54860201,319.19226336)
\curveto(489.53859911,319.08226308)(489.50859914,318.99226317)(489.45860201,318.92226336)
\curveto(489.43859921,318.89226327)(489.40359924,318.8672633)(489.35360201,318.84726336)
\curveto(489.31359933,318.83726333)(489.26859938,318.82726334)(489.21860201,318.81726336)
\lineto(489.14360201,318.81726336)
\curveto(489.09359955,318.80726336)(489.03859961,318.80226336)(488.97860201,318.80226336)
\lineto(488.81360201,318.80226336)
\lineto(488.16860201,318.80226336)
\curveto(488.10860054,318.81226335)(488.0436006,318.81726335)(487.97360201,318.81726336)
\lineto(487.77860201,318.81726336)
\curveto(487.72860092,318.83726333)(487.67860097,318.85226331)(487.62860201,318.86226336)
\curveto(487.57860107,318.88226328)(487.5436011,318.91726325)(487.52360201,318.96726336)
\curveto(487.48360116,319.01726315)(487.45860119,319.08726308)(487.44860201,319.17726336)
\lineto(487.44860201,319.47726336)
\lineto(487.44860201,320.49726336)
\lineto(487.44860201,324.72726336)
\lineto(487.44860201,325.83726336)
\lineto(487.44860201,326.12226336)
\curveto(487.4486012,326.22225594)(487.46860118,326.30225586)(487.50860201,326.36226336)
\curveto(487.55860109,326.44225572)(487.63360101,326.49225567)(487.73360201,326.51226336)
\curveto(487.83360081,326.53225563)(487.95360069,326.54225562)(488.09360201,326.54226336)
\lineto(488.85860201,326.54226336)
\curveto(488.97859967,326.54225562)(489.08359956,326.53225563)(489.17360201,326.51226336)
\curveto(489.26359938,326.50225566)(489.33359931,326.45725571)(489.38360201,326.37726336)
\curveto(489.41359923,326.32725584)(489.42859922,326.25725591)(489.42860201,326.16726336)
\lineto(489.45860201,325.89726336)
\curveto(489.46859918,325.81725635)(489.48359916,325.74225642)(489.50360201,325.67226336)
\curveto(489.53359911,325.60225656)(489.58359906,325.5672566)(489.65360201,325.56726336)
\curveto(489.67359897,325.58725658)(489.69359895,325.59725657)(489.71360201,325.59726336)
\curveto(489.73359891,325.59725657)(489.75359889,325.60725656)(489.77360201,325.62726336)
\curveto(489.83359881,325.67725649)(489.88359876,325.73225643)(489.92360201,325.79226336)
\curveto(489.97359867,325.8622563)(490.03359861,325.92225624)(490.10360201,325.97226336)
\curveto(490.1435985,326.00225616)(490.17859847,326.03225613)(490.20860201,326.06226336)
\curveto(490.23859841,326.10225606)(490.27359837,326.13725603)(490.31360201,326.16726336)
\lineto(490.58360201,326.34726336)
\curveto(490.68359796,326.40725576)(490.78359786,326.4622557)(490.88360201,326.51226336)
\curveto(490.98359766,326.55225561)(491.08359756,326.58725558)(491.18360201,326.61726336)
\lineto(491.51360201,326.70726336)
\curveto(491.5435971,326.71725545)(491.59859705,326.71725545)(491.67860201,326.70726336)
\curveto(491.76859688,326.70725546)(491.82359682,326.71725545)(491.84360201,326.73726336)
}
}
{
\newrgbcolor{curcolor}{0 0 0}
\pscustom[linestyle=none,fillstyle=solid,fillcolor=curcolor]
{
\newpath
\moveto(500.35000826,322.74726336)
\curveto(500.37000009,322.6672595)(500.37000009,322.57725959)(500.35000826,322.47726336)
\curveto(500.33000013,322.37725979)(500.29500017,322.31225985)(500.24500826,322.28226336)
\curveto(500.19500027,322.24225992)(500.12000034,322.21225995)(500.02000826,322.19226336)
\curveto(499.93000053,322.18225998)(499.82500064,322.17225999)(499.70500826,322.16226336)
\lineto(499.36000826,322.16226336)
\curveto(499.25000121,322.17225999)(499.15000131,322.17725999)(499.06000826,322.17726336)
\lineto(495.40000826,322.17726336)
\lineto(495.19000826,322.17726336)
\curveto(495.13000533,322.17725999)(495.07500539,322.16726)(495.02500826,322.14726336)
\curveto(494.94500552,322.10726006)(494.89500557,322.0672601)(494.87500826,322.02726336)
\curveto(494.85500561,322.00726016)(494.83500563,321.9672602)(494.81500826,321.90726336)
\curveto(494.79500567,321.85726031)(494.79000567,321.80726036)(494.80000826,321.75726336)
\curveto(494.82000564,321.69726047)(494.83000563,321.63726053)(494.83000826,321.57726336)
\curveto(494.84000562,321.52726064)(494.85500561,321.47226069)(494.87500826,321.41226336)
\curveto(494.95500551,321.17226099)(495.05000541,320.97226119)(495.16000826,320.81226336)
\curveto(495.28000518,320.6622615)(495.44000502,320.52726164)(495.64000826,320.40726336)
\curveto(495.72000474,320.35726181)(495.80000466,320.32226184)(495.88000826,320.30226336)
\curveto(495.97000449,320.29226187)(496.0600044,320.27226189)(496.15000826,320.24226336)
\curveto(496.23000423,320.22226194)(496.34000412,320.20726196)(496.48000826,320.19726336)
\curveto(496.62000384,320.18726198)(496.74000372,320.19226197)(496.84000826,320.21226336)
\lineto(496.97500826,320.21226336)
\curveto(497.07500339,320.23226193)(497.1650033,320.25226191)(497.24500826,320.27226336)
\curveto(497.33500313,320.30226186)(497.42000304,320.33226183)(497.50000826,320.36226336)
\curveto(497.60000286,320.41226175)(497.71000275,320.47726169)(497.83000826,320.55726336)
\curveto(497.9600025,320.63726153)(498.05500241,320.71726145)(498.11500826,320.79726336)
\curveto(498.1650023,320.8672613)(498.21500225,320.93226123)(498.26500826,320.99226336)
\curveto(498.32500214,321.0622611)(498.39500207,321.11226105)(498.47500826,321.14226336)
\curveto(498.57500189,321.19226097)(498.70000176,321.21226095)(498.85000826,321.20226336)
\lineto(499.28500826,321.20226336)
\lineto(499.46500826,321.20226336)
\curveto(499.53500093,321.21226095)(499.59500087,321.20726096)(499.64500826,321.18726336)
\lineto(499.79500826,321.18726336)
\curveto(499.89500057,321.167261)(499.9650005,321.14226102)(500.00500826,321.11226336)
\curveto(500.04500042,321.09226107)(500.0650004,321.04726112)(500.06500826,320.97726336)
\curveto(500.07500039,320.90726126)(500.07000039,320.84726132)(500.05000826,320.79726336)
\curveto(500.00000046,320.65726151)(499.94500052,320.53226163)(499.88500826,320.42226336)
\curveto(499.82500064,320.31226185)(499.75500071,320.20226196)(499.67500826,320.09226336)
\curveto(499.45500101,319.7622624)(499.20500126,319.49726267)(498.92500826,319.29726336)
\curveto(498.64500182,319.09726307)(498.29500217,318.92726324)(497.87500826,318.78726336)
\curveto(497.7650027,318.74726342)(497.65500281,318.72226344)(497.54500826,318.71226336)
\curveto(497.43500303,318.70226346)(497.32000314,318.68226348)(497.20000826,318.65226336)
\curveto(497.1600033,318.64226352)(497.11500335,318.64226352)(497.06500826,318.65226336)
\curveto(497.02500344,318.65226351)(496.98500348,318.64726352)(496.94500826,318.63726336)
\lineto(496.78000826,318.63726336)
\curveto(496.73000373,318.61726355)(496.67000379,318.61226355)(496.60000826,318.62226336)
\curveto(496.54000392,318.62226354)(496.48500398,318.62726354)(496.43500826,318.63726336)
\curveto(496.35500411,318.64726352)(496.28500418,318.64726352)(496.22500826,318.63726336)
\curveto(496.1650043,318.62726354)(496.10000436,318.63226353)(496.03000826,318.65226336)
\curveto(495.98000448,318.67226349)(495.92500454,318.68226348)(495.86500826,318.68226336)
\curveto(495.80500466,318.68226348)(495.75000471,318.69226347)(495.70000826,318.71226336)
\curveto(495.59000487,318.73226343)(495.48000498,318.75726341)(495.37000826,318.78726336)
\curveto(495.2600052,318.80726336)(495.1600053,318.84226332)(495.07000826,318.89226336)
\curveto(494.9600055,318.93226323)(494.85500561,318.9672632)(494.75500826,318.99726336)
\curveto(494.6650058,319.03726313)(494.58000588,319.08226308)(494.50000826,319.13226336)
\curveto(494.18000628,319.33226283)(493.89500657,319.5622626)(493.64500826,319.82226336)
\curveto(493.39500707,320.09226207)(493.19000727,320.40226176)(493.03000826,320.75226336)
\curveto(492.98000748,320.8622613)(492.94000752,320.97226119)(492.91000826,321.08226336)
\curveto(492.88000758,321.20226096)(492.84000762,321.32226084)(492.79000826,321.44226336)
\curveto(492.78000768,321.48226068)(492.77500769,321.51726065)(492.77500826,321.54726336)
\curveto(492.77500769,321.58726058)(492.77000769,321.62726054)(492.76000826,321.66726336)
\curveto(492.72000774,321.78726038)(492.69500777,321.91726025)(492.68500826,322.05726336)
\lineto(492.65500826,322.47726336)
\curveto(492.65500781,322.52725964)(492.65000781,322.58225958)(492.64000826,322.64226336)
\curveto(492.64000782,322.70225946)(492.64500782,322.75725941)(492.65500826,322.80726336)
\lineto(492.65500826,322.98726336)
\lineto(492.70000826,323.34726336)
\curveto(492.74000772,323.51725865)(492.77500769,323.68225848)(492.80500826,323.84226336)
\curveto(492.83500763,324.00225816)(492.88000758,324.15225801)(492.94000826,324.29226336)
\curveto(493.37000709,325.33225683)(494.10000636,326.0672561)(495.13000826,326.49726336)
\curveto(495.27000519,326.55725561)(495.41000505,326.59725557)(495.55000826,326.61726336)
\curveto(495.70000476,326.64725552)(495.85500461,326.68225548)(496.01500826,326.72226336)
\curveto(496.09500437,326.73225543)(496.17000429,326.73725543)(496.24000826,326.73726336)
\curveto(496.31000415,326.73725543)(496.38500408,326.74225542)(496.46500826,326.75226336)
\curveto(496.97500349,326.7622554)(497.41000305,326.70225546)(497.77000826,326.57226336)
\curveto(498.14000232,326.45225571)(498.47000199,326.29225587)(498.76000826,326.09226336)
\curveto(498.85000161,326.03225613)(498.94000152,325.9622562)(499.03000826,325.88226336)
\curveto(499.12000134,325.81225635)(499.20000126,325.73725643)(499.27000826,325.65726336)
\curveto(499.30000116,325.60725656)(499.34000112,325.5672566)(499.39000826,325.53726336)
\curveto(499.47000099,325.42725674)(499.54500092,325.31225685)(499.61500826,325.19226336)
\curveto(499.68500078,325.08225708)(499.7600007,324.9672572)(499.84000826,324.84726336)
\curveto(499.89000057,324.75725741)(499.93000053,324.6622575)(499.96000826,324.56226336)
\curveto(500.00000046,324.47225769)(500.04000042,324.37225779)(500.08000826,324.26226336)
\curveto(500.13000033,324.13225803)(500.17000029,323.99725817)(500.20000826,323.85726336)
\curveto(500.23000023,323.71725845)(500.2650002,323.57725859)(500.30500826,323.43726336)
\curveto(500.32500014,323.35725881)(500.33000013,323.2672589)(500.32000826,323.16726336)
\curveto(500.32000014,323.07725909)(500.33000013,322.99225917)(500.35000826,322.91226336)
\lineto(500.35000826,322.74726336)
\moveto(498.10000826,323.63226336)
\curveto(498.17000229,323.73225843)(498.17500229,323.85225831)(498.11500826,323.99226336)
\curveto(498.0650024,324.14225802)(498.02500244,324.25225791)(497.99500826,324.32226336)
\curveto(497.85500261,324.59225757)(497.67000279,324.79725737)(497.44000826,324.93726336)
\curveto(497.21000325,325.08725708)(496.89000357,325.167257)(496.48000826,325.17726336)
\curveto(496.45000401,325.15725701)(496.41500405,325.15225701)(496.37500826,325.16226336)
\curveto(496.33500413,325.17225699)(496.30000416,325.17225699)(496.27000826,325.16226336)
\curveto(496.22000424,325.14225702)(496.1650043,325.12725704)(496.10500826,325.11726336)
\curveto(496.04500442,325.11725705)(495.99000447,325.10725706)(495.94000826,325.08726336)
\curveto(495.50000496,324.94725722)(495.17500529,324.67225749)(494.96500826,324.26226336)
\curveto(494.94500552,324.22225794)(494.92000554,324.167258)(494.89000826,324.09726336)
\curveto(494.87000559,324.03725813)(494.85500561,323.97225819)(494.84500826,323.90226336)
\curveto(494.83500563,323.84225832)(494.83500563,323.78225838)(494.84500826,323.72226336)
\curveto(494.8650056,323.6622585)(494.90000556,323.61225855)(494.95000826,323.57226336)
\curveto(495.03000543,323.52225864)(495.14000532,323.49725867)(495.28000826,323.49726336)
\lineto(495.68500826,323.49726336)
\lineto(497.35000826,323.49726336)
\lineto(497.78500826,323.49726336)
\curveto(497.94500252,323.50725866)(498.05000241,323.55225861)(498.10000826,323.63226336)
}
}
{
\newrgbcolor{curcolor}{0 0 0}
\pscustom[linestyle=none,fillstyle=solid,fillcolor=curcolor]
{
\newpath
\moveto(505.16828951,326.75226336)
\curveto(505.97828435,326.77225539)(506.65328367,326.65225551)(507.19328951,326.39226336)
\curveto(507.74328258,326.13225603)(508.17828215,325.7622564)(508.49828951,325.28226336)
\curveto(508.65828167,325.04225712)(508.77828155,324.7672574)(508.85828951,324.45726336)
\curveto(508.87828145,324.40725776)(508.89328143,324.34225782)(508.90328951,324.26226336)
\curveto(508.9232814,324.18225798)(508.9232814,324.11225805)(508.90328951,324.05226336)
\curveto(508.86328146,323.94225822)(508.79328153,323.87725829)(508.69328951,323.85726336)
\curveto(508.59328173,323.84725832)(508.47328185,323.84225832)(508.33328951,323.84226336)
\lineto(507.55328951,323.84226336)
\lineto(507.26828951,323.84226336)
\curveto(507.17828315,323.84225832)(507.10328322,323.8622583)(507.04328951,323.90226336)
\curveto(506.96328336,323.94225822)(506.90828342,324.00225816)(506.87828951,324.08226336)
\curveto(506.84828348,324.17225799)(506.80828352,324.2622579)(506.75828951,324.35226336)
\curveto(506.69828363,324.4622577)(506.63328369,324.5622576)(506.56328951,324.65226336)
\curveto(506.49328383,324.74225742)(506.41328391,324.82225734)(506.32328951,324.89226336)
\curveto(506.18328414,324.98225718)(506.0282843,325.05225711)(505.85828951,325.10226336)
\curveto(505.79828453,325.12225704)(505.73828459,325.13225703)(505.67828951,325.13226336)
\curveto(505.61828471,325.13225703)(505.56328476,325.14225702)(505.51328951,325.16226336)
\lineto(505.36328951,325.16226336)
\curveto(505.16328516,325.162257)(505.00328532,325.14225702)(504.88328951,325.10226336)
\curveto(504.59328573,325.01225715)(504.35828597,324.87225729)(504.17828951,324.68226336)
\curveto(503.99828633,324.50225766)(503.85328647,324.28225788)(503.74328951,324.02226336)
\curveto(503.69328663,323.91225825)(503.65328667,323.79225837)(503.62328951,323.66226336)
\curveto(503.60328672,323.54225862)(503.57828675,323.41225875)(503.54828951,323.27226336)
\curveto(503.53828679,323.23225893)(503.53328679,323.19225897)(503.53328951,323.15226336)
\curveto(503.53328679,323.11225905)(503.5282868,323.07225909)(503.51828951,323.03226336)
\curveto(503.49828683,322.93225923)(503.48828684,322.79225937)(503.48828951,322.61226336)
\curveto(503.49828683,322.43225973)(503.51328681,322.29225987)(503.53328951,322.19226336)
\curveto(503.53328679,322.11226005)(503.53828679,322.05726011)(503.54828951,322.02726336)
\curveto(503.56828676,321.95726021)(503.57828675,321.88726028)(503.57828951,321.81726336)
\curveto(503.58828674,321.74726042)(503.60328672,321.67726049)(503.62328951,321.60726336)
\curveto(503.70328662,321.37726079)(503.79828653,321.167261)(503.90828951,320.97726336)
\curveto(504.01828631,320.78726138)(504.15828617,320.62726154)(504.32828951,320.49726336)
\curveto(504.36828596,320.4672617)(504.4282859,320.43226173)(504.50828951,320.39226336)
\curveto(504.61828571,320.32226184)(504.7282856,320.27726189)(504.83828951,320.25726336)
\curveto(504.95828537,320.23726193)(505.10328522,320.21726195)(505.27328951,320.19726336)
\lineto(505.36328951,320.19726336)
\curveto(505.40328492,320.19726197)(505.43328489,320.20226196)(505.45328951,320.21226336)
\lineto(505.58828951,320.21226336)
\curveto(505.65828467,320.23226193)(505.7232846,320.24726192)(505.78328951,320.25726336)
\curveto(505.85328447,320.27726189)(505.91828441,320.29726187)(505.97828951,320.31726336)
\curveto(506.27828405,320.44726172)(506.50828382,320.63726153)(506.66828951,320.88726336)
\curveto(506.70828362,320.93726123)(506.74328358,320.99226117)(506.77328951,321.05226336)
\curveto(506.80328352,321.12226104)(506.8282835,321.18226098)(506.84828951,321.23226336)
\curveto(506.88828344,321.34226082)(506.9232834,321.43726073)(506.95328951,321.51726336)
\curveto(506.98328334,321.60726056)(507.05328327,321.67726049)(507.16328951,321.72726336)
\curveto(507.25328307,321.7672604)(507.39828293,321.78226038)(507.59828951,321.77226336)
\lineto(508.09328951,321.77226336)
\lineto(508.30328951,321.77226336)
\curveto(508.38328194,321.78226038)(508.44828188,321.77726039)(508.49828951,321.75726336)
\lineto(508.61828951,321.75726336)
\lineto(508.73828951,321.72726336)
\curveto(508.77828155,321.72726044)(508.80828152,321.71726045)(508.82828951,321.69726336)
\curveto(508.87828145,321.65726051)(508.90828142,321.59726057)(508.91828951,321.51726336)
\curveto(508.93828139,321.44726072)(508.93828139,321.37226079)(508.91828951,321.29226336)
\curveto(508.8282815,320.9622612)(508.71828161,320.6672615)(508.58828951,320.40726336)
\curveto(508.17828215,319.63726253)(507.5232828,319.10226306)(506.62328951,318.80226336)
\curveto(506.5232838,318.77226339)(506.41828391,318.75226341)(506.30828951,318.74226336)
\curveto(506.19828413,318.72226344)(506.08828424,318.69726347)(505.97828951,318.66726336)
\curveto(505.91828441,318.65726351)(505.85828447,318.65226351)(505.79828951,318.65226336)
\curveto(505.73828459,318.65226351)(505.67828465,318.64726352)(505.61828951,318.63726336)
\lineto(505.45328951,318.63726336)
\curveto(505.40328492,318.61726355)(505.328285,318.61226355)(505.22828951,318.62226336)
\curveto(505.1282852,318.62226354)(505.05328527,318.62726354)(505.00328951,318.63726336)
\curveto(504.9232854,318.65726351)(504.84828548,318.6672635)(504.77828951,318.66726336)
\curveto(504.71828561,318.65726351)(504.65328567,318.6622635)(504.58328951,318.68226336)
\lineto(504.43328951,318.71226336)
\curveto(504.38328594,318.71226345)(504.33328599,318.71726345)(504.28328951,318.72726336)
\curveto(504.17328615,318.75726341)(504.06828626,318.78726338)(503.96828951,318.81726336)
\curveto(503.86828646,318.84726332)(503.77328655,318.88226328)(503.68328951,318.92226336)
\curveto(503.21328711,319.12226304)(502.81828751,319.37726279)(502.49828951,319.68726336)
\curveto(502.17828815,320.00726216)(501.91828841,320.40226176)(501.71828951,320.87226336)
\curveto(501.66828866,320.9622612)(501.6282887,321.05726111)(501.59828951,321.15726336)
\lineto(501.50828951,321.48726336)
\curveto(501.49828883,321.52726064)(501.49328883,321.5622606)(501.49328951,321.59226336)
\curveto(501.49328883,321.63226053)(501.48328884,321.67726049)(501.46328951,321.72726336)
\curveto(501.44328888,321.79726037)(501.43328889,321.8672603)(501.43328951,321.93726336)
\curveto(501.43328889,322.01726015)(501.4232889,322.09226007)(501.40328951,322.16226336)
\lineto(501.40328951,322.41726336)
\curveto(501.38328894,322.4672597)(501.37328895,322.52225964)(501.37328951,322.58226336)
\curveto(501.37328895,322.65225951)(501.38328894,322.71225945)(501.40328951,322.76226336)
\curveto(501.41328891,322.81225935)(501.41328891,322.85725931)(501.40328951,322.89726336)
\curveto(501.39328893,322.93725923)(501.39328893,322.97725919)(501.40328951,323.01726336)
\curveto(501.4232889,323.08725908)(501.4282889,323.15225901)(501.41828951,323.21226336)
\curveto(501.41828891,323.27225889)(501.4282889,323.33225883)(501.44828951,323.39226336)
\curveto(501.49828883,323.57225859)(501.53828879,323.74225842)(501.56828951,323.90226336)
\curveto(501.59828873,324.07225809)(501.64328868,324.23725793)(501.70328951,324.39726336)
\curveto(501.9232884,324.90725726)(502.19828813,325.33225683)(502.52828951,325.67226336)
\curveto(502.86828746,326.01225615)(503.29828703,326.28725588)(503.81828951,326.49726336)
\curveto(503.95828637,326.55725561)(504.10328622,326.59725557)(504.25328951,326.61726336)
\curveto(504.40328592,326.64725552)(504.55828577,326.68225548)(504.71828951,326.72226336)
\curveto(504.79828553,326.73225543)(504.87328545,326.73725543)(504.94328951,326.73726336)
\curveto(505.01328531,326.73725543)(505.08828524,326.74225542)(505.16828951,326.75226336)
}
}
{
\newrgbcolor{curcolor}{0 0 0}
\pscustom[linestyle=none,fillstyle=solid,fillcolor=curcolor]
{
\newpath
\moveto(510.63157076,326.52726336)
\lineto(511.75657076,326.52726336)
\curveto(511.86656832,326.52725564)(511.96656822,326.52225564)(512.05657076,326.51226336)
\curveto(512.14656804,326.50225566)(512.21156798,326.4672557)(512.25157076,326.40726336)
\curveto(512.30156789,326.34725582)(512.33156786,326.2622559)(512.34157076,326.15226336)
\curveto(512.35156784,326.05225611)(512.35656783,325.94725622)(512.35657076,325.83726336)
\lineto(512.35657076,324.78726336)
\lineto(512.35657076,322.55226336)
\curveto(512.35656783,322.19225997)(512.37156782,321.85226031)(512.40157076,321.53226336)
\curveto(512.43156776,321.21226095)(512.52156767,320.94726122)(512.67157076,320.73726336)
\curveto(512.81156738,320.52726164)(513.03656715,320.37726179)(513.34657076,320.28726336)
\curveto(513.39656679,320.27726189)(513.43656675,320.27226189)(513.46657076,320.27226336)
\curveto(513.50656668,320.27226189)(513.55156664,320.2672619)(513.60157076,320.25726336)
\curveto(513.65156654,320.24726192)(513.70656648,320.24226192)(513.76657076,320.24226336)
\curveto(513.82656636,320.24226192)(513.87156632,320.24726192)(513.90157076,320.25726336)
\curveto(513.95156624,320.27726189)(513.9915662,320.28226188)(514.02157076,320.27226336)
\curveto(514.06156613,320.2622619)(514.10156609,320.2672619)(514.14157076,320.28726336)
\curveto(514.35156584,320.33726183)(514.51656567,320.40226176)(514.63657076,320.48226336)
\curveto(514.81656537,320.59226157)(514.95656523,320.73226143)(515.05657076,320.90226336)
\curveto(515.16656502,321.08226108)(515.24156495,321.27726089)(515.28157076,321.48726336)
\curveto(515.33156486,321.70726046)(515.36156483,321.94726022)(515.37157076,322.20726336)
\curveto(515.38156481,322.47725969)(515.3865648,322.75725941)(515.38657076,323.04726336)
\lineto(515.38657076,324.86226336)
\lineto(515.38657076,325.83726336)
\lineto(515.38657076,326.10726336)
\curveto(515.3865648,326.20725596)(515.40656478,326.28725588)(515.44657076,326.34726336)
\curveto(515.49656469,326.43725573)(515.57156462,326.48725568)(515.67157076,326.49726336)
\curveto(515.77156442,326.51725565)(515.8915643,326.52725564)(516.03157076,326.52726336)
\lineto(516.82657076,326.52726336)
\lineto(517.11157076,326.52726336)
\curveto(517.20156299,326.52725564)(517.27656291,326.50725566)(517.33657076,326.46726336)
\curveto(517.41656277,326.41725575)(517.46156273,326.34225582)(517.47157076,326.24226336)
\curveto(517.48156271,326.14225602)(517.4865627,326.02725614)(517.48657076,325.89726336)
\lineto(517.48657076,324.75726336)
\lineto(517.48657076,320.54226336)
\lineto(517.48657076,319.47726336)
\lineto(517.48657076,319.17726336)
\curveto(517.4865627,319.07726309)(517.46656272,319.00226316)(517.42657076,318.95226336)
\curveto(517.37656281,318.87226329)(517.30156289,318.82726334)(517.20157076,318.81726336)
\curveto(517.10156309,318.80726336)(516.99656319,318.80226336)(516.88657076,318.80226336)
\lineto(516.07657076,318.80226336)
\curveto(515.96656422,318.80226336)(515.86656432,318.80726336)(515.77657076,318.81726336)
\curveto(515.69656449,318.82726334)(515.63156456,318.8672633)(515.58157076,318.93726336)
\curveto(515.56156463,318.9672632)(515.54156465,319.01226315)(515.52157076,319.07226336)
\curveto(515.51156468,319.13226303)(515.49656469,319.19226297)(515.47657076,319.25226336)
\curveto(515.46656472,319.31226285)(515.45156474,319.3672628)(515.43157076,319.41726336)
\curveto(515.41156478,319.4672627)(515.38156481,319.49726267)(515.34157076,319.50726336)
\curveto(515.32156487,319.52726264)(515.29656489,319.53226263)(515.26657076,319.52226336)
\curveto(515.23656495,319.51226265)(515.21156498,319.50226266)(515.19157076,319.49226336)
\curveto(515.12156507,319.45226271)(515.06156513,319.40726276)(515.01157076,319.35726336)
\curveto(514.96156523,319.30726286)(514.90656528,319.2622629)(514.84657076,319.22226336)
\curveto(514.80656538,319.19226297)(514.76656542,319.15726301)(514.72657076,319.11726336)
\curveto(514.69656549,319.08726308)(514.65656553,319.05726311)(514.60657076,319.02726336)
\curveto(514.37656581,318.88726328)(514.10656608,318.77726339)(513.79657076,318.69726336)
\curveto(513.72656646,318.67726349)(513.65656653,318.6672635)(513.58657076,318.66726336)
\curveto(513.51656667,318.65726351)(513.44156675,318.64226352)(513.36157076,318.62226336)
\curveto(513.32156687,318.61226355)(513.27656691,318.61226355)(513.22657076,318.62226336)
\curveto(513.186567,318.62226354)(513.14656704,318.61726355)(513.10657076,318.60726336)
\curveto(513.07656711,318.59726357)(513.01156718,318.59726357)(512.91157076,318.60726336)
\curveto(512.82156737,318.60726356)(512.76156743,318.61226355)(512.73157076,318.62226336)
\curveto(512.68156751,318.62226354)(512.63156756,318.62726354)(512.58157076,318.63726336)
\lineto(512.43157076,318.63726336)
\curveto(512.31156788,318.6672635)(512.19656799,318.69226347)(512.08657076,318.71226336)
\curveto(511.97656821,318.73226343)(511.86656832,318.7622634)(511.75657076,318.80226336)
\curveto(511.70656848,318.82226334)(511.66156853,318.83726333)(511.62157076,318.84726336)
\curveto(511.5915686,318.8672633)(511.55156864,318.88726328)(511.50157076,318.90726336)
\curveto(511.15156904,319.09726307)(510.87156932,319.3622628)(510.66157076,319.70226336)
\curveto(510.53156966,319.91226225)(510.43656975,320.162262)(510.37657076,320.45226336)
\curveto(510.31656987,320.75226141)(510.27656991,321.0672611)(510.25657076,321.39726336)
\curveto(510.24656994,321.73726043)(510.24156995,322.08226008)(510.24157076,322.43226336)
\curveto(510.25156994,322.79225937)(510.25656993,323.14725902)(510.25657076,323.49726336)
\lineto(510.25657076,325.53726336)
\curveto(510.25656993,325.6672565)(510.25156994,325.81725635)(510.24157076,325.98726336)
\curveto(510.24156995,326.167256)(510.26656992,326.29725587)(510.31657076,326.37726336)
\curveto(510.34656984,326.42725574)(510.40656978,326.47225569)(510.49657076,326.51226336)
\curveto(510.55656963,326.51225565)(510.60156959,326.51725565)(510.63157076,326.52726336)
}
}
{
\newrgbcolor{curcolor}{0 0 0}
\pscustom[linestyle=none,fillstyle=solid,fillcolor=curcolor]
{
\newpath
\moveto(523.54282076,326.73726336)
\curveto(523.65281544,326.73725543)(523.74781535,326.72725544)(523.82782076,326.70726336)
\curveto(523.91781518,326.68725548)(523.98781511,326.64225552)(524.03782076,326.57226336)
\curveto(524.097815,326.49225567)(524.12781497,326.35225581)(524.12782076,326.15226336)
\lineto(524.12782076,325.64226336)
\lineto(524.12782076,325.26726336)
\curveto(524.13781496,325.12725704)(524.12281497,325.01725715)(524.08282076,324.93726336)
\curveto(524.04281505,324.8672573)(523.98281511,324.82225734)(523.90282076,324.80226336)
\curveto(523.83281526,324.78225738)(523.74781535,324.77225739)(523.64782076,324.77226336)
\curveto(523.55781554,324.77225739)(523.45781564,324.77725739)(523.34782076,324.78726336)
\curveto(523.24781585,324.79725737)(523.15281594,324.79225737)(523.06282076,324.77226336)
\curveto(522.9928161,324.75225741)(522.92281617,324.73725743)(522.85282076,324.72726336)
\curveto(522.78281631,324.72725744)(522.71781638,324.71725745)(522.65782076,324.69726336)
\curveto(522.4978166,324.64725752)(522.33781676,324.57225759)(522.17782076,324.47226336)
\curveto(522.01781708,324.38225778)(521.8928172,324.27725789)(521.80282076,324.15726336)
\curveto(521.75281734,324.07725809)(521.6978174,323.99225817)(521.63782076,323.90226336)
\curveto(521.58781751,323.82225834)(521.53781756,323.73725843)(521.48782076,323.64726336)
\curveto(521.45781764,323.5672586)(521.42781767,323.48225868)(521.39782076,323.39226336)
\lineto(521.33782076,323.15226336)
\curveto(521.31781778,323.08225908)(521.30781779,323.00725916)(521.30782076,322.92726336)
\curveto(521.30781779,322.85725931)(521.2978178,322.78725938)(521.27782076,322.71726336)
\curveto(521.26781783,322.67725949)(521.26281783,322.63725953)(521.26282076,322.59726336)
\curveto(521.27281782,322.5672596)(521.27281782,322.53725963)(521.26282076,322.50726336)
\lineto(521.26282076,322.26726336)
\curveto(521.24281785,322.19725997)(521.23781786,322.11726005)(521.24782076,322.02726336)
\curveto(521.25781784,321.94726022)(521.26281783,321.8672603)(521.26282076,321.78726336)
\lineto(521.26282076,320.82726336)
\lineto(521.26282076,319.55226336)
\curveto(521.26281783,319.42226274)(521.25781784,319.30226286)(521.24782076,319.19226336)
\curveto(521.23781786,319.08226308)(521.20781789,318.99226317)(521.15782076,318.92226336)
\curveto(521.13781796,318.89226327)(521.10281799,318.8672633)(521.05282076,318.84726336)
\curveto(521.01281808,318.83726333)(520.96781813,318.82726334)(520.91782076,318.81726336)
\lineto(520.84282076,318.81726336)
\curveto(520.7928183,318.80726336)(520.73781836,318.80226336)(520.67782076,318.80226336)
\lineto(520.51282076,318.80226336)
\lineto(519.86782076,318.80226336)
\curveto(519.80781929,318.81226335)(519.74281935,318.81726335)(519.67282076,318.81726336)
\lineto(519.47782076,318.81726336)
\curveto(519.42781967,318.83726333)(519.37781972,318.85226331)(519.32782076,318.86226336)
\curveto(519.27781982,318.88226328)(519.24281985,318.91726325)(519.22282076,318.96726336)
\curveto(519.18281991,319.01726315)(519.15781994,319.08726308)(519.14782076,319.17726336)
\lineto(519.14782076,319.47726336)
\lineto(519.14782076,320.49726336)
\lineto(519.14782076,324.72726336)
\lineto(519.14782076,325.83726336)
\lineto(519.14782076,326.12226336)
\curveto(519.14781995,326.22225594)(519.16781993,326.30225586)(519.20782076,326.36226336)
\curveto(519.25781984,326.44225572)(519.33281976,326.49225567)(519.43282076,326.51226336)
\curveto(519.53281956,326.53225563)(519.65281944,326.54225562)(519.79282076,326.54226336)
\lineto(520.55782076,326.54226336)
\curveto(520.67781842,326.54225562)(520.78281831,326.53225563)(520.87282076,326.51226336)
\curveto(520.96281813,326.50225566)(521.03281806,326.45725571)(521.08282076,326.37726336)
\curveto(521.11281798,326.32725584)(521.12781797,326.25725591)(521.12782076,326.16726336)
\lineto(521.15782076,325.89726336)
\curveto(521.16781793,325.81725635)(521.18281791,325.74225642)(521.20282076,325.67226336)
\curveto(521.23281786,325.60225656)(521.28281781,325.5672566)(521.35282076,325.56726336)
\curveto(521.37281772,325.58725658)(521.3928177,325.59725657)(521.41282076,325.59726336)
\curveto(521.43281766,325.59725657)(521.45281764,325.60725656)(521.47282076,325.62726336)
\curveto(521.53281756,325.67725649)(521.58281751,325.73225643)(521.62282076,325.79226336)
\curveto(521.67281742,325.8622563)(521.73281736,325.92225624)(521.80282076,325.97226336)
\curveto(521.84281725,326.00225616)(521.87781722,326.03225613)(521.90782076,326.06226336)
\curveto(521.93781716,326.10225606)(521.97281712,326.13725603)(522.01282076,326.16726336)
\lineto(522.28282076,326.34726336)
\curveto(522.38281671,326.40725576)(522.48281661,326.4622557)(522.58282076,326.51226336)
\curveto(522.68281641,326.55225561)(522.78281631,326.58725558)(522.88282076,326.61726336)
\lineto(523.21282076,326.70726336)
\curveto(523.24281585,326.71725545)(523.2978158,326.71725545)(523.37782076,326.70726336)
\curveto(523.46781563,326.70725546)(523.52281557,326.71725545)(523.54282076,326.73726336)
}
}
{
\newrgbcolor{curcolor}{0 0 0}
\pscustom[linestyle=none,fillstyle=solid,fillcolor=curcolor]
{
\newpath
\moveto(527.91789888,326.75226336)
\curveto(528.66789438,326.77225539)(529.31789373,326.68725548)(529.86789888,326.49726336)
\curveto(530.42789262,326.31725585)(530.8528922,326.00225616)(531.14289888,325.55226336)
\curveto(531.21289184,325.44225672)(531.27289178,325.32725684)(531.32289888,325.20726336)
\curveto(531.38289167,325.09725707)(531.43289162,324.97225719)(531.47289888,324.83226336)
\curveto(531.49289156,324.77225739)(531.50289155,324.70725746)(531.50289888,324.63726336)
\curveto(531.50289155,324.5672576)(531.49289156,324.50725766)(531.47289888,324.45726336)
\curveto(531.43289162,324.39725777)(531.37789167,324.35725781)(531.30789888,324.33726336)
\curveto(531.25789179,324.31725785)(531.19789185,324.30725786)(531.12789888,324.30726336)
\lineto(530.91789888,324.30726336)
\lineto(530.25789888,324.30726336)
\curveto(530.18789286,324.30725786)(530.11789293,324.30225786)(530.04789888,324.29226336)
\curveto(529.97789307,324.29225787)(529.91289314,324.30225786)(529.85289888,324.32226336)
\curveto(529.7528933,324.34225782)(529.67789337,324.38225778)(529.62789888,324.44226336)
\curveto(529.57789347,324.50225766)(529.53289352,324.5622576)(529.49289888,324.62226336)
\lineto(529.37289888,324.83226336)
\curveto(529.34289371,324.91225725)(529.29289376,324.97725719)(529.22289888,325.02726336)
\curveto(529.12289393,325.10725706)(529.02289403,325.167257)(528.92289888,325.20726336)
\curveto(528.83289422,325.24725692)(528.71789433,325.28225688)(528.57789888,325.31226336)
\curveto(528.50789454,325.33225683)(528.40289465,325.34725682)(528.26289888,325.35726336)
\curveto(528.13289492,325.3672568)(528.03289502,325.3622568)(527.96289888,325.34226336)
\lineto(527.85789888,325.34226336)
\lineto(527.70789888,325.31226336)
\curveto(527.66789538,325.31225685)(527.62289543,325.30725686)(527.57289888,325.29726336)
\curveto(527.40289565,325.24725692)(527.26289579,325.17725699)(527.15289888,325.08726336)
\curveto(527.052896,325.00725716)(526.98289607,324.88225728)(526.94289888,324.71226336)
\curveto(526.92289613,324.64225752)(526.92289613,324.57725759)(526.94289888,324.51726336)
\curveto(526.96289609,324.45725771)(526.98289607,324.40725776)(527.00289888,324.36726336)
\curveto(527.07289598,324.24725792)(527.1528959,324.15225801)(527.24289888,324.08226336)
\curveto(527.34289571,324.01225815)(527.45789559,323.95225821)(527.58789888,323.90226336)
\curveto(527.77789527,323.82225834)(527.98289507,323.75225841)(528.20289888,323.69226336)
\lineto(528.89289888,323.54226336)
\curveto(529.13289392,323.50225866)(529.36289369,323.45225871)(529.58289888,323.39226336)
\curveto(529.81289324,323.34225882)(530.02789302,323.27725889)(530.22789888,323.19726336)
\curveto(530.31789273,323.15725901)(530.40289265,323.12225904)(530.48289888,323.09226336)
\curveto(530.57289248,323.07225909)(530.65789239,323.03725913)(530.73789888,322.98726336)
\curveto(530.92789212,322.8672593)(531.09789195,322.73725943)(531.24789888,322.59726336)
\curveto(531.40789164,322.45725971)(531.53289152,322.28225988)(531.62289888,322.07226336)
\curveto(531.6528914,322.00226016)(531.67789137,321.93226023)(531.69789888,321.86226336)
\curveto(531.71789133,321.79226037)(531.73789131,321.71726045)(531.75789888,321.63726336)
\curveto(531.76789128,321.57726059)(531.77289128,321.48226068)(531.77289888,321.35226336)
\curveto(531.78289127,321.23226093)(531.78289127,321.13726103)(531.77289888,321.06726336)
\lineto(531.77289888,320.99226336)
\curveto(531.7528913,320.93226123)(531.73789131,320.87226129)(531.72789888,320.81226336)
\curveto(531.72789132,320.7622614)(531.72289133,320.71226145)(531.71289888,320.66226336)
\curveto(531.64289141,320.3622618)(531.53289152,320.09726207)(531.38289888,319.86726336)
\curveto(531.22289183,319.62726254)(531.02789202,319.43226273)(530.79789888,319.28226336)
\curveto(530.56789248,319.13226303)(530.30789274,319.00226316)(530.01789888,318.89226336)
\curveto(529.90789314,318.84226332)(529.78789326,318.80726336)(529.65789888,318.78726336)
\curveto(529.53789351,318.7672634)(529.41789363,318.74226342)(529.29789888,318.71226336)
\curveto(529.20789384,318.69226347)(529.11289394,318.68226348)(529.01289888,318.68226336)
\curveto(528.92289413,318.67226349)(528.83289422,318.65726351)(528.74289888,318.63726336)
\lineto(528.47289888,318.63726336)
\curveto(528.41289464,318.61726355)(528.30789474,318.60726356)(528.15789888,318.60726336)
\curveto(528.01789503,318.60726356)(527.91789513,318.61726355)(527.85789888,318.63726336)
\curveto(527.82789522,318.63726353)(527.79289526,318.64226352)(527.75289888,318.65226336)
\lineto(527.64789888,318.65226336)
\curveto(527.52789552,318.67226349)(527.40789564,318.68726348)(527.28789888,318.69726336)
\curveto(527.16789588,318.70726346)(527.052896,318.72726344)(526.94289888,318.75726336)
\curveto(526.5528965,318.8672633)(526.20789684,318.99226317)(525.90789888,319.13226336)
\curveto(525.60789744,319.28226288)(525.3528977,319.50226266)(525.14289888,319.79226336)
\curveto(525.00289805,319.98226218)(524.88289817,320.20226196)(524.78289888,320.45226336)
\curveto(524.76289829,320.51226165)(524.74289831,320.59226157)(524.72289888,320.69226336)
\curveto(524.70289835,320.74226142)(524.68789836,320.81226135)(524.67789888,320.90226336)
\curveto(524.66789838,320.99226117)(524.67289838,321.0672611)(524.69289888,321.12726336)
\curveto(524.72289833,321.19726097)(524.77289828,321.24726092)(524.84289888,321.27726336)
\curveto(524.89289816,321.29726087)(524.9528981,321.30726086)(525.02289888,321.30726336)
\lineto(525.24789888,321.30726336)
\lineto(525.95289888,321.30726336)
\lineto(526.19289888,321.30726336)
\curveto(526.27289678,321.30726086)(526.34289671,321.29726087)(526.40289888,321.27726336)
\curveto(526.51289654,321.23726093)(526.58289647,321.17226099)(526.61289888,321.08226336)
\curveto(526.6528964,320.99226117)(526.69789635,320.89726127)(526.74789888,320.79726336)
\curveto(526.76789628,320.74726142)(526.80289625,320.68226148)(526.85289888,320.60226336)
\curveto(526.91289614,320.52226164)(526.96289609,320.47226169)(527.00289888,320.45226336)
\curveto(527.12289593,320.35226181)(527.23789581,320.27226189)(527.34789888,320.21226336)
\curveto(527.45789559,320.162262)(527.59789545,320.11226205)(527.76789888,320.06226336)
\curveto(527.81789523,320.04226212)(527.86789518,320.03226213)(527.91789888,320.03226336)
\curveto(527.96789508,320.04226212)(528.01789503,320.04226212)(528.06789888,320.03226336)
\curveto(528.1478949,320.01226215)(528.23289482,320.00226216)(528.32289888,320.00226336)
\curveto(528.42289463,320.01226215)(528.50789454,320.02726214)(528.57789888,320.04726336)
\curveto(528.62789442,320.05726211)(528.67289438,320.0622621)(528.71289888,320.06226336)
\curveto(528.76289429,320.0622621)(528.81289424,320.07226209)(528.86289888,320.09226336)
\curveto(529.00289405,320.14226202)(529.12789392,320.20226196)(529.23789888,320.27226336)
\curveto(529.35789369,320.34226182)(529.4528936,320.43226173)(529.52289888,320.54226336)
\curveto(529.57289348,320.62226154)(529.61289344,320.74726142)(529.64289888,320.91726336)
\curveto(529.66289339,320.98726118)(529.66289339,321.05226111)(529.64289888,321.11226336)
\curveto(529.62289343,321.17226099)(529.60289345,321.22226094)(529.58289888,321.26226336)
\curveto(529.51289354,321.40226076)(529.42289363,321.50726066)(529.31289888,321.57726336)
\curveto(529.21289384,321.64726052)(529.09289396,321.71226045)(528.95289888,321.77226336)
\curveto(528.76289429,321.85226031)(528.56289449,321.91726025)(528.35289888,321.96726336)
\curveto(528.14289491,322.01726015)(527.93289512,322.07226009)(527.72289888,322.13226336)
\curveto(527.64289541,322.15226001)(527.55789549,322.16726)(527.46789888,322.17726336)
\curveto(527.38789566,322.18725998)(527.30789574,322.20225996)(527.22789888,322.22226336)
\curveto(526.90789614,322.31225985)(526.60289645,322.39725977)(526.31289888,322.47726336)
\curveto(526.02289703,322.5672596)(525.75789729,322.69725947)(525.51789888,322.86726336)
\curveto(525.23789781,323.0672591)(525.03289802,323.33725883)(524.90289888,323.67726336)
\curveto(524.88289817,323.74725842)(524.86289819,323.84225832)(524.84289888,323.96226336)
\curveto(524.82289823,324.03225813)(524.80789824,324.11725805)(524.79789888,324.21726336)
\curveto(524.78789826,324.31725785)(524.79289826,324.40725776)(524.81289888,324.48726336)
\curveto(524.83289822,324.53725763)(524.83789821,324.57725759)(524.82789888,324.60726336)
\curveto(524.81789823,324.64725752)(524.82289823,324.69225747)(524.84289888,324.74226336)
\curveto(524.86289819,324.85225731)(524.88289817,324.95225721)(524.90289888,325.04226336)
\curveto(524.93289812,325.14225702)(524.96789808,325.23725693)(525.00789888,325.32726336)
\curveto(525.13789791,325.61725655)(525.31789773,325.85225631)(525.54789888,326.03226336)
\curveto(525.77789727,326.21225595)(526.03789701,326.35725581)(526.32789888,326.46726336)
\curveto(526.43789661,326.51725565)(526.5528965,326.55225561)(526.67289888,326.57226336)
\curveto(526.79289626,326.60225556)(526.91789613,326.63225553)(527.04789888,326.66226336)
\curveto(527.10789594,326.68225548)(527.16789588,326.69225547)(527.22789888,326.69226336)
\lineto(527.40789888,326.72226336)
\curveto(527.48789556,326.73225543)(527.57289548,326.73725543)(527.66289888,326.73726336)
\curveto(527.7528953,326.73725543)(527.83789521,326.74225542)(527.91789888,326.75226336)
}
}
{
\newrgbcolor{curcolor}{0 0 0}
\pscustom[linestyle=none,fillstyle=solid,fillcolor=curcolor]
{
\newpath
\moveto(540.77453951,322.98726336)
\curveto(540.79453094,322.92725924)(540.80453093,322.84225932)(540.80453951,322.73226336)
\curveto(540.80453093,322.62225954)(540.79453094,322.53725963)(540.77453951,322.47726336)
\lineto(540.77453951,322.32726336)
\curveto(540.75453098,322.24725992)(540.74453099,322.16726)(540.74453951,322.08726336)
\curveto(540.75453098,322.00726016)(540.74953098,321.92726024)(540.72953951,321.84726336)
\curveto(540.70953102,321.77726039)(540.69453104,321.71226045)(540.68453951,321.65226336)
\curveto(540.67453106,321.59226057)(540.66453107,321.52726064)(540.65453951,321.45726336)
\curveto(540.61453112,321.34726082)(540.57953115,321.23226093)(540.54953951,321.11226336)
\curveto(540.51953121,321.00226116)(540.47953125,320.89726127)(540.42953951,320.79726336)
\curveto(540.21953151,320.31726185)(539.94453179,319.92726224)(539.60453951,319.62726336)
\curveto(539.26453247,319.32726284)(538.85453288,319.07726309)(538.37453951,318.87726336)
\curveto(538.25453348,318.82726334)(538.1295336,318.79226337)(537.99953951,318.77226336)
\curveto(537.87953385,318.74226342)(537.75453398,318.71226345)(537.62453951,318.68226336)
\curveto(537.57453416,318.6622635)(537.51953421,318.65226351)(537.45953951,318.65226336)
\curveto(537.39953433,318.65226351)(537.34453439,318.64726352)(537.29453951,318.63726336)
\lineto(537.18953951,318.63726336)
\curveto(537.15953457,318.62726354)(537.1295346,318.62226354)(537.09953951,318.62226336)
\curveto(537.04953468,318.61226355)(536.96953476,318.60726356)(536.85953951,318.60726336)
\curveto(536.74953498,318.59726357)(536.66453507,318.60226356)(536.60453951,318.62226336)
\lineto(536.45453951,318.62226336)
\curveto(536.40453533,318.63226353)(536.34953538,318.63726353)(536.28953951,318.63726336)
\curveto(536.23953549,318.62726354)(536.18953554,318.63226353)(536.13953951,318.65226336)
\curveto(536.09953563,318.6622635)(536.05953567,318.6672635)(536.01953951,318.66726336)
\curveto(535.98953574,318.6672635)(535.94953578,318.67226349)(535.89953951,318.68226336)
\curveto(535.79953593,318.71226345)(535.69953603,318.73726343)(535.59953951,318.75726336)
\curveto(535.49953623,318.77726339)(535.40453633,318.80726336)(535.31453951,318.84726336)
\curveto(535.19453654,318.88726328)(535.07953665,318.92726324)(534.96953951,318.96726336)
\curveto(534.86953686,319.00726316)(534.76453697,319.05726311)(534.65453951,319.11726336)
\curveto(534.30453743,319.32726284)(534.00453773,319.57226259)(533.75453951,319.85226336)
\curveto(533.50453823,320.13226203)(533.29453844,320.4672617)(533.12453951,320.85726336)
\curveto(533.07453866,320.94726122)(533.0345387,321.04226112)(533.00453951,321.14226336)
\curveto(532.98453875,321.24226092)(532.95953877,321.34726082)(532.92953951,321.45726336)
\curveto(532.90953882,321.50726066)(532.89953883,321.55226061)(532.89953951,321.59226336)
\curveto(532.89953883,321.63226053)(532.88953884,321.67726049)(532.86953951,321.72726336)
\curveto(532.84953888,321.80726036)(532.83953889,321.88726028)(532.83953951,321.96726336)
\curveto(532.83953889,322.05726011)(532.8295389,322.14226002)(532.80953951,322.22226336)
\curveto(532.79953893,322.27225989)(532.79453894,322.31725985)(532.79453951,322.35726336)
\lineto(532.79453951,322.49226336)
\curveto(532.77453896,322.55225961)(532.76453897,322.63725953)(532.76453951,322.74726336)
\curveto(532.77453896,322.85725931)(532.78953894,322.94225922)(532.80953951,323.00226336)
\lineto(532.80953951,323.10726336)
\curveto(532.81953891,323.15725901)(532.81953891,323.20725896)(532.80953951,323.25726336)
\curveto(532.80953892,323.31725885)(532.81953891,323.37225879)(532.83953951,323.42226336)
\curveto(532.84953888,323.47225869)(532.85453888,323.51725865)(532.85453951,323.55726336)
\curveto(532.85453888,323.60725856)(532.86453887,323.65725851)(532.88453951,323.70726336)
\curveto(532.92453881,323.83725833)(532.95953877,323.9622582)(532.98953951,324.08226336)
\curveto(533.01953871,324.21225795)(533.05953867,324.33725783)(533.10953951,324.45726336)
\curveto(533.28953844,324.8672573)(533.50453823,325.20725696)(533.75453951,325.47726336)
\curveto(534.00453773,325.75725641)(534.30953742,326.01225615)(534.66953951,326.24226336)
\curveto(534.76953696,326.29225587)(534.87453686,326.33725583)(534.98453951,326.37726336)
\curveto(535.09453664,326.41725575)(535.20453653,326.4622557)(535.31453951,326.51226336)
\curveto(535.44453629,326.5622556)(535.57953615,326.59725557)(535.71953951,326.61726336)
\curveto(535.85953587,326.63725553)(536.00453573,326.6672555)(536.15453951,326.70726336)
\curveto(536.2345355,326.71725545)(536.30953542,326.72225544)(536.37953951,326.72226336)
\curveto(536.44953528,326.72225544)(536.51953521,326.72725544)(536.58953951,326.73726336)
\curveto(537.16953456,326.74725542)(537.66953406,326.68725548)(538.08953951,326.55726336)
\curveto(538.51953321,326.42725574)(538.89953283,326.24725592)(539.22953951,326.01726336)
\curveto(539.33953239,325.93725623)(539.44953228,325.84725632)(539.55953951,325.74726336)
\curveto(539.67953205,325.65725651)(539.77953195,325.55725661)(539.85953951,325.44726336)
\curveto(539.93953179,325.34725682)(540.00953172,325.24725692)(540.06953951,325.14726336)
\curveto(540.13953159,325.04725712)(540.20953152,324.94225722)(540.27953951,324.83226336)
\curveto(540.34953138,324.72225744)(540.40453133,324.60225756)(540.44453951,324.47226336)
\curveto(540.48453125,324.35225781)(540.5295312,324.22225794)(540.57953951,324.08226336)
\curveto(540.60953112,324.00225816)(540.6345311,323.91725825)(540.65453951,323.82726336)
\lineto(540.71453951,323.55726336)
\curveto(540.72453101,323.51725865)(540.729531,323.47725869)(540.72953951,323.43726336)
\curveto(540.729531,323.39725877)(540.734531,323.35725881)(540.74453951,323.31726336)
\curveto(540.76453097,323.2672589)(540.76953096,323.21225895)(540.75953951,323.15226336)
\curveto(540.74953098,323.09225907)(540.75453098,323.03725913)(540.77453951,322.98726336)
\moveto(538.67453951,322.44726336)
\curveto(538.68453305,322.49725967)(538.68953304,322.5672596)(538.68953951,322.65726336)
\curveto(538.68953304,322.75725941)(538.68453305,322.83225933)(538.67453951,322.88226336)
\lineto(538.67453951,323.00226336)
\curveto(538.65453308,323.05225911)(538.64453309,323.10725906)(538.64453951,323.16726336)
\curveto(538.64453309,323.22725894)(538.63953309,323.28225888)(538.62953951,323.33226336)
\curveto(538.6295331,323.37225879)(538.62453311,323.40225876)(538.61453951,323.42226336)
\lineto(538.55453951,323.66226336)
\curveto(538.54453319,323.75225841)(538.52453321,323.83725833)(538.49453951,323.91726336)
\curveto(538.38453335,324.17725799)(538.25453348,324.39725777)(538.10453951,324.57726336)
\curveto(537.95453378,324.7672574)(537.75453398,324.91725725)(537.50453951,325.02726336)
\curveto(537.44453429,325.04725712)(537.38453435,325.0622571)(537.32453951,325.07226336)
\curveto(537.26453447,325.09225707)(537.19953453,325.11225705)(537.12953951,325.13226336)
\curveto(537.04953468,325.15225701)(536.96453477,325.15725701)(536.87453951,325.14726336)
\lineto(536.60453951,325.14726336)
\curveto(536.57453516,325.12725704)(536.53953519,325.11725705)(536.49953951,325.11726336)
\curveto(536.45953527,325.12725704)(536.42453531,325.12725704)(536.39453951,325.11726336)
\lineto(536.18453951,325.05726336)
\curveto(536.12453561,325.04725712)(536.06953566,325.02725714)(536.01953951,324.99726336)
\curveto(535.76953596,324.88725728)(535.56453617,324.72725744)(535.40453951,324.51726336)
\curveto(535.25453648,324.31725785)(535.1345366,324.08225808)(535.04453951,323.81226336)
\curveto(535.01453672,323.71225845)(534.98953674,323.60725856)(534.96953951,323.49726336)
\curveto(534.95953677,323.38725878)(534.94453679,323.27725889)(534.92453951,323.16726336)
\curveto(534.91453682,323.11725905)(534.90953682,323.0672591)(534.90953951,323.01726336)
\lineto(534.90953951,322.86726336)
\curveto(534.88953684,322.79725937)(534.87953685,322.69225947)(534.87953951,322.55226336)
\curveto(534.88953684,322.41225975)(534.90453683,322.30725986)(534.92453951,322.23726336)
\lineto(534.92453951,322.10226336)
\curveto(534.94453679,322.02226014)(534.95953677,321.94226022)(534.96953951,321.86226336)
\curveto(534.97953675,321.79226037)(534.99453674,321.71726045)(535.01453951,321.63726336)
\curveto(535.11453662,321.33726083)(535.21953651,321.09226107)(535.32953951,320.90226336)
\curveto(535.44953628,320.72226144)(535.6345361,320.55726161)(535.88453951,320.40726336)
\curveto(535.95453578,320.35726181)(536.0295357,320.31726185)(536.10953951,320.28726336)
\curveto(536.19953553,320.25726191)(536.28953544,320.23226193)(536.37953951,320.21226336)
\curveto(536.41953531,320.20226196)(536.45453528,320.19726197)(536.48453951,320.19726336)
\curveto(536.51453522,320.20726196)(536.54953518,320.20726196)(536.58953951,320.19726336)
\lineto(536.70953951,320.16726336)
\curveto(536.75953497,320.167262)(536.80453493,320.17226199)(536.84453951,320.18226336)
\lineto(536.96453951,320.18226336)
\curveto(537.04453469,320.20226196)(537.12453461,320.21726195)(537.20453951,320.22726336)
\curveto(537.28453445,320.23726193)(537.35953437,320.25726191)(537.42953951,320.28726336)
\curveto(537.68953404,320.38726178)(537.89953383,320.52226164)(538.05953951,320.69226336)
\curveto(538.21953351,320.8622613)(538.35453338,321.07226109)(538.46453951,321.32226336)
\curveto(538.50453323,321.42226074)(538.5345332,321.52226064)(538.55453951,321.62226336)
\curveto(538.57453316,321.72226044)(538.59953313,321.82726034)(538.62953951,321.93726336)
\curveto(538.63953309,321.97726019)(538.64453309,322.01226015)(538.64453951,322.04226336)
\curveto(538.64453309,322.08226008)(538.64953308,322.12226004)(538.65953951,322.16226336)
\lineto(538.65953951,322.29726336)
\curveto(538.65953307,322.34725982)(538.66453307,322.39725977)(538.67453951,322.44726336)
}
}
{
\newrgbcolor{curcolor}{0.80000001 0.80000001 0.80000001}
\pscustom[linestyle=none,fillstyle=solid,fillcolor=curcolor]
{
\newpath
\moveto(118.65951904,414.52157101)
\curveto(73.49932006,455.49594901)(58.33004643,520.09999194)(80.53437162,576.891815)
\lineto(217.42099381,523.37222524)
\closepath
}
}
{
\newrgbcolor{curcolor}{0.90196079 0.90196079 0.90196079}
\pscustom[linestyle=none,fillstyle=solid,fillcolor=curcolor]
{
\newpath
\moveto(217.42099669,670.34941335)
\curveto(298.59425626,670.34941175)(364.39818351,604.54548192)(364.39818192,523.37222235)
\curveto(364.39818033,442.19896279)(298.59425049,376.39503553)(217.42099093,376.39503713)
\curveto(180.77407546,376.39503784)(145.4490839,390.08556265)(118.37343246,414.7818278)
\lineto(217.42099381,523.37222524)
\closepath
}
}
{
\newrgbcolor{curcolor}{0.7019608 0.7019608 0.7019608}
\pscustom[linestyle=none,fillstyle=solid,fillcolor=curcolor]
{
\newpath
\moveto(80.23284328,576.11409834)
\curveto(86.82093395,593.25053892)(96.56128519,609.00098904)(108.94981728,622.55033616)
\lineto(217.42099381,523.37222524)
\closepath
}
}
{
\newrgbcolor{curcolor}{0.60000002 0.60000002 0.60000002}
\pscustom[linestyle=none,fillstyle=solid,fillcolor=curcolor]
{
\newpath
\moveto(108.68177944,622.25638516)
\curveto(117.79795225,632.28109964)(128.25658038,640.99705557)(139.76050587,648.15660902)
\lineto(217.42099381,523.37222524)
\closepath
}
}
{
\newrgbcolor{curcolor}{0.50196081 0.50196081 0.50196081}
\pscustom[linestyle=none,fillstyle=solid,fillcolor=curcolor]
{
\newpath
\moveto(139.70655636,648.12301696)
\curveto(159.14991326,660.23540137)(181.15530701,667.63534516)(203.96663592,669.73230886)
\lineto(217.42099381,523.37222524)
\closepath
}
}
{
\newrgbcolor{curcolor}{0.40000001 0.40000001 0.40000001}
\pscustom[linestyle=none,fillstyle=solid,fillcolor=curcolor]
{
\newpath
\moveto(203.94254523,669.73009229)
\curveto(208.42351273,670.14275538)(212.92106775,670.34941343)(217.42099669,670.34941335)
\lineto(217.42099381,523.37222524)
\closepath
}
}
{
\newrgbcolor{curcolor}{0.80000001 0.80000001 0.80000001}
\pscustom[linestyle=none,fillstyle=solid,fillcolor=curcolor]
{
\newpath
\moveto(704.72646361,635.350286)
\curveto(712.53948867,628.56461731)(719.61402493,620.97278015)(725.83236598,612.70108777)
\lineto(608.35003181,524.38237524)
\closepath
}
}
{
\newrgbcolor{curcolor}{0.90196079 0.90196079 0.90196079}
\pscustom[linestyle=none,fillstyle=solid,fillcolor=curcolor]
{
\newpath
\moveto(608.35003469,671.35956335)
\curveto(643.8877702,671.35956265)(678.22199624,658.48357977)(704.99622946,635.11541595)
\lineto(608.35003181,524.38237524)
\closepath
}
}
{
\newrgbcolor{curcolor}{0.7019608 0.7019608 0.7019608}
\pscustom[linestyle=none,fillstyle=solid,fillcolor=curcolor]
{
\newpath
\moveto(725.83125315,612.70256804)
\curveto(774.60914874,547.81948111)(761.55331155,455.67904949)(696.67022462,406.9011539)
\curveto(669.07295791,386.15404309)(635.05267376,375.78454728)(600.57491475,377.61098368)
\lineto(608.35003181,524.38237524)
\closepath
}
}
{
\newrgbcolor{curcolor}{0.60000002 0.60000002 0.60000002}
\pscustom[linestyle=none,fillstyle=solid,fillcolor=curcolor]
{
\newpath
\moveto(600.67182853,377.60588182)
\curveto(594.40598237,377.9336621)(588.16755517,378.66233636)(581.99483875,379.78743195)
\lineto(608.35003181,524.38237524)
\closepath
}
}
{
\newrgbcolor{curcolor}{0.50196081 0.50196081 0.50196081}
\pscustom[linestyle=none,fillstyle=solid,fillcolor=curcolor]
{
\newpath
\moveto(582.09966254,379.76836506)
\curveto(502.23155011,394.26604369)(449.23834301,470.76463207)(463.73602164,550.6327445)
\curveto(475.99085466,618.14495849)(533.3278146,668.19489458)(601.8766819,671.21694053)
\lineto(608.35003181,524.38237524)
\closepath
}
}
{
\newrgbcolor{curcolor}{0.40000001 0.40000001 0.40000001}
\pscustom[linestyle=none,fillstyle=solid,fillcolor=curcolor]
{
\newpath
\moveto(601.74948156,671.21127757)
\curveto(603.94827809,671.31012233)(606.14901755,671.35956339)(608.35003469,671.35956335)
\lineto(608.35003181,524.38237524)
\closepath
}
}
{
\newrgbcolor{curcolor}{0.80000001 0.80000001 0.80000001}
\pscustom[linestyle=none,fillstyle=solid,fillcolor=curcolor]
{
\newpath
\moveto(563.57744453,23.76609671)
\curveto(486.26209806,48.49331388)(443.63099612,131.21515604)(468.35821329,208.53050252)
\curveto(476.07417851,232.65624705)(489.89413762,254.3784572)(508.4776025,271.59015387)
\lineto(608.35003181,163.75791524)
\closepath
}
}
{
\newrgbcolor{curcolor}{0.90196079 0.90196079 0.90196079}
\pscustom[linestyle=none,fillstyle=solid,fillcolor=curcolor]
{
\newpath
\moveto(608.35003469,310.73510335)
\curveto(689.52329426,310.73510175)(755.32722151,244.93117192)(755.32721992,163.75791235)
\curveto(755.32721833,82.58465279)(689.52328849,16.78072553)(608.35002893,16.78072713)
\curveto(593.14703499,16.78072742)(578.03607959,19.1394892)(563.55633528,23.77284969)
\lineto(608.35003181,163.75791524)
\closepath
}
}
{
\newrgbcolor{curcolor}{0.50196081 0.50196081 0.50196081}
\pscustom[linestyle=none,fillstyle=solid,fillcolor=curcolor]
{
\newpath
\moveto(508.18492676,271.3183427)
\curveto(535.38765694,296.65074783)(571.17854994,310.73510408)(608.35003469,310.73510335)
\lineto(608.35003181,163.75791524)
\closepath
}
}
{
\newrgbcolor{curcolor}{0.80000001 0.80000001 0.80000001}
\pscustom[linestyle=none,fillstyle=solid,fillcolor=curcolor]
{
\newpath
\moveto(287.43904147,292.98549614)
\curveto(358.80946367,254.31559952)(385.31847734,165.11029576)(346.64858072,93.73987357)
\curveto(312.26175498,30.27442668)(236.87849306,1.17250225)(168.76582939,25.06772735)
\lineto(217.42099981,163.75791524)
\closepath
}
}
{
\newrgbcolor{curcolor}{0.90196079 0.90196079 0.90196079}
\pscustom[linestyle=none,fillstyle=solid,fillcolor=curcolor]
{
\newpath
\moveto(217.42100269,310.73510335)
\curveto(242.05075825,310.73510286)(266.28560427,304.54558911)(287.89902103,292.73520916)
\lineto(217.42099981,163.75791524)
\closepath
}
}
{
\newrgbcolor{curcolor}{0.7019608 0.7019608 0.7019608}
\pscustom[linestyle=none,fillstyle=solid,fillcolor=curcolor]
{
\newpath
\moveto(168.96141599,24.99926664)
\curveto(159.99590911,28.13034882)(151.35496264,32.12297859)(143.15975622,36.92116376)
\lineto(217.42099981,163.75791524)
\closepath
}
}
{
\newrgbcolor{curcolor}{0.60000002 0.60000002 0.60000002}
\pscustom[linestyle=none,fillstyle=solid,fillcolor=curcolor]
{
\newpath
\moveto(143.31857592,36.82831034)
\curveto(73.21729083,77.75394898)(49.56575628,167.75905404)(90.49139492,237.86033913)
\curveto(110.59831307,272.30136066)(143.83768808,297.09306507)(182.58182663,306.54631055)
\lineto(217.42099981,163.75791524)
\closepath
}
}
{
\newrgbcolor{curcolor}{0.50196081 0.50196081 0.50196081}
\pscustom[linestyle=none,fillstyle=solid,fillcolor=curcolor]
{
\newpath
\moveto(182.53069634,306.53382549)
\curveto(193.95091974,309.32459782)(205.66473148,310.73510358)(217.42100269,310.73510335)
\lineto(217.42099981,163.75791524)
\closepath
}
}
{
\newrgbcolor{curcolor}{0 0 0}
\pscustom[linestyle=none,fillstyle=solid,fillcolor=curcolor]
{
\newpath
\moveto(100.48703462,590.54351702)
\lineto(104.08703462,590.54351702)
\lineto(104.73203462,590.54351702)
\curveto(104.81202809,590.5435066)(104.88702802,590.5385066)(104.95703462,590.52851702)
\curveto(105.02702788,590.52850661)(105.08702782,590.51850662)(105.13703462,590.49851702)
\curveto(105.2070277,590.46850667)(105.26202764,590.40850673)(105.30203462,590.31851702)
\curveto(105.32202758,590.28850685)(105.33202757,590.24850689)(105.33203462,590.19851702)
\lineto(105.33203462,590.06351702)
\curveto(105.34202756,589.95350719)(105.33702757,589.84850729)(105.31703462,589.74851702)
\curveto(105.3070276,589.64850749)(105.27202763,589.57850756)(105.21203462,589.53851702)
\curveto(105.12202778,589.46850767)(104.98702792,589.43350771)(104.80703462,589.43351702)
\curveto(104.62702828,589.4435077)(104.46202844,589.44850769)(104.31203462,589.44851702)
\lineto(102.31703462,589.44851702)
\lineto(101.82203462,589.44851702)
\lineto(101.68703462,589.44851702)
\curveto(101.64703126,589.44850769)(101.6070313,589.4435077)(101.56703462,589.43351702)
\lineto(101.35703462,589.43351702)
\curveto(101.24703166,589.40350774)(101.16703174,589.36350778)(101.11703462,589.31351702)
\curveto(101.06703184,589.27350787)(101.03203187,589.21850792)(101.01203462,589.14851702)
\curveto(100.99203191,589.08850805)(100.97703193,589.01850812)(100.96703462,588.93851702)
\curveto(100.95703195,588.85850828)(100.93703197,588.76850837)(100.90703462,588.66851702)
\curveto(100.85703205,588.46850867)(100.81703209,588.26350888)(100.78703462,588.05351702)
\curveto(100.75703215,587.8435093)(100.71703219,587.6385095)(100.66703462,587.43851702)
\curveto(100.64703226,587.36850977)(100.63703227,587.29850984)(100.63703462,587.22851702)
\curveto(100.63703227,587.16850997)(100.62703228,587.10351004)(100.60703462,587.03351702)
\curveto(100.59703231,587.00351014)(100.58703232,586.96351018)(100.57703462,586.91351702)
\curveto(100.57703233,586.87351027)(100.58203232,586.83351031)(100.59203462,586.79351702)
\curveto(100.61203229,586.7435104)(100.63703227,586.69851044)(100.66703462,586.65851702)
\curveto(100.7070322,586.62851051)(100.76703214,586.62351052)(100.84703462,586.64351702)
\curveto(100.907032,586.66351048)(100.96703194,586.68851045)(101.02703462,586.71851702)
\curveto(101.08703182,586.75851038)(101.14703176,586.79351035)(101.20703462,586.82351702)
\curveto(101.26703164,586.8435103)(101.31703159,586.85851028)(101.35703462,586.86851702)
\curveto(101.54703136,586.94851019)(101.75203115,587.00351014)(101.97203462,587.03351702)
\curveto(102.2020307,587.06351008)(102.43203047,587.07351007)(102.66203462,587.06351702)
\curveto(102.90203,587.06351008)(103.13202977,587.0385101)(103.35203462,586.98851702)
\curveto(103.57202933,586.94851019)(103.77202913,586.88851025)(103.95203462,586.80851702)
\curveto(104.0020289,586.78851035)(104.04702886,586.76851037)(104.08703462,586.74851702)
\curveto(104.13702877,586.72851041)(104.18702872,586.70351044)(104.23703462,586.67351702)
\curveto(104.58702832,586.46351068)(104.86702804,586.23351091)(105.07703462,585.98351702)
\curveto(105.29702761,585.73351141)(105.49202741,585.40851173)(105.66203462,585.00851702)
\curveto(105.71202719,584.89851224)(105.74702716,584.78851235)(105.76703462,584.67851702)
\curveto(105.78702712,584.56851257)(105.81202709,584.45351269)(105.84203462,584.33351702)
\curveto(105.85202705,584.30351284)(105.85702705,584.25851288)(105.85703462,584.19851702)
\curveto(105.87702703,584.138513)(105.88702702,584.06851307)(105.88703462,583.98851702)
\curveto(105.88702702,583.91851322)(105.89702701,583.85351329)(105.91703462,583.79351702)
\lineto(105.91703462,583.62851702)
\curveto(105.92702698,583.57851356)(105.93202697,583.50851363)(105.93203462,583.41851702)
\curveto(105.93202697,583.32851381)(105.92202698,583.25851388)(105.90203462,583.20851702)
\curveto(105.88202702,583.14851399)(105.87702703,583.08851405)(105.88703462,583.02851702)
\curveto(105.89702701,582.97851416)(105.89202701,582.92851421)(105.87203462,582.87851702)
\curveto(105.83202707,582.71851442)(105.79702711,582.56851457)(105.76703462,582.42851702)
\curveto(105.73702717,582.28851485)(105.69202721,582.15351499)(105.63203462,582.02351702)
\curveto(105.47202743,581.65351549)(105.25202765,581.31851582)(104.97203462,581.01851702)
\curveto(104.69202821,580.71851642)(104.37202853,580.48851665)(104.01203462,580.32851702)
\curveto(103.84202906,580.24851689)(103.64202926,580.17351697)(103.41203462,580.10351702)
\curveto(103.3020296,580.06351708)(103.18702972,580.0385171)(103.06703462,580.02851702)
\curveto(102.94702996,580.01851712)(102.82703008,579.99851714)(102.70703462,579.96851702)
\curveto(102.65703025,579.94851719)(102.6020303,579.94851719)(102.54203462,579.96851702)
\curveto(102.48203042,579.97851716)(102.42203048,579.97351717)(102.36203462,579.95351702)
\curveto(102.26203064,579.93351721)(102.16203074,579.93351721)(102.06203462,579.95351702)
\lineto(101.92703462,579.95351702)
\curveto(101.87703103,579.97351717)(101.81703109,579.98351716)(101.74703462,579.98351702)
\curveto(101.68703122,579.97351717)(101.63203127,579.97851716)(101.58203462,579.99851702)
\curveto(101.54203136,580.00851713)(101.5070314,580.01351713)(101.47703462,580.01351702)
\curveto(101.44703146,580.01351713)(101.41203149,580.01851712)(101.37203462,580.02851702)
\lineto(101.10203462,580.08851702)
\curveto(101.01203189,580.10851703)(100.92703198,580.138517)(100.84703462,580.17851702)
\curveto(100.5070324,580.31851682)(100.21703269,580.47351667)(99.97703462,580.64351702)
\curveto(99.73703317,580.82351632)(99.51703339,581.05351609)(99.31703462,581.33351702)
\curveto(99.16703374,581.56351558)(99.05203385,581.80351534)(98.97203462,582.05351702)
\curveto(98.95203395,582.10351504)(98.94203396,582.14851499)(98.94203462,582.18851702)
\curveto(98.94203396,582.2385149)(98.93203397,582.28851485)(98.91203462,582.33851702)
\curveto(98.89203401,582.39851474)(98.87703403,582.47851466)(98.86703462,582.57851702)
\curveto(98.86703404,582.67851446)(98.88703402,582.75351439)(98.92703462,582.80351702)
\curveto(98.97703393,582.88351426)(99.05703385,582.92851421)(99.16703462,582.93851702)
\curveto(99.27703363,582.94851419)(99.39203351,582.95351419)(99.51203462,582.95351702)
\lineto(99.67703462,582.95351702)
\curveto(99.73703317,582.95351419)(99.79203311,582.9435142)(99.84203462,582.92351702)
\curveto(99.93203297,582.90351424)(100.0020329,582.86351428)(100.05203462,582.80351702)
\curveto(100.12203278,582.71351443)(100.16703274,582.60351454)(100.18703462,582.47351702)
\curveto(100.21703269,582.35351479)(100.26203264,582.24851489)(100.32203462,582.15851702)
\curveto(100.51203239,581.81851532)(100.77203213,581.54851559)(101.10203462,581.34851702)
\curveto(101.2020317,581.28851585)(101.3070316,581.2385159)(101.41703462,581.19851702)
\curveto(101.53703137,581.16851597)(101.65703125,581.13351601)(101.77703462,581.09351702)
\curveto(101.94703096,581.0435161)(102.15203075,581.02351612)(102.39203462,581.03351702)
\curveto(102.64203026,581.05351609)(102.84203006,581.08851605)(102.99203462,581.13851702)
\curveto(103.36202954,581.25851588)(103.65202925,581.41851572)(103.86203462,581.61851702)
\curveto(104.08202882,581.82851531)(104.26202864,582.10851503)(104.40203462,582.45851702)
\curveto(104.45202845,582.55851458)(104.48202842,582.66351448)(104.49203462,582.77351702)
\curveto(104.51202839,582.88351426)(104.53702837,582.99851414)(104.56703462,583.11851702)
\lineto(104.56703462,583.22351702)
\curveto(104.57702833,583.26351388)(104.58202832,583.30351384)(104.58203462,583.34351702)
\curveto(104.59202831,583.37351377)(104.59202831,583.40851373)(104.58203462,583.44851702)
\lineto(104.58203462,583.56851702)
\curveto(104.58202832,583.82851331)(104.55202835,584.07351307)(104.49203462,584.30351702)
\curveto(104.38202852,584.65351249)(104.22702868,584.94851219)(104.02703462,585.18851702)
\curveto(103.82702908,585.4385117)(103.56702934,585.63351151)(103.24703462,585.77351702)
\lineto(103.06703462,585.83351702)
\curveto(103.01702989,585.85351129)(102.95702995,585.87351127)(102.88703462,585.89351702)
\curveto(102.83703007,585.91351123)(102.77703013,585.92351122)(102.70703462,585.92351702)
\curveto(102.64703026,585.93351121)(102.58203032,585.94851119)(102.51203462,585.96851702)
\lineto(102.36203462,585.96851702)
\curveto(102.32203058,585.98851115)(102.26703064,585.99851114)(102.19703462,585.99851702)
\curveto(102.13703077,585.99851114)(102.08203082,585.98851115)(102.03203462,585.96851702)
\lineto(101.92703462,585.96851702)
\curveto(101.89703101,585.96851117)(101.86203104,585.96351118)(101.82203462,585.95351702)
\lineto(101.58203462,585.89351702)
\curveto(101.5020314,585.88351126)(101.42203148,585.86351128)(101.34203462,585.83351702)
\curveto(101.1020318,585.73351141)(100.87203203,585.59851154)(100.65203462,585.42851702)
\curveto(100.56203234,585.35851178)(100.47703243,585.28351186)(100.39703462,585.20351702)
\curveto(100.31703259,585.13351201)(100.21703269,585.07851206)(100.09703462,585.03851702)
\curveto(100.0070329,585.00851213)(99.86703304,584.99851214)(99.67703462,585.00851702)
\curveto(99.49703341,585.01851212)(99.37703353,585.0435121)(99.31703462,585.08351702)
\curveto(99.26703364,585.12351202)(99.22703368,585.18351196)(99.19703462,585.26351702)
\curveto(99.17703373,585.3435118)(99.17703373,585.42851171)(99.19703462,585.51851702)
\curveto(99.22703368,585.6385115)(99.24703366,585.75851138)(99.25703462,585.87851702)
\curveto(99.27703363,586.00851113)(99.3020336,586.13351101)(99.33203462,586.25351702)
\curveto(99.35203355,586.29351085)(99.35703355,586.32851081)(99.34703462,586.35851702)
\curveto(99.34703356,586.39851074)(99.35703355,586.4435107)(99.37703462,586.49351702)
\curveto(99.39703351,586.58351056)(99.41203349,586.67351047)(99.42203462,586.76351702)
\curveto(99.43203347,586.86351028)(99.45203345,586.95851018)(99.48203462,587.04851702)
\curveto(99.49203341,587.10851003)(99.49703341,587.16850997)(99.49703462,587.22851702)
\curveto(99.5070334,587.28850985)(99.52203338,587.34850979)(99.54203462,587.40851702)
\curveto(99.59203331,587.60850953)(99.62703328,587.81350933)(99.64703462,588.02351702)
\curveto(99.67703323,588.2435089)(99.71703319,588.45350869)(99.76703462,588.65351702)
\curveto(99.79703311,588.75350839)(99.81703309,588.85350829)(99.82703462,588.95351702)
\curveto(99.83703307,589.05350809)(99.85203305,589.15350799)(99.87203462,589.25351702)
\curveto(99.88203302,589.28350786)(99.88703302,589.32350782)(99.88703462,589.37351702)
\curveto(99.91703299,589.48350766)(99.93703297,589.58850755)(99.94703462,589.68851702)
\curveto(99.96703294,589.79850734)(99.99203291,589.90850723)(100.02203462,590.01851702)
\curveto(100.04203286,590.09850704)(100.05703285,590.16850697)(100.06703462,590.22851702)
\curveto(100.07703283,590.29850684)(100.1020328,590.35850678)(100.14203462,590.40851702)
\curveto(100.16203274,590.4385067)(100.19203271,590.45850668)(100.23203462,590.46851702)
\curveto(100.27203263,590.48850665)(100.31703259,590.50850663)(100.36703462,590.52851702)
\curveto(100.42703248,590.52850661)(100.46703244,590.53350661)(100.48703462,590.54351702)
}
}
{
\newrgbcolor{curcolor}{0 0 0}
\pscustom[linestyle=none,fillstyle=solid,fillcolor=curcolor]
{
\newpath
\moveto(108.281644,581.76851702)
\lineto(108.581644,581.76851702)
\curveto(108.69164194,581.77851536)(108.79664183,581.77851536)(108.896644,581.76851702)
\curveto(109.00664162,581.76851537)(109.10664152,581.75851538)(109.196644,581.73851702)
\curveto(109.28664134,581.72851541)(109.35664127,581.70351544)(109.406644,581.66351702)
\curveto(109.4266412,581.6435155)(109.44164119,581.61351553)(109.451644,581.57351702)
\curveto(109.47164116,581.53351561)(109.49164114,581.48851565)(109.511644,581.43851702)
\lineto(109.511644,581.36351702)
\curveto(109.52164111,581.31351583)(109.52164111,581.25851588)(109.511644,581.19851702)
\lineto(109.511644,581.04851702)
\lineto(109.511644,580.56851702)
\curveto(109.51164112,580.39851674)(109.47164116,580.27851686)(109.391644,580.20851702)
\curveto(109.32164131,580.15851698)(109.2316414,580.13351701)(109.121644,580.13351702)
\lineto(108.791644,580.13351702)
\lineto(108.341644,580.13351702)
\curveto(108.19164244,580.13351701)(108.07664255,580.16351698)(107.996644,580.22351702)
\curveto(107.95664267,580.25351689)(107.9266427,580.30351684)(107.906644,580.37351702)
\curveto(107.88664274,580.45351669)(107.87164276,580.5385166)(107.861644,580.62851702)
\lineto(107.861644,580.91351702)
\curveto(107.87164276,581.01351613)(107.87664275,581.09851604)(107.876644,581.16851702)
\lineto(107.876644,581.36351702)
\curveto(107.87664275,581.42351572)(107.88664274,581.47851566)(107.906644,581.52851702)
\curveto(107.94664268,581.6385155)(108.01664261,581.70851543)(108.116644,581.73851702)
\curveto(108.14664248,581.7385154)(108.20164243,581.74851539)(108.281644,581.76851702)
}
}
{
\newrgbcolor{curcolor}{0 0 0}
\pscustom[linestyle=none,fillstyle=solid,fillcolor=curcolor]
{
\newpath
\moveto(118.42680025,585.72851702)
\curveto(118.42679261,585.64851149)(118.43179261,585.56851157)(118.44180025,585.48851702)
\curveto(118.45179259,585.40851173)(118.44679259,585.33351181)(118.42680025,585.26351702)
\curveto(118.40679263,585.22351192)(118.40179264,585.17851196)(118.41180025,585.12851702)
\curveto(118.42179262,585.08851205)(118.42179262,585.04851209)(118.41180025,585.00851702)
\lineto(118.41180025,584.85851702)
\curveto(118.40179264,584.76851237)(118.39679264,584.67851246)(118.39680025,584.58851702)
\curveto(118.39679264,584.50851263)(118.39179265,584.42851271)(118.38180025,584.34851702)
\lineto(118.35180025,584.10851702)
\curveto(118.3417927,584.0385131)(118.33179271,583.96351318)(118.32180025,583.88351702)
\curveto(118.31179273,583.8435133)(118.30679273,583.80351334)(118.30680025,583.76351702)
\curveto(118.30679273,583.72351342)(118.30179274,583.67851346)(118.29180025,583.62851702)
\curveto(118.25179279,583.48851365)(118.22179282,583.34851379)(118.20180025,583.20851702)
\curveto(118.19179285,583.06851407)(118.16179288,582.93351421)(118.11180025,582.80351702)
\curveto(118.06179298,582.63351451)(118.00679303,582.46851467)(117.94680025,582.30851702)
\curveto(117.89679314,582.14851499)(117.8367932,581.99351515)(117.76680025,581.84351702)
\curveto(117.74679329,581.78351536)(117.71679332,581.72351542)(117.67680025,581.66351702)
\lineto(117.58680025,581.51351702)
\curveto(117.38679365,581.19351595)(117.17179387,580.92851621)(116.94180025,580.71851702)
\curveto(116.71179433,580.50851663)(116.41679462,580.32851681)(116.05680025,580.17851702)
\curveto(115.9367951,580.12851701)(115.80679523,580.09351705)(115.66680025,580.07351702)
\curveto(115.5367955,580.05351709)(115.40179564,580.02851711)(115.26180025,579.99851702)
\curveto(115.20179584,579.98851715)(115.1417959,579.98351716)(115.08180025,579.98351702)
\curveto(115.02179602,579.98351716)(114.95679608,579.97851716)(114.88680025,579.96851702)
\curveto(114.85679618,579.95851718)(114.80679623,579.95851718)(114.73680025,579.96851702)
\lineto(114.58680025,579.96851702)
\lineto(114.43680025,579.96851702)
\curveto(114.35679668,579.98851715)(114.27179677,580.00351714)(114.18180025,580.01351702)
\curveto(114.10179694,580.01351713)(114.02679701,580.02351712)(113.95680025,580.04351702)
\curveto(113.91679712,580.05351709)(113.88179716,580.05851708)(113.85180025,580.05851702)
\curveto(113.83179721,580.04851709)(113.80679723,580.05351709)(113.77680025,580.07351702)
\lineto(113.50680025,580.13351702)
\curveto(113.41679762,580.16351698)(113.33179771,580.19351695)(113.25180025,580.22351702)
\curveto(112.67179837,580.46351668)(112.2367988,580.83351631)(111.94680025,581.33351702)
\curveto(111.86679917,581.46351568)(111.80179924,581.59851554)(111.75180025,581.73851702)
\curveto(111.71179933,581.87851526)(111.66679937,582.02851511)(111.61680025,582.18851702)
\curveto(111.59679944,582.26851487)(111.59179945,582.34851479)(111.60180025,582.42851702)
\curveto(111.62179942,582.50851463)(111.65679938,582.56351458)(111.70680025,582.59351702)
\curveto(111.7367993,582.61351453)(111.79179925,582.62851451)(111.87180025,582.63851702)
\curveto(111.95179909,582.65851448)(112.036799,582.66851447)(112.12680025,582.66851702)
\curveto(112.21679882,582.67851446)(112.30179874,582.67851446)(112.38180025,582.66851702)
\curveto(112.47179857,582.65851448)(112.5417985,582.64851449)(112.59180025,582.63851702)
\curveto(112.61179843,582.62851451)(112.6367984,582.61351453)(112.66680025,582.59351702)
\curveto(112.70679833,582.57351457)(112.7367983,582.55351459)(112.75680025,582.53351702)
\curveto(112.81679822,582.45351469)(112.86179818,582.35851478)(112.89180025,582.24851702)
\curveto(112.93179811,582.138515)(112.97679806,582.0385151)(113.02680025,581.94851702)
\curveto(113.27679776,581.55851558)(113.64679739,581.28851585)(114.13680025,581.13851702)
\curveto(114.20679683,581.11851602)(114.27679676,581.10351604)(114.34680025,581.09351702)
\curveto(114.42679661,581.09351605)(114.50679653,581.08351606)(114.58680025,581.06351702)
\curveto(114.62679641,581.05351609)(114.68179636,581.04851609)(114.75180025,581.04851702)
\curveto(114.83179621,581.04851609)(114.88679615,581.05351609)(114.91680025,581.06351702)
\curveto(114.94679609,581.07351607)(114.97679606,581.07851606)(115.00680025,581.07851702)
\lineto(115.11180025,581.07851702)
\curveto(115.19179585,581.09851604)(115.26679577,581.11851602)(115.33680025,581.13851702)
\curveto(115.41679562,581.15851598)(115.49179555,581.18351596)(115.56180025,581.21351702)
\curveto(115.91179513,581.36351578)(116.18179486,581.57851556)(116.37180025,581.85851702)
\curveto(116.56179448,582.138515)(116.71679432,582.46351468)(116.83680025,582.83351702)
\curveto(116.86679417,582.91351423)(116.88679415,582.98851415)(116.89680025,583.05851702)
\curveto(116.91679412,583.12851401)(116.9367941,583.20351394)(116.95680025,583.28351702)
\curveto(116.97679406,583.37351377)(116.99179405,583.46851367)(117.00180025,583.56851702)
\curveto(117.02179402,583.67851346)(117.041794,583.78351336)(117.06180025,583.88351702)
\curveto(117.07179397,583.93351321)(117.07679396,583.98351316)(117.07680025,584.03351702)
\curveto(117.08679395,584.09351305)(117.09179395,584.14851299)(117.09180025,584.19851702)
\curveto(117.11179393,584.25851288)(117.12179392,584.33351281)(117.12180025,584.42351702)
\curveto(117.12179392,584.52351262)(117.11179393,584.60351254)(117.09180025,584.66351702)
\curveto(117.06179398,584.75351239)(117.01179403,584.79351235)(116.94180025,584.78351702)
\curveto(116.88179416,584.77351237)(116.82679421,584.7435124)(116.77680025,584.69351702)
\curveto(116.69679434,584.6435125)(116.62679441,584.58351256)(116.56680025,584.51351702)
\curveto(116.51679452,584.4435127)(116.45179459,584.38351276)(116.37180025,584.33351702)
\curveto(116.21179483,584.22351292)(116.04679499,584.12351302)(115.87680025,584.03351702)
\curveto(115.70679533,583.95351319)(115.51179553,583.88351326)(115.29180025,583.82351702)
\curveto(115.19179585,583.79351335)(115.09179595,583.77851336)(114.99180025,583.77851702)
\curveto(114.90179614,583.77851336)(114.80179624,583.76851337)(114.69180025,583.74851702)
\lineto(114.54180025,583.74851702)
\curveto(114.49179655,583.76851337)(114.4417966,583.77351337)(114.39180025,583.76351702)
\curveto(114.35179669,583.75351339)(114.31179673,583.75351339)(114.27180025,583.76351702)
\curveto(114.2417968,583.77351337)(114.19679684,583.77851336)(114.13680025,583.77851702)
\curveto(114.07679696,583.78851335)(114.01179703,583.79851334)(113.94180025,583.80851702)
\lineto(113.76180025,583.83851702)
\curveto(113.31179773,583.95851318)(112.93179811,584.12351302)(112.62180025,584.33351702)
\curveto(112.35179869,584.52351262)(112.12179892,584.75351239)(111.93180025,585.02351702)
\curveto(111.75179929,585.30351184)(111.60679943,585.61851152)(111.49680025,585.96851702)
\lineto(111.43680025,586.17851702)
\curveto(111.42679961,586.25851088)(111.41179963,586.3385108)(111.39180025,586.41851702)
\curveto(111.38179966,586.44851069)(111.37679966,586.47851066)(111.37680025,586.50851702)
\curveto(111.37679966,586.5385106)(111.37179967,586.56851057)(111.36180025,586.59851702)
\curveto(111.35179969,586.65851048)(111.34679969,586.71851042)(111.34680025,586.77851702)
\curveto(111.34679969,586.84851029)(111.3367997,586.90851023)(111.31680025,586.95851702)
\lineto(111.31680025,587.13851702)
\curveto(111.30679973,587.18850995)(111.30179974,587.25850988)(111.30180025,587.34851702)
\curveto(111.30179974,587.4385097)(111.31179973,587.50850963)(111.33180025,587.55851702)
\lineto(111.33180025,587.72351702)
\curveto(111.35179969,587.80350934)(111.36179968,587.87850926)(111.36180025,587.94851702)
\curveto(111.37179967,588.01850912)(111.38679965,588.08850905)(111.40680025,588.15851702)
\curveto(111.46679957,588.35850878)(111.52679951,588.54850859)(111.58680025,588.72851702)
\curveto(111.65679938,588.90850823)(111.74679929,589.07850806)(111.85680025,589.23851702)
\curveto(111.89679914,589.30850783)(111.9367991,589.37350777)(111.97680025,589.43351702)
\lineto(112.12680025,589.61351702)
\curveto(112.14679889,589.62350752)(112.16679887,589.6385075)(112.18680025,589.65851702)
\curveto(112.27679876,589.78850735)(112.38679865,589.89850724)(112.51680025,589.98851702)
\curveto(112.77679826,590.18850695)(113.041798,590.3435068)(113.31180025,590.45351702)
\curveto(113.39179765,590.49350665)(113.47179757,590.52350662)(113.55180025,590.54351702)
\curveto(113.6417974,590.57350657)(113.73179731,590.59850654)(113.82180025,590.61851702)
\curveto(113.92179712,590.64850649)(114.02179702,590.66850647)(114.12180025,590.67851702)
\curveto(114.22179682,590.68850645)(114.32679671,590.70350644)(114.43680025,590.72351702)
\curveto(114.46679657,590.73350641)(114.50679653,590.73350641)(114.55680025,590.72351702)
\curveto(114.61679642,590.71350643)(114.65679638,590.71850642)(114.67680025,590.73851702)
\curveto(115.39679564,590.75850638)(115.99679504,590.6435065)(116.47680025,590.39351702)
\curveto(116.95679408,590.143507)(117.33179371,589.80350734)(117.60180025,589.37351702)
\curveto(117.69179335,589.23350791)(117.77179327,589.08850805)(117.84180025,588.93851702)
\curveto(117.91179313,588.78850835)(117.98179306,588.62850851)(118.05180025,588.45851702)
\curveto(118.10179294,588.31850882)(118.1417929,588.16850897)(118.17180025,588.00851702)
\curveto(118.20179284,587.84850929)(118.2367928,587.68850945)(118.27680025,587.52851702)
\curveto(118.29679274,587.47850966)(118.30679273,587.42350972)(118.30680025,587.36351702)
\curveto(118.30679273,587.31350983)(118.31179273,587.26350988)(118.32180025,587.21351702)
\curveto(118.3417927,587.15350999)(118.35179269,587.08851005)(118.35180025,587.01851702)
\curveto(118.35179269,586.95851018)(118.36179268,586.90351024)(118.38180025,586.85351702)
\lineto(118.38180025,586.68851702)
\curveto(118.40179264,586.6385105)(118.40679263,586.58851055)(118.39680025,586.53851702)
\curveto(118.38679265,586.48851065)(118.39179265,586.4385107)(118.41180025,586.38851702)
\curveto(118.41179263,586.36851077)(118.40679263,586.3435108)(118.39680025,586.31351702)
\curveto(118.39679264,586.28351086)(118.40179264,586.25851088)(118.41180025,586.23851702)
\curveto(118.42179262,586.20851093)(118.42179262,586.17351097)(118.41180025,586.13351702)
\curveto(118.41179263,586.09351105)(118.41679262,586.05351109)(118.42680025,586.01351702)
\curveto(118.4367926,585.97351117)(118.4367926,585.92851121)(118.42680025,585.87851702)
\lineto(118.42680025,585.72851702)
\moveto(116.92680025,587.03351702)
\curveto(116.9367941,587.08351006)(116.9417941,587.14351)(116.94180025,587.21351702)
\curveto(116.9417941,587.28350986)(116.9367941,587.3435098)(116.92680025,587.39351702)
\curveto(116.91679412,587.4435097)(116.91179413,587.51850962)(116.91180025,587.61851702)
\curveto(116.89179415,587.69850944)(116.87179417,587.77350937)(116.85180025,587.84351702)
\curveto(116.8417942,587.91350923)(116.82679421,587.98350916)(116.80680025,588.05351702)
\curveto(116.66679437,588.48350866)(116.47179457,588.81850832)(116.22180025,589.05851702)
\curveto(115.98179506,589.29850784)(115.6367954,589.47850766)(115.18680025,589.59851702)
\curveto(115.09679594,589.61850752)(114.99679604,589.62850751)(114.88680025,589.62851702)
\lineto(114.55680025,589.62851702)
\curveto(114.5367965,589.60850753)(114.50179654,589.59850754)(114.45180025,589.59851702)
\curveto(114.40179664,589.60850753)(114.35679668,589.60850753)(114.31680025,589.59851702)
\curveto(114.2367968,589.57850756)(114.16179688,589.55850758)(114.09180025,589.53851702)
\lineto(113.88180025,589.47851702)
\curveto(113.59179745,589.34850779)(113.36179768,589.16850797)(113.19180025,588.93851702)
\curveto(113.02179802,588.71850842)(112.88679815,588.45850868)(112.78680025,588.15851702)
\curveto(112.75679828,588.06850907)(112.73179831,587.97350917)(112.71180025,587.87351702)
\curveto(112.70179834,587.78350936)(112.68679835,587.68850945)(112.66680025,587.58851702)
\lineto(112.66680025,587.45351702)
\curveto(112.6367984,587.3435098)(112.62679841,587.20350994)(112.63680025,587.03351702)
\curveto(112.65679838,586.87351027)(112.67679836,586.7435104)(112.69680025,586.64351702)
\curveto(112.71679832,586.58351056)(112.73179831,586.52351062)(112.74180025,586.46351702)
\curveto(112.75179829,586.41351073)(112.76679827,586.36351078)(112.78680025,586.31351702)
\curveto(112.86679817,586.11351103)(112.96179808,585.92351122)(113.07180025,585.74351702)
\curveto(113.19179785,585.56351158)(113.33179771,585.41851172)(113.49180025,585.30851702)
\curveto(113.5417975,585.25851188)(113.59679744,585.21851192)(113.65680025,585.18851702)
\curveto(113.71679732,585.15851198)(113.77679726,585.12351202)(113.83680025,585.08351702)
\curveto(113.98679705,585.00351214)(114.17179687,584.9385122)(114.39180025,584.88851702)
\curveto(114.4417966,584.86851227)(114.48179656,584.86351228)(114.51180025,584.87351702)
\curveto(114.55179649,584.88351226)(114.59679644,584.87851226)(114.64680025,584.85851702)
\curveto(114.68679635,584.84851229)(114.7417963,584.8435123)(114.81180025,584.84351702)
\curveto(114.88179616,584.8435123)(114.9417961,584.84851229)(114.99180025,584.85851702)
\curveto(115.09179595,584.87851226)(115.18679585,584.89351225)(115.27680025,584.90351702)
\curveto(115.36679567,584.92351222)(115.45679558,584.95351219)(115.54680025,584.99351702)
\curveto(116.08679495,585.21351193)(116.48179456,585.60851153)(116.73180025,586.17851702)
\curveto(116.78179426,586.27851086)(116.81679422,586.37851076)(116.83680025,586.47851702)
\curveto(116.85679418,586.58851055)(116.88179416,586.69851044)(116.91180025,586.80851702)
\curveto(116.91179413,586.90851023)(116.91679412,586.98351016)(116.92680025,587.03351702)
}
}
{
\newrgbcolor{curcolor}{0 0 0}
\pscustom[linestyle=none,fillstyle=solid,fillcolor=curcolor]
{
\newpath
\moveto(129.64140962,588.65351702)
\curveto(129.44139932,588.36350878)(129.23139953,588.07850906)(129.01140962,587.79851702)
\curveto(128.80139996,587.51850962)(128.59640017,587.23350991)(128.39640962,586.94351702)
\curveto(127.79640097,586.09351105)(127.19140157,585.25351189)(126.58140962,584.42351702)
\curveto(125.97140279,583.60351354)(125.3664034,582.76851437)(124.76640962,581.91851702)
\lineto(124.25640962,581.19851702)
\lineto(123.74640962,580.50851702)
\curveto(123.6664051,580.39851674)(123.58640518,580.28351686)(123.50640962,580.16351702)
\curveto(123.42640534,580.0435171)(123.33140543,579.94851719)(123.22140962,579.87851702)
\curveto(123.18140558,579.85851728)(123.11640565,579.8435173)(123.02640962,579.83351702)
\curveto(122.94640582,579.81351733)(122.85640591,579.80351734)(122.75640962,579.80351702)
\curveto(122.65640611,579.80351734)(122.5614062,579.80851733)(122.47140962,579.81851702)
\curveto(122.39140637,579.82851731)(122.33140643,579.84851729)(122.29140962,579.87851702)
\curveto(122.2614065,579.89851724)(122.23640653,579.93351721)(122.21640962,579.98351702)
\curveto(122.20640656,580.02351712)(122.21140655,580.06851707)(122.23140962,580.11851702)
\curveto(122.27140649,580.19851694)(122.31640645,580.27351687)(122.36640962,580.34351702)
\curveto(122.42640634,580.42351672)(122.48140628,580.50351664)(122.53140962,580.58351702)
\curveto(122.77140599,580.92351622)(123.01640575,581.25851588)(123.26640962,581.58851702)
\curveto(123.51640525,581.91851522)(123.75640501,582.25351489)(123.98640962,582.59351702)
\curveto(124.14640462,582.81351433)(124.30640446,583.02851411)(124.46640962,583.23851702)
\curveto(124.62640414,583.44851369)(124.78640398,583.66351348)(124.94640962,583.88351702)
\curveto(125.30640346,584.40351274)(125.67140309,584.91351223)(126.04140962,585.41351702)
\curveto(126.41140235,585.91351123)(126.78140198,586.42351072)(127.15140962,586.94351702)
\curveto(127.29140147,587.14351)(127.43140133,587.3385098)(127.57140962,587.52851702)
\curveto(127.72140104,587.71850942)(127.8664009,587.91350923)(128.00640962,588.11351702)
\curveto(128.21640055,588.41350873)(128.43140033,588.71350843)(128.65140962,589.01351702)
\lineto(129.31140962,589.91351702)
\lineto(129.49140962,590.18351702)
\lineto(129.70140962,590.45351702)
\lineto(129.82140962,590.63351702)
\curveto(129.87139889,590.69350645)(129.92139884,590.74850639)(129.97140962,590.79851702)
\curveto(130.04139872,590.84850629)(130.11639865,590.88350626)(130.19640962,590.90351702)
\curveto(130.21639855,590.91350623)(130.24139852,590.91350623)(130.27140962,590.90351702)
\curveto(130.31139845,590.90350624)(130.34139842,590.91350623)(130.36140962,590.93351702)
\curveto(130.48139828,590.93350621)(130.61639815,590.92850621)(130.76640962,590.91851702)
\curveto(130.91639785,590.91850622)(131.00639776,590.87350627)(131.03640962,590.78351702)
\curveto(131.05639771,590.75350639)(131.0613977,590.71850642)(131.05140962,590.67851702)
\curveto(131.04139772,590.6385065)(131.02639774,590.60850653)(131.00640962,590.58851702)
\curveto(130.9663978,590.50850663)(130.92639784,590.4385067)(130.88640962,590.37851702)
\curveto(130.84639792,590.31850682)(130.80139796,590.25850688)(130.75140962,590.19851702)
\lineto(130.18140962,589.41851702)
\curveto(130.00139876,589.16850797)(129.82139894,588.91350823)(129.64140962,588.65351702)
\moveto(122.78640962,584.75351702)
\curveto(122.73640603,584.77351237)(122.68640608,584.77851236)(122.63640962,584.76851702)
\curveto(122.58640618,584.75851238)(122.53640623,584.76351238)(122.48640962,584.78351702)
\curveto(122.37640639,584.80351234)(122.27140649,584.82351232)(122.17140962,584.84351702)
\curveto(122.08140668,584.87351227)(121.98640678,584.91351223)(121.88640962,584.96351702)
\curveto(121.55640721,585.10351204)(121.30140746,585.29851184)(121.12140962,585.54851702)
\curveto(120.94140782,585.80851133)(120.79640797,586.11851102)(120.68640962,586.47851702)
\curveto(120.65640811,586.55851058)(120.63640813,586.6385105)(120.62640962,586.71851702)
\curveto(120.61640815,586.80851033)(120.60140816,586.89351025)(120.58140962,586.97351702)
\curveto(120.57140819,587.02351012)(120.5664082,587.08851005)(120.56640962,587.16851702)
\curveto(120.55640821,587.19850994)(120.55140821,587.22850991)(120.55140962,587.25851702)
\curveto(120.55140821,587.29850984)(120.54640822,587.33350981)(120.53640962,587.36351702)
\lineto(120.53640962,587.51351702)
\curveto(120.52640824,587.56350958)(120.52140824,587.62350952)(120.52140962,587.69351702)
\curveto(120.52140824,587.77350937)(120.52640824,587.8385093)(120.53640962,587.88851702)
\lineto(120.53640962,588.05351702)
\curveto(120.55640821,588.10350904)(120.5614082,588.14850899)(120.55140962,588.18851702)
\curveto(120.55140821,588.2385089)(120.55640821,588.28350886)(120.56640962,588.32351702)
\curveto(120.57640819,588.36350878)(120.58140818,588.39850874)(120.58140962,588.42851702)
\curveto(120.58140818,588.46850867)(120.58640818,588.50850863)(120.59640962,588.54851702)
\curveto(120.62640814,588.65850848)(120.64640812,588.76850837)(120.65640962,588.87851702)
\curveto(120.67640809,588.99850814)(120.71140805,589.11350803)(120.76140962,589.22351702)
\curveto(120.90140786,589.56350758)(121.0614077,589.8385073)(121.24140962,590.04851702)
\curveto(121.43140733,590.26850687)(121.70140706,590.44850669)(122.05140962,590.58851702)
\curveto(122.13140663,590.61850652)(122.21640655,590.6385065)(122.30640962,590.64851702)
\curveto(122.39640637,590.66850647)(122.49140627,590.68850645)(122.59140962,590.70851702)
\curveto(122.62140614,590.71850642)(122.67640609,590.71850642)(122.75640962,590.70851702)
\curveto(122.83640593,590.70850643)(122.88640588,590.71850642)(122.90640962,590.73851702)
\curveto(123.4664053,590.74850639)(123.91640485,590.6385065)(124.25640962,590.40851702)
\curveto(124.60640416,590.17850696)(124.8664039,589.87350727)(125.03640962,589.49351702)
\curveto(125.07640369,589.40350774)(125.11140365,589.30850783)(125.14140962,589.20851702)
\curveto(125.17140359,589.10850803)(125.19640357,589.00850813)(125.21640962,588.90851702)
\curveto(125.23640353,588.87850826)(125.24140352,588.84850829)(125.23140962,588.81851702)
\curveto(125.23140353,588.78850835)(125.23640353,588.75850838)(125.24640962,588.72851702)
\curveto(125.27640349,588.61850852)(125.29640347,588.49350865)(125.30640962,588.35351702)
\curveto(125.31640345,588.22350892)(125.32640344,588.08850905)(125.33640962,587.94851702)
\lineto(125.33640962,587.78351702)
\curveto(125.34640342,587.72350942)(125.34640342,587.66850947)(125.33640962,587.61851702)
\curveto(125.32640344,587.56850957)(125.32140344,587.51850962)(125.32140962,587.46851702)
\lineto(125.32140962,587.33351702)
\curveto(125.31140345,587.29350985)(125.30640346,587.25350989)(125.30640962,587.21351702)
\curveto(125.31640345,587.17350997)(125.31140345,587.12851001)(125.29140962,587.07851702)
\curveto(125.27140349,586.96851017)(125.25140351,586.86351028)(125.23140962,586.76351702)
\curveto(125.22140354,586.66351048)(125.20140356,586.56351058)(125.17140962,586.46351702)
\curveto(125.04140372,586.10351104)(124.87640389,585.78851135)(124.67640962,585.51851702)
\curveto(124.47640429,585.24851189)(124.20140456,585.0435121)(123.85140962,584.90351702)
\curveto(123.77140499,584.87351227)(123.68640508,584.84851229)(123.59640962,584.82851702)
\lineto(123.32640962,584.76851702)
\curveto(123.27640549,584.75851238)(123.23140553,584.75351239)(123.19140962,584.75351702)
\curveto(123.15140561,584.76351238)(123.11140565,584.76351238)(123.07140962,584.75351702)
\curveto(122.97140579,584.73351241)(122.87640589,584.73351241)(122.78640962,584.75351702)
\moveto(121.94640962,586.14851702)
\curveto(121.98640678,586.07851106)(122.02640674,586.01351113)(122.06640962,585.95351702)
\curveto(122.10640666,585.90351124)(122.15640661,585.85351129)(122.21640962,585.80351702)
\lineto(122.36640962,585.68351702)
\curveto(122.42640634,585.65351149)(122.49140627,585.62851151)(122.56140962,585.60851702)
\curveto(122.60140616,585.58851155)(122.63640613,585.57851156)(122.66640962,585.57851702)
\curveto(122.70640606,585.58851155)(122.74640602,585.58351156)(122.78640962,585.56351702)
\curveto(122.81640595,585.56351158)(122.85640591,585.55851158)(122.90640962,585.54851702)
\curveto(122.95640581,585.54851159)(122.99640577,585.55351159)(123.02640962,585.56351702)
\lineto(123.25140962,585.60851702)
\curveto(123.50140526,585.68851145)(123.68640508,585.81351133)(123.80640962,585.98351702)
\curveto(123.88640488,586.08351106)(123.95640481,586.21351093)(124.01640962,586.37351702)
\curveto(124.09640467,586.55351059)(124.15640461,586.77851036)(124.19640962,587.04851702)
\curveto(124.23640453,587.32850981)(124.25140451,587.60850953)(124.24140962,587.88851702)
\curveto(124.23140453,588.17850896)(124.20140456,588.45350869)(124.15140962,588.71351702)
\curveto(124.10140466,588.97350817)(124.02640474,589.18350796)(123.92640962,589.34351702)
\curveto(123.80640496,589.5435076)(123.65640511,589.69350745)(123.47640962,589.79351702)
\curveto(123.39640537,589.8435073)(123.30640546,589.87350727)(123.20640962,589.88351702)
\curveto(123.10640566,589.90350724)(123.00140576,589.91350723)(122.89140962,589.91351702)
\curveto(122.87140589,589.90350724)(122.84640592,589.89850724)(122.81640962,589.89851702)
\curveto(122.79640597,589.90850723)(122.77640599,589.90850723)(122.75640962,589.89851702)
\curveto(122.70640606,589.88850725)(122.6614061,589.87850726)(122.62140962,589.86851702)
\curveto(122.58140618,589.86850727)(122.54140622,589.85850728)(122.50140962,589.83851702)
\curveto(122.32140644,589.75850738)(122.17140659,589.6385075)(122.05140962,589.47851702)
\curveto(121.94140682,589.31850782)(121.85140691,589.138508)(121.78140962,588.93851702)
\curveto(121.72140704,588.74850839)(121.67640709,588.52350862)(121.64640962,588.26351702)
\curveto(121.62640714,588.00350914)(121.62140714,587.7385094)(121.63140962,587.46851702)
\curveto(121.64140712,587.20850993)(121.67140709,586.95851018)(121.72140962,586.71851702)
\curveto(121.78140698,586.48851065)(121.85640691,586.29851084)(121.94640962,586.14851702)
\moveto(132.74640962,583.16351702)
\curveto(132.75639601,583.11351403)(132.761396,583.02351412)(132.76140962,582.89351702)
\curveto(132.761396,582.76351438)(132.75139601,582.67351447)(132.73140962,582.62351702)
\curveto(132.71139605,582.57351457)(132.70639606,582.51851462)(132.71640962,582.45851702)
\curveto(132.72639604,582.40851473)(132.72639604,582.35851478)(132.71640962,582.30851702)
\curveto(132.67639609,582.16851497)(132.64639612,582.03351511)(132.62640962,581.90351702)
\curveto(132.61639615,581.77351537)(132.58639618,581.65351549)(132.53640962,581.54351702)
\curveto(132.39639637,581.19351595)(132.23139653,580.89851624)(132.04140962,580.65851702)
\curveto(131.85139691,580.42851671)(131.58139718,580.2435169)(131.23140962,580.10351702)
\curveto(131.15139761,580.07351707)(131.0663977,580.05351709)(130.97640962,580.04351702)
\curveto(130.88639788,580.02351712)(130.80139796,580.00351714)(130.72140962,579.98351702)
\curveto(130.67139809,579.97351717)(130.62139814,579.96851717)(130.57140962,579.96851702)
\curveto(130.52139824,579.96851717)(130.47139829,579.96351718)(130.42140962,579.95351702)
\curveto(130.39139837,579.9435172)(130.34139842,579.9435172)(130.27140962,579.95351702)
\curveto(130.20139856,579.95351719)(130.15139861,579.95851718)(130.12140962,579.96851702)
\curveto(130.0613987,579.98851715)(130.00139876,579.99851714)(129.94140962,579.99851702)
\curveto(129.89139887,579.98851715)(129.84139892,579.99351715)(129.79140962,580.01351702)
\curveto(129.70139906,580.03351711)(129.61139915,580.05851708)(129.52140962,580.08851702)
\curveto(129.44139932,580.10851703)(129.3613994,580.138517)(129.28140962,580.17851702)
\curveto(128.9613998,580.31851682)(128.71140005,580.51351663)(128.53140962,580.76351702)
\curveto(128.35140041,581.02351612)(128.20140056,581.32851581)(128.08140962,581.67851702)
\curveto(128.0614007,581.75851538)(128.04640072,581.8435153)(128.03640962,581.93351702)
\curveto(128.02640074,582.02351512)(128.01140075,582.10851503)(127.99140962,582.18851702)
\curveto(127.98140078,582.21851492)(127.97640079,582.24851489)(127.97640962,582.27851702)
\lineto(127.97640962,582.38351702)
\curveto(127.95640081,582.46351468)(127.94640082,582.5435146)(127.94640962,582.62351702)
\lineto(127.94640962,582.75851702)
\curveto(127.92640084,582.85851428)(127.92640084,582.95851418)(127.94640962,583.05851702)
\lineto(127.94640962,583.23851702)
\curveto(127.95640081,583.28851385)(127.9614008,583.33351381)(127.96140962,583.37351702)
\curveto(127.9614008,583.42351372)(127.9664008,583.46851367)(127.97640962,583.50851702)
\curveto(127.98640078,583.54851359)(127.99140077,583.58351356)(127.99140962,583.61351702)
\curveto(127.99140077,583.65351349)(127.99640077,583.69351345)(128.00640962,583.73351702)
\lineto(128.06640962,584.06351702)
\curveto(128.08640068,584.18351296)(128.11640065,584.29351285)(128.15640962,584.39351702)
\curveto(128.29640047,584.72351242)(128.45640031,584.99851214)(128.63640962,585.21851702)
\curveto(128.82639994,585.44851169)(129.08639968,585.63351151)(129.41640962,585.77351702)
\curveto(129.49639927,585.81351133)(129.58139918,585.8385113)(129.67140962,585.84851702)
\lineto(129.97140962,585.90851702)
\lineto(130.10640962,585.90851702)
\curveto(130.15639861,585.91851122)(130.20639856,585.92351122)(130.25640962,585.92351702)
\curveto(130.82639794,585.9435112)(131.28639748,585.8385113)(131.63640962,585.60851702)
\curveto(131.99639677,585.38851175)(132.2613965,585.08851205)(132.43140962,584.70851702)
\curveto(132.48139628,584.60851253)(132.52139624,584.50851263)(132.55140962,584.40851702)
\curveto(132.58139618,584.30851283)(132.61139615,584.20351294)(132.64140962,584.09351702)
\curveto(132.65139611,584.05351309)(132.65639611,584.01851312)(132.65640962,583.98851702)
\curveto(132.65639611,583.96851317)(132.6613961,583.9385132)(132.67140962,583.89851702)
\curveto(132.69139607,583.82851331)(132.70139606,583.75351339)(132.70140962,583.67351702)
\curveto(132.70139606,583.59351355)(132.71139605,583.51351363)(132.73140962,583.43351702)
\curveto(132.73139603,583.38351376)(132.73139603,583.3385138)(132.73140962,583.29851702)
\curveto(132.73139603,583.25851388)(132.73639603,583.21351393)(132.74640962,583.16351702)
\moveto(131.63640962,582.72851702)
\curveto(131.64639712,582.77851436)(131.65139711,582.85351429)(131.65140962,582.95351702)
\curveto(131.6613971,583.05351409)(131.65639711,583.12851401)(131.63640962,583.17851702)
\curveto(131.61639715,583.2385139)(131.61139715,583.29351385)(131.62140962,583.34351702)
\curveto(131.64139712,583.40351374)(131.64139712,583.46351368)(131.62140962,583.52351702)
\curveto(131.61139715,583.55351359)(131.60639716,583.58851355)(131.60640962,583.62851702)
\curveto(131.60639716,583.66851347)(131.60139716,583.70851343)(131.59140962,583.74851702)
\curveto(131.57139719,583.82851331)(131.55139721,583.90351324)(131.53140962,583.97351702)
\curveto(131.52139724,584.05351309)(131.50639726,584.13351301)(131.48640962,584.21351702)
\curveto(131.45639731,584.27351287)(131.43139733,584.33351281)(131.41140962,584.39351702)
\curveto(131.39139737,584.45351269)(131.3613974,584.51351263)(131.32140962,584.57351702)
\curveto(131.22139754,584.7435124)(131.09139767,584.87851226)(130.93140962,584.97851702)
\curveto(130.85139791,585.02851211)(130.75639801,585.06351208)(130.64640962,585.08351702)
\curveto(130.53639823,585.10351204)(130.41139835,585.11351203)(130.27140962,585.11351702)
\curveto(130.25139851,585.10351204)(130.22639854,585.09851204)(130.19640962,585.09851702)
\curveto(130.1663986,585.10851203)(130.13639863,585.10851203)(130.10640962,585.09851702)
\lineto(129.95640962,585.03851702)
\curveto(129.90639886,585.02851211)(129.8613989,585.01351213)(129.82140962,584.99351702)
\curveto(129.63139913,584.88351226)(129.48639928,584.7385124)(129.38640962,584.55851702)
\curveto(129.29639947,584.37851276)(129.21639955,584.17351297)(129.14640962,583.94351702)
\curveto(129.10639966,583.81351333)(129.08639968,583.67851346)(129.08640962,583.53851702)
\curveto(129.08639968,583.40851373)(129.07639969,583.26351388)(129.05640962,583.10351702)
\curveto(129.04639972,583.05351409)(129.03639973,582.99351415)(129.02640962,582.92351702)
\curveto(129.02639974,582.85351429)(129.03639973,582.79351435)(129.05640962,582.74351702)
\lineto(129.05640962,582.57851702)
\lineto(129.05640962,582.39851702)
\curveto(129.0663997,582.34851479)(129.07639969,582.29351485)(129.08640962,582.23351702)
\curveto(129.09639967,582.18351496)(129.10139966,582.12851501)(129.10140962,582.06851702)
\curveto(129.11139965,582.00851513)(129.12639964,581.95351519)(129.14640962,581.90351702)
\curveto(129.19639957,581.71351543)(129.25639951,581.5385156)(129.32640962,581.37851702)
\curveto(129.39639937,581.21851592)(129.50139926,581.08851605)(129.64140962,580.98851702)
\curveto(129.77139899,580.88851625)(129.91139885,580.81851632)(130.06140962,580.77851702)
\curveto(130.09139867,580.76851637)(130.11639865,580.76351638)(130.13640962,580.76351702)
\curveto(130.1663986,580.77351637)(130.19639857,580.77351637)(130.22640962,580.76351702)
\curveto(130.24639852,580.76351638)(130.27639849,580.75851638)(130.31640962,580.74851702)
\curveto(130.35639841,580.74851639)(130.39139837,580.75351639)(130.42140962,580.76351702)
\curveto(130.4613983,580.77351637)(130.50139826,580.77851636)(130.54140962,580.77851702)
\curveto(130.58139818,580.77851636)(130.62139814,580.78851635)(130.66140962,580.80851702)
\curveto(130.90139786,580.88851625)(131.09639767,581.02351612)(131.24640962,581.21351702)
\curveto(131.3663974,581.39351575)(131.45639731,581.59851554)(131.51640962,581.82851702)
\curveto(131.53639723,581.89851524)(131.55139721,581.96851517)(131.56140962,582.03851702)
\curveto(131.57139719,582.11851502)(131.58639718,582.19851494)(131.60640962,582.27851702)
\curveto(131.60639716,582.3385148)(131.61139715,582.38351476)(131.62140962,582.41351702)
\curveto(131.62139714,582.43351471)(131.62139714,582.45851468)(131.62140962,582.48851702)
\curveto(131.62139714,582.52851461)(131.62639714,582.55851458)(131.63640962,582.57851702)
\lineto(131.63640962,582.72851702)
}
}
{
\newrgbcolor{curcolor}{0 0 0}
\pscustom[linestyle=none,fillstyle=solid,fillcolor=curcolor]
{
\newpath
\moveto(92.93551027,503.86543963)
\curveto(93.03550542,503.86542901)(93.13050532,503.85542902)(93.22051027,503.83543963)
\curveto(93.31050514,503.82542905)(93.37550508,503.79542908)(93.41551027,503.74543963)
\curveto(93.47550498,503.66542921)(93.50550495,503.56042932)(93.50551027,503.43043963)
\lineto(93.50551027,503.04043963)
\lineto(93.50551027,501.54043963)
\lineto(93.50551027,495.15043963)
\lineto(93.50551027,493.98043963)
\lineto(93.50551027,493.66543963)
\curveto(93.51550494,493.56543931)(93.50050495,493.48543939)(93.46051027,493.42543963)
\curveto(93.41050504,493.34543953)(93.33550512,493.29543958)(93.23551027,493.27543963)
\curveto(93.14550531,493.26543961)(93.03550542,493.26043962)(92.90551027,493.26043963)
\lineto(92.68051027,493.26043963)
\curveto(92.60050585,493.2804396)(92.53050592,493.29543958)(92.47051027,493.30543963)
\curveto(92.41050604,493.32543955)(92.36050609,493.36543951)(92.32051027,493.42543963)
\curveto(92.28050617,493.48543939)(92.26050619,493.56043932)(92.26051027,493.65043963)
\lineto(92.26051027,493.95043963)
\lineto(92.26051027,495.04543963)
\lineto(92.26051027,500.38543963)
\curveto(92.24050621,500.4754324)(92.22550623,500.55043233)(92.21551027,500.61043963)
\curveto(92.21550624,500.6804322)(92.18550627,500.74043214)(92.12551027,500.79043963)
\curveto(92.0555064,500.84043204)(91.96550649,500.86543201)(91.85551027,500.86543963)
\curveto(91.7555067,500.875432)(91.64550681,500.880432)(91.52551027,500.88043963)
\lineto(90.38551027,500.88043963)
\lineto(89.89051027,500.88043963)
\curveto(89.73050872,500.89043199)(89.62050883,500.95043193)(89.56051027,501.06043963)
\curveto(89.54050891,501.09043179)(89.53050892,501.12043176)(89.53051027,501.15043963)
\curveto(89.53050892,501.19043169)(89.52550893,501.23543164)(89.51551027,501.28543963)
\curveto(89.49550896,501.40543147)(89.50050895,501.51543136)(89.53051027,501.61543963)
\curveto(89.57050888,501.71543116)(89.62550883,501.78543109)(89.69551027,501.82543963)
\curveto(89.77550868,501.875431)(89.89550856,501.90043098)(90.05551027,501.90043963)
\curveto(90.21550824,501.90043098)(90.3505081,501.91543096)(90.46051027,501.94543963)
\curveto(90.51050794,501.95543092)(90.56550789,501.96043092)(90.62551027,501.96043963)
\curveto(90.68550777,501.97043091)(90.74550771,501.98543089)(90.80551027,502.00543963)
\curveto(90.9555075,502.05543082)(91.10050735,502.10543077)(91.24051027,502.15543963)
\curveto(91.38050707,502.21543066)(91.51550694,502.28543059)(91.64551027,502.36543963)
\curveto(91.78550667,502.45543042)(91.90550655,502.56043032)(92.00551027,502.68043963)
\curveto(92.10550635,502.80043008)(92.20050625,502.93042995)(92.29051027,503.07043963)
\curveto(92.3505061,503.17042971)(92.39550606,503.2804296)(92.42551027,503.40043963)
\curveto(92.46550599,503.52042936)(92.51550594,503.62542925)(92.57551027,503.71543963)
\curveto(92.62550583,503.7754291)(92.69550576,503.81542906)(92.78551027,503.83543963)
\curveto(92.80550565,503.84542903)(92.83050562,503.85042903)(92.86051027,503.85043963)
\curveto(92.89050556,503.85042903)(92.91550554,503.85542902)(92.93551027,503.86543963)
}
}
{
\newrgbcolor{curcolor}{0 0 0}
\pscustom[linestyle=none,fillstyle=solid,fillcolor=curcolor]
{
\newpath
\moveto(104.16511965,498.85543963)
\curveto(104.16511201,498.7754341)(104.17011201,498.69543418)(104.18011965,498.61543963)
\curveto(104.19011199,498.53543434)(104.18511199,498.46043442)(104.16511965,498.39043963)
\curveto(104.14511203,498.35043453)(104.14011204,498.30543457)(104.15011965,498.25543963)
\curveto(104.16011202,498.21543466)(104.16011202,498.1754347)(104.15011965,498.13543963)
\lineto(104.15011965,497.98543963)
\curveto(104.14011204,497.89543498)(104.13511204,497.80543507)(104.13511965,497.71543963)
\curveto(104.13511204,497.63543524)(104.13011205,497.55543532)(104.12011965,497.47543963)
\lineto(104.09011965,497.23543963)
\curveto(104.0801121,497.16543571)(104.07011211,497.09043579)(104.06011965,497.01043963)
\curveto(104.05011213,496.97043591)(104.04511213,496.93043595)(104.04511965,496.89043963)
\curveto(104.04511213,496.85043603)(104.04011214,496.80543607)(104.03011965,496.75543963)
\curveto(103.99011219,496.61543626)(103.96011222,496.4754364)(103.94011965,496.33543963)
\curveto(103.93011225,496.19543668)(103.90011228,496.06043682)(103.85011965,495.93043963)
\curveto(103.80011238,495.76043712)(103.74511243,495.59543728)(103.68511965,495.43543963)
\curveto(103.63511254,495.2754376)(103.5751126,495.12043776)(103.50511965,494.97043963)
\curveto(103.48511269,494.91043797)(103.45511272,494.85043803)(103.41511965,494.79043963)
\lineto(103.32511965,494.64043963)
\curveto(103.12511305,494.32043856)(102.91011327,494.05543882)(102.68011965,493.84543963)
\curveto(102.45011373,493.63543924)(102.15511402,493.45543942)(101.79511965,493.30543963)
\curveto(101.6751145,493.25543962)(101.54511463,493.22043966)(101.40511965,493.20043963)
\curveto(101.2751149,493.1804397)(101.14011504,493.15543972)(101.00011965,493.12543963)
\curveto(100.94011524,493.11543976)(100.8801153,493.11043977)(100.82011965,493.11043963)
\curveto(100.76011542,493.11043977)(100.69511548,493.10543977)(100.62511965,493.09543963)
\curveto(100.59511558,493.08543979)(100.54511563,493.08543979)(100.47511965,493.09543963)
\lineto(100.32511965,493.09543963)
\lineto(100.17511965,493.09543963)
\curveto(100.09511608,493.11543976)(100.01011617,493.13043975)(99.92011965,493.14043963)
\curveto(99.84011634,493.14043974)(99.76511641,493.15043973)(99.69511965,493.17043963)
\curveto(99.65511652,493.1804397)(99.62011656,493.18543969)(99.59011965,493.18543963)
\curveto(99.57011661,493.1754397)(99.54511663,493.1804397)(99.51511965,493.20043963)
\lineto(99.24511965,493.26043963)
\curveto(99.15511702,493.29043959)(99.07011711,493.32043956)(98.99011965,493.35043963)
\curveto(98.41011777,493.59043929)(97.9751182,493.96043892)(97.68511965,494.46043963)
\curveto(97.60511857,494.59043829)(97.54011864,494.72543815)(97.49011965,494.86543963)
\curveto(97.45011873,495.00543787)(97.40511877,495.15543772)(97.35511965,495.31543963)
\curveto(97.33511884,495.39543748)(97.33011885,495.4754374)(97.34011965,495.55543963)
\curveto(97.36011882,495.63543724)(97.39511878,495.69043719)(97.44511965,495.72043963)
\curveto(97.4751187,495.74043714)(97.53011865,495.75543712)(97.61011965,495.76543963)
\curveto(97.69011849,495.78543709)(97.7751184,495.79543708)(97.86511965,495.79543963)
\curveto(97.95511822,495.80543707)(98.04011814,495.80543707)(98.12011965,495.79543963)
\curveto(98.21011797,495.78543709)(98.2801179,495.7754371)(98.33011965,495.76543963)
\curveto(98.35011783,495.75543712)(98.3751178,495.74043714)(98.40511965,495.72043963)
\curveto(98.44511773,495.70043718)(98.4751177,495.6804372)(98.49511965,495.66043963)
\curveto(98.55511762,495.5804373)(98.60011758,495.48543739)(98.63011965,495.37543963)
\curveto(98.67011751,495.26543761)(98.71511746,495.16543771)(98.76511965,495.07543963)
\curveto(99.01511716,494.68543819)(99.38511679,494.41543846)(99.87511965,494.26543963)
\curveto(99.94511623,494.24543863)(100.01511616,494.23043865)(100.08511965,494.22043963)
\curveto(100.16511601,494.22043866)(100.24511593,494.21043867)(100.32511965,494.19043963)
\curveto(100.36511581,494.1804387)(100.42011576,494.1754387)(100.49011965,494.17543963)
\curveto(100.57011561,494.1754387)(100.62511555,494.1804387)(100.65511965,494.19043963)
\curveto(100.68511549,494.20043868)(100.71511546,494.20543867)(100.74511965,494.20543963)
\lineto(100.85011965,494.20543963)
\curveto(100.93011525,494.22543865)(101.00511517,494.24543863)(101.07511965,494.26543963)
\curveto(101.15511502,494.28543859)(101.23011495,494.31043857)(101.30011965,494.34043963)
\curveto(101.65011453,494.49043839)(101.92011426,494.70543817)(102.11011965,494.98543963)
\curveto(102.30011388,495.26543761)(102.45511372,495.59043729)(102.57511965,495.96043963)
\curveto(102.60511357,496.04043684)(102.62511355,496.11543676)(102.63511965,496.18543963)
\curveto(102.65511352,496.25543662)(102.6751135,496.33043655)(102.69511965,496.41043963)
\curveto(102.71511346,496.50043638)(102.73011345,496.59543628)(102.74011965,496.69543963)
\curveto(102.76011342,496.80543607)(102.7801134,496.91043597)(102.80011965,497.01043963)
\curveto(102.81011337,497.06043582)(102.81511336,497.11043577)(102.81511965,497.16043963)
\curveto(102.82511335,497.22043566)(102.83011335,497.2754356)(102.83011965,497.32543963)
\curveto(102.85011333,497.38543549)(102.86011332,497.46043542)(102.86011965,497.55043963)
\curveto(102.86011332,497.65043523)(102.85011333,497.73043515)(102.83011965,497.79043963)
\curveto(102.80011338,497.880435)(102.75011343,497.92043496)(102.68011965,497.91043963)
\curveto(102.62011356,497.90043498)(102.56511361,497.87043501)(102.51511965,497.82043963)
\curveto(102.43511374,497.77043511)(102.36511381,497.71043517)(102.30511965,497.64043963)
\curveto(102.25511392,497.57043531)(102.19011399,497.51043537)(102.11011965,497.46043963)
\curveto(101.95011423,497.35043553)(101.78511439,497.25043563)(101.61511965,497.16043963)
\curveto(101.44511473,497.0804358)(101.25011493,497.01043587)(101.03011965,496.95043963)
\curveto(100.93011525,496.92043596)(100.83011535,496.90543597)(100.73011965,496.90543963)
\curveto(100.64011554,496.90543597)(100.54011564,496.89543598)(100.43011965,496.87543963)
\lineto(100.28011965,496.87543963)
\curveto(100.23011595,496.89543598)(100.180116,496.90043598)(100.13011965,496.89043963)
\curveto(100.09011609,496.880436)(100.05011613,496.880436)(100.01011965,496.89043963)
\curveto(99.9801162,496.90043598)(99.93511624,496.90543597)(99.87511965,496.90543963)
\curveto(99.81511636,496.91543596)(99.75011643,496.92543595)(99.68011965,496.93543963)
\lineto(99.50011965,496.96543963)
\curveto(99.05011713,497.08543579)(98.67011751,497.25043563)(98.36011965,497.46043963)
\curveto(98.09011809,497.65043523)(97.86011832,497.880435)(97.67011965,498.15043963)
\curveto(97.49011869,498.43043445)(97.34511883,498.74543413)(97.23511965,499.09543963)
\lineto(97.17511965,499.30543963)
\curveto(97.16511901,499.38543349)(97.15011903,499.46543341)(97.13011965,499.54543963)
\curveto(97.12011906,499.5754333)(97.11511906,499.60543327)(97.11511965,499.63543963)
\curveto(97.11511906,499.66543321)(97.11011907,499.69543318)(97.10011965,499.72543963)
\curveto(97.09011909,499.78543309)(97.08511909,499.84543303)(97.08511965,499.90543963)
\curveto(97.08511909,499.9754329)(97.0751191,500.03543284)(97.05511965,500.08543963)
\lineto(97.05511965,500.26543963)
\curveto(97.04511913,500.31543256)(97.04011914,500.38543249)(97.04011965,500.47543963)
\curveto(97.04011914,500.56543231)(97.05011913,500.63543224)(97.07011965,500.68543963)
\lineto(97.07011965,500.85043963)
\curveto(97.09011909,500.93043195)(97.10011908,501.00543187)(97.10011965,501.07543963)
\curveto(97.11011907,501.14543173)(97.12511905,501.21543166)(97.14511965,501.28543963)
\curveto(97.20511897,501.48543139)(97.26511891,501.6754312)(97.32511965,501.85543963)
\curveto(97.39511878,502.03543084)(97.48511869,502.20543067)(97.59511965,502.36543963)
\curveto(97.63511854,502.43543044)(97.6751185,502.50043038)(97.71511965,502.56043963)
\lineto(97.86511965,502.74043963)
\curveto(97.88511829,502.75043013)(97.90511827,502.76543011)(97.92511965,502.78543963)
\curveto(98.01511816,502.91542996)(98.12511805,503.02542985)(98.25511965,503.11543963)
\curveto(98.51511766,503.31542956)(98.7801174,503.47042941)(99.05011965,503.58043963)
\curveto(99.13011705,503.62042926)(99.21011697,503.65042923)(99.29011965,503.67043963)
\curveto(99.3801168,503.70042918)(99.47011671,503.72542915)(99.56011965,503.74543963)
\curveto(99.66011652,503.7754291)(99.76011642,503.79542908)(99.86011965,503.80543963)
\curveto(99.96011622,503.81542906)(100.06511611,503.83042905)(100.17511965,503.85043963)
\curveto(100.20511597,503.86042902)(100.24511593,503.86042902)(100.29511965,503.85043963)
\curveto(100.35511582,503.84042904)(100.39511578,503.84542903)(100.41511965,503.86543963)
\curveto(101.13511504,503.88542899)(101.73511444,503.77042911)(102.21511965,503.52043963)
\curveto(102.69511348,503.27042961)(103.07011311,502.93042995)(103.34011965,502.50043963)
\curveto(103.43011275,502.36043052)(103.51011267,502.21543066)(103.58011965,502.06543963)
\curveto(103.65011253,501.91543096)(103.72011246,501.75543112)(103.79011965,501.58543963)
\curveto(103.84011234,501.44543143)(103.8801123,501.29543158)(103.91011965,501.13543963)
\curveto(103.94011224,500.9754319)(103.9751122,500.81543206)(104.01511965,500.65543963)
\curveto(104.03511214,500.60543227)(104.04511213,500.55043233)(104.04511965,500.49043963)
\curveto(104.04511213,500.44043244)(104.05011213,500.39043249)(104.06011965,500.34043963)
\curveto(104.0801121,500.2804326)(104.09011209,500.21543266)(104.09011965,500.14543963)
\curveto(104.09011209,500.08543279)(104.10011208,500.03043285)(104.12011965,499.98043963)
\lineto(104.12011965,499.81543963)
\curveto(104.14011204,499.76543311)(104.14511203,499.71543316)(104.13511965,499.66543963)
\curveto(104.12511205,499.61543326)(104.13011205,499.56543331)(104.15011965,499.51543963)
\curveto(104.15011203,499.49543338)(104.14511203,499.47043341)(104.13511965,499.44043963)
\curveto(104.13511204,499.41043347)(104.14011204,499.38543349)(104.15011965,499.36543963)
\curveto(104.16011202,499.33543354)(104.16011202,499.30043358)(104.15011965,499.26043963)
\curveto(104.15011203,499.22043366)(104.15511202,499.1804337)(104.16511965,499.14043963)
\curveto(104.175112,499.10043378)(104.175112,499.05543382)(104.16511965,499.00543963)
\lineto(104.16511965,498.85543963)
\moveto(102.66511965,500.16043963)
\curveto(102.6751135,500.21043267)(102.6801135,500.27043261)(102.68011965,500.34043963)
\curveto(102.6801135,500.41043247)(102.6751135,500.47043241)(102.66511965,500.52043963)
\curveto(102.65511352,500.57043231)(102.65011353,500.64543223)(102.65011965,500.74543963)
\curveto(102.63011355,500.82543205)(102.61011357,500.90043198)(102.59011965,500.97043963)
\curveto(102.5801136,501.04043184)(102.56511361,501.11043177)(102.54511965,501.18043963)
\curveto(102.40511377,501.61043127)(102.21011397,501.94543093)(101.96011965,502.18543963)
\curveto(101.72011446,502.42543045)(101.3751148,502.60543027)(100.92511965,502.72543963)
\curveto(100.83511534,502.74543013)(100.73511544,502.75543012)(100.62511965,502.75543963)
\lineto(100.29511965,502.75543963)
\curveto(100.2751159,502.73543014)(100.24011594,502.72543015)(100.19011965,502.72543963)
\curveto(100.14011604,502.73543014)(100.09511608,502.73543014)(100.05511965,502.72543963)
\curveto(99.9751162,502.70543017)(99.90011628,502.68543019)(99.83011965,502.66543963)
\lineto(99.62011965,502.60543963)
\curveto(99.33011685,502.4754304)(99.10011708,502.29543058)(98.93011965,502.06543963)
\curveto(98.76011742,501.84543103)(98.62511755,501.58543129)(98.52511965,501.28543963)
\curveto(98.49511768,501.19543168)(98.47011771,501.10043178)(98.45011965,501.00043963)
\curveto(98.44011774,500.91043197)(98.42511775,500.81543206)(98.40511965,500.71543963)
\lineto(98.40511965,500.58043963)
\curveto(98.3751178,500.47043241)(98.36511781,500.33043255)(98.37511965,500.16043963)
\curveto(98.39511778,500.00043288)(98.41511776,499.87043301)(98.43511965,499.77043963)
\curveto(98.45511772,499.71043317)(98.47011771,499.65043323)(98.48011965,499.59043963)
\curveto(98.49011769,499.54043334)(98.50511767,499.49043339)(98.52511965,499.44043963)
\curveto(98.60511757,499.24043364)(98.70011748,499.05043383)(98.81011965,498.87043963)
\curveto(98.93011725,498.69043419)(99.07011711,498.54543433)(99.23011965,498.43543963)
\curveto(99.2801169,498.38543449)(99.33511684,498.34543453)(99.39511965,498.31543963)
\curveto(99.45511672,498.28543459)(99.51511666,498.25043463)(99.57511965,498.21043963)
\curveto(99.72511645,498.13043475)(99.91011627,498.06543481)(100.13011965,498.01543963)
\curveto(100.180116,497.99543488)(100.22011596,497.99043489)(100.25011965,498.00043963)
\curveto(100.29011589,498.01043487)(100.33511584,498.00543487)(100.38511965,497.98543963)
\curveto(100.42511575,497.9754349)(100.4801157,497.97043491)(100.55011965,497.97043963)
\curveto(100.62011556,497.97043491)(100.6801155,497.9754349)(100.73011965,497.98543963)
\curveto(100.83011535,498.00543487)(100.92511525,498.02043486)(101.01511965,498.03043963)
\curveto(101.10511507,498.05043483)(101.19511498,498.0804348)(101.28511965,498.12043963)
\curveto(101.82511435,498.34043454)(102.22011396,498.73543414)(102.47011965,499.30543963)
\curveto(102.52011366,499.40543347)(102.55511362,499.50543337)(102.57511965,499.60543963)
\curveto(102.59511358,499.71543316)(102.62011356,499.82543305)(102.65011965,499.93543963)
\curveto(102.65011353,500.03543284)(102.65511352,500.11043277)(102.66511965,500.16043963)
}
}
{
\newrgbcolor{curcolor}{0 0 0}
\pscustom[linestyle=none,fillstyle=solid,fillcolor=curcolor]
{
\newpath
\moveto(106.52972902,494.89543963)
\lineto(106.82972902,494.89543963)
\curveto(106.93972696,494.90543797)(107.04472686,494.90543797)(107.14472902,494.89543963)
\curveto(107.25472665,494.89543798)(107.35472655,494.88543799)(107.44472902,494.86543963)
\curveto(107.53472637,494.85543802)(107.6047263,494.83043805)(107.65472902,494.79043963)
\curveto(107.67472623,494.77043811)(107.68972621,494.74043814)(107.69972902,494.70043963)
\curveto(107.71972618,494.66043822)(107.73972616,494.61543826)(107.75972902,494.56543963)
\lineto(107.75972902,494.49043963)
\curveto(107.76972613,494.44043844)(107.76972613,494.38543849)(107.75972902,494.32543963)
\lineto(107.75972902,494.17543963)
\lineto(107.75972902,493.69543963)
\curveto(107.75972614,493.52543935)(107.71972618,493.40543947)(107.63972902,493.33543963)
\curveto(107.56972633,493.28543959)(107.47972642,493.26043962)(107.36972902,493.26043963)
\lineto(107.03972902,493.26043963)
\lineto(106.58972902,493.26043963)
\curveto(106.43972746,493.26043962)(106.32472758,493.29043959)(106.24472902,493.35043963)
\curveto(106.2047277,493.3804395)(106.17472773,493.43043945)(106.15472902,493.50043963)
\curveto(106.13472777,493.5804393)(106.11972778,493.66543921)(106.10972902,493.75543963)
\lineto(106.10972902,494.04043963)
\curveto(106.11972778,494.14043874)(106.12472778,494.22543865)(106.12472902,494.29543963)
\lineto(106.12472902,494.49043963)
\curveto(106.12472778,494.55043833)(106.13472777,494.60543827)(106.15472902,494.65543963)
\curveto(106.19472771,494.76543811)(106.26472764,494.83543804)(106.36472902,494.86543963)
\curveto(106.39472751,494.86543801)(106.44972745,494.875438)(106.52972902,494.89543963)
}
}
{
\newrgbcolor{curcolor}{0 0 0}
\pscustom[linestyle=none,fillstyle=solid,fillcolor=curcolor]
{
\newpath
\moveto(113.79488527,503.86543963)
\curveto(113.89488042,503.86542901)(113.98988032,503.85542902)(114.07988527,503.83543963)
\curveto(114.16988014,503.82542905)(114.23488008,503.79542908)(114.27488527,503.74543963)
\curveto(114.33487998,503.66542921)(114.36487995,503.56042932)(114.36488527,503.43043963)
\lineto(114.36488527,503.04043963)
\lineto(114.36488527,501.54043963)
\lineto(114.36488527,495.15043963)
\lineto(114.36488527,493.98043963)
\lineto(114.36488527,493.66543963)
\curveto(114.37487994,493.56543931)(114.35987995,493.48543939)(114.31988527,493.42543963)
\curveto(114.26988004,493.34543953)(114.19488012,493.29543958)(114.09488527,493.27543963)
\curveto(114.00488031,493.26543961)(113.89488042,493.26043962)(113.76488527,493.26043963)
\lineto(113.53988527,493.26043963)
\curveto(113.45988085,493.2804396)(113.38988092,493.29543958)(113.32988527,493.30543963)
\curveto(113.26988104,493.32543955)(113.21988109,493.36543951)(113.17988527,493.42543963)
\curveto(113.13988117,493.48543939)(113.11988119,493.56043932)(113.11988527,493.65043963)
\lineto(113.11988527,493.95043963)
\lineto(113.11988527,495.04543963)
\lineto(113.11988527,500.38543963)
\curveto(113.09988121,500.4754324)(113.08488123,500.55043233)(113.07488527,500.61043963)
\curveto(113.07488124,500.6804322)(113.04488127,500.74043214)(112.98488527,500.79043963)
\curveto(112.9148814,500.84043204)(112.82488149,500.86543201)(112.71488527,500.86543963)
\curveto(112.6148817,500.875432)(112.50488181,500.880432)(112.38488527,500.88043963)
\lineto(111.24488527,500.88043963)
\lineto(110.74988527,500.88043963)
\curveto(110.58988372,500.89043199)(110.47988383,500.95043193)(110.41988527,501.06043963)
\curveto(110.39988391,501.09043179)(110.38988392,501.12043176)(110.38988527,501.15043963)
\curveto(110.38988392,501.19043169)(110.38488393,501.23543164)(110.37488527,501.28543963)
\curveto(110.35488396,501.40543147)(110.35988395,501.51543136)(110.38988527,501.61543963)
\curveto(110.42988388,501.71543116)(110.48488383,501.78543109)(110.55488527,501.82543963)
\curveto(110.63488368,501.875431)(110.75488356,501.90043098)(110.91488527,501.90043963)
\curveto(111.07488324,501.90043098)(111.2098831,501.91543096)(111.31988527,501.94543963)
\curveto(111.36988294,501.95543092)(111.42488289,501.96043092)(111.48488527,501.96043963)
\curveto(111.54488277,501.97043091)(111.60488271,501.98543089)(111.66488527,502.00543963)
\curveto(111.8148825,502.05543082)(111.95988235,502.10543077)(112.09988527,502.15543963)
\curveto(112.23988207,502.21543066)(112.37488194,502.28543059)(112.50488527,502.36543963)
\curveto(112.64488167,502.45543042)(112.76488155,502.56043032)(112.86488527,502.68043963)
\curveto(112.96488135,502.80043008)(113.05988125,502.93042995)(113.14988527,503.07043963)
\curveto(113.2098811,503.17042971)(113.25488106,503.2804296)(113.28488527,503.40043963)
\curveto(113.32488099,503.52042936)(113.37488094,503.62542925)(113.43488527,503.71543963)
\curveto(113.48488083,503.7754291)(113.55488076,503.81542906)(113.64488527,503.83543963)
\curveto(113.66488065,503.84542903)(113.68988062,503.85042903)(113.71988527,503.85043963)
\curveto(113.74988056,503.85042903)(113.77488054,503.85542902)(113.79488527,503.86543963)
}
}
{
\newrgbcolor{curcolor}{0 0 0}
\pscustom[linestyle=none,fillstyle=solid,fillcolor=curcolor]
{
\newpath
\moveto(127.88949465,501.78043963)
\curveto(127.68948435,501.49043139)(127.47948456,501.20543167)(127.25949465,500.92543963)
\curveto(127.04948499,500.64543223)(126.84448519,500.36043252)(126.64449465,500.07043963)
\curveto(126.04448599,499.22043366)(125.4394866,498.3804345)(124.82949465,497.55043963)
\curveto(124.21948782,496.73043615)(123.61448842,495.89543698)(123.01449465,495.04543963)
\lineto(122.50449465,494.32543963)
\lineto(121.99449465,493.63543963)
\curveto(121.91449012,493.52543935)(121.8344902,493.41043947)(121.75449465,493.29043963)
\curveto(121.67449036,493.17043971)(121.57949046,493.0754398)(121.46949465,493.00543963)
\curveto(121.42949061,492.98543989)(121.36449067,492.97043991)(121.27449465,492.96043963)
\curveto(121.19449084,492.94043994)(121.10449093,492.93043995)(121.00449465,492.93043963)
\curveto(120.90449113,492.93043995)(120.80949123,492.93543994)(120.71949465,492.94543963)
\curveto(120.6394914,492.95543992)(120.57949146,492.9754399)(120.53949465,493.00543963)
\curveto(120.50949153,493.02543985)(120.48449155,493.06043982)(120.46449465,493.11043963)
\curveto(120.45449158,493.15043973)(120.45949158,493.19543968)(120.47949465,493.24543963)
\curveto(120.51949152,493.32543955)(120.56449147,493.40043948)(120.61449465,493.47043963)
\curveto(120.67449136,493.55043933)(120.72949131,493.63043925)(120.77949465,493.71043963)
\curveto(121.01949102,494.05043883)(121.26449077,494.38543849)(121.51449465,494.71543963)
\curveto(121.76449027,495.04543783)(122.00449003,495.3804375)(122.23449465,495.72043963)
\curveto(122.39448964,495.94043694)(122.55448948,496.15543672)(122.71449465,496.36543963)
\curveto(122.87448916,496.5754363)(123.034489,496.79043609)(123.19449465,497.01043963)
\curveto(123.55448848,497.53043535)(123.91948812,498.04043484)(124.28949465,498.54043963)
\curveto(124.65948738,499.04043384)(125.02948701,499.55043333)(125.39949465,500.07043963)
\curveto(125.5394865,500.27043261)(125.67948636,500.46543241)(125.81949465,500.65543963)
\curveto(125.96948607,500.84543203)(126.11448592,501.04043184)(126.25449465,501.24043963)
\curveto(126.46448557,501.54043134)(126.67948536,501.84043104)(126.89949465,502.14043963)
\lineto(127.55949465,503.04043963)
\lineto(127.73949465,503.31043963)
\lineto(127.94949465,503.58043963)
\lineto(128.06949465,503.76043963)
\curveto(128.11948392,503.82042906)(128.16948387,503.875429)(128.21949465,503.92543963)
\curveto(128.28948375,503.9754289)(128.36448367,504.01042887)(128.44449465,504.03043963)
\curveto(128.46448357,504.04042884)(128.48948355,504.04042884)(128.51949465,504.03043963)
\curveto(128.55948348,504.03042885)(128.58948345,504.04042884)(128.60949465,504.06043963)
\curveto(128.72948331,504.06042882)(128.86448317,504.05542882)(129.01449465,504.04543963)
\curveto(129.16448287,504.04542883)(129.25448278,504.00042888)(129.28449465,503.91043963)
\curveto(129.30448273,503.880429)(129.30948273,503.84542903)(129.29949465,503.80543963)
\curveto(129.28948275,503.76542911)(129.27448276,503.73542914)(129.25449465,503.71543963)
\curveto(129.21448282,503.63542924)(129.17448286,503.56542931)(129.13449465,503.50543963)
\curveto(129.09448294,503.44542943)(129.04948299,503.38542949)(128.99949465,503.32543963)
\lineto(128.42949465,502.54543963)
\curveto(128.24948379,502.29543058)(128.06948397,502.04043084)(127.88949465,501.78043963)
\moveto(121.03449465,497.88043963)
\curveto(120.98449105,497.90043498)(120.9344911,497.90543497)(120.88449465,497.89543963)
\curveto(120.8344912,497.88543499)(120.78449125,497.89043499)(120.73449465,497.91043963)
\curveto(120.62449141,497.93043495)(120.51949152,497.95043493)(120.41949465,497.97043963)
\curveto(120.32949171,498.00043488)(120.2344918,498.04043484)(120.13449465,498.09043963)
\curveto(119.80449223,498.23043465)(119.54949249,498.42543445)(119.36949465,498.67543963)
\curveto(119.18949285,498.93543394)(119.04449299,499.24543363)(118.93449465,499.60543963)
\curveto(118.90449313,499.68543319)(118.88449315,499.76543311)(118.87449465,499.84543963)
\curveto(118.86449317,499.93543294)(118.84949319,500.02043286)(118.82949465,500.10043963)
\curveto(118.81949322,500.15043273)(118.81449322,500.21543266)(118.81449465,500.29543963)
\curveto(118.80449323,500.32543255)(118.79949324,500.35543252)(118.79949465,500.38543963)
\curveto(118.79949324,500.42543245)(118.79449324,500.46043242)(118.78449465,500.49043963)
\lineto(118.78449465,500.64043963)
\curveto(118.77449326,500.69043219)(118.76949327,500.75043213)(118.76949465,500.82043963)
\curveto(118.76949327,500.90043198)(118.77449326,500.96543191)(118.78449465,501.01543963)
\lineto(118.78449465,501.18043963)
\curveto(118.80449323,501.23043165)(118.80949323,501.2754316)(118.79949465,501.31543963)
\curveto(118.79949324,501.36543151)(118.80449323,501.41043147)(118.81449465,501.45043963)
\curveto(118.82449321,501.49043139)(118.82949321,501.52543135)(118.82949465,501.55543963)
\curveto(118.82949321,501.59543128)(118.8344932,501.63543124)(118.84449465,501.67543963)
\curveto(118.87449316,501.78543109)(118.89449314,501.89543098)(118.90449465,502.00543963)
\curveto(118.92449311,502.12543075)(118.95949308,502.24043064)(119.00949465,502.35043963)
\curveto(119.14949289,502.69043019)(119.30949273,502.96542991)(119.48949465,503.17543963)
\curveto(119.67949236,503.39542948)(119.94949209,503.5754293)(120.29949465,503.71543963)
\curveto(120.37949166,503.74542913)(120.46449157,503.76542911)(120.55449465,503.77543963)
\curveto(120.64449139,503.79542908)(120.7394913,503.81542906)(120.83949465,503.83543963)
\curveto(120.86949117,503.84542903)(120.92449111,503.84542903)(121.00449465,503.83543963)
\curveto(121.08449095,503.83542904)(121.1344909,503.84542903)(121.15449465,503.86543963)
\curveto(121.71449032,503.875429)(122.16448987,503.76542911)(122.50449465,503.53543963)
\curveto(122.85448918,503.30542957)(123.11448892,503.00042988)(123.28449465,502.62043963)
\curveto(123.32448871,502.53043035)(123.35948868,502.43543044)(123.38949465,502.33543963)
\curveto(123.41948862,502.23543064)(123.44448859,502.13543074)(123.46449465,502.03543963)
\curveto(123.48448855,502.00543087)(123.48948855,501.9754309)(123.47949465,501.94543963)
\curveto(123.47948856,501.91543096)(123.48448855,501.88543099)(123.49449465,501.85543963)
\curveto(123.52448851,501.74543113)(123.54448849,501.62043126)(123.55449465,501.48043963)
\curveto(123.56448847,501.35043153)(123.57448846,501.21543166)(123.58449465,501.07543963)
\lineto(123.58449465,500.91043963)
\curveto(123.59448844,500.85043203)(123.59448844,500.79543208)(123.58449465,500.74543963)
\curveto(123.57448846,500.69543218)(123.56948847,500.64543223)(123.56949465,500.59543963)
\lineto(123.56949465,500.46043963)
\curveto(123.55948848,500.42043246)(123.55448848,500.3804325)(123.55449465,500.34043963)
\curveto(123.56448847,500.30043258)(123.55948848,500.25543262)(123.53949465,500.20543963)
\curveto(123.51948852,500.09543278)(123.49948854,499.99043289)(123.47949465,499.89043963)
\curveto(123.46948857,499.79043309)(123.44948859,499.69043319)(123.41949465,499.59043963)
\curveto(123.28948875,499.23043365)(123.12448891,498.91543396)(122.92449465,498.64543963)
\curveto(122.72448931,498.3754345)(122.44948959,498.17043471)(122.09949465,498.03043963)
\curveto(122.01949002,498.00043488)(121.9344901,497.9754349)(121.84449465,497.95543963)
\lineto(121.57449465,497.89543963)
\curveto(121.52449051,497.88543499)(121.47949056,497.880435)(121.43949465,497.88043963)
\curveto(121.39949064,497.89043499)(121.35949068,497.89043499)(121.31949465,497.88043963)
\curveto(121.21949082,497.86043502)(121.12449091,497.86043502)(121.03449465,497.88043963)
\moveto(120.19449465,499.27543963)
\curveto(120.2344918,499.20543367)(120.27449176,499.14043374)(120.31449465,499.08043963)
\curveto(120.35449168,499.03043385)(120.40449163,498.9804339)(120.46449465,498.93043963)
\lineto(120.61449465,498.81043963)
\curveto(120.67449136,498.7804341)(120.7394913,498.75543412)(120.80949465,498.73543963)
\curveto(120.84949119,498.71543416)(120.88449115,498.70543417)(120.91449465,498.70543963)
\curveto(120.95449108,498.71543416)(120.99449104,498.71043417)(121.03449465,498.69043963)
\curveto(121.06449097,498.69043419)(121.10449093,498.68543419)(121.15449465,498.67543963)
\curveto(121.20449083,498.6754342)(121.24449079,498.6804342)(121.27449465,498.69043963)
\lineto(121.49949465,498.73543963)
\curveto(121.74949029,498.81543406)(121.9344901,498.94043394)(122.05449465,499.11043963)
\curveto(122.1344899,499.21043367)(122.20448983,499.34043354)(122.26449465,499.50043963)
\curveto(122.34448969,499.6804332)(122.40448963,499.90543297)(122.44449465,500.17543963)
\curveto(122.48448955,500.45543242)(122.49948954,500.73543214)(122.48949465,501.01543963)
\curveto(122.47948956,501.30543157)(122.44948959,501.5804313)(122.39949465,501.84043963)
\curveto(122.34948969,502.10043078)(122.27448976,502.31043057)(122.17449465,502.47043963)
\curveto(122.05448998,502.67043021)(121.90449013,502.82043006)(121.72449465,502.92043963)
\curveto(121.64449039,502.97042991)(121.55449048,503.00042988)(121.45449465,503.01043963)
\curveto(121.35449068,503.03042985)(121.24949079,503.04042984)(121.13949465,503.04043963)
\curveto(121.11949092,503.03042985)(121.09449094,503.02542985)(121.06449465,503.02543963)
\curveto(121.04449099,503.03542984)(121.02449101,503.03542984)(121.00449465,503.02543963)
\curveto(120.95449108,503.01542986)(120.90949113,503.00542987)(120.86949465,502.99543963)
\curveto(120.82949121,502.99542988)(120.78949125,502.98542989)(120.74949465,502.96543963)
\curveto(120.56949147,502.88542999)(120.41949162,502.76543011)(120.29949465,502.60543963)
\curveto(120.18949185,502.44543043)(120.09949194,502.26543061)(120.02949465,502.06543963)
\curveto(119.96949207,501.875431)(119.92449211,501.65043123)(119.89449465,501.39043963)
\curveto(119.87449216,501.13043175)(119.86949217,500.86543201)(119.87949465,500.59543963)
\curveto(119.88949215,500.33543254)(119.91949212,500.08543279)(119.96949465,499.84543963)
\curveto(120.02949201,499.61543326)(120.10449193,499.42543345)(120.19449465,499.27543963)
\moveto(130.99449465,496.29043963)
\curveto(131.00448103,496.24043664)(131.00948103,496.15043673)(131.00949465,496.02043963)
\curveto(131.00948103,495.89043699)(130.99948104,495.80043708)(130.97949465,495.75043963)
\curveto(130.95948108,495.70043718)(130.95448108,495.64543723)(130.96449465,495.58543963)
\curveto(130.97448106,495.53543734)(130.97448106,495.48543739)(130.96449465,495.43543963)
\curveto(130.92448111,495.29543758)(130.89448114,495.16043772)(130.87449465,495.03043963)
\curveto(130.86448117,494.90043798)(130.8344812,494.7804381)(130.78449465,494.67043963)
\curveto(130.64448139,494.32043856)(130.47948156,494.02543885)(130.28949465,493.78543963)
\curveto(130.09948194,493.55543932)(129.82948221,493.37043951)(129.47949465,493.23043963)
\curveto(129.39948264,493.20043968)(129.31448272,493.1804397)(129.22449465,493.17043963)
\curveto(129.1344829,493.15043973)(129.04948299,493.13043975)(128.96949465,493.11043963)
\curveto(128.91948312,493.10043978)(128.86948317,493.09543978)(128.81949465,493.09543963)
\curveto(128.76948327,493.09543978)(128.71948332,493.09043979)(128.66949465,493.08043963)
\curveto(128.6394834,493.07043981)(128.58948345,493.07043981)(128.51949465,493.08043963)
\curveto(128.44948359,493.0804398)(128.39948364,493.08543979)(128.36949465,493.09543963)
\curveto(128.30948373,493.11543976)(128.24948379,493.12543975)(128.18949465,493.12543963)
\curveto(128.1394839,493.11543976)(128.08948395,493.12043976)(128.03949465,493.14043963)
\curveto(127.94948409,493.16043972)(127.85948418,493.18543969)(127.76949465,493.21543963)
\curveto(127.68948435,493.23543964)(127.60948443,493.26543961)(127.52949465,493.30543963)
\curveto(127.20948483,493.44543943)(126.95948508,493.64043924)(126.77949465,493.89043963)
\curveto(126.59948544,494.15043873)(126.44948559,494.45543842)(126.32949465,494.80543963)
\curveto(126.30948573,494.88543799)(126.29448574,494.97043791)(126.28449465,495.06043963)
\curveto(126.27448576,495.15043773)(126.25948578,495.23543764)(126.23949465,495.31543963)
\curveto(126.22948581,495.34543753)(126.22448581,495.3754375)(126.22449465,495.40543963)
\lineto(126.22449465,495.51043963)
\curveto(126.20448583,495.59043729)(126.19448584,495.67043721)(126.19449465,495.75043963)
\lineto(126.19449465,495.88543963)
\curveto(126.17448586,495.98543689)(126.17448586,496.08543679)(126.19449465,496.18543963)
\lineto(126.19449465,496.36543963)
\curveto(126.20448583,496.41543646)(126.20948583,496.46043642)(126.20949465,496.50043963)
\curveto(126.20948583,496.55043633)(126.21448582,496.59543628)(126.22449465,496.63543963)
\curveto(126.2344858,496.6754362)(126.2394858,496.71043617)(126.23949465,496.74043963)
\curveto(126.2394858,496.7804361)(126.24448579,496.82043606)(126.25449465,496.86043963)
\lineto(126.31449465,497.19043963)
\curveto(126.3344857,497.31043557)(126.36448567,497.42043546)(126.40449465,497.52043963)
\curveto(126.54448549,497.85043503)(126.70448533,498.12543475)(126.88449465,498.34543963)
\curveto(127.07448496,498.5754343)(127.3344847,498.76043412)(127.66449465,498.90043963)
\curveto(127.74448429,498.94043394)(127.82948421,498.96543391)(127.91949465,498.97543963)
\lineto(128.21949465,499.03543963)
\lineto(128.35449465,499.03543963)
\curveto(128.40448363,499.04543383)(128.45448358,499.05043383)(128.50449465,499.05043963)
\curveto(129.07448296,499.07043381)(129.5344825,498.96543391)(129.88449465,498.73543963)
\curveto(130.24448179,498.51543436)(130.50948153,498.21543466)(130.67949465,497.83543963)
\curveto(130.72948131,497.73543514)(130.76948127,497.63543524)(130.79949465,497.53543963)
\curveto(130.82948121,497.43543544)(130.85948118,497.33043555)(130.88949465,497.22043963)
\curveto(130.89948114,497.1804357)(130.90448113,497.14543573)(130.90449465,497.11543963)
\curveto(130.90448113,497.09543578)(130.90948113,497.06543581)(130.91949465,497.02543963)
\curveto(130.9394811,496.95543592)(130.94948109,496.880436)(130.94949465,496.80043963)
\curveto(130.94948109,496.72043616)(130.95948108,496.64043624)(130.97949465,496.56043963)
\curveto(130.97948106,496.51043637)(130.97948106,496.46543641)(130.97949465,496.42543963)
\curveto(130.97948106,496.38543649)(130.98448105,496.34043654)(130.99449465,496.29043963)
\moveto(129.88449465,495.85543963)
\curveto(129.89448214,495.90543697)(129.89948214,495.9804369)(129.89949465,496.08043963)
\curveto(129.90948213,496.1804367)(129.90448213,496.25543662)(129.88449465,496.30543963)
\curveto(129.86448217,496.36543651)(129.85948218,496.42043646)(129.86949465,496.47043963)
\curveto(129.88948215,496.53043635)(129.88948215,496.59043629)(129.86949465,496.65043963)
\curveto(129.85948218,496.6804362)(129.85448218,496.71543616)(129.85449465,496.75543963)
\curveto(129.85448218,496.79543608)(129.84948219,496.83543604)(129.83949465,496.87543963)
\curveto(129.81948222,496.95543592)(129.79948224,497.03043585)(129.77949465,497.10043963)
\curveto(129.76948227,497.1804357)(129.75448228,497.26043562)(129.73449465,497.34043963)
\curveto(129.70448233,497.40043548)(129.67948236,497.46043542)(129.65949465,497.52043963)
\curveto(129.6394824,497.5804353)(129.60948243,497.64043524)(129.56949465,497.70043963)
\curveto(129.46948257,497.87043501)(129.3394827,498.00543487)(129.17949465,498.10543963)
\curveto(129.09948294,498.15543472)(129.00448303,498.19043469)(128.89449465,498.21043963)
\curveto(128.78448325,498.23043465)(128.65948338,498.24043464)(128.51949465,498.24043963)
\curveto(128.49948354,498.23043465)(128.47448356,498.22543465)(128.44449465,498.22543963)
\curveto(128.41448362,498.23543464)(128.38448365,498.23543464)(128.35449465,498.22543963)
\lineto(128.20449465,498.16543963)
\curveto(128.15448388,498.15543472)(128.10948393,498.14043474)(128.06949465,498.12043963)
\curveto(127.87948416,498.01043487)(127.7344843,497.86543501)(127.63449465,497.68543963)
\curveto(127.54448449,497.50543537)(127.46448457,497.30043558)(127.39449465,497.07043963)
\curveto(127.35448468,496.94043594)(127.3344847,496.80543607)(127.33449465,496.66543963)
\curveto(127.3344847,496.53543634)(127.32448471,496.39043649)(127.30449465,496.23043963)
\curveto(127.29448474,496.1804367)(127.28448475,496.12043676)(127.27449465,496.05043963)
\curveto(127.27448476,495.9804369)(127.28448475,495.92043696)(127.30449465,495.87043963)
\lineto(127.30449465,495.70543963)
\lineto(127.30449465,495.52543963)
\curveto(127.31448472,495.4754374)(127.32448471,495.42043746)(127.33449465,495.36043963)
\curveto(127.34448469,495.31043757)(127.34948469,495.25543762)(127.34949465,495.19543963)
\curveto(127.35948468,495.13543774)(127.37448466,495.0804378)(127.39449465,495.03043963)
\curveto(127.44448459,494.84043804)(127.50448453,494.66543821)(127.57449465,494.50543963)
\curveto(127.64448439,494.34543853)(127.74948429,494.21543866)(127.88949465,494.11543963)
\curveto(128.01948402,494.01543886)(128.15948388,493.94543893)(128.30949465,493.90543963)
\curveto(128.3394837,493.89543898)(128.36448367,493.89043899)(128.38449465,493.89043963)
\curveto(128.41448362,493.90043898)(128.44448359,493.90043898)(128.47449465,493.89043963)
\curveto(128.49448354,493.89043899)(128.52448351,493.88543899)(128.56449465,493.87543963)
\curveto(128.60448343,493.875439)(128.6394834,493.880439)(128.66949465,493.89043963)
\curveto(128.70948333,493.90043898)(128.74948329,493.90543897)(128.78949465,493.90543963)
\curveto(128.82948321,493.90543897)(128.86948317,493.91543896)(128.90949465,493.93543963)
\curveto(129.14948289,494.01543886)(129.34448269,494.15043873)(129.49449465,494.34043963)
\curveto(129.61448242,494.52043836)(129.70448233,494.72543815)(129.76449465,494.95543963)
\curveto(129.78448225,495.02543785)(129.79948224,495.09543778)(129.80949465,495.16543963)
\curveto(129.81948222,495.24543763)(129.8344822,495.32543755)(129.85449465,495.40543963)
\curveto(129.85448218,495.46543741)(129.85948218,495.51043737)(129.86949465,495.54043963)
\curveto(129.86948217,495.56043732)(129.86948217,495.58543729)(129.86949465,495.61543963)
\curveto(129.86948217,495.65543722)(129.87448216,495.68543719)(129.88449465,495.70543963)
\lineto(129.88449465,495.85543963)
}
}
{
\newrgbcolor{curcolor}{0 0 0}
\pscustom[linestyle=none,fillstyle=solid,fillcolor=curcolor]
{
\newpath
\moveto(309.01768404,481.41796404)
\curveto(309.04767631,481.29795983)(309.07267629,481.15795997)(309.09268404,480.99796404)
\curveto(309.11267625,480.83796029)(309.12267624,480.67296045)(309.12268404,480.50296404)
\curveto(309.12267624,480.33296079)(309.11267625,480.16796096)(309.09268404,480.00796404)
\curveto(309.07267629,479.84796128)(309.04767631,479.70796142)(309.01768404,479.58796404)
\curveto(308.97767638,479.44796168)(308.94267642,479.3229618)(308.91268404,479.21296404)
\curveto(308.88267648,479.10296202)(308.84267652,478.99296213)(308.79268404,478.88296404)
\curveto(308.52267684,478.24296288)(308.10767725,477.75796337)(307.54768404,477.42796404)
\curveto(307.46767789,477.36796376)(307.38267798,477.31796381)(307.29268404,477.27796404)
\curveto(307.20267816,477.24796388)(307.10267826,477.21296391)(306.99268404,477.17296404)
\curveto(306.88267848,477.122964)(306.7626786,477.08796404)(306.63268404,477.06796404)
\curveto(306.51267885,477.03796409)(306.38267898,477.00796412)(306.24268404,476.97796404)
\curveto(306.18267918,476.95796417)(306.12267924,476.95296417)(306.06268404,476.96296404)
\curveto(306.01267935,476.97296415)(305.95267941,476.96796416)(305.88268404,476.94796404)
\curveto(305.8626795,476.93796419)(305.83767952,476.93796419)(305.80768404,476.94796404)
\curveto(305.77767958,476.94796418)(305.75267961,476.94296418)(305.73268404,476.93296404)
\lineto(305.58268404,476.93296404)
\curveto(305.51267985,476.9229642)(305.4626799,476.9229642)(305.43268404,476.93296404)
\curveto(305.39267997,476.94296418)(305.34768001,476.94796418)(305.29768404,476.94796404)
\curveto(305.2576801,476.93796419)(305.21768014,476.93796419)(305.17768404,476.94796404)
\curveto(305.08768027,476.96796416)(304.99768036,476.98296414)(304.90768404,476.99296404)
\curveto(304.81768054,476.99296413)(304.72768063,477.00296412)(304.63768404,477.02296404)
\curveto(304.54768081,477.05296407)(304.4576809,477.07796405)(304.36768404,477.09796404)
\curveto(304.27768108,477.11796401)(304.19268117,477.14796398)(304.11268404,477.18796404)
\curveto(303.87268149,477.29796383)(303.64768171,477.4279637)(303.43768404,477.57796404)
\curveto(303.22768213,477.73796339)(303.04768231,477.91796321)(302.89768404,478.11796404)
\curveto(302.77768258,478.28796284)(302.67268269,478.46296266)(302.58268404,478.64296404)
\curveto(302.49268287,478.8229623)(302.40268296,479.01296211)(302.31268404,479.21296404)
\curveto(302.27268309,479.31296181)(302.23768312,479.41296171)(302.20768404,479.51296404)
\curveto(302.18768317,479.6229615)(302.1626832,479.73296139)(302.13268404,479.84296404)
\curveto(302.09268327,479.98296114)(302.06768329,480.122961)(302.05768404,480.26296404)
\curveto(302.04768331,480.40296072)(302.02768333,480.54296058)(301.99768404,480.68296404)
\curveto(301.98768337,480.79296033)(301.97768338,480.89296023)(301.96768404,480.98296404)
\curveto(301.96768339,481.08296004)(301.9576834,481.18295994)(301.93768404,481.28296404)
\lineto(301.93768404,481.37296404)
\curveto(301.94768341,481.40295972)(301.94768341,481.4279597)(301.93768404,481.44796404)
\lineto(301.93768404,481.65796404)
\curveto(301.91768344,481.71795941)(301.90768345,481.78295934)(301.90768404,481.85296404)
\curveto(301.91768344,481.93295919)(301.92268344,482.00795912)(301.92268404,482.07796404)
\lineto(301.92268404,482.22796404)
\curveto(301.92268344,482.27795885)(301.92768343,482.3279588)(301.93768404,482.37796404)
\lineto(301.93768404,482.75296404)
\curveto(301.94768341,482.78295834)(301.94768341,482.81795831)(301.93768404,482.85796404)
\curveto(301.93768342,482.89795823)(301.94268342,482.93795819)(301.95268404,482.97796404)
\curveto(301.97268339,483.08795804)(301.98768337,483.19795793)(301.99768404,483.30796404)
\curveto(302.00768335,483.4279577)(302.01768334,483.54295758)(302.02768404,483.65296404)
\curveto(302.06768329,483.80295732)(302.09268327,483.94795718)(302.10268404,484.08796404)
\curveto(302.12268324,484.23795689)(302.15268321,484.38295674)(302.19268404,484.52296404)
\curveto(302.28268308,484.8229563)(302.37768298,485.10795602)(302.47768404,485.37796404)
\curveto(302.57768278,485.64795548)(302.70268266,485.89795523)(302.85268404,486.12796404)
\curveto(303.05268231,486.44795468)(303.29768206,486.7279544)(303.58768404,486.96796404)
\curveto(303.87768148,487.20795392)(304.21768114,487.39295373)(304.60768404,487.52296404)
\curveto(304.71768064,487.56295356)(304.82768053,487.58795354)(304.93768404,487.59796404)
\curveto(305.0576803,487.61795351)(305.17768018,487.64295348)(305.29768404,487.67296404)
\curveto(305.36767999,487.68295344)(305.43267993,487.68795344)(305.49268404,487.68796404)
\curveto(305.55267981,487.68795344)(305.61767974,487.69295343)(305.68768404,487.70296404)
\curveto(306.38767897,487.7229534)(306.9626784,487.60795352)(307.41268404,487.35796404)
\curveto(307.8626775,487.10795402)(308.20767715,486.75795437)(308.44768404,486.30796404)
\curveto(308.5576768,486.07795505)(308.6576767,485.80295532)(308.74768404,485.48296404)
\curveto(308.76767659,485.41295571)(308.76767659,485.33795579)(308.74768404,485.25796404)
\curveto(308.73767662,485.18795594)(308.71267665,485.13795599)(308.67268404,485.10796404)
\curveto(308.64267672,485.07795605)(308.58267678,485.05295607)(308.49268404,485.03296404)
\curveto(308.40267696,485.0229561)(308.30267706,485.01295611)(308.19268404,485.00296404)
\curveto(308.09267727,485.00295612)(307.99267737,485.00795612)(307.89268404,485.01796404)
\curveto(307.80267756,485.0279561)(307.73767762,485.04795608)(307.69768404,485.07796404)
\curveto(307.58767777,485.14795598)(307.50767785,485.25795587)(307.45768404,485.40796404)
\curveto(307.41767794,485.55795557)(307.362678,485.68795544)(307.29268404,485.79796404)
\curveto(307.10267826,486.10795502)(306.82267854,486.33795479)(306.45268404,486.48796404)
\curveto(306.38267898,486.51795461)(306.30767905,486.53795459)(306.22768404,486.54796404)
\curveto(306.1576792,486.55795457)(306.08267928,486.57295455)(306.00268404,486.59296404)
\curveto(305.95267941,486.60295452)(305.88267948,486.60795452)(305.79268404,486.60796404)
\curveto(305.71267965,486.60795452)(305.64767971,486.60295452)(305.59768404,486.59296404)
\curveto(305.5576798,486.57295455)(305.52267984,486.56795456)(305.49268404,486.57796404)
\curveto(305.4626799,486.58795454)(305.42767993,486.58795454)(305.38768404,486.57796404)
\lineto(305.14768404,486.51796404)
\curveto(305.07768028,486.49795463)(305.00768035,486.47295465)(304.93768404,486.44296404)
\curveto(304.5576808,486.28295484)(304.26768109,486.07295505)(304.06768404,485.81296404)
\curveto(303.87768148,485.55295557)(303.70268166,485.23795589)(303.54268404,484.86796404)
\curveto(303.51268185,484.78795634)(303.48768187,484.70795642)(303.46768404,484.62796404)
\curveto(303.4576819,484.54795658)(303.43768192,484.46795666)(303.40768404,484.38796404)
\curveto(303.37768198,484.27795685)(303.35268201,484.16295696)(303.33268404,484.04296404)
\curveto(303.32268204,483.9229572)(303.30268206,483.80295732)(303.27268404,483.68296404)
\curveto(303.25268211,483.63295749)(303.24268212,483.58295754)(303.24268404,483.53296404)
\curveto(303.25268211,483.48295764)(303.24768211,483.43295769)(303.22768404,483.38296404)
\curveto(303.21768214,483.3229578)(303.21768214,483.24295788)(303.22768404,483.14296404)
\curveto(303.23768212,483.05295807)(303.25268211,482.99795813)(303.27268404,482.97796404)
\curveto(303.29268207,482.93795819)(303.32268204,482.91795821)(303.36268404,482.91796404)
\curveto(303.41268195,482.91795821)(303.4576819,482.9279582)(303.49768404,482.94796404)
\curveto(303.56768179,482.98795814)(303.62768173,483.03295809)(303.67768404,483.08296404)
\curveto(303.72768163,483.13295799)(303.78768157,483.18295794)(303.85768404,483.23296404)
\lineto(303.91768404,483.29296404)
\curveto(303.94768141,483.3229578)(303.97768138,483.34795778)(304.00768404,483.36796404)
\curveto(304.23768112,483.5279576)(304.51268085,483.66295746)(304.83268404,483.77296404)
\curveto(304.90268046,483.79295733)(304.97268039,483.80795732)(305.04268404,483.81796404)
\curveto(305.11268025,483.8279573)(305.18768017,483.84295728)(305.26768404,483.86296404)
\curveto(305.30768005,483.86295726)(305.34268002,483.86795726)(305.37268404,483.87796404)
\curveto(305.40267996,483.88795724)(305.43767992,483.88795724)(305.47768404,483.87796404)
\curveto(305.52767983,483.87795725)(305.56767979,483.88795724)(305.59768404,483.90796404)
\lineto(305.76268404,483.90796404)
\lineto(305.85268404,483.90796404)
\curveto(305.90267946,483.91795721)(305.94267942,483.91795721)(305.97268404,483.90796404)
\curveto(306.02267934,483.89795723)(306.07267929,483.89295723)(306.12268404,483.89296404)
\curveto(306.18267918,483.90295722)(306.23767912,483.90295722)(306.28768404,483.89296404)
\curveto(306.39767896,483.86295726)(306.50267886,483.84295728)(306.60268404,483.83296404)
\curveto(306.71267865,483.8229573)(306.81767854,483.79795733)(306.91768404,483.75796404)
\curveto(307.33767802,483.61795751)(307.68267768,483.43295769)(307.95268404,483.20296404)
\curveto(308.22267714,482.98295814)(308.4626769,482.69795843)(308.67268404,482.34796404)
\curveto(308.75267661,482.20795892)(308.81767654,482.05795907)(308.86768404,481.89796404)
\curveto(308.91767644,481.74795938)(308.96767639,481.58795954)(309.01768404,481.41796404)
\moveto(307.77268404,480.11296404)
\curveto(307.78267758,480.16296096)(307.78767757,480.20796092)(307.78768404,480.24796404)
\lineto(307.78768404,480.39796404)
\curveto(307.78767757,480.70796042)(307.74767761,480.99296013)(307.66768404,481.25296404)
\curveto(307.64767771,481.31295981)(307.62767773,481.36795976)(307.60768404,481.41796404)
\curveto(307.59767776,481.47795965)(307.58267778,481.53295959)(307.56268404,481.58296404)
\curveto(307.34267802,482.07295905)(306.99767836,482.4229587)(306.52768404,482.63296404)
\curveto(306.44767891,482.66295846)(306.36767899,482.68795844)(306.28768404,482.70796404)
\lineto(306.04768404,482.76796404)
\curveto(305.96767939,482.78795834)(305.87767948,482.79795833)(305.77768404,482.79796404)
\lineto(305.46268404,482.79796404)
\curveto(305.44267992,482.77795835)(305.40267996,482.76795836)(305.34268404,482.76796404)
\curveto(305.29268007,482.77795835)(305.24768011,482.77795835)(305.20768404,482.76796404)
\lineto(304.96768404,482.70796404)
\curveto(304.89768046,482.69795843)(304.82768053,482.67795845)(304.75768404,482.64796404)
\curveto(304.1576812,482.38795874)(303.75268161,481.9229592)(303.54268404,481.25296404)
\curveto(303.51268185,481.17295995)(303.49268187,481.09296003)(303.48268404,481.01296404)
\curveto(303.47268189,480.93296019)(303.4576819,480.84796028)(303.43768404,480.75796404)
\lineto(303.43768404,480.60796404)
\curveto(303.42768193,480.56796056)(303.42268194,480.49796063)(303.42268404,480.39796404)
\curveto(303.42268194,480.16796096)(303.44268192,479.97296115)(303.48268404,479.81296404)
\curveto(303.50268186,479.74296138)(303.51768184,479.67796145)(303.52768404,479.61796404)
\curveto(303.53768182,479.55796157)(303.5576818,479.49296163)(303.58768404,479.42296404)
\curveto(303.69768166,479.14296198)(303.84268152,478.89796223)(304.02268404,478.68796404)
\curveto(304.20268116,478.48796264)(304.43768092,478.3279628)(304.72768404,478.20796404)
\lineto(304.96768404,478.11796404)
\lineto(305.20768404,478.05796404)
\curveto(305.2576801,478.03796309)(305.29768006,478.03296309)(305.32768404,478.04296404)
\curveto(305.36767999,478.05296307)(305.41267995,478.04796308)(305.46268404,478.02796404)
\curveto(305.49267987,478.01796311)(305.54767981,478.01296311)(305.62768404,478.01296404)
\curveto(305.70767965,478.01296311)(305.76767959,478.01796311)(305.80768404,478.02796404)
\curveto(305.91767944,478.04796308)(306.02267934,478.06296306)(306.12268404,478.07296404)
\curveto(306.22267914,478.08296304)(306.31767904,478.11296301)(306.40768404,478.16296404)
\curveto(306.93767842,478.36296276)(307.32767803,478.73796239)(307.57768404,479.28796404)
\curveto(307.61767774,479.38796174)(307.64767771,479.49296163)(307.66768404,479.60296404)
\lineto(307.75768404,479.93296404)
\curveto(307.7576776,480.01296111)(307.7626776,480.07296105)(307.77268404,480.11296404)
}
}
{
\newrgbcolor{curcolor}{0 0 0}
\pscustom[linestyle=none,fillstyle=solid,fillcolor=curcolor]
{
\newpath
\moveto(314.42729341,487.70296404)
\curveto(314.52728856,487.70295342)(314.62228846,487.69295343)(314.71229341,487.67296404)
\curveto(314.80228828,487.66295346)(314.86728822,487.63295349)(314.90729341,487.58296404)
\curveto(314.96728812,487.50295362)(314.99728809,487.39795373)(314.99729341,487.26796404)
\lineto(314.99729341,486.87796404)
\lineto(314.99729341,485.37796404)
\lineto(314.99729341,478.98796404)
\lineto(314.99729341,477.81796404)
\lineto(314.99729341,477.50296404)
\curveto(315.00728808,477.40296372)(314.99228809,477.3229638)(314.95229341,477.26296404)
\curveto(314.90228818,477.18296394)(314.82728826,477.13296399)(314.72729341,477.11296404)
\curveto(314.63728845,477.10296402)(314.52728856,477.09796403)(314.39729341,477.09796404)
\lineto(314.17229341,477.09796404)
\curveto(314.09228899,477.11796401)(314.02228906,477.13296399)(313.96229341,477.14296404)
\curveto(313.90228918,477.16296396)(313.85228923,477.20296392)(313.81229341,477.26296404)
\curveto(313.77228931,477.3229638)(313.75228933,477.39796373)(313.75229341,477.48796404)
\lineto(313.75229341,477.78796404)
\lineto(313.75229341,478.88296404)
\lineto(313.75229341,484.22296404)
\curveto(313.73228935,484.31295681)(313.71728937,484.38795674)(313.70729341,484.44796404)
\curveto(313.70728938,484.51795661)(313.67728941,484.57795655)(313.61729341,484.62796404)
\curveto(313.54728954,484.67795645)(313.45728963,484.70295642)(313.34729341,484.70296404)
\curveto(313.24728984,484.71295641)(313.13728995,484.71795641)(313.01729341,484.71796404)
\lineto(311.87729341,484.71796404)
\lineto(311.38229341,484.71796404)
\curveto(311.22229186,484.7279564)(311.11229197,484.78795634)(311.05229341,484.89796404)
\curveto(311.03229205,484.9279562)(311.02229206,484.95795617)(311.02229341,484.98796404)
\curveto(311.02229206,485.0279561)(311.01729207,485.07295605)(311.00729341,485.12296404)
\curveto(310.9872921,485.24295588)(310.99229209,485.35295577)(311.02229341,485.45296404)
\curveto(311.06229202,485.55295557)(311.11729197,485.6229555)(311.18729341,485.66296404)
\curveto(311.26729182,485.71295541)(311.3872917,485.73795539)(311.54729341,485.73796404)
\curveto(311.70729138,485.73795539)(311.84229124,485.75295537)(311.95229341,485.78296404)
\curveto(312.00229108,485.79295533)(312.05729103,485.79795533)(312.11729341,485.79796404)
\curveto(312.17729091,485.80795532)(312.23729085,485.8229553)(312.29729341,485.84296404)
\curveto(312.44729064,485.89295523)(312.59229049,485.94295518)(312.73229341,485.99296404)
\curveto(312.87229021,486.05295507)(313.00729008,486.122955)(313.13729341,486.20296404)
\curveto(313.27728981,486.29295483)(313.39728969,486.39795473)(313.49729341,486.51796404)
\curveto(313.59728949,486.63795449)(313.69228939,486.76795436)(313.78229341,486.90796404)
\curveto(313.84228924,487.00795412)(313.8872892,487.11795401)(313.91729341,487.23796404)
\curveto(313.95728913,487.35795377)(314.00728908,487.46295366)(314.06729341,487.55296404)
\curveto(314.11728897,487.61295351)(314.1872889,487.65295347)(314.27729341,487.67296404)
\curveto(314.29728879,487.68295344)(314.32228876,487.68795344)(314.35229341,487.68796404)
\curveto(314.3822887,487.68795344)(314.40728868,487.69295343)(314.42729341,487.70296404)
}
}
{
\newrgbcolor{curcolor}{0 0 0}
\pscustom[linestyle=none,fillstyle=solid,fillcolor=curcolor]
{
\newpath
\moveto(319.67190279,478.73296404)
\lineto(319.97190279,478.73296404)
\curveto(320.08190073,478.74296238)(320.18690062,478.74296238)(320.28690279,478.73296404)
\curveto(320.39690041,478.73296239)(320.49690031,478.7229624)(320.58690279,478.70296404)
\curveto(320.67690013,478.69296243)(320.74690006,478.66796246)(320.79690279,478.62796404)
\curveto(320.81689999,478.60796252)(320.83189998,478.57796255)(320.84190279,478.53796404)
\curveto(320.86189995,478.49796263)(320.88189993,478.45296267)(320.90190279,478.40296404)
\lineto(320.90190279,478.32796404)
\curveto(320.9118999,478.27796285)(320.9118999,478.2229629)(320.90190279,478.16296404)
\lineto(320.90190279,478.01296404)
\lineto(320.90190279,477.53296404)
\curveto(320.90189991,477.36296376)(320.86189995,477.24296388)(320.78190279,477.17296404)
\curveto(320.7119001,477.122964)(320.62190019,477.09796403)(320.51190279,477.09796404)
\lineto(320.18190279,477.09796404)
\lineto(319.73190279,477.09796404)
\curveto(319.58190123,477.09796403)(319.46690134,477.127964)(319.38690279,477.18796404)
\curveto(319.34690146,477.21796391)(319.31690149,477.26796386)(319.29690279,477.33796404)
\curveto(319.27690153,477.41796371)(319.26190155,477.50296362)(319.25190279,477.59296404)
\lineto(319.25190279,477.87796404)
\curveto(319.26190155,477.97796315)(319.26690154,478.06296306)(319.26690279,478.13296404)
\lineto(319.26690279,478.32796404)
\curveto(319.26690154,478.38796274)(319.27690153,478.44296268)(319.29690279,478.49296404)
\curveto(319.33690147,478.60296252)(319.4069014,478.67296245)(319.50690279,478.70296404)
\curveto(319.53690127,478.70296242)(319.59190122,478.71296241)(319.67190279,478.73296404)
}
}
{
\newrgbcolor{curcolor}{0 0 0}
\pscustom[linestyle=none,fillstyle=solid,fillcolor=curcolor]
{
\newpath
\moveto(329.89205904,480.17296404)
\curveto(329.90205132,480.13296099)(329.90205132,480.08296104)(329.89205904,480.02296404)
\curveto(329.89205133,479.96296116)(329.88705133,479.91296121)(329.87705904,479.87296404)
\curveto(329.87705134,479.83296129)(329.87205135,479.79296133)(329.86205904,479.75296404)
\lineto(329.86205904,479.64796404)
\curveto(329.84205138,479.56796156)(329.82705139,479.48796164)(329.81705904,479.40796404)
\curveto(329.80705141,479.3279618)(329.78705143,479.25296187)(329.75705904,479.18296404)
\curveto(329.73705148,479.10296202)(329.7170515,479.0279621)(329.69705904,478.95796404)
\curveto(329.67705154,478.88796224)(329.64705157,478.81296231)(329.60705904,478.73296404)
\curveto(329.42705179,478.31296281)(329.17205205,477.97296315)(328.84205904,477.71296404)
\curveto(328.51205271,477.45296367)(328.1220531,477.24796388)(327.67205904,477.09796404)
\curveto(327.55205367,477.05796407)(327.42705379,477.03296409)(327.29705904,477.02296404)
\curveto(327.17705404,477.00296412)(327.05205417,476.97796415)(326.92205904,476.94796404)
\curveto(326.86205436,476.93796419)(326.79705442,476.93296419)(326.72705904,476.93296404)
\curveto(326.66705455,476.93296419)(326.60205462,476.9279642)(326.53205904,476.91796404)
\lineto(326.41205904,476.91796404)
\lineto(326.21705904,476.91796404)
\curveto(326.15705506,476.90796422)(326.10205512,476.91296421)(326.05205904,476.93296404)
\curveto(325.98205524,476.95296417)(325.9170553,476.95796417)(325.85705904,476.94796404)
\curveto(325.79705542,476.93796419)(325.73705548,476.94296418)(325.67705904,476.96296404)
\curveto(325.62705559,476.97296415)(325.58205564,476.97796415)(325.54205904,476.97796404)
\curveto(325.50205572,476.97796415)(325.45705576,476.98796414)(325.40705904,477.00796404)
\curveto(325.32705589,477.0279641)(325.25205597,477.04796408)(325.18205904,477.06796404)
\curveto(325.11205611,477.07796405)(325.04205618,477.09296403)(324.97205904,477.11296404)
\curveto(324.49205673,477.28296384)(324.09205713,477.49296363)(323.77205904,477.74296404)
\curveto(323.46205776,478.00296312)(323.21205801,478.35796277)(323.02205904,478.80796404)
\curveto(322.99205823,478.86796226)(322.96705825,478.9279622)(322.94705904,478.98796404)
\curveto(322.93705828,479.05796207)(322.9220583,479.13296199)(322.90205904,479.21296404)
\curveto(322.88205834,479.27296185)(322.86705835,479.33796179)(322.85705904,479.40796404)
\curveto(322.84705837,479.47796165)(322.83205839,479.54796158)(322.81205904,479.61796404)
\curveto(322.80205842,479.66796146)(322.79705842,479.70796142)(322.79705904,479.73796404)
\lineto(322.79705904,479.85796404)
\curveto(322.78705843,479.89796123)(322.77705844,479.94796118)(322.76705904,480.00796404)
\curveto(322.76705845,480.06796106)(322.77205845,480.11796101)(322.78205904,480.15796404)
\lineto(322.78205904,480.29296404)
\curveto(322.79205843,480.34296078)(322.79705842,480.39296073)(322.79705904,480.44296404)
\curveto(322.8170584,480.54296058)(322.83205839,480.63796049)(322.84205904,480.72796404)
\curveto(322.85205837,480.8279603)(322.87205835,480.9229602)(322.90205904,481.01296404)
\curveto(322.95205827,481.16295996)(323.00705821,481.30295982)(323.06705904,481.43296404)
\curveto(323.12705809,481.56295956)(323.19705802,481.68295944)(323.27705904,481.79296404)
\curveto(323.30705791,481.84295928)(323.33705788,481.88295924)(323.36705904,481.91296404)
\curveto(323.40705781,481.94295918)(323.44205778,481.97795915)(323.47205904,482.01796404)
\curveto(323.53205769,482.09795903)(323.60205762,482.16795896)(323.68205904,482.22796404)
\curveto(323.74205748,482.27795885)(323.80205742,482.3229588)(323.86205904,482.36296404)
\lineto(324.07205904,482.51296404)
\curveto(324.1220571,482.55295857)(324.17205705,482.58795854)(324.22205904,482.61796404)
\curveto(324.27205695,482.65795847)(324.30705691,482.71295841)(324.32705904,482.78296404)
\curveto(324.32705689,482.81295831)(324.3170569,482.83795829)(324.29705904,482.85796404)
\curveto(324.28705693,482.88795824)(324.27705694,482.91295821)(324.26705904,482.93296404)
\curveto(324.22705699,482.98295814)(324.17705704,483.0279581)(324.11705904,483.06796404)
\curveto(324.06705715,483.11795801)(324.0170572,483.16295796)(323.96705904,483.20296404)
\curveto(323.92705729,483.23295789)(323.87705734,483.28795784)(323.81705904,483.36796404)
\curveto(323.79705742,483.39795773)(323.76705745,483.4229577)(323.72705904,483.44296404)
\curveto(323.69705752,483.47295765)(323.67205755,483.50795762)(323.65205904,483.54796404)
\curveto(323.48205774,483.75795737)(323.35205787,484.00295712)(323.26205904,484.28296404)
\curveto(323.24205798,484.36295676)(323.22705799,484.44295668)(323.21705904,484.52296404)
\curveto(323.20705801,484.60295652)(323.19205803,484.68295644)(323.17205904,484.76296404)
\curveto(323.15205807,484.81295631)(323.14205808,484.87795625)(323.14205904,484.95796404)
\curveto(323.14205808,485.04795608)(323.15205807,485.11795601)(323.17205904,485.16796404)
\curveto(323.17205805,485.26795586)(323.17705804,485.33795579)(323.18705904,485.37796404)
\curveto(323.20705801,485.45795567)(323.222058,485.5279556)(323.23205904,485.58796404)
\curveto(323.24205798,485.65795547)(323.25705796,485.7279554)(323.27705904,485.79796404)
\curveto(323.42705779,486.2279549)(323.64205758,486.57295455)(323.92205904,486.83296404)
\curveto(324.21205701,487.09295403)(324.56205666,487.30795382)(324.97205904,487.47796404)
\curveto(325.08205614,487.5279536)(325.19705602,487.55795357)(325.31705904,487.56796404)
\curveto(325.44705577,487.58795354)(325.57705564,487.61795351)(325.70705904,487.65796404)
\curveto(325.78705543,487.65795347)(325.85705536,487.65795347)(325.91705904,487.65796404)
\curveto(325.98705523,487.66795346)(326.06205516,487.67795345)(326.14205904,487.68796404)
\curveto(326.93205429,487.70795342)(327.58705363,487.57795355)(328.10705904,487.29796404)
\curveto(328.63705258,487.01795411)(329.0170522,486.60795452)(329.24705904,486.06796404)
\curveto(329.35705186,485.83795529)(329.42705179,485.55295557)(329.45705904,485.21296404)
\curveto(329.49705172,484.88295624)(329.46705175,484.57795655)(329.36705904,484.29796404)
\curveto(329.32705189,484.16795696)(329.27705194,484.04795708)(329.21705904,483.93796404)
\curveto(329.16705205,483.8279573)(329.10705211,483.7229574)(329.03705904,483.62296404)
\curveto(329.0170522,483.58295754)(328.98705223,483.54795758)(328.94705904,483.51796404)
\lineto(328.85705904,483.42796404)
\curveto(328.80705241,483.33795779)(328.74705247,483.27295785)(328.67705904,483.23296404)
\curveto(328.62705259,483.18295794)(328.57205265,483.13295799)(328.51205904,483.08296404)
\curveto(328.46205276,483.04295808)(328.4170528,482.99795813)(328.37705904,482.94796404)
\curveto(328.35705286,482.9279582)(328.33705288,482.90295822)(328.31705904,482.87296404)
\curveto(328.30705291,482.85295827)(328.30705291,482.8279583)(328.31705904,482.79796404)
\curveto(328.32705289,482.74795838)(328.35705286,482.69795843)(328.40705904,482.64796404)
\curveto(328.45705276,482.60795852)(328.51205271,482.56795856)(328.57205904,482.52796404)
\lineto(328.75205904,482.40796404)
\curveto(328.81205241,482.37795875)(328.86205236,482.34795878)(328.90205904,482.31796404)
\curveto(329.23205199,482.07795905)(329.48205174,481.76795936)(329.65205904,481.38796404)
\curveto(329.69205153,481.30795982)(329.7220515,481.2229599)(329.74205904,481.13296404)
\curveto(329.77205145,481.04296008)(329.79705142,480.95296017)(329.81705904,480.86296404)
\curveto(329.82705139,480.81296031)(329.83705138,480.75796037)(329.84705904,480.69796404)
\lineto(329.87705904,480.54796404)
\curveto(329.88705133,480.48796064)(329.88705133,480.4229607)(329.87705904,480.35296404)
\curveto(329.86705135,480.29296083)(329.87205135,480.23296089)(329.89205904,480.17296404)
\moveto(324.50705904,485.21296404)
\curveto(324.47705674,485.10295602)(324.47205675,484.96295616)(324.49205904,484.79296404)
\curveto(324.51205671,484.63295649)(324.53705668,484.50795662)(324.56705904,484.41796404)
\curveto(324.67705654,484.09795703)(324.82705639,483.85295727)(325.01705904,483.68296404)
\curveto(325.20705601,483.5229576)(325.47205575,483.39295773)(325.81205904,483.29296404)
\curveto(325.94205528,483.26295786)(326.10705511,483.23795789)(326.30705904,483.21796404)
\curveto(326.50705471,483.20795792)(326.67705454,483.2229579)(326.81705904,483.26296404)
\curveto(327.10705411,483.34295778)(327.34705387,483.45295767)(327.53705904,483.59296404)
\curveto(327.73705348,483.74295738)(327.89205333,483.94295718)(328.00205904,484.19296404)
\curveto(328.0220532,484.24295688)(328.03205319,484.28795684)(328.03205904,484.32796404)
\curveto(328.04205318,484.36795676)(328.05705316,484.41295671)(328.07705904,484.46296404)
\curveto(328.10705311,484.57295655)(328.12705309,484.71295641)(328.13705904,484.88296404)
\curveto(328.14705307,485.05295607)(328.13705308,485.19795593)(328.10705904,485.31796404)
\curveto(328.08705313,485.40795572)(328.06205316,485.49295563)(328.03205904,485.57296404)
\curveto(328.01205321,485.65295547)(327.97705324,485.73295539)(327.92705904,485.81296404)
\curveto(327.75705346,486.08295504)(327.53205369,486.27795485)(327.25205904,486.39796404)
\curveto(326.98205424,486.51795461)(326.6220546,486.57795455)(326.17205904,486.57796404)
\curveto(326.15205507,486.55795457)(326.1220551,486.55295457)(326.08205904,486.56296404)
\curveto(326.04205518,486.57295455)(326.00705521,486.57295455)(325.97705904,486.56296404)
\curveto(325.92705529,486.54295458)(325.87205535,486.5279546)(325.81205904,486.51796404)
\curveto(325.76205546,486.51795461)(325.71205551,486.50795462)(325.66205904,486.48796404)
\curveto(325.4220558,486.39795473)(325.21205601,486.28295484)(325.03205904,486.14296404)
\curveto(324.85205637,486.01295511)(324.71205651,485.83295529)(324.61205904,485.60296404)
\curveto(324.59205663,485.54295558)(324.57205665,485.47795565)(324.55205904,485.40796404)
\curveto(324.54205668,485.34795578)(324.52705669,485.28295584)(324.50705904,485.21296404)
\moveto(328.52705904,479.67796404)
\curveto(328.57705264,479.86796126)(328.58205264,480.07296105)(328.54205904,480.29296404)
\curveto(328.51205271,480.51296061)(328.46705275,480.69296043)(328.40705904,480.83296404)
\curveto(328.23705298,481.20295992)(327.97705324,481.50795962)(327.62705904,481.74796404)
\curveto(327.28705393,481.98795914)(326.85205437,482.10795902)(326.32205904,482.10796404)
\curveto(326.29205493,482.08795904)(326.25205497,482.08295904)(326.20205904,482.09296404)
\curveto(326.15205507,482.11295901)(326.11205511,482.11795901)(326.08205904,482.10796404)
\lineto(325.81205904,482.04796404)
\curveto(325.73205549,482.03795909)(325.65205557,482.0229591)(325.57205904,482.00296404)
\curveto(325.27205595,481.89295923)(325.00705621,481.74795938)(324.77705904,481.56796404)
\curveto(324.55705666,481.38795974)(324.38705683,481.15795997)(324.26705904,480.87796404)
\curveto(324.23705698,480.79796033)(324.21205701,480.71796041)(324.19205904,480.63796404)
\curveto(324.17205705,480.55796057)(324.15205707,480.47296065)(324.13205904,480.38296404)
\curveto(324.10205712,480.26296086)(324.09205713,480.11296101)(324.10205904,479.93296404)
\curveto(324.1220571,479.75296137)(324.14705707,479.61296151)(324.17705904,479.51296404)
\curveto(324.19705702,479.46296166)(324.20705701,479.41796171)(324.20705904,479.37796404)
\curveto(324.217057,479.34796178)(324.23205699,479.30796182)(324.25205904,479.25796404)
\curveto(324.35205687,479.03796209)(324.48205674,478.83796229)(324.64205904,478.65796404)
\curveto(324.81205641,478.47796265)(325.00705621,478.34296278)(325.22705904,478.25296404)
\curveto(325.29705592,478.21296291)(325.39205583,478.17796295)(325.51205904,478.14796404)
\curveto(325.73205549,478.05796307)(325.98705523,478.01296311)(326.27705904,478.01296404)
\lineto(326.56205904,478.01296404)
\curveto(326.66205456,478.03296309)(326.75705446,478.04796308)(326.84705904,478.05796404)
\curveto(326.93705428,478.06796306)(327.02705419,478.08796304)(327.11705904,478.11796404)
\curveto(327.37705384,478.19796293)(327.6170536,478.3279628)(327.83705904,478.50796404)
\curveto(328.06705315,478.69796243)(328.23705298,478.91296221)(328.34705904,479.15296404)
\curveto(328.38705283,479.23296189)(328.4170528,479.31296181)(328.43705904,479.39296404)
\curveto(328.46705275,479.48296164)(328.49705272,479.57796155)(328.52705904,479.67796404)
}
}
{
\newrgbcolor{curcolor}{0 0 0}
\pscustom[linestyle=none,fillstyle=solid,fillcolor=curcolor]
{
\newpath
\moveto(341.03166841,485.61796404)
\curveto(340.83165811,485.3279558)(340.62165832,485.04295608)(340.40166841,484.76296404)
\curveto(340.19165875,484.48295664)(339.98665896,484.19795693)(339.78666841,483.90796404)
\curveto(339.18665976,483.05795807)(338.58166036,482.21795891)(337.97166841,481.38796404)
\curveto(337.36166158,480.56796056)(336.75666219,479.73296139)(336.15666841,478.88296404)
\lineto(335.64666841,478.16296404)
\lineto(335.13666841,477.47296404)
\curveto(335.05666389,477.36296376)(334.97666397,477.24796388)(334.89666841,477.12796404)
\curveto(334.81666413,477.00796412)(334.72166422,476.91296421)(334.61166841,476.84296404)
\curveto(334.57166437,476.8229643)(334.50666444,476.80796432)(334.41666841,476.79796404)
\curveto(334.33666461,476.77796435)(334.2466647,476.76796436)(334.14666841,476.76796404)
\curveto(334.0466649,476.76796436)(333.95166499,476.77296435)(333.86166841,476.78296404)
\curveto(333.78166516,476.79296433)(333.72166522,476.81296431)(333.68166841,476.84296404)
\curveto(333.65166529,476.86296426)(333.62666532,476.89796423)(333.60666841,476.94796404)
\curveto(333.59666535,476.98796414)(333.60166534,477.03296409)(333.62166841,477.08296404)
\curveto(333.66166528,477.16296396)(333.70666524,477.23796389)(333.75666841,477.30796404)
\curveto(333.81666513,477.38796374)(333.87166507,477.46796366)(333.92166841,477.54796404)
\curveto(334.16166478,477.88796324)(334.40666454,478.2229629)(334.65666841,478.55296404)
\curveto(334.90666404,478.88296224)(335.1466638,479.21796191)(335.37666841,479.55796404)
\curveto(335.53666341,479.77796135)(335.69666325,479.99296113)(335.85666841,480.20296404)
\curveto(336.01666293,480.41296071)(336.17666277,480.6279605)(336.33666841,480.84796404)
\curveto(336.69666225,481.36795976)(337.06166188,481.87795925)(337.43166841,482.37796404)
\curveto(337.80166114,482.87795825)(338.17166077,483.38795774)(338.54166841,483.90796404)
\curveto(338.68166026,484.10795702)(338.82166012,484.30295682)(338.96166841,484.49296404)
\curveto(339.11165983,484.68295644)(339.25665969,484.87795625)(339.39666841,485.07796404)
\curveto(339.60665934,485.37795575)(339.82165912,485.67795545)(340.04166841,485.97796404)
\lineto(340.70166841,486.87796404)
\lineto(340.88166841,487.14796404)
\lineto(341.09166841,487.41796404)
\lineto(341.21166841,487.59796404)
\curveto(341.26165768,487.65795347)(341.31165763,487.71295341)(341.36166841,487.76296404)
\curveto(341.43165751,487.81295331)(341.50665744,487.84795328)(341.58666841,487.86796404)
\curveto(341.60665734,487.87795325)(341.63165731,487.87795325)(341.66166841,487.86796404)
\curveto(341.70165724,487.86795326)(341.73165721,487.87795325)(341.75166841,487.89796404)
\curveto(341.87165707,487.89795323)(342.00665694,487.89295323)(342.15666841,487.88296404)
\curveto(342.30665664,487.88295324)(342.39665655,487.83795329)(342.42666841,487.74796404)
\curveto(342.4466565,487.71795341)(342.45165649,487.68295344)(342.44166841,487.64296404)
\curveto(342.43165651,487.60295352)(342.41665653,487.57295355)(342.39666841,487.55296404)
\curveto(342.35665659,487.47295365)(342.31665663,487.40295372)(342.27666841,487.34296404)
\curveto(342.23665671,487.28295384)(342.19165675,487.2229539)(342.14166841,487.16296404)
\lineto(341.57166841,486.38296404)
\curveto(341.39165755,486.13295499)(341.21165773,485.87795525)(341.03166841,485.61796404)
\moveto(334.17666841,481.71796404)
\curveto(334.12666482,481.73795939)(334.07666487,481.74295938)(334.02666841,481.73296404)
\curveto(333.97666497,481.7229594)(333.92666502,481.7279594)(333.87666841,481.74796404)
\curveto(333.76666518,481.76795936)(333.66166528,481.78795934)(333.56166841,481.80796404)
\curveto(333.47166547,481.83795929)(333.37666557,481.87795925)(333.27666841,481.92796404)
\curveto(332.946666,482.06795906)(332.69166625,482.26295886)(332.51166841,482.51296404)
\curveto(332.33166661,482.77295835)(332.18666676,483.08295804)(332.07666841,483.44296404)
\curveto(332.0466669,483.5229576)(332.02666692,483.60295752)(332.01666841,483.68296404)
\curveto(332.00666694,483.77295735)(331.99166695,483.85795727)(331.97166841,483.93796404)
\curveto(331.96166698,483.98795714)(331.95666699,484.05295707)(331.95666841,484.13296404)
\curveto(331.946667,484.16295696)(331.941667,484.19295693)(331.94166841,484.22296404)
\curveto(331.941667,484.26295686)(331.93666701,484.29795683)(331.92666841,484.32796404)
\lineto(331.92666841,484.47796404)
\curveto(331.91666703,484.5279566)(331.91166703,484.58795654)(331.91166841,484.65796404)
\curveto(331.91166703,484.73795639)(331.91666703,484.80295632)(331.92666841,484.85296404)
\lineto(331.92666841,485.01796404)
\curveto(331.946667,485.06795606)(331.95166699,485.11295601)(331.94166841,485.15296404)
\curveto(331.941667,485.20295592)(331.946667,485.24795588)(331.95666841,485.28796404)
\curveto(331.96666698,485.3279558)(331.97166697,485.36295576)(331.97166841,485.39296404)
\curveto(331.97166697,485.43295569)(331.97666697,485.47295565)(331.98666841,485.51296404)
\curveto(332.01666693,485.6229555)(332.03666691,485.73295539)(332.04666841,485.84296404)
\curveto(332.06666688,485.96295516)(332.10166684,486.07795505)(332.15166841,486.18796404)
\curveto(332.29166665,486.5279546)(332.45166649,486.80295432)(332.63166841,487.01296404)
\curveto(332.82166612,487.23295389)(333.09166585,487.41295371)(333.44166841,487.55296404)
\curveto(333.52166542,487.58295354)(333.60666534,487.60295352)(333.69666841,487.61296404)
\curveto(333.78666516,487.63295349)(333.88166506,487.65295347)(333.98166841,487.67296404)
\curveto(334.01166493,487.68295344)(334.06666488,487.68295344)(334.14666841,487.67296404)
\curveto(334.22666472,487.67295345)(334.27666467,487.68295344)(334.29666841,487.70296404)
\curveto(334.85666409,487.71295341)(335.30666364,487.60295352)(335.64666841,487.37296404)
\curveto(335.99666295,487.14295398)(336.25666269,486.83795429)(336.42666841,486.45796404)
\curveto(336.46666248,486.36795476)(336.50166244,486.27295485)(336.53166841,486.17296404)
\curveto(336.56166238,486.07295505)(336.58666236,485.97295515)(336.60666841,485.87296404)
\curveto(336.62666232,485.84295528)(336.63166231,485.81295531)(336.62166841,485.78296404)
\curveto(336.62166232,485.75295537)(336.62666232,485.7229554)(336.63666841,485.69296404)
\curveto(336.66666228,485.58295554)(336.68666226,485.45795567)(336.69666841,485.31796404)
\curveto(336.70666224,485.18795594)(336.71666223,485.05295607)(336.72666841,484.91296404)
\lineto(336.72666841,484.74796404)
\curveto(336.73666221,484.68795644)(336.73666221,484.63295649)(336.72666841,484.58296404)
\curveto(336.71666223,484.53295659)(336.71166223,484.48295664)(336.71166841,484.43296404)
\lineto(336.71166841,484.29796404)
\curveto(336.70166224,484.25795687)(336.69666225,484.21795691)(336.69666841,484.17796404)
\curveto(336.70666224,484.13795699)(336.70166224,484.09295703)(336.68166841,484.04296404)
\curveto(336.66166228,483.93295719)(336.6416623,483.8279573)(336.62166841,483.72796404)
\curveto(336.61166233,483.6279575)(336.59166235,483.5279576)(336.56166841,483.42796404)
\curveto(336.43166251,483.06795806)(336.26666268,482.75295837)(336.06666841,482.48296404)
\curveto(335.86666308,482.21295891)(335.59166335,482.00795912)(335.24166841,481.86796404)
\curveto(335.16166378,481.83795929)(335.07666387,481.81295931)(334.98666841,481.79296404)
\lineto(334.71666841,481.73296404)
\curveto(334.66666428,481.7229594)(334.62166432,481.71795941)(334.58166841,481.71796404)
\curveto(334.5416644,481.7279594)(334.50166444,481.7279594)(334.46166841,481.71796404)
\curveto(334.36166458,481.69795943)(334.26666468,481.69795943)(334.17666841,481.71796404)
\moveto(333.33666841,483.11296404)
\curveto(333.37666557,483.04295808)(333.41666553,482.97795815)(333.45666841,482.91796404)
\curveto(333.49666545,482.86795826)(333.5466654,482.81795831)(333.60666841,482.76796404)
\lineto(333.75666841,482.64796404)
\curveto(333.81666513,482.61795851)(333.88166506,482.59295853)(333.95166841,482.57296404)
\curveto(333.99166495,482.55295857)(334.02666492,482.54295858)(334.05666841,482.54296404)
\curveto(334.09666485,482.55295857)(334.13666481,482.54795858)(334.17666841,482.52796404)
\curveto(334.20666474,482.5279586)(334.2466647,482.5229586)(334.29666841,482.51296404)
\curveto(334.3466646,482.51295861)(334.38666456,482.51795861)(334.41666841,482.52796404)
\lineto(334.64166841,482.57296404)
\curveto(334.89166405,482.65295847)(335.07666387,482.77795835)(335.19666841,482.94796404)
\curveto(335.27666367,483.04795808)(335.3466636,483.17795795)(335.40666841,483.33796404)
\curveto(335.48666346,483.51795761)(335.5466634,483.74295738)(335.58666841,484.01296404)
\curveto(335.62666332,484.29295683)(335.6416633,484.57295655)(335.63166841,484.85296404)
\curveto(335.62166332,485.14295598)(335.59166335,485.41795571)(335.54166841,485.67796404)
\curveto(335.49166345,485.93795519)(335.41666353,486.14795498)(335.31666841,486.30796404)
\curveto(335.19666375,486.50795462)(335.0466639,486.65795447)(334.86666841,486.75796404)
\curveto(334.78666416,486.80795432)(334.69666425,486.83795429)(334.59666841,486.84796404)
\curveto(334.49666445,486.86795426)(334.39166455,486.87795425)(334.28166841,486.87796404)
\curveto(334.26166468,486.86795426)(334.23666471,486.86295426)(334.20666841,486.86296404)
\curveto(334.18666476,486.87295425)(334.16666478,486.87295425)(334.14666841,486.86296404)
\curveto(334.09666485,486.85295427)(334.05166489,486.84295428)(334.01166841,486.83296404)
\curveto(333.97166497,486.83295429)(333.93166501,486.8229543)(333.89166841,486.80296404)
\curveto(333.71166523,486.7229544)(333.56166538,486.60295452)(333.44166841,486.44296404)
\curveto(333.33166561,486.28295484)(333.2416657,486.10295502)(333.17166841,485.90296404)
\curveto(333.11166583,485.71295541)(333.06666588,485.48795564)(333.03666841,485.22796404)
\curveto(333.01666593,484.96795616)(333.01166593,484.70295642)(333.02166841,484.43296404)
\curveto(333.03166591,484.17295695)(333.06166588,483.9229572)(333.11166841,483.68296404)
\curveto(333.17166577,483.45295767)(333.2466657,483.26295786)(333.33666841,483.11296404)
\moveto(344.13666841,480.12796404)
\curveto(344.1466548,480.07796105)(344.15165479,479.98796114)(344.15166841,479.85796404)
\curveto(344.15165479,479.7279614)(344.1416548,479.63796149)(344.12166841,479.58796404)
\curveto(344.10165484,479.53796159)(344.09665485,479.48296164)(344.10666841,479.42296404)
\curveto(344.11665483,479.37296175)(344.11665483,479.3229618)(344.10666841,479.27296404)
\curveto(344.06665488,479.13296199)(344.03665491,478.99796213)(344.01666841,478.86796404)
\curveto(344.00665494,478.73796239)(343.97665497,478.61796251)(343.92666841,478.50796404)
\curveto(343.78665516,478.15796297)(343.62165532,477.86296326)(343.43166841,477.62296404)
\curveto(343.2416557,477.39296373)(342.97165597,477.20796392)(342.62166841,477.06796404)
\curveto(342.5416564,477.03796409)(342.45665649,477.01796411)(342.36666841,477.00796404)
\curveto(342.27665667,476.98796414)(342.19165675,476.96796416)(342.11166841,476.94796404)
\curveto(342.06165688,476.93796419)(342.01165693,476.93296419)(341.96166841,476.93296404)
\curveto(341.91165703,476.93296419)(341.86165708,476.9279642)(341.81166841,476.91796404)
\curveto(341.78165716,476.90796422)(341.73165721,476.90796422)(341.66166841,476.91796404)
\curveto(341.59165735,476.91796421)(341.5416574,476.9229642)(341.51166841,476.93296404)
\curveto(341.45165749,476.95296417)(341.39165755,476.96296416)(341.33166841,476.96296404)
\curveto(341.28165766,476.95296417)(341.23165771,476.95796417)(341.18166841,476.97796404)
\curveto(341.09165785,476.99796413)(341.00165794,477.0229641)(340.91166841,477.05296404)
\curveto(340.83165811,477.07296405)(340.75165819,477.10296402)(340.67166841,477.14296404)
\curveto(340.35165859,477.28296384)(340.10165884,477.47796365)(339.92166841,477.72796404)
\curveto(339.7416592,477.98796314)(339.59165935,478.29296283)(339.47166841,478.64296404)
\curveto(339.45165949,478.7229624)(339.43665951,478.80796232)(339.42666841,478.89796404)
\curveto(339.41665953,478.98796214)(339.40165954,479.07296205)(339.38166841,479.15296404)
\curveto(339.37165957,479.18296194)(339.36665958,479.21296191)(339.36666841,479.24296404)
\lineto(339.36666841,479.34796404)
\curveto(339.3466596,479.4279617)(339.33665961,479.50796162)(339.33666841,479.58796404)
\lineto(339.33666841,479.72296404)
\curveto(339.31665963,479.8229613)(339.31665963,479.9229612)(339.33666841,480.02296404)
\lineto(339.33666841,480.20296404)
\curveto(339.3466596,480.25296087)(339.35165959,480.29796083)(339.35166841,480.33796404)
\curveto(339.35165959,480.38796074)(339.35665959,480.43296069)(339.36666841,480.47296404)
\curveto(339.37665957,480.51296061)(339.38165956,480.54796058)(339.38166841,480.57796404)
\curveto(339.38165956,480.61796051)(339.38665956,480.65796047)(339.39666841,480.69796404)
\lineto(339.45666841,481.02796404)
\curveto(339.47665947,481.14795998)(339.50665944,481.25795987)(339.54666841,481.35796404)
\curveto(339.68665926,481.68795944)(339.8466591,481.96295916)(340.02666841,482.18296404)
\curveto(340.21665873,482.41295871)(340.47665847,482.59795853)(340.80666841,482.73796404)
\curveto(340.88665806,482.77795835)(340.97165797,482.80295832)(341.06166841,482.81296404)
\lineto(341.36166841,482.87296404)
\lineto(341.49666841,482.87296404)
\curveto(341.5466574,482.88295824)(341.59665735,482.88795824)(341.64666841,482.88796404)
\curveto(342.21665673,482.90795822)(342.67665627,482.80295832)(343.02666841,482.57296404)
\curveto(343.38665556,482.35295877)(343.65165529,482.05295907)(343.82166841,481.67296404)
\curveto(343.87165507,481.57295955)(343.91165503,481.47295965)(343.94166841,481.37296404)
\curveto(343.97165497,481.27295985)(344.00165494,481.16795996)(344.03166841,481.05796404)
\curveto(344.0416549,481.01796011)(344.0466549,480.98296014)(344.04666841,480.95296404)
\curveto(344.0466549,480.93296019)(344.05165489,480.90296022)(344.06166841,480.86296404)
\curveto(344.08165486,480.79296033)(344.09165485,480.71796041)(344.09166841,480.63796404)
\curveto(344.09165485,480.55796057)(344.10165484,480.47796065)(344.12166841,480.39796404)
\curveto(344.12165482,480.34796078)(344.12165482,480.30296082)(344.12166841,480.26296404)
\curveto(344.12165482,480.2229609)(344.12665482,480.17796095)(344.13666841,480.12796404)
\moveto(343.02666841,479.69296404)
\curveto(343.03665591,479.74296138)(343.0416559,479.81796131)(343.04166841,479.91796404)
\curveto(343.05165589,480.01796111)(343.0466559,480.09296103)(343.02666841,480.14296404)
\curveto(343.00665594,480.20296092)(343.00165594,480.25796087)(343.01166841,480.30796404)
\curveto(343.03165591,480.36796076)(343.03165591,480.4279607)(343.01166841,480.48796404)
\curveto(343.00165594,480.51796061)(342.99665595,480.55296057)(342.99666841,480.59296404)
\curveto(342.99665595,480.63296049)(342.99165595,480.67296045)(342.98166841,480.71296404)
\curveto(342.96165598,480.79296033)(342.941656,480.86796026)(342.92166841,480.93796404)
\curveto(342.91165603,481.01796011)(342.89665605,481.09796003)(342.87666841,481.17796404)
\curveto(342.8466561,481.23795989)(342.82165612,481.29795983)(342.80166841,481.35796404)
\curveto(342.78165616,481.41795971)(342.75165619,481.47795965)(342.71166841,481.53796404)
\curveto(342.61165633,481.70795942)(342.48165646,481.84295928)(342.32166841,481.94296404)
\curveto(342.2416567,481.99295913)(342.1466568,482.0279591)(342.03666841,482.04796404)
\curveto(341.92665702,482.06795906)(341.80165714,482.07795905)(341.66166841,482.07796404)
\curveto(341.6416573,482.06795906)(341.61665733,482.06295906)(341.58666841,482.06296404)
\curveto(341.55665739,482.07295905)(341.52665742,482.07295905)(341.49666841,482.06296404)
\lineto(341.34666841,482.00296404)
\curveto(341.29665765,481.99295913)(341.25165769,481.97795915)(341.21166841,481.95796404)
\curveto(341.02165792,481.84795928)(340.87665807,481.70295942)(340.77666841,481.52296404)
\curveto(340.68665826,481.34295978)(340.60665834,481.13795999)(340.53666841,480.90796404)
\curveto(340.49665845,480.77796035)(340.47665847,480.64296048)(340.47666841,480.50296404)
\curveto(340.47665847,480.37296075)(340.46665848,480.2279609)(340.44666841,480.06796404)
\curveto(340.43665851,480.01796111)(340.42665852,479.95796117)(340.41666841,479.88796404)
\curveto(340.41665853,479.81796131)(340.42665852,479.75796137)(340.44666841,479.70796404)
\lineto(340.44666841,479.54296404)
\lineto(340.44666841,479.36296404)
\curveto(340.45665849,479.31296181)(340.46665848,479.25796187)(340.47666841,479.19796404)
\curveto(340.48665846,479.14796198)(340.49165845,479.09296203)(340.49166841,479.03296404)
\curveto(340.50165844,478.97296215)(340.51665843,478.91796221)(340.53666841,478.86796404)
\curveto(340.58665836,478.67796245)(340.6466583,478.50296262)(340.71666841,478.34296404)
\curveto(340.78665816,478.18296294)(340.89165805,478.05296307)(341.03166841,477.95296404)
\curveto(341.16165778,477.85296327)(341.30165764,477.78296334)(341.45166841,477.74296404)
\curveto(341.48165746,477.73296339)(341.50665744,477.7279634)(341.52666841,477.72796404)
\curveto(341.55665739,477.73796339)(341.58665736,477.73796339)(341.61666841,477.72796404)
\curveto(341.63665731,477.7279634)(341.66665728,477.7229634)(341.70666841,477.71296404)
\curveto(341.7466572,477.71296341)(341.78165716,477.71796341)(341.81166841,477.72796404)
\curveto(341.85165709,477.73796339)(341.89165705,477.74296338)(341.93166841,477.74296404)
\curveto(341.97165697,477.74296338)(342.01165693,477.75296337)(342.05166841,477.77296404)
\curveto(342.29165665,477.85296327)(342.48665646,477.98796314)(342.63666841,478.17796404)
\curveto(342.75665619,478.35796277)(342.8466561,478.56296256)(342.90666841,478.79296404)
\curveto(342.92665602,478.86296226)(342.941656,478.93296219)(342.95166841,479.00296404)
\curveto(342.96165598,479.08296204)(342.97665597,479.16296196)(342.99666841,479.24296404)
\curveto(342.99665595,479.30296182)(343.00165594,479.34796178)(343.01166841,479.37796404)
\curveto(343.01165593,479.39796173)(343.01165593,479.4229617)(343.01166841,479.45296404)
\curveto(343.01165593,479.49296163)(343.01665593,479.5229616)(343.02666841,479.54296404)
\lineto(343.02666841,479.69296404)
}
}
{
\newrgbcolor{curcolor}{0 0 0}
\pscustom[linestyle=none,fillstyle=solid,fillcolor=curcolor]
{
\newpath
\moveto(632.35698091,642.25630877)
\curveto(632.45697606,642.25629815)(632.55197596,642.24629816)(632.64198091,642.22630877)
\curveto(632.73197578,642.21629819)(632.79697572,642.18629822)(632.83698091,642.13630877)
\curveto(632.89697562,642.05629835)(632.92697559,641.95129846)(632.92698091,641.82130877)
\lineto(632.92698091,641.43130877)
\lineto(632.92698091,639.93130877)
\lineto(632.92698091,633.54130877)
\lineto(632.92698091,632.37130877)
\lineto(632.92698091,632.05630877)
\curveto(632.93697558,631.95630845)(632.92197559,631.87630853)(632.88198091,631.81630877)
\curveto(632.83197568,631.73630867)(632.75697576,631.68630872)(632.65698091,631.66630877)
\curveto(632.56697595,631.65630875)(632.45697606,631.65130876)(632.32698091,631.65130877)
\lineto(632.10198091,631.65130877)
\curveto(632.02197649,631.67130874)(631.95197656,631.68630872)(631.89198091,631.69630877)
\curveto(631.83197668,631.71630869)(631.78197673,631.75630865)(631.74198091,631.81630877)
\curveto(631.70197681,631.87630853)(631.68197683,631.95130846)(631.68198091,632.04130877)
\lineto(631.68198091,632.34130877)
\lineto(631.68198091,633.43630877)
\lineto(631.68198091,638.77630877)
\curveto(631.66197685,638.86630154)(631.64697687,638.94130147)(631.63698091,639.00130877)
\curveto(631.63697688,639.07130134)(631.60697691,639.13130128)(631.54698091,639.18130877)
\curveto(631.47697704,639.23130118)(631.38697713,639.25630115)(631.27698091,639.25630877)
\curveto(631.17697734,639.26630114)(631.06697745,639.27130114)(630.94698091,639.27130877)
\lineto(629.80698091,639.27130877)
\lineto(629.31198091,639.27130877)
\curveto(629.15197936,639.28130113)(629.04197947,639.34130107)(628.98198091,639.45130877)
\curveto(628.96197955,639.48130093)(628.95197956,639.5113009)(628.95198091,639.54130877)
\curveto(628.95197956,639.58130083)(628.94697957,639.62630078)(628.93698091,639.67630877)
\curveto(628.9169796,639.79630061)(628.92197959,639.9063005)(628.95198091,640.00630877)
\curveto(628.99197952,640.1063003)(629.04697947,640.17630023)(629.11698091,640.21630877)
\curveto(629.19697932,640.26630014)(629.3169792,640.29130012)(629.47698091,640.29130877)
\curveto(629.63697888,640.29130012)(629.77197874,640.3063001)(629.88198091,640.33630877)
\curveto(629.93197858,640.34630006)(629.98697853,640.35130006)(630.04698091,640.35130877)
\curveto(630.10697841,640.36130005)(630.16697835,640.37630003)(630.22698091,640.39630877)
\curveto(630.37697814,640.44629996)(630.52197799,640.49629991)(630.66198091,640.54630877)
\curveto(630.80197771,640.6062998)(630.93697758,640.67629973)(631.06698091,640.75630877)
\curveto(631.20697731,640.84629956)(631.32697719,640.95129946)(631.42698091,641.07130877)
\curveto(631.52697699,641.19129922)(631.62197689,641.32129909)(631.71198091,641.46130877)
\curveto(631.77197674,641.56129885)(631.8169767,641.67129874)(631.84698091,641.79130877)
\curveto(631.88697663,641.9112985)(631.93697658,642.01629839)(631.99698091,642.10630877)
\curveto(632.04697647,642.16629824)(632.1169764,642.2062982)(632.20698091,642.22630877)
\curveto(632.22697629,642.23629817)(632.25197626,642.24129817)(632.28198091,642.24130877)
\curveto(632.3119762,642.24129817)(632.33697618,642.24629816)(632.35698091,642.25630877)
}
}
{
\newrgbcolor{curcolor}{0 0 0}
\pscustom[linestyle=none,fillstyle=solid,fillcolor=curcolor]
{
\newpath
\moveto(640.70659029,642.25630877)
\curveto(640.80658543,642.25629815)(640.90158534,642.24629816)(640.99159029,642.22630877)
\curveto(641.08158516,642.21629819)(641.14658509,642.18629822)(641.18659029,642.13630877)
\curveto(641.24658499,642.05629835)(641.27658496,641.95129846)(641.27659029,641.82130877)
\lineto(641.27659029,641.43130877)
\lineto(641.27659029,639.93130877)
\lineto(641.27659029,633.54130877)
\lineto(641.27659029,632.37130877)
\lineto(641.27659029,632.05630877)
\curveto(641.28658495,631.95630845)(641.27158497,631.87630853)(641.23159029,631.81630877)
\curveto(641.18158506,631.73630867)(641.10658513,631.68630872)(641.00659029,631.66630877)
\curveto(640.91658532,631.65630875)(640.80658543,631.65130876)(640.67659029,631.65130877)
\lineto(640.45159029,631.65130877)
\curveto(640.37158587,631.67130874)(640.30158594,631.68630872)(640.24159029,631.69630877)
\curveto(640.18158606,631.71630869)(640.13158611,631.75630865)(640.09159029,631.81630877)
\curveto(640.05158619,631.87630853)(640.03158621,631.95130846)(640.03159029,632.04130877)
\lineto(640.03159029,632.34130877)
\lineto(640.03159029,633.43630877)
\lineto(640.03159029,638.77630877)
\curveto(640.01158623,638.86630154)(639.99658624,638.94130147)(639.98659029,639.00130877)
\curveto(639.98658625,639.07130134)(639.95658628,639.13130128)(639.89659029,639.18130877)
\curveto(639.82658641,639.23130118)(639.7365865,639.25630115)(639.62659029,639.25630877)
\curveto(639.52658671,639.26630114)(639.41658682,639.27130114)(639.29659029,639.27130877)
\lineto(638.15659029,639.27130877)
\lineto(637.66159029,639.27130877)
\curveto(637.50158874,639.28130113)(637.39158885,639.34130107)(637.33159029,639.45130877)
\curveto(637.31158893,639.48130093)(637.30158894,639.5113009)(637.30159029,639.54130877)
\curveto(637.30158894,639.58130083)(637.29658894,639.62630078)(637.28659029,639.67630877)
\curveto(637.26658897,639.79630061)(637.27158897,639.9063005)(637.30159029,640.00630877)
\curveto(637.3415889,640.1063003)(637.39658884,640.17630023)(637.46659029,640.21630877)
\curveto(637.54658869,640.26630014)(637.66658857,640.29130012)(637.82659029,640.29130877)
\curveto(637.98658825,640.29130012)(638.12158812,640.3063001)(638.23159029,640.33630877)
\curveto(638.28158796,640.34630006)(638.3365879,640.35130006)(638.39659029,640.35130877)
\curveto(638.45658778,640.36130005)(638.51658772,640.37630003)(638.57659029,640.39630877)
\curveto(638.72658751,640.44629996)(638.87158737,640.49629991)(639.01159029,640.54630877)
\curveto(639.15158709,640.6062998)(639.28658695,640.67629973)(639.41659029,640.75630877)
\curveto(639.55658668,640.84629956)(639.67658656,640.95129946)(639.77659029,641.07130877)
\curveto(639.87658636,641.19129922)(639.97158627,641.32129909)(640.06159029,641.46130877)
\curveto(640.12158612,641.56129885)(640.16658607,641.67129874)(640.19659029,641.79130877)
\curveto(640.236586,641.9112985)(640.28658595,642.01629839)(640.34659029,642.10630877)
\curveto(640.39658584,642.16629824)(640.46658577,642.2062982)(640.55659029,642.22630877)
\curveto(640.57658566,642.23629817)(640.60158564,642.24129817)(640.63159029,642.24130877)
\curveto(640.66158558,642.24129817)(640.68658555,642.24629816)(640.70659029,642.25630877)
}
}
{
\newrgbcolor{curcolor}{0 0 0}
\pscustom[linestyle=none,fillstyle=solid,fillcolor=curcolor]
{
\newpath
\moveto(645.95119966,633.28630877)
\lineto(646.25119966,633.28630877)
\curveto(646.3611976,633.29630711)(646.4661975,633.29630711)(646.56619966,633.28630877)
\curveto(646.67619729,633.28630712)(646.77619719,633.27630713)(646.86619966,633.25630877)
\curveto(646.95619701,633.24630716)(647.02619694,633.22130719)(647.07619966,633.18130877)
\curveto(647.09619687,633.16130725)(647.11119685,633.13130728)(647.12119966,633.09130877)
\curveto(647.14119682,633.05130736)(647.1611968,633.0063074)(647.18119966,632.95630877)
\lineto(647.18119966,632.88130877)
\curveto(647.19119677,632.83130758)(647.19119677,632.77630763)(647.18119966,632.71630877)
\lineto(647.18119966,632.56630877)
\lineto(647.18119966,632.08630877)
\curveto(647.18119678,631.91630849)(647.14119682,631.79630861)(647.06119966,631.72630877)
\curveto(646.99119697,631.67630873)(646.90119706,631.65130876)(646.79119966,631.65130877)
\lineto(646.46119966,631.65130877)
\lineto(646.01119966,631.65130877)
\curveto(645.8611981,631.65130876)(645.74619822,631.68130873)(645.66619966,631.74130877)
\curveto(645.62619834,631.77130864)(645.59619837,631.82130859)(645.57619966,631.89130877)
\curveto(645.55619841,631.97130844)(645.54119842,632.05630835)(645.53119966,632.14630877)
\lineto(645.53119966,632.43130877)
\curveto(645.54119842,632.53130788)(645.54619842,632.61630779)(645.54619966,632.68630877)
\lineto(645.54619966,632.88130877)
\curveto(645.54619842,632.94130747)(645.55619841,632.99630741)(645.57619966,633.04630877)
\curveto(645.61619835,633.15630725)(645.68619828,633.22630718)(645.78619966,633.25630877)
\curveto(645.81619815,633.25630715)(645.87119809,633.26630714)(645.95119966,633.28630877)
}
}
{
\newrgbcolor{curcolor}{0 0 0}
\pscustom[linestyle=none,fillstyle=solid,fillcolor=curcolor]
{
\newpath
\moveto(656.03635591,635.14630877)
\curveto(656.10634827,635.09630531)(656.14634823,635.02630538)(656.15635591,634.93630877)
\curveto(656.1763482,634.84630556)(656.18634819,634.74130567)(656.18635591,634.62130877)
\curveto(656.18634819,634.57130584)(656.18134819,634.52130589)(656.17135591,634.47130877)
\curveto(656.1713482,634.42130599)(656.16134821,634.37630603)(656.14135591,634.33630877)
\curveto(656.11134826,634.24630616)(656.05134832,634.18630622)(655.96135591,634.15630877)
\curveto(655.88134849,634.13630627)(655.78634859,634.12630628)(655.67635591,634.12630877)
\lineto(655.36135591,634.12630877)
\curveto(655.25134912,634.13630627)(655.14634923,634.12630628)(655.04635591,634.09630877)
\curveto(654.90634947,634.06630634)(654.81634956,633.98630642)(654.77635591,633.85630877)
\curveto(654.75634962,633.78630662)(654.74634963,633.70130671)(654.74635591,633.60130877)
\lineto(654.74635591,633.33130877)
\lineto(654.74635591,632.38630877)
\lineto(654.74635591,632.05630877)
\curveto(654.74634963,631.94630846)(654.72634965,631.86130855)(654.68635591,631.80130877)
\curveto(654.64634973,631.74130867)(654.59634978,631.70130871)(654.53635591,631.68130877)
\curveto(654.48634989,631.67130874)(654.42134995,631.65630875)(654.34135591,631.63630877)
\lineto(654.14635591,631.63630877)
\curveto(654.02635035,631.63630877)(653.92135045,631.64130877)(653.83135591,631.65130877)
\curveto(653.74135063,631.67130874)(653.6713507,631.72130869)(653.62135591,631.80130877)
\curveto(653.59135078,631.85130856)(653.5763508,631.92130849)(653.57635591,632.01130877)
\lineto(653.57635591,632.31130877)
\lineto(653.57635591,633.34630877)
\curveto(653.5763508,633.5063069)(653.56635081,633.65130676)(653.54635591,633.78130877)
\curveto(653.53635084,633.92130649)(653.48135089,634.01630639)(653.38135591,634.06630877)
\curveto(653.33135104,634.08630632)(653.26135111,634.10130631)(653.17135591,634.11130877)
\curveto(653.09135128,634.12130629)(653.00135137,634.12630628)(652.90135591,634.12630877)
\lineto(652.61635591,634.12630877)
\lineto(652.37635591,634.12630877)
\lineto(650.11135591,634.12630877)
\curveto(650.02135435,634.12630628)(649.91635446,634.12130629)(649.79635591,634.11130877)
\lineto(649.46635591,634.11130877)
\curveto(649.35635502,634.1113063)(649.25635512,634.12130629)(649.16635591,634.14130877)
\curveto(649.0763553,634.16130625)(649.01635536,634.19630621)(648.98635591,634.24630877)
\curveto(648.93635544,634.31630609)(648.91135546,634.411306)(648.91135591,634.53130877)
\lineto(648.91135591,634.87630877)
\lineto(648.91135591,635.14630877)
\curveto(648.95135542,635.31630509)(649.00635537,635.45630495)(649.07635591,635.56630877)
\curveto(649.14635523,635.67630473)(649.22635515,635.79130462)(649.31635591,635.91130877)
\lineto(649.67635591,636.45130877)
\curveto(650.11635426,637.08130333)(650.55135382,637.70130271)(650.98135591,638.31130877)
\lineto(652.30135591,640.17130877)
\curveto(652.46135191,640.40130001)(652.61635176,640.62129979)(652.76635591,640.83130877)
\curveto(652.91635146,641.05129936)(653.0713513,641.27629913)(653.23135591,641.50630877)
\curveto(653.28135109,641.57629883)(653.33135104,641.64129877)(653.38135591,641.70130877)
\curveto(653.43135094,641.77129864)(653.48135089,641.84629856)(653.53135591,641.92630877)
\lineto(653.59135591,642.01630877)
\curveto(653.62135075,642.05629835)(653.65135072,642.08629832)(653.68135591,642.10630877)
\curveto(653.72135065,642.13629827)(653.76135061,642.15629825)(653.80135591,642.16630877)
\curveto(653.84135053,642.18629822)(653.88635049,642.2062982)(653.93635591,642.22630877)
\curveto(653.95635042,642.22629818)(653.9763504,642.22129819)(653.99635591,642.21130877)
\curveto(654.02635035,642.2112982)(654.05135032,642.22129819)(654.07135591,642.24130877)
\curveto(654.20135017,642.24129817)(654.32135005,642.23629817)(654.43135591,642.22630877)
\curveto(654.54134983,642.21629819)(654.62134975,642.17129824)(654.67135591,642.09130877)
\curveto(654.71134966,642.04129837)(654.73134964,641.97129844)(654.73135591,641.88130877)
\curveto(654.74134963,641.79129862)(654.74634963,641.69629871)(654.74635591,641.59630877)
\lineto(654.74635591,636.13630877)
\curveto(654.74634963,636.06630434)(654.74134963,635.99130442)(654.73135591,635.91130877)
\curveto(654.73134964,635.84130457)(654.73634964,635.77130464)(654.74635591,635.70130877)
\lineto(654.74635591,635.59630877)
\curveto(654.76634961,635.54630486)(654.78134959,635.49130492)(654.79135591,635.43130877)
\curveto(654.80134957,635.38130503)(654.82634955,635.34130507)(654.86635591,635.31130877)
\curveto(654.93634944,635.26130515)(655.02134935,635.23130518)(655.12135591,635.22130877)
\lineto(655.45135591,635.22130877)
\curveto(655.56134881,635.22130519)(655.66634871,635.21630519)(655.76635591,635.20630877)
\curveto(655.8763485,635.2063052)(655.96634841,635.18630522)(656.03635591,635.14630877)
\moveto(653.47135591,635.34130877)
\curveto(653.55135082,635.45130496)(653.58635079,635.62130479)(653.57635591,635.85130877)
\lineto(653.57635591,636.46630877)
\lineto(653.57635591,638.94130877)
\lineto(653.57635591,639.25630877)
\curveto(653.58635079,639.37630103)(653.58135079,639.47630093)(653.56135591,639.55630877)
\lineto(653.56135591,639.70630877)
\curveto(653.56135081,639.79630061)(653.54635083,639.88130053)(653.51635591,639.96130877)
\curveto(653.50635087,639.98130043)(653.49635088,639.99130042)(653.48635591,639.99130877)
\lineto(653.44135591,640.03630877)
\curveto(653.42135095,640.04630036)(653.39135098,640.05130036)(653.35135591,640.05130877)
\curveto(653.33135104,640.03130038)(653.31135106,640.01630039)(653.29135591,640.00630877)
\curveto(653.28135109,640.0063004)(653.26635111,640.00130041)(653.24635591,639.99130877)
\curveto(653.18635119,639.94130047)(653.12635125,639.87130054)(653.06635591,639.78130877)
\curveto(653.00635137,639.69130072)(652.95135142,639.6113008)(652.90135591,639.54130877)
\curveto(652.80135157,639.40130101)(652.70635167,639.25630115)(652.61635591,639.10630877)
\curveto(652.52635185,638.96630144)(652.43135194,638.82630158)(652.33135591,638.68630877)
\lineto(651.79135591,637.90630877)
\curveto(651.62135275,637.64630276)(651.44635293,637.38630302)(651.26635591,637.12630877)
\curveto(651.18635319,637.01630339)(651.11135326,636.9113035)(651.04135591,636.81130877)
\lineto(650.83135591,636.51130877)
\curveto(650.78135359,636.43130398)(650.73135364,636.35630405)(650.68135591,636.28630877)
\curveto(650.64135373,636.21630419)(650.59635378,636.14130427)(650.54635591,636.06130877)
\curveto(650.49635388,636.00130441)(650.44635393,635.93630447)(650.39635591,635.86630877)
\curveto(650.35635402,635.8063046)(650.31635406,635.73630467)(650.27635591,635.65630877)
\curveto(650.23635414,635.59630481)(650.21135416,635.52630488)(650.20135591,635.44630877)
\curveto(650.19135418,635.37630503)(650.22635415,635.32130509)(650.30635591,635.28130877)
\curveto(650.376354,635.23130518)(650.48635389,635.2063052)(650.63635591,635.20630877)
\curveto(650.79635358,635.21630519)(650.93135344,635.22130519)(651.04135591,635.22130877)
\lineto(652.72135591,635.22130877)
\lineto(653.15635591,635.22130877)
\curveto(653.30635107,635.22130519)(653.41135096,635.26130515)(653.47135591,635.34130877)
}
}
{
\newrgbcolor{curcolor}{0 0 0}
\pscustom[linestyle=none,fillstyle=solid,fillcolor=curcolor]
{
\newpath
\moveto(667.31096529,640.17130877)
\curveto(667.11095499,639.88130053)(666.9009552,639.59630081)(666.68096529,639.31630877)
\curveto(666.47095563,639.03630137)(666.26595583,638.75130166)(666.06596529,638.46130877)
\curveto(665.46595663,637.6113028)(664.86095724,636.77130364)(664.25096529,635.94130877)
\curveto(663.64095846,635.12130529)(663.03595906,634.28630612)(662.43596529,633.43630877)
\lineto(661.92596529,632.71630877)
\lineto(661.41596529,632.02630877)
\curveto(661.33596076,631.91630849)(661.25596084,631.80130861)(661.17596529,631.68130877)
\curveto(661.095961,631.56130885)(661.0009611,631.46630894)(660.89096529,631.39630877)
\curveto(660.85096125,631.37630903)(660.78596131,631.36130905)(660.69596529,631.35130877)
\curveto(660.61596148,631.33130908)(660.52596157,631.32130909)(660.42596529,631.32130877)
\curveto(660.32596177,631.32130909)(660.23096187,631.32630908)(660.14096529,631.33630877)
\curveto(660.06096204,631.34630906)(660.0009621,631.36630904)(659.96096529,631.39630877)
\curveto(659.93096217,631.41630899)(659.90596219,631.45130896)(659.88596529,631.50130877)
\curveto(659.87596222,631.54130887)(659.88096222,631.58630882)(659.90096529,631.63630877)
\curveto(659.94096216,631.71630869)(659.98596211,631.79130862)(660.03596529,631.86130877)
\curveto(660.095962,631.94130847)(660.15096195,632.02130839)(660.20096529,632.10130877)
\curveto(660.44096166,632.44130797)(660.68596141,632.77630763)(660.93596529,633.10630877)
\curveto(661.18596091,633.43630697)(661.42596067,633.77130664)(661.65596529,634.11130877)
\curveto(661.81596028,634.33130608)(661.97596012,634.54630586)(662.13596529,634.75630877)
\curveto(662.2959598,634.96630544)(662.45595964,635.18130523)(662.61596529,635.40130877)
\curveto(662.97595912,635.92130449)(663.34095876,636.43130398)(663.71096529,636.93130877)
\curveto(664.08095802,637.43130298)(664.45095765,637.94130247)(664.82096529,638.46130877)
\curveto(664.96095714,638.66130175)(665.100957,638.85630155)(665.24096529,639.04630877)
\curveto(665.39095671,639.23630117)(665.53595656,639.43130098)(665.67596529,639.63130877)
\curveto(665.88595621,639.93130048)(666.100956,640.23130018)(666.32096529,640.53130877)
\lineto(666.98096529,641.43130877)
\lineto(667.16096529,641.70130877)
\lineto(667.37096529,641.97130877)
\lineto(667.49096529,642.15130877)
\curveto(667.54095456,642.2112982)(667.59095451,642.26629814)(667.64096529,642.31630877)
\curveto(667.71095439,642.36629804)(667.78595431,642.40129801)(667.86596529,642.42130877)
\curveto(667.88595421,642.43129798)(667.91095419,642.43129798)(667.94096529,642.42130877)
\curveto(667.98095412,642.42129799)(668.01095409,642.43129798)(668.03096529,642.45130877)
\curveto(668.15095395,642.45129796)(668.28595381,642.44629796)(668.43596529,642.43630877)
\curveto(668.58595351,642.43629797)(668.67595342,642.39129802)(668.70596529,642.30130877)
\curveto(668.72595337,642.27129814)(668.73095337,642.23629817)(668.72096529,642.19630877)
\curveto(668.71095339,642.15629825)(668.6959534,642.12629828)(668.67596529,642.10630877)
\curveto(668.63595346,642.02629838)(668.5959535,641.95629845)(668.55596529,641.89630877)
\curveto(668.51595358,641.83629857)(668.47095363,641.77629863)(668.42096529,641.71630877)
\lineto(667.85096529,640.93630877)
\curveto(667.67095443,640.68629972)(667.49095461,640.43129998)(667.31096529,640.17130877)
\moveto(660.45596529,636.27130877)
\curveto(660.40596169,636.29130412)(660.35596174,636.29630411)(660.30596529,636.28630877)
\curveto(660.25596184,636.27630413)(660.20596189,636.28130413)(660.15596529,636.30130877)
\curveto(660.04596205,636.32130409)(659.94096216,636.34130407)(659.84096529,636.36130877)
\curveto(659.75096235,636.39130402)(659.65596244,636.43130398)(659.55596529,636.48130877)
\curveto(659.22596287,636.62130379)(658.97096313,636.81630359)(658.79096529,637.06630877)
\curveto(658.61096349,637.32630308)(658.46596363,637.63630277)(658.35596529,637.99630877)
\curveto(658.32596377,638.07630233)(658.30596379,638.15630225)(658.29596529,638.23630877)
\curveto(658.28596381,638.32630208)(658.27096383,638.411302)(658.25096529,638.49130877)
\curveto(658.24096386,638.54130187)(658.23596386,638.6063018)(658.23596529,638.68630877)
\curveto(658.22596387,638.71630169)(658.22096388,638.74630166)(658.22096529,638.77630877)
\curveto(658.22096388,638.81630159)(658.21596388,638.85130156)(658.20596529,638.88130877)
\lineto(658.20596529,639.03130877)
\curveto(658.1959639,639.08130133)(658.19096391,639.14130127)(658.19096529,639.21130877)
\curveto(658.19096391,639.29130112)(658.1959639,639.35630105)(658.20596529,639.40630877)
\lineto(658.20596529,639.57130877)
\curveto(658.22596387,639.62130079)(658.23096387,639.66630074)(658.22096529,639.70630877)
\curveto(658.22096388,639.75630065)(658.22596387,639.80130061)(658.23596529,639.84130877)
\curveto(658.24596385,639.88130053)(658.25096385,639.91630049)(658.25096529,639.94630877)
\curveto(658.25096385,639.98630042)(658.25596384,640.02630038)(658.26596529,640.06630877)
\curveto(658.2959638,640.17630023)(658.31596378,640.28630012)(658.32596529,640.39630877)
\curveto(658.34596375,640.51629989)(658.38096372,640.63129978)(658.43096529,640.74130877)
\curveto(658.57096353,641.08129933)(658.73096337,641.35629905)(658.91096529,641.56630877)
\curveto(659.100963,641.78629862)(659.37096273,641.96629844)(659.72096529,642.10630877)
\curveto(659.8009623,642.13629827)(659.88596221,642.15629825)(659.97596529,642.16630877)
\curveto(660.06596203,642.18629822)(660.16096194,642.2062982)(660.26096529,642.22630877)
\curveto(660.29096181,642.23629817)(660.34596175,642.23629817)(660.42596529,642.22630877)
\curveto(660.50596159,642.22629818)(660.55596154,642.23629817)(660.57596529,642.25630877)
\curveto(661.13596096,642.26629814)(661.58596051,642.15629825)(661.92596529,641.92630877)
\curveto(662.27595982,641.69629871)(662.53595956,641.39129902)(662.70596529,641.01130877)
\curveto(662.74595935,640.92129949)(662.78095932,640.82629958)(662.81096529,640.72630877)
\curveto(662.84095926,640.62629978)(662.86595923,640.52629988)(662.88596529,640.42630877)
\curveto(662.90595919,640.39630001)(662.91095919,640.36630004)(662.90096529,640.33630877)
\curveto(662.9009592,640.3063001)(662.90595919,640.27630013)(662.91596529,640.24630877)
\curveto(662.94595915,640.13630027)(662.96595913,640.0113004)(662.97596529,639.87130877)
\curveto(662.98595911,639.74130067)(662.9959591,639.6063008)(663.00596529,639.46630877)
\lineto(663.00596529,639.30130877)
\curveto(663.01595908,639.24130117)(663.01595908,639.18630122)(663.00596529,639.13630877)
\curveto(662.9959591,639.08630132)(662.99095911,639.03630137)(662.99096529,638.98630877)
\lineto(662.99096529,638.85130877)
\curveto(662.98095912,638.8113016)(662.97595912,638.77130164)(662.97596529,638.73130877)
\curveto(662.98595911,638.69130172)(662.98095912,638.64630176)(662.96096529,638.59630877)
\curveto(662.94095916,638.48630192)(662.92095918,638.38130203)(662.90096529,638.28130877)
\curveto(662.89095921,638.18130223)(662.87095923,638.08130233)(662.84096529,637.98130877)
\curveto(662.71095939,637.62130279)(662.54595955,637.3063031)(662.34596529,637.03630877)
\curveto(662.14595995,636.76630364)(661.87096023,636.56130385)(661.52096529,636.42130877)
\curveto(661.44096066,636.39130402)(661.35596074,636.36630404)(661.26596529,636.34630877)
\lineto(660.99596529,636.28630877)
\curveto(660.94596115,636.27630413)(660.9009612,636.27130414)(660.86096529,636.27130877)
\curveto(660.82096128,636.28130413)(660.78096132,636.28130413)(660.74096529,636.27130877)
\curveto(660.64096146,636.25130416)(660.54596155,636.25130416)(660.45596529,636.27130877)
\moveto(659.61596529,637.66630877)
\curveto(659.65596244,637.59630281)(659.6959624,637.53130288)(659.73596529,637.47130877)
\curveto(659.77596232,637.42130299)(659.82596227,637.37130304)(659.88596529,637.32130877)
\lineto(660.03596529,637.20130877)
\curveto(660.095962,637.17130324)(660.16096194,637.14630326)(660.23096529,637.12630877)
\curveto(660.27096183,637.1063033)(660.30596179,637.09630331)(660.33596529,637.09630877)
\curveto(660.37596172,637.1063033)(660.41596168,637.10130331)(660.45596529,637.08130877)
\curveto(660.48596161,637.08130333)(660.52596157,637.07630333)(660.57596529,637.06630877)
\curveto(660.62596147,637.06630334)(660.66596143,637.07130334)(660.69596529,637.08130877)
\lineto(660.92096529,637.12630877)
\curveto(661.17096093,637.2063032)(661.35596074,637.33130308)(661.47596529,637.50130877)
\curveto(661.55596054,637.60130281)(661.62596047,637.73130268)(661.68596529,637.89130877)
\curveto(661.76596033,638.07130234)(661.82596027,638.29630211)(661.86596529,638.56630877)
\curveto(661.90596019,638.84630156)(661.92096018,639.12630128)(661.91096529,639.40630877)
\curveto(661.9009602,639.69630071)(661.87096023,639.97130044)(661.82096529,640.23130877)
\curveto(661.77096033,640.49129992)(661.6959604,640.70129971)(661.59596529,640.86130877)
\curveto(661.47596062,641.06129935)(661.32596077,641.2112992)(661.14596529,641.31130877)
\curveto(661.06596103,641.36129905)(660.97596112,641.39129902)(660.87596529,641.40130877)
\curveto(660.77596132,641.42129899)(660.67096143,641.43129898)(660.56096529,641.43130877)
\curveto(660.54096156,641.42129899)(660.51596158,641.41629899)(660.48596529,641.41630877)
\curveto(660.46596163,641.42629898)(660.44596165,641.42629898)(660.42596529,641.41630877)
\curveto(660.37596172,641.406299)(660.33096177,641.39629901)(660.29096529,641.38630877)
\curveto(660.25096185,641.38629902)(660.21096189,641.37629903)(660.17096529,641.35630877)
\curveto(659.99096211,641.27629913)(659.84096226,641.15629925)(659.72096529,640.99630877)
\curveto(659.61096249,640.83629957)(659.52096258,640.65629975)(659.45096529,640.45630877)
\curveto(659.39096271,640.26630014)(659.34596275,640.04130037)(659.31596529,639.78130877)
\curveto(659.2959628,639.52130089)(659.29096281,639.25630115)(659.30096529,638.98630877)
\curveto(659.31096279,638.72630168)(659.34096276,638.47630193)(659.39096529,638.23630877)
\curveto(659.45096265,638.0063024)(659.52596257,637.81630259)(659.61596529,637.66630877)
\moveto(670.41596529,634.68130877)
\curveto(670.42595167,634.63130578)(670.43095167,634.54130587)(670.43096529,634.41130877)
\curveto(670.43095167,634.28130613)(670.42095168,634.19130622)(670.40096529,634.14130877)
\curveto(670.38095172,634.09130632)(670.37595172,634.03630637)(670.38596529,633.97630877)
\curveto(670.3959517,633.92630648)(670.3959517,633.87630653)(670.38596529,633.82630877)
\curveto(670.34595175,633.68630672)(670.31595178,633.55130686)(670.29596529,633.42130877)
\curveto(670.28595181,633.29130712)(670.25595184,633.17130724)(670.20596529,633.06130877)
\curveto(670.06595203,632.7113077)(669.9009522,632.41630799)(669.71096529,632.17630877)
\curveto(669.52095258,631.94630846)(669.25095285,631.76130865)(668.90096529,631.62130877)
\curveto(668.82095328,631.59130882)(668.73595336,631.57130884)(668.64596529,631.56130877)
\curveto(668.55595354,631.54130887)(668.47095363,631.52130889)(668.39096529,631.50130877)
\curveto(668.34095376,631.49130892)(668.29095381,631.48630892)(668.24096529,631.48630877)
\curveto(668.19095391,631.48630892)(668.14095396,631.48130893)(668.09096529,631.47130877)
\curveto(668.06095404,631.46130895)(668.01095409,631.46130895)(667.94096529,631.47130877)
\curveto(667.87095423,631.47130894)(667.82095428,631.47630893)(667.79096529,631.48630877)
\curveto(667.73095437,631.5063089)(667.67095443,631.51630889)(667.61096529,631.51630877)
\curveto(667.56095454,631.5063089)(667.51095459,631.5113089)(667.46096529,631.53130877)
\curveto(667.37095473,631.55130886)(667.28095482,631.57630883)(667.19096529,631.60630877)
\curveto(667.11095499,631.62630878)(667.03095507,631.65630875)(666.95096529,631.69630877)
\curveto(666.63095547,631.83630857)(666.38095572,632.03130838)(666.20096529,632.28130877)
\curveto(666.02095608,632.54130787)(665.87095623,632.84630756)(665.75096529,633.19630877)
\curveto(665.73095637,633.27630713)(665.71595638,633.36130705)(665.70596529,633.45130877)
\curveto(665.6959564,633.54130687)(665.68095642,633.62630678)(665.66096529,633.70630877)
\curveto(665.65095645,633.73630667)(665.64595645,633.76630664)(665.64596529,633.79630877)
\lineto(665.64596529,633.90130877)
\curveto(665.62595647,633.98130643)(665.61595648,634.06130635)(665.61596529,634.14130877)
\lineto(665.61596529,634.27630877)
\curveto(665.5959565,634.37630603)(665.5959565,634.47630593)(665.61596529,634.57630877)
\lineto(665.61596529,634.75630877)
\curveto(665.62595647,634.8063056)(665.63095647,634.85130556)(665.63096529,634.89130877)
\curveto(665.63095647,634.94130547)(665.63595646,634.98630542)(665.64596529,635.02630877)
\curveto(665.65595644,635.06630534)(665.66095644,635.10130531)(665.66096529,635.13130877)
\curveto(665.66095644,635.17130524)(665.66595643,635.2113052)(665.67596529,635.25130877)
\lineto(665.73596529,635.58130877)
\curveto(665.75595634,635.70130471)(665.78595631,635.8113046)(665.82596529,635.91130877)
\curveto(665.96595613,636.24130417)(666.12595597,636.51630389)(666.30596529,636.73630877)
\curveto(666.4959556,636.96630344)(666.75595534,637.15130326)(667.08596529,637.29130877)
\curveto(667.16595493,637.33130308)(667.25095485,637.35630305)(667.34096529,637.36630877)
\lineto(667.64096529,637.42630877)
\lineto(667.77596529,637.42630877)
\curveto(667.82595427,637.43630297)(667.87595422,637.44130297)(667.92596529,637.44130877)
\curveto(668.4959536,637.46130295)(668.95595314,637.35630305)(669.30596529,637.12630877)
\curveto(669.66595243,636.9063035)(669.93095217,636.6063038)(670.10096529,636.22630877)
\curveto(670.15095195,636.12630428)(670.19095191,636.02630438)(670.22096529,635.92630877)
\curveto(670.25095185,635.82630458)(670.28095182,635.72130469)(670.31096529,635.61130877)
\curveto(670.32095178,635.57130484)(670.32595177,635.53630487)(670.32596529,635.50630877)
\curveto(670.32595177,635.48630492)(670.33095177,635.45630495)(670.34096529,635.41630877)
\curveto(670.36095174,635.34630506)(670.37095173,635.27130514)(670.37096529,635.19130877)
\curveto(670.37095173,635.1113053)(670.38095172,635.03130538)(670.40096529,634.95130877)
\curveto(670.4009517,634.90130551)(670.4009517,634.85630555)(670.40096529,634.81630877)
\curveto(670.4009517,634.77630563)(670.40595169,634.73130568)(670.41596529,634.68130877)
\moveto(669.30596529,634.24630877)
\curveto(669.31595278,634.29630611)(669.32095278,634.37130604)(669.32096529,634.47130877)
\curveto(669.33095277,634.57130584)(669.32595277,634.64630576)(669.30596529,634.69630877)
\curveto(669.28595281,634.75630565)(669.28095282,634.8113056)(669.29096529,634.86130877)
\curveto(669.31095279,634.92130549)(669.31095279,634.98130543)(669.29096529,635.04130877)
\curveto(669.28095282,635.07130534)(669.27595282,635.1063053)(669.27596529,635.14630877)
\curveto(669.27595282,635.18630522)(669.27095283,635.22630518)(669.26096529,635.26630877)
\curveto(669.24095286,635.34630506)(669.22095288,635.42130499)(669.20096529,635.49130877)
\curveto(669.19095291,635.57130484)(669.17595292,635.65130476)(669.15596529,635.73130877)
\curveto(669.12595297,635.79130462)(669.100953,635.85130456)(669.08096529,635.91130877)
\curveto(669.06095304,635.97130444)(669.03095307,636.03130438)(668.99096529,636.09130877)
\curveto(668.89095321,636.26130415)(668.76095334,636.39630401)(668.60096529,636.49630877)
\curveto(668.52095358,636.54630386)(668.42595367,636.58130383)(668.31596529,636.60130877)
\curveto(668.20595389,636.62130379)(668.08095402,636.63130378)(667.94096529,636.63130877)
\curveto(667.92095418,636.62130379)(667.8959542,636.61630379)(667.86596529,636.61630877)
\curveto(667.83595426,636.62630378)(667.80595429,636.62630378)(667.77596529,636.61630877)
\lineto(667.62596529,636.55630877)
\curveto(667.57595452,636.54630386)(667.53095457,636.53130388)(667.49096529,636.51130877)
\curveto(667.3009548,636.40130401)(667.15595494,636.25630415)(667.05596529,636.07630877)
\curveto(666.96595513,635.89630451)(666.88595521,635.69130472)(666.81596529,635.46130877)
\curveto(666.77595532,635.33130508)(666.75595534,635.19630521)(666.75596529,635.05630877)
\curveto(666.75595534,634.92630548)(666.74595535,634.78130563)(666.72596529,634.62130877)
\curveto(666.71595538,634.57130584)(666.70595539,634.5113059)(666.69596529,634.44130877)
\curveto(666.6959554,634.37130604)(666.70595539,634.3113061)(666.72596529,634.26130877)
\lineto(666.72596529,634.09630877)
\lineto(666.72596529,633.91630877)
\curveto(666.73595536,633.86630654)(666.74595535,633.8113066)(666.75596529,633.75130877)
\curveto(666.76595533,633.70130671)(666.77095533,633.64630676)(666.77096529,633.58630877)
\curveto(666.78095532,633.52630688)(666.7959553,633.47130694)(666.81596529,633.42130877)
\curveto(666.86595523,633.23130718)(666.92595517,633.05630735)(666.99596529,632.89630877)
\curveto(667.06595503,632.73630767)(667.17095493,632.6063078)(667.31096529,632.50630877)
\curveto(667.44095466,632.406308)(667.58095452,632.33630807)(667.73096529,632.29630877)
\curveto(667.76095434,632.28630812)(667.78595431,632.28130813)(667.80596529,632.28130877)
\curveto(667.83595426,632.29130812)(667.86595423,632.29130812)(667.89596529,632.28130877)
\curveto(667.91595418,632.28130813)(667.94595415,632.27630813)(667.98596529,632.26630877)
\curveto(668.02595407,632.26630814)(668.06095404,632.27130814)(668.09096529,632.28130877)
\curveto(668.13095397,632.29130812)(668.17095393,632.29630811)(668.21096529,632.29630877)
\curveto(668.25095385,632.29630811)(668.29095381,632.3063081)(668.33096529,632.32630877)
\curveto(668.57095353,632.406308)(668.76595333,632.54130787)(668.91596529,632.73130877)
\curveto(669.03595306,632.9113075)(669.12595297,633.11630729)(669.18596529,633.34630877)
\curveto(669.20595289,633.41630699)(669.22095288,633.48630692)(669.23096529,633.55630877)
\curveto(669.24095286,633.63630677)(669.25595284,633.71630669)(669.27596529,633.79630877)
\curveto(669.27595282,633.85630655)(669.28095282,633.90130651)(669.29096529,633.93130877)
\curveto(669.29095281,633.95130646)(669.29095281,633.97630643)(669.29096529,634.00630877)
\curveto(669.29095281,634.04630636)(669.2959528,634.07630633)(669.30596529,634.09630877)
\lineto(669.30596529,634.24630877)
}
}
{
\newrgbcolor{curcolor}{0 0 0}
\pscustom[linestyle=none,fillstyle=solid,fillcolor=curcolor]
{
\newpath
\moveto(701.1124814,475.58113787)
\curveto(702.74247596,475.61112722)(703.79247491,475.05612778)(704.2624814,473.91613787)
\curveto(704.36247434,473.68612915)(704.42747428,473.39612944)(704.4574814,473.04613787)
\curveto(704.49747421,472.70613013)(704.47247423,472.39613044)(704.3824814,472.11613787)
\curveto(704.29247441,471.85613098)(704.17247453,471.6311312)(704.0224814,471.44113787)
\curveto(704.0024747,471.40113143)(703.97747473,471.36613147)(703.9474814,471.33613787)
\curveto(703.91747479,471.31613152)(703.89247481,471.29113154)(703.8724814,471.26113787)
\lineto(703.7824814,471.14113787)
\curveto(703.75247495,471.11113172)(703.71747499,471.08613175)(703.6774814,471.06613787)
\curveto(703.62747508,471.01613182)(703.57247513,470.97113186)(703.5124814,470.93113787)
\curveto(703.46247524,470.89113194)(703.41747529,470.84113199)(703.3774814,470.78113787)
\curveto(703.33747537,470.74113209)(703.32247538,470.69113214)(703.3324814,470.63113787)
\curveto(703.34247536,470.58113225)(703.37247533,470.5361323)(703.4224814,470.49613787)
\curveto(703.47247523,470.45613238)(703.52747518,470.41613242)(703.5874814,470.37613787)
\curveto(703.65747505,470.34613249)(703.72247498,470.31613252)(703.7824814,470.28613787)
\curveto(703.84247486,470.25613258)(703.89247481,470.22613261)(703.9324814,470.19613787)
\curveto(704.25247445,469.97613286)(704.5074742,469.66613317)(704.6974814,469.26613787)
\curveto(704.73747397,469.17613366)(704.76747394,469.08113375)(704.7874814,468.98113787)
\curveto(704.81747389,468.89113394)(704.84247386,468.80113403)(704.8624814,468.71113787)
\curveto(704.87247383,468.66113417)(704.87747383,468.61113422)(704.8774814,468.56113787)
\curveto(704.88747382,468.52113431)(704.89747381,468.47613436)(704.9074814,468.42613787)
\curveto(704.91747379,468.37613446)(704.91747379,468.32613451)(704.9074814,468.27613787)
\curveto(704.89747381,468.22613461)(704.9024738,468.17613466)(704.9224814,468.12613787)
\curveto(704.93247377,468.07613476)(704.93747377,468.01613482)(704.9374814,467.94613787)
\curveto(704.93747377,467.87613496)(704.92747378,467.81613502)(704.9074814,467.76613787)
\lineto(704.9074814,467.54113787)
\lineto(704.8474814,467.30113787)
\curveto(704.83747387,467.2311356)(704.82247388,467.16113567)(704.8024814,467.09113787)
\curveto(704.77247393,467.00113583)(704.74247396,466.91613592)(704.7124814,466.83613787)
\curveto(704.69247401,466.75613608)(704.66247404,466.67613616)(704.6224814,466.59613787)
\curveto(704.6024741,466.5361363)(704.57247413,466.47613636)(704.5324814,466.41613787)
\curveto(704.5024742,466.36613647)(704.46747424,466.31613652)(704.4274814,466.26613787)
\curveto(704.22747448,465.95613688)(703.97747473,465.69613714)(703.6774814,465.48613787)
\curveto(703.37747533,465.28613755)(703.03247567,465.12113771)(702.6424814,464.99113787)
\curveto(702.52247618,464.95113788)(702.39247631,464.92613791)(702.2524814,464.91613787)
\curveto(702.12247658,464.89613794)(701.98747672,464.87113796)(701.8474814,464.84113787)
\curveto(701.77747693,464.831138)(701.707477,464.82613801)(701.6374814,464.82613787)
\curveto(701.57747713,464.82613801)(701.51247719,464.82113801)(701.4424814,464.81113787)
\curveto(701.4024773,464.80113803)(701.34247736,464.79613804)(701.2624814,464.79613787)
\curveto(701.19247751,464.79613804)(701.14247756,464.80113803)(701.1124814,464.81113787)
\curveto(701.06247764,464.82113801)(701.01747769,464.82613801)(700.9774814,464.82613787)
\lineto(700.8574814,464.82613787)
\curveto(700.75747795,464.84613799)(700.65747805,464.86113797)(700.5574814,464.87113787)
\curveto(700.45747825,464.88113795)(700.36247834,464.89613794)(700.2724814,464.91613787)
\curveto(700.16247854,464.94613789)(700.05247865,464.97113786)(699.9424814,464.99113787)
\curveto(699.84247886,465.02113781)(699.73747897,465.06113777)(699.6274814,465.11113787)
\curveto(699.25747945,465.27113756)(698.94247976,465.47113736)(698.6824814,465.71113787)
\curveto(698.42248028,465.96113687)(698.21248049,466.27113656)(698.0524814,466.64113787)
\curveto(698.01248069,466.7311361)(697.97748073,466.82613601)(697.9474814,466.92613787)
\curveto(697.91748079,467.02613581)(697.88748082,467.1311357)(697.8574814,467.24113787)
\curveto(697.83748087,467.29113554)(697.82748088,467.34113549)(697.8274814,467.39113787)
\curveto(697.82748088,467.45113538)(697.81748089,467.51113532)(697.7974814,467.57113787)
\curveto(697.77748093,467.6311352)(697.76748094,467.71113512)(697.7674814,467.81113787)
\curveto(697.76748094,467.91113492)(697.78248092,467.98613485)(697.8124814,468.03613787)
\curveto(697.82248088,468.06613477)(697.83748087,468.09113474)(697.8574814,468.11113787)
\lineto(697.9174814,468.17113787)
\curveto(697.95748075,468.19113464)(698.01748069,468.20613463)(698.0974814,468.21613787)
\curveto(698.18748052,468.22613461)(698.27748043,468.2311346)(698.3674814,468.23113787)
\curveto(698.45748025,468.2311346)(698.54248016,468.22613461)(698.6224814,468.21613787)
\curveto(698.71247999,468.20613463)(698.77747993,468.19613464)(698.8174814,468.18613787)
\curveto(698.83747987,468.16613467)(698.85747985,468.15113468)(698.8774814,468.14113787)
\curveto(698.89747981,468.14113469)(698.91747979,468.1311347)(698.9374814,468.11113787)
\curveto(699.0074797,468.02113481)(699.04747966,467.90613493)(699.0574814,467.76613787)
\curveto(699.07747963,467.62613521)(699.1074796,467.50113533)(699.1474814,467.39113787)
\lineto(699.2974814,467.03113787)
\curveto(699.34747936,466.92113591)(699.41247929,466.81613602)(699.4924814,466.71613787)
\curveto(699.51247919,466.68613615)(699.53247917,466.66113617)(699.5524814,466.64113787)
\curveto(699.58247912,466.62113621)(699.6074791,466.59613624)(699.6274814,466.56613787)
\curveto(699.66747904,466.50613633)(699.702479,466.46113637)(699.7324814,466.43113787)
\curveto(699.77247893,466.40113643)(699.8074789,466.37113646)(699.8374814,466.34113787)
\curveto(699.87747883,466.31113652)(699.92247878,466.28113655)(699.9724814,466.25113787)
\curveto(700.06247864,466.19113664)(700.15747855,466.14113669)(700.2574814,466.10113787)
\lineto(700.5874814,465.98113787)
\curveto(700.73747797,465.9311369)(700.93747777,465.90113693)(701.1874814,465.89113787)
\curveto(701.43747727,465.88113695)(701.64747706,465.90113693)(701.8174814,465.95113787)
\curveto(701.89747681,465.97113686)(701.96747674,465.98613685)(702.0274814,465.99613787)
\lineto(702.2374814,466.05613787)
\curveto(702.51747619,466.17613666)(702.75747595,466.32613651)(702.9574814,466.50613787)
\curveto(703.16747554,466.68613615)(703.33247537,466.91613592)(703.4524814,467.19613787)
\curveto(703.48247522,467.26613557)(703.5024752,467.3361355)(703.5124814,467.40613787)
\lineto(703.5724814,467.64613787)
\curveto(703.61247509,467.78613505)(703.62247508,467.94613489)(703.6024814,468.12613787)
\curveto(703.58247512,468.31613452)(703.55247515,468.46613437)(703.5124814,468.57613787)
\curveto(703.38247532,468.95613388)(703.19747551,469.24613359)(702.9574814,469.44613787)
\curveto(702.72747598,469.64613319)(702.41747629,469.80613303)(702.0274814,469.92613787)
\curveto(701.91747679,469.95613288)(701.79747691,469.97613286)(701.6674814,469.98613787)
\curveto(701.54747716,469.99613284)(701.42247728,470.00113283)(701.2924814,470.00113787)
\curveto(701.13247757,470.00113283)(700.99247771,470.00613283)(700.8724814,470.01613787)
\curveto(700.75247795,470.02613281)(700.66747804,470.08613275)(700.6174814,470.19613787)
\curveto(700.59747811,470.22613261)(700.58747812,470.26113257)(700.5874814,470.30113787)
\lineto(700.5874814,470.43613787)
\curveto(700.57747813,470.5361323)(700.57747813,470.6311322)(700.5874814,470.72113787)
\curveto(700.6074781,470.81113202)(700.64747806,470.87613196)(700.7074814,470.91613787)
\curveto(700.74747796,470.94613189)(700.78747792,470.96613187)(700.8274814,470.97613787)
\curveto(700.87747783,470.98613185)(700.93247777,470.99613184)(700.9924814,471.00613787)
\curveto(701.01247769,471.01613182)(701.03747767,471.01613182)(701.0674814,471.00613787)
\curveto(701.09747761,471.00613183)(701.12247758,471.01113182)(701.1424814,471.02113787)
\lineto(701.2774814,471.02113787)
\curveto(701.38747732,471.04113179)(701.48747722,471.05113178)(701.5774814,471.05113787)
\curveto(701.67747703,471.06113177)(701.77247693,471.08113175)(701.8624814,471.11113787)
\curveto(702.18247652,471.22113161)(702.43747627,471.36613147)(702.6274814,471.54613787)
\curveto(702.81747589,471.72613111)(702.96747574,471.97613086)(703.0774814,472.29613787)
\curveto(703.1074756,472.39613044)(703.12747558,472.52113031)(703.1374814,472.67113787)
\curveto(703.15747555,472.83113)(703.15247555,472.97612986)(703.1224814,473.10613787)
\curveto(703.1024756,473.17612966)(703.08247562,473.24112959)(703.0624814,473.30113787)
\curveto(703.05247565,473.37112946)(703.03247567,473.4361294)(703.0024814,473.49613787)
\curveto(702.9024758,473.7361291)(702.75747595,473.92612891)(702.5674814,474.06613787)
\curveto(702.37747633,474.20612863)(702.15247655,474.31612852)(701.8924814,474.39613787)
\curveto(701.83247687,474.41612842)(701.77247693,474.42612841)(701.7124814,474.42613787)
\curveto(701.65247705,474.42612841)(701.58747712,474.4361284)(701.5174814,474.45613787)
\curveto(701.43747727,474.47612836)(701.34247736,474.48612835)(701.2324814,474.48613787)
\curveto(701.12247758,474.48612835)(701.02747768,474.47612836)(700.9474814,474.45613787)
\curveto(700.89747781,474.4361284)(700.84747786,474.42612841)(700.7974814,474.42613787)
\curveto(700.75747795,474.42612841)(700.71247799,474.41612842)(700.6624814,474.39613787)
\curveto(700.48247822,474.34612849)(700.31247839,474.27112856)(700.1524814,474.17113787)
\curveto(700.0024787,474.08112875)(699.87247883,473.96612887)(699.7624814,473.82613787)
\curveto(699.67247903,473.70612913)(699.59247911,473.57612926)(699.5224814,473.43613787)
\curveto(699.45247925,473.29612954)(699.38747932,473.14112969)(699.3274814,472.97113787)
\curveto(699.29747941,472.86112997)(699.27747943,472.74113009)(699.2674814,472.61113787)
\curveto(699.25747945,472.49113034)(699.22247948,472.39113044)(699.1624814,472.31113787)
\curveto(699.14247956,472.27113056)(699.08247962,472.2311306)(698.9824814,472.19113787)
\curveto(698.94247976,472.18113065)(698.88247982,472.17113066)(698.8024814,472.16113787)
\lineto(698.5474814,472.16113787)
\curveto(698.45748025,472.17113066)(698.37248033,472.18113065)(698.2924814,472.19113787)
\curveto(698.22248048,472.20113063)(698.17248053,472.21613062)(698.1424814,472.23613787)
\curveto(698.1024806,472.26613057)(698.06748064,472.32113051)(698.0374814,472.40113787)
\curveto(698.0074807,472.48113035)(698.0024807,472.56613027)(698.0224814,472.65613787)
\curveto(698.03248067,472.70613013)(698.03748067,472.75613008)(698.0374814,472.80613787)
\lineto(698.0674814,472.98613787)
\curveto(698.09748061,473.08612975)(698.12248058,473.18612965)(698.1424814,473.28613787)
\curveto(698.17248053,473.38612945)(698.2074805,473.47612936)(698.2474814,473.55613787)
\curveto(698.29748041,473.66612917)(698.34248036,473.77112906)(698.3824814,473.87113787)
\curveto(698.42248028,473.98112885)(698.47248023,474.08612875)(698.5324814,474.18613787)
\curveto(698.86247984,474.72612811)(699.33247937,475.12112771)(699.9424814,475.37113787)
\curveto(700.06247864,475.42112741)(700.18747852,475.45612738)(700.3174814,475.47613787)
\curveto(700.45747825,475.49612734)(700.59747811,475.52112731)(700.7374814,475.55113787)
\curveto(700.79747791,475.56112727)(700.85747785,475.56612727)(700.9174814,475.56613787)
\curveto(700.98747772,475.56612727)(701.05247765,475.57112726)(701.1124814,475.58113787)
}
}
{
\newrgbcolor{curcolor}{0 0 0}
\pscustom[linestyle=none,fillstyle=solid,fillcolor=curcolor]
{
\newpath
\moveto(713.34709078,469.29613787)
\curveto(713.37708305,469.17613366)(713.40208303,469.0361338)(713.42209078,468.87613787)
\curveto(713.44208299,468.71613412)(713.45208298,468.55113428)(713.45209078,468.38113787)
\curveto(713.45208298,468.21113462)(713.44208299,468.04613479)(713.42209078,467.88613787)
\curveto(713.40208303,467.72613511)(713.37708305,467.58613525)(713.34709078,467.46613787)
\curveto(713.30708312,467.32613551)(713.27208316,467.20113563)(713.24209078,467.09113787)
\curveto(713.21208322,466.98113585)(713.17208326,466.87113596)(713.12209078,466.76113787)
\curveto(712.85208358,466.12113671)(712.43708399,465.6361372)(711.87709078,465.30613787)
\curveto(711.79708463,465.24613759)(711.71208472,465.19613764)(711.62209078,465.15613787)
\curveto(711.5320849,465.12613771)(711.432085,465.09113774)(711.32209078,465.05113787)
\curveto(711.21208522,465.00113783)(711.09208534,464.96613787)(710.96209078,464.94613787)
\curveto(710.84208559,464.91613792)(710.71208572,464.88613795)(710.57209078,464.85613787)
\curveto(710.51208592,464.836138)(710.45208598,464.831138)(710.39209078,464.84113787)
\curveto(710.34208609,464.85113798)(710.28208615,464.84613799)(710.21209078,464.82613787)
\curveto(710.19208624,464.81613802)(710.16708626,464.81613802)(710.13709078,464.82613787)
\curveto(710.10708632,464.82613801)(710.08208635,464.82113801)(710.06209078,464.81113787)
\lineto(709.91209078,464.81113787)
\curveto(709.84208659,464.80113803)(709.79208664,464.80113803)(709.76209078,464.81113787)
\curveto(709.72208671,464.82113801)(709.67708675,464.82613801)(709.62709078,464.82613787)
\curveto(709.58708684,464.81613802)(709.54708688,464.81613802)(709.50709078,464.82613787)
\curveto(709.41708701,464.84613799)(709.3270871,464.86113797)(709.23709078,464.87113787)
\curveto(709.14708728,464.87113796)(709.05708737,464.88113795)(708.96709078,464.90113787)
\curveto(708.87708755,464.9311379)(708.78708764,464.95613788)(708.69709078,464.97613787)
\curveto(708.60708782,464.99613784)(708.52208791,465.02613781)(708.44209078,465.06613787)
\curveto(708.20208823,465.17613766)(707.97708845,465.30613753)(707.76709078,465.45613787)
\curveto(707.55708887,465.61613722)(707.37708905,465.79613704)(707.22709078,465.99613787)
\curveto(707.10708932,466.16613667)(707.00208943,466.34113649)(706.91209078,466.52113787)
\curveto(706.82208961,466.70113613)(706.7320897,466.89113594)(706.64209078,467.09113787)
\curveto(706.60208983,467.19113564)(706.56708986,467.29113554)(706.53709078,467.39113787)
\curveto(706.51708991,467.50113533)(706.49208994,467.61113522)(706.46209078,467.72113787)
\curveto(706.42209001,467.86113497)(706.39709003,468.00113483)(706.38709078,468.14113787)
\curveto(706.37709005,468.28113455)(706.35709007,468.42113441)(706.32709078,468.56113787)
\curveto(706.31709011,468.67113416)(706.30709012,468.77113406)(706.29709078,468.86113787)
\curveto(706.29709013,468.96113387)(706.28709014,469.06113377)(706.26709078,469.16113787)
\lineto(706.26709078,469.25113787)
\curveto(706.27709015,469.28113355)(706.27709015,469.30613353)(706.26709078,469.32613787)
\lineto(706.26709078,469.53613787)
\curveto(706.24709018,469.59613324)(706.23709019,469.66113317)(706.23709078,469.73113787)
\curveto(706.24709018,469.81113302)(706.25209018,469.88613295)(706.25209078,469.95613787)
\lineto(706.25209078,470.10613787)
\curveto(706.25209018,470.15613268)(706.25709017,470.20613263)(706.26709078,470.25613787)
\lineto(706.26709078,470.63113787)
\curveto(706.27709015,470.66113217)(706.27709015,470.69613214)(706.26709078,470.73613787)
\curveto(706.26709016,470.77613206)(706.27209016,470.81613202)(706.28209078,470.85613787)
\curveto(706.30209013,470.96613187)(706.31709011,471.07613176)(706.32709078,471.18613787)
\curveto(706.33709009,471.30613153)(706.34709008,471.42113141)(706.35709078,471.53113787)
\curveto(706.39709003,471.68113115)(706.42209001,471.82613101)(706.43209078,471.96613787)
\curveto(706.45208998,472.11613072)(706.48208995,472.26113057)(706.52209078,472.40113787)
\curveto(706.61208982,472.70113013)(706.70708972,472.98612985)(706.80709078,473.25613787)
\curveto(706.90708952,473.52612931)(707.0320894,473.77612906)(707.18209078,474.00613787)
\curveto(707.38208905,474.32612851)(707.6270888,474.60612823)(707.91709078,474.84613787)
\curveto(708.20708822,475.08612775)(708.54708788,475.27112756)(708.93709078,475.40113787)
\curveto(709.04708738,475.44112739)(709.15708727,475.46612737)(709.26709078,475.47613787)
\curveto(709.38708704,475.49612734)(709.50708692,475.52112731)(709.62709078,475.55113787)
\curveto(709.69708673,475.56112727)(709.76208667,475.56612727)(709.82209078,475.56613787)
\curveto(709.88208655,475.56612727)(709.94708648,475.57112726)(710.01709078,475.58113787)
\curveto(710.71708571,475.60112723)(711.29208514,475.48612735)(711.74209078,475.23613787)
\curveto(712.19208424,474.98612785)(712.53708389,474.6361282)(712.77709078,474.18613787)
\curveto(712.88708354,473.95612888)(712.98708344,473.68112915)(713.07709078,473.36113787)
\curveto(713.09708333,473.29112954)(713.09708333,473.21612962)(713.07709078,473.13613787)
\curveto(713.06708336,473.06612977)(713.04208339,473.01612982)(713.00209078,472.98613787)
\curveto(712.97208346,472.95612988)(712.91208352,472.9311299)(712.82209078,472.91113787)
\curveto(712.7320837,472.90112993)(712.6320838,472.89112994)(712.52209078,472.88113787)
\curveto(712.42208401,472.88112995)(712.32208411,472.88612995)(712.22209078,472.89613787)
\curveto(712.1320843,472.90612993)(712.06708436,472.92612991)(712.02709078,472.95613787)
\curveto(711.91708451,473.02612981)(711.83708459,473.1361297)(711.78709078,473.28613787)
\curveto(711.74708468,473.4361294)(711.69208474,473.56612927)(711.62209078,473.67613787)
\curveto(711.432085,473.98612885)(711.15208528,474.21612862)(710.78209078,474.36613787)
\curveto(710.71208572,474.39612844)(710.63708579,474.41612842)(710.55709078,474.42613787)
\curveto(710.48708594,474.4361284)(710.41208602,474.45112838)(710.33209078,474.47113787)
\curveto(710.28208615,474.48112835)(710.21208622,474.48612835)(710.12209078,474.48613787)
\curveto(710.04208639,474.48612835)(709.97708645,474.48112835)(709.92709078,474.47113787)
\curveto(709.88708654,474.45112838)(709.85208658,474.44612839)(709.82209078,474.45613787)
\curveto(709.79208664,474.46612837)(709.75708667,474.46612837)(709.71709078,474.45613787)
\lineto(709.47709078,474.39613787)
\curveto(709.40708702,474.37612846)(709.33708709,474.35112848)(709.26709078,474.32113787)
\curveto(708.88708754,474.16112867)(708.59708783,473.95112888)(708.39709078,473.69113787)
\curveto(708.20708822,473.4311294)(708.0320884,473.11612972)(707.87209078,472.74613787)
\curveto(707.84208859,472.66613017)(707.81708861,472.58613025)(707.79709078,472.50613787)
\curveto(707.78708864,472.42613041)(707.76708866,472.34613049)(707.73709078,472.26613787)
\curveto(707.70708872,472.15613068)(707.68208875,472.04113079)(707.66209078,471.92113787)
\curveto(707.65208878,471.80113103)(707.6320888,471.68113115)(707.60209078,471.56113787)
\curveto(707.58208885,471.51113132)(707.57208886,471.46113137)(707.57209078,471.41113787)
\curveto(707.58208885,471.36113147)(707.57708885,471.31113152)(707.55709078,471.26113787)
\curveto(707.54708888,471.20113163)(707.54708888,471.12113171)(707.55709078,471.02113787)
\curveto(707.56708886,470.9311319)(707.58208885,470.87613196)(707.60209078,470.85613787)
\curveto(707.62208881,470.81613202)(707.65208878,470.79613204)(707.69209078,470.79613787)
\curveto(707.74208869,470.79613204)(707.78708864,470.80613203)(707.82709078,470.82613787)
\curveto(707.89708853,470.86613197)(707.95708847,470.91113192)(708.00709078,470.96113787)
\curveto(708.05708837,471.01113182)(708.11708831,471.06113177)(708.18709078,471.11113787)
\lineto(708.24709078,471.17113787)
\curveto(708.27708815,471.20113163)(708.30708812,471.22613161)(708.33709078,471.24613787)
\curveto(708.56708786,471.40613143)(708.84208759,471.54113129)(709.16209078,471.65113787)
\curveto(709.2320872,471.67113116)(709.30208713,471.68613115)(709.37209078,471.69613787)
\curveto(709.44208699,471.70613113)(709.51708691,471.72113111)(709.59709078,471.74113787)
\curveto(709.63708679,471.74113109)(709.67208676,471.74613109)(709.70209078,471.75613787)
\curveto(709.7320867,471.76613107)(709.76708666,471.76613107)(709.80709078,471.75613787)
\curveto(709.85708657,471.75613108)(709.89708653,471.76613107)(709.92709078,471.78613787)
\lineto(710.09209078,471.78613787)
\lineto(710.18209078,471.78613787)
\curveto(710.2320862,471.79613104)(710.27208616,471.79613104)(710.30209078,471.78613787)
\curveto(710.35208608,471.77613106)(710.40208603,471.77113106)(710.45209078,471.77113787)
\curveto(710.51208592,471.78113105)(710.56708586,471.78113105)(710.61709078,471.77113787)
\curveto(710.7270857,471.74113109)(710.8320856,471.72113111)(710.93209078,471.71113787)
\curveto(711.04208539,471.70113113)(711.14708528,471.67613116)(711.24709078,471.63613787)
\curveto(711.66708476,471.49613134)(712.01208442,471.31113152)(712.28209078,471.08113787)
\curveto(712.55208388,470.86113197)(712.79208364,470.57613226)(713.00209078,470.22613787)
\curveto(713.08208335,470.08613275)(713.14708328,469.9361329)(713.19709078,469.77613787)
\curveto(713.24708318,469.62613321)(713.29708313,469.46613337)(713.34709078,469.29613787)
\moveto(712.10209078,467.99113787)
\curveto(712.11208432,468.04113479)(712.11708431,468.08613475)(712.11709078,468.12613787)
\lineto(712.11709078,468.27613787)
\curveto(712.11708431,468.58613425)(712.07708435,468.87113396)(711.99709078,469.13113787)
\curveto(711.97708445,469.19113364)(711.95708447,469.24613359)(711.93709078,469.29613787)
\curveto(711.9270845,469.35613348)(711.91208452,469.41113342)(711.89209078,469.46113787)
\curveto(711.67208476,469.95113288)(711.3270851,470.30113253)(710.85709078,470.51113787)
\curveto(710.77708565,470.54113229)(710.69708573,470.56613227)(710.61709078,470.58613787)
\lineto(710.37709078,470.64613787)
\curveto(710.29708613,470.66613217)(710.20708622,470.67613216)(710.10709078,470.67613787)
\lineto(709.79209078,470.67613787)
\curveto(709.77208666,470.65613218)(709.7320867,470.64613219)(709.67209078,470.64613787)
\curveto(709.62208681,470.65613218)(709.57708685,470.65613218)(709.53709078,470.64613787)
\lineto(709.29709078,470.58613787)
\curveto(709.2270872,470.57613226)(709.15708727,470.55613228)(709.08709078,470.52613787)
\curveto(708.48708794,470.26613257)(708.08208835,469.80113303)(707.87209078,469.13113787)
\curveto(707.84208859,469.05113378)(707.82208861,468.97113386)(707.81209078,468.89113787)
\curveto(707.80208863,468.81113402)(707.78708864,468.72613411)(707.76709078,468.63613787)
\lineto(707.76709078,468.48613787)
\curveto(707.75708867,468.44613439)(707.75208868,468.37613446)(707.75209078,468.27613787)
\curveto(707.75208868,468.04613479)(707.77208866,467.85113498)(707.81209078,467.69113787)
\curveto(707.8320886,467.62113521)(707.84708858,467.55613528)(707.85709078,467.49613787)
\curveto(707.86708856,467.4361354)(707.88708854,467.37113546)(707.91709078,467.30113787)
\curveto(708.0270884,467.02113581)(708.17208826,466.77613606)(708.35209078,466.56613787)
\curveto(708.5320879,466.36613647)(708.76708766,466.20613663)(709.05709078,466.08613787)
\lineto(709.29709078,465.99613787)
\lineto(709.53709078,465.93613787)
\curveto(709.58708684,465.91613692)(709.6270868,465.91113692)(709.65709078,465.92113787)
\curveto(709.69708673,465.9311369)(709.74208669,465.92613691)(709.79209078,465.90613787)
\curveto(709.82208661,465.89613694)(709.87708655,465.89113694)(709.95709078,465.89113787)
\curveto(710.03708639,465.89113694)(710.09708633,465.89613694)(710.13709078,465.90613787)
\curveto(710.24708618,465.92613691)(710.35208608,465.94113689)(710.45209078,465.95113787)
\curveto(710.55208588,465.96113687)(710.64708578,465.99113684)(710.73709078,466.04113787)
\curveto(711.26708516,466.24113659)(711.65708477,466.61613622)(711.90709078,467.16613787)
\curveto(711.94708448,467.26613557)(711.97708445,467.37113546)(711.99709078,467.48113787)
\lineto(712.08709078,467.81113787)
\curveto(712.08708434,467.89113494)(712.09208434,467.95113488)(712.10209078,467.99113787)
}
}
{
\newrgbcolor{curcolor}{0 0 0}
\pscustom[linestyle=none,fillstyle=solid,fillcolor=curcolor]
{
\newpath
\moveto(724.50170015,473.49613787)
\curveto(724.30168985,473.20612963)(724.09169006,472.92112991)(723.87170015,472.64113787)
\curveto(723.66169049,472.36113047)(723.4566907,472.07613076)(723.25670015,471.78613787)
\curveto(722.6566915,470.9361319)(722.0516921,470.09613274)(721.44170015,469.26613787)
\curveto(720.83169332,468.44613439)(720.22669393,467.61113522)(719.62670015,466.76113787)
\lineto(719.11670015,466.04113787)
\lineto(718.60670015,465.35113787)
\curveto(718.52669563,465.24113759)(718.44669571,465.12613771)(718.36670015,465.00613787)
\curveto(718.28669587,464.88613795)(718.19169596,464.79113804)(718.08170015,464.72113787)
\curveto(718.04169611,464.70113813)(717.97669618,464.68613815)(717.88670015,464.67613787)
\curveto(717.80669635,464.65613818)(717.71669644,464.64613819)(717.61670015,464.64613787)
\curveto(717.51669664,464.64613819)(717.42169673,464.65113818)(717.33170015,464.66113787)
\curveto(717.2516969,464.67113816)(717.19169696,464.69113814)(717.15170015,464.72113787)
\curveto(717.12169703,464.74113809)(717.09669706,464.77613806)(717.07670015,464.82613787)
\curveto(717.06669709,464.86613797)(717.07169708,464.91113792)(717.09170015,464.96113787)
\curveto(717.13169702,465.04113779)(717.17669698,465.11613772)(717.22670015,465.18613787)
\curveto(717.28669687,465.26613757)(717.34169681,465.34613749)(717.39170015,465.42613787)
\curveto(717.63169652,465.76613707)(717.87669628,466.10113673)(718.12670015,466.43113787)
\curveto(718.37669578,466.76113607)(718.61669554,467.09613574)(718.84670015,467.43613787)
\curveto(719.00669515,467.65613518)(719.16669499,467.87113496)(719.32670015,468.08113787)
\curveto(719.48669467,468.29113454)(719.64669451,468.50613433)(719.80670015,468.72613787)
\curveto(720.16669399,469.24613359)(720.53169362,469.75613308)(720.90170015,470.25613787)
\curveto(721.27169288,470.75613208)(721.64169251,471.26613157)(722.01170015,471.78613787)
\curveto(722.151692,471.98613085)(722.29169186,472.18113065)(722.43170015,472.37113787)
\curveto(722.58169157,472.56113027)(722.72669143,472.75613008)(722.86670015,472.95613787)
\curveto(723.07669108,473.25612958)(723.29169086,473.55612928)(723.51170015,473.85613787)
\lineto(724.17170015,474.75613787)
\lineto(724.35170015,475.02613787)
\lineto(724.56170015,475.29613787)
\lineto(724.68170015,475.47613787)
\curveto(724.73168942,475.5361273)(724.78168937,475.59112724)(724.83170015,475.64113787)
\curveto(724.90168925,475.69112714)(724.97668918,475.72612711)(725.05670015,475.74613787)
\curveto(725.07668908,475.75612708)(725.10168905,475.75612708)(725.13170015,475.74613787)
\curveto(725.17168898,475.74612709)(725.20168895,475.75612708)(725.22170015,475.77613787)
\curveto(725.34168881,475.77612706)(725.47668868,475.77112706)(725.62670015,475.76113787)
\curveto(725.77668838,475.76112707)(725.86668829,475.71612712)(725.89670015,475.62613787)
\curveto(725.91668824,475.59612724)(725.92168823,475.56112727)(725.91170015,475.52113787)
\curveto(725.90168825,475.48112735)(725.88668827,475.45112738)(725.86670015,475.43113787)
\curveto(725.82668833,475.35112748)(725.78668837,475.28112755)(725.74670015,475.22113787)
\curveto(725.70668845,475.16112767)(725.66168849,475.10112773)(725.61170015,475.04113787)
\lineto(725.04170015,474.26113787)
\curveto(724.86168929,474.01112882)(724.68168947,473.75612908)(724.50170015,473.49613787)
\moveto(717.64670015,469.59613787)
\curveto(717.59669656,469.61613322)(717.54669661,469.62113321)(717.49670015,469.61113787)
\curveto(717.44669671,469.60113323)(717.39669676,469.60613323)(717.34670015,469.62613787)
\curveto(717.23669692,469.64613319)(717.13169702,469.66613317)(717.03170015,469.68613787)
\curveto(716.94169721,469.71613312)(716.84669731,469.75613308)(716.74670015,469.80613787)
\curveto(716.41669774,469.94613289)(716.16169799,470.14113269)(715.98170015,470.39113787)
\curveto(715.80169835,470.65113218)(715.6566985,470.96113187)(715.54670015,471.32113787)
\curveto(715.51669864,471.40113143)(715.49669866,471.48113135)(715.48670015,471.56113787)
\curveto(715.47669868,471.65113118)(715.46169869,471.7361311)(715.44170015,471.81613787)
\curveto(715.43169872,471.86613097)(715.42669873,471.9311309)(715.42670015,472.01113787)
\curveto(715.41669874,472.04113079)(715.41169874,472.07113076)(715.41170015,472.10113787)
\curveto(715.41169874,472.14113069)(715.40669875,472.17613066)(715.39670015,472.20613787)
\lineto(715.39670015,472.35613787)
\curveto(715.38669877,472.40613043)(715.38169877,472.46613037)(715.38170015,472.53613787)
\curveto(715.38169877,472.61613022)(715.38669877,472.68113015)(715.39670015,472.73113787)
\lineto(715.39670015,472.89613787)
\curveto(715.41669874,472.94612989)(715.42169873,472.99112984)(715.41170015,473.03113787)
\curveto(715.41169874,473.08112975)(715.41669874,473.12612971)(715.42670015,473.16613787)
\curveto(715.43669872,473.20612963)(715.44169871,473.24112959)(715.44170015,473.27113787)
\curveto(715.44169871,473.31112952)(715.44669871,473.35112948)(715.45670015,473.39113787)
\curveto(715.48669867,473.50112933)(715.50669865,473.61112922)(715.51670015,473.72113787)
\curveto(715.53669862,473.84112899)(715.57169858,473.95612888)(715.62170015,474.06613787)
\curveto(715.76169839,474.40612843)(715.92169823,474.68112815)(716.10170015,474.89113787)
\curveto(716.29169786,475.11112772)(716.56169759,475.29112754)(716.91170015,475.43113787)
\curveto(716.99169716,475.46112737)(717.07669708,475.48112735)(717.16670015,475.49113787)
\curveto(717.2566969,475.51112732)(717.3516968,475.5311273)(717.45170015,475.55113787)
\curveto(717.48169667,475.56112727)(717.53669662,475.56112727)(717.61670015,475.55113787)
\curveto(717.69669646,475.55112728)(717.74669641,475.56112727)(717.76670015,475.58113787)
\curveto(718.32669583,475.59112724)(718.77669538,475.48112735)(719.11670015,475.25113787)
\curveto(719.46669469,475.02112781)(719.72669443,474.71612812)(719.89670015,474.33613787)
\curveto(719.93669422,474.24612859)(719.97169418,474.15112868)(720.00170015,474.05113787)
\curveto(720.03169412,473.95112888)(720.0566941,473.85112898)(720.07670015,473.75113787)
\curveto(720.09669406,473.72112911)(720.10169405,473.69112914)(720.09170015,473.66113787)
\curveto(720.09169406,473.6311292)(720.09669406,473.60112923)(720.10670015,473.57113787)
\curveto(720.13669402,473.46112937)(720.156694,473.3361295)(720.16670015,473.19613787)
\curveto(720.17669398,473.06612977)(720.18669397,472.9311299)(720.19670015,472.79113787)
\lineto(720.19670015,472.62613787)
\curveto(720.20669395,472.56613027)(720.20669395,472.51113032)(720.19670015,472.46113787)
\curveto(720.18669397,472.41113042)(720.18169397,472.36113047)(720.18170015,472.31113787)
\lineto(720.18170015,472.17613787)
\curveto(720.17169398,472.1361307)(720.16669399,472.09613074)(720.16670015,472.05613787)
\curveto(720.17669398,472.01613082)(720.17169398,471.97113086)(720.15170015,471.92113787)
\curveto(720.13169402,471.81113102)(720.11169404,471.70613113)(720.09170015,471.60613787)
\curveto(720.08169407,471.50613133)(720.06169409,471.40613143)(720.03170015,471.30613787)
\curveto(719.90169425,470.94613189)(719.73669442,470.6311322)(719.53670015,470.36113787)
\curveto(719.33669482,470.09113274)(719.06169509,469.88613295)(718.71170015,469.74613787)
\curveto(718.63169552,469.71613312)(718.54669561,469.69113314)(718.45670015,469.67113787)
\lineto(718.18670015,469.61113787)
\curveto(718.13669602,469.60113323)(718.09169606,469.59613324)(718.05170015,469.59613787)
\curveto(718.01169614,469.60613323)(717.97169618,469.60613323)(717.93170015,469.59613787)
\curveto(717.83169632,469.57613326)(717.73669642,469.57613326)(717.64670015,469.59613787)
\moveto(716.80670015,470.99113787)
\curveto(716.84669731,470.92113191)(716.88669727,470.85613198)(716.92670015,470.79613787)
\curveto(716.96669719,470.74613209)(717.01669714,470.69613214)(717.07670015,470.64613787)
\lineto(717.22670015,470.52613787)
\curveto(717.28669687,470.49613234)(717.3516968,470.47113236)(717.42170015,470.45113787)
\curveto(717.46169669,470.4311324)(717.49669666,470.42113241)(717.52670015,470.42113787)
\curveto(717.56669659,470.4311324)(717.60669655,470.42613241)(717.64670015,470.40613787)
\curveto(717.67669648,470.40613243)(717.71669644,470.40113243)(717.76670015,470.39113787)
\curveto(717.81669634,470.39113244)(717.8566963,470.39613244)(717.88670015,470.40613787)
\lineto(718.11170015,470.45113787)
\curveto(718.36169579,470.5311323)(718.54669561,470.65613218)(718.66670015,470.82613787)
\curveto(718.74669541,470.92613191)(718.81669534,471.05613178)(718.87670015,471.21613787)
\curveto(718.9566952,471.39613144)(719.01669514,471.62113121)(719.05670015,471.89113787)
\curveto(719.09669506,472.17113066)(719.11169504,472.45113038)(719.10170015,472.73113787)
\curveto(719.09169506,473.02112981)(719.06169509,473.29612954)(719.01170015,473.55613787)
\curveto(718.96169519,473.81612902)(718.88669527,474.02612881)(718.78670015,474.18613787)
\curveto(718.66669549,474.38612845)(718.51669564,474.5361283)(718.33670015,474.63613787)
\curveto(718.2566959,474.68612815)(718.16669599,474.71612812)(718.06670015,474.72613787)
\curveto(717.96669619,474.74612809)(717.86169629,474.75612808)(717.75170015,474.75613787)
\curveto(717.73169642,474.74612809)(717.70669645,474.74112809)(717.67670015,474.74113787)
\curveto(717.6566965,474.75112808)(717.63669652,474.75112808)(717.61670015,474.74113787)
\curveto(717.56669659,474.7311281)(717.52169663,474.72112811)(717.48170015,474.71113787)
\curveto(717.44169671,474.71112812)(717.40169675,474.70112813)(717.36170015,474.68113787)
\curveto(717.18169697,474.60112823)(717.03169712,474.48112835)(716.91170015,474.32113787)
\curveto(716.80169735,474.16112867)(716.71169744,473.98112885)(716.64170015,473.78113787)
\curveto(716.58169757,473.59112924)(716.53669762,473.36612947)(716.50670015,473.10613787)
\curveto(716.48669767,472.84612999)(716.48169767,472.58113025)(716.49170015,472.31113787)
\curveto(716.50169765,472.05113078)(716.53169762,471.80113103)(716.58170015,471.56113787)
\curveto(716.64169751,471.3311315)(716.71669744,471.14113169)(716.80670015,470.99113787)
\moveto(727.60670015,468.00613787)
\curveto(727.61668654,467.95613488)(727.62168653,467.86613497)(727.62170015,467.73613787)
\curveto(727.62168653,467.60613523)(727.61168654,467.51613532)(727.59170015,467.46613787)
\curveto(727.57168658,467.41613542)(727.56668659,467.36113547)(727.57670015,467.30113787)
\curveto(727.58668657,467.25113558)(727.58668657,467.20113563)(727.57670015,467.15113787)
\curveto(727.53668662,467.01113582)(727.50668665,466.87613596)(727.48670015,466.74613787)
\curveto(727.47668668,466.61613622)(727.44668671,466.49613634)(727.39670015,466.38613787)
\curveto(727.2566869,466.0361368)(727.09168706,465.74113709)(726.90170015,465.50113787)
\curveto(726.71168744,465.27113756)(726.44168771,465.08613775)(726.09170015,464.94613787)
\curveto(726.01168814,464.91613792)(725.92668823,464.89613794)(725.83670015,464.88613787)
\curveto(725.74668841,464.86613797)(725.66168849,464.84613799)(725.58170015,464.82613787)
\curveto(725.53168862,464.81613802)(725.48168867,464.81113802)(725.43170015,464.81113787)
\curveto(725.38168877,464.81113802)(725.33168882,464.80613803)(725.28170015,464.79613787)
\curveto(725.2516889,464.78613805)(725.20168895,464.78613805)(725.13170015,464.79613787)
\curveto(725.06168909,464.79613804)(725.01168914,464.80113803)(724.98170015,464.81113787)
\curveto(724.92168923,464.831138)(724.86168929,464.84113799)(724.80170015,464.84113787)
\curveto(724.7516894,464.831138)(724.70168945,464.836138)(724.65170015,464.85613787)
\curveto(724.56168959,464.87613796)(724.47168968,464.90113793)(724.38170015,464.93113787)
\curveto(724.30168985,464.95113788)(724.22168993,464.98113785)(724.14170015,465.02113787)
\curveto(723.82169033,465.16113767)(723.57169058,465.35613748)(723.39170015,465.60613787)
\curveto(723.21169094,465.86613697)(723.06169109,466.17113666)(722.94170015,466.52113787)
\curveto(722.92169123,466.60113623)(722.90669125,466.68613615)(722.89670015,466.77613787)
\curveto(722.88669127,466.86613597)(722.87169128,466.95113588)(722.85170015,467.03113787)
\curveto(722.84169131,467.06113577)(722.83669132,467.09113574)(722.83670015,467.12113787)
\lineto(722.83670015,467.22613787)
\curveto(722.81669134,467.30613553)(722.80669135,467.38613545)(722.80670015,467.46613787)
\lineto(722.80670015,467.60113787)
\curveto(722.78669137,467.70113513)(722.78669137,467.80113503)(722.80670015,467.90113787)
\lineto(722.80670015,468.08113787)
\curveto(722.81669134,468.1311347)(722.82169133,468.17613466)(722.82170015,468.21613787)
\curveto(722.82169133,468.26613457)(722.82669133,468.31113452)(722.83670015,468.35113787)
\curveto(722.84669131,468.39113444)(722.8516913,468.42613441)(722.85170015,468.45613787)
\curveto(722.8516913,468.49613434)(722.8566913,468.5361343)(722.86670015,468.57613787)
\lineto(722.92670015,468.90613787)
\curveto(722.94669121,469.02613381)(722.97669118,469.1361337)(723.01670015,469.23613787)
\curveto(723.156691,469.56613327)(723.31669084,469.84113299)(723.49670015,470.06113787)
\curveto(723.68669047,470.29113254)(723.94669021,470.47613236)(724.27670015,470.61613787)
\curveto(724.3566898,470.65613218)(724.44168971,470.68113215)(724.53170015,470.69113787)
\lineto(724.83170015,470.75113787)
\lineto(724.96670015,470.75113787)
\curveto(725.01668914,470.76113207)(725.06668909,470.76613207)(725.11670015,470.76613787)
\curveto(725.68668847,470.78613205)(726.14668801,470.68113215)(726.49670015,470.45113787)
\curveto(726.8566873,470.2311326)(727.12168703,469.9311329)(727.29170015,469.55113787)
\curveto(727.34168681,469.45113338)(727.38168677,469.35113348)(727.41170015,469.25113787)
\curveto(727.44168671,469.15113368)(727.47168668,469.04613379)(727.50170015,468.93613787)
\curveto(727.51168664,468.89613394)(727.51668664,468.86113397)(727.51670015,468.83113787)
\curveto(727.51668664,468.81113402)(727.52168663,468.78113405)(727.53170015,468.74113787)
\curveto(727.5516866,468.67113416)(727.56168659,468.59613424)(727.56170015,468.51613787)
\curveto(727.56168659,468.4361344)(727.57168658,468.35613448)(727.59170015,468.27613787)
\curveto(727.59168656,468.22613461)(727.59168656,468.18113465)(727.59170015,468.14113787)
\curveto(727.59168656,468.10113473)(727.59668656,468.05613478)(727.60670015,468.00613787)
\moveto(726.49670015,467.57113787)
\curveto(726.50668765,467.62113521)(726.51168764,467.69613514)(726.51170015,467.79613787)
\curveto(726.52168763,467.89613494)(726.51668764,467.97113486)(726.49670015,468.02113787)
\curveto(726.47668768,468.08113475)(726.47168768,468.1361347)(726.48170015,468.18613787)
\curveto(726.50168765,468.24613459)(726.50168765,468.30613453)(726.48170015,468.36613787)
\curveto(726.47168768,468.39613444)(726.46668769,468.4311344)(726.46670015,468.47113787)
\curveto(726.46668769,468.51113432)(726.46168769,468.55113428)(726.45170015,468.59113787)
\curveto(726.43168772,468.67113416)(726.41168774,468.74613409)(726.39170015,468.81613787)
\curveto(726.38168777,468.89613394)(726.36668779,468.97613386)(726.34670015,469.05613787)
\curveto(726.31668784,469.11613372)(726.29168786,469.17613366)(726.27170015,469.23613787)
\curveto(726.2516879,469.29613354)(726.22168793,469.35613348)(726.18170015,469.41613787)
\curveto(726.08168807,469.58613325)(725.9516882,469.72113311)(725.79170015,469.82113787)
\curveto(725.71168844,469.87113296)(725.61668854,469.90613293)(725.50670015,469.92613787)
\curveto(725.39668876,469.94613289)(725.27168888,469.95613288)(725.13170015,469.95613787)
\curveto(725.11168904,469.94613289)(725.08668907,469.94113289)(725.05670015,469.94113787)
\curveto(725.02668913,469.95113288)(724.99668916,469.95113288)(724.96670015,469.94113787)
\lineto(724.81670015,469.88113787)
\curveto(724.76668939,469.87113296)(724.72168943,469.85613298)(724.68170015,469.83613787)
\curveto(724.49168966,469.72613311)(724.34668981,469.58113325)(724.24670015,469.40113787)
\curveto(724.15669,469.22113361)(724.07669008,469.01613382)(724.00670015,468.78613787)
\curveto(723.96669019,468.65613418)(723.94669021,468.52113431)(723.94670015,468.38113787)
\curveto(723.94669021,468.25113458)(723.93669022,468.10613473)(723.91670015,467.94613787)
\curveto(723.90669025,467.89613494)(723.89669026,467.836135)(723.88670015,467.76613787)
\curveto(723.88669027,467.69613514)(723.89669026,467.6361352)(723.91670015,467.58613787)
\lineto(723.91670015,467.42113787)
\lineto(723.91670015,467.24113787)
\curveto(723.92669023,467.19113564)(723.93669022,467.1361357)(723.94670015,467.07613787)
\curveto(723.9566902,467.02613581)(723.96169019,466.97113586)(723.96170015,466.91113787)
\curveto(723.97169018,466.85113598)(723.98669017,466.79613604)(724.00670015,466.74613787)
\curveto(724.0566901,466.55613628)(724.11669004,466.38113645)(724.18670015,466.22113787)
\curveto(724.2566899,466.06113677)(724.36168979,465.9311369)(724.50170015,465.83113787)
\curveto(724.63168952,465.7311371)(724.77168938,465.66113717)(724.92170015,465.62113787)
\curveto(724.9516892,465.61113722)(724.97668918,465.60613723)(724.99670015,465.60613787)
\curveto(725.02668913,465.61613722)(725.0566891,465.61613722)(725.08670015,465.60613787)
\curveto(725.10668905,465.60613723)(725.13668902,465.60113723)(725.17670015,465.59113787)
\curveto(725.21668894,465.59113724)(725.2516889,465.59613724)(725.28170015,465.60613787)
\curveto(725.32168883,465.61613722)(725.36168879,465.62113721)(725.40170015,465.62113787)
\curveto(725.44168871,465.62113721)(725.48168867,465.6311372)(725.52170015,465.65113787)
\curveto(725.76168839,465.7311371)(725.9566882,465.86613697)(726.10670015,466.05613787)
\curveto(726.22668793,466.2361366)(726.31668784,466.44113639)(726.37670015,466.67113787)
\curveto(726.39668776,466.74113609)(726.41168774,466.81113602)(726.42170015,466.88113787)
\curveto(726.43168772,466.96113587)(726.44668771,467.04113579)(726.46670015,467.12113787)
\curveto(726.46668769,467.18113565)(726.47168768,467.22613561)(726.48170015,467.25613787)
\curveto(726.48168767,467.27613556)(726.48168767,467.30113553)(726.48170015,467.33113787)
\curveto(726.48168767,467.37113546)(726.48668767,467.40113543)(726.49670015,467.42113787)
\lineto(726.49670015,467.57113787)
}
}
{
\newrgbcolor{curcolor}{0 0 0}
\pscustom[linestyle=none,fillstyle=solid,fillcolor=curcolor]
{
\newpath
\moveto(479.61347383,529.08032977)
\curveto(479.68346619,529.03032631)(479.72346615,528.96032638)(479.73347383,528.87032977)
\curveto(479.75346612,528.78032656)(479.76346611,528.67532666)(479.76347383,528.55532977)
\curveto(479.76346611,528.50532683)(479.75846611,528.45532688)(479.74847383,528.40532977)
\curveto(479.74846612,528.35532698)(479.73846613,528.31032703)(479.71847383,528.27032977)
\curveto(479.68846618,528.18032716)(479.62846624,528.12032722)(479.53847383,528.09032977)
\curveto(479.45846641,528.07032727)(479.36346651,528.06032728)(479.25347383,528.06032977)
\lineto(478.93847383,528.06032977)
\curveto(478.82846704,528.07032727)(478.72346715,528.06032728)(478.62347383,528.03032977)
\curveto(478.48346739,528.00032734)(478.39346748,527.92032742)(478.35347383,527.79032977)
\curveto(478.33346754,527.72032762)(478.32346755,527.6353277)(478.32347383,527.53532977)
\lineto(478.32347383,527.26532977)
\lineto(478.32347383,526.32032977)
\lineto(478.32347383,525.99032977)
\curveto(478.32346755,525.88032946)(478.30346757,525.79532954)(478.26347383,525.73532977)
\curveto(478.22346765,525.67532966)(478.1734677,525.6353297)(478.11347383,525.61532977)
\curveto(478.06346781,525.60532973)(477.99846787,525.59032975)(477.91847383,525.57032977)
\lineto(477.72347383,525.57032977)
\curveto(477.60346827,525.57032977)(477.49846837,525.57532976)(477.40847383,525.58532977)
\curveto(477.31846855,525.60532973)(477.24846862,525.65532968)(477.19847383,525.73532977)
\curveto(477.1684687,525.78532955)(477.15346872,525.85532948)(477.15347383,525.94532977)
\lineto(477.15347383,526.24532977)
\lineto(477.15347383,527.28032977)
\curveto(477.15346872,527.4403279)(477.14346873,527.58532775)(477.12347383,527.71532977)
\curveto(477.11346876,527.85532748)(477.05846881,527.95032739)(476.95847383,528.00032977)
\curveto(476.90846896,528.02032732)(476.83846903,528.0353273)(476.74847383,528.04532977)
\curveto(476.6684692,528.05532728)(476.57846929,528.06032728)(476.47847383,528.06032977)
\lineto(476.19347383,528.06032977)
\lineto(475.95347383,528.06032977)
\lineto(473.68847383,528.06032977)
\curveto(473.59847227,528.06032728)(473.49347238,528.05532728)(473.37347383,528.04532977)
\lineto(473.04347383,528.04532977)
\curveto(472.93347294,528.04532729)(472.83347304,528.05532728)(472.74347383,528.07532977)
\curveto(472.65347322,528.09532724)(472.59347328,528.13032721)(472.56347383,528.18032977)
\curveto(472.51347336,528.25032709)(472.48847338,528.34532699)(472.48847383,528.46532977)
\lineto(472.48847383,528.81032977)
\lineto(472.48847383,529.08032977)
\curveto(472.52847334,529.25032609)(472.58347329,529.39032595)(472.65347383,529.50032977)
\curveto(472.72347315,529.61032573)(472.80347307,529.72532561)(472.89347383,529.84532977)
\lineto(473.25347383,530.38532977)
\curveto(473.69347218,531.01532432)(474.12847174,531.6353237)(474.55847383,532.24532977)
\lineto(475.87847383,534.10532977)
\curveto(476.03846983,534.335321)(476.19346968,534.55532078)(476.34347383,534.76532977)
\curveto(476.49346938,534.98532035)(476.64846922,535.21032013)(476.80847383,535.44032977)
\curveto(476.85846901,535.51031983)(476.90846896,535.57531976)(476.95847383,535.63532977)
\curveto(477.00846886,535.70531963)(477.05846881,535.78031956)(477.10847383,535.86032977)
\lineto(477.16847383,535.95032977)
\curveto(477.19846867,535.99031935)(477.22846864,536.02031932)(477.25847383,536.04032977)
\curveto(477.29846857,536.07031927)(477.33846853,536.09031925)(477.37847383,536.10032977)
\curveto(477.41846845,536.12031922)(477.46346841,536.1403192)(477.51347383,536.16032977)
\curveto(477.53346834,536.16031918)(477.55346832,536.15531918)(477.57347383,536.14532977)
\curveto(477.60346827,536.14531919)(477.62846824,536.15531918)(477.64847383,536.17532977)
\curveto(477.77846809,536.17531916)(477.89846797,536.17031917)(478.00847383,536.16032977)
\curveto(478.11846775,536.15031919)(478.19846767,536.10531923)(478.24847383,536.02532977)
\curveto(478.28846758,535.97531936)(478.30846756,535.90531943)(478.30847383,535.81532977)
\curveto(478.31846755,535.72531961)(478.32346755,535.63031971)(478.32347383,535.53032977)
\lineto(478.32347383,530.07032977)
\curveto(478.32346755,530.00032534)(478.31846755,529.92532541)(478.30847383,529.84532977)
\curveto(478.30846756,529.77532556)(478.31346756,529.70532563)(478.32347383,529.63532977)
\lineto(478.32347383,529.53032977)
\curveto(478.34346753,529.48032586)(478.35846751,529.42532591)(478.36847383,529.36532977)
\curveto(478.37846749,529.31532602)(478.40346747,529.27532606)(478.44347383,529.24532977)
\curveto(478.51346736,529.19532614)(478.59846727,529.16532617)(478.69847383,529.15532977)
\lineto(479.02847383,529.15532977)
\curveto(479.13846673,529.15532618)(479.24346663,529.15032619)(479.34347383,529.14032977)
\curveto(479.45346642,529.1403262)(479.54346633,529.12032622)(479.61347383,529.08032977)
\moveto(477.04847383,529.27532977)
\curveto(477.12846874,529.38532595)(477.16346871,529.55532578)(477.15347383,529.78532977)
\lineto(477.15347383,530.40032977)
\lineto(477.15347383,532.87532977)
\lineto(477.15347383,533.19032977)
\curveto(477.16346871,533.31032203)(477.15846871,533.41032193)(477.13847383,533.49032977)
\lineto(477.13847383,533.64032977)
\curveto(477.13846873,533.73032161)(477.12346875,533.81532152)(477.09347383,533.89532977)
\curveto(477.08346879,533.91532142)(477.0734688,533.92532141)(477.06347383,533.92532977)
\lineto(477.01847383,533.97032977)
\curveto(476.99846887,533.98032136)(476.9684689,533.98532135)(476.92847383,533.98532977)
\curveto(476.90846896,533.96532137)(476.88846898,533.95032139)(476.86847383,533.94032977)
\curveto(476.85846901,533.9403214)(476.84346903,533.9353214)(476.82347383,533.92532977)
\curveto(476.76346911,533.87532146)(476.70346917,533.80532153)(476.64347383,533.71532977)
\curveto(476.58346929,533.62532171)(476.52846934,533.54532179)(476.47847383,533.47532977)
\curveto(476.37846949,533.335322)(476.28346959,533.19032215)(476.19347383,533.04032977)
\curveto(476.10346977,532.90032244)(476.00846986,532.76032258)(475.90847383,532.62032977)
\lineto(475.36847383,531.84032977)
\curveto(475.19847067,531.58032376)(475.02347085,531.32032402)(474.84347383,531.06032977)
\curveto(474.76347111,530.95032439)(474.68847118,530.84532449)(474.61847383,530.74532977)
\lineto(474.40847383,530.44532977)
\curveto(474.35847151,530.36532497)(474.30847156,530.29032505)(474.25847383,530.22032977)
\curveto(474.21847165,530.15032519)(474.1734717,530.07532526)(474.12347383,529.99532977)
\curveto(474.0734718,529.9353254)(474.02347185,529.87032547)(473.97347383,529.80032977)
\curveto(473.93347194,529.7403256)(473.89347198,529.67032567)(473.85347383,529.59032977)
\curveto(473.81347206,529.53032581)(473.78847208,529.46032588)(473.77847383,529.38032977)
\curveto(473.7684721,529.31032603)(473.80347207,529.25532608)(473.88347383,529.21532977)
\curveto(473.95347192,529.16532617)(474.06347181,529.1403262)(474.21347383,529.14032977)
\curveto(474.3734715,529.15032619)(474.50847136,529.15532618)(474.61847383,529.15532977)
\lineto(476.29847383,529.15532977)
\lineto(476.73347383,529.15532977)
\curveto(476.88346899,529.15532618)(476.98846888,529.19532614)(477.04847383,529.27532977)
}
}
{
\newrgbcolor{curcolor}{0 0 0}
\pscustom[linestyle=none,fillstyle=solid,fillcolor=curcolor]
{
\newpath
\moveto(488.08308321,529.90532977)
\curveto(488.11307548,529.78532555)(488.13807546,529.64532569)(488.15808321,529.48532977)
\curveto(488.17807542,529.32532601)(488.18807541,529.16032618)(488.18808321,528.99032977)
\curveto(488.18807541,528.82032652)(488.17807542,528.65532668)(488.15808321,528.49532977)
\curveto(488.13807546,528.335327)(488.11307548,528.19532714)(488.08308321,528.07532977)
\curveto(488.04307555,527.9353274)(488.00807559,527.81032753)(487.97808321,527.70032977)
\curveto(487.94807565,527.59032775)(487.90807569,527.48032786)(487.85808321,527.37032977)
\curveto(487.58807601,526.73032861)(487.17307642,526.24532909)(486.61308321,525.91532977)
\curveto(486.53307706,525.85532948)(486.44807715,525.80532953)(486.35808321,525.76532977)
\curveto(486.26807733,525.7353296)(486.16807743,525.70032964)(486.05808321,525.66032977)
\curveto(485.94807765,525.61032973)(485.82807777,525.57532976)(485.69808321,525.55532977)
\curveto(485.57807802,525.52532981)(485.44807815,525.49532984)(485.30808321,525.46532977)
\curveto(485.24807835,525.44532989)(485.18807841,525.4403299)(485.12808321,525.45032977)
\curveto(485.07807852,525.46032988)(485.01807858,525.45532988)(484.94808321,525.43532977)
\curveto(484.92807867,525.42532991)(484.90307869,525.42532991)(484.87308321,525.43532977)
\curveto(484.84307875,525.4353299)(484.81807878,525.43032991)(484.79808321,525.42032977)
\lineto(484.64808321,525.42032977)
\curveto(484.57807902,525.41032993)(484.52807907,525.41032993)(484.49808321,525.42032977)
\curveto(484.45807914,525.43032991)(484.41307918,525.4353299)(484.36308321,525.43532977)
\curveto(484.32307927,525.42532991)(484.28307931,525.42532991)(484.24308321,525.43532977)
\curveto(484.15307944,525.45532988)(484.06307953,525.47032987)(483.97308321,525.48032977)
\curveto(483.88307971,525.48032986)(483.7930798,525.49032985)(483.70308321,525.51032977)
\curveto(483.61307998,525.5403298)(483.52308007,525.56532977)(483.43308321,525.58532977)
\curveto(483.34308025,525.60532973)(483.25808034,525.6353297)(483.17808321,525.67532977)
\curveto(482.93808066,525.78532955)(482.71308088,525.91532942)(482.50308321,526.06532977)
\curveto(482.2930813,526.22532911)(482.11308148,526.40532893)(481.96308321,526.60532977)
\curveto(481.84308175,526.77532856)(481.73808186,526.95032839)(481.64808321,527.13032977)
\curveto(481.55808204,527.31032803)(481.46808213,527.50032784)(481.37808321,527.70032977)
\curveto(481.33808226,527.80032754)(481.30308229,527.90032744)(481.27308321,528.00032977)
\curveto(481.25308234,528.11032723)(481.22808237,528.22032712)(481.19808321,528.33032977)
\curveto(481.15808244,528.47032687)(481.13308246,528.61032673)(481.12308321,528.75032977)
\curveto(481.11308248,528.89032645)(481.0930825,529.03032631)(481.06308321,529.17032977)
\curveto(481.05308254,529.28032606)(481.04308255,529.38032596)(481.03308321,529.47032977)
\curveto(481.03308256,529.57032577)(481.02308257,529.67032567)(481.00308321,529.77032977)
\lineto(481.00308321,529.86032977)
\curveto(481.01308258,529.89032545)(481.01308258,529.91532542)(481.00308321,529.93532977)
\lineto(481.00308321,530.14532977)
\curveto(480.98308261,530.20532513)(480.97308262,530.27032507)(480.97308321,530.34032977)
\curveto(480.98308261,530.42032492)(480.98808261,530.49532484)(480.98808321,530.56532977)
\lineto(480.98808321,530.71532977)
\curveto(480.98808261,530.76532457)(480.9930826,530.81532452)(481.00308321,530.86532977)
\lineto(481.00308321,531.24032977)
\curveto(481.01308258,531.27032407)(481.01308258,531.30532403)(481.00308321,531.34532977)
\curveto(481.00308259,531.38532395)(481.00808259,531.42532391)(481.01808321,531.46532977)
\curveto(481.03808256,531.57532376)(481.05308254,531.68532365)(481.06308321,531.79532977)
\curveto(481.07308252,531.91532342)(481.08308251,532.03032331)(481.09308321,532.14032977)
\curveto(481.13308246,532.29032305)(481.15808244,532.4353229)(481.16808321,532.57532977)
\curveto(481.18808241,532.72532261)(481.21808238,532.87032247)(481.25808321,533.01032977)
\curveto(481.34808225,533.31032203)(481.44308215,533.59532174)(481.54308321,533.86532977)
\curveto(481.64308195,534.1353212)(481.76808183,534.38532095)(481.91808321,534.61532977)
\curveto(482.11808148,534.9353204)(482.36308123,535.21532012)(482.65308321,535.45532977)
\curveto(482.94308065,535.69531964)(483.28308031,535.88031946)(483.67308321,536.01032977)
\curveto(483.78307981,536.05031929)(483.8930797,536.07531926)(484.00308321,536.08532977)
\curveto(484.12307947,536.10531923)(484.24307935,536.13031921)(484.36308321,536.16032977)
\curveto(484.43307916,536.17031917)(484.4980791,536.17531916)(484.55808321,536.17532977)
\curveto(484.61807898,536.17531916)(484.68307891,536.18031916)(484.75308321,536.19032977)
\curveto(485.45307814,536.21031913)(486.02807757,536.09531924)(486.47808321,535.84532977)
\curveto(486.92807667,535.59531974)(487.27307632,535.24532009)(487.51308321,534.79532977)
\curveto(487.62307597,534.56532077)(487.72307587,534.29032105)(487.81308321,533.97032977)
\curveto(487.83307576,533.90032144)(487.83307576,533.82532151)(487.81308321,533.74532977)
\curveto(487.80307579,533.67532166)(487.77807582,533.62532171)(487.73808321,533.59532977)
\curveto(487.70807589,533.56532177)(487.64807595,533.5403218)(487.55808321,533.52032977)
\curveto(487.46807613,533.51032183)(487.36807623,533.50032184)(487.25808321,533.49032977)
\curveto(487.15807644,533.49032185)(487.05807654,533.49532184)(486.95808321,533.50532977)
\curveto(486.86807673,533.51532182)(486.80307679,533.5353218)(486.76308321,533.56532977)
\curveto(486.65307694,533.6353217)(486.57307702,533.74532159)(486.52308321,533.89532977)
\curveto(486.48307711,534.04532129)(486.42807717,534.17532116)(486.35808321,534.28532977)
\curveto(486.16807743,534.59532074)(485.88807771,534.82532051)(485.51808321,534.97532977)
\curveto(485.44807815,535.00532033)(485.37307822,535.02532031)(485.29308321,535.03532977)
\curveto(485.22307837,535.04532029)(485.14807845,535.06032028)(485.06808321,535.08032977)
\curveto(485.01807858,535.09032025)(484.94807865,535.09532024)(484.85808321,535.09532977)
\curveto(484.77807882,535.09532024)(484.71307888,535.09032025)(484.66308321,535.08032977)
\curveto(484.62307897,535.06032028)(484.58807901,535.05532028)(484.55808321,535.06532977)
\curveto(484.52807907,535.07532026)(484.4930791,535.07532026)(484.45308321,535.06532977)
\lineto(484.21308321,535.00532977)
\curveto(484.14307945,534.98532035)(484.07307952,534.96032038)(484.00308321,534.93032977)
\curveto(483.62307997,534.77032057)(483.33308026,534.56032078)(483.13308321,534.30032977)
\curveto(482.94308065,534.0403213)(482.76808083,533.72532161)(482.60808321,533.35532977)
\curveto(482.57808102,533.27532206)(482.55308104,533.19532214)(482.53308321,533.11532977)
\curveto(482.52308107,533.0353223)(482.50308109,532.95532238)(482.47308321,532.87532977)
\curveto(482.44308115,532.76532257)(482.41808118,532.65032269)(482.39808321,532.53032977)
\curveto(482.38808121,532.41032293)(482.36808123,532.29032305)(482.33808321,532.17032977)
\curveto(482.31808128,532.12032322)(482.30808129,532.07032327)(482.30808321,532.02032977)
\curveto(482.31808128,531.97032337)(482.31308128,531.92032342)(482.29308321,531.87032977)
\curveto(482.28308131,531.81032353)(482.28308131,531.73032361)(482.29308321,531.63032977)
\curveto(482.30308129,531.5403238)(482.31808128,531.48532385)(482.33808321,531.46532977)
\curveto(482.35808124,531.42532391)(482.38808121,531.40532393)(482.42808321,531.40532977)
\curveto(482.47808112,531.40532393)(482.52308107,531.41532392)(482.56308321,531.43532977)
\curveto(482.63308096,531.47532386)(482.6930809,531.52032382)(482.74308321,531.57032977)
\curveto(482.7930808,531.62032372)(482.85308074,531.67032367)(482.92308321,531.72032977)
\lineto(482.98308321,531.78032977)
\curveto(483.01308058,531.81032353)(483.04308055,531.8353235)(483.07308321,531.85532977)
\curveto(483.30308029,532.01532332)(483.57808002,532.15032319)(483.89808321,532.26032977)
\curveto(483.96807963,532.28032306)(484.03807956,532.29532304)(484.10808321,532.30532977)
\curveto(484.17807942,532.31532302)(484.25307934,532.33032301)(484.33308321,532.35032977)
\curveto(484.37307922,532.35032299)(484.40807919,532.35532298)(484.43808321,532.36532977)
\curveto(484.46807913,532.37532296)(484.50307909,532.37532296)(484.54308321,532.36532977)
\curveto(484.593079,532.36532297)(484.63307896,532.37532296)(484.66308321,532.39532977)
\lineto(484.82808321,532.39532977)
\lineto(484.91808321,532.39532977)
\curveto(484.96807863,532.40532293)(485.00807859,532.40532293)(485.03808321,532.39532977)
\curveto(485.08807851,532.38532295)(485.13807846,532.38032296)(485.18808321,532.38032977)
\curveto(485.24807835,532.39032295)(485.30307829,532.39032295)(485.35308321,532.38032977)
\curveto(485.46307813,532.35032299)(485.56807803,532.33032301)(485.66808321,532.32032977)
\curveto(485.77807782,532.31032303)(485.88307771,532.28532305)(485.98308321,532.24532977)
\curveto(486.40307719,532.10532323)(486.74807685,531.92032342)(487.01808321,531.69032977)
\curveto(487.28807631,531.47032387)(487.52807607,531.18532415)(487.73808321,530.83532977)
\curveto(487.81807578,530.69532464)(487.88307571,530.54532479)(487.93308321,530.38532977)
\curveto(487.98307561,530.2353251)(488.03307556,530.07532526)(488.08308321,529.90532977)
\moveto(486.83808321,528.60032977)
\curveto(486.84807675,528.65032669)(486.85307674,528.69532664)(486.85308321,528.73532977)
\lineto(486.85308321,528.88532977)
\curveto(486.85307674,529.19532614)(486.81307678,529.48032586)(486.73308321,529.74032977)
\curveto(486.71307688,529.80032554)(486.6930769,529.85532548)(486.67308321,529.90532977)
\curveto(486.66307693,529.96532537)(486.64807695,530.02032532)(486.62808321,530.07032977)
\curveto(486.40807719,530.56032478)(486.06307753,530.91032443)(485.59308321,531.12032977)
\curveto(485.51307808,531.15032419)(485.43307816,531.17532416)(485.35308321,531.19532977)
\lineto(485.11308321,531.25532977)
\curveto(485.03307856,531.27532406)(484.94307865,531.28532405)(484.84308321,531.28532977)
\lineto(484.52808321,531.28532977)
\curveto(484.50807909,531.26532407)(484.46807913,531.25532408)(484.40808321,531.25532977)
\curveto(484.35807924,531.26532407)(484.31307928,531.26532407)(484.27308321,531.25532977)
\lineto(484.03308321,531.19532977)
\curveto(483.96307963,531.18532415)(483.8930797,531.16532417)(483.82308321,531.13532977)
\curveto(483.22308037,530.87532446)(482.81808078,530.41032493)(482.60808321,529.74032977)
\curveto(482.57808102,529.66032568)(482.55808104,529.58032576)(482.54808321,529.50032977)
\curveto(482.53808106,529.42032592)(482.52308107,529.335326)(482.50308321,529.24532977)
\lineto(482.50308321,529.09532977)
\curveto(482.4930811,529.05532628)(482.48808111,528.98532635)(482.48808321,528.88532977)
\curveto(482.48808111,528.65532668)(482.50808109,528.46032688)(482.54808321,528.30032977)
\curveto(482.56808103,528.23032711)(482.58308101,528.16532717)(482.59308321,528.10532977)
\curveto(482.60308099,528.04532729)(482.62308097,527.98032736)(482.65308321,527.91032977)
\curveto(482.76308083,527.63032771)(482.90808069,527.38532795)(483.08808321,527.17532977)
\curveto(483.26808033,526.97532836)(483.50308009,526.81532852)(483.79308321,526.69532977)
\lineto(484.03308321,526.60532977)
\lineto(484.27308321,526.54532977)
\curveto(484.32307927,526.52532881)(484.36307923,526.52032882)(484.39308321,526.53032977)
\curveto(484.43307916,526.5403288)(484.47807912,526.5353288)(484.52808321,526.51532977)
\curveto(484.55807904,526.50532883)(484.61307898,526.50032884)(484.69308321,526.50032977)
\curveto(484.77307882,526.50032884)(484.83307876,526.50532883)(484.87308321,526.51532977)
\curveto(484.98307861,526.5353288)(485.08807851,526.55032879)(485.18808321,526.56032977)
\curveto(485.28807831,526.57032877)(485.38307821,526.60032874)(485.47308321,526.65032977)
\curveto(486.00307759,526.85032849)(486.3930772,527.22532811)(486.64308321,527.77532977)
\curveto(486.68307691,527.87532746)(486.71307688,527.98032736)(486.73308321,528.09032977)
\lineto(486.82308321,528.42032977)
\curveto(486.82307677,528.50032684)(486.82807677,528.56032678)(486.83808321,528.60032977)
}
}
{
\newrgbcolor{curcolor}{0 0 0}
\pscustom[linestyle=none,fillstyle=solid,fillcolor=curcolor]
{
\newpath
\moveto(490.38769258,527.22032977)
\lineto(490.68769258,527.22032977)
\curveto(490.79769052,527.23032811)(490.90269042,527.23032811)(491.00269258,527.22032977)
\curveto(491.11269021,527.22032812)(491.21269011,527.21032813)(491.30269258,527.19032977)
\curveto(491.39268993,527.18032816)(491.46268986,527.15532818)(491.51269258,527.11532977)
\curveto(491.53268979,527.09532824)(491.54768977,527.06532827)(491.55769258,527.02532977)
\curveto(491.57768974,526.98532835)(491.59768972,526.9403284)(491.61769258,526.89032977)
\lineto(491.61769258,526.81532977)
\curveto(491.62768969,526.76532857)(491.62768969,526.71032863)(491.61769258,526.65032977)
\lineto(491.61769258,526.50032977)
\lineto(491.61769258,526.02032977)
\curveto(491.6176897,525.85032949)(491.57768974,525.73032961)(491.49769258,525.66032977)
\curveto(491.42768989,525.61032973)(491.33768998,525.58532975)(491.22769258,525.58532977)
\lineto(490.89769258,525.58532977)
\lineto(490.44769258,525.58532977)
\curveto(490.29769102,525.58532975)(490.18269114,525.61532972)(490.10269258,525.67532977)
\curveto(490.06269126,525.70532963)(490.03269129,525.75532958)(490.01269258,525.82532977)
\curveto(489.99269133,525.90532943)(489.97769134,525.99032935)(489.96769258,526.08032977)
\lineto(489.96769258,526.36532977)
\curveto(489.97769134,526.46532887)(489.98269134,526.55032879)(489.98269258,526.62032977)
\lineto(489.98269258,526.81532977)
\curveto(489.98269134,526.87532846)(489.99269133,526.93032841)(490.01269258,526.98032977)
\curveto(490.05269127,527.09032825)(490.1226912,527.16032818)(490.22269258,527.19032977)
\curveto(490.25269107,527.19032815)(490.30769101,527.20032814)(490.38769258,527.22032977)
}
}
{
\newrgbcolor{curcolor}{0 0 0}
\pscustom[linestyle=none,fillstyle=solid,fillcolor=curcolor]
{
\newpath
\moveto(494.05284883,535.99532977)
\lineto(498.85284883,535.99532977)
\lineto(499.85784883,535.99532977)
\curveto(499.99784173,535.99531934)(500.11784161,535.98531935)(500.21784883,535.96532977)
\curveto(500.3278414,535.95531938)(500.40784132,535.91031943)(500.45784883,535.83032977)
\curveto(500.47784125,535.79031955)(500.48784124,535.7403196)(500.48784883,535.68032977)
\curveto(500.49784123,535.62031972)(500.50284123,535.55531978)(500.50284883,535.48532977)
\lineto(500.50284883,535.21532977)
\curveto(500.50284123,535.12532021)(500.49284124,535.04532029)(500.47284883,534.97532977)
\curveto(500.4328413,534.89532044)(500.38784134,534.82532051)(500.33784883,534.76532977)
\lineto(500.18784883,534.58532977)
\curveto(500.15784157,534.5353208)(500.12284161,534.49532084)(500.08284883,534.46532977)
\curveto(500.04284169,534.4353209)(500.00284173,534.39532094)(499.96284883,534.34532977)
\curveto(499.88284185,534.2353211)(499.79784193,534.12532121)(499.70784883,534.01532977)
\curveto(499.61784211,533.91532142)(499.5328422,533.81032153)(499.45284883,533.70032977)
\curveto(499.31284242,533.50032184)(499.17284256,533.29032205)(499.03284883,533.07032977)
\curveto(498.89284284,532.86032248)(498.75284298,532.64532269)(498.61284883,532.42532977)
\curveto(498.56284317,532.335323)(498.51284322,532.2403231)(498.46284883,532.14032977)
\curveto(498.41284332,532.0403233)(498.35784337,531.94532339)(498.29784883,531.85532977)
\curveto(498.27784345,531.8353235)(498.26784346,531.81032353)(498.26784883,531.78032977)
\curveto(498.26784346,531.75032359)(498.25784347,531.72532361)(498.23784883,531.70532977)
\curveto(498.16784356,531.60532373)(498.10284363,531.49032385)(498.04284883,531.36032977)
\curveto(497.98284375,531.2403241)(497.9278438,531.12532421)(497.87784883,531.01532977)
\curveto(497.77784395,530.78532455)(497.68284405,530.55032479)(497.59284883,530.31032977)
\curveto(497.50284423,530.07032527)(497.40284433,529.83032551)(497.29284883,529.59032977)
\curveto(497.27284446,529.5403258)(497.25784447,529.49532584)(497.24784883,529.45532977)
\curveto(497.24784448,529.41532592)(497.23784449,529.37032597)(497.21784883,529.32032977)
\curveto(497.16784456,529.20032614)(497.12284461,529.07532626)(497.08284883,528.94532977)
\curveto(497.05284468,528.82532651)(497.01784471,528.70532663)(496.97784883,528.58532977)
\curveto(496.89784483,528.35532698)(496.8328449,528.11532722)(496.78284883,527.86532977)
\curveto(496.74284499,527.62532771)(496.69284504,527.38532795)(496.63284883,527.14532977)
\curveto(496.59284514,526.99532834)(496.56784516,526.84532849)(496.55784883,526.69532977)
\curveto(496.54784518,526.54532879)(496.5278452,526.39532894)(496.49784883,526.24532977)
\curveto(496.48784524,526.20532913)(496.48284525,526.14532919)(496.48284883,526.06532977)
\curveto(496.45284528,525.94532939)(496.42284531,525.84532949)(496.39284883,525.76532977)
\curveto(496.36284537,525.68532965)(496.29284544,525.63032971)(496.18284883,525.60032977)
\curveto(496.1328456,525.58032976)(496.07784565,525.57032977)(496.01784883,525.57032977)
\lineto(495.82284883,525.57032977)
\curveto(495.68284605,525.57032977)(495.54284619,525.57532976)(495.40284883,525.58532977)
\curveto(495.27284646,525.59532974)(495.17784655,525.6403297)(495.11784883,525.72032977)
\curveto(495.07784665,525.78032956)(495.05784667,525.86532947)(495.05784883,525.97532977)
\curveto(495.06784666,526.08532925)(495.08284665,526.18032916)(495.10284883,526.26032977)
\lineto(495.10284883,526.33532977)
\curveto(495.11284662,526.36532897)(495.11784661,526.39532894)(495.11784883,526.42532977)
\curveto(495.13784659,526.50532883)(495.14784658,526.58032876)(495.14784883,526.65032977)
\curveto(495.14784658,526.72032862)(495.15784657,526.79032855)(495.17784883,526.86032977)
\curveto(495.2278465,527.05032829)(495.26784646,527.2353281)(495.29784883,527.41532977)
\curveto(495.3278464,527.60532773)(495.36784636,527.78532755)(495.41784883,527.95532977)
\curveto(495.43784629,528.00532733)(495.44784628,528.04532729)(495.44784883,528.07532977)
\curveto(495.44784628,528.10532723)(495.45284628,528.1403272)(495.46284883,528.18032977)
\curveto(495.56284617,528.48032686)(495.65284608,528.77532656)(495.73284883,529.06532977)
\curveto(495.82284591,529.35532598)(495.9278458,529.6353257)(496.04784883,529.90532977)
\curveto(496.30784542,530.48532485)(496.57784515,531.0353243)(496.85784883,531.55532977)
\curveto(497.13784459,532.08532325)(497.44784428,532.59032275)(497.78784883,533.07032977)
\curveto(497.9278438,533.27032207)(498.07784365,533.46032188)(498.23784883,533.64032977)
\curveto(498.39784333,533.83032151)(498.54784318,534.02032132)(498.68784883,534.21032977)
\curveto(498.727843,534.26032108)(498.76284297,534.30532103)(498.79284883,534.34532977)
\curveto(498.8328429,534.39532094)(498.86784286,534.44532089)(498.89784883,534.49532977)
\curveto(498.90784282,534.51532082)(498.91784281,534.5403208)(498.92784883,534.57032977)
\curveto(498.94784278,534.60032074)(498.94784278,534.63032071)(498.92784883,534.66032977)
\curveto(498.90784282,534.72032062)(498.87284286,534.75532058)(498.82284883,534.76532977)
\curveto(498.77284296,534.78532055)(498.72284301,534.80532053)(498.67284883,534.82532977)
\lineto(498.56784883,534.82532977)
\curveto(498.5278432,534.8353205)(498.47784325,534.8353205)(498.41784883,534.82532977)
\lineto(498.26784883,534.82532977)
\lineto(497.66784883,534.82532977)
\lineto(495.02784883,534.82532977)
\lineto(494.29284883,534.82532977)
\lineto(494.05284883,534.82532977)
\curveto(493.98284775,534.8353205)(493.92284781,534.85032049)(493.87284883,534.87032977)
\curveto(493.78284795,534.91032043)(493.72284801,534.97032037)(493.69284883,535.05032977)
\curveto(493.64284809,535.15032019)(493.6278481,535.29532004)(493.64784883,535.48532977)
\curveto(493.66784806,535.68531965)(493.70284803,535.82031952)(493.75284883,535.89032977)
\curveto(493.77284796,535.91031943)(493.79784793,535.92531941)(493.82784883,535.93532977)
\lineto(493.94784883,535.99532977)
\curveto(493.96784776,535.99531934)(493.98284775,535.99031935)(493.99284883,535.98032977)
\curveto(494.01284772,535.98031936)(494.0328477,535.98531935)(494.05284883,535.99532977)
}
}
{
\newrgbcolor{curcolor}{0 0 0}
\pscustom[linestyle=none,fillstyle=solid,fillcolor=curcolor]
{
\newpath
\moveto(511.74745821,534.10532977)
\curveto(511.54744791,533.81532152)(511.33744812,533.53032181)(511.11745821,533.25032977)
\curveto(510.90744855,532.97032237)(510.70244875,532.68532265)(510.50245821,532.39532977)
\curveto(509.90244955,531.54532379)(509.29745016,530.70532463)(508.68745821,529.87532977)
\curveto(508.07745138,529.05532628)(507.47245198,528.22032712)(506.87245821,527.37032977)
\lineto(506.36245821,526.65032977)
\lineto(505.85245821,525.96032977)
\curveto(505.77245368,525.85032949)(505.69245376,525.7353296)(505.61245821,525.61532977)
\curveto(505.53245392,525.49532984)(505.43745402,525.40032994)(505.32745821,525.33032977)
\curveto(505.28745417,525.31033003)(505.22245423,525.29533004)(505.13245821,525.28532977)
\curveto(505.0524544,525.26533007)(504.96245449,525.25533008)(504.86245821,525.25532977)
\curveto(504.76245469,525.25533008)(504.66745479,525.26033008)(504.57745821,525.27032977)
\curveto(504.49745496,525.28033006)(504.43745502,525.30033004)(504.39745821,525.33032977)
\curveto(504.36745509,525.35032999)(504.34245511,525.38532995)(504.32245821,525.43532977)
\curveto(504.31245514,525.47532986)(504.31745514,525.52032982)(504.33745821,525.57032977)
\curveto(504.37745508,525.65032969)(504.42245503,525.72532961)(504.47245821,525.79532977)
\curveto(504.53245492,525.87532946)(504.58745487,525.95532938)(504.63745821,526.03532977)
\curveto(504.87745458,526.37532896)(505.12245433,526.71032863)(505.37245821,527.04032977)
\curveto(505.62245383,527.37032797)(505.86245359,527.70532763)(506.09245821,528.04532977)
\curveto(506.2524532,528.26532707)(506.41245304,528.48032686)(506.57245821,528.69032977)
\curveto(506.73245272,528.90032644)(506.89245256,529.11532622)(507.05245821,529.33532977)
\curveto(507.41245204,529.85532548)(507.77745168,530.36532497)(508.14745821,530.86532977)
\curveto(508.51745094,531.36532397)(508.88745057,531.87532346)(509.25745821,532.39532977)
\curveto(509.39745006,532.59532274)(509.53744992,532.79032255)(509.67745821,532.98032977)
\curveto(509.82744963,533.17032217)(509.97244948,533.36532197)(510.11245821,533.56532977)
\curveto(510.32244913,533.86532147)(510.53744892,534.16532117)(510.75745821,534.46532977)
\lineto(511.41745821,535.36532977)
\lineto(511.59745821,535.63532977)
\lineto(511.80745821,535.90532977)
\lineto(511.92745821,536.08532977)
\curveto(511.97744748,536.14531919)(512.02744743,536.20031914)(512.07745821,536.25032977)
\curveto(512.14744731,536.30031904)(512.22244723,536.335319)(512.30245821,536.35532977)
\curveto(512.32244713,536.36531897)(512.34744711,536.36531897)(512.37745821,536.35532977)
\curveto(512.41744704,536.35531898)(512.44744701,536.36531897)(512.46745821,536.38532977)
\curveto(512.58744687,536.38531895)(512.72244673,536.38031896)(512.87245821,536.37032977)
\curveto(513.02244643,536.37031897)(513.11244634,536.32531901)(513.14245821,536.23532977)
\curveto(513.16244629,536.20531913)(513.16744629,536.17031917)(513.15745821,536.13032977)
\curveto(513.14744631,536.09031925)(513.13244632,536.06031928)(513.11245821,536.04032977)
\curveto(513.07244638,535.96031938)(513.03244642,535.89031945)(512.99245821,535.83032977)
\curveto(512.9524465,535.77031957)(512.90744655,535.71031963)(512.85745821,535.65032977)
\lineto(512.28745821,534.87032977)
\curveto(512.10744735,534.62032072)(511.92744753,534.36532097)(511.74745821,534.10532977)
\moveto(504.89245821,530.20532977)
\curveto(504.84245461,530.22532511)(504.79245466,530.23032511)(504.74245821,530.22032977)
\curveto(504.69245476,530.21032513)(504.64245481,530.21532512)(504.59245821,530.23532977)
\curveto(504.48245497,530.25532508)(504.37745508,530.27532506)(504.27745821,530.29532977)
\curveto(504.18745527,530.32532501)(504.09245536,530.36532497)(503.99245821,530.41532977)
\curveto(503.66245579,530.55532478)(503.40745605,530.75032459)(503.22745821,531.00032977)
\curveto(503.04745641,531.26032408)(502.90245655,531.57032377)(502.79245821,531.93032977)
\curveto(502.76245669,532.01032333)(502.74245671,532.09032325)(502.73245821,532.17032977)
\curveto(502.72245673,532.26032308)(502.70745675,532.34532299)(502.68745821,532.42532977)
\curveto(502.67745678,532.47532286)(502.67245678,532.5403228)(502.67245821,532.62032977)
\curveto(502.66245679,532.65032269)(502.6574568,532.68032266)(502.65745821,532.71032977)
\curveto(502.6574568,532.75032259)(502.6524568,532.78532255)(502.64245821,532.81532977)
\lineto(502.64245821,532.96532977)
\curveto(502.63245682,533.01532232)(502.62745683,533.07532226)(502.62745821,533.14532977)
\curveto(502.62745683,533.22532211)(502.63245682,533.29032205)(502.64245821,533.34032977)
\lineto(502.64245821,533.50532977)
\curveto(502.66245679,533.55532178)(502.66745679,533.60032174)(502.65745821,533.64032977)
\curveto(502.6574568,533.69032165)(502.66245679,533.7353216)(502.67245821,533.77532977)
\curveto(502.68245677,533.81532152)(502.68745677,533.85032149)(502.68745821,533.88032977)
\curveto(502.68745677,533.92032142)(502.69245676,533.96032138)(502.70245821,534.00032977)
\curveto(502.73245672,534.11032123)(502.7524567,534.22032112)(502.76245821,534.33032977)
\curveto(502.78245667,534.45032089)(502.81745664,534.56532077)(502.86745821,534.67532977)
\curveto(503.00745645,535.01532032)(503.16745629,535.29032005)(503.34745821,535.50032977)
\curveto(503.53745592,535.72031962)(503.80745565,535.90031944)(504.15745821,536.04032977)
\curveto(504.23745522,536.07031927)(504.32245513,536.09031925)(504.41245821,536.10032977)
\curveto(504.50245495,536.12031922)(504.59745486,536.1403192)(504.69745821,536.16032977)
\curveto(504.72745473,536.17031917)(504.78245467,536.17031917)(504.86245821,536.16032977)
\curveto(504.94245451,536.16031918)(504.99245446,536.17031917)(505.01245821,536.19032977)
\curveto(505.57245388,536.20031914)(506.02245343,536.09031925)(506.36245821,535.86032977)
\curveto(506.71245274,535.63031971)(506.97245248,535.32532001)(507.14245821,534.94532977)
\curveto(507.18245227,534.85532048)(507.21745224,534.76032058)(507.24745821,534.66032977)
\curveto(507.27745218,534.56032078)(507.30245215,534.46032088)(507.32245821,534.36032977)
\curveto(507.34245211,534.33032101)(507.34745211,534.30032104)(507.33745821,534.27032977)
\curveto(507.33745212,534.2403211)(507.34245211,534.21032113)(507.35245821,534.18032977)
\curveto(507.38245207,534.07032127)(507.40245205,533.94532139)(507.41245821,533.80532977)
\curveto(507.42245203,533.67532166)(507.43245202,533.5403218)(507.44245821,533.40032977)
\lineto(507.44245821,533.23532977)
\curveto(507.452452,533.17532216)(507.452452,533.12032222)(507.44245821,533.07032977)
\curveto(507.43245202,533.02032232)(507.42745203,532.97032237)(507.42745821,532.92032977)
\lineto(507.42745821,532.78532977)
\curveto(507.41745204,532.74532259)(507.41245204,532.70532263)(507.41245821,532.66532977)
\curveto(507.42245203,532.62532271)(507.41745204,532.58032276)(507.39745821,532.53032977)
\curveto(507.37745208,532.42032292)(507.3574521,532.31532302)(507.33745821,532.21532977)
\curveto(507.32745213,532.11532322)(507.30745215,532.01532332)(507.27745821,531.91532977)
\curveto(507.14745231,531.55532378)(506.98245247,531.2403241)(506.78245821,530.97032977)
\curveto(506.58245287,530.70032464)(506.30745315,530.49532484)(505.95745821,530.35532977)
\curveto(505.87745358,530.32532501)(505.79245366,530.30032504)(505.70245821,530.28032977)
\lineto(505.43245821,530.22032977)
\curveto(505.38245407,530.21032513)(505.33745412,530.20532513)(505.29745821,530.20532977)
\curveto(505.2574542,530.21532512)(505.21745424,530.21532512)(505.17745821,530.20532977)
\curveto(505.07745438,530.18532515)(504.98245447,530.18532515)(504.89245821,530.20532977)
\moveto(504.05245821,531.60032977)
\curveto(504.09245536,531.53032381)(504.13245532,531.46532387)(504.17245821,531.40532977)
\curveto(504.21245524,531.35532398)(504.26245519,531.30532403)(504.32245821,531.25532977)
\lineto(504.47245821,531.13532977)
\curveto(504.53245492,531.10532423)(504.59745486,531.08032426)(504.66745821,531.06032977)
\curveto(504.70745475,531.0403243)(504.74245471,531.03032431)(504.77245821,531.03032977)
\curveto(504.81245464,531.0403243)(504.8524546,531.0353243)(504.89245821,531.01532977)
\curveto(504.92245453,531.01532432)(504.96245449,531.01032433)(505.01245821,531.00032977)
\curveto(505.06245439,531.00032434)(505.10245435,531.00532433)(505.13245821,531.01532977)
\lineto(505.35745821,531.06032977)
\curveto(505.60745385,531.1403242)(505.79245366,531.26532407)(505.91245821,531.43532977)
\curveto(505.99245346,531.5353238)(506.06245339,531.66532367)(506.12245821,531.82532977)
\curveto(506.20245325,532.00532333)(506.26245319,532.23032311)(506.30245821,532.50032977)
\curveto(506.34245311,532.78032256)(506.3574531,533.06032228)(506.34745821,533.34032977)
\curveto(506.33745312,533.63032171)(506.30745315,533.90532143)(506.25745821,534.16532977)
\curveto(506.20745325,534.42532091)(506.13245332,534.6353207)(506.03245821,534.79532977)
\curveto(505.91245354,534.99532034)(505.76245369,535.14532019)(505.58245821,535.24532977)
\curveto(505.50245395,535.29532004)(505.41245404,535.32532001)(505.31245821,535.33532977)
\curveto(505.21245424,535.35531998)(505.10745435,535.36531997)(504.99745821,535.36532977)
\curveto(504.97745448,535.35531998)(504.9524545,535.35031999)(504.92245821,535.35032977)
\curveto(504.90245455,535.36031998)(504.88245457,535.36031998)(504.86245821,535.35032977)
\curveto(504.81245464,535.34032)(504.76745469,535.33032001)(504.72745821,535.32032977)
\curveto(504.68745477,535.32032002)(504.64745481,535.31032003)(504.60745821,535.29032977)
\curveto(504.42745503,535.21032013)(504.27745518,535.09032025)(504.15745821,534.93032977)
\curveto(504.04745541,534.77032057)(503.9574555,534.59032075)(503.88745821,534.39032977)
\curveto(503.82745563,534.20032114)(503.78245567,533.97532136)(503.75245821,533.71532977)
\curveto(503.73245572,533.45532188)(503.72745573,533.19032215)(503.73745821,532.92032977)
\curveto(503.74745571,532.66032268)(503.77745568,532.41032293)(503.82745821,532.17032977)
\curveto(503.88745557,531.9403234)(503.96245549,531.75032359)(504.05245821,531.60032977)
\moveto(514.85245821,528.61532977)
\curveto(514.86244459,528.56532677)(514.86744459,528.47532686)(514.86745821,528.34532977)
\curveto(514.86744459,528.21532712)(514.8574446,528.12532721)(514.83745821,528.07532977)
\curveto(514.81744464,528.02532731)(514.81244464,527.97032737)(514.82245821,527.91032977)
\curveto(514.83244462,527.86032748)(514.83244462,527.81032753)(514.82245821,527.76032977)
\curveto(514.78244467,527.62032772)(514.7524447,527.48532785)(514.73245821,527.35532977)
\curveto(514.72244473,527.22532811)(514.69244476,527.10532823)(514.64245821,526.99532977)
\curveto(514.50244495,526.64532869)(514.33744512,526.35032899)(514.14745821,526.11032977)
\curveto(513.9574455,525.88032946)(513.68744577,525.69532964)(513.33745821,525.55532977)
\curveto(513.2574462,525.52532981)(513.17244628,525.50532983)(513.08245821,525.49532977)
\curveto(512.99244646,525.47532986)(512.90744655,525.45532988)(512.82745821,525.43532977)
\curveto(512.77744668,525.42532991)(512.72744673,525.42032992)(512.67745821,525.42032977)
\curveto(512.62744683,525.42032992)(512.57744688,525.41532992)(512.52745821,525.40532977)
\curveto(512.49744696,525.39532994)(512.44744701,525.39532994)(512.37745821,525.40532977)
\curveto(512.30744715,525.40532993)(512.2574472,525.41032993)(512.22745821,525.42032977)
\curveto(512.16744729,525.4403299)(512.10744735,525.45032989)(512.04745821,525.45032977)
\curveto(511.99744746,525.4403299)(511.94744751,525.44532989)(511.89745821,525.46532977)
\curveto(511.80744765,525.48532985)(511.71744774,525.51032983)(511.62745821,525.54032977)
\curveto(511.54744791,525.56032978)(511.46744799,525.59032975)(511.38745821,525.63032977)
\curveto(511.06744839,525.77032957)(510.81744864,525.96532937)(510.63745821,526.21532977)
\curveto(510.457449,526.47532886)(510.30744915,526.78032856)(510.18745821,527.13032977)
\curveto(510.16744929,527.21032813)(510.1524493,527.29532804)(510.14245821,527.38532977)
\curveto(510.13244932,527.47532786)(510.11744934,527.56032778)(510.09745821,527.64032977)
\curveto(510.08744937,527.67032767)(510.08244937,527.70032764)(510.08245821,527.73032977)
\lineto(510.08245821,527.83532977)
\curveto(510.06244939,527.91532742)(510.0524494,527.99532734)(510.05245821,528.07532977)
\lineto(510.05245821,528.21032977)
\curveto(510.03244942,528.31032703)(510.03244942,528.41032693)(510.05245821,528.51032977)
\lineto(510.05245821,528.69032977)
\curveto(510.06244939,528.7403266)(510.06744939,528.78532655)(510.06745821,528.82532977)
\curveto(510.06744939,528.87532646)(510.07244938,528.92032642)(510.08245821,528.96032977)
\curveto(510.09244936,529.00032634)(510.09744936,529.0353263)(510.09745821,529.06532977)
\curveto(510.09744936,529.10532623)(510.10244935,529.14532619)(510.11245821,529.18532977)
\lineto(510.17245821,529.51532977)
\curveto(510.19244926,529.6353257)(510.22244923,529.74532559)(510.26245821,529.84532977)
\curveto(510.40244905,530.17532516)(510.56244889,530.45032489)(510.74245821,530.67032977)
\curveto(510.93244852,530.90032444)(511.19244826,531.08532425)(511.52245821,531.22532977)
\curveto(511.60244785,531.26532407)(511.68744777,531.29032405)(511.77745821,531.30032977)
\lineto(512.07745821,531.36032977)
\lineto(512.21245821,531.36032977)
\curveto(512.26244719,531.37032397)(512.31244714,531.37532396)(512.36245821,531.37532977)
\curveto(512.93244652,531.39532394)(513.39244606,531.29032405)(513.74245821,531.06032977)
\curveto(514.10244535,530.8403245)(514.36744509,530.5403248)(514.53745821,530.16032977)
\curveto(514.58744487,530.06032528)(514.62744483,529.96032538)(514.65745821,529.86032977)
\curveto(514.68744477,529.76032558)(514.71744474,529.65532568)(514.74745821,529.54532977)
\curveto(514.7574447,529.50532583)(514.76244469,529.47032587)(514.76245821,529.44032977)
\curveto(514.76244469,529.42032592)(514.76744469,529.39032595)(514.77745821,529.35032977)
\curveto(514.79744466,529.28032606)(514.80744465,529.20532613)(514.80745821,529.12532977)
\curveto(514.80744465,529.04532629)(514.81744464,528.96532637)(514.83745821,528.88532977)
\curveto(514.83744462,528.8353265)(514.83744462,528.79032655)(514.83745821,528.75032977)
\curveto(514.83744462,528.71032663)(514.84244461,528.66532667)(514.85245821,528.61532977)
\moveto(513.74245821,528.18032977)
\curveto(513.7524457,528.23032711)(513.7574457,528.30532703)(513.75745821,528.40532977)
\curveto(513.76744569,528.50532683)(513.76244569,528.58032676)(513.74245821,528.63032977)
\curveto(513.72244573,528.69032665)(513.71744574,528.74532659)(513.72745821,528.79532977)
\curveto(513.74744571,528.85532648)(513.74744571,528.91532642)(513.72745821,528.97532977)
\curveto(513.71744574,529.00532633)(513.71244574,529.0403263)(513.71245821,529.08032977)
\curveto(513.71244574,529.12032622)(513.70744575,529.16032618)(513.69745821,529.20032977)
\curveto(513.67744578,529.28032606)(513.6574458,529.35532598)(513.63745821,529.42532977)
\curveto(513.62744583,529.50532583)(513.61244584,529.58532575)(513.59245821,529.66532977)
\curveto(513.56244589,529.72532561)(513.53744592,529.78532555)(513.51745821,529.84532977)
\curveto(513.49744596,529.90532543)(513.46744599,529.96532537)(513.42745821,530.02532977)
\curveto(513.32744613,530.19532514)(513.19744626,530.33032501)(513.03745821,530.43032977)
\curveto(512.9574465,530.48032486)(512.86244659,530.51532482)(512.75245821,530.53532977)
\curveto(512.64244681,530.55532478)(512.51744694,530.56532477)(512.37745821,530.56532977)
\curveto(512.3574471,530.55532478)(512.33244712,530.55032479)(512.30245821,530.55032977)
\curveto(512.27244718,530.56032478)(512.24244721,530.56032478)(512.21245821,530.55032977)
\lineto(512.06245821,530.49032977)
\curveto(512.01244744,530.48032486)(511.96744749,530.46532487)(511.92745821,530.44532977)
\curveto(511.73744772,530.335325)(511.59244786,530.19032515)(511.49245821,530.01032977)
\curveto(511.40244805,529.83032551)(511.32244813,529.62532571)(511.25245821,529.39532977)
\curveto(511.21244824,529.26532607)(511.19244826,529.13032621)(511.19245821,528.99032977)
\curveto(511.19244826,528.86032648)(511.18244827,528.71532662)(511.16245821,528.55532977)
\curveto(511.1524483,528.50532683)(511.14244831,528.44532689)(511.13245821,528.37532977)
\curveto(511.13244832,528.30532703)(511.14244831,528.24532709)(511.16245821,528.19532977)
\lineto(511.16245821,528.03032977)
\lineto(511.16245821,527.85032977)
\curveto(511.17244828,527.80032754)(511.18244827,527.74532759)(511.19245821,527.68532977)
\curveto(511.20244825,527.6353277)(511.20744825,527.58032776)(511.20745821,527.52032977)
\curveto(511.21744824,527.46032788)(511.23244822,527.40532793)(511.25245821,527.35532977)
\curveto(511.30244815,527.16532817)(511.36244809,526.99032835)(511.43245821,526.83032977)
\curveto(511.50244795,526.67032867)(511.60744785,526.5403288)(511.74745821,526.44032977)
\curveto(511.87744758,526.340329)(512.01744744,526.27032907)(512.16745821,526.23032977)
\curveto(512.19744726,526.22032912)(512.22244723,526.21532912)(512.24245821,526.21532977)
\curveto(512.27244718,526.22532911)(512.30244715,526.22532911)(512.33245821,526.21532977)
\curveto(512.3524471,526.21532912)(512.38244707,526.21032913)(512.42245821,526.20032977)
\curveto(512.46244699,526.20032914)(512.49744696,526.20532913)(512.52745821,526.21532977)
\curveto(512.56744689,526.22532911)(512.60744685,526.23032911)(512.64745821,526.23032977)
\curveto(512.68744677,526.23032911)(512.72744673,526.2403291)(512.76745821,526.26032977)
\curveto(513.00744645,526.340329)(513.20244625,526.47532886)(513.35245821,526.66532977)
\curveto(513.47244598,526.84532849)(513.56244589,527.05032829)(513.62245821,527.28032977)
\curveto(513.64244581,527.35032799)(513.6574458,527.42032792)(513.66745821,527.49032977)
\curveto(513.67744578,527.57032777)(513.69244576,527.65032769)(513.71245821,527.73032977)
\curveto(513.71244574,527.79032755)(513.71744574,527.8353275)(513.72745821,527.86532977)
\curveto(513.72744573,527.88532745)(513.72744573,527.91032743)(513.72745821,527.94032977)
\curveto(513.72744573,527.98032736)(513.73244572,528.01032733)(513.74245821,528.03032977)
\lineto(513.74245821,528.18032977)
}
}
{
\newrgbcolor{curcolor}{0 0 0}
\pscustom[linestyle=none,fillstyle=solid,fillcolor=curcolor]
{
\newpath
\moveto(548.51432955,281.63185809)
\curveto(548.61432469,281.63184747)(548.7093246,281.62184748)(548.79932955,281.60185809)
\curveto(548.88932442,281.59184751)(548.95432435,281.56184754)(548.99432955,281.51185809)
\curveto(549.05432425,281.43184767)(549.08432422,281.32684777)(549.08432955,281.19685809)
\lineto(549.08432955,280.80685809)
\lineto(549.08432955,279.30685809)
\lineto(549.08432955,272.91685809)
\lineto(549.08432955,271.74685809)
\lineto(549.08432955,271.43185809)
\curveto(549.09432421,271.33185777)(549.07932423,271.25185785)(549.03932955,271.19185809)
\curveto(548.98932432,271.11185799)(548.91432439,271.06185804)(548.81432955,271.04185809)
\curveto(548.72432458,271.03185807)(548.61432469,271.02685807)(548.48432955,271.02685809)
\lineto(548.25932955,271.02685809)
\curveto(548.17932513,271.04685805)(548.1093252,271.06185804)(548.04932955,271.07185809)
\curveto(547.98932532,271.09185801)(547.93932537,271.13185797)(547.89932955,271.19185809)
\curveto(547.85932545,271.25185785)(547.83932547,271.32685777)(547.83932955,271.41685809)
\lineto(547.83932955,271.71685809)
\lineto(547.83932955,272.81185809)
\lineto(547.83932955,278.15185809)
\curveto(547.81932549,278.24185086)(547.8043255,278.31685078)(547.79432955,278.37685809)
\curveto(547.79432551,278.44685065)(547.76432554,278.50685059)(547.70432955,278.55685809)
\curveto(547.63432567,278.60685049)(547.54432576,278.63185047)(547.43432955,278.63185809)
\curveto(547.33432597,278.64185046)(547.22432608,278.64685045)(547.10432955,278.64685809)
\lineto(545.96432955,278.64685809)
\lineto(545.46932955,278.64685809)
\curveto(545.309328,278.65685044)(545.19932811,278.71685038)(545.13932955,278.82685809)
\curveto(545.11932819,278.85685024)(545.1093282,278.88685021)(545.10932955,278.91685809)
\curveto(545.1093282,278.95685014)(545.1043282,279.0018501)(545.09432955,279.05185809)
\curveto(545.07432823,279.17184993)(545.07932823,279.28184982)(545.10932955,279.38185809)
\curveto(545.14932816,279.48184962)(545.2043281,279.55184955)(545.27432955,279.59185809)
\curveto(545.35432795,279.64184946)(545.47432783,279.66684943)(545.63432955,279.66685809)
\curveto(545.79432751,279.66684943)(545.92932738,279.68184942)(546.03932955,279.71185809)
\curveto(546.08932722,279.72184938)(546.14432716,279.72684937)(546.20432955,279.72685809)
\curveto(546.26432704,279.73684936)(546.32432698,279.75184935)(546.38432955,279.77185809)
\curveto(546.53432677,279.82184928)(546.67932663,279.87184923)(546.81932955,279.92185809)
\curveto(546.95932635,279.98184912)(547.09432621,280.05184905)(547.22432955,280.13185809)
\curveto(547.36432594,280.22184888)(547.48432582,280.32684877)(547.58432955,280.44685809)
\curveto(547.68432562,280.56684853)(547.77932553,280.6968484)(547.86932955,280.83685809)
\curveto(547.92932538,280.93684816)(547.97432533,281.04684805)(548.00432955,281.16685809)
\curveto(548.04432526,281.28684781)(548.09432521,281.39184771)(548.15432955,281.48185809)
\curveto(548.2043251,281.54184756)(548.27432503,281.58184752)(548.36432955,281.60185809)
\curveto(548.38432492,281.61184749)(548.4093249,281.61684748)(548.43932955,281.61685809)
\curveto(548.46932484,281.61684748)(548.49432481,281.62184748)(548.51432955,281.63185809)
}
}
{
\newrgbcolor{curcolor}{0 0 0}
\pscustom[linestyle=none,fillstyle=solid,fillcolor=curcolor]
{
\newpath
\moveto(556.86393892,281.63185809)
\curveto(556.96393407,281.63184747)(557.05893397,281.62184748)(557.14893892,281.60185809)
\curveto(557.23893379,281.59184751)(557.30393373,281.56184754)(557.34393892,281.51185809)
\curveto(557.40393363,281.43184767)(557.4339336,281.32684777)(557.43393892,281.19685809)
\lineto(557.43393892,280.80685809)
\lineto(557.43393892,279.30685809)
\lineto(557.43393892,272.91685809)
\lineto(557.43393892,271.74685809)
\lineto(557.43393892,271.43185809)
\curveto(557.44393359,271.33185777)(557.4289336,271.25185785)(557.38893892,271.19185809)
\curveto(557.33893369,271.11185799)(557.26393377,271.06185804)(557.16393892,271.04185809)
\curveto(557.07393396,271.03185807)(556.96393407,271.02685807)(556.83393892,271.02685809)
\lineto(556.60893892,271.02685809)
\curveto(556.5289345,271.04685805)(556.45893457,271.06185804)(556.39893892,271.07185809)
\curveto(556.33893469,271.09185801)(556.28893474,271.13185797)(556.24893892,271.19185809)
\curveto(556.20893482,271.25185785)(556.18893484,271.32685777)(556.18893892,271.41685809)
\lineto(556.18893892,271.71685809)
\lineto(556.18893892,272.81185809)
\lineto(556.18893892,278.15185809)
\curveto(556.16893486,278.24185086)(556.15393488,278.31685078)(556.14393892,278.37685809)
\curveto(556.14393489,278.44685065)(556.11393492,278.50685059)(556.05393892,278.55685809)
\curveto(555.98393505,278.60685049)(555.89393514,278.63185047)(555.78393892,278.63185809)
\curveto(555.68393535,278.64185046)(555.57393546,278.64685045)(555.45393892,278.64685809)
\lineto(554.31393892,278.64685809)
\lineto(553.81893892,278.64685809)
\curveto(553.65893737,278.65685044)(553.54893748,278.71685038)(553.48893892,278.82685809)
\curveto(553.46893756,278.85685024)(553.45893757,278.88685021)(553.45893892,278.91685809)
\curveto(553.45893757,278.95685014)(553.45393758,279.0018501)(553.44393892,279.05185809)
\curveto(553.42393761,279.17184993)(553.4289376,279.28184982)(553.45893892,279.38185809)
\curveto(553.49893753,279.48184962)(553.55393748,279.55184955)(553.62393892,279.59185809)
\curveto(553.70393733,279.64184946)(553.82393721,279.66684943)(553.98393892,279.66685809)
\curveto(554.14393689,279.66684943)(554.27893675,279.68184942)(554.38893892,279.71185809)
\curveto(554.43893659,279.72184938)(554.49393654,279.72684937)(554.55393892,279.72685809)
\curveto(554.61393642,279.73684936)(554.67393636,279.75184935)(554.73393892,279.77185809)
\curveto(554.88393615,279.82184928)(555.028936,279.87184923)(555.16893892,279.92185809)
\curveto(555.30893572,279.98184912)(555.44393559,280.05184905)(555.57393892,280.13185809)
\curveto(555.71393532,280.22184888)(555.8339352,280.32684877)(555.93393892,280.44685809)
\curveto(556.033935,280.56684853)(556.1289349,280.6968484)(556.21893892,280.83685809)
\curveto(556.27893475,280.93684816)(556.32393471,281.04684805)(556.35393892,281.16685809)
\curveto(556.39393464,281.28684781)(556.44393459,281.39184771)(556.50393892,281.48185809)
\curveto(556.55393448,281.54184756)(556.62393441,281.58184752)(556.71393892,281.60185809)
\curveto(556.7339343,281.61184749)(556.75893427,281.61684748)(556.78893892,281.61685809)
\curveto(556.81893421,281.61684748)(556.84393419,281.62184748)(556.86393892,281.63185809)
}
}
{
\newrgbcolor{curcolor}{0 0 0}
\pscustom[linestyle=none,fillstyle=solid,fillcolor=curcolor]
{
\newpath
\moveto(562.1085483,272.66185809)
\lineto(562.4085483,272.66185809)
\curveto(562.51854624,272.67185643)(562.62354613,272.67185643)(562.7235483,272.66185809)
\curveto(562.83354592,272.66185644)(562.93354582,272.65185645)(563.0235483,272.63185809)
\curveto(563.11354564,272.62185648)(563.18354557,272.5968565)(563.2335483,272.55685809)
\curveto(563.2535455,272.53685656)(563.26854549,272.50685659)(563.2785483,272.46685809)
\curveto(563.29854546,272.42685667)(563.31854544,272.38185672)(563.3385483,272.33185809)
\lineto(563.3385483,272.25685809)
\curveto(563.34854541,272.20685689)(563.34854541,272.15185695)(563.3385483,272.09185809)
\lineto(563.3385483,271.94185809)
\lineto(563.3385483,271.46185809)
\curveto(563.33854542,271.29185781)(563.29854546,271.17185793)(563.2185483,271.10185809)
\curveto(563.14854561,271.05185805)(563.0585457,271.02685807)(562.9485483,271.02685809)
\lineto(562.6185483,271.02685809)
\lineto(562.1685483,271.02685809)
\curveto(562.01854674,271.02685807)(561.90354685,271.05685804)(561.8235483,271.11685809)
\curveto(561.78354697,271.14685795)(561.753547,271.1968579)(561.7335483,271.26685809)
\curveto(561.71354704,271.34685775)(561.69854706,271.43185767)(561.6885483,271.52185809)
\lineto(561.6885483,271.80685809)
\curveto(561.69854706,271.90685719)(561.70354705,271.99185711)(561.7035483,272.06185809)
\lineto(561.7035483,272.25685809)
\curveto(561.70354705,272.31685678)(561.71354704,272.37185673)(561.7335483,272.42185809)
\curveto(561.77354698,272.53185657)(561.84354691,272.6018565)(561.9435483,272.63185809)
\curveto(561.97354678,272.63185647)(562.02854673,272.64185646)(562.1085483,272.66185809)
}
}
{
\newrgbcolor{curcolor}{0 0 0}
\pscustom[linestyle=none,fillstyle=solid,fillcolor=curcolor]
{
\newpath
\moveto(572.25370455,276.62185809)
\curveto(572.25369691,276.54185256)(572.25869691,276.46185264)(572.26870455,276.38185809)
\curveto(572.27869689,276.3018528)(572.27369689,276.22685287)(572.25370455,276.15685809)
\curveto(572.23369693,276.11685298)(572.22869694,276.07185303)(572.23870455,276.02185809)
\curveto(572.24869692,275.98185312)(572.24869692,275.94185316)(572.23870455,275.90185809)
\lineto(572.23870455,275.75185809)
\curveto(572.22869694,275.66185344)(572.22369694,275.57185353)(572.22370455,275.48185809)
\curveto(572.22369694,275.4018537)(572.21869695,275.32185378)(572.20870455,275.24185809)
\lineto(572.17870455,275.00185809)
\curveto(572.168697,274.93185417)(572.15869701,274.85685424)(572.14870455,274.77685809)
\curveto(572.13869703,274.73685436)(572.13369703,274.6968544)(572.13370455,274.65685809)
\curveto(572.13369703,274.61685448)(572.12869704,274.57185453)(572.11870455,274.52185809)
\curveto(572.07869709,274.38185472)(572.04869712,274.24185486)(572.02870455,274.10185809)
\curveto(572.01869715,273.96185514)(571.98869718,273.82685527)(571.93870455,273.69685809)
\curveto(571.88869728,273.52685557)(571.83369733,273.36185574)(571.77370455,273.20185809)
\curveto(571.72369744,273.04185606)(571.6636975,272.88685621)(571.59370455,272.73685809)
\curveto(571.57369759,272.67685642)(571.54369762,272.61685648)(571.50370455,272.55685809)
\lineto(571.41370455,272.40685809)
\curveto(571.21369795,272.08685701)(570.99869817,271.82185728)(570.76870455,271.61185809)
\curveto(570.53869863,271.4018577)(570.24369892,271.22185788)(569.88370455,271.07185809)
\curveto(569.7636994,271.02185808)(569.63369953,270.98685811)(569.49370455,270.96685809)
\curveto(569.3636998,270.94685815)(569.22869994,270.92185818)(569.08870455,270.89185809)
\curveto(569.02870014,270.88185822)(568.9687002,270.87685822)(568.90870455,270.87685809)
\curveto(568.84870032,270.87685822)(568.78370038,270.87185823)(568.71370455,270.86185809)
\curveto(568.68370048,270.85185825)(568.63370053,270.85185825)(568.56370455,270.86185809)
\lineto(568.41370455,270.86185809)
\lineto(568.26370455,270.86185809)
\curveto(568.18370098,270.88185822)(568.09870107,270.8968582)(568.00870455,270.90685809)
\curveto(567.92870124,270.90685819)(567.85370131,270.91685818)(567.78370455,270.93685809)
\curveto(567.74370142,270.94685815)(567.70870146,270.95185815)(567.67870455,270.95185809)
\curveto(567.65870151,270.94185816)(567.63370153,270.94685815)(567.60370455,270.96685809)
\lineto(567.33370455,271.02685809)
\curveto(567.24370192,271.05685804)(567.15870201,271.08685801)(567.07870455,271.11685809)
\curveto(566.49870267,271.35685774)(566.0637031,271.72685737)(565.77370455,272.22685809)
\curveto(565.69370347,272.35685674)(565.62870354,272.49185661)(565.57870455,272.63185809)
\curveto(565.53870363,272.77185633)(565.49370367,272.92185618)(565.44370455,273.08185809)
\curveto(565.42370374,273.16185594)(565.41870375,273.24185586)(565.42870455,273.32185809)
\curveto(565.44870372,273.4018557)(565.48370368,273.45685564)(565.53370455,273.48685809)
\curveto(565.5637036,273.50685559)(565.61870355,273.52185558)(565.69870455,273.53185809)
\curveto(565.77870339,273.55185555)(565.8637033,273.56185554)(565.95370455,273.56185809)
\curveto(566.04370312,273.57185553)(566.12870304,273.57185553)(566.20870455,273.56185809)
\curveto(566.29870287,273.55185555)(566.3687028,273.54185556)(566.41870455,273.53185809)
\curveto(566.43870273,273.52185558)(566.4637027,273.50685559)(566.49370455,273.48685809)
\curveto(566.53370263,273.46685563)(566.5637026,273.44685565)(566.58370455,273.42685809)
\curveto(566.64370252,273.34685575)(566.68870248,273.25185585)(566.71870455,273.14185809)
\curveto(566.75870241,273.03185607)(566.80370236,272.93185617)(566.85370455,272.84185809)
\curveto(567.10370206,272.45185665)(567.47370169,272.18185692)(567.96370455,272.03185809)
\curveto(568.03370113,272.01185709)(568.10370106,271.9968571)(568.17370455,271.98685809)
\curveto(568.25370091,271.98685711)(568.33370083,271.97685712)(568.41370455,271.95685809)
\curveto(568.45370071,271.94685715)(568.50870066,271.94185716)(568.57870455,271.94185809)
\curveto(568.65870051,271.94185716)(568.71370045,271.94685715)(568.74370455,271.95685809)
\curveto(568.77370039,271.96685713)(568.80370036,271.97185713)(568.83370455,271.97185809)
\lineto(568.93870455,271.97185809)
\curveto(569.01870015,271.99185711)(569.09370007,272.01185709)(569.16370455,272.03185809)
\curveto(569.24369992,272.05185705)(569.31869985,272.07685702)(569.38870455,272.10685809)
\curveto(569.73869943,272.25685684)(570.00869916,272.47185663)(570.19870455,272.75185809)
\curveto(570.38869878,273.03185607)(570.54369862,273.35685574)(570.66370455,273.72685809)
\curveto(570.69369847,273.80685529)(570.71369845,273.88185522)(570.72370455,273.95185809)
\curveto(570.74369842,274.02185508)(570.7636984,274.096855)(570.78370455,274.17685809)
\curveto(570.80369836,274.26685483)(570.81869835,274.36185474)(570.82870455,274.46185809)
\curveto(570.84869832,274.57185453)(570.8686983,274.67685442)(570.88870455,274.77685809)
\curveto(570.89869827,274.82685427)(570.90369826,274.87685422)(570.90370455,274.92685809)
\curveto(570.91369825,274.98685411)(570.91869825,275.04185406)(570.91870455,275.09185809)
\curveto(570.93869823,275.15185395)(570.94869822,275.22685387)(570.94870455,275.31685809)
\curveto(570.94869822,275.41685368)(570.93869823,275.4968536)(570.91870455,275.55685809)
\curveto(570.88869828,275.64685345)(570.83869833,275.68685341)(570.76870455,275.67685809)
\curveto(570.70869846,275.66685343)(570.65369851,275.63685346)(570.60370455,275.58685809)
\curveto(570.52369864,275.53685356)(570.45369871,275.47685362)(570.39370455,275.40685809)
\curveto(570.34369882,275.33685376)(570.27869889,275.27685382)(570.19870455,275.22685809)
\curveto(570.03869913,275.11685398)(569.87369929,275.01685408)(569.70370455,274.92685809)
\curveto(569.53369963,274.84685425)(569.33869983,274.77685432)(569.11870455,274.71685809)
\curveto(569.01870015,274.68685441)(568.91870025,274.67185443)(568.81870455,274.67185809)
\curveto(568.72870044,274.67185443)(568.62870054,274.66185444)(568.51870455,274.64185809)
\lineto(568.36870455,274.64185809)
\curveto(568.31870085,274.66185444)(568.2687009,274.66685443)(568.21870455,274.65685809)
\curveto(568.17870099,274.64685445)(568.13870103,274.64685445)(568.09870455,274.65685809)
\curveto(568.0687011,274.66685443)(568.02370114,274.67185443)(567.96370455,274.67185809)
\curveto(567.90370126,274.68185442)(567.83870133,274.69185441)(567.76870455,274.70185809)
\lineto(567.58870455,274.73185809)
\curveto(567.13870203,274.85185425)(566.75870241,275.01685408)(566.44870455,275.22685809)
\curveto(566.17870299,275.41685368)(565.94870322,275.64685345)(565.75870455,275.91685809)
\curveto(565.57870359,276.1968529)(565.43370373,276.51185259)(565.32370455,276.86185809)
\lineto(565.26370455,277.07185809)
\curveto(565.25370391,277.15185195)(565.23870393,277.23185187)(565.21870455,277.31185809)
\curveto(565.20870396,277.34185176)(565.20370396,277.37185173)(565.20370455,277.40185809)
\curveto(565.20370396,277.43185167)(565.19870397,277.46185164)(565.18870455,277.49185809)
\curveto(565.17870399,277.55185155)(565.17370399,277.61185149)(565.17370455,277.67185809)
\curveto(565.17370399,277.74185136)(565.163704,277.8018513)(565.14370455,277.85185809)
\lineto(565.14370455,278.03185809)
\curveto(565.13370403,278.08185102)(565.12870404,278.15185095)(565.12870455,278.24185809)
\curveto(565.12870404,278.33185077)(565.13870403,278.4018507)(565.15870455,278.45185809)
\lineto(565.15870455,278.61685809)
\curveto(565.17870399,278.6968504)(565.18870398,278.77185033)(565.18870455,278.84185809)
\curveto(565.19870397,278.91185019)(565.21370395,278.98185012)(565.23370455,279.05185809)
\curveto(565.29370387,279.25184985)(565.35370381,279.44184966)(565.41370455,279.62185809)
\curveto(565.48370368,279.8018493)(565.57370359,279.97184913)(565.68370455,280.13185809)
\curveto(565.72370344,280.2018489)(565.7637034,280.26684883)(565.80370455,280.32685809)
\lineto(565.95370455,280.50685809)
\curveto(565.97370319,280.51684858)(565.99370317,280.53184857)(566.01370455,280.55185809)
\curveto(566.10370306,280.68184842)(566.21370295,280.79184831)(566.34370455,280.88185809)
\curveto(566.60370256,281.08184802)(566.8687023,281.23684786)(567.13870455,281.34685809)
\curveto(567.21870195,281.38684771)(567.29870187,281.41684768)(567.37870455,281.43685809)
\curveto(567.4687017,281.46684763)(567.55870161,281.49184761)(567.64870455,281.51185809)
\curveto(567.74870142,281.54184756)(567.84870132,281.56184754)(567.94870455,281.57185809)
\curveto(568.04870112,281.58184752)(568.15370101,281.5968475)(568.26370455,281.61685809)
\curveto(568.29370087,281.62684747)(568.33370083,281.62684747)(568.38370455,281.61685809)
\curveto(568.44370072,281.60684749)(568.48370068,281.61184749)(568.50370455,281.63185809)
\curveto(569.22369994,281.65184745)(569.82369934,281.53684756)(570.30370455,281.28685809)
\curveto(570.78369838,281.03684806)(571.15869801,280.6968484)(571.42870455,280.26685809)
\curveto(571.51869765,280.12684897)(571.59869757,279.98184912)(571.66870455,279.83185809)
\curveto(571.73869743,279.68184942)(571.80869736,279.52184958)(571.87870455,279.35185809)
\curveto(571.92869724,279.21184989)(571.9686972,279.06185004)(571.99870455,278.90185809)
\curveto(572.02869714,278.74185036)(572.0636971,278.58185052)(572.10370455,278.42185809)
\curveto(572.12369704,278.37185073)(572.13369703,278.31685078)(572.13370455,278.25685809)
\curveto(572.13369703,278.20685089)(572.13869703,278.15685094)(572.14870455,278.10685809)
\curveto(572.168697,278.04685105)(572.17869699,277.98185112)(572.17870455,277.91185809)
\curveto(572.17869699,277.85185125)(572.18869698,277.7968513)(572.20870455,277.74685809)
\lineto(572.20870455,277.58185809)
\curveto(572.22869694,277.53185157)(572.23369693,277.48185162)(572.22370455,277.43185809)
\curveto(572.21369695,277.38185172)(572.21869695,277.33185177)(572.23870455,277.28185809)
\curveto(572.23869693,277.26185184)(572.23369693,277.23685186)(572.22370455,277.20685809)
\curveto(572.22369694,277.17685192)(572.22869694,277.15185195)(572.23870455,277.13185809)
\curveto(572.24869692,277.101852)(572.24869692,277.06685203)(572.23870455,277.02685809)
\curveto(572.23869693,276.98685211)(572.24369692,276.94685215)(572.25370455,276.90685809)
\curveto(572.2636969,276.86685223)(572.2636969,276.82185228)(572.25370455,276.77185809)
\lineto(572.25370455,276.62185809)
\moveto(570.75370455,277.92685809)
\curveto(570.7636984,277.97685112)(570.7686984,278.03685106)(570.76870455,278.10685809)
\curveto(570.7686984,278.17685092)(570.7636984,278.23685086)(570.75370455,278.28685809)
\curveto(570.74369842,278.33685076)(570.73869843,278.41185069)(570.73870455,278.51185809)
\curveto(570.71869845,278.59185051)(570.69869847,278.66685043)(570.67870455,278.73685809)
\curveto(570.6686985,278.80685029)(570.65369851,278.87685022)(570.63370455,278.94685809)
\curveto(570.49369867,279.37684972)(570.29869887,279.71184939)(570.04870455,279.95185809)
\curveto(569.80869936,280.19184891)(569.4636997,280.37184873)(569.01370455,280.49185809)
\curveto(568.92370024,280.51184859)(568.82370034,280.52184858)(568.71370455,280.52185809)
\lineto(568.38370455,280.52185809)
\curveto(568.3637008,280.5018486)(568.32870084,280.49184861)(568.27870455,280.49185809)
\curveto(568.22870094,280.5018486)(568.18370098,280.5018486)(568.14370455,280.49185809)
\curveto(568.0637011,280.47184863)(567.98870118,280.45184865)(567.91870455,280.43185809)
\lineto(567.70870455,280.37185809)
\curveto(567.41870175,280.24184886)(567.18870198,280.06184904)(567.01870455,279.83185809)
\curveto(566.84870232,279.61184949)(566.71370245,279.35184975)(566.61370455,279.05185809)
\curveto(566.58370258,278.96185014)(566.55870261,278.86685023)(566.53870455,278.76685809)
\curveto(566.52870264,278.67685042)(566.51370265,278.58185052)(566.49370455,278.48185809)
\lineto(566.49370455,278.34685809)
\curveto(566.4637027,278.23685086)(566.45370271,278.096851)(566.46370455,277.92685809)
\curveto(566.48370268,277.76685133)(566.50370266,277.63685146)(566.52370455,277.53685809)
\curveto(566.54370262,277.47685162)(566.55870261,277.41685168)(566.56870455,277.35685809)
\curveto(566.57870259,277.30685179)(566.59370257,277.25685184)(566.61370455,277.20685809)
\curveto(566.69370247,277.00685209)(566.78870238,276.81685228)(566.89870455,276.63685809)
\curveto(567.01870215,276.45685264)(567.15870201,276.31185279)(567.31870455,276.20185809)
\curveto(567.3687018,276.15185295)(567.42370174,276.11185299)(567.48370455,276.08185809)
\curveto(567.54370162,276.05185305)(567.60370156,276.01685308)(567.66370455,275.97685809)
\curveto(567.81370135,275.8968532)(567.99870117,275.83185327)(568.21870455,275.78185809)
\curveto(568.2687009,275.76185334)(568.30870086,275.75685334)(568.33870455,275.76685809)
\curveto(568.37870079,275.77685332)(568.42370074,275.77185333)(568.47370455,275.75185809)
\curveto(568.51370065,275.74185336)(568.5687006,275.73685336)(568.63870455,275.73685809)
\curveto(568.70870046,275.73685336)(568.7687004,275.74185336)(568.81870455,275.75185809)
\curveto(568.91870025,275.77185333)(569.01370015,275.78685331)(569.10370455,275.79685809)
\curveto(569.19369997,275.81685328)(569.28369988,275.84685325)(569.37370455,275.88685809)
\curveto(569.91369925,276.10685299)(570.30869886,276.5018526)(570.55870455,277.07185809)
\curveto(570.60869856,277.17185193)(570.64369852,277.27185183)(570.66370455,277.37185809)
\curveto(570.68369848,277.48185162)(570.70869846,277.59185151)(570.73870455,277.70185809)
\curveto(570.73869843,277.8018513)(570.74369842,277.87685122)(570.75370455,277.92685809)
}
}
{
\newrgbcolor{curcolor}{0 0 0}
\pscustom[linestyle=none,fillstyle=solid,fillcolor=curcolor]
{
\newpath
\moveto(583.46831392,279.54685809)
\curveto(583.26830362,279.25684984)(583.05830383,278.97185013)(582.83831392,278.69185809)
\curveto(582.62830426,278.41185069)(582.42330447,278.12685097)(582.22331392,277.83685809)
\curveto(581.62330527,276.98685211)(581.01830587,276.14685295)(580.40831392,275.31685809)
\curveto(579.79830709,274.4968546)(579.1933077,273.66185544)(578.59331392,272.81185809)
\lineto(578.08331392,272.09185809)
\lineto(577.57331392,271.40185809)
\curveto(577.4933094,271.29185781)(577.41330948,271.17685792)(577.33331392,271.05685809)
\curveto(577.25330964,270.93685816)(577.15830973,270.84185826)(577.04831392,270.77185809)
\curveto(577.00830988,270.75185835)(576.94330995,270.73685836)(576.85331392,270.72685809)
\curveto(576.77331012,270.70685839)(576.68331021,270.6968584)(576.58331392,270.69685809)
\curveto(576.48331041,270.6968584)(576.3883105,270.7018584)(576.29831392,270.71185809)
\curveto(576.21831067,270.72185838)(576.15831073,270.74185836)(576.11831392,270.77185809)
\curveto(576.0883108,270.79185831)(576.06331083,270.82685827)(576.04331392,270.87685809)
\curveto(576.03331086,270.91685818)(576.03831085,270.96185814)(576.05831392,271.01185809)
\curveto(576.09831079,271.09185801)(576.14331075,271.16685793)(576.19331392,271.23685809)
\curveto(576.25331064,271.31685778)(576.30831058,271.3968577)(576.35831392,271.47685809)
\curveto(576.59831029,271.81685728)(576.84331005,272.15185695)(577.09331392,272.48185809)
\curveto(577.34330955,272.81185629)(577.58330931,273.14685595)(577.81331392,273.48685809)
\curveto(577.97330892,273.70685539)(578.13330876,273.92185518)(578.29331392,274.13185809)
\curveto(578.45330844,274.34185476)(578.61330828,274.55685454)(578.77331392,274.77685809)
\curveto(579.13330776,275.2968538)(579.49830739,275.80685329)(579.86831392,276.30685809)
\curveto(580.23830665,276.80685229)(580.60830628,277.31685178)(580.97831392,277.83685809)
\curveto(581.11830577,278.03685106)(581.25830563,278.23185087)(581.39831392,278.42185809)
\curveto(581.54830534,278.61185049)(581.6933052,278.80685029)(581.83331392,279.00685809)
\curveto(582.04330485,279.30684979)(582.25830463,279.60684949)(582.47831392,279.90685809)
\lineto(583.13831392,280.80685809)
\lineto(583.31831392,281.07685809)
\lineto(583.52831392,281.34685809)
\lineto(583.64831392,281.52685809)
\curveto(583.69830319,281.58684751)(583.74830314,281.64184746)(583.79831392,281.69185809)
\curveto(583.86830302,281.74184736)(583.94330295,281.77684732)(584.02331392,281.79685809)
\curveto(584.04330285,281.80684729)(584.06830282,281.80684729)(584.09831392,281.79685809)
\curveto(584.13830275,281.7968473)(584.16830272,281.80684729)(584.18831392,281.82685809)
\curveto(584.30830258,281.82684727)(584.44330245,281.82184728)(584.59331392,281.81185809)
\curveto(584.74330215,281.81184729)(584.83330206,281.76684733)(584.86331392,281.67685809)
\curveto(584.88330201,281.64684745)(584.888302,281.61184749)(584.87831392,281.57185809)
\curveto(584.86830202,281.53184757)(584.85330204,281.5018476)(584.83331392,281.48185809)
\curveto(584.7933021,281.4018477)(584.75330214,281.33184777)(584.71331392,281.27185809)
\curveto(584.67330222,281.21184789)(584.62830226,281.15184795)(584.57831392,281.09185809)
\lineto(584.00831392,280.31185809)
\curveto(583.82830306,280.06184904)(583.64830324,279.80684929)(583.46831392,279.54685809)
\moveto(576.61331392,275.64685809)
\curveto(576.56331033,275.66685343)(576.51331038,275.67185343)(576.46331392,275.66185809)
\curveto(576.41331048,275.65185345)(576.36331053,275.65685344)(576.31331392,275.67685809)
\curveto(576.20331069,275.6968534)(576.09831079,275.71685338)(575.99831392,275.73685809)
\curveto(575.90831098,275.76685333)(575.81331108,275.80685329)(575.71331392,275.85685809)
\curveto(575.38331151,275.9968531)(575.12831176,276.19185291)(574.94831392,276.44185809)
\curveto(574.76831212,276.7018524)(574.62331227,277.01185209)(574.51331392,277.37185809)
\curveto(574.48331241,277.45185165)(574.46331243,277.53185157)(574.45331392,277.61185809)
\curveto(574.44331245,277.7018514)(574.42831246,277.78685131)(574.40831392,277.86685809)
\curveto(574.39831249,277.91685118)(574.3933125,277.98185112)(574.39331392,278.06185809)
\curveto(574.38331251,278.09185101)(574.37831251,278.12185098)(574.37831392,278.15185809)
\curveto(574.37831251,278.19185091)(574.37331252,278.22685087)(574.36331392,278.25685809)
\lineto(574.36331392,278.40685809)
\curveto(574.35331254,278.45685064)(574.34831254,278.51685058)(574.34831392,278.58685809)
\curveto(574.34831254,278.66685043)(574.35331254,278.73185037)(574.36331392,278.78185809)
\lineto(574.36331392,278.94685809)
\curveto(574.38331251,278.9968501)(574.3883125,279.04185006)(574.37831392,279.08185809)
\curveto(574.37831251,279.13184997)(574.38331251,279.17684992)(574.39331392,279.21685809)
\curveto(574.40331249,279.25684984)(574.40831248,279.29184981)(574.40831392,279.32185809)
\curveto(574.40831248,279.36184974)(574.41331248,279.4018497)(574.42331392,279.44185809)
\curveto(574.45331244,279.55184955)(574.47331242,279.66184944)(574.48331392,279.77185809)
\curveto(574.50331239,279.89184921)(574.53831235,280.00684909)(574.58831392,280.11685809)
\curveto(574.72831216,280.45684864)(574.888312,280.73184837)(575.06831392,280.94185809)
\curveto(575.25831163,281.16184794)(575.52831136,281.34184776)(575.87831392,281.48185809)
\curveto(575.95831093,281.51184759)(576.04331085,281.53184757)(576.13331392,281.54185809)
\curveto(576.22331067,281.56184754)(576.31831057,281.58184752)(576.41831392,281.60185809)
\curveto(576.44831044,281.61184749)(576.50331039,281.61184749)(576.58331392,281.60185809)
\curveto(576.66331023,281.6018475)(576.71331018,281.61184749)(576.73331392,281.63185809)
\curveto(577.2933096,281.64184746)(577.74330915,281.53184757)(578.08331392,281.30185809)
\curveto(578.43330846,281.07184803)(578.6933082,280.76684833)(578.86331392,280.38685809)
\curveto(578.90330799,280.2968488)(578.93830795,280.2018489)(578.96831392,280.10185809)
\curveto(578.99830789,280.0018491)(579.02330787,279.9018492)(579.04331392,279.80185809)
\curveto(579.06330783,279.77184933)(579.06830782,279.74184936)(579.05831392,279.71185809)
\curveto(579.05830783,279.68184942)(579.06330783,279.65184945)(579.07331392,279.62185809)
\curveto(579.10330779,279.51184959)(579.12330777,279.38684971)(579.13331392,279.24685809)
\curveto(579.14330775,279.11684998)(579.15330774,278.98185012)(579.16331392,278.84185809)
\lineto(579.16331392,278.67685809)
\curveto(579.17330772,278.61685048)(579.17330772,278.56185054)(579.16331392,278.51185809)
\curveto(579.15330774,278.46185064)(579.14830774,278.41185069)(579.14831392,278.36185809)
\lineto(579.14831392,278.22685809)
\curveto(579.13830775,278.18685091)(579.13330776,278.14685095)(579.13331392,278.10685809)
\curveto(579.14330775,278.06685103)(579.13830775,278.02185108)(579.11831392,277.97185809)
\curveto(579.09830779,277.86185124)(579.07830781,277.75685134)(579.05831392,277.65685809)
\curveto(579.04830784,277.55685154)(579.02830786,277.45685164)(578.99831392,277.35685809)
\curveto(578.86830802,276.9968521)(578.70330819,276.68185242)(578.50331392,276.41185809)
\curveto(578.30330859,276.14185296)(578.02830886,275.93685316)(577.67831392,275.79685809)
\curveto(577.59830929,275.76685333)(577.51330938,275.74185336)(577.42331392,275.72185809)
\lineto(577.15331392,275.66185809)
\curveto(577.10330979,275.65185345)(577.05830983,275.64685345)(577.01831392,275.64685809)
\curveto(576.97830991,275.65685344)(576.93830995,275.65685344)(576.89831392,275.64685809)
\curveto(576.79831009,275.62685347)(576.70331019,275.62685347)(576.61331392,275.64685809)
\moveto(575.77331392,277.04185809)
\curveto(575.81331108,276.97185213)(575.85331104,276.90685219)(575.89331392,276.84685809)
\curveto(575.93331096,276.7968523)(575.98331091,276.74685235)(576.04331392,276.69685809)
\lineto(576.19331392,276.57685809)
\curveto(576.25331064,276.54685255)(576.31831057,276.52185258)(576.38831392,276.50185809)
\curveto(576.42831046,276.48185262)(576.46331043,276.47185263)(576.49331392,276.47185809)
\curveto(576.53331036,276.48185262)(576.57331032,276.47685262)(576.61331392,276.45685809)
\curveto(576.64331025,276.45685264)(576.68331021,276.45185265)(576.73331392,276.44185809)
\curveto(576.78331011,276.44185266)(576.82331007,276.44685265)(576.85331392,276.45685809)
\lineto(577.07831392,276.50185809)
\curveto(577.32830956,276.58185252)(577.51330938,276.70685239)(577.63331392,276.87685809)
\curveto(577.71330918,276.97685212)(577.78330911,277.10685199)(577.84331392,277.26685809)
\curveto(577.92330897,277.44685165)(577.98330891,277.67185143)(578.02331392,277.94185809)
\curveto(578.06330883,278.22185088)(578.07830881,278.5018506)(578.06831392,278.78185809)
\curveto(578.05830883,279.07185003)(578.02830886,279.34684975)(577.97831392,279.60685809)
\curveto(577.92830896,279.86684923)(577.85330904,280.07684902)(577.75331392,280.23685809)
\curveto(577.63330926,280.43684866)(577.48330941,280.58684851)(577.30331392,280.68685809)
\curveto(577.22330967,280.73684836)(577.13330976,280.76684833)(577.03331392,280.77685809)
\curveto(576.93330996,280.7968483)(576.82831006,280.80684829)(576.71831392,280.80685809)
\curveto(576.69831019,280.7968483)(576.67331022,280.79184831)(576.64331392,280.79185809)
\curveto(576.62331027,280.8018483)(576.60331029,280.8018483)(576.58331392,280.79185809)
\curveto(576.53331036,280.78184832)(576.4883104,280.77184833)(576.44831392,280.76185809)
\curveto(576.40831048,280.76184834)(576.36831052,280.75184835)(576.32831392,280.73185809)
\curveto(576.14831074,280.65184845)(575.99831089,280.53184857)(575.87831392,280.37185809)
\curveto(575.76831112,280.21184889)(575.67831121,280.03184907)(575.60831392,279.83185809)
\curveto(575.54831134,279.64184946)(575.50331139,279.41684968)(575.47331392,279.15685809)
\curveto(575.45331144,278.8968502)(575.44831144,278.63185047)(575.45831392,278.36185809)
\curveto(575.46831142,278.101851)(575.49831139,277.85185125)(575.54831392,277.61185809)
\curveto(575.60831128,277.38185172)(575.68331121,277.19185191)(575.77331392,277.04185809)
\moveto(586.57331392,274.05685809)
\curveto(586.58330031,274.00685509)(586.5883003,273.91685518)(586.58831392,273.78685809)
\curveto(586.5883003,273.65685544)(586.57830031,273.56685553)(586.55831392,273.51685809)
\curveto(586.53830035,273.46685563)(586.53330036,273.41185569)(586.54331392,273.35185809)
\curveto(586.55330034,273.3018558)(586.55330034,273.25185585)(586.54331392,273.20185809)
\curveto(586.50330039,273.06185604)(586.47330042,272.92685617)(586.45331392,272.79685809)
\curveto(586.44330045,272.66685643)(586.41330048,272.54685655)(586.36331392,272.43685809)
\curveto(586.22330067,272.08685701)(586.05830083,271.79185731)(585.86831392,271.55185809)
\curveto(585.67830121,271.32185778)(585.40830148,271.13685796)(585.05831392,270.99685809)
\curveto(584.97830191,270.96685813)(584.893302,270.94685815)(584.80331392,270.93685809)
\curveto(584.71330218,270.91685818)(584.62830226,270.8968582)(584.54831392,270.87685809)
\curveto(584.49830239,270.86685823)(584.44830244,270.86185824)(584.39831392,270.86185809)
\curveto(584.34830254,270.86185824)(584.29830259,270.85685824)(584.24831392,270.84685809)
\curveto(584.21830267,270.83685826)(584.16830272,270.83685826)(584.09831392,270.84685809)
\curveto(584.02830286,270.84685825)(583.97830291,270.85185825)(583.94831392,270.86185809)
\curveto(583.888303,270.88185822)(583.82830306,270.89185821)(583.76831392,270.89185809)
\curveto(583.71830317,270.88185822)(583.66830322,270.88685821)(583.61831392,270.90685809)
\curveto(583.52830336,270.92685817)(583.43830345,270.95185815)(583.34831392,270.98185809)
\curveto(583.26830362,271.0018581)(583.1883037,271.03185807)(583.10831392,271.07185809)
\curveto(582.7883041,271.21185789)(582.53830435,271.40685769)(582.35831392,271.65685809)
\curveto(582.17830471,271.91685718)(582.02830486,272.22185688)(581.90831392,272.57185809)
\curveto(581.888305,272.65185645)(581.87330502,272.73685636)(581.86331392,272.82685809)
\curveto(581.85330504,272.91685618)(581.83830505,273.0018561)(581.81831392,273.08185809)
\curveto(581.80830508,273.11185599)(581.80330509,273.14185596)(581.80331392,273.17185809)
\lineto(581.80331392,273.27685809)
\curveto(581.78330511,273.35685574)(581.77330512,273.43685566)(581.77331392,273.51685809)
\lineto(581.77331392,273.65185809)
\curveto(581.75330514,273.75185535)(581.75330514,273.85185525)(581.77331392,273.95185809)
\lineto(581.77331392,274.13185809)
\curveto(581.78330511,274.18185492)(581.7883051,274.22685487)(581.78831392,274.26685809)
\curveto(581.7883051,274.31685478)(581.7933051,274.36185474)(581.80331392,274.40185809)
\curveto(581.81330508,274.44185466)(581.81830507,274.47685462)(581.81831392,274.50685809)
\curveto(581.81830507,274.54685455)(581.82330507,274.58685451)(581.83331392,274.62685809)
\lineto(581.89331392,274.95685809)
\curveto(581.91330498,275.07685402)(581.94330495,275.18685391)(581.98331392,275.28685809)
\curveto(582.12330477,275.61685348)(582.28330461,275.89185321)(582.46331392,276.11185809)
\curveto(582.65330424,276.34185276)(582.91330398,276.52685257)(583.24331392,276.66685809)
\curveto(583.32330357,276.70685239)(583.40830348,276.73185237)(583.49831392,276.74185809)
\lineto(583.79831392,276.80185809)
\lineto(583.93331392,276.80185809)
\curveto(583.98330291,276.81185229)(584.03330286,276.81685228)(584.08331392,276.81685809)
\curveto(584.65330224,276.83685226)(585.11330178,276.73185237)(585.46331392,276.50185809)
\curveto(585.82330107,276.28185282)(586.0883008,275.98185312)(586.25831392,275.60185809)
\curveto(586.30830058,275.5018536)(586.34830054,275.4018537)(586.37831392,275.30185809)
\curveto(586.40830048,275.2018539)(586.43830045,275.096854)(586.46831392,274.98685809)
\curveto(586.47830041,274.94685415)(586.48330041,274.91185419)(586.48331392,274.88185809)
\curveto(586.48330041,274.86185424)(586.4883004,274.83185427)(586.49831392,274.79185809)
\curveto(586.51830037,274.72185438)(586.52830036,274.64685445)(586.52831392,274.56685809)
\curveto(586.52830036,274.48685461)(586.53830035,274.40685469)(586.55831392,274.32685809)
\curveto(586.55830033,274.27685482)(586.55830033,274.23185487)(586.55831392,274.19185809)
\curveto(586.55830033,274.15185495)(586.56330033,274.10685499)(586.57331392,274.05685809)
\moveto(585.46331392,273.62185809)
\curveto(585.47330142,273.67185543)(585.47830141,273.74685535)(585.47831392,273.84685809)
\curveto(585.4883014,273.94685515)(585.48330141,274.02185508)(585.46331392,274.07185809)
\curveto(585.44330145,274.13185497)(585.43830145,274.18685491)(585.44831392,274.23685809)
\curveto(585.46830142,274.2968548)(585.46830142,274.35685474)(585.44831392,274.41685809)
\curveto(585.43830145,274.44685465)(585.43330146,274.48185462)(585.43331392,274.52185809)
\curveto(585.43330146,274.56185454)(585.42830146,274.6018545)(585.41831392,274.64185809)
\curveto(585.39830149,274.72185438)(585.37830151,274.7968543)(585.35831392,274.86685809)
\curveto(585.34830154,274.94685415)(585.33330156,275.02685407)(585.31331392,275.10685809)
\curveto(585.28330161,275.16685393)(585.25830163,275.22685387)(585.23831392,275.28685809)
\curveto(585.21830167,275.34685375)(585.1883017,275.40685369)(585.14831392,275.46685809)
\curveto(585.04830184,275.63685346)(584.91830197,275.77185333)(584.75831392,275.87185809)
\curveto(584.67830221,275.92185318)(584.58330231,275.95685314)(584.47331392,275.97685809)
\curveto(584.36330253,275.9968531)(584.23830265,276.00685309)(584.09831392,276.00685809)
\curveto(584.07830281,275.9968531)(584.05330284,275.99185311)(584.02331392,275.99185809)
\curveto(583.9933029,276.0018531)(583.96330293,276.0018531)(583.93331392,275.99185809)
\lineto(583.78331392,275.93185809)
\curveto(583.73330316,275.92185318)(583.6883032,275.90685319)(583.64831392,275.88685809)
\curveto(583.45830343,275.77685332)(583.31330358,275.63185347)(583.21331392,275.45185809)
\curveto(583.12330377,275.27185383)(583.04330385,275.06685403)(582.97331392,274.83685809)
\curveto(582.93330396,274.70685439)(582.91330398,274.57185453)(582.91331392,274.43185809)
\curveto(582.91330398,274.3018548)(582.90330399,274.15685494)(582.88331392,273.99685809)
\curveto(582.87330402,273.94685515)(582.86330403,273.88685521)(582.85331392,273.81685809)
\curveto(582.85330404,273.74685535)(582.86330403,273.68685541)(582.88331392,273.63685809)
\lineto(582.88331392,273.47185809)
\lineto(582.88331392,273.29185809)
\curveto(582.893304,273.24185586)(582.90330399,273.18685591)(582.91331392,273.12685809)
\curveto(582.92330397,273.07685602)(582.92830396,273.02185608)(582.92831392,272.96185809)
\curveto(582.93830395,272.9018562)(582.95330394,272.84685625)(582.97331392,272.79685809)
\curveto(583.02330387,272.60685649)(583.08330381,272.43185667)(583.15331392,272.27185809)
\curveto(583.22330367,272.11185699)(583.32830356,271.98185712)(583.46831392,271.88185809)
\curveto(583.59830329,271.78185732)(583.73830315,271.71185739)(583.88831392,271.67185809)
\curveto(583.91830297,271.66185744)(583.94330295,271.65685744)(583.96331392,271.65685809)
\curveto(583.9933029,271.66685743)(584.02330287,271.66685743)(584.05331392,271.65685809)
\curveto(584.07330282,271.65685744)(584.10330279,271.65185745)(584.14331392,271.64185809)
\curveto(584.18330271,271.64185746)(584.21830267,271.64685745)(584.24831392,271.65685809)
\curveto(584.2883026,271.66685743)(584.32830256,271.67185743)(584.36831392,271.67185809)
\curveto(584.40830248,271.67185743)(584.44830244,271.68185742)(584.48831392,271.70185809)
\curveto(584.72830216,271.78185732)(584.92330197,271.91685718)(585.07331392,272.10685809)
\curveto(585.1933017,272.28685681)(585.28330161,272.49185661)(585.34331392,272.72185809)
\curveto(585.36330153,272.79185631)(585.37830151,272.86185624)(585.38831392,272.93185809)
\curveto(585.39830149,273.01185609)(585.41330148,273.09185601)(585.43331392,273.17185809)
\curveto(585.43330146,273.23185587)(585.43830145,273.27685582)(585.44831392,273.30685809)
\curveto(585.44830144,273.32685577)(585.44830144,273.35185575)(585.44831392,273.38185809)
\curveto(585.44830144,273.42185568)(585.45330144,273.45185565)(585.46331392,273.47185809)
\lineto(585.46331392,273.62185809)
}
}
{
\newrgbcolor{curcolor}{0 0 0}
\pscustom[linestyle=none,fillstyle=solid,fillcolor=curcolor]
{
\newpath
\moveto(478.87893038,144.25112078)
\curveto(480.50892494,144.28111013)(481.55892389,143.72611069)(482.02893038,142.58612078)
\curveto(482.12892332,142.35611206)(482.19392325,142.06611235)(482.22393038,141.71612078)
\curveto(482.26392318,141.37611304)(482.23892321,141.06611335)(482.14893038,140.78612078)
\curveto(482.05892339,140.52611389)(481.93892351,140.30111411)(481.78893038,140.11112078)
\curveto(481.76892368,140.07111434)(481.7439237,140.03611438)(481.71393038,140.00612078)
\curveto(481.68392376,139.98611443)(481.65892379,139.96111445)(481.63893038,139.93112078)
\lineto(481.54893038,139.81112078)
\curveto(481.51892393,139.78111463)(481.48392396,139.75611466)(481.44393038,139.73612078)
\curveto(481.39392405,139.68611473)(481.33892411,139.64111477)(481.27893038,139.60112078)
\curveto(481.22892422,139.56111485)(481.18392426,139.5111149)(481.14393038,139.45112078)
\curveto(481.10392434,139.411115)(481.08892436,139.36111505)(481.09893038,139.30112078)
\curveto(481.10892434,139.25111516)(481.13892431,139.20611521)(481.18893038,139.16612078)
\curveto(481.23892421,139.12611529)(481.29392415,139.08611533)(481.35393038,139.04612078)
\curveto(481.42392402,139.0161154)(481.48892396,138.98611543)(481.54893038,138.95612078)
\curveto(481.60892384,138.92611549)(481.65892379,138.89611552)(481.69893038,138.86612078)
\curveto(482.01892343,138.64611577)(482.27392317,138.33611608)(482.46393038,137.93612078)
\curveto(482.50392294,137.84611657)(482.53392291,137.75111666)(482.55393038,137.65112078)
\curveto(482.58392286,137.56111685)(482.60892284,137.47111694)(482.62893038,137.38112078)
\curveto(482.63892281,137.33111708)(482.6439228,137.28111713)(482.64393038,137.23112078)
\curveto(482.65392279,137.19111722)(482.66392278,137.14611727)(482.67393038,137.09612078)
\curveto(482.68392276,137.04611737)(482.68392276,136.99611742)(482.67393038,136.94612078)
\curveto(482.66392278,136.89611752)(482.66892278,136.84611757)(482.68893038,136.79612078)
\curveto(482.69892275,136.74611767)(482.70392274,136.68611773)(482.70393038,136.61612078)
\curveto(482.70392274,136.54611787)(482.69392275,136.48611793)(482.67393038,136.43612078)
\lineto(482.67393038,136.21112078)
\lineto(482.61393038,135.97112078)
\curveto(482.60392284,135.90111851)(482.58892286,135.83111858)(482.56893038,135.76112078)
\curveto(482.53892291,135.67111874)(482.50892294,135.58611883)(482.47893038,135.50612078)
\curveto(482.45892299,135.42611899)(482.42892302,135.34611907)(482.38893038,135.26612078)
\curveto(482.36892308,135.20611921)(482.33892311,135.14611927)(482.29893038,135.08612078)
\curveto(482.26892318,135.03611938)(482.23392321,134.98611943)(482.19393038,134.93612078)
\curveto(481.99392345,134.62611979)(481.7439237,134.36612005)(481.44393038,134.15612078)
\curveto(481.1439243,133.95612046)(480.79892465,133.79112062)(480.40893038,133.66112078)
\curveto(480.28892516,133.62112079)(480.15892529,133.59612082)(480.01893038,133.58612078)
\curveto(479.88892556,133.56612085)(479.75392569,133.54112087)(479.61393038,133.51112078)
\curveto(479.5439259,133.50112091)(479.47392597,133.49612092)(479.40393038,133.49612078)
\curveto(479.3439261,133.49612092)(479.27892617,133.49112092)(479.20893038,133.48112078)
\curveto(479.16892628,133.47112094)(479.10892634,133.46612095)(479.02893038,133.46612078)
\curveto(478.95892649,133.46612095)(478.90892654,133.47112094)(478.87893038,133.48112078)
\curveto(478.82892662,133.49112092)(478.78392666,133.49612092)(478.74393038,133.49612078)
\lineto(478.62393038,133.49612078)
\curveto(478.52392692,133.5161209)(478.42392702,133.53112088)(478.32393038,133.54112078)
\curveto(478.22392722,133.55112086)(478.12892732,133.56612085)(478.03893038,133.58612078)
\curveto(477.92892752,133.6161208)(477.81892763,133.64112077)(477.70893038,133.66112078)
\curveto(477.60892784,133.69112072)(477.50392794,133.73112068)(477.39393038,133.78112078)
\curveto(477.02392842,133.94112047)(476.70892874,134.14112027)(476.44893038,134.38112078)
\curveto(476.18892926,134.63111978)(475.97892947,134.94111947)(475.81893038,135.31112078)
\curveto(475.77892967,135.40111901)(475.7439297,135.49611892)(475.71393038,135.59612078)
\curveto(475.68392976,135.69611872)(475.65392979,135.80111861)(475.62393038,135.91112078)
\curveto(475.60392984,135.96111845)(475.59392985,136.0111184)(475.59393038,136.06112078)
\curveto(475.59392985,136.12111829)(475.58392986,136.18111823)(475.56393038,136.24112078)
\curveto(475.5439299,136.30111811)(475.53392991,136.38111803)(475.53393038,136.48112078)
\curveto(475.53392991,136.58111783)(475.5489299,136.65611776)(475.57893038,136.70612078)
\curveto(475.58892986,136.73611768)(475.60392984,136.76111765)(475.62393038,136.78112078)
\lineto(475.68393038,136.84112078)
\curveto(475.72392972,136.86111755)(475.78392966,136.87611754)(475.86393038,136.88612078)
\curveto(475.95392949,136.89611752)(476.0439294,136.90111751)(476.13393038,136.90112078)
\curveto(476.22392922,136.90111751)(476.30892914,136.89611752)(476.38893038,136.88612078)
\curveto(476.47892897,136.87611754)(476.5439289,136.86611755)(476.58393038,136.85612078)
\curveto(476.60392884,136.83611758)(476.62392882,136.82111759)(476.64393038,136.81112078)
\curveto(476.66392878,136.8111176)(476.68392876,136.80111761)(476.70393038,136.78112078)
\curveto(476.77392867,136.69111772)(476.81392863,136.57611784)(476.82393038,136.43612078)
\curveto(476.8439286,136.29611812)(476.87392857,136.17111824)(476.91393038,136.06112078)
\lineto(477.06393038,135.70112078)
\curveto(477.11392833,135.59111882)(477.17892827,135.48611893)(477.25893038,135.38612078)
\curveto(477.27892817,135.35611906)(477.29892815,135.33111908)(477.31893038,135.31112078)
\curveto(477.3489281,135.29111912)(477.37392807,135.26611915)(477.39393038,135.23612078)
\curveto(477.43392801,135.17611924)(477.46892798,135.13111928)(477.49893038,135.10112078)
\curveto(477.53892791,135.07111934)(477.57392787,135.04111937)(477.60393038,135.01112078)
\curveto(477.6439278,134.98111943)(477.68892776,134.95111946)(477.73893038,134.92112078)
\curveto(477.82892762,134.86111955)(477.92392752,134.8111196)(478.02393038,134.77112078)
\lineto(478.35393038,134.65112078)
\curveto(478.50392694,134.60111981)(478.70392674,134.57111984)(478.95393038,134.56112078)
\curveto(479.20392624,134.55111986)(479.41392603,134.57111984)(479.58393038,134.62112078)
\curveto(479.66392578,134.64111977)(479.73392571,134.65611976)(479.79393038,134.66612078)
\lineto(480.00393038,134.72612078)
\curveto(480.28392516,134.84611957)(480.52392492,134.99611942)(480.72393038,135.17612078)
\curveto(480.93392451,135.35611906)(481.09892435,135.58611883)(481.21893038,135.86612078)
\curveto(481.2489242,135.93611848)(481.26892418,136.00611841)(481.27893038,136.07612078)
\lineto(481.33893038,136.31612078)
\curveto(481.37892407,136.45611796)(481.38892406,136.6161178)(481.36893038,136.79612078)
\curveto(481.3489241,136.98611743)(481.31892413,137.13611728)(481.27893038,137.24612078)
\curveto(481.1489243,137.62611679)(480.96392448,137.9161165)(480.72393038,138.11612078)
\curveto(480.49392495,138.3161161)(480.18392526,138.47611594)(479.79393038,138.59612078)
\curveto(479.68392576,138.62611579)(479.56392588,138.64611577)(479.43393038,138.65612078)
\curveto(479.31392613,138.66611575)(479.18892626,138.67111574)(479.05893038,138.67112078)
\curveto(478.89892655,138.67111574)(478.75892669,138.67611574)(478.63893038,138.68612078)
\curveto(478.51892693,138.69611572)(478.43392701,138.75611566)(478.38393038,138.86612078)
\curveto(478.36392708,138.89611552)(478.35392709,138.93111548)(478.35393038,138.97112078)
\lineto(478.35393038,139.10612078)
\curveto(478.3439271,139.20611521)(478.3439271,139.30111511)(478.35393038,139.39112078)
\curveto(478.37392707,139.48111493)(478.41392703,139.54611487)(478.47393038,139.58612078)
\curveto(478.51392693,139.6161148)(478.55392689,139.63611478)(478.59393038,139.64612078)
\curveto(478.6439268,139.65611476)(478.69892675,139.66611475)(478.75893038,139.67612078)
\curveto(478.77892667,139.68611473)(478.80392664,139.68611473)(478.83393038,139.67612078)
\curveto(478.86392658,139.67611474)(478.88892656,139.68111473)(478.90893038,139.69112078)
\lineto(479.04393038,139.69112078)
\curveto(479.15392629,139.7111147)(479.25392619,139.72111469)(479.34393038,139.72112078)
\curveto(479.443926,139.73111468)(479.53892591,139.75111466)(479.62893038,139.78112078)
\curveto(479.9489255,139.89111452)(480.20392524,140.03611438)(480.39393038,140.21612078)
\curveto(480.58392486,140.39611402)(480.73392471,140.64611377)(480.84393038,140.96612078)
\curveto(480.87392457,141.06611335)(480.89392455,141.19111322)(480.90393038,141.34112078)
\curveto(480.92392452,141.50111291)(480.91892453,141.64611277)(480.88893038,141.77612078)
\curveto(480.86892458,141.84611257)(480.8489246,141.9111125)(480.82893038,141.97112078)
\curveto(480.81892463,142.04111237)(480.79892465,142.10611231)(480.76893038,142.16612078)
\curveto(480.66892478,142.40611201)(480.52392492,142.59611182)(480.33393038,142.73612078)
\curveto(480.1439253,142.87611154)(479.91892553,142.98611143)(479.65893038,143.06612078)
\curveto(479.59892585,143.08611133)(479.53892591,143.09611132)(479.47893038,143.09612078)
\curveto(479.41892603,143.09611132)(479.35392609,143.10611131)(479.28393038,143.12612078)
\curveto(479.20392624,143.14611127)(479.10892634,143.15611126)(478.99893038,143.15612078)
\curveto(478.88892656,143.15611126)(478.79392665,143.14611127)(478.71393038,143.12612078)
\curveto(478.66392678,143.10611131)(478.61392683,143.09611132)(478.56393038,143.09612078)
\curveto(478.52392692,143.09611132)(478.47892697,143.08611133)(478.42893038,143.06612078)
\curveto(478.2489272,143.0161114)(478.07892737,142.94111147)(477.91893038,142.84112078)
\curveto(477.76892768,142.75111166)(477.63892781,142.63611178)(477.52893038,142.49612078)
\curveto(477.43892801,142.37611204)(477.35892809,142.24611217)(477.28893038,142.10612078)
\curveto(477.21892823,141.96611245)(477.15392829,141.8111126)(477.09393038,141.64112078)
\curveto(477.06392838,141.53111288)(477.0439284,141.411113)(477.03393038,141.28112078)
\curveto(477.02392842,141.16111325)(476.98892846,141.06111335)(476.92893038,140.98112078)
\curveto(476.90892854,140.94111347)(476.8489286,140.90111351)(476.74893038,140.86112078)
\curveto(476.70892874,140.85111356)(476.6489288,140.84111357)(476.56893038,140.83112078)
\lineto(476.31393038,140.83112078)
\curveto(476.22392922,140.84111357)(476.13892931,140.85111356)(476.05893038,140.86112078)
\curveto(475.98892946,140.87111354)(475.93892951,140.88611353)(475.90893038,140.90612078)
\curveto(475.86892958,140.93611348)(475.83392961,140.99111342)(475.80393038,141.07112078)
\curveto(475.77392967,141.15111326)(475.76892968,141.23611318)(475.78893038,141.32612078)
\curveto(475.79892965,141.37611304)(475.80392964,141.42611299)(475.80393038,141.47612078)
\lineto(475.83393038,141.65612078)
\curveto(475.86392958,141.75611266)(475.88892956,141.85611256)(475.90893038,141.95612078)
\curveto(475.93892951,142.05611236)(475.97392947,142.14611227)(476.01393038,142.22612078)
\curveto(476.06392938,142.33611208)(476.10892934,142.44111197)(476.14893038,142.54112078)
\curveto(476.18892926,142.65111176)(476.23892921,142.75611166)(476.29893038,142.85612078)
\curveto(476.62892882,143.39611102)(477.09892835,143.79111062)(477.70893038,144.04112078)
\curveto(477.82892762,144.09111032)(477.95392749,144.12611029)(478.08393038,144.14612078)
\curveto(478.22392722,144.16611025)(478.36392708,144.19111022)(478.50393038,144.22112078)
\curveto(478.56392688,144.23111018)(478.62392682,144.23611018)(478.68393038,144.23612078)
\curveto(478.75392669,144.23611018)(478.81892663,144.24111017)(478.87893038,144.25112078)
}
}
{
\newrgbcolor{curcolor}{0 0 0}
\pscustom[linestyle=none,fillstyle=solid,fillcolor=curcolor]
{
\newpath
\moveto(487.22853975,144.25112078)
\curveto(488.85853431,144.28111013)(489.90853326,143.72611069)(490.37853975,142.58612078)
\curveto(490.47853269,142.35611206)(490.54353263,142.06611235)(490.57353975,141.71612078)
\curveto(490.61353256,141.37611304)(490.58853258,141.06611335)(490.49853975,140.78612078)
\curveto(490.40853276,140.52611389)(490.28853288,140.30111411)(490.13853975,140.11112078)
\curveto(490.11853305,140.07111434)(490.09353308,140.03611438)(490.06353975,140.00612078)
\curveto(490.03353314,139.98611443)(490.00853316,139.96111445)(489.98853975,139.93112078)
\lineto(489.89853975,139.81112078)
\curveto(489.8685333,139.78111463)(489.83353334,139.75611466)(489.79353975,139.73612078)
\curveto(489.74353343,139.68611473)(489.68853348,139.64111477)(489.62853975,139.60112078)
\curveto(489.57853359,139.56111485)(489.53353364,139.5111149)(489.49353975,139.45112078)
\curveto(489.45353372,139.411115)(489.43853373,139.36111505)(489.44853975,139.30112078)
\curveto(489.45853371,139.25111516)(489.48853368,139.20611521)(489.53853975,139.16612078)
\curveto(489.58853358,139.12611529)(489.64353353,139.08611533)(489.70353975,139.04612078)
\curveto(489.7735334,139.0161154)(489.83853333,138.98611543)(489.89853975,138.95612078)
\curveto(489.95853321,138.92611549)(490.00853316,138.89611552)(490.04853975,138.86612078)
\curveto(490.3685328,138.64611577)(490.62353255,138.33611608)(490.81353975,137.93612078)
\curveto(490.85353232,137.84611657)(490.88353229,137.75111666)(490.90353975,137.65112078)
\curveto(490.93353224,137.56111685)(490.95853221,137.47111694)(490.97853975,137.38112078)
\curveto(490.98853218,137.33111708)(490.99353218,137.28111713)(490.99353975,137.23112078)
\curveto(491.00353217,137.19111722)(491.01353216,137.14611727)(491.02353975,137.09612078)
\curveto(491.03353214,137.04611737)(491.03353214,136.99611742)(491.02353975,136.94612078)
\curveto(491.01353216,136.89611752)(491.01853215,136.84611757)(491.03853975,136.79612078)
\curveto(491.04853212,136.74611767)(491.05353212,136.68611773)(491.05353975,136.61612078)
\curveto(491.05353212,136.54611787)(491.04353213,136.48611793)(491.02353975,136.43612078)
\lineto(491.02353975,136.21112078)
\lineto(490.96353975,135.97112078)
\curveto(490.95353222,135.90111851)(490.93853223,135.83111858)(490.91853975,135.76112078)
\curveto(490.88853228,135.67111874)(490.85853231,135.58611883)(490.82853975,135.50612078)
\curveto(490.80853236,135.42611899)(490.77853239,135.34611907)(490.73853975,135.26612078)
\curveto(490.71853245,135.20611921)(490.68853248,135.14611927)(490.64853975,135.08612078)
\curveto(490.61853255,135.03611938)(490.58353259,134.98611943)(490.54353975,134.93612078)
\curveto(490.34353283,134.62611979)(490.09353308,134.36612005)(489.79353975,134.15612078)
\curveto(489.49353368,133.95612046)(489.14853402,133.79112062)(488.75853975,133.66112078)
\curveto(488.63853453,133.62112079)(488.50853466,133.59612082)(488.36853975,133.58612078)
\curveto(488.23853493,133.56612085)(488.10353507,133.54112087)(487.96353975,133.51112078)
\curveto(487.89353528,133.50112091)(487.82353535,133.49612092)(487.75353975,133.49612078)
\curveto(487.69353548,133.49612092)(487.62853554,133.49112092)(487.55853975,133.48112078)
\curveto(487.51853565,133.47112094)(487.45853571,133.46612095)(487.37853975,133.46612078)
\curveto(487.30853586,133.46612095)(487.25853591,133.47112094)(487.22853975,133.48112078)
\curveto(487.17853599,133.49112092)(487.13353604,133.49612092)(487.09353975,133.49612078)
\lineto(486.97353975,133.49612078)
\curveto(486.8735363,133.5161209)(486.7735364,133.53112088)(486.67353975,133.54112078)
\curveto(486.5735366,133.55112086)(486.47853669,133.56612085)(486.38853975,133.58612078)
\curveto(486.27853689,133.6161208)(486.168537,133.64112077)(486.05853975,133.66112078)
\curveto(485.95853721,133.69112072)(485.85353732,133.73112068)(485.74353975,133.78112078)
\curveto(485.3735378,133.94112047)(485.05853811,134.14112027)(484.79853975,134.38112078)
\curveto(484.53853863,134.63111978)(484.32853884,134.94111947)(484.16853975,135.31112078)
\curveto(484.12853904,135.40111901)(484.09353908,135.49611892)(484.06353975,135.59612078)
\curveto(484.03353914,135.69611872)(484.00353917,135.80111861)(483.97353975,135.91112078)
\curveto(483.95353922,135.96111845)(483.94353923,136.0111184)(483.94353975,136.06112078)
\curveto(483.94353923,136.12111829)(483.93353924,136.18111823)(483.91353975,136.24112078)
\curveto(483.89353928,136.30111811)(483.88353929,136.38111803)(483.88353975,136.48112078)
\curveto(483.88353929,136.58111783)(483.89853927,136.65611776)(483.92853975,136.70612078)
\curveto(483.93853923,136.73611768)(483.95353922,136.76111765)(483.97353975,136.78112078)
\lineto(484.03353975,136.84112078)
\curveto(484.0735391,136.86111755)(484.13353904,136.87611754)(484.21353975,136.88612078)
\curveto(484.30353887,136.89611752)(484.39353878,136.90111751)(484.48353975,136.90112078)
\curveto(484.5735386,136.90111751)(484.65853851,136.89611752)(484.73853975,136.88612078)
\curveto(484.82853834,136.87611754)(484.89353828,136.86611755)(484.93353975,136.85612078)
\curveto(484.95353822,136.83611758)(484.9735382,136.82111759)(484.99353975,136.81112078)
\curveto(485.01353816,136.8111176)(485.03353814,136.80111761)(485.05353975,136.78112078)
\curveto(485.12353805,136.69111772)(485.16353801,136.57611784)(485.17353975,136.43612078)
\curveto(485.19353798,136.29611812)(485.22353795,136.17111824)(485.26353975,136.06112078)
\lineto(485.41353975,135.70112078)
\curveto(485.46353771,135.59111882)(485.52853764,135.48611893)(485.60853975,135.38612078)
\curveto(485.62853754,135.35611906)(485.64853752,135.33111908)(485.66853975,135.31112078)
\curveto(485.69853747,135.29111912)(485.72353745,135.26611915)(485.74353975,135.23612078)
\curveto(485.78353739,135.17611924)(485.81853735,135.13111928)(485.84853975,135.10112078)
\curveto(485.88853728,135.07111934)(485.92353725,135.04111937)(485.95353975,135.01112078)
\curveto(485.99353718,134.98111943)(486.03853713,134.95111946)(486.08853975,134.92112078)
\curveto(486.17853699,134.86111955)(486.2735369,134.8111196)(486.37353975,134.77112078)
\lineto(486.70353975,134.65112078)
\curveto(486.85353632,134.60111981)(487.05353612,134.57111984)(487.30353975,134.56112078)
\curveto(487.55353562,134.55111986)(487.76353541,134.57111984)(487.93353975,134.62112078)
\curveto(488.01353516,134.64111977)(488.08353509,134.65611976)(488.14353975,134.66612078)
\lineto(488.35353975,134.72612078)
\curveto(488.63353454,134.84611957)(488.8735343,134.99611942)(489.07353975,135.17612078)
\curveto(489.28353389,135.35611906)(489.44853372,135.58611883)(489.56853975,135.86612078)
\curveto(489.59853357,135.93611848)(489.61853355,136.00611841)(489.62853975,136.07612078)
\lineto(489.68853975,136.31612078)
\curveto(489.72853344,136.45611796)(489.73853343,136.6161178)(489.71853975,136.79612078)
\curveto(489.69853347,136.98611743)(489.6685335,137.13611728)(489.62853975,137.24612078)
\curveto(489.49853367,137.62611679)(489.31353386,137.9161165)(489.07353975,138.11612078)
\curveto(488.84353433,138.3161161)(488.53353464,138.47611594)(488.14353975,138.59612078)
\curveto(488.03353514,138.62611579)(487.91353526,138.64611577)(487.78353975,138.65612078)
\curveto(487.66353551,138.66611575)(487.53853563,138.67111574)(487.40853975,138.67112078)
\curveto(487.24853592,138.67111574)(487.10853606,138.67611574)(486.98853975,138.68612078)
\curveto(486.8685363,138.69611572)(486.78353639,138.75611566)(486.73353975,138.86612078)
\curveto(486.71353646,138.89611552)(486.70353647,138.93111548)(486.70353975,138.97112078)
\lineto(486.70353975,139.10612078)
\curveto(486.69353648,139.20611521)(486.69353648,139.30111511)(486.70353975,139.39112078)
\curveto(486.72353645,139.48111493)(486.76353641,139.54611487)(486.82353975,139.58612078)
\curveto(486.86353631,139.6161148)(486.90353627,139.63611478)(486.94353975,139.64612078)
\curveto(486.99353618,139.65611476)(487.04853612,139.66611475)(487.10853975,139.67612078)
\curveto(487.12853604,139.68611473)(487.15353602,139.68611473)(487.18353975,139.67612078)
\curveto(487.21353596,139.67611474)(487.23853593,139.68111473)(487.25853975,139.69112078)
\lineto(487.39353975,139.69112078)
\curveto(487.50353567,139.7111147)(487.60353557,139.72111469)(487.69353975,139.72112078)
\curveto(487.79353538,139.73111468)(487.88853528,139.75111466)(487.97853975,139.78112078)
\curveto(488.29853487,139.89111452)(488.55353462,140.03611438)(488.74353975,140.21612078)
\curveto(488.93353424,140.39611402)(489.08353409,140.64611377)(489.19353975,140.96612078)
\curveto(489.22353395,141.06611335)(489.24353393,141.19111322)(489.25353975,141.34112078)
\curveto(489.2735339,141.50111291)(489.2685339,141.64611277)(489.23853975,141.77612078)
\curveto(489.21853395,141.84611257)(489.19853397,141.9111125)(489.17853975,141.97112078)
\curveto(489.168534,142.04111237)(489.14853402,142.10611231)(489.11853975,142.16612078)
\curveto(489.01853415,142.40611201)(488.8735343,142.59611182)(488.68353975,142.73612078)
\curveto(488.49353468,142.87611154)(488.2685349,142.98611143)(488.00853975,143.06612078)
\curveto(487.94853522,143.08611133)(487.88853528,143.09611132)(487.82853975,143.09612078)
\curveto(487.7685354,143.09611132)(487.70353547,143.10611131)(487.63353975,143.12612078)
\curveto(487.55353562,143.14611127)(487.45853571,143.15611126)(487.34853975,143.15612078)
\curveto(487.23853593,143.15611126)(487.14353603,143.14611127)(487.06353975,143.12612078)
\curveto(487.01353616,143.10611131)(486.96353621,143.09611132)(486.91353975,143.09612078)
\curveto(486.8735363,143.09611132)(486.82853634,143.08611133)(486.77853975,143.06612078)
\curveto(486.59853657,143.0161114)(486.42853674,142.94111147)(486.26853975,142.84112078)
\curveto(486.11853705,142.75111166)(485.98853718,142.63611178)(485.87853975,142.49612078)
\curveto(485.78853738,142.37611204)(485.70853746,142.24611217)(485.63853975,142.10612078)
\curveto(485.5685376,141.96611245)(485.50353767,141.8111126)(485.44353975,141.64112078)
\curveto(485.41353776,141.53111288)(485.39353778,141.411113)(485.38353975,141.28112078)
\curveto(485.3735378,141.16111325)(485.33853783,141.06111335)(485.27853975,140.98112078)
\curveto(485.25853791,140.94111347)(485.19853797,140.90111351)(485.09853975,140.86112078)
\curveto(485.05853811,140.85111356)(484.99853817,140.84111357)(484.91853975,140.83112078)
\lineto(484.66353975,140.83112078)
\curveto(484.5735386,140.84111357)(484.48853868,140.85111356)(484.40853975,140.86112078)
\curveto(484.33853883,140.87111354)(484.28853888,140.88611353)(484.25853975,140.90612078)
\curveto(484.21853895,140.93611348)(484.18353899,140.99111342)(484.15353975,141.07112078)
\curveto(484.12353905,141.15111326)(484.11853905,141.23611318)(484.13853975,141.32612078)
\curveto(484.14853902,141.37611304)(484.15353902,141.42611299)(484.15353975,141.47612078)
\lineto(484.18353975,141.65612078)
\curveto(484.21353896,141.75611266)(484.23853893,141.85611256)(484.25853975,141.95612078)
\curveto(484.28853888,142.05611236)(484.32353885,142.14611227)(484.36353975,142.22612078)
\curveto(484.41353876,142.33611208)(484.45853871,142.44111197)(484.49853975,142.54112078)
\curveto(484.53853863,142.65111176)(484.58853858,142.75611166)(484.64853975,142.85612078)
\curveto(484.97853819,143.39611102)(485.44853772,143.79111062)(486.05853975,144.04112078)
\curveto(486.17853699,144.09111032)(486.30353687,144.12611029)(486.43353975,144.14612078)
\curveto(486.5735366,144.16611025)(486.71353646,144.19111022)(486.85353975,144.22112078)
\curveto(486.91353626,144.23111018)(486.9735362,144.23611018)(487.03353975,144.23612078)
\curveto(487.10353607,144.23611018)(487.168536,144.24111017)(487.22853975,144.25112078)
}
}
{
\newrgbcolor{curcolor}{0 0 0}
\pscustom[linestyle=none,fillstyle=solid,fillcolor=curcolor]
{
\newpath
\moveto(493.41814913,135.28112078)
\lineto(493.71814913,135.28112078)
\curveto(493.82814707,135.29111912)(493.93314696,135.29111912)(494.03314913,135.28112078)
\curveto(494.14314675,135.28111913)(494.24314665,135.27111914)(494.33314913,135.25112078)
\curveto(494.42314647,135.24111917)(494.4931464,135.2161192)(494.54314913,135.17612078)
\curveto(494.56314633,135.15611926)(494.57814632,135.12611929)(494.58814913,135.08612078)
\curveto(494.60814629,135.04611937)(494.62814627,135.00111941)(494.64814913,134.95112078)
\lineto(494.64814913,134.87612078)
\curveto(494.65814624,134.82611959)(494.65814624,134.77111964)(494.64814913,134.71112078)
\lineto(494.64814913,134.56112078)
\lineto(494.64814913,134.08112078)
\curveto(494.64814625,133.9111205)(494.60814629,133.79112062)(494.52814913,133.72112078)
\curveto(494.45814644,133.67112074)(494.36814653,133.64612077)(494.25814913,133.64612078)
\lineto(493.92814913,133.64612078)
\lineto(493.47814913,133.64612078)
\curveto(493.32814757,133.64612077)(493.21314768,133.67612074)(493.13314913,133.73612078)
\curveto(493.0931478,133.76612065)(493.06314783,133.8161206)(493.04314913,133.88612078)
\curveto(493.02314787,133.96612045)(493.00814789,134.05112036)(492.99814913,134.14112078)
\lineto(492.99814913,134.42612078)
\curveto(493.00814789,134.52611989)(493.01314788,134.6111198)(493.01314913,134.68112078)
\lineto(493.01314913,134.87612078)
\curveto(493.01314788,134.93611948)(493.02314787,134.99111942)(493.04314913,135.04112078)
\curveto(493.08314781,135.15111926)(493.15314774,135.22111919)(493.25314913,135.25112078)
\curveto(493.28314761,135.25111916)(493.33814756,135.26111915)(493.41814913,135.28112078)
}
}
{
\newrgbcolor{curcolor}{0 0 0}
\pscustom[linestyle=none,fillstyle=solid,fillcolor=curcolor]
{
\newpath
\moveto(499.73830538,144.25112078)
\curveto(501.36829994,144.28111013)(502.41829889,143.72611069)(502.88830538,142.58612078)
\curveto(502.98829832,142.35611206)(503.05329825,142.06611235)(503.08330538,141.71612078)
\curveto(503.12329818,141.37611304)(503.09829821,141.06611335)(503.00830538,140.78612078)
\curveto(502.91829839,140.52611389)(502.79829851,140.30111411)(502.64830538,140.11112078)
\curveto(502.62829868,140.07111434)(502.6032987,140.03611438)(502.57330538,140.00612078)
\curveto(502.54329876,139.98611443)(502.51829879,139.96111445)(502.49830538,139.93112078)
\lineto(502.40830538,139.81112078)
\curveto(502.37829893,139.78111463)(502.34329896,139.75611466)(502.30330538,139.73612078)
\curveto(502.25329905,139.68611473)(502.19829911,139.64111477)(502.13830538,139.60112078)
\curveto(502.08829922,139.56111485)(502.04329926,139.5111149)(502.00330538,139.45112078)
\curveto(501.96329934,139.411115)(501.94829936,139.36111505)(501.95830538,139.30112078)
\curveto(501.96829934,139.25111516)(501.99829931,139.20611521)(502.04830538,139.16612078)
\curveto(502.09829921,139.12611529)(502.15329915,139.08611533)(502.21330538,139.04612078)
\curveto(502.28329902,139.0161154)(502.34829896,138.98611543)(502.40830538,138.95612078)
\curveto(502.46829884,138.92611549)(502.51829879,138.89611552)(502.55830538,138.86612078)
\curveto(502.87829843,138.64611577)(503.13329817,138.33611608)(503.32330538,137.93612078)
\curveto(503.36329794,137.84611657)(503.39329791,137.75111666)(503.41330538,137.65112078)
\curveto(503.44329786,137.56111685)(503.46829784,137.47111694)(503.48830538,137.38112078)
\curveto(503.49829781,137.33111708)(503.5032978,137.28111713)(503.50330538,137.23112078)
\curveto(503.51329779,137.19111722)(503.52329778,137.14611727)(503.53330538,137.09612078)
\curveto(503.54329776,137.04611737)(503.54329776,136.99611742)(503.53330538,136.94612078)
\curveto(503.52329778,136.89611752)(503.52829778,136.84611757)(503.54830538,136.79612078)
\curveto(503.55829775,136.74611767)(503.56329774,136.68611773)(503.56330538,136.61612078)
\curveto(503.56329774,136.54611787)(503.55329775,136.48611793)(503.53330538,136.43612078)
\lineto(503.53330538,136.21112078)
\lineto(503.47330538,135.97112078)
\curveto(503.46329784,135.90111851)(503.44829786,135.83111858)(503.42830538,135.76112078)
\curveto(503.39829791,135.67111874)(503.36829794,135.58611883)(503.33830538,135.50612078)
\curveto(503.31829799,135.42611899)(503.28829802,135.34611907)(503.24830538,135.26612078)
\curveto(503.22829808,135.20611921)(503.19829811,135.14611927)(503.15830538,135.08612078)
\curveto(503.12829818,135.03611938)(503.09329821,134.98611943)(503.05330538,134.93612078)
\curveto(502.85329845,134.62611979)(502.6032987,134.36612005)(502.30330538,134.15612078)
\curveto(502.0032993,133.95612046)(501.65829965,133.79112062)(501.26830538,133.66112078)
\curveto(501.14830016,133.62112079)(501.01830029,133.59612082)(500.87830538,133.58612078)
\curveto(500.74830056,133.56612085)(500.61330069,133.54112087)(500.47330538,133.51112078)
\curveto(500.4033009,133.50112091)(500.33330097,133.49612092)(500.26330538,133.49612078)
\curveto(500.2033011,133.49612092)(500.13830117,133.49112092)(500.06830538,133.48112078)
\curveto(500.02830128,133.47112094)(499.96830134,133.46612095)(499.88830538,133.46612078)
\curveto(499.81830149,133.46612095)(499.76830154,133.47112094)(499.73830538,133.48112078)
\curveto(499.68830162,133.49112092)(499.64330166,133.49612092)(499.60330538,133.49612078)
\lineto(499.48330538,133.49612078)
\curveto(499.38330192,133.5161209)(499.28330202,133.53112088)(499.18330538,133.54112078)
\curveto(499.08330222,133.55112086)(498.98830232,133.56612085)(498.89830538,133.58612078)
\curveto(498.78830252,133.6161208)(498.67830263,133.64112077)(498.56830538,133.66112078)
\curveto(498.46830284,133.69112072)(498.36330294,133.73112068)(498.25330538,133.78112078)
\curveto(497.88330342,133.94112047)(497.56830374,134.14112027)(497.30830538,134.38112078)
\curveto(497.04830426,134.63111978)(496.83830447,134.94111947)(496.67830538,135.31112078)
\curveto(496.63830467,135.40111901)(496.6033047,135.49611892)(496.57330538,135.59612078)
\curveto(496.54330476,135.69611872)(496.51330479,135.80111861)(496.48330538,135.91112078)
\curveto(496.46330484,135.96111845)(496.45330485,136.0111184)(496.45330538,136.06112078)
\curveto(496.45330485,136.12111829)(496.44330486,136.18111823)(496.42330538,136.24112078)
\curveto(496.4033049,136.30111811)(496.39330491,136.38111803)(496.39330538,136.48112078)
\curveto(496.39330491,136.58111783)(496.4083049,136.65611776)(496.43830538,136.70612078)
\curveto(496.44830486,136.73611768)(496.46330484,136.76111765)(496.48330538,136.78112078)
\lineto(496.54330538,136.84112078)
\curveto(496.58330472,136.86111755)(496.64330466,136.87611754)(496.72330538,136.88612078)
\curveto(496.81330449,136.89611752)(496.9033044,136.90111751)(496.99330538,136.90112078)
\curveto(497.08330422,136.90111751)(497.16830414,136.89611752)(497.24830538,136.88612078)
\curveto(497.33830397,136.87611754)(497.4033039,136.86611755)(497.44330538,136.85612078)
\curveto(497.46330384,136.83611758)(497.48330382,136.82111759)(497.50330538,136.81112078)
\curveto(497.52330378,136.8111176)(497.54330376,136.80111761)(497.56330538,136.78112078)
\curveto(497.63330367,136.69111772)(497.67330363,136.57611784)(497.68330538,136.43612078)
\curveto(497.7033036,136.29611812)(497.73330357,136.17111824)(497.77330538,136.06112078)
\lineto(497.92330538,135.70112078)
\curveto(497.97330333,135.59111882)(498.03830327,135.48611893)(498.11830538,135.38612078)
\curveto(498.13830317,135.35611906)(498.15830315,135.33111908)(498.17830538,135.31112078)
\curveto(498.2083031,135.29111912)(498.23330307,135.26611915)(498.25330538,135.23612078)
\curveto(498.29330301,135.17611924)(498.32830298,135.13111928)(498.35830538,135.10112078)
\curveto(498.39830291,135.07111934)(498.43330287,135.04111937)(498.46330538,135.01112078)
\curveto(498.5033028,134.98111943)(498.54830276,134.95111946)(498.59830538,134.92112078)
\curveto(498.68830262,134.86111955)(498.78330252,134.8111196)(498.88330538,134.77112078)
\lineto(499.21330538,134.65112078)
\curveto(499.36330194,134.60111981)(499.56330174,134.57111984)(499.81330538,134.56112078)
\curveto(500.06330124,134.55111986)(500.27330103,134.57111984)(500.44330538,134.62112078)
\curveto(500.52330078,134.64111977)(500.59330071,134.65611976)(500.65330538,134.66612078)
\lineto(500.86330538,134.72612078)
\curveto(501.14330016,134.84611957)(501.38329992,134.99611942)(501.58330538,135.17612078)
\curveto(501.79329951,135.35611906)(501.95829935,135.58611883)(502.07830538,135.86612078)
\curveto(502.1082992,135.93611848)(502.12829918,136.00611841)(502.13830538,136.07612078)
\lineto(502.19830538,136.31612078)
\curveto(502.23829907,136.45611796)(502.24829906,136.6161178)(502.22830538,136.79612078)
\curveto(502.2082991,136.98611743)(502.17829913,137.13611728)(502.13830538,137.24612078)
\curveto(502.0082993,137.62611679)(501.82329948,137.9161165)(501.58330538,138.11612078)
\curveto(501.35329995,138.3161161)(501.04330026,138.47611594)(500.65330538,138.59612078)
\curveto(500.54330076,138.62611579)(500.42330088,138.64611577)(500.29330538,138.65612078)
\curveto(500.17330113,138.66611575)(500.04830126,138.67111574)(499.91830538,138.67112078)
\curveto(499.75830155,138.67111574)(499.61830169,138.67611574)(499.49830538,138.68612078)
\curveto(499.37830193,138.69611572)(499.29330201,138.75611566)(499.24330538,138.86612078)
\curveto(499.22330208,138.89611552)(499.21330209,138.93111548)(499.21330538,138.97112078)
\lineto(499.21330538,139.10612078)
\curveto(499.2033021,139.20611521)(499.2033021,139.30111511)(499.21330538,139.39112078)
\curveto(499.23330207,139.48111493)(499.27330203,139.54611487)(499.33330538,139.58612078)
\curveto(499.37330193,139.6161148)(499.41330189,139.63611478)(499.45330538,139.64612078)
\curveto(499.5033018,139.65611476)(499.55830175,139.66611475)(499.61830538,139.67612078)
\curveto(499.63830167,139.68611473)(499.66330164,139.68611473)(499.69330538,139.67612078)
\curveto(499.72330158,139.67611474)(499.74830156,139.68111473)(499.76830538,139.69112078)
\lineto(499.90330538,139.69112078)
\curveto(500.01330129,139.7111147)(500.11330119,139.72111469)(500.20330538,139.72112078)
\curveto(500.303301,139.73111468)(500.39830091,139.75111466)(500.48830538,139.78112078)
\curveto(500.8083005,139.89111452)(501.06330024,140.03611438)(501.25330538,140.21612078)
\curveto(501.44329986,140.39611402)(501.59329971,140.64611377)(501.70330538,140.96612078)
\curveto(501.73329957,141.06611335)(501.75329955,141.19111322)(501.76330538,141.34112078)
\curveto(501.78329952,141.50111291)(501.77829953,141.64611277)(501.74830538,141.77612078)
\curveto(501.72829958,141.84611257)(501.7082996,141.9111125)(501.68830538,141.97112078)
\curveto(501.67829963,142.04111237)(501.65829965,142.10611231)(501.62830538,142.16612078)
\curveto(501.52829978,142.40611201)(501.38329992,142.59611182)(501.19330538,142.73612078)
\curveto(501.0033003,142.87611154)(500.77830053,142.98611143)(500.51830538,143.06612078)
\curveto(500.45830085,143.08611133)(500.39830091,143.09611132)(500.33830538,143.09612078)
\curveto(500.27830103,143.09611132)(500.21330109,143.10611131)(500.14330538,143.12612078)
\curveto(500.06330124,143.14611127)(499.96830134,143.15611126)(499.85830538,143.15612078)
\curveto(499.74830156,143.15611126)(499.65330165,143.14611127)(499.57330538,143.12612078)
\curveto(499.52330178,143.10611131)(499.47330183,143.09611132)(499.42330538,143.09612078)
\curveto(499.38330192,143.09611132)(499.33830197,143.08611133)(499.28830538,143.06612078)
\curveto(499.1083022,143.0161114)(498.93830237,142.94111147)(498.77830538,142.84112078)
\curveto(498.62830268,142.75111166)(498.49830281,142.63611178)(498.38830538,142.49612078)
\curveto(498.29830301,142.37611204)(498.21830309,142.24611217)(498.14830538,142.10612078)
\curveto(498.07830323,141.96611245)(498.01330329,141.8111126)(497.95330538,141.64112078)
\curveto(497.92330338,141.53111288)(497.9033034,141.411113)(497.89330538,141.28112078)
\curveto(497.88330342,141.16111325)(497.84830346,141.06111335)(497.78830538,140.98112078)
\curveto(497.76830354,140.94111347)(497.7083036,140.90111351)(497.60830538,140.86112078)
\curveto(497.56830374,140.85111356)(497.5083038,140.84111357)(497.42830538,140.83112078)
\lineto(497.17330538,140.83112078)
\curveto(497.08330422,140.84111357)(496.99830431,140.85111356)(496.91830538,140.86112078)
\curveto(496.84830446,140.87111354)(496.79830451,140.88611353)(496.76830538,140.90612078)
\curveto(496.72830458,140.93611348)(496.69330461,140.99111342)(496.66330538,141.07112078)
\curveto(496.63330467,141.15111326)(496.62830468,141.23611318)(496.64830538,141.32612078)
\curveto(496.65830465,141.37611304)(496.66330464,141.42611299)(496.66330538,141.47612078)
\lineto(496.69330538,141.65612078)
\curveto(496.72330458,141.75611266)(496.74830456,141.85611256)(496.76830538,141.95612078)
\curveto(496.79830451,142.05611236)(496.83330447,142.14611227)(496.87330538,142.22612078)
\curveto(496.92330438,142.33611208)(496.96830434,142.44111197)(497.00830538,142.54112078)
\curveto(497.04830426,142.65111176)(497.09830421,142.75611166)(497.15830538,142.85612078)
\curveto(497.48830382,143.39611102)(497.95830335,143.79111062)(498.56830538,144.04112078)
\curveto(498.68830262,144.09111032)(498.81330249,144.12611029)(498.94330538,144.14612078)
\curveto(499.08330222,144.16611025)(499.22330208,144.19111022)(499.36330538,144.22112078)
\curveto(499.42330188,144.23111018)(499.48330182,144.23611018)(499.54330538,144.23612078)
\curveto(499.61330169,144.23611018)(499.67830163,144.24111017)(499.73830538,144.25112078)
}
}
{
\newrgbcolor{curcolor}{0 0 0}
\pscustom[linestyle=none,fillstyle=solid,fillcolor=curcolor]
{
\newpath
\moveto(514.77791475,142.16612078)
\curveto(514.57790445,141.87611254)(514.36790466,141.59111282)(514.14791475,141.31112078)
\curveto(513.93790509,141.03111338)(513.7329053,140.74611367)(513.53291475,140.45612078)
\curveto(512.9329061,139.60611481)(512.3279067,138.76611565)(511.71791475,137.93612078)
\curveto(511.10790792,137.1161173)(510.50290853,136.28111813)(509.90291475,135.43112078)
\lineto(509.39291475,134.71112078)
\lineto(508.88291475,134.02112078)
\curveto(508.80291023,133.9111205)(508.72291031,133.79612062)(508.64291475,133.67612078)
\curveto(508.56291047,133.55612086)(508.46791056,133.46112095)(508.35791475,133.39112078)
\curveto(508.31791071,133.37112104)(508.25291078,133.35612106)(508.16291475,133.34612078)
\curveto(508.08291095,133.32612109)(507.99291104,133.3161211)(507.89291475,133.31612078)
\curveto(507.79291124,133.3161211)(507.69791133,133.32112109)(507.60791475,133.33112078)
\curveto(507.5279115,133.34112107)(507.46791156,133.36112105)(507.42791475,133.39112078)
\curveto(507.39791163,133.411121)(507.37291166,133.44612097)(507.35291475,133.49612078)
\curveto(507.34291169,133.53612088)(507.34791168,133.58112083)(507.36791475,133.63112078)
\curveto(507.40791162,133.7111207)(507.45291158,133.78612063)(507.50291475,133.85612078)
\curveto(507.56291147,133.93612048)(507.61791141,134.0161204)(507.66791475,134.09612078)
\curveto(507.90791112,134.43611998)(508.15291088,134.77111964)(508.40291475,135.10112078)
\curveto(508.65291038,135.43111898)(508.89291014,135.76611865)(509.12291475,136.10612078)
\curveto(509.28290975,136.32611809)(509.44290959,136.54111787)(509.60291475,136.75112078)
\curveto(509.76290927,136.96111745)(509.92290911,137.17611724)(510.08291475,137.39612078)
\curveto(510.44290859,137.9161165)(510.80790822,138.42611599)(511.17791475,138.92612078)
\curveto(511.54790748,139.42611499)(511.91790711,139.93611448)(512.28791475,140.45612078)
\curveto(512.4279066,140.65611376)(512.56790646,140.85111356)(512.70791475,141.04112078)
\curveto(512.85790617,141.23111318)(513.00290603,141.42611299)(513.14291475,141.62612078)
\curveto(513.35290568,141.92611249)(513.56790546,142.22611219)(513.78791475,142.52612078)
\lineto(514.44791475,143.42612078)
\lineto(514.62791475,143.69612078)
\lineto(514.83791475,143.96612078)
\lineto(514.95791475,144.14612078)
\curveto(515.00790402,144.20611021)(515.05790397,144.26111015)(515.10791475,144.31112078)
\curveto(515.17790385,144.36111005)(515.25290378,144.39611002)(515.33291475,144.41612078)
\curveto(515.35290368,144.42610999)(515.37790365,144.42610999)(515.40791475,144.41612078)
\curveto(515.44790358,144.41611)(515.47790355,144.42610999)(515.49791475,144.44612078)
\curveto(515.61790341,144.44610997)(515.75290328,144.44110997)(515.90291475,144.43112078)
\curveto(516.05290298,144.43110998)(516.14290289,144.38611003)(516.17291475,144.29612078)
\curveto(516.19290284,144.26611015)(516.19790283,144.23111018)(516.18791475,144.19112078)
\curveto(516.17790285,144.15111026)(516.16290287,144.12111029)(516.14291475,144.10112078)
\curveto(516.10290293,144.02111039)(516.06290297,143.95111046)(516.02291475,143.89112078)
\curveto(515.98290305,143.83111058)(515.93790309,143.77111064)(515.88791475,143.71112078)
\lineto(515.31791475,142.93112078)
\curveto(515.13790389,142.68111173)(514.95790407,142.42611199)(514.77791475,142.16612078)
\moveto(507.92291475,138.26612078)
\curveto(507.87291116,138.28611613)(507.82291121,138.29111612)(507.77291475,138.28112078)
\curveto(507.72291131,138.27111614)(507.67291136,138.27611614)(507.62291475,138.29612078)
\curveto(507.51291152,138.3161161)(507.40791162,138.33611608)(507.30791475,138.35612078)
\curveto(507.21791181,138.38611603)(507.12291191,138.42611599)(507.02291475,138.47612078)
\curveto(506.69291234,138.6161158)(506.43791259,138.8111156)(506.25791475,139.06112078)
\curveto(506.07791295,139.32111509)(505.9329131,139.63111478)(505.82291475,139.99112078)
\curveto(505.79291324,140.07111434)(505.77291326,140.15111426)(505.76291475,140.23112078)
\curveto(505.75291328,140.32111409)(505.73791329,140.40611401)(505.71791475,140.48612078)
\curveto(505.70791332,140.53611388)(505.70291333,140.60111381)(505.70291475,140.68112078)
\curveto(505.69291334,140.7111137)(505.68791334,140.74111367)(505.68791475,140.77112078)
\curveto(505.68791334,140.8111136)(505.68291335,140.84611357)(505.67291475,140.87612078)
\lineto(505.67291475,141.02612078)
\curveto(505.66291337,141.07611334)(505.65791337,141.13611328)(505.65791475,141.20612078)
\curveto(505.65791337,141.28611313)(505.66291337,141.35111306)(505.67291475,141.40112078)
\lineto(505.67291475,141.56612078)
\curveto(505.69291334,141.6161128)(505.69791333,141.66111275)(505.68791475,141.70112078)
\curveto(505.68791334,141.75111266)(505.69291334,141.79611262)(505.70291475,141.83612078)
\curveto(505.71291332,141.87611254)(505.71791331,141.9111125)(505.71791475,141.94112078)
\curveto(505.71791331,141.98111243)(505.72291331,142.02111239)(505.73291475,142.06112078)
\curveto(505.76291327,142.17111224)(505.78291325,142.28111213)(505.79291475,142.39112078)
\curveto(505.81291322,142.5111119)(505.84791318,142.62611179)(505.89791475,142.73612078)
\curveto(506.03791299,143.07611134)(506.19791283,143.35111106)(506.37791475,143.56112078)
\curveto(506.56791246,143.78111063)(506.83791219,143.96111045)(507.18791475,144.10112078)
\curveto(507.26791176,144.13111028)(507.35291168,144.15111026)(507.44291475,144.16112078)
\curveto(507.5329115,144.18111023)(507.6279114,144.20111021)(507.72791475,144.22112078)
\curveto(507.75791127,144.23111018)(507.81291122,144.23111018)(507.89291475,144.22112078)
\curveto(507.97291106,144.22111019)(508.02291101,144.23111018)(508.04291475,144.25112078)
\curveto(508.60291043,144.26111015)(509.05290998,144.15111026)(509.39291475,143.92112078)
\curveto(509.74290929,143.69111072)(510.00290903,143.38611103)(510.17291475,143.00612078)
\curveto(510.21290882,142.9161115)(510.24790878,142.82111159)(510.27791475,142.72112078)
\curveto(510.30790872,142.62111179)(510.3329087,142.52111189)(510.35291475,142.42112078)
\curveto(510.37290866,142.39111202)(510.37790865,142.36111205)(510.36791475,142.33112078)
\curveto(510.36790866,142.30111211)(510.37290866,142.27111214)(510.38291475,142.24112078)
\curveto(510.41290862,142.13111228)(510.4329086,142.00611241)(510.44291475,141.86612078)
\curveto(510.45290858,141.73611268)(510.46290857,141.60111281)(510.47291475,141.46112078)
\lineto(510.47291475,141.29612078)
\curveto(510.48290855,141.23611318)(510.48290855,141.18111323)(510.47291475,141.13112078)
\curveto(510.46290857,141.08111333)(510.45790857,141.03111338)(510.45791475,140.98112078)
\lineto(510.45791475,140.84612078)
\curveto(510.44790858,140.80611361)(510.44290859,140.76611365)(510.44291475,140.72612078)
\curveto(510.45290858,140.68611373)(510.44790858,140.64111377)(510.42791475,140.59112078)
\curveto(510.40790862,140.48111393)(510.38790864,140.37611404)(510.36791475,140.27612078)
\curveto(510.35790867,140.17611424)(510.33790869,140.07611434)(510.30791475,139.97612078)
\curveto(510.17790885,139.6161148)(510.01290902,139.30111511)(509.81291475,139.03112078)
\curveto(509.61290942,138.76111565)(509.33790969,138.55611586)(508.98791475,138.41612078)
\curveto(508.90791012,138.38611603)(508.82291021,138.36111605)(508.73291475,138.34112078)
\lineto(508.46291475,138.28112078)
\curveto(508.41291062,138.27111614)(508.36791066,138.26611615)(508.32791475,138.26612078)
\curveto(508.28791074,138.27611614)(508.24791078,138.27611614)(508.20791475,138.26612078)
\curveto(508.10791092,138.24611617)(508.01291102,138.24611617)(507.92291475,138.26612078)
\moveto(507.08291475,139.66112078)
\curveto(507.12291191,139.59111482)(507.16291187,139.52611489)(507.20291475,139.46612078)
\curveto(507.24291179,139.416115)(507.29291174,139.36611505)(507.35291475,139.31612078)
\lineto(507.50291475,139.19612078)
\curveto(507.56291147,139.16611525)(507.6279114,139.14111527)(507.69791475,139.12112078)
\curveto(507.73791129,139.10111531)(507.77291126,139.09111532)(507.80291475,139.09112078)
\curveto(507.84291119,139.10111531)(507.88291115,139.09611532)(507.92291475,139.07612078)
\curveto(507.95291108,139.07611534)(507.99291104,139.07111534)(508.04291475,139.06112078)
\curveto(508.09291094,139.06111535)(508.1329109,139.06611535)(508.16291475,139.07612078)
\lineto(508.38791475,139.12112078)
\curveto(508.63791039,139.20111521)(508.82291021,139.32611509)(508.94291475,139.49612078)
\curveto(509.02291001,139.59611482)(509.09290994,139.72611469)(509.15291475,139.88612078)
\curveto(509.2329098,140.06611435)(509.29290974,140.29111412)(509.33291475,140.56112078)
\curveto(509.37290966,140.84111357)(509.38790964,141.12111329)(509.37791475,141.40112078)
\curveto(509.36790966,141.69111272)(509.33790969,141.96611245)(509.28791475,142.22612078)
\curveto(509.23790979,142.48611193)(509.16290987,142.69611172)(509.06291475,142.85612078)
\curveto(508.94291009,143.05611136)(508.79291024,143.20611121)(508.61291475,143.30612078)
\curveto(508.5329105,143.35611106)(508.44291059,143.38611103)(508.34291475,143.39612078)
\curveto(508.24291079,143.416111)(508.13791089,143.42611099)(508.02791475,143.42612078)
\curveto(508.00791102,143.416111)(507.98291105,143.411111)(507.95291475,143.41112078)
\curveto(507.9329111,143.42111099)(507.91291112,143.42111099)(507.89291475,143.41112078)
\curveto(507.84291119,143.40111101)(507.79791123,143.39111102)(507.75791475,143.38112078)
\curveto(507.71791131,143.38111103)(507.67791135,143.37111104)(507.63791475,143.35112078)
\curveto(507.45791157,143.27111114)(507.30791172,143.15111126)(507.18791475,142.99112078)
\curveto(507.07791195,142.83111158)(506.98791204,142.65111176)(506.91791475,142.45112078)
\curveto(506.85791217,142.26111215)(506.81291222,142.03611238)(506.78291475,141.77612078)
\curveto(506.76291227,141.5161129)(506.75791227,141.25111316)(506.76791475,140.98112078)
\curveto(506.77791225,140.72111369)(506.80791222,140.47111394)(506.85791475,140.23112078)
\curveto(506.91791211,140.00111441)(506.99291204,139.8111146)(507.08291475,139.66112078)
\moveto(517.88291475,136.67612078)
\curveto(517.89290114,136.62611779)(517.89790113,136.53611788)(517.89791475,136.40612078)
\curveto(517.89790113,136.27611814)(517.88790114,136.18611823)(517.86791475,136.13612078)
\curveto(517.84790118,136.08611833)(517.84290119,136.03111838)(517.85291475,135.97112078)
\curveto(517.86290117,135.92111849)(517.86290117,135.87111854)(517.85291475,135.82112078)
\curveto(517.81290122,135.68111873)(517.78290125,135.54611887)(517.76291475,135.41612078)
\curveto(517.75290128,135.28611913)(517.72290131,135.16611925)(517.67291475,135.05612078)
\curveto(517.5329015,134.70611971)(517.36790166,134.41112)(517.17791475,134.17112078)
\curveto(516.98790204,133.94112047)(516.71790231,133.75612066)(516.36791475,133.61612078)
\curveto(516.28790274,133.58612083)(516.20290283,133.56612085)(516.11291475,133.55612078)
\curveto(516.02290301,133.53612088)(515.93790309,133.5161209)(515.85791475,133.49612078)
\curveto(515.80790322,133.48612093)(515.75790327,133.48112093)(515.70791475,133.48112078)
\curveto(515.65790337,133.48112093)(515.60790342,133.47612094)(515.55791475,133.46612078)
\curveto(515.5279035,133.45612096)(515.47790355,133.45612096)(515.40791475,133.46612078)
\curveto(515.33790369,133.46612095)(515.28790374,133.47112094)(515.25791475,133.48112078)
\curveto(515.19790383,133.50112091)(515.13790389,133.5111209)(515.07791475,133.51112078)
\curveto(515.027904,133.50112091)(514.97790405,133.50612091)(514.92791475,133.52612078)
\curveto(514.83790419,133.54612087)(514.74790428,133.57112084)(514.65791475,133.60112078)
\curveto(514.57790445,133.62112079)(514.49790453,133.65112076)(514.41791475,133.69112078)
\curveto(514.09790493,133.83112058)(513.84790518,134.02612039)(513.66791475,134.27612078)
\curveto(513.48790554,134.53611988)(513.33790569,134.84111957)(513.21791475,135.19112078)
\curveto(513.19790583,135.27111914)(513.18290585,135.35611906)(513.17291475,135.44612078)
\curveto(513.16290587,135.53611888)(513.14790588,135.62111879)(513.12791475,135.70112078)
\curveto(513.11790591,135.73111868)(513.11290592,135.76111865)(513.11291475,135.79112078)
\lineto(513.11291475,135.89612078)
\curveto(513.09290594,135.97611844)(513.08290595,136.05611836)(513.08291475,136.13612078)
\lineto(513.08291475,136.27112078)
\curveto(513.06290597,136.37111804)(513.06290597,136.47111794)(513.08291475,136.57112078)
\lineto(513.08291475,136.75112078)
\curveto(513.09290594,136.80111761)(513.09790593,136.84611757)(513.09791475,136.88612078)
\curveto(513.09790593,136.93611748)(513.10290593,136.98111743)(513.11291475,137.02112078)
\curveto(513.12290591,137.06111735)(513.1279059,137.09611732)(513.12791475,137.12612078)
\curveto(513.1279059,137.16611725)(513.1329059,137.20611721)(513.14291475,137.24612078)
\lineto(513.20291475,137.57612078)
\curveto(513.22290581,137.69611672)(513.25290578,137.80611661)(513.29291475,137.90612078)
\curveto(513.4329056,138.23611618)(513.59290544,138.5111159)(513.77291475,138.73112078)
\curveto(513.96290507,138.96111545)(514.22290481,139.14611527)(514.55291475,139.28612078)
\curveto(514.6329044,139.32611509)(514.71790431,139.35111506)(514.80791475,139.36112078)
\lineto(515.10791475,139.42112078)
\lineto(515.24291475,139.42112078)
\curveto(515.29290374,139.43111498)(515.34290369,139.43611498)(515.39291475,139.43612078)
\curveto(515.96290307,139.45611496)(516.42290261,139.35111506)(516.77291475,139.12112078)
\curveto(517.1329019,138.90111551)(517.39790163,138.60111581)(517.56791475,138.22112078)
\curveto(517.61790141,138.12111629)(517.65790137,138.02111639)(517.68791475,137.92112078)
\curveto(517.71790131,137.82111659)(517.74790128,137.7161167)(517.77791475,137.60612078)
\curveto(517.78790124,137.56611685)(517.79290124,137.53111688)(517.79291475,137.50112078)
\curveto(517.79290124,137.48111693)(517.79790123,137.45111696)(517.80791475,137.41112078)
\curveto(517.8279012,137.34111707)(517.83790119,137.26611715)(517.83791475,137.18612078)
\curveto(517.83790119,137.10611731)(517.84790118,137.02611739)(517.86791475,136.94612078)
\curveto(517.86790116,136.89611752)(517.86790116,136.85111756)(517.86791475,136.81112078)
\curveto(517.86790116,136.77111764)(517.87290116,136.72611769)(517.88291475,136.67612078)
\moveto(516.77291475,136.24112078)
\curveto(516.78290225,136.29111812)(516.78790224,136.36611805)(516.78791475,136.46612078)
\curveto(516.79790223,136.56611785)(516.79290224,136.64111777)(516.77291475,136.69112078)
\curveto(516.75290228,136.75111766)(516.74790228,136.80611761)(516.75791475,136.85612078)
\curveto(516.77790225,136.9161175)(516.77790225,136.97611744)(516.75791475,137.03612078)
\curveto(516.74790228,137.06611735)(516.74290229,137.10111731)(516.74291475,137.14112078)
\curveto(516.74290229,137.18111723)(516.73790229,137.22111719)(516.72791475,137.26112078)
\curveto(516.70790232,137.34111707)(516.68790234,137.416117)(516.66791475,137.48612078)
\curveto(516.65790237,137.56611685)(516.64290239,137.64611677)(516.62291475,137.72612078)
\curveto(516.59290244,137.78611663)(516.56790246,137.84611657)(516.54791475,137.90612078)
\curveto(516.5279025,137.96611645)(516.49790253,138.02611639)(516.45791475,138.08612078)
\curveto(516.35790267,138.25611616)(516.2279028,138.39111602)(516.06791475,138.49112078)
\curveto(515.98790304,138.54111587)(515.89290314,138.57611584)(515.78291475,138.59612078)
\curveto(515.67290336,138.6161158)(515.54790348,138.62611579)(515.40791475,138.62612078)
\curveto(515.38790364,138.6161158)(515.36290367,138.6111158)(515.33291475,138.61112078)
\curveto(515.30290373,138.62111579)(515.27290376,138.62111579)(515.24291475,138.61112078)
\lineto(515.09291475,138.55112078)
\curveto(515.04290399,138.54111587)(514.99790403,138.52611589)(514.95791475,138.50612078)
\curveto(514.76790426,138.39611602)(514.62290441,138.25111616)(514.52291475,138.07112078)
\curveto(514.4329046,137.89111652)(514.35290468,137.68611673)(514.28291475,137.45612078)
\curveto(514.24290479,137.32611709)(514.22290481,137.19111722)(514.22291475,137.05112078)
\curveto(514.22290481,136.92111749)(514.21290482,136.77611764)(514.19291475,136.61612078)
\curveto(514.18290485,136.56611785)(514.17290486,136.50611791)(514.16291475,136.43612078)
\curveto(514.16290487,136.36611805)(514.17290486,136.30611811)(514.19291475,136.25612078)
\lineto(514.19291475,136.09112078)
\lineto(514.19291475,135.91112078)
\curveto(514.20290483,135.86111855)(514.21290482,135.80611861)(514.22291475,135.74612078)
\curveto(514.2329048,135.69611872)(514.23790479,135.64111877)(514.23791475,135.58112078)
\curveto(514.24790478,135.52111889)(514.26290477,135.46611895)(514.28291475,135.41612078)
\curveto(514.3329047,135.22611919)(514.39290464,135.05111936)(514.46291475,134.89112078)
\curveto(514.5329045,134.73111968)(514.63790439,134.60111981)(514.77791475,134.50112078)
\curveto(514.90790412,134.40112001)(515.04790398,134.33112008)(515.19791475,134.29112078)
\curveto(515.2279038,134.28112013)(515.25290378,134.27612014)(515.27291475,134.27612078)
\curveto(515.30290373,134.28612013)(515.3329037,134.28612013)(515.36291475,134.27612078)
\curveto(515.38290365,134.27612014)(515.41290362,134.27112014)(515.45291475,134.26112078)
\curveto(515.49290354,134.26112015)(515.5279035,134.26612015)(515.55791475,134.27612078)
\curveto(515.59790343,134.28612013)(515.63790339,134.29112012)(515.67791475,134.29112078)
\curveto(515.71790331,134.29112012)(515.75790327,134.30112011)(515.79791475,134.32112078)
\curveto(516.03790299,134.40112001)(516.2329028,134.53611988)(516.38291475,134.72612078)
\curveto(516.50290253,134.90611951)(516.59290244,135.1111193)(516.65291475,135.34112078)
\curveto(516.67290236,135.411119)(516.68790234,135.48111893)(516.69791475,135.55112078)
\curveto(516.70790232,135.63111878)(516.72290231,135.7111187)(516.74291475,135.79112078)
\curveto(516.74290229,135.85111856)(516.74790228,135.89611852)(516.75791475,135.92612078)
\curveto(516.75790227,135.94611847)(516.75790227,135.97111844)(516.75791475,136.00112078)
\curveto(516.75790227,136.04111837)(516.76290227,136.07111834)(516.77291475,136.09112078)
\lineto(516.77291475,136.24112078)
}
}
{
\newrgbcolor{curcolor}{0 0 0}
\pscustom[linestyle=none,fillstyle=solid,fillcolor=curcolor]
{
\newpath
\moveto(698.49734956,154.15762225)
\lineto(702.09734956,154.15762225)
\lineto(702.74234956,154.15762225)
\curveto(702.82234303,154.15761182)(702.89734296,154.15261183)(702.96734956,154.14262225)
\curveto(703.03734282,154.14261184)(703.09734276,154.13261185)(703.14734956,154.11262225)
\curveto(703.21734264,154.0826119)(703.27234258,154.02261196)(703.31234956,153.93262225)
\curveto(703.33234252,153.90261208)(703.34234251,153.86261212)(703.34234956,153.81262225)
\lineto(703.34234956,153.67762225)
\curveto(703.3523425,153.56761241)(703.34734251,153.46261252)(703.32734956,153.36262225)
\curveto(703.31734254,153.26261272)(703.28234257,153.19261279)(703.22234956,153.15262225)
\curveto(703.13234272,153.0826129)(702.99734286,153.04761293)(702.81734956,153.04762225)
\curveto(702.63734322,153.05761292)(702.47234338,153.06261292)(702.32234956,153.06262225)
\lineto(700.32734956,153.06262225)
\lineto(699.83234956,153.06262225)
\lineto(699.69734956,153.06262225)
\curveto(699.6573462,153.06261292)(699.61734624,153.05761292)(699.57734956,153.04762225)
\lineto(699.36734956,153.04762225)
\curveto(699.2573466,153.01761296)(699.17734668,152.977613)(699.12734956,152.92762225)
\curveto(699.07734678,152.88761309)(699.04234681,152.83261315)(699.02234956,152.76262225)
\curveto(699.00234685,152.70261328)(698.98734687,152.63261335)(698.97734956,152.55262225)
\curveto(698.96734689,152.47261351)(698.94734691,152.3826136)(698.91734956,152.28262225)
\curveto(698.86734699,152.0826139)(698.82734703,151.8776141)(698.79734956,151.66762225)
\curveto(698.76734709,151.45761452)(698.72734713,151.25261473)(698.67734956,151.05262225)
\curveto(698.6573472,150.982615)(698.64734721,150.91261507)(698.64734956,150.84262225)
\curveto(698.64734721,150.7826152)(698.63734722,150.71761526)(698.61734956,150.64762225)
\curveto(698.60734725,150.61761536)(698.59734726,150.5776154)(698.58734956,150.52762225)
\curveto(698.58734727,150.48761549)(698.59234726,150.44761553)(698.60234956,150.40762225)
\curveto(698.62234723,150.35761562)(698.64734721,150.31261567)(698.67734956,150.27262225)
\curveto(698.71734714,150.24261574)(698.77734708,150.23761574)(698.85734956,150.25762225)
\curveto(698.91734694,150.2776157)(698.97734688,150.30261568)(699.03734956,150.33262225)
\curveto(699.09734676,150.37261561)(699.1573467,150.40761557)(699.21734956,150.43762225)
\curveto(699.27734658,150.45761552)(699.32734653,150.47261551)(699.36734956,150.48262225)
\curveto(699.5573463,150.56261542)(699.76234609,150.61761536)(699.98234956,150.64762225)
\curveto(700.21234564,150.6776153)(700.44234541,150.68761529)(700.67234956,150.67762225)
\curveto(700.91234494,150.6776153)(701.14234471,150.65261533)(701.36234956,150.60262225)
\curveto(701.58234427,150.56261542)(701.78234407,150.50261548)(701.96234956,150.42262225)
\curveto(702.01234384,150.40261558)(702.0573438,150.3826156)(702.09734956,150.36262225)
\curveto(702.14734371,150.34261564)(702.19734366,150.31761566)(702.24734956,150.28762225)
\curveto(702.59734326,150.0776159)(702.87734298,149.84761613)(703.08734956,149.59762225)
\curveto(703.30734255,149.34761663)(703.50234235,149.02261696)(703.67234956,148.62262225)
\curveto(703.72234213,148.51261747)(703.7573421,148.40261758)(703.77734956,148.29262225)
\curveto(703.79734206,148.1826178)(703.82234203,148.06761791)(703.85234956,147.94762225)
\curveto(703.86234199,147.91761806)(703.86734199,147.87261811)(703.86734956,147.81262225)
\curveto(703.88734197,147.75261823)(703.89734196,147.6826183)(703.89734956,147.60262225)
\curveto(703.89734196,147.53261845)(703.90734195,147.46761851)(703.92734956,147.40762225)
\lineto(703.92734956,147.24262225)
\curveto(703.93734192,147.19261879)(703.94234191,147.12261886)(703.94234956,147.03262225)
\curveto(703.94234191,146.94261904)(703.93234192,146.87261911)(703.91234956,146.82262225)
\curveto(703.89234196,146.76261922)(703.88734197,146.70261928)(703.89734956,146.64262225)
\curveto(703.90734195,146.59261939)(703.90234195,146.54261944)(703.88234956,146.49262225)
\curveto(703.84234201,146.33261965)(703.80734205,146.1826198)(703.77734956,146.04262225)
\curveto(703.74734211,145.90262008)(703.70234215,145.76762021)(703.64234956,145.63762225)
\curveto(703.48234237,145.26762071)(703.26234259,144.93262105)(702.98234956,144.63262225)
\curveto(702.70234315,144.33262165)(702.38234347,144.10262188)(702.02234956,143.94262225)
\curveto(701.852344,143.86262212)(701.6523442,143.78762219)(701.42234956,143.71762225)
\curveto(701.31234454,143.6776223)(701.19734466,143.65262233)(701.07734956,143.64262225)
\curveto(700.9573449,143.63262235)(700.83734502,143.61262237)(700.71734956,143.58262225)
\curveto(700.66734519,143.56262242)(700.61234524,143.56262242)(700.55234956,143.58262225)
\curveto(700.49234536,143.59262239)(700.43234542,143.58762239)(700.37234956,143.56762225)
\curveto(700.27234558,143.54762243)(700.17234568,143.54762243)(700.07234956,143.56762225)
\lineto(699.93734956,143.56762225)
\curveto(699.88734597,143.58762239)(699.82734603,143.59762238)(699.75734956,143.59762225)
\curveto(699.69734616,143.58762239)(699.64234621,143.59262239)(699.59234956,143.61262225)
\curveto(699.5523463,143.62262236)(699.51734634,143.62762235)(699.48734956,143.62762225)
\curveto(699.4573464,143.62762235)(699.42234643,143.63262235)(699.38234956,143.64262225)
\lineto(699.11234956,143.70262225)
\curveto(699.02234683,143.72262226)(698.93734692,143.75262223)(698.85734956,143.79262225)
\curveto(698.51734734,143.93262205)(698.22734763,144.08762189)(697.98734956,144.25762225)
\curveto(697.74734811,144.43762154)(697.52734833,144.66762131)(697.32734956,144.94762225)
\curveto(697.17734868,145.1776208)(697.06234879,145.41762056)(696.98234956,145.66762225)
\curveto(696.96234889,145.71762026)(696.9523489,145.76262022)(696.95234956,145.80262225)
\curveto(696.9523489,145.85262013)(696.94234891,145.90262008)(696.92234956,145.95262225)
\curveto(696.90234895,146.01261997)(696.88734897,146.09261989)(696.87734956,146.19262225)
\curveto(696.87734898,146.29261969)(696.89734896,146.36761961)(696.93734956,146.41762225)
\curveto(696.98734887,146.49761948)(697.06734879,146.54261944)(697.17734956,146.55262225)
\curveto(697.28734857,146.56261942)(697.40234845,146.56761941)(697.52234956,146.56762225)
\lineto(697.68734956,146.56762225)
\curveto(697.74734811,146.56761941)(697.80234805,146.55761942)(697.85234956,146.53762225)
\curveto(697.94234791,146.51761946)(698.01234784,146.4776195)(698.06234956,146.41762225)
\curveto(698.13234772,146.32761965)(698.17734768,146.21761976)(698.19734956,146.08762225)
\curveto(698.22734763,145.96762001)(698.27234758,145.86262012)(698.33234956,145.77262225)
\curveto(698.52234733,145.43262055)(698.78234707,145.16262082)(699.11234956,144.96262225)
\curveto(699.21234664,144.90262108)(699.31734654,144.85262113)(699.42734956,144.81262225)
\curveto(699.54734631,144.7826212)(699.66734619,144.74762123)(699.78734956,144.70762225)
\curveto(699.9573459,144.65762132)(700.16234569,144.63762134)(700.40234956,144.64762225)
\curveto(700.6523452,144.66762131)(700.852345,144.70262128)(701.00234956,144.75262225)
\curveto(701.37234448,144.87262111)(701.66234419,145.03262095)(701.87234956,145.23262225)
\curveto(702.09234376,145.44262054)(702.27234358,145.72262026)(702.41234956,146.07262225)
\curveto(702.46234339,146.17261981)(702.49234336,146.2776197)(702.50234956,146.38762225)
\curveto(702.52234333,146.49761948)(702.54734331,146.61261937)(702.57734956,146.73262225)
\lineto(702.57734956,146.83762225)
\curveto(702.58734327,146.8776191)(702.59234326,146.91761906)(702.59234956,146.95762225)
\curveto(702.60234325,146.98761899)(702.60234325,147.02261896)(702.59234956,147.06262225)
\lineto(702.59234956,147.18262225)
\curveto(702.59234326,147.44261854)(702.56234329,147.68761829)(702.50234956,147.91762225)
\curveto(702.39234346,148.26761771)(702.23734362,148.56261742)(702.03734956,148.80262225)
\curveto(701.83734402,149.05261693)(701.57734428,149.24761673)(701.25734956,149.38762225)
\lineto(701.07734956,149.44762225)
\curveto(701.02734483,149.46761651)(700.96734489,149.48761649)(700.89734956,149.50762225)
\curveto(700.84734501,149.52761645)(700.78734507,149.53761644)(700.71734956,149.53762225)
\curveto(700.6573452,149.54761643)(700.59234526,149.56261642)(700.52234956,149.58262225)
\lineto(700.37234956,149.58262225)
\curveto(700.33234552,149.60261638)(700.27734558,149.61261637)(700.20734956,149.61262225)
\curveto(700.14734571,149.61261637)(700.09234576,149.60261638)(700.04234956,149.58262225)
\lineto(699.93734956,149.58262225)
\curveto(699.90734595,149.5826164)(699.87234598,149.5776164)(699.83234956,149.56762225)
\lineto(699.59234956,149.50762225)
\curveto(699.51234634,149.49761648)(699.43234642,149.4776165)(699.35234956,149.44762225)
\curveto(699.11234674,149.34761663)(698.88234697,149.21261677)(698.66234956,149.04262225)
\curveto(698.57234728,148.97261701)(698.48734737,148.89761708)(698.40734956,148.81762225)
\curveto(698.32734753,148.74761723)(698.22734763,148.69261729)(698.10734956,148.65262225)
\curveto(698.01734784,148.62261736)(697.87734798,148.61261737)(697.68734956,148.62262225)
\curveto(697.50734835,148.63261735)(697.38734847,148.65761732)(697.32734956,148.69762225)
\curveto(697.27734858,148.73761724)(697.23734862,148.79761718)(697.20734956,148.87762225)
\curveto(697.18734867,148.95761702)(697.18734867,149.04261694)(697.20734956,149.13262225)
\curveto(697.23734862,149.25261673)(697.2573486,149.37261661)(697.26734956,149.49262225)
\curveto(697.28734857,149.62261636)(697.31234854,149.74761623)(697.34234956,149.86762225)
\curveto(697.36234849,149.90761607)(697.36734849,149.94261604)(697.35734956,149.97262225)
\curveto(697.3573485,150.01261597)(697.36734849,150.05761592)(697.38734956,150.10762225)
\curveto(697.40734845,150.19761578)(697.42234843,150.28761569)(697.43234956,150.37762225)
\curveto(697.44234841,150.4776155)(697.46234839,150.57261541)(697.49234956,150.66262225)
\curveto(697.50234835,150.72261526)(697.50734835,150.7826152)(697.50734956,150.84262225)
\curveto(697.51734834,150.90261508)(697.53234832,150.96261502)(697.55234956,151.02262225)
\curveto(697.60234825,151.22261476)(697.63734822,151.42761455)(697.65734956,151.63762225)
\curveto(697.68734817,151.85761412)(697.72734813,152.06761391)(697.77734956,152.26762225)
\curveto(697.80734805,152.36761361)(697.82734803,152.46761351)(697.83734956,152.56762225)
\curveto(697.84734801,152.66761331)(697.86234799,152.76761321)(697.88234956,152.86762225)
\curveto(697.89234796,152.89761308)(697.89734796,152.93761304)(697.89734956,152.98762225)
\curveto(697.92734793,153.09761288)(697.94734791,153.20261278)(697.95734956,153.30262225)
\curveto(697.97734788,153.41261257)(698.00234785,153.52261246)(698.03234956,153.63262225)
\curveto(698.0523478,153.71261227)(698.06734779,153.7826122)(698.07734956,153.84262225)
\curveto(698.08734777,153.91261207)(698.11234774,153.97261201)(698.15234956,154.02262225)
\curveto(698.17234768,154.05261193)(698.20234765,154.07261191)(698.24234956,154.08262225)
\curveto(698.28234757,154.10261188)(698.32734753,154.12261186)(698.37734956,154.14262225)
\curveto(698.43734742,154.14261184)(698.47734738,154.14761183)(698.49734956,154.15762225)
}
}
{
\newrgbcolor{curcolor}{0 0 0}
\pscustom[linestyle=none,fillstyle=solid,fillcolor=curcolor]
{
\newpath
\moveto(712.21695894,147.24262225)
\curveto(712.28695129,147.19261879)(712.32695125,147.12261886)(712.33695894,147.03262225)
\curveto(712.35695122,146.94261904)(712.36695121,146.83761914)(712.36695894,146.71762225)
\curveto(712.36695121,146.66761931)(712.36195122,146.61761936)(712.35195894,146.56762225)
\curveto(712.35195123,146.51761946)(712.34195124,146.47261951)(712.32195894,146.43262225)
\curveto(712.29195129,146.34261964)(712.23195135,146.2826197)(712.14195894,146.25262225)
\curveto(712.06195152,146.23261975)(711.96695161,146.22261976)(711.85695894,146.22262225)
\lineto(711.54195894,146.22262225)
\curveto(711.43195215,146.23261975)(711.32695225,146.22261976)(711.22695894,146.19262225)
\curveto(711.08695249,146.16261982)(710.99695258,146.0826199)(710.95695894,145.95262225)
\curveto(710.93695264,145.8826201)(710.92695265,145.79762018)(710.92695894,145.69762225)
\lineto(710.92695894,145.42762225)
\lineto(710.92695894,144.48262225)
\lineto(710.92695894,144.15262225)
\curveto(710.92695265,144.04262194)(710.90695267,143.95762202)(710.86695894,143.89762225)
\curveto(710.82695275,143.83762214)(710.7769528,143.79762218)(710.71695894,143.77762225)
\curveto(710.66695291,143.76762221)(710.60195298,143.75262223)(710.52195894,143.73262225)
\lineto(710.32695894,143.73262225)
\curveto(710.20695337,143.73262225)(710.10195348,143.73762224)(710.01195894,143.74762225)
\curveto(709.92195366,143.76762221)(709.85195373,143.81762216)(709.80195894,143.89762225)
\curveto(709.77195381,143.94762203)(709.75695382,144.01762196)(709.75695894,144.10762225)
\lineto(709.75695894,144.40762225)
\lineto(709.75695894,145.44262225)
\curveto(709.75695382,145.60262038)(709.74695383,145.74762023)(709.72695894,145.87762225)
\curveto(709.71695386,146.01761996)(709.66195392,146.11261987)(709.56195894,146.16262225)
\curveto(709.51195407,146.1826198)(709.44195414,146.19761978)(709.35195894,146.20762225)
\curveto(709.27195431,146.21761976)(709.1819544,146.22261976)(709.08195894,146.22262225)
\lineto(708.79695894,146.22262225)
\lineto(708.55695894,146.22262225)
\lineto(706.29195894,146.22262225)
\curveto(706.20195738,146.22261976)(706.09695748,146.21761976)(705.97695894,146.20762225)
\lineto(705.64695894,146.20762225)
\curveto(705.53695804,146.20761977)(705.43695814,146.21761976)(705.34695894,146.23762225)
\curveto(705.25695832,146.25761972)(705.19695838,146.29261969)(705.16695894,146.34262225)
\curveto(705.11695846,146.41261957)(705.09195849,146.50761947)(705.09195894,146.62762225)
\lineto(705.09195894,146.97262225)
\lineto(705.09195894,147.24262225)
\curveto(705.13195845,147.41261857)(705.18695839,147.55261843)(705.25695894,147.66262225)
\curveto(705.32695825,147.77261821)(705.40695817,147.88761809)(705.49695894,148.00762225)
\lineto(705.85695894,148.54762225)
\curveto(706.29695728,149.1776168)(706.73195685,149.79761618)(707.16195894,150.40762225)
\lineto(708.48195894,152.26762225)
\curveto(708.64195494,152.49761348)(708.79695478,152.71761326)(708.94695894,152.92762225)
\curveto(709.09695448,153.14761283)(709.25195433,153.37261261)(709.41195894,153.60262225)
\curveto(709.46195412,153.67261231)(709.51195407,153.73761224)(709.56195894,153.79762225)
\curveto(709.61195397,153.86761211)(709.66195392,153.94261204)(709.71195894,154.02262225)
\lineto(709.77195894,154.11262225)
\curveto(709.80195378,154.15261183)(709.83195375,154.1826118)(709.86195894,154.20262225)
\curveto(709.90195368,154.23261175)(709.94195364,154.25261173)(709.98195894,154.26262225)
\curveto(710.02195356,154.2826117)(710.06695351,154.30261168)(710.11695894,154.32262225)
\curveto(710.13695344,154.32261166)(710.15695342,154.31761166)(710.17695894,154.30762225)
\curveto(710.20695337,154.30761167)(710.23195335,154.31761166)(710.25195894,154.33762225)
\curveto(710.3819532,154.33761164)(710.50195308,154.33261165)(710.61195894,154.32262225)
\curveto(710.72195286,154.31261167)(710.80195278,154.26761171)(710.85195894,154.18762225)
\curveto(710.89195269,154.13761184)(710.91195267,154.06761191)(710.91195894,153.97762225)
\curveto(710.92195266,153.88761209)(710.92695265,153.79261219)(710.92695894,153.69262225)
\lineto(710.92695894,148.23262225)
\curveto(710.92695265,148.16261782)(710.92195266,148.08761789)(710.91195894,148.00762225)
\curveto(710.91195267,147.93761804)(710.91695266,147.86761811)(710.92695894,147.79762225)
\lineto(710.92695894,147.69262225)
\curveto(710.94695263,147.64261834)(710.96195262,147.58761839)(710.97195894,147.52762225)
\curveto(710.9819526,147.4776185)(711.00695257,147.43761854)(711.04695894,147.40762225)
\curveto(711.11695246,147.35761862)(711.20195238,147.32761865)(711.30195894,147.31762225)
\lineto(711.63195894,147.31762225)
\curveto(711.74195184,147.31761866)(711.84695173,147.31261867)(711.94695894,147.30262225)
\curveto(712.05695152,147.30261868)(712.14695143,147.2826187)(712.21695894,147.24262225)
\moveto(709.65195894,147.43762225)
\curveto(709.73195385,147.54761843)(709.76695381,147.71761826)(709.75695894,147.94762225)
\lineto(709.75695894,148.56262225)
\lineto(709.75695894,151.03762225)
\lineto(709.75695894,151.35262225)
\curveto(709.76695381,151.47261451)(709.76195382,151.57261441)(709.74195894,151.65262225)
\lineto(709.74195894,151.80262225)
\curveto(709.74195384,151.89261409)(709.72695385,151.977614)(709.69695894,152.05762225)
\curveto(709.68695389,152.0776139)(709.6769539,152.08761389)(709.66695894,152.08762225)
\lineto(709.62195894,152.13262225)
\curveto(709.60195398,152.14261384)(709.57195401,152.14761383)(709.53195894,152.14762225)
\curveto(709.51195407,152.12761385)(709.49195409,152.11261387)(709.47195894,152.10262225)
\curveto(709.46195412,152.10261388)(709.44695413,152.09761388)(709.42695894,152.08762225)
\curveto(709.36695421,152.03761394)(709.30695427,151.96761401)(709.24695894,151.87762225)
\curveto(709.18695439,151.78761419)(709.13195445,151.70761427)(709.08195894,151.63762225)
\curveto(708.9819546,151.49761448)(708.88695469,151.35261463)(708.79695894,151.20262225)
\curveto(708.70695487,151.06261492)(708.61195497,150.92261506)(708.51195894,150.78262225)
\lineto(707.97195894,150.00262225)
\curveto(707.80195578,149.74261624)(707.62695595,149.4826165)(707.44695894,149.22262225)
\curveto(707.36695621,149.11261687)(707.29195629,149.00761697)(707.22195894,148.90762225)
\lineto(707.01195894,148.60762225)
\curveto(706.96195662,148.52761745)(706.91195667,148.45261753)(706.86195894,148.38262225)
\curveto(706.82195676,148.31261767)(706.7769568,148.23761774)(706.72695894,148.15762225)
\curveto(706.6769569,148.09761788)(706.62695695,148.03261795)(706.57695894,147.96262225)
\curveto(706.53695704,147.90261808)(706.49695708,147.83261815)(706.45695894,147.75262225)
\curveto(706.41695716,147.69261829)(706.39195719,147.62261836)(706.38195894,147.54262225)
\curveto(706.37195721,147.47261851)(706.40695717,147.41761856)(706.48695894,147.37762225)
\curveto(706.55695702,147.32761865)(706.66695691,147.30261868)(706.81695894,147.30262225)
\curveto(706.9769566,147.31261867)(707.11195647,147.31761866)(707.22195894,147.31762225)
\lineto(708.90195894,147.31762225)
\lineto(709.33695894,147.31762225)
\curveto(709.48695409,147.31761866)(709.59195399,147.35761862)(709.65195894,147.43762225)
}
}
{
\newrgbcolor{curcolor}{0 0 0}
\pscustom[linestyle=none,fillstyle=solid,fillcolor=curcolor]
{
\newpath
\moveto(714.64156831,145.38262225)
\lineto(714.94156831,145.38262225)
\curveto(715.05156625,145.39262059)(715.15656615,145.39262059)(715.25656831,145.38262225)
\curveto(715.36656594,145.3826206)(715.46656584,145.37262061)(715.55656831,145.35262225)
\curveto(715.64656566,145.34262064)(715.71656559,145.31762066)(715.76656831,145.27762225)
\curveto(715.78656552,145.25762072)(715.8015655,145.22762075)(715.81156831,145.18762225)
\curveto(715.83156547,145.14762083)(715.85156545,145.10262088)(715.87156831,145.05262225)
\lineto(715.87156831,144.97762225)
\curveto(715.88156542,144.92762105)(715.88156542,144.87262111)(715.87156831,144.81262225)
\lineto(715.87156831,144.66262225)
\lineto(715.87156831,144.18262225)
\curveto(715.87156543,144.01262197)(715.83156547,143.89262209)(715.75156831,143.82262225)
\curveto(715.68156562,143.77262221)(715.59156571,143.74762223)(715.48156831,143.74762225)
\lineto(715.15156831,143.74762225)
\lineto(714.70156831,143.74762225)
\curveto(714.55156675,143.74762223)(714.43656687,143.7776222)(714.35656831,143.83762225)
\curveto(714.31656699,143.86762211)(714.28656702,143.91762206)(714.26656831,143.98762225)
\curveto(714.24656706,144.06762191)(714.23156707,144.15262183)(714.22156831,144.24262225)
\lineto(714.22156831,144.52762225)
\curveto(714.23156707,144.62762135)(714.23656707,144.71262127)(714.23656831,144.78262225)
\lineto(714.23656831,144.97762225)
\curveto(714.23656707,145.03762094)(714.24656706,145.09262089)(714.26656831,145.14262225)
\curveto(714.306567,145.25262073)(714.37656693,145.32262066)(714.47656831,145.35262225)
\curveto(714.5065668,145.35262063)(714.56156674,145.36262062)(714.64156831,145.38262225)
}
}
{
\newrgbcolor{curcolor}{0 0 0}
\pscustom[linestyle=none,fillstyle=solid,fillcolor=curcolor]
{
\newpath
\moveto(724.86172456,146.82262225)
\curveto(724.87171684,146.7826192)(724.87171684,146.73261925)(724.86172456,146.67262225)
\curveto(724.86171685,146.61261937)(724.85671686,146.56261942)(724.84672456,146.52262225)
\curveto(724.84671687,146.4826195)(724.84171687,146.44261954)(724.83172456,146.40262225)
\lineto(724.83172456,146.29762225)
\curveto(724.8117169,146.21761976)(724.79671692,146.13761984)(724.78672456,146.05762225)
\curveto(724.77671694,145.97762)(724.75671696,145.90262008)(724.72672456,145.83262225)
\curveto(724.70671701,145.75262023)(724.68671703,145.6776203)(724.66672456,145.60762225)
\curveto(724.64671707,145.53762044)(724.6167171,145.46262052)(724.57672456,145.38262225)
\curveto(724.39671732,144.96262102)(724.14171757,144.62262136)(723.81172456,144.36262225)
\curveto(723.48171823,144.10262188)(723.09171862,143.89762208)(722.64172456,143.74762225)
\curveto(722.52171919,143.70762227)(722.39671932,143.6826223)(722.26672456,143.67262225)
\curveto(722.14671957,143.65262233)(722.02171969,143.62762235)(721.89172456,143.59762225)
\curveto(721.83171988,143.58762239)(721.76671995,143.5826224)(721.69672456,143.58262225)
\curveto(721.63672008,143.5826224)(721.57172014,143.5776224)(721.50172456,143.56762225)
\lineto(721.38172456,143.56762225)
\lineto(721.18672456,143.56762225)
\curveto(721.12672059,143.55762242)(721.07172064,143.56262242)(721.02172456,143.58262225)
\curveto(720.95172076,143.60262238)(720.88672083,143.60762237)(720.82672456,143.59762225)
\curveto(720.76672095,143.58762239)(720.70672101,143.59262239)(720.64672456,143.61262225)
\curveto(720.59672112,143.62262236)(720.55172116,143.62762235)(720.51172456,143.62762225)
\curveto(720.47172124,143.62762235)(720.42672129,143.63762234)(720.37672456,143.65762225)
\curveto(720.29672142,143.6776223)(720.22172149,143.69762228)(720.15172456,143.71762225)
\curveto(720.08172163,143.72762225)(720.0117217,143.74262224)(719.94172456,143.76262225)
\curveto(719.46172225,143.93262205)(719.06172265,144.14262184)(718.74172456,144.39262225)
\curveto(718.43172328,144.65262133)(718.18172353,145.00762097)(717.99172456,145.45762225)
\curveto(717.96172375,145.51762046)(717.93672378,145.5776204)(717.91672456,145.63762225)
\curveto(717.90672381,145.70762027)(717.89172382,145.7826202)(717.87172456,145.86262225)
\curveto(717.85172386,145.92262006)(717.83672388,145.98761999)(717.82672456,146.05762225)
\curveto(717.8167239,146.12761985)(717.80172391,146.19761978)(717.78172456,146.26762225)
\curveto(717.77172394,146.31761966)(717.76672395,146.35761962)(717.76672456,146.38762225)
\lineto(717.76672456,146.50762225)
\curveto(717.75672396,146.54761943)(717.74672397,146.59761938)(717.73672456,146.65762225)
\curveto(717.73672398,146.71761926)(717.74172397,146.76761921)(717.75172456,146.80762225)
\lineto(717.75172456,146.94262225)
\curveto(717.76172395,146.99261899)(717.76672395,147.04261894)(717.76672456,147.09262225)
\curveto(717.78672393,147.19261879)(717.80172391,147.28761869)(717.81172456,147.37762225)
\curveto(717.82172389,147.4776185)(717.84172387,147.57261841)(717.87172456,147.66262225)
\curveto(717.92172379,147.81261817)(717.97672374,147.95261803)(718.03672456,148.08262225)
\curveto(718.09672362,148.21261777)(718.16672355,148.33261765)(718.24672456,148.44262225)
\curveto(718.27672344,148.49261749)(718.30672341,148.53261745)(718.33672456,148.56262225)
\curveto(718.37672334,148.59261739)(718.4117233,148.62761735)(718.44172456,148.66762225)
\curveto(718.50172321,148.74761723)(718.57172314,148.81761716)(718.65172456,148.87762225)
\curveto(718.711723,148.92761705)(718.77172294,148.97261701)(718.83172456,149.01262225)
\lineto(719.04172456,149.16262225)
\curveto(719.09172262,149.20261678)(719.14172257,149.23761674)(719.19172456,149.26762225)
\curveto(719.24172247,149.30761667)(719.27672244,149.36261662)(719.29672456,149.43262225)
\curveto(719.29672242,149.46261652)(719.28672243,149.48761649)(719.26672456,149.50762225)
\curveto(719.25672246,149.53761644)(719.24672247,149.56261642)(719.23672456,149.58262225)
\curveto(719.19672252,149.63261635)(719.14672257,149.6776163)(719.08672456,149.71762225)
\curveto(719.03672268,149.76761621)(718.98672273,149.81261617)(718.93672456,149.85262225)
\curveto(718.89672282,149.8826161)(718.84672287,149.93761604)(718.78672456,150.01762225)
\curveto(718.76672295,150.04761593)(718.73672298,150.07261591)(718.69672456,150.09262225)
\curveto(718.66672305,150.12261586)(718.64172307,150.15761582)(718.62172456,150.19762225)
\curveto(718.45172326,150.40761557)(718.32172339,150.65261533)(718.23172456,150.93262225)
\curveto(718.2117235,151.01261497)(718.19672352,151.09261489)(718.18672456,151.17262225)
\curveto(718.17672354,151.25261473)(718.16172355,151.33261465)(718.14172456,151.41262225)
\curveto(718.12172359,151.46261452)(718.1117236,151.52761445)(718.11172456,151.60762225)
\curveto(718.1117236,151.69761428)(718.12172359,151.76761421)(718.14172456,151.81762225)
\curveto(718.14172357,151.91761406)(718.14672357,151.98761399)(718.15672456,152.02762225)
\curveto(718.17672354,152.10761387)(718.19172352,152.1776138)(718.20172456,152.23762225)
\curveto(718.2117235,152.30761367)(718.22672349,152.3776136)(718.24672456,152.44762225)
\curveto(718.39672332,152.8776131)(718.6117231,153.22261276)(718.89172456,153.48262225)
\curveto(719.18172253,153.74261224)(719.53172218,153.95761202)(719.94172456,154.12762225)
\curveto(720.05172166,154.1776118)(720.16672155,154.20761177)(720.28672456,154.21762225)
\curveto(720.4167213,154.23761174)(720.54672117,154.26761171)(720.67672456,154.30762225)
\curveto(720.75672096,154.30761167)(720.82672089,154.30761167)(720.88672456,154.30762225)
\curveto(720.95672076,154.31761166)(721.03172068,154.32761165)(721.11172456,154.33762225)
\curveto(721.90171981,154.35761162)(722.55671916,154.22761175)(723.07672456,153.94762225)
\curveto(723.60671811,153.66761231)(723.98671773,153.25761272)(724.21672456,152.71762225)
\curveto(724.32671739,152.48761349)(724.39671732,152.20261378)(724.42672456,151.86262225)
\curveto(724.46671725,151.53261445)(724.43671728,151.22761475)(724.33672456,150.94762225)
\curveto(724.29671742,150.81761516)(724.24671747,150.69761528)(724.18672456,150.58762225)
\curveto(724.13671758,150.4776155)(724.07671764,150.37261561)(724.00672456,150.27262225)
\curveto(723.98671773,150.23261575)(723.95671776,150.19761578)(723.91672456,150.16762225)
\lineto(723.82672456,150.07762225)
\curveto(723.77671794,149.98761599)(723.716718,149.92261606)(723.64672456,149.88262225)
\curveto(723.59671812,149.83261615)(723.54171817,149.7826162)(723.48172456,149.73262225)
\curveto(723.43171828,149.69261629)(723.38671833,149.64761633)(723.34672456,149.59762225)
\curveto(723.32671839,149.5776164)(723.30671841,149.55261643)(723.28672456,149.52262225)
\curveto(723.27671844,149.50261648)(723.27671844,149.4776165)(723.28672456,149.44762225)
\curveto(723.29671842,149.39761658)(723.32671839,149.34761663)(723.37672456,149.29762225)
\curveto(723.42671829,149.25761672)(723.48171823,149.21761676)(723.54172456,149.17762225)
\lineto(723.72172456,149.05762225)
\curveto(723.78171793,149.02761695)(723.83171788,148.99761698)(723.87172456,148.96762225)
\curveto(724.20171751,148.72761725)(724.45171726,148.41761756)(724.62172456,148.03762225)
\curveto(724.66171705,147.95761802)(724.69171702,147.87261811)(724.71172456,147.78262225)
\curveto(724.74171697,147.69261829)(724.76671695,147.60261838)(724.78672456,147.51262225)
\curveto(724.79671692,147.46261852)(724.80671691,147.40761857)(724.81672456,147.34762225)
\lineto(724.84672456,147.19762225)
\curveto(724.85671686,147.13761884)(724.85671686,147.07261891)(724.84672456,147.00262225)
\curveto(724.83671688,146.94261904)(724.84171687,146.8826191)(724.86172456,146.82262225)
\moveto(719.47672456,151.86262225)
\curveto(719.44672227,151.75261423)(719.44172227,151.61261437)(719.46172456,151.44262225)
\curveto(719.48172223,151.2826147)(719.50672221,151.15761482)(719.53672456,151.06762225)
\curveto(719.64672207,150.74761523)(719.79672192,150.50261548)(719.98672456,150.33262225)
\curveto(720.17672154,150.17261581)(720.44172127,150.04261594)(720.78172456,149.94262225)
\curveto(720.9117208,149.91261607)(721.07672064,149.88761609)(721.27672456,149.86762225)
\curveto(721.47672024,149.85761612)(721.64672007,149.87261611)(721.78672456,149.91262225)
\curveto(722.07671964,149.99261599)(722.3167194,150.10261588)(722.50672456,150.24262225)
\curveto(722.70671901,150.39261559)(722.86171885,150.59261539)(722.97172456,150.84262225)
\curveto(722.99171872,150.89261509)(723.00171871,150.93761504)(723.00172456,150.97762225)
\curveto(723.0117187,151.01761496)(723.02671869,151.06261492)(723.04672456,151.11262225)
\curveto(723.07671864,151.22261476)(723.09671862,151.36261462)(723.10672456,151.53262225)
\curveto(723.1167186,151.70261428)(723.10671861,151.84761413)(723.07672456,151.96762225)
\curveto(723.05671866,152.05761392)(723.03171868,152.14261384)(723.00172456,152.22262225)
\curveto(722.98171873,152.30261368)(722.94671877,152.3826136)(722.89672456,152.46262225)
\curveto(722.72671899,152.73261325)(722.50171921,152.92761305)(722.22172456,153.04762225)
\curveto(721.95171976,153.16761281)(721.59172012,153.22761275)(721.14172456,153.22762225)
\curveto(721.12172059,153.20761277)(721.09172062,153.20261278)(721.05172456,153.21262225)
\curveto(721.0117207,153.22261276)(720.97672074,153.22261276)(720.94672456,153.21262225)
\curveto(720.89672082,153.19261279)(720.84172087,153.1776128)(720.78172456,153.16762225)
\curveto(720.73172098,153.16761281)(720.68172103,153.15761282)(720.63172456,153.13762225)
\curveto(720.39172132,153.04761293)(720.18172153,152.93261305)(720.00172456,152.79262225)
\curveto(719.82172189,152.66261332)(719.68172203,152.4826135)(719.58172456,152.25262225)
\curveto(719.56172215,152.19261379)(719.54172217,152.12761385)(719.52172456,152.05762225)
\curveto(719.5117222,151.99761398)(719.49672222,151.93261405)(719.47672456,151.86262225)
\moveto(723.49672456,146.32762225)
\curveto(723.54671817,146.51761946)(723.55171816,146.72261926)(723.51172456,146.94262225)
\curveto(723.48171823,147.16261882)(723.43671828,147.34261864)(723.37672456,147.48262225)
\curveto(723.20671851,147.85261813)(722.94671877,148.15761782)(722.59672456,148.39762225)
\curveto(722.25671946,148.63761734)(721.82171989,148.75761722)(721.29172456,148.75762225)
\curveto(721.26172045,148.73761724)(721.22172049,148.73261725)(721.17172456,148.74262225)
\curveto(721.12172059,148.76261722)(721.08172063,148.76761721)(721.05172456,148.75762225)
\lineto(720.78172456,148.69762225)
\curveto(720.70172101,148.68761729)(720.62172109,148.67261731)(720.54172456,148.65262225)
\curveto(720.24172147,148.54261744)(719.97672174,148.39761758)(719.74672456,148.21762225)
\curveto(719.52672219,148.03761794)(719.35672236,147.80761817)(719.23672456,147.52762225)
\curveto(719.20672251,147.44761853)(719.18172253,147.36761861)(719.16172456,147.28762225)
\curveto(719.14172257,147.20761877)(719.12172259,147.12261886)(719.10172456,147.03262225)
\curveto(719.07172264,146.91261907)(719.06172265,146.76261922)(719.07172456,146.58262225)
\curveto(719.09172262,146.40261958)(719.1167226,146.26261972)(719.14672456,146.16262225)
\curveto(719.16672255,146.11261987)(719.17672254,146.06761991)(719.17672456,146.02762225)
\curveto(719.18672253,145.99761998)(719.20172251,145.95762002)(719.22172456,145.90762225)
\curveto(719.32172239,145.68762029)(719.45172226,145.48762049)(719.61172456,145.30762225)
\curveto(719.78172193,145.12762085)(719.97672174,144.99262099)(720.19672456,144.90262225)
\curveto(720.26672145,144.86262112)(720.36172135,144.82762115)(720.48172456,144.79762225)
\curveto(720.70172101,144.70762127)(720.95672076,144.66262132)(721.24672456,144.66262225)
\lineto(721.53172456,144.66262225)
\curveto(721.63172008,144.6826213)(721.72671999,144.69762128)(721.81672456,144.70762225)
\curveto(721.90671981,144.71762126)(721.99671972,144.73762124)(722.08672456,144.76762225)
\curveto(722.34671937,144.84762113)(722.58671913,144.977621)(722.80672456,145.15762225)
\curveto(723.03671868,145.34762063)(723.20671851,145.56262042)(723.31672456,145.80262225)
\curveto(723.35671836,145.8826201)(723.38671833,145.96262002)(723.40672456,146.04262225)
\curveto(723.43671828,146.13261985)(723.46671825,146.22761975)(723.49672456,146.32762225)
}
}
{
\newrgbcolor{curcolor}{0 0 0}
\pscustom[linestyle=none,fillstyle=solid,fillcolor=curcolor]
{
\newpath
\moveto(736.00133394,152.26762225)
\curveto(735.80132364,151.977614)(735.59132385,151.69261429)(735.37133394,151.41262225)
\curveto(735.16132428,151.13261485)(734.95632448,150.84761513)(734.75633394,150.55762225)
\curveto(734.15632528,149.70761627)(733.55132589,148.86761711)(732.94133394,148.03762225)
\curveto(732.33132711,147.21761876)(731.72632771,146.3826196)(731.12633394,145.53262225)
\lineto(730.61633394,144.81262225)
\lineto(730.10633394,144.12262225)
\curveto(730.02632941,144.01262197)(729.94632949,143.89762208)(729.86633394,143.77762225)
\curveto(729.78632965,143.65762232)(729.69132975,143.56262242)(729.58133394,143.49262225)
\curveto(729.5413299,143.47262251)(729.47632996,143.45762252)(729.38633394,143.44762225)
\curveto(729.30633013,143.42762255)(729.21633022,143.41762256)(729.11633394,143.41762225)
\curveto(729.01633042,143.41762256)(728.92133052,143.42262256)(728.83133394,143.43262225)
\curveto(728.75133069,143.44262254)(728.69133075,143.46262252)(728.65133394,143.49262225)
\curveto(728.62133082,143.51262247)(728.59633084,143.54762243)(728.57633394,143.59762225)
\curveto(728.56633087,143.63762234)(728.57133087,143.6826223)(728.59133394,143.73262225)
\curveto(728.63133081,143.81262217)(728.67633076,143.88762209)(728.72633394,143.95762225)
\curveto(728.78633065,144.03762194)(728.8413306,144.11762186)(728.89133394,144.19762225)
\curveto(729.13133031,144.53762144)(729.37633006,144.87262111)(729.62633394,145.20262225)
\curveto(729.87632956,145.53262045)(730.11632932,145.86762011)(730.34633394,146.20762225)
\curveto(730.50632893,146.42761955)(730.66632877,146.64261934)(730.82633394,146.85262225)
\curveto(730.98632845,147.06261892)(731.14632829,147.2776187)(731.30633394,147.49762225)
\curveto(731.66632777,148.01761796)(732.03132741,148.52761745)(732.40133394,149.02762225)
\curveto(732.77132667,149.52761645)(733.1413263,150.03761594)(733.51133394,150.55762225)
\curveto(733.65132579,150.75761522)(733.79132565,150.95261503)(733.93133394,151.14262225)
\curveto(734.08132536,151.33261465)(734.22632521,151.52761445)(734.36633394,151.72762225)
\curveto(734.57632486,152.02761395)(734.79132465,152.32761365)(735.01133394,152.62762225)
\lineto(735.67133394,153.52762225)
\lineto(735.85133394,153.79762225)
\lineto(736.06133394,154.06762225)
\lineto(736.18133394,154.24762225)
\curveto(736.23132321,154.30761167)(736.28132316,154.36261162)(736.33133394,154.41262225)
\curveto(736.40132304,154.46261152)(736.47632296,154.49761148)(736.55633394,154.51762225)
\curveto(736.57632286,154.52761145)(736.60132284,154.52761145)(736.63133394,154.51762225)
\curveto(736.67132277,154.51761146)(736.70132274,154.52761145)(736.72133394,154.54762225)
\curveto(736.8413226,154.54761143)(736.97632246,154.54261144)(737.12633394,154.53262225)
\curveto(737.27632216,154.53261145)(737.36632207,154.48761149)(737.39633394,154.39762225)
\curveto(737.41632202,154.36761161)(737.42132202,154.33261165)(737.41133394,154.29262225)
\curveto(737.40132204,154.25261173)(737.38632205,154.22261176)(737.36633394,154.20262225)
\curveto(737.32632211,154.12261186)(737.28632215,154.05261193)(737.24633394,153.99262225)
\curveto(737.20632223,153.93261205)(737.16132228,153.87261211)(737.11133394,153.81262225)
\lineto(736.54133394,153.03262225)
\curveto(736.36132308,152.7826132)(736.18132326,152.52761345)(736.00133394,152.26762225)
\moveto(729.14633394,148.36762225)
\curveto(729.09633034,148.38761759)(729.04633039,148.39261759)(728.99633394,148.38262225)
\curveto(728.94633049,148.37261761)(728.89633054,148.3776176)(728.84633394,148.39762225)
\curveto(728.7363307,148.41761756)(728.63133081,148.43761754)(728.53133394,148.45762225)
\curveto(728.441331,148.48761749)(728.34633109,148.52761745)(728.24633394,148.57762225)
\curveto(727.91633152,148.71761726)(727.66133178,148.91261707)(727.48133394,149.16262225)
\curveto(727.30133214,149.42261656)(727.15633228,149.73261625)(727.04633394,150.09262225)
\curveto(727.01633242,150.17261581)(726.99633244,150.25261573)(726.98633394,150.33262225)
\curveto(726.97633246,150.42261556)(726.96133248,150.50761547)(726.94133394,150.58762225)
\curveto(726.93133251,150.63761534)(726.92633251,150.70261528)(726.92633394,150.78262225)
\curveto(726.91633252,150.81261517)(726.91133253,150.84261514)(726.91133394,150.87262225)
\curveto(726.91133253,150.91261507)(726.90633253,150.94761503)(726.89633394,150.97762225)
\lineto(726.89633394,151.12762225)
\curveto(726.88633255,151.1776148)(726.88133256,151.23761474)(726.88133394,151.30762225)
\curveto(726.88133256,151.38761459)(726.88633255,151.45261453)(726.89633394,151.50262225)
\lineto(726.89633394,151.66762225)
\curveto(726.91633252,151.71761426)(726.92133252,151.76261422)(726.91133394,151.80262225)
\curveto(726.91133253,151.85261413)(726.91633252,151.89761408)(726.92633394,151.93762225)
\curveto(726.9363325,151.977614)(726.9413325,152.01261397)(726.94133394,152.04262225)
\curveto(726.9413325,152.0826139)(726.94633249,152.12261386)(726.95633394,152.16262225)
\curveto(726.98633245,152.27261371)(727.00633243,152.3826136)(727.01633394,152.49262225)
\curveto(727.0363324,152.61261337)(727.07133237,152.72761325)(727.12133394,152.83762225)
\curveto(727.26133218,153.1776128)(727.42133202,153.45261253)(727.60133394,153.66262225)
\curveto(727.79133165,153.8826121)(728.06133138,154.06261192)(728.41133394,154.20262225)
\curveto(728.49133095,154.23261175)(728.57633086,154.25261173)(728.66633394,154.26262225)
\curveto(728.75633068,154.2826117)(728.85133059,154.30261168)(728.95133394,154.32262225)
\curveto(728.98133046,154.33261165)(729.0363304,154.33261165)(729.11633394,154.32262225)
\curveto(729.19633024,154.32261166)(729.24633019,154.33261165)(729.26633394,154.35262225)
\curveto(729.82632961,154.36261162)(730.27632916,154.25261173)(730.61633394,154.02262225)
\curveto(730.96632847,153.79261219)(731.22632821,153.48761249)(731.39633394,153.10762225)
\curveto(731.436328,153.01761296)(731.47132797,152.92261306)(731.50133394,152.82262225)
\curveto(731.53132791,152.72261326)(731.55632788,152.62261336)(731.57633394,152.52262225)
\curveto(731.59632784,152.49261349)(731.60132784,152.46261352)(731.59133394,152.43262225)
\curveto(731.59132785,152.40261358)(731.59632784,152.37261361)(731.60633394,152.34262225)
\curveto(731.6363278,152.23261375)(731.65632778,152.10761387)(731.66633394,151.96762225)
\curveto(731.67632776,151.83761414)(731.68632775,151.70261428)(731.69633394,151.56262225)
\lineto(731.69633394,151.39762225)
\curveto(731.70632773,151.33761464)(731.70632773,151.2826147)(731.69633394,151.23262225)
\curveto(731.68632775,151.1826148)(731.68132776,151.13261485)(731.68133394,151.08262225)
\lineto(731.68133394,150.94762225)
\curveto(731.67132777,150.90761507)(731.66632777,150.86761511)(731.66633394,150.82762225)
\curveto(731.67632776,150.78761519)(731.67132777,150.74261524)(731.65133394,150.69262225)
\curveto(731.63132781,150.5826154)(731.61132783,150.4776155)(731.59133394,150.37762225)
\curveto(731.58132786,150.2776157)(731.56132788,150.1776158)(731.53133394,150.07762225)
\curveto(731.40132804,149.71761626)(731.2363282,149.40261658)(731.03633394,149.13262225)
\curveto(730.8363286,148.86261712)(730.56132888,148.65761732)(730.21133394,148.51762225)
\curveto(730.13132931,148.48761749)(730.04632939,148.46261752)(729.95633394,148.44262225)
\lineto(729.68633394,148.38262225)
\curveto(729.6363298,148.37261761)(729.59132985,148.36761761)(729.55133394,148.36762225)
\curveto(729.51132993,148.3776176)(729.47132997,148.3776176)(729.43133394,148.36762225)
\curveto(729.33133011,148.34761763)(729.2363302,148.34761763)(729.14633394,148.36762225)
\moveto(728.30633394,149.76262225)
\curveto(728.34633109,149.69261629)(728.38633105,149.62761635)(728.42633394,149.56762225)
\curveto(728.46633097,149.51761646)(728.51633092,149.46761651)(728.57633394,149.41762225)
\lineto(728.72633394,149.29762225)
\curveto(728.78633065,149.26761671)(728.85133059,149.24261674)(728.92133394,149.22262225)
\curveto(728.96133048,149.20261678)(728.99633044,149.19261679)(729.02633394,149.19262225)
\curveto(729.06633037,149.20261678)(729.10633033,149.19761678)(729.14633394,149.17762225)
\curveto(729.17633026,149.1776168)(729.21633022,149.17261681)(729.26633394,149.16262225)
\curveto(729.31633012,149.16261682)(729.35633008,149.16761681)(729.38633394,149.17762225)
\lineto(729.61133394,149.22262225)
\curveto(729.86132958,149.30261668)(730.04632939,149.42761655)(730.16633394,149.59762225)
\curveto(730.24632919,149.69761628)(730.31632912,149.82761615)(730.37633394,149.98762225)
\curveto(730.45632898,150.16761581)(730.51632892,150.39261559)(730.55633394,150.66262225)
\curveto(730.59632884,150.94261504)(730.61132883,151.22261476)(730.60133394,151.50262225)
\curveto(730.59132885,151.79261419)(730.56132888,152.06761391)(730.51133394,152.32762225)
\curveto(730.46132898,152.58761339)(730.38632905,152.79761318)(730.28633394,152.95762225)
\curveto(730.16632927,153.15761282)(730.01632942,153.30761267)(729.83633394,153.40762225)
\curveto(729.75632968,153.45761252)(729.66632977,153.48761249)(729.56633394,153.49762225)
\curveto(729.46632997,153.51761246)(729.36133008,153.52761245)(729.25133394,153.52762225)
\curveto(729.23133021,153.51761246)(729.20633023,153.51261247)(729.17633394,153.51262225)
\curveto(729.15633028,153.52261246)(729.1363303,153.52261246)(729.11633394,153.51262225)
\curveto(729.06633037,153.50261248)(729.02133042,153.49261249)(728.98133394,153.48262225)
\curveto(728.9413305,153.4826125)(728.90133054,153.47261251)(728.86133394,153.45262225)
\curveto(728.68133076,153.37261261)(728.53133091,153.25261273)(728.41133394,153.09262225)
\curveto(728.30133114,152.93261305)(728.21133123,152.75261323)(728.14133394,152.55262225)
\curveto(728.08133136,152.36261362)(728.0363314,152.13761384)(728.00633394,151.87762225)
\curveto(727.98633145,151.61761436)(727.98133146,151.35261463)(727.99133394,151.08262225)
\curveto(728.00133144,150.82261516)(728.03133141,150.57261541)(728.08133394,150.33262225)
\curveto(728.1413313,150.10261588)(728.21633122,149.91261607)(728.30633394,149.76262225)
\moveto(739.10633394,146.77762225)
\curveto(739.11632032,146.72761925)(739.12132032,146.63761934)(739.12133394,146.50762225)
\curveto(739.12132032,146.3776196)(739.11132033,146.28761969)(739.09133394,146.23762225)
\curveto(739.07132037,146.18761979)(739.06632037,146.13261985)(739.07633394,146.07262225)
\curveto(739.08632035,146.02261996)(739.08632035,145.97262001)(739.07633394,145.92262225)
\curveto(739.0363204,145.7826202)(739.00632043,145.64762033)(738.98633394,145.51762225)
\curveto(738.97632046,145.38762059)(738.94632049,145.26762071)(738.89633394,145.15762225)
\curveto(738.75632068,144.80762117)(738.59132085,144.51262147)(738.40133394,144.27262225)
\curveto(738.21132123,144.04262194)(737.9413215,143.85762212)(737.59133394,143.71762225)
\curveto(737.51132193,143.68762229)(737.42632201,143.66762231)(737.33633394,143.65762225)
\curveto(737.24632219,143.63762234)(737.16132228,143.61762236)(737.08133394,143.59762225)
\curveto(737.03132241,143.58762239)(736.98132246,143.5826224)(736.93133394,143.58262225)
\curveto(736.88132256,143.5826224)(736.83132261,143.5776224)(736.78133394,143.56762225)
\curveto(736.75132269,143.55762242)(736.70132274,143.55762242)(736.63133394,143.56762225)
\curveto(736.56132288,143.56762241)(736.51132293,143.57262241)(736.48133394,143.58262225)
\curveto(736.42132302,143.60262238)(736.36132308,143.61262237)(736.30133394,143.61262225)
\curveto(736.25132319,143.60262238)(736.20132324,143.60762237)(736.15133394,143.62762225)
\curveto(736.06132338,143.64762233)(735.97132347,143.67262231)(735.88133394,143.70262225)
\curveto(735.80132364,143.72262226)(735.72132372,143.75262223)(735.64133394,143.79262225)
\curveto(735.32132412,143.93262205)(735.07132437,144.12762185)(734.89133394,144.37762225)
\curveto(734.71132473,144.63762134)(734.56132488,144.94262104)(734.44133394,145.29262225)
\curveto(734.42132502,145.37262061)(734.40632503,145.45762052)(734.39633394,145.54762225)
\curveto(734.38632505,145.63762034)(734.37132507,145.72262026)(734.35133394,145.80262225)
\curveto(734.3413251,145.83262015)(734.3363251,145.86262012)(734.33633394,145.89262225)
\lineto(734.33633394,145.99762225)
\curveto(734.31632512,146.0776199)(734.30632513,146.15761982)(734.30633394,146.23762225)
\lineto(734.30633394,146.37262225)
\curveto(734.28632515,146.47261951)(734.28632515,146.57261941)(734.30633394,146.67262225)
\lineto(734.30633394,146.85262225)
\curveto(734.31632512,146.90261908)(734.32132512,146.94761903)(734.32133394,146.98762225)
\curveto(734.32132512,147.03761894)(734.32632511,147.0826189)(734.33633394,147.12262225)
\curveto(734.34632509,147.16261882)(734.35132509,147.19761878)(734.35133394,147.22762225)
\curveto(734.35132509,147.26761871)(734.35632508,147.30761867)(734.36633394,147.34762225)
\lineto(734.42633394,147.67762225)
\curveto(734.44632499,147.79761818)(734.47632496,147.90761807)(734.51633394,148.00762225)
\curveto(734.65632478,148.33761764)(734.81632462,148.61261737)(734.99633394,148.83262225)
\curveto(735.18632425,149.06261692)(735.44632399,149.24761673)(735.77633394,149.38762225)
\curveto(735.85632358,149.42761655)(735.9413235,149.45261653)(736.03133394,149.46262225)
\lineto(736.33133394,149.52262225)
\lineto(736.46633394,149.52262225)
\curveto(736.51632292,149.53261645)(736.56632287,149.53761644)(736.61633394,149.53762225)
\curveto(737.18632225,149.55761642)(737.64632179,149.45261653)(737.99633394,149.22262225)
\curveto(738.35632108,149.00261698)(738.62132082,148.70261728)(738.79133394,148.32262225)
\curveto(738.8413206,148.22261776)(738.88132056,148.12261786)(738.91133394,148.02262225)
\curveto(738.9413205,147.92261806)(738.97132047,147.81761816)(739.00133394,147.70762225)
\curveto(739.01132043,147.66761831)(739.01632042,147.63261835)(739.01633394,147.60262225)
\curveto(739.01632042,147.5826184)(739.02132042,147.55261843)(739.03133394,147.51262225)
\curveto(739.05132039,147.44261854)(739.06132038,147.36761861)(739.06133394,147.28762225)
\curveto(739.06132038,147.20761877)(739.07132037,147.12761885)(739.09133394,147.04762225)
\curveto(739.09132035,146.99761898)(739.09132035,146.95261903)(739.09133394,146.91262225)
\curveto(739.09132035,146.87261911)(739.09632034,146.82761915)(739.10633394,146.77762225)
\moveto(737.99633394,146.34262225)
\curveto(738.00632143,146.39261959)(738.01132143,146.46761951)(738.01133394,146.56762225)
\curveto(738.02132142,146.66761931)(738.01632142,146.74261924)(737.99633394,146.79262225)
\curveto(737.97632146,146.85261913)(737.97132147,146.90761907)(737.98133394,146.95762225)
\curveto(738.00132144,147.01761896)(738.00132144,147.0776189)(737.98133394,147.13762225)
\curveto(737.97132147,147.16761881)(737.96632147,147.20261878)(737.96633394,147.24262225)
\curveto(737.96632147,147.2826187)(737.96132148,147.32261866)(737.95133394,147.36262225)
\curveto(737.93132151,147.44261854)(737.91132153,147.51761846)(737.89133394,147.58762225)
\curveto(737.88132156,147.66761831)(737.86632157,147.74761823)(737.84633394,147.82762225)
\curveto(737.81632162,147.88761809)(737.79132165,147.94761803)(737.77133394,148.00762225)
\curveto(737.75132169,148.06761791)(737.72132172,148.12761785)(737.68133394,148.18762225)
\curveto(737.58132186,148.35761762)(737.45132199,148.49261749)(737.29133394,148.59262225)
\curveto(737.21132223,148.64261734)(737.11632232,148.6776173)(737.00633394,148.69762225)
\curveto(736.89632254,148.71761726)(736.77132267,148.72761725)(736.63133394,148.72762225)
\curveto(736.61132283,148.71761726)(736.58632285,148.71261727)(736.55633394,148.71262225)
\curveto(736.52632291,148.72261726)(736.49632294,148.72261726)(736.46633394,148.71262225)
\lineto(736.31633394,148.65262225)
\curveto(736.26632317,148.64261734)(736.22132322,148.62761735)(736.18133394,148.60762225)
\curveto(735.99132345,148.49761748)(735.84632359,148.35261763)(735.74633394,148.17262225)
\curveto(735.65632378,147.99261799)(735.57632386,147.78761819)(735.50633394,147.55762225)
\curveto(735.46632397,147.42761855)(735.44632399,147.29261869)(735.44633394,147.15262225)
\curveto(735.44632399,147.02261896)(735.436324,146.8776191)(735.41633394,146.71762225)
\curveto(735.40632403,146.66761931)(735.39632404,146.60761937)(735.38633394,146.53762225)
\curveto(735.38632405,146.46761951)(735.39632404,146.40761957)(735.41633394,146.35762225)
\lineto(735.41633394,146.19262225)
\lineto(735.41633394,146.01262225)
\curveto(735.42632401,145.96262002)(735.436324,145.90762007)(735.44633394,145.84762225)
\curveto(735.45632398,145.79762018)(735.46132398,145.74262024)(735.46133394,145.68262225)
\curveto(735.47132397,145.62262036)(735.48632395,145.56762041)(735.50633394,145.51762225)
\curveto(735.55632388,145.32762065)(735.61632382,145.15262083)(735.68633394,144.99262225)
\curveto(735.75632368,144.83262115)(735.86132358,144.70262128)(736.00133394,144.60262225)
\curveto(736.13132331,144.50262148)(736.27132317,144.43262155)(736.42133394,144.39262225)
\curveto(736.45132299,144.3826216)(736.47632296,144.3776216)(736.49633394,144.37762225)
\curveto(736.52632291,144.38762159)(736.55632288,144.38762159)(736.58633394,144.37762225)
\curveto(736.60632283,144.3776216)(736.6363228,144.37262161)(736.67633394,144.36262225)
\curveto(736.71632272,144.36262162)(736.75132269,144.36762161)(736.78133394,144.37762225)
\curveto(736.82132262,144.38762159)(736.86132258,144.39262159)(736.90133394,144.39262225)
\curveto(736.9413225,144.39262159)(736.98132246,144.40262158)(737.02133394,144.42262225)
\curveto(737.26132218,144.50262148)(737.45632198,144.63762134)(737.60633394,144.82762225)
\curveto(737.72632171,145.00762097)(737.81632162,145.21262077)(737.87633394,145.44262225)
\curveto(737.89632154,145.51262047)(737.91132153,145.5826204)(737.92133394,145.65262225)
\curveto(737.93132151,145.73262025)(737.94632149,145.81262017)(737.96633394,145.89262225)
\curveto(737.96632147,145.95262003)(737.97132147,145.99761998)(737.98133394,146.02762225)
\curveto(737.98132146,146.04761993)(737.98132146,146.07261991)(737.98133394,146.10262225)
\curveto(737.98132146,146.14261984)(737.98632145,146.17261981)(737.99633394,146.19262225)
\lineto(737.99633394,146.34262225)
}
}
{
\newrgbcolor{curcolor}{0 0 0}
\pscustom[linestyle=none,fillstyle=solid,fillcolor=curcolor]
{
\newpath
\moveto(244.38293428,284.20342059)
\curveto(244.39292656,284.16341754)(244.39292656,284.11341759)(244.38293428,284.05342059)
\curveto(244.38292657,283.99341771)(244.37792658,283.94341776)(244.36793428,283.90342059)
\curveto(244.36792659,283.86341784)(244.36292659,283.82341788)(244.35293428,283.78342059)
\lineto(244.35293428,283.67842059)
\curveto(244.33292662,283.5984181)(244.31792664,283.51841818)(244.30793428,283.43842059)
\curveto(244.29792666,283.35841834)(244.27792668,283.28341842)(244.24793428,283.21342059)
\curveto(244.22792673,283.13341857)(244.20792675,283.05841864)(244.18793428,282.98842059)
\curveto(244.16792679,282.91841878)(244.13792682,282.84341886)(244.09793428,282.76342059)
\curveto(243.91792704,282.34341936)(243.66292729,282.0034197)(243.33293428,281.74342059)
\curveto(243.00292795,281.48342022)(242.61292834,281.27842042)(242.16293428,281.12842059)
\curveto(242.04292891,281.08842061)(241.91792904,281.06342064)(241.78793428,281.05342059)
\curveto(241.66792929,281.03342067)(241.54292941,281.00842069)(241.41293428,280.97842059)
\curveto(241.3529296,280.96842073)(241.28792967,280.96342074)(241.21793428,280.96342059)
\curveto(241.1579298,280.96342074)(241.09292986,280.95842074)(241.02293428,280.94842059)
\lineto(240.90293428,280.94842059)
\lineto(240.70793428,280.94842059)
\curveto(240.64793031,280.93842076)(240.59293036,280.94342076)(240.54293428,280.96342059)
\curveto(240.47293048,280.98342072)(240.40793055,280.98842071)(240.34793428,280.97842059)
\curveto(240.28793067,280.96842073)(240.22793073,280.97342073)(240.16793428,280.99342059)
\curveto(240.11793084,281.0034207)(240.07293088,281.00842069)(240.03293428,281.00842059)
\curveto(239.99293096,281.00842069)(239.94793101,281.01842068)(239.89793428,281.03842059)
\curveto(239.81793114,281.05842064)(239.74293121,281.07842062)(239.67293428,281.09842059)
\curveto(239.60293135,281.10842059)(239.53293142,281.12342058)(239.46293428,281.14342059)
\curveto(238.98293197,281.31342039)(238.58293237,281.52342018)(238.26293428,281.77342059)
\curveto(237.952933,282.03341967)(237.70293325,282.38841931)(237.51293428,282.83842059)
\curveto(237.48293347,282.8984188)(237.4579335,282.95841874)(237.43793428,283.01842059)
\curveto(237.42793353,283.08841861)(237.41293354,283.16341854)(237.39293428,283.24342059)
\curveto(237.37293358,283.3034184)(237.3579336,283.36841833)(237.34793428,283.43842059)
\curveto(237.33793362,283.50841819)(237.32293363,283.57841812)(237.30293428,283.64842059)
\curveto(237.29293366,283.698418)(237.28793367,283.73841796)(237.28793428,283.76842059)
\lineto(237.28793428,283.88842059)
\curveto(237.27793368,283.92841777)(237.26793369,283.97841772)(237.25793428,284.03842059)
\curveto(237.2579337,284.0984176)(237.26293369,284.14841755)(237.27293428,284.18842059)
\lineto(237.27293428,284.32342059)
\curveto(237.28293367,284.37341733)(237.28793367,284.42341728)(237.28793428,284.47342059)
\curveto(237.30793365,284.57341713)(237.32293363,284.66841703)(237.33293428,284.75842059)
\curveto(237.34293361,284.85841684)(237.36293359,284.95341675)(237.39293428,285.04342059)
\curveto(237.44293351,285.19341651)(237.49793346,285.33341637)(237.55793428,285.46342059)
\curveto(237.61793334,285.59341611)(237.68793327,285.71341599)(237.76793428,285.82342059)
\curveto(237.79793316,285.87341583)(237.82793313,285.91341579)(237.85793428,285.94342059)
\curveto(237.89793306,285.97341573)(237.93293302,286.00841569)(237.96293428,286.04842059)
\curveto(238.02293293,286.12841557)(238.09293286,286.1984155)(238.17293428,286.25842059)
\curveto(238.23293272,286.30841539)(238.29293266,286.35341535)(238.35293428,286.39342059)
\lineto(238.56293428,286.54342059)
\curveto(238.61293234,286.58341512)(238.66293229,286.61841508)(238.71293428,286.64842059)
\curveto(238.76293219,286.68841501)(238.79793216,286.74341496)(238.81793428,286.81342059)
\curveto(238.81793214,286.84341486)(238.80793215,286.86841483)(238.78793428,286.88842059)
\curveto(238.77793218,286.91841478)(238.76793219,286.94341476)(238.75793428,286.96342059)
\curveto(238.71793224,287.01341469)(238.66793229,287.05841464)(238.60793428,287.09842059)
\curveto(238.5579324,287.14841455)(238.50793245,287.19341451)(238.45793428,287.23342059)
\curveto(238.41793254,287.26341444)(238.36793259,287.31841438)(238.30793428,287.39842059)
\curveto(238.28793267,287.42841427)(238.2579327,287.45341425)(238.21793428,287.47342059)
\curveto(238.18793277,287.5034142)(238.16293279,287.53841416)(238.14293428,287.57842059)
\curveto(237.97293298,287.78841391)(237.84293311,288.03341367)(237.75293428,288.31342059)
\curveto(237.73293322,288.39341331)(237.71793324,288.47341323)(237.70793428,288.55342059)
\curveto(237.69793326,288.63341307)(237.68293327,288.71341299)(237.66293428,288.79342059)
\curveto(237.64293331,288.84341286)(237.63293332,288.90841279)(237.63293428,288.98842059)
\curveto(237.63293332,289.07841262)(237.64293331,289.14841255)(237.66293428,289.19842059)
\curveto(237.66293329,289.2984124)(237.66793329,289.36841233)(237.67793428,289.40842059)
\curveto(237.69793326,289.48841221)(237.71293324,289.55841214)(237.72293428,289.61842059)
\curveto(237.73293322,289.68841201)(237.74793321,289.75841194)(237.76793428,289.82842059)
\curveto(237.91793304,290.25841144)(238.13293282,290.6034111)(238.41293428,290.86342059)
\curveto(238.70293225,291.12341058)(239.0529319,291.33841036)(239.46293428,291.50842059)
\curveto(239.57293138,291.55841014)(239.68793127,291.58841011)(239.80793428,291.59842059)
\curveto(239.93793102,291.61841008)(240.06793089,291.64841005)(240.19793428,291.68842059)
\curveto(240.27793068,291.68841001)(240.34793061,291.68841001)(240.40793428,291.68842059)
\curveto(240.47793048,291.69841)(240.5529304,291.70840999)(240.63293428,291.71842059)
\curveto(241.42292953,291.73840996)(242.07792888,291.60841009)(242.59793428,291.32842059)
\curveto(243.12792783,291.04841065)(243.50792745,290.63841106)(243.73793428,290.09842059)
\curveto(243.84792711,289.86841183)(243.91792704,289.58341212)(243.94793428,289.24342059)
\curveto(243.98792697,288.91341279)(243.957927,288.60841309)(243.85793428,288.32842059)
\curveto(243.81792714,288.1984135)(243.76792719,288.07841362)(243.70793428,287.96842059)
\curveto(243.6579273,287.85841384)(243.59792736,287.75341395)(243.52793428,287.65342059)
\curveto(243.50792745,287.61341409)(243.47792748,287.57841412)(243.43793428,287.54842059)
\lineto(243.34793428,287.45842059)
\curveto(243.29792766,287.36841433)(243.23792772,287.3034144)(243.16793428,287.26342059)
\curveto(243.11792784,287.21341449)(243.06292789,287.16341454)(243.00293428,287.11342059)
\curveto(242.952928,287.07341463)(242.90792805,287.02841467)(242.86793428,286.97842059)
\curveto(242.84792811,286.95841474)(242.82792813,286.93341477)(242.80793428,286.90342059)
\curveto(242.79792816,286.88341482)(242.79792816,286.85841484)(242.80793428,286.82842059)
\curveto(242.81792814,286.77841492)(242.84792811,286.72841497)(242.89793428,286.67842059)
\curveto(242.94792801,286.63841506)(243.00292795,286.5984151)(243.06293428,286.55842059)
\lineto(243.24293428,286.43842059)
\curveto(243.30292765,286.40841529)(243.3529276,286.37841532)(243.39293428,286.34842059)
\curveto(243.72292723,286.10841559)(243.97292698,285.7984159)(244.14293428,285.41842059)
\curveto(244.18292677,285.33841636)(244.21292674,285.25341645)(244.23293428,285.16342059)
\curveto(244.26292669,285.07341663)(244.28792667,284.98341672)(244.30793428,284.89342059)
\curveto(244.31792664,284.84341686)(244.32792663,284.78841691)(244.33793428,284.72842059)
\lineto(244.36793428,284.57842059)
\curveto(244.37792658,284.51841718)(244.37792658,284.45341725)(244.36793428,284.38342059)
\curveto(244.3579266,284.32341738)(244.36292659,284.26341744)(244.38293428,284.20342059)
\moveto(238.99793428,289.24342059)
\curveto(238.96793199,289.13341257)(238.96293199,288.99341271)(238.98293428,288.82342059)
\curveto(239.00293195,288.66341304)(239.02793193,288.53841316)(239.05793428,288.44842059)
\curveto(239.16793179,288.12841357)(239.31793164,287.88341382)(239.50793428,287.71342059)
\curveto(239.69793126,287.55341415)(239.96293099,287.42341428)(240.30293428,287.32342059)
\curveto(240.43293052,287.29341441)(240.59793036,287.26841443)(240.79793428,287.24842059)
\curveto(240.99792996,287.23841446)(241.16792979,287.25341445)(241.30793428,287.29342059)
\curveto(241.59792936,287.37341433)(241.83792912,287.48341422)(242.02793428,287.62342059)
\curveto(242.22792873,287.77341393)(242.38292857,287.97341373)(242.49293428,288.22342059)
\curveto(242.51292844,288.27341343)(242.52292843,288.31841338)(242.52293428,288.35842059)
\curveto(242.53292842,288.3984133)(242.54792841,288.44341326)(242.56793428,288.49342059)
\curveto(242.59792836,288.6034131)(242.61792834,288.74341296)(242.62793428,288.91342059)
\curveto(242.63792832,289.08341262)(242.62792833,289.22841247)(242.59793428,289.34842059)
\curveto(242.57792838,289.43841226)(242.5529284,289.52341218)(242.52293428,289.60342059)
\curveto(242.50292845,289.68341202)(242.46792849,289.76341194)(242.41793428,289.84342059)
\curveto(242.24792871,290.11341159)(242.02292893,290.30841139)(241.74293428,290.42842059)
\curveto(241.47292948,290.54841115)(241.11292984,290.60841109)(240.66293428,290.60842059)
\curveto(240.64293031,290.58841111)(240.61293034,290.58341112)(240.57293428,290.59342059)
\curveto(240.53293042,290.6034111)(240.49793046,290.6034111)(240.46793428,290.59342059)
\curveto(240.41793054,290.57341113)(240.36293059,290.55841114)(240.30293428,290.54842059)
\curveto(240.2529307,290.54841115)(240.20293075,290.53841116)(240.15293428,290.51842059)
\curveto(239.91293104,290.42841127)(239.70293125,290.31341139)(239.52293428,290.17342059)
\curveto(239.34293161,290.04341166)(239.20293175,289.86341184)(239.10293428,289.63342059)
\curveto(239.08293187,289.57341213)(239.06293189,289.50841219)(239.04293428,289.43842059)
\curveto(239.03293192,289.37841232)(239.01793194,289.31341239)(238.99793428,289.24342059)
\moveto(243.01793428,283.70842059)
\curveto(243.06792789,283.8984178)(243.07292788,284.1034176)(243.03293428,284.32342059)
\curveto(243.00292795,284.54341716)(242.957928,284.72341698)(242.89793428,284.86342059)
\curveto(242.72792823,285.23341647)(242.46792849,285.53841616)(242.11793428,285.77842059)
\curveto(241.77792918,286.01841568)(241.34292961,286.13841556)(240.81293428,286.13842059)
\curveto(240.78293017,286.11841558)(240.74293021,286.11341559)(240.69293428,286.12342059)
\curveto(240.64293031,286.14341556)(240.60293035,286.14841555)(240.57293428,286.13842059)
\lineto(240.30293428,286.07842059)
\curveto(240.22293073,286.06841563)(240.14293081,286.05341565)(240.06293428,286.03342059)
\curveto(239.76293119,285.92341578)(239.49793146,285.77841592)(239.26793428,285.59842059)
\curveto(239.04793191,285.41841628)(238.87793208,285.18841651)(238.75793428,284.90842059)
\curveto(238.72793223,284.82841687)(238.70293225,284.74841695)(238.68293428,284.66842059)
\curveto(238.66293229,284.58841711)(238.64293231,284.5034172)(238.62293428,284.41342059)
\curveto(238.59293236,284.29341741)(238.58293237,284.14341756)(238.59293428,283.96342059)
\curveto(238.61293234,283.78341792)(238.63793232,283.64341806)(238.66793428,283.54342059)
\curveto(238.68793227,283.49341821)(238.69793226,283.44841825)(238.69793428,283.40842059)
\curveto(238.70793225,283.37841832)(238.72293223,283.33841836)(238.74293428,283.28842059)
\curveto(238.84293211,283.06841863)(238.97293198,282.86841883)(239.13293428,282.68842059)
\curveto(239.30293165,282.50841919)(239.49793146,282.37341933)(239.71793428,282.28342059)
\curveto(239.78793117,282.24341946)(239.88293107,282.20841949)(240.00293428,282.17842059)
\curveto(240.22293073,282.08841961)(240.47793048,282.04341966)(240.76793428,282.04342059)
\lineto(241.05293428,282.04342059)
\curveto(241.1529298,282.06341964)(241.24792971,282.07841962)(241.33793428,282.08842059)
\curveto(241.42792953,282.0984196)(241.51792944,282.11841958)(241.60793428,282.14842059)
\curveto(241.86792909,282.22841947)(242.10792885,282.35841934)(242.32793428,282.53842059)
\curveto(242.5579284,282.72841897)(242.72792823,282.94341876)(242.83793428,283.18342059)
\curveto(242.87792808,283.26341844)(242.90792805,283.34341836)(242.92793428,283.42342059)
\curveto(242.957928,283.51341819)(242.98792797,283.60841809)(243.01793428,283.70842059)
}
}
{
\newrgbcolor{curcolor}{0 0 0}
\pscustom[linestyle=none,fillstyle=solid,fillcolor=curcolor]
{
\newpath
\moveto(255.52254366,289.64842059)
\curveto(255.32253336,289.35841234)(255.11253357,289.07341263)(254.89254366,288.79342059)
\curveto(254.682534,288.51341319)(254.4775342,288.22841347)(254.27754366,287.93842059)
\curveto(253.677535,287.08841461)(253.07253561,286.24841545)(252.46254366,285.41842059)
\curveto(251.85253683,284.5984171)(251.24753743,283.76341794)(250.64754366,282.91342059)
\lineto(250.13754366,282.19342059)
\lineto(249.62754366,281.50342059)
\curveto(249.54753913,281.39342031)(249.46753921,281.27842042)(249.38754366,281.15842059)
\curveto(249.30753937,281.03842066)(249.21253947,280.94342076)(249.10254366,280.87342059)
\curveto(249.06253962,280.85342085)(248.99753968,280.83842086)(248.90754366,280.82842059)
\curveto(248.82753985,280.80842089)(248.73753994,280.7984209)(248.63754366,280.79842059)
\curveto(248.53754014,280.7984209)(248.44254024,280.8034209)(248.35254366,280.81342059)
\curveto(248.27254041,280.82342088)(248.21254047,280.84342086)(248.17254366,280.87342059)
\curveto(248.14254054,280.89342081)(248.11754056,280.92842077)(248.09754366,280.97842059)
\curveto(248.08754059,281.01842068)(248.09254059,281.06342064)(248.11254366,281.11342059)
\curveto(248.15254053,281.19342051)(248.19754048,281.26842043)(248.24754366,281.33842059)
\curveto(248.30754037,281.41842028)(248.36254032,281.4984202)(248.41254366,281.57842059)
\curveto(248.65254003,281.91841978)(248.89753978,282.25341945)(249.14754366,282.58342059)
\curveto(249.39753928,282.91341879)(249.63753904,283.24841845)(249.86754366,283.58842059)
\curveto(250.02753865,283.80841789)(250.18753849,284.02341768)(250.34754366,284.23342059)
\curveto(250.50753817,284.44341726)(250.66753801,284.65841704)(250.82754366,284.87842059)
\curveto(251.18753749,285.3984163)(251.55253713,285.90841579)(251.92254366,286.40842059)
\curveto(252.29253639,286.90841479)(252.66253602,287.41841428)(253.03254366,287.93842059)
\curveto(253.17253551,288.13841356)(253.31253537,288.33341337)(253.45254366,288.52342059)
\curveto(253.60253508,288.71341299)(253.74753493,288.90841279)(253.88754366,289.10842059)
\curveto(254.09753458,289.40841229)(254.31253437,289.70841199)(254.53254366,290.00842059)
\lineto(255.19254366,290.90842059)
\lineto(255.37254366,291.17842059)
\lineto(255.58254366,291.44842059)
\lineto(255.70254366,291.62842059)
\curveto(255.75253293,291.68841001)(255.80253288,291.74340996)(255.85254366,291.79342059)
\curveto(255.92253276,291.84340986)(255.99753268,291.87840982)(256.07754366,291.89842059)
\curveto(256.09753258,291.90840979)(256.12253256,291.90840979)(256.15254366,291.89842059)
\curveto(256.19253249,291.8984098)(256.22253246,291.90840979)(256.24254366,291.92842059)
\curveto(256.36253232,291.92840977)(256.49753218,291.92340978)(256.64754366,291.91342059)
\curveto(256.79753188,291.91340979)(256.88753179,291.86840983)(256.91754366,291.77842059)
\curveto(256.93753174,291.74840995)(256.94253174,291.71340999)(256.93254366,291.67342059)
\curveto(256.92253176,291.63341007)(256.90753177,291.6034101)(256.88754366,291.58342059)
\curveto(256.84753183,291.5034102)(256.80753187,291.43341027)(256.76754366,291.37342059)
\curveto(256.72753195,291.31341039)(256.682532,291.25341045)(256.63254366,291.19342059)
\lineto(256.06254366,290.41342059)
\curveto(255.8825328,290.16341154)(255.70253298,289.90841179)(255.52254366,289.64842059)
\moveto(248.66754366,285.74842059)
\curveto(248.61754006,285.76841593)(248.56754011,285.77341593)(248.51754366,285.76342059)
\curveto(248.46754021,285.75341595)(248.41754026,285.75841594)(248.36754366,285.77842059)
\curveto(248.25754042,285.7984159)(248.15254053,285.81841588)(248.05254366,285.83842059)
\curveto(247.96254072,285.86841583)(247.86754081,285.90841579)(247.76754366,285.95842059)
\curveto(247.43754124,286.0984156)(247.1825415,286.29341541)(247.00254366,286.54342059)
\curveto(246.82254186,286.8034149)(246.677542,287.11341459)(246.56754366,287.47342059)
\curveto(246.53754214,287.55341415)(246.51754216,287.63341407)(246.50754366,287.71342059)
\curveto(246.49754218,287.8034139)(246.4825422,287.88841381)(246.46254366,287.96842059)
\curveto(246.45254223,288.01841368)(246.44754223,288.08341362)(246.44754366,288.16342059)
\curveto(246.43754224,288.19341351)(246.43254225,288.22341348)(246.43254366,288.25342059)
\curveto(246.43254225,288.29341341)(246.42754225,288.32841337)(246.41754366,288.35842059)
\lineto(246.41754366,288.50842059)
\curveto(246.40754227,288.55841314)(246.40254228,288.61841308)(246.40254366,288.68842059)
\curveto(246.40254228,288.76841293)(246.40754227,288.83341287)(246.41754366,288.88342059)
\lineto(246.41754366,289.04842059)
\curveto(246.43754224,289.0984126)(246.44254224,289.14341256)(246.43254366,289.18342059)
\curveto(246.43254225,289.23341247)(246.43754224,289.27841242)(246.44754366,289.31842059)
\curveto(246.45754222,289.35841234)(246.46254222,289.39341231)(246.46254366,289.42342059)
\curveto(246.46254222,289.46341224)(246.46754221,289.5034122)(246.47754366,289.54342059)
\curveto(246.50754217,289.65341205)(246.52754215,289.76341194)(246.53754366,289.87342059)
\curveto(246.55754212,289.99341171)(246.59254209,290.10841159)(246.64254366,290.21842059)
\curveto(246.7825419,290.55841114)(246.94254174,290.83341087)(247.12254366,291.04342059)
\curveto(247.31254137,291.26341044)(247.5825411,291.44341026)(247.93254366,291.58342059)
\curveto(248.01254067,291.61341009)(248.09754058,291.63341007)(248.18754366,291.64342059)
\curveto(248.2775404,291.66341004)(248.37254031,291.68341002)(248.47254366,291.70342059)
\curveto(248.50254018,291.71340999)(248.55754012,291.71340999)(248.63754366,291.70342059)
\curveto(248.71753996,291.70341)(248.76753991,291.71340999)(248.78754366,291.73342059)
\curveto(249.34753933,291.74340996)(249.79753888,291.63341007)(250.13754366,291.40342059)
\curveto(250.48753819,291.17341053)(250.74753793,290.86841083)(250.91754366,290.48842059)
\curveto(250.95753772,290.3984113)(250.99253769,290.3034114)(251.02254366,290.20342059)
\curveto(251.05253763,290.1034116)(251.0775376,290.0034117)(251.09754366,289.90342059)
\curveto(251.11753756,289.87341183)(251.12253756,289.84341186)(251.11254366,289.81342059)
\curveto(251.11253757,289.78341192)(251.11753756,289.75341195)(251.12754366,289.72342059)
\curveto(251.15753752,289.61341209)(251.1775375,289.48841221)(251.18754366,289.34842059)
\curveto(251.19753748,289.21841248)(251.20753747,289.08341262)(251.21754366,288.94342059)
\lineto(251.21754366,288.77842059)
\curveto(251.22753745,288.71841298)(251.22753745,288.66341304)(251.21754366,288.61342059)
\curveto(251.20753747,288.56341314)(251.20253748,288.51341319)(251.20254366,288.46342059)
\lineto(251.20254366,288.32842059)
\curveto(251.19253749,288.28841341)(251.18753749,288.24841345)(251.18754366,288.20842059)
\curveto(251.19753748,288.16841353)(251.19253749,288.12341358)(251.17254366,288.07342059)
\curveto(251.15253753,287.96341374)(251.13253755,287.85841384)(251.11254366,287.75842059)
\curveto(251.10253758,287.65841404)(251.0825376,287.55841414)(251.05254366,287.45842059)
\curveto(250.92253776,287.0984146)(250.75753792,286.78341492)(250.55754366,286.51342059)
\curveto(250.35753832,286.24341546)(250.0825386,286.03841566)(249.73254366,285.89842059)
\curveto(249.65253903,285.86841583)(249.56753911,285.84341586)(249.47754366,285.82342059)
\lineto(249.20754366,285.76342059)
\curveto(249.15753952,285.75341595)(249.11253957,285.74841595)(249.07254366,285.74842059)
\curveto(249.03253965,285.75841594)(248.99253969,285.75841594)(248.95254366,285.74842059)
\curveto(248.85253983,285.72841597)(248.75753992,285.72841597)(248.66754366,285.74842059)
\moveto(247.82754366,287.14342059)
\curveto(247.86754081,287.07341463)(247.90754077,287.00841469)(247.94754366,286.94842059)
\curveto(247.98754069,286.8984148)(248.03754064,286.84841485)(248.09754366,286.79842059)
\lineto(248.24754366,286.67842059)
\curveto(248.30754037,286.64841505)(248.37254031,286.62341508)(248.44254366,286.60342059)
\curveto(248.4825402,286.58341512)(248.51754016,286.57341513)(248.54754366,286.57342059)
\curveto(248.58754009,286.58341512)(248.62754005,286.57841512)(248.66754366,286.55842059)
\curveto(248.69753998,286.55841514)(248.73753994,286.55341515)(248.78754366,286.54342059)
\curveto(248.83753984,286.54341516)(248.8775398,286.54841515)(248.90754366,286.55842059)
\lineto(249.13254366,286.60342059)
\curveto(249.3825393,286.68341502)(249.56753911,286.80841489)(249.68754366,286.97842059)
\curveto(249.76753891,287.07841462)(249.83753884,287.20841449)(249.89754366,287.36842059)
\curveto(249.9775387,287.54841415)(250.03753864,287.77341393)(250.07754366,288.04342059)
\curveto(250.11753856,288.32341338)(250.13253855,288.6034131)(250.12254366,288.88342059)
\curveto(250.11253857,289.17341253)(250.0825386,289.44841225)(250.03254366,289.70842059)
\curveto(249.9825387,289.96841173)(249.90753877,290.17841152)(249.80754366,290.33842059)
\curveto(249.68753899,290.53841116)(249.53753914,290.68841101)(249.35754366,290.78842059)
\curveto(249.2775394,290.83841086)(249.18753949,290.86841083)(249.08754366,290.87842059)
\curveto(248.98753969,290.8984108)(248.8825398,290.90841079)(248.77254366,290.90842059)
\curveto(248.75253993,290.8984108)(248.72753995,290.89341081)(248.69754366,290.89342059)
\curveto(248.67754,290.9034108)(248.65754002,290.9034108)(248.63754366,290.89342059)
\curveto(248.58754009,290.88341082)(248.54254014,290.87341083)(248.50254366,290.86342059)
\curveto(248.46254022,290.86341084)(248.42254026,290.85341085)(248.38254366,290.83342059)
\curveto(248.20254048,290.75341095)(248.05254063,290.63341107)(247.93254366,290.47342059)
\curveto(247.82254086,290.31341139)(247.73254095,290.13341157)(247.66254366,289.93342059)
\curveto(247.60254108,289.74341196)(247.55754112,289.51841218)(247.52754366,289.25842059)
\curveto(247.50754117,288.9984127)(247.50254118,288.73341297)(247.51254366,288.46342059)
\curveto(247.52254116,288.2034135)(247.55254113,287.95341375)(247.60254366,287.71342059)
\curveto(247.66254102,287.48341422)(247.73754094,287.29341441)(247.82754366,287.14342059)
\moveto(258.62754366,284.15842059)
\curveto(258.63753004,284.10841759)(258.64253004,284.01841768)(258.64254366,283.88842059)
\curveto(258.64253004,283.75841794)(258.63253005,283.66841803)(258.61254366,283.61842059)
\curveto(258.59253009,283.56841813)(258.58753009,283.51341819)(258.59754366,283.45342059)
\curveto(258.60753007,283.4034183)(258.60753007,283.35341835)(258.59754366,283.30342059)
\curveto(258.55753012,283.16341854)(258.52753015,283.02841867)(258.50754366,282.89842059)
\curveto(258.49753018,282.76841893)(258.46753021,282.64841905)(258.41754366,282.53842059)
\curveto(258.2775304,282.18841951)(258.11253057,281.89341981)(257.92254366,281.65342059)
\curveto(257.73253095,281.42342028)(257.46253122,281.23842046)(257.11254366,281.09842059)
\curveto(257.03253165,281.06842063)(256.94753173,281.04842065)(256.85754366,281.03842059)
\curveto(256.76753191,281.01842068)(256.682532,280.9984207)(256.60254366,280.97842059)
\curveto(256.55253213,280.96842073)(256.50253218,280.96342074)(256.45254366,280.96342059)
\curveto(256.40253228,280.96342074)(256.35253233,280.95842074)(256.30254366,280.94842059)
\curveto(256.27253241,280.93842076)(256.22253246,280.93842076)(256.15254366,280.94842059)
\curveto(256.0825326,280.94842075)(256.03253265,280.95342075)(256.00254366,280.96342059)
\curveto(255.94253274,280.98342072)(255.8825328,280.99342071)(255.82254366,280.99342059)
\curveto(255.77253291,280.98342072)(255.72253296,280.98842071)(255.67254366,281.00842059)
\curveto(255.5825331,281.02842067)(255.49253319,281.05342065)(255.40254366,281.08342059)
\curveto(255.32253336,281.1034206)(255.24253344,281.13342057)(255.16254366,281.17342059)
\curveto(254.84253384,281.31342039)(254.59253409,281.50842019)(254.41254366,281.75842059)
\curveto(254.23253445,282.01841968)(254.0825346,282.32341938)(253.96254366,282.67342059)
\curveto(253.94253474,282.75341895)(253.92753475,282.83841886)(253.91754366,282.92842059)
\curveto(253.90753477,283.01841868)(253.89253479,283.1034186)(253.87254366,283.18342059)
\curveto(253.86253482,283.21341849)(253.85753482,283.24341846)(253.85754366,283.27342059)
\lineto(253.85754366,283.37842059)
\curveto(253.83753484,283.45841824)(253.82753485,283.53841816)(253.82754366,283.61842059)
\lineto(253.82754366,283.75342059)
\curveto(253.80753487,283.85341785)(253.80753487,283.95341775)(253.82754366,284.05342059)
\lineto(253.82754366,284.23342059)
\curveto(253.83753484,284.28341742)(253.84253484,284.32841737)(253.84254366,284.36842059)
\curveto(253.84253484,284.41841728)(253.84753483,284.46341724)(253.85754366,284.50342059)
\curveto(253.86753481,284.54341716)(253.87253481,284.57841712)(253.87254366,284.60842059)
\curveto(253.87253481,284.64841705)(253.8775348,284.68841701)(253.88754366,284.72842059)
\lineto(253.94754366,285.05842059)
\curveto(253.96753471,285.17841652)(253.99753468,285.28841641)(254.03754366,285.38842059)
\curveto(254.1775345,285.71841598)(254.33753434,285.99341571)(254.51754366,286.21342059)
\curveto(254.70753397,286.44341526)(254.96753371,286.62841507)(255.29754366,286.76842059)
\curveto(255.3775333,286.80841489)(255.46253322,286.83341487)(255.55254366,286.84342059)
\lineto(255.85254366,286.90342059)
\lineto(255.98754366,286.90342059)
\curveto(256.03753264,286.91341479)(256.08753259,286.91841478)(256.13754366,286.91842059)
\curveto(256.70753197,286.93841476)(257.16753151,286.83341487)(257.51754366,286.60342059)
\curveto(257.8775308,286.38341532)(258.14253054,286.08341562)(258.31254366,285.70342059)
\curveto(258.36253032,285.6034161)(258.40253028,285.5034162)(258.43254366,285.40342059)
\curveto(258.46253022,285.3034164)(258.49253019,285.1984165)(258.52254366,285.08842059)
\curveto(258.53253015,285.04841665)(258.53753014,285.01341669)(258.53754366,284.98342059)
\curveto(258.53753014,284.96341674)(258.54253014,284.93341677)(258.55254366,284.89342059)
\curveto(258.57253011,284.82341688)(258.5825301,284.74841695)(258.58254366,284.66842059)
\curveto(258.5825301,284.58841711)(258.59253009,284.50841719)(258.61254366,284.42842059)
\curveto(258.61253007,284.37841732)(258.61253007,284.33341737)(258.61254366,284.29342059)
\curveto(258.61253007,284.25341745)(258.61753006,284.20841749)(258.62754366,284.15842059)
\moveto(257.51754366,283.72342059)
\curveto(257.52753115,283.77341793)(257.53253115,283.84841785)(257.53254366,283.94842059)
\curveto(257.54253114,284.04841765)(257.53753114,284.12341758)(257.51754366,284.17342059)
\curveto(257.49753118,284.23341747)(257.49253119,284.28841741)(257.50254366,284.33842059)
\curveto(257.52253116,284.3984173)(257.52253116,284.45841724)(257.50254366,284.51842059)
\curveto(257.49253119,284.54841715)(257.48753119,284.58341712)(257.48754366,284.62342059)
\curveto(257.48753119,284.66341704)(257.4825312,284.703417)(257.47254366,284.74342059)
\curveto(257.45253123,284.82341688)(257.43253125,284.8984168)(257.41254366,284.96842059)
\curveto(257.40253128,285.04841665)(257.38753129,285.12841657)(257.36754366,285.20842059)
\curveto(257.33753134,285.26841643)(257.31253137,285.32841637)(257.29254366,285.38842059)
\curveto(257.27253141,285.44841625)(257.24253144,285.50841619)(257.20254366,285.56842059)
\curveto(257.10253158,285.73841596)(256.97253171,285.87341583)(256.81254366,285.97342059)
\curveto(256.73253195,286.02341568)(256.63753204,286.05841564)(256.52754366,286.07842059)
\curveto(256.41753226,286.0984156)(256.29253239,286.10841559)(256.15254366,286.10842059)
\curveto(256.13253255,286.0984156)(256.10753257,286.09341561)(256.07754366,286.09342059)
\curveto(256.04753263,286.1034156)(256.01753266,286.1034156)(255.98754366,286.09342059)
\lineto(255.83754366,286.03342059)
\curveto(255.78753289,286.02341568)(255.74253294,286.00841569)(255.70254366,285.98842059)
\curveto(255.51253317,285.87841582)(255.36753331,285.73341597)(255.26754366,285.55342059)
\curveto(255.1775335,285.37341633)(255.09753358,285.16841653)(255.02754366,284.93842059)
\curveto(254.98753369,284.80841689)(254.96753371,284.67341703)(254.96754366,284.53342059)
\curveto(254.96753371,284.4034173)(254.95753372,284.25841744)(254.93754366,284.09842059)
\curveto(254.92753375,284.04841765)(254.91753376,283.98841771)(254.90754366,283.91842059)
\curveto(254.90753377,283.84841785)(254.91753376,283.78841791)(254.93754366,283.73842059)
\lineto(254.93754366,283.57342059)
\lineto(254.93754366,283.39342059)
\curveto(254.94753373,283.34341836)(254.95753372,283.28841841)(254.96754366,283.22842059)
\curveto(254.9775337,283.17841852)(254.9825337,283.12341858)(254.98254366,283.06342059)
\curveto(254.99253369,283.0034187)(255.00753367,282.94841875)(255.02754366,282.89842059)
\curveto(255.0775336,282.70841899)(255.13753354,282.53341917)(255.20754366,282.37342059)
\curveto(255.2775334,282.21341949)(255.3825333,282.08341962)(255.52254366,281.98342059)
\curveto(255.65253303,281.88341982)(255.79253289,281.81341989)(255.94254366,281.77342059)
\curveto(255.97253271,281.76341994)(255.99753268,281.75841994)(256.01754366,281.75842059)
\curveto(256.04753263,281.76841993)(256.0775326,281.76841993)(256.10754366,281.75842059)
\curveto(256.12753255,281.75841994)(256.15753252,281.75341995)(256.19754366,281.74342059)
\curveto(256.23753244,281.74341996)(256.27253241,281.74841995)(256.30254366,281.75842059)
\curveto(256.34253234,281.76841993)(256.3825323,281.77341993)(256.42254366,281.77342059)
\curveto(256.46253222,281.77341993)(256.50253218,281.78341992)(256.54254366,281.80342059)
\curveto(256.7825319,281.88341982)(256.9775317,282.01841968)(257.12754366,282.20842059)
\curveto(257.24753143,282.38841931)(257.33753134,282.59341911)(257.39754366,282.82342059)
\curveto(257.41753126,282.89341881)(257.43253125,282.96341874)(257.44254366,283.03342059)
\curveto(257.45253123,283.11341859)(257.46753121,283.19341851)(257.48754366,283.27342059)
\curveto(257.48753119,283.33341837)(257.49253119,283.37841832)(257.50254366,283.40842059)
\curveto(257.50253118,283.42841827)(257.50253118,283.45341825)(257.50254366,283.48342059)
\curveto(257.50253118,283.52341818)(257.50753117,283.55341815)(257.51754366,283.57342059)
\lineto(257.51754366,283.72342059)
}
}
{
\newrgbcolor{curcolor}{0 0 0}
\pscustom[linestyle=none,fillstyle=solid,fillcolor=curcolor]
{
\newpath
\moveto(306.87738985,120.9786452)
\curveto(306.9473822,120.92864174)(306.98738216,120.85864181)(306.99738985,120.7686452)
\curveto(307.01738213,120.67864199)(307.02738212,120.57364209)(307.02738985,120.4536452)
\curveto(307.02738212,120.40364226)(307.02238213,120.35364231)(307.01238985,120.3036452)
\curveto(307.01238214,120.25364241)(307.00238215,120.20864246)(306.98238985,120.1686452)
\curveto(306.9523822,120.07864259)(306.89238226,120.01864265)(306.80238985,119.9886452)
\curveto(306.72238243,119.9686427)(306.62738252,119.95864271)(306.51738985,119.9586452)
\lineto(306.20238985,119.9586452)
\curveto(306.09238306,119.9686427)(305.98738316,119.95864271)(305.88738985,119.9286452)
\curveto(305.7473834,119.89864277)(305.65738349,119.81864285)(305.61738985,119.6886452)
\curveto(305.59738355,119.61864305)(305.58738356,119.53364313)(305.58738985,119.4336452)
\lineto(305.58738985,119.1636452)
\lineto(305.58738985,118.2186452)
\lineto(305.58738985,117.8886452)
\curveto(305.58738356,117.77864489)(305.56738358,117.69364497)(305.52738985,117.6336452)
\curveto(305.48738366,117.57364509)(305.43738371,117.53364513)(305.37738985,117.5136452)
\curveto(305.32738382,117.50364516)(305.26238389,117.48864518)(305.18238985,117.4686452)
\lineto(304.98738985,117.4686452)
\curveto(304.86738428,117.4686452)(304.76238439,117.47364519)(304.67238985,117.4836452)
\curveto(304.58238457,117.50364516)(304.51238464,117.55364511)(304.46238985,117.6336452)
\curveto(304.43238472,117.68364498)(304.41738473,117.75364491)(304.41738985,117.8436452)
\lineto(304.41738985,118.1436452)
\lineto(304.41738985,119.1786452)
\curveto(304.41738473,119.33864333)(304.40738474,119.48364318)(304.38738985,119.6136452)
\curveto(304.37738477,119.75364291)(304.32238483,119.84864282)(304.22238985,119.8986452)
\curveto(304.17238498,119.91864275)(304.10238505,119.93364273)(304.01238985,119.9436452)
\curveto(303.93238522,119.95364271)(303.84238531,119.95864271)(303.74238985,119.9586452)
\lineto(303.45738985,119.9586452)
\lineto(303.21738985,119.9586452)
\lineto(300.95238985,119.9586452)
\curveto(300.86238829,119.95864271)(300.75738839,119.95364271)(300.63738985,119.9436452)
\lineto(300.30738985,119.9436452)
\curveto(300.19738895,119.94364272)(300.09738905,119.95364271)(300.00738985,119.9736452)
\curveto(299.91738923,119.99364267)(299.85738929,120.02864264)(299.82738985,120.0786452)
\curveto(299.77738937,120.14864252)(299.7523894,120.24364242)(299.75238985,120.3636452)
\lineto(299.75238985,120.7086452)
\lineto(299.75238985,120.9786452)
\curveto(299.79238936,121.14864152)(299.8473893,121.28864138)(299.91738985,121.3986452)
\curveto(299.98738916,121.50864116)(300.06738908,121.62364104)(300.15738985,121.7436452)
\lineto(300.51738985,122.2836452)
\curveto(300.95738819,122.91363975)(301.39238776,123.53363913)(301.82238985,124.1436452)
\lineto(303.14238985,126.0036452)
\curveto(303.30238585,126.23363643)(303.45738569,126.45363621)(303.60738985,126.6636452)
\curveto(303.75738539,126.88363578)(303.91238524,127.10863556)(304.07238985,127.3386452)
\curveto(304.12238503,127.40863526)(304.17238498,127.47363519)(304.22238985,127.5336452)
\curveto(304.27238488,127.60363506)(304.32238483,127.67863499)(304.37238985,127.7586452)
\lineto(304.43238985,127.8486452)
\curveto(304.46238469,127.88863478)(304.49238466,127.91863475)(304.52238985,127.9386452)
\curveto(304.56238459,127.9686347)(304.60238455,127.98863468)(304.64238985,127.9986452)
\curveto(304.68238447,128.01863465)(304.72738442,128.03863463)(304.77738985,128.0586452)
\curveto(304.79738435,128.05863461)(304.81738433,128.05363461)(304.83738985,128.0436452)
\curveto(304.86738428,128.04363462)(304.89238426,128.05363461)(304.91238985,128.0736452)
\curveto(305.04238411,128.07363459)(305.16238399,128.0686346)(305.27238985,128.0586452)
\curveto(305.38238377,128.04863462)(305.46238369,128.00363466)(305.51238985,127.9236452)
\curveto(305.5523836,127.87363479)(305.57238358,127.80363486)(305.57238985,127.7136452)
\curveto(305.58238357,127.62363504)(305.58738356,127.52863514)(305.58738985,127.4286452)
\lineto(305.58738985,121.9686452)
\curveto(305.58738356,121.89864077)(305.58238357,121.82364084)(305.57238985,121.7436452)
\curveto(305.57238358,121.67364099)(305.57738357,121.60364106)(305.58738985,121.5336452)
\lineto(305.58738985,121.4286452)
\curveto(305.60738354,121.37864129)(305.62238353,121.32364134)(305.63238985,121.2636452)
\curveto(305.64238351,121.21364145)(305.66738348,121.17364149)(305.70738985,121.1436452)
\curveto(305.77738337,121.09364157)(305.86238329,121.0636416)(305.96238985,121.0536452)
\lineto(306.29238985,121.0536452)
\curveto(306.40238275,121.05364161)(306.50738264,121.04864162)(306.60738985,121.0386452)
\curveto(306.71738243,121.03864163)(306.80738234,121.01864165)(306.87738985,120.9786452)
\moveto(304.31238985,121.1736452)
\curveto(304.39238476,121.28364138)(304.42738472,121.45364121)(304.41738985,121.6836452)
\lineto(304.41738985,122.2986452)
\lineto(304.41738985,124.7736452)
\lineto(304.41738985,125.0886452)
\curveto(304.42738472,125.20863746)(304.42238473,125.30863736)(304.40238985,125.3886452)
\lineto(304.40238985,125.5386452)
\curveto(304.40238475,125.62863704)(304.38738476,125.71363695)(304.35738985,125.7936452)
\curveto(304.3473848,125.81363685)(304.33738481,125.82363684)(304.32738985,125.8236452)
\lineto(304.28238985,125.8686452)
\curveto(304.26238489,125.87863679)(304.23238492,125.88363678)(304.19238985,125.8836452)
\curveto(304.17238498,125.8636368)(304.152385,125.84863682)(304.13238985,125.8386452)
\curveto(304.12238503,125.83863683)(304.10738504,125.83363683)(304.08738985,125.8236452)
\curveto(304.02738512,125.77363689)(303.96738518,125.70363696)(303.90738985,125.6136452)
\curveto(303.8473853,125.52363714)(303.79238536,125.44363722)(303.74238985,125.3736452)
\curveto(303.64238551,125.23363743)(303.5473856,125.08863758)(303.45738985,124.9386452)
\curveto(303.36738578,124.79863787)(303.27238588,124.65863801)(303.17238985,124.5186452)
\lineto(302.63238985,123.7386452)
\curveto(302.46238669,123.47863919)(302.28738686,123.21863945)(302.10738985,122.9586452)
\curveto(302.02738712,122.84863982)(301.9523872,122.74363992)(301.88238985,122.6436452)
\lineto(301.67238985,122.3436452)
\curveto(301.62238753,122.2636404)(301.57238758,122.18864048)(301.52238985,122.1186452)
\curveto(301.48238767,122.04864062)(301.43738771,121.97364069)(301.38738985,121.8936452)
\curveto(301.33738781,121.83364083)(301.28738786,121.7686409)(301.23738985,121.6986452)
\curveto(301.19738795,121.63864103)(301.15738799,121.5686411)(301.11738985,121.4886452)
\curveto(301.07738807,121.42864124)(301.0523881,121.35864131)(301.04238985,121.2786452)
\curveto(301.03238812,121.20864146)(301.06738808,121.15364151)(301.14738985,121.1136452)
\curveto(301.21738793,121.0636416)(301.32738782,121.03864163)(301.47738985,121.0386452)
\curveto(301.63738751,121.04864162)(301.77238738,121.05364161)(301.88238985,121.0536452)
\lineto(303.56238985,121.0536452)
\lineto(303.99738985,121.0536452)
\curveto(304.147385,121.05364161)(304.2523849,121.09364157)(304.31238985,121.1736452)
}
}
{
\newrgbcolor{curcolor}{0 0 0}
\pscustom[linestyle=none,fillstyle=solid,fillcolor=curcolor]
{
\newpath
\moveto(308.80699922,127.8936452)
\lineto(313.60699922,127.8936452)
\lineto(314.61199922,127.8936452)
\curveto(314.75199212,127.89363477)(314.871992,127.88363478)(314.97199922,127.8636452)
\curveto(315.08199179,127.85363481)(315.16199171,127.80863486)(315.21199922,127.7286452)
\curveto(315.23199164,127.68863498)(315.24199163,127.63863503)(315.24199922,127.5786452)
\curveto(315.25199162,127.51863515)(315.25699162,127.45363521)(315.25699922,127.3836452)
\lineto(315.25699922,127.1136452)
\curveto(315.25699162,127.02363564)(315.24699163,126.94363572)(315.22699922,126.8736452)
\curveto(315.18699169,126.79363587)(315.14199173,126.72363594)(315.09199922,126.6636452)
\lineto(314.94199922,126.4836452)
\curveto(314.91199196,126.43363623)(314.876992,126.39363627)(314.83699922,126.3636452)
\curveto(314.79699208,126.33363633)(314.75699212,126.29363637)(314.71699922,126.2436452)
\curveto(314.63699224,126.13363653)(314.55199232,126.02363664)(314.46199922,125.9136452)
\curveto(314.3719925,125.81363685)(314.28699259,125.70863696)(314.20699922,125.5986452)
\curveto(314.06699281,125.39863727)(313.92699295,125.18863748)(313.78699922,124.9686452)
\curveto(313.64699323,124.75863791)(313.50699337,124.54363812)(313.36699922,124.3236452)
\curveto(313.31699356,124.23363843)(313.26699361,124.13863853)(313.21699922,124.0386452)
\curveto(313.16699371,123.93863873)(313.11199376,123.84363882)(313.05199922,123.7536452)
\curveto(313.03199384,123.73363893)(313.02199385,123.70863896)(313.02199922,123.6786452)
\curveto(313.02199385,123.64863902)(313.01199386,123.62363904)(312.99199922,123.6036452)
\curveto(312.92199395,123.50363916)(312.85699402,123.38863928)(312.79699922,123.2586452)
\curveto(312.73699414,123.13863953)(312.68199419,123.02363964)(312.63199922,122.9136452)
\curveto(312.53199434,122.68363998)(312.43699444,122.44864022)(312.34699922,122.2086452)
\curveto(312.25699462,121.9686407)(312.15699472,121.72864094)(312.04699922,121.4886452)
\curveto(312.02699485,121.43864123)(312.01199486,121.39364127)(312.00199922,121.3536452)
\curveto(312.00199487,121.31364135)(311.99199488,121.2686414)(311.97199922,121.2186452)
\curveto(311.92199495,121.09864157)(311.876995,120.97364169)(311.83699922,120.8436452)
\curveto(311.80699507,120.72364194)(311.7719951,120.60364206)(311.73199922,120.4836452)
\curveto(311.65199522,120.25364241)(311.58699529,120.01364265)(311.53699922,119.7636452)
\curveto(311.49699538,119.52364314)(311.44699543,119.28364338)(311.38699922,119.0436452)
\curveto(311.34699553,118.89364377)(311.32199555,118.74364392)(311.31199922,118.5936452)
\curveto(311.30199557,118.44364422)(311.28199559,118.29364437)(311.25199922,118.1436452)
\curveto(311.24199563,118.10364456)(311.23699564,118.04364462)(311.23699922,117.9636452)
\curveto(311.20699567,117.84364482)(311.1769957,117.74364492)(311.14699922,117.6636452)
\curveto(311.11699576,117.58364508)(311.04699583,117.52864514)(310.93699922,117.4986452)
\curveto(310.88699599,117.47864519)(310.83199604,117.4686452)(310.77199922,117.4686452)
\lineto(310.57699922,117.4686452)
\curveto(310.43699644,117.4686452)(310.29699658,117.47364519)(310.15699922,117.4836452)
\curveto(310.02699685,117.49364517)(309.93199694,117.53864513)(309.87199922,117.6186452)
\curveto(309.83199704,117.67864499)(309.81199706,117.7636449)(309.81199922,117.8736452)
\curveto(309.82199705,117.98364468)(309.83699704,118.07864459)(309.85699922,118.1586452)
\lineto(309.85699922,118.2336452)
\curveto(309.86699701,118.2636444)(309.871997,118.29364437)(309.87199922,118.3236452)
\curveto(309.89199698,118.40364426)(309.90199697,118.47864419)(309.90199922,118.5486452)
\curveto(309.90199697,118.61864405)(309.91199696,118.68864398)(309.93199922,118.7586452)
\curveto(309.98199689,118.94864372)(310.02199685,119.13364353)(310.05199922,119.3136452)
\curveto(310.08199679,119.50364316)(310.12199675,119.68364298)(310.17199922,119.8536452)
\curveto(310.19199668,119.90364276)(310.20199667,119.94364272)(310.20199922,119.9736452)
\curveto(310.20199667,120.00364266)(310.20699667,120.03864263)(310.21699922,120.0786452)
\curveto(310.31699656,120.37864229)(310.40699647,120.67364199)(310.48699922,120.9636452)
\curveto(310.5769963,121.25364141)(310.68199619,121.53364113)(310.80199922,121.8036452)
\curveto(311.06199581,122.38364028)(311.33199554,122.93363973)(311.61199922,123.4536452)
\curveto(311.89199498,123.98363868)(312.20199467,124.48863818)(312.54199922,124.9686452)
\curveto(312.68199419,125.1686375)(312.83199404,125.35863731)(312.99199922,125.5386452)
\curveto(313.15199372,125.72863694)(313.30199357,125.91863675)(313.44199922,126.1086452)
\curveto(313.48199339,126.15863651)(313.51699336,126.20363646)(313.54699922,126.2436452)
\curveto(313.58699329,126.29363637)(313.62199325,126.34363632)(313.65199922,126.3936452)
\curveto(313.66199321,126.41363625)(313.6719932,126.43863623)(313.68199922,126.4686452)
\curveto(313.70199317,126.49863617)(313.70199317,126.52863614)(313.68199922,126.5586452)
\curveto(313.66199321,126.61863605)(313.62699325,126.65363601)(313.57699922,126.6636452)
\curveto(313.52699335,126.68363598)(313.4769934,126.70363596)(313.42699922,126.7236452)
\lineto(313.32199922,126.7236452)
\curveto(313.28199359,126.73363593)(313.23199364,126.73363593)(313.17199922,126.7236452)
\lineto(313.02199922,126.7236452)
\lineto(312.42199922,126.7236452)
\lineto(309.78199922,126.7236452)
\lineto(309.04699922,126.7236452)
\lineto(308.80699922,126.7236452)
\curveto(308.73699814,126.73363593)(308.6769982,126.74863592)(308.62699922,126.7686452)
\curveto(308.53699834,126.80863586)(308.4769984,126.8686358)(308.44699922,126.9486452)
\curveto(308.39699848,127.04863562)(308.38199849,127.19363547)(308.40199922,127.3836452)
\curveto(308.42199845,127.58363508)(308.45699842,127.71863495)(308.50699922,127.7886452)
\curveto(308.52699835,127.80863486)(308.55199832,127.82363484)(308.58199922,127.8336452)
\lineto(308.70199922,127.8936452)
\curveto(308.72199815,127.89363477)(308.73699814,127.88863478)(308.74699922,127.8786452)
\curveto(308.76699811,127.87863479)(308.78699809,127.88363478)(308.80699922,127.8936452)
}
}
{
\newrgbcolor{curcolor}{0 0 0}
\pscustom[linestyle=none,fillstyle=solid,fillcolor=curcolor]
{
\newpath
\moveto(317.6516086,119.1186452)
\lineto(317.9516086,119.1186452)
\curveto(318.06160654,119.12864354)(318.16660643,119.12864354)(318.2666086,119.1186452)
\curveto(318.37660622,119.11864355)(318.47660612,119.10864356)(318.5666086,119.0886452)
\curveto(318.65660594,119.07864359)(318.72660587,119.05364361)(318.7766086,119.0136452)
\curveto(318.7966058,118.99364367)(318.81160579,118.9636437)(318.8216086,118.9236452)
\curveto(318.84160576,118.88364378)(318.86160574,118.83864383)(318.8816086,118.7886452)
\lineto(318.8816086,118.7136452)
\curveto(318.89160571,118.663644)(318.89160571,118.60864406)(318.8816086,118.5486452)
\lineto(318.8816086,118.3986452)
\lineto(318.8816086,117.9186452)
\curveto(318.88160572,117.74864492)(318.84160576,117.62864504)(318.7616086,117.5586452)
\curveto(318.69160591,117.50864516)(318.601606,117.48364518)(318.4916086,117.4836452)
\lineto(318.1616086,117.4836452)
\lineto(317.7116086,117.4836452)
\curveto(317.56160704,117.48364518)(317.44660715,117.51364515)(317.3666086,117.5736452)
\curveto(317.32660727,117.60364506)(317.2966073,117.65364501)(317.2766086,117.7236452)
\curveto(317.25660734,117.80364486)(317.24160736,117.88864478)(317.2316086,117.9786452)
\lineto(317.2316086,118.2636452)
\curveto(317.24160736,118.3636443)(317.24660735,118.44864422)(317.2466086,118.5186452)
\lineto(317.2466086,118.7136452)
\curveto(317.24660735,118.77364389)(317.25660734,118.82864384)(317.2766086,118.8786452)
\curveto(317.31660728,118.98864368)(317.38660721,119.05864361)(317.4866086,119.0886452)
\curveto(317.51660708,119.08864358)(317.57160703,119.09864357)(317.6516086,119.1186452)
}
}
{
\newrgbcolor{curcolor}{0 0 0}
\pscustom[linestyle=none,fillstyle=solid,fillcolor=curcolor]
{
\newpath
\moveto(323.97176485,128.0886452)
\curveto(325.60175941,128.11863455)(326.65175836,127.5636351)(327.12176485,126.4236452)
\curveto(327.22175779,126.19363647)(327.28675772,125.90363676)(327.31676485,125.5536452)
\curveto(327.35675765,125.21363745)(327.33175768,124.90363776)(327.24176485,124.6236452)
\curveto(327.15175786,124.3636383)(327.03175798,124.13863853)(326.88176485,123.9486452)
\curveto(326.86175815,123.90863876)(326.83675817,123.87363879)(326.80676485,123.8436452)
\curveto(326.77675823,123.82363884)(326.75175826,123.79863887)(326.73176485,123.7686452)
\lineto(326.64176485,123.6486452)
\curveto(326.6117584,123.61863905)(326.57675843,123.59363907)(326.53676485,123.5736452)
\curveto(326.48675852,123.52363914)(326.43175858,123.47863919)(326.37176485,123.4386452)
\curveto(326.32175869,123.39863927)(326.27675873,123.34863932)(326.23676485,123.2886452)
\curveto(326.19675881,123.24863942)(326.18175883,123.19863947)(326.19176485,123.1386452)
\curveto(326.20175881,123.08863958)(326.23175878,123.04363962)(326.28176485,123.0036452)
\curveto(326.33175868,122.9636397)(326.38675862,122.92363974)(326.44676485,122.8836452)
\curveto(326.51675849,122.85363981)(326.58175843,122.82363984)(326.64176485,122.7936452)
\curveto(326.70175831,122.7636399)(326.75175826,122.73363993)(326.79176485,122.7036452)
\curveto(327.1117579,122.48364018)(327.36675764,122.17364049)(327.55676485,121.7736452)
\curveto(327.59675741,121.68364098)(327.62675738,121.58864108)(327.64676485,121.4886452)
\curveto(327.67675733,121.39864127)(327.70175731,121.30864136)(327.72176485,121.2186452)
\curveto(327.73175728,121.1686415)(327.73675727,121.11864155)(327.73676485,121.0686452)
\curveto(327.74675726,121.02864164)(327.75675725,120.98364168)(327.76676485,120.9336452)
\curveto(327.77675723,120.88364178)(327.77675723,120.83364183)(327.76676485,120.7836452)
\curveto(327.75675725,120.73364193)(327.76175725,120.68364198)(327.78176485,120.6336452)
\curveto(327.79175722,120.58364208)(327.79675721,120.52364214)(327.79676485,120.4536452)
\curveto(327.79675721,120.38364228)(327.78675722,120.32364234)(327.76676485,120.2736452)
\lineto(327.76676485,120.0486452)
\lineto(327.70676485,119.8086452)
\curveto(327.69675731,119.73864293)(327.68175733,119.668643)(327.66176485,119.5986452)
\curveto(327.63175738,119.50864316)(327.60175741,119.42364324)(327.57176485,119.3436452)
\curveto(327.55175746,119.2636434)(327.52175749,119.18364348)(327.48176485,119.1036452)
\curveto(327.46175755,119.04364362)(327.43175758,118.98364368)(327.39176485,118.9236452)
\curveto(327.36175765,118.87364379)(327.32675768,118.82364384)(327.28676485,118.7736452)
\curveto(327.08675792,118.4636442)(326.83675817,118.20364446)(326.53676485,117.9936452)
\curveto(326.23675877,117.79364487)(325.89175912,117.62864504)(325.50176485,117.4986452)
\curveto(325.38175963,117.45864521)(325.25175976,117.43364523)(325.11176485,117.4236452)
\curveto(324.98176003,117.40364526)(324.84676016,117.37864529)(324.70676485,117.3486452)
\curveto(324.63676037,117.33864533)(324.56676044,117.33364533)(324.49676485,117.3336452)
\curveto(324.43676057,117.33364533)(324.37176064,117.32864534)(324.30176485,117.3186452)
\curveto(324.26176075,117.30864536)(324.20176081,117.30364536)(324.12176485,117.3036452)
\curveto(324.05176096,117.30364536)(324.00176101,117.30864536)(323.97176485,117.3186452)
\curveto(323.92176109,117.32864534)(323.87676113,117.33364533)(323.83676485,117.3336452)
\lineto(323.71676485,117.3336452)
\curveto(323.61676139,117.35364531)(323.51676149,117.3686453)(323.41676485,117.3786452)
\curveto(323.31676169,117.38864528)(323.22176179,117.40364526)(323.13176485,117.4236452)
\curveto(323.02176199,117.45364521)(322.9117621,117.47864519)(322.80176485,117.4986452)
\curveto(322.70176231,117.52864514)(322.59676241,117.5686451)(322.48676485,117.6186452)
\curveto(322.11676289,117.77864489)(321.80176321,117.97864469)(321.54176485,118.2186452)
\curveto(321.28176373,118.4686442)(321.07176394,118.77864389)(320.91176485,119.1486452)
\curveto(320.87176414,119.23864343)(320.83676417,119.33364333)(320.80676485,119.4336452)
\curveto(320.77676423,119.53364313)(320.74676426,119.63864303)(320.71676485,119.7486452)
\curveto(320.69676431,119.79864287)(320.68676432,119.84864282)(320.68676485,119.8986452)
\curveto(320.68676432,119.95864271)(320.67676433,120.01864265)(320.65676485,120.0786452)
\curveto(320.63676437,120.13864253)(320.62676438,120.21864245)(320.62676485,120.3186452)
\curveto(320.62676438,120.41864225)(320.64176437,120.49364217)(320.67176485,120.5436452)
\curveto(320.68176433,120.57364209)(320.69676431,120.59864207)(320.71676485,120.6186452)
\lineto(320.77676485,120.6786452)
\curveto(320.81676419,120.69864197)(320.87676413,120.71364195)(320.95676485,120.7236452)
\curveto(321.04676396,120.73364193)(321.13676387,120.73864193)(321.22676485,120.7386452)
\curveto(321.31676369,120.73864193)(321.40176361,120.73364193)(321.48176485,120.7236452)
\curveto(321.57176344,120.71364195)(321.63676337,120.70364196)(321.67676485,120.6936452)
\curveto(321.69676331,120.67364199)(321.71676329,120.65864201)(321.73676485,120.6486452)
\curveto(321.75676325,120.64864202)(321.77676323,120.63864203)(321.79676485,120.6186452)
\curveto(321.86676314,120.52864214)(321.9067631,120.41364225)(321.91676485,120.2736452)
\curveto(321.93676307,120.13364253)(321.96676304,120.00864266)(322.00676485,119.8986452)
\lineto(322.15676485,119.5386452)
\curveto(322.2067628,119.42864324)(322.27176274,119.32364334)(322.35176485,119.2236452)
\curveto(322.37176264,119.19364347)(322.39176262,119.1686435)(322.41176485,119.1486452)
\curveto(322.44176257,119.12864354)(322.46676254,119.10364356)(322.48676485,119.0736452)
\curveto(322.52676248,119.01364365)(322.56176245,118.9686437)(322.59176485,118.9386452)
\curveto(322.63176238,118.90864376)(322.66676234,118.87864379)(322.69676485,118.8486452)
\curveto(322.73676227,118.81864385)(322.78176223,118.78864388)(322.83176485,118.7586452)
\curveto(322.92176209,118.69864397)(323.01676199,118.64864402)(323.11676485,118.6086452)
\lineto(323.44676485,118.4886452)
\curveto(323.59676141,118.43864423)(323.79676121,118.40864426)(324.04676485,118.3986452)
\curveto(324.29676071,118.38864428)(324.5067605,118.40864426)(324.67676485,118.4586452)
\curveto(324.75676025,118.47864419)(324.82676018,118.49364417)(324.88676485,118.5036452)
\lineto(325.09676485,118.5636452)
\curveto(325.37675963,118.68364398)(325.61675939,118.83364383)(325.81676485,119.0136452)
\curveto(326.02675898,119.19364347)(326.19175882,119.42364324)(326.31176485,119.7036452)
\curveto(326.34175867,119.77364289)(326.36175865,119.84364282)(326.37176485,119.9136452)
\lineto(326.43176485,120.1536452)
\curveto(326.47175854,120.29364237)(326.48175853,120.45364221)(326.46176485,120.6336452)
\curveto(326.44175857,120.82364184)(326.4117586,120.97364169)(326.37176485,121.0836452)
\curveto(326.24175877,121.4636412)(326.05675895,121.75364091)(325.81676485,121.9536452)
\curveto(325.58675942,122.15364051)(325.27675973,122.31364035)(324.88676485,122.4336452)
\curveto(324.77676023,122.4636402)(324.65676035,122.48364018)(324.52676485,122.4936452)
\curveto(324.4067606,122.50364016)(324.28176073,122.50864016)(324.15176485,122.5086452)
\curveto(323.99176102,122.50864016)(323.85176116,122.51364015)(323.73176485,122.5236452)
\curveto(323.6117614,122.53364013)(323.52676148,122.59364007)(323.47676485,122.7036452)
\curveto(323.45676155,122.73363993)(323.44676156,122.7686399)(323.44676485,122.8086452)
\lineto(323.44676485,122.9436452)
\curveto(323.43676157,123.04363962)(323.43676157,123.13863953)(323.44676485,123.2286452)
\curveto(323.46676154,123.31863935)(323.5067615,123.38363928)(323.56676485,123.4236452)
\curveto(323.6067614,123.45363921)(323.64676136,123.47363919)(323.68676485,123.4836452)
\curveto(323.73676127,123.49363917)(323.79176122,123.50363916)(323.85176485,123.5136452)
\curveto(323.87176114,123.52363914)(323.89676111,123.52363914)(323.92676485,123.5136452)
\curveto(323.95676105,123.51363915)(323.98176103,123.51863915)(324.00176485,123.5286452)
\lineto(324.13676485,123.5286452)
\curveto(324.24676076,123.54863912)(324.34676066,123.55863911)(324.43676485,123.5586452)
\curveto(324.53676047,123.5686391)(324.63176038,123.58863908)(324.72176485,123.6186452)
\curveto(325.04175997,123.72863894)(325.29675971,123.87363879)(325.48676485,124.0536452)
\curveto(325.67675933,124.23363843)(325.82675918,124.48363818)(325.93676485,124.8036452)
\curveto(325.96675904,124.90363776)(325.98675902,125.02863764)(325.99676485,125.1786452)
\curveto(326.01675899,125.33863733)(326.011759,125.48363718)(325.98176485,125.6136452)
\curveto(325.96175905,125.68363698)(325.94175907,125.74863692)(325.92176485,125.8086452)
\curveto(325.9117591,125.87863679)(325.89175912,125.94363672)(325.86176485,126.0036452)
\curveto(325.76175925,126.24363642)(325.61675939,126.43363623)(325.42676485,126.5736452)
\curveto(325.23675977,126.71363595)(325.01176,126.82363584)(324.75176485,126.9036452)
\curveto(324.69176032,126.92363574)(324.63176038,126.93363573)(324.57176485,126.9336452)
\curveto(324.5117605,126.93363573)(324.44676056,126.94363572)(324.37676485,126.9636452)
\curveto(324.29676071,126.98363568)(324.20176081,126.99363567)(324.09176485,126.9936452)
\curveto(323.98176103,126.99363567)(323.88676112,126.98363568)(323.80676485,126.9636452)
\curveto(323.75676125,126.94363572)(323.7067613,126.93363573)(323.65676485,126.9336452)
\curveto(323.61676139,126.93363573)(323.57176144,126.92363574)(323.52176485,126.9036452)
\curveto(323.34176167,126.85363581)(323.17176184,126.77863589)(323.01176485,126.6786452)
\curveto(322.86176215,126.58863608)(322.73176228,126.47363619)(322.62176485,126.3336452)
\curveto(322.53176248,126.21363645)(322.45176256,126.08363658)(322.38176485,125.9436452)
\curveto(322.3117627,125.80363686)(322.24676276,125.64863702)(322.18676485,125.4786452)
\curveto(322.15676285,125.3686373)(322.13676287,125.24863742)(322.12676485,125.1186452)
\curveto(322.11676289,124.99863767)(322.08176293,124.89863777)(322.02176485,124.8186452)
\curveto(322.00176301,124.77863789)(321.94176307,124.73863793)(321.84176485,124.6986452)
\curveto(321.80176321,124.68863798)(321.74176327,124.67863799)(321.66176485,124.6686452)
\lineto(321.40676485,124.6686452)
\curveto(321.31676369,124.67863799)(321.23176378,124.68863798)(321.15176485,124.6986452)
\curveto(321.08176393,124.70863796)(321.03176398,124.72363794)(321.00176485,124.7436452)
\curveto(320.96176405,124.77363789)(320.92676408,124.82863784)(320.89676485,124.9086452)
\curveto(320.86676414,124.98863768)(320.86176415,125.07363759)(320.88176485,125.1636452)
\curveto(320.89176412,125.21363745)(320.89676411,125.2636374)(320.89676485,125.3136452)
\lineto(320.92676485,125.4936452)
\curveto(320.95676405,125.59363707)(320.98176403,125.69363697)(321.00176485,125.7936452)
\curveto(321.03176398,125.89363677)(321.06676394,125.98363668)(321.10676485,126.0636452)
\curveto(321.15676385,126.17363649)(321.20176381,126.27863639)(321.24176485,126.3786452)
\curveto(321.28176373,126.48863618)(321.33176368,126.59363607)(321.39176485,126.6936452)
\curveto(321.72176329,127.23363543)(322.19176282,127.62863504)(322.80176485,127.8786452)
\curveto(322.92176209,127.92863474)(323.04676196,127.9636347)(323.17676485,127.9836452)
\curveto(323.31676169,128.00363466)(323.45676155,128.02863464)(323.59676485,128.0586452)
\curveto(323.65676135,128.0686346)(323.71676129,128.07363459)(323.77676485,128.0736452)
\curveto(323.84676116,128.07363459)(323.9117611,128.07863459)(323.97176485,128.0886452)
}
}
{
\newrgbcolor{curcolor}{0 0 0}
\pscustom[linestyle=none,fillstyle=solid,fillcolor=curcolor]
{
\newpath
\moveto(339.01137422,126.0036452)
\curveto(338.81136392,125.71363695)(338.60136413,125.42863724)(338.38137422,125.1486452)
\curveto(338.17136456,124.8686378)(337.96636477,124.58363808)(337.76637422,124.2936452)
\curveto(337.16636557,123.44363922)(336.56136617,122.60364006)(335.95137422,121.7736452)
\curveto(335.34136739,120.95364171)(334.736368,120.11864255)(334.13637422,119.2686452)
\lineto(333.62637422,118.5486452)
\lineto(333.11637422,117.8586452)
\curveto(333.0363697,117.74864492)(332.95636978,117.63364503)(332.87637422,117.5136452)
\curveto(332.79636994,117.39364527)(332.70137003,117.29864537)(332.59137422,117.2286452)
\curveto(332.55137018,117.20864546)(332.48637025,117.19364547)(332.39637422,117.1836452)
\curveto(332.31637042,117.1636455)(332.22637051,117.15364551)(332.12637422,117.1536452)
\curveto(332.02637071,117.15364551)(331.9313708,117.15864551)(331.84137422,117.1686452)
\curveto(331.76137097,117.17864549)(331.70137103,117.19864547)(331.66137422,117.2286452)
\curveto(331.6313711,117.24864542)(331.60637113,117.28364538)(331.58637422,117.3336452)
\curveto(331.57637116,117.37364529)(331.58137115,117.41864525)(331.60137422,117.4686452)
\curveto(331.64137109,117.54864512)(331.68637105,117.62364504)(331.73637422,117.6936452)
\curveto(331.79637094,117.77364489)(331.85137088,117.85364481)(331.90137422,117.9336452)
\curveto(332.14137059,118.27364439)(332.38637035,118.60864406)(332.63637422,118.9386452)
\curveto(332.88636985,119.2686434)(333.12636961,119.60364306)(333.35637422,119.9436452)
\curveto(333.51636922,120.1636425)(333.67636906,120.37864229)(333.83637422,120.5886452)
\curveto(333.99636874,120.79864187)(334.15636858,121.01364165)(334.31637422,121.2336452)
\curveto(334.67636806,121.75364091)(335.04136769,122.2636404)(335.41137422,122.7636452)
\curveto(335.78136695,123.2636394)(336.15136658,123.77363889)(336.52137422,124.2936452)
\curveto(336.66136607,124.49363817)(336.80136593,124.68863798)(336.94137422,124.8786452)
\curveto(337.09136564,125.0686376)(337.2363655,125.2636374)(337.37637422,125.4636452)
\curveto(337.58636515,125.7636369)(337.80136493,126.0636366)(338.02137422,126.3636452)
\lineto(338.68137422,127.2636452)
\lineto(338.86137422,127.5336452)
\lineto(339.07137422,127.8036452)
\lineto(339.19137422,127.9836452)
\curveto(339.24136349,128.04363462)(339.29136344,128.09863457)(339.34137422,128.1486452)
\curveto(339.41136332,128.19863447)(339.48636325,128.23363443)(339.56637422,128.2536452)
\curveto(339.58636315,128.2636344)(339.61136312,128.2636344)(339.64137422,128.2536452)
\curveto(339.68136305,128.25363441)(339.71136302,128.2636344)(339.73137422,128.2836452)
\curveto(339.85136288,128.28363438)(339.98636275,128.27863439)(340.13637422,128.2686452)
\curveto(340.28636245,128.2686344)(340.37636236,128.22363444)(340.40637422,128.1336452)
\curveto(340.42636231,128.10363456)(340.4313623,128.0686346)(340.42137422,128.0286452)
\curveto(340.41136232,127.98863468)(340.39636234,127.95863471)(340.37637422,127.9386452)
\curveto(340.3363624,127.85863481)(340.29636244,127.78863488)(340.25637422,127.7286452)
\curveto(340.21636252,127.668635)(340.17136256,127.60863506)(340.12137422,127.5486452)
\lineto(339.55137422,126.7686452)
\curveto(339.37136336,126.51863615)(339.19136354,126.2636364)(339.01137422,126.0036452)
\moveto(332.15637422,122.1036452)
\curveto(332.10637063,122.12364054)(332.05637068,122.12864054)(332.00637422,122.1186452)
\curveto(331.95637078,122.10864056)(331.90637083,122.11364055)(331.85637422,122.1336452)
\curveto(331.74637099,122.15364051)(331.64137109,122.17364049)(331.54137422,122.1936452)
\curveto(331.45137128,122.22364044)(331.35637138,122.2636404)(331.25637422,122.3136452)
\curveto(330.92637181,122.45364021)(330.67137206,122.64864002)(330.49137422,122.8986452)
\curveto(330.31137242,123.15863951)(330.16637257,123.4686392)(330.05637422,123.8286452)
\curveto(330.02637271,123.90863876)(330.00637273,123.98863868)(329.99637422,124.0686452)
\curveto(329.98637275,124.15863851)(329.97137276,124.24363842)(329.95137422,124.3236452)
\curveto(329.94137279,124.37363829)(329.9363728,124.43863823)(329.93637422,124.5186452)
\curveto(329.92637281,124.54863812)(329.92137281,124.57863809)(329.92137422,124.6086452)
\curveto(329.92137281,124.64863802)(329.91637282,124.68363798)(329.90637422,124.7136452)
\lineto(329.90637422,124.8636452)
\curveto(329.89637284,124.91363775)(329.89137284,124.97363769)(329.89137422,125.0436452)
\curveto(329.89137284,125.12363754)(329.89637284,125.18863748)(329.90637422,125.2386452)
\lineto(329.90637422,125.4036452)
\curveto(329.92637281,125.45363721)(329.9313728,125.49863717)(329.92137422,125.5386452)
\curveto(329.92137281,125.58863708)(329.92637281,125.63363703)(329.93637422,125.6736452)
\curveto(329.94637279,125.71363695)(329.95137278,125.74863692)(329.95137422,125.7786452)
\curveto(329.95137278,125.81863685)(329.95637278,125.85863681)(329.96637422,125.8986452)
\curveto(329.99637274,126.00863666)(330.01637272,126.11863655)(330.02637422,126.2286452)
\curveto(330.04637269,126.34863632)(330.08137265,126.4636362)(330.13137422,126.5736452)
\curveto(330.27137246,126.91363575)(330.4313723,127.18863548)(330.61137422,127.3986452)
\curveto(330.80137193,127.61863505)(331.07137166,127.79863487)(331.42137422,127.9386452)
\curveto(331.50137123,127.9686347)(331.58637115,127.98863468)(331.67637422,127.9986452)
\curveto(331.76637097,128.01863465)(331.86137087,128.03863463)(331.96137422,128.0586452)
\curveto(331.99137074,128.0686346)(332.04637069,128.0686346)(332.12637422,128.0586452)
\curveto(332.20637053,128.05863461)(332.25637048,128.0686346)(332.27637422,128.0886452)
\curveto(332.8363699,128.09863457)(333.28636945,127.98863468)(333.62637422,127.7586452)
\curveto(333.97636876,127.52863514)(334.2363685,127.22363544)(334.40637422,126.8436452)
\curveto(334.44636829,126.75363591)(334.48136825,126.65863601)(334.51137422,126.5586452)
\curveto(334.54136819,126.45863621)(334.56636817,126.35863631)(334.58637422,126.2586452)
\curveto(334.60636813,126.22863644)(334.61136812,126.19863647)(334.60137422,126.1686452)
\curveto(334.60136813,126.13863653)(334.60636813,126.10863656)(334.61637422,126.0786452)
\curveto(334.64636809,125.9686367)(334.66636807,125.84363682)(334.67637422,125.7036452)
\curveto(334.68636805,125.57363709)(334.69636804,125.43863723)(334.70637422,125.2986452)
\lineto(334.70637422,125.1336452)
\curveto(334.71636802,125.07363759)(334.71636802,125.01863765)(334.70637422,124.9686452)
\curveto(334.69636804,124.91863775)(334.69136804,124.8686378)(334.69137422,124.8186452)
\lineto(334.69137422,124.6836452)
\curveto(334.68136805,124.64363802)(334.67636806,124.60363806)(334.67637422,124.5636452)
\curveto(334.68636805,124.52363814)(334.68136805,124.47863819)(334.66137422,124.4286452)
\curveto(334.64136809,124.31863835)(334.62136811,124.21363845)(334.60137422,124.1136452)
\curveto(334.59136814,124.01363865)(334.57136816,123.91363875)(334.54137422,123.8136452)
\curveto(334.41136832,123.45363921)(334.24636849,123.13863953)(334.04637422,122.8686452)
\curveto(333.84636889,122.59864007)(333.57136916,122.39364027)(333.22137422,122.2536452)
\curveto(333.14136959,122.22364044)(333.05636968,122.19864047)(332.96637422,122.1786452)
\lineto(332.69637422,122.1186452)
\curveto(332.64637009,122.10864056)(332.60137013,122.10364056)(332.56137422,122.1036452)
\curveto(332.52137021,122.11364055)(332.48137025,122.11364055)(332.44137422,122.1036452)
\curveto(332.34137039,122.08364058)(332.24637049,122.08364058)(332.15637422,122.1036452)
\moveto(331.31637422,123.4986452)
\curveto(331.35637138,123.42863924)(331.39637134,123.3636393)(331.43637422,123.3036452)
\curveto(331.47637126,123.25363941)(331.52637121,123.20363946)(331.58637422,123.1536452)
\lineto(331.73637422,123.0336452)
\curveto(331.79637094,123.00363966)(331.86137087,122.97863969)(331.93137422,122.9586452)
\curveto(331.97137076,122.93863973)(332.00637073,122.92863974)(332.03637422,122.9286452)
\curveto(332.07637066,122.93863973)(332.11637062,122.93363973)(332.15637422,122.9136452)
\curveto(332.18637055,122.91363975)(332.22637051,122.90863976)(332.27637422,122.8986452)
\curveto(332.32637041,122.89863977)(332.36637037,122.90363976)(332.39637422,122.9136452)
\lineto(332.62137422,122.9586452)
\curveto(332.87136986,123.03863963)(333.05636968,123.1636395)(333.17637422,123.3336452)
\curveto(333.25636948,123.43363923)(333.32636941,123.5636391)(333.38637422,123.7236452)
\curveto(333.46636927,123.90363876)(333.52636921,124.12863854)(333.56637422,124.3986452)
\curveto(333.60636913,124.67863799)(333.62136911,124.95863771)(333.61137422,125.2386452)
\curveto(333.60136913,125.52863714)(333.57136916,125.80363686)(333.52137422,126.0636452)
\curveto(333.47136926,126.32363634)(333.39636934,126.53363613)(333.29637422,126.6936452)
\curveto(333.17636956,126.89363577)(333.02636971,127.04363562)(332.84637422,127.1436452)
\curveto(332.76636997,127.19363547)(332.67637006,127.22363544)(332.57637422,127.2336452)
\curveto(332.47637026,127.25363541)(332.37137036,127.2636354)(332.26137422,127.2636452)
\curveto(332.24137049,127.25363541)(332.21637052,127.24863542)(332.18637422,127.2486452)
\curveto(332.16637057,127.25863541)(332.14637059,127.25863541)(332.12637422,127.2486452)
\curveto(332.07637066,127.23863543)(332.0313707,127.22863544)(331.99137422,127.2186452)
\curveto(331.95137078,127.21863545)(331.91137082,127.20863546)(331.87137422,127.1886452)
\curveto(331.69137104,127.10863556)(331.54137119,126.98863568)(331.42137422,126.8286452)
\curveto(331.31137142,126.668636)(331.22137151,126.48863618)(331.15137422,126.2886452)
\curveto(331.09137164,126.09863657)(331.04637169,125.87363679)(331.01637422,125.6136452)
\curveto(330.99637174,125.35363731)(330.99137174,125.08863758)(331.00137422,124.8186452)
\curveto(331.01137172,124.55863811)(331.04137169,124.30863836)(331.09137422,124.0686452)
\curveto(331.15137158,123.83863883)(331.22637151,123.64863902)(331.31637422,123.4986452)
\moveto(342.11637422,120.5136452)
\curveto(342.12636061,120.4636422)(342.1313606,120.37364229)(342.13137422,120.2436452)
\curveto(342.1313606,120.11364255)(342.12136061,120.02364264)(342.10137422,119.9736452)
\curveto(342.08136065,119.92364274)(342.07636066,119.8686428)(342.08637422,119.8086452)
\curveto(342.09636064,119.75864291)(342.09636064,119.70864296)(342.08637422,119.6586452)
\curveto(342.04636069,119.51864315)(342.01636072,119.38364328)(341.99637422,119.2536452)
\curveto(341.98636075,119.12364354)(341.95636078,119.00364366)(341.90637422,118.8936452)
\curveto(341.76636097,118.54364412)(341.60136113,118.24864442)(341.41137422,118.0086452)
\curveto(341.22136151,117.77864489)(340.95136178,117.59364507)(340.60137422,117.4536452)
\curveto(340.52136221,117.42364524)(340.4363623,117.40364526)(340.34637422,117.3936452)
\curveto(340.25636248,117.37364529)(340.17136256,117.35364531)(340.09137422,117.3336452)
\curveto(340.04136269,117.32364534)(339.99136274,117.31864535)(339.94137422,117.3186452)
\curveto(339.89136284,117.31864535)(339.84136289,117.31364535)(339.79137422,117.3036452)
\curveto(339.76136297,117.29364537)(339.71136302,117.29364537)(339.64137422,117.3036452)
\curveto(339.57136316,117.30364536)(339.52136321,117.30864536)(339.49137422,117.3186452)
\curveto(339.4313633,117.33864533)(339.37136336,117.34864532)(339.31137422,117.3486452)
\curveto(339.26136347,117.33864533)(339.21136352,117.34364532)(339.16137422,117.3636452)
\curveto(339.07136366,117.38364528)(338.98136375,117.40864526)(338.89137422,117.4386452)
\curveto(338.81136392,117.45864521)(338.731364,117.48864518)(338.65137422,117.5286452)
\curveto(338.3313644,117.668645)(338.08136465,117.8636448)(337.90137422,118.1136452)
\curveto(337.72136501,118.37364429)(337.57136516,118.67864399)(337.45137422,119.0286452)
\curveto(337.4313653,119.10864356)(337.41636532,119.19364347)(337.40637422,119.2836452)
\curveto(337.39636534,119.37364329)(337.38136535,119.45864321)(337.36137422,119.5386452)
\curveto(337.35136538,119.5686431)(337.34636539,119.59864307)(337.34637422,119.6286452)
\lineto(337.34637422,119.7336452)
\curveto(337.32636541,119.81364285)(337.31636542,119.89364277)(337.31637422,119.9736452)
\lineto(337.31637422,120.1086452)
\curveto(337.29636544,120.20864246)(337.29636544,120.30864236)(337.31637422,120.4086452)
\lineto(337.31637422,120.5886452)
\curveto(337.32636541,120.63864203)(337.3313654,120.68364198)(337.33137422,120.7236452)
\curveto(337.3313654,120.77364189)(337.3363654,120.81864185)(337.34637422,120.8586452)
\curveto(337.35636538,120.89864177)(337.36136537,120.93364173)(337.36137422,120.9636452)
\curveto(337.36136537,121.00364166)(337.36636537,121.04364162)(337.37637422,121.0836452)
\lineto(337.43637422,121.4136452)
\curveto(337.45636528,121.53364113)(337.48636525,121.64364102)(337.52637422,121.7436452)
\curveto(337.66636507,122.07364059)(337.82636491,122.34864032)(338.00637422,122.5686452)
\curveto(338.19636454,122.79863987)(338.45636428,122.98363968)(338.78637422,123.1236452)
\curveto(338.86636387,123.1636395)(338.95136378,123.18863948)(339.04137422,123.1986452)
\lineto(339.34137422,123.2586452)
\lineto(339.47637422,123.2586452)
\curveto(339.52636321,123.2686394)(339.57636316,123.27363939)(339.62637422,123.2736452)
\curveto(340.19636254,123.29363937)(340.65636208,123.18863948)(341.00637422,122.9586452)
\curveto(341.36636137,122.73863993)(341.6313611,122.43864023)(341.80137422,122.0586452)
\curveto(341.85136088,121.95864071)(341.89136084,121.85864081)(341.92137422,121.7586452)
\curveto(341.95136078,121.65864101)(341.98136075,121.55364111)(342.01137422,121.4436452)
\curveto(342.02136071,121.40364126)(342.02636071,121.3686413)(342.02637422,121.3386452)
\curveto(342.02636071,121.31864135)(342.0313607,121.28864138)(342.04137422,121.2486452)
\curveto(342.06136067,121.17864149)(342.07136066,121.10364156)(342.07137422,121.0236452)
\curveto(342.07136066,120.94364172)(342.08136065,120.8636418)(342.10137422,120.7836452)
\curveto(342.10136063,120.73364193)(342.10136063,120.68864198)(342.10137422,120.6486452)
\curveto(342.10136063,120.60864206)(342.10636063,120.5636421)(342.11637422,120.5136452)
\moveto(341.00637422,120.0786452)
\curveto(341.01636172,120.12864254)(341.02136171,120.20364246)(341.02137422,120.3036452)
\curveto(341.0313617,120.40364226)(341.02636171,120.47864219)(341.00637422,120.5286452)
\curveto(340.98636175,120.58864208)(340.98136175,120.64364202)(340.99137422,120.6936452)
\curveto(341.01136172,120.75364191)(341.01136172,120.81364185)(340.99137422,120.8736452)
\curveto(340.98136175,120.90364176)(340.97636176,120.93864173)(340.97637422,120.9786452)
\curveto(340.97636176,121.01864165)(340.97136176,121.05864161)(340.96137422,121.0986452)
\curveto(340.94136179,121.17864149)(340.92136181,121.25364141)(340.90137422,121.3236452)
\curveto(340.89136184,121.40364126)(340.87636186,121.48364118)(340.85637422,121.5636452)
\curveto(340.82636191,121.62364104)(340.80136193,121.68364098)(340.78137422,121.7436452)
\curveto(340.76136197,121.80364086)(340.731362,121.8636408)(340.69137422,121.9236452)
\curveto(340.59136214,122.09364057)(340.46136227,122.22864044)(340.30137422,122.3286452)
\curveto(340.22136251,122.37864029)(340.12636261,122.41364025)(340.01637422,122.4336452)
\curveto(339.90636283,122.45364021)(339.78136295,122.4636402)(339.64137422,122.4636452)
\curveto(339.62136311,122.45364021)(339.59636314,122.44864022)(339.56637422,122.4486452)
\curveto(339.5363632,122.45864021)(339.50636323,122.45864021)(339.47637422,122.4486452)
\lineto(339.32637422,122.3886452)
\curveto(339.27636346,122.37864029)(339.2313635,122.3636403)(339.19137422,122.3436452)
\curveto(339.00136373,122.23364043)(338.85636388,122.08864058)(338.75637422,121.9086452)
\curveto(338.66636407,121.72864094)(338.58636415,121.52364114)(338.51637422,121.2936452)
\curveto(338.47636426,121.1636415)(338.45636428,121.02864164)(338.45637422,120.8886452)
\curveto(338.45636428,120.75864191)(338.44636429,120.61364205)(338.42637422,120.4536452)
\curveto(338.41636432,120.40364226)(338.40636433,120.34364232)(338.39637422,120.2736452)
\curveto(338.39636434,120.20364246)(338.40636433,120.14364252)(338.42637422,120.0936452)
\lineto(338.42637422,119.9286452)
\lineto(338.42637422,119.7486452)
\curveto(338.4363643,119.69864297)(338.44636429,119.64364302)(338.45637422,119.5836452)
\curveto(338.46636427,119.53364313)(338.47136426,119.47864319)(338.47137422,119.4186452)
\curveto(338.48136425,119.35864331)(338.49636424,119.30364336)(338.51637422,119.2536452)
\curveto(338.56636417,119.0636436)(338.62636411,118.88864378)(338.69637422,118.7286452)
\curveto(338.76636397,118.5686441)(338.87136386,118.43864423)(339.01137422,118.3386452)
\curveto(339.14136359,118.23864443)(339.28136345,118.1686445)(339.43137422,118.1286452)
\curveto(339.46136327,118.11864455)(339.48636325,118.11364455)(339.50637422,118.1136452)
\curveto(339.5363632,118.12364454)(339.56636317,118.12364454)(339.59637422,118.1136452)
\curveto(339.61636312,118.11364455)(339.64636309,118.10864456)(339.68637422,118.0986452)
\curveto(339.72636301,118.09864457)(339.76136297,118.10364456)(339.79137422,118.1136452)
\curveto(339.8313629,118.12364454)(339.87136286,118.12864454)(339.91137422,118.1286452)
\curveto(339.95136278,118.12864454)(339.99136274,118.13864453)(340.03137422,118.1586452)
\curveto(340.27136246,118.23864443)(340.46636227,118.37364429)(340.61637422,118.5636452)
\curveto(340.736362,118.74364392)(340.82636191,118.94864372)(340.88637422,119.1786452)
\curveto(340.90636183,119.24864342)(340.92136181,119.31864335)(340.93137422,119.3886452)
\curveto(340.94136179,119.4686432)(340.95636178,119.54864312)(340.97637422,119.6286452)
\curveto(340.97636176,119.68864298)(340.98136175,119.73364293)(340.99137422,119.7636452)
\curveto(340.99136174,119.78364288)(340.99136174,119.80864286)(340.99137422,119.8386452)
\curveto(340.99136174,119.87864279)(340.99636174,119.90864276)(341.00637422,119.9286452)
\lineto(341.00637422,120.0786452)
}
}
{
\newrgbcolor{curcolor}{0 0 0}
\pscustom[linestyle=none,fillstyle=solid,fillcolor=curcolor]
{
\newpath
\moveto(84.91944246,187.68764422)
\curveto(86.54943702,187.71763357)(87.59943597,187.16263413)(88.06944246,186.02264422)
\curveto(88.1694354,185.7926355)(88.23443534,185.50263579)(88.26444246,185.15264422)
\curveto(88.30443527,184.81263648)(88.27943529,184.50263679)(88.18944246,184.22264422)
\curveto(88.09943547,183.96263733)(87.97943559,183.73763755)(87.82944246,183.54764422)
\curveto(87.80943576,183.50763778)(87.78443579,183.47263782)(87.75444246,183.44264422)
\curveto(87.72443585,183.42263787)(87.69943587,183.39763789)(87.67944246,183.36764422)
\lineto(87.58944246,183.24764422)
\curveto(87.55943601,183.21763807)(87.52443605,183.1926381)(87.48444246,183.17264422)
\curveto(87.43443614,183.12263817)(87.37943619,183.07763821)(87.31944246,183.03764422)
\curveto(87.2694363,182.99763829)(87.22443635,182.94763834)(87.18444246,182.88764422)
\curveto(87.14443643,182.84763844)(87.12943644,182.79763849)(87.13944246,182.73764422)
\curveto(87.14943642,182.6876386)(87.17943639,182.64263865)(87.22944246,182.60264422)
\curveto(87.27943629,182.56263873)(87.33443624,182.52263877)(87.39444246,182.48264422)
\curveto(87.46443611,182.45263884)(87.52943604,182.42263887)(87.58944246,182.39264422)
\curveto(87.64943592,182.36263893)(87.69943587,182.33263896)(87.73944246,182.30264422)
\curveto(88.05943551,182.08263921)(88.31443526,181.77263952)(88.50444246,181.37264422)
\curveto(88.54443503,181.28264001)(88.574435,181.1876401)(88.59444246,181.08764422)
\curveto(88.62443495,180.99764029)(88.64943492,180.90764038)(88.66944246,180.81764422)
\curveto(88.67943489,180.76764052)(88.68443489,180.71764057)(88.68444246,180.66764422)
\curveto(88.69443488,180.62764066)(88.70443487,180.58264071)(88.71444246,180.53264422)
\curveto(88.72443485,180.48264081)(88.72443485,180.43264086)(88.71444246,180.38264422)
\curveto(88.70443487,180.33264096)(88.70943486,180.28264101)(88.72944246,180.23264422)
\curveto(88.73943483,180.18264111)(88.74443483,180.12264117)(88.74444246,180.05264422)
\curveto(88.74443483,179.98264131)(88.73443484,179.92264137)(88.71444246,179.87264422)
\lineto(88.71444246,179.64764422)
\lineto(88.65444246,179.40764422)
\curveto(88.64443493,179.33764195)(88.62943494,179.26764202)(88.60944246,179.19764422)
\curveto(88.57943499,179.10764218)(88.54943502,179.02264227)(88.51944246,178.94264422)
\curveto(88.49943507,178.86264243)(88.4694351,178.78264251)(88.42944246,178.70264422)
\curveto(88.40943516,178.64264265)(88.37943519,178.58264271)(88.33944246,178.52264422)
\curveto(88.30943526,178.47264282)(88.2744353,178.42264287)(88.23444246,178.37264422)
\curveto(88.03443554,178.06264323)(87.78443579,177.80264349)(87.48444246,177.59264422)
\curveto(87.18443639,177.3926439)(86.83943673,177.22764406)(86.44944246,177.09764422)
\curveto(86.32943724,177.05764423)(86.19943737,177.03264426)(86.05944246,177.02264422)
\curveto(85.92943764,177.00264429)(85.79443778,176.97764431)(85.65444246,176.94764422)
\curveto(85.58443799,176.93764435)(85.51443806,176.93264436)(85.44444246,176.93264422)
\curveto(85.38443819,176.93264436)(85.31943825,176.92764436)(85.24944246,176.91764422)
\curveto(85.20943836,176.90764438)(85.14943842,176.90264439)(85.06944246,176.90264422)
\curveto(84.99943857,176.90264439)(84.94943862,176.90764438)(84.91944246,176.91764422)
\curveto(84.8694387,176.92764436)(84.82443875,176.93264436)(84.78444246,176.93264422)
\lineto(84.66444246,176.93264422)
\curveto(84.56443901,176.95264434)(84.46443911,176.96764432)(84.36444246,176.97764422)
\curveto(84.26443931,176.9876443)(84.1694394,177.00264429)(84.07944246,177.02264422)
\curveto(83.9694396,177.05264424)(83.85943971,177.07764421)(83.74944246,177.09764422)
\curveto(83.64943992,177.12764416)(83.54444003,177.16764412)(83.43444246,177.21764422)
\curveto(83.06444051,177.37764391)(82.74944082,177.57764371)(82.48944246,177.81764422)
\curveto(82.22944134,178.06764322)(82.01944155,178.37764291)(81.85944246,178.74764422)
\curveto(81.81944175,178.83764245)(81.78444179,178.93264236)(81.75444246,179.03264422)
\curveto(81.72444185,179.13264216)(81.69444188,179.23764205)(81.66444246,179.34764422)
\curveto(81.64444193,179.39764189)(81.63444194,179.44764184)(81.63444246,179.49764422)
\curveto(81.63444194,179.55764173)(81.62444195,179.61764167)(81.60444246,179.67764422)
\curveto(81.58444199,179.73764155)(81.574442,179.81764147)(81.57444246,179.91764422)
\curveto(81.574442,180.01764127)(81.58944198,180.0926412)(81.61944246,180.14264422)
\curveto(81.62944194,180.17264112)(81.64444193,180.19764109)(81.66444246,180.21764422)
\lineto(81.72444246,180.27764422)
\curveto(81.76444181,180.29764099)(81.82444175,180.31264098)(81.90444246,180.32264422)
\curveto(81.99444158,180.33264096)(82.08444149,180.33764095)(82.17444246,180.33764422)
\curveto(82.26444131,180.33764095)(82.34944122,180.33264096)(82.42944246,180.32264422)
\curveto(82.51944105,180.31264098)(82.58444099,180.30264099)(82.62444246,180.29264422)
\curveto(82.64444093,180.27264102)(82.66444091,180.25764103)(82.68444246,180.24764422)
\curveto(82.70444087,180.24764104)(82.72444085,180.23764105)(82.74444246,180.21764422)
\curveto(82.81444076,180.12764116)(82.85444072,180.01264128)(82.86444246,179.87264422)
\curveto(82.88444069,179.73264156)(82.91444066,179.60764168)(82.95444246,179.49764422)
\lineto(83.10444246,179.13764422)
\curveto(83.15444042,179.02764226)(83.21944035,178.92264237)(83.29944246,178.82264422)
\curveto(83.31944025,178.7926425)(83.33944023,178.76764252)(83.35944246,178.74764422)
\curveto(83.38944018,178.72764256)(83.41444016,178.70264259)(83.43444246,178.67264422)
\curveto(83.4744401,178.61264268)(83.50944006,178.56764272)(83.53944246,178.53764422)
\curveto(83.57943999,178.50764278)(83.61443996,178.47764281)(83.64444246,178.44764422)
\curveto(83.68443989,178.41764287)(83.72943984,178.3876429)(83.77944246,178.35764422)
\curveto(83.8694397,178.29764299)(83.96443961,178.24764304)(84.06444246,178.20764422)
\lineto(84.39444246,178.08764422)
\curveto(84.54443903,178.03764325)(84.74443883,178.00764328)(84.99444246,177.99764422)
\curveto(85.24443833,177.9876433)(85.45443812,178.00764328)(85.62444246,178.05764422)
\curveto(85.70443787,178.07764321)(85.7744378,178.0926432)(85.83444246,178.10264422)
\lineto(86.04444246,178.16264422)
\curveto(86.32443725,178.28264301)(86.56443701,178.43264286)(86.76444246,178.61264422)
\curveto(86.9744366,178.7926425)(87.13943643,179.02264227)(87.25944246,179.30264422)
\curveto(87.28943628,179.37264192)(87.30943626,179.44264185)(87.31944246,179.51264422)
\lineto(87.37944246,179.75264422)
\curveto(87.41943615,179.8926414)(87.42943614,180.05264124)(87.40944246,180.23264422)
\curveto(87.38943618,180.42264087)(87.35943621,180.57264072)(87.31944246,180.68264422)
\curveto(87.18943638,181.06264023)(87.00443657,181.35263994)(86.76444246,181.55264422)
\curveto(86.53443704,181.75263954)(86.22443735,181.91263938)(85.83444246,182.03264422)
\curveto(85.72443785,182.06263923)(85.60443797,182.08263921)(85.47444246,182.09264422)
\curveto(85.35443822,182.10263919)(85.22943834,182.10763918)(85.09944246,182.10764422)
\curveto(84.93943863,182.10763918)(84.79943877,182.11263918)(84.67944246,182.12264422)
\curveto(84.55943901,182.13263916)(84.4744391,182.1926391)(84.42444246,182.30264422)
\curveto(84.40443917,182.33263896)(84.39443918,182.36763892)(84.39444246,182.40764422)
\lineto(84.39444246,182.54264422)
\curveto(84.38443919,182.64263865)(84.38443919,182.73763855)(84.39444246,182.82764422)
\curveto(84.41443916,182.91763837)(84.45443912,182.98263831)(84.51444246,183.02264422)
\curveto(84.55443902,183.05263824)(84.59443898,183.07263822)(84.63444246,183.08264422)
\curveto(84.68443889,183.0926382)(84.73943883,183.10263819)(84.79944246,183.11264422)
\curveto(84.81943875,183.12263817)(84.84443873,183.12263817)(84.87444246,183.11264422)
\curveto(84.90443867,183.11263818)(84.92943864,183.11763817)(84.94944246,183.12764422)
\lineto(85.08444246,183.12764422)
\curveto(85.19443838,183.14763814)(85.29443828,183.15763813)(85.38444246,183.15764422)
\curveto(85.48443809,183.16763812)(85.57943799,183.1876381)(85.66944246,183.21764422)
\curveto(85.98943758,183.32763796)(86.24443733,183.47263782)(86.43444246,183.65264422)
\curveto(86.62443695,183.83263746)(86.7744368,184.08263721)(86.88444246,184.40264422)
\curveto(86.91443666,184.50263679)(86.93443664,184.62763666)(86.94444246,184.77764422)
\curveto(86.96443661,184.93763635)(86.95943661,185.08263621)(86.92944246,185.21264422)
\curveto(86.90943666,185.28263601)(86.88943668,185.34763594)(86.86944246,185.40764422)
\curveto(86.85943671,185.47763581)(86.83943673,185.54263575)(86.80944246,185.60264422)
\curveto(86.70943686,185.84263545)(86.56443701,186.03263526)(86.37444246,186.17264422)
\curveto(86.18443739,186.31263498)(85.95943761,186.42263487)(85.69944246,186.50264422)
\curveto(85.63943793,186.52263477)(85.57943799,186.53263476)(85.51944246,186.53264422)
\curveto(85.45943811,186.53263476)(85.39443818,186.54263475)(85.32444246,186.56264422)
\curveto(85.24443833,186.58263471)(85.14943842,186.5926347)(85.03944246,186.59264422)
\curveto(84.92943864,186.5926347)(84.83443874,186.58263471)(84.75444246,186.56264422)
\curveto(84.70443887,186.54263475)(84.65443892,186.53263476)(84.60444246,186.53264422)
\curveto(84.56443901,186.53263476)(84.51943905,186.52263477)(84.46944246,186.50264422)
\curveto(84.28943928,186.45263484)(84.11943945,186.37763491)(83.95944246,186.27764422)
\curveto(83.80943976,186.1876351)(83.67943989,186.07263522)(83.56944246,185.93264422)
\curveto(83.47944009,185.81263548)(83.39944017,185.68263561)(83.32944246,185.54264422)
\curveto(83.25944031,185.40263589)(83.19444038,185.24763604)(83.13444246,185.07764422)
\curveto(83.10444047,184.96763632)(83.08444049,184.84763644)(83.07444246,184.71764422)
\curveto(83.06444051,184.59763669)(83.02944054,184.49763679)(82.96944246,184.41764422)
\curveto(82.94944062,184.37763691)(82.88944068,184.33763695)(82.78944246,184.29764422)
\curveto(82.74944082,184.287637)(82.68944088,184.27763701)(82.60944246,184.26764422)
\lineto(82.35444246,184.26764422)
\curveto(82.26444131,184.27763701)(82.17944139,184.287637)(82.09944246,184.29764422)
\curveto(82.02944154,184.30763698)(81.97944159,184.32263697)(81.94944246,184.34264422)
\curveto(81.90944166,184.37263692)(81.8744417,184.42763686)(81.84444246,184.50764422)
\curveto(81.81444176,184.5876367)(81.80944176,184.67263662)(81.82944246,184.76264422)
\curveto(81.83944173,184.81263648)(81.84444173,184.86263643)(81.84444246,184.91264422)
\lineto(81.87444246,185.09264422)
\curveto(81.90444167,185.1926361)(81.92944164,185.292636)(81.94944246,185.39264422)
\curveto(81.97944159,185.4926358)(82.01444156,185.58263571)(82.05444246,185.66264422)
\curveto(82.10444147,185.77263552)(82.14944142,185.87763541)(82.18944246,185.97764422)
\curveto(82.22944134,186.0876352)(82.27944129,186.1926351)(82.33944246,186.29264422)
\curveto(82.6694409,186.83263446)(83.13944043,187.22763406)(83.74944246,187.47764422)
\curveto(83.8694397,187.52763376)(83.99443958,187.56263373)(84.12444246,187.58264422)
\curveto(84.26443931,187.60263369)(84.40443917,187.62763366)(84.54444246,187.65764422)
\curveto(84.60443897,187.66763362)(84.66443891,187.67263362)(84.72444246,187.67264422)
\curveto(84.79443878,187.67263362)(84.85943871,187.67763361)(84.91944246,187.68764422)
}
}
{
\newrgbcolor{curcolor}{0 0 0}
\pscustom[linestyle=none,fillstyle=solid,fillcolor=curcolor]
{
\newpath
\moveto(90.61405184,187.49264422)
\lineto(95.41405184,187.49264422)
\lineto(96.41905184,187.49264422)
\curveto(96.55904474,187.4926338)(96.67904462,187.48263381)(96.77905184,187.46264422)
\curveto(96.88904441,187.45263384)(96.96904433,187.40763388)(97.01905184,187.32764422)
\curveto(97.03904426,187.287634)(97.04904425,187.23763405)(97.04905184,187.17764422)
\curveto(97.05904424,187.11763417)(97.06404423,187.05263424)(97.06405184,186.98264422)
\lineto(97.06405184,186.71264422)
\curveto(97.06404423,186.62263467)(97.05404424,186.54263475)(97.03405184,186.47264422)
\curveto(96.9940443,186.3926349)(96.94904435,186.32263497)(96.89905184,186.26264422)
\lineto(96.74905184,186.08264422)
\curveto(96.71904458,186.03263526)(96.68404461,185.9926353)(96.64405184,185.96264422)
\curveto(96.60404469,185.93263536)(96.56404473,185.8926354)(96.52405184,185.84264422)
\curveto(96.44404485,185.73263556)(96.35904494,185.62263567)(96.26905184,185.51264422)
\curveto(96.17904512,185.41263588)(96.0940452,185.30763598)(96.01405184,185.19764422)
\curveto(95.87404542,184.99763629)(95.73404556,184.7876365)(95.59405184,184.56764422)
\curveto(95.45404584,184.35763693)(95.31404598,184.14263715)(95.17405184,183.92264422)
\curveto(95.12404617,183.83263746)(95.07404622,183.73763755)(95.02405184,183.63764422)
\curveto(94.97404632,183.53763775)(94.91904638,183.44263785)(94.85905184,183.35264422)
\curveto(94.83904646,183.33263796)(94.82904647,183.30763798)(94.82905184,183.27764422)
\curveto(94.82904647,183.24763804)(94.81904648,183.22263807)(94.79905184,183.20264422)
\curveto(94.72904657,183.10263819)(94.66404663,182.9876383)(94.60405184,182.85764422)
\curveto(94.54404675,182.73763855)(94.48904681,182.62263867)(94.43905184,182.51264422)
\curveto(94.33904696,182.28263901)(94.24404705,182.04763924)(94.15405184,181.80764422)
\curveto(94.06404723,181.56763972)(93.96404733,181.32763996)(93.85405184,181.08764422)
\curveto(93.83404746,181.03764025)(93.81904748,180.9926403)(93.80905184,180.95264422)
\curveto(93.80904749,180.91264038)(93.7990475,180.86764042)(93.77905184,180.81764422)
\curveto(93.72904757,180.69764059)(93.68404761,180.57264072)(93.64405184,180.44264422)
\curveto(93.61404768,180.32264097)(93.57904772,180.20264109)(93.53905184,180.08264422)
\curveto(93.45904784,179.85264144)(93.3940479,179.61264168)(93.34405184,179.36264422)
\curveto(93.30404799,179.12264217)(93.25404804,178.88264241)(93.19405184,178.64264422)
\curveto(93.15404814,178.4926428)(93.12904817,178.34264295)(93.11905184,178.19264422)
\curveto(93.10904819,178.04264325)(93.08904821,177.8926434)(93.05905184,177.74264422)
\curveto(93.04904825,177.70264359)(93.04404825,177.64264365)(93.04405184,177.56264422)
\curveto(93.01404828,177.44264385)(92.98404831,177.34264395)(92.95405184,177.26264422)
\curveto(92.92404837,177.18264411)(92.85404844,177.12764416)(92.74405184,177.09764422)
\curveto(92.6940486,177.07764421)(92.63904866,177.06764422)(92.57905184,177.06764422)
\lineto(92.38405184,177.06764422)
\curveto(92.24404905,177.06764422)(92.10404919,177.07264422)(91.96405184,177.08264422)
\curveto(91.83404946,177.0926442)(91.73904956,177.13764415)(91.67905184,177.21764422)
\curveto(91.63904966,177.27764401)(91.61904968,177.36264393)(91.61905184,177.47264422)
\curveto(91.62904967,177.58264371)(91.64404965,177.67764361)(91.66405184,177.75764422)
\lineto(91.66405184,177.83264422)
\curveto(91.67404962,177.86264343)(91.67904962,177.8926434)(91.67905184,177.92264422)
\curveto(91.6990496,178.00264329)(91.70904959,178.07764321)(91.70905184,178.14764422)
\curveto(91.70904959,178.21764307)(91.71904958,178.287643)(91.73905184,178.35764422)
\curveto(91.78904951,178.54764274)(91.82904947,178.73264256)(91.85905184,178.91264422)
\curveto(91.88904941,179.10264219)(91.92904937,179.28264201)(91.97905184,179.45264422)
\curveto(91.9990493,179.50264179)(92.00904929,179.54264175)(92.00905184,179.57264422)
\curveto(92.00904929,179.60264169)(92.01404928,179.63764165)(92.02405184,179.67764422)
\curveto(92.12404917,179.97764131)(92.21404908,180.27264102)(92.29405184,180.56264422)
\curveto(92.38404891,180.85264044)(92.48904881,181.13264016)(92.60905184,181.40264422)
\curveto(92.86904843,181.98263931)(93.13904816,182.53263876)(93.41905184,183.05264422)
\curveto(93.6990476,183.58263771)(94.00904729,184.0876372)(94.34905184,184.56764422)
\curveto(94.48904681,184.76763652)(94.63904666,184.95763633)(94.79905184,185.13764422)
\curveto(94.95904634,185.32763596)(95.10904619,185.51763577)(95.24905184,185.70764422)
\curveto(95.28904601,185.75763553)(95.32404597,185.80263549)(95.35405184,185.84264422)
\curveto(95.3940459,185.8926354)(95.42904587,185.94263535)(95.45905184,185.99264422)
\curveto(95.46904583,186.01263528)(95.47904582,186.03763525)(95.48905184,186.06764422)
\curveto(95.50904579,186.09763519)(95.50904579,186.12763516)(95.48905184,186.15764422)
\curveto(95.46904583,186.21763507)(95.43404586,186.25263504)(95.38405184,186.26264422)
\curveto(95.33404596,186.28263501)(95.28404601,186.30263499)(95.23405184,186.32264422)
\lineto(95.12905184,186.32264422)
\curveto(95.08904621,186.33263496)(95.03904626,186.33263496)(94.97905184,186.32264422)
\lineto(94.82905184,186.32264422)
\lineto(94.22905184,186.32264422)
\lineto(91.58905184,186.32264422)
\lineto(90.85405184,186.32264422)
\lineto(90.61405184,186.32264422)
\curveto(90.54405075,186.33263496)(90.48405081,186.34763494)(90.43405184,186.36764422)
\curveto(90.34405095,186.40763488)(90.28405101,186.46763482)(90.25405184,186.54764422)
\curveto(90.20405109,186.64763464)(90.18905111,186.7926345)(90.20905184,186.98264422)
\curveto(90.22905107,187.18263411)(90.26405103,187.31763397)(90.31405184,187.38764422)
\curveto(90.33405096,187.40763388)(90.35905094,187.42263387)(90.38905184,187.43264422)
\lineto(90.50905184,187.49264422)
\curveto(90.52905077,187.4926338)(90.54405075,187.4876338)(90.55405184,187.47764422)
\curveto(90.57405072,187.47763381)(90.5940507,187.48263381)(90.61405184,187.49264422)
}
}
{
\newrgbcolor{curcolor}{0 0 0}
\pscustom[linestyle=none,fillstyle=solid,fillcolor=curcolor]
{
\newpath
\moveto(99.45866121,178.71764422)
\lineto(99.75866121,178.71764422)
\curveto(99.86865915,178.72764256)(99.97365905,178.72764256)(100.07366121,178.71764422)
\curveto(100.18365884,178.71764257)(100.28365874,178.70764258)(100.37366121,178.68764422)
\curveto(100.46365856,178.67764261)(100.53365849,178.65264264)(100.58366121,178.61264422)
\curveto(100.60365842,178.5926427)(100.6186584,178.56264273)(100.62866121,178.52264422)
\curveto(100.64865837,178.48264281)(100.66865835,178.43764285)(100.68866121,178.38764422)
\lineto(100.68866121,178.31264422)
\curveto(100.69865832,178.26264303)(100.69865832,178.20764308)(100.68866121,178.14764422)
\lineto(100.68866121,177.99764422)
\lineto(100.68866121,177.51764422)
\curveto(100.68865833,177.34764394)(100.64865837,177.22764406)(100.56866121,177.15764422)
\curveto(100.49865852,177.10764418)(100.40865861,177.08264421)(100.29866121,177.08264422)
\lineto(99.96866121,177.08264422)
\lineto(99.51866121,177.08264422)
\curveto(99.36865965,177.08264421)(99.25365977,177.11264418)(99.17366121,177.17264422)
\curveto(99.13365989,177.20264409)(99.10365992,177.25264404)(99.08366121,177.32264422)
\curveto(99.06365996,177.40264389)(99.04865997,177.4876438)(99.03866121,177.57764422)
\lineto(99.03866121,177.86264422)
\curveto(99.04865997,177.96264333)(99.05365997,178.04764324)(99.05366121,178.11764422)
\lineto(99.05366121,178.31264422)
\curveto(99.05365997,178.37264292)(99.06365996,178.42764286)(99.08366121,178.47764422)
\curveto(99.1236599,178.5876427)(99.19365983,178.65764263)(99.29366121,178.68764422)
\curveto(99.3236597,178.6876426)(99.37865964,178.69764259)(99.45866121,178.71764422)
}
}
{
\newrgbcolor{curcolor}{0 0 0}
\pscustom[linestyle=none,fillstyle=solid,fillcolor=curcolor]
{
\newpath
\moveto(109.60381746,182.67764422)
\curveto(109.60380983,182.59763869)(109.60880982,182.51763877)(109.61881746,182.43764422)
\curveto(109.6288098,182.35763893)(109.62380981,182.28263901)(109.60381746,182.21264422)
\curveto(109.58380985,182.17263912)(109.57880985,182.12763916)(109.58881746,182.07764422)
\curveto(109.59880983,182.03763925)(109.59880983,181.99763929)(109.58881746,181.95764422)
\lineto(109.58881746,181.80764422)
\curveto(109.57880985,181.71763957)(109.57380986,181.62763966)(109.57381746,181.53764422)
\curveto(109.57380986,181.45763983)(109.56880986,181.37763991)(109.55881746,181.29764422)
\lineto(109.52881746,181.05764422)
\curveto(109.51880991,180.9876403)(109.50880992,180.91264038)(109.49881746,180.83264422)
\curveto(109.48880994,180.7926405)(109.48380995,180.75264054)(109.48381746,180.71264422)
\curveto(109.48380995,180.67264062)(109.47880995,180.62764066)(109.46881746,180.57764422)
\curveto(109.42881,180.43764085)(109.39881003,180.29764099)(109.37881746,180.15764422)
\curveto(109.36881006,180.01764127)(109.33881009,179.88264141)(109.28881746,179.75264422)
\curveto(109.23881019,179.58264171)(109.18381025,179.41764187)(109.12381746,179.25764422)
\curveto(109.07381036,179.09764219)(109.01381042,178.94264235)(108.94381746,178.79264422)
\curveto(108.92381051,178.73264256)(108.89381054,178.67264262)(108.85381746,178.61264422)
\lineto(108.76381746,178.46264422)
\curveto(108.56381087,178.14264315)(108.34881108,177.87764341)(108.11881746,177.66764422)
\curveto(107.88881154,177.45764383)(107.59381184,177.27764401)(107.23381746,177.12764422)
\curveto(107.11381232,177.07764421)(106.98381245,177.04264425)(106.84381746,177.02264422)
\curveto(106.71381272,177.00264429)(106.57881285,176.97764431)(106.43881746,176.94764422)
\curveto(106.37881305,176.93764435)(106.31881311,176.93264436)(106.25881746,176.93264422)
\curveto(106.19881323,176.93264436)(106.1338133,176.92764436)(106.06381746,176.91764422)
\curveto(106.0338134,176.90764438)(105.98381345,176.90764438)(105.91381746,176.91764422)
\lineto(105.76381746,176.91764422)
\lineto(105.61381746,176.91764422)
\curveto(105.5338139,176.93764435)(105.44881398,176.95264434)(105.35881746,176.96264422)
\curveto(105.27881415,176.96264433)(105.20381423,176.97264432)(105.13381746,176.99264422)
\curveto(105.09381434,177.00264429)(105.05881437,177.00764428)(105.02881746,177.00764422)
\curveto(105.00881442,176.99764429)(104.98381445,177.00264429)(104.95381746,177.02264422)
\lineto(104.68381746,177.08264422)
\curveto(104.59381484,177.11264418)(104.50881492,177.14264415)(104.42881746,177.17264422)
\curveto(103.84881558,177.41264388)(103.41381602,177.78264351)(103.12381746,178.28264422)
\curveto(103.04381639,178.41264288)(102.97881645,178.54764274)(102.92881746,178.68764422)
\curveto(102.88881654,178.82764246)(102.84381659,178.97764231)(102.79381746,179.13764422)
\curveto(102.77381666,179.21764207)(102.76881666,179.29764199)(102.77881746,179.37764422)
\curveto(102.79881663,179.45764183)(102.8338166,179.51264178)(102.88381746,179.54264422)
\curveto(102.91381652,179.56264173)(102.96881646,179.57764171)(103.04881746,179.58764422)
\curveto(103.1288163,179.60764168)(103.21381622,179.61764167)(103.30381746,179.61764422)
\curveto(103.39381604,179.62764166)(103.47881595,179.62764166)(103.55881746,179.61764422)
\curveto(103.64881578,179.60764168)(103.71881571,179.59764169)(103.76881746,179.58764422)
\curveto(103.78881564,179.57764171)(103.81381562,179.56264173)(103.84381746,179.54264422)
\curveto(103.88381555,179.52264177)(103.91381552,179.50264179)(103.93381746,179.48264422)
\curveto(103.99381544,179.40264189)(104.03881539,179.30764198)(104.06881746,179.19764422)
\curveto(104.10881532,179.0876422)(104.15381528,178.9876423)(104.20381746,178.89764422)
\curveto(104.45381498,178.50764278)(104.82381461,178.23764305)(105.31381746,178.08764422)
\curveto(105.38381405,178.06764322)(105.45381398,178.05264324)(105.52381746,178.04264422)
\curveto(105.60381383,178.04264325)(105.68381375,178.03264326)(105.76381746,178.01264422)
\curveto(105.80381363,178.00264329)(105.85881357,177.99764329)(105.92881746,177.99764422)
\curveto(106.00881342,177.99764329)(106.06381337,178.00264329)(106.09381746,178.01264422)
\curveto(106.12381331,178.02264327)(106.15381328,178.02764326)(106.18381746,178.02764422)
\lineto(106.28881746,178.02764422)
\curveto(106.36881306,178.04764324)(106.44381299,178.06764322)(106.51381746,178.08764422)
\curveto(106.59381284,178.10764318)(106.66881276,178.13264316)(106.73881746,178.16264422)
\curveto(107.08881234,178.31264298)(107.35881207,178.52764276)(107.54881746,178.80764422)
\curveto(107.73881169,179.0876422)(107.89381154,179.41264188)(108.01381746,179.78264422)
\curveto(108.04381139,179.86264143)(108.06381137,179.93764135)(108.07381746,180.00764422)
\curveto(108.09381134,180.07764121)(108.11381132,180.15264114)(108.13381746,180.23264422)
\curveto(108.15381128,180.32264097)(108.16881126,180.41764087)(108.17881746,180.51764422)
\curveto(108.19881123,180.62764066)(108.21881121,180.73264056)(108.23881746,180.83264422)
\curveto(108.24881118,180.88264041)(108.25381118,180.93264036)(108.25381746,180.98264422)
\curveto(108.26381117,181.04264025)(108.26881116,181.09764019)(108.26881746,181.14764422)
\curveto(108.28881114,181.20764008)(108.29881113,181.28264001)(108.29881746,181.37264422)
\curveto(108.29881113,181.47263982)(108.28881114,181.55263974)(108.26881746,181.61264422)
\curveto(108.23881119,181.70263959)(108.18881124,181.74263955)(108.11881746,181.73264422)
\curveto(108.05881137,181.72263957)(108.00381143,181.6926396)(107.95381746,181.64264422)
\curveto(107.87381156,181.5926397)(107.80381163,181.53263976)(107.74381746,181.46264422)
\curveto(107.69381174,181.3926399)(107.6288118,181.33263996)(107.54881746,181.28264422)
\curveto(107.38881204,181.17264012)(107.22381221,181.07264022)(107.05381746,180.98264422)
\curveto(106.88381255,180.90264039)(106.68881274,180.83264046)(106.46881746,180.77264422)
\curveto(106.36881306,180.74264055)(106.26881316,180.72764056)(106.16881746,180.72764422)
\curveto(106.07881335,180.72764056)(105.97881345,180.71764057)(105.86881746,180.69764422)
\lineto(105.71881746,180.69764422)
\curveto(105.66881376,180.71764057)(105.61881381,180.72264057)(105.56881746,180.71264422)
\curveto(105.5288139,180.70264059)(105.48881394,180.70264059)(105.44881746,180.71264422)
\curveto(105.41881401,180.72264057)(105.37381406,180.72764056)(105.31381746,180.72764422)
\curveto(105.25381418,180.73764055)(105.18881424,180.74764054)(105.11881746,180.75764422)
\lineto(104.93881746,180.78764422)
\curveto(104.48881494,180.90764038)(104.10881532,181.07264022)(103.79881746,181.28264422)
\curveto(103.5288159,181.47263982)(103.29881613,181.70263959)(103.10881746,181.97264422)
\curveto(102.9288165,182.25263904)(102.78381665,182.56763872)(102.67381746,182.91764422)
\lineto(102.61381746,183.12764422)
\curveto(102.60381683,183.20763808)(102.58881684,183.287638)(102.56881746,183.36764422)
\curveto(102.55881687,183.39763789)(102.55381688,183.42763786)(102.55381746,183.45764422)
\curveto(102.55381688,183.4876378)(102.54881688,183.51763777)(102.53881746,183.54764422)
\curveto(102.5288169,183.60763768)(102.52381691,183.66763762)(102.52381746,183.72764422)
\curveto(102.52381691,183.79763749)(102.51381692,183.85763743)(102.49381746,183.90764422)
\lineto(102.49381746,184.08764422)
\curveto(102.48381695,184.13763715)(102.47881695,184.20763708)(102.47881746,184.29764422)
\curveto(102.47881695,184.3876369)(102.48881694,184.45763683)(102.50881746,184.50764422)
\lineto(102.50881746,184.67264422)
\curveto(102.5288169,184.75263654)(102.53881689,184.82763646)(102.53881746,184.89764422)
\curveto(102.54881688,184.96763632)(102.56381687,185.03763625)(102.58381746,185.10764422)
\curveto(102.64381679,185.30763598)(102.70381673,185.49763579)(102.76381746,185.67764422)
\curveto(102.8338166,185.85763543)(102.92381651,186.02763526)(103.03381746,186.18764422)
\curveto(103.07381636,186.25763503)(103.11381632,186.32263497)(103.15381746,186.38264422)
\lineto(103.30381746,186.56264422)
\curveto(103.32381611,186.57263472)(103.34381609,186.5876347)(103.36381746,186.60764422)
\curveto(103.45381598,186.73763455)(103.56381587,186.84763444)(103.69381746,186.93764422)
\curveto(103.95381548,187.13763415)(104.21881521,187.292634)(104.48881746,187.40264422)
\curveto(104.56881486,187.44263385)(104.64881478,187.47263382)(104.72881746,187.49264422)
\curveto(104.81881461,187.52263377)(104.90881452,187.54763374)(104.99881746,187.56764422)
\curveto(105.09881433,187.59763369)(105.19881423,187.61763367)(105.29881746,187.62764422)
\curveto(105.39881403,187.63763365)(105.50381393,187.65263364)(105.61381746,187.67264422)
\curveto(105.64381379,187.68263361)(105.68381375,187.68263361)(105.73381746,187.67264422)
\curveto(105.79381364,187.66263363)(105.8338136,187.66763362)(105.85381746,187.68764422)
\curveto(106.57381286,187.70763358)(107.17381226,187.5926337)(107.65381746,187.34264422)
\curveto(108.1338113,187.0926342)(108.50881092,186.75263454)(108.77881746,186.32264422)
\curveto(108.86881056,186.18263511)(108.94881048,186.03763525)(109.01881746,185.88764422)
\curveto(109.08881034,185.73763555)(109.15881027,185.57763571)(109.22881746,185.40764422)
\curveto(109.27881015,185.26763602)(109.31881011,185.11763617)(109.34881746,184.95764422)
\curveto(109.37881005,184.79763649)(109.41381002,184.63763665)(109.45381746,184.47764422)
\curveto(109.47380996,184.42763686)(109.48380995,184.37263692)(109.48381746,184.31264422)
\curveto(109.48380995,184.26263703)(109.48880994,184.21263708)(109.49881746,184.16264422)
\curveto(109.51880991,184.10263719)(109.5288099,184.03763725)(109.52881746,183.96764422)
\curveto(109.5288099,183.90763738)(109.53880989,183.85263744)(109.55881746,183.80264422)
\lineto(109.55881746,183.63764422)
\curveto(109.57880985,183.5876377)(109.58380985,183.53763775)(109.57381746,183.48764422)
\curveto(109.56380987,183.43763785)(109.56880986,183.3876379)(109.58881746,183.33764422)
\curveto(109.58880984,183.31763797)(109.58380985,183.292638)(109.57381746,183.26264422)
\curveto(109.57380986,183.23263806)(109.57880985,183.20763808)(109.58881746,183.18764422)
\curveto(109.59880983,183.15763813)(109.59880983,183.12263817)(109.58881746,183.08264422)
\curveto(109.58880984,183.04263825)(109.59380984,183.00263829)(109.60381746,182.96264422)
\curveto(109.61380982,182.92263837)(109.61380982,182.87763841)(109.60381746,182.82764422)
\lineto(109.60381746,182.67764422)
\moveto(108.10381746,183.98264422)
\curveto(108.11381132,184.03263726)(108.11881131,184.0926372)(108.11881746,184.16264422)
\curveto(108.11881131,184.23263706)(108.11381132,184.292637)(108.10381746,184.34264422)
\curveto(108.09381134,184.3926369)(108.08881134,184.46763682)(108.08881746,184.56764422)
\curveto(108.06881136,184.64763664)(108.04881138,184.72263657)(108.02881746,184.79264422)
\curveto(108.01881141,184.86263643)(108.00381143,184.93263636)(107.98381746,185.00264422)
\curveto(107.84381159,185.43263586)(107.64881178,185.76763552)(107.39881746,186.00764422)
\curveto(107.15881227,186.24763504)(106.81381262,186.42763486)(106.36381746,186.54764422)
\curveto(106.27381316,186.56763472)(106.17381326,186.57763471)(106.06381746,186.57764422)
\lineto(105.73381746,186.57764422)
\curveto(105.71381372,186.55763473)(105.67881375,186.54763474)(105.62881746,186.54764422)
\curveto(105.57881385,186.55763473)(105.5338139,186.55763473)(105.49381746,186.54764422)
\curveto(105.41381402,186.52763476)(105.33881409,186.50763478)(105.26881746,186.48764422)
\lineto(105.05881746,186.42764422)
\curveto(104.76881466,186.29763499)(104.53881489,186.11763517)(104.36881746,185.88764422)
\curveto(104.19881523,185.66763562)(104.06381537,185.40763588)(103.96381746,185.10764422)
\curveto(103.9338155,185.01763627)(103.90881552,184.92263637)(103.88881746,184.82264422)
\curveto(103.87881555,184.73263656)(103.86381557,184.63763665)(103.84381746,184.53764422)
\lineto(103.84381746,184.40264422)
\curveto(103.81381562,184.292637)(103.80381563,184.15263714)(103.81381746,183.98264422)
\curveto(103.8338156,183.82263747)(103.85381558,183.6926376)(103.87381746,183.59264422)
\curveto(103.89381554,183.53263776)(103.90881552,183.47263782)(103.91881746,183.41264422)
\curveto(103.9288155,183.36263793)(103.94381549,183.31263798)(103.96381746,183.26264422)
\curveto(104.04381539,183.06263823)(104.13881529,182.87263842)(104.24881746,182.69264422)
\curveto(104.36881506,182.51263878)(104.50881492,182.36763892)(104.66881746,182.25764422)
\curveto(104.71881471,182.20763908)(104.77381466,182.16763912)(104.83381746,182.13764422)
\curveto(104.89381454,182.10763918)(104.95381448,182.07263922)(105.01381746,182.03264422)
\curveto(105.16381427,181.95263934)(105.34881408,181.8876394)(105.56881746,181.83764422)
\curveto(105.61881381,181.81763947)(105.65881377,181.81263948)(105.68881746,181.82264422)
\curveto(105.7288137,181.83263946)(105.77381366,181.82763946)(105.82381746,181.80764422)
\curveto(105.86381357,181.79763949)(105.91881351,181.7926395)(105.98881746,181.79264422)
\curveto(106.05881337,181.7926395)(106.11881331,181.79763949)(106.16881746,181.80764422)
\curveto(106.26881316,181.82763946)(106.36381307,181.84263945)(106.45381746,181.85264422)
\curveto(106.54381289,181.87263942)(106.6338128,181.90263939)(106.72381746,181.94264422)
\curveto(107.26381217,182.16263913)(107.65881177,182.55763873)(107.90881746,183.12764422)
\curveto(107.95881147,183.22763806)(107.99381144,183.32763796)(108.01381746,183.42764422)
\curveto(108.0338114,183.53763775)(108.05881137,183.64763764)(108.08881746,183.75764422)
\curveto(108.08881134,183.85763743)(108.09381134,183.93263736)(108.10381746,183.98264422)
}
}
{
\newrgbcolor{curcolor}{0 0 0}
\pscustom[linestyle=none,fillstyle=solid,fillcolor=curcolor]
{
\newpath
\moveto(120.81842684,185.60264422)
\curveto(120.61841654,185.31263598)(120.40841675,185.02763626)(120.18842684,184.74764422)
\curveto(119.97841718,184.46763682)(119.77341738,184.18263711)(119.57342684,183.89264422)
\curveto(118.97341818,183.04263825)(118.36841879,182.20263909)(117.75842684,181.37264422)
\curveto(117.14842001,180.55264074)(116.54342061,179.71764157)(115.94342684,178.86764422)
\lineto(115.43342684,178.14764422)
\lineto(114.92342684,177.45764422)
\curveto(114.84342231,177.34764394)(114.76342239,177.23264406)(114.68342684,177.11264422)
\curveto(114.60342255,176.9926443)(114.50842265,176.89764439)(114.39842684,176.82764422)
\curveto(114.3584228,176.80764448)(114.29342286,176.7926445)(114.20342684,176.78264422)
\curveto(114.12342303,176.76264453)(114.03342312,176.75264454)(113.93342684,176.75264422)
\curveto(113.83342332,176.75264454)(113.73842342,176.75764453)(113.64842684,176.76764422)
\curveto(113.56842359,176.77764451)(113.50842365,176.79764449)(113.46842684,176.82764422)
\curveto(113.43842372,176.84764444)(113.41342374,176.88264441)(113.39342684,176.93264422)
\curveto(113.38342377,176.97264432)(113.38842377,177.01764427)(113.40842684,177.06764422)
\curveto(113.44842371,177.14764414)(113.49342366,177.22264407)(113.54342684,177.29264422)
\curveto(113.60342355,177.37264392)(113.6584235,177.45264384)(113.70842684,177.53264422)
\curveto(113.94842321,177.87264342)(114.19342296,178.20764308)(114.44342684,178.53764422)
\curveto(114.69342246,178.86764242)(114.93342222,179.20264209)(115.16342684,179.54264422)
\curveto(115.32342183,179.76264153)(115.48342167,179.97764131)(115.64342684,180.18764422)
\curveto(115.80342135,180.39764089)(115.96342119,180.61264068)(116.12342684,180.83264422)
\curveto(116.48342067,181.35263994)(116.84842031,181.86263943)(117.21842684,182.36264422)
\curveto(117.58841957,182.86263843)(117.9584192,183.37263792)(118.32842684,183.89264422)
\curveto(118.46841869,184.0926372)(118.60841855,184.287637)(118.74842684,184.47764422)
\curveto(118.89841826,184.66763662)(119.04341811,184.86263643)(119.18342684,185.06264422)
\curveto(119.39341776,185.36263593)(119.60841755,185.66263563)(119.82842684,185.96264422)
\lineto(120.48842684,186.86264422)
\lineto(120.66842684,187.13264422)
\lineto(120.87842684,187.40264422)
\lineto(120.99842684,187.58264422)
\curveto(121.04841611,187.64263365)(121.09841606,187.69763359)(121.14842684,187.74764422)
\curveto(121.21841594,187.79763349)(121.29341586,187.83263346)(121.37342684,187.85264422)
\curveto(121.39341576,187.86263343)(121.41841574,187.86263343)(121.44842684,187.85264422)
\curveto(121.48841567,187.85263344)(121.51841564,187.86263343)(121.53842684,187.88264422)
\curveto(121.6584155,187.88263341)(121.79341536,187.87763341)(121.94342684,187.86764422)
\curveto(122.09341506,187.86763342)(122.18341497,187.82263347)(122.21342684,187.73264422)
\curveto(122.23341492,187.70263359)(122.23841492,187.66763362)(122.22842684,187.62764422)
\curveto(122.21841494,187.5876337)(122.20341495,187.55763373)(122.18342684,187.53764422)
\curveto(122.14341501,187.45763383)(122.10341505,187.3876339)(122.06342684,187.32764422)
\curveto(122.02341513,187.26763402)(121.97841518,187.20763408)(121.92842684,187.14764422)
\lineto(121.35842684,186.36764422)
\curveto(121.17841598,186.11763517)(120.99841616,185.86263543)(120.81842684,185.60264422)
\moveto(113.96342684,181.70264422)
\curveto(113.91342324,181.72263957)(113.86342329,181.72763956)(113.81342684,181.71764422)
\curveto(113.76342339,181.70763958)(113.71342344,181.71263958)(113.66342684,181.73264422)
\curveto(113.5534236,181.75263954)(113.44842371,181.77263952)(113.34842684,181.79264422)
\curveto(113.2584239,181.82263947)(113.16342399,181.86263943)(113.06342684,181.91264422)
\curveto(112.73342442,182.05263924)(112.47842468,182.24763904)(112.29842684,182.49764422)
\curveto(112.11842504,182.75763853)(111.97342518,183.06763822)(111.86342684,183.42764422)
\curveto(111.83342532,183.50763778)(111.81342534,183.5876377)(111.80342684,183.66764422)
\curveto(111.79342536,183.75763753)(111.77842538,183.84263745)(111.75842684,183.92264422)
\curveto(111.74842541,183.97263732)(111.74342541,184.03763725)(111.74342684,184.11764422)
\curveto(111.73342542,184.14763714)(111.72842543,184.17763711)(111.72842684,184.20764422)
\curveto(111.72842543,184.24763704)(111.72342543,184.28263701)(111.71342684,184.31264422)
\lineto(111.71342684,184.46264422)
\curveto(111.70342545,184.51263678)(111.69842546,184.57263672)(111.69842684,184.64264422)
\curveto(111.69842546,184.72263657)(111.70342545,184.7876365)(111.71342684,184.83764422)
\lineto(111.71342684,185.00264422)
\curveto(111.73342542,185.05263624)(111.73842542,185.09763619)(111.72842684,185.13764422)
\curveto(111.72842543,185.1876361)(111.73342542,185.23263606)(111.74342684,185.27264422)
\curveto(111.7534254,185.31263598)(111.7584254,185.34763594)(111.75842684,185.37764422)
\curveto(111.7584254,185.41763587)(111.76342539,185.45763583)(111.77342684,185.49764422)
\curveto(111.80342535,185.60763568)(111.82342533,185.71763557)(111.83342684,185.82764422)
\curveto(111.8534253,185.94763534)(111.88842527,186.06263523)(111.93842684,186.17264422)
\curveto(112.07842508,186.51263478)(112.23842492,186.7876345)(112.41842684,186.99764422)
\curveto(112.60842455,187.21763407)(112.87842428,187.39763389)(113.22842684,187.53764422)
\curveto(113.30842385,187.56763372)(113.39342376,187.5876337)(113.48342684,187.59764422)
\curveto(113.57342358,187.61763367)(113.66842349,187.63763365)(113.76842684,187.65764422)
\curveto(113.79842336,187.66763362)(113.8534233,187.66763362)(113.93342684,187.65764422)
\curveto(114.01342314,187.65763363)(114.06342309,187.66763362)(114.08342684,187.68764422)
\curveto(114.64342251,187.69763359)(115.09342206,187.5876337)(115.43342684,187.35764422)
\curveto(115.78342137,187.12763416)(116.04342111,186.82263447)(116.21342684,186.44264422)
\curveto(116.2534209,186.35263494)(116.28842087,186.25763503)(116.31842684,186.15764422)
\curveto(116.34842081,186.05763523)(116.37342078,185.95763533)(116.39342684,185.85764422)
\curveto(116.41342074,185.82763546)(116.41842074,185.79763549)(116.40842684,185.76764422)
\curveto(116.40842075,185.73763555)(116.41342074,185.70763558)(116.42342684,185.67764422)
\curveto(116.4534207,185.56763572)(116.47342068,185.44263585)(116.48342684,185.30264422)
\curveto(116.49342066,185.17263612)(116.50342065,185.03763625)(116.51342684,184.89764422)
\lineto(116.51342684,184.73264422)
\curveto(116.52342063,184.67263662)(116.52342063,184.61763667)(116.51342684,184.56764422)
\curveto(116.50342065,184.51763677)(116.49842066,184.46763682)(116.49842684,184.41764422)
\lineto(116.49842684,184.28264422)
\curveto(116.48842067,184.24263705)(116.48342067,184.20263709)(116.48342684,184.16264422)
\curveto(116.49342066,184.12263717)(116.48842067,184.07763721)(116.46842684,184.02764422)
\curveto(116.44842071,183.91763737)(116.42842073,183.81263748)(116.40842684,183.71264422)
\curveto(116.39842076,183.61263768)(116.37842078,183.51263778)(116.34842684,183.41264422)
\curveto(116.21842094,183.05263824)(116.0534211,182.73763855)(115.85342684,182.46764422)
\curveto(115.6534215,182.19763909)(115.37842178,181.9926393)(115.02842684,181.85264422)
\curveto(114.94842221,181.82263947)(114.86342229,181.79763949)(114.77342684,181.77764422)
\lineto(114.50342684,181.71764422)
\curveto(114.4534227,181.70763958)(114.40842275,181.70263959)(114.36842684,181.70264422)
\curveto(114.32842283,181.71263958)(114.28842287,181.71263958)(114.24842684,181.70264422)
\curveto(114.14842301,181.68263961)(114.0534231,181.68263961)(113.96342684,181.70264422)
\moveto(113.12342684,183.09764422)
\curveto(113.16342399,183.02763826)(113.20342395,182.96263833)(113.24342684,182.90264422)
\curveto(113.28342387,182.85263844)(113.33342382,182.80263849)(113.39342684,182.75264422)
\lineto(113.54342684,182.63264422)
\curveto(113.60342355,182.60263869)(113.66842349,182.57763871)(113.73842684,182.55764422)
\curveto(113.77842338,182.53763875)(113.81342334,182.52763876)(113.84342684,182.52764422)
\curveto(113.88342327,182.53763875)(113.92342323,182.53263876)(113.96342684,182.51264422)
\curveto(113.99342316,182.51263878)(114.03342312,182.50763878)(114.08342684,182.49764422)
\curveto(114.13342302,182.49763879)(114.17342298,182.50263879)(114.20342684,182.51264422)
\lineto(114.42842684,182.55764422)
\curveto(114.67842248,182.63763865)(114.86342229,182.76263853)(114.98342684,182.93264422)
\curveto(115.06342209,183.03263826)(115.13342202,183.16263813)(115.19342684,183.32264422)
\curveto(115.27342188,183.50263779)(115.33342182,183.72763756)(115.37342684,183.99764422)
\curveto(115.41342174,184.27763701)(115.42842173,184.55763673)(115.41842684,184.83764422)
\curveto(115.40842175,185.12763616)(115.37842178,185.40263589)(115.32842684,185.66264422)
\curveto(115.27842188,185.92263537)(115.20342195,186.13263516)(115.10342684,186.29264422)
\curveto(114.98342217,186.4926348)(114.83342232,186.64263465)(114.65342684,186.74264422)
\curveto(114.57342258,186.7926345)(114.48342267,186.82263447)(114.38342684,186.83264422)
\curveto(114.28342287,186.85263444)(114.17842298,186.86263443)(114.06842684,186.86264422)
\curveto(114.04842311,186.85263444)(114.02342313,186.84763444)(113.99342684,186.84764422)
\curveto(113.97342318,186.85763443)(113.9534232,186.85763443)(113.93342684,186.84764422)
\curveto(113.88342327,186.83763445)(113.83842332,186.82763446)(113.79842684,186.81764422)
\curveto(113.7584234,186.81763447)(113.71842344,186.80763448)(113.67842684,186.78764422)
\curveto(113.49842366,186.70763458)(113.34842381,186.5876347)(113.22842684,186.42764422)
\curveto(113.11842404,186.26763502)(113.02842413,186.0876352)(112.95842684,185.88764422)
\curveto(112.89842426,185.69763559)(112.8534243,185.47263582)(112.82342684,185.21264422)
\curveto(112.80342435,184.95263634)(112.79842436,184.6876366)(112.80842684,184.41764422)
\curveto(112.81842434,184.15763713)(112.84842431,183.90763738)(112.89842684,183.66764422)
\curveto(112.9584242,183.43763785)(113.03342412,183.24763804)(113.12342684,183.09764422)
\moveto(123.92342684,180.11264422)
\curveto(123.93341322,180.06264123)(123.93841322,179.97264132)(123.93842684,179.84264422)
\curveto(123.93841322,179.71264158)(123.92841323,179.62264167)(123.90842684,179.57264422)
\curveto(123.88841327,179.52264177)(123.88341327,179.46764182)(123.89342684,179.40764422)
\curveto(123.90341325,179.35764193)(123.90341325,179.30764198)(123.89342684,179.25764422)
\curveto(123.8534133,179.11764217)(123.82341333,178.98264231)(123.80342684,178.85264422)
\curveto(123.79341336,178.72264257)(123.76341339,178.60264269)(123.71342684,178.49264422)
\curveto(123.57341358,178.14264315)(123.40841375,177.84764344)(123.21842684,177.60764422)
\curveto(123.02841413,177.37764391)(122.7584144,177.1926441)(122.40842684,177.05264422)
\curveto(122.32841483,177.02264427)(122.24341491,177.00264429)(122.15342684,176.99264422)
\curveto(122.06341509,176.97264432)(121.97841518,176.95264434)(121.89842684,176.93264422)
\curveto(121.84841531,176.92264437)(121.79841536,176.91764437)(121.74842684,176.91764422)
\curveto(121.69841546,176.91764437)(121.64841551,176.91264438)(121.59842684,176.90264422)
\curveto(121.56841559,176.8926444)(121.51841564,176.8926444)(121.44842684,176.90264422)
\curveto(121.37841578,176.90264439)(121.32841583,176.90764438)(121.29842684,176.91764422)
\curveto(121.23841592,176.93764435)(121.17841598,176.94764434)(121.11842684,176.94764422)
\curveto(121.06841609,176.93764435)(121.01841614,176.94264435)(120.96842684,176.96264422)
\curveto(120.87841628,176.98264431)(120.78841637,177.00764428)(120.69842684,177.03764422)
\curveto(120.61841654,177.05764423)(120.53841662,177.0876442)(120.45842684,177.12764422)
\curveto(120.13841702,177.26764402)(119.88841727,177.46264383)(119.70842684,177.71264422)
\curveto(119.52841763,177.97264332)(119.37841778,178.27764301)(119.25842684,178.62764422)
\curveto(119.23841792,178.70764258)(119.22341793,178.7926425)(119.21342684,178.88264422)
\curveto(119.20341795,178.97264232)(119.18841797,179.05764223)(119.16842684,179.13764422)
\curveto(119.158418,179.16764212)(119.153418,179.19764209)(119.15342684,179.22764422)
\lineto(119.15342684,179.33264422)
\curveto(119.13341802,179.41264188)(119.12341803,179.4926418)(119.12342684,179.57264422)
\lineto(119.12342684,179.70764422)
\curveto(119.10341805,179.80764148)(119.10341805,179.90764138)(119.12342684,180.00764422)
\lineto(119.12342684,180.18764422)
\curveto(119.13341802,180.23764105)(119.13841802,180.28264101)(119.13842684,180.32264422)
\curveto(119.13841802,180.37264092)(119.14341801,180.41764087)(119.15342684,180.45764422)
\curveto(119.16341799,180.49764079)(119.16841799,180.53264076)(119.16842684,180.56264422)
\curveto(119.16841799,180.60264069)(119.17341798,180.64264065)(119.18342684,180.68264422)
\lineto(119.24342684,181.01264422)
\curveto(119.26341789,181.13264016)(119.29341786,181.24264005)(119.33342684,181.34264422)
\curveto(119.47341768,181.67263962)(119.63341752,181.94763934)(119.81342684,182.16764422)
\curveto(120.00341715,182.39763889)(120.26341689,182.58263871)(120.59342684,182.72264422)
\curveto(120.67341648,182.76263853)(120.7584164,182.7876385)(120.84842684,182.79764422)
\lineto(121.14842684,182.85764422)
\lineto(121.28342684,182.85764422)
\curveto(121.33341582,182.86763842)(121.38341577,182.87263842)(121.43342684,182.87264422)
\curveto(122.00341515,182.8926384)(122.46341469,182.7876385)(122.81342684,182.55764422)
\curveto(123.17341398,182.33763895)(123.43841372,182.03763925)(123.60842684,181.65764422)
\curveto(123.6584135,181.55763973)(123.69841346,181.45763983)(123.72842684,181.35764422)
\curveto(123.7584134,181.25764003)(123.78841337,181.15264014)(123.81842684,181.04264422)
\curveto(123.82841333,181.00264029)(123.83341332,180.96764032)(123.83342684,180.93764422)
\curveto(123.83341332,180.91764037)(123.83841332,180.8876404)(123.84842684,180.84764422)
\curveto(123.86841329,180.77764051)(123.87841328,180.70264059)(123.87842684,180.62264422)
\curveto(123.87841328,180.54264075)(123.88841327,180.46264083)(123.90842684,180.38264422)
\curveto(123.90841325,180.33264096)(123.90841325,180.287641)(123.90842684,180.24764422)
\curveto(123.90841325,180.20764108)(123.91341324,180.16264113)(123.92342684,180.11264422)
\moveto(122.81342684,179.67764422)
\curveto(122.82341433,179.72764156)(122.82841433,179.80264149)(122.82842684,179.90264422)
\curveto(122.83841432,180.00264129)(122.83341432,180.07764121)(122.81342684,180.12764422)
\curveto(122.79341436,180.1876411)(122.78841437,180.24264105)(122.79842684,180.29264422)
\curveto(122.81841434,180.35264094)(122.81841434,180.41264088)(122.79842684,180.47264422)
\curveto(122.78841437,180.50264079)(122.78341437,180.53764075)(122.78342684,180.57764422)
\curveto(122.78341437,180.61764067)(122.77841438,180.65764063)(122.76842684,180.69764422)
\curveto(122.74841441,180.77764051)(122.72841443,180.85264044)(122.70842684,180.92264422)
\curveto(122.69841446,181.00264029)(122.68341447,181.08264021)(122.66342684,181.16264422)
\curveto(122.63341452,181.22264007)(122.60841455,181.28264001)(122.58842684,181.34264422)
\curveto(122.56841459,181.40263989)(122.53841462,181.46263983)(122.49842684,181.52264422)
\curveto(122.39841476,181.6926396)(122.26841489,181.82763946)(122.10842684,181.92764422)
\curveto(122.02841513,181.97763931)(121.93341522,182.01263928)(121.82342684,182.03264422)
\curveto(121.71341544,182.05263924)(121.58841557,182.06263923)(121.44842684,182.06264422)
\curveto(121.42841573,182.05263924)(121.40341575,182.04763924)(121.37342684,182.04764422)
\curveto(121.34341581,182.05763923)(121.31341584,182.05763923)(121.28342684,182.04764422)
\lineto(121.13342684,181.98764422)
\curveto(121.08341607,181.97763931)(121.03841612,181.96263933)(120.99842684,181.94264422)
\curveto(120.80841635,181.83263946)(120.66341649,181.6876396)(120.56342684,181.50764422)
\curveto(120.47341668,181.32763996)(120.39341676,181.12264017)(120.32342684,180.89264422)
\curveto(120.28341687,180.76264053)(120.26341689,180.62764066)(120.26342684,180.48764422)
\curveto(120.26341689,180.35764093)(120.2534169,180.21264108)(120.23342684,180.05264422)
\curveto(120.22341693,180.00264129)(120.21341694,179.94264135)(120.20342684,179.87264422)
\curveto(120.20341695,179.80264149)(120.21341694,179.74264155)(120.23342684,179.69264422)
\lineto(120.23342684,179.52764422)
\lineto(120.23342684,179.34764422)
\curveto(120.24341691,179.29764199)(120.2534169,179.24264205)(120.26342684,179.18264422)
\curveto(120.27341688,179.13264216)(120.27841688,179.07764221)(120.27842684,179.01764422)
\curveto(120.28841687,178.95764233)(120.30341685,178.90264239)(120.32342684,178.85264422)
\curveto(120.37341678,178.66264263)(120.43341672,178.4876428)(120.50342684,178.32764422)
\curveto(120.57341658,178.16764312)(120.67841648,178.03764325)(120.81842684,177.93764422)
\curveto(120.94841621,177.83764345)(121.08841607,177.76764352)(121.23842684,177.72764422)
\curveto(121.26841589,177.71764357)(121.29341586,177.71264358)(121.31342684,177.71264422)
\curveto(121.34341581,177.72264357)(121.37341578,177.72264357)(121.40342684,177.71264422)
\curveto(121.42341573,177.71264358)(121.4534157,177.70764358)(121.49342684,177.69764422)
\curveto(121.53341562,177.69764359)(121.56841559,177.70264359)(121.59842684,177.71264422)
\curveto(121.63841552,177.72264357)(121.67841548,177.72764356)(121.71842684,177.72764422)
\curveto(121.7584154,177.72764356)(121.79841536,177.73764355)(121.83842684,177.75764422)
\curveto(122.07841508,177.83764345)(122.27341488,177.97264332)(122.42342684,178.16264422)
\curveto(122.54341461,178.34264295)(122.63341452,178.54764274)(122.69342684,178.77764422)
\curveto(122.71341444,178.84764244)(122.72841443,178.91764237)(122.73842684,178.98764422)
\curveto(122.74841441,179.06764222)(122.76341439,179.14764214)(122.78342684,179.22764422)
\curveto(122.78341437,179.287642)(122.78841437,179.33264196)(122.79842684,179.36264422)
\curveto(122.79841436,179.38264191)(122.79841436,179.40764188)(122.79842684,179.43764422)
\curveto(122.79841436,179.47764181)(122.80341435,179.50764178)(122.81342684,179.52764422)
\lineto(122.81342684,179.67764422)
}
}
{
\newrgbcolor{curcolor}{0 0 0}
\pscustom[linestyle=none,fillstyle=solid,fillcolor=curcolor]
{
\newpath
\moveto(164.08679964,644.08163348)
\lineto(168.88679964,644.08163348)
\lineto(169.89179964,644.08163348)
\curveto(170.03179254,644.08162305)(170.15179242,644.07162306)(170.25179964,644.05163348)
\curveto(170.36179221,644.04162309)(170.44179213,643.99662314)(170.49179964,643.91663348)
\curveto(170.51179206,643.87662326)(170.52179205,643.82662331)(170.52179964,643.76663348)
\curveto(170.53179204,643.70662343)(170.53679203,643.64162349)(170.53679964,643.57163348)
\lineto(170.53679964,643.30163348)
\curveto(170.53679203,643.21162392)(170.52679204,643.131624)(170.50679964,643.06163348)
\curveto(170.4667921,642.98162415)(170.42179215,642.91162422)(170.37179964,642.85163348)
\lineto(170.22179964,642.67163348)
\curveto(170.19179238,642.62162451)(170.15679241,642.58162455)(170.11679964,642.55163348)
\curveto(170.07679249,642.52162461)(170.03679253,642.48162465)(169.99679964,642.43163348)
\curveto(169.91679265,642.32162481)(169.83179274,642.21162492)(169.74179964,642.10163348)
\curveto(169.65179292,642.00162513)(169.566793,641.89662524)(169.48679964,641.78663348)
\curveto(169.34679322,641.58662555)(169.20679336,641.37662576)(169.06679964,641.15663348)
\curveto(168.92679364,640.94662619)(168.78679378,640.7316264)(168.64679964,640.51163348)
\curveto(168.59679397,640.42162671)(168.54679402,640.32662681)(168.49679964,640.22663348)
\curveto(168.44679412,640.12662701)(168.39179418,640.0316271)(168.33179964,639.94163348)
\curveto(168.31179426,639.92162721)(168.30179427,639.89662724)(168.30179964,639.86663348)
\curveto(168.30179427,639.8366273)(168.29179428,639.81162732)(168.27179964,639.79163348)
\curveto(168.20179437,639.69162744)(168.13679443,639.57662756)(168.07679964,639.44663348)
\curveto(168.01679455,639.32662781)(167.96179461,639.21162792)(167.91179964,639.10163348)
\curveto(167.81179476,638.87162826)(167.71679485,638.6366285)(167.62679964,638.39663348)
\curveto(167.53679503,638.15662898)(167.43679513,637.91662922)(167.32679964,637.67663348)
\curveto(167.30679526,637.62662951)(167.29179528,637.58162955)(167.28179964,637.54163348)
\curveto(167.28179529,637.50162963)(167.2717953,637.45662968)(167.25179964,637.40663348)
\curveto(167.20179537,637.28662985)(167.15679541,637.16162997)(167.11679964,637.03163348)
\curveto(167.08679548,636.91163022)(167.05179552,636.79163034)(167.01179964,636.67163348)
\curveto(166.93179564,636.44163069)(166.8667957,636.20163093)(166.81679964,635.95163348)
\curveto(166.77679579,635.71163142)(166.72679584,635.47163166)(166.66679964,635.23163348)
\curveto(166.62679594,635.08163205)(166.60179597,634.9316322)(166.59179964,634.78163348)
\curveto(166.58179599,634.6316325)(166.56179601,634.48163265)(166.53179964,634.33163348)
\curveto(166.52179605,634.29163284)(166.51679605,634.2316329)(166.51679964,634.15163348)
\curveto(166.48679608,634.0316331)(166.45679611,633.9316332)(166.42679964,633.85163348)
\curveto(166.39679617,633.77163336)(166.32679624,633.71663342)(166.21679964,633.68663348)
\curveto(166.1667964,633.66663347)(166.11179646,633.65663348)(166.05179964,633.65663348)
\lineto(165.85679964,633.65663348)
\curveto(165.71679685,633.65663348)(165.57679699,633.66163347)(165.43679964,633.67163348)
\curveto(165.30679726,633.68163345)(165.21179736,633.72663341)(165.15179964,633.80663348)
\curveto(165.11179746,633.86663327)(165.09179748,633.95163318)(165.09179964,634.06163348)
\curveto(165.10179747,634.17163296)(165.11679745,634.26663287)(165.13679964,634.34663348)
\lineto(165.13679964,634.42163348)
\curveto(165.14679742,634.45163268)(165.15179742,634.48163265)(165.15179964,634.51163348)
\curveto(165.1717974,634.59163254)(165.18179739,634.66663247)(165.18179964,634.73663348)
\curveto(165.18179739,634.80663233)(165.19179738,634.87663226)(165.21179964,634.94663348)
\curveto(165.26179731,635.136632)(165.30179727,635.32163181)(165.33179964,635.50163348)
\curveto(165.36179721,635.69163144)(165.40179717,635.87163126)(165.45179964,636.04163348)
\curveto(165.4717971,636.09163104)(165.48179709,636.131631)(165.48179964,636.16163348)
\curveto(165.48179709,636.19163094)(165.48679708,636.22663091)(165.49679964,636.26663348)
\curveto(165.59679697,636.56663057)(165.68679688,636.86163027)(165.76679964,637.15163348)
\curveto(165.85679671,637.44162969)(165.96179661,637.72162941)(166.08179964,637.99163348)
\curveto(166.34179623,638.57162856)(166.61179596,639.12162801)(166.89179964,639.64163348)
\curveto(167.1717954,640.17162696)(167.48179509,640.67662646)(167.82179964,641.15663348)
\curveto(167.96179461,641.35662578)(168.11179446,641.54662559)(168.27179964,641.72663348)
\curveto(168.43179414,641.91662522)(168.58179399,642.10662503)(168.72179964,642.29663348)
\curveto(168.76179381,642.34662479)(168.79679377,642.39162474)(168.82679964,642.43163348)
\curveto(168.8667937,642.48162465)(168.90179367,642.5316246)(168.93179964,642.58163348)
\curveto(168.94179363,642.60162453)(168.95179362,642.62662451)(168.96179964,642.65663348)
\curveto(168.98179359,642.68662445)(168.98179359,642.71662442)(168.96179964,642.74663348)
\curveto(168.94179363,642.80662433)(168.90679366,642.84162429)(168.85679964,642.85163348)
\curveto(168.80679376,642.87162426)(168.75679381,642.89162424)(168.70679964,642.91163348)
\lineto(168.60179964,642.91163348)
\curveto(168.56179401,642.92162421)(168.51179406,642.92162421)(168.45179964,642.91163348)
\lineto(168.30179964,642.91163348)
\lineto(167.70179964,642.91163348)
\lineto(165.06179964,642.91163348)
\lineto(164.32679964,642.91163348)
\lineto(164.08679964,642.91163348)
\curveto(164.01679855,642.92162421)(163.95679861,642.9366242)(163.90679964,642.95663348)
\curveto(163.81679875,642.99662414)(163.75679881,643.05662408)(163.72679964,643.13663348)
\curveto(163.67679889,643.2366239)(163.66179891,643.38162375)(163.68179964,643.57163348)
\curveto(163.70179887,643.77162336)(163.73679883,643.90662323)(163.78679964,643.97663348)
\curveto(163.80679876,643.99662314)(163.83179874,644.01162312)(163.86179964,644.02163348)
\lineto(163.98179964,644.08163348)
\curveto(164.00179857,644.08162305)(164.01679855,644.07662306)(164.02679964,644.06663348)
\curveto(164.04679852,644.06662307)(164.0667985,644.07162306)(164.08679964,644.08163348)
}
}
{
\newrgbcolor{curcolor}{0 0 0}
\pscustom[linestyle=none,fillstyle=solid,fillcolor=curcolor]
{
\newpath
\moveto(172.93140901,635.30663348)
\lineto(173.23140901,635.30663348)
\curveto(173.34140695,635.31663182)(173.44640685,635.31663182)(173.54640901,635.30663348)
\curveto(173.65640664,635.30663183)(173.75640654,635.29663184)(173.84640901,635.27663348)
\curveto(173.93640636,635.26663187)(174.00640629,635.24163189)(174.05640901,635.20163348)
\curveto(174.07640622,635.18163195)(174.0914062,635.15163198)(174.10140901,635.11163348)
\curveto(174.12140617,635.07163206)(174.14140615,635.02663211)(174.16140901,634.97663348)
\lineto(174.16140901,634.90163348)
\curveto(174.17140612,634.85163228)(174.17140612,634.79663234)(174.16140901,634.73663348)
\lineto(174.16140901,634.58663348)
\lineto(174.16140901,634.10663348)
\curveto(174.16140613,633.9366332)(174.12140617,633.81663332)(174.04140901,633.74663348)
\curveto(173.97140632,633.69663344)(173.88140641,633.67163346)(173.77140901,633.67163348)
\lineto(173.44140901,633.67163348)
\lineto(172.99140901,633.67163348)
\curveto(172.84140745,633.67163346)(172.72640757,633.70163343)(172.64640901,633.76163348)
\curveto(172.60640769,633.79163334)(172.57640772,633.84163329)(172.55640901,633.91163348)
\curveto(172.53640776,633.99163314)(172.52140777,634.07663306)(172.51140901,634.16663348)
\lineto(172.51140901,634.45163348)
\curveto(172.52140777,634.55163258)(172.52640777,634.6366325)(172.52640901,634.70663348)
\lineto(172.52640901,634.90163348)
\curveto(172.52640777,634.96163217)(172.53640776,635.01663212)(172.55640901,635.06663348)
\curveto(172.5964077,635.17663196)(172.66640763,635.24663189)(172.76640901,635.27663348)
\curveto(172.7964075,635.27663186)(172.85140744,635.28663185)(172.93140901,635.30663348)
}
}
{
\newrgbcolor{curcolor}{0 0 0}
\pscustom[linestyle=none,fillstyle=solid,fillcolor=curcolor]
{
\newpath
\moveto(183.01656526,637.16663348)
\curveto(183.08655762,637.11663002)(183.12655758,637.04663009)(183.13656526,636.95663348)
\curveto(183.15655755,636.86663027)(183.16655754,636.76163037)(183.16656526,636.64163348)
\curveto(183.16655754,636.59163054)(183.16155754,636.54163059)(183.15156526,636.49163348)
\curveto(183.15155755,636.44163069)(183.14155756,636.39663074)(183.12156526,636.35663348)
\curveto(183.09155761,636.26663087)(183.03155767,636.20663093)(182.94156526,636.17663348)
\curveto(182.86155784,636.15663098)(182.76655794,636.14663099)(182.65656526,636.14663348)
\lineto(182.34156526,636.14663348)
\curveto(182.23155847,636.15663098)(182.12655858,636.14663099)(182.02656526,636.11663348)
\curveto(181.88655882,636.08663105)(181.79655891,636.00663113)(181.75656526,635.87663348)
\curveto(181.73655897,635.80663133)(181.72655898,635.72163141)(181.72656526,635.62163348)
\lineto(181.72656526,635.35163348)
\lineto(181.72656526,634.40663348)
\lineto(181.72656526,634.07663348)
\curveto(181.72655898,633.96663317)(181.706559,633.88163325)(181.66656526,633.82163348)
\curveto(181.62655908,633.76163337)(181.57655913,633.72163341)(181.51656526,633.70163348)
\curveto(181.46655924,633.69163344)(181.4015593,633.67663346)(181.32156526,633.65663348)
\lineto(181.12656526,633.65663348)
\curveto(181.0065597,633.65663348)(180.9015598,633.66163347)(180.81156526,633.67163348)
\curveto(180.72155998,633.69163344)(180.65156005,633.74163339)(180.60156526,633.82163348)
\curveto(180.57156013,633.87163326)(180.55656015,633.94163319)(180.55656526,634.03163348)
\lineto(180.55656526,634.33163348)
\lineto(180.55656526,635.36663348)
\curveto(180.55656015,635.52663161)(180.54656016,635.67163146)(180.52656526,635.80163348)
\curveto(180.51656019,635.94163119)(180.46156024,636.0366311)(180.36156526,636.08663348)
\curveto(180.31156039,636.10663103)(180.24156046,636.12163101)(180.15156526,636.13163348)
\curveto(180.07156063,636.14163099)(179.98156072,636.14663099)(179.88156526,636.14663348)
\lineto(179.59656526,636.14663348)
\lineto(179.35656526,636.14663348)
\lineto(177.09156526,636.14663348)
\curveto(177.0015637,636.14663099)(176.89656381,636.14163099)(176.77656526,636.13163348)
\lineto(176.44656526,636.13163348)
\curveto(176.33656437,636.131631)(176.23656447,636.14163099)(176.14656526,636.16163348)
\curveto(176.05656465,636.18163095)(175.99656471,636.21663092)(175.96656526,636.26663348)
\curveto(175.91656479,636.3366308)(175.89156481,636.4316307)(175.89156526,636.55163348)
\lineto(175.89156526,636.89663348)
\lineto(175.89156526,637.16663348)
\curveto(175.93156477,637.3366298)(175.98656472,637.47662966)(176.05656526,637.58663348)
\curveto(176.12656458,637.69662944)(176.2065645,637.81162932)(176.29656526,637.93163348)
\lineto(176.65656526,638.47163348)
\curveto(177.09656361,639.10162803)(177.53156317,639.72162741)(177.96156526,640.33163348)
\lineto(179.28156526,642.19163348)
\curveto(179.44156126,642.42162471)(179.59656111,642.64162449)(179.74656526,642.85163348)
\curveto(179.89656081,643.07162406)(180.05156065,643.29662384)(180.21156526,643.52663348)
\curveto(180.26156044,643.59662354)(180.31156039,643.66162347)(180.36156526,643.72163348)
\curveto(180.41156029,643.79162334)(180.46156024,643.86662327)(180.51156526,643.94663348)
\lineto(180.57156526,644.03663348)
\curveto(180.6015601,644.07662306)(180.63156007,644.10662303)(180.66156526,644.12663348)
\curveto(180.70156,644.15662298)(180.74155996,644.17662296)(180.78156526,644.18663348)
\curveto(180.82155988,644.20662293)(180.86655984,644.22662291)(180.91656526,644.24663348)
\curveto(180.93655977,644.24662289)(180.95655975,644.24162289)(180.97656526,644.23163348)
\curveto(181.0065597,644.2316229)(181.03155967,644.24162289)(181.05156526,644.26163348)
\curveto(181.18155952,644.26162287)(181.3015594,644.25662288)(181.41156526,644.24663348)
\curveto(181.52155918,644.2366229)(181.6015591,644.19162294)(181.65156526,644.11163348)
\curveto(181.69155901,644.06162307)(181.71155899,643.99162314)(181.71156526,643.90163348)
\curveto(181.72155898,643.81162332)(181.72655898,643.71662342)(181.72656526,643.61663348)
\lineto(181.72656526,638.15663348)
\curveto(181.72655898,638.08662905)(181.72155898,638.01162912)(181.71156526,637.93163348)
\curveto(181.71155899,637.86162927)(181.71655899,637.79162934)(181.72656526,637.72163348)
\lineto(181.72656526,637.61663348)
\curveto(181.74655896,637.56662957)(181.76155894,637.51162962)(181.77156526,637.45163348)
\curveto(181.78155892,637.40162973)(181.8065589,637.36162977)(181.84656526,637.33163348)
\curveto(181.91655879,637.28162985)(182.0015587,637.25162988)(182.10156526,637.24163348)
\lineto(182.43156526,637.24163348)
\curveto(182.54155816,637.24162989)(182.64655806,637.2366299)(182.74656526,637.22663348)
\curveto(182.85655785,637.22662991)(182.94655776,637.20662993)(183.01656526,637.16663348)
\moveto(180.45156526,637.36163348)
\curveto(180.53156017,637.47162966)(180.56656014,637.64162949)(180.55656526,637.87163348)
\lineto(180.55656526,638.48663348)
\lineto(180.55656526,640.96163348)
\lineto(180.55656526,641.27663348)
\curveto(180.56656014,641.39662574)(180.56156014,641.49662564)(180.54156526,641.57663348)
\lineto(180.54156526,641.72663348)
\curveto(180.54156016,641.81662532)(180.52656018,641.90162523)(180.49656526,641.98163348)
\curveto(180.48656022,642.00162513)(180.47656023,642.01162512)(180.46656526,642.01163348)
\lineto(180.42156526,642.05663348)
\curveto(180.4015603,642.06662507)(180.37156033,642.07162506)(180.33156526,642.07163348)
\curveto(180.31156039,642.05162508)(180.29156041,642.0366251)(180.27156526,642.02663348)
\curveto(180.26156044,642.02662511)(180.24656046,642.02162511)(180.22656526,642.01163348)
\curveto(180.16656054,641.96162517)(180.1065606,641.89162524)(180.04656526,641.80163348)
\curveto(179.98656072,641.71162542)(179.93156077,641.6316255)(179.88156526,641.56163348)
\curveto(179.78156092,641.42162571)(179.68656102,641.27662586)(179.59656526,641.12663348)
\curveto(179.5065612,640.98662615)(179.41156129,640.84662629)(179.31156526,640.70663348)
\lineto(178.77156526,639.92663348)
\curveto(178.6015621,639.66662747)(178.42656228,639.40662773)(178.24656526,639.14663348)
\curveto(178.16656254,639.0366281)(178.09156261,638.9316282)(178.02156526,638.83163348)
\lineto(177.81156526,638.53163348)
\curveto(177.76156294,638.45162868)(177.71156299,638.37662876)(177.66156526,638.30663348)
\curveto(177.62156308,638.2366289)(177.57656313,638.16162897)(177.52656526,638.08163348)
\curveto(177.47656323,638.02162911)(177.42656328,637.95662918)(177.37656526,637.88663348)
\curveto(177.33656337,637.82662931)(177.29656341,637.75662938)(177.25656526,637.67663348)
\curveto(177.21656349,637.61662952)(177.19156351,637.54662959)(177.18156526,637.46663348)
\curveto(177.17156353,637.39662974)(177.2065635,637.34162979)(177.28656526,637.30163348)
\curveto(177.35656335,637.25162988)(177.46656324,637.22662991)(177.61656526,637.22663348)
\curveto(177.77656293,637.2366299)(177.91156279,637.24162989)(178.02156526,637.24163348)
\lineto(179.70156526,637.24163348)
\lineto(180.13656526,637.24163348)
\curveto(180.28656042,637.24162989)(180.39156031,637.28162985)(180.45156526,637.36163348)
}
}
{
\newrgbcolor{curcolor}{0 0 0}
\pscustom[linestyle=none,fillstyle=solid,fillcolor=curcolor]
{
\newpath
\moveto(194.29117464,642.19163348)
\curveto(194.09116434,641.90162523)(193.88116455,641.61662552)(193.66117464,641.33663348)
\curveto(193.45116498,641.05662608)(193.24616518,640.77162636)(193.04617464,640.48163348)
\curveto(192.44616598,639.6316275)(191.84116659,638.79162834)(191.23117464,637.96163348)
\curveto(190.62116781,637.14162999)(190.01616841,636.30663083)(189.41617464,635.45663348)
\lineto(188.90617464,634.73663348)
\lineto(188.39617464,634.04663348)
\curveto(188.31617011,633.9366332)(188.23617019,633.82163331)(188.15617464,633.70163348)
\curveto(188.07617035,633.58163355)(187.98117045,633.48663365)(187.87117464,633.41663348)
\curveto(187.8311706,633.39663374)(187.76617066,633.38163375)(187.67617464,633.37163348)
\curveto(187.59617083,633.35163378)(187.50617092,633.34163379)(187.40617464,633.34163348)
\curveto(187.30617112,633.34163379)(187.21117122,633.34663379)(187.12117464,633.35663348)
\curveto(187.04117139,633.36663377)(186.98117145,633.38663375)(186.94117464,633.41663348)
\curveto(186.91117152,633.4366337)(186.88617154,633.47163366)(186.86617464,633.52163348)
\curveto(186.85617157,633.56163357)(186.86117157,633.60663353)(186.88117464,633.65663348)
\curveto(186.92117151,633.7366334)(186.96617146,633.81163332)(187.01617464,633.88163348)
\curveto(187.07617135,633.96163317)(187.1311713,634.04163309)(187.18117464,634.12163348)
\curveto(187.42117101,634.46163267)(187.66617076,634.79663234)(187.91617464,635.12663348)
\curveto(188.16617026,635.45663168)(188.40617002,635.79163134)(188.63617464,636.13163348)
\curveto(188.79616963,636.35163078)(188.95616947,636.56663057)(189.11617464,636.77663348)
\curveto(189.27616915,636.98663015)(189.43616899,637.20162993)(189.59617464,637.42163348)
\curveto(189.95616847,637.94162919)(190.32116811,638.45162868)(190.69117464,638.95163348)
\curveto(191.06116737,639.45162768)(191.431167,639.96162717)(191.80117464,640.48163348)
\curveto(191.94116649,640.68162645)(192.08116635,640.87662626)(192.22117464,641.06663348)
\curveto(192.37116606,641.25662588)(192.51616591,641.45162568)(192.65617464,641.65163348)
\curveto(192.86616556,641.95162518)(193.08116535,642.25162488)(193.30117464,642.55163348)
\lineto(193.96117464,643.45163348)
\lineto(194.14117464,643.72163348)
\lineto(194.35117464,643.99163348)
\lineto(194.47117464,644.17163348)
\curveto(194.52116391,644.2316229)(194.57116386,644.28662285)(194.62117464,644.33663348)
\curveto(194.69116374,644.38662275)(194.76616366,644.42162271)(194.84617464,644.44163348)
\curveto(194.86616356,644.45162268)(194.89116354,644.45162268)(194.92117464,644.44163348)
\curveto(194.96116347,644.44162269)(194.99116344,644.45162268)(195.01117464,644.47163348)
\curveto(195.1311633,644.47162266)(195.26616316,644.46662267)(195.41617464,644.45663348)
\curveto(195.56616286,644.45662268)(195.65616277,644.41162272)(195.68617464,644.32163348)
\curveto(195.70616272,644.29162284)(195.71116272,644.25662288)(195.70117464,644.21663348)
\curveto(195.69116274,644.17662296)(195.67616275,644.14662299)(195.65617464,644.12663348)
\curveto(195.61616281,644.04662309)(195.57616285,643.97662316)(195.53617464,643.91663348)
\curveto(195.49616293,643.85662328)(195.45116298,643.79662334)(195.40117464,643.73663348)
\lineto(194.83117464,642.95663348)
\curveto(194.65116378,642.70662443)(194.47116396,642.45162468)(194.29117464,642.19163348)
\moveto(187.43617464,638.29163348)
\curveto(187.38617104,638.31162882)(187.33617109,638.31662882)(187.28617464,638.30663348)
\curveto(187.23617119,638.29662884)(187.18617124,638.30162883)(187.13617464,638.32163348)
\curveto(187.0261714,638.34162879)(186.92117151,638.36162877)(186.82117464,638.38163348)
\curveto(186.7311717,638.41162872)(186.63617179,638.45162868)(186.53617464,638.50163348)
\curveto(186.20617222,638.64162849)(185.95117248,638.8366283)(185.77117464,639.08663348)
\curveto(185.59117284,639.34662779)(185.44617298,639.65662748)(185.33617464,640.01663348)
\curveto(185.30617312,640.09662704)(185.28617314,640.17662696)(185.27617464,640.25663348)
\curveto(185.26617316,640.34662679)(185.25117318,640.4316267)(185.23117464,640.51163348)
\curveto(185.22117321,640.56162657)(185.21617321,640.62662651)(185.21617464,640.70663348)
\curveto(185.20617322,640.7366264)(185.20117323,640.76662637)(185.20117464,640.79663348)
\curveto(185.20117323,640.8366263)(185.19617323,640.87162626)(185.18617464,640.90163348)
\lineto(185.18617464,641.05163348)
\curveto(185.17617325,641.10162603)(185.17117326,641.16162597)(185.17117464,641.23163348)
\curveto(185.17117326,641.31162582)(185.17617325,641.37662576)(185.18617464,641.42663348)
\lineto(185.18617464,641.59163348)
\curveto(185.20617322,641.64162549)(185.21117322,641.68662545)(185.20117464,641.72663348)
\curveto(185.20117323,641.77662536)(185.20617322,641.82162531)(185.21617464,641.86163348)
\curveto(185.2261732,641.90162523)(185.2311732,641.9366252)(185.23117464,641.96663348)
\curveto(185.2311732,642.00662513)(185.23617319,642.04662509)(185.24617464,642.08663348)
\curveto(185.27617315,642.19662494)(185.29617313,642.30662483)(185.30617464,642.41663348)
\curveto(185.3261731,642.5366246)(185.36117307,642.65162448)(185.41117464,642.76163348)
\curveto(185.55117288,643.10162403)(185.71117272,643.37662376)(185.89117464,643.58663348)
\curveto(186.08117235,643.80662333)(186.35117208,643.98662315)(186.70117464,644.12663348)
\curveto(186.78117165,644.15662298)(186.86617156,644.17662296)(186.95617464,644.18663348)
\curveto(187.04617138,644.20662293)(187.14117129,644.22662291)(187.24117464,644.24663348)
\curveto(187.27117116,644.25662288)(187.3261711,644.25662288)(187.40617464,644.24663348)
\curveto(187.48617094,644.24662289)(187.53617089,644.25662288)(187.55617464,644.27663348)
\curveto(188.11617031,644.28662285)(188.56616986,644.17662296)(188.90617464,643.94663348)
\curveto(189.25616917,643.71662342)(189.51616891,643.41162372)(189.68617464,643.03163348)
\curveto(189.7261687,642.94162419)(189.76116867,642.84662429)(189.79117464,642.74663348)
\curveto(189.82116861,642.64662449)(189.84616858,642.54662459)(189.86617464,642.44663348)
\curveto(189.88616854,642.41662472)(189.89116854,642.38662475)(189.88117464,642.35663348)
\curveto(189.88116855,642.32662481)(189.88616854,642.29662484)(189.89617464,642.26663348)
\curveto(189.9261685,642.15662498)(189.94616848,642.0316251)(189.95617464,641.89163348)
\curveto(189.96616846,641.76162537)(189.97616845,641.62662551)(189.98617464,641.48663348)
\lineto(189.98617464,641.32163348)
\curveto(189.99616843,641.26162587)(189.99616843,641.20662593)(189.98617464,641.15663348)
\curveto(189.97616845,641.10662603)(189.97116846,641.05662608)(189.97117464,641.00663348)
\lineto(189.97117464,640.87163348)
\curveto(189.96116847,640.8316263)(189.95616847,640.79162634)(189.95617464,640.75163348)
\curveto(189.96616846,640.71162642)(189.96116847,640.66662647)(189.94117464,640.61663348)
\curveto(189.92116851,640.50662663)(189.90116853,640.40162673)(189.88117464,640.30163348)
\curveto(189.87116856,640.20162693)(189.85116858,640.10162703)(189.82117464,640.00163348)
\curveto(189.69116874,639.64162749)(189.5261689,639.32662781)(189.32617464,639.05663348)
\curveto(189.1261693,638.78662835)(188.85116958,638.58162855)(188.50117464,638.44163348)
\curveto(188.42117001,638.41162872)(188.33617009,638.38662875)(188.24617464,638.36663348)
\lineto(187.97617464,638.30663348)
\curveto(187.9261705,638.29662884)(187.88117055,638.29162884)(187.84117464,638.29163348)
\curveto(187.80117063,638.30162883)(187.76117067,638.30162883)(187.72117464,638.29163348)
\curveto(187.62117081,638.27162886)(187.5261709,638.27162886)(187.43617464,638.29163348)
\moveto(186.59617464,639.68663348)
\curveto(186.63617179,639.61662752)(186.67617175,639.55162758)(186.71617464,639.49163348)
\curveto(186.75617167,639.44162769)(186.80617162,639.39162774)(186.86617464,639.34163348)
\lineto(187.01617464,639.22163348)
\curveto(187.07617135,639.19162794)(187.14117129,639.16662797)(187.21117464,639.14663348)
\curveto(187.25117118,639.12662801)(187.28617114,639.11662802)(187.31617464,639.11663348)
\curveto(187.35617107,639.12662801)(187.39617103,639.12162801)(187.43617464,639.10163348)
\curveto(187.46617096,639.10162803)(187.50617092,639.09662804)(187.55617464,639.08663348)
\curveto(187.60617082,639.08662805)(187.64617078,639.09162804)(187.67617464,639.10163348)
\lineto(187.90117464,639.14663348)
\curveto(188.15117028,639.22662791)(188.33617009,639.35162778)(188.45617464,639.52163348)
\curveto(188.53616989,639.62162751)(188.60616982,639.75162738)(188.66617464,639.91163348)
\curveto(188.74616968,640.09162704)(188.80616962,640.31662682)(188.84617464,640.58663348)
\curveto(188.88616954,640.86662627)(188.90116953,641.14662599)(188.89117464,641.42663348)
\curveto(188.88116955,641.71662542)(188.85116958,641.99162514)(188.80117464,642.25163348)
\curveto(188.75116968,642.51162462)(188.67616975,642.72162441)(188.57617464,642.88163348)
\curveto(188.45616997,643.08162405)(188.30617012,643.2316239)(188.12617464,643.33163348)
\curveto(188.04617038,643.38162375)(187.95617047,643.41162372)(187.85617464,643.42163348)
\curveto(187.75617067,643.44162369)(187.65117078,643.45162368)(187.54117464,643.45163348)
\curveto(187.52117091,643.44162369)(187.49617093,643.4366237)(187.46617464,643.43663348)
\curveto(187.44617098,643.44662369)(187.426171,643.44662369)(187.40617464,643.43663348)
\curveto(187.35617107,643.42662371)(187.31117112,643.41662372)(187.27117464,643.40663348)
\curveto(187.2311712,643.40662373)(187.19117124,643.39662374)(187.15117464,643.37663348)
\curveto(186.97117146,643.29662384)(186.82117161,643.17662396)(186.70117464,643.01663348)
\curveto(186.59117184,642.85662428)(186.50117193,642.67662446)(186.43117464,642.47663348)
\curveto(186.37117206,642.28662485)(186.3261721,642.06162507)(186.29617464,641.80163348)
\curveto(186.27617215,641.54162559)(186.27117216,641.27662586)(186.28117464,641.00663348)
\curveto(186.29117214,640.74662639)(186.32117211,640.49662664)(186.37117464,640.25663348)
\curveto(186.431172,640.02662711)(186.50617192,639.8366273)(186.59617464,639.68663348)
\moveto(197.39617464,636.70163348)
\curveto(197.40616102,636.65163048)(197.41116102,636.56163057)(197.41117464,636.43163348)
\curveto(197.41116102,636.30163083)(197.40116103,636.21163092)(197.38117464,636.16163348)
\curveto(197.36116107,636.11163102)(197.35616107,636.05663108)(197.36617464,635.99663348)
\curveto(197.37616105,635.94663119)(197.37616105,635.89663124)(197.36617464,635.84663348)
\curveto(197.3261611,635.70663143)(197.29616113,635.57163156)(197.27617464,635.44163348)
\curveto(197.26616116,635.31163182)(197.23616119,635.19163194)(197.18617464,635.08163348)
\curveto(197.04616138,634.7316324)(196.88116155,634.4366327)(196.69117464,634.19663348)
\curveto(196.50116193,633.96663317)(196.2311622,633.78163335)(195.88117464,633.64163348)
\curveto(195.80116263,633.61163352)(195.71616271,633.59163354)(195.62617464,633.58163348)
\curveto(195.53616289,633.56163357)(195.45116298,633.54163359)(195.37117464,633.52163348)
\curveto(195.32116311,633.51163362)(195.27116316,633.50663363)(195.22117464,633.50663348)
\curveto(195.17116326,633.50663363)(195.12116331,633.50163363)(195.07117464,633.49163348)
\curveto(195.04116339,633.48163365)(194.99116344,633.48163365)(194.92117464,633.49163348)
\curveto(194.85116358,633.49163364)(194.80116363,633.49663364)(194.77117464,633.50663348)
\curveto(194.71116372,633.52663361)(194.65116378,633.5366336)(194.59117464,633.53663348)
\curveto(194.54116389,633.52663361)(194.49116394,633.5316336)(194.44117464,633.55163348)
\curveto(194.35116408,633.57163356)(194.26116417,633.59663354)(194.17117464,633.62663348)
\curveto(194.09116434,633.64663349)(194.01116442,633.67663346)(193.93117464,633.71663348)
\curveto(193.61116482,633.85663328)(193.36116507,634.05163308)(193.18117464,634.30163348)
\curveto(193.00116543,634.56163257)(192.85116558,634.86663227)(192.73117464,635.21663348)
\curveto(192.71116572,635.29663184)(192.69616573,635.38163175)(192.68617464,635.47163348)
\curveto(192.67616575,635.56163157)(192.66116577,635.64663149)(192.64117464,635.72663348)
\curveto(192.6311658,635.75663138)(192.6261658,635.78663135)(192.62617464,635.81663348)
\lineto(192.62617464,635.92163348)
\curveto(192.60616582,636.00163113)(192.59616583,636.08163105)(192.59617464,636.16163348)
\lineto(192.59617464,636.29663348)
\curveto(192.57616585,636.39663074)(192.57616585,636.49663064)(192.59617464,636.59663348)
\lineto(192.59617464,636.77663348)
\curveto(192.60616582,636.82663031)(192.61116582,636.87163026)(192.61117464,636.91163348)
\curveto(192.61116582,636.96163017)(192.61616581,637.00663013)(192.62617464,637.04663348)
\curveto(192.63616579,637.08663005)(192.64116579,637.12163001)(192.64117464,637.15163348)
\curveto(192.64116579,637.19162994)(192.64616578,637.2316299)(192.65617464,637.27163348)
\lineto(192.71617464,637.60163348)
\curveto(192.73616569,637.72162941)(192.76616566,637.8316293)(192.80617464,637.93163348)
\curveto(192.94616548,638.26162887)(193.10616532,638.5366286)(193.28617464,638.75663348)
\curveto(193.47616495,638.98662815)(193.73616469,639.17162796)(194.06617464,639.31163348)
\curveto(194.14616428,639.35162778)(194.2311642,639.37662776)(194.32117464,639.38663348)
\lineto(194.62117464,639.44663348)
\lineto(194.75617464,639.44663348)
\curveto(194.80616362,639.45662768)(194.85616357,639.46162767)(194.90617464,639.46163348)
\curveto(195.47616295,639.48162765)(195.93616249,639.37662776)(196.28617464,639.14663348)
\curveto(196.64616178,638.92662821)(196.91116152,638.62662851)(197.08117464,638.24663348)
\curveto(197.1311613,638.14662899)(197.17116126,638.04662909)(197.20117464,637.94663348)
\curveto(197.2311612,637.84662929)(197.26116117,637.74162939)(197.29117464,637.63163348)
\curveto(197.30116113,637.59162954)(197.30616112,637.55662958)(197.30617464,637.52663348)
\curveto(197.30616112,637.50662963)(197.31116112,637.47662966)(197.32117464,637.43663348)
\curveto(197.34116109,637.36662977)(197.35116108,637.29162984)(197.35117464,637.21163348)
\curveto(197.35116108,637.13163)(197.36116107,637.05163008)(197.38117464,636.97163348)
\curveto(197.38116105,636.92163021)(197.38116105,636.87663026)(197.38117464,636.83663348)
\curveto(197.38116105,636.79663034)(197.38616104,636.75163038)(197.39617464,636.70163348)
\moveto(196.28617464,636.26663348)
\curveto(196.29616213,636.31663082)(196.30116213,636.39163074)(196.30117464,636.49163348)
\curveto(196.31116212,636.59163054)(196.30616212,636.66663047)(196.28617464,636.71663348)
\curveto(196.26616216,636.77663036)(196.26116217,636.8316303)(196.27117464,636.88163348)
\curveto(196.29116214,636.94163019)(196.29116214,637.00163013)(196.27117464,637.06163348)
\curveto(196.26116217,637.09163004)(196.25616217,637.12663001)(196.25617464,637.16663348)
\curveto(196.25616217,637.20662993)(196.25116218,637.24662989)(196.24117464,637.28663348)
\curveto(196.22116221,637.36662977)(196.20116223,637.44162969)(196.18117464,637.51163348)
\curveto(196.17116226,637.59162954)(196.15616227,637.67162946)(196.13617464,637.75163348)
\curveto(196.10616232,637.81162932)(196.08116235,637.87162926)(196.06117464,637.93163348)
\curveto(196.04116239,637.99162914)(196.01116242,638.05162908)(195.97117464,638.11163348)
\curveto(195.87116256,638.28162885)(195.74116269,638.41662872)(195.58117464,638.51663348)
\curveto(195.50116293,638.56662857)(195.40616302,638.60162853)(195.29617464,638.62163348)
\curveto(195.18616324,638.64162849)(195.06116337,638.65162848)(194.92117464,638.65163348)
\curveto(194.90116353,638.64162849)(194.87616355,638.6366285)(194.84617464,638.63663348)
\curveto(194.81616361,638.64662849)(194.78616364,638.64662849)(194.75617464,638.63663348)
\lineto(194.60617464,638.57663348)
\curveto(194.55616387,638.56662857)(194.51116392,638.55162858)(194.47117464,638.53163348)
\curveto(194.28116415,638.42162871)(194.13616429,638.27662886)(194.03617464,638.09663348)
\curveto(193.94616448,637.91662922)(193.86616456,637.71162942)(193.79617464,637.48163348)
\curveto(193.75616467,637.35162978)(193.73616469,637.21662992)(193.73617464,637.07663348)
\curveto(193.73616469,636.94663019)(193.7261647,636.80163033)(193.70617464,636.64163348)
\curveto(193.69616473,636.59163054)(193.68616474,636.5316306)(193.67617464,636.46163348)
\curveto(193.67616475,636.39163074)(193.68616474,636.3316308)(193.70617464,636.28163348)
\lineto(193.70617464,636.11663348)
\lineto(193.70617464,635.93663348)
\curveto(193.71616471,635.88663125)(193.7261647,635.8316313)(193.73617464,635.77163348)
\curveto(193.74616468,635.72163141)(193.75116468,635.66663147)(193.75117464,635.60663348)
\curveto(193.76116467,635.54663159)(193.77616465,635.49163164)(193.79617464,635.44163348)
\curveto(193.84616458,635.25163188)(193.90616452,635.07663206)(193.97617464,634.91663348)
\curveto(194.04616438,634.75663238)(194.15116428,634.62663251)(194.29117464,634.52663348)
\curveto(194.42116401,634.42663271)(194.56116387,634.35663278)(194.71117464,634.31663348)
\curveto(194.74116369,634.30663283)(194.76616366,634.30163283)(194.78617464,634.30163348)
\curveto(194.81616361,634.31163282)(194.84616358,634.31163282)(194.87617464,634.30163348)
\curveto(194.89616353,634.30163283)(194.9261635,634.29663284)(194.96617464,634.28663348)
\curveto(195.00616342,634.28663285)(195.04116339,634.29163284)(195.07117464,634.30163348)
\curveto(195.11116332,634.31163282)(195.15116328,634.31663282)(195.19117464,634.31663348)
\curveto(195.2311632,634.31663282)(195.27116316,634.32663281)(195.31117464,634.34663348)
\curveto(195.55116288,634.42663271)(195.74616268,634.56163257)(195.89617464,634.75163348)
\curveto(196.01616241,634.9316322)(196.10616232,635.136632)(196.16617464,635.36663348)
\curveto(196.18616224,635.4366317)(196.20116223,635.50663163)(196.21117464,635.57663348)
\curveto(196.22116221,635.65663148)(196.23616219,635.7366314)(196.25617464,635.81663348)
\curveto(196.25616217,635.87663126)(196.26116217,635.92163121)(196.27117464,635.95163348)
\curveto(196.27116216,635.97163116)(196.27116216,635.99663114)(196.27117464,636.02663348)
\curveto(196.27116216,636.06663107)(196.27616215,636.09663104)(196.28617464,636.11663348)
\lineto(196.28617464,636.26663348)
}
}
\end{pspicture}

\caption{Porcentajes de los recursos según su tipo}
\label{recursos_pie_1}
\end{figure}

\subsection{Linea de tiempo}
Finalizados los elementos propios del sistema, ahora se verá la ultima variable
que se ha considerado importante para los objetivos del proyecto, estas son las
lineas de tiempo, donde se presentan los tiempos en cuales los elementos del
sistema han sido creados.

En la figura \ref{tiempos_area_1} puede apreciarse las cuatro lineas de creación
estas son:

\begin{figure}
\centering
\input{graphics/cap5_10.tex}
\caption{Linea de tiempo de la creación de elementos en el sistema}
\label{tiempos_area_1}
\end{figure}

\begin{description}
\item [Usuarios] Linea que compila las actividades de creacion de usuarios en el
sistema.
\item [Contactos] Linea que compila las actividades de creacion de contactos por
parte de los usuarios.
\item [Espacios] Linea que compila las actividades de creacion de espacios
virtuales por parte de los usuarios.
\item [Recursos] Linea que compila las actividades de creacion de recursos por
parte de los usuarios.
\end{description}

Puede notarse como la linea predominante es la de creacion de usaurios, con
creaciones automaticas de estudiantes por parte de los docentes representada por
grandes saltos en periodos muy cortos.

En la figura \ref{tiempos_area_2}, se ve con mas claridad la misma linea de
tiempo, pero a escala menor, resalta la curiosa relación entre las lineas
de creación de recursos y la de creación de contactos, siendo esta la linea que
determina todo el objeto de investigación.

\begin{figure}
\centering
%LaTeX with PSTricks extensions
%%Creator: inkscape 0.48.5
%%Please note this file requires PSTricks extensions
\psset{xunit=.5pt,yunit=.5pt,runit=.5pt}
\begin{pspicture}(1052.35998535,480)
{
\newrgbcolor{curcolor}{0 0 0}
\pscustom[linestyle=none,fillstyle=solid,fillcolor=curcolor]
{
\newpath
\moveto(36.46913239,445.95727663)
\curveto(36.47912467,445.91727358)(36.47912467,445.86727363)(36.46913239,445.80727663)
\curveto(36.46912468,445.74727375)(36.46412468,445.6972738)(36.45413239,445.65727663)
\curveto(36.45412469,445.61727388)(36.4491247,445.57727392)(36.43913239,445.53727663)
\lineto(36.43913239,445.43227663)
\curveto(36.41912473,445.35227415)(36.40412474,445.27227423)(36.39413239,445.19227663)
\curveto(36.38412476,445.11227439)(36.36412478,445.03727446)(36.33413239,444.96727663)
\curveto(36.31412483,444.88727461)(36.29412485,444.81227469)(36.27413239,444.74227663)
\curveto(36.25412489,444.67227483)(36.22412492,444.5972749)(36.18413239,444.51727663)
\curveto(36.00412514,444.0972754)(35.7491254,443.75727574)(35.41913239,443.49727663)
\curveto(35.08912606,443.23727626)(34.69912645,443.03227647)(34.24913239,442.88227663)
\curveto(34.12912702,442.84227666)(34.00412714,442.81727668)(33.87413239,442.80727663)
\curveto(33.75412739,442.78727671)(33.62912752,442.76227674)(33.49913239,442.73227663)
\curveto(33.43912771,442.72227678)(33.37412777,442.71727678)(33.30413239,442.71727663)
\curveto(33.2441279,442.71727678)(33.17912797,442.71227679)(33.10913239,442.70227663)
\lineto(32.98913239,442.70227663)
\lineto(32.79413239,442.70227663)
\curveto(32.73412841,442.69227681)(32.67912847,442.6972768)(32.62913239,442.71727663)
\curveto(32.55912859,442.73727676)(32.49412865,442.74227676)(32.43413239,442.73227663)
\curveto(32.37412877,442.72227678)(32.31412883,442.72727677)(32.25413239,442.74727663)
\curveto(32.20412894,442.75727674)(32.15912899,442.76227674)(32.11913239,442.76227663)
\curveto(32.07912907,442.76227674)(32.03412911,442.77227673)(31.98413239,442.79227663)
\curveto(31.90412924,442.81227669)(31.82912932,442.83227667)(31.75913239,442.85227663)
\curveto(31.68912946,442.86227664)(31.61912953,442.87727662)(31.54913239,442.89727663)
\curveto(31.06913008,443.06727643)(30.66913048,443.27727622)(30.34913239,443.52727663)
\curveto(30.03913111,443.78727571)(29.78913136,444.14227536)(29.59913239,444.59227663)
\curveto(29.56913158,444.65227485)(29.5441316,444.71227479)(29.52413239,444.77227663)
\curveto(29.51413163,444.84227466)(29.49913165,444.91727458)(29.47913239,444.99727663)
\curveto(29.45913169,445.05727444)(29.4441317,445.12227438)(29.43413239,445.19227663)
\curveto(29.42413172,445.26227424)(29.40913174,445.33227417)(29.38913239,445.40227663)
\curveto(29.37913177,445.45227405)(29.37413177,445.49227401)(29.37413239,445.52227663)
\lineto(29.37413239,445.64227663)
\curveto(29.36413178,445.68227382)(29.35413179,445.73227377)(29.34413239,445.79227663)
\curveto(29.3441318,445.85227365)(29.3491318,445.9022736)(29.35913239,445.94227663)
\lineto(29.35913239,446.07727663)
\curveto(29.36913178,446.12727337)(29.37413177,446.17727332)(29.37413239,446.22727663)
\curveto(29.39413175,446.32727317)(29.40913174,446.42227308)(29.41913239,446.51227663)
\curveto(29.42913172,446.61227289)(29.4491317,446.70727279)(29.47913239,446.79727663)
\curveto(29.52913162,446.94727255)(29.58413156,447.08727241)(29.64413239,447.21727663)
\curveto(29.70413144,447.34727215)(29.77413137,447.46727203)(29.85413239,447.57727663)
\curveto(29.88413126,447.62727187)(29.91413123,447.66727183)(29.94413239,447.69727663)
\curveto(29.98413116,447.72727177)(30.01913113,447.76227174)(30.04913239,447.80227663)
\curveto(30.10913104,447.88227162)(30.17913097,447.95227155)(30.25913239,448.01227663)
\curveto(30.31913083,448.06227144)(30.37913077,448.10727139)(30.43913239,448.14727663)
\lineto(30.64913239,448.29727663)
\curveto(30.69913045,448.33727116)(30.7491304,448.37227113)(30.79913239,448.40227663)
\curveto(30.8491303,448.44227106)(30.88413026,448.497271)(30.90413239,448.56727663)
\curveto(30.90413024,448.5972709)(30.89413025,448.62227088)(30.87413239,448.64227663)
\curveto(30.86413028,448.67227083)(30.85413029,448.6972708)(30.84413239,448.71727663)
\curveto(30.80413034,448.76727073)(30.75413039,448.81227069)(30.69413239,448.85227663)
\curveto(30.6441305,448.9022706)(30.59413055,448.94727055)(30.54413239,448.98727663)
\curveto(30.50413064,449.01727048)(30.45413069,449.07227043)(30.39413239,449.15227663)
\curveto(30.37413077,449.18227032)(30.3441308,449.20727029)(30.30413239,449.22727663)
\curveto(30.27413087,449.25727024)(30.2491309,449.29227021)(30.22913239,449.33227663)
\curveto(30.05913109,449.54226996)(29.92913122,449.78726971)(29.83913239,450.06727663)
\curveto(29.81913133,450.14726935)(29.80413134,450.22726927)(29.79413239,450.30727663)
\curveto(29.78413136,450.38726911)(29.76913138,450.46726903)(29.74913239,450.54727663)
\curveto(29.72913142,450.5972689)(29.71913143,450.66226884)(29.71913239,450.74227663)
\curveto(29.71913143,450.83226867)(29.72913142,450.9022686)(29.74913239,450.95227663)
\curveto(29.7491314,451.05226845)(29.75413139,451.12226838)(29.76413239,451.16227663)
\curveto(29.78413136,451.24226826)(29.79913135,451.31226819)(29.80913239,451.37227663)
\curveto(29.81913133,451.44226806)(29.83413131,451.51226799)(29.85413239,451.58227663)
\curveto(30.00413114,452.01226749)(30.21913093,452.35726714)(30.49913239,452.61727663)
\curveto(30.78913036,452.87726662)(31.13913001,453.09226641)(31.54913239,453.26227663)
\curveto(31.65912949,453.31226619)(31.77412937,453.34226616)(31.89413239,453.35227663)
\curveto(32.02412912,453.37226613)(32.15412899,453.4022661)(32.28413239,453.44227663)
\curveto(32.36412878,453.44226606)(32.43412871,453.44226606)(32.49413239,453.44227663)
\curveto(32.56412858,453.45226605)(32.63912851,453.46226604)(32.71913239,453.47227663)
\curveto(33.50912764,453.49226601)(34.16412698,453.36226614)(34.68413239,453.08227663)
\curveto(35.21412593,452.8022667)(35.59412555,452.39226711)(35.82413239,451.85227663)
\curveto(35.93412521,451.62226788)(36.00412514,451.33726816)(36.03413239,450.99727663)
\curveto(36.07412507,450.66726883)(36.0441251,450.36226914)(35.94413239,450.08227663)
\curveto(35.90412524,449.95226955)(35.85412529,449.83226967)(35.79413239,449.72227663)
\curveto(35.7441254,449.61226989)(35.68412546,449.50726999)(35.61413239,449.40727663)
\curveto(35.59412555,449.36727013)(35.56412558,449.33227017)(35.52413239,449.30227663)
\lineto(35.43413239,449.21227663)
\curveto(35.38412576,449.12227038)(35.32412582,449.05727044)(35.25413239,449.01727663)
\curveto(35.20412594,448.96727053)(35.149126,448.91727058)(35.08913239,448.86727663)
\curveto(35.03912611,448.82727067)(34.99412615,448.78227072)(34.95413239,448.73227663)
\curveto(34.93412621,448.71227079)(34.91412623,448.68727081)(34.89413239,448.65727663)
\curveto(34.88412626,448.63727086)(34.88412626,448.61227089)(34.89413239,448.58227663)
\curveto(34.90412624,448.53227097)(34.93412621,448.48227102)(34.98413239,448.43227663)
\curveto(35.03412611,448.39227111)(35.08912606,448.35227115)(35.14913239,448.31227663)
\lineto(35.32913239,448.19227663)
\curveto(35.38912576,448.16227134)(35.43912571,448.13227137)(35.47913239,448.10227663)
\curveto(35.80912534,447.86227164)(36.05912509,447.55227195)(36.22913239,447.17227663)
\curveto(36.26912488,447.09227241)(36.29912485,447.00727249)(36.31913239,446.91727663)
\curveto(36.3491248,446.82727267)(36.37412477,446.73727276)(36.39413239,446.64727663)
\curveto(36.40412474,446.5972729)(36.41412473,446.54227296)(36.42413239,446.48227663)
\lineto(36.45413239,446.33227663)
\curveto(36.46412468,446.27227323)(36.46412468,446.20727329)(36.45413239,446.13727663)
\curveto(36.4441247,446.07727342)(36.4491247,446.01727348)(36.46913239,445.95727663)
\moveto(31.08413239,450.99727663)
\curveto(31.05413009,450.88726861)(31.0491301,450.74726875)(31.06913239,450.57727663)
\curveto(31.08913006,450.41726908)(31.11413003,450.29226921)(31.14413239,450.20227663)
\curveto(31.25412989,449.88226962)(31.40412974,449.63726986)(31.59413239,449.46727663)
\curveto(31.78412936,449.30727019)(32.0491291,449.17727032)(32.38913239,449.07727663)
\curveto(32.51912863,449.04727045)(32.68412846,449.02227048)(32.88413239,449.00227663)
\curveto(33.08412806,448.99227051)(33.25412789,449.00727049)(33.39413239,449.04727663)
\curveto(33.68412746,449.12727037)(33.92412722,449.23727026)(34.11413239,449.37727663)
\curveto(34.31412683,449.52726997)(34.46912668,449.72726977)(34.57913239,449.97727663)
\curveto(34.59912655,450.02726947)(34.60912654,450.07226943)(34.60913239,450.11227663)
\curveto(34.61912653,450.15226935)(34.63412651,450.1972693)(34.65413239,450.24727663)
\curveto(34.68412646,450.35726914)(34.70412644,450.497269)(34.71413239,450.66727663)
\curveto(34.72412642,450.83726866)(34.71412643,450.98226852)(34.68413239,451.10227663)
\curveto(34.66412648,451.19226831)(34.63912651,451.27726822)(34.60913239,451.35727663)
\curveto(34.58912656,451.43726806)(34.55412659,451.51726798)(34.50413239,451.59727663)
\curveto(34.33412681,451.86726763)(34.10912704,452.06226744)(33.82913239,452.18227663)
\curveto(33.55912759,452.3022672)(33.19912795,452.36226714)(32.74913239,452.36227663)
\curveto(32.72912842,452.34226716)(32.69912845,452.33726716)(32.65913239,452.34727663)
\curveto(32.61912853,452.35726714)(32.58412856,452.35726714)(32.55413239,452.34727663)
\curveto(32.50412864,452.32726717)(32.4491287,452.31226719)(32.38913239,452.30227663)
\curveto(32.33912881,452.3022672)(32.28912886,452.29226721)(32.23913239,452.27227663)
\curveto(31.99912915,452.18226732)(31.78912936,452.06726743)(31.60913239,451.92727663)
\curveto(31.42912972,451.7972677)(31.28912986,451.61726788)(31.18913239,451.38727663)
\curveto(31.16912998,451.32726817)(31.14913,451.26226824)(31.12913239,451.19227663)
\curveto(31.11913003,451.13226837)(31.10413004,451.06726843)(31.08413239,450.99727663)
\moveto(35.10413239,445.46227663)
\curveto(35.15412599,445.65227385)(35.15912599,445.85727364)(35.11913239,446.07727663)
\curveto(35.08912606,446.2972732)(35.0441261,446.47727302)(34.98413239,446.61727663)
\curveto(34.81412633,446.98727251)(34.55412659,447.29227221)(34.20413239,447.53227663)
\curveto(33.86412728,447.77227173)(33.42912772,447.89227161)(32.89913239,447.89227663)
\curveto(32.86912828,447.87227163)(32.82912832,447.86727163)(32.77913239,447.87727663)
\curveto(32.72912842,447.8972716)(32.68912846,447.9022716)(32.65913239,447.89227663)
\lineto(32.38913239,447.83227663)
\curveto(32.30912884,447.82227168)(32.22912892,447.80727169)(32.14913239,447.78727663)
\curveto(31.8491293,447.67727182)(31.58412956,447.53227197)(31.35413239,447.35227663)
\curveto(31.13413001,447.17227233)(30.96413018,446.94227256)(30.84413239,446.66227663)
\curveto(30.81413033,446.58227292)(30.78913036,446.502273)(30.76913239,446.42227663)
\curveto(30.7491304,446.34227316)(30.72913042,446.25727324)(30.70913239,446.16727663)
\curveto(30.67913047,446.04727345)(30.66913048,445.8972736)(30.67913239,445.71727663)
\curveto(30.69913045,445.53727396)(30.72413042,445.3972741)(30.75413239,445.29727663)
\curveto(30.77413037,445.24727425)(30.78413036,445.2022743)(30.78413239,445.16227663)
\curveto(30.79413035,445.13227437)(30.80913034,445.09227441)(30.82913239,445.04227663)
\curveto(30.92913022,444.82227468)(31.05913009,444.62227488)(31.21913239,444.44227663)
\curveto(31.38912976,444.26227524)(31.58412956,444.12727537)(31.80413239,444.03727663)
\curveto(31.87412927,443.9972755)(31.96912918,443.96227554)(32.08913239,443.93227663)
\curveto(32.30912884,443.84227566)(32.56412858,443.7972757)(32.85413239,443.79727663)
\lineto(33.13913239,443.79727663)
\curveto(33.23912791,443.81727568)(33.33412781,443.83227567)(33.42413239,443.84227663)
\curveto(33.51412763,443.85227565)(33.60412754,443.87227563)(33.69413239,443.90227663)
\curveto(33.95412719,443.98227552)(34.19412695,444.11227539)(34.41413239,444.29227663)
\curveto(34.6441265,444.48227502)(34.81412633,444.6972748)(34.92413239,444.93727663)
\curveto(34.96412618,445.01727448)(34.99412615,445.0972744)(35.01413239,445.17727663)
\curveto(35.0441261,445.26727423)(35.07412607,445.36227414)(35.10413239,445.46227663)
}
}
{
\newrgbcolor{curcolor}{0 0 0}
\pscustom[linestyle=none,fillstyle=solid,fillcolor=curcolor]
{
\newpath
\moveto(44.80374176,447.96727663)
\lineto(44.80374176,447.71227663)
\curveto(44.81373406,447.63227187)(44.80873406,447.55727194)(44.78874176,447.48727663)
\lineto(44.78874176,447.24727663)
\lineto(44.78874176,447.08227663)
\curveto(44.7687341,446.98227252)(44.75873411,446.87727262)(44.75874176,446.76727663)
\curveto(44.75873411,446.66727283)(44.74873412,446.56727293)(44.72874176,446.46727663)
\lineto(44.72874176,446.31727663)
\curveto(44.69873417,446.17727332)(44.67873419,446.03727346)(44.66874176,445.89727663)
\curveto(44.65873421,445.76727373)(44.63373424,445.63727386)(44.59374176,445.50727663)
\curveto(44.5737343,445.42727407)(44.55373432,445.34227416)(44.53374176,445.25227663)
\lineto(44.47374176,445.01227663)
\lineto(44.35374176,444.71227663)
\curveto(44.32373455,444.62227488)(44.28873458,444.53227497)(44.24874176,444.44227663)
\curveto(44.14873472,444.22227528)(44.01373486,444.00727549)(43.84374176,443.79727663)
\curveto(43.68373519,443.58727591)(43.50873536,443.41727608)(43.31874176,443.28727663)
\curveto(43.2687356,443.24727625)(43.20873566,443.20727629)(43.13874176,443.16727663)
\curveto(43.07873579,443.13727636)(43.01873585,443.1022764)(42.95874176,443.06227663)
\curveto(42.87873599,443.01227649)(42.78373609,442.97227653)(42.67374176,442.94227663)
\curveto(42.56373631,442.91227659)(42.45873641,442.88227662)(42.35874176,442.85227663)
\curveto(42.24873662,442.81227669)(42.13873673,442.78727671)(42.02874176,442.77727663)
\curveto(41.91873695,442.76727673)(41.80373707,442.75227675)(41.68374176,442.73227663)
\curveto(41.64373723,442.72227678)(41.59873727,442.72227678)(41.54874176,442.73227663)
\curveto(41.50873736,442.73227677)(41.4687374,442.72727677)(41.42874176,442.71727663)
\curveto(41.38873748,442.70727679)(41.33373754,442.7022768)(41.26374176,442.70227663)
\curveto(41.19373768,442.7022768)(41.14373773,442.70727679)(41.11374176,442.71727663)
\curveto(41.06373781,442.73727676)(41.01873785,442.74227676)(40.97874176,442.73227663)
\curveto(40.93873793,442.72227678)(40.90373797,442.72227678)(40.87374176,442.73227663)
\lineto(40.78374176,442.73227663)
\curveto(40.72373815,442.75227675)(40.65873821,442.76727673)(40.58874176,442.77727663)
\curveto(40.52873834,442.77727672)(40.46373841,442.78227672)(40.39374176,442.79227663)
\curveto(40.22373865,442.84227666)(40.06373881,442.89227661)(39.91374176,442.94227663)
\curveto(39.76373911,442.99227651)(39.61873925,443.05727644)(39.47874176,443.13727663)
\curveto(39.42873944,443.17727632)(39.3737395,443.20727629)(39.31374176,443.22727663)
\curveto(39.26373961,443.25727624)(39.21373966,443.29227621)(39.16374176,443.33227663)
\curveto(38.92373995,443.51227599)(38.72374015,443.73227577)(38.56374176,443.99227663)
\curveto(38.40374047,444.25227525)(38.26374061,444.53727496)(38.14374176,444.84727663)
\curveto(38.08374079,444.98727451)(38.03874083,445.12727437)(38.00874176,445.26727663)
\curveto(37.97874089,445.41727408)(37.94374093,445.57227393)(37.90374176,445.73227663)
\curveto(37.88374099,445.84227366)(37.868741,445.95227355)(37.85874176,446.06227663)
\curveto(37.84874102,446.17227333)(37.83374104,446.28227322)(37.81374176,446.39227663)
\curveto(37.80374107,446.43227307)(37.79874107,446.47227303)(37.79874176,446.51227663)
\curveto(37.80874106,446.55227295)(37.80874106,446.59227291)(37.79874176,446.63227663)
\curveto(37.78874108,446.68227282)(37.78374109,446.73227277)(37.78374176,446.78227663)
\lineto(37.78374176,446.94727663)
\curveto(37.76374111,446.9972725)(37.75874111,447.04727245)(37.76874176,447.09727663)
\curveto(37.77874109,447.15727234)(37.77874109,447.21227229)(37.76874176,447.26227663)
\curveto(37.75874111,447.3022722)(37.75874111,447.34727215)(37.76874176,447.39727663)
\curveto(37.77874109,447.44727205)(37.7737411,447.497272)(37.75374176,447.54727663)
\curveto(37.73374114,447.61727188)(37.72874114,447.69227181)(37.73874176,447.77227663)
\curveto(37.74874112,447.86227164)(37.75374112,447.94727155)(37.75374176,448.02727663)
\curveto(37.75374112,448.11727138)(37.74874112,448.21727128)(37.73874176,448.32727663)
\curveto(37.72874114,448.44727105)(37.73374114,448.54727095)(37.75374176,448.62727663)
\lineto(37.75374176,448.91227663)
\lineto(37.79874176,449.54227663)
\curveto(37.80874106,449.64226986)(37.81874105,449.73726976)(37.82874176,449.82727663)
\lineto(37.85874176,450.12727663)
\curveto(37.87874099,450.17726932)(37.88374099,450.22726927)(37.87374176,450.27727663)
\curveto(37.873741,450.33726916)(37.88374099,450.39226911)(37.90374176,450.44227663)
\curveto(37.95374092,450.61226889)(37.99374088,450.77726872)(38.02374176,450.93727663)
\curveto(38.05374082,451.10726839)(38.10374077,451.26726823)(38.17374176,451.41727663)
\curveto(38.36374051,451.87726762)(38.58374029,452.25226725)(38.83374176,452.54227663)
\curveto(39.09373978,452.83226667)(39.45373942,453.07726642)(39.91374176,453.27727663)
\curveto(40.04373883,453.32726617)(40.1737387,453.36226614)(40.30374176,453.38227663)
\curveto(40.44373843,453.4022661)(40.58373829,453.42726607)(40.72374176,453.45727663)
\curveto(40.79373808,453.46726603)(40.85873801,453.47226603)(40.91874176,453.47227663)
\curveto(40.97873789,453.47226603)(41.04373783,453.47726602)(41.11374176,453.48727663)
\curveto(41.94373693,453.50726599)(42.61373626,453.35726614)(43.12374176,453.03727663)
\curveto(43.63373524,452.72726677)(44.01373486,452.28726721)(44.26374176,451.71727663)
\curveto(44.31373456,451.5972679)(44.35873451,451.47226803)(44.39874176,451.34227663)
\curveto(44.43873443,451.21226829)(44.48373439,451.07726842)(44.53374176,450.93727663)
\curveto(44.55373432,450.85726864)(44.5687343,450.77226873)(44.57874176,450.68227663)
\lineto(44.63874176,450.44227663)
\curveto(44.6687342,450.33226917)(44.68373419,450.22226928)(44.68374176,450.11227663)
\curveto(44.69373418,450.0022695)(44.70873416,449.89226961)(44.72874176,449.78227663)
\curveto(44.74873412,449.73226977)(44.75373412,449.68726981)(44.74374176,449.64727663)
\curveto(44.74373413,449.60726989)(44.74873412,449.56726993)(44.75874176,449.52727663)
\curveto(44.7687341,449.47727002)(44.7687341,449.42227008)(44.75874176,449.36227663)
\curveto(44.75873411,449.31227019)(44.76373411,449.26227024)(44.77374176,449.21227663)
\lineto(44.77374176,449.07727663)
\curveto(44.79373408,449.01727048)(44.79373408,448.94727055)(44.77374176,448.86727663)
\curveto(44.76373411,448.7972707)(44.7687341,448.73227077)(44.78874176,448.67227663)
\curveto(44.79873407,448.64227086)(44.80373407,448.6022709)(44.80374176,448.55227663)
\lineto(44.80374176,448.43227663)
\lineto(44.80374176,447.96727663)
\moveto(43.25874176,445.64227663)
\curveto(43.35873551,445.96227354)(43.41873545,446.32727317)(43.43874176,446.73727663)
\curveto(43.45873541,447.14727235)(43.4687354,447.55727194)(43.46874176,447.96727663)
\curveto(43.4687354,448.3972711)(43.45873541,448.81727068)(43.43874176,449.22727663)
\curveto(43.41873545,449.63726986)(43.3737355,450.02226948)(43.30374176,450.38227663)
\curveto(43.23373564,450.74226876)(43.12373575,451.06226844)(42.97374176,451.34227663)
\curveto(42.83373604,451.63226787)(42.63873623,451.86726763)(42.38874176,452.04727663)
\curveto(42.22873664,452.15726734)(42.04873682,452.23726726)(41.84874176,452.28727663)
\curveto(41.64873722,452.34726715)(41.40373747,452.37726712)(41.11374176,452.37727663)
\curveto(41.09373778,452.35726714)(41.05873781,452.34726715)(41.00874176,452.34727663)
\curveto(40.95873791,452.35726714)(40.91873795,452.35726714)(40.88874176,452.34727663)
\curveto(40.80873806,452.32726717)(40.73373814,452.30726719)(40.66374176,452.28727663)
\curveto(40.60373827,452.27726722)(40.53873833,452.25726724)(40.46874176,452.22727663)
\curveto(40.19873867,452.10726739)(39.97873889,451.93726756)(39.80874176,451.71727663)
\curveto(39.64873922,451.50726799)(39.51373936,451.26226824)(39.40374176,450.98227663)
\curveto(39.35373952,450.87226863)(39.31373956,450.75226875)(39.28374176,450.62227663)
\curveto(39.26373961,450.502269)(39.23873963,450.37726912)(39.20874176,450.24727663)
\curveto(39.18873968,450.1972693)(39.17873969,450.14226936)(39.17874176,450.08227663)
\curveto(39.17873969,450.03226947)(39.1737397,449.98226952)(39.16374176,449.93227663)
\curveto(39.15373972,449.84226966)(39.14373973,449.74726975)(39.13374176,449.64727663)
\curveto(39.12373975,449.55726994)(39.11373976,449.46227004)(39.10374176,449.36227663)
\curveto(39.10373977,449.28227022)(39.09873977,449.1972703)(39.08874176,449.10727663)
\lineto(39.08874176,448.86727663)
\lineto(39.08874176,448.68727663)
\curveto(39.07873979,448.65727084)(39.0737398,448.62227088)(39.07374176,448.58227663)
\lineto(39.07374176,448.44727663)
\lineto(39.07374176,447.99727663)
\curveto(39.0737398,447.91727158)(39.0687398,447.83227167)(39.05874176,447.74227663)
\curveto(39.05873981,447.66227184)(39.0687398,447.58727191)(39.08874176,447.51727663)
\lineto(39.08874176,447.24727663)
\curveto(39.08873978,447.22727227)(39.08373979,447.1972723)(39.07374176,447.15727663)
\curveto(39.0737398,447.12727237)(39.07873979,447.1022724)(39.08874176,447.08227663)
\curveto(39.09873977,446.98227252)(39.10373977,446.88227262)(39.10374176,446.78227663)
\curveto(39.11373976,446.69227281)(39.12373975,446.59227291)(39.13374176,446.48227663)
\curveto(39.16373971,446.36227314)(39.17873969,446.23727326)(39.17874176,446.10727663)
\curveto(39.18873968,445.98727351)(39.21373966,445.87227363)(39.25374176,445.76227663)
\curveto(39.33373954,445.46227404)(39.41873945,445.1972743)(39.50874176,444.96727663)
\curveto(39.60873926,444.73727476)(39.75373912,444.52227498)(39.94374176,444.32227663)
\curveto(40.15373872,444.12227538)(40.41873845,443.97227553)(40.73874176,443.87227663)
\curveto(40.77873809,443.85227565)(40.81373806,443.84227566)(40.84374176,443.84227663)
\curveto(40.88373799,443.85227565)(40.92873794,443.84727565)(40.97874176,443.82727663)
\curveto(41.01873785,443.81727568)(41.08873778,443.80727569)(41.18874176,443.79727663)
\curveto(41.29873757,443.78727571)(41.38373749,443.79227571)(41.44374176,443.81227663)
\curveto(41.51373736,443.83227567)(41.58373729,443.84227566)(41.65374176,443.84227663)
\curveto(41.72373715,443.85227565)(41.78873708,443.86727563)(41.84874176,443.88727663)
\curveto(42.04873682,443.94727555)(42.22873664,444.03227547)(42.38874176,444.14227663)
\curveto(42.41873645,444.16227534)(42.44373643,444.18227532)(42.46374176,444.20227663)
\lineto(42.52374176,444.26227663)
\curveto(42.56373631,444.28227522)(42.61373626,444.32227518)(42.67374176,444.38227663)
\curveto(42.7737361,444.52227498)(42.85873601,444.65227485)(42.92874176,444.77227663)
\curveto(42.99873587,444.89227461)(43.0687358,445.03727446)(43.13874176,445.20727663)
\curveto(43.1687357,445.27727422)(43.18873568,445.34727415)(43.19874176,445.41727663)
\curveto(43.21873565,445.48727401)(43.23873563,445.56227394)(43.25874176,445.64227663)
}
}
{
\newrgbcolor{curcolor}{0 0 0}
\pscustom[linestyle=none,fillstyle=solid,fillcolor=curcolor]
{
\newpath
\moveto(36.45413239,367.04016726)
\curveto(36.48412466,366.92016304)(36.50912464,366.78016318)(36.52913239,366.62016726)
\curveto(36.5491246,366.4601635)(36.55912459,366.29516367)(36.55913239,366.12516726)
\curveto(36.55912459,365.95516401)(36.5491246,365.79016417)(36.52913239,365.63016726)
\curveto(36.50912464,365.47016449)(36.48412466,365.33016463)(36.45413239,365.21016726)
\curveto(36.41412473,365.07016489)(36.37912477,364.94516502)(36.34913239,364.83516726)
\curveto(36.31912483,364.72516524)(36.27912487,364.61516535)(36.22913239,364.50516726)
\curveto(35.95912519,363.8651661)(35.5441256,363.38016658)(34.98413239,363.05016726)
\curveto(34.90412624,362.99016697)(34.81912633,362.94016702)(34.72913239,362.90016726)
\curveto(34.63912651,362.87016709)(34.53912661,362.83516713)(34.42913239,362.79516726)
\curveto(34.31912683,362.74516722)(34.19912695,362.71016725)(34.06913239,362.69016726)
\curveto(33.9491272,362.6601673)(33.81912733,362.63016733)(33.67913239,362.60016726)
\curveto(33.61912753,362.58016738)(33.55912759,362.57516739)(33.49913239,362.58516726)
\curveto(33.4491277,362.59516737)(33.38912776,362.59016737)(33.31913239,362.57016726)
\curveto(33.29912785,362.5601674)(33.27412787,362.5601674)(33.24413239,362.57016726)
\curveto(33.21412793,362.57016739)(33.18912796,362.5651674)(33.16913239,362.55516726)
\lineto(33.01913239,362.55516726)
\curveto(32.9491282,362.54516742)(32.89912825,362.54516742)(32.86913239,362.55516726)
\curveto(32.82912832,362.5651674)(32.78412836,362.57016739)(32.73413239,362.57016726)
\curveto(32.69412845,362.5601674)(32.65412849,362.5601674)(32.61413239,362.57016726)
\curveto(32.52412862,362.59016737)(32.43412871,362.60516736)(32.34413239,362.61516726)
\curveto(32.25412889,362.61516735)(32.16412898,362.62516734)(32.07413239,362.64516726)
\curveto(31.98412916,362.67516729)(31.89412925,362.70016726)(31.80413239,362.72016726)
\curveto(31.71412943,362.74016722)(31.62912952,362.77016719)(31.54913239,362.81016726)
\curveto(31.30912984,362.92016704)(31.08413006,363.05016691)(30.87413239,363.20016726)
\curveto(30.66413048,363.3601666)(30.48413066,363.54016642)(30.33413239,363.74016726)
\curveto(30.21413093,363.91016605)(30.10913104,364.08516588)(30.01913239,364.26516726)
\curveto(29.92913122,364.44516552)(29.83913131,364.63516533)(29.74913239,364.83516726)
\curveto(29.70913144,364.93516503)(29.67413147,365.03516493)(29.64413239,365.13516726)
\curveto(29.62413152,365.24516472)(29.59913155,365.35516461)(29.56913239,365.46516726)
\curveto(29.52913162,365.60516436)(29.50413164,365.74516422)(29.49413239,365.88516726)
\curveto(29.48413166,366.02516394)(29.46413168,366.1651638)(29.43413239,366.30516726)
\curveto(29.42413172,366.41516355)(29.41413173,366.51516345)(29.40413239,366.60516726)
\curveto(29.40413174,366.70516326)(29.39413175,366.80516316)(29.37413239,366.90516726)
\lineto(29.37413239,366.99516726)
\curveto(29.38413176,367.02516294)(29.38413176,367.05016291)(29.37413239,367.07016726)
\lineto(29.37413239,367.28016726)
\curveto(29.35413179,367.34016262)(29.3441318,367.40516256)(29.34413239,367.47516726)
\curveto(29.35413179,367.55516241)(29.35913179,367.63016233)(29.35913239,367.70016726)
\lineto(29.35913239,367.85016726)
\curveto(29.35913179,367.90016206)(29.36413178,367.95016201)(29.37413239,368.00016726)
\lineto(29.37413239,368.37516726)
\curveto(29.38413176,368.40516156)(29.38413176,368.44016152)(29.37413239,368.48016726)
\curveto(29.37413177,368.52016144)(29.37913177,368.5601614)(29.38913239,368.60016726)
\curveto(29.40913174,368.71016125)(29.42413172,368.82016114)(29.43413239,368.93016726)
\curveto(29.4441317,369.05016091)(29.45413169,369.1651608)(29.46413239,369.27516726)
\curveto(29.50413164,369.42516054)(29.52913162,369.57016039)(29.53913239,369.71016726)
\curveto(29.55913159,369.8601601)(29.58913156,370.00515996)(29.62913239,370.14516726)
\curveto(29.71913143,370.44515952)(29.81413133,370.73015923)(29.91413239,371.00016726)
\curveto(30.01413113,371.27015869)(30.13913101,371.52015844)(30.28913239,371.75016726)
\curveto(30.48913066,372.07015789)(30.73413041,372.35015761)(31.02413239,372.59016726)
\curveto(31.31412983,372.83015713)(31.65412949,373.01515695)(32.04413239,373.14516726)
\curveto(32.15412899,373.18515678)(32.26412888,373.21015675)(32.37413239,373.22016726)
\curveto(32.49412865,373.24015672)(32.61412853,373.2651567)(32.73413239,373.29516726)
\curveto(32.80412834,373.30515666)(32.86912828,373.31015665)(32.92913239,373.31016726)
\curveto(32.98912816,373.31015665)(33.05412809,373.31515665)(33.12413239,373.32516726)
\curveto(33.82412732,373.34515662)(34.39912675,373.23015673)(34.84913239,372.98016726)
\curveto(35.29912585,372.73015723)(35.6441255,372.38015758)(35.88413239,371.93016726)
\curveto(35.99412515,371.70015826)(36.09412505,371.42515854)(36.18413239,371.10516726)
\curveto(36.20412494,371.03515893)(36.20412494,370.960159)(36.18413239,370.88016726)
\curveto(36.17412497,370.81015915)(36.149125,370.7601592)(36.10913239,370.73016726)
\curveto(36.07912507,370.70015926)(36.01912513,370.67515929)(35.92913239,370.65516726)
\curveto(35.83912531,370.64515932)(35.73912541,370.63515933)(35.62913239,370.62516726)
\curveto(35.52912562,370.62515934)(35.42912572,370.63015933)(35.32913239,370.64016726)
\curveto(35.23912591,370.65015931)(35.17412597,370.67015929)(35.13413239,370.70016726)
\curveto(35.02412612,370.77015919)(34.9441262,370.88015908)(34.89413239,371.03016726)
\curveto(34.85412629,371.18015878)(34.79912635,371.31015865)(34.72913239,371.42016726)
\curveto(34.53912661,371.73015823)(34.25912689,371.960158)(33.88913239,372.11016726)
\curveto(33.81912733,372.14015782)(33.7441274,372.1601578)(33.66413239,372.17016726)
\curveto(33.59412755,372.18015778)(33.51912763,372.19515777)(33.43913239,372.21516726)
\curveto(33.38912776,372.22515774)(33.31912783,372.23015773)(33.22913239,372.23016726)
\curveto(33.149128,372.23015773)(33.08412806,372.22515774)(33.03413239,372.21516726)
\curveto(32.99412815,372.19515777)(32.95912819,372.19015777)(32.92913239,372.20016726)
\curveto(32.89912825,372.21015775)(32.86412828,372.21015775)(32.82413239,372.20016726)
\lineto(32.58413239,372.14016726)
\curveto(32.51412863,372.12015784)(32.4441287,372.09515787)(32.37413239,372.06516726)
\curveto(31.99412915,371.90515806)(31.70412944,371.69515827)(31.50413239,371.43516726)
\curveto(31.31412983,371.17515879)(31.13913001,370.8601591)(30.97913239,370.49016726)
\curveto(30.9491302,370.41015955)(30.92413022,370.33015963)(30.90413239,370.25016726)
\curveto(30.89413025,370.17015979)(30.87413027,370.09015987)(30.84413239,370.01016726)
\curveto(30.81413033,369.90016006)(30.78913036,369.78516018)(30.76913239,369.66516726)
\curveto(30.75913039,369.54516042)(30.73913041,369.42516054)(30.70913239,369.30516726)
\curveto(30.68913046,369.25516071)(30.67913047,369.20516076)(30.67913239,369.15516726)
\curveto(30.68913046,369.10516086)(30.68413046,369.05516091)(30.66413239,369.00516726)
\curveto(30.65413049,368.94516102)(30.65413049,368.8651611)(30.66413239,368.76516726)
\curveto(30.67413047,368.67516129)(30.68913046,368.62016134)(30.70913239,368.60016726)
\curveto(30.72913042,368.5601614)(30.75913039,368.54016142)(30.79913239,368.54016726)
\curveto(30.8491303,368.54016142)(30.89413025,368.55016141)(30.93413239,368.57016726)
\curveto(31.00413014,368.61016135)(31.06413008,368.65516131)(31.11413239,368.70516726)
\curveto(31.16412998,368.75516121)(31.22412992,368.80516116)(31.29413239,368.85516726)
\lineto(31.35413239,368.91516726)
\curveto(31.38412976,368.94516102)(31.41412973,368.97016099)(31.44413239,368.99016726)
\curveto(31.67412947,369.15016081)(31.9491292,369.28516068)(32.26913239,369.39516726)
\curveto(32.33912881,369.41516055)(32.40912874,369.43016053)(32.47913239,369.44016726)
\curveto(32.5491286,369.45016051)(32.62412852,369.4651605)(32.70413239,369.48516726)
\curveto(32.7441284,369.48516048)(32.77912837,369.49016047)(32.80913239,369.50016726)
\curveto(32.83912831,369.51016045)(32.87412827,369.51016045)(32.91413239,369.50016726)
\curveto(32.96412818,369.50016046)(33.00412814,369.51016045)(33.03413239,369.53016726)
\lineto(33.19913239,369.53016726)
\lineto(33.28913239,369.53016726)
\curveto(33.33912781,369.54016042)(33.37912777,369.54016042)(33.40913239,369.53016726)
\curveto(33.45912769,369.52016044)(33.50912764,369.51516045)(33.55913239,369.51516726)
\curveto(33.61912753,369.52516044)(33.67412747,369.52516044)(33.72413239,369.51516726)
\curveto(33.83412731,369.48516048)(33.93912721,369.4651605)(34.03913239,369.45516726)
\curveto(34.149127,369.44516052)(34.25412689,369.42016054)(34.35413239,369.38016726)
\curveto(34.77412637,369.24016072)(35.11912603,369.05516091)(35.38913239,368.82516726)
\curveto(35.65912549,368.60516136)(35.89912525,368.32016164)(36.10913239,367.97016726)
\curveto(36.18912496,367.83016213)(36.25412489,367.68016228)(36.30413239,367.52016726)
\curveto(36.35412479,367.37016259)(36.40412474,367.21016275)(36.45413239,367.04016726)
\moveto(35.20913239,365.73516726)
\curveto(35.21912593,365.78516418)(35.22412592,365.83016413)(35.22413239,365.87016726)
\lineto(35.22413239,366.02016726)
\curveto(35.22412592,366.33016363)(35.18412596,366.61516335)(35.10413239,366.87516726)
\curveto(35.08412606,366.93516303)(35.06412608,366.99016297)(35.04413239,367.04016726)
\curveto(35.03412611,367.10016286)(35.01912613,367.15516281)(34.99913239,367.20516726)
\curveto(34.77912637,367.69516227)(34.43412671,368.04516192)(33.96413239,368.25516726)
\curveto(33.88412726,368.28516168)(33.80412734,368.31016165)(33.72413239,368.33016726)
\lineto(33.48413239,368.39016726)
\curveto(33.40412774,368.41016155)(33.31412783,368.42016154)(33.21413239,368.42016726)
\lineto(32.89913239,368.42016726)
\curveto(32.87912827,368.40016156)(32.83912831,368.39016157)(32.77913239,368.39016726)
\curveto(32.72912842,368.40016156)(32.68412846,368.40016156)(32.64413239,368.39016726)
\lineto(32.40413239,368.33016726)
\curveto(32.33412881,368.32016164)(32.26412888,368.30016166)(32.19413239,368.27016726)
\curveto(31.59412955,368.01016195)(31.18912996,367.54516242)(30.97913239,366.87516726)
\curveto(30.9491302,366.79516317)(30.92913022,366.71516325)(30.91913239,366.63516726)
\curveto(30.90913024,366.55516341)(30.89413025,366.47016349)(30.87413239,366.38016726)
\lineto(30.87413239,366.23016726)
\curveto(30.86413028,366.19016377)(30.85913029,366.12016384)(30.85913239,366.02016726)
\curveto(30.85913029,365.79016417)(30.87913027,365.59516437)(30.91913239,365.43516726)
\curveto(30.93913021,365.3651646)(30.95413019,365.30016466)(30.96413239,365.24016726)
\curveto(30.97413017,365.18016478)(30.99413015,365.11516485)(31.02413239,365.04516726)
\curveto(31.13413001,364.7651652)(31.27912987,364.52016544)(31.45913239,364.31016726)
\curveto(31.63912951,364.11016585)(31.87412927,363.95016601)(32.16413239,363.83016726)
\lineto(32.40413239,363.74016726)
\lineto(32.64413239,363.68016726)
\curveto(32.69412845,363.6601663)(32.73412841,363.65516631)(32.76413239,363.66516726)
\curveto(32.80412834,363.67516629)(32.8491283,363.67016629)(32.89913239,363.65016726)
\curveto(32.92912822,363.64016632)(32.98412816,363.63516633)(33.06413239,363.63516726)
\curveto(33.144128,363.63516633)(33.20412794,363.64016632)(33.24413239,363.65016726)
\curveto(33.35412779,363.67016629)(33.45912769,363.68516628)(33.55913239,363.69516726)
\curveto(33.65912749,363.70516626)(33.75412739,363.73516623)(33.84413239,363.78516726)
\curveto(34.37412677,363.98516598)(34.76412638,364.3601656)(35.01413239,364.91016726)
\curveto(35.05412609,365.01016495)(35.08412606,365.11516485)(35.10413239,365.22516726)
\lineto(35.19413239,365.55516726)
\curveto(35.19412595,365.63516433)(35.19912595,365.69516427)(35.20913239,365.73516726)
}
}
{
\newrgbcolor{curcolor}{0 0 0}
\pscustom[linestyle=none,fillstyle=solid,fillcolor=curcolor]
{
\newpath
\moveto(44.80374176,367.80516726)
\lineto(44.80374176,367.55016726)
\curveto(44.81373406,367.47016249)(44.80873406,367.39516257)(44.78874176,367.32516726)
\lineto(44.78874176,367.08516726)
\lineto(44.78874176,366.92016726)
\curveto(44.7687341,366.82016314)(44.75873411,366.71516325)(44.75874176,366.60516726)
\curveto(44.75873411,366.50516346)(44.74873412,366.40516356)(44.72874176,366.30516726)
\lineto(44.72874176,366.15516726)
\curveto(44.69873417,366.01516395)(44.67873419,365.87516409)(44.66874176,365.73516726)
\curveto(44.65873421,365.60516436)(44.63373424,365.47516449)(44.59374176,365.34516726)
\curveto(44.5737343,365.2651647)(44.55373432,365.18016478)(44.53374176,365.09016726)
\lineto(44.47374176,364.85016726)
\lineto(44.35374176,364.55016726)
\curveto(44.32373455,364.4601655)(44.28873458,364.37016559)(44.24874176,364.28016726)
\curveto(44.14873472,364.0601659)(44.01373486,363.84516612)(43.84374176,363.63516726)
\curveto(43.68373519,363.42516654)(43.50873536,363.25516671)(43.31874176,363.12516726)
\curveto(43.2687356,363.08516688)(43.20873566,363.04516692)(43.13874176,363.00516726)
\curveto(43.07873579,362.97516699)(43.01873585,362.94016702)(42.95874176,362.90016726)
\curveto(42.87873599,362.85016711)(42.78373609,362.81016715)(42.67374176,362.78016726)
\curveto(42.56373631,362.75016721)(42.45873641,362.72016724)(42.35874176,362.69016726)
\curveto(42.24873662,362.65016731)(42.13873673,362.62516734)(42.02874176,362.61516726)
\curveto(41.91873695,362.60516736)(41.80373707,362.59016737)(41.68374176,362.57016726)
\curveto(41.64373723,362.5601674)(41.59873727,362.5601674)(41.54874176,362.57016726)
\curveto(41.50873736,362.57016739)(41.4687374,362.5651674)(41.42874176,362.55516726)
\curveto(41.38873748,362.54516742)(41.33373754,362.54016742)(41.26374176,362.54016726)
\curveto(41.19373768,362.54016742)(41.14373773,362.54516742)(41.11374176,362.55516726)
\curveto(41.06373781,362.57516739)(41.01873785,362.58016738)(40.97874176,362.57016726)
\curveto(40.93873793,362.5601674)(40.90373797,362.5601674)(40.87374176,362.57016726)
\lineto(40.78374176,362.57016726)
\curveto(40.72373815,362.59016737)(40.65873821,362.60516736)(40.58874176,362.61516726)
\curveto(40.52873834,362.61516735)(40.46373841,362.62016734)(40.39374176,362.63016726)
\curveto(40.22373865,362.68016728)(40.06373881,362.73016723)(39.91374176,362.78016726)
\curveto(39.76373911,362.83016713)(39.61873925,362.89516707)(39.47874176,362.97516726)
\curveto(39.42873944,363.01516695)(39.3737395,363.04516692)(39.31374176,363.06516726)
\curveto(39.26373961,363.09516687)(39.21373966,363.13016683)(39.16374176,363.17016726)
\curveto(38.92373995,363.35016661)(38.72374015,363.57016639)(38.56374176,363.83016726)
\curveto(38.40374047,364.09016587)(38.26374061,364.37516559)(38.14374176,364.68516726)
\curveto(38.08374079,364.82516514)(38.03874083,364.965165)(38.00874176,365.10516726)
\curveto(37.97874089,365.25516471)(37.94374093,365.41016455)(37.90374176,365.57016726)
\curveto(37.88374099,365.68016428)(37.868741,365.79016417)(37.85874176,365.90016726)
\curveto(37.84874102,366.01016395)(37.83374104,366.12016384)(37.81374176,366.23016726)
\curveto(37.80374107,366.27016369)(37.79874107,366.31016365)(37.79874176,366.35016726)
\curveto(37.80874106,366.39016357)(37.80874106,366.43016353)(37.79874176,366.47016726)
\curveto(37.78874108,366.52016344)(37.78374109,366.57016339)(37.78374176,366.62016726)
\lineto(37.78374176,366.78516726)
\curveto(37.76374111,366.83516313)(37.75874111,366.88516308)(37.76874176,366.93516726)
\curveto(37.77874109,366.99516297)(37.77874109,367.05016291)(37.76874176,367.10016726)
\curveto(37.75874111,367.14016282)(37.75874111,367.18516278)(37.76874176,367.23516726)
\curveto(37.77874109,367.28516268)(37.7737411,367.33516263)(37.75374176,367.38516726)
\curveto(37.73374114,367.45516251)(37.72874114,367.53016243)(37.73874176,367.61016726)
\curveto(37.74874112,367.70016226)(37.75374112,367.78516218)(37.75374176,367.86516726)
\curveto(37.75374112,367.95516201)(37.74874112,368.05516191)(37.73874176,368.16516726)
\curveto(37.72874114,368.28516168)(37.73374114,368.38516158)(37.75374176,368.46516726)
\lineto(37.75374176,368.75016726)
\lineto(37.79874176,369.38016726)
\curveto(37.80874106,369.48016048)(37.81874105,369.57516039)(37.82874176,369.66516726)
\lineto(37.85874176,369.96516726)
\curveto(37.87874099,370.01515995)(37.88374099,370.0651599)(37.87374176,370.11516726)
\curveto(37.873741,370.17515979)(37.88374099,370.23015973)(37.90374176,370.28016726)
\curveto(37.95374092,370.45015951)(37.99374088,370.61515935)(38.02374176,370.77516726)
\curveto(38.05374082,370.94515902)(38.10374077,371.10515886)(38.17374176,371.25516726)
\curveto(38.36374051,371.71515825)(38.58374029,372.09015787)(38.83374176,372.38016726)
\curveto(39.09373978,372.67015729)(39.45373942,372.91515705)(39.91374176,373.11516726)
\curveto(40.04373883,373.1651568)(40.1737387,373.20015676)(40.30374176,373.22016726)
\curveto(40.44373843,373.24015672)(40.58373829,373.2651567)(40.72374176,373.29516726)
\curveto(40.79373808,373.30515666)(40.85873801,373.31015665)(40.91874176,373.31016726)
\curveto(40.97873789,373.31015665)(41.04373783,373.31515665)(41.11374176,373.32516726)
\curveto(41.94373693,373.34515662)(42.61373626,373.19515677)(43.12374176,372.87516726)
\curveto(43.63373524,372.5651574)(44.01373486,372.12515784)(44.26374176,371.55516726)
\curveto(44.31373456,371.43515853)(44.35873451,371.31015865)(44.39874176,371.18016726)
\curveto(44.43873443,371.05015891)(44.48373439,370.91515905)(44.53374176,370.77516726)
\curveto(44.55373432,370.69515927)(44.5687343,370.61015935)(44.57874176,370.52016726)
\lineto(44.63874176,370.28016726)
\curveto(44.6687342,370.17015979)(44.68373419,370.0601599)(44.68374176,369.95016726)
\curveto(44.69373418,369.84016012)(44.70873416,369.73016023)(44.72874176,369.62016726)
\curveto(44.74873412,369.57016039)(44.75373412,369.52516044)(44.74374176,369.48516726)
\curveto(44.74373413,369.44516052)(44.74873412,369.40516056)(44.75874176,369.36516726)
\curveto(44.7687341,369.31516065)(44.7687341,369.2601607)(44.75874176,369.20016726)
\curveto(44.75873411,369.15016081)(44.76373411,369.10016086)(44.77374176,369.05016726)
\lineto(44.77374176,368.91516726)
\curveto(44.79373408,368.85516111)(44.79373408,368.78516118)(44.77374176,368.70516726)
\curveto(44.76373411,368.63516133)(44.7687341,368.57016139)(44.78874176,368.51016726)
\curveto(44.79873407,368.48016148)(44.80373407,368.44016152)(44.80374176,368.39016726)
\lineto(44.80374176,368.27016726)
\lineto(44.80374176,367.80516726)
\moveto(43.25874176,365.48016726)
\curveto(43.35873551,365.80016416)(43.41873545,366.1651638)(43.43874176,366.57516726)
\curveto(43.45873541,366.98516298)(43.4687354,367.39516257)(43.46874176,367.80516726)
\curveto(43.4687354,368.23516173)(43.45873541,368.65516131)(43.43874176,369.06516726)
\curveto(43.41873545,369.47516049)(43.3737355,369.8601601)(43.30374176,370.22016726)
\curveto(43.23373564,370.58015938)(43.12373575,370.90015906)(42.97374176,371.18016726)
\curveto(42.83373604,371.47015849)(42.63873623,371.70515826)(42.38874176,371.88516726)
\curveto(42.22873664,371.99515797)(42.04873682,372.07515789)(41.84874176,372.12516726)
\curveto(41.64873722,372.18515778)(41.40373747,372.21515775)(41.11374176,372.21516726)
\curveto(41.09373778,372.19515777)(41.05873781,372.18515778)(41.00874176,372.18516726)
\curveto(40.95873791,372.19515777)(40.91873795,372.19515777)(40.88874176,372.18516726)
\curveto(40.80873806,372.1651578)(40.73373814,372.14515782)(40.66374176,372.12516726)
\curveto(40.60373827,372.11515785)(40.53873833,372.09515787)(40.46874176,372.06516726)
\curveto(40.19873867,371.94515802)(39.97873889,371.77515819)(39.80874176,371.55516726)
\curveto(39.64873922,371.34515862)(39.51373936,371.10015886)(39.40374176,370.82016726)
\curveto(39.35373952,370.71015925)(39.31373956,370.59015937)(39.28374176,370.46016726)
\curveto(39.26373961,370.34015962)(39.23873963,370.21515975)(39.20874176,370.08516726)
\curveto(39.18873968,370.03515993)(39.17873969,369.98015998)(39.17874176,369.92016726)
\curveto(39.17873969,369.87016009)(39.1737397,369.82016014)(39.16374176,369.77016726)
\curveto(39.15373972,369.68016028)(39.14373973,369.58516038)(39.13374176,369.48516726)
\curveto(39.12373975,369.39516057)(39.11373976,369.30016066)(39.10374176,369.20016726)
\curveto(39.10373977,369.12016084)(39.09873977,369.03516093)(39.08874176,368.94516726)
\lineto(39.08874176,368.70516726)
\lineto(39.08874176,368.52516726)
\curveto(39.07873979,368.49516147)(39.0737398,368.4601615)(39.07374176,368.42016726)
\lineto(39.07374176,368.28516726)
\lineto(39.07374176,367.83516726)
\curveto(39.0737398,367.75516221)(39.0687398,367.67016229)(39.05874176,367.58016726)
\curveto(39.05873981,367.50016246)(39.0687398,367.42516254)(39.08874176,367.35516726)
\lineto(39.08874176,367.08516726)
\curveto(39.08873978,367.0651629)(39.08373979,367.03516293)(39.07374176,366.99516726)
\curveto(39.0737398,366.965163)(39.07873979,366.94016302)(39.08874176,366.92016726)
\curveto(39.09873977,366.82016314)(39.10373977,366.72016324)(39.10374176,366.62016726)
\curveto(39.11373976,366.53016343)(39.12373975,366.43016353)(39.13374176,366.32016726)
\curveto(39.16373971,366.20016376)(39.17873969,366.07516389)(39.17874176,365.94516726)
\curveto(39.18873968,365.82516414)(39.21373966,365.71016425)(39.25374176,365.60016726)
\curveto(39.33373954,365.30016466)(39.41873945,365.03516493)(39.50874176,364.80516726)
\curveto(39.60873926,364.57516539)(39.75373912,364.3601656)(39.94374176,364.16016726)
\curveto(40.15373872,363.960166)(40.41873845,363.81016615)(40.73874176,363.71016726)
\curveto(40.77873809,363.69016627)(40.81373806,363.68016628)(40.84374176,363.68016726)
\curveto(40.88373799,363.69016627)(40.92873794,363.68516628)(40.97874176,363.66516726)
\curveto(41.01873785,363.65516631)(41.08873778,363.64516632)(41.18874176,363.63516726)
\curveto(41.29873757,363.62516634)(41.38373749,363.63016633)(41.44374176,363.65016726)
\curveto(41.51373736,363.67016629)(41.58373729,363.68016628)(41.65374176,363.68016726)
\curveto(41.72373715,363.69016627)(41.78873708,363.70516626)(41.84874176,363.72516726)
\curveto(42.04873682,363.78516618)(42.22873664,363.87016609)(42.38874176,363.98016726)
\curveto(42.41873645,364.00016596)(42.44373643,364.02016594)(42.46374176,364.04016726)
\lineto(42.52374176,364.10016726)
\curveto(42.56373631,364.12016584)(42.61373626,364.1601658)(42.67374176,364.22016726)
\curveto(42.7737361,364.3601656)(42.85873601,364.49016547)(42.92874176,364.61016726)
\curveto(42.99873587,364.73016523)(43.0687358,364.87516509)(43.13874176,365.04516726)
\curveto(43.1687357,365.11516485)(43.18873568,365.18516478)(43.19874176,365.25516726)
\curveto(43.21873565,365.32516464)(43.23873563,365.40016456)(43.25874176,365.48016726)
}
}
{
\newrgbcolor{curcolor}{0 0 0}
\pscustom[linestyle=none,fillstyle=solid,fillcolor=curcolor]
{
\newpath
\moveto(36.33413239,286.00807497)
\curveto(36.40412474,285.95807151)(36.4441247,285.88807158)(36.45413239,285.79807497)
\curveto(36.47412467,285.70807176)(36.48412466,285.60307187)(36.48413239,285.48307497)
\curveto(36.48412466,285.43307204)(36.47912467,285.38307209)(36.46913239,285.33307497)
\curveto(36.46912468,285.28307219)(36.45912469,285.23807223)(36.43913239,285.19807497)
\curveto(36.40912474,285.10807236)(36.3491248,285.04807242)(36.25913239,285.01807497)
\curveto(36.17912497,284.99807247)(36.08412506,284.98807248)(35.97413239,284.98807497)
\lineto(35.65913239,284.98807497)
\curveto(35.5491256,284.99807247)(35.4441257,284.98807248)(35.34413239,284.95807497)
\curveto(35.20412594,284.92807254)(35.11412603,284.84807262)(35.07413239,284.71807497)
\curveto(35.05412609,284.64807282)(35.0441261,284.56307291)(35.04413239,284.46307497)
\lineto(35.04413239,284.19307497)
\lineto(35.04413239,283.24807497)
\lineto(35.04413239,282.91807497)
\curveto(35.0441261,282.80807466)(35.02412612,282.72307475)(34.98413239,282.66307497)
\curveto(34.9441262,282.60307487)(34.89412625,282.56307491)(34.83413239,282.54307497)
\curveto(34.78412636,282.53307494)(34.71912643,282.51807495)(34.63913239,282.49807497)
\lineto(34.44413239,282.49807497)
\curveto(34.32412682,282.49807497)(34.21912693,282.50307497)(34.12913239,282.51307497)
\curveto(34.03912711,282.53307494)(33.96912718,282.58307489)(33.91913239,282.66307497)
\curveto(33.88912726,282.71307476)(33.87412727,282.78307469)(33.87413239,282.87307497)
\lineto(33.87413239,283.17307497)
\lineto(33.87413239,284.20807497)
\curveto(33.87412727,284.3680731)(33.86412728,284.51307296)(33.84413239,284.64307497)
\curveto(33.83412731,284.78307269)(33.77912737,284.87807259)(33.67913239,284.92807497)
\curveto(33.62912752,284.94807252)(33.55912759,284.96307251)(33.46913239,284.97307497)
\curveto(33.38912776,284.98307249)(33.29912785,284.98807248)(33.19913239,284.98807497)
\lineto(32.91413239,284.98807497)
\lineto(32.67413239,284.98807497)
\lineto(30.40913239,284.98807497)
\curveto(30.31913083,284.98807248)(30.21413093,284.98307249)(30.09413239,284.97307497)
\lineto(29.76413239,284.97307497)
\curveto(29.65413149,284.9730725)(29.55413159,284.98307249)(29.46413239,285.00307497)
\curveto(29.37413177,285.02307245)(29.31413183,285.05807241)(29.28413239,285.10807497)
\curveto(29.23413191,285.17807229)(29.20913194,285.2730722)(29.20913239,285.39307497)
\lineto(29.20913239,285.73807497)
\lineto(29.20913239,286.00807497)
\curveto(29.2491319,286.17807129)(29.30413184,286.31807115)(29.37413239,286.42807497)
\curveto(29.4441317,286.53807093)(29.52413162,286.65307082)(29.61413239,286.77307497)
\lineto(29.97413239,287.31307497)
\curveto(30.41413073,287.94306953)(30.8491303,288.56306891)(31.27913239,289.17307497)
\lineto(32.59913239,291.03307497)
\curveto(32.75912839,291.26306621)(32.91412823,291.48306599)(33.06413239,291.69307497)
\curveto(33.21412793,291.91306556)(33.36912778,292.13806533)(33.52913239,292.36807497)
\curveto(33.57912757,292.43806503)(33.62912752,292.50306497)(33.67913239,292.56307497)
\curveto(33.72912742,292.63306484)(33.77912737,292.70806476)(33.82913239,292.78807497)
\lineto(33.88913239,292.87807497)
\curveto(33.91912723,292.91806455)(33.9491272,292.94806452)(33.97913239,292.96807497)
\curveto(34.01912713,292.99806447)(34.05912709,293.01806445)(34.09913239,293.02807497)
\curveto(34.13912701,293.04806442)(34.18412696,293.0680644)(34.23413239,293.08807497)
\curveto(34.25412689,293.08806438)(34.27412687,293.08306439)(34.29413239,293.07307497)
\curveto(34.32412682,293.0730644)(34.3491268,293.08306439)(34.36913239,293.10307497)
\curveto(34.49912665,293.10306437)(34.61912653,293.09806437)(34.72913239,293.08807497)
\curveto(34.83912631,293.07806439)(34.91912623,293.03306444)(34.96913239,292.95307497)
\curveto(35.00912614,292.90306457)(35.02912612,292.83306464)(35.02913239,292.74307497)
\curveto(35.03912611,292.65306482)(35.0441261,292.55806491)(35.04413239,292.45807497)
\lineto(35.04413239,286.99807497)
\curveto(35.0441261,286.92807054)(35.03912611,286.85307062)(35.02913239,286.77307497)
\curveto(35.02912612,286.70307077)(35.03412611,286.63307084)(35.04413239,286.56307497)
\lineto(35.04413239,286.45807497)
\curveto(35.06412608,286.40807106)(35.07912607,286.35307112)(35.08913239,286.29307497)
\curveto(35.09912605,286.24307123)(35.12412602,286.20307127)(35.16413239,286.17307497)
\curveto(35.23412591,286.12307135)(35.31912583,286.09307138)(35.41913239,286.08307497)
\lineto(35.74913239,286.08307497)
\curveto(35.85912529,286.08307139)(35.96412518,286.07807139)(36.06413239,286.06807497)
\curveto(36.17412497,286.0680714)(36.26412488,286.04807142)(36.33413239,286.00807497)
\moveto(33.76913239,286.20307497)
\curveto(33.8491273,286.31307116)(33.88412726,286.48307099)(33.87413239,286.71307497)
\lineto(33.87413239,287.32807497)
\lineto(33.87413239,289.80307497)
\lineto(33.87413239,290.11807497)
\curveto(33.88412726,290.23806723)(33.87912727,290.33806713)(33.85913239,290.41807497)
\lineto(33.85913239,290.56807497)
\curveto(33.85912729,290.65806681)(33.8441273,290.74306673)(33.81413239,290.82307497)
\curveto(33.80412734,290.84306663)(33.79412735,290.85306662)(33.78413239,290.85307497)
\lineto(33.73913239,290.89807497)
\curveto(33.71912743,290.90806656)(33.68912746,290.91306656)(33.64913239,290.91307497)
\curveto(33.62912752,290.89306658)(33.60912754,290.87806659)(33.58913239,290.86807497)
\curveto(33.57912757,290.8680666)(33.56412758,290.86306661)(33.54413239,290.85307497)
\curveto(33.48412766,290.80306667)(33.42412772,290.73306674)(33.36413239,290.64307497)
\curveto(33.30412784,290.55306692)(33.2491279,290.473067)(33.19913239,290.40307497)
\curveto(33.09912805,290.26306721)(33.00412814,290.11806735)(32.91413239,289.96807497)
\curveto(32.82412832,289.82806764)(32.72912842,289.68806778)(32.62913239,289.54807497)
\lineto(32.08913239,288.76807497)
\curveto(31.91912923,288.50806896)(31.7441294,288.24806922)(31.56413239,287.98807497)
\curveto(31.48412966,287.87806959)(31.40912974,287.7730697)(31.33913239,287.67307497)
\lineto(31.12913239,287.37307497)
\curveto(31.07913007,287.29307018)(31.02913012,287.21807025)(30.97913239,287.14807497)
\curveto(30.93913021,287.07807039)(30.89413025,287.00307047)(30.84413239,286.92307497)
\curveto(30.79413035,286.86307061)(30.7441304,286.79807067)(30.69413239,286.72807497)
\curveto(30.65413049,286.6680708)(30.61413053,286.59807087)(30.57413239,286.51807497)
\curveto(30.53413061,286.45807101)(30.50913064,286.38807108)(30.49913239,286.30807497)
\curveto(30.48913066,286.23807123)(30.52413062,286.18307129)(30.60413239,286.14307497)
\curveto(30.67413047,286.09307138)(30.78413036,286.0680714)(30.93413239,286.06807497)
\curveto(31.09413005,286.07807139)(31.22912992,286.08307139)(31.33913239,286.08307497)
\lineto(33.01913239,286.08307497)
\lineto(33.45413239,286.08307497)
\curveto(33.60412754,286.08307139)(33.70912744,286.12307135)(33.76913239,286.20307497)
}
}
{
\newrgbcolor{curcolor}{0 0 0}
\pscustom[linestyle=none,fillstyle=solid,fillcolor=curcolor]
{
\newpath
\moveto(44.80374176,287.59807497)
\lineto(44.80374176,287.34307497)
\curveto(44.81373406,287.26307021)(44.80873406,287.18807028)(44.78874176,287.11807497)
\lineto(44.78874176,286.87807497)
\lineto(44.78874176,286.71307497)
\curveto(44.7687341,286.61307086)(44.75873411,286.50807096)(44.75874176,286.39807497)
\curveto(44.75873411,286.29807117)(44.74873412,286.19807127)(44.72874176,286.09807497)
\lineto(44.72874176,285.94807497)
\curveto(44.69873417,285.80807166)(44.67873419,285.6680718)(44.66874176,285.52807497)
\curveto(44.65873421,285.39807207)(44.63373424,285.2680722)(44.59374176,285.13807497)
\curveto(44.5737343,285.05807241)(44.55373432,284.9730725)(44.53374176,284.88307497)
\lineto(44.47374176,284.64307497)
\lineto(44.35374176,284.34307497)
\curveto(44.32373455,284.25307322)(44.28873458,284.16307331)(44.24874176,284.07307497)
\curveto(44.14873472,283.85307362)(44.01373486,283.63807383)(43.84374176,283.42807497)
\curveto(43.68373519,283.21807425)(43.50873536,283.04807442)(43.31874176,282.91807497)
\curveto(43.2687356,282.87807459)(43.20873566,282.83807463)(43.13874176,282.79807497)
\curveto(43.07873579,282.7680747)(43.01873585,282.73307474)(42.95874176,282.69307497)
\curveto(42.87873599,282.64307483)(42.78373609,282.60307487)(42.67374176,282.57307497)
\curveto(42.56373631,282.54307493)(42.45873641,282.51307496)(42.35874176,282.48307497)
\curveto(42.24873662,282.44307503)(42.13873673,282.41807505)(42.02874176,282.40807497)
\curveto(41.91873695,282.39807507)(41.80373707,282.38307509)(41.68374176,282.36307497)
\curveto(41.64373723,282.35307512)(41.59873727,282.35307512)(41.54874176,282.36307497)
\curveto(41.50873736,282.36307511)(41.4687374,282.35807511)(41.42874176,282.34807497)
\curveto(41.38873748,282.33807513)(41.33373754,282.33307514)(41.26374176,282.33307497)
\curveto(41.19373768,282.33307514)(41.14373773,282.33807513)(41.11374176,282.34807497)
\curveto(41.06373781,282.3680751)(41.01873785,282.3730751)(40.97874176,282.36307497)
\curveto(40.93873793,282.35307512)(40.90373797,282.35307512)(40.87374176,282.36307497)
\lineto(40.78374176,282.36307497)
\curveto(40.72373815,282.38307509)(40.65873821,282.39807507)(40.58874176,282.40807497)
\curveto(40.52873834,282.40807506)(40.46373841,282.41307506)(40.39374176,282.42307497)
\curveto(40.22373865,282.473075)(40.06373881,282.52307495)(39.91374176,282.57307497)
\curveto(39.76373911,282.62307485)(39.61873925,282.68807478)(39.47874176,282.76807497)
\curveto(39.42873944,282.80807466)(39.3737395,282.83807463)(39.31374176,282.85807497)
\curveto(39.26373961,282.88807458)(39.21373966,282.92307455)(39.16374176,282.96307497)
\curveto(38.92373995,283.14307433)(38.72374015,283.36307411)(38.56374176,283.62307497)
\curveto(38.40374047,283.88307359)(38.26374061,284.1680733)(38.14374176,284.47807497)
\curveto(38.08374079,284.61807285)(38.03874083,284.75807271)(38.00874176,284.89807497)
\curveto(37.97874089,285.04807242)(37.94374093,285.20307227)(37.90374176,285.36307497)
\curveto(37.88374099,285.473072)(37.868741,285.58307189)(37.85874176,285.69307497)
\curveto(37.84874102,285.80307167)(37.83374104,285.91307156)(37.81374176,286.02307497)
\curveto(37.80374107,286.06307141)(37.79874107,286.10307137)(37.79874176,286.14307497)
\curveto(37.80874106,286.18307129)(37.80874106,286.22307125)(37.79874176,286.26307497)
\curveto(37.78874108,286.31307116)(37.78374109,286.36307111)(37.78374176,286.41307497)
\lineto(37.78374176,286.57807497)
\curveto(37.76374111,286.62807084)(37.75874111,286.67807079)(37.76874176,286.72807497)
\curveto(37.77874109,286.78807068)(37.77874109,286.84307063)(37.76874176,286.89307497)
\curveto(37.75874111,286.93307054)(37.75874111,286.97807049)(37.76874176,287.02807497)
\curveto(37.77874109,287.07807039)(37.7737411,287.12807034)(37.75374176,287.17807497)
\curveto(37.73374114,287.24807022)(37.72874114,287.32307015)(37.73874176,287.40307497)
\curveto(37.74874112,287.49306998)(37.75374112,287.57806989)(37.75374176,287.65807497)
\curveto(37.75374112,287.74806972)(37.74874112,287.84806962)(37.73874176,287.95807497)
\curveto(37.72874114,288.07806939)(37.73374114,288.17806929)(37.75374176,288.25807497)
\lineto(37.75374176,288.54307497)
\lineto(37.79874176,289.17307497)
\curveto(37.80874106,289.2730682)(37.81874105,289.3680681)(37.82874176,289.45807497)
\lineto(37.85874176,289.75807497)
\curveto(37.87874099,289.80806766)(37.88374099,289.85806761)(37.87374176,289.90807497)
\curveto(37.873741,289.9680675)(37.88374099,290.02306745)(37.90374176,290.07307497)
\curveto(37.95374092,290.24306723)(37.99374088,290.40806706)(38.02374176,290.56807497)
\curveto(38.05374082,290.73806673)(38.10374077,290.89806657)(38.17374176,291.04807497)
\curveto(38.36374051,291.50806596)(38.58374029,291.88306559)(38.83374176,292.17307497)
\curveto(39.09373978,292.46306501)(39.45373942,292.70806476)(39.91374176,292.90807497)
\curveto(40.04373883,292.95806451)(40.1737387,292.99306448)(40.30374176,293.01307497)
\curveto(40.44373843,293.03306444)(40.58373829,293.05806441)(40.72374176,293.08807497)
\curveto(40.79373808,293.09806437)(40.85873801,293.10306437)(40.91874176,293.10307497)
\curveto(40.97873789,293.10306437)(41.04373783,293.10806436)(41.11374176,293.11807497)
\curveto(41.94373693,293.13806433)(42.61373626,292.98806448)(43.12374176,292.66807497)
\curveto(43.63373524,292.35806511)(44.01373486,291.91806555)(44.26374176,291.34807497)
\curveto(44.31373456,291.22806624)(44.35873451,291.10306637)(44.39874176,290.97307497)
\curveto(44.43873443,290.84306663)(44.48373439,290.70806676)(44.53374176,290.56807497)
\curveto(44.55373432,290.48806698)(44.5687343,290.40306707)(44.57874176,290.31307497)
\lineto(44.63874176,290.07307497)
\curveto(44.6687342,289.96306751)(44.68373419,289.85306762)(44.68374176,289.74307497)
\curveto(44.69373418,289.63306784)(44.70873416,289.52306795)(44.72874176,289.41307497)
\curveto(44.74873412,289.36306811)(44.75373412,289.31806815)(44.74374176,289.27807497)
\curveto(44.74373413,289.23806823)(44.74873412,289.19806827)(44.75874176,289.15807497)
\curveto(44.7687341,289.10806836)(44.7687341,289.05306842)(44.75874176,288.99307497)
\curveto(44.75873411,288.94306853)(44.76373411,288.89306858)(44.77374176,288.84307497)
\lineto(44.77374176,288.70807497)
\curveto(44.79373408,288.64806882)(44.79373408,288.57806889)(44.77374176,288.49807497)
\curveto(44.76373411,288.42806904)(44.7687341,288.36306911)(44.78874176,288.30307497)
\curveto(44.79873407,288.2730692)(44.80373407,288.23306924)(44.80374176,288.18307497)
\lineto(44.80374176,288.06307497)
\lineto(44.80374176,287.59807497)
\moveto(43.25874176,285.27307497)
\curveto(43.35873551,285.59307188)(43.41873545,285.95807151)(43.43874176,286.36807497)
\curveto(43.45873541,286.77807069)(43.4687354,287.18807028)(43.46874176,287.59807497)
\curveto(43.4687354,288.02806944)(43.45873541,288.44806902)(43.43874176,288.85807497)
\curveto(43.41873545,289.2680682)(43.3737355,289.65306782)(43.30374176,290.01307497)
\curveto(43.23373564,290.3730671)(43.12373575,290.69306678)(42.97374176,290.97307497)
\curveto(42.83373604,291.26306621)(42.63873623,291.49806597)(42.38874176,291.67807497)
\curveto(42.22873664,291.78806568)(42.04873682,291.8680656)(41.84874176,291.91807497)
\curveto(41.64873722,291.97806549)(41.40373747,292.00806546)(41.11374176,292.00807497)
\curveto(41.09373778,291.98806548)(41.05873781,291.97806549)(41.00874176,291.97807497)
\curveto(40.95873791,291.98806548)(40.91873795,291.98806548)(40.88874176,291.97807497)
\curveto(40.80873806,291.95806551)(40.73373814,291.93806553)(40.66374176,291.91807497)
\curveto(40.60373827,291.90806556)(40.53873833,291.88806558)(40.46874176,291.85807497)
\curveto(40.19873867,291.73806573)(39.97873889,291.5680659)(39.80874176,291.34807497)
\curveto(39.64873922,291.13806633)(39.51373936,290.89306658)(39.40374176,290.61307497)
\curveto(39.35373952,290.50306697)(39.31373956,290.38306709)(39.28374176,290.25307497)
\curveto(39.26373961,290.13306734)(39.23873963,290.00806746)(39.20874176,289.87807497)
\curveto(39.18873968,289.82806764)(39.17873969,289.7730677)(39.17874176,289.71307497)
\curveto(39.17873969,289.66306781)(39.1737397,289.61306786)(39.16374176,289.56307497)
\curveto(39.15373972,289.473068)(39.14373973,289.37806809)(39.13374176,289.27807497)
\curveto(39.12373975,289.18806828)(39.11373976,289.09306838)(39.10374176,288.99307497)
\curveto(39.10373977,288.91306856)(39.09873977,288.82806864)(39.08874176,288.73807497)
\lineto(39.08874176,288.49807497)
\lineto(39.08874176,288.31807497)
\curveto(39.07873979,288.28806918)(39.0737398,288.25306922)(39.07374176,288.21307497)
\lineto(39.07374176,288.07807497)
\lineto(39.07374176,287.62807497)
\curveto(39.0737398,287.54806992)(39.0687398,287.46307001)(39.05874176,287.37307497)
\curveto(39.05873981,287.29307018)(39.0687398,287.21807025)(39.08874176,287.14807497)
\lineto(39.08874176,286.87807497)
\curveto(39.08873978,286.85807061)(39.08373979,286.82807064)(39.07374176,286.78807497)
\curveto(39.0737398,286.75807071)(39.07873979,286.73307074)(39.08874176,286.71307497)
\curveto(39.09873977,286.61307086)(39.10373977,286.51307096)(39.10374176,286.41307497)
\curveto(39.11373976,286.32307115)(39.12373975,286.22307125)(39.13374176,286.11307497)
\curveto(39.16373971,285.99307148)(39.17873969,285.8680716)(39.17874176,285.73807497)
\curveto(39.18873968,285.61807185)(39.21373966,285.50307197)(39.25374176,285.39307497)
\curveto(39.33373954,285.09307238)(39.41873945,284.82807264)(39.50874176,284.59807497)
\curveto(39.60873926,284.3680731)(39.75373912,284.15307332)(39.94374176,283.95307497)
\curveto(40.15373872,283.75307372)(40.41873845,283.60307387)(40.73874176,283.50307497)
\curveto(40.77873809,283.48307399)(40.81373806,283.473074)(40.84374176,283.47307497)
\curveto(40.88373799,283.48307399)(40.92873794,283.47807399)(40.97874176,283.45807497)
\curveto(41.01873785,283.44807402)(41.08873778,283.43807403)(41.18874176,283.42807497)
\curveto(41.29873757,283.41807405)(41.38373749,283.42307405)(41.44374176,283.44307497)
\curveto(41.51373736,283.46307401)(41.58373729,283.473074)(41.65374176,283.47307497)
\curveto(41.72373715,283.48307399)(41.78873708,283.49807397)(41.84874176,283.51807497)
\curveto(42.04873682,283.57807389)(42.22873664,283.66307381)(42.38874176,283.77307497)
\curveto(42.41873645,283.79307368)(42.44373643,283.81307366)(42.46374176,283.83307497)
\lineto(42.52374176,283.89307497)
\curveto(42.56373631,283.91307356)(42.61373626,283.95307352)(42.67374176,284.01307497)
\curveto(42.7737361,284.15307332)(42.85873601,284.28307319)(42.92874176,284.40307497)
\curveto(42.99873587,284.52307295)(43.0687358,284.6680728)(43.13874176,284.83807497)
\curveto(43.1687357,284.90807256)(43.18873568,284.97807249)(43.19874176,285.04807497)
\curveto(43.21873565,285.11807235)(43.23873563,285.19307228)(43.25874176,285.27307497)
}
}
{
\newrgbcolor{curcolor}{0 0 0}
\pscustom[linestyle=none,fillstyle=solid,fillcolor=curcolor]
{
\newpath
\moveto(32.70413239,212.95593508)
\curveto(33.39412775,212.96592445)(33.99412715,212.84592457)(34.50413239,212.59593508)
\curveto(35.02412612,212.34592507)(35.41912573,212.0109254)(35.68913239,211.59093508)
\curveto(35.73912541,211.5109259)(35.78412536,211.42092599)(35.82413239,211.32093508)
\curveto(35.86412528,211.23092618)(35.90912524,211.13592628)(35.95913239,211.03593508)
\curveto(35.99912515,210.93592648)(36.02912512,210.83592658)(36.04913239,210.73593508)
\curveto(36.06912508,210.63592678)(36.08912506,210.53092688)(36.10913239,210.42093508)
\curveto(36.12912502,210.37092704)(36.13412501,210.32592709)(36.12413239,210.28593508)
\curveto(36.11412503,210.24592717)(36.11912503,210.20092721)(36.13913239,210.15093508)
\curveto(36.149125,210.10092731)(36.15412499,210.0159274)(36.15413239,209.89593508)
\curveto(36.15412499,209.78592763)(36.149125,209.70092771)(36.13913239,209.64093508)
\curveto(36.11912503,209.58092783)(36.10912504,209.52092789)(36.10913239,209.46093508)
\curveto(36.11912503,209.40092801)(36.11412503,209.34092807)(36.09413239,209.28093508)
\curveto(36.05412509,209.14092827)(36.01912513,209.00592841)(35.98913239,208.87593508)
\curveto(35.95912519,208.74592867)(35.91912523,208.62092879)(35.86913239,208.50093508)
\curveto(35.80912534,208.36092905)(35.73912541,208.23592918)(35.65913239,208.12593508)
\curveto(35.58912556,208.0159294)(35.51412563,207.90592951)(35.43413239,207.79593508)
\lineto(35.37413239,207.73593508)
\curveto(35.36412578,207.7159297)(35.3491258,207.69592972)(35.32913239,207.67593508)
\curveto(35.20912594,207.5159299)(35.07412607,207.37093004)(34.92413239,207.24093508)
\curveto(34.77412637,207.1109303)(34.61412653,206.98593043)(34.44413239,206.86593508)
\curveto(34.13412701,206.64593077)(33.83912731,206.44093097)(33.55913239,206.25093508)
\curveto(33.32912782,206.1109313)(33.09912805,205.97593144)(32.86913239,205.84593508)
\curveto(32.6491285,205.7159317)(32.42912872,205.58093183)(32.20913239,205.44093508)
\curveto(31.95912919,205.27093214)(31.71912943,205.09093232)(31.48913239,204.90093508)
\curveto(31.26912988,204.7109327)(31.07913007,204.48593293)(30.91913239,204.22593508)
\curveto(30.87913027,204.16593325)(30.8441303,204.10593331)(30.81413239,204.04593508)
\curveto(30.78413036,203.99593342)(30.75413039,203.93093348)(30.72413239,203.85093508)
\curveto(30.70413044,203.78093363)(30.69913045,203.72093369)(30.70913239,203.67093508)
\curveto(30.72913042,203.60093381)(30.76413038,203.54593387)(30.81413239,203.50593508)
\curveto(30.86413028,203.47593394)(30.92413022,203.45593396)(30.99413239,203.44593508)
\lineto(31.23413239,203.44593508)
\lineto(31.98413239,203.44593508)
\lineto(34.78913239,203.44593508)
\lineto(35.44913239,203.44593508)
\curveto(35.53912561,203.44593397)(35.62412552,203.44093397)(35.70413239,203.43093508)
\curveto(35.78412536,203.43093398)(35.8491253,203.410934)(35.89913239,203.37093508)
\curveto(35.9491252,203.33093408)(35.98912516,203.25593416)(36.01913239,203.14593508)
\curveto(36.05912509,203.04593437)(36.06912508,202.94593447)(36.04913239,202.84593508)
\lineto(36.04913239,202.71093508)
\curveto(36.02912512,202.64093477)(36.00912514,202.58093483)(35.98913239,202.53093508)
\curveto(35.96912518,202.48093493)(35.93412521,202.44093497)(35.88413239,202.41093508)
\curveto(35.83412531,202.37093504)(35.76412538,202.35093506)(35.67413239,202.35093508)
\lineto(35.40413239,202.35093508)
\lineto(34.50413239,202.35093508)
\lineto(30.99413239,202.35093508)
\lineto(29.92913239,202.35093508)
\curveto(29.8491313,202.35093506)(29.75913139,202.34593507)(29.65913239,202.33593508)
\curveto(29.55913159,202.33593508)(29.47413167,202.34593507)(29.40413239,202.36593508)
\curveto(29.19413195,202.43593498)(29.12913202,202.6159348)(29.20913239,202.90593508)
\curveto(29.21913193,202.94593447)(29.21913193,202.98093443)(29.20913239,203.01093508)
\curveto(29.20913194,203.05093436)(29.21913193,203.09593432)(29.23913239,203.14593508)
\curveto(29.25913189,203.22593419)(29.27913187,203.3109341)(29.29913239,203.40093508)
\curveto(29.31913183,203.49093392)(29.3441318,203.57593384)(29.37413239,203.65593508)
\curveto(29.53413161,204.14593327)(29.73413141,204.56093285)(29.97413239,204.90093508)
\curveto(30.15413099,205.15093226)(30.35913079,205.37593204)(30.58913239,205.57593508)
\curveto(30.81913033,205.78593163)(31.05913009,205.98093143)(31.30913239,206.16093508)
\curveto(31.56912958,206.34093107)(31.83412931,206.5109309)(32.10413239,206.67093508)
\curveto(32.38412876,206.84093057)(32.65412849,207.0159304)(32.91413239,207.19593508)
\curveto(33.02412812,207.27593014)(33.12912802,207.35093006)(33.22913239,207.42093508)
\curveto(33.33912781,207.49092992)(33.4491277,207.56592985)(33.55913239,207.64593508)
\curveto(33.59912755,207.67592974)(33.63412751,207.70592971)(33.66413239,207.73593508)
\curveto(33.70412744,207.77592964)(33.7441274,207.80592961)(33.78413239,207.82593508)
\curveto(33.92412722,207.93592948)(34.0491271,208.06092935)(34.15913239,208.20093508)
\curveto(34.17912697,208.23092918)(34.20412694,208.25592916)(34.23413239,208.27593508)
\curveto(34.26412688,208.30592911)(34.28912686,208.33592908)(34.30913239,208.36593508)
\curveto(34.38912676,208.46592895)(34.45412669,208.56592885)(34.50413239,208.66593508)
\curveto(34.56412658,208.76592865)(34.61912653,208.87592854)(34.66913239,208.99593508)
\curveto(34.69912645,209.06592835)(34.71912643,209.14092827)(34.72913239,209.22093508)
\lineto(34.78913239,209.46093508)
\lineto(34.78913239,209.55093508)
\curveto(34.79912635,209.58092783)(34.80412634,209.6109278)(34.80413239,209.64093508)
\curveto(34.82412632,209.7109277)(34.82912632,209.80592761)(34.81913239,209.92593508)
\curveto(34.81912633,210.05592736)(34.80912634,210.15592726)(34.78913239,210.22593508)
\curveto(34.76912638,210.30592711)(34.7491264,210.38092703)(34.72913239,210.45093508)
\curveto(34.71912643,210.53092688)(34.69912645,210.6109268)(34.66913239,210.69093508)
\curveto(34.55912659,210.93092648)(34.40912674,211.13092628)(34.21913239,211.29093508)
\curveto(34.03912711,211.46092595)(33.81912733,211.60092581)(33.55913239,211.71093508)
\curveto(33.48912766,211.73092568)(33.41912773,211.74592567)(33.34913239,211.75593508)
\curveto(33.27912787,211.77592564)(33.20412794,211.79592562)(33.12413239,211.81593508)
\curveto(33.0441281,211.83592558)(32.93412821,211.84592557)(32.79413239,211.84593508)
\curveto(32.66412848,211.84592557)(32.55912859,211.83592558)(32.47913239,211.81593508)
\curveto(32.41912873,211.80592561)(32.36412878,211.80092561)(32.31413239,211.80093508)
\curveto(32.26412888,211.80092561)(32.21412893,211.79092562)(32.16413239,211.77093508)
\curveto(32.06412908,211.73092568)(31.96912918,211.69092572)(31.87913239,211.65093508)
\curveto(31.79912935,211.6109258)(31.71912943,211.56592585)(31.63913239,211.51593508)
\curveto(31.60912954,211.49592592)(31.57912957,211.47092594)(31.54913239,211.44093508)
\curveto(31.52912962,211.410926)(31.50412964,211.38592603)(31.47413239,211.36593508)
\lineto(31.39913239,211.29093508)
\curveto(31.36912978,211.27092614)(31.3441298,211.25092616)(31.32413239,211.23093508)
\lineto(31.17413239,211.02093508)
\curveto(31.13413001,210.96092645)(31.08913006,210.89592652)(31.03913239,210.82593508)
\curveto(30.97913017,210.73592668)(30.92913022,210.63092678)(30.88913239,210.51093508)
\curveto(30.85913029,210.40092701)(30.82413032,210.29092712)(30.78413239,210.18093508)
\curveto(30.7441304,210.07092734)(30.71913043,209.92592749)(30.70913239,209.74593508)
\curveto(30.69913045,209.57592784)(30.66913048,209.45092796)(30.61913239,209.37093508)
\curveto(30.56913058,209.29092812)(30.49413065,209.24592817)(30.39413239,209.23593508)
\curveto(30.29413085,209.22592819)(30.18413096,209.22092819)(30.06413239,209.22093508)
\curveto(30.02413112,209.22092819)(29.98413116,209.2159282)(29.94413239,209.20593508)
\curveto(29.90413124,209.20592821)(29.86913128,209.2109282)(29.83913239,209.22093508)
\curveto(29.78913136,209.24092817)(29.73913141,209.25092816)(29.68913239,209.25093508)
\curveto(29.6491315,209.25092816)(29.60913154,209.26092815)(29.56913239,209.28093508)
\curveto(29.47913167,209.34092807)(29.43413171,209.47592794)(29.43413239,209.68593508)
\lineto(29.43413239,209.80593508)
\curveto(29.4441317,209.86592755)(29.4491317,209.92592749)(29.44913239,209.98593508)
\curveto(29.45913169,210.05592736)(29.46913168,210.12092729)(29.47913239,210.18093508)
\curveto(29.49913165,210.29092712)(29.51913163,210.39092702)(29.53913239,210.48093508)
\curveto(29.55913159,210.58092683)(29.58913156,210.67592674)(29.62913239,210.76593508)
\curveto(29.6491315,210.83592658)(29.66913148,210.89592652)(29.68913239,210.94593508)
\lineto(29.74913239,211.12593508)
\curveto(29.86913128,211.38592603)(30.02413112,211.63092578)(30.21413239,211.86093508)
\curveto(30.41413073,212.09092532)(30.62913052,212.27592514)(30.85913239,212.41593508)
\curveto(30.96913018,212.49592492)(31.08413006,212.56092485)(31.20413239,212.61093508)
\lineto(31.59413239,212.76093508)
\curveto(31.70412944,212.8109246)(31.81912933,212.84092457)(31.93913239,212.85093508)
\curveto(32.05912909,212.87092454)(32.18412896,212.89592452)(32.31413239,212.92593508)
\curveto(32.38412876,212.92592449)(32.4491287,212.92592449)(32.50913239,212.92593508)
\curveto(32.56912858,212.93592448)(32.63412851,212.94592447)(32.70413239,212.95593508)
}
}
{
\newrgbcolor{curcolor}{0 0 0}
\pscustom[linestyle=none,fillstyle=solid,fillcolor=curcolor]
{
\newpath
\moveto(44.80374176,207.43593508)
\lineto(44.80374176,207.18093508)
\curveto(44.81373406,207.10093031)(44.80873406,207.02593039)(44.78874176,206.95593508)
\lineto(44.78874176,206.71593508)
\lineto(44.78874176,206.55093508)
\curveto(44.7687341,206.45093096)(44.75873411,206.34593107)(44.75874176,206.23593508)
\curveto(44.75873411,206.13593128)(44.74873412,206.03593138)(44.72874176,205.93593508)
\lineto(44.72874176,205.78593508)
\curveto(44.69873417,205.64593177)(44.67873419,205.50593191)(44.66874176,205.36593508)
\curveto(44.65873421,205.23593218)(44.63373424,205.10593231)(44.59374176,204.97593508)
\curveto(44.5737343,204.89593252)(44.55373432,204.8109326)(44.53374176,204.72093508)
\lineto(44.47374176,204.48093508)
\lineto(44.35374176,204.18093508)
\curveto(44.32373455,204.09093332)(44.28873458,204.00093341)(44.24874176,203.91093508)
\curveto(44.14873472,203.69093372)(44.01373486,203.47593394)(43.84374176,203.26593508)
\curveto(43.68373519,203.05593436)(43.50873536,202.88593453)(43.31874176,202.75593508)
\curveto(43.2687356,202.7159347)(43.20873566,202.67593474)(43.13874176,202.63593508)
\curveto(43.07873579,202.60593481)(43.01873585,202.57093484)(42.95874176,202.53093508)
\curveto(42.87873599,202.48093493)(42.78373609,202.44093497)(42.67374176,202.41093508)
\curveto(42.56373631,202.38093503)(42.45873641,202.35093506)(42.35874176,202.32093508)
\curveto(42.24873662,202.28093513)(42.13873673,202.25593516)(42.02874176,202.24593508)
\curveto(41.91873695,202.23593518)(41.80373707,202.22093519)(41.68374176,202.20093508)
\curveto(41.64373723,202.19093522)(41.59873727,202.19093522)(41.54874176,202.20093508)
\curveto(41.50873736,202.20093521)(41.4687374,202.19593522)(41.42874176,202.18593508)
\curveto(41.38873748,202.17593524)(41.33373754,202.17093524)(41.26374176,202.17093508)
\curveto(41.19373768,202.17093524)(41.14373773,202.17593524)(41.11374176,202.18593508)
\curveto(41.06373781,202.20593521)(41.01873785,202.2109352)(40.97874176,202.20093508)
\curveto(40.93873793,202.19093522)(40.90373797,202.19093522)(40.87374176,202.20093508)
\lineto(40.78374176,202.20093508)
\curveto(40.72373815,202.22093519)(40.65873821,202.23593518)(40.58874176,202.24593508)
\curveto(40.52873834,202.24593517)(40.46373841,202.25093516)(40.39374176,202.26093508)
\curveto(40.22373865,202.3109351)(40.06373881,202.36093505)(39.91374176,202.41093508)
\curveto(39.76373911,202.46093495)(39.61873925,202.52593489)(39.47874176,202.60593508)
\curveto(39.42873944,202.64593477)(39.3737395,202.67593474)(39.31374176,202.69593508)
\curveto(39.26373961,202.72593469)(39.21373966,202.76093465)(39.16374176,202.80093508)
\curveto(38.92373995,202.98093443)(38.72374015,203.20093421)(38.56374176,203.46093508)
\curveto(38.40374047,203.72093369)(38.26374061,204.00593341)(38.14374176,204.31593508)
\curveto(38.08374079,204.45593296)(38.03874083,204.59593282)(38.00874176,204.73593508)
\curveto(37.97874089,204.88593253)(37.94374093,205.04093237)(37.90374176,205.20093508)
\curveto(37.88374099,205.3109321)(37.868741,205.42093199)(37.85874176,205.53093508)
\curveto(37.84874102,205.64093177)(37.83374104,205.75093166)(37.81374176,205.86093508)
\curveto(37.80374107,205.90093151)(37.79874107,205.94093147)(37.79874176,205.98093508)
\curveto(37.80874106,206.02093139)(37.80874106,206.06093135)(37.79874176,206.10093508)
\curveto(37.78874108,206.15093126)(37.78374109,206.20093121)(37.78374176,206.25093508)
\lineto(37.78374176,206.41593508)
\curveto(37.76374111,206.46593095)(37.75874111,206.5159309)(37.76874176,206.56593508)
\curveto(37.77874109,206.62593079)(37.77874109,206.68093073)(37.76874176,206.73093508)
\curveto(37.75874111,206.77093064)(37.75874111,206.8159306)(37.76874176,206.86593508)
\curveto(37.77874109,206.9159305)(37.7737411,206.96593045)(37.75374176,207.01593508)
\curveto(37.73374114,207.08593033)(37.72874114,207.16093025)(37.73874176,207.24093508)
\curveto(37.74874112,207.33093008)(37.75374112,207.41593)(37.75374176,207.49593508)
\curveto(37.75374112,207.58592983)(37.74874112,207.68592973)(37.73874176,207.79593508)
\curveto(37.72874114,207.9159295)(37.73374114,208.0159294)(37.75374176,208.09593508)
\lineto(37.75374176,208.38093508)
\lineto(37.79874176,209.01093508)
\curveto(37.80874106,209.1109283)(37.81874105,209.20592821)(37.82874176,209.29593508)
\lineto(37.85874176,209.59593508)
\curveto(37.87874099,209.64592777)(37.88374099,209.69592772)(37.87374176,209.74593508)
\curveto(37.873741,209.80592761)(37.88374099,209.86092755)(37.90374176,209.91093508)
\curveto(37.95374092,210.08092733)(37.99374088,210.24592717)(38.02374176,210.40593508)
\curveto(38.05374082,210.57592684)(38.10374077,210.73592668)(38.17374176,210.88593508)
\curveto(38.36374051,211.34592607)(38.58374029,211.72092569)(38.83374176,212.01093508)
\curveto(39.09373978,212.30092511)(39.45373942,212.54592487)(39.91374176,212.74593508)
\curveto(40.04373883,212.79592462)(40.1737387,212.83092458)(40.30374176,212.85093508)
\curveto(40.44373843,212.87092454)(40.58373829,212.89592452)(40.72374176,212.92593508)
\curveto(40.79373808,212.93592448)(40.85873801,212.94092447)(40.91874176,212.94093508)
\curveto(40.97873789,212.94092447)(41.04373783,212.94592447)(41.11374176,212.95593508)
\curveto(41.94373693,212.97592444)(42.61373626,212.82592459)(43.12374176,212.50593508)
\curveto(43.63373524,212.19592522)(44.01373486,211.75592566)(44.26374176,211.18593508)
\curveto(44.31373456,211.06592635)(44.35873451,210.94092647)(44.39874176,210.81093508)
\curveto(44.43873443,210.68092673)(44.48373439,210.54592687)(44.53374176,210.40593508)
\curveto(44.55373432,210.32592709)(44.5687343,210.24092717)(44.57874176,210.15093508)
\lineto(44.63874176,209.91093508)
\curveto(44.6687342,209.80092761)(44.68373419,209.69092772)(44.68374176,209.58093508)
\curveto(44.69373418,209.47092794)(44.70873416,209.36092805)(44.72874176,209.25093508)
\curveto(44.74873412,209.20092821)(44.75373412,209.15592826)(44.74374176,209.11593508)
\curveto(44.74373413,209.07592834)(44.74873412,209.03592838)(44.75874176,208.99593508)
\curveto(44.7687341,208.94592847)(44.7687341,208.89092852)(44.75874176,208.83093508)
\curveto(44.75873411,208.78092863)(44.76373411,208.73092868)(44.77374176,208.68093508)
\lineto(44.77374176,208.54593508)
\curveto(44.79373408,208.48592893)(44.79373408,208.415929)(44.77374176,208.33593508)
\curveto(44.76373411,208.26592915)(44.7687341,208.20092921)(44.78874176,208.14093508)
\curveto(44.79873407,208.1109293)(44.80373407,208.07092934)(44.80374176,208.02093508)
\lineto(44.80374176,207.90093508)
\lineto(44.80374176,207.43593508)
\moveto(43.25874176,205.11093508)
\curveto(43.35873551,205.43093198)(43.41873545,205.79593162)(43.43874176,206.20593508)
\curveto(43.45873541,206.6159308)(43.4687354,207.02593039)(43.46874176,207.43593508)
\curveto(43.4687354,207.86592955)(43.45873541,208.28592913)(43.43874176,208.69593508)
\curveto(43.41873545,209.10592831)(43.3737355,209.49092792)(43.30374176,209.85093508)
\curveto(43.23373564,210.2109272)(43.12373575,210.53092688)(42.97374176,210.81093508)
\curveto(42.83373604,211.10092631)(42.63873623,211.33592608)(42.38874176,211.51593508)
\curveto(42.22873664,211.62592579)(42.04873682,211.70592571)(41.84874176,211.75593508)
\curveto(41.64873722,211.8159256)(41.40373747,211.84592557)(41.11374176,211.84593508)
\curveto(41.09373778,211.82592559)(41.05873781,211.8159256)(41.00874176,211.81593508)
\curveto(40.95873791,211.82592559)(40.91873795,211.82592559)(40.88874176,211.81593508)
\curveto(40.80873806,211.79592562)(40.73373814,211.77592564)(40.66374176,211.75593508)
\curveto(40.60373827,211.74592567)(40.53873833,211.72592569)(40.46874176,211.69593508)
\curveto(40.19873867,211.57592584)(39.97873889,211.40592601)(39.80874176,211.18593508)
\curveto(39.64873922,210.97592644)(39.51373936,210.73092668)(39.40374176,210.45093508)
\curveto(39.35373952,210.34092707)(39.31373956,210.22092719)(39.28374176,210.09093508)
\curveto(39.26373961,209.97092744)(39.23873963,209.84592757)(39.20874176,209.71593508)
\curveto(39.18873968,209.66592775)(39.17873969,209.6109278)(39.17874176,209.55093508)
\curveto(39.17873969,209.50092791)(39.1737397,209.45092796)(39.16374176,209.40093508)
\curveto(39.15373972,209.3109281)(39.14373973,209.2159282)(39.13374176,209.11593508)
\curveto(39.12373975,209.02592839)(39.11373976,208.93092848)(39.10374176,208.83093508)
\curveto(39.10373977,208.75092866)(39.09873977,208.66592875)(39.08874176,208.57593508)
\lineto(39.08874176,208.33593508)
\lineto(39.08874176,208.15593508)
\curveto(39.07873979,208.12592929)(39.0737398,208.09092932)(39.07374176,208.05093508)
\lineto(39.07374176,207.91593508)
\lineto(39.07374176,207.46593508)
\curveto(39.0737398,207.38593003)(39.0687398,207.30093011)(39.05874176,207.21093508)
\curveto(39.05873981,207.13093028)(39.0687398,207.05593036)(39.08874176,206.98593508)
\lineto(39.08874176,206.71593508)
\curveto(39.08873978,206.69593072)(39.08373979,206.66593075)(39.07374176,206.62593508)
\curveto(39.0737398,206.59593082)(39.07873979,206.57093084)(39.08874176,206.55093508)
\curveto(39.09873977,206.45093096)(39.10373977,206.35093106)(39.10374176,206.25093508)
\curveto(39.11373976,206.16093125)(39.12373975,206.06093135)(39.13374176,205.95093508)
\curveto(39.16373971,205.83093158)(39.17873969,205.70593171)(39.17874176,205.57593508)
\curveto(39.18873968,205.45593196)(39.21373966,205.34093207)(39.25374176,205.23093508)
\curveto(39.33373954,204.93093248)(39.41873945,204.66593275)(39.50874176,204.43593508)
\curveto(39.60873926,204.20593321)(39.75373912,203.99093342)(39.94374176,203.79093508)
\curveto(40.15373872,203.59093382)(40.41873845,203.44093397)(40.73874176,203.34093508)
\curveto(40.77873809,203.32093409)(40.81373806,203.3109341)(40.84374176,203.31093508)
\curveto(40.88373799,203.32093409)(40.92873794,203.3159341)(40.97874176,203.29593508)
\curveto(41.01873785,203.28593413)(41.08873778,203.27593414)(41.18874176,203.26593508)
\curveto(41.29873757,203.25593416)(41.38373749,203.26093415)(41.44374176,203.28093508)
\curveto(41.51373736,203.30093411)(41.58373729,203.3109341)(41.65374176,203.31093508)
\curveto(41.72373715,203.32093409)(41.78873708,203.33593408)(41.84874176,203.35593508)
\curveto(42.04873682,203.415934)(42.22873664,203.50093391)(42.38874176,203.61093508)
\curveto(42.41873645,203.63093378)(42.44373643,203.65093376)(42.46374176,203.67093508)
\lineto(42.52374176,203.73093508)
\curveto(42.56373631,203.75093366)(42.61373626,203.79093362)(42.67374176,203.85093508)
\curveto(42.7737361,203.99093342)(42.85873601,204.12093329)(42.92874176,204.24093508)
\curveto(42.99873587,204.36093305)(43.0687358,204.50593291)(43.13874176,204.67593508)
\curveto(43.1687357,204.74593267)(43.18873568,204.8159326)(43.19874176,204.88593508)
\curveto(43.21873565,204.95593246)(43.23873563,205.03093238)(43.25874176,205.11093508)
}
}
{
\newrgbcolor{curcolor}{0 0 0}
\pscustom[linestyle=none,fillstyle=solid,fillcolor=curcolor]
{
\newpath
\moveto(44.80374176,127.24385744)
\lineto(44.80374176,126.98885744)
\curveto(44.81373406,126.90885268)(44.80873406,126.83385275)(44.78874176,126.76385744)
\lineto(44.78874176,126.52385744)
\lineto(44.78874176,126.35885744)
\curveto(44.7687341,126.25885333)(44.75873411,126.15385343)(44.75874176,126.04385744)
\curveto(44.75873411,125.94385364)(44.74873412,125.84385374)(44.72874176,125.74385744)
\lineto(44.72874176,125.59385744)
\curveto(44.69873417,125.45385413)(44.67873419,125.31385427)(44.66874176,125.17385744)
\curveto(44.65873421,125.04385454)(44.63373424,124.91385467)(44.59374176,124.78385744)
\curveto(44.5737343,124.70385488)(44.55373432,124.61885497)(44.53374176,124.52885744)
\lineto(44.47374176,124.28885744)
\lineto(44.35374176,123.98885744)
\curveto(44.32373455,123.89885569)(44.28873458,123.80885578)(44.24874176,123.71885744)
\curveto(44.14873472,123.49885609)(44.01373486,123.2838563)(43.84374176,123.07385744)
\curveto(43.68373519,122.86385672)(43.50873536,122.69385689)(43.31874176,122.56385744)
\curveto(43.2687356,122.52385706)(43.20873566,122.4838571)(43.13874176,122.44385744)
\curveto(43.07873579,122.41385717)(43.01873585,122.37885721)(42.95874176,122.33885744)
\curveto(42.87873599,122.2888573)(42.78373609,122.24885734)(42.67374176,122.21885744)
\curveto(42.56373631,122.1888574)(42.45873641,122.15885743)(42.35874176,122.12885744)
\curveto(42.24873662,122.0888575)(42.13873673,122.06385752)(42.02874176,122.05385744)
\curveto(41.91873695,122.04385754)(41.80373707,122.02885756)(41.68374176,122.00885744)
\curveto(41.64373723,121.99885759)(41.59873727,121.99885759)(41.54874176,122.00885744)
\curveto(41.50873736,122.00885758)(41.4687374,122.00385758)(41.42874176,121.99385744)
\curveto(41.38873748,121.9838576)(41.33373754,121.97885761)(41.26374176,121.97885744)
\curveto(41.19373768,121.97885761)(41.14373773,121.9838576)(41.11374176,121.99385744)
\curveto(41.06373781,122.01385757)(41.01873785,122.01885757)(40.97874176,122.00885744)
\curveto(40.93873793,121.99885759)(40.90373797,121.99885759)(40.87374176,122.00885744)
\lineto(40.78374176,122.00885744)
\curveto(40.72373815,122.02885756)(40.65873821,122.04385754)(40.58874176,122.05385744)
\curveto(40.52873834,122.05385753)(40.46373841,122.05885753)(40.39374176,122.06885744)
\curveto(40.22373865,122.11885747)(40.06373881,122.16885742)(39.91374176,122.21885744)
\curveto(39.76373911,122.26885732)(39.61873925,122.33385725)(39.47874176,122.41385744)
\curveto(39.42873944,122.45385713)(39.3737395,122.4838571)(39.31374176,122.50385744)
\curveto(39.26373961,122.53385705)(39.21373966,122.56885702)(39.16374176,122.60885744)
\curveto(38.92373995,122.7888568)(38.72374015,123.00885658)(38.56374176,123.26885744)
\curveto(38.40374047,123.52885606)(38.26374061,123.81385577)(38.14374176,124.12385744)
\curveto(38.08374079,124.26385532)(38.03874083,124.40385518)(38.00874176,124.54385744)
\curveto(37.97874089,124.69385489)(37.94374093,124.84885474)(37.90374176,125.00885744)
\curveto(37.88374099,125.11885447)(37.868741,125.22885436)(37.85874176,125.33885744)
\curveto(37.84874102,125.44885414)(37.83374104,125.55885403)(37.81374176,125.66885744)
\curveto(37.80374107,125.70885388)(37.79874107,125.74885384)(37.79874176,125.78885744)
\curveto(37.80874106,125.82885376)(37.80874106,125.86885372)(37.79874176,125.90885744)
\curveto(37.78874108,125.95885363)(37.78374109,126.00885358)(37.78374176,126.05885744)
\lineto(37.78374176,126.22385744)
\curveto(37.76374111,126.27385331)(37.75874111,126.32385326)(37.76874176,126.37385744)
\curveto(37.77874109,126.43385315)(37.77874109,126.4888531)(37.76874176,126.53885744)
\curveto(37.75874111,126.57885301)(37.75874111,126.62385296)(37.76874176,126.67385744)
\curveto(37.77874109,126.72385286)(37.7737411,126.77385281)(37.75374176,126.82385744)
\curveto(37.73374114,126.89385269)(37.72874114,126.96885262)(37.73874176,127.04885744)
\curveto(37.74874112,127.13885245)(37.75374112,127.22385236)(37.75374176,127.30385744)
\curveto(37.75374112,127.39385219)(37.74874112,127.49385209)(37.73874176,127.60385744)
\curveto(37.72874114,127.72385186)(37.73374114,127.82385176)(37.75374176,127.90385744)
\lineto(37.75374176,128.18885744)
\lineto(37.79874176,128.81885744)
\curveto(37.80874106,128.91885067)(37.81874105,129.01385057)(37.82874176,129.10385744)
\lineto(37.85874176,129.40385744)
\curveto(37.87874099,129.45385013)(37.88374099,129.50385008)(37.87374176,129.55385744)
\curveto(37.873741,129.61384997)(37.88374099,129.66884992)(37.90374176,129.71885744)
\curveto(37.95374092,129.8888497)(37.99374088,130.05384953)(38.02374176,130.21385744)
\curveto(38.05374082,130.3838492)(38.10374077,130.54384904)(38.17374176,130.69385744)
\curveto(38.36374051,131.15384843)(38.58374029,131.52884806)(38.83374176,131.81885744)
\curveto(39.09373978,132.10884748)(39.45373942,132.35384723)(39.91374176,132.55385744)
\curveto(40.04373883,132.60384698)(40.1737387,132.63884695)(40.30374176,132.65885744)
\curveto(40.44373843,132.67884691)(40.58373829,132.70384688)(40.72374176,132.73385744)
\curveto(40.79373808,132.74384684)(40.85873801,132.74884684)(40.91874176,132.74885744)
\curveto(40.97873789,132.74884684)(41.04373783,132.75384683)(41.11374176,132.76385744)
\curveto(41.94373693,132.7838468)(42.61373626,132.63384695)(43.12374176,132.31385744)
\curveto(43.63373524,132.00384758)(44.01373486,131.56384802)(44.26374176,130.99385744)
\curveto(44.31373456,130.87384871)(44.35873451,130.74884884)(44.39874176,130.61885744)
\curveto(44.43873443,130.4888491)(44.48373439,130.35384923)(44.53374176,130.21385744)
\curveto(44.55373432,130.13384945)(44.5687343,130.04884954)(44.57874176,129.95885744)
\lineto(44.63874176,129.71885744)
\curveto(44.6687342,129.60884998)(44.68373419,129.49885009)(44.68374176,129.38885744)
\curveto(44.69373418,129.27885031)(44.70873416,129.16885042)(44.72874176,129.05885744)
\curveto(44.74873412,129.00885058)(44.75373412,128.96385062)(44.74374176,128.92385744)
\curveto(44.74373413,128.8838507)(44.74873412,128.84385074)(44.75874176,128.80385744)
\curveto(44.7687341,128.75385083)(44.7687341,128.69885089)(44.75874176,128.63885744)
\curveto(44.75873411,128.588851)(44.76373411,128.53885105)(44.77374176,128.48885744)
\lineto(44.77374176,128.35385744)
\curveto(44.79373408,128.29385129)(44.79373408,128.22385136)(44.77374176,128.14385744)
\curveto(44.76373411,128.07385151)(44.7687341,128.00885158)(44.78874176,127.94885744)
\curveto(44.79873407,127.91885167)(44.80373407,127.87885171)(44.80374176,127.82885744)
\lineto(44.80374176,127.70885744)
\lineto(44.80374176,127.24385744)
\moveto(43.25874176,124.91885744)
\curveto(43.35873551,125.23885435)(43.41873545,125.60385398)(43.43874176,126.01385744)
\curveto(43.45873541,126.42385316)(43.4687354,126.83385275)(43.46874176,127.24385744)
\curveto(43.4687354,127.67385191)(43.45873541,128.09385149)(43.43874176,128.50385744)
\curveto(43.41873545,128.91385067)(43.3737355,129.29885029)(43.30374176,129.65885744)
\curveto(43.23373564,130.01884957)(43.12373575,130.33884925)(42.97374176,130.61885744)
\curveto(42.83373604,130.90884868)(42.63873623,131.14384844)(42.38874176,131.32385744)
\curveto(42.22873664,131.43384815)(42.04873682,131.51384807)(41.84874176,131.56385744)
\curveto(41.64873722,131.62384796)(41.40373747,131.65384793)(41.11374176,131.65385744)
\curveto(41.09373778,131.63384795)(41.05873781,131.62384796)(41.00874176,131.62385744)
\curveto(40.95873791,131.63384795)(40.91873795,131.63384795)(40.88874176,131.62385744)
\curveto(40.80873806,131.60384798)(40.73373814,131.583848)(40.66374176,131.56385744)
\curveto(40.60373827,131.55384803)(40.53873833,131.53384805)(40.46874176,131.50385744)
\curveto(40.19873867,131.3838482)(39.97873889,131.21384837)(39.80874176,130.99385744)
\curveto(39.64873922,130.7838488)(39.51373936,130.53884905)(39.40374176,130.25885744)
\curveto(39.35373952,130.14884944)(39.31373956,130.02884956)(39.28374176,129.89885744)
\curveto(39.26373961,129.77884981)(39.23873963,129.65384993)(39.20874176,129.52385744)
\curveto(39.18873968,129.47385011)(39.17873969,129.41885017)(39.17874176,129.35885744)
\curveto(39.17873969,129.30885028)(39.1737397,129.25885033)(39.16374176,129.20885744)
\curveto(39.15373972,129.11885047)(39.14373973,129.02385056)(39.13374176,128.92385744)
\curveto(39.12373975,128.83385075)(39.11373976,128.73885085)(39.10374176,128.63885744)
\curveto(39.10373977,128.55885103)(39.09873977,128.47385111)(39.08874176,128.38385744)
\lineto(39.08874176,128.14385744)
\lineto(39.08874176,127.96385744)
\curveto(39.07873979,127.93385165)(39.0737398,127.89885169)(39.07374176,127.85885744)
\lineto(39.07374176,127.72385744)
\lineto(39.07374176,127.27385744)
\curveto(39.0737398,127.19385239)(39.0687398,127.10885248)(39.05874176,127.01885744)
\curveto(39.05873981,126.93885265)(39.0687398,126.86385272)(39.08874176,126.79385744)
\lineto(39.08874176,126.52385744)
\curveto(39.08873978,126.50385308)(39.08373979,126.47385311)(39.07374176,126.43385744)
\curveto(39.0737398,126.40385318)(39.07873979,126.37885321)(39.08874176,126.35885744)
\curveto(39.09873977,126.25885333)(39.10373977,126.15885343)(39.10374176,126.05885744)
\curveto(39.11373976,125.96885362)(39.12373975,125.86885372)(39.13374176,125.75885744)
\curveto(39.16373971,125.63885395)(39.17873969,125.51385407)(39.17874176,125.38385744)
\curveto(39.18873968,125.26385432)(39.21373966,125.14885444)(39.25374176,125.03885744)
\curveto(39.33373954,124.73885485)(39.41873945,124.47385511)(39.50874176,124.24385744)
\curveto(39.60873926,124.01385557)(39.75373912,123.79885579)(39.94374176,123.59885744)
\curveto(40.15373872,123.39885619)(40.41873845,123.24885634)(40.73874176,123.14885744)
\curveto(40.77873809,123.12885646)(40.81373806,123.11885647)(40.84374176,123.11885744)
\curveto(40.88373799,123.12885646)(40.92873794,123.12385646)(40.97874176,123.10385744)
\curveto(41.01873785,123.09385649)(41.08873778,123.0838565)(41.18874176,123.07385744)
\curveto(41.29873757,123.06385652)(41.38373749,123.06885652)(41.44374176,123.08885744)
\curveto(41.51373736,123.10885648)(41.58373729,123.11885647)(41.65374176,123.11885744)
\curveto(41.72373715,123.12885646)(41.78873708,123.14385644)(41.84874176,123.16385744)
\curveto(42.04873682,123.22385636)(42.22873664,123.30885628)(42.38874176,123.41885744)
\curveto(42.41873645,123.43885615)(42.44373643,123.45885613)(42.46374176,123.47885744)
\lineto(42.52374176,123.53885744)
\curveto(42.56373631,123.55885603)(42.61373626,123.59885599)(42.67374176,123.65885744)
\curveto(42.7737361,123.79885579)(42.85873601,123.92885566)(42.92874176,124.04885744)
\curveto(42.99873587,124.16885542)(43.0687358,124.31385527)(43.13874176,124.48385744)
\curveto(43.1687357,124.55385503)(43.18873568,124.62385496)(43.19874176,124.69385744)
\curveto(43.21873565,124.76385482)(43.23873563,124.83885475)(43.25874176,124.91885744)
}
}
{
\newrgbcolor{curcolor}{0 0 0}
\pscustom[linewidth=1,linecolor=curcolor]
{
\newpath
\moveto(67.142857,287.5)
\lineto(1024.2857,287.5)
}
}
{
\newrgbcolor{curcolor}{0 0 0}
\pscustom[linewidth=1,linecolor=curcolor]
{
\newpath
\moveto(67.142857,207.54464)
\lineto(1024.2857,207.54464)
}
}
{
\newrgbcolor{curcolor}{0 0 0}
\pscustom[linewidth=1,linecolor=curcolor]
{
\newpath
\moveto(67.142857,127.5)
\lineto(1024.2857,127.5)
}
}
{
\newrgbcolor{curcolor}{0 0 0}
\pscustom[linewidth=1,linecolor=curcolor]
{
\newpath
\moveto(67.142857,367.5)
\lineto(1024.2857,367.5)
}
}
{
\newrgbcolor{curcolor}{1 1 1}
\pscustom[linestyle=none,fillstyle=solid,fillcolor=curcolor]
{
\newpath
\moveto(107.83379364,399.88729616)
\lineto(221.22342682,399.88729616)
\lineto(221.22342682,314.40996309)
\lineto(107.83379364,314.40996309)
\closepath
}
}
{
\newrgbcolor{curcolor}{0 0 0}
\pscustom[linewidth=0.89664584,linecolor=curcolor]
{
\newpath
\moveto(107.83379364,399.88729616)
\lineto(221.22342682,399.88729616)
\lineto(221.22342682,314.40996309)
\lineto(107.83379364,314.40996309)
\closepath
}
}
{
\newrgbcolor{curcolor}{0 0 0}
\pscustom[linewidth=1,linecolor=curcolor]
{
\newpath
\moveto(67.142857,447.45536)
\lineto(1024.2857,447.45536)
}
}
{
\newrgbcolor{curcolor}{0 0 0}
\pscustom[linestyle=none,fillstyle=solid,fillcolor=curcolor]
{
\newpath
\moveto(144.77876512,384.77531497)
\curveto(145.75875862,384.79530402)(146.5787578,384.63530418)(147.23876512,384.29531497)
\curveto(147.90875647,383.96530485)(148.42875595,383.50530531)(148.79876512,382.91531497)
\curveto(148.89875548,382.75530606)(148.9787554,382.60030621)(149.03876512,382.45031497)
\curveto(149.10875527,382.3103065)(149.1737552,382.14030667)(149.23376512,381.94031497)
\curveto(149.25375512,381.89030692)(149.2737551,381.82030699)(149.29376512,381.73031497)
\curveto(149.31375506,381.65030716)(149.30875507,381.57530724)(149.27876512,381.50531497)
\curveto(149.25875512,381.44530737)(149.21875516,381.40530741)(149.15876512,381.38531497)
\curveto(149.10875527,381.37530744)(149.05375532,381.36030745)(148.99376512,381.34031497)
\lineto(148.84376512,381.34031497)
\curveto(148.81375556,381.33030748)(148.7737556,381.32530749)(148.72376512,381.32531497)
\lineto(148.60376512,381.32531497)
\curveto(148.46375591,381.32530749)(148.33375604,381.33030748)(148.21376512,381.34031497)
\curveto(148.10375627,381.36030745)(148.02375635,381.4103074)(147.97376512,381.49031497)
\curveto(147.90375647,381.59030722)(147.84875653,381.70530711)(147.80876512,381.83531497)
\curveto(147.76875661,381.96530685)(147.71375666,382.08530673)(147.64376512,382.19531497)
\curveto(147.51375686,382.4153064)(147.36375701,382.60530621)(147.19376512,382.76531497)
\curveto(147.03375734,382.92530589)(146.84375753,383.07530574)(146.62376512,383.21531497)
\curveto(146.50375787,383.29530552)(146.36875801,383.35530546)(146.21876512,383.39531497)
\curveto(146.0787583,383.43530538)(145.93375844,383.47530534)(145.78376512,383.51531497)
\curveto(145.6737587,383.54530527)(145.54875883,383.56530525)(145.40876512,383.57531497)
\curveto(145.26875911,383.59530522)(145.11875926,383.60530521)(144.95876512,383.60531497)
\curveto(144.80875957,383.60530521)(144.65875972,383.59530522)(144.50876512,383.57531497)
\curveto(144.36876001,383.56530525)(144.24876013,383.54530527)(144.14876512,383.51531497)
\curveto(144.04876033,383.49530532)(143.95376042,383.47530534)(143.86376512,383.45531497)
\curveto(143.7737606,383.43530538)(143.68376069,383.40530541)(143.59376512,383.36531497)
\curveto(142.75376162,383.0153058)(142.14876223,382.4153064)(141.77876512,381.56531497)
\curveto(141.70876267,381.42530739)(141.64876273,381.27530754)(141.59876512,381.11531497)
\curveto(141.55876282,380.96530785)(141.51376286,380.810308)(141.46376512,380.65031497)
\curveto(141.44376293,380.59030822)(141.43376294,380.52530829)(141.43376512,380.45531497)
\curveto(141.43376294,380.39530842)(141.42376295,380.33530848)(141.40376512,380.27531497)
\curveto(141.39376298,380.23530858)(141.38876299,380.20030861)(141.38876512,380.17031497)
\curveto(141.38876299,380.14030867)(141.38376299,380.10530871)(141.37376512,380.06531497)
\curveto(141.35376302,379.95530886)(141.33876304,379.84030897)(141.32876512,379.72031497)
\lineto(141.32876512,379.37531497)
\curveto(141.32876305,379.30530951)(141.32376305,379.23030958)(141.31376512,379.15031497)
\curveto(141.31376306,379.08030973)(141.31876306,379.0153098)(141.32876512,378.95531497)
\lineto(141.32876512,378.80531497)
\curveto(141.34876303,378.73531008)(141.35376302,378.66531015)(141.34376512,378.59531497)
\curveto(141.34376303,378.52531029)(141.35376302,378.45531036)(141.37376512,378.38531497)
\curveto(141.39376298,378.32531049)(141.39876298,378.26531055)(141.38876512,378.20531497)
\curveto(141.38876299,378.14531067)(141.39876298,378.09031072)(141.41876512,378.04031497)
\curveto(141.44876293,377.9103109)(141.4737629,377.78031103)(141.49376512,377.65031497)
\curveto(141.52376285,377.53031128)(141.55876282,377.4103114)(141.59876512,377.29031497)
\curveto(141.76876261,376.79031202)(141.98876239,376.36031245)(142.25876512,376.00031497)
\curveto(142.52876185,375.65031316)(142.88376149,375.36031345)(143.32376512,375.13031497)
\curveto(143.46376091,375.06031375)(143.60376077,375.00531381)(143.74376512,374.96531497)
\curveto(143.89376048,374.92531389)(144.05376032,374.88031393)(144.22376512,374.83031497)
\curveto(144.29376008,374.810314)(144.35876002,374.80031401)(144.41876512,374.80031497)
\curveto(144.4787599,374.810314)(144.54875983,374.80531401)(144.62876512,374.78531497)
\curveto(144.6787597,374.77531404)(144.76875961,374.76531405)(144.89876512,374.75531497)
\curveto(145.02875935,374.75531406)(145.12375925,374.76531405)(145.18376512,374.78531497)
\lineto(145.28876512,374.78531497)
\curveto(145.32875905,374.79531402)(145.36875901,374.79531402)(145.40876512,374.78531497)
\curveto(145.44875893,374.78531403)(145.48875889,374.79531402)(145.52876512,374.81531497)
\curveto(145.62875875,374.83531398)(145.72375865,374.85031396)(145.81376512,374.86031497)
\curveto(145.91375846,374.88031393)(146.00875837,374.9103139)(146.09876512,374.95031497)
\curveto(146.8787575,375.27031354)(147.42875695,375.79531302)(147.74876512,376.52531497)
\curveto(147.82875655,376.70531211)(147.90375647,376.92031189)(147.97376512,377.17031497)
\curveto(147.99375638,377.26031155)(148.00875637,377.35031146)(148.01876512,377.44031497)
\curveto(148.03875634,377.54031127)(148.0737563,377.63031118)(148.12376512,377.71031497)
\curveto(148.1737562,377.79031102)(148.25375612,377.83531098)(148.36376512,377.84531497)
\curveto(148.4737559,377.85531096)(148.59375578,377.86031095)(148.72376512,377.86031497)
\lineto(148.87376512,377.86031497)
\curveto(148.92375545,377.86031095)(148.96875541,377.85531096)(149.00876512,377.84531497)
\lineto(149.11376512,377.84531497)
\lineto(149.20376512,377.81531497)
\curveto(149.24375513,377.815311)(149.2737551,377.80531101)(149.29376512,377.78531497)
\curveto(149.36375501,377.74531107)(149.40375497,377.67031114)(149.41376512,377.56031497)
\curveto(149.42375495,377.46031135)(149.41375496,377.36031145)(149.38376512,377.26031497)
\curveto(149.32375505,377.03031178)(149.26875511,376.810312)(149.21876512,376.60031497)
\curveto(149.16875521,376.39031242)(149.09375528,376.19031262)(148.99376512,376.00031497)
\curveto(148.91375546,375.87031294)(148.83875554,375.74531307)(148.76876512,375.62531497)
\curveto(148.70875567,375.50531331)(148.63875574,375.38531343)(148.55876512,375.26531497)
\curveto(148.378756,375.00531381)(148.15375622,374.76531405)(147.88376512,374.54531497)
\curveto(147.62375675,374.33531448)(147.33875704,374.16031465)(147.02876512,374.02031497)
\curveto(146.91875746,373.97031484)(146.80875757,373.93031488)(146.69876512,373.90031497)
\curveto(146.59875778,373.87031494)(146.49375788,373.83531498)(146.38376512,373.79531497)
\curveto(146.2737581,373.75531506)(146.15875822,373.73031508)(146.03876512,373.72031497)
\curveto(145.92875845,373.70031511)(145.81375856,373.68031513)(145.69376512,373.66031497)
\curveto(145.64375873,373.64031517)(145.59875878,373.63531518)(145.55876512,373.64531497)
\curveto(145.51875886,373.64531517)(145.4787589,373.64031517)(145.43876512,373.63031497)
\curveto(145.378759,373.62031519)(145.31875906,373.6153152)(145.25876512,373.61531497)
\curveto(145.19875918,373.6153152)(145.13375924,373.6103152)(145.06376512,373.60031497)
\curveto(145.03375934,373.59031522)(144.96375941,373.59031522)(144.85376512,373.60031497)
\curveto(144.75375962,373.60031521)(144.68875969,373.60531521)(144.65876512,373.61531497)
\curveto(144.60875977,373.62531519)(144.55875982,373.63031518)(144.50876512,373.63031497)
\curveto(144.46875991,373.62031519)(144.42375995,373.62031519)(144.37376512,373.63031497)
\lineto(144.22376512,373.63031497)
\curveto(144.14376023,373.65031516)(144.06876031,373.66531515)(143.99876512,373.67531497)
\curveto(143.92876045,373.67531514)(143.85376052,373.68531513)(143.77376512,373.70531497)
\lineto(143.50376512,373.76531497)
\curveto(143.41376096,373.77531504)(143.32876105,373.79531502)(143.24876512,373.82531497)
\curveto(143.03876134,373.88531493)(142.84876153,373.96031485)(142.67876512,374.05031497)
\curveto(142.04876233,374.32031449)(141.53876284,374.70531411)(141.14876512,375.20531497)
\curveto(140.75876362,375.70531311)(140.44876393,376.29531252)(140.21876512,376.97531497)
\curveto(140.1787642,377.09531172)(140.14376423,377.22031159)(140.11376512,377.35031497)
\curveto(140.09376428,377.48031133)(140.06876431,377.6153112)(140.03876512,377.75531497)
\curveto(140.01876436,377.80531101)(140.00876437,377.85031096)(140.00876512,377.89031497)
\curveto(140.01876436,377.93031088)(140.01876436,377.97531084)(140.00876512,378.02531497)
\curveto(139.98876439,378.1153107)(139.9737644,378.2103106)(139.96376512,378.31031497)
\curveto(139.96376441,378.4103104)(139.95376442,378.50531031)(139.93376512,378.59531497)
\lineto(139.93376512,378.88031497)
\curveto(139.91376446,378.93030988)(139.90376447,379.0153098)(139.90376512,379.13531497)
\curveto(139.90376447,379.25530956)(139.91376446,379.34030947)(139.93376512,379.39031497)
\curveto(139.94376443,379.42030939)(139.94376443,379.45030936)(139.93376512,379.48031497)
\curveto(139.92376445,379.52030929)(139.92376445,379.55030926)(139.93376512,379.57031497)
\lineto(139.93376512,379.70531497)
\curveto(139.94376443,379.78530903)(139.94876443,379.86530895)(139.94876512,379.94531497)
\curveto(139.95876442,380.03530878)(139.9737644,380.12030869)(139.99376512,380.20031497)
\curveto(140.01376436,380.26030855)(140.02376435,380.32030849)(140.02376512,380.38031497)
\curveto(140.02376435,380.45030836)(140.03376434,380.52030829)(140.05376512,380.59031497)
\curveto(140.10376427,380.76030805)(140.14376423,380.92530789)(140.17376512,381.08531497)
\curveto(140.20376417,381.24530757)(140.24876413,381.39530742)(140.30876512,381.53531497)
\lineto(140.45876512,381.92531497)
\curveto(140.51876386,382.06530675)(140.58376379,382.19030662)(140.65376512,382.30031497)
\curveto(140.80376357,382.56030625)(140.95376342,382.79530602)(141.10376512,383.00531497)
\curveto(141.13376324,383.05530576)(141.16876321,383.09530572)(141.20876512,383.12531497)
\curveto(141.25876312,383.16530565)(141.29876308,383.2103056)(141.32876512,383.26031497)
\curveto(141.38876299,383.34030547)(141.44876293,383.4103054)(141.50876512,383.47031497)
\lineto(141.71876512,383.65031497)
\curveto(141.7787626,383.70030511)(141.83376254,383.74530507)(141.88376512,383.78531497)
\curveto(141.94376243,383.83530498)(142.00876237,383.88530493)(142.07876512,383.93531497)
\curveto(142.22876215,384.04530477)(142.38376199,384.14030467)(142.54376512,384.22031497)
\curveto(142.71376166,384.30030451)(142.88876149,384.38030443)(143.06876512,384.46031497)
\curveto(143.1787612,384.5103043)(143.29376108,384.54530427)(143.41376512,384.56531497)
\curveto(143.54376083,384.59530422)(143.66876071,384.63030418)(143.78876512,384.67031497)
\curveto(143.85876052,384.68030413)(143.92376045,384.69030412)(143.98376512,384.70031497)
\lineto(144.16376512,384.73031497)
\curveto(144.24376013,384.74030407)(144.31876006,384.74530407)(144.38876512,384.74531497)
\curveto(144.46875991,384.75530406)(144.54875983,384.76530405)(144.62876512,384.77531497)
\curveto(144.64875973,384.78530403)(144.6737597,384.78530403)(144.70376512,384.77531497)
\curveto(144.73375964,384.76530405)(144.75875962,384.76530405)(144.77876512,384.77531497)
}
}
{
\newrgbcolor{curcolor}{0 0 0}
\pscustom[linestyle=none,fillstyle=solid,fillcolor=curcolor]
{
\newpath
\moveto(158.13860887,378.05531497)
\curveto(158.15860081,377.99531082)(158.1686008,377.90031091)(158.16860887,377.77031497)
\curveto(158.1686008,377.65031116)(158.1636008,377.56531125)(158.15360887,377.51531497)
\lineto(158.15360887,377.36531497)
\curveto(158.14360082,377.28531153)(158.13360083,377.2103116)(158.12360887,377.14031497)
\curveto(158.12360084,377.08031173)(158.11860085,377.0103118)(158.10860887,376.93031497)
\curveto(158.08860088,376.87031194)(158.07360089,376.810312)(158.06360887,376.75031497)
\curveto(158.0636009,376.69031212)(158.05360091,376.63031218)(158.03360887,376.57031497)
\curveto(157.99360097,376.44031237)(157.95860101,376.3103125)(157.92860887,376.18031497)
\curveto(157.89860107,376.05031276)(157.85860111,375.93031288)(157.80860887,375.82031497)
\curveto(157.59860137,375.34031347)(157.31860165,374.93531388)(156.96860887,374.60531497)
\curveto(156.61860235,374.28531453)(156.18860278,374.04031477)(155.67860887,373.87031497)
\curveto(155.5686034,373.83031498)(155.44860352,373.80031501)(155.31860887,373.78031497)
\curveto(155.19860377,373.76031505)(155.07360389,373.74031507)(154.94360887,373.72031497)
\curveto(154.88360408,373.7103151)(154.81860415,373.70531511)(154.74860887,373.70531497)
\curveto(154.68860428,373.69531512)(154.62860434,373.69031512)(154.56860887,373.69031497)
\curveto(154.52860444,373.68031513)(154.4686045,373.67531514)(154.38860887,373.67531497)
\curveto(154.31860465,373.67531514)(154.2686047,373.68031513)(154.23860887,373.69031497)
\curveto(154.19860477,373.70031511)(154.15860481,373.70531511)(154.11860887,373.70531497)
\curveto(154.07860489,373.69531512)(154.04360492,373.69531512)(154.01360887,373.70531497)
\lineto(153.92360887,373.70531497)
\lineto(153.56360887,373.75031497)
\curveto(153.42360554,373.79031502)(153.28860568,373.83031498)(153.15860887,373.87031497)
\curveto(153.02860594,373.9103149)(152.90360606,373.95531486)(152.78360887,374.00531497)
\curveto(152.33360663,374.20531461)(151.963607,374.46531435)(151.67360887,374.78531497)
\curveto(151.38360758,375.10531371)(151.14360782,375.49531332)(150.95360887,375.95531497)
\curveto(150.90360806,376.05531276)(150.8636081,376.15531266)(150.83360887,376.25531497)
\curveto(150.81360815,376.35531246)(150.79360817,376.46031235)(150.77360887,376.57031497)
\curveto(150.75360821,376.6103122)(150.74360822,376.64031217)(150.74360887,376.66031497)
\curveto(150.75360821,376.69031212)(150.75360821,376.72531209)(150.74360887,376.76531497)
\curveto(150.72360824,376.84531197)(150.70860826,376.92531189)(150.69860887,377.00531497)
\curveto(150.69860827,377.09531172)(150.68860828,377.18031163)(150.66860887,377.26031497)
\lineto(150.66860887,377.38031497)
\curveto(150.6686083,377.42031139)(150.6636083,377.46531135)(150.65360887,377.51531497)
\curveto(150.64360832,377.56531125)(150.63860833,377.65031116)(150.63860887,377.77031497)
\curveto(150.63860833,377.90031091)(150.64860832,377.99531082)(150.66860887,378.05531497)
\curveto(150.68860828,378.12531069)(150.69360827,378.19531062)(150.68360887,378.26531497)
\curveto(150.67360829,378.33531048)(150.67860829,378.40531041)(150.69860887,378.47531497)
\curveto(150.70860826,378.52531029)(150.71360825,378.56531025)(150.71360887,378.59531497)
\curveto(150.72360824,378.63531018)(150.73360823,378.68031013)(150.74360887,378.73031497)
\curveto(150.77360819,378.85030996)(150.79860817,378.97030984)(150.81860887,379.09031497)
\curveto(150.84860812,379.2103096)(150.88860808,379.32530949)(150.93860887,379.43531497)
\curveto(151.08860788,379.80530901)(151.2686077,380.13530868)(151.47860887,380.42531497)
\curveto(151.69860727,380.72530809)(151.963607,380.97530784)(152.27360887,381.17531497)
\curveto(152.39360657,381.25530756)(152.51860645,381.32030749)(152.64860887,381.37031497)
\curveto(152.77860619,381.43030738)(152.91360605,381.49030732)(153.05360887,381.55031497)
\curveto(153.17360579,381.60030721)(153.30360566,381.63030718)(153.44360887,381.64031497)
\curveto(153.58360538,381.66030715)(153.72360524,381.69030712)(153.86360887,381.73031497)
\lineto(154.05860887,381.73031497)
\curveto(154.12860484,381.74030707)(154.19360477,381.75030706)(154.25360887,381.76031497)
\curveto(155.14360382,381.77030704)(155.88360308,381.58530723)(156.47360887,381.20531497)
\curveto(157.0636019,380.82530799)(157.48860148,380.33030848)(157.74860887,379.72031497)
\curveto(157.79860117,379.62030919)(157.83860113,379.52030929)(157.86860887,379.42031497)
\curveto(157.89860107,379.32030949)(157.93360103,379.2153096)(157.97360887,379.10531497)
\curveto(158.00360096,378.99530982)(158.02860094,378.87530994)(158.04860887,378.74531497)
\curveto(158.0686009,378.62531019)(158.09360087,378.50031031)(158.12360887,378.37031497)
\curveto(158.13360083,378.32031049)(158.13360083,378.26531055)(158.12360887,378.20531497)
\curveto(158.12360084,378.15531066)(158.12860084,378.10531071)(158.13860887,378.05531497)
\moveto(156.80360887,377.20031497)
\curveto(156.82360214,377.27031154)(156.82860214,377.35031146)(156.81860887,377.44031497)
\lineto(156.81860887,377.69531497)
\curveto(156.81860215,378.08531073)(156.78360218,378.4153104)(156.71360887,378.68531497)
\curveto(156.68360228,378.76531005)(156.65860231,378.84530997)(156.63860887,378.92531497)
\curveto(156.61860235,379.00530981)(156.59360237,379.08030973)(156.56360887,379.15031497)
\curveto(156.28360268,379.80030901)(155.83860313,380.25030856)(155.22860887,380.50031497)
\curveto(155.15860381,380.53030828)(155.08360388,380.55030826)(155.00360887,380.56031497)
\lineto(154.76360887,380.62031497)
\curveto(154.68360428,380.64030817)(154.59860437,380.65030816)(154.50860887,380.65031497)
\lineto(154.23860887,380.65031497)
\lineto(153.96860887,380.60531497)
\curveto(153.8686051,380.58530823)(153.77360519,380.56030825)(153.68360887,380.53031497)
\curveto(153.60360536,380.5103083)(153.52360544,380.48030833)(153.44360887,380.44031497)
\curveto(153.37360559,380.42030839)(153.30860566,380.39030842)(153.24860887,380.35031497)
\curveto(153.18860578,380.3103085)(153.13360583,380.27030854)(153.08360887,380.23031497)
\curveto(152.84360612,380.06030875)(152.64860632,379.85530896)(152.49860887,379.61531497)
\curveto(152.34860662,379.37530944)(152.21860675,379.09530972)(152.10860887,378.77531497)
\curveto(152.07860689,378.67531014)(152.05860691,378.57031024)(152.04860887,378.46031497)
\curveto(152.03860693,378.36031045)(152.02360694,378.25531056)(152.00360887,378.14531497)
\curveto(151.99360697,378.10531071)(151.98860698,378.04031077)(151.98860887,377.95031497)
\curveto(151.97860699,377.92031089)(151.97360699,377.88531093)(151.97360887,377.84531497)
\curveto(151.98360698,377.80531101)(151.98860698,377.76031105)(151.98860887,377.71031497)
\lineto(151.98860887,377.41031497)
\curveto(151.98860698,377.3103115)(151.99860697,377.22031159)(152.01860887,377.14031497)
\lineto(152.04860887,376.96031497)
\curveto(152.0686069,376.86031195)(152.08360688,376.76031205)(152.09360887,376.66031497)
\curveto(152.11360685,376.57031224)(152.14360682,376.48531233)(152.18360887,376.40531497)
\curveto(152.28360668,376.16531265)(152.39860657,375.94031287)(152.52860887,375.73031497)
\curveto(152.6686063,375.52031329)(152.83860613,375.34531347)(153.03860887,375.20531497)
\curveto(153.08860588,375.17531364)(153.13360583,375.15031366)(153.17360887,375.13031497)
\curveto(153.21360575,375.1103137)(153.25860571,375.08531373)(153.30860887,375.05531497)
\curveto(153.38860558,375.00531381)(153.47360549,374.96031385)(153.56360887,374.92031497)
\curveto(153.6636053,374.89031392)(153.7686052,374.86031395)(153.87860887,374.83031497)
\curveto(153.92860504,374.810314)(153.97360499,374.80031401)(154.01360887,374.80031497)
\curveto(154.0636049,374.810314)(154.11360485,374.810314)(154.16360887,374.80031497)
\curveto(154.19360477,374.79031402)(154.25360471,374.78031403)(154.34360887,374.77031497)
\curveto(154.44360452,374.76031405)(154.51860445,374.76531405)(154.56860887,374.78531497)
\curveto(154.60860436,374.79531402)(154.64860432,374.79531402)(154.68860887,374.78531497)
\curveto(154.72860424,374.78531403)(154.7686042,374.79531402)(154.80860887,374.81531497)
\curveto(154.88860408,374.83531398)(154.968604,374.85031396)(155.04860887,374.86031497)
\curveto(155.12860384,374.88031393)(155.20360376,374.90531391)(155.27360887,374.93531497)
\curveto(155.61360335,375.07531374)(155.88860308,375.27031354)(156.09860887,375.52031497)
\curveto(156.30860266,375.77031304)(156.48360248,376.06531275)(156.62360887,376.40531497)
\curveto(156.67360229,376.52531229)(156.70360226,376.65031216)(156.71360887,376.78031497)
\curveto(156.73360223,376.92031189)(156.7636022,377.06031175)(156.80360887,377.20031497)
}
}
{
\newrgbcolor{curcolor}{0 0 0}
\pscustom[linestyle=none,fillstyle=solid,fillcolor=curcolor]
{
\newpath
\moveto(163.31689012,381.73031497)
\curveto(163.94688488,381.75030706)(164.45188438,381.66530715)(164.83189012,381.47531497)
\curveto(165.21188362,381.28530753)(165.51688331,381.00030781)(165.74689012,380.62031497)
\curveto(165.80688302,380.52030829)(165.85188298,380.4103084)(165.88189012,380.29031497)
\curveto(165.92188291,380.18030863)(165.95688287,380.06530875)(165.98689012,379.94531497)
\curveto(166.03688279,379.75530906)(166.06688276,379.55030926)(166.07689012,379.33031497)
\curveto(166.08688274,379.1103097)(166.09188274,378.88530993)(166.09189012,378.65531497)
\lineto(166.09189012,377.05031497)
\lineto(166.09189012,374.71031497)
\curveto(166.09188274,374.54031427)(166.08688274,374.37031444)(166.07689012,374.20031497)
\curveto(166.07688275,374.03031478)(166.01188282,373.92031489)(165.88189012,373.87031497)
\curveto(165.831883,373.85031496)(165.77688305,373.84031497)(165.71689012,373.84031497)
\curveto(165.66688316,373.83031498)(165.61188322,373.82531499)(165.55189012,373.82531497)
\curveto(165.42188341,373.82531499)(165.29688353,373.83031498)(165.17689012,373.84031497)
\curveto(165.05688377,373.84031497)(164.97188386,373.88031493)(164.92189012,373.96031497)
\curveto(164.87188396,374.03031478)(164.84688398,374.12031469)(164.84689012,374.23031497)
\lineto(164.84689012,374.56031497)
\lineto(164.84689012,375.85031497)
\lineto(164.84689012,378.29531497)
\curveto(164.84688398,378.56531025)(164.84188399,378.83030998)(164.83189012,379.09031497)
\curveto(164.82188401,379.36030945)(164.77688405,379.59030922)(164.69689012,379.78031497)
\curveto(164.61688421,379.98030883)(164.49688433,380.14030867)(164.33689012,380.26031497)
\curveto(164.17688465,380.39030842)(163.99188484,380.49030832)(163.78189012,380.56031497)
\curveto(163.72188511,380.58030823)(163.65688517,380.59030822)(163.58689012,380.59031497)
\curveto(163.5268853,380.60030821)(163.46688536,380.6153082)(163.40689012,380.63531497)
\curveto(163.35688547,380.64530817)(163.27688555,380.64530817)(163.16689012,380.63531497)
\curveto(163.06688576,380.63530818)(162.99688583,380.63030818)(162.95689012,380.62031497)
\curveto(162.91688591,380.60030821)(162.88188595,380.59030822)(162.85189012,380.59031497)
\curveto(162.82188601,380.60030821)(162.78688604,380.60030821)(162.74689012,380.59031497)
\curveto(162.61688621,380.56030825)(162.49188634,380.52530829)(162.37189012,380.48531497)
\curveto(162.26188657,380.45530836)(162.15688667,380.4103084)(162.05689012,380.35031497)
\curveto(162.01688681,380.33030848)(161.98188685,380.3103085)(161.95189012,380.29031497)
\curveto(161.92188691,380.27030854)(161.88688694,380.25030856)(161.84689012,380.23031497)
\curveto(161.49688733,379.98030883)(161.24188759,379.60530921)(161.08189012,379.10531497)
\curveto(161.05188778,379.02530979)(161.0318878,378.94030987)(161.02189012,378.85031497)
\curveto(161.01188782,378.77031004)(160.99688783,378.69031012)(160.97689012,378.61031497)
\curveto(160.95688787,378.56031025)(160.95188788,378.5103103)(160.96189012,378.46031497)
\curveto(160.97188786,378.42031039)(160.96688786,378.38031043)(160.94689012,378.34031497)
\lineto(160.94689012,378.02531497)
\curveto(160.93688789,377.99531082)(160.9318879,377.96031085)(160.93189012,377.92031497)
\curveto(160.94188789,377.88031093)(160.94688788,377.83531098)(160.94689012,377.78531497)
\lineto(160.94689012,377.33531497)
\lineto(160.94689012,375.89531497)
\lineto(160.94689012,374.57531497)
\lineto(160.94689012,374.23031497)
\curveto(160.94688788,374.12031469)(160.92188791,374.03031478)(160.87189012,373.96031497)
\curveto(160.82188801,373.88031493)(160.7318881,373.84031497)(160.60189012,373.84031497)
\curveto(160.48188835,373.83031498)(160.35688847,373.82531499)(160.22689012,373.82531497)
\curveto(160.14688868,373.82531499)(160.07188876,373.83031498)(160.00189012,373.84031497)
\curveto(159.9318889,373.85031496)(159.87188896,373.87531494)(159.82189012,373.91531497)
\curveto(159.74188909,373.96531485)(159.70188913,374.06031475)(159.70189012,374.20031497)
\lineto(159.70189012,374.60531497)
\lineto(159.70189012,376.37531497)
\lineto(159.70189012,380.00531497)
\lineto(159.70189012,380.92031497)
\lineto(159.70189012,381.19031497)
\curveto(159.70188913,381.28030753)(159.72188911,381.35030746)(159.76189012,381.40031497)
\curveto(159.79188904,381.46030735)(159.84188899,381.50030731)(159.91189012,381.52031497)
\curveto(159.95188888,381.53030728)(160.00688882,381.54030727)(160.07689012,381.55031497)
\curveto(160.15688867,381.56030725)(160.23688859,381.56530725)(160.31689012,381.56531497)
\curveto(160.39688843,381.56530725)(160.47188836,381.56030725)(160.54189012,381.55031497)
\curveto(160.62188821,381.54030727)(160.67688815,381.52530729)(160.70689012,381.50531497)
\curveto(160.81688801,381.43530738)(160.86688796,381.34530747)(160.85689012,381.23531497)
\curveto(160.84688798,381.13530768)(160.86188797,381.02030779)(160.90189012,380.89031497)
\curveto(160.92188791,380.83030798)(160.96188787,380.78030803)(161.02189012,380.74031497)
\curveto(161.14188769,380.73030808)(161.23688759,380.77530804)(161.30689012,380.87531497)
\curveto(161.38688744,380.97530784)(161.46688736,381.05530776)(161.54689012,381.11531497)
\curveto(161.68688714,381.2153076)(161.826887,381.30530751)(161.96689012,381.38531497)
\curveto(162.11688671,381.47530734)(162.28688654,381.55030726)(162.47689012,381.61031497)
\curveto(162.55688627,381.64030717)(162.64188619,381.66030715)(162.73189012,381.67031497)
\curveto(162.831886,381.68030713)(162.9268859,381.69530712)(163.01689012,381.71531497)
\curveto(163.06688576,381.72530709)(163.11688571,381.73030708)(163.16689012,381.73031497)
\lineto(163.31689012,381.73031497)
}
}
{
\newrgbcolor{curcolor}{0 0 0}
\pscustom[linestyle=none,fillstyle=solid,fillcolor=curcolor]
{
\newpath
\moveto(168.92149949,383.92031497)
\curveto(169.07149748,383.92030489)(169.22149733,383.9153049)(169.37149949,383.90531497)
\curveto(169.52149703,383.90530491)(169.62649693,383.86530495)(169.68649949,383.78531497)
\curveto(169.73649682,383.72530509)(169.76149679,383.64030517)(169.76149949,383.53031497)
\curveto(169.77149678,383.43030538)(169.77649678,383.32530549)(169.77649949,383.21531497)
\lineto(169.77649949,382.34531497)
\curveto(169.77649678,382.26530655)(169.77149678,382.18030663)(169.76149949,382.09031497)
\curveto(169.76149679,382.0103068)(169.77149678,381.94030687)(169.79149949,381.88031497)
\curveto(169.83149672,381.74030707)(169.92149663,381.65030716)(170.06149949,381.61031497)
\curveto(170.11149644,381.60030721)(170.1564964,381.59530722)(170.19649949,381.59531497)
\lineto(170.34649949,381.59531497)
\lineto(170.75149949,381.59531497)
\curveto(170.91149564,381.60530721)(171.02649553,381.59530722)(171.09649949,381.56531497)
\curveto(171.18649537,381.50530731)(171.24649531,381.44530737)(171.27649949,381.38531497)
\curveto(171.29649526,381.34530747)(171.30649525,381.30030751)(171.30649949,381.25031497)
\lineto(171.30649949,381.10031497)
\curveto(171.30649525,380.99030782)(171.30149525,380.88530793)(171.29149949,380.78531497)
\curveto(171.28149527,380.69530812)(171.24649531,380.62530819)(171.18649949,380.57531497)
\curveto(171.12649543,380.52530829)(171.04149551,380.49530832)(170.93149949,380.48531497)
\lineto(170.60149949,380.48531497)
\curveto(170.49149606,380.49530832)(170.38149617,380.50030831)(170.27149949,380.50031497)
\curveto(170.16149639,380.50030831)(170.06649649,380.48530833)(169.98649949,380.45531497)
\curveto(169.91649664,380.42530839)(169.86649669,380.37530844)(169.83649949,380.30531497)
\curveto(169.80649675,380.23530858)(169.78649677,380.15030866)(169.77649949,380.05031497)
\curveto(169.76649679,379.96030885)(169.76149679,379.86030895)(169.76149949,379.75031497)
\curveto(169.77149678,379.65030916)(169.77649678,379.55030926)(169.77649949,379.45031497)
\lineto(169.77649949,376.48031497)
\curveto(169.77649678,376.26031255)(169.77149678,376.02531279)(169.76149949,375.77531497)
\curveto(169.76149679,375.53531328)(169.80649675,375.35031346)(169.89649949,375.22031497)
\curveto(169.94649661,375.14031367)(170.01149654,375.08531373)(170.09149949,375.05531497)
\curveto(170.17149638,375.02531379)(170.26649629,375.00031381)(170.37649949,374.98031497)
\curveto(170.40649615,374.97031384)(170.43649612,374.96531385)(170.46649949,374.96531497)
\curveto(170.50649605,374.97531384)(170.54149601,374.97531384)(170.57149949,374.96531497)
\lineto(170.76649949,374.96531497)
\curveto(170.86649569,374.96531385)(170.9564956,374.95531386)(171.03649949,374.93531497)
\curveto(171.12649543,374.92531389)(171.19149536,374.89031392)(171.23149949,374.83031497)
\curveto(171.2514953,374.80031401)(171.26649529,374.74531407)(171.27649949,374.66531497)
\curveto(171.29649526,374.59531422)(171.30649525,374.52031429)(171.30649949,374.44031497)
\curveto(171.31649524,374.36031445)(171.31649524,374.28031453)(171.30649949,374.20031497)
\curveto(171.29649526,374.13031468)(171.27649528,374.07531474)(171.24649949,374.03531497)
\curveto(171.20649535,373.96531485)(171.13149542,373.9153149)(171.02149949,373.88531497)
\curveto(170.94149561,373.86531495)(170.8514957,373.85531496)(170.75149949,373.85531497)
\curveto(170.6514959,373.86531495)(170.56149599,373.87031494)(170.48149949,373.87031497)
\curveto(170.42149613,373.87031494)(170.36149619,373.86531495)(170.30149949,373.85531497)
\curveto(170.24149631,373.85531496)(170.18649637,373.86031495)(170.13649949,373.87031497)
\lineto(169.95649949,373.87031497)
\curveto(169.90649665,373.88031493)(169.8564967,373.88531493)(169.80649949,373.88531497)
\curveto(169.76649679,373.89531492)(169.72149683,373.90031491)(169.67149949,373.90031497)
\curveto(169.47149708,373.95031486)(169.29649726,374.00531481)(169.14649949,374.06531497)
\curveto(169.00649755,374.12531469)(168.88649767,374.23031458)(168.78649949,374.38031497)
\curveto(168.64649791,374.58031423)(168.56649799,374.83031398)(168.54649949,375.13031497)
\curveto(168.52649803,375.44031337)(168.51649804,375.77031304)(168.51649949,376.12031497)
\lineto(168.51649949,380.05031497)
\curveto(168.48649807,380.18030863)(168.4564981,380.27530854)(168.42649949,380.33531497)
\curveto(168.40649815,380.39530842)(168.33649822,380.44530837)(168.21649949,380.48531497)
\curveto(168.17649838,380.49530832)(168.13649842,380.49530832)(168.09649949,380.48531497)
\curveto(168.0564985,380.47530834)(168.01649854,380.48030833)(167.97649949,380.50031497)
\lineto(167.73649949,380.50031497)
\curveto(167.60649895,380.50030831)(167.49649906,380.5103083)(167.40649949,380.53031497)
\curveto(167.32649923,380.56030825)(167.27149928,380.62030819)(167.24149949,380.71031497)
\curveto(167.22149933,380.75030806)(167.20649935,380.79530802)(167.19649949,380.84531497)
\lineto(167.19649949,380.99531497)
\curveto(167.19649936,381.13530768)(167.20649935,381.25030756)(167.22649949,381.34031497)
\curveto(167.24649931,381.44030737)(167.30649925,381.5153073)(167.40649949,381.56531497)
\curveto(167.51649904,381.60530721)(167.6564989,381.6153072)(167.82649949,381.59531497)
\curveto(168.00649855,381.57530724)(168.1564984,381.58530723)(168.27649949,381.62531497)
\curveto(168.36649819,381.67530714)(168.43649812,381.74530707)(168.48649949,381.83531497)
\curveto(168.50649805,381.89530692)(168.51649804,381.97030684)(168.51649949,382.06031497)
\lineto(168.51649949,382.31531497)
\lineto(168.51649949,383.24531497)
\lineto(168.51649949,383.48531497)
\curveto(168.51649804,383.57530524)(168.52649803,383.65030516)(168.54649949,383.71031497)
\curveto(168.58649797,383.79030502)(168.66149789,383.85530496)(168.77149949,383.90531497)
\curveto(168.80149775,383.90530491)(168.82649773,383.90530491)(168.84649949,383.90531497)
\curveto(168.87649768,383.9153049)(168.90149765,383.92030489)(168.92149949,383.92031497)
}
}
{
\newrgbcolor{curcolor}{0 0 0}
\pscustom[linestyle=none,fillstyle=solid,fillcolor=curcolor]
{
\newpath
\moveto(179.57829637,374.41031497)
\curveto(179.60828854,374.25031456)(179.59328855,374.1153147)(179.53329637,374.00531497)
\curveto(179.47328867,373.90531491)(179.39328875,373.83031498)(179.29329637,373.78031497)
\curveto(179.2432889,373.76031505)(179.18828896,373.75031506)(179.12829637,373.75031497)
\curveto(179.07828907,373.75031506)(179.02328912,373.74031507)(178.96329637,373.72031497)
\curveto(178.7432894,373.67031514)(178.52328962,373.68531513)(178.30329637,373.76531497)
\curveto(178.09329005,373.83531498)(177.9482902,373.92531489)(177.86829637,374.03531497)
\curveto(177.81829033,374.10531471)(177.77329037,374.18531463)(177.73329637,374.27531497)
\curveto(177.69329045,374.37531444)(177.6432905,374.45531436)(177.58329637,374.51531497)
\curveto(177.56329058,374.53531428)(177.53829061,374.55531426)(177.50829637,374.57531497)
\curveto(177.48829066,374.59531422)(177.45829069,374.60031421)(177.41829637,374.59031497)
\curveto(177.30829084,374.56031425)(177.20329094,374.50531431)(177.10329637,374.42531497)
\curveto(177.01329113,374.34531447)(176.92329122,374.27531454)(176.83329637,374.21531497)
\curveto(176.70329144,374.13531468)(176.56329158,374.06031475)(176.41329637,373.99031497)
\curveto(176.26329188,373.93031488)(176.10329204,373.87531494)(175.93329637,373.82531497)
\curveto(175.83329231,373.79531502)(175.72329242,373.77531504)(175.60329637,373.76531497)
\curveto(175.49329265,373.75531506)(175.38329276,373.74031507)(175.27329637,373.72031497)
\curveto(175.22329292,373.7103151)(175.17829297,373.70531511)(175.13829637,373.70531497)
\lineto(175.03329637,373.70531497)
\curveto(174.92329322,373.68531513)(174.81829333,373.68531513)(174.71829637,373.70531497)
\lineto(174.58329637,373.70531497)
\curveto(174.53329361,373.7153151)(174.48329366,373.72031509)(174.43329637,373.72031497)
\curveto(174.38329376,373.72031509)(174.33829381,373.73031508)(174.29829637,373.75031497)
\curveto(174.25829389,373.76031505)(174.22329392,373.76531505)(174.19329637,373.76531497)
\curveto(174.17329397,373.75531506)(174.148294,373.75531506)(174.11829637,373.76531497)
\lineto(173.87829637,373.82531497)
\curveto(173.79829435,373.83531498)(173.72329442,373.85531496)(173.65329637,373.88531497)
\curveto(173.35329479,374.0153148)(173.10829504,374.16031465)(172.91829637,374.32031497)
\curveto(172.73829541,374.49031432)(172.58829556,374.72531409)(172.46829637,375.02531497)
\curveto(172.37829577,375.24531357)(172.33329581,375.5103133)(172.33329637,375.82031497)
\lineto(172.33329637,376.13531497)
\curveto(172.3432958,376.18531263)(172.3482958,376.23531258)(172.34829637,376.28531497)
\lineto(172.37829637,376.46531497)
\lineto(172.49829637,376.79531497)
\curveto(172.53829561,376.90531191)(172.58829556,377.00531181)(172.64829637,377.09531497)
\curveto(172.82829532,377.38531143)(173.07329507,377.60031121)(173.38329637,377.74031497)
\curveto(173.69329445,377.88031093)(174.03329411,378.00531081)(174.40329637,378.11531497)
\curveto(174.5432936,378.15531066)(174.68829346,378.18531063)(174.83829637,378.20531497)
\curveto(174.98829316,378.22531059)(175.13829301,378.25031056)(175.28829637,378.28031497)
\curveto(175.35829279,378.30031051)(175.42329272,378.3103105)(175.48329637,378.31031497)
\curveto(175.55329259,378.3103105)(175.62829252,378.32031049)(175.70829637,378.34031497)
\curveto(175.77829237,378.36031045)(175.8482923,378.37031044)(175.91829637,378.37031497)
\curveto(175.98829216,378.38031043)(176.06329208,378.39531042)(176.14329637,378.41531497)
\curveto(176.39329175,378.47531034)(176.62829152,378.52531029)(176.84829637,378.56531497)
\curveto(177.06829108,378.6153102)(177.2432909,378.73031008)(177.37329637,378.91031497)
\curveto(177.43329071,378.99030982)(177.48329066,379.09030972)(177.52329637,379.21031497)
\curveto(177.56329058,379.34030947)(177.56329058,379.48030933)(177.52329637,379.63031497)
\curveto(177.46329068,379.87030894)(177.37329077,380.06030875)(177.25329637,380.20031497)
\curveto(177.143291,380.34030847)(176.98329116,380.45030836)(176.77329637,380.53031497)
\curveto(176.65329149,380.58030823)(176.50829164,380.6153082)(176.33829637,380.63531497)
\curveto(176.17829197,380.65530816)(176.00829214,380.66530815)(175.82829637,380.66531497)
\curveto(175.6482925,380.66530815)(175.47329267,380.65530816)(175.30329637,380.63531497)
\curveto(175.13329301,380.6153082)(174.98829316,380.58530823)(174.86829637,380.54531497)
\curveto(174.69829345,380.48530833)(174.53329361,380.40030841)(174.37329637,380.29031497)
\curveto(174.29329385,380.23030858)(174.21829393,380.15030866)(174.14829637,380.05031497)
\curveto(174.08829406,379.96030885)(174.03329411,379.86030895)(173.98329637,379.75031497)
\curveto(173.95329419,379.67030914)(173.92329422,379.58530923)(173.89329637,379.49531497)
\curveto(173.87329427,379.40530941)(173.82829432,379.33530948)(173.75829637,379.28531497)
\curveto(173.71829443,379.25530956)(173.6482945,379.23030958)(173.54829637,379.21031497)
\curveto(173.45829469,379.20030961)(173.36329478,379.19530962)(173.26329637,379.19531497)
\curveto(173.16329498,379.19530962)(173.06329508,379.20030961)(172.96329637,379.21031497)
\curveto(172.87329527,379.23030958)(172.80829534,379.25530956)(172.76829637,379.28531497)
\curveto(172.72829542,379.3153095)(172.69829545,379.36530945)(172.67829637,379.43531497)
\curveto(172.65829549,379.50530931)(172.65829549,379.58030923)(172.67829637,379.66031497)
\curveto(172.70829544,379.79030902)(172.73829541,379.9103089)(172.76829637,380.02031497)
\curveto(172.80829534,380.14030867)(172.85329529,380.25530856)(172.90329637,380.36531497)
\curveto(173.09329505,380.7153081)(173.33329481,380.98530783)(173.62329637,381.17531497)
\curveto(173.91329423,381.37530744)(174.27329387,381.53530728)(174.70329637,381.65531497)
\curveto(174.80329334,381.67530714)(174.90329324,381.69030712)(175.00329637,381.70031497)
\curveto(175.11329303,381.7103071)(175.22329292,381.72530709)(175.33329637,381.74531497)
\curveto(175.37329277,381.75530706)(175.43829271,381.75530706)(175.52829637,381.74531497)
\curveto(175.61829253,381.74530707)(175.67329247,381.75530706)(175.69329637,381.77531497)
\curveto(176.39329175,381.78530703)(177.00329114,381.70530711)(177.52329637,381.53531497)
\curveto(178.0432901,381.36530745)(178.40828974,381.04030777)(178.61829637,380.56031497)
\curveto(178.70828944,380.36030845)(178.75828939,380.12530869)(178.76829637,379.85531497)
\curveto(178.78828936,379.59530922)(178.79828935,379.32030949)(178.79829637,379.03031497)
\lineto(178.79829637,375.71531497)
\curveto(178.79828935,375.57531324)(178.80328934,375.44031337)(178.81329637,375.31031497)
\curveto(178.82328932,375.18031363)(178.85328929,375.07531374)(178.90329637,374.99531497)
\curveto(178.95328919,374.92531389)(179.01828913,374.87531394)(179.09829637,374.84531497)
\curveto(179.18828896,374.80531401)(179.27328887,374.77531404)(179.35329637,374.75531497)
\curveto(179.43328871,374.74531407)(179.49328865,374.70031411)(179.53329637,374.62031497)
\curveto(179.55328859,374.59031422)(179.56328858,374.56031425)(179.56329637,374.53031497)
\curveto(179.56328858,374.50031431)(179.56828858,374.46031435)(179.57829637,374.41031497)
\moveto(177.43329637,376.07531497)
\curveto(177.49329065,376.2153126)(177.52329062,376.37531244)(177.52329637,376.55531497)
\curveto(177.53329061,376.74531207)(177.53829061,376.94031187)(177.53829637,377.14031497)
\curveto(177.53829061,377.25031156)(177.53329061,377.35031146)(177.52329637,377.44031497)
\curveto(177.51329063,377.53031128)(177.47329067,377.60031121)(177.40329637,377.65031497)
\curveto(177.37329077,377.67031114)(177.30329084,377.68031113)(177.19329637,377.68031497)
\curveto(177.17329097,377.66031115)(177.13829101,377.65031116)(177.08829637,377.65031497)
\curveto(177.03829111,377.65031116)(176.99329115,377.64031117)(176.95329637,377.62031497)
\curveto(176.87329127,377.60031121)(176.78329136,377.58031123)(176.68329637,377.56031497)
\lineto(176.38329637,377.50031497)
\curveto(176.35329179,377.50031131)(176.31829183,377.49531132)(176.27829637,377.48531497)
\lineto(176.17329637,377.48531497)
\curveto(176.02329212,377.44531137)(175.85829229,377.42031139)(175.67829637,377.41031497)
\curveto(175.50829264,377.4103114)(175.3482928,377.39031142)(175.19829637,377.35031497)
\curveto(175.11829303,377.33031148)(175.0432931,377.3103115)(174.97329637,377.29031497)
\curveto(174.91329323,377.28031153)(174.8432933,377.26531155)(174.76329637,377.24531497)
\curveto(174.60329354,377.19531162)(174.45329369,377.13031168)(174.31329637,377.05031497)
\curveto(174.17329397,376.98031183)(174.05329409,376.89031192)(173.95329637,376.78031497)
\curveto(173.85329429,376.67031214)(173.77829437,376.53531228)(173.72829637,376.37531497)
\curveto(173.67829447,376.22531259)(173.65829449,376.04031277)(173.66829637,375.82031497)
\curveto(173.66829448,375.72031309)(173.68329446,375.62531319)(173.71329637,375.53531497)
\curveto(173.75329439,375.45531336)(173.79829435,375.38031343)(173.84829637,375.31031497)
\curveto(173.92829422,375.20031361)(174.03329411,375.10531371)(174.16329637,375.02531497)
\curveto(174.29329385,374.95531386)(174.43329371,374.89531392)(174.58329637,374.84531497)
\curveto(174.63329351,374.83531398)(174.68329346,374.83031398)(174.73329637,374.83031497)
\curveto(174.78329336,374.83031398)(174.83329331,374.82531399)(174.88329637,374.81531497)
\curveto(174.95329319,374.79531402)(175.03829311,374.78031403)(175.13829637,374.77031497)
\curveto(175.2482929,374.77031404)(175.33829281,374.78031403)(175.40829637,374.80031497)
\curveto(175.46829268,374.82031399)(175.52829262,374.82531399)(175.58829637,374.81531497)
\curveto(175.6482925,374.815314)(175.70829244,374.82531399)(175.76829637,374.84531497)
\curveto(175.8482923,374.86531395)(175.92329222,374.88031393)(175.99329637,374.89031497)
\curveto(176.07329207,374.90031391)(176.148292,374.92031389)(176.21829637,374.95031497)
\curveto(176.50829164,375.07031374)(176.75329139,375.2153136)(176.95329637,375.38531497)
\curveto(177.16329098,375.55531326)(177.32329082,375.78531303)(177.43329637,376.07531497)
}
}
{
\newrgbcolor{curcolor}{0 0 0}
\pscustom[linestyle=none,fillstyle=solid,fillcolor=curcolor]
{
\newpath
\moveto(183.88493699,381.76031497)
\curveto(184.6249322,381.77030704)(185.23993159,381.66030715)(185.72993699,381.43031497)
\curveto(186.2299306,381.2103076)(186.6249302,380.87530794)(186.91493699,380.42531497)
\curveto(187.04492978,380.22530859)(187.15492967,379.98030883)(187.24493699,379.69031497)
\curveto(187.26492956,379.64030917)(187.27992955,379.57530924)(187.28993699,379.49531497)
\curveto(187.29992953,379.4153094)(187.29492953,379.34530947)(187.27493699,379.28531497)
\curveto(187.24492958,379.23530958)(187.19492963,379.19030962)(187.12493699,379.15031497)
\curveto(187.09492973,379.13030968)(187.06492976,379.12030969)(187.03493699,379.12031497)
\curveto(187.00492982,379.13030968)(186.96992986,379.13030968)(186.92993699,379.12031497)
\curveto(186.88992994,379.1103097)(186.84992998,379.10530971)(186.80993699,379.10531497)
\curveto(186.76993006,379.1153097)(186.7299301,379.12030969)(186.68993699,379.12031497)
\lineto(186.37493699,379.12031497)
\curveto(186.27493055,379.13030968)(186.18993064,379.16030965)(186.11993699,379.21031497)
\curveto(186.03993079,379.27030954)(185.98493084,379.35530946)(185.95493699,379.46531497)
\curveto(185.9249309,379.57530924)(185.88493094,379.67030914)(185.83493699,379.75031497)
\curveto(185.68493114,380.0103088)(185.48993134,380.2153086)(185.24993699,380.36531497)
\curveto(185.16993166,380.4153084)(185.08493174,380.45530836)(184.99493699,380.48531497)
\curveto(184.90493192,380.52530829)(184.80993202,380.56030825)(184.70993699,380.59031497)
\curveto(184.56993226,380.63030818)(184.38493244,380.65030816)(184.15493699,380.65031497)
\curveto(183.9249329,380.66030815)(183.73493309,380.64030817)(183.58493699,380.59031497)
\curveto(183.51493331,380.57030824)(183.44993338,380.55530826)(183.38993699,380.54531497)
\curveto(183.3299335,380.53530828)(183.26493356,380.52030829)(183.19493699,380.50031497)
\curveto(182.93493389,380.39030842)(182.70493412,380.24030857)(182.50493699,380.05031497)
\curveto(182.30493452,379.86030895)(182.14993468,379.63530918)(182.03993699,379.37531497)
\curveto(181.99993483,379.28530953)(181.96493486,379.19030962)(181.93493699,379.09031497)
\curveto(181.90493492,379.00030981)(181.87493495,378.90030991)(181.84493699,378.79031497)
\lineto(181.75493699,378.38531497)
\curveto(181.74493508,378.33531048)(181.73993509,378.28031053)(181.73993699,378.22031497)
\curveto(181.74993508,378.16031065)(181.74493508,378.10531071)(181.72493699,378.05531497)
\lineto(181.72493699,377.93531497)
\curveto(181.71493511,377.89531092)(181.70493512,377.83031098)(181.69493699,377.74031497)
\curveto(181.69493513,377.65031116)(181.70493512,377.58531123)(181.72493699,377.54531497)
\curveto(181.73493509,377.49531132)(181.73493509,377.44531137)(181.72493699,377.39531497)
\curveto(181.71493511,377.34531147)(181.71493511,377.29531152)(181.72493699,377.24531497)
\curveto(181.73493509,377.20531161)(181.73993509,377.13531168)(181.73993699,377.03531497)
\curveto(181.75993507,376.95531186)(181.77493505,376.87031194)(181.78493699,376.78031497)
\curveto(181.80493502,376.69031212)(181.824935,376.60531221)(181.84493699,376.52531497)
\curveto(181.95493487,376.20531261)(182.07993475,375.92531289)(182.21993699,375.68531497)
\curveto(182.36993446,375.45531336)(182.57493425,375.25531356)(182.83493699,375.08531497)
\curveto(182.9249339,375.03531378)(183.01493381,374.99031382)(183.10493699,374.95031497)
\curveto(183.20493362,374.9103139)(183.30993352,374.87031394)(183.41993699,374.83031497)
\curveto(183.46993336,374.82031399)(183.50993332,374.815314)(183.53993699,374.81531497)
\curveto(183.56993326,374.815314)(183.60993322,374.810314)(183.65993699,374.80031497)
\curveto(183.68993314,374.79031402)(183.73993309,374.78531403)(183.80993699,374.78531497)
\lineto(183.97493699,374.78531497)
\curveto(183.97493285,374.77531404)(183.99493283,374.77031404)(184.03493699,374.77031497)
\curveto(184.05493277,374.78031403)(184.07993275,374.78031403)(184.10993699,374.77031497)
\curveto(184.13993269,374.77031404)(184.16993266,374.77531404)(184.19993699,374.78531497)
\curveto(184.26993256,374.80531401)(184.33493249,374.810314)(184.39493699,374.80031497)
\curveto(184.46493236,374.80031401)(184.53493229,374.810314)(184.60493699,374.83031497)
\curveto(184.86493196,374.9103139)(185.08993174,375.0103138)(185.27993699,375.13031497)
\curveto(185.46993136,375.26031355)(185.6299312,375.42531339)(185.75993699,375.62531497)
\curveto(185.80993102,375.70531311)(185.85493097,375.79031302)(185.89493699,375.88031497)
\lineto(186.01493699,376.15031497)
\curveto(186.03493079,376.23031258)(186.05493077,376.30531251)(186.07493699,376.37531497)
\curveto(186.10493072,376.45531236)(186.15493067,376.52031229)(186.22493699,376.57031497)
\curveto(186.25493057,376.60031221)(186.31493051,376.62031219)(186.40493699,376.63031497)
\curveto(186.49493033,376.65031216)(186.58993024,376.66031215)(186.68993699,376.66031497)
\curveto(186.79993003,376.67031214)(186.89992993,376.67031214)(186.98993699,376.66031497)
\curveto(187.08992974,376.65031216)(187.15992967,376.63031218)(187.19993699,376.60031497)
\curveto(187.25992957,376.56031225)(187.29492953,376.50031231)(187.30493699,376.42031497)
\curveto(187.3249295,376.34031247)(187.3249295,376.25531256)(187.30493699,376.16531497)
\curveto(187.25492957,376.0153128)(187.20492962,375.87031294)(187.15493699,375.73031497)
\curveto(187.11492971,375.60031321)(187.05992977,375.47031334)(186.98993699,375.34031497)
\curveto(186.83992999,375.04031377)(186.64993018,374.77531404)(186.41993699,374.54531497)
\curveto(186.19993063,374.3153145)(185.9299309,374.13031468)(185.60993699,373.99031497)
\curveto(185.5299313,373.95031486)(185.44493138,373.9153149)(185.35493699,373.88531497)
\curveto(185.26493156,373.86531495)(185.16993166,373.84031497)(185.06993699,373.81031497)
\curveto(184.95993187,373.77031504)(184.84993198,373.75031506)(184.73993699,373.75031497)
\curveto(184.6299322,373.74031507)(184.51993231,373.72531509)(184.40993699,373.70531497)
\curveto(184.36993246,373.68531513)(184.3299325,373.68031513)(184.28993699,373.69031497)
\curveto(184.24993258,373.70031511)(184.20993262,373.70031511)(184.16993699,373.69031497)
\lineto(184.03493699,373.69031497)
\lineto(183.79493699,373.69031497)
\curveto(183.7249331,373.68031513)(183.65993317,373.68531513)(183.59993699,373.70531497)
\lineto(183.52493699,373.70531497)
\lineto(183.16493699,373.75031497)
\curveto(183.03493379,373.79031502)(182.90993392,373.82531499)(182.78993699,373.85531497)
\curveto(182.66993416,373.88531493)(182.55493427,373.92531489)(182.44493699,373.97531497)
\curveto(182.08493474,374.13531468)(181.78493504,374.32531449)(181.54493699,374.54531497)
\curveto(181.31493551,374.76531405)(181.09993573,375.03531378)(180.89993699,375.35531497)
\curveto(180.84993598,375.43531338)(180.80493602,375.52531329)(180.76493699,375.62531497)
\lineto(180.64493699,375.92531497)
\curveto(180.59493623,376.03531278)(180.55993627,376.15031266)(180.53993699,376.27031497)
\curveto(180.51993631,376.39031242)(180.49493633,376.5103123)(180.46493699,376.63031497)
\curveto(180.45493637,376.67031214)(180.44993638,376.7103121)(180.44993699,376.75031497)
\curveto(180.44993638,376.79031202)(180.44493638,376.83031198)(180.43493699,376.87031497)
\curveto(180.41493641,376.93031188)(180.40493642,376.99531182)(180.40493699,377.06531497)
\curveto(180.41493641,377.13531168)(180.40993642,377.20031161)(180.38993699,377.26031497)
\lineto(180.38993699,377.41031497)
\curveto(180.37993645,377.46031135)(180.37493645,377.53031128)(180.37493699,377.62031497)
\curveto(180.37493645,377.7103111)(180.37993645,377.78031103)(180.38993699,377.83031497)
\curveto(180.39993643,377.88031093)(180.39993643,377.92531089)(180.38993699,377.96531497)
\curveto(180.38993644,378.00531081)(180.39493643,378.04531077)(180.40493699,378.08531497)
\curveto(180.4249364,378.15531066)(180.4299364,378.22531059)(180.41993699,378.29531497)
\curveto(180.41993641,378.36531045)(180.4299364,378.43031038)(180.44993699,378.49031497)
\curveto(180.48993634,378.66031015)(180.5249363,378.83030998)(180.55493699,379.00031497)
\curveto(180.58493624,379.17030964)(180.6299362,379.33030948)(180.68993699,379.48031497)
\curveto(180.89993593,380.00030881)(181.15493567,380.42030839)(181.45493699,380.74031497)
\curveto(181.75493507,381.06030775)(182.16493466,381.32530749)(182.68493699,381.53531497)
\curveto(182.79493403,381.58530723)(182.91493391,381.62030719)(183.04493699,381.64031497)
\curveto(183.17493365,381.66030715)(183.30993352,381.68530713)(183.44993699,381.71531497)
\curveto(183.51993331,381.72530709)(183.58993324,381.73030708)(183.65993699,381.73031497)
\curveto(183.7299331,381.74030707)(183.80493302,381.75030706)(183.88493699,381.76031497)
}
}
{
\newrgbcolor{curcolor}{0 0 0}
\pscustom[linestyle=none,fillstyle=solid,fillcolor=curcolor]
{
\newpath
\moveto(189.75157762,383.92031497)
\curveto(189.90157561,383.92030489)(190.05157546,383.9153049)(190.20157762,383.90531497)
\curveto(190.35157516,383.90530491)(190.45657505,383.86530495)(190.51657762,383.78531497)
\curveto(190.56657494,383.72530509)(190.59157492,383.64030517)(190.59157762,383.53031497)
\curveto(190.60157491,383.43030538)(190.6065749,383.32530549)(190.60657762,383.21531497)
\lineto(190.60657762,382.34531497)
\curveto(190.6065749,382.26530655)(190.60157491,382.18030663)(190.59157762,382.09031497)
\curveto(190.59157492,382.0103068)(190.60157491,381.94030687)(190.62157762,381.88031497)
\curveto(190.66157485,381.74030707)(190.75157476,381.65030716)(190.89157762,381.61031497)
\curveto(190.94157457,381.60030721)(190.98657452,381.59530722)(191.02657762,381.59531497)
\lineto(191.17657762,381.59531497)
\lineto(191.58157762,381.59531497)
\curveto(191.74157377,381.60530721)(191.85657365,381.59530722)(191.92657762,381.56531497)
\curveto(192.01657349,381.50530731)(192.07657343,381.44530737)(192.10657762,381.38531497)
\curveto(192.12657338,381.34530747)(192.13657337,381.30030751)(192.13657762,381.25031497)
\lineto(192.13657762,381.10031497)
\curveto(192.13657337,380.99030782)(192.13157338,380.88530793)(192.12157762,380.78531497)
\curveto(192.1115734,380.69530812)(192.07657343,380.62530819)(192.01657762,380.57531497)
\curveto(191.95657355,380.52530829)(191.87157364,380.49530832)(191.76157762,380.48531497)
\lineto(191.43157762,380.48531497)
\curveto(191.32157419,380.49530832)(191.2115743,380.50030831)(191.10157762,380.50031497)
\curveto(190.99157452,380.50030831)(190.89657461,380.48530833)(190.81657762,380.45531497)
\curveto(190.74657476,380.42530839)(190.69657481,380.37530844)(190.66657762,380.30531497)
\curveto(190.63657487,380.23530858)(190.61657489,380.15030866)(190.60657762,380.05031497)
\curveto(190.59657491,379.96030885)(190.59157492,379.86030895)(190.59157762,379.75031497)
\curveto(190.60157491,379.65030916)(190.6065749,379.55030926)(190.60657762,379.45031497)
\lineto(190.60657762,376.48031497)
\curveto(190.6065749,376.26031255)(190.60157491,376.02531279)(190.59157762,375.77531497)
\curveto(190.59157492,375.53531328)(190.63657487,375.35031346)(190.72657762,375.22031497)
\curveto(190.77657473,375.14031367)(190.84157467,375.08531373)(190.92157762,375.05531497)
\curveto(191.00157451,375.02531379)(191.09657441,375.00031381)(191.20657762,374.98031497)
\curveto(191.23657427,374.97031384)(191.26657424,374.96531385)(191.29657762,374.96531497)
\curveto(191.33657417,374.97531384)(191.37157414,374.97531384)(191.40157762,374.96531497)
\lineto(191.59657762,374.96531497)
\curveto(191.69657381,374.96531385)(191.78657372,374.95531386)(191.86657762,374.93531497)
\curveto(191.95657355,374.92531389)(192.02157349,374.89031392)(192.06157762,374.83031497)
\curveto(192.08157343,374.80031401)(192.09657341,374.74531407)(192.10657762,374.66531497)
\curveto(192.12657338,374.59531422)(192.13657337,374.52031429)(192.13657762,374.44031497)
\curveto(192.14657336,374.36031445)(192.14657336,374.28031453)(192.13657762,374.20031497)
\curveto(192.12657338,374.13031468)(192.1065734,374.07531474)(192.07657762,374.03531497)
\curveto(192.03657347,373.96531485)(191.96157355,373.9153149)(191.85157762,373.88531497)
\curveto(191.77157374,373.86531495)(191.68157383,373.85531496)(191.58157762,373.85531497)
\curveto(191.48157403,373.86531495)(191.39157412,373.87031494)(191.31157762,373.87031497)
\curveto(191.25157426,373.87031494)(191.19157432,373.86531495)(191.13157762,373.85531497)
\curveto(191.07157444,373.85531496)(191.01657449,373.86031495)(190.96657762,373.87031497)
\lineto(190.78657762,373.87031497)
\curveto(190.73657477,373.88031493)(190.68657482,373.88531493)(190.63657762,373.88531497)
\curveto(190.59657491,373.89531492)(190.55157496,373.90031491)(190.50157762,373.90031497)
\curveto(190.30157521,373.95031486)(190.12657538,374.00531481)(189.97657762,374.06531497)
\curveto(189.83657567,374.12531469)(189.71657579,374.23031458)(189.61657762,374.38031497)
\curveto(189.47657603,374.58031423)(189.39657611,374.83031398)(189.37657762,375.13031497)
\curveto(189.35657615,375.44031337)(189.34657616,375.77031304)(189.34657762,376.12031497)
\lineto(189.34657762,380.05031497)
\curveto(189.31657619,380.18030863)(189.28657622,380.27530854)(189.25657762,380.33531497)
\curveto(189.23657627,380.39530842)(189.16657634,380.44530837)(189.04657762,380.48531497)
\curveto(189.0065765,380.49530832)(188.96657654,380.49530832)(188.92657762,380.48531497)
\curveto(188.88657662,380.47530834)(188.84657666,380.48030833)(188.80657762,380.50031497)
\lineto(188.56657762,380.50031497)
\curveto(188.43657707,380.50030831)(188.32657718,380.5103083)(188.23657762,380.53031497)
\curveto(188.15657735,380.56030825)(188.10157741,380.62030819)(188.07157762,380.71031497)
\curveto(188.05157746,380.75030806)(188.03657747,380.79530802)(188.02657762,380.84531497)
\lineto(188.02657762,380.99531497)
\curveto(188.02657748,381.13530768)(188.03657747,381.25030756)(188.05657762,381.34031497)
\curveto(188.07657743,381.44030737)(188.13657737,381.5153073)(188.23657762,381.56531497)
\curveto(188.34657716,381.60530721)(188.48657702,381.6153072)(188.65657762,381.59531497)
\curveto(188.83657667,381.57530724)(188.98657652,381.58530723)(189.10657762,381.62531497)
\curveto(189.19657631,381.67530714)(189.26657624,381.74530707)(189.31657762,381.83531497)
\curveto(189.33657617,381.89530692)(189.34657616,381.97030684)(189.34657762,382.06031497)
\lineto(189.34657762,382.31531497)
\lineto(189.34657762,383.24531497)
\lineto(189.34657762,383.48531497)
\curveto(189.34657616,383.57530524)(189.35657615,383.65030516)(189.37657762,383.71031497)
\curveto(189.41657609,383.79030502)(189.49157602,383.85530496)(189.60157762,383.90531497)
\curveto(189.63157588,383.90530491)(189.65657585,383.90530491)(189.67657762,383.90531497)
\curveto(189.7065758,383.9153049)(189.73157578,383.92030489)(189.75157762,383.92031497)
}
}
{
\newrgbcolor{curcolor}{0 0 0}
\pscustom[linestyle=none,fillstyle=solid,fillcolor=curcolor]
{
\newpath
\moveto(200.64837449,378.05531497)
\curveto(200.66836643,377.99531082)(200.67836642,377.90031091)(200.67837449,377.77031497)
\curveto(200.67836642,377.65031116)(200.67336643,377.56531125)(200.66337449,377.51531497)
\lineto(200.66337449,377.36531497)
\curveto(200.65336645,377.28531153)(200.64336646,377.2103116)(200.63337449,377.14031497)
\curveto(200.63336647,377.08031173)(200.62836647,377.0103118)(200.61837449,376.93031497)
\curveto(200.5983665,376.87031194)(200.58336652,376.810312)(200.57337449,376.75031497)
\curveto(200.57336653,376.69031212)(200.56336654,376.63031218)(200.54337449,376.57031497)
\curveto(200.5033666,376.44031237)(200.46836663,376.3103125)(200.43837449,376.18031497)
\curveto(200.40836669,376.05031276)(200.36836673,375.93031288)(200.31837449,375.82031497)
\curveto(200.10836699,375.34031347)(199.82836727,374.93531388)(199.47837449,374.60531497)
\curveto(199.12836797,374.28531453)(198.6983684,374.04031477)(198.18837449,373.87031497)
\curveto(198.07836902,373.83031498)(197.95836914,373.80031501)(197.82837449,373.78031497)
\curveto(197.70836939,373.76031505)(197.58336952,373.74031507)(197.45337449,373.72031497)
\curveto(197.39336971,373.7103151)(197.32836977,373.70531511)(197.25837449,373.70531497)
\curveto(197.1983699,373.69531512)(197.13836996,373.69031512)(197.07837449,373.69031497)
\curveto(197.03837006,373.68031513)(196.97837012,373.67531514)(196.89837449,373.67531497)
\curveto(196.82837027,373.67531514)(196.77837032,373.68031513)(196.74837449,373.69031497)
\curveto(196.70837039,373.70031511)(196.66837043,373.70531511)(196.62837449,373.70531497)
\curveto(196.58837051,373.69531512)(196.55337055,373.69531512)(196.52337449,373.70531497)
\lineto(196.43337449,373.70531497)
\lineto(196.07337449,373.75031497)
\curveto(195.93337117,373.79031502)(195.7983713,373.83031498)(195.66837449,373.87031497)
\curveto(195.53837156,373.9103149)(195.41337169,373.95531486)(195.29337449,374.00531497)
\curveto(194.84337226,374.20531461)(194.47337263,374.46531435)(194.18337449,374.78531497)
\curveto(193.89337321,375.10531371)(193.65337345,375.49531332)(193.46337449,375.95531497)
\curveto(193.41337369,376.05531276)(193.37337373,376.15531266)(193.34337449,376.25531497)
\curveto(193.32337378,376.35531246)(193.3033738,376.46031235)(193.28337449,376.57031497)
\curveto(193.26337384,376.6103122)(193.25337385,376.64031217)(193.25337449,376.66031497)
\curveto(193.26337384,376.69031212)(193.26337384,376.72531209)(193.25337449,376.76531497)
\curveto(193.23337387,376.84531197)(193.21837388,376.92531189)(193.20837449,377.00531497)
\curveto(193.20837389,377.09531172)(193.1983739,377.18031163)(193.17837449,377.26031497)
\lineto(193.17837449,377.38031497)
\curveto(193.17837392,377.42031139)(193.17337393,377.46531135)(193.16337449,377.51531497)
\curveto(193.15337395,377.56531125)(193.14837395,377.65031116)(193.14837449,377.77031497)
\curveto(193.14837395,377.90031091)(193.15837394,377.99531082)(193.17837449,378.05531497)
\curveto(193.1983739,378.12531069)(193.2033739,378.19531062)(193.19337449,378.26531497)
\curveto(193.18337392,378.33531048)(193.18837391,378.40531041)(193.20837449,378.47531497)
\curveto(193.21837388,378.52531029)(193.22337388,378.56531025)(193.22337449,378.59531497)
\curveto(193.23337387,378.63531018)(193.24337386,378.68031013)(193.25337449,378.73031497)
\curveto(193.28337382,378.85030996)(193.30837379,378.97030984)(193.32837449,379.09031497)
\curveto(193.35837374,379.2103096)(193.3983737,379.32530949)(193.44837449,379.43531497)
\curveto(193.5983735,379.80530901)(193.77837332,380.13530868)(193.98837449,380.42531497)
\curveto(194.20837289,380.72530809)(194.47337263,380.97530784)(194.78337449,381.17531497)
\curveto(194.9033722,381.25530756)(195.02837207,381.32030749)(195.15837449,381.37031497)
\curveto(195.28837181,381.43030738)(195.42337168,381.49030732)(195.56337449,381.55031497)
\curveto(195.68337142,381.60030721)(195.81337129,381.63030718)(195.95337449,381.64031497)
\curveto(196.09337101,381.66030715)(196.23337087,381.69030712)(196.37337449,381.73031497)
\lineto(196.56837449,381.73031497)
\curveto(196.63837046,381.74030707)(196.7033704,381.75030706)(196.76337449,381.76031497)
\curveto(197.65336945,381.77030704)(198.39336871,381.58530723)(198.98337449,381.20531497)
\curveto(199.57336753,380.82530799)(199.9983671,380.33030848)(200.25837449,379.72031497)
\curveto(200.30836679,379.62030919)(200.34836675,379.52030929)(200.37837449,379.42031497)
\curveto(200.40836669,379.32030949)(200.44336666,379.2153096)(200.48337449,379.10531497)
\curveto(200.51336659,378.99530982)(200.53836656,378.87530994)(200.55837449,378.74531497)
\curveto(200.57836652,378.62531019)(200.6033665,378.50031031)(200.63337449,378.37031497)
\curveto(200.64336646,378.32031049)(200.64336646,378.26531055)(200.63337449,378.20531497)
\curveto(200.63336647,378.15531066)(200.63836646,378.10531071)(200.64837449,378.05531497)
\moveto(199.31337449,377.20031497)
\curveto(199.33336777,377.27031154)(199.33836776,377.35031146)(199.32837449,377.44031497)
\lineto(199.32837449,377.69531497)
\curveto(199.32836777,378.08531073)(199.29336781,378.4153104)(199.22337449,378.68531497)
\curveto(199.19336791,378.76531005)(199.16836793,378.84530997)(199.14837449,378.92531497)
\curveto(199.12836797,379.00530981)(199.103368,379.08030973)(199.07337449,379.15031497)
\curveto(198.79336831,379.80030901)(198.34836875,380.25030856)(197.73837449,380.50031497)
\curveto(197.66836943,380.53030828)(197.59336951,380.55030826)(197.51337449,380.56031497)
\lineto(197.27337449,380.62031497)
\curveto(197.19336991,380.64030817)(197.10836999,380.65030816)(197.01837449,380.65031497)
\lineto(196.74837449,380.65031497)
\lineto(196.47837449,380.60531497)
\curveto(196.37837072,380.58530823)(196.28337082,380.56030825)(196.19337449,380.53031497)
\curveto(196.11337099,380.5103083)(196.03337107,380.48030833)(195.95337449,380.44031497)
\curveto(195.88337122,380.42030839)(195.81837128,380.39030842)(195.75837449,380.35031497)
\curveto(195.6983714,380.3103085)(195.64337146,380.27030854)(195.59337449,380.23031497)
\curveto(195.35337175,380.06030875)(195.15837194,379.85530896)(195.00837449,379.61531497)
\curveto(194.85837224,379.37530944)(194.72837237,379.09530972)(194.61837449,378.77531497)
\curveto(194.58837251,378.67531014)(194.56837253,378.57031024)(194.55837449,378.46031497)
\curveto(194.54837255,378.36031045)(194.53337257,378.25531056)(194.51337449,378.14531497)
\curveto(194.5033726,378.10531071)(194.4983726,378.04031077)(194.49837449,377.95031497)
\curveto(194.48837261,377.92031089)(194.48337262,377.88531093)(194.48337449,377.84531497)
\curveto(194.49337261,377.80531101)(194.4983726,377.76031105)(194.49837449,377.71031497)
\lineto(194.49837449,377.41031497)
\curveto(194.4983726,377.3103115)(194.50837259,377.22031159)(194.52837449,377.14031497)
\lineto(194.55837449,376.96031497)
\curveto(194.57837252,376.86031195)(194.59337251,376.76031205)(194.60337449,376.66031497)
\curveto(194.62337248,376.57031224)(194.65337245,376.48531233)(194.69337449,376.40531497)
\curveto(194.79337231,376.16531265)(194.90837219,375.94031287)(195.03837449,375.73031497)
\curveto(195.17837192,375.52031329)(195.34837175,375.34531347)(195.54837449,375.20531497)
\curveto(195.5983715,375.17531364)(195.64337146,375.15031366)(195.68337449,375.13031497)
\curveto(195.72337138,375.1103137)(195.76837133,375.08531373)(195.81837449,375.05531497)
\curveto(195.8983712,375.00531381)(195.98337112,374.96031385)(196.07337449,374.92031497)
\curveto(196.17337093,374.89031392)(196.27837082,374.86031395)(196.38837449,374.83031497)
\curveto(196.43837066,374.810314)(196.48337062,374.80031401)(196.52337449,374.80031497)
\curveto(196.57337053,374.810314)(196.62337048,374.810314)(196.67337449,374.80031497)
\curveto(196.7033704,374.79031402)(196.76337034,374.78031403)(196.85337449,374.77031497)
\curveto(196.95337015,374.76031405)(197.02837007,374.76531405)(197.07837449,374.78531497)
\curveto(197.11836998,374.79531402)(197.15836994,374.79531402)(197.19837449,374.78531497)
\curveto(197.23836986,374.78531403)(197.27836982,374.79531402)(197.31837449,374.81531497)
\curveto(197.3983697,374.83531398)(197.47836962,374.85031396)(197.55837449,374.86031497)
\curveto(197.63836946,374.88031393)(197.71336939,374.90531391)(197.78337449,374.93531497)
\curveto(198.12336898,375.07531374)(198.3983687,375.27031354)(198.60837449,375.52031497)
\curveto(198.81836828,375.77031304)(198.99336811,376.06531275)(199.13337449,376.40531497)
\curveto(199.18336792,376.52531229)(199.21336789,376.65031216)(199.22337449,376.78031497)
\curveto(199.24336786,376.92031189)(199.27336783,377.06031175)(199.31337449,377.20031497)
}
}
{
\newrgbcolor{curcolor}{0 0 0}
\pscustom[linestyle=none,fillstyle=solid,fillcolor=curcolor]
{
\newpath
\moveto(204.56665574,381.76031497)
\curveto(205.28665168,381.77030704)(205.89165107,381.68530713)(206.38165574,381.50531497)
\curveto(206.87165009,381.33530748)(207.25164971,381.03030778)(207.52165574,380.59031497)
\curveto(207.59164937,380.48030833)(207.64664932,380.36530845)(207.68665574,380.24531497)
\curveto(207.72664924,380.13530868)(207.7666492,380.0103088)(207.80665574,379.87031497)
\curveto(207.82664914,379.80030901)(207.83164913,379.72530909)(207.82165574,379.64531497)
\curveto(207.81164915,379.57530924)(207.79664917,379.52030929)(207.77665574,379.48031497)
\curveto(207.75664921,379.46030935)(207.73164923,379.44030937)(207.70165574,379.42031497)
\curveto(207.67164929,379.4103094)(207.64664932,379.39530942)(207.62665574,379.37531497)
\curveto(207.57664939,379.35530946)(207.52664944,379.35030946)(207.47665574,379.36031497)
\curveto(207.42664954,379.37030944)(207.37664959,379.37030944)(207.32665574,379.36031497)
\curveto(207.24664972,379.34030947)(207.14164982,379.33530948)(207.01165574,379.34531497)
\curveto(206.88165008,379.36530945)(206.79165017,379.39030942)(206.74165574,379.42031497)
\curveto(206.6616503,379.47030934)(206.60665036,379.53530928)(206.57665574,379.61531497)
\curveto(206.55665041,379.70530911)(206.52165044,379.79030902)(206.47165574,379.87031497)
\curveto(206.38165058,380.03030878)(206.25665071,380.17530864)(206.09665574,380.30531497)
\curveto(205.98665098,380.38530843)(205.8666511,380.44530837)(205.73665574,380.48531497)
\curveto(205.60665136,380.52530829)(205.4666515,380.56530825)(205.31665574,380.60531497)
\curveto(205.2666517,380.62530819)(205.21665175,380.63030818)(205.16665574,380.62031497)
\curveto(205.11665185,380.62030819)(205.0666519,380.62530819)(205.01665574,380.63531497)
\curveto(204.95665201,380.65530816)(204.88165208,380.66530815)(204.79165574,380.66531497)
\curveto(204.70165226,380.66530815)(204.62665234,380.65530816)(204.56665574,380.63531497)
\lineto(204.47665574,380.63531497)
\lineto(204.32665574,380.60531497)
\curveto(204.27665269,380.60530821)(204.22665274,380.60030821)(204.17665574,380.59031497)
\curveto(203.91665305,380.53030828)(203.70165326,380.44530837)(203.53165574,380.33531497)
\curveto(203.3616536,380.22530859)(203.24665372,380.04030877)(203.18665574,379.78031497)
\curveto(203.1666538,379.7103091)(203.1616538,379.64030917)(203.17165574,379.57031497)
\curveto(203.19165377,379.50030931)(203.21165375,379.44030937)(203.23165574,379.39031497)
\curveto(203.29165367,379.24030957)(203.3616536,379.13030968)(203.44165574,379.06031497)
\curveto(203.53165343,379.00030981)(203.64165332,378.93030988)(203.77165574,378.85031497)
\curveto(203.93165303,378.75031006)(204.11165285,378.67531014)(204.31165574,378.62531497)
\curveto(204.51165245,378.58531023)(204.71165225,378.53531028)(204.91165574,378.47531497)
\curveto(205.04165192,378.43531038)(205.17165179,378.40531041)(205.30165574,378.38531497)
\curveto(205.43165153,378.36531045)(205.5616514,378.33531048)(205.69165574,378.29531497)
\curveto(205.90165106,378.23531058)(206.10665086,378.17531064)(206.30665574,378.11531497)
\curveto(206.50665046,378.06531075)(206.70665026,378.00031081)(206.90665574,377.92031497)
\lineto(207.05665574,377.86031497)
\curveto(207.10664986,377.84031097)(207.15664981,377.815311)(207.20665574,377.78531497)
\curveto(207.40664956,377.66531115)(207.58164938,377.53031128)(207.73165574,377.38031497)
\curveto(207.88164908,377.23031158)(208.00664896,377.04031177)(208.10665574,376.81031497)
\curveto(208.12664884,376.74031207)(208.14664882,376.64531217)(208.16665574,376.52531497)
\curveto(208.18664878,376.45531236)(208.19664877,376.38031243)(208.19665574,376.30031497)
\curveto(208.20664876,376.23031258)(208.21164875,376.15031266)(208.21165574,376.06031497)
\lineto(208.21165574,375.91031497)
\curveto(208.19164877,375.84031297)(208.18164878,375.77031304)(208.18165574,375.70031497)
\curveto(208.18164878,375.63031318)(208.17164879,375.56031325)(208.15165574,375.49031497)
\curveto(208.12164884,375.38031343)(208.08664888,375.27531354)(208.04665574,375.17531497)
\curveto(208.00664896,375.07531374)(207.961649,374.98531383)(207.91165574,374.90531497)
\curveto(207.75164921,374.64531417)(207.54664942,374.43531438)(207.29665574,374.27531497)
\curveto(207.04664992,374.12531469)(206.7666502,373.99531482)(206.45665574,373.88531497)
\curveto(206.3666506,373.85531496)(206.27165069,373.83531498)(206.17165574,373.82531497)
\curveto(206.08165088,373.80531501)(205.99165097,373.78031503)(205.90165574,373.75031497)
\curveto(205.80165116,373.73031508)(205.70165126,373.72031509)(205.60165574,373.72031497)
\curveto(205.50165146,373.72031509)(205.40165156,373.7103151)(205.30165574,373.69031497)
\lineto(205.15165574,373.69031497)
\curveto(205.10165186,373.68031513)(205.03165193,373.67531514)(204.94165574,373.67531497)
\curveto(204.85165211,373.67531514)(204.78165218,373.68031513)(204.73165574,373.69031497)
\lineto(204.56665574,373.69031497)
\curveto(204.50665246,373.7103151)(204.44165252,373.72031509)(204.37165574,373.72031497)
\curveto(204.30165266,373.7103151)(204.24165272,373.7153151)(204.19165574,373.73531497)
\curveto(204.14165282,373.74531507)(204.07665289,373.75031506)(203.99665574,373.75031497)
\lineto(203.75665574,373.81031497)
\curveto(203.68665328,373.82031499)(203.61165335,373.84031497)(203.53165574,373.87031497)
\curveto(203.22165374,373.97031484)(202.95165401,374.09531472)(202.72165574,374.24531497)
\curveto(202.49165447,374.39531442)(202.29165467,374.59031422)(202.12165574,374.83031497)
\curveto(202.03165493,374.96031385)(201.95665501,375.09531372)(201.89665574,375.23531497)
\curveto(201.83665513,375.37531344)(201.78165518,375.53031328)(201.73165574,375.70031497)
\curveto(201.71165525,375.76031305)(201.70165526,375.83031298)(201.70165574,375.91031497)
\curveto(201.71165525,376.00031281)(201.72665524,376.07031274)(201.74665574,376.12031497)
\curveto(201.77665519,376.16031265)(201.82665514,376.20031261)(201.89665574,376.24031497)
\curveto(201.94665502,376.26031255)(202.01665495,376.27031254)(202.10665574,376.27031497)
\curveto(202.19665477,376.28031253)(202.28665468,376.28031253)(202.37665574,376.27031497)
\curveto(202.4666545,376.26031255)(202.55165441,376.24531257)(202.63165574,376.22531497)
\curveto(202.72165424,376.2153126)(202.78165418,376.20031261)(202.81165574,376.18031497)
\curveto(202.88165408,376.13031268)(202.92665404,376.05531276)(202.94665574,375.95531497)
\curveto(202.97665399,375.86531295)(203.01165395,375.78031303)(203.05165574,375.70031497)
\curveto(203.15165381,375.48031333)(203.28665368,375.3103135)(203.45665574,375.19031497)
\curveto(203.57665339,375.10031371)(203.71165325,375.03031378)(203.86165574,374.98031497)
\curveto(204.01165295,374.93031388)(204.17165279,374.88031393)(204.34165574,374.83031497)
\lineto(204.65665574,374.78531497)
\lineto(204.74665574,374.78531497)
\curveto(204.81665215,374.76531405)(204.90665206,374.75531406)(205.01665574,374.75531497)
\curveto(205.13665183,374.75531406)(205.23665173,374.76531405)(205.31665574,374.78531497)
\curveto(205.38665158,374.78531403)(205.44165152,374.79031402)(205.48165574,374.80031497)
\curveto(205.54165142,374.810314)(205.60165136,374.815314)(205.66165574,374.81531497)
\curveto(205.72165124,374.82531399)(205.77665119,374.83531398)(205.82665574,374.84531497)
\curveto(206.11665085,374.92531389)(206.34665062,375.03031378)(206.51665574,375.16031497)
\curveto(206.68665028,375.29031352)(206.80665016,375.5103133)(206.87665574,375.82031497)
\curveto(206.89665007,375.87031294)(206.90165006,375.92531289)(206.89165574,375.98531497)
\curveto(206.88165008,376.04531277)(206.87165009,376.09031272)(206.86165574,376.12031497)
\curveto(206.81165015,376.3103125)(206.74165022,376.45031236)(206.65165574,376.54031497)
\curveto(206.5616504,376.64031217)(206.44665052,376.73031208)(206.30665574,376.81031497)
\curveto(206.21665075,376.87031194)(206.11665085,376.92031189)(206.00665574,376.96031497)
\lineto(205.67665574,377.08031497)
\curveto(205.64665132,377.09031172)(205.61665135,377.09531172)(205.58665574,377.09531497)
\curveto(205.5666514,377.09531172)(205.54165142,377.10531171)(205.51165574,377.12531497)
\curveto(205.17165179,377.23531158)(204.81665215,377.3153115)(204.44665574,377.36531497)
\curveto(204.08665288,377.42531139)(203.74665322,377.52031129)(203.42665574,377.65031497)
\curveto(203.32665364,377.69031112)(203.23165373,377.72531109)(203.14165574,377.75531497)
\curveto(203.05165391,377.78531103)(202.966654,377.82531099)(202.88665574,377.87531497)
\curveto(202.69665427,377.98531083)(202.52165444,378.1103107)(202.36165574,378.25031497)
\curveto(202.20165476,378.39031042)(202.07665489,378.56531025)(201.98665574,378.77531497)
\curveto(201.95665501,378.84530997)(201.93165503,378.9153099)(201.91165574,378.98531497)
\curveto(201.90165506,379.05530976)(201.88665508,379.13030968)(201.86665574,379.21031497)
\curveto(201.83665513,379.33030948)(201.82665514,379.46530935)(201.83665574,379.61531497)
\curveto(201.84665512,379.77530904)(201.8616551,379.9103089)(201.88165574,380.02031497)
\curveto(201.90165506,380.07030874)(201.91165505,380.1103087)(201.91165574,380.14031497)
\curveto(201.92165504,380.18030863)(201.93665503,380.22030859)(201.95665574,380.26031497)
\curveto(202.04665492,380.49030832)(202.1666548,380.69030812)(202.31665574,380.86031497)
\curveto(202.47665449,381.03030778)(202.65665431,381.18030763)(202.85665574,381.31031497)
\curveto(203.00665396,381.40030741)(203.17165379,381.47030734)(203.35165574,381.52031497)
\curveto(203.53165343,381.58030723)(203.72165324,381.63530718)(203.92165574,381.68531497)
\curveto(203.99165297,381.69530712)(204.05665291,381.70530711)(204.11665574,381.71531497)
\curveto(204.18665278,381.72530709)(204.2616527,381.73530708)(204.34165574,381.74531497)
\curveto(204.37165259,381.75530706)(204.41165255,381.75530706)(204.46165574,381.74531497)
\curveto(204.51165245,381.73530708)(204.54665242,381.74030707)(204.56665574,381.76031497)
}
}
{
\newrgbcolor{curcolor}{0.7019608 0.7019608 0.7019608}
\pscustom[linestyle=none,fillstyle=solid,fillcolor=curcolor]
{
\newpath
\moveto(120.84556474,387.06538211)
\lineto(135.84556474,387.06538211)
\lineto(135.84556474,372.06538211)
\lineto(120.84556474,372.06538211)
\closepath
}
}
{
\newrgbcolor{curcolor}{0 0 0}
\pscustom[linestyle=none,fillstyle=solid,fillcolor=curcolor]
{
\newpath
\moveto(148.82876512,351.37313724)
\curveto(148.8787555,351.24313698)(148.85875552,351.14313708)(148.76876512,351.07313724)
\curveto(148.71875566,351.04313718)(148.65375572,351.0231372)(148.57376512,351.01313724)
\lineto(148.34876512,351.01313724)
\lineto(147.86876512,351.01313724)
\curveto(147.70875667,351.01313721)(147.58375679,351.04813717)(147.49376512,351.11813724)
\curveto(147.41375696,351.16813705)(147.35875702,351.24313698)(147.32876512,351.34313724)
\lineto(147.26876512,351.67313724)
\curveto(147.25875712,351.71313651)(147.25375712,351.74813647)(147.25376512,351.77813724)
\lineto(147.25376512,351.88313724)
\curveto(147.23375714,351.93313629)(147.22875715,351.97813624)(147.23876512,352.01813724)
\curveto(147.24875713,352.05813616)(147.24875713,352.09813612)(147.23876512,352.13813724)
\curveto(147.22875715,352.19813602)(147.22375715,352.25813596)(147.22376512,352.31813724)
\lineto(147.22376512,352.49813724)
\lineto(147.17876512,353.17313724)
\curveto(147.15875722,353.24313498)(147.14875723,353.31313491)(147.14876512,353.38313724)
\curveto(147.14875723,353.45313477)(147.13875724,353.52813469)(147.11876512,353.60813724)
\curveto(147.06875731,353.78813443)(147.02875735,353.96813425)(146.99876512,354.14813724)
\curveto(146.9787574,354.32813389)(146.93375744,354.49813372)(146.86376512,354.65813724)
\curveto(146.6737577,355.07813314)(146.35875802,355.35813286)(145.91876512,355.49813724)
\curveto(145.78875859,355.54813267)(145.64375873,355.57313265)(145.48376512,355.57313724)
\curveto(145.33375904,355.58313264)(145.1737592,355.58813263)(145.00376512,355.58813724)
\lineto(142.24376512,355.58813724)
\curveto(142.1737622,355.56813265)(142.10876227,355.54813267)(142.04876512,355.52813724)
\curveto(141.99876238,355.5181327)(141.95376242,355.48813273)(141.91376512,355.43813724)
\curveto(141.84376253,355.33813288)(141.80876257,355.17313305)(141.80876512,354.94313724)
\curveto(141.81876256,354.7231335)(141.82376255,354.52813369)(141.82376512,354.35813724)
\lineto(141.82376512,352.18313724)
\curveto(141.82376255,352.04313618)(141.82876255,351.86813635)(141.83876512,351.65813724)
\curveto(141.84876253,351.45813676)(141.82876255,351.30813691)(141.77876512,351.20813724)
\curveto(141.75876262,351.13813708)(141.71876266,351.09313713)(141.65876512,351.07313724)
\curveto(141.61876276,351.05313717)(141.5787628,351.04313718)(141.53876512,351.04313724)
\curveto(141.50876287,351.04313718)(141.46876291,351.03313719)(141.41876512,351.01313724)
\curveto(141.378763,351.00313722)(141.33376304,350.99813722)(141.28376512,350.99813724)
\curveto(141.23376314,351.00813721)(141.18376319,351.01313721)(141.13376512,351.01313724)
\lineto(140.80376512,351.01313724)
\curveto(140.70376367,351.0231372)(140.61876376,351.05313717)(140.54876512,351.10313724)
\curveto(140.46876391,351.15313707)(140.42876395,351.24313698)(140.42876512,351.37313724)
\lineto(140.42876512,351.77813724)
\lineto(140.42876512,360.89813724)
\curveto(140.42876395,361.00812721)(140.42376395,361.1231271)(140.41376512,361.24313724)
\curveto(140.41376396,361.36312686)(140.43876394,361.45812676)(140.48876512,361.52813724)
\curveto(140.52876385,361.58812663)(140.60376377,361.63812658)(140.71376512,361.67813724)
\curveto(140.73376364,361.68812653)(140.75376362,361.68812653)(140.77376512,361.67813724)
\curveto(140.79376358,361.67812654)(140.81376356,361.68312654)(140.83376512,361.69313724)
\lineto(145.18376512,361.69313724)
\curveto(145.25375912,361.69312653)(145.32875905,361.69312653)(145.40876512,361.69313724)
\curveto(145.48875889,361.70312652)(145.55875882,361.70312652)(145.61876512,361.69313724)
\lineto(145.78376512,361.69313724)
\curveto(145.84375853,361.68312654)(145.90375847,361.67312655)(145.96376512,361.66313724)
\curveto(146.02375835,361.66312656)(146.08875829,361.65812656)(146.15876512,361.64813724)
\curveto(146.23875814,361.62812659)(146.31875806,361.61312661)(146.39876512,361.60313724)
\curveto(146.48875789,361.59312663)(146.5737578,361.57812664)(146.65376512,361.55813724)
\curveto(146.84375753,361.49812672)(147.01875736,361.43312679)(147.17876512,361.36313724)
\curveto(147.33875704,361.29312693)(147.48875689,361.20812701)(147.62876512,361.10813724)
\curveto(147.8787565,360.93812728)(148.0787563,360.72812749)(148.22876512,360.47813724)
\curveto(148.38875599,360.23812798)(148.51875586,359.95312827)(148.61876512,359.62313724)
\curveto(148.63875574,359.54312868)(148.64875573,359.45812876)(148.64876512,359.36813724)
\curveto(148.65875572,359.28812893)(148.6737557,359.20812901)(148.69376512,359.12813724)
\lineto(148.69376512,358.97813724)
\curveto(148.70375567,358.92812929)(148.70375567,358.86812935)(148.69376512,358.79813724)
\curveto(148.69375568,358.73812948)(148.68875569,358.68312954)(148.67876512,358.63313724)
\lineto(148.67876512,358.46813724)
\curveto(148.65875572,358.38812983)(148.64375573,358.31312991)(148.63376512,358.24313724)
\curveto(148.63375574,358.17313005)(148.62375575,358.10313012)(148.60376512,358.03313724)
\curveto(148.55375582,357.88313034)(148.50375587,357.73813048)(148.45376512,357.59813724)
\curveto(148.41375596,357.46813075)(148.35375602,357.34313088)(148.27376512,357.22313724)
\curveto(148.24375613,357.17313105)(148.20875617,357.12813109)(148.16876512,357.08813724)
\curveto(148.13875624,357.04813117)(148.10875627,357.00313122)(148.07876512,356.95313724)
\lineto(148.04876512,356.92313724)
\curveto(148.03875634,356.9231313)(148.02875635,356.9181313)(148.01876512,356.90813724)
\lineto(147.94376512,356.83313724)
\curveto(147.92375645,356.80313142)(147.90375647,356.77813144)(147.88376512,356.75813724)
\curveto(147.80375657,356.69813152)(147.72875665,356.63813158)(147.65876512,356.57813724)
\curveto(147.58875679,356.52813169)(147.51375686,356.47813174)(147.43376512,356.42813724)
\curveto(147.38375699,356.39813182)(147.33875704,356.36313186)(147.29876512,356.32313724)
\curveto(147.25875712,356.29313193)(147.23375714,356.24813197)(147.22376512,356.18813724)
\curveto(147.21375716,356.12813209)(147.23375714,356.07813214)(147.28376512,356.03813724)
\curveto(147.34375703,355.99813222)(147.39375698,355.96813225)(147.43376512,355.94813724)
\curveto(147.54375683,355.87813234)(147.64375673,355.80313242)(147.73376512,355.72313724)
\curveto(147.83375654,355.64313258)(147.91875646,355.54813267)(147.98876512,355.43813724)
\curveto(148.09875628,355.29813292)(148.1787562,355.13813308)(148.22876512,354.95813724)
\curveto(148.2787561,354.78813343)(148.32875605,354.60313362)(148.37876512,354.40313724)
\lineto(148.40876512,354.16313724)
\curveto(148.41875596,354.09313413)(148.42875595,354.0181342)(148.43876512,353.93813724)
\curveto(148.45875592,353.86813435)(148.46375591,353.79813442)(148.45376512,353.72813724)
\curveto(148.44375593,353.65813456)(148.44875593,353.58813463)(148.46876512,353.51813724)
\lineto(148.46876512,353.38313724)
\curveto(148.48875589,353.31313491)(148.49375588,353.23813498)(148.48376512,353.15813724)
\curveto(148.4737559,353.07813514)(148.4787559,352.99813522)(148.49876512,352.91813724)
\curveto(148.50875587,352.87813534)(148.50875587,352.83813538)(148.49876512,352.79813724)
\curveto(148.49875588,352.76813545)(148.50375587,352.72813549)(148.51376512,352.67813724)
\curveto(148.53375584,352.57813564)(148.54875583,352.47313575)(148.55876512,352.36313724)
\curveto(148.56875581,352.26313596)(148.58875579,352.16813605)(148.61876512,352.07813724)
\curveto(148.63875574,352.0181362)(148.64875573,351.95813626)(148.64876512,351.89813724)
\curveto(148.65875572,351.84813637)(148.6737557,351.79313643)(148.69376512,351.73313724)
\lineto(148.82876512,351.37313724)
\moveto(147.01376512,357.59813724)
\curveto(147.08375729,357.70813051)(147.13375724,357.8231304)(147.16376512,357.94313724)
\curveto(147.20375717,358.06313016)(147.23875714,358.19313003)(147.26876512,358.33313724)
\lineto(147.26876512,358.46813724)
\curveto(147.29875708,358.60812961)(147.30375707,358.75812946)(147.28376512,358.91813724)
\curveto(147.26375711,359.08812913)(147.23375714,359.22812899)(147.19376512,359.33813724)
\curveto(147.03375734,359.83812838)(146.71875766,360.18312804)(146.24876512,360.37313724)
\curveto(146.04875833,360.45312777)(145.81375856,360.49812772)(145.54376512,360.50813724)
\curveto(145.28375909,360.5181277)(145.01375936,360.5231277)(144.73376512,360.52313724)
\lineto(142.25876512,360.52313724)
\curveto(142.23876214,360.51312771)(142.21376216,360.50812771)(142.18376512,360.50813724)
\curveto(142.16376221,360.50812771)(142.13876224,360.50312772)(142.10876512,360.49313724)
\curveto(141.98876239,360.46312776)(141.90876247,360.39812782)(141.86876512,360.29813724)
\curveto(141.82876255,360.20812801)(141.80876257,360.08312814)(141.80876512,359.92313724)
\curveto(141.81876256,359.76312846)(141.82376255,359.6181286)(141.82376512,359.48813724)
\lineto(141.82376512,357.76313724)
\curveto(141.82376255,357.61313061)(141.81876256,357.45313077)(141.80876512,357.28313724)
\curveto(141.80876257,357.1231311)(141.84376253,356.99813122)(141.91376512,356.90813724)
\curveto(141.96376241,356.83813138)(142.03876234,356.79313143)(142.13876512,356.77313724)
\curveto(142.23876214,356.76313146)(142.34876203,356.75813146)(142.46876512,356.75813724)
\lineto(143.39876512,356.75813724)
\curveto(143.78876059,356.75813146)(144.16876021,356.75313147)(144.53876512,356.74313724)
\curveto(144.90875947,356.74313148)(145.24875913,356.76313146)(145.55876512,356.80313724)
\curveto(145.8787585,356.85313137)(146.16375821,356.93813128)(146.41376512,357.05813724)
\curveto(146.66375771,357.17813104)(146.86375751,357.35813086)(147.01376512,357.59813724)
}
}
{
\newrgbcolor{curcolor}{0 0 0}
\pscustom[linestyle=none,fillstyle=solid,fillcolor=curcolor]
{
\newpath
\moveto(157.20696824,355.16813724)
\curveto(157.22696056,355.06813315)(157.22696056,354.95313327)(157.20696824,354.82313724)
\curveto(157.19696059,354.70313352)(157.16696062,354.6181336)(157.11696824,354.56813724)
\curveto(157.06696072,354.52813369)(156.99196079,354.49813372)(156.89196824,354.47813724)
\curveto(156.80196098,354.46813375)(156.69696109,354.46313376)(156.57696824,354.46313724)
\lineto(156.21696824,354.46313724)
\curveto(156.09696169,354.47313375)(155.99196179,354.47813374)(155.90196824,354.47813724)
\lineto(152.06196824,354.47813724)
\curveto(151.9819658,354.47813374)(151.90196588,354.47313375)(151.82196824,354.46313724)
\curveto(151.74196604,354.46313376)(151.67696611,354.44813377)(151.62696824,354.41813724)
\curveto(151.5869662,354.39813382)(151.54696624,354.35813386)(151.50696824,354.29813724)
\curveto(151.4869663,354.26813395)(151.46696632,354.223134)(151.44696824,354.16313724)
\curveto(151.42696636,354.11313411)(151.42696636,354.06313416)(151.44696824,354.01313724)
\curveto(151.45696633,353.96313426)(151.46196632,353.9181343)(151.46196824,353.87813724)
\curveto(151.46196632,353.83813438)(151.46696632,353.79813442)(151.47696824,353.75813724)
\curveto(151.49696629,353.67813454)(151.51696627,353.59313463)(151.53696824,353.50313724)
\curveto(151.55696623,353.4231348)(151.5869662,353.34313488)(151.62696824,353.26313724)
\curveto(151.85696593,352.7231355)(152.23696555,352.33813588)(152.76696824,352.10813724)
\curveto(152.82696496,352.07813614)(152.89196489,352.05313617)(152.96196824,352.03313724)
\lineto(153.17196824,351.97313724)
\curveto(153.20196458,351.96313626)(153.25196453,351.95813626)(153.32196824,351.95813724)
\curveto(153.46196432,351.9181363)(153.64696414,351.89813632)(153.87696824,351.89813724)
\curveto(154.10696368,351.89813632)(154.29196349,351.9181363)(154.43196824,351.95813724)
\curveto(154.57196321,351.99813622)(154.69696309,352.03813618)(154.80696824,352.07813724)
\curveto(154.92696286,352.12813609)(155.03696275,352.18813603)(155.13696824,352.25813724)
\curveto(155.24696254,352.32813589)(155.34196244,352.40813581)(155.42196824,352.49813724)
\curveto(155.50196228,352.59813562)(155.57196221,352.70313552)(155.63196824,352.81313724)
\curveto(155.69196209,352.91313531)(155.74196204,353.0181352)(155.78196824,353.12813724)
\curveto(155.83196195,353.23813498)(155.91196187,353.3181349)(156.02196824,353.36813724)
\curveto(156.06196172,353.38813483)(156.12696166,353.40313482)(156.21696824,353.41313724)
\curveto(156.30696148,353.4231348)(156.39696139,353.4231348)(156.48696824,353.41313724)
\curveto(156.57696121,353.41313481)(156.66196112,353.40813481)(156.74196824,353.39813724)
\curveto(156.82196096,353.38813483)(156.87696091,353.36813485)(156.90696824,353.33813724)
\curveto(157.00696078,353.26813495)(157.03196075,353.15313507)(156.98196824,352.99313724)
\curveto(156.90196088,352.7231355)(156.79696099,352.48313574)(156.66696824,352.27313724)
\curveto(156.46696132,351.95313627)(156.23696155,351.68813653)(155.97696824,351.47813724)
\curveto(155.72696206,351.27813694)(155.40696238,351.11313711)(155.01696824,350.98313724)
\curveto(154.91696287,350.94313728)(154.81696297,350.9181373)(154.71696824,350.90813724)
\curveto(154.61696317,350.88813733)(154.51196327,350.86813735)(154.40196824,350.84813724)
\curveto(154.35196343,350.83813738)(154.30196348,350.83313739)(154.25196824,350.83313724)
\curveto(154.21196357,350.83313739)(154.16696362,350.82813739)(154.11696824,350.81813724)
\lineto(153.96696824,350.81813724)
\curveto(153.91696387,350.80813741)(153.85696393,350.80313742)(153.78696824,350.80313724)
\curveto(153.72696406,350.80313742)(153.67696411,350.80813741)(153.63696824,350.81813724)
\lineto(153.50196824,350.81813724)
\curveto(153.45196433,350.82813739)(153.40696438,350.83313739)(153.36696824,350.83313724)
\curveto(153.32696446,350.83313739)(153.2869645,350.83813738)(153.24696824,350.84813724)
\curveto(153.19696459,350.85813736)(153.14196464,350.86813735)(153.08196824,350.87813724)
\curveto(153.02196476,350.87813734)(152.96696482,350.88313734)(152.91696824,350.89313724)
\curveto(152.82696496,350.91313731)(152.73696505,350.93813728)(152.64696824,350.96813724)
\curveto(152.55696523,350.98813723)(152.47196531,351.01313721)(152.39196824,351.04313724)
\curveto(152.35196543,351.06313716)(152.31696547,351.07313715)(152.28696824,351.07313724)
\curveto(152.25696553,351.08313714)(152.22196556,351.09813712)(152.18196824,351.11813724)
\curveto(152.03196575,351.18813703)(151.87196591,351.27313695)(151.70196824,351.37313724)
\curveto(151.41196637,351.56313666)(151.16196662,351.79313643)(150.95196824,352.06313724)
\curveto(150.75196703,352.34313588)(150.5819672,352.65313557)(150.44196824,352.99313724)
\curveto(150.39196739,353.10313512)(150.35196743,353.218135)(150.32196824,353.33813724)
\curveto(150.30196748,353.45813476)(150.27196751,353.57813464)(150.23196824,353.69813724)
\curveto(150.22196756,353.73813448)(150.21696757,353.77313445)(150.21696824,353.80313724)
\curveto(150.21696757,353.83313439)(150.21196757,353.87313435)(150.20196824,353.92313724)
\curveto(150.1819676,354.00313422)(150.16696762,354.08813413)(150.15696824,354.17813724)
\curveto(150.14696764,354.26813395)(150.13196765,354.35813386)(150.11196824,354.44813724)
\lineto(150.11196824,354.65813724)
\curveto(150.10196768,354.69813352)(150.09196769,354.75313347)(150.08196824,354.82313724)
\curveto(150.0819677,354.90313332)(150.0869677,354.96813325)(150.09696824,355.01813724)
\lineto(150.09696824,355.18313724)
\curveto(150.11696767,355.23313299)(150.12196766,355.28313294)(150.11196824,355.33313724)
\curveto(150.11196767,355.39313283)(150.11696767,355.44813277)(150.12696824,355.49813724)
\curveto(150.16696762,355.65813256)(150.19696759,355.8181324)(150.21696824,355.97813724)
\curveto(150.24696754,356.13813208)(150.29196749,356.28813193)(150.35196824,356.42813724)
\curveto(150.40196738,356.53813168)(150.44696734,356.64813157)(150.48696824,356.75813724)
\curveto(150.53696725,356.87813134)(150.59196719,356.99313123)(150.65196824,357.10313724)
\curveto(150.87196691,357.45313077)(151.12196666,357.75313047)(151.40196824,358.00313724)
\curveto(151.6819661,358.26312996)(152.02696576,358.47812974)(152.43696824,358.64813724)
\curveto(152.55696523,358.69812952)(152.67696511,358.73312949)(152.79696824,358.75313724)
\curveto(152.92696486,358.78312944)(153.06196472,358.81312941)(153.20196824,358.84313724)
\curveto(153.25196453,358.85312937)(153.29696449,358.85812936)(153.33696824,358.85813724)
\curveto(153.37696441,358.86812935)(153.42196436,358.87312935)(153.47196824,358.87313724)
\curveto(153.49196429,358.88312934)(153.51696427,358.88312934)(153.54696824,358.87313724)
\curveto(153.57696421,358.86312936)(153.60196418,358.86812935)(153.62196824,358.88813724)
\curveto(154.04196374,358.89812932)(154.40696338,358.85312937)(154.71696824,358.75313724)
\curveto(155.02696276,358.66312956)(155.30696248,358.53812968)(155.55696824,358.37813724)
\curveto(155.60696218,358.35812986)(155.64696214,358.32812989)(155.67696824,358.28813724)
\curveto(155.70696208,358.25812996)(155.74196204,358.23312999)(155.78196824,358.21313724)
\curveto(155.86196192,358.15313007)(155.94196184,358.08313014)(156.02196824,358.00313724)
\curveto(156.11196167,357.9231303)(156.1869616,357.84313038)(156.24696824,357.76313724)
\curveto(156.40696138,357.55313067)(156.54196124,357.35313087)(156.65196824,357.16313724)
\curveto(156.72196106,357.05313117)(156.77696101,356.93313129)(156.81696824,356.80313724)
\curveto(156.85696093,356.67313155)(156.90196088,356.54313168)(156.95196824,356.41313724)
\curveto(157.00196078,356.28313194)(157.03696075,356.14813207)(157.05696824,356.00813724)
\curveto(157.0869607,355.86813235)(157.12196066,355.72813249)(157.16196824,355.58813724)
\curveto(157.17196061,355.5181327)(157.17696061,355.44813277)(157.17696824,355.37813724)
\lineto(157.20696824,355.16813724)
\moveto(155.75196824,355.67813724)
\curveto(155.781962,355.7181325)(155.80696198,355.76813245)(155.82696824,355.82813724)
\curveto(155.84696194,355.89813232)(155.84696194,355.96813225)(155.82696824,356.03813724)
\curveto(155.76696202,356.25813196)(155.6819621,356.46313176)(155.57196824,356.65313724)
\curveto(155.43196235,356.88313134)(155.27696251,357.07813114)(155.10696824,357.23813724)
\curveto(154.93696285,357.39813082)(154.71696307,357.53313069)(154.44696824,357.64313724)
\curveto(154.37696341,357.66313056)(154.30696348,357.67813054)(154.23696824,357.68813724)
\curveto(154.16696362,357.70813051)(154.09196369,357.72813049)(154.01196824,357.74813724)
\curveto(153.93196385,357.76813045)(153.84696394,357.77813044)(153.75696824,357.77813724)
\lineto(153.50196824,357.77813724)
\curveto(153.47196431,357.75813046)(153.43696435,357.74813047)(153.39696824,357.74813724)
\curveto(153.35696443,357.75813046)(153.32196446,357.75813046)(153.29196824,357.74813724)
\lineto(153.05196824,357.68813724)
\curveto(152.9819648,357.67813054)(152.91196487,357.66313056)(152.84196824,357.64313724)
\curveto(152.55196523,357.5231307)(152.31696547,357.37313085)(152.13696824,357.19313724)
\curveto(151.96696582,357.01313121)(151.81196597,356.78813143)(151.67196824,356.51813724)
\curveto(151.64196614,356.46813175)(151.61196617,356.40313182)(151.58196824,356.32313724)
\curveto(151.55196623,356.25313197)(151.52696626,356.17313205)(151.50696824,356.08313724)
\curveto(151.4869663,355.99313223)(151.4819663,355.90813231)(151.49196824,355.82813724)
\curveto(151.50196628,355.74813247)(151.53696625,355.68813253)(151.59696824,355.64813724)
\curveto(151.67696611,355.58813263)(151.81196597,355.55813266)(152.00196824,355.55813724)
\curveto(152.20196558,355.56813265)(152.37196541,355.57313265)(152.51196824,355.57313724)
\lineto(154.79196824,355.57313724)
\curveto(154.94196284,355.57313265)(155.12196266,355.56813265)(155.33196824,355.55813724)
\curveto(155.54196224,355.55813266)(155.6819621,355.59813262)(155.75196824,355.67813724)
}
}
{
\newrgbcolor{curcolor}{0 0 0}
\pscustom[linestyle=none,fillstyle=solid,fillcolor=curcolor]
{
\newpath
\moveto(161.64860887,358.90313724)
\curveto(162.38860408,358.91312931)(163.00360346,358.80312942)(163.49360887,358.57313724)
\curveto(163.99360247,358.35312987)(164.38860208,358.0181302)(164.67860887,357.56813724)
\curveto(164.80860166,357.36813085)(164.91860155,357.1231311)(165.00860887,356.83313724)
\curveto(165.02860144,356.78313144)(165.04360142,356.7181315)(165.05360887,356.63813724)
\curveto(165.0636014,356.55813166)(165.05860141,356.48813173)(165.03860887,356.42813724)
\curveto(165.00860146,356.37813184)(164.95860151,356.33313189)(164.88860887,356.29313724)
\curveto(164.85860161,356.27313195)(164.82860164,356.26313196)(164.79860887,356.26313724)
\curveto(164.7686017,356.27313195)(164.73360173,356.27313195)(164.69360887,356.26313724)
\curveto(164.65360181,356.25313197)(164.61360185,356.24813197)(164.57360887,356.24813724)
\curveto(164.53360193,356.25813196)(164.49360197,356.26313196)(164.45360887,356.26313724)
\lineto(164.13860887,356.26313724)
\curveto(164.03860243,356.27313195)(163.95360251,356.30313192)(163.88360887,356.35313724)
\curveto(163.80360266,356.41313181)(163.74860272,356.49813172)(163.71860887,356.60813724)
\curveto(163.68860278,356.7181315)(163.64860282,356.81313141)(163.59860887,356.89313724)
\curveto(163.44860302,357.15313107)(163.25360321,357.35813086)(163.01360887,357.50813724)
\curveto(162.93360353,357.55813066)(162.84860362,357.59813062)(162.75860887,357.62813724)
\curveto(162.6686038,357.66813055)(162.57360389,357.70313052)(162.47360887,357.73313724)
\curveto(162.33360413,357.77313045)(162.14860432,357.79313043)(161.91860887,357.79313724)
\curveto(161.68860478,357.80313042)(161.49860497,357.78313044)(161.34860887,357.73313724)
\curveto(161.27860519,357.71313051)(161.21360525,357.69813052)(161.15360887,357.68813724)
\curveto(161.09360537,357.67813054)(161.02860544,357.66313056)(160.95860887,357.64313724)
\curveto(160.69860577,357.53313069)(160.468606,357.38313084)(160.26860887,357.19313724)
\curveto(160.0686064,357.00313122)(159.91360655,356.77813144)(159.80360887,356.51813724)
\curveto(159.7636067,356.42813179)(159.72860674,356.33313189)(159.69860887,356.23313724)
\curveto(159.6686068,356.14313208)(159.63860683,356.04313218)(159.60860887,355.93313724)
\lineto(159.51860887,355.52813724)
\curveto(159.50860696,355.47813274)(159.50360696,355.4231328)(159.50360887,355.36313724)
\curveto(159.51360695,355.30313292)(159.50860696,355.24813297)(159.48860887,355.19813724)
\lineto(159.48860887,355.07813724)
\curveto(159.47860699,355.03813318)(159.468607,354.97313325)(159.45860887,354.88313724)
\curveto(159.45860701,354.79313343)(159.468607,354.72813349)(159.48860887,354.68813724)
\curveto(159.49860697,354.63813358)(159.49860697,354.58813363)(159.48860887,354.53813724)
\curveto(159.47860699,354.48813373)(159.47860699,354.43813378)(159.48860887,354.38813724)
\curveto(159.49860697,354.34813387)(159.50360696,354.27813394)(159.50360887,354.17813724)
\curveto(159.52360694,354.09813412)(159.53860693,354.01313421)(159.54860887,353.92313724)
\curveto(159.5686069,353.83313439)(159.58860688,353.74813447)(159.60860887,353.66813724)
\curveto(159.71860675,353.34813487)(159.84360662,353.06813515)(159.98360887,352.82813724)
\curveto(160.13360633,352.59813562)(160.33860613,352.39813582)(160.59860887,352.22813724)
\curveto(160.68860578,352.17813604)(160.77860569,352.13313609)(160.86860887,352.09313724)
\curveto(160.9686055,352.05313617)(161.07360539,352.01313621)(161.18360887,351.97313724)
\curveto(161.23360523,351.96313626)(161.27360519,351.95813626)(161.30360887,351.95813724)
\curveto(161.33360513,351.95813626)(161.37360509,351.95313627)(161.42360887,351.94313724)
\curveto(161.45360501,351.93313629)(161.50360496,351.92813629)(161.57360887,351.92813724)
\lineto(161.73860887,351.92813724)
\curveto(161.73860473,351.9181363)(161.75860471,351.91313631)(161.79860887,351.91313724)
\curveto(161.81860465,351.9231363)(161.84360462,351.9231363)(161.87360887,351.91313724)
\curveto(161.90360456,351.91313631)(161.93360453,351.9181363)(161.96360887,351.92813724)
\curveto(162.03360443,351.94813627)(162.09860437,351.95313627)(162.15860887,351.94313724)
\curveto(162.22860424,351.94313628)(162.29860417,351.95313627)(162.36860887,351.97313724)
\curveto(162.62860384,352.05313617)(162.85360361,352.15313607)(163.04360887,352.27313724)
\curveto(163.23360323,352.40313582)(163.39360307,352.56813565)(163.52360887,352.76813724)
\curveto(163.57360289,352.84813537)(163.61860285,352.93313529)(163.65860887,353.02313724)
\lineto(163.77860887,353.29313724)
\curveto(163.79860267,353.37313485)(163.81860265,353.44813477)(163.83860887,353.51813724)
\curveto(163.8686026,353.59813462)(163.91860255,353.66313456)(163.98860887,353.71313724)
\curveto(164.01860245,353.74313448)(164.07860239,353.76313446)(164.16860887,353.77313724)
\curveto(164.25860221,353.79313443)(164.35360211,353.80313442)(164.45360887,353.80313724)
\curveto(164.5636019,353.81313441)(164.6636018,353.81313441)(164.75360887,353.80313724)
\curveto(164.85360161,353.79313443)(164.92360154,353.77313445)(164.96360887,353.74313724)
\curveto(165.02360144,353.70313452)(165.05860141,353.64313458)(165.06860887,353.56313724)
\curveto(165.08860138,353.48313474)(165.08860138,353.39813482)(165.06860887,353.30813724)
\curveto(165.01860145,353.15813506)(164.9686015,353.01313521)(164.91860887,352.87313724)
\curveto(164.87860159,352.74313548)(164.82360164,352.61313561)(164.75360887,352.48313724)
\curveto(164.60360186,352.18313604)(164.41360205,351.9181363)(164.18360887,351.68813724)
\curveto(163.9636025,351.45813676)(163.69360277,351.27313695)(163.37360887,351.13313724)
\curveto(163.29360317,351.09313713)(163.20860326,351.05813716)(163.11860887,351.02813724)
\curveto(163.02860344,351.00813721)(162.93360353,350.98313724)(162.83360887,350.95313724)
\curveto(162.72360374,350.91313731)(162.61360385,350.89313733)(162.50360887,350.89313724)
\curveto(162.39360407,350.88313734)(162.28360418,350.86813735)(162.17360887,350.84813724)
\curveto(162.13360433,350.82813739)(162.09360437,350.8231374)(162.05360887,350.83313724)
\curveto(162.01360445,350.84313738)(161.97360449,350.84313738)(161.93360887,350.83313724)
\lineto(161.79860887,350.83313724)
\lineto(161.55860887,350.83313724)
\curveto(161.48860498,350.8231374)(161.42360504,350.82813739)(161.36360887,350.84813724)
\lineto(161.28860887,350.84813724)
\lineto(160.92860887,350.89313724)
\curveto(160.79860567,350.93313729)(160.67360579,350.96813725)(160.55360887,350.99813724)
\curveto(160.43360603,351.02813719)(160.31860615,351.06813715)(160.20860887,351.11813724)
\curveto(159.84860662,351.27813694)(159.54860692,351.46813675)(159.30860887,351.68813724)
\curveto(159.07860739,351.90813631)(158.8636076,352.17813604)(158.66360887,352.49813724)
\curveto(158.61360785,352.57813564)(158.5686079,352.66813555)(158.52860887,352.76813724)
\lineto(158.40860887,353.06813724)
\curveto(158.35860811,353.17813504)(158.32360814,353.29313493)(158.30360887,353.41313724)
\curveto(158.28360818,353.53313469)(158.25860821,353.65313457)(158.22860887,353.77313724)
\curveto(158.21860825,353.81313441)(158.21360825,353.85313437)(158.21360887,353.89313724)
\curveto(158.21360825,353.93313429)(158.20860826,353.97313425)(158.19860887,354.01313724)
\curveto(158.17860829,354.07313415)(158.1686083,354.13813408)(158.16860887,354.20813724)
\curveto(158.17860829,354.27813394)(158.17360829,354.34313388)(158.15360887,354.40313724)
\lineto(158.15360887,354.55313724)
\curveto(158.14360832,354.60313362)(158.13860833,354.67313355)(158.13860887,354.76313724)
\curveto(158.13860833,354.85313337)(158.14360832,354.9231333)(158.15360887,354.97313724)
\curveto(158.1636083,355.0231332)(158.1636083,355.06813315)(158.15360887,355.10813724)
\curveto(158.15360831,355.14813307)(158.15860831,355.18813303)(158.16860887,355.22813724)
\curveto(158.18860828,355.29813292)(158.19360827,355.36813285)(158.18360887,355.43813724)
\curveto(158.18360828,355.50813271)(158.19360827,355.57313265)(158.21360887,355.63313724)
\curveto(158.25360821,355.80313242)(158.28860818,355.97313225)(158.31860887,356.14313724)
\curveto(158.34860812,356.31313191)(158.39360807,356.47313175)(158.45360887,356.62313724)
\curveto(158.6636078,357.14313108)(158.91860755,357.56313066)(159.21860887,357.88313724)
\curveto(159.51860695,358.20313002)(159.92860654,358.46812975)(160.44860887,358.67813724)
\curveto(160.55860591,358.72812949)(160.67860579,358.76312946)(160.80860887,358.78313724)
\curveto(160.93860553,358.80312942)(161.07360539,358.82812939)(161.21360887,358.85813724)
\curveto(161.28360518,358.86812935)(161.35360511,358.87312935)(161.42360887,358.87313724)
\curveto(161.49360497,358.88312934)(161.5686049,358.89312933)(161.64860887,358.90313724)
}
}
{
\newrgbcolor{curcolor}{0 0 0}
\pscustom[linestyle=none,fillstyle=solid,fillcolor=curcolor]
{
\newpath
\moveto(167.03524949,358.72313724)
\lineto(167.47024949,358.72313724)
\curveto(167.62024753,358.7231295)(167.72524742,358.68312954)(167.78524949,358.60313724)
\curveto(167.83524731,358.5231297)(167.86024729,358.4231298)(167.86024949,358.30313724)
\curveto(167.87024728,358.18313004)(167.87524727,358.06313016)(167.87524949,357.94313724)
\lineto(167.87524949,356.51813724)
\lineto(167.87524949,354.25313724)
\lineto(167.87524949,353.56313724)
\curveto(167.87524727,353.33313489)(167.90024725,353.13313509)(167.95024949,352.96313724)
\curveto(168.11024704,352.51313571)(168.41024674,352.19813602)(168.85024949,352.01813724)
\curveto(169.07024608,351.92813629)(169.33524581,351.89313633)(169.64524949,351.91313724)
\curveto(169.95524519,351.94313628)(170.20524494,351.99813622)(170.39524949,352.07813724)
\curveto(170.72524442,352.218136)(170.98524416,352.39313583)(171.17524949,352.60313724)
\curveto(171.37524377,352.8231354)(171.53024362,353.10813511)(171.64024949,353.45813724)
\curveto(171.67024348,353.53813468)(171.69024346,353.6181346)(171.70024949,353.69813724)
\curveto(171.71024344,353.77813444)(171.72524342,353.86313436)(171.74524949,353.95313724)
\curveto(171.75524339,354.00313422)(171.75524339,354.04813417)(171.74524949,354.08813724)
\curveto(171.7452434,354.12813409)(171.75524339,354.17313405)(171.77524949,354.22313724)
\lineto(171.77524949,354.53813724)
\curveto(171.79524335,354.6181336)(171.80024335,354.70813351)(171.79024949,354.80813724)
\curveto(171.78024337,354.9181333)(171.77524337,355.0181332)(171.77524949,355.10813724)
\lineto(171.77524949,356.27813724)
\lineto(171.77524949,357.86813724)
\curveto(171.77524337,357.98813023)(171.77024338,358.11313011)(171.76024949,358.24313724)
\curveto(171.76024339,358.38312984)(171.78524336,358.49312973)(171.83524949,358.57313724)
\curveto(171.87524327,358.6231296)(171.92024323,358.65312957)(171.97024949,358.66313724)
\curveto(172.03024312,358.68312954)(172.10024305,358.70312952)(172.18024949,358.72313724)
\lineto(172.40524949,358.72313724)
\curveto(172.52524262,358.7231295)(172.63024252,358.7181295)(172.72024949,358.70813724)
\curveto(172.82024233,358.69812952)(172.89524225,358.65312957)(172.94524949,358.57313724)
\curveto(172.99524215,358.5231297)(173.02024213,358.44812977)(173.02024949,358.34813724)
\lineto(173.02024949,358.06313724)
\lineto(173.02024949,357.04313724)
\lineto(173.02024949,353.00813724)
\lineto(173.02024949,351.65813724)
\curveto(173.02024213,351.53813668)(173.01524213,351.4231368)(173.00524949,351.31313724)
\curveto(173.00524214,351.21313701)(172.97024218,351.13813708)(172.90024949,351.08813724)
\curveto(172.86024229,351.05813716)(172.80024235,351.03313719)(172.72024949,351.01313724)
\curveto(172.64024251,351.00313722)(172.5502426,350.99313723)(172.45024949,350.98313724)
\curveto(172.36024279,350.98313724)(172.27024288,350.98813723)(172.18024949,350.99813724)
\curveto(172.10024305,351.00813721)(172.04024311,351.02813719)(172.00024949,351.05813724)
\curveto(171.9502432,351.09813712)(171.90524324,351.16313706)(171.86524949,351.25313724)
\curveto(171.85524329,351.29313693)(171.8452433,351.34813687)(171.83524949,351.41813724)
\curveto(171.83524331,351.48813673)(171.83024332,351.55313667)(171.82024949,351.61313724)
\curveto(171.81024334,351.68313654)(171.79024336,351.73813648)(171.76024949,351.77813724)
\curveto(171.73024342,351.8181364)(171.68524346,351.83313639)(171.62524949,351.82313724)
\curveto(171.5452436,351.80313642)(171.46524368,351.74313648)(171.38524949,351.64313724)
\curveto(171.30524384,351.55313667)(171.23024392,351.48313674)(171.16024949,351.43313724)
\curveto(170.94024421,351.27313695)(170.69024446,351.13313709)(170.41024949,351.01313724)
\curveto(170.30024485,350.96313726)(170.18524496,350.93313729)(170.06524949,350.92313724)
\curveto(169.95524519,350.90313732)(169.84024531,350.87813734)(169.72024949,350.84813724)
\curveto(169.67024548,350.83813738)(169.61524553,350.83813738)(169.55524949,350.84813724)
\curveto(169.50524564,350.85813736)(169.45524569,350.85313737)(169.40524949,350.83313724)
\curveto(169.30524584,350.81313741)(169.21524593,350.81313741)(169.13524949,350.83313724)
\lineto(168.98524949,350.83313724)
\curveto(168.93524621,350.85313737)(168.87524627,350.86313736)(168.80524949,350.86313724)
\curveto(168.7452464,350.86313736)(168.69024646,350.86813735)(168.64024949,350.87813724)
\curveto(168.60024655,350.89813732)(168.56024659,350.90813731)(168.52024949,350.90813724)
\curveto(168.49024666,350.89813732)(168.4502467,350.90313732)(168.40024949,350.92313724)
\lineto(168.16024949,350.98313724)
\curveto(168.09024706,351.00313722)(168.01524713,351.03313719)(167.93524949,351.07313724)
\curveto(167.67524747,351.18313704)(167.45524769,351.32813689)(167.27524949,351.50813724)
\curveto(167.10524804,351.69813652)(166.96524818,351.9231363)(166.85524949,352.18313724)
\curveto(166.81524833,352.27313595)(166.78524836,352.36313586)(166.76524949,352.45313724)
\lineto(166.70524949,352.75313724)
\curveto(166.68524846,352.81313541)(166.67524847,352.86813535)(166.67524949,352.91813724)
\curveto(166.68524846,352.97813524)(166.68024847,353.04313518)(166.66024949,353.11313724)
\curveto(166.6502485,353.13313509)(166.6452485,353.15813506)(166.64524949,353.18813724)
\curveto(166.6452485,353.22813499)(166.64024851,353.26313496)(166.63024949,353.29313724)
\lineto(166.63024949,353.44313724)
\curveto(166.62024853,353.48313474)(166.61524853,353.52813469)(166.61524949,353.57813724)
\curveto(166.62524852,353.63813458)(166.63024852,353.69313453)(166.63024949,353.74313724)
\lineto(166.63024949,354.34313724)
\lineto(166.63024949,357.10313724)
\lineto(166.63024949,358.06313724)
\lineto(166.63024949,358.33313724)
\curveto(166.63024852,358.4231298)(166.6502485,358.49812972)(166.69024949,358.55813724)
\curveto(166.73024842,358.62812959)(166.80524834,358.67812954)(166.91524949,358.70813724)
\curveto(166.93524821,358.7181295)(166.95524819,358.7181295)(166.97524949,358.70813724)
\curveto(166.99524815,358.70812951)(167.01524813,358.71312951)(167.03524949,358.72313724)
}
}
{
\newrgbcolor{curcolor}{0 0 0}
\pscustom[linestyle=none,fillstyle=solid,fillcolor=curcolor]
{
\newpath
\moveto(178.56485887,358.90313724)
\curveto(178.79485408,358.90312932)(178.92485395,358.84312938)(178.95485887,358.72313724)
\curveto(178.98485389,358.61312961)(178.99985387,358.44812977)(178.99985887,358.22813724)
\lineto(178.99985887,357.94313724)
\curveto(178.99985387,357.85313037)(178.9748539,357.77813044)(178.92485887,357.71813724)
\curveto(178.86485401,357.63813058)(178.77985409,357.59313063)(178.66985887,357.58313724)
\curveto(178.55985431,357.58313064)(178.44985442,357.56813065)(178.33985887,357.53813724)
\curveto(178.19985467,357.50813071)(178.06485481,357.47813074)(177.93485887,357.44813724)
\curveto(177.81485506,357.4181308)(177.69985517,357.37813084)(177.58985887,357.32813724)
\curveto(177.29985557,357.19813102)(177.06485581,357.0181312)(176.88485887,356.78813724)
\curveto(176.70485617,356.56813165)(176.54985632,356.31313191)(176.41985887,356.02313724)
\curveto(176.37985649,355.91313231)(176.34985652,355.79813242)(176.32985887,355.67813724)
\curveto(176.30985656,355.56813265)(176.28485659,355.45313277)(176.25485887,355.33313724)
\curveto(176.24485663,355.28313294)(176.23985663,355.23313299)(176.23985887,355.18313724)
\curveto(176.24985662,355.13313309)(176.24985662,355.08313314)(176.23985887,355.03313724)
\curveto(176.20985666,354.91313331)(176.19485668,354.77313345)(176.19485887,354.61313724)
\curveto(176.20485667,354.46313376)(176.20985666,354.3181339)(176.20985887,354.17813724)
\lineto(176.20985887,352.33313724)
\lineto(176.20985887,351.98813724)
\curveto(176.20985666,351.86813635)(176.20485667,351.75313647)(176.19485887,351.64313724)
\curveto(176.18485669,351.53313669)(176.17985669,351.43813678)(176.17985887,351.35813724)
\curveto(176.18985668,351.27813694)(176.1698567,351.20813701)(176.11985887,351.14813724)
\curveto(176.0698568,351.07813714)(175.98985688,351.03813718)(175.87985887,351.02813724)
\curveto(175.77985709,351.0181372)(175.6698572,351.01313721)(175.54985887,351.01313724)
\lineto(175.27985887,351.01313724)
\curveto(175.22985764,351.03313719)(175.17985769,351.04813717)(175.12985887,351.05813724)
\curveto(175.08985778,351.07813714)(175.05985781,351.10313712)(175.03985887,351.13313724)
\curveto(174.98985788,351.20313702)(174.95985791,351.28813693)(174.94985887,351.38813724)
\lineto(174.94985887,351.71813724)
\lineto(174.94985887,352.87313724)
\lineto(174.94985887,357.02813724)
\lineto(174.94985887,358.06313724)
\lineto(174.94985887,358.36313724)
\curveto(174.95985791,358.46312976)(174.98985788,358.54812967)(175.03985887,358.61813724)
\curveto(175.0698578,358.65812956)(175.11985775,358.68812953)(175.18985887,358.70813724)
\curveto(175.2698576,358.72812949)(175.35485752,358.73812948)(175.44485887,358.73813724)
\curveto(175.53485734,358.74812947)(175.62485725,358.74812947)(175.71485887,358.73813724)
\curveto(175.80485707,358.72812949)(175.874857,358.71312951)(175.92485887,358.69313724)
\curveto(176.00485687,358.66312956)(176.05485682,358.60312962)(176.07485887,358.51313724)
\curveto(176.10485677,358.43312979)(176.11985675,358.34312988)(176.11985887,358.24313724)
\lineto(176.11985887,357.94313724)
\curveto(176.11985675,357.84313038)(176.13985673,357.75313047)(176.17985887,357.67313724)
\curveto(176.18985668,357.65313057)(176.19985667,357.63813058)(176.20985887,357.62813724)
\lineto(176.25485887,357.58313724)
\curveto(176.36485651,357.58313064)(176.45485642,357.62813059)(176.52485887,357.71813724)
\curveto(176.59485628,357.8181304)(176.65485622,357.89813032)(176.70485887,357.95813724)
\lineto(176.79485887,358.04813724)
\curveto(176.88485599,358.15813006)(177.00985586,358.27312995)(177.16985887,358.39313724)
\curveto(177.32985554,358.51312971)(177.47985539,358.60312962)(177.61985887,358.66313724)
\curveto(177.70985516,358.71312951)(177.80485507,358.74812947)(177.90485887,358.76813724)
\curveto(178.00485487,358.79812942)(178.10985476,358.82812939)(178.21985887,358.85813724)
\curveto(178.27985459,358.86812935)(178.33985453,358.87312935)(178.39985887,358.87313724)
\curveto(178.45985441,358.88312934)(178.51485436,358.89312933)(178.56485887,358.90313724)
}
}
{
\newrgbcolor{curcolor}{0 0 0}
\pscustom[linestyle=none,fillstyle=solid,fillcolor=curcolor]
{
\newpath
\moveto(182.35962449,358.90313724)
\curveto(183.07962043,358.91312931)(183.68461982,358.82812939)(184.17462449,358.64813724)
\curveto(184.66461884,358.47812974)(185.04461846,358.17313005)(185.31462449,357.73313724)
\curveto(185.38461812,357.6231306)(185.43961807,357.50813071)(185.47962449,357.38813724)
\curveto(185.51961799,357.27813094)(185.55961795,357.15313107)(185.59962449,357.01313724)
\curveto(185.61961789,356.94313128)(185.62461788,356.86813135)(185.61462449,356.78813724)
\curveto(185.6046179,356.7181315)(185.58961792,356.66313156)(185.56962449,356.62313724)
\curveto(185.54961796,356.60313162)(185.52461798,356.58313164)(185.49462449,356.56313724)
\curveto(185.46461804,356.55313167)(185.43961807,356.53813168)(185.41962449,356.51813724)
\curveto(185.36961814,356.49813172)(185.31961819,356.49313173)(185.26962449,356.50313724)
\curveto(185.21961829,356.51313171)(185.16961834,356.51313171)(185.11962449,356.50313724)
\curveto(185.03961847,356.48313174)(184.93461857,356.47813174)(184.80462449,356.48813724)
\curveto(184.67461883,356.50813171)(184.58461892,356.53313169)(184.53462449,356.56313724)
\curveto(184.45461905,356.61313161)(184.39961911,356.67813154)(184.36962449,356.75813724)
\curveto(184.34961916,356.84813137)(184.31461919,356.93313129)(184.26462449,357.01313724)
\curveto(184.17461933,357.17313105)(184.04961946,357.3181309)(183.88962449,357.44813724)
\curveto(183.77961973,357.52813069)(183.65961985,357.58813063)(183.52962449,357.62813724)
\curveto(183.39962011,357.66813055)(183.25962025,357.70813051)(183.10962449,357.74813724)
\curveto(183.05962045,357.76813045)(183.0096205,357.77313045)(182.95962449,357.76313724)
\curveto(182.9096206,357.76313046)(182.85962065,357.76813045)(182.80962449,357.77813724)
\curveto(182.74962076,357.79813042)(182.67462083,357.80813041)(182.58462449,357.80813724)
\curveto(182.49462101,357.80813041)(182.41962109,357.79813042)(182.35962449,357.77813724)
\lineto(182.26962449,357.77813724)
\lineto(182.11962449,357.74813724)
\curveto(182.06962144,357.74813047)(182.01962149,357.74313048)(181.96962449,357.73313724)
\curveto(181.7096218,357.67313055)(181.49462201,357.58813063)(181.32462449,357.47813724)
\curveto(181.15462235,357.36813085)(181.03962247,357.18313104)(180.97962449,356.92313724)
\curveto(180.95962255,356.85313137)(180.95462255,356.78313144)(180.96462449,356.71313724)
\curveto(180.98462252,356.64313158)(181.0046225,356.58313164)(181.02462449,356.53313724)
\curveto(181.08462242,356.38313184)(181.15462235,356.27313195)(181.23462449,356.20313724)
\curveto(181.32462218,356.14313208)(181.43462207,356.07313215)(181.56462449,355.99313724)
\curveto(181.72462178,355.89313233)(181.9046216,355.8181324)(182.10462449,355.76813724)
\curveto(182.3046212,355.72813249)(182.504621,355.67813254)(182.70462449,355.61813724)
\curveto(182.83462067,355.57813264)(182.96462054,355.54813267)(183.09462449,355.52813724)
\curveto(183.22462028,355.50813271)(183.35462015,355.47813274)(183.48462449,355.43813724)
\curveto(183.69461981,355.37813284)(183.89961961,355.3181329)(184.09962449,355.25813724)
\curveto(184.29961921,355.20813301)(184.49961901,355.14313308)(184.69962449,355.06313724)
\lineto(184.84962449,355.00313724)
\curveto(184.89961861,354.98313324)(184.94961856,354.95813326)(184.99962449,354.92813724)
\curveto(185.19961831,354.80813341)(185.37461813,354.67313355)(185.52462449,354.52313724)
\curveto(185.67461783,354.37313385)(185.79961771,354.18313404)(185.89962449,353.95313724)
\curveto(185.91961759,353.88313434)(185.93961757,353.78813443)(185.95962449,353.66813724)
\curveto(185.97961753,353.59813462)(185.98961752,353.5231347)(185.98962449,353.44313724)
\curveto(185.99961751,353.37313485)(186.0046175,353.29313493)(186.00462449,353.20313724)
\lineto(186.00462449,353.05313724)
\curveto(185.98461752,352.98313524)(185.97461753,352.91313531)(185.97462449,352.84313724)
\curveto(185.97461753,352.77313545)(185.96461754,352.70313552)(185.94462449,352.63313724)
\curveto(185.91461759,352.5231357)(185.87961763,352.4181358)(185.83962449,352.31813724)
\curveto(185.79961771,352.218136)(185.75461775,352.12813609)(185.70462449,352.04813724)
\curveto(185.54461796,351.78813643)(185.33961817,351.57813664)(185.08962449,351.41813724)
\curveto(184.83961867,351.26813695)(184.55961895,351.13813708)(184.24962449,351.02813724)
\curveto(184.15961935,350.99813722)(184.06461944,350.97813724)(183.96462449,350.96813724)
\curveto(183.87461963,350.94813727)(183.78461972,350.9231373)(183.69462449,350.89313724)
\curveto(183.59461991,350.87313735)(183.49462001,350.86313736)(183.39462449,350.86313724)
\curveto(183.29462021,350.86313736)(183.19462031,350.85313737)(183.09462449,350.83313724)
\lineto(182.94462449,350.83313724)
\curveto(182.89462061,350.8231374)(182.82462068,350.8181374)(182.73462449,350.81813724)
\curveto(182.64462086,350.8181374)(182.57462093,350.8231374)(182.52462449,350.83313724)
\lineto(182.35962449,350.83313724)
\curveto(182.29962121,350.85313737)(182.23462127,350.86313736)(182.16462449,350.86313724)
\curveto(182.09462141,350.85313737)(182.03462147,350.85813736)(181.98462449,350.87813724)
\curveto(181.93462157,350.88813733)(181.86962164,350.89313733)(181.78962449,350.89313724)
\lineto(181.54962449,350.95313724)
\curveto(181.47962203,350.96313726)(181.4046221,350.98313724)(181.32462449,351.01313724)
\curveto(181.01462249,351.11313711)(180.74462276,351.23813698)(180.51462449,351.38813724)
\curveto(180.28462322,351.53813668)(180.08462342,351.73313649)(179.91462449,351.97313724)
\curveto(179.82462368,352.10313612)(179.74962376,352.23813598)(179.68962449,352.37813724)
\curveto(179.62962388,352.5181357)(179.57462393,352.67313555)(179.52462449,352.84313724)
\curveto(179.504624,352.90313532)(179.49462401,352.97313525)(179.49462449,353.05313724)
\curveto(179.504624,353.14313508)(179.51962399,353.21313501)(179.53962449,353.26313724)
\curveto(179.56962394,353.30313492)(179.61962389,353.34313488)(179.68962449,353.38313724)
\curveto(179.73962377,353.40313482)(179.8096237,353.41313481)(179.89962449,353.41313724)
\curveto(179.98962352,353.4231348)(180.07962343,353.4231348)(180.16962449,353.41313724)
\curveto(180.25962325,353.40313482)(180.34462316,353.38813483)(180.42462449,353.36813724)
\curveto(180.51462299,353.35813486)(180.57462293,353.34313488)(180.60462449,353.32313724)
\curveto(180.67462283,353.27313495)(180.71962279,353.19813502)(180.73962449,353.09813724)
\curveto(180.76962274,353.00813521)(180.8046227,352.9231353)(180.84462449,352.84313724)
\curveto(180.94462256,352.6231356)(181.07962243,352.45313577)(181.24962449,352.33313724)
\curveto(181.36962214,352.24313598)(181.504622,352.17313605)(181.65462449,352.12313724)
\curveto(181.8046217,352.07313615)(181.96462154,352.0231362)(182.13462449,351.97313724)
\lineto(182.44962449,351.92813724)
\lineto(182.53962449,351.92813724)
\curveto(182.6096209,351.90813631)(182.69962081,351.89813632)(182.80962449,351.89813724)
\curveto(182.92962058,351.89813632)(183.02962048,351.90813631)(183.10962449,351.92813724)
\curveto(183.17962033,351.92813629)(183.23462027,351.93313629)(183.27462449,351.94313724)
\curveto(183.33462017,351.95313627)(183.39462011,351.95813626)(183.45462449,351.95813724)
\curveto(183.51461999,351.96813625)(183.56961994,351.97813624)(183.61962449,351.98813724)
\curveto(183.9096196,352.06813615)(184.13961937,352.17313605)(184.30962449,352.30313724)
\curveto(184.47961903,352.43313579)(184.59961891,352.65313557)(184.66962449,352.96313724)
\curveto(184.68961882,353.01313521)(184.69461881,353.06813515)(184.68462449,353.12813724)
\curveto(184.67461883,353.18813503)(184.66461884,353.23313499)(184.65462449,353.26313724)
\curveto(184.6046189,353.45313477)(184.53461897,353.59313463)(184.44462449,353.68313724)
\curveto(184.35461915,353.78313444)(184.23961927,353.87313435)(184.09962449,353.95313724)
\curveto(184.0096195,354.01313421)(183.9096196,354.06313416)(183.79962449,354.10313724)
\lineto(183.46962449,354.22313724)
\curveto(183.43962007,354.23313399)(183.4096201,354.23813398)(183.37962449,354.23813724)
\curveto(183.35962015,354.23813398)(183.33462017,354.24813397)(183.30462449,354.26813724)
\curveto(182.96462054,354.37813384)(182.6096209,354.45813376)(182.23962449,354.50813724)
\curveto(181.87962163,354.56813365)(181.53962197,354.66313356)(181.21962449,354.79313724)
\curveto(181.11962239,354.83313339)(181.02462248,354.86813335)(180.93462449,354.89813724)
\curveto(180.84462266,354.92813329)(180.75962275,354.96813325)(180.67962449,355.01813724)
\curveto(180.48962302,355.12813309)(180.31462319,355.25313297)(180.15462449,355.39313724)
\curveto(179.99462351,355.53313269)(179.86962364,355.70813251)(179.77962449,355.91813724)
\curveto(179.74962376,355.98813223)(179.72462378,356.05813216)(179.70462449,356.12813724)
\curveto(179.69462381,356.19813202)(179.67962383,356.27313195)(179.65962449,356.35313724)
\curveto(179.62962388,356.47313175)(179.61962389,356.60813161)(179.62962449,356.75813724)
\curveto(179.63962387,356.9181313)(179.65462385,357.05313117)(179.67462449,357.16313724)
\curveto(179.69462381,357.21313101)(179.7046238,357.25313097)(179.70462449,357.28313724)
\curveto(179.71462379,357.3231309)(179.72962378,357.36313086)(179.74962449,357.40313724)
\curveto(179.83962367,357.63313059)(179.95962355,357.83313039)(180.10962449,358.00313724)
\curveto(180.26962324,358.17313005)(180.44962306,358.3231299)(180.64962449,358.45313724)
\curveto(180.79962271,358.54312968)(180.96462254,358.61312961)(181.14462449,358.66313724)
\curveto(181.32462218,358.7231295)(181.51462199,358.77812944)(181.71462449,358.82813724)
\curveto(181.78462172,358.83812938)(181.84962166,358.84812937)(181.90962449,358.85813724)
\curveto(181.97962153,358.86812935)(182.05462145,358.87812934)(182.13462449,358.88813724)
\curveto(182.16462134,358.89812932)(182.2046213,358.89812932)(182.25462449,358.88813724)
\curveto(182.3046212,358.87812934)(182.33962117,358.88312934)(182.35962449,358.90313724)
}
}
{
\newrgbcolor{curcolor}{0 0 0}
\pscustom[linestyle=none,fillstyle=solid,fillcolor=curcolor]
{
\newpath
\moveto(194.55462449,355.19813724)
\curveto(194.57461643,355.13813308)(194.58461642,355.04313318)(194.58462449,354.91313724)
\curveto(194.58461642,354.79313343)(194.57961643,354.70813351)(194.56962449,354.65813724)
\lineto(194.56962449,354.50813724)
\curveto(194.55961645,354.42813379)(194.54961646,354.35313387)(194.53962449,354.28313724)
\curveto(194.53961647,354.223134)(194.53461647,354.15313407)(194.52462449,354.07313724)
\curveto(194.5046165,354.01313421)(194.48961652,353.95313427)(194.47962449,353.89313724)
\curveto(194.47961653,353.83313439)(194.46961654,353.77313445)(194.44962449,353.71313724)
\curveto(194.4096166,353.58313464)(194.37461663,353.45313477)(194.34462449,353.32313724)
\curveto(194.31461669,353.19313503)(194.27461673,353.07313515)(194.22462449,352.96313724)
\curveto(194.01461699,352.48313574)(193.73461727,352.07813614)(193.38462449,351.74813724)
\curveto(193.03461797,351.42813679)(192.6046184,351.18313704)(192.09462449,351.01313724)
\curveto(191.98461902,350.97313725)(191.86461914,350.94313728)(191.73462449,350.92313724)
\curveto(191.61461939,350.90313732)(191.48961952,350.88313734)(191.35962449,350.86313724)
\curveto(191.29961971,350.85313737)(191.23461977,350.84813737)(191.16462449,350.84813724)
\curveto(191.1046199,350.83813738)(191.04461996,350.83313739)(190.98462449,350.83313724)
\curveto(190.94462006,350.8231374)(190.88462012,350.8181374)(190.80462449,350.81813724)
\curveto(190.73462027,350.8181374)(190.68462032,350.8231374)(190.65462449,350.83313724)
\curveto(190.61462039,350.84313738)(190.57462043,350.84813737)(190.53462449,350.84813724)
\curveto(190.49462051,350.83813738)(190.45962055,350.83813738)(190.42962449,350.84813724)
\lineto(190.33962449,350.84813724)
\lineto(189.97962449,350.89313724)
\curveto(189.83962117,350.93313729)(189.7046213,350.97313725)(189.57462449,351.01313724)
\curveto(189.44462156,351.05313717)(189.31962169,351.09813712)(189.19962449,351.14813724)
\curveto(188.74962226,351.34813687)(188.37962263,351.60813661)(188.08962449,351.92813724)
\curveto(187.79962321,352.24813597)(187.55962345,352.63813558)(187.36962449,353.09813724)
\curveto(187.31962369,353.19813502)(187.27962373,353.29813492)(187.24962449,353.39813724)
\curveto(187.22962378,353.49813472)(187.2096238,353.60313462)(187.18962449,353.71313724)
\curveto(187.16962384,353.75313447)(187.15962385,353.78313444)(187.15962449,353.80313724)
\curveto(187.16962384,353.83313439)(187.16962384,353.86813435)(187.15962449,353.90813724)
\curveto(187.13962387,353.98813423)(187.12462388,354.06813415)(187.11462449,354.14813724)
\curveto(187.11462389,354.23813398)(187.1046239,354.3231339)(187.08462449,354.40313724)
\lineto(187.08462449,354.52313724)
\curveto(187.08462392,354.56313366)(187.07962393,354.60813361)(187.06962449,354.65813724)
\curveto(187.05962395,354.70813351)(187.05462395,354.79313343)(187.05462449,354.91313724)
\curveto(187.05462395,355.04313318)(187.06462394,355.13813308)(187.08462449,355.19813724)
\curveto(187.1046239,355.26813295)(187.1096239,355.33813288)(187.09962449,355.40813724)
\curveto(187.08962392,355.47813274)(187.09462391,355.54813267)(187.11462449,355.61813724)
\curveto(187.12462388,355.66813255)(187.12962388,355.70813251)(187.12962449,355.73813724)
\curveto(187.13962387,355.77813244)(187.14962386,355.8231324)(187.15962449,355.87313724)
\curveto(187.18962382,355.99313223)(187.21462379,356.11313211)(187.23462449,356.23313724)
\curveto(187.26462374,356.35313187)(187.3046237,356.46813175)(187.35462449,356.57813724)
\curveto(187.5046235,356.94813127)(187.68462332,357.27813094)(187.89462449,357.56813724)
\curveto(188.11462289,357.86813035)(188.37962263,358.1181301)(188.68962449,358.31813724)
\curveto(188.8096222,358.39812982)(188.93462207,358.46312976)(189.06462449,358.51313724)
\curveto(189.19462181,358.57312965)(189.32962168,358.63312959)(189.46962449,358.69313724)
\curveto(189.58962142,358.74312948)(189.71962129,358.77312945)(189.85962449,358.78313724)
\curveto(189.99962101,358.80312942)(190.13962087,358.83312939)(190.27962449,358.87313724)
\lineto(190.47462449,358.87313724)
\curveto(190.54462046,358.88312934)(190.6096204,358.89312933)(190.66962449,358.90313724)
\curveto(191.55961945,358.91312931)(192.29961871,358.72812949)(192.88962449,358.34813724)
\curveto(193.47961753,357.96813025)(193.9046171,357.47313075)(194.16462449,356.86313724)
\curveto(194.21461679,356.76313146)(194.25461675,356.66313156)(194.28462449,356.56313724)
\curveto(194.31461669,356.46313176)(194.34961666,356.35813186)(194.38962449,356.24813724)
\curveto(194.41961659,356.13813208)(194.44461656,356.0181322)(194.46462449,355.88813724)
\curveto(194.48461652,355.76813245)(194.5096165,355.64313258)(194.53962449,355.51313724)
\curveto(194.54961646,355.46313276)(194.54961646,355.40813281)(194.53962449,355.34813724)
\curveto(194.53961647,355.29813292)(194.54461646,355.24813297)(194.55462449,355.19813724)
\moveto(193.21962449,354.34313724)
\curveto(193.23961777,354.41313381)(193.24461776,354.49313373)(193.23462449,354.58313724)
\lineto(193.23462449,354.83813724)
\curveto(193.23461777,355.22813299)(193.19961781,355.55813266)(193.12962449,355.82813724)
\curveto(193.09961791,355.90813231)(193.07461793,355.98813223)(193.05462449,356.06813724)
\curveto(193.03461797,356.14813207)(193.009618,356.223132)(192.97962449,356.29313724)
\curveto(192.69961831,356.94313128)(192.25461875,357.39313083)(191.64462449,357.64313724)
\curveto(191.57461943,357.67313055)(191.49961951,357.69313053)(191.41962449,357.70313724)
\lineto(191.17962449,357.76313724)
\curveto(191.09961991,357.78313044)(191.01461999,357.79313043)(190.92462449,357.79313724)
\lineto(190.65462449,357.79313724)
\lineto(190.38462449,357.74813724)
\curveto(190.28462072,357.72813049)(190.18962082,357.70313052)(190.09962449,357.67313724)
\curveto(190.01962099,357.65313057)(189.93962107,357.6231306)(189.85962449,357.58313724)
\curveto(189.78962122,357.56313066)(189.72462128,357.53313069)(189.66462449,357.49313724)
\curveto(189.6046214,357.45313077)(189.54962146,357.41313081)(189.49962449,357.37313724)
\curveto(189.25962175,357.20313102)(189.06462194,356.99813122)(188.91462449,356.75813724)
\curveto(188.76462224,356.5181317)(188.63462237,356.23813198)(188.52462449,355.91813724)
\curveto(188.49462251,355.8181324)(188.47462253,355.71313251)(188.46462449,355.60313724)
\curveto(188.45462255,355.50313272)(188.43962257,355.39813282)(188.41962449,355.28813724)
\curveto(188.4096226,355.24813297)(188.4046226,355.18313304)(188.40462449,355.09313724)
\curveto(188.39462261,355.06313316)(188.38962262,355.02813319)(188.38962449,354.98813724)
\curveto(188.39962261,354.94813327)(188.4046226,354.90313332)(188.40462449,354.85313724)
\lineto(188.40462449,354.55313724)
\curveto(188.4046226,354.45313377)(188.41462259,354.36313386)(188.43462449,354.28313724)
\lineto(188.46462449,354.10313724)
\curveto(188.48462252,354.00313422)(188.49962251,353.90313432)(188.50962449,353.80313724)
\curveto(188.52962248,353.71313451)(188.55962245,353.62813459)(188.59962449,353.54813724)
\curveto(188.69962231,353.30813491)(188.81462219,353.08313514)(188.94462449,352.87313724)
\curveto(189.08462192,352.66313556)(189.25462175,352.48813573)(189.45462449,352.34813724)
\curveto(189.5046215,352.3181359)(189.54962146,352.29313593)(189.58962449,352.27313724)
\curveto(189.62962138,352.25313597)(189.67462133,352.22813599)(189.72462449,352.19813724)
\curveto(189.8046212,352.14813607)(189.88962112,352.10313612)(189.97962449,352.06313724)
\curveto(190.07962093,352.03313619)(190.18462082,352.00313622)(190.29462449,351.97313724)
\curveto(190.34462066,351.95313627)(190.38962062,351.94313628)(190.42962449,351.94313724)
\curveto(190.47962053,351.95313627)(190.52962048,351.95313627)(190.57962449,351.94313724)
\curveto(190.6096204,351.93313629)(190.66962034,351.9231363)(190.75962449,351.91313724)
\curveto(190.85962015,351.90313632)(190.93462007,351.90813631)(190.98462449,351.92813724)
\curveto(191.02461998,351.93813628)(191.06461994,351.93813628)(191.10462449,351.92813724)
\curveto(191.14461986,351.92813629)(191.18461982,351.93813628)(191.22462449,351.95813724)
\curveto(191.3046197,351.97813624)(191.38461962,351.99313623)(191.46462449,352.00313724)
\curveto(191.54461946,352.0231362)(191.61961939,352.04813617)(191.68962449,352.07813724)
\curveto(192.02961898,352.218136)(192.3046187,352.41313581)(192.51462449,352.66313724)
\curveto(192.72461828,352.91313531)(192.89961811,353.20813501)(193.03962449,353.54813724)
\curveto(193.08961792,353.66813455)(193.11961789,353.79313443)(193.12962449,353.92313724)
\curveto(193.14961786,354.06313416)(193.17961783,354.20313402)(193.21962449,354.34313724)
}
}
{
\newrgbcolor{curcolor}{0 0 0}
\pscustom[linestyle=none,fillstyle=solid,fillcolor=curcolor]
{
\newpath
\moveto(198.47290574,358.90313724)
\curveto(199.19290168,358.91312931)(199.79790107,358.82812939)(200.28790574,358.64813724)
\curveto(200.77790009,358.47812974)(201.15789971,358.17313005)(201.42790574,357.73313724)
\curveto(201.49789937,357.6231306)(201.55289932,357.50813071)(201.59290574,357.38813724)
\curveto(201.63289924,357.27813094)(201.6728992,357.15313107)(201.71290574,357.01313724)
\curveto(201.73289914,356.94313128)(201.73789913,356.86813135)(201.72790574,356.78813724)
\curveto(201.71789915,356.7181315)(201.70289917,356.66313156)(201.68290574,356.62313724)
\curveto(201.66289921,356.60313162)(201.63789923,356.58313164)(201.60790574,356.56313724)
\curveto(201.57789929,356.55313167)(201.55289932,356.53813168)(201.53290574,356.51813724)
\curveto(201.48289939,356.49813172)(201.43289944,356.49313173)(201.38290574,356.50313724)
\curveto(201.33289954,356.51313171)(201.28289959,356.51313171)(201.23290574,356.50313724)
\curveto(201.15289972,356.48313174)(201.04789982,356.47813174)(200.91790574,356.48813724)
\curveto(200.78790008,356.50813171)(200.69790017,356.53313169)(200.64790574,356.56313724)
\curveto(200.5679003,356.61313161)(200.51290036,356.67813154)(200.48290574,356.75813724)
\curveto(200.46290041,356.84813137)(200.42790044,356.93313129)(200.37790574,357.01313724)
\curveto(200.28790058,357.17313105)(200.16290071,357.3181309)(200.00290574,357.44813724)
\curveto(199.89290098,357.52813069)(199.7729011,357.58813063)(199.64290574,357.62813724)
\curveto(199.51290136,357.66813055)(199.3729015,357.70813051)(199.22290574,357.74813724)
\curveto(199.1729017,357.76813045)(199.12290175,357.77313045)(199.07290574,357.76313724)
\curveto(199.02290185,357.76313046)(198.9729019,357.76813045)(198.92290574,357.77813724)
\curveto(198.86290201,357.79813042)(198.78790208,357.80813041)(198.69790574,357.80813724)
\curveto(198.60790226,357.80813041)(198.53290234,357.79813042)(198.47290574,357.77813724)
\lineto(198.38290574,357.77813724)
\lineto(198.23290574,357.74813724)
\curveto(198.18290269,357.74813047)(198.13290274,357.74313048)(198.08290574,357.73313724)
\curveto(197.82290305,357.67313055)(197.60790326,357.58813063)(197.43790574,357.47813724)
\curveto(197.2679036,357.36813085)(197.15290372,357.18313104)(197.09290574,356.92313724)
\curveto(197.0729038,356.85313137)(197.0679038,356.78313144)(197.07790574,356.71313724)
\curveto(197.09790377,356.64313158)(197.11790375,356.58313164)(197.13790574,356.53313724)
\curveto(197.19790367,356.38313184)(197.2679036,356.27313195)(197.34790574,356.20313724)
\curveto(197.43790343,356.14313208)(197.54790332,356.07313215)(197.67790574,355.99313724)
\curveto(197.83790303,355.89313233)(198.01790285,355.8181324)(198.21790574,355.76813724)
\curveto(198.41790245,355.72813249)(198.61790225,355.67813254)(198.81790574,355.61813724)
\curveto(198.94790192,355.57813264)(199.07790179,355.54813267)(199.20790574,355.52813724)
\curveto(199.33790153,355.50813271)(199.4679014,355.47813274)(199.59790574,355.43813724)
\curveto(199.80790106,355.37813284)(200.01290086,355.3181329)(200.21290574,355.25813724)
\curveto(200.41290046,355.20813301)(200.61290026,355.14313308)(200.81290574,355.06313724)
\lineto(200.96290574,355.00313724)
\curveto(201.01289986,354.98313324)(201.06289981,354.95813326)(201.11290574,354.92813724)
\curveto(201.31289956,354.80813341)(201.48789938,354.67313355)(201.63790574,354.52313724)
\curveto(201.78789908,354.37313385)(201.91289896,354.18313404)(202.01290574,353.95313724)
\curveto(202.03289884,353.88313434)(202.05289882,353.78813443)(202.07290574,353.66813724)
\curveto(202.09289878,353.59813462)(202.10289877,353.5231347)(202.10290574,353.44313724)
\curveto(202.11289876,353.37313485)(202.11789875,353.29313493)(202.11790574,353.20313724)
\lineto(202.11790574,353.05313724)
\curveto(202.09789877,352.98313524)(202.08789878,352.91313531)(202.08790574,352.84313724)
\curveto(202.08789878,352.77313545)(202.07789879,352.70313552)(202.05790574,352.63313724)
\curveto(202.02789884,352.5231357)(201.99289888,352.4181358)(201.95290574,352.31813724)
\curveto(201.91289896,352.218136)(201.867899,352.12813609)(201.81790574,352.04813724)
\curveto(201.65789921,351.78813643)(201.45289942,351.57813664)(201.20290574,351.41813724)
\curveto(200.95289992,351.26813695)(200.6729002,351.13813708)(200.36290574,351.02813724)
\curveto(200.2729006,350.99813722)(200.17790069,350.97813724)(200.07790574,350.96813724)
\curveto(199.98790088,350.94813727)(199.89790097,350.9231373)(199.80790574,350.89313724)
\curveto(199.70790116,350.87313735)(199.60790126,350.86313736)(199.50790574,350.86313724)
\curveto(199.40790146,350.86313736)(199.30790156,350.85313737)(199.20790574,350.83313724)
\lineto(199.05790574,350.83313724)
\curveto(199.00790186,350.8231374)(198.93790193,350.8181374)(198.84790574,350.81813724)
\curveto(198.75790211,350.8181374)(198.68790218,350.8231374)(198.63790574,350.83313724)
\lineto(198.47290574,350.83313724)
\curveto(198.41290246,350.85313737)(198.34790252,350.86313736)(198.27790574,350.86313724)
\curveto(198.20790266,350.85313737)(198.14790272,350.85813736)(198.09790574,350.87813724)
\curveto(198.04790282,350.88813733)(197.98290289,350.89313733)(197.90290574,350.89313724)
\lineto(197.66290574,350.95313724)
\curveto(197.59290328,350.96313726)(197.51790335,350.98313724)(197.43790574,351.01313724)
\curveto(197.12790374,351.11313711)(196.85790401,351.23813698)(196.62790574,351.38813724)
\curveto(196.39790447,351.53813668)(196.19790467,351.73313649)(196.02790574,351.97313724)
\curveto(195.93790493,352.10313612)(195.86290501,352.23813598)(195.80290574,352.37813724)
\curveto(195.74290513,352.5181357)(195.68790518,352.67313555)(195.63790574,352.84313724)
\curveto(195.61790525,352.90313532)(195.60790526,352.97313525)(195.60790574,353.05313724)
\curveto(195.61790525,353.14313508)(195.63290524,353.21313501)(195.65290574,353.26313724)
\curveto(195.68290519,353.30313492)(195.73290514,353.34313488)(195.80290574,353.38313724)
\curveto(195.85290502,353.40313482)(195.92290495,353.41313481)(196.01290574,353.41313724)
\curveto(196.10290477,353.4231348)(196.19290468,353.4231348)(196.28290574,353.41313724)
\curveto(196.3729045,353.40313482)(196.45790441,353.38813483)(196.53790574,353.36813724)
\curveto(196.62790424,353.35813486)(196.68790418,353.34313488)(196.71790574,353.32313724)
\curveto(196.78790408,353.27313495)(196.83290404,353.19813502)(196.85290574,353.09813724)
\curveto(196.88290399,353.00813521)(196.91790395,352.9231353)(196.95790574,352.84313724)
\curveto(197.05790381,352.6231356)(197.19290368,352.45313577)(197.36290574,352.33313724)
\curveto(197.48290339,352.24313598)(197.61790325,352.17313605)(197.76790574,352.12313724)
\curveto(197.91790295,352.07313615)(198.07790279,352.0231362)(198.24790574,351.97313724)
\lineto(198.56290574,351.92813724)
\lineto(198.65290574,351.92813724)
\curveto(198.72290215,351.90813631)(198.81290206,351.89813632)(198.92290574,351.89813724)
\curveto(199.04290183,351.89813632)(199.14290173,351.90813631)(199.22290574,351.92813724)
\curveto(199.29290158,351.92813629)(199.34790152,351.93313629)(199.38790574,351.94313724)
\curveto(199.44790142,351.95313627)(199.50790136,351.95813626)(199.56790574,351.95813724)
\curveto(199.62790124,351.96813625)(199.68290119,351.97813624)(199.73290574,351.98813724)
\curveto(200.02290085,352.06813615)(200.25290062,352.17313605)(200.42290574,352.30313724)
\curveto(200.59290028,352.43313579)(200.71290016,352.65313557)(200.78290574,352.96313724)
\curveto(200.80290007,353.01313521)(200.80790006,353.06813515)(200.79790574,353.12813724)
\curveto(200.78790008,353.18813503)(200.77790009,353.23313499)(200.76790574,353.26313724)
\curveto(200.71790015,353.45313477)(200.64790022,353.59313463)(200.55790574,353.68313724)
\curveto(200.4679004,353.78313444)(200.35290052,353.87313435)(200.21290574,353.95313724)
\curveto(200.12290075,354.01313421)(200.02290085,354.06313416)(199.91290574,354.10313724)
\lineto(199.58290574,354.22313724)
\curveto(199.55290132,354.23313399)(199.52290135,354.23813398)(199.49290574,354.23813724)
\curveto(199.4729014,354.23813398)(199.44790142,354.24813397)(199.41790574,354.26813724)
\curveto(199.07790179,354.37813384)(198.72290215,354.45813376)(198.35290574,354.50813724)
\curveto(197.99290288,354.56813365)(197.65290322,354.66313356)(197.33290574,354.79313724)
\curveto(197.23290364,354.83313339)(197.13790373,354.86813335)(197.04790574,354.89813724)
\curveto(196.95790391,354.92813329)(196.872904,354.96813325)(196.79290574,355.01813724)
\curveto(196.60290427,355.12813309)(196.42790444,355.25313297)(196.26790574,355.39313724)
\curveto(196.10790476,355.53313269)(195.98290489,355.70813251)(195.89290574,355.91813724)
\curveto(195.86290501,355.98813223)(195.83790503,356.05813216)(195.81790574,356.12813724)
\curveto(195.80790506,356.19813202)(195.79290508,356.27313195)(195.77290574,356.35313724)
\curveto(195.74290513,356.47313175)(195.73290514,356.60813161)(195.74290574,356.75813724)
\curveto(195.75290512,356.9181313)(195.7679051,357.05313117)(195.78790574,357.16313724)
\curveto(195.80790506,357.21313101)(195.81790505,357.25313097)(195.81790574,357.28313724)
\curveto(195.82790504,357.3231309)(195.84290503,357.36313086)(195.86290574,357.40313724)
\curveto(195.95290492,357.63313059)(196.0729048,357.83313039)(196.22290574,358.00313724)
\curveto(196.38290449,358.17313005)(196.56290431,358.3231299)(196.76290574,358.45313724)
\curveto(196.91290396,358.54312968)(197.07790379,358.61312961)(197.25790574,358.66313724)
\curveto(197.43790343,358.7231295)(197.62790324,358.77812944)(197.82790574,358.82813724)
\curveto(197.89790297,358.83812938)(197.96290291,358.84812937)(198.02290574,358.85813724)
\curveto(198.09290278,358.86812935)(198.1679027,358.87812934)(198.24790574,358.88813724)
\curveto(198.27790259,358.89812932)(198.31790255,358.89812932)(198.36790574,358.88813724)
\curveto(198.41790245,358.87812934)(198.45290242,358.88312934)(198.47290574,358.90313724)
}
}
{
\newrgbcolor{curcolor}{0.60000002 0.60000002 0.60000002}
\pscustom[linestyle=none,fillstyle=solid,fillcolor=curcolor]
{
\newpath
\moveto(120.84556474,364.20820438)
\lineto(135.84556474,364.20820438)
\lineto(135.84556474,349.20820438)
\lineto(120.84556474,349.20820438)
\closepath
}
}
{
\newrgbcolor{curcolor}{0 0 0}
\pscustom[linestyle=none,fillstyle=solid,fillcolor=curcolor]
{
\newpath
\moveto(140.83376512,338.64237064)
\lineto(145.73876512,338.64237064)
\lineto(147.02876512,338.64237064)
\curveto(147.13875724,338.64235994)(147.24875713,338.64235994)(147.35876512,338.64237064)
\curveto(147.46875691,338.65235993)(147.55875682,338.63235995)(147.62876512,338.58237064)
\curveto(147.65875672,338.56236002)(147.68375669,338.53736005)(147.70376512,338.50737064)
\curveto(147.72375665,338.47736011)(147.74375663,338.44736014)(147.76376512,338.41737064)
\curveto(147.78375659,338.34736024)(147.79375658,338.23236035)(147.79376512,338.07237064)
\curveto(147.79375658,337.92236066)(147.78375659,337.80736078)(147.76376512,337.72737064)
\curveto(147.72375665,337.587361)(147.63875674,337.50736108)(147.50876512,337.48737064)
\curveto(147.378757,337.47736111)(147.22375715,337.47236111)(147.04376512,337.47237064)
\lineto(145.54376512,337.47237064)
\lineto(143.02376512,337.47237064)
\lineto(142.45376512,337.47237064)
\curveto(142.24376213,337.4823611)(142.08876229,337.45736113)(141.98876512,337.39737064)
\curveto(141.88876249,337.33736125)(141.83376254,337.23236135)(141.82376512,337.08237064)
\lineto(141.82376512,336.61737064)
\lineto(141.82376512,335.08737064)
\curveto(141.82376255,334.97736361)(141.81876256,334.84736374)(141.80876512,334.69737064)
\curveto(141.80876257,334.54736404)(141.81876256,334.42736416)(141.83876512,334.33737064)
\curveto(141.86876251,334.21736437)(141.92876245,334.13736445)(142.01876512,334.09737064)
\curveto(142.05876232,334.07736451)(142.12876225,334.05736453)(142.22876512,334.03737064)
\lineto(142.37876512,334.03737064)
\curveto(142.41876196,334.02736456)(142.45876192,334.02236456)(142.49876512,334.02237064)
\curveto(142.54876183,334.03236455)(142.59876178,334.03736455)(142.64876512,334.03737064)
\lineto(143.15876512,334.03737064)
\lineto(146.09876512,334.03737064)
\lineto(146.39876512,334.03737064)
\curveto(146.50875787,334.04736454)(146.61875776,334.04736454)(146.72876512,334.03737064)
\curveto(146.84875753,334.03736455)(146.95375742,334.02736456)(147.04376512,334.00737064)
\curveto(147.14375723,333.99736459)(147.21875716,333.97736461)(147.26876512,333.94737064)
\curveto(147.29875708,333.92736466)(147.32375705,333.8823647)(147.34376512,333.81237064)
\curveto(147.36375701,333.74236484)(147.378757,333.66736492)(147.38876512,333.58737064)
\curveto(147.39875698,333.50736508)(147.39875698,333.42236516)(147.38876512,333.33237064)
\curveto(147.38875699,333.25236533)(147.378757,333.1823654)(147.35876512,333.12237064)
\curveto(147.33875704,333.03236555)(147.29375708,332.96736562)(147.22376512,332.92737064)
\curveto(147.20375717,332.90736568)(147.1737572,332.89236569)(147.13376512,332.88237064)
\curveto(147.10375727,332.8823657)(147.0737573,332.87736571)(147.04376512,332.86737064)
\lineto(146.95376512,332.86737064)
\curveto(146.90375747,332.85736573)(146.85375752,332.85236573)(146.80376512,332.85237064)
\curveto(146.75375762,332.86236572)(146.70375767,332.86736572)(146.65376512,332.86737064)
\lineto(146.09876512,332.86737064)
\lineto(142.93376512,332.86737064)
\lineto(142.57376512,332.86737064)
\curveto(142.46376191,332.87736571)(142.35876202,332.87236571)(142.25876512,332.85237064)
\curveto(142.15876222,332.84236574)(142.06876231,332.81736577)(141.98876512,332.77737064)
\curveto(141.91876246,332.73736585)(141.86876251,332.66736592)(141.83876512,332.56737064)
\curveto(141.81876256,332.50736608)(141.80876257,332.43736615)(141.80876512,332.35737064)
\curveto(141.81876256,332.27736631)(141.82376255,332.19736639)(141.82376512,332.11737064)
\lineto(141.82376512,331.27737064)
\lineto(141.82376512,329.85237064)
\curveto(141.82376255,329.71236887)(141.82876255,329.582369)(141.83876512,329.46237064)
\curveto(141.84876253,329.35236923)(141.88876249,329.27236931)(141.95876512,329.22237064)
\curveto(142.02876235,329.17236941)(142.10876227,329.14236944)(142.19876512,329.13237064)
\lineto(142.49876512,329.13237064)
\lineto(143.45876512,329.13237064)
\lineto(146.23376512,329.13237064)
\lineto(147.08876512,329.13237064)
\lineto(147.32876512,329.13237064)
\curveto(147.40875697,329.14236944)(147.4787569,329.13736945)(147.53876512,329.11737064)
\curveto(147.65875672,329.07736951)(147.73875664,329.02236956)(147.77876512,328.95237064)
\curveto(147.79875658,328.92236966)(147.81375656,328.87236971)(147.82376512,328.80237064)
\curveto(147.83375654,328.73236985)(147.83875654,328.65736993)(147.83876512,328.57737064)
\curveto(147.84875653,328.50737008)(147.84875653,328.43237015)(147.83876512,328.35237064)
\curveto(147.82875655,328.2823703)(147.81875656,328.22737036)(147.80876512,328.18737064)
\curveto(147.76875661,328.10737048)(147.72375665,328.05237053)(147.67376512,328.02237064)
\curveto(147.61375676,327.9823706)(147.53375684,327.96237062)(147.43376512,327.96237064)
\lineto(147.16376512,327.96237064)
\lineto(146.11376512,327.96237064)
\lineto(142.12376512,327.96237064)
\lineto(141.07376512,327.96237064)
\curveto(140.93376344,327.96237062)(140.81376356,327.96737062)(140.71376512,327.97737064)
\curveto(140.61376376,327.99737059)(140.53876384,328.04737054)(140.48876512,328.12737064)
\curveto(140.44876393,328.1873704)(140.42876395,328.26237032)(140.42876512,328.35237064)
\lineto(140.42876512,328.63737064)
\lineto(140.42876512,329.68737064)
\lineto(140.42876512,333.70737064)
\lineto(140.42876512,337.06737064)
\lineto(140.42876512,337.99737064)
\lineto(140.42876512,338.26737064)
\curveto(140.42876395,338.35736023)(140.44876393,338.42736016)(140.48876512,338.47737064)
\curveto(140.52876385,338.54736004)(140.60376377,338.59735999)(140.71376512,338.62737064)
\curveto(140.73376364,338.63735995)(140.75376362,338.63735995)(140.77376512,338.62737064)
\curveto(140.79376358,338.62735996)(140.81376356,338.63235995)(140.83376512,338.64237064)
}
}
{
\newrgbcolor{curcolor}{0 0 0}
\pscustom[linestyle=none,fillstyle=solid,fillcolor=curcolor]
{
\newpath
\moveto(151.77368699,335.86737064)
\curveto(152.49368293,335.87736271)(153.09868232,335.79236279)(153.58868699,335.61237064)
\curveto(154.07868134,335.44236314)(154.45868096,335.13736345)(154.72868699,334.69737064)
\curveto(154.79868062,334.587364)(154.85368057,334.47236411)(154.89368699,334.35237064)
\curveto(154.93368049,334.24236434)(154.97368045,334.11736447)(155.01368699,333.97737064)
\curveto(155.03368039,333.90736468)(155.03868038,333.83236475)(155.02868699,333.75237064)
\curveto(155.0186804,333.6823649)(155.00368042,333.62736496)(154.98368699,333.58737064)
\curveto(154.96368046,333.56736502)(154.93868048,333.54736504)(154.90868699,333.52737064)
\curveto(154.87868054,333.51736507)(154.85368057,333.50236508)(154.83368699,333.48237064)
\curveto(154.78368064,333.46236512)(154.73368069,333.45736513)(154.68368699,333.46737064)
\curveto(154.63368079,333.47736511)(154.58368084,333.47736511)(154.53368699,333.46737064)
\curveto(154.45368097,333.44736514)(154.34868107,333.44236514)(154.21868699,333.45237064)
\curveto(154.08868133,333.47236511)(153.99868142,333.49736509)(153.94868699,333.52737064)
\curveto(153.86868155,333.57736501)(153.81368161,333.64236494)(153.78368699,333.72237064)
\curveto(153.76368166,333.81236477)(153.72868169,333.89736469)(153.67868699,333.97737064)
\curveto(153.58868183,334.13736445)(153.46368196,334.2823643)(153.30368699,334.41237064)
\curveto(153.19368223,334.49236409)(153.07368235,334.55236403)(152.94368699,334.59237064)
\curveto(152.81368261,334.63236395)(152.67368275,334.67236391)(152.52368699,334.71237064)
\curveto(152.47368295,334.73236385)(152.423683,334.73736385)(152.37368699,334.72737064)
\curveto(152.3236831,334.72736386)(152.27368315,334.73236385)(152.22368699,334.74237064)
\curveto(152.16368326,334.76236382)(152.08868333,334.77236381)(151.99868699,334.77237064)
\curveto(151.90868351,334.77236381)(151.83368359,334.76236382)(151.77368699,334.74237064)
\lineto(151.68368699,334.74237064)
\lineto(151.53368699,334.71237064)
\curveto(151.48368394,334.71236387)(151.43368399,334.70736388)(151.38368699,334.69737064)
\curveto(151.1236843,334.63736395)(150.90868451,334.55236403)(150.73868699,334.44237064)
\curveto(150.56868485,334.33236425)(150.45368497,334.14736444)(150.39368699,333.88737064)
\curveto(150.37368505,333.81736477)(150.36868505,333.74736484)(150.37868699,333.67737064)
\curveto(150.39868502,333.60736498)(150.418685,333.54736504)(150.43868699,333.49737064)
\curveto(150.49868492,333.34736524)(150.56868485,333.23736535)(150.64868699,333.16737064)
\curveto(150.73868468,333.10736548)(150.84868457,333.03736555)(150.97868699,332.95737064)
\curveto(151.13868428,332.85736573)(151.3186841,332.7823658)(151.51868699,332.73237064)
\curveto(151.7186837,332.69236589)(151.9186835,332.64236594)(152.11868699,332.58237064)
\curveto(152.24868317,332.54236604)(152.37868304,332.51236607)(152.50868699,332.49237064)
\curveto(152.63868278,332.47236611)(152.76868265,332.44236614)(152.89868699,332.40237064)
\curveto(153.10868231,332.34236624)(153.31368211,332.2823663)(153.51368699,332.22237064)
\curveto(153.71368171,332.17236641)(153.91368151,332.10736648)(154.11368699,332.02737064)
\lineto(154.26368699,331.96737064)
\curveto(154.31368111,331.94736664)(154.36368106,331.92236666)(154.41368699,331.89237064)
\curveto(154.61368081,331.77236681)(154.78868063,331.63736695)(154.93868699,331.48737064)
\curveto(155.08868033,331.33736725)(155.21368021,331.14736744)(155.31368699,330.91737064)
\curveto(155.33368009,330.84736774)(155.35368007,330.75236783)(155.37368699,330.63237064)
\curveto(155.39368003,330.56236802)(155.40368002,330.4873681)(155.40368699,330.40737064)
\curveto(155.41368001,330.33736825)(155.41868,330.25736833)(155.41868699,330.16737064)
\lineto(155.41868699,330.01737064)
\curveto(155.39868002,329.94736864)(155.38868003,329.87736871)(155.38868699,329.80737064)
\curveto(155.38868003,329.73736885)(155.37868004,329.66736892)(155.35868699,329.59737064)
\curveto(155.32868009,329.4873691)(155.29368013,329.3823692)(155.25368699,329.28237064)
\curveto(155.21368021,329.1823694)(155.16868025,329.09236949)(155.11868699,329.01237064)
\curveto(154.95868046,328.75236983)(154.75368067,328.54237004)(154.50368699,328.38237064)
\curveto(154.25368117,328.23237035)(153.97368145,328.10237048)(153.66368699,327.99237064)
\curveto(153.57368185,327.96237062)(153.47868194,327.94237064)(153.37868699,327.93237064)
\curveto(153.28868213,327.91237067)(153.19868222,327.8873707)(153.10868699,327.85737064)
\curveto(153.00868241,327.83737075)(152.90868251,327.82737076)(152.80868699,327.82737064)
\curveto(152.70868271,327.82737076)(152.60868281,327.81737077)(152.50868699,327.79737064)
\lineto(152.35868699,327.79737064)
\curveto(152.30868311,327.7873708)(152.23868318,327.7823708)(152.14868699,327.78237064)
\curveto(152.05868336,327.7823708)(151.98868343,327.7873708)(151.93868699,327.79737064)
\lineto(151.77368699,327.79737064)
\curveto(151.71368371,327.81737077)(151.64868377,327.82737076)(151.57868699,327.82737064)
\curveto(151.50868391,327.81737077)(151.44868397,327.82237076)(151.39868699,327.84237064)
\curveto(151.34868407,327.85237073)(151.28368414,327.85737073)(151.20368699,327.85737064)
\lineto(150.96368699,327.91737064)
\curveto(150.89368453,327.92737066)(150.8186846,327.94737064)(150.73868699,327.97737064)
\curveto(150.42868499,328.07737051)(150.15868526,328.20237038)(149.92868699,328.35237064)
\curveto(149.69868572,328.50237008)(149.49868592,328.69736989)(149.32868699,328.93737064)
\curveto(149.23868618,329.06736952)(149.16368626,329.20236938)(149.10368699,329.34237064)
\curveto(149.04368638,329.4823691)(148.98868643,329.63736895)(148.93868699,329.80737064)
\curveto(148.9186865,329.86736872)(148.90868651,329.93736865)(148.90868699,330.01737064)
\curveto(148.9186865,330.10736848)(148.93368649,330.17736841)(148.95368699,330.22737064)
\curveto(148.98368644,330.26736832)(149.03368639,330.30736828)(149.10368699,330.34737064)
\curveto(149.15368627,330.36736822)(149.2236862,330.37736821)(149.31368699,330.37737064)
\curveto(149.40368602,330.3873682)(149.49368593,330.3873682)(149.58368699,330.37737064)
\curveto(149.67368575,330.36736822)(149.75868566,330.35236823)(149.83868699,330.33237064)
\curveto(149.92868549,330.32236826)(149.98868543,330.30736828)(150.01868699,330.28737064)
\curveto(150.08868533,330.23736835)(150.13368529,330.16236842)(150.15368699,330.06237064)
\curveto(150.18368524,329.97236861)(150.2186852,329.8873687)(150.25868699,329.80737064)
\curveto(150.35868506,329.587369)(150.49368493,329.41736917)(150.66368699,329.29737064)
\curveto(150.78368464,329.20736938)(150.9186845,329.13736945)(151.06868699,329.08737064)
\curveto(151.2186842,329.03736955)(151.37868404,328.9873696)(151.54868699,328.93737064)
\lineto(151.86368699,328.89237064)
\lineto(151.95368699,328.89237064)
\curveto(152.0236834,328.87236971)(152.11368331,328.86236972)(152.22368699,328.86237064)
\curveto(152.34368308,328.86236972)(152.44368298,328.87236971)(152.52368699,328.89237064)
\curveto(152.59368283,328.89236969)(152.64868277,328.89736969)(152.68868699,328.90737064)
\curveto(152.74868267,328.91736967)(152.80868261,328.92236966)(152.86868699,328.92237064)
\curveto(152.92868249,328.93236965)(152.98368244,328.94236964)(153.03368699,328.95237064)
\curveto(153.3236821,329.03236955)(153.55368187,329.13736945)(153.72368699,329.26737064)
\curveto(153.89368153,329.39736919)(154.01368141,329.61736897)(154.08368699,329.92737064)
\curveto(154.10368132,329.97736861)(154.10868131,330.03236855)(154.09868699,330.09237064)
\curveto(154.08868133,330.15236843)(154.07868134,330.19736839)(154.06868699,330.22737064)
\curveto(154.0186814,330.41736817)(153.94868147,330.55736803)(153.85868699,330.64737064)
\curveto(153.76868165,330.74736784)(153.65368177,330.83736775)(153.51368699,330.91737064)
\curveto(153.423682,330.97736761)(153.3236821,331.02736756)(153.21368699,331.06737064)
\lineto(152.88368699,331.18737064)
\curveto(152.85368257,331.19736739)(152.8236826,331.20236738)(152.79368699,331.20237064)
\curveto(152.77368265,331.20236738)(152.74868267,331.21236737)(152.71868699,331.23237064)
\curveto(152.37868304,331.34236724)(152.0236834,331.42236716)(151.65368699,331.47237064)
\curveto(151.29368413,331.53236705)(150.95368447,331.62736696)(150.63368699,331.75737064)
\curveto(150.53368489,331.79736679)(150.43868498,331.83236675)(150.34868699,331.86237064)
\curveto(150.25868516,331.89236669)(150.17368525,331.93236665)(150.09368699,331.98237064)
\curveto(149.90368552,332.09236649)(149.72868569,332.21736637)(149.56868699,332.35737064)
\curveto(149.40868601,332.49736609)(149.28368614,332.67236591)(149.19368699,332.88237064)
\curveto(149.16368626,332.95236563)(149.13868628,333.02236556)(149.11868699,333.09237064)
\curveto(149.10868631,333.16236542)(149.09368633,333.23736535)(149.07368699,333.31737064)
\curveto(149.04368638,333.43736515)(149.03368639,333.57236501)(149.04368699,333.72237064)
\curveto(149.05368637,333.8823647)(149.06868635,334.01736457)(149.08868699,334.12737064)
\curveto(149.10868631,334.17736441)(149.1186863,334.21736437)(149.11868699,334.24737064)
\curveto(149.12868629,334.2873643)(149.14368628,334.32736426)(149.16368699,334.36737064)
\curveto(149.25368617,334.59736399)(149.37368605,334.79736379)(149.52368699,334.96737064)
\curveto(149.68368574,335.13736345)(149.86368556,335.2873633)(150.06368699,335.41737064)
\curveto(150.21368521,335.50736308)(150.37868504,335.57736301)(150.55868699,335.62737064)
\curveto(150.73868468,335.6873629)(150.92868449,335.74236284)(151.12868699,335.79237064)
\curveto(151.19868422,335.80236278)(151.26368416,335.81236277)(151.32368699,335.82237064)
\curveto(151.39368403,335.83236275)(151.46868395,335.84236274)(151.54868699,335.85237064)
\curveto(151.57868384,335.86236272)(151.6186838,335.86236272)(151.66868699,335.85237064)
\curveto(151.7186837,335.84236274)(151.75368367,335.84736274)(151.77368699,335.86737064)
}
}
{
\newrgbcolor{curcolor}{0 0 0}
\pscustom[linestyle=none,fillstyle=solid,fillcolor=curcolor]
{
\newpath
\moveto(164.26868699,332.02737064)
\curveto(164.27867864,331.97736661)(164.28367864,331.91236667)(164.28368699,331.83237064)
\curveto(164.28367864,331.75236683)(164.27867864,331.6873669)(164.26868699,331.63737064)
\curveto(164.24867867,331.587367)(164.24367868,331.53736705)(164.25368699,331.48737064)
\curveto(164.26367866,331.44736714)(164.26367866,331.40736718)(164.25368699,331.36737064)
\curveto(164.25367867,331.29736729)(164.24867867,331.24236734)(164.23868699,331.20237064)
\curveto(164.2186787,331.11236747)(164.20367872,331.02236756)(164.19368699,330.93237064)
\curveto(164.19367873,330.84236774)(164.18367874,330.75236783)(164.16368699,330.66237064)
\lineto(164.10368699,330.42237064)
\curveto(164.08367884,330.35236823)(164.05867886,330.27736831)(164.02868699,330.19737064)
\curveto(163.90867901,329.82736876)(163.74367918,329.49236909)(163.53368699,329.19237064)
\curveto(163.47367945,329.10236948)(163.40867951,329.01236957)(163.33868699,328.92237064)
\curveto(163.26867965,328.84236974)(163.19367973,328.76736982)(163.11368699,328.69737064)
\lineto(163.03868699,328.62237064)
\curveto(162.96867995,328.57237001)(162.90368002,328.52237006)(162.84368699,328.47237064)
\curveto(162.78368014,328.42237016)(162.71368021,328.37237021)(162.63368699,328.32237064)
\curveto(162.5236804,328.24237034)(162.39868052,328.17237041)(162.25868699,328.11237064)
\curveto(162.12868079,328.06237052)(161.99368093,328.01237057)(161.85368699,327.96237064)
\curveto(161.77368115,327.94237064)(161.69368123,327.92737066)(161.61368699,327.91737064)
\curveto(161.54368138,327.90737068)(161.46868145,327.89237069)(161.38868699,327.87237064)
\lineto(161.32868699,327.87237064)
\curveto(161.3186816,327.86237072)(161.30368162,327.85737073)(161.28368699,327.85737064)
\curveto(161.19368173,327.83737075)(161.05868186,327.82737076)(160.87868699,327.82737064)
\curveto(160.70868221,327.81737077)(160.57368235,327.82237076)(160.47368699,327.84237064)
\lineto(160.39868699,327.84237064)
\curveto(160.32868259,327.85237073)(160.26368266,327.86237072)(160.20368699,327.87237064)
\curveto(160.14368278,327.87237071)(160.08368284,327.8823707)(160.02368699,327.90237064)
\curveto(159.85368307,327.95237063)(159.69368323,327.99737059)(159.54368699,328.03737064)
\curveto(159.39368353,328.07737051)(159.25368367,328.13737045)(159.12368699,328.21737064)
\curveto(158.96368396,328.30737028)(158.8236841,328.40237018)(158.70368699,328.50237064)
\curveto(158.66368426,328.53237005)(158.60368432,328.57237001)(158.52368699,328.62237064)
\curveto(158.44368448,328.6823699)(158.36868455,328.6873699)(158.29868699,328.63737064)
\curveto(158.25868466,328.60736998)(158.23868468,328.56737002)(158.23868699,328.51737064)
\curveto(158.23868468,328.46737012)(158.22868469,328.41237017)(158.20868699,328.35237064)
\curveto(158.19868472,328.32237026)(158.19868472,328.2873703)(158.20868699,328.24737064)
\curveto(158.2186847,328.21737037)(158.2186847,328.1823704)(158.20868699,328.14237064)
\curveto(158.18868473,328.0823705)(158.17868474,328.01737057)(158.17868699,327.94737064)
\curveto(158.18868473,327.86737072)(158.19368473,327.79737079)(158.19368699,327.73737064)
\lineto(158.19368699,325.93737064)
\lineto(158.19368699,325.50237064)
\curveto(158.19368473,325.35237323)(158.16368476,325.23737335)(158.10368699,325.15737064)
\curveto(158.05368487,325.0873735)(157.97368495,325.05237353)(157.86368699,325.05237064)
\curveto(157.75368517,325.04237354)(157.64368528,325.03737355)(157.53368699,325.03737064)
\lineto(157.29368699,325.03737064)
\curveto(157.2236857,325.05737353)(157.16368576,325.07737351)(157.11368699,325.09737064)
\curveto(157.07368585,325.11737347)(157.03868588,325.15237343)(157.00868699,325.20237064)
\curveto(156.95868596,325.27237331)(156.93368599,325.3823732)(156.93368699,325.53237064)
\curveto(156.94368598,325.6823729)(156.94868597,325.81237277)(156.94868699,325.92237064)
\lineto(156.94868699,334.92237064)
\lineto(156.94868699,335.28237064)
\curveto(156.95868596,335.41236317)(156.98868593,335.51736307)(157.03868699,335.59737064)
\curveto(157.06868585,335.63736295)(157.13368579,335.66736292)(157.23368699,335.68737064)
\curveto(157.34368558,335.71736287)(157.45868546,335.72736286)(157.57868699,335.71737064)
\curveto(157.69868522,335.71736287)(157.80868511,335.70236288)(157.90868699,335.67237064)
\curveto(158.0186849,335.65236293)(158.08868483,335.62236296)(158.11868699,335.58237064)
\curveto(158.15868476,335.53236305)(158.17868474,335.47236311)(158.17868699,335.40237064)
\curveto(158.18868473,335.33236325)(158.20868471,335.26236332)(158.23868699,335.19237064)
\curveto(158.25868466,335.16236342)(158.27368465,335.13736345)(158.28368699,335.11737064)
\curveto(158.30368462,335.10736348)(158.3236846,335.09236349)(158.34368699,335.07237064)
\curveto(158.45368447,335.06236352)(158.54368438,335.09736349)(158.61368699,335.17737064)
\curveto(158.69368423,335.25736333)(158.76868415,335.32236326)(158.83868699,335.37237064)
\curveto(159.09868382,335.55236303)(159.40868351,335.69236289)(159.76868699,335.79237064)
\curveto(159.85868306,335.81236277)(159.94868297,335.82736276)(160.03868699,335.83737064)
\curveto(160.13868278,335.84736274)(160.23868268,335.86236272)(160.33868699,335.88237064)
\curveto(160.37868254,335.89236269)(160.42868249,335.89236269)(160.48868699,335.88237064)
\curveto(160.54868237,335.87236271)(160.58868233,335.87736271)(160.60868699,335.89737064)
\curveto(161.03868188,335.90736268)(161.4186815,335.86236272)(161.74868699,335.76237064)
\curveto(162.07868084,335.67236291)(162.37368055,335.54236304)(162.63368699,335.37237064)
\lineto(162.78368699,335.25237064)
\curveto(162.83368009,335.22236336)(162.88368004,335.1873634)(162.93368699,335.14737064)
\curveto(162.95367997,335.12736346)(162.96867995,335.10736348)(162.97868699,335.08737064)
\curveto(162.99867992,335.07736351)(163.0186799,335.06236352)(163.03868699,335.04237064)
\curveto(163.08867983,334.99236359)(163.14367978,334.93736365)(163.20368699,334.87737064)
\curveto(163.26367966,334.81736377)(163.3186796,334.75736383)(163.36868699,334.69737064)
\curveto(163.48867943,334.52736406)(163.61367931,334.34236424)(163.74368699,334.14237064)
\curveto(163.8236791,334.01236457)(163.88867903,333.86736472)(163.93868699,333.70737064)
\curveto(163.99867892,333.54736504)(164.05367887,333.3873652)(164.10368699,333.22737064)
\curveto(164.1236788,333.14736544)(164.13867878,333.06236552)(164.14868699,332.97237064)
\curveto(164.16867875,332.8823657)(164.18867873,332.79736579)(164.20868699,332.71737064)
\lineto(164.20868699,332.59737064)
\curveto(164.2186787,332.56736602)(164.2236787,332.53736605)(164.22368699,332.50737064)
\curveto(164.24367868,332.45736613)(164.24867867,332.40236618)(164.23868699,332.34237064)
\curveto(164.23867868,332.2823663)(164.24867867,332.22736636)(164.26868699,332.17737064)
\lineto(164.26868699,332.02737064)
\moveto(162.93368699,331.62237064)
\curveto(162.95367997,331.67236691)(162.95867996,331.73236685)(162.94868699,331.80237064)
\curveto(162.93867998,331.8823667)(162.93367999,331.95236663)(162.93368699,332.01237064)
\curveto(162.93367999,332.1823664)(162.92368,332.34236624)(162.90368699,332.49237064)
\curveto(162.89368003,332.64236594)(162.86368006,332.7873658)(162.81368699,332.92737064)
\lineto(162.75368699,333.10737064)
\curveto(162.74368018,333.17736541)(162.7236802,333.24236534)(162.69368699,333.30237064)
\curveto(162.58368034,333.57236501)(162.40868051,333.83236475)(162.16868699,334.08237064)
\curveto(161.93868098,334.33236425)(161.7186812,334.50236408)(161.50868699,334.59237064)
\curveto(161.42868149,334.63236395)(161.34368158,334.66236392)(161.25368699,334.68237064)
\curveto(161.17368175,334.70236388)(161.08868183,334.72736386)(160.99868699,334.75737064)
\curveto(160.90868201,334.77736381)(160.80368212,334.7873638)(160.68368699,334.78737064)
\lineto(160.35368699,334.78737064)
\curveto(160.33368259,334.76736382)(160.29368263,334.75736383)(160.23368699,334.75737064)
\curveto(160.18368274,334.76736382)(160.13868278,334.76736382)(160.09868699,334.75737064)
\lineto(159.82868699,334.69737064)
\curveto(159.74868317,334.67736391)(159.66868325,334.64736394)(159.58868699,334.60737064)
\curveto(159.26868365,334.46736412)(159.00368392,334.26236432)(158.79368699,333.99237064)
\curveto(158.59368433,333.73236485)(158.43868448,333.42736516)(158.32868699,333.07737064)
\curveto(158.28868463,332.96736562)(158.25868466,332.85736573)(158.23868699,332.74737064)
\curveto(158.22868469,332.63736595)(158.21368471,332.52736606)(158.19368699,332.41737064)
\curveto(158.18368474,332.37736621)(158.17868474,332.33736625)(158.17868699,332.29737064)
\curveto(158.17868474,332.26736632)(158.17368475,332.23236635)(158.16368699,332.19237064)
\lineto(158.16368699,332.07237064)
\curveto(158.15368477,332.02236656)(158.14868477,331.94736664)(158.14868699,331.84737064)
\curveto(158.14868477,331.75736683)(158.15368477,331.6873669)(158.16368699,331.63737064)
\lineto(158.16368699,331.51737064)
\curveto(158.17368475,331.47736711)(158.17868474,331.43736715)(158.17868699,331.39737064)
\curveto(158.17868474,331.35736723)(158.18368474,331.32236726)(158.19368699,331.29237064)
\curveto(158.20368472,331.26236732)(158.20868471,331.23236735)(158.20868699,331.20237064)
\curveto(158.20868471,331.17236741)(158.21368471,331.13736745)(158.22368699,331.09737064)
\curveto(158.24368468,331.01736757)(158.25868466,330.93736765)(158.26868699,330.85737064)
\lineto(158.32868699,330.61737064)
\curveto(158.43868448,330.27736831)(158.58868433,329.97736861)(158.77868699,329.71737064)
\curveto(158.97868394,329.46736912)(159.23868368,329.27236931)(159.55868699,329.13237064)
\curveto(159.74868317,329.05236953)(159.94368298,328.99236959)(160.14368699,328.95237064)
\curveto(160.18368274,328.93236965)(160.2236827,328.92236966)(160.26368699,328.92237064)
\curveto(160.30368262,328.93236965)(160.34368258,328.93236965)(160.38368699,328.92237064)
\lineto(160.50368699,328.92237064)
\curveto(160.57368235,328.90236968)(160.64368228,328.90236968)(160.71368699,328.92237064)
\lineto(160.83368699,328.92237064)
\curveto(160.94368198,328.94236964)(161.04868187,328.95736963)(161.14868699,328.96737064)
\curveto(161.24868167,328.97736961)(161.34868157,329.00236958)(161.44868699,329.04237064)
\curveto(161.75868116,329.17236941)(162.00868091,329.34236924)(162.19868699,329.55237064)
\curveto(162.39868052,329.77236881)(162.56368036,330.03736855)(162.69368699,330.34737064)
\curveto(162.74368018,330.4873681)(162.77868014,330.62736796)(162.79868699,330.76737064)
\curveto(162.82868009,330.91736767)(162.86368006,331.07236751)(162.90368699,331.23237064)
\curveto(162.91368001,331.2823673)(162.91868,331.32736726)(162.91868699,331.36737064)
\curveto(162.91868,331.40736718)(162.92368,331.45236713)(162.93368699,331.50237064)
\lineto(162.93368699,331.62237064)
}
}
{
\newrgbcolor{curcolor}{0 0 0}
\pscustom[linestyle=none,fillstyle=solid,fillcolor=curcolor]
{
\newpath
\moveto(172.63493699,328.51737064)
\curveto(172.66492916,328.35737023)(172.64992918,328.22237036)(172.58993699,328.11237064)
\curveto(172.5299293,328.01237057)(172.44992938,327.93737065)(172.34993699,327.88737064)
\curveto(172.29992953,327.86737072)(172.24492958,327.85737073)(172.18493699,327.85737064)
\curveto(172.13492969,327.85737073)(172.07992975,327.84737074)(172.01993699,327.82737064)
\curveto(171.79993003,327.77737081)(171.57993025,327.79237079)(171.35993699,327.87237064)
\curveto(171.14993068,327.94237064)(171.00493082,328.03237055)(170.92493699,328.14237064)
\curveto(170.87493095,328.21237037)(170.829931,328.29237029)(170.78993699,328.38237064)
\curveto(170.74993108,328.4823701)(170.69993113,328.56237002)(170.63993699,328.62237064)
\curveto(170.61993121,328.64236994)(170.59493123,328.66236992)(170.56493699,328.68237064)
\curveto(170.54493128,328.70236988)(170.51493131,328.70736988)(170.47493699,328.69737064)
\curveto(170.36493146,328.66736992)(170.25993157,328.61236997)(170.15993699,328.53237064)
\curveto(170.06993176,328.45237013)(169.97993185,328.3823702)(169.88993699,328.32237064)
\curveto(169.75993207,328.24237034)(169.61993221,328.16737042)(169.46993699,328.09737064)
\curveto(169.31993251,328.03737055)(169.15993267,327.9823706)(168.98993699,327.93237064)
\curveto(168.88993294,327.90237068)(168.77993305,327.8823707)(168.65993699,327.87237064)
\curveto(168.54993328,327.86237072)(168.43993339,327.84737074)(168.32993699,327.82737064)
\curveto(168.27993355,327.81737077)(168.23493359,327.81237077)(168.19493699,327.81237064)
\lineto(168.08993699,327.81237064)
\curveto(167.97993385,327.79237079)(167.87493395,327.79237079)(167.77493699,327.81237064)
\lineto(167.63993699,327.81237064)
\curveto(167.58993424,327.82237076)(167.53993429,327.82737076)(167.48993699,327.82737064)
\curveto(167.43993439,327.82737076)(167.39493443,327.83737075)(167.35493699,327.85737064)
\curveto(167.31493451,327.86737072)(167.27993455,327.87237071)(167.24993699,327.87237064)
\curveto(167.2299346,327.86237072)(167.20493462,327.86237072)(167.17493699,327.87237064)
\lineto(166.93493699,327.93237064)
\curveto(166.85493497,327.94237064)(166.77993505,327.96237062)(166.70993699,327.99237064)
\curveto(166.40993542,328.12237046)(166.16493566,328.26737032)(165.97493699,328.42737064)
\curveto(165.79493603,328.59736999)(165.64493618,328.83236975)(165.52493699,329.13237064)
\curveto(165.43493639,329.35236923)(165.38993644,329.61736897)(165.38993699,329.92737064)
\lineto(165.38993699,330.24237064)
\curveto(165.39993643,330.29236829)(165.40493642,330.34236824)(165.40493699,330.39237064)
\lineto(165.43493699,330.57237064)
\lineto(165.55493699,330.90237064)
\curveto(165.59493623,331.01236757)(165.64493618,331.11236747)(165.70493699,331.20237064)
\curveto(165.88493594,331.49236709)(166.1299357,331.70736688)(166.43993699,331.84737064)
\curveto(166.74993508,331.9873666)(167.08993474,332.11236647)(167.45993699,332.22237064)
\curveto(167.59993423,332.26236632)(167.74493408,332.29236629)(167.89493699,332.31237064)
\curveto(168.04493378,332.33236625)(168.19493363,332.35736623)(168.34493699,332.38737064)
\curveto(168.41493341,332.40736618)(168.47993335,332.41736617)(168.53993699,332.41737064)
\curveto(168.60993322,332.41736617)(168.68493314,332.42736616)(168.76493699,332.44737064)
\curveto(168.83493299,332.46736612)(168.90493292,332.47736611)(168.97493699,332.47737064)
\curveto(169.04493278,332.4873661)(169.11993271,332.50236608)(169.19993699,332.52237064)
\curveto(169.44993238,332.582366)(169.68493214,332.63236595)(169.90493699,332.67237064)
\curveto(170.1249317,332.72236586)(170.29993153,332.83736575)(170.42993699,333.01737064)
\curveto(170.48993134,333.09736549)(170.53993129,333.19736539)(170.57993699,333.31737064)
\curveto(170.61993121,333.44736514)(170.61993121,333.587365)(170.57993699,333.73737064)
\curveto(170.51993131,333.97736461)(170.4299314,334.16736442)(170.30993699,334.30737064)
\curveto(170.19993163,334.44736414)(170.03993179,334.55736403)(169.82993699,334.63737064)
\curveto(169.70993212,334.6873639)(169.56493226,334.72236386)(169.39493699,334.74237064)
\curveto(169.23493259,334.76236382)(169.06493276,334.77236381)(168.88493699,334.77237064)
\curveto(168.70493312,334.77236381)(168.5299333,334.76236382)(168.35993699,334.74237064)
\curveto(168.18993364,334.72236386)(168.04493378,334.69236389)(167.92493699,334.65237064)
\curveto(167.75493407,334.59236399)(167.58993424,334.50736408)(167.42993699,334.39737064)
\curveto(167.34993448,334.33736425)(167.27493455,334.25736433)(167.20493699,334.15737064)
\curveto(167.14493468,334.06736452)(167.08993474,333.96736462)(167.03993699,333.85737064)
\curveto(167.00993482,333.77736481)(166.97993485,333.69236489)(166.94993699,333.60237064)
\curveto(166.9299349,333.51236507)(166.88493494,333.44236514)(166.81493699,333.39237064)
\curveto(166.77493505,333.36236522)(166.70493512,333.33736525)(166.60493699,333.31737064)
\curveto(166.51493531,333.30736528)(166.41993541,333.30236528)(166.31993699,333.30237064)
\curveto(166.21993561,333.30236528)(166.11993571,333.30736528)(166.01993699,333.31737064)
\curveto(165.9299359,333.33736525)(165.86493596,333.36236522)(165.82493699,333.39237064)
\curveto(165.78493604,333.42236516)(165.75493607,333.47236511)(165.73493699,333.54237064)
\curveto(165.71493611,333.61236497)(165.71493611,333.6873649)(165.73493699,333.76737064)
\curveto(165.76493606,333.89736469)(165.79493603,334.01736457)(165.82493699,334.12737064)
\curveto(165.86493596,334.24736434)(165.90993592,334.36236422)(165.95993699,334.47237064)
\curveto(166.14993568,334.82236376)(166.38993544,335.09236349)(166.67993699,335.28237064)
\curveto(166.96993486,335.4823631)(167.3299345,335.64236294)(167.75993699,335.76237064)
\curveto(167.85993397,335.7823628)(167.95993387,335.79736279)(168.05993699,335.80737064)
\curveto(168.16993366,335.81736277)(168.27993355,335.83236275)(168.38993699,335.85237064)
\curveto(168.4299334,335.86236272)(168.49493333,335.86236272)(168.58493699,335.85237064)
\curveto(168.67493315,335.85236273)(168.7299331,335.86236272)(168.74993699,335.88237064)
\curveto(169.44993238,335.89236269)(170.05993177,335.81236277)(170.57993699,335.64237064)
\curveto(171.09993073,335.47236311)(171.46493036,335.14736344)(171.67493699,334.66737064)
\curveto(171.76493006,334.46736412)(171.81493001,334.23236435)(171.82493699,333.96237064)
\curveto(171.84492998,333.70236488)(171.85492997,333.42736516)(171.85493699,333.13737064)
\lineto(171.85493699,329.82237064)
\curveto(171.85492997,329.6823689)(171.85992997,329.54736904)(171.86993699,329.41737064)
\curveto(171.87992995,329.2873693)(171.90992992,329.1823694)(171.95993699,329.10237064)
\curveto(172.00992982,329.03236955)(172.07492975,328.9823696)(172.15493699,328.95237064)
\curveto(172.24492958,328.91236967)(172.3299295,328.8823697)(172.40993699,328.86237064)
\curveto(172.48992934,328.85236973)(172.54992928,328.80736978)(172.58993699,328.72737064)
\curveto(172.60992922,328.69736989)(172.61992921,328.66736992)(172.61993699,328.63737064)
\curveto(172.61992921,328.60736998)(172.6249292,328.56737002)(172.63493699,328.51737064)
\moveto(170.48993699,330.18237064)
\curveto(170.54993128,330.32236826)(170.57993125,330.4823681)(170.57993699,330.66237064)
\curveto(170.58993124,330.85236773)(170.59493123,331.04736754)(170.59493699,331.24737064)
\curveto(170.59493123,331.35736723)(170.58993124,331.45736713)(170.57993699,331.54737064)
\curveto(170.56993126,331.63736695)(170.5299313,331.70736688)(170.45993699,331.75737064)
\curveto(170.4299314,331.77736681)(170.35993147,331.7873668)(170.24993699,331.78737064)
\curveto(170.2299316,331.76736682)(170.19493163,331.75736683)(170.14493699,331.75737064)
\curveto(170.09493173,331.75736683)(170.04993178,331.74736684)(170.00993699,331.72737064)
\curveto(169.9299319,331.70736688)(169.83993199,331.6873669)(169.73993699,331.66737064)
\lineto(169.43993699,331.60737064)
\curveto(169.40993242,331.60736698)(169.37493245,331.60236698)(169.33493699,331.59237064)
\lineto(169.22993699,331.59237064)
\curveto(169.07993275,331.55236703)(168.91493291,331.52736706)(168.73493699,331.51737064)
\curveto(168.56493326,331.51736707)(168.40493342,331.49736709)(168.25493699,331.45737064)
\curveto(168.17493365,331.43736715)(168.09993373,331.41736717)(168.02993699,331.39737064)
\curveto(167.96993386,331.3873672)(167.89993393,331.37236721)(167.81993699,331.35237064)
\curveto(167.65993417,331.30236728)(167.50993432,331.23736735)(167.36993699,331.15737064)
\curveto(167.2299346,331.0873675)(167.10993472,330.99736759)(167.00993699,330.88737064)
\curveto(166.90993492,330.77736781)(166.83493499,330.64236794)(166.78493699,330.48237064)
\curveto(166.73493509,330.33236825)(166.71493511,330.14736844)(166.72493699,329.92737064)
\curveto(166.7249351,329.82736876)(166.73993509,329.73236885)(166.76993699,329.64237064)
\curveto(166.80993502,329.56236902)(166.85493497,329.4873691)(166.90493699,329.41737064)
\curveto(166.98493484,329.30736928)(167.08993474,329.21236937)(167.21993699,329.13237064)
\curveto(167.34993448,329.06236952)(167.48993434,329.00236958)(167.63993699,328.95237064)
\curveto(167.68993414,328.94236964)(167.73993409,328.93736965)(167.78993699,328.93737064)
\curveto(167.83993399,328.93736965)(167.88993394,328.93236965)(167.93993699,328.92237064)
\curveto(168.00993382,328.90236968)(168.09493373,328.8873697)(168.19493699,328.87737064)
\curveto(168.30493352,328.87736971)(168.39493343,328.8873697)(168.46493699,328.90737064)
\curveto(168.5249333,328.92736966)(168.58493324,328.93236965)(168.64493699,328.92237064)
\curveto(168.70493312,328.92236966)(168.76493306,328.93236965)(168.82493699,328.95237064)
\curveto(168.90493292,328.97236961)(168.97993285,328.9873696)(169.04993699,328.99737064)
\curveto(169.1299327,329.00736958)(169.20493262,329.02736956)(169.27493699,329.05737064)
\curveto(169.56493226,329.17736941)(169.80993202,329.32236926)(170.00993699,329.49237064)
\curveto(170.21993161,329.66236892)(170.37993145,329.89236869)(170.48993699,330.18237064)
}
}
{
\newrgbcolor{curcolor}{0 0 0}
\pscustom[linestyle=none,fillstyle=solid,fillcolor=curcolor]
{
\newpath
\moveto(176.94157762,335.86737064)
\curveto(177.68157283,335.87736271)(178.29657221,335.76736282)(178.78657762,335.53737064)
\curveto(179.28657122,335.31736327)(179.68157083,334.9823636)(179.97157762,334.53237064)
\curveto(180.10157041,334.33236425)(180.2115703,334.0873645)(180.30157762,333.79737064)
\curveto(180.32157019,333.74736484)(180.33657017,333.6823649)(180.34657762,333.60237064)
\curveto(180.35657015,333.52236506)(180.35157016,333.45236513)(180.33157762,333.39237064)
\curveto(180.30157021,333.34236524)(180.25157026,333.29736529)(180.18157762,333.25737064)
\curveto(180.15157036,333.23736535)(180.12157039,333.22736536)(180.09157762,333.22737064)
\curveto(180.06157045,333.23736535)(180.02657048,333.23736535)(179.98657762,333.22737064)
\curveto(179.94657056,333.21736537)(179.9065706,333.21236537)(179.86657762,333.21237064)
\curveto(179.82657068,333.22236536)(179.78657072,333.22736536)(179.74657762,333.22737064)
\lineto(179.43157762,333.22737064)
\curveto(179.33157118,333.23736535)(179.24657126,333.26736532)(179.17657762,333.31737064)
\curveto(179.09657141,333.37736521)(179.04157147,333.46236512)(179.01157762,333.57237064)
\curveto(178.98157153,333.6823649)(178.94157157,333.77736481)(178.89157762,333.85737064)
\curveto(178.74157177,334.11736447)(178.54657196,334.32236426)(178.30657762,334.47237064)
\curveto(178.22657228,334.52236406)(178.14157237,334.56236402)(178.05157762,334.59237064)
\curveto(177.96157255,334.63236395)(177.86657264,334.66736392)(177.76657762,334.69737064)
\curveto(177.62657288,334.73736385)(177.44157307,334.75736383)(177.21157762,334.75737064)
\curveto(176.98157353,334.76736382)(176.79157372,334.74736384)(176.64157762,334.69737064)
\curveto(176.57157394,334.67736391)(176.506574,334.66236392)(176.44657762,334.65237064)
\curveto(176.38657412,334.64236394)(176.32157419,334.62736396)(176.25157762,334.60737064)
\curveto(175.99157452,334.49736409)(175.76157475,334.34736424)(175.56157762,334.15737064)
\curveto(175.36157515,333.96736462)(175.2065753,333.74236484)(175.09657762,333.48237064)
\curveto(175.05657545,333.39236519)(175.02157549,333.29736529)(174.99157762,333.19737064)
\curveto(174.96157555,333.10736548)(174.93157558,333.00736558)(174.90157762,332.89737064)
\lineto(174.81157762,332.49237064)
\curveto(174.80157571,332.44236614)(174.79657571,332.3873662)(174.79657762,332.32737064)
\curveto(174.8065757,332.26736632)(174.80157571,332.21236637)(174.78157762,332.16237064)
\lineto(174.78157762,332.04237064)
\curveto(174.77157574,332.00236658)(174.76157575,331.93736665)(174.75157762,331.84737064)
\curveto(174.75157576,331.75736683)(174.76157575,331.69236689)(174.78157762,331.65237064)
\curveto(174.79157572,331.60236698)(174.79157572,331.55236703)(174.78157762,331.50237064)
\curveto(174.77157574,331.45236713)(174.77157574,331.40236718)(174.78157762,331.35237064)
\curveto(174.79157572,331.31236727)(174.79657571,331.24236734)(174.79657762,331.14237064)
\curveto(174.81657569,331.06236752)(174.83157568,330.97736761)(174.84157762,330.88737064)
\curveto(174.86157565,330.79736779)(174.88157563,330.71236787)(174.90157762,330.63237064)
\curveto(175.0115755,330.31236827)(175.13657537,330.03236855)(175.27657762,329.79237064)
\curveto(175.42657508,329.56236902)(175.63157488,329.36236922)(175.89157762,329.19237064)
\curveto(175.98157453,329.14236944)(176.07157444,329.09736949)(176.16157762,329.05737064)
\curveto(176.26157425,329.01736957)(176.36657414,328.97736961)(176.47657762,328.93737064)
\curveto(176.52657398,328.92736966)(176.56657394,328.92236966)(176.59657762,328.92237064)
\curveto(176.62657388,328.92236966)(176.66657384,328.91736967)(176.71657762,328.90737064)
\curveto(176.74657376,328.89736969)(176.79657371,328.89236969)(176.86657762,328.89237064)
\lineto(177.03157762,328.89237064)
\curveto(177.03157348,328.8823697)(177.05157346,328.87736971)(177.09157762,328.87737064)
\curveto(177.1115734,328.8873697)(177.13657337,328.8873697)(177.16657762,328.87737064)
\curveto(177.19657331,328.87736971)(177.22657328,328.8823697)(177.25657762,328.89237064)
\curveto(177.32657318,328.91236967)(177.39157312,328.91736967)(177.45157762,328.90737064)
\curveto(177.52157299,328.90736968)(177.59157292,328.91736967)(177.66157762,328.93737064)
\curveto(177.92157259,329.01736957)(178.14657236,329.11736947)(178.33657762,329.23737064)
\curveto(178.52657198,329.36736922)(178.68657182,329.53236905)(178.81657762,329.73237064)
\curveto(178.86657164,329.81236877)(178.9115716,329.89736869)(178.95157762,329.98737064)
\lineto(179.07157762,330.25737064)
\curveto(179.09157142,330.33736825)(179.1115714,330.41236817)(179.13157762,330.48237064)
\curveto(179.16157135,330.56236802)(179.2115713,330.62736796)(179.28157762,330.67737064)
\curveto(179.3115712,330.70736788)(179.37157114,330.72736786)(179.46157762,330.73737064)
\curveto(179.55157096,330.75736783)(179.64657086,330.76736782)(179.74657762,330.76737064)
\curveto(179.85657065,330.77736781)(179.95657055,330.77736781)(180.04657762,330.76737064)
\curveto(180.14657036,330.75736783)(180.21657029,330.73736785)(180.25657762,330.70737064)
\curveto(180.31657019,330.66736792)(180.35157016,330.60736798)(180.36157762,330.52737064)
\curveto(180.38157013,330.44736814)(180.38157013,330.36236822)(180.36157762,330.27237064)
\curveto(180.3115702,330.12236846)(180.26157025,329.97736861)(180.21157762,329.83737064)
\curveto(180.17157034,329.70736888)(180.11657039,329.57736901)(180.04657762,329.44737064)
\curveto(179.89657061,329.14736944)(179.7065708,328.8823697)(179.47657762,328.65237064)
\curveto(179.25657125,328.42237016)(178.98657152,328.23737035)(178.66657762,328.09737064)
\curveto(178.58657192,328.05737053)(178.50157201,328.02237056)(178.41157762,327.99237064)
\curveto(178.32157219,327.97237061)(178.22657228,327.94737064)(178.12657762,327.91737064)
\curveto(178.01657249,327.87737071)(177.9065726,327.85737073)(177.79657762,327.85737064)
\curveto(177.68657282,327.84737074)(177.57657293,327.83237075)(177.46657762,327.81237064)
\curveto(177.42657308,327.79237079)(177.38657312,327.7873708)(177.34657762,327.79737064)
\curveto(177.3065732,327.80737078)(177.26657324,327.80737078)(177.22657762,327.79737064)
\lineto(177.09157762,327.79737064)
\lineto(176.85157762,327.79737064)
\curveto(176.78157373,327.7873708)(176.71657379,327.79237079)(176.65657762,327.81237064)
\lineto(176.58157762,327.81237064)
\lineto(176.22157762,327.85737064)
\curveto(176.09157442,327.89737069)(175.96657454,327.93237065)(175.84657762,327.96237064)
\curveto(175.72657478,327.99237059)(175.6115749,328.03237055)(175.50157762,328.08237064)
\curveto(175.14157537,328.24237034)(174.84157567,328.43237015)(174.60157762,328.65237064)
\curveto(174.37157614,328.87236971)(174.15657635,329.14236944)(173.95657762,329.46237064)
\curveto(173.9065766,329.54236904)(173.86157665,329.63236895)(173.82157762,329.73237064)
\lineto(173.70157762,330.03237064)
\curveto(173.65157686,330.14236844)(173.61657689,330.25736833)(173.59657762,330.37737064)
\curveto(173.57657693,330.49736809)(173.55157696,330.61736797)(173.52157762,330.73737064)
\curveto(173.511577,330.77736781)(173.506577,330.81736777)(173.50657762,330.85737064)
\curveto(173.506577,330.89736769)(173.50157701,330.93736765)(173.49157762,330.97737064)
\curveto(173.47157704,331.03736755)(173.46157705,331.10236748)(173.46157762,331.17237064)
\curveto(173.47157704,331.24236734)(173.46657704,331.30736728)(173.44657762,331.36737064)
\lineto(173.44657762,331.51737064)
\curveto(173.43657707,331.56736702)(173.43157708,331.63736695)(173.43157762,331.72737064)
\curveto(173.43157708,331.81736677)(173.43657707,331.8873667)(173.44657762,331.93737064)
\curveto(173.45657705,331.9873666)(173.45657705,332.03236655)(173.44657762,332.07237064)
\curveto(173.44657706,332.11236647)(173.45157706,332.15236643)(173.46157762,332.19237064)
\curveto(173.48157703,332.26236632)(173.48657702,332.33236625)(173.47657762,332.40237064)
\curveto(173.47657703,332.47236611)(173.48657702,332.53736605)(173.50657762,332.59737064)
\curveto(173.54657696,332.76736582)(173.58157693,332.93736565)(173.61157762,333.10737064)
\curveto(173.64157687,333.27736531)(173.68657682,333.43736515)(173.74657762,333.58737064)
\curveto(173.95657655,334.10736448)(174.2115763,334.52736406)(174.51157762,334.84737064)
\curveto(174.8115757,335.16736342)(175.22157529,335.43236315)(175.74157762,335.64237064)
\curveto(175.85157466,335.69236289)(175.97157454,335.72736286)(176.10157762,335.74737064)
\curveto(176.23157428,335.76736282)(176.36657414,335.79236279)(176.50657762,335.82237064)
\curveto(176.57657393,335.83236275)(176.64657386,335.83736275)(176.71657762,335.83737064)
\curveto(176.78657372,335.84736274)(176.86157365,335.85736273)(176.94157762,335.86737064)
}
}
{
\newrgbcolor{curcolor}{0 0 0}
\pscustom[linestyle=none,fillstyle=solid,fillcolor=curcolor]
{
\newpath
\moveto(182.14821824,337.18737064)
\curveto(182.06821712,337.24736134)(182.02321717,337.35236123)(182.01321824,337.50237064)
\lineto(182.01321824,337.96737064)
\lineto(182.01321824,338.22237064)
\curveto(182.01321718,338.31236027)(182.02821716,338.3873602)(182.05821824,338.44737064)
\curveto(182.09821709,338.52736006)(182.17821701,338.58736)(182.29821824,338.62737064)
\curveto(182.31821687,338.63735995)(182.33821685,338.63735995)(182.35821824,338.62737064)
\curveto(182.3882168,338.62735996)(182.41321678,338.63235995)(182.43321824,338.64237064)
\curveto(182.60321659,338.64235994)(182.76321643,338.63735995)(182.91321824,338.62737064)
\curveto(183.06321613,338.61735997)(183.16321603,338.55736003)(183.21321824,338.44737064)
\curveto(183.24321595,338.3873602)(183.25821593,338.31236027)(183.25821824,338.22237064)
\lineto(183.25821824,337.96737064)
\curveto(183.25821593,337.7873608)(183.25321594,337.61736097)(183.24321824,337.45737064)
\curveto(183.24321595,337.29736129)(183.17821601,337.19236139)(183.04821824,337.14237064)
\curveto(182.99821619,337.12236146)(182.94321625,337.11236147)(182.88321824,337.11237064)
\lineto(182.71821824,337.11237064)
\lineto(182.40321824,337.11237064)
\curveto(182.30321689,337.11236147)(182.21821697,337.13736145)(182.14821824,337.18737064)
\moveto(183.25821824,328.68237064)
\lineto(183.25821824,328.36737064)
\curveto(183.26821592,328.26737032)(183.24821594,328.1873704)(183.19821824,328.12737064)
\curveto(183.16821602,328.06737052)(183.12321607,328.02737056)(183.06321824,328.00737064)
\curveto(183.00321619,327.99737059)(182.93321626,327.9823706)(182.85321824,327.96237064)
\lineto(182.62821824,327.96237064)
\curveto(182.49821669,327.96237062)(182.38321681,327.96737062)(182.28321824,327.97737064)
\curveto(182.193217,327.99737059)(182.12321707,328.04737054)(182.07321824,328.12737064)
\curveto(182.03321716,328.1873704)(182.01321718,328.26237032)(182.01321824,328.35237064)
\lineto(182.01321824,328.63737064)
\lineto(182.01321824,334.98237064)
\lineto(182.01321824,335.29737064)
\curveto(182.01321718,335.40736318)(182.03821715,335.49236309)(182.08821824,335.55237064)
\curveto(182.11821707,335.60236298)(182.15821703,335.63236295)(182.20821824,335.64237064)
\curveto(182.25821693,335.65236293)(182.31321688,335.66736292)(182.37321824,335.68737064)
\curveto(182.3932168,335.6873629)(182.41321678,335.6823629)(182.43321824,335.67237064)
\curveto(182.46321673,335.67236291)(182.4882167,335.67736291)(182.50821824,335.68737064)
\curveto(182.63821655,335.6873629)(182.76821642,335.6823629)(182.89821824,335.67237064)
\curveto(183.03821615,335.67236291)(183.13321606,335.63236295)(183.18321824,335.55237064)
\curveto(183.23321596,335.49236309)(183.25821593,335.41236317)(183.25821824,335.31237064)
\lineto(183.25821824,335.02737064)
\lineto(183.25821824,328.68237064)
}
}
{
\newrgbcolor{curcolor}{0 0 0}
\pscustom[linestyle=none,fillstyle=solid,fillcolor=curcolor]
{
\newpath
\moveto(192.32806199,332.16237064)
\curveto(192.34805393,332.10236648)(192.35805392,332.00736658)(192.35806199,331.87737064)
\curveto(192.35805392,331.75736683)(192.35305393,331.67236691)(192.34306199,331.62237064)
\lineto(192.34306199,331.47237064)
\curveto(192.33305395,331.39236719)(192.32305396,331.31736727)(192.31306199,331.24737064)
\curveto(192.31305397,331.1873674)(192.30805397,331.11736747)(192.29806199,331.03737064)
\curveto(192.278054,330.97736761)(192.26305402,330.91736767)(192.25306199,330.85737064)
\curveto(192.25305403,330.79736779)(192.24305404,330.73736785)(192.22306199,330.67737064)
\curveto(192.1830541,330.54736804)(192.14805413,330.41736817)(192.11806199,330.28737064)
\curveto(192.08805419,330.15736843)(192.04805423,330.03736855)(191.99806199,329.92737064)
\curveto(191.78805449,329.44736914)(191.50805477,329.04236954)(191.15806199,328.71237064)
\curveto(190.80805547,328.39237019)(190.3780559,328.14737044)(189.86806199,327.97737064)
\curveto(189.75805652,327.93737065)(189.63805664,327.90737068)(189.50806199,327.88737064)
\curveto(189.38805689,327.86737072)(189.26305702,327.84737074)(189.13306199,327.82737064)
\curveto(189.07305721,327.81737077)(189.00805727,327.81237077)(188.93806199,327.81237064)
\curveto(188.8780574,327.80237078)(188.81805746,327.79737079)(188.75806199,327.79737064)
\curveto(188.71805756,327.7873708)(188.65805762,327.7823708)(188.57806199,327.78237064)
\curveto(188.50805777,327.7823708)(188.45805782,327.7873708)(188.42806199,327.79737064)
\curveto(188.38805789,327.80737078)(188.34805793,327.81237077)(188.30806199,327.81237064)
\curveto(188.26805801,327.80237078)(188.23305805,327.80237078)(188.20306199,327.81237064)
\lineto(188.11306199,327.81237064)
\lineto(187.75306199,327.85737064)
\curveto(187.61305867,327.89737069)(187.4780588,327.93737065)(187.34806199,327.97737064)
\curveto(187.21805906,328.01737057)(187.09305919,328.06237052)(186.97306199,328.11237064)
\curveto(186.52305976,328.31237027)(186.15306013,328.57237001)(185.86306199,328.89237064)
\curveto(185.57306071,329.21236937)(185.33306095,329.60236898)(185.14306199,330.06237064)
\curveto(185.09306119,330.16236842)(185.05306123,330.26236832)(185.02306199,330.36237064)
\curveto(185.00306128,330.46236812)(184.9830613,330.56736802)(184.96306199,330.67737064)
\curveto(184.94306134,330.71736787)(184.93306135,330.74736784)(184.93306199,330.76737064)
\curveto(184.94306134,330.79736779)(184.94306134,330.83236775)(184.93306199,330.87237064)
\curveto(184.91306137,330.95236763)(184.89806138,331.03236755)(184.88806199,331.11237064)
\curveto(184.88806139,331.20236738)(184.8780614,331.2873673)(184.85806199,331.36737064)
\lineto(184.85806199,331.48737064)
\curveto(184.85806142,331.52736706)(184.85306143,331.57236701)(184.84306199,331.62237064)
\curveto(184.83306145,331.67236691)(184.82806145,331.75736683)(184.82806199,331.87737064)
\curveto(184.82806145,332.00736658)(184.83806144,332.10236648)(184.85806199,332.16237064)
\curveto(184.8780614,332.23236635)(184.8830614,332.30236628)(184.87306199,332.37237064)
\curveto(184.86306142,332.44236614)(184.86806141,332.51236607)(184.88806199,332.58237064)
\curveto(184.89806138,332.63236595)(184.90306138,332.67236591)(184.90306199,332.70237064)
\curveto(184.91306137,332.74236584)(184.92306136,332.7873658)(184.93306199,332.83737064)
\curveto(184.96306132,332.95736563)(184.98806129,333.07736551)(185.00806199,333.19737064)
\curveto(185.03806124,333.31736527)(185.0780612,333.43236515)(185.12806199,333.54237064)
\curveto(185.278061,333.91236467)(185.45806082,334.24236434)(185.66806199,334.53237064)
\curveto(185.88806039,334.83236375)(186.15306013,335.0823635)(186.46306199,335.28237064)
\curveto(186.5830597,335.36236322)(186.70805957,335.42736316)(186.83806199,335.47737064)
\curveto(186.96805931,335.53736305)(187.10305918,335.59736299)(187.24306199,335.65737064)
\curveto(187.36305892,335.70736288)(187.49305879,335.73736285)(187.63306199,335.74737064)
\curveto(187.77305851,335.76736282)(187.91305837,335.79736279)(188.05306199,335.83737064)
\lineto(188.24806199,335.83737064)
\curveto(188.31805796,335.84736274)(188.3830579,335.85736273)(188.44306199,335.86737064)
\curveto(189.33305695,335.87736271)(190.07305621,335.69236289)(190.66306199,335.31237064)
\curveto(191.25305503,334.93236365)(191.6780546,334.43736415)(191.93806199,333.82737064)
\curveto(191.98805429,333.72736486)(192.02805425,333.62736496)(192.05806199,333.52737064)
\curveto(192.08805419,333.42736516)(192.12305416,333.32236526)(192.16306199,333.21237064)
\curveto(192.19305409,333.10236548)(192.21805406,332.9823656)(192.23806199,332.85237064)
\curveto(192.25805402,332.73236585)(192.283054,332.60736598)(192.31306199,332.47737064)
\curveto(192.32305396,332.42736616)(192.32305396,332.37236621)(192.31306199,332.31237064)
\curveto(192.31305397,332.26236632)(192.31805396,332.21236637)(192.32806199,332.16237064)
\moveto(190.99306199,331.30737064)
\curveto(191.01305527,331.37736721)(191.01805526,331.45736713)(191.00806199,331.54737064)
\lineto(191.00806199,331.80237064)
\curveto(191.00805527,332.19236639)(190.97305531,332.52236606)(190.90306199,332.79237064)
\curveto(190.87305541,332.87236571)(190.84805543,332.95236563)(190.82806199,333.03237064)
\curveto(190.80805547,333.11236547)(190.7830555,333.1873654)(190.75306199,333.25737064)
\curveto(190.47305581,333.90736468)(190.02805625,334.35736423)(189.41806199,334.60737064)
\curveto(189.34805693,334.63736395)(189.27305701,334.65736393)(189.19306199,334.66737064)
\lineto(188.95306199,334.72737064)
\curveto(188.87305741,334.74736384)(188.78805749,334.75736383)(188.69806199,334.75737064)
\lineto(188.42806199,334.75737064)
\lineto(188.15806199,334.71237064)
\curveto(188.05805822,334.69236389)(187.96305832,334.66736392)(187.87306199,334.63737064)
\curveto(187.79305849,334.61736397)(187.71305857,334.587364)(187.63306199,334.54737064)
\curveto(187.56305872,334.52736406)(187.49805878,334.49736409)(187.43806199,334.45737064)
\curveto(187.3780589,334.41736417)(187.32305896,334.37736421)(187.27306199,334.33737064)
\curveto(187.03305925,334.16736442)(186.83805944,333.96236462)(186.68806199,333.72237064)
\curveto(186.53805974,333.4823651)(186.40805987,333.20236538)(186.29806199,332.88237064)
\curveto(186.26806001,332.7823658)(186.24806003,332.67736591)(186.23806199,332.56737064)
\curveto(186.22806005,332.46736612)(186.21306007,332.36236622)(186.19306199,332.25237064)
\curveto(186.1830601,332.21236637)(186.1780601,332.14736644)(186.17806199,332.05737064)
\curveto(186.16806011,332.02736656)(186.16306012,331.99236659)(186.16306199,331.95237064)
\curveto(186.17306011,331.91236667)(186.1780601,331.86736672)(186.17806199,331.81737064)
\lineto(186.17806199,331.51737064)
\curveto(186.1780601,331.41736717)(186.18806009,331.32736726)(186.20806199,331.24737064)
\lineto(186.23806199,331.06737064)
\curveto(186.25806002,330.96736762)(186.27306001,330.86736772)(186.28306199,330.76737064)
\curveto(186.30305998,330.67736791)(186.33305995,330.59236799)(186.37306199,330.51237064)
\curveto(186.47305981,330.27236831)(186.58805969,330.04736854)(186.71806199,329.83737064)
\curveto(186.85805942,329.62736896)(187.02805925,329.45236913)(187.22806199,329.31237064)
\curveto(187.278059,329.2823693)(187.32305896,329.25736933)(187.36306199,329.23737064)
\curveto(187.40305888,329.21736937)(187.44805883,329.19236939)(187.49806199,329.16237064)
\curveto(187.5780587,329.11236947)(187.66305862,329.06736952)(187.75306199,329.02737064)
\curveto(187.85305843,328.99736959)(187.95805832,328.96736962)(188.06806199,328.93737064)
\curveto(188.11805816,328.91736967)(188.16305812,328.90736968)(188.20306199,328.90737064)
\curveto(188.25305803,328.91736967)(188.30305798,328.91736967)(188.35306199,328.90737064)
\curveto(188.3830579,328.89736969)(188.44305784,328.8873697)(188.53306199,328.87737064)
\curveto(188.63305765,328.86736972)(188.70805757,328.87236971)(188.75806199,328.89237064)
\curveto(188.79805748,328.90236968)(188.83805744,328.90236968)(188.87806199,328.89237064)
\curveto(188.91805736,328.89236969)(188.95805732,328.90236968)(188.99806199,328.92237064)
\curveto(189.0780572,328.94236964)(189.15805712,328.95736963)(189.23806199,328.96737064)
\curveto(189.31805696,328.9873696)(189.39305689,329.01236957)(189.46306199,329.04237064)
\curveto(189.80305648,329.1823694)(190.0780562,329.37736921)(190.28806199,329.62737064)
\curveto(190.49805578,329.87736871)(190.67305561,330.17236841)(190.81306199,330.51237064)
\curveto(190.86305542,330.63236795)(190.89305539,330.75736783)(190.90306199,330.88737064)
\curveto(190.92305536,331.02736756)(190.95305533,331.16736742)(190.99306199,331.30737064)
}
}
{
\newrgbcolor{curcolor}{0 0 0}
\pscustom[linestyle=none,fillstyle=solid,fillcolor=curcolor]
{
\newpath
\moveto(196.24634324,335.86737064)
\curveto(196.96633918,335.87736271)(197.57133857,335.79236279)(198.06134324,335.61237064)
\curveto(198.55133759,335.44236314)(198.93133721,335.13736345)(199.20134324,334.69737064)
\curveto(199.27133687,334.587364)(199.32633682,334.47236411)(199.36634324,334.35237064)
\curveto(199.40633674,334.24236434)(199.4463367,334.11736447)(199.48634324,333.97737064)
\curveto(199.50633664,333.90736468)(199.51133663,333.83236475)(199.50134324,333.75237064)
\curveto(199.49133665,333.6823649)(199.47633667,333.62736496)(199.45634324,333.58737064)
\curveto(199.43633671,333.56736502)(199.41133673,333.54736504)(199.38134324,333.52737064)
\curveto(199.35133679,333.51736507)(199.32633682,333.50236508)(199.30634324,333.48237064)
\curveto(199.25633689,333.46236512)(199.20633694,333.45736513)(199.15634324,333.46737064)
\curveto(199.10633704,333.47736511)(199.05633709,333.47736511)(199.00634324,333.46737064)
\curveto(198.92633722,333.44736514)(198.82133732,333.44236514)(198.69134324,333.45237064)
\curveto(198.56133758,333.47236511)(198.47133767,333.49736509)(198.42134324,333.52737064)
\curveto(198.3413378,333.57736501)(198.28633786,333.64236494)(198.25634324,333.72237064)
\curveto(198.23633791,333.81236477)(198.20133794,333.89736469)(198.15134324,333.97737064)
\curveto(198.06133808,334.13736445)(197.93633821,334.2823643)(197.77634324,334.41237064)
\curveto(197.66633848,334.49236409)(197.5463386,334.55236403)(197.41634324,334.59237064)
\curveto(197.28633886,334.63236395)(197.146339,334.67236391)(196.99634324,334.71237064)
\curveto(196.9463392,334.73236385)(196.89633925,334.73736385)(196.84634324,334.72737064)
\curveto(196.79633935,334.72736386)(196.7463394,334.73236385)(196.69634324,334.74237064)
\curveto(196.63633951,334.76236382)(196.56133958,334.77236381)(196.47134324,334.77237064)
\curveto(196.38133976,334.77236381)(196.30633984,334.76236382)(196.24634324,334.74237064)
\lineto(196.15634324,334.74237064)
\lineto(196.00634324,334.71237064)
\curveto(195.95634019,334.71236387)(195.90634024,334.70736388)(195.85634324,334.69737064)
\curveto(195.59634055,334.63736395)(195.38134076,334.55236403)(195.21134324,334.44237064)
\curveto(195.0413411,334.33236425)(194.92634122,334.14736444)(194.86634324,333.88737064)
\curveto(194.8463413,333.81736477)(194.8413413,333.74736484)(194.85134324,333.67737064)
\curveto(194.87134127,333.60736498)(194.89134125,333.54736504)(194.91134324,333.49737064)
\curveto(194.97134117,333.34736524)(195.0413411,333.23736535)(195.12134324,333.16737064)
\curveto(195.21134093,333.10736548)(195.32134082,333.03736555)(195.45134324,332.95737064)
\curveto(195.61134053,332.85736573)(195.79134035,332.7823658)(195.99134324,332.73237064)
\curveto(196.19133995,332.69236589)(196.39133975,332.64236594)(196.59134324,332.58237064)
\curveto(196.72133942,332.54236604)(196.85133929,332.51236607)(196.98134324,332.49237064)
\curveto(197.11133903,332.47236611)(197.2413389,332.44236614)(197.37134324,332.40237064)
\curveto(197.58133856,332.34236624)(197.78633836,332.2823663)(197.98634324,332.22237064)
\curveto(198.18633796,332.17236641)(198.38633776,332.10736648)(198.58634324,332.02737064)
\lineto(198.73634324,331.96737064)
\curveto(198.78633736,331.94736664)(198.83633731,331.92236666)(198.88634324,331.89237064)
\curveto(199.08633706,331.77236681)(199.26133688,331.63736695)(199.41134324,331.48737064)
\curveto(199.56133658,331.33736725)(199.68633646,331.14736744)(199.78634324,330.91737064)
\curveto(199.80633634,330.84736774)(199.82633632,330.75236783)(199.84634324,330.63237064)
\curveto(199.86633628,330.56236802)(199.87633627,330.4873681)(199.87634324,330.40737064)
\curveto(199.88633626,330.33736825)(199.89133625,330.25736833)(199.89134324,330.16737064)
\lineto(199.89134324,330.01737064)
\curveto(199.87133627,329.94736864)(199.86133628,329.87736871)(199.86134324,329.80737064)
\curveto(199.86133628,329.73736885)(199.85133629,329.66736892)(199.83134324,329.59737064)
\curveto(199.80133634,329.4873691)(199.76633638,329.3823692)(199.72634324,329.28237064)
\curveto(199.68633646,329.1823694)(199.6413365,329.09236949)(199.59134324,329.01237064)
\curveto(199.43133671,328.75236983)(199.22633692,328.54237004)(198.97634324,328.38237064)
\curveto(198.72633742,328.23237035)(198.4463377,328.10237048)(198.13634324,327.99237064)
\curveto(198.0463381,327.96237062)(197.95133819,327.94237064)(197.85134324,327.93237064)
\curveto(197.76133838,327.91237067)(197.67133847,327.8873707)(197.58134324,327.85737064)
\curveto(197.48133866,327.83737075)(197.38133876,327.82737076)(197.28134324,327.82737064)
\curveto(197.18133896,327.82737076)(197.08133906,327.81737077)(196.98134324,327.79737064)
\lineto(196.83134324,327.79737064)
\curveto(196.78133936,327.7873708)(196.71133943,327.7823708)(196.62134324,327.78237064)
\curveto(196.53133961,327.7823708)(196.46133968,327.7873708)(196.41134324,327.79737064)
\lineto(196.24634324,327.79737064)
\curveto(196.18633996,327.81737077)(196.12134002,327.82737076)(196.05134324,327.82737064)
\curveto(195.98134016,327.81737077)(195.92134022,327.82237076)(195.87134324,327.84237064)
\curveto(195.82134032,327.85237073)(195.75634039,327.85737073)(195.67634324,327.85737064)
\lineto(195.43634324,327.91737064)
\curveto(195.36634078,327.92737066)(195.29134085,327.94737064)(195.21134324,327.97737064)
\curveto(194.90134124,328.07737051)(194.63134151,328.20237038)(194.40134324,328.35237064)
\curveto(194.17134197,328.50237008)(193.97134217,328.69736989)(193.80134324,328.93737064)
\curveto(193.71134243,329.06736952)(193.63634251,329.20236938)(193.57634324,329.34237064)
\curveto(193.51634263,329.4823691)(193.46134268,329.63736895)(193.41134324,329.80737064)
\curveto(193.39134275,329.86736872)(193.38134276,329.93736865)(193.38134324,330.01737064)
\curveto(193.39134275,330.10736848)(193.40634274,330.17736841)(193.42634324,330.22737064)
\curveto(193.45634269,330.26736832)(193.50634264,330.30736828)(193.57634324,330.34737064)
\curveto(193.62634252,330.36736822)(193.69634245,330.37736821)(193.78634324,330.37737064)
\curveto(193.87634227,330.3873682)(193.96634218,330.3873682)(194.05634324,330.37737064)
\curveto(194.146342,330.36736822)(194.23134191,330.35236823)(194.31134324,330.33237064)
\curveto(194.40134174,330.32236826)(194.46134168,330.30736828)(194.49134324,330.28737064)
\curveto(194.56134158,330.23736835)(194.60634154,330.16236842)(194.62634324,330.06237064)
\curveto(194.65634149,329.97236861)(194.69134145,329.8873687)(194.73134324,329.80737064)
\curveto(194.83134131,329.587369)(194.96634118,329.41736917)(195.13634324,329.29737064)
\curveto(195.25634089,329.20736938)(195.39134075,329.13736945)(195.54134324,329.08737064)
\curveto(195.69134045,329.03736955)(195.85134029,328.9873696)(196.02134324,328.93737064)
\lineto(196.33634324,328.89237064)
\lineto(196.42634324,328.89237064)
\curveto(196.49633965,328.87236971)(196.58633956,328.86236972)(196.69634324,328.86237064)
\curveto(196.81633933,328.86236972)(196.91633923,328.87236971)(196.99634324,328.89237064)
\curveto(197.06633908,328.89236969)(197.12133902,328.89736969)(197.16134324,328.90737064)
\curveto(197.22133892,328.91736967)(197.28133886,328.92236966)(197.34134324,328.92237064)
\curveto(197.40133874,328.93236965)(197.45633869,328.94236964)(197.50634324,328.95237064)
\curveto(197.79633835,329.03236955)(198.02633812,329.13736945)(198.19634324,329.26737064)
\curveto(198.36633778,329.39736919)(198.48633766,329.61736897)(198.55634324,329.92737064)
\curveto(198.57633757,329.97736861)(198.58133756,330.03236855)(198.57134324,330.09237064)
\curveto(198.56133758,330.15236843)(198.55133759,330.19736839)(198.54134324,330.22737064)
\curveto(198.49133765,330.41736817)(198.42133772,330.55736803)(198.33134324,330.64737064)
\curveto(198.2413379,330.74736784)(198.12633802,330.83736775)(197.98634324,330.91737064)
\curveto(197.89633825,330.97736761)(197.79633835,331.02736756)(197.68634324,331.06737064)
\lineto(197.35634324,331.18737064)
\curveto(197.32633882,331.19736739)(197.29633885,331.20236738)(197.26634324,331.20237064)
\curveto(197.2463389,331.20236738)(197.22133892,331.21236737)(197.19134324,331.23237064)
\curveto(196.85133929,331.34236724)(196.49633965,331.42236716)(196.12634324,331.47237064)
\curveto(195.76634038,331.53236705)(195.42634072,331.62736696)(195.10634324,331.75737064)
\curveto(195.00634114,331.79736679)(194.91134123,331.83236675)(194.82134324,331.86237064)
\curveto(194.73134141,331.89236669)(194.6463415,331.93236665)(194.56634324,331.98237064)
\curveto(194.37634177,332.09236649)(194.20134194,332.21736637)(194.04134324,332.35737064)
\curveto(193.88134226,332.49736609)(193.75634239,332.67236591)(193.66634324,332.88237064)
\curveto(193.63634251,332.95236563)(193.61134253,333.02236556)(193.59134324,333.09237064)
\curveto(193.58134256,333.16236542)(193.56634258,333.23736535)(193.54634324,333.31737064)
\curveto(193.51634263,333.43736515)(193.50634264,333.57236501)(193.51634324,333.72237064)
\curveto(193.52634262,333.8823647)(193.5413426,334.01736457)(193.56134324,334.12737064)
\curveto(193.58134256,334.17736441)(193.59134255,334.21736437)(193.59134324,334.24737064)
\curveto(193.60134254,334.2873643)(193.61634253,334.32736426)(193.63634324,334.36737064)
\curveto(193.72634242,334.59736399)(193.8463423,334.79736379)(193.99634324,334.96737064)
\curveto(194.15634199,335.13736345)(194.33634181,335.2873633)(194.53634324,335.41737064)
\curveto(194.68634146,335.50736308)(194.85134129,335.57736301)(195.03134324,335.62737064)
\curveto(195.21134093,335.6873629)(195.40134074,335.74236284)(195.60134324,335.79237064)
\curveto(195.67134047,335.80236278)(195.73634041,335.81236277)(195.79634324,335.82237064)
\curveto(195.86634028,335.83236275)(195.9413402,335.84236274)(196.02134324,335.85237064)
\curveto(196.05134009,335.86236272)(196.09134005,335.86236272)(196.14134324,335.85237064)
\curveto(196.19133995,335.84236274)(196.22633992,335.84736274)(196.24634324,335.86737064)
}
}
{
\newrgbcolor{curcolor}{0.50196081 0.50196081 0.50196081}
\pscustom[linestyle=none,fillstyle=solid,fillcolor=curcolor]
{
\newpath
\moveto(120.84556474,341.17243777)
\lineto(135.84556474,341.17243777)
\lineto(135.84556474,326.17243777)
\lineto(120.84556474,326.17243777)
\closepath
}
}
{
\newrgbcolor{curcolor}{0 0 0}
\pscustom[linestyle=none,fillstyle=solid,fillcolor=curcolor]
{
\newpath
\moveto(263.84668701,31.67142873)
\lineto(263.84668701,32.58642873)
\curveto(263.84669771,32.68642608)(263.84669771,32.78142599)(263.84668701,32.87142873)
\curveto(263.84669771,32.96142581)(263.86669769,33.03642573)(263.90668701,33.09642873)
\curveto(263.96669759,33.18642558)(264.04669751,33.24642552)(264.14668701,33.27642873)
\curveto(264.24669731,33.31642545)(264.3516972,33.36142541)(264.46168701,33.41142873)
\curveto(264.6516969,33.49142528)(264.84169671,33.56142521)(265.03168701,33.62142873)
\curveto(265.22169633,33.69142508)(265.41169614,33.766425)(265.60168701,33.84642873)
\curveto(265.78169577,33.91642485)(265.96669559,33.98142479)(266.15668701,34.04142873)
\curveto(266.33669522,34.10142467)(266.51669504,34.1714246)(266.69668701,34.25142873)
\curveto(266.83669472,34.31142446)(266.98169457,34.3664244)(267.13168701,34.41642873)
\curveto(267.28169427,34.4664243)(267.42669413,34.52142425)(267.56668701,34.58142873)
\curveto(268.01669354,34.76142401)(268.47169308,34.93142384)(268.93168701,35.09142873)
\curveto(269.38169217,35.25142352)(269.83169172,35.42142335)(270.28168701,35.60142873)
\curveto(270.33169122,35.62142315)(270.38169117,35.63642313)(270.43168701,35.64642873)
\lineto(270.58168701,35.70642873)
\curveto(270.80169075,35.79642297)(271.02669053,35.88142289)(271.25668701,35.96142873)
\curveto(271.47669008,36.04142273)(271.69668986,36.12642264)(271.91668701,36.21642873)
\curveto(272.00668955,36.25642251)(272.11668944,36.29642247)(272.24668701,36.33642873)
\curveto(272.36668919,36.37642239)(272.43668912,36.44142233)(272.45668701,36.53142873)
\curveto(272.46668909,36.5714222)(272.46668909,36.60142217)(272.45668701,36.62142873)
\lineto(272.39668701,36.68142873)
\curveto(272.34668921,36.73142204)(272.29168926,36.766422)(272.23168701,36.78642873)
\curveto(272.17168938,36.81642195)(272.10668945,36.84642192)(272.03668701,36.87642873)
\lineto(271.40668701,37.11642873)
\curveto(271.18669037,37.19642157)(270.97169058,37.27642149)(270.76168701,37.35642873)
\lineto(270.61168701,37.41642873)
\lineto(270.43168701,37.47642873)
\curveto(270.24169131,37.55642121)(270.0516915,37.62642114)(269.86168701,37.68642873)
\curveto(269.66169189,37.75642101)(269.46169209,37.83142094)(269.26168701,37.91142873)
\curveto(268.68169287,38.15142062)(268.09669346,38.3714204)(267.50668701,38.57142873)
\curveto(266.91669464,38.78141999)(266.33169522,39.00641976)(265.75168701,39.24642873)
\curveto(265.551696,39.32641944)(265.34669621,39.40141937)(265.13668701,39.47142873)
\curveto(264.92669663,39.55141922)(264.72169683,39.63141914)(264.52168701,39.71142873)
\curveto(264.44169711,39.75141902)(264.34169721,39.78641898)(264.22168701,39.81642873)
\curveto(264.10169745,39.85641891)(264.01669754,39.91141886)(263.96668701,39.98142873)
\curveto(263.92669763,40.04141873)(263.89669766,40.11641865)(263.87668701,40.20642873)
\curveto(263.8566977,40.30641846)(263.84669771,40.41641835)(263.84668701,40.53642873)
\curveto(263.83669772,40.65641811)(263.83669772,40.77641799)(263.84668701,40.89642873)
\curveto(263.84669771,41.01641775)(263.84669771,41.12641764)(263.84668701,41.22642873)
\curveto(263.84669771,41.31641745)(263.84669771,41.40641736)(263.84668701,41.49642873)
\curveto(263.84669771,41.59641717)(263.86669769,41.6714171)(263.90668701,41.72142873)
\curveto(263.9566976,41.81141696)(264.04669751,41.86141691)(264.17668701,41.87142873)
\curveto(264.30669725,41.88141689)(264.44669711,41.88641688)(264.59668701,41.88642873)
\lineto(266.24668701,41.88642873)
\lineto(272.51668701,41.88642873)
\lineto(273.77668701,41.88642873)
\curveto(273.88668767,41.88641688)(273.99668756,41.88641688)(274.10668701,41.88642873)
\curveto(274.21668734,41.89641687)(274.30168725,41.87641689)(274.36168701,41.82642873)
\curveto(274.42168713,41.79641697)(274.46168709,41.75141702)(274.48168701,41.69142873)
\curveto(274.49168706,41.63141714)(274.50668705,41.56141721)(274.52668701,41.48142873)
\lineto(274.52668701,41.24142873)
\lineto(274.52668701,40.88142873)
\curveto(274.51668704,40.771418)(274.47168708,40.69141808)(274.39168701,40.64142873)
\curveto(274.36168719,40.62141815)(274.33168722,40.60641816)(274.30168701,40.59642873)
\curveto(274.26168729,40.59641817)(274.21668734,40.58641818)(274.16668701,40.56642873)
\lineto(274.00168701,40.56642873)
\curveto(273.94168761,40.55641821)(273.87168768,40.55141822)(273.79168701,40.55142873)
\curveto(273.71168784,40.56141821)(273.63668792,40.5664182)(273.56668701,40.56642873)
\lineto(272.72668701,40.56642873)
\lineto(268.30168701,40.56642873)
\curveto(268.0516935,40.5664182)(267.80169375,40.5664182)(267.55168701,40.56642873)
\curveto(267.29169426,40.5664182)(267.04169451,40.56141821)(266.80168701,40.55142873)
\curveto(266.70169485,40.55141822)(266.59169496,40.54641822)(266.47168701,40.53642873)
\curveto(266.3516952,40.52641824)(266.29169526,40.4714183)(266.29168701,40.37142873)
\lineto(266.30668701,40.37142873)
\curveto(266.32669523,40.30141847)(266.39169516,40.24141853)(266.50168701,40.19142873)
\curveto(266.61169494,40.15141862)(266.70669485,40.11641865)(266.78668701,40.08642873)
\curveto(266.9566946,40.01641875)(267.13169442,39.95141882)(267.31168701,39.89142873)
\curveto(267.48169407,39.83141894)(267.6516939,39.76141901)(267.82168701,39.68142873)
\curveto(267.87169368,39.66141911)(267.91669364,39.64641912)(267.95668701,39.63642873)
\curveto(267.99669356,39.62641914)(268.04169351,39.61141916)(268.09168701,39.59142873)
\curveto(268.27169328,39.51141926)(268.4566931,39.44141933)(268.64668701,39.38142873)
\curveto(268.82669273,39.33141944)(269.00669255,39.2664195)(269.18668701,39.18642873)
\curveto(269.33669222,39.11641965)(269.49169206,39.05641971)(269.65168701,39.00642873)
\curveto(269.80169175,38.95641981)(269.9516916,38.90141987)(270.10168701,38.84142873)
\curveto(270.57169098,38.64142013)(271.04669051,38.46142031)(271.52668701,38.30142873)
\curveto(271.99668956,38.14142063)(272.46168909,37.9664208)(272.92168701,37.77642873)
\curveto(273.10168845,37.69642107)(273.28168827,37.62642114)(273.46168701,37.56642873)
\curveto(273.64168791,37.50642126)(273.82168773,37.44142133)(274.00168701,37.37142873)
\curveto(274.11168744,37.32142145)(274.21668734,37.2714215)(274.31668701,37.22142873)
\curveto(274.40668715,37.18142159)(274.47168708,37.09642167)(274.51168701,36.96642873)
\curveto(274.52168703,36.94642182)(274.52668703,36.92142185)(274.52668701,36.89142873)
\curveto(274.51668704,36.8714219)(274.51668704,36.84642192)(274.52668701,36.81642873)
\curveto(274.53668702,36.78642198)(274.54168701,36.75142202)(274.54168701,36.71142873)
\curveto(274.53168702,36.6714221)(274.52668703,36.63142214)(274.52668701,36.59142873)
\lineto(274.52668701,36.29142873)
\curveto(274.52668703,36.19142258)(274.50168705,36.11142266)(274.45168701,36.05142873)
\curveto(274.40168715,35.9714228)(274.33168722,35.91142286)(274.24168701,35.87142873)
\curveto(274.14168741,35.84142293)(274.04168751,35.80142297)(273.94168701,35.75142873)
\curveto(273.74168781,35.6714231)(273.53668802,35.59142318)(273.32668701,35.51142873)
\curveto(273.10668845,35.44142333)(272.89668866,35.3664234)(272.69668701,35.28642873)
\curveto(272.51668904,35.20642356)(272.33668922,35.13642363)(272.15668701,35.07642873)
\curveto(271.96668959,35.02642374)(271.78168977,34.96142381)(271.60168701,34.88142873)
\curveto(271.04169051,34.65142412)(270.47669108,34.43642433)(269.90668701,34.23642873)
\curveto(269.33669222,34.03642473)(268.77169278,33.82142495)(268.21168701,33.59142873)
\lineto(267.58168701,33.35142873)
\curveto(267.36169419,33.28142549)(267.1516944,33.20642556)(266.95168701,33.12642873)
\curveto(266.84169471,33.07642569)(266.73669482,33.03142574)(266.63668701,32.99142873)
\curveto(266.52669503,32.96142581)(266.43169512,32.91142586)(266.35168701,32.84142873)
\curveto(266.33169522,32.83142594)(266.32169523,32.82142595)(266.32168701,32.81142873)
\lineto(266.29168701,32.78142873)
\lineto(266.29168701,32.70642873)
\lineto(266.32168701,32.67642873)
\curveto(266.32169523,32.6664261)(266.32669523,32.65642611)(266.33668701,32.64642873)
\curveto(266.38669517,32.62642614)(266.44169511,32.61642615)(266.50168701,32.61642873)
\curveto(266.56169499,32.61642615)(266.62169493,32.60642616)(266.68168701,32.58642873)
\lineto(266.84668701,32.58642873)
\curveto(266.90669465,32.5664262)(266.97169458,32.56142621)(267.04168701,32.57142873)
\curveto(267.11169444,32.58142619)(267.18169437,32.58642618)(267.25168701,32.58642873)
\lineto(268.06168701,32.58642873)
\lineto(272.62168701,32.58642873)
\lineto(273.80668701,32.58642873)
\curveto(273.91668764,32.58642618)(274.02668753,32.58142619)(274.13668701,32.57142873)
\curveto(274.24668731,32.5714262)(274.33168722,32.54642622)(274.39168701,32.49642873)
\curveto(274.47168708,32.44642632)(274.51668704,32.35642641)(274.52668701,32.22642873)
\lineto(274.52668701,31.83642873)
\lineto(274.52668701,31.64142873)
\curveto(274.52668703,31.59142718)(274.51668704,31.54142723)(274.49668701,31.49142873)
\curveto(274.4566871,31.36142741)(274.37168718,31.28642748)(274.24168701,31.26642873)
\curveto(274.11168744,31.25642751)(273.96168759,31.25142752)(273.79168701,31.25142873)
\lineto(272.05168701,31.25142873)
\lineto(266.05168701,31.25142873)
\lineto(264.64168701,31.25142873)
\curveto(264.53169702,31.25142752)(264.41669714,31.24642752)(264.29668701,31.23642873)
\curveto(264.17669738,31.23642753)(264.08169747,31.26142751)(264.01168701,31.31142873)
\curveto(263.9516976,31.35142742)(263.90169765,31.42642734)(263.86168701,31.53642873)
\curveto(263.8516977,31.55642721)(263.8516977,31.57642719)(263.86168701,31.59642873)
\curveto(263.86169769,31.62642714)(263.8566977,31.65142712)(263.84668701,31.67142873)
}
}
{
\newrgbcolor{curcolor}{0 0 0}
\pscustom[linestyle=none,fillstyle=solid,fillcolor=curcolor]
{
\newpath
\moveto(273.97168701,50.87353811)
\curveto(274.13168742,50.90353028)(274.26668729,50.88853029)(274.37668701,50.82853811)
\curveto(274.47668708,50.76853041)(274.551687,50.68853049)(274.60168701,50.58853811)
\curveto(274.62168693,50.53853064)(274.63168692,50.4835307)(274.63168701,50.42353811)
\curveto(274.63168692,50.37353081)(274.64168691,50.31853086)(274.66168701,50.25853811)
\curveto(274.71168684,50.03853114)(274.69668686,49.81853136)(274.61668701,49.59853811)
\curveto(274.54668701,49.38853179)(274.4566871,49.24353194)(274.34668701,49.16353811)
\curveto(274.27668728,49.11353207)(274.19668736,49.06853211)(274.10668701,49.02853811)
\curveto(274.00668755,48.98853219)(273.92668763,48.93853224)(273.86668701,48.87853811)
\curveto(273.84668771,48.85853232)(273.82668773,48.83353235)(273.80668701,48.80353811)
\curveto(273.78668777,48.7835324)(273.78168777,48.75353243)(273.79168701,48.71353811)
\curveto(273.82168773,48.60353258)(273.87668768,48.49853268)(273.95668701,48.39853811)
\curveto(274.03668752,48.30853287)(274.10668745,48.21853296)(274.16668701,48.12853811)
\curveto(274.24668731,47.99853318)(274.32168723,47.85853332)(274.39168701,47.70853811)
\curveto(274.4516871,47.55853362)(274.50668705,47.39853378)(274.55668701,47.22853811)
\curveto(274.58668697,47.12853405)(274.60668695,47.01853416)(274.61668701,46.89853811)
\curveto(274.62668693,46.78853439)(274.64168691,46.6785345)(274.66168701,46.56853811)
\curveto(274.67168688,46.51853466)(274.67668688,46.47353471)(274.67668701,46.43353811)
\lineto(274.67668701,46.32853811)
\curveto(274.69668686,46.21853496)(274.69668686,46.11353507)(274.67668701,46.01353811)
\lineto(274.67668701,45.87853811)
\curveto(274.66668689,45.82853535)(274.66168689,45.7785354)(274.66168701,45.72853811)
\curveto(274.66168689,45.6785355)(274.6516869,45.63353555)(274.63168701,45.59353811)
\curveto(274.62168693,45.55353563)(274.61668694,45.51853566)(274.61668701,45.48853811)
\curveto(274.62668693,45.46853571)(274.62668693,45.44353574)(274.61668701,45.41353811)
\lineto(274.55668701,45.17353811)
\curveto(274.54668701,45.09353609)(274.52668703,45.01853616)(274.49668701,44.94853811)
\curveto(274.36668719,44.64853653)(274.22168733,44.40353678)(274.06168701,44.21353811)
\curveto(273.89168766,44.03353715)(273.6566879,43.8835373)(273.35668701,43.76353811)
\curveto(273.13668842,43.67353751)(272.87168868,43.62853755)(272.56168701,43.62853811)
\lineto(272.24668701,43.62853811)
\curveto(272.19668936,43.63853754)(272.14668941,43.64353754)(272.09668701,43.64353811)
\lineto(271.91668701,43.67353811)
\lineto(271.58668701,43.79353811)
\curveto(271.47669008,43.83353735)(271.37669018,43.8835373)(271.28668701,43.94353811)
\curveto(270.99669056,44.12353706)(270.78169077,44.36853681)(270.64168701,44.67853811)
\curveto(270.50169105,44.98853619)(270.37669118,45.32853585)(270.26668701,45.69853811)
\curveto(270.22669133,45.83853534)(270.19669136,45.9835352)(270.17668701,46.13353811)
\curveto(270.1566914,46.2835349)(270.13169142,46.43353475)(270.10168701,46.58353811)
\curveto(270.08169147,46.65353453)(270.07169148,46.71853446)(270.07168701,46.77853811)
\curveto(270.07169148,46.84853433)(270.06169149,46.92353426)(270.04168701,47.00353811)
\curveto(270.02169153,47.07353411)(270.01169154,47.14353404)(270.01168701,47.21353811)
\curveto(270.00169155,47.2835339)(269.98669157,47.35853382)(269.96668701,47.43853811)
\curveto(269.90669165,47.68853349)(269.8566917,47.92353326)(269.81668701,48.14353811)
\curveto(269.76669179,48.36353282)(269.6516919,48.53853264)(269.47168701,48.66853811)
\curveto(269.39169216,48.72853245)(269.29169226,48.7785324)(269.17168701,48.81853811)
\curveto(269.04169251,48.85853232)(268.90169265,48.85853232)(268.75168701,48.81853811)
\curveto(268.51169304,48.75853242)(268.32169323,48.66853251)(268.18168701,48.54853811)
\curveto(268.04169351,48.43853274)(267.93169362,48.2785329)(267.85168701,48.06853811)
\curveto(267.80169375,47.94853323)(267.76669379,47.80353338)(267.74668701,47.63353811)
\curveto(267.72669383,47.47353371)(267.71669384,47.30353388)(267.71668701,47.12353811)
\curveto(267.71669384,46.94353424)(267.72669383,46.76853441)(267.74668701,46.59853811)
\curveto(267.76669379,46.42853475)(267.79669376,46.2835349)(267.83668701,46.16353811)
\curveto(267.89669366,45.99353519)(267.98169357,45.82853535)(268.09168701,45.66853811)
\curveto(268.1516934,45.58853559)(268.23169332,45.51353567)(268.33168701,45.44353811)
\curveto(268.42169313,45.3835358)(268.52169303,45.32853585)(268.63168701,45.27853811)
\curveto(268.71169284,45.24853593)(268.79669276,45.21853596)(268.88668701,45.18853811)
\curveto(268.97669258,45.16853601)(269.04669251,45.12353606)(269.09668701,45.05353811)
\curveto(269.12669243,45.01353617)(269.1516924,44.94353624)(269.17168701,44.84353811)
\curveto(269.18169237,44.75353643)(269.18669237,44.65853652)(269.18668701,44.55853811)
\curveto(269.18669237,44.45853672)(269.18169237,44.35853682)(269.17168701,44.25853811)
\curveto(269.1516924,44.16853701)(269.12669243,44.10353708)(269.09668701,44.06353811)
\curveto(269.06669249,44.02353716)(269.01669254,43.99353719)(268.94668701,43.97353811)
\curveto(268.87669268,43.95353723)(268.80169275,43.95353723)(268.72168701,43.97353811)
\curveto(268.59169296,44.00353718)(268.47169308,44.03353715)(268.36168701,44.06353811)
\curveto(268.24169331,44.10353708)(268.12669343,44.14853703)(268.01668701,44.19853811)
\curveto(267.66669389,44.38853679)(267.39669416,44.62853655)(267.20668701,44.91853811)
\curveto(267.00669455,45.20853597)(266.84669471,45.56853561)(266.72668701,45.99853811)
\curveto(266.70669485,46.09853508)(266.69169486,46.19853498)(266.68168701,46.29853811)
\curveto(266.67169488,46.40853477)(266.6566949,46.51853466)(266.63668701,46.62853811)
\curveto(266.62669493,46.66853451)(266.62669493,46.73353445)(266.63668701,46.82353811)
\curveto(266.63669492,46.91353427)(266.62669493,46.96853421)(266.60668701,46.98853811)
\curveto(266.59669496,47.68853349)(266.67669488,48.29853288)(266.84668701,48.81853811)
\curveto(267.01669454,49.33853184)(267.34169421,49.70353148)(267.82168701,49.91353811)
\curveto(268.02169353,50.00353118)(268.2566933,50.05353113)(268.52668701,50.06353811)
\curveto(268.78669277,50.0835311)(269.06169249,50.09353109)(269.35168701,50.09353811)
\lineto(272.66668701,50.09353811)
\curveto(272.80668875,50.09353109)(272.94168861,50.09853108)(273.07168701,50.10853811)
\curveto(273.20168835,50.11853106)(273.30668825,50.14853103)(273.38668701,50.19853811)
\curveto(273.4566881,50.24853093)(273.50668805,50.31353087)(273.53668701,50.39353811)
\curveto(273.57668798,50.4835307)(273.60668795,50.56853061)(273.62668701,50.64853811)
\curveto(273.63668792,50.72853045)(273.68168787,50.78853039)(273.76168701,50.82853811)
\curveto(273.79168776,50.84853033)(273.82168773,50.85853032)(273.85168701,50.85853811)
\curveto(273.88168767,50.85853032)(273.92168763,50.86353032)(273.97168701,50.87353811)
\moveto(272.30668701,48.72853811)
\curveto(272.16668939,48.78853239)(272.00668955,48.81853236)(271.82668701,48.81853811)
\curveto(271.63668992,48.82853235)(271.44169011,48.83353235)(271.24168701,48.83353811)
\curveto(271.13169042,48.83353235)(271.03169052,48.82853235)(270.94168701,48.81853811)
\curveto(270.8516907,48.80853237)(270.78169077,48.76853241)(270.73168701,48.69853811)
\curveto(270.71169084,48.66853251)(270.70169085,48.59853258)(270.70168701,48.48853811)
\curveto(270.72169083,48.46853271)(270.73169082,48.43353275)(270.73168701,48.38353811)
\curveto(270.73169082,48.33353285)(270.74169081,48.28853289)(270.76168701,48.24853811)
\curveto(270.78169077,48.16853301)(270.80169075,48.0785331)(270.82168701,47.97853811)
\lineto(270.88168701,47.67853811)
\curveto(270.88169067,47.64853353)(270.88669067,47.61353357)(270.89668701,47.57353811)
\lineto(270.89668701,47.46853811)
\curveto(270.93669062,47.31853386)(270.96169059,47.15353403)(270.97168701,46.97353811)
\curveto(270.97169058,46.80353438)(270.99169056,46.64353454)(271.03168701,46.49353811)
\curveto(271.0516905,46.41353477)(271.07169048,46.33853484)(271.09168701,46.26853811)
\curveto(271.10169045,46.20853497)(271.11669044,46.13853504)(271.13668701,46.05853811)
\curveto(271.18669037,45.89853528)(271.2516903,45.74853543)(271.33168701,45.60853811)
\curveto(271.40169015,45.46853571)(271.49169006,45.34853583)(271.60168701,45.24853811)
\curveto(271.71168984,45.14853603)(271.84668971,45.07353611)(272.00668701,45.02353811)
\curveto(272.1566894,44.97353621)(272.34168921,44.95353623)(272.56168701,44.96353811)
\curveto(272.66168889,44.96353622)(272.7566888,44.9785362)(272.84668701,45.00853811)
\curveto(272.92668863,45.04853613)(273.00168855,45.09353609)(273.07168701,45.14353811)
\curveto(273.18168837,45.22353596)(273.27668828,45.32853585)(273.35668701,45.45853811)
\curveto(273.42668813,45.58853559)(273.48668807,45.72853545)(273.53668701,45.87853811)
\curveto(273.54668801,45.92853525)(273.551688,45.9785352)(273.55168701,46.02853811)
\curveto(273.551688,46.0785351)(273.556688,46.12853505)(273.56668701,46.17853811)
\curveto(273.58668797,46.24853493)(273.60168795,46.33353485)(273.61168701,46.43353811)
\curveto(273.61168794,46.54353464)(273.60168795,46.63353455)(273.58168701,46.70353811)
\curveto(273.56168799,46.76353442)(273.556688,46.82353436)(273.56668701,46.88353811)
\curveto(273.56668799,46.94353424)(273.556688,47.00353418)(273.53668701,47.06353811)
\curveto(273.51668804,47.14353404)(273.50168805,47.21853396)(273.49168701,47.28853811)
\curveto(273.48168807,47.36853381)(273.46168809,47.44353374)(273.43168701,47.51353811)
\curveto(273.31168824,47.80353338)(273.16668839,48.04853313)(272.99668701,48.24853811)
\curveto(272.82668873,48.45853272)(272.59668896,48.61853256)(272.30668701,48.72853811)
}
}
{
\newrgbcolor{curcolor}{0 0 0}
\pscustom[linestyle=none,fillstyle=solid,fillcolor=curcolor]
{
\newpath
\moveto(266.62168701,55.69017873)
\curveto(266.62169493,55.92017394)(266.68169487,56.05017381)(266.80168701,56.08017873)
\curveto(266.91169464,56.11017375)(267.07669448,56.12517374)(267.29668701,56.12517873)
\lineto(267.58168701,56.12517873)
\curveto(267.67169388,56.12517374)(267.74669381,56.10017376)(267.80668701,56.05017873)
\curveto(267.88669367,55.99017387)(267.93169362,55.90517396)(267.94168701,55.79517873)
\curveto(267.94169361,55.68517418)(267.9566936,55.57517429)(267.98668701,55.46517873)
\curveto(268.01669354,55.32517454)(268.04669351,55.19017467)(268.07668701,55.06017873)
\curveto(268.10669345,54.94017492)(268.14669341,54.82517504)(268.19668701,54.71517873)
\curveto(268.32669323,54.42517544)(268.50669305,54.19017567)(268.73668701,54.01017873)
\curveto(268.9566926,53.83017603)(269.21169234,53.67517619)(269.50168701,53.54517873)
\curveto(269.61169194,53.50517636)(269.72669183,53.47517639)(269.84668701,53.45517873)
\curveto(269.9566916,53.43517643)(270.07169148,53.41017645)(270.19168701,53.38017873)
\curveto(270.24169131,53.37017649)(270.29169126,53.3651765)(270.34168701,53.36517873)
\curveto(270.39169116,53.37517649)(270.44169111,53.37517649)(270.49168701,53.36517873)
\curveto(270.61169094,53.33517653)(270.7516908,53.32017654)(270.91168701,53.32017873)
\curveto(271.06169049,53.33017653)(271.20669035,53.33517653)(271.34668701,53.33517873)
\lineto(273.19168701,53.33517873)
\lineto(273.53668701,53.33517873)
\curveto(273.6566879,53.33517653)(273.77168778,53.33017653)(273.88168701,53.32017873)
\curveto(273.99168756,53.31017655)(274.08668747,53.30517656)(274.16668701,53.30517873)
\curveto(274.24668731,53.31517655)(274.31668724,53.29517657)(274.37668701,53.24517873)
\curveto(274.44668711,53.19517667)(274.48668707,53.11517675)(274.49668701,53.00517873)
\curveto(274.50668705,52.90517696)(274.51168704,52.79517707)(274.51168701,52.67517873)
\lineto(274.51168701,52.40517873)
\curveto(274.49168706,52.35517751)(274.47668708,52.30517756)(274.46668701,52.25517873)
\curveto(274.44668711,52.21517765)(274.42168713,52.18517768)(274.39168701,52.16517873)
\curveto(274.32168723,52.11517775)(274.23668732,52.08517778)(274.13668701,52.07517873)
\lineto(273.80668701,52.07517873)
\lineto(272.65168701,52.07517873)
\lineto(268.49668701,52.07517873)
\lineto(267.46168701,52.07517873)
\lineto(267.16168701,52.07517873)
\curveto(267.06169449,52.08517778)(266.97669458,52.11517775)(266.90668701,52.16517873)
\curveto(266.86669469,52.19517767)(266.83669472,52.24517762)(266.81668701,52.31517873)
\curveto(266.79669476,52.39517747)(266.78669477,52.48017738)(266.78668701,52.57017873)
\curveto(266.77669478,52.6601772)(266.77669478,52.75017711)(266.78668701,52.84017873)
\curveto(266.79669476,52.93017693)(266.81169474,53.00017686)(266.83168701,53.05017873)
\curveto(266.86169469,53.13017673)(266.92169463,53.18017668)(267.01168701,53.20017873)
\curveto(267.09169446,53.23017663)(267.18169437,53.24517662)(267.28168701,53.24517873)
\lineto(267.58168701,53.24517873)
\curveto(267.68169387,53.24517662)(267.77169378,53.2651766)(267.85168701,53.30517873)
\curveto(267.87169368,53.31517655)(267.88669367,53.32517654)(267.89668701,53.33517873)
\lineto(267.94168701,53.38017873)
\curveto(267.94169361,53.49017637)(267.89669366,53.58017628)(267.80668701,53.65017873)
\curveto(267.70669385,53.72017614)(267.62669393,53.78017608)(267.56668701,53.83017873)
\lineto(267.47668701,53.92017873)
\curveto(267.36669419,54.01017585)(267.2516943,54.13517573)(267.13168701,54.29517873)
\curveto(267.01169454,54.45517541)(266.92169463,54.60517526)(266.86168701,54.74517873)
\curveto(266.81169474,54.83517503)(266.77669478,54.93017493)(266.75668701,55.03017873)
\curveto(266.72669483,55.13017473)(266.69669486,55.23517463)(266.66668701,55.34517873)
\curveto(266.6566949,55.40517446)(266.6516949,55.4651744)(266.65168701,55.52517873)
\curveto(266.64169491,55.58517428)(266.63169492,55.64017422)(266.62168701,55.69017873)
}
}
{
\newrgbcolor{curcolor}{0 0 0}
\pscustom[linestyle=none,fillstyle=solid,fillcolor=curcolor]
{
}
}
{
\newrgbcolor{curcolor}{0 0 0}
\pscustom[linestyle=none,fillstyle=solid,fillcolor=curcolor]
{
\newpath
\moveto(263.92168701,64.24510061)
\curveto(263.91169764,64.93509597)(264.03169752,65.53509537)(264.28168701,66.04510061)
\curveto(264.53169702,66.56509434)(264.86669669,66.96009395)(265.28668701,67.23010061)
\curveto(265.36669619,67.28009363)(265.4566961,67.32509358)(265.55668701,67.36510061)
\curveto(265.64669591,67.4050935)(265.74169581,67.45009346)(265.84168701,67.50010061)
\curveto(265.94169561,67.54009337)(266.04169551,67.57009334)(266.14168701,67.59010061)
\curveto(266.24169531,67.6100933)(266.34669521,67.63009328)(266.45668701,67.65010061)
\curveto(266.50669505,67.67009324)(266.551695,67.67509323)(266.59168701,67.66510061)
\curveto(266.63169492,67.65509325)(266.67669488,67.66009325)(266.72668701,67.68010061)
\curveto(266.77669478,67.69009322)(266.86169469,67.69509321)(266.98168701,67.69510061)
\curveto(267.09169446,67.69509321)(267.17669438,67.69009322)(267.23668701,67.68010061)
\curveto(267.29669426,67.66009325)(267.3566942,67.65009326)(267.41668701,67.65010061)
\curveto(267.47669408,67.66009325)(267.53669402,67.65509325)(267.59668701,67.63510061)
\curveto(267.73669382,67.59509331)(267.87169368,67.56009335)(268.00168701,67.53010061)
\curveto(268.13169342,67.50009341)(268.2566933,67.46009345)(268.37668701,67.41010061)
\curveto(268.51669304,67.35009356)(268.64169291,67.28009363)(268.75168701,67.20010061)
\curveto(268.86169269,67.13009378)(268.97169258,67.05509385)(269.08168701,66.97510061)
\lineto(269.14168701,66.91510061)
\curveto(269.16169239,66.905094)(269.18169237,66.89009402)(269.20168701,66.87010061)
\curveto(269.36169219,66.75009416)(269.50669205,66.61509429)(269.63668701,66.46510061)
\curveto(269.76669179,66.31509459)(269.89169166,66.15509475)(270.01168701,65.98510061)
\curveto(270.23169132,65.67509523)(270.43669112,65.38009553)(270.62668701,65.10010061)
\curveto(270.76669079,64.87009604)(270.90169065,64.64009627)(271.03168701,64.41010061)
\curveto(271.16169039,64.19009672)(271.29669026,63.97009694)(271.43668701,63.75010061)
\curveto(271.60668995,63.50009741)(271.78668977,63.26009765)(271.97668701,63.03010061)
\curveto(272.16668939,62.8100981)(272.39168916,62.62009829)(272.65168701,62.46010061)
\curveto(272.71168884,62.42009849)(272.77168878,62.38509852)(272.83168701,62.35510061)
\curveto(272.88168867,62.32509858)(272.94668861,62.29509861)(273.02668701,62.26510061)
\curveto(273.09668846,62.24509866)(273.1566884,62.24009867)(273.20668701,62.25010061)
\curveto(273.27668828,62.27009864)(273.33168822,62.3050986)(273.37168701,62.35510061)
\curveto(273.40168815,62.4050985)(273.42168813,62.46509844)(273.43168701,62.53510061)
\lineto(273.43168701,62.77510061)
\lineto(273.43168701,63.52510061)
\lineto(273.43168701,66.33010061)
\lineto(273.43168701,66.99010061)
\curveto(273.43168812,67.08009383)(273.43668812,67.16509374)(273.44668701,67.24510061)
\curveto(273.44668811,67.32509358)(273.46668809,67.39009352)(273.50668701,67.44010061)
\curveto(273.54668801,67.49009342)(273.62168793,67.53009338)(273.73168701,67.56010061)
\curveto(273.83168772,67.60009331)(273.93168762,67.6100933)(274.03168701,67.59010061)
\lineto(274.16668701,67.59010061)
\curveto(274.23668732,67.57009334)(274.29668726,67.55009336)(274.34668701,67.53010061)
\curveto(274.39668716,67.5100934)(274.43668712,67.47509343)(274.46668701,67.42510061)
\curveto(274.50668705,67.37509353)(274.52668703,67.3050936)(274.52668701,67.21510061)
\lineto(274.52668701,66.94510061)
\lineto(274.52668701,66.04510061)
\lineto(274.52668701,62.53510061)
\lineto(274.52668701,61.47010061)
\curveto(274.52668703,61.39009952)(274.53168702,61.30009961)(274.54168701,61.20010061)
\curveto(274.54168701,61.10009981)(274.53168702,61.01509989)(274.51168701,60.94510061)
\curveto(274.44168711,60.73510017)(274.26168729,60.67010024)(273.97168701,60.75010061)
\curveto(273.93168762,60.76010015)(273.89668766,60.76010015)(273.86668701,60.75010061)
\curveto(273.82668773,60.75010016)(273.78168777,60.76010015)(273.73168701,60.78010061)
\curveto(273.6516879,60.80010011)(273.56668799,60.82010009)(273.47668701,60.84010061)
\curveto(273.38668817,60.86010005)(273.30168825,60.88510002)(273.22168701,60.91510061)
\curveto(272.73168882,61.07509983)(272.31668924,61.27509963)(271.97668701,61.51510061)
\curveto(271.72668983,61.69509921)(271.50169005,61.90009901)(271.30168701,62.13010061)
\curveto(271.09169046,62.36009855)(270.89669066,62.60009831)(270.71668701,62.85010061)
\curveto(270.53669102,63.1100978)(270.36669119,63.37509753)(270.20668701,63.64510061)
\curveto(270.03669152,63.92509698)(269.86169169,64.19509671)(269.68168701,64.45510061)
\curveto(269.60169195,64.56509634)(269.52669203,64.67009624)(269.45668701,64.77010061)
\curveto(269.38669217,64.88009603)(269.31169224,64.99009592)(269.23168701,65.10010061)
\curveto(269.20169235,65.14009577)(269.17169238,65.17509573)(269.14168701,65.20510061)
\curveto(269.10169245,65.24509566)(269.07169248,65.28509562)(269.05168701,65.32510061)
\curveto(268.94169261,65.46509544)(268.81669274,65.59009532)(268.67668701,65.70010061)
\curveto(268.64669291,65.72009519)(268.62169293,65.74509516)(268.60168701,65.77510061)
\curveto(268.57169298,65.8050951)(268.54169301,65.83009508)(268.51168701,65.85010061)
\curveto(268.41169314,65.93009498)(268.31169324,65.99509491)(268.21168701,66.04510061)
\curveto(268.11169344,66.1050948)(268.00169355,66.16009475)(267.88168701,66.21010061)
\curveto(267.81169374,66.24009467)(267.73669382,66.26009465)(267.65668701,66.27010061)
\lineto(267.41668701,66.33010061)
\lineto(267.32668701,66.33010061)
\curveto(267.29669426,66.34009457)(267.26669429,66.34509456)(267.23668701,66.34510061)
\curveto(267.16669439,66.36509454)(267.07169448,66.37009454)(266.95168701,66.36010061)
\curveto(266.82169473,66.36009455)(266.72169483,66.35009456)(266.65168701,66.33010061)
\curveto(266.57169498,66.3100946)(266.49669506,66.29009462)(266.42668701,66.27010061)
\curveto(266.34669521,66.26009465)(266.26669529,66.24009467)(266.18668701,66.21010061)
\curveto(265.94669561,66.10009481)(265.74669581,65.95009496)(265.58668701,65.76010061)
\curveto(265.41669614,65.58009533)(265.27669628,65.36009555)(265.16668701,65.10010061)
\curveto(265.14669641,65.03009588)(265.13169642,64.96009595)(265.12168701,64.89010061)
\curveto(265.10169645,64.82009609)(265.08169647,64.74509616)(265.06168701,64.66510061)
\curveto(265.04169651,64.58509632)(265.03169652,64.47509643)(265.03168701,64.33510061)
\curveto(265.03169652,64.2050967)(265.04169651,64.10009681)(265.06168701,64.02010061)
\curveto(265.07169648,63.96009695)(265.07669648,63.905097)(265.07668701,63.85510061)
\curveto(265.07669648,63.8050971)(265.08669647,63.75509715)(265.10668701,63.70510061)
\curveto(265.14669641,63.6050973)(265.18669637,63.5100974)(265.22668701,63.42010061)
\curveto(265.26669629,63.34009757)(265.31169624,63.26009765)(265.36168701,63.18010061)
\curveto(265.38169617,63.15009776)(265.40669615,63.12009779)(265.43668701,63.09010061)
\curveto(265.46669609,63.07009784)(265.49169606,63.04509786)(265.51168701,63.01510061)
\lineto(265.58668701,62.94010061)
\curveto(265.60669595,62.910098)(265.62669593,62.88509802)(265.64668701,62.86510061)
\lineto(265.85668701,62.71510061)
\curveto(265.91669564,62.67509823)(265.98169557,62.63009828)(266.05168701,62.58010061)
\curveto(266.14169541,62.52009839)(266.24669531,62.47009844)(266.36668701,62.43010061)
\curveto(266.47669508,62.40009851)(266.58669497,62.36509854)(266.69668701,62.32510061)
\curveto(266.80669475,62.28509862)(266.9516946,62.26009865)(267.13168701,62.25010061)
\curveto(267.30169425,62.24009867)(267.42669413,62.2100987)(267.50668701,62.16010061)
\curveto(267.58669397,62.1100988)(267.63169392,62.03509887)(267.64168701,61.93510061)
\curveto(267.6516939,61.83509907)(267.6566939,61.72509918)(267.65668701,61.60510061)
\curveto(267.6566939,61.56509934)(267.66169389,61.52509938)(267.67168701,61.48510061)
\curveto(267.67169388,61.44509946)(267.66669389,61.4100995)(267.65668701,61.38010061)
\curveto(267.63669392,61.33009958)(267.62669393,61.28009963)(267.62668701,61.23010061)
\curveto(267.62669393,61.19009972)(267.61669394,61.15009976)(267.59668701,61.11010061)
\curveto(267.53669402,61.02009989)(267.40169415,60.97509993)(267.19168701,60.97510061)
\lineto(267.07168701,60.97510061)
\curveto(267.01169454,60.98509992)(266.9516946,60.99009992)(266.89168701,60.99010061)
\curveto(266.82169473,61.00009991)(266.7566948,61.0100999)(266.69668701,61.02010061)
\curveto(266.58669497,61.04009987)(266.48669507,61.06009985)(266.39668701,61.08010061)
\curveto(266.29669526,61.10009981)(266.20169535,61.13009978)(266.11168701,61.17010061)
\curveto(266.04169551,61.19009972)(265.98169557,61.2100997)(265.93168701,61.23010061)
\lineto(265.75168701,61.29010061)
\curveto(265.49169606,61.4100995)(265.24669631,61.56509934)(265.01668701,61.75510061)
\curveto(264.78669677,61.95509895)(264.60169695,62.17009874)(264.46168701,62.40010061)
\curveto(264.38169717,62.5100984)(264.31669724,62.62509828)(264.26668701,62.74510061)
\lineto(264.11668701,63.13510061)
\curveto(264.06669749,63.24509766)(264.03669752,63.36009755)(264.02668701,63.48010061)
\curveto(264.00669755,63.60009731)(263.98169757,63.72509718)(263.95168701,63.85510061)
\curveto(263.9516976,63.92509698)(263.9516976,63.99009692)(263.95168701,64.05010061)
\curveto(263.94169761,64.1100968)(263.93169762,64.17509673)(263.92168701,64.24510061)
}
}
{
\newrgbcolor{curcolor}{0 0 0}
\pscustom[linestyle=none,fillstyle=solid,fillcolor=curcolor]
{
\newpath
\moveto(263.92168701,72.59470998)
\curveto(263.91169764,73.28470535)(264.03169752,73.88470475)(264.28168701,74.39470998)
\curveto(264.53169702,74.91470372)(264.86669669,75.30970332)(265.28668701,75.57970998)
\curveto(265.36669619,75.629703)(265.4566961,75.67470296)(265.55668701,75.71470998)
\curveto(265.64669591,75.75470288)(265.74169581,75.79970283)(265.84168701,75.84970998)
\curveto(265.94169561,75.88970274)(266.04169551,75.91970271)(266.14168701,75.93970998)
\curveto(266.24169531,75.95970267)(266.34669521,75.97970265)(266.45668701,75.99970998)
\curveto(266.50669505,76.01970261)(266.551695,76.02470261)(266.59168701,76.01470998)
\curveto(266.63169492,76.00470263)(266.67669488,76.00970262)(266.72668701,76.02970998)
\curveto(266.77669478,76.03970259)(266.86169469,76.04470259)(266.98168701,76.04470998)
\curveto(267.09169446,76.04470259)(267.17669438,76.03970259)(267.23668701,76.02970998)
\curveto(267.29669426,76.00970262)(267.3566942,75.99970263)(267.41668701,75.99970998)
\curveto(267.47669408,76.00970262)(267.53669402,76.00470263)(267.59668701,75.98470998)
\curveto(267.73669382,75.94470269)(267.87169368,75.90970272)(268.00168701,75.87970998)
\curveto(268.13169342,75.84970278)(268.2566933,75.80970282)(268.37668701,75.75970998)
\curveto(268.51669304,75.69970293)(268.64169291,75.629703)(268.75168701,75.54970998)
\curveto(268.86169269,75.47970315)(268.97169258,75.40470323)(269.08168701,75.32470998)
\lineto(269.14168701,75.26470998)
\curveto(269.16169239,75.25470338)(269.18169237,75.23970339)(269.20168701,75.21970998)
\curveto(269.36169219,75.09970353)(269.50669205,74.96470367)(269.63668701,74.81470998)
\curveto(269.76669179,74.66470397)(269.89169166,74.50470413)(270.01168701,74.33470998)
\curveto(270.23169132,74.02470461)(270.43669112,73.7297049)(270.62668701,73.44970998)
\curveto(270.76669079,73.21970541)(270.90169065,72.98970564)(271.03168701,72.75970998)
\curveto(271.16169039,72.53970609)(271.29669026,72.31970631)(271.43668701,72.09970998)
\curveto(271.60668995,71.84970678)(271.78668977,71.60970702)(271.97668701,71.37970998)
\curveto(272.16668939,71.15970747)(272.39168916,70.96970766)(272.65168701,70.80970998)
\curveto(272.71168884,70.76970786)(272.77168878,70.7347079)(272.83168701,70.70470998)
\curveto(272.88168867,70.67470796)(272.94668861,70.64470799)(273.02668701,70.61470998)
\curveto(273.09668846,70.59470804)(273.1566884,70.58970804)(273.20668701,70.59970998)
\curveto(273.27668828,70.61970801)(273.33168822,70.65470798)(273.37168701,70.70470998)
\curveto(273.40168815,70.75470788)(273.42168813,70.81470782)(273.43168701,70.88470998)
\lineto(273.43168701,71.12470998)
\lineto(273.43168701,71.87470998)
\lineto(273.43168701,74.67970998)
\lineto(273.43168701,75.33970998)
\curveto(273.43168812,75.4297032)(273.43668812,75.51470312)(273.44668701,75.59470998)
\curveto(273.44668811,75.67470296)(273.46668809,75.73970289)(273.50668701,75.78970998)
\curveto(273.54668801,75.83970279)(273.62168793,75.87970275)(273.73168701,75.90970998)
\curveto(273.83168772,75.94970268)(273.93168762,75.95970267)(274.03168701,75.93970998)
\lineto(274.16668701,75.93970998)
\curveto(274.23668732,75.91970271)(274.29668726,75.89970273)(274.34668701,75.87970998)
\curveto(274.39668716,75.85970277)(274.43668712,75.82470281)(274.46668701,75.77470998)
\curveto(274.50668705,75.72470291)(274.52668703,75.65470298)(274.52668701,75.56470998)
\lineto(274.52668701,75.29470998)
\lineto(274.52668701,74.39470998)
\lineto(274.52668701,70.88470998)
\lineto(274.52668701,69.81970998)
\curveto(274.52668703,69.73970889)(274.53168702,69.64970898)(274.54168701,69.54970998)
\curveto(274.54168701,69.44970918)(274.53168702,69.36470927)(274.51168701,69.29470998)
\curveto(274.44168711,69.08470955)(274.26168729,69.01970961)(273.97168701,69.09970998)
\curveto(273.93168762,69.10970952)(273.89668766,69.10970952)(273.86668701,69.09970998)
\curveto(273.82668773,69.09970953)(273.78168777,69.10970952)(273.73168701,69.12970998)
\curveto(273.6516879,69.14970948)(273.56668799,69.16970946)(273.47668701,69.18970998)
\curveto(273.38668817,69.20970942)(273.30168825,69.2347094)(273.22168701,69.26470998)
\curveto(272.73168882,69.42470921)(272.31668924,69.62470901)(271.97668701,69.86470998)
\curveto(271.72668983,70.04470859)(271.50169005,70.24970838)(271.30168701,70.47970998)
\curveto(271.09169046,70.70970792)(270.89669066,70.94970768)(270.71668701,71.19970998)
\curveto(270.53669102,71.45970717)(270.36669119,71.72470691)(270.20668701,71.99470998)
\curveto(270.03669152,72.27470636)(269.86169169,72.54470609)(269.68168701,72.80470998)
\curveto(269.60169195,72.91470572)(269.52669203,73.01970561)(269.45668701,73.11970998)
\curveto(269.38669217,73.2297054)(269.31169224,73.33970529)(269.23168701,73.44970998)
\curveto(269.20169235,73.48970514)(269.17169238,73.52470511)(269.14168701,73.55470998)
\curveto(269.10169245,73.59470504)(269.07169248,73.634705)(269.05168701,73.67470998)
\curveto(268.94169261,73.81470482)(268.81669274,73.93970469)(268.67668701,74.04970998)
\curveto(268.64669291,74.06970456)(268.62169293,74.09470454)(268.60168701,74.12470998)
\curveto(268.57169298,74.15470448)(268.54169301,74.17970445)(268.51168701,74.19970998)
\curveto(268.41169314,74.27970435)(268.31169324,74.34470429)(268.21168701,74.39470998)
\curveto(268.11169344,74.45470418)(268.00169355,74.50970412)(267.88168701,74.55970998)
\curveto(267.81169374,74.58970404)(267.73669382,74.60970402)(267.65668701,74.61970998)
\lineto(267.41668701,74.67970998)
\lineto(267.32668701,74.67970998)
\curveto(267.29669426,74.68970394)(267.26669429,74.69470394)(267.23668701,74.69470998)
\curveto(267.16669439,74.71470392)(267.07169448,74.71970391)(266.95168701,74.70970998)
\curveto(266.82169473,74.70970392)(266.72169483,74.69970393)(266.65168701,74.67970998)
\curveto(266.57169498,74.65970397)(266.49669506,74.63970399)(266.42668701,74.61970998)
\curveto(266.34669521,74.60970402)(266.26669529,74.58970404)(266.18668701,74.55970998)
\curveto(265.94669561,74.44970418)(265.74669581,74.29970433)(265.58668701,74.10970998)
\curveto(265.41669614,73.9297047)(265.27669628,73.70970492)(265.16668701,73.44970998)
\curveto(265.14669641,73.37970525)(265.13169642,73.30970532)(265.12168701,73.23970998)
\curveto(265.10169645,73.16970546)(265.08169647,73.09470554)(265.06168701,73.01470998)
\curveto(265.04169651,72.9347057)(265.03169652,72.82470581)(265.03168701,72.68470998)
\curveto(265.03169652,72.55470608)(265.04169651,72.44970618)(265.06168701,72.36970998)
\curveto(265.07169648,72.30970632)(265.07669648,72.25470638)(265.07668701,72.20470998)
\curveto(265.07669648,72.15470648)(265.08669647,72.10470653)(265.10668701,72.05470998)
\curveto(265.14669641,71.95470668)(265.18669637,71.85970677)(265.22668701,71.76970998)
\curveto(265.26669629,71.68970694)(265.31169624,71.60970702)(265.36168701,71.52970998)
\curveto(265.38169617,71.49970713)(265.40669615,71.46970716)(265.43668701,71.43970998)
\curveto(265.46669609,71.41970721)(265.49169606,71.39470724)(265.51168701,71.36470998)
\lineto(265.58668701,71.28970998)
\curveto(265.60669595,71.25970737)(265.62669593,71.2347074)(265.64668701,71.21470998)
\lineto(265.85668701,71.06470998)
\curveto(265.91669564,71.02470761)(265.98169557,70.97970765)(266.05168701,70.92970998)
\curveto(266.14169541,70.86970776)(266.24669531,70.81970781)(266.36668701,70.77970998)
\curveto(266.47669508,70.74970788)(266.58669497,70.71470792)(266.69668701,70.67470998)
\curveto(266.80669475,70.634708)(266.9516946,70.60970802)(267.13168701,70.59970998)
\curveto(267.30169425,70.58970804)(267.42669413,70.55970807)(267.50668701,70.50970998)
\curveto(267.58669397,70.45970817)(267.63169392,70.38470825)(267.64168701,70.28470998)
\curveto(267.6516939,70.18470845)(267.6566939,70.07470856)(267.65668701,69.95470998)
\curveto(267.6566939,69.91470872)(267.66169389,69.87470876)(267.67168701,69.83470998)
\curveto(267.67169388,69.79470884)(267.66669389,69.75970887)(267.65668701,69.72970998)
\curveto(267.63669392,69.67970895)(267.62669393,69.629709)(267.62668701,69.57970998)
\curveto(267.62669393,69.53970909)(267.61669394,69.49970913)(267.59668701,69.45970998)
\curveto(267.53669402,69.36970926)(267.40169415,69.32470931)(267.19168701,69.32470998)
\lineto(267.07168701,69.32470998)
\curveto(267.01169454,69.3347093)(266.9516946,69.33970929)(266.89168701,69.33970998)
\curveto(266.82169473,69.34970928)(266.7566948,69.35970927)(266.69668701,69.36970998)
\curveto(266.58669497,69.38970924)(266.48669507,69.40970922)(266.39668701,69.42970998)
\curveto(266.29669526,69.44970918)(266.20169535,69.47970915)(266.11168701,69.51970998)
\curveto(266.04169551,69.53970909)(265.98169557,69.55970907)(265.93168701,69.57970998)
\lineto(265.75168701,69.63970998)
\curveto(265.49169606,69.75970887)(265.24669631,69.91470872)(265.01668701,70.10470998)
\curveto(264.78669677,70.30470833)(264.60169695,70.51970811)(264.46168701,70.74970998)
\curveto(264.38169717,70.85970777)(264.31669724,70.97470766)(264.26668701,71.09470998)
\lineto(264.11668701,71.48470998)
\curveto(264.06669749,71.59470704)(264.03669752,71.70970692)(264.02668701,71.82970998)
\curveto(264.00669755,71.94970668)(263.98169757,72.07470656)(263.95168701,72.20470998)
\curveto(263.9516976,72.27470636)(263.9516976,72.33970629)(263.95168701,72.39970998)
\curveto(263.94169761,72.45970617)(263.93169762,72.52470611)(263.92168701,72.59470998)
}
}
{
\newrgbcolor{curcolor}{0 0 0}
\pscustom[linestyle=none,fillstyle=solid,fillcolor=curcolor]
{
\newpath
\moveto(272.89168701,78.63431936)
\lineto(272.89168701,79.26431936)
\lineto(272.89168701,79.45931936)
\curveto(272.89168866,79.52931683)(272.90168865,79.58931677)(272.92168701,79.63931936)
\curveto(272.96168859,79.70931665)(273.00168855,79.7593166)(273.04168701,79.78931936)
\curveto(273.09168846,79.82931653)(273.1566884,79.84931651)(273.23668701,79.84931936)
\curveto(273.31668824,79.8593165)(273.40168815,79.86431649)(273.49168701,79.86431936)
\lineto(274.21168701,79.86431936)
\curveto(274.69168686,79.86431649)(275.10168645,79.80431655)(275.44168701,79.68431936)
\curveto(275.78168577,79.56431679)(276.0566855,79.36931699)(276.26668701,79.09931936)
\curveto(276.31668524,79.02931733)(276.36168519,78.9593174)(276.40168701,78.88931936)
\curveto(276.4516851,78.82931753)(276.49668506,78.7543176)(276.53668701,78.66431936)
\curveto(276.54668501,78.64431771)(276.556685,78.61431774)(276.56668701,78.57431936)
\curveto(276.58668497,78.53431782)(276.59168496,78.48931787)(276.58168701,78.43931936)
\curveto(276.551685,78.34931801)(276.47668508,78.29431806)(276.35668701,78.27431936)
\curveto(276.24668531,78.2543181)(276.1516854,78.26931809)(276.07168701,78.31931936)
\curveto(276.00168555,78.34931801)(275.93668562,78.39431796)(275.87668701,78.45431936)
\curveto(275.82668573,78.52431783)(275.77668578,78.58931777)(275.72668701,78.64931936)
\curveto(275.67668588,78.71931764)(275.60168595,78.77931758)(275.50168701,78.82931936)
\curveto(275.41168614,78.88931747)(275.32168623,78.93931742)(275.23168701,78.97931936)
\curveto(275.20168635,78.99931736)(275.14168641,79.02431733)(275.05168701,79.05431936)
\curveto(274.97168658,79.08431727)(274.90168665,79.08931727)(274.84168701,79.06931936)
\curveto(274.70168685,79.03931732)(274.61168694,78.97931738)(274.57168701,78.88931936)
\curveto(274.54168701,78.80931755)(274.52668703,78.71931764)(274.52668701,78.61931936)
\curveto(274.52668703,78.51931784)(274.50168705,78.43431792)(274.45168701,78.36431936)
\curveto(274.38168717,78.27431808)(274.24168731,78.22931813)(274.03168701,78.22931936)
\lineto(273.47668701,78.22931936)
\lineto(273.25168701,78.22931936)
\curveto(273.17168838,78.23931812)(273.10668845,78.2593181)(273.05668701,78.28931936)
\curveto(272.97668858,78.34931801)(272.93168862,78.41931794)(272.92168701,78.49931936)
\curveto(272.91168864,78.51931784)(272.90668865,78.53931782)(272.90668701,78.55931936)
\curveto(272.90668865,78.58931777)(272.90168865,78.61431774)(272.89168701,78.63431936)
}
}
{
\newrgbcolor{curcolor}{0 0 0}
\pscustom[linestyle=none,fillstyle=solid,fillcolor=curcolor]
{
}
}
{
\newrgbcolor{curcolor}{0 0 0}
\pscustom[linestyle=none,fillstyle=solid,fillcolor=curcolor]
{
\newpath
\moveto(263.92168701,89.26463186)
\curveto(263.91169764,89.95462722)(264.03169752,90.55462662)(264.28168701,91.06463186)
\curveto(264.53169702,91.58462559)(264.86669669,91.9796252)(265.28668701,92.24963186)
\curveto(265.36669619,92.29962488)(265.4566961,92.34462483)(265.55668701,92.38463186)
\curveto(265.64669591,92.42462475)(265.74169581,92.46962471)(265.84168701,92.51963186)
\curveto(265.94169561,92.55962462)(266.04169551,92.58962459)(266.14168701,92.60963186)
\curveto(266.24169531,92.62962455)(266.34669521,92.64962453)(266.45668701,92.66963186)
\curveto(266.50669505,92.68962449)(266.551695,92.69462448)(266.59168701,92.68463186)
\curveto(266.63169492,92.6746245)(266.67669488,92.6796245)(266.72668701,92.69963186)
\curveto(266.77669478,92.70962447)(266.86169469,92.71462446)(266.98168701,92.71463186)
\curveto(267.09169446,92.71462446)(267.17669438,92.70962447)(267.23668701,92.69963186)
\curveto(267.29669426,92.6796245)(267.3566942,92.66962451)(267.41668701,92.66963186)
\curveto(267.47669408,92.6796245)(267.53669402,92.6746245)(267.59668701,92.65463186)
\curveto(267.73669382,92.61462456)(267.87169368,92.5796246)(268.00168701,92.54963186)
\curveto(268.13169342,92.51962466)(268.2566933,92.4796247)(268.37668701,92.42963186)
\curveto(268.51669304,92.36962481)(268.64169291,92.29962488)(268.75168701,92.21963186)
\curveto(268.86169269,92.14962503)(268.97169258,92.0746251)(269.08168701,91.99463186)
\lineto(269.14168701,91.93463186)
\curveto(269.16169239,91.92462525)(269.18169237,91.90962527)(269.20168701,91.88963186)
\curveto(269.36169219,91.76962541)(269.50669205,91.63462554)(269.63668701,91.48463186)
\curveto(269.76669179,91.33462584)(269.89169166,91.174626)(270.01168701,91.00463186)
\curveto(270.23169132,90.69462648)(270.43669112,90.39962678)(270.62668701,90.11963186)
\curveto(270.76669079,89.88962729)(270.90169065,89.65962752)(271.03168701,89.42963186)
\curveto(271.16169039,89.20962797)(271.29669026,88.98962819)(271.43668701,88.76963186)
\curveto(271.60668995,88.51962866)(271.78668977,88.2796289)(271.97668701,88.04963186)
\curveto(272.16668939,87.82962935)(272.39168916,87.63962954)(272.65168701,87.47963186)
\curveto(272.71168884,87.43962974)(272.77168878,87.40462977)(272.83168701,87.37463186)
\curveto(272.88168867,87.34462983)(272.94668861,87.31462986)(273.02668701,87.28463186)
\curveto(273.09668846,87.26462991)(273.1566884,87.25962992)(273.20668701,87.26963186)
\curveto(273.27668828,87.28962989)(273.33168822,87.32462985)(273.37168701,87.37463186)
\curveto(273.40168815,87.42462975)(273.42168813,87.48462969)(273.43168701,87.55463186)
\lineto(273.43168701,87.79463186)
\lineto(273.43168701,88.54463186)
\lineto(273.43168701,91.34963186)
\lineto(273.43168701,92.00963186)
\curveto(273.43168812,92.09962508)(273.43668812,92.18462499)(273.44668701,92.26463186)
\curveto(273.44668811,92.34462483)(273.46668809,92.40962477)(273.50668701,92.45963186)
\curveto(273.54668801,92.50962467)(273.62168793,92.54962463)(273.73168701,92.57963186)
\curveto(273.83168772,92.61962456)(273.93168762,92.62962455)(274.03168701,92.60963186)
\lineto(274.16668701,92.60963186)
\curveto(274.23668732,92.58962459)(274.29668726,92.56962461)(274.34668701,92.54963186)
\curveto(274.39668716,92.52962465)(274.43668712,92.49462468)(274.46668701,92.44463186)
\curveto(274.50668705,92.39462478)(274.52668703,92.32462485)(274.52668701,92.23463186)
\lineto(274.52668701,91.96463186)
\lineto(274.52668701,91.06463186)
\lineto(274.52668701,87.55463186)
\lineto(274.52668701,86.48963186)
\curveto(274.52668703,86.40963077)(274.53168702,86.31963086)(274.54168701,86.21963186)
\curveto(274.54168701,86.11963106)(274.53168702,86.03463114)(274.51168701,85.96463186)
\curveto(274.44168711,85.75463142)(274.26168729,85.68963149)(273.97168701,85.76963186)
\curveto(273.93168762,85.7796314)(273.89668766,85.7796314)(273.86668701,85.76963186)
\curveto(273.82668773,85.76963141)(273.78168777,85.7796314)(273.73168701,85.79963186)
\curveto(273.6516879,85.81963136)(273.56668799,85.83963134)(273.47668701,85.85963186)
\curveto(273.38668817,85.8796313)(273.30168825,85.90463127)(273.22168701,85.93463186)
\curveto(272.73168882,86.09463108)(272.31668924,86.29463088)(271.97668701,86.53463186)
\curveto(271.72668983,86.71463046)(271.50169005,86.91963026)(271.30168701,87.14963186)
\curveto(271.09169046,87.3796298)(270.89669066,87.61962956)(270.71668701,87.86963186)
\curveto(270.53669102,88.12962905)(270.36669119,88.39462878)(270.20668701,88.66463186)
\curveto(270.03669152,88.94462823)(269.86169169,89.21462796)(269.68168701,89.47463186)
\curveto(269.60169195,89.58462759)(269.52669203,89.68962749)(269.45668701,89.78963186)
\curveto(269.38669217,89.89962728)(269.31169224,90.00962717)(269.23168701,90.11963186)
\curveto(269.20169235,90.15962702)(269.17169238,90.19462698)(269.14168701,90.22463186)
\curveto(269.10169245,90.26462691)(269.07169248,90.30462687)(269.05168701,90.34463186)
\curveto(268.94169261,90.48462669)(268.81669274,90.60962657)(268.67668701,90.71963186)
\curveto(268.64669291,90.73962644)(268.62169293,90.76462641)(268.60168701,90.79463186)
\curveto(268.57169298,90.82462635)(268.54169301,90.84962633)(268.51168701,90.86963186)
\curveto(268.41169314,90.94962623)(268.31169324,91.01462616)(268.21168701,91.06463186)
\curveto(268.11169344,91.12462605)(268.00169355,91.179626)(267.88168701,91.22963186)
\curveto(267.81169374,91.25962592)(267.73669382,91.2796259)(267.65668701,91.28963186)
\lineto(267.41668701,91.34963186)
\lineto(267.32668701,91.34963186)
\curveto(267.29669426,91.35962582)(267.26669429,91.36462581)(267.23668701,91.36463186)
\curveto(267.16669439,91.38462579)(267.07169448,91.38962579)(266.95168701,91.37963186)
\curveto(266.82169473,91.3796258)(266.72169483,91.36962581)(266.65168701,91.34963186)
\curveto(266.57169498,91.32962585)(266.49669506,91.30962587)(266.42668701,91.28963186)
\curveto(266.34669521,91.2796259)(266.26669529,91.25962592)(266.18668701,91.22963186)
\curveto(265.94669561,91.11962606)(265.74669581,90.96962621)(265.58668701,90.77963186)
\curveto(265.41669614,90.59962658)(265.27669628,90.3796268)(265.16668701,90.11963186)
\curveto(265.14669641,90.04962713)(265.13169642,89.9796272)(265.12168701,89.90963186)
\curveto(265.10169645,89.83962734)(265.08169647,89.76462741)(265.06168701,89.68463186)
\curveto(265.04169651,89.60462757)(265.03169652,89.49462768)(265.03168701,89.35463186)
\curveto(265.03169652,89.22462795)(265.04169651,89.11962806)(265.06168701,89.03963186)
\curveto(265.07169648,88.9796282)(265.07669648,88.92462825)(265.07668701,88.87463186)
\curveto(265.07669648,88.82462835)(265.08669647,88.7746284)(265.10668701,88.72463186)
\curveto(265.14669641,88.62462855)(265.18669637,88.52962865)(265.22668701,88.43963186)
\curveto(265.26669629,88.35962882)(265.31169624,88.2796289)(265.36168701,88.19963186)
\curveto(265.38169617,88.16962901)(265.40669615,88.13962904)(265.43668701,88.10963186)
\curveto(265.46669609,88.08962909)(265.49169606,88.06462911)(265.51168701,88.03463186)
\lineto(265.58668701,87.95963186)
\curveto(265.60669595,87.92962925)(265.62669593,87.90462927)(265.64668701,87.88463186)
\lineto(265.85668701,87.73463186)
\curveto(265.91669564,87.69462948)(265.98169557,87.64962953)(266.05168701,87.59963186)
\curveto(266.14169541,87.53962964)(266.24669531,87.48962969)(266.36668701,87.44963186)
\curveto(266.47669508,87.41962976)(266.58669497,87.38462979)(266.69668701,87.34463186)
\curveto(266.80669475,87.30462987)(266.9516946,87.2796299)(267.13168701,87.26963186)
\curveto(267.30169425,87.25962992)(267.42669413,87.22962995)(267.50668701,87.17963186)
\curveto(267.58669397,87.12963005)(267.63169392,87.05463012)(267.64168701,86.95463186)
\curveto(267.6516939,86.85463032)(267.6566939,86.74463043)(267.65668701,86.62463186)
\curveto(267.6566939,86.58463059)(267.66169389,86.54463063)(267.67168701,86.50463186)
\curveto(267.67169388,86.46463071)(267.66669389,86.42963075)(267.65668701,86.39963186)
\curveto(267.63669392,86.34963083)(267.62669393,86.29963088)(267.62668701,86.24963186)
\curveto(267.62669393,86.20963097)(267.61669394,86.16963101)(267.59668701,86.12963186)
\curveto(267.53669402,86.03963114)(267.40169415,85.99463118)(267.19168701,85.99463186)
\lineto(267.07168701,85.99463186)
\curveto(267.01169454,86.00463117)(266.9516946,86.00963117)(266.89168701,86.00963186)
\curveto(266.82169473,86.01963116)(266.7566948,86.02963115)(266.69668701,86.03963186)
\curveto(266.58669497,86.05963112)(266.48669507,86.0796311)(266.39668701,86.09963186)
\curveto(266.29669526,86.11963106)(266.20169535,86.14963103)(266.11168701,86.18963186)
\curveto(266.04169551,86.20963097)(265.98169557,86.22963095)(265.93168701,86.24963186)
\lineto(265.75168701,86.30963186)
\curveto(265.49169606,86.42963075)(265.24669631,86.58463059)(265.01668701,86.77463186)
\curveto(264.78669677,86.9746302)(264.60169695,87.18962999)(264.46168701,87.41963186)
\curveto(264.38169717,87.52962965)(264.31669724,87.64462953)(264.26668701,87.76463186)
\lineto(264.11668701,88.15463186)
\curveto(264.06669749,88.26462891)(264.03669752,88.3796288)(264.02668701,88.49963186)
\curveto(264.00669755,88.61962856)(263.98169757,88.74462843)(263.95168701,88.87463186)
\curveto(263.9516976,88.94462823)(263.9516976,89.00962817)(263.95168701,89.06963186)
\curveto(263.94169761,89.12962805)(263.93169762,89.19462798)(263.92168701,89.26463186)
}
}
{
\newrgbcolor{curcolor}{0 0 0}
\pscustom[linestyle=none,fillstyle=solid,fillcolor=curcolor]
{
\newpath
\moveto(269.44168701,101.36424123)
\lineto(269.69668701,101.36424123)
\curveto(269.77669178,101.37423353)(269.8516917,101.36923353)(269.92168701,101.34924123)
\lineto(270.16168701,101.34924123)
\lineto(270.32668701,101.34924123)
\curveto(270.42669113,101.32923357)(270.53169102,101.31923358)(270.64168701,101.31924123)
\curveto(270.74169081,101.31923358)(270.84169071,101.30923359)(270.94168701,101.28924123)
\lineto(271.09168701,101.28924123)
\curveto(271.23169032,101.25923364)(271.37169018,101.23923366)(271.51168701,101.22924123)
\curveto(271.64168991,101.21923368)(271.77168978,101.19423371)(271.90168701,101.15424123)
\curveto(271.98168957,101.13423377)(272.06668949,101.11423379)(272.15668701,101.09424123)
\lineto(272.39668701,101.03424123)
\lineto(272.69668701,100.91424123)
\curveto(272.78668877,100.88423402)(272.87668868,100.84923405)(272.96668701,100.80924123)
\curveto(273.18668837,100.70923419)(273.40168815,100.57423433)(273.61168701,100.40424123)
\curveto(273.82168773,100.24423466)(273.99168756,100.06923483)(274.12168701,99.87924123)
\curveto(274.16168739,99.82923507)(274.20168735,99.76923513)(274.24168701,99.69924123)
\curveto(274.27168728,99.63923526)(274.30668725,99.57923532)(274.34668701,99.51924123)
\curveto(274.39668716,99.43923546)(274.43668712,99.34423556)(274.46668701,99.23424123)
\curveto(274.49668706,99.12423578)(274.52668703,99.01923588)(274.55668701,98.91924123)
\curveto(274.59668696,98.80923609)(274.62168693,98.6992362)(274.63168701,98.58924123)
\curveto(274.64168691,98.47923642)(274.6566869,98.36423654)(274.67668701,98.24424123)
\curveto(274.68668687,98.2042367)(274.68668687,98.15923674)(274.67668701,98.10924123)
\curveto(274.67668688,98.06923683)(274.68168687,98.02923687)(274.69168701,97.98924123)
\curveto(274.70168685,97.94923695)(274.70668685,97.89423701)(274.70668701,97.82424123)
\curveto(274.70668685,97.75423715)(274.70168685,97.7042372)(274.69168701,97.67424123)
\curveto(274.67168688,97.62423728)(274.66668689,97.57923732)(274.67668701,97.53924123)
\curveto(274.68668687,97.4992374)(274.68668687,97.46423744)(274.67668701,97.43424123)
\lineto(274.67668701,97.34424123)
\curveto(274.6566869,97.28423762)(274.64168691,97.21923768)(274.63168701,97.14924123)
\curveto(274.63168692,97.08923781)(274.62668693,97.02423788)(274.61668701,96.95424123)
\curveto(274.56668699,96.78423812)(274.51668704,96.62423828)(274.46668701,96.47424123)
\curveto(274.41668714,96.32423858)(274.3516872,96.17923872)(274.27168701,96.03924123)
\curveto(274.23168732,95.98923891)(274.20168735,95.93423897)(274.18168701,95.87424123)
\curveto(274.1516874,95.82423908)(274.11668744,95.77423913)(274.07668701,95.72424123)
\curveto(273.89668766,95.48423942)(273.67668788,95.28423962)(273.41668701,95.12424123)
\curveto(273.1566884,94.96423994)(272.87168868,94.82424008)(272.56168701,94.70424123)
\curveto(272.42168913,94.64424026)(272.28168927,94.5992403)(272.14168701,94.56924123)
\curveto(271.99168956,94.53924036)(271.83668972,94.5042404)(271.67668701,94.46424123)
\curveto(271.56668999,94.44424046)(271.4566901,94.42924047)(271.34668701,94.41924123)
\curveto(271.23669032,94.40924049)(271.12669043,94.39424051)(271.01668701,94.37424123)
\curveto(270.97669058,94.36424054)(270.93669062,94.35924054)(270.89668701,94.35924123)
\curveto(270.8566907,94.36924053)(270.81669074,94.36924053)(270.77668701,94.35924123)
\curveto(270.72669083,94.34924055)(270.67669088,94.34424056)(270.62668701,94.34424123)
\lineto(270.46168701,94.34424123)
\curveto(270.41169114,94.32424058)(270.36169119,94.31924058)(270.31168701,94.32924123)
\curveto(270.2516913,94.33924056)(270.19669136,94.33924056)(270.14668701,94.32924123)
\curveto(270.10669145,94.31924058)(270.06169149,94.31924058)(270.01168701,94.32924123)
\curveto(269.96169159,94.33924056)(269.91169164,94.33424057)(269.86168701,94.31424123)
\curveto(269.79169176,94.29424061)(269.71669184,94.28924061)(269.63668701,94.29924123)
\curveto(269.54669201,94.30924059)(269.46169209,94.31424059)(269.38168701,94.31424123)
\curveto(269.29169226,94.31424059)(269.19169236,94.30924059)(269.08168701,94.29924123)
\curveto(268.96169259,94.28924061)(268.86169269,94.29424061)(268.78168701,94.31424123)
\lineto(268.49668701,94.31424123)
\lineto(267.86668701,94.35924123)
\curveto(267.76669379,94.36924053)(267.67169388,94.37924052)(267.58168701,94.38924123)
\lineto(267.28168701,94.41924123)
\curveto(267.23169432,94.43924046)(267.18169437,94.44424046)(267.13168701,94.43424123)
\curveto(267.07169448,94.43424047)(267.01669454,94.44424046)(266.96668701,94.46424123)
\curveto(266.79669476,94.51424039)(266.63169492,94.55424035)(266.47168701,94.58424123)
\curveto(266.30169525,94.61424029)(266.14169541,94.66424024)(265.99168701,94.73424123)
\curveto(265.53169602,94.92423998)(265.1566964,95.14423976)(264.86668701,95.39424123)
\curveto(264.57669698,95.65423925)(264.33169722,96.01423889)(264.13168701,96.47424123)
\curveto(264.08169747,96.6042383)(264.04669751,96.73423817)(264.02668701,96.86424123)
\curveto(264.00669755,97.0042379)(263.98169757,97.14423776)(263.95168701,97.28424123)
\curveto(263.94169761,97.35423755)(263.93669762,97.41923748)(263.93668701,97.47924123)
\curveto(263.93669762,97.53923736)(263.93169762,97.6042373)(263.92168701,97.67424123)
\curveto(263.90169765,98.5042364)(264.0516975,99.17423573)(264.37168701,99.68424123)
\curveto(264.68169687,100.19423471)(265.12169643,100.57423433)(265.69168701,100.82424123)
\curveto(265.81169574,100.87423403)(265.93669562,100.91923398)(266.06668701,100.95924123)
\curveto(266.19669536,100.9992339)(266.33169522,101.04423386)(266.47168701,101.09424123)
\curveto(266.551695,101.11423379)(266.63669492,101.12923377)(266.72668701,101.13924123)
\lineto(266.96668701,101.19924123)
\curveto(267.07669448,101.22923367)(267.18669437,101.24423366)(267.29668701,101.24424123)
\curveto(267.40669415,101.25423365)(267.51669404,101.26923363)(267.62668701,101.28924123)
\curveto(267.67669388,101.30923359)(267.72169383,101.31423359)(267.76168701,101.30424123)
\curveto(267.80169375,101.3042336)(267.84169371,101.30923359)(267.88168701,101.31924123)
\curveto(267.93169362,101.32923357)(267.98669357,101.32923357)(268.04668701,101.31924123)
\curveto(268.09669346,101.31923358)(268.14669341,101.32423358)(268.19668701,101.33424123)
\lineto(268.33168701,101.33424123)
\curveto(268.39169316,101.35423355)(268.46169309,101.35423355)(268.54168701,101.33424123)
\curveto(268.61169294,101.32423358)(268.67669288,101.32923357)(268.73668701,101.34924123)
\curveto(268.76669279,101.35923354)(268.80669275,101.36423354)(268.85668701,101.36424123)
\lineto(268.97668701,101.36424123)
\lineto(269.44168701,101.36424123)
\moveto(271.76668701,99.81924123)
\curveto(271.44669011,99.91923498)(271.08169047,99.97923492)(270.67168701,99.99924123)
\curveto(270.26169129,100.01923488)(269.8516917,100.02923487)(269.44168701,100.02924123)
\curveto(269.01169254,100.02923487)(268.59169296,100.01923488)(268.18168701,99.99924123)
\curveto(267.77169378,99.97923492)(267.38669417,99.93423497)(267.02668701,99.86424123)
\curveto(266.66669489,99.79423511)(266.34669521,99.68423522)(266.06668701,99.53424123)
\curveto(265.77669578,99.39423551)(265.54169601,99.1992357)(265.36168701,98.94924123)
\curveto(265.2516963,98.78923611)(265.17169638,98.60923629)(265.12168701,98.40924123)
\curveto(265.06169649,98.20923669)(265.03169652,97.96423694)(265.03168701,97.67424123)
\curveto(265.0516965,97.65423725)(265.06169649,97.61923728)(265.06168701,97.56924123)
\curveto(265.0516965,97.51923738)(265.0516965,97.47923742)(265.06168701,97.44924123)
\curveto(265.08169647,97.36923753)(265.10169645,97.29423761)(265.12168701,97.22424123)
\curveto(265.13169642,97.16423774)(265.1516964,97.0992378)(265.18168701,97.02924123)
\curveto(265.30169625,96.75923814)(265.47169608,96.53923836)(265.69168701,96.36924123)
\curveto(265.90169565,96.20923869)(266.14669541,96.07423883)(266.42668701,95.96424123)
\curveto(266.53669502,95.91423899)(266.6566949,95.87423903)(266.78668701,95.84424123)
\curveto(266.90669465,95.82423908)(267.03169452,95.7992391)(267.16168701,95.76924123)
\curveto(267.21169434,95.74923915)(267.26669429,95.73923916)(267.32668701,95.73924123)
\curveto(267.37669418,95.73923916)(267.42669413,95.73423917)(267.47668701,95.72424123)
\curveto(267.56669399,95.71423919)(267.66169389,95.7042392)(267.76168701,95.69424123)
\curveto(267.8516937,95.68423922)(267.94669361,95.67423923)(268.04668701,95.66424123)
\curveto(268.12669343,95.66423924)(268.21169334,95.65923924)(268.30168701,95.64924123)
\lineto(268.54168701,95.64924123)
\lineto(268.72168701,95.64924123)
\curveto(268.7516928,95.63923926)(268.78669277,95.63423927)(268.82668701,95.63424123)
\lineto(268.96168701,95.63424123)
\lineto(269.41168701,95.63424123)
\curveto(269.49169206,95.63423927)(269.57669198,95.62923927)(269.66668701,95.61924123)
\curveto(269.74669181,95.61923928)(269.82169173,95.62923927)(269.89168701,95.64924123)
\lineto(270.16168701,95.64924123)
\curveto(270.18169137,95.64923925)(270.21169134,95.64423926)(270.25168701,95.63424123)
\curveto(270.28169127,95.63423927)(270.30669125,95.63923926)(270.32668701,95.64924123)
\curveto(270.42669113,95.65923924)(270.52669103,95.66423924)(270.62668701,95.66424123)
\curveto(270.71669084,95.67423923)(270.81669074,95.68423922)(270.92668701,95.69424123)
\curveto(271.04669051,95.72423918)(271.17169038,95.73923916)(271.30168701,95.73924123)
\curveto(271.42169013,95.74923915)(271.53669002,95.77423913)(271.64668701,95.81424123)
\curveto(271.94668961,95.89423901)(272.21168934,95.97923892)(272.44168701,96.06924123)
\curveto(272.67168888,96.16923873)(272.88668867,96.31423859)(273.08668701,96.50424123)
\curveto(273.28668827,96.71423819)(273.43668812,96.97923792)(273.53668701,97.29924123)
\curveto(273.556688,97.33923756)(273.56668799,97.37423753)(273.56668701,97.40424123)
\curveto(273.556688,97.44423746)(273.56168799,97.48923741)(273.58168701,97.53924123)
\curveto(273.59168796,97.57923732)(273.60168795,97.64923725)(273.61168701,97.74924123)
\curveto(273.62168793,97.85923704)(273.61668794,97.94423696)(273.59668701,98.00424123)
\curveto(273.57668798,98.07423683)(273.56668799,98.14423676)(273.56668701,98.21424123)
\curveto(273.556688,98.28423662)(273.54168801,98.34923655)(273.52168701,98.40924123)
\curveto(273.46168809,98.60923629)(273.37668818,98.78923611)(273.26668701,98.94924123)
\curveto(273.24668831,98.97923592)(273.22668833,99.0042359)(273.20668701,99.02424123)
\lineto(273.14668701,99.08424123)
\curveto(273.12668843,99.12423578)(273.08668847,99.17423573)(273.02668701,99.23424123)
\curveto(272.88668867,99.33423557)(272.7566888,99.41923548)(272.63668701,99.48924123)
\curveto(272.51668904,99.55923534)(272.37168918,99.62923527)(272.20168701,99.69924123)
\curveto(272.13168942,99.72923517)(272.06168949,99.74923515)(271.99168701,99.75924123)
\curveto(271.92168963,99.77923512)(271.84668971,99.7992351)(271.76668701,99.81924123)
}
}
{
\newrgbcolor{curcolor}{0 0 0}
\pscustom[linestyle=none,fillstyle=solid,fillcolor=curcolor]
{
\newpath
\moveto(263.92168701,106.77385061)
\curveto(263.92169763,106.87384575)(263.93169762,106.96884566)(263.95168701,107.05885061)
\curveto(263.96169759,107.14884548)(263.99169756,107.21384541)(264.04168701,107.25385061)
\curveto(264.12169743,107.31384531)(264.22669733,107.34384528)(264.35668701,107.34385061)
\lineto(264.74668701,107.34385061)
\lineto(266.24668701,107.34385061)
\lineto(272.63668701,107.34385061)
\lineto(273.80668701,107.34385061)
\lineto(274.12168701,107.34385061)
\curveto(274.22168733,107.35384527)(274.30168725,107.33884529)(274.36168701,107.29885061)
\curveto(274.44168711,107.24884538)(274.49168706,107.17384545)(274.51168701,107.07385061)
\curveto(274.52168703,106.98384564)(274.52668703,106.87384575)(274.52668701,106.74385061)
\lineto(274.52668701,106.51885061)
\curveto(274.50668705,106.43884619)(274.49168706,106.36884626)(274.48168701,106.30885061)
\curveto(274.46168709,106.24884638)(274.42168713,106.19884643)(274.36168701,106.15885061)
\curveto(274.30168725,106.11884651)(274.22668733,106.09884653)(274.13668701,106.09885061)
\lineto(273.83668701,106.09885061)
\lineto(272.74168701,106.09885061)
\lineto(267.40168701,106.09885061)
\curveto(267.31169424,106.07884655)(267.23669432,106.06384656)(267.17668701,106.05385061)
\curveto(267.10669445,106.05384657)(267.04669451,106.0238466)(266.99668701,105.96385061)
\curveto(266.94669461,105.89384673)(266.92169463,105.80384682)(266.92168701,105.69385061)
\curveto(266.91169464,105.59384703)(266.90669465,105.48384714)(266.90668701,105.36385061)
\lineto(266.90668701,104.22385061)
\lineto(266.90668701,103.72885061)
\curveto(266.89669466,103.56884906)(266.83669472,103.45884917)(266.72668701,103.39885061)
\curveto(266.69669486,103.37884925)(266.66669489,103.36884926)(266.63668701,103.36885061)
\curveto(266.59669496,103.36884926)(266.551695,103.36384926)(266.50168701,103.35385061)
\curveto(266.38169517,103.33384929)(266.27169528,103.33884929)(266.17168701,103.36885061)
\curveto(266.07169548,103.40884922)(266.00169555,103.46384916)(265.96168701,103.53385061)
\curveto(265.91169564,103.61384901)(265.88669567,103.73384889)(265.88668701,103.89385061)
\curveto(265.88669567,104.05384857)(265.87169568,104.18884844)(265.84168701,104.29885061)
\curveto(265.83169572,104.34884828)(265.82669573,104.40384822)(265.82668701,104.46385061)
\curveto(265.81669574,104.5238481)(265.80169575,104.58384804)(265.78168701,104.64385061)
\curveto(265.73169582,104.79384783)(265.68169587,104.93884769)(265.63168701,105.07885061)
\curveto(265.57169598,105.21884741)(265.50169605,105.35384727)(265.42168701,105.48385061)
\curveto(265.33169622,105.623847)(265.22669633,105.74384688)(265.10668701,105.84385061)
\curveto(264.98669657,105.94384668)(264.8566967,106.03884659)(264.71668701,106.12885061)
\curveto(264.61669694,106.18884644)(264.50669705,106.23384639)(264.38668701,106.26385061)
\curveto(264.26669729,106.30384632)(264.16169739,106.35384627)(264.07168701,106.41385061)
\curveto(264.01169754,106.46384616)(263.97169758,106.53384609)(263.95168701,106.62385061)
\curveto(263.94169761,106.64384598)(263.93669762,106.66884596)(263.93668701,106.69885061)
\curveto(263.93669762,106.7288459)(263.93169762,106.75384587)(263.92168701,106.77385061)
}
}
{
\newrgbcolor{curcolor}{0 0 0}
\pscustom[linestyle=none,fillstyle=solid,fillcolor=curcolor]
{
\newpath
\moveto(263.92168701,115.12345998)
\curveto(263.92169763,115.22345513)(263.93169762,115.31845503)(263.95168701,115.40845998)
\curveto(263.96169759,115.49845485)(263.99169756,115.56345479)(264.04168701,115.60345998)
\curveto(264.12169743,115.66345469)(264.22669733,115.69345466)(264.35668701,115.69345998)
\lineto(264.74668701,115.69345998)
\lineto(266.24668701,115.69345998)
\lineto(272.63668701,115.69345998)
\lineto(273.80668701,115.69345998)
\lineto(274.12168701,115.69345998)
\curveto(274.22168733,115.70345465)(274.30168725,115.68845466)(274.36168701,115.64845998)
\curveto(274.44168711,115.59845475)(274.49168706,115.52345483)(274.51168701,115.42345998)
\curveto(274.52168703,115.33345502)(274.52668703,115.22345513)(274.52668701,115.09345998)
\lineto(274.52668701,114.86845998)
\curveto(274.50668705,114.78845556)(274.49168706,114.71845563)(274.48168701,114.65845998)
\curveto(274.46168709,114.59845575)(274.42168713,114.5484558)(274.36168701,114.50845998)
\curveto(274.30168725,114.46845588)(274.22668733,114.4484559)(274.13668701,114.44845998)
\lineto(273.83668701,114.44845998)
\lineto(272.74168701,114.44845998)
\lineto(267.40168701,114.44845998)
\curveto(267.31169424,114.42845592)(267.23669432,114.41345594)(267.17668701,114.40345998)
\curveto(267.10669445,114.40345595)(267.04669451,114.37345598)(266.99668701,114.31345998)
\curveto(266.94669461,114.24345611)(266.92169463,114.1534562)(266.92168701,114.04345998)
\curveto(266.91169464,113.94345641)(266.90669465,113.83345652)(266.90668701,113.71345998)
\lineto(266.90668701,112.57345998)
\lineto(266.90668701,112.07845998)
\curveto(266.89669466,111.91845843)(266.83669472,111.80845854)(266.72668701,111.74845998)
\curveto(266.69669486,111.72845862)(266.66669489,111.71845863)(266.63668701,111.71845998)
\curveto(266.59669496,111.71845863)(266.551695,111.71345864)(266.50168701,111.70345998)
\curveto(266.38169517,111.68345867)(266.27169528,111.68845866)(266.17168701,111.71845998)
\curveto(266.07169548,111.75845859)(266.00169555,111.81345854)(265.96168701,111.88345998)
\curveto(265.91169564,111.96345839)(265.88669567,112.08345827)(265.88668701,112.24345998)
\curveto(265.88669567,112.40345795)(265.87169568,112.53845781)(265.84168701,112.64845998)
\curveto(265.83169572,112.69845765)(265.82669573,112.7534576)(265.82668701,112.81345998)
\curveto(265.81669574,112.87345748)(265.80169575,112.93345742)(265.78168701,112.99345998)
\curveto(265.73169582,113.14345721)(265.68169587,113.28845706)(265.63168701,113.42845998)
\curveto(265.57169598,113.56845678)(265.50169605,113.70345665)(265.42168701,113.83345998)
\curveto(265.33169622,113.97345638)(265.22669633,114.09345626)(265.10668701,114.19345998)
\curveto(264.98669657,114.29345606)(264.8566967,114.38845596)(264.71668701,114.47845998)
\curveto(264.61669694,114.53845581)(264.50669705,114.58345577)(264.38668701,114.61345998)
\curveto(264.26669729,114.6534557)(264.16169739,114.70345565)(264.07168701,114.76345998)
\curveto(264.01169754,114.81345554)(263.97169758,114.88345547)(263.95168701,114.97345998)
\curveto(263.94169761,114.99345536)(263.93669762,115.01845533)(263.93668701,115.04845998)
\curveto(263.93669762,115.07845527)(263.93169762,115.10345525)(263.92168701,115.12345998)
}
}
{
\newrgbcolor{curcolor}{0 0 0}
\pscustom[linestyle=none,fillstyle=solid,fillcolor=curcolor]
{
\newpath
\moveto(284.75803345,31.67142873)
\lineto(284.75803345,32.58642873)
\curveto(284.75804414,32.68642608)(284.75804414,32.78142599)(284.75803345,32.87142873)
\curveto(284.75804414,32.96142581)(284.77804412,33.03642573)(284.81803345,33.09642873)
\curveto(284.87804402,33.18642558)(284.95804394,33.24642552)(285.05803345,33.27642873)
\curveto(285.15804374,33.31642545)(285.26304364,33.36142541)(285.37303345,33.41142873)
\curveto(285.56304334,33.49142528)(285.75304315,33.56142521)(285.94303345,33.62142873)
\curveto(286.13304277,33.69142508)(286.32304258,33.766425)(286.51303345,33.84642873)
\curveto(286.69304221,33.91642485)(286.87804202,33.98142479)(287.06803345,34.04142873)
\curveto(287.24804165,34.10142467)(287.42804147,34.1714246)(287.60803345,34.25142873)
\curveto(287.74804115,34.31142446)(287.89304101,34.3664244)(288.04303345,34.41642873)
\curveto(288.19304071,34.4664243)(288.33804056,34.52142425)(288.47803345,34.58142873)
\curveto(288.92803997,34.76142401)(289.38303952,34.93142384)(289.84303345,35.09142873)
\curveto(290.29303861,35.25142352)(290.74303816,35.42142335)(291.19303345,35.60142873)
\curveto(291.24303766,35.62142315)(291.29303761,35.63642313)(291.34303345,35.64642873)
\lineto(291.49303345,35.70642873)
\curveto(291.71303719,35.79642297)(291.93803696,35.88142289)(292.16803345,35.96142873)
\curveto(292.38803651,36.04142273)(292.60803629,36.12642264)(292.82803345,36.21642873)
\curveto(292.91803598,36.25642251)(293.02803587,36.29642247)(293.15803345,36.33642873)
\curveto(293.27803562,36.37642239)(293.34803555,36.44142233)(293.36803345,36.53142873)
\curveto(293.37803552,36.5714222)(293.37803552,36.60142217)(293.36803345,36.62142873)
\lineto(293.30803345,36.68142873)
\curveto(293.25803564,36.73142204)(293.2030357,36.766422)(293.14303345,36.78642873)
\curveto(293.08303582,36.81642195)(293.01803588,36.84642192)(292.94803345,36.87642873)
\lineto(292.31803345,37.11642873)
\curveto(292.0980368,37.19642157)(291.88303702,37.27642149)(291.67303345,37.35642873)
\lineto(291.52303345,37.41642873)
\lineto(291.34303345,37.47642873)
\curveto(291.15303775,37.55642121)(290.96303794,37.62642114)(290.77303345,37.68642873)
\curveto(290.57303833,37.75642101)(290.37303853,37.83142094)(290.17303345,37.91142873)
\curveto(289.59303931,38.15142062)(289.00803989,38.3714204)(288.41803345,38.57142873)
\curveto(287.82804107,38.78141999)(287.24304166,39.00641976)(286.66303345,39.24642873)
\curveto(286.46304244,39.32641944)(286.25804264,39.40141937)(286.04803345,39.47142873)
\curveto(285.83804306,39.55141922)(285.63304327,39.63141914)(285.43303345,39.71142873)
\curveto(285.35304355,39.75141902)(285.25304365,39.78641898)(285.13303345,39.81642873)
\curveto(285.01304389,39.85641891)(284.92804397,39.91141886)(284.87803345,39.98142873)
\curveto(284.83804406,40.04141873)(284.80804409,40.11641865)(284.78803345,40.20642873)
\curveto(284.76804413,40.30641846)(284.75804414,40.41641835)(284.75803345,40.53642873)
\curveto(284.74804415,40.65641811)(284.74804415,40.77641799)(284.75803345,40.89642873)
\curveto(284.75804414,41.01641775)(284.75804414,41.12641764)(284.75803345,41.22642873)
\curveto(284.75804414,41.31641745)(284.75804414,41.40641736)(284.75803345,41.49642873)
\curveto(284.75804414,41.59641717)(284.77804412,41.6714171)(284.81803345,41.72142873)
\curveto(284.86804403,41.81141696)(284.95804394,41.86141691)(285.08803345,41.87142873)
\curveto(285.21804368,41.88141689)(285.35804354,41.88641688)(285.50803345,41.88642873)
\lineto(287.15803345,41.88642873)
\lineto(293.42803345,41.88642873)
\lineto(294.68803345,41.88642873)
\curveto(294.7980341,41.88641688)(294.90803399,41.88641688)(295.01803345,41.88642873)
\curveto(295.12803377,41.89641687)(295.21303369,41.87641689)(295.27303345,41.82642873)
\curveto(295.33303357,41.79641697)(295.37303353,41.75141702)(295.39303345,41.69142873)
\curveto(295.4030335,41.63141714)(295.41803348,41.56141721)(295.43803345,41.48142873)
\lineto(295.43803345,41.24142873)
\lineto(295.43803345,40.88142873)
\curveto(295.42803347,40.771418)(295.38303352,40.69141808)(295.30303345,40.64142873)
\curveto(295.27303363,40.62141815)(295.24303366,40.60641816)(295.21303345,40.59642873)
\curveto(295.17303373,40.59641817)(295.12803377,40.58641818)(295.07803345,40.56642873)
\lineto(294.91303345,40.56642873)
\curveto(294.85303405,40.55641821)(294.78303412,40.55141822)(294.70303345,40.55142873)
\curveto(294.62303428,40.56141821)(294.54803435,40.5664182)(294.47803345,40.56642873)
\lineto(293.63803345,40.56642873)
\lineto(289.21303345,40.56642873)
\curveto(288.96303994,40.5664182)(288.71304019,40.5664182)(288.46303345,40.56642873)
\curveto(288.2030407,40.5664182)(287.95304095,40.56141821)(287.71303345,40.55142873)
\curveto(287.61304129,40.55141822)(287.5030414,40.54641822)(287.38303345,40.53642873)
\curveto(287.26304164,40.52641824)(287.2030417,40.4714183)(287.20303345,40.37142873)
\lineto(287.21803345,40.37142873)
\curveto(287.23804166,40.30141847)(287.3030416,40.24141853)(287.41303345,40.19142873)
\curveto(287.52304138,40.15141862)(287.61804128,40.11641865)(287.69803345,40.08642873)
\curveto(287.86804103,40.01641875)(288.04304086,39.95141882)(288.22303345,39.89142873)
\curveto(288.39304051,39.83141894)(288.56304034,39.76141901)(288.73303345,39.68142873)
\curveto(288.78304012,39.66141911)(288.82804007,39.64641912)(288.86803345,39.63642873)
\curveto(288.90803999,39.62641914)(288.95303995,39.61141916)(289.00303345,39.59142873)
\curveto(289.18303972,39.51141926)(289.36803953,39.44141933)(289.55803345,39.38142873)
\curveto(289.73803916,39.33141944)(289.91803898,39.2664195)(290.09803345,39.18642873)
\curveto(290.24803865,39.11641965)(290.4030385,39.05641971)(290.56303345,39.00642873)
\curveto(290.71303819,38.95641981)(290.86303804,38.90141987)(291.01303345,38.84142873)
\curveto(291.48303742,38.64142013)(291.95803694,38.46142031)(292.43803345,38.30142873)
\curveto(292.90803599,38.14142063)(293.37303553,37.9664208)(293.83303345,37.77642873)
\curveto(294.01303489,37.69642107)(294.19303471,37.62642114)(294.37303345,37.56642873)
\curveto(294.55303435,37.50642126)(294.73303417,37.44142133)(294.91303345,37.37142873)
\curveto(295.02303388,37.32142145)(295.12803377,37.2714215)(295.22803345,37.22142873)
\curveto(295.31803358,37.18142159)(295.38303352,37.09642167)(295.42303345,36.96642873)
\curveto(295.43303347,36.94642182)(295.43803346,36.92142185)(295.43803345,36.89142873)
\curveto(295.42803347,36.8714219)(295.42803347,36.84642192)(295.43803345,36.81642873)
\curveto(295.44803345,36.78642198)(295.45303345,36.75142202)(295.45303345,36.71142873)
\curveto(295.44303346,36.6714221)(295.43803346,36.63142214)(295.43803345,36.59142873)
\lineto(295.43803345,36.29142873)
\curveto(295.43803346,36.19142258)(295.41303349,36.11142266)(295.36303345,36.05142873)
\curveto(295.31303359,35.9714228)(295.24303366,35.91142286)(295.15303345,35.87142873)
\curveto(295.05303385,35.84142293)(294.95303395,35.80142297)(294.85303345,35.75142873)
\curveto(294.65303425,35.6714231)(294.44803445,35.59142318)(294.23803345,35.51142873)
\curveto(294.01803488,35.44142333)(293.80803509,35.3664234)(293.60803345,35.28642873)
\curveto(293.42803547,35.20642356)(293.24803565,35.13642363)(293.06803345,35.07642873)
\curveto(292.87803602,35.02642374)(292.69303621,34.96142381)(292.51303345,34.88142873)
\curveto(291.95303695,34.65142412)(291.38803751,34.43642433)(290.81803345,34.23642873)
\curveto(290.24803865,34.03642473)(289.68303922,33.82142495)(289.12303345,33.59142873)
\lineto(288.49303345,33.35142873)
\curveto(288.27304063,33.28142549)(288.06304084,33.20642556)(287.86303345,33.12642873)
\curveto(287.75304115,33.07642569)(287.64804125,33.03142574)(287.54803345,32.99142873)
\curveto(287.43804146,32.96142581)(287.34304156,32.91142586)(287.26303345,32.84142873)
\curveto(287.24304166,32.83142594)(287.23304167,32.82142595)(287.23303345,32.81142873)
\lineto(287.20303345,32.78142873)
\lineto(287.20303345,32.70642873)
\lineto(287.23303345,32.67642873)
\curveto(287.23304167,32.6664261)(287.23804166,32.65642611)(287.24803345,32.64642873)
\curveto(287.2980416,32.62642614)(287.35304155,32.61642615)(287.41303345,32.61642873)
\curveto(287.47304143,32.61642615)(287.53304137,32.60642616)(287.59303345,32.58642873)
\lineto(287.75803345,32.58642873)
\curveto(287.81804108,32.5664262)(287.88304102,32.56142621)(287.95303345,32.57142873)
\curveto(288.02304088,32.58142619)(288.09304081,32.58642618)(288.16303345,32.58642873)
\lineto(288.97303345,32.58642873)
\lineto(293.53303345,32.58642873)
\lineto(294.71803345,32.58642873)
\curveto(294.82803407,32.58642618)(294.93803396,32.58142619)(295.04803345,32.57142873)
\curveto(295.15803374,32.5714262)(295.24303366,32.54642622)(295.30303345,32.49642873)
\curveto(295.38303352,32.44642632)(295.42803347,32.35642641)(295.43803345,32.22642873)
\lineto(295.43803345,31.83642873)
\lineto(295.43803345,31.64142873)
\curveto(295.43803346,31.59142718)(295.42803347,31.54142723)(295.40803345,31.49142873)
\curveto(295.36803353,31.36142741)(295.28303362,31.28642748)(295.15303345,31.26642873)
\curveto(295.02303388,31.25642751)(294.87303403,31.25142752)(294.70303345,31.25142873)
\lineto(292.96303345,31.25142873)
\lineto(286.96303345,31.25142873)
\lineto(285.55303345,31.25142873)
\curveto(285.44304346,31.25142752)(285.32804357,31.24642752)(285.20803345,31.23642873)
\curveto(285.08804381,31.23642753)(284.99304391,31.26142751)(284.92303345,31.31142873)
\curveto(284.86304404,31.35142742)(284.81304409,31.42642734)(284.77303345,31.53642873)
\curveto(284.76304414,31.55642721)(284.76304414,31.57642719)(284.77303345,31.59642873)
\curveto(284.77304413,31.62642714)(284.76804413,31.65142712)(284.75803345,31.67142873)
}
}
{
\newrgbcolor{curcolor}{0 0 0}
\pscustom[linestyle=none,fillstyle=solid,fillcolor=curcolor]
{
\newpath
\moveto(294.88303345,50.87353811)
\curveto(295.04303386,50.90353028)(295.17803372,50.88853029)(295.28803345,50.82853811)
\curveto(295.38803351,50.76853041)(295.46303344,50.68853049)(295.51303345,50.58853811)
\curveto(295.53303337,50.53853064)(295.54303336,50.4835307)(295.54303345,50.42353811)
\curveto(295.54303336,50.37353081)(295.55303335,50.31853086)(295.57303345,50.25853811)
\curveto(295.62303328,50.03853114)(295.60803329,49.81853136)(295.52803345,49.59853811)
\curveto(295.45803344,49.38853179)(295.36803353,49.24353194)(295.25803345,49.16353811)
\curveto(295.18803371,49.11353207)(295.10803379,49.06853211)(295.01803345,49.02853811)
\curveto(294.91803398,48.98853219)(294.83803406,48.93853224)(294.77803345,48.87853811)
\curveto(294.75803414,48.85853232)(294.73803416,48.83353235)(294.71803345,48.80353811)
\curveto(294.6980342,48.7835324)(294.69303421,48.75353243)(294.70303345,48.71353811)
\curveto(294.73303417,48.60353258)(294.78803411,48.49853268)(294.86803345,48.39853811)
\curveto(294.94803395,48.30853287)(295.01803388,48.21853296)(295.07803345,48.12853811)
\curveto(295.15803374,47.99853318)(295.23303367,47.85853332)(295.30303345,47.70853811)
\curveto(295.36303354,47.55853362)(295.41803348,47.39853378)(295.46803345,47.22853811)
\curveto(295.4980334,47.12853405)(295.51803338,47.01853416)(295.52803345,46.89853811)
\curveto(295.53803336,46.78853439)(295.55303335,46.6785345)(295.57303345,46.56853811)
\curveto(295.58303332,46.51853466)(295.58803331,46.47353471)(295.58803345,46.43353811)
\lineto(295.58803345,46.32853811)
\curveto(295.60803329,46.21853496)(295.60803329,46.11353507)(295.58803345,46.01353811)
\lineto(295.58803345,45.87853811)
\curveto(295.57803332,45.82853535)(295.57303333,45.7785354)(295.57303345,45.72853811)
\curveto(295.57303333,45.6785355)(295.56303334,45.63353555)(295.54303345,45.59353811)
\curveto(295.53303337,45.55353563)(295.52803337,45.51853566)(295.52803345,45.48853811)
\curveto(295.53803336,45.46853571)(295.53803336,45.44353574)(295.52803345,45.41353811)
\lineto(295.46803345,45.17353811)
\curveto(295.45803344,45.09353609)(295.43803346,45.01853616)(295.40803345,44.94853811)
\curveto(295.27803362,44.64853653)(295.13303377,44.40353678)(294.97303345,44.21353811)
\curveto(294.8030341,44.03353715)(294.56803433,43.8835373)(294.26803345,43.76353811)
\curveto(294.04803485,43.67353751)(293.78303512,43.62853755)(293.47303345,43.62853811)
\lineto(293.15803345,43.62853811)
\curveto(293.10803579,43.63853754)(293.05803584,43.64353754)(293.00803345,43.64353811)
\lineto(292.82803345,43.67353811)
\lineto(292.49803345,43.79353811)
\curveto(292.38803651,43.83353735)(292.28803661,43.8835373)(292.19803345,43.94353811)
\curveto(291.90803699,44.12353706)(291.69303721,44.36853681)(291.55303345,44.67853811)
\curveto(291.41303749,44.98853619)(291.28803761,45.32853585)(291.17803345,45.69853811)
\curveto(291.13803776,45.83853534)(291.10803779,45.9835352)(291.08803345,46.13353811)
\curveto(291.06803783,46.2835349)(291.04303786,46.43353475)(291.01303345,46.58353811)
\curveto(290.99303791,46.65353453)(290.98303792,46.71853446)(290.98303345,46.77853811)
\curveto(290.98303792,46.84853433)(290.97303793,46.92353426)(290.95303345,47.00353811)
\curveto(290.93303797,47.07353411)(290.92303798,47.14353404)(290.92303345,47.21353811)
\curveto(290.91303799,47.2835339)(290.898038,47.35853382)(290.87803345,47.43853811)
\curveto(290.81803808,47.68853349)(290.76803813,47.92353326)(290.72803345,48.14353811)
\curveto(290.67803822,48.36353282)(290.56303834,48.53853264)(290.38303345,48.66853811)
\curveto(290.3030386,48.72853245)(290.2030387,48.7785324)(290.08303345,48.81853811)
\curveto(289.95303895,48.85853232)(289.81303909,48.85853232)(289.66303345,48.81853811)
\curveto(289.42303948,48.75853242)(289.23303967,48.66853251)(289.09303345,48.54853811)
\curveto(288.95303995,48.43853274)(288.84304006,48.2785329)(288.76303345,48.06853811)
\curveto(288.71304019,47.94853323)(288.67804022,47.80353338)(288.65803345,47.63353811)
\curveto(288.63804026,47.47353371)(288.62804027,47.30353388)(288.62803345,47.12353811)
\curveto(288.62804027,46.94353424)(288.63804026,46.76853441)(288.65803345,46.59853811)
\curveto(288.67804022,46.42853475)(288.70804019,46.2835349)(288.74803345,46.16353811)
\curveto(288.80804009,45.99353519)(288.89304001,45.82853535)(289.00303345,45.66853811)
\curveto(289.06303984,45.58853559)(289.14303976,45.51353567)(289.24303345,45.44353811)
\curveto(289.33303957,45.3835358)(289.43303947,45.32853585)(289.54303345,45.27853811)
\curveto(289.62303928,45.24853593)(289.70803919,45.21853596)(289.79803345,45.18853811)
\curveto(289.88803901,45.16853601)(289.95803894,45.12353606)(290.00803345,45.05353811)
\curveto(290.03803886,45.01353617)(290.06303884,44.94353624)(290.08303345,44.84353811)
\curveto(290.09303881,44.75353643)(290.0980388,44.65853652)(290.09803345,44.55853811)
\curveto(290.0980388,44.45853672)(290.09303881,44.35853682)(290.08303345,44.25853811)
\curveto(290.06303884,44.16853701)(290.03803886,44.10353708)(290.00803345,44.06353811)
\curveto(289.97803892,44.02353716)(289.92803897,43.99353719)(289.85803345,43.97353811)
\curveto(289.78803911,43.95353723)(289.71303919,43.95353723)(289.63303345,43.97353811)
\curveto(289.5030394,44.00353718)(289.38303952,44.03353715)(289.27303345,44.06353811)
\curveto(289.15303975,44.10353708)(289.03803986,44.14853703)(288.92803345,44.19853811)
\curveto(288.57804032,44.38853679)(288.30804059,44.62853655)(288.11803345,44.91853811)
\curveto(287.91804098,45.20853597)(287.75804114,45.56853561)(287.63803345,45.99853811)
\curveto(287.61804128,46.09853508)(287.6030413,46.19853498)(287.59303345,46.29853811)
\curveto(287.58304132,46.40853477)(287.56804133,46.51853466)(287.54803345,46.62853811)
\curveto(287.53804136,46.66853451)(287.53804136,46.73353445)(287.54803345,46.82353811)
\curveto(287.54804135,46.91353427)(287.53804136,46.96853421)(287.51803345,46.98853811)
\curveto(287.50804139,47.68853349)(287.58804131,48.29853288)(287.75803345,48.81853811)
\curveto(287.92804097,49.33853184)(288.25304065,49.70353148)(288.73303345,49.91353811)
\curveto(288.93303997,50.00353118)(289.16803973,50.05353113)(289.43803345,50.06353811)
\curveto(289.6980392,50.0835311)(289.97303893,50.09353109)(290.26303345,50.09353811)
\lineto(293.57803345,50.09353811)
\curveto(293.71803518,50.09353109)(293.85303505,50.09853108)(293.98303345,50.10853811)
\curveto(294.11303479,50.11853106)(294.21803468,50.14853103)(294.29803345,50.19853811)
\curveto(294.36803453,50.24853093)(294.41803448,50.31353087)(294.44803345,50.39353811)
\curveto(294.48803441,50.4835307)(294.51803438,50.56853061)(294.53803345,50.64853811)
\curveto(294.54803435,50.72853045)(294.59303431,50.78853039)(294.67303345,50.82853811)
\curveto(294.7030342,50.84853033)(294.73303417,50.85853032)(294.76303345,50.85853811)
\curveto(294.79303411,50.85853032)(294.83303407,50.86353032)(294.88303345,50.87353811)
\moveto(293.21803345,48.72853811)
\curveto(293.07803582,48.78853239)(292.91803598,48.81853236)(292.73803345,48.81853811)
\curveto(292.54803635,48.82853235)(292.35303655,48.83353235)(292.15303345,48.83353811)
\curveto(292.04303686,48.83353235)(291.94303696,48.82853235)(291.85303345,48.81853811)
\curveto(291.76303714,48.80853237)(291.69303721,48.76853241)(291.64303345,48.69853811)
\curveto(291.62303728,48.66853251)(291.61303729,48.59853258)(291.61303345,48.48853811)
\curveto(291.63303727,48.46853271)(291.64303726,48.43353275)(291.64303345,48.38353811)
\curveto(291.64303726,48.33353285)(291.65303725,48.28853289)(291.67303345,48.24853811)
\curveto(291.69303721,48.16853301)(291.71303719,48.0785331)(291.73303345,47.97853811)
\lineto(291.79303345,47.67853811)
\curveto(291.79303711,47.64853353)(291.7980371,47.61353357)(291.80803345,47.57353811)
\lineto(291.80803345,47.46853811)
\curveto(291.84803705,47.31853386)(291.87303703,47.15353403)(291.88303345,46.97353811)
\curveto(291.88303702,46.80353438)(291.903037,46.64353454)(291.94303345,46.49353811)
\curveto(291.96303694,46.41353477)(291.98303692,46.33853484)(292.00303345,46.26853811)
\curveto(292.01303689,46.20853497)(292.02803687,46.13853504)(292.04803345,46.05853811)
\curveto(292.0980368,45.89853528)(292.16303674,45.74853543)(292.24303345,45.60853811)
\curveto(292.31303659,45.46853571)(292.4030365,45.34853583)(292.51303345,45.24853811)
\curveto(292.62303628,45.14853603)(292.75803614,45.07353611)(292.91803345,45.02353811)
\curveto(293.06803583,44.97353621)(293.25303565,44.95353623)(293.47303345,44.96353811)
\curveto(293.57303533,44.96353622)(293.66803523,44.9785362)(293.75803345,45.00853811)
\curveto(293.83803506,45.04853613)(293.91303499,45.09353609)(293.98303345,45.14353811)
\curveto(294.09303481,45.22353596)(294.18803471,45.32853585)(294.26803345,45.45853811)
\curveto(294.33803456,45.58853559)(294.3980345,45.72853545)(294.44803345,45.87853811)
\curveto(294.45803444,45.92853525)(294.46303444,45.9785352)(294.46303345,46.02853811)
\curveto(294.46303444,46.0785351)(294.46803443,46.12853505)(294.47803345,46.17853811)
\curveto(294.4980344,46.24853493)(294.51303439,46.33353485)(294.52303345,46.43353811)
\curveto(294.52303438,46.54353464)(294.51303439,46.63353455)(294.49303345,46.70353811)
\curveto(294.47303443,46.76353442)(294.46803443,46.82353436)(294.47803345,46.88353811)
\curveto(294.47803442,46.94353424)(294.46803443,47.00353418)(294.44803345,47.06353811)
\curveto(294.42803447,47.14353404)(294.41303449,47.21853396)(294.40303345,47.28853811)
\curveto(294.39303451,47.36853381)(294.37303453,47.44353374)(294.34303345,47.51353811)
\curveto(294.22303468,47.80353338)(294.07803482,48.04853313)(293.90803345,48.24853811)
\curveto(293.73803516,48.45853272)(293.50803539,48.61853256)(293.21803345,48.72853811)
}
}
{
\newrgbcolor{curcolor}{0 0 0}
\pscustom[linestyle=none,fillstyle=solid,fillcolor=curcolor]
{
\newpath
\moveto(287.53303345,55.69017873)
\curveto(287.53304137,55.92017394)(287.59304131,56.05017381)(287.71303345,56.08017873)
\curveto(287.82304108,56.11017375)(287.98804091,56.12517374)(288.20803345,56.12517873)
\lineto(288.49303345,56.12517873)
\curveto(288.58304032,56.12517374)(288.65804024,56.10017376)(288.71803345,56.05017873)
\curveto(288.7980401,55.99017387)(288.84304006,55.90517396)(288.85303345,55.79517873)
\curveto(288.85304005,55.68517418)(288.86804003,55.57517429)(288.89803345,55.46517873)
\curveto(288.92803997,55.32517454)(288.95803994,55.19017467)(288.98803345,55.06017873)
\curveto(289.01803988,54.94017492)(289.05803984,54.82517504)(289.10803345,54.71517873)
\curveto(289.23803966,54.42517544)(289.41803948,54.19017567)(289.64803345,54.01017873)
\curveto(289.86803903,53.83017603)(290.12303878,53.67517619)(290.41303345,53.54517873)
\curveto(290.52303838,53.50517636)(290.63803826,53.47517639)(290.75803345,53.45517873)
\curveto(290.86803803,53.43517643)(290.98303792,53.41017645)(291.10303345,53.38017873)
\curveto(291.15303775,53.37017649)(291.2030377,53.3651765)(291.25303345,53.36517873)
\curveto(291.3030376,53.37517649)(291.35303755,53.37517649)(291.40303345,53.36517873)
\curveto(291.52303738,53.33517653)(291.66303724,53.32017654)(291.82303345,53.32017873)
\curveto(291.97303693,53.33017653)(292.11803678,53.33517653)(292.25803345,53.33517873)
\lineto(294.10303345,53.33517873)
\lineto(294.44803345,53.33517873)
\curveto(294.56803433,53.33517653)(294.68303422,53.33017653)(294.79303345,53.32017873)
\curveto(294.903034,53.31017655)(294.9980339,53.30517656)(295.07803345,53.30517873)
\curveto(295.15803374,53.31517655)(295.22803367,53.29517657)(295.28803345,53.24517873)
\curveto(295.35803354,53.19517667)(295.3980335,53.11517675)(295.40803345,53.00517873)
\curveto(295.41803348,52.90517696)(295.42303348,52.79517707)(295.42303345,52.67517873)
\lineto(295.42303345,52.40517873)
\curveto(295.4030335,52.35517751)(295.38803351,52.30517756)(295.37803345,52.25517873)
\curveto(295.35803354,52.21517765)(295.33303357,52.18517768)(295.30303345,52.16517873)
\curveto(295.23303367,52.11517775)(295.14803375,52.08517778)(295.04803345,52.07517873)
\lineto(294.71803345,52.07517873)
\lineto(293.56303345,52.07517873)
\lineto(289.40803345,52.07517873)
\lineto(288.37303345,52.07517873)
\lineto(288.07303345,52.07517873)
\curveto(287.97304093,52.08517778)(287.88804101,52.11517775)(287.81803345,52.16517873)
\curveto(287.77804112,52.19517767)(287.74804115,52.24517762)(287.72803345,52.31517873)
\curveto(287.70804119,52.39517747)(287.6980412,52.48017738)(287.69803345,52.57017873)
\curveto(287.68804121,52.6601772)(287.68804121,52.75017711)(287.69803345,52.84017873)
\curveto(287.70804119,52.93017693)(287.72304118,53.00017686)(287.74303345,53.05017873)
\curveto(287.77304113,53.13017673)(287.83304107,53.18017668)(287.92303345,53.20017873)
\curveto(288.0030409,53.23017663)(288.09304081,53.24517662)(288.19303345,53.24517873)
\lineto(288.49303345,53.24517873)
\curveto(288.59304031,53.24517662)(288.68304022,53.2651766)(288.76303345,53.30517873)
\curveto(288.78304012,53.31517655)(288.7980401,53.32517654)(288.80803345,53.33517873)
\lineto(288.85303345,53.38017873)
\curveto(288.85304005,53.49017637)(288.80804009,53.58017628)(288.71803345,53.65017873)
\curveto(288.61804028,53.72017614)(288.53804036,53.78017608)(288.47803345,53.83017873)
\lineto(288.38803345,53.92017873)
\curveto(288.27804062,54.01017585)(288.16304074,54.13517573)(288.04303345,54.29517873)
\curveto(287.92304098,54.45517541)(287.83304107,54.60517526)(287.77303345,54.74517873)
\curveto(287.72304118,54.83517503)(287.68804121,54.93017493)(287.66803345,55.03017873)
\curveto(287.63804126,55.13017473)(287.60804129,55.23517463)(287.57803345,55.34517873)
\curveto(287.56804133,55.40517446)(287.56304134,55.4651744)(287.56303345,55.52517873)
\curveto(287.55304135,55.58517428)(287.54304136,55.64017422)(287.53303345,55.69017873)
}
}
{
\newrgbcolor{curcolor}{0 0 0}
\pscustom[linestyle=none,fillstyle=solid,fillcolor=curcolor]
{
}
}
{
\newrgbcolor{curcolor}{0 0 0}
\pscustom[linestyle=none,fillstyle=solid,fillcolor=curcolor]
{
\newpath
\moveto(284.83303345,64.24510061)
\curveto(284.82304408,64.93509597)(284.94304396,65.53509537)(285.19303345,66.04510061)
\curveto(285.44304346,66.56509434)(285.77804312,66.96009395)(286.19803345,67.23010061)
\curveto(286.27804262,67.28009363)(286.36804253,67.32509358)(286.46803345,67.36510061)
\curveto(286.55804234,67.4050935)(286.65304225,67.45009346)(286.75303345,67.50010061)
\curveto(286.85304205,67.54009337)(286.95304195,67.57009334)(287.05303345,67.59010061)
\curveto(287.15304175,67.6100933)(287.25804164,67.63009328)(287.36803345,67.65010061)
\curveto(287.41804148,67.67009324)(287.46304144,67.67509323)(287.50303345,67.66510061)
\curveto(287.54304136,67.65509325)(287.58804131,67.66009325)(287.63803345,67.68010061)
\curveto(287.68804121,67.69009322)(287.77304113,67.69509321)(287.89303345,67.69510061)
\curveto(288.0030409,67.69509321)(288.08804081,67.69009322)(288.14803345,67.68010061)
\curveto(288.20804069,67.66009325)(288.26804063,67.65009326)(288.32803345,67.65010061)
\curveto(288.38804051,67.66009325)(288.44804045,67.65509325)(288.50803345,67.63510061)
\curveto(288.64804025,67.59509331)(288.78304012,67.56009335)(288.91303345,67.53010061)
\curveto(289.04303986,67.50009341)(289.16803973,67.46009345)(289.28803345,67.41010061)
\curveto(289.42803947,67.35009356)(289.55303935,67.28009363)(289.66303345,67.20010061)
\curveto(289.77303913,67.13009378)(289.88303902,67.05509385)(289.99303345,66.97510061)
\lineto(290.05303345,66.91510061)
\curveto(290.07303883,66.905094)(290.09303881,66.89009402)(290.11303345,66.87010061)
\curveto(290.27303863,66.75009416)(290.41803848,66.61509429)(290.54803345,66.46510061)
\curveto(290.67803822,66.31509459)(290.8030381,66.15509475)(290.92303345,65.98510061)
\curveto(291.14303776,65.67509523)(291.34803755,65.38009553)(291.53803345,65.10010061)
\curveto(291.67803722,64.87009604)(291.81303709,64.64009627)(291.94303345,64.41010061)
\curveto(292.07303683,64.19009672)(292.20803669,63.97009694)(292.34803345,63.75010061)
\curveto(292.51803638,63.50009741)(292.6980362,63.26009765)(292.88803345,63.03010061)
\curveto(293.07803582,62.8100981)(293.3030356,62.62009829)(293.56303345,62.46010061)
\curveto(293.62303528,62.42009849)(293.68303522,62.38509852)(293.74303345,62.35510061)
\curveto(293.79303511,62.32509858)(293.85803504,62.29509861)(293.93803345,62.26510061)
\curveto(294.00803489,62.24509866)(294.06803483,62.24009867)(294.11803345,62.25010061)
\curveto(294.18803471,62.27009864)(294.24303466,62.3050986)(294.28303345,62.35510061)
\curveto(294.31303459,62.4050985)(294.33303457,62.46509844)(294.34303345,62.53510061)
\lineto(294.34303345,62.77510061)
\lineto(294.34303345,63.52510061)
\lineto(294.34303345,66.33010061)
\lineto(294.34303345,66.99010061)
\curveto(294.34303456,67.08009383)(294.34803455,67.16509374)(294.35803345,67.24510061)
\curveto(294.35803454,67.32509358)(294.37803452,67.39009352)(294.41803345,67.44010061)
\curveto(294.45803444,67.49009342)(294.53303437,67.53009338)(294.64303345,67.56010061)
\curveto(294.74303416,67.60009331)(294.84303406,67.6100933)(294.94303345,67.59010061)
\lineto(295.07803345,67.59010061)
\curveto(295.14803375,67.57009334)(295.20803369,67.55009336)(295.25803345,67.53010061)
\curveto(295.30803359,67.5100934)(295.34803355,67.47509343)(295.37803345,67.42510061)
\curveto(295.41803348,67.37509353)(295.43803346,67.3050936)(295.43803345,67.21510061)
\lineto(295.43803345,66.94510061)
\lineto(295.43803345,66.04510061)
\lineto(295.43803345,62.53510061)
\lineto(295.43803345,61.47010061)
\curveto(295.43803346,61.39009952)(295.44303346,61.30009961)(295.45303345,61.20010061)
\curveto(295.45303345,61.10009981)(295.44303346,61.01509989)(295.42303345,60.94510061)
\curveto(295.35303355,60.73510017)(295.17303373,60.67010024)(294.88303345,60.75010061)
\curveto(294.84303406,60.76010015)(294.80803409,60.76010015)(294.77803345,60.75010061)
\curveto(294.73803416,60.75010016)(294.69303421,60.76010015)(294.64303345,60.78010061)
\curveto(294.56303434,60.80010011)(294.47803442,60.82010009)(294.38803345,60.84010061)
\curveto(294.2980346,60.86010005)(294.21303469,60.88510002)(294.13303345,60.91510061)
\curveto(293.64303526,61.07509983)(293.22803567,61.27509963)(292.88803345,61.51510061)
\curveto(292.63803626,61.69509921)(292.41303649,61.90009901)(292.21303345,62.13010061)
\curveto(292.0030369,62.36009855)(291.80803709,62.60009831)(291.62803345,62.85010061)
\curveto(291.44803745,63.1100978)(291.27803762,63.37509753)(291.11803345,63.64510061)
\curveto(290.94803795,63.92509698)(290.77303813,64.19509671)(290.59303345,64.45510061)
\curveto(290.51303839,64.56509634)(290.43803846,64.67009624)(290.36803345,64.77010061)
\curveto(290.2980386,64.88009603)(290.22303868,64.99009592)(290.14303345,65.10010061)
\curveto(290.11303879,65.14009577)(290.08303882,65.17509573)(290.05303345,65.20510061)
\curveto(290.01303889,65.24509566)(289.98303892,65.28509562)(289.96303345,65.32510061)
\curveto(289.85303905,65.46509544)(289.72803917,65.59009532)(289.58803345,65.70010061)
\curveto(289.55803934,65.72009519)(289.53303937,65.74509516)(289.51303345,65.77510061)
\curveto(289.48303942,65.8050951)(289.45303945,65.83009508)(289.42303345,65.85010061)
\curveto(289.32303958,65.93009498)(289.22303968,65.99509491)(289.12303345,66.04510061)
\curveto(289.02303988,66.1050948)(288.91303999,66.16009475)(288.79303345,66.21010061)
\curveto(288.72304018,66.24009467)(288.64804025,66.26009465)(288.56803345,66.27010061)
\lineto(288.32803345,66.33010061)
\lineto(288.23803345,66.33010061)
\curveto(288.20804069,66.34009457)(288.17804072,66.34509456)(288.14803345,66.34510061)
\curveto(288.07804082,66.36509454)(287.98304092,66.37009454)(287.86303345,66.36010061)
\curveto(287.73304117,66.36009455)(287.63304127,66.35009456)(287.56303345,66.33010061)
\curveto(287.48304142,66.3100946)(287.40804149,66.29009462)(287.33803345,66.27010061)
\curveto(287.25804164,66.26009465)(287.17804172,66.24009467)(287.09803345,66.21010061)
\curveto(286.85804204,66.10009481)(286.65804224,65.95009496)(286.49803345,65.76010061)
\curveto(286.32804257,65.58009533)(286.18804271,65.36009555)(286.07803345,65.10010061)
\curveto(286.05804284,65.03009588)(286.04304286,64.96009595)(286.03303345,64.89010061)
\curveto(286.01304289,64.82009609)(285.99304291,64.74509616)(285.97303345,64.66510061)
\curveto(285.95304295,64.58509632)(285.94304296,64.47509643)(285.94303345,64.33510061)
\curveto(285.94304296,64.2050967)(285.95304295,64.10009681)(285.97303345,64.02010061)
\curveto(285.98304292,63.96009695)(285.98804291,63.905097)(285.98803345,63.85510061)
\curveto(285.98804291,63.8050971)(285.9980429,63.75509715)(286.01803345,63.70510061)
\curveto(286.05804284,63.6050973)(286.0980428,63.5100974)(286.13803345,63.42010061)
\curveto(286.17804272,63.34009757)(286.22304268,63.26009765)(286.27303345,63.18010061)
\curveto(286.29304261,63.15009776)(286.31804258,63.12009779)(286.34803345,63.09010061)
\curveto(286.37804252,63.07009784)(286.4030425,63.04509786)(286.42303345,63.01510061)
\lineto(286.49803345,62.94010061)
\curveto(286.51804238,62.910098)(286.53804236,62.88509802)(286.55803345,62.86510061)
\lineto(286.76803345,62.71510061)
\curveto(286.82804207,62.67509823)(286.89304201,62.63009828)(286.96303345,62.58010061)
\curveto(287.05304185,62.52009839)(287.15804174,62.47009844)(287.27803345,62.43010061)
\curveto(287.38804151,62.40009851)(287.4980414,62.36509854)(287.60803345,62.32510061)
\curveto(287.71804118,62.28509862)(287.86304104,62.26009865)(288.04303345,62.25010061)
\curveto(288.21304069,62.24009867)(288.33804056,62.2100987)(288.41803345,62.16010061)
\curveto(288.4980404,62.1100988)(288.54304036,62.03509887)(288.55303345,61.93510061)
\curveto(288.56304034,61.83509907)(288.56804033,61.72509918)(288.56803345,61.60510061)
\curveto(288.56804033,61.56509934)(288.57304033,61.52509938)(288.58303345,61.48510061)
\curveto(288.58304032,61.44509946)(288.57804032,61.4100995)(288.56803345,61.38010061)
\curveto(288.54804035,61.33009958)(288.53804036,61.28009963)(288.53803345,61.23010061)
\curveto(288.53804036,61.19009972)(288.52804037,61.15009976)(288.50803345,61.11010061)
\curveto(288.44804045,61.02009989)(288.31304059,60.97509993)(288.10303345,60.97510061)
\lineto(287.98303345,60.97510061)
\curveto(287.92304098,60.98509992)(287.86304104,60.99009992)(287.80303345,60.99010061)
\curveto(287.73304117,61.00009991)(287.66804123,61.0100999)(287.60803345,61.02010061)
\curveto(287.4980414,61.04009987)(287.3980415,61.06009985)(287.30803345,61.08010061)
\curveto(287.20804169,61.10009981)(287.11304179,61.13009978)(287.02303345,61.17010061)
\curveto(286.95304195,61.19009972)(286.89304201,61.2100997)(286.84303345,61.23010061)
\lineto(286.66303345,61.29010061)
\curveto(286.4030425,61.4100995)(286.15804274,61.56509934)(285.92803345,61.75510061)
\curveto(285.6980432,61.95509895)(285.51304339,62.17009874)(285.37303345,62.40010061)
\curveto(285.29304361,62.5100984)(285.22804367,62.62509828)(285.17803345,62.74510061)
\lineto(285.02803345,63.13510061)
\curveto(284.97804392,63.24509766)(284.94804395,63.36009755)(284.93803345,63.48010061)
\curveto(284.91804398,63.60009731)(284.89304401,63.72509718)(284.86303345,63.85510061)
\curveto(284.86304404,63.92509698)(284.86304404,63.99009692)(284.86303345,64.05010061)
\curveto(284.85304405,64.1100968)(284.84304406,64.17509673)(284.83303345,64.24510061)
}
}
{
\newrgbcolor{curcolor}{0 0 0}
\pscustom[linestyle=none,fillstyle=solid,fillcolor=curcolor]
{
\newpath
\moveto(291.11803345,76.34470998)
\curveto(291.23803766,76.37470226)(291.37803752,76.39970223)(291.53803345,76.41970998)
\curveto(291.6980372,76.43970219)(291.86303704,76.44970218)(292.03303345,76.44970998)
\curveto(292.2030367,76.44970218)(292.36803653,76.43970219)(292.52803345,76.41970998)
\curveto(292.68803621,76.39970223)(292.82803607,76.37470226)(292.94803345,76.34470998)
\curveto(293.08803581,76.30470233)(293.21303569,76.26970236)(293.32303345,76.23970998)
\curveto(293.43303547,76.20970242)(293.54303536,76.16970246)(293.65303345,76.11970998)
\curveto(294.29303461,75.84970278)(294.77803412,75.4347032)(295.10803345,74.87470998)
\curveto(295.16803373,74.79470384)(295.21803368,74.70970392)(295.25803345,74.61970998)
\curveto(295.28803361,74.5297041)(295.32303358,74.4297042)(295.36303345,74.31970998)
\curveto(295.41303349,74.20970442)(295.44803345,74.08970454)(295.46803345,73.95970998)
\curveto(295.4980334,73.83970479)(295.52803337,73.70970492)(295.55803345,73.56970998)
\curveto(295.57803332,73.50970512)(295.58303332,73.44970518)(295.57303345,73.38970998)
\curveto(295.56303334,73.33970529)(295.56803333,73.27970535)(295.58803345,73.20970998)
\curveto(295.5980333,73.18970544)(295.5980333,73.16470547)(295.58803345,73.13470998)
\curveto(295.58803331,73.10470553)(295.59303331,73.07970555)(295.60303345,73.05970998)
\lineto(295.60303345,72.90970998)
\curveto(295.61303329,72.83970579)(295.61303329,72.78970584)(295.60303345,72.75970998)
\curveto(295.59303331,72.71970591)(295.58803331,72.67470596)(295.58803345,72.62470998)
\curveto(295.5980333,72.58470605)(295.5980333,72.54470609)(295.58803345,72.50470998)
\curveto(295.56803333,72.41470622)(295.55303335,72.32470631)(295.54303345,72.23470998)
\curveto(295.54303336,72.14470649)(295.53303337,72.05470658)(295.51303345,71.96470998)
\curveto(295.48303342,71.87470676)(295.45803344,71.78470685)(295.43803345,71.69470998)
\curveto(295.41803348,71.60470703)(295.38803351,71.51970711)(295.34803345,71.43970998)
\curveto(295.23803366,71.19970743)(295.10803379,70.97470766)(294.95803345,70.76470998)
\curveto(294.7980341,70.55470808)(294.61803428,70.37470826)(294.41803345,70.22470998)
\curveto(294.24803465,70.10470853)(294.07303483,69.99970863)(293.89303345,69.90970998)
\curveto(293.71303519,69.81970881)(293.52303538,69.7297089)(293.32303345,69.63970998)
\curveto(293.22303568,69.59970903)(293.12303578,69.56470907)(293.02303345,69.53470998)
\curveto(292.91303599,69.51470912)(292.8030361,69.48970914)(292.69303345,69.45970998)
\curveto(292.55303635,69.41970921)(292.41303649,69.39470924)(292.27303345,69.38470998)
\curveto(292.13303677,69.37470926)(291.99303691,69.35470928)(291.85303345,69.32470998)
\curveto(291.74303716,69.31470932)(291.64303726,69.30470933)(291.55303345,69.29470998)
\curveto(291.45303745,69.29470934)(291.35303755,69.28470935)(291.25303345,69.26470998)
\lineto(291.16303345,69.26470998)
\curveto(291.13303777,69.27470936)(291.10803779,69.27470936)(291.08803345,69.26470998)
\lineto(290.87803345,69.26470998)
\curveto(290.81803808,69.24470939)(290.75303815,69.2347094)(290.68303345,69.23470998)
\curveto(290.6030383,69.24470939)(290.52803837,69.24970938)(290.45803345,69.24970998)
\lineto(290.30803345,69.24970998)
\curveto(290.25803864,69.24970938)(290.20803869,69.25470938)(290.15803345,69.26470998)
\lineto(289.78303345,69.26470998)
\curveto(289.75303915,69.27470936)(289.71803918,69.27470936)(289.67803345,69.26470998)
\curveto(289.63803926,69.26470937)(289.5980393,69.26970936)(289.55803345,69.27970998)
\curveto(289.44803945,69.29970933)(289.33803956,69.31470932)(289.22803345,69.32470998)
\curveto(289.10803979,69.3347093)(288.99303991,69.34470929)(288.88303345,69.35470998)
\curveto(288.73304017,69.39470924)(288.58804031,69.41970921)(288.44803345,69.42970998)
\curveto(288.2980406,69.44970918)(288.15304075,69.47970915)(288.01303345,69.51970998)
\curveto(287.71304119,69.60970902)(287.42804147,69.70470893)(287.15803345,69.80470998)
\curveto(286.88804201,69.90470873)(286.63804226,70.0297086)(286.40803345,70.17970998)
\curveto(286.08804281,70.37970825)(285.80804309,70.62470801)(285.56803345,70.91470998)
\curveto(285.32804357,71.20470743)(285.14304376,71.54470709)(285.01303345,71.93470998)
\curveto(284.97304393,72.04470659)(284.94804395,72.15470648)(284.93803345,72.26470998)
\curveto(284.91804398,72.38470625)(284.89304401,72.50470613)(284.86303345,72.62470998)
\curveto(284.85304405,72.69470594)(284.84804405,72.75970587)(284.84803345,72.81970998)
\curveto(284.84804405,72.87970575)(284.84304406,72.94470569)(284.83303345,73.01470998)
\curveto(284.81304409,73.71470492)(284.92804397,74.28970434)(285.17803345,74.73970998)
\curveto(285.42804347,75.18970344)(285.77804312,75.5347031)(286.22803345,75.77470998)
\curveto(286.45804244,75.88470275)(286.73304217,75.98470265)(287.05303345,76.07470998)
\curveto(287.12304178,76.09470254)(287.1980417,76.09470254)(287.27803345,76.07470998)
\curveto(287.34804155,76.06470257)(287.3980415,76.03970259)(287.42803345,75.99970998)
\curveto(287.45804144,75.96970266)(287.48304142,75.90970272)(287.50303345,75.81970998)
\curveto(287.51304139,75.7297029)(287.52304138,75.629703)(287.53303345,75.51970998)
\curveto(287.53304137,75.41970321)(287.52804137,75.31970331)(287.51803345,75.21970998)
\curveto(287.50804139,75.1297035)(287.48804141,75.06470357)(287.45803345,75.02470998)
\curveto(287.38804151,74.91470372)(287.27804162,74.8347038)(287.12803345,74.78470998)
\curveto(286.97804192,74.74470389)(286.84804205,74.68970394)(286.73803345,74.61970998)
\curveto(286.42804247,74.4297042)(286.1980427,74.14970448)(286.04803345,73.77970998)
\curveto(286.01804288,73.70970492)(285.9980429,73.634705)(285.98803345,73.55470998)
\curveto(285.97804292,73.48470515)(285.96304294,73.40970522)(285.94303345,73.32970998)
\curveto(285.93304297,73.27970535)(285.92804297,73.20970542)(285.92803345,73.11970998)
\curveto(285.92804297,73.03970559)(285.93304297,72.97470566)(285.94303345,72.92470998)
\curveto(285.96304294,72.88470575)(285.96804293,72.84970578)(285.95803345,72.81970998)
\curveto(285.94804295,72.78970584)(285.94804295,72.75470588)(285.95803345,72.71470998)
\lineto(286.01803345,72.47470998)
\curveto(286.03804286,72.40470623)(286.06304284,72.3347063)(286.09303345,72.26470998)
\curveto(286.25304265,71.88470675)(286.46304244,71.59470704)(286.72303345,71.39470998)
\curveto(286.98304192,71.20470743)(287.2980416,71.0297076)(287.66803345,70.86970998)
\curveto(287.74804115,70.83970779)(287.82804107,70.81470782)(287.90803345,70.79470998)
\curveto(287.98804091,70.78470785)(288.06804083,70.76470787)(288.14803345,70.73470998)
\curveto(288.25804064,70.70470793)(288.37304053,70.67970795)(288.49303345,70.65970998)
\curveto(288.61304029,70.64970798)(288.73304017,70.629708)(288.85303345,70.59970998)
\curveto(288.90304,70.57970805)(288.95303995,70.56970806)(289.00303345,70.56970998)
\curveto(289.05303985,70.57970805)(289.1030398,70.57470806)(289.15303345,70.55470998)
\curveto(289.21303969,70.54470809)(289.29303961,70.54470809)(289.39303345,70.55470998)
\curveto(289.48303942,70.56470807)(289.53803936,70.57970805)(289.55803345,70.59970998)
\curveto(289.5980393,70.61970801)(289.61803928,70.64970798)(289.61803345,70.68970998)
\curveto(289.61803928,70.73970789)(289.60803929,70.78470785)(289.58803345,70.82470998)
\curveto(289.54803935,70.89470774)(289.5030394,70.95470768)(289.45303345,71.00470998)
\curveto(289.4030395,71.05470758)(289.35303955,71.11470752)(289.30303345,71.18470998)
\lineto(289.24303345,71.24470998)
\curveto(289.21303969,71.27470736)(289.18803971,71.30470733)(289.16803345,71.33470998)
\curveto(289.00803989,71.56470707)(288.87304003,71.83970679)(288.76303345,72.15970998)
\curveto(288.74304016,72.2297064)(288.72804017,72.29970633)(288.71803345,72.36970998)
\curveto(288.70804019,72.43970619)(288.69304021,72.51470612)(288.67303345,72.59470998)
\curveto(288.67304023,72.634706)(288.66804023,72.66970596)(288.65803345,72.69970998)
\curveto(288.64804025,72.7297059)(288.64804025,72.76470587)(288.65803345,72.80470998)
\curveto(288.65804024,72.85470578)(288.64804025,72.89470574)(288.62803345,72.92470998)
\lineto(288.62803345,73.08970998)
\lineto(288.62803345,73.17970998)
\curveto(288.61804028,73.2297054)(288.61804028,73.26970536)(288.62803345,73.29970998)
\curveto(288.63804026,73.34970528)(288.64304026,73.39970523)(288.64303345,73.44970998)
\curveto(288.63304027,73.50970512)(288.63304027,73.56470507)(288.64303345,73.61470998)
\curveto(288.67304023,73.72470491)(288.69304021,73.8297048)(288.70303345,73.92970998)
\curveto(288.71304019,74.03970459)(288.73804016,74.14470449)(288.77803345,74.24470998)
\curveto(288.91803998,74.66470397)(289.1030398,75.00970362)(289.33303345,75.27970998)
\curveto(289.55303935,75.54970308)(289.83803906,75.78970284)(290.18803345,75.99970998)
\curveto(290.32803857,76.07970255)(290.47803842,76.14470249)(290.63803345,76.19470998)
\curveto(290.78803811,76.24470239)(290.94803795,76.29470234)(291.11803345,76.34470998)
\moveto(292.42303345,75.09970998)
\curveto(292.37303653,75.10970352)(292.32803657,75.11470352)(292.28803345,75.11470998)
\lineto(292.13803345,75.11470998)
\curveto(291.82803707,75.11470352)(291.54303736,75.07470356)(291.28303345,74.99470998)
\curveto(291.22303768,74.97470366)(291.16803773,74.95470368)(291.11803345,74.93470998)
\curveto(291.05803784,74.92470371)(291.0030379,74.90970372)(290.95303345,74.88970998)
\curveto(290.46303844,74.66970396)(290.11303879,74.32470431)(289.90303345,73.85470998)
\curveto(289.87303903,73.77470486)(289.84803905,73.69470494)(289.82803345,73.61470998)
\lineto(289.76803345,73.37470998)
\curveto(289.74803915,73.29470534)(289.73803916,73.20470543)(289.73803345,73.10470998)
\lineto(289.73803345,72.78970998)
\curveto(289.75803914,72.76970586)(289.76803913,72.7297059)(289.76803345,72.66970998)
\curveto(289.75803914,72.61970601)(289.75803914,72.57470606)(289.76803345,72.53470998)
\lineto(289.82803345,72.29470998)
\curveto(289.83803906,72.22470641)(289.85803904,72.15470648)(289.88803345,72.08470998)
\curveto(290.14803875,71.48470715)(290.61303829,71.07970755)(291.28303345,70.86970998)
\curveto(291.36303754,70.83970779)(291.44303746,70.81970781)(291.52303345,70.80970998)
\curveto(291.6030373,70.79970783)(291.68803721,70.78470785)(291.77803345,70.76470998)
\lineto(291.92803345,70.76470998)
\curveto(291.96803693,70.75470788)(292.03803686,70.74970788)(292.13803345,70.74970998)
\curveto(292.36803653,70.74970788)(292.56303634,70.76970786)(292.72303345,70.80970998)
\curveto(292.79303611,70.8297078)(292.85803604,70.84470779)(292.91803345,70.85470998)
\curveto(292.97803592,70.86470777)(293.04303586,70.88470775)(293.11303345,70.91470998)
\curveto(293.39303551,71.02470761)(293.63803526,71.16970746)(293.84803345,71.34970998)
\curveto(294.04803485,71.5297071)(294.20803469,71.76470687)(294.32803345,72.05470998)
\lineto(294.41803345,72.29470998)
\lineto(294.47803345,72.53470998)
\curveto(294.4980344,72.58470605)(294.5030344,72.62470601)(294.49303345,72.65470998)
\curveto(294.48303442,72.69470594)(294.48803441,72.73970589)(294.50803345,72.78970998)
\curveto(294.51803438,72.81970581)(294.52303438,72.87470576)(294.52303345,72.95470998)
\curveto(294.52303438,73.0347056)(294.51803438,73.09470554)(294.50803345,73.13470998)
\curveto(294.48803441,73.24470539)(294.47303443,73.34970528)(294.46303345,73.44970998)
\curveto(294.45303445,73.54970508)(294.42303448,73.64470499)(294.37303345,73.73470998)
\curveto(294.17303473,74.26470437)(293.7980351,74.65470398)(293.24803345,74.90470998)
\curveto(293.14803575,74.94470369)(293.04303586,74.97470366)(292.93303345,74.99470998)
\lineto(292.60303345,75.08470998)
\curveto(292.52303638,75.08470355)(292.46303644,75.08970354)(292.42303345,75.09970998)
}
}
{
\newrgbcolor{curcolor}{0 0 0}
\pscustom[linestyle=none,fillstyle=solid,fillcolor=curcolor]
{
\newpath
\moveto(293.80303345,78.63431936)
\lineto(293.80303345,79.26431936)
\lineto(293.80303345,79.45931936)
\curveto(293.8030351,79.52931683)(293.81303509,79.58931677)(293.83303345,79.63931936)
\curveto(293.87303503,79.70931665)(293.91303499,79.7593166)(293.95303345,79.78931936)
\curveto(294.0030349,79.82931653)(294.06803483,79.84931651)(294.14803345,79.84931936)
\curveto(294.22803467,79.8593165)(294.31303459,79.86431649)(294.40303345,79.86431936)
\lineto(295.12303345,79.86431936)
\curveto(295.6030333,79.86431649)(296.01303289,79.80431655)(296.35303345,79.68431936)
\curveto(296.69303221,79.56431679)(296.96803193,79.36931699)(297.17803345,79.09931936)
\curveto(297.22803167,79.02931733)(297.27303163,78.9593174)(297.31303345,78.88931936)
\curveto(297.36303154,78.82931753)(297.40803149,78.7543176)(297.44803345,78.66431936)
\curveto(297.45803144,78.64431771)(297.46803143,78.61431774)(297.47803345,78.57431936)
\curveto(297.4980314,78.53431782)(297.5030314,78.48931787)(297.49303345,78.43931936)
\curveto(297.46303144,78.34931801)(297.38803151,78.29431806)(297.26803345,78.27431936)
\curveto(297.15803174,78.2543181)(297.06303184,78.26931809)(296.98303345,78.31931936)
\curveto(296.91303199,78.34931801)(296.84803205,78.39431796)(296.78803345,78.45431936)
\curveto(296.73803216,78.52431783)(296.68803221,78.58931777)(296.63803345,78.64931936)
\curveto(296.58803231,78.71931764)(296.51303239,78.77931758)(296.41303345,78.82931936)
\curveto(296.32303258,78.88931747)(296.23303267,78.93931742)(296.14303345,78.97931936)
\curveto(296.11303279,78.99931736)(296.05303285,79.02431733)(295.96303345,79.05431936)
\curveto(295.88303302,79.08431727)(295.81303309,79.08931727)(295.75303345,79.06931936)
\curveto(295.61303329,79.03931732)(295.52303338,78.97931738)(295.48303345,78.88931936)
\curveto(295.45303345,78.80931755)(295.43803346,78.71931764)(295.43803345,78.61931936)
\curveto(295.43803346,78.51931784)(295.41303349,78.43431792)(295.36303345,78.36431936)
\curveto(295.29303361,78.27431808)(295.15303375,78.22931813)(294.94303345,78.22931936)
\lineto(294.38803345,78.22931936)
\lineto(294.16303345,78.22931936)
\curveto(294.08303482,78.23931812)(294.01803488,78.2593181)(293.96803345,78.28931936)
\curveto(293.88803501,78.34931801)(293.84303506,78.41931794)(293.83303345,78.49931936)
\curveto(293.82303508,78.51931784)(293.81803508,78.53931782)(293.81803345,78.55931936)
\curveto(293.81803508,78.58931777)(293.81303509,78.61431774)(293.80303345,78.63431936)
}
}
{
\newrgbcolor{curcolor}{0 0 0}
\pscustom[linestyle=none,fillstyle=solid,fillcolor=curcolor]
{
}
}
{
\newrgbcolor{curcolor}{0 0 0}
\pscustom[linestyle=none,fillstyle=solid,fillcolor=curcolor]
{
\newpath
\moveto(284.83303345,89.26463186)
\curveto(284.82304408,89.95462722)(284.94304396,90.55462662)(285.19303345,91.06463186)
\curveto(285.44304346,91.58462559)(285.77804312,91.9796252)(286.19803345,92.24963186)
\curveto(286.27804262,92.29962488)(286.36804253,92.34462483)(286.46803345,92.38463186)
\curveto(286.55804234,92.42462475)(286.65304225,92.46962471)(286.75303345,92.51963186)
\curveto(286.85304205,92.55962462)(286.95304195,92.58962459)(287.05303345,92.60963186)
\curveto(287.15304175,92.62962455)(287.25804164,92.64962453)(287.36803345,92.66963186)
\curveto(287.41804148,92.68962449)(287.46304144,92.69462448)(287.50303345,92.68463186)
\curveto(287.54304136,92.6746245)(287.58804131,92.6796245)(287.63803345,92.69963186)
\curveto(287.68804121,92.70962447)(287.77304113,92.71462446)(287.89303345,92.71463186)
\curveto(288.0030409,92.71462446)(288.08804081,92.70962447)(288.14803345,92.69963186)
\curveto(288.20804069,92.6796245)(288.26804063,92.66962451)(288.32803345,92.66963186)
\curveto(288.38804051,92.6796245)(288.44804045,92.6746245)(288.50803345,92.65463186)
\curveto(288.64804025,92.61462456)(288.78304012,92.5796246)(288.91303345,92.54963186)
\curveto(289.04303986,92.51962466)(289.16803973,92.4796247)(289.28803345,92.42963186)
\curveto(289.42803947,92.36962481)(289.55303935,92.29962488)(289.66303345,92.21963186)
\curveto(289.77303913,92.14962503)(289.88303902,92.0746251)(289.99303345,91.99463186)
\lineto(290.05303345,91.93463186)
\curveto(290.07303883,91.92462525)(290.09303881,91.90962527)(290.11303345,91.88963186)
\curveto(290.27303863,91.76962541)(290.41803848,91.63462554)(290.54803345,91.48463186)
\curveto(290.67803822,91.33462584)(290.8030381,91.174626)(290.92303345,91.00463186)
\curveto(291.14303776,90.69462648)(291.34803755,90.39962678)(291.53803345,90.11963186)
\curveto(291.67803722,89.88962729)(291.81303709,89.65962752)(291.94303345,89.42963186)
\curveto(292.07303683,89.20962797)(292.20803669,88.98962819)(292.34803345,88.76963186)
\curveto(292.51803638,88.51962866)(292.6980362,88.2796289)(292.88803345,88.04963186)
\curveto(293.07803582,87.82962935)(293.3030356,87.63962954)(293.56303345,87.47963186)
\curveto(293.62303528,87.43962974)(293.68303522,87.40462977)(293.74303345,87.37463186)
\curveto(293.79303511,87.34462983)(293.85803504,87.31462986)(293.93803345,87.28463186)
\curveto(294.00803489,87.26462991)(294.06803483,87.25962992)(294.11803345,87.26963186)
\curveto(294.18803471,87.28962989)(294.24303466,87.32462985)(294.28303345,87.37463186)
\curveto(294.31303459,87.42462975)(294.33303457,87.48462969)(294.34303345,87.55463186)
\lineto(294.34303345,87.79463186)
\lineto(294.34303345,88.54463186)
\lineto(294.34303345,91.34963186)
\lineto(294.34303345,92.00963186)
\curveto(294.34303456,92.09962508)(294.34803455,92.18462499)(294.35803345,92.26463186)
\curveto(294.35803454,92.34462483)(294.37803452,92.40962477)(294.41803345,92.45963186)
\curveto(294.45803444,92.50962467)(294.53303437,92.54962463)(294.64303345,92.57963186)
\curveto(294.74303416,92.61962456)(294.84303406,92.62962455)(294.94303345,92.60963186)
\lineto(295.07803345,92.60963186)
\curveto(295.14803375,92.58962459)(295.20803369,92.56962461)(295.25803345,92.54963186)
\curveto(295.30803359,92.52962465)(295.34803355,92.49462468)(295.37803345,92.44463186)
\curveto(295.41803348,92.39462478)(295.43803346,92.32462485)(295.43803345,92.23463186)
\lineto(295.43803345,91.96463186)
\lineto(295.43803345,91.06463186)
\lineto(295.43803345,87.55463186)
\lineto(295.43803345,86.48963186)
\curveto(295.43803346,86.40963077)(295.44303346,86.31963086)(295.45303345,86.21963186)
\curveto(295.45303345,86.11963106)(295.44303346,86.03463114)(295.42303345,85.96463186)
\curveto(295.35303355,85.75463142)(295.17303373,85.68963149)(294.88303345,85.76963186)
\curveto(294.84303406,85.7796314)(294.80803409,85.7796314)(294.77803345,85.76963186)
\curveto(294.73803416,85.76963141)(294.69303421,85.7796314)(294.64303345,85.79963186)
\curveto(294.56303434,85.81963136)(294.47803442,85.83963134)(294.38803345,85.85963186)
\curveto(294.2980346,85.8796313)(294.21303469,85.90463127)(294.13303345,85.93463186)
\curveto(293.64303526,86.09463108)(293.22803567,86.29463088)(292.88803345,86.53463186)
\curveto(292.63803626,86.71463046)(292.41303649,86.91963026)(292.21303345,87.14963186)
\curveto(292.0030369,87.3796298)(291.80803709,87.61962956)(291.62803345,87.86963186)
\curveto(291.44803745,88.12962905)(291.27803762,88.39462878)(291.11803345,88.66463186)
\curveto(290.94803795,88.94462823)(290.77303813,89.21462796)(290.59303345,89.47463186)
\curveto(290.51303839,89.58462759)(290.43803846,89.68962749)(290.36803345,89.78963186)
\curveto(290.2980386,89.89962728)(290.22303868,90.00962717)(290.14303345,90.11963186)
\curveto(290.11303879,90.15962702)(290.08303882,90.19462698)(290.05303345,90.22463186)
\curveto(290.01303889,90.26462691)(289.98303892,90.30462687)(289.96303345,90.34463186)
\curveto(289.85303905,90.48462669)(289.72803917,90.60962657)(289.58803345,90.71963186)
\curveto(289.55803934,90.73962644)(289.53303937,90.76462641)(289.51303345,90.79463186)
\curveto(289.48303942,90.82462635)(289.45303945,90.84962633)(289.42303345,90.86963186)
\curveto(289.32303958,90.94962623)(289.22303968,91.01462616)(289.12303345,91.06463186)
\curveto(289.02303988,91.12462605)(288.91303999,91.179626)(288.79303345,91.22963186)
\curveto(288.72304018,91.25962592)(288.64804025,91.2796259)(288.56803345,91.28963186)
\lineto(288.32803345,91.34963186)
\lineto(288.23803345,91.34963186)
\curveto(288.20804069,91.35962582)(288.17804072,91.36462581)(288.14803345,91.36463186)
\curveto(288.07804082,91.38462579)(287.98304092,91.38962579)(287.86303345,91.37963186)
\curveto(287.73304117,91.3796258)(287.63304127,91.36962581)(287.56303345,91.34963186)
\curveto(287.48304142,91.32962585)(287.40804149,91.30962587)(287.33803345,91.28963186)
\curveto(287.25804164,91.2796259)(287.17804172,91.25962592)(287.09803345,91.22963186)
\curveto(286.85804204,91.11962606)(286.65804224,90.96962621)(286.49803345,90.77963186)
\curveto(286.32804257,90.59962658)(286.18804271,90.3796268)(286.07803345,90.11963186)
\curveto(286.05804284,90.04962713)(286.04304286,89.9796272)(286.03303345,89.90963186)
\curveto(286.01304289,89.83962734)(285.99304291,89.76462741)(285.97303345,89.68463186)
\curveto(285.95304295,89.60462757)(285.94304296,89.49462768)(285.94303345,89.35463186)
\curveto(285.94304296,89.22462795)(285.95304295,89.11962806)(285.97303345,89.03963186)
\curveto(285.98304292,88.9796282)(285.98804291,88.92462825)(285.98803345,88.87463186)
\curveto(285.98804291,88.82462835)(285.9980429,88.7746284)(286.01803345,88.72463186)
\curveto(286.05804284,88.62462855)(286.0980428,88.52962865)(286.13803345,88.43963186)
\curveto(286.17804272,88.35962882)(286.22304268,88.2796289)(286.27303345,88.19963186)
\curveto(286.29304261,88.16962901)(286.31804258,88.13962904)(286.34803345,88.10963186)
\curveto(286.37804252,88.08962909)(286.4030425,88.06462911)(286.42303345,88.03463186)
\lineto(286.49803345,87.95963186)
\curveto(286.51804238,87.92962925)(286.53804236,87.90462927)(286.55803345,87.88463186)
\lineto(286.76803345,87.73463186)
\curveto(286.82804207,87.69462948)(286.89304201,87.64962953)(286.96303345,87.59963186)
\curveto(287.05304185,87.53962964)(287.15804174,87.48962969)(287.27803345,87.44963186)
\curveto(287.38804151,87.41962976)(287.4980414,87.38462979)(287.60803345,87.34463186)
\curveto(287.71804118,87.30462987)(287.86304104,87.2796299)(288.04303345,87.26963186)
\curveto(288.21304069,87.25962992)(288.33804056,87.22962995)(288.41803345,87.17963186)
\curveto(288.4980404,87.12963005)(288.54304036,87.05463012)(288.55303345,86.95463186)
\curveto(288.56304034,86.85463032)(288.56804033,86.74463043)(288.56803345,86.62463186)
\curveto(288.56804033,86.58463059)(288.57304033,86.54463063)(288.58303345,86.50463186)
\curveto(288.58304032,86.46463071)(288.57804032,86.42963075)(288.56803345,86.39963186)
\curveto(288.54804035,86.34963083)(288.53804036,86.29963088)(288.53803345,86.24963186)
\curveto(288.53804036,86.20963097)(288.52804037,86.16963101)(288.50803345,86.12963186)
\curveto(288.44804045,86.03963114)(288.31304059,85.99463118)(288.10303345,85.99463186)
\lineto(287.98303345,85.99463186)
\curveto(287.92304098,86.00463117)(287.86304104,86.00963117)(287.80303345,86.00963186)
\curveto(287.73304117,86.01963116)(287.66804123,86.02963115)(287.60803345,86.03963186)
\curveto(287.4980414,86.05963112)(287.3980415,86.0796311)(287.30803345,86.09963186)
\curveto(287.20804169,86.11963106)(287.11304179,86.14963103)(287.02303345,86.18963186)
\curveto(286.95304195,86.20963097)(286.89304201,86.22963095)(286.84303345,86.24963186)
\lineto(286.66303345,86.30963186)
\curveto(286.4030425,86.42963075)(286.15804274,86.58463059)(285.92803345,86.77463186)
\curveto(285.6980432,86.9746302)(285.51304339,87.18962999)(285.37303345,87.41963186)
\curveto(285.29304361,87.52962965)(285.22804367,87.64462953)(285.17803345,87.76463186)
\lineto(285.02803345,88.15463186)
\curveto(284.97804392,88.26462891)(284.94804395,88.3796288)(284.93803345,88.49963186)
\curveto(284.91804398,88.61962856)(284.89304401,88.74462843)(284.86303345,88.87463186)
\curveto(284.86304404,88.94462823)(284.86304404,89.00962817)(284.86303345,89.06963186)
\curveto(284.85304405,89.12962805)(284.84304406,89.19462798)(284.83303345,89.26463186)
}
}
{
\newrgbcolor{curcolor}{0 0 0}
\pscustom[linestyle=none,fillstyle=solid,fillcolor=curcolor]
{
\newpath
\moveto(290.35303345,101.36424123)
\lineto(290.60803345,101.36424123)
\curveto(290.68803821,101.37423353)(290.76303814,101.36923353)(290.83303345,101.34924123)
\lineto(291.07303345,101.34924123)
\lineto(291.23803345,101.34924123)
\curveto(291.33803756,101.32923357)(291.44303746,101.31923358)(291.55303345,101.31924123)
\curveto(291.65303725,101.31923358)(291.75303715,101.30923359)(291.85303345,101.28924123)
\lineto(292.00303345,101.28924123)
\curveto(292.14303676,101.25923364)(292.28303662,101.23923366)(292.42303345,101.22924123)
\curveto(292.55303635,101.21923368)(292.68303622,101.19423371)(292.81303345,101.15424123)
\curveto(292.89303601,101.13423377)(292.97803592,101.11423379)(293.06803345,101.09424123)
\lineto(293.30803345,101.03424123)
\lineto(293.60803345,100.91424123)
\curveto(293.6980352,100.88423402)(293.78803511,100.84923405)(293.87803345,100.80924123)
\curveto(294.0980348,100.70923419)(294.31303459,100.57423433)(294.52303345,100.40424123)
\curveto(294.73303417,100.24423466)(294.903034,100.06923483)(295.03303345,99.87924123)
\curveto(295.07303383,99.82923507)(295.11303379,99.76923513)(295.15303345,99.69924123)
\curveto(295.18303372,99.63923526)(295.21803368,99.57923532)(295.25803345,99.51924123)
\curveto(295.30803359,99.43923546)(295.34803355,99.34423556)(295.37803345,99.23424123)
\curveto(295.40803349,99.12423578)(295.43803346,99.01923588)(295.46803345,98.91924123)
\curveto(295.50803339,98.80923609)(295.53303337,98.6992362)(295.54303345,98.58924123)
\curveto(295.55303335,98.47923642)(295.56803333,98.36423654)(295.58803345,98.24424123)
\curveto(295.5980333,98.2042367)(295.5980333,98.15923674)(295.58803345,98.10924123)
\curveto(295.58803331,98.06923683)(295.59303331,98.02923687)(295.60303345,97.98924123)
\curveto(295.61303329,97.94923695)(295.61803328,97.89423701)(295.61803345,97.82424123)
\curveto(295.61803328,97.75423715)(295.61303329,97.7042372)(295.60303345,97.67424123)
\curveto(295.58303332,97.62423728)(295.57803332,97.57923732)(295.58803345,97.53924123)
\curveto(295.5980333,97.4992374)(295.5980333,97.46423744)(295.58803345,97.43424123)
\lineto(295.58803345,97.34424123)
\curveto(295.56803333,97.28423762)(295.55303335,97.21923768)(295.54303345,97.14924123)
\curveto(295.54303336,97.08923781)(295.53803336,97.02423788)(295.52803345,96.95424123)
\curveto(295.47803342,96.78423812)(295.42803347,96.62423828)(295.37803345,96.47424123)
\curveto(295.32803357,96.32423858)(295.26303364,96.17923872)(295.18303345,96.03924123)
\curveto(295.14303376,95.98923891)(295.11303379,95.93423897)(295.09303345,95.87424123)
\curveto(295.06303384,95.82423908)(295.02803387,95.77423913)(294.98803345,95.72424123)
\curveto(294.80803409,95.48423942)(294.58803431,95.28423962)(294.32803345,95.12424123)
\curveto(294.06803483,94.96423994)(293.78303512,94.82424008)(293.47303345,94.70424123)
\curveto(293.33303557,94.64424026)(293.19303571,94.5992403)(293.05303345,94.56924123)
\curveto(292.903036,94.53924036)(292.74803615,94.5042404)(292.58803345,94.46424123)
\curveto(292.47803642,94.44424046)(292.36803653,94.42924047)(292.25803345,94.41924123)
\curveto(292.14803675,94.40924049)(292.03803686,94.39424051)(291.92803345,94.37424123)
\curveto(291.88803701,94.36424054)(291.84803705,94.35924054)(291.80803345,94.35924123)
\curveto(291.76803713,94.36924053)(291.72803717,94.36924053)(291.68803345,94.35924123)
\curveto(291.63803726,94.34924055)(291.58803731,94.34424056)(291.53803345,94.34424123)
\lineto(291.37303345,94.34424123)
\curveto(291.32303758,94.32424058)(291.27303763,94.31924058)(291.22303345,94.32924123)
\curveto(291.16303774,94.33924056)(291.10803779,94.33924056)(291.05803345,94.32924123)
\curveto(291.01803788,94.31924058)(290.97303793,94.31924058)(290.92303345,94.32924123)
\curveto(290.87303803,94.33924056)(290.82303808,94.33424057)(290.77303345,94.31424123)
\curveto(290.7030382,94.29424061)(290.62803827,94.28924061)(290.54803345,94.29924123)
\curveto(290.45803844,94.30924059)(290.37303853,94.31424059)(290.29303345,94.31424123)
\curveto(290.2030387,94.31424059)(290.1030388,94.30924059)(289.99303345,94.29924123)
\curveto(289.87303903,94.28924061)(289.77303913,94.29424061)(289.69303345,94.31424123)
\lineto(289.40803345,94.31424123)
\lineto(288.77803345,94.35924123)
\curveto(288.67804022,94.36924053)(288.58304032,94.37924052)(288.49303345,94.38924123)
\lineto(288.19303345,94.41924123)
\curveto(288.14304076,94.43924046)(288.09304081,94.44424046)(288.04303345,94.43424123)
\curveto(287.98304092,94.43424047)(287.92804097,94.44424046)(287.87803345,94.46424123)
\curveto(287.70804119,94.51424039)(287.54304136,94.55424035)(287.38303345,94.58424123)
\curveto(287.21304169,94.61424029)(287.05304185,94.66424024)(286.90303345,94.73424123)
\curveto(286.44304246,94.92423998)(286.06804283,95.14423976)(285.77803345,95.39424123)
\curveto(285.48804341,95.65423925)(285.24304366,96.01423889)(285.04303345,96.47424123)
\curveto(284.99304391,96.6042383)(284.95804394,96.73423817)(284.93803345,96.86424123)
\curveto(284.91804398,97.0042379)(284.89304401,97.14423776)(284.86303345,97.28424123)
\curveto(284.85304405,97.35423755)(284.84804405,97.41923748)(284.84803345,97.47924123)
\curveto(284.84804405,97.53923736)(284.84304406,97.6042373)(284.83303345,97.67424123)
\curveto(284.81304409,98.5042364)(284.96304394,99.17423573)(285.28303345,99.68424123)
\curveto(285.59304331,100.19423471)(286.03304287,100.57423433)(286.60303345,100.82424123)
\curveto(286.72304218,100.87423403)(286.84804205,100.91923398)(286.97803345,100.95924123)
\curveto(287.10804179,100.9992339)(287.24304166,101.04423386)(287.38303345,101.09424123)
\curveto(287.46304144,101.11423379)(287.54804135,101.12923377)(287.63803345,101.13924123)
\lineto(287.87803345,101.19924123)
\curveto(287.98804091,101.22923367)(288.0980408,101.24423366)(288.20803345,101.24424123)
\curveto(288.31804058,101.25423365)(288.42804047,101.26923363)(288.53803345,101.28924123)
\curveto(288.58804031,101.30923359)(288.63304027,101.31423359)(288.67303345,101.30424123)
\curveto(288.71304019,101.3042336)(288.75304015,101.30923359)(288.79303345,101.31924123)
\curveto(288.84304006,101.32923357)(288.89804,101.32923357)(288.95803345,101.31924123)
\curveto(289.00803989,101.31923358)(289.05803984,101.32423358)(289.10803345,101.33424123)
\lineto(289.24303345,101.33424123)
\curveto(289.3030396,101.35423355)(289.37303953,101.35423355)(289.45303345,101.33424123)
\curveto(289.52303938,101.32423358)(289.58803931,101.32923357)(289.64803345,101.34924123)
\curveto(289.67803922,101.35923354)(289.71803918,101.36423354)(289.76803345,101.36424123)
\lineto(289.88803345,101.36424123)
\lineto(290.35303345,101.36424123)
\moveto(292.67803345,99.81924123)
\curveto(292.35803654,99.91923498)(291.99303691,99.97923492)(291.58303345,99.99924123)
\curveto(291.17303773,100.01923488)(290.76303814,100.02923487)(290.35303345,100.02924123)
\curveto(289.92303898,100.02923487)(289.5030394,100.01923488)(289.09303345,99.99924123)
\curveto(288.68304022,99.97923492)(288.2980406,99.93423497)(287.93803345,99.86424123)
\curveto(287.57804132,99.79423511)(287.25804164,99.68423522)(286.97803345,99.53424123)
\curveto(286.68804221,99.39423551)(286.45304245,99.1992357)(286.27303345,98.94924123)
\curveto(286.16304274,98.78923611)(286.08304282,98.60923629)(286.03303345,98.40924123)
\curveto(285.97304293,98.20923669)(285.94304296,97.96423694)(285.94303345,97.67424123)
\curveto(285.96304294,97.65423725)(285.97304293,97.61923728)(285.97303345,97.56924123)
\curveto(285.96304294,97.51923738)(285.96304294,97.47923742)(285.97303345,97.44924123)
\curveto(285.99304291,97.36923753)(286.01304289,97.29423761)(286.03303345,97.22424123)
\curveto(286.04304286,97.16423774)(286.06304284,97.0992378)(286.09303345,97.02924123)
\curveto(286.21304269,96.75923814)(286.38304252,96.53923836)(286.60303345,96.36924123)
\curveto(286.81304209,96.20923869)(287.05804184,96.07423883)(287.33803345,95.96424123)
\curveto(287.44804145,95.91423899)(287.56804133,95.87423903)(287.69803345,95.84424123)
\curveto(287.81804108,95.82423908)(287.94304096,95.7992391)(288.07303345,95.76924123)
\curveto(288.12304078,95.74923915)(288.17804072,95.73923916)(288.23803345,95.73924123)
\curveto(288.28804061,95.73923916)(288.33804056,95.73423917)(288.38803345,95.72424123)
\curveto(288.47804042,95.71423919)(288.57304033,95.7042392)(288.67303345,95.69424123)
\curveto(288.76304014,95.68423922)(288.85804004,95.67423923)(288.95803345,95.66424123)
\curveto(289.03803986,95.66423924)(289.12303978,95.65923924)(289.21303345,95.64924123)
\lineto(289.45303345,95.64924123)
\lineto(289.63303345,95.64924123)
\curveto(289.66303924,95.63923926)(289.6980392,95.63423927)(289.73803345,95.63424123)
\lineto(289.87303345,95.63424123)
\lineto(290.32303345,95.63424123)
\curveto(290.4030385,95.63423927)(290.48803841,95.62923927)(290.57803345,95.61924123)
\curveto(290.65803824,95.61923928)(290.73303817,95.62923927)(290.80303345,95.64924123)
\lineto(291.07303345,95.64924123)
\curveto(291.09303781,95.64923925)(291.12303778,95.64423926)(291.16303345,95.63424123)
\curveto(291.19303771,95.63423927)(291.21803768,95.63923926)(291.23803345,95.64924123)
\curveto(291.33803756,95.65923924)(291.43803746,95.66423924)(291.53803345,95.66424123)
\curveto(291.62803727,95.67423923)(291.72803717,95.68423922)(291.83803345,95.69424123)
\curveto(291.95803694,95.72423918)(292.08303682,95.73923916)(292.21303345,95.73924123)
\curveto(292.33303657,95.74923915)(292.44803645,95.77423913)(292.55803345,95.81424123)
\curveto(292.85803604,95.89423901)(293.12303578,95.97923892)(293.35303345,96.06924123)
\curveto(293.58303532,96.16923873)(293.7980351,96.31423859)(293.99803345,96.50424123)
\curveto(294.1980347,96.71423819)(294.34803455,96.97923792)(294.44803345,97.29924123)
\curveto(294.46803443,97.33923756)(294.47803442,97.37423753)(294.47803345,97.40424123)
\curveto(294.46803443,97.44423746)(294.47303443,97.48923741)(294.49303345,97.53924123)
\curveto(294.5030344,97.57923732)(294.51303439,97.64923725)(294.52303345,97.74924123)
\curveto(294.53303437,97.85923704)(294.52803437,97.94423696)(294.50803345,98.00424123)
\curveto(294.48803441,98.07423683)(294.47803442,98.14423676)(294.47803345,98.21424123)
\curveto(294.46803443,98.28423662)(294.45303445,98.34923655)(294.43303345,98.40924123)
\curveto(294.37303453,98.60923629)(294.28803461,98.78923611)(294.17803345,98.94924123)
\curveto(294.15803474,98.97923592)(294.13803476,99.0042359)(294.11803345,99.02424123)
\lineto(294.05803345,99.08424123)
\curveto(294.03803486,99.12423578)(293.9980349,99.17423573)(293.93803345,99.23424123)
\curveto(293.7980351,99.33423557)(293.66803523,99.41923548)(293.54803345,99.48924123)
\curveto(293.42803547,99.55923534)(293.28303562,99.62923527)(293.11303345,99.69924123)
\curveto(293.04303586,99.72923517)(292.97303593,99.74923515)(292.90303345,99.75924123)
\curveto(292.83303607,99.77923512)(292.75803614,99.7992351)(292.67803345,99.81924123)
}
}
{
\newrgbcolor{curcolor}{0 0 0}
\pscustom[linestyle=none,fillstyle=solid,fillcolor=curcolor]
{
\newpath
\moveto(284.83303345,106.77385061)
\curveto(284.83304407,106.87384575)(284.84304406,106.96884566)(284.86303345,107.05885061)
\curveto(284.87304403,107.14884548)(284.903044,107.21384541)(284.95303345,107.25385061)
\curveto(285.03304387,107.31384531)(285.13804376,107.34384528)(285.26803345,107.34385061)
\lineto(285.65803345,107.34385061)
\lineto(287.15803345,107.34385061)
\lineto(293.54803345,107.34385061)
\lineto(294.71803345,107.34385061)
\lineto(295.03303345,107.34385061)
\curveto(295.13303377,107.35384527)(295.21303369,107.33884529)(295.27303345,107.29885061)
\curveto(295.35303355,107.24884538)(295.4030335,107.17384545)(295.42303345,107.07385061)
\curveto(295.43303347,106.98384564)(295.43803346,106.87384575)(295.43803345,106.74385061)
\lineto(295.43803345,106.51885061)
\curveto(295.41803348,106.43884619)(295.4030335,106.36884626)(295.39303345,106.30885061)
\curveto(295.37303353,106.24884638)(295.33303357,106.19884643)(295.27303345,106.15885061)
\curveto(295.21303369,106.11884651)(295.13803376,106.09884653)(295.04803345,106.09885061)
\lineto(294.74803345,106.09885061)
\lineto(293.65303345,106.09885061)
\lineto(288.31303345,106.09885061)
\curveto(288.22304068,106.07884655)(288.14804075,106.06384656)(288.08803345,106.05385061)
\curveto(288.01804088,106.05384657)(287.95804094,106.0238466)(287.90803345,105.96385061)
\curveto(287.85804104,105.89384673)(287.83304107,105.80384682)(287.83303345,105.69385061)
\curveto(287.82304108,105.59384703)(287.81804108,105.48384714)(287.81803345,105.36385061)
\lineto(287.81803345,104.22385061)
\lineto(287.81803345,103.72885061)
\curveto(287.80804109,103.56884906)(287.74804115,103.45884917)(287.63803345,103.39885061)
\curveto(287.60804129,103.37884925)(287.57804132,103.36884926)(287.54803345,103.36885061)
\curveto(287.50804139,103.36884926)(287.46304144,103.36384926)(287.41303345,103.35385061)
\curveto(287.29304161,103.33384929)(287.18304172,103.33884929)(287.08303345,103.36885061)
\curveto(286.98304192,103.40884922)(286.91304199,103.46384916)(286.87303345,103.53385061)
\curveto(286.82304208,103.61384901)(286.7980421,103.73384889)(286.79803345,103.89385061)
\curveto(286.7980421,104.05384857)(286.78304212,104.18884844)(286.75303345,104.29885061)
\curveto(286.74304216,104.34884828)(286.73804216,104.40384822)(286.73803345,104.46385061)
\curveto(286.72804217,104.5238481)(286.71304219,104.58384804)(286.69303345,104.64385061)
\curveto(286.64304226,104.79384783)(286.59304231,104.93884769)(286.54303345,105.07885061)
\curveto(286.48304242,105.21884741)(286.41304249,105.35384727)(286.33303345,105.48385061)
\curveto(286.24304266,105.623847)(286.13804276,105.74384688)(286.01803345,105.84385061)
\curveto(285.898043,105.94384668)(285.76804313,106.03884659)(285.62803345,106.12885061)
\curveto(285.52804337,106.18884644)(285.41804348,106.23384639)(285.29803345,106.26385061)
\curveto(285.17804372,106.30384632)(285.07304383,106.35384627)(284.98303345,106.41385061)
\curveto(284.92304398,106.46384616)(284.88304402,106.53384609)(284.86303345,106.62385061)
\curveto(284.85304405,106.64384598)(284.84804405,106.66884596)(284.84803345,106.69885061)
\curveto(284.84804405,106.7288459)(284.84304406,106.75384587)(284.83303345,106.77385061)
}
}
{
\newrgbcolor{curcolor}{0 0 0}
\pscustom[linestyle=none,fillstyle=solid,fillcolor=curcolor]
{
\newpath
\moveto(284.83303345,115.12345998)
\curveto(284.83304407,115.22345513)(284.84304406,115.31845503)(284.86303345,115.40845998)
\curveto(284.87304403,115.49845485)(284.903044,115.56345479)(284.95303345,115.60345998)
\curveto(285.03304387,115.66345469)(285.13804376,115.69345466)(285.26803345,115.69345998)
\lineto(285.65803345,115.69345998)
\lineto(287.15803345,115.69345998)
\lineto(293.54803345,115.69345998)
\lineto(294.71803345,115.69345998)
\lineto(295.03303345,115.69345998)
\curveto(295.13303377,115.70345465)(295.21303369,115.68845466)(295.27303345,115.64845998)
\curveto(295.35303355,115.59845475)(295.4030335,115.52345483)(295.42303345,115.42345998)
\curveto(295.43303347,115.33345502)(295.43803346,115.22345513)(295.43803345,115.09345998)
\lineto(295.43803345,114.86845998)
\curveto(295.41803348,114.78845556)(295.4030335,114.71845563)(295.39303345,114.65845998)
\curveto(295.37303353,114.59845575)(295.33303357,114.5484558)(295.27303345,114.50845998)
\curveto(295.21303369,114.46845588)(295.13803376,114.4484559)(295.04803345,114.44845998)
\lineto(294.74803345,114.44845998)
\lineto(293.65303345,114.44845998)
\lineto(288.31303345,114.44845998)
\curveto(288.22304068,114.42845592)(288.14804075,114.41345594)(288.08803345,114.40345998)
\curveto(288.01804088,114.40345595)(287.95804094,114.37345598)(287.90803345,114.31345998)
\curveto(287.85804104,114.24345611)(287.83304107,114.1534562)(287.83303345,114.04345998)
\curveto(287.82304108,113.94345641)(287.81804108,113.83345652)(287.81803345,113.71345998)
\lineto(287.81803345,112.57345998)
\lineto(287.81803345,112.07845998)
\curveto(287.80804109,111.91845843)(287.74804115,111.80845854)(287.63803345,111.74845998)
\curveto(287.60804129,111.72845862)(287.57804132,111.71845863)(287.54803345,111.71845998)
\curveto(287.50804139,111.71845863)(287.46304144,111.71345864)(287.41303345,111.70345998)
\curveto(287.29304161,111.68345867)(287.18304172,111.68845866)(287.08303345,111.71845998)
\curveto(286.98304192,111.75845859)(286.91304199,111.81345854)(286.87303345,111.88345998)
\curveto(286.82304208,111.96345839)(286.7980421,112.08345827)(286.79803345,112.24345998)
\curveto(286.7980421,112.40345795)(286.78304212,112.53845781)(286.75303345,112.64845998)
\curveto(286.74304216,112.69845765)(286.73804216,112.7534576)(286.73803345,112.81345998)
\curveto(286.72804217,112.87345748)(286.71304219,112.93345742)(286.69303345,112.99345998)
\curveto(286.64304226,113.14345721)(286.59304231,113.28845706)(286.54303345,113.42845998)
\curveto(286.48304242,113.56845678)(286.41304249,113.70345665)(286.33303345,113.83345998)
\curveto(286.24304266,113.97345638)(286.13804276,114.09345626)(286.01803345,114.19345998)
\curveto(285.898043,114.29345606)(285.76804313,114.38845596)(285.62803345,114.47845998)
\curveto(285.52804337,114.53845581)(285.41804348,114.58345577)(285.29803345,114.61345998)
\curveto(285.17804372,114.6534557)(285.07304383,114.70345565)(284.98303345,114.76345998)
\curveto(284.92304398,114.81345554)(284.88304402,114.88345547)(284.86303345,114.97345998)
\curveto(284.85304405,114.99345536)(284.84804405,115.01845533)(284.84803345,115.04845998)
\curveto(284.84804405,115.07845527)(284.84304406,115.10345525)(284.83303345,115.12345998)
}
}
{
\newrgbcolor{curcolor}{0 0 0}
\pscustom[linestyle=none,fillstyle=solid,fillcolor=curcolor]
{
\newpath
\moveto(305.66937988,31.67142873)
\lineto(305.66937988,32.58642873)
\curveto(305.66939058,32.68642608)(305.66939058,32.78142599)(305.66937988,32.87142873)
\curveto(305.66939058,32.96142581)(305.68939056,33.03642573)(305.72937988,33.09642873)
\curveto(305.78939046,33.18642558)(305.86939038,33.24642552)(305.96937988,33.27642873)
\curveto(306.06939018,33.31642545)(306.17439007,33.36142541)(306.28437988,33.41142873)
\curveto(306.47438977,33.49142528)(306.66438958,33.56142521)(306.85437988,33.62142873)
\curveto(307.0443892,33.69142508)(307.23438901,33.766425)(307.42437988,33.84642873)
\curveto(307.60438864,33.91642485)(307.78938846,33.98142479)(307.97937988,34.04142873)
\curveto(308.15938809,34.10142467)(308.33938791,34.1714246)(308.51937988,34.25142873)
\curveto(308.65938759,34.31142446)(308.80438744,34.3664244)(308.95437988,34.41642873)
\curveto(309.10438714,34.4664243)(309.249387,34.52142425)(309.38937988,34.58142873)
\curveto(309.83938641,34.76142401)(310.29438595,34.93142384)(310.75437988,35.09142873)
\curveto(311.20438504,35.25142352)(311.65438459,35.42142335)(312.10437988,35.60142873)
\curveto(312.15438409,35.62142315)(312.20438404,35.63642313)(312.25437988,35.64642873)
\lineto(312.40437988,35.70642873)
\curveto(312.62438362,35.79642297)(312.8493834,35.88142289)(313.07937988,35.96142873)
\curveto(313.29938295,36.04142273)(313.51938273,36.12642264)(313.73937988,36.21642873)
\curveto(313.82938242,36.25642251)(313.93938231,36.29642247)(314.06937988,36.33642873)
\curveto(314.18938206,36.37642239)(314.25938199,36.44142233)(314.27937988,36.53142873)
\curveto(314.28938196,36.5714222)(314.28938196,36.60142217)(314.27937988,36.62142873)
\lineto(314.21937988,36.68142873)
\curveto(314.16938208,36.73142204)(314.11438213,36.766422)(314.05437988,36.78642873)
\curveto(313.99438225,36.81642195)(313.92938232,36.84642192)(313.85937988,36.87642873)
\lineto(313.22937988,37.11642873)
\curveto(313.00938324,37.19642157)(312.79438345,37.27642149)(312.58437988,37.35642873)
\lineto(312.43437988,37.41642873)
\lineto(312.25437988,37.47642873)
\curveto(312.06438418,37.55642121)(311.87438437,37.62642114)(311.68437988,37.68642873)
\curveto(311.48438476,37.75642101)(311.28438496,37.83142094)(311.08437988,37.91142873)
\curveto(310.50438574,38.15142062)(309.91938633,38.3714204)(309.32937988,38.57142873)
\curveto(308.73938751,38.78141999)(308.15438809,39.00641976)(307.57437988,39.24642873)
\curveto(307.37438887,39.32641944)(307.16938908,39.40141937)(306.95937988,39.47142873)
\curveto(306.7493895,39.55141922)(306.5443897,39.63141914)(306.34437988,39.71142873)
\curveto(306.26438998,39.75141902)(306.16439008,39.78641898)(306.04437988,39.81642873)
\curveto(305.92439032,39.85641891)(305.83939041,39.91141886)(305.78937988,39.98142873)
\curveto(305.7493905,40.04141873)(305.71939053,40.11641865)(305.69937988,40.20642873)
\curveto(305.67939057,40.30641846)(305.66939058,40.41641835)(305.66937988,40.53642873)
\curveto(305.65939059,40.65641811)(305.65939059,40.77641799)(305.66937988,40.89642873)
\curveto(305.66939058,41.01641775)(305.66939058,41.12641764)(305.66937988,41.22642873)
\curveto(305.66939058,41.31641745)(305.66939058,41.40641736)(305.66937988,41.49642873)
\curveto(305.66939058,41.59641717)(305.68939056,41.6714171)(305.72937988,41.72142873)
\curveto(305.77939047,41.81141696)(305.86939038,41.86141691)(305.99937988,41.87142873)
\curveto(306.12939012,41.88141689)(306.26938998,41.88641688)(306.41937988,41.88642873)
\lineto(308.06937988,41.88642873)
\lineto(314.33937988,41.88642873)
\lineto(315.59937988,41.88642873)
\curveto(315.70938054,41.88641688)(315.81938043,41.88641688)(315.92937988,41.88642873)
\curveto(316.03938021,41.89641687)(316.12438012,41.87641689)(316.18437988,41.82642873)
\curveto(316.24438,41.79641697)(316.28437996,41.75141702)(316.30437988,41.69142873)
\curveto(316.31437993,41.63141714)(316.32937992,41.56141721)(316.34937988,41.48142873)
\lineto(316.34937988,41.24142873)
\lineto(316.34937988,40.88142873)
\curveto(316.33937991,40.771418)(316.29437995,40.69141808)(316.21437988,40.64142873)
\curveto(316.18438006,40.62141815)(316.15438009,40.60641816)(316.12437988,40.59642873)
\curveto(316.08438016,40.59641817)(316.03938021,40.58641818)(315.98937988,40.56642873)
\lineto(315.82437988,40.56642873)
\curveto(315.76438048,40.55641821)(315.69438055,40.55141822)(315.61437988,40.55142873)
\curveto(315.53438071,40.56141821)(315.45938079,40.5664182)(315.38937988,40.56642873)
\lineto(314.54937988,40.56642873)
\lineto(310.12437988,40.56642873)
\curveto(309.87438637,40.5664182)(309.62438662,40.5664182)(309.37437988,40.56642873)
\curveto(309.11438713,40.5664182)(308.86438738,40.56141821)(308.62437988,40.55142873)
\curveto(308.52438772,40.55141822)(308.41438783,40.54641822)(308.29437988,40.53642873)
\curveto(308.17438807,40.52641824)(308.11438813,40.4714183)(308.11437988,40.37142873)
\lineto(308.12937988,40.37142873)
\curveto(308.1493881,40.30141847)(308.21438803,40.24141853)(308.32437988,40.19142873)
\curveto(308.43438781,40.15141862)(308.52938772,40.11641865)(308.60937988,40.08642873)
\curveto(308.77938747,40.01641875)(308.95438729,39.95141882)(309.13437988,39.89142873)
\curveto(309.30438694,39.83141894)(309.47438677,39.76141901)(309.64437988,39.68142873)
\curveto(309.69438655,39.66141911)(309.73938651,39.64641912)(309.77937988,39.63642873)
\curveto(309.81938643,39.62641914)(309.86438638,39.61141916)(309.91437988,39.59142873)
\curveto(310.09438615,39.51141926)(310.27938597,39.44141933)(310.46937988,39.38142873)
\curveto(310.6493856,39.33141944)(310.82938542,39.2664195)(311.00937988,39.18642873)
\curveto(311.15938509,39.11641965)(311.31438493,39.05641971)(311.47437988,39.00642873)
\curveto(311.62438462,38.95641981)(311.77438447,38.90141987)(311.92437988,38.84142873)
\curveto(312.39438385,38.64142013)(312.86938338,38.46142031)(313.34937988,38.30142873)
\curveto(313.81938243,38.14142063)(314.28438196,37.9664208)(314.74437988,37.77642873)
\curveto(314.92438132,37.69642107)(315.10438114,37.62642114)(315.28437988,37.56642873)
\curveto(315.46438078,37.50642126)(315.6443806,37.44142133)(315.82437988,37.37142873)
\curveto(315.93438031,37.32142145)(316.03938021,37.2714215)(316.13937988,37.22142873)
\curveto(316.22938002,37.18142159)(316.29437995,37.09642167)(316.33437988,36.96642873)
\curveto(316.3443799,36.94642182)(316.3493799,36.92142185)(316.34937988,36.89142873)
\curveto(316.33937991,36.8714219)(316.33937991,36.84642192)(316.34937988,36.81642873)
\curveto(316.35937989,36.78642198)(316.36437988,36.75142202)(316.36437988,36.71142873)
\curveto(316.35437989,36.6714221)(316.3493799,36.63142214)(316.34937988,36.59142873)
\lineto(316.34937988,36.29142873)
\curveto(316.3493799,36.19142258)(316.32437992,36.11142266)(316.27437988,36.05142873)
\curveto(316.22438002,35.9714228)(316.15438009,35.91142286)(316.06437988,35.87142873)
\curveto(315.96438028,35.84142293)(315.86438038,35.80142297)(315.76437988,35.75142873)
\curveto(315.56438068,35.6714231)(315.35938089,35.59142318)(315.14937988,35.51142873)
\curveto(314.92938132,35.44142333)(314.71938153,35.3664234)(314.51937988,35.28642873)
\curveto(314.33938191,35.20642356)(314.15938209,35.13642363)(313.97937988,35.07642873)
\curveto(313.78938246,35.02642374)(313.60438264,34.96142381)(313.42437988,34.88142873)
\curveto(312.86438338,34.65142412)(312.29938395,34.43642433)(311.72937988,34.23642873)
\curveto(311.15938509,34.03642473)(310.59438565,33.82142495)(310.03437988,33.59142873)
\lineto(309.40437988,33.35142873)
\curveto(309.18438706,33.28142549)(308.97438727,33.20642556)(308.77437988,33.12642873)
\curveto(308.66438758,33.07642569)(308.55938769,33.03142574)(308.45937988,32.99142873)
\curveto(308.3493879,32.96142581)(308.25438799,32.91142586)(308.17437988,32.84142873)
\curveto(308.15438809,32.83142594)(308.1443881,32.82142595)(308.14437988,32.81142873)
\lineto(308.11437988,32.78142873)
\lineto(308.11437988,32.70642873)
\lineto(308.14437988,32.67642873)
\curveto(308.1443881,32.6664261)(308.1493881,32.65642611)(308.15937988,32.64642873)
\curveto(308.20938804,32.62642614)(308.26438798,32.61642615)(308.32437988,32.61642873)
\curveto(308.38438786,32.61642615)(308.4443878,32.60642616)(308.50437988,32.58642873)
\lineto(308.66937988,32.58642873)
\curveto(308.72938752,32.5664262)(308.79438745,32.56142621)(308.86437988,32.57142873)
\curveto(308.93438731,32.58142619)(309.00438724,32.58642618)(309.07437988,32.58642873)
\lineto(309.88437988,32.58642873)
\lineto(314.44437988,32.58642873)
\lineto(315.62937988,32.58642873)
\curveto(315.73938051,32.58642618)(315.8493804,32.58142619)(315.95937988,32.57142873)
\curveto(316.06938018,32.5714262)(316.15438009,32.54642622)(316.21437988,32.49642873)
\curveto(316.29437995,32.44642632)(316.33937991,32.35642641)(316.34937988,32.22642873)
\lineto(316.34937988,31.83642873)
\lineto(316.34937988,31.64142873)
\curveto(316.3493799,31.59142718)(316.33937991,31.54142723)(316.31937988,31.49142873)
\curveto(316.27937997,31.36142741)(316.19438005,31.28642748)(316.06437988,31.26642873)
\curveto(315.93438031,31.25642751)(315.78438046,31.25142752)(315.61437988,31.25142873)
\lineto(313.87437988,31.25142873)
\lineto(307.87437988,31.25142873)
\lineto(306.46437988,31.25142873)
\curveto(306.35438989,31.25142752)(306.23939001,31.24642752)(306.11937988,31.23642873)
\curveto(305.99939025,31.23642753)(305.90439034,31.26142751)(305.83437988,31.31142873)
\curveto(305.77439047,31.35142742)(305.72439052,31.42642734)(305.68437988,31.53642873)
\curveto(305.67439057,31.55642721)(305.67439057,31.57642719)(305.68437988,31.59642873)
\curveto(305.68439056,31.62642714)(305.67939057,31.65142712)(305.66937988,31.67142873)
}
}
{
\newrgbcolor{curcolor}{0 0 0}
\pscustom[linestyle=none,fillstyle=solid,fillcolor=curcolor]
{
\newpath
\moveto(315.79437988,50.87353811)
\curveto(315.95438029,50.90353028)(316.08938016,50.88853029)(316.19937988,50.82853811)
\curveto(316.29937995,50.76853041)(316.37437987,50.68853049)(316.42437988,50.58853811)
\curveto(316.4443798,50.53853064)(316.45437979,50.4835307)(316.45437988,50.42353811)
\curveto(316.45437979,50.37353081)(316.46437978,50.31853086)(316.48437988,50.25853811)
\curveto(316.53437971,50.03853114)(316.51937973,49.81853136)(316.43937988,49.59853811)
\curveto(316.36937988,49.38853179)(316.27937997,49.24353194)(316.16937988,49.16353811)
\curveto(316.09938015,49.11353207)(316.01938023,49.06853211)(315.92937988,49.02853811)
\curveto(315.82938042,48.98853219)(315.7493805,48.93853224)(315.68937988,48.87853811)
\curveto(315.66938058,48.85853232)(315.6493806,48.83353235)(315.62937988,48.80353811)
\curveto(315.60938064,48.7835324)(315.60438064,48.75353243)(315.61437988,48.71353811)
\curveto(315.6443806,48.60353258)(315.69938055,48.49853268)(315.77937988,48.39853811)
\curveto(315.85938039,48.30853287)(315.92938032,48.21853296)(315.98937988,48.12853811)
\curveto(316.06938018,47.99853318)(316.1443801,47.85853332)(316.21437988,47.70853811)
\curveto(316.27437997,47.55853362)(316.32937992,47.39853378)(316.37937988,47.22853811)
\curveto(316.40937984,47.12853405)(316.42937982,47.01853416)(316.43937988,46.89853811)
\curveto(316.4493798,46.78853439)(316.46437978,46.6785345)(316.48437988,46.56853811)
\curveto(316.49437975,46.51853466)(316.49937975,46.47353471)(316.49937988,46.43353811)
\lineto(316.49937988,46.32853811)
\curveto(316.51937973,46.21853496)(316.51937973,46.11353507)(316.49937988,46.01353811)
\lineto(316.49937988,45.87853811)
\curveto(316.48937976,45.82853535)(316.48437976,45.7785354)(316.48437988,45.72853811)
\curveto(316.48437976,45.6785355)(316.47437977,45.63353555)(316.45437988,45.59353811)
\curveto(316.4443798,45.55353563)(316.43937981,45.51853566)(316.43937988,45.48853811)
\curveto(316.4493798,45.46853571)(316.4493798,45.44353574)(316.43937988,45.41353811)
\lineto(316.37937988,45.17353811)
\curveto(316.36937988,45.09353609)(316.3493799,45.01853616)(316.31937988,44.94853811)
\curveto(316.18938006,44.64853653)(316.0443802,44.40353678)(315.88437988,44.21353811)
\curveto(315.71438053,44.03353715)(315.47938077,43.8835373)(315.17937988,43.76353811)
\curveto(314.95938129,43.67353751)(314.69438155,43.62853755)(314.38437988,43.62853811)
\lineto(314.06937988,43.62853811)
\curveto(314.01938223,43.63853754)(313.96938228,43.64353754)(313.91937988,43.64353811)
\lineto(313.73937988,43.67353811)
\lineto(313.40937988,43.79353811)
\curveto(313.29938295,43.83353735)(313.19938305,43.8835373)(313.10937988,43.94353811)
\curveto(312.81938343,44.12353706)(312.60438364,44.36853681)(312.46437988,44.67853811)
\curveto(312.32438392,44.98853619)(312.19938405,45.32853585)(312.08937988,45.69853811)
\curveto(312.0493842,45.83853534)(312.01938423,45.9835352)(311.99937988,46.13353811)
\curveto(311.97938427,46.2835349)(311.95438429,46.43353475)(311.92437988,46.58353811)
\curveto(311.90438434,46.65353453)(311.89438435,46.71853446)(311.89437988,46.77853811)
\curveto(311.89438435,46.84853433)(311.88438436,46.92353426)(311.86437988,47.00353811)
\curveto(311.8443844,47.07353411)(311.83438441,47.14353404)(311.83437988,47.21353811)
\curveto(311.82438442,47.2835339)(311.80938444,47.35853382)(311.78937988,47.43853811)
\curveto(311.72938452,47.68853349)(311.67938457,47.92353326)(311.63937988,48.14353811)
\curveto(311.58938466,48.36353282)(311.47438477,48.53853264)(311.29437988,48.66853811)
\curveto(311.21438503,48.72853245)(311.11438513,48.7785324)(310.99437988,48.81853811)
\curveto(310.86438538,48.85853232)(310.72438552,48.85853232)(310.57437988,48.81853811)
\curveto(310.33438591,48.75853242)(310.1443861,48.66853251)(310.00437988,48.54853811)
\curveto(309.86438638,48.43853274)(309.75438649,48.2785329)(309.67437988,48.06853811)
\curveto(309.62438662,47.94853323)(309.58938666,47.80353338)(309.56937988,47.63353811)
\curveto(309.5493867,47.47353371)(309.53938671,47.30353388)(309.53937988,47.12353811)
\curveto(309.53938671,46.94353424)(309.5493867,46.76853441)(309.56937988,46.59853811)
\curveto(309.58938666,46.42853475)(309.61938663,46.2835349)(309.65937988,46.16353811)
\curveto(309.71938653,45.99353519)(309.80438644,45.82853535)(309.91437988,45.66853811)
\curveto(309.97438627,45.58853559)(310.05438619,45.51353567)(310.15437988,45.44353811)
\curveto(310.244386,45.3835358)(310.3443859,45.32853585)(310.45437988,45.27853811)
\curveto(310.53438571,45.24853593)(310.61938563,45.21853596)(310.70937988,45.18853811)
\curveto(310.79938545,45.16853601)(310.86938538,45.12353606)(310.91937988,45.05353811)
\curveto(310.9493853,45.01353617)(310.97438527,44.94353624)(310.99437988,44.84353811)
\curveto(311.00438524,44.75353643)(311.00938524,44.65853652)(311.00937988,44.55853811)
\curveto(311.00938524,44.45853672)(311.00438524,44.35853682)(310.99437988,44.25853811)
\curveto(310.97438527,44.16853701)(310.9493853,44.10353708)(310.91937988,44.06353811)
\curveto(310.88938536,44.02353716)(310.83938541,43.99353719)(310.76937988,43.97353811)
\curveto(310.69938555,43.95353723)(310.62438562,43.95353723)(310.54437988,43.97353811)
\curveto(310.41438583,44.00353718)(310.29438595,44.03353715)(310.18437988,44.06353811)
\curveto(310.06438618,44.10353708)(309.9493863,44.14853703)(309.83937988,44.19853811)
\curveto(309.48938676,44.38853679)(309.21938703,44.62853655)(309.02937988,44.91853811)
\curveto(308.82938742,45.20853597)(308.66938758,45.56853561)(308.54937988,45.99853811)
\curveto(308.52938772,46.09853508)(308.51438773,46.19853498)(308.50437988,46.29853811)
\curveto(308.49438775,46.40853477)(308.47938777,46.51853466)(308.45937988,46.62853811)
\curveto(308.4493878,46.66853451)(308.4493878,46.73353445)(308.45937988,46.82353811)
\curveto(308.45938779,46.91353427)(308.4493878,46.96853421)(308.42937988,46.98853811)
\curveto(308.41938783,47.68853349)(308.49938775,48.29853288)(308.66937988,48.81853811)
\curveto(308.83938741,49.33853184)(309.16438708,49.70353148)(309.64437988,49.91353811)
\curveto(309.8443864,50.00353118)(310.07938617,50.05353113)(310.34937988,50.06353811)
\curveto(310.60938564,50.0835311)(310.88438536,50.09353109)(311.17437988,50.09353811)
\lineto(314.48937988,50.09353811)
\curveto(314.62938162,50.09353109)(314.76438148,50.09853108)(314.89437988,50.10853811)
\curveto(315.02438122,50.11853106)(315.12938112,50.14853103)(315.20937988,50.19853811)
\curveto(315.27938097,50.24853093)(315.32938092,50.31353087)(315.35937988,50.39353811)
\curveto(315.39938085,50.4835307)(315.42938082,50.56853061)(315.44937988,50.64853811)
\curveto(315.45938079,50.72853045)(315.50438074,50.78853039)(315.58437988,50.82853811)
\curveto(315.61438063,50.84853033)(315.6443806,50.85853032)(315.67437988,50.85853811)
\curveto(315.70438054,50.85853032)(315.7443805,50.86353032)(315.79437988,50.87353811)
\moveto(314.12937988,48.72853811)
\curveto(313.98938226,48.78853239)(313.82938242,48.81853236)(313.64937988,48.81853811)
\curveto(313.45938279,48.82853235)(313.26438298,48.83353235)(313.06437988,48.83353811)
\curveto(312.95438329,48.83353235)(312.85438339,48.82853235)(312.76437988,48.81853811)
\curveto(312.67438357,48.80853237)(312.60438364,48.76853241)(312.55437988,48.69853811)
\curveto(312.53438371,48.66853251)(312.52438372,48.59853258)(312.52437988,48.48853811)
\curveto(312.5443837,48.46853271)(312.55438369,48.43353275)(312.55437988,48.38353811)
\curveto(312.55438369,48.33353285)(312.56438368,48.28853289)(312.58437988,48.24853811)
\curveto(312.60438364,48.16853301)(312.62438362,48.0785331)(312.64437988,47.97853811)
\lineto(312.70437988,47.67853811)
\curveto(312.70438354,47.64853353)(312.70938354,47.61353357)(312.71937988,47.57353811)
\lineto(312.71937988,47.46853811)
\curveto(312.75938349,47.31853386)(312.78438346,47.15353403)(312.79437988,46.97353811)
\curveto(312.79438345,46.80353438)(312.81438343,46.64353454)(312.85437988,46.49353811)
\curveto(312.87438337,46.41353477)(312.89438335,46.33853484)(312.91437988,46.26853811)
\curveto(312.92438332,46.20853497)(312.93938331,46.13853504)(312.95937988,46.05853811)
\curveto(313.00938324,45.89853528)(313.07438317,45.74853543)(313.15437988,45.60853811)
\curveto(313.22438302,45.46853571)(313.31438293,45.34853583)(313.42437988,45.24853811)
\curveto(313.53438271,45.14853603)(313.66938258,45.07353611)(313.82937988,45.02353811)
\curveto(313.97938227,44.97353621)(314.16438208,44.95353623)(314.38437988,44.96353811)
\curveto(314.48438176,44.96353622)(314.57938167,44.9785362)(314.66937988,45.00853811)
\curveto(314.7493815,45.04853613)(314.82438142,45.09353609)(314.89437988,45.14353811)
\curveto(315.00438124,45.22353596)(315.09938115,45.32853585)(315.17937988,45.45853811)
\curveto(315.249381,45.58853559)(315.30938094,45.72853545)(315.35937988,45.87853811)
\curveto(315.36938088,45.92853525)(315.37438087,45.9785352)(315.37437988,46.02853811)
\curveto(315.37438087,46.0785351)(315.37938087,46.12853505)(315.38937988,46.17853811)
\curveto(315.40938084,46.24853493)(315.42438082,46.33353485)(315.43437988,46.43353811)
\curveto(315.43438081,46.54353464)(315.42438082,46.63353455)(315.40437988,46.70353811)
\curveto(315.38438086,46.76353442)(315.37938087,46.82353436)(315.38937988,46.88353811)
\curveto(315.38938086,46.94353424)(315.37938087,47.00353418)(315.35937988,47.06353811)
\curveto(315.33938091,47.14353404)(315.32438092,47.21853396)(315.31437988,47.28853811)
\curveto(315.30438094,47.36853381)(315.28438096,47.44353374)(315.25437988,47.51353811)
\curveto(315.13438111,47.80353338)(314.98938126,48.04853313)(314.81937988,48.24853811)
\curveto(314.6493816,48.45853272)(314.41938183,48.61853256)(314.12937988,48.72853811)
}
}
{
\newrgbcolor{curcolor}{0 0 0}
\pscustom[linestyle=none,fillstyle=solid,fillcolor=curcolor]
{
\newpath
\moveto(308.44437988,55.69017873)
\curveto(308.4443878,55.92017394)(308.50438774,56.05017381)(308.62437988,56.08017873)
\curveto(308.73438751,56.11017375)(308.89938735,56.12517374)(309.11937988,56.12517873)
\lineto(309.40437988,56.12517873)
\curveto(309.49438675,56.12517374)(309.56938668,56.10017376)(309.62937988,56.05017873)
\curveto(309.70938654,55.99017387)(309.75438649,55.90517396)(309.76437988,55.79517873)
\curveto(309.76438648,55.68517418)(309.77938647,55.57517429)(309.80937988,55.46517873)
\curveto(309.83938641,55.32517454)(309.86938638,55.19017467)(309.89937988,55.06017873)
\curveto(309.92938632,54.94017492)(309.96938628,54.82517504)(310.01937988,54.71517873)
\curveto(310.1493861,54.42517544)(310.32938592,54.19017567)(310.55937988,54.01017873)
\curveto(310.77938547,53.83017603)(311.03438521,53.67517619)(311.32437988,53.54517873)
\curveto(311.43438481,53.50517636)(311.5493847,53.47517639)(311.66937988,53.45517873)
\curveto(311.77938447,53.43517643)(311.89438435,53.41017645)(312.01437988,53.38017873)
\curveto(312.06438418,53.37017649)(312.11438413,53.3651765)(312.16437988,53.36517873)
\curveto(312.21438403,53.37517649)(312.26438398,53.37517649)(312.31437988,53.36517873)
\curveto(312.43438381,53.33517653)(312.57438367,53.32017654)(312.73437988,53.32017873)
\curveto(312.88438336,53.33017653)(313.02938322,53.33517653)(313.16937988,53.33517873)
\lineto(315.01437988,53.33517873)
\lineto(315.35937988,53.33517873)
\curveto(315.47938077,53.33517653)(315.59438065,53.33017653)(315.70437988,53.32017873)
\curveto(315.81438043,53.31017655)(315.90938034,53.30517656)(315.98937988,53.30517873)
\curveto(316.06938018,53.31517655)(316.13938011,53.29517657)(316.19937988,53.24517873)
\curveto(316.26937998,53.19517667)(316.30937994,53.11517675)(316.31937988,53.00517873)
\curveto(316.32937992,52.90517696)(316.33437991,52.79517707)(316.33437988,52.67517873)
\lineto(316.33437988,52.40517873)
\curveto(316.31437993,52.35517751)(316.29937995,52.30517756)(316.28937988,52.25517873)
\curveto(316.26937998,52.21517765)(316.24438,52.18517768)(316.21437988,52.16517873)
\curveto(316.1443801,52.11517775)(316.05938019,52.08517778)(315.95937988,52.07517873)
\lineto(315.62937988,52.07517873)
\lineto(314.47437988,52.07517873)
\lineto(310.31937988,52.07517873)
\lineto(309.28437988,52.07517873)
\lineto(308.98437988,52.07517873)
\curveto(308.88438736,52.08517778)(308.79938745,52.11517775)(308.72937988,52.16517873)
\curveto(308.68938756,52.19517767)(308.65938759,52.24517762)(308.63937988,52.31517873)
\curveto(308.61938763,52.39517747)(308.60938764,52.48017738)(308.60937988,52.57017873)
\curveto(308.59938765,52.6601772)(308.59938765,52.75017711)(308.60937988,52.84017873)
\curveto(308.61938763,52.93017693)(308.63438761,53.00017686)(308.65437988,53.05017873)
\curveto(308.68438756,53.13017673)(308.7443875,53.18017668)(308.83437988,53.20017873)
\curveto(308.91438733,53.23017663)(309.00438724,53.24517662)(309.10437988,53.24517873)
\lineto(309.40437988,53.24517873)
\curveto(309.50438674,53.24517662)(309.59438665,53.2651766)(309.67437988,53.30517873)
\curveto(309.69438655,53.31517655)(309.70938654,53.32517654)(309.71937988,53.33517873)
\lineto(309.76437988,53.38017873)
\curveto(309.76438648,53.49017637)(309.71938653,53.58017628)(309.62937988,53.65017873)
\curveto(309.52938672,53.72017614)(309.4493868,53.78017608)(309.38937988,53.83017873)
\lineto(309.29937988,53.92017873)
\curveto(309.18938706,54.01017585)(309.07438717,54.13517573)(308.95437988,54.29517873)
\curveto(308.83438741,54.45517541)(308.7443875,54.60517526)(308.68437988,54.74517873)
\curveto(308.63438761,54.83517503)(308.59938765,54.93017493)(308.57937988,55.03017873)
\curveto(308.5493877,55.13017473)(308.51938773,55.23517463)(308.48937988,55.34517873)
\curveto(308.47938777,55.40517446)(308.47438777,55.4651744)(308.47437988,55.52517873)
\curveto(308.46438778,55.58517428)(308.45438779,55.64017422)(308.44437988,55.69017873)
}
}
{
\newrgbcolor{curcolor}{0 0 0}
\pscustom[linestyle=none,fillstyle=solid,fillcolor=curcolor]
{
}
}
{
\newrgbcolor{curcolor}{0 0 0}
\pscustom[linestyle=none,fillstyle=solid,fillcolor=curcolor]
{
\newpath
\moveto(305.74437988,64.11010061)
\curveto(305.71439053,65.74009517)(306.26938998,66.79009412)(307.40937988,67.26010061)
\curveto(307.63938861,67.36009355)(307.92938832,67.42509348)(308.27937988,67.45510061)
\curveto(308.61938763,67.49509341)(308.92938732,67.47009344)(309.20937988,67.38010061)
\curveto(309.46938678,67.29009362)(309.69438655,67.17009374)(309.88437988,67.02010061)
\curveto(309.92438632,67.00009391)(309.95938629,66.97509393)(309.98937988,66.94510061)
\curveto(310.00938624,66.91509399)(310.03438621,66.89009402)(310.06437988,66.87010061)
\lineto(310.18437988,66.78010061)
\curveto(310.21438603,66.75009416)(310.23938601,66.71509419)(310.25937988,66.67510061)
\curveto(310.30938594,66.62509428)(310.35438589,66.57009434)(310.39437988,66.51010061)
\curveto(310.43438581,66.46009445)(310.48438576,66.41509449)(310.54437988,66.37510061)
\curveto(310.58438566,66.33509457)(310.63438561,66.32009459)(310.69437988,66.33010061)
\curveto(310.7443855,66.34009457)(310.78938546,66.37009454)(310.82937988,66.42010061)
\curveto(310.86938538,66.47009444)(310.90938534,66.52509438)(310.94937988,66.58510061)
\curveto(310.97938527,66.65509425)(311.00938524,66.72009419)(311.03937988,66.78010061)
\curveto(311.06938518,66.84009407)(311.09938515,66.89009402)(311.12937988,66.93010061)
\curveto(311.3493849,67.25009366)(311.65938459,67.5050934)(312.05937988,67.69510061)
\curveto(312.1493841,67.73509317)(312.244384,67.76509314)(312.34437988,67.78510061)
\curveto(312.43438381,67.81509309)(312.52438372,67.84009307)(312.61437988,67.86010061)
\curveto(312.66438358,67.87009304)(312.71438353,67.87509303)(312.76437988,67.87510061)
\curveto(312.80438344,67.88509302)(312.8493834,67.89509301)(312.89937988,67.90510061)
\curveto(312.9493833,67.91509299)(312.99938325,67.91509299)(313.04937988,67.90510061)
\curveto(313.09938315,67.89509301)(313.1493831,67.90009301)(313.19937988,67.92010061)
\curveto(313.249383,67.93009298)(313.30938294,67.93509297)(313.37937988,67.93510061)
\curveto(313.4493828,67.93509297)(313.50938274,67.92509298)(313.55937988,67.90510061)
\lineto(313.78437988,67.90510061)
\lineto(314.02437988,67.84510061)
\curveto(314.09438215,67.83509307)(314.16438208,67.82009309)(314.23437988,67.80010061)
\curveto(314.32438192,67.77009314)(314.40938184,67.74009317)(314.48937988,67.71010061)
\curveto(314.56938168,67.69009322)(314.6493816,67.66009325)(314.72937988,67.62010061)
\curveto(314.78938146,67.60009331)(314.8493814,67.57009334)(314.90937988,67.53010061)
\curveto(314.95938129,67.50009341)(315.00938124,67.46509344)(315.05937988,67.42510061)
\curveto(315.36938088,67.22509368)(315.62938062,66.97509393)(315.83937988,66.67510061)
\curveto(316.03938021,66.37509453)(316.20438004,66.03009488)(316.33437988,65.64010061)
\curveto(316.37437987,65.52009539)(316.39937985,65.39009552)(316.40937988,65.25010061)
\curveto(316.42937982,65.12009579)(316.45437979,64.98509592)(316.48437988,64.84510061)
\curveto(316.49437975,64.77509613)(316.49937975,64.7050962)(316.49937988,64.63510061)
\curveto(316.49937975,64.57509633)(316.50437974,64.5100964)(316.51437988,64.44010061)
\curveto(316.52437972,64.40009651)(316.52937972,64.34009657)(316.52937988,64.26010061)
\curveto(316.52937972,64.19009672)(316.52437972,64.14009677)(316.51437988,64.11010061)
\curveto(316.50437974,64.06009685)(316.49937975,64.01509689)(316.49937988,63.97510061)
\lineto(316.49937988,63.85510061)
\curveto(316.47937977,63.75509715)(316.46437978,63.65509725)(316.45437988,63.55510061)
\curveto(316.4443798,63.45509745)(316.42937982,63.36009755)(316.40937988,63.27010061)
\curveto(316.37937987,63.16009775)(316.35437989,63.05009786)(316.33437988,62.94010061)
\curveto(316.30437994,62.84009807)(316.26437998,62.73509817)(316.21437988,62.62510061)
\curveto(316.05438019,62.25509865)(315.85438039,61.94009897)(315.61437988,61.68010061)
\curveto(315.36438088,61.42009949)(315.05438119,61.2100997)(314.68437988,61.05010061)
\curveto(314.59438165,61.0100999)(314.49938175,60.97509993)(314.39937988,60.94510061)
\curveto(314.29938195,60.91509999)(314.19438205,60.88510002)(314.08437988,60.85510061)
\curveto(314.03438221,60.83510007)(313.98438226,60.82510008)(313.93437988,60.82510061)
\curveto(313.87438237,60.82510008)(313.81438243,60.81510009)(313.75437988,60.79510061)
\curveto(313.69438255,60.77510013)(313.61438263,60.76510014)(313.51437988,60.76510061)
\curveto(313.41438283,60.76510014)(313.33938291,60.78010013)(313.28937988,60.81010061)
\curveto(313.25938299,60.82010009)(313.23438301,60.83510007)(313.21437988,60.85510061)
\lineto(313.15437988,60.91510061)
\curveto(313.13438311,60.95509995)(313.11938313,61.01509989)(313.10937988,61.09510061)
\curveto(313.09938315,61.18509972)(313.09438315,61.27509963)(313.09437988,61.36510061)
\curveto(313.09438315,61.45509945)(313.09938315,61.54009937)(313.10937988,61.62010061)
\curveto(313.11938313,61.7100992)(313.12938312,61.77509913)(313.13937988,61.81510061)
\curveto(313.15938309,61.83509907)(313.17438307,61.85509905)(313.18437988,61.87510061)
\curveto(313.18438306,61.89509901)(313.19438305,61.91509899)(313.21437988,61.93510061)
\curveto(313.30438294,62.0050989)(313.41938283,62.04509886)(313.55937988,62.05510061)
\curveto(313.69938255,62.07509883)(313.82438242,62.1050988)(313.93437988,62.14510061)
\lineto(314.29437988,62.29510061)
\curveto(314.40438184,62.34509856)(314.50938174,62.4100985)(314.60937988,62.49010061)
\curveto(314.63938161,62.5100984)(314.66438158,62.53009838)(314.68437988,62.55010061)
\curveto(314.70438154,62.58009833)(314.72938152,62.6050983)(314.75937988,62.62510061)
\curveto(314.81938143,62.66509824)(314.86438138,62.70009821)(314.89437988,62.73010061)
\curveto(314.92438132,62.77009814)(314.95438129,62.8050981)(314.98437988,62.83510061)
\curveto(315.01438123,62.87509803)(315.0443812,62.92009799)(315.07437988,62.97010061)
\curveto(315.13438111,63.06009785)(315.18438106,63.15509775)(315.22437988,63.25510061)
\lineto(315.34437988,63.58510061)
\curveto(315.39438085,63.73509717)(315.42438082,63.93509697)(315.43437988,64.18510061)
\curveto(315.4443808,64.43509647)(315.42438082,64.64509626)(315.37437988,64.81510061)
\curveto(315.35438089,64.89509601)(315.33938091,64.96509594)(315.32937988,65.02510061)
\lineto(315.26937988,65.23510061)
\curveto(315.1493811,65.51509539)(314.99938125,65.75509515)(314.81937988,65.95510061)
\curveto(314.63938161,66.16509474)(314.40938184,66.33009458)(314.12937988,66.45010061)
\curveto(314.05938219,66.48009443)(313.98938226,66.50009441)(313.91937988,66.51010061)
\lineto(313.67937988,66.57010061)
\curveto(313.53938271,66.6100943)(313.37938287,66.62009429)(313.19937988,66.60010061)
\curveto(313.00938324,66.58009433)(312.85938339,66.55009436)(312.74937988,66.51010061)
\curveto(312.36938388,66.38009453)(312.07938417,66.19509471)(311.87937988,65.95510061)
\curveto(311.67938457,65.72509518)(311.51938473,65.41509549)(311.39937988,65.02510061)
\curveto(311.36938488,64.91509599)(311.3493849,64.79509611)(311.33937988,64.66510061)
\curveto(311.32938492,64.54509636)(311.32438492,64.42009649)(311.32437988,64.29010061)
\curveto(311.32438492,64.13009678)(311.31938493,63.99009692)(311.30937988,63.87010061)
\curveto(311.29938495,63.75009716)(311.23938501,63.66509724)(311.12937988,63.61510061)
\curveto(311.09938515,63.59509731)(311.06438518,63.58509732)(311.02437988,63.58510061)
\lineto(310.88937988,63.58510061)
\curveto(310.78938546,63.57509733)(310.69438555,63.57509733)(310.60437988,63.58510061)
\curveto(310.51438573,63.6050973)(310.4493858,63.64509726)(310.40937988,63.70510061)
\curveto(310.37938587,63.74509716)(310.35938589,63.78509712)(310.34937988,63.82510061)
\curveto(310.33938591,63.87509703)(310.32938592,63.93009698)(310.31937988,63.99010061)
\curveto(310.30938594,64.0100969)(310.30938594,64.03509687)(310.31937988,64.06510061)
\curveto(310.31938593,64.09509681)(310.31438593,64.12009679)(310.30437988,64.14010061)
\lineto(310.30437988,64.27510061)
\curveto(310.28438596,64.38509652)(310.27438597,64.48509642)(310.27437988,64.57510061)
\curveto(310.26438598,64.67509623)(310.244386,64.77009614)(310.21437988,64.86010061)
\curveto(310.10438614,65.18009573)(309.95938629,65.43509547)(309.77937988,65.62510061)
\curveto(309.59938665,65.81509509)(309.3493869,65.96509494)(309.02937988,66.07510061)
\curveto(308.92938732,66.1050948)(308.80438744,66.12509478)(308.65437988,66.13510061)
\curveto(308.49438775,66.15509475)(308.3493879,66.15009476)(308.21937988,66.12010061)
\curveto(308.1493881,66.10009481)(308.08438816,66.08009483)(308.02437988,66.06010061)
\curveto(307.95438829,66.05009486)(307.88938836,66.03009488)(307.82937988,66.00010061)
\curveto(307.58938866,65.90009501)(307.39938885,65.75509515)(307.25937988,65.56510061)
\curveto(307.11938913,65.37509553)(307.00938924,65.15009576)(306.92937988,64.89010061)
\curveto(306.90938934,64.83009608)(306.89938935,64.77009614)(306.89937988,64.71010061)
\curveto(306.89938935,64.65009626)(306.88938936,64.58509632)(306.86937988,64.51510061)
\curveto(306.8493894,64.43509647)(306.83938941,64.34009657)(306.83937988,64.23010061)
\curveto(306.83938941,64.12009679)(306.8493894,64.02509688)(306.86937988,63.94510061)
\curveto(306.88938936,63.89509701)(306.89938935,63.84509706)(306.89937988,63.79510061)
\curveto(306.89938935,63.75509715)(306.90938934,63.7100972)(306.92937988,63.66010061)
\curveto(306.97938927,63.48009743)(307.05438919,63.3100976)(307.15437988,63.15010061)
\curveto(307.244389,63.00009791)(307.35938889,62.87009804)(307.49937988,62.76010061)
\curveto(307.61938863,62.67009824)(307.7493885,62.59009832)(307.88937988,62.52010061)
\curveto(308.02938822,62.45009846)(308.18438806,62.38509852)(308.35437988,62.32510061)
\curveto(308.46438778,62.29509861)(308.58438766,62.27509863)(308.71437988,62.26510061)
\curveto(308.83438741,62.25509865)(308.93438731,62.22009869)(309.01437988,62.16010061)
\curveto(309.05438719,62.14009877)(309.09438715,62.08009883)(309.13437988,61.98010061)
\curveto(309.1443871,61.94009897)(309.15438709,61.88009903)(309.16437988,61.80010061)
\lineto(309.16437988,61.54510061)
\curveto(309.15438709,61.45509945)(309.1443871,61.37009954)(309.13437988,61.29010061)
\curveto(309.12438712,61.22009969)(309.10938714,61.17009974)(309.08937988,61.14010061)
\curveto(309.05938719,61.10009981)(309.00438724,61.06509984)(308.92437988,61.03510061)
\curveto(308.8443874,61.0050999)(308.75938749,61.00009991)(308.66937988,61.02010061)
\curveto(308.61938763,61.03009988)(308.56938768,61.03509987)(308.51937988,61.03510061)
\lineto(308.33937988,61.06510061)
\curveto(308.23938801,61.09509981)(308.13938811,61.12009979)(308.03937988,61.14010061)
\curveto(307.93938831,61.17009974)(307.8493884,61.2050997)(307.76937988,61.24510061)
\curveto(307.65938859,61.29509961)(307.55438869,61.34009957)(307.45437988,61.38010061)
\curveto(307.3443889,61.42009949)(307.23938901,61.47009944)(307.13937988,61.53010061)
\curveto(306.59938965,61.86009905)(306.20439004,62.33009858)(305.95437988,62.94010061)
\curveto(305.90439034,63.06009785)(305.86939038,63.18509772)(305.84937988,63.31510061)
\curveto(305.82939042,63.45509745)(305.80439044,63.59509731)(305.77437988,63.73510061)
\curveto(305.76439048,63.79509711)(305.75939049,63.85509705)(305.75937988,63.91510061)
\curveto(305.75939049,63.98509692)(305.75439049,64.05009686)(305.74437988,64.11010061)
}
}
{
\newrgbcolor{curcolor}{0 0 0}
\pscustom[linestyle=none,fillstyle=solid,fillcolor=curcolor]
{
\newpath
\moveto(311.26437988,76.34470998)
\lineto(311.51937988,76.34470998)
\curveto(311.59938465,76.35470228)(311.67438457,76.34970228)(311.74437988,76.32970998)
\lineto(311.98437988,76.32970998)
\lineto(312.14937988,76.32970998)
\curveto(312.249384,76.30970232)(312.35438389,76.29970233)(312.46437988,76.29970998)
\curveto(312.56438368,76.29970233)(312.66438358,76.28970234)(312.76437988,76.26970998)
\lineto(312.91437988,76.26970998)
\curveto(313.05438319,76.23970239)(313.19438305,76.21970241)(313.33437988,76.20970998)
\curveto(313.46438278,76.19970243)(313.59438265,76.17470246)(313.72437988,76.13470998)
\curveto(313.80438244,76.11470252)(313.88938236,76.09470254)(313.97937988,76.07470998)
\lineto(314.21937988,76.01470998)
\lineto(314.51937988,75.89470998)
\curveto(314.60938164,75.86470277)(314.69938155,75.8297028)(314.78937988,75.78970998)
\curveto(315.00938124,75.68970294)(315.22438102,75.55470308)(315.43437988,75.38470998)
\curveto(315.6443806,75.22470341)(315.81438043,75.04970358)(315.94437988,74.85970998)
\curveto(315.98438026,74.80970382)(316.02438022,74.74970388)(316.06437988,74.67970998)
\curveto(316.09438015,74.61970401)(316.12938012,74.55970407)(316.16937988,74.49970998)
\curveto(316.21938003,74.41970421)(316.25937999,74.32470431)(316.28937988,74.21470998)
\curveto(316.31937993,74.10470453)(316.3493799,73.99970463)(316.37937988,73.89970998)
\curveto(316.41937983,73.78970484)(316.4443798,73.67970495)(316.45437988,73.56970998)
\curveto(316.46437978,73.45970517)(316.47937977,73.34470529)(316.49937988,73.22470998)
\curveto(316.50937974,73.18470545)(316.50937974,73.13970549)(316.49937988,73.08970998)
\curveto(316.49937975,73.04970558)(316.50437974,73.00970562)(316.51437988,72.96970998)
\curveto(316.52437972,72.9297057)(316.52937972,72.87470576)(316.52937988,72.80470998)
\curveto(316.52937972,72.7347059)(316.52437972,72.68470595)(316.51437988,72.65470998)
\curveto(316.49437975,72.60470603)(316.48937976,72.55970607)(316.49937988,72.51970998)
\curveto(316.50937974,72.47970615)(316.50937974,72.44470619)(316.49937988,72.41470998)
\lineto(316.49937988,72.32470998)
\curveto(316.47937977,72.26470637)(316.46437978,72.19970643)(316.45437988,72.12970998)
\curveto(316.45437979,72.06970656)(316.4493798,72.00470663)(316.43937988,71.93470998)
\curveto(316.38937986,71.76470687)(316.33937991,71.60470703)(316.28937988,71.45470998)
\curveto(316.23938001,71.30470733)(316.17438007,71.15970747)(316.09437988,71.01970998)
\curveto(316.05438019,70.96970766)(316.02438022,70.91470772)(316.00437988,70.85470998)
\curveto(315.97438027,70.80470783)(315.93938031,70.75470788)(315.89937988,70.70470998)
\curveto(315.71938053,70.46470817)(315.49938075,70.26470837)(315.23937988,70.10470998)
\curveto(314.97938127,69.94470869)(314.69438155,69.80470883)(314.38437988,69.68470998)
\curveto(314.244382,69.62470901)(314.10438214,69.57970905)(313.96437988,69.54970998)
\curveto(313.81438243,69.51970911)(313.65938259,69.48470915)(313.49937988,69.44470998)
\curveto(313.38938286,69.42470921)(313.27938297,69.40970922)(313.16937988,69.39970998)
\curveto(313.05938319,69.38970924)(312.9493833,69.37470926)(312.83937988,69.35470998)
\curveto(312.79938345,69.34470929)(312.75938349,69.33970929)(312.71937988,69.33970998)
\curveto(312.67938357,69.34970928)(312.63938361,69.34970928)(312.59937988,69.33970998)
\curveto(312.5493837,69.3297093)(312.49938375,69.32470931)(312.44937988,69.32470998)
\lineto(312.28437988,69.32470998)
\curveto(312.23438401,69.30470933)(312.18438406,69.29970933)(312.13437988,69.30970998)
\curveto(312.07438417,69.31970931)(312.01938423,69.31970931)(311.96937988,69.30970998)
\curveto(311.92938432,69.29970933)(311.88438436,69.29970933)(311.83437988,69.30970998)
\curveto(311.78438446,69.31970931)(311.73438451,69.31470932)(311.68437988,69.29470998)
\curveto(311.61438463,69.27470936)(311.53938471,69.26970936)(311.45937988,69.27970998)
\curveto(311.36938488,69.28970934)(311.28438496,69.29470934)(311.20437988,69.29470998)
\curveto(311.11438513,69.29470934)(311.01438523,69.28970934)(310.90437988,69.27970998)
\curveto(310.78438546,69.26970936)(310.68438556,69.27470936)(310.60437988,69.29470998)
\lineto(310.31937988,69.29470998)
\lineto(309.68937988,69.33970998)
\curveto(309.58938666,69.34970928)(309.49438675,69.35970927)(309.40437988,69.36970998)
\lineto(309.10437988,69.39970998)
\curveto(309.05438719,69.41970921)(309.00438724,69.42470921)(308.95437988,69.41470998)
\curveto(308.89438735,69.41470922)(308.83938741,69.42470921)(308.78937988,69.44470998)
\curveto(308.61938763,69.49470914)(308.45438779,69.5347091)(308.29437988,69.56470998)
\curveto(308.12438812,69.59470904)(307.96438828,69.64470899)(307.81437988,69.71470998)
\curveto(307.35438889,69.90470873)(306.97938927,70.12470851)(306.68937988,70.37470998)
\curveto(306.39938985,70.634708)(306.15439009,70.99470764)(305.95437988,71.45470998)
\curveto(305.90439034,71.58470705)(305.86939038,71.71470692)(305.84937988,71.84470998)
\curveto(305.82939042,71.98470665)(305.80439044,72.12470651)(305.77437988,72.26470998)
\curveto(305.76439048,72.3347063)(305.75939049,72.39970623)(305.75937988,72.45970998)
\curveto(305.75939049,72.51970611)(305.75439049,72.58470605)(305.74437988,72.65470998)
\curveto(305.72439052,73.48470515)(305.87439037,74.15470448)(306.19437988,74.66470998)
\curveto(306.50438974,75.17470346)(306.9443893,75.55470308)(307.51437988,75.80470998)
\curveto(307.63438861,75.85470278)(307.75938849,75.89970273)(307.88937988,75.93970998)
\curveto(308.01938823,75.97970265)(308.15438809,76.02470261)(308.29437988,76.07470998)
\curveto(308.37438787,76.09470254)(308.45938779,76.10970252)(308.54937988,76.11970998)
\lineto(308.78937988,76.17970998)
\curveto(308.89938735,76.20970242)(309.00938724,76.22470241)(309.11937988,76.22470998)
\curveto(309.22938702,76.2347024)(309.33938691,76.24970238)(309.44937988,76.26970998)
\curveto(309.49938675,76.28970234)(309.5443867,76.29470234)(309.58437988,76.28470998)
\curveto(309.62438662,76.28470235)(309.66438658,76.28970234)(309.70437988,76.29970998)
\curveto(309.75438649,76.30970232)(309.80938644,76.30970232)(309.86937988,76.29970998)
\curveto(309.91938633,76.29970233)(309.96938628,76.30470233)(310.01937988,76.31470998)
\lineto(310.15437988,76.31470998)
\curveto(310.21438603,76.3347023)(310.28438596,76.3347023)(310.36437988,76.31470998)
\curveto(310.43438581,76.30470233)(310.49938575,76.30970232)(310.55937988,76.32970998)
\curveto(310.58938566,76.33970229)(310.62938562,76.34470229)(310.67937988,76.34470998)
\lineto(310.79937988,76.34470998)
\lineto(311.26437988,76.34470998)
\moveto(313.58937988,74.79970998)
\curveto(313.26938298,74.89970373)(312.90438334,74.95970367)(312.49437988,74.97970998)
\curveto(312.08438416,74.99970363)(311.67438457,75.00970362)(311.26437988,75.00970998)
\curveto(310.83438541,75.00970362)(310.41438583,74.99970363)(310.00437988,74.97970998)
\curveto(309.59438665,74.95970367)(309.20938704,74.91470372)(308.84937988,74.84470998)
\curveto(308.48938776,74.77470386)(308.16938808,74.66470397)(307.88937988,74.51470998)
\curveto(307.59938865,74.37470426)(307.36438888,74.17970445)(307.18437988,73.92970998)
\curveto(307.07438917,73.76970486)(306.99438925,73.58970504)(306.94437988,73.38970998)
\curveto(306.88438936,73.18970544)(306.85438939,72.94470569)(306.85437988,72.65470998)
\curveto(306.87438937,72.634706)(306.88438936,72.59970603)(306.88437988,72.54970998)
\curveto(306.87438937,72.49970613)(306.87438937,72.45970617)(306.88437988,72.42970998)
\curveto(306.90438934,72.34970628)(306.92438932,72.27470636)(306.94437988,72.20470998)
\curveto(306.95438929,72.14470649)(306.97438927,72.07970655)(307.00437988,72.00970998)
\curveto(307.12438912,71.73970689)(307.29438895,71.51970711)(307.51437988,71.34970998)
\curveto(307.72438852,71.18970744)(307.96938828,71.05470758)(308.24937988,70.94470998)
\curveto(308.35938789,70.89470774)(308.47938777,70.85470778)(308.60937988,70.82470998)
\curveto(308.72938752,70.80470783)(308.85438739,70.77970785)(308.98437988,70.74970998)
\curveto(309.03438721,70.7297079)(309.08938716,70.71970791)(309.14937988,70.71970998)
\curveto(309.19938705,70.71970791)(309.249387,70.71470792)(309.29937988,70.70470998)
\curveto(309.38938686,70.69470794)(309.48438676,70.68470795)(309.58437988,70.67470998)
\curveto(309.67438657,70.66470797)(309.76938648,70.65470798)(309.86937988,70.64470998)
\curveto(309.9493863,70.64470799)(310.03438621,70.63970799)(310.12437988,70.62970998)
\lineto(310.36437988,70.62970998)
\lineto(310.54437988,70.62970998)
\curveto(310.57438567,70.61970801)(310.60938564,70.61470802)(310.64937988,70.61470998)
\lineto(310.78437988,70.61470998)
\lineto(311.23437988,70.61470998)
\curveto(311.31438493,70.61470802)(311.39938485,70.60970802)(311.48937988,70.59970998)
\curveto(311.56938468,70.59970803)(311.6443846,70.60970802)(311.71437988,70.62970998)
\lineto(311.98437988,70.62970998)
\curveto(312.00438424,70.629708)(312.03438421,70.62470801)(312.07437988,70.61470998)
\curveto(312.10438414,70.61470802)(312.12938412,70.61970801)(312.14937988,70.62970998)
\curveto(312.249384,70.63970799)(312.3493839,70.64470799)(312.44937988,70.64470998)
\curveto(312.53938371,70.65470798)(312.63938361,70.66470797)(312.74937988,70.67470998)
\curveto(312.86938338,70.70470793)(312.99438325,70.71970791)(313.12437988,70.71970998)
\curveto(313.244383,70.7297079)(313.35938289,70.75470788)(313.46937988,70.79470998)
\curveto(313.76938248,70.87470776)(314.03438221,70.95970767)(314.26437988,71.04970998)
\curveto(314.49438175,71.14970748)(314.70938154,71.29470734)(314.90937988,71.48470998)
\curveto(315.10938114,71.69470694)(315.25938099,71.95970667)(315.35937988,72.27970998)
\curveto(315.37938087,72.31970631)(315.38938086,72.35470628)(315.38937988,72.38470998)
\curveto(315.37938087,72.42470621)(315.38438086,72.46970616)(315.40437988,72.51970998)
\curveto(315.41438083,72.55970607)(315.42438082,72.629706)(315.43437988,72.72970998)
\curveto(315.4443808,72.83970579)(315.43938081,72.92470571)(315.41937988,72.98470998)
\curveto(315.39938085,73.05470558)(315.38938086,73.12470551)(315.38937988,73.19470998)
\curveto(315.37938087,73.26470537)(315.36438088,73.3297053)(315.34437988,73.38970998)
\curveto(315.28438096,73.58970504)(315.19938105,73.76970486)(315.08937988,73.92970998)
\curveto(315.06938118,73.95970467)(315.0493812,73.98470465)(315.02937988,74.00470998)
\lineto(314.96937988,74.06470998)
\curveto(314.9493813,74.10470453)(314.90938134,74.15470448)(314.84937988,74.21470998)
\curveto(314.70938154,74.31470432)(314.57938167,74.39970423)(314.45937988,74.46970998)
\curveto(314.33938191,74.53970409)(314.19438205,74.60970402)(314.02437988,74.67970998)
\curveto(313.95438229,74.70970392)(313.88438236,74.7297039)(313.81437988,74.73970998)
\curveto(313.7443825,74.75970387)(313.66938258,74.77970385)(313.58937988,74.79970998)
}
}
{
\newrgbcolor{curcolor}{0 0 0}
\pscustom[linestyle=none,fillstyle=solid,fillcolor=curcolor]
{
\newpath
\moveto(314.71437988,78.63431936)
\lineto(314.71437988,79.26431936)
\lineto(314.71437988,79.45931936)
\curveto(314.71438153,79.52931683)(314.72438152,79.58931677)(314.74437988,79.63931936)
\curveto(314.78438146,79.70931665)(314.82438142,79.7593166)(314.86437988,79.78931936)
\curveto(314.91438133,79.82931653)(314.97938127,79.84931651)(315.05937988,79.84931936)
\curveto(315.13938111,79.8593165)(315.22438102,79.86431649)(315.31437988,79.86431936)
\lineto(316.03437988,79.86431936)
\curveto(316.51437973,79.86431649)(316.92437932,79.80431655)(317.26437988,79.68431936)
\curveto(317.60437864,79.56431679)(317.87937837,79.36931699)(318.08937988,79.09931936)
\curveto(318.13937811,79.02931733)(318.18437806,78.9593174)(318.22437988,78.88931936)
\curveto(318.27437797,78.82931753)(318.31937793,78.7543176)(318.35937988,78.66431936)
\curveto(318.36937788,78.64431771)(318.37937787,78.61431774)(318.38937988,78.57431936)
\curveto(318.40937784,78.53431782)(318.41437783,78.48931787)(318.40437988,78.43931936)
\curveto(318.37437787,78.34931801)(318.29937795,78.29431806)(318.17937988,78.27431936)
\curveto(318.06937818,78.2543181)(317.97437827,78.26931809)(317.89437988,78.31931936)
\curveto(317.82437842,78.34931801)(317.75937849,78.39431796)(317.69937988,78.45431936)
\curveto(317.6493786,78.52431783)(317.59937865,78.58931777)(317.54937988,78.64931936)
\curveto(317.49937875,78.71931764)(317.42437882,78.77931758)(317.32437988,78.82931936)
\curveto(317.23437901,78.88931747)(317.1443791,78.93931742)(317.05437988,78.97931936)
\curveto(317.02437922,78.99931736)(316.96437928,79.02431733)(316.87437988,79.05431936)
\curveto(316.79437945,79.08431727)(316.72437952,79.08931727)(316.66437988,79.06931936)
\curveto(316.52437972,79.03931732)(316.43437981,78.97931738)(316.39437988,78.88931936)
\curveto(316.36437988,78.80931755)(316.3493799,78.71931764)(316.34937988,78.61931936)
\curveto(316.3493799,78.51931784)(316.32437992,78.43431792)(316.27437988,78.36431936)
\curveto(316.20438004,78.27431808)(316.06438018,78.22931813)(315.85437988,78.22931936)
\lineto(315.29937988,78.22931936)
\lineto(315.07437988,78.22931936)
\curveto(314.99438125,78.23931812)(314.92938132,78.2593181)(314.87937988,78.28931936)
\curveto(314.79938145,78.34931801)(314.75438149,78.41931794)(314.74437988,78.49931936)
\curveto(314.73438151,78.51931784)(314.72938152,78.53931782)(314.72937988,78.55931936)
\curveto(314.72938152,78.58931777)(314.72438152,78.61431774)(314.71437988,78.63431936)
}
}
{
\newrgbcolor{curcolor}{0 0 0}
\pscustom[linestyle=none,fillstyle=solid,fillcolor=curcolor]
{
}
}
{
\newrgbcolor{curcolor}{0 0 0}
\pscustom[linestyle=none,fillstyle=solid,fillcolor=curcolor]
{
\newpath
\moveto(305.74437988,89.26463186)
\curveto(305.73439051,89.95462722)(305.85439039,90.55462662)(306.10437988,91.06463186)
\curveto(306.35438989,91.58462559)(306.68938956,91.9796252)(307.10937988,92.24963186)
\curveto(307.18938906,92.29962488)(307.27938897,92.34462483)(307.37937988,92.38463186)
\curveto(307.46938878,92.42462475)(307.56438868,92.46962471)(307.66437988,92.51963186)
\curveto(307.76438848,92.55962462)(307.86438838,92.58962459)(307.96437988,92.60963186)
\curveto(308.06438818,92.62962455)(308.16938808,92.64962453)(308.27937988,92.66963186)
\curveto(308.32938792,92.68962449)(308.37438787,92.69462448)(308.41437988,92.68463186)
\curveto(308.45438779,92.6746245)(308.49938775,92.6796245)(308.54937988,92.69963186)
\curveto(308.59938765,92.70962447)(308.68438756,92.71462446)(308.80437988,92.71463186)
\curveto(308.91438733,92.71462446)(308.99938725,92.70962447)(309.05937988,92.69963186)
\curveto(309.11938713,92.6796245)(309.17938707,92.66962451)(309.23937988,92.66963186)
\curveto(309.29938695,92.6796245)(309.35938689,92.6746245)(309.41937988,92.65463186)
\curveto(309.55938669,92.61462456)(309.69438655,92.5796246)(309.82437988,92.54963186)
\curveto(309.95438629,92.51962466)(310.07938617,92.4796247)(310.19937988,92.42963186)
\curveto(310.33938591,92.36962481)(310.46438578,92.29962488)(310.57437988,92.21963186)
\curveto(310.68438556,92.14962503)(310.79438545,92.0746251)(310.90437988,91.99463186)
\lineto(310.96437988,91.93463186)
\curveto(310.98438526,91.92462525)(311.00438524,91.90962527)(311.02437988,91.88963186)
\curveto(311.18438506,91.76962541)(311.32938492,91.63462554)(311.45937988,91.48463186)
\curveto(311.58938466,91.33462584)(311.71438453,91.174626)(311.83437988,91.00463186)
\curveto(312.05438419,90.69462648)(312.25938399,90.39962678)(312.44937988,90.11963186)
\curveto(312.58938366,89.88962729)(312.72438352,89.65962752)(312.85437988,89.42963186)
\curveto(312.98438326,89.20962797)(313.11938313,88.98962819)(313.25937988,88.76963186)
\curveto(313.42938282,88.51962866)(313.60938264,88.2796289)(313.79937988,88.04963186)
\curveto(313.98938226,87.82962935)(314.21438203,87.63962954)(314.47437988,87.47963186)
\curveto(314.53438171,87.43962974)(314.59438165,87.40462977)(314.65437988,87.37463186)
\curveto(314.70438154,87.34462983)(314.76938148,87.31462986)(314.84937988,87.28463186)
\curveto(314.91938133,87.26462991)(314.97938127,87.25962992)(315.02937988,87.26963186)
\curveto(315.09938115,87.28962989)(315.15438109,87.32462985)(315.19437988,87.37463186)
\curveto(315.22438102,87.42462975)(315.244381,87.48462969)(315.25437988,87.55463186)
\lineto(315.25437988,87.79463186)
\lineto(315.25437988,88.54463186)
\lineto(315.25437988,91.34963186)
\lineto(315.25437988,92.00963186)
\curveto(315.25438099,92.09962508)(315.25938099,92.18462499)(315.26937988,92.26463186)
\curveto(315.26938098,92.34462483)(315.28938096,92.40962477)(315.32937988,92.45963186)
\curveto(315.36938088,92.50962467)(315.4443808,92.54962463)(315.55437988,92.57963186)
\curveto(315.65438059,92.61962456)(315.75438049,92.62962455)(315.85437988,92.60963186)
\lineto(315.98937988,92.60963186)
\curveto(316.05938019,92.58962459)(316.11938013,92.56962461)(316.16937988,92.54963186)
\curveto(316.21938003,92.52962465)(316.25937999,92.49462468)(316.28937988,92.44463186)
\curveto(316.32937992,92.39462478)(316.3493799,92.32462485)(316.34937988,92.23463186)
\lineto(316.34937988,91.96463186)
\lineto(316.34937988,91.06463186)
\lineto(316.34937988,87.55463186)
\lineto(316.34937988,86.48963186)
\curveto(316.3493799,86.40963077)(316.35437989,86.31963086)(316.36437988,86.21963186)
\curveto(316.36437988,86.11963106)(316.35437989,86.03463114)(316.33437988,85.96463186)
\curveto(316.26437998,85.75463142)(316.08438016,85.68963149)(315.79437988,85.76963186)
\curveto(315.75438049,85.7796314)(315.71938053,85.7796314)(315.68937988,85.76963186)
\curveto(315.6493806,85.76963141)(315.60438064,85.7796314)(315.55437988,85.79963186)
\curveto(315.47438077,85.81963136)(315.38938086,85.83963134)(315.29937988,85.85963186)
\curveto(315.20938104,85.8796313)(315.12438112,85.90463127)(315.04437988,85.93463186)
\curveto(314.55438169,86.09463108)(314.13938211,86.29463088)(313.79937988,86.53463186)
\curveto(313.5493827,86.71463046)(313.32438292,86.91963026)(313.12437988,87.14963186)
\curveto(312.91438333,87.3796298)(312.71938353,87.61962956)(312.53937988,87.86963186)
\curveto(312.35938389,88.12962905)(312.18938406,88.39462878)(312.02937988,88.66463186)
\curveto(311.85938439,88.94462823)(311.68438456,89.21462796)(311.50437988,89.47463186)
\curveto(311.42438482,89.58462759)(311.3493849,89.68962749)(311.27937988,89.78963186)
\curveto(311.20938504,89.89962728)(311.13438511,90.00962717)(311.05437988,90.11963186)
\curveto(311.02438522,90.15962702)(310.99438525,90.19462698)(310.96437988,90.22463186)
\curveto(310.92438532,90.26462691)(310.89438535,90.30462687)(310.87437988,90.34463186)
\curveto(310.76438548,90.48462669)(310.63938561,90.60962657)(310.49937988,90.71963186)
\curveto(310.46938578,90.73962644)(310.4443858,90.76462641)(310.42437988,90.79463186)
\curveto(310.39438585,90.82462635)(310.36438588,90.84962633)(310.33437988,90.86963186)
\curveto(310.23438601,90.94962623)(310.13438611,91.01462616)(310.03437988,91.06463186)
\curveto(309.93438631,91.12462605)(309.82438642,91.179626)(309.70437988,91.22963186)
\curveto(309.63438661,91.25962592)(309.55938669,91.2796259)(309.47937988,91.28963186)
\lineto(309.23937988,91.34963186)
\lineto(309.14937988,91.34963186)
\curveto(309.11938713,91.35962582)(309.08938716,91.36462581)(309.05937988,91.36463186)
\curveto(308.98938726,91.38462579)(308.89438735,91.38962579)(308.77437988,91.37963186)
\curveto(308.6443876,91.3796258)(308.5443877,91.36962581)(308.47437988,91.34963186)
\curveto(308.39438785,91.32962585)(308.31938793,91.30962587)(308.24937988,91.28963186)
\curveto(308.16938808,91.2796259)(308.08938816,91.25962592)(308.00937988,91.22963186)
\curveto(307.76938848,91.11962606)(307.56938868,90.96962621)(307.40937988,90.77963186)
\curveto(307.23938901,90.59962658)(307.09938915,90.3796268)(306.98937988,90.11963186)
\curveto(306.96938928,90.04962713)(306.95438929,89.9796272)(306.94437988,89.90963186)
\curveto(306.92438932,89.83962734)(306.90438934,89.76462741)(306.88437988,89.68463186)
\curveto(306.86438938,89.60462757)(306.85438939,89.49462768)(306.85437988,89.35463186)
\curveto(306.85438939,89.22462795)(306.86438938,89.11962806)(306.88437988,89.03963186)
\curveto(306.89438935,88.9796282)(306.89938935,88.92462825)(306.89937988,88.87463186)
\curveto(306.89938935,88.82462835)(306.90938934,88.7746284)(306.92937988,88.72463186)
\curveto(306.96938928,88.62462855)(307.00938924,88.52962865)(307.04937988,88.43963186)
\curveto(307.08938916,88.35962882)(307.13438911,88.2796289)(307.18437988,88.19963186)
\curveto(307.20438904,88.16962901)(307.22938902,88.13962904)(307.25937988,88.10963186)
\curveto(307.28938896,88.08962909)(307.31438893,88.06462911)(307.33437988,88.03463186)
\lineto(307.40937988,87.95963186)
\curveto(307.42938882,87.92962925)(307.4493888,87.90462927)(307.46937988,87.88463186)
\lineto(307.67937988,87.73463186)
\curveto(307.73938851,87.69462948)(307.80438844,87.64962953)(307.87437988,87.59963186)
\curveto(307.96438828,87.53962964)(308.06938818,87.48962969)(308.18937988,87.44963186)
\curveto(308.29938795,87.41962976)(308.40938784,87.38462979)(308.51937988,87.34463186)
\curveto(308.62938762,87.30462987)(308.77438747,87.2796299)(308.95437988,87.26963186)
\curveto(309.12438712,87.25962992)(309.249387,87.22962995)(309.32937988,87.17963186)
\curveto(309.40938684,87.12963005)(309.45438679,87.05463012)(309.46437988,86.95463186)
\curveto(309.47438677,86.85463032)(309.47938677,86.74463043)(309.47937988,86.62463186)
\curveto(309.47938677,86.58463059)(309.48438676,86.54463063)(309.49437988,86.50463186)
\curveto(309.49438675,86.46463071)(309.48938676,86.42963075)(309.47937988,86.39963186)
\curveto(309.45938679,86.34963083)(309.4493868,86.29963088)(309.44937988,86.24963186)
\curveto(309.4493868,86.20963097)(309.43938681,86.16963101)(309.41937988,86.12963186)
\curveto(309.35938689,86.03963114)(309.22438702,85.99463118)(309.01437988,85.99463186)
\lineto(308.89437988,85.99463186)
\curveto(308.83438741,86.00463117)(308.77438747,86.00963117)(308.71437988,86.00963186)
\curveto(308.6443876,86.01963116)(308.57938767,86.02963115)(308.51937988,86.03963186)
\curveto(308.40938784,86.05963112)(308.30938794,86.0796311)(308.21937988,86.09963186)
\curveto(308.11938813,86.11963106)(308.02438822,86.14963103)(307.93437988,86.18963186)
\curveto(307.86438838,86.20963097)(307.80438844,86.22963095)(307.75437988,86.24963186)
\lineto(307.57437988,86.30963186)
\curveto(307.31438893,86.42963075)(307.06938918,86.58463059)(306.83937988,86.77463186)
\curveto(306.60938964,86.9746302)(306.42438982,87.18962999)(306.28437988,87.41963186)
\curveto(306.20439004,87.52962965)(306.13939011,87.64462953)(306.08937988,87.76463186)
\lineto(305.93937988,88.15463186)
\curveto(305.88939036,88.26462891)(305.85939039,88.3796288)(305.84937988,88.49963186)
\curveto(305.82939042,88.61962856)(305.80439044,88.74462843)(305.77437988,88.87463186)
\curveto(305.77439047,88.94462823)(305.77439047,89.00962817)(305.77437988,89.06963186)
\curveto(305.76439048,89.12962805)(305.75439049,89.19462798)(305.74437988,89.26463186)
}
}
{
\newrgbcolor{curcolor}{0 0 0}
\pscustom[linestyle=none,fillstyle=solid,fillcolor=curcolor]
{
\newpath
\moveto(311.26437988,101.36424123)
\lineto(311.51937988,101.36424123)
\curveto(311.59938465,101.37423353)(311.67438457,101.36923353)(311.74437988,101.34924123)
\lineto(311.98437988,101.34924123)
\lineto(312.14937988,101.34924123)
\curveto(312.249384,101.32923357)(312.35438389,101.31923358)(312.46437988,101.31924123)
\curveto(312.56438368,101.31923358)(312.66438358,101.30923359)(312.76437988,101.28924123)
\lineto(312.91437988,101.28924123)
\curveto(313.05438319,101.25923364)(313.19438305,101.23923366)(313.33437988,101.22924123)
\curveto(313.46438278,101.21923368)(313.59438265,101.19423371)(313.72437988,101.15424123)
\curveto(313.80438244,101.13423377)(313.88938236,101.11423379)(313.97937988,101.09424123)
\lineto(314.21937988,101.03424123)
\lineto(314.51937988,100.91424123)
\curveto(314.60938164,100.88423402)(314.69938155,100.84923405)(314.78937988,100.80924123)
\curveto(315.00938124,100.70923419)(315.22438102,100.57423433)(315.43437988,100.40424123)
\curveto(315.6443806,100.24423466)(315.81438043,100.06923483)(315.94437988,99.87924123)
\curveto(315.98438026,99.82923507)(316.02438022,99.76923513)(316.06437988,99.69924123)
\curveto(316.09438015,99.63923526)(316.12938012,99.57923532)(316.16937988,99.51924123)
\curveto(316.21938003,99.43923546)(316.25937999,99.34423556)(316.28937988,99.23424123)
\curveto(316.31937993,99.12423578)(316.3493799,99.01923588)(316.37937988,98.91924123)
\curveto(316.41937983,98.80923609)(316.4443798,98.6992362)(316.45437988,98.58924123)
\curveto(316.46437978,98.47923642)(316.47937977,98.36423654)(316.49937988,98.24424123)
\curveto(316.50937974,98.2042367)(316.50937974,98.15923674)(316.49937988,98.10924123)
\curveto(316.49937975,98.06923683)(316.50437974,98.02923687)(316.51437988,97.98924123)
\curveto(316.52437972,97.94923695)(316.52937972,97.89423701)(316.52937988,97.82424123)
\curveto(316.52937972,97.75423715)(316.52437972,97.7042372)(316.51437988,97.67424123)
\curveto(316.49437975,97.62423728)(316.48937976,97.57923732)(316.49937988,97.53924123)
\curveto(316.50937974,97.4992374)(316.50937974,97.46423744)(316.49937988,97.43424123)
\lineto(316.49937988,97.34424123)
\curveto(316.47937977,97.28423762)(316.46437978,97.21923768)(316.45437988,97.14924123)
\curveto(316.45437979,97.08923781)(316.4493798,97.02423788)(316.43937988,96.95424123)
\curveto(316.38937986,96.78423812)(316.33937991,96.62423828)(316.28937988,96.47424123)
\curveto(316.23938001,96.32423858)(316.17438007,96.17923872)(316.09437988,96.03924123)
\curveto(316.05438019,95.98923891)(316.02438022,95.93423897)(316.00437988,95.87424123)
\curveto(315.97438027,95.82423908)(315.93938031,95.77423913)(315.89937988,95.72424123)
\curveto(315.71938053,95.48423942)(315.49938075,95.28423962)(315.23937988,95.12424123)
\curveto(314.97938127,94.96423994)(314.69438155,94.82424008)(314.38437988,94.70424123)
\curveto(314.244382,94.64424026)(314.10438214,94.5992403)(313.96437988,94.56924123)
\curveto(313.81438243,94.53924036)(313.65938259,94.5042404)(313.49937988,94.46424123)
\curveto(313.38938286,94.44424046)(313.27938297,94.42924047)(313.16937988,94.41924123)
\curveto(313.05938319,94.40924049)(312.9493833,94.39424051)(312.83937988,94.37424123)
\curveto(312.79938345,94.36424054)(312.75938349,94.35924054)(312.71937988,94.35924123)
\curveto(312.67938357,94.36924053)(312.63938361,94.36924053)(312.59937988,94.35924123)
\curveto(312.5493837,94.34924055)(312.49938375,94.34424056)(312.44937988,94.34424123)
\lineto(312.28437988,94.34424123)
\curveto(312.23438401,94.32424058)(312.18438406,94.31924058)(312.13437988,94.32924123)
\curveto(312.07438417,94.33924056)(312.01938423,94.33924056)(311.96937988,94.32924123)
\curveto(311.92938432,94.31924058)(311.88438436,94.31924058)(311.83437988,94.32924123)
\curveto(311.78438446,94.33924056)(311.73438451,94.33424057)(311.68437988,94.31424123)
\curveto(311.61438463,94.29424061)(311.53938471,94.28924061)(311.45937988,94.29924123)
\curveto(311.36938488,94.30924059)(311.28438496,94.31424059)(311.20437988,94.31424123)
\curveto(311.11438513,94.31424059)(311.01438523,94.30924059)(310.90437988,94.29924123)
\curveto(310.78438546,94.28924061)(310.68438556,94.29424061)(310.60437988,94.31424123)
\lineto(310.31937988,94.31424123)
\lineto(309.68937988,94.35924123)
\curveto(309.58938666,94.36924053)(309.49438675,94.37924052)(309.40437988,94.38924123)
\lineto(309.10437988,94.41924123)
\curveto(309.05438719,94.43924046)(309.00438724,94.44424046)(308.95437988,94.43424123)
\curveto(308.89438735,94.43424047)(308.83938741,94.44424046)(308.78937988,94.46424123)
\curveto(308.61938763,94.51424039)(308.45438779,94.55424035)(308.29437988,94.58424123)
\curveto(308.12438812,94.61424029)(307.96438828,94.66424024)(307.81437988,94.73424123)
\curveto(307.35438889,94.92423998)(306.97938927,95.14423976)(306.68937988,95.39424123)
\curveto(306.39938985,95.65423925)(306.15439009,96.01423889)(305.95437988,96.47424123)
\curveto(305.90439034,96.6042383)(305.86939038,96.73423817)(305.84937988,96.86424123)
\curveto(305.82939042,97.0042379)(305.80439044,97.14423776)(305.77437988,97.28424123)
\curveto(305.76439048,97.35423755)(305.75939049,97.41923748)(305.75937988,97.47924123)
\curveto(305.75939049,97.53923736)(305.75439049,97.6042373)(305.74437988,97.67424123)
\curveto(305.72439052,98.5042364)(305.87439037,99.17423573)(306.19437988,99.68424123)
\curveto(306.50438974,100.19423471)(306.9443893,100.57423433)(307.51437988,100.82424123)
\curveto(307.63438861,100.87423403)(307.75938849,100.91923398)(307.88937988,100.95924123)
\curveto(308.01938823,100.9992339)(308.15438809,101.04423386)(308.29437988,101.09424123)
\curveto(308.37438787,101.11423379)(308.45938779,101.12923377)(308.54937988,101.13924123)
\lineto(308.78937988,101.19924123)
\curveto(308.89938735,101.22923367)(309.00938724,101.24423366)(309.11937988,101.24424123)
\curveto(309.22938702,101.25423365)(309.33938691,101.26923363)(309.44937988,101.28924123)
\curveto(309.49938675,101.30923359)(309.5443867,101.31423359)(309.58437988,101.30424123)
\curveto(309.62438662,101.3042336)(309.66438658,101.30923359)(309.70437988,101.31924123)
\curveto(309.75438649,101.32923357)(309.80938644,101.32923357)(309.86937988,101.31924123)
\curveto(309.91938633,101.31923358)(309.96938628,101.32423358)(310.01937988,101.33424123)
\lineto(310.15437988,101.33424123)
\curveto(310.21438603,101.35423355)(310.28438596,101.35423355)(310.36437988,101.33424123)
\curveto(310.43438581,101.32423358)(310.49938575,101.32923357)(310.55937988,101.34924123)
\curveto(310.58938566,101.35923354)(310.62938562,101.36423354)(310.67937988,101.36424123)
\lineto(310.79937988,101.36424123)
\lineto(311.26437988,101.36424123)
\moveto(313.58937988,99.81924123)
\curveto(313.26938298,99.91923498)(312.90438334,99.97923492)(312.49437988,99.99924123)
\curveto(312.08438416,100.01923488)(311.67438457,100.02923487)(311.26437988,100.02924123)
\curveto(310.83438541,100.02923487)(310.41438583,100.01923488)(310.00437988,99.99924123)
\curveto(309.59438665,99.97923492)(309.20938704,99.93423497)(308.84937988,99.86424123)
\curveto(308.48938776,99.79423511)(308.16938808,99.68423522)(307.88937988,99.53424123)
\curveto(307.59938865,99.39423551)(307.36438888,99.1992357)(307.18437988,98.94924123)
\curveto(307.07438917,98.78923611)(306.99438925,98.60923629)(306.94437988,98.40924123)
\curveto(306.88438936,98.20923669)(306.85438939,97.96423694)(306.85437988,97.67424123)
\curveto(306.87438937,97.65423725)(306.88438936,97.61923728)(306.88437988,97.56924123)
\curveto(306.87438937,97.51923738)(306.87438937,97.47923742)(306.88437988,97.44924123)
\curveto(306.90438934,97.36923753)(306.92438932,97.29423761)(306.94437988,97.22424123)
\curveto(306.95438929,97.16423774)(306.97438927,97.0992378)(307.00437988,97.02924123)
\curveto(307.12438912,96.75923814)(307.29438895,96.53923836)(307.51437988,96.36924123)
\curveto(307.72438852,96.20923869)(307.96938828,96.07423883)(308.24937988,95.96424123)
\curveto(308.35938789,95.91423899)(308.47938777,95.87423903)(308.60937988,95.84424123)
\curveto(308.72938752,95.82423908)(308.85438739,95.7992391)(308.98437988,95.76924123)
\curveto(309.03438721,95.74923915)(309.08938716,95.73923916)(309.14937988,95.73924123)
\curveto(309.19938705,95.73923916)(309.249387,95.73423917)(309.29937988,95.72424123)
\curveto(309.38938686,95.71423919)(309.48438676,95.7042392)(309.58437988,95.69424123)
\curveto(309.67438657,95.68423922)(309.76938648,95.67423923)(309.86937988,95.66424123)
\curveto(309.9493863,95.66423924)(310.03438621,95.65923924)(310.12437988,95.64924123)
\lineto(310.36437988,95.64924123)
\lineto(310.54437988,95.64924123)
\curveto(310.57438567,95.63923926)(310.60938564,95.63423927)(310.64937988,95.63424123)
\lineto(310.78437988,95.63424123)
\lineto(311.23437988,95.63424123)
\curveto(311.31438493,95.63423927)(311.39938485,95.62923927)(311.48937988,95.61924123)
\curveto(311.56938468,95.61923928)(311.6443846,95.62923927)(311.71437988,95.64924123)
\lineto(311.98437988,95.64924123)
\curveto(312.00438424,95.64923925)(312.03438421,95.64423926)(312.07437988,95.63424123)
\curveto(312.10438414,95.63423927)(312.12938412,95.63923926)(312.14937988,95.64924123)
\curveto(312.249384,95.65923924)(312.3493839,95.66423924)(312.44937988,95.66424123)
\curveto(312.53938371,95.67423923)(312.63938361,95.68423922)(312.74937988,95.69424123)
\curveto(312.86938338,95.72423918)(312.99438325,95.73923916)(313.12437988,95.73924123)
\curveto(313.244383,95.74923915)(313.35938289,95.77423913)(313.46937988,95.81424123)
\curveto(313.76938248,95.89423901)(314.03438221,95.97923892)(314.26437988,96.06924123)
\curveto(314.49438175,96.16923873)(314.70938154,96.31423859)(314.90937988,96.50424123)
\curveto(315.10938114,96.71423819)(315.25938099,96.97923792)(315.35937988,97.29924123)
\curveto(315.37938087,97.33923756)(315.38938086,97.37423753)(315.38937988,97.40424123)
\curveto(315.37938087,97.44423746)(315.38438086,97.48923741)(315.40437988,97.53924123)
\curveto(315.41438083,97.57923732)(315.42438082,97.64923725)(315.43437988,97.74924123)
\curveto(315.4443808,97.85923704)(315.43938081,97.94423696)(315.41937988,98.00424123)
\curveto(315.39938085,98.07423683)(315.38938086,98.14423676)(315.38937988,98.21424123)
\curveto(315.37938087,98.28423662)(315.36438088,98.34923655)(315.34437988,98.40924123)
\curveto(315.28438096,98.60923629)(315.19938105,98.78923611)(315.08937988,98.94924123)
\curveto(315.06938118,98.97923592)(315.0493812,99.0042359)(315.02937988,99.02424123)
\lineto(314.96937988,99.08424123)
\curveto(314.9493813,99.12423578)(314.90938134,99.17423573)(314.84937988,99.23424123)
\curveto(314.70938154,99.33423557)(314.57938167,99.41923548)(314.45937988,99.48924123)
\curveto(314.33938191,99.55923534)(314.19438205,99.62923527)(314.02437988,99.69924123)
\curveto(313.95438229,99.72923517)(313.88438236,99.74923515)(313.81437988,99.75924123)
\curveto(313.7443825,99.77923512)(313.66938258,99.7992351)(313.58937988,99.81924123)
}
}
{
\newrgbcolor{curcolor}{0 0 0}
\pscustom[linestyle=none,fillstyle=solid,fillcolor=curcolor]
{
\newpath
\moveto(305.74437988,106.77385061)
\curveto(305.7443905,106.87384575)(305.75439049,106.96884566)(305.77437988,107.05885061)
\curveto(305.78439046,107.14884548)(305.81439043,107.21384541)(305.86437988,107.25385061)
\curveto(305.9443903,107.31384531)(306.0493902,107.34384528)(306.17937988,107.34385061)
\lineto(306.56937988,107.34385061)
\lineto(308.06937988,107.34385061)
\lineto(314.45937988,107.34385061)
\lineto(315.62937988,107.34385061)
\lineto(315.94437988,107.34385061)
\curveto(316.0443802,107.35384527)(316.12438012,107.33884529)(316.18437988,107.29885061)
\curveto(316.26437998,107.24884538)(316.31437993,107.17384545)(316.33437988,107.07385061)
\curveto(316.3443799,106.98384564)(316.3493799,106.87384575)(316.34937988,106.74385061)
\lineto(316.34937988,106.51885061)
\curveto(316.32937992,106.43884619)(316.31437993,106.36884626)(316.30437988,106.30885061)
\curveto(316.28437996,106.24884638)(316.24438,106.19884643)(316.18437988,106.15885061)
\curveto(316.12438012,106.11884651)(316.0493802,106.09884653)(315.95937988,106.09885061)
\lineto(315.65937988,106.09885061)
\lineto(314.56437988,106.09885061)
\lineto(309.22437988,106.09885061)
\curveto(309.13438711,106.07884655)(309.05938719,106.06384656)(308.99937988,106.05385061)
\curveto(308.92938732,106.05384657)(308.86938738,106.0238466)(308.81937988,105.96385061)
\curveto(308.76938748,105.89384673)(308.7443875,105.80384682)(308.74437988,105.69385061)
\curveto(308.73438751,105.59384703)(308.72938752,105.48384714)(308.72937988,105.36385061)
\lineto(308.72937988,104.22385061)
\lineto(308.72937988,103.72885061)
\curveto(308.71938753,103.56884906)(308.65938759,103.45884917)(308.54937988,103.39885061)
\curveto(308.51938773,103.37884925)(308.48938776,103.36884926)(308.45937988,103.36885061)
\curveto(308.41938783,103.36884926)(308.37438787,103.36384926)(308.32437988,103.35385061)
\curveto(308.20438804,103.33384929)(308.09438815,103.33884929)(307.99437988,103.36885061)
\curveto(307.89438835,103.40884922)(307.82438842,103.46384916)(307.78437988,103.53385061)
\curveto(307.73438851,103.61384901)(307.70938854,103.73384889)(307.70937988,103.89385061)
\curveto(307.70938854,104.05384857)(307.69438855,104.18884844)(307.66437988,104.29885061)
\curveto(307.65438859,104.34884828)(307.6493886,104.40384822)(307.64937988,104.46385061)
\curveto(307.63938861,104.5238481)(307.62438862,104.58384804)(307.60437988,104.64385061)
\curveto(307.55438869,104.79384783)(307.50438874,104.93884769)(307.45437988,105.07885061)
\curveto(307.39438885,105.21884741)(307.32438892,105.35384727)(307.24437988,105.48385061)
\curveto(307.15438909,105.623847)(307.0493892,105.74384688)(306.92937988,105.84385061)
\curveto(306.80938944,105.94384668)(306.67938957,106.03884659)(306.53937988,106.12885061)
\curveto(306.43938981,106.18884644)(306.32938992,106.23384639)(306.20937988,106.26385061)
\curveto(306.08939016,106.30384632)(305.98439026,106.35384627)(305.89437988,106.41385061)
\curveto(305.83439041,106.46384616)(305.79439045,106.53384609)(305.77437988,106.62385061)
\curveto(305.76439048,106.64384598)(305.75939049,106.66884596)(305.75937988,106.69885061)
\curveto(305.75939049,106.7288459)(305.75439049,106.75384587)(305.74437988,106.77385061)
}
}
{
\newrgbcolor{curcolor}{0 0 0}
\pscustom[linestyle=none,fillstyle=solid,fillcolor=curcolor]
{
\newpath
\moveto(305.74437988,115.12345998)
\curveto(305.7443905,115.22345513)(305.75439049,115.31845503)(305.77437988,115.40845998)
\curveto(305.78439046,115.49845485)(305.81439043,115.56345479)(305.86437988,115.60345998)
\curveto(305.9443903,115.66345469)(306.0493902,115.69345466)(306.17937988,115.69345998)
\lineto(306.56937988,115.69345998)
\lineto(308.06937988,115.69345998)
\lineto(314.45937988,115.69345998)
\lineto(315.62937988,115.69345998)
\lineto(315.94437988,115.69345998)
\curveto(316.0443802,115.70345465)(316.12438012,115.68845466)(316.18437988,115.64845998)
\curveto(316.26437998,115.59845475)(316.31437993,115.52345483)(316.33437988,115.42345998)
\curveto(316.3443799,115.33345502)(316.3493799,115.22345513)(316.34937988,115.09345998)
\lineto(316.34937988,114.86845998)
\curveto(316.32937992,114.78845556)(316.31437993,114.71845563)(316.30437988,114.65845998)
\curveto(316.28437996,114.59845575)(316.24438,114.5484558)(316.18437988,114.50845998)
\curveto(316.12438012,114.46845588)(316.0493802,114.4484559)(315.95937988,114.44845998)
\lineto(315.65937988,114.44845998)
\lineto(314.56437988,114.44845998)
\lineto(309.22437988,114.44845998)
\curveto(309.13438711,114.42845592)(309.05938719,114.41345594)(308.99937988,114.40345998)
\curveto(308.92938732,114.40345595)(308.86938738,114.37345598)(308.81937988,114.31345998)
\curveto(308.76938748,114.24345611)(308.7443875,114.1534562)(308.74437988,114.04345998)
\curveto(308.73438751,113.94345641)(308.72938752,113.83345652)(308.72937988,113.71345998)
\lineto(308.72937988,112.57345998)
\lineto(308.72937988,112.07845998)
\curveto(308.71938753,111.91845843)(308.65938759,111.80845854)(308.54937988,111.74845998)
\curveto(308.51938773,111.72845862)(308.48938776,111.71845863)(308.45937988,111.71845998)
\curveto(308.41938783,111.71845863)(308.37438787,111.71345864)(308.32437988,111.70345998)
\curveto(308.20438804,111.68345867)(308.09438815,111.68845866)(307.99437988,111.71845998)
\curveto(307.89438835,111.75845859)(307.82438842,111.81345854)(307.78437988,111.88345998)
\curveto(307.73438851,111.96345839)(307.70938854,112.08345827)(307.70937988,112.24345998)
\curveto(307.70938854,112.40345795)(307.69438855,112.53845781)(307.66437988,112.64845998)
\curveto(307.65438859,112.69845765)(307.6493886,112.7534576)(307.64937988,112.81345998)
\curveto(307.63938861,112.87345748)(307.62438862,112.93345742)(307.60437988,112.99345998)
\curveto(307.55438869,113.14345721)(307.50438874,113.28845706)(307.45437988,113.42845998)
\curveto(307.39438885,113.56845678)(307.32438892,113.70345665)(307.24437988,113.83345998)
\curveto(307.15438909,113.97345638)(307.0493892,114.09345626)(306.92937988,114.19345998)
\curveto(306.80938944,114.29345606)(306.67938957,114.38845596)(306.53937988,114.47845998)
\curveto(306.43938981,114.53845581)(306.32938992,114.58345577)(306.20937988,114.61345998)
\curveto(306.08939016,114.6534557)(305.98439026,114.70345565)(305.89437988,114.76345998)
\curveto(305.83439041,114.81345554)(305.79439045,114.88345547)(305.77437988,114.97345998)
\curveto(305.76439048,114.99345536)(305.75939049,115.01845533)(305.75937988,115.04845998)
\curveto(305.75939049,115.07845527)(305.75439049,115.10345525)(305.74437988,115.12345998)
}
}
{
\newrgbcolor{curcolor}{0 0 0}
\pscustom[linestyle=none,fillstyle=solid,fillcolor=curcolor]
{
\newpath
\moveto(336.48072632,42.02236623)
\curveto(336.53072706,42.04235669)(336.590727,42.06735666)(336.66072632,42.09736623)
\curveto(336.73072686,42.1273566)(336.80572679,42.14735658)(336.88572632,42.15736623)
\curveto(336.95572664,42.17735655)(337.02572657,42.17735655)(337.09572632,42.15736623)
\curveto(337.15572644,42.14735658)(337.20072639,42.10735662)(337.23072632,42.03736623)
\curveto(337.25072634,41.98735674)(337.26072633,41.9273568)(337.26072632,41.85736623)
\lineto(337.26072632,41.64736623)
\lineto(337.26072632,41.19736623)
\curveto(337.26072633,41.04735768)(337.23572636,40.9273578)(337.18572632,40.83736623)
\curveto(337.12572647,40.73735799)(337.02072657,40.66235807)(336.87072632,40.61236623)
\curveto(336.72072687,40.57235816)(336.58572701,40.5273582)(336.46572632,40.47736623)
\curveto(336.20572739,40.36735836)(335.93572766,40.26735846)(335.65572632,40.17736623)
\curveto(335.37572822,40.08735864)(335.10072849,39.98735874)(334.83072632,39.87736623)
\curveto(334.74072885,39.84735888)(334.65572894,39.81735891)(334.57572632,39.78736623)
\curveto(334.4957291,39.76735896)(334.42072917,39.73735899)(334.35072632,39.69736623)
\curveto(334.28072931,39.66735906)(334.22072937,39.62235911)(334.17072632,39.56236623)
\curveto(334.12072947,39.50235923)(334.08072951,39.42235931)(334.05072632,39.32236623)
\curveto(334.03072956,39.27235946)(334.02572957,39.21235952)(334.03572632,39.14236623)
\lineto(334.03572632,38.94736623)
\lineto(334.03572632,36.11236623)
\lineto(334.03572632,35.81236623)
\curveto(334.02572957,35.70236303)(334.02572957,35.59736313)(334.03572632,35.49736623)
\curveto(334.04572955,35.39736333)(334.06072953,35.30236343)(334.08072632,35.21236623)
\curveto(334.10072949,35.1323636)(334.14072945,35.07236366)(334.20072632,35.03236623)
\curveto(334.30072929,34.95236378)(334.41572918,34.89236384)(334.54572632,34.85236623)
\curveto(334.66572893,34.82236391)(334.7907288,34.78236395)(334.92072632,34.73236623)
\curveto(335.15072844,34.6323641)(335.3907282,34.53736419)(335.64072632,34.44736623)
\curveto(335.8907277,34.36736436)(336.13072746,34.27736445)(336.36072632,34.17736623)
\curveto(336.42072717,34.15736457)(336.4907271,34.1323646)(336.57072632,34.10236623)
\curveto(336.64072695,34.08236465)(336.71572688,34.05736467)(336.79572632,34.02736623)
\curveto(336.87572672,33.99736473)(336.95072664,33.96236477)(337.02072632,33.92236623)
\curveto(337.08072651,33.89236484)(337.12572647,33.85736487)(337.15572632,33.81736623)
\curveto(337.21572638,33.73736499)(337.25072634,33.6273651)(337.26072632,33.48736623)
\lineto(337.26072632,33.06736623)
\lineto(337.26072632,32.82736623)
\curveto(337.25072634,32.75736597)(337.22572637,32.69736603)(337.18572632,32.64736623)
\curveto(337.15572644,32.59736613)(337.11072648,32.56736616)(337.05072632,32.55736623)
\curveto(336.9907266,32.55736617)(336.93072666,32.56236617)(336.87072632,32.57236623)
\curveto(336.80072679,32.59236614)(336.73572686,32.61236612)(336.67572632,32.63236623)
\curveto(336.60572699,32.66236607)(336.55572704,32.68736604)(336.52572632,32.70736623)
\curveto(336.20572739,32.84736588)(335.8907277,32.97236576)(335.58072632,33.08236623)
\curveto(335.26072833,33.19236554)(334.94072865,33.31236542)(334.62072632,33.44236623)
\curveto(334.40072919,33.5323652)(334.18572941,33.61736511)(333.97572632,33.69736623)
\curveto(333.75572984,33.77736495)(333.53573006,33.86236487)(333.31572632,33.95236623)
\curveto(332.595731,34.25236448)(331.87073172,34.53736419)(331.14072632,34.80736623)
\curveto(330.40073319,35.07736365)(329.66573393,35.36236337)(328.93572632,35.66236623)
\curveto(328.67573492,35.77236296)(328.41073518,35.87236286)(328.14072632,35.96236623)
\curveto(327.87073572,36.06236267)(327.60573599,36.16736256)(327.34572632,36.27736623)
\curveto(327.23573636,36.3273624)(327.11573648,36.37236236)(326.98572632,36.41236623)
\curveto(326.84573675,36.46236227)(326.74573685,36.5323622)(326.68572632,36.62236623)
\curveto(326.64573695,36.66236207)(326.61573698,36.727362)(326.59572632,36.81736623)
\curveto(326.58573701,36.83736189)(326.58573701,36.85736187)(326.59572632,36.87736623)
\curveto(326.595737,36.90736182)(326.590737,36.9323618)(326.58072632,36.95236623)
\curveto(326.58073701,37.1323616)(326.58073701,37.34236139)(326.58072632,37.58236623)
\curveto(326.57073702,37.82236091)(326.60573699,37.99736073)(326.68572632,38.10736623)
\curveto(326.74573685,38.18736054)(326.84573675,38.24736048)(326.98572632,38.28736623)
\curveto(327.11573648,38.33736039)(327.23573636,38.38736034)(327.34572632,38.43736623)
\curveto(327.57573602,38.53736019)(327.80573579,38.6273601)(328.03572632,38.70736623)
\curveto(328.26573533,38.78735994)(328.4957351,38.87735985)(328.72572632,38.97736623)
\curveto(328.92573467,39.05735967)(329.13073446,39.1323596)(329.34072632,39.20236623)
\curveto(329.55073404,39.28235945)(329.75573384,39.36735936)(329.95572632,39.45736623)
\curveto(330.68573291,39.75735897)(331.42573217,40.04235869)(332.17572632,40.31236623)
\curveto(332.91573068,40.59235814)(333.65072994,40.88735784)(334.38072632,41.19736623)
\curveto(334.47072912,41.23735749)(334.55572904,41.26735746)(334.63572632,41.28736623)
\curveto(334.71572888,41.31735741)(334.80072879,41.34735738)(334.89072632,41.37736623)
\curveto(335.15072844,41.48735724)(335.41572818,41.59235714)(335.68572632,41.69236623)
\curveto(335.95572764,41.80235693)(336.22072737,41.91235682)(336.48072632,42.02236623)
\moveto(332.83572632,38.81236623)
\curveto(332.80573079,38.90235983)(332.75573084,38.95735977)(332.68572632,38.97736623)
\curveto(332.61573098,39.00735972)(332.54073105,39.01235972)(332.46072632,38.99236623)
\curveto(332.37073122,38.98235975)(332.28573131,38.95735977)(332.20572632,38.91736623)
\curveto(332.11573148,38.88735984)(332.04073155,38.85735987)(331.98072632,38.82736623)
\curveto(331.94073165,38.80735992)(331.90573169,38.79735993)(331.87572632,38.79736623)
\curveto(331.84573175,38.79735993)(331.81073178,38.78735994)(331.77072632,38.76736623)
\lineto(331.53072632,38.67736623)
\curveto(331.44073215,38.65736007)(331.35073224,38.6273601)(331.26072632,38.58736623)
\curveto(330.90073269,38.43736029)(330.53573306,38.30236043)(330.16572632,38.18236623)
\curveto(329.78573381,38.07236066)(329.41573418,37.94236079)(329.05572632,37.79236623)
\curveto(328.94573465,37.74236099)(328.83573476,37.69736103)(328.72572632,37.65736623)
\curveto(328.61573498,37.6273611)(328.51073508,37.58736114)(328.41072632,37.53736623)
\curveto(328.36073523,37.51736121)(328.31573528,37.49236124)(328.27572632,37.46236623)
\curveto(328.22573537,37.44236129)(328.20073539,37.39236134)(328.20072632,37.31236623)
\curveto(328.22073537,37.29236144)(328.23573536,37.27236146)(328.24572632,37.25236623)
\curveto(328.25573534,37.2323615)(328.27073532,37.21236152)(328.29072632,37.19236623)
\curveto(328.34073525,37.15236158)(328.3957352,37.12236161)(328.45572632,37.10236623)
\curveto(328.50573509,37.08236165)(328.56073503,37.06236167)(328.62072632,37.04236623)
\curveto(328.73073486,36.99236174)(328.84073475,36.95236178)(328.95072632,36.92236623)
\curveto(329.06073453,36.89236184)(329.17073442,36.85236188)(329.28072632,36.80236623)
\curveto(329.67073392,36.6323621)(330.06573353,36.48236225)(330.46572632,36.35236623)
\curveto(330.86573273,36.2323625)(331.25573234,36.09236264)(331.63572632,35.93236623)
\lineto(331.78572632,35.87236623)
\curveto(331.83573176,35.86236287)(331.88573171,35.84736288)(331.93572632,35.82736623)
\lineto(332.17572632,35.73736623)
\curveto(332.25573134,35.70736302)(332.33573126,35.68236305)(332.41572632,35.66236623)
\curveto(332.46573113,35.64236309)(332.52073107,35.6323631)(332.58072632,35.63236623)
\curveto(332.64073095,35.64236309)(332.6907309,35.65736307)(332.73072632,35.67736623)
\curveto(332.81073078,35.727363)(332.85573074,35.8323629)(332.86572632,35.99236623)
\lineto(332.86572632,36.44236623)
\lineto(332.86572632,38.04736623)
\curveto(332.86573073,38.15736057)(332.87073072,38.29236044)(332.88072632,38.45236623)
\curveto(332.88073071,38.61236012)(332.86573073,38.73236)(332.83572632,38.81236623)
}
}
{
\newrgbcolor{curcolor}{0 0 0}
\pscustom[linestyle=none,fillstyle=solid,fillcolor=curcolor]
{
\newpath
\moveto(333.22572632,50.56392873)
\curveto(333.27573032,50.57392038)(333.34573025,50.57892038)(333.43572632,50.57892873)
\curveto(333.51573008,50.57892038)(333.58073001,50.57392038)(333.63072632,50.56392873)
\curveto(333.67072992,50.56392039)(333.71072988,50.5589204)(333.75072632,50.54892873)
\lineto(333.87072632,50.54892873)
\curveto(333.95072964,50.52892043)(334.03072956,50.51892044)(334.11072632,50.51892873)
\curveto(334.1907294,50.51892044)(334.27072932,50.50892045)(334.35072632,50.48892873)
\curveto(334.3907292,50.47892048)(334.43072916,50.47392048)(334.47072632,50.47392873)
\curveto(334.50072909,50.47392048)(334.53572906,50.46892049)(334.57572632,50.45892873)
\curveto(334.68572891,50.42892053)(334.7907288,50.39892056)(334.89072632,50.36892873)
\curveto(334.9907286,50.34892061)(335.0907285,50.31892064)(335.19072632,50.27892873)
\curveto(335.54072805,50.13892082)(335.85572774,49.96892099)(336.13572632,49.76892873)
\curveto(336.41572718,49.56892139)(336.65572694,49.31892164)(336.85572632,49.01892873)
\curveto(336.95572664,48.86892209)(337.04072655,48.72392223)(337.11072632,48.58392873)
\curveto(337.16072643,48.47392248)(337.20072639,48.36392259)(337.23072632,48.25392873)
\curveto(337.26072633,48.1539228)(337.2907263,48.04892291)(337.32072632,47.93892873)
\curveto(337.34072625,47.86892309)(337.35072624,47.80392315)(337.35072632,47.74392873)
\curveto(337.36072623,47.68392327)(337.37572622,47.62392333)(337.39572632,47.56392873)
\lineto(337.39572632,47.41392873)
\curveto(337.41572618,47.36392359)(337.42572617,47.28892367)(337.42572632,47.18892873)
\curveto(337.43572616,47.08892387)(337.43072616,47.00892395)(337.41072632,46.94892873)
\lineto(337.41072632,46.79892873)
\curveto(337.40072619,46.7589242)(337.3957262,46.71392424)(337.39572632,46.66392873)
\curveto(337.3957262,46.62392433)(337.3907262,46.57892438)(337.38072632,46.52892873)
\curveto(337.34072625,46.37892458)(337.30572629,46.22892473)(337.27572632,46.07892873)
\curveto(337.24572635,45.93892502)(337.20072639,45.79892516)(337.14072632,45.65892873)
\curveto(337.06072653,45.4589255)(336.96072663,45.27892568)(336.84072632,45.11892873)
\lineto(336.69072632,44.93892873)
\curveto(336.63072696,44.87892608)(336.590727,44.80892615)(336.57072632,44.72892873)
\curveto(336.56072703,44.66892629)(336.57572702,44.61892634)(336.61572632,44.57892873)
\curveto(336.64572695,44.54892641)(336.6907269,44.52392643)(336.75072632,44.50392873)
\curveto(336.81072678,44.49392646)(336.87572672,44.48392647)(336.94572632,44.47392873)
\curveto(337.00572659,44.47392648)(337.05072654,44.46392649)(337.08072632,44.44392873)
\curveto(337.13072646,44.40392655)(337.17572642,44.3589266)(337.21572632,44.30892873)
\curveto(337.23572636,44.2589267)(337.25072634,44.18892677)(337.26072632,44.09892873)
\lineto(337.26072632,43.82892873)
\curveto(337.26072633,43.73892722)(337.25572634,43.6539273)(337.24572632,43.57392873)
\curveto(337.22572637,43.49392746)(337.20572639,43.43392752)(337.18572632,43.39392873)
\curveto(337.16572643,43.37392758)(337.14072645,43.3539276)(337.11072632,43.33392873)
\lineto(337.02072632,43.27392873)
\curveto(336.94072665,43.24392771)(336.82072677,43.22892773)(336.66072632,43.22892873)
\curveto(336.50072709,43.23892772)(336.36572723,43.24392771)(336.25572632,43.24392873)
\lineto(327.45072632,43.24392873)
\curveto(327.33073626,43.24392771)(327.20573639,43.23892772)(327.07572632,43.22892873)
\curveto(326.93573666,43.22892773)(326.82573677,43.2539277)(326.74572632,43.30392873)
\curveto(326.68573691,43.34392761)(326.63573696,43.40892755)(326.59572632,43.49892873)
\curveto(326.595737,43.51892744)(326.595737,43.54392741)(326.59572632,43.57392873)
\curveto(326.58573701,43.60392735)(326.58073701,43.62892733)(326.58072632,43.64892873)
\curveto(326.57073702,43.78892717)(326.57073702,43.93392702)(326.58072632,44.08392873)
\curveto(326.58073701,44.24392671)(326.62073697,44.3539266)(326.70072632,44.41392873)
\curveto(326.78073681,44.46392649)(326.8957367,44.48892647)(327.04572632,44.48892873)
\lineto(327.45072632,44.48892873)
\lineto(329.20572632,44.48892873)
\lineto(329.46072632,44.48892873)
\lineto(329.74572632,44.48892873)
\curveto(329.83573376,44.49892646)(329.92073367,44.50892645)(330.00072632,44.51892873)
\curveto(330.07073352,44.53892642)(330.12073347,44.56892639)(330.15072632,44.60892873)
\curveto(330.18073341,44.64892631)(330.18573341,44.69392626)(330.16572632,44.74392873)
\curveto(330.14573345,44.79392616)(330.12573347,44.83392612)(330.10572632,44.86392873)
\curveto(330.06573353,44.91392604)(330.02573357,44.958926)(329.98572632,44.99892873)
\lineto(329.86572632,45.14892873)
\curveto(329.81573378,45.21892574)(329.77073382,45.28892567)(329.73072632,45.35892873)
\lineto(329.61072632,45.59892873)
\curveto(329.52073407,45.77892518)(329.45573414,45.99392496)(329.41572632,46.24392873)
\curveto(329.37573422,46.49392446)(329.35573424,46.74892421)(329.35572632,47.00892873)
\curveto(329.35573424,47.26892369)(329.38073421,47.52392343)(329.43072632,47.77392873)
\curveto(329.47073412,48.02392293)(329.53073406,48.24392271)(329.61072632,48.43392873)
\curveto(329.78073381,48.83392212)(330.01573358,49.17892178)(330.31572632,49.46892873)
\curveto(330.61573298,49.7589212)(330.96573263,49.98892097)(331.36572632,50.15892873)
\curveto(331.47573212,50.20892075)(331.58573201,50.24892071)(331.69572632,50.27892873)
\curveto(331.7957318,50.31892064)(331.90073169,50.3589206)(332.01072632,50.39892873)
\curveto(332.12073147,50.42892053)(332.23573136,50.44892051)(332.35572632,50.45892873)
\lineto(332.68572632,50.51892873)
\curveto(332.71573088,50.52892043)(332.75073084,50.53392042)(332.79072632,50.53392873)
\curveto(332.82073077,50.53392042)(332.85073074,50.53892042)(332.88072632,50.54892873)
\curveto(332.94073065,50.56892039)(333.00073059,50.56892039)(333.06072632,50.54892873)
\curveto(333.11073048,50.53892042)(333.16573043,50.54392041)(333.22572632,50.56392873)
\moveto(333.61572632,49.22892873)
\curveto(333.56573003,49.24892171)(333.50573009,49.2539217)(333.43572632,49.24392873)
\curveto(333.36573023,49.23392172)(333.30073029,49.22892173)(333.24072632,49.22892873)
\curveto(333.07073052,49.22892173)(332.91073068,49.21892174)(332.76072632,49.19892873)
\curveto(332.61073098,49.18892177)(332.47573112,49.1589218)(332.35572632,49.10892873)
\curveto(332.25573134,49.07892188)(332.16573143,49.0539219)(332.08572632,49.03392873)
\curveto(332.00573159,49.01392194)(331.92573167,48.98392197)(331.84572632,48.94392873)
\curveto(331.595732,48.83392212)(331.36573223,48.68392227)(331.15572632,48.49392873)
\curveto(330.93573266,48.30392265)(330.77073282,48.08392287)(330.66072632,47.83392873)
\curveto(330.63073296,47.7539232)(330.60573299,47.67392328)(330.58572632,47.59392873)
\curveto(330.55573304,47.52392343)(330.53073306,47.44892351)(330.51072632,47.36892873)
\curveto(330.48073311,47.2589237)(330.46573313,47.14892381)(330.46572632,47.03892873)
\curveto(330.45573314,46.92892403)(330.45073314,46.80892415)(330.45072632,46.67892873)
\curveto(330.46073313,46.62892433)(330.47073312,46.58392437)(330.48072632,46.54392873)
\lineto(330.48072632,46.40892873)
\lineto(330.54072632,46.13892873)
\curveto(330.56073303,46.0589249)(330.590733,45.97892498)(330.63072632,45.89892873)
\curveto(330.77073282,45.5589254)(330.98073261,45.28892567)(331.26072632,45.08892873)
\curveto(331.53073206,44.88892607)(331.85073174,44.72892623)(332.22072632,44.60892873)
\curveto(332.33073126,44.56892639)(332.44073115,44.54392641)(332.55072632,44.53392873)
\curveto(332.66073093,44.52392643)(332.77573082,44.50392645)(332.89572632,44.47392873)
\curveto(332.94573065,44.46392649)(332.9907306,44.46392649)(333.03072632,44.47392873)
\curveto(333.07073052,44.48392647)(333.11573048,44.47892648)(333.16572632,44.45892873)
\curveto(333.21573038,44.44892651)(333.2907303,44.44392651)(333.39072632,44.44392873)
\curveto(333.48073011,44.44392651)(333.55073004,44.44892651)(333.60072632,44.45892873)
\lineto(333.72072632,44.45892873)
\curveto(333.76072983,44.46892649)(333.80072979,44.47392648)(333.84072632,44.47392873)
\curveto(333.88072971,44.47392648)(333.91572968,44.47892648)(333.94572632,44.48892873)
\curveto(333.97572962,44.49892646)(334.01072958,44.50392645)(334.05072632,44.50392873)
\curveto(334.08072951,44.50392645)(334.11072948,44.50892645)(334.14072632,44.51892873)
\curveto(334.22072937,44.53892642)(334.30072929,44.5539264)(334.38072632,44.56392873)
\lineto(334.62072632,44.62392873)
\curveto(334.96072863,44.73392622)(335.25072834,44.88392607)(335.49072632,45.07392873)
\curveto(335.73072786,45.27392568)(335.93072766,45.51892544)(336.09072632,45.80892873)
\curveto(336.14072745,45.89892506)(336.18072741,45.99392496)(336.21072632,46.09392873)
\curveto(336.23072736,46.19392476)(336.25572734,46.29892466)(336.28572632,46.40892873)
\curveto(336.30572729,46.4589245)(336.31572728,46.50392445)(336.31572632,46.54392873)
\curveto(336.30572729,46.59392436)(336.30572729,46.64392431)(336.31572632,46.69392873)
\curveto(336.32572727,46.73392422)(336.33072726,46.77892418)(336.33072632,46.82892873)
\lineto(336.33072632,46.96392873)
\lineto(336.33072632,47.09892873)
\curveto(336.32072727,47.13892382)(336.31572728,47.17392378)(336.31572632,47.20392873)
\curveto(336.31572728,47.23392372)(336.31072728,47.26892369)(336.30072632,47.30892873)
\curveto(336.28072731,47.38892357)(336.26572733,47.46392349)(336.25572632,47.53392873)
\curveto(336.23572736,47.60392335)(336.21072738,47.67892328)(336.18072632,47.75892873)
\curveto(336.05072754,48.06892289)(335.88072771,48.31892264)(335.67072632,48.50892873)
\curveto(335.45072814,48.69892226)(335.18572841,48.8589221)(334.87572632,48.98892873)
\curveto(334.73572886,49.03892192)(334.595729,49.07392188)(334.45572632,49.09392873)
\curveto(334.30572929,49.12392183)(334.15572944,49.1589218)(334.00572632,49.19892873)
\curveto(333.95572964,49.21892174)(333.91072968,49.22392173)(333.87072632,49.21392873)
\curveto(333.82072977,49.21392174)(333.77072982,49.21892174)(333.72072632,49.22892873)
\lineto(333.61572632,49.22892873)
}
}
{
\newrgbcolor{curcolor}{0 0 0}
\pscustom[linestyle=none,fillstyle=solid,fillcolor=curcolor]
{
\newpath
\moveto(329.35572632,55.69017873)
\curveto(329.35573424,55.92017394)(329.41573418,56.05017381)(329.53572632,56.08017873)
\curveto(329.64573395,56.11017375)(329.81073378,56.12517374)(330.03072632,56.12517873)
\lineto(330.31572632,56.12517873)
\curveto(330.40573319,56.12517374)(330.48073311,56.10017376)(330.54072632,56.05017873)
\curveto(330.62073297,55.99017387)(330.66573293,55.90517396)(330.67572632,55.79517873)
\curveto(330.67573292,55.68517418)(330.6907329,55.57517429)(330.72072632,55.46517873)
\curveto(330.75073284,55.32517454)(330.78073281,55.19017467)(330.81072632,55.06017873)
\curveto(330.84073275,54.94017492)(330.88073271,54.82517504)(330.93072632,54.71517873)
\curveto(331.06073253,54.42517544)(331.24073235,54.19017567)(331.47072632,54.01017873)
\curveto(331.6907319,53.83017603)(331.94573165,53.67517619)(332.23572632,53.54517873)
\curveto(332.34573125,53.50517636)(332.46073113,53.47517639)(332.58072632,53.45517873)
\curveto(332.6907309,53.43517643)(332.80573079,53.41017645)(332.92572632,53.38017873)
\curveto(332.97573062,53.37017649)(333.02573057,53.3651765)(333.07572632,53.36517873)
\curveto(333.12573047,53.37517649)(333.17573042,53.37517649)(333.22572632,53.36517873)
\curveto(333.34573025,53.33517653)(333.48573011,53.32017654)(333.64572632,53.32017873)
\curveto(333.7957298,53.33017653)(333.94072965,53.33517653)(334.08072632,53.33517873)
\lineto(335.92572632,53.33517873)
\lineto(336.27072632,53.33517873)
\curveto(336.3907272,53.33517653)(336.50572709,53.33017653)(336.61572632,53.32017873)
\curveto(336.72572687,53.31017655)(336.82072677,53.30517656)(336.90072632,53.30517873)
\curveto(336.98072661,53.31517655)(337.05072654,53.29517657)(337.11072632,53.24517873)
\curveto(337.18072641,53.19517667)(337.22072637,53.11517675)(337.23072632,53.00517873)
\curveto(337.24072635,52.90517696)(337.24572635,52.79517707)(337.24572632,52.67517873)
\lineto(337.24572632,52.40517873)
\curveto(337.22572637,52.35517751)(337.21072638,52.30517756)(337.20072632,52.25517873)
\curveto(337.18072641,52.21517765)(337.15572644,52.18517768)(337.12572632,52.16517873)
\curveto(337.05572654,52.11517775)(336.97072662,52.08517778)(336.87072632,52.07517873)
\lineto(336.54072632,52.07517873)
\lineto(335.38572632,52.07517873)
\lineto(331.23072632,52.07517873)
\lineto(330.19572632,52.07517873)
\lineto(329.89572632,52.07517873)
\curveto(329.7957338,52.08517778)(329.71073388,52.11517775)(329.64072632,52.16517873)
\curveto(329.60073399,52.19517767)(329.57073402,52.24517762)(329.55072632,52.31517873)
\curveto(329.53073406,52.39517747)(329.52073407,52.48017738)(329.52072632,52.57017873)
\curveto(329.51073408,52.6601772)(329.51073408,52.75017711)(329.52072632,52.84017873)
\curveto(329.53073406,52.93017693)(329.54573405,53.00017686)(329.56572632,53.05017873)
\curveto(329.595734,53.13017673)(329.65573394,53.18017668)(329.74572632,53.20017873)
\curveto(329.82573377,53.23017663)(329.91573368,53.24517662)(330.01572632,53.24517873)
\lineto(330.31572632,53.24517873)
\curveto(330.41573318,53.24517662)(330.50573309,53.2651766)(330.58572632,53.30517873)
\curveto(330.60573299,53.31517655)(330.62073297,53.32517654)(330.63072632,53.33517873)
\lineto(330.67572632,53.38017873)
\curveto(330.67573292,53.49017637)(330.63073296,53.58017628)(330.54072632,53.65017873)
\curveto(330.44073315,53.72017614)(330.36073323,53.78017608)(330.30072632,53.83017873)
\lineto(330.21072632,53.92017873)
\curveto(330.10073349,54.01017585)(329.98573361,54.13517573)(329.86572632,54.29517873)
\curveto(329.74573385,54.45517541)(329.65573394,54.60517526)(329.59572632,54.74517873)
\curveto(329.54573405,54.83517503)(329.51073408,54.93017493)(329.49072632,55.03017873)
\curveto(329.46073413,55.13017473)(329.43073416,55.23517463)(329.40072632,55.34517873)
\curveto(329.3907342,55.40517446)(329.38573421,55.4651744)(329.38572632,55.52517873)
\curveto(329.37573422,55.58517428)(329.36573423,55.64017422)(329.35572632,55.69017873)
}
}
{
\newrgbcolor{curcolor}{0 0 0}
\pscustom[linestyle=none,fillstyle=solid,fillcolor=curcolor]
{
}
}
{
\newrgbcolor{curcolor}{0 0 0}
\pscustom[linestyle=none,fillstyle=solid,fillcolor=curcolor]
{
\newpath
\moveto(332.17572632,67.99510061)
\lineto(332.43072632,67.99510061)
\curveto(332.51073108,68.0050929)(332.58573101,68.00009291)(332.65572632,67.98010061)
\lineto(332.89572632,67.98010061)
\lineto(333.06072632,67.98010061)
\curveto(333.16073043,67.96009295)(333.26573033,67.95009296)(333.37572632,67.95010061)
\curveto(333.47573012,67.95009296)(333.57573002,67.94009297)(333.67572632,67.92010061)
\lineto(333.82572632,67.92010061)
\curveto(333.96572963,67.89009302)(334.10572949,67.87009304)(334.24572632,67.86010061)
\curveto(334.37572922,67.85009306)(334.50572909,67.82509308)(334.63572632,67.78510061)
\curveto(334.71572888,67.76509314)(334.80072879,67.74509316)(334.89072632,67.72510061)
\lineto(335.13072632,67.66510061)
\lineto(335.43072632,67.54510061)
\curveto(335.52072807,67.51509339)(335.61072798,67.48009343)(335.70072632,67.44010061)
\curveto(335.92072767,67.34009357)(336.13572746,67.2050937)(336.34572632,67.03510061)
\curveto(336.55572704,66.87509403)(336.72572687,66.70009421)(336.85572632,66.51010061)
\curveto(336.8957267,66.46009445)(336.93572666,66.40009451)(336.97572632,66.33010061)
\curveto(337.00572659,66.27009464)(337.04072655,66.2100947)(337.08072632,66.15010061)
\curveto(337.13072646,66.07009484)(337.17072642,65.97509493)(337.20072632,65.86510061)
\curveto(337.23072636,65.75509515)(337.26072633,65.65009526)(337.29072632,65.55010061)
\curveto(337.33072626,65.44009547)(337.35572624,65.33009558)(337.36572632,65.22010061)
\curveto(337.37572622,65.1100958)(337.3907262,64.99509591)(337.41072632,64.87510061)
\curveto(337.42072617,64.83509607)(337.42072617,64.79009612)(337.41072632,64.74010061)
\curveto(337.41072618,64.70009621)(337.41572618,64.66009625)(337.42572632,64.62010061)
\curveto(337.43572616,64.58009633)(337.44072615,64.52509638)(337.44072632,64.45510061)
\curveto(337.44072615,64.38509652)(337.43572616,64.33509657)(337.42572632,64.30510061)
\curveto(337.40572619,64.25509665)(337.40072619,64.2100967)(337.41072632,64.17010061)
\curveto(337.42072617,64.13009678)(337.42072617,64.09509681)(337.41072632,64.06510061)
\lineto(337.41072632,63.97510061)
\curveto(337.3907262,63.91509699)(337.37572622,63.85009706)(337.36572632,63.78010061)
\curveto(337.36572623,63.72009719)(337.36072623,63.65509725)(337.35072632,63.58510061)
\curveto(337.30072629,63.41509749)(337.25072634,63.25509765)(337.20072632,63.10510061)
\curveto(337.15072644,62.95509795)(337.08572651,62.8100981)(337.00572632,62.67010061)
\curveto(336.96572663,62.62009829)(336.93572666,62.56509834)(336.91572632,62.50510061)
\curveto(336.88572671,62.45509845)(336.85072674,62.4050985)(336.81072632,62.35510061)
\curveto(336.63072696,62.11509879)(336.41072718,61.91509899)(336.15072632,61.75510061)
\curveto(335.8907277,61.59509931)(335.60572799,61.45509945)(335.29572632,61.33510061)
\curveto(335.15572844,61.27509963)(335.01572858,61.23009968)(334.87572632,61.20010061)
\curveto(334.72572887,61.17009974)(334.57072902,61.13509977)(334.41072632,61.09510061)
\curveto(334.30072929,61.07509983)(334.1907294,61.06009985)(334.08072632,61.05010061)
\curveto(333.97072962,61.04009987)(333.86072973,61.02509988)(333.75072632,61.00510061)
\curveto(333.71072988,60.99509991)(333.67072992,60.99009992)(333.63072632,60.99010061)
\curveto(333.59073,61.00009991)(333.55073004,61.00009991)(333.51072632,60.99010061)
\curveto(333.46073013,60.98009993)(333.41073018,60.97509993)(333.36072632,60.97510061)
\lineto(333.19572632,60.97510061)
\curveto(333.14573045,60.95509995)(333.0957305,60.95009996)(333.04572632,60.96010061)
\curveto(332.98573061,60.97009994)(332.93073066,60.97009994)(332.88072632,60.96010061)
\curveto(332.84073075,60.95009996)(332.7957308,60.95009996)(332.74572632,60.96010061)
\curveto(332.6957309,60.97009994)(332.64573095,60.96509994)(332.59572632,60.94510061)
\curveto(332.52573107,60.92509998)(332.45073114,60.92009999)(332.37072632,60.93010061)
\curveto(332.28073131,60.94009997)(332.1957314,60.94509996)(332.11572632,60.94510061)
\curveto(332.02573157,60.94509996)(331.92573167,60.94009997)(331.81572632,60.93010061)
\curveto(331.6957319,60.92009999)(331.595732,60.92509998)(331.51572632,60.94510061)
\lineto(331.23072632,60.94510061)
\lineto(330.60072632,60.99010061)
\curveto(330.50073309,61.00009991)(330.40573319,61.0100999)(330.31572632,61.02010061)
\lineto(330.01572632,61.05010061)
\curveto(329.96573363,61.07009984)(329.91573368,61.07509983)(329.86572632,61.06510061)
\curveto(329.80573379,61.06509984)(329.75073384,61.07509983)(329.70072632,61.09510061)
\curveto(329.53073406,61.14509976)(329.36573423,61.18509972)(329.20572632,61.21510061)
\curveto(329.03573456,61.24509966)(328.87573472,61.29509961)(328.72572632,61.36510061)
\curveto(328.26573533,61.55509935)(327.8907357,61.77509913)(327.60072632,62.02510061)
\curveto(327.31073628,62.28509862)(327.06573653,62.64509826)(326.86572632,63.10510061)
\curveto(326.81573678,63.23509767)(326.78073681,63.36509754)(326.76072632,63.49510061)
\curveto(326.74073685,63.63509727)(326.71573688,63.77509713)(326.68572632,63.91510061)
\curveto(326.67573692,63.98509692)(326.67073692,64.05009686)(326.67072632,64.11010061)
\curveto(326.67073692,64.17009674)(326.66573693,64.23509667)(326.65572632,64.30510061)
\curveto(326.63573696,65.13509577)(326.78573681,65.8050951)(327.10572632,66.31510061)
\curveto(327.41573618,66.82509408)(327.85573574,67.2050937)(328.42572632,67.45510061)
\curveto(328.54573505,67.5050934)(328.67073492,67.55009336)(328.80072632,67.59010061)
\curveto(328.93073466,67.63009328)(329.06573453,67.67509323)(329.20572632,67.72510061)
\curveto(329.28573431,67.74509316)(329.37073422,67.76009315)(329.46072632,67.77010061)
\lineto(329.70072632,67.83010061)
\curveto(329.81073378,67.86009305)(329.92073367,67.87509303)(330.03072632,67.87510061)
\curveto(330.14073345,67.88509302)(330.25073334,67.90009301)(330.36072632,67.92010061)
\curveto(330.41073318,67.94009297)(330.45573314,67.94509296)(330.49572632,67.93510061)
\curveto(330.53573306,67.93509297)(330.57573302,67.94009297)(330.61572632,67.95010061)
\curveto(330.66573293,67.96009295)(330.72073287,67.96009295)(330.78072632,67.95010061)
\curveto(330.83073276,67.95009296)(330.88073271,67.95509295)(330.93072632,67.96510061)
\lineto(331.06572632,67.96510061)
\curveto(331.12573247,67.98509292)(331.1957324,67.98509292)(331.27572632,67.96510061)
\curveto(331.34573225,67.95509295)(331.41073218,67.96009295)(331.47072632,67.98010061)
\curveto(331.50073209,67.99009292)(331.54073205,67.99509291)(331.59072632,67.99510061)
\lineto(331.71072632,67.99510061)
\lineto(332.17572632,67.99510061)
\moveto(334.50072632,66.45010061)
\curveto(334.18072941,66.55009436)(333.81572978,66.6100943)(333.40572632,66.63010061)
\curveto(332.9957306,66.65009426)(332.58573101,66.66009425)(332.17572632,66.66010061)
\curveto(331.74573185,66.66009425)(331.32573227,66.65009426)(330.91572632,66.63010061)
\curveto(330.50573309,66.6100943)(330.12073347,66.56509434)(329.76072632,66.49510061)
\curveto(329.40073419,66.42509448)(329.08073451,66.31509459)(328.80072632,66.16510061)
\curveto(328.51073508,66.02509488)(328.27573532,65.83009508)(328.09572632,65.58010061)
\curveto(327.98573561,65.42009549)(327.90573569,65.24009567)(327.85572632,65.04010061)
\curveto(327.7957358,64.84009607)(327.76573583,64.59509631)(327.76572632,64.30510061)
\curveto(327.78573581,64.28509662)(327.7957358,64.25009666)(327.79572632,64.20010061)
\curveto(327.78573581,64.15009676)(327.78573581,64.1100968)(327.79572632,64.08010061)
\curveto(327.81573578,64.00009691)(327.83573576,63.92509698)(327.85572632,63.85510061)
\curveto(327.86573573,63.79509711)(327.88573571,63.73009718)(327.91572632,63.66010061)
\curveto(328.03573556,63.39009752)(328.20573539,63.17009774)(328.42572632,63.00010061)
\curveto(328.63573496,62.84009807)(328.88073471,62.7050982)(329.16072632,62.59510061)
\curveto(329.27073432,62.54509836)(329.3907342,62.5050984)(329.52072632,62.47510061)
\curveto(329.64073395,62.45509845)(329.76573383,62.43009848)(329.89572632,62.40010061)
\curveto(329.94573365,62.38009853)(330.00073359,62.37009854)(330.06072632,62.37010061)
\curveto(330.11073348,62.37009854)(330.16073343,62.36509854)(330.21072632,62.35510061)
\curveto(330.30073329,62.34509856)(330.3957332,62.33509857)(330.49572632,62.32510061)
\curveto(330.58573301,62.31509859)(330.68073291,62.3050986)(330.78072632,62.29510061)
\curveto(330.86073273,62.29509861)(330.94573265,62.29009862)(331.03572632,62.28010061)
\lineto(331.27572632,62.28010061)
\lineto(331.45572632,62.28010061)
\curveto(331.48573211,62.27009864)(331.52073207,62.26509864)(331.56072632,62.26510061)
\lineto(331.69572632,62.26510061)
\lineto(332.14572632,62.26510061)
\curveto(332.22573137,62.26509864)(332.31073128,62.26009865)(332.40072632,62.25010061)
\curveto(332.48073111,62.25009866)(332.55573104,62.26009865)(332.62572632,62.28010061)
\lineto(332.89572632,62.28010061)
\curveto(332.91573068,62.28009863)(332.94573065,62.27509863)(332.98572632,62.26510061)
\curveto(333.01573058,62.26509864)(333.04073055,62.27009864)(333.06072632,62.28010061)
\curveto(333.16073043,62.29009862)(333.26073033,62.29509861)(333.36072632,62.29510061)
\curveto(333.45073014,62.3050986)(333.55073004,62.31509859)(333.66072632,62.32510061)
\curveto(333.78072981,62.35509855)(333.90572969,62.37009854)(334.03572632,62.37010061)
\curveto(334.15572944,62.38009853)(334.27072932,62.4050985)(334.38072632,62.44510061)
\curveto(334.68072891,62.52509838)(334.94572865,62.6100983)(335.17572632,62.70010061)
\curveto(335.40572819,62.80009811)(335.62072797,62.94509796)(335.82072632,63.13510061)
\curveto(336.02072757,63.34509756)(336.17072742,63.6100973)(336.27072632,63.93010061)
\curveto(336.2907273,63.97009694)(336.30072729,64.0050969)(336.30072632,64.03510061)
\curveto(336.2907273,64.07509683)(336.2957273,64.12009679)(336.31572632,64.17010061)
\curveto(336.32572727,64.2100967)(336.33572726,64.28009663)(336.34572632,64.38010061)
\curveto(336.35572724,64.49009642)(336.35072724,64.57509633)(336.33072632,64.63510061)
\curveto(336.31072728,64.7050962)(336.30072729,64.77509613)(336.30072632,64.84510061)
\curveto(336.2907273,64.91509599)(336.27572732,64.98009593)(336.25572632,65.04010061)
\curveto(336.1957274,65.24009567)(336.11072748,65.42009549)(336.00072632,65.58010061)
\curveto(335.98072761,65.6100953)(335.96072763,65.63509527)(335.94072632,65.65510061)
\lineto(335.88072632,65.71510061)
\curveto(335.86072773,65.75509515)(335.82072777,65.8050951)(335.76072632,65.86510061)
\curveto(335.62072797,65.96509494)(335.4907281,66.05009486)(335.37072632,66.12010061)
\curveto(335.25072834,66.19009472)(335.10572849,66.26009465)(334.93572632,66.33010061)
\curveto(334.86572873,66.36009455)(334.7957288,66.38009453)(334.72572632,66.39010061)
\curveto(334.65572894,66.4100945)(334.58072901,66.43009448)(334.50072632,66.45010061)
}
}
{
\newrgbcolor{curcolor}{0 0 0}
\pscustom[linestyle=none,fillstyle=solid,fillcolor=curcolor]
{
\newpath
\moveto(326.65572632,72.45970998)
\curveto(326.62573697,74.08970454)(327.18073641,75.13970349)(328.32072632,75.60970998)
\curveto(328.55073504,75.70970292)(328.84073475,75.77470286)(329.19072632,75.80470998)
\curveto(329.53073406,75.84470279)(329.84073375,75.81970281)(330.12072632,75.72970998)
\curveto(330.38073321,75.63970299)(330.60573299,75.51970311)(330.79572632,75.36970998)
\curveto(330.83573276,75.34970328)(330.87073272,75.32470331)(330.90072632,75.29470998)
\curveto(330.92073267,75.26470337)(330.94573265,75.23970339)(330.97572632,75.21970998)
\lineto(331.09572632,75.12970998)
\curveto(331.12573247,75.09970353)(331.15073244,75.06470357)(331.17072632,75.02470998)
\curveto(331.22073237,74.97470366)(331.26573233,74.91970371)(331.30572632,74.85970998)
\curveto(331.34573225,74.80970382)(331.3957322,74.76470387)(331.45572632,74.72470998)
\curveto(331.4957321,74.68470395)(331.54573205,74.66970396)(331.60572632,74.67970998)
\curveto(331.65573194,74.68970394)(331.70073189,74.71970391)(331.74072632,74.76970998)
\curveto(331.78073181,74.81970381)(331.82073177,74.87470376)(331.86072632,74.93470998)
\curveto(331.8907317,75.00470363)(331.92073167,75.06970356)(331.95072632,75.12970998)
\curveto(331.98073161,75.18970344)(332.01073158,75.23970339)(332.04072632,75.27970998)
\curveto(332.26073133,75.59970303)(332.57073102,75.85470278)(332.97072632,76.04470998)
\curveto(333.06073053,76.08470255)(333.15573044,76.11470252)(333.25572632,76.13470998)
\curveto(333.34573025,76.16470247)(333.43573016,76.18970244)(333.52572632,76.20970998)
\curveto(333.57573002,76.21970241)(333.62572997,76.22470241)(333.67572632,76.22470998)
\curveto(333.71572988,76.2347024)(333.76072983,76.24470239)(333.81072632,76.25470998)
\curveto(333.86072973,76.26470237)(333.91072968,76.26470237)(333.96072632,76.25470998)
\curveto(334.01072958,76.24470239)(334.06072953,76.24970238)(334.11072632,76.26970998)
\curveto(334.16072943,76.27970235)(334.22072937,76.28470235)(334.29072632,76.28470998)
\curveto(334.36072923,76.28470235)(334.42072917,76.27470236)(334.47072632,76.25470998)
\lineto(334.69572632,76.25470998)
\lineto(334.93572632,76.19470998)
\curveto(335.00572859,76.18470245)(335.07572852,76.16970246)(335.14572632,76.14970998)
\curveto(335.23572836,76.11970251)(335.32072827,76.08970254)(335.40072632,76.05970998)
\curveto(335.48072811,76.03970259)(335.56072803,76.00970262)(335.64072632,75.96970998)
\curveto(335.70072789,75.94970268)(335.76072783,75.91970271)(335.82072632,75.87970998)
\curveto(335.87072772,75.84970278)(335.92072767,75.81470282)(335.97072632,75.77470998)
\curveto(336.28072731,75.57470306)(336.54072705,75.32470331)(336.75072632,75.02470998)
\curveto(336.95072664,74.72470391)(337.11572648,74.37970425)(337.24572632,73.98970998)
\curveto(337.28572631,73.86970476)(337.31072628,73.73970489)(337.32072632,73.59970998)
\curveto(337.34072625,73.46970516)(337.36572623,73.3347053)(337.39572632,73.19470998)
\curveto(337.40572619,73.12470551)(337.41072618,73.05470558)(337.41072632,72.98470998)
\curveto(337.41072618,72.92470571)(337.41572618,72.85970577)(337.42572632,72.78970998)
\curveto(337.43572616,72.74970588)(337.44072615,72.68970594)(337.44072632,72.60970998)
\curveto(337.44072615,72.53970609)(337.43572616,72.48970614)(337.42572632,72.45970998)
\curveto(337.41572618,72.40970622)(337.41072618,72.36470627)(337.41072632,72.32470998)
\lineto(337.41072632,72.20470998)
\curveto(337.3907262,72.10470653)(337.37572622,72.00470663)(337.36572632,71.90470998)
\curveto(337.35572624,71.80470683)(337.34072625,71.70970692)(337.32072632,71.61970998)
\curveto(337.2907263,71.50970712)(337.26572633,71.39970723)(337.24572632,71.28970998)
\curveto(337.21572638,71.18970744)(337.17572642,71.08470755)(337.12572632,70.97470998)
\curveto(336.96572663,70.60470803)(336.76572683,70.28970834)(336.52572632,70.02970998)
\curveto(336.27572732,69.76970886)(335.96572763,69.55970907)(335.59572632,69.39970998)
\curveto(335.50572809,69.35970927)(335.41072818,69.32470931)(335.31072632,69.29470998)
\curveto(335.21072838,69.26470937)(335.10572849,69.2347094)(334.99572632,69.20470998)
\curveto(334.94572865,69.18470945)(334.8957287,69.17470946)(334.84572632,69.17470998)
\curveto(334.78572881,69.17470946)(334.72572887,69.16470947)(334.66572632,69.14470998)
\curveto(334.60572899,69.12470951)(334.52572907,69.11470952)(334.42572632,69.11470998)
\curveto(334.32572927,69.11470952)(334.25072934,69.1297095)(334.20072632,69.15970998)
\curveto(334.17072942,69.16970946)(334.14572945,69.18470945)(334.12572632,69.20470998)
\lineto(334.06572632,69.26470998)
\curveto(334.04572955,69.30470933)(334.03072956,69.36470927)(334.02072632,69.44470998)
\curveto(334.01072958,69.5347091)(334.00572959,69.62470901)(334.00572632,69.71470998)
\curveto(334.00572959,69.80470883)(334.01072958,69.88970874)(334.02072632,69.96970998)
\curveto(334.03072956,70.05970857)(334.04072955,70.12470851)(334.05072632,70.16470998)
\curveto(334.07072952,70.18470845)(334.08572951,70.20470843)(334.09572632,70.22470998)
\curveto(334.0957295,70.24470839)(334.10572949,70.26470837)(334.12572632,70.28470998)
\curveto(334.21572938,70.35470828)(334.33072926,70.39470824)(334.47072632,70.40470998)
\curveto(334.61072898,70.42470821)(334.73572886,70.45470818)(334.84572632,70.49470998)
\lineto(335.20572632,70.64470998)
\curveto(335.31572828,70.69470794)(335.42072817,70.75970787)(335.52072632,70.83970998)
\curveto(335.55072804,70.85970777)(335.57572802,70.87970775)(335.59572632,70.89970998)
\curveto(335.61572798,70.9297077)(335.64072795,70.95470768)(335.67072632,70.97470998)
\curveto(335.73072786,71.01470762)(335.77572782,71.04970758)(335.80572632,71.07970998)
\curveto(335.83572776,71.11970751)(335.86572773,71.15470748)(335.89572632,71.18470998)
\curveto(335.92572767,71.22470741)(335.95572764,71.26970736)(335.98572632,71.31970998)
\curveto(336.04572755,71.40970722)(336.0957275,71.50470713)(336.13572632,71.60470998)
\lineto(336.25572632,71.93470998)
\curveto(336.30572729,72.08470655)(336.33572726,72.28470635)(336.34572632,72.53470998)
\curveto(336.35572724,72.78470585)(336.33572726,72.99470564)(336.28572632,73.16470998)
\curveto(336.26572733,73.24470539)(336.25072734,73.31470532)(336.24072632,73.37470998)
\lineto(336.18072632,73.58470998)
\curveto(336.06072753,73.86470477)(335.91072768,74.10470453)(335.73072632,74.30470998)
\curveto(335.55072804,74.51470412)(335.32072827,74.67970395)(335.04072632,74.79970998)
\curveto(334.97072862,74.8297038)(334.90072869,74.84970378)(334.83072632,74.85970998)
\lineto(334.59072632,74.91970998)
\curveto(334.45072914,74.95970367)(334.2907293,74.96970366)(334.11072632,74.94970998)
\curveto(333.92072967,74.9297037)(333.77072982,74.89970373)(333.66072632,74.85970998)
\curveto(333.28073031,74.7297039)(332.9907306,74.54470409)(332.79072632,74.30470998)
\curveto(332.590731,74.07470456)(332.43073116,73.76470487)(332.31072632,73.37470998)
\curveto(332.28073131,73.26470537)(332.26073133,73.14470549)(332.25072632,73.01470998)
\curveto(332.24073135,72.89470574)(332.23573136,72.76970586)(332.23572632,72.63970998)
\curveto(332.23573136,72.47970615)(332.23073136,72.33970629)(332.22072632,72.21970998)
\curveto(332.21073138,72.09970653)(332.15073144,72.01470662)(332.04072632,71.96470998)
\curveto(332.01073158,71.94470669)(331.97573162,71.9347067)(331.93572632,71.93470998)
\lineto(331.80072632,71.93470998)
\curveto(331.70073189,71.92470671)(331.60573199,71.92470671)(331.51572632,71.93470998)
\curveto(331.42573217,71.95470668)(331.36073223,71.99470664)(331.32072632,72.05470998)
\curveto(331.2907323,72.09470654)(331.27073232,72.1347065)(331.26072632,72.17470998)
\curveto(331.25073234,72.22470641)(331.24073235,72.27970635)(331.23072632,72.33970998)
\curveto(331.22073237,72.35970627)(331.22073237,72.38470625)(331.23072632,72.41470998)
\curveto(331.23073236,72.44470619)(331.22573237,72.46970616)(331.21572632,72.48970998)
\lineto(331.21572632,72.62470998)
\curveto(331.1957324,72.7347059)(331.18573241,72.8347058)(331.18572632,72.92470998)
\curveto(331.17573242,73.02470561)(331.15573244,73.11970551)(331.12572632,73.20970998)
\curveto(331.01573258,73.5297051)(330.87073272,73.78470485)(330.69072632,73.97470998)
\curveto(330.51073308,74.16470447)(330.26073333,74.31470432)(329.94072632,74.42470998)
\curveto(329.84073375,74.45470418)(329.71573388,74.47470416)(329.56572632,74.48470998)
\curveto(329.40573419,74.50470413)(329.26073433,74.49970413)(329.13072632,74.46970998)
\curveto(329.06073453,74.44970418)(328.9957346,74.4297042)(328.93572632,74.40970998)
\curveto(328.86573473,74.39970423)(328.80073479,74.37970425)(328.74072632,74.34970998)
\curveto(328.50073509,74.24970438)(328.31073528,74.10470453)(328.17072632,73.91470998)
\curveto(328.03073556,73.72470491)(327.92073567,73.49970513)(327.84072632,73.23970998)
\curveto(327.82073577,73.17970545)(327.81073578,73.11970551)(327.81072632,73.05970998)
\curveto(327.81073578,72.99970563)(327.80073579,72.9347057)(327.78072632,72.86470998)
\curveto(327.76073583,72.78470585)(327.75073584,72.68970594)(327.75072632,72.57970998)
\curveto(327.75073584,72.46970616)(327.76073583,72.37470626)(327.78072632,72.29470998)
\curveto(327.80073579,72.24470639)(327.81073578,72.19470644)(327.81072632,72.14470998)
\curveto(327.81073578,72.10470653)(327.82073577,72.05970657)(327.84072632,72.00970998)
\curveto(327.8907357,71.8297068)(327.96573563,71.65970697)(328.06572632,71.49970998)
\curveto(328.15573544,71.34970728)(328.27073532,71.21970741)(328.41072632,71.10970998)
\curveto(328.53073506,71.01970761)(328.66073493,70.93970769)(328.80072632,70.86970998)
\curveto(328.94073465,70.79970783)(329.0957345,70.7347079)(329.26572632,70.67470998)
\curveto(329.37573422,70.64470799)(329.4957341,70.62470801)(329.62572632,70.61470998)
\curveto(329.74573385,70.60470803)(329.84573375,70.56970806)(329.92572632,70.50970998)
\curveto(329.96573363,70.48970814)(330.00573359,70.4297082)(330.04572632,70.32970998)
\curveto(330.05573354,70.28970834)(330.06573353,70.2297084)(330.07572632,70.14970998)
\lineto(330.07572632,69.89470998)
\curveto(330.06573353,69.80470883)(330.05573354,69.71970891)(330.04572632,69.63970998)
\curveto(330.03573356,69.56970906)(330.02073357,69.51970911)(330.00072632,69.48970998)
\curveto(329.97073362,69.44970918)(329.91573368,69.41470922)(329.83572632,69.38470998)
\curveto(329.75573384,69.35470928)(329.67073392,69.34970928)(329.58072632,69.36970998)
\curveto(329.53073406,69.37970925)(329.48073411,69.38470925)(329.43072632,69.38470998)
\lineto(329.25072632,69.41470998)
\curveto(329.15073444,69.44470919)(329.05073454,69.46970916)(328.95072632,69.48970998)
\curveto(328.85073474,69.51970911)(328.76073483,69.55470908)(328.68072632,69.59470998)
\curveto(328.57073502,69.64470899)(328.46573513,69.68970894)(328.36572632,69.72970998)
\curveto(328.25573534,69.76970886)(328.15073544,69.81970881)(328.05072632,69.87970998)
\curveto(327.51073608,70.20970842)(327.11573648,70.67970795)(326.86572632,71.28970998)
\curveto(326.81573678,71.40970722)(326.78073681,71.5347071)(326.76072632,71.66470998)
\curveto(326.74073685,71.80470683)(326.71573688,71.94470669)(326.68572632,72.08470998)
\curveto(326.67573692,72.14470649)(326.67073692,72.20470643)(326.67072632,72.26470998)
\curveto(326.67073692,72.3347063)(326.66573693,72.39970623)(326.65572632,72.45970998)
}
}
{
\newrgbcolor{curcolor}{0 0 0}
\pscustom[linestyle=none,fillstyle=solid,fillcolor=curcolor]
{
\newpath
\moveto(335.62572632,78.63431936)
\lineto(335.62572632,79.26431936)
\lineto(335.62572632,79.45931936)
\curveto(335.62572797,79.52931683)(335.63572796,79.58931677)(335.65572632,79.63931936)
\curveto(335.6957279,79.70931665)(335.73572786,79.7593166)(335.77572632,79.78931936)
\curveto(335.82572777,79.82931653)(335.8907277,79.84931651)(335.97072632,79.84931936)
\curveto(336.05072754,79.8593165)(336.13572746,79.86431649)(336.22572632,79.86431936)
\lineto(336.94572632,79.86431936)
\curveto(337.42572617,79.86431649)(337.83572576,79.80431655)(338.17572632,79.68431936)
\curveto(338.51572508,79.56431679)(338.7907248,79.36931699)(339.00072632,79.09931936)
\curveto(339.05072454,79.02931733)(339.0957245,78.9593174)(339.13572632,78.88931936)
\curveto(339.18572441,78.82931753)(339.23072436,78.7543176)(339.27072632,78.66431936)
\curveto(339.28072431,78.64431771)(339.2907243,78.61431774)(339.30072632,78.57431936)
\curveto(339.32072427,78.53431782)(339.32572427,78.48931787)(339.31572632,78.43931936)
\curveto(339.28572431,78.34931801)(339.21072438,78.29431806)(339.09072632,78.27431936)
\curveto(338.98072461,78.2543181)(338.88572471,78.26931809)(338.80572632,78.31931936)
\curveto(338.73572486,78.34931801)(338.67072492,78.39431796)(338.61072632,78.45431936)
\curveto(338.56072503,78.52431783)(338.51072508,78.58931777)(338.46072632,78.64931936)
\curveto(338.41072518,78.71931764)(338.33572526,78.77931758)(338.23572632,78.82931936)
\curveto(338.14572545,78.88931747)(338.05572554,78.93931742)(337.96572632,78.97931936)
\curveto(337.93572566,78.99931736)(337.87572572,79.02431733)(337.78572632,79.05431936)
\curveto(337.70572589,79.08431727)(337.63572596,79.08931727)(337.57572632,79.06931936)
\curveto(337.43572616,79.03931732)(337.34572625,78.97931738)(337.30572632,78.88931936)
\curveto(337.27572632,78.80931755)(337.26072633,78.71931764)(337.26072632,78.61931936)
\curveto(337.26072633,78.51931784)(337.23572636,78.43431792)(337.18572632,78.36431936)
\curveto(337.11572648,78.27431808)(336.97572662,78.22931813)(336.76572632,78.22931936)
\lineto(336.21072632,78.22931936)
\lineto(335.98572632,78.22931936)
\curveto(335.90572769,78.23931812)(335.84072775,78.2593181)(335.79072632,78.28931936)
\curveto(335.71072788,78.34931801)(335.66572793,78.41931794)(335.65572632,78.49931936)
\curveto(335.64572795,78.51931784)(335.64072795,78.53931782)(335.64072632,78.55931936)
\curveto(335.64072795,78.58931777)(335.63572796,78.61431774)(335.62572632,78.63431936)
}
}
{
\newrgbcolor{curcolor}{0 0 0}
\pscustom[linestyle=none,fillstyle=solid,fillcolor=curcolor]
{
}
}
{
\newrgbcolor{curcolor}{0 0 0}
\pscustom[linestyle=none,fillstyle=solid,fillcolor=curcolor]
{
\newpath
\moveto(326.65572632,89.26463186)
\curveto(326.64573695,89.95462722)(326.76573683,90.55462662)(327.01572632,91.06463186)
\curveto(327.26573633,91.58462559)(327.60073599,91.9796252)(328.02072632,92.24963186)
\curveto(328.10073549,92.29962488)(328.1907354,92.34462483)(328.29072632,92.38463186)
\curveto(328.38073521,92.42462475)(328.47573512,92.46962471)(328.57572632,92.51963186)
\curveto(328.67573492,92.55962462)(328.77573482,92.58962459)(328.87572632,92.60963186)
\curveto(328.97573462,92.62962455)(329.08073451,92.64962453)(329.19072632,92.66963186)
\curveto(329.24073435,92.68962449)(329.28573431,92.69462448)(329.32572632,92.68463186)
\curveto(329.36573423,92.6746245)(329.41073418,92.6796245)(329.46072632,92.69963186)
\curveto(329.51073408,92.70962447)(329.595734,92.71462446)(329.71572632,92.71463186)
\curveto(329.82573377,92.71462446)(329.91073368,92.70962447)(329.97072632,92.69963186)
\curveto(330.03073356,92.6796245)(330.0907335,92.66962451)(330.15072632,92.66963186)
\curveto(330.21073338,92.6796245)(330.27073332,92.6746245)(330.33072632,92.65463186)
\curveto(330.47073312,92.61462456)(330.60573299,92.5796246)(330.73572632,92.54963186)
\curveto(330.86573273,92.51962466)(330.9907326,92.4796247)(331.11072632,92.42963186)
\curveto(331.25073234,92.36962481)(331.37573222,92.29962488)(331.48572632,92.21963186)
\curveto(331.595732,92.14962503)(331.70573189,92.0746251)(331.81572632,91.99463186)
\lineto(331.87572632,91.93463186)
\curveto(331.8957317,91.92462525)(331.91573168,91.90962527)(331.93572632,91.88963186)
\curveto(332.0957315,91.76962541)(332.24073135,91.63462554)(332.37072632,91.48463186)
\curveto(332.50073109,91.33462584)(332.62573097,91.174626)(332.74572632,91.00463186)
\curveto(332.96573063,90.69462648)(333.17073042,90.39962678)(333.36072632,90.11963186)
\curveto(333.50073009,89.88962729)(333.63572996,89.65962752)(333.76572632,89.42963186)
\curveto(333.8957297,89.20962797)(334.03072956,88.98962819)(334.17072632,88.76963186)
\curveto(334.34072925,88.51962866)(334.52072907,88.2796289)(334.71072632,88.04963186)
\curveto(334.90072869,87.82962935)(335.12572847,87.63962954)(335.38572632,87.47963186)
\curveto(335.44572815,87.43962974)(335.50572809,87.40462977)(335.56572632,87.37463186)
\curveto(335.61572798,87.34462983)(335.68072791,87.31462986)(335.76072632,87.28463186)
\curveto(335.83072776,87.26462991)(335.8907277,87.25962992)(335.94072632,87.26963186)
\curveto(336.01072758,87.28962989)(336.06572753,87.32462985)(336.10572632,87.37463186)
\curveto(336.13572746,87.42462975)(336.15572744,87.48462969)(336.16572632,87.55463186)
\lineto(336.16572632,87.79463186)
\lineto(336.16572632,88.54463186)
\lineto(336.16572632,91.34963186)
\lineto(336.16572632,92.00963186)
\curveto(336.16572743,92.09962508)(336.17072742,92.18462499)(336.18072632,92.26463186)
\curveto(336.18072741,92.34462483)(336.20072739,92.40962477)(336.24072632,92.45963186)
\curveto(336.28072731,92.50962467)(336.35572724,92.54962463)(336.46572632,92.57963186)
\curveto(336.56572703,92.61962456)(336.66572693,92.62962455)(336.76572632,92.60963186)
\lineto(336.90072632,92.60963186)
\curveto(336.97072662,92.58962459)(337.03072656,92.56962461)(337.08072632,92.54963186)
\curveto(337.13072646,92.52962465)(337.17072642,92.49462468)(337.20072632,92.44463186)
\curveto(337.24072635,92.39462478)(337.26072633,92.32462485)(337.26072632,92.23463186)
\lineto(337.26072632,91.96463186)
\lineto(337.26072632,91.06463186)
\lineto(337.26072632,87.55463186)
\lineto(337.26072632,86.48963186)
\curveto(337.26072633,86.40963077)(337.26572633,86.31963086)(337.27572632,86.21963186)
\curveto(337.27572632,86.11963106)(337.26572633,86.03463114)(337.24572632,85.96463186)
\curveto(337.17572642,85.75463142)(336.9957266,85.68963149)(336.70572632,85.76963186)
\curveto(336.66572693,85.7796314)(336.63072696,85.7796314)(336.60072632,85.76963186)
\curveto(336.56072703,85.76963141)(336.51572708,85.7796314)(336.46572632,85.79963186)
\curveto(336.38572721,85.81963136)(336.30072729,85.83963134)(336.21072632,85.85963186)
\curveto(336.12072747,85.8796313)(336.03572756,85.90463127)(335.95572632,85.93463186)
\curveto(335.46572813,86.09463108)(335.05072854,86.29463088)(334.71072632,86.53463186)
\curveto(334.46072913,86.71463046)(334.23572936,86.91963026)(334.03572632,87.14963186)
\curveto(333.82572977,87.3796298)(333.63072996,87.61962956)(333.45072632,87.86963186)
\curveto(333.27073032,88.12962905)(333.10073049,88.39462878)(332.94072632,88.66463186)
\curveto(332.77073082,88.94462823)(332.595731,89.21462796)(332.41572632,89.47463186)
\curveto(332.33573126,89.58462759)(332.26073133,89.68962749)(332.19072632,89.78963186)
\curveto(332.12073147,89.89962728)(332.04573155,90.00962717)(331.96572632,90.11963186)
\curveto(331.93573166,90.15962702)(331.90573169,90.19462698)(331.87572632,90.22463186)
\curveto(331.83573176,90.26462691)(331.80573179,90.30462687)(331.78572632,90.34463186)
\curveto(331.67573192,90.48462669)(331.55073204,90.60962657)(331.41072632,90.71963186)
\curveto(331.38073221,90.73962644)(331.35573224,90.76462641)(331.33572632,90.79463186)
\curveto(331.30573229,90.82462635)(331.27573232,90.84962633)(331.24572632,90.86963186)
\curveto(331.14573245,90.94962623)(331.04573255,91.01462616)(330.94572632,91.06463186)
\curveto(330.84573275,91.12462605)(330.73573286,91.179626)(330.61572632,91.22963186)
\curveto(330.54573305,91.25962592)(330.47073312,91.2796259)(330.39072632,91.28963186)
\lineto(330.15072632,91.34963186)
\lineto(330.06072632,91.34963186)
\curveto(330.03073356,91.35962582)(330.00073359,91.36462581)(329.97072632,91.36463186)
\curveto(329.90073369,91.38462579)(329.80573379,91.38962579)(329.68572632,91.37963186)
\curveto(329.55573404,91.3796258)(329.45573414,91.36962581)(329.38572632,91.34963186)
\curveto(329.30573429,91.32962585)(329.23073436,91.30962587)(329.16072632,91.28963186)
\curveto(329.08073451,91.2796259)(329.00073459,91.25962592)(328.92072632,91.22963186)
\curveto(328.68073491,91.11962606)(328.48073511,90.96962621)(328.32072632,90.77963186)
\curveto(328.15073544,90.59962658)(328.01073558,90.3796268)(327.90072632,90.11963186)
\curveto(327.88073571,90.04962713)(327.86573573,89.9796272)(327.85572632,89.90963186)
\curveto(327.83573576,89.83962734)(327.81573578,89.76462741)(327.79572632,89.68463186)
\curveto(327.77573582,89.60462757)(327.76573583,89.49462768)(327.76572632,89.35463186)
\curveto(327.76573583,89.22462795)(327.77573582,89.11962806)(327.79572632,89.03963186)
\curveto(327.80573579,88.9796282)(327.81073578,88.92462825)(327.81072632,88.87463186)
\curveto(327.81073578,88.82462835)(327.82073577,88.7746284)(327.84072632,88.72463186)
\curveto(327.88073571,88.62462855)(327.92073567,88.52962865)(327.96072632,88.43963186)
\curveto(328.00073559,88.35962882)(328.04573555,88.2796289)(328.09572632,88.19963186)
\curveto(328.11573548,88.16962901)(328.14073545,88.13962904)(328.17072632,88.10963186)
\curveto(328.20073539,88.08962909)(328.22573537,88.06462911)(328.24572632,88.03463186)
\lineto(328.32072632,87.95963186)
\curveto(328.34073525,87.92962925)(328.36073523,87.90462927)(328.38072632,87.88463186)
\lineto(328.59072632,87.73463186)
\curveto(328.65073494,87.69462948)(328.71573488,87.64962953)(328.78572632,87.59963186)
\curveto(328.87573472,87.53962964)(328.98073461,87.48962969)(329.10072632,87.44963186)
\curveto(329.21073438,87.41962976)(329.32073427,87.38462979)(329.43072632,87.34463186)
\curveto(329.54073405,87.30462987)(329.68573391,87.2796299)(329.86572632,87.26963186)
\curveto(330.03573356,87.25962992)(330.16073343,87.22962995)(330.24072632,87.17963186)
\curveto(330.32073327,87.12963005)(330.36573323,87.05463012)(330.37572632,86.95463186)
\curveto(330.38573321,86.85463032)(330.3907332,86.74463043)(330.39072632,86.62463186)
\curveto(330.3907332,86.58463059)(330.3957332,86.54463063)(330.40572632,86.50463186)
\curveto(330.40573319,86.46463071)(330.40073319,86.42963075)(330.39072632,86.39963186)
\curveto(330.37073322,86.34963083)(330.36073323,86.29963088)(330.36072632,86.24963186)
\curveto(330.36073323,86.20963097)(330.35073324,86.16963101)(330.33072632,86.12963186)
\curveto(330.27073332,86.03963114)(330.13573346,85.99463118)(329.92572632,85.99463186)
\lineto(329.80572632,85.99463186)
\curveto(329.74573385,86.00463117)(329.68573391,86.00963117)(329.62572632,86.00963186)
\curveto(329.55573404,86.01963116)(329.4907341,86.02963115)(329.43072632,86.03963186)
\curveto(329.32073427,86.05963112)(329.22073437,86.0796311)(329.13072632,86.09963186)
\curveto(329.03073456,86.11963106)(328.93573466,86.14963103)(328.84572632,86.18963186)
\curveto(328.77573482,86.20963097)(328.71573488,86.22963095)(328.66572632,86.24963186)
\lineto(328.48572632,86.30963186)
\curveto(328.22573537,86.42963075)(327.98073561,86.58463059)(327.75072632,86.77463186)
\curveto(327.52073607,86.9746302)(327.33573626,87.18962999)(327.19572632,87.41963186)
\curveto(327.11573648,87.52962965)(327.05073654,87.64462953)(327.00072632,87.76463186)
\lineto(326.85072632,88.15463186)
\curveto(326.80073679,88.26462891)(326.77073682,88.3796288)(326.76072632,88.49963186)
\curveto(326.74073685,88.61962856)(326.71573688,88.74462843)(326.68572632,88.87463186)
\curveto(326.68573691,88.94462823)(326.68573691,89.00962817)(326.68572632,89.06963186)
\curveto(326.67573692,89.12962805)(326.66573693,89.19462798)(326.65572632,89.26463186)
}
}
{
\newrgbcolor{curcolor}{0 0 0}
\pscustom[linestyle=none,fillstyle=solid,fillcolor=curcolor]
{
\newpath
\moveto(332.17572632,101.36424123)
\lineto(332.43072632,101.36424123)
\curveto(332.51073108,101.37423353)(332.58573101,101.36923353)(332.65572632,101.34924123)
\lineto(332.89572632,101.34924123)
\lineto(333.06072632,101.34924123)
\curveto(333.16073043,101.32923357)(333.26573033,101.31923358)(333.37572632,101.31924123)
\curveto(333.47573012,101.31923358)(333.57573002,101.30923359)(333.67572632,101.28924123)
\lineto(333.82572632,101.28924123)
\curveto(333.96572963,101.25923364)(334.10572949,101.23923366)(334.24572632,101.22924123)
\curveto(334.37572922,101.21923368)(334.50572909,101.19423371)(334.63572632,101.15424123)
\curveto(334.71572888,101.13423377)(334.80072879,101.11423379)(334.89072632,101.09424123)
\lineto(335.13072632,101.03424123)
\lineto(335.43072632,100.91424123)
\curveto(335.52072807,100.88423402)(335.61072798,100.84923405)(335.70072632,100.80924123)
\curveto(335.92072767,100.70923419)(336.13572746,100.57423433)(336.34572632,100.40424123)
\curveto(336.55572704,100.24423466)(336.72572687,100.06923483)(336.85572632,99.87924123)
\curveto(336.8957267,99.82923507)(336.93572666,99.76923513)(336.97572632,99.69924123)
\curveto(337.00572659,99.63923526)(337.04072655,99.57923532)(337.08072632,99.51924123)
\curveto(337.13072646,99.43923546)(337.17072642,99.34423556)(337.20072632,99.23424123)
\curveto(337.23072636,99.12423578)(337.26072633,99.01923588)(337.29072632,98.91924123)
\curveto(337.33072626,98.80923609)(337.35572624,98.6992362)(337.36572632,98.58924123)
\curveto(337.37572622,98.47923642)(337.3907262,98.36423654)(337.41072632,98.24424123)
\curveto(337.42072617,98.2042367)(337.42072617,98.15923674)(337.41072632,98.10924123)
\curveto(337.41072618,98.06923683)(337.41572618,98.02923687)(337.42572632,97.98924123)
\curveto(337.43572616,97.94923695)(337.44072615,97.89423701)(337.44072632,97.82424123)
\curveto(337.44072615,97.75423715)(337.43572616,97.7042372)(337.42572632,97.67424123)
\curveto(337.40572619,97.62423728)(337.40072619,97.57923732)(337.41072632,97.53924123)
\curveto(337.42072617,97.4992374)(337.42072617,97.46423744)(337.41072632,97.43424123)
\lineto(337.41072632,97.34424123)
\curveto(337.3907262,97.28423762)(337.37572622,97.21923768)(337.36572632,97.14924123)
\curveto(337.36572623,97.08923781)(337.36072623,97.02423788)(337.35072632,96.95424123)
\curveto(337.30072629,96.78423812)(337.25072634,96.62423828)(337.20072632,96.47424123)
\curveto(337.15072644,96.32423858)(337.08572651,96.17923872)(337.00572632,96.03924123)
\curveto(336.96572663,95.98923891)(336.93572666,95.93423897)(336.91572632,95.87424123)
\curveto(336.88572671,95.82423908)(336.85072674,95.77423913)(336.81072632,95.72424123)
\curveto(336.63072696,95.48423942)(336.41072718,95.28423962)(336.15072632,95.12424123)
\curveto(335.8907277,94.96423994)(335.60572799,94.82424008)(335.29572632,94.70424123)
\curveto(335.15572844,94.64424026)(335.01572858,94.5992403)(334.87572632,94.56924123)
\curveto(334.72572887,94.53924036)(334.57072902,94.5042404)(334.41072632,94.46424123)
\curveto(334.30072929,94.44424046)(334.1907294,94.42924047)(334.08072632,94.41924123)
\curveto(333.97072962,94.40924049)(333.86072973,94.39424051)(333.75072632,94.37424123)
\curveto(333.71072988,94.36424054)(333.67072992,94.35924054)(333.63072632,94.35924123)
\curveto(333.59073,94.36924053)(333.55073004,94.36924053)(333.51072632,94.35924123)
\curveto(333.46073013,94.34924055)(333.41073018,94.34424056)(333.36072632,94.34424123)
\lineto(333.19572632,94.34424123)
\curveto(333.14573045,94.32424058)(333.0957305,94.31924058)(333.04572632,94.32924123)
\curveto(332.98573061,94.33924056)(332.93073066,94.33924056)(332.88072632,94.32924123)
\curveto(332.84073075,94.31924058)(332.7957308,94.31924058)(332.74572632,94.32924123)
\curveto(332.6957309,94.33924056)(332.64573095,94.33424057)(332.59572632,94.31424123)
\curveto(332.52573107,94.29424061)(332.45073114,94.28924061)(332.37072632,94.29924123)
\curveto(332.28073131,94.30924059)(332.1957314,94.31424059)(332.11572632,94.31424123)
\curveto(332.02573157,94.31424059)(331.92573167,94.30924059)(331.81572632,94.29924123)
\curveto(331.6957319,94.28924061)(331.595732,94.29424061)(331.51572632,94.31424123)
\lineto(331.23072632,94.31424123)
\lineto(330.60072632,94.35924123)
\curveto(330.50073309,94.36924053)(330.40573319,94.37924052)(330.31572632,94.38924123)
\lineto(330.01572632,94.41924123)
\curveto(329.96573363,94.43924046)(329.91573368,94.44424046)(329.86572632,94.43424123)
\curveto(329.80573379,94.43424047)(329.75073384,94.44424046)(329.70072632,94.46424123)
\curveto(329.53073406,94.51424039)(329.36573423,94.55424035)(329.20572632,94.58424123)
\curveto(329.03573456,94.61424029)(328.87573472,94.66424024)(328.72572632,94.73424123)
\curveto(328.26573533,94.92423998)(327.8907357,95.14423976)(327.60072632,95.39424123)
\curveto(327.31073628,95.65423925)(327.06573653,96.01423889)(326.86572632,96.47424123)
\curveto(326.81573678,96.6042383)(326.78073681,96.73423817)(326.76072632,96.86424123)
\curveto(326.74073685,97.0042379)(326.71573688,97.14423776)(326.68572632,97.28424123)
\curveto(326.67573692,97.35423755)(326.67073692,97.41923748)(326.67072632,97.47924123)
\curveto(326.67073692,97.53923736)(326.66573693,97.6042373)(326.65572632,97.67424123)
\curveto(326.63573696,98.5042364)(326.78573681,99.17423573)(327.10572632,99.68424123)
\curveto(327.41573618,100.19423471)(327.85573574,100.57423433)(328.42572632,100.82424123)
\curveto(328.54573505,100.87423403)(328.67073492,100.91923398)(328.80072632,100.95924123)
\curveto(328.93073466,100.9992339)(329.06573453,101.04423386)(329.20572632,101.09424123)
\curveto(329.28573431,101.11423379)(329.37073422,101.12923377)(329.46072632,101.13924123)
\lineto(329.70072632,101.19924123)
\curveto(329.81073378,101.22923367)(329.92073367,101.24423366)(330.03072632,101.24424123)
\curveto(330.14073345,101.25423365)(330.25073334,101.26923363)(330.36072632,101.28924123)
\curveto(330.41073318,101.30923359)(330.45573314,101.31423359)(330.49572632,101.30424123)
\curveto(330.53573306,101.3042336)(330.57573302,101.30923359)(330.61572632,101.31924123)
\curveto(330.66573293,101.32923357)(330.72073287,101.32923357)(330.78072632,101.31924123)
\curveto(330.83073276,101.31923358)(330.88073271,101.32423358)(330.93072632,101.33424123)
\lineto(331.06572632,101.33424123)
\curveto(331.12573247,101.35423355)(331.1957324,101.35423355)(331.27572632,101.33424123)
\curveto(331.34573225,101.32423358)(331.41073218,101.32923357)(331.47072632,101.34924123)
\curveto(331.50073209,101.35923354)(331.54073205,101.36423354)(331.59072632,101.36424123)
\lineto(331.71072632,101.36424123)
\lineto(332.17572632,101.36424123)
\moveto(334.50072632,99.81924123)
\curveto(334.18072941,99.91923498)(333.81572978,99.97923492)(333.40572632,99.99924123)
\curveto(332.9957306,100.01923488)(332.58573101,100.02923487)(332.17572632,100.02924123)
\curveto(331.74573185,100.02923487)(331.32573227,100.01923488)(330.91572632,99.99924123)
\curveto(330.50573309,99.97923492)(330.12073347,99.93423497)(329.76072632,99.86424123)
\curveto(329.40073419,99.79423511)(329.08073451,99.68423522)(328.80072632,99.53424123)
\curveto(328.51073508,99.39423551)(328.27573532,99.1992357)(328.09572632,98.94924123)
\curveto(327.98573561,98.78923611)(327.90573569,98.60923629)(327.85572632,98.40924123)
\curveto(327.7957358,98.20923669)(327.76573583,97.96423694)(327.76572632,97.67424123)
\curveto(327.78573581,97.65423725)(327.7957358,97.61923728)(327.79572632,97.56924123)
\curveto(327.78573581,97.51923738)(327.78573581,97.47923742)(327.79572632,97.44924123)
\curveto(327.81573578,97.36923753)(327.83573576,97.29423761)(327.85572632,97.22424123)
\curveto(327.86573573,97.16423774)(327.88573571,97.0992378)(327.91572632,97.02924123)
\curveto(328.03573556,96.75923814)(328.20573539,96.53923836)(328.42572632,96.36924123)
\curveto(328.63573496,96.20923869)(328.88073471,96.07423883)(329.16072632,95.96424123)
\curveto(329.27073432,95.91423899)(329.3907342,95.87423903)(329.52072632,95.84424123)
\curveto(329.64073395,95.82423908)(329.76573383,95.7992391)(329.89572632,95.76924123)
\curveto(329.94573365,95.74923915)(330.00073359,95.73923916)(330.06072632,95.73924123)
\curveto(330.11073348,95.73923916)(330.16073343,95.73423917)(330.21072632,95.72424123)
\curveto(330.30073329,95.71423919)(330.3957332,95.7042392)(330.49572632,95.69424123)
\curveto(330.58573301,95.68423922)(330.68073291,95.67423923)(330.78072632,95.66424123)
\curveto(330.86073273,95.66423924)(330.94573265,95.65923924)(331.03572632,95.64924123)
\lineto(331.27572632,95.64924123)
\lineto(331.45572632,95.64924123)
\curveto(331.48573211,95.63923926)(331.52073207,95.63423927)(331.56072632,95.63424123)
\lineto(331.69572632,95.63424123)
\lineto(332.14572632,95.63424123)
\curveto(332.22573137,95.63423927)(332.31073128,95.62923927)(332.40072632,95.61924123)
\curveto(332.48073111,95.61923928)(332.55573104,95.62923927)(332.62572632,95.64924123)
\lineto(332.89572632,95.64924123)
\curveto(332.91573068,95.64923925)(332.94573065,95.64423926)(332.98572632,95.63424123)
\curveto(333.01573058,95.63423927)(333.04073055,95.63923926)(333.06072632,95.64924123)
\curveto(333.16073043,95.65923924)(333.26073033,95.66423924)(333.36072632,95.66424123)
\curveto(333.45073014,95.67423923)(333.55073004,95.68423922)(333.66072632,95.69424123)
\curveto(333.78072981,95.72423918)(333.90572969,95.73923916)(334.03572632,95.73924123)
\curveto(334.15572944,95.74923915)(334.27072932,95.77423913)(334.38072632,95.81424123)
\curveto(334.68072891,95.89423901)(334.94572865,95.97923892)(335.17572632,96.06924123)
\curveto(335.40572819,96.16923873)(335.62072797,96.31423859)(335.82072632,96.50424123)
\curveto(336.02072757,96.71423819)(336.17072742,96.97923792)(336.27072632,97.29924123)
\curveto(336.2907273,97.33923756)(336.30072729,97.37423753)(336.30072632,97.40424123)
\curveto(336.2907273,97.44423746)(336.2957273,97.48923741)(336.31572632,97.53924123)
\curveto(336.32572727,97.57923732)(336.33572726,97.64923725)(336.34572632,97.74924123)
\curveto(336.35572724,97.85923704)(336.35072724,97.94423696)(336.33072632,98.00424123)
\curveto(336.31072728,98.07423683)(336.30072729,98.14423676)(336.30072632,98.21424123)
\curveto(336.2907273,98.28423662)(336.27572732,98.34923655)(336.25572632,98.40924123)
\curveto(336.1957274,98.60923629)(336.11072748,98.78923611)(336.00072632,98.94924123)
\curveto(335.98072761,98.97923592)(335.96072763,99.0042359)(335.94072632,99.02424123)
\lineto(335.88072632,99.08424123)
\curveto(335.86072773,99.12423578)(335.82072777,99.17423573)(335.76072632,99.23424123)
\curveto(335.62072797,99.33423557)(335.4907281,99.41923548)(335.37072632,99.48924123)
\curveto(335.25072834,99.55923534)(335.10572849,99.62923527)(334.93572632,99.69924123)
\curveto(334.86572873,99.72923517)(334.7957288,99.74923515)(334.72572632,99.75924123)
\curveto(334.65572894,99.77923512)(334.58072901,99.7992351)(334.50072632,99.81924123)
}
}
{
\newrgbcolor{curcolor}{0 0 0}
\pscustom[linestyle=none,fillstyle=solid,fillcolor=curcolor]
{
\newpath
\moveto(326.65572632,106.77385061)
\curveto(326.65573694,106.87384575)(326.66573693,106.96884566)(326.68572632,107.05885061)
\curveto(326.6957369,107.14884548)(326.72573687,107.21384541)(326.77572632,107.25385061)
\curveto(326.85573674,107.31384531)(326.96073663,107.34384528)(327.09072632,107.34385061)
\lineto(327.48072632,107.34385061)
\lineto(328.98072632,107.34385061)
\lineto(335.37072632,107.34385061)
\lineto(336.54072632,107.34385061)
\lineto(336.85572632,107.34385061)
\curveto(336.95572664,107.35384527)(337.03572656,107.33884529)(337.09572632,107.29885061)
\curveto(337.17572642,107.24884538)(337.22572637,107.17384545)(337.24572632,107.07385061)
\curveto(337.25572634,106.98384564)(337.26072633,106.87384575)(337.26072632,106.74385061)
\lineto(337.26072632,106.51885061)
\curveto(337.24072635,106.43884619)(337.22572637,106.36884626)(337.21572632,106.30885061)
\curveto(337.1957264,106.24884638)(337.15572644,106.19884643)(337.09572632,106.15885061)
\curveto(337.03572656,106.11884651)(336.96072663,106.09884653)(336.87072632,106.09885061)
\lineto(336.57072632,106.09885061)
\lineto(335.47572632,106.09885061)
\lineto(330.13572632,106.09885061)
\curveto(330.04573355,106.07884655)(329.97073362,106.06384656)(329.91072632,106.05385061)
\curveto(329.84073375,106.05384657)(329.78073381,106.0238466)(329.73072632,105.96385061)
\curveto(329.68073391,105.89384673)(329.65573394,105.80384682)(329.65572632,105.69385061)
\curveto(329.64573395,105.59384703)(329.64073395,105.48384714)(329.64072632,105.36385061)
\lineto(329.64072632,104.22385061)
\lineto(329.64072632,103.72885061)
\curveto(329.63073396,103.56884906)(329.57073402,103.45884917)(329.46072632,103.39885061)
\curveto(329.43073416,103.37884925)(329.40073419,103.36884926)(329.37072632,103.36885061)
\curveto(329.33073426,103.36884926)(329.28573431,103.36384926)(329.23572632,103.35385061)
\curveto(329.11573448,103.33384929)(329.00573459,103.33884929)(328.90572632,103.36885061)
\curveto(328.80573479,103.40884922)(328.73573486,103.46384916)(328.69572632,103.53385061)
\curveto(328.64573495,103.61384901)(328.62073497,103.73384889)(328.62072632,103.89385061)
\curveto(328.62073497,104.05384857)(328.60573499,104.18884844)(328.57572632,104.29885061)
\curveto(328.56573503,104.34884828)(328.56073503,104.40384822)(328.56072632,104.46385061)
\curveto(328.55073504,104.5238481)(328.53573506,104.58384804)(328.51572632,104.64385061)
\curveto(328.46573513,104.79384783)(328.41573518,104.93884769)(328.36572632,105.07885061)
\curveto(328.30573529,105.21884741)(328.23573536,105.35384727)(328.15572632,105.48385061)
\curveto(328.06573553,105.623847)(327.96073563,105.74384688)(327.84072632,105.84385061)
\curveto(327.72073587,105.94384668)(327.590736,106.03884659)(327.45072632,106.12885061)
\curveto(327.35073624,106.18884644)(327.24073635,106.23384639)(327.12072632,106.26385061)
\curveto(327.00073659,106.30384632)(326.8957367,106.35384627)(326.80572632,106.41385061)
\curveto(326.74573685,106.46384616)(326.70573689,106.53384609)(326.68572632,106.62385061)
\curveto(326.67573692,106.64384598)(326.67073692,106.66884596)(326.67072632,106.69885061)
\curveto(326.67073692,106.7288459)(326.66573693,106.75384587)(326.65572632,106.77385061)
}
}
{
\newrgbcolor{curcolor}{0 0 0}
\pscustom[linestyle=none,fillstyle=solid,fillcolor=curcolor]
{
\newpath
\moveto(326.65572632,115.12345998)
\curveto(326.65573694,115.22345513)(326.66573693,115.31845503)(326.68572632,115.40845998)
\curveto(326.6957369,115.49845485)(326.72573687,115.56345479)(326.77572632,115.60345998)
\curveto(326.85573674,115.66345469)(326.96073663,115.69345466)(327.09072632,115.69345998)
\lineto(327.48072632,115.69345998)
\lineto(328.98072632,115.69345998)
\lineto(335.37072632,115.69345998)
\lineto(336.54072632,115.69345998)
\lineto(336.85572632,115.69345998)
\curveto(336.95572664,115.70345465)(337.03572656,115.68845466)(337.09572632,115.64845998)
\curveto(337.17572642,115.59845475)(337.22572637,115.52345483)(337.24572632,115.42345998)
\curveto(337.25572634,115.33345502)(337.26072633,115.22345513)(337.26072632,115.09345998)
\lineto(337.26072632,114.86845998)
\curveto(337.24072635,114.78845556)(337.22572637,114.71845563)(337.21572632,114.65845998)
\curveto(337.1957264,114.59845575)(337.15572644,114.5484558)(337.09572632,114.50845998)
\curveto(337.03572656,114.46845588)(336.96072663,114.4484559)(336.87072632,114.44845998)
\lineto(336.57072632,114.44845998)
\lineto(335.47572632,114.44845998)
\lineto(330.13572632,114.44845998)
\curveto(330.04573355,114.42845592)(329.97073362,114.41345594)(329.91072632,114.40345998)
\curveto(329.84073375,114.40345595)(329.78073381,114.37345598)(329.73072632,114.31345998)
\curveto(329.68073391,114.24345611)(329.65573394,114.1534562)(329.65572632,114.04345998)
\curveto(329.64573395,113.94345641)(329.64073395,113.83345652)(329.64072632,113.71345998)
\lineto(329.64072632,112.57345998)
\lineto(329.64072632,112.07845998)
\curveto(329.63073396,111.91845843)(329.57073402,111.80845854)(329.46072632,111.74845998)
\curveto(329.43073416,111.72845862)(329.40073419,111.71845863)(329.37072632,111.71845998)
\curveto(329.33073426,111.71845863)(329.28573431,111.71345864)(329.23572632,111.70345998)
\curveto(329.11573448,111.68345867)(329.00573459,111.68845866)(328.90572632,111.71845998)
\curveto(328.80573479,111.75845859)(328.73573486,111.81345854)(328.69572632,111.88345998)
\curveto(328.64573495,111.96345839)(328.62073497,112.08345827)(328.62072632,112.24345998)
\curveto(328.62073497,112.40345795)(328.60573499,112.53845781)(328.57572632,112.64845998)
\curveto(328.56573503,112.69845765)(328.56073503,112.7534576)(328.56072632,112.81345998)
\curveto(328.55073504,112.87345748)(328.53573506,112.93345742)(328.51572632,112.99345998)
\curveto(328.46573513,113.14345721)(328.41573518,113.28845706)(328.36572632,113.42845998)
\curveto(328.30573529,113.56845678)(328.23573536,113.70345665)(328.15572632,113.83345998)
\curveto(328.06573553,113.97345638)(327.96073563,114.09345626)(327.84072632,114.19345998)
\curveto(327.72073587,114.29345606)(327.590736,114.38845596)(327.45072632,114.47845998)
\curveto(327.35073624,114.53845581)(327.24073635,114.58345577)(327.12072632,114.61345998)
\curveto(327.00073659,114.6534557)(326.8957367,114.70345565)(326.80572632,114.76345998)
\curveto(326.74573685,114.81345554)(326.70573689,114.88345547)(326.68572632,114.97345998)
\curveto(326.67573692,114.99345536)(326.67073692,115.01845533)(326.67072632,115.04845998)
\curveto(326.67073692,115.07845527)(326.66573693,115.10345525)(326.65572632,115.12345998)
}
}
{
\newrgbcolor{curcolor}{0 0 0}
\pscustom[linestyle=none,fillstyle=solid,fillcolor=curcolor]
{
\newpath
\moveto(357.39207275,42.02236623)
\curveto(357.4420735,42.04235669)(357.50207344,42.06735666)(357.57207275,42.09736623)
\curveto(357.6420733,42.1273566)(357.71707322,42.14735658)(357.79707275,42.15736623)
\curveto(357.86707307,42.17735655)(357.937073,42.17735655)(358.00707275,42.15736623)
\curveto(358.06707287,42.14735658)(358.11207283,42.10735662)(358.14207275,42.03736623)
\curveto(358.16207278,41.98735674)(358.17207277,41.9273568)(358.17207275,41.85736623)
\lineto(358.17207275,41.64736623)
\lineto(358.17207275,41.19736623)
\curveto(358.17207277,41.04735768)(358.14707279,40.9273578)(358.09707275,40.83736623)
\curveto(358.0370729,40.73735799)(357.93207301,40.66235807)(357.78207275,40.61236623)
\curveto(357.63207331,40.57235816)(357.49707344,40.5273582)(357.37707275,40.47736623)
\curveto(357.11707382,40.36735836)(356.84707409,40.26735846)(356.56707275,40.17736623)
\curveto(356.28707465,40.08735864)(356.01207493,39.98735874)(355.74207275,39.87736623)
\curveto(355.65207529,39.84735888)(355.56707537,39.81735891)(355.48707275,39.78736623)
\curveto(355.40707553,39.76735896)(355.33207561,39.73735899)(355.26207275,39.69736623)
\curveto(355.19207575,39.66735906)(355.13207581,39.62235911)(355.08207275,39.56236623)
\curveto(355.03207591,39.50235923)(354.99207595,39.42235931)(354.96207275,39.32236623)
\curveto(354.942076,39.27235946)(354.937076,39.21235952)(354.94707275,39.14236623)
\lineto(354.94707275,38.94736623)
\lineto(354.94707275,36.11236623)
\lineto(354.94707275,35.81236623)
\curveto(354.937076,35.70236303)(354.937076,35.59736313)(354.94707275,35.49736623)
\curveto(354.95707598,35.39736333)(354.97207597,35.30236343)(354.99207275,35.21236623)
\curveto(355.01207593,35.1323636)(355.05207589,35.07236366)(355.11207275,35.03236623)
\curveto(355.21207573,34.95236378)(355.32707561,34.89236384)(355.45707275,34.85236623)
\curveto(355.57707536,34.82236391)(355.70207524,34.78236395)(355.83207275,34.73236623)
\curveto(356.06207488,34.6323641)(356.30207464,34.53736419)(356.55207275,34.44736623)
\curveto(356.80207414,34.36736436)(357.0420739,34.27736445)(357.27207275,34.17736623)
\curveto(357.33207361,34.15736457)(357.40207354,34.1323646)(357.48207275,34.10236623)
\curveto(357.55207339,34.08236465)(357.62707331,34.05736467)(357.70707275,34.02736623)
\curveto(357.78707315,33.99736473)(357.86207308,33.96236477)(357.93207275,33.92236623)
\curveto(357.99207295,33.89236484)(358.0370729,33.85736487)(358.06707275,33.81736623)
\curveto(358.12707281,33.73736499)(358.16207278,33.6273651)(358.17207275,33.48736623)
\lineto(358.17207275,33.06736623)
\lineto(358.17207275,32.82736623)
\curveto(358.16207278,32.75736597)(358.1370728,32.69736603)(358.09707275,32.64736623)
\curveto(358.06707287,32.59736613)(358.02207292,32.56736616)(357.96207275,32.55736623)
\curveto(357.90207304,32.55736617)(357.8420731,32.56236617)(357.78207275,32.57236623)
\curveto(357.71207323,32.59236614)(357.64707329,32.61236612)(357.58707275,32.63236623)
\curveto(357.51707342,32.66236607)(357.46707347,32.68736604)(357.43707275,32.70736623)
\curveto(357.11707382,32.84736588)(356.80207414,32.97236576)(356.49207275,33.08236623)
\curveto(356.17207477,33.19236554)(355.85207509,33.31236542)(355.53207275,33.44236623)
\curveto(355.31207563,33.5323652)(355.09707584,33.61736511)(354.88707275,33.69736623)
\curveto(354.66707627,33.77736495)(354.44707649,33.86236487)(354.22707275,33.95236623)
\curveto(353.50707743,34.25236448)(352.78207816,34.53736419)(352.05207275,34.80736623)
\curveto(351.31207963,35.07736365)(350.57708036,35.36236337)(349.84707275,35.66236623)
\curveto(349.58708135,35.77236296)(349.32208162,35.87236286)(349.05207275,35.96236623)
\curveto(348.78208216,36.06236267)(348.51708242,36.16736256)(348.25707275,36.27736623)
\curveto(348.14708279,36.3273624)(348.02708291,36.37236236)(347.89707275,36.41236623)
\curveto(347.75708318,36.46236227)(347.65708328,36.5323622)(347.59707275,36.62236623)
\curveto(347.55708338,36.66236207)(347.52708341,36.727362)(347.50707275,36.81736623)
\curveto(347.49708344,36.83736189)(347.49708344,36.85736187)(347.50707275,36.87736623)
\curveto(347.50708343,36.90736182)(347.50208344,36.9323618)(347.49207275,36.95236623)
\curveto(347.49208345,37.1323616)(347.49208345,37.34236139)(347.49207275,37.58236623)
\curveto(347.48208346,37.82236091)(347.51708342,37.99736073)(347.59707275,38.10736623)
\curveto(347.65708328,38.18736054)(347.75708318,38.24736048)(347.89707275,38.28736623)
\curveto(348.02708291,38.33736039)(348.14708279,38.38736034)(348.25707275,38.43736623)
\curveto(348.48708245,38.53736019)(348.71708222,38.6273601)(348.94707275,38.70736623)
\curveto(349.17708176,38.78735994)(349.40708153,38.87735985)(349.63707275,38.97736623)
\curveto(349.8370811,39.05735967)(350.0420809,39.1323596)(350.25207275,39.20236623)
\curveto(350.46208048,39.28235945)(350.66708027,39.36735936)(350.86707275,39.45736623)
\curveto(351.59707934,39.75735897)(352.3370786,40.04235869)(353.08707275,40.31236623)
\curveto(353.82707711,40.59235814)(354.56207638,40.88735784)(355.29207275,41.19736623)
\curveto(355.38207556,41.23735749)(355.46707547,41.26735746)(355.54707275,41.28736623)
\curveto(355.62707531,41.31735741)(355.71207523,41.34735738)(355.80207275,41.37736623)
\curveto(356.06207488,41.48735724)(356.32707461,41.59235714)(356.59707275,41.69236623)
\curveto(356.86707407,41.80235693)(357.13207381,41.91235682)(357.39207275,42.02236623)
\moveto(353.74707275,38.81236623)
\curveto(353.71707722,38.90235983)(353.66707727,38.95735977)(353.59707275,38.97736623)
\curveto(353.52707741,39.00735972)(353.45207749,39.01235972)(353.37207275,38.99236623)
\curveto(353.28207766,38.98235975)(353.19707774,38.95735977)(353.11707275,38.91736623)
\curveto(353.02707791,38.88735984)(352.95207799,38.85735987)(352.89207275,38.82736623)
\curveto(352.85207809,38.80735992)(352.81707812,38.79735993)(352.78707275,38.79736623)
\curveto(352.75707818,38.79735993)(352.72207822,38.78735994)(352.68207275,38.76736623)
\lineto(352.44207275,38.67736623)
\curveto(352.35207859,38.65736007)(352.26207868,38.6273601)(352.17207275,38.58736623)
\curveto(351.81207913,38.43736029)(351.44707949,38.30236043)(351.07707275,38.18236623)
\curveto(350.69708024,38.07236066)(350.32708061,37.94236079)(349.96707275,37.79236623)
\curveto(349.85708108,37.74236099)(349.74708119,37.69736103)(349.63707275,37.65736623)
\curveto(349.52708141,37.6273611)(349.42208152,37.58736114)(349.32207275,37.53736623)
\curveto(349.27208167,37.51736121)(349.22708171,37.49236124)(349.18707275,37.46236623)
\curveto(349.1370818,37.44236129)(349.11208183,37.39236134)(349.11207275,37.31236623)
\curveto(349.13208181,37.29236144)(349.14708179,37.27236146)(349.15707275,37.25236623)
\curveto(349.16708177,37.2323615)(349.18208176,37.21236152)(349.20207275,37.19236623)
\curveto(349.25208169,37.15236158)(349.30708163,37.12236161)(349.36707275,37.10236623)
\curveto(349.41708152,37.08236165)(349.47208147,37.06236167)(349.53207275,37.04236623)
\curveto(349.6420813,36.99236174)(349.75208119,36.95236178)(349.86207275,36.92236623)
\curveto(349.97208097,36.89236184)(350.08208086,36.85236188)(350.19207275,36.80236623)
\curveto(350.58208036,36.6323621)(350.97707996,36.48236225)(351.37707275,36.35236623)
\curveto(351.77707916,36.2323625)(352.16707877,36.09236264)(352.54707275,35.93236623)
\lineto(352.69707275,35.87236623)
\curveto(352.74707819,35.86236287)(352.79707814,35.84736288)(352.84707275,35.82736623)
\lineto(353.08707275,35.73736623)
\curveto(353.16707777,35.70736302)(353.24707769,35.68236305)(353.32707275,35.66236623)
\curveto(353.37707756,35.64236309)(353.43207751,35.6323631)(353.49207275,35.63236623)
\curveto(353.55207739,35.64236309)(353.60207734,35.65736307)(353.64207275,35.67736623)
\curveto(353.72207722,35.727363)(353.76707717,35.8323629)(353.77707275,35.99236623)
\lineto(353.77707275,36.44236623)
\lineto(353.77707275,38.04736623)
\curveto(353.77707716,38.15736057)(353.78207716,38.29236044)(353.79207275,38.45236623)
\curveto(353.79207715,38.61236012)(353.77707716,38.73236)(353.74707275,38.81236623)
}
}
{
\newrgbcolor{curcolor}{0 0 0}
\pscustom[linestyle=none,fillstyle=solid,fillcolor=curcolor]
{
\newpath
\moveto(354.13707275,50.56392873)
\curveto(354.18707675,50.57392038)(354.25707668,50.57892038)(354.34707275,50.57892873)
\curveto(354.42707651,50.57892038)(354.49207645,50.57392038)(354.54207275,50.56392873)
\curveto(354.58207636,50.56392039)(354.62207632,50.5589204)(354.66207275,50.54892873)
\lineto(354.78207275,50.54892873)
\curveto(354.86207608,50.52892043)(354.942076,50.51892044)(355.02207275,50.51892873)
\curveto(355.10207584,50.51892044)(355.18207576,50.50892045)(355.26207275,50.48892873)
\curveto(355.30207564,50.47892048)(355.3420756,50.47392048)(355.38207275,50.47392873)
\curveto(355.41207553,50.47392048)(355.44707549,50.46892049)(355.48707275,50.45892873)
\curveto(355.59707534,50.42892053)(355.70207524,50.39892056)(355.80207275,50.36892873)
\curveto(355.90207504,50.34892061)(356.00207494,50.31892064)(356.10207275,50.27892873)
\curveto(356.45207449,50.13892082)(356.76707417,49.96892099)(357.04707275,49.76892873)
\curveto(357.32707361,49.56892139)(357.56707337,49.31892164)(357.76707275,49.01892873)
\curveto(357.86707307,48.86892209)(357.95207299,48.72392223)(358.02207275,48.58392873)
\curveto(358.07207287,48.47392248)(358.11207283,48.36392259)(358.14207275,48.25392873)
\curveto(358.17207277,48.1539228)(358.20207274,48.04892291)(358.23207275,47.93892873)
\curveto(358.25207269,47.86892309)(358.26207268,47.80392315)(358.26207275,47.74392873)
\curveto(358.27207267,47.68392327)(358.28707265,47.62392333)(358.30707275,47.56392873)
\lineto(358.30707275,47.41392873)
\curveto(358.32707261,47.36392359)(358.3370726,47.28892367)(358.33707275,47.18892873)
\curveto(358.34707259,47.08892387)(358.3420726,47.00892395)(358.32207275,46.94892873)
\lineto(358.32207275,46.79892873)
\curveto(358.31207263,46.7589242)(358.30707263,46.71392424)(358.30707275,46.66392873)
\curveto(358.30707263,46.62392433)(358.30207264,46.57892438)(358.29207275,46.52892873)
\curveto(358.25207269,46.37892458)(358.21707272,46.22892473)(358.18707275,46.07892873)
\curveto(358.15707278,45.93892502)(358.11207283,45.79892516)(358.05207275,45.65892873)
\curveto(357.97207297,45.4589255)(357.87207307,45.27892568)(357.75207275,45.11892873)
\lineto(357.60207275,44.93892873)
\curveto(357.5420734,44.87892608)(357.50207344,44.80892615)(357.48207275,44.72892873)
\curveto(357.47207347,44.66892629)(357.48707345,44.61892634)(357.52707275,44.57892873)
\curveto(357.55707338,44.54892641)(357.60207334,44.52392643)(357.66207275,44.50392873)
\curveto(357.72207322,44.49392646)(357.78707315,44.48392647)(357.85707275,44.47392873)
\curveto(357.91707302,44.47392648)(357.96207298,44.46392649)(357.99207275,44.44392873)
\curveto(358.0420729,44.40392655)(358.08707285,44.3589266)(358.12707275,44.30892873)
\curveto(358.14707279,44.2589267)(358.16207278,44.18892677)(358.17207275,44.09892873)
\lineto(358.17207275,43.82892873)
\curveto(358.17207277,43.73892722)(358.16707277,43.6539273)(358.15707275,43.57392873)
\curveto(358.1370728,43.49392746)(358.11707282,43.43392752)(358.09707275,43.39392873)
\curveto(358.07707286,43.37392758)(358.05207289,43.3539276)(358.02207275,43.33392873)
\lineto(357.93207275,43.27392873)
\curveto(357.85207309,43.24392771)(357.73207321,43.22892773)(357.57207275,43.22892873)
\curveto(357.41207353,43.23892772)(357.27707366,43.24392771)(357.16707275,43.24392873)
\lineto(348.36207275,43.24392873)
\curveto(348.2420827,43.24392771)(348.11708282,43.23892772)(347.98707275,43.22892873)
\curveto(347.84708309,43.22892773)(347.7370832,43.2539277)(347.65707275,43.30392873)
\curveto(347.59708334,43.34392761)(347.54708339,43.40892755)(347.50707275,43.49892873)
\curveto(347.50708343,43.51892744)(347.50708343,43.54392741)(347.50707275,43.57392873)
\curveto(347.49708344,43.60392735)(347.49208345,43.62892733)(347.49207275,43.64892873)
\curveto(347.48208346,43.78892717)(347.48208346,43.93392702)(347.49207275,44.08392873)
\curveto(347.49208345,44.24392671)(347.53208341,44.3539266)(347.61207275,44.41392873)
\curveto(347.69208325,44.46392649)(347.80708313,44.48892647)(347.95707275,44.48892873)
\lineto(348.36207275,44.48892873)
\lineto(350.11707275,44.48892873)
\lineto(350.37207275,44.48892873)
\lineto(350.65707275,44.48892873)
\curveto(350.74708019,44.49892646)(350.83208011,44.50892645)(350.91207275,44.51892873)
\curveto(350.98207996,44.53892642)(351.03207991,44.56892639)(351.06207275,44.60892873)
\curveto(351.09207985,44.64892631)(351.09707984,44.69392626)(351.07707275,44.74392873)
\curveto(351.05707988,44.79392616)(351.0370799,44.83392612)(351.01707275,44.86392873)
\curveto(350.97707996,44.91392604)(350.93708,44.958926)(350.89707275,44.99892873)
\lineto(350.77707275,45.14892873)
\curveto(350.72708021,45.21892574)(350.68208026,45.28892567)(350.64207275,45.35892873)
\lineto(350.52207275,45.59892873)
\curveto(350.43208051,45.77892518)(350.36708057,45.99392496)(350.32707275,46.24392873)
\curveto(350.28708065,46.49392446)(350.26708067,46.74892421)(350.26707275,47.00892873)
\curveto(350.26708067,47.26892369)(350.29208065,47.52392343)(350.34207275,47.77392873)
\curveto(350.38208056,48.02392293)(350.4420805,48.24392271)(350.52207275,48.43392873)
\curveto(350.69208025,48.83392212)(350.92708001,49.17892178)(351.22707275,49.46892873)
\curveto(351.52707941,49.7589212)(351.87707906,49.98892097)(352.27707275,50.15892873)
\curveto(352.38707855,50.20892075)(352.49707844,50.24892071)(352.60707275,50.27892873)
\curveto(352.70707823,50.31892064)(352.81207813,50.3589206)(352.92207275,50.39892873)
\curveto(353.03207791,50.42892053)(353.14707779,50.44892051)(353.26707275,50.45892873)
\lineto(353.59707275,50.51892873)
\curveto(353.62707731,50.52892043)(353.66207728,50.53392042)(353.70207275,50.53392873)
\curveto(353.73207721,50.53392042)(353.76207718,50.53892042)(353.79207275,50.54892873)
\curveto(353.85207709,50.56892039)(353.91207703,50.56892039)(353.97207275,50.54892873)
\curveto(354.02207692,50.53892042)(354.07707686,50.54392041)(354.13707275,50.56392873)
\moveto(354.52707275,49.22892873)
\curveto(354.47707646,49.24892171)(354.41707652,49.2539217)(354.34707275,49.24392873)
\curveto(354.27707666,49.23392172)(354.21207673,49.22892173)(354.15207275,49.22892873)
\curveto(353.98207696,49.22892173)(353.82207712,49.21892174)(353.67207275,49.19892873)
\curveto(353.52207742,49.18892177)(353.38707755,49.1589218)(353.26707275,49.10892873)
\curveto(353.16707777,49.07892188)(353.07707786,49.0539219)(352.99707275,49.03392873)
\curveto(352.91707802,49.01392194)(352.8370781,48.98392197)(352.75707275,48.94392873)
\curveto(352.50707843,48.83392212)(352.27707866,48.68392227)(352.06707275,48.49392873)
\curveto(351.84707909,48.30392265)(351.68207926,48.08392287)(351.57207275,47.83392873)
\curveto(351.5420794,47.7539232)(351.51707942,47.67392328)(351.49707275,47.59392873)
\curveto(351.46707947,47.52392343)(351.4420795,47.44892351)(351.42207275,47.36892873)
\curveto(351.39207955,47.2589237)(351.37707956,47.14892381)(351.37707275,47.03892873)
\curveto(351.36707957,46.92892403)(351.36207958,46.80892415)(351.36207275,46.67892873)
\curveto(351.37207957,46.62892433)(351.38207956,46.58392437)(351.39207275,46.54392873)
\lineto(351.39207275,46.40892873)
\lineto(351.45207275,46.13892873)
\curveto(351.47207947,46.0589249)(351.50207944,45.97892498)(351.54207275,45.89892873)
\curveto(351.68207926,45.5589254)(351.89207905,45.28892567)(352.17207275,45.08892873)
\curveto(352.4420785,44.88892607)(352.76207818,44.72892623)(353.13207275,44.60892873)
\curveto(353.2420777,44.56892639)(353.35207759,44.54392641)(353.46207275,44.53392873)
\curveto(353.57207737,44.52392643)(353.68707725,44.50392645)(353.80707275,44.47392873)
\curveto(353.85707708,44.46392649)(353.90207704,44.46392649)(353.94207275,44.47392873)
\curveto(353.98207696,44.48392647)(354.02707691,44.47892648)(354.07707275,44.45892873)
\curveto(354.12707681,44.44892651)(354.20207674,44.44392651)(354.30207275,44.44392873)
\curveto(354.39207655,44.44392651)(354.46207648,44.44892651)(354.51207275,44.45892873)
\lineto(354.63207275,44.45892873)
\curveto(354.67207627,44.46892649)(354.71207623,44.47392648)(354.75207275,44.47392873)
\curveto(354.79207615,44.47392648)(354.82707611,44.47892648)(354.85707275,44.48892873)
\curveto(354.88707605,44.49892646)(354.92207602,44.50392645)(354.96207275,44.50392873)
\curveto(354.99207595,44.50392645)(355.02207592,44.50892645)(355.05207275,44.51892873)
\curveto(355.13207581,44.53892642)(355.21207573,44.5539264)(355.29207275,44.56392873)
\lineto(355.53207275,44.62392873)
\curveto(355.87207507,44.73392622)(356.16207478,44.88392607)(356.40207275,45.07392873)
\curveto(356.6420743,45.27392568)(356.8420741,45.51892544)(357.00207275,45.80892873)
\curveto(357.05207389,45.89892506)(357.09207385,45.99392496)(357.12207275,46.09392873)
\curveto(357.1420738,46.19392476)(357.16707377,46.29892466)(357.19707275,46.40892873)
\curveto(357.21707372,46.4589245)(357.22707371,46.50392445)(357.22707275,46.54392873)
\curveto(357.21707372,46.59392436)(357.21707372,46.64392431)(357.22707275,46.69392873)
\curveto(357.2370737,46.73392422)(357.2420737,46.77892418)(357.24207275,46.82892873)
\lineto(357.24207275,46.96392873)
\lineto(357.24207275,47.09892873)
\curveto(357.23207371,47.13892382)(357.22707371,47.17392378)(357.22707275,47.20392873)
\curveto(357.22707371,47.23392372)(357.22207372,47.26892369)(357.21207275,47.30892873)
\curveto(357.19207375,47.38892357)(357.17707376,47.46392349)(357.16707275,47.53392873)
\curveto(357.14707379,47.60392335)(357.12207382,47.67892328)(357.09207275,47.75892873)
\curveto(356.96207398,48.06892289)(356.79207415,48.31892264)(356.58207275,48.50892873)
\curveto(356.36207458,48.69892226)(356.09707484,48.8589221)(355.78707275,48.98892873)
\curveto(355.64707529,49.03892192)(355.50707543,49.07392188)(355.36707275,49.09392873)
\curveto(355.21707572,49.12392183)(355.06707587,49.1589218)(354.91707275,49.19892873)
\curveto(354.86707607,49.21892174)(354.82207612,49.22392173)(354.78207275,49.21392873)
\curveto(354.73207621,49.21392174)(354.68207626,49.21892174)(354.63207275,49.22892873)
\lineto(354.52707275,49.22892873)
}
}
{
\newrgbcolor{curcolor}{0 0 0}
\pscustom[linestyle=none,fillstyle=solid,fillcolor=curcolor]
{
\newpath
\moveto(350.26707275,55.69017873)
\curveto(350.26708067,55.92017394)(350.32708061,56.05017381)(350.44707275,56.08017873)
\curveto(350.55708038,56.11017375)(350.72208022,56.12517374)(350.94207275,56.12517873)
\lineto(351.22707275,56.12517873)
\curveto(351.31707962,56.12517374)(351.39207955,56.10017376)(351.45207275,56.05017873)
\curveto(351.53207941,55.99017387)(351.57707936,55.90517396)(351.58707275,55.79517873)
\curveto(351.58707935,55.68517418)(351.60207934,55.57517429)(351.63207275,55.46517873)
\curveto(351.66207928,55.32517454)(351.69207925,55.19017467)(351.72207275,55.06017873)
\curveto(351.75207919,54.94017492)(351.79207915,54.82517504)(351.84207275,54.71517873)
\curveto(351.97207897,54.42517544)(352.15207879,54.19017567)(352.38207275,54.01017873)
\curveto(352.60207834,53.83017603)(352.85707808,53.67517619)(353.14707275,53.54517873)
\curveto(353.25707768,53.50517636)(353.37207757,53.47517639)(353.49207275,53.45517873)
\curveto(353.60207734,53.43517643)(353.71707722,53.41017645)(353.83707275,53.38017873)
\curveto(353.88707705,53.37017649)(353.937077,53.3651765)(353.98707275,53.36517873)
\curveto(354.0370769,53.37517649)(354.08707685,53.37517649)(354.13707275,53.36517873)
\curveto(354.25707668,53.33517653)(354.39707654,53.32017654)(354.55707275,53.32017873)
\curveto(354.70707623,53.33017653)(354.85207609,53.33517653)(354.99207275,53.33517873)
\lineto(356.83707275,53.33517873)
\lineto(357.18207275,53.33517873)
\curveto(357.30207364,53.33517653)(357.41707352,53.33017653)(357.52707275,53.32017873)
\curveto(357.6370733,53.31017655)(357.73207321,53.30517656)(357.81207275,53.30517873)
\curveto(357.89207305,53.31517655)(357.96207298,53.29517657)(358.02207275,53.24517873)
\curveto(358.09207285,53.19517667)(358.13207281,53.11517675)(358.14207275,53.00517873)
\curveto(358.15207279,52.90517696)(358.15707278,52.79517707)(358.15707275,52.67517873)
\lineto(358.15707275,52.40517873)
\curveto(358.1370728,52.35517751)(358.12207282,52.30517756)(358.11207275,52.25517873)
\curveto(358.09207285,52.21517765)(358.06707287,52.18517768)(358.03707275,52.16517873)
\curveto(357.96707297,52.11517775)(357.88207306,52.08517778)(357.78207275,52.07517873)
\lineto(357.45207275,52.07517873)
\lineto(356.29707275,52.07517873)
\lineto(352.14207275,52.07517873)
\lineto(351.10707275,52.07517873)
\lineto(350.80707275,52.07517873)
\curveto(350.70708023,52.08517778)(350.62208032,52.11517775)(350.55207275,52.16517873)
\curveto(350.51208043,52.19517767)(350.48208046,52.24517762)(350.46207275,52.31517873)
\curveto(350.4420805,52.39517747)(350.43208051,52.48017738)(350.43207275,52.57017873)
\curveto(350.42208052,52.6601772)(350.42208052,52.75017711)(350.43207275,52.84017873)
\curveto(350.4420805,52.93017693)(350.45708048,53.00017686)(350.47707275,53.05017873)
\curveto(350.50708043,53.13017673)(350.56708037,53.18017668)(350.65707275,53.20017873)
\curveto(350.7370802,53.23017663)(350.82708011,53.24517662)(350.92707275,53.24517873)
\lineto(351.22707275,53.24517873)
\curveto(351.32707961,53.24517662)(351.41707952,53.2651766)(351.49707275,53.30517873)
\curveto(351.51707942,53.31517655)(351.53207941,53.32517654)(351.54207275,53.33517873)
\lineto(351.58707275,53.38017873)
\curveto(351.58707935,53.49017637)(351.5420794,53.58017628)(351.45207275,53.65017873)
\curveto(351.35207959,53.72017614)(351.27207967,53.78017608)(351.21207275,53.83017873)
\lineto(351.12207275,53.92017873)
\curveto(351.01207993,54.01017585)(350.89708004,54.13517573)(350.77707275,54.29517873)
\curveto(350.65708028,54.45517541)(350.56708037,54.60517526)(350.50707275,54.74517873)
\curveto(350.45708048,54.83517503)(350.42208052,54.93017493)(350.40207275,55.03017873)
\curveto(350.37208057,55.13017473)(350.3420806,55.23517463)(350.31207275,55.34517873)
\curveto(350.30208064,55.40517446)(350.29708064,55.4651744)(350.29707275,55.52517873)
\curveto(350.28708065,55.58517428)(350.27708066,55.64017422)(350.26707275,55.69017873)
}
}
{
\newrgbcolor{curcolor}{0 0 0}
\pscustom[linestyle=none,fillstyle=solid,fillcolor=curcolor]
{
}
}
{
\newrgbcolor{curcolor}{0 0 0}
\pscustom[linestyle=none,fillstyle=solid,fillcolor=curcolor]
{
\newpath
\moveto(353.08707275,67.99510061)
\lineto(353.34207275,67.99510061)
\curveto(353.42207752,68.0050929)(353.49707744,68.00009291)(353.56707275,67.98010061)
\lineto(353.80707275,67.98010061)
\lineto(353.97207275,67.98010061)
\curveto(354.07207687,67.96009295)(354.17707676,67.95009296)(354.28707275,67.95010061)
\curveto(354.38707655,67.95009296)(354.48707645,67.94009297)(354.58707275,67.92010061)
\lineto(354.73707275,67.92010061)
\curveto(354.87707606,67.89009302)(355.01707592,67.87009304)(355.15707275,67.86010061)
\curveto(355.28707565,67.85009306)(355.41707552,67.82509308)(355.54707275,67.78510061)
\curveto(355.62707531,67.76509314)(355.71207523,67.74509316)(355.80207275,67.72510061)
\lineto(356.04207275,67.66510061)
\lineto(356.34207275,67.54510061)
\curveto(356.43207451,67.51509339)(356.52207442,67.48009343)(356.61207275,67.44010061)
\curveto(356.83207411,67.34009357)(357.04707389,67.2050937)(357.25707275,67.03510061)
\curveto(357.46707347,66.87509403)(357.6370733,66.70009421)(357.76707275,66.51010061)
\curveto(357.80707313,66.46009445)(357.84707309,66.40009451)(357.88707275,66.33010061)
\curveto(357.91707302,66.27009464)(357.95207299,66.2100947)(357.99207275,66.15010061)
\curveto(358.0420729,66.07009484)(358.08207286,65.97509493)(358.11207275,65.86510061)
\curveto(358.1420728,65.75509515)(358.17207277,65.65009526)(358.20207275,65.55010061)
\curveto(358.2420727,65.44009547)(358.26707267,65.33009558)(358.27707275,65.22010061)
\curveto(358.28707265,65.1100958)(358.30207264,64.99509591)(358.32207275,64.87510061)
\curveto(358.33207261,64.83509607)(358.33207261,64.79009612)(358.32207275,64.74010061)
\curveto(358.32207262,64.70009621)(358.32707261,64.66009625)(358.33707275,64.62010061)
\curveto(358.34707259,64.58009633)(358.35207259,64.52509638)(358.35207275,64.45510061)
\curveto(358.35207259,64.38509652)(358.34707259,64.33509657)(358.33707275,64.30510061)
\curveto(358.31707262,64.25509665)(358.31207263,64.2100967)(358.32207275,64.17010061)
\curveto(358.33207261,64.13009678)(358.33207261,64.09509681)(358.32207275,64.06510061)
\lineto(358.32207275,63.97510061)
\curveto(358.30207264,63.91509699)(358.28707265,63.85009706)(358.27707275,63.78010061)
\curveto(358.27707266,63.72009719)(358.27207267,63.65509725)(358.26207275,63.58510061)
\curveto(358.21207273,63.41509749)(358.16207278,63.25509765)(358.11207275,63.10510061)
\curveto(358.06207288,62.95509795)(357.99707294,62.8100981)(357.91707275,62.67010061)
\curveto(357.87707306,62.62009829)(357.84707309,62.56509834)(357.82707275,62.50510061)
\curveto(357.79707314,62.45509845)(357.76207318,62.4050985)(357.72207275,62.35510061)
\curveto(357.5420734,62.11509879)(357.32207362,61.91509899)(357.06207275,61.75510061)
\curveto(356.80207414,61.59509931)(356.51707442,61.45509945)(356.20707275,61.33510061)
\curveto(356.06707487,61.27509963)(355.92707501,61.23009968)(355.78707275,61.20010061)
\curveto(355.6370753,61.17009974)(355.48207546,61.13509977)(355.32207275,61.09510061)
\curveto(355.21207573,61.07509983)(355.10207584,61.06009985)(354.99207275,61.05010061)
\curveto(354.88207606,61.04009987)(354.77207617,61.02509988)(354.66207275,61.00510061)
\curveto(354.62207632,60.99509991)(354.58207636,60.99009992)(354.54207275,60.99010061)
\curveto(354.50207644,61.00009991)(354.46207648,61.00009991)(354.42207275,60.99010061)
\curveto(354.37207657,60.98009993)(354.32207662,60.97509993)(354.27207275,60.97510061)
\lineto(354.10707275,60.97510061)
\curveto(354.05707688,60.95509995)(354.00707693,60.95009996)(353.95707275,60.96010061)
\curveto(353.89707704,60.97009994)(353.8420771,60.97009994)(353.79207275,60.96010061)
\curveto(353.75207719,60.95009996)(353.70707723,60.95009996)(353.65707275,60.96010061)
\curveto(353.60707733,60.97009994)(353.55707738,60.96509994)(353.50707275,60.94510061)
\curveto(353.4370775,60.92509998)(353.36207758,60.92009999)(353.28207275,60.93010061)
\curveto(353.19207775,60.94009997)(353.10707783,60.94509996)(353.02707275,60.94510061)
\curveto(352.937078,60.94509996)(352.8370781,60.94009997)(352.72707275,60.93010061)
\curveto(352.60707833,60.92009999)(352.50707843,60.92509998)(352.42707275,60.94510061)
\lineto(352.14207275,60.94510061)
\lineto(351.51207275,60.99010061)
\curveto(351.41207953,61.00009991)(351.31707962,61.0100999)(351.22707275,61.02010061)
\lineto(350.92707275,61.05010061)
\curveto(350.87708006,61.07009984)(350.82708011,61.07509983)(350.77707275,61.06510061)
\curveto(350.71708022,61.06509984)(350.66208028,61.07509983)(350.61207275,61.09510061)
\curveto(350.4420805,61.14509976)(350.27708066,61.18509972)(350.11707275,61.21510061)
\curveto(349.94708099,61.24509966)(349.78708115,61.29509961)(349.63707275,61.36510061)
\curveto(349.17708176,61.55509935)(348.80208214,61.77509913)(348.51207275,62.02510061)
\curveto(348.22208272,62.28509862)(347.97708296,62.64509826)(347.77707275,63.10510061)
\curveto(347.72708321,63.23509767)(347.69208325,63.36509754)(347.67207275,63.49510061)
\curveto(347.65208329,63.63509727)(347.62708331,63.77509713)(347.59707275,63.91510061)
\curveto(347.58708335,63.98509692)(347.58208336,64.05009686)(347.58207275,64.11010061)
\curveto(347.58208336,64.17009674)(347.57708336,64.23509667)(347.56707275,64.30510061)
\curveto(347.54708339,65.13509577)(347.69708324,65.8050951)(348.01707275,66.31510061)
\curveto(348.32708261,66.82509408)(348.76708217,67.2050937)(349.33707275,67.45510061)
\curveto(349.45708148,67.5050934)(349.58208136,67.55009336)(349.71207275,67.59010061)
\curveto(349.8420811,67.63009328)(349.97708096,67.67509323)(350.11707275,67.72510061)
\curveto(350.19708074,67.74509316)(350.28208066,67.76009315)(350.37207275,67.77010061)
\lineto(350.61207275,67.83010061)
\curveto(350.72208022,67.86009305)(350.83208011,67.87509303)(350.94207275,67.87510061)
\curveto(351.05207989,67.88509302)(351.16207978,67.90009301)(351.27207275,67.92010061)
\curveto(351.32207962,67.94009297)(351.36707957,67.94509296)(351.40707275,67.93510061)
\curveto(351.44707949,67.93509297)(351.48707945,67.94009297)(351.52707275,67.95010061)
\curveto(351.57707936,67.96009295)(351.63207931,67.96009295)(351.69207275,67.95010061)
\curveto(351.7420792,67.95009296)(351.79207915,67.95509295)(351.84207275,67.96510061)
\lineto(351.97707275,67.96510061)
\curveto(352.0370789,67.98509292)(352.10707883,67.98509292)(352.18707275,67.96510061)
\curveto(352.25707868,67.95509295)(352.32207862,67.96009295)(352.38207275,67.98010061)
\curveto(352.41207853,67.99009292)(352.45207849,67.99509291)(352.50207275,67.99510061)
\lineto(352.62207275,67.99510061)
\lineto(353.08707275,67.99510061)
\moveto(355.41207275,66.45010061)
\curveto(355.09207585,66.55009436)(354.72707621,66.6100943)(354.31707275,66.63010061)
\curveto(353.90707703,66.65009426)(353.49707744,66.66009425)(353.08707275,66.66010061)
\curveto(352.65707828,66.66009425)(352.2370787,66.65009426)(351.82707275,66.63010061)
\curveto(351.41707952,66.6100943)(351.03207991,66.56509434)(350.67207275,66.49510061)
\curveto(350.31208063,66.42509448)(349.99208095,66.31509459)(349.71207275,66.16510061)
\curveto(349.42208152,66.02509488)(349.18708175,65.83009508)(349.00707275,65.58010061)
\curveto(348.89708204,65.42009549)(348.81708212,65.24009567)(348.76707275,65.04010061)
\curveto(348.70708223,64.84009607)(348.67708226,64.59509631)(348.67707275,64.30510061)
\curveto(348.69708224,64.28509662)(348.70708223,64.25009666)(348.70707275,64.20010061)
\curveto(348.69708224,64.15009676)(348.69708224,64.1100968)(348.70707275,64.08010061)
\curveto(348.72708221,64.00009691)(348.74708219,63.92509698)(348.76707275,63.85510061)
\curveto(348.77708216,63.79509711)(348.79708214,63.73009718)(348.82707275,63.66010061)
\curveto(348.94708199,63.39009752)(349.11708182,63.17009774)(349.33707275,63.00010061)
\curveto(349.54708139,62.84009807)(349.79208115,62.7050982)(350.07207275,62.59510061)
\curveto(350.18208076,62.54509836)(350.30208064,62.5050984)(350.43207275,62.47510061)
\curveto(350.55208039,62.45509845)(350.67708026,62.43009848)(350.80707275,62.40010061)
\curveto(350.85708008,62.38009853)(350.91208003,62.37009854)(350.97207275,62.37010061)
\curveto(351.02207992,62.37009854)(351.07207987,62.36509854)(351.12207275,62.35510061)
\curveto(351.21207973,62.34509856)(351.30707963,62.33509857)(351.40707275,62.32510061)
\curveto(351.49707944,62.31509859)(351.59207935,62.3050986)(351.69207275,62.29510061)
\curveto(351.77207917,62.29509861)(351.85707908,62.29009862)(351.94707275,62.28010061)
\lineto(352.18707275,62.28010061)
\lineto(352.36707275,62.28010061)
\curveto(352.39707854,62.27009864)(352.43207851,62.26509864)(352.47207275,62.26510061)
\lineto(352.60707275,62.26510061)
\lineto(353.05707275,62.26510061)
\curveto(353.1370778,62.26509864)(353.22207772,62.26009865)(353.31207275,62.25010061)
\curveto(353.39207755,62.25009866)(353.46707747,62.26009865)(353.53707275,62.28010061)
\lineto(353.80707275,62.28010061)
\curveto(353.82707711,62.28009863)(353.85707708,62.27509863)(353.89707275,62.26510061)
\curveto(353.92707701,62.26509864)(353.95207699,62.27009864)(353.97207275,62.28010061)
\curveto(354.07207687,62.29009862)(354.17207677,62.29509861)(354.27207275,62.29510061)
\curveto(354.36207658,62.3050986)(354.46207648,62.31509859)(354.57207275,62.32510061)
\curveto(354.69207625,62.35509855)(354.81707612,62.37009854)(354.94707275,62.37010061)
\curveto(355.06707587,62.38009853)(355.18207576,62.4050985)(355.29207275,62.44510061)
\curveto(355.59207535,62.52509838)(355.85707508,62.6100983)(356.08707275,62.70010061)
\curveto(356.31707462,62.80009811)(356.53207441,62.94509796)(356.73207275,63.13510061)
\curveto(356.93207401,63.34509756)(357.08207386,63.6100973)(357.18207275,63.93010061)
\curveto(357.20207374,63.97009694)(357.21207373,64.0050969)(357.21207275,64.03510061)
\curveto(357.20207374,64.07509683)(357.20707373,64.12009679)(357.22707275,64.17010061)
\curveto(357.2370737,64.2100967)(357.24707369,64.28009663)(357.25707275,64.38010061)
\curveto(357.26707367,64.49009642)(357.26207368,64.57509633)(357.24207275,64.63510061)
\curveto(357.22207372,64.7050962)(357.21207373,64.77509613)(357.21207275,64.84510061)
\curveto(357.20207374,64.91509599)(357.18707375,64.98009593)(357.16707275,65.04010061)
\curveto(357.10707383,65.24009567)(357.02207392,65.42009549)(356.91207275,65.58010061)
\curveto(356.89207405,65.6100953)(356.87207407,65.63509527)(356.85207275,65.65510061)
\lineto(356.79207275,65.71510061)
\curveto(356.77207417,65.75509515)(356.73207421,65.8050951)(356.67207275,65.86510061)
\curveto(356.53207441,65.96509494)(356.40207454,66.05009486)(356.28207275,66.12010061)
\curveto(356.16207478,66.19009472)(356.01707492,66.26009465)(355.84707275,66.33010061)
\curveto(355.77707516,66.36009455)(355.70707523,66.38009453)(355.63707275,66.39010061)
\curveto(355.56707537,66.4100945)(355.49207545,66.43009448)(355.41207275,66.45010061)
}
}
{
\newrgbcolor{curcolor}{0 0 0}
\pscustom[linestyle=none,fillstyle=solid,fillcolor=curcolor]
{
\newpath
\moveto(347.76207275,69.80470998)
\lineto(347.76207275,74.60470998)
\lineto(347.76207275,75.60970998)
\curveto(347.76208318,75.74970288)(347.77208317,75.86970276)(347.79207275,75.96970998)
\curveto(347.80208314,76.07970255)(347.84708309,76.15970247)(347.92707275,76.20970998)
\curveto(347.96708297,76.2297024)(348.01708292,76.23970239)(348.07707275,76.23970998)
\curveto(348.1370828,76.24970238)(348.20208274,76.25470238)(348.27207275,76.25470998)
\lineto(348.54207275,76.25470998)
\curveto(348.63208231,76.25470238)(348.71208223,76.24470239)(348.78207275,76.22470998)
\curveto(348.86208208,76.18470245)(348.93208201,76.13970249)(348.99207275,76.08970998)
\lineto(349.17207275,75.93970998)
\curveto(349.22208172,75.90970272)(349.26208168,75.87470276)(349.29207275,75.83470998)
\curveto(349.32208162,75.79470284)(349.36208158,75.75470288)(349.41207275,75.71470998)
\curveto(349.52208142,75.634703)(349.63208131,75.54970308)(349.74207275,75.45970998)
\curveto(349.8420811,75.36970326)(349.94708099,75.28470335)(350.05707275,75.20470998)
\curveto(350.25708068,75.06470357)(350.46708047,74.92470371)(350.68707275,74.78470998)
\curveto(350.89708004,74.64470399)(351.11207983,74.50470413)(351.33207275,74.36470998)
\curveto(351.42207952,74.31470432)(351.51707942,74.26470437)(351.61707275,74.21470998)
\curveto(351.71707922,74.16470447)(351.81207913,74.10970452)(351.90207275,74.04970998)
\curveto(351.92207902,74.0297046)(351.94707899,74.01970461)(351.97707275,74.01970998)
\curveto(352.00707893,74.01970461)(352.03207891,74.00970462)(352.05207275,73.98970998)
\curveto(352.15207879,73.91970471)(352.26707867,73.85470478)(352.39707275,73.79470998)
\curveto(352.51707842,73.7347049)(352.63207831,73.67970495)(352.74207275,73.62970998)
\curveto(352.97207797,73.5297051)(353.20707773,73.4347052)(353.44707275,73.34470998)
\curveto(353.68707725,73.25470538)(353.92707701,73.15470548)(354.16707275,73.04470998)
\curveto(354.21707672,73.02470561)(354.26207668,73.00970562)(354.30207275,72.99970998)
\curveto(354.3420766,72.99970563)(354.38707655,72.98970564)(354.43707275,72.96970998)
\curveto(354.55707638,72.91970571)(354.68207626,72.87470576)(354.81207275,72.83470998)
\curveto(354.93207601,72.80470583)(355.05207589,72.76970586)(355.17207275,72.72970998)
\curveto(355.40207554,72.64970598)(355.6420753,72.58470605)(355.89207275,72.53470998)
\curveto(356.13207481,72.49470614)(356.37207457,72.44470619)(356.61207275,72.38470998)
\curveto(356.76207418,72.34470629)(356.91207403,72.31970631)(357.06207275,72.30970998)
\curveto(357.21207373,72.29970633)(357.36207358,72.27970635)(357.51207275,72.24970998)
\curveto(357.55207339,72.23970639)(357.61207333,72.2347064)(357.69207275,72.23470998)
\curveto(357.81207313,72.20470643)(357.91207303,72.17470646)(357.99207275,72.14470998)
\curveto(358.07207287,72.11470652)(358.12707281,72.04470659)(358.15707275,71.93470998)
\curveto(358.17707276,71.88470675)(358.18707275,71.8297068)(358.18707275,71.76970998)
\lineto(358.18707275,71.57470998)
\curveto(358.18707275,71.4347072)(358.18207276,71.29470734)(358.17207275,71.15470998)
\curveto(358.16207278,71.02470761)(358.11707282,70.9297077)(358.03707275,70.86970998)
\curveto(357.97707296,70.8297078)(357.89207305,70.80970782)(357.78207275,70.80970998)
\curveto(357.67207327,70.81970781)(357.57707336,70.8347078)(357.49707275,70.85470998)
\lineto(357.42207275,70.85470998)
\curveto(357.39207355,70.86470777)(357.36207358,70.86970776)(357.33207275,70.86970998)
\curveto(357.25207369,70.88970774)(357.17707376,70.89970773)(357.10707275,70.89970998)
\curveto(357.0370739,70.89970773)(356.96707397,70.90970772)(356.89707275,70.92970998)
\curveto(356.70707423,70.97970765)(356.52207442,71.01970761)(356.34207275,71.04970998)
\curveto(356.15207479,71.07970755)(355.97207497,71.11970751)(355.80207275,71.16970998)
\curveto(355.75207519,71.18970744)(355.71207523,71.19970743)(355.68207275,71.19970998)
\curveto(355.65207529,71.19970743)(355.61707532,71.20470743)(355.57707275,71.21470998)
\curveto(355.27707566,71.31470732)(354.98207596,71.40470723)(354.69207275,71.48470998)
\curveto(354.40207654,71.57470706)(354.12207682,71.67970695)(353.85207275,71.79970998)
\curveto(353.27207767,72.05970657)(352.72207822,72.3297063)(352.20207275,72.60970998)
\curveto(351.67207927,72.88970574)(351.16707977,73.19970543)(350.68707275,73.53970998)
\curveto(350.48708045,73.67970495)(350.29708064,73.8297048)(350.11707275,73.98970998)
\curveto(349.92708101,74.14970448)(349.7370812,74.29970433)(349.54707275,74.43970998)
\curveto(349.49708144,74.47970415)(349.45208149,74.51470412)(349.41207275,74.54470998)
\curveto(349.36208158,74.58470405)(349.31208163,74.61970401)(349.26207275,74.64970998)
\curveto(349.2420817,74.65970397)(349.21708172,74.66970396)(349.18707275,74.67970998)
\curveto(349.15708178,74.69970393)(349.12708181,74.69970393)(349.09707275,74.67970998)
\curveto(349.0370819,74.65970397)(349.00208194,74.62470401)(348.99207275,74.57470998)
\curveto(348.97208197,74.52470411)(348.95208199,74.47470416)(348.93207275,74.42470998)
\lineto(348.93207275,74.31970998)
\curveto(348.92208202,74.27970435)(348.92208202,74.2297044)(348.93207275,74.16970998)
\lineto(348.93207275,74.01970998)
\lineto(348.93207275,73.41970998)
\lineto(348.93207275,70.77970998)
\lineto(348.93207275,70.04470998)
\lineto(348.93207275,69.80470998)
\curveto(348.92208202,69.7347089)(348.90708203,69.67470896)(348.88707275,69.62470998)
\curveto(348.84708209,69.5347091)(348.78708215,69.47470916)(348.70707275,69.44470998)
\curveto(348.60708233,69.39470924)(348.46208248,69.37970925)(348.27207275,69.39970998)
\curveto(348.07208287,69.41970921)(347.937083,69.45470918)(347.86707275,69.50470998)
\curveto(347.84708309,69.52470911)(347.83208311,69.54970908)(347.82207275,69.57970998)
\lineto(347.76207275,69.69970998)
\curveto(347.76208318,69.71970891)(347.76708317,69.7347089)(347.77707275,69.74470998)
\curveto(347.77708316,69.76470887)(347.77208317,69.78470885)(347.76207275,69.80470998)
}
}
{
\newrgbcolor{curcolor}{0 0 0}
\pscustom[linestyle=none,fillstyle=solid,fillcolor=curcolor]
{
\newpath
\moveto(356.53707275,78.63431936)
\lineto(356.53707275,79.26431936)
\lineto(356.53707275,79.45931936)
\curveto(356.5370744,79.52931683)(356.54707439,79.58931677)(356.56707275,79.63931936)
\curveto(356.60707433,79.70931665)(356.64707429,79.7593166)(356.68707275,79.78931936)
\curveto(356.7370742,79.82931653)(356.80207414,79.84931651)(356.88207275,79.84931936)
\curveto(356.96207398,79.8593165)(357.04707389,79.86431649)(357.13707275,79.86431936)
\lineto(357.85707275,79.86431936)
\curveto(358.3370726,79.86431649)(358.74707219,79.80431655)(359.08707275,79.68431936)
\curveto(359.42707151,79.56431679)(359.70207124,79.36931699)(359.91207275,79.09931936)
\curveto(359.96207098,79.02931733)(360.00707093,78.9593174)(360.04707275,78.88931936)
\curveto(360.09707084,78.82931753)(360.1420708,78.7543176)(360.18207275,78.66431936)
\curveto(360.19207075,78.64431771)(360.20207074,78.61431774)(360.21207275,78.57431936)
\curveto(360.23207071,78.53431782)(360.2370707,78.48931787)(360.22707275,78.43931936)
\curveto(360.19707074,78.34931801)(360.12207082,78.29431806)(360.00207275,78.27431936)
\curveto(359.89207105,78.2543181)(359.79707114,78.26931809)(359.71707275,78.31931936)
\curveto(359.64707129,78.34931801)(359.58207136,78.39431796)(359.52207275,78.45431936)
\curveto(359.47207147,78.52431783)(359.42207152,78.58931777)(359.37207275,78.64931936)
\curveto(359.32207162,78.71931764)(359.24707169,78.77931758)(359.14707275,78.82931936)
\curveto(359.05707188,78.88931747)(358.96707197,78.93931742)(358.87707275,78.97931936)
\curveto(358.84707209,78.99931736)(358.78707215,79.02431733)(358.69707275,79.05431936)
\curveto(358.61707232,79.08431727)(358.54707239,79.08931727)(358.48707275,79.06931936)
\curveto(358.34707259,79.03931732)(358.25707268,78.97931738)(358.21707275,78.88931936)
\curveto(358.18707275,78.80931755)(358.17207277,78.71931764)(358.17207275,78.61931936)
\curveto(358.17207277,78.51931784)(358.14707279,78.43431792)(358.09707275,78.36431936)
\curveto(358.02707291,78.27431808)(357.88707305,78.22931813)(357.67707275,78.22931936)
\lineto(357.12207275,78.22931936)
\lineto(356.89707275,78.22931936)
\curveto(356.81707412,78.23931812)(356.75207419,78.2593181)(356.70207275,78.28931936)
\curveto(356.62207432,78.34931801)(356.57707436,78.41931794)(356.56707275,78.49931936)
\curveto(356.55707438,78.51931784)(356.55207439,78.53931782)(356.55207275,78.55931936)
\curveto(356.55207439,78.58931777)(356.54707439,78.61431774)(356.53707275,78.63431936)
}
}
{
\newrgbcolor{curcolor}{0 0 0}
\pscustom[linestyle=none,fillstyle=solid,fillcolor=curcolor]
{
}
}
{
\newrgbcolor{curcolor}{0 0 0}
\pscustom[linestyle=none,fillstyle=solid,fillcolor=curcolor]
{
\newpath
\moveto(347.56707275,89.26463186)
\curveto(347.55708338,89.95462722)(347.67708326,90.55462662)(347.92707275,91.06463186)
\curveto(348.17708276,91.58462559)(348.51208243,91.9796252)(348.93207275,92.24963186)
\curveto(349.01208193,92.29962488)(349.10208184,92.34462483)(349.20207275,92.38463186)
\curveto(349.29208165,92.42462475)(349.38708155,92.46962471)(349.48707275,92.51963186)
\curveto(349.58708135,92.55962462)(349.68708125,92.58962459)(349.78707275,92.60963186)
\curveto(349.88708105,92.62962455)(349.99208095,92.64962453)(350.10207275,92.66963186)
\curveto(350.15208079,92.68962449)(350.19708074,92.69462448)(350.23707275,92.68463186)
\curveto(350.27708066,92.6746245)(350.32208062,92.6796245)(350.37207275,92.69963186)
\curveto(350.42208052,92.70962447)(350.50708043,92.71462446)(350.62707275,92.71463186)
\curveto(350.7370802,92.71462446)(350.82208012,92.70962447)(350.88207275,92.69963186)
\curveto(350.94208,92.6796245)(351.00207994,92.66962451)(351.06207275,92.66963186)
\curveto(351.12207982,92.6796245)(351.18207976,92.6746245)(351.24207275,92.65463186)
\curveto(351.38207956,92.61462456)(351.51707942,92.5796246)(351.64707275,92.54963186)
\curveto(351.77707916,92.51962466)(351.90207904,92.4796247)(352.02207275,92.42963186)
\curveto(352.16207878,92.36962481)(352.28707865,92.29962488)(352.39707275,92.21963186)
\curveto(352.50707843,92.14962503)(352.61707832,92.0746251)(352.72707275,91.99463186)
\lineto(352.78707275,91.93463186)
\curveto(352.80707813,91.92462525)(352.82707811,91.90962527)(352.84707275,91.88963186)
\curveto(353.00707793,91.76962541)(353.15207779,91.63462554)(353.28207275,91.48463186)
\curveto(353.41207753,91.33462584)(353.5370774,91.174626)(353.65707275,91.00463186)
\curveto(353.87707706,90.69462648)(354.08207686,90.39962678)(354.27207275,90.11963186)
\curveto(354.41207653,89.88962729)(354.54707639,89.65962752)(354.67707275,89.42963186)
\curveto(354.80707613,89.20962797)(354.942076,88.98962819)(355.08207275,88.76963186)
\curveto(355.25207569,88.51962866)(355.43207551,88.2796289)(355.62207275,88.04963186)
\curveto(355.81207513,87.82962935)(356.0370749,87.63962954)(356.29707275,87.47963186)
\curveto(356.35707458,87.43962974)(356.41707452,87.40462977)(356.47707275,87.37463186)
\curveto(356.52707441,87.34462983)(356.59207435,87.31462986)(356.67207275,87.28463186)
\curveto(356.7420742,87.26462991)(356.80207414,87.25962992)(356.85207275,87.26963186)
\curveto(356.92207402,87.28962989)(356.97707396,87.32462985)(357.01707275,87.37463186)
\curveto(357.04707389,87.42462975)(357.06707387,87.48462969)(357.07707275,87.55463186)
\lineto(357.07707275,87.79463186)
\lineto(357.07707275,88.54463186)
\lineto(357.07707275,91.34963186)
\lineto(357.07707275,92.00963186)
\curveto(357.07707386,92.09962508)(357.08207386,92.18462499)(357.09207275,92.26463186)
\curveto(357.09207385,92.34462483)(357.11207383,92.40962477)(357.15207275,92.45963186)
\curveto(357.19207375,92.50962467)(357.26707367,92.54962463)(357.37707275,92.57963186)
\curveto(357.47707346,92.61962456)(357.57707336,92.62962455)(357.67707275,92.60963186)
\lineto(357.81207275,92.60963186)
\curveto(357.88207306,92.58962459)(357.942073,92.56962461)(357.99207275,92.54963186)
\curveto(358.0420729,92.52962465)(358.08207286,92.49462468)(358.11207275,92.44463186)
\curveto(358.15207279,92.39462478)(358.17207277,92.32462485)(358.17207275,92.23463186)
\lineto(358.17207275,91.96463186)
\lineto(358.17207275,91.06463186)
\lineto(358.17207275,87.55463186)
\lineto(358.17207275,86.48963186)
\curveto(358.17207277,86.40963077)(358.17707276,86.31963086)(358.18707275,86.21963186)
\curveto(358.18707275,86.11963106)(358.17707276,86.03463114)(358.15707275,85.96463186)
\curveto(358.08707285,85.75463142)(357.90707303,85.68963149)(357.61707275,85.76963186)
\curveto(357.57707336,85.7796314)(357.5420734,85.7796314)(357.51207275,85.76963186)
\curveto(357.47207347,85.76963141)(357.42707351,85.7796314)(357.37707275,85.79963186)
\curveto(357.29707364,85.81963136)(357.21207373,85.83963134)(357.12207275,85.85963186)
\curveto(357.03207391,85.8796313)(356.94707399,85.90463127)(356.86707275,85.93463186)
\curveto(356.37707456,86.09463108)(355.96207498,86.29463088)(355.62207275,86.53463186)
\curveto(355.37207557,86.71463046)(355.14707579,86.91963026)(354.94707275,87.14963186)
\curveto(354.7370762,87.3796298)(354.5420764,87.61962956)(354.36207275,87.86963186)
\curveto(354.18207676,88.12962905)(354.01207693,88.39462878)(353.85207275,88.66463186)
\curveto(353.68207726,88.94462823)(353.50707743,89.21462796)(353.32707275,89.47463186)
\curveto(353.24707769,89.58462759)(353.17207777,89.68962749)(353.10207275,89.78963186)
\curveto(353.03207791,89.89962728)(352.95707798,90.00962717)(352.87707275,90.11963186)
\curveto(352.84707809,90.15962702)(352.81707812,90.19462698)(352.78707275,90.22463186)
\curveto(352.74707819,90.26462691)(352.71707822,90.30462687)(352.69707275,90.34463186)
\curveto(352.58707835,90.48462669)(352.46207848,90.60962657)(352.32207275,90.71963186)
\curveto(352.29207865,90.73962644)(352.26707867,90.76462641)(352.24707275,90.79463186)
\curveto(352.21707872,90.82462635)(352.18707875,90.84962633)(352.15707275,90.86963186)
\curveto(352.05707888,90.94962623)(351.95707898,91.01462616)(351.85707275,91.06463186)
\curveto(351.75707918,91.12462605)(351.64707929,91.179626)(351.52707275,91.22963186)
\curveto(351.45707948,91.25962592)(351.38207956,91.2796259)(351.30207275,91.28963186)
\lineto(351.06207275,91.34963186)
\lineto(350.97207275,91.34963186)
\curveto(350.94208,91.35962582)(350.91208003,91.36462581)(350.88207275,91.36463186)
\curveto(350.81208013,91.38462579)(350.71708022,91.38962579)(350.59707275,91.37963186)
\curveto(350.46708047,91.3796258)(350.36708057,91.36962581)(350.29707275,91.34963186)
\curveto(350.21708072,91.32962585)(350.1420808,91.30962587)(350.07207275,91.28963186)
\curveto(349.99208095,91.2796259)(349.91208103,91.25962592)(349.83207275,91.22963186)
\curveto(349.59208135,91.11962606)(349.39208155,90.96962621)(349.23207275,90.77963186)
\curveto(349.06208188,90.59962658)(348.92208202,90.3796268)(348.81207275,90.11963186)
\curveto(348.79208215,90.04962713)(348.77708216,89.9796272)(348.76707275,89.90963186)
\curveto(348.74708219,89.83962734)(348.72708221,89.76462741)(348.70707275,89.68463186)
\curveto(348.68708225,89.60462757)(348.67708226,89.49462768)(348.67707275,89.35463186)
\curveto(348.67708226,89.22462795)(348.68708225,89.11962806)(348.70707275,89.03963186)
\curveto(348.71708222,88.9796282)(348.72208222,88.92462825)(348.72207275,88.87463186)
\curveto(348.72208222,88.82462835)(348.73208221,88.7746284)(348.75207275,88.72463186)
\curveto(348.79208215,88.62462855)(348.83208211,88.52962865)(348.87207275,88.43963186)
\curveto(348.91208203,88.35962882)(348.95708198,88.2796289)(349.00707275,88.19963186)
\curveto(349.02708191,88.16962901)(349.05208189,88.13962904)(349.08207275,88.10963186)
\curveto(349.11208183,88.08962909)(349.1370818,88.06462911)(349.15707275,88.03463186)
\lineto(349.23207275,87.95963186)
\curveto(349.25208169,87.92962925)(349.27208167,87.90462927)(349.29207275,87.88463186)
\lineto(349.50207275,87.73463186)
\curveto(349.56208138,87.69462948)(349.62708131,87.64962953)(349.69707275,87.59963186)
\curveto(349.78708115,87.53962964)(349.89208105,87.48962969)(350.01207275,87.44963186)
\curveto(350.12208082,87.41962976)(350.23208071,87.38462979)(350.34207275,87.34463186)
\curveto(350.45208049,87.30462987)(350.59708034,87.2796299)(350.77707275,87.26963186)
\curveto(350.94707999,87.25962992)(351.07207987,87.22962995)(351.15207275,87.17963186)
\curveto(351.23207971,87.12963005)(351.27707966,87.05463012)(351.28707275,86.95463186)
\curveto(351.29707964,86.85463032)(351.30207964,86.74463043)(351.30207275,86.62463186)
\curveto(351.30207964,86.58463059)(351.30707963,86.54463063)(351.31707275,86.50463186)
\curveto(351.31707962,86.46463071)(351.31207963,86.42963075)(351.30207275,86.39963186)
\curveto(351.28207966,86.34963083)(351.27207967,86.29963088)(351.27207275,86.24963186)
\curveto(351.27207967,86.20963097)(351.26207968,86.16963101)(351.24207275,86.12963186)
\curveto(351.18207976,86.03963114)(351.04707989,85.99463118)(350.83707275,85.99463186)
\lineto(350.71707275,85.99463186)
\curveto(350.65708028,86.00463117)(350.59708034,86.00963117)(350.53707275,86.00963186)
\curveto(350.46708047,86.01963116)(350.40208054,86.02963115)(350.34207275,86.03963186)
\curveto(350.23208071,86.05963112)(350.13208081,86.0796311)(350.04207275,86.09963186)
\curveto(349.942081,86.11963106)(349.84708109,86.14963103)(349.75707275,86.18963186)
\curveto(349.68708125,86.20963097)(349.62708131,86.22963095)(349.57707275,86.24963186)
\lineto(349.39707275,86.30963186)
\curveto(349.1370818,86.42963075)(348.89208205,86.58463059)(348.66207275,86.77463186)
\curveto(348.43208251,86.9746302)(348.24708269,87.18962999)(348.10707275,87.41963186)
\curveto(348.02708291,87.52962965)(347.96208298,87.64462953)(347.91207275,87.76463186)
\lineto(347.76207275,88.15463186)
\curveto(347.71208323,88.26462891)(347.68208326,88.3796288)(347.67207275,88.49963186)
\curveto(347.65208329,88.61962856)(347.62708331,88.74462843)(347.59707275,88.87463186)
\curveto(347.59708334,88.94462823)(347.59708334,89.00962817)(347.59707275,89.06963186)
\curveto(347.58708335,89.12962805)(347.57708336,89.19462798)(347.56707275,89.26463186)
}
}
{
\newrgbcolor{curcolor}{0 0 0}
\pscustom[linestyle=none,fillstyle=solid,fillcolor=curcolor]
{
\newpath
\moveto(353.08707275,101.36424123)
\lineto(353.34207275,101.36424123)
\curveto(353.42207752,101.37423353)(353.49707744,101.36923353)(353.56707275,101.34924123)
\lineto(353.80707275,101.34924123)
\lineto(353.97207275,101.34924123)
\curveto(354.07207687,101.32923357)(354.17707676,101.31923358)(354.28707275,101.31924123)
\curveto(354.38707655,101.31923358)(354.48707645,101.30923359)(354.58707275,101.28924123)
\lineto(354.73707275,101.28924123)
\curveto(354.87707606,101.25923364)(355.01707592,101.23923366)(355.15707275,101.22924123)
\curveto(355.28707565,101.21923368)(355.41707552,101.19423371)(355.54707275,101.15424123)
\curveto(355.62707531,101.13423377)(355.71207523,101.11423379)(355.80207275,101.09424123)
\lineto(356.04207275,101.03424123)
\lineto(356.34207275,100.91424123)
\curveto(356.43207451,100.88423402)(356.52207442,100.84923405)(356.61207275,100.80924123)
\curveto(356.83207411,100.70923419)(357.04707389,100.57423433)(357.25707275,100.40424123)
\curveto(357.46707347,100.24423466)(357.6370733,100.06923483)(357.76707275,99.87924123)
\curveto(357.80707313,99.82923507)(357.84707309,99.76923513)(357.88707275,99.69924123)
\curveto(357.91707302,99.63923526)(357.95207299,99.57923532)(357.99207275,99.51924123)
\curveto(358.0420729,99.43923546)(358.08207286,99.34423556)(358.11207275,99.23424123)
\curveto(358.1420728,99.12423578)(358.17207277,99.01923588)(358.20207275,98.91924123)
\curveto(358.2420727,98.80923609)(358.26707267,98.6992362)(358.27707275,98.58924123)
\curveto(358.28707265,98.47923642)(358.30207264,98.36423654)(358.32207275,98.24424123)
\curveto(358.33207261,98.2042367)(358.33207261,98.15923674)(358.32207275,98.10924123)
\curveto(358.32207262,98.06923683)(358.32707261,98.02923687)(358.33707275,97.98924123)
\curveto(358.34707259,97.94923695)(358.35207259,97.89423701)(358.35207275,97.82424123)
\curveto(358.35207259,97.75423715)(358.34707259,97.7042372)(358.33707275,97.67424123)
\curveto(358.31707262,97.62423728)(358.31207263,97.57923732)(358.32207275,97.53924123)
\curveto(358.33207261,97.4992374)(358.33207261,97.46423744)(358.32207275,97.43424123)
\lineto(358.32207275,97.34424123)
\curveto(358.30207264,97.28423762)(358.28707265,97.21923768)(358.27707275,97.14924123)
\curveto(358.27707266,97.08923781)(358.27207267,97.02423788)(358.26207275,96.95424123)
\curveto(358.21207273,96.78423812)(358.16207278,96.62423828)(358.11207275,96.47424123)
\curveto(358.06207288,96.32423858)(357.99707294,96.17923872)(357.91707275,96.03924123)
\curveto(357.87707306,95.98923891)(357.84707309,95.93423897)(357.82707275,95.87424123)
\curveto(357.79707314,95.82423908)(357.76207318,95.77423913)(357.72207275,95.72424123)
\curveto(357.5420734,95.48423942)(357.32207362,95.28423962)(357.06207275,95.12424123)
\curveto(356.80207414,94.96423994)(356.51707442,94.82424008)(356.20707275,94.70424123)
\curveto(356.06707487,94.64424026)(355.92707501,94.5992403)(355.78707275,94.56924123)
\curveto(355.6370753,94.53924036)(355.48207546,94.5042404)(355.32207275,94.46424123)
\curveto(355.21207573,94.44424046)(355.10207584,94.42924047)(354.99207275,94.41924123)
\curveto(354.88207606,94.40924049)(354.77207617,94.39424051)(354.66207275,94.37424123)
\curveto(354.62207632,94.36424054)(354.58207636,94.35924054)(354.54207275,94.35924123)
\curveto(354.50207644,94.36924053)(354.46207648,94.36924053)(354.42207275,94.35924123)
\curveto(354.37207657,94.34924055)(354.32207662,94.34424056)(354.27207275,94.34424123)
\lineto(354.10707275,94.34424123)
\curveto(354.05707688,94.32424058)(354.00707693,94.31924058)(353.95707275,94.32924123)
\curveto(353.89707704,94.33924056)(353.8420771,94.33924056)(353.79207275,94.32924123)
\curveto(353.75207719,94.31924058)(353.70707723,94.31924058)(353.65707275,94.32924123)
\curveto(353.60707733,94.33924056)(353.55707738,94.33424057)(353.50707275,94.31424123)
\curveto(353.4370775,94.29424061)(353.36207758,94.28924061)(353.28207275,94.29924123)
\curveto(353.19207775,94.30924059)(353.10707783,94.31424059)(353.02707275,94.31424123)
\curveto(352.937078,94.31424059)(352.8370781,94.30924059)(352.72707275,94.29924123)
\curveto(352.60707833,94.28924061)(352.50707843,94.29424061)(352.42707275,94.31424123)
\lineto(352.14207275,94.31424123)
\lineto(351.51207275,94.35924123)
\curveto(351.41207953,94.36924053)(351.31707962,94.37924052)(351.22707275,94.38924123)
\lineto(350.92707275,94.41924123)
\curveto(350.87708006,94.43924046)(350.82708011,94.44424046)(350.77707275,94.43424123)
\curveto(350.71708022,94.43424047)(350.66208028,94.44424046)(350.61207275,94.46424123)
\curveto(350.4420805,94.51424039)(350.27708066,94.55424035)(350.11707275,94.58424123)
\curveto(349.94708099,94.61424029)(349.78708115,94.66424024)(349.63707275,94.73424123)
\curveto(349.17708176,94.92423998)(348.80208214,95.14423976)(348.51207275,95.39424123)
\curveto(348.22208272,95.65423925)(347.97708296,96.01423889)(347.77707275,96.47424123)
\curveto(347.72708321,96.6042383)(347.69208325,96.73423817)(347.67207275,96.86424123)
\curveto(347.65208329,97.0042379)(347.62708331,97.14423776)(347.59707275,97.28424123)
\curveto(347.58708335,97.35423755)(347.58208336,97.41923748)(347.58207275,97.47924123)
\curveto(347.58208336,97.53923736)(347.57708336,97.6042373)(347.56707275,97.67424123)
\curveto(347.54708339,98.5042364)(347.69708324,99.17423573)(348.01707275,99.68424123)
\curveto(348.32708261,100.19423471)(348.76708217,100.57423433)(349.33707275,100.82424123)
\curveto(349.45708148,100.87423403)(349.58208136,100.91923398)(349.71207275,100.95924123)
\curveto(349.8420811,100.9992339)(349.97708096,101.04423386)(350.11707275,101.09424123)
\curveto(350.19708074,101.11423379)(350.28208066,101.12923377)(350.37207275,101.13924123)
\lineto(350.61207275,101.19924123)
\curveto(350.72208022,101.22923367)(350.83208011,101.24423366)(350.94207275,101.24424123)
\curveto(351.05207989,101.25423365)(351.16207978,101.26923363)(351.27207275,101.28924123)
\curveto(351.32207962,101.30923359)(351.36707957,101.31423359)(351.40707275,101.30424123)
\curveto(351.44707949,101.3042336)(351.48707945,101.30923359)(351.52707275,101.31924123)
\curveto(351.57707936,101.32923357)(351.63207931,101.32923357)(351.69207275,101.31924123)
\curveto(351.7420792,101.31923358)(351.79207915,101.32423358)(351.84207275,101.33424123)
\lineto(351.97707275,101.33424123)
\curveto(352.0370789,101.35423355)(352.10707883,101.35423355)(352.18707275,101.33424123)
\curveto(352.25707868,101.32423358)(352.32207862,101.32923357)(352.38207275,101.34924123)
\curveto(352.41207853,101.35923354)(352.45207849,101.36423354)(352.50207275,101.36424123)
\lineto(352.62207275,101.36424123)
\lineto(353.08707275,101.36424123)
\moveto(355.41207275,99.81924123)
\curveto(355.09207585,99.91923498)(354.72707621,99.97923492)(354.31707275,99.99924123)
\curveto(353.90707703,100.01923488)(353.49707744,100.02923487)(353.08707275,100.02924123)
\curveto(352.65707828,100.02923487)(352.2370787,100.01923488)(351.82707275,99.99924123)
\curveto(351.41707952,99.97923492)(351.03207991,99.93423497)(350.67207275,99.86424123)
\curveto(350.31208063,99.79423511)(349.99208095,99.68423522)(349.71207275,99.53424123)
\curveto(349.42208152,99.39423551)(349.18708175,99.1992357)(349.00707275,98.94924123)
\curveto(348.89708204,98.78923611)(348.81708212,98.60923629)(348.76707275,98.40924123)
\curveto(348.70708223,98.20923669)(348.67708226,97.96423694)(348.67707275,97.67424123)
\curveto(348.69708224,97.65423725)(348.70708223,97.61923728)(348.70707275,97.56924123)
\curveto(348.69708224,97.51923738)(348.69708224,97.47923742)(348.70707275,97.44924123)
\curveto(348.72708221,97.36923753)(348.74708219,97.29423761)(348.76707275,97.22424123)
\curveto(348.77708216,97.16423774)(348.79708214,97.0992378)(348.82707275,97.02924123)
\curveto(348.94708199,96.75923814)(349.11708182,96.53923836)(349.33707275,96.36924123)
\curveto(349.54708139,96.20923869)(349.79208115,96.07423883)(350.07207275,95.96424123)
\curveto(350.18208076,95.91423899)(350.30208064,95.87423903)(350.43207275,95.84424123)
\curveto(350.55208039,95.82423908)(350.67708026,95.7992391)(350.80707275,95.76924123)
\curveto(350.85708008,95.74923915)(350.91208003,95.73923916)(350.97207275,95.73924123)
\curveto(351.02207992,95.73923916)(351.07207987,95.73423917)(351.12207275,95.72424123)
\curveto(351.21207973,95.71423919)(351.30707963,95.7042392)(351.40707275,95.69424123)
\curveto(351.49707944,95.68423922)(351.59207935,95.67423923)(351.69207275,95.66424123)
\curveto(351.77207917,95.66423924)(351.85707908,95.65923924)(351.94707275,95.64924123)
\lineto(352.18707275,95.64924123)
\lineto(352.36707275,95.64924123)
\curveto(352.39707854,95.63923926)(352.43207851,95.63423927)(352.47207275,95.63424123)
\lineto(352.60707275,95.63424123)
\lineto(353.05707275,95.63424123)
\curveto(353.1370778,95.63423927)(353.22207772,95.62923927)(353.31207275,95.61924123)
\curveto(353.39207755,95.61923928)(353.46707747,95.62923927)(353.53707275,95.64924123)
\lineto(353.80707275,95.64924123)
\curveto(353.82707711,95.64923925)(353.85707708,95.64423926)(353.89707275,95.63424123)
\curveto(353.92707701,95.63423927)(353.95207699,95.63923926)(353.97207275,95.64924123)
\curveto(354.07207687,95.65923924)(354.17207677,95.66423924)(354.27207275,95.66424123)
\curveto(354.36207658,95.67423923)(354.46207648,95.68423922)(354.57207275,95.69424123)
\curveto(354.69207625,95.72423918)(354.81707612,95.73923916)(354.94707275,95.73924123)
\curveto(355.06707587,95.74923915)(355.18207576,95.77423913)(355.29207275,95.81424123)
\curveto(355.59207535,95.89423901)(355.85707508,95.97923892)(356.08707275,96.06924123)
\curveto(356.31707462,96.16923873)(356.53207441,96.31423859)(356.73207275,96.50424123)
\curveto(356.93207401,96.71423819)(357.08207386,96.97923792)(357.18207275,97.29924123)
\curveto(357.20207374,97.33923756)(357.21207373,97.37423753)(357.21207275,97.40424123)
\curveto(357.20207374,97.44423746)(357.20707373,97.48923741)(357.22707275,97.53924123)
\curveto(357.2370737,97.57923732)(357.24707369,97.64923725)(357.25707275,97.74924123)
\curveto(357.26707367,97.85923704)(357.26207368,97.94423696)(357.24207275,98.00424123)
\curveto(357.22207372,98.07423683)(357.21207373,98.14423676)(357.21207275,98.21424123)
\curveto(357.20207374,98.28423662)(357.18707375,98.34923655)(357.16707275,98.40924123)
\curveto(357.10707383,98.60923629)(357.02207392,98.78923611)(356.91207275,98.94924123)
\curveto(356.89207405,98.97923592)(356.87207407,99.0042359)(356.85207275,99.02424123)
\lineto(356.79207275,99.08424123)
\curveto(356.77207417,99.12423578)(356.73207421,99.17423573)(356.67207275,99.23424123)
\curveto(356.53207441,99.33423557)(356.40207454,99.41923548)(356.28207275,99.48924123)
\curveto(356.16207478,99.55923534)(356.01707492,99.62923527)(355.84707275,99.69924123)
\curveto(355.77707516,99.72923517)(355.70707523,99.74923515)(355.63707275,99.75924123)
\curveto(355.56707537,99.77923512)(355.49207545,99.7992351)(355.41207275,99.81924123)
}
}
{
\newrgbcolor{curcolor}{0 0 0}
\pscustom[linestyle=none,fillstyle=solid,fillcolor=curcolor]
{
\newpath
\moveto(347.56707275,106.77385061)
\curveto(347.56708337,106.87384575)(347.57708336,106.96884566)(347.59707275,107.05885061)
\curveto(347.60708333,107.14884548)(347.6370833,107.21384541)(347.68707275,107.25385061)
\curveto(347.76708317,107.31384531)(347.87208307,107.34384528)(348.00207275,107.34385061)
\lineto(348.39207275,107.34385061)
\lineto(349.89207275,107.34385061)
\lineto(356.28207275,107.34385061)
\lineto(357.45207275,107.34385061)
\lineto(357.76707275,107.34385061)
\curveto(357.86707307,107.35384527)(357.94707299,107.33884529)(358.00707275,107.29885061)
\curveto(358.08707285,107.24884538)(358.1370728,107.17384545)(358.15707275,107.07385061)
\curveto(358.16707277,106.98384564)(358.17207277,106.87384575)(358.17207275,106.74385061)
\lineto(358.17207275,106.51885061)
\curveto(358.15207279,106.43884619)(358.1370728,106.36884626)(358.12707275,106.30885061)
\curveto(358.10707283,106.24884638)(358.06707287,106.19884643)(358.00707275,106.15885061)
\curveto(357.94707299,106.11884651)(357.87207307,106.09884653)(357.78207275,106.09885061)
\lineto(357.48207275,106.09885061)
\lineto(356.38707275,106.09885061)
\lineto(351.04707275,106.09885061)
\curveto(350.95707998,106.07884655)(350.88208006,106.06384656)(350.82207275,106.05385061)
\curveto(350.75208019,106.05384657)(350.69208025,106.0238466)(350.64207275,105.96385061)
\curveto(350.59208035,105.89384673)(350.56708037,105.80384682)(350.56707275,105.69385061)
\curveto(350.55708038,105.59384703)(350.55208039,105.48384714)(350.55207275,105.36385061)
\lineto(350.55207275,104.22385061)
\lineto(350.55207275,103.72885061)
\curveto(350.5420804,103.56884906)(350.48208046,103.45884917)(350.37207275,103.39885061)
\curveto(350.3420806,103.37884925)(350.31208063,103.36884926)(350.28207275,103.36885061)
\curveto(350.2420807,103.36884926)(350.19708074,103.36384926)(350.14707275,103.35385061)
\curveto(350.02708091,103.33384929)(349.91708102,103.33884929)(349.81707275,103.36885061)
\curveto(349.71708122,103.40884922)(349.64708129,103.46384916)(349.60707275,103.53385061)
\curveto(349.55708138,103.61384901)(349.53208141,103.73384889)(349.53207275,103.89385061)
\curveto(349.53208141,104.05384857)(349.51708142,104.18884844)(349.48707275,104.29885061)
\curveto(349.47708146,104.34884828)(349.47208147,104.40384822)(349.47207275,104.46385061)
\curveto(349.46208148,104.5238481)(349.44708149,104.58384804)(349.42707275,104.64385061)
\curveto(349.37708156,104.79384783)(349.32708161,104.93884769)(349.27707275,105.07885061)
\curveto(349.21708172,105.21884741)(349.14708179,105.35384727)(349.06707275,105.48385061)
\curveto(348.97708196,105.623847)(348.87208207,105.74384688)(348.75207275,105.84385061)
\curveto(348.63208231,105.94384668)(348.50208244,106.03884659)(348.36207275,106.12885061)
\curveto(348.26208268,106.18884644)(348.15208279,106.23384639)(348.03207275,106.26385061)
\curveto(347.91208303,106.30384632)(347.80708313,106.35384627)(347.71707275,106.41385061)
\curveto(347.65708328,106.46384616)(347.61708332,106.53384609)(347.59707275,106.62385061)
\curveto(347.58708335,106.64384598)(347.58208336,106.66884596)(347.58207275,106.69885061)
\curveto(347.58208336,106.7288459)(347.57708336,106.75384587)(347.56707275,106.77385061)
}
}
{
\newrgbcolor{curcolor}{0 0 0}
\pscustom[linestyle=none,fillstyle=solid,fillcolor=curcolor]
{
\newpath
\moveto(347.56707275,115.12345998)
\curveto(347.56708337,115.22345513)(347.57708336,115.31845503)(347.59707275,115.40845998)
\curveto(347.60708333,115.49845485)(347.6370833,115.56345479)(347.68707275,115.60345998)
\curveto(347.76708317,115.66345469)(347.87208307,115.69345466)(348.00207275,115.69345998)
\lineto(348.39207275,115.69345998)
\lineto(349.89207275,115.69345998)
\lineto(356.28207275,115.69345998)
\lineto(357.45207275,115.69345998)
\lineto(357.76707275,115.69345998)
\curveto(357.86707307,115.70345465)(357.94707299,115.68845466)(358.00707275,115.64845998)
\curveto(358.08707285,115.59845475)(358.1370728,115.52345483)(358.15707275,115.42345998)
\curveto(358.16707277,115.33345502)(358.17207277,115.22345513)(358.17207275,115.09345998)
\lineto(358.17207275,114.86845998)
\curveto(358.15207279,114.78845556)(358.1370728,114.71845563)(358.12707275,114.65845998)
\curveto(358.10707283,114.59845575)(358.06707287,114.5484558)(358.00707275,114.50845998)
\curveto(357.94707299,114.46845588)(357.87207307,114.4484559)(357.78207275,114.44845998)
\lineto(357.48207275,114.44845998)
\lineto(356.38707275,114.44845998)
\lineto(351.04707275,114.44845998)
\curveto(350.95707998,114.42845592)(350.88208006,114.41345594)(350.82207275,114.40345998)
\curveto(350.75208019,114.40345595)(350.69208025,114.37345598)(350.64207275,114.31345998)
\curveto(350.59208035,114.24345611)(350.56708037,114.1534562)(350.56707275,114.04345998)
\curveto(350.55708038,113.94345641)(350.55208039,113.83345652)(350.55207275,113.71345998)
\lineto(350.55207275,112.57345998)
\lineto(350.55207275,112.07845998)
\curveto(350.5420804,111.91845843)(350.48208046,111.80845854)(350.37207275,111.74845998)
\curveto(350.3420806,111.72845862)(350.31208063,111.71845863)(350.28207275,111.71845998)
\curveto(350.2420807,111.71845863)(350.19708074,111.71345864)(350.14707275,111.70345998)
\curveto(350.02708091,111.68345867)(349.91708102,111.68845866)(349.81707275,111.71845998)
\curveto(349.71708122,111.75845859)(349.64708129,111.81345854)(349.60707275,111.88345998)
\curveto(349.55708138,111.96345839)(349.53208141,112.08345827)(349.53207275,112.24345998)
\curveto(349.53208141,112.40345795)(349.51708142,112.53845781)(349.48707275,112.64845998)
\curveto(349.47708146,112.69845765)(349.47208147,112.7534576)(349.47207275,112.81345998)
\curveto(349.46208148,112.87345748)(349.44708149,112.93345742)(349.42707275,112.99345998)
\curveto(349.37708156,113.14345721)(349.32708161,113.28845706)(349.27707275,113.42845998)
\curveto(349.21708172,113.56845678)(349.14708179,113.70345665)(349.06707275,113.83345998)
\curveto(348.97708196,113.97345638)(348.87208207,114.09345626)(348.75207275,114.19345998)
\curveto(348.63208231,114.29345606)(348.50208244,114.38845596)(348.36207275,114.47845998)
\curveto(348.26208268,114.53845581)(348.15208279,114.58345577)(348.03207275,114.61345998)
\curveto(347.91208303,114.6534557)(347.80708313,114.70345565)(347.71707275,114.76345998)
\curveto(347.65708328,114.81345554)(347.61708332,114.88345547)(347.59707275,114.97345998)
\curveto(347.58708335,114.99345536)(347.58208336,115.01845533)(347.58207275,115.04845998)
\curveto(347.58208336,115.07845527)(347.57708336,115.10345525)(347.56707275,115.12345998)
}
}
{
\newrgbcolor{curcolor}{0 0 0}
\pscustom[linestyle=none,fillstyle=solid,fillcolor=curcolor]
{
\newpath
\moveto(378.30341919,42.02236623)
\curveto(378.35341993,42.04235669)(378.41341987,42.06735666)(378.48341919,42.09736623)
\curveto(378.55341973,42.1273566)(378.62841966,42.14735658)(378.70841919,42.15736623)
\curveto(378.77841951,42.17735655)(378.84841944,42.17735655)(378.91841919,42.15736623)
\curveto(378.97841931,42.14735658)(379.02341926,42.10735662)(379.05341919,42.03736623)
\curveto(379.07341921,41.98735674)(379.0834192,41.9273568)(379.08341919,41.85736623)
\lineto(379.08341919,41.64736623)
\lineto(379.08341919,41.19736623)
\curveto(379.0834192,41.04735768)(379.05841923,40.9273578)(379.00841919,40.83736623)
\curveto(378.94841934,40.73735799)(378.84341944,40.66235807)(378.69341919,40.61236623)
\curveto(378.54341974,40.57235816)(378.40841988,40.5273582)(378.28841919,40.47736623)
\curveto(378.02842026,40.36735836)(377.75842053,40.26735846)(377.47841919,40.17736623)
\curveto(377.19842109,40.08735864)(376.92342136,39.98735874)(376.65341919,39.87736623)
\curveto(376.56342172,39.84735888)(376.47842181,39.81735891)(376.39841919,39.78736623)
\curveto(376.31842197,39.76735896)(376.24342204,39.73735899)(376.17341919,39.69736623)
\curveto(376.10342218,39.66735906)(376.04342224,39.62235911)(375.99341919,39.56236623)
\curveto(375.94342234,39.50235923)(375.90342238,39.42235931)(375.87341919,39.32236623)
\curveto(375.85342243,39.27235946)(375.84842244,39.21235952)(375.85841919,39.14236623)
\lineto(375.85841919,38.94736623)
\lineto(375.85841919,36.11236623)
\lineto(375.85841919,35.81236623)
\curveto(375.84842244,35.70236303)(375.84842244,35.59736313)(375.85841919,35.49736623)
\curveto(375.86842242,35.39736333)(375.8834224,35.30236343)(375.90341919,35.21236623)
\curveto(375.92342236,35.1323636)(375.96342232,35.07236366)(376.02341919,35.03236623)
\curveto(376.12342216,34.95236378)(376.23842205,34.89236384)(376.36841919,34.85236623)
\curveto(376.4884218,34.82236391)(376.61342167,34.78236395)(376.74341919,34.73236623)
\curveto(376.97342131,34.6323641)(377.21342107,34.53736419)(377.46341919,34.44736623)
\curveto(377.71342057,34.36736436)(377.95342033,34.27736445)(378.18341919,34.17736623)
\curveto(378.24342004,34.15736457)(378.31341997,34.1323646)(378.39341919,34.10236623)
\curveto(378.46341982,34.08236465)(378.53841975,34.05736467)(378.61841919,34.02736623)
\curveto(378.69841959,33.99736473)(378.77341951,33.96236477)(378.84341919,33.92236623)
\curveto(378.90341938,33.89236484)(378.94841934,33.85736487)(378.97841919,33.81736623)
\curveto(379.03841925,33.73736499)(379.07341921,33.6273651)(379.08341919,33.48736623)
\lineto(379.08341919,33.06736623)
\lineto(379.08341919,32.82736623)
\curveto(379.07341921,32.75736597)(379.04841924,32.69736603)(379.00841919,32.64736623)
\curveto(378.97841931,32.59736613)(378.93341935,32.56736616)(378.87341919,32.55736623)
\curveto(378.81341947,32.55736617)(378.75341953,32.56236617)(378.69341919,32.57236623)
\curveto(378.62341966,32.59236614)(378.55841973,32.61236612)(378.49841919,32.63236623)
\curveto(378.42841986,32.66236607)(378.37841991,32.68736604)(378.34841919,32.70736623)
\curveto(378.02842026,32.84736588)(377.71342057,32.97236576)(377.40341919,33.08236623)
\curveto(377.0834212,33.19236554)(376.76342152,33.31236542)(376.44341919,33.44236623)
\curveto(376.22342206,33.5323652)(376.00842228,33.61736511)(375.79841919,33.69736623)
\curveto(375.57842271,33.77736495)(375.35842293,33.86236487)(375.13841919,33.95236623)
\curveto(374.41842387,34.25236448)(373.69342459,34.53736419)(372.96341919,34.80736623)
\curveto(372.22342606,35.07736365)(371.4884268,35.36236337)(370.75841919,35.66236623)
\curveto(370.49842779,35.77236296)(370.23342805,35.87236286)(369.96341919,35.96236623)
\curveto(369.69342859,36.06236267)(369.42842886,36.16736256)(369.16841919,36.27736623)
\curveto(369.05842923,36.3273624)(368.93842935,36.37236236)(368.80841919,36.41236623)
\curveto(368.66842962,36.46236227)(368.56842972,36.5323622)(368.50841919,36.62236623)
\curveto(368.46842982,36.66236207)(368.43842985,36.727362)(368.41841919,36.81736623)
\curveto(368.40842988,36.83736189)(368.40842988,36.85736187)(368.41841919,36.87736623)
\curveto(368.41842987,36.90736182)(368.41342987,36.9323618)(368.40341919,36.95236623)
\curveto(368.40342988,37.1323616)(368.40342988,37.34236139)(368.40341919,37.58236623)
\curveto(368.39342989,37.82236091)(368.42842986,37.99736073)(368.50841919,38.10736623)
\curveto(368.56842972,38.18736054)(368.66842962,38.24736048)(368.80841919,38.28736623)
\curveto(368.93842935,38.33736039)(369.05842923,38.38736034)(369.16841919,38.43736623)
\curveto(369.39842889,38.53736019)(369.62842866,38.6273601)(369.85841919,38.70736623)
\curveto(370.0884282,38.78735994)(370.31842797,38.87735985)(370.54841919,38.97736623)
\curveto(370.74842754,39.05735967)(370.95342733,39.1323596)(371.16341919,39.20236623)
\curveto(371.37342691,39.28235945)(371.57842671,39.36735936)(371.77841919,39.45736623)
\curveto(372.50842578,39.75735897)(373.24842504,40.04235869)(373.99841919,40.31236623)
\curveto(374.73842355,40.59235814)(375.47342281,40.88735784)(376.20341919,41.19736623)
\curveto(376.29342199,41.23735749)(376.37842191,41.26735746)(376.45841919,41.28736623)
\curveto(376.53842175,41.31735741)(376.62342166,41.34735738)(376.71341919,41.37736623)
\curveto(376.97342131,41.48735724)(377.23842105,41.59235714)(377.50841919,41.69236623)
\curveto(377.77842051,41.80235693)(378.04342024,41.91235682)(378.30341919,42.02236623)
\moveto(374.65841919,38.81236623)
\curveto(374.62842366,38.90235983)(374.57842371,38.95735977)(374.50841919,38.97736623)
\curveto(374.43842385,39.00735972)(374.36342392,39.01235972)(374.28341919,38.99236623)
\curveto(374.19342409,38.98235975)(374.10842418,38.95735977)(374.02841919,38.91736623)
\curveto(373.93842435,38.88735984)(373.86342442,38.85735987)(373.80341919,38.82736623)
\curveto(373.76342452,38.80735992)(373.72842456,38.79735993)(373.69841919,38.79736623)
\curveto(373.66842462,38.79735993)(373.63342465,38.78735994)(373.59341919,38.76736623)
\lineto(373.35341919,38.67736623)
\curveto(373.26342502,38.65736007)(373.17342511,38.6273601)(373.08341919,38.58736623)
\curveto(372.72342556,38.43736029)(372.35842593,38.30236043)(371.98841919,38.18236623)
\curveto(371.60842668,38.07236066)(371.23842705,37.94236079)(370.87841919,37.79236623)
\curveto(370.76842752,37.74236099)(370.65842763,37.69736103)(370.54841919,37.65736623)
\curveto(370.43842785,37.6273611)(370.33342795,37.58736114)(370.23341919,37.53736623)
\curveto(370.1834281,37.51736121)(370.13842815,37.49236124)(370.09841919,37.46236623)
\curveto(370.04842824,37.44236129)(370.02342826,37.39236134)(370.02341919,37.31236623)
\curveto(370.04342824,37.29236144)(370.05842823,37.27236146)(370.06841919,37.25236623)
\curveto(370.07842821,37.2323615)(370.09342819,37.21236152)(370.11341919,37.19236623)
\curveto(370.16342812,37.15236158)(370.21842807,37.12236161)(370.27841919,37.10236623)
\curveto(370.32842796,37.08236165)(370.3834279,37.06236167)(370.44341919,37.04236623)
\curveto(370.55342773,36.99236174)(370.66342762,36.95236178)(370.77341919,36.92236623)
\curveto(370.8834274,36.89236184)(370.99342729,36.85236188)(371.10341919,36.80236623)
\curveto(371.49342679,36.6323621)(371.8884264,36.48236225)(372.28841919,36.35236623)
\curveto(372.6884256,36.2323625)(373.07842521,36.09236264)(373.45841919,35.93236623)
\lineto(373.60841919,35.87236623)
\curveto(373.65842463,35.86236287)(373.70842458,35.84736288)(373.75841919,35.82736623)
\lineto(373.99841919,35.73736623)
\curveto(374.07842421,35.70736302)(374.15842413,35.68236305)(374.23841919,35.66236623)
\curveto(374.288424,35.64236309)(374.34342394,35.6323631)(374.40341919,35.63236623)
\curveto(374.46342382,35.64236309)(374.51342377,35.65736307)(374.55341919,35.67736623)
\curveto(374.63342365,35.727363)(374.67842361,35.8323629)(374.68841919,35.99236623)
\lineto(374.68841919,36.44236623)
\lineto(374.68841919,38.04736623)
\curveto(374.6884236,38.15736057)(374.69342359,38.29236044)(374.70341919,38.45236623)
\curveto(374.70342358,38.61236012)(374.6884236,38.73236)(374.65841919,38.81236623)
}
}
{
\newrgbcolor{curcolor}{0 0 0}
\pscustom[linestyle=none,fillstyle=solid,fillcolor=curcolor]
{
\newpath
\moveto(375.04841919,50.56392873)
\curveto(375.09842319,50.57392038)(375.16842312,50.57892038)(375.25841919,50.57892873)
\curveto(375.33842295,50.57892038)(375.40342288,50.57392038)(375.45341919,50.56392873)
\curveto(375.49342279,50.56392039)(375.53342275,50.5589204)(375.57341919,50.54892873)
\lineto(375.69341919,50.54892873)
\curveto(375.77342251,50.52892043)(375.85342243,50.51892044)(375.93341919,50.51892873)
\curveto(376.01342227,50.51892044)(376.09342219,50.50892045)(376.17341919,50.48892873)
\curveto(376.21342207,50.47892048)(376.25342203,50.47392048)(376.29341919,50.47392873)
\curveto(376.32342196,50.47392048)(376.35842193,50.46892049)(376.39841919,50.45892873)
\curveto(376.50842178,50.42892053)(376.61342167,50.39892056)(376.71341919,50.36892873)
\curveto(376.81342147,50.34892061)(376.91342137,50.31892064)(377.01341919,50.27892873)
\curveto(377.36342092,50.13892082)(377.67842061,49.96892099)(377.95841919,49.76892873)
\curveto(378.23842005,49.56892139)(378.47841981,49.31892164)(378.67841919,49.01892873)
\curveto(378.77841951,48.86892209)(378.86341942,48.72392223)(378.93341919,48.58392873)
\curveto(378.9834193,48.47392248)(379.02341926,48.36392259)(379.05341919,48.25392873)
\curveto(379.0834192,48.1539228)(379.11341917,48.04892291)(379.14341919,47.93892873)
\curveto(379.16341912,47.86892309)(379.17341911,47.80392315)(379.17341919,47.74392873)
\curveto(379.1834191,47.68392327)(379.19841909,47.62392333)(379.21841919,47.56392873)
\lineto(379.21841919,47.41392873)
\curveto(379.23841905,47.36392359)(379.24841904,47.28892367)(379.24841919,47.18892873)
\curveto(379.25841903,47.08892387)(379.25341903,47.00892395)(379.23341919,46.94892873)
\lineto(379.23341919,46.79892873)
\curveto(379.22341906,46.7589242)(379.21841907,46.71392424)(379.21841919,46.66392873)
\curveto(379.21841907,46.62392433)(379.21341907,46.57892438)(379.20341919,46.52892873)
\curveto(379.16341912,46.37892458)(379.12841916,46.22892473)(379.09841919,46.07892873)
\curveto(379.06841922,45.93892502)(379.02341926,45.79892516)(378.96341919,45.65892873)
\curveto(378.8834194,45.4589255)(378.7834195,45.27892568)(378.66341919,45.11892873)
\lineto(378.51341919,44.93892873)
\curveto(378.45341983,44.87892608)(378.41341987,44.80892615)(378.39341919,44.72892873)
\curveto(378.3834199,44.66892629)(378.39841989,44.61892634)(378.43841919,44.57892873)
\curveto(378.46841982,44.54892641)(378.51341977,44.52392643)(378.57341919,44.50392873)
\curveto(378.63341965,44.49392646)(378.69841959,44.48392647)(378.76841919,44.47392873)
\curveto(378.82841946,44.47392648)(378.87341941,44.46392649)(378.90341919,44.44392873)
\curveto(378.95341933,44.40392655)(378.99841929,44.3589266)(379.03841919,44.30892873)
\curveto(379.05841923,44.2589267)(379.07341921,44.18892677)(379.08341919,44.09892873)
\lineto(379.08341919,43.82892873)
\curveto(379.0834192,43.73892722)(379.07841921,43.6539273)(379.06841919,43.57392873)
\curveto(379.04841924,43.49392746)(379.02841926,43.43392752)(379.00841919,43.39392873)
\curveto(378.9884193,43.37392758)(378.96341932,43.3539276)(378.93341919,43.33392873)
\lineto(378.84341919,43.27392873)
\curveto(378.76341952,43.24392771)(378.64341964,43.22892773)(378.48341919,43.22892873)
\curveto(378.32341996,43.23892772)(378.1884201,43.24392771)(378.07841919,43.24392873)
\lineto(369.27341919,43.24392873)
\curveto(369.15342913,43.24392771)(369.02842926,43.23892772)(368.89841919,43.22892873)
\curveto(368.75842953,43.22892773)(368.64842964,43.2539277)(368.56841919,43.30392873)
\curveto(368.50842978,43.34392761)(368.45842983,43.40892755)(368.41841919,43.49892873)
\curveto(368.41842987,43.51892744)(368.41842987,43.54392741)(368.41841919,43.57392873)
\curveto(368.40842988,43.60392735)(368.40342988,43.62892733)(368.40341919,43.64892873)
\curveto(368.39342989,43.78892717)(368.39342989,43.93392702)(368.40341919,44.08392873)
\curveto(368.40342988,44.24392671)(368.44342984,44.3539266)(368.52341919,44.41392873)
\curveto(368.60342968,44.46392649)(368.71842957,44.48892647)(368.86841919,44.48892873)
\lineto(369.27341919,44.48892873)
\lineto(371.02841919,44.48892873)
\lineto(371.28341919,44.48892873)
\lineto(371.56841919,44.48892873)
\curveto(371.65842663,44.49892646)(371.74342654,44.50892645)(371.82341919,44.51892873)
\curveto(371.89342639,44.53892642)(371.94342634,44.56892639)(371.97341919,44.60892873)
\curveto(372.00342628,44.64892631)(372.00842628,44.69392626)(371.98841919,44.74392873)
\curveto(371.96842632,44.79392616)(371.94842634,44.83392612)(371.92841919,44.86392873)
\curveto(371.8884264,44.91392604)(371.84842644,44.958926)(371.80841919,44.99892873)
\lineto(371.68841919,45.14892873)
\curveto(371.63842665,45.21892574)(371.59342669,45.28892567)(371.55341919,45.35892873)
\lineto(371.43341919,45.59892873)
\curveto(371.34342694,45.77892518)(371.27842701,45.99392496)(371.23841919,46.24392873)
\curveto(371.19842709,46.49392446)(371.17842711,46.74892421)(371.17841919,47.00892873)
\curveto(371.17842711,47.26892369)(371.20342708,47.52392343)(371.25341919,47.77392873)
\curveto(371.29342699,48.02392293)(371.35342693,48.24392271)(371.43341919,48.43392873)
\curveto(371.60342668,48.83392212)(371.83842645,49.17892178)(372.13841919,49.46892873)
\curveto(372.43842585,49.7589212)(372.7884255,49.98892097)(373.18841919,50.15892873)
\curveto(373.29842499,50.20892075)(373.40842488,50.24892071)(373.51841919,50.27892873)
\curveto(373.61842467,50.31892064)(373.72342456,50.3589206)(373.83341919,50.39892873)
\curveto(373.94342434,50.42892053)(374.05842423,50.44892051)(374.17841919,50.45892873)
\lineto(374.50841919,50.51892873)
\curveto(374.53842375,50.52892043)(374.57342371,50.53392042)(374.61341919,50.53392873)
\curveto(374.64342364,50.53392042)(374.67342361,50.53892042)(374.70341919,50.54892873)
\curveto(374.76342352,50.56892039)(374.82342346,50.56892039)(374.88341919,50.54892873)
\curveto(374.93342335,50.53892042)(374.9884233,50.54392041)(375.04841919,50.56392873)
\moveto(375.43841919,49.22892873)
\curveto(375.3884229,49.24892171)(375.32842296,49.2539217)(375.25841919,49.24392873)
\curveto(375.1884231,49.23392172)(375.12342316,49.22892173)(375.06341919,49.22892873)
\curveto(374.89342339,49.22892173)(374.73342355,49.21892174)(374.58341919,49.19892873)
\curveto(374.43342385,49.18892177)(374.29842399,49.1589218)(374.17841919,49.10892873)
\curveto(374.07842421,49.07892188)(373.9884243,49.0539219)(373.90841919,49.03392873)
\curveto(373.82842446,49.01392194)(373.74842454,48.98392197)(373.66841919,48.94392873)
\curveto(373.41842487,48.83392212)(373.1884251,48.68392227)(372.97841919,48.49392873)
\curveto(372.75842553,48.30392265)(372.59342569,48.08392287)(372.48341919,47.83392873)
\curveto(372.45342583,47.7539232)(372.42842586,47.67392328)(372.40841919,47.59392873)
\curveto(372.37842591,47.52392343)(372.35342593,47.44892351)(372.33341919,47.36892873)
\curveto(372.30342598,47.2589237)(372.288426,47.14892381)(372.28841919,47.03892873)
\curveto(372.27842601,46.92892403)(372.27342601,46.80892415)(372.27341919,46.67892873)
\curveto(372.283426,46.62892433)(372.29342599,46.58392437)(372.30341919,46.54392873)
\lineto(372.30341919,46.40892873)
\lineto(372.36341919,46.13892873)
\curveto(372.3834259,46.0589249)(372.41342587,45.97892498)(372.45341919,45.89892873)
\curveto(372.59342569,45.5589254)(372.80342548,45.28892567)(373.08341919,45.08892873)
\curveto(373.35342493,44.88892607)(373.67342461,44.72892623)(374.04341919,44.60892873)
\curveto(374.15342413,44.56892639)(374.26342402,44.54392641)(374.37341919,44.53392873)
\curveto(374.4834238,44.52392643)(374.59842369,44.50392645)(374.71841919,44.47392873)
\curveto(374.76842352,44.46392649)(374.81342347,44.46392649)(374.85341919,44.47392873)
\curveto(374.89342339,44.48392647)(374.93842335,44.47892648)(374.98841919,44.45892873)
\curveto(375.03842325,44.44892651)(375.11342317,44.44392651)(375.21341919,44.44392873)
\curveto(375.30342298,44.44392651)(375.37342291,44.44892651)(375.42341919,44.45892873)
\lineto(375.54341919,44.45892873)
\curveto(375.5834227,44.46892649)(375.62342266,44.47392648)(375.66341919,44.47392873)
\curveto(375.70342258,44.47392648)(375.73842255,44.47892648)(375.76841919,44.48892873)
\curveto(375.79842249,44.49892646)(375.83342245,44.50392645)(375.87341919,44.50392873)
\curveto(375.90342238,44.50392645)(375.93342235,44.50892645)(375.96341919,44.51892873)
\curveto(376.04342224,44.53892642)(376.12342216,44.5539264)(376.20341919,44.56392873)
\lineto(376.44341919,44.62392873)
\curveto(376.7834215,44.73392622)(377.07342121,44.88392607)(377.31341919,45.07392873)
\curveto(377.55342073,45.27392568)(377.75342053,45.51892544)(377.91341919,45.80892873)
\curveto(377.96342032,45.89892506)(378.00342028,45.99392496)(378.03341919,46.09392873)
\curveto(378.05342023,46.19392476)(378.07842021,46.29892466)(378.10841919,46.40892873)
\curveto(378.12842016,46.4589245)(378.13842015,46.50392445)(378.13841919,46.54392873)
\curveto(378.12842016,46.59392436)(378.12842016,46.64392431)(378.13841919,46.69392873)
\curveto(378.14842014,46.73392422)(378.15342013,46.77892418)(378.15341919,46.82892873)
\lineto(378.15341919,46.96392873)
\lineto(378.15341919,47.09892873)
\curveto(378.14342014,47.13892382)(378.13842015,47.17392378)(378.13841919,47.20392873)
\curveto(378.13842015,47.23392372)(378.13342015,47.26892369)(378.12341919,47.30892873)
\curveto(378.10342018,47.38892357)(378.0884202,47.46392349)(378.07841919,47.53392873)
\curveto(378.05842023,47.60392335)(378.03342025,47.67892328)(378.00341919,47.75892873)
\curveto(377.87342041,48.06892289)(377.70342058,48.31892264)(377.49341919,48.50892873)
\curveto(377.27342101,48.69892226)(377.00842128,48.8589221)(376.69841919,48.98892873)
\curveto(376.55842173,49.03892192)(376.41842187,49.07392188)(376.27841919,49.09392873)
\curveto(376.12842216,49.12392183)(375.97842231,49.1589218)(375.82841919,49.19892873)
\curveto(375.77842251,49.21892174)(375.73342255,49.22392173)(375.69341919,49.21392873)
\curveto(375.64342264,49.21392174)(375.59342269,49.21892174)(375.54341919,49.22892873)
\lineto(375.43841919,49.22892873)
}
}
{
\newrgbcolor{curcolor}{0 0 0}
\pscustom[linestyle=none,fillstyle=solid,fillcolor=curcolor]
{
\newpath
\moveto(371.17841919,55.69017873)
\curveto(371.17842711,55.92017394)(371.23842705,56.05017381)(371.35841919,56.08017873)
\curveto(371.46842682,56.11017375)(371.63342665,56.12517374)(371.85341919,56.12517873)
\lineto(372.13841919,56.12517873)
\curveto(372.22842606,56.12517374)(372.30342598,56.10017376)(372.36341919,56.05017873)
\curveto(372.44342584,55.99017387)(372.4884258,55.90517396)(372.49841919,55.79517873)
\curveto(372.49842579,55.68517418)(372.51342577,55.57517429)(372.54341919,55.46517873)
\curveto(372.57342571,55.32517454)(372.60342568,55.19017467)(372.63341919,55.06017873)
\curveto(372.66342562,54.94017492)(372.70342558,54.82517504)(372.75341919,54.71517873)
\curveto(372.8834254,54.42517544)(373.06342522,54.19017567)(373.29341919,54.01017873)
\curveto(373.51342477,53.83017603)(373.76842452,53.67517619)(374.05841919,53.54517873)
\curveto(374.16842412,53.50517636)(374.283424,53.47517639)(374.40341919,53.45517873)
\curveto(374.51342377,53.43517643)(374.62842366,53.41017645)(374.74841919,53.38017873)
\curveto(374.79842349,53.37017649)(374.84842344,53.3651765)(374.89841919,53.36517873)
\curveto(374.94842334,53.37517649)(374.99842329,53.37517649)(375.04841919,53.36517873)
\curveto(375.16842312,53.33517653)(375.30842298,53.32017654)(375.46841919,53.32017873)
\curveto(375.61842267,53.33017653)(375.76342252,53.33517653)(375.90341919,53.33517873)
\lineto(377.74841919,53.33517873)
\lineto(378.09341919,53.33517873)
\curveto(378.21342007,53.33517653)(378.32841996,53.33017653)(378.43841919,53.32017873)
\curveto(378.54841974,53.31017655)(378.64341964,53.30517656)(378.72341919,53.30517873)
\curveto(378.80341948,53.31517655)(378.87341941,53.29517657)(378.93341919,53.24517873)
\curveto(379.00341928,53.19517667)(379.04341924,53.11517675)(379.05341919,53.00517873)
\curveto(379.06341922,52.90517696)(379.06841922,52.79517707)(379.06841919,52.67517873)
\lineto(379.06841919,52.40517873)
\curveto(379.04841924,52.35517751)(379.03341925,52.30517756)(379.02341919,52.25517873)
\curveto(379.00341928,52.21517765)(378.97841931,52.18517768)(378.94841919,52.16517873)
\curveto(378.87841941,52.11517775)(378.79341949,52.08517778)(378.69341919,52.07517873)
\lineto(378.36341919,52.07517873)
\lineto(377.20841919,52.07517873)
\lineto(373.05341919,52.07517873)
\lineto(372.01841919,52.07517873)
\lineto(371.71841919,52.07517873)
\curveto(371.61842667,52.08517778)(371.53342675,52.11517775)(371.46341919,52.16517873)
\curveto(371.42342686,52.19517767)(371.39342689,52.24517762)(371.37341919,52.31517873)
\curveto(371.35342693,52.39517747)(371.34342694,52.48017738)(371.34341919,52.57017873)
\curveto(371.33342695,52.6601772)(371.33342695,52.75017711)(371.34341919,52.84017873)
\curveto(371.35342693,52.93017693)(371.36842692,53.00017686)(371.38841919,53.05017873)
\curveto(371.41842687,53.13017673)(371.47842681,53.18017668)(371.56841919,53.20017873)
\curveto(371.64842664,53.23017663)(371.73842655,53.24517662)(371.83841919,53.24517873)
\lineto(372.13841919,53.24517873)
\curveto(372.23842605,53.24517662)(372.32842596,53.2651766)(372.40841919,53.30517873)
\curveto(372.42842586,53.31517655)(372.44342584,53.32517654)(372.45341919,53.33517873)
\lineto(372.49841919,53.38017873)
\curveto(372.49842579,53.49017637)(372.45342583,53.58017628)(372.36341919,53.65017873)
\curveto(372.26342602,53.72017614)(372.1834261,53.78017608)(372.12341919,53.83017873)
\lineto(372.03341919,53.92017873)
\curveto(371.92342636,54.01017585)(371.80842648,54.13517573)(371.68841919,54.29517873)
\curveto(371.56842672,54.45517541)(371.47842681,54.60517526)(371.41841919,54.74517873)
\curveto(371.36842692,54.83517503)(371.33342695,54.93017493)(371.31341919,55.03017873)
\curveto(371.283427,55.13017473)(371.25342703,55.23517463)(371.22341919,55.34517873)
\curveto(371.21342707,55.40517446)(371.20842708,55.4651744)(371.20841919,55.52517873)
\curveto(371.19842709,55.58517428)(371.1884271,55.64017422)(371.17841919,55.69017873)
}
}
{
\newrgbcolor{curcolor}{0 0 0}
\pscustom[linestyle=none,fillstyle=solid,fillcolor=curcolor]
{
}
}
{
\newrgbcolor{curcolor}{0 0 0}
\pscustom[linestyle=none,fillstyle=solid,fillcolor=curcolor]
{
\newpath
\moveto(368.47841919,65.05510061)
\curveto(368.47842981,65.15509575)(368.4884298,65.25009566)(368.50841919,65.34010061)
\curveto(368.51842977,65.43009548)(368.54842974,65.49509541)(368.59841919,65.53510061)
\curveto(368.67842961,65.59509531)(368.7834295,65.62509528)(368.91341919,65.62510061)
\lineto(369.30341919,65.62510061)
\lineto(370.80341919,65.62510061)
\lineto(377.19341919,65.62510061)
\lineto(378.36341919,65.62510061)
\lineto(378.67841919,65.62510061)
\curveto(378.77841951,65.63509527)(378.85841943,65.62009529)(378.91841919,65.58010061)
\curveto(378.99841929,65.53009538)(379.04841924,65.45509545)(379.06841919,65.35510061)
\curveto(379.07841921,65.26509564)(379.0834192,65.15509575)(379.08341919,65.02510061)
\lineto(379.08341919,64.80010061)
\curveto(379.06341922,64.72009619)(379.04841924,64.65009626)(379.03841919,64.59010061)
\curveto(379.01841927,64.53009638)(378.97841931,64.48009643)(378.91841919,64.44010061)
\curveto(378.85841943,64.40009651)(378.7834195,64.38009653)(378.69341919,64.38010061)
\lineto(378.39341919,64.38010061)
\lineto(377.29841919,64.38010061)
\lineto(371.95841919,64.38010061)
\curveto(371.86842642,64.36009655)(371.79342649,64.34509656)(371.73341919,64.33510061)
\curveto(371.66342662,64.33509657)(371.60342668,64.3050966)(371.55341919,64.24510061)
\curveto(371.50342678,64.17509673)(371.47842681,64.08509682)(371.47841919,63.97510061)
\curveto(371.46842682,63.87509703)(371.46342682,63.76509714)(371.46341919,63.64510061)
\lineto(371.46341919,62.50510061)
\lineto(371.46341919,62.01010061)
\curveto(371.45342683,61.85009906)(371.39342689,61.74009917)(371.28341919,61.68010061)
\curveto(371.25342703,61.66009925)(371.22342706,61.65009926)(371.19341919,61.65010061)
\curveto(371.15342713,61.65009926)(371.10842718,61.64509926)(371.05841919,61.63510061)
\curveto(370.93842735,61.61509929)(370.82842746,61.62009929)(370.72841919,61.65010061)
\curveto(370.62842766,61.69009922)(370.55842773,61.74509916)(370.51841919,61.81510061)
\curveto(370.46842782,61.89509901)(370.44342784,62.01509889)(370.44341919,62.17510061)
\curveto(370.44342784,62.33509857)(370.42842786,62.47009844)(370.39841919,62.58010061)
\curveto(370.3884279,62.63009828)(370.3834279,62.68509822)(370.38341919,62.74510061)
\curveto(370.37342791,62.8050981)(370.35842793,62.86509804)(370.33841919,62.92510061)
\curveto(370.288428,63.07509783)(370.23842805,63.22009769)(370.18841919,63.36010061)
\curveto(370.12842816,63.50009741)(370.05842823,63.63509727)(369.97841919,63.76510061)
\curveto(369.8884284,63.905097)(369.7834285,64.02509688)(369.66341919,64.12510061)
\curveto(369.54342874,64.22509668)(369.41342887,64.32009659)(369.27341919,64.41010061)
\curveto(369.17342911,64.47009644)(369.06342922,64.51509639)(368.94341919,64.54510061)
\curveto(368.82342946,64.58509632)(368.71842957,64.63509627)(368.62841919,64.69510061)
\curveto(368.56842972,64.74509616)(368.52842976,64.81509609)(368.50841919,64.90510061)
\curveto(368.49842979,64.92509598)(368.49342979,64.95009596)(368.49341919,64.98010061)
\curveto(368.49342979,65.0100959)(368.4884298,65.03509587)(368.47841919,65.05510061)
}
}
{
\newrgbcolor{curcolor}{0 0 0}
\pscustom[linestyle=none,fillstyle=solid,fillcolor=curcolor]
{
\newpath
\moveto(368.47841919,73.40470998)
\curveto(368.47842981,73.50470513)(368.4884298,73.59970503)(368.50841919,73.68970998)
\curveto(368.51842977,73.77970485)(368.54842974,73.84470479)(368.59841919,73.88470998)
\curveto(368.67842961,73.94470469)(368.7834295,73.97470466)(368.91341919,73.97470998)
\lineto(369.30341919,73.97470998)
\lineto(370.80341919,73.97470998)
\lineto(377.19341919,73.97470998)
\lineto(378.36341919,73.97470998)
\lineto(378.67841919,73.97470998)
\curveto(378.77841951,73.98470465)(378.85841943,73.96970466)(378.91841919,73.92970998)
\curveto(378.99841929,73.87970475)(379.04841924,73.80470483)(379.06841919,73.70470998)
\curveto(379.07841921,73.61470502)(379.0834192,73.50470513)(379.08341919,73.37470998)
\lineto(379.08341919,73.14970998)
\curveto(379.06341922,73.06970556)(379.04841924,72.99970563)(379.03841919,72.93970998)
\curveto(379.01841927,72.87970575)(378.97841931,72.8297058)(378.91841919,72.78970998)
\curveto(378.85841943,72.74970588)(378.7834195,72.7297059)(378.69341919,72.72970998)
\lineto(378.39341919,72.72970998)
\lineto(377.29841919,72.72970998)
\lineto(371.95841919,72.72970998)
\curveto(371.86842642,72.70970592)(371.79342649,72.69470594)(371.73341919,72.68470998)
\curveto(371.66342662,72.68470595)(371.60342668,72.65470598)(371.55341919,72.59470998)
\curveto(371.50342678,72.52470611)(371.47842681,72.4347062)(371.47841919,72.32470998)
\curveto(371.46842682,72.22470641)(371.46342682,72.11470652)(371.46341919,71.99470998)
\lineto(371.46341919,70.85470998)
\lineto(371.46341919,70.35970998)
\curveto(371.45342683,70.19970843)(371.39342689,70.08970854)(371.28341919,70.02970998)
\curveto(371.25342703,70.00970862)(371.22342706,69.99970863)(371.19341919,69.99970998)
\curveto(371.15342713,69.99970863)(371.10842718,69.99470864)(371.05841919,69.98470998)
\curveto(370.93842735,69.96470867)(370.82842746,69.96970866)(370.72841919,69.99970998)
\curveto(370.62842766,70.03970859)(370.55842773,70.09470854)(370.51841919,70.16470998)
\curveto(370.46842782,70.24470839)(370.44342784,70.36470827)(370.44341919,70.52470998)
\curveto(370.44342784,70.68470795)(370.42842786,70.81970781)(370.39841919,70.92970998)
\curveto(370.3884279,70.97970765)(370.3834279,71.0347076)(370.38341919,71.09470998)
\curveto(370.37342791,71.15470748)(370.35842793,71.21470742)(370.33841919,71.27470998)
\curveto(370.288428,71.42470721)(370.23842805,71.56970706)(370.18841919,71.70970998)
\curveto(370.12842816,71.84970678)(370.05842823,71.98470665)(369.97841919,72.11470998)
\curveto(369.8884284,72.25470638)(369.7834285,72.37470626)(369.66341919,72.47470998)
\curveto(369.54342874,72.57470606)(369.41342887,72.66970596)(369.27341919,72.75970998)
\curveto(369.17342911,72.81970581)(369.06342922,72.86470577)(368.94341919,72.89470998)
\curveto(368.82342946,72.9347057)(368.71842957,72.98470565)(368.62841919,73.04470998)
\curveto(368.56842972,73.09470554)(368.52842976,73.16470547)(368.50841919,73.25470998)
\curveto(368.49842979,73.27470536)(368.49342979,73.29970533)(368.49341919,73.32970998)
\curveto(368.49342979,73.35970527)(368.4884298,73.38470525)(368.47841919,73.40470998)
}
}
{
\newrgbcolor{curcolor}{0 0 0}
\pscustom[linestyle=none,fillstyle=solid,fillcolor=curcolor]
{
\newpath
\moveto(377.44841919,78.63431936)
\lineto(377.44841919,79.26431936)
\lineto(377.44841919,79.45931936)
\curveto(377.44842084,79.52931683)(377.45842083,79.58931677)(377.47841919,79.63931936)
\curveto(377.51842077,79.70931665)(377.55842073,79.7593166)(377.59841919,79.78931936)
\curveto(377.64842064,79.82931653)(377.71342057,79.84931651)(377.79341919,79.84931936)
\curveto(377.87342041,79.8593165)(377.95842033,79.86431649)(378.04841919,79.86431936)
\lineto(378.76841919,79.86431936)
\curveto(379.24841904,79.86431649)(379.65841863,79.80431655)(379.99841919,79.68431936)
\curveto(380.33841795,79.56431679)(380.61341767,79.36931699)(380.82341919,79.09931936)
\curveto(380.87341741,79.02931733)(380.91841737,78.9593174)(380.95841919,78.88931936)
\curveto(381.00841728,78.82931753)(381.05341723,78.7543176)(381.09341919,78.66431936)
\curveto(381.10341718,78.64431771)(381.11341717,78.61431774)(381.12341919,78.57431936)
\curveto(381.14341714,78.53431782)(381.14841714,78.48931787)(381.13841919,78.43931936)
\curveto(381.10841718,78.34931801)(381.03341725,78.29431806)(380.91341919,78.27431936)
\curveto(380.80341748,78.2543181)(380.70841758,78.26931809)(380.62841919,78.31931936)
\curveto(380.55841773,78.34931801)(380.49341779,78.39431796)(380.43341919,78.45431936)
\curveto(380.3834179,78.52431783)(380.33341795,78.58931777)(380.28341919,78.64931936)
\curveto(380.23341805,78.71931764)(380.15841813,78.77931758)(380.05841919,78.82931936)
\curveto(379.96841832,78.88931747)(379.87841841,78.93931742)(379.78841919,78.97931936)
\curveto(379.75841853,78.99931736)(379.69841859,79.02431733)(379.60841919,79.05431936)
\curveto(379.52841876,79.08431727)(379.45841883,79.08931727)(379.39841919,79.06931936)
\curveto(379.25841903,79.03931732)(379.16841912,78.97931738)(379.12841919,78.88931936)
\curveto(379.09841919,78.80931755)(379.0834192,78.71931764)(379.08341919,78.61931936)
\curveto(379.0834192,78.51931784)(379.05841923,78.43431792)(379.00841919,78.36431936)
\curveto(378.93841935,78.27431808)(378.79841949,78.22931813)(378.58841919,78.22931936)
\lineto(378.03341919,78.22931936)
\lineto(377.80841919,78.22931936)
\curveto(377.72842056,78.23931812)(377.66342062,78.2593181)(377.61341919,78.28931936)
\curveto(377.53342075,78.34931801)(377.4884208,78.41931794)(377.47841919,78.49931936)
\curveto(377.46842082,78.51931784)(377.46342082,78.53931782)(377.46341919,78.55931936)
\curveto(377.46342082,78.58931777)(377.45842083,78.61431774)(377.44841919,78.63431936)
}
}
{
\newrgbcolor{curcolor}{0 0 0}
\pscustom[linestyle=none,fillstyle=solid,fillcolor=curcolor]
{
}
}
{
\newrgbcolor{curcolor}{0 0 0}
\pscustom[linestyle=none,fillstyle=solid,fillcolor=curcolor]
{
\newpath
\moveto(368.47841919,89.26463186)
\curveto(368.46842982,89.95462722)(368.5884297,90.55462662)(368.83841919,91.06463186)
\curveto(369.0884292,91.58462559)(369.42342886,91.9796252)(369.84341919,92.24963186)
\curveto(369.92342836,92.29962488)(370.01342827,92.34462483)(370.11341919,92.38463186)
\curveto(370.20342808,92.42462475)(370.29842799,92.46962471)(370.39841919,92.51963186)
\curveto(370.49842779,92.55962462)(370.59842769,92.58962459)(370.69841919,92.60963186)
\curveto(370.79842749,92.62962455)(370.90342738,92.64962453)(371.01341919,92.66963186)
\curveto(371.06342722,92.68962449)(371.10842718,92.69462448)(371.14841919,92.68463186)
\curveto(371.1884271,92.6746245)(371.23342705,92.6796245)(371.28341919,92.69963186)
\curveto(371.33342695,92.70962447)(371.41842687,92.71462446)(371.53841919,92.71463186)
\curveto(371.64842664,92.71462446)(371.73342655,92.70962447)(371.79341919,92.69963186)
\curveto(371.85342643,92.6796245)(371.91342637,92.66962451)(371.97341919,92.66963186)
\curveto(372.03342625,92.6796245)(372.09342619,92.6746245)(372.15341919,92.65463186)
\curveto(372.29342599,92.61462456)(372.42842586,92.5796246)(372.55841919,92.54963186)
\curveto(372.6884256,92.51962466)(372.81342547,92.4796247)(372.93341919,92.42963186)
\curveto(373.07342521,92.36962481)(373.19842509,92.29962488)(373.30841919,92.21963186)
\curveto(373.41842487,92.14962503)(373.52842476,92.0746251)(373.63841919,91.99463186)
\lineto(373.69841919,91.93463186)
\curveto(373.71842457,91.92462525)(373.73842455,91.90962527)(373.75841919,91.88963186)
\curveto(373.91842437,91.76962541)(374.06342422,91.63462554)(374.19341919,91.48463186)
\curveto(374.32342396,91.33462584)(374.44842384,91.174626)(374.56841919,91.00463186)
\curveto(374.7884235,90.69462648)(374.99342329,90.39962678)(375.18341919,90.11963186)
\curveto(375.32342296,89.88962729)(375.45842283,89.65962752)(375.58841919,89.42963186)
\curveto(375.71842257,89.20962797)(375.85342243,88.98962819)(375.99341919,88.76963186)
\curveto(376.16342212,88.51962866)(376.34342194,88.2796289)(376.53341919,88.04963186)
\curveto(376.72342156,87.82962935)(376.94842134,87.63962954)(377.20841919,87.47963186)
\curveto(377.26842102,87.43962974)(377.32842096,87.40462977)(377.38841919,87.37463186)
\curveto(377.43842085,87.34462983)(377.50342078,87.31462986)(377.58341919,87.28463186)
\curveto(377.65342063,87.26462991)(377.71342057,87.25962992)(377.76341919,87.26963186)
\curveto(377.83342045,87.28962989)(377.8884204,87.32462985)(377.92841919,87.37463186)
\curveto(377.95842033,87.42462975)(377.97842031,87.48462969)(377.98841919,87.55463186)
\lineto(377.98841919,87.79463186)
\lineto(377.98841919,88.54463186)
\lineto(377.98841919,91.34963186)
\lineto(377.98841919,92.00963186)
\curveto(377.9884203,92.09962508)(377.99342029,92.18462499)(378.00341919,92.26463186)
\curveto(378.00342028,92.34462483)(378.02342026,92.40962477)(378.06341919,92.45963186)
\curveto(378.10342018,92.50962467)(378.17842011,92.54962463)(378.28841919,92.57963186)
\curveto(378.3884199,92.61962456)(378.4884198,92.62962455)(378.58841919,92.60963186)
\lineto(378.72341919,92.60963186)
\curveto(378.79341949,92.58962459)(378.85341943,92.56962461)(378.90341919,92.54963186)
\curveto(378.95341933,92.52962465)(378.99341929,92.49462468)(379.02341919,92.44463186)
\curveto(379.06341922,92.39462478)(379.0834192,92.32462485)(379.08341919,92.23463186)
\lineto(379.08341919,91.96463186)
\lineto(379.08341919,91.06463186)
\lineto(379.08341919,87.55463186)
\lineto(379.08341919,86.48963186)
\curveto(379.0834192,86.40963077)(379.0884192,86.31963086)(379.09841919,86.21963186)
\curveto(379.09841919,86.11963106)(379.0884192,86.03463114)(379.06841919,85.96463186)
\curveto(378.99841929,85.75463142)(378.81841947,85.68963149)(378.52841919,85.76963186)
\curveto(378.4884198,85.7796314)(378.45341983,85.7796314)(378.42341919,85.76963186)
\curveto(378.3834199,85.76963141)(378.33841995,85.7796314)(378.28841919,85.79963186)
\curveto(378.20842008,85.81963136)(378.12342016,85.83963134)(378.03341919,85.85963186)
\curveto(377.94342034,85.8796313)(377.85842043,85.90463127)(377.77841919,85.93463186)
\curveto(377.288421,86.09463108)(376.87342141,86.29463088)(376.53341919,86.53463186)
\curveto(376.283422,86.71463046)(376.05842223,86.91963026)(375.85841919,87.14963186)
\curveto(375.64842264,87.3796298)(375.45342283,87.61962956)(375.27341919,87.86963186)
\curveto(375.09342319,88.12962905)(374.92342336,88.39462878)(374.76341919,88.66463186)
\curveto(374.59342369,88.94462823)(374.41842387,89.21462796)(374.23841919,89.47463186)
\curveto(374.15842413,89.58462759)(374.0834242,89.68962749)(374.01341919,89.78963186)
\curveto(373.94342434,89.89962728)(373.86842442,90.00962717)(373.78841919,90.11963186)
\curveto(373.75842453,90.15962702)(373.72842456,90.19462698)(373.69841919,90.22463186)
\curveto(373.65842463,90.26462691)(373.62842466,90.30462687)(373.60841919,90.34463186)
\curveto(373.49842479,90.48462669)(373.37342491,90.60962657)(373.23341919,90.71963186)
\curveto(373.20342508,90.73962644)(373.17842511,90.76462641)(373.15841919,90.79463186)
\curveto(373.12842516,90.82462635)(373.09842519,90.84962633)(373.06841919,90.86963186)
\curveto(372.96842532,90.94962623)(372.86842542,91.01462616)(372.76841919,91.06463186)
\curveto(372.66842562,91.12462605)(372.55842573,91.179626)(372.43841919,91.22963186)
\curveto(372.36842592,91.25962592)(372.29342599,91.2796259)(372.21341919,91.28963186)
\lineto(371.97341919,91.34963186)
\lineto(371.88341919,91.34963186)
\curveto(371.85342643,91.35962582)(371.82342646,91.36462581)(371.79341919,91.36463186)
\curveto(371.72342656,91.38462579)(371.62842666,91.38962579)(371.50841919,91.37963186)
\curveto(371.37842691,91.3796258)(371.27842701,91.36962581)(371.20841919,91.34963186)
\curveto(371.12842716,91.32962585)(371.05342723,91.30962587)(370.98341919,91.28963186)
\curveto(370.90342738,91.2796259)(370.82342746,91.25962592)(370.74341919,91.22963186)
\curveto(370.50342778,91.11962606)(370.30342798,90.96962621)(370.14341919,90.77963186)
\curveto(369.97342831,90.59962658)(369.83342845,90.3796268)(369.72341919,90.11963186)
\curveto(369.70342858,90.04962713)(369.6884286,89.9796272)(369.67841919,89.90963186)
\curveto(369.65842863,89.83962734)(369.63842865,89.76462741)(369.61841919,89.68463186)
\curveto(369.59842869,89.60462757)(369.5884287,89.49462768)(369.58841919,89.35463186)
\curveto(369.5884287,89.22462795)(369.59842869,89.11962806)(369.61841919,89.03963186)
\curveto(369.62842866,88.9796282)(369.63342865,88.92462825)(369.63341919,88.87463186)
\curveto(369.63342865,88.82462835)(369.64342864,88.7746284)(369.66341919,88.72463186)
\curveto(369.70342858,88.62462855)(369.74342854,88.52962865)(369.78341919,88.43963186)
\curveto(369.82342846,88.35962882)(369.86842842,88.2796289)(369.91841919,88.19963186)
\curveto(369.93842835,88.16962901)(369.96342832,88.13962904)(369.99341919,88.10963186)
\curveto(370.02342826,88.08962909)(370.04842824,88.06462911)(370.06841919,88.03463186)
\lineto(370.14341919,87.95963186)
\curveto(370.16342812,87.92962925)(370.1834281,87.90462927)(370.20341919,87.88463186)
\lineto(370.41341919,87.73463186)
\curveto(370.47342781,87.69462948)(370.53842775,87.64962953)(370.60841919,87.59963186)
\curveto(370.69842759,87.53962964)(370.80342748,87.48962969)(370.92341919,87.44963186)
\curveto(371.03342725,87.41962976)(371.14342714,87.38462979)(371.25341919,87.34463186)
\curveto(371.36342692,87.30462987)(371.50842678,87.2796299)(371.68841919,87.26963186)
\curveto(371.85842643,87.25962992)(371.9834263,87.22962995)(372.06341919,87.17963186)
\curveto(372.14342614,87.12963005)(372.1884261,87.05463012)(372.19841919,86.95463186)
\curveto(372.20842608,86.85463032)(372.21342607,86.74463043)(372.21341919,86.62463186)
\curveto(372.21342607,86.58463059)(372.21842607,86.54463063)(372.22841919,86.50463186)
\curveto(372.22842606,86.46463071)(372.22342606,86.42963075)(372.21341919,86.39963186)
\curveto(372.19342609,86.34963083)(372.1834261,86.29963088)(372.18341919,86.24963186)
\curveto(372.1834261,86.20963097)(372.17342611,86.16963101)(372.15341919,86.12963186)
\curveto(372.09342619,86.03963114)(371.95842633,85.99463118)(371.74841919,85.99463186)
\lineto(371.62841919,85.99463186)
\curveto(371.56842672,86.00463117)(371.50842678,86.00963117)(371.44841919,86.00963186)
\curveto(371.37842691,86.01963116)(371.31342697,86.02963115)(371.25341919,86.03963186)
\curveto(371.14342714,86.05963112)(371.04342724,86.0796311)(370.95341919,86.09963186)
\curveto(370.85342743,86.11963106)(370.75842753,86.14963103)(370.66841919,86.18963186)
\curveto(370.59842769,86.20963097)(370.53842775,86.22963095)(370.48841919,86.24963186)
\lineto(370.30841919,86.30963186)
\curveto(370.04842824,86.42963075)(369.80342848,86.58463059)(369.57341919,86.77463186)
\curveto(369.34342894,86.9746302)(369.15842913,87.18962999)(369.01841919,87.41963186)
\curveto(368.93842935,87.52962965)(368.87342941,87.64462953)(368.82341919,87.76463186)
\lineto(368.67341919,88.15463186)
\curveto(368.62342966,88.26462891)(368.59342969,88.3796288)(368.58341919,88.49963186)
\curveto(368.56342972,88.61962856)(368.53842975,88.74462843)(368.50841919,88.87463186)
\curveto(368.50842978,88.94462823)(368.50842978,89.00962817)(368.50841919,89.06963186)
\curveto(368.49842979,89.12962805)(368.4884298,89.19462798)(368.47841919,89.26463186)
}
}
{
\newrgbcolor{curcolor}{0 0 0}
\pscustom[linestyle=none,fillstyle=solid,fillcolor=curcolor]
{
\newpath
\moveto(373.99841919,101.36424123)
\lineto(374.25341919,101.36424123)
\curveto(374.33342395,101.37423353)(374.40842388,101.36923353)(374.47841919,101.34924123)
\lineto(374.71841919,101.34924123)
\lineto(374.88341919,101.34924123)
\curveto(374.9834233,101.32923357)(375.0884232,101.31923358)(375.19841919,101.31924123)
\curveto(375.29842299,101.31923358)(375.39842289,101.30923359)(375.49841919,101.28924123)
\lineto(375.64841919,101.28924123)
\curveto(375.7884225,101.25923364)(375.92842236,101.23923366)(376.06841919,101.22924123)
\curveto(376.19842209,101.21923368)(376.32842196,101.19423371)(376.45841919,101.15424123)
\curveto(376.53842175,101.13423377)(376.62342166,101.11423379)(376.71341919,101.09424123)
\lineto(376.95341919,101.03424123)
\lineto(377.25341919,100.91424123)
\curveto(377.34342094,100.88423402)(377.43342085,100.84923405)(377.52341919,100.80924123)
\curveto(377.74342054,100.70923419)(377.95842033,100.57423433)(378.16841919,100.40424123)
\curveto(378.37841991,100.24423466)(378.54841974,100.06923483)(378.67841919,99.87924123)
\curveto(378.71841957,99.82923507)(378.75841953,99.76923513)(378.79841919,99.69924123)
\curveto(378.82841946,99.63923526)(378.86341942,99.57923532)(378.90341919,99.51924123)
\curveto(378.95341933,99.43923546)(378.99341929,99.34423556)(379.02341919,99.23424123)
\curveto(379.05341923,99.12423578)(379.0834192,99.01923588)(379.11341919,98.91924123)
\curveto(379.15341913,98.80923609)(379.17841911,98.6992362)(379.18841919,98.58924123)
\curveto(379.19841909,98.47923642)(379.21341907,98.36423654)(379.23341919,98.24424123)
\curveto(379.24341904,98.2042367)(379.24341904,98.15923674)(379.23341919,98.10924123)
\curveto(379.23341905,98.06923683)(379.23841905,98.02923687)(379.24841919,97.98924123)
\curveto(379.25841903,97.94923695)(379.26341902,97.89423701)(379.26341919,97.82424123)
\curveto(379.26341902,97.75423715)(379.25841903,97.7042372)(379.24841919,97.67424123)
\curveto(379.22841906,97.62423728)(379.22341906,97.57923732)(379.23341919,97.53924123)
\curveto(379.24341904,97.4992374)(379.24341904,97.46423744)(379.23341919,97.43424123)
\lineto(379.23341919,97.34424123)
\curveto(379.21341907,97.28423762)(379.19841909,97.21923768)(379.18841919,97.14924123)
\curveto(379.1884191,97.08923781)(379.1834191,97.02423788)(379.17341919,96.95424123)
\curveto(379.12341916,96.78423812)(379.07341921,96.62423828)(379.02341919,96.47424123)
\curveto(378.97341931,96.32423858)(378.90841938,96.17923872)(378.82841919,96.03924123)
\curveto(378.7884195,95.98923891)(378.75841953,95.93423897)(378.73841919,95.87424123)
\curveto(378.70841958,95.82423908)(378.67341961,95.77423913)(378.63341919,95.72424123)
\curveto(378.45341983,95.48423942)(378.23342005,95.28423962)(377.97341919,95.12424123)
\curveto(377.71342057,94.96423994)(377.42842086,94.82424008)(377.11841919,94.70424123)
\curveto(376.97842131,94.64424026)(376.83842145,94.5992403)(376.69841919,94.56924123)
\curveto(376.54842174,94.53924036)(376.39342189,94.5042404)(376.23341919,94.46424123)
\curveto(376.12342216,94.44424046)(376.01342227,94.42924047)(375.90341919,94.41924123)
\curveto(375.79342249,94.40924049)(375.6834226,94.39424051)(375.57341919,94.37424123)
\curveto(375.53342275,94.36424054)(375.49342279,94.35924054)(375.45341919,94.35924123)
\curveto(375.41342287,94.36924053)(375.37342291,94.36924053)(375.33341919,94.35924123)
\curveto(375.283423,94.34924055)(375.23342305,94.34424056)(375.18341919,94.34424123)
\lineto(375.01841919,94.34424123)
\curveto(374.96842332,94.32424058)(374.91842337,94.31924058)(374.86841919,94.32924123)
\curveto(374.80842348,94.33924056)(374.75342353,94.33924056)(374.70341919,94.32924123)
\curveto(374.66342362,94.31924058)(374.61842367,94.31924058)(374.56841919,94.32924123)
\curveto(374.51842377,94.33924056)(374.46842382,94.33424057)(374.41841919,94.31424123)
\curveto(374.34842394,94.29424061)(374.27342401,94.28924061)(374.19341919,94.29924123)
\curveto(374.10342418,94.30924059)(374.01842427,94.31424059)(373.93841919,94.31424123)
\curveto(373.84842444,94.31424059)(373.74842454,94.30924059)(373.63841919,94.29924123)
\curveto(373.51842477,94.28924061)(373.41842487,94.29424061)(373.33841919,94.31424123)
\lineto(373.05341919,94.31424123)
\lineto(372.42341919,94.35924123)
\curveto(372.32342596,94.36924053)(372.22842606,94.37924052)(372.13841919,94.38924123)
\lineto(371.83841919,94.41924123)
\curveto(371.7884265,94.43924046)(371.73842655,94.44424046)(371.68841919,94.43424123)
\curveto(371.62842666,94.43424047)(371.57342671,94.44424046)(371.52341919,94.46424123)
\curveto(371.35342693,94.51424039)(371.1884271,94.55424035)(371.02841919,94.58424123)
\curveto(370.85842743,94.61424029)(370.69842759,94.66424024)(370.54841919,94.73424123)
\curveto(370.0884282,94.92423998)(369.71342857,95.14423976)(369.42341919,95.39424123)
\curveto(369.13342915,95.65423925)(368.8884294,96.01423889)(368.68841919,96.47424123)
\curveto(368.63842965,96.6042383)(368.60342968,96.73423817)(368.58341919,96.86424123)
\curveto(368.56342972,97.0042379)(368.53842975,97.14423776)(368.50841919,97.28424123)
\curveto(368.49842979,97.35423755)(368.49342979,97.41923748)(368.49341919,97.47924123)
\curveto(368.49342979,97.53923736)(368.4884298,97.6042373)(368.47841919,97.67424123)
\curveto(368.45842983,98.5042364)(368.60842968,99.17423573)(368.92841919,99.68424123)
\curveto(369.23842905,100.19423471)(369.67842861,100.57423433)(370.24841919,100.82424123)
\curveto(370.36842792,100.87423403)(370.49342779,100.91923398)(370.62341919,100.95924123)
\curveto(370.75342753,100.9992339)(370.8884274,101.04423386)(371.02841919,101.09424123)
\curveto(371.10842718,101.11423379)(371.19342709,101.12923377)(371.28341919,101.13924123)
\lineto(371.52341919,101.19924123)
\curveto(371.63342665,101.22923367)(371.74342654,101.24423366)(371.85341919,101.24424123)
\curveto(371.96342632,101.25423365)(372.07342621,101.26923363)(372.18341919,101.28924123)
\curveto(372.23342605,101.30923359)(372.27842601,101.31423359)(372.31841919,101.30424123)
\curveto(372.35842593,101.3042336)(372.39842589,101.30923359)(372.43841919,101.31924123)
\curveto(372.4884258,101.32923357)(372.54342574,101.32923357)(372.60341919,101.31924123)
\curveto(372.65342563,101.31923358)(372.70342558,101.32423358)(372.75341919,101.33424123)
\lineto(372.88841919,101.33424123)
\curveto(372.94842534,101.35423355)(373.01842527,101.35423355)(373.09841919,101.33424123)
\curveto(373.16842512,101.32423358)(373.23342505,101.32923357)(373.29341919,101.34924123)
\curveto(373.32342496,101.35923354)(373.36342492,101.36423354)(373.41341919,101.36424123)
\lineto(373.53341919,101.36424123)
\lineto(373.99841919,101.36424123)
\moveto(376.32341919,99.81924123)
\curveto(376.00342228,99.91923498)(375.63842265,99.97923492)(375.22841919,99.99924123)
\curveto(374.81842347,100.01923488)(374.40842388,100.02923487)(373.99841919,100.02924123)
\curveto(373.56842472,100.02923487)(373.14842514,100.01923488)(372.73841919,99.99924123)
\curveto(372.32842596,99.97923492)(371.94342634,99.93423497)(371.58341919,99.86424123)
\curveto(371.22342706,99.79423511)(370.90342738,99.68423522)(370.62341919,99.53424123)
\curveto(370.33342795,99.39423551)(370.09842819,99.1992357)(369.91841919,98.94924123)
\curveto(369.80842848,98.78923611)(369.72842856,98.60923629)(369.67841919,98.40924123)
\curveto(369.61842867,98.20923669)(369.5884287,97.96423694)(369.58841919,97.67424123)
\curveto(369.60842868,97.65423725)(369.61842867,97.61923728)(369.61841919,97.56924123)
\curveto(369.60842868,97.51923738)(369.60842868,97.47923742)(369.61841919,97.44924123)
\curveto(369.63842865,97.36923753)(369.65842863,97.29423761)(369.67841919,97.22424123)
\curveto(369.6884286,97.16423774)(369.70842858,97.0992378)(369.73841919,97.02924123)
\curveto(369.85842843,96.75923814)(370.02842826,96.53923836)(370.24841919,96.36924123)
\curveto(370.45842783,96.20923869)(370.70342758,96.07423883)(370.98341919,95.96424123)
\curveto(371.09342719,95.91423899)(371.21342707,95.87423903)(371.34341919,95.84424123)
\curveto(371.46342682,95.82423908)(371.5884267,95.7992391)(371.71841919,95.76924123)
\curveto(371.76842652,95.74923915)(371.82342646,95.73923916)(371.88341919,95.73924123)
\curveto(371.93342635,95.73923916)(371.9834263,95.73423917)(372.03341919,95.72424123)
\curveto(372.12342616,95.71423919)(372.21842607,95.7042392)(372.31841919,95.69424123)
\curveto(372.40842588,95.68423922)(372.50342578,95.67423923)(372.60341919,95.66424123)
\curveto(372.6834256,95.66423924)(372.76842552,95.65923924)(372.85841919,95.64924123)
\lineto(373.09841919,95.64924123)
\lineto(373.27841919,95.64924123)
\curveto(373.30842498,95.63923926)(373.34342494,95.63423927)(373.38341919,95.63424123)
\lineto(373.51841919,95.63424123)
\lineto(373.96841919,95.63424123)
\curveto(374.04842424,95.63423927)(374.13342415,95.62923927)(374.22341919,95.61924123)
\curveto(374.30342398,95.61923928)(374.37842391,95.62923927)(374.44841919,95.64924123)
\lineto(374.71841919,95.64924123)
\curveto(374.73842355,95.64923925)(374.76842352,95.64423926)(374.80841919,95.63424123)
\curveto(374.83842345,95.63423927)(374.86342342,95.63923926)(374.88341919,95.64924123)
\curveto(374.9834233,95.65923924)(375.0834232,95.66423924)(375.18341919,95.66424123)
\curveto(375.27342301,95.67423923)(375.37342291,95.68423922)(375.48341919,95.69424123)
\curveto(375.60342268,95.72423918)(375.72842256,95.73923916)(375.85841919,95.73924123)
\curveto(375.97842231,95.74923915)(376.09342219,95.77423913)(376.20341919,95.81424123)
\curveto(376.50342178,95.89423901)(376.76842152,95.97923892)(376.99841919,96.06924123)
\curveto(377.22842106,96.16923873)(377.44342084,96.31423859)(377.64341919,96.50424123)
\curveto(377.84342044,96.71423819)(377.99342029,96.97923792)(378.09341919,97.29924123)
\curveto(378.11342017,97.33923756)(378.12342016,97.37423753)(378.12341919,97.40424123)
\curveto(378.11342017,97.44423746)(378.11842017,97.48923741)(378.13841919,97.53924123)
\curveto(378.14842014,97.57923732)(378.15842013,97.64923725)(378.16841919,97.74924123)
\curveto(378.17842011,97.85923704)(378.17342011,97.94423696)(378.15341919,98.00424123)
\curveto(378.13342015,98.07423683)(378.12342016,98.14423676)(378.12341919,98.21424123)
\curveto(378.11342017,98.28423662)(378.09842019,98.34923655)(378.07841919,98.40924123)
\curveto(378.01842027,98.60923629)(377.93342035,98.78923611)(377.82341919,98.94924123)
\curveto(377.80342048,98.97923592)(377.7834205,99.0042359)(377.76341919,99.02424123)
\lineto(377.70341919,99.08424123)
\curveto(377.6834206,99.12423578)(377.64342064,99.17423573)(377.58341919,99.23424123)
\curveto(377.44342084,99.33423557)(377.31342097,99.41923548)(377.19341919,99.48924123)
\curveto(377.07342121,99.55923534)(376.92842136,99.62923527)(376.75841919,99.69924123)
\curveto(376.6884216,99.72923517)(376.61842167,99.74923515)(376.54841919,99.75924123)
\curveto(376.47842181,99.77923512)(376.40342188,99.7992351)(376.32341919,99.81924123)
}
}
{
\newrgbcolor{curcolor}{0 0 0}
\pscustom[linestyle=none,fillstyle=solid,fillcolor=curcolor]
{
\newpath
\moveto(368.47841919,106.77385061)
\curveto(368.47842981,106.87384575)(368.4884298,106.96884566)(368.50841919,107.05885061)
\curveto(368.51842977,107.14884548)(368.54842974,107.21384541)(368.59841919,107.25385061)
\curveto(368.67842961,107.31384531)(368.7834295,107.34384528)(368.91341919,107.34385061)
\lineto(369.30341919,107.34385061)
\lineto(370.80341919,107.34385061)
\lineto(377.19341919,107.34385061)
\lineto(378.36341919,107.34385061)
\lineto(378.67841919,107.34385061)
\curveto(378.77841951,107.35384527)(378.85841943,107.33884529)(378.91841919,107.29885061)
\curveto(378.99841929,107.24884538)(379.04841924,107.17384545)(379.06841919,107.07385061)
\curveto(379.07841921,106.98384564)(379.0834192,106.87384575)(379.08341919,106.74385061)
\lineto(379.08341919,106.51885061)
\curveto(379.06341922,106.43884619)(379.04841924,106.36884626)(379.03841919,106.30885061)
\curveto(379.01841927,106.24884638)(378.97841931,106.19884643)(378.91841919,106.15885061)
\curveto(378.85841943,106.11884651)(378.7834195,106.09884653)(378.69341919,106.09885061)
\lineto(378.39341919,106.09885061)
\lineto(377.29841919,106.09885061)
\lineto(371.95841919,106.09885061)
\curveto(371.86842642,106.07884655)(371.79342649,106.06384656)(371.73341919,106.05385061)
\curveto(371.66342662,106.05384657)(371.60342668,106.0238466)(371.55341919,105.96385061)
\curveto(371.50342678,105.89384673)(371.47842681,105.80384682)(371.47841919,105.69385061)
\curveto(371.46842682,105.59384703)(371.46342682,105.48384714)(371.46341919,105.36385061)
\lineto(371.46341919,104.22385061)
\lineto(371.46341919,103.72885061)
\curveto(371.45342683,103.56884906)(371.39342689,103.45884917)(371.28341919,103.39885061)
\curveto(371.25342703,103.37884925)(371.22342706,103.36884926)(371.19341919,103.36885061)
\curveto(371.15342713,103.36884926)(371.10842718,103.36384926)(371.05841919,103.35385061)
\curveto(370.93842735,103.33384929)(370.82842746,103.33884929)(370.72841919,103.36885061)
\curveto(370.62842766,103.40884922)(370.55842773,103.46384916)(370.51841919,103.53385061)
\curveto(370.46842782,103.61384901)(370.44342784,103.73384889)(370.44341919,103.89385061)
\curveto(370.44342784,104.05384857)(370.42842786,104.18884844)(370.39841919,104.29885061)
\curveto(370.3884279,104.34884828)(370.3834279,104.40384822)(370.38341919,104.46385061)
\curveto(370.37342791,104.5238481)(370.35842793,104.58384804)(370.33841919,104.64385061)
\curveto(370.288428,104.79384783)(370.23842805,104.93884769)(370.18841919,105.07885061)
\curveto(370.12842816,105.21884741)(370.05842823,105.35384727)(369.97841919,105.48385061)
\curveto(369.8884284,105.623847)(369.7834285,105.74384688)(369.66341919,105.84385061)
\curveto(369.54342874,105.94384668)(369.41342887,106.03884659)(369.27341919,106.12885061)
\curveto(369.17342911,106.18884644)(369.06342922,106.23384639)(368.94341919,106.26385061)
\curveto(368.82342946,106.30384632)(368.71842957,106.35384627)(368.62841919,106.41385061)
\curveto(368.56842972,106.46384616)(368.52842976,106.53384609)(368.50841919,106.62385061)
\curveto(368.49842979,106.64384598)(368.49342979,106.66884596)(368.49341919,106.69885061)
\curveto(368.49342979,106.7288459)(368.4884298,106.75384587)(368.47841919,106.77385061)
}
}
{
\newrgbcolor{curcolor}{0 0 0}
\pscustom[linestyle=none,fillstyle=solid,fillcolor=curcolor]
{
\newpath
\moveto(368.47841919,115.12345998)
\curveto(368.47842981,115.22345513)(368.4884298,115.31845503)(368.50841919,115.40845998)
\curveto(368.51842977,115.49845485)(368.54842974,115.56345479)(368.59841919,115.60345998)
\curveto(368.67842961,115.66345469)(368.7834295,115.69345466)(368.91341919,115.69345998)
\lineto(369.30341919,115.69345998)
\lineto(370.80341919,115.69345998)
\lineto(377.19341919,115.69345998)
\lineto(378.36341919,115.69345998)
\lineto(378.67841919,115.69345998)
\curveto(378.77841951,115.70345465)(378.85841943,115.68845466)(378.91841919,115.64845998)
\curveto(378.99841929,115.59845475)(379.04841924,115.52345483)(379.06841919,115.42345998)
\curveto(379.07841921,115.33345502)(379.0834192,115.22345513)(379.08341919,115.09345998)
\lineto(379.08341919,114.86845998)
\curveto(379.06341922,114.78845556)(379.04841924,114.71845563)(379.03841919,114.65845998)
\curveto(379.01841927,114.59845575)(378.97841931,114.5484558)(378.91841919,114.50845998)
\curveto(378.85841943,114.46845588)(378.7834195,114.4484559)(378.69341919,114.44845998)
\lineto(378.39341919,114.44845998)
\lineto(377.29841919,114.44845998)
\lineto(371.95841919,114.44845998)
\curveto(371.86842642,114.42845592)(371.79342649,114.41345594)(371.73341919,114.40345998)
\curveto(371.66342662,114.40345595)(371.60342668,114.37345598)(371.55341919,114.31345998)
\curveto(371.50342678,114.24345611)(371.47842681,114.1534562)(371.47841919,114.04345998)
\curveto(371.46842682,113.94345641)(371.46342682,113.83345652)(371.46341919,113.71345998)
\lineto(371.46341919,112.57345998)
\lineto(371.46341919,112.07845998)
\curveto(371.45342683,111.91845843)(371.39342689,111.80845854)(371.28341919,111.74845998)
\curveto(371.25342703,111.72845862)(371.22342706,111.71845863)(371.19341919,111.71845998)
\curveto(371.15342713,111.71845863)(371.10842718,111.71345864)(371.05841919,111.70345998)
\curveto(370.93842735,111.68345867)(370.82842746,111.68845866)(370.72841919,111.71845998)
\curveto(370.62842766,111.75845859)(370.55842773,111.81345854)(370.51841919,111.88345998)
\curveto(370.46842782,111.96345839)(370.44342784,112.08345827)(370.44341919,112.24345998)
\curveto(370.44342784,112.40345795)(370.42842786,112.53845781)(370.39841919,112.64845998)
\curveto(370.3884279,112.69845765)(370.3834279,112.7534576)(370.38341919,112.81345998)
\curveto(370.37342791,112.87345748)(370.35842793,112.93345742)(370.33841919,112.99345998)
\curveto(370.288428,113.14345721)(370.23842805,113.28845706)(370.18841919,113.42845998)
\curveto(370.12842816,113.56845678)(370.05842823,113.70345665)(369.97841919,113.83345998)
\curveto(369.8884284,113.97345638)(369.7834285,114.09345626)(369.66341919,114.19345998)
\curveto(369.54342874,114.29345606)(369.41342887,114.38845596)(369.27341919,114.47845998)
\curveto(369.17342911,114.53845581)(369.06342922,114.58345577)(368.94341919,114.61345998)
\curveto(368.82342946,114.6534557)(368.71842957,114.70345565)(368.62841919,114.76345998)
\curveto(368.56842972,114.81345554)(368.52842976,114.88345547)(368.50841919,114.97345998)
\curveto(368.49842979,114.99345536)(368.49342979,115.01845533)(368.49341919,115.04845998)
\curveto(368.49342979,115.07845527)(368.4884298,115.10345525)(368.47841919,115.12345998)
}
}
{
\newrgbcolor{curcolor}{0 0 0}
\pscustom[linestyle=none,fillstyle=solid,fillcolor=curcolor]
{
\newpath
\moveto(399.21476562,42.02236623)
\curveto(399.26476637,42.04235669)(399.32476631,42.06735666)(399.39476562,42.09736623)
\curveto(399.46476617,42.1273566)(399.5397661,42.14735658)(399.61976563,42.15736623)
\curveto(399.68976594,42.17735655)(399.75976588,42.17735655)(399.82976562,42.15736623)
\curveto(399.88976574,42.14735658)(399.9347657,42.10735662)(399.96476562,42.03736623)
\curveto(399.98476565,41.98735674)(399.99476564,41.9273568)(399.99476563,41.85736623)
\lineto(399.99476563,41.64736623)
\lineto(399.99476563,41.19736623)
\curveto(399.99476564,41.04735768)(399.96976567,40.9273578)(399.91976563,40.83736623)
\curveto(399.85976578,40.73735799)(399.75476588,40.66235807)(399.60476563,40.61236623)
\curveto(399.45476618,40.57235816)(399.31976631,40.5273582)(399.19976562,40.47736623)
\curveto(398.9397667,40.36735836)(398.66976697,40.26735846)(398.38976562,40.17736623)
\curveto(398.10976752,40.08735864)(397.8347678,39.98735874)(397.56476563,39.87736623)
\curveto(397.47476816,39.84735888)(397.38976825,39.81735891)(397.30976563,39.78736623)
\curveto(397.22976841,39.76735896)(397.15476848,39.73735899)(397.08476562,39.69736623)
\curveto(397.01476862,39.66735906)(396.95476868,39.62235911)(396.90476562,39.56236623)
\curveto(396.85476878,39.50235923)(396.81476882,39.42235931)(396.78476562,39.32236623)
\curveto(396.76476887,39.27235946)(396.75976888,39.21235952)(396.76976562,39.14236623)
\lineto(396.76976562,38.94736623)
\lineto(396.76976562,36.11236623)
\lineto(396.76976562,35.81236623)
\curveto(396.75976888,35.70236303)(396.75976888,35.59736313)(396.76976562,35.49736623)
\curveto(396.77976885,35.39736333)(396.79476884,35.30236343)(396.81476563,35.21236623)
\curveto(396.8347688,35.1323636)(396.87476876,35.07236366)(396.93476563,35.03236623)
\curveto(397.0347686,34.95236378)(397.14976848,34.89236384)(397.27976562,34.85236623)
\curveto(397.39976824,34.82236391)(397.52476811,34.78236395)(397.65476562,34.73236623)
\curveto(397.88476775,34.6323641)(398.12476751,34.53736419)(398.37476563,34.44736623)
\curveto(398.62476701,34.36736436)(398.86476677,34.27736445)(399.09476562,34.17736623)
\curveto(399.15476648,34.15736457)(399.22476641,34.1323646)(399.30476563,34.10236623)
\curveto(399.37476626,34.08236465)(399.44976618,34.05736467)(399.52976562,34.02736623)
\curveto(399.60976602,33.99736473)(399.68476595,33.96236477)(399.75476563,33.92236623)
\curveto(399.81476582,33.89236484)(399.85976578,33.85736487)(399.88976562,33.81736623)
\curveto(399.94976568,33.73736499)(399.98476565,33.6273651)(399.99476563,33.48736623)
\lineto(399.99476563,33.06736623)
\lineto(399.99476563,32.82736623)
\curveto(399.98476565,32.75736597)(399.95976568,32.69736603)(399.91976563,32.64736623)
\curveto(399.88976574,32.59736613)(399.84476579,32.56736616)(399.78476562,32.55736623)
\curveto(399.72476591,32.55736617)(399.66476597,32.56236617)(399.60476563,32.57236623)
\curveto(399.5347661,32.59236614)(399.46976617,32.61236612)(399.40976562,32.63236623)
\curveto(399.3397663,32.66236607)(399.28976634,32.68736604)(399.25976562,32.70736623)
\curveto(398.9397667,32.84736588)(398.62476701,32.97236576)(398.31476563,33.08236623)
\curveto(397.99476764,33.19236554)(397.67476796,33.31236542)(397.35476563,33.44236623)
\curveto(397.1347685,33.5323652)(396.91976871,33.61736511)(396.70976562,33.69736623)
\curveto(396.48976915,33.77736495)(396.26976937,33.86236487)(396.04976563,33.95236623)
\curveto(395.32977031,34.25236448)(394.60477103,34.53736419)(393.87476563,34.80736623)
\curveto(393.1347725,35.07736365)(392.39977323,35.36236337)(391.66976563,35.66236623)
\curveto(391.40977422,35.77236296)(391.14477449,35.87236286)(390.87476563,35.96236623)
\curveto(390.60477503,36.06236267)(390.33977529,36.16736256)(390.07976562,36.27736623)
\curveto(389.96977566,36.3273624)(389.84977579,36.37236236)(389.71976562,36.41236623)
\curveto(389.57977606,36.46236227)(389.47977616,36.5323622)(389.41976563,36.62236623)
\curveto(389.37977626,36.66236207)(389.34977629,36.727362)(389.32976562,36.81736623)
\curveto(389.31977632,36.83736189)(389.31977632,36.85736187)(389.32976562,36.87736623)
\curveto(389.3297763,36.90736182)(389.32477631,36.9323618)(389.31476563,36.95236623)
\curveto(389.31477632,37.1323616)(389.31477632,37.34236139)(389.31476563,37.58236623)
\curveto(389.30477633,37.82236091)(389.33977629,37.99736073)(389.41976563,38.10736623)
\curveto(389.47977616,38.18736054)(389.57977606,38.24736048)(389.71976562,38.28736623)
\curveto(389.84977579,38.33736039)(389.96977566,38.38736034)(390.07976562,38.43736623)
\curveto(390.30977532,38.53736019)(390.53977509,38.6273601)(390.76976562,38.70736623)
\curveto(390.99977463,38.78735994)(391.22977441,38.87735985)(391.45976562,38.97736623)
\curveto(391.65977398,39.05735967)(391.86477377,39.1323596)(392.07476562,39.20236623)
\curveto(392.28477335,39.28235945)(392.48977315,39.36735936)(392.68976563,39.45736623)
\curveto(393.41977222,39.75735897)(394.15977148,40.04235869)(394.90976562,40.31236623)
\curveto(395.64976998,40.59235814)(396.38476925,40.88735784)(397.11476563,41.19736623)
\curveto(397.20476843,41.23735749)(397.28976835,41.26735746)(397.36976563,41.28736623)
\curveto(397.44976818,41.31735741)(397.5347681,41.34735738)(397.62476563,41.37736623)
\curveto(397.88476775,41.48735724)(398.14976748,41.59235714)(398.41976563,41.69236623)
\curveto(398.68976694,41.80235693)(398.95476668,41.91235682)(399.21476562,42.02236623)
\moveto(395.56976563,38.81236623)
\curveto(395.53977009,38.90235983)(395.48977015,38.95735977)(395.41976563,38.97736623)
\curveto(395.34977028,39.00735972)(395.27477036,39.01235972)(395.19476563,38.99236623)
\curveto(395.10477053,38.98235975)(395.01977062,38.95735977)(394.93976563,38.91736623)
\curveto(394.84977078,38.88735984)(394.77477086,38.85735987)(394.71476562,38.82736623)
\curveto(394.67477096,38.80735992)(394.63977099,38.79735993)(394.60976563,38.79736623)
\curveto(394.57977105,38.79735993)(394.54477109,38.78735994)(394.50476563,38.76736623)
\lineto(394.26476562,38.67736623)
\curveto(394.17477146,38.65736007)(394.08477155,38.6273601)(393.99476563,38.58736623)
\curveto(393.634772,38.43736029)(393.26977236,38.30236043)(392.89976562,38.18236623)
\curveto(392.51977312,38.07236066)(392.14977349,37.94236079)(391.78976562,37.79236623)
\curveto(391.67977395,37.74236099)(391.56977406,37.69736103)(391.45976562,37.65736623)
\curveto(391.34977429,37.6273611)(391.24477439,37.58736114)(391.14476562,37.53736623)
\curveto(391.09477454,37.51736121)(391.04977459,37.49236124)(391.00976562,37.46236623)
\curveto(390.95977468,37.44236129)(390.9347747,37.39236134)(390.93476563,37.31236623)
\curveto(390.95477468,37.29236144)(390.96977466,37.27236146)(390.97976563,37.25236623)
\curveto(390.98977465,37.2323615)(391.00477463,37.21236152)(391.02476562,37.19236623)
\curveto(391.07477456,37.15236158)(391.12977451,37.12236161)(391.18976563,37.10236623)
\curveto(391.23977439,37.08236165)(391.29477434,37.06236167)(391.35476563,37.04236623)
\curveto(391.46477417,36.99236174)(391.57477406,36.95236178)(391.68476563,36.92236623)
\curveto(391.79477384,36.89236184)(391.90477373,36.85236188)(392.01476562,36.80236623)
\curveto(392.40477323,36.6323621)(392.79977283,36.48236225)(393.19976562,36.35236623)
\curveto(393.59977204,36.2323625)(393.98977165,36.09236264)(394.36976563,35.93236623)
\lineto(394.51976562,35.87236623)
\curveto(394.56977106,35.86236287)(394.61977102,35.84736288)(394.66976563,35.82736623)
\lineto(394.90976562,35.73736623)
\curveto(394.98977065,35.70736302)(395.06977056,35.68236305)(395.14976562,35.66236623)
\curveto(395.19977044,35.64236309)(395.25477038,35.6323631)(395.31476563,35.63236623)
\curveto(395.37477026,35.64236309)(395.42477021,35.65736307)(395.46476562,35.67736623)
\curveto(395.54477009,35.727363)(395.58977005,35.8323629)(395.59976562,35.99236623)
\lineto(395.59976562,36.44236623)
\lineto(395.59976562,38.04736623)
\curveto(395.59977004,38.15736057)(395.60477003,38.29236044)(395.61476563,38.45236623)
\curveto(395.61477002,38.61236012)(395.59977004,38.73236)(395.56976563,38.81236623)
}
}
{
\newrgbcolor{curcolor}{0 0 0}
\pscustom[linestyle=none,fillstyle=solid,fillcolor=curcolor]
{
\newpath
\moveto(395.95976562,50.56392873)
\curveto(396.00976962,50.57392038)(396.07976955,50.57892038)(396.16976563,50.57892873)
\curveto(396.24976938,50.57892038)(396.31476932,50.57392038)(396.36476563,50.56392873)
\curveto(396.40476923,50.56392039)(396.44476919,50.5589204)(396.48476563,50.54892873)
\lineto(396.60476563,50.54892873)
\curveto(396.68476895,50.52892043)(396.76476887,50.51892044)(396.84476562,50.51892873)
\curveto(396.92476871,50.51892044)(397.00476863,50.50892045)(397.08476562,50.48892873)
\curveto(397.12476851,50.47892048)(397.16476847,50.47392048)(397.20476562,50.47392873)
\curveto(397.2347684,50.47392048)(397.26976837,50.46892049)(397.30976563,50.45892873)
\curveto(397.41976821,50.42892053)(397.52476811,50.39892056)(397.62476563,50.36892873)
\curveto(397.72476791,50.34892061)(397.82476781,50.31892064)(397.92476563,50.27892873)
\curveto(398.27476736,50.13892082)(398.58976704,49.96892099)(398.86976563,49.76892873)
\curveto(399.14976648,49.56892139)(399.38976624,49.31892164)(399.58976562,49.01892873)
\curveto(399.68976594,48.86892209)(399.77476586,48.72392223)(399.84476562,48.58392873)
\curveto(399.89476574,48.47392248)(399.9347657,48.36392259)(399.96476562,48.25392873)
\curveto(399.99476564,48.1539228)(400.02476561,48.04892291)(400.05476563,47.93892873)
\curveto(400.07476556,47.86892309)(400.08476555,47.80392315)(400.08476562,47.74392873)
\curveto(400.09476554,47.68392327)(400.10976553,47.62392333)(400.12976563,47.56392873)
\lineto(400.12976563,47.41392873)
\curveto(400.14976548,47.36392359)(400.15976547,47.28892367)(400.15976562,47.18892873)
\curveto(400.16976547,47.08892387)(400.16476547,47.00892395)(400.14476562,46.94892873)
\lineto(400.14476562,46.79892873)
\curveto(400.1347655,46.7589242)(400.12976551,46.71392424)(400.12976563,46.66392873)
\curveto(400.12976551,46.62392433)(400.12476551,46.57892438)(400.11476563,46.52892873)
\curveto(400.07476556,46.37892458)(400.0397656,46.22892473)(400.00976562,46.07892873)
\curveto(399.97976565,45.93892502)(399.9347657,45.79892516)(399.87476563,45.65892873)
\curveto(399.79476584,45.4589255)(399.69476594,45.27892568)(399.57476562,45.11892873)
\lineto(399.42476563,44.93892873)
\curveto(399.36476627,44.87892608)(399.32476631,44.80892615)(399.30476563,44.72892873)
\curveto(399.29476634,44.66892629)(399.30976632,44.61892634)(399.34976562,44.57892873)
\curveto(399.37976625,44.54892641)(399.42476621,44.52392643)(399.48476563,44.50392873)
\curveto(399.54476609,44.49392646)(399.60976602,44.48392647)(399.67976563,44.47392873)
\curveto(399.7397659,44.47392648)(399.78476585,44.46392649)(399.81476563,44.44392873)
\curveto(399.86476577,44.40392655)(399.90976572,44.3589266)(399.94976562,44.30892873)
\curveto(399.96976567,44.2589267)(399.98476565,44.18892677)(399.99476563,44.09892873)
\lineto(399.99476563,43.82892873)
\curveto(399.99476564,43.73892722)(399.98976564,43.6539273)(399.97976563,43.57392873)
\curveto(399.95976568,43.49392746)(399.9397657,43.43392752)(399.91976563,43.39392873)
\curveto(399.89976574,43.37392758)(399.87476576,43.3539276)(399.84476562,43.33392873)
\lineto(399.75476563,43.27392873)
\curveto(399.67476596,43.24392771)(399.55476608,43.22892773)(399.39476562,43.22892873)
\curveto(399.2347664,43.23892772)(399.09976654,43.24392771)(398.98976563,43.24392873)
\lineto(390.18476563,43.24392873)
\curveto(390.06477557,43.24392771)(389.93977569,43.23892772)(389.80976563,43.22892873)
\curveto(389.66977596,43.22892773)(389.55977608,43.2539277)(389.47976563,43.30392873)
\curveto(389.41977622,43.34392761)(389.36977626,43.40892755)(389.32976562,43.49892873)
\curveto(389.3297763,43.51892744)(389.3297763,43.54392741)(389.32976562,43.57392873)
\curveto(389.31977632,43.60392735)(389.31477632,43.62892733)(389.31476563,43.64892873)
\curveto(389.30477633,43.78892717)(389.30477633,43.93392702)(389.31476563,44.08392873)
\curveto(389.31477632,44.24392671)(389.35477628,44.3539266)(389.43476563,44.41392873)
\curveto(389.51477612,44.46392649)(389.629776,44.48892647)(389.77976562,44.48892873)
\lineto(390.18476563,44.48892873)
\lineto(391.93976563,44.48892873)
\lineto(392.19476563,44.48892873)
\lineto(392.47976563,44.48892873)
\curveto(392.56977306,44.49892646)(392.65477298,44.50892645)(392.73476563,44.51892873)
\curveto(392.80477283,44.53892642)(392.85477278,44.56892639)(392.88476562,44.60892873)
\curveto(392.91477272,44.64892631)(392.91977272,44.69392626)(392.89976562,44.74392873)
\curveto(392.87977275,44.79392616)(392.85977278,44.83392612)(392.83976562,44.86392873)
\curveto(392.79977283,44.91392604)(392.75977288,44.958926)(392.71976562,44.99892873)
\lineto(392.59976562,45.14892873)
\curveto(392.54977309,45.21892574)(392.50477313,45.28892567)(392.46476562,45.35892873)
\lineto(392.34476562,45.59892873)
\curveto(392.25477338,45.77892518)(392.18977345,45.99392496)(392.14976562,46.24392873)
\curveto(392.10977352,46.49392446)(392.08977355,46.74892421)(392.08976562,47.00892873)
\curveto(392.08977355,47.26892369)(392.11477352,47.52392343)(392.16476562,47.77392873)
\curveto(392.20477343,48.02392293)(392.26477337,48.24392271)(392.34476562,48.43392873)
\curveto(392.51477312,48.83392212)(392.74977289,49.17892178)(393.04976563,49.46892873)
\curveto(393.34977228,49.7589212)(393.69977194,49.98892097)(394.09976562,50.15892873)
\curveto(394.20977142,50.20892075)(394.31977132,50.24892071)(394.42976563,50.27892873)
\curveto(394.52977111,50.31892064)(394.634771,50.3589206)(394.74476563,50.39892873)
\curveto(394.85477078,50.42892053)(394.96977066,50.44892051)(395.08976562,50.45892873)
\lineto(395.41976563,50.51892873)
\curveto(395.44977018,50.52892043)(395.48477015,50.53392042)(395.52476562,50.53392873)
\curveto(395.55477008,50.53392042)(395.58477005,50.53892042)(395.61476563,50.54892873)
\curveto(395.67476996,50.56892039)(395.7347699,50.56892039)(395.79476563,50.54892873)
\curveto(395.84476979,50.53892042)(395.89976974,50.54392041)(395.95976562,50.56392873)
\moveto(396.34976562,49.22892873)
\curveto(396.29976934,49.24892171)(396.23976939,49.2539217)(396.16976563,49.24392873)
\curveto(396.09976954,49.23392172)(396.0347696,49.22892173)(395.97476562,49.22892873)
\curveto(395.80476983,49.22892173)(395.64476999,49.21892174)(395.49476563,49.19892873)
\curveto(395.34477029,49.18892177)(395.20977042,49.1589218)(395.08976562,49.10892873)
\curveto(394.98977065,49.07892188)(394.89977074,49.0539219)(394.81976563,49.03392873)
\curveto(394.73977089,49.01392194)(394.65977098,48.98392197)(394.57976562,48.94392873)
\curveto(394.32977131,48.83392212)(394.09977154,48.68392227)(393.88976562,48.49392873)
\curveto(393.66977196,48.30392265)(393.50477213,48.08392287)(393.39476562,47.83392873)
\curveto(393.36477227,47.7539232)(393.33977229,47.67392328)(393.31976563,47.59392873)
\curveto(393.28977235,47.52392343)(393.26477237,47.44892351)(393.24476563,47.36892873)
\curveto(393.21477242,47.2589237)(393.19977243,47.14892381)(393.19976562,47.03892873)
\curveto(393.18977245,46.92892403)(393.18477245,46.80892415)(393.18476563,46.67892873)
\curveto(393.19477244,46.62892433)(393.20477243,46.58392437)(393.21476562,46.54392873)
\lineto(393.21476562,46.40892873)
\lineto(393.27476562,46.13892873)
\curveto(393.29477234,46.0589249)(393.32477231,45.97892498)(393.36476563,45.89892873)
\curveto(393.50477213,45.5589254)(393.71477192,45.28892567)(393.99476563,45.08892873)
\curveto(394.26477137,44.88892607)(394.58477105,44.72892623)(394.95476562,44.60892873)
\curveto(395.06477057,44.56892639)(395.17477046,44.54392641)(395.28476562,44.53392873)
\curveto(395.39477024,44.52392643)(395.50977012,44.50392645)(395.62976563,44.47392873)
\curveto(395.67976995,44.46392649)(395.72476991,44.46392649)(395.76476562,44.47392873)
\curveto(395.80476983,44.48392647)(395.84976978,44.47892648)(395.89976562,44.45892873)
\curveto(395.94976968,44.44892651)(396.02476961,44.44392651)(396.12476563,44.44392873)
\curveto(396.21476942,44.44392651)(396.28476935,44.44892651)(396.33476562,44.45892873)
\lineto(396.45476562,44.45892873)
\curveto(396.49476914,44.46892649)(396.5347691,44.47392648)(396.57476562,44.47392873)
\curveto(396.61476902,44.47392648)(396.64976898,44.47892648)(396.67976563,44.48892873)
\curveto(396.70976892,44.49892646)(396.74476889,44.50392645)(396.78476562,44.50392873)
\curveto(396.81476882,44.50392645)(396.84476879,44.50892645)(396.87476563,44.51892873)
\curveto(396.95476868,44.53892642)(397.0347686,44.5539264)(397.11476563,44.56392873)
\lineto(397.35476563,44.62392873)
\curveto(397.69476794,44.73392622)(397.98476765,44.88392607)(398.22476562,45.07392873)
\curveto(398.46476717,45.27392568)(398.66476697,45.51892544)(398.82476562,45.80892873)
\curveto(398.87476676,45.89892506)(398.91476672,45.99392496)(398.94476563,46.09392873)
\curveto(398.96476667,46.19392476)(398.98976664,46.29892466)(399.01976562,46.40892873)
\curveto(399.0397666,46.4589245)(399.04976658,46.50392445)(399.04976563,46.54392873)
\curveto(399.0397666,46.59392436)(399.0397666,46.64392431)(399.04976563,46.69392873)
\curveto(399.05976658,46.73392422)(399.06476657,46.77892418)(399.06476563,46.82892873)
\lineto(399.06476563,46.96392873)
\lineto(399.06476563,47.09892873)
\curveto(399.05476658,47.13892382)(399.04976658,47.17392378)(399.04976563,47.20392873)
\curveto(399.04976658,47.23392372)(399.04476659,47.26892369)(399.03476562,47.30892873)
\curveto(399.01476662,47.38892357)(398.99976664,47.46392349)(398.98976563,47.53392873)
\curveto(398.96976667,47.60392335)(398.94476669,47.67892328)(398.91476562,47.75892873)
\curveto(398.78476685,48.06892289)(398.61476702,48.31892264)(398.40476562,48.50892873)
\curveto(398.18476745,48.69892226)(397.91976771,48.8589221)(397.60976563,48.98892873)
\curveto(397.46976817,49.03892192)(397.32976831,49.07392188)(397.18976563,49.09392873)
\curveto(397.03976859,49.12392183)(396.88976875,49.1589218)(396.73976563,49.19892873)
\curveto(396.68976895,49.21892174)(396.64476899,49.22392173)(396.60476563,49.21392873)
\curveto(396.55476908,49.21392174)(396.50476913,49.21892174)(396.45476562,49.22892873)
\lineto(396.34976562,49.22892873)
}
}
{
\newrgbcolor{curcolor}{0 0 0}
\pscustom[linestyle=none,fillstyle=solid,fillcolor=curcolor]
{
\newpath
\moveto(392.08976562,55.69017873)
\curveto(392.08977355,55.92017394)(392.14977349,56.05017381)(392.26976562,56.08017873)
\curveto(392.37977325,56.11017375)(392.54477309,56.12517374)(392.76476562,56.12517873)
\lineto(393.04976563,56.12517873)
\curveto(393.13977249,56.12517374)(393.21477242,56.10017376)(393.27476562,56.05017873)
\curveto(393.35477228,55.99017387)(393.39977224,55.90517396)(393.40976562,55.79517873)
\curveto(393.40977222,55.68517418)(393.42477221,55.57517429)(393.45476562,55.46517873)
\curveto(393.48477215,55.32517454)(393.51477212,55.19017467)(393.54476563,55.06017873)
\curveto(393.57477206,54.94017492)(393.61477202,54.82517504)(393.66476562,54.71517873)
\curveto(393.79477184,54.42517544)(393.97477166,54.19017567)(394.20476562,54.01017873)
\curveto(394.42477121,53.83017603)(394.67977095,53.67517619)(394.96976562,53.54517873)
\curveto(395.07977055,53.50517636)(395.19477044,53.47517639)(395.31476563,53.45517873)
\curveto(395.42477021,53.43517643)(395.53977009,53.41017645)(395.65976562,53.38017873)
\curveto(395.70976992,53.37017649)(395.75976988,53.3651765)(395.80976563,53.36517873)
\curveto(395.85976978,53.37517649)(395.90976972,53.37517649)(395.95976562,53.36517873)
\curveto(396.07976955,53.33517653)(396.21976941,53.32017654)(396.37976563,53.32017873)
\curveto(396.52976911,53.33017653)(396.67476896,53.33517653)(396.81476563,53.33517873)
\lineto(398.65976562,53.33517873)
\lineto(399.00476563,53.33517873)
\curveto(399.12476651,53.33517653)(399.2397664,53.33017653)(399.34976562,53.32017873)
\curveto(399.45976618,53.31017655)(399.55476608,53.30517656)(399.63476562,53.30517873)
\curveto(399.71476592,53.31517655)(399.78476585,53.29517657)(399.84476562,53.24517873)
\curveto(399.91476572,53.19517667)(399.95476568,53.11517675)(399.96476562,53.00517873)
\curveto(399.97476566,52.90517696)(399.97976565,52.79517707)(399.97976563,52.67517873)
\lineto(399.97976563,52.40517873)
\curveto(399.95976568,52.35517751)(399.94476569,52.30517756)(399.93476563,52.25517873)
\curveto(399.91476572,52.21517765)(399.88976574,52.18517768)(399.85976563,52.16517873)
\curveto(399.78976584,52.11517775)(399.70476593,52.08517778)(399.60476563,52.07517873)
\lineto(399.27476562,52.07517873)
\lineto(398.11976563,52.07517873)
\lineto(393.96476562,52.07517873)
\lineto(392.92976563,52.07517873)
\lineto(392.62976563,52.07517873)
\curveto(392.52977311,52.08517778)(392.44477319,52.11517775)(392.37476563,52.16517873)
\curveto(392.3347733,52.19517767)(392.30477333,52.24517762)(392.28476562,52.31517873)
\curveto(392.26477337,52.39517747)(392.25477338,52.48017738)(392.25476563,52.57017873)
\curveto(392.24477339,52.6601772)(392.24477339,52.75017711)(392.25476563,52.84017873)
\curveto(392.26477337,52.93017693)(392.27977335,53.00017686)(392.29976563,53.05017873)
\curveto(392.32977331,53.13017673)(392.38977325,53.18017668)(392.47976563,53.20017873)
\curveto(392.55977308,53.23017663)(392.64977299,53.24517662)(392.74976563,53.24517873)
\lineto(393.04976563,53.24517873)
\curveto(393.14977248,53.24517662)(393.23977239,53.2651766)(393.31976563,53.30517873)
\curveto(393.33977229,53.31517655)(393.35477228,53.32517654)(393.36476563,53.33517873)
\lineto(393.40976562,53.38017873)
\curveto(393.40977222,53.49017637)(393.36477227,53.58017628)(393.27476562,53.65017873)
\curveto(393.17477246,53.72017614)(393.09477254,53.78017608)(393.03476562,53.83017873)
\lineto(392.94476563,53.92017873)
\curveto(392.8347728,54.01017585)(392.71977292,54.13517573)(392.59976562,54.29517873)
\curveto(392.47977315,54.45517541)(392.38977325,54.60517526)(392.32976562,54.74517873)
\curveto(392.27977335,54.83517503)(392.24477339,54.93017493)(392.22476562,55.03017873)
\curveto(392.19477344,55.13017473)(392.16477347,55.23517463)(392.13476562,55.34517873)
\curveto(392.12477351,55.40517446)(392.11977352,55.4651744)(392.11976563,55.52517873)
\curveto(392.10977352,55.58517428)(392.09977353,55.64017422)(392.08976562,55.69017873)
}
}
{
\newrgbcolor{curcolor}{0 0 0}
\pscustom[linestyle=none,fillstyle=solid,fillcolor=curcolor]
{
}
}
{
\newrgbcolor{curcolor}{0 0 0}
\pscustom[linestyle=none,fillstyle=solid,fillcolor=curcolor]
{
\newpath
\moveto(389.38976562,65.05510061)
\curveto(389.38977625,65.15509575)(389.39977623,65.25009566)(389.41976563,65.34010061)
\curveto(389.4297762,65.43009548)(389.45977618,65.49509541)(389.50976562,65.53510061)
\curveto(389.58977605,65.59509531)(389.69477594,65.62509528)(389.82476562,65.62510061)
\lineto(390.21476562,65.62510061)
\lineto(391.71476562,65.62510061)
\lineto(398.10476563,65.62510061)
\lineto(399.27476562,65.62510061)
\lineto(399.58976562,65.62510061)
\curveto(399.68976594,65.63509527)(399.76976587,65.62009529)(399.82976562,65.58010061)
\curveto(399.90976572,65.53009538)(399.95976568,65.45509545)(399.97976563,65.35510061)
\curveto(399.98976564,65.26509564)(399.99476564,65.15509575)(399.99476563,65.02510061)
\lineto(399.99476563,64.80010061)
\curveto(399.97476566,64.72009619)(399.95976568,64.65009626)(399.94976562,64.59010061)
\curveto(399.92976571,64.53009638)(399.88976574,64.48009643)(399.82976562,64.44010061)
\curveto(399.76976587,64.40009651)(399.69476594,64.38009653)(399.60476563,64.38010061)
\lineto(399.30476563,64.38010061)
\lineto(398.20976562,64.38010061)
\lineto(392.86976563,64.38010061)
\curveto(392.77977285,64.36009655)(392.70477293,64.34509656)(392.64476562,64.33510061)
\curveto(392.57477306,64.33509657)(392.51477312,64.3050966)(392.46476562,64.24510061)
\curveto(392.41477322,64.17509673)(392.38977325,64.08509682)(392.38976562,63.97510061)
\curveto(392.37977325,63.87509703)(392.37477326,63.76509714)(392.37476563,63.64510061)
\lineto(392.37476563,62.50510061)
\lineto(392.37476563,62.01010061)
\curveto(392.36477327,61.85009906)(392.30477333,61.74009917)(392.19476563,61.68010061)
\curveto(392.16477347,61.66009925)(392.1347735,61.65009926)(392.10476563,61.65010061)
\curveto(392.06477357,61.65009926)(392.01977362,61.64509926)(391.96976562,61.63510061)
\curveto(391.84977379,61.61509929)(391.73977389,61.62009929)(391.63976562,61.65010061)
\curveto(391.53977409,61.69009922)(391.46977416,61.74509916)(391.42976563,61.81510061)
\curveto(391.37977426,61.89509901)(391.35477428,62.01509889)(391.35476563,62.17510061)
\curveto(391.35477428,62.33509857)(391.33977429,62.47009844)(391.30976563,62.58010061)
\curveto(391.29977433,62.63009828)(391.29477434,62.68509822)(391.29476563,62.74510061)
\curveto(391.28477435,62.8050981)(391.26977436,62.86509804)(391.24976563,62.92510061)
\curveto(391.19977443,63.07509783)(391.14977449,63.22009769)(391.09976562,63.36010061)
\curveto(391.03977459,63.50009741)(390.96977466,63.63509727)(390.88976562,63.76510061)
\curveto(390.79977483,63.905097)(390.69477494,64.02509688)(390.57476562,64.12510061)
\curveto(390.45477518,64.22509668)(390.32477531,64.32009659)(390.18476563,64.41010061)
\curveto(390.08477555,64.47009644)(389.97477566,64.51509639)(389.85476563,64.54510061)
\curveto(389.7347759,64.58509632)(389.629776,64.63509627)(389.53976562,64.69510061)
\curveto(389.47977616,64.74509616)(389.43977619,64.81509609)(389.41976563,64.90510061)
\curveto(389.40977622,64.92509598)(389.40477623,64.95009596)(389.40476562,64.98010061)
\curveto(389.40477623,65.0100959)(389.39977623,65.03509587)(389.38976562,65.05510061)
}
}
{
\newrgbcolor{curcolor}{0 0 0}
\pscustom[linestyle=none,fillstyle=solid,fillcolor=curcolor]
{
\newpath
\moveto(389.58476562,70.85470998)
\lineto(389.58476562,74.45470998)
\lineto(389.58476562,75.09970998)
\curveto(389.58477605,75.17970345)(389.58977605,75.25470338)(389.59976562,75.32470998)
\curveto(389.59977603,75.39470324)(389.60977602,75.45470318)(389.62976563,75.50470998)
\curveto(389.65977598,75.57470306)(389.71977592,75.629703)(389.80976563,75.66970998)
\curveto(389.83977579,75.68970294)(389.87977576,75.69970293)(389.92976563,75.69970998)
\lineto(390.06476563,75.69970998)
\curveto(390.17477546,75.70970292)(390.27977536,75.70470293)(390.37976563,75.68470998)
\curveto(390.47977515,75.67470296)(390.54977509,75.63970299)(390.58976562,75.57970998)
\curveto(390.65977498,75.48970314)(390.69477494,75.35470328)(390.69476563,75.17470998)
\curveto(390.68477495,74.99470364)(390.67977495,74.8297038)(390.67976563,74.67970998)
\lineto(390.67976563,72.68470998)
\lineto(390.67976563,72.18970998)
\lineto(390.67976563,72.05470998)
\curveto(390.67977495,72.01470662)(390.68477495,71.97470666)(390.69476563,71.93470998)
\lineto(390.69476563,71.72470998)
\curveto(390.72477491,71.61470702)(390.76477487,71.5347071)(390.81476563,71.48470998)
\curveto(390.85477478,71.4347072)(390.90977472,71.39970723)(390.97976563,71.37970998)
\curveto(391.03977459,71.35970727)(391.10977452,71.34470729)(391.18976563,71.33470998)
\curveto(391.26977436,71.32470731)(391.35977428,71.30470733)(391.45976562,71.27470998)
\curveto(391.65977398,71.22470741)(391.86477377,71.18470745)(392.07476562,71.15470998)
\curveto(392.28477335,71.12470751)(392.48977315,71.08470755)(392.68976563,71.03470998)
\curveto(392.75977288,71.01470762)(392.82977281,71.00470763)(392.89976562,71.00470998)
\curveto(392.95977268,71.00470763)(393.02477261,70.99470764)(393.09476562,70.97470998)
\curveto(393.12477251,70.96470767)(393.16477247,70.95470768)(393.21476562,70.94470998)
\curveto(393.25477238,70.94470769)(393.29477234,70.94970768)(393.33476562,70.95970998)
\curveto(393.38477225,70.97970765)(393.42977221,71.00470763)(393.46976562,71.03470998)
\curveto(393.49977214,71.07470756)(393.50477213,71.1347075)(393.48476563,71.21470998)
\curveto(393.46477217,71.27470736)(393.43977219,71.3347073)(393.40976562,71.39470998)
\curveto(393.36977226,71.45470718)(393.3347723,71.51470712)(393.30476563,71.57470998)
\curveto(393.28477235,71.634707)(393.26977236,71.68470695)(393.25976562,71.72470998)
\curveto(393.17977245,71.91470672)(393.12477251,72.11970651)(393.09476562,72.33970998)
\curveto(393.06477257,72.56970606)(393.05477258,72.79970583)(393.06476563,73.02970998)
\curveto(393.06477257,73.26970536)(393.08977255,73.49970513)(393.13976562,73.71970998)
\curveto(393.17977245,73.93970469)(393.23977239,74.13970449)(393.31976563,74.31970998)
\curveto(393.33977229,74.36970426)(393.35977228,74.41470422)(393.37976563,74.45470998)
\curveto(393.39977224,74.50470413)(393.42477221,74.55470408)(393.45476562,74.60470998)
\curveto(393.66477197,74.95470368)(393.89477174,75.2347034)(394.14476562,75.44470998)
\curveto(394.39477124,75.66470297)(394.71977092,75.85970277)(395.11976563,76.02970998)
\curveto(395.22977041,76.07970255)(395.33977029,76.11470252)(395.44976562,76.13470998)
\curveto(395.55977008,76.15470248)(395.67476996,76.17970245)(395.79476563,76.20970998)
\curveto(395.82476981,76.21970241)(395.86976977,76.22470241)(395.92976563,76.22470998)
\curveto(395.98976965,76.24470239)(396.05976958,76.25470238)(396.13976562,76.25470998)
\curveto(396.20976942,76.25470238)(396.27476936,76.26470237)(396.33476562,76.28470998)
\lineto(396.49976563,76.28470998)
\curveto(396.54976908,76.29470234)(396.61976901,76.29970233)(396.70976562,76.29970998)
\curveto(396.79976884,76.29970233)(396.86976877,76.28970234)(396.91976563,76.26970998)
\curveto(396.97976865,76.24970238)(397.03976859,76.24470239)(397.09976562,76.25470998)
\curveto(397.14976848,76.26470237)(397.19976844,76.25970237)(397.24976563,76.23970998)
\curveto(397.40976822,76.19970243)(397.55976808,76.16470247)(397.69976562,76.13470998)
\curveto(397.8397678,76.10470253)(397.97476766,76.05970257)(398.10476563,75.99970998)
\curveto(398.47476716,75.83970279)(398.80976682,75.61970301)(399.10976563,75.33970998)
\curveto(399.40976622,75.05970357)(399.639766,74.73970389)(399.79976563,74.37970998)
\curveto(399.87976575,74.20970442)(399.95476568,74.00970462)(400.02476562,73.77970998)
\curveto(400.06476557,73.66970496)(400.08976554,73.55470508)(400.09976562,73.43470998)
\curveto(400.10976553,73.31470532)(400.12976551,73.19470544)(400.15976562,73.07470998)
\curveto(400.17976545,73.02470561)(400.17976545,72.96970566)(400.15976562,72.90970998)
\curveto(400.14976548,72.84970578)(400.15476548,72.78970584)(400.17476563,72.72970998)
\curveto(400.19476544,72.629706)(400.19476544,72.5297061)(400.17476563,72.42970998)
\lineto(400.17476563,72.29470998)
\curveto(400.15476548,72.24470639)(400.14476549,72.18470645)(400.14476562,72.11470998)
\curveto(400.15476548,72.05470658)(400.14976548,71.99970663)(400.12976563,71.94970998)
\curveto(400.11976551,71.90970672)(400.11476552,71.87470676)(400.11476563,71.84470998)
\curveto(400.11476552,71.81470682)(400.10976553,71.77970685)(400.09976562,71.73970998)
\lineto(400.03976562,71.46970998)
\curveto(400.01976561,71.37970725)(399.98976564,71.29470734)(399.94976562,71.21470998)
\curveto(399.80976582,70.87470776)(399.65476598,70.58470805)(399.48476563,70.34470998)
\curveto(399.30476633,70.10470853)(399.07476656,69.88470875)(398.79476563,69.68470998)
\curveto(398.56476707,69.5347091)(398.32476731,69.41970921)(398.07476562,69.33970998)
\curveto(398.02476761,69.31970931)(397.97976765,69.30970932)(397.93976563,69.30970998)
\curveto(397.88976774,69.30970932)(397.8397678,69.29970933)(397.78976562,69.27970998)
\curveto(397.72976791,69.25970937)(397.64976798,69.24470939)(397.54976563,69.23470998)
\curveto(397.44976818,69.2347094)(397.37476826,69.25470938)(397.32476562,69.29470998)
\curveto(397.24476839,69.34470929)(397.19976844,69.42470921)(397.18976563,69.53470998)
\curveto(397.17976845,69.64470899)(397.17476846,69.75970887)(397.17476563,69.87970998)
\lineto(397.17476563,70.04470998)
\curveto(397.17476846,70.10470853)(397.18476845,70.15970847)(397.20476562,70.20970998)
\curveto(397.22476841,70.29970833)(397.26476837,70.36970826)(397.32476562,70.41970998)
\curveto(397.41476822,70.48970814)(397.52476811,70.5347081)(397.65476562,70.55470998)
\curveto(397.77476786,70.58470805)(397.87976775,70.629708)(397.96976562,70.68970998)
\curveto(398.30976732,70.87970775)(398.57976705,71.13970749)(398.77976562,71.46970998)
\curveto(398.8397668,71.56970706)(398.88976674,71.67470696)(398.92976563,71.78470998)
\curveto(398.95976668,71.90470673)(398.99476664,72.02470661)(399.03476562,72.14470998)
\curveto(399.08476655,72.31470632)(399.10476653,72.51970611)(399.09476562,72.75970998)
\curveto(399.07476656,73.00970562)(399.0397666,73.20970542)(398.98976563,73.35970998)
\curveto(398.86976677,73.7297049)(398.70976692,74.01970461)(398.50976562,74.22970998)
\curveto(398.29976734,74.44970418)(398.01976761,74.629704)(397.66976563,74.76970998)
\curveto(397.56976807,74.81970381)(397.46476817,74.84970378)(397.35476563,74.85970998)
\curveto(397.24476839,74.87970375)(397.12976851,74.90470373)(397.00976562,74.93470998)
\lineto(396.90476562,74.93470998)
\curveto(396.86476877,74.94470369)(396.82476881,74.94970368)(396.78476562,74.94970998)
\curveto(396.75476888,74.95970367)(396.71976891,74.95970367)(396.67976563,74.94970998)
\lineto(396.55976563,74.94970998)
\curveto(396.29976934,74.94970368)(396.05476958,74.91970371)(395.82476562,74.85970998)
\curveto(395.47477016,74.74970388)(395.17977045,74.59470404)(394.93976563,74.39470998)
\curveto(394.68977095,74.19470444)(394.49477114,73.9347047)(394.35476563,73.61470998)
\lineto(394.29476563,73.43470998)
\curveto(394.27477136,73.38470525)(394.25477138,73.32470531)(394.23476563,73.25470998)
\curveto(394.21477142,73.20470543)(394.20477143,73.14470549)(394.20476562,73.07470998)
\curveto(394.19477144,73.01470562)(394.17977145,72.94970568)(394.15976562,72.87970998)
\lineto(394.15976562,72.72970998)
\curveto(394.13977149,72.68970594)(394.12977151,72.634706)(394.12976563,72.56470998)
\curveto(394.12977151,72.50470613)(394.13977149,72.44970618)(394.15976562,72.39970998)
\lineto(394.15976562,72.29470998)
\curveto(394.15977148,72.26470637)(394.16477147,72.2297064)(394.17476563,72.18970998)
\lineto(394.23476563,71.94970998)
\curveto(394.24477139,71.86970676)(394.26477137,71.78970684)(394.29476563,71.70970998)
\curveto(394.39477124,71.46970716)(394.52977111,71.23970739)(394.69976562,71.01970998)
\curveto(394.76977086,70.9297077)(394.84477079,70.84470779)(394.92476563,70.76470998)
\curveto(394.99477064,70.68470795)(395.04977058,70.58470805)(395.08976562,70.46470998)
\curveto(395.11977052,70.37470826)(395.12977051,70.2347084)(395.11976563,70.04470998)
\curveto(395.10977052,69.86470877)(395.08477055,69.74470889)(395.04476563,69.68470998)
\curveto(395.00477063,69.634709)(394.94477069,69.59470904)(394.86476563,69.56470998)
\curveto(394.78477085,69.54470909)(394.69977094,69.54470909)(394.60976563,69.56470998)
\curveto(394.48977115,69.59470904)(394.36977126,69.61470902)(394.24976563,69.62470998)
\curveto(394.11977152,69.64470899)(393.99477164,69.66970896)(393.87476563,69.69970998)
\curveto(393.8347718,69.71970891)(393.79977184,69.72470891)(393.76976562,69.71470998)
\curveto(393.72977191,69.71470892)(393.68477195,69.72470891)(393.63476562,69.74470998)
\curveto(393.54477209,69.76470887)(393.45477218,69.77970885)(393.36476563,69.78970998)
\curveto(393.26477237,69.79970883)(393.16977246,69.81970881)(393.07976562,69.84970998)
\curveto(393.01977262,69.85970877)(392.95977268,69.86470877)(392.89976562,69.86470998)
\curveto(392.83977279,69.87470876)(392.77977285,69.88970874)(392.71976562,69.90970998)
\curveto(392.51977312,69.95970867)(392.31477332,69.99470864)(392.10476563,70.01470998)
\curveto(391.88477375,70.04470859)(391.67477396,70.08470855)(391.47476562,70.13470998)
\curveto(391.37477426,70.16470847)(391.27477436,70.18470845)(391.17476563,70.19470998)
\curveto(391.07477456,70.20470843)(390.97477466,70.21970841)(390.87476563,70.23970998)
\curveto(390.84477479,70.24970838)(390.80477483,70.25470838)(390.75476563,70.25470998)
\curveto(390.64477499,70.28470835)(390.53977509,70.30470833)(390.43976563,70.31470998)
\curveto(390.3297753,70.3347083)(390.21977542,70.35970827)(390.10976563,70.38970998)
\curveto(390.0297756,70.40970822)(389.95977568,70.42470821)(389.89976562,70.43470998)
\curveto(389.8297758,70.44470819)(389.76977586,70.46970816)(389.71976562,70.50970998)
\curveto(389.68977595,70.5297081)(389.66977596,70.55970807)(389.65976562,70.59970998)
\curveto(389.63977599,70.63970799)(389.61977602,70.68470795)(389.59976562,70.73470998)
\curveto(389.59977603,70.79470784)(389.59477604,70.8347078)(389.58476562,70.85470998)
}
}
{
\newrgbcolor{curcolor}{0 0 0}
\pscustom[linestyle=none,fillstyle=solid,fillcolor=curcolor]
{
\newpath
\moveto(398.35976563,78.63431936)
\lineto(398.35976563,79.26431936)
\lineto(398.35976563,79.45931936)
\curveto(398.35976728,79.52931683)(398.36976727,79.58931677)(398.38976562,79.63931936)
\curveto(398.42976721,79.70931665)(398.46976717,79.7593166)(398.50976562,79.78931936)
\curveto(398.55976708,79.82931653)(398.62476701,79.84931651)(398.70476562,79.84931936)
\curveto(398.78476685,79.8593165)(398.86976677,79.86431649)(398.95976562,79.86431936)
\lineto(399.67976563,79.86431936)
\curveto(400.15976547,79.86431649)(400.56976507,79.80431655)(400.90976562,79.68431936)
\curveto(401.24976438,79.56431679)(401.52476411,79.36931699)(401.73476563,79.09931936)
\curveto(401.78476385,79.02931733)(401.82976381,78.9593174)(401.86976563,78.88931936)
\curveto(401.91976371,78.82931753)(401.96476367,78.7543176)(402.00476563,78.66431936)
\curveto(402.01476362,78.64431771)(402.02476361,78.61431774)(402.03476562,78.57431936)
\curveto(402.05476358,78.53431782)(402.05976357,78.48931787)(402.04976563,78.43931936)
\curveto(402.01976361,78.34931801)(401.94476369,78.29431806)(401.82476562,78.27431936)
\curveto(401.71476392,78.2543181)(401.61976401,78.26931809)(401.53976562,78.31931936)
\curveto(401.46976417,78.34931801)(401.40476423,78.39431796)(401.34476562,78.45431936)
\curveto(401.29476434,78.52431783)(401.24476439,78.58931777)(401.19476563,78.64931936)
\curveto(401.14476449,78.71931764)(401.06976457,78.77931758)(400.96976562,78.82931936)
\curveto(400.87976475,78.88931747)(400.78976484,78.93931742)(400.69976562,78.97931936)
\curveto(400.66976497,78.99931736)(400.60976503,79.02431733)(400.51976562,79.05431936)
\curveto(400.4397652,79.08431727)(400.36976527,79.08931727)(400.30976563,79.06931936)
\curveto(400.16976547,79.03931732)(400.07976555,78.97931738)(400.03976562,78.88931936)
\curveto(400.00976562,78.80931755)(399.99476564,78.71931764)(399.99476563,78.61931936)
\curveto(399.99476564,78.51931784)(399.96976567,78.43431792)(399.91976563,78.36431936)
\curveto(399.84976578,78.27431808)(399.70976592,78.22931813)(399.49976563,78.22931936)
\lineto(398.94476563,78.22931936)
\lineto(398.71976562,78.22931936)
\curveto(398.639767,78.23931812)(398.57476706,78.2593181)(398.52476562,78.28931936)
\curveto(398.44476719,78.34931801)(398.39976724,78.41931794)(398.38976562,78.49931936)
\curveto(398.37976725,78.51931784)(398.37476726,78.53931782)(398.37476563,78.55931936)
\curveto(398.37476726,78.58931777)(398.36976727,78.61431774)(398.35976563,78.63431936)
}
}
{
\newrgbcolor{curcolor}{0 0 0}
\pscustom[linestyle=none,fillstyle=solid,fillcolor=curcolor]
{
}
}
{
\newrgbcolor{curcolor}{0 0 0}
\pscustom[linestyle=none,fillstyle=solid,fillcolor=curcolor]
{
\newpath
\moveto(389.38976562,89.26463186)
\curveto(389.37977626,89.95462722)(389.49977613,90.55462662)(389.74976563,91.06463186)
\curveto(389.99977563,91.58462559)(390.3347753,91.9796252)(390.75476563,92.24963186)
\curveto(390.8347748,92.29962488)(390.92477471,92.34462483)(391.02476562,92.38463186)
\curveto(391.11477452,92.42462475)(391.20977442,92.46962471)(391.30976563,92.51963186)
\curveto(391.40977422,92.55962462)(391.50977412,92.58962459)(391.60976563,92.60963186)
\curveto(391.70977392,92.62962455)(391.81477382,92.64962453)(391.92476563,92.66963186)
\curveto(391.97477366,92.68962449)(392.01977362,92.69462448)(392.05976563,92.68463186)
\curveto(392.09977353,92.6746245)(392.14477349,92.6796245)(392.19476563,92.69963186)
\curveto(392.24477339,92.70962447)(392.32977331,92.71462446)(392.44976562,92.71463186)
\curveto(392.55977308,92.71462446)(392.64477299,92.70962447)(392.70476562,92.69963186)
\curveto(392.76477287,92.6796245)(392.82477281,92.66962451)(392.88476562,92.66963186)
\curveto(392.94477269,92.6796245)(393.00477263,92.6746245)(393.06476563,92.65463186)
\curveto(393.20477243,92.61462456)(393.33977229,92.5796246)(393.46976562,92.54963186)
\curveto(393.59977204,92.51962466)(393.72477191,92.4796247)(393.84476562,92.42963186)
\curveto(393.98477165,92.36962481)(394.10977152,92.29962488)(394.21976562,92.21963186)
\curveto(394.32977131,92.14962503)(394.43977119,92.0746251)(394.54976563,91.99463186)
\lineto(394.60976563,91.93463186)
\curveto(394.62977101,91.92462525)(394.64977098,91.90962527)(394.66976563,91.88963186)
\curveto(394.82977081,91.76962541)(394.97477066,91.63462554)(395.10476563,91.48463186)
\curveto(395.2347704,91.33462584)(395.35977028,91.174626)(395.47976563,91.00463186)
\curveto(395.69976994,90.69462648)(395.90476973,90.39962678)(396.09476562,90.11963186)
\curveto(396.2347694,89.88962729)(396.36976927,89.65962752)(396.49976563,89.42963186)
\curveto(396.62976901,89.20962797)(396.76476887,88.98962819)(396.90476562,88.76963186)
\curveto(397.07476856,88.51962866)(397.25476838,88.2796289)(397.44476563,88.04963186)
\curveto(397.634768,87.82962935)(397.85976778,87.63962954)(398.11976563,87.47963186)
\curveto(398.17976745,87.43962974)(398.2397674,87.40462977)(398.29976563,87.37463186)
\curveto(398.34976728,87.34462983)(398.41476722,87.31462986)(398.49476563,87.28463186)
\curveto(398.56476707,87.26462991)(398.62476701,87.25962992)(398.67476563,87.26963186)
\curveto(398.74476689,87.28962989)(398.79976684,87.32462985)(398.83976562,87.37463186)
\curveto(398.86976677,87.42462975)(398.88976674,87.48462969)(398.89976562,87.55463186)
\lineto(398.89976562,87.79463186)
\lineto(398.89976562,88.54463186)
\lineto(398.89976562,91.34963186)
\lineto(398.89976562,92.00963186)
\curveto(398.89976674,92.09962508)(398.90476673,92.18462499)(398.91476562,92.26463186)
\curveto(398.91476672,92.34462483)(398.9347667,92.40962477)(398.97476562,92.45963186)
\curveto(399.01476662,92.50962467)(399.08976654,92.54962463)(399.19976562,92.57963186)
\curveto(399.29976634,92.61962456)(399.39976624,92.62962455)(399.49976563,92.60963186)
\lineto(399.63476562,92.60963186)
\curveto(399.70476593,92.58962459)(399.76476587,92.56962461)(399.81476563,92.54963186)
\curveto(399.86476577,92.52962465)(399.90476573,92.49462468)(399.93476563,92.44463186)
\curveto(399.97476566,92.39462478)(399.99476564,92.32462485)(399.99476563,92.23463186)
\lineto(399.99476563,91.96463186)
\lineto(399.99476563,91.06463186)
\lineto(399.99476563,87.55463186)
\lineto(399.99476563,86.48963186)
\curveto(399.99476564,86.40963077)(399.99976564,86.31963086)(400.00976562,86.21963186)
\curveto(400.00976562,86.11963106)(399.99976564,86.03463114)(399.97976563,85.96463186)
\curveto(399.90976572,85.75463142)(399.72976591,85.68963149)(399.43976563,85.76963186)
\curveto(399.39976624,85.7796314)(399.36476627,85.7796314)(399.33476562,85.76963186)
\curveto(399.29476634,85.76963141)(399.24976638,85.7796314)(399.19976562,85.79963186)
\curveto(399.11976651,85.81963136)(399.0347666,85.83963134)(398.94476563,85.85963186)
\curveto(398.85476678,85.8796313)(398.76976687,85.90463127)(398.68976563,85.93463186)
\curveto(398.19976744,86.09463108)(397.78476785,86.29463088)(397.44476563,86.53463186)
\curveto(397.19476844,86.71463046)(396.96976867,86.91963026)(396.76976562,87.14963186)
\curveto(396.55976908,87.3796298)(396.36476927,87.61962956)(396.18476563,87.86963186)
\curveto(396.00476963,88.12962905)(395.8347698,88.39462878)(395.67476563,88.66463186)
\curveto(395.50477013,88.94462823)(395.32977031,89.21462796)(395.14976562,89.47463186)
\curveto(395.06977056,89.58462759)(394.99477064,89.68962749)(394.92476563,89.78963186)
\curveto(394.85477078,89.89962728)(394.77977085,90.00962717)(394.69976562,90.11963186)
\curveto(394.66977096,90.15962702)(394.63977099,90.19462698)(394.60976563,90.22463186)
\curveto(394.56977106,90.26462691)(394.53977109,90.30462687)(394.51976562,90.34463186)
\curveto(394.40977122,90.48462669)(394.28477135,90.60962657)(394.14476562,90.71963186)
\curveto(394.11477152,90.73962644)(394.08977155,90.76462641)(394.06976563,90.79463186)
\curveto(394.03977159,90.82462635)(394.00977162,90.84962633)(393.97976563,90.86963186)
\curveto(393.87977175,90.94962623)(393.77977185,91.01462616)(393.67976563,91.06463186)
\curveto(393.57977205,91.12462605)(393.46977216,91.179626)(393.34976562,91.22963186)
\curveto(393.27977235,91.25962592)(393.20477243,91.2796259)(393.12476563,91.28963186)
\lineto(392.88476562,91.34963186)
\lineto(392.79476563,91.34963186)
\curveto(392.76477287,91.35962582)(392.7347729,91.36462581)(392.70476562,91.36463186)
\curveto(392.634773,91.38462579)(392.53977309,91.38962579)(392.41976563,91.37963186)
\curveto(392.28977335,91.3796258)(392.18977345,91.36962581)(392.11976563,91.34963186)
\curveto(392.03977359,91.32962585)(391.96477367,91.30962587)(391.89476562,91.28963186)
\curveto(391.81477382,91.2796259)(391.7347739,91.25962592)(391.65476562,91.22963186)
\curveto(391.41477422,91.11962606)(391.21477442,90.96962621)(391.05476563,90.77963186)
\curveto(390.88477475,90.59962658)(390.74477489,90.3796268)(390.63476562,90.11963186)
\curveto(390.61477502,90.04962713)(390.59977503,89.9796272)(390.58976562,89.90963186)
\curveto(390.56977506,89.83962734)(390.54977509,89.76462741)(390.52976562,89.68463186)
\curveto(390.50977512,89.60462757)(390.49977513,89.49462768)(390.49976563,89.35463186)
\curveto(390.49977513,89.22462795)(390.50977512,89.11962806)(390.52976562,89.03963186)
\curveto(390.53977509,88.9796282)(390.54477509,88.92462825)(390.54476563,88.87463186)
\curveto(390.54477509,88.82462835)(390.55477508,88.7746284)(390.57476562,88.72463186)
\curveto(390.61477502,88.62462855)(390.65477498,88.52962865)(390.69476563,88.43963186)
\curveto(390.7347749,88.35962882)(390.77977486,88.2796289)(390.82976562,88.19963186)
\curveto(390.84977479,88.16962901)(390.87477476,88.13962904)(390.90476562,88.10963186)
\curveto(390.9347747,88.08962909)(390.95977468,88.06462911)(390.97976563,88.03463186)
\lineto(391.05476563,87.95963186)
\curveto(391.07477456,87.92962925)(391.09477454,87.90462927)(391.11476563,87.88463186)
\lineto(391.32476562,87.73463186)
\curveto(391.38477425,87.69462948)(391.44977419,87.64962953)(391.51976562,87.59963186)
\curveto(391.60977402,87.53962964)(391.71477392,87.48962969)(391.83476562,87.44963186)
\curveto(391.94477369,87.41962976)(392.05477358,87.38462979)(392.16476562,87.34463186)
\curveto(392.27477336,87.30462987)(392.41977322,87.2796299)(392.59976562,87.26963186)
\curveto(392.76977286,87.25962992)(392.89477274,87.22962995)(392.97476562,87.17963186)
\curveto(393.05477258,87.12963005)(393.09977253,87.05463012)(393.10976563,86.95463186)
\curveto(393.11977252,86.85463032)(393.12477251,86.74463043)(393.12476563,86.62463186)
\curveto(393.12477251,86.58463059)(393.12977251,86.54463063)(393.13976562,86.50463186)
\curveto(393.13977249,86.46463071)(393.1347725,86.42963075)(393.12476563,86.39963186)
\curveto(393.10477253,86.34963083)(393.09477254,86.29963088)(393.09476562,86.24963186)
\curveto(393.09477254,86.20963097)(393.08477255,86.16963101)(393.06476563,86.12963186)
\curveto(393.00477263,86.03963114)(392.86977276,85.99463118)(392.65976562,85.99463186)
\lineto(392.53976562,85.99463186)
\curveto(392.47977315,86.00463117)(392.41977322,86.00963117)(392.35976563,86.00963186)
\curveto(392.28977335,86.01963116)(392.22477341,86.02963115)(392.16476562,86.03963186)
\curveto(392.05477358,86.05963112)(391.95477368,86.0796311)(391.86476563,86.09963186)
\curveto(391.76477387,86.11963106)(391.66977396,86.14963103)(391.57976562,86.18963186)
\curveto(391.50977412,86.20963097)(391.44977419,86.22963095)(391.39976562,86.24963186)
\lineto(391.21976562,86.30963186)
\curveto(390.95977468,86.42963075)(390.71477492,86.58463059)(390.48476563,86.77463186)
\curveto(390.25477538,86.9746302)(390.06977556,87.18962999)(389.92976563,87.41963186)
\curveto(389.84977579,87.52962965)(389.78477585,87.64462953)(389.73476563,87.76463186)
\lineto(389.58476562,88.15463186)
\curveto(389.5347761,88.26462891)(389.50477613,88.3796288)(389.49476563,88.49963186)
\curveto(389.47477616,88.61962856)(389.44977619,88.74462843)(389.41976563,88.87463186)
\curveto(389.41977622,88.94462823)(389.41977622,89.00962817)(389.41976563,89.06963186)
\curveto(389.40977622,89.12962805)(389.39977623,89.19462798)(389.38976562,89.26463186)
}
}
{
\newrgbcolor{curcolor}{0 0 0}
\pscustom[linestyle=none,fillstyle=solid,fillcolor=curcolor]
{
\newpath
\moveto(394.90976562,101.36424123)
\lineto(395.16476562,101.36424123)
\curveto(395.24477039,101.37423353)(395.31977032,101.36923353)(395.38976562,101.34924123)
\lineto(395.62976563,101.34924123)
\lineto(395.79476563,101.34924123)
\curveto(395.89476974,101.32923357)(395.99976964,101.31923358)(396.10976563,101.31924123)
\curveto(396.20976942,101.31923358)(396.30976932,101.30923359)(396.40976562,101.28924123)
\lineto(396.55976563,101.28924123)
\curveto(396.69976894,101.25923364)(396.83976879,101.23923366)(396.97976563,101.22924123)
\curveto(397.10976852,101.21923368)(397.23976839,101.19423371)(397.36976563,101.15424123)
\curveto(397.44976818,101.13423377)(397.5347681,101.11423379)(397.62476563,101.09424123)
\lineto(397.86476563,101.03424123)
\lineto(398.16476562,100.91424123)
\curveto(398.25476738,100.88423402)(398.34476729,100.84923405)(398.43476563,100.80924123)
\curveto(398.65476698,100.70923419)(398.86976677,100.57423433)(399.07976562,100.40424123)
\curveto(399.28976634,100.24423466)(399.45976618,100.06923483)(399.58976562,99.87924123)
\curveto(399.62976601,99.82923507)(399.66976597,99.76923513)(399.70976562,99.69924123)
\curveto(399.7397659,99.63923526)(399.77476586,99.57923532)(399.81476563,99.51924123)
\curveto(399.86476577,99.43923546)(399.90476573,99.34423556)(399.93476563,99.23424123)
\curveto(399.96476567,99.12423578)(399.99476564,99.01923588)(400.02476562,98.91924123)
\curveto(400.06476557,98.80923609)(400.08976554,98.6992362)(400.09976562,98.58924123)
\curveto(400.10976553,98.47923642)(400.12476551,98.36423654)(400.14476562,98.24424123)
\curveto(400.15476548,98.2042367)(400.15476548,98.15923674)(400.14476562,98.10924123)
\curveto(400.14476549,98.06923683)(400.14976548,98.02923687)(400.15976562,97.98924123)
\curveto(400.16976547,97.94923695)(400.17476546,97.89423701)(400.17476563,97.82424123)
\curveto(400.17476546,97.75423715)(400.16976547,97.7042372)(400.15976562,97.67424123)
\curveto(400.1397655,97.62423728)(400.1347655,97.57923732)(400.14476562,97.53924123)
\curveto(400.15476548,97.4992374)(400.15476548,97.46423744)(400.14476562,97.43424123)
\lineto(400.14476562,97.34424123)
\curveto(400.12476551,97.28423762)(400.10976553,97.21923768)(400.09976562,97.14924123)
\curveto(400.09976554,97.08923781)(400.09476554,97.02423788)(400.08476562,96.95424123)
\curveto(400.0347656,96.78423812)(399.98476565,96.62423828)(399.93476563,96.47424123)
\curveto(399.88476575,96.32423858)(399.81976581,96.17923872)(399.73976563,96.03924123)
\curveto(399.69976594,95.98923891)(399.66976597,95.93423897)(399.64976562,95.87424123)
\curveto(399.61976601,95.82423908)(399.58476605,95.77423913)(399.54476563,95.72424123)
\curveto(399.36476627,95.48423942)(399.14476649,95.28423962)(398.88476562,95.12424123)
\curveto(398.62476701,94.96423994)(398.3397673,94.82424008)(398.02976562,94.70424123)
\curveto(397.88976774,94.64424026)(397.74976788,94.5992403)(397.60976563,94.56924123)
\curveto(397.45976818,94.53924036)(397.30476833,94.5042404)(397.14476562,94.46424123)
\curveto(397.0347686,94.44424046)(396.92476871,94.42924047)(396.81476563,94.41924123)
\curveto(396.70476893,94.40924049)(396.59476904,94.39424051)(396.48476563,94.37424123)
\curveto(396.44476919,94.36424054)(396.40476923,94.35924054)(396.36476563,94.35924123)
\curveto(396.32476931,94.36924053)(396.28476935,94.36924053)(396.24476563,94.35924123)
\curveto(396.19476944,94.34924055)(396.14476949,94.34424056)(396.09476562,94.34424123)
\lineto(395.92976563,94.34424123)
\curveto(395.87976975,94.32424058)(395.82976981,94.31924058)(395.77976562,94.32924123)
\curveto(395.71976991,94.33924056)(395.66476997,94.33924056)(395.61476563,94.32924123)
\curveto(395.57477006,94.31924058)(395.52977011,94.31924058)(395.47976563,94.32924123)
\curveto(395.42977021,94.33924056)(395.37977025,94.33424057)(395.32976562,94.31424123)
\curveto(395.25977038,94.29424061)(395.18477045,94.28924061)(395.10476563,94.29924123)
\curveto(395.01477062,94.30924059)(394.92977071,94.31424059)(394.84976562,94.31424123)
\curveto(394.75977088,94.31424059)(394.65977098,94.30924059)(394.54976563,94.29924123)
\curveto(394.42977121,94.28924061)(394.32977131,94.29424061)(394.24976563,94.31424123)
\lineto(393.96476562,94.31424123)
\lineto(393.33476562,94.35924123)
\curveto(393.2347724,94.36924053)(393.13977249,94.37924052)(393.04976563,94.38924123)
\lineto(392.74976563,94.41924123)
\curveto(392.69977293,94.43924046)(392.64977299,94.44424046)(392.59976562,94.43424123)
\curveto(392.53977309,94.43424047)(392.48477315,94.44424046)(392.43476563,94.46424123)
\curveto(392.26477337,94.51424039)(392.09977353,94.55424035)(391.93976563,94.58424123)
\curveto(391.76977386,94.61424029)(391.60977402,94.66424024)(391.45976562,94.73424123)
\curveto(390.99977463,94.92423998)(390.62477501,95.14423976)(390.33476562,95.39424123)
\curveto(390.04477559,95.65423925)(389.79977583,96.01423889)(389.59976562,96.47424123)
\curveto(389.54977609,96.6042383)(389.51477612,96.73423817)(389.49476563,96.86424123)
\curveto(389.47477616,97.0042379)(389.44977619,97.14423776)(389.41976563,97.28424123)
\curveto(389.40977622,97.35423755)(389.40477623,97.41923748)(389.40476562,97.47924123)
\curveto(389.40477623,97.53923736)(389.39977623,97.6042373)(389.38976562,97.67424123)
\curveto(389.36977626,98.5042364)(389.51977612,99.17423573)(389.83976562,99.68424123)
\curveto(390.14977549,100.19423471)(390.58977505,100.57423433)(391.15976562,100.82424123)
\curveto(391.27977435,100.87423403)(391.40477423,100.91923398)(391.53476562,100.95924123)
\curveto(391.66477397,100.9992339)(391.79977383,101.04423386)(391.93976563,101.09424123)
\curveto(392.01977362,101.11423379)(392.10477353,101.12923377)(392.19476563,101.13924123)
\lineto(392.43476563,101.19924123)
\curveto(392.54477309,101.22923367)(392.65477298,101.24423366)(392.76476562,101.24424123)
\curveto(392.87477276,101.25423365)(392.98477265,101.26923363)(393.09476562,101.28924123)
\curveto(393.14477249,101.30923359)(393.18977245,101.31423359)(393.22976563,101.30424123)
\curveto(393.26977236,101.3042336)(393.30977232,101.30923359)(393.34976562,101.31924123)
\curveto(393.39977224,101.32923357)(393.45477218,101.32923357)(393.51476562,101.31924123)
\curveto(393.56477207,101.31923358)(393.61477202,101.32423358)(393.66476562,101.33424123)
\lineto(393.79976563,101.33424123)
\curveto(393.85977178,101.35423355)(393.92977171,101.35423355)(394.00976562,101.33424123)
\curveto(394.07977155,101.32423358)(394.14477149,101.32923357)(394.20476562,101.34924123)
\curveto(394.2347714,101.35923354)(394.27477136,101.36423354)(394.32476562,101.36424123)
\lineto(394.44476563,101.36424123)
\lineto(394.90976562,101.36424123)
\moveto(397.23476563,99.81924123)
\curveto(396.91476872,99.91923498)(396.54976908,99.97923492)(396.13976562,99.99924123)
\curveto(395.72976991,100.01923488)(395.31977032,100.02923487)(394.90976562,100.02924123)
\curveto(394.47977115,100.02923487)(394.05977158,100.01923488)(393.64976562,99.99924123)
\curveto(393.23977239,99.97923492)(392.85477278,99.93423497)(392.49476563,99.86424123)
\curveto(392.1347735,99.79423511)(391.81477382,99.68423522)(391.53476562,99.53424123)
\curveto(391.24477439,99.39423551)(391.00977462,99.1992357)(390.82976562,98.94924123)
\curveto(390.71977492,98.78923611)(390.63977499,98.60923629)(390.58976562,98.40924123)
\curveto(390.5297751,98.20923669)(390.49977513,97.96423694)(390.49976563,97.67424123)
\curveto(390.51977512,97.65423725)(390.5297751,97.61923728)(390.52976562,97.56924123)
\curveto(390.51977512,97.51923738)(390.51977512,97.47923742)(390.52976562,97.44924123)
\curveto(390.54977509,97.36923753)(390.56977506,97.29423761)(390.58976562,97.22424123)
\curveto(390.59977503,97.16423774)(390.61977502,97.0992378)(390.64976562,97.02924123)
\curveto(390.76977486,96.75923814)(390.93977469,96.53923836)(391.15976562,96.36924123)
\curveto(391.36977426,96.20923869)(391.61477402,96.07423883)(391.89476562,95.96424123)
\curveto(392.00477363,95.91423899)(392.12477351,95.87423903)(392.25476563,95.84424123)
\curveto(392.37477326,95.82423908)(392.49977313,95.7992391)(392.62976563,95.76924123)
\curveto(392.67977295,95.74923915)(392.7347729,95.73923916)(392.79476563,95.73924123)
\curveto(392.84477279,95.73923916)(392.89477274,95.73423917)(392.94476563,95.72424123)
\curveto(393.0347726,95.71423919)(393.12977251,95.7042392)(393.22976563,95.69424123)
\curveto(393.31977232,95.68423922)(393.41477222,95.67423923)(393.51476562,95.66424123)
\curveto(393.59477204,95.66423924)(393.67977195,95.65923924)(393.76976562,95.64924123)
\lineto(394.00976562,95.64924123)
\lineto(394.18976563,95.64924123)
\curveto(394.21977142,95.63923926)(394.25477138,95.63423927)(394.29476563,95.63424123)
\lineto(394.42976563,95.63424123)
\lineto(394.87976563,95.63424123)
\curveto(394.95977068,95.63423927)(395.04477059,95.62923927)(395.13476562,95.61924123)
\curveto(395.21477042,95.61923928)(395.28977035,95.62923927)(395.35976563,95.64924123)
\lineto(395.62976563,95.64924123)
\curveto(395.64976998,95.64923925)(395.67976995,95.64423926)(395.71976562,95.63424123)
\curveto(395.74976988,95.63423927)(395.77476986,95.63923926)(395.79476563,95.64924123)
\curveto(395.89476974,95.65923924)(395.99476964,95.66423924)(396.09476562,95.66424123)
\curveto(396.18476945,95.67423923)(396.28476935,95.68423922)(396.39476562,95.69424123)
\curveto(396.51476912,95.72423918)(396.63976899,95.73923916)(396.76976562,95.73924123)
\curveto(396.88976875,95.74923915)(397.00476863,95.77423913)(397.11476563,95.81424123)
\curveto(397.41476822,95.89423901)(397.67976795,95.97923892)(397.90976562,96.06924123)
\curveto(398.1397675,96.16923873)(398.35476728,96.31423859)(398.55476563,96.50424123)
\curveto(398.75476688,96.71423819)(398.90476673,96.97923792)(399.00476563,97.29924123)
\curveto(399.02476661,97.33923756)(399.0347666,97.37423753)(399.03476562,97.40424123)
\curveto(399.02476661,97.44423746)(399.02976661,97.48923741)(399.04976563,97.53924123)
\curveto(399.05976658,97.57923732)(399.06976657,97.64923725)(399.07976562,97.74924123)
\curveto(399.08976654,97.85923704)(399.08476655,97.94423696)(399.06476563,98.00424123)
\curveto(399.04476659,98.07423683)(399.0347666,98.14423676)(399.03476562,98.21424123)
\curveto(399.02476661,98.28423662)(399.00976662,98.34923655)(398.98976563,98.40924123)
\curveto(398.92976671,98.60923629)(398.84476679,98.78923611)(398.73476563,98.94924123)
\curveto(398.71476692,98.97923592)(398.69476694,99.0042359)(398.67476563,99.02424123)
\lineto(398.61476563,99.08424123)
\curveto(398.59476704,99.12423578)(398.55476708,99.17423573)(398.49476563,99.23424123)
\curveto(398.35476728,99.33423557)(398.22476741,99.41923548)(398.10476563,99.48924123)
\curveto(397.98476765,99.55923534)(397.8397678,99.62923527)(397.66976563,99.69924123)
\curveto(397.59976804,99.72923517)(397.52976811,99.74923515)(397.45976562,99.75924123)
\curveto(397.38976825,99.77923512)(397.31476832,99.7992351)(397.23476563,99.81924123)
}
}
{
\newrgbcolor{curcolor}{0 0 0}
\pscustom[linestyle=none,fillstyle=solid,fillcolor=curcolor]
{
\newpath
\moveto(389.38976562,106.77385061)
\curveto(389.38977625,106.87384575)(389.39977623,106.96884566)(389.41976563,107.05885061)
\curveto(389.4297762,107.14884548)(389.45977618,107.21384541)(389.50976562,107.25385061)
\curveto(389.58977605,107.31384531)(389.69477594,107.34384528)(389.82476562,107.34385061)
\lineto(390.21476562,107.34385061)
\lineto(391.71476562,107.34385061)
\lineto(398.10476563,107.34385061)
\lineto(399.27476562,107.34385061)
\lineto(399.58976562,107.34385061)
\curveto(399.68976594,107.35384527)(399.76976587,107.33884529)(399.82976562,107.29885061)
\curveto(399.90976572,107.24884538)(399.95976568,107.17384545)(399.97976563,107.07385061)
\curveto(399.98976564,106.98384564)(399.99476564,106.87384575)(399.99476563,106.74385061)
\lineto(399.99476563,106.51885061)
\curveto(399.97476566,106.43884619)(399.95976568,106.36884626)(399.94976562,106.30885061)
\curveto(399.92976571,106.24884638)(399.88976574,106.19884643)(399.82976562,106.15885061)
\curveto(399.76976587,106.11884651)(399.69476594,106.09884653)(399.60476563,106.09885061)
\lineto(399.30476563,106.09885061)
\lineto(398.20976562,106.09885061)
\lineto(392.86976563,106.09885061)
\curveto(392.77977285,106.07884655)(392.70477293,106.06384656)(392.64476562,106.05385061)
\curveto(392.57477306,106.05384657)(392.51477312,106.0238466)(392.46476562,105.96385061)
\curveto(392.41477322,105.89384673)(392.38977325,105.80384682)(392.38976562,105.69385061)
\curveto(392.37977325,105.59384703)(392.37477326,105.48384714)(392.37476563,105.36385061)
\lineto(392.37476563,104.22385061)
\lineto(392.37476563,103.72885061)
\curveto(392.36477327,103.56884906)(392.30477333,103.45884917)(392.19476563,103.39885061)
\curveto(392.16477347,103.37884925)(392.1347735,103.36884926)(392.10476563,103.36885061)
\curveto(392.06477357,103.36884926)(392.01977362,103.36384926)(391.96976562,103.35385061)
\curveto(391.84977379,103.33384929)(391.73977389,103.33884929)(391.63976562,103.36885061)
\curveto(391.53977409,103.40884922)(391.46977416,103.46384916)(391.42976563,103.53385061)
\curveto(391.37977426,103.61384901)(391.35477428,103.73384889)(391.35476563,103.89385061)
\curveto(391.35477428,104.05384857)(391.33977429,104.18884844)(391.30976563,104.29885061)
\curveto(391.29977433,104.34884828)(391.29477434,104.40384822)(391.29476563,104.46385061)
\curveto(391.28477435,104.5238481)(391.26977436,104.58384804)(391.24976563,104.64385061)
\curveto(391.19977443,104.79384783)(391.14977449,104.93884769)(391.09976562,105.07885061)
\curveto(391.03977459,105.21884741)(390.96977466,105.35384727)(390.88976562,105.48385061)
\curveto(390.79977483,105.623847)(390.69477494,105.74384688)(390.57476562,105.84385061)
\curveto(390.45477518,105.94384668)(390.32477531,106.03884659)(390.18476563,106.12885061)
\curveto(390.08477555,106.18884644)(389.97477566,106.23384639)(389.85476563,106.26385061)
\curveto(389.7347759,106.30384632)(389.629776,106.35384627)(389.53976562,106.41385061)
\curveto(389.47977616,106.46384616)(389.43977619,106.53384609)(389.41976563,106.62385061)
\curveto(389.40977622,106.64384598)(389.40477623,106.66884596)(389.40476562,106.69885061)
\curveto(389.40477623,106.7288459)(389.39977623,106.75384587)(389.38976562,106.77385061)
}
}
{
\newrgbcolor{curcolor}{0 0 0}
\pscustom[linestyle=none,fillstyle=solid,fillcolor=curcolor]
{
\newpath
\moveto(389.38976562,115.12345998)
\curveto(389.38977625,115.22345513)(389.39977623,115.31845503)(389.41976563,115.40845998)
\curveto(389.4297762,115.49845485)(389.45977618,115.56345479)(389.50976562,115.60345998)
\curveto(389.58977605,115.66345469)(389.69477594,115.69345466)(389.82476562,115.69345998)
\lineto(390.21476562,115.69345998)
\lineto(391.71476562,115.69345998)
\lineto(398.10476563,115.69345998)
\lineto(399.27476562,115.69345998)
\lineto(399.58976562,115.69345998)
\curveto(399.68976594,115.70345465)(399.76976587,115.68845466)(399.82976562,115.64845998)
\curveto(399.90976572,115.59845475)(399.95976568,115.52345483)(399.97976563,115.42345998)
\curveto(399.98976564,115.33345502)(399.99476564,115.22345513)(399.99476563,115.09345998)
\lineto(399.99476563,114.86845998)
\curveto(399.97476566,114.78845556)(399.95976568,114.71845563)(399.94976562,114.65845998)
\curveto(399.92976571,114.59845575)(399.88976574,114.5484558)(399.82976562,114.50845998)
\curveto(399.76976587,114.46845588)(399.69476594,114.4484559)(399.60476563,114.44845998)
\lineto(399.30476563,114.44845998)
\lineto(398.20976562,114.44845998)
\lineto(392.86976563,114.44845998)
\curveto(392.77977285,114.42845592)(392.70477293,114.41345594)(392.64476562,114.40345998)
\curveto(392.57477306,114.40345595)(392.51477312,114.37345598)(392.46476562,114.31345998)
\curveto(392.41477322,114.24345611)(392.38977325,114.1534562)(392.38976562,114.04345998)
\curveto(392.37977325,113.94345641)(392.37477326,113.83345652)(392.37476563,113.71345998)
\lineto(392.37476563,112.57345998)
\lineto(392.37476563,112.07845998)
\curveto(392.36477327,111.91845843)(392.30477333,111.80845854)(392.19476563,111.74845998)
\curveto(392.16477347,111.72845862)(392.1347735,111.71845863)(392.10476563,111.71845998)
\curveto(392.06477357,111.71845863)(392.01977362,111.71345864)(391.96976562,111.70345998)
\curveto(391.84977379,111.68345867)(391.73977389,111.68845866)(391.63976562,111.71845998)
\curveto(391.53977409,111.75845859)(391.46977416,111.81345854)(391.42976563,111.88345998)
\curveto(391.37977426,111.96345839)(391.35477428,112.08345827)(391.35476563,112.24345998)
\curveto(391.35477428,112.40345795)(391.33977429,112.53845781)(391.30976563,112.64845998)
\curveto(391.29977433,112.69845765)(391.29477434,112.7534576)(391.29476563,112.81345998)
\curveto(391.28477435,112.87345748)(391.26977436,112.93345742)(391.24976563,112.99345998)
\curveto(391.19977443,113.14345721)(391.14977449,113.28845706)(391.09976562,113.42845998)
\curveto(391.03977459,113.56845678)(390.96977466,113.70345665)(390.88976562,113.83345998)
\curveto(390.79977483,113.97345638)(390.69477494,114.09345626)(390.57476562,114.19345998)
\curveto(390.45477518,114.29345606)(390.32477531,114.38845596)(390.18476563,114.47845998)
\curveto(390.08477555,114.53845581)(389.97477566,114.58345577)(389.85476563,114.61345998)
\curveto(389.7347759,114.6534557)(389.629776,114.70345565)(389.53976562,114.76345998)
\curveto(389.47977616,114.81345554)(389.43977619,114.88345547)(389.41976563,114.97345998)
\curveto(389.40977622,114.99345536)(389.40477623,115.01845533)(389.40476562,115.04845998)
\curveto(389.40477623,115.07845527)(389.39977623,115.10345525)(389.38976562,115.12345998)
}
}
{
\newrgbcolor{curcolor}{0 0 0}
\pscustom[linestyle=none,fillstyle=solid,fillcolor=curcolor]
{
\newpath
\moveto(420.12611206,42.02236623)
\curveto(420.17611281,42.04235669)(420.23611275,42.06735666)(420.30611206,42.09736623)
\curveto(420.37611261,42.1273566)(420.45111253,42.14735658)(420.53111206,42.15736623)
\curveto(420.60111238,42.17735655)(420.67111231,42.17735655)(420.74111206,42.15736623)
\curveto(420.80111218,42.14735658)(420.84611214,42.10735662)(420.87611206,42.03736623)
\curveto(420.89611209,41.98735674)(420.90611208,41.9273568)(420.90611206,41.85736623)
\lineto(420.90611206,41.64736623)
\lineto(420.90611206,41.19736623)
\curveto(420.90611208,41.04735768)(420.8811121,40.9273578)(420.83111206,40.83736623)
\curveto(420.77111221,40.73735799)(420.66611232,40.66235807)(420.51611206,40.61236623)
\curveto(420.36611262,40.57235816)(420.23111275,40.5273582)(420.11111206,40.47736623)
\curveto(419.85111313,40.36735836)(419.5811134,40.26735846)(419.30111206,40.17736623)
\curveto(419.02111396,40.08735864)(418.74611424,39.98735874)(418.47611206,39.87736623)
\curveto(418.3861146,39.84735888)(418.30111468,39.81735891)(418.22111206,39.78736623)
\curveto(418.14111484,39.76735896)(418.06611492,39.73735899)(417.99611206,39.69736623)
\curveto(417.92611506,39.66735906)(417.86611512,39.62235911)(417.81611206,39.56236623)
\curveto(417.76611522,39.50235923)(417.72611526,39.42235931)(417.69611206,39.32236623)
\curveto(417.67611531,39.27235946)(417.67111531,39.21235952)(417.68111206,39.14236623)
\lineto(417.68111206,38.94736623)
\lineto(417.68111206,36.11236623)
\lineto(417.68111206,35.81236623)
\curveto(417.67111531,35.70236303)(417.67111531,35.59736313)(417.68111206,35.49736623)
\curveto(417.69111529,35.39736333)(417.70611528,35.30236343)(417.72611206,35.21236623)
\curveto(417.74611524,35.1323636)(417.7861152,35.07236366)(417.84611206,35.03236623)
\curveto(417.94611504,34.95236378)(418.06111492,34.89236384)(418.19111206,34.85236623)
\curveto(418.31111467,34.82236391)(418.43611455,34.78236395)(418.56611206,34.73236623)
\curveto(418.79611419,34.6323641)(419.03611395,34.53736419)(419.28611206,34.44736623)
\curveto(419.53611345,34.36736436)(419.77611321,34.27736445)(420.00611206,34.17736623)
\curveto(420.06611292,34.15736457)(420.13611285,34.1323646)(420.21611206,34.10236623)
\curveto(420.2861127,34.08236465)(420.36111262,34.05736467)(420.44111206,34.02736623)
\curveto(420.52111246,33.99736473)(420.59611239,33.96236477)(420.66611206,33.92236623)
\curveto(420.72611226,33.89236484)(420.77111221,33.85736487)(420.80111206,33.81736623)
\curveto(420.86111212,33.73736499)(420.89611209,33.6273651)(420.90611206,33.48736623)
\lineto(420.90611206,33.06736623)
\lineto(420.90611206,32.82736623)
\curveto(420.89611209,32.75736597)(420.87111211,32.69736603)(420.83111206,32.64736623)
\curveto(420.80111218,32.59736613)(420.75611223,32.56736616)(420.69611206,32.55736623)
\curveto(420.63611235,32.55736617)(420.57611241,32.56236617)(420.51611206,32.57236623)
\curveto(420.44611254,32.59236614)(420.3811126,32.61236612)(420.32111206,32.63236623)
\curveto(420.25111273,32.66236607)(420.20111278,32.68736604)(420.17111206,32.70736623)
\curveto(419.85111313,32.84736588)(419.53611345,32.97236576)(419.22611206,33.08236623)
\curveto(418.90611408,33.19236554)(418.5861144,33.31236542)(418.26611206,33.44236623)
\curveto(418.04611494,33.5323652)(417.83111515,33.61736511)(417.62111206,33.69736623)
\curveto(417.40111558,33.77736495)(417.1811158,33.86236487)(416.96111206,33.95236623)
\curveto(416.24111674,34.25236448)(415.51611747,34.53736419)(414.78611206,34.80736623)
\curveto(414.04611894,35.07736365)(413.31111967,35.36236337)(412.58111206,35.66236623)
\curveto(412.32112066,35.77236296)(412.05612093,35.87236286)(411.78611206,35.96236623)
\curveto(411.51612147,36.06236267)(411.25112173,36.16736256)(410.99111206,36.27736623)
\curveto(410.8811221,36.3273624)(410.76112222,36.37236236)(410.63111206,36.41236623)
\curveto(410.49112249,36.46236227)(410.39112259,36.5323622)(410.33111206,36.62236623)
\curveto(410.29112269,36.66236207)(410.26112272,36.727362)(410.24111206,36.81736623)
\curveto(410.23112275,36.83736189)(410.23112275,36.85736187)(410.24111206,36.87736623)
\curveto(410.24112274,36.90736182)(410.23612275,36.9323618)(410.22611206,36.95236623)
\curveto(410.22612276,37.1323616)(410.22612276,37.34236139)(410.22611206,37.58236623)
\curveto(410.21612277,37.82236091)(410.25112273,37.99736073)(410.33111206,38.10736623)
\curveto(410.39112259,38.18736054)(410.49112249,38.24736048)(410.63111206,38.28736623)
\curveto(410.76112222,38.33736039)(410.8811221,38.38736034)(410.99111206,38.43736623)
\curveto(411.22112176,38.53736019)(411.45112153,38.6273601)(411.68111206,38.70736623)
\curveto(411.91112107,38.78735994)(412.14112084,38.87735985)(412.37111206,38.97736623)
\curveto(412.57112041,39.05735967)(412.77612021,39.1323596)(412.98611206,39.20236623)
\curveto(413.19611979,39.28235945)(413.40111958,39.36735936)(413.60111206,39.45736623)
\curveto(414.33111865,39.75735897)(415.07111791,40.04235869)(415.82111206,40.31236623)
\curveto(416.56111642,40.59235814)(417.29611569,40.88735784)(418.02611206,41.19736623)
\curveto(418.11611487,41.23735749)(418.20111478,41.26735746)(418.28111206,41.28736623)
\curveto(418.36111462,41.31735741)(418.44611454,41.34735738)(418.53611206,41.37736623)
\curveto(418.79611419,41.48735724)(419.06111392,41.59235714)(419.33111206,41.69236623)
\curveto(419.60111338,41.80235693)(419.86611312,41.91235682)(420.12611206,42.02236623)
\moveto(416.48111206,38.81236623)
\curveto(416.45111653,38.90235983)(416.40111658,38.95735977)(416.33111206,38.97736623)
\curveto(416.26111672,39.00735972)(416.1861168,39.01235972)(416.10611206,38.99236623)
\curveto(416.01611697,38.98235975)(415.93111705,38.95735977)(415.85111206,38.91736623)
\curveto(415.76111722,38.88735984)(415.6861173,38.85735987)(415.62611206,38.82736623)
\curveto(415.5861174,38.80735992)(415.55111743,38.79735993)(415.52111206,38.79736623)
\curveto(415.49111749,38.79735993)(415.45611753,38.78735994)(415.41611206,38.76736623)
\lineto(415.17611206,38.67736623)
\curveto(415.0861179,38.65736007)(414.99611799,38.6273601)(414.90611206,38.58736623)
\curveto(414.54611844,38.43736029)(414.1811188,38.30236043)(413.81111206,38.18236623)
\curveto(413.43111955,38.07236066)(413.06111992,37.94236079)(412.70111206,37.79236623)
\curveto(412.59112039,37.74236099)(412.4811205,37.69736103)(412.37111206,37.65736623)
\curveto(412.26112072,37.6273611)(412.15612083,37.58736114)(412.05611206,37.53736623)
\curveto(412.00612098,37.51736121)(411.96112102,37.49236124)(411.92111206,37.46236623)
\curveto(411.87112111,37.44236129)(411.84612114,37.39236134)(411.84611206,37.31236623)
\curveto(411.86612112,37.29236144)(411.8811211,37.27236146)(411.89111206,37.25236623)
\curveto(411.90112108,37.2323615)(411.91612107,37.21236152)(411.93611206,37.19236623)
\curveto(411.986121,37.15236158)(412.04112094,37.12236161)(412.10111206,37.10236623)
\curveto(412.15112083,37.08236165)(412.20612078,37.06236167)(412.26611206,37.04236623)
\curveto(412.37612061,36.99236174)(412.4861205,36.95236178)(412.59611206,36.92236623)
\curveto(412.70612028,36.89236184)(412.81612017,36.85236188)(412.92611206,36.80236623)
\curveto(413.31611967,36.6323621)(413.71111927,36.48236225)(414.11111206,36.35236623)
\curveto(414.51111847,36.2323625)(414.90111808,36.09236264)(415.28111206,35.93236623)
\lineto(415.43111206,35.87236623)
\curveto(415.4811175,35.86236287)(415.53111745,35.84736288)(415.58111206,35.82736623)
\lineto(415.82111206,35.73736623)
\curveto(415.90111708,35.70736302)(415.981117,35.68236305)(416.06111206,35.66236623)
\curveto(416.11111687,35.64236309)(416.16611682,35.6323631)(416.22611206,35.63236623)
\curveto(416.2861167,35.64236309)(416.33611665,35.65736307)(416.37611206,35.67736623)
\curveto(416.45611653,35.727363)(416.50111648,35.8323629)(416.51111206,35.99236623)
\lineto(416.51111206,36.44236623)
\lineto(416.51111206,38.04736623)
\curveto(416.51111647,38.15736057)(416.51611647,38.29236044)(416.52611206,38.45236623)
\curveto(416.52611646,38.61236012)(416.51111647,38.73236)(416.48111206,38.81236623)
}
}
{
\newrgbcolor{curcolor}{0 0 0}
\pscustom[linestyle=none,fillstyle=solid,fillcolor=curcolor]
{
\newpath
\moveto(416.87111206,50.56392873)
\curveto(416.92111606,50.57392038)(416.99111599,50.57892038)(417.08111206,50.57892873)
\curveto(417.16111582,50.57892038)(417.22611576,50.57392038)(417.27611206,50.56392873)
\curveto(417.31611567,50.56392039)(417.35611563,50.5589204)(417.39611206,50.54892873)
\lineto(417.51611206,50.54892873)
\curveto(417.59611539,50.52892043)(417.67611531,50.51892044)(417.75611206,50.51892873)
\curveto(417.83611515,50.51892044)(417.91611507,50.50892045)(417.99611206,50.48892873)
\curveto(418.03611495,50.47892048)(418.07611491,50.47392048)(418.11611206,50.47392873)
\curveto(418.14611484,50.47392048)(418.1811148,50.46892049)(418.22111206,50.45892873)
\curveto(418.33111465,50.42892053)(418.43611455,50.39892056)(418.53611206,50.36892873)
\curveto(418.63611435,50.34892061)(418.73611425,50.31892064)(418.83611206,50.27892873)
\curveto(419.1861138,50.13892082)(419.50111348,49.96892099)(419.78111206,49.76892873)
\curveto(420.06111292,49.56892139)(420.30111268,49.31892164)(420.50111206,49.01892873)
\curveto(420.60111238,48.86892209)(420.6861123,48.72392223)(420.75611206,48.58392873)
\curveto(420.80611218,48.47392248)(420.84611214,48.36392259)(420.87611206,48.25392873)
\curveto(420.90611208,48.1539228)(420.93611205,48.04892291)(420.96611206,47.93892873)
\curveto(420.986112,47.86892309)(420.99611199,47.80392315)(420.99611206,47.74392873)
\curveto(421.00611198,47.68392327)(421.02111196,47.62392333)(421.04111206,47.56392873)
\lineto(421.04111206,47.41392873)
\curveto(421.06111192,47.36392359)(421.07111191,47.28892367)(421.07111206,47.18892873)
\curveto(421.0811119,47.08892387)(421.07611191,47.00892395)(421.05611206,46.94892873)
\lineto(421.05611206,46.79892873)
\curveto(421.04611194,46.7589242)(421.04111194,46.71392424)(421.04111206,46.66392873)
\curveto(421.04111194,46.62392433)(421.03611195,46.57892438)(421.02611206,46.52892873)
\curveto(420.986112,46.37892458)(420.95111203,46.22892473)(420.92111206,46.07892873)
\curveto(420.89111209,45.93892502)(420.84611214,45.79892516)(420.78611206,45.65892873)
\curveto(420.70611228,45.4589255)(420.60611238,45.27892568)(420.48611206,45.11892873)
\lineto(420.33611206,44.93892873)
\curveto(420.27611271,44.87892608)(420.23611275,44.80892615)(420.21611206,44.72892873)
\curveto(420.20611278,44.66892629)(420.22111276,44.61892634)(420.26111206,44.57892873)
\curveto(420.29111269,44.54892641)(420.33611265,44.52392643)(420.39611206,44.50392873)
\curveto(420.45611253,44.49392646)(420.52111246,44.48392647)(420.59111206,44.47392873)
\curveto(420.65111233,44.47392648)(420.69611229,44.46392649)(420.72611206,44.44392873)
\curveto(420.77611221,44.40392655)(420.82111216,44.3589266)(420.86111206,44.30892873)
\curveto(420.8811121,44.2589267)(420.89611209,44.18892677)(420.90611206,44.09892873)
\lineto(420.90611206,43.82892873)
\curveto(420.90611208,43.73892722)(420.90111208,43.6539273)(420.89111206,43.57392873)
\curveto(420.87111211,43.49392746)(420.85111213,43.43392752)(420.83111206,43.39392873)
\curveto(420.81111217,43.37392758)(420.7861122,43.3539276)(420.75611206,43.33392873)
\lineto(420.66611206,43.27392873)
\curveto(420.5861124,43.24392771)(420.46611252,43.22892773)(420.30611206,43.22892873)
\curveto(420.14611284,43.23892772)(420.01111297,43.24392771)(419.90111206,43.24392873)
\lineto(411.09611206,43.24392873)
\curveto(410.97612201,43.24392771)(410.85112213,43.23892772)(410.72111206,43.22892873)
\curveto(410.5811224,43.22892773)(410.47112251,43.2539277)(410.39111206,43.30392873)
\curveto(410.33112265,43.34392761)(410.2811227,43.40892755)(410.24111206,43.49892873)
\curveto(410.24112274,43.51892744)(410.24112274,43.54392741)(410.24111206,43.57392873)
\curveto(410.23112275,43.60392735)(410.22612276,43.62892733)(410.22611206,43.64892873)
\curveto(410.21612277,43.78892717)(410.21612277,43.93392702)(410.22611206,44.08392873)
\curveto(410.22612276,44.24392671)(410.26612272,44.3539266)(410.34611206,44.41392873)
\curveto(410.42612256,44.46392649)(410.54112244,44.48892647)(410.69111206,44.48892873)
\lineto(411.09611206,44.48892873)
\lineto(412.85111206,44.48892873)
\lineto(413.10611206,44.48892873)
\lineto(413.39111206,44.48892873)
\curveto(413.4811195,44.49892646)(413.56611942,44.50892645)(413.64611206,44.51892873)
\curveto(413.71611927,44.53892642)(413.76611922,44.56892639)(413.79611206,44.60892873)
\curveto(413.82611916,44.64892631)(413.83111915,44.69392626)(413.81111206,44.74392873)
\curveto(413.79111919,44.79392616)(413.77111921,44.83392612)(413.75111206,44.86392873)
\curveto(413.71111927,44.91392604)(413.67111931,44.958926)(413.63111206,44.99892873)
\lineto(413.51111206,45.14892873)
\curveto(413.46111952,45.21892574)(413.41611957,45.28892567)(413.37611206,45.35892873)
\lineto(413.25611206,45.59892873)
\curveto(413.16611982,45.77892518)(413.10111988,45.99392496)(413.06111206,46.24392873)
\curveto(413.02111996,46.49392446)(413.00111998,46.74892421)(413.00111206,47.00892873)
\curveto(413.00111998,47.26892369)(413.02611996,47.52392343)(413.07611206,47.77392873)
\curveto(413.11611987,48.02392293)(413.17611981,48.24392271)(413.25611206,48.43392873)
\curveto(413.42611956,48.83392212)(413.66111932,49.17892178)(413.96111206,49.46892873)
\curveto(414.26111872,49.7589212)(414.61111837,49.98892097)(415.01111206,50.15892873)
\curveto(415.12111786,50.20892075)(415.23111775,50.24892071)(415.34111206,50.27892873)
\curveto(415.44111754,50.31892064)(415.54611744,50.3589206)(415.65611206,50.39892873)
\curveto(415.76611722,50.42892053)(415.8811171,50.44892051)(416.00111206,50.45892873)
\lineto(416.33111206,50.51892873)
\curveto(416.36111662,50.52892043)(416.39611659,50.53392042)(416.43611206,50.53392873)
\curveto(416.46611652,50.53392042)(416.49611649,50.53892042)(416.52611206,50.54892873)
\curveto(416.5861164,50.56892039)(416.64611634,50.56892039)(416.70611206,50.54892873)
\curveto(416.75611623,50.53892042)(416.81111617,50.54392041)(416.87111206,50.56392873)
\moveto(417.26111206,49.22892873)
\curveto(417.21111577,49.24892171)(417.15111583,49.2539217)(417.08111206,49.24392873)
\curveto(417.01111597,49.23392172)(416.94611604,49.22892173)(416.88611206,49.22892873)
\curveto(416.71611627,49.22892173)(416.55611643,49.21892174)(416.40611206,49.19892873)
\curveto(416.25611673,49.18892177)(416.12111686,49.1589218)(416.00111206,49.10892873)
\curveto(415.90111708,49.07892188)(415.81111717,49.0539219)(415.73111206,49.03392873)
\curveto(415.65111733,49.01392194)(415.57111741,48.98392197)(415.49111206,48.94392873)
\curveto(415.24111774,48.83392212)(415.01111797,48.68392227)(414.80111206,48.49392873)
\curveto(414.5811184,48.30392265)(414.41611857,48.08392287)(414.30611206,47.83392873)
\curveto(414.27611871,47.7539232)(414.25111873,47.67392328)(414.23111206,47.59392873)
\curveto(414.20111878,47.52392343)(414.17611881,47.44892351)(414.15611206,47.36892873)
\curveto(414.12611886,47.2589237)(414.11111887,47.14892381)(414.11111206,47.03892873)
\curveto(414.10111888,46.92892403)(414.09611889,46.80892415)(414.09611206,46.67892873)
\curveto(414.10611888,46.62892433)(414.11611887,46.58392437)(414.12611206,46.54392873)
\lineto(414.12611206,46.40892873)
\lineto(414.18611206,46.13892873)
\curveto(414.20611878,46.0589249)(414.23611875,45.97892498)(414.27611206,45.89892873)
\curveto(414.41611857,45.5589254)(414.62611836,45.28892567)(414.90611206,45.08892873)
\curveto(415.17611781,44.88892607)(415.49611749,44.72892623)(415.86611206,44.60892873)
\curveto(415.97611701,44.56892639)(416.0861169,44.54392641)(416.19611206,44.53392873)
\curveto(416.30611668,44.52392643)(416.42111656,44.50392645)(416.54111206,44.47392873)
\curveto(416.59111639,44.46392649)(416.63611635,44.46392649)(416.67611206,44.47392873)
\curveto(416.71611627,44.48392647)(416.76111622,44.47892648)(416.81111206,44.45892873)
\curveto(416.86111612,44.44892651)(416.93611605,44.44392651)(417.03611206,44.44392873)
\curveto(417.12611586,44.44392651)(417.19611579,44.44892651)(417.24611206,44.45892873)
\lineto(417.36611206,44.45892873)
\curveto(417.40611558,44.46892649)(417.44611554,44.47392648)(417.48611206,44.47392873)
\curveto(417.52611546,44.47392648)(417.56111542,44.47892648)(417.59111206,44.48892873)
\curveto(417.62111536,44.49892646)(417.65611533,44.50392645)(417.69611206,44.50392873)
\curveto(417.72611526,44.50392645)(417.75611523,44.50892645)(417.78611206,44.51892873)
\curveto(417.86611512,44.53892642)(417.94611504,44.5539264)(418.02611206,44.56392873)
\lineto(418.26611206,44.62392873)
\curveto(418.60611438,44.73392622)(418.89611409,44.88392607)(419.13611206,45.07392873)
\curveto(419.37611361,45.27392568)(419.57611341,45.51892544)(419.73611206,45.80892873)
\curveto(419.7861132,45.89892506)(419.82611316,45.99392496)(419.85611206,46.09392873)
\curveto(419.87611311,46.19392476)(419.90111308,46.29892466)(419.93111206,46.40892873)
\curveto(419.95111303,46.4589245)(419.96111302,46.50392445)(419.96111206,46.54392873)
\curveto(419.95111303,46.59392436)(419.95111303,46.64392431)(419.96111206,46.69392873)
\curveto(419.97111301,46.73392422)(419.97611301,46.77892418)(419.97611206,46.82892873)
\lineto(419.97611206,46.96392873)
\lineto(419.97611206,47.09892873)
\curveto(419.96611302,47.13892382)(419.96111302,47.17392378)(419.96111206,47.20392873)
\curveto(419.96111302,47.23392372)(419.95611303,47.26892369)(419.94611206,47.30892873)
\curveto(419.92611306,47.38892357)(419.91111307,47.46392349)(419.90111206,47.53392873)
\curveto(419.8811131,47.60392335)(419.85611313,47.67892328)(419.82611206,47.75892873)
\curveto(419.69611329,48.06892289)(419.52611346,48.31892264)(419.31611206,48.50892873)
\curveto(419.09611389,48.69892226)(418.83111415,48.8589221)(418.52111206,48.98892873)
\curveto(418.3811146,49.03892192)(418.24111474,49.07392188)(418.10111206,49.09392873)
\curveto(417.95111503,49.12392183)(417.80111518,49.1589218)(417.65111206,49.19892873)
\curveto(417.60111538,49.21892174)(417.55611543,49.22392173)(417.51611206,49.21392873)
\curveto(417.46611552,49.21392174)(417.41611557,49.21892174)(417.36611206,49.22892873)
\lineto(417.26111206,49.22892873)
}
}
{
\newrgbcolor{curcolor}{0 0 0}
\pscustom[linestyle=none,fillstyle=solid,fillcolor=curcolor]
{
\newpath
\moveto(413.00111206,55.69017873)
\curveto(413.00111998,55.92017394)(413.06111992,56.05017381)(413.18111206,56.08017873)
\curveto(413.29111969,56.11017375)(413.45611953,56.12517374)(413.67611206,56.12517873)
\lineto(413.96111206,56.12517873)
\curveto(414.05111893,56.12517374)(414.12611886,56.10017376)(414.18611206,56.05017873)
\curveto(414.26611872,55.99017387)(414.31111867,55.90517396)(414.32111206,55.79517873)
\curveto(414.32111866,55.68517418)(414.33611865,55.57517429)(414.36611206,55.46517873)
\curveto(414.39611859,55.32517454)(414.42611856,55.19017467)(414.45611206,55.06017873)
\curveto(414.4861185,54.94017492)(414.52611846,54.82517504)(414.57611206,54.71517873)
\curveto(414.70611828,54.42517544)(414.8861181,54.19017567)(415.11611206,54.01017873)
\curveto(415.33611765,53.83017603)(415.59111739,53.67517619)(415.88111206,53.54517873)
\curveto(415.99111699,53.50517636)(416.10611688,53.47517639)(416.22611206,53.45517873)
\curveto(416.33611665,53.43517643)(416.45111653,53.41017645)(416.57111206,53.38017873)
\curveto(416.62111636,53.37017649)(416.67111631,53.3651765)(416.72111206,53.36517873)
\curveto(416.77111621,53.37517649)(416.82111616,53.37517649)(416.87111206,53.36517873)
\curveto(416.99111599,53.33517653)(417.13111585,53.32017654)(417.29111206,53.32017873)
\curveto(417.44111554,53.33017653)(417.5861154,53.33517653)(417.72611206,53.33517873)
\lineto(419.57111206,53.33517873)
\lineto(419.91611206,53.33517873)
\curveto(420.03611295,53.33517653)(420.15111283,53.33017653)(420.26111206,53.32017873)
\curveto(420.37111261,53.31017655)(420.46611252,53.30517656)(420.54611206,53.30517873)
\curveto(420.62611236,53.31517655)(420.69611229,53.29517657)(420.75611206,53.24517873)
\curveto(420.82611216,53.19517667)(420.86611212,53.11517675)(420.87611206,53.00517873)
\curveto(420.8861121,52.90517696)(420.89111209,52.79517707)(420.89111206,52.67517873)
\lineto(420.89111206,52.40517873)
\curveto(420.87111211,52.35517751)(420.85611213,52.30517756)(420.84611206,52.25517873)
\curveto(420.82611216,52.21517765)(420.80111218,52.18517768)(420.77111206,52.16517873)
\curveto(420.70111228,52.11517775)(420.61611237,52.08517778)(420.51611206,52.07517873)
\lineto(420.18611206,52.07517873)
\lineto(419.03111206,52.07517873)
\lineto(414.87611206,52.07517873)
\lineto(413.84111206,52.07517873)
\lineto(413.54111206,52.07517873)
\curveto(413.44111954,52.08517778)(413.35611963,52.11517775)(413.28611206,52.16517873)
\curveto(413.24611974,52.19517767)(413.21611977,52.24517762)(413.19611206,52.31517873)
\curveto(413.17611981,52.39517747)(413.16611982,52.48017738)(413.16611206,52.57017873)
\curveto(413.15611983,52.6601772)(413.15611983,52.75017711)(413.16611206,52.84017873)
\curveto(413.17611981,52.93017693)(413.19111979,53.00017686)(413.21111206,53.05017873)
\curveto(413.24111974,53.13017673)(413.30111968,53.18017668)(413.39111206,53.20017873)
\curveto(413.47111951,53.23017663)(413.56111942,53.24517662)(413.66111206,53.24517873)
\lineto(413.96111206,53.24517873)
\curveto(414.06111892,53.24517662)(414.15111883,53.2651766)(414.23111206,53.30517873)
\curveto(414.25111873,53.31517655)(414.26611872,53.32517654)(414.27611206,53.33517873)
\lineto(414.32111206,53.38017873)
\curveto(414.32111866,53.49017637)(414.27611871,53.58017628)(414.18611206,53.65017873)
\curveto(414.0861189,53.72017614)(414.00611898,53.78017608)(413.94611206,53.83017873)
\lineto(413.85611206,53.92017873)
\curveto(413.74611924,54.01017585)(413.63111935,54.13517573)(413.51111206,54.29517873)
\curveto(413.39111959,54.45517541)(413.30111968,54.60517526)(413.24111206,54.74517873)
\curveto(413.19111979,54.83517503)(413.15611983,54.93017493)(413.13611206,55.03017873)
\curveto(413.10611988,55.13017473)(413.07611991,55.23517463)(413.04611206,55.34517873)
\curveto(413.03611995,55.40517446)(413.03111995,55.4651744)(413.03111206,55.52517873)
\curveto(413.02111996,55.58517428)(413.01111997,55.64017422)(413.00111206,55.69017873)
}
}
{
\newrgbcolor{curcolor}{0 0 0}
\pscustom[linestyle=none,fillstyle=solid,fillcolor=curcolor]
{
}
}
{
\newrgbcolor{curcolor}{0 0 0}
\pscustom[linestyle=none,fillstyle=solid,fillcolor=curcolor]
{
\newpath
\moveto(410.30111206,65.05510061)
\curveto(410.30112268,65.15509575)(410.31112267,65.25009566)(410.33111206,65.34010061)
\curveto(410.34112264,65.43009548)(410.37112261,65.49509541)(410.42111206,65.53510061)
\curveto(410.50112248,65.59509531)(410.60612238,65.62509528)(410.73611206,65.62510061)
\lineto(411.12611206,65.62510061)
\lineto(412.62611206,65.62510061)
\lineto(419.01611206,65.62510061)
\lineto(420.18611206,65.62510061)
\lineto(420.50111206,65.62510061)
\curveto(420.60111238,65.63509527)(420.6811123,65.62009529)(420.74111206,65.58010061)
\curveto(420.82111216,65.53009538)(420.87111211,65.45509545)(420.89111206,65.35510061)
\curveto(420.90111208,65.26509564)(420.90611208,65.15509575)(420.90611206,65.02510061)
\lineto(420.90611206,64.80010061)
\curveto(420.8861121,64.72009619)(420.87111211,64.65009626)(420.86111206,64.59010061)
\curveto(420.84111214,64.53009638)(420.80111218,64.48009643)(420.74111206,64.44010061)
\curveto(420.6811123,64.40009651)(420.60611238,64.38009653)(420.51611206,64.38010061)
\lineto(420.21611206,64.38010061)
\lineto(419.12111206,64.38010061)
\lineto(413.78111206,64.38010061)
\curveto(413.69111929,64.36009655)(413.61611937,64.34509656)(413.55611206,64.33510061)
\curveto(413.4861195,64.33509657)(413.42611956,64.3050966)(413.37611206,64.24510061)
\curveto(413.32611966,64.17509673)(413.30111968,64.08509682)(413.30111206,63.97510061)
\curveto(413.29111969,63.87509703)(413.2861197,63.76509714)(413.28611206,63.64510061)
\lineto(413.28611206,62.50510061)
\lineto(413.28611206,62.01010061)
\curveto(413.27611971,61.85009906)(413.21611977,61.74009917)(413.10611206,61.68010061)
\curveto(413.07611991,61.66009925)(413.04611994,61.65009926)(413.01611206,61.65010061)
\curveto(412.97612001,61.65009926)(412.93112005,61.64509926)(412.88111206,61.63510061)
\curveto(412.76112022,61.61509929)(412.65112033,61.62009929)(412.55111206,61.65010061)
\curveto(412.45112053,61.69009922)(412.3811206,61.74509916)(412.34111206,61.81510061)
\curveto(412.29112069,61.89509901)(412.26612072,62.01509889)(412.26611206,62.17510061)
\curveto(412.26612072,62.33509857)(412.25112073,62.47009844)(412.22111206,62.58010061)
\curveto(412.21112077,62.63009828)(412.20612078,62.68509822)(412.20611206,62.74510061)
\curveto(412.19612079,62.8050981)(412.1811208,62.86509804)(412.16111206,62.92510061)
\curveto(412.11112087,63.07509783)(412.06112092,63.22009769)(412.01111206,63.36010061)
\curveto(411.95112103,63.50009741)(411.8811211,63.63509727)(411.80111206,63.76510061)
\curveto(411.71112127,63.905097)(411.60612138,64.02509688)(411.48611206,64.12510061)
\curveto(411.36612162,64.22509668)(411.23612175,64.32009659)(411.09611206,64.41010061)
\curveto(410.99612199,64.47009644)(410.8861221,64.51509639)(410.76611206,64.54510061)
\curveto(410.64612234,64.58509632)(410.54112244,64.63509627)(410.45111206,64.69510061)
\curveto(410.39112259,64.74509616)(410.35112263,64.81509609)(410.33111206,64.90510061)
\curveto(410.32112266,64.92509598)(410.31612267,64.95009596)(410.31611206,64.98010061)
\curveto(410.31612267,65.0100959)(410.31112267,65.03509587)(410.30111206,65.05510061)
}
}
{
\newrgbcolor{curcolor}{0 0 0}
\pscustom[linestyle=none,fillstyle=solid,fillcolor=curcolor]
{
\newpath
\moveto(415.31111206,76.28470998)
\curveto(415.39111759,76.28470235)(415.47111751,76.28970234)(415.55111206,76.29970998)
\curveto(415.63111735,76.30970232)(415.70611728,76.30470233)(415.77611206,76.28470998)
\curveto(415.81611717,76.26470237)(415.86111712,76.25970237)(415.91111206,76.26970998)
\curveto(415.95111703,76.27970235)(415.99111699,76.27970235)(416.03111206,76.26970998)
\lineto(416.18111206,76.26970998)
\curveto(416.27111671,76.25970237)(416.36111662,76.25470238)(416.45111206,76.25470998)
\curveto(416.53111645,76.25470238)(416.61111637,76.24970238)(416.69111206,76.23970998)
\lineto(416.93111206,76.20970998)
\curveto(417.00111598,76.19970243)(417.07611591,76.18970244)(417.15611206,76.17970998)
\curveto(417.19611579,76.16970246)(417.23611575,76.16470247)(417.27611206,76.16470998)
\curveto(417.31611567,76.16470247)(417.36111562,76.15970247)(417.41111206,76.14970998)
\curveto(417.55111543,76.10970252)(417.69111529,76.07970255)(417.83111206,76.05970998)
\curveto(417.97111501,76.04970258)(418.10611488,76.01970261)(418.23611206,75.96970998)
\curveto(418.40611458,75.91970271)(418.57111441,75.86470277)(418.73111206,75.80470998)
\curveto(418.89111409,75.75470288)(419.04611394,75.69470294)(419.19611206,75.62470998)
\curveto(419.25611373,75.60470303)(419.31611367,75.57470306)(419.37611206,75.53470998)
\lineto(419.52611206,75.44470998)
\curveto(419.84611314,75.24470339)(420.11111287,75.0297036)(420.32111206,74.79970998)
\curveto(420.53111245,74.56970406)(420.71111227,74.27470436)(420.86111206,73.91470998)
\curveto(420.91111207,73.79470484)(420.94611204,73.66470497)(420.96611206,73.52470998)
\curveto(420.986112,73.39470524)(421.01111197,73.25970537)(421.04111206,73.11970998)
\curveto(421.05111193,73.05970557)(421.05611193,72.99970563)(421.05611206,72.93970998)
\curveto(421.05611193,72.87970575)(421.06111192,72.81470582)(421.07111206,72.74470998)
\curveto(421.0811119,72.71470592)(421.0811119,72.66470597)(421.07111206,72.59470998)
\lineto(421.07111206,72.44470998)
\lineto(421.07111206,72.29470998)
\curveto(421.05111193,72.21470642)(421.03611195,72.1297065)(421.02611206,72.03970998)
\curveto(421.02611196,71.95970667)(421.01611197,71.88470675)(420.99611206,71.81470998)
\curveto(420.986112,71.77470686)(420.981112,71.73970689)(420.98111206,71.70970998)
\curveto(420.99111199,71.68970694)(420.986112,71.66470697)(420.96611206,71.63470998)
\lineto(420.90611206,71.36470998)
\curveto(420.87611211,71.27470736)(420.84611214,71.18970744)(420.81611206,71.10970998)
\curveto(420.57611241,70.5297081)(420.20611278,70.09470854)(419.70611206,69.80470998)
\curveto(419.57611341,69.72470891)(419.44111354,69.65970897)(419.30111206,69.60970998)
\curveto(419.16111382,69.56970906)(419.01111397,69.52470911)(418.85111206,69.47470998)
\curveto(418.77111421,69.45470918)(418.69111429,69.44970918)(418.61111206,69.45970998)
\curveto(418.53111445,69.47970915)(418.47611451,69.51470912)(418.44611206,69.56470998)
\curveto(418.42611456,69.59470904)(418.41111457,69.64970898)(418.40111206,69.72970998)
\curveto(418.3811146,69.80970882)(418.37111461,69.89470874)(418.37111206,69.98470998)
\curveto(418.36111462,70.07470856)(418.36111462,70.15970847)(418.37111206,70.23970998)
\curveto(418.3811146,70.3297083)(418.39111459,70.39970823)(418.40111206,70.44970998)
\curveto(418.41111457,70.46970816)(418.42611456,70.49470814)(418.44611206,70.52470998)
\curveto(418.46611452,70.56470807)(418.4861145,70.59470804)(418.50611206,70.61470998)
\curveto(418.5861144,70.67470796)(418.6811143,70.71970791)(418.79111206,70.74970998)
\curveto(418.90111408,70.78970784)(419.00111398,70.8347078)(419.09111206,70.88470998)
\curveto(419.4811135,71.1347075)(419.75111323,71.50470713)(419.90111206,71.99470998)
\curveto(419.92111306,72.06470657)(419.93611305,72.1347065)(419.94611206,72.20470998)
\curveto(419.94611304,72.28470635)(419.95611303,72.36470627)(419.97611206,72.44470998)
\curveto(419.986113,72.48470615)(419.99111299,72.53970609)(419.99111206,72.60970998)
\curveto(419.99111299,72.68970594)(419.986113,72.74470589)(419.97611206,72.77470998)
\curveto(419.96611302,72.80470583)(419.96111302,72.8347058)(419.96111206,72.86470998)
\lineto(419.96111206,72.96970998)
\curveto(419.94111304,73.04970558)(419.92111306,73.12470551)(419.90111206,73.19470998)
\curveto(419.8811131,73.27470536)(419.85611313,73.34970528)(419.82611206,73.41970998)
\curveto(419.67611331,73.76970486)(419.46111352,74.03970459)(419.18111206,74.22970998)
\curveto(418.90111408,74.41970421)(418.57611441,74.57470406)(418.20611206,74.69470998)
\curveto(418.12611486,74.72470391)(418.05111493,74.74470389)(417.98111206,74.75470998)
\curveto(417.91111507,74.77470386)(417.83611515,74.79470384)(417.75611206,74.81470998)
\curveto(417.66611532,74.8347038)(417.57111541,74.84970378)(417.47111206,74.85970998)
\curveto(417.36111562,74.87970375)(417.25611573,74.89970373)(417.15611206,74.91970998)
\curveto(417.10611588,74.9297037)(417.05611593,74.9347037)(417.00611206,74.93470998)
\curveto(416.94611604,74.94470369)(416.89111609,74.94970368)(416.84111206,74.94970998)
\curveto(416.7811162,74.96970366)(416.70611628,74.97970365)(416.61611206,74.97970998)
\curveto(416.51611647,74.97970365)(416.43611655,74.96970366)(416.37611206,74.94970998)
\curveto(416.2861167,74.91970371)(416.24611674,74.86970376)(416.25611206,74.79970998)
\curveto(416.26611672,74.73970389)(416.29611669,74.68470395)(416.34611206,74.63470998)
\curveto(416.39611659,74.55470408)(416.45611653,74.48470415)(416.52611206,74.42470998)
\curveto(416.59611639,74.37470426)(416.65611633,74.30970432)(416.70611206,74.22970998)
\curveto(416.81611617,74.06970456)(416.91611607,73.90470473)(417.00611206,73.73470998)
\curveto(417.0861159,73.56470507)(417.15611583,73.36970526)(417.21611206,73.14970998)
\curveto(417.24611574,73.04970558)(417.26111572,72.94970568)(417.26111206,72.84970998)
\curveto(417.26111572,72.75970587)(417.27111571,72.65970597)(417.29111206,72.54970998)
\lineto(417.29111206,72.39970998)
\curveto(417.27111571,72.34970628)(417.26611572,72.29970633)(417.27611206,72.24970998)
\curveto(417.2861157,72.20970642)(417.2861157,72.16970646)(417.27611206,72.12970998)
\curveto(417.26611572,72.09970653)(417.26111572,72.05470658)(417.26111206,71.99470998)
\curveto(417.25111573,71.9347067)(417.24111574,71.86970676)(417.23111206,71.79970998)
\lineto(417.20111206,71.61970998)
\curveto(417.0811159,71.16970746)(416.91611607,70.78970784)(416.70611206,70.47970998)
\curveto(416.51611647,70.20970842)(416.2861167,69.97970865)(416.01611206,69.78970998)
\curveto(415.73611725,69.60970902)(415.42111756,69.46470917)(415.07111206,69.35470998)
\lineto(414.86111206,69.29470998)
\curveto(414.7811182,69.28470935)(414.70111828,69.26970936)(414.62111206,69.24970998)
\curveto(414.59111839,69.23970939)(414.56111842,69.2347094)(414.53111206,69.23470998)
\curveto(414.50111848,69.2347094)(414.47111851,69.2297094)(414.44111206,69.21970998)
\curveto(414.3811186,69.20970942)(414.32111866,69.20470943)(414.26111206,69.20470998)
\curveto(414.19111879,69.20470943)(414.13111885,69.19470944)(414.08111206,69.17470998)
\lineto(413.90111206,69.17470998)
\curveto(413.85111913,69.16470947)(413.7811192,69.15970947)(413.69111206,69.15970998)
\curveto(413.60111938,69.15970947)(413.53111945,69.16970946)(413.48111206,69.18970998)
\lineto(413.31611206,69.18970998)
\curveto(413.23611975,69.20970942)(413.16111982,69.21970941)(413.09111206,69.21970998)
\curveto(413.02111996,69.2297094)(412.95112003,69.24470939)(412.88111206,69.26470998)
\curveto(412.6811203,69.32470931)(412.49112049,69.38470925)(412.31111206,69.44470998)
\curveto(412.13112085,69.51470912)(411.96112102,69.60470903)(411.80111206,69.71470998)
\curveto(411.73112125,69.75470888)(411.66612132,69.79470884)(411.60611206,69.83470998)
\lineto(411.42611206,69.98470998)
\curveto(411.41612157,70.00470863)(411.40112158,70.02470861)(411.38111206,70.04470998)
\curveto(411.25112173,70.1347085)(411.14112184,70.24470839)(411.05111206,70.37470998)
\curveto(410.85112213,70.634708)(410.69612229,70.89970773)(410.58611206,71.16970998)
\curveto(410.54612244,71.24970738)(410.51612247,71.3297073)(410.49611206,71.40970998)
\curveto(410.46612252,71.49970713)(410.44112254,71.58970704)(410.42111206,71.67970998)
\curveto(410.39112259,71.77970685)(410.37112261,71.87970675)(410.36111206,71.97970998)
\curveto(410.35112263,72.07970655)(410.33612265,72.18470645)(410.31611206,72.29470998)
\curveto(410.30612268,72.32470631)(410.30612268,72.36470627)(410.31611206,72.41470998)
\curveto(410.32612266,72.47470616)(410.32112266,72.51470612)(410.30111206,72.53470998)
\curveto(410.2811227,73.25470538)(410.39612259,73.85470478)(410.64611206,74.33470998)
\curveto(410.89612209,74.81470382)(411.23612175,75.18970344)(411.66611206,75.45970998)
\curveto(411.80612118,75.54970308)(411.95112103,75.629703)(412.10111206,75.69970998)
\curveto(412.25112073,75.76970286)(412.41112057,75.83970279)(412.58111206,75.90970998)
\curveto(412.72112026,75.95970267)(412.87112011,75.99970263)(413.03111206,76.02970998)
\curveto(413.19111979,76.05970257)(413.35111963,76.09470254)(413.51111206,76.13470998)
\curveto(413.56111942,76.15470248)(413.61611937,76.16470247)(413.67611206,76.16470998)
\curveto(413.72611926,76.16470247)(413.77611921,76.16970246)(413.82611206,76.17970998)
\curveto(413.8861191,76.19970243)(413.95111903,76.20970242)(414.02111206,76.20970998)
\curveto(414.0811189,76.20970242)(414.13611885,76.21970241)(414.18611206,76.23970998)
\lineto(414.35111206,76.23970998)
\curveto(414.40111858,76.25970237)(414.45111853,76.26470237)(414.50111206,76.25470998)
\curveto(414.55111843,76.24470239)(414.60111838,76.24970238)(414.65111206,76.26970998)
\curveto(414.67111831,76.26970236)(414.69611829,76.26470237)(414.72611206,76.25470998)
\curveto(414.75611823,76.25470238)(414.7811182,76.25970237)(414.80111206,76.26970998)
\curveto(414.83111815,76.27970235)(414.86611812,76.27970235)(414.90611206,76.26970998)
\curveto(414.94611804,76.26970236)(414.986118,76.27470236)(415.02611206,76.28470998)
\curveto(415.06611792,76.29470234)(415.11111787,76.29470234)(415.16111206,76.28470998)
\lineto(415.31111206,76.28470998)
\moveto(414.00611206,74.78470998)
\curveto(413.95611903,74.79470384)(413.89611909,74.79970383)(413.82611206,74.79970998)
\curveto(413.75611923,74.79970383)(413.69611929,74.79470384)(413.64611206,74.78470998)
\curveto(413.59611939,74.77470386)(413.52111946,74.76970386)(413.42111206,74.76970998)
\curveto(413.34111964,74.74970388)(413.26611972,74.7297039)(413.19611206,74.70970998)
\curveto(413.12611986,74.69970393)(413.05611993,74.68470395)(412.98611206,74.66470998)
\curveto(412.55612043,74.52470411)(412.22112076,74.3297043)(411.98111206,74.07970998)
\curveto(411.74112124,73.83970479)(411.56112142,73.49470514)(411.44111206,73.04470998)
\curveto(411.42112156,72.95470568)(411.41112157,72.85470578)(411.41111206,72.74470998)
\lineto(411.41111206,72.41470998)
\curveto(411.43112155,72.39470624)(411.44112154,72.35970627)(411.44111206,72.30970998)
\curveto(411.43112155,72.25970637)(411.43112155,72.21470642)(411.44111206,72.17470998)
\curveto(411.46112152,72.09470654)(411.4811215,72.01970661)(411.50111206,71.94970998)
\lineto(411.56111206,71.73970998)
\curveto(411.69112129,71.44970718)(411.87112111,71.21970741)(412.10111206,71.04970998)
\curveto(412.32112066,70.87970775)(412.5811204,70.74470789)(412.88111206,70.64470998)
\curveto(412.97112001,70.61470802)(413.06611992,70.58970804)(413.16611206,70.56970998)
\curveto(413.25611973,70.55970807)(413.35111963,70.54470809)(413.45111206,70.52470998)
\lineto(413.58611206,70.52470998)
\curveto(413.69611929,70.49470814)(413.83611915,70.48470815)(414.00611206,70.49470998)
\curveto(414.16611882,70.51470812)(414.29611869,70.5347081)(414.39611206,70.55470998)
\curveto(414.45611853,70.57470806)(414.51611847,70.58970804)(414.57611206,70.59970998)
\curveto(414.62611836,70.60970802)(414.67611831,70.62470801)(414.72611206,70.64470998)
\curveto(414.92611806,70.72470791)(415.11611787,70.81970781)(415.29611206,70.92970998)
\curveto(415.47611751,71.04970758)(415.62111736,71.18970744)(415.73111206,71.34970998)
\curveto(415.7811172,71.39970723)(415.82111716,71.45470718)(415.85111206,71.51470998)
\curveto(415.8811171,71.57470706)(415.91611707,71.634707)(415.95611206,71.69470998)
\curveto(416.03611695,71.84470679)(416.10111688,72.0297066)(416.15111206,72.24970998)
\curveto(416.17111681,72.29970633)(416.17611681,72.33970629)(416.16611206,72.36970998)
\curveto(416.15611683,72.40970622)(416.16111682,72.45470618)(416.18111206,72.50470998)
\curveto(416.19111679,72.54470609)(416.19611679,72.59970603)(416.19611206,72.66970998)
\curveto(416.19611679,72.73970589)(416.19111679,72.79970583)(416.18111206,72.84970998)
\curveto(416.16111682,72.94970568)(416.14611684,73.04470559)(416.13611206,73.13470998)
\curveto(416.11611687,73.22470541)(416.0861169,73.31470532)(416.04611206,73.40470998)
\curveto(415.82611716,73.94470469)(415.43111755,74.33970429)(414.86111206,74.58970998)
\curveto(414.76111822,74.63970399)(414.66111832,74.67470396)(414.56111206,74.69470998)
\curveto(414.45111853,74.71470392)(414.34111864,74.73970389)(414.23111206,74.76970998)
\curveto(414.13111885,74.76970386)(414.05611893,74.77470386)(414.00611206,74.78470998)
}
}
{
\newrgbcolor{curcolor}{0 0 0}
\pscustom[linestyle=none,fillstyle=solid,fillcolor=curcolor]
{
\newpath
\moveto(419.27111206,78.63431936)
\lineto(419.27111206,79.26431936)
\lineto(419.27111206,79.45931936)
\curveto(419.27111371,79.52931683)(419.2811137,79.58931677)(419.30111206,79.63931936)
\curveto(419.34111364,79.70931665)(419.3811136,79.7593166)(419.42111206,79.78931936)
\curveto(419.47111351,79.82931653)(419.53611345,79.84931651)(419.61611206,79.84931936)
\curveto(419.69611329,79.8593165)(419.7811132,79.86431649)(419.87111206,79.86431936)
\lineto(420.59111206,79.86431936)
\curveto(421.07111191,79.86431649)(421.4811115,79.80431655)(421.82111206,79.68431936)
\curveto(422.16111082,79.56431679)(422.43611055,79.36931699)(422.64611206,79.09931936)
\curveto(422.69611029,79.02931733)(422.74111024,78.9593174)(422.78111206,78.88931936)
\curveto(422.83111015,78.82931753)(422.87611011,78.7543176)(422.91611206,78.66431936)
\curveto(422.92611006,78.64431771)(422.93611005,78.61431774)(422.94611206,78.57431936)
\curveto(422.96611002,78.53431782)(422.97111001,78.48931787)(422.96111206,78.43931936)
\curveto(422.93111005,78.34931801)(422.85611013,78.29431806)(422.73611206,78.27431936)
\curveto(422.62611036,78.2543181)(422.53111045,78.26931809)(422.45111206,78.31931936)
\curveto(422.3811106,78.34931801)(422.31611067,78.39431796)(422.25611206,78.45431936)
\curveto(422.20611078,78.52431783)(422.15611083,78.58931777)(422.10611206,78.64931936)
\curveto(422.05611093,78.71931764)(421.981111,78.77931758)(421.88111206,78.82931936)
\curveto(421.79111119,78.88931747)(421.70111128,78.93931742)(421.61111206,78.97931936)
\curveto(421.5811114,78.99931736)(421.52111146,79.02431733)(421.43111206,79.05431936)
\curveto(421.35111163,79.08431727)(421.2811117,79.08931727)(421.22111206,79.06931936)
\curveto(421.0811119,79.03931732)(420.99111199,78.97931738)(420.95111206,78.88931936)
\curveto(420.92111206,78.80931755)(420.90611208,78.71931764)(420.90611206,78.61931936)
\curveto(420.90611208,78.51931784)(420.8811121,78.43431792)(420.83111206,78.36431936)
\curveto(420.76111222,78.27431808)(420.62111236,78.22931813)(420.41111206,78.22931936)
\lineto(419.85611206,78.22931936)
\lineto(419.63111206,78.22931936)
\curveto(419.55111343,78.23931812)(419.4861135,78.2593181)(419.43611206,78.28931936)
\curveto(419.35611363,78.34931801)(419.31111367,78.41931794)(419.30111206,78.49931936)
\curveto(419.29111369,78.51931784)(419.2861137,78.53931782)(419.28611206,78.55931936)
\curveto(419.2861137,78.58931777)(419.2811137,78.61431774)(419.27111206,78.63431936)
}
}
{
\newrgbcolor{curcolor}{0 0 0}
\pscustom[linestyle=none,fillstyle=solid,fillcolor=curcolor]
{
}
}
{
\newrgbcolor{curcolor}{0 0 0}
\pscustom[linestyle=none,fillstyle=solid,fillcolor=curcolor]
{
\newpath
\moveto(410.30111206,89.26463186)
\curveto(410.29112269,89.95462722)(410.41112257,90.55462662)(410.66111206,91.06463186)
\curveto(410.91112207,91.58462559)(411.24612174,91.9796252)(411.66611206,92.24963186)
\curveto(411.74612124,92.29962488)(411.83612115,92.34462483)(411.93611206,92.38463186)
\curveto(412.02612096,92.42462475)(412.12112086,92.46962471)(412.22111206,92.51963186)
\curveto(412.32112066,92.55962462)(412.42112056,92.58962459)(412.52111206,92.60963186)
\curveto(412.62112036,92.62962455)(412.72612026,92.64962453)(412.83611206,92.66963186)
\curveto(412.8861201,92.68962449)(412.93112005,92.69462448)(412.97111206,92.68463186)
\curveto(413.01111997,92.6746245)(413.05611993,92.6796245)(413.10611206,92.69963186)
\curveto(413.15611983,92.70962447)(413.24111974,92.71462446)(413.36111206,92.71463186)
\curveto(413.47111951,92.71462446)(413.55611943,92.70962447)(413.61611206,92.69963186)
\curveto(413.67611931,92.6796245)(413.73611925,92.66962451)(413.79611206,92.66963186)
\curveto(413.85611913,92.6796245)(413.91611907,92.6746245)(413.97611206,92.65463186)
\curveto(414.11611887,92.61462456)(414.25111873,92.5796246)(414.38111206,92.54963186)
\curveto(414.51111847,92.51962466)(414.63611835,92.4796247)(414.75611206,92.42963186)
\curveto(414.89611809,92.36962481)(415.02111796,92.29962488)(415.13111206,92.21963186)
\curveto(415.24111774,92.14962503)(415.35111763,92.0746251)(415.46111206,91.99463186)
\lineto(415.52111206,91.93463186)
\curveto(415.54111744,91.92462525)(415.56111742,91.90962527)(415.58111206,91.88963186)
\curveto(415.74111724,91.76962541)(415.8861171,91.63462554)(416.01611206,91.48463186)
\curveto(416.14611684,91.33462584)(416.27111671,91.174626)(416.39111206,91.00463186)
\curveto(416.61111637,90.69462648)(416.81611617,90.39962678)(417.00611206,90.11963186)
\curveto(417.14611584,89.88962729)(417.2811157,89.65962752)(417.41111206,89.42963186)
\curveto(417.54111544,89.20962797)(417.67611531,88.98962819)(417.81611206,88.76963186)
\curveto(417.986115,88.51962866)(418.16611482,88.2796289)(418.35611206,88.04963186)
\curveto(418.54611444,87.82962935)(418.77111421,87.63962954)(419.03111206,87.47963186)
\curveto(419.09111389,87.43962974)(419.15111383,87.40462977)(419.21111206,87.37463186)
\curveto(419.26111372,87.34462983)(419.32611366,87.31462986)(419.40611206,87.28463186)
\curveto(419.47611351,87.26462991)(419.53611345,87.25962992)(419.58611206,87.26963186)
\curveto(419.65611333,87.28962989)(419.71111327,87.32462985)(419.75111206,87.37463186)
\curveto(419.7811132,87.42462975)(419.80111318,87.48462969)(419.81111206,87.55463186)
\lineto(419.81111206,87.79463186)
\lineto(419.81111206,88.54463186)
\lineto(419.81111206,91.34963186)
\lineto(419.81111206,92.00963186)
\curveto(419.81111317,92.09962508)(419.81611317,92.18462499)(419.82611206,92.26463186)
\curveto(419.82611316,92.34462483)(419.84611314,92.40962477)(419.88611206,92.45963186)
\curveto(419.92611306,92.50962467)(420.00111298,92.54962463)(420.11111206,92.57963186)
\curveto(420.21111277,92.61962456)(420.31111267,92.62962455)(420.41111206,92.60963186)
\lineto(420.54611206,92.60963186)
\curveto(420.61611237,92.58962459)(420.67611231,92.56962461)(420.72611206,92.54963186)
\curveto(420.77611221,92.52962465)(420.81611217,92.49462468)(420.84611206,92.44463186)
\curveto(420.8861121,92.39462478)(420.90611208,92.32462485)(420.90611206,92.23463186)
\lineto(420.90611206,91.96463186)
\lineto(420.90611206,91.06463186)
\lineto(420.90611206,87.55463186)
\lineto(420.90611206,86.48963186)
\curveto(420.90611208,86.40963077)(420.91111207,86.31963086)(420.92111206,86.21963186)
\curveto(420.92111206,86.11963106)(420.91111207,86.03463114)(420.89111206,85.96463186)
\curveto(420.82111216,85.75463142)(420.64111234,85.68963149)(420.35111206,85.76963186)
\curveto(420.31111267,85.7796314)(420.27611271,85.7796314)(420.24611206,85.76963186)
\curveto(420.20611278,85.76963141)(420.16111282,85.7796314)(420.11111206,85.79963186)
\curveto(420.03111295,85.81963136)(419.94611304,85.83963134)(419.85611206,85.85963186)
\curveto(419.76611322,85.8796313)(419.6811133,85.90463127)(419.60111206,85.93463186)
\curveto(419.11111387,86.09463108)(418.69611429,86.29463088)(418.35611206,86.53463186)
\curveto(418.10611488,86.71463046)(417.8811151,86.91963026)(417.68111206,87.14963186)
\curveto(417.47111551,87.3796298)(417.27611571,87.61962956)(417.09611206,87.86963186)
\curveto(416.91611607,88.12962905)(416.74611624,88.39462878)(416.58611206,88.66463186)
\curveto(416.41611657,88.94462823)(416.24111674,89.21462796)(416.06111206,89.47463186)
\curveto(415.981117,89.58462759)(415.90611708,89.68962749)(415.83611206,89.78963186)
\curveto(415.76611722,89.89962728)(415.69111729,90.00962717)(415.61111206,90.11963186)
\curveto(415.5811174,90.15962702)(415.55111743,90.19462698)(415.52111206,90.22463186)
\curveto(415.4811175,90.26462691)(415.45111753,90.30462687)(415.43111206,90.34463186)
\curveto(415.32111766,90.48462669)(415.19611779,90.60962657)(415.05611206,90.71963186)
\curveto(415.02611796,90.73962644)(415.00111798,90.76462641)(414.98111206,90.79463186)
\curveto(414.95111803,90.82462635)(414.92111806,90.84962633)(414.89111206,90.86963186)
\curveto(414.79111819,90.94962623)(414.69111829,91.01462616)(414.59111206,91.06463186)
\curveto(414.49111849,91.12462605)(414.3811186,91.179626)(414.26111206,91.22963186)
\curveto(414.19111879,91.25962592)(414.11611887,91.2796259)(414.03611206,91.28963186)
\lineto(413.79611206,91.34963186)
\lineto(413.70611206,91.34963186)
\curveto(413.67611931,91.35962582)(413.64611934,91.36462581)(413.61611206,91.36463186)
\curveto(413.54611944,91.38462579)(413.45111953,91.38962579)(413.33111206,91.37963186)
\curveto(413.20111978,91.3796258)(413.10111988,91.36962581)(413.03111206,91.34963186)
\curveto(412.95112003,91.32962585)(412.87612011,91.30962587)(412.80611206,91.28963186)
\curveto(412.72612026,91.2796259)(412.64612034,91.25962592)(412.56611206,91.22963186)
\curveto(412.32612066,91.11962606)(412.12612086,90.96962621)(411.96611206,90.77963186)
\curveto(411.79612119,90.59962658)(411.65612133,90.3796268)(411.54611206,90.11963186)
\curveto(411.52612146,90.04962713)(411.51112147,89.9796272)(411.50111206,89.90963186)
\curveto(411.4811215,89.83962734)(411.46112152,89.76462741)(411.44111206,89.68463186)
\curveto(411.42112156,89.60462757)(411.41112157,89.49462768)(411.41111206,89.35463186)
\curveto(411.41112157,89.22462795)(411.42112156,89.11962806)(411.44111206,89.03963186)
\curveto(411.45112153,88.9796282)(411.45612153,88.92462825)(411.45611206,88.87463186)
\curveto(411.45612153,88.82462835)(411.46612152,88.7746284)(411.48611206,88.72463186)
\curveto(411.52612146,88.62462855)(411.56612142,88.52962865)(411.60611206,88.43963186)
\curveto(411.64612134,88.35962882)(411.69112129,88.2796289)(411.74111206,88.19963186)
\curveto(411.76112122,88.16962901)(411.7861212,88.13962904)(411.81611206,88.10963186)
\curveto(411.84612114,88.08962909)(411.87112111,88.06462911)(411.89111206,88.03463186)
\lineto(411.96611206,87.95963186)
\curveto(411.986121,87.92962925)(412.00612098,87.90462927)(412.02611206,87.88463186)
\lineto(412.23611206,87.73463186)
\curveto(412.29612069,87.69462948)(412.36112062,87.64962953)(412.43111206,87.59963186)
\curveto(412.52112046,87.53962964)(412.62612036,87.48962969)(412.74611206,87.44963186)
\curveto(412.85612013,87.41962976)(412.96612002,87.38462979)(413.07611206,87.34463186)
\curveto(413.1861198,87.30462987)(413.33111965,87.2796299)(413.51111206,87.26963186)
\curveto(413.6811193,87.25962992)(413.80611918,87.22962995)(413.88611206,87.17963186)
\curveto(413.96611902,87.12963005)(414.01111897,87.05463012)(414.02111206,86.95463186)
\curveto(414.03111895,86.85463032)(414.03611895,86.74463043)(414.03611206,86.62463186)
\curveto(414.03611895,86.58463059)(414.04111894,86.54463063)(414.05111206,86.50463186)
\curveto(414.05111893,86.46463071)(414.04611894,86.42963075)(414.03611206,86.39963186)
\curveto(414.01611897,86.34963083)(414.00611898,86.29963088)(414.00611206,86.24963186)
\curveto(414.00611898,86.20963097)(413.99611899,86.16963101)(413.97611206,86.12963186)
\curveto(413.91611907,86.03963114)(413.7811192,85.99463118)(413.57111206,85.99463186)
\lineto(413.45111206,85.99463186)
\curveto(413.39111959,86.00463117)(413.33111965,86.00963117)(413.27111206,86.00963186)
\curveto(413.20111978,86.01963116)(413.13611985,86.02963115)(413.07611206,86.03963186)
\curveto(412.96612002,86.05963112)(412.86612012,86.0796311)(412.77611206,86.09963186)
\curveto(412.67612031,86.11963106)(412.5811204,86.14963103)(412.49111206,86.18963186)
\curveto(412.42112056,86.20963097)(412.36112062,86.22963095)(412.31111206,86.24963186)
\lineto(412.13111206,86.30963186)
\curveto(411.87112111,86.42963075)(411.62612136,86.58463059)(411.39611206,86.77463186)
\curveto(411.16612182,86.9746302)(410.981122,87.18962999)(410.84111206,87.41963186)
\curveto(410.76112222,87.52962965)(410.69612229,87.64462953)(410.64611206,87.76463186)
\lineto(410.49611206,88.15463186)
\curveto(410.44612254,88.26462891)(410.41612257,88.3796288)(410.40611206,88.49963186)
\curveto(410.3861226,88.61962856)(410.36112262,88.74462843)(410.33111206,88.87463186)
\curveto(410.33112265,88.94462823)(410.33112265,89.00962817)(410.33111206,89.06963186)
\curveto(410.32112266,89.12962805)(410.31112267,89.19462798)(410.30111206,89.26463186)
}
}
{
\newrgbcolor{curcolor}{0 0 0}
\pscustom[linestyle=none,fillstyle=solid,fillcolor=curcolor]
{
\newpath
\moveto(415.82111206,101.36424123)
\lineto(416.07611206,101.36424123)
\curveto(416.15611683,101.37423353)(416.23111675,101.36923353)(416.30111206,101.34924123)
\lineto(416.54111206,101.34924123)
\lineto(416.70611206,101.34924123)
\curveto(416.80611618,101.32923357)(416.91111607,101.31923358)(417.02111206,101.31924123)
\curveto(417.12111586,101.31923358)(417.22111576,101.30923359)(417.32111206,101.28924123)
\lineto(417.47111206,101.28924123)
\curveto(417.61111537,101.25923364)(417.75111523,101.23923366)(417.89111206,101.22924123)
\curveto(418.02111496,101.21923368)(418.15111483,101.19423371)(418.28111206,101.15424123)
\curveto(418.36111462,101.13423377)(418.44611454,101.11423379)(418.53611206,101.09424123)
\lineto(418.77611206,101.03424123)
\lineto(419.07611206,100.91424123)
\curveto(419.16611382,100.88423402)(419.25611373,100.84923405)(419.34611206,100.80924123)
\curveto(419.56611342,100.70923419)(419.7811132,100.57423433)(419.99111206,100.40424123)
\curveto(420.20111278,100.24423466)(420.37111261,100.06923483)(420.50111206,99.87924123)
\curveto(420.54111244,99.82923507)(420.5811124,99.76923513)(420.62111206,99.69924123)
\curveto(420.65111233,99.63923526)(420.6861123,99.57923532)(420.72611206,99.51924123)
\curveto(420.77611221,99.43923546)(420.81611217,99.34423556)(420.84611206,99.23424123)
\curveto(420.87611211,99.12423578)(420.90611208,99.01923588)(420.93611206,98.91924123)
\curveto(420.97611201,98.80923609)(421.00111198,98.6992362)(421.01111206,98.58924123)
\curveto(421.02111196,98.47923642)(421.03611195,98.36423654)(421.05611206,98.24424123)
\curveto(421.06611192,98.2042367)(421.06611192,98.15923674)(421.05611206,98.10924123)
\curveto(421.05611193,98.06923683)(421.06111192,98.02923687)(421.07111206,97.98924123)
\curveto(421.0811119,97.94923695)(421.0861119,97.89423701)(421.08611206,97.82424123)
\curveto(421.0861119,97.75423715)(421.0811119,97.7042372)(421.07111206,97.67424123)
\curveto(421.05111193,97.62423728)(421.04611194,97.57923732)(421.05611206,97.53924123)
\curveto(421.06611192,97.4992374)(421.06611192,97.46423744)(421.05611206,97.43424123)
\lineto(421.05611206,97.34424123)
\curveto(421.03611195,97.28423762)(421.02111196,97.21923768)(421.01111206,97.14924123)
\curveto(421.01111197,97.08923781)(421.00611198,97.02423788)(420.99611206,96.95424123)
\curveto(420.94611204,96.78423812)(420.89611209,96.62423828)(420.84611206,96.47424123)
\curveto(420.79611219,96.32423858)(420.73111225,96.17923872)(420.65111206,96.03924123)
\curveto(420.61111237,95.98923891)(420.5811124,95.93423897)(420.56111206,95.87424123)
\curveto(420.53111245,95.82423908)(420.49611249,95.77423913)(420.45611206,95.72424123)
\curveto(420.27611271,95.48423942)(420.05611293,95.28423962)(419.79611206,95.12424123)
\curveto(419.53611345,94.96423994)(419.25111373,94.82424008)(418.94111206,94.70424123)
\curveto(418.80111418,94.64424026)(418.66111432,94.5992403)(418.52111206,94.56924123)
\curveto(418.37111461,94.53924036)(418.21611477,94.5042404)(418.05611206,94.46424123)
\curveto(417.94611504,94.44424046)(417.83611515,94.42924047)(417.72611206,94.41924123)
\curveto(417.61611537,94.40924049)(417.50611548,94.39424051)(417.39611206,94.37424123)
\curveto(417.35611563,94.36424054)(417.31611567,94.35924054)(417.27611206,94.35924123)
\curveto(417.23611575,94.36924053)(417.19611579,94.36924053)(417.15611206,94.35924123)
\curveto(417.10611588,94.34924055)(417.05611593,94.34424056)(417.00611206,94.34424123)
\lineto(416.84111206,94.34424123)
\curveto(416.79111619,94.32424058)(416.74111624,94.31924058)(416.69111206,94.32924123)
\curveto(416.63111635,94.33924056)(416.57611641,94.33924056)(416.52611206,94.32924123)
\curveto(416.4861165,94.31924058)(416.44111654,94.31924058)(416.39111206,94.32924123)
\curveto(416.34111664,94.33924056)(416.29111669,94.33424057)(416.24111206,94.31424123)
\curveto(416.17111681,94.29424061)(416.09611689,94.28924061)(416.01611206,94.29924123)
\curveto(415.92611706,94.30924059)(415.84111714,94.31424059)(415.76111206,94.31424123)
\curveto(415.67111731,94.31424059)(415.57111741,94.30924059)(415.46111206,94.29924123)
\curveto(415.34111764,94.28924061)(415.24111774,94.29424061)(415.16111206,94.31424123)
\lineto(414.87611206,94.31424123)
\lineto(414.24611206,94.35924123)
\curveto(414.14611884,94.36924053)(414.05111893,94.37924052)(413.96111206,94.38924123)
\lineto(413.66111206,94.41924123)
\curveto(413.61111937,94.43924046)(413.56111942,94.44424046)(413.51111206,94.43424123)
\curveto(413.45111953,94.43424047)(413.39611959,94.44424046)(413.34611206,94.46424123)
\curveto(413.17611981,94.51424039)(413.01111997,94.55424035)(412.85111206,94.58424123)
\curveto(412.6811203,94.61424029)(412.52112046,94.66424024)(412.37111206,94.73424123)
\curveto(411.91112107,94.92423998)(411.53612145,95.14423976)(411.24611206,95.39424123)
\curveto(410.95612203,95.65423925)(410.71112227,96.01423889)(410.51111206,96.47424123)
\curveto(410.46112252,96.6042383)(410.42612256,96.73423817)(410.40611206,96.86424123)
\curveto(410.3861226,97.0042379)(410.36112262,97.14423776)(410.33111206,97.28424123)
\curveto(410.32112266,97.35423755)(410.31612267,97.41923748)(410.31611206,97.47924123)
\curveto(410.31612267,97.53923736)(410.31112267,97.6042373)(410.30111206,97.67424123)
\curveto(410.2811227,98.5042364)(410.43112255,99.17423573)(410.75111206,99.68424123)
\curveto(411.06112192,100.19423471)(411.50112148,100.57423433)(412.07111206,100.82424123)
\curveto(412.19112079,100.87423403)(412.31612067,100.91923398)(412.44611206,100.95924123)
\curveto(412.57612041,100.9992339)(412.71112027,101.04423386)(412.85111206,101.09424123)
\curveto(412.93112005,101.11423379)(413.01611997,101.12923377)(413.10611206,101.13924123)
\lineto(413.34611206,101.19924123)
\curveto(413.45611953,101.22923367)(413.56611942,101.24423366)(413.67611206,101.24424123)
\curveto(413.7861192,101.25423365)(413.89611909,101.26923363)(414.00611206,101.28924123)
\curveto(414.05611893,101.30923359)(414.10111888,101.31423359)(414.14111206,101.30424123)
\curveto(414.1811188,101.3042336)(414.22111876,101.30923359)(414.26111206,101.31924123)
\curveto(414.31111867,101.32923357)(414.36611862,101.32923357)(414.42611206,101.31924123)
\curveto(414.47611851,101.31923358)(414.52611846,101.32423358)(414.57611206,101.33424123)
\lineto(414.71111206,101.33424123)
\curveto(414.77111821,101.35423355)(414.84111814,101.35423355)(414.92111206,101.33424123)
\curveto(414.99111799,101.32423358)(415.05611793,101.32923357)(415.11611206,101.34924123)
\curveto(415.14611784,101.35923354)(415.1861178,101.36423354)(415.23611206,101.36424123)
\lineto(415.35611206,101.36424123)
\lineto(415.82111206,101.36424123)
\moveto(418.14611206,99.81924123)
\curveto(417.82611516,99.91923498)(417.46111552,99.97923492)(417.05111206,99.99924123)
\curveto(416.64111634,100.01923488)(416.23111675,100.02923487)(415.82111206,100.02924123)
\curveto(415.39111759,100.02923487)(414.97111801,100.01923488)(414.56111206,99.99924123)
\curveto(414.15111883,99.97923492)(413.76611922,99.93423497)(413.40611206,99.86424123)
\curveto(413.04611994,99.79423511)(412.72612026,99.68423522)(412.44611206,99.53424123)
\curveto(412.15612083,99.39423551)(411.92112106,99.1992357)(411.74111206,98.94924123)
\curveto(411.63112135,98.78923611)(411.55112143,98.60923629)(411.50111206,98.40924123)
\curveto(411.44112154,98.20923669)(411.41112157,97.96423694)(411.41111206,97.67424123)
\curveto(411.43112155,97.65423725)(411.44112154,97.61923728)(411.44111206,97.56924123)
\curveto(411.43112155,97.51923738)(411.43112155,97.47923742)(411.44111206,97.44924123)
\curveto(411.46112152,97.36923753)(411.4811215,97.29423761)(411.50111206,97.22424123)
\curveto(411.51112147,97.16423774)(411.53112145,97.0992378)(411.56111206,97.02924123)
\curveto(411.6811213,96.75923814)(411.85112113,96.53923836)(412.07111206,96.36924123)
\curveto(412.2811207,96.20923869)(412.52612046,96.07423883)(412.80611206,95.96424123)
\curveto(412.91612007,95.91423899)(413.03611995,95.87423903)(413.16611206,95.84424123)
\curveto(413.2861197,95.82423908)(413.41111957,95.7992391)(413.54111206,95.76924123)
\curveto(413.59111939,95.74923915)(413.64611934,95.73923916)(413.70611206,95.73924123)
\curveto(413.75611923,95.73923916)(413.80611918,95.73423917)(413.85611206,95.72424123)
\curveto(413.94611904,95.71423919)(414.04111894,95.7042392)(414.14111206,95.69424123)
\curveto(414.23111875,95.68423922)(414.32611866,95.67423923)(414.42611206,95.66424123)
\curveto(414.50611848,95.66423924)(414.59111839,95.65923924)(414.68111206,95.64924123)
\lineto(414.92111206,95.64924123)
\lineto(415.10111206,95.64924123)
\curveto(415.13111785,95.63923926)(415.16611782,95.63423927)(415.20611206,95.63424123)
\lineto(415.34111206,95.63424123)
\lineto(415.79111206,95.63424123)
\curveto(415.87111711,95.63423927)(415.95611703,95.62923927)(416.04611206,95.61924123)
\curveto(416.12611686,95.61923928)(416.20111678,95.62923927)(416.27111206,95.64924123)
\lineto(416.54111206,95.64924123)
\curveto(416.56111642,95.64923925)(416.59111639,95.64423926)(416.63111206,95.63424123)
\curveto(416.66111632,95.63423927)(416.6861163,95.63923926)(416.70611206,95.64924123)
\curveto(416.80611618,95.65923924)(416.90611608,95.66423924)(417.00611206,95.66424123)
\curveto(417.09611589,95.67423923)(417.19611579,95.68423922)(417.30611206,95.69424123)
\curveto(417.42611556,95.72423918)(417.55111543,95.73923916)(417.68111206,95.73924123)
\curveto(417.80111518,95.74923915)(417.91611507,95.77423913)(418.02611206,95.81424123)
\curveto(418.32611466,95.89423901)(418.59111439,95.97923892)(418.82111206,96.06924123)
\curveto(419.05111393,96.16923873)(419.26611372,96.31423859)(419.46611206,96.50424123)
\curveto(419.66611332,96.71423819)(419.81611317,96.97923792)(419.91611206,97.29924123)
\curveto(419.93611305,97.33923756)(419.94611304,97.37423753)(419.94611206,97.40424123)
\curveto(419.93611305,97.44423746)(419.94111304,97.48923741)(419.96111206,97.53924123)
\curveto(419.97111301,97.57923732)(419.981113,97.64923725)(419.99111206,97.74924123)
\curveto(420.00111298,97.85923704)(419.99611299,97.94423696)(419.97611206,98.00424123)
\curveto(419.95611303,98.07423683)(419.94611304,98.14423676)(419.94611206,98.21424123)
\curveto(419.93611305,98.28423662)(419.92111306,98.34923655)(419.90111206,98.40924123)
\curveto(419.84111314,98.60923629)(419.75611323,98.78923611)(419.64611206,98.94924123)
\curveto(419.62611336,98.97923592)(419.60611338,99.0042359)(419.58611206,99.02424123)
\lineto(419.52611206,99.08424123)
\curveto(419.50611348,99.12423578)(419.46611352,99.17423573)(419.40611206,99.23424123)
\curveto(419.26611372,99.33423557)(419.13611385,99.41923548)(419.01611206,99.48924123)
\curveto(418.89611409,99.55923534)(418.75111423,99.62923527)(418.58111206,99.69924123)
\curveto(418.51111447,99.72923517)(418.44111454,99.74923515)(418.37111206,99.75924123)
\curveto(418.30111468,99.77923512)(418.22611476,99.7992351)(418.14611206,99.81924123)
}
}
{
\newrgbcolor{curcolor}{0 0 0}
\pscustom[linestyle=none,fillstyle=solid,fillcolor=curcolor]
{
\newpath
\moveto(410.30111206,106.77385061)
\curveto(410.30112268,106.87384575)(410.31112267,106.96884566)(410.33111206,107.05885061)
\curveto(410.34112264,107.14884548)(410.37112261,107.21384541)(410.42111206,107.25385061)
\curveto(410.50112248,107.31384531)(410.60612238,107.34384528)(410.73611206,107.34385061)
\lineto(411.12611206,107.34385061)
\lineto(412.62611206,107.34385061)
\lineto(419.01611206,107.34385061)
\lineto(420.18611206,107.34385061)
\lineto(420.50111206,107.34385061)
\curveto(420.60111238,107.35384527)(420.6811123,107.33884529)(420.74111206,107.29885061)
\curveto(420.82111216,107.24884538)(420.87111211,107.17384545)(420.89111206,107.07385061)
\curveto(420.90111208,106.98384564)(420.90611208,106.87384575)(420.90611206,106.74385061)
\lineto(420.90611206,106.51885061)
\curveto(420.8861121,106.43884619)(420.87111211,106.36884626)(420.86111206,106.30885061)
\curveto(420.84111214,106.24884638)(420.80111218,106.19884643)(420.74111206,106.15885061)
\curveto(420.6811123,106.11884651)(420.60611238,106.09884653)(420.51611206,106.09885061)
\lineto(420.21611206,106.09885061)
\lineto(419.12111206,106.09885061)
\lineto(413.78111206,106.09885061)
\curveto(413.69111929,106.07884655)(413.61611937,106.06384656)(413.55611206,106.05385061)
\curveto(413.4861195,106.05384657)(413.42611956,106.0238466)(413.37611206,105.96385061)
\curveto(413.32611966,105.89384673)(413.30111968,105.80384682)(413.30111206,105.69385061)
\curveto(413.29111969,105.59384703)(413.2861197,105.48384714)(413.28611206,105.36385061)
\lineto(413.28611206,104.22385061)
\lineto(413.28611206,103.72885061)
\curveto(413.27611971,103.56884906)(413.21611977,103.45884917)(413.10611206,103.39885061)
\curveto(413.07611991,103.37884925)(413.04611994,103.36884926)(413.01611206,103.36885061)
\curveto(412.97612001,103.36884926)(412.93112005,103.36384926)(412.88111206,103.35385061)
\curveto(412.76112022,103.33384929)(412.65112033,103.33884929)(412.55111206,103.36885061)
\curveto(412.45112053,103.40884922)(412.3811206,103.46384916)(412.34111206,103.53385061)
\curveto(412.29112069,103.61384901)(412.26612072,103.73384889)(412.26611206,103.89385061)
\curveto(412.26612072,104.05384857)(412.25112073,104.18884844)(412.22111206,104.29885061)
\curveto(412.21112077,104.34884828)(412.20612078,104.40384822)(412.20611206,104.46385061)
\curveto(412.19612079,104.5238481)(412.1811208,104.58384804)(412.16111206,104.64385061)
\curveto(412.11112087,104.79384783)(412.06112092,104.93884769)(412.01111206,105.07885061)
\curveto(411.95112103,105.21884741)(411.8811211,105.35384727)(411.80111206,105.48385061)
\curveto(411.71112127,105.623847)(411.60612138,105.74384688)(411.48611206,105.84385061)
\curveto(411.36612162,105.94384668)(411.23612175,106.03884659)(411.09611206,106.12885061)
\curveto(410.99612199,106.18884644)(410.8861221,106.23384639)(410.76611206,106.26385061)
\curveto(410.64612234,106.30384632)(410.54112244,106.35384627)(410.45111206,106.41385061)
\curveto(410.39112259,106.46384616)(410.35112263,106.53384609)(410.33111206,106.62385061)
\curveto(410.32112266,106.64384598)(410.31612267,106.66884596)(410.31611206,106.69885061)
\curveto(410.31612267,106.7288459)(410.31112267,106.75384587)(410.30111206,106.77385061)
}
}
{
\newrgbcolor{curcolor}{0 0 0}
\pscustom[linestyle=none,fillstyle=solid,fillcolor=curcolor]
{
\newpath
\moveto(410.30111206,115.12345998)
\curveto(410.30112268,115.22345513)(410.31112267,115.31845503)(410.33111206,115.40845998)
\curveto(410.34112264,115.49845485)(410.37112261,115.56345479)(410.42111206,115.60345998)
\curveto(410.50112248,115.66345469)(410.60612238,115.69345466)(410.73611206,115.69345998)
\lineto(411.12611206,115.69345998)
\lineto(412.62611206,115.69345998)
\lineto(419.01611206,115.69345998)
\lineto(420.18611206,115.69345998)
\lineto(420.50111206,115.69345998)
\curveto(420.60111238,115.70345465)(420.6811123,115.68845466)(420.74111206,115.64845998)
\curveto(420.82111216,115.59845475)(420.87111211,115.52345483)(420.89111206,115.42345998)
\curveto(420.90111208,115.33345502)(420.90611208,115.22345513)(420.90611206,115.09345998)
\lineto(420.90611206,114.86845998)
\curveto(420.8861121,114.78845556)(420.87111211,114.71845563)(420.86111206,114.65845998)
\curveto(420.84111214,114.59845575)(420.80111218,114.5484558)(420.74111206,114.50845998)
\curveto(420.6811123,114.46845588)(420.60611238,114.4484559)(420.51611206,114.44845998)
\lineto(420.21611206,114.44845998)
\lineto(419.12111206,114.44845998)
\lineto(413.78111206,114.44845998)
\curveto(413.69111929,114.42845592)(413.61611937,114.41345594)(413.55611206,114.40345998)
\curveto(413.4861195,114.40345595)(413.42611956,114.37345598)(413.37611206,114.31345998)
\curveto(413.32611966,114.24345611)(413.30111968,114.1534562)(413.30111206,114.04345998)
\curveto(413.29111969,113.94345641)(413.2861197,113.83345652)(413.28611206,113.71345998)
\lineto(413.28611206,112.57345998)
\lineto(413.28611206,112.07845998)
\curveto(413.27611971,111.91845843)(413.21611977,111.80845854)(413.10611206,111.74845998)
\curveto(413.07611991,111.72845862)(413.04611994,111.71845863)(413.01611206,111.71845998)
\curveto(412.97612001,111.71845863)(412.93112005,111.71345864)(412.88111206,111.70345998)
\curveto(412.76112022,111.68345867)(412.65112033,111.68845866)(412.55111206,111.71845998)
\curveto(412.45112053,111.75845859)(412.3811206,111.81345854)(412.34111206,111.88345998)
\curveto(412.29112069,111.96345839)(412.26612072,112.08345827)(412.26611206,112.24345998)
\curveto(412.26612072,112.40345795)(412.25112073,112.53845781)(412.22111206,112.64845998)
\curveto(412.21112077,112.69845765)(412.20612078,112.7534576)(412.20611206,112.81345998)
\curveto(412.19612079,112.87345748)(412.1811208,112.93345742)(412.16111206,112.99345998)
\curveto(412.11112087,113.14345721)(412.06112092,113.28845706)(412.01111206,113.42845998)
\curveto(411.95112103,113.56845678)(411.8811211,113.70345665)(411.80111206,113.83345998)
\curveto(411.71112127,113.97345638)(411.60612138,114.09345626)(411.48611206,114.19345998)
\curveto(411.36612162,114.29345606)(411.23612175,114.38845596)(411.09611206,114.47845998)
\curveto(410.99612199,114.53845581)(410.8861221,114.58345577)(410.76611206,114.61345998)
\curveto(410.64612234,114.6534557)(410.54112244,114.70345565)(410.45111206,114.76345998)
\curveto(410.39112259,114.81345554)(410.35112263,114.88345547)(410.33111206,114.97345998)
\curveto(410.32112266,114.99345536)(410.31612267,115.01845533)(410.31611206,115.04845998)
\curveto(410.31612267,115.07845527)(410.31112267,115.10345525)(410.30111206,115.12345998)
}
}
{
\newrgbcolor{curcolor}{0 0 0}
\pscustom[linestyle=none,fillstyle=solid,fillcolor=curcolor]
{
\newpath
\moveto(441.0374585,42.02236623)
\curveto(441.08745924,42.04235669)(441.14745918,42.06735666)(441.2174585,42.09736623)
\curveto(441.28745904,42.1273566)(441.36245897,42.14735658)(441.4424585,42.15736623)
\curveto(441.51245882,42.17735655)(441.58245875,42.17735655)(441.6524585,42.15736623)
\curveto(441.71245862,42.14735658)(441.75745857,42.10735662)(441.7874585,42.03736623)
\curveto(441.80745852,41.98735674)(441.81745851,41.9273568)(441.8174585,41.85736623)
\lineto(441.8174585,41.64736623)
\lineto(441.8174585,41.19736623)
\curveto(441.81745851,41.04735768)(441.79245854,40.9273578)(441.7424585,40.83736623)
\curveto(441.68245865,40.73735799)(441.57745875,40.66235807)(441.4274585,40.61236623)
\curveto(441.27745905,40.57235816)(441.14245919,40.5273582)(441.0224585,40.47736623)
\curveto(440.76245957,40.36735836)(440.49245984,40.26735846)(440.2124585,40.17736623)
\curveto(439.9324604,40.08735864)(439.65746067,39.98735874)(439.3874585,39.87736623)
\curveto(439.29746103,39.84735888)(439.21246112,39.81735891)(439.1324585,39.78736623)
\curveto(439.05246128,39.76735896)(438.97746135,39.73735899)(438.9074585,39.69736623)
\curveto(438.83746149,39.66735906)(438.77746155,39.62235911)(438.7274585,39.56236623)
\curveto(438.67746165,39.50235923)(438.63746169,39.42235931)(438.6074585,39.32236623)
\curveto(438.58746174,39.27235946)(438.58246175,39.21235952)(438.5924585,39.14236623)
\lineto(438.5924585,38.94736623)
\lineto(438.5924585,36.11236623)
\lineto(438.5924585,35.81236623)
\curveto(438.58246175,35.70236303)(438.58246175,35.59736313)(438.5924585,35.49736623)
\curveto(438.60246173,35.39736333)(438.61746171,35.30236343)(438.6374585,35.21236623)
\curveto(438.65746167,35.1323636)(438.69746163,35.07236366)(438.7574585,35.03236623)
\curveto(438.85746147,34.95236378)(438.97246136,34.89236384)(439.1024585,34.85236623)
\curveto(439.22246111,34.82236391)(439.34746098,34.78236395)(439.4774585,34.73236623)
\curveto(439.70746062,34.6323641)(439.94746038,34.53736419)(440.1974585,34.44736623)
\curveto(440.44745988,34.36736436)(440.68745964,34.27736445)(440.9174585,34.17736623)
\curveto(440.97745935,34.15736457)(441.04745928,34.1323646)(441.1274585,34.10236623)
\curveto(441.19745913,34.08236465)(441.27245906,34.05736467)(441.3524585,34.02736623)
\curveto(441.4324589,33.99736473)(441.50745882,33.96236477)(441.5774585,33.92236623)
\curveto(441.63745869,33.89236484)(441.68245865,33.85736487)(441.7124585,33.81736623)
\curveto(441.77245856,33.73736499)(441.80745852,33.6273651)(441.8174585,33.48736623)
\lineto(441.8174585,33.06736623)
\lineto(441.8174585,32.82736623)
\curveto(441.80745852,32.75736597)(441.78245855,32.69736603)(441.7424585,32.64736623)
\curveto(441.71245862,32.59736613)(441.66745866,32.56736616)(441.6074585,32.55736623)
\curveto(441.54745878,32.55736617)(441.48745884,32.56236617)(441.4274585,32.57236623)
\curveto(441.35745897,32.59236614)(441.29245904,32.61236612)(441.2324585,32.63236623)
\curveto(441.16245917,32.66236607)(441.11245922,32.68736604)(441.0824585,32.70736623)
\curveto(440.76245957,32.84736588)(440.44745988,32.97236576)(440.1374585,33.08236623)
\curveto(439.81746051,33.19236554)(439.49746083,33.31236542)(439.1774585,33.44236623)
\curveto(438.95746137,33.5323652)(438.74246159,33.61736511)(438.5324585,33.69736623)
\curveto(438.31246202,33.77736495)(438.09246224,33.86236487)(437.8724585,33.95236623)
\curveto(437.15246318,34.25236448)(436.4274639,34.53736419)(435.6974585,34.80736623)
\curveto(434.95746537,35.07736365)(434.22246611,35.36236337)(433.4924585,35.66236623)
\curveto(433.2324671,35.77236296)(432.96746736,35.87236286)(432.6974585,35.96236623)
\curveto(432.4274679,36.06236267)(432.16246817,36.16736256)(431.9024585,36.27736623)
\curveto(431.79246854,36.3273624)(431.67246866,36.37236236)(431.5424585,36.41236623)
\curveto(431.40246893,36.46236227)(431.30246903,36.5323622)(431.2424585,36.62236623)
\curveto(431.20246913,36.66236207)(431.17246916,36.727362)(431.1524585,36.81736623)
\curveto(431.14246919,36.83736189)(431.14246919,36.85736187)(431.1524585,36.87736623)
\curveto(431.15246918,36.90736182)(431.14746918,36.9323618)(431.1374585,36.95236623)
\curveto(431.13746919,37.1323616)(431.13746919,37.34236139)(431.1374585,37.58236623)
\curveto(431.1274692,37.82236091)(431.16246917,37.99736073)(431.2424585,38.10736623)
\curveto(431.30246903,38.18736054)(431.40246893,38.24736048)(431.5424585,38.28736623)
\curveto(431.67246866,38.33736039)(431.79246854,38.38736034)(431.9024585,38.43736623)
\curveto(432.1324682,38.53736019)(432.36246797,38.6273601)(432.5924585,38.70736623)
\curveto(432.82246751,38.78735994)(433.05246728,38.87735985)(433.2824585,38.97736623)
\curveto(433.48246685,39.05735967)(433.68746664,39.1323596)(433.8974585,39.20236623)
\curveto(434.10746622,39.28235945)(434.31246602,39.36735936)(434.5124585,39.45736623)
\curveto(435.24246509,39.75735897)(435.98246435,40.04235869)(436.7324585,40.31236623)
\curveto(437.47246286,40.59235814)(438.20746212,40.88735784)(438.9374585,41.19736623)
\curveto(439.0274613,41.23735749)(439.11246122,41.26735746)(439.1924585,41.28736623)
\curveto(439.27246106,41.31735741)(439.35746097,41.34735738)(439.4474585,41.37736623)
\curveto(439.70746062,41.48735724)(439.97246036,41.59235714)(440.2424585,41.69236623)
\curveto(440.51245982,41.80235693)(440.77745955,41.91235682)(441.0374585,42.02236623)
\moveto(437.3924585,38.81236623)
\curveto(437.36246297,38.90235983)(437.31246302,38.95735977)(437.2424585,38.97736623)
\curveto(437.17246316,39.00735972)(437.09746323,39.01235972)(437.0174585,38.99236623)
\curveto(436.9274634,38.98235975)(436.84246349,38.95735977)(436.7624585,38.91736623)
\curveto(436.67246366,38.88735984)(436.59746373,38.85735987)(436.5374585,38.82736623)
\curveto(436.49746383,38.80735992)(436.46246387,38.79735993)(436.4324585,38.79736623)
\curveto(436.40246393,38.79735993)(436.36746396,38.78735994)(436.3274585,38.76736623)
\lineto(436.0874585,38.67736623)
\curveto(435.99746433,38.65736007)(435.90746442,38.6273601)(435.8174585,38.58736623)
\curveto(435.45746487,38.43736029)(435.09246524,38.30236043)(434.7224585,38.18236623)
\curveto(434.34246599,38.07236066)(433.97246636,37.94236079)(433.6124585,37.79236623)
\curveto(433.50246683,37.74236099)(433.39246694,37.69736103)(433.2824585,37.65736623)
\curveto(433.17246716,37.6273611)(433.06746726,37.58736114)(432.9674585,37.53736623)
\curveto(432.91746741,37.51736121)(432.87246746,37.49236124)(432.8324585,37.46236623)
\curveto(432.78246755,37.44236129)(432.75746757,37.39236134)(432.7574585,37.31236623)
\curveto(432.77746755,37.29236144)(432.79246754,37.27236146)(432.8024585,37.25236623)
\curveto(432.81246752,37.2323615)(432.8274675,37.21236152)(432.8474585,37.19236623)
\curveto(432.89746743,37.15236158)(432.95246738,37.12236161)(433.0124585,37.10236623)
\curveto(433.06246727,37.08236165)(433.11746721,37.06236167)(433.1774585,37.04236623)
\curveto(433.28746704,36.99236174)(433.39746693,36.95236178)(433.5074585,36.92236623)
\curveto(433.61746671,36.89236184)(433.7274666,36.85236188)(433.8374585,36.80236623)
\curveto(434.2274661,36.6323621)(434.62246571,36.48236225)(435.0224585,36.35236623)
\curveto(435.42246491,36.2323625)(435.81246452,36.09236264)(436.1924585,35.93236623)
\lineto(436.3424585,35.87236623)
\curveto(436.39246394,35.86236287)(436.44246389,35.84736288)(436.4924585,35.82736623)
\lineto(436.7324585,35.73736623)
\curveto(436.81246352,35.70736302)(436.89246344,35.68236305)(436.9724585,35.66236623)
\curveto(437.02246331,35.64236309)(437.07746325,35.6323631)(437.1374585,35.63236623)
\curveto(437.19746313,35.64236309)(437.24746308,35.65736307)(437.2874585,35.67736623)
\curveto(437.36746296,35.727363)(437.41246292,35.8323629)(437.4224585,35.99236623)
\lineto(437.4224585,36.44236623)
\lineto(437.4224585,38.04736623)
\curveto(437.42246291,38.15736057)(437.4274629,38.29236044)(437.4374585,38.45236623)
\curveto(437.43746289,38.61236012)(437.42246291,38.73236)(437.3924585,38.81236623)
}
}
{
\newrgbcolor{curcolor}{0 0 0}
\pscustom[linestyle=none,fillstyle=solid,fillcolor=curcolor]
{
\newpath
\moveto(437.7824585,50.56392873)
\curveto(437.8324625,50.57392038)(437.90246243,50.57892038)(437.9924585,50.57892873)
\curveto(438.07246226,50.57892038)(438.13746219,50.57392038)(438.1874585,50.56392873)
\curveto(438.2274621,50.56392039)(438.26746206,50.5589204)(438.3074585,50.54892873)
\lineto(438.4274585,50.54892873)
\curveto(438.50746182,50.52892043)(438.58746174,50.51892044)(438.6674585,50.51892873)
\curveto(438.74746158,50.51892044)(438.8274615,50.50892045)(438.9074585,50.48892873)
\curveto(438.94746138,50.47892048)(438.98746134,50.47392048)(439.0274585,50.47392873)
\curveto(439.05746127,50.47392048)(439.09246124,50.46892049)(439.1324585,50.45892873)
\curveto(439.24246109,50.42892053)(439.34746098,50.39892056)(439.4474585,50.36892873)
\curveto(439.54746078,50.34892061)(439.64746068,50.31892064)(439.7474585,50.27892873)
\curveto(440.09746023,50.13892082)(440.41245992,49.96892099)(440.6924585,49.76892873)
\curveto(440.97245936,49.56892139)(441.21245912,49.31892164)(441.4124585,49.01892873)
\curveto(441.51245882,48.86892209)(441.59745873,48.72392223)(441.6674585,48.58392873)
\curveto(441.71745861,48.47392248)(441.75745857,48.36392259)(441.7874585,48.25392873)
\curveto(441.81745851,48.1539228)(441.84745848,48.04892291)(441.8774585,47.93892873)
\curveto(441.89745843,47.86892309)(441.90745842,47.80392315)(441.9074585,47.74392873)
\curveto(441.91745841,47.68392327)(441.9324584,47.62392333)(441.9524585,47.56392873)
\lineto(441.9524585,47.41392873)
\curveto(441.97245836,47.36392359)(441.98245835,47.28892367)(441.9824585,47.18892873)
\curveto(441.99245834,47.08892387)(441.98745834,47.00892395)(441.9674585,46.94892873)
\lineto(441.9674585,46.79892873)
\curveto(441.95745837,46.7589242)(441.95245838,46.71392424)(441.9524585,46.66392873)
\curveto(441.95245838,46.62392433)(441.94745838,46.57892438)(441.9374585,46.52892873)
\curveto(441.89745843,46.37892458)(441.86245847,46.22892473)(441.8324585,46.07892873)
\curveto(441.80245853,45.93892502)(441.75745857,45.79892516)(441.6974585,45.65892873)
\curveto(441.61745871,45.4589255)(441.51745881,45.27892568)(441.3974585,45.11892873)
\lineto(441.2474585,44.93892873)
\curveto(441.18745914,44.87892608)(441.14745918,44.80892615)(441.1274585,44.72892873)
\curveto(441.11745921,44.66892629)(441.1324592,44.61892634)(441.1724585,44.57892873)
\curveto(441.20245913,44.54892641)(441.24745908,44.52392643)(441.3074585,44.50392873)
\curveto(441.36745896,44.49392646)(441.4324589,44.48392647)(441.5024585,44.47392873)
\curveto(441.56245877,44.47392648)(441.60745872,44.46392649)(441.6374585,44.44392873)
\curveto(441.68745864,44.40392655)(441.7324586,44.3589266)(441.7724585,44.30892873)
\curveto(441.79245854,44.2589267)(441.80745852,44.18892677)(441.8174585,44.09892873)
\lineto(441.8174585,43.82892873)
\curveto(441.81745851,43.73892722)(441.81245852,43.6539273)(441.8024585,43.57392873)
\curveto(441.78245855,43.49392746)(441.76245857,43.43392752)(441.7424585,43.39392873)
\curveto(441.72245861,43.37392758)(441.69745863,43.3539276)(441.6674585,43.33392873)
\lineto(441.5774585,43.27392873)
\curveto(441.49745883,43.24392771)(441.37745895,43.22892773)(441.2174585,43.22892873)
\curveto(441.05745927,43.23892772)(440.92245941,43.24392771)(440.8124585,43.24392873)
\lineto(432.0074585,43.24392873)
\curveto(431.88746844,43.24392771)(431.76246857,43.23892772)(431.6324585,43.22892873)
\curveto(431.49246884,43.22892773)(431.38246895,43.2539277)(431.3024585,43.30392873)
\curveto(431.24246909,43.34392761)(431.19246914,43.40892755)(431.1524585,43.49892873)
\curveto(431.15246918,43.51892744)(431.15246918,43.54392741)(431.1524585,43.57392873)
\curveto(431.14246919,43.60392735)(431.13746919,43.62892733)(431.1374585,43.64892873)
\curveto(431.1274692,43.78892717)(431.1274692,43.93392702)(431.1374585,44.08392873)
\curveto(431.13746919,44.24392671)(431.17746915,44.3539266)(431.2574585,44.41392873)
\curveto(431.33746899,44.46392649)(431.45246888,44.48892647)(431.6024585,44.48892873)
\lineto(432.0074585,44.48892873)
\lineto(433.7624585,44.48892873)
\lineto(434.0174585,44.48892873)
\lineto(434.3024585,44.48892873)
\curveto(434.39246594,44.49892646)(434.47746585,44.50892645)(434.5574585,44.51892873)
\curveto(434.6274657,44.53892642)(434.67746565,44.56892639)(434.7074585,44.60892873)
\curveto(434.73746559,44.64892631)(434.74246559,44.69392626)(434.7224585,44.74392873)
\curveto(434.70246563,44.79392616)(434.68246565,44.83392612)(434.6624585,44.86392873)
\curveto(434.62246571,44.91392604)(434.58246575,44.958926)(434.5424585,44.99892873)
\lineto(434.4224585,45.14892873)
\curveto(434.37246596,45.21892574)(434.327466,45.28892567)(434.2874585,45.35892873)
\lineto(434.1674585,45.59892873)
\curveto(434.07746625,45.77892518)(434.01246632,45.99392496)(433.9724585,46.24392873)
\curveto(433.9324664,46.49392446)(433.91246642,46.74892421)(433.9124585,47.00892873)
\curveto(433.91246642,47.26892369)(433.93746639,47.52392343)(433.9874585,47.77392873)
\curveto(434.0274663,48.02392293)(434.08746624,48.24392271)(434.1674585,48.43392873)
\curveto(434.33746599,48.83392212)(434.57246576,49.17892178)(434.8724585,49.46892873)
\curveto(435.17246516,49.7589212)(435.52246481,49.98892097)(435.9224585,50.15892873)
\curveto(436.0324643,50.20892075)(436.14246419,50.24892071)(436.2524585,50.27892873)
\curveto(436.35246398,50.31892064)(436.45746387,50.3589206)(436.5674585,50.39892873)
\curveto(436.67746365,50.42892053)(436.79246354,50.44892051)(436.9124585,50.45892873)
\lineto(437.2424585,50.51892873)
\curveto(437.27246306,50.52892043)(437.30746302,50.53392042)(437.3474585,50.53392873)
\curveto(437.37746295,50.53392042)(437.40746292,50.53892042)(437.4374585,50.54892873)
\curveto(437.49746283,50.56892039)(437.55746277,50.56892039)(437.6174585,50.54892873)
\curveto(437.66746266,50.53892042)(437.72246261,50.54392041)(437.7824585,50.56392873)
\moveto(438.1724585,49.22892873)
\curveto(438.12246221,49.24892171)(438.06246227,49.2539217)(437.9924585,49.24392873)
\curveto(437.92246241,49.23392172)(437.85746247,49.22892173)(437.7974585,49.22892873)
\curveto(437.6274627,49.22892173)(437.46746286,49.21892174)(437.3174585,49.19892873)
\curveto(437.16746316,49.18892177)(437.0324633,49.1589218)(436.9124585,49.10892873)
\curveto(436.81246352,49.07892188)(436.72246361,49.0539219)(436.6424585,49.03392873)
\curveto(436.56246377,49.01392194)(436.48246385,48.98392197)(436.4024585,48.94392873)
\curveto(436.15246418,48.83392212)(435.92246441,48.68392227)(435.7124585,48.49392873)
\curveto(435.49246484,48.30392265)(435.327465,48.08392287)(435.2174585,47.83392873)
\curveto(435.18746514,47.7539232)(435.16246517,47.67392328)(435.1424585,47.59392873)
\curveto(435.11246522,47.52392343)(435.08746524,47.44892351)(435.0674585,47.36892873)
\curveto(435.03746529,47.2589237)(435.02246531,47.14892381)(435.0224585,47.03892873)
\curveto(435.01246532,46.92892403)(435.00746532,46.80892415)(435.0074585,46.67892873)
\curveto(435.01746531,46.62892433)(435.0274653,46.58392437)(435.0374585,46.54392873)
\lineto(435.0374585,46.40892873)
\lineto(435.0974585,46.13892873)
\curveto(435.11746521,46.0589249)(435.14746518,45.97892498)(435.1874585,45.89892873)
\curveto(435.327465,45.5589254)(435.53746479,45.28892567)(435.8174585,45.08892873)
\curveto(436.08746424,44.88892607)(436.40746392,44.72892623)(436.7774585,44.60892873)
\curveto(436.88746344,44.56892639)(436.99746333,44.54392641)(437.1074585,44.53392873)
\curveto(437.21746311,44.52392643)(437.332463,44.50392645)(437.4524585,44.47392873)
\curveto(437.50246283,44.46392649)(437.54746278,44.46392649)(437.5874585,44.47392873)
\curveto(437.6274627,44.48392647)(437.67246266,44.47892648)(437.7224585,44.45892873)
\curveto(437.77246256,44.44892651)(437.84746248,44.44392651)(437.9474585,44.44392873)
\curveto(438.03746229,44.44392651)(438.10746222,44.44892651)(438.1574585,44.45892873)
\lineto(438.2774585,44.45892873)
\curveto(438.31746201,44.46892649)(438.35746197,44.47392648)(438.3974585,44.47392873)
\curveto(438.43746189,44.47392648)(438.47246186,44.47892648)(438.5024585,44.48892873)
\curveto(438.5324618,44.49892646)(438.56746176,44.50392645)(438.6074585,44.50392873)
\curveto(438.63746169,44.50392645)(438.66746166,44.50892645)(438.6974585,44.51892873)
\curveto(438.77746155,44.53892642)(438.85746147,44.5539264)(438.9374585,44.56392873)
\lineto(439.1774585,44.62392873)
\curveto(439.51746081,44.73392622)(439.80746052,44.88392607)(440.0474585,45.07392873)
\curveto(440.28746004,45.27392568)(440.48745984,45.51892544)(440.6474585,45.80892873)
\curveto(440.69745963,45.89892506)(440.73745959,45.99392496)(440.7674585,46.09392873)
\curveto(440.78745954,46.19392476)(440.81245952,46.29892466)(440.8424585,46.40892873)
\curveto(440.86245947,46.4589245)(440.87245946,46.50392445)(440.8724585,46.54392873)
\curveto(440.86245947,46.59392436)(440.86245947,46.64392431)(440.8724585,46.69392873)
\curveto(440.88245945,46.73392422)(440.88745944,46.77892418)(440.8874585,46.82892873)
\lineto(440.8874585,46.96392873)
\lineto(440.8874585,47.09892873)
\curveto(440.87745945,47.13892382)(440.87245946,47.17392378)(440.8724585,47.20392873)
\curveto(440.87245946,47.23392372)(440.86745946,47.26892369)(440.8574585,47.30892873)
\curveto(440.83745949,47.38892357)(440.82245951,47.46392349)(440.8124585,47.53392873)
\curveto(440.79245954,47.60392335)(440.76745956,47.67892328)(440.7374585,47.75892873)
\curveto(440.60745972,48.06892289)(440.43745989,48.31892264)(440.2274585,48.50892873)
\curveto(440.00746032,48.69892226)(439.74246059,48.8589221)(439.4324585,48.98892873)
\curveto(439.29246104,49.03892192)(439.15246118,49.07392188)(439.0124585,49.09392873)
\curveto(438.86246147,49.12392183)(438.71246162,49.1589218)(438.5624585,49.19892873)
\curveto(438.51246182,49.21892174)(438.46746186,49.22392173)(438.4274585,49.21392873)
\curveto(438.37746195,49.21392174)(438.327462,49.21892174)(438.2774585,49.22892873)
\lineto(438.1724585,49.22892873)
}
}
{
\newrgbcolor{curcolor}{0 0 0}
\pscustom[linestyle=none,fillstyle=solid,fillcolor=curcolor]
{
\newpath
\moveto(433.9124585,55.69017873)
\curveto(433.91246642,55.92017394)(433.97246636,56.05017381)(434.0924585,56.08017873)
\curveto(434.20246613,56.11017375)(434.36746596,56.12517374)(434.5874585,56.12517873)
\lineto(434.8724585,56.12517873)
\curveto(434.96246537,56.12517374)(435.03746529,56.10017376)(435.0974585,56.05017873)
\curveto(435.17746515,55.99017387)(435.22246511,55.90517396)(435.2324585,55.79517873)
\curveto(435.2324651,55.68517418)(435.24746508,55.57517429)(435.2774585,55.46517873)
\curveto(435.30746502,55.32517454)(435.33746499,55.19017467)(435.3674585,55.06017873)
\curveto(435.39746493,54.94017492)(435.43746489,54.82517504)(435.4874585,54.71517873)
\curveto(435.61746471,54.42517544)(435.79746453,54.19017567)(436.0274585,54.01017873)
\curveto(436.24746408,53.83017603)(436.50246383,53.67517619)(436.7924585,53.54517873)
\curveto(436.90246343,53.50517636)(437.01746331,53.47517639)(437.1374585,53.45517873)
\curveto(437.24746308,53.43517643)(437.36246297,53.41017645)(437.4824585,53.38017873)
\curveto(437.5324628,53.37017649)(437.58246275,53.3651765)(437.6324585,53.36517873)
\curveto(437.68246265,53.37517649)(437.7324626,53.37517649)(437.7824585,53.36517873)
\curveto(437.90246243,53.33517653)(438.04246229,53.32017654)(438.2024585,53.32017873)
\curveto(438.35246198,53.33017653)(438.49746183,53.33517653)(438.6374585,53.33517873)
\lineto(440.4824585,53.33517873)
\lineto(440.8274585,53.33517873)
\curveto(440.94745938,53.33517653)(441.06245927,53.33017653)(441.1724585,53.32017873)
\curveto(441.28245905,53.31017655)(441.37745895,53.30517656)(441.4574585,53.30517873)
\curveto(441.53745879,53.31517655)(441.60745872,53.29517657)(441.6674585,53.24517873)
\curveto(441.73745859,53.19517667)(441.77745855,53.11517675)(441.7874585,53.00517873)
\curveto(441.79745853,52.90517696)(441.80245853,52.79517707)(441.8024585,52.67517873)
\lineto(441.8024585,52.40517873)
\curveto(441.78245855,52.35517751)(441.76745856,52.30517756)(441.7574585,52.25517873)
\curveto(441.73745859,52.21517765)(441.71245862,52.18517768)(441.6824585,52.16517873)
\curveto(441.61245872,52.11517775)(441.5274588,52.08517778)(441.4274585,52.07517873)
\lineto(441.0974585,52.07517873)
\lineto(439.9424585,52.07517873)
\lineto(435.7874585,52.07517873)
\lineto(434.7524585,52.07517873)
\lineto(434.4524585,52.07517873)
\curveto(434.35246598,52.08517778)(434.26746606,52.11517775)(434.1974585,52.16517873)
\curveto(434.15746617,52.19517767)(434.1274662,52.24517762)(434.1074585,52.31517873)
\curveto(434.08746624,52.39517747)(434.07746625,52.48017738)(434.0774585,52.57017873)
\curveto(434.06746626,52.6601772)(434.06746626,52.75017711)(434.0774585,52.84017873)
\curveto(434.08746624,52.93017693)(434.10246623,53.00017686)(434.1224585,53.05017873)
\curveto(434.15246618,53.13017673)(434.21246612,53.18017668)(434.3024585,53.20017873)
\curveto(434.38246595,53.23017663)(434.47246586,53.24517662)(434.5724585,53.24517873)
\lineto(434.8724585,53.24517873)
\curveto(434.97246536,53.24517662)(435.06246527,53.2651766)(435.1424585,53.30517873)
\curveto(435.16246517,53.31517655)(435.17746515,53.32517654)(435.1874585,53.33517873)
\lineto(435.2324585,53.38017873)
\curveto(435.2324651,53.49017637)(435.18746514,53.58017628)(435.0974585,53.65017873)
\curveto(434.99746533,53.72017614)(434.91746541,53.78017608)(434.8574585,53.83017873)
\lineto(434.7674585,53.92017873)
\curveto(434.65746567,54.01017585)(434.54246579,54.13517573)(434.4224585,54.29517873)
\curveto(434.30246603,54.45517541)(434.21246612,54.60517526)(434.1524585,54.74517873)
\curveto(434.10246623,54.83517503)(434.06746626,54.93017493)(434.0474585,55.03017873)
\curveto(434.01746631,55.13017473)(433.98746634,55.23517463)(433.9574585,55.34517873)
\curveto(433.94746638,55.40517446)(433.94246639,55.4651744)(433.9424585,55.52517873)
\curveto(433.9324664,55.58517428)(433.92246641,55.64017422)(433.9124585,55.69017873)
}
}
{
\newrgbcolor{curcolor}{0 0 0}
\pscustom[linestyle=none,fillstyle=solid,fillcolor=curcolor]
{
}
}
{
\newrgbcolor{curcolor}{0 0 0}
\pscustom[linestyle=none,fillstyle=solid,fillcolor=curcolor]
{
\newpath
\moveto(431.2124585,64.24510061)
\curveto(431.20246913,64.93509597)(431.32246901,65.53509537)(431.5724585,66.04510061)
\curveto(431.82246851,66.56509434)(432.15746817,66.96009395)(432.5774585,67.23010061)
\curveto(432.65746767,67.28009363)(432.74746758,67.32509358)(432.8474585,67.36510061)
\curveto(432.93746739,67.4050935)(433.0324673,67.45009346)(433.1324585,67.50010061)
\curveto(433.2324671,67.54009337)(433.332467,67.57009334)(433.4324585,67.59010061)
\curveto(433.5324668,67.6100933)(433.63746669,67.63009328)(433.7474585,67.65010061)
\curveto(433.79746653,67.67009324)(433.84246649,67.67509323)(433.8824585,67.66510061)
\curveto(433.92246641,67.65509325)(433.96746636,67.66009325)(434.0174585,67.68010061)
\curveto(434.06746626,67.69009322)(434.15246618,67.69509321)(434.2724585,67.69510061)
\curveto(434.38246595,67.69509321)(434.46746586,67.69009322)(434.5274585,67.68010061)
\curveto(434.58746574,67.66009325)(434.64746568,67.65009326)(434.7074585,67.65010061)
\curveto(434.76746556,67.66009325)(434.8274655,67.65509325)(434.8874585,67.63510061)
\curveto(435.0274653,67.59509331)(435.16246517,67.56009335)(435.2924585,67.53010061)
\curveto(435.42246491,67.50009341)(435.54746478,67.46009345)(435.6674585,67.41010061)
\curveto(435.80746452,67.35009356)(435.9324644,67.28009363)(436.0424585,67.20010061)
\curveto(436.15246418,67.13009378)(436.26246407,67.05509385)(436.3724585,66.97510061)
\lineto(436.4324585,66.91510061)
\curveto(436.45246388,66.905094)(436.47246386,66.89009402)(436.4924585,66.87010061)
\curveto(436.65246368,66.75009416)(436.79746353,66.61509429)(436.9274585,66.46510061)
\curveto(437.05746327,66.31509459)(437.18246315,66.15509475)(437.3024585,65.98510061)
\curveto(437.52246281,65.67509523)(437.7274626,65.38009553)(437.9174585,65.10010061)
\curveto(438.05746227,64.87009604)(438.19246214,64.64009627)(438.3224585,64.41010061)
\curveto(438.45246188,64.19009672)(438.58746174,63.97009694)(438.7274585,63.75010061)
\curveto(438.89746143,63.50009741)(439.07746125,63.26009765)(439.2674585,63.03010061)
\curveto(439.45746087,62.8100981)(439.68246065,62.62009829)(439.9424585,62.46010061)
\curveto(440.00246033,62.42009849)(440.06246027,62.38509852)(440.1224585,62.35510061)
\curveto(440.17246016,62.32509858)(440.23746009,62.29509861)(440.3174585,62.26510061)
\curveto(440.38745994,62.24509866)(440.44745988,62.24009867)(440.4974585,62.25010061)
\curveto(440.56745976,62.27009864)(440.62245971,62.3050986)(440.6624585,62.35510061)
\curveto(440.69245964,62.4050985)(440.71245962,62.46509844)(440.7224585,62.53510061)
\lineto(440.7224585,62.77510061)
\lineto(440.7224585,63.52510061)
\lineto(440.7224585,66.33010061)
\lineto(440.7224585,66.99010061)
\curveto(440.72245961,67.08009383)(440.7274596,67.16509374)(440.7374585,67.24510061)
\curveto(440.73745959,67.32509358)(440.75745957,67.39009352)(440.7974585,67.44010061)
\curveto(440.83745949,67.49009342)(440.91245942,67.53009338)(441.0224585,67.56010061)
\curveto(441.12245921,67.60009331)(441.22245911,67.6100933)(441.3224585,67.59010061)
\lineto(441.4574585,67.59010061)
\curveto(441.5274588,67.57009334)(441.58745874,67.55009336)(441.6374585,67.53010061)
\curveto(441.68745864,67.5100934)(441.7274586,67.47509343)(441.7574585,67.42510061)
\curveto(441.79745853,67.37509353)(441.81745851,67.3050936)(441.8174585,67.21510061)
\lineto(441.8174585,66.94510061)
\lineto(441.8174585,66.04510061)
\lineto(441.8174585,62.53510061)
\lineto(441.8174585,61.47010061)
\curveto(441.81745851,61.39009952)(441.82245851,61.30009961)(441.8324585,61.20010061)
\curveto(441.8324585,61.10009981)(441.82245851,61.01509989)(441.8024585,60.94510061)
\curveto(441.7324586,60.73510017)(441.55245878,60.67010024)(441.2624585,60.75010061)
\curveto(441.22245911,60.76010015)(441.18745914,60.76010015)(441.1574585,60.75010061)
\curveto(441.11745921,60.75010016)(441.07245926,60.76010015)(441.0224585,60.78010061)
\curveto(440.94245939,60.80010011)(440.85745947,60.82010009)(440.7674585,60.84010061)
\curveto(440.67745965,60.86010005)(440.59245974,60.88510002)(440.5124585,60.91510061)
\curveto(440.02246031,61.07509983)(439.60746072,61.27509963)(439.2674585,61.51510061)
\curveto(439.01746131,61.69509921)(438.79246154,61.90009901)(438.5924585,62.13010061)
\curveto(438.38246195,62.36009855)(438.18746214,62.60009831)(438.0074585,62.85010061)
\curveto(437.8274625,63.1100978)(437.65746267,63.37509753)(437.4974585,63.64510061)
\curveto(437.327463,63.92509698)(437.15246318,64.19509671)(436.9724585,64.45510061)
\curveto(436.89246344,64.56509634)(436.81746351,64.67009624)(436.7474585,64.77010061)
\curveto(436.67746365,64.88009603)(436.60246373,64.99009592)(436.5224585,65.10010061)
\curveto(436.49246384,65.14009577)(436.46246387,65.17509573)(436.4324585,65.20510061)
\curveto(436.39246394,65.24509566)(436.36246397,65.28509562)(436.3424585,65.32510061)
\curveto(436.2324641,65.46509544)(436.10746422,65.59009532)(435.9674585,65.70010061)
\curveto(435.93746439,65.72009519)(435.91246442,65.74509516)(435.8924585,65.77510061)
\curveto(435.86246447,65.8050951)(435.8324645,65.83009508)(435.8024585,65.85010061)
\curveto(435.70246463,65.93009498)(435.60246473,65.99509491)(435.5024585,66.04510061)
\curveto(435.40246493,66.1050948)(435.29246504,66.16009475)(435.1724585,66.21010061)
\curveto(435.10246523,66.24009467)(435.0274653,66.26009465)(434.9474585,66.27010061)
\lineto(434.7074585,66.33010061)
\lineto(434.6174585,66.33010061)
\curveto(434.58746574,66.34009457)(434.55746577,66.34509456)(434.5274585,66.34510061)
\curveto(434.45746587,66.36509454)(434.36246597,66.37009454)(434.2424585,66.36010061)
\curveto(434.11246622,66.36009455)(434.01246632,66.35009456)(433.9424585,66.33010061)
\curveto(433.86246647,66.3100946)(433.78746654,66.29009462)(433.7174585,66.27010061)
\curveto(433.63746669,66.26009465)(433.55746677,66.24009467)(433.4774585,66.21010061)
\curveto(433.23746709,66.10009481)(433.03746729,65.95009496)(432.8774585,65.76010061)
\curveto(432.70746762,65.58009533)(432.56746776,65.36009555)(432.4574585,65.10010061)
\curveto(432.43746789,65.03009588)(432.42246791,64.96009595)(432.4124585,64.89010061)
\curveto(432.39246794,64.82009609)(432.37246796,64.74509616)(432.3524585,64.66510061)
\curveto(432.332468,64.58509632)(432.32246801,64.47509643)(432.3224585,64.33510061)
\curveto(432.32246801,64.2050967)(432.332468,64.10009681)(432.3524585,64.02010061)
\curveto(432.36246797,63.96009695)(432.36746796,63.905097)(432.3674585,63.85510061)
\curveto(432.36746796,63.8050971)(432.37746795,63.75509715)(432.3974585,63.70510061)
\curveto(432.43746789,63.6050973)(432.47746785,63.5100974)(432.5174585,63.42010061)
\curveto(432.55746777,63.34009757)(432.60246773,63.26009765)(432.6524585,63.18010061)
\curveto(432.67246766,63.15009776)(432.69746763,63.12009779)(432.7274585,63.09010061)
\curveto(432.75746757,63.07009784)(432.78246755,63.04509786)(432.8024585,63.01510061)
\lineto(432.8774585,62.94010061)
\curveto(432.89746743,62.910098)(432.91746741,62.88509802)(432.9374585,62.86510061)
\lineto(433.1474585,62.71510061)
\curveto(433.20746712,62.67509823)(433.27246706,62.63009828)(433.3424585,62.58010061)
\curveto(433.4324669,62.52009839)(433.53746679,62.47009844)(433.6574585,62.43010061)
\curveto(433.76746656,62.40009851)(433.87746645,62.36509854)(433.9874585,62.32510061)
\curveto(434.09746623,62.28509862)(434.24246609,62.26009865)(434.4224585,62.25010061)
\curveto(434.59246574,62.24009867)(434.71746561,62.2100987)(434.7974585,62.16010061)
\curveto(434.87746545,62.1100988)(434.92246541,62.03509887)(434.9324585,61.93510061)
\curveto(434.94246539,61.83509907)(434.94746538,61.72509918)(434.9474585,61.60510061)
\curveto(434.94746538,61.56509934)(434.95246538,61.52509938)(434.9624585,61.48510061)
\curveto(434.96246537,61.44509946)(434.95746537,61.4100995)(434.9474585,61.38010061)
\curveto(434.9274654,61.33009958)(434.91746541,61.28009963)(434.9174585,61.23010061)
\curveto(434.91746541,61.19009972)(434.90746542,61.15009976)(434.8874585,61.11010061)
\curveto(434.8274655,61.02009989)(434.69246564,60.97509993)(434.4824585,60.97510061)
\lineto(434.3624585,60.97510061)
\curveto(434.30246603,60.98509992)(434.24246609,60.99009992)(434.1824585,60.99010061)
\curveto(434.11246622,61.00009991)(434.04746628,61.0100999)(433.9874585,61.02010061)
\curveto(433.87746645,61.04009987)(433.77746655,61.06009985)(433.6874585,61.08010061)
\curveto(433.58746674,61.10009981)(433.49246684,61.13009978)(433.4024585,61.17010061)
\curveto(433.332467,61.19009972)(433.27246706,61.2100997)(433.2224585,61.23010061)
\lineto(433.0424585,61.29010061)
\curveto(432.78246755,61.4100995)(432.53746779,61.56509934)(432.3074585,61.75510061)
\curveto(432.07746825,61.95509895)(431.89246844,62.17009874)(431.7524585,62.40010061)
\curveto(431.67246866,62.5100984)(431.60746872,62.62509828)(431.5574585,62.74510061)
\lineto(431.4074585,63.13510061)
\curveto(431.35746897,63.24509766)(431.327469,63.36009755)(431.3174585,63.48010061)
\curveto(431.29746903,63.60009731)(431.27246906,63.72509718)(431.2424585,63.85510061)
\curveto(431.24246909,63.92509698)(431.24246909,63.99009692)(431.2424585,64.05010061)
\curveto(431.2324691,64.1100968)(431.22246911,64.17509673)(431.2124585,64.24510061)
}
}
{
\newrgbcolor{curcolor}{0 0 0}
\pscustom[linestyle=none,fillstyle=solid,fillcolor=curcolor]
{
\newpath
\moveto(431.2124585,72.45970998)
\curveto(431.18246915,74.08970454)(431.73746859,75.13970349)(432.8774585,75.60970998)
\curveto(433.10746722,75.70970292)(433.39746693,75.77470286)(433.7474585,75.80470998)
\curveto(434.08746624,75.84470279)(434.39746593,75.81970281)(434.6774585,75.72970998)
\curveto(434.93746539,75.63970299)(435.16246517,75.51970311)(435.3524585,75.36970998)
\curveto(435.39246494,75.34970328)(435.4274649,75.32470331)(435.4574585,75.29470998)
\curveto(435.47746485,75.26470337)(435.50246483,75.23970339)(435.5324585,75.21970998)
\lineto(435.6524585,75.12970998)
\curveto(435.68246465,75.09970353)(435.70746462,75.06470357)(435.7274585,75.02470998)
\curveto(435.77746455,74.97470366)(435.82246451,74.91970371)(435.8624585,74.85970998)
\curveto(435.90246443,74.80970382)(435.95246438,74.76470387)(436.0124585,74.72470998)
\curveto(436.05246428,74.68470395)(436.10246423,74.66970396)(436.1624585,74.67970998)
\curveto(436.21246412,74.68970394)(436.25746407,74.71970391)(436.2974585,74.76970998)
\curveto(436.33746399,74.81970381)(436.37746395,74.87470376)(436.4174585,74.93470998)
\curveto(436.44746388,75.00470363)(436.47746385,75.06970356)(436.5074585,75.12970998)
\curveto(436.53746379,75.18970344)(436.56746376,75.23970339)(436.5974585,75.27970998)
\curveto(436.81746351,75.59970303)(437.1274632,75.85470278)(437.5274585,76.04470998)
\curveto(437.61746271,76.08470255)(437.71246262,76.11470252)(437.8124585,76.13470998)
\curveto(437.90246243,76.16470247)(437.99246234,76.18970244)(438.0824585,76.20970998)
\curveto(438.1324622,76.21970241)(438.18246215,76.22470241)(438.2324585,76.22470998)
\curveto(438.27246206,76.2347024)(438.31746201,76.24470239)(438.3674585,76.25470998)
\curveto(438.41746191,76.26470237)(438.46746186,76.26470237)(438.5174585,76.25470998)
\curveto(438.56746176,76.24470239)(438.61746171,76.24970238)(438.6674585,76.26970998)
\curveto(438.71746161,76.27970235)(438.77746155,76.28470235)(438.8474585,76.28470998)
\curveto(438.91746141,76.28470235)(438.97746135,76.27470236)(439.0274585,76.25470998)
\lineto(439.2524585,76.25470998)
\lineto(439.4924585,76.19470998)
\curveto(439.56246077,76.18470245)(439.6324607,76.16970246)(439.7024585,76.14970998)
\curveto(439.79246054,76.11970251)(439.87746045,76.08970254)(439.9574585,76.05970998)
\curveto(440.03746029,76.03970259)(440.11746021,76.00970262)(440.1974585,75.96970998)
\curveto(440.25746007,75.94970268)(440.31746001,75.91970271)(440.3774585,75.87970998)
\curveto(440.4274599,75.84970278)(440.47745985,75.81470282)(440.5274585,75.77470998)
\curveto(440.83745949,75.57470306)(441.09745923,75.32470331)(441.3074585,75.02470998)
\curveto(441.50745882,74.72470391)(441.67245866,74.37970425)(441.8024585,73.98970998)
\curveto(441.84245849,73.86970476)(441.86745846,73.73970489)(441.8774585,73.59970998)
\curveto(441.89745843,73.46970516)(441.92245841,73.3347053)(441.9524585,73.19470998)
\curveto(441.96245837,73.12470551)(441.96745836,73.05470558)(441.9674585,72.98470998)
\curveto(441.96745836,72.92470571)(441.97245836,72.85970577)(441.9824585,72.78970998)
\curveto(441.99245834,72.74970588)(441.99745833,72.68970594)(441.9974585,72.60970998)
\curveto(441.99745833,72.53970609)(441.99245834,72.48970614)(441.9824585,72.45970998)
\curveto(441.97245836,72.40970622)(441.96745836,72.36470627)(441.9674585,72.32470998)
\lineto(441.9674585,72.20470998)
\curveto(441.94745838,72.10470653)(441.9324584,72.00470663)(441.9224585,71.90470998)
\curveto(441.91245842,71.80470683)(441.89745843,71.70970692)(441.8774585,71.61970998)
\curveto(441.84745848,71.50970712)(441.82245851,71.39970723)(441.8024585,71.28970998)
\curveto(441.77245856,71.18970744)(441.7324586,71.08470755)(441.6824585,70.97470998)
\curveto(441.52245881,70.60470803)(441.32245901,70.28970834)(441.0824585,70.02970998)
\curveto(440.8324595,69.76970886)(440.52245981,69.55970907)(440.1524585,69.39970998)
\curveto(440.06246027,69.35970927)(439.96746036,69.32470931)(439.8674585,69.29470998)
\curveto(439.76746056,69.26470937)(439.66246067,69.2347094)(439.5524585,69.20470998)
\curveto(439.50246083,69.18470945)(439.45246088,69.17470946)(439.4024585,69.17470998)
\curveto(439.34246099,69.17470946)(439.28246105,69.16470947)(439.2224585,69.14470998)
\curveto(439.16246117,69.12470951)(439.08246125,69.11470952)(438.9824585,69.11470998)
\curveto(438.88246145,69.11470952)(438.80746152,69.1297095)(438.7574585,69.15970998)
\curveto(438.7274616,69.16970946)(438.70246163,69.18470945)(438.6824585,69.20470998)
\lineto(438.6224585,69.26470998)
\curveto(438.60246173,69.30470933)(438.58746174,69.36470927)(438.5774585,69.44470998)
\curveto(438.56746176,69.5347091)(438.56246177,69.62470901)(438.5624585,69.71470998)
\curveto(438.56246177,69.80470883)(438.56746176,69.88970874)(438.5774585,69.96970998)
\curveto(438.58746174,70.05970857)(438.59746173,70.12470851)(438.6074585,70.16470998)
\curveto(438.6274617,70.18470845)(438.64246169,70.20470843)(438.6524585,70.22470998)
\curveto(438.65246168,70.24470839)(438.66246167,70.26470837)(438.6824585,70.28470998)
\curveto(438.77246156,70.35470828)(438.88746144,70.39470824)(439.0274585,70.40470998)
\curveto(439.16746116,70.42470821)(439.29246104,70.45470818)(439.4024585,70.49470998)
\lineto(439.7624585,70.64470998)
\curveto(439.87246046,70.69470794)(439.97746035,70.75970787)(440.0774585,70.83970998)
\curveto(440.10746022,70.85970777)(440.1324602,70.87970775)(440.1524585,70.89970998)
\curveto(440.17246016,70.9297077)(440.19746013,70.95470768)(440.2274585,70.97470998)
\curveto(440.28746004,71.01470762)(440.33246,71.04970758)(440.3624585,71.07970998)
\curveto(440.39245994,71.11970751)(440.42245991,71.15470748)(440.4524585,71.18470998)
\curveto(440.48245985,71.22470741)(440.51245982,71.26970736)(440.5424585,71.31970998)
\curveto(440.60245973,71.40970722)(440.65245968,71.50470713)(440.6924585,71.60470998)
\lineto(440.8124585,71.93470998)
\curveto(440.86245947,72.08470655)(440.89245944,72.28470635)(440.9024585,72.53470998)
\curveto(440.91245942,72.78470585)(440.89245944,72.99470564)(440.8424585,73.16470998)
\curveto(440.82245951,73.24470539)(440.80745952,73.31470532)(440.7974585,73.37470998)
\lineto(440.7374585,73.58470998)
\curveto(440.61745971,73.86470477)(440.46745986,74.10470453)(440.2874585,74.30470998)
\curveto(440.10746022,74.51470412)(439.87746045,74.67970395)(439.5974585,74.79970998)
\curveto(439.5274608,74.8297038)(439.45746087,74.84970378)(439.3874585,74.85970998)
\lineto(439.1474585,74.91970998)
\curveto(439.00746132,74.95970367)(438.84746148,74.96970366)(438.6674585,74.94970998)
\curveto(438.47746185,74.9297037)(438.327462,74.89970373)(438.2174585,74.85970998)
\curveto(437.83746249,74.7297039)(437.54746278,74.54470409)(437.3474585,74.30470998)
\curveto(437.14746318,74.07470456)(436.98746334,73.76470487)(436.8674585,73.37470998)
\curveto(436.83746349,73.26470537)(436.81746351,73.14470549)(436.8074585,73.01470998)
\curveto(436.79746353,72.89470574)(436.79246354,72.76970586)(436.7924585,72.63970998)
\curveto(436.79246354,72.47970615)(436.78746354,72.33970629)(436.7774585,72.21970998)
\curveto(436.76746356,72.09970653)(436.70746362,72.01470662)(436.5974585,71.96470998)
\curveto(436.56746376,71.94470669)(436.5324638,71.9347067)(436.4924585,71.93470998)
\lineto(436.3574585,71.93470998)
\curveto(436.25746407,71.92470671)(436.16246417,71.92470671)(436.0724585,71.93470998)
\curveto(435.98246435,71.95470668)(435.91746441,71.99470664)(435.8774585,72.05470998)
\curveto(435.84746448,72.09470654)(435.8274645,72.1347065)(435.8174585,72.17470998)
\curveto(435.80746452,72.22470641)(435.79746453,72.27970635)(435.7874585,72.33970998)
\curveto(435.77746455,72.35970627)(435.77746455,72.38470625)(435.7874585,72.41470998)
\curveto(435.78746454,72.44470619)(435.78246455,72.46970616)(435.7724585,72.48970998)
\lineto(435.7724585,72.62470998)
\curveto(435.75246458,72.7347059)(435.74246459,72.8347058)(435.7424585,72.92470998)
\curveto(435.7324646,73.02470561)(435.71246462,73.11970551)(435.6824585,73.20970998)
\curveto(435.57246476,73.5297051)(435.4274649,73.78470485)(435.2474585,73.97470998)
\curveto(435.06746526,74.16470447)(434.81746551,74.31470432)(434.4974585,74.42470998)
\curveto(434.39746593,74.45470418)(434.27246606,74.47470416)(434.1224585,74.48470998)
\curveto(433.96246637,74.50470413)(433.81746651,74.49970413)(433.6874585,74.46970998)
\curveto(433.61746671,74.44970418)(433.55246678,74.4297042)(433.4924585,74.40970998)
\curveto(433.42246691,74.39970423)(433.35746697,74.37970425)(433.2974585,74.34970998)
\curveto(433.05746727,74.24970438)(432.86746746,74.10470453)(432.7274585,73.91470998)
\curveto(432.58746774,73.72470491)(432.47746785,73.49970513)(432.3974585,73.23970998)
\curveto(432.37746795,73.17970545)(432.36746796,73.11970551)(432.3674585,73.05970998)
\curveto(432.36746796,72.99970563)(432.35746797,72.9347057)(432.3374585,72.86470998)
\curveto(432.31746801,72.78470585)(432.30746802,72.68970594)(432.3074585,72.57970998)
\curveto(432.30746802,72.46970616)(432.31746801,72.37470626)(432.3374585,72.29470998)
\curveto(432.35746797,72.24470639)(432.36746796,72.19470644)(432.3674585,72.14470998)
\curveto(432.36746796,72.10470653)(432.37746795,72.05970657)(432.3974585,72.00970998)
\curveto(432.44746788,71.8297068)(432.52246781,71.65970697)(432.6224585,71.49970998)
\curveto(432.71246762,71.34970728)(432.8274675,71.21970741)(432.9674585,71.10970998)
\curveto(433.08746724,71.01970761)(433.21746711,70.93970769)(433.3574585,70.86970998)
\curveto(433.49746683,70.79970783)(433.65246668,70.7347079)(433.8224585,70.67470998)
\curveto(433.9324664,70.64470799)(434.05246628,70.62470801)(434.1824585,70.61470998)
\curveto(434.30246603,70.60470803)(434.40246593,70.56970806)(434.4824585,70.50970998)
\curveto(434.52246581,70.48970814)(434.56246577,70.4297082)(434.6024585,70.32970998)
\curveto(434.61246572,70.28970834)(434.62246571,70.2297084)(434.6324585,70.14970998)
\lineto(434.6324585,69.89470998)
\curveto(434.62246571,69.80470883)(434.61246572,69.71970891)(434.6024585,69.63970998)
\curveto(434.59246574,69.56970906)(434.57746575,69.51970911)(434.5574585,69.48970998)
\curveto(434.5274658,69.44970918)(434.47246586,69.41470922)(434.3924585,69.38470998)
\curveto(434.31246602,69.35470928)(434.2274661,69.34970928)(434.1374585,69.36970998)
\curveto(434.08746624,69.37970925)(434.03746629,69.38470925)(433.9874585,69.38470998)
\lineto(433.8074585,69.41470998)
\curveto(433.70746662,69.44470919)(433.60746672,69.46970916)(433.5074585,69.48970998)
\curveto(433.40746692,69.51970911)(433.31746701,69.55470908)(433.2374585,69.59470998)
\curveto(433.1274672,69.64470899)(433.02246731,69.68970894)(432.9224585,69.72970998)
\curveto(432.81246752,69.76970886)(432.70746762,69.81970881)(432.6074585,69.87970998)
\curveto(432.06746826,70.20970842)(431.67246866,70.67970795)(431.4224585,71.28970998)
\curveto(431.37246896,71.40970722)(431.33746899,71.5347071)(431.3174585,71.66470998)
\curveto(431.29746903,71.80470683)(431.27246906,71.94470669)(431.2424585,72.08470998)
\curveto(431.2324691,72.14470649)(431.2274691,72.20470643)(431.2274585,72.26470998)
\curveto(431.2274691,72.3347063)(431.22246911,72.39970623)(431.2124585,72.45970998)
}
}
{
\newrgbcolor{curcolor}{0 0 0}
\pscustom[linestyle=none,fillstyle=solid,fillcolor=curcolor]
{
\newpath
\moveto(440.1824585,78.63431936)
\lineto(440.1824585,79.26431936)
\lineto(440.1824585,79.45931936)
\curveto(440.18246015,79.52931683)(440.19246014,79.58931677)(440.2124585,79.63931936)
\curveto(440.25246008,79.70931665)(440.29246004,79.7593166)(440.3324585,79.78931936)
\curveto(440.38245995,79.82931653)(440.44745988,79.84931651)(440.5274585,79.84931936)
\curveto(440.60745972,79.8593165)(440.69245964,79.86431649)(440.7824585,79.86431936)
\lineto(441.5024585,79.86431936)
\curveto(441.98245835,79.86431649)(442.39245794,79.80431655)(442.7324585,79.68431936)
\curveto(443.07245726,79.56431679)(443.34745698,79.36931699)(443.5574585,79.09931936)
\curveto(443.60745672,79.02931733)(443.65245668,78.9593174)(443.6924585,78.88931936)
\curveto(443.74245659,78.82931753)(443.78745654,78.7543176)(443.8274585,78.66431936)
\curveto(443.83745649,78.64431771)(443.84745648,78.61431774)(443.8574585,78.57431936)
\curveto(443.87745645,78.53431782)(443.88245645,78.48931787)(443.8724585,78.43931936)
\curveto(443.84245649,78.34931801)(443.76745656,78.29431806)(443.6474585,78.27431936)
\curveto(443.53745679,78.2543181)(443.44245689,78.26931809)(443.3624585,78.31931936)
\curveto(443.29245704,78.34931801)(443.2274571,78.39431796)(443.1674585,78.45431936)
\curveto(443.11745721,78.52431783)(443.06745726,78.58931777)(443.0174585,78.64931936)
\curveto(442.96745736,78.71931764)(442.89245744,78.77931758)(442.7924585,78.82931936)
\curveto(442.70245763,78.88931747)(442.61245772,78.93931742)(442.5224585,78.97931936)
\curveto(442.49245784,78.99931736)(442.4324579,79.02431733)(442.3424585,79.05431936)
\curveto(442.26245807,79.08431727)(442.19245814,79.08931727)(442.1324585,79.06931936)
\curveto(441.99245834,79.03931732)(441.90245843,78.97931738)(441.8624585,78.88931936)
\curveto(441.8324585,78.80931755)(441.81745851,78.71931764)(441.8174585,78.61931936)
\curveto(441.81745851,78.51931784)(441.79245854,78.43431792)(441.7424585,78.36431936)
\curveto(441.67245866,78.27431808)(441.5324588,78.22931813)(441.3224585,78.22931936)
\lineto(440.7674585,78.22931936)
\lineto(440.5424585,78.22931936)
\curveto(440.46245987,78.23931812)(440.39745993,78.2593181)(440.3474585,78.28931936)
\curveto(440.26746006,78.34931801)(440.22246011,78.41931794)(440.2124585,78.49931936)
\curveto(440.20246013,78.51931784)(440.19746013,78.53931782)(440.1974585,78.55931936)
\curveto(440.19746013,78.58931777)(440.19246014,78.61431774)(440.1824585,78.63431936)
}
}
{
\newrgbcolor{curcolor}{0 0 0}
\pscustom[linestyle=none,fillstyle=solid,fillcolor=curcolor]
{
}
}
{
\newrgbcolor{curcolor}{0 0 0}
\pscustom[linestyle=none,fillstyle=solid,fillcolor=curcolor]
{
\newpath
\moveto(431.2124585,89.26463186)
\curveto(431.20246913,89.95462722)(431.32246901,90.55462662)(431.5724585,91.06463186)
\curveto(431.82246851,91.58462559)(432.15746817,91.9796252)(432.5774585,92.24963186)
\curveto(432.65746767,92.29962488)(432.74746758,92.34462483)(432.8474585,92.38463186)
\curveto(432.93746739,92.42462475)(433.0324673,92.46962471)(433.1324585,92.51963186)
\curveto(433.2324671,92.55962462)(433.332467,92.58962459)(433.4324585,92.60963186)
\curveto(433.5324668,92.62962455)(433.63746669,92.64962453)(433.7474585,92.66963186)
\curveto(433.79746653,92.68962449)(433.84246649,92.69462448)(433.8824585,92.68463186)
\curveto(433.92246641,92.6746245)(433.96746636,92.6796245)(434.0174585,92.69963186)
\curveto(434.06746626,92.70962447)(434.15246618,92.71462446)(434.2724585,92.71463186)
\curveto(434.38246595,92.71462446)(434.46746586,92.70962447)(434.5274585,92.69963186)
\curveto(434.58746574,92.6796245)(434.64746568,92.66962451)(434.7074585,92.66963186)
\curveto(434.76746556,92.6796245)(434.8274655,92.6746245)(434.8874585,92.65463186)
\curveto(435.0274653,92.61462456)(435.16246517,92.5796246)(435.2924585,92.54963186)
\curveto(435.42246491,92.51962466)(435.54746478,92.4796247)(435.6674585,92.42963186)
\curveto(435.80746452,92.36962481)(435.9324644,92.29962488)(436.0424585,92.21963186)
\curveto(436.15246418,92.14962503)(436.26246407,92.0746251)(436.3724585,91.99463186)
\lineto(436.4324585,91.93463186)
\curveto(436.45246388,91.92462525)(436.47246386,91.90962527)(436.4924585,91.88963186)
\curveto(436.65246368,91.76962541)(436.79746353,91.63462554)(436.9274585,91.48463186)
\curveto(437.05746327,91.33462584)(437.18246315,91.174626)(437.3024585,91.00463186)
\curveto(437.52246281,90.69462648)(437.7274626,90.39962678)(437.9174585,90.11963186)
\curveto(438.05746227,89.88962729)(438.19246214,89.65962752)(438.3224585,89.42963186)
\curveto(438.45246188,89.20962797)(438.58746174,88.98962819)(438.7274585,88.76963186)
\curveto(438.89746143,88.51962866)(439.07746125,88.2796289)(439.2674585,88.04963186)
\curveto(439.45746087,87.82962935)(439.68246065,87.63962954)(439.9424585,87.47963186)
\curveto(440.00246033,87.43962974)(440.06246027,87.40462977)(440.1224585,87.37463186)
\curveto(440.17246016,87.34462983)(440.23746009,87.31462986)(440.3174585,87.28463186)
\curveto(440.38745994,87.26462991)(440.44745988,87.25962992)(440.4974585,87.26963186)
\curveto(440.56745976,87.28962989)(440.62245971,87.32462985)(440.6624585,87.37463186)
\curveto(440.69245964,87.42462975)(440.71245962,87.48462969)(440.7224585,87.55463186)
\lineto(440.7224585,87.79463186)
\lineto(440.7224585,88.54463186)
\lineto(440.7224585,91.34963186)
\lineto(440.7224585,92.00963186)
\curveto(440.72245961,92.09962508)(440.7274596,92.18462499)(440.7374585,92.26463186)
\curveto(440.73745959,92.34462483)(440.75745957,92.40962477)(440.7974585,92.45963186)
\curveto(440.83745949,92.50962467)(440.91245942,92.54962463)(441.0224585,92.57963186)
\curveto(441.12245921,92.61962456)(441.22245911,92.62962455)(441.3224585,92.60963186)
\lineto(441.4574585,92.60963186)
\curveto(441.5274588,92.58962459)(441.58745874,92.56962461)(441.6374585,92.54963186)
\curveto(441.68745864,92.52962465)(441.7274586,92.49462468)(441.7574585,92.44463186)
\curveto(441.79745853,92.39462478)(441.81745851,92.32462485)(441.8174585,92.23463186)
\lineto(441.8174585,91.96463186)
\lineto(441.8174585,91.06463186)
\lineto(441.8174585,87.55463186)
\lineto(441.8174585,86.48963186)
\curveto(441.81745851,86.40963077)(441.82245851,86.31963086)(441.8324585,86.21963186)
\curveto(441.8324585,86.11963106)(441.82245851,86.03463114)(441.8024585,85.96463186)
\curveto(441.7324586,85.75463142)(441.55245878,85.68963149)(441.2624585,85.76963186)
\curveto(441.22245911,85.7796314)(441.18745914,85.7796314)(441.1574585,85.76963186)
\curveto(441.11745921,85.76963141)(441.07245926,85.7796314)(441.0224585,85.79963186)
\curveto(440.94245939,85.81963136)(440.85745947,85.83963134)(440.7674585,85.85963186)
\curveto(440.67745965,85.8796313)(440.59245974,85.90463127)(440.5124585,85.93463186)
\curveto(440.02246031,86.09463108)(439.60746072,86.29463088)(439.2674585,86.53463186)
\curveto(439.01746131,86.71463046)(438.79246154,86.91963026)(438.5924585,87.14963186)
\curveto(438.38246195,87.3796298)(438.18746214,87.61962956)(438.0074585,87.86963186)
\curveto(437.8274625,88.12962905)(437.65746267,88.39462878)(437.4974585,88.66463186)
\curveto(437.327463,88.94462823)(437.15246318,89.21462796)(436.9724585,89.47463186)
\curveto(436.89246344,89.58462759)(436.81746351,89.68962749)(436.7474585,89.78963186)
\curveto(436.67746365,89.89962728)(436.60246373,90.00962717)(436.5224585,90.11963186)
\curveto(436.49246384,90.15962702)(436.46246387,90.19462698)(436.4324585,90.22463186)
\curveto(436.39246394,90.26462691)(436.36246397,90.30462687)(436.3424585,90.34463186)
\curveto(436.2324641,90.48462669)(436.10746422,90.60962657)(435.9674585,90.71963186)
\curveto(435.93746439,90.73962644)(435.91246442,90.76462641)(435.8924585,90.79463186)
\curveto(435.86246447,90.82462635)(435.8324645,90.84962633)(435.8024585,90.86963186)
\curveto(435.70246463,90.94962623)(435.60246473,91.01462616)(435.5024585,91.06463186)
\curveto(435.40246493,91.12462605)(435.29246504,91.179626)(435.1724585,91.22963186)
\curveto(435.10246523,91.25962592)(435.0274653,91.2796259)(434.9474585,91.28963186)
\lineto(434.7074585,91.34963186)
\lineto(434.6174585,91.34963186)
\curveto(434.58746574,91.35962582)(434.55746577,91.36462581)(434.5274585,91.36463186)
\curveto(434.45746587,91.38462579)(434.36246597,91.38962579)(434.2424585,91.37963186)
\curveto(434.11246622,91.3796258)(434.01246632,91.36962581)(433.9424585,91.34963186)
\curveto(433.86246647,91.32962585)(433.78746654,91.30962587)(433.7174585,91.28963186)
\curveto(433.63746669,91.2796259)(433.55746677,91.25962592)(433.4774585,91.22963186)
\curveto(433.23746709,91.11962606)(433.03746729,90.96962621)(432.8774585,90.77963186)
\curveto(432.70746762,90.59962658)(432.56746776,90.3796268)(432.4574585,90.11963186)
\curveto(432.43746789,90.04962713)(432.42246791,89.9796272)(432.4124585,89.90963186)
\curveto(432.39246794,89.83962734)(432.37246796,89.76462741)(432.3524585,89.68463186)
\curveto(432.332468,89.60462757)(432.32246801,89.49462768)(432.3224585,89.35463186)
\curveto(432.32246801,89.22462795)(432.332468,89.11962806)(432.3524585,89.03963186)
\curveto(432.36246797,88.9796282)(432.36746796,88.92462825)(432.3674585,88.87463186)
\curveto(432.36746796,88.82462835)(432.37746795,88.7746284)(432.3974585,88.72463186)
\curveto(432.43746789,88.62462855)(432.47746785,88.52962865)(432.5174585,88.43963186)
\curveto(432.55746777,88.35962882)(432.60246773,88.2796289)(432.6524585,88.19963186)
\curveto(432.67246766,88.16962901)(432.69746763,88.13962904)(432.7274585,88.10963186)
\curveto(432.75746757,88.08962909)(432.78246755,88.06462911)(432.8024585,88.03463186)
\lineto(432.8774585,87.95963186)
\curveto(432.89746743,87.92962925)(432.91746741,87.90462927)(432.9374585,87.88463186)
\lineto(433.1474585,87.73463186)
\curveto(433.20746712,87.69462948)(433.27246706,87.64962953)(433.3424585,87.59963186)
\curveto(433.4324669,87.53962964)(433.53746679,87.48962969)(433.6574585,87.44963186)
\curveto(433.76746656,87.41962976)(433.87746645,87.38462979)(433.9874585,87.34463186)
\curveto(434.09746623,87.30462987)(434.24246609,87.2796299)(434.4224585,87.26963186)
\curveto(434.59246574,87.25962992)(434.71746561,87.22962995)(434.7974585,87.17963186)
\curveto(434.87746545,87.12963005)(434.92246541,87.05463012)(434.9324585,86.95463186)
\curveto(434.94246539,86.85463032)(434.94746538,86.74463043)(434.9474585,86.62463186)
\curveto(434.94746538,86.58463059)(434.95246538,86.54463063)(434.9624585,86.50463186)
\curveto(434.96246537,86.46463071)(434.95746537,86.42963075)(434.9474585,86.39963186)
\curveto(434.9274654,86.34963083)(434.91746541,86.29963088)(434.9174585,86.24963186)
\curveto(434.91746541,86.20963097)(434.90746542,86.16963101)(434.8874585,86.12963186)
\curveto(434.8274655,86.03963114)(434.69246564,85.99463118)(434.4824585,85.99463186)
\lineto(434.3624585,85.99463186)
\curveto(434.30246603,86.00463117)(434.24246609,86.00963117)(434.1824585,86.00963186)
\curveto(434.11246622,86.01963116)(434.04746628,86.02963115)(433.9874585,86.03963186)
\curveto(433.87746645,86.05963112)(433.77746655,86.0796311)(433.6874585,86.09963186)
\curveto(433.58746674,86.11963106)(433.49246684,86.14963103)(433.4024585,86.18963186)
\curveto(433.332467,86.20963097)(433.27246706,86.22963095)(433.2224585,86.24963186)
\lineto(433.0424585,86.30963186)
\curveto(432.78246755,86.42963075)(432.53746779,86.58463059)(432.3074585,86.77463186)
\curveto(432.07746825,86.9746302)(431.89246844,87.18962999)(431.7524585,87.41963186)
\curveto(431.67246866,87.52962965)(431.60746872,87.64462953)(431.5574585,87.76463186)
\lineto(431.4074585,88.15463186)
\curveto(431.35746897,88.26462891)(431.327469,88.3796288)(431.3174585,88.49963186)
\curveto(431.29746903,88.61962856)(431.27246906,88.74462843)(431.2424585,88.87463186)
\curveto(431.24246909,88.94462823)(431.24246909,89.00962817)(431.2424585,89.06963186)
\curveto(431.2324691,89.12962805)(431.22246911,89.19462798)(431.2124585,89.26463186)
}
}
{
\newrgbcolor{curcolor}{0 0 0}
\pscustom[linestyle=none,fillstyle=solid,fillcolor=curcolor]
{
\newpath
\moveto(436.7324585,101.36424123)
\lineto(436.9874585,101.36424123)
\curveto(437.06746326,101.37423353)(437.14246319,101.36923353)(437.2124585,101.34924123)
\lineto(437.4524585,101.34924123)
\lineto(437.6174585,101.34924123)
\curveto(437.71746261,101.32923357)(437.82246251,101.31923358)(437.9324585,101.31924123)
\curveto(438.0324623,101.31923358)(438.1324622,101.30923359)(438.2324585,101.28924123)
\lineto(438.3824585,101.28924123)
\curveto(438.52246181,101.25923364)(438.66246167,101.23923366)(438.8024585,101.22924123)
\curveto(438.9324614,101.21923368)(439.06246127,101.19423371)(439.1924585,101.15424123)
\curveto(439.27246106,101.13423377)(439.35746097,101.11423379)(439.4474585,101.09424123)
\lineto(439.6874585,101.03424123)
\lineto(439.9874585,100.91424123)
\curveto(440.07746025,100.88423402)(440.16746016,100.84923405)(440.2574585,100.80924123)
\curveto(440.47745985,100.70923419)(440.69245964,100.57423433)(440.9024585,100.40424123)
\curveto(441.11245922,100.24423466)(441.28245905,100.06923483)(441.4124585,99.87924123)
\curveto(441.45245888,99.82923507)(441.49245884,99.76923513)(441.5324585,99.69924123)
\curveto(441.56245877,99.63923526)(441.59745873,99.57923532)(441.6374585,99.51924123)
\curveto(441.68745864,99.43923546)(441.7274586,99.34423556)(441.7574585,99.23424123)
\curveto(441.78745854,99.12423578)(441.81745851,99.01923588)(441.8474585,98.91924123)
\curveto(441.88745844,98.80923609)(441.91245842,98.6992362)(441.9224585,98.58924123)
\curveto(441.9324584,98.47923642)(441.94745838,98.36423654)(441.9674585,98.24424123)
\curveto(441.97745835,98.2042367)(441.97745835,98.15923674)(441.9674585,98.10924123)
\curveto(441.96745836,98.06923683)(441.97245836,98.02923687)(441.9824585,97.98924123)
\curveto(441.99245834,97.94923695)(441.99745833,97.89423701)(441.9974585,97.82424123)
\curveto(441.99745833,97.75423715)(441.99245834,97.7042372)(441.9824585,97.67424123)
\curveto(441.96245837,97.62423728)(441.95745837,97.57923732)(441.9674585,97.53924123)
\curveto(441.97745835,97.4992374)(441.97745835,97.46423744)(441.9674585,97.43424123)
\lineto(441.9674585,97.34424123)
\curveto(441.94745838,97.28423762)(441.9324584,97.21923768)(441.9224585,97.14924123)
\curveto(441.92245841,97.08923781)(441.91745841,97.02423788)(441.9074585,96.95424123)
\curveto(441.85745847,96.78423812)(441.80745852,96.62423828)(441.7574585,96.47424123)
\curveto(441.70745862,96.32423858)(441.64245869,96.17923872)(441.5624585,96.03924123)
\curveto(441.52245881,95.98923891)(441.49245884,95.93423897)(441.4724585,95.87424123)
\curveto(441.44245889,95.82423908)(441.40745892,95.77423913)(441.3674585,95.72424123)
\curveto(441.18745914,95.48423942)(440.96745936,95.28423962)(440.7074585,95.12424123)
\curveto(440.44745988,94.96423994)(440.16246017,94.82424008)(439.8524585,94.70424123)
\curveto(439.71246062,94.64424026)(439.57246076,94.5992403)(439.4324585,94.56924123)
\curveto(439.28246105,94.53924036)(439.1274612,94.5042404)(438.9674585,94.46424123)
\curveto(438.85746147,94.44424046)(438.74746158,94.42924047)(438.6374585,94.41924123)
\curveto(438.5274618,94.40924049)(438.41746191,94.39424051)(438.3074585,94.37424123)
\curveto(438.26746206,94.36424054)(438.2274621,94.35924054)(438.1874585,94.35924123)
\curveto(438.14746218,94.36924053)(438.10746222,94.36924053)(438.0674585,94.35924123)
\curveto(438.01746231,94.34924055)(437.96746236,94.34424056)(437.9174585,94.34424123)
\lineto(437.7524585,94.34424123)
\curveto(437.70246263,94.32424058)(437.65246268,94.31924058)(437.6024585,94.32924123)
\curveto(437.54246279,94.33924056)(437.48746284,94.33924056)(437.4374585,94.32924123)
\curveto(437.39746293,94.31924058)(437.35246298,94.31924058)(437.3024585,94.32924123)
\curveto(437.25246308,94.33924056)(437.20246313,94.33424057)(437.1524585,94.31424123)
\curveto(437.08246325,94.29424061)(437.00746332,94.28924061)(436.9274585,94.29924123)
\curveto(436.83746349,94.30924059)(436.75246358,94.31424059)(436.6724585,94.31424123)
\curveto(436.58246375,94.31424059)(436.48246385,94.30924059)(436.3724585,94.29924123)
\curveto(436.25246408,94.28924061)(436.15246418,94.29424061)(436.0724585,94.31424123)
\lineto(435.7874585,94.31424123)
\lineto(435.1574585,94.35924123)
\curveto(435.05746527,94.36924053)(434.96246537,94.37924052)(434.8724585,94.38924123)
\lineto(434.5724585,94.41924123)
\curveto(434.52246581,94.43924046)(434.47246586,94.44424046)(434.4224585,94.43424123)
\curveto(434.36246597,94.43424047)(434.30746602,94.44424046)(434.2574585,94.46424123)
\curveto(434.08746624,94.51424039)(433.92246641,94.55424035)(433.7624585,94.58424123)
\curveto(433.59246674,94.61424029)(433.4324669,94.66424024)(433.2824585,94.73424123)
\curveto(432.82246751,94.92423998)(432.44746788,95.14423976)(432.1574585,95.39424123)
\curveto(431.86746846,95.65423925)(431.62246871,96.01423889)(431.4224585,96.47424123)
\curveto(431.37246896,96.6042383)(431.33746899,96.73423817)(431.3174585,96.86424123)
\curveto(431.29746903,97.0042379)(431.27246906,97.14423776)(431.2424585,97.28424123)
\curveto(431.2324691,97.35423755)(431.2274691,97.41923748)(431.2274585,97.47924123)
\curveto(431.2274691,97.53923736)(431.22246911,97.6042373)(431.2124585,97.67424123)
\curveto(431.19246914,98.5042364)(431.34246899,99.17423573)(431.6624585,99.68424123)
\curveto(431.97246836,100.19423471)(432.41246792,100.57423433)(432.9824585,100.82424123)
\curveto(433.10246723,100.87423403)(433.2274671,100.91923398)(433.3574585,100.95924123)
\curveto(433.48746684,100.9992339)(433.62246671,101.04423386)(433.7624585,101.09424123)
\curveto(433.84246649,101.11423379)(433.9274664,101.12923377)(434.0174585,101.13924123)
\lineto(434.2574585,101.19924123)
\curveto(434.36746596,101.22923367)(434.47746585,101.24423366)(434.5874585,101.24424123)
\curveto(434.69746563,101.25423365)(434.80746552,101.26923363)(434.9174585,101.28924123)
\curveto(434.96746536,101.30923359)(435.01246532,101.31423359)(435.0524585,101.30424123)
\curveto(435.09246524,101.3042336)(435.1324652,101.30923359)(435.1724585,101.31924123)
\curveto(435.22246511,101.32923357)(435.27746505,101.32923357)(435.3374585,101.31924123)
\curveto(435.38746494,101.31923358)(435.43746489,101.32423358)(435.4874585,101.33424123)
\lineto(435.6224585,101.33424123)
\curveto(435.68246465,101.35423355)(435.75246458,101.35423355)(435.8324585,101.33424123)
\curveto(435.90246443,101.32423358)(435.96746436,101.32923357)(436.0274585,101.34924123)
\curveto(436.05746427,101.35923354)(436.09746423,101.36423354)(436.1474585,101.36424123)
\lineto(436.2674585,101.36424123)
\lineto(436.7324585,101.36424123)
\moveto(439.0574585,99.81924123)
\curveto(438.73746159,99.91923498)(438.37246196,99.97923492)(437.9624585,99.99924123)
\curveto(437.55246278,100.01923488)(437.14246319,100.02923487)(436.7324585,100.02924123)
\curveto(436.30246403,100.02923487)(435.88246445,100.01923488)(435.4724585,99.99924123)
\curveto(435.06246527,99.97923492)(434.67746565,99.93423497)(434.3174585,99.86424123)
\curveto(433.95746637,99.79423511)(433.63746669,99.68423522)(433.3574585,99.53424123)
\curveto(433.06746726,99.39423551)(432.8324675,99.1992357)(432.6524585,98.94924123)
\curveto(432.54246779,98.78923611)(432.46246787,98.60923629)(432.4124585,98.40924123)
\curveto(432.35246798,98.20923669)(432.32246801,97.96423694)(432.3224585,97.67424123)
\curveto(432.34246799,97.65423725)(432.35246798,97.61923728)(432.3524585,97.56924123)
\curveto(432.34246799,97.51923738)(432.34246799,97.47923742)(432.3524585,97.44924123)
\curveto(432.37246796,97.36923753)(432.39246794,97.29423761)(432.4124585,97.22424123)
\curveto(432.42246791,97.16423774)(432.44246789,97.0992378)(432.4724585,97.02924123)
\curveto(432.59246774,96.75923814)(432.76246757,96.53923836)(432.9824585,96.36924123)
\curveto(433.19246714,96.20923869)(433.43746689,96.07423883)(433.7174585,95.96424123)
\curveto(433.8274665,95.91423899)(433.94746638,95.87423903)(434.0774585,95.84424123)
\curveto(434.19746613,95.82423908)(434.32246601,95.7992391)(434.4524585,95.76924123)
\curveto(434.50246583,95.74923915)(434.55746577,95.73923916)(434.6174585,95.73924123)
\curveto(434.66746566,95.73923916)(434.71746561,95.73423917)(434.7674585,95.72424123)
\curveto(434.85746547,95.71423919)(434.95246538,95.7042392)(435.0524585,95.69424123)
\curveto(435.14246519,95.68423922)(435.23746509,95.67423923)(435.3374585,95.66424123)
\curveto(435.41746491,95.66423924)(435.50246483,95.65923924)(435.5924585,95.64924123)
\lineto(435.8324585,95.64924123)
\lineto(436.0124585,95.64924123)
\curveto(436.04246429,95.63923926)(436.07746425,95.63423927)(436.1174585,95.63424123)
\lineto(436.2524585,95.63424123)
\lineto(436.7024585,95.63424123)
\curveto(436.78246355,95.63423927)(436.86746346,95.62923927)(436.9574585,95.61924123)
\curveto(437.03746329,95.61923928)(437.11246322,95.62923927)(437.1824585,95.64924123)
\lineto(437.4524585,95.64924123)
\curveto(437.47246286,95.64923925)(437.50246283,95.64423926)(437.5424585,95.63424123)
\curveto(437.57246276,95.63423927)(437.59746273,95.63923926)(437.6174585,95.64924123)
\curveto(437.71746261,95.65923924)(437.81746251,95.66423924)(437.9174585,95.66424123)
\curveto(438.00746232,95.67423923)(438.10746222,95.68423922)(438.2174585,95.69424123)
\curveto(438.33746199,95.72423918)(438.46246187,95.73923916)(438.5924585,95.73924123)
\curveto(438.71246162,95.74923915)(438.8274615,95.77423913)(438.9374585,95.81424123)
\curveto(439.23746109,95.89423901)(439.50246083,95.97923892)(439.7324585,96.06924123)
\curveto(439.96246037,96.16923873)(440.17746015,96.31423859)(440.3774585,96.50424123)
\curveto(440.57745975,96.71423819)(440.7274596,96.97923792)(440.8274585,97.29924123)
\curveto(440.84745948,97.33923756)(440.85745947,97.37423753)(440.8574585,97.40424123)
\curveto(440.84745948,97.44423746)(440.85245948,97.48923741)(440.8724585,97.53924123)
\curveto(440.88245945,97.57923732)(440.89245944,97.64923725)(440.9024585,97.74924123)
\curveto(440.91245942,97.85923704)(440.90745942,97.94423696)(440.8874585,98.00424123)
\curveto(440.86745946,98.07423683)(440.85745947,98.14423676)(440.8574585,98.21424123)
\curveto(440.84745948,98.28423662)(440.8324595,98.34923655)(440.8124585,98.40924123)
\curveto(440.75245958,98.60923629)(440.66745966,98.78923611)(440.5574585,98.94924123)
\curveto(440.53745979,98.97923592)(440.51745981,99.0042359)(440.4974585,99.02424123)
\lineto(440.4374585,99.08424123)
\curveto(440.41745991,99.12423578)(440.37745995,99.17423573)(440.3174585,99.23424123)
\curveto(440.17746015,99.33423557)(440.04746028,99.41923548)(439.9274585,99.48924123)
\curveto(439.80746052,99.55923534)(439.66246067,99.62923527)(439.4924585,99.69924123)
\curveto(439.42246091,99.72923517)(439.35246098,99.74923515)(439.2824585,99.75924123)
\curveto(439.21246112,99.77923512)(439.13746119,99.7992351)(439.0574585,99.81924123)
}
}
{
\newrgbcolor{curcolor}{0 0 0}
\pscustom[linestyle=none,fillstyle=solid,fillcolor=curcolor]
{
\newpath
\moveto(431.2124585,106.77385061)
\curveto(431.21246912,106.87384575)(431.22246911,106.96884566)(431.2424585,107.05885061)
\curveto(431.25246908,107.14884548)(431.28246905,107.21384541)(431.3324585,107.25385061)
\curveto(431.41246892,107.31384531)(431.51746881,107.34384528)(431.6474585,107.34385061)
\lineto(432.0374585,107.34385061)
\lineto(433.5374585,107.34385061)
\lineto(439.9274585,107.34385061)
\lineto(441.0974585,107.34385061)
\lineto(441.4124585,107.34385061)
\curveto(441.51245882,107.35384527)(441.59245874,107.33884529)(441.6524585,107.29885061)
\curveto(441.7324586,107.24884538)(441.78245855,107.17384545)(441.8024585,107.07385061)
\curveto(441.81245852,106.98384564)(441.81745851,106.87384575)(441.8174585,106.74385061)
\lineto(441.8174585,106.51885061)
\curveto(441.79745853,106.43884619)(441.78245855,106.36884626)(441.7724585,106.30885061)
\curveto(441.75245858,106.24884638)(441.71245862,106.19884643)(441.6524585,106.15885061)
\curveto(441.59245874,106.11884651)(441.51745881,106.09884653)(441.4274585,106.09885061)
\lineto(441.1274585,106.09885061)
\lineto(440.0324585,106.09885061)
\lineto(434.6924585,106.09885061)
\curveto(434.60246573,106.07884655)(434.5274658,106.06384656)(434.4674585,106.05385061)
\curveto(434.39746593,106.05384657)(434.33746599,106.0238466)(434.2874585,105.96385061)
\curveto(434.23746609,105.89384673)(434.21246612,105.80384682)(434.2124585,105.69385061)
\curveto(434.20246613,105.59384703)(434.19746613,105.48384714)(434.1974585,105.36385061)
\lineto(434.1974585,104.22385061)
\lineto(434.1974585,103.72885061)
\curveto(434.18746614,103.56884906)(434.1274662,103.45884917)(434.0174585,103.39885061)
\curveto(433.98746634,103.37884925)(433.95746637,103.36884926)(433.9274585,103.36885061)
\curveto(433.88746644,103.36884926)(433.84246649,103.36384926)(433.7924585,103.35385061)
\curveto(433.67246666,103.33384929)(433.56246677,103.33884929)(433.4624585,103.36885061)
\curveto(433.36246697,103.40884922)(433.29246704,103.46384916)(433.2524585,103.53385061)
\curveto(433.20246713,103.61384901)(433.17746715,103.73384889)(433.1774585,103.89385061)
\curveto(433.17746715,104.05384857)(433.16246717,104.18884844)(433.1324585,104.29885061)
\curveto(433.12246721,104.34884828)(433.11746721,104.40384822)(433.1174585,104.46385061)
\curveto(433.10746722,104.5238481)(433.09246724,104.58384804)(433.0724585,104.64385061)
\curveto(433.02246731,104.79384783)(432.97246736,104.93884769)(432.9224585,105.07885061)
\curveto(432.86246747,105.21884741)(432.79246754,105.35384727)(432.7124585,105.48385061)
\curveto(432.62246771,105.623847)(432.51746781,105.74384688)(432.3974585,105.84385061)
\curveto(432.27746805,105.94384668)(432.14746818,106.03884659)(432.0074585,106.12885061)
\curveto(431.90746842,106.18884644)(431.79746853,106.23384639)(431.6774585,106.26385061)
\curveto(431.55746877,106.30384632)(431.45246888,106.35384627)(431.3624585,106.41385061)
\curveto(431.30246903,106.46384616)(431.26246907,106.53384609)(431.2424585,106.62385061)
\curveto(431.2324691,106.64384598)(431.2274691,106.66884596)(431.2274585,106.69885061)
\curveto(431.2274691,106.7288459)(431.22246911,106.75384587)(431.2124585,106.77385061)
}
}
{
\newrgbcolor{curcolor}{0 0 0}
\pscustom[linestyle=none,fillstyle=solid,fillcolor=curcolor]
{
\newpath
\moveto(431.2124585,115.12345998)
\curveto(431.21246912,115.22345513)(431.22246911,115.31845503)(431.2424585,115.40845998)
\curveto(431.25246908,115.49845485)(431.28246905,115.56345479)(431.3324585,115.60345998)
\curveto(431.41246892,115.66345469)(431.51746881,115.69345466)(431.6474585,115.69345998)
\lineto(432.0374585,115.69345998)
\lineto(433.5374585,115.69345998)
\lineto(439.9274585,115.69345998)
\lineto(441.0974585,115.69345998)
\lineto(441.4124585,115.69345998)
\curveto(441.51245882,115.70345465)(441.59245874,115.68845466)(441.6524585,115.64845998)
\curveto(441.7324586,115.59845475)(441.78245855,115.52345483)(441.8024585,115.42345998)
\curveto(441.81245852,115.33345502)(441.81745851,115.22345513)(441.8174585,115.09345998)
\lineto(441.8174585,114.86845998)
\curveto(441.79745853,114.78845556)(441.78245855,114.71845563)(441.7724585,114.65845998)
\curveto(441.75245858,114.59845575)(441.71245862,114.5484558)(441.6524585,114.50845998)
\curveto(441.59245874,114.46845588)(441.51745881,114.4484559)(441.4274585,114.44845998)
\lineto(441.1274585,114.44845998)
\lineto(440.0324585,114.44845998)
\lineto(434.6924585,114.44845998)
\curveto(434.60246573,114.42845592)(434.5274658,114.41345594)(434.4674585,114.40345998)
\curveto(434.39746593,114.40345595)(434.33746599,114.37345598)(434.2874585,114.31345998)
\curveto(434.23746609,114.24345611)(434.21246612,114.1534562)(434.2124585,114.04345998)
\curveto(434.20246613,113.94345641)(434.19746613,113.83345652)(434.1974585,113.71345998)
\lineto(434.1974585,112.57345998)
\lineto(434.1974585,112.07845998)
\curveto(434.18746614,111.91845843)(434.1274662,111.80845854)(434.0174585,111.74845998)
\curveto(433.98746634,111.72845862)(433.95746637,111.71845863)(433.9274585,111.71845998)
\curveto(433.88746644,111.71845863)(433.84246649,111.71345864)(433.7924585,111.70345998)
\curveto(433.67246666,111.68345867)(433.56246677,111.68845866)(433.4624585,111.71845998)
\curveto(433.36246697,111.75845859)(433.29246704,111.81345854)(433.2524585,111.88345998)
\curveto(433.20246713,111.96345839)(433.17746715,112.08345827)(433.1774585,112.24345998)
\curveto(433.17746715,112.40345795)(433.16246717,112.53845781)(433.1324585,112.64845998)
\curveto(433.12246721,112.69845765)(433.11746721,112.7534576)(433.1174585,112.81345998)
\curveto(433.10746722,112.87345748)(433.09246724,112.93345742)(433.0724585,112.99345998)
\curveto(433.02246731,113.14345721)(432.97246736,113.28845706)(432.9224585,113.42845998)
\curveto(432.86246747,113.56845678)(432.79246754,113.70345665)(432.7124585,113.83345998)
\curveto(432.62246771,113.97345638)(432.51746781,114.09345626)(432.3974585,114.19345998)
\curveto(432.27746805,114.29345606)(432.14746818,114.38845596)(432.0074585,114.47845998)
\curveto(431.90746842,114.53845581)(431.79746853,114.58345577)(431.6774585,114.61345998)
\curveto(431.55746877,114.6534557)(431.45246888,114.70345565)(431.3624585,114.76345998)
\curveto(431.30246903,114.81345554)(431.26246907,114.88345547)(431.2424585,114.97345998)
\curveto(431.2324691,114.99345536)(431.2274691,115.01845533)(431.2274585,115.04845998)
\curveto(431.2274691,115.07845527)(431.22246911,115.10345525)(431.2124585,115.12345998)
}
}
{
\newrgbcolor{curcolor}{0 0 0}
\pscustom[linestyle=none,fillstyle=solid,fillcolor=curcolor]
{
\newpath
\moveto(461.94880493,42.02236623)
\curveto(461.99880568,42.04235669)(462.05880562,42.06735666)(462.12880493,42.09736623)
\curveto(462.19880548,42.1273566)(462.2738054,42.14735658)(462.35380493,42.15736623)
\curveto(462.42380525,42.17735655)(462.49380518,42.17735655)(462.56380493,42.15736623)
\curveto(462.62380505,42.14735658)(462.66880501,42.10735662)(462.69880493,42.03736623)
\curveto(462.71880496,41.98735674)(462.72880495,41.9273568)(462.72880493,41.85736623)
\lineto(462.72880493,41.64736623)
\lineto(462.72880493,41.19736623)
\curveto(462.72880495,41.04735768)(462.70380497,40.9273578)(462.65380493,40.83736623)
\curveto(462.59380508,40.73735799)(462.48880519,40.66235807)(462.33880493,40.61236623)
\curveto(462.18880549,40.57235816)(462.05380562,40.5273582)(461.93380493,40.47736623)
\curveto(461.673806,40.36735836)(461.40380627,40.26735846)(461.12380493,40.17736623)
\curveto(460.84380683,40.08735864)(460.56880711,39.98735874)(460.29880493,39.87736623)
\curveto(460.20880747,39.84735888)(460.12380755,39.81735891)(460.04380493,39.78736623)
\curveto(459.96380771,39.76735896)(459.88880779,39.73735899)(459.81880493,39.69736623)
\curveto(459.74880793,39.66735906)(459.68880799,39.62235911)(459.63880493,39.56236623)
\curveto(459.58880809,39.50235923)(459.54880813,39.42235931)(459.51880493,39.32236623)
\curveto(459.49880818,39.27235946)(459.49380818,39.21235952)(459.50380493,39.14236623)
\lineto(459.50380493,38.94736623)
\lineto(459.50380493,36.11236623)
\lineto(459.50380493,35.81236623)
\curveto(459.49380818,35.70236303)(459.49380818,35.59736313)(459.50380493,35.49736623)
\curveto(459.51380816,35.39736333)(459.52880815,35.30236343)(459.54880493,35.21236623)
\curveto(459.56880811,35.1323636)(459.60880807,35.07236366)(459.66880493,35.03236623)
\curveto(459.76880791,34.95236378)(459.88380779,34.89236384)(460.01380493,34.85236623)
\curveto(460.13380754,34.82236391)(460.25880742,34.78236395)(460.38880493,34.73236623)
\curveto(460.61880706,34.6323641)(460.85880682,34.53736419)(461.10880493,34.44736623)
\curveto(461.35880632,34.36736436)(461.59880608,34.27736445)(461.82880493,34.17736623)
\curveto(461.88880579,34.15736457)(461.95880572,34.1323646)(462.03880493,34.10236623)
\curveto(462.10880557,34.08236465)(462.18380549,34.05736467)(462.26380493,34.02736623)
\curveto(462.34380533,33.99736473)(462.41880526,33.96236477)(462.48880493,33.92236623)
\curveto(462.54880513,33.89236484)(462.59380508,33.85736487)(462.62380493,33.81736623)
\curveto(462.68380499,33.73736499)(462.71880496,33.6273651)(462.72880493,33.48736623)
\lineto(462.72880493,33.06736623)
\lineto(462.72880493,32.82736623)
\curveto(462.71880496,32.75736597)(462.69380498,32.69736603)(462.65380493,32.64736623)
\curveto(462.62380505,32.59736613)(462.5788051,32.56736616)(462.51880493,32.55736623)
\curveto(462.45880522,32.55736617)(462.39880528,32.56236617)(462.33880493,32.57236623)
\curveto(462.26880541,32.59236614)(462.20380547,32.61236612)(462.14380493,32.63236623)
\curveto(462.0738056,32.66236607)(462.02380565,32.68736604)(461.99380493,32.70736623)
\curveto(461.673806,32.84736588)(461.35880632,32.97236576)(461.04880493,33.08236623)
\curveto(460.72880695,33.19236554)(460.40880727,33.31236542)(460.08880493,33.44236623)
\curveto(459.86880781,33.5323652)(459.65380802,33.61736511)(459.44380493,33.69736623)
\curveto(459.22380845,33.77736495)(459.00380867,33.86236487)(458.78380493,33.95236623)
\curveto(458.06380961,34.25236448)(457.33881034,34.53736419)(456.60880493,34.80736623)
\curveto(455.86881181,35.07736365)(455.13381254,35.36236337)(454.40380493,35.66236623)
\curveto(454.14381353,35.77236296)(453.8788138,35.87236286)(453.60880493,35.96236623)
\curveto(453.33881434,36.06236267)(453.0738146,36.16736256)(452.81380493,36.27736623)
\curveto(452.70381497,36.3273624)(452.58381509,36.37236236)(452.45380493,36.41236623)
\curveto(452.31381536,36.46236227)(452.21381546,36.5323622)(452.15380493,36.62236623)
\curveto(452.11381556,36.66236207)(452.08381559,36.727362)(452.06380493,36.81736623)
\curveto(452.05381562,36.83736189)(452.05381562,36.85736187)(452.06380493,36.87736623)
\curveto(452.06381561,36.90736182)(452.05881562,36.9323618)(452.04880493,36.95236623)
\curveto(452.04881563,37.1323616)(452.04881563,37.34236139)(452.04880493,37.58236623)
\curveto(452.03881564,37.82236091)(452.0738156,37.99736073)(452.15380493,38.10736623)
\curveto(452.21381546,38.18736054)(452.31381536,38.24736048)(452.45380493,38.28736623)
\curveto(452.58381509,38.33736039)(452.70381497,38.38736034)(452.81380493,38.43736623)
\curveto(453.04381463,38.53736019)(453.2738144,38.6273601)(453.50380493,38.70736623)
\curveto(453.73381394,38.78735994)(453.96381371,38.87735985)(454.19380493,38.97736623)
\curveto(454.39381328,39.05735967)(454.59881308,39.1323596)(454.80880493,39.20236623)
\curveto(455.01881266,39.28235945)(455.22381245,39.36735936)(455.42380493,39.45736623)
\curveto(456.15381152,39.75735897)(456.89381078,40.04235869)(457.64380493,40.31236623)
\curveto(458.38380929,40.59235814)(459.11880856,40.88735784)(459.84880493,41.19736623)
\curveto(459.93880774,41.23735749)(460.02380765,41.26735746)(460.10380493,41.28736623)
\curveto(460.18380749,41.31735741)(460.26880741,41.34735738)(460.35880493,41.37736623)
\curveto(460.61880706,41.48735724)(460.88380679,41.59235714)(461.15380493,41.69236623)
\curveto(461.42380625,41.80235693)(461.68880599,41.91235682)(461.94880493,42.02236623)
\moveto(458.30380493,38.81236623)
\curveto(458.2738094,38.90235983)(458.22380945,38.95735977)(458.15380493,38.97736623)
\curveto(458.08380959,39.00735972)(458.00880967,39.01235972)(457.92880493,38.99236623)
\curveto(457.83880984,38.98235975)(457.75380992,38.95735977)(457.67380493,38.91736623)
\curveto(457.58381009,38.88735984)(457.50881017,38.85735987)(457.44880493,38.82736623)
\curveto(457.40881027,38.80735992)(457.3738103,38.79735993)(457.34380493,38.79736623)
\curveto(457.31381036,38.79735993)(457.2788104,38.78735994)(457.23880493,38.76736623)
\lineto(456.99880493,38.67736623)
\curveto(456.90881077,38.65736007)(456.81881086,38.6273601)(456.72880493,38.58736623)
\curveto(456.36881131,38.43736029)(456.00381167,38.30236043)(455.63380493,38.18236623)
\curveto(455.25381242,38.07236066)(454.88381279,37.94236079)(454.52380493,37.79236623)
\curveto(454.41381326,37.74236099)(454.30381337,37.69736103)(454.19380493,37.65736623)
\curveto(454.08381359,37.6273611)(453.9788137,37.58736114)(453.87880493,37.53736623)
\curveto(453.82881385,37.51736121)(453.78381389,37.49236124)(453.74380493,37.46236623)
\curveto(453.69381398,37.44236129)(453.66881401,37.39236134)(453.66880493,37.31236623)
\curveto(453.68881399,37.29236144)(453.70381397,37.27236146)(453.71380493,37.25236623)
\curveto(453.72381395,37.2323615)(453.73881394,37.21236152)(453.75880493,37.19236623)
\curveto(453.80881387,37.15236158)(453.86381381,37.12236161)(453.92380493,37.10236623)
\curveto(453.9738137,37.08236165)(454.02881365,37.06236167)(454.08880493,37.04236623)
\curveto(454.19881348,36.99236174)(454.30881337,36.95236178)(454.41880493,36.92236623)
\curveto(454.52881315,36.89236184)(454.63881304,36.85236188)(454.74880493,36.80236623)
\curveto(455.13881254,36.6323621)(455.53381214,36.48236225)(455.93380493,36.35236623)
\curveto(456.33381134,36.2323625)(456.72381095,36.09236264)(457.10380493,35.93236623)
\lineto(457.25380493,35.87236623)
\curveto(457.30381037,35.86236287)(457.35381032,35.84736288)(457.40380493,35.82736623)
\lineto(457.64380493,35.73736623)
\curveto(457.72380995,35.70736302)(457.80380987,35.68236305)(457.88380493,35.66236623)
\curveto(457.93380974,35.64236309)(457.98880969,35.6323631)(458.04880493,35.63236623)
\curveto(458.10880957,35.64236309)(458.15880952,35.65736307)(458.19880493,35.67736623)
\curveto(458.2788094,35.727363)(458.32380935,35.8323629)(458.33380493,35.99236623)
\lineto(458.33380493,36.44236623)
\lineto(458.33380493,38.04736623)
\curveto(458.33380934,38.15736057)(458.33880934,38.29236044)(458.34880493,38.45236623)
\curveto(458.34880933,38.61236012)(458.33380934,38.73236)(458.30380493,38.81236623)
}
}
{
\newrgbcolor{curcolor}{0 0 0}
\pscustom[linestyle=none,fillstyle=solid,fillcolor=curcolor]
{
\newpath
\moveto(458.69380493,50.56392873)
\curveto(458.74380893,50.57392038)(458.81380886,50.57892038)(458.90380493,50.57892873)
\curveto(458.98380869,50.57892038)(459.04880863,50.57392038)(459.09880493,50.56392873)
\curveto(459.13880854,50.56392039)(459.1788085,50.5589204)(459.21880493,50.54892873)
\lineto(459.33880493,50.54892873)
\curveto(459.41880826,50.52892043)(459.49880818,50.51892044)(459.57880493,50.51892873)
\curveto(459.65880802,50.51892044)(459.73880794,50.50892045)(459.81880493,50.48892873)
\curveto(459.85880782,50.47892048)(459.89880778,50.47392048)(459.93880493,50.47392873)
\curveto(459.96880771,50.47392048)(460.00380767,50.46892049)(460.04380493,50.45892873)
\curveto(460.15380752,50.42892053)(460.25880742,50.39892056)(460.35880493,50.36892873)
\curveto(460.45880722,50.34892061)(460.55880712,50.31892064)(460.65880493,50.27892873)
\curveto(461.00880667,50.13892082)(461.32380635,49.96892099)(461.60380493,49.76892873)
\curveto(461.88380579,49.56892139)(462.12380555,49.31892164)(462.32380493,49.01892873)
\curveto(462.42380525,48.86892209)(462.50880517,48.72392223)(462.57880493,48.58392873)
\curveto(462.62880505,48.47392248)(462.66880501,48.36392259)(462.69880493,48.25392873)
\curveto(462.72880495,48.1539228)(462.75880492,48.04892291)(462.78880493,47.93892873)
\curveto(462.80880487,47.86892309)(462.81880486,47.80392315)(462.81880493,47.74392873)
\curveto(462.82880485,47.68392327)(462.84380483,47.62392333)(462.86380493,47.56392873)
\lineto(462.86380493,47.41392873)
\curveto(462.88380479,47.36392359)(462.89380478,47.28892367)(462.89380493,47.18892873)
\curveto(462.90380477,47.08892387)(462.89880478,47.00892395)(462.87880493,46.94892873)
\lineto(462.87880493,46.79892873)
\curveto(462.86880481,46.7589242)(462.86380481,46.71392424)(462.86380493,46.66392873)
\curveto(462.86380481,46.62392433)(462.85880482,46.57892438)(462.84880493,46.52892873)
\curveto(462.80880487,46.37892458)(462.7738049,46.22892473)(462.74380493,46.07892873)
\curveto(462.71380496,45.93892502)(462.66880501,45.79892516)(462.60880493,45.65892873)
\curveto(462.52880515,45.4589255)(462.42880525,45.27892568)(462.30880493,45.11892873)
\lineto(462.15880493,44.93892873)
\curveto(462.09880558,44.87892608)(462.05880562,44.80892615)(462.03880493,44.72892873)
\curveto(462.02880565,44.66892629)(462.04380563,44.61892634)(462.08380493,44.57892873)
\curveto(462.11380556,44.54892641)(462.15880552,44.52392643)(462.21880493,44.50392873)
\curveto(462.2788054,44.49392646)(462.34380533,44.48392647)(462.41380493,44.47392873)
\curveto(462.4738052,44.47392648)(462.51880516,44.46392649)(462.54880493,44.44392873)
\curveto(462.59880508,44.40392655)(462.64380503,44.3589266)(462.68380493,44.30892873)
\curveto(462.70380497,44.2589267)(462.71880496,44.18892677)(462.72880493,44.09892873)
\lineto(462.72880493,43.82892873)
\curveto(462.72880495,43.73892722)(462.72380495,43.6539273)(462.71380493,43.57392873)
\curveto(462.69380498,43.49392746)(462.673805,43.43392752)(462.65380493,43.39392873)
\curveto(462.63380504,43.37392758)(462.60880507,43.3539276)(462.57880493,43.33392873)
\lineto(462.48880493,43.27392873)
\curveto(462.40880527,43.24392771)(462.28880539,43.22892773)(462.12880493,43.22892873)
\curveto(461.96880571,43.23892772)(461.83380584,43.24392771)(461.72380493,43.24392873)
\lineto(452.91880493,43.24392873)
\curveto(452.79881488,43.24392771)(452.673815,43.23892772)(452.54380493,43.22892873)
\curveto(452.40381527,43.22892773)(452.29381538,43.2539277)(452.21380493,43.30392873)
\curveto(452.15381552,43.34392761)(452.10381557,43.40892755)(452.06380493,43.49892873)
\curveto(452.06381561,43.51892744)(452.06381561,43.54392741)(452.06380493,43.57392873)
\curveto(452.05381562,43.60392735)(452.04881563,43.62892733)(452.04880493,43.64892873)
\curveto(452.03881564,43.78892717)(452.03881564,43.93392702)(452.04880493,44.08392873)
\curveto(452.04881563,44.24392671)(452.08881559,44.3539266)(452.16880493,44.41392873)
\curveto(452.24881543,44.46392649)(452.36381531,44.48892647)(452.51380493,44.48892873)
\lineto(452.91880493,44.48892873)
\lineto(454.67380493,44.48892873)
\lineto(454.92880493,44.48892873)
\lineto(455.21380493,44.48892873)
\curveto(455.30381237,44.49892646)(455.38881229,44.50892645)(455.46880493,44.51892873)
\curveto(455.53881214,44.53892642)(455.58881209,44.56892639)(455.61880493,44.60892873)
\curveto(455.64881203,44.64892631)(455.65381202,44.69392626)(455.63380493,44.74392873)
\curveto(455.61381206,44.79392616)(455.59381208,44.83392612)(455.57380493,44.86392873)
\curveto(455.53381214,44.91392604)(455.49381218,44.958926)(455.45380493,44.99892873)
\lineto(455.33380493,45.14892873)
\curveto(455.28381239,45.21892574)(455.23881244,45.28892567)(455.19880493,45.35892873)
\lineto(455.07880493,45.59892873)
\curveto(454.98881269,45.77892518)(454.92381275,45.99392496)(454.88380493,46.24392873)
\curveto(454.84381283,46.49392446)(454.82381285,46.74892421)(454.82380493,47.00892873)
\curveto(454.82381285,47.26892369)(454.84881283,47.52392343)(454.89880493,47.77392873)
\curveto(454.93881274,48.02392293)(454.99881268,48.24392271)(455.07880493,48.43392873)
\curveto(455.24881243,48.83392212)(455.48381219,49.17892178)(455.78380493,49.46892873)
\curveto(456.08381159,49.7589212)(456.43381124,49.98892097)(456.83380493,50.15892873)
\curveto(456.94381073,50.20892075)(457.05381062,50.24892071)(457.16380493,50.27892873)
\curveto(457.26381041,50.31892064)(457.36881031,50.3589206)(457.47880493,50.39892873)
\curveto(457.58881009,50.42892053)(457.70380997,50.44892051)(457.82380493,50.45892873)
\lineto(458.15380493,50.51892873)
\curveto(458.18380949,50.52892043)(458.21880946,50.53392042)(458.25880493,50.53392873)
\curveto(458.28880939,50.53392042)(458.31880936,50.53892042)(458.34880493,50.54892873)
\curveto(458.40880927,50.56892039)(458.46880921,50.56892039)(458.52880493,50.54892873)
\curveto(458.5788091,50.53892042)(458.63380904,50.54392041)(458.69380493,50.56392873)
\moveto(459.08380493,49.22892873)
\curveto(459.03380864,49.24892171)(458.9738087,49.2539217)(458.90380493,49.24392873)
\curveto(458.83380884,49.23392172)(458.76880891,49.22892173)(458.70880493,49.22892873)
\curveto(458.53880914,49.22892173)(458.3788093,49.21892174)(458.22880493,49.19892873)
\curveto(458.0788096,49.18892177)(457.94380973,49.1589218)(457.82380493,49.10892873)
\curveto(457.72380995,49.07892188)(457.63381004,49.0539219)(457.55380493,49.03392873)
\curveto(457.4738102,49.01392194)(457.39381028,48.98392197)(457.31380493,48.94392873)
\curveto(457.06381061,48.83392212)(456.83381084,48.68392227)(456.62380493,48.49392873)
\curveto(456.40381127,48.30392265)(456.23881144,48.08392287)(456.12880493,47.83392873)
\curveto(456.09881158,47.7539232)(456.0738116,47.67392328)(456.05380493,47.59392873)
\curveto(456.02381165,47.52392343)(455.99881168,47.44892351)(455.97880493,47.36892873)
\curveto(455.94881173,47.2589237)(455.93381174,47.14892381)(455.93380493,47.03892873)
\curveto(455.92381175,46.92892403)(455.91881176,46.80892415)(455.91880493,46.67892873)
\curveto(455.92881175,46.62892433)(455.93881174,46.58392437)(455.94880493,46.54392873)
\lineto(455.94880493,46.40892873)
\lineto(456.00880493,46.13892873)
\curveto(456.02881165,46.0589249)(456.05881162,45.97892498)(456.09880493,45.89892873)
\curveto(456.23881144,45.5589254)(456.44881123,45.28892567)(456.72880493,45.08892873)
\curveto(456.99881068,44.88892607)(457.31881036,44.72892623)(457.68880493,44.60892873)
\curveto(457.79880988,44.56892639)(457.90880977,44.54392641)(458.01880493,44.53392873)
\curveto(458.12880955,44.52392643)(458.24380943,44.50392645)(458.36380493,44.47392873)
\curveto(458.41380926,44.46392649)(458.45880922,44.46392649)(458.49880493,44.47392873)
\curveto(458.53880914,44.48392647)(458.58380909,44.47892648)(458.63380493,44.45892873)
\curveto(458.68380899,44.44892651)(458.75880892,44.44392651)(458.85880493,44.44392873)
\curveto(458.94880873,44.44392651)(459.01880866,44.44892651)(459.06880493,44.45892873)
\lineto(459.18880493,44.45892873)
\curveto(459.22880845,44.46892649)(459.26880841,44.47392648)(459.30880493,44.47392873)
\curveto(459.34880833,44.47392648)(459.38380829,44.47892648)(459.41380493,44.48892873)
\curveto(459.44380823,44.49892646)(459.4788082,44.50392645)(459.51880493,44.50392873)
\curveto(459.54880813,44.50392645)(459.5788081,44.50892645)(459.60880493,44.51892873)
\curveto(459.68880799,44.53892642)(459.76880791,44.5539264)(459.84880493,44.56392873)
\lineto(460.08880493,44.62392873)
\curveto(460.42880725,44.73392622)(460.71880696,44.88392607)(460.95880493,45.07392873)
\curveto(461.19880648,45.27392568)(461.39880628,45.51892544)(461.55880493,45.80892873)
\curveto(461.60880607,45.89892506)(461.64880603,45.99392496)(461.67880493,46.09392873)
\curveto(461.69880598,46.19392476)(461.72380595,46.29892466)(461.75380493,46.40892873)
\curveto(461.7738059,46.4589245)(461.78380589,46.50392445)(461.78380493,46.54392873)
\curveto(461.7738059,46.59392436)(461.7738059,46.64392431)(461.78380493,46.69392873)
\curveto(461.79380588,46.73392422)(461.79880588,46.77892418)(461.79880493,46.82892873)
\lineto(461.79880493,46.96392873)
\lineto(461.79880493,47.09892873)
\curveto(461.78880589,47.13892382)(461.78380589,47.17392378)(461.78380493,47.20392873)
\curveto(461.78380589,47.23392372)(461.7788059,47.26892369)(461.76880493,47.30892873)
\curveto(461.74880593,47.38892357)(461.73380594,47.46392349)(461.72380493,47.53392873)
\curveto(461.70380597,47.60392335)(461.678806,47.67892328)(461.64880493,47.75892873)
\curveto(461.51880616,48.06892289)(461.34880633,48.31892264)(461.13880493,48.50892873)
\curveto(460.91880676,48.69892226)(460.65380702,48.8589221)(460.34380493,48.98892873)
\curveto(460.20380747,49.03892192)(460.06380761,49.07392188)(459.92380493,49.09392873)
\curveto(459.7738079,49.12392183)(459.62380805,49.1589218)(459.47380493,49.19892873)
\curveto(459.42380825,49.21892174)(459.3788083,49.22392173)(459.33880493,49.21392873)
\curveto(459.28880839,49.21392174)(459.23880844,49.21892174)(459.18880493,49.22892873)
\lineto(459.08380493,49.22892873)
}
}
{
\newrgbcolor{curcolor}{0 0 0}
\pscustom[linestyle=none,fillstyle=solid,fillcolor=curcolor]
{
\newpath
\moveto(454.82380493,55.69017873)
\curveto(454.82381285,55.92017394)(454.88381279,56.05017381)(455.00380493,56.08017873)
\curveto(455.11381256,56.11017375)(455.2788124,56.12517374)(455.49880493,56.12517873)
\lineto(455.78380493,56.12517873)
\curveto(455.8738118,56.12517374)(455.94881173,56.10017376)(456.00880493,56.05017873)
\curveto(456.08881159,55.99017387)(456.13381154,55.90517396)(456.14380493,55.79517873)
\curveto(456.14381153,55.68517418)(456.15881152,55.57517429)(456.18880493,55.46517873)
\curveto(456.21881146,55.32517454)(456.24881143,55.19017467)(456.27880493,55.06017873)
\curveto(456.30881137,54.94017492)(456.34881133,54.82517504)(456.39880493,54.71517873)
\curveto(456.52881115,54.42517544)(456.70881097,54.19017567)(456.93880493,54.01017873)
\curveto(457.15881052,53.83017603)(457.41381026,53.67517619)(457.70380493,53.54517873)
\curveto(457.81380986,53.50517636)(457.92880975,53.47517639)(458.04880493,53.45517873)
\curveto(458.15880952,53.43517643)(458.2738094,53.41017645)(458.39380493,53.38017873)
\curveto(458.44380923,53.37017649)(458.49380918,53.3651765)(458.54380493,53.36517873)
\curveto(458.59380908,53.37517649)(458.64380903,53.37517649)(458.69380493,53.36517873)
\curveto(458.81380886,53.33517653)(458.95380872,53.32017654)(459.11380493,53.32017873)
\curveto(459.26380841,53.33017653)(459.40880827,53.33517653)(459.54880493,53.33517873)
\lineto(461.39380493,53.33517873)
\lineto(461.73880493,53.33517873)
\curveto(461.85880582,53.33517653)(461.9738057,53.33017653)(462.08380493,53.32017873)
\curveto(462.19380548,53.31017655)(462.28880539,53.30517656)(462.36880493,53.30517873)
\curveto(462.44880523,53.31517655)(462.51880516,53.29517657)(462.57880493,53.24517873)
\curveto(462.64880503,53.19517667)(462.68880499,53.11517675)(462.69880493,53.00517873)
\curveto(462.70880497,52.90517696)(462.71380496,52.79517707)(462.71380493,52.67517873)
\lineto(462.71380493,52.40517873)
\curveto(462.69380498,52.35517751)(462.678805,52.30517756)(462.66880493,52.25517873)
\curveto(462.64880503,52.21517765)(462.62380505,52.18517768)(462.59380493,52.16517873)
\curveto(462.52380515,52.11517775)(462.43880524,52.08517778)(462.33880493,52.07517873)
\lineto(462.00880493,52.07517873)
\lineto(460.85380493,52.07517873)
\lineto(456.69880493,52.07517873)
\lineto(455.66380493,52.07517873)
\lineto(455.36380493,52.07517873)
\curveto(455.26381241,52.08517778)(455.1788125,52.11517775)(455.10880493,52.16517873)
\curveto(455.06881261,52.19517767)(455.03881264,52.24517762)(455.01880493,52.31517873)
\curveto(454.99881268,52.39517747)(454.98881269,52.48017738)(454.98880493,52.57017873)
\curveto(454.9788127,52.6601772)(454.9788127,52.75017711)(454.98880493,52.84017873)
\curveto(454.99881268,52.93017693)(455.01381266,53.00017686)(455.03380493,53.05017873)
\curveto(455.06381261,53.13017673)(455.12381255,53.18017668)(455.21380493,53.20017873)
\curveto(455.29381238,53.23017663)(455.38381229,53.24517662)(455.48380493,53.24517873)
\lineto(455.78380493,53.24517873)
\curveto(455.88381179,53.24517662)(455.9738117,53.2651766)(456.05380493,53.30517873)
\curveto(456.0738116,53.31517655)(456.08881159,53.32517654)(456.09880493,53.33517873)
\lineto(456.14380493,53.38017873)
\curveto(456.14381153,53.49017637)(456.09881158,53.58017628)(456.00880493,53.65017873)
\curveto(455.90881177,53.72017614)(455.82881185,53.78017608)(455.76880493,53.83017873)
\lineto(455.67880493,53.92017873)
\curveto(455.56881211,54.01017585)(455.45381222,54.13517573)(455.33380493,54.29517873)
\curveto(455.21381246,54.45517541)(455.12381255,54.60517526)(455.06380493,54.74517873)
\curveto(455.01381266,54.83517503)(454.9788127,54.93017493)(454.95880493,55.03017873)
\curveto(454.92881275,55.13017473)(454.89881278,55.23517463)(454.86880493,55.34517873)
\curveto(454.85881282,55.40517446)(454.85381282,55.4651744)(454.85380493,55.52517873)
\curveto(454.84381283,55.58517428)(454.83381284,55.64017422)(454.82380493,55.69017873)
}
}
{
\newrgbcolor{curcolor}{0 0 0}
\pscustom[linestyle=none,fillstyle=solid,fillcolor=curcolor]
{
}
}
{
\newrgbcolor{curcolor}{0 0 0}
\pscustom[linestyle=none,fillstyle=solid,fillcolor=curcolor]
{
\newpath
\moveto(452.12380493,64.24510061)
\curveto(452.11381556,64.93509597)(452.23381544,65.53509537)(452.48380493,66.04510061)
\curveto(452.73381494,66.56509434)(453.06881461,66.96009395)(453.48880493,67.23010061)
\curveto(453.56881411,67.28009363)(453.65881402,67.32509358)(453.75880493,67.36510061)
\curveto(453.84881383,67.4050935)(453.94381373,67.45009346)(454.04380493,67.50010061)
\curveto(454.14381353,67.54009337)(454.24381343,67.57009334)(454.34380493,67.59010061)
\curveto(454.44381323,67.6100933)(454.54881313,67.63009328)(454.65880493,67.65010061)
\curveto(454.70881297,67.67009324)(454.75381292,67.67509323)(454.79380493,67.66510061)
\curveto(454.83381284,67.65509325)(454.8788128,67.66009325)(454.92880493,67.68010061)
\curveto(454.9788127,67.69009322)(455.06381261,67.69509321)(455.18380493,67.69510061)
\curveto(455.29381238,67.69509321)(455.3788123,67.69009322)(455.43880493,67.68010061)
\curveto(455.49881218,67.66009325)(455.55881212,67.65009326)(455.61880493,67.65010061)
\curveto(455.678812,67.66009325)(455.73881194,67.65509325)(455.79880493,67.63510061)
\curveto(455.93881174,67.59509331)(456.0738116,67.56009335)(456.20380493,67.53010061)
\curveto(456.33381134,67.50009341)(456.45881122,67.46009345)(456.57880493,67.41010061)
\curveto(456.71881096,67.35009356)(456.84381083,67.28009363)(456.95380493,67.20010061)
\curveto(457.06381061,67.13009378)(457.1738105,67.05509385)(457.28380493,66.97510061)
\lineto(457.34380493,66.91510061)
\curveto(457.36381031,66.905094)(457.38381029,66.89009402)(457.40380493,66.87010061)
\curveto(457.56381011,66.75009416)(457.70880997,66.61509429)(457.83880493,66.46510061)
\curveto(457.96880971,66.31509459)(458.09380958,66.15509475)(458.21380493,65.98510061)
\curveto(458.43380924,65.67509523)(458.63880904,65.38009553)(458.82880493,65.10010061)
\curveto(458.96880871,64.87009604)(459.10380857,64.64009627)(459.23380493,64.41010061)
\curveto(459.36380831,64.19009672)(459.49880818,63.97009694)(459.63880493,63.75010061)
\curveto(459.80880787,63.50009741)(459.98880769,63.26009765)(460.17880493,63.03010061)
\curveto(460.36880731,62.8100981)(460.59380708,62.62009829)(460.85380493,62.46010061)
\curveto(460.91380676,62.42009849)(460.9738067,62.38509852)(461.03380493,62.35510061)
\curveto(461.08380659,62.32509858)(461.14880653,62.29509861)(461.22880493,62.26510061)
\curveto(461.29880638,62.24509866)(461.35880632,62.24009867)(461.40880493,62.25010061)
\curveto(461.4788062,62.27009864)(461.53380614,62.3050986)(461.57380493,62.35510061)
\curveto(461.60380607,62.4050985)(461.62380605,62.46509844)(461.63380493,62.53510061)
\lineto(461.63380493,62.77510061)
\lineto(461.63380493,63.52510061)
\lineto(461.63380493,66.33010061)
\lineto(461.63380493,66.99010061)
\curveto(461.63380604,67.08009383)(461.63880604,67.16509374)(461.64880493,67.24510061)
\curveto(461.64880603,67.32509358)(461.66880601,67.39009352)(461.70880493,67.44010061)
\curveto(461.74880593,67.49009342)(461.82380585,67.53009338)(461.93380493,67.56010061)
\curveto(462.03380564,67.60009331)(462.13380554,67.6100933)(462.23380493,67.59010061)
\lineto(462.36880493,67.59010061)
\curveto(462.43880524,67.57009334)(462.49880518,67.55009336)(462.54880493,67.53010061)
\curveto(462.59880508,67.5100934)(462.63880504,67.47509343)(462.66880493,67.42510061)
\curveto(462.70880497,67.37509353)(462.72880495,67.3050936)(462.72880493,67.21510061)
\lineto(462.72880493,66.94510061)
\lineto(462.72880493,66.04510061)
\lineto(462.72880493,62.53510061)
\lineto(462.72880493,61.47010061)
\curveto(462.72880495,61.39009952)(462.73380494,61.30009961)(462.74380493,61.20010061)
\curveto(462.74380493,61.10009981)(462.73380494,61.01509989)(462.71380493,60.94510061)
\curveto(462.64380503,60.73510017)(462.46380521,60.67010024)(462.17380493,60.75010061)
\curveto(462.13380554,60.76010015)(462.09880558,60.76010015)(462.06880493,60.75010061)
\curveto(462.02880565,60.75010016)(461.98380569,60.76010015)(461.93380493,60.78010061)
\curveto(461.85380582,60.80010011)(461.76880591,60.82010009)(461.67880493,60.84010061)
\curveto(461.58880609,60.86010005)(461.50380617,60.88510002)(461.42380493,60.91510061)
\curveto(460.93380674,61.07509983)(460.51880716,61.27509963)(460.17880493,61.51510061)
\curveto(459.92880775,61.69509921)(459.70380797,61.90009901)(459.50380493,62.13010061)
\curveto(459.29380838,62.36009855)(459.09880858,62.60009831)(458.91880493,62.85010061)
\curveto(458.73880894,63.1100978)(458.56880911,63.37509753)(458.40880493,63.64510061)
\curveto(458.23880944,63.92509698)(458.06380961,64.19509671)(457.88380493,64.45510061)
\curveto(457.80380987,64.56509634)(457.72880995,64.67009624)(457.65880493,64.77010061)
\curveto(457.58881009,64.88009603)(457.51381016,64.99009592)(457.43380493,65.10010061)
\curveto(457.40381027,65.14009577)(457.3738103,65.17509573)(457.34380493,65.20510061)
\curveto(457.30381037,65.24509566)(457.2738104,65.28509562)(457.25380493,65.32510061)
\curveto(457.14381053,65.46509544)(457.01881066,65.59009532)(456.87880493,65.70010061)
\curveto(456.84881083,65.72009519)(456.82381085,65.74509516)(456.80380493,65.77510061)
\curveto(456.7738109,65.8050951)(456.74381093,65.83009508)(456.71380493,65.85010061)
\curveto(456.61381106,65.93009498)(456.51381116,65.99509491)(456.41380493,66.04510061)
\curveto(456.31381136,66.1050948)(456.20381147,66.16009475)(456.08380493,66.21010061)
\curveto(456.01381166,66.24009467)(455.93881174,66.26009465)(455.85880493,66.27010061)
\lineto(455.61880493,66.33010061)
\lineto(455.52880493,66.33010061)
\curveto(455.49881218,66.34009457)(455.46881221,66.34509456)(455.43880493,66.34510061)
\curveto(455.36881231,66.36509454)(455.2738124,66.37009454)(455.15380493,66.36010061)
\curveto(455.02381265,66.36009455)(454.92381275,66.35009456)(454.85380493,66.33010061)
\curveto(454.7738129,66.3100946)(454.69881298,66.29009462)(454.62880493,66.27010061)
\curveto(454.54881313,66.26009465)(454.46881321,66.24009467)(454.38880493,66.21010061)
\curveto(454.14881353,66.10009481)(453.94881373,65.95009496)(453.78880493,65.76010061)
\curveto(453.61881406,65.58009533)(453.4788142,65.36009555)(453.36880493,65.10010061)
\curveto(453.34881433,65.03009588)(453.33381434,64.96009595)(453.32380493,64.89010061)
\curveto(453.30381437,64.82009609)(453.28381439,64.74509616)(453.26380493,64.66510061)
\curveto(453.24381443,64.58509632)(453.23381444,64.47509643)(453.23380493,64.33510061)
\curveto(453.23381444,64.2050967)(453.24381443,64.10009681)(453.26380493,64.02010061)
\curveto(453.2738144,63.96009695)(453.2788144,63.905097)(453.27880493,63.85510061)
\curveto(453.2788144,63.8050971)(453.28881439,63.75509715)(453.30880493,63.70510061)
\curveto(453.34881433,63.6050973)(453.38881429,63.5100974)(453.42880493,63.42010061)
\curveto(453.46881421,63.34009757)(453.51381416,63.26009765)(453.56380493,63.18010061)
\curveto(453.58381409,63.15009776)(453.60881407,63.12009779)(453.63880493,63.09010061)
\curveto(453.66881401,63.07009784)(453.69381398,63.04509786)(453.71380493,63.01510061)
\lineto(453.78880493,62.94010061)
\curveto(453.80881387,62.910098)(453.82881385,62.88509802)(453.84880493,62.86510061)
\lineto(454.05880493,62.71510061)
\curveto(454.11881356,62.67509823)(454.18381349,62.63009828)(454.25380493,62.58010061)
\curveto(454.34381333,62.52009839)(454.44881323,62.47009844)(454.56880493,62.43010061)
\curveto(454.678813,62.40009851)(454.78881289,62.36509854)(454.89880493,62.32510061)
\curveto(455.00881267,62.28509862)(455.15381252,62.26009865)(455.33380493,62.25010061)
\curveto(455.50381217,62.24009867)(455.62881205,62.2100987)(455.70880493,62.16010061)
\curveto(455.78881189,62.1100988)(455.83381184,62.03509887)(455.84380493,61.93510061)
\curveto(455.85381182,61.83509907)(455.85881182,61.72509918)(455.85880493,61.60510061)
\curveto(455.85881182,61.56509934)(455.86381181,61.52509938)(455.87380493,61.48510061)
\curveto(455.8738118,61.44509946)(455.86881181,61.4100995)(455.85880493,61.38010061)
\curveto(455.83881184,61.33009958)(455.82881185,61.28009963)(455.82880493,61.23010061)
\curveto(455.82881185,61.19009972)(455.81881186,61.15009976)(455.79880493,61.11010061)
\curveto(455.73881194,61.02009989)(455.60381207,60.97509993)(455.39380493,60.97510061)
\lineto(455.27380493,60.97510061)
\curveto(455.21381246,60.98509992)(455.15381252,60.99009992)(455.09380493,60.99010061)
\curveto(455.02381265,61.00009991)(454.95881272,61.0100999)(454.89880493,61.02010061)
\curveto(454.78881289,61.04009987)(454.68881299,61.06009985)(454.59880493,61.08010061)
\curveto(454.49881318,61.10009981)(454.40381327,61.13009978)(454.31380493,61.17010061)
\curveto(454.24381343,61.19009972)(454.18381349,61.2100997)(454.13380493,61.23010061)
\lineto(453.95380493,61.29010061)
\curveto(453.69381398,61.4100995)(453.44881423,61.56509934)(453.21880493,61.75510061)
\curveto(452.98881469,61.95509895)(452.80381487,62.17009874)(452.66380493,62.40010061)
\curveto(452.58381509,62.5100984)(452.51881516,62.62509828)(452.46880493,62.74510061)
\lineto(452.31880493,63.13510061)
\curveto(452.26881541,63.24509766)(452.23881544,63.36009755)(452.22880493,63.48010061)
\curveto(452.20881547,63.60009731)(452.18381549,63.72509718)(452.15380493,63.85510061)
\curveto(452.15381552,63.92509698)(452.15381552,63.99009692)(452.15380493,64.05010061)
\curveto(452.14381553,64.1100968)(452.13381554,64.17509673)(452.12380493,64.24510061)
}
}
{
\newrgbcolor{curcolor}{0 0 0}
\pscustom[linestyle=none,fillstyle=solid,fillcolor=curcolor]
{
\newpath
\moveto(452.31880493,69.80470998)
\lineto(452.31880493,74.60470998)
\lineto(452.31880493,75.60970998)
\curveto(452.31881536,75.74970288)(452.32881535,75.86970276)(452.34880493,75.96970998)
\curveto(452.35881532,76.07970255)(452.40381527,76.15970247)(452.48380493,76.20970998)
\curveto(452.52381515,76.2297024)(452.5738151,76.23970239)(452.63380493,76.23970998)
\curveto(452.69381498,76.24970238)(452.75881492,76.25470238)(452.82880493,76.25470998)
\lineto(453.09880493,76.25470998)
\curveto(453.18881449,76.25470238)(453.26881441,76.24470239)(453.33880493,76.22470998)
\curveto(453.41881426,76.18470245)(453.48881419,76.13970249)(453.54880493,76.08970998)
\lineto(453.72880493,75.93970998)
\curveto(453.7788139,75.90970272)(453.81881386,75.87470276)(453.84880493,75.83470998)
\curveto(453.8788138,75.79470284)(453.91881376,75.75470288)(453.96880493,75.71470998)
\curveto(454.0788136,75.634703)(454.18881349,75.54970308)(454.29880493,75.45970998)
\curveto(454.39881328,75.36970326)(454.50381317,75.28470335)(454.61380493,75.20470998)
\curveto(454.81381286,75.06470357)(455.02381265,74.92470371)(455.24380493,74.78470998)
\curveto(455.45381222,74.64470399)(455.66881201,74.50470413)(455.88880493,74.36470998)
\curveto(455.9788117,74.31470432)(456.0738116,74.26470437)(456.17380493,74.21470998)
\curveto(456.2738114,74.16470447)(456.36881131,74.10970452)(456.45880493,74.04970998)
\curveto(456.4788112,74.0297046)(456.50381117,74.01970461)(456.53380493,74.01970998)
\curveto(456.56381111,74.01970461)(456.58881109,74.00970462)(456.60880493,73.98970998)
\curveto(456.70881097,73.91970471)(456.82381085,73.85470478)(456.95380493,73.79470998)
\curveto(457.0738106,73.7347049)(457.18881049,73.67970495)(457.29880493,73.62970998)
\curveto(457.52881015,73.5297051)(457.76380991,73.4347052)(458.00380493,73.34470998)
\curveto(458.24380943,73.25470538)(458.48380919,73.15470548)(458.72380493,73.04470998)
\curveto(458.7738089,73.02470561)(458.81880886,73.00970562)(458.85880493,72.99970998)
\curveto(458.89880878,72.99970563)(458.94380873,72.98970564)(458.99380493,72.96970998)
\curveto(459.11380856,72.91970571)(459.23880844,72.87470576)(459.36880493,72.83470998)
\curveto(459.48880819,72.80470583)(459.60880807,72.76970586)(459.72880493,72.72970998)
\curveto(459.95880772,72.64970598)(460.19880748,72.58470605)(460.44880493,72.53470998)
\curveto(460.68880699,72.49470614)(460.92880675,72.44470619)(461.16880493,72.38470998)
\curveto(461.31880636,72.34470629)(461.46880621,72.31970631)(461.61880493,72.30970998)
\curveto(461.76880591,72.29970633)(461.91880576,72.27970635)(462.06880493,72.24970998)
\curveto(462.10880557,72.23970639)(462.16880551,72.2347064)(462.24880493,72.23470998)
\curveto(462.36880531,72.20470643)(462.46880521,72.17470646)(462.54880493,72.14470998)
\curveto(462.62880505,72.11470652)(462.68380499,72.04470659)(462.71380493,71.93470998)
\curveto(462.73380494,71.88470675)(462.74380493,71.8297068)(462.74380493,71.76970998)
\lineto(462.74380493,71.57470998)
\curveto(462.74380493,71.4347072)(462.73880494,71.29470734)(462.72880493,71.15470998)
\curveto(462.71880496,71.02470761)(462.673805,70.9297077)(462.59380493,70.86970998)
\curveto(462.53380514,70.8297078)(462.44880523,70.80970782)(462.33880493,70.80970998)
\curveto(462.22880545,70.81970781)(462.13380554,70.8347078)(462.05380493,70.85470998)
\lineto(461.97880493,70.85470998)
\curveto(461.94880573,70.86470777)(461.91880576,70.86970776)(461.88880493,70.86970998)
\curveto(461.80880587,70.88970774)(461.73380594,70.89970773)(461.66380493,70.89970998)
\curveto(461.59380608,70.89970773)(461.52380615,70.90970772)(461.45380493,70.92970998)
\curveto(461.26380641,70.97970765)(461.0788066,71.01970761)(460.89880493,71.04970998)
\curveto(460.70880697,71.07970755)(460.52880715,71.11970751)(460.35880493,71.16970998)
\curveto(460.30880737,71.18970744)(460.26880741,71.19970743)(460.23880493,71.19970998)
\curveto(460.20880747,71.19970743)(460.1738075,71.20470743)(460.13380493,71.21470998)
\curveto(459.83380784,71.31470732)(459.53880814,71.40470723)(459.24880493,71.48470998)
\curveto(458.95880872,71.57470706)(458.678809,71.67970695)(458.40880493,71.79970998)
\curveto(457.82880985,72.05970657)(457.2788104,72.3297063)(456.75880493,72.60970998)
\curveto(456.22881145,72.88970574)(455.72381195,73.19970543)(455.24380493,73.53970998)
\curveto(455.04381263,73.67970495)(454.85381282,73.8297048)(454.67380493,73.98970998)
\curveto(454.48381319,74.14970448)(454.29381338,74.29970433)(454.10380493,74.43970998)
\curveto(454.05381362,74.47970415)(454.00881367,74.51470412)(453.96880493,74.54470998)
\curveto(453.91881376,74.58470405)(453.86881381,74.61970401)(453.81880493,74.64970998)
\curveto(453.79881388,74.65970397)(453.7738139,74.66970396)(453.74380493,74.67970998)
\curveto(453.71381396,74.69970393)(453.68381399,74.69970393)(453.65380493,74.67970998)
\curveto(453.59381408,74.65970397)(453.55881412,74.62470401)(453.54880493,74.57470998)
\curveto(453.52881415,74.52470411)(453.50881417,74.47470416)(453.48880493,74.42470998)
\lineto(453.48880493,74.31970998)
\curveto(453.4788142,74.27970435)(453.4788142,74.2297044)(453.48880493,74.16970998)
\lineto(453.48880493,74.01970998)
\lineto(453.48880493,73.41970998)
\lineto(453.48880493,70.77970998)
\lineto(453.48880493,70.04470998)
\lineto(453.48880493,69.80470998)
\curveto(453.4788142,69.7347089)(453.46381421,69.67470896)(453.44380493,69.62470998)
\curveto(453.40381427,69.5347091)(453.34381433,69.47470916)(453.26380493,69.44470998)
\curveto(453.16381451,69.39470924)(453.01881466,69.37970925)(452.82880493,69.39970998)
\curveto(452.62881505,69.41970921)(452.49381518,69.45470918)(452.42380493,69.50470998)
\curveto(452.40381527,69.52470911)(452.38881529,69.54970908)(452.37880493,69.57970998)
\lineto(452.31880493,69.69970998)
\curveto(452.31881536,69.71970891)(452.32381535,69.7347089)(452.33380493,69.74470998)
\curveto(452.33381534,69.76470887)(452.32881535,69.78470885)(452.31880493,69.80470998)
}
}
{
\newrgbcolor{curcolor}{0 0 0}
\pscustom[linestyle=none,fillstyle=solid,fillcolor=curcolor]
{
\newpath
\moveto(461.09380493,78.63431936)
\lineto(461.09380493,79.26431936)
\lineto(461.09380493,79.45931936)
\curveto(461.09380658,79.52931683)(461.10380657,79.58931677)(461.12380493,79.63931936)
\curveto(461.16380651,79.70931665)(461.20380647,79.7593166)(461.24380493,79.78931936)
\curveto(461.29380638,79.82931653)(461.35880632,79.84931651)(461.43880493,79.84931936)
\curveto(461.51880616,79.8593165)(461.60380607,79.86431649)(461.69380493,79.86431936)
\lineto(462.41380493,79.86431936)
\curveto(462.89380478,79.86431649)(463.30380437,79.80431655)(463.64380493,79.68431936)
\curveto(463.98380369,79.56431679)(464.25880342,79.36931699)(464.46880493,79.09931936)
\curveto(464.51880316,79.02931733)(464.56380311,78.9593174)(464.60380493,78.88931936)
\curveto(464.65380302,78.82931753)(464.69880298,78.7543176)(464.73880493,78.66431936)
\curveto(464.74880293,78.64431771)(464.75880292,78.61431774)(464.76880493,78.57431936)
\curveto(464.78880289,78.53431782)(464.79380288,78.48931787)(464.78380493,78.43931936)
\curveto(464.75380292,78.34931801)(464.678803,78.29431806)(464.55880493,78.27431936)
\curveto(464.44880323,78.2543181)(464.35380332,78.26931809)(464.27380493,78.31931936)
\curveto(464.20380347,78.34931801)(464.13880354,78.39431796)(464.07880493,78.45431936)
\curveto(464.02880365,78.52431783)(463.9788037,78.58931777)(463.92880493,78.64931936)
\curveto(463.8788038,78.71931764)(463.80380387,78.77931758)(463.70380493,78.82931936)
\curveto(463.61380406,78.88931747)(463.52380415,78.93931742)(463.43380493,78.97931936)
\curveto(463.40380427,78.99931736)(463.34380433,79.02431733)(463.25380493,79.05431936)
\curveto(463.1738045,79.08431727)(463.10380457,79.08931727)(463.04380493,79.06931936)
\curveto(462.90380477,79.03931732)(462.81380486,78.97931738)(462.77380493,78.88931936)
\curveto(462.74380493,78.80931755)(462.72880495,78.71931764)(462.72880493,78.61931936)
\curveto(462.72880495,78.51931784)(462.70380497,78.43431792)(462.65380493,78.36431936)
\curveto(462.58380509,78.27431808)(462.44380523,78.22931813)(462.23380493,78.22931936)
\lineto(461.67880493,78.22931936)
\lineto(461.45380493,78.22931936)
\curveto(461.3738063,78.23931812)(461.30880637,78.2593181)(461.25880493,78.28931936)
\curveto(461.1788065,78.34931801)(461.13380654,78.41931794)(461.12380493,78.49931936)
\curveto(461.11380656,78.51931784)(461.10880657,78.53931782)(461.10880493,78.55931936)
\curveto(461.10880657,78.58931777)(461.10380657,78.61431774)(461.09380493,78.63431936)
}
}
{
\newrgbcolor{curcolor}{0 0 0}
\pscustom[linestyle=none,fillstyle=solid,fillcolor=curcolor]
{
}
}
{
\newrgbcolor{curcolor}{0 0 0}
\pscustom[linestyle=none,fillstyle=solid,fillcolor=curcolor]
{
\newpath
\moveto(452.12380493,89.26463186)
\curveto(452.11381556,89.95462722)(452.23381544,90.55462662)(452.48380493,91.06463186)
\curveto(452.73381494,91.58462559)(453.06881461,91.9796252)(453.48880493,92.24963186)
\curveto(453.56881411,92.29962488)(453.65881402,92.34462483)(453.75880493,92.38463186)
\curveto(453.84881383,92.42462475)(453.94381373,92.46962471)(454.04380493,92.51963186)
\curveto(454.14381353,92.55962462)(454.24381343,92.58962459)(454.34380493,92.60963186)
\curveto(454.44381323,92.62962455)(454.54881313,92.64962453)(454.65880493,92.66963186)
\curveto(454.70881297,92.68962449)(454.75381292,92.69462448)(454.79380493,92.68463186)
\curveto(454.83381284,92.6746245)(454.8788128,92.6796245)(454.92880493,92.69963186)
\curveto(454.9788127,92.70962447)(455.06381261,92.71462446)(455.18380493,92.71463186)
\curveto(455.29381238,92.71462446)(455.3788123,92.70962447)(455.43880493,92.69963186)
\curveto(455.49881218,92.6796245)(455.55881212,92.66962451)(455.61880493,92.66963186)
\curveto(455.678812,92.6796245)(455.73881194,92.6746245)(455.79880493,92.65463186)
\curveto(455.93881174,92.61462456)(456.0738116,92.5796246)(456.20380493,92.54963186)
\curveto(456.33381134,92.51962466)(456.45881122,92.4796247)(456.57880493,92.42963186)
\curveto(456.71881096,92.36962481)(456.84381083,92.29962488)(456.95380493,92.21963186)
\curveto(457.06381061,92.14962503)(457.1738105,92.0746251)(457.28380493,91.99463186)
\lineto(457.34380493,91.93463186)
\curveto(457.36381031,91.92462525)(457.38381029,91.90962527)(457.40380493,91.88963186)
\curveto(457.56381011,91.76962541)(457.70880997,91.63462554)(457.83880493,91.48463186)
\curveto(457.96880971,91.33462584)(458.09380958,91.174626)(458.21380493,91.00463186)
\curveto(458.43380924,90.69462648)(458.63880904,90.39962678)(458.82880493,90.11963186)
\curveto(458.96880871,89.88962729)(459.10380857,89.65962752)(459.23380493,89.42963186)
\curveto(459.36380831,89.20962797)(459.49880818,88.98962819)(459.63880493,88.76963186)
\curveto(459.80880787,88.51962866)(459.98880769,88.2796289)(460.17880493,88.04963186)
\curveto(460.36880731,87.82962935)(460.59380708,87.63962954)(460.85380493,87.47963186)
\curveto(460.91380676,87.43962974)(460.9738067,87.40462977)(461.03380493,87.37463186)
\curveto(461.08380659,87.34462983)(461.14880653,87.31462986)(461.22880493,87.28463186)
\curveto(461.29880638,87.26462991)(461.35880632,87.25962992)(461.40880493,87.26963186)
\curveto(461.4788062,87.28962989)(461.53380614,87.32462985)(461.57380493,87.37463186)
\curveto(461.60380607,87.42462975)(461.62380605,87.48462969)(461.63380493,87.55463186)
\lineto(461.63380493,87.79463186)
\lineto(461.63380493,88.54463186)
\lineto(461.63380493,91.34963186)
\lineto(461.63380493,92.00963186)
\curveto(461.63380604,92.09962508)(461.63880604,92.18462499)(461.64880493,92.26463186)
\curveto(461.64880603,92.34462483)(461.66880601,92.40962477)(461.70880493,92.45963186)
\curveto(461.74880593,92.50962467)(461.82380585,92.54962463)(461.93380493,92.57963186)
\curveto(462.03380564,92.61962456)(462.13380554,92.62962455)(462.23380493,92.60963186)
\lineto(462.36880493,92.60963186)
\curveto(462.43880524,92.58962459)(462.49880518,92.56962461)(462.54880493,92.54963186)
\curveto(462.59880508,92.52962465)(462.63880504,92.49462468)(462.66880493,92.44463186)
\curveto(462.70880497,92.39462478)(462.72880495,92.32462485)(462.72880493,92.23463186)
\lineto(462.72880493,91.96463186)
\lineto(462.72880493,91.06463186)
\lineto(462.72880493,87.55463186)
\lineto(462.72880493,86.48963186)
\curveto(462.72880495,86.40963077)(462.73380494,86.31963086)(462.74380493,86.21963186)
\curveto(462.74380493,86.11963106)(462.73380494,86.03463114)(462.71380493,85.96463186)
\curveto(462.64380503,85.75463142)(462.46380521,85.68963149)(462.17380493,85.76963186)
\curveto(462.13380554,85.7796314)(462.09880558,85.7796314)(462.06880493,85.76963186)
\curveto(462.02880565,85.76963141)(461.98380569,85.7796314)(461.93380493,85.79963186)
\curveto(461.85380582,85.81963136)(461.76880591,85.83963134)(461.67880493,85.85963186)
\curveto(461.58880609,85.8796313)(461.50380617,85.90463127)(461.42380493,85.93463186)
\curveto(460.93380674,86.09463108)(460.51880716,86.29463088)(460.17880493,86.53463186)
\curveto(459.92880775,86.71463046)(459.70380797,86.91963026)(459.50380493,87.14963186)
\curveto(459.29380838,87.3796298)(459.09880858,87.61962956)(458.91880493,87.86963186)
\curveto(458.73880894,88.12962905)(458.56880911,88.39462878)(458.40880493,88.66463186)
\curveto(458.23880944,88.94462823)(458.06380961,89.21462796)(457.88380493,89.47463186)
\curveto(457.80380987,89.58462759)(457.72880995,89.68962749)(457.65880493,89.78963186)
\curveto(457.58881009,89.89962728)(457.51381016,90.00962717)(457.43380493,90.11963186)
\curveto(457.40381027,90.15962702)(457.3738103,90.19462698)(457.34380493,90.22463186)
\curveto(457.30381037,90.26462691)(457.2738104,90.30462687)(457.25380493,90.34463186)
\curveto(457.14381053,90.48462669)(457.01881066,90.60962657)(456.87880493,90.71963186)
\curveto(456.84881083,90.73962644)(456.82381085,90.76462641)(456.80380493,90.79463186)
\curveto(456.7738109,90.82462635)(456.74381093,90.84962633)(456.71380493,90.86963186)
\curveto(456.61381106,90.94962623)(456.51381116,91.01462616)(456.41380493,91.06463186)
\curveto(456.31381136,91.12462605)(456.20381147,91.179626)(456.08380493,91.22963186)
\curveto(456.01381166,91.25962592)(455.93881174,91.2796259)(455.85880493,91.28963186)
\lineto(455.61880493,91.34963186)
\lineto(455.52880493,91.34963186)
\curveto(455.49881218,91.35962582)(455.46881221,91.36462581)(455.43880493,91.36463186)
\curveto(455.36881231,91.38462579)(455.2738124,91.38962579)(455.15380493,91.37963186)
\curveto(455.02381265,91.3796258)(454.92381275,91.36962581)(454.85380493,91.34963186)
\curveto(454.7738129,91.32962585)(454.69881298,91.30962587)(454.62880493,91.28963186)
\curveto(454.54881313,91.2796259)(454.46881321,91.25962592)(454.38880493,91.22963186)
\curveto(454.14881353,91.11962606)(453.94881373,90.96962621)(453.78880493,90.77963186)
\curveto(453.61881406,90.59962658)(453.4788142,90.3796268)(453.36880493,90.11963186)
\curveto(453.34881433,90.04962713)(453.33381434,89.9796272)(453.32380493,89.90963186)
\curveto(453.30381437,89.83962734)(453.28381439,89.76462741)(453.26380493,89.68463186)
\curveto(453.24381443,89.60462757)(453.23381444,89.49462768)(453.23380493,89.35463186)
\curveto(453.23381444,89.22462795)(453.24381443,89.11962806)(453.26380493,89.03963186)
\curveto(453.2738144,88.9796282)(453.2788144,88.92462825)(453.27880493,88.87463186)
\curveto(453.2788144,88.82462835)(453.28881439,88.7746284)(453.30880493,88.72463186)
\curveto(453.34881433,88.62462855)(453.38881429,88.52962865)(453.42880493,88.43963186)
\curveto(453.46881421,88.35962882)(453.51381416,88.2796289)(453.56380493,88.19963186)
\curveto(453.58381409,88.16962901)(453.60881407,88.13962904)(453.63880493,88.10963186)
\curveto(453.66881401,88.08962909)(453.69381398,88.06462911)(453.71380493,88.03463186)
\lineto(453.78880493,87.95963186)
\curveto(453.80881387,87.92962925)(453.82881385,87.90462927)(453.84880493,87.88463186)
\lineto(454.05880493,87.73463186)
\curveto(454.11881356,87.69462948)(454.18381349,87.64962953)(454.25380493,87.59963186)
\curveto(454.34381333,87.53962964)(454.44881323,87.48962969)(454.56880493,87.44963186)
\curveto(454.678813,87.41962976)(454.78881289,87.38462979)(454.89880493,87.34463186)
\curveto(455.00881267,87.30462987)(455.15381252,87.2796299)(455.33380493,87.26963186)
\curveto(455.50381217,87.25962992)(455.62881205,87.22962995)(455.70880493,87.17963186)
\curveto(455.78881189,87.12963005)(455.83381184,87.05463012)(455.84380493,86.95463186)
\curveto(455.85381182,86.85463032)(455.85881182,86.74463043)(455.85880493,86.62463186)
\curveto(455.85881182,86.58463059)(455.86381181,86.54463063)(455.87380493,86.50463186)
\curveto(455.8738118,86.46463071)(455.86881181,86.42963075)(455.85880493,86.39963186)
\curveto(455.83881184,86.34963083)(455.82881185,86.29963088)(455.82880493,86.24963186)
\curveto(455.82881185,86.20963097)(455.81881186,86.16963101)(455.79880493,86.12963186)
\curveto(455.73881194,86.03963114)(455.60381207,85.99463118)(455.39380493,85.99463186)
\lineto(455.27380493,85.99463186)
\curveto(455.21381246,86.00463117)(455.15381252,86.00963117)(455.09380493,86.00963186)
\curveto(455.02381265,86.01963116)(454.95881272,86.02963115)(454.89880493,86.03963186)
\curveto(454.78881289,86.05963112)(454.68881299,86.0796311)(454.59880493,86.09963186)
\curveto(454.49881318,86.11963106)(454.40381327,86.14963103)(454.31380493,86.18963186)
\curveto(454.24381343,86.20963097)(454.18381349,86.22963095)(454.13380493,86.24963186)
\lineto(453.95380493,86.30963186)
\curveto(453.69381398,86.42963075)(453.44881423,86.58463059)(453.21880493,86.77463186)
\curveto(452.98881469,86.9746302)(452.80381487,87.18962999)(452.66380493,87.41963186)
\curveto(452.58381509,87.52962965)(452.51881516,87.64462953)(452.46880493,87.76463186)
\lineto(452.31880493,88.15463186)
\curveto(452.26881541,88.26462891)(452.23881544,88.3796288)(452.22880493,88.49963186)
\curveto(452.20881547,88.61962856)(452.18381549,88.74462843)(452.15380493,88.87463186)
\curveto(452.15381552,88.94462823)(452.15381552,89.00962817)(452.15380493,89.06963186)
\curveto(452.14381553,89.12962805)(452.13381554,89.19462798)(452.12380493,89.26463186)
}
}
{
\newrgbcolor{curcolor}{0 0 0}
\pscustom[linestyle=none,fillstyle=solid,fillcolor=curcolor]
{
\newpath
\moveto(457.64380493,101.36424123)
\lineto(457.89880493,101.36424123)
\curveto(457.9788097,101.37423353)(458.05380962,101.36923353)(458.12380493,101.34924123)
\lineto(458.36380493,101.34924123)
\lineto(458.52880493,101.34924123)
\curveto(458.62880905,101.32923357)(458.73380894,101.31923358)(458.84380493,101.31924123)
\curveto(458.94380873,101.31923358)(459.04380863,101.30923359)(459.14380493,101.28924123)
\lineto(459.29380493,101.28924123)
\curveto(459.43380824,101.25923364)(459.5738081,101.23923366)(459.71380493,101.22924123)
\curveto(459.84380783,101.21923368)(459.9738077,101.19423371)(460.10380493,101.15424123)
\curveto(460.18380749,101.13423377)(460.26880741,101.11423379)(460.35880493,101.09424123)
\lineto(460.59880493,101.03424123)
\lineto(460.89880493,100.91424123)
\curveto(460.98880669,100.88423402)(461.0788066,100.84923405)(461.16880493,100.80924123)
\curveto(461.38880629,100.70923419)(461.60380607,100.57423433)(461.81380493,100.40424123)
\curveto(462.02380565,100.24423466)(462.19380548,100.06923483)(462.32380493,99.87924123)
\curveto(462.36380531,99.82923507)(462.40380527,99.76923513)(462.44380493,99.69924123)
\curveto(462.4738052,99.63923526)(462.50880517,99.57923532)(462.54880493,99.51924123)
\curveto(462.59880508,99.43923546)(462.63880504,99.34423556)(462.66880493,99.23424123)
\curveto(462.69880498,99.12423578)(462.72880495,99.01923588)(462.75880493,98.91924123)
\curveto(462.79880488,98.80923609)(462.82380485,98.6992362)(462.83380493,98.58924123)
\curveto(462.84380483,98.47923642)(462.85880482,98.36423654)(462.87880493,98.24424123)
\curveto(462.88880479,98.2042367)(462.88880479,98.15923674)(462.87880493,98.10924123)
\curveto(462.8788048,98.06923683)(462.88380479,98.02923687)(462.89380493,97.98924123)
\curveto(462.90380477,97.94923695)(462.90880477,97.89423701)(462.90880493,97.82424123)
\curveto(462.90880477,97.75423715)(462.90380477,97.7042372)(462.89380493,97.67424123)
\curveto(462.8738048,97.62423728)(462.86880481,97.57923732)(462.87880493,97.53924123)
\curveto(462.88880479,97.4992374)(462.88880479,97.46423744)(462.87880493,97.43424123)
\lineto(462.87880493,97.34424123)
\curveto(462.85880482,97.28423762)(462.84380483,97.21923768)(462.83380493,97.14924123)
\curveto(462.83380484,97.08923781)(462.82880485,97.02423788)(462.81880493,96.95424123)
\curveto(462.76880491,96.78423812)(462.71880496,96.62423828)(462.66880493,96.47424123)
\curveto(462.61880506,96.32423858)(462.55380512,96.17923872)(462.47380493,96.03924123)
\curveto(462.43380524,95.98923891)(462.40380527,95.93423897)(462.38380493,95.87424123)
\curveto(462.35380532,95.82423908)(462.31880536,95.77423913)(462.27880493,95.72424123)
\curveto(462.09880558,95.48423942)(461.8788058,95.28423962)(461.61880493,95.12424123)
\curveto(461.35880632,94.96423994)(461.0738066,94.82424008)(460.76380493,94.70424123)
\curveto(460.62380705,94.64424026)(460.48380719,94.5992403)(460.34380493,94.56924123)
\curveto(460.19380748,94.53924036)(460.03880764,94.5042404)(459.87880493,94.46424123)
\curveto(459.76880791,94.44424046)(459.65880802,94.42924047)(459.54880493,94.41924123)
\curveto(459.43880824,94.40924049)(459.32880835,94.39424051)(459.21880493,94.37424123)
\curveto(459.1788085,94.36424054)(459.13880854,94.35924054)(459.09880493,94.35924123)
\curveto(459.05880862,94.36924053)(459.01880866,94.36924053)(458.97880493,94.35924123)
\curveto(458.92880875,94.34924055)(458.8788088,94.34424056)(458.82880493,94.34424123)
\lineto(458.66380493,94.34424123)
\curveto(458.61380906,94.32424058)(458.56380911,94.31924058)(458.51380493,94.32924123)
\curveto(458.45380922,94.33924056)(458.39880928,94.33924056)(458.34880493,94.32924123)
\curveto(458.30880937,94.31924058)(458.26380941,94.31924058)(458.21380493,94.32924123)
\curveto(458.16380951,94.33924056)(458.11380956,94.33424057)(458.06380493,94.31424123)
\curveto(457.99380968,94.29424061)(457.91880976,94.28924061)(457.83880493,94.29924123)
\curveto(457.74880993,94.30924059)(457.66381001,94.31424059)(457.58380493,94.31424123)
\curveto(457.49381018,94.31424059)(457.39381028,94.30924059)(457.28380493,94.29924123)
\curveto(457.16381051,94.28924061)(457.06381061,94.29424061)(456.98380493,94.31424123)
\lineto(456.69880493,94.31424123)
\lineto(456.06880493,94.35924123)
\curveto(455.96881171,94.36924053)(455.8738118,94.37924052)(455.78380493,94.38924123)
\lineto(455.48380493,94.41924123)
\curveto(455.43381224,94.43924046)(455.38381229,94.44424046)(455.33380493,94.43424123)
\curveto(455.2738124,94.43424047)(455.21881246,94.44424046)(455.16880493,94.46424123)
\curveto(454.99881268,94.51424039)(454.83381284,94.55424035)(454.67380493,94.58424123)
\curveto(454.50381317,94.61424029)(454.34381333,94.66424024)(454.19380493,94.73424123)
\curveto(453.73381394,94.92423998)(453.35881432,95.14423976)(453.06880493,95.39424123)
\curveto(452.7788149,95.65423925)(452.53381514,96.01423889)(452.33380493,96.47424123)
\curveto(452.28381539,96.6042383)(452.24881543,96.73423817)(452.22880493,96.86424123)
\curveto(452.20881547,97.0042379)(452.18381549,97.14423776)(452.15380493,97.28424123)
\curveto(452.14381553,97.35423755)(452.13881554,97.41923748)(452.13880493,97.47924123)
\curveto(452.13881554,97.53923736)(452.13381554,97.6042373)(452.12380493,97.67424123)
\curveto(452.10381557,98.5042364)(452.25381542,99.17423573)(452.57380493,99.68424123)
\curveto(452.88381479,100.19423471)(453.32381435,100.57423433)(453.89380493,100.82424123)
\curveto(454.01381366,100.87423403)(454.13881354,100.91923398)(454.26880493,100.95924123)
\curveto(454.39881328,100.9992339)(454.53381314,101.04423386)(454.67380493,101.09424123)
\curveto(454.75381292,101.11423379)(454.83881284,101.12923377)(454.92880493,101.13924123)
\lineto(455.16880493,101.19924123)
\curveto(455.2788124,101.22923367)(455.38881229,101.24423366)(455.49880493,101.24424123)
\curveto(455.60881207,101.25423365)(455.71881196,101.26923363)(455.82880493,101.28924123)
\curveto(455.8788118,101.30923359)(455.92381175,101.31423359)(455.96380493,101.30424123)
\curveto(456.00381167,101.3042336)(456.04381163,101.30923359)(456.08380493,101.31924123)
\curveto(456.13381154,101.32923357)(456.18881149,101.32923357)(456.24880493,101.31924123)
\curveto(456.29881138,101.31923358)(456.34881133,101.32423358)(456.39880493,101.33424123)
\lineto(456.53380493,101.33424123)
\curveto(456.59381108,101.35423355)(456.66381101,101.35423355)(456.74380493,101.33424123)
\curveto(456.81381086,101.32423358)(456.8788108,101.32923357)(456.93880493,101.34924123)
\curveto(456.96881071,101.35923354)(457.00881067,101.36423354)(457.05880493,101.36424123)
\lineto(457.17880493,101.36424123)
\lineto(457.64380493,101.36424123)
\moveto(459.96880493,99.81924123)
\curveto(459.64880803,99.91923498)(459.28380839,99.97923492)(458.87380493,99.99924123)
\curveto(458.46380921,100.01923488)(458.05380962,100.02923487)(457.64380493,100.02924123)
\curveto(457.21381046,100.02923487)(456.79381088,100.01923488)(456.38380493,99.99924123)
\curveto(455.9738117,99.97923492)(455.58881209,99.93423497)(455.22880493,99.86424123)
\curveto(454.86881281,99.79423511)(454.54881313,99.68423522)(454.26880493,99.53424123)
\curveto(453.9788137,99.39423551)(453.74381393,99.1992357)(453.56380493,98.94924123)
\curveto(453.45381422,98.78923611)(453.3738143,98.60923629)(453.32380493,98.40924123)
\curveto(453.26381441,98.20923669)(453.23381444,97.96423694)(453.23380493,97.67424123)
\curveto(453.25381442,97.65423725)(453.26381441,97.61923728)(453.26380493,97.56924123)
\curveto(453.25381442,97.51923738)(453.25381442,97.47923742)(453.26380493,97.44924123)
\curveto(453.28381439,97.36923753)(453.30381437,97.29423761)(453.32380493,97.22424123)
\curveto(453.33381434,97.16423774)(453.35381432,97.0992378)(453.38380493,97.02924123)
\curveto(453.50381417,96.75923814)(453.673814,96.53923836)(453.89380493,96.36924123)
\curveto(454.10381357,96.20923869)(454.34881333,96.07423883)(454.62880493,95.96424123)
\curveto(454.73881294,95.91423899)(454.85881282,95.87423903)(454.98880493,95.84424123)
\curveto(455.10881257,95.82423908)(455.23381244,95.7992391)(455.36380493,95.76924123)
\curveto(455.41381226,95.74923915)(455.46881221,95.73923916)(455.52880493,95.73924123)
\curveto(455.5788121,95.73923916)(455.62881205,95.73423917)(455.67880493,95.72424123)
\curveto(455.76881191,95.71423919)(455.86381181,95.7042392)(455.96380493,95.69424123)
\curveto(456.05381162,95.68423922)(456.14881153,95.67423923)(456.24880493,95.66424123)
\curveto(456.32881135,95.66423924)(456.41381126,95.65923924)(456.50380493,95.64924123)
\lineto(456.74380493,95.64924123)
\lineto(456.92380493,95.64924123)
\curveto(456.95381072,95.63923926)(456.98881069,95.63423927)(457.02880493,95.63424123)
\lineto(457.16380493,95.63424123)
\lineto(457.61380493,95.63424123)
\curveto(457.69380998,95.63423927)(457.7788099,95.62923927)(457.86880493,95.61924123)
\curveto(457.94880973,95.61923928)(458.02380965,95.62923927)(458.09380493,95.64924123)
\lineto(458.36380493,95.64924123)
\curveto(458.38380929,95.64923925)(458.41380926,95.64423926)(458.45380493,95.63424123)
\curveto(458.48380919,95.63423927)(458.50880917,95.63923926)(458.52880493,95.64924123)
\curveto(458.62880905,95.65923924)(458.72880895,95.66423924)(458.82880493,95.66424123)
\curveto(458.91880876,95.67423923)(459.01880866,95.68423922)(459.12880493,95.69424123)
\curveto(459.24880843,95.72423918)(459.3738083,95.73923916)(459.50380493,95.73924123)
\curveto(459.62380805,95.74923915)(459.73880794,95.77423913)(459.84880493,95.81424123)
\curveto(460.14880753,95.89423901)(460.41380726,95.97923892)(460.64380493,96.06924123)
\curveto(460.8738068,96.16923873)(461.08880659,96.31423859)(461.28880493,96.50424123)
\curveto(461.48880619,96.71423819)(461.63880604,96.97923792)(461.73880493,97.29924123)
\curveto(461.75880592,97.33923756)(461.76880591,97.37423753)(461.76880493,97.40424123)
\curveto(461.75880592,97.44423746)(461.76380591,97.48923741)(461.78380493,97.53924123)
\curveto(461.79380588,97.57923732)(461.80380587,97.64923725)(461.81380493,97.74924123)
\curveto(461.82380585,97.85923704)(461.81880586,97.94423696)(461.79880493,98.00424123)
\curveto(461.7788059,98.07423683)(461.76880591,98.14423676)(461.76880493,98.21424123)
\curveto(461.75880592,98.28423662)(461.74380593,98.34923655)(461.72380493,98.40924123)
\curveto(461.66380601,98.60923629)(461.5788061,98.78923611)(461.46880493,98.94924123)
\curveto(461.44880623,98.97923592)(461.42880625,99.0042359)(461.40880493,99.02424123)
\lineto(461.34880493,99.08424123)
\curveto(461.32880635,99.12423578)(461.28880639,99.17423573)(461.22880493,99.23424123)
\curveto(461.08880659,99.33423557)(460.95880672,99.41923548)(460.83880493,99.48924123)
\curveto(460.71880696,99.55923534)(460.5738071,99.62923527)(460.40380493,99.69924123)
\curveto(460.33380734,99.72923517)(460.26380741,99.74923515)(460.19380493,99.75924123)
\curveto(460.12380755,99.77923512)(460.04880763,99.7992351)(459.96880493,99.81924123)
}
}
{
\newrgbcolor{curcolor}{0 0 0}
\pscustom[linestyle=none,fillstyle=solid,fillcolor=curcolor]
{
\newpath
\moveto(452.12380493,106.77385061)
\curveto(452.12381555,106.87384575)(452.13381554,106.96884566)(452.15380493,107.05885061)
\curveto(452.16381551,107.14884548)(452.19381548,107.21384541)(452.24380493,107.25385061)
\curveto(452.32381535,107.31384531)(452.42881525,107.34384528)(452.55880493,107.34385061)
\lineto(452.94880493,107.34385061)
\lineto(454.44880493,107.34385061)
\lineto(460.83880493,107.34385061)
\lineto(462.00880493,107.34385061)
\lineto(462.32380493,107.34385061)
\curveto(462.42380525,107.35384527)(462.50380517,107.33884529)(462.56380493,107.29885061)
\curveto(462.64380503,107.24884538)(462.69380498,107.17384545)(462.71380493,107.07385061)
\curveto(462.72380495,106.98384564)(462.72880495,106.87384575)(462.72880493,106.74385061)
\lineto(462.72880493,106.51885061)
\curveto(462.70880497,106.43884619)(462.69380498,106.36884626)(462.68380493,106.30885061)
\curveto(462.66380501,106.24884638)(462.62380505,106.19884643)(462.56380493,106.15885061)
\curveto(462.50380517,106.11884651)(462.42880525,106.09884653)(462.33880493,106.09885061)
\lineto(462.03880493,106.09885061)
\lineto(460.94380493,106.09885061)
\lineto(455.60380493,106.09885061)
\curveto(455.51381216,106.07884655)(455.43881224,106.06384656)(455.37880493,106.05385061)
\curveto(455.30881237,106.05384657)(455.24881243,106.0238466)(455.19880493,105.96385061)
\curveto(455.14881253,105.89384673)(455.12381255,105.80384682)(455.12380493,105.69385061)
\curveto(455.11381256,105.59384703)(455.10881257,105.48384714)(455.10880493,105.36385061)
\lineto(455.10880493,104.22385061)
\lineto(455.10880493,103.72885061)
\curveto(455.09881258,103.56884906)(455.03881264,103.45884917)(454.92880493,103.39885061)
\curveto(454.89881278,103.37884925)(454.86881281,103.36884926)(454.83880493,103.36885061)
\curveto(454.79881288,103.36884926)(454.75381292,103.36384926)(454.70380493,103.35385061)
\curveto(454.58381309,103.33384929)(454.4738132,103.33884929)(454.37380493,103.36885061)
\curveto(454.2738134,103.40884922)(454.20381347,103.46384916)(454.16380493,103.53385061)
\curveto(454.11381356,103.61384901)(454.08881359,103.73384889)(454.08880493,103.89385061)
\curveto(454.08881359,104.05384857)(454.0738136,104.18884844)(454.04380493,104.29885061)
\curveto(454.03381364,104.34884828)(454.02881365,104.40384822)(454.02880493,104.46385061)
\curveto(454.01881366,104.5238481)(454.00381367,104.58384804)(453.98380493,104.64385061)
\curveto(453.93381374,104.79384783)(453.88381379,104.93884769)(453.83380493,105.07885061)
\curveto(453.7738139,105.21884741)(453.70381397,105.35384727)(453.62380493,105.48385061)
\curveto(453.53381414,105.623847)(453.42881425,105.74384688)(453.30880493,105.84385061)
\curveto(453.18881449,105.94384668)(453.05881462,106.03884659)(452.91880493,106.12885061)
\curveto(452.81881486,106.18884644)(452.70881497,106.23384639)(452.58880493,106.26385061)
\curveto(452.46881521,106.30384632)(452.36381531,106.35384627)(452.27380493,106.41385061)
\curveto(452.21381546,106.46384616)(452.1738155,106.53384609)(452.15380493,106.62385061)
\curveto(452.14381553,106.64384598)(452.13881554,106.66884596)(452.13880493,106.69885061)
\curveto(452.13881554,106.7288459)(452.13381554,106.75384587)(452.12380493,106.77385061)
}
}
{
\newrgbcolor{curcolor}{0 0 0}
\pscustom[linestyle=none,fillstyle=solid,fillcolor=curcolor]
{
\newpath
\moveto(452.12380493,115.12345998)
\curveto(452.12381555,115.22345513)(452.13381554,115.31845503)(452.15380493,115.40845998)
\curveto(452.16381551,115.49845485)(452.19381548,115.56345479)(452.24380493,115.60345998)
\curveto(452.32381535,115.66345469)(452.42881525,115.69345466)(452.55880493,115.69345998)
\lineto(452.94880493,115.69345998)
\lineto(454.44880493,115.69345998)
\lineto(460.83880493,115.69345998)
\lineto(462.00880493,115.69345998)
\lineto(462.32380493,115.69345998)
\curveto(462.42380525,115.70345465)(462.50380517,115.68845466)(462.56380493,115.64845998)
\curveto(462.64380503,115.59845475)(462.69380498,115.52345483)(462.71380493,115.42345998)
\curveto(462.72380495,115.33345502)(462.72880495,115.22345513)(462.72880493,115.09345998)
\lineto(462.72880493,114.86845998)
\curveto(462.70880497,114.78845556)(462.69380498,114.71845563)(462.68380493,114.65845998)
\curveto(462.66380501,114.59845575)(462.62380505,114.5484558)(462.56380493,114.50845998)
\curveto(462.50380517,114.46845588)(462.42880525,114.4484559)(462.33880493,114.44845998)
\lineto(462.03880493,114.44845998)
\lineto(460.94380493,114.44845998)
\lineto(455.60380493,114.44845998)
\curveto(455.51381216,114.42845592)(455.43881224,114.41345594)(455.37880493,114.40345998)
\curveto(455.30881237,114.40345595)(455.24881243,114.37345598)(455.19880493,114.31345998)
\curveto(455.14881253,114.24345611)(455.12381255,114.1534562)(455.12380493,114.04345998)
\curveto(455.11381256,113.94345641)(455.10881257,113.83345652)(455.10880493,113.71345998)
\lineto(455.10880493,112.57345998)
\lineto(455.10880493,112.07845998)
\curveto(455.09881258,111.91845843)(455.03881264,111.80845854)(454.92880493,111.74845998)
\curveto(454.89881278,111.72845862)(454.86881281,111.71845863)(454.83880493,111.71845998)
\curveto(454.79881288,111.71845863)(454.75381292,111.71345864)(454.70380493,111.70345998)
\curveto(454.58381309,111.68345867)(454.4738132,111.68845866)(454.37380493,111.71845998)
\curveto(454.2738134,111.75845859)(454.20381347,111.81345854)(454.16380493,111.88345998)
\curveto(454.11381356,111.96345839)(454.08881359,112.08345827)(454.08880493,112.24345998)
\curveto(454.08881359,112.40345795)(454.0738136,112.53845781)(454.04380493,112.64845998)
\curveto(454.03381364,112.69845765)(454.02881365,112.7534576)(454.02880493,112.81345998)
\curveto(454.01881366,112.87345748)(454.00381367,112.93345742)(453.98380493,112.99345998)
\curveto(453.93381374,113.14345721)(453.88381379,113.28845706)(453.83380493,113.42845998)
\curveto(453.7738139,113.56845678)(453.70381397,113.70345665)(453.62380493,113.83345998)
\curveto(453.53381414,113.97345638)(453.42881425,114.09345626)(453.30880493,114.19345998)
\curveto(453.18881449,114.29345606)(453.05881462,114.38845596)(452.91880493,114.47845998)
\curveto(452.81881486,114.53845581)(452.70881497,114.58345577)(452.58880493,114.61345998)
\curveto(452.46881521,114.6534557)(452.36381531,114.70345565)(452.27380493,114.76345998)
\curveto(452.21381546,114.81345554)(452.1738155,114.88345547)(452.15380493,114.97345998)
\curveto(452.14381553,114.99345536)(452.13881554,115.01845533)(452.13880493,115.04845998)
\curveto(452.13881554,115.07845527)(452.13381554,115.10345525)(452.12380493,115.12345998)
}
}
{
\newrgbcolor{curcolor}{0 0 0}
\pscustom[linestyle=none,fillstyle=solid,fillcolor=curcolor]
{
\newpath
\moveto(472.96015137,29.18119436)
\lineto(472.96015137,30.09619436)
\curveto(472.96016206,30.19619171)(472.96016206,30.29119161)(472.96015137,30.38119436)
\curveto(472.96016206,30.47119143)(472.98016204,30.54619136)(473.02015137,30.60619436)
\curveto(473.08016194,30.69619121)(473.16016186,30.75619115)(473.26015137,30.78619436)
\curveto(473.36016166,30.82619108)(473.46516156,30.87119103)(473.57515137,30.92119436)
\curveto(473.76516126,31.0011909)(473.95516107,31.07119083)(474.14515137,31.13119436)
\curveto(474.33516069,31.2011907)(474.5251605,31.27619063)(474.71515137,31.35619436)
\curveto(474.89516013,31.42619048)(475.08015994,31.49119041)(475.27015137,31.55119436)
\curveto(475.45015957,31.61119029)(475.63015939,31.68119022)(475.81015137,31.76119436)
\curveto(475.95015907,31.82119008)(476.09515893,31.87619003)(476.24515137,31.92619436)
\curveto(476.39515863,31.97618993)(476.54015848,32.03118987)(476.68015137,32.09119436)
\curveto(477.13015789,32.27118963)(477.58515744,32.44118946)(478.04515137,32.60119436)
\curveto(478.49515653,32.76118914)(478.94515608,32.93118897)(479.39515137,33.11119436)
\curveto(479.44515558,33.13118877)(479.49515553,33.14618876)(479.54515137,33.15619436)
\lineto(479.69515137,33.21619436)
\curveto(479.91515511,33.3061886)(480.14015488,33.39118851)(480.37015137,33.47119436)
\curveto(480.59015443,33.55118835)(480.81015421,33.63618827)(481.03015137,33.72619436)
\curveto(481.1201539,33.76618814)(481.23015379,33.8061881)(481.36015137,33.84619436)
\curveto(481.48015354,33.88618802)(481.55015347,33.95118795)(481.57015137,34.04119436)
\curveto(481.58015344,34.08118782)(481.58015344,34.11118779)(481.57015137,34.13119436)
\lineto(481.51015137,34.19119436)
\curveto(481.46015356,34.24118766)(481.40515362,34.27618763)(481.34515137,34.29619436)
\curveto(481.28515374,34.32618758)(481.2201538,34.35618755)(481.15015137,34.38619436)
\lineto(480.52015137,34.62619436)
\curveto(480.30015472,34.7061872)(480.08515494,34.78618712)(479.87515137,34.86619436)
\lineto(479.72515137,34.92619436)
\lineto(479.54515137,34.98619436)
\curveto(479.35515567,35.06618684)(479.16515586,35.13618677)(478.97515137,35.19619436)
\curveto(478.77515625,35.26618664)(478.57515645,35.34118656)(478.37515137,35.42119436)
\curveto(477.79515723,35.66118624)(477.21015781,35.88118602)(476.62015137,36.08119436)
\curveto(476.03015899,36.29118561)(475.44515958,36.51618539)(474.86515137,36.75619436)
\curveto(474.66516036,36.83618507)(474.46016056,36.91118499)(474.25015137,36.98119436)
\curveto(474.04016098,37.06118484)(473.83516119,37.14118476)(473.63515137,37.22119436)
\curveto(473.55516147,37.26118464)(473.45516157,37.29618461)(473.33515137,37.32619436)
\curveto(473.21516181,37.36618454)(473.13016189,37.42118448)(473.08015137,37.49119436)
\curveto(473.04016198,37.55118435)(473.01016201,37.62618428)(472.99015137,37.71619436)
\curveto(472.97016205,37.81618409)(472.96016206,37.92618398)(472.96015137,38.04619436)
\curveto(472.95016207,38.16618374)(472.95016207,38.28618362)(472.96015137,38.40619436)
\curveto(472.96016206,38.52618338)(472.96016206,38.63618327)(472.96015137,38.73619436)
\curveto(472.96016206,38.82618308)(472.96016206,38.91618299)(472.96015137,39.00619436)
\curveto(472.96016206,39.1061828)(472.98016204,39.18118272)(473.02015137,39.23119436)
\curveto(473.07016195,39.32118258)(473.16016186,39.37118253)(473.29015137,39.38119436)
\curveto(473.4201616,39.39118251)(473.56016146,39.39618251)(473.71015137,39.39619436)
\lineto(475.36015137,39.39619436)
\lineto(481.63015137,39.39619436)
\lineto(482.89015137,39.39619436)
\curveto(483.00015202,39.39618251)(483.11015191,39.39618251)(483.22015137,39.39619436)
\curveto(483.33015169,39.4061825)(483.41515161,39.38618252)(483.47515137,39.33619436)
\curveto(483.53515149,39.3061826)(483.57515145,39.26118264)(483.59515137,39.20119436)
\curveto(483.60515142,39.14118276)(483.6201514,39.07118283)(483.64015137,38.99119436)
\lineto(483.64015137,38.75119436)
\lineto(483.64015137,38.39119436)
\curveto(483.63015139,38.28118362)(483.58515144,38.2011837)(483.50515137,38.15119436)
\curveto(483.47515155,38.13118377)(483.44515158,38.11618379)(483.41515137,38.10619436)
\curveto(483.37515165,38.1061838)(483.33015169,38.09618381)(483.28015137,38.07619436)
\lineto(483.11515137,38.07619436)
\curveto(483.05515197,38.06618384)(482.98515204,38.06118384)(482.90515137,38.06119436)
\curveto(482.8251522,38.07118383)(482.75015227,38.07618383)(482.68015137,38.07619436)
\lineto(481.84015137,38.07619436)
\lineto(477.41515137,38.07619436)
\curveto(477.16515786,38.07618383)(476.91515811,38.07618383)(476.66515137,38.07619436)
\curveto(476.40515862,38.07618383)(476.15515887,38.07118383)(475.91515137,38.06119436)
\curveto(475.81515921,38.06118384)(475.70515932,38.05618385)(475.58515137,38.04619436)
\curveto(475.46515956,38.03618387)(475.40515962,37.98118392)(475.40515137,37.88119436)
\lineto(475.42015137,37.88119436)
\curveto(475.44015958,37.81118409)(475.50515952,37.75118415)(475.61515137,37.70119436)
\curveto(475.7251593,37.66118424)(475.8201592,37.62618428)(475.90015137,37.59619436)
\curveto(476.07015895,37.52618438)(476.24515878,37.46118444)(476.42515137,37.40119436)
\curveto(476.59515843,37.34118456)(476.76515826,37.27118463)(476.93515137,37.19119436)
\curveto(476.98515804,37.17118473)(477.03015799,37.15618475)(477.07015137,37.14619436)
\curveto(477.11015791,37.13618477)(477.15515787,37.12118478)(477.20515137,37.10119436)
\curveto(477.38515764,37.02118488)(477.57015745,36.95118495)(477.76015137,36.89119436)
\curveto(477.94015708,36.84118506)(478.1201569,36.77618513)(478.30015137,36.69619436)
\curveto(478.45015657,36.62618528)(478.60515642,36.56618534)(478.76515137,36.51619436)
\curveto(478.91515611,36.46618544)(479.06515596,36.41118549)(479.21515137,36.35119436)
\curveto(479.68515534,36.15118575)(480.16015486,35.97118593)(480.64015137,35.81119436)
\curveto(481.11015391,35.65118625)(481.57515345,35.47618643)(482.03515137,35.28619436)
\curveto(482.21515281,35.2061867)(482.39515263,35.13618677)(482.57515137,35.07619436)
\curveto(482.75515227,35.01618689)(482.93515209,34.95118695)(483.11515137,34.88119436)
\curveto(483.2251518,34.83118707)(483.33015169,34.78118712)(483.43015137,34.73119436)
\curveto(483.5201515,34.69118721)(483.58515144,34.6061873)(483.62515137,34.47619436)
\curveto(483.63515139,34.45618745)(483.64015138,34.43118747)(483.64015137,34.40119436)
\curveto(483.63015139,34.38118752)(483.63015139,34.35618755)(483.64015137,34.32619436)
\curveto(483.65015137,34.29618761)(483.65515137,34.26118764)(483.65515137,34.22119436)
\curveto(483.64515138,34.18118772)(483.64015138,34.14118776)(483.64015137,34.10119436)
\lineto(483.64015137,33.80119436)
\curveto(483.64015138,33.7011882)(483.61515141,33.62118828)(483.56515137,33.56119436)
\curveto(483.51515151,33.48118842)(483.44515158,33.42118848)(483.35515137,33.38119436)
\curveto(483.25515177,33.35118855)(483.15515187,33.31118859)(483.05515137,33.26119436)
\curveto(482.85515217,33.18118872)(482.65015237,33.1011888)(482.44015137,33.02119436)
\curveto(482.2201528,32.95118895)(482.01015301,32.87618903)(481.81015137,32.79619436)
\curveto(481.63015339,32.71618919)(481.45015357,32.64618926)(481.27015137,32.58619436)
\curveto(481.08015394,32.53618937)(480.89515413,32.47118943)(480.71515137,32.39119436)
\curveto(480.15515487,32.16118974)(479.59015543,31.94618996)(479.02015137,31.74619436)
\curveto(478.45015657,31.54619036)(477.88515714,31.33119057)(477.32515137,31.10119436)
\lineto(476.69515137,30.86119436)
\curveto(476.47515855,30.79119111)(476.26515876,30.71619119)(476.06515137,30.63619436)
\curveto(475.95515907,30.58619132)(475.85015917,30.54119136)(475.75015137,30.50119436)
\curveto(475.64015938,30.47119143)(475.54515948,30.42119148)(475.46515137,30.35119436)
\curveto(475.44515958,30.34119156)(475.43515959,30.33119157)(475.43515137,30.32119436)
\lineto(475.40515137,30.29119436)
\lineto(475.40515137,30.21619436)
\lineto(475.43515137,30.18619436)
\curveto(475.43515959,30.17619173)(475.44015958,30.16619174)(475.45015137,30.15619436)
\curveto(475.50015952,30.13619177)(475.55515947,30.12619178)(475.61515137,30.12619436)
\curveto(475.67515935,30.12619178)(475.73515929,30.11619179)(475.79515137,30.09619436)
\lineto(475.96015137,30.09619436)
\curveto(476.020159,30.07619183)(476.08515894,30.07119183)(476.15515137,30.08119436)
\curveto(476.2251588,30.09119181)(476.29515873,30.09619181)(476.36515137,30.09619436)
\lineto(477.17515137,30.09619436)
\lineto(481.73515137,30.09619436)
\lineto(482.92015137,30.09619436)
\curveto(483.03015199,30.09619181)(483.14015188,30.09119181)(483.25015137,30.08119436)
\curveto(483.36015166,30.08119182)(483.44515158,30.05619185)(483.50515137,30.00619436)
\curveto(483.58515144,29.95619195)(483.63015139,29.86619204)(483.64015137,29.73619436)
\lineto(483.64015137,29.34619436)
\lineto(483.64015137,29.15119436)
\curveto(483.64015138,29.1011928)(483.63015139,29.05119285)(483.61015137,29.00119436)
\curveto(483.57015145,28.87119303)(483.48515154,28.79619311)(483.35515137,28.77619436)
\curveto(483.2251518,28.76619314)(483.07515195,28.76119314)(482.90515137,28.76119436)
\lineto(481.16515137,28.76119436)
\lineto(475.16515137,28.76119436)
\lineto(473.75515137,28.76119436)
\curveto(473.64516138,28.76119314)(473.53016149,28.75619315)(473.41015137,28.74619436)
\curveto(473.29016173,28.74619316)(473.19516183,28.77119313)(473.12515137,28.82119436)
\curveto(473.06516196,28.86119304)(473.01516201,28.93619297)(472.97515137,29.04619436)
\curveto(472.96516206,29.06619284)(472.96516206,29.08619282)(472.97515137,29.10619436)
\curveto(472.97516205,29.13619277)(472.97016205,29.16119274)(472.96015137,29.18119436)
}
}
{
\newrgbcolor{curcolor}{0 0 0}
\pscustom[linestyle=none,fillstyle=solid,fillcolor=curcolor]
{
\newpath
\moveto(483.08515137,48.38330373)
\curveto(483.24515178,48.4132959)(483.38015164,48.39829592)(483.49015137,48.33830373)
\curveto(483.59015143,48.27829604)(483.66515136,48.19829612)(483.71515137,48.09830373)
\curveto(483.73515129,48.04829627)(483.74515128,47.99329632)(483.74515137,47.93330373)
\curveto(483.74515128,47.88329643)(483.75515127,47.82829649)(483.77515137,47.76830373)
\curveto(483.8251512,47.54829677)(483.81015121,47.32829699)(483.73015137,47.10830373)
\curveto(483.66015136,46.89829742)(483.57015145,46.75329756)(483.46015137,46.67330373)
\curveto(483.39015163,46.62329769)(483.31015171,46.57829774)(483.22015137,46.53830373)
\curveto(483.1201519,46.49829782)(483.04015198,46.44829787)(482.98015137,46.38830373)
\curveto(482.96015206,46.36829795)(482.94015208,46.34329797)(482.92015137,46.31330373)
\curveto(482.90015212,46.29329802)(482.89515213,46.26329805)(482.90515137,46.22330373)
\curveto(482.93515209,46.1132982)(482.99015203,46.00829831)(483.07015137,45.90830373)
\curveto(483.15015187,45.8182985)(483.2201518,45.72829859)(483.28015137,45.63830373)
\curveto(483.36015166,45.50829881)(483.43515159,45.36829895)(483.50515137,45.21830373)
\curveto(483.56515146,45.06829925)(483.6201514,44.90829941)(483.67015137,44.73830373)
\curveto(483.70015132,44.63829968)(483.7201513,44.52829979)(483.73015137,44.40830373)
\curveto(483.74015128,44.29830002)(483.75515127,44.18830013)(483.77515137,44.07830373)
\curveto(483.78515124,44.02830029)(483.79015123,43.98330033)(483.79015137,43.94330373)
\lineto(483.79015137,43.83830373)
\curveto(483.81015121,43.72830059)(483.81015121,43.62330069)(483.79015137,43.52330373)
\lineto(483.79015137,43.38830373)
\curveto(483.78015124,43.33830098)(483.77515125,43.28830103)(483.77515137,43.23830373)
\curveto(483.77515125,43.18830113)(483.76515126,43.14330117)(483.74515137,43.10330373)
\curveto(483.73515129,43.06330125)(483.73015129,43.02830129)(483.73015137,42.99830373)
\curveto(483.74015128,42.97830134)(483.74015128,42.95330136)(483.73015137,42.92330373)
\lineto(483.67015137,42.68330373)
\curveto(483.66015136,42.60330171)(483.64015138,42.52830179)(483.61015137,42.45830373)
\curveto(483.48015154,42.15830216)(483.33515169,41.9133024)(483.17515137,41.72330373)
\curveto(483.00515202,41.54330277)(482.77015225,41.39330292)(482.47015137,41.27330373)
\curveto(482.25015277,41.18330313)(481.98515304,41.13830318)(481.67515137,41.13830373)
\lineto(481.36015137,41.13830373)
\curveto(481.31015371,41.14830317)(481.26015376,41.15330316)(481.21015137,41.15330373)
\lineto(481.03015137,41.18330373)
\lineto(480.70015137,41.30330373)
\curveto(480.59015443,41.34330297)(480.49015453,41.39330292)(480.40015137,41.45330373)
\curveto(480.11015491,41.63330268)(479.89515513,41.87830244)(479.75515137,42.18830373)
\curveto(479.61515541,42.49830182)(479.49015553,42.83830148)(479.38015137,43.20830373)
\curveto(479.34015568,43.34830097)(479.31015571,43.49330082)(479.29015137,43.64330373)
\curveto(479.27015575,43.79330052)(479.24515578,43.94330037)(479.21515137,44.09330373)
\curveto(479.19515583,44.16330015)(479.18515584,44.22830009)(479.18515137,44.28830373)
\curveto(479.18515584,44.35829996)(479.17515585,44.43329988)(479.15515137,44.51330373)
\curveto(479.13515589,44.58329973)(479.1251559,44.65329966)(479.12515137,44.72330373)
\curveto(479.11515591,44.79329952)(479.10015592,44.86829945)(479.08015137,44.94830373)
\curveto(479.020156,45.19829912)(478.97015605,45.43329888)(478.93015137,45.65330373)
\curveto(478.88015614,45.87329844)(478.76515626,46.04829827)(478.58515137,46.17830373)
\curveto(478.50515652,46.23829808)(478.40515662,46.28829803)(478.28515137,46.32830373)
\curveto(478.15515687,46.36829795)(478.01515701,46.36829795)(477.86515137,46.32830373)
\curveto(477.6251574,46.26829805)(477.43515759,46.17829814)(477.29515137,46.05830373)
\curveto(477.15515787,45.94829837)(477.04515798,45.78829853)(476.96515137,45.57830373)
\curveto(476.91515811,45.45829886)(476.88015814,45.313299)(476.86015137,45.14330373)
\curveto(476.84015818,44.98329933)(476.83015819,44.8132995)(476.83015137,44.63330373)
\curveto(476.83015819,44.45329986)(476.84015818,44.27830004)(476.86015137,44.10830373)
\curveto(476.88015814,43.93830038)(476.91015811,43.79330052)(476.95015137,43.67330373)
\curveto(477.01015801,43.50330081)(477.09515793,43.33830098)(477.20515137,43.17830373)
\curveto(477.26515776,43.09830122)(477.34515768,43.02330129)(477.44515137,42.95330373)
\curveto(477.53515749,42.89330142)(477.63515739,42.83830148)(477.74515137,42.78830373)
\curveto(477.8251572,42.75830156)(477.91015711,42.72830159)(478.00015137,42.69830373)
\curveto(478.09015693,42.67830164)(478.16015686,42.63330168)(478.21015137,42.56330373)
\curveto(478.24015678,42.52330179)(478.26515676,42.45330186)(478.28515137,42.35330373)
\curveto(478.29515673,42.26330205)(478.30015672,42.16830215)(478.30015137,42.06830373)
\curveto(478.30015672,41.96830235)(478.29515673,41.86830245)(478.28515137,41.76830373)
\curveto(478.26515676,41.67830264)(478.24015678,41.6133027)(478.21015137,41.57330373)
\curveto(478.18015684,41.53330278)(478.13015689,41.50330281)(478.06015137,41.48330373)
\curveto(477.99015703,41.46330285)(477.91515711,41.46330285)(477.83515137,41.48330373)
\curveto(477.70515732,41.5133028)(477.58515744,41.54330277)(477.47515137,41.57330373)
\curveto(477.35515767,41.6133027)(477.24015778,41.65830266)(477.13015137,41.70830373)
\curveto(476.78015824,41.89830242)(476.51015851,42.13830218)(476.32015137,42.42830373)
\curveto(476.1201589,42.7183016)(475.96015906,43.07830124)(475.84015137,43.50830373)
\curveto(475.8201592,43.60830071)(475.80515922,43.70830061)(475.79515137,43.80830373)
\curveto(475.78515924,43.9183004)(475.77015925,44.02830029)(475.75015137,44.13830373)
\curveto(475.74015928,44.17830014)(475.74015928,44.24330007)(475.75015137,44.33330373)
\curveto(475.75015927,44.42329989)(475.74015928,44.47829984)(475.72015137,44.49830373)
\curveto(475.71015931,45.19829912)(475.79015923,45.80829851)(475.96015137,46.32830373)
\curveto(476.13015889,46.84829747)(476.45515857,47.2132971)(476.93515137,47.42330373)
\curveto(477.13515789,47.5132968)(477.37015765,47.56329675)(477.64015137,47.57330373)
\curveto(477.90015712,47.59329672)(478.17515685,47.60329671)(478.46515137,47.60330373)
\lineto(481.78015137,47.60330373)
\curveto(481.9201531,47.60329671)(482.05515297,47.60829671)(482.18515137,47.61830373)
\curveto(482.31515271,47.62829669)(482.4201526,47.65829666)(482.50015137,47.70830373)
\curveto(482.57015245,47.75829656)(482.6201524,47.82329649)(482.65015137,47.90330373)
\curveto(482.69015233,47.99329632)(482.7201523,48.07829624)(482.74015137,48.15830373)
\curveto(482.75015227,48.23829608)(482.79515223,48.29829602)(482.87515137,48.33830373)
\curveto(482.90515212,48.35829596)(482.93515209,48.36829595)(482.96515137,48.36830373)
\curveto(482.99515203,48.36829595)(483.03515199,48.37329594)(483.08515137,48.38330373)
\moveto(481.42015137,46.23830373)
\curveto(481.28015374,46.29829802)(481.1201539,46.32829799)(480.94015137,46.32830373)
\curveto(480.75015427,46.33829798)(480.55515447,46.34329797)(480.35515137,46.34330373)
\curveto(480.24515478,46.34329797)(480.14515488,46.33829798)(480.05515137,46.32830373)
\curveto(479.96515506,46.318298)(479.89515513,46.27829804)(479.84515137,46.20830373)
\curveto(479.8251552,46.17829814)(479.81515521,46.10829821)(479.81515137,45.99830373)
\curveto(479.83515519,45.97829834)(479.84515518,45.94329837)(479.84515137,45.89330373)
\curveto(479.84515518,45.84329847)(479.85515517,45.79829852)(479.87515137,45.75830373)
\curveto(479.89515513,45.67829864)(479.91515511,45.58829873)(479.93515137,45.48830373)
\lineto(479.99515137,45.18830373)
\curveto(479.99515503,45.15829916)(480.00015502,45.12329919)(480.01015137,45.08330373)
\lineto(480.01015137,44.97830373)
\curveto(480.05015497,44.82829949)(480.07515495,44.66329965)(480.08515137,44.48330373)
\curveto(480.08515494,44.3133)(480.10515492,44.15330016)(480.14515137,44.00330373)
\curveto(480.16515486,43.92330039)(480.18515484,43.84830047)(480.20515137,43.77830373)
\curveto(480.21515481,43.7183006)(480.23015479,43.64830067)(480.25015137,43.56830373)
\curveto(480.30015472,43.40830091)(480.36515466,43.25830106)(480.44515137,43.11830373)
\curveto(480.51515451,42.97830134)(480.60515442,42.85830146)(480.71515137,42.75830373)
\curveto(480.8251542,42.65830166)(480.96015406,42.58330173)(481.12015137,42.53330373)
\curveto(481.27015375,42.48330183)(481.45515357,42.46330185)(481.67515137,42.47330373)
\curveto(481.77515325,42.47330184)(481.87015315,42.48830183)(481.96015137,42.51830373)
\curveto(482.04015298,42.55830176)(482.11515291,42.60330171)(482.18515137,42.65330373)
\curveto(482.29515273,42.73330158)(482.39015263,42.83830148)(482.47015137,42.96830373)
\curveto(482.54015248,43.09830122)(482.60015242,43.23830108)(482.65015137,43.38830373)
\curveto(482.66015236,43.43830088)(482.66515236,43.48830083)(482.66515137,43.53830373)
\curveto(482.66515236,43.58830073)(482.67015235,43.63830068)(482.68015137,43.68830373)
\curveto(482.70015232,43.75830056)(482.71515231,43.84330047)(482.72515137,43.94330373)
\curveto(482.7251523,44.05330026)(482.71515231,44.14330017)(482.69515137,44.21330373)
\curveto(482.67515235,44.27330004)(482.67015235,44.33329998)(482.68015137,44.39330373)
\curveto(482.68015234,44.45329986)(482.67015235,44.5132998)(482.65015137,44.57330373)
\curveto(482.63015239,44.65329966)(482.61515241,44.72829959)(482.60515137,44.79830373)
\curveto(482.59515243,44.87829944)(482.57515245,44.95329936)(482.54515137,45.02330373)
\curveto(482.4251526,45.313299)(482.28015274,45.55829876)(482.11015137,45.75830373)
\curveto(481.94015308,45.96829835)(481.71015331,46.12829819)(481.42015137,46.23830373)
}
}
{
\newrgbcolor{curcolor}{0 0 0}
\pscustom[linestyle=none,fillstyle=solid,fillcolor=curcolor]
{
\newpath
\moveto(475.91515137,49.26994436)
\lineto(475.91515137,49.71994436)
\curveto(475.90515912,49.88994311)(475.9251591,50.01494298)(475.97515137,50.09494436)
\curveto(476.025159,50.17494282)(476.09015893,50.22994277)(476.17015137,50.25994436)
\curveto(476.25015877,50.2999427)(476.33515869,50.33994266)(476.42515137,50.37994436)
\curveto(476.55515847,50.42994257)(476.68515834,50.47494252)(476.81515137,50.51494436)
\curveto(476.94515808,50.55494244)(477.07515795,50.5999424)(477.20515137,50.64994436)
\curveto(477.3251577,50.6999423)(477.45015757,50.74494225)(477.58015137,50.78494436)
\curveto(477.70015732,50.82494217)(477.8201572,50.86994213)(477.94015137,50.91994436)
\curveto(478.05015697,50.96994203)(478.16515686,51.00994199)(478.28515137,51.03994436)
\curveto(478.40515662,51.06994193)(478.5251565,51.10994189)(478.64515137,51.15994436)
\curveto(478.93515609,51.27994172)(479.23515579,51.38994161)(479.54515137,51.48994436)
\curveto(479.85515517,51.58994141)(480.15515487,51.6999413)(480.44515137,51.81994436)
\curveto(480.48515454,51.83994116)(480.5251545,51.84994115)(480.56515137,51.84994436)
\curveto(480.59515443,51.84994115)(480.6251544,51.85994114)(480.65515137,51.87994436)
\curveto(480.79515423,51.93994106)(480.94015408,51.994941)(481.09015137,52.04494436)
\lineto(481.51015137,52.19494436)
\curveto(481.58015344,52.22494077)(481.65515337,52.25494074)(481.73515137,52.28494436)
\curveto(481.80515322,52.31494068)(481.85015317,52.36494063)(481.87015137,52.43494436)
\curveto(481.90015312,52.51494048)(481.87515315,52.57494042)(481.79515137,52.61494436)
\curveto(481.70515332,52.66494033)(481.63515339,52.6999403)(481.58515137,52.71994436)
\curveto(481.41515361,52.7999402)(481.23515379,52.86494013)(481.04515137,52.91494436)
\curveto(480.85515417,52.96494003)(480.67015435,53.02493997)(480.49015137,53.09494436)
\curveto(480.26015476,53.18493981)(480.03015499,53.26493973)(479.80015137,53.33494436)
\curveto(479.56015546,53.40493959)(479.33015569,53.48993951)(479.11015137,53.58994436)
\curveto(479.06015596,53.5999394)(478.99515603,53.61493938)(478.91515137,53.63494436)
\curveto(478.8251562,53.67493932)(478.73515629,53.70993929)(478.64515137,53.73994436)
\curveto(478.54515648,53.76993923)(478.45515657,53.7999392)(478.37515137,53.82994436)
\curveto(478.3251567,53.84993915)(478.28015674,53.86493913)(478.24015137,53.87494436)
\curveto(478.20015682,53.88493911)(478.15515687,53.8999391)(478.10515137,53.91994436)
\curveto(477.98515704,53.96993903)(477.86515716,54.00993899)(477.74515137,54.03994436)
\curveto(477.61515741,54.07993892)(477.49015753,54.12493887)(477.37015137,54.17494436)
\curveto(477.3201577,54.1949388)(477.27515775,54.20993879)(477.23515137,54.21994436)
\curveto(477.19515783,54.22993877)(477.15015787,54.24493875)(477.10015137,54.26494436)
\curveto(477.01015801,54.30493869)(476.9201581,54.33993866)(476.83015137,54.36994436)
\curveto(476.73015829,54.3999386)(476.63515839,54.42993857)(476.54515137,54.45994436)
\curveto(476.46515856,54.48993851)(476.38515864,54.51493848)(476.30515137,54.53494436)
\curveto(476.21515881,54.56493843)(476.14015888,54.60493839)(476.08015137,54.65494436)
\curveto(475.99015903,54.72493827)(475.94015908,54.81993818)(475.93015137,54.93994436)
\curveto(475.9201591,55.06993793)(475.91515911,55.20993779)(475.91515137,55.35994436)
\curveto(475.91515911,55.43993756)(475.9201591,55.51493748)(475.93015137,55.58494436)
\curveto(475.93015909,55.66493733)(475.94515908,55.72993727)(475.97515137,55.77994436)
\curveto(476.03515899,55.86993713)(476.13015889,55.8949371)(476.26015137,55.85494436)
\curveto(476.39015863,55.81493718)(476.49015853,55.77993722)(476.56015137,55.74994436)
\lineto(476.62015137,55.71994436)
\curveto(476.64015838,55.71993728)(476.66015836,55.71493728)(476.68015137,55.70494436)
\curveto(476.96015806,55.5949374)(477.24515778,55.48493751)(477.53515137,55.37494436)
\lineto(478.37515137,55.04494436)
\curveto(478.45515657,55.01493798)(478.53015649,54.98993801)(478.60015137,54.96994436)
\curveto(478.66015636,54.94993805)(478.7251563,54.92493807)(478.79515137,54.89494436)
\curveto(478.99515603,54.81493818)(479.20015582,54.73493826)(479.41015137,54.65494436)
\curveto(479.61015541,54.58493841)(479.81015521,54.50993849)(480.01015137,54.42994436)
\curveto(480.70015432,54.13993886)(481.39515363,53.86993913)(482.09515137,53.61994436)
\curveto(482.79515223,53.36993963)(483.49015153,53.0999399)(484.18015137,52.80994436)
\lineto(484.33015137,52.74994436)
\curveto(484.39015063,52.73994026)(484.45015057,52.72494027)(484.51015137,52.70494436)
\curveto(484.88015014,52.54494045)(485.24514978,52.37494062)(485.60515137,52.19494436)
\curveto(485.97514905,52.01494098)(486.26014876,51.76494123)(486.46015137,51.44494436)
\curveto(486.5201485,51.33494166)(486.56514846,51.22494177)(486.59515137,51.11494436)
\curveto(486.63514839,51.00494199)(486.67014835,50.87994212)(486.70015137,50.73994436)
\curveto(486.7201483,50.68994231)(486.7251483,50.63494236)(486.71515137,50.57494436)
\curveto(486.70514832,50.52494247)(486.70514832,50.46994253)(486.71515137,50.40994436)
\curveto(486.73514829,50.32994267)(486.73514829,50.24994275)(486.71515137,50.16994436)
\curveto(486.70514832,50.12994287)(486.70014832,50.07994292)(486.70015137,50.01994436)
\lineto(486.64015137,49.77994436)
\curveto(486.6201484,49.70994329)(486.58014844,49.65494334)(486.52015137,49.61494436)
\curveto(486.46014856,49.56494343)(486.38514864,49.53494346)(486.29515137,49.52494436)
\lineto(486.02515137,49.52494436)
\lineto(485.81515137,49.52494436)
\curveto(485.75514927,49.53494346)(485.70514932,49.55494344)(485.66515137,49.58494436)
\curveto(485.55514947,49.65494334)(485.5251495,49.77494322)(485.57515137,49.94494436)
\curveto(485.59514943,50.05494294)(485.60514942,50.17494282)(485.60515137,50.30494436)
\curveto(485.60514942,50.43494256)(485.58514944,50.54994245)(485.54515137,50.64994436)
\curveto(485.49514953,50.7999422)(485.4201496,50.91994208)(485.32015137,51.00994436)
\curveto(485.2201498,51.10994189)(485.10514992,51.1949418)(484.97515137,51.26494436)
\curveto(484.85515017,51.33494166)(484.7251503,51.3949416)(484.58515137,51.44494436)
\lineto(484.16515137,51.62494436)
\curveto(484.07515095,51.66494133)(483.96515106,51.70494129)(483.83515137,51.74494436)
\curveto(483.70515132,51.7949412)(483.57015145,51.7999412)(483.43015137,51.75994436)
\curveto(483.27015175,51.70994129)(483.1201519,51.65494134)(482.98015137,51.59494436)
\curveto(482.84015218,51.54494145)(482.70015232,51.48994151)(482.56015137,51.42994436)
\curveto(482.35015267,51.33994166)(482.14015288,51.25494174)(481.93015137,51.17494436)
\curveto(481.7201533,51.0949419)(481.51515351,51.01494198)(481.31515137,50.93494436)
\curveto(481.17515385,50.87494212)(481.04015398,50.81994218)(480.91015137,50.76994436)
\curveto(480.78015424,50.71994228)(480.64515438,50.66994233)(480.50515137,50.61994436)
\lineto(479.18515137,50.07994436)
\curveto(478.74515628,49.90994309)(478.30515672,49.73494326)(477.86515137,49.55494436)
\curveto(477.63515739,49.45494354)(477.41515761,49.36494363)(477.20515137,49.28494436)
\curveto(476.98515804,49.20494379)(476.76515826,49.11994388)(476.54515137,49.02994436)
\curveto(476.48515854,49.00994399)(476.40515862,48.97994402)(476.30515137,48.93994436)
\curveto(476.19515883,48.8999441)(476.10515892,48.90494409)(476.03515137,48.95494436)
\curveto(475.98515904,48.98494401)(475.95015907,49.04494395)(475.93015137,49.13494436)
\curveto(475.9201591,49.15494384)(475.9201591,49.17494382)(475.93015137,49.19494436)
\curveto(475.93015909,49.22494377)(475.9251591,49.24994375)(475.91515137,49.26994436)
}
}
{
\newrgbcolor{curcolor}{0 0 0}
\pscustom[linestyle=none,fillstyle=solid,fillcolor=curcolor]
{
}
}
{
\newrgbcolor{curcolor}{0 0 0}
\pscustom[linestyle=none,fillstyle=solid,fillcolor=curcolor]
{
\newpath
\moveto(478.55515137,67.99510061)
\lineto(478.81015137,67.99510061)
\curveto(478.89015613,68.0050929)(478.96515606,68.00009291)(479.03515137,67.98010061)
\lineto(479.27515137,67.98010061)
\lineto(479.44015137,67.98010061)
\curveto(479.54015548,67.96009295)(479.64515538,67.95009296)(479.75515137,67.95010061)
\curveto(479.85515517,67.95009296)(479.95515507,67.94009297)(480.05515137,67.92010061)
\lineto(480.20515137,67.92010061)
\curveto(480.34515468,67.89009302)(480.48515454,67.87009304)(480.62515137,67.86010061)
\curveto(480.75515427,67.85009306)(480.88515414,67.82509308)(481.01515137,67.78510061)
\curveto(481.09515393,67.76509314)(481.18015384,67.74509316)(481.27015137,67.72510061)
\lineto(481.51015137,67.66510061)
\lineto(481.81015137,67.54510061)
\curveto(481.90015312,67.51509339)(481.99015303,67.48009343)(482.08015137,67.44010061)
\curveto(482.30015272,67.34009357)(482.51515251,67.2050937)(482.72515137,67.03510061)
\curveto(482.93515209,66.87509403)(483.10515192,66.70009421)(483.23515137,66.51010061)
\curveto(483.27515175,66.46009445)(483.31515171,66.40009451)(483.35515137,66.33010061)
\curveto(483.38515164,66.27009464)(483.4201516,66.2100947)(483.46015137,66.15010061)
\curveto(483.51015151,66.07009484)(483.55015147,65.97509493)(483.58015137,65.86510061)
\curveto(483.61015141,65.75509515)(483.64015138,65.65009526)(483.67015137,65.55010061)
\curveto(483.71015131,65.44009547)(483.73515129,65.33009558)(483.74515137,65.22010061)
\curveto(483.75515127,65.1100958)(483.77015125,64.99509591)(483.79015137,64.87510061)
\curveto(483.80015122,64.83509607)(483.80015122,64.79009612)(483.79015137,64.74010061)
\curveto(483.79015123,64.70009621)(483.79515123,64.66009625)(483.80515137,64.62010061)
\curveto(483.81515121,64.58009633)(483.8201512,64.52509638)(483.82015137,64.45510061)
\curveto(483.8201512,64.38509652)(483.81515121,64.33509657)(483.80515137,64.30510061)
\curveto(483.78515124,64.25509665)(483.78015124,64.2100967)(483.79015137,64.17010061)
\curveto(483.80015122,64.13009678)(483.80015122,64.09509681)(483.79015137,64.06510061)
\lineto(483.79015137,63.97510061)
\curveto(483.77015125,63.91509699)(483.75515127,63.85009706)(483.74515137,63.78010061)
\curveto(483.74515128,63.72009719)(483.74015128,63.65509725)(483.73015137,63.58510061)
\curveto(483.68015134,63.41509749)(483.63015139,63.25509765)(483.58015137,63.10510061)
\curveto(483.53015149,62.95509795)(483.46515156,62.8100981)(483.38515137,62.67010061)
\curveto(483.34515168,62.62009829)(483.31515171,62.56509834)(483.29515137,62.50510061)
\curveto(483.26515176,62.45509845)(483.23015179,62.4050985)(483.19015137,62.35510061)
\curveto(483.01015201,62.11509879)(482.79015223,61.91509899)(482.53015137,61.75510061)
\curveto(482.27015275,61.59509931)(481.98515304,61.45509945)(481.67515137,61.33510061)
\curveto(481.53515349,61.27509963)(481.39515363,61.23009968)(481.25515137,61.20010061)
\curveto(481.10515392,61.17009974)(480.95015407,61.13509977)(480.79015137,61.09510061)
\curveto(480.68015434,61.07509983)(480.57015445,61.06009985)(480.46015137,61.05010061)
\curveto(480.35015467,61.04009987)(480.24015478,61.02509988)(480.13015137,61.00510061)
\curveto(480.09015493,60.99509991)(480.05015497,60.99009992)(480.01015137,60.99010061)
\curveto(479.97015505,61.00009991)(479.93015509,61.00009991)(479.89015137,60.99010061)
\curveto(479.84015518,60.98009993)(479.79015523,60.97509993)(479.74015137,60.97510061)
\lineto(479.57515137,60.97510061)
\curveto(479.5251555,60.95509995)(479.47515555,60.95009996)(479.42515137,60.96010061)
\curveto(479.36515566,60.97009994)(479.31015571,60.97009994)(479.26015137,60.96010061)
\curveto(479.2201558,60.95009996)(479.17515585,60.95009996)(479.12515137,60.96010061)
\curveto(479.07515595,60.97009994)(479.025156,60.96509994)(478.97515137,60.94510061)
\curveto(478.90515612,60.92509998)(478.83015619,60.92009999)(478.75015137,60.93010061)
\curveto(478.66015636,60.94009997)(478.57515645,60.94509996)(478.49515137,60.94510061)
\curveto(478.40515662,60.94509996)(478.30515672,60.94009997)(478.19515137,60.93010061)
\curveto(478.07515695,60.92009999)(477.97515705,60.92509998)(477.89515137,60.94510061)
\lineto(477.61015137,60.94510061)
\lineto(476.98015137,60.99010061)
\curveto(476.88015814,61.00009991)(476.78515824,61.0100999)(476.69515137,61.02010061)
\lineto(476.39515137,61.05010061)
\curveto(476.34515868,61.07009984)(476.29515873,61.07509983)(476.24515137,61.06510061)
\curveto(476.18515884,61.06509984)(476.13015889,61.07509983)(476.08015137,61.09510061)
\curveto(475.91015911,61.14509976)(475.74515928,61.18509972)(475.58515137,61.21510061)
\curveto(475.41515961,61.24509966)(475.25515977,61.29509961)(475.10515137,61.36510061)
\curveto(474.64516038,61.55509935)(474.27016075,61.77509913)(473.98015137,62.02510061)
\curveto(473.69016133,62.28509862)(473.44516158,62.64509826)(473.24515137,63.10510061)
\curveto(473.19516183,63.23509767)(473.16016186,63.36509754)(473.14015137,63.49510061)
\curveto(473.1201619,63.63509727)(473.09516193,63.77509713)(473.06515137,63.91510061)
\curveto(473.05516197,63.98509692)(473.05016197,64.05009686)(473.05015137,64.11010061)
\curveto(473.05016197,64.17009674)(473.04516198,64.23509667)(473.03515137,64.30510061)
\curveto(473.01516201,65.13509577)(473.16516186,65.8050951)(473.48515137,66.31510061)
\curveto(473.79516123,66.82509408)(474.23516079,67.2050937)(474.80515137,67.45510061)
\curveto(474.9251601,67.5050934)(475.05015997,67.55009336)(475.18015137,67.59010061)
\curveto(475.31015971,67.63009328)(475.44515958,67.67509323)(475.58515137,67.72510061)
\curveto(475.66515936,67.74509316)(475.75015927,67.76009315)(475.84015137,67.77010061)
\lineto(476.08015137,67.83010061)
\curveto(476.19015883,67.86009305)(476.30015872,67.87509303)(476.41015137,67.87510061)
\curveto(476.5201585,67.88509302)(476.63015839,67.90009301)(476.74015137,67.92010061)
\curveto(476.79015823,67.94009297)(476.83515819,67.94509296)(476.87515137,67.93510061)
\curveto(476.91515811,67.93509297)(476.95515807,67.94009297)(476.99515137,67.95010061)
\curveto(477.04515798,67.96009295)(477.10015792,67.96009295)(477.16015137,67.95010061)
\curveto(477.21015781,67.95009296)(477.26015776,67.95509295)(477.31015137,67.96510061)
\lineto(477.44515137,67.96510061)
\curveto(477.50515752,67.98509292)(477.57515745,67.98509292)(477.65515137,67.96510061)
\curveto(477.7251573,67.95509295)(477.79015723,67.96009295)(477.85015137,67.98010061)
\curveto(477.88015714,67.99009292)(477.9201571,67.99509291)(477.97015137,67.99510061)
\lineto(478.09015137,67.99510061)
\lineto(478.55515137,67.99510061)
\moveto(480.88015137,66.45010061)
\curveto(480.56015446,66.55009436)(480.19515483,66.6100943)(479.78515137,66.63010061)
\curveto(479.37515565,66.65009426)(478.96515606,66.66009425)(478.55515137,66.66010061)
\curveto(478.1251569,66.66009425)(477.70515732,66.65009426)(477.29515137,66.63010061)
\curveto(476.88515814,66.6100943)(476.50015852,66.56509434)(476.14015137,66.49510061)
\curveto(475.78015924,66.42509448)(475.46015956,66.31509459)(475.18015137,66.16510061)
\curveto(474.89016013,66.02509488)(474.65516037,65.83009508)(474.47515137,65.58010061)
\curveto(474.36516066,65.42009549)(474.28516074,65.24009567)(474.23515137,65.04010061)
\curveto(474.17516085,64.84009607)(474.14516088,64.59509631)(474.14515137,64.30510061)
\curveto(474.16516086,64.28509662)(474.17516085,64.25009666)(474.17515137,64.20010061)
\curveto(474.16516086,64.15009676)(474.16516086,64.1100968)(474.17515137,64.08010061)
\curveto(474.19516083,64.00009691)(474.21516081,63.92509698)(474.23515137,63.85510061)
\curveto(474.24516078,63.79509711)(474.26516076,63.73009718)(474.29515137,63.66010061)
\curveto(474.41516061,63.39009752)(474.58516044,63.17009774)(474.80515137,63.00010061)
\curveto(475.01516001,62.84009807)(475.26015976,62.7050982)(475.54015137,62.59510061)
\curveto(475.65015937,62.54509836)(475.77015925,62.5050984)(475.90015137,62.47510061)
\curveto(476.020159,62.45509845)(476.14515888,62.43009848)(476.27515137,62.40010061)
\curveto(476.3251587,62.38009853)(476.38015864,62.37009854)(476.44015137,62.37010061)
\curveto(476.49015853,62.37009854)(476.54015848,62.36509854)(476.59015137,62.35510061)
\curveto(476.68015834,62.34509856)(476.77515825,62.33509857)(476.87515137,62.32510061)
\curveto(476.96515806,62.31509859)(477.06015796,62.3050986)(477.16015137,62.29510061)
\curveto(477.24015778,62.29509861)(477.3251577,62.29009862)(477.41515137,62.28010061)
\lineto(477.65515137,62.28010061)
\lineto(477.83515137,62.28010061)
\curveto(477.86515716,62.27009864)(477.90015712,62.26509864)(477.94015137,62.26510061)
\lineto(478.07515137,62.26510061)
\lineto(478.52515137,62.26510061)
\curveto(478.60515642,62.26509864)(478.69015633,62.26009865)(478.78015137,62.25010061)
\curveto(478.86015616,62.25009866)(478.93515609,62.26009865)(479.00515137,62.28010061)
\lineto(479.27515137,62.28010061)
\curveto(479.29515573,62.28009863)(479.3251557,62.27509863)(479.36515137,62.26510061)
\curveto(479.39515563,62.26509864)(479.4201556,62.27009864)(479.44015137,62.28010061)
\curveto(479.54015548,62.29009862)(479.64015538,62.29509861)(479.74015137,62.29510061)
\curveto(479.83015519,62.3050986)(479.93015509,62.31509859)(480.04015137,62.32510061)
\curveto(480.16015486,62.35509855)(480.28515474,62.37009854)(480.41515137,62.37010061)
\curveto(480.53515449,62.38009853)(480.65015437,62.4050985)(480.76015137,62.44510061)
\curveto(481.06015396,62.52509838)(481.3251537,62.6100983)(481.55515137,62.70010061)
\curveto(481.78515324,62.80009811)(482.00015302,62.94509796)(482.20015137,63.13510061)
\curveto(482.40015262,63.34509756)(482.55015247,63.6100973)(482.65015137,63.93010061)
\curveto(482.67015235,63.97009694)(482.68015234,64.0050969)(482.68015137,64.03510061)
\curveto(482.67015235,64.07509683)(482.67515235,64.12009679)(482.69515137,64.17010061)
\curveto(482.70515232,64.2100967)(482.71515231,64.28009663)(482.72515137,64.38010061)
\curveto(482.73515229,64.49009642)(482.73015229,64.57509633)(482.71015137,64.63510061)
\curveto(482.69015233,64.7050962)(482.68015234,64.77509613)(482.68015137,64.84510061)
\curveto(482.67015235,64.91509599)(482.65515237,64.98009593)(482.63515137,65.04010061)
\curveto(482.57515245,65.24009567)(482.49015253,65.42009549)(482.38015137,65.58010061)
\curveto(482.36015266,65.6100953)(482.34015268,65.63509527)(482.32015137,65.65510061)
\lineto(482.26015137,65.71510061)
\curveto(482.24015278,65.75509515)(482.20015282,65.8050951)(482.14015137,65.86510061)
\curveto(482.00015302,65.96509494)(481.87015315,66.05009486)(481.75015137,66.12010061)
\curveto(481.63015339,66.19009472)(481.48515354,66.26009465)(481.31515137,66.33010061)
\curveto(481.24515378,66.36009455)(481.17515385,66.38009453)(481.10515137,66.39010061)
\curveto(481.03515399,66.4100945)(480.96015406,66.43009448)(480.88015137,66.45010061)
}
}
{
\newrgbcolor{curcolor}{0 0 0}
\pscustom[linestyle=none,fillstyle=solid,fillcolor=curcolor]
{
\newpath
\moveto(473.03515137,73.40470998)
\curveto(473.03516199,73.50470513)(473.04516198,73.59970503)(473.06515137,73.68970998)
\curveto(473.07516195,73.77970485)(473.10516192,73.84470479)(473.15515137,73.88470998)
\curveto(473.23516179,73.94470469)(473.34016168,73.97470466)(473.47015137,73.97470998)
\lineto(473.86015137,73.97470998)
\lineto(475.36015137,73.97470998)
\lineto(481.75015137,73.97470998)
\lineto(482.92015137,73.97470998)
\lineto(483.23515137,73.97470998)
\curveto(483.33515169,73.98470465)(483.41515161,73.96970466)(483.47515137,73.92970998)
\curveto(483.55515147,73.87970475)(483.60515142,73.80470483)(483.62515137,73.70470998)
\curveto(483.63515139,73.61470502)(483.64015138,73.50470513)(483.64015137,73.37470998)
\lineto(483.64015137,73.14970998)
\curveto(483.6201514,73.06970556)(483.60515142,72.99970563)(483.59515137,72.93970998)
\curveto(483.57515145,72.87970575)(483.53515149,72.8297058)(483.47515137,72.78970998)
\curveto(483.41515161,72.74970588)(483.34015168,72.7297059)(483.25015137,72.72970998)
\lineto(482.95015137,72.72970998)
\lineto(481.85515137,72.72970998)
\lineto(476.51515137,72.72970998)
\curveto(476.4251586,72.70970592)(476.35015867,72.69470594)(476.29015137,72.68470998)
\curveto(476.2201588,72.68470595)(476.16015886,72.65470598)(476.11015137,72.59470998)
\curveto(476.06015896,72.52470611)(476.03515899,72.4347062)(476.03515137,72.32470998)
\curveto(476.025159,72.22470641)(476.020159,72.11470652)(476.02015137,71.99470998)
\lineto(476.02015137,70.85470998)
\lineto(476.02015137,70.35970998)
\curveto(476.01015901,70.19970843)(475.95015907,70.08970854)(475.84015137,70.02970998)
\curveto(475.81015921,70.00970862)(475.78015924,69.99970863)(475.75015137,69.99970998)
\curveto(475.71015931,69.99970863)(475.66515936,69.99470864)(475.61515137,69.98470998)
\curveto(475.49515953,69.96470867)(475.38515964,69.96970866)(475.28515137,69.99970998)
\curveto(475.18515984,70.03970859)(475.11515991,70.09470854)(475.07515137,70.16470998)
\curveto(475.02516,70.24470839)(475.00016002,70.36470827)(475.00015137,70.52470998)
\curveto(475.00016002,70.68470795)(474.98516004,70.81970781)(474.95515137,70.92970998)
\curveto(474.94516008,70.97970765)(474.94016008,71.0347076)(474.94015137,71.09470998)
\curveto(474.93016009,71.15470748)(474.91516011,71.21470742)(474.89515137,71.27470998)
\curveto(474.84516018,71.42470721)(474.79516023,71.56970706)(474.74515137,71.70970998)
\curveto(474.68516034,71.84970678)(474.61516041,71.98470665)(474.53515137,72.11470998)
\curveto(474.44516058,72.25470638)(474.34016068,72.37470626)(474.22015137,72.47470998)
\curveto(474.10016092,72.57470606)(473.97016105,72.66970596)(473.83015137,72.75970998)
\curveto(473.73016129,72.81970581)(473.6201614,72.86470577)(473.50015137,72.89470998)
\curveto(473.38016164,72.9347057)(473.27516175,72.98470565)(473.18515137,73.04470998)
\curveto(473.1251619,73.09470554)(473.08516194,73.16470547)(473.06515137,73.25470998)
\curveto(473.05516197,73.27470536)(473.05016197,73.29970533)(473.05015137,73.32970998)
\curveto(473.05016197,73.35970527)(473.04516198,73.38470525)(473.03515137,73.40470998)
}
}
{
\newrgbcolor{curcolor}{0 0 0}
\pscustom[linestyle=none,fillstyle=solid,fillcolor=curcolor]
{
\newpath
\moveto(482.00515137,78.63431936)
\lineto(482.00515137,79.26431936)
\lineto(482.00515137,79.45931936)
\curveto(482.00515302,79.52931683)(482.01515301,79.58931677)(482.03515137,79.63931936)
\curveto(482.07515295,79.70931665)(482.11515291,79.7593166)(482.15515137,79.78931936)
\curveto(482.20515282,79.82931653)(482.27015275,79.84931651)(482.35015137,79.84931936)
\curveto(482.43015259,79.8593165)(482.51515251,79.86431649)(482.60515137,79.86431936)
\lineto(483.32515137,79.86431936)
\curveto(483.80515122,79.86431649)(484.21515081,79.80431655)(484.55515137,79.68431936)
\curveto(484.89515013,79.56431679)(485.17014985,79.36931699)(485.38015137,79.09931936)
\curveto(485.43014959,79.02931733)(485.47514955,78.9593174)(485.51515137,78.88931936)
\curveto(485.56514946,78.82931753)(485.61014941,78.7543176)(485.65015137,78.66431936)
\curveto(485.66014936,78.64431771)(485.67014935,78.61431774)(485.68015137,78.57431936)
\curveto(485.70014932,78.53431782)(485.70514932,78.48931787)(485.69515137,78.43931936)
\curveto(485.66514936,78.34931801)(485.59014943,78.29431806)(485.47015137,78.27431936)
\curveto(485.36014966,78.2543181)(485.26514976,78.26931809)(485.18515137,78.31931936)
\curveto(485.11514991,78.34931801)(485.05014997,78.39431796)(484.99015137,78.45431936)
\curveto(484.94015008,78.52431783)(484.89015013,78.58931777)(484.84015137,78.64931936)
\curveto(484.79015023,78.71931764)(484.71515031,78.77931758)(484.61515137,78.82931936)
\curveto(484.5251505,78.88931747)(484.43515059,78.93931742)(484.34515137,78.97931936)
\curveto(484.31515071,78.99931736)(484.25515077,79.02431733)(484.16515137,79.05431936)
\curveto(484.08515094,79.08431727)(484.01515101,79.08931727)(483.95515137,79.06931936)
\curveto(483.81515121,79.03931732)(483.7251513,78.97931738)(483.68515137,78.88931936)
\curveto(483.65515137,78.80931755)(483.64015138,78.71931764)(483.64015137,78.61931936)
\curveto(483.64015138,78.51931784)(483.61515141,78.43431792)(483.56515137,78.36431936)
\curveto(483.49515153,78.27431808)(483.35515167,78.22931813)(483.14515137,78.22931936)
\lineto(482.59015137,78.22931936)
\lineto(482.36515137,78.22931936)
\curveto(482.28515274,78.23931812)(482.2201528,78.2593181)(482.17015137,78.28931936)
\curveto(482.09015293,78.34931801)(482.04515298,78.41931794)(482.03515137,78.49931936)
\curveto(482.025153,78.51931784)(482.020153,78.53931782)(482.02015137,78.55931936)
\curveto(482.020153,78.58931777)(482.01515301,78.61431774)(482.00515137,78.63431936)
}
}
{
\newrgbcolor{curcolor}{0 0 0}
\pscustom[linestyle=none,fillstyle=solid,fillcolor=curcolor]
{
}
}
{
\newrgbcolor{curcolor}{0 0 0}
\pscustom[linestyle=none,fillstyle=solid,fillcolor=curcolor]
{
\newpath
\moveto(473.03515137,89.26463186)
\curveto(473.025162,89.95462722)(473.14516188,90.55462662)(473.39515137,91.06463186)
\curveto(473.64516138,91.58462559)(473.98016104,91.9796252)(474.40015137,92.24963186)
\curveto(474.48016054,92.29962488)(474.57016045,92.34462483)(474.67015137,92.38463186)
\curveto(474.76016026,92.42462475)(474.85516017,92.46962471)(474.95515137,92.51963186)
\curveto(475.05515997,92.55962462)(475.15515987,92.58962459)(475.25515137,92.60963186)
\curveto(475.35515967,92.62962455)(475.46015956,92.64962453)(475.57015137,92.66963186)
\curveto(475.6201594,92.68962449)(475.66515936,92.69462448)(475.70515137,92.68463186)
\curveto(475.74515928,92.6746245)(475.79015923,92.6796245)(475.84015137,92.69963186)
\curveto(475.89015913,92.70962447)(475.97515905,92.71462446)(476.09515137,92.71463186)
\curveto(476.20515882,92.71462446)(476.29015873,92.70962447)(476.35015137,92.69963186)
\curveto(476.41015861,92.6796245)(476.47015855,92.66962451)(476.53015137,92.66963186)
\curveto(476.59015843,92.6796245)(476.65015837,92.6746245)(476.71015137,92.65463186)
\curveto(476.85015817,92.61462456)(476.98515804,92.5796246)(477.11515137,92.54963186)
\curveto(477.24515778,92.51962466)(477.37015765,92.4796247)(477.49015137,92.42963186)
\curveto(477.63015739,92.36962481)(477.75515727,92.29962488)(477.86515137,92.21963186)
\curveto(477.97515705,92.14962503)(478.08515694,92.0746251)(478.19515137,91.99463186)
\lineto(478.25515137,91.93463186)
\curveto(478.27515675,91.92462525)(478.29515673,91.90962527)(478.31515137,91.88963186)
\curveto(478.47515655,91.76962541)(478.6201564,91.63462554)(478.75015137,91.48463186)
\curveto(478.88015614,91.33462584)(479.00515602,91.174626)(479.12515137,91.00463186)
\curveto(479.34515568,90.69462648)(479.55015547,90.39962678)(479.74015137,90.11963186)
\curveto(479.88015514,89.88962729)(480.01515501,89.65962752)(480.14515137,89.42963186)
\curveto(480.27515475,89.20962797)(480.41015461,88.98962819)(480.55015137,88.76963186)
\curveto(480.7201543,88.51962866)(480.90015412,88.2796289)(481.09015137,88.04963186)
\curveto(481.28015374,87.82962935)(481.50515352,87.63962954)(481.76515137,87.47963186)
\curveto(481.8251532,87.43962974)(481.88515314,87.40462977)(481.94515137,87.37463186)
\curveto(481.99515303,87.34462983)(482.06015296,87.31462986)(482.14015137,87.28463186)
\curveto(482.21015281,87.26462991)(482.27015275,87.25962992)(482.32015137,87.26963186)
\curveto(482.39015263,87.28962989)(482.44515258,87.32462985)(482.48515137,87.37463186)
\curveto(482.51515251,87.42462975)(482.53515249,87.48462969)(482.54515137,87.55463186)
\lineto(482.54515137,87.79463186)
\lineto(482.54515137,88.54463186)
\lineto(482.54515137,91.34963186)
\lineto(482.54515137,92.00963186)
\curveto(482.54515248,92.09962508)(482.55015247,92.18462499)(482.56015137,92.26463186)
\curveto(482.56015246,92.34462483)(482.58015244,92.40962477)(482.62015137,92.45963186)
\curveto(482.66015236,92.50962467)(482.73515229,92.54962463)(482.84515137,92.57963186)
\curveto(482.94515208,92.61962456)(483.04515198,92.62962455)(483.14515137,92.60963186)
\lineto(483.28015137,92.60963186)
\curveto(483.35015167,92.58962459)(483.41015161,92.56962461)(483.46015137,92.54963186)
\curveto(483.51015151,92.52962465)(483.55015147,92.49462468)(483.58015137,92.44463186)
\curveto(483.6201514,92.39462478)(483.64015138,92.32462485)(483.64015137,92.23463186)
\lineto(483.64015137,91.96463186)
\lineto(483.64015137,91.06463186)
\lineto(483.64015137,87.55463186)
\lineto(483.64015137,86.48963186)
\curveto(483.64015138,86.40963077)(483.64515138,86.31963086)(483.65515137,86.21963186)
\curveto(483.65515137,86.11963106)(483.64515138,86.03463114)(483.62515137,85.96463186)
\curveto(483.55515147,85.75463142)(483.37515165,85.68963149)(483.08515137,85.76963186)
\curveto(483.04515198,85.7796314)(483.01015201,85.7796314)(482.98015137,85.76963186)
\curveto(482.94015208,85.76963141)(482.89515213,85.7796314)(482.84515137,85.79963186)
\curveto(482.76515226,85.81963136)(482.68015234,85.83963134)(482.59015137,85.85963186)
\curveto(482.50015252,85.8796313)(482.41515261,85.90463127)(482.33515137,85.93463186)
\curveto(481.84515318,86.09463108)(481.43015359,86.29463088)(481.09015137,86.53463186)
\curveto(480.84015418,86.71463046)(480.61515441,86.91963026)(480.41515137,87.14963186)
\curveto(480.20515482,87.3796298)(480.01015501,87.61962956)(479.83015137,87.86963186)
\curveto(479.65015537,88.12962905)(479.48015554,88.39462878)(479.32015137,88.66463186)
\curveto(479.15015587,88.94462823)(478.97515605,89.21462796)(478.79515137,89.47463186)
\curveto(478.71515631,89.58462759)(478.64015638,89.68962749)(478.57015137,89.78963186)
\curveto(478.50015652,89.89962728)(478.4251566,90.00962717)(478.34515137,90.11963186)
\curveto(478.31515671,90.15962702)(478.28515674,90.19462698)(478.25515137,90.22463186)
\curveto(478.21515681,90.26462691)(478.18515684,90.30462687)(478.16515137,90.34463186)
\curveto(478.05515697,90.48462669)(477.93015709,90.60962657)(477.79015137,90.71963186)
\curveto(477.76015726,90.73962644)(477.73515729,90.76462641)(477.71515137,90.79463186)
\curveto(477.68515734,90.82462635)(477.65515737,90.84962633)(477.62515137,90.86963186)
\curveto(477.5251575,90.94962623)(477.4251576,91.01462616)(477.32515137,91.06463186)
\curveto(477.2251578,91.12462605)(477.11515791,91.179626)(476.99515137,91.22963186)
\curveto(476.9251581,91.25962592)(476.85015817,91.2796259)(476.77015137,91.28963186)
\lineto(476.53015137,91.34963186)
\lineto(476.44015137,91.34963186)
\curveto(476.41015861,91.35962582)(476.38015864,91.36462581)(476.35015137,91.36463186)
\curveto(476.28015874,91.38462579)(476.18515884,91.38962579)(476.06515137,91.37963186)
\curveto(475.93515909,91.3796258)(475.83515919,91.36962581)(475.76515137,91.34963186)
\curveto(475.68515934,91.32962585)(475.61015941,91.30962587)(475.54015137,91.28963186)
\curveto(475.46015956,91.2796259)(475.38015964,91.25962592)(475.30015137,91.22963186)
\curveto(475.06015996,91.11962606)(474.86016016,90.96962621)(474.70015137,90.77963186)
\curveto(474.53016049,90.59962658)(474.39016063,90.3796268)(474.28015137,90.11963186)
\curveto(474.26016076,90.04962713)(474.24516078,89.9796272)(474.23515137,89.90963186)
\curveto(474.21516081,89.83962734)(474.19516083,89.76462741)(474.17515137,89.68463186)
\curveto(474.15516087,89.60462757)(474.14516088,89.49462768)(474.14515137,89.35463186)
\curveto(474.14516088,89.22462795)(474.15516087,89.11962806)(474.17515137,89.03963186)
\curveto(474.18516084,88.9796282)(474.19016083,88.92462825)(474.19015137,88.87463186)
\curveto(474.19016083,88.82462835)(474.20016082,88.7746284)(474.22015137,88.72463186)
\curveto(474.26016076,88.62462855)(474.30016072,88.52962865)(474.34015137,88.43963186)
\curveto(474.38016064,88.35962882)(474.4251606,88.2796289)(474.47515137,88.19963186)
\curveto(474.49516053,88.16962901)(474.5201605,88.13962904)(474.55015137,88.10963186)
\curveto(474.58016044,88.08962909)(474.60516042,88.06462911)(474.62515137,88.03463186)
\lineto(474.70015137,87.95963186)
\curveto(474.7201603,87.92962925)(474.74016028,87.90462927)(474.76015137,87.88463186)
\lineto(474.97015137,87.73463186)
\curveto(475.03015999,87.69462948)(475.09515993,87.64962953)(475.16515137,87.59963186)
\curveto(475.25515977,87.53962964)(475.36015966,87.48962969)(475.48015137,87.44963186)
\curveto(475.59015943,87.41962976)(475.70015932,87.38462979)(475.81015137,87.34463186)
\curveto(475.9201591,87.30462987)(476.06515896,87.2796299)(476.24515137,87.26963186)
\curveto(476.41515861,87.25962992)(476.54015848,87.22962995)(476.62015137,87.17963186)
\curveto(476.70015832,87.12963005)(476.74515828,87.05463012)(476.75515137,86.95463186)
\curveto(476.76515826,86.85463032)(476.77015825,86.74463043)(476.77015137,86.62463186)
\curveto(476.77015825,86.58463059)(476.77515825,86.54463063)(476.78515137,86.50463186)
\curveto(476.78515824,86.46463071)(476.78015824,86.42963075)(476.77015137,86.39963186)
\curveto(476.75015827,86.34963083)(476.74015828,86.29963088)(476.74015137,86.24963186)
\curveto(476.74015828,86.20963097)(476.73015829,86.16963101)(476.71015137,86.12963186)
\curveto(476.65015837,86.03963114)(476.51515851,85.99463118)(476.30515137,85.99463186)
\lineto(476.18515137,85.99463186)
\curveto(476.1251589,86.00463117)(476.06515896,86.00963117)(476.00515137,86.00963186)
\curveto(475.93515909,86.01963116)(475.87015915,86.02963115)(475.81015137,86.03963186)
\curveto(475.70015932,86.05963112)(475.60015942,86.0796311)(475.51015137,86.09963186)
\curveto(475.41015961,86.11963106)(475.31515971,86.14963103)(475.22515137,86.18963186)
\curveto(475.15515987,86.20963097)(475.09515993,86.22963095)(475.04515137,86.24963186)
\lineto(474.86515137,86.30963186)
\curveto(474.60516042,86.42963075)(474.36016066,86.58463059)(474.13015137,86.77463186)
\curveto(473.90016112,86.9746302)(473.71516131,87.18962999)(473.57515137,87.41963186)
\curveto(473.49516153,87.52962965)(473.43016159,87.64462953)(473.38015137,87.76463186)
\lineto(473.23015137,88.15463186)
\curveto(473.18016184,88.26462891)(473.15016187,88.3796288)(473.14015137,88.49963186)
\curveto(473.1201619,88.61962856)(473.09516193,88.74462843)(473.06515137,88.87463186)
\curveto(473.06516196,88.94462823)(473.06516196,89.00962817)(473.06515137,89.06963186)
\curveto(473.05516197,89.12962805)(473.04516198,89.19462798)(473.03515137,89.26463186)
}
}
{
\newrgbcolor{curcolor}{0 0 0}
\pscustom[linestyle=none,fillstyle=solid,fillcolor=curcolor]
{
\newpath
\moveto(478.55515137,101.36424123)
\lineto(478.81015137,101.36424123)
\curveto(478.89015613,101.37423353)(478.96515606,101.36923353)(479.03515137,101.34924123)
\lineto(479.27515137,101.34924123)
\lineto(479.44015137,101.34924123)
\curveto(479.54015548,101.32923357)(479.64515538,101.31923358)(479.75515137,101.31924123)
\curveto(479.85515517,101.31923358)(479.95515507,101.30923359)(480.05515137,101.28924123)
\lineto(480.20515137,101.28924123)
\curveto(480.34515468,101.25923364)(480.48515454,101.23923366)(480.62515137,101.22924123)
\curveto(480.75515427,101.21923368)(480.88515414,101.19423371)(481.01515137,101.15424123)
\curveto(481.09515393,101.13423377)(481.18015384,101.11423379)(481.27015137,101.09424123)
\lineto(481.51015137,101.03424123)
\lineto(481.81015137,100.91424123)
\curveto(481.90015312,100.88423402)(481.99015303,100.84923405)(482.08015137,100.80924123)
\curveto(482.30015272,100.70923419)(482.51515251,100.57423433)(482.72515137,100.40424123)
\curveto(482.93515209,100.24423466)(483.10515192,100.06923483)(483.23515137,99.87924123)
\curveto(483.27515175,99.82923507)(483.31515171,99.76923513)(483.35515137,99.69924123)
\curveto(483.38515164,99.63923526)(483.4201516,99.57923532)(483.46015137,99.51924123)
\curveto(483.51015151,99.43923546)(483.55015147,99.34423556)(483.58015137,99.23424123)
\curveto(483.61015141,99.12423578)(483.64015138,99.01923588)(483.67015137,98.91924123)
\curveto(483.71015131,98.80923609)(483.73515129,98.6992362)(483.74515137,98.58924123)
\curveto(483.75515127,98.47923642)(483.77015125,98.36423654)(483.79015137,98.24424123)
\curveto(483.80015122,98.2042367)(483.80015122,98.15923674)(483.79015137,98.10924123)
\curveto(483.79015123,98.06923683)(483.79515123,98.02923687)(483.80515137,97.98924123)
\curveto(483.81515121,97.94923695)(483.8201512,97.89423701)(483.82015137,97.82424123)
\curveto(483.8201512,97.75423715)(483.81515121,97.7042372)(483.80515137,97.67424123)
\curveto(483.78515124,97.62423728)(483.78015124,97.57923732)(483.79015137,97.53924123)
\curveto(483.80015122,97.4992374)(483.80015122,97.46423744)(483.79015137,97.43424123)
\lineto(483.79015137,97.34424123)
\curveto(483.77015125,97.28423762)(483.75515127,97.21923768)(483.74515137,97.14924123)
\curveto(483.74515128,97.08923781)(483.74015128,97.02423788)(483.73015137,96.95424123)
\curveto(483.68015134,96.78423812)(483.63015139,96.62423828)(483.58015137,96.47424123)
\curveto(483.53015149,96.32423858)(483.46515156,96.17923872)(483.38515137,96.03924123)
\curveto(483.34515168,95.98923891)(483.31515171,95.93423897)(483.29515137,95.87424123)
\curveto(483.26515176,95.82423908)(483.23015179,95.77423913)(483.19015137,95.72424123)
\curveto(483.01015201,95.48423942)(482.79015223,95.28423962)(482.53015137,95.12424123)
\curveto(482.27015275,94.96423994)(481.98515304,94.82424008)(481.67515137,94.70424123)
\curveto(481.53515349,94.64424026)(481.39515363,94.5992403)(481.25515137,94.56924123)
\curveto(481.10515392,94.53924036)(480.95015407,94.5042404)(480.79015137,94.46424123)
\curveto(480.68015434,94.44424046)(480.57015445,94.42924047)(480.46015137,94.41924123)
\curveto(480.35015467,94.40924049)(480.24015478,94.39424051)(480.13015137,94.37424123)
\curveto(480.09015493,94.36424054)(480.05015497,94.35924054)(480.01015137,94.35924123)
\curveto(479.97015505,94.36924053)(479.93015509,94.36924053)(479.89015137,94.35924123)
\curveto(479.84015518,94.34924055)(479.79015523,94.34424056)(479.74015137,94.34424123)
\lineto(479.57515137,94.34424123)
\curveto(479.5251555,94.32424058)(479.47515555,94.31924058)(479.42515137,94.32924123)
\curveto(479.36515566,94.33924056)(479.31015571,94.33924056)(479.26015137,94.32924123)
\curveto(479.2201558,94.31924058)(479.17515585,94.31924058)(479.12515137,94.32924123)
\curveto(479.07515595,94.33924056)(479.025156,94.33424057)(478.97515137,94.31424123)
\curveto(478.90515612,94.29424061)(478.83015619,94.28924061)(478.75015137,94.29924123)
\curveto(478.66015636,94.30924059)(478.57515645,94.31424059)(478.49515137,94.31424123)
\curveto(478.40515662,94.31424059)(478.30515672,94.30924059)(478.19515137,94.29924123)
\curveto(478.07515695,94.28924061)(477.97515705,94.29424061)(477.89515137,94.31424123)
\lineto(477.61015137,94.31424123)
\lineto(476.98015137,94.35924123)
\curveto(476.88015814,94.36924053)(476.78515824,94.37924052)(476.69515137,94.38924123)
\lineto(476.39515137,94.41924123)
\curveto(476.34515868,94.43924046)(476.29515873,94.44424046)(476.24515137,94.43424123)
\curveto(476.18515884,94.43424047)(476.13015889,94.44424046)(476.08015137,94.46424123)
\curveto(475.91015911,94.51424039)(475.74515928,94.55424035)(475.58515137,94.58424123)
\curveto(475.41515961,94.61424029)(475.25515977,94.66424024)(475.10515137,94.73424123)
\curveto(474.64516038,94.92423998)(474.27016075,95.14423976)(473.98015137,95.39424123)
\curveto(473.69016133,95.65423925)(473.44516158,96.01423889)(473.24515137,96.47424123)
\curveto(473.19516183,96.6042383)(473.16016186,96.73423817)(473.14015137,96.86424123)
\curveto(473.1201619,97.0042379)(473.09516193,97.14423776)(473.06515137,97.28424123)
\curveto(473.05516197,97.35423755)(473.05016197,97.41923748)(473.05015137,97.47924123)
\curveto(473.05016197,97.53923736)(473.04516198,97.6042373)(473.03515137,97.67424123)
\curveto(473.01516201,98.5042364)(473.16516186,99.17423573)(473.48515137,99.68424123)
\curveto(473.79516123,100.19423471)(474.23516079,100.57423433)(474.80515137,100.82424123)
\curveto(474.9251601,100.87423403)(475.05015997,100.91923398)(475.18015137,100.95924123)
\curveto(475.31015971,100.9992339)(475.44515958,101.04423386)(475.58515137,101.09424123)
\curveto(475.66515936,101.11423379)(475.75015927,101.12923377)(475.84015137,101.13924123)
\lineto(476.08015137,101.19924123)
\curveto(476.19015883,101.22923367)(476.30015872,101.24423366)(476.41015137,101.24424123)
\curveto(476.5201585,101.25423365)(476.63015839,101.26923363)(476.74015137,101.28924123)
\curveto(476.79015823,101.30923359)(476.83515819,101.31423359)(476.87515137,101.30424123)
\curveto(476.91515811,101.3042336)(476.95515807,101.30923359)(476.99515137,101.31924123)
\curveto(477.04515798,101.32923357)(477.10015792,101.32923357)(477.16015137,101.31924123)
\curveto(477.21015781,101.31923358)(477.26015776,101.32423358)(477.31015137,101.33424123)
\lineto(477.44515137,101.33424123)
\curveto(477.50515752,101.35423355)(477.57515745,101.35423355)(477.65515137,101.33424123)
\curveto(477.7251573,101.32423358)(477.79015723,101.32923357)(477.85015137,101.34924123)
\curveto(477.88015714,101.35923354)(477.9201571,101.36423354)(477.97015137,101.36424123)
\lineto(478.09015137,101.36424123)
\lineto(478.55515137,101.36424123)
\moveto(480.88015137,99.81924123)
\curveto(480.56015446,99.91923498)(480.19515483,99.97923492)(479.78515137,99.99924123)
\curveto(479.37515565,100.01923488)(478.96515606,100.02923487)(478.55515137,100.02924123)
\curveto(478.1251569,100.02923487)(477.70515732,100.01923488)(477.29515137,99.99924123)
\curveto(476.88515814,99.97923492)(476.50015852,99.93423497)(476.14015137,99.86424123)
\curveto(475.78015924,99.79423511)(475.46015956,99.68423522)(475.18015137,99.53424123)
\curveto(474.89016013,99.39423551)(474.65516037,99.1992357)(474.47515137,98.94924123)
\curveto(474.36516066,98.78923611)(474.28516074,98.60923629)(474.23515137,98.40924123)
\curveto(474.17516085,98.20923669)(474.14516088,97.96423694)(474.14515137,97.67424123)
\curveto(474.16516086,97.65423725)(474.17516085,97.61923728)(474.17515137,97.56924123)
\curveto(474.16516086,97.51923738)(474.16516086,97.47923742)(474.17515137,97.44924123)
\curveto(474.19516083,97.36923753)(474.21516081,97.29423761)(474.23515137,97.22424123)
\curveto(474.24516078,97.16423774)(474.26516076,97.0992378)(474.29515137,97.02924123)
\curveto(474.41516061,96.75923814)(474.58516044,96.53923836)(474.80515137,96.36924123)
\curveto(475.01516001,96.20923869)(475.26015976,96.07423883)(475.54015137,95.96424123)
\curveto(475.65015937,95.91423899)(475.77015925,95.87423903)(475.90015137,95.84424123)
\curveto(476.020159,95.82423908)(476.14515888,95.7992391)(476.27515137,95.76924123)
\curveto(476.3251587,95.74923915)(476.38015864,95.73923916)(476.44015137,95.73924123)
\curveto(476.49015853,95.73923916)(476.54015848,95.73423917)(476.59015137,95.72424123)
\curveto(476.68015834,95.71423919)(476.77515825,95.7042392)(476.87515137,95.69424123)
\curveto(476.96515806,95.68423922)(477.06015796,95.67423923)(477.16015137,95.66424123)
\curveto(477.24015778,95.66423924)(477.3251577,95.65923924)(477.41515137,95.64924123)
\lineto(477.65515137,95.64924123)
\lineto(477.83515137,95.64924123)
\curveto(477.86515716,95.63923926)(477.90015712,95.63423927)(477.94015137,95.63424123)
\lineto(478.07515137,95.63424123)
\lineto(478.52515137,95.63424123)
\curveto(478.60515642,95.63423927)(478.69015633,95.62923927)(478.78015137,95.61924123)
\curveto(478.86015616,95.61923928)(478.93515609,95.62923927)(479.00515137,95.64924123)
\lineto(479.27515137,95.64924123)
\curveto(479.29515573,95.64923925)(479.3251557,95.64423926)(479.36515137,95.63424123)
\curveto(479.39515563,95.63423927)(479.4201556,95.63923926)(479.44015137,95.64924123)
\curveto(479.54015548,95.65923924)(479.64015538,95.66423924)(479.74015137,95.66424123)
\curveto(479.83015519,95.67423923)(479.93015509,95.68423922)(480.04015137,95.69424123)
\curveto(480.16015486,95.72423918)(480.28515474,95.73923916)(480.41515137,95.73924123)
\curveto(480.53515449,95.74923915)(480.65015437,95.77423913)(480.76015137,95.81424123)
\curveto(481.06015396,95.89423901)(481.3251537,95.97923892)(481.55515137,96.06924123)
\curveto(481.78515324,96.16923873)(482.00015302,96.31423859)(482.20015137,96.50424123)
\curveto(482.40015262,96.71423819)(482.55015247,96.97923792)(482.65015137,97.29924123)
\curveto(482.67015235,97.33923756)(482.68015234,97.37423753)(482.68015137,97.40424123)
\curveto(482.67015235,97.44423746)(482.67515235,97.48923741)(482.69515137,97.53924123)
\curveto(482.70515232,97.57923732)(482.71515231,97.64923725)(482.72515137,97.74924123)
\curveto(482.73515229,97.85923704)(482.73015229,97.94423696)(482.71015137,98.00424123)
\curveto(482.69015233,98.07423683)(482.68015234,98.14423676)(482.68015137,98.21424123)
\curveto(482.67015235,98.28423662)(482.65515237,98.34923655)(482.63515137,98.40924123)
\curveto(482.57515245,98.60923629)(482.49015253,98.78923611)(482.38015137,98.94924123)
\curveto(482.36015266,98.97923592)(482.34015268,99.0042359)(482.32015137,99.02424123)
\lineto(482.26015137,99.08424123)
\curveto(482.24015278,99.12423578)(482.20015282,99.17423573)(482.14015137,99.23424123)
\curveto(482.00015302,99.33423557)(481.87015315,99.41923548)(481.75015137,99.48924123)
\curveto(481.63015339,99.55923534)(481.48515354,99.62923527)(481.31515137,99.69924123)
\curveto(481.24515378,99.72923517)(481.17515385,99.74923515)(481.10515137,99.75924123)
\curveto(481.03515399,99.77923512)(480.96015406,99.7992351)(480.88015137,99.81924123)
}
}
{
\newrgbcolor{curcolor}{0 0 0}
\pscustom[linestyle=none,fillstyle=solid,fillcolor=curcolor]
{
\newpath
\moveto(473.03515137,106.77385061)
\curveto(473.03516199,106.87384575)(473.04516198,106.96884566)(473.06515137,107.05885061)
\curveto(473.07516195,107.14884548)(473.10516192,107.21384541)(473.15515137,107.25385061)
\curveto(473.23516179,107.31384531)(473.34016168,107.34384528)(473.47015137,107.34385061)
\lineto(473.86015137,107.34385061)
\lineto(475.36015137,107.34385061)
\lineto(481.75015137,107.34385061)
\lineto(482.92015137,107.34385061)
\lineto(483.23515137,107.34385061)
\curveto(483.33515169,107.35384527)(483.41515161,107.33884529)(483.47515137,107.29885061)
\curveto(483.55515147,107.24884538)(483.60515142,107.17384545)(483.62515137,107.07385061)
\curveto(483.63515139,106.98384564)(483.64015138,106.87384575)(483.64015137,106.74385061)
\lineto(483.64015137,106.51885061)
\curveto(483.6201514,106.43884619)(483.60515142,106.36884626)(483.59515137,106.30885061)
\curveto(483.57515145,106.24884638)(483.53515149,106.19884643)(483.47515137,106.15885061)
\curveto(483.41515161,106.11884651)(483.34015168,106.09884653)(483.25015137,106.09885061)
\lineto(482.95015137,106.09885061)
\lineto(481.85515137,106.09885061)
\lineto(476.51515137,106.09885061)
\curveto(476.4251586,106.07884655)(476.35015867,106.06384656)(476.29015137,106.05385061)
\curveto(476.2201588,106.05384657)(476.16015886,106.0238466)(476.11015137,105.96385061)
\curveto(476.06015896,105.89384673)(476.03515899,105.80384682)(476.03515137,105.69385061)
\curveto(476.025159,105.59384703)(476.020159,105.48384714)(476.02015137,105.36385061)
\lineto(476.02015137,104.22385061)
\lineto(476.02015137,103.72885061)
\curveto(476.01015901,103.56884906)(475.95015907,103.45884917)(475.84015137,103.39885061)
\curveto(475.81015921,103.37884925)(475.78015924,103.36884926)(475.75015137,103.36885061)
\curveto(475.71015931,103.36884926)(475.66515936,103.36384926)(475.61515137,103.35385061)
\curveto(475.49515953,103.33384929)(475.38515964,103.33884929)(475.28515137,103.36885061)
\curveto(475.18515984,103.40884922)(475.11515991,103.46384916)(475.07515137,103.53385061)
\curveto(475.02516,103.61384901)(475.00016002,103.73384889)(475.00015137,103.89385061)
\curveto(475.00016002,104.05384857)(474.98516004,104.18884844)(474.95515137,104.29885061)
\curveto(474.94516008,104.34884828)(474.94016008,104.40384822)(474.94015137,104.46385061)
\curveto(474.93016009,104.5238481)(474.91516011,104.58384804)(474.89515137,104.64385061)
\curveto(474.84516018,104.79384783)(474.79516023,104.93884769)(474.74515137,105.07885061)
\curveto(474.68516034,105.21884741)(474.61516041,105.35384727)(474.53515137,105.48385061)
\curveto(474.44516058,105.623847)(474.34016068,105.74384688)(474.22015137,105.84385061)
\curveto(474.10016092,105.94384668)(473.97016105,106.03884659)(473.83015137,106.12885061)
\curveto(473.73016129,106.18884644)(473.6201614,106.23384639)(473.50015137,106.26385061)
\curveto(473.38016164,106.30384632)(473.27516175,106.35384627)(473.18515137,106.41385061)
\curveto(473.1251619,106.46384616)(473.08516194,106.53384609)(473.06515137,106.62385061)
\curveto(473.05516197,106.64384598)(473.05016197,106.66884596)(473.05015137,106.69885061)
\curveto(473.05016197,106.7288459)(473.04516198,106.75384587)(473.03515137,106.77385061)
}
}
{
\newrgbcolor{curcolor}{0 0 0}
\pscustom[linestyle=none,fillstyle=solid,fillcolor=curcolor]
{
\newpath
\moveto(473.03515137,115.12345998)
\curveto(473.03516199,115.22345513)(473.04516198,115.31845503)(473.06515137,115.40845998)
\curveto(473.07516195,115.49845485)(473.10516192,115.56345479)(473.15515137,115.60345998)
\curveto(473.23516179,115.66345469)(473.34016168,115.69345466)(473.47015137,115.69345998)
\lineto(473.86015137,115.69345998)
\lineto(475.36015137,115.69345998)
\lineto(481.75015137,115.69345998)
\lineto(482.92015137,115.69345998)
\lineto(483.23515137,115.69345998)
\curveto(483.33515169,115.70345465)(483.41515161,115.68845466)(483.47515137,115.64845998)
\curveto(483.55515147,115.59845475)(483.60515142,115.52345483)(483.62515137,115.42345998)
\curveto(483.63515139,115.33345502)(483.64015138,115.22345513)(483.64015137,115.09345998)
\lineto(483.64015137,114.86845998)
\curveto(483.6201514,114.78845556)(483.60515142,114.71845563)(483.59515137,114.65845998)
\curveto(483.57515145,114.59845575)(483.53515149,114.5484558)(483.47515137,114.50845998)
\curveto(483.41515161,114.46845588)(483.34015168,114.4484559)(483.25015137,114.44845998)
\lineto(482.95015137,114.44845998)
\lineto(481.85515137,114.44845998)
\lineto(476.51515137,114.44845998)
\curveto(476.4251586,114.42845592)(476.35015867,114.41345594)(476.29015137,114.40345998)
\curveto(476.2201588,114.40345595)(476.16015886,114.37345598)(476.11015137,114.31345998)
\curveto(476.06015896,114.24345611)(476.03515899,114.1534562)(476.03515137,114.04345998)
\curveto(476.025159,113.94345641)(476.020159,113.83345652)(476.02015137,113.71345998)
\lineto(476.02015137,112.57345998)
\lineto(476.02015137,112.07845998)
\curveto(476.01015901,111.91845843)(475.95015907,111.80845854)(475.84015137,111.74845998)
\curveto(475.81015921,111.72845862)(475.78015924,111.71845863)(475.75015137,111.71845998)
\curveto(475.71015931,111.71845863)(475.66515936,111.71345864)(475.61515137,111.70345998)
\curveto(475.49515953,111.68345867)(475.38515964,111.68845866)(475.28515137,111.71845998)
\curveto(475.18515984,111.75845859)(475.11515991,111.81345854)(475.07515137,111.88345998)
\curveto(475.02516,111.96345839)(475.00016002,112.08345827)(475.00015137,112.24345998)
\curveto(475.00016002,112.40345795)(474.98516004,112.53845781)(474.95515137,112.64845998)
\curveto(474.94516008,112.69845765)(474.94016008,112.7534576)(474.94015137,112.81345998)
\curveto(474.93016009,112.87345748)(474.91516011,112.93345742)(474.89515137,112.99345998)
\curveto(474.84516018,113.14345721)(474.79516023,113.28845706)(474.74515137,113.42845998)
\curveto(474.68516034,113.56845678)(474.61516041,113.70345665)(474.53515137,113.83345998)
\curveto(474.44516058,113.97345638)(474.34016068,114.09345626)(474.22015137,114.19345998)
\curveto(474.10016092,114.29345606)(473.97016105,114.38845596)(473.83015137,114.47845998)
\curveto(473.73016129,114.53845581)(473.6201614,114.58345577)(473.50015137,114.61345998)
\curveto(473.38016164,114.6534557)(473.27516175,114.70345565)(473.18515137,114.76345998)
\curveto(473.1251619,114.81345554)(473.08516194,114.88345547)(473.06515137,114.97345998)
\curveto(473.05516197,114.99345536)(473.05016197,115.01845533)(473.05015137,115.04845998)
\curveto(473.05016197,115.07845527)(473.04516198,115.10345525)(473.03515137,115.12345998)
}
}
{
\newrgbcolor{curcolor}{0 0 0}
\pscustom[linestyle=none,fillstyle=solid,fillcolor=curcolor]
{
\newpath
\moveto(494.90647095,29.18119436)
\lineto(494.90647095,30.09619436)
\curveto(494.90648164,30.19619171)(494.90648164,30.29119161)(494.90647095,30.38119436)
\curveto(494.90648164,30.47119143)(494.92648162,30.54619136)(494.96647095,30.60619436)
\curveto(495.02648152,30.69619121)(495.10648144,30.75619115)(495.20647095,30.78619436)
\curveto(495.30648124,30.82619108)(495.41148114,30.87119103)(495.52147095,30.92119436)
\curveto(495.71148084,31.0011909)(495.90148065,31.07119083)(496.09147095,31.13119436)
\curveto(496.28148027,31.2011907)(496.47148008,31.27619063)(496.66147095,31.35619436)
\curveto(496.84147971,31.42619048)(497.02647952,31.49119041)(497.21647095,31.55119436)
\curveto(497.39647915,31.61119029)(497.57647897,31.68119022)(497.75647095,31.76119436)
\curveto(497.89647865,31.82119008)(498.04147851,31.87619003)(498.19147095,31.92619436)
\curveto(498.34147821,31.97618993)(498.48647806,32.03118987)(498.62647095,32.09119436)
\curveto(499.07647747,32.27118963)(499.53147702,32.44118946)(499.99147095,32.60119436)
\curveto(500.44147611,32.76118914)(500.89147566,32.93118897)(501.34147095,33.11119436)
\curveto(501.39147516,33.13118877)(501.44147511,33.14618876)(501.49147095,33.15619436)
\lineto(501.64147095,33.21619436)
\curveto(501.86147469,33.3061886)(502.08647446,33.39118851)(502.31647095,33.47119436)
\curveto(502.53647401,33.55118835)(502.75647379,33.63618827)(502.97647095,33.72619436)
\curveto(503.06647348,33.76618814)(503.17647337,33.8061881)(503.30647095,33.84619436)
\curveto(503.42647312,33.88618802)(503.49647305,33.95118795)(503.51647095,34.04119436)
\curveto(503.52647302,34.08118782)(503.52647302,34.11118779)(503.51647095,34.13119436)
\lineto(503.45647095,34.19119436)
\curveto(503.40647314,34.24118766)(503.3514732,34.27618763)(503.29147095,34.29619436)
\curveto(503.23147332,34.32618758)(503.16647338,34.35618755)(503.09647095,34.38619436)
\lineto(502.46647095,34.62619436)
\curveto(502.2464743,34.7061872)(502.03147452,34.78618712)(501.82147095,34.86619436)
\lineto(501.67147095,34.92619436)
\lineto(501.49147095,34.98619436)
\curveto(501.30147525,35.06618684)(501.11147544,35.13618677)(500.92147095,35.19619436)
\curveto(500.72147583,35.26618664)(500.52147603,35.34118656)(500.32147095,35.42119436)
\curveto(499.74147681,35.66118624)(499.15647739,35.88118602)(498.56647095,36.08119436)
\curveto(497.97647857,36.29118561)(497.39147916,36.51618539)(496.81147095,36.75619436)
\curveto(496.61147994,36.83618507)(496.40648014,36.91118499)(496.19647095,36.98119436)
\curveto(495.98648056,37.06118484)(495.78148077,37.14118476)(495.58147095,37.22119436)
\curveto(495.50148105,37.26118464)(495.40148115,37.29618461)(495.28147095,37.32619436)
\curveto(495.16148139,37.36618454)(495.07648147,37.42118448)(495.02647095,37.49119436)
\curveto(494.98648156,37.55118435)(494.95648159,37.62618428)(494.93647095,37.71619436)
\curveto(494.91648163,37.81618409)(494.90648164,37.92618398)(494.90647095,38.04619436)
\curveto(494.89648165,38.16618374)(494.89648165,38.28618362)(494.90647095,38.40619436)
\curveto(494.90648164,38.52618338)(494.90648164,38.63618327)(494.90647095,38.73619436)
\curveto(494.90648164,38.82618308)(494.90648164,38.91618299)(494.90647095,39.00619436)
\curveto(494.90648164,39.1061828)(494.92648162,39.18118272)(494.96647095,39.23119436)
\curveto(495.01648153,39.32118258)(495.10648144,39.37118253)(495.23647095,39.38119436)
\curveto(495.36648118,39.39118251)(495.50648104,39.39618251)(495.65647095,39.39619436)
\lineto(497.30647095,39.39619436)
\lineto(503.57647095,39.39619436)
\lineto(504.83647095,39.39619436)
\curveto(504.9464716,39.39618251)(505.05647149,39.39618251)(505.16647095,39.39619436)
\curveto(505.27647127,39.4061825)(505.36147119,39.38618252)(505.42147095,39.33619436)
\curveto(505.48147107,39.3061826)(505.52147103,39.26118264)(505.54147095,39.20119436)
\curveto(505.551471,39.14118276)(505.56647098,39.07118283)(505.58647095,38.99119436)
\lineto(505.58647095,38.75119436)
\lineto(505.58647095,38.39119436)
\curveto(505.57647097,38.28118362)(505.53147102,38.2011837)(505.45147095,38.15119436)
\curveto(505.42147113,38.13118377)(505.39147116,38.11618379)(505.36147095,38.10619436)
\curveto(505.32147123,38.1061838)(505.27647127,38.09618381)(505.22647095,38.07619436)
\lineto(505.06147095,38.07619436)
\curveto(505.00147155,38.06618384)(504.93147162,38.06118384)(504.85147095,38.06119436)
\curveto(504.77147178,38.07118383)(504.69647185,38.07618383)(504.62647095,38.07619436)
\lineto(503.78647095,38.07619436)
\lineto(499.36147095,38.07619436)
\curveto(499.11147744,38.07618383)(498.86147769,38.07618383)(498.61147095,38.07619436)
\curveto(498.3514782,38.07618383)(498.10147845,38.07118383)(497.86147095,38.06119436)
\curveto(497.76147879,38.06118384)(497.6514789,38.05618385)(497.53147095,38.04619436)
\curveto(497.41147914,38.03618387)(497.3514792,37.98118392)(497.35147095,37.88119436)
\lineto(497.36647095,37.88119436)
\curveto(497.38647916,37.81118409)(497.4514791,37.75118415)(497.56147095,37.70119436)
\curveto(497.67147888,37.66118424)(497.76647878,37.62618428)(497.84647095,37.59619436)
\curveto(498.01647853,37.52618438)(498.19147836,37.46118444)(498.37147095,37.40119436)
\curveto(498.54147801,37.34118456)(498.71147784,37.27118463)(498.88147095,37.19119436)
\curveto(498.93147762,37.17118473)(498.97647757,37.15618475)(499.01647095,37.14619436)
\curveto(499.05647749,37.13618477)(499.10147745,37.12118478)(499.15147095,37.10119436)
\curveto(499.33147722,37.02118488)(499.51647703,36.95118495)(499.70647095,36.89119436)
\curveto(499.88647666,36.84118506)(500.06647648,36.77618513)(500.24647095,36.69619436)
\curveto(500.39647615,36.62618528)(500.551476,36.56618534)(500.71147095,36.51619436)
\curveto(500.86147569,36.46618544)(501.01147554,36.41118549)(501.16147095,36.35119436)
\curveto(501.63147492,36.15118575)(502.10647444,35.97118593)(502.58647095,35.81119436)
\curveto(503.05647349,35.65118625)(503.52147303,35.47618643)(503.98147095,35.28619436)
\curveto(504.16147239,35.2061867)(504.34147221,35.13618677)(504.52147095,35.07619436)
\curveto(504.70147185,35.01618689)(504.88147167,34.95118695)(505.06147095,34.88119436)
\curveto(505.17147138,34.83118707)(505.27647127,34.78118712)(505.37647095,34.73119436)
\curveto(505.46647108,34.69118721)(505.53147102,34.6061873)(505.57147095,34.47619436)
\curveto(505.58147097,34.45618745)(505.58647096,34.43118747)(505.58647095,34.40119436)
\curveto(505.57647097,34.38118752)(505.57647097,34.35618755)(505.58647095,34.32619436)
\curveto(505.59647095,34.29618761)(505.60147095,34.26118764)(505.60147095,34.22119436)
\curveto(505.59147096,34.18118772)(505.58647096,34.14118776)(505.58647095,34.10119436)
\lineto(505.58647095,33.80119436)
\curveto(505.58647096,33.7011882)(505.56147099,33.62118828)(505.51147095,33.56119436)
\curveto(505.46147109,33.48118842)(505.39147116,33.42118848)(505.30147095,33.38119436)
\curveto(505.20147135,33.35118855)(505.10147145,33.31118859)(505.00147095,33.26119436)
\curveto(504.80147175,33.18118872)(504.59647195,33.1011888)(504.38647095,33.02119436)
\curveto(504.16647238,32.95118895)(503.95647259,32.87618903)(503.75647095,32.79619436)
\curveto(503.57647297,32.71618919)(503.39647315,32.64618926)(503.21647095,32.58619436)
\curveto(503.02647352,32.53618937)(502.84147371,32.47118943)(502.66147095,32.39119436)
\curveto(502.10147445,32.16118974)(501.53647501,31.94618996)(500.96647095,31.74619436)
\curveto(500.39647615,31.54619036)(499.83147672,31.33119057)(499.27147095,31.10119436)
\lineto(498.64147095,30.86119436)
\curveto(498.42147813,30.79119111)(498.21147834,30.71619119)(498.01147095,30.63619436)
\curveto(497.90147865,30.58619132)(497.79647875,30.54119136)(497.69647095,30.50119436)
\curveto(497.58647896,30.47119143)(497.49147906,30.42119148)(497.41147095,30.35119436)
\curveto(497.39147916,30.34119156)(497.38147917,30.33119157)(497.38147095,30.32119436)
\lineto(497.35147095,30.29119436)
\lineto(497.35147095,30.21619436)
\lineto(497.38147095,30.18619436)
\curveto(497.38147917,30.17619173)(497.38647916,30.16619174)(497.39647095,30.15619436)
\curveto(497.4464791,30.13619177)(497.50147905,30.12619178)(497.56147095,30.12619436)
\curveto(497.62147893,30.12619178)(497.68147887,30.11619179)(497.74147095,30.09619436)
\lineto(497.90647095,30.09619436)
\curveto(497.96647858,30.07619183)(498.03147852,30.07119183)(498.10147095,30.08119436)
\curveto(498.17147838,30.09119181)(498.24147831,30.09619181)(498.31147095,30.09619436)
\lineto(499.12147095,30.09619436)
\lineto(503.68147095,30.09619436)
\lineto(504.86647095,30.09619436)
\curveto(504.97647157,30.09619181)(505.08647146,30.09119181)(505.19647095,30.08119436)
\curveto(505.30647124,30.08119182)(505.39147116,30.05619185)(505.45147095,30.00619436)
\curveto(505.53147102,29.95619195)(505.57647097,29.86619204)(505.58647095,29.73619436)
\lineto(505.58647095,29.34619436)
\lineto(505.58647095,29.15119436)
\curveto(505.58647096,29.1011928)(505.57647097,29.05119285)(505.55647095,29.00119436)
\curveto(505.51647103,28.87119303)(505.43147112,28.79619311)(505.30147095,28.77619436)
\curveto(505.17147138,28.76619314)(505.02147153,28.76119314)(504.85147095,28.76119436)
\lineto(503.11147095,28.76119436)
\lineto(497.11147095,28.76119436)
\lineto(495.70147095,28.76119436)
\curveto(495.59148096,28.76119314)(495.47648107,28.75619315)(495.35647095,28.74619436)
\curveto(495.23648131,28.74619316)(495.14148141,28.77119313)(495.07147095,28.82119436)
\curveto(495.01148154,28.86119304)(494.96148159,28.93619297)(494.92147095,29.04619436)
\curveto(494.91148164,29.06619284)(494.91148164,29.08619282)(494.92147095,29.10619436)
\curveto(494.92148163,29.13619277)(494.91648163,29.16119274)(494.90647095,29.18119436)
}
}
{
\newrgbcolor{curcolor}{0 0 0}
\pscustom[linestyle=none,fillstyle=solid,fillcolor=curcolor]
{
\newpath
\moveto(505.03147095,48.38330373)
\curveto(505.19147136,48.4132959)(505.32647122,48.39829592)(505.43647095,48.33830373)
\curveto(505.53647101,48.27829604)(505.61147094,48.19829612)(505.66147095,48.09830373)
\curveto(505.68147087,48.04829627)(505.69147086,47.99329632)(505.69147095,47.93330373)
\curveto(505.69147086,47.88329643)(505.70147085,47.82829649)(505.72147095,47.76830373)
\curveto(505.77147078,47.54829677)(505.75647079,47.32829699)(505.67647095,47.10830373)
\curveto(505.60647094,46.89829742)(505.51647103,46.75329756)(505.40647095,46.67330373)
\curveto(505.33647121,46.62329769)(505.25647129,46.57829774)(505.16647095,46.53830373)
\curveto(505.06647148,46.49829782)(504.98647156,46.44829787)(504.92647095,46.38830373)
\curveto(504.90647164,46.36829795)(504.88647166,46.34329797)(504.86647095,46.31330373)
\curveto(504.8464717,46.29329802)(504.84147171,46.26329805)(504.85147095,46.22330373)
\curveto(504.88147167,46.1132982)(504.93647161,46.00829831)(505.01647095,45.90830373)
\curveto(505.09647145,45.8182985)(505.16647138,45.72829859)(505.22647095,45.63830373)
\curveto(505.30647124,45.50829881)(505.38147117,45.36829895)(505.45147095,45.21830373)
\curveto(505.51147104,45.06829925)(505.56647098,44.90829941)(505.61647095,44.73830373)
\curveto(505.6464709,44.63829968)(505.66647088,44.52829979)(505.67647095,44.40830373)
\curveto(505.68647086,44.29830002)(505.70147085,44.18830013)(505.72147095,44.07830373)
\curveto(505.73147082,44.02830029)(505.73647081,43.98330033)(505.73647095,43.94330373)
\lineto(505.73647095,43.83830373)
\curveto(505.75647079,43.72830059)(505.75647079,43.62330069)(505.73647095,43.52330373)
\lineto(505.73647095,43.38830373)
\curveto(505.72647082,43.33830098)(505.72147083,43.28830103)(505.72147095,43.23830373)
\curveto(505.72147083,43.18830113)(505.71147084,43.14330117)(505.69147095,43.10330373)
\curveto(505.68147087,43.06330125)(505.67647087,43.02830129)(505.67647095,42.99830373)
\curveto(505.68647086,42.97830134)(505.68647086,42.95330136)(505.67647095,42.92330373)
\lineto(505.61647095,42.68330373)
\curveto(505.60647094,42.60330171)(505.58647096,42.52830179)(505.55647095,42.45830373)
\curveto(505.42647112,42.15830216)(505.28147127,41.9133024)(505.12147095,41.72330373)
\curveto(504.9514716,41.54330277)(504.71647183,41.39330292)(504.41647095,41.27330373)
\curveto(504.19647235,41.18330313)(503.93147262,41.13830318)(503.62147095,41.13830373)
\lineto(503.30647095,41.13830373)
\curveto(503.25647329,41.14830317)(503.20647334,41.15330316)(503.15647095,41.15330373)
\lineto(502.97647095,41.18330373)
\lineto(502.64647095,41.30330373)
\curveto(502.53647401,41.34330297)(502.43647411,41.39330292)(502.34647095,41.45330373)
\curveto(502.05647449,41.63330268)(501.84147471,41.87830244)(501.70147095,42.18830373)
\curveto(501.56147499,42.49830182)(501.43647511,42.83830148)(501.32647095,43.20830373)
\curveto(501.28647526,43.34830097)(501.25647529,43.49330082)(501.23647095,43.64330373)
\curveto(501.21647533,43.79330052)(501.19147536,43.94330037)(501.16147095,44.09330373)
\curveto(501.14147541,44.16330015)(501.13147542,44.22830009)(501.13147095,44.28830373)
\curveto(501.13147542,44.35829996)(501.12147543,44.43329988)(501.10147095,44.51330373)
\curveto(501.08147547,44.58329973)(501.07147548,44.65329966)(501.07147095,44.72330373)
\curveto(501.06147549,44.79329952)(501.0464755,44.86829945)(501.02647095,44.94830373)
\curveto(500.96647558,45.19829912)(500.91647563,45.43329888)(500.87647095,45.65330373)
\curveto(500.82647572,45.87329844)(500.71147584,46.04829827)(500.53147095,46.17830373)
\curveto(500.4514761,46.23829808)(500.3514762,46.28829803)(500.23147095,46.32830373)
\curveto(500.10147645,46.36829795)(499.96147659,46.36829795)(499.81147095,46.32830373)
\curveto(499.57147698,46.26829805)(499.38147717,46.17829814)(499.24147095,46.05830373)
\curveto(499.10147745,45.94829837)(498.99147756,45.78829853)(498.91147095,45.57830373)
\curveto(498.86147769,45.45829886)(498.82647772,45.313299)(498.80647095,45.14330373)
\curveto(498.78647776,44.98329933)(498.77647777,44.8132995)(498.77647095,44.63330373)
\curveto(498.77647777,44.45329986)(498.78647776,44.27830004)(498.80647095,44.10830373)
\curveto(498.82647772,43.93830038)(498.85647769,43.79330052)(498.89647095,43.67330373)
\curveto(498.95647759,43.50330081)(499.04147751,43.33830098)(499.15147095,43.17830373)
\curveto(499.21147734,43.09830122)(499.29147726,43.02330129)(499.39147095,42.95330373)
\curveto(499.48147707,42.89330142)(499.58147697,42.83830148)(499.69147095,42.78830373)
\curveto(499.77147678,42.75830156)(499.85647669,42.72830159)(499.94647095,42.69830373)
\curveto(500.03647651,42.67830164)(500.10647644,42.63330168)(500.15647095,42.56330373)
\curveto(500.18647636,42.52330179)(500.21147634,42.45330186)(500.23147095,42.35330373)
\curveto(500.24147631,42.26330205)(500.2464763,42.16830215)(500.24647095,42.06830373)
\curveto(500.2464763,41.96830235)(500.24147631,41.86830245)(500.23147095,41.76830373)
\curveto(500.21147634,41.67830264)(500.18647636,41.6133027)(500.15647095,41.57330373)
\curveto(500.12647642,41.53330278)(500.07647647,41.50330281)(500.00647095,41.48330373)
\curveto(499.93647661,41.46330285)(499.86147669,41.46330285)(499.78147095,41.48330373)
\curveto(499.6514769,41.5133028)(499.53147702,41.54330277)(499.42147095,41.57330373)
\curveto(499.30147725,41.6133027)(499.18647736,41.65830266)(499.07647095,41.70830373)
\curveto(498.72647782,41.89830242)(498.45647809,42.13830218)(498.26647095,42.42830373)
\curveto(498.06647848,42.7183016)(497.90647864,43.07830124)(497.78647095,43.50830373)
\curveto(497.76647878,43.60830071)(497.7514788,43.70830061)(497.74147095,43.80830373)
\curveto(497.73147882,43.9183004)(497.71647883,44.02830029)(497.69647095,44.13830373)
\curveto(497.68647886,44.17830014)(497.68647886,44.24330007)(497.69647095,44.33330373)
\curveto(497.69647885,44.42329989)(497.68647886,44.47829984)(497.66647095,44.49830373)
\curveto(497.65647889,45.19829912)(497.73647881,45.80829851)(497.90647095,46.32830373)
\curveto(498.07647847,46.84829747)(498.40147815,47.2132971)(498.88147095,47.42330373)
\curveto(499.08147747,47.5132968)(499.31647723,47.56329675)(499.58647095,47.57330373)
\curveto(499.8464767,47.59329672)(500.12147643,47.60329671)(500.41147095,47.60330373)
\lineto(503.72647095,47.60330373)
\curveto(503.86647268,47.60329671)(504.00147255,47.60829671)(504.13147095,47.61830373)
\curveto(504.26147229,47.62829669)(504.36647218,47.65829666)(504.44647095,47.70830373)
\curveto(504.51647203,47.75829656)(504.56647198,47.82329649)(504.59647095,47.90330373)
\curveto(504.63647191,47.99329632)(504.66647188,48.07829624)(504.68647095,48.15830373)
\curveto(504.69647185,48.23829608)(504.74147181,48.29829602)(504.82147095,48.33830373)
\curveto(504.8514717,48.35829596)(504.88147167,48.36829595)(504.91147095,48.36830373)
\curveto(504.94147161,48.36829595)(504.98147157,48.37329594)(505.03147095,48.38330373)
\moveto(503.36647095,46.23830373)
\curveto(503.22647332,46.29829802)(503.06647348,46.32829799)(502.88647095,46.32830373)
\curveto(502.69647385,46.33829798)(502.50147405,46.34329797)(502.30147095,46.34330373)
\curveto(502.19147436,46.34329797)(502.09147446,46.33829798)(502.00147095,46.32830373)
\curveto(501.91147464,46.318298)(501.84147471,46.27829804)(501.79147095,46.20830373)
\curveto(501.77147478,46.17829814)(501.76147479,46.10829821)(501.76147095,45.99830373)
\curveto(501.78147477,45.97829834)(501.79147476,45.94329837)(501.79147095,45.89330373)
\curveto(501.79147476,45.84329847)(501.80147475,45.79829852)(501.82147095,45.75830373)
\curveto(501.84147471,45.67829864)(501.86147469,45.58829873)(501.88147095,45.48830373)
\lineto(501.94147095,45.18830373)
\curveto(501.94147461,45.15829916)(501.9464746,45.12329919)(501.95647095,45.08330373)
\lineto(501.95647095,44.97830373)
\curveto(501.99647455,44.82829949)(502.02147453,44.66329965)(502.03147095,44.48330373)
\curveto(502.03147452,44.3133)(502.0514745,44.15330016)(502.09147095,44.00330373)
\curveto(502.11147444,43.92330039)(502.13147442,43.84830047)(502.15147095,43.77830373)
\curveto(502.16147439,43.7183006)(502.17647437,43.64830067)(502.19647095,43.56830373)
\curveto(502.2464743,43.40830091)(502.31147424,43.25830106)(502.39147095,43.11830373)
\curveto(502.46147409,42.97830134)(502.551474,42.85830146)(502.66147095,42.75830373)
\curveto(502.77147378,42.65830166)(502.90647364,42.58330173)(503.06647095,42.53330373)
\curveto(503.21647333,42.48330183)(503.40147315,42.46330185)(503.62147095,42.47330373)
\curveto(503.72147283,42.47330184)(503.81647273,42.48830183)(503.90647095,42.51830373)
\curveto(503.98647256,42.55830176)(504.06147249,42.60330171)(504.13147095,42.65330373)
\curveto(504.24147231,42.73330158)(504.33647221,42.83830148)(504.41647095,42.96830373)
\curveto(504.48647206,43.09830122)(504.546472,43.23830108)(504.59647095,43.38830373)
\curveto(504.60647194,43.43830088)(504.61147194,43.48830083)(504.61147095,43.53830373)
\curveto(504.61147194,43.58830073)(504.61647193,43.63830068)(504.62647095,43.68830373)
\curveto(504.6464719,43.75830056)(504.66147189,43.84330047)(504.67147095,43.94330373)
\curveto(504.67147188,44.05330026)(504.66147189,44.14330017)(504.64147095,44.21330373)
\curveto(504.62147193,44.27330004)(504.61647193,44.33329998)(504.62647095,44.39330373)
\curveto(504.62647192,44.45329986)(504.61647193,44.5132998)(504.59647095,44.57330373)
\curveto(504.57647197,44.65329966)(504.56147199,44.72829959)(504.55147095,44.79830373)
\curveto(504.54147201,44.87829944)(504.52147203,44.95329936)(504.49147095,45.02330373)
\curveto(504.37147218,45.313299)(504.22647232,45.55829876)(504.05647095,45.75830373)
\curveto(503.88647266,45.96829835)(503.65647289,46.12829819)(503.36647095,46.23830373)
}
}
{
\newrgbcolor{curcolor}{0 0 0}
\pscustom[linestyle=none,fillstyle=solid,fillcolor=curcolor]
{
\newpath
\moveto(497.86147095,49.26994436)
\lineto(497.86147095,49.71994436)
\curveto(497.8514787,49.88994311)(497.87147868,50.01494298)(497.92147095,50.09494436)
\curveto(497.97147858,50.17494282)(498.03647851,50.22994277)(498.11647095,50.25994436)
\curveto(498.19647835,50.2999427)(498.28147827,50.33994266)(498.37147095,50.37994436)
\curveto(498.50147805,50.42994257)(498.63147792,50.47494252)(498.76147095,50.51494436)
\curveto(498.89147766,50.55494244)(499.02147753,50.5999424)(499.15147095,50.64994436)
\curveto(499.27147728,50.6999423)(499.39647715,50.74494225)(499.52647095,50.78494436)
\curveto(499.6464769,50.82494217)(499.76647678,50.86994213)(499.88647095,50.91994436)
\curveto(499.99647655,50.96994203)(500.11147644,51.00994199)(500.23147095,51.03994436)
\curveto(500.3514762,51.06994193)(500.47147608,51.10994189)(500.59147095,51.15994436)
\curveto(500.88147567,51.27994172)(501.18147537,51.38994161)(501.49147095,51.48994436)
\curveto(501.80147475,51.58994141)(502.10147445,51.6999413)(502.39147095,51.81994436)
\curveto(502.43147412,51.83994116)(502.47147408,51.84994115)(502.51147095,51.84994436)
\curveto(502.54147401,51.84994115)(502.57147398,51.85994114)(502.60147095,51.87994436)
\curveto(502.74147381,51.93994106)(502.88647366,51.994941)(503.03647095,52.04494436)
\lineto(503.45647095,52.19494436)
\curveto(503.52647302,52.22494077)(503.60147295,52.25494074)(503.68147095,52.28494436)
\curveto(503.7514728,52.31494068)(503.79647275,52.36494063)(503.81647095,52.43494436)
\curveto(503.8464727,52.51494048)(503.82147273,52.57494042)(503.74147095,52.61494436)
\curveto(503.6514729,52.66494033)(503.58147297,52.6999403)(503.53147095,52.71994436)
\curveto(503.36147319,52.7999402)(503.18147337,52.86494013)(502.99147095,52.91494436)
\curveto(502.80147375,52.96494003)(502.61647393,53.02493997)(502.43647095,53.09494436)
\curveto(502.20647434,53.18493981)(501.97647457,53.26493973)(501.74647095,53.33494436)
\curveto(501.50647504,53.40493959)(501.27647527,53.48993951)(501.05647095,53.58994436)
\curveto(501.00647554,53.5999394)(500.94147561,53.61493938)(500.86147095,53.63494436)
\curveto(500.77147578,53.67493932)(500.68147587,53.70993929)(500.59147095,53.73994436)
\curveto(500.49147606,53.76993923)(500.40147615,53.7999392)(500.32147095,53.82994436)
\curveto(500.27147628,53.84993915)(500.22647632,53.86493913)(500.18647095,53.87494436)
\curveto(500.1464764,53.88493911)(500.10147645,53.8999391)(500.05147095,53.91994436)
\curveto(499.93147662,53.96993903)(499.81147674,54.00993899)(499.69147095,54.03994436)
\curveto(499.56147699,54.07993892)(499.43647711,54.12493887)(499.31647095,54.17494436)
\curveto(499.26647728,54.1949388)(499.22147733,54.20993879)(499.18147095,54.21994436)
\curveto(499.14147741,54.22993877)(499.09647745,54.24493875)(499.04647095,54.26494436)
\curveto(498.95647759,54.30493869)(498.86647768,54.33993866)(498.77647095,54.36994436)
\curveto(498.67647787,54.3999386)(498.58147797,54.42993857)(498.49147095,54.45994436)
\curveto(498.41147814,54.48993851)(498.33147822,54.51493848)(498.25147095,54.53494436)
\curveto(498.16147839,54.56493843)(498.08647846,54.60493839)(498.02647095,54.65494436)
\curveto(497.93647861,54.72493827)(497.88647866,54.81993818)(497.87647095,54.93994436)
\curveto(497.86647868,55.06993793)(497.86147869,55.20993779)(497.86147095,55.35994436)
\curveto(497.86147869,55.43993756)(497.86647868,55.51493748)(497.87647095,55.58494436)
\curveto(497.87647867,55.66493733)(497.89147866,55.72993727)(497.92147095,55.77994436)
\curveto(497.98147857,55.86993713)(498.07647847,55.8949371)(498.20647095,55.85494436)
\curveto(498.33647821,55.81493718)(498.43647811,55.77993722)(498.50647095,55.74994436)
\lineto(498.56647095,55.71994436)
\curveto(498.58647796,55.71993728)(498.60647794,55.71493728)(498.62647095,55.70494436)
\curveto(498.90647764,55.5949374)(499.19147736,55.48493751)(499.48147095,55.37494436)
\lineto(500.32147095,55.04494436)
\curveto(500.40147615,55.01493798)(500.47647607,54.98993801)(500.54647095,54.96994436)
\curveto(500.60647594,54.94993805)(500.67147588,54.92493807)(500.74147095,54.89494436)
\curveto(500.94147561,54.81493818)(501.1464754,54.73493826)(501.35647095,54.65494436)
\curveto(501.55647499,54.58493841)(501.75647479,54.50993849)(501.95647095,54.42994436)
\curveto(502.6464739,54.13993886)(503.34147321,53.86993913)(504.04147095,53.61994436)
\curveto(504.74147181,53.36993963)(505.43647111,53.0999399)(506.12647095,52.80994436)
\lineto(506.27647095,52.74994436)
\curveto(506.33647021,52.73994026)(506.39647015,52.72494027)(506.45647095,52.70494436)
\curveto(506.82646972,52.54494045)(507.19146936,52.37494062)(507.55147095,52.19494436)
\curveto(507.92146863,52.01494098)(508.20646834,51.76494123)(508.40647095,51.44494436)
\curveto(508.46646808,51.33494166)(508.51146804,51.22494177)(508.54147095,51.11494436)
\curveto(508.58146797,51.00494199)(508.61646793,50.87994212)(508.64647095,50.73994436)
\curveto(508.66646788,50.68994231)(508.67146788,50.63494236)(508.66147095,50.57494436)
\curveto(508.6514679,50.52494247)(508.6514679,50.46994253)(508.66147095,50.40994436)
\curveto(508.68146787,50.32994267)(508.68146787,50.24994275)(508.66147095,50.16994436)
\curveto(508.6514679,50.12994287)(508.6464679,50.07994292)(508.64647095,50.01994436)
\lineto(508.58647095,49.77994436)
\curveto(508.56646798,49.70994329)(508.52646802,49.65494334)(508.46647095,49.61494436)
\curveto(508.40646814,49.56494343)(508.33146822,49.53494346)(508.24147095,49.52494436)
\lineto(507.97147095,49.52494436)
\lineto(507.76147095,49.52494436)
\curveto(507.70146885,49.53494346)(507.6514689,49.55494344)(507.61147095,49.58494436)
\curveto(507.50146905,49.65494334)(507.47146908,49.77494322)(507.52147095,49.94494436)
\curveto(507.54146901,50.05494294)(507.551469,50.17494282)(507.55147095,50.30494436)
\curveto(507.551469,50.43494256)(507.53146902,50.54994245)(507.49147095,50.64994436)
\curveto(507.44146911,50.7999422)(507.36646918,50.91994208)(507.26647095,51.00994436)
\curveto(507.16646938,51.10994189)(507.0514695,51.1949418)(506.92147095,51.26494436)
\curveto(506.80146975,51.33494166)(506.67146988,51.3949416)(506.53147095,51.44494436)
\lineto(506.11147095,51.62494436)
\curveto(506.02147053,51.66494133)(505.91147064,51.70494129)(505.78147095,51.74494436)
\curveto(505.6514709,51.7949412)(505.51647103,51.7999412)(505.37647095,51.75994436)
\curveto(505.21647133,51.70994129)(505.06647148,51.65494134)(504.92647095,51.59494436)
\curveto(504.78647176,51.54494145)(504.6464719,51.48994151)(504.50647095,51.42994436)
\curveto(504.29647225,51.33994166)(504.08647246,51.25494174)(503.87647095,51.17494436)
\curveto(503.66647288,51.0949419)(503.46147309,51.01494198)(503.26147095,50.93494436)
\curveto(503.12147343,50.87494212)(502.98647356,50.81994218)(502.85647095,50.76994436)
\curveto(502.72647382,50.71994228)(502.59147396,50.66994233)(502.45147095,50.61994436)
\lineto(501.13147095,50.07994436)
\curveto(500.69147586,49.90994309)(500.2514763,49.73494326)(499.81147095,49.55494436)
\curveto(499.58147697,49.45494354)(499.36147719,49.36494363)(499.15147095,49.28494436)
\curveto(498.93147762,49.20494379)(498.71147784,49.11994388)(498.49147095,49.02994436)
\curveto(498.43147812,49.00994399)(498.3514782,48.97994402)(498.25147095,48.93994436)
\curveto(498.14147841,48.8999441)(498.0514785,48.90494409)(497.98147095,48.95494436)
\curveto(497.93147862,48.98494401)(497.89647865,49.04494395)(497.87647095,49.13494436)
\curveto(497.86647868,49.15494384)(497.86647868,49.17494382)(497.87647095,49.19494436)
\curveto(497.87647867,49.22494377)(497.87147868,49.24994375)(497.86147095,49.26994436)
}
}
{
\newrgbcolor{curcolor}{0 0 0}
\pscustom[linestyle=none,fillstyle=solid,fillcolor=curcolor]
{
}
}
{
\newrgbcolor{curcolor}{0 0 0}
\pscustom[linestyle=none,fillstyle=solid,fillcolor=curcolor]
{
\newpath
\moveto(500.50147095,67.99510061)
\lineto(500.75647095,67.99510061)
\curveto(500.83647571,68.0050929)(500.91147564,68.00009291)(500.98147095,67.98010061)
\lineto(501.22147095,67.98010061)
\lineto(501.38647095,67.98010061)
\curveto(501.48647506,67.96009295)(501.59147496,67.95009296)(501.70147095,67.95010061)
\curveto(501.80147475,67.95009296)(501.90147465,67.94009297)(502.00147095,67.92010061)
\lineto(502.15147095,67.92010061)
\curveto(502.29147426,67.89009302)(502.43147412,67.87009304)(502.57147095,67.86010061)
\curveto(502.70147385,67.85009306)(502.83147372,67.82509308)(502.96147095,67.78510061)
\curveto(503.04147351,67.76509314)(503.12647342,67.74509316)(503.21647095,67.72510061)
\lineto(503.45647095,67.66510061)
\lineto(503.75647095,67.54510061)
\curveto(503.8464727,67.51509339)(503.93647261,67.48009343)(504.02647095,67.44010061)
\curveto(504.2464723,67.34009357)(504.46147209,67.2050937)(504.67147095,67.03510061)
\curveto(504.88147167,66.87509403)(505.0514715,66.70009421)(505.18147095,66.51010061)
\curveto(505.22147133,66.46009445)(505.26147129,66.40009451)(505.30147095,66.33010061)
\curveto(505.33147122,66.27009464)(505.36647118,66.2100947)(505.40647095,66.15010061)
\curveto(505.45647109,66.07009484)(505.49647105,65.97509493)(505.52647095,65.86510061)
\curveto(505.55647099,65.75509515)(505.58647096,65.65009526)(505.61647095,65.55010061)
\curveto(505.65647089,65.44009547)(505.68147087,65.33009558)(505.69147095,65.22010061)
\curveto(505.70147085,65.1100958)(505.71647083,64.99509591)(505.73647095,64.87510061)
\curveto(505.7464708,64.83509607)(505.7464708,64.79009612)(505.73647095,64.74010061)
\curveto(505.73647081,64.70009621)(505.74147081,64.66009625)(505.75147095,64.62010061)
\curveto(505.76147079,64.58009633)(505.76647078,64.52509638)(505.76647095,64.45510061)
\curveto(505.76647078,64.38509652)(505.76147079,64.33509657)(505.75147095,64.30510061)
\curveto(505.73147082,64.25509665)(505.72647082,64.2100967)(505.73647095,64.17010061)
\curveto(505.7464708,64.13009678)(505.7464708,64.09509681)(505.73647095,64.06510061)
\lineto(505.73647095,63.97510061)
\curveto(505.71647083,63.91509699)(505.70147085,63.85009706)(505.69147095,63.78010061)
\curveto(505.69147086,63.72009719)(505.68647086,63.65509725)(505.67647095,63.58510061)
\curveto(505.62647092,63.41509749)(505.57647097,63.25509765)(505.52647095,63.10510061)
\curveto(505.47647107,62.95509795)(505.41147114,62.8100981)(505.33147095,62.67010061)
\curveto(505.29147126,62.62009829)(505.26147129,62.56509834)(505.24147095,62.50510061)
\curveto(505.21147134,62.45509845)(505.17647137,62.4050985)(505.13647095,62.35510061)
\curveto(504.95647159,62.11509879)(504.73647181,61.91509899)(504.47647095,61.75510061)
\curveto(504.21647233,61.59509931)(503.93147262,61.45509945)(503.62147095,61.33510061)
\curveto(503.48147307,61.27509963)(503.34147321,61.23009968)(503.20147095,61.20010061)
\curveto(503.0514735,61.17009974)(502.89647365,61.13509977)(502.73647095,61.09510061)
\curveto(502.62647392,61.07509983)(502.51647403,61.06009985)(502.40647095,61.05010061)
\curveto(502.29647425,61.04009987)(502.18647436,61.02509988)(502.07647095,61.00510061)
\curveto(502.03647451,60.99509991)(501.99647455,60.99009992)(501.95647095,60.99010061)
\curveto(501.91647463,61.00009991)(501.87647467,61.00009991)(501.83647095,60.99010061)
\curveto(501.78647476,60.98009993)(501.73647481,60.97509993)(501.68647095,60.97510061)
\lineto(501.52147095,60.97510061)
\curveto(501.47147508,60.95509995)(501.42147513,60.95009996)(501.37147095,60.96010061)
\curveto(501.31147524,60.97009994)(501.25647529,60.97009994)(501.20647095,60.96010061)
\curveto(501.16647538,60.95009996)(501.12147543,60.95009996)(501.07147095,60.96010061)
\curveto(501.02147553,60.97009994)(500.97147558,60.96509994)(500.92147095,60.94510061)
\curveto(500.8514757,60.92509998)(500.77647577,60.92009999)(500.69647095,60.93010061)
\curveto(500.60647594,60.94009997)(500.52147603,60.94509996)(500.44147095,60.94510061)
\curveto(500.3514762,60.94509996)(500.2514763,60.94009997)(500.14147095,60.93010061)
\curveto(500.02147653,60.92009999)(499.92147663,60.92509998)(499.84147095,60.94510061)
\lineto(499.55647095,60.94510061)
\lineto(498.92647095,60.99010061)
\curveto(498.82647772,61.00009991)(498.73147782,61.0100999)(498.64147095,61.02010061)
\lineto(498.34147095,61.05010061)
\curveto(498.29147826,61.07009984)(498.24147831,61.07509983)(498.19147095,61.06510061)
\curveto(498.13147842,61.06509984)(498.07647847,61.07509983)(498.02647095,61.09510061)
\curveto(497.85647869,61.14509976)(497.69147886,61.18509972)(497.53147095,61.21510061)
\curveto(497.36147919,61.24509966)(497.20147935,61.29509961)(497.05147095,61.36510061)
\curveto(496.59147996,61.55509935)(496.21648033,61.77509913)(495.92647095,62.02510061)
\curveto(495.63648091,62.28509862)(495.39148116,62.64509826)(495.19147095,63.10510061)
\curveto(495.14148141,63.23509767)(495.10648144,63.36509754)(495.08647095,63.49510061)
\curveto(495.06648148,63.63509727)(495.04148151,63.77509713)(495.01147095,63.91510061)
\curveto(495.00148155,63.98509692)(494.99648155,64.05009686)(494.99647095,64.11010061)
\curveto(494.99648155,64.17009674)(494.99148156,64.23509667)(494.98147095,64.30510061)
\curveto(494.96148159,65.13509577)(495.11148144,65.8050951)(495.43147095,66.31510061)
\curveto(495.74148081,66.82509408)(496.18148037,67.2050937)(496.75147095,67.45510061)
\curveto(496.87147968,67.5050934)(496.99647955,67.55009336)(497.12647095,67.59010061)
\curveto(497.25647929,67.63009328)(497.39147916,67.67509323)(497.53147095,67.72510061)
\curveto(497.61147894,67.74509316)(497.69647885,67.76009315)(497.78647095,67.77010061)
\lineto(498.02647095,67.83010061)
\curveto(498.13647841,67.86009305)(498.2464783,67.87509303)(498.35647095,67.87510061)
\curveto(498.46647808,67.88509302)(498.57647797,67.90009301)(498.68647095,67.92010061)
\curveto(498.73647781,67.94009297)(498.78147777,67.94509296)(498.82147095,67.93510061)
\curveto(498.86147769,67.93509297)(498.90147765,67.94009297)(498.94147095,67.95010061)
\curveto(498.99147756,67.96009295)(499.0464775,67.96009295)(499.10647095,67.95010061)
\curveto(499.15647739,67.95009296)(499.20647734,67.95509295)(499.25647095,67.96510061)
\lineto(499.39147095,67.96510061)
\curveto(499.4514771,67.98509292)(499.52147703,67.98509292)(499.60147095,67.96510061)
\curveto(499.67147688,67.95509295)(499.73647681,67.96009295)(499.79647095,67.98010061)
\curveto(499.82647672,67.99009292)(499.86647668,67.99509291)(499.91647095,67.99510061)
\lineto(500.03647095,67.99510061)
\lineto(500.50147095,67.99510061)
\moveto(502.82647095,66.45010061)
\curveto(502.50647404,66.55009436)(502.14147441,66.6100943)(501.73147095,66.63010061)
\curveto(501.32147523,66.65009426)(500.91147564,66.66009425)(500.50147095,66.66010061)
\curveto(500.07147648,66.66009425)(499.6514769,66.65009426)(499.24147095,66.63010061)
\curveto(498.83147772,66.6100943)(498.4464781,66.56509434)(498.08647095,66.49510061)
\curveto(497.72647882,66.42509448)(497.40647914,66.31509459)(497.12647095,66.16510061)
\curveto(496.83647971,66.02509488)(496.60147995,65.83009508)(496.42147095,65.58010061)
\curveto(496.31148024,65.42009549)(496.23148032,65.24009567)(496.18147095,65.04010061)
\curveto(496.12148043,64.84009607)(496.09148046,64.59509631)(496.09147095,64.30510061)
\curveto(496.11148044,64.28509662)(496.12148043,64.25009666)(496.12147095,64.20010061)
\curveto(496.11148044,64.15009676)(496.11148044,64.1100968)(496.12147095,64.08010061)
\curveto(496.14148041,64.00009691)(496.16148039,63.92509698)(496.18147095,63.85510061)
\curveto(496.19148036,63.79509711)(496.21148034,63.73009718)(496.24147095,63.66010061)
\curveto(496.36148019,63.39009752)(496.53148002,63.17009774)(496.75147095,63.00010061)
\curveto(496.96147959,62.84009807)(497.20647934,62.7050982)(497.48647095,62.59510061)
\curveto(497.59647895,62.54509836)(497.71647883,62.5050984)(497.84647095,62.47510061)
\curveto(497.96647858,62.45509845)(498.09147846,62.43009848)(498.22147095,62.40010061)
\curveto(498.27147828,62.38009853)(498.32647822,62.37009854)(498.38647095,62.37010061)
\curveto(498.43647811,62.37009854)(498.48647806,62.36509854)(498.53647095,62.35510061)
\curveto(498.62647792,62.34509856)(498.72147783,62.33509857)(498.82147095,62.32510061)
\curveto(498.91147764,62.31509859)(499.00647754,62.3050986)(499.10647095,62.29510061)
\curveto(499.18647736,62.29509861)(499.27147728,62.29009862)(499.36147095,62.28010061)
\lineto(499.60147095,62.28010061)
\lineto(499.78147095,62.28010061)
\curveto(499.81147674,62.27009864)(499.8464767,62.26509864)(499.88647095,62.26510061)
\lineto(500.02147095,62.26510061)
\lineto(500.47147095,62.26510061)
\curveto(500.551476,62.26509864)(500.63647591,62.26009865)(500.72647095,62.25010061)
\curveto(500.80647574,62.25009866)(500.88147567,62.26009865)(500.95147095,62.28010061)
\lineto(501.22147095,62.28010061)
\curveto(501.24147531,62.28009863)(501.27147528,62.27509863)(501.31147095,62.26510061)
\curveto(501.34147521,62.26509864)(501.36647518,62.27009864)(501.38647095,62.28010061)
\curveto(501.48647506,62.29009862)(501.58647496,62.29509861)(501.68647095,62.29510061)
\curveto(501.77647477,62.3050986)(501.87647467,62.31509859)(501.98647095,62.32510061)
\curveto(502.10647444,62.35509855)(502.23147432,62.37009854)(502.36147095,62.37010061)
\curveto(502.48147407,62.38009853)(502.59647395,62.4050985)(502.70647095,62.44510061)
\curveto(503.00647354,62.52509838)(503.27147328,62.6100983)(503.50147095,62.70010061)
\curveto(503.73147282,62.80009811)(503.9464726,62.94509796)(504.14647095,63.13510061)
\curveto(504.3464722,63.34509756)(504.49647205,63.6100973)(504.59647095,63.93010061)
\curveto(504.61647193,63.97009694)(504.62647192,64.0050969)(504.62647095,64.03510061)
\curveto(504.61647193,64.07509683)(504.62147193,64.12009679)(504.64147095,64.17010061)
\curveto(504.6514719,64.2100967)(504.66147189,64.28009663)(504.67147095,64.38010061)
\curveto(504.68147187,64.49009642)(504.67647187,64.57509633)(504.65647095,64.63510061)
\curveto(504.63647191,64.7050962)(504.62647192,64.77509613)(504.62647095,64.84510061)
\curveto(504.61647193,64.91509599)(504.60147195,64.98009593)(504.58147095,65.04010061)
\curveto(504.52147203,65.24009567)(504.43647211,65.42009549)(504.32647095,65.58010061)
\curveto(504.30647224,65.6100953)(504.28647226,65.63509527)(504.26647095,65.65510061)
\lineto(504.20647095,65.71510061)
\curveto(504.18647236,65.75509515)(504.1464724,65.8050951)(504.08647095,65.86510061)
\curveto(503.9464726,65.96509494)(503.81647273,66.05009486)(503.69647095,66.12010061)
\curveto(503.57647297,66.19009472)(503.43147312,66.26009465)(503.26147095,66.33010061)
\curveto(503.19147336,66.36009455)(503.12147343,66.38009453)(503.05147095,66.39010061)
\curveto(502.98147357,66.4100945)(502.90647364,66.43009448)(502.82647095,66.45010061)
}
}
{
\newrgbcolor{curcolor}{0 0 0}
\pscustom[linestyle=none,fillstyle=solid,fillcolor=curcolor]
{
\newpath
\moveto(495.17647095,70.85470998)
\lineto(495.17647095,74.45470998)
\lineto(495.17647095,75.09970998)
\curveto(495.17648137,75.17970345)(495.18148137,75.25470338)(495.19147095,75.32470998)
\curveto(495.19148136,75.39470324)(495.20148135,75.45470318)(495.22147095,75.50470998)
\curveto(495.2514813,75.57470306)(495.31148124,75.629703)(495.40147095,75.66970998)
\curveto(495.43148112,75.68970294)(495.47148108,75.69970293)(495.52147095,75.69970998)
\lineto(495.65647095,75.69970998)
\curveto(495.76648078,75.70970292)(495.87148068,75.70470293)(495.97147095,75.68470998)
\curveto(496.07148048,75.67470296)(496.14148041,75.63970299)(496.18147095,75.57970998)
\curveto(496.2514803,75.48970314)(496.28648026,75.35470328)(496.28647095,75.17470998)
\curveto(496.27648027,74.99470364)(496.27148028,74.8297038)(496.27147095,74.67970998)
\lineto(496.27147095,72.68470998)
\lineto(496.27147095,72.18970998)
\lineto(496.27147095,72.05470998)
\curveto(496.27148028,72.01470662)(496.27648027,71.97470666)(496.28647095,71.93470998)
\lineto(496.28647095,71.72470998)
\curveto(496.31648023,71.61470702)(496.35648019,71.5347071)(496.40647095,71.48470998)
\curveto(496.4464801,71.4347072)(496.50148005,71.39970723)(496.57147095,71.37970998)
\curveto(496.63147992,71.35970727)(496.70147985,71.34470729)(496.78147095,71.33470998)
\curveto(496.86147969,71.32470731)(496.9514796,71.30470733)(497.05147095,71.27470998)
\curveto(497.2514793,71.22470741)(497.45647909,71.18470745)(497.66647095,71.15470998)
\curveto(497.87647867,71.12470751)(498.08147847,71.08470755)(498.28147095,71.03470998)
\curveto(498.3514782,71.01470762)(498.42147813,71.00470763)(498.49147095,71.00470998)
\curveto(498.551478,71.00470763)(498.61647793,70.99470764)(498.68647095,70.97470998)
\curveto(498.71647783,70.96470767)(498.75647779,70.95470768)(498.80647095,70.94470998)
\curveto(498.8464777,70.94470769)(498.88647766,70.94970768)(498.92647095,70.95970998)
\curveto(498.97647757,70.97970765)(499.02147753,71.00470763)(499.06147095,71.03470998)
\curveto(499.09147746,71.07470756)(499.09647745,71.1347075)(499.07647095,71.21470998)
\curveto(499.05647749,71.27470736)(499.03147752,71.3347073)(499.00147095,71.39470998)
\curveto(498.96147759,71.45470718)(498.92647762,71.51470712)(498.89647095,71.57470998)
\curveto(498.87647767,71.634707)(498.86147769,71.68470695)(498.85147095,71.72470998)
\curveto(498.77147778,71.91470672)(498.71647783,72.11970651)(498.68647095,72.33970998)
\curveto(498.65647789,72.56970606)(498.6464779,72.79970583)(498.65647095,73.02970998)
\curveto(498.65647789,73.26970536)(498.68147787,73.49970513)(498.73147095,73.71970998)
\curveto(498.77147778,73.93970469)(498.83147772,74.13970449)(498.91147095,74.31970998)
\curveto(498.93147762,74.36970426)(498.9514776,74.41470422)(498.97147095,74.45470998)
\curveto(498.99147756,74.50470413)(499.01647753,74.55470408)(499.04647095,74.60470998)
\curveto(499.25647729,74.95470368)(499.48647706,75.2347034)(499.73647095,75.44470998)
\curveto(499.98647656,75.66470297)(500.31147624,75.85970277)(500.71147095,76.02970998)
\curveto(500.82147573,76.07970255)(500.93147562,76.11470252)(501.04147095,76.13470998)
\curveto(501.1514754,76.15470248)(501.26647528,76.17970245)(501.38647095,76.20970998)
\curveto(501.41647513,76.21970241)(501.46147509,76.22470241)(501.52147095,76.22470998)
\curveto(501.58147497,76.24470239)(501.6514749,76.25470238)(501.73147095,76.25470998)
\curveto(501.80147475,76.25470238)(501.86647468,76.26470237)(501.92647095,76.28470998)
\lineto(502.09147095,76.28470998)
\curveto(502.14147441,76.29470234)(502.21147434,76.29970233)(502.30147095,76.29970998)
\curveto(502.39147416,76.29970233)(502.46147409,76.28970234)(502.51147095,76.26970998)
\curveto(502.57147398,76.24970238)(502.63147392,76.24470239)(502.69147095,76.25470998)
\curveto(502.74147381,76.26470237)(502.79147376,76.25970237)(502.84147095,76.23970998)
\curveto(503.00147355,76.19970243)(503.1514734,76.16470247)(503.29147095,76.13470998)
\curveto(503.43147312,76.10470253)(503.56647298,76.05970257)(503.69647095,75.99970998)
\curveto(504.06647248,75.83970279)(504.40147215,75.61970301)(504.70147095,75.33970998)
\curveto(505.00147155,75.05970357)(505.23147132,74.73970389)(505.39147095,74.37970998)
\curveto(505.47147108,74.20970442)(505.546471,74.00970462)(505.61647095,73.77970998)
\curveto(505.65647089,73.66970496)(505.68147087,73.55470508)(505.69147095,73.43470998)
\curveto(505.70147085,73.31470532)(505.72147083,73.19470544)(505.75147095,73.07470998)
\curveto(505.77147078,73.02470561)(505.77147078,72.96970566)(505.75147095,72.90970998)
\curveto(505.74147081,72.84970578)(505.7464708,72.78970584)(505.76647095,72.72970998)
\curveto(505.78647076,72.629706)(505.78647076,72.5297061)(505.76647095,72.42970998)
\lineto(505.76647095,72.29470998)
\curveto(505.7464708,72.24470639)(505.73647081,72.18470645)(505.73647095,72.11470998)
\curveto(505.7464708,72.05470658)(505.74147081,71.99970663)(505.72147095,71.94970998)
\curveto(505.71147084,71.90970672)(505.70647084,71.87470676)(505.70647095,71.84470998)
\curveto(505.70647084,71.81470682)(505.70147085,71.77970685)(505.69147095,71.73970998)
\lineto(505.63147095,71.46970998)
\curveto(505.61147094,71.37970725)(505.58147097,71.29470734)(505.54147095,71.21470998)
\curveto(505.40147115,70.87470776)(505.2464713,70.58470805)(505.07647095,70.34470998)
\curveto(504.89647165,70.10470853)(504.66647188,69.88470875)(504.38647095,69.68470998)
\curveto(504.15647239,69.5347091)(503.91647263,69.41970921)(503.66647095,69.33970998)
\curveto(503.61647293,69.31970931)(503.57147298,69.30970932)(503.53147095,69.30970998)
\curveto(503.48147307,69.30970932)(503.43147312,69.29970933)(503.38147095,69.27970998)
\curveto(503.32147323,69.25970937)(503.24147331,69.24470939)(503.14147095,69.23470998)
\curveto(503.04147351,69.2347094)(502.96647358,69.25470938)(502.91647095,69.29470998)
\curveto(502.83647371,69.34470929)(502.79147376,69.42470921)(502.78147095,69.53470998)
\curveto(502.77147378,69.64470899)(502.76647378,69.75970887)(502.76647095,69.87970998)
\lineto(502.76647095,70.04470998)
\curveto(502.76647378,70.10470853)(502.77647377,70.15970847)(502.79647095,70.20970998)
\curveto(502.81647373,70.29970833)(502.85647369,70.36970826)(502.91647095,70.41970998)
\curveto(503.00647354,70.48970814)(503.11647343,70.5347081)(503.24647095,70.55470998)
\curveto(503.36647318,70.58470805)(503.47147308,70.629708)(503.56147095,70.68970998)
\curveto(503.90147265,70.87970775)(504.17147238,71.13970749)(504.37147095,71.46970998)
\curveto(504.43147212,71.56970706)(504.48147207,71.67470696)(504.52147095,71.78470998)
\curveto(504.551472,71.90470673)(504.58647196,72.02470661)(504.62647095,72.14470998)
\curveto(504.67647187,72.31470632)(504.69647185,72.51970611)(504.68647095,72.75970998)
\curveto(504.66647188,73.00970562)(504.63147192,73.20970542)(504.58147095,73.35970998)
\curveto(504.46147209,73.7297049)(504.30147225,74.01970461)(504.10147095,74.22970998)
\curveto(503.89147266,74.44970418)(503.61147294,74.629704)(503.26147095,74.76970998)
\curveto(503.16147339,74.81970381)(503.05647349,74.84970378)(502.94647095,74.85970998)
\curveto(502.83647371,74.87970375)(502.72147383,74.90470373)(502.60147095,74.93470998)
\lineto(502.49647095,74.93470998)
\curveto(502.45647409,74.94470369)(502.41647413,74.94970368)(502.37647095,74.94970998)
\curveto(502.3464742,74.95970367)(502.31147424,74.95970367)(502.27147095,74.94970998)
\lineto(502.15147095,74.94970998)
\curveto(501.89147466,74.94970368)(501.6464749,74.91970371)(501.41647095,74.85970998)
\curveto(501.06647548,74.74970388)(500.77147578,74.59470404)(500.53147095,74.39470998)
\curveto(500.28147627,74.19470444)(500.08647646,73.9347047)(499.94647095,73.61470998)
\lineto(499.88647095,73.43470998)
\curveto(499.86647668,73.38470525)(499.8464767,73.32470531)(499.82647095,73.25470998)
\curveto(499.80647674,73.20470543)(499.79647675,73.14470549)(499.79647095,73.07470998)
\curveto(499.78647676,73.01470562)(499.77147678,72.94970568)(499.75147095,72.87970998)
\lineto(499.75147095,72.72970998)
\curveto(499.73147682,72.68970594)(499.72147683,72.634706)(499.72147095,72.56470998)
\curveto(499.72147683,72.50470613)(499.73147682,72.44970618)(499.75147095,72.39970998)
\lineto(499.75147095,72.29470998)
\curveto(499.7514768,72.26470637)(499.75647679,72.2297064)(499.76647095,72.18970998)
\lineto(499.82647095,71.94970998)
\curveto(499.83647671,71.86970676)(499.85647669,71.78970684)(499.88647095,71.70970998)
\curveto(499.98647656,71.46970716)(500.12147643,71.23970739)(500.29147095,71.01970998)
\curveto(500.36147619,70.9297077)(500.43647611,70.84470779)(500.51647095,70.76470998)
\curveto(500.58647596,70.68470795)(500.64147591,70.58470805)(500.68147095,70.46470998)
\curveto(500.71147584,70.37470826)(500.72147583,70.2347084)(500.71147095,70.04470998)
\curveto(500.70147585,69.86470877)(500.67647587,69.74470889)(500.63647095,69.68470998)
\curveto(500.59647595,69.634709)(500.53647601,69.59470904)(500.45647095,69.56470998)
\curveto(500.37647617,69.54470909)(500.29147626,69.54470909)(500.20147095,69.56470998)
\curveto(500.08147647,69.59470904)(499.96147659,69.61470902)(499.84147095,69.62470998)
\curveto(499.71147684,69.64470899)(499.58647696,69.66970896)(499.46647095,69.69970998)
\curveto(499.42647712,69.71970891)(499.39147716,69.72470891)(499.36147095,69.71470998)
\curveto(499.32147723,69.71470892)(499.27647727,69.72470891)(499.22647095,69.74470998)
\curveto(499.13647741,69.76470887)(499.0464775,69.77970885)(498.95647095,69.78970998)
\curveto(498.85647769,69.79970883)(498.76147779,69.81970881)(498.67147095,69.84970998)
\curveto(498.61147794,69.85970877)(498.551478,69.86470877)(498.49147095,69.86470998)
\curveto(498.43147812,69.87470876)(498.37147818,69.88970874)(498.31147095,69.90970998)
\curveto(498.11147844,69.95970867)(497.90647864,69.99470864)(497.69647095,70.01470998)
\curveto(497.47647907,70.04470859)(497.26647928,70.08470855)(497.06647095,70.13470998)
\curveto(496.96647958,70.16470847)(496.86647968,70.18470845)(496.76647095,70.19470998)
\curveto(496.66647988,70.20470843)(496.56647998,70.21970841)(496.46647095,70.23970998)
\curveto(496.43648011,70.24970838)(496.39648015,70.25470838)(496.34647095,70.25470998)
\curveto(496.23648031,70.28470835)(496.13148042,70.30470833)(496.03147095,70.31470998)
\curveto(495.92148063,70.3347083)(495.81148074,70.35970827)(495.70147095,70.38970998)
\curveto(495.62148093,70.40970822)(495.551481,70.42470821)(495.49147095,70.43470998)
\curveto(495.42148113,70.44470819)(495.36148119,70.46970816)(495.31147095,70.50970998)
\curveto(495.28148127,70.5297081)(495.26148129,70.55970807)(495.25147095,70.59970998)
\curveto(495.23148132,70.63970799)(495.21148134,70.68470795)(495.19147095,70.73470998)
\curveto(495.19148136,70.79470784)(495.18648136,70.8347078)(495.17647095,70.85470998)
}
}
{
\newrgbcolor{curcolor}{0 0 0}
\pscustom[linestyle=none,fillstyle=solid,fillcolor=curcolor]
{
\newpath
\moveto(503.95147095,78.63431936)
\lineto(503.95147095,79.26431936)
\lineto(503.95147095,79.45931936)
\curveto(503.9514726,79.52931683)(503.96147259,79.58931677)(503.98147095,79.63931936)
\curveto(504.02147253,79.70931665)(504.06147249,79.7593166)(504.10147095,79.78931936)
\curveto(504.1514724,79.82931653)(504.21647233,79.84931651)(504.29647095,79.84931936)
\curveto(504.37647217,79.8593165)(504.46147209,79.86431649)(504.55147095,79.86431936)
\lineto(505.27147095,79.86431936)
\curveto(505.7514708,79.86431649)(506.16147039,79.80431655)(506.50147095,79.68431936)
\curveto(506.84146971,79.56431679)(507.11646943,79.36931699)(507.32647095,79.09931936)
\curveto(507.37646917,79.02931733)(507.42146913,78.9593174)(507.46147095,78.88931936)
\curveto(507.51146904,78.82931753)(507.55646899,78.7543176)(507.59647095,78.66431936)
\curveto(507.60646894,78.64431771)(507.61646893,78.61431774)(507.62647095,78.57431936)
\curveto(507.6464689,78.53431782)(507.6514689,78.48931787)(507.64147095,78.43931936)
\curveto(507.61146894,78.34931801)(507.53646901,78.29431806)(507.41647095,78.27431936)
\curveto(507.30646924,78.2543181)(507.21146934,78.26931809)(507.13147095,78.31931936)
\curveto(507.06146949,78.34931801)(506.99646955,78.39431796)(506.93647095,78.45431936)
\curveto(506.88646966,78.52431783)(506.83646971,78.58931777)(506.78647095,78.64931936)
\curveto(506.73646981,78.71931764)(506.66146989,78.77931758)(506.56147095,78.82931936)
\curveto(506.47147008,78.88931747)(506.38147017,78.93931742)(506.29147095,78.97931936)
\curveto(506.26147029,78.99931736)(506.20147035,79.02431733)(506.11147095,79.05431936)
\curveto(506.03147052,79.08431727)(505.96147059,79.08931727)(505.90147095,79.06931936)
\curveto(505.76147079,79.03931732)(505.67147088,78.97931738)(505.63147095,78.88931936)
\curveto(505.60147095,78.80931755)(505.58647096,78.71931764)(505.58647095,78.61931936)
\curveto(505.58647096,78.51931784)(505.56147099,78.43431792)(505.51147095,78.36431936)
\curveto(505.44147111,78.27431808)(505.30147125,78.22931813)(505.09147095,78.22931936)
\lineto(504.53647095,78.22931936)
\lineto(504.31147095,78.22931936)
\curveto(504.23147232,78.23931812)(504.16647238,78.2593181)(504.11647095,78.28931936)
\curveto(504.03647251,78.34931801)(503.99147256,78.41931794)(503.98147095,78.49931936)
\curveto(503.97147258,78.51931784)(503.96647258,78.53931782)(503.96647095,78.55931936)
\curveto(503.96647258,78.58931777)(503.96147259,78.61431774)(503.95147095,78.63431936)
}
}
{
\newrgbcolor{curcolor}{0 0 0}
\pscustom[linestyle=none,fillstyle=solid,fillcolor=curcolor]
{
}
}
{
\newrgbcolor{curcolor}{0 0 0}
\pscustom[linestyle=none,fillstyle=solid,fillcolor=curcolor]
{
\newpath
\moveto(494.98147095,89.26463186)
\curveto(494.97148158,89.95462722)(495.09148146,90.55462662)(495.34147095,91.06463186)
\curveto(495.59148096,91.58462559)(495.92648062,91.9796252)(496.34647095,92.24963186)
\curveto(496.42648012,92.29962488)(496.51648003,92.34462483)(496.61647095,92.38463186)
\curveto(496.70647984,92.42462475)(496.80147975,92.46962471)(496.90147095,92.51963186)
\curveto(497.00147955,92.55962462)(497.10147945,92.58962459)(497.20147095,92.60963186)
\curveto(497.30147925,92.62962455)(497.40647914,92.64962453)(497.51647095,92.66963186)
\curveto(497.56647898,92.68962449)(497.61147894,92.69462448)(497.65147095,92.68463186)
\curveto(497.69147886,92.6746245)(497.73647881,92.6796245)(497.78647095,92.69963186)
\curveto(497.83647871,92.70962447)(497.92147863,92.71462446)(498.04147095,92.71463186)
\curveto(498.1514784,92.71462446)(498.23647831,92.70962447)(498.29647095,92.69963186)
\curveto(498.35647819,92.6796245)(498.41647813,92.66962451)(498.47647095,92.66963186)
\curveto(498.53647801,92.6796245)(498.59647795,92.6746245)(498.65647095,92.65463186)
\curveto(498.79647775,92.61462456)(498.93147762,92.5796246)(499.06147095,92.54963186)
\curveto(499.19147736,92.51962466)(499.31647723,92.4796247)(499.43647095,92.42963186)
\curveto(499.57647697,92.36962481)(499.70147685,92.29962488)(499.81147095,92.21963186)
\curveto(499.92147663,92.14962503)(500.03147652,92.0746251)(500.14147095,91.99463186)
\lineto(500.20147095,91.93463186)
\curveto(500.22147633,91.92462525)(500.24147631,91.90962527)(500.26147095,91.88963186)
\curveto(500.42147613,91.76962541)(500.56647598,91.63462554)(500.69647095,91.48463186)
\curveto(500.82647572,91.33462584)(500.9514756,91.174626)(501.07147095,91.00463186)
\curveto(501.29147526,90.69462648)(501.49647505,90.39962678)(501.68647095,90.11963186)
\curveto(501.82647472,89.88962729)(501.96147459,89.65962752)(502.09147095,89.42963186)
\curveto(502.22147433,89.20962797)(502.35647419,88.98962819)(502.49647095,88.76963186)
\curveto(502.66647388,88.51962866)(502.8464737,88.2796289)(503.03647095,88.04963186)
\curveto(503.22647332,87.82962935)(503.4514731,87.63962954)(503.71147095,87.47963186)
\curveto(503.77147278,87.43962974)(503.83147272,87.40462977)(503.89147095,87.37463186)
\curveto(503.94147261,87.34462983)(504.00647254,87.31462986)(504.08647095,87.28463186)
\curveto(504.15647239,87.26462991)(504.21647233,87.25962992)(504.26647095,87.26963186)
\curveto(504.33647221,87.28962989)(504.39147216,87.32462985)(504.43147095,87.37463186)
\curveto(504.46147209,87.42462975)(504.48147207,87.48462969)(504.49147095,87.55463186)
\lineto(504.49147095,87.79463186)
\lineto(504.49147095,88.54463186)
\lineto(504.49147095,91.34963186)
\lineto(504.49147095,92.00963186)
\curveto(504.49147206,92.09962508)(504.49647205,92.18462499)(504.50647095,92.26463186)
\curveto(504.50647204,92.34462483)(504.52647202,92.40962477)(504.56647095,92.45963186)
\curveto(504.60647194,92.50962467)(504.68147187,92.54962463)(504.79147095,92.57963186)
\curveto(504.89147166,92.61962456)(504.99147156,92.62962455)(505.09147095,92.60963186)
\lineto(505.22647095,92.60963186)
\curveto(505.29647125,92.58962459)(505.35647119,92.56962461)(505.40647095,92.54963186)
\curveto(505.45647109,92.52962465)(505.49647105,92.49462468)(505.52647095,92.44463186)
\curveto(505.56647098,92.39462478)(505.58647096,92.32462485)(505.58647095,92.23463186)
\lineto(505.58647095,91.96463186)
\lineto(505.58647095,91.06463186)
\lineto(505.58647095,87.55463186)
\lineto(505.58647095,86.48963186)
\curveto(505.58647096,86.40963077)(505.59147096,86.31963086)(505.60147095,86.21963186)
\curveto(505.60147095,86.11963106)(505.59147096,86.03463114)(505.57147095,85.96463186)
\curveto(505.50147105,85.75463142)(505.32147123,85.68963149)(505.03147095,85.76963186)
\curveto(504.99147156,85.7796314)(504.95647159,85.7796314)(504.92647095,85.76963186)
\curveto(504.88647166,85.76963141)(504.84147171,85.7796314)(504.79147095,85.79963186)
\curveto(504.71147184,85.81963136)(504.62647192,85.83963134)(504.53647095,85.85963186)
\curveto(504.4464721,85.8796313)(504.36147219,85.90463127)(504.28147095,85.93463186)
\curveto(503.79147276,86.09463108)(503.37647317,86.29463088)(503.03647095,86.53463186)
\curveto(502.78647376,86.71463046)(502.56147399,86.91963026)(502.36147095,87.14963186)
\curveto(502.1514744,87.3796298)(501.95647459,87.61962956)(501.77647095,87.86963186)
\curveto(501.59647495,88.12962905)(501.42647512,88.39462878)(501.26647095,88.66463186)
\curveto(501.09647545,88.94462823)(500.92147563,89.21462796)(500.74147095,89.47463186)
\curveto(500.66147589,89.58462759)(500.58647596,89.68962749)(500.51647095,89.78963186)
\curveto(500.4464761,89.89962728)(500.37147618,90.00962717)(500.29147095,90.11963186)
\curveto(500.26147629,90.15962702)(500.23147632,90.19462698)(500.20147095,90.22463186)
\curveto(500.16147639,90.26462691)(500.13147642,90.30462687)(500.11147095,90.34463186)
\curveto(500.00147655,90.48462669)(499.87647667,90.60962657)(499.73647095,90.71963186)
\curveto(499.70647684,90.73962644)(499.68147687,90.76462641)(499.66147095,90.79463186)
\curveto(499.63147692,90.82462635)(499.60147695,90.84962633)(499.57147095,90.86963186)
\curveto(499.47147708,90.94962623)(499.37147718,91.01462616)(499.27147095,91.06463186)
\curveto(499.17147738,91.12462605)(499.06147749,91.179626)(498.94147095,91.22963186)
\curveto(498.87147768,91.25962592)(498.79647775,91.2796259)(498.71647095,91.28963186)
\lineto(498.47647095,91.34963186)
\lineto(498.38647095,91.34963186)
\curveto(498.35647819,91.35962582)(498.32647822,91.36462581)(498.29647095,91.36463186)
\curveto(498.22647832,91.38462579)(498.13147842,91.38962579)(498.01147095,91.37963186)
\curveto(497.88147867,91.3796258)(497.78147877,91.36962581)(497.71147095,91.34963186)
\curveto(497.63147892,91.32962585)(497.55647899,91.30962587)(497.48647095,91.28963186)
\curveto(497.40647914,91.2796259)(497.32647922,91.25962592)(497.24647095,91.22963186)
\curveto(497.00647954,91.11962606)(496.80647974,90.96962621)(496.64647095,90.77963186)
\curveto(496.47648007,90.59962658)(496.33648021,90.3796268)(496.22647095,90.11963186)
\curveto(496.20648034,90.04962713)(496.19148036,89.9796272)(496.18147095,89.90963186)
\curveto(496.16148039,89.83962734)(496.14148041,89.76462741)(496.12147095,89.68463186)
\curveto(496.10148045,89.60462757)(496.09148046,89.49462768)(496.09147095,89.35463186)
\curveto(496.09148046,89.22462795)(496.10148045,89.11962806)(496.12147095,89.03963186)
\curveto(496.13148042,88.9796282)(496.13648041,88.92462825)(496.13647095,88.87463186)
\curveto(496.13648041,88.82462835)(496.1464804,88.7746284)(496.16647095,88.72463186)
\curveto(496.20648034,88.62462855)(496.2464803,88.52962865)(496.28647095,88.43963186)
\curveto(496.32648022,88.35962882)(496.37148018,88.2796289)(496.42147095,88.19963186)
\curveto(496.44148011,88.16962901)(496.46648008,88.13962904)(496.49647095,88.10963186)
\curveto(496.52648002,88.08962909)(496.55148,88.06462911)(496.57147095,88.03463186)
\lineto(496.64647095,87.95963186)
\curveto(496.66647988,87.92962925)(496.68647986,87.90462927)(496.70647095,87.88463186)
\lineto(496.91647095,87.73463186)
\curveto(496.97647957,87.69462948)(497.04147951,87.64962953)(497.11147095,87.59963186)
\curveto(497.20147935,87.53962964)(497.30647924,87.48962969)(497.42647095,87.44963186)
\curveto(497.53647901,87.41962976)(497.6464789,87.38462979)(497.75647095,87.34463186)
\curveto(497.86647868,87.30462987)(498.01147854,87.2796299)(498.19147095,87.26963186)
\curveto(498.36147819,87.25962992)(498.48647806,87.22962995)(498.56647095,87.17963186)
\curveto(498.6464779,87.12963005)(498.69147786,87.05463012)(498.70147095,86.95463186)
\curveto(498.71147784,86.85463032)(498.71647783,86.74463043)(498.71647095,86.62463186)
\curveto(498.71647783,86.58463059)(498.72147783,86.54463063)(498.73147095,86.50463186)
\curveto(498.73147782,86.46463071)(498.72647782,86.42963075)(498.71647095,86.39963186)
\curveto(498.69647785,86.34963083)(498.68647786,86.29963088)(498.68647095,86.24963186)
\curveto(498.68647786,86.20963097)(498.67647787,86.16963101)(498.65647095,86.12963186)
\curveto(498.59647795,86.03963114)(498.46147809,85.99463118)(498.25147095,85.99463186)
\lineto(498.13147095,85.99463186)
\curveto(498.07147848,86.00463117)(498.01147854,86.00963117)(497.95147095,86.00963186)
\curveto(497.88147867,86.01963116)(497.81647873,86.02963115)(497.75647095,86.03963186)
\curveto(497.6464789,86.05963112)(497.546479,86.0796311)(497.45647095,86.09963186)
\curveto(497.35647919,86.11963106)(497.26147929,86.14963103)(497.17147095,86.18963186)
\curveto(497.10147945,86.20963097)(497.04147951,86.22963095)(496.99147095,86.24963186)
\lineto(496.81147095,86.30963186)
\curveto(496.55148,86.42963075)(496.30648024,86.58463059)(496.07647095,86.77463186)
\curveto(495.8464807,86.9746302)(495.66148089,87.18962999)(495.52147095,87.41963186)
\curveto(495.44148111,87.52962965)(495.37648117,87.64462953)(495.32647095,87.76463186)
\lineto(495.17647095,88.15463186)
\curveto(495.12648142,88.26462891)(495.09648145,88.3796288)(495.08647095,88.49963186)
\curveto(495.06648148,88.61962856)(495.04148151,88.74462843)(495.01147095,88.87463186)
\curveto(495.01148154,88.94462823)(495.01148154,89.00962817)(495.01147095,89.06963186)
\curveto(495.00148155,89.12962805)(494.99148156,89.19462798)(494.98147095,89.26463186)
}
}
{
\newrgbcolor{curcolor}{0 0 0}
\pscustom[linestyle=none,fillstyle=solid,fillcolor=curcolor]
{
\newpath
\moveto(500.50147095,101.36424123)
\lineto(500.75647095,101.36424123)
\curveto(500.83647571,101.37423353)(500.91147564,101.36923353)(500.98147095,101.34924123)
\lineto(501.22147095,101.34924123)
\lineto(501.38647095,101.34924123)
\curveto(501.48647506,101.32923357)(501.59147496,101.31923358)(501.70147095,101.31924123)
\curveto(501.80147475,101.31923358)(501.90147465,101.30923359)(502.00147095,101.28924123)
\lineto(502.15147095,101.28924123)
\curveto(502.29147426,101.25923364)(502.43147412,101.23923366)(502.57147095,101.22924123)
\curveto(502.70147385,101.21923368)(502.83147372,101.19423371)(502.96147095,101.15424123)
\curveto(503.04147351,101.13423377)(503.12647342,101.11423379)(503.21647095,101.09424123)
\lineto(503.45647095,101.03424123)
\lineto(503.75647095,100.91424123)
\curveto(503.8464727,100.88423402)(503.93647261,100.84923405)(504.02647095,100.80924123)
\curveto(504.2464723,100.70923419)(504.46147209,100.57423433)(504.67147095,100.40424123)
\curveto(504.88147167,100.24423466)(505.0514715,100.06923483)(505.18147095,99.87924123)
\curveto(505.22147133,99.82923507)(505.26147129,99.76923513)(505.30147095,99.69924123)
\curveto(505.33147122,99.63923526)(505.36647118,99.57923532)(505.40647095,99.51924123)
\curveto(505.45647109,99.43923546)(505.49647105,99.34423556)(505.52647095,99.23424123)
\curveto(505.55647099,99.12423578)(505.58647096,99.01923588)(505.61647095,98.91924123)
\curveto(505.65647089,98.80923609)(505.68147087,98.6992362)(505.69147095,98.58924123)
\curveto(505.70147085,98.47923642)(505.71647083,98.36423654)(505.73647095,98.24424123)
\curveto(505.7464708,98.2042367)(505.7464708,98.15923674)(505.73647095,98.10924123)
\curveto(505.73647081,98.06923683)(505.74147081,98.02923687)(505.75147095,97.98924123)
\curveto(505.76147079,97.94923695)(505.76647078,97.89423701)(505.76647095,97.82424123)
\curveto(505.76647078,97.75423715)(505.76147079,97.7042372)(505.75147095,97.67424123)
\curveto(505.73147082,97.62423728)(505.72647082,97.57923732)(505.73647095,97.53924123)
\curveto(505.7464708,97.4992374)(505.7464708,97.46423744)(505.73647095,97.43424123)
\lineto(505.73647095,97.34424123)
\curveto(505.71647083,97.28423762)(505.70147085,97.21923768)(505.69147095,97.14924123)
\curveto(505.69147086,97.08923781)(505.68647086,97.02423788)(505.67647095,96.95424123)
\curveto(505.62647092,96.78423812)(505.57647097,96.62423828)(505.52647095,96.47424123)
\curveto(505.47647107,96.32423858)(505.41147114,96.17923872)(505.33147095,96.03924123)
\curveto(505.29147126,95.98923891)(505.26147129,95.93423897)(505.24147095,95.87424123)
\curveto(505.21147134,95.82423908)(505.17647137,95.77423913)(505.13647095,95.72424123)
\curveto(504.95647159,95.48423942)(504.73647181,95.28423962)(504.47647095,95.12424123)
\curveto(504.21647233,94.96423994)(503.93147262,94.82424008)(503.62147095,94.70424123)
\curveto(503.48147307,94.64424026)(503.34147321,94.5992403)(503.20147095,94.56924123)
\curveto(503.0514735,94.53924036)(502.89647365,94.5042404)(502.73647095,94.46424123)
\curveto(502.62647392,94.44424046)(502.51647403,94.42924047)(502.40647095,94.41924123)
\curveto(502.29647425,94.40924049)(502.18647436,94.39424051)(502.07647095,94.37424123)
\curveto(502.03647451,94.36424054)(501.99647455,94.35924054)(501.95647095,94.35924123)
\curveto(501.91647463,94.36924053)(501.87647467,94.36924053)(501.83647095,94.35924123)
\curveto(501.78647476,94.34924055)(501.73647481,94.34424056)(501.68647095,94.34424123)
\lineto(501.52147095,94.34424123)
\curveto(501.47147508,94.32424058)(501.42147513,94.31924058)(501.37147095,94.32924123)
\curveto(501.31147524,94.33924056)(501.25647529,94.33924056)(501.20647095,94.32924123)
\curveto(501.16647538,94.31924058)(501.12147543,94.31924058)(501.07147095,94.32924123)
\curveto(501.02147553,94.33924056)(500.97147558,94.33424057)(500.92147095,94.31424123)
\curveto(500.8514757,94.29424061)(500.77647577,94.28924061)(500.69647095,94.29924123)
\curveto(500.60647594,94.30924059)(500.52147603,94.31424059)(500.44147095,94.31424123)
\curveto(500.3514762,94.31424059)(500.2514763,94.30924059)(500.14147095,94.29924123)
\curveto(500.02147653,94.28924061)(499.92147663,94.29424061)(499.84147095,94.31424123)
\lineto(499.55647095,94.31424123)
\lineto(498.92647095,94.35924123)
\curveto(498.82647772,94.36924053)(498.73147782,94.37924052)(498.64147095,94.38924123)
\lineto(498.34147095,94.41924123)
\curveto(498.29147826,94.43924046)(498.24147831,94.44424046)(498.19147095,94.43424123)
\curveto(498.13147842,94.43424047)(498.07647847,94.44424046)(498.02647095,94.46424123)
\curveto(497.85647869,94.51424039)(497.69147886,94.55424035)(497.53147095,94.58424123)
\curveto(497.36147919,94.61424029)(497.20147935,94.66424024)(497.05147095,94.73424123)
\curveto(496.59147996,94.92423998)(496.21648033,95.14423976)(495.92647095,95.39424123)
\curveto(495.63648091,95.65423925)(495.39148116,96.01423889)(495.19147095,96.47424123)
\curveto(495.14148141,96.6042383)(495.10648144,96.73423817)(495.08647095,96.86424123)
\curveto(495.06648148,97.0042379)(495.04148151,97.14423776)(495.01147095,97.28424123)
\curveto(495.00148155,97.35423755)(494.99648155,97.41923748)(494.99647095,97.47924123)
\curveto(494.99648155,97.53923736)(494.99148156,97.6042373)(494.98147095,97.67424123)
\curveto(494.96148159,98.5042364)(495.11148144,99.17423573)(495.43147095,99.68424123)
\curveto(495.74148081,100.19423471)(496.18148037,100.57423433)(496.75147095,100.82424123)
\curveto(496.87147968,100.87423403)(496.99647955,100.91923398)(497.12647095,100.95924123)
\curveto(497.25647929,100.9992339)(497.39147916,101.04423386)(497.53147095,101.09424123)
\curveto(497.61147894,101.11423379)(497.69647885,101.12923377)(497.78647095,101.13924123)
\lineto(498.02647095,101.19924123)
\curveto(498.13647841,101.22923367)(498.2464783,101.24423366)(498.35647095,101.24424123)
\curveto(498.46647808,101.25423365)(498.57647797,101.26923363)(498.68647095,101.28924123)
\curveto(498.73647781,101.30923359)(498.78147777,101.31423359)(498.82147095,101.30424123)
\curveto(498.86147769,101.3042336)(498.90147765,101.30923359)(498.94147095,101.31924123)
\curveto(498.99147756,101.32923357)(499.0464775,101.32923357)(499.10647095,101.31924123)
\curveto(499.15647739,101.31923358)(499.20647734,101.32423358)(499.25647095,101.33424123)
\lineto(499.39147095,101.33424123)
\curveto(499.4514771,101.35423355)(499.52147703,101.35423355)(499.60147095,101.33424123)
\curveto(499.67147688,101.32423358)(499.73647681,101.32923357)(499.79647095,101.34924123)
\curveto(499.82647672,101.35923354)(499.86647668,101.36423354)(499.91647095,101.36424123)
\lineto(500.03647095,101.36424123)
\lineto(500.50147095,101.36424123)
\moveto(502.82647095,99.81924123)
\curveto(502.50647404,99.91923498)(502.14147441,99.97923492)(501.73147095,99.99924123)
\curveto(501.32147523,100.01923488)(500.91147564,100.02923487)(500.50147095,100.02924123)
\curveto(500.07147648,100.02923487)(499.6514769,100.01923488)(499.24147095,99.99924123)
\curveto(498.83147772,99.97923492)(498.4464781,99.93423497)(498.08647095,99.86424123)
\curveto(497.72647882,99.79423511)(497.40647914,99.68423522)(497.12647095,99.53424123)
\curveto(496.83647971,99.39423551)(496.60147995,99.1992357)(496.42147095,98.94924123)
\curveto(496.31148024,98.78923611)(496.23148032,98.60923629)(496.18147095,98.40924123)
\curveto(496.12148043,98.20923669)(496.09148046,97.96423694)(496.09147095,97.67424123)
\curveto(496.11148044,97.65423725)(496.12148043,97.61923728)(496.12147095,97.56924123)
\curveto(496.11148044,97.51923738)(496.11148044,97.47923742)(496.12147095,97.44924123)
\curveto(496.14148041,97.36923753)(496.16148039,97.29423761)(496.18147095,97.22424123)
\curveto(496.19148036,97.16423774)(496.21148034,97.0992378)(496.24147095,97.02924123)
\curveto(496.36148019,96.75923814)(496.53148002,96.53923836)(496.75147095,96.36924123)
\curveto(496.96147959,96.20923869)(497.20647934,96.07423883)(497.48647095,95.96424123)
\curveto(497.59647895,95.91423899)(497.71647883,95.87423903)(497.84647095,95.84424123)
\curveto(497.96647858,95.82423908)(498.09147846,95.7992391)(498.22147095,95.76924123)
\curveto(498.27147828,95.74923915)(498.32647822,95.73923916)(498.38647095,95.73924123)
\curveto(498.43647811,95.73923916)(498.48647806,95.73423917)(498.53647095,95.72424123)
\curveto(498.62647792,95.71423919)(498.72147783,95.7042392)(498.82147095,95.69424123)
\curveto(498.91147764,95.68423922)(499.00647754,95.67423923)(499.10647095,95.66424123)
\curveto(499.18647736,95.66423924)(499.27147728,95.65923924)(499.36147095,95.64924123)
\lineto(499.60147095,95.64924123)
\lineto(499.78147095,95.64924123)
\curveto(499.81147674,95.63923926)(499.8464767,95.63423927)(499.88647095,95.63424123)
\lineto(500.02147095,95.63424123)
\lineto(500.47147095,95.63424123)
\curveto(500.551476,95.63423927)(500.63647591,95.62923927)(500.72647095,95.61924123)
\curveto(500.80647574,95.61923928)(500.88147567,95.62923927)(500.95147095,95.64924123)
\lineto(501.22147095,95.64924123)
\curveto(501.24147531,95.64923925)(501.27147528,95.64423926)(501.31147095,95.63424123)
\curveto(501.34147521,95.63423927)(501.36647518,95.63923926)(501.38647095,95.64924123)
\curveto(501.48647506,95.65923924)(501.58647496,95.66423924)(501.68647095,95.66424123)
\curveto(501.77647477,95.67423923)(501.87647467,95.68423922)(501.98647095,95.69424123)
\curveto(502.10647444,95.72423918)(502.23147432,95.73923916)(502.36147095,95.73924123)
\curveto(502.48147407,95.74923915)(502.59647395,95.77423913)(502.70647095,95.81424123)
\curveto(503.00647354,95.89423901)(503.27147328,95.97923892)(503.50147095,96.06924123)
\curveto(503.73147282,96.16923873)(503.9464726,96.31423859)(504.14647095,96.50424123)
\curveto(504.3464722,96.71423819)(504.49647205,96.97923792)(504.59647095,97.29924123)
\curveto(504.61647193,97.33923756)(504.62647192,97.37423753)(504.62647095,97.40424123)
\curveto(504.61647193,97.44423746)(504.62147193,97.48923741)(504.64147095,97.53924123)
\curveto(504.6514719,97.57923732)(504.66147189,97.64923725)(504.67147095,97.74924123)
\curveto(504.68147187,97.85923704)(504.67647187,97.94423696)(504.65647095,98.00424123)
\curveto(504.63647191,98.07423683)(504.62647192,98.14423676)(504.62647095,98.21424123)
\curveto(504.61647193,98.28423662)(504.60147195,98.34923655)(504.58147095,98.40924123)
\curveto(504.52147203,98.60923629)(504.43647211,98.78923611)(504.32647095,98.94924123)
\curveto(504.30647224,98.97923592)(504.28647226,99.0042359)(504.26647095,99.02424123)
\lineto(504.20647095,99.08424123)
\curveto(504.18647236,99.12423578)(504.1464724,99.17423573)(504.08647095,99.23424123)
\curveto(503.9464726,99.33423557)(503.81647273,99.41923548)(503.69647095,99.48924123)
\curveto(503.57647297,99.55923534)(503.43147312,99.62923527)(503.26147095,99.69924123)
\curveto(503.19147336,99.72923517)(503.12147343,99.74923515)(503.05147095,99.75924123)
\curveto(502.98147357,99.77923512)(502.90647364,99.7992351)(502.82647095,99.81924123)
}
}
{
\newrgbcolor{curcolor}{0 0 0}
\pscustom[linestyle=none,fillstyle=solid,fillcolor=curcolor]
{
\newpath
\moveto(494.98147095,106.77385061)
\curveto(494.98148157,106.87384575)(494.99148156,106.96884566)(495.01147095,107.05885061)
\curveto(495.02148153,107.14884548)(495.0514815,107.21384541)(495.10147095,107.25385061)
\curveto(495.18148137,107.31384531)(495.28648126,107.34384528)(495.41647095,107.34385061)
\lineto(495.80647095,107.34385061)
\lineto(497.30647095,107.34385061)
\lineto(503.69647095,107.34385061)
\lineto(504.86647095,107.34385061)
\lineto(505.18147095,107.34385061)
\curveto(505.28147127,107.35384527)(505.36147119,107.33884529)(505.42147095,107.29885061)
\curveto(505.50147105,107.24884538)(505.551471,107.17384545)(505.57147095,107.07385061)
\curveto(505.58147097,106.98384564)(505.58647096,106.87384575)(505.58647095,106.74385061)
\lineto(505.58647095,106.51885061)
\curveto(505.56647098,106.43884619)(505.551471,106.36884626)(505.54147095,106.30885061)
\curveto(505.52147103,106.24884638)(505.48147107,106.19884643)(505.42147095,106.15885061)
\curveto(505.36147119,106.11884651)(505.28647126,106.09884653)(505.19647095,106.09885061)
\lineto(504.89647095,106.09885061)
\lineto(503.80147095,106.09885061)
\lineto(498.46147095,106.09885061)
\curveto(498.37147818,106.07884655)(498.29647825,106.06384656)(498.23647095,106.05385061)
\curveto(498.16647838,106.05384657)(498.10647844,106.0238466)(498.05647095,105.96385061)
\curveto(498.00647854,105.89384673)(497.98147857,105.80384682)(497.98147095,105.69385061)
\curveto(497.97147858,105.59384703)(497.96647858,105.48384714)(497.96647095,105.36385061)
\lineto(497.96647095,104.22385061)
\lineto(497.96647095,103.72885061)
\curveto(497.95647859,103.56884906)(497.89647865,103.45884917)(497.78647095,103.39885061)
\curveto(497.75647879,103.37884925)(497.72647882,103.36884926)(497.69647095,103.36885061)
\curveto(497.65647889,103.36884926)(497.61147894,103.36384926)(497.56147095,103.35385061)
\curveto(497.44147911,103.33384929)(497.33147922,103.33884929)(497.23147095,103.36885061)
\curveto(497.13147942,103.40884922)(497.06147949,103.46384916)(497.02147095,103.53385061)
\curveto(496.97147958,103.61384901)(496.9464796,103.73384889)(496.94647095,103.89385061)
\curveto(496.9464796,104.05384857)(496.93147962,104.18884844)(496.90147095,104.29885061)
\curveto(496.89147966,104.34884828)(496.88647966,104.40384822)(496.88647095,104.46385061)
\curveto(496.87647967,104.5238481)(496.86147969,104.58384804)(496.84147095,104.64385061)
\curveto(496.79147976,104.79384783)(496.74147981,104.93884769)(496.69147095,105.07885061)
\curveto(496.63147992,105.21884741)(496.56147999,105.35384727)(496.48147095,105.48385061)
\curveto(496.39148016,105.623847)(496.28648026,105.74384688)(496.16647095,105.84385061)
\curveto(496.0464805,105.94384668)(495.91648063,106.03884659)(495.77647095,106.12885061)
\curveto(495.67648087,106.18884644)(495.56648098,106.23384639)(495.44647095,106.26385061)
\curveto(495.32648122,106.30384632)(495.22148133,106.35384627)(495.13147095,106.41385061)
\curveto(495.07148148,106.46384616)(495.03148152,106.53384609)(495.01147095,106.62385061)
\curveto(495.00148155,106.64384598)(494.99648155,106.66884596)(494.99647095,106.69885061)
\curveto(494.99648155,106.7288459)(494.99148156,106.75384587)(494.98147095,106.77385061)
}
}
{
\newrgbcolor{curcolor}{0 0 0}
\pscustom[linestyle=none,fillstyle=solid,fillcolor=curcolor]
{
\newpath
\moveto(494.98147095,115.12345998)
\curveto(494.98148157,115.22345513)(494.99148156,115.31845503)(495.01147095,115.40845998)
\curveto(495.02148153,115.49845485)(495.0514815,115.56345479)(495.10147095,115.60345998)
\curveto(495.18148137,115.66345469)(495.28648126,115.69345466)(495.41647095,115.69345998)
\lineto(495.80647095,115.69345998)
\lineto(497.30647095,115.69345998)
\lineto(503.69647095,115.69345998)
\lineto(504.86647095,115.69345998)
\lineto(505.18147095,115.69345998)
\curveto(505.28147127,115.70345465)(505.36147119,115.68845466)(505.42147095,115.64845998)
\curveto(505.50147105,115.59845475)(505.551471,115.52345483)(505.57147095,115.42345998)
\curveto(505.58147097,115.33345502)(505.58647096,115.22345513)(505.58647095,115.09345998)
\lineto(505.58647095,114.86845998)
\curveto(505.56647098,114.78845556)(505.551471,114.71845563)(505.54147095,114.65845998)
\curveto(505.52147103,114.59845575)(505.48147107,114.5484558)(505.42147095,114.50845998)
\curveto(505.36147119,114.46845588)(505.28647126,114.4484559)(505.19647095,114.44845998)
\lineto(504.89647095,114.44845998)
\lineto(503.80147095,114.44845998)
\lineto(498.46147095,114.44845998)
\curveto(498.37147818,114.42845592)(498.29647825,114.41345594)(498.23647095,114.40345998)
\curveto(498.16647838,114.40345595)(498.10647844,114.37345598)(498.05647095,114.31345998)
\curveto(498.00647854,114.24345611)(497.98147857,114.1534562)(497.98147095,114.04345998)
\curveto(497.97147858,113.94345641)(497.96647858,113.83345652)(497.96647095,113.71345998)
\lineto(497.96647095,112.57345998)
\lineto(497.96647095,112.07845998)
\curveto(497.95647859,111.91845843)(497.89647865,111.80845854)(497.78647095,111.74845998)
\curveto(497.75647879,111.72845862)(497.72647882,111.71845863)(497.69647095,111.71845998)
\curveto(497.65647889,111.71845863)(497.61147894,111.71345864)(497.56147095,111.70345998)
\curveto(497.44147911,111.68345867)(497.33147922,111.68845866)(497.23147095,111.71845998)
\curveto(497.13147942,111.75845859)(497.06147949,111.81345854)(497.02147095,111.88345998)
\curveto(496.97147958,111.96345839)(496.9464796,112.08345827)(496.94647095,112.24345998)
\curveto(496.9464796,112.40345795)(496.93147962,112.53845781)(496.90147095,112.64845998)
\curveto(496.89147966,112.69845765)(496.88647966,112.7534576)(496.88647095,112.81345998)
\curveto(496.87647967,112.87345748)(496.86147969,112.93345742)(496.84147095,112.99345998)
\curveto(496.79147976,113.14345721)(496.74147981,113.28845706)(496.69147095,113.42845998)
\curveto(496.63147992,113.56845678)(496.56147999,113.70345665)(496.48147095,113.83345998)
\curveto(496.39148016,113.97345638)(496.28648026,114.09345626)(496.16647095,114.19345998)
\curveto(496.0464805,114.29345606)(495.91648063,114.38845596)(495.77647095,114.47845998)
\curveto(495.67648087,114.53845581)(495.56648098,114.58345577)(495.44647095,114.61345998)
\curveto(495.32648122,114.6534557)(495.22148133,114.70345565)(495.13147095,114.76345998)
\curveto(495.07148148,114.81345554)(495.03148152,114.88345547)(495.01147095,114.97345998)
\curveto(495.00148155,114.99345536)(494.99648155,115.01845533)(494.99647095,115.04845998)
\curveto(494.99648155,115.07845527)(494.99148156,115.10345525)(494.98147095,115.12345998)
}
}
{
\newrgbcolor{curcolor}{0 0 0}
\pscustom[linestyle=none,fillstyle=solid,fillcolor=curcolor]
{
\newpath
\moveto(516.85279053,29.18119436)
\lineto(516.85279053,30.09619436)
\curveto(516.85280122,30.19619171)(516.85280122,30.29119161)(516.85279053,30.38119436)
\curveto(516.85280122,30.47119143)(516.8728012,30.54619136)(516.91279053,30.60619436)
\curveto(516.9728011,30.69619121)(517.05280102,30.75619115)(517.15279053,30.78619436)
\curveto(517.25280082,30.82619108)(517.35780072,30.87119103)(517.46779053,30.92119436)
\curveto(517.65780042,31.0011909)(517.84780023,31.07119083)(518.03779053,31.13119436)
\curveto(518.22779985,31.2011907)(518.41779966,31.27619063)(518.60779053,31.35619436)
\curveto(518.78779929,31.42619048)(518.9727991,31.49119041)(519.16279053,31.55119436)
\curveto(519.34279873,31.61119029)(519.52279855,31.68119022)(519.70279053,31.76119436)
\curveto(519.84279823,31.82119008)(519.98779809,31.87619003)(520.13779053,31.92619436)
\curveto(520.28779779,31.97618993)(520.43279764,32.03118987)(520.57279053,32.09119436)
\curveto(521.02279705,32.27118963)(521.4777966,32.44118946)(521.93779053,32.60119436)
\curveto(522.38779569,32.76118914)(522.83779524,32.93118897)(523.28779053,33.11119436)
\curveto(523.33779474,33.13118877)(523.38779469,33.14618876)(523.43779053,33.15619436)
\lineto(523.58779053,33.21619436)
\curveto(523.80779427,33.3061886)(524.03279404,33.39118851)(524.26279053,33.47119436)
\curveto(524.48279359,33.55118835)(524.70279337,33.63618827)(524.92279053,33.72619436)
\curveto(525.01279306,33.76618814)(525.12279295,33.8061881)(525.25279053,33.84619436)
\curveto(525.3727927,33.88618802)(525.44279263,33.95118795)(525.46279053,34.04119436)
\curveto(525.4727926,34.08118782)(525.4727926,34.11118779)(525.46279053,34.13119436)
\lineto(525.40279053,34.19119436)
\curveto(525.35279272,34.24118766)(525.29779278,34.27618763)(525.23779053,34.29619436)
\curveto(525.1777929,34.32618758)(525.11279296,34.35618755)(525.04279053,34.38619436)
\lineto(524.41279053,34.62619436)
\curveto(524.19279388,34.7061872)(523.9777941,34.78618712)(523.76779053,34.86619436)
\lineto(523.61779053,34.92619436)
\lineto(523.43779053,34.98619436)
\curveto(523.24779483,35.06618684)(523.05779502,35.13618677)(522.86779053,35.19619436)
\curveto(522.66779541,35.26618664)(522.46779561,35.34118656)(522.26779053,35.42119436)
\curveto(521.68779639,35.66118624)(521.10279697,35.88118602)(520.51279053,36.08119436)
\curveto(519.92279815,36.29118561)(519.33779874,36.51618539)(518.75779053,36.75619436)
\curveto(518.55779952,36.83618507)(518.35279972,36.91118499)(518.14279053,36.98119436)
\curveto(517.93280014,37.06118484)(517.72780035,37.14118476)(517.52779053,37.22119436)
\curveto(517.44780063,37.26118464)(517.34780073,37.29618461)(517.22779053,37.32619436)
\curveto(517.10780097,37.36618454)(517.02280105,37.42118448)(516.97279053,37.49119436)
\curveto(516.93280114,37.55118435)(516.90280117,37.62618428)(516.88279053,37.71619436)
\curveto(516.86280121,37.81618409)(516.85280122,37.92618398)(516.85279053,38.04619436)
\curveto(516.84280123,38.16618374)(516.84280123,38.28618362)(516.85279053,38.40619436)
\curveto(516.85280122,38.52618338)(516.85280122,38.63618327)(516.85279053,38.73619436)
\curveto(516.85280122,38.82618308)(516.85280122,38.91618299)(516.85279053,39.00619436)
\curveto(516.85280122,39.1061828)(516.8728012,39.18118272)(516.91279053,39.23119436)
\curveto(516.96280111,39.32118258)(517.05280102,39.37118253)(517.18279053,39.38119436)
\curveto(517.31280076,39.39118251)(517.45280062,39.39618251)(517.60279053,39.39619436)
\lineto(519.25279053,39.39619436)
\lineto(525.52279053,39.39619436)
\lineto(526.78279053,39.39619436)
\curveto(526.89279118,39.39618251)(527.00279107,39.39618251)(527.11279053,39.39619436)
\curveto(527.22279085,39.4061825)(527.30779077,39.38618252)(527.36779053,39.33619436)
\curveto(527.42779065,39.3061826)(527.46779061,39.26118264)(527.48779053,39.20119436)
\curveto(527.49779058,39.14118276)(527.51279056,39.07118283)(527.53279053,38.99119436)
\lineto(527.53279053,38.75119436)
\lineto(527.53279053,38.39119436)
\curveto(527.52279055,38.28118362)(527.4777906,38.2011837)(527.39779053,38.15119436)
\curveto(527.36779071,38.13118377)(527.33779074,38.11618379)(527.30779053,38.10619436)
\curveto(527.26779081,38.1061838)(527.22279085,38.09618381)(527.17279053,38.07619436)
\lineto(527.00779053,38.07619436)
\curveto(526.94779113,38.06618384)(526.8777912,38.06118384)(526.79779053,38.06119436)
\curveto(526.71779136,38.07118383)(526.64279143,38.07618383)(526.57279053,38.07619436)
\lineto(525.73279053,38.07619436)
\lineto(521.30779053,38.07619436)
\curveto(521.05779702,38.07618383)(520.80779727,38.07618383)(520.55779053,38.07619436)
\curveto(520.29779778,38.07618383)(520.04779803,38.07118383)(519.80779053,38.06119436)
\curveto(519.70779837,38.06118384)(519.59779848,38.05618385)(519.47779053,38.04619436)
\curveto(519.35779872,38.03618387)(519.29779878,37.98118392)(519.29779053,37.88119436)
\lineto(519.31279053,37.88119436)
\curveto(519.33279874,37.81118409)(519.39779868,37.75118415)(519.50779053,37.70119436)
\curveto(519.61779846,37.66118424)(519.71279836,37.62618428)(519.79279053,37.59619436)
\curveto(519.96279811,37.52618438)(520.13779794,37.46118444)(520.31779053,37.40119436)
\curveto(520.48779759,37.34118456)(520.65779742,37.27118463)(520.82779053,37.19119436)
\curveto(520.8777972,37.17118473)(520.92279715,37.15618475)(520.96279053,37.14619436)
\curveto(521.00279707,37.13618477)(521.04779703,37.12118478)(521.09779053,37.10119436)
\curveto(521.2777968,37.02118488)(521.46279661,36.95118495)(521.65279053,36.89119436)
\curveto(521.83279624,36.84118506)(522.01279606,36.77618513)(522.19279053,36.69619436)
\curveto(522.34279573,36.62618528)(522.49779558,36.56618534)(522.65779053,36.51619436)
\curveto(522.80779527,36.46618544)(522.95779512,36.41118549)(523.10779053,36.35119436)
\curveto(523.5777945,36.15118575)(524.05279402,35.97118593)(524.53279053,35.81119436)
\curveto(525.00279307,35.65118625)(525.46779261,35.47618643)(525.92779053,35.28619436)
\curveto(526.10779197,35.2061867)(526.28779179,35.13618677)(526.46779053,35.07619436)
\curveto(526.64779143,35.01618689)(526.82779125,34.95118695)(527.00779053,34.88119436)
\curveto(527.11779096,34.83118707)(527.22279085,34.78118712)(527.32279053,34.73119436)
\curveto(527.41279066,34.69118721)(527.4777906,34.6061873)(527.51779053,34.47619436)
\curveto(527.52779055,34.45618745)(527.53279054,34.43118747)(527.53279053,34.40119436)
\curveto(527.52279055,34.38118752)(527.52279055,34.35618755)(527.53279053,34.32619436)
\curveto(527.54279053,34.29618761)(527.54779053,34.26118764)(527.54779053,34.22119436)
\curveto(527.53779054,34.18118772)(527.53279054,34.14118776)(527.53279053,34.10119436)
\lineto(527.53279053,33.80119436)
\curveto(527.53279054,33.7011882)(527.50779057,33.62118828)(527.45779053,33.56119436)
\curveto(527.40779067,33.48118842)(527.33779074,33.42118848)(527.24779053,33.38119436)
\curveto(527.14779093,33.35118855)(527.04779103,33.31118859)(526.94779053,33.26119436)
\curveto(526.74779133,33.18118872)(526.54279153,33.1011888)(526.33279053,33.02119436)
\curveto(526.11279196,32.95118895)(525.90279217,32.87618903)(525.70279053,32.79619436)
\curveto(525.52279255,32.71618919)(525.34279273,32.64618926)(525.16279053,32.58619436)
\curveto(524.9727931,32.53618937)(524.78779329,32.47118943)(524.60779053,32.39119436)
\curveto(524.04779403,32.16118974)(523.48279459,31.94618996)(522.91279053,31.74619436)
\curveto(522.34279573,31.54619036)(521.7777963,31.33119057)(521.21779053,31.10119436)
\lineto(520.58779053,30.86119436)
\curveto(520.36779771,30.79119111)(520.15779792,30.71619119)(519.95779053,30.63619436)
\curveto(519.84779823,30.58619132)(519.74279833,30.54119136)(519.64279053,30.50119436)
\curveto(519.53279854,30.47119143)(519.43779864,30.42119148)(519.35779053,30.35119436)
\curveto(519.33779874,30.34119156)(519.32779875,30.33119157)(519.32779053,30.32119436)
\lineto(519.29779053,30.29119436)
\lineto(519.29779053,30.21619436)
\lineto(519.32779053,30.18619436)
\curveto(519.32779875,30.17619173)(519.33279874,30.16619174)(519.34279053,30.15619436)
\curveto(519.39279868,30.13619177)(519.44779863,30.12619178)(519.50779053,30.12619436)
\curveto(519.56779851,30.12619178)(519.62779845,30.11619179)(519.68779053,30.09619436)
\lineto(519.85279053,30.09619436)
\curveto(519.91279816,30.07619183)(519.9777981,30.07119183)(520.04779053,30.08119436)
\curveto(520.11779796,30.09119181)(520.18779789,30.09619181)(520.25779053,30.09619436)
\lineto(521.06779053,30.09619436)
\lineto(525.62779053,30.09619436)
\lineto(526.81279053,30.09619436)
\curveto(526.92279115,30.09619181)(527.03279104,30.09119181)(527.14279053,30.08119436)
\curveto(527.25279082,30.08119182)(527.33779074,30.05619185)(527.39779053,30.00619436)
\curveto(527.4777906,29.95619195)(527.52279055,29.86619204)(527.53279053,29.73619436)
\lineto(527.53279053,29.34619436)
\lineto(527.53279053,29.15119436)
\curveto(527.53279054,29.1011928)(527.52279055,29.05119285)(527.50279053,29.00119436)
\curveto(527.46279061,28.87119303)(527.3777907,28.79619311)(527.24779053,28.77619436)
\curveto(527.11779096,28.76619314)(526.96779111,28.76119314)(526.79779053,28.76119436)
\lineto(525.05779053,28.76119436)
\lineto(519.05779053,28.76119436)
\lineto(517.64779053,28.76119436)
\curveto(517.53780054,28.76119314)(517.42280065,28.75619315)(517.30279053,28.74619436)
\curveto(517.18280089,28.74619316)(517.08780099,28.77119313)(517.01779053,28.82119436)
\curveto(516.95780112,28.86119304)(516.90780117,28.93619297)(516.86779053,29.04619436)
\curveto(516.85780122,29.06619284)(516.85780122,29.08619282)(516.86779053,29.10619436)
\curveto(516.86780121,29.13619277)(516.86280121,29.16119274)(516.85279053,29.18119436)
}
}
{
\newrgbcolor{curcolor}{0 0 0}
\pscustom[linestyle=none,fillstyle=solid,fillcolor=curcolor]
{
\newpath
\moveto(526.97779053,48.38330373)
\curveto(527.13779094,48.4132959)(527.2727908,48.39829592)(527.38279053,48.33830373)
\curveto(527.48279059,48.27829604)(527.55779052,48.19829612)(527.60779053,48.09830373)
\curveto(527.62779045,48.04829627)(527.63779044,47.99329632)(527.63779053,47.93330373)
\curveto(527.63779044,47.88329643)(527.64779043,47.82829649)(527.66779053,47.76830373)
\curveto(527.71779036,47.54829677)(527.70279037,47.32829699)(527.62279053,47.10830373)
\curveto(527.55279052,46.89829742)(527.46279061,46.75329756)(527.35279053,46.67330373)
\curveto(527.28279079,46.62329769)(527.20279087,46.57829774)(527.11279053,46.53830373)
\curveto(527.01279106,46.49829782)(526.93279114,46.44829787)(526.87279053,46.38830373)
\curveto(526.85279122,46.36829795)(526.83279124,46.34329797)(526.81279053,46.31330373)
\curveto(526.79279128,46.29329802)(526.78779129,46.26329805)(526.79779053,46.22330373)
\curveto(526.82779125,46.1132982)(526.88279119,46.00829831)(526.96279053,45.90830373)
\curveto(527.04279103,45.8182985)(527.11279096,45.72829859)(527.17279053,45.63830373)
\curveto(527.25279082,45.50829881)(527.32779075,45.36829895)(527.39779053,45.21830373)
\curveto(527.45779062,45.06829925)(527.51279056,44.90829941)(527.56279053,44.73830373)
\curveto(527.59279048,44.63829968)(527.61279046,44.52829979)(527.62279053,44.40830373)
\curveto(527.63279044,44.29830002)(527.64779043,44.18830013)(527.66779053,44.07830373)
\curveto(527.6777904,44.02830029)(527.68279039,43.98330033)(527.68279053,43.94330373)
\lineto(527.68279053,43.83830373)
\curveto(527.70279037,43.72830059)(527.70279037,43.62330069)(527.68279053,43.52330373)
\lineto(527.68279053,43.38830373)
\curveto(527.6727904,43.33830098)(527.66779041,43.28830103)(527.66779053,43.23830373)
\curveto(527.66779041,43.18830113)(527.65779042,43.14330117)(527.63779053,43.10330373)
\curveto(527.62779045,43.06330125)(527.62279045,43.02830129)(527.62279053,42.99830373)
\curveto(527.63279044,42.97830134)(527.63279044,42.95330136)(527.62279053,42.92330373)
\lineto(527.56279053,42.68330373)
\curveto(527.55279052,42.60330171)(527.53279054,42.52830179)(527.50279053,42.45830373)
\curveto(527.3727907,42.15830216)(527.22779085,41.9133024)(527.06779053,41.72330373)
\curveto(526.89779118,41.54330277)(526.66279141,41.39330292)(526.36279053,41.27330373)
\curveto(526.14279193,41.18330313)(525.8777922,41.13830318)(525.56779053,41.13830373)
\lineto(525.25279053,41.13830373)
\curveto(525.20279287,41.14830317)(525.15279292,41.15330316)(525.10279053,41.15330373)
\lineto(524.92279053,41.18330373)
\lineto(524.59279053,41.30330373)
\curveto(524.48279359,41.34330297)(524.38279369,41.39330292)(524.29279053,41.45330373)
\curveto(524.00279407,41.63330268)(523.78779429,41.87830244)(523.64779053,42.18830373)
\curveto(523.50779457,42.49830182)(523.38279469,42.83830148)(523.27279053,43.20830373)
\curveto(523.23279484,43.34830097)(523.20279487,43.49330082)(523.18279053,43.64330373)
\curveto(523.16279491,43.79330052)(523.13779494,43.94330037)(523.10779053,44.09330373)
\curveto(523.08779499,44.16330015)(523.077795,44.22830009)(523.07779053,44.28830373)
\curveto(523.077795,44.35829996)(523.06779501,44.43329988)(523.04779053,44.51330373)
\curveto(523.02779505,44.58329973)(523.01779506,44.65329966)(523.01779053,44.72330373)
\curveto(523.00779507,44.79329952)(522.99279508,44.86829945)(522.97279053,44.94830373)
\curveto(522.91279516,45.19829912)(522.86279521,45.43329888)(522.82279053,45.65330373)
\curveto(522.7727953,45.87329844)(522.65779542,46.04829827)(522.47779053,46.17830373)
\curveto(522.39779568,46.23829808)(522.29779578,46.28829803)(522.17779053,46.32830373)
\curveto(522.04779603,46.36829795)(521.90779617,46.36829795)(521.75779053,46.32830373)
\curveto(521.51779656,46.26829805)(521.32779675,46.17829814)(521.18779053,46.05830373)
\curveto(521.04779703,45.94829837)(520.93779714,45.78829853)(520.85779053,45.57830373)
\curveto(520.80779727,45.45829886)(520.7727973,45.313299)(520.75279053,45.14330373)
\curveto(520.73279734,44.98329933)(520.72279735,44.8132995)(520.72279053,44.63330373)
\curveto(520.72279735,44.45329986)(520.73279734,44.27830004)(520.75279053,44.10830373)
\curveto(520.7727973,43.93830038)(520.80279727,43.79330052)(520.84279053,43.67330373)
\curveto(520.90279717,43.50330081)(520.98779709,43.33830098)(521.09779053,43.17830373)
\curveto(521.15779692,43.09830122)(521.23779684,43.02330129)(521.33779053,42.95330373)
\curveto(521.42779665,42.89330142)(521.52779655,42.83830148)(521.63779053,42.78830373)
\curveto(521.71779636,42.75830156)(521.80279627,42.72830159)(521.89279053,42.69830373)
\curveto(521.98279609,42.67830164)(522.05279602,42.63330168)(522.10279053,42.56330373)
\curveto(522.13279594,42.52330179)(522.15779592,42.45330186)(522.17779053,42.35330373)
\curveto(522.18779589,42.26330205)(522.19279588,42.16830215)(522.19279053,42.06830373)
\curveto(522.19279588,41.96830235)(522.18779589,41.86830245)(522.17779053,41.76830373)
\curveto(522.15779592,41.67830264)(522.13279594,41.6133027)(522.10279053,41.57330373)
\curveto(522.072796,41.53330278)(522.02279605,41.50330281)(521.95279053,41.48330373)
\curveto(521.88279619,41.46330285)(521.80779627,41.46330285)(521.72779053,41.48330373)
\curveto(521.59779648,41.5133028)(521.4777966,41.54330277)(521.36779053,41.57330373)
\curveto(521.24779683,41.6133027)(521.13279694,41.65830266)(521.02279053,41.70830373)
\curveto(520.6727974,41.89830242)(520.40279767,42.13830218)(520.21279053,42.42830373)
\curveto(520.01279806,42.7183016)(519.85279822,43.07830124)(519.73279053,43.50830373)
\curveto(519.71279836,43.60830071)(519.69779838,43.70830061)(519.68779053,43.80830373)
\curveto(519.6777984,43.9183004)(519.66279841,44.02830029)(519.64279053,44.13830373)
\curveto(519.63279844,44.17830014)(519.63279844,44.24330007)(519.64279053,44.33330373)
\curveto(519.64279843,44.42329989)(519.63279844,44.47829984)(519.61279053,44.49830373)
\curveto(519.60279847,45.19829912)(519.68279839,45.80829851)(519.85279053,46.32830373)
\curveto(520.02279805,46.84829747)(520.34779773,47.2132971)(520.82779053,47.42330373)
\curveto(521.02779705,47.5132968)(521.26279681,47.56329675)(521.53279053,47.57330373)
\curveto(521.79279628,47.59329672)(522.06779601,47.60329671)(522.35779053,47.60330373)
\lineto(525.67279053,47.60330373)
\curveto(525.81279226,47.60329671)(525.94779213,47.60829671)(526.07779053,47.61830373)
\curveto(526.20779187,47.62829669)(526.31279176,47.65829666)(526.39279053,47.70830373)
\curveto(526.46279161,47.75829656)(526.51279156,47.82329649)(526.54279053,47.90330373)
\curveto(526.58279149,47.99329632)(526.61279146,48.07829624)(526.63279053,48.15830373)
\curveto(526.64279143,48.23829608)(526.68779139,48.29829602)(526.76779053,48.33830373)
\curveto(526.79779128,48.35829596)(526.82779125,48.36829595)(526.85779053,48.36830373)
\curveto(526.88779119,48.36829595)(526.92779115,48.37329594)(526.97779053,48.38330373)
\moveto(525.31279053,46.23830373)
\curveto(525.1727929,46.29829802)(525.01279306,46.32829799)(524.83279053,46.32830373)
\curveto(524.64279343,46.33829798)(524.44779363,46.34329797)(524.24779053,46.34330373)
\curveto(524.13779394,46.34329797)(524.03779404,46.33829798)(523.94779053,46.32830373)
\curveto(523.85779422,46.318298)(523.78779429,46.27829804)(523.73779053,46.20830373)
\curveto(523.71779436,46.17829814)(523.70779437,46.10829821)(523.70779053,45.99830373)
\curveto(523.72779435,45.97829834)(523.73779434,45.94329837)(523.73779053,45.89330373)
\curveto(523.73779434,45.84329847)(523.74779433,45.79829852)(523.76779053,45.75830373)
\curveto(523.78779429,45.67829864)(523.80779427,45.58829873)(523.82779053,45.48830373)
\lineto(523.88779053,45.18830373)
\curveto(523.88779419,45.15829916)(523.89279418,45.12329919)(523.90279053,45.08330373)
\lineto(523.90279053,44.97830373)
\curveto(523.94279413,44.82829949)(523.96779411,44.66329965)(523.97779053,44.48330373)
\curveto(523.9777941,44.3133)(523.99779408,44.15330016)(524.03779053,44.00330373)
\curveto(524.05779402,43.92330039)(524.077794,43.84830047)(524.09779053,43.77830373)
\curveto(524.10779397,43.7183006)(524.12279395,43.64830067)(524.14279053,43.56830373)
\curveto(524.19279388,43.40830091)(524.25779382,43.25830106)(524.33779053,43.11830373)
\curveto(524.40779367,42.97830134)(524.49779358,42.85830146)(524.60779053,42.75830373)
\curveto(524.71779336,42.65830166)(524.85279322,42.58330173)(525.01279053,42.53330373)
\curveto(525.16279291,42.48330183)(525.34779273,42.46330185)(525.56779053,42.47330373)
\curveto(525.66779241,42.47330184)(525.76279231,42.48830183)(525.85279053,42.51830373)
\curveto(525.93279214,42.55830176)(526.00779207,42.60330171)(526.07779053,42.65330373)
\curveto(526.18779189,42.73330158)(526.28279179,42.83830148)(526.36279053,42.96830373)
\curveto(526.43279164,43.09830122)(526.49279158,43.23830108)(526.54279053,43.38830373)
\curveto(526.55279152,43.43830088)(526.55779152,43.48830083)(526.55779053,43.53830373)
\curveto(526.55779152,43.58830073)(526.56279151,43.63830068)(526.57279053,43.68830373)
\curveto(526.59279148,43.75830056)(526.60779147,43.84330047)(526.61779053,43.94330373)
\curveto(526.61779146,44.05330026)(526.60779147,44.14330017)(526.58779053,44.21330373)
\curveto(526.56779151,44.27330004)(526.56279151,44.33329998)(526.57279053,44.39330373)
\curveto(526.5727915,44.45329986)(526.56279151,44.5132998)(526.54279053,44.57330373)
\curveto(526.52279155,44.65329966)(526.50779157,44.72829959)(526.49779053,44.79830373)
\curveto(526.48779159,44.87829944)(526.46779161,44.95329936)(526.43779053,45.02330373)
\curveto(526.31779176,45.313299)(526.1727919,45.55829876)(526.00279053,45.75830373)
\curveto(525.83279224,45.96829835)(525.60279247,46.12829819)(525.31279053,46.23830373)
}
}
{
\newrgbcolor{curcolor}{0 0 0}
\pscustom[linestyle=none,fillstyle=solid,fillcolor=curcolor]
{
\newpath
\moveto(519.80779053,49.26994436)
\lineto(519.80779053,49.71994436)
\curveto(519.79779828,49.88994311)(519.81779826,50.01494298)(519.86779053,50.09494436)
\curveto(519.91779816,50.17494282)(519.98279809,50.22994277)(520.06279053,50.25994436)
\curveto(520.14279793,50.2999427)(520.22779785,50.33994266)(520.31779053,50.37994436)
\curveto(520.44779763,50.42994257)(520.5777975,50.47494252)(520.70779053,50.51494436)
\curveto(520.83779724,50.55494244)(520.96779711,50.5999424)(521.09779053,50.64994436)
\curveto(521.21779686,50.6999423)(521.34279673,50.74494225)(521.47279053,50.78494436)
\curveto(521.59279648,50.82494217)(521.71279636,50.86994213)(521.83279053,50.91994436)
\curveto(521.94279613,50.96994203)(522.05779602,51.00994199)(522.17779053,51.03994436)
\curveto(522.29779578,51.06994193)(522.41779566,51.10994189)(522.53779053,51.15994436)
\curveto(522.82779525,51.27994172)(523.12779495,51.38994161)(523.43779053,51.48994436)
\curveto(523.74779433,51.58994141)(524.04779403,51.6999413)(524.33779053,51.81994436)
\curveto(524.3777937,51.83994116)(524.41779366,51.84994115)(524.45779053,51.84994436)
\curveto(524.48779359,51.84994115)(524.51779356,51.85994114)(524.54779053,51.87994436)
\curveto(524.68779339,51.93994106)(524.83279324,51.994941)(524.98279053,52.04494436)
\lineto(525.40279053,52.19494436)
\curveto(525.4727926,52.22494077)(525.54779253,52.25494074)(525.62779053,52.28494436)
\curveto(525.69779238,52.31494068)(525.74279233,52.36494063)(525.76279053,52.43494436)
\curveto(525.79279228,52.51494048)(525.76779231,52.57494042)(525.68779053,52.61494436)
\curveto(525.59779248,52.66494033)(525.52779255,52.6999403)(525.47779053,52.71994436)
\curveto(525.30779277,52.7999402)(525.12779295,52.86494013)(524.93779053,52.91494436)
\curveto(524.74779333,52.96494003)(524.56279351,53.02493997)(524.38279053,53.09494436)
\curveto(524.15279392,53.18493981)(523.92279415,53.26493973)(523.69279053,53.33494436)
\curveto(523.45279462,53.40493959)(523.22279485,53.48993951)(523.00279053,53.58994436)
\curveto(522.95279512,53.5999394)(522.88779519,53.61493938)(522.80779053,53.63494436)
\curveto(522.71779536,53.67493932)(522.62779545,53.70993929)(522.53779053,53.73994436)
\curveto(522.43779564,53.76993923)(522.34779573,53.7999392)(522.26779053,53.82994436)
\curveto(522.21779586,53.84993915)(522.1727959,53.86493913)(522.13279053,53.87494436)
\curveto(522.09279598,53.88493911)(522.04779603,53.8999391)(521.99779053,53.91994436)
\curveto(521.8777962,53.96993903)(521.75779632,54.00993899)(521.63779053,54.03994436)
\curveto(521.50779657,54.07993892)(521.38279669,54.12493887)(521.26279053,54.17494436)
\curveto(521.21279686,54.1949388)(521.16779691,54.20993879)(521.12779053,54.21994436)
\curveto(521.08779699,54.22993877)(521.04279703,54.24493875)(520.99279053,54.26494436)
\curveto(520.90279717,54.30493869)(520.81279726,54.33993866)(520.72279053,54.36994436)
\curveto(520.62279745,54.3999386)(520.52779755,54.42993857)(520.43779053,54.45994436)
\curveto(520.35779772,54.48993851)(520.2777978,54.51493848)(520.19779053,54.53494436)
\curveto(520.10779797,54.56493843)(520.03279804,54.60493839)(519.97279053,54.65494436)
\curveto(519.88279819,54.72493827)(519.83279824,54.81993818)(519.82279053,54.93994436)
\curveto(519.81279826,55.06993793)(519.80779827,55.20993779)(519.80779053,55.35994436)
\curveto(519.80779827,55.43993756)(519.81279826,55.51493748)(519.82279053,55.58494436)
\curveto(519.82279825,55.66493733)(519.83779824,55.72993727)(519.86779053,55.77994436)
\curveto(519.92779815,55.86993713)(520.02279805,55.8949371)(520.15279053,55.85494436)
\curveto(520.28279779,55.81493718)(520.38279769,55.77993722)(520.45279053,55.74994436)
\lineto(520.51279053,55.71994436)
\curveto(520.53279754,55.71993728)(520.55279752,55.71493728)(520.57279053,55.70494436)
\curveto(520.85279722,55.5949374)(521.13779694,55.48493751)(521.42779053,55.37494436)
\lineto(522.26779053,55.04494436)
\curveto(522.34779573,55.01493798)(522.42279565,54.98993801)(522.49279053,54.96994436)
\curveto(522.55279552,54.94993805)(522.61779546,54.92493807)(522.68779053,54.89494436)
\curveto(522.88779519,54.81493818)(523.09279498,54.73493826)(523.30279053,54.65494436)
\curveto(523.50279457,54.58493841)(523.70279437,54.50993849)(523.90279053,54.42994436)
\curveto(524.59279348,54.13993886)(525.28779279,53.86993913)(525.98779053,53.61994436)
\curveto(526.68779139,53.36993963)(527.38279069,53.0999399)(528.07279053,52.80994436)
\lineto(528.22279053,52.74994436)
\curveto(528.28278979,52.73994026)(528.34278973,52.72494027)(528.40279053,52.70494436)
\curveto(528.7727893,52.54494045)(529.13778894,52.37494062)(529.49779053,52.19494436)
\curveto(529.86778821,52.01494098)(530.15278792,51.76494123)(530.35279053,51.44494436)
\curveto(530.41278766,51.33494166)(530.45778762,51.22494177)(530.48779053,51.11494436)
\curveto(530.52778755,51.00494199)(530.56278751,50.87994212)(530.59279053,50.73994436)
\curveto(530.61278746,50.68994231)(530.61778746,50.63494236)(530.60779053,50.57494436)
\curveto(530.59778748,50.52494247)(530.59778748,50.46994253)(530.60779053,50.40994436)
\curveto(530.62778745,50.32994267)(530.62778745,50.24994275)(530.60779053,50.16994436)
\curveto(530.59778748,50.12994287)(530.59278748,50.07994292)(530.59279053,50.01994436)
\lineto(530.53279053,49.77994436)
\curveto(530.51278756,49.70994329)(530.4727876,49.65494334)(530.41279053,49.61494436)
\curveto(530.35278772,49.56494343)(530.2777878,49.53494346)(530.18779053,49.52494436)
\lineto(529.91779053,49.52494436)
\lineto(529.70779053,49.52494436)
\curveto(529.64778843,49.53494346)(529.59778848,49.55494344)(529.55779053,49.58494436)
\curveto(529.44778863,49.65494334)(529.41778866,49.77494322)(529.46779053,49.94494436)
\curveto(529.48778859,50.05494294)(529.49778858,50.17494282)(529.49779053,50.30494436)
\curveto(529.49778858,50.43494256)(529.4777886,50.54994245)(529.43779053,50.64994436)
\curveto(529.38778869,50.7999422)(529.31278876,50.91994208)(529.21279053,51.00994436)
\curveto(529.11278896,51.10994189)(528.99778908,51.1949418)(528.86779053,51.26494436)
\curveto(528.74778933,51.33494166)(528.61778946,51.3949416)(528.47779053,51.44494436)
\lineto(528.05779053,51.62494436)
\curveto(527.96779011,51.66494133)(527.85779022,51.70494129)(527.72779053,51.74494436)
\curveto(527.59779048,51.7949412)(527.46279061,51.7999412)(527.32279053,51.75994436)
\curveto(527.16279091,51.70994129)(527.01279106,51.65494134)(526.87279053,51.59494436)
\curveto(526.73279134,51.54494145)(526.59279148,51.48994151)(526.45279053,51.42994436)
\curveto(526.24279183,51.33994166)(526.03279204,51.25494174)(525.82279053,51.17494436)
\curveto(525.61279246,51.0949419)(525.40779267,51.01494198)(525.20779053,50.93494436)
\curveto(525.06779301,50.87494212)(524.93279314,50.81994218)(524.80279053,50.76994436)
\curveto(524.6727934,50.71994228)(524.53779354,50.66994233)(524.39779053,50.61994436)
\lineto(523.07779053,50.07994436)
\curveto(522.63779544,49.90994309)(522.19779588,49.73494326)(521.75779053,49.55494436)
\curveto(521.52779655,49.45494354)(521.30779677,49.36494363)(521.09779053,49.28494436)
\curveto(520.8777972,49.20494379)(520.65779742,49.11994388)(520.43779053,49.02994436)
\curveto(520.3777977,49.00994399)(520.29779778,48.97994402)(520.19779053,48.93994436)
\curveto(520.08779799,48.8999441)(519.99779808,48.90494409)(519.92779053,48.95494436)
\curveto(519.8777982,48.98494401)(519.84279823,49.04494395)(519.82279053,49.13494436)
\curveto(519.81279826,49.15494384)(519.81279826,49.17494382)(519.82279053,49.19494436)
\curveto(519.82279825,49.22494377)(519.81779826,49.24994375)(519.80779053,49.26994436)
}
}
{
\newrgbcolor{curcolor}{0 0 0}
\pscustom[linestyle=none,fillstyle=solid,fillcolor=curcolor]
{
}
}
{
\newrgbcolor{curcolor}{0 0 0}
\pscustom[linestyle=none,fillstyle=solid,fillcolor=curcolor]
{
\newpath
\moveto(522.44779053,67.99510061)
\lineto(522.70279053,67.99510061)
\curveto(522.78279529,68.0050929)(522.85779522,68.00009291)(522.92779053,67.98010061)
\lineto(523.16779053,67.98010061)
\lineto(523.33279053,67.98010061)
\curveto(523.43279464,67.96009295)(523.53779454,67.95009296)(523.64779053,67.95010061)
\curveto(523.74779433,67.95009296)(523.84779423,67.94009297)(523.94779053,67.92010061)
\lineto(524.09779053,67.92010061)
\curveto(524.23779384,67.89009302)(524.3777937,67.87009304)(524.51779053,67.86010061)
\curveto(524.64779343,67.85009306)(524.7777933,67.82509308)(524.90779053,67.78510061)
\curveto(524.98779309,67.76509314)(525.072793,67.74509316)(525.16279053,67.72510061)
\lineto(525.40279053,67.66510061)
\lineto(525.70279053,67.54510061)
\curveto(525.79279228,67.51509339)(525.88279219,67.48009343)(525.97279053,67.44010061)
\curveto(526.19279188,67.34009357)(526.40779167,67.2050937)(526.61779053,67.03510061)
\curveto(526.82779125,66.87509403)(526.99779108,66.70009421)(527.12779053,66.51010061)
\curveto(527.16779091,66.46009445)(527.20779087,66.40009451)(527.24779053,66.33010061)
\curveto(527.2777908,66.27009464)(527.31279076,66.2100947)(527.35279053,66.15010061)
\curveto(527.40279067,66.07009484)(527.44279063,65.97509493)(527.47279053,65.86510061)
\curveto(527.50279057,65.75509515)(527.53279054,65.65009526)(527.56279053,65.55010061)
\curveto(527.60279047,65.44009547)(527.62779045,65.33009558)(527.63779053,65.22010061)
\curveto(527.64779043,65.1100958)(527.66279041,64.99509591)(527.68279053,64.87510061)
\curveto(527.69279038,64.83509607)(527.69279038,64.79009612)(527.68279053,64.74010061)
\curveto(527.68279039,64.70009621)(527.68779039,64.66009625)(527.69779053,64.62010061)
\curveto(527.70779037,64.58009633)(527.71279036,64.52509638)(527.71279053,64.45510061)
\curveto(527.71279036,64.38509652)(527.70779037,64.33509657)(527.69779053,64.30510061)
\curveto(527.6777904,64.25509665)(527.6727904,64.2100967)(527.68279053,64.17010061)
\curveto(527.69279038,64.13009678)(527.69279038,64.09509681)(527.68279053,64.06510061)
\lineto(527.68279053,63.97510061)
\curveto(527.66279041,63.91509699)(527.64779043,63.85009706)(527.63779053,63.78010061)
\curveto(527.63779044,63.72009719)(527.63279044,63.65509725)(527.62279053,63.58510061)
\curveto(527.5727905,63.41509749)(527.52279055,63.25509765)(527.47279053,63.10510061)
\curveto(527.42279065,62.95509795)(527.35779072,62.8100981)(527.27779053,62.67010061)
\curveto(527.23779084,62.62009829)(527.20779087,62.56509834)(527.18779053,62.50510061)
\curveto(527.15779092,62.45509845)(527.12279095,62.4050985)(527.08279053,62.35510061)
\curveto(526.90279117,62.11509879)(526.68279139,61.91509899)(526.42279053,61.75510061)
\curveto(526.16279191,61.59509931)(525.8777922,61.45509945)(525.56779053,61.33510061)
\curveto(525.42779265,61.27509963)(525.28779279,61.23009968)(525.14779053,61.20010061)
\curveto(524.99779308,61.17009974)(524.84279323,61.13509977)(524.68279053,61.09510061)
\curveto(524.5727935,61.07509983)(524.46279361,61.06009985)(524.35279053,61.05010061)
\curveto(524.24279383,61.04009987)(524.13279394,61.02509988)(524.02279053,61.00510061)
\curveto(523.98279409,60.99509991)(523.94279413,60.99009992)(523.90279053,60.99010061)
\curveto(523.86279421,61.00009991)(523.82279425,61.00009991)(523.78279053,60.99010061)
\curveto(523.73279434,60.98009993)(523.68279439,60.97509993)(523.63279053,60.97510061)
\lineto(523.46779053,60.97510061)
\curveto(523.41779466,60.95509995)(523.36779471,60.95009996)(523.31779053,60.96010061)
\curveto(523.25779482,60.97009994)(523.20279487,60.97009994)(523.15279053,60.96010061)
\curveto(523.11279496,60.95009996)(523.06779501,60.95009996)(523.01779053,60.96010061)
\curveto(522.96779511,60.97009994)(522.91779516,60.96509994)(522.86779053,60.94510061)
\curveto(522.79779528,60.92509998)(522.72279535,60.92009999)(522.64279053,60.93010061)
\curveto(522.55279552,60.94009997)(522.46779561,60.94509996)(522.38779053,60.94510061)
\curveto(522.29779578,60.94509996)(522.19779588,60.94009997)(522.08779053,60.93010061)
\curveto(521.96779611,60.92009999)(521.86779621,60.92509998)(521.78779053,60.94510061)
\lineto(521.50279053,60.94510061)
\lineto(520.87279053,60.99010061)
\curveto(520.7727973,61.00009991)(520.6777974,61.0100999)(520.58779053,61.02010061)
\lineto(520.28779053,61.05010061)
\curveto(520.23779784,61.07009984)(520.18779789,61.07509983)(520.13779053,61.06510061)
\curveto(520.077798,61.06509984)(520.02279805,61.07509983)(519.97279053,61.09510061)
\curveto(519.80279827,61.14509976)(519.63779844,61.18509972)(519.47779053,61.21510061)
\curveto(519.30779877,61.24509966)(519.14779893,61.29509961)(518.99779053,61.36510061)
\curveto(518.53779954,61.55509935)(518.16279991,61.77509913)(517.87279053,62.02510061)
\curveto(517.58280049,62.28509862)(517.33780074,62.64509826)(517.13779053,63.10510061)
\curveto(517.08780099,63.23509767)(517.05280102,63.36509754)(517.03279053,63.49510061)
\curveto(517.01280106,63.63509727)(516.98780109,63.77509713)(516.95779053,63.91510061)
\curveto(516.94780113,63.98509692)(516.94280113,64.05009686)(516.94279053,64.11010061)
\curveto(516.94280113,64.17009674)(516.93780114,64.23509667)(516.92779053,64.30510061)
\curveto(516.90780117,65.13509577)(517.05780102,65.8050951)(517.37779053,66.31510061)
\curveto(517.68780039,66.82509408)(518.12779995,67.2050937)(518.69779053,67.45510061)
\curveto(518.81779926,67.5050934)(518.94279913,67.55009336)(519.07279053,67.59010061)
\curveto(519.20279887,67.63009328)(519.33779874,67.67509323)(519.47779053,67.72510061)
\curveto(519.55779852,67.74509316)(519.64279843,67.76009315)(519.73279053,67.77010061)
\lineto(519.97279053,67.83010061)
\curveto(520.08279799,67.86009305)(520.19279788,67.87509303)(520.30279053,67.87510061)
\curveto(520.41279766,67.88509302)(520.52279755,67.90009301)(520.63279053,67.92010061)
\curveto(520.68279739,67.94009297)(520.72779735,67.94509296)(520.76779053,67.93510061)
\curveto(520.80779727,67.93509297)(520.84779723,67.94009297)(520.88779053,67.95010061)
\curveto(520.93779714,67.96009295)(520.99279708,67.96009295)(521.05279053,67.95010061)
\curveto(521.10279697,67.95009296)(521.15279692,67.95509295)(521.20279053,67.96510061)
\lineto(521.33779053,67.96510061)
\curveto(521.39779668,67.98509292)(521.46779661,67.98509292)(521.54779053,67.96510061)
\curveto(521.61779646,67.95509295)(521.68279639,67.96009295)(521.74279053,67.98010061)
\curveto(521.7727963,67.99009292)(521.81279626,67.99509291)(521.86279053,67.99510061)
\lineto(521.98279053,67.99510061)
\lineto(522.44779053,67.99510061)
\moveto(524.77279053,66.45010061)
\curveto(524.45279362,66.55009436)(524.08779399,66.6100943)(523.67779053,66.63010061)
\curveto(523.26779481,66.65009426)(522.85779522,66.66009425)(522.44779053,66.66010061)
\curveto(522.01779606,66.66009425)(521.59779648,66.65009426)(521.18779053,66.63010061)
\curveto(520.7777973,66.6100943)(520.39279768,66.56509434)(520.03279053,66.49510061)
\curveto(519.6727984,66.42509448)(519.35279872,66.31509459)(519.07279053,66.16510061)
\curveto(518.78279929,66.02509488)(518.54779953,65.83009508)(518.36779053,65.58010061)
\curveto(518.25779982,65.42009549)(518.1777999,65.24009567)(518.12779053,65.04010061)
\curveto(518.06780001,64.84009607)(518.03780004,64.59509631)(518.03779053,64.30510061)
\curveto(518.05780002,64.28509662)(518.06780001,64.25009666)(518.06779053,64.20010061)
\curveto(518.05780002,64.15009676)(518.05780002,64.1100968)(518.06779053,64.08010061)
\curveto(518.08779999,64.00009691)(518.10779997,63.92509698)(518.12779053,63.85510061)
\curveto(518.13779994,63.79509711)(518.15779992,63.73009718)(518.18779053,63.66010061)
\curveto(518.30779977,63.39009752)(518.4777996,63.17009774)(518.69779053,63.00010061)
\curveto(518.90779917,62.84009807)(519.15279892,62.7050982)(519.43279053,62.59510061)
\curveto(519.54279853,62.54509836)(519.66279841,62.5050984)(519.79279053,62.47510061)
\curveto(519.91279816,62.45509845)(520.03779804,62.43009848)(520.16779053,62.40010061)
\curveto(520.21779786,62.38009853)(520.2727978,62.37009854)(520.33279053,62.37010061)
\curveto(520.38279769,62.37009854)(520.43279764,62.36509854)(520.48279053,62.35510061)
\curveto(520.5727975,62.34509856)(520.66779741,62.33509857)(520.76779053,62.32510061)
\curveto(520.85779722,62.31509859)(520.95279712,62.3050986)(521.05279053,62.29510061)
\curveto(521.13279694,62.29509861)(521.21779686,62.29009862)(521.30779053,62.28010061)
\lineto(521.54779053,62.28010061)
\lineto(521.72779053,62.28010061)
\curveto(521.75779632,62.27009864)(521.79279628,62.26509864)(521.83279053,62.26510061)
\lineto(521.96779053,62.26510061)
\lineto(522.41779053,62.26510061)
\curveto(522.49779558,62.26509864)(522.58279549,62.26009865)(522.67279053,62.25010061)
\curveto(522.75279532,62.25009866)(522.82779525,62.26009865)(522.89779053,62.28010061)
\lineto(523.16779053,62.28010061)
\curveto(523.18779489,62.28009863)(523.21779486,62.27509863)(523.25779053,62.26510061)
\curveto(523.28779479,62.26509864)(523.31279476,62.27009864)(523.33279053,62.28010061)
\curveto(523.43279464,62.29009862)(523.53279454,62.29509861)(523.63279053,62.29510061)
\curveto(523.72279435,62.3050986)(523.82279425,62.31509859)(523.93279053,62.32510061)
\curveto(524.05279402,62.35509855)(524.1777939,62.37009854)(524.30779053,62.37010061)
\curveto(524.42779365,62.38009853)(524.54279353,62.4050985)(524.65279053,62.44510061)
\curveto(524.95279312,62.52509838)(525.21779286,62.6100983)(525.44779053,62.70010061)
\curveto(525.6777924,62.80009811)(525.89279218,62.94509796)(526.09279053,63.13510061)
\curveto(526.29279178,63.34509756)(526.44279163,63.6100973)(526.54279053,63.93010061)
\curveto(526.56279151,63.97009694)(526.5727915,64.0050969)(526.57279053,64.03510061)
\curveto(526.56279151,64.07509683)(526.56779151,64.12009679)(526.58779053,64.17010061)
\curveto(526.59779148,64.2100967)(526.60779147,64.28009663)(526.61779053,64.38010061)
\curveto(526.62779145,64.49009642)(526.62279145,64.57509633)(526.60279053,64.63510061)
\curveto(526.58279149,64.7050962)(526.5727915,64.77509613)(526.57279053,64.84510061)
\curveto(526.56279151,64.91509599)(526.54779153,64.98009593)(526.52779053,65.04010061)
\curveto(526.46779161,65.24009567)(526.38279169,65.42009549)(526.27279053,65.58010061)
\curveto(526.25279182,65.6100953)(526.23279184,65.63509527)(526.21279053,65.65510061)
\lineto(526.15279053,65.71510061)
\curveto(526.13279194,65.75509515)(526.09279198,65.8050951)(526.03279053,65.86510061)
\curveto(525.89279218,65.96509494)(525.76279231,66.05009486)(525.64279053,66.12010061)
\curveto(525.52279255,66.19009472)(525.3777927,66.26009465)(525.20779053,66.33010061)
\curveto(525.13779294,66.36009455)(525.06779301,66.38009453)(524.99779053,66.39010061)
\curveto(524.92779315,66.4100945)(524.85279322,66.43009448)(524.77279053,66.45010061)
}
}
{
\newrgbcolor{curcolor}{0 0 0}
\pscustom[linestyle=none,fillstyle=solid,fillcolor=curcolor]
{
\newpath
\moveto(521.93779053,76.28470998)
\curveto(522.01779606,76.28470235)(522.09779598,76.28970234)(522.17779053,76.29970998)
\curveto(522.25779582,76.30970232)(522.33279574,76.30470233)(522.40279053,76.28470998)
\curveto(522.44279563,76.26470237)(522.48779559,76.25970237)(522.53779053,76.26970998)
\curveto(522.5777955,76.27970235)(522.61779546,76.27970235)(522.65779053,76.26970998)
\lineto(522.80779053,76.26970998)
\curveto(522.89779518,76.25970237)(522.98779509,76.25470238)(523.07779053,76.25470998)
\curveto(523.15779492,76.25470238)(523.23779484,76.24970238)(523.31779053,76.23970998)
\lineto(523.55779053,76.20970998)
\curveto(523.62779445,76.19970243)(523.70279437,76.18970244)(523.78279053,76.17970998)
\curveto(523.82279425,76.16970246)(523.86279421,76.16470247)(523.90279053,76.16470998)
\curveto(523.94279413,76.16470247)(523.98779409,76.15970247)(524.03779053,76.14970998)
\curveto(524.1777939,76.10970252)(524.31779376,76.07970255)(524.45779053,76.05970998)
\curveto(524.59779348,76.04970258)(524.73279334,76.01970261)(524.86279053,75.96970998)
\curveto(525.03279304,75.91970271)(525.19779288,75.86470277)(525.35779053,75.80470998)
\curveto(525.51779256,75.75470288)(525.6727924,75.69470294)(525.82279053,75.62470998)
\curveto(525.88279219,75.60470303)(525.94279213,75.57470306)(526.00279053,75.53470998)
\lineto(526.15279053,75.44470998)
\curveto(526.4727916,75.24470339)(526.73779134,75.0297036)(526.94779053,74.79970998)
\curveto(527.15779092,74.56970406)(527.33779074,74.27470436)(527.48779053,73.91470998)
\curveto(527.53779054,73.79470484)(527.5727905,73.66470497)(527.59279053,73.52470998)
\curveto(527.61279046,73.39470524)(527.63779044,73.25970537)(527.66779053,73.11970998)
\curveto(527.6777904,73.05970557)(527.68279039,72.99970563)(527.68279053,72.93970998)
\curveto(527.68279039,72.87970575)(527.68779039,72.81470582)(527.69779053,72.74470998)
\curveto(527.70779037,72.71470592)(527.70779037,72.66470597)(527.69779053,72.59470998)
\lineto(527.69779053,72.44470998)
\lineto(527.69779053,72.29470998)
\curveto(527.6777904,72.21470642)(527.66279041,72.1297065)(527.65279053,72.03970998)
\curveto(527.65279042,71.95970667)(527.64279043,71.88470675)(527.62279053,71.81470998)
\curveto(527.61279046,71.77470686)(527.60779047,71.73970689)(527.60779053,71.70970998)
\curveto(527.61779046,71.68970694)(527.61279046,71.66470697)(527.59279053,71.63470998)
\lineto(527.53279053,71.36470998)
\curveto(527.50279057,71.27470736)(527.4727906,71.18970744)(527.44279053,71.10970998)
\curveto(527.20279087,70.5297081)(526.83279124,70.09470854)(526.33279053,69.80470998)
\curveto(526.20279187,69.72470891)(526.06779201,69.65970897)(525.92779053,69.60970998)
\curveto(525.78779229,69.56970906)(525.63779244,69.52470911)(525.47779053,69.47470998)
\curveto(525.39779268,69.45470918)(525.31779276,69.44970918)(525.23779053,69.45970998)
\curveto(525.15779292,69.47970915)(525.10279297,69.51470912)(525.07279053,69.56470998)
\curveto(525.05279302,69.59470904)(525.03779304,69.64970898)(525.02779053,69.72970998)
\curveto(525.00779307,69.80970882)(524.99779308,69.89470874)(524.99779053,69.98470998)
\curveto(524.98779309,70.07470856)(524.98779309,70.15970847)(524.99779053,70.23970998)
\curveto(525.00779307,70.3297083)(525.01779306,70.39970823)(525.02779053,70.44970998)
\curveto(525.03779304,70.46970816)(525.05279302,70.49470814)(525.07279053,70.52470998)
\curveto(525.09279298,70.56470807)(525.11279296,70.59470804)(525.13279053,70.61470998)
\curveto(525.21279286,70.67470796)(525.30779277,70.71970791)(525.41779053,70.74970998)
\curveto(525.52779255,70.78970784)(525.62779245,70.8347078)(525.71779053,70.88470998)
\curveto(526.10779197,71.1347075)(526.3777917,71.50470713)(526.52779053,71.99470998)
\curveto(526.54779153,72.06470657)(526.56279151,72.1347065)(526.57279053,72.20470998)
\curveto(526.5727915,72.28470635)(526.58279149,72.36470627)(526.60279053,72.44470998)
\curveto(526.61279146,72.48470615)(526.61779146,72.53970609)(526.61779053,72.60970998)
\curveto(526.61779146,72.68970594)(526.61279146,72.74470589)(526.60279053,72.77470998)
\curveto(526.59279148,72.80470583)(526.58779149,72.8347058)(526.58779053,72.86470998)
\lineto(526.58779053,72.96970998)
\curveto(526.56779151,73.04970558)(526.54779153,73.12470551)(526.52779053,73.19470998)
\curveto(526.50779157,73.27470536)(526.48279159,73.34970528)(526.45279053,73.41970998)
\curveto(526.30279177,73.76970486)(526.08779199,74.03970459)(525.80779053,74.22970998)
\curveto(525.52779255,74.41970421)(525.20279287,74.57470406)(524.83279053,74.69470998)
\curveto(524.75279332,74.72470391)(524.6777934,74.74470389)(524.60779053,74.75470998)
\curveto(524.53779354,74.77470386)(524.46279361,74.79470384)(524.38279053,74.81470998)
\curveto(524.29279378,74.8347038)(524.19779388,74.84970378)(524.09779053,74.85970998)
\curveto(523.98779409,74.87970375)(523.88279419,74.89970373)(523.78279053,74.91970998)
\curveto(523.73279434,74.9297037)(523.68279439,74.9347037)(523.63279053,74.93470998)
\curveto(523.5727945,74.94470369)(523.51779456,74.94970368)(523.46779053,74.94970998)
\curveto(523.40779467,74.96970366)(523.33279474,74.97970365)(523.24279053,74.97970998)
\curveto(523.14279493,74.97970365)(523.06279501,74.96970366)(523.00279053,74.94970998)
\curveto(522.91279516,74.91970371)(522.8727952,74.86970376)(522.88279053,74.79970998)
\curveto(522.89279518,74.73970389)(522.92279515,74.68470395)(522.97279053,74.63470998)
\curveto(523.02279505,74.55470408)(523.08279499,74.48470415)(523.15279053,74.42470998)
\curveto(523.22279485,74.37470426)(523.28279479,74.30970432)(523.33279053,74.22970998)
\curveto(523.44279463,74.06970456)(523.54279453,73.90470473)(523.63279053,73.73470998)
\curveto(523.71279436,73.56470507)(523.78279429,73.36970526)(523.84279053,73.14970998)
\curveto(523.8727942,73.04970558)(523.88779419,72.94970568)(523.88779053,72.84970998)
\curveto(523.88779419,72.75970587)(523.89779418,72.65970597)(523.91779053,72.54970998)
\lineto(523.91779053,72.39970998)
\curveto(523.89779418,72.34970628)(523.89279418,72.29970633)(523.90279053,72.24970998)
\curveto(523.91279416,72.20970642)(523.91279416,72.16970646)(523.90279053,72.12970998)
\curveto(523.89279418,72.09970653)(523.88779419,72.05470658)(523.88779053,71.99470998)
\curveto(523.8777942,71.9347067)(523.86779421,71.86970676)(523.85779053,71.79970998)
\lineto(523.82779053,71.61970998)
\curveto(523.70779437,71.16970746)(523.54279453,70.78970784)(523.33279053,70.47970998)
\curveto(523.14279493,70.20970842)(522.91279516,69.97970865)(522.64279053,69.78970998)
\curveto(522.36279571,69.60970902)(522.04779603,69.46470917)(521.69779053,69.35470998)
\lineto(521.48779053,69.29470998)
\curveto(521.40779667,69.28470935)(521.32779675,69.26970936)(521.24779053,69.24970998)
\curveto(521.21779686,69.23970939)(521.18779689,69.2347094)(521.15779053,69.23470998)
\curveto(521.12779695,69.2347094)(521.09779698,69.2297094)(521.06779053,69.21970998)
\curveto(521.00779707,69.20970942)(520.94779713,69.20470943)(520.88779053,69.20470998)
\curveto(520.81779726,69.20470943)(520.75779732,69.19470944)(520.70779053,69.17470998)
\lineto(520.52779053,69.17470998)
\curveto(520.4777976,69.16470947)(520.40779767,69.15970947)(520.31779053,69.15970998)
\curveto(520.22779785,69.15970947)(520.15779792,69.16970946)(520.10779053,69.18970998)
\lineto(519.94279053,69.18970998)
\curveto(519.86279821,69.20970942)(519.78779829,69.21970941)(519.71779053,69.21970998)
\curveto(519.64779843,69.2297094)(519.5777985,69.24470939)(519.50779053,69.26470998)
\curveto(519.30779877,69.32470931)(519.11779896,69.38470925)(518.93779053,69.44470998)
\curveto(518.75779932,69.51470912)(518.58779949,69.60470903)(518.42779053,69.71470998)
\curveto(518.35779972,69.75470888)(518.29279978,69.79470884)(518.23279053,69.83470998)
\lineto(518.05279053,69.98470998)
\curveto(518.04280003,70.00470863)(518.02780005,70.02470861)(518.00779053,70.04470998)
\curveto(517.8778002,70.1347085)(517.76780031,70.24470839)(517.67779053,70.37470998)
\curveto(517.4778006,70.634708)(517.32280075,70.89970773)(517.21279053,71.16970998)
\curveto(517.1728009,71.24970738)(517.14280093,71.3297073)(517.12279053,71.40970998)
\curveto(517.09280098,71.49970713)(517.06780101,71.58970704)(517.04779053,71.67970998)
\curveto(517.01780106,71.77970685)(516.99780108,71.87970675)(516.98779053,71.97970998)
\curveto(516.9778011,72.07970655)(516.96280111,72.18470645)(516.94279053,72.29470998)
\curveto(516.93280114,72.32470631)(516.93280114,72.36470627)(516.94279053,72.41470998)
\curveto(516.95280112,72.47470616)(516.94780113,72.51470612)(516.92779053,72.53470998)
\curveto(516.90780117,73.25470538)(517.02280105,73.85470478)(517.27279053,74.33470998)
\curveto(517.52280055,74.81470382)(517.86280021,75.18970344)(518.29279053,75.45970998)
\curveto(518.43279964,75.54970308)(518.5777995,75.629703)(518.72779053,75.69970998)
\curveto(518.8777992,75.76970286)(519.03779904,75.83970279)(519.20779053,75.90970998)
\curveto(519.34779873,75.95970267)(519.49779858,75.99970263)(519.65779053,76.02970998)
\curveto(519.81779826,76.05970257)(519.9777981,76.09470254)(520.13779053,76.13470998)
\curveto(520.18779789,76.15470248)(520.24279783,76.16470247)(520.30279053,76.16470998)
\curveto(520.35279772,76.16470247)(520.40279767,76.16970246)(520.45279053,76.17970998)
\curveto(520.51279756,76.19970243)(520.5777975,76.20970242)(520.64779053,76.20970998)
\curveto(520.70779737,76.20970242)(520.76279731,76.21970241)(520.81279053,76.23970998)
\lineto(520.97779053,76.23970998)
\curveto(521.02779705,76.25970237)(521.077797,76.26470237)(521.12779053,76.25470998)
\curveto(521.1777969,76.24470239)(521.22779685,76.24970238)(521.27779053,76.26970998)
\curveto(521.29779678,76.26970236)(521.32279675,76.26470237)(521.35279053,76.25470998)
\curveto(521.38279669,76.25470238)(521.40779667,76.25970237)(521.42779053,76.26970998)
\curveto(521.45779662,76.27970235)(521.49279658,76.27970235)(521.53279053,76.26970998)
\curveto(521.5727965,76.26970236)(521.61279646,76.27470236)(521.65279053,76.28470998)
\curveto(521.69279638,76.29470234)(521.73779634,76.29470234)(521.78779053,76.28470998)
\lineto(521.93779053,76.28470998)
\moveto(520.63279053,74.78470998)
\curveto(520.58279749,74.79470384)(520.52279755,74.79970383)(520.45279053,74.79970998)
\curveto(520.38279769,74.79970383)(520.32279775,74.79470384)(520.27279053,74.78470998)
\curveto(520.22279785,74.77470386)(520.14779793,74.76970386)(520.04779053,74.76970998)
\curveto(519.96779811,74.74970388)(519.89279818,74.7297039)(519.82279053,74.70970998)
\curveto(519.75279832,74.69970393)(519.68279839,74.68470395)(519.61279053,74.66470998)
\curveto(519.18279889,74.52470411)(518.84779923,74.3297043)(518.60779053,74.07970998)
\curveto(518.36779971,73.83970479)(518.18779989,73.49470514)(518.06779053,73.04470998)
\curveto(518.04780003,72.95470568)(518.03780004,72.85470578)(518.03779053,72.74470998)
\lineto(518.03779053,72.41470998)
\curveto(518.05780002,72.39470624)(518.06780001,72.35970627)(518.06779053,72.30970998)
\curveto(518.05780002,72.25970637)(518.05780002,72.21470642)(518.06779053,72.17470998)
\curveto(518.08779999,72.09470654)(518.10779997,72.01970661)(518.12779053,71.94970998)
\lineto(518.18779053,71.73970998)
\curveto(518.31779976,71.44970718)(518.49779958,71.21970741)(518.72779053,71.04970998)
\curveto(518.94779913,70.87970775)(519.20779887,70.74470789)(519.50779053,70.64470998)
\curveto(519.59779848,70.61470802)(519.69279838,70.58970804)(519.79279053,70.56970998)
\curveto(519.88279819,70.55970807)(519.9777981,70.54470809)(520.07779053,70.52470998)
\lineto(520.21279053,70.52470998)
\curveto(520.32279775,70.49470814)(520.46279761,70.48470815)(520.63279053,70.49470998)
\curveto(520.79279728,70.51470812)(520.92279715,70.5347081)(521.02279053,70.55470998)
\curveto(521.08279699,70.57470806)(521.14279693,70.58970804)(521.20279053,70.59970998)
\curveto(521.25279682,70.60970802)(521.30279677,70.62470801)(521.35279053,70.64470998)
\curveto(521.55279652,70.72470791)(521.74279633,70.81970781)(521.92279053,70.92970998)
\curveto(522.10279597,71.04970758)(522.24779583,71.18970744)(522.35779053,71.34970998)
\curveto(522.40779567,71.39970723)(522.44779563,71.45470718)(522.47779053,71.51470998)
\curveto(522.50779557,71.57470706)(522.54279553,71.634707)(522.58279053,71.69470998)
\curveto(522.66279541,71.84470679)(522.72779535,72.0297066)(522.77779053,72.24970998)
\curveto(522.79779528,72.29970633)(522.80279527,72.33970629)(522.79279053,72.36970998)
\curveto(522.78279529,72.40970622)(522.78779529,72.45470618)(522.80779053,72.50470998)
\curveto(522.81779526,72.54470609)(522.82279525,72.59970603)(522.82279053,72.66970998)
\curveto(522.82279525,72.73970589)(522.81779526,72.79970583)(522.80779053,72.84970998)
\curveto(522.78779529,72.94970568)(522.7727953,73.04470559)(522.76279053,73.13470998)
\curveto(522.74279533,73.22470541)(522.71279536,73.31470532)(522.67279053,73.40470998)
\curveto(522.45279562,73.94470469)(522.05779602,74.33970429)(521.48779053,74.58970998)
\curveto(521.38779669,74.63970399)(521.28779679,74.67470396)(521.18779053,74.69470998)
\curveto(521.077797,74.71470392)(520.96779711,74.73970389)(520.85779053,74.76970998)
\curveto(520.75779732,74.76970386)(520.68279739,74.77470386)(520.63279053,74.78470998)
}
}
{
\newrgbcolor{curcolor}{0 0 0}
\pscustom[linestyle=none,fillstyle=solid,fillcolor=curcolor]
{
\newpath
\moveto(525.89779053,78.63431936)
\lineto(525.89779053,79.26431936)
\lineto(525.89779053,79.45931936)
\curveto(525.89779218,79.52931683)(525.90779217,79.58931677)(525.92779053,79.63931936)
\curveto(525.96779211,79.70931665)(526.00779207,79.7593166)(526.04779053,79.78931936)
\curveto(526.09779198,79.82931653)(526.16279191,79.84931651)(526.24279053,79.84931936)
\curveto(526.32279175,79.8593165)(526.40779167,79.86431649)(526.49779053,79.86431936)
\lineto(527.21779053,79.86431936)
\curveto(527.69779038,79.86431649)(528.10778997,79.80431655)(528.44779053,79.68431936)
\curveto(528.78778929,79.56431679)(529.06278901,79.36931699)(529.27279053,79.09931936)
\curveto(529.32278875,79.02931733)(529.36778871,78.9593174)(529.40779053,78.88931936)
\curveto(529.45778862,78.82931753)(529.50278857,78.7543176)(529.54279053,78.66431936)
\curveto(529.55278852,78.64431771)(529.56278851,78.61431774)(529.57279053,78.57431936)
\curveto(529.59278848,78.53431782)(529.59778848,78.48931787)(529.58779053,78.43931936)
\curveto(529.55778852,78.34931801)(529.48278859,78.29431806)(529.36279053,78.27431936)
\curveto(529.25278882,78.2543181)(529.15778892,78.26931809)(529.07779053,78.31931936)
\curveto(529.00778907,78.34931801)(528.94278913,78.39431796)(528.88279053,78.45431936)
\curveto(528.83278924,78.52431783)(528.78278929,78.58931777)(528.73279053,78.64931936)
\curveto(528.68278939,78.71931764)(528.60778947,78.77931758)(528.50779053,78.82931936)
\curveto(528.41778966,78.88931747)(528.32778975,78.93931742)(528.23779053,78.97931936)
\curveto(528.20778987,78.99931736)(528.14778993,79.02431733)(528.05779053,79.05431936)
\curveto(527.9777901,79.08431727)(527.90779017,79.08931727)(527.84779053,79.06931936)
\curveto(527.70779037,79.03931732)(527.61779046,78.97931738)(527.57779053,78.88931936)
\curveto(527.54779053,78.80931755)(527.53279054,78.71931764)(527.53279053,78.61931936)
\curveto(527.53279054,78.51931784)(527.50779057,78.43431792)(527.45779053,78.36431936)
\curveto(527.38779069,78.27431808)(527.24779083,78.22931813)(527.03779053,78.22931936)
\lineto(526.48279053,78.22931936)
\lineto(526.25779053,78.22931936)
\curveto(526.1777919,78.23931812)(526.11279196,78.2593181)(526.06279053,78.28931936)
\curveto(525.98279209,78.34931801)(525.93779214,78.41931794)(525.92779053,78.49931936)
\curveto(525.91779216,78.51931784)(525.91279216,78.53931782)(525.91279053,78.55931936)
\curveto(525.91279216,78.58931777)(525.90779217,78.61431774)(525.89779053,78.63431936)
}
}
{
\newrgbcolor{curcolor}{0 0 0}
\pscustom[linestyle=none,fillstyle=solid,fillcolor=curcolor]
{
}
}
{
\newrgbcolor{curcolor}{0 0 0}
\pscustom[linestyle=none,fillstyle=solid,fillcolor=curcolor]
{
\newpath
\moveto(516.92779053,89.26463186)
\curveto(516.91780116,89.95462722)(517.03780104,90.55462662)(517.28779053,91.06463186)
\curveto(517.53780054,91.58462559)(517.8728002,91.9796252)(518.29279053,92.24963186)
\curveto(518.3727997,92.29962488)(518.46279961,92.34462483)(518.56279053,92.38463186)
\curveto(518.65279942,92.42462475)(518.74779933,92.46962471)(518.84779053,92.51963186)
\curveto(518.94779913,92.55962462)(519.04779903,92.58962459)(519.14779053,92.60963186)
\curveto(519.24779883,92.62962455)(519.35279872,92.64962453)(519.46279053,92.66963186)
\curveto(519.51279856,92.68962449)(519.55779852,92.69462448)(519.59779053,92.68463186)
\curveto(519.63779844,92.6746245)(519.68279839,92.6796245)(519.73279053,92.69963186)
\curveto(519.78279829,92.70962447)(519.86779821,92.71462446)(519.98779053,92.71463186)
\curveto(520.09779798,92.71462446)(520.18279789,92.70962447)(520.24279053,92.69963186)
\curveto(520.30279777,92.6796245)(520.36279771,92.66962451)(520.42279053,92.66963186)
\curveto(520.48279759,92.6796245)(520.54279753,92.6746245)(520.60279053,92.65463186)
\curveto(520.74279733,92.61462456)(520.8777972,92.5796246)(521.00779053,92.54963186)
\curveto(521.13779694,92.51962466)(521.26279681,92.4796247)(521.38279053,92.42963186)
\curveto(521.52279655,92.36962481)(521.64779643,92.29962488)(521.75779053,92.21963186)
\curveto(521.86779621,92.14962503)(521.9777961,92.0746251)(522.08779053,91.99463186)
\lineto(522.14779053,91.93463186)
\curveto(522.16779591,91.92462525)(522.18779589,91.90962527)(522.20779053,91.88963186)
\curveto(522.36779571,91.76962541)(522.51279556,91.63462554)(522.64279053,91.48463186)
\curveto(522.7727953,91.33462584)(522.89779518,91.174626)(523.01779053,91.00463186)
\curveto(523.23779484,90.69462648)(523.44279463,90.39962678)(523.63279053,90.11963186)
\curveto(523.7727943,89.88962729)(523.90779417,89.65962752)(524.03779053,89.42963186)
\curveto(524.16779391,89.20962797)(524.30279377,88.98962819)(524.44279053,88.76963186)
\curveto(524.61279346,88.51962866)(524.79279328,88.2796289)(524.98279053,88.04963186)
\curveto(525.1727929,87.82962935)(525.39779268,87.63962954)(525.65779053,87.47963186)
\curveto(525.71779236,87.43962974)(525.7777923,87.40462977)(525.83779053,87.37463186)
\curveto(525.88779219,87.34462983)(525.95279212,87.31462986)(526.03279053,87.28463186)
\curveto(526.10279197,87.26462991)(526.16279191,87.25962992)(526.21279053,87.26963186)
\curveto(526.28279179,87.28962989)(526.33779174,87.32462985)(526.37779053,87.37463186)
\curveto(526.40779167,87.42462975)(526.42779165,87.48462969)(526.43779053,87.55463186)
\lineto(526.43779053,87.79463186)
\lineto(526.43779053,88.54463186)
\lineto(526.43779053,91.34963186)
\lineto(526.43779053,92.00963186)
\curveto(526.43779164,92.09962508)(526.44279163,92.18462499)(526.45279053,92.26463186)
\curveto(526.45279162,92.34462483)(526.4727916,92.40962477)(526.51279053,92.45963186)
\curveto(526.55279152,92.50962467)(526.62779145,92.54962463)(526.73779053,92.57963186)
\curveto(526.83779124,92.61962456)(526.93779114,92.62962455)(527.03779053,92.60963186)
\lineto(527.17279053,92.60963186)
\curveto(527.24279083,92.58962459)(527.30279077,92.56962461)(527.35279053,92.54963186)
\curveto(527.40279067,92.52962465)(527.44279063,92.49462468)(527.47279053,92.44463186)
\curveto(527.51279056,92.39462478)(527.53279054,92.32462485)(527.53279053,92.23463186)
\lineto(527.53279053,91.96463186)
\lineto(527.53279053,91.06463186)
\lineto(527.53279053,87.55463186)
\lineto(527.53279053,86.48963186)
\curveto(527.53279054,86.40963077)(527.53779054,86.31963086)(527.54779053,86.21963186)
\curveto(527.54779053,86.11963106)(527.53779054,86.03463114)(527.51779053,85.96463186)
\curveto(527.44779063,85.75463142)(527.26779081,85.68963149)(526.97779053,85.76963186)
\curveto(526.93779114,85.7796314)(526.90279117,85.7796314)(526.87279053,85.76963186)
\curveto(526.83279124,85.76963141)(526.78779129,85.7796314)(526.73779053,85.79963186)
\curveto(526.65779142,85.81963136)(526.5727915,85.83963134)(526.48279053,85.85963186)
\curveto(526.39279168,85.8796313)(526.30779177,85.90463127)(526.22779053,85.93463186)
\curveto(525.73779234,86.09463108)(525.32279275,86.29463088)(524.98279053,86.53463186)
\curveto(524.73279334,86.71463046)(524.50779357,86.91963026)(524.30779053,87.14963186)
\curveto(524.09779398,87.3796298)(523.90279417,87.61962956)(523.72279053,87.86963186)
\curveto(523.54279453,88.12962905)(523.3727947,88.39462878)(523.21279053,88.66463186)
\curveto(523.04279503,88.94462823)(522.86779521,89.21462796)(522.68779053,89.47463186)
\curveto(522.60779547,89.58462759)(522.53279554,89.68962749)(522.46279053,89.78963186)
\curveto(522.39279568,89.89962728)(522.31779576,90.00962717)(522.23779053,90.11963186)
\curveto(522.20779587,90.15962702)(522.1777959,90.19462698)(522.14779053,90.22463186)
\curveto(522.10779597,90.26462691)(522.077796,90.30462687)(522.05779053,90.34463186)
\curveto(521.94779613,90.48462669)(521.82279625,90.60962657)(521.68279053,90.71963186)
\curveto(521.65279642,90.73962644)(521.62779645,90.76462641)(521.60779053,90.79463186)
\curveto(521.5777965,90.82462635)(521.54779653,90.84962633)(521.51779053,90.86963186)
\curveto(521.41779666,90.94962623)(521.31779676,91.01462616)(521.21779053,91.06463186)
\curveto(521.11779696,91.12462605)(521.00779707,91.179626)(520.88779053,91.22963186)
\curveto(520.81779726,91.25962592)(520.74279733,91.2796259)(520.66279053,91.28963186)
\lineto(520.42279053,91.34963186)
\lineto(520.33279053,91.34963186)
\curveto(520.30279777,91.35962582)(520.2727978,91.36462581)(520.24279053,91.36463186)
\curveto(520.1727979,91.38462579)(520.077798,91.38962579)(519.95779053,91.37963186)
\curveto(519.82779825,91.3796258)(519.72779835,91.36962581)(519.65779053,91.34963186)
\curveto(519.5777985,91.32962585)(519.50279857,91.30962587)(519.43279053,91.28963186)
\curveto(519.35279872,91.2796259)(519.2727988,91.25962592)(519.19279053,91.22963186)
\curveto(518.95279912,91.11962606)(518.75279932,90.96962621)(518.59279053,90.77963186)
\curveto(518.42279965,90.59962658)(518.28279979,90.3796268)(518.17279053,90.11963186)
\curveto(518.15279992,90.04962713)(518.13779994,89.9796272)(518.12779053,89.90963186)
\curveto(518.10779997,89.83962734)(518.08779999,89.76462741)(518.06779053,89.68463186)
\curveto(518.04780003,89.60462757)(518.03780004,89.49462768)(518.03779053,89.35463186)
\curveto(518.03780004,89.22462795)(518.04780003,89.11962806)(518.06779053,89.03963186)
\curveto(518.0778,88.9796282)(518.08279999,88.92462825)(518.08279053,88.87463186)
\curveto(518.08279999,88.82462835)(518.09279998,88.7746284)(518.11279053,88.72463186)
\curveto(518.15279992,88.62462855)(518.19279988,88.52962865)(518.23279053,88.43963186)
\curveto(518.2727998,88.35962882)(518.31779976,88.2796289)(518.36779053,88.19963186)
\curveto(518.38779969,88.16962901)(518.41279966,88.13962904)(518.44279053,88.10963186)
\curveto(518.4727996,88.08962909)(518.49779958,88.06462911)(518.51779053,88.03463186)
\lineto(518.59279053,87.95963186)
\curveto(518.61279946,87.92962925)(518.63279944,87.90462927)(518.65279053,87.88463186)
\lineto(518.86279053,87.73463186)
\curveto(518.92279915,87.69462948)(518.98779909,87.64962953)(519.05779053,87.59963186)
\curveto(519.14779893,87.53962964)(519.25279882,87.48962969)(519.37279053,87.44963186)
\curveto(519.48279859,87.41962976)(519.59279848,87.38462979)(519.70279053,87.34463186)
\curveto(519.81279826,87.30462987)(519.95779812,87.2796299)(520.13779053,87.26963186)
\curveto(520.30779777,87.25962992)(520.43279764,87.22962995)(520.51279053,87.17963186)
\curveto(520.59279748,87.12963005)(520.63779744,87.05463012)(520.64779053,86.95463186)
\curveto(520.65779742,86.85463032)(520.66279741,86.74463043)(520.66279053,86.62463186)
\curveto(520.66279741,86.58463059)(520.66779741,86.54463063)(520.67779053,86.50463186)
\curveto(520.6777974,86.46463071)(520.6727974,86.42963075)(520.66279053,86.39963186)
\curveto(520.64279743,86.34963083)(520.63279744,86.29963088)(520.63279053,86.24963186)
\curveto(520.63279744,86.20963097)(520.62279745,86.16963101)(520.60279053,86.12963186)
\curveto(520.54279753,86.03963114)(520.40779767,85.99463118)(520.19779053,85.99463186)
\lineto(520.07779053,85.99463186)
\curveto(520.01779806,86.00463117)(519.95779812,86.00963117)(519.89779053,86.00963186)
\curveto(519.82779825,86.01963116)(519.76279831,86.02963115)(519.70279053,86.03963186)
\curveto(519.59279848,86.05963112)(519.49279858,86.0796311)(519.40279053,86.09963186)
\curveto(519.30279877,86.11963106)(519.20779887,86.14963103)(519.11779053,86.18963186)
\curveto(519.04779903,86.20963097)(518.98779909,86.22963095)(518.93779053,86.24963186)
\lineto(518.75779053,86.30963186)
\curveto(518.49779958,86.42963075)(518.25279982,86.58463059)(518.02279053,86.77463186)
\curveto(517.79280028,86.9746302)(517.60780047,87.18962999)(517.46779053,87.41963186)
\curveto(517.38780069,87.52962965)(517.32280075,87.64462953)(517.27279053,87.76463186)
\lineto(517.12279053,88.15463186)
\curveto(517.072801,88.26462891)(517.04280103,88.3796288)(517.03279053,88.49963186)
\curveto(517.01280106,88.61962856)(516.98780109,88.74462843)(516.95779053,88.87463186)
\curveto(516.95780112,88.94462823)(516.95780112,89.00962817)(516.95779053,89.06963186)
\curveto(516.94780113,89.12962805)(516.93780114,89.19462798)(516.92779053,89.26463186)
}
}
{
\newrgbcolor{curcolor}{0 0 0}
\pscustom[linestyle=none,fillstyle=solid,fillcolor=curcolor]
{
\newpath
\moveto(522.44779053,101.36424123)
\lineto(522.70279053,101.36424123)
\curveto(522.78279529,101.37423353)(522.85779522,101.36923353)(522.92779053,101.34924123)
\lineto(523.16779053,101.34924123)
\lineto(523.33279053,101.34924123)
\curveto(523.43279464,101.32923357)(523.53779454,101.31923358)(523.64779053,101.31924123)
\curveto(523.74779433,101.31923358)(523.84779423,101.30923359)(523.94779053,101.28924123)
\lineto(524.09779053,101.28924123)
\curveto(524.23779384,101.25923364)(524.3777937,101.23923366)(524.51779053,101.22924123)
\curveto(524.64779343,101.21923368)(524.7777933,101.19423371)(524.90779053,101.15424123)
\curveto(524.98779309,101.13423377)(525.072793,101.11423379)(525.16279053,101.09424123)
\lineto(525.40279053,101.03424123)
\lineto(525.70279053,100.91424123)
\curveto(525.79279228,100.88423402)(525.88279219,100.84923405)(525.97279053,100.80924123)
\curveto(526.19279188,100.70923419)(526.40779167,100.57423433)(526.61779053,100.40424123)
\curveto(526.82779125,100.24423466)(526.99779108,100.06923483)(527.12779053,99.87924123)
\curveto(527.16779091,99.82923507)(527.20779087,99.76923513)(527.24779053,99.69924123)
\curveto(527.2777908,99.63923526)(527.31279076,99.57923532)(527.35279053,99.51924123)
\curveto(527.40279067,99.43923546)(527.44279063,99.34423556)(527.47279053,99.23424123)
\curveto(527.50279057,99.12423578)(527.53279054,99.01923588)(527.56279053,98.91924123)
\curveto(527.60279047,98.80923609)(527.62779045,98.6992362)(527.63779053,98.58924123)
\curveto(527.64779043,98.47923642)(527.66279041,98.36423654)(527.68279053,98.24424123)
\curveto(527.69279038,98.2042367)(527.69279038,98.15923674)(527.68279053,98.10924123)
\curveto(527.68279039,98.06923683)(527.68779039,98.02923687)(527.69779053,97.98924123)
\curveto(527.70779037,97.94923695)(527.71279036,97.89423701)(527.71279053,97.82424123)
\curveto(527.71279036,97.75423715)(527.70779037,97.7042372)(527.69779053,97.67424123)
\curveto(527.6777904,97.62423728)(527.6727904,97.57923732)(527.68279053,97.53924123)
\curveto(527.69279038,97.4992374)(527.69279038,97.46423744)(527.68279053,97.43424123)
\lineto(527.68279053,97.34424123)
\curveto(527.66279041,97.28423762)(527.64779043,97.21923768)(527.63779053,97.14924123)
\curveto(527.63779044,97.08923781)(527.63279044,97.02423788)(527.62279053,96.95424123)
\curveto(527.5727905,96.78423812)(527.52279055,96.62423828)(527.47279053,96.47424123)
\curveto(527.42279065,96.32423858)(527.35779072,96.17923872)(527.27779053,96.03924123)
\curveto(527.23779084,95.98923891)(527.20779087,95.93423897)(527.18779053,95.87424123)
\curveto(527.15779092,95.82423908)(527.12279095,95.77423913)(527.08279053,95.72424123)
\curveto(526.90279117,95.48423942)(526.68279139,95.28423962)(526.42279053,95.12424123)
\curveto(526.16279191,94.96423994)(525.8777922,94.82424008)(525.56779053,94.70424123)
\curveto(525.42779265,94.64424026)(525.28779279,94.5992403)(525.14779053,94.56924123)
\curveto(524.99779308,94.53924036)(524.84279323,94.5042404)(524.68279053,94.46424123)
\curveto(524.5727935,94.44424046)(524.46279361,94.42924047)(524.35279053,94.41924123)
\curveto(524.24279383,94.40924049)(524.13279394,94.39424051)(524.02279053,94.37424123)
\curveto(523.98279409,94.36424054)(523.94279413,94.35924054)(523.90279053,94.35924123)
\curveto(523.86279421,94.36924053)(523.82279425,94.36924053)(523.78279053,94.35924123)
\curveto(523.73279434,94.34924055)(523.68279439,94.34424056)(523.63279053,94.34424123)
\lineto(523.46779053,94.34424123)
\curveto(523.41779466,94.32424058)(523.36779471,94.31924058)(523.31779053,94.32924123)
\curveto(523.25779482,94.33924056)(523.20279487,94.33924056)(523.15279053,94.32924123)
\curveto(523.11279496,94.31924058)(523.06779501,94.31924058)(523.01779053,94.32924123)
\curveto(522.96779511,94.33924056)(522.91779516,94.33424057)(522.86779053,94.31424123)
\curveto(522.79779528,94.29424061)(522.72279535,94.28924061)(522.64279053,94.29924123)
\curveto(522.55279552,94.30924059)(522.46779561,94.31424059)(522.38779053,94.31424123)
\curveto(522.29779578,94.31424059)(522.19779588,94.30924059)(522.08779053,94.29924123)
\curveto(521.96779611,94.28924061)(521.86779621,94.29424061)(521.78779053,94.31424123)
\lineto(521.50279053,94.31424123)
\lineto(520.87279053,94.35924123)
\curveto(520.7727973,94.36924053)(520.6777974,94.37924052)(520.58779053,94.38924123)
\lineto(520.28779053,94.41924123)
\curveto(520.23779784,94.43924046)(520.18779789,94.44424046)(520.13779053,94.43424123)
\curveto(520.077798,94.43424047)(520.02279805,94.44424046)(519.97279053,94.46424123)
\curveto(519.80279827,94.51424039)(519.63779844,94.55424035)(519.47779053,94.58424123)
\curveto(519.30779877,94.61424029)(519.14779893,94.66424024)(518.99779053,94.73424123)
\curveto(518.53779954,94.92423998)(518.16279991,95.14423976)(517.87279053,95.39424123)
\curveto(517.58280049,95.65423925)(517.33780074,96.01423889)(517.13779053,96.47424123)
\curveto(517.08780099,96.6042383)(517.05280102,96.73423817)(517.03279053,96.86424123)
\curveto(517.01280106,97.0042379)(516.98780109,97.14423776)(516.95779053,97.28424123)
\curveto(516.94780113,97.35423755)(516.94280113,97.41923748)(516.94279053,97.47924123)
\curveto(516.94280113,97.53923736)(516.93780114,97.6042373)(516.92779053,97.67424123)
\curveto(516.90780117,98.5042364)(517.05780102,99.17423573)(517.37779053,99.68424123)
\curveto(517.68780039,100.19423471)(518.12779995,100.57423433)(518.69779053,100.82424123)
\curveto(518.81779926,100.87423403)(518.94279913,100.91923398)(519.07279053,100.95924123)
\curveto(519.20279887,100.9992339)(519.33779874,101.04423386)(519.47779053,101.09424123)
\curveto(519.55779852,101.11423379)(519.64279843,101.12923377)(519.73279053,101.13924123)
\lineto(519.97279053,101.19924123)
\curveto(520.08279799,101.22923367)(520.19279788,101.24423366)(520.30279053,101.24424123)
\curveto(520.41279766,101.25423365)(520.52279755,101.26923363)(520.63279053,101.28924123)
\curveto(520.68279739,101.30923359)(520.72779735,101.31423359)(520.76779053,101.30424123)
\curveto(520.80779727,101.3042336)(520.84779723,101.30923359)(520.88779053,101.31924123)
\curveto(520.93779714,101.32923357)(520.99279708,101.32923357)(521.05279053,101.31924123)
\curveto(521.10279697,101.31923358)(521.15279692,101.32423358)(521.20279053,101.33424123)
\lineto(521.33779053,101.33424123)
\curveto(521.39779668,101.35423355)(521.46779661,101.35423355)(521.54779053,101.33424123)
\curveto(521.61779646,101.32423358)(521.68279639,101.32923357)(521.74279053,101.34924123)
\curveto(521.7727963,101.35923354)(521.81279626,101.36423354)(521.86279053,101.36424123)
\lineto(521.98279053,101.36424123)
\lineto(522.44779053,101.36424123)
\moveto(524.77279053,99.81924123)
\curveto(524.45279362,99.91923498)(524.08779399,99.97923492)(523.67779053,99.99924123)
\curveto(523.26779481,100.01923488)(522.85779522,100.02923487)(522.44779053,100.02924123)
\curveto(522.01779606,100.02923487)(521.59779648,100.01923488)(521.18779053,99.99924123)
\curveto(520.7777973,99.97923492)(520.39279768,99.93423497)(520.03279053,99.86424123)
\curveto(519.6727984,99.79423511)(519.35279872,99.68423522)(519.07279053,99.53424123)
\curveto(518.78279929,99.39423551)(518.54779953,99.1992357)(518.36779053,98.94924123)
\curveto(518.25779982,98.78923611)(518.1777999,98.60923629)(518.12779053,98.40924123)
\curveto(518.06780001,98.20923669)(518.03780004,97.96423694)(518.03779053,97.67424123)
\curveto(518.05780002,97.65423725)(518.06780001,97.61923728)(518.06779053,97.56924123)
\curveto(518.05780002,97.51923738)(518.05780002,97.47923742)(518.06779053,97.44924123)
\curveto(518.08779999,97.36923753)(518.10779997,97.29423761)(518.12779053,97.22424123)
\curveto(518.13779994,97.16423774)(518.15779992,97.0992378)(518.18779053,97.02924123)
\curveto(518.30779977,96.75923814)(518.4777996,96.53923836)(518.69779053,96.36924123)
\curveto(518.90779917,96.20923869)(519.15279892,96.07423883)(519.43279053,95.96424123)
\curveto(519.54279853,95.91423899)(519.66279841,95.87423903)(519.79279053,95.84424123)
\curveto(519.91279816,95.82423908)(520.03779804,95.7992391)(520.16779053,95.76924123)
\curveto(520.21779786,95.74923915)(520.2727978,95.73923916)(520.33279053,95.73924123)
\curveto(520.38279769,95.73923916)(520.43279764,95.73423917)(520.48279053,95.72424123)
\curveto(520.5727975,95.71423919)(520.66779741,95.7042392)(520.76779053,95.69424123)
\curveto(520.85779722,95.68423922)(520.95279712,95.67423923)(521.05279053,95.66424123)
\curveto(521.13279694,95.66423924)(521.21779686,95.65923924)(521.30779053,95.64924123)
\lineto(521.54779053,95.64924123)
\lineto(521.72779053,95.64924123)
\curveto(521.75779632,95.63923926)(521.79279628,95.63423927)(521.83279053,95.63424123)
\lineto(521.96779053,95.63424123)
\lineto(522.41779053,95.63424123)
\curveto(522.49779558,95.63423927)(522.58279549,95.62923927)(522.67279053,95.61924123)
\curveto(522.75279532,95.61923928)(522.82779525,95.62923927)(522.89779053,95.64924123)
\lineto(523.16779053,95.64924123)
\curveto(523.18779489,95.64923925)(523.21779486,95.64423926)(523.25779053,95.63424123)
\curveto(523.28779479,95.63423927)(523.31279476,95.63923926)(523.33279053,95.64924123)
\curveto(523.43279464,95.65923924)(523.53279454,95.66423924)(523.63279053,95.66424123)
\curveto(523.72279435,95.67423923)(523.82279425,95.68423922)(523.93279053,95.69424123)
\curveto(524.05279402,95.72423918)(524.1777939,95.73923916)(524.30779053,95.73924123)
\curveto(524.42779365,95.74923915)(524.54279353,95.77423913)(524.65279053,95.81424123)
\curveto(524.95279312,95.89423901)(525.21779286,95.97923892)(525.44779053,96.06924123)
\curveto(525.6777924,96.16923873)(525.89279218,96.31423859)(526.09279053,96.50424123)
\curveto(526.29279178,96.71423819)(526.44279163,96.97923792)(526.54279053,97.29924123)
\curveto(526.56279151,97.33923756)(526.5727915,97.37423753)(526.57279053,97.40424123)
\curveto(526.56279151,97.44423746)(526.56779151,97.48923741)(526.58779053,97.53924123)
\curveto(526.59779148,97.57923732)(526.60779147,97.64923725)(526.61779053,97.74924123)
\curveto(526.62779145,97.85923704)(526.62279145,97.94423696)(526.60279053,98.00424123)
\curveto(526.58279149,98.07423683)(526.5727915,98.14423676)(526.57279053,98.21424123)
\curveto(526.56279151,98.28423662)(526.54779153,98.34923655)(526.52779053,98.40924123)
\curveto(526.46779161,98.60923629)(526.38279169,98.78923611)(526.27279053,98.94924123)
\curveto(526.25279182,98.97923592)(526.23279184,99.0042359)(526.21279053,99.02424123)
\lineto(526.15279053,99.08424123)
\curveto(526.13279194,99.12423578)(526.09279198,99.17423573)(526.03279053,99.23424123)
\curveto(525.89279218,99.33423557)(525.76279231,99.41923548)(525.64279053,99.48924123)
\curveto(525.52279255,99.55923534)(525.3777927,99.62923527)(525.20779053,99.69924123)
\curveto(525.13779294,99.72923517)(525.06779301,99.74923515)(524.99779053,99.75924123)
\curveto(524.92779315,99.77923512)(524.85279322,99.7992351)(524.77279053,99.81924123)
}
}
{
\newrgbcolor{curcolor}{0 0 0}
\pscustom[linestyle=none,fillstyle=solid,fillcolor=curcolor]
{
\newpath
\moveto(516.92779053,106.77385061)
\curveto(516.92780115,106.87384575)(516.93780114,106.96884566)(516.95779053,107.05885061)
\curveto(516.96780111,107.14884548)(516.99780108,107.21384541)(517.04779053,107.25385061)
\curveto(517.12780095,107.31384531)(517.23280084,107.34384528)(517.36279053,107.34385061)
\lineto(517.75279053,107.34385061)
\lineto(519.25279053,107.34385061)
\lineto(525.64279053,107.34385061)
\lineto(526.81279053,107.34385061)
\lineto(527.12779053,107.34385061)
\curveto(527.22779085,107.35384527)(527.30779077,107.33884529)(527.36779053,107.29885061)
\curveto(527.44779063,107.24884538)(527.49779058,107.17384545)(527.51779053,107.07385061)
\curveto(527.52779055,106.98384564)(527.53279054,106.87384575)(527.53279053,106.74385061)
\lineto(527.53279053,106.51885061)
\curveto(527.51279056,106.43884619)(527.49779058,106.36884626)(527.48779053,106.30885061)
\curveto(527.46779061,106.24884638)(527.42779065,106.19884643)(527.36779053,106.15885061)
\curveto(527.30779077,106.11884651)(527.23279084,106.09884653)(527.14279053,106.09885061)
\lineto(526.84279053,106.09885061)
\lineto(525.74779053,106.09885061)
\lineto(520.40779053,106.09885061)
\curveto(520.31779776,106.07884655)(520.24279783,106.06384656)(520.18279053,106.05385061)
\curveto(520.11279796,106.05384657)(520.05279802,106.0238466)(520.00279053,105.96385061)
\curveto(519.95279812,105.89384673)(519.92779815,105.80384682)(519.92779053,105.69385061)
\curveto(519.91779816,105.59384703)(519.91279816,105.48384714)(519.91279053,105.36385061)
\lineto(519.91279053,104.22385061)
\lineto(519.91279053,103.72885061)
\curveto(519.90279817,103.56884906)(519.84279823,103.45884917)(519.73279053,103.39885061)
\curveto(519.70279837,103.37884925)(519.6727984,103.36884926)(519.64279053,103.36885061)
\curveto(519.60279847,103.36884926)(519.55779852,103.36384926)(519.50779053,103.35385061)
\curveto(519.38779869,103.33384929)(519.2777988,103.33884929)(519.17779053,103.36885061)
\curveto(519.077799,103.40884922)(519.00779907,103.46384916)(518.96779053,103.53385061)
\curveto(518.91779916,103.61384901)(518.89279918,103.73384889)(518.89279053,103.89385061)
\curveto(518.89279918,104.05384857)(518.8777992,104.18884844)(518.84779053,104.29885061)
\curveto(518.83779924,104.34884828)(518.83279924,104.40384822)(518.83279053,104.46385061)
\curveto(518.82279925,104.5238481)(518.80779927,104.58384804)(518.78779053,104.64385061)
\curveto(518.73779934,104.79384783)(518.68779939,104.93884769)(518.63779053,105.07885061)
\curveto(518.5777995,105.21884741)(518.50779957,105.35384727)(518.42779053,105.48385061)
\curveto(518.33779974,105.623847)(518.23279984,105.74384688)(518.11279053,105.84385061)
\curveto(517.99280008,105.94384668)(517.86280021,106.03884659)(517.72279053,106.12885061)
\curveto(517.62280045,106.18884644)(517.51280056,106.23384639)(517.39279053,106.26385061)
\curveto(517.2728008,106.30384632)(517.16780091,106.35384627)(517.07779053,106.41385061)
\curveto(517.01780106,106.46384616)(516.9778011,106.53384609)(516.95779053,106.62385061)
\curveto(516.94780113,106.64384598)(516.94280113,106.66884596)(516.94279053,106.69885061)
\curveto(516.94280113,106.7288459)(516.93780114,106.75384587)(516.92779053,106.77385061)
}
}
{
\newrgbcolor{curcolor}{0 0 0}
\pscustom[linestyle=none,fillstyle=solid,fillcolor=curcolor]
{
\newpath
\moveto(516.92779053,115.12345998)
\curveto(516.92780115,115.22345513)(516.93780114,115.31845503)(516.95779053,115.40845998)
\curveto(516.96780111,115.49845485)(516.99780108,115.56345479)(517.04779053,115.60345998)
\curveto(517.12780095,115.66345469)(517.23280084,115.69345466)(517.36279053,115.69345998)
\lineto(517.75279053,115.69345998)
\lineto(519.25279053,115.69345998)
\lineto(525.64279053,115.69345998)
\lineto(526.81279053,115.69345998)
\lineto(527.12779053,115.69345998)
\curveto(527.22779085,115.70345465)(527.30779077,115.68845466)(527.36779053,115.64845998)
\curveto(527.44779063,115.59845475)(527.49779058,115.52345483)(527.51779053,115.42345998)
\curveto(527.52779055,115.33345502)(527.53279054,115.22345513)(527.53279053,115.09345998)
\lineto(527.53279053,114.86845998)
\curveto(527.51279056,114.78845556)(527.49779058,114.71845563)(527.48779053,114.65845998)
\curveto(527.46779061,114.59845575)(527.42779065,114.5484558)(527.36779053,114.50845998)
\curveto(527.30779077,114.46845588)(527.23279084,114.4484559)(527.14279053,114.44845998)
\lineto(526.84279053,114.44845998)
\lineto(525.74779053,114.44845998)
\lineto(520.40779053,114.44845998)
\curveto(520.31779776,114.42845592)(520.24279783,114.41345594)(520.18279053,114.40345998)
\curveto(520.11279796,114.40345595)(520.05279802,114.37345598)(520.00279053,114.31345998)
\curveto(519.95279812,114.24345611)(519.92779815,114.1534562)(519.92779053,114.04345998)
\curveto(519.91779816,113.94345641)(519.91279816,113.83345652)(519.91279053,113.71345998)
\lineto(519.91279053,112.57345998)
\lineto(519.91279053,112.07845998)
\curveto(519.90279817,111.91845843)(519.84279823,111.80845854)(519.73279053,111.74845998)
\curveto(519.70279837,111.72845862)(519.6727984,111.71845863)(519.64279053,111.71845998)
\curveto(519.60279847,111.71845863)(519.55779852,111.71345864)(519.50779053,111.70345998)
\curveto(519.38779869,111.68345867)(519.2777988,111.68845866)(519.17779053,111.71845998)
\curveto(519.077799,111.75845859)(519.00779907,111.81345854)(518.96779053,111.88345998)
\curveto(518.91779916,111.96345839)(518.89279918,112.08345827)(518.89279053,112.24345998)
\curveto(518.89279918,112.40345795)(518.8777992,112.53845781)(518.84779053,112.64845998)
\curveto(518.83779924,112.69845765)(518.83279924,112.7534576)(518.83279053,112.81345998)
\curveto(518.82279925,112.87345748)(518.80779927,112.93345742)(518.78779053,112.99345998)
\curveto(518.73779934,113.14345721)(518.68779939,113.28845706)(518.63779053,113.42845998)
\curveto(518.5777995,113.56845678)(518.50779957,113.70345665)(518.42779053,113.83345998)
\curveto(518.33779974,113.97345638)(518.23279984,114.09345626)(518.11279053,114.19345998)
\curveto(517.99280008,114.29345606)(517.86280021,114.38845596)(517.72279053,114.47845998)
\curveto(517.62280045,114.53845581)(517.51280056,114.58345577)(517.39279053,114.61345998)
\curveto(517.2728008,114.6534557)(517.16780091,114.70345565)(517.07779053,114.76345998)
\curveto(517.01780106,114.81345554)(516.9778011,114.88345547)(516.95779053,114.97345998)
\curveto(516.94780113,114.99345536)(516.94280113,115.01845533)(516.94279053,115.04845998)
\curveto(516.94280113,115.07845527)(516.93780114,115.10345525)(516.92779053,115.12345998)
}
}
{
\newrgbcolor{curcolor}{0 0 0}
\pscustom[linestyle=none,fillstyle=solid,fillcolor=curcolor]
{
\newpath
\moveto(538.79907959,29.18119436)
\lineto(538.79907959,30.09619436)
\curveto(538.79909028,30.19619171)(538.79909028,30.29119161)(538.79907959,30.38119436)
\curveto(538.79909028,30.47119143)(538.81909026,30.54619136)(538.85907959,30.60619436)
\curveto(538.91909016,30.69619121)(538.99909008,30.75619115)(539.09907959,30.78619436)
\curveto(539.19908988,30.82619108)(539.30408978,30.87119103)(539.41407959,30.92119436)
\curveto(539.60408948,31.0011909)(539.79408929,31.07119083)(539.98407959,31.13119436)
\curveto(540.17408891,31.2011907)(540.36408872,31.27619063)(540.55407959,31.35619436)
\curveto(540.73408835,31.42619048)(540.91908816,31.49119041)(541.10907959,31.55119436)
\curveto(541.28908779,31.61119029)(541.46908761,31.68119022)(541.64907959,31.76119436)
\curveto(541.78908729,31.82119008)(541.93408715,31.87619003)(542.08407959,31.92619436)
\curveto(542.23408685,31.97618993)(542.3790867,32.03118987)(542.51907959,32.09119436)
\curveto(542.96908611,32.27118963)(543.42408566,32.44118946)(543.88407959,32.60119436)
\curveto(544.33408475,32.76118914)(544.7840843,32.93118897)(545.23407959,33.11119436)
\curveto(545.2840838,33.13118877)(545.33408375,33.14618876)(545.38407959,33.15619436)
\lineto(545.53407959,33.21619436)
\curveto(545.75408333,33.3061886)(545.9790831,33.39118851)(546.20907959,33.47119436)
\curveto(546.42908265,33.55118835)(546.64908243,33.63618827)(546.86907959,33.72619436)
\curveto(546.95908212,33.76618814)(547.06908201,33.8061881)(547.19907959,33.84619436)
\curveto(547.31908176,33.88618802)(547.38908169,33.95118795)(547.40907959,34.04119436)
\curveto(547.41908166,34.08118782)(547.41908166,34.11118779)(547.40907959,34.13119436)
\lineto(547.34907959,34.19119436)
\curveto(547.29908178,34.24118766)(547.24408184,34.27618763)(547.18407959,34.29619436)
\curveto(547.12408196,34.32618758)(547.05908202,34.35618755)(546.98907959,34.38619436)
\lineto(546.35907959,34.62619436)
\curveto(546.13908294,34.7061872)(545.92408316,34.78618712)(545.71407959,34.86619436)
\lineto(545.56407959,34.92619436)
\lineto(545.38407959,34.98619436)
\curveto(545.19408389,35.06618684)(545.00408408,35.13618677)(544.81407959,35.19619436)
\curveto(544.61408447,35.26618664)(544.41408467,35.34118656)(544.21407959,35.42119436)
\curveto(543.63408545,35.66118624)(543.04908603,35.88118602)(542.45907959,36.08119436)
\curveto(541.86908721,36.29118561)(541.2840878,36.51618539)(540.70407959,36.75619436)
\curveto(540.50408858,36.83618507)(540.29908878,36.91118499)(540.08907959,36.98119436)
\curveto(539.8790892,37.06118484)(539.67408941,37.14118476)(539.47407959,37.22119436)
\curveto(539.39408969,37.26118464)(539.29408979,37.29618461)(539.17407959,37.32619436)
\curveto(539.05409003,37.36618454)(538.96909011,37.42118448)(538.91907959,37.49119436)
\curveto(538.8790902,37.55118435)(538.84909023,37.62618428)(538.82907959,37.71619436)
\curveto(538.80909027,37.81618409)(538.79909028,37.92618398)(538.79907959,38.04619436)
\curveto(538.78909029,38.16618374)(538.78909029,38.28618362)(538.79907959,38.40619436)
\curveto(538.79909028,38.52618338)(538.79909028,38.63618327)(538.79907959,38.73619436)
\curveto(538.79909028,38.82618308)(538.79909028,38.91618299)(538.79907959,39.00619436)
\curveto(538.79909028,39.1061828)(538.81909026,39.18118272)(538.85907959,39.23119436)
\curveto(538.90909017,39.32118258)(538.99909008,39.37118253)(539.12907959,39.38119436)
\curveto(539.25908982,39.39118251)(539.39908968,39.39618251)(539.54907959,39.39619436)
\lineto(541.19907959,39.39619436)
\lineto(547.46907959,39.39619436)
\lineto(548.72907959,39.39619436)
\curveto(548.83908024,39.39618251)(548.94908013,39.39618251)(549.05907959,39.39619436)
\curveto(549.16907991,39.4061825)(549.25407983,39.38618252)(549.31407959,39.33619436)
\curveto(549.37407971,39.3061826)(549.41407967,39.26118264)(549.43407959,39.20119436)
\curveto(549.44407964,39.14118276)(549.45907962,39.07118283)(549.47907959,38.99119436)
\lineto(549.47907959,38.75119436)
\lineto(549.47907959,38.39119436)
\curveto(549.46907961,38.28118362)(549.42407966,38.2011837)(549.34407959,38.15119436)
\curveto(549.31407977,38.13118377)(549.2840798,38.11618379)(549.25407959,38.10619436)
\curveto(549.21407987,38.1061838)(549.16907991,38.09618381)(549.11907959,38.07619436)
\lineto(548.95407959,38.07619436)
\curveto(548.89408019,38.06618384)(548.82408026,38.06118384)(548.74407959,38.06119436)
\curveto(548.66408042,38.07118383)(548.58908049,38.07618383)(548.51907959,38.07619436)
\lineto(547.67907959,38.07619436)
\lineto(543.25407959,38.07619436)
\curveto(543.00408608,38.07618383)(542.75408633,38.07618383)(542.50407959,38.07619436)
\curveto(542.24408684,38.07618383)(541.99408709,38.07118383)(541.75407959,38.06119436)
\curveto(541.65408743,38.06118384)(541.54408754,38.05618385)(541.42407959,38.04619436)
\curveto(541.30408778,38.03618387)(541.24408784,37.98118392)(541.24407959,37.88119436)
\lineto(541.25907959,37.88119436)
\curveto(541.2790878,37.81118409)(541.34408774,37.75118415)(541.45407959,37.70119436)
\curveto(541.56408752,37.66118424)(541.65908742,37.62618428)(541.73907959,37.59619436)
\curveto(541.90908717,37.52618438)(542.084087,37.46118444)(542.26407959,37.40119436)
\curveto(542.43408665,37.34118456)(542.60408648,37.27118463)(542.77407959,37.19119436)
\curveto(542.82408626,37.17118473)(542.86908621,37.15618475)(542.90907959,37.14619436)
\curveto(542.94908613,37.13618477)(542.99408609,37.12118478)(543.04407959,37.10119436)
\curveto(543.22408586,37.02118488)(543.40908567,36.95118495)(543.59907959,36.89119436)
\curveto(543.7790853,36.84118506)(543.95908512,36.77618513)(544.13907959,36.69619436)
\curveto(544.28908479,36.62618528)(544.44408464,36.56618534)(544.60407959,36.51619436)
\curveto(544.75408433,36.46618544)(544.90408418,36.41118549)(545.05407959,36.35119436)
\curveto(545.52408356,36.15118575)(545.99908308,35.97118593)(546.47907959,35.81119436)
\curveto(546.94908213,35.65118625)(547.41408167,35.47618643)(547.87407959,35.28619436)
\curveto(548.05408103,35.2061867)(548.23408085,35.13618677)(548.41407959,35.07619436)
\curveto(548.59408049,35.01618689)(548.77408031,34.95118695)(548.95407959,34.88119436)
\curveto(549.06408002,34.83118707)(549.16907991,34.78118712)(549.26907959,34.73119436)
\curveto(549.35907972,34.69118721)(549.42407966,34.6061873)(549.46407959,34.47619436)
\curveto(549.47407961,34.45618745)(549.4790796,34.43118747)(549.47907959,34.40119436)
\curveto(549.46907961,34.38118752)(549.46907961,34.35618755)(549.47907959,34.32619436)
\curveto(549.48907959,34.29618761)(549.49407959,34.26118764)(549.49407959,34.22119436)
\curveto(549.4840796,34.18118772)(549.4790796,34.14118776)(549.47907959,34.10119436)
\lineto(549.47907959,33.80119436)
\curveto(549.4790796,33.7011882)(549.45407963,33.62118828)(549.40407959,33.56119436)
\curveto(549.35407973,33.48118842)(549.2840798,33.42118848)(549.19407959,33.38119436)
\curveto(549.09407999,33.35118855)(548.99408009,33.31118859)(548.89407959,33.26119436)
\curveto(548.69408039,33.18118872)(548.48908059,33.1011888)(548.27907959,33.02119436)
\curveto(548.05908102,32.95118895)(547.84908123,32.87618903)(547.64907959,32.79619436)
\curveto(547.46908161,32.71618919)(547.28908179,32.64618926)(547.10907959,32.58619436)
\curveto(546.91908216,32.53618937)(546.73408235,32.47118943)(546.55407959,32.39119436)
\curveto(545.99408309,32.16118974)(545.42908365,31.94618996)(544.85907959,31.74619436)
\curveto(544.28908479,31.54619036)(543.72408536,31.33119057)(543.16407959,31.10119436)
\lineto(542.53407959,30.86119436)
\curveto(542.31408677,30.79119111)(542.10408698,30.71619119)(541.90407959,30.63619436)
\curveto(541.79408729,30.58619132)(541.68908739,30.54119136)(541.58907959,30.50119436)
\curveto(541.4790876,30.47119143)(541.3840877,30.42119148)(541.30407959,30.35119436)
\curveto(541.2840878,30.34119156)(541.27408781,30.33119157)(541.27407959,30.32119436)
\lineto(541.24407959,30.29119436)
\lineto(541.24407959,30.21619436)
\lineto(541.27407959,30.18619436)
\curveto(541.27408781,30.17619173)(541.2790878,30.16619174)(541.28907959,30.15619436)
\curveto(541.33908774,30.13619177)(541.39408769,30.12619178)(541.45407959,30.12619436)
\curveto(541.51408757,30.12619178)(541.57408751,30.11619179)(541.63407959,30.09619436)
\lineto(541.79907959,30.09619436)
\curveto(541.85908722,30.07619183)(541.92408716,30.07119183)(541.99407959,30.08119436)
\curveto(542.06408702,30.09119181)(542.13408695,30.09619181)(542.20407959,30.09619436)
\lineto(543.01407959,30.09619436)
\lineto(547.57407959,30.09619436)
\lineto(548.75907959,30.09619436)
\curveto(548.86908021,30.09619181)(548.9790801,30.09119181)(549.08907959,30.08119436)
\curveto(549.19907988,30.08119182)(549.2840798,30.05619185)(549.34407959,30.00619436)
\curveto(549.42407966,29.95619195)(549.46907961,29.86619204)(549.47907959,29.73619436)
\lineto(549.47907959,29.34619436)
\lineto(549.47907959,29.15119436)
\curveto(549.4790796,29.1011928)(549.46907961,29.05119285)(549.44907959,29.00119436)
\curveto(549.40907967,28.87119303)(549.32407976,28.79619311)(549.19407959,28.77619436)
\curveto(549.06408002,28.76619314)(548.91408017,28.76119314)(548.74407959,28.76119436)
\lineto(547.00407959,28.76119436)
\lineto(541.00407959,28.76119436)
\lineto(539.59407959,28.76119436)
\curveto(539.4840896,28.76119314)(539.36908971,28.75619315)(539.24907959,28.74619436)
\curveto(539.12908995,28.74619316)(539.03409005,28.77119313)(538.96407959,28.82119436)
\curveto(538.90409018,28.86119304)(538.85409023,28.93619297)(538.81407959,29.04619436)
\curveto(538.80409028,29.06619284)(538.80409028,29.08619282)(538.81407959,29.10619436)
\curveto(538.81409027,29.13619277)(538.80909027,29.16119274)(538.79907959,29.18119436)
}
}
{
\newrgbcolor{curcolor}{0 0 0}
\pscustom[linestyle=none,fillstyle=solid,fillcolor=curcolor]
{
\newpath
\moveto(548.92407959,48.38330373)
\curveto(549.08408,48.4132959)(549.21907986,48.39829592)(549.32907959,48.33830373)
\curveto(549.42907965,48.27829604)(549.50407958,48.19829612)(549.55407959,48.09830373)
\curveto(549.57407951,48.04829627)(549.5840795,47.99329632)(549.58407959,47.93330373)
\curveto(549.5840795,47.88329643)(549.59407949,47.82829649)(549.61407959,47.76830373)
\curveto(549.66407942,47.54829677)(549.64907943,47.32829699)(549.56907959,47.10830373)
\curveto(549.49907958,46.89829742)(549.40907967,46.75329756)(549.29907959,46.67330373)
\curveto(549.22907985,46.62329769)(549.14907993,46.57829774)(549.05907959,46.53830373)
\curveto(548.95908012,46.49829782)(548.8790802,46.44829787)(548.81907959,46.38830373)
\curveto(548.79908028,46.36829795)(548.7790803,46.34329797)(548.75907959,46.31330373)
\curveto(548.73908034,46.29329802)(548.73408035,46.26329805)(548.74407959,46.22330373)
\curveto(548.77408031,46.1132982)(548.82908025,46.00829831)(548.90907959,45.90830373)
\curveto(548.98908009,45.8182985)(549.05908002,45.72829859)(549.11907959,45.63830373)
\curveto(549.19907988,45.50829881)(549.27407981,45.36829895)(549.34407959,45.21830373)
\curveto(549.40407968,45.06829925)(549.45907962,44.90829941)(549.50907959,44.73830373)
\curveto(549.53907954,44.63829968)(549.55907952,44.52829979)(549.56907959,44.40830373)
\curveto(549.5790795,44.29830002)(549.59407949,44.18830013)(549.61407959,44.07830373)
\curveto(549.62407946,44.02830029)(549.62907945,43.98330033)(549.62907959,43.94330373)
\lineto(549.62907959,43.83830373)
\curveto(549.64907943,43.72830059)(549.64907943,43.62330069)(549.62907959,43.52330373)
\lineto(549.62907959,43.38830373)
\curveto(549.61907946,43.33830098)(549.61407947,43.28830103)(549.61407959,43.23830373)
\curveto(549.61407947,43.18830113)(549.60407948,43.14330117)(549.58407959,43.10330373)
\curveto(549.57407951,43.06330125)(549.56907951,43.02830129)(549.56907959,42.99830373)
\curveto(549.5790795,42.97830134)(549.5790795,42.95330136)(549.56907959,42.92330373)
\lineto(549.50907959,42.68330373)
\curveto(549.49907958,42.60330171)(549.4790796,42.52830179)(549.44907959,42.45830373)
\curveto(549.31907976,42.15830216)(549.17407991,41.9133024)(549.01407959,41.72330373)
\curveto(548.84408024,41.54330277)(548.60908047,41.39330292)(548.30907959,41.27330373)
\curveto(548.08908099,41.18330313)(547.82408126,41.13830318)(547.51407959,41.13830373)
\lineto(547.19907959,41.13830373)
\curveto(547.14908193,41.14830317)(547.09908198,41.15330316)(547.04907959,41.15330373)
\lineto(546.86907959,41.18330373)
\lineto(546.53907959,41.30330373)
\curveto(546.42908265,41.34330297)(546.32908275,41.39330292)(546.23907959,41.45330373)
\curveto(545.94908313,41.63330268)(545.73408335,41.87830244)(545.59407959,42.18830373)
\curveto(545.45408363,42.49830182)(545.32908375,42.83830148)(545.21907959,43.20830373)
\curveto(545.1790839,43.34830097)(545.14908393,43.49330082)(545.12907959,43.64330373)
\curveto(545.10908397,43.79330052)(545.084084,43.94330037)(545.05407959,44.09330373)
\curveto(545.03408405,44.16330015)(545.02408406,44.22830009)(545.02407959,44.28830373)
\curveto(545.02408406,44.35829996)(545.01408407,44.43329988)(544.99407959,44.51330373)
\curveto(544.97408411,44.58329973)(544.96408412,44.65329966)(544.96407959,44.72330373)
\curveto(544.95408413,44.79329952)(544.93908414,44.86829945)(544.91907959,44.94830373)
\curveto(544.85908422,45.19829912)(544.80908427,45.43329888)(544.76907959,45.65330373)
\curveto(544.71908436,45.87329844)(544.60408448,46.04829827)(544.42407959,46.17830373)
\curveto(544.34408474,46.23829808)(544.24408484,46.28829803)(544.12407959,46.32830373)
\curveto(543.99408509,46.36829795)(543.85408523,46.36829795)(543.70407959,46.32830373)
\curveto(543.46408562,46.26829805)(543.27408581,46.17829814)(543.13407959,46.05830373)
\curveto(542.99408609,45.94829837)(542.8840862,45.78829853)(542.80407959,45.57830373)
\curveto(542.75408633,45.45829886)(542.71908636,45.313299)(542.69907959,45.14330373)
\curveto(542.6790864,44.98329933)(542.66908641,44.8132995)(542.66907959,44.63330373)
\curveto(542.66908641,44.45329986)(542.6790864,44.27830004)(542.69907959,44.10830373)
\curveto(542.71908636,43.93830038)(542.74908633,43.79330052)(542.78907959,43.67330373)
\curveto(542.84908623,43.50330081)(542.93408615,43.33830098)(543.04407959,43.17830373)
\curveto(543.10408598,43.09830122)(543.1840859,43.02330129)(543.28407959,42.95330373)
\curveto(543.37408571,42.89330142)(543.47408561,42.83830148)(543.58407959,42.78830373)
\curveto(543.66408542,42.75830156)(543.74908533,42.72830159)(543.83907959,42.69830373)
\curveto(543.92908515,42.67830164)(543.99908508,42.63330168)(544.04907959,42.56330373)
\curveto(544.079085,42.52330179)(544.10408498,42.45330186)(544.12407959,42.35330373)
\curveto(544.13408495,42.26330205)(544.13908494,42.16830215)(544.13907959,42.06830373)
\curveto(544.13908494,41.96830235)(544.13408495,41.86830245)(544.12407959,41.76830373)
\curveto(544.10408498,41.67830264)(544.079085,41.6133027)(544.04907959,41.57330373)
\curveto(544.01908506,41.53330278)(543.96908511,41.50330281)(543.89907959,41.48330373)
\curveto(543.82908525,41.46330285)(543.75408533,41.46330285)(543.67407959,41.48330373)
\curveto(543.54408554,41.5133028)(543.42408566,41.54330277)(543.31407959,41.57330373)
\curveto(543.19408589,41.6133027)(543.079086,41.65830266)(542.96907959,41.70830373)
\curveto(542.61908646,41.89830242)(542.34908673,42.13830218)(542.15907959,42.42830373)
\curveto(541.95908712,42.7183016)(541.79908728,43.07830124)(541.67907959,43.50830373)
\curveto(541.65908742,43.60830071)(541.64408744,43.70830061)(541.63407959,43.80830373)
\curveto(541.62408746,43.9183004)(541.60908747,44.02830029)(541.58907959,44.13830373)
\curveto(541.5790875,44.17830014)(541.5790875,44.24330007)(541.58907959,44.33330373)
\curveto(541.58908749,44.42329989)(541.5790875,44.47829984)(541.55907959,44.49830373)
\curveto(541.54908753,45.19829912)(541.62908745,45.80829851)(541.79907959,46.32830373)
\curveto(541.96908711,46.84829747)(542.29408679,47.2132971)(542.77407959,47.42330373)
\curveto(542.97408611,47.5132968)(543.20908587,47.56329675)(543.47907959,47.57330373)
\curveto(543.73908534,47.59329672)(544.01408507,47.60329671)(544.30407959,47.60330373)
\lineto(547.61907959,47.60330373)
\curveto(547.75908132,47.60329671)(547.89408119,47.60829671)(548.02407959,47.61830373)
\curveto(548.15408093,47.62829669)(548.25908082,47.65829666)(548.33907959,47.70830373)
\curveto(548.40908067,47.75829656)(548.45908062,47.82329649)(548.48907959,47.90330373)
\curveto(548.52908055,47.99329632)(548.55908052,48.07829624)(548.57907959,48.15830373)
\curveto(548.58908049,48.23829608)(548.63408045,48.29829602)(548.71407959,48.33830373)
\curveto(548.74408034,48.35829596)(548.77408031,48.36829595)(548.80407959,48.36830373)
\curveto(548.83408025,48.36829595)(548.87408021,48.37329594)(548.92407959,48.38330373)
\moveto(547.25907959,46.23830373)
\curveto(547.11908196,46.29829802)(546.95908212,46.32829799)(546.77907959,46.32830373)
\curveto(546.58908249,46.33829798)(546.39408269,46.34329797)(546.19407959,46.34330373)
\curveto(546.084083,46.34329797)(545.9840831,46.33829798)(545.89407959,46.32830373)
\curveto(545.80408328,46.318298)(545.73408335,46.27829804)(545.68407959,46.20830373)
\curveto(545.66408342,46.17829814)(545.65408343,46.10829821)(545.65407959,45.99830373)
\curveto(545.67408341,45.97829834)(545.6840834,45.94329837)(545.68407959,45.89330373)
\curveto(545.6840834,45.84329847)(545.69408339,45.79829852)(545.71407959,45.75830373)
\curveto(545.73408335,45.67829864)(545.75408333,45.58829873)(545.77407959,45.48830373)
\lineto(545.83407959,45.18830373)
\curveto(545.83408325,45.15829916)(545.83908324,45.12329919)(545.84907959,45.08330373)
\lineto(545.84907959,44.97830373)
\curveto(545.88908319,44.82829949)(545.91408317,44.66329965)(545.92407959,44.48330373)
\curveto(545.92408316,44.3133)(545.94408314,44.15330016)(545.98407959,44.00330373)
\curveto(546.00408308,43.92330039)(546.02408306,43.84830047)(546.04407959,43.77830373)
\curveto(546.05408303,43.7183006)(546.06908301,43.64830067)(546.08907959,43.56830373)
\curveto(546.13908294,43.40830091)(546.20408288,43.25830106)(546.28407959,43.11830373)
\curveto(546.35408273,42.97830134)(546.44408264,42.85830146)(546.55407959,42.75830373)
\curveto(546.66408242,42.65830166)(546.79908228,42.58330173)(546.95907959,42.53330373)
\curveto(547.10908197,42.48330183)(547.29408179,42.46330185)(547.51407959,42.47330373)
\curveto(547.61408147,42.47330184)(547.70908137,42.48830183)(547.79907959,42.51830373)
\curveto(547.8790812,42.55830176)(547.95408113,42.60330171)(548.02407959,42.65330373)
\curveto(548.13408095,42.73330158)(548.22908085,42.83830148)(548.30907959,42.96830373)
\curveto(548.3790807,43.09830122)(548.43908064,43.23830108)(548.48907959,43.38830373)
\curveto(548.49908058,43.43830088)(548.50408058,43.48830083)(548.50407959,43.53830373)
\curveto(548.50408058,43.58830073)(548.50908057,43.63830068)(548.51907959,43.68830373)
\curveto(548.53908054,43.75830056)(548.55408053,43.84330047)(548.56407959,43.94330373)
\curveto(548.56408052,44.05330026)(548.55408053,44.14330017)(548.53407959,44.21330373)
\curveto(548.51408057,44.27330004)(548.50908057,44.33329998)(548.51907959,44.39330373)
\curveto(548.51908056,44.45329986)(548.50908057,44.5132998)(548.48907959,44.57330373)
\curveto(548.46908061,44.65329966)(548.45408063,44.72829959)(548.44407959,44.79830373)
\curveto(548.43408065,44.87829944)(548.41408067,44.95329936)(548.38407959,45.02330373)
\curveto(548.26408082,45.313299)(548.11908096,45.55829876)(547.94907959,45.75830373)
\curveto(547.7790813,45.96829835)(547.54908153,46.12829819)(547.25907959,46.23830373)
}
}
{
\newrgbcolor{curcolor}{0 0 0}
\pscustom[linestyle=none,fillstyle=solid,fillcolor=curcolor]
{
\newpath
\moveto(541.75407959,49.26994436)
\lineto(541.75407959,49.71994436)
\curveto(541.74408734,49.88994311)(541.76408732,50.01494298)(541.81407959,50.09494436)
\curveto(541.86408722,50.17494282)(541.92908715,50.22994277)(542.00907959,50.25994436)
\curveto(542.08908699,50.2999427)(542.17408691,50.33994266)(542.26407959,50.37994436)
\curveto(542.39408669,50.42994257)(542.52408656,50.47494252)(542.65407959,50.51494436)
\curveto(542.7840863,50.55494244)(542.91408617,50.5999424)(543.04407959,50.64994436)
\curveto(543.16408592,50.6999423)(543.28908579,50.74494225)(543.41907959,50.78494436)
\curveto(543.53908554,50.82494217)(543.65908542,50.86994213)(543.77907959,50.91994436)
\curveto(543.88908519,50.96994203)(544.00408508,51.00994199)(544.12407959,51.03994436)
\curveto(544.24408484,51.06994193)(544.36408472,51.10994189)(544.48407959,51.15994436)
\curveto(544.77408431,51.27994172)(545.07408401,51.38994161)(545.38407959,51.48994436)
\curveto(545.69408339,51.58994141)(545.99408309,51.6999413)(546.28407959,51.81994436)
\curveto(546.32408276,51.83994116)(546.36408272,51.84994115)(546.40407959,51.84994436)
\curveto(546.43408265,51.84994115)(546.46408262,51.85994114)(546.49407959,51.87994436)
\curveto(546.63408245,51.93994106)(546.7790823,51.994941)(546.92907959,52.04494436)
\lineto(547.34907959,52.19494436)
\curveto(547.41908166,52.22494077)(547.49408159,52.25494074)(547.57407959,52.28494436)
\curveto(547.64408144,52.31494068)(547.68908139,52.36494063)(547.70907959,52.43494436)
\curveto(547.73908134,52.51494048)(547.71408137,52.57494042)(547.63407959,52.61494436)
\curveto(547.54408154,52.66494033)(547.47408161,52.6999403)(547.42407959,52.71994436)
\curveto(547.25408183,52.7999402)(547.07408201,52.86494013)(546.88407959,52.91494436)
\curveto(546.69408239,52.96494003)(546.50908257,53.02493997)(546.32907959,53.09494436)
\curveto(546.09908298,53.18493981)(545.86908321,53.26493973)(545.63907959,53.33494436)
\curveto(545.39908368,53.40493959)(545.16908391,53.48993951)(544.94907959,53.58994436)
\curveto(544.89908418,53.5999394)(544.83408425,53.61493938)(544.75407959,53.63494436)
\curveto(544.66408442,53.67493932)(544.57408451,53.70993929)(544.48407959,53.73994436)
\curveto(544.3840847,53.76993923)(544.29408479,53.7999392)(544.21407959,53.82994436)
\curveto(544.16408492,53.84993915)(544.11908496,53.86493913)(544.07907959,53.87494436)
\curveto(544.03908504,53.88493911)(543.99408509,53.8999391)(543.94407959,53.91994436)
\curveto(543.82408526,53.96993903)(543.70408538,54.00993899)(543.58407959,54.03994436)
\curveto(543.45408563,54.07993892)(543.32908575,54.12493887)(543.20907959,54.17494436)
\curveto(543.15908592,54.1949388)(543.11408597,54.20993879)(543.07407959,54.21994436)
\curveto(543.03408605,54.22993877)(542.98908609,54.24493875)(542.93907959,54.26494436)
\curveto(542.84908623,54.30493869)(542.75908632,54.33993866)(542.66907959,54.36994436)
\curveto(542.56908651,54.3999386)(542.47408661,54.42993857)(542.38407959,54.45994436)
\curveto(542.30408678,54.48993851)(542.22408686,54.51493848)(542.14407959,54.53494436)
\curveto(542.05408703,54.56493843)(541.9790871,54.60493839)(541.91907959,54.65494436)
\curveto(541.82908725,54.72493827)(541.7790873,54.81993818)(541.76907959,54.93994436)
\curveto(541.75908732,55.06993793)(541.75408733,55.20993779)(541.75407959,55.35994436)
\curveto(541.75408733,55.43993756)(541.75908732,55.51493748)(541.76907959,55.58494436)
\curveto(541.76908731,55.66493733)(541.7840873,55.72993727)(541.81407959,55.77994436)
\curveto(541.87408721,55.86993713)(541.96908711,55.8949371)(542.09907959,55.85494436)
\curveto(542.22908685,55.81493718)(542.32908675,55.77993722)(542.39907959,55.74994436)
\lineto(542.45907959,55.71994436)
\curveto(542.4790866,55.71993728)(542.49908658,55.71493728)(542.51907959,55.70494436)
\curveto(542.79908628,55.5949374)(543.084086,55.48493751)(543.37407959,55.37494436)
\lineto(544.21407959,55.04494436)
\curveto(544.29408479,55.01493798)(544.36908471,54.98993801)(544.43907959,54.96994436)
\curveto(544.49908458,54.94993805)(544.56408452,54.92493807)(544.63407959,54.89494436)
\curveto(544.83408425,54.81493818)(545.03908404,54.73493826)(545.24907959,54.65494436)
\curveto(545.44908363,54.58493841)(545.64908343,54.50993849)(545.84907959,54.42994436)
\curveto(546.53908254,54.13993886)(547.23408185,53.86993913)(547.93407959,53.61994436)
\curveto(548.63408045,53.36993963)(549.32907975,53.0999399)(550.01907959,52.80994436)
\lineto(550.16907959,52.74994436)
\curveto(550.22907885,52.73994026)(550.28907879,52.72494027)(550.34907959,52.70494436)
\curveto(550.71907836,52.54494045)(551.084078,52.37494062)(551.44407959,52.19494436)
\curveto(551.81407727,52.01494098)(552.09907698,51.76494123)(552.29907959,51.44494436)
\curveto(552.35907672,51.33494166)(552.40407668,51.22494177)(552.43407959,51.11494436)
\curveto(552.47407661,51.00494199)(552.50907657,50.87994212)(552.53907959,50.73994436)
\curveto(552.55907652,50.68994231)(552.56407652,50.63494236)(552.55407959,50.57494436)
\curveto(552.54407654,50.52494247)(552.54407654,50.46994253)(552.55407959,50.40994436)
\curveto(552.57407651,50.32994267)(552.57407651,50.24994275)(552.55407959,50.16994436)
\curveto(552.54407654,50.12994287)(552.53907654,50.07994292)(552.53907959,50.01994436)
\lineto(552.47907959,49.77994436)
\curveto(552.45907662,49.70994329)(552.41907666,49.65494334)(552.35907959,49.61494436)
\curveto(552.29907678,49.56494343)(552.22407686,49.53494346)(552.13407959,49.52494436)
\lineto(551.86407959,49.52494436)
\lineto(551.65407959,49.52494436)
\curveto(551.59407749,49.53494346)(551.54407754,49.55494344)(551.50407959,49.58494436)
\curveto(551.39407769,49.65494334)(551.36407772,49.77494322)(551.41407959,49.94494436)
\curveto(551.43407765,50.05494294)(551.44407764,50.17494282)(551.44407959,50.30494436)
\curveto(551.44407764,50.43494256)(551.42407766,50.54994245)(551.38407959,50.64994436)
\curveto(551.33407775,50.7999422)(551.25907782,50.91994208)(551.15907959,51.00994436)
\curveto(551.05907802,51.10994189)(550.94407814,51.1949418)(550.81407959,51.26494436)
\curveto(550.69407839,51.33494166)(550.56407852,51.3949416)(550.42407959,51.44494436)
\lineto(550.00407959,51.62494436)
\curveto(549.91407917,51.66494133)(549.80407928,51.70494129)(549.67407959,51.74494436)
\curveto(549.54407954,51.7949412)(549.40907967,51.7999412)(549.26907959,51.75994436)
\curveto(549.10907997,51.70994129)(548.95908012,51.65494134)(548.81907959,51.59494436)
\curveto(548.6790804,51.54494145)(548.53908054,51.48994151)(548.39907959,51.42994436)
\curveto(548.18908089,51.33994166)(547.9790811,51.25494174)(547.76907959,51.17494436)
\curveto(547.55908152,51.0949419)(547.35408173,51.01494198)(547.15407959,50.93494436)
\curveto(547.01408207,50.87494212)(546.8790822,50.81994218)(546.74907959,50.76994436)
\curveto(546.61908246,50.71994228)(546.4840826,50.66994233)(546.34407959,50.61994436)
\lineto(545.02407959,50.07994436)
\curveto(544.5840845,49.90994309)(544.14408494,49.73494326)(543.70407959,49.55494436)
\curveto(543.47408561,49.45494354)(543.25408583,49.36494363)(543.04407959,49.28494436)
\curveto(542.82408626,49.20494379)(542.60408648,49.11994388)(542.38407959,49.02994436)
\curveto(542.32408676,49.00994399)(542.24408684,48.97994402)(542.14407959,48.93994436)
\curveto(542.03408705,48.8999441)(541.94408714,48.90494409)(541.87407959,48.95494436)
\curveto(541.82408726,48.98494401)(541.78908729,49.04494395)(541.76907959,49.13494436)
\curveto(541.75908732,49.15494384)(541.75908732,49.17494382)(541.76907959,49.19494436)
\curveto(541.76908731,49.22494377)(541.76408732,49.24994375)(541.75407959,49.26994436)
}
}
{
\newrgbcolor{curcolor}{0 0 0}
\pscustom[linestyle=none,fillstyle=solid,fillcolor=curcolor]
{
}
}
{
\newrgbcolor{curcolor}{0 0 0}
\pscustom[linestyle=none,fillstyle=solid,fillcolor=curcolor]
{
\newpath
\moveto(538.87407959,65.05510061)
\curveto(538.87409021,65.15509575)(538.8840902,65.25009566)(538.90407959,65.34010061)
\curveto(538.91409017,65.43009548)(538.94409014,65.49509541)(538.99407959,65.53510061)
\curveto(539.07409001,65.59509531)(539.1790899,65.62509528)(539.30907959,65.62510061)
\lineto(539.69907959,65.62510061)
\lineto(541.19907959,65.62510061)
\lineto(547.58907959,65.62510061)
\lineto(548.75907959,65.62510061)
\lineto(549.07407959,65.62510061)
\curveto(549.17407991,65.63509527)(549.25407983,65.62009529)(549.31407959,65.58010061)
\curveto(549.39407969,65.53009538)(549.44407964,65.45509545)(549.46407959,65.35510061)
\curveto(549.47407961,65.26509564)(549.4790796,65.15509575)(549.47907959,65.02510061)
\lineto(549.47907959,64.80010061)
\curveto(549.45907962,64.72009619)(549.44407964,64.65009626)(549.43407959,64.59010061)
\curveto(549.41407967,64.53009638)(549.37407971,64.48009643)(549.31407959,64.44010061)
\curveto(549.25407983,64.40009651)(549.1790799,64.38009653)(549.08907959,64.38010061)
\lineto(548.78907959,64.38010061)
\lineto(547.69407959,64.38010061)
\lineto(542.35407959,64.38010061)
\curveto(542.26408682,64.36009655)(542.18908689,64.34509656)(542.12907959,64.33510061)
\curveto(542.05908702,64.33509657)(541.99908708,64.3050966)(541.94907959,64.24510061)
\curveto(541.89908718,64.17509673)(541.87408721,64.08509682)(541.87407959,63.97510061)
\curveto(541.86408722,63.87509703)(541.85908722,63.76509714)(541.85907959,63.64510061)
\lineto(541.85907959,62.50510061)
\lineto(541.85907959,62.01010061)
\curveto(541.84908723,61.85009906)(541.78908729,61.74009917)(541.67907959,61.68010061)
\curveto(541.64908743,61.66009925)(541.61908746,61.65009926)(541.58907959,61.65010061)
\curveto(541.54908753,61.65009926)(541.50408758,61.64509926)(541.45407959,61.63510061)
\curveto(541.33408775,61.61509929)(541.22408786,61.62009929)(541.12407959,61.65010061)
\curveto(541.02408806,61.69009922)(540.95408813,61.74509916)(540.91407959,61.81510061)
\curveto(540.86408822,61.89509901)(540.83908824,62.01509889)(540.83907959,62.17510061)
\curveto(540.83908824,62.33509857)(540.82408826,62.47009844)(540.79407959,62.58010061)
\curveto(540.7840883,62.63009828)(540.7790883,62.68509822)(540.77907959,62.74510061)
\curveto(540.76908831,62.8050981)(540.75408833,62.86509804)(540.73407959,62.92510061)
\curveto(540.6840884,63.07509783)(540.63408845,63.22009769)(540.58407959,63.36010061)
\curveto(540.52408856,63.50009741)(540.45408863,63.63509727)(540.37407959,63.76510061)
\curveto(540.2840888,63.905097)(540.1790889,64.02509688)(540.05907959,64.12510061)
\curveto(539.93908914,64.22509668)(539.80908927,64.32009659)(539.66907959,64.41010061)
\curveto(539.56908951,64.47009644)(539.45908962,64.51509639)(539.33907959,64.54510061)
\curveto(539.21908986,64.58509632)(539.11408997,64.63509627)(539.02407959,64.69510061)
\curveto(538.96409012,64.74509616)(538.92409016,64.81509609)(538.90407959,64.90510061)
\curveto(538.89409019,64.92509598)(538.88909019,64.95009596)(538.88907959,64.98010061)
\curveto(538.88909019,65.0100959)(538.8840902,65.03509587)(538.87407959,65.05510061)
}
}
{
\newrgbcolor{curcolor}{0 0 0}
\pscustom[linestyle=none,fillstyle=solid,fillcolor=curcolor]
{
\newpath
\moveto(538.87407959,72.45970998)
\curveto(538.84409024,74.08970454)(539.39908968,75.13970349)(540.53907959,75.60970998)
\curveto(540.76908831,75.70970292)(541.05908802,75.77470286)(541.40907959,75.80470998)
\curveto(541.74908733,75.84470279)(542.05908702,75.81970281)(542.33907959,75.72970998)
\curveto(542.59908648,75.63970299)(542.82408626,75.51970311)(543.01407959,75.36970998)
\curveto(543.05408603,75.34970328)(543.08908599,75.32470331)(543.11907959,75.29470998)
\curveto(543.13908594,75.26470337)(543.16408592,75.23970339)(543.19407959,75.21970998)
\lineto(543.31407959,75.12970998)
\curveto(543.34408574,75.09970353)(543.36908571,75.06470357)(543.38907959,75.02470998)
\curveto(543.43908564,74.97470366)(543.4840856,74.91970371)(543.52407959,74.85970998)
\curveto(543.56408552,74.80970382)(543.61408547,74.76470387)(543.67407959,74.72470998)
\curveto(543.71408537,74.68470395)(543.76408532,74.66970396)(543.82407959,74.67970998)
\curveto(543.87408521,74.68970394)(543.91908516,74.71970391)(543.95907959,74.76970998)
\curveto(543.99908508,74.81970381)(544.03908504,74.87470376)(544.07907959,74.93470998)
\curveto(544.10908497,75.00470363)(544.13908494,75.06970356)(544.16907959,75.12970998)
\curveto(544.19908488,75.18970344)(544.22908485,75.23970339)(544.25907959,75.27970998)
\curveto(544.4790846,75.59970303)(544.78908429,75.85470278)(545.18907959,76.04470998)
\curveto(545.2790838,76.08470255)(545.37408371,76.11470252)(545.47407959,76.13470998)
\curveto(545.56408352,76.16470247)(545.65408343,76.18970244)(545.74407959,76.20970998)
\curveto(545.79408329,76.21970241)(545.84408324,76.22470241)(545.89407959,76.22470998)
\curveto(545.93408315,76.2347024)(545.9790831,76.24470239)(546.02907959,76.25470998)
\curveto(546.079083,76.26470237)(546.12908295,76.26470237)(546.17907959,76.25470998)
\curveto(546.22908285,76.24470239)(546.2790828,76.24970238)(546.32907959,76.26970998)
\curveto(546.3790827,76.27970235)(546.43908264,76.28470235)(546.50907959,76.28470998)
\curveto(546.5790825,76.28470235)(546.63908244,76.27470236)(546.68907959,76.25470998)
\lineto(546.91407959,76.25470998)
\lineto(547.15407959,76.19470998)
\curveto(547.22408186,76.18470245)(547.29408179,76.16970246)(547.36407959,76.14970998)
\curveto(547.45408163,76.11970251)(547.53908154,76.08970254)(547.61907959,76.05970998)
\curveto(547.69908138,76.03970259)(547.7790813,76.00970262)(547.85907959,75.96970998)
\curveto(547.91908116,75.94970268)(547.9790811,75.91970271)(548.03907959,75.87970998)
\curveto(548.08908099,75.84970278)(548.13908094,75.81470282)(548.18907959,75.77470998)
\curveto(548.49908058,75.57470306)(548.75908032,75.32470331)(548.96907959,75.02470998)
\curveto(549.16907991,74.72470391)(549.33407975,74.37970425)(549.46407959,73.98970998)
\curveto(549.50407958,73.86970476)(549.52907955,73.73970489)(549.53907959,73.59970998)
\curveto(549.55907952,73.46970516)(549.5840795,73.3347053)(549.61407959,73.19470998)
\curveto(549.62407946,73.12470551)(549.62907945,73.05470558)(549.62907959,72.98470998)
\curveto(549.62907945,72.92470571)(549.63407945,72.85970577)(549.64407959,72.78970998)
\curveto(549.65407943,72.74970588)(549.65907942,72.68970594)(549.65907959,72.60970998)
\curveto(549.65907942,72.53970609)(549.65407943,72.48970614)(549.64407959,72.45970998)
\curveto(549.63407945,72.40970622)(549.62907945,72.36470627)(549.62907959,72.32470998)
\lineto(549.62907959,72.20470998)
\curveto(549.60907947,72.10470653)(549.59407949,72.00470663)(549.58407959,71.90470998)
\curveto(549.57407951,71.80470683)(549.55907952,71.70970692)(549.53907959,71.61970998)
\curveto(549.50907957,71.50970712)(549.4840796,71.39970723)(549.46407959,71.28970998)
\curveto(549.43407965,71.18970744)(549.39407969,71.08470755)(549.34407959,70.97470998)
\curveto(549.1840799,70.60470803)(548.9840801,70.28970834)(548.74407959,70.02970998)
\curveto(548.49408059,69.76970886)(548.1840809,69.55970907)(547.81407959,69.39970998)
\curveto(547.72408136,69.35970927)(547.62908145,69.32470931)(547.52907959,69.29470998)
\curveto(547.42908165,69.26470937)(547.32408176,69.2347094)(547.21407959,69.20470998)
\curveto(547.16408192,69.18470945)(547.11408197,69.17470946)(547.06407959,69.17470998)
\curveto(547.00408208,69.17470946)(546.94408214,69.16470947)(546.88407959,69.14470998)
\curveto(546.82408226,69.12470951)(546.74408234,69.11470952)(546.64407959,69.11470998)
\curveto(546.54408254,69.11470952)(546.46908261,69.1297095)(546.41907959,69.15970998)
\curveto(546.38908269,69.16970946)(546.36408272,69.18470945)(546.34407959,69.20470998)
\lineto(546.28407959,69.26470998)
\curveto(546.26408282,69.30470933)(546.24908283,69.36470927)(546.23907959,69.44470998)
\curveto(546.22908285,69.5347091)(546.22408286,69.62470901)(546.22407959,69.71470998)
\curveto(546.22408286,69.80470883)(546.22908285,69.88970874)(546.23907959,69.96970998)
\curveto(546.24908283,70.05970857)(546.25908282,70.12470851)(546.26907959,70.16470998)
\curveto(546.28908279,70.18470845)(546.30408278,70.20470843)(546.31407959,70.22470998)
\curveto(546.31408277,70.24470839)(546.32408276,70.26470837)(546.34407959,70.28470998)
\curveto(546.43408265,70.35470828)(546.54908253,70.39470824)(546.68907959,70.40470998)
\curveto(546.82908225,70.42470821)(546.95408213,70.45470818)(547.06407959,70.49470998)
\lineto(547.42407959,70.64470998)
\curveto(547.53408155,70.69470794)(547.63908144,70.75970787)(547.73907959,70.83970998)
\curveto(547.76908131,70.85970777)(547.79408129,70.87970775)(547.81407959,70.89970998)
\curveto(547.83408125,70.9297077)(547.85908122,70.95470768)(547.88907959,70.97470998)
\curveto(547.94908113,71.01470762)(547.99408109,71.04970758)(548.02407959,71.07970998)
\curveto(548.05408103,71.11970751)(548.084081,71.15470748)(548.11407959,71.18470998)
\curveto(548.14408094,71.22470741)(548.17408091,71.26970736)(548.20407959,71.31970998)
\curveto(548.26408082,71.40970722)(548.31408077,71.50470713)(548.35407959,71.60470998)
\lineto(548.47407959,71.93470998)
\curveto(548.52408056,72.08470655)(548.55408053,72.28470635)(548.56407959,72.53470998)
\curveto(548.57408051,72.78470585)(548.55408053,72.99470564)(548.50407959,73.16470998)
\curveto(548.4840806,73.24470539)(548.46908061,73.31470532)(548.45907959,73.37470998)
\lineto(548.39907959,73.58470998)
\curveto(548.2790808,73.86470477)(548.12908095,74.10470453)(547.94907959,74.30470998)
\curveto(547.76908131,74.51470412)(547.53908154,74.67970395)(547.25907959,74.79970998)
\curveto(547.18908189,74.8297038)(547.11908196,74.84970378)(547.04907959,74.85970998)
\lineto(546.80907959,74.91970998)
\curveto(546.66908241,74.95970367)(546.50908257,74.96970366)(546.32907959,74.94970998)
\curveto(546.13908294,74.9297037)(545.98908309,74.89970373)(545.87907959,74.85970998)
\curveto(545.49908358,74.7297039)(545.20908387,74.54470409)(545.00907959,74.30470998)
\curveto(544.80908427,74.07470456)(544.64908443,73.76470487)(544.52907959,73.37470998)
\curveto(544.49908458,73.26470537)(544.4790846,73.14470549)(544.46907959,73.01470998)
\curveto(544.45908462,72.89470574)(544.45408463,72.76970586)(544.45407959,72.63970998)
\curveto(544.45408463,72.47970615)(544.44908463,72.33970629)(544.43907959,72.21970998)
\curveto(544.42908465,72.09970653)(544.36908471,72.01470662)(544.25907959,71.96470998)
\curveto(544.22908485,71.94470669)(544.19408489,71.9347067)(544.15407959,71.93470998)
\lineto(544.01907959,71.93470998)
\curveto(543.91908516,71.92470671)(543.82408526,71.92470671)(543.73407959,71.93470998)
\curveto(543.64408544,71.95470668)(543.5790855,71.99470664)(543.53907959,72.05470998)
\curveto(543.50908557,72.09470654)(543.48908559,72.1347065)(543.47907959,72.17470998)
\curveto(543.46908561,72.22470641)(543.45908562,72.27970635)(543.44907959,72.33970998)
\curveto(543.43908564,72.35970627)(543.43908564,72.38470625)(543.44907959,72.41470998)
\curveto(543.44908563,72.44470619)(543.44408564,72.46970616)(543.43407959,72.48970998)
\lineto(543.43407959,72.62470998)
\curveto(543.41408567,72.7347059)(543.40408568,72.8347058)(543.40407959,72.92470998)
\curveto(543.39408569,73.02470561)(543.37408571,73.11970551)(543.34407959,73.20970998)
\curveto(543.23408585,73.5297051)(543.08908599,73.78470485)(542.90907959,73.97470998)
\curveto(542.72908635,74.16470447)(542.4790866,74.31470432)(542.15907959,74.42470998)
\curveto(542.05908702,74.45470418)(541.93408715,74.47470416)(541.78407959,74.48470998)
\curveto(541.62408746,74.50470413)(541.4790876,74.49970413)(541.34907959,74.46970998)
\curveto(541.2790878,74.44970418)(541.21408787,74.4297042)(541.15407959,74.40970998)
\curveto(541.084088,74.39970423)(541.01908806,74.37970425)(540.95907959,74.34970998)
\curveto(540.71908836,74.24970438)(540.52908855,74.10470453)(540.38907959,73.91470998)
\curveto(540.24908883,73.72470491)(540.13908894,73.49970513)(540.05907959,73.23970998)
\curveto(540.03908904,73.17970545)(540.02908905,73.11970551)(540.02907959,73.05970998)
\curveto(540.02908905,72.99970563)(540.01908906,72.9347057)(539.99907959,72.86470998)
\curveto(539.9790891,72.78470585)(539.96908911,72.68970594)(539.96907959,72.57970998)
\curveto(539.96908911,72.46970616)(539.9790891,72.37470626)(539.99907959,72.29470998)
\curveto(540.01908906,72.24470639)(540.02908905,72.19470644)(540.02907959,72.14470998)
\curveto(540.02908905,72.10470653)(540.03908904,72.05970657)(540.05907959,72.00970998)
\curveto(540.10908897,71.8297068)(540.1840889,71.65970697)(540.28407959,71.49970998)
\curveto(540.37408871,71.34970728)(540.48908859,71.21970741)(540.62907959,71.10970998)
\curveto(540.74908833,71.01970761)(540.8790882,70.93970769)(541.01907959,70.86970998)
\curveto(541.15908792,70.79970783)(541.31408777,70.7347079)(541.48407959,70.67470998)
\curveto(541.59408749,70.64470799)(541.71408737,70.62470801)(541.84407959,70.61470998)
\curveto(541.96408712,70.60470803)(542.06408702,70.56970806)(542.14407959,70.50970998)
\curveto(542.1840869,70.48970814)(542.22408686,70.4297082)(542.26407959,70.32970998)
\curveto(542.27408681,70.28970834)(542.2840868,70.2297084)(542.29407959,70.14970998)
\lineto(542.29407959,69.89470998)
\curveto(542.2840868,69.80470883)(542.27408681,69.71970891)(542.26407959,69.63970998)
\curveto(542.25408683,69.56970906)(542.23908684,69.51970911)(542.21907959,69.48970998)
\curveto(542.18908689,69.44970918)(542.13408695,69.41470922)(542.05407959,69.38470998)
\curveto(541.97408711,69.35470928)(541.88908719,69.34970928)(541.79907959,69.36970998)
\curveto(541.74908733,69.37970925)(541.69908738,69.38470925)(541.64907959,69.38470998)
\lineto(541.46907959,69.41470998)
\curveto(541.36908771,69.44470919)(541.26908781,69.46970916)(541.16907959,69.48970998)
\curveto(541.06908801,69.51970911)(540.9790881,69.55470908)(540.89907959,69.59470998)
\curveto(540.78908829,69.64470899)(540.6840884,69.68970894)(540.58407959,69.72970998)
\curveto(540.47408861,69.76970886)(540.36908871,69.81970881)(540.26907959,69.87970998)
\curveto(539.72908935,70.20970842)(539.33408975,70.67970795)(539.08407959,71.28970998)
\curveto(539.03409005,71.40970722)(538.99909008,71.5347071)(538.97907959,71.66470998)
\curveto(538.95909012,71.80470683)(538.93409015,71.94470669)(538.90407959,72.08470998)
\curveto(538.89409019,72.14470649)(538.88909019,72.20470643)(538.88907959,72.26470998)
\curveto(538.88909019,72.3347063)(538.8840902,72.39970623)(538.87407959,72.45970998)
}
}
{
\newrgbcolor{curcolor}{0 0 0}
\pscustom[linestyle=none,fillstyle=solid,fillcolor=curcolor]
{
\newpath
\moveto(547.84407959,78.63431936)
\lineto(547.84407959,79.26431936)
\lineto(547.84407959,79.45931936)
\curveto(547.84408124,79.52931683)(547.85408123,79.58931677)(547.87407959,79.63931936)
\curveto(547.91408117,79.70931665)(547.95408113,79.7593166)(547.99407959,79.78931936)
\curveto(548.04408104,79.82931653)(548.10908097,79.84931651)(548.18907959,79.84931936)
\curveto(548.26908081,79.8593165)(548.35408073,79.86431649)(548.44407959,79.86431936)
\lineto(549.16407959,79.86431936)
\curveto(549.64407944,79.86431649)(550.05407903,79.80431655)(550.39407959,79.68431936)
\curveto(550.73407835,79.56431679)(551.00907807,79.36931699)(551.21907959,79.09931936)
\curveto(551.26907781,79.02931733)(551.31407777,78.9593174)(551.35407959,78.88931936)
\curveto(551.40407768,78.82931753)(551.44907763,78.7543176)(551.48907959,78.66431936)
\curveto(551.49907758,78.64431771)(551.50907757,78.61431774)(551.51907959,78.57431936)
\curveto(551.53907754,78.53431782)(551.54407754,78.48931787)(551.53407959,78.43931936)
\curveto(551.50407758,78.34931801)(551.42907765,78.29431806)(551.30907959,78.27431936)
\curveto(551.19907788,78.2543181)(551.10407798,78.26931809)(551.02407959,78.31931936)
\curveto(550.95407813,78.34931801)(550.88907819,78.39431796)(550.82907959,78.45431936)
\curveto(550.7790783,78.52431783)(550.72907835,78.58931777)(550.67907959,78.64931936)
\curveto(550.62907845,78.71931764)(550.55407853,78.77931758)(550.45407959,78.82931936)
\curveto(550.36407872,78.88931747)(550.27407881,78.93931742)(550.18407959,78.97931936)
\curveto(550.15407893,78.99931736)(550.09407899,79.02431733)(550.00407959,79.05431936)
\curveto(549.92407916,79.08431727)(549.85407923,79.08931727)(549.79407959,79.06931936)
\curveto(549.65407943,79.03931732)(549.56407952,78.97931738)(549.52407959,78.88931936)
\curveto(549.49407959,78.80931755)(549.4790796,78.71931764)(549.47907959,78.61931936)
\curveto(549.4790796,78.51931784)(549.45407963,78.43431792)(549.40407959,78.36431936)
\curveto(549.33407975,78.27431808)(549.19407989,78.22931813)(548.98407959,78.22931936)
\lineto(548.42907959,78.22931936)
\lineto(548.20407959,78.22931936)
\curveto(548.12408096,78.23931812)(548.05908102,78.2593181)(548.00907959,78.28931936)
\curveto(547.92908115,78.34931801)(547.8840812,78.41931794)(547.87407959,78.49931936)
\curveto(547.86408122,78.51931784)(547.85908122,78.53931782)(547.85907959,78.55931936)
\curveto(547.85908122,78.58931777)(547.85408123,78.61431774)(547.84407959,78.63431936)
}
}
{
\newrgbcolor{curcolor}{0 0 0}
\pscustom[linestyle=none,fillstyle=solid,fillcolor=curcolor]
{
}
}
{
\newrgbcolor{curcolor}{0 0 0}
\pscustom[linestyle=none,fillstyle=solid,fillcolor=curcolor]
{
\newpath
\moveto(538.87407959,89.26463186)
\curveto(538.86409022,89.95462722)(538.9840901,90.55462662)(539.23407959,91.06463186)
\curveto(539.4840896,91.58462559)(539.81908926,91.9796252)(540.23907959,92.24963186)
\curveto(540.31908876,92.29962488)(540.40908867,92.34462483)(540.50907959,92.38463186)
\curveto(540.59908848,92.42462475)(540.69408839,92.46962471)(540.79407959,92.51963186)
\curveto(540.89408819,92.55962462)(540.99408809,92.58962459)(541.09407959,92.60963186)
\curveto(541.19408789,92.62962455)(541.29908778,92.64962453)(541.40907959,92.66963186)
\curveto(541.45908762,92.68962449)(541.50408758,92.69462448)(541.54407959,92.68463186)
\curveto(541.5840875,92.6746245)(541.62908745,92.6796245)(541.67907959,92.69963186)
\curveto(541.72908735,92.70962447)(541.81408727,92.71462446)(541.93407959,92.71463186)
\curveto(542.04408704,92.71462446)(542.12908695,92.70962447)(542.18907959,92.69963186)
\curveto(542.24908683,92.6796245)(542.30908677,92.66962451)(542.36907959,92.66963186)
\curveto(542.42908665,92.6796245)(542.48908659,92.6746245)(542.54907959,92.65463186)
\curveto(542.68908639,92.61462456)(542.82408626,92.5796246)(542.95407959,92.54963186)
\curveto(543.084086,92.51962466)(543.20908587,92.4796247)(543.32907959,92.42963186)
\curveto(543.46908561,92.36962481)(543.59408549,92.29962488)(543.70407959,92.21963186)
\curveto(543.81408527,92.14962503)(543.92408516,92.0746251)(544.03407959,91.99463186)
\lineto(544.09407959,91.93463186)
\curveto(544.11408497,91.92462525)(544.13408495,91.90962527)(544.15407959,91.88963186)
\curveto(544.31408477,91.76962541)(544.45908462,91.63462554)(544.58907959,91.48463186)
\curveto(544.71908436,91.33462584)(544.84408424,91.174626)(544.96407959,91.00463186)
\curveto(545.1840839,90.69462648)(545.38908369,90.39962678)(545.57907959,90.11963186)
\curveto(545.71908336,89.88962729)(545.85408323,89.65962752)(545.98407959,89.42963186)
\curveto(546.11408297,89.20962797)(546.24908283,88.98962819)(546.38907959,88.76963186)
\curveto(546.55908252,88.51962866)(546.73908234,88.2796289)(546.92907959,88.04963186)
\curveto(547.11908196,87.82962935)(547.34408174,87.63962954)(547.60407959,87.47963186)
\curveto(547.66408142,87.43962974)(547.72408136,87.40462977)(547.78407959,87.37463186)
\curveto(547.83408125,87.34462983)(547.89908118,87.31462986)(547.97907959,87.28463186)
\curveto(548.04908103,87.26462991)(548.10908097,87.25962992)(548.15907959,87.26963186)
\curveto(548.22908085,87.28962989)(548.2840808,87.32462985)(548.32407959,87.37463186)
\curveto(548.35408073,87.42462975)(548.37408071,87.48462969)(548.38407959,87.55463186)
\lineto(548.38407959,87.79463186)
\lineto(548.38407959,88.54463186)
\lineto(548.38407959,91.34963186)
\lineto(548.38407959,92.00963186)
\curveto(548.3840807,92.09962508)(548.38908069,92.18462499)(548.39907959,92.26463186)
\curveto(548.39908068,92.34462483)(548.41908066,92.40962477)(548.45907959,92.45963186)
\curveto(548.49908058,92.50962467)(548.57408051,92.54962463)(548.68407959,92.57963186)
\curveto(548.7840803,92.61962456)(548.8840802,92.62962455)(548.98407959,92.60963186)
\lineto(549.11907959,92.60963186)
\curveto(549.18907989,92.58962459)(549.24907983,92.56962461)(549.29907959,92.54963186)
\curveto(549.34907973,92.52962465)(549.38907969,92.49462468)(549.41907959,92.44463186)
\curveto(549.45907962,92.39462478)(549.4790796,92.32462485)(549.47907959,92.23463186)
\lineto(549.47907959,91.96463186)
\lineto(549.47907959,91.06463186)
\lineto(549.47907959,87.55463186)
\lineto(549.47907959,86.48963186)
\curveto(549.4790796,86.40963077)(549.4840796,86.31963086)(549.49407959,86.21963186)
\curveto(549.49407959,86.11963106)(549.4840796,86.03463114)(549.46407959,85.96463186)
\curveto(549.39407969,85.75463142)(549.21407987,85.68963149)(548.92407959,85.76963186)
\curveto(548.8840802,85.7796314)(548.84908023,85.7796314)(548.81907959,85.76963186)
\curveto(548.7790803,85.76963141)(548.73408035,85.7796314)(548.68407959,85.79963186)
\curveto(548.60408048,85.81963136)(548.51908056,85.83963134)(548.42907959,85.85963186)
\curveto(548.33908074,85.8796313)(548.25408083,85.90463127)(548.17407959,85.93463186)
\curveto(547.6840814,86.09463108)(547.26908181,86.29463088)(546.92907959,86.53463186)
\curveto(546.6790824,86.71463046)(546.45408263,86.91963026)(546.25407959,87.14963186)
\curveto(546.04408304,87.3796298)(545.84908323,87.61962956)(545.66907959,87.86963186)
\curveto(545.48908359,88.12962905)(545.31908376,88.39462878)(545.15907959,88.66463186)
\curveto(544.98908409,88.94462823)(544.81408427,89.21462796)(544.63407959,89.47463186)
\curveto(544.55408453,89.58462759)(544.4790846,89.68962749)(544.40907959,89.78963186)
\curveto(544.33908474,89.89962728)(544.26408482,90.00962717)(544.18407959,90.11963186)
\curveto(544.15408493,90.15962702)(544.12408496,90.19462698)(544.09407959,90.22463186)
\curveto(544.05408503,90.26462691)(544.02408506,90.30462687)(544.00407959,90.34463186)
\curveto(543.89408519,90.48462669)(543.76908531,90.60962657)(543.62907959,90.71963186)
\curveto(543.59908548,90.73962644)(543.57408551,90.76462641)(543.55407959,90.79463186)
\curveto(543.52408556,90.82462635)(543.49408559,90.84962633)(543.46407959,90.86963186)
\curveto(543.36408572,90.94962623)(543.26408582,91.01462616)(543.16407959,91.06463186)
\curveto(543.06408602,91.12462605)(542.95408613,91.179626)(542.83407959,91.22963186)
\curveto(542.76408632,91.25962592)(542.68908639,91.2796259)(542.60907959,91.28963186)
\lineto(542.36907959,91.34963186)
\lineto(542.27907959,91.34963186)
\curveto(542.24908683,91.35962582)(542.21908686,91.36462581)(542.18907959,91.36463186)
\curveto(542.11908696,91.38462579)(542.02408706,91.38962579)(541.90407959,91.37963186)
\curveto(541.77408731,91.3796258)(541.67408741,91.36962581)(541.60407959,91.34963186)
\curveto(541.52408756,91.32962585)(541.44908763,91.30962587)(541.37907959,91.28963186)
\curveto(541.29908778,91.2796259)(541.21908786,91.25962592)(541.13907959,91.22963186)
\curveto(540.89908818,91.11962606)(540.69908838,90.96962621)(540.53907959,90.77963186)
\curveto(540.36908871,90.59962658)(540.22908885,90.3796268)(540.11907959,90.11963186)
\curveto(540.09908898,90.04962713)(540.084089,89.9796272)(540.07407959,89.90963186)
\curveto(540.05408903,89.83962734)(540.03408905,89.76462741)(540.01407959,89.68463186)
\curveto(539.99408909,89.60462757)(539.9840891,89.49462768)(539.98407959,89.35463186)
\curveto(539.9840891,89.22462795)(539.99408909,89.11962806)(540.01407959,89.03963186)
\curveto(540.02408906,88.9796282)(540.02908905,88.92462825)(540.02907959,88.87463186)
\curveto(540.02908905,88.82462835)(540.03908904,88.7746284)(540.05907959,88.72463186)
\curveto(540.09908898,88.62462855)(540.13908894,88.52962865)(540.17907959,88.43963186)
\curveto(540.21908886,88.35962882)(540.26408882,88.2796289)(540.31407959,88.19963186)
\curveto(540.33408875,88.16962901)(540.35908872,88.13962904)(540.38907959,88.10963186)
\curveto(540.41908866,88.08962909)(540.44408864,88.06462911)(540.46407959,88.03463186)
\lineto(540.53907959,87.95963186)
\curveto(540.55908852,87.92962925)(540.5790885,87.90462927)(540.59907959,87.88463186)
\lineto(540.80907959,87.73463186)
\curveto(540.86908821,87.69462948)(540.93408815,87.64962953)(541.00407959,87.59963186)
\curveto(541.09408799,87.53962964)(541.19908788,87.48962969)(541.31907959,87.44963186)
\curveto(541.42908765,87.41962976)(541.53908754,87.38462979)(541.64907959,87.34463186)
\curveto(541.75908732,87.30462987)(541.90408718,87.2796299)(542.08407959,87.26963186)
\curveto(542.25408683,87.25962992)(542.3790867,87.22962995)(542.45907959,87.17963186)
\curveto(542.53908654,87.12963005)(542.5840865,87.05463012)(542.59407959,86.95463186)
\curveto(542.60408648,86.85463032)(542.60908647,86.74463043)(542.60907959,86.62463186)
\curveto(542.60908647,86.58463059)(542.61408647,86.54463063)(542.62407959,86.50463186)
\curveto(542.62408646,86.46463071)(542.61908646,86.42963075)(542.60907959,86.39963186)
\curveto(542.58908649,86.34963083)(542.5790865,86.29963088)(542.57907959,86.24963186)
\curveto(542.5790865,86.20963097)(542.56908651,86.16963101)(542.54907959,86.12963186)
\curveto(542.48908659,86.03963114)(542.35408673,85.99463118)(542.14407959,85.99463186)
\lineto(542.02407959,85.99463186)
\curveto(541.96408712,86.00463117)(541.90408718,86.00963117)(541.84407959,86.00963186)
\curveto(541.77408731,86.01963116)(541.70908737,86.02963115)(541.64907959,86.03963186)
\curveto(541.53908754,86.05963112)(541.43908764,86.0796311)(541.34907959,86.09963186)
\curveto(541.24908783,86.11963106)(541.15408793,86.14963103)(541.06407959,86.18963186)
\curveto(540.99408809,86.20963097)(540.93408815,86.22963095)(540.88407959,86.24963186)
\lineto(540.70407959,86.30963186)
\curveto(540.44408864,86.42963075)(540.19908888,86.58463059)(539.96907959,86.77463186)
\curveto(539.73908934,86.9746302)(539.55408953,87.18962999)(539.41407959,87.41963186)
\curveto(539.33408975,87.52962965)(539.26908981,87.64462953)(539.21907959,87.76463186)
\lineto(539.06907959,88.15463186)
\curveto(539.01909006,88.26462891)(538.98909009,88.3796288)(538.97907959,88.49963186)
\curveto(538.95909012,88.61962856)(538.93409015,88.74462843)(538.90407959,88.87463186)
\curveto(538.90409018,88.94462823)(538.90409018,89.00962817)(538.90407959,89.06963186)
\curveto(538.89409019,89.12962805)(538.8840902,89.19462798)(538.87407959,89.26463186)
}
}
{
\newrgbcolor{curcolor}{0 0 0}
\pscustom[linestyle=none,fillstyle=solid,fillcolor=curcolor]
{
\newpath
\moveto(544.39407959,101.36424123)
\lineto(544.64907959,101.36424123)
\curveto(544.72908435,101.37423353)(544.80408428,101.36923353)(544.87407959,101.34924123)
\lineto(545.11407959,101.34924123)
\lineto(545.27907959,101.34924123)
\curveto(545.3790837,101.32923357)(545.4840836,101.31923358)(545.59407959,101.31924123)
\curveto(545.69408339,101.31923358)(545.79408329,101.30923359)(545.89407959,101.28924123)
\lineto(546.04407959,101.28924123)
\curveto(546.1840829,101.25923364)(546.32408276,101.23923366)(546.46407959,101.22924123)
\curveto(546.59408249,101.21923368)(546.72408236,101.19423371)(546.85407959,101.15424123)
\curveto(546.93408215,101.13423377)(547.01908206,101.11423379)(547.10907959,101.09424123)
\lineto(547.34907959,101.03424123)
\lineto(547.64907959,100.91424123)
\curveto(547.73908134,100.88423402)(547.82908125,100.84923405)(547.91907959,100.80924123)
\curveto(548.13908094,100.70923419)(548.35408073,100.57423433)(548.56407959,100.40424123)
\curveto(548.77408031,100.24423466)(548.94408014,100.06923483)(549.07407959,99.87924123)
\curveto(549.11407997,99.82923507)(549.15407993,99.76923513)(549.19407959,99.69924123)
\curveto(549.22407986,99.63923526)(549.25907982,99.57923532)(549.29907959,99.51924123)
\curveto(549.34907973,99.43923546)(549.38907969,99.34423556)(549.41907959,99.23424123)
\curveto(549.44907963,99.12423578)(549.4790796,99.01923588)(549.50907959,98.91924123)
\curveto(549.54907953,98.80923609)(549.57407951,98.6992362)(549.58407959,98.58924123)
\curveto(549.59407949,98.47923642)(549.60907947,98.36423654)(549.62907959,98.24424123)
\curveto(549.63907944,98.2042367)(549.63907944,98.15923674)(549.62907959,98.10924123)
\curveto(549.62907945,98.06923683)(549.63407945,98.02923687)(549.64407959,97.98924123)
\curveto(549.65407943,97.94923695)(549.65907942,97.89423701)(549.65907959,97.82424123)
\curveto(549.65907942,97.75423715)(549.65407943,97.7042372)(549.64407959,97.67424123)
\curveto(549.62407946,97.62423728)(549.61907946,97.57923732)(549.62907959,97.53924123)
\curveto(549.63907944,97.4992374)(549.63907944,97.46423744)(549.62907959,97.43424123)
\lineto(549.62907959,97.34424123)
\curveto(549.60907947,97.28423762)(549.59407949,97.21923768)(549.58407959,97.14924123)
\curveto(549.5840795,97.08923781)(549.5790795,97.02423788)(549.56907959,96.95424123)
\curveto(549.51907956,96.78423812)(549.46907961,96.62423828)(549.41907959,96.47424123)
\curveto(549.36907971,96.32423858)(549.30407978,96.17923872)(549.22407959,96.03924123)
\curveto(549.1840799,95.98923891)(549.15407993,95.93423897)(549.13407959,95.87424123)
\curveto(549.10407998,95.82423908)(549.06908001,95.77423913)(549.02907959,95.72424123)
\curveto(548.84908023,95.48423942)(548.62908045,95.28423962)(548.36907959,95.12424123)
\curveto(548.10908097,94.96423994)(547.82408126,94.82424008)(547.51407959,94.70424123)
\curveto(547.37408171,94.64424026)(547.23408185,94.5992403)(547.09407959,94.56924123)
\curveto(546.94408214,94.53924036)(546.78908229,94.5042404)(546.62907959,94.46424123)
\curveto(546.51908256,94.44424046)(546.40908267,94.42924047)(546.29907959,94.41924123)
\curveto(546.18908289,94.40924049)(546.079083,94.39424051)(545.96907959,94.37424123)
\curveto(545.92908315,94.36424054)(545.88908319,94.35924054)(545.84907959,94.35924123)
\curveto(545.80908327,94.36924053)(545.76908331,94.36924053)(545.72907959,94.35924123)
\curveto(545.6790834,94.34924055)(545.62908345,94.34424056)(545.57907959,94.34424123)
\lineto(545.41407959,94.34424123)
\curveto(545.36408372,94.32424058)(545.31408377,94.31924058)(545.26407959,94.32924123)
\curveto(545.20408388,94.33924056)(545.14908393,94.33924056)(545.09907959,94.32924123)
\curveto(545.05908402,94.31924058)(545.01408407,94.31924058)(544.96407959,94.32924123)
\curveto(544.91408417,94.33924056)(544.86408422,94.33424057)(544.81407959,94.31424123)
\curveto(544.74408434,94.29424061)(544.66908441,94.28924061)(544.58907959,94.29924123)
\curveto(544.49908458,94.30924059)(544.41408467,94.31424059)(544.33407959,94.31424123)
\curveto(544.24408484,94.31424059)(544.14408494,94.30924059)(544.03407959,94.29924123)
\curveto(543.91408517,94.28924061)(543.81408527,94.29424061)(543.73407959,94.31424123)
\lineto(543.44907959,94.31424123)
\lineto(542.81907959,94.35924123)
\curveto(542.71908636,94.36924053)(542.62408646,94.37924052)(542.53407959,94.38924123)
\lineto(542.23407959,94.41924123)
\curveto(542.1840869,94.43924046)(542.13408695,94.44424046)(542.08407959,94.43424123)
\curveto(542.02408706,94.43424047)(541.96908711,94.44424046)(541.91907959,94.46424123)
\curveto(541.74908733,94.51424039)(541.5840875,94.55424035)(541.42407959,94.58424123)
\curveto(541.25408783,94.61424029)(541.09408799,94.66424024)(540.94407959,94.73424123)
\curveto(540.4840886,94.92423998)(540.10908897,95.14423976)(539.81907959,95.39424123)
\curveto(539.52908955,95.65423925)(539.2840898,96.01423889)(539.08407959,96.47424123)
\curveto(539.03409005,96.6042383)(538.99909008,96.73423817)(538.97907959,96.86424123)
\curveto(538.95909012,97.0042379)(538.93409015,97.14423776)(538.90407959,97.28424123)
\curveto(538.89409019,97.35423755)(538.88909019,97.41923748)(538.88907959,97.47924123)
\curveto(538.88909019,97.53923736)(538.8840902,97.6042373)(538.87407959,97.67424123)
\curveto(538.85409023,98.5042364)(539.00409008,99.17423573)(539.32407959,99.68424123)
\curveto(539.63408945,100.19423471)(540.07408901,100.57423433)(540.64407959,100.82424123)
\curveto(540.76408832,100.87423403)(540.88908819,100.91923398)(541.01907959,100.95924123)
\curveto(541.14908793,100.9992339)(541.2840878,101.04423386)(541.42407959,101.09424123)
\curveto(541.50408758,101.11423379)(541.58908749,101.12923377)(541.67907959,101.13924123)
\lineto(541.91907959,101.19924123)
\curveto(542.02908705,101.22923367)(542.13908694,101.24423366)(542.24907959,101.24424123)
\curveto(542.35908672,101.25423365)(542.46908661,101.26923363)(542.57907959,101.28924123)
\curveto(542.62908645,101.30923359)(542.67408641,101.31423359)(542.71407959,101.30424123)
\curveto(542.75408633,101.3042336)(542.79408629,101.30923359)(542.83407959,101.31924123)
\curveto(542.8840862,101.32923357)(542.93908614,101.32923357)(542.99907959,101.31924123)
\curveto(543.04908603,101.31923358)(543.09908598,101.32423358)(543.14907959,101.33424123)
\lineto(543.28407959,101.33424123)
\curveto(543.34408574,101.35423355)(543.41408567,101.35423355)(543.49407959,101.33424123)
\curveto(543.56408552,101.32423358)(543.62908545,101.32923357)(543.68907959,101.34924123)
\curveto(543.71908536,101.35923354)(543.75908532,101.36423354)(543.80907959,101.36424123)
\lineto(543.92907959,101.36424123)
\lineto(544.39407959,101.36424123)
\moveto(546.71907959,99.81924123)
\curveto(546.39908268,99.91923498)(546.03408305,99.97923492)(545.62407959,99.99924123)
\curveto(545.21408387,100.01923488)(544.80408428,100.02923487)(544.39407959,100.02924123)
\curveto(543.96408512,100.02923487)(543.54408554,100.01923488)(543.13407959,99.99924123)
\curveto(542.72408636,99.97923492)(542.33908674,99.93423497)(541.97907959,99.86424123)
\curveto(541.61908746,99.79423511)(541.29908778,99.68423522)(541.01907959,99.53424123)
\curveto(540.72908835,99.39423551)(540.49408859,99.1992357)(540.31407959,98.94924123)
\curveto(540.20408888,98.78923611)(540.12408896,98.60923629)(540.07407959,98.40924123)
\curveto(540.01408907,98.20923669)(539.9840891,97.96423694)(539.98407959,97.67424123)
\curveto(540.00408908,97.65423725)(540.01408907,97.61923728)(540.01407959,97.56924123)
\curveto(540.00408908,97.51923738)(540.00408908,97.47923742)(540.01407959,97.44924123)
\curveto(540.03408905,97.36923753)(540.05408903,97.29423761)(540.07407959,97.22424123)
\curveto(540.084089,97.16423774)(540.10408898,97.0992378)(540.13407959,97.02924123)
\curveto(540.25408883,96.75923814)(540.42408866,96.53923836)(540.64407959,96.36924123)
\curveto(540.85408823,96.20923869)(541.09908798,96.07423883)(541.37907959,95.96424123)
\curveto(541.48908759,95.91423899)(541.60908747,95.87423903)(541.73907959,95.84424123)
\curveto(541.85908722,95.82423908)(541.9840871,95.7992391)(542.11407959,95.76924123)
\curveto(542.16408692,95.74923915)(542.21908686,95.73923916)(542.27907959,95.73924123)
\curveto(542.32908675,95.73923916)(542.3790867,95.73423917)(542.42907959,95.72424123)
\curveto(542.51908656,95.71423919)(542.61408647,95.7042392)(542.71407959,95.69424123)
\curveto(542.80408628,95.68423922)(542.89908618,95.67423923)(542.99907959,95.66424123)
\curveto(543.079086,95.66423924)(543.16408592,95.65923924)(543.25407959,95.64924123)
\lineto(543.49407959,95.64924123)
\lineto(543.67407959,95.64924123)
\curveto(543.70408538,95.63923926)(543.73908534,95.63423927)(543.77907959,95.63424123)
\lineto(543.91407959,95.63424123)
\lineto(544.36407959,95.63424123)
\curveto(544.44408464,95.63423927)(544.52908455,95.62923927)(544.61907959,95.61924123)
\curveto(544.69908438,95.61923928)(544.77408431,95.62923927)(544.84407959,95.64924123)
\lineto(545.11407959,95.64924123)
\curveto(545.13408395,95.64923925)(545.16408392,95.64423926)(545.20407959,95.63424123)
\curveto(545.23408385,95.63423927)(545.25908382,95.63923926)(545.27907959,95.64924123)
\curveto(545.3790837,95.65923924)(545.4790836,95.66423924)(545.57907959,95.66424123)
\curveto(545.66908341,95.67423923)(545.76908331,95.68423922)(545.87907959,95.69424123)
\curveto(545.99908308,95.72423918)(546.12408296,95.73923916)(546.25407959,95.73924123)
\curveto(546.37408271,95.74923915)(546.48908259,95.77423913)(546.59907959,95.81424123)
\curveto(546.89908218,95.89423901)(547.16408192,95.97923892)(547.39407959,96.06924123)
\curveto(547.62408146,96.16923873)(547.83908124,96.31423859)(548.03907959,96.50424123)
\curveto(548.23908084,96.71423819)(548.38908069,96.97923792)(548.48907959,97.29924123)
\curveto(548.50908057,97.33923756)(548.51908056,97.37423753)(548.51907959,97.40424123)
\curveto(548.50908057,97.44423746)(548.51408057,97.48923741)(548.53407959,97.53924123)
\curveto(548.54408054,97.57923732)(548.55408053,97.64923725)(548.56407959,97.74924123)
\curveto(548.57408051,97.85923704)(548.56908051,97.94423696)(548.54907959,98.00424123)
\curveto(548.52908055,98.07423683)(548.51908056,98.14423676)(548.51907959,98.21424123)
\curveto(548.50908057,98.28423662)(548.49408059,98.34923655)(548.47407959,98.40924123)
\curveto(548.41408067,98.60923629)(548.32908075,98.78923611)(548.21907959,98.94924123)
\curveto(548.19908088,98.97923592)(548.1790809,99.0042359)(548.15907959,99.02424123)
\lineto(548.09907959,99.08424123)
\curveto(548.079081,99.12423578)(548.03908104,99.17423573)(547.97907959,99.23424123)
\curveto(547.83908124,99.33423557)(547.70908137,99.41923548)(547.58907959,99.48924123)
\curveto(547.46908161,99.55923534)(547.32408176,99.62923527)(547.15407959,99.69924123)
\curveto(547.084082,99.72923517)(547.01408207,99.74923515)(546.94407959,99.75924123)
\curveto(546.87408221,99.77923512)(546.79908228,99.7992351)(546.71907959,99.81924123)
}
}
{
\newrgbcolor{curcolor}{0 0 0}
\pscustom[linestyle=none,fillstyle=solid,fillcolor=curcolor]
{
\newpath
\moveto(538.87407959,106.77385061)
\curveto(538.87409021,106.87384575)(538.8840902,106.96884566)(538.90407959,107.05885061)
\curveto(538.91409017,107.14884548)(538.94409014,107.21384541)(538.99407959,107.25385061)
\curveto(539.07409001,107.31384531)(539.1790899,107.34384528)(539.30907959,107.34385061)
\lineto(539.69907959,107.34385061)
\lineto(541.19907959,107.34385061)
\lineto(547.58907959,107.34385061)
\lineto(548.75907959,107.34385061)
\lineto(549.07407959,107.34385061)
\curveto(549.17407991,107.35384527)(549.25407983,107.33884529)(549.31407959,107.29885061)
\curveto(549.39407969,107.24884538)(549.44407964,107.17384545)(549.46407959,107.07385061)
\curveto(549.47407961,106.98384564)(549.4790796,106.87384575)(549.47907959,106.74385061)
\lineto(549.47907959,106.51885061)
\curveto(549.45907962,106.43884619)(549.44407964,106.36884626)(549.43407959,106.30885061)
\curveto(549.41407967,106.24884638)(549.37407971,106.19884643)(549.31407959,106.15885061)
\curveto(549.25407983,106.11884651)(549.1790799,106.09884653)(549.08907959,106.09885061)
\lineto(548.78907959,106.09885061)
\lineto(547.69407959,106.09885061)
\lineto(542.35407959,106.09885061)
\curveto(542.26408682,106.07884655)(542.18908689,106.06384656)(542.12907959,106.05385061)
\curveto(542.05908702,106.05384657)(541.99908708,106.0238466)(541.94907959,105.96385061)
\curveto(541.89908718,105.89384673)(541.87408721,105.80384682)(541.87407959,105.69385061)
\curveto(541.86408722,105.59384703)(541.85908722,105.48384714)(541.85907959,105.36385061)
\lineto(541.85907959,104.22385061)
\lineto(541.85907959,103.72885061)
\curveto(541.84908723,103.56884906)(541.78908729,103.45884917)(541.67907959,103.39885061)
\curveto(541.64908743,103.37884925)(541.61908746,103.36884926)(541.58907959,103.36885061)
\curveto(541.54908753,103.36884926)(541.50408758,103.36384926)(541.45407959,103.35385061)
\curveto(541.33408775,103.33384929)(541.22408786,103.33884929)(541.12407959,103.36885061)
\curveto(541.02408806,103.40884922)(540.95408813,103.46384916)(540.91407959,103.53385061)
\curveto(540.86408822,103.61384901)(540.83908824,103.73384889)(540.83907959,103.89385061)
\curveto(540.83908824,104.05384857)(540.82408826,104.18884844)(540.79407959,104.29885061)
\curveto(540.7840883,104.34884828)(540.7790883,104.40384822)(540.77907959,104.46385061)
\curveto(540.76908831,104.5238481)(540.75408833,104.58384804)(540.73407959,104.64385061)
\curveto(540.6840884,104.79384783)(540.63408845,104.93884769)(540.58407959,105.07885061)
\curveto(540.52408856,105.21884741)(540.45408863,105.35384727)(540.37407959,105.48385061)
\curveto(540.2840888,105.623847)(540.1790889,105.74384688)(540.05907959,105.84385061)
\curveto(539.93908914,105.94384668)(539.80908927,106.03884659)(539.66907959,106.12885061)
\curveto(539.56908951,106.18884644)(539.45908962,106.23384639)(539.33907959,106.26385061)
\curveto(539.21908986,106.30384632)(539.11408997,106.35384627)(539.02407959,106.41385061)
\curveto(538.96409012,106.46384616)(538.92409016,106.53384609)(538.90407959,106.62385061)
\curveto(538.89409019,106.64384598)(538.88909019,106.66884596)(538.88907959,106.69885061)
\curveto(538.88909019,106.7288459)(538.8840902,106.75384587)(538.87407959,106.77385061)
}
}
{
\newrgbcolor{curcolor}{0 0 0}
\pscustom[linestyle=none,fillstyle=solid,fillcolor=curcolor]
{
\newpath
\moveto(538.87407959,115.12345998)
\curveto(538.87409021,115.22345513)(538.8840902,115.31845503)(538.90407959,115.40845998)
\curveto(538.91409017,115.49845485)(538.94409014,115.56345479)(538.99407959,115.60345998)
\curveto(539.07409001,115.66345469)(539.1790899,115.69345466)(539.30907959,115.69345998)
\lineto(539.69907959,115.69345998)
\lineto(541.19907959,115.69345998)
\lineto(547.58907959,115.69345998)
\lineto(548.75907959,115.69345998)
\lineto(549.07407959,115.69345998)
\curveto(549.17407991,115.70345465)(549.25407983,115.68845466)(549.31407959,115.64845998)
\curveto(549.39407969,115.59845475)(549.44407964,115.52345483)(549.46407959,115.42345998)
\curveto(549.47407961,115.33345502)(549.4790796,115.22345513)(549.47907959,115.09345998)
\lineto(549.47907959,114.86845998)
\curveto(549.45907962,114.78845556)(549.44407964,114.71845563)(549.43407959,114.65845998)
\curveto(549.41407967,114.59845575)(549.37407971,114.5484558)(549.31407959,114.50845998)
\curveto(549.25407983,114.46845588)(549.1790799,114.4484559)(549.08907959,114.44845998)
\lineto(548.78907959,114.44845998)
\lineto(547.69407959,114.44845998)
\lineto(542.35407959,114.44845998)
\curveto(542.26408682,114.42845592)(542.18908689,114.41345594)(542.12907959,114.40345998)
\curveto(542.05908702,114.40345595)(541.99908708,114.37345598)(541.94907959,114.31345998)
\curveto(541.89908718,114.24345611)(541.87408721,114.1534562)(541.87407959,114.04345998)
\curveto(541.86408722,113.94345641)(541.85908722,113.83345652)(541.85907959,113.71345998)
\lineto(541.85907959,112.57345998)
\lineto(541.85907959,112.07845998)
\curveto(541.84908723,111.91845843)(541.78908729,111.80845854)(541.67907959,111.74845998)
\curveto(541.64908743,111.72845862)(541.61908746,111.71845863)(541.58907959,111.71845998)
\curveto(541.54908753,111.71845863)(541.50408758,111.71345864)(541.45407959,111.70345998)
\curveto(541.33408775,111.68345867)(541.22408786,111.68845866)(541.12407959,111.71845998)
\curveto(541.02408806,111.75845859)(540.95408813,111.81345854)(540.91407959,111.88345998)
\curveto(540.86408822,111.96345839)(540.83908824,112.08345827)(540.83907959,112.24345998)
\curveto(540.83908824,112.40345795)(540.82408826,112.53845781)(540.79407959,112.64845998)
\curveto(540.7840883,112.69845765)(540.7790883,112.7534576)(540.77907959,112.81345998)
\curveto(540.76908831,112.87345748)(540.75408833,112.93345742)(540.73407959,112.99345998)
\curveto(540.6840884,113.14345721)(540.63408845,113.28845706)(540.58407959,113.42845998)
\curveto(540.52408856,113.56845678)(540.45408863,113.70345665)(540.37407959,113.83345998)
\curveto(540.2840888,113.97345638)(540.1790889,114.09345626)(540.05907959,114.19345998)
\curveto(539.93908914,114.29345606)(539.80908927,114.38845596)(539.66907959,114.47845998)
\curveto(539.56908951,114.53845581)(539.45908962,114.58345577)(539.33907959,114.61345998)
\curveto(539.21908986,114.6534557)(539.11408997,114.70345565)(539.02407959,114.76345998)
\curveto(538.96409012,114.81345554)(538.92409016,114.88345547)(538.90407959,114.97345998)
\curveto(538.89409019,114.99345536)(538.88909019,115.01845533)(538.88907959,115.04845998)
\curveto(538.88909019,115.07845527)(538.8840902,115.10345525)(538.87407959,115.12345998)
}
}
{
\newrgbcolor{curcolor}{0 0 0}
\pscustom[linestyle=none,fillstyle=solid,fillcolor=curcolor]
{
\newpath
\moveto(560.74536865,29.18119436)
\lineto(560.74536865,30.09619436)
\curveto(560.74537935,30.19619171)(560.74537935,30.29119161)(560.74536865,30.38119436)
\curveto(560.74537935,30.47119143)(560.76537933,30.54619136)(560.80536865,30.60619436)
\curveto(560.86537923,30.69619121)(560.94537915,30.75619115)(561.04536865,30.78619436)
\curveto(561.14537895,30.82619108)(561.25037884,30.87119103)(561.36036865,30.92119436)
\curveto(561.55037854,31.0011909)(561.74037835,31.07119083)(561.93036865,31.13119436)
\curveto(562.12037797,31.2011907)(562.31037778,31.27619063)(562.50036865,31.35619436)
\curveto(562.68037741,31.42619048)(562.86537723,31.49119041)(563.05536865,31.55119436)
\curveto(563.23537686,31.61119029)(563.41537668,31.68119022)(563.59536865,31.76119436)
\curveto(563.73537636,31.82119008)(563.88037621,31.87619003)(564.03036865,31.92619436)
\curveto(564.18037591,31.97618993)(564.32537577,32.03118987)(564.46536865,32.09119436)
\curveto(564.91537518,32.27118963)(565.37037472,32.44118946)(565.83036865,32.60119436)
\curveto(566.28037381,32.76118914)(566.73037336,32.93118897)(567.18036865,33.11119436)
\curveto(567.23037286,33.13118877)(567.28037281,33.14618876)(567.33036865,33.15619436)
\lineto(567.48036865,33.21619436)
\curveto(567.70037239,33.3061886)(567.92537217,33.39118851)(568.15536865,33.47119436)
\curveto(568.37537172,33.55118835)(568.5953715,33.63618827)(568.81536865,33.72619436)
\curveto(568.90537119,33.76618814)(569.01537108,33.8061881)(569.14536865,33.84619436)
\curveto(569.26537083,33.88618802)(569.33537076,33.95118795)(569.35536865,34.04119436)
\curveto(569.36537073,34.08118782)(569.36537073,34.11118779)(569.35536865,34.13119436)
\lineto(569.29536865,34.19119436)
\curveto(569.24537085,34.24118766)(569.1903709,34.27618763)(569.13036865,34.29619436)
\curveto(569.07037102,34.32618758)(569.00537109,34.35618755)(568.93536865,34.38619436)
\lineto(568.30536865,34.62619436)
\curveto(568.08537201,34.7061872)(567.87037222,34.78618712)(567.66036865,34.86619436)
\lineto(567.51036865,34.92619436)
\lineto(567.33036865,34.98619436)
\curveto(567.14037295,35.06618684)(566.95037314,35.13618677)(566.76036865,35.19619436)
\curveto(566.56037353,35.26618664)(566.36037373,35.34118656)(566.16036865,35.42119436)
\curveto(565.58037451,35.66118624)(564.9953751,35.88118602)(564.40536865,36.08119436)
\curveto(563.81537628,36.29118561)(563.23037686,36.51618539)(562.65036865,36.75619436)
\curveto(562.45037764,36.83618507)(562.24537785,36.91118499)(562.03536865,36.98119436)
\curveto(561.82537827,37.06118484)(561.62037847,37.14118476)(561.42036865,37.22119436)
\curveto(561.34037875,37.26118464)(561.24037885,37.29618461)(561.12036865,37.32619436)
\curveto(561.00037909,37.36618454)(560.91537918,37.42118448)(560.86536865,37.49119436)
\curveto(560.82537927,37.55118435)(560.7953793,37.62618428)(560.77536865,37.71619436)
\curveto(560.75537934,37.81618409)(560.74537935,37.92618398)(560.74536865,38.04619436)
\curveto(560.73537936,38.16618374)(560.73537936,38.28618362)(560.74536865,38.40619436)
\curveto(560.74537935,38.52618338)(560.74537935,38.63618327)(560.74536865,38.73619436)
\curveto(560.74537935,38.82618308)(560.74537935,38.91618299)(560.74536865,39.00619436)
\curveto(560.74537935,39.1061828)(560.76537933,39.18118272)(560.80536865,39.23119436)
\curveto(560.85537924,39.32118258)(560.94537915,39.37118253)(561.07536865,39.38119436)
\curveto(561.20537889,39.39118251)(561.34537875,39.39618251)(561.49536865,39.39619436)
\lineto(563.14536865,39.39619436)
\lineto(569.41536865,39.39619436)
\lineto(570.67536865,39.39619436)
\curveto(570.78536931,39.39618251)(570.8953692,39.39618251)(571.00536865,39.39619436)
\curveto(571.11536898,39.4061825)(571.20036889,39.38618252)(571.26036865,39.33619436)
\curveto(571.32036877,39.3061826)(571.36036873,39.26118264)(571.38036865,39.20119436)
\curveto(571.3903687,39.14118276)(571.40536869,39.07118283)(571.42536865,38.99119436)
\lineto(571.42536865,38.75119436)
\lineto(571.42536865,38.39119436)
\curveto(571.41536868,38.28118362)(571.37036872,38.2011837)(571.29036865,38.15119436)
\curveto(571.26036883,38.13118377)(571.23036886,38.11618379)(571.20036865,38.10619436)
\curveto(571.16036893,38.1061838)(571.11536898,38.09618381)(571.06536865,38.07619436)
\lineto(570.90036865,38.07619436)
\curveto(570.84036925,38.06618384)(570.77036932,38.06118384)(570.69036865,38.06119436)
\curveto(570.61036948,38.07118383)(570.53536956,38.07618383)(570.46536865,38.07619436)
\lineto(569.62536865,38.07619436)
\lineto(565.20036865,38.07619436)
\curveto(564.95037514,38.07618383)(564.70037539,38.07618383)(564.45036865,38.07619436)
\curveto(564.1903759,38.07618383)(563.94037615,38.07118383)(563.70036865,38.06119436)
\curveto(563.60037649,38.06118384)(563.4903766,38.05618385)(563.37036865,38.04619436)
\curveto(563.25037684,38.03618387)(563.1903769,37.98118392)(563.19036865,37.88119436)
\lineto(563.20536865,37.88119436)
\curveto(563.22537687,37.81118409)(563.2903768,37.75118415)(563.40036865,37.70119436)
\curveto(563.51037658,37.66118424)(563.60537649,37.62618428)(563.68536865,37.59619436)
\curveto(563.85537624,37.52618438)(564.03037606,37.46118444)(564.21036865,37.40119436)
\curveto(564.38037571,37.34118456)(564.55037554,37.27118463)(564.72036865,37.19119436)
\curveto(564.77037532,37.17118473)(564.81537528,37.15618475)(564.85536865,37.14619436)
\curveto(564.8953752,37.13618477)(564.94037515,37.12118478)(564.99036865,37.10119436)
\curveto(565.17037492,37.02118488)(565.35537474,36.95118495)(565.54536865,36.89119436)
\curveto(565.72537437,36.84118506)(565.90537419,36.77618513)(566.08536865,36.69619436)
\curveto(566.23537386,36.62618528)(566.3903737,36.56618534)(566.55036865,36.51619436)
\curveto(566.70037339,36.46618544)(566.85037324,36.41118549)(567.00036865,36.35119436)
\curveto(567.47037262,36.15118575)(567.94537215,35.97118593)(568.42536865,35.81119436)
\curveto(568.8953712,35.65118625)(569.36037073,35.47618643)(569.82036865,35.28619436)
\curveto(570.00037009,35.2061867)(570.18036991,35.13618677)(570.36036865,35.07619436)
\curveto(570.54036955,35.01618689)(570.72036937,34.95118695)(570.90036865,34.88119436)
\curveto(571.01036908,34.83118707)(571.11536898,34.78118712)(571.21536865,34.73119436)
\curveto(571.30536879,34.69118721)(571.37036872,34.6061873)(571.41036865,34.47619436)
\curveto(571.42036867,34.45618745)(571.42536867,34.43118747)(571.42536865,34.40119436)
\curveto(571.41536868,34.38118752)(571.41536868,34.35618755)(571.42536865,34.32619436)
\curveto(571.43536866,34.29618761)(571.44036865,34.26118764)(571.44036865,34.22119436)
\curveto(571.43036866,34.18118772)(571.42536867,34.14118776)(571.42536865,34.10119436)
\lineto(571.42536865,33.80119436)
\curveto(571.42536867,33.7011882)(571.40036869,33.62118828)(571.35036865,33.56119436)
\curveto(571.30036879,33.48118842)(571.23036886,33.42118848)(571.14036865,33.38119436)
\curveto(571.04036905,33.35118855)(570.94036915,33.31118859)(570.84036865,33.26119436)
\curveto(570.64036945,33.18118872)(570.43536966,33.1011888)(570.22536865,33.02119436)
\curveto(570.00537009,32.95118895)(569.7953703,32.87618903)(569.59536865,32.79619436)
\curveto(569.41537068,32.71618919)(569.23537086,32.64618926)(569.05536865,32.58619436)
\curveto(568.86537123,32.53618937)(568.68037141,32.47118943)(568.50036865,32.39119436)
\curveto(567.94037215,32.16118974)(567.37537272,31.94618996)(566.80536865,31.74619436)
\curveto(566.23537386,31.54619036)(565.67037442,31.33119057)(565.11036865,31.10119436)
\lineto(564.48036865,30.86119436)
\curveto(564.26037583,30.79119111)(564.05037604,30.71619119)(563.85036865,30.63619436)
\curveto(563.74037635,30.58619132)(563.63537646,30.54119136)(563.53536865,30.50119436)
\curveto(563.42537667,30.47119143)(563.33037676,30.42119148)(563.25036865,30.35119436)
\curveto(563.23037686,30.34119156)(563.22037687,30.33119157)(563.22036865,30.32119436)
\lineto(563.19036865,30.29119436)
\lineto(563.19036865,30.21619436)
\lineto(563.22036865,30.18619436)
\curveto(563.22037687,30.17619173)(563.22537687,30.16619174)(563.23536865,30.15619436)
\curveto(563.28537681,30.13619177)(563.34037675,30.12619178)(563.40036865,30.12619436)
\curveto(563.46037663,30.12619178)(563.52037657,30.11619179)(563.58036865,30.09619436)
\lineto(563.74536865,30.09619436)
\curveto(563.80537629,30.07619183)(563.87037622,30.07119183)(563.94036865,30.08119436)
\curveto(564.01037608,30.09119181)(564.08037601,30.09619181)(564.15036865,30.09619436)
\lineto(564.96036865,30.09619436)
\lineto(569.52036865,30.09619436)
\lineto(570.70536865,30.09619436)
\curveto(570.81536928,30.09619181)(570.92536917,30.09119181)(571.03536865,30.08119436)
\curveto(571.14536895,30.08119182)(571.23036886,30.05619185)(571.29036865,30.00619436)
\curveto(571.37036872,29.95619195)(571.41536868,29.86619204)(571.42536865,29.73619436)
\lineto(571.42536865,29.34619436)
\lineto(571.42536865,29.15119436)
\curveto(571.42536867,29.1011928)(571.41536868,29.05119285)(571.39536865,29.00119436)
\curveto(571.35536874,28.87119303)(571.27036882,28.79619311)(571.14036865,28.77619436)
\curveto(571.01036908,28.76619314)(570.86036923,28.76119314)(570.69036865,28.76119436)
\lineto(568.95036865,28.76119436)
\lineto(562.95036865,28.76119436)
\lineto(561.54036865,28.76119436)
\curveto(561.43037866,28.76119314)(561.31537878,28.75619315)(561.19536865,28.74619436)
\curveto(561.07537902,28.74619316)(560.98037911,28.77119313)(560.91036865,28.82119436)
\curveto(560.85037924,28.86119304)(560.80037929,28.93619297)(560.76036865,29.04619436)
\curveto(560.75037934,29.06619284)(560.75037934,29.08619282)(560.76036865,29.10619436)
\curveto(560.76037933,29.13619277)(560.75537934,29.16119274)(560.74536865,29.18119436)
}
}
{
\newrgbcolor{curcolor}{0 0 0}
\pscustom[linestyle=none,fillstyle=solid,fillcolor=curcolor]
{
\newpath
\moveto(570.87036865,48.38330373)
\curveto(571.03036906,48.4132959)(571.16536893,48.39829592)(571.27536865,48.33830373)
\curveto(571.37536872,48.27829604)(571.45036864,48.19829612)(571.50036865,48.09830373)
\curveto(571.52036857,48.04829627)(571.53036856,47.99329632)(571.53036865,47.93330373)
\curveto(571.53036856,47.88329643)(571.54036855,47.82829649)(571.56036865,47.76830373)
\curveto(571.61036848,47.54829677)(571.5953685,47.32829699)(571.51536865,47.10830373)
\curveto(571.44536865,46.89829742)(571.35536874,46.75329756)(571.24536865,46.67330373)
\curveto(571.17536892,46.62329769)(571.095369,46.57829774)(571.00536865,46.53830373)
\curveto(570.90536919,46.49829782)(570.82536927,46.44829787)(570.76536865,46.38830373)
\curveto(570.74536935,46.36829795)(570.72536937,46.34329797)(570.70536865,46.31330373)
\curveto(570.68536941,46.29329802)(570.68036941,46.26329805)(570.69036865,46.22330373)
\curveto(570.72036937,46.1132982)(570.77536932,46.00829831)(570.85536865,45.90830373)
\curveto(570.93536916,45.8182985)(571.00536909,45.72829859)(571.06536865,45.63830373)
\curveto(571.14536895,45.50829881)(571.22036887,45.36829895)(571.29036865,45.21830373)
\curveto(571.35036874,45.06829925)(571.40536869,44.90829941)(571.45536865,44.73830373)
\curveto(571.48536861,44.63829968)(571.50536859,44.52829979)(571.51536865,44.40830373)
\curveto(571.52536857,44.29830002)(571.54036855,44.18830013)(571.56036865,44.07830373)
\curveto(571.57036852,44.02830029)(571.57536852,43.98330033)(571.57536865,43.94330373)
\lineto(571.57536865,43.83830373)
\curveto(571.5953685,43.72830059)(571.5953685,43.62330069)(571.57536865,43.52330373)
\lineto(571.57536865,43.38830373)
\curveto(571.56536853,43.33830098)(571.56036853,43.28830103)(571.56036865,43.23830373)
\curveto(571.56036853,43.18830113)(571.55036854,43.14330117)(571.53036865,43.10330373)
\curveto(571.52036857,43.06330125)(571.51536858,43.02830129)(571.51536865,42.99830373)
\curveto(571.52536857,42.97830134)(571.52536857,42.95330136)(571.51536865,42.92330373)
\lineto(571.45536865,42.68330373)
\curveto(571.44536865,42.60330171)(571.42536867,42.52830179)(571.39536865,42.45830373)
\curveto(571.26536883,42.15830216)(571.12036897,41.9133024)(570.96036865,41.72330373)
\curveto(570.7903693,41.54330277)(570.55536954,41.39330292)(570.25536865,41.27330373)
\curveto(570.03537006,41.18330313)(569.77037032,41.13830318)(569.46036865,41.13830373)
\lineto(569.14536865,41.13830373)
\curveto(569.095371,41.14830317)(569.04537105,41.15330316)(568.99536865,41.15330373)
\lineto(568.81536865,41.18330373)
\lineto(568.48536865,41.30330373)
\curveto(568.37537172,41.34330297)(568.27537182,41.39330292)(568.18536865,41.45330373)
\curveto(567.8953722,41.63330268)(567.68037241,41.87830244)(567.54036865,42.18830373)
\curveto(567.40037269,42.49830182)(567.27537282,42.83830148)(567.16536865,43.20830373)
\curveto(567.12537297,43.34830097)(567.095373,43.49330082)(567.07536865,43.64330373)
\curveto(567.05537304,43.79330052)(567.03037306,43.94330037)(567.00036865,44.09330373)
\curveto(566.98037311,44.16330015)(566.97037312,44.22830009)(566.97036865,44.28830373)
\curveto(566.97037312,44.35829996)(566.96037313,44.43329988)(566.94036865,44.51330373)
\curveto(566.92037317,44.58329973)(566.91037318,44.65329966)(566.91036865,44.72330373)
\curveto(566.90037319,44.79329952)(566.88537321,44.86829945)(566.86536865,44.94830373)
\curveto(566.80537329,45.19829912)(566.75537334,45.43329888)(566.71536865,45.65330373)
\curveto(566.66537343,45.87329844)(566.55037354,46.04829827)(566.37036865,46.17830373)
\curveto(566.2903738,46.23829808)(566.1903739,46.28829803)(566.07036865,46.32830373)
\curveto(565.94037415,46.36829795)(565.80037429,46.36829795)(565.65036865,46.32830373)
\curveto(565.41037468,46.26829805)(565.22037487,46.17829814)(565.08036865,46.05830373)
\curveto(564.94037515,45.94829837)(564.83037526,45.78829853)(564.75036865,45.57830373)
\curveto(564.70037539,45.45829886)(564.66537543,45.313299)(564.64536865,45.14330373)
\curveto(564.62537547,44.98329933)(564.61537548,44.8132995)(564.61536865,44.63330373)
\curveto(564.61537548,44.45329986)(564.62537547,44.27830004)(564.64536865,44.10830373)
\curveto(564.66537543,43.93830038)(564.6953754,43.79330052)(564.73536865,43.67330373)
\curveto(564.7953753,43.50330081)(564.88037521,43.33830098)(564.99036865,43.17830373)
\curveto(565.05037504,43.09830122)(565.13037496,43.02330129)(565.23036865,42.95330373)
\curveto(565.32037477,42.89330142)(565.42037467,42.83830148)(565.53036865,42.78830373)
\curveto(565.61037448,42.75830156)(565.6953744,42.72830159)(565.78536865,42.69830373)
\curveto(565.87537422,42.67830164)(565.94537415,42.63330168)(565.99536865,42.56330373)
\curveto(566.02537407,42.52330179)(566.05037404,42.45330186)(566.07036865,42.35330373)
\curveto(566.08037401,42.26330205)(566.08537401,42.16830215)(566.08536865,42.06830373)
\curveto(566.08537401,41.96830235)(566.08037401,41.86830245)(566.07036865,41.76830373)
\curveto(566.05037404,41.67830264)(566.02537407,41.6133027)(565.99536865,41.57330373)
\curveto(565.96537413,41.53330278)(565.91537418,41.50330281)(565.84536865,41.48330373)
\curveto(565.77537432,41.46330285)(565.70037439,41.46330285)(565.62036865,41.48330373)
\curveto(565.4903746,41.5133028)(565.37037472,41.54330277)(565.26036865,41.57330373)
\curveto(565.14037495,41.6133027)(565.02537507,41.65830266)(564.91536865,41.70830373)
\curveto(564.56537553,41.89830242)(564.2953758,42.13830218)(564.10536865,42.42830373)
\curveto(563.90537619,42.7183016)(563.74537635,43.07830124)(563.62536865,43.50830373)
\curveto(563.60537649,43.60830071)(563.5903765,43.70830061)(563.58036865,43.80830373)
\curveto(563.57037652,43.9183004)(563.55537654,44.02830029)(563.53536865,44.13830373)
\curveto(563.52537657,44.17830014)(563.52537657,44.24330007)(563.53536865,44.33330373)
\curveto(563.53537656,44.42329989)(563.52537657,44.47829984)(563.50536865,44.49830373)
\curveto(563.4953766,45.19829912)(563.57537652,45.80829851)(563.74536865,46.32830373)
\curveto(563.91537618,46.84829747)(564.24037585,47.2132971)(564.72036865,47.42330373)
\curveto(564.92037517,47.5132968)(565.15537494,47.56329675)(565.42536865,47.57330373)
\curveto(565.68537441,47.59329672)(565.96037413,47.60329671)(566.25036865,47.60330373)
\lineto(569.56536865,47.60330373)
\curveto(569.70537039,47.60329671)(569.84037025,47.60829671)(569.97036865,47.61830373)
\curveto(570.10036999,47.62829669)(570.20536989,47.65829666)(570.28536865,47.70830373)
\curveto(570.35536974,47.75829656)(570.40536969,47.82329649)(570.43536865,47.90330373)
\curveto(570.47536962,47.99329632)(570.50536959,48.07829624)(570.52536865,48.15830373)
\curveto(570.53536956,48.23829608)(570.58036951,48.29829602)(570.66036865,48.33830373)
\curveto(570.6903694,48.35829596)(570.72036937,48.36829595)(570.75036865,48.36830373)
\curveto(570.78036931,48.36829595)(570.82036927,48.37329594)(570.87036865,48.38330373)
\moveto(569.20536865,46.23830373)
\curveto(569.06537103,46.29829802)(568.90537119,46.32829799)(568.72536865,46.32830373)
\curveto(568.53537156,46.33829798)(568.34037175,46.34329797)(568.14036865,46.34330373)
\curveto(568.03037206,46.34329797)(567.93037216,46.33829798)(567.84036865,46.32830373)
\curveto(567.75037234,46.318298)(567.68037241,46.27829804)(567.63036865,46.20830373)
\curveto(567.61037248,46.17829814)(567.60037249,46.10829821)(567.60036865,45.99830373)
\curveto(567.62037247,45.97829834)(567.63037246,45.94329837)(567.63036865,45.89330373)
\curveto(567.63037246,45.84329847)(567.64037245,45.79829852)(567.66036865,45.75830373)
\curveto(567.68037241,45.67829864)(567.70037239,45.58829873)(567.72036865,45.48830373)
\lineto(567.78036865,45.18830373)
\curveto(567.78037231,45.15829916)(567.78537231,45.12329919)(567.79536865,45.08330373)
\lineto(567.79536865,44.97830373)
\curveto(567.83537226,44.82829949)(567.86037223,44.66329965)(567.87036865,44.48330373)
\curveto(567.87037222,44.3133)(567.8903722,44.15330016)(567.93036865,44.00330373)
\curveto(567.95037214,43.92330039)(567.97037212,43.84830047)(567.99036865,43.77830373)
\curveto(568.00037209,43.7183006)(568.01537208,43.64830067)(568.03536865,43.56830373)
\curveto(568.08537201,43.40830091)(568.15037194,43.25830106)(568.23036865,43.11830373)
\curveto(568.30037179,42.97830134)(568.3903717,42.85830146)(568.50036865,42.75830373)
\curveto(568.61037148,42.65830166)(568.74537135,42.58330173)(568.90536865,42.53330373)
\curveto(569.05537104,42.48330183)(569.24037085,42.46330185)(569.46036865,42.47330373)
\curveto(569.56037053,42.47330184)(569.65537044,42.48830183)(569.74536865,42.51830373)
\curveto(569.82537027,42.55830176)(569.90037019,42.60330171)(569.97036865,42.65330373)
\curveto(570.08037001,42.73330158)(570.17536992,42.83830148)(570.25536865,42.96830373)
\curveto(570.32536977,43.09830122)(570.38536971,43.23830108)(570.43536865,43.38830373)
\curveto(570.44536965,43.43830088)(570.45036964,43.48830083)(570.45036865,43.53830373)
\curveto(570.45036964,43.58830073)(570.45536964,43.63830068)(570.46536865,43.68830373)
\curveto(570.48536961,43.75830056)(570.50036959,43.84330047)(570.51036865,43.94330373)
\curveto(570.51036958,44.05330026)(570.50036959,44.14330017)(570.48036865,44.21330373)
\curveto(570.46036963,44.27330004)(570.45536964,44.33329998)(570.46536865,44.39330373)
\curveto(570.46536963,44.45329986)(570.45536964,44.5132998)(570.43536865,44.57330373)
\curveto(570.41536968,44.65329966)(570.40036969,44.72829959)(570.39036865,44.79830373)
\curveto(570.38036971,44.87829944)(570.36036973,44.95329936)(570.33036865,45.02330373)
\curveto(570.21036988,45.313299)(570.06537003,45.55829876)(569.89536865,45.75830373)
\curveto(569.72537037,45.96829835)(569.4953706,46.12829819)(569.20536865,46.23830373)
}
}
{
\newrgbcolor{curcolor}{0 0 0}
\pscustom[linestyle=none,fillstyle=solid,fillcolor=curcolor]
{
\newpath
\moveto(563.70036865,49.26994436)
\lineto(563.70036865,49.71994436)
\curveto(563.6903764,49.88994311)(563.71037638,50.01494298)(563.76036865,50.09494436)
\curveto(563.81037628,50.17494282)(563.87537622,50.22994277)(563.95536865,50.25994436)
\curveto(564.03537606,50.2999427)(564.12037597,50.33994266)(564.21036865,50.37994436)
\curveto(564.34037575,50.42994257)(564.47037562,50.47494252)(564.60036865,50.51494436)
\curveto(564.73037536,50.55494244)(564.86037523,50.5999424)(564.99036865,50.64994436)
\curveto(565.11037498,50.6999423)(565.23537486,50.74494225)(565.36536865,50.78494436)
\curveto(565.48537461,50.82494217)(565.60537449,50.86994213)(565.72536865,50.91994436)
\curveto(565.83537426,50.96994203)(565.95037414,51.00994199)(566.07036865,51.03994436)
\curveto(566.1903739,51.06994193)(566.31037378,51.10994189)(566.43036865,51.15994436)
\curveto(566.72037337,51.27994172)(567.02037307,51.38994161)(567.33036865,51.48994436)
\curveto(567.64037245,51.58994141)(567.94037215,51.6999413)(568.23036865,51.81994436)
\curveto(568.27037182,51.83994116)(568.31037178,51.84994115)(568.35036865,51.84994436)
\curveto(568.38037171,51.84994115)(568.41037168,51.85994114)(568.44036865,51.87994436)
\curveto(568.58037151,51.93994106)(568.72537137,51.994941)(568.87536865,52.04494436)
\lineto(569.29536865,52.19494436)
\curveto(569.36537073,52.22494077)(569.44037065,52.25494074)(569.52036865,52.28494436)
\curveto(569.5903705,52.31494068)(569.63537046,52.36494063)(569.65536865,52.43494436)
\curveto(569.68537041,52.51494048)(569.66037043,52.57494042)(569.58036865,52.61494436)
\curveto(569.4903706,52.66494033)(569.42037067,52.6999403)(569.37036865,52.71994436)
\curveto(569.20037089,52.7999402)(569.02037107,52.86494013)(568.83036865,52.91494436)
\curveto(568.64037145,52.96494003)(568.45537164,53.02493997)(568.27536865,53.09494436)
\curveto(568.04537205,53.18493981)(567.81537228,53.26493973)(567.58536865,53.33494436)
\curveto(567.34537275,53.40493959)(567.11537298,53.48993951)(566.89536865,53.58994436)
\curveto(566.84537325,53.5999394)(566.78037331,53.61493938)(566.70036865,53.63494436)
\curveto(566.61037348,53.67493932)(566.52037357,53.70993929)(566.43036865,53.73994436)
\curveto(566.33037376,53.76993923)(566.24037385,53.7999392)(566.16036865,53.82994436)
\curveto(566.11037398,53.84993915)(566.06537403,53.86493913)(566.02536865,53.87494436)
\curveto(565.98537411,53.88493911)(565.94037415,53.8999391)(565.89036865,53.91994436)
\curveto(565.77037432,53.96993903)(565.65037444,54.00993899)(565.53036865,54.03994436)
\curveto(565.40037469,54.07993892)(565.27537482,54.12493887)(565.15536865,54.17494436)
\curveto(565.10537499,54.1949388)(565.06037503,54.20993879)(565.02036865,54.21994436)
\curveto(564.98037511,54.22993877)(564.93537516,54.24493875)(564.88536865,54.26494436)
\curveto(564.7953753,54.30493869)(564.70537539,54.33993866)(564.61536865,54.36994436)
\curveto(564.51537558,54.3999386)(564.42037567,54.42993857)(564.33036865,54.45994436)
\curveto(564.25037584,54.48993851)(564.17037592,54.51493848)(564.09036865,54.53494436)
\curveto(564.00037609,54.56493843)(563.92537617,54.60493839)(563.86536865,54.65494436)
\curveto(563.77537632,54.72493827)(563.72537637,54.81993818)(563.71536865,54.93994436)
\curveto(563.70537639,55.06993793)(563.70037639,55.20993779)(563.70036865,55.35994436)
\curveto(563.70037639,55.43993756)(563.70537639,55.51493748)(563.71536865,55.58494436)
\curveto(563.71537638,55.66493733)(563.73037636,55.72993727)(563.76036865,55.77994436)
\curveto(563.82037627,55.86993713)(563.91537618,55.8949371)(564.04536865,55.85494436)
\curveto(564.17537592,55.81493718)(564.27537582,55.77993722)(564.34536865,55.74994436)
\lineto(564.40536865,55.71994436)
\curveto(564.42537567,55.71993728)(564.44537565,55.71493728)(564.46536865,55.70494436)
\curveto(564.74537535,55.5949374)(565.03037506,55.48493751)(565.32036865,55.37494436)
\lineto(566.16036865,55.04494436)
\curveto(566.24037385,55.01493798)(566.31537378,54.98993801)(566.38536865,54.96994436)
\curveto(566.44537365,54.94993805)(566.51037358,54.92493807)(566.58036865,54.89494436)
\curveto(566.78037331,54.81493818)(566.98537311,54.73493826)(567.19536865,54.65494436)
\curveto(567.3953727,54.58493841)(567.5953725,54.50993849)(567.79536865,54.42994436)
\curveto(568.48537161,54.13993886)(569.18037091,53.86993913)(569.88036865,53.61994436)
\curveto(570.58036951,53.36993963)(571.27536882,53.0999399)(571.96536865,52.80994436)
\lineto(572.11536865,52.74994436)
\curveto(572.17536792,52.73994026)(572.23536786,52.72494027)(572.29536865,52.70494436)
\curveto(572.66536743,52.54494045)(573.03036706,52.37494062)(573.39036865,52.19494436)
\curveto(573.76036633,52.01494098)(574.04536605,51.76494123)(574.24536865,51.44494436)
\curveto(574.30536579,51.33494166)(574.35036574,51.22494177)(574.38036865,51.11494436)
\curveto(574.42036567,51.00494199)(574.45536564,50.87994212)(574.48536865,50.73994436)
\curveto(574.50536559,50.68994231)(574.51036558,50.63494236)(574.50036865,50.57494436)
\curveto(574.4903656,50.52494247)(574.4903656,50.46994253)(574.50036865,50.40994436)
\curveto(574.52036557,50.32994267)(574.52036557,50.24994275)(574.50036865,50.16994436)
\curveto(574.4903656,50.12994287)(574.48536561,50.07994292)(574.48536865,50.01994436)
\lineto(574.42536865,49.77994436)
\curveto(574.40536569,49.70994329)(574.36536573,49.65494334)(574.30536865,49.61494436)
\curveto(574.24536585,49.56494343)(574.17036592,49.53494346)(574.08036865,49.52494436)
\lineto(573.81036865,49.52494436)
\lineto(573.60036865,49.52494436)
\curveto(573.54036655,49.53494346)(573.4903666,49.55494344)(573.45036865,49.58494436)
\curveto(573.34036675,49.65494334)(573.31036678,49.77494322)(573.36036865,49.94494436)
\curveto(573.38036671,50.05494294)(573.3903667,50.17494282)(573.39036865,50.30494436)
\curveto(573.3903667,50.43494256)(573.37036672,50.54994245)(573.33036865,50.64994436)
\curveto(573.28036681,50.7999422)(573.20536689,50.91994208)(573.10536865,51.00994436)
\curveto(573.00536709,51.10994189)(572.8903672,51.1949418)(572.76036865,51.26494436)
\curveto(572.64036745,51.33494166)(572.51036758,51.3949416)(572.37036865,51.44494436)
\lineto(571.95036865,51.62494436)
\curveto(571.86036823,51.66494133)(571.75036834,51.70494129)(571.62036865,51.74494436)
\curveto(571.4903686,51.7949412)(571.35536874,51.7999412)(571.21536865,51.75994436)
\curveto(571.05536904,51.70994129)(570.90536919,51.65494134)(570.76536865,51.59494436)
\curveto(570.62536947,51.54494145)(570.48536961,51.48994151)(570.34536865,51.42994436)
\curveto(570.13536996,51.33994166)(569.92537017,51.25494174)(569.71536865,51.17494436)
\curveto(569.50537059,51.0949419)(569.30037079,51.01494198)(569.10036865,50.93494436)
\curveto(568.96037113,50.87494212)(568.82537127,50.81994218)(568.69536865,50.76994436)
\curveto(568.56537153,50.71994228)(568.43037166,50.66994233)(568.29036865,50.61994436)
\lineto(566.97036865,50.07994436)
\curveto(566.53037356,49.90994309)(566.090374,49.73494326)(565.65036865,49.55494436)
\curveto(565.42037467,49.45494354)(565.20037489,49.36494363)(564.99036865,49.28494436)
\curveto(564.77037532,49.20494379)(564.55037554,49.11994388)(564.33036865,49.02994436)
\curveto(564.27037582,49.00994399)(564.1903759,48.97994402)(564.09036865,48.93994436)
\curveto(563.98037611,48.8999441)(563.8903762,48.90494409)(563.82036865,48.95494436)
\curveto(563.77037632,48.98494401)(563.73537636,49.04494395)(563.71536865,49.13494436)
\curveto(563.70537639,49.15494384)(563.70537639,49.17494382)(563.71536865,49.19494436)
\curveto(563.71537638,49.22494377)(563.71037638,49.24994375)(563.70036865,49.26994436)
}
}
{
\newrgbcolor{curcolor}{0 0 0}
\pscustom[linestyle=none,fillstyle=solid,fillcolor=curcolor]
{
}
}
{
\newrgbcolor{curcolor}{0 0 0}
\pscustom[linestyle=none,fillstyle=solid,fillcolor=curcolor]
{
\newpath
\moveto(560.82036865,65.05510061)
\curveto(560.82037927,65.15509575)(560.83037926,65.25009566)(560.85036865,65.34010061)
\curveto(560.86037923,65.43009548)(560.8903792,65.49509541)(560.94036865,65.53510061)
\curveto(561.02037907,65.59509531)(561.12537897,65.62509528)(561.25536865,65.62510061)
\lineto(561.64536865,65.62510061)
\lineto(563.14536865,65.62510061)
\lineto(569.53536865,65.62510061)
\lineto(570.70536865,65.62510061)
\lineto(571.02036865,65.62510061)
\curveto(571.12036897,65.63509527)(571.20036889,65.62009529)(571.26036865,65.58010061)
\curveto(571.34036875,65.53009538)(571.3903687,65.45509545)(571.41036865,65.35510061)
\curveto(571.42036867,65.26509564)(571.42536867,65.15509575)(571.42536865,65.02510061)
\lineto(571.42536865,64.80010061)
\curveto(571.40536869,64.72009619)(571.3903687,64.65009626)(571.38036865,64.59010061)
\curveto(571.36036873,64.53009638)(571.32036877,64.48009643)(571.26036865,64.44010061)
\curveto(571.20036889,64.40009651)(571.12536897,64.38009653)(571.03536865,64.38010061)
\lineto(570.73536865,64.38010061)
\lineto(569.64036865,64.38010061)
\lineto(564.30036865,64.38010061)
\curveto(564.21037588,64.36009655)(564.13537596,64.34509656)(564.07536865,64.33510061)
\curveto(564.00537609,64.33509657)(563.94537615,64.3050966)(563.89536865,64.24510061)
\curveto(563.84537625,64.17509673)(563.82037627,64.08509682)(563.82036865,63.97510061)
\curveto(563.81037628,63.87509703)(563.80537629,63.76509714)(563.80536865,63.64510061)
\lineto(563.80536865,62.50510061)
\lineto(563.80536865,62.01010061)
\curveto(563.7953763,61.85009906)(563.73537636,61.74009917)(563.62536865,61.68010061)
\curveto(563.5953765,61.66009925)(563.56537653,61.65009926)(563.53536865,61.65010061)
\curveto(563.4953766,61.65009926)(563.45037664,61.64509926)(563.40036865,61.63510061)
\curveto(563.28037681,61.61509929)(563.17037692,61.62009929)(563.07036865,61.65010061)
\curveto(562.97037712,61.69009922)(562.90037719,61.74509916)(562.86036865,61.81510061)
\curveto(562.81037728,61.89509901)(562.78537731,62.01509889)(562.78536865,62.17510061)
\curveto(562.78537731,62.33509857)(562.77037732,62.47009844)(562.74036865,62.58010061)
\curveto(562.73037736,62.63009828)(562.72537737,62.68509822)(562.72536865,62.74510061)
\curveto(562.71537738,62.8050981)(562.70037739,62.86509804)(562.68036865,62.92510061)
\curveto(562.63037746,63.07509783)(562.58037751,63.22009769)(562.53036865,63.36010061)
\curveto(562.47037762,63.50009741)(562.40037769,63.63509727)(562.32036865,63.76510061)
\curveto(562.23037786,63.905097)(562.12537797,64.02509688)(562.00536865,64.12510061)
\curveto(561.88537821,64.22509668)(561.75537834,64.32009659)(561.61536865,64.41010061)
\curveto(561.51537858,64.47009644)(561.40537869,64.51509639)(561.28536865,64.54510061)
\curveto(561.16537893,64.58509632)(561.06037903,64.63509627)(560.97036865,64.69510061)
\curveto(560.91037918,64.74509616)(560.87037922,64.81509609)(560.85036865,64.90510061)
\curveto(560.84037925,64.92509598)(560.83537926,64.95009596)(560.83536865,64.98010061)
\curveto(560.83537926,65.0100959)(560.83037926,65.03509587)(560.82036865,65.05510061)
}
}
{
\newrgbcolor{curcolor}{0 0 0}
\pscustom[linestyle=none,fillstyle=solid,fillcolor=curcolor]
{
\newpath
\moveto(561.01536865,69.80470998)
\lineto(561.01536865,74.60470998)
\lineto(561.01536865,75.60970998)
\curveto(561.01537908,75.74970288)(561.02537907,75.86970276)(561.04536865,75.96970998)
\curveto(561.05537904,76.07970255)(561.10037899,76.15970247)(561.18036865,76.20970998)
\curveto(561.22037887,76.2297024)(561.27037882,76.23970239)(561.33036865,76.23970998)
\curveto(561.3903787,76.24970238)(561.45537864,76.25470238)(561.52536865,76.25470998)
\lineto(561.79536865,76.25470998)
\curveto(561.88537821,76.25470238)(561.96537813,76.24470239)(562.03536865,76.22470998)
\curveto(562.11537798,76.18470245)(562.18537791,76.13970249)(562.24536865,76.08970998)
\lineto(562.42536865,75.93970998)
\curveto(562.47537762,75.90970272)(562.51537758,75.87470276)(562.54536865,75.83470998)
\curveto(562.57537752,75.79470284)(562.61537748,75.75470288)(562.66536865,75.71470998)
\curveto(562.77537732,75.634703)(562.88537721,75.54970308)(562.99536865,75.45970998)
\curveto(563.095377,75.36970326)(563.20037689,75.28470335)(563.31036865,75.20470998)
\curveto(563.51037658,75.06470357)(563.72037637,74.92470371)(563.94036865,74.78470998)
\curveto(564.15037594,74.64470399)(564.36537573,74.50470413)(564.58536865,74.36470998)
\curveto(564.67537542,74.31470432)(564.77037532,74.26470437)(564.87036865,74.21470998)
\curveto(564.97037512,74.16470447)(565.06537503,74.10970452)(565.15536865,74.04970998)
\curveto(565.17537492,74.0297046)(565.20037489,74.01970461)(565.23036865,74.01970998)
\curveto(565.26037483,74.01970461)(565.28537481,74.00970462)(565.30536865,73.98970998)
\curveto(565.40537469,73.91970471)(565.52037457,73.85470478)(565.65036865,73.79470998)
\curveto(565.77037432,73.7347049)(565.88537421,73.67970495)(565.99536865,73.62970998)
\curveto(566.22537387,73.5297051)(566.46037363,73.4347052)(566.70036865,73.34470998)
\curveto(566.94037315,73.25470538)(567.18037291,73.15470548)(567.42036865,73.04470998)
\curveto(567.47037262,73.02470561)(567.51537258,73.00970562)(567.55536865,72.99970998)
\curveto(567.5953725,72.99970563)(567.64037245,72.98970564)(567.69036865,72.96970998)
\curveto(567.81037228,72.91970571)(567.93537216,72.87470576)(568.06536865,72.83470998)
\curveto(568.18537191,72.80470583)(568.30537179,72.76970586)(568.42536865,72.72970998)
\curveto(568.65537144,72.64970598)(568.8953712,72.58470605)(569.14536865,72.53470998)
\curveto(569.38537071,72.49470614)(569.62537047,72.44470619)(569.86536865,72.38470998)
\curveto(570.01537008,72.34470629)(570.16536993,72.31970631)(570.31536865,72.30970998)
\curveto(570.46536963,72.29970633)(570.61536948,72.27970635)(570.76536865,72.24970998)
\curveto(570.80536929,72.23970639)(570.86536923,72.2347064)(570.94536865,72.23470998)
\curveto(571.06536903,72.20470643)(571.16536893,72.17470646)(571.24536865,72.14470998)
\curveto(571.32536877,72.11470652)(571.38036871,72.04470659)(571.41036865,71.93470998)
\curveto(571.43036866,71.88470675)(571.44036865,71.8297068)(571.44036865,71.76970998)
\lineto(571.44036865,71.57470998)
\curveto(571.44036865,71.4347072)(571.43536866,71.29470734)(571.42536865,71.15470998)
\curveto(571.41536868,71.02470761)(571.37036872,70.9297077)(571.29036865,70.86970998)
\curveto(571.23036886,70.8297078)(571.14536895,70.80970782)(571.03536865,70.80970998)
\curveto(570.92536917,70.81970781)(570.83036926,70.8347078)(570.75036865,70.85470998)
\lineto(570.67536865,70.85470998)
\curveto(570.64536945,70.86470777)(570.61536948,70.86970776)(570.58536865,70.86970998)
\curveto(570.50536959,70.88970774)(570.43036966,70.89970773)(570.36036865,70.89970998)
\curveto(570.2903698,70.89970773)(570.22036987,70.90970772)(570.15036865,70.92970998)
\curveto(569.96037013,70.97970765)(569.77537032,71.01970761)(569.59536865,71.04970998)
\curveto(569.40537069,71.07970755)(569.22537087,71.11970751)(569.05536865,71.16970998)
\curveto(569.00537109,71.18970744)(568.96537113,71.19970743)(568.93536865,71.19970998)
\curveto(568.90537119,71.19970743)(568.87037122,71.20470743)(568.83036865,71.21470998)
\curveto(568.53037156,71.31470732)(568.23537186,71.40470723)(567.94536865,71.48470998)
\curveto(567.65537244,71.57470706)(567.37537272,71.67970695)(567.10536865,71.79970998)
\curveto(566.52537357,72.05970657)(565.97537412,72.3297063)(565.45536865,72.60970998)
\curveto(564.92537517,72.88970574)(564.42037567,73.19970543)(563.94036865,73.53970998)
\curveto(563.74037635,73.67970495)(563.55037654,73.8297048)(563.37036865,73.98970998)
\curveto(563.18037691,74.14970448)(562.9903771,74.29970433)(562.80036865,74.43970998)
\curveto(562.75037734,74.47970415)(562.70537739,74.51470412)(562.66536865,74.54470998)
\curveto(562.61537748,74.58470405)(562.56537753,74.61970401)(562.51536865,74.64970998)
\curveto(562.4953776,74.65970397)(562.47037762,74.66970396)(562.44036865,74.67970998)
\curveto(562.41037768,74.69970393)(562.38037771,74.69970393)(562.35036865,74.67970998)
\curveto(562.2903778,74.65970397)(562.25537784,74.62470401)(562.24536865,74.57470998)
\curveto(562.22537787,74.52470411)(562.20537789,74.47470416)(562.18536865,74.42470998)
\lineto(562.18536865,74.31970998)
\curveto(562.17537792,74.27970435)(562.17537792,74.2297044)(562.18536865,74.16970998)
\lineto(562.18536865,74.01970998)
\lineto(562.18536865,73.41970998)
\lineto(562.18536865,70.77970998)
\lineto(562.18536865,70.04470998)
\lineto(562.18536865,69.80470998)
\curveto(562.17537792,69.7347089)(562.16037793,69.67470896)(562.14036865,69.62470998)
\curveto(562.10037799,69.5347091)(562.04037805,69.47470916)(561.96036865,69.44470998)
\curveto(561.86037823,69.39470924)(561.71537838,69.37970925)(561.52536865,69.39970998)
\curveto(561.32537877,69.41970921)(561.1903789,69.45470918)(561.12036865,69.50470998)
\curveto(561.10037899,69.52470911)(561.08537901,69.54970908)(561.07536865,69.57970998)
\lineto(561.01536865,69.69970998)
\curveto(561.01537908,69.71970891)(561.02037907,69.7347089)(561.03036865,69.74470998)
\curveto(561.03037906,69.76470887)(561.02537907,69.78470885)(561.01536865,69.80470998)
}
}
{
\newrgbcolor{curcolor}{0 0 0}
\pscustom[linestyle=none,fillstyle=solid,fillcolor=curcolor]
{
\newpath
\moveto(569.79036865,78.63431936)
\lineto(569.79036865,79.26431936)
\lineto(569.79036865,79.45931936)
\curveto(569.7903703,79.52931683)(569.80037029,79.58931677)(569.82036865,79.63931936)
\curveto(569.86037023,79.70931665)(569.90037019,79.7593166)(569.94036865,79.78931936)
\curveto(569.9903701,79.82931653)(570.05537004,79.84931651)(570.13536865,79.84931936)
\curveto(570.21536988,79.8593165)(570.30036979,79.86431649)(570.39036865,79.86431936)
\lineto(571.11036865,79.86431936)
\curveto(571.5903685,79.86431649)(572.00036809,79.80431655)(572.34036865,79.68431936)
\curveto(572.68036741,79.56431679)(572.95536714,79.36931699)(573.16536865,79.09931936)
\curveto(573.21536688,79.02931733)(573.26036683,78.9593174)(573.30036865,78.88931936)
\curveto(573.35036674,78.82931753)(573.3953667,78.7543176)(573.43536865,78.66431936)
\curveto(573.44536665,78.64431771)(573.45536664,78.61431774)(573.46536865,78.57431936)
\curveto(573.48536661,78.53431782)(573.4903666,78.48931787)(573.48036865,78.43931936)
\curveto(573.45036664,78.34931801)(573.37536672,78.29431806)(573.25536865,78.27431936)
\curveto(573.14536695,78.2543181)(573.05036704,78.26931809)(572.97036865,78.31931936)
\curveto(572.90036719,78.34931801)(572.83536726,78.39431796)(572.77536865,78.45431936)
\curveto(572.72536737,78.52431783)(572.67536742,78.58931777)(572.62536865,78.64931936)
\curveto(572.57536752,78.71931764)(572.50036759,78.77931758)(572.40036865,78.82931936)
\curveto(572.31036778,78.88931747)(572.22036787,78.93931742)(572.13036865,78.97931936)
\curveto(572.10036799,78.99931736)(572.04036805,79.02431733)(571.95036865,79.05431936)
\curveto(571.87036822,79.08431727)(571.80036829,79.08931727)(571.74036865,79.06931936)
\curveto(571.60036849,79.03931732)(571.51036858,78.97931738)(571.47036865,78.88931936)
\curveto(571.44036865,78.80931755)(571.42536867,78.71931764)(571.42536865,78.61931936)
\curveto(571.42536867,78.51931784)(571.40036869,78.43431792)(571.35036865,78.36431936)
\curveto(571.28036881,78.27431808)(571.14036895,78.22931813)(570.93036865,78.22931936)
\lineto(570.37536865,78.22931936)
\lineto(570.15036865,78.22931936)
\curveto(570.07037002,78.23931812)(570.00537009,78.2593181)(569.95536865,78.28931936)
\curveto(569.87537022,78.34931801)(569.83037026,78.41931794)(569.82036865,78.49931936)
\curveto(569.81037028,78.51931784)(569.80537029,78.53931782)(569.80536865,78.55931936)
\curveto(569.80537029,78.58931777)(569.80037029,78.61431774)(569.79036865,78.63431936)
}
}
{
\newrgbcolor{curcolor}{0 0 0}
\pscustom[linestyle=none,fillstyle=solid,fillcolor=curcolor]
{
}
}
{
\newrgbcolor{curcolor}{0 0 0}
\pscustom[linestyle=none,fillstyle=solid,fillcolor=curcolor]
{
\newpath
\moveto(560.82036865,89.26463186)
\curveto(560.81037928,89.95462722)(560.93037916,90.55462662)(561.18036865,91.06463186)
\curveto(561.43037866,91.58462559)(561.76537833,91.9796252)(562.18536865,92.24963186)
\curveto(562.26537783,92.29962488)(562.35537774,92.34462483)(562.45536865,92.38463186)
\curveto(562.54537755,92.42462475)(562.64037745,92.46962471)(562.74036865,92.51963186)
\curveto(562.84037725,92.55962462)(562.94037715,92.58962459)(563.04036865,92.60963186)
\curveto(563.14037695,92.62962455)(563.24537685,92.64962453)(563.35536865,92.66963186)
\curveto(563.40537669,92.68962449)(563.45037664,92.69462448)(563.49036865,92.68463186)
\curveto(563.53037656,92.6746245)(563.57537652,92.6796245)(563.62536865,92.69963186)
\curveto(563.67537642,92.70962447)(563.76037633,92.71462446)(563.88036865,92.71463186)
\curveto(563.9903761,92.71462446)(564.07537602,92.70962447)(564.13536865,92.69963186)
\curveto(564.1953759,92.6796245)(564.25537584,92.66962451)(564.31536865,92.66963186)
\curveto(564.37537572,92.6796245)(564.43537566,92.6746245)(564.49536865,92.65463186)
\curveto(564.63537546,92.61462456)(564.77037532,92.5796246)(564.90036865,92.54963186)
\curveto(565.03037506,92.51962466)(565.15537494,92.4796247)(565.27536865,92.42963186)
\curveto(565.41537468,92.36962481)(565.54037455,92.29962488)(565.65036865,92.21963186)
\curveto(565.76037433,92.14962503)(565.87037422,92.0746251)(565.98036865,91.99463186)
\lineto(566.04036865,91.93463186)
\curveto(566.06037403,91.92462525)(566.08037401,91.90962527)(566.10036865,91.88963186)
\curveto(566.26037383,91.76962541)(566.40537369,91.63462554)(566.53536865,91.48463186)
\curveto(566.66537343,91.33462584)(566.7903733,91.174626)(566.91036865,91.00463186)
\curveto(567.13037296,90.69462648)(567.33537276,90.39962678)(567.52536865,90.11963186)
\curveto(567.66537243,89.88962729)(567.80037229,89.65962752)(567.93036865,89.42963186)
\curveto(568.06037203,89.20962797)(568.1953719,88.98962819)(568.33536865,88.76963186)
\curveto(568.50537159,88.51962866)(568.68537141,88.2796289)(568.87536865,88.04963186)
\curveto(569.06537103,87.82962935)(569.2903708,87.63962954)(569.55036865,87.47963186)
\curveto(569.61037048,87.43962974)(569.67037042,87.40462977)(569.73036865,87.37463186)
\curveto(569.78037031,87.34462983)(569.84537025,87.31462986)(569.92536865,87.28463186)
\curveto(569.9953701,87.26462991)(570.05537004,87.25962992)(570.10536865,87.26963186)
\curveto(570.17536992,87.28962989)(570.23036986,87.32462985)(570.27036865,87.37463186)
\curveto(570.30036979,87.42462975)(570.32036977,87.48462969)(570.33036865,87.55463186)
\lineto(570.33036865,87.79463186)
\lineto(570.33036865,88.54463186)
\lineto(570.33036865,91.34963186)
\lineto(570.33036865,92.00963186)
\curveto(570.33036976,92.09962508)(570.33536976,92.18462499)(570.34536865,92.26463186)
\curveto(570.34536975,92.34462483)(570.36536973,92.40962477)(570.40536865,92.45963186)
\curveto(570.44536965,92.50962467)(570.52036957,92.54962463)(570.63036865,92.57963186)
\curveto(570.73036936,92.61962456)(570.83036926,92.62962455)(570.93036865,92.60963186)
\lineto(571.06536865,92.60963186)
\curveto(571.13536896,92.58962459)(571.1953689,92.56962461)(571.24536865,92.54963186)
\curveto(571.2953688,92.52962465)(571.33536876,92.49462468)(571.36536865,92.44463186)
\curveto(571.40536869,92.39462478)(571.42536867,92.32462485)(571.42536865,92.23463186)
\lineto(571.42536865,91.96463186)
\lineto(571.42536865,91.06463186)
\lineto(571.42536865,87.55463186)
\lineto(571.42536865,86.48963186)
\curveto(571.42536867,86.40963077)(571.43036866,86.31963086)(571.44036865,86.21963186)
\curveto(571.44036865,86.11963106)(571.43036866,86.03463114)(571.41036865,85.96463186)
\curveto(571.34036875,85.75463142)(571.16036893,85.68963149)(570.87036865,85.76963186)
\curveto(570.83036926,85.7796314)(570.7953693,85.7796314)(570.76536865,85.76963186)
\curveto(570.72536937,85.76963141)(570.68036941,85.7796314)(570.63036865,85.79963186)
\curveto(570.55036954,85.81963136)(570.46536963,85.83963134)(570.37536865,85.85963186)
\curveto(570.28536981,85.8796313)(570.20036989,85.90463127)(570.12036865,85.93463186)
\curveto(569.63037046,86.09463108)(569.21537088,86.29463088)(568.87536865,86.53463186)
\curveto(568.62537147,86.71463046)(568.40037169,86.91963026)(568.20036865,87.14963186)
\curveto(567.9903721,87.3796298)(567.7953723,87.61962956)(567.61536865,87.86963186)
\curveto(567.43537266,88.12962905)(567.26537283,88.39462878)(567.10536865,88.66463186)
\curveto(566.93537316,88.94462823)(566.76037333,89.21462796)(566.58036865,89.47463186)
\curveto(566.50037359,89.58462759)(566.42537367,89.68962749)(566.35536865,89.78963186)
\curveto(566.28537381,89.89962728)(566.21037388,90.00962717)(566.13036865,90.11963186)
\curveto(566.10037399,90.15962702)(566.07037402,90.19462698)(566.04036865,90.22463186)
\curveto(566.00037409,90.26462691)(565.97037412,90.30462687)(565.95036865,90.34463186)
\curveto(565.84037425,90.48462669)(565.71537438,90.60962657)(565.57536865,90.71963186)
\curveto(565.54537455,90.73962644)(565.52037457,90.76462641)(565.50036865,90.79463186)
\curveto(565.47037462,90.82462635)(565.44037465,90.84962633)(565.41036865,90.86963186)
\curveto(565.31037478,90.94962623)(565.21037488,91.01462616)(565.11036865,91.06463186)
\curveto(565.01037508,91.12462605)(564.90037519,91.179626)(564.78036865,91.22963186)
\curveto(564.71037538,91.25962592)(564.63537546,91.2796259)(564.55536865,91.28963186)
\lineto(564.31536865,91.34963186)
\lineto(564.22536865,91.34963186)
\curveto(564.1953759,91.35962582)(564.16537593,91.36462581)(564.13536865,91.36463186)
\curveto(564.06537603,91.38462579)(563.97037612,91.38962579)(563.85036865,91.37963186)
\curveto(563.72037637,91.3796258)(563.62037647,91.36962581)(563.55036865,91.34963186)
\curveto(563.47037662,91.32962585)(563.3953767,91.30962587)(563.32536865,91.28963186)
\curveto(563.24537685,91.2796259)(563.16537693,91.25962592)(563.08536865,91.22963186)
\curveto(562.84537725,91.11962606)(562.64537745,90.96962621)(562.48536865,90.77963186)
\curveto(562.31537778,90.59962658)(562.17537792,90.3796268)(562.06536865,90.11963186)
\curveto(562.04537805,90.04962713)(562.03037806,89.9796272)(562.02036865,89.90963186)
\curveto(562.00037809,89.83962734)(561.98037811,89.76462741)(561.96036865,89.68463186)
\curveto(561.94037815,89.60462757)(561.93037816,89.49462768)(561.93036865,89.35463186)
\curveto(561.93037816,89.22462795)(561.94037815,89.11962806)(561.96036865,89.03963186)
\curveto(561.97037812,88.9796282)(561.97537812,88.92462825)(561.97536865,88.87463186)
\curveto(561.97537812,88.82462835)(561.98537811,88.7746284)(562.00536865,88.72463186)
\curveto(562.04537805,88.62462855)(562.08537801,88.52962865)(562.12536865,88.43963186)
\curveto(562.16537793,88.35962882)(562.21037788,88.2796289)(562.26036865,88.19963186)
\curveto(562.28037781,88.16962901)(562.30537779,88.13962904)(562.33536865,88.10963186)
\curveto(562.36537773,88.08962909)(562.3903777,88.06462911)(562.41036865,88.03463186)
\lineto(562.48536865,87.95963186)
\curveto(562.50537759,87.92962925)(562.52537757,87.90462927)(562.54536865,87.88463186)
\lineto(562.75536865,87.73463186)
\curveto(562.81537728,87.69462948)(562.88037721,87.64962953)(562.95036865,87.59963186)
\curveto(563.04037705,87.53962964)(563.14537695,87.48962969)(563.26536865,87.44963186)
\curveto(563.37537672,87.41962976)(563.48537661,87.38462979)(563.59536865,87.34463186)
\curveto(563.70537639,87.30462987)(563.85037624,87.2796299)(564.03036865,87.26963186)
\curveto(564.20037589,87.25962992)(564.32537577,87.22962995)(564.40536865,87.17963186)
\curveto(564.48537561,87.12963005)(564.53037556,87.05463012)(564.54036865,86.95463186)
\curveto(564.55037554,86.85463032)(564.55537554,86.74463043)(564.55536865,86.62463186)
\curveto(564.55537554,86.58463059)(564.56037553,86.54463063)(564.57036865,86.50463186)
\curveto(564.57037552,86.46463071)(564.56537553,86.42963075)(564.55536865,86.39963186)
\curveto(564.53537556,86.34963083)(564.52537557,86.29963088)(564.52536865,86.24963186)
\curveto(564.52537557,86.20963097)(564.51537558,86.16963101)(564.49536865,86.12963186)
\curveto(564.43537566,86.03963114)(564.30037579,85.99463118)(564.09036865,85.99463186)
\lineto(563.97036865,85.99463186)
\curveto(563.91037618,86.00463117)(563.85037624,86.00963117)(563.79036865,86.00963186)
\curveto(563.72037637,86.01963116)(563.65537644,86.02963115)(563.59536865,86.03963186)
\curveto(563.48537661,86.05963112)(563.38537671,86.0796311)(563.29536865,86.09963186)
\curveto(563.1953769,86.11963106)(563.10037699,86.14963103)(563.01036865,86.18963186)
\curveto(562.94037715,86.20963097)(562.88037721,86.22963095)(562.83036865,86.24963186)
\lineto(562.65036865,86.30963186)
\curveto(562.3903777,86.42963075)(562.14537795,86.58463059)(561.91536865,86.77463186)
\curveto(561.68537841,86.9746302)(561.50037859,87.18962999)(561.36036865,87.41963186)
\curveto(561.28037881,87.52962965)(561.21537888,87.64462953)(561.16536865,87.76463186)
\lineto(561.01536865,88.15463186)
\curveto(560.96537913,88.26462891)(560.93537916,88.3796288)(560.92536865,88.49963186)
\curveto(560.90537919,88.61962856)(560.88037921,88.74462843)(560.85036865,88.87463186)
\curveto(560.85037924,88.94462823)(560.85037924,89.00962817)(560.85036865,89.06963186)
\curveto(560.84037925,89.12962805)(560.83037926,89.19462798)(560.82036865,89.26463186)
}
}
{
\newrgbcolor{curcolor}{0 0 0}
\pscustom[linestyle=none,fillstyle=solid,fillcolor=curcolor]
{
\newpath
\moveto(566.34036865,101.36424123)
\lineto(566.59536865,101.36424123)
\curveto(566.67537342,101.37423353)(566.75037334,101.36923353)(566.82036865,101.34924123)
\lineto(567.06036865,101.34924123)
\lineto(567.22536865,101.34924123)
\curveto(567.32537277,101.32923357)(567.43037266,101.31923358)(567.54036865,101.31924123)
\curveto(567.64037245,101.31923358)(567.74037235,101.30923359)(567.84036865,101.28924123)
\lineto(567.99036865,101.28924123)
\curveto(568.13037196,101.25923364)(568.27037182,101.23923366)(568.41036865,101.22924123)
\curveto(568.54037155,101.21923368)(568.67037142,101.19423371)(568.80036865,101.15424123)
\curveto(568.88037121,101.13423377)(568.96537113,101.11423379)(569.05536865,101.09424123)
\lineto(569.29536865,101.03424123)
\lineto(569.59536865,100.91424123)
\curveto(569.68537041,100.88423402)(569.77537032,100.84923405)(569.86536865,100.80924123)
\curveto(570.08537001,100.70923419)(570.30036979,100.57423433)(570.51036865,100.40424123)
\curveto(570.72036937,100.24423466)(570.8903692,100.06923483)(571.02036865,99.87924123)
\curveto(571.06036903,99.82923507)(571.10036899,99.76923513)(571.14036865,99.69924123)
\curveto(571.17036892,99.63923526)(571.20536889,99.57923532)(571.24536865,99.51924123)
\curveto(571.2953688,99.43923546)(571.33536876,99.34423556)(571.36536865,99.23424123)
\curveto(571.3953687,99.12423578)(571.42536867,99.01923588)(571.45536865,98.91924123)
\curveto(571.4953686,98.80923609)(571.52036857,98.6992362)(571.53036865,98.58924123)
\curveto(571.54036855,98.47923642)(571.55536854,98.36423654)(571.57536865,98.24424123)
\curveto(571.58536851,98.2042367)(571.58536851,98.15923674)(571.57536865,98.10924123)
\curveto(571.57536852,98.06923683)(571.58036851,98.02923687)(571.59036865,97.98924123)
\curveto(571.60036849,97.94923695)(571.60536849,97.89423701)(571.60536865,97.82424123)
\curveto(571.60536849,97.75423715)(571.60036849,97.7042372)(571.59036865,97.67424123)
\curveto(571.57036852,97.62423728)(571.56536853,97.57923732)(571.57536865,97.53924123)
\curveto(571.58536851,97.4992374)(571.58536851,97.46423744)(571.57536865,97.43424123)
\lineto(571.57536865,97.34424123)
\curveto(571.55536854,97.28423762)(571.54036855,97.21923768)(571.53036865,97.14924123)
\curveto(571.53036856,97.08923781)(571.52536857,97.02423788)(571.51536865,96.95424123)
\curveto(571.46536863,96.78423812)(571.41536868,96.62423828)(571.36536865,96.47424123)
\curveto(571.31536878,96.32423858)(571.25036884,96.17923872)(571.17036865,96.03924123)
\curveto(571.13036896,95.98923891)(571.10036899,95.93423897)(571.08036865,95.87424123)
\curveto(571.05036904,95.82423908)(571.01536908,95.77423913)(570.97536865,95.72424123)
\curveto(570.7953693,95.48423942)(570.57536952,95.28423962)(570.31536865,95.12424123)
\curveto(570.05537004,94.96423994)(569.77037032,94.82424008)(569.46036865,94.70424123)
\curveto(569.32037077,94.64424026)(569.18037091,94.5992403)(569.04036865,94.56924123)
\curveto(568.8903712,94.53924036)(568.73537136,94.5042404)(568.57536865,94.46424123)
\curveto(568.46537163,94.44424046)(568.35537174,94.42924047)(568.24536865,94.41924123)
\curveto(568.13537196,94.40924049)(568.02537207,94.39424051)(567.91536865,94.37424123)
\curveto(567.87537222,94.36424054)(567.83537226,94.35924054)(567.79536865,94.35924123)
\curveto(567.75537234,94.36924053)(567.71537238,94.36924053)(567.67536865,94.35924123)
\curveto(567.62537247,94.34924055)(567.57537252,94.34424056)(567.52536865,94.34424123)
\lineto(567.36036865,94.34424123)
\curveto(567.31037278,94.32424058)(567.26037283,94.31924058)(567.21036865,94.32924123)
\curveto(567.15037294,94.33924056)(567.095373,94.33924056)(567.04536865,94.32924123)
\curveto(567.00537309,94.31924058)(566.96037313,94.31924058)(566.91036865,94.32924123)
\curveto(566.86037323,94.33924056)(566.81037328,94.33424057)(566.76036865,94.31424123)
\curveto(566.6903734,94.29424061)(566.61537348,94.28924061)(566.53536865,94.29924123)
\curveto(566.44537365,94.30924059)(566.36037373,94.31424059)(566.28036865,94.31424123)
\curveto(566.1903739,94.31424059)(566.090374,94.30924059)(565.98036865,94.29924123)
\curveto(565.86037423,94.28924061)(565.76037433,94.29424061)(565.68036865,94.31424123)
\lineto(565.39536865,94.31424123)
\lineto(564.76536865,94.35924123)
\curveto(564.66537543,94.36924053)(564.57037552,94.37924052)(564.48036865,94.38924123)
\lineto(564.18036865,94.41924123)
\curveto(564.13037596,94.43924046)(564.08037601,94.44424046)(564.03036865,94.43424123)
\curveto(563.97037612,94.43424047)(563.91537618,94.44424046)(563.86536865,94.46424123)
\curveto(563.6953764,94.51424039)(563.53037656,94.55424035)(563.37036865,94.58424123)
\curveto(563.20037689,94.61424029)(563.04037705,94.66424024)(562.89036865,94.73424123)
\curveto(562.43037766,94.92423998)(562.05537804,95.14423976)(561.76536865,95.39424123)
\curveto(561.47537862,95.65423925)(561.23037886,96.01423889)(561.03036865,96.47424123)
\curveto(560.98037911,96.6042383)(560.94537915,96.73423817)(560.92536865,96.86424123)
\curveto(560.90537919,97.0042379)(560.88037921,97.14423776)(560.85036865,97.28424123)
\curveto(560.84037925,97.35423755)(560.83537926,97.41923748)(560.83536865,97.47924123)
\curveto(560.83537926,97.53923736)(560.83037926,97.6042373)(560.82036865,97.67424123)
\curveto(560.80037929,98.5042364)(560.95037914,99.17423573)(561.27036865,99.68424123)
\curveto(561.58037851,100.19423471)(562.02037807,100.57423433)(562.59036865,100.82424123)
\curveto(562.71037738,100.87423403)(562.83537726,100.91923398)(562.96536865,100.95924123)
\curveto(563.095377,100.9992339)(563.23037686,101.04423386)(563.37036865,101.09424123)
\curveto(563.45037664,101.11423379)(563.53537656,101.12923377)(563.62536865,101.13924123)
\lineto(563.86536865,101.19924123)
\curveto(563.97537612,101.22923367)(564.08537601,101.24423366)(564.19536865,101.24424123)
\curveto(564.30537579,101.25423365)(564.41537568,101.26923363)(564.52536865,101.28924123)
\curveto(564.57537552,101.30923359)(564.62037547,101.31423359)(564.66036865,101.30424123)
\curveto(564.70037539,101.3042336)(564.74037535,101.30923359)(564.78036865,101.31924123)
\curveto(564.83037526,101.32923357)(564.88537521,101.32923357)(564.94536865,101.31924123)
\curveto(564.9953751,101.31923358)(565.04537505,101.32423358)(565.09536865,101.33424123)
\lineto(565.23036865,101.33424123)
\curveto(565.2903748,101.35423355)(565.36037473,101.35423355)(565.44036865,101.33424123)
\curveto(565.51037458,101.32423358)(565.57537452,101.32923357)(565.63536865,101.34924123)
\curveto(565.66537443,101.35923354)(565.70537439,101.36423354)(565.75536865,101.36424123)
\lineto(565.87536865,101.36424123)
\lineto(566.34036865,101.36424123)
\moveto(568.66536865,99.81924123)
\curveto(568.34537175,99.91923498)(567.98037211,99.97923492)(567.57036865,99.99924123)
\curveto(567.16037293,100.01923488)(566.75037334,100.02923487)(566.34036865,100.02924123)
\curveto(565.91037418,100.02923487)(565.4903746,100.01923488)(565.08036865,99.99924123)
\curveto(564.67037542,99.97923492)(564.28537581,99.93423497)(563.92536865,99.86424123)
\curveto(563.56537653,99.79423511)(563.24537685,99.68423522)(562.96536865,99.53424123)
\curveto(562.67537742,99.39423551)(562.44037765,99.1992357)(562.26036865,98.94924123)
\curveto(562.15037794,98.78923611)(562.07037802,98.60923629)(562.02036865,98.40924123)
\curveto(561.96037813,98.20923669)(561.93037816,97.96423694)(561.93036865,97.67424123)
\curveto(561.95037814,97.65423725)(561.96037813,97.61923728)(561.96036865,97.56924123)
\curveto(561.95037814,97.51923738)(561.95037814,97.47923742)(561.96036865,97.44924123)
\curveto(561.98037811,97.36923753)(562.00037809,97.29423761)(562.02036865,97.22424123)
\curveto(562.03037806,97.16423774)(562.05037804,97.0992378)(562.08036865,97.02924123)
\curveto(562.20037789,96.75923814)(562.37037772,96.53923836)(562.59036865,96.36924123)
\curveto(562.80037729,96.20923869)(563.04537705,96.07423883)(563.32536865,95.96424123)
\curveto(563.43537666,95.91423899)(563.55537654,95.87423903)(563.68536865,95.84424123)
\curveto(563.80537629,95.82423908)(563.93037616,95.7992391)(564.06036865,95.76924123)
\curveto(564.11037598,95.74923915)(564.16537593,95.73923916)(564.22536865,95.73924123)
\curveto(564.27537582,95.73923916)(564.32537577,95.73423917)(564.37536865,95.72424123)
\curveto(564.46537563,95.71423919)(564.56037553,95.7042392)(564.66036865,95.69424123)
\curveto(564.75037534,95.68423922)(564.84537525,95.67423923)(564.94536865,95.66424123)
\curveto(565.02537507,95.66423924)(565.11037498,95.65923924)(565.20036865,95.64924123)
\lineto(565.44036865,95.64924123)
\lineto(565.62036865,95.64924123)
\curveto(565.65037444,95.63923926)(565.68537441,95.63423927)(565.72536865,95.63424123)
\lineto(565.86036865,95.63424123)
\lineto(566.31036865,95.63424123)
\curveto(566.3903737,95.63423927)(566.47537362,95.62923927)(566.56536865,95.61924123)
\curveto(566.64537345,95.61923928)(566.72037337,95.62923927)(566.79036865,95.64924123)
\lineto(567.06036865,95.64924123)
\curveto(567.08037301,95.64923925)(567.11037298,95.64423926)(567.15036865,95.63424123)
\curveto(567.18037291,95.63423927)(567.20537289,95.63923926)(567.22536865,95.64924123)
\curveto(567.32537277,95.65923924)(567.42537267,95.66423924)(567.52536865,95.66424123)
\curveto(567.61537248,95.67423923)(567.71537238,95.68423922)(567.82536865,95.69424123)
\curveto(567.94537215,95.72423918)(568.07037202,95.73923916)(568.20036865,95.73924123)
\curveto(568.32037177,95.74923915)(568.43537166,95.77423913)(568.54536865,95.81424123)
\curveto(568.84537125,95.89423901)(569.11037098,95.97923892)(569.34036865,96.06924123)
\curveto(569.57037052,96.16923873)(569.78537031,96.31423859)(569.98536865,96.50424123)
\curveto(570.18536991,96.71423819)(570.33536976,96.97923792)(570.43536865,97.29924123)
\curveto(570.45536964,97.33923756)(570.46536963,97.37423753)(570.46536865,97.40424123)
\curveto(570.45536964,97.44423746)(570.46036963,97.48923741)(570.48036865,97.53924123)
\curveto(570.4903696,97.57923732)(570.50036959,97.64923725)(570.51036865,97.74924123)
\curveto(570.52036957,97.85923704)(570.51536958,97.94423696)(570.49536865,98.00424123)
\curveto(570.47536962,98.07423683)(570.46536963,98.14423676)(570.46536865,98.21424123)
\curveto(570.45536964,98.28423662)(570.44036965,98.34923655)(570.42036865,98.40924123)
\curveto(570.36036973,98.60923629)(570.27536982,98.78923611)(570.16536865,98.94924123)
\curveto(570.14536995,98.97923592)(570.12536997,99.0042359)(570.10536865,99.02424123)
\lineto(570.04536865,99.08424123)
\curveto(570.02537007,99.12423578)(569.98537011,99.17423573)(569.92536865,99.23424123)
\curveto(569.78537031,99.33423557)(569.65537044,99.41923548)(569.53536865,99.48924123)
\curveto(569.41537068,99.55923534)(569.27037082,99.62923527)(569.10036865,99.69924123)
\curveto(569.03037106,99.72923517)(568.96037113,99.74923515)(568.89036865,99.75924123)
\curveto(568.82037127,99.77923512)(568.74537135,99.7992351)(568.66536865,99.81924123)
}
}
{
\newrgbcolor{curcolor}{0 0 0}
\pscustom[linestyle=none,fillstyle=solid,fillcolor=curcolor]
{
\newpath
\moveto(560.82036865,106.77385061)
\curveto(560.82037927,106.87384575)(560.83037926,106.96884566)(560.85036865,107.05885061)
\curveto(560.86037923,107.14884548)(560.8903792,107.21384541)(560.94036865,107.25385061)
\curveto(561.02037907,107.31384531)(561.12537897,107.34384528)(561.25536865,107.34385061)
\lineto(561.64536865,107.34385061)
\lineto(563.14536865,107.34385061)
\lineto(569.53536865,107.34385061)
\lineto(570.70536865,107.34385061)
\lineto(571.02036865,107.34385061)
\curveto(571.12036897,107.35384527)(571.20036889,107.33884529)(571.26036865,107.29885061)
\curveto(571.34036875,107.24884538)(571.3903687,107.17384545)(571.41036865,107.07385061)
\curveto(571.42036867,106.98384564)(571.42536867,106.87384575)(571.42536865,106.74385061)
\lineto(571.42536865,106.51885061)
\curveto(571.40536869,106.43884619)(571.3903687,106.36884626)(571.38036865,106.30885061)
\curveto(571.36036873,106.24884638)(571.32036877,106.19884643)(571.26036865,106.15885061)
\curveto(571.20036889,106.11884651)(571.12536897,106.09884653)(571.03536865,106.09885061)
\lineto(570.73536865,106.09885061)
\lineto(569.64036865,106.09885061)
\lineto(564.30036865,106.09885061)
\curveto(564.21037588,106.07884655)(564.13537596,106.06384656)(564.07536865,106.05385061)
\curveto(564.00537609,106.05384657)(563.94537615,106.0238466)(563.89536865,105.96385061)
\curveto(563.84537625,105.89384673)(563.82037627,105.80384682)(563.82036865,105.69385061)
\curveto(563.81037628,105.59384703)(563.80537629,105.48384714)(563.80536865,105.36385061)
\lineto(563.80536865,104.22385061)
\lineto(563.80536865,103.72885061)
\curveto(563.7953763,103.56884906)(563.73537636,103.45884917)(563.62536865,103.39885061)
\curveto(563.5953765,103.37884925)(563.56537653,103.36884926)(563.53536865,103.36885061)
\curveto(563.4953766,103.36884926)(563.45037664,103.36384926)(563.40036865,103.35385061)
\curveto(563.28037681,103.33384929)(563.17037692,103.33884929)(563.07036865,103.36885061)
\curveto(562.97037712,103.40884922)(562.90037719,103.46384916)(562.86036865,103.53385061)
\curveto(562.81037728,103.61384901)(562.78537731,103.73384889)(562.78536865,103.89385061)
\curveto(562.78537731,104.05384857)(562.77037732,104.18884844)(562.74036865,104.29885061)
\curveto(562.73037736,104.34884828)(562.72537737,104.40384822)(562.72536865,104.46385061)
\curveto(562.71537738,104.5238481)(562.70037739,104.58384804)(562.68036865,104.64385061)
\curveto(562.63037746,104.79384783)(562.58037751,104.93884769)(562.53036865,105.07885061)
\curveto(562.47037762,105.21884741)(562.40037769,105.35384727)(562.32036865,105.48385061)
\curveto(562.23037786,105.623847)(562.12537797,105.74384688)(562.00536865,105.84385061)
\curveto(561.88537821,105.94384668)(561.75537834,106.03884659)(561.61536865,106.12885061)
\curveto(561.51537858,106.18884644)(561.40537869,106.23384639)(561.28536865,106.26385061)
\curveto(561.16537893,106.30384632)(561.06037903,106.35384627)(560.97036865,106.41385061)
\curveto(560.91037918,106.46384616)(560.87037922,106.53384609)(560.85036865,106.62385061)
\curveto(560.84037925,106.64384598)(560.83537926,106.66884596)(560.83536865,106.69885061)
\curveto(560.83537926,106.7288459)(560.83037926,106.75384587)(560.82036865,106.77385061)
}
}
{
\newrgbcolor{curcolor}{0 0 0}
\pscustom[linestyle=none,fillstyle=solid,fillcolor=curcolor]
{
\newpath
\moveto(560.82036865,115.12345998)
\curveto(560.82037927,115.22345513)(560.83037926,115.31845503)(560.85036865,115.40845998)
\curveto(560.86037923,115.49845485)(560.8903792,115.56345479)(560.94036865,115.60345998)
\curveto(561.02037907,115.66345469)(561.12537897,115.69345466)(561.25536865,115.69345998)
\lineto(561.64536865,115.69345998)
\lineto(563.14536865,115.69345998)
\lineto(569.53536865,115.69345998)
\lineto(570.70536865,115.69345998)
\lineto(571.02036865,115.69345998)
\curveto(571.12036897,115.70345465)(571.20036889,115.68845466)(571.26036865,115.64845998)
\curveto(571.34036875,115.59845475)(571.3903687,115.52345483)(571.41036865,115.42345998)
\curveto(571.42036867,115.33345502)(571.42536867,115.22345513)(571.42536865,115.09345998)
\lineto(571.42536865,114.86845998)
\curveto(571.40536869,114.78845556)(571.3903687,114.71845563)(571.38036865,114.65845998)
\curveto(571.36036873,114.59845575)(571.32036877,114.5484558)(571.26036865,114.50845998)
\curveto(571.20036889,114.46845588)(571.12536897,114.4484559)(571.03536865,114.44845998)
\lineto(570.73536865,114.44845998)
\lineto(569.64036865,114.44845998)
\lineto(564.30036865,114.44845998)
\curveto(564.21037588,114.42845592)(564.13537596,114.41345594)(564.07536865,114.40345998)
\curveto(564.00537609,114.40345595)(563.94537615,114.37345598)(563.89536865,114.31345998)
\curveto(563.84537625,114.24345611)(563.82037627,114.1534562)(563.82036865,114.04345998)
\curveto(563.81037628,113.94345641)(563.80537629,113.83345652)(563.80536865,113.71345998)
\lineto(563.80536865,112.57345998)
\lineto(563.80536865,112.07845998)
\curveto(563.7953763,111.91845843)(563.73537636,111.80845854)(563.62536865,111.74845998)
\curveto(563.5953765,111.72845862)(563.56537653,111.71845863)(563.53536865,111.71845998)
\curveto(563.4953766,111.71845863)(563.45037664,111.71345864)(563.40036865,111.70345998)
\curveto(563.28037681,111.68345867)(563.17037692,111.68845866)(563.07036865,111.71845998)
\curveto(562.97037712,111.75845859)(562.90037719,111.81345854)(562.86036865,111.88345998)
\curveto(562.81037728,111.96345839)(562.78537731,112.08345827)(562.78536865,112.24345998)
\curveto(562.78537731,112.40345795)(562.77037732,112.53845781)(562.74036865,112.64845998)
\curveto(562.73037736,112.69845765)(562.72537737,112.7534576)(562.72536865,112.81345998)
\curveto(562.71537738,112.87345748)(562.70037739,112.93345742)(562.68036865,112.99345998)
\curveto(562.63037746,113.14345721)(562.58037751,113.28845706)(562.53036865,113.42845998)
\curveto(562.47037762,113.56845678)(562.40037769,113.70345665)(562.32036865,113.83345998)
\curveto(562.23037786,113.97345638)(562.12537797,114.09345626)(562.00536865,114.19345998)
\curveto(561.88537821,114.29345606)(561.75537834,114.38845596)(561.61536865,114.47845998)
\curveto(561.51537858,114.53845581)(561.40537869,114.58345577)(561.28536865,114.61345998)
\curveto(561.16537893,114.6534557)(561.06037903,114.70345565)(560.97036865,114.76345998)
\curveto(560.91037918,114.81345554)(560.87037922,114.88345547)(560.85036865,114.97345998)
\curveto(560.84037925,114.99345536)(560.83537926,115.01845533)(560.83536865,115.04845998)
\curveto(560.83537926,115.07845527)(560.83037926,115.10345525)(560.82036865,115.12345998)
}
}
{
\newrgbcolor{curcolor}{0 0 0}
\pscustom[linestyle=none,fillstyle=solid,fillcolor=curcolor]
{
\newpath
\moveto(582.69165771,29.18119436)
\lineto(582.69165771,30.09619436)
\curveto(582.69166841,30.19619171)(582.69166841,30.29119161)(582.69165771,30.38119436)
\curveto(582.69166841,30.47119143)(582.71166839,30.54619136)(582.75165771,30.60619436)
\curveto(582.81166829,30.69619121)(582.89166821,30.75619115)(582.99165771,30.78619436)
\curveto(583.09166801,30.82619108)(583.1966679,30.87119103)(583.30665771,30.92119436)
\curveto(583.4966676,31.0011909)(583.68666741,31.07119083)(583.87665771,31.13119436)
\curveto(584.06666703,31.2011907)(584.25666684,31.27619063)(584.44665771,31.35619436)
\curveto(584.62666647,31.42619048)(584.81166629,31.49119041)(585.00165771,31.55119436)
\curveto(585.18166592,31.61119029)(585.36166574,31.68119022)(585.54165771,31.76119436)
\curveto(585.68166542,31.82119008)(585.82666527,31.87619003)(585.97665771,31.92619436)
\curveto(586.12666497,31.97618993)(586.27166483,32.03118987)(586.41165771,32.09119436)
\curveto(586.86166424,32.27118963)(587.31666378,32.44118946)(587.77665771,32.60119436)
\curveto(588.22666287,32.76118914)(588.67666242,32.93118897)(589.12665771,33.11119436)
\curveto(589.17666192,33.13118877)(589.22666187,33.14618876)(589.27665771,33.15619436)
\lineto(589.42665771,33.21619436)
\curveto(589.64666145,33.3061886)(589.87166123,33.39118851)(590.10165771,33.47119436)
\curveto(590.32166078,33.55118835)(590.54166056,33.63618827)(590.76165771,33.72619436)
\curveto(590.85166025,33.76618814)(590.96166014,33.8061881)(591.09165771,33.84619436)
\curveto(591.21165989,33.88618802)(591.28165982,33.95118795)(591.30165771,34.04119436)
\curveto(591.31165979,34.08118782)(591.31165979,34.11118779)(591.30165771,34.13119436)
\lineto(591.24165771,34.19119436)
\curveto(591.19165991,34.24118766)(591.13665996,34.27618763)(591.07665771,34.29619436)
\curveto(591.01666008,34.32618758)(590.95166015,34.35618755)(590.88165771,34.38619436)
\lineto(590.25165771,34.62619436)
\curveto(590.03166107,34.7061872)(589.81666128,34.78618712)(589.60665771,34.86619436)
\lineto(589.45665771,34.92619436)
\lineto(589.27665771,34.98619436)
\curveto(589.08666201,35.06618684)(588.8966622,35.13618677)(588.70665771,35.19619436)
\curveto(588.50666259,35.26618664)(588.30666279,35.34118656)(588.10665771,35.42119436)
\curveto(587.52666357,35.66118624)(586.94166416,35.88118602)(586.35165771,36.08119436)
\curveto(585.76166534,36.29118561)(585.17666592,36.51618539)(584.59665771,36.75619436)
\curveto(584.3966667,36.83618507)(584.19166691,36.91118499)(583.98165771,36.98119436)
\curveto(583.77166733,37.06118484)(583.56666753,37.14118476)(583.36665771,37.22119436)
\curveto(583.28666781,37.26118464)(583.18666791,37.29618461)(583.06665771,37.32619436)
\curveto(582.94666815,37.36618454)(582.86166824,37.42118448)(582.81165771,37.49119436)
\curveto(582.77166833,37.55118435)(582.74166836,37.62618428)(582.72165771,37.71619436)
\curveto(582.7016684,37.81618409)(582.69166841,37.92618398)(582.69165771,38.04619436)
\curveto(582.68166842,38.16618374)(582.68166842,38.28618362)(582.69165771,38.40619436)
\curveto(582.69166841,38.52618338)(582.69166841,38.63618327)(582.69165771,38.73619436)
\curveto(582.69166841,38.82618308)(582.69166841,38.91618299)(582.69165771,39.00619436)
\curveto(582.69166841,39.1061828)(582.71166839,39.18118272)(582.75165771,39.23119436)
\curveto(582.8016683,39.32118258)(582.89166821,39.37118253)(583.02165771,39.38119436)
\curveto(583.15166795,39.39118251)(583.29166781,39.39618251)(583.44165771,39.39619436)
\lineto(585.09165771,39.39619436)
\lineto(591.36165771,39.39619436)
\lineto(592.62165771,39.39619436)
\curveto(592.73165837,39.39618251)(592.84165826,39.39618251)(592.95165771,39.39619436)
\curveto(593.06165804,39.4061825)(593.14665795,39.38618252)(593.20665771,39.33619436)
\curveto(593.26665783,39.3061826)(593.30665779,39.26118264)(593.32665771,39.20119436)
\curveto(593.33665776,39.14118276)(593.35165775,39.07118283)(593.37165771,38.99119436)
\lineto(593.37165771,38.75119436)
\lineto(593.37165771,38.39119436)
\curveto(593.36165774,38.28118362)(593.31665778,38.2011837)(593.23665771,38.15119436)
\curveto(593.20665789,38.13118377)(593.17665792,38.11618379)(593.14665771,38.10619436)
\curveto(593.10665799,38.1061838)(593.06165804,38.09618381)(593.01165771,38.07619436)
\lineto(592.84665771,38.07619436)
\curveto(592.78665831,38.06618384)(592.71665838,38.06118384)(592.63665771,38.06119436)
\curveto(592.55665854,38.07118383)(592.48165862,38.07618383)(592.41165771,38.07619436)
\lineto(591.57165771,38.07619436)
\lineto(587.14665771,38.07619436)
\curveto(586.8966642,38.07618383)(586.64666445,38.07618383)(586.39665771,38.07619436)
\curveto(586.13666496,38.07618383)(585.88666521,38.07118383)(585.64665771,38.06119436)
\curveto(585.54666555,38.06118384)(585.43666566,38.05618385)(585.31665771,38.04619436)
\curveto(585.1966659,38.03618387)(585.13666596,37.98118392)(585.13665771,37.88119436)
\lineto(585.15165771,37.88119436)
\curveto(585.17166593,37.81118409)(585.23666586,37.75118415)(585.34665771,37.70119436)
\curveto(585.45666564,37.66118424)(585.55166555,37.62618428)(585.63165771,37.59619436)
\curveto(585.8016653,37.52618438)(585.97666512,37.46118444)(586.15665771,37.40119436)
\curveto(586.32666477,37.34118456)(586.4966646,37.27118463)(586.66665771,37.19119436)
\curveto(586.71666438,37.17118473)(586.76166434,37.15618475)(586.80165771,37.14619436)
\curveto(586.84166426,37.13618477)(586.88666421,37.12118478)(586.93665771,37.10119436)
\curveto(587.11666398,37.02118488)(587.3016638,36.95118495)(587.49165771,36.89119436)
\curveto(587.67166343,36.84118506)(587.85166325,36.77618513)(588.03165771,36.69619436)
\curveto(588.18166292,36.62618528)(588.33666276,36.56618534)(588.49665771,36.51619436)
\curveto(588.64666245,36.46618544)(588.7966623,36.41118549)(588.94665771,36.35119436)
\curveto(589.41666168,36.15118575)(589.89166121,35.97118593)(590.37165771,35.81119436)
\curveto(590.84166026,35.65118625)(591.30665979,35.47618643)(591.76665771,35.28619436)
\curveto(591.94665915,35.2061867)(592.12665897,35.13618677)(592.30665771,35.07619436)
\curveto(592.48665861,35.01618689)(592.66665843,34.95118695)(592.84665771,34.88119436)
\curveto(592.95665814,34.83118707)(593.06165804,34.78118712)(593.16165771,34.73119436)
\curveto(593.25165785,34.69118721)(593.31665778,34.6061873)(593.35665771,34.47619436)
\curveto(593.36665773,34.45618745)(593.37165773,34.43118747)(593.37165771,34.40119436)
\curveto(593.36165774,34.38118752)(593.36165774,34.35618755)(593.37165771,34.32619436)
\curveto(593.38165772,34.29618761)(593.38665771,34.26118764)(593.38665771,34.22119436)
\curveto(593.37665772,34.18118772)(593.37165773,34.14118776)(593.37165771,34.10119436)
\lineto(593.37165771,33.80119436)
\curveto(593.37165773,33.7011882)(593.34665775,33.62118828)(593.29665771,33.56119436)
\curveto(593.24665785,33.48118842)(593.17665792,33.42118848)(593.08665771,33.38119436)
\curveto(592.98665811,33.35118855)(592.88665821,33.31118859)(592.78665771,33.26119436)
\curveto(592.58665851,33.18118872)(592.38165872,33.1011888)(592.17165771,33.02119436)
\curveto(591.95165915,32.95118895)(591.74165936,32.87618903)(591.54165771,32.79619436)
\curveto(591.36165974,32.71618919)(591.18165992,32.64618926)(591.00165771,32.58619436)
\curveto(590.81166029,32.53618937)(590.62666047,32.47118943)(590.44665771,32.39119436)
\curveto(589.88666121,32.16118974)(589.32166178,31.94618996)(588.75165771,31.74619436)
\curveto(588.18166292,31.54619036)(587.61666348,31.33119057)(587.05665771,31.10119436)
\lineto(586.42665771,30.86119436)
\curveto(586.20666489,30.79119111)(585.9966651,30.71619119)(585.79665771,30.63619436)
\curveto(585.68666541,30.58619132)(585.58166552,30.54119136)(585.48165771,30.50119436)
\curveto(585.37166573,30.47119143)(585.27666582,30.42119148)(585.19665771,30.35119436)
\curveto(585.17666592,30.34119156)(585.16666593,30.33119157)(585.16665771,30.32119436)
\lineto(585.13665771,30.29119436)
\lineto(585.13665771,30.21619436)
\lineto(585.16665771,30.18619436)
\curveto(585.16666593,30.17619173)(585.17166593,30.16619174)(585.18165771,30.15619436)
\curveto(585.23166587,30.13619177)(585.28666581,30.12619178)(585.34665771,30.12619436)
\curveto(585.40666569,30.12619178)(585.46666563,30.11619179)(585.52665771,30.09619436)
\lineto(585.69165771,30.09619436)
\curveto(585.75166535,30.07619183)(585.81666528,30.07119183)(585.88665771,30.08119436)
\curveto(585.95666514,30.09119181)(586.02666507,30.09619181)(586.09665771,30.09619436)
\lineto(586.90665771,30.09619436)
\lineto(591.46665771,30.09619436)
\lineto(592.65165771,30.09619436)
\curveto(592.76165834,30.09619181)(592.87165823,30.09119181)(592.98165771,30.08119436)
\curveto(593.09165801,30.08119182)(593.17665792,30.05619185)(593.23665771,30.00619436)
\curveto(593.31665778,29.95619195)(593.36165774,29.86619204)(593.37165771,29.73619436)
\lineto(593.37165771,29.34619436)
\lineto(593.37165771,29.15119436)
\curveto(593.37165773,29.1011928)(593.36165774,29.05119285)(593.34165771,29.00119436)
\curveto(593.3016578,28.87119303)(593.21665788,28.79619311)(593.08665771,28.77619436)
\curveto(592.95665814,28.76619314)(592.80665829,28.76119314)(592.63665771,28.76119436)
\lineto(590.89665771,28.76119436)
\lineto(584.89665771,28.76119436)
\lineto(583.48665771,28.76119436)
\curveto(583.37666772,28.76119314)(583.26166784,28.75619315)(583.14165771,28.74619436)
\curveto(583.02166808,28.74619316)(582.92666817,28.77119313)(582.85665771,28.82119436)
\curveto(582.7966683,28.86119304)(582.74666835,28.93619297)(582.70665771,29.04619436)
\curveto(582.6966684,29.06619284)(582.6966684,29.08619282)(582.70665771,29.10619436)
\curveto(582.70666839,29.13619277)(582.7016684,29.16119274)(582.69165771,29.18119436)
}
}
{
\newrgbcolor{curcolor}{0 0 0}
\pscustom[linestyle=none,fillstyle=solid,fillcolor=curcolor]
{
\newpath
\moveto(592.81665771,48.38330373)
\curveto(592.97665812,48.4132959)(593.11165799,48.39829592)(593.22165771,48.33830373)
\curveto(593.32165778,48.27829604)(593.3966577,48.19829612)(593.44665771,48.09830373)
\curveto(593.46665763,48.04829627)(593.47665762,47.99329632)(593.47665771,47.93330373)
\curveto(593.47665762,47.88329643)(593.48665761,47.82829649)(593.50665771,47.76830373)
\curveto(593.55665754,47.54829677)(593.54165756,47.32829699)(593.46165771,47.10830373)
\curveto(593.39165771,46.89829742)(593.3016578,46.75329756)(593.19165771,46.67330373)
\curveto(593.12165798,46.62329769)(593.04165806,46.57829774)(592.95165771,46.53830373)
\curveto(592.85165825,46.49829782)(592.77165833,46.44829787)(592.71165771,46.38830373)
\curveto(592.69165841,46.36829795)(592.67165843,46.34329797)(592.65165771,46.31330373)
\curveto(592.63165847,46.29329802)(592.62665847,46.26329805)(592.63665771,46.22330373)
\curveto(592.66665843,46.1132982)(592.72165838,46.00829831)(592.80165771,45.90830373)
\curveto(592.88165822,45.8182985)(592.95165815,45.72829859)(593.01165771,45.63830373)
\curveto(593.09165801,45.50829881)(593.16665793,45.36829895)(593.23665771,45.21830373)
\curveto(593.2966578,45.06829925)(593.35165775,44.90829941)(593.40165771,44.73830373)
\curveto(593.43165767,44.63829968)(593.45165765,44.52829979)(593.46165771,44.40830373)
\curveto(593.47165763,44.29830002)(593.48665761,44.18830013)(593.50665771,44.07830373)
\curveto(593.51665758,44.02830029)(593.52165758,43.98330033)(593.52165771,43.94330373)
\lineto(593.52165771,43.83830373)
\curveto(593.54165756,43.72830059)(593.54165756,43.62330069)(593.52165771,43.52330373)
\lineto(593.52165771,43.38830373)
\curveto(593.51165759,43.33830098)(593.50665759,43.28830103)(593.50665771,43.23830373)
\curveto(593.50665759,43.18830113)(593.4966576,43.14330117)(593.47665771,43.10330373)
\curveto(593.46665763,43.06330125)(593.46165764,43.02830129)(593.46165771,42.99830373)
\curveto(593.47165763,42.97830134)(593.47165763,42.95330136)(593.46165771,42.92330373)
\lineto(593.40165771,42.68330373)
\curveto(593.39165771,42.60330171)(593.37165773,42.52830179)(593.34165771,42.45830373)
\curveto(593.21165789,42.15830216)(593.06665803,41.9133024)(592.90665771,41.72330373)
\curveto(592.73665836,41.54330277)(592.5016586,41.39330292)(592.20165771,41.27330373)
\curveto(591.98165912,41.18330313)(591.71665938,41.13830318)(591.40665771,41.13830373)
\lineto(591.09165771,41.13830373)
\curveto(591.04166006,41.14830317)(590.99166011,41.15330316)(590.94165771,41.15330373)
\lineto(590.76165771,41.18330373)
\lineto(590.43165771,41.30330373)
\curveto(590.32166078,41.34330297)(590.22166088,41.39330292)(590.13165771,41.45330373)
\curveto(589.84166126,41.63330268)(589.62666147,41.87830244)(589.48665771,42.18830373)
\curveto(589.34666175,42.49830182)(589.22166188,42.83830148)(589.11165771,43.20830373)
\curveto(589.07166203,43.34830097)(589.04166206,43.49330082)(589.02165771,43.64330373)
\curveto(589.0016621,43.79330052)(588.97666212,43.94330037)(588.94665771,44.09330373)
\curveto(588.92666217,44.16330015)(588.91666218,44.22830009)(588.91665771,44.28830373)
\curveto(588.91666218,44.35829996)(588.90666219,44.43329988)(588.88665771,44.51330373)
\curveto(588.86666223,44.58329973)(588.85666224,44.65329966)(588.85665771,44.72330373)
\curveto(588.84666225,44.79329952)(588.83166227,44.86829945)(588.81165771,44.94830373)
\curveto(588.75166235,45.19829912)(588.7016624,45.43329888)(588.66165771,45.65330373)
\curveto(588.61166249,45.87329844)(588.4966626,46.04829827)(588.31665771,46.17830373)
\curveto(588.23666286,46.23829808)(588.13666296,46.28829803)(588.01665771,46.32830373)
\curveto(587.88666321,46.36829795)(587.74666335,46.36829795)(587.59665771,46.32830373)
\curveto(587.35666374,46.26829805)(587.16666393,46.17829814)(587.02665771,46.05830373)
\curveto(586.88666421,45.94829837)(586.77666432,45.78829853)(586.69665771,45.57830373)
\curveto(586.64666445,45.45829886)(586.61166449,45.313299)(586.59165771,45.14330373)
\curveto(586.57166453,44.98329933)(586.56166454,44.8132995)(586.56165771,44.63330373)
\curveto(586.56166454,44.45329986)(586.57166453,44.27830004)(586.59165771,44.10830373)
\curveto(586.61166449,43.93830038)(586.64166446,43.79330052)(586.68165771,43.67330373)
\curveto(586.74166436,43.50330081)(586.82666427,43.33830098)(586.93665771,43.17830373)
\curveto(586.9966641,43.09830122)(587.07666402,43.02330129)(587.17665771,42.95330373)
\curveto(587.26666383,42.89330142)(587.36666373,42.83830148)(587.47665771,42.78830373)
\curveto(587.55666354,42.75830156)(587.64166346,42.72830159)(587.73165771,42.69830373)
\curveto(587.82166328,42.67830164)(587.89166321,42.63330168)(587.94165771,42.56330373)
\curveto(587.97166313,42.52330179)(587.9966631,42.45330186)(588.01665771,42.35330373)
\curveto(588.02666307,42.26330205)(588.03166307,42.16830215)(588.03165771,42.06830373)
\curveto(588.03166307,41.96830235)(588.02666307,41.86830245)(588.01665771,41.76830373)
\curveto(587.9966631,41.67830264)(587.97166313,41.6133027)(587.94165771,41.57330373)
\curveto(587.91166319,41.53330278)(587.86166324,41.50330281)(587.79165771,41.48330373)
\curveto(587.72166338,41.46330285)(587.64666345,41.46330285)(587.56665771,41.48330373)
\curveto(587.43666366,41.5133028)(587.31666378,41.54330277)(587.20665771,41.57330373)
\curveto(587.08666401,41.6133027)(586.97166413,41.65830266)(586.86165771,41.70830373)
\curveto(586.51166459,41.89830242)(586.24166486,42.13830218)(586.05165771,42.42830373)
\curveto(585.85166525,42.7183016)(585.69166541,43.07830124)(585.57165771,43.50830373)
\curveto(585.55166555,43.60830071)(585.53666556,43.70830061)(585.52665771,43.80830373)
\curveto(585.51666558,43.9183004)(585.5016656,44.02830029)(585.48165771,44.13830373)
\curveto(585.47166563,44.17830014)(585.47166563,44.24330007)(585.48165771,44.33330373)
\curveto(585.48166562,44.42329989)(585.47166563,44.47829984)(585.45165771,44.49830373)
\curveto(585.44166566,45.19829912)(585.52166558,45.80829851)(585.69165771,46.32830373)
\curveto(585.86166524,46.84829747)(586.18666491,47.2132971)(586.66665771,47.42330373)
\curveto(586.86666423,47.5132968)(587.101664,47.56329675)(587.37165771,47.57330373)
\curveto(587.63166347,47.59329672)(587.90666319,47.60329671)(588.19665771,47.60330373)
\lineto(591.51165771,47.60330373)
\curveto(591.65165945,47.60329671)(591.78665931,47.60829671)(591.91665771,47.61830373)
\curveto(592.04665905,47.62829669)(592.15165895,47.65829666)(592.23165771,47.70830373)
\curveto(592.3016588,47.75829656)(592.35165875,47.82329649)(592.38165771,47.90330373)
\curveto(592.42165868,47.99329632)(592.45165865,48.07829624)(592.47165771,48.15830373)
\curveto(592.48165862,48.23829608)(592.52665857,48.29829602)(592.60665771,48.33830373)
\curveto(592.63665846,48.35829596)(592.66665843,48.36829595)(592.69665771,48.36830373)
\curveto(592.72665837,48.36829595)(592.76665833,48.37329594)(592.81665771,48.38330373)
\moveto(591.15165771,46.23830373)
\curveto(591.01166009,46.29829802)(590.85166025,46.32829799)(590.67165771,46.32830373)
\curveto(590.48166062,46.33829798)(590.28666081,46.34329797)(590.08665771,46.34330373)
\curveto(589.97666112,46.34329797)(589.87666122,46.33829798)(589.78665771,46.32830373)
\curveto(589.6966614,46.318298)(589.62666147,46.27829804)(589.57665771,46.20830373)
\curveto(589.55666154,46.17829814)(589.54666155,46.10829821)(589.54665771,45.99830373)
\curveto(589.56666153,45.97829834)(589.57666152,45.94329837)(589.57665771,45.89330373)
\curveto(589.57666152,45.84329847)(589.58666151,45.79829852)(589.60665771,45.75830373)
\curveto(589.62666147,45.67829864)(589.64666145,45.58829873)(589.66665771,45.48830373)
\lineto(589.72665771,45.18830373)
\curveto(589.72666137,45.15829916)(589.73166137,45.12329919)(589.74165771,45.08330373)
\lineto(589.74165771,44.97830373)
\curveto(589.78166132,44.82829949)(589.80666129,44.66329965)(589.81665771,44.48330373)
\curveto(589.81666128,44.3133)(589.83666126,44.15330016)(589.87665771,44.00330373)
\curveto(589.8966612,43.92330039)(589.91666118,43.84830047)(589.93665771,43.77830373)
\curveto(589.94666115,43.7183006)(589.96166114,43.64830067)(589.98165771,43.56830373)
\curveto(590.03166107,43.40830091)(590.096661,43.25830106)(590.17665771,43.11830373)
\curveto(590.24666085,42.97830134)(590.33666076,42.85830146)(590.44665771,42.75830373)
\curveto(590.55666054,42.65830166)(590.69166041,42.58330173)(590.85165771,42.53330373)
\curveto(591.0016601,42.48330183)(591.18665991,42.46330185)(591.40665771,42.47330373)
\curveto(591.50665959,42.47330184)(591.6016595,42.48830183)(591.69165771,42.51830373)
\curveto(591.77165933,42.55830176)(591.84665925,42.60330171)(591.91665771,42.65330373)
\curveto(592.02665907,42.73330158)(592.12165898,42.83830148)(592.20165771,42.96830373)
\curveto(592.27165883,43.09830122)(592.33165877,43.23830108)(592.38165771,43.38830373)
\curveto(592.39165871,43.43830088)(592.3966587,43.48830083)(592.39665771,43.53830373)
\curveto(592.3966587,43.58830073)(592.4016587,43.63830068)(592.41165771,43.68830373)
\curveto(592.43165867,43.75830056)(592.44665865,43.84330047)(592.45665771,43.94330373)
\curveto(592.45665864,44.05330026)(592.44665865,44.14330017)(592.42665771,44.21330373)
\curveto(592.40665869,44.27330004)(592.4016587,44.33329998)(592.41165771,44.39330373)
\curveto(592.41165869,44.45329986)(592.4016587,44.5132998)(592.38165771,44.57330373)
\curveto(592.36165874,44.65329966)(592.34665875,44.72829959)(592.33665771,44.79830373)
\curveto(592.32665877,44.87829944)(592.30665879,44.95329936)(592.27665771,45.02330373)
\curveto(592.15665894,45.313299)(592.01165909,45.55829876)(591.84165771,45.75830373)
\curveto(591.67165943,45.96829835)(591.44165966,46.12829819)(591.15165771,46.23830373)
}
}
{
\newrgbcolor{curcolor}{0 0 0}
\pscustom[linestyle=none,fillstyle=solid,fillcolor=curcolor]
{
\newpath
\moveto(585.64665771,49.26994436)
\lineto(585.64665771,49.71994436)
\curveto(585.63666546,49.88994311)(585.65666544,50.01494298)(585.70665771,50.09494436)
\curveto(585.75666534,50.17494282)(585.82166528,50.22994277)(585.90165771,50.25994436)
\curveto(585.98166512,50.2999427)(586.06666503,50.33994266)(586.15665771,50.37994436)
\curveto(586.28666481,50.42994257)(586.41666468,50.47494252)(586.54665771,50.51494436)
\curveto(586.67666442,50.55494244)(586.80666429,50.5999424)(586.93665771,50.64994436)
\curveto(587.05666404,50.6999423)(587.18166392,50.74494225)(587.31165771,50.78494436)
\curveto(587.43166367,50.82494217)(587.55166355,50.86994213)(587.67165771,50.91994436)
\curveto(587.78166332,50.96994203)(587.8966632,51.00994199)(588.01665771,51.03994436)
\curveto(588.13666296,51.06994193)(588.25666284,51.10994189)(588.37665771,51.15994436)
\curveto(588.66666243,51.27994172)(588.96666213,51.38994161)(589.27665771,51.48994436)
\curveto(589.58666151,51.58994141)(589.88666121,51.6999413)(590.17665771,51.81994436)
\curveto(590.21666088,51.83994116)(590.25666084,51.84994115)(590.29665771,51.84994436)
\curveto(590.32666077,51.84994115)(590.35666074,51.85994114)(590.38665771,51.87994436)
\curveto(590.52666057,51.93994106)(590.67166043,51.994941)(590.82165771,52.04494436)
\lineto(591.24165771,52.19494436)
\curveto(591.31165979,52.22494077)(591.38665971,52.25494074)(591.46665771,52.28494436)
\curveto(591.53665956,52.31494068)(591.58165952,52.36494063)(591.60165771,52.43494436)
\curveto(591.63165947,52.51494048)(591.60665949,52.57494042)(591.52665771,52.61494436)
\curveto(591.43665966,52.66494033)(591.36665973,52.6999403)(591.31665771,52.71994436)
\curveto(591.14665995,52.7999402)(590.96666013,52.86494013)(590.77665771,52.91494436)
\curveto(590.58666051,52.96494003)(590.4016607,53.02493997)(590.22165771,53.09494436)
\curveto(589.99166111,53.18493981)(589.76166134,53.26493973)(589.53165771,53.33494436)
\curveto(589.29166181,53.40493959)(589.06166204,53.48993951)(588.84165771,53.58994436)
\curveto(588.79166231,53.5999394)(588.72666237,53.61493938)(588.64665771,53.63494436)
\curveto(588.55666254,53.67493932)(588.46666263,53.70993929)(588.37665771,53.73994436)
\curveto(588.27666282,53.76993923)(588.18666291,53.7999392)(588.10665771,53.82994436)
\curveto(588.05666304,53.84993915)(588.01166309,53.86493913)(587.97165771,53.87494436)
\curveto(587.93166317,53.88493911)(587.88666321,53.8999391)(587.83665771,53.91994436)
\curveto(587.71666338,53.96993903)(587.5966635,54.00993899)(587.47665771,54.03994436)
\curveto(587.34666375,54.07993892)(587.22166388,54.12493887)(587.10165771,54.17494436)
\curveto(587.05166405,54.1949388)(587.00666409,54.20993879)(586.96665771,54.21994436)
\curveto(586.92666417,54.22993877)(586.88166422,54.24493875)(586.83165771,54.26494436)
\curveto(586.74166436,54.30493869)(586.65166445,54.33993866)(586.56165771,54.36994436)
\curveto(586.46166464,54.3999386)(586.36666473,54.42993857)(586.27665771,54.45994436)
\curveto(586.1966649,54.48993851)(586.11666498,54.51493848)(586.03665771,54.53494436)
\curveto(585.94666515,54.56493843)(585.87166523,54.60493839)(585.81165771,54.65494436)
\curveto(585.72166538,54.72493827)(585.67166543,54.81993818)(585.66165771,54.93994436)
\curveto(585.65166545,55.06993793)(585.64666545,55.20993779)(585.64665771,55.35994436)
\curveto(585.64666545,55.43993756)(585.65166545,55.51493748)(585.66165771,55.58494436)
\curveto(585.66166544,55.66493733)(585.67666542,55.72993727)(585.70665771,55.77994436)
\curveto(585.76666533,55.86993713)(585.86166524,55.8949371)(585.99165771,55.85494436)
\curveto(586.12166498,55.81493718)(586.22166488,55.77993722)(586.29165771,55.74994436)
\lineto(586.35165771,55.71994436)
\curveto(586.37166473,55.71993728)(586.39166471,55.71493728)(586.41165771,55.70494436)
\curveto(586.69166441,55.5949374)(586.97666412,55.48493751)(587.26665771,55.37494436)
\lineto(588.10665771,55.04494436)
\curveto(588.18666291,55.01493798)(588.26166284,54.98993801)(588.33165771,54.96994436)
\curveto(588.39166271,54.94993805)(588.45666264,54.92493807)(588.52665771,54.89494436)
\curveto(588.72666237,54.81493818)(588.93166217,54.73493826)(589.14165771,54.65494436)
\curveto(589.34166176,54.58493841)(589.54166156,54.50993849)(589.74165771,54.42994436)
\curveto(590.43166067,54.13993886)(591.12665997,53.86993913)(591.82665771,53.61994436)
\curveto(592.52665857,53.36993963)(593.22165788,53.0999399)(593.91165771,52.80994436)
\lineto(594.06165771,52.74994436)
\curveto(594.12165698,52.73994026)(594.18165692,52.72494027)(594.24165771,52.70494436)
\curveto(594.61165649,52.54494045)(594.97665612,52.37494062)(595.33665771,52.19494436)
\curveto(595.70665539,52.01494098)(595.99165511,51.76494123)(596.19165771,51.44494436)
\curveto(596.25165485,51.33494166)(596.2966548,51.22494177)(596.32665771,51.11494436)
\curveto(596.36665473,51.00494199)(596.4016547,50.87994212)(596.43165771,50.73994436)
\curveto(596.45165465,50.68994231)(596.45665464,50.63494236)(596.44665771,50.57494436)
\curveto(596.43665466,50.52494247)(596.43665466,50.46994253)(596.44665771,50.40994436)
\curveto(596.46665463,50.32994267)(596.46665463,50.24994275)(596.44665771,50.16994436)
\curveto(596.43665466,50.12994287)(596.43165467,50.07994292)(596.43165771,50.01994436)
\lineto(596.37165771,49.77994436)
\curveto(596.35165475,49.70994329)(596.31165479,49.65494334)(596.25165771,49.61494436)
\curveto(596.19165491,49.56494343)(596.11665498,49.53494346)(596.02665771,49.52494436)
\lineto(595.75665771,49.52494436)
\lineto(595.54665771,49.52494436)
\curveto(595.48665561,49.53494346)(595.43665566,49.55494344)(595.39665771,49.58494436)
\curveto(595.28665581,49.65494334)(595.25665584,49.77494322)(595.30665771,49.94494436)
\curveto(595.32665577,50.05494294)(595.33665576,50.17494282)(595.33665771,50.30494436)
\curveto(595.33665576,50.43494256)(595.31665578,50.54994245)(595.27665771,50.64994436)
\curveto(595.22665587,50.7999422)(595.15165595,50.91994208)(595.05165771,51.00994436)
\curveto(594.95165615,51.10994189)(594.83665626,51.1949418)(594.70665771,51.26494436)
\curveto(594.58665651,51.33494166)(594.45665664,51.3949416)(594.31665771,51.44494436)
\lineto(593.89665771,51.62494436)
\curveto(593.80665729,51.66494133)(593.6966574,51.70494129)(593.56665771,51.74494436)
\curveto(593.43665766,51.7949412)(593.3016578,51.7999412)(593.16165771,51.75994436)
\curveto(593.0016581,51.70994129)(592.85165825,51.65494134)(592.71165771,51.59494436)
\curveto(592.57165853,51.54494145)(592.43165867,51.48994151)(592.29165771,51.42994436)
\curveto(592.08165902,51.33994166)(591.87165923,51.25494174)(591.66165771,51.17494436)
\curveto(591.45165965,51.0949419)(591.24665985,51.01494198)(591.04665771,50.93494436)
\curveto(590.90666019,50.87494212)(590.77166033,50.81994218)(590.64165771,50.76994436)
\curveto(590.51166059,50.71994228)(590.37666072,50.66994233)(590.23665771,50.61994436)
\lineto(588.91665771,50.07994436)
\curveto(588.47666262,49.90994309)(588.03666306,49.73494326)(587.59665771,49.55494436)
\curveto(587.36666373,49.45494354)(587.14666395,49.36494363)(586.93665771,49.28494436)
\curveto(586.71666438,49.20494379)(586.4966646,49.11994388)(586.27665771,49.02994436)
\curveto(586.21666488,49.00994399)(586.13666496,48.97994402)(586.03665771,48.93994436)
\curveto(585.92666517,48.8999441)(585.83666526,48.90494409)(585.76665771,48.95494436)
\curveto(585.71666538,48.98494401)(585.68166542,49.04494395)(585.66165771,49.13494436)
\curveto(585.65166545,49.15494384)(585.65166545,49.17494382)(585.66165771,49.19494436)
\curveto(585.66166544,49.22494377)(585.65666544,49.24994375)(585.64665771,49.26994436)
}
}
{
\newrgbcolor{curcolor}{0 0 0}
\pscustom[linestyle=none,fillstyle=solid,fillcolor=curcolor]
{
}
}
{
\newrgbcolor{curcolor}{0 0 0}
\pscustom[linestyle=none,fillstyle=solid,fillcolor=curcolor]
{
\newpath
\moveto(582.76665771,64.24510061)
\curveto(582.75666834,64.93509597)(582.87666822,65.53509537)(583.12665771,66.04510061)
\curveto(583.37666772,66.56509434)(583.71166739,66.96009395)(584.13165771,67.23010061)
\curveto(584.21166689,67.28009363)(584.3016668,67.32509358)(584.40165771,67.36510061)
\curveto(584.49166661,67.4050935)(584.58666651,67.45009346)(584.68665771,67.50010061)
\curveto(584.78666631,67.54009337)(584.88666621,67.57009334)(584.98665771,67.59010061)
\curveto(585.08666601,67.6100933)(585.19166591,67.63009328)(585.30165771,67.65010061)
\curveto(585.35166575,67.67009324)(585.3966657,67.67509323)(585.43665771,67.66510061)
\curveto(585.47666562,67.65509325)(585.52166558,67.66009325)(585.57165771,67.68010061)
\curveto(585.62166548,67.69009322)(585.70666539,67.69509321)(585.82665771,67.69510061)
\curveto(585.93666516,67.69509321)(586.02166508,67.69009322)(586.08165771,67.68010061)
\curveto(586.14166496,67.66009325)(586.2016649,67.65009326)(586.26165771,67.65010061)
\curveto(586.32166478,67.66009325)(586.38166472,67.65509325)(586.44165771,67.63510061)
\curveto(586.58166452,67.59509331)(586.71666438,67.56009335)(586.84665771,67.53010061)
\curveto(586.97666412,67.50009341)(587.101664,67.46009345)(587.22165771,67.41010061)
\curveto(587.36166374,67.35009356)(587.48666361,67.28009363)(587.59665771,67.20010061)
\curveto(587.70666339,67.13009378)(587.81666328,67.05509385)(587.92665771,66.97510061)
\lineto(587.98665771,66.91510061)
\curveto(588.00666309,66.905094)(588.02666307,66.89009402)(588.04665771,66.87010061)
\curveto(588.20666289,66.75009416)(588.35166275,66.61509429)(588.48165771,66.46510061)
\curveto(588.61166249,66.31509459)(588.73666236,66.15509475)(588.85665771,65.98510061)
\curveto(589.07666202,65.67509523)(589.28166182,65.38009553)(589.47165771,65.10010061)
\curveto(589.61166149,64.87009604)(589.74666135,64.64009627)(589.87665771,64.41010061)
\curveto(590.00666109,64.19009672)(590.14166096,63.97009694)(590.28165771,63.75010061)
\curveto(590.45166065,63.50009741)(590.63166047,63.26009765)(590.82165771,63.03010061)
\curveto(591.01166009,62.8100981)(591.23665986,62.62009829)(591.49665771,62.46010061)
\curveto(591.55665954,62.42009849)(591.61665948,62.38509852)(591.67665771,62.35510061)
\curveto(591.72665937,62.32509858)(591.79165931,62.29509861)(591.87165771,62.26510061)
\curveto(591.94165916,62.24509866)(592.0016591,62.24009867)(592.05165771,62.25010061)
\curveto(592.12165898,62.27009864)(592.17665892,62.3050986)(592.21665771,62.35510061)
\curveto(592.24665885,62.4050985)(592.26665883,62.46509844)(592.27665771,62.53510061)
\lineto(592.27665771,62.77510061)
\lineto(592.27665771,63.52510061)
\lineto(592.27665771,66.33010061)
\lineto(592.27665771,66.99010061)
\curveto(592.27665882,67.08009383)(592.28165882,67.16509374)(592.29165771,67.24510061)
\curveto(592.29165881,67.32509358)(592.31165879,67.39009352)(592.35165771,67.44010061)
\curveto(592.39165871,67.49009342)(592.46665863,67.53009338)(592.57665771,67.56010061)
\curveto(592.67665842,67.60009331)(592.77665832,67.6100933)(592.87665771,67.59010061)
\lineto(593.01165771,67.59010061)
\curveto(593.08165802,67.57009334)(593.14165796,67.55009336)(593.19165771,67.53010061)
\curveto(593.24165786,67.5100934)(593.28165782,67.47509343)(593.31165771,67.42510061)
\curveto(593.35165775,67.37509353)(593.37165773,67.3050936)(593.37165771,67.21510061)
\lineto(593.37165771,66.94510061)
\lineto(593.37165771,66.04510061)
\lineto(593.37165771,62.53510061)
\lineto(593.37165771,61.47010061)
\curveto(593.37165773,61.39009952)(593.37665772,61.30009961)(593.38665771,61.20010061)
\curveto(593.38665771,61.10009981)(593.37665772,61.01509989)(593.35665771,60.94510061)
\curveto(593.28665781,60.73510017)(593.10665799,60.67010024)(592.81665771,60.75010061)
\curveto(592.77665832,60.76010015)(592.74165836,60.76010015)(592.71165771,60.75010061)
\curveto(592.67165843,60.75010016)(592.62665847,60.76010015)(592.57665771,60.78010061)
\curveto(592.4966586,60.80010011)(592.41165869,60.82010009)(592.32165771,60.84010061)
\curveto(592.23165887,60.86010005)(592.14665895,60.88510002)(592.06665771,60.91510061)
\curveto(591.57665952,61.07509983)(591.16165994,61.27509963)(590.82165771,61.51510061)
\curveto(590.57166053,61.69509921)(590.34666075,61.90009901)(590.14665771,62.13010061)
\curveto(589.93666116,62.36009855)(589.74166136,62.60009831)(589.56165771,62.85010061)
\curveto(589.38166172,63.1100978)(589.21166189,63.37509753)(589.05165771,63.64510061)
\curveto(588.88166222,63.92509698)(588.70666239,64.19509671)(588.52665771,64.45510061)
\curveto(588.44666265,64.56509634)(588.37166273,64.67009624)(588.30165771,64.77010061)
\curveto(588.23166287,64.88009603)(588.15666294,64.99009592)(588.07665771,65.10010061)
\curveto(588.04666305,65.14009577)(588.01666308,65.17509573)(587.98665771,65.20510061)
\curveto(587.94666315,65.24509566)(587.91666318,65.28509562)(587.89665771,65.32510061)
\curveto(587.78666331,65.46509544)(587.66166344,65.59009532)(587.52165771,65.70010061)
\curveto(587.49166361,65.72009519)(587.46666363,65.74509516)(587.44665771,65.77510061)
\curveto(587.41666368,65.8050951)(587.38666371,65.83009508)(587.35665771,65.85010061)
\curveto(587.25666384,65.93009498)(587.15666394,65.99509491)(587.05665771,66.04510061)
\curveto(586.95666414,66.1050948)(586.84666425,66.16009475)(586.72665771,66.21010061)
\curveto(586.65666444,66.24009467)(586.58166452,66.26009465)(586.50165771,66.27010061)
\lineto(586.26165771,66.33010061)
\lineto(586.17165771,66.33010061)
\curveto(586.14166496,66.34009457)(586.11166499,66.34509456)(586.08165771,66.34510061)
\curveto(586.01166509,66.36509454)(585.91666518,66.37009454)(585.79665771,66.36010061)
\curveto(585.66666543,66.36009455)(585.56666553,66.35009456)(585.49665771,66.33010061)
\curveto(585.41666568,66.3100946)(585.34166576,66.29009462)(585.27165771,66.27010061)
\curveto(585.19166591,66.26009465)(585.11166599,66.24009467)(585.03165771,66.21010061)
\curveto(584.79166631,66.10009481)(584.59166651,65.95009496)(584.43165771,65.76010061)
\curveto(584.26166684,65.58009533)(584.12166698,65.36009555)(584.01165771,65.10010061)
\curveto(583.99166711,65.03009588)(583.97666712,64.96009595)(583.96665771,64.89010061)
\curveto(583.94666715,64.82009609)(583.92666717,64.74509616)(583.90665771,64.66510061)
\curveto(583.88666721,64.58509632)(583.87666722,64.47509643)(583.87665771,64.33510061)
\curveto(583.87666722,64.2050967)(583.88666721,64.10009681)(583.90665771,64.02010061)
\curveto(583.91666718,63.96009695)(583.92166718,63.905097)(583.92165771,63.85510061)
\curveto(583.92166718,63.8050971)(583.93166717,63.75509715)(583.95165771,63.70510061)
\curveto(583.99166711,63.6050973)(584.03166707,63.5100974)(584.07165771,63.42010061)
\curveto(584.11166699,63.34009757)(584.15666694,63.26009765)(584.20665771,63.18010061)
\curveto(584.22666687,63.15009776)(584.25166685,63.12009779)(584.28165771,63.09010061)
\curveto(584.31166679,63.07009784)(584.33666676,63.04509786)(584.35665771,63.01510061)
\lineto(584.43165771,62.94010061)
\curveto(584.45166665,62.910098)(584.47166663,62.88509802)(584.49165771,62.86510061)
\lineto(584.70165771,62.71510061)
\curveto(584.76166634,62.67509823)(584.82666627,62.63009828)(584.89665771,62.58010061)
\curveto(584.98666611,62.52009839)(585.09166601,62.47009844)(585.21165771,62.43010061)
\curveto(585.32166578,62.40009851)(585.43166567,62.36509854)(585.54165771,62.32510061)
\curveto(585.65166545,62.28509862)(585.7966653,62.26009865)(585.97665771,62.25010061)
\curveto(586.14666495,62.24009867)(586.27166483,62.2100987)(586.35165771,62.16010061)
\curveto(586.43166467,62.1100988)(586.47666462,62.03509887)(586.48665771,61.93510061)
\curveto(586.4966646,61.83509907)(586.5016646,61.72509918)(586.50165771,61.60510061)
\curveto(586.5016646,61.56509934)(586.50666459,61.52509938)(586.51665771,61.48510061)
\curveto(586.51666458,61.44509946)(586.51166459,61.4100995)(586.50165771,61.38010061)
\curveto(586.48166462,61.33009958)(586.47166463,61.28009963)(586.47165771,61.23010061)
\curveto(586.47166463,61.19009972)(586.46166464,61.15009976)(586.44165771,61.11010061)
\curveto(586.38166472,61.02009989)(586.24666485,60.97509993)(586.03665771,60.97510061)
\lineto(585.91665771,60.97510061)
\curveto(585.85666524,60.98509992)(585.7966653,60.99009992)(585.73665771,60.99010061)
\curveto(585.66666543,61.00009991)(585.6016655,61.0100999)(585.54165771,61.02010061)
\curveto(585.43166567,61.04009987)(585.33166577,61.06009985)(585.24165771,61.08010061)
\curveto(585.14166596,61.10009981)(585.04666605,61.13009978)(584.95665771,61.17010061)
\curveto(584.88666621,61.19009972)(584.82666627,61.2100997)(584.77665771,61.23010061)
\lineto(584.59665771,61.29010061)
\curveto(584.33666676,61.4100995)(584.09166701,61.56509934)(583.86165771,61.75510061)
\curveto(583.63166747,61.95509895)(583.44666765,62.17009874)(583.30665771,62.40010061)
\curveto(583.22666787,62.5100984)(583.16166794,62.62509828)(583.11165771,62.74510061)
\lineto(582.96165771,63.13510061)
\curveto(582.91166819,63.24509766)(582.88166822,63.36009755)(582.87165771,63.48010061)
\curveto(582.85166825,63.60009731)(582.82666827,63.72509718)(582.79665771,63.85510061)
\curveto(582.7966683,63.92509698)(582.7966683,63.99009692)(582.79665771,64.05010061)
\curveto(582.78666831,64.1100968)(582.77666832,64.17509673)(582.76665771,64.24510061)
}
}
{
\newrgbcolor{curcolor}{0 0 0}
\pscustom[linestyle=none,fillstyle=solid,fillcolor=curcolor]
{
\newpath
\moveto(582.76665771,73.40470998)
\curveto(582.76666833,73.50470513)(582.77666832,73.59970503)(582.79665771,73.68970998)
\curveto(582.80666829,73.77970485)(582.83666826,73.84470479)(582.88665771,73.88470998)
\curveto(582.96666813,73.94470469)(583.07166803,73.97470466)(583.20165771,73.97470998)
\lineto(583.59165771,73.97470998)
\lineto(585.09165771,73.97470998)
\lineto(591.48165771,73.97470998)
\lineto(592.65165771,73.97470998)
\lineto(592.96665771,73.97470998)
\curveto(593.06665803,73.98470465)(593.14665795,73.96970466)(593.20665771,73.92970998)
\curveto(593.28665781,73.87970475)(593.33665776,73.80470483)(593.35665771,73.70470998)
\curveto(593.36665773,73.61470502)(593.37165773,73.50470513)(593.37165771,73.37470998)
\lineto(593.37165771,73.14970998)
\curveto(593.35165775,73.06970556)(593.33665776,72.99970563)(593.32665771,72.93970998)
\curveto(593.30665779,72.87970575)(593.26665783,72.8297058)(593.20665771,72.78970998)
\curveto(593.14665795,72.74970588)(593.07165803,72.7297059)(592.98165771,72.72970998)
\lineto(592.68165771,72.72970998)
\lineto(591.58665771,72.72970998)
\lineto(586.24665771,72.72970998)
\curveto(586.15666494,72.70970592)(586.08166502,72.69470594)(586.02165771,72.68470998)
\curveto(585.95166515,72.68470595)(585.89166521,72.65470598)(585.84165771,72.59470998)
\curveto(585.79166531,72.52470611)(585.76666533,72.4347062)(585.76665771,72.32470998)
\curveto(585.75666534,72.22470641)(585.75166535,72.11470652)(585.75165771,71.99470998)
\lineto(585.75165771,70.85470998)
\lineto(585.75165771,70.35970998)
\curveto(585.74166536,70.19970843)(585.68166542,70.08970854)(585.57165771,70.02970998)
\curveto(585.54166556,70.00970862)(585.51166559,69.99970863)(585.48165771,69.99970998)
\curveto(585.44166566,69.99970863)(585.3966657,69.99470864)(585.34665771,69.98470998)
\curveto(585.22666587,69.96470867)(585.11666598,69.96970866)(585.01665771,69.99970998)
\curveto(584.91666618,70.03970859)(584.84666625,70.09470854)(584.80665771,70.16470998)
\curveto(584.75666634,70.24470839)(584.73166637,70.36470827)(584.73165771,70.52470998)
\curveto(584.73166637,70.68470795)(584.71666638,70.81970781)(584.68665771,70.92970998)
\curveto(584.67666642,70.97970765)(584.67166643,71.0347076)(584.67165771,71.09470998)
\curveto(584.66166644,71.15470748)(584.64666645,71.21470742)(584.62665771,71.27470998)
\curveto(584.57666652,71.42470721)(584.52666657,71.56970706)(584.47665771,71.70970998)
\curveto(584.41666668,71.84970678)(584.34666675,71.98470665)(584.26665771,72.11470998)
\curveto(584.17666692,72.25470638)(584.07166703,72.37470626)(583.95165771,72.47470998)
\curveto(583.83166727,72.57470606)(583.7016674,72.66970596)(583.56165771,72.75970998)
\curveto(583.46166764,72.81970581)(583.35166775,72.86470577)(583.23165771,72.89470998)
\curveto(583.11166799,72.9347057)(583.00666809,72.98470565)(582.91665771,73.04470998)
\curveto(582.85666824,73.09470554)(582.81666828,73.16470547)(582.79665771,73.25470998)
\curveto(582.78666831,73.27470536)(582.78166832,73.29970533)(582.78165771,73.32970998)
\curveto(582.78166832,73.35970527)(582.77666832,73.38470525)(582.76665771,73.40470998)
}
}
{
\newrgbcolor{curcolor}{0 0 0}
\pscustom[linestyle=none,fillstyle=solid,fillcolor=curcolor]
{
\newpath
\moveto(591.73665771,78.63431936)
\lineto(591.73665771,79.26431936)
\lineto(591.73665771,79.45931936)
\curveto(591.73665936,79.52931683)(591.74665935,79.58931677)(591.76665771,79.63931936)
\curveto(591.80665929,79.70931665)(591.84665925,79.7593166)(591.88665771,79.78931936)
\curveto(591.93665916,79.82931653)(592.0016591,79.84931651)(592.08165771,79.84931936)
\curveto(592.16165894,79.8593165)(592.24665885,79.86431649)(592.33665771,79.86431936)
\lineto(593.05665771,79.86431936)
\curveto(593.53665756,79.86431649)(593.94665715,79.80431655)(594.28665771,79.68431936)
\curveto(594.62665647,79.56431679)(594.9016562,79.36931699)(595.11165771,79.09931936)
\curveto(595.16165594,79.02931733)(595.20665589,78.9593174)(595.24665771,78.88931936)
\curveto(595.2966558,78.82931753)(595.34165576,78.7543176)(595.38165771,78.66431936)
\curveto(595.39165571,78.64431771)(595.4016557,78.61431774)(595.41165771,78.57431936)
\curveto(595.43165567,78.53431782)(595.43665566,78.48931787)(595.42665771,78.43931936)
\curveto(595.3966557,78.34931801)(595.32165578,78.29431806)(595.20165771,78.27431936)
\curveto(595.09165601,78.2543181)(594.9966561,78.26931809)(594.91665771,78.31931936)
\curveto(594.84665625,78.34931801)(594.78165632,78.39431796)(594.72165771,78.45431936)
\curveto(594.67165643,78.52431783)(594.62165648,78.58931777)(594.57165771,78.64931936)
\curveto(594.52165658,78.71931764)(594.44665665,78.77931758)(594.34665771,78.82931936)
\curveto(594.25665684,78.88931747)(594.16665693,78.93931742)(594.07665771,78.97931936)
\curveto(594.04665705,78.99931736)(593.98665711,79.02431733)(593.89665771,79.05431936)
\curveto(593.81665728,79.08431727)(593.74665735,79.08931727)(593.68665771,79.06931936)
\curveto(593.54665755,79.03931732)(593.45665764,78.97931738)(593.41665771,78.88931936)
\curveto(593.38665771,78.80931755)(593.37165773,78.71931764)(593.37165771,78.61931936)
\curveto(593.37165773,78.51931784)(593.34665775,78.43431792)(593.29665771,78.36431936)
\curveto(593.22665787,78.27431808)(593.08665801,78.22931813)(592.87665771,78.22931936)
\lineto(592.32165771,78.22931936)
\lineto(592.09665771,78.22931936)
\curveto(592.01665908,78.23931812)(591.95165915,78.2593181)(591.90165771,78.28931936)
\curveto(591.82165928,78.34931801)(591.77665932,78.41931794)(591.76665771,78.49931936)
\curveto(591.75665934,78.51931784)(591.75165935,78.53931782)(591.75165771,78.55931936)
\curveto(591.75165935,78.58931777)(591.74665935,78.61431774)(591.73665771,78.63431936)
}
}
{
\newrgbcolor{curcolor}{0 0 0}
\pscustom[linestyle=none,fillstyle=solid,fillcolor=curcolor]
{
}
}
{
\newrgbcolor{curcolor}{0 0 0}
\pscustom[linestyle=none,fillstyle=solid,fillcolor=curcolor]
{
\newpath
\moveto(582.76665771,89.26463186)
\curveto(582.75666834,89.95462722)(582.87666822,90.55462662)(583.12665771,91.06463186)
\curveto(583.37666772,91.58462559)(583.71166739,91.9796252)(584.13165771,92.24963186)
\curveto(584.21166689,92.29962488)(584.3016668,92.34462483)(584.40165771,92.38463186)
\curveto(584.49166661,92.42462475)(584.58666651,92.46962471)(584.68665771,92.51963186)
\curveto(584.78666631,92.55962462)(584.88666621,92.58962459)(584.98665771,92.60963186)
\curveto(585.08666601,92.62962455)(585.19166591,92.64962453)(585.30165771,92.66963186)
\curveto(585.35166575,92.68962449)(585.3966657,92.69462448)(585.43665771,92.68463186)
\curveto(585.47666562,92.6746245)(585.52166558,92.6796245)(585.57165771,92.69963186)
\curveto(585.62166548,92.70962447)(585.70666539,92.71462446)(585.82665771,92.71463186)
\curveto(585.93666516,92.71462446)(586.02166508,92.70962447)(586.08165771,92.69963186)
\curveto(586.14166496,92.6796245)(586.2016649,92.66962451)(586.26165771,92.66963186)
\curveto(586.32166478,92.6796245)(586.38166472,92.6746245)(586.44165771,92.65463186)
\curveto(586.58166452,92.61462456)(586.71666438,92.5796246)(586.84665771,92.54963186)
\curveto(586.97666412,92.51962466)(587.101664,92.4796247)(587.22165771,92.42963186)
\curveto(587.36166374,92.36962481)(587.48666361,92.29962488)(587.59665771,92.21963186)
\curveto(587.70666339,92.14962503)(587.81666328,92.0746251)(587.92665771,91.99463186)
\lineto(587.98665771,91.93463186)
\curveto(588.00666309,91.92462525)(588.02666307,91.90962527)(588.04665771,91.88963186)
\curveto(588.20666289,91.76962541)(588.35166275,91.63462554)(588.48165771,91.48463186)
\curveto(588.61166249,91.33462584)(588.73666236,91.174626)(588.85665771,91.00463186)
\curveto(589.07666202,90.69462648)(589.28166182,90.39962678)(589.47165771,90.11963186)
\curveto(589.61166149,89.88962729)(589.74666135,89.65962752)(589.87665771,89.42963186)
\curveto(590.00666109,89.20962797)(590.14166096,88.98962819)(590.28165771,88.76963186)
\curveto(590.45166065,88.51962866)(590.63166047,88.2796289)(590.82165771,88.04963186)
\curveto(591.01166009,87.82962935)(591.23665986,87.63962954)(591.49665771,87.47963186)
\curveto(591.55665954,87.43962974)(591.61665948,87.40462977)(591.67665771,87.37463186)
\curveto(591.72665937,87.34462983)(591.79165931,87.31462986)(591.87165771,87.28463186)
\curveto(591.94165916,87.26462991)(592.0016591,87.25962992)(592.05165771,87.26963186)
\curveto(592.12165898,87.28962989)(592.17665892,87.32462985)(592.21665771,87.37463186)
\curveto(592.24665885,87.42462975)(592.26665883,87.48462969)(592.27665771,87.55463186)
\lineto(592.27665771,87.79463186)
\lineto(592.27665771,88.54463186)
\lineto(592.27665771,91.34963186)
\lineto(592.27665771,92.00963186)
\curveto(592.27665882,92.09962508)(592.28165882,92.18462499)(592.29165771,92.26463186)
\curveto(592.29165881,92.34462483)(592.31165879,92.40962477)(592.35165771,92.45963186)
\curveto(592.39165871,92.50962467)(592.46665863,92.54962463)(592.57665771,92.57963186)
\curveto(592.67665842,92.61962456)(592.77665832,92.62962455)(592.87665771,92.60963186)
\lineto(593.01165771,92.60963186)
\curveto(593.08165802,92.58962459)(593.14165796,92.56962461)(593.19165771,92.54963186)
\curveto(593.24165786,92.52962465)(593.28165782,92.49462468)(593.31165771,92.44463186)
\curveto(593.35165775,92.39462478)(593.37165773,92.32462485)(593.37165771,92.23463186)
\lineto(593.37165771,91.96463186)
\lineto(593.37165771,91.06463186)
\lineto(593.37165771,87.55463186)
\lineto(593.37165771,86.48963186)
\curveto(593.37165773,86.40963077)(593.37665772,86.31963086)(593.38665771,86.21963186)
\curveto(593.38665771,86.11963106)(593.37665772,86.03463114)(593.35665771,85.96463186)
\curveto(593.28665781,85.75463142)(593.10665799,85.68963149)(592.81665771,85.76963186)
\curveto(592.77665832,85.7796314)(592.74165836,85.7796314)(592.71165771,85.76963186)
\curveto(592.67165843,85.76963141)(592.62665847,85.7796314)(592.57665771,85.79963186)
\curveto(592.4966586,85.81963136)(592.41165869,85.83963134)(592.32165771,85.85963186)
\curveto(592.23165887,85.8796313)(592.14665895,85.90463127)(592.06665771,85.93463186)
\curveto(591.57665952,86.09463108)(591.16165994,86.29463088)(590.82165771,86.53463186)
\curveto(590.57166053,86.71463046)(590.34666075,86.91963026)(590.14665771,87.14963186)
\curveto(589.93666116,87.3796298)(589.74166136,87.61962956)(589.56165771,87.86963186)
\curveto(589.38166172,88.12962905)(589.21166189,88.39462878)(589.05165771,88.66463186)
\curveto(588.88166222,88.94462823)(588.70666239,89.21462796)(588.52665771,89.47463186)
\curveto(588.44666265,89.58462759)(588.37166273,89.68962749)(588.30165771,89.78963186)
\curveto(588.23166287,89.89962728)(588.15666294,90.00962717)(588.07665771,90.11963186)
\curveto(588.04666305,90.15962702)(588.01666308,90.19462698)(587.98665771,90.22463186)
\curveto(587.94666315,90.26462691)(587.91666318,90.30462687)(587.89665771,90.34463186)
\curveto(587.78666331,90.48462669)(587.66166344,90.60962657)(587.52165771,90.71963186)
\curveto(587.49166361,90.73962644)(587.46666363,90.76462641)(587.44665771,90.79463186)
\curveto(587.41666368,90.82462635)(587.38666371,90.84962633)(587.35665771,90.86963186)
\curveto(587.25666384,90.94962623)(587.15666394,91.01462616)(587.05665771,91.06463186)
\curveto(586.95666414,91.12462605)(586.84666425,91.179626)(586.72665771,91.22963186)
\curveto(586.65666444,91.25962592)(586.58166452,91.2796259)(586.50165771,91.28963186)
\lineto(586.26165771,91.34963186)
\lineto(586.17165771,91.34963186)
\curveto(586.14166496,91.35962582)(586.11166499,91.36462581)(586.08165771,91.36463186)
\curveto(586.01166509,91.38462579)(585.91666518,91.38962579)(585.79665771,91.37963186)
\curveto(585.66666543,91.3796258)(585.56666553,91.36962581)(585.49665771,91.34963186)
\curveto(585.41666568,91.32962585)(585.34166576,91.30962587)(585.27165771,91.28963186)
\curveto(585.19166591,91.2796259)(585.11166599,91.25962592)(585.03165771,91.22963186)
\curveto(584.79166631,91.11962606)(584.59166651,90.96962621)(584.43165771,90.77963186)
\curveto(584.26166684,90.59962658)(584.12166698,90.3796268)(584.01165771,90.11963186)
\curveto(583.99166711,90.04962713)(583.97666712,89.9796272)(583.96665771,89.90963186)
\curveto(583.94666715,89.83962734)(583.92666717,89.76462741)(583.90665771,89.68463186)
\curveto(583.88666721,89.60462757)(583.87666722,89.49462768)(583.87665771,89.35463186)
\curveto(583.87666722,89.22462795)(583.88666721,89.11962806)(583.90665771,89.03963186)
\curveto(583.91666718,88.9796282)(583.92166718,88.92462825)(583.92165771,88.87463186)
\curveto(583.92166718,88.82462835)(583.93166717,88.7746284)(583.95165771,88.72463186)
\curveto(583.99166711,88.62462855)(584.03166707,88.52962865)(584.07165771,88.43963186)
\curveto(584.11166699,88.35962882)(584.15666694,88.2796289)(584.20665771,88.19963186)
\curveto(584.22666687,88.16962901)(584.25166685,88.13962904)(584.28165771,88.10963186)
\curveto(584.31166679,88.08962909)(584.33666676,88.06462911)(584.35665771,88.03463186)
\lineto(584.43165771,87.95963186)
\curveto(584.45166665,87.92962925)(584.47166663,87.90462927)(584.49165771,87.88463186)
\lineto(584.70165771,87.73463186)
\curveto(584.76166634,87.69462948)(584.82666627,87.64962953)(584.89665771,87.59963186)
\curveto(584.98666611,87.53962964)(585.09166601,87.48962969)(585.21165771,87.44963186)
\curveto(585.32166578,87.41962976)(585.43166567,87.38462979)(585.54165771,87.34463186)
\curveto(585.65166545,87.30462987)(585.7966653,87.2796299)(585.97665771,87.26963186)
\curveto(586.14666495,87.25962992)(586.27166483,87.22962995)(586.35165771,87.17963186)
\curveto(586.43166467,87.12963005)(586.47666462,87.05463012)(586.48665771,86.95463186)
\curveto(586.4966646,86.85463032)(586.5016646,86.74463043)(586.50165771,86.62463186)
\curveto(586.5016646,86.58463059)(586.50666459,86.54463063)(586.51665771,86.50463186)
\curveto(586.51666458,86.46463071)(586.51166459,86.42963075)(586.50165771,86.39963186)
\curveto(586.48166462,86.34963083)(586.47166463,86.29963088)(586.47165771,86.24963186)
\curveto(586.47166463,86.20963097)(586.46166464,86.16963101)(586.44165771,86.12963186)
\curveto(586.38166472,86.03963114)(586.24666485,85.99463118)(586.03665771,85.99463186)
\lineto(585.91665771,85.99463186)
\curveto(585.85666524,86.00463117)(585.7966653,86.00963117)(585.73665771,86.00963186)
\curveto(585.66666543,86.01963116)(585.6016655,86.02963115)(585.54165771,86.03963186)
\curveto(585.43166567,86.05963112)(585.33166577,86.0796311)(585.24165771,86.09963186)
\curveto(585.14166596,86.11963106)(585.04666605,86.14963103)(584.95665771,86.18963186)
\curveto(584.88666621,86.20963097)(584.82666627,86.22963095)(584.77665771,86.24963186)
\lineto(584.59665771,86.30963186)
\curveto(584.33666676,86.42963075)(584.09166701,86.58463059)(583.86165771,86.77463186)
\curveto(583.63166747,86.9746302)(583.44666765,87.18962999)(583.30665771,87.41963186)
\curveto(583.22666787,87.52962965)(583.16166794,87.64462953)(583.11165771,87.76463186)
\lineto(582.96165771,88.15463186)
\curveto(582.91166819,88.26462891)(582.88166822,88.3796288)(582.87165771,88.49963186)
\curveto(582.85166825,88.61962856)(582.82666827,88.74462843)(582.79665771,88.87463186)
\curveto(582.7966683,88.94462823)(582.7966683,89.00962817)(582.79665771,89.06963186)
\curveto(582.78666831,89.12962805)(582.77666832,89.19462798)(582.76665771,89.26463186)
}
}
{
\newrgbcolor{curcolor}{0 0 0}
\pscustom[linestyle=none,fillstyle=solid,fillcolor=curcolor]
{
\newpath
\moveto(588.28665771,101.36424123)
\lineto(588.54165771,101.36424123)
\curveto(588.62166248,101.37423353)(588.6966624,101.36923353)(588.76665771,101.34924123)
\lineto(589.00665771,101.34924123)
\lineto(589.17165771,101.34924123)
\curveto(589.27166183,101.32923357)(589.37666172,101.31923358)(589.48665771,101.31924123)
\curveto(589.58666151,101.31923358)(589.68666141,101.30923359)(589.78665771,101.28924123)
\lineto(589.93665771,101.28924123)
\curveto(590.07666102,101.25923364)(590.21666088,101.23923366)(590.35665771,101.22924123)
\curveto(590.48666061,101.21923368)(590.61666048,101.19423371)(590.74665771,101.15424123)
\curveto(590.82666027,101.13423377)(590.91166019,101.11423379)(591.00165771,101.09424123)
\lineto(591.24165771,101.03424123)
\lineto(591.54165771,100.91424123)
\curveto(591.63165947,100.88423402)(591.72165938,100.84923405)(591.81165771,100.80924123)
\curveto(592.03165907,100.70923419)(592.24665885,100.57423433)(592.45665771,100.40424123)
\curveto(592.66665843,100.24423466)(592.83665826,100.06923483)(592.96665771,99.87924123)
\curveto(593.00665809,99.82923507)(593.04665805,99.76923513)(593.08665771,99.69924123)
\curveto(593.11665798,99.63923526)(593.15165795,99.57923532)(593.19165771,99.51924123)
\curveto(593.24165786,99.43923546)(593.28165782,99.34423556)(593.31165771,99.23424123)
\curveto(593.34165776,99.12423578)(593.37165773,99.01923588)(593.40165771,98.91924123)
\curveto(593.44165766,98.80923609)(593.46665763,98.6992362)(593.47665771,98.58924123)
\curveto(593.48665761,98.47923642)(593.5016576,98.36423654)(593.52165771,98.24424123)
\curveto(593.53165757,98.2042367)(593.53165757,98.15923674)(593.52165771,98.10924123)
\curveto(593.52165758,98.06923683)(593.52665757,98.02923687)(593.53665771,97.98924123)
\curveto(593.54665755,97.94923695)(593.55165755,97.89423701)(593.55165771,97.82424123)
\curveto(593.55165755,97.75423715)(593.54665755,97.7042372)(593.53665771,97.67424123)
\curveto(593.51665758,97.62423728)(593.51165759,97.57923732)(593.52165771,97.53924123)
\curveto(593.53165757,97.4992374)(593.53165757,97.46423744)(593.52165771,97.43424123)
\lineto(593.52165771,97.34424123)
\curveto(593.5016576,97.28423762)(593.48665761,97.21923768)(593.47665771,97.14924123)
\curveto(593.47665762,97.08923781)(593.47165763,97.02423788)(593.46165771,96.95424123)
\curveto(593.41165769,96.78423812)(593.36165774,96.62423828)(593.31165771,96.47424123)
\curveto(593.26165784,96.32423858)(593.1966579,96.17923872)(593.11665771,96.03924123)
\curveto(593.07665802,95.98923891)(593.04665805,95.93423897)(593.02665771,95.87424123)
\curveto(592.9966581,95.82423908)(592.96165814,95.77423913)(592.92165771,95.72424123)
\curveto(592.74165836,95.48423942)(592.52165858,95.28423962)(592.26165771,95.12424123)
\curveto(592.0016591,94.96423994)(591.71665938,94.82424008)(591.40665771,94.70424123)
\curveto(591.26665983,94.64424026)(591.12665997,94.5992403)(590.98665771,94.56924123)
\curveto(590.83666026,94.53924036)(590.68166042,94.5042404)(590.52165771,94.46424123)
\curveto(590.41166069,94.44424046)(590.3016608,94.42924047)(590.19165771,94.41924123)
\curveto(590.08166102,94.40924049)(589.97166113,94.39424051)(589.86165771,94.37424123)
\curveto(589.82166128,94.36424054)(589.78166132,94.35924054)(589.74165771,94.35924123)
\curveto(589.7016614,94.36924053)(589.66166144,94.36924053)(589.62165771,94.35924123)
\curveto(589.57166153,94.34924055)(589.52166158,94.34424056)(589.47165771,94.34424123)
\lineto(589.30665771,94.34424123)
\curveto(589.25666184,94.32424058)(589.20666189,94.31924058)(589.15665771,94.32924123)
\curveto(589.096662,94.33924056)(589.04166206,94.33924056)(588.99165771,94.32924123)
\curveto(588.95166215,94.31924058)(588.90666219,94.31924058)(588.85665771,94.32924123)
\curveto(588.80666229,94.33924056)(588.75666234,94.33424057)(588.70665771,94.31424123)
\curveto(588.63666246,94.29424061)(588.56166254,94.28924061)(588.48165771,94.29924123)
\curveto(588.39166271,94.30924059)(588.30666279,94.31424059)(588.22665771,94.31424123)
\curveto(588.13666296,94.31424059)(588.03666306,94.30924059)(587.92665771,94.29924123)
\curveto(587.80666329,94.28924061)(587.70666339,94.29424061)(587.62665771,94.31424123)
\lineto(587.34165771,94.31424123)
\lineto(586.71165771,94.35924123)
\curveto(586.61166449,94.36924053)(586.51666458,94.37924052)(586.42665771,94.38924123)
\lineto(586.12665771,94.41924123)
\curveto(586.07666502,94.43924046)(586.02666507,94.44424046)(585.97665771,94.43424123)
\curveto(585.91666518,94.43424047)(585.86166524,94.44424046)(585.81165771,94.46424123)
\curveto(585.64166546,94.51424039)(585.47666562,94.55424035)(585.31665771,94.58424123)
\curveto(585.14666595,94.61424029)(584.98666611,94.66424024)(584.83665771,94.73424123)
\curveto(584.37666672,94.92423998)(584.0016671,95.14423976)(583.71165771,95.39424123)
\curveto(583.42166768,95.65423925)(583.17666792,96.01423889)(582.97665771,96.47424123)
\curveto(582.92666817,96.6042383)(582.89166821,96.73423817)(582.87165771,96.86424123)
\curveto(582.85166825,97.0042379)(582.82666827,97.14423776)(582.79665771,97.28424123)
\curveto(582.78666831,97.35423755)(582.78166832,97.41923748)(582.78165771,97.47924123)
\curveto(582.78166832,97.53923736)(582.77666832,97.6042373)(582.76665771,97.67424123)
\curveto(582.74666835,98.5042364)(582.8966682,99.17423573)(583.21665771,99.68424123)
\curveto(583.52666757,100.19423471)(583.96666713,100.57423433)(584.53665771,100.82424123)
\curveto(584.65666644,100.87423403)(584.78166632,100.91923398)(584.91165771,100.95924123)
\curveto(585.04166606,100.9992339)(585.17666592,101.04423386)(585.31665771,101.09424123)
\curveto(585.3966657,101.11423379)(585.48166562,101.12923377)(585.57165771,101.13924123)
\lineto(585.81165771,101.19924123)
\curveto(585.92166518,101.22923367)(586.03166507,101.24423366)(586.14165771,101.24424123)
\curveto(586.25166485,101.25423365)(586.36166474,101.26923363)(586.47165771,101.28924123)
\curveto(586.52166458,101.30923359)(586.56666453,101.31423359)(586.60665771,101.30424123)
\curveto(586.64666445,101.3042336)(586.68666441,101.30923359)(586.72665771,101.31924123)
\curveto(586.77666432,101.32923357)(586.83166427,101.32923357)(586.89165771,101.31924123)
\curveto(586.94166416,101.31923358)(586.99166411,101.32423358)(587.04165771,101.33424123)
\lineto(587.17665771,101.33424123)
\curveto(587.23666386,101.35423355)(587.30666379,101.35423355)(587.38665771,101.33424123)
\curveto(587.45666364,101.32423358)(587.52166358,101.32923357)(587.58165771,101.34924123)
\curveto(587.61166349,101.35923354)(587.65166345,101.36423354)(587.70165771,101.36424123)
\lineto(587.82165771,101.36424123)
\lineto(588.28665771,101.36424123)
\moveto(590.61165771,99.81924123)
\curveto(590.29166081,99.91923498)(589.92666117,99.97923492)(589.51665771,99.99924123)
\curveto(589.10666199,100.01923488)(588.6966624,100.02923487)(588.28665771,100.02924123)
\curveto(587.85666324,100.02923487)(587.43666366,100.01923488)(587.02665771,99.99924123)
\curveto(586.61666448,99.97923492)(586.23166487,99.93423497)(585.87165771,99.86424123)
\curveto(585.51166559,99.79423511)(585.19166591,99.68423522)(584.91165771,99.53424123)
\curveto(584.62166648,99.39423551)(584.38666671,99.1992357)(584.20665771,98.94924123)
\curveto(584.096667,98.78923611)(584.01666708,98.60923629)(583.96665771,98.40924123)
\curveto(583.90666719,98.20923669)(583.87666722,97.96423694)(583.87665771,97.67424123)
\curveto(583.8966672,97.65423725)(583.90666719,97.61923728)(583.90665771,97.56924123)
\curveto(583.8966672,97.51923738)(583.8966672,97.47923742)(583.90665771,97.44924123)
\curveto(583.92666717,97.36923753)(583.94666715,97.29423761)(583.96665771,97.22424123)
\curveto(583.97666712,97.16423774)(583.9966671,97.0992378)(584.02665771,97.02924123)
\curveto(584.14666695,96.75923814)(584.31666678,96.53923836)(584.53665771,96.36924123)
\curveto(584.74666635,96.20923869)(584.99166611,96.07423883)(585.27165771,95.96424123)
\curveto(585.38166572,95.91423899)(585.5016656,95.87423903)(585.63165771,95.84424123)
\curveto(585.75166535,95.82423908)(585.87666522,95.7992391)(586.00665771,95.76924123)
\curveto(586.05666504,95.74923915)(586.11166499,95.73923916)(586.17165771,95.73924123)
\curveto(586.22166488,95.73923916)(586.27166483,95.73423917)(586.32165771,95.72424123)
\curveto(586.41166469,95.71423919)(586.50666459,95.7042392)(586.60665771,95.69424123)
\curveto(586.6966644,95.68423922)(586.79166431,95.67423923)(586.89165771,95.66424123)
\curveto(586.97166413,95.66423924)(587.05666404,95.65923924)(587.14665771,95.64924123)
\lineto(587.38665771,95.64924123)
\lineto(587.56665771,95.64924123)
\curveto(587.5966635,95.63923926)(587.63166347,95.63423927)(587.67165771,95.63424123)
\lineto(587.80665771,95.63424123)
\lineto(588.25665771,95.63424123)
\curveto(588.33666276,95.63423927)(588.42166268,95.62923927)(588.51165771,95.61924123)
\curveto(588.59166251,95.61923928)(588.66666243,95.62923927)(588.73665771,95.64924123)
\lineto(589.00665771,95.64924123)
\curveto(589.02666207,95.64923925)(589.05666204,95.64423926)(589.09665771,95.63424123)
\curveto(589.12666197,95.63423927)(589.15166195,95.63923926)(589.17165771,95.64924123)
\curveto(589.27166183,95.65923924)(589.37166173,95.66423924)(589.47165771,95.66424123)
\curveto(589.56166154,95.67423923)(589.66166144,95.68423922)(589.77165771,95.69424123)
\curveto(589.89166121,95.72423918)(590.01666108,95.73923916)(590.14665771,95.73924123)
\curveto(590.26666083,95.74923915)(590.38166072,95.77423913)(590.49165771,95.81424123)
\curveto(590.79166031,95.89423901)(591.05666004,95.97923892)(591.28665771,96.06924123)
\curveto(591.51665958,96.16923873)(591.73165937,96.31423859)(591.93165771,96.50424123)
\curveto(592.13165897,96.71423819)(592.28165882,96.97923792)(592.38165771,97.29924123)
\curveto(592.4016587,97.33923756)(592.41165869,97.37423753)(592.41165771,97.40424123)
\curveto(592.4016587,97.44423746)(592.40665869,97.48923741)(592.42665771,97.53924123)
\curveto(592.43665866,97.57923732)(592.44665865,97.64923725)(592.45665771,97.74924123)
\curveto(592.46665863,97.85923704)(592.46165864,97.94423696)(592.44165771,98.00424123)
\curveto(592.42165868,98.07423683)(592.41165869,98.14423676)(592.41165771,98.21424123)
\curveto(592.4016587,98.28423662)(592.38665871,98.34923655)(592.36665771,98.40924123)
\curveto(592.30665879,98.60923629)(592.22165888,98.78923611)(592.11165771,98.94924123)
\curveto(592.09165901,98.97923592)(592.07165903,99.0042359)(592.05165771,99.02424123)
\lineto(591.99165771,99.08424123)
\curveto(591.97165913,99.12423578)(591.93165917,99.17423573)(591.87165771,99.23424123)
\curveto(591.73165937,99.33423557)(591.6016595,99.41923548)(591.48165771,99.48924123)
\curveto(591.36165974,99.55923534)(591.21665988,99.62923527)(591.04665771,99.69924123)
\curveto(590.97666012,99.72923517)(590.90666019,99.74923515)(590.83665771,99.75924123)
\curveto(590.76666033,99.77923512)(590.69166041,99.7992351)(590.61165771,99.81924123)
}
}
{
\newrgbcolor{curcolor}{0 0 0}
\pscustom[linestyle=none,fillstyle=solid,fillcolor=curcolor]
{
\newpath
\moveto(582.76665771,106.77385061)
\curveto(582.76666833,106.87384575)(582.77666832,106.96884566)(582.79665771,107.05885061)
\curveto(582.80666829,107.14884548)(582.83666826,107.21384541)(582.88665771,107.25385061)
\curveto(582.96666813,107.31384531)(583.07166803,107.34384528)(583.20165771,107.34385061)
\lineto(583.59165771,107.34385061)
\lineto(585.09165771,107.34385061)
\lineto(591.48165771,107.34385061)
\lineto(592.65165771,107.34385061)
\lineto(592.96665771,107.34385061)
\curveto(593.06665803,107.35384527)(593.14665795,107.33884529)(593.20665771,107.29885061)
\curveto(593.28665781,107.24884538)(593.33665776,107.17384545)(593.35665771,107.07385061)
\curveto(593.36665773,106.98384564)(593.37165773,106.87384575)(593.37165771,106.74385061)
\lineto(593.37165771,106.51885061)
\curveto(593.35165775,106.43884619)(593.33665776,106.36884626)(593.32665771,106.30885061)
\curveto(593.30665779,106.24884638)(593.26665783,106.19884643)(593.20665771,106.15885061)
\curveto(593.14665795,106.11884651)(593.07165803,106.09884653)(592.98165771,106.09885061)
\lineto(592.68165771,106.09885061)
\lineto(591.58665771,106.09885061)
\lineto(586.24665771,106.09885061)
\curveto(586.15666494,106.07884655)(586.08166502,106.06384656)(586.02165771,106.05385061)
\curveto(585.95166515,106.05384657)(585.89166521,106.0238466)(585.84165771,105.96385061)
\curveto(585.79166531,105.89384673)(585.76666533,105.80384682)(585.76665771,105.69385061)
\curveto(585.75666534,105.59384703)(585.75166535,105.48384714)(585.75165771,105.36385061)
\lineto(585.75165771,104.22385061)
\lineto(585.75165771,103.72885061)
\curveto(585.74166536,103.56884906)(585.68166542,103.45884917)(585.57165771,103.39885061)
\curveto(585.54166556,103.37884925)(585.51166559,103.36884926)(585.48165771,103.36885061)
\curveto(585.44166566,103.36884926)(585.3966657,103.36384926)(585.34665771,103.35385061)
\curveto(585.22666587,103.33384929)(585.11666598,103.33884929)(585.01665771,103.36885061)
\curveto(584.91666618,103.40884922)(584.84666625,103.46384916)(584.80665771,103.53385061)
\curveto(584.75666634,103.61384901)(584.73166637,103.73384889)(584.73165771,103.89385061)
\curveto(584.73166637,104.05384857)(584.71666638,104.18884844)(584.68665771,104.29885061)
\curveto(584.67666642,104.34884828)(584.67166643,104.40384822)(584.67165771,104.46385061)
\curveto(584.66166644,104.5238481)(584.64666645,104.58384804)(584.62665771,104.64385061)
\curveto(584.57666652,104.79384783)(584.52666657,104.93884769)(584.47665771,105.07885061)
\curveto(584.41666668,105.21884741)(584.34666675,105.35384727)(584.26665771,105.48385061)
\curveto(584.17666692,105.623847)(584.07166703,105.74384688)(583.95165771,105.84385061)
\curveto(583.83166727,105.94384668)(583.7016674,106.03884659)(583.56165771,106.12885061)
\curveto(583.46166764,106.18884644)(583.35166775,106.23384639)(583.23165771,106.26385061)
\curveto(583.11166799,106.30384632)(583.00666809,106.35384627)(582.91665771,106.41385061)
\curveto(582.85666824,106.46384616)(582.81666828,106.53384609)(582.79665771,106.62385061)
\curveto(582.78666831,106.64384598)(582.78166832,106.66884596)(582.78165771,106.69885061)
\curveto(582.78166832,106.7288459)(582.77666832,106.75384587)(582.76665771,106.77385061)
}
}
{
\newrgbcolor{curcolor}{0 0 0}
\pscustom[linestyle=none,fillstyle=solid,fillcolor=curcolor]
{
\newpath
\moveto(582.76665771,115.12345998)
\curveto(582.76666833,115.22345513)(582.77666832,115.31845503)(582.79665771,115.40845998)
\curveto(582.80666829,115.49845485)(582.83666826,115.56345479)(582.88665771,115.60345998)
\curveto(582.96666813,115.66345469)(583.07166803,115.69345466)(583.20165771,115.69345998)
\lineto(583.59165771,115.69345998)
\lineto(585.09165771,115.69345998)
\lineto(591.48165771,115.69345998)
\lineto(592.65165771,115.69345998)
\lineto(592.96665771,115.69345998)
\curveto(593.06665803,115.70345465)(593.14665795,115.68845466)(593.20665771,115.64845998)
\curveto(593.28665781,115.59845475)(593.33665776,115.52345483)(593.35665771,115.42345998)
\curveto(593.36665773,115.33345502)(593.37165773,115.22345513)(593.37165771,115.09345998)
\lineto(593.37165771,114.86845998)
\curveto(593.35165775,114.78845556)(593.33665776,114.71845563)(593.32665771,114.65845998)
\curveto(593.30665779,114.59845575)(593.26665783,114.5484558)(593.20665771,114.50845998)
\curveto(593.14665795,114.46845588)(593.07165803,114.4484559)(592.98165771,114.44845998)
\lineto(592.68165771,114.44845998)
\lineto(591.58665771,114.44845998)
\lineto(586.24665771,114.44845998)
\curveto(586.15666494,114.42845592)(586.08166502,114.41345594)(586.02165771,114.40345998)
\curveto(585.95166515,114.40345595)(585.89166521,114.37345598)(585.84165771,114.31345998)
\curveto(585.79166531,114.24345611)(585.76666533,114.1534562)(585.76665771,114.04345998)
\curveto(585.75666534,113.94345641)(585.75166535,113.83345652)(585.75165771,113.71345998)
\lineto(585.75165771,112.57345998)
\lineto(585.75165771,112.07845998)
\curveto(585.74166536,111.91845843)(585.68166542,111.80845854)(585.57165771,111.74845998)
\curveto(585.54166556,111.72845862)(585.51166559,111.71845863)(585.48165771,111.71845998)
\curveto(585.44166566,111.71845863)(585.3966657,111.71345864)(585.34665771,111.70345998)
\curveto(585.22666587,111.68345867)(585.11666598,111.68845866)(585.01665771,111.71845998)
\curveto(584.91666618,111.75845859)(584.84666625,111.81345854)(584.80665771,111.88345998)
\curveto(584.75666634,111.96345839)(584.73166637,112.08345827)(584.73165771,112.24345998)
\curveto(584.73166637,112.40345795)(584.71666638,112.53845781)(584.68665771,112.64845998)
\curveto(584.67666642,112.69845765)(584.67166643,112.7534576)(584.67165771,112.81345998)
\curveto(584.66166644,112.87345748)(584.64666645,112.93345742)(584.62665771,112.99345998)
\curveto(584.57666652,113.14345721)(584.52666657,113.28845706)(584.47665771,113.42845998)
\curveto(584.41666668,113.56845678)(584.34666675,113.70345665)(584.26665771,113.83345998)
\curveto(584.17666692,113.97345638)(584.07166703,114.09345626)(583.95165771,114.19345998)
\curveto(583.83166727,114.29345606)(583.7016674,114.38845596)(583.56165771,114.47845998)
\curveto(583.46166764,114.53845581)(583.35166775,114.58345577)(583.23165771,114.61345998)
\curveto(583.11166799,114.6534557)(583.00666809,114.70345565)(582.91665771,114.76345998)
\curveto(582.85666824,114.81345554)(582.81666828,114.88345547)(582.79665771,114.97345998)
\curveto(582.78666831,114.99345536)(582.78166832,115.01845533)(582.78165771,115.04845998)
\curveto(582.78166832,115.07845527)(582.77666832,115.10345525)(582.76665771,115.12345998)
}
}
{
\newrgbcolor{curcolor}{0 0 0}
\pscustom[linestyle=none,fillstyle=solid,fillcolor=curcolor]
{
\newpath
\moveto(604.63794678,29.18119436)
\lineto(604.63794678,30.09619436)
\curveto(604.63795747,30.19619171)(604.63795747,30.29119161)(604.63794678,30.38119436)
\curveto(604.63795747,30.47119143)(604.65795745,30.54619136)(604.69794678,30.60619436)
\curveto(604.75795735,30.69619121)(604.83795727,30.75619115)(604.93794678,30.78619436)
\curveto(605.03795707,30.82619108)(605.14295697,30.87119103)(605.25294678,30.92119436)
\curveto(605.44295667,31.0011909)(605.63295648,31.07119083)(605.82294678,31.13119436)
\curveto(606.0129561,31.2011907)(606.20295591,31.27619063)(606.39294678,31.35619436)
\curveto(606.57295554,31.42619048)(606.75795535,31.49119041)(606.94794678,31.55119436)
\curveto(607.12795498,31.61119029)(607.3079548,31.68119022)(607.48794678,31.76119436)
\curveto(607.62795448,31.82119008)(607.77295434,31.87619003)(607.92294678,31.92619436)
\curveto(608.07295404,31.97618993)(608.21795389,32.03118987)(608.35794678,32.09119436)
\curveto(608.8079533,32.27118963)(609.26295285,32.44118946)(609.72294678,32.60119436)
\curveto(610.17295194,32.76118914)(610.62295149,32.93118897)(611.07294678,33.11119436)
\curveto(611.12295099,33.13118877)(611.17295094,33.14618876)(611.22294678,33.15619436)
\lineto(611.37294678,33.21619436)
\curveto(611.59295052,33.3061886)(611.81795029,33.39118851)(612.04794678,33.47119436)
\curveto(612.26794984,33.55118835)(612.48794962,33.63618827)(612.70794678,33.72619436)
\curveto(612.79794931,33.76618814)(612.9079492,33.8061881)(613.03794678,33.84619436)
\curveto(613.15794895,33.88618802)(613.22794888,33.95118795)(613.24794678,34.04119436)
\curveto(613.25794885,34.08118782)(613.25794885,34.11118779)(613.24794678,34.13119436)
\lineto(613.18794678,34.19119436)
\curveto(613.13794897,34.24118766)(613.08294903,34.27618763)(613.02294678,34.29619436)
\curveto(612.96294915,34.32618758)(612.89794921,34.35618755)(612.82794678,34.38619436)
\lineto(612.19794678,34.62619436)
\curveto(611.97795013,34.7061872)(611.76295035,34.78618712)(611.55294678,34.86619436)
\lineto(611.40294678,34.92619436)
\lineto(611.22294678,34.98619436)
\curveto(611.03295108,35.06618684)(610.84295127,35.13618677)(610.65294678,35.19619436)
\curveto(610.45295166,35.26618664)(610.25295186,35.34118656)(610.05294678,35.42119436)
\curveto(609.47295264,35.66118624)(608.88795322,35.88118602)(608.29794678,36.08119436)
\curveto(607.7079544,36.29118561)(607.12295499,36.51618539)(606.54294678,36.75619436)
\curveto(606.34295577,36.83618507)(606.13795597,36.91118499)(605.92794678,36.98119436)
\curveto(605.71795639,37.06118484)(605.5129566,37.14118476)(605.31294678,37.22119436)
\curveto(605.23295688,37.26118464)(605.13295698,37.29618461)(605.01294678,37.32619436)
\curveto(604.89295722,37.36618454)(604.8079573,37.42118448)(604.75794678,37.49119436)
\curveto(604.71795739,37.55118435)(604.68795742,37.62618428)(604.66794678,37.71619436)
\curveto(604.64795746,37.81618409)(604.63795747,37.92618398)(604.63794678,38.04619436)
\curveto(604.62795748,38.16618374)(604.62795748,38.28618362)(604.63794678,38.40619436)
\curveto(604.63795747,38.52618338)(604.63795747,38.63618327)(604.63794678,38.73619436)
\curveto(604.63795747,38.82618308)(604.63795747,38.91618299)(604.63794678,39.00619436)
\curveto(604.63795747,39.1061828)(604.65795745,39.18118272)(604.69794678,39.23119436)
\curveto(604.74795736,39.32118258)(604.83795727,39.37118253)(604.96794678,39.38119436)
\curveto(605.09795701,39.39118251)(605.23795687,39.39618251)(605.38794678,39.39619436)
\lineto(607.03794678,39.39619436)
\lineto(613.30794678,39.39619436)
\lineto(614.56794678,39.39619436)
\curveto(614.67794743,39.39618251)(614.78794732,39.39618251)(614.89794678,39.39619436)
\curveto(615.0079471,39.4061825)(615.09294702,39.38618252)(615.15294678,39.33619436)
\curveto(615.2129469,39.3061826)(615.25294686,39.26118264)(615.27294678,39.20119436)
\curveto(615.28294683,39.14118276)(615.29794681,39.07118283)(615.31794678,38.99119436)
\lineto(615.31794678,38.75119436)
\lineto(615.31794678,38.39119436)
\curveto(615.3079468,38.28118362)(615.26294685,38.2011837)(615.18294678,38.15119436)
\curveto(615.15294696,38.13118377)(615.12294699,38.11618379)(615.09294678,38.10619436)
\curveto(615.05294706,38.1061838)(615.0079471,38.09618381)(614.95794678,38.07619436)
\lineto(614.79294678,38.07619436)
\curveto(614.73294738,38.06618384)(614.66294745,38.06118384)(614.58294678,38.06119436)
\curveto(614.50294761,38.07118383)(614.42794768,38.07618383)(614.35794678,38.07619436)
\lineto(613.51794678,38.07619436)
\lineto(609.09294678,38.07619436)
\curveto(608.84295327,38.07618383)(608.59295352,38.07618383)(608.34294678,38.07619436)
\curveto(608.08295403,38.07618383)(607.83295428,38.07118383)(607.59294678,38.06119436)
\curveto(607.49295462,38.06118384)(607.38295473,38.05618385)(607.26294678,38.04619436)
\curveto(607.14295497,38.03618387)(607.08295503,37.98118392)(607.08294678,37.88119436)
\lineto(607.09794678,37.88119436)
\curveto(607.11795499,37.81118409)(607.18295493,37.75118415)(607.29294678,37.70119436)
\curveto(607.40295471,37.66118424)(607.49795461,37.62618428)(607.57794678,37.59619436)
\curveto(607.74795436,37.52618438)(607.92295419,37.46118444)(608.10294678,37.40119436)
\curveto(608.27295384,37.34118456)(608.44295367,37.27118463)(608.61294678,37.19119436)
\curveto(608.66295345,37.17118473)(608.7079534,37.15618475)(608.74794678,37.14619436)
\curveto(608.78795332,37.13618477)(608.83295328,37.12118478)(608.88294678,37.10119436)
\curveto(609.06295305,37.02118488)(609.24795286,36.95118495)(609.43794678,36.89119436)
\curveto(609.61795249,36.84118506)(609.79795231,36.77618513)(609.97794678,36.69619436)
\curveto(610.12795198,36.62618528)(610.28295183,36.56618534)(610.44294678,36.51619436)
\curveto(610.59295152,36.46618544)(610.74295137,36.41118549)(610.89294678,36.35119436)
\curveto(611.36295075,36.15118575)(611.83795027,35.97118593)(612.31794678,35.81119436)
\curveto(612.78794932,35.65118625)(613.25294886,35.47618643)(613.71294678,35.28619436)
\curveto(613.89294822,35.2061867)(614.07294804,35.13618677)(614.25294678,35.07619436)
\curveto(614.43294768,35.01618689)(614.6129475,34.95118695)(614.79294678,34.88119436)
\curveto(614.90294721,34.83118707)(615.0079471,34.78118712)(615.10794678,34.73119436)
\curveto(615.19794691,34.69118721)(615.26294685,34.6061873)(615.30294678,34.47619436)
\curveto(615.3129468,34.45618745)(615.31794679,34.43118747)(615.31794678,34.40119436)
\curveto(615.3079468,34.38118752)(615.3079468,34.35618755)(615.31794678,34.32619436)
\curveto(615.32794678,34.29618761)(615.33294678,34.26118764)(615.33294678,34.22119436)
\curveto(615.32294679,34.18118772)(615.31794679,34.14118776)(615.31794678,34.10119436)
\lineto(615.31794678,33.80119436)
\curveto(615.31794679,33.7011882)(615.29294682,33.62118828)(615.24294678,33.56119436)
\curveto(615.19294692,33.48118842)(615.12294699,33.42118848)(615.03294678,33.38119436)
\curveto(614.93294718,33.35118855)(614.83294728,33.31118859)(614.73294678,33.26119436)
\curveto(614.53294758,33.18118872)(614.32794778,33.1011888)(614.11794678,33.02119436)
\curveto(613.89794821,32.95118895)(613.68794842,32.87618903)(613.48794678,32.79619436)
\curveto(613.3079488,32.71618919)(613.12794898,32.64618926)(612.94794678,32.58619436)
\curveto(612.75794935,32.53618937)(612.57294954,32.47118943)(612.39294678,32.39119436)
\curveto(611.83295028,32.16118974)(611.26795084,31.94618996)(610.69794678,31.74619436)
\curveto(610.12795198,31.54619036)(609.56295255,31.33119057)(609.00294678,31.10119436)
\lineto(608.37294678,30.86119436)
\curveto(608.15295396,30.79119111)(607.94295417,30.71619119)(607.74294678,30.63619436)
\curveto(607.63295448,30.58619132)(607.52795458,30.54119136)(607.42794678,30.50119436)
\curveto(607.31795479,30.47119143)(607.22295489,30.42119148)(607.14294678,30.35119436)
\curveto(607.12295499,30.34119156)(607.112955,30.33119157)(607.11294678,30.32119436)
\lineto(607.08294678,30.29119436)
\lineto(607.08294678,30.21619436)
\lineto(607.11294678,30.18619436)
\curveto(607.112955,30.17619173)(607.11795499,30.16619174)(607.12794678,30.15619436)
\curveto(607.17795493,30.13619177)(607.23295488,30.12619178)(607.29294678,30.12619436)
\curveto(607.35295476,30.12619178)(607.4129547,30.11619179)(607.47294678,30.09619436)
\lineto(607.63794678,30.09619436)
\curveto(607.69795441,30.07619183)(607.76295435,30.07119183)(607.83294678,30.08119436)
\curveto(607.90295421,30.09119181)(607.97295414,30.09619181)(608.04294678,30.09619436)
\lineto(608.85294678,30.09619436)
\lineto(613.41294678,30.09619436)
\lineto(614.59794678,30.09619436)
\curveto(614.7079474,30.09619181)(614.81794729,30.09119181)(614.92794678,30.08119436)
\curveto(615.03794707,30.08119182)(615.12294699,30.05619185)(615.18294678,30.00619436)
\curveto(615.26294685,29.95619195)(615.3079468,29.86619204)(615.31794678,29.73619436)
\lineto(615.31794678,29.34619436)
\lineto(615.31794678,29.15119436)
\curveto(615.31794679,29.1011928)(615.3079468,29.05119285)(615.28794678,29.00119436)
\curveto(615.24794686,28.87119303)(615.16294695,28.79619311)(615.03294678,28.77619436)
\curveto(614.90294721,28.76619314)(614.75294736,28.76119314)(614.58294678,28.76119436)
\lineto(612.84294678,28.76119436)
\lineto(606.84294678,28.76119436)
\lineto(605.43294678,28.76119436)
\curveto(605.32295679,28.76119314)(605.2079569,28.75619315)(605.08794678,28.74619436)
\curveto(604.96795714,28.74619316)(604.87295724,28.77119313)(604.80294678,28.82119436)
\curveto(604.74295737,28.86119304)(604.69295742,28.93619297)(604.65294678,29.04619436)
\curveto(604.64295747,29.06619284)(604.64295747,29.08619282)(604.65294678,29.10619436)
\curveto(604.65295746,29.13619277)(604.64795746,29.16119274)(604.63794678,29.18119436)
}
}
{
\newrgbcolor{curcolor}{0 0 0}
\pscustom[linestyle=none,fillstyle=solid,fillcolor=curcolor]
{
\newpath
\moveto(614.76294678,48.38330373)
\curveto(614.92294719,48.4132959)(615.05794705,48.39829592)(615.16794678,48.33830373)
\curveto(615.26794684,48.27829604)(615.34294677,48.19829612)(615.39294678,48.09830373)
\curveto(615.4129467,48.04829627)(615.42294669,47.99329632)(615.42294678,47.93330373)
\curveto(615.42294669,47.88329643)(615.43294668,47.82829649)(615.45294678,47.76830373)
\curveto(615.50294661,47.54829677)(615.48794662,47.32829699)(615.40794678,47.10830373)
\curveto(615.33794677,46.89829742)(615.24794686,46.75329756)(615.13794678,46.67330373)
\curveto(615.06794704,46.62329769)(614.98794712,46.57829774)(614.89794678,46.53830373)
\curveto(614.79794731,46.49829782)(614.71794739,46.44829787)(614.65794678,46.38830373)
\curveto(614.63794747,46.36829795)(614.61794749,46.34329797)(614.59794678,46.31330373)
\curveto(614.57794753,46.29329802)(614.57294754,46.26329805)(614.58294678,46.22330373)
\curveto(614.6129475,46.1132982)(614.66794744,46.00829831)(614.74794678,45.90830373)
\curveto(614.82794728,45.8182985)(614.89794721,45.72829859)(614.95794678,45.63830373)
\curveto(615.03794707,45.50829881)(615.112947,45.36829895)(615.18294678,45.21830373)
\curveto(615.24294687,45.06829925)(615.29794681,44.90829941)(615.34794678,44.73830373)
\curveto(615.37794673,44.63829968)(615.39794671,44.52829979)(615.40794678,44.40830373)
\curveto(615.41794669,44.29830002)(615.43294668,44.18830013)(615.45294678,44.07830373)
\curveto(615.46294665,44.02830029)(615.46794664,43.98330033)(615.46794678,43.94330373)
\lineto(615.46794678,43.83830373)
\curveto(615.48794662,43.72830059)(615.48794662,43.62330069)(615.46794678,43.52330373)
\lineto(615.46794678,43.38830373)
\curveto(615.45794665,43.33830098)(615.45294666,43.28830103)(615.45294678,43.23830373)
\curveto(615.45294666,43.18830113)(615.44294667,43.14330117)(615.42294678,43.10330373)
\curveto(615.4129467,43.06330125)(615.4079467,43.02830129)(615.40794678,42.99830373)
\curveto(615.41794669,42.97830134)(615.41794669,42.95330136)(615.40794678,42.92330373)
\lineto(615.34794678,42.68330373)
\curveto(615.33794677,42.60330171)(615.31794679,42.52830179)(615.28794678,42.45830373)
\curveto(615.15794695,42.15830216)(615.0129471,41.9133024)(614.85294678,41.72330373)
\curveto(614.68294743,41.54330277)(614.44794766,41.39330292)(614.14794678,41.27330373)
\curveto(613.92794818,41.18330313)(613.66294845,41.13830318)(613.35294678,41.13830373)
\lineto(613.03794678,41.13830373)
\curveto(612.98794912,41.14830317)(612.93794917,41.15330316)(612.88794678,41.15330373)
\lineto(612.70794678,41.18330373)
\lineto(612.37794678,41.30330373)
\curveto(612.26794984,41.34330297)(612.16794994,41.39330292)(612.07794678,41.45330373)
\curveto(611.78795032,41.63330268)(611.57295054,41.87830244)(611.43294678,42.18830373)
\curveto(611.29295082,42.49830182)(611.16795094,42.83830148)(611.05794678,43.20830373)
\curveto(611.01795109,43.34830097)(610.98795112,43.49330082)(610.96794678,43.64330373)
\curveto(610.94795116,43.79330052)(610.92295119,43.94330037)(610.89294678,44.09330373)
\curveto(610.87295124,44.16330015)(610.86295125,44.22830009)(610.86294678,44.28830373)
\curveto(610.86295125,44.35829996)(610.85295126,44.43329988)(610.83294678,44.51330373)
\curveto(610.8129513,44.58329973)(610.80295131,44.65329966)(610.80294678,44.72330373)
\curveto(610.79295132,44.79329952)(610.77795133,44.86829945)(610.75794678,44.94830373)
\curveto(610.69795141,45.19829912)(610.64795146,45.43329888)(610.60794678,45.65330373)
\curveto(610.55795155,45.87329844)(610.44295167,46.04829827)(610.26294678,46.17830373)
\curveto(610.18295193,46.23829808)(610.08295203,46.28829803)(609.96294678,46.32830373)
\curveto(609.83295228,46.36829795)(609.69295242,46.36829795)(609.54294678,46.32830373)
\curveto(609.30295281,46.26829805)(609.112953,46.17829814)(608.97294678,46.05830373)
\curveto(608.83295328,45.94829837)(608.72295339,45.78829853)(608.64294678,45.57830373)
\curveto(608.59295352,45.45829886)(608.55795355,45.313299)(608.53794678,45.14330373)
\curveto(608.51795359,44.98329933)(608.5079536,44.8132995)(608.50794678,44.63330373)
\curveto(608.5079536,44.45329986)(608.51795359,44.27830004)(608.53794678,44.10830373)
\curveto(608.55795355,43.93830038)(608.58795352,43.79330052)(608.62794678,43.67330373)
\curveto(608.68795342,43.50330081)(608.77295334,43.33830098)(608.88294678,43.17830373)
\curveto(608.94295317,43.09830122)(609.02295309,43.02330129)(609.12294678,42.95330373)
\curveto(609.2129529,42.89330142)(609.3129528,42.83830148)(609.42294678,42.78830373)
\curveto(609.50295261,42.75830156)(609.58795252,42.72830159)(609.67794678,42.69830373)
\curveto(609.76795234,42.67830164)(609.83795227,42.63330168)(609.88794678,42.56330373)
\curveto(609.91795219,42.52330179)(609.94295217,42.45330186)(609.96294678,42.35330373)
\curveto(609.97295214,42.26330205)(609.97795213,42.16830215)(609.97794678,42.06830373)
\curveto(609.97795213,41.96830235)(609.97295214,41.86830245)(609.96294678,41.76830373)
\curveto(609.94295217,41.67830264)(609.91795219,41.6133027)(609.88794678,41.57330373)
\curveto(609.85795225,41.53330278)(609.8079523,41.50330281)(609.73794678,41.48330373)
\curveto(609.66795244,41.46330285)(609.59295252,41.46330285)(609.51294678,41.48330373)
\curveto(609.38295273,41.5133028)(609.26295285,41.54330277)(609.15294678,41.57330373)
\curveto(609.03295308,41.6133027)(608.91795319,41.65830266)(608.80794678,41.70830373)
\curveto(608.45795365,41.89830242)(608.18795392,42.13830218)(607.99794678,42.42830373)
\curveto(607.79795431,42.7183016)(607.63795447,43.07830124)(607.51794678,43.50830373)
\curveto(607.49795461,43.60830071)(607.48295463,43.70830061)(607.47294678,43.80830373)
\curveto(607.46295465,43.9183004)(607.44795466,44.02830029)(607.42794678,44.13830373)
\curveto(607.41795469,44.17830014)(607.41795469,44.24330007)(607.42794678,44.33330373)
\curveto(607.42795468,44.42329989)(607.41795469,44.47829984)(607.39794678,44.49830373)
\curveto(607.38795472,45.19829912)(607.46795464,45.80829851)(607.63794678,46.32830373)
\curveto(607.8079543,46.84829747)(608.13295398,47.2132971)(608.61294678,47.42330373)
\curveto(608.8129533,47.5132968)(609.04795306,47.56329675)(609.31794678,47.57330373)
\curveto(609.57795253,47.59329672)(609.85295226,47.60329671)(610.14294678,47.60330373)
\lineto(613.45794678,47.60330373)
\curveto(613.59794851,47.60329671)(613.73294838,47.60829671)(613.86294678,47.61830373)
\curveto(613.99294812,47.62829669)(614.09794801,47.65829666)(614.17794678,47.70830373)
\curveto(614.24794786,47.75829656)(614.29794781,47.82329649)(614.32794678,47.90330373)
\curveto(614.36794774,47.99329632)(614.39794771,48.07829624)(614.41794678,48.15830373)
\curveto(614.42794768,48.23829608)(614.47294764,48.29829602)(614.55294678,48.33830373)
\curveto(614.58294753,48.35829596)(614.6129475,48.36829595)(614.64294678,48.36830373)
\curveto(614.67294744,48.36829595)(614.7129474,48.37329594)(614.76294678,48.38330373)
\moveto(613.09794678,46.23830373)
\curveto(612.95794915,46.29829802)(612.79794931,46.32829799)(612.61794678,46.32830373)
\curveto(612.42794968,46.33829798)(612.23294988,46.34329797)(612.03294678,46.34330373)
\curveto(611.92295019,46.34329797)(611.82295029,46.33829798)(611.73294678,46.32830373)
\curveto(611.64295047,46.318298)(611.57295054,46.27829804)(611.52294678,46.20830373)
\curveto(611.50295061,46.17829814)(611.49295062,46.10829821)(611.49294678,45.99830373)
\curveto(611.5129506,45.97829834)(611.52295059,45.94329837)(611.52294678,45.89330373)
\curveto(611.52295059,45.84329847)(611.53295058,45.79829852)(611.55294678,45.75830373)
\curveto(611.57295054,45.67829864)(611.59295052,45.58829873)(611.61294678,45.48830373)
\lineto(611.67294678,45.18830373)
\curveto(611.67295044,45.15829916)(611.67795043,45.12329919)(611.68794678,45.08330373)
\lineto(611.68794678,44.97830373)
\curveto(611.72795038,44.82829949)(611.75295036,44.66329965)(611.76294678,44.48330373)
\curveto(611.76295035,44.3133)(611.78295033,44.15330016)(611.82294678,44.00330373)
\curveto(611.84295027,43.92330039)(611.86295025,43.84830047)(611.88294678,43.77830373)
\curveto(611.89295022,43.7183006)(611.9079502,43.64830067)(611.92794678,43.56830373)
\curveto(611.97795013,43.40830091)(612.04295007,43.25830106)(612.12294678,43.11830373)
\curveto(612.19294992,42.97830134)(612.28294983,42.85830146)(612.39294678,42.75830373)
\curveto(612.50294961,42.65830166)(612.63794947,42.58330173)(612.79794678,42.53330373)
\curveto(612.94794916,42.48330183)(613.13294898,42.46330185)(613.35294678,42.47330373)
\curveto(613.45294866,42.47330184)(613.54794856,42.48830183)(613.63794678,42.51830373)
\curveto(613.71794839,42.55830176)(613.79294832,42.60330171)(613.86294678,42.65330373)
\curveto(613.97294814,42.73330158)(614.06794804,42.83830148)(614.14794678,42.96830373)
\curveto(614.21794789,43.09830122)(614.27794783,43.23830108)(614.32794678,43.38830373)
\curveto(614.33794777,43.43830088)(614.34294777,43.48830083)(614.34294678,43.53830373)
\curveto(614.34294777,43.58830073)(614.34794776,43.63830068)(614.35794678,43.68830373)
\curveto(614.37794773,43.75830056)(614.39294772,43.84330047)(614.40294678,43.94330373)
\curveto(614.40294771,44.05330026)(614.39294772,44.14330017)(614.37294678,44.21330373)
\curveto(614.35294776,44.27330004)(614.34794776,44.33329998)(614.35794678,44.39330373)
\curveto(614.35794775,44.45329986)(614.34794776,44.5132998)(614.32794678,44.57330373)
\curveto(614.3079478,44.65329966)(614.29294782,44.72829959)(614.28294678,44.79830373)
\curveto(614.27294784,44.87829944)(614.25294786,44.95329936)(614.22294678,45.02330373)
\curveto(614.10294801,45.313299)(613.95794815,45.55829876)(613.78794678,45.75830373)
\curveto(613.61794849,45.96829835)(613.38794872,46.12829819)(613.09794678,46.23830373)
}
}
{
\newrgbcolor{curcolor}{0 0 0}
\pscustom[linestyle=none,fillstyle=solid,fillcolor=curcolor]
{
\newpath
\moveto(607.59294678,49.26994436)
\lineto(607.59294678,49.71994436)
\curveto(607.58295453,49.88994311)(607.60295451,50.01494298)(607.65294678,50.09494436)
\curveto(607.70295441,50.17494282)(607.76795434,50.22994277)(607.84794678,50.25994436)
\curveto(607.92795418,50.2999427)(608.0129541,50.33994266)(608.10294678,50.37994436)
\curveto(608.23295388,50.42994257)(608.36295375,50.47494252)(608.49294678,50.51494436)
\curveto(608.62295349,50.55494244)(608.75295336,50.5999424)(608.88294678,50.64994436)
\curveto(609.00295311,50.6999423)(609.12795298,50.74494225)(609.25794678,50.78494436)
\curveto(609.37795273,50.82494217)(609.49795261,50.86994213)(609.61794678,50.91994436)
\curveto(609.72795238,50.96994203)(609.84295227,51.00994199)(609.96294678,51.03994436)
\curveto(610.08295203,51.06994193)(610.20295191,51.10994189)(610.32294678,51.15994436)
\curveto(610.6129515,51.27994172)(610.9129512,51.38994161)(611.22294678,51.48994436)
\curveto(611.53295058,51.58994141)(611.83295028,51.6999413)(612.12294678,51.81994436)
\curveto(612.16294995,51.83994116)(612.20294991,51.84994115)(612.24294678,51.84994436)
\curveto(612.27294984,51.84994115)(612.30294981,51.85994114)(612.33294678,51.87994436)
\curveto(612.47294964,51.93994106)(612.61794949,51.994941)(612.76794678,52.04494436)
\lineto(613.18794678,52.19494436)
\curveto(613.25794885,52.22494077)(613.33294878,52.25494074)(613.41294678,52.28494436)
\curveto(613.48294863,52.31494068)(613.52794858,52.36494063)(613.54794678,52.43494436)
\curveto(613.57794853,52.51494048)(613.55294856,52.57494042)(613.47294678,52.61494436)
\curveto(613.38294873,52.66494033)(613.3129488,52.6999403)(613.26294678,52.71994436)
\curveto(613.09294902,52.7999402)(612.9129492,52.86494013)(612.72294678,52.91494436)
\curveto(612.53294958,52.96494003)(612.34794976,53.02493997)(612.16794678,53.09494436)
\curveto(611.93795017,53.18493981)(611.7079504,53.26493973)(611.47794678,53.33494436)
\curveto(611.23795087,53.40493959)(611.0079511,53.48993951)(610.78794678,53.58994436)
\curveto(610.73795137,53.5999394)(610.67295144,53.61493938)(610.59294678,53.63494436)
\curveto(610.50295161,53.67493932)(610.4129517,53.70993929)(610.32294678,53.73994436)
\curveto(610.22295189,53.76993923)(610.13295198,53.7999392)(610.05294678,53.82994436)
\curveto(610.00295211,53.84993915)(609.95795215,53.86493913)(609.91794678,53.87494436)
\curveto(609.87795223,53.88493911)(609.83295228,53.8999391)(609.78294678,53.91994436)
\curveto(609.66295245,53.96993903)(609.54295257,54.00993899)(609.42294678,54.03994436)
\curveto(609.29295282,54.07993892)(609.16795294,54.12493887)(609.04794678,54.17494436)
\curveto(608.99795311,54.1949388)(608.95295316,54.20993879)(608.91294678,54.21994436)
\curveto(608.87295324,54.22993877)(608.82795328,54.24493875)(608.77794678,54.26494436)
\curveto(608.68795342,54.30493869)(608.59795351,54.33993866)(608.50794678,54.36994436)
\curveto(608.4079537,54.3999386)(608.3129538,54.42993857)(608.22294678,54.45994436)
\curveto(608.14295397,54.48993851)(608.06295405,54.51493848)(607.98294678,54.53494436)
\curveto(607.89295422,54.56493843)(607.81795429,54.60493839)(607.75794678,54.65494436)
\curveto(607.66795444,54.72493827)(607.61795449,54.81993818)(607.60794678,54.93994436)
\curveto(607.59795451,55.06993793)(607.59295452,55.20993779)(607.59294678,55.35994436)
\curveto(607.59295452,55.43993756)(607.59795451,55.51493748)(607.60794678,55.58494436)
\curveto(607.6079545,55.66493733)(607.62295449,55.72993727)(607.65294678,55.77994436)
\curveto(607.7129544,55.86993713)(607.8079543,55.8949371)(607.93794678,55.85494436)
\curveto(608.06795404,55.81493718)(608.16795394,55.77993722)(608.23794678,55.74994436)
\lineto(608.29794678,55.71994436)
\curveto(608.31795379,55.71993728)(608.33795377,55.71493728)(608.35794678,55.70494436)
\curveto(608.63795347,55.5949374)(608.92295319,55.48493751)(609.21294678,55.37494436)
\lineto(610.05294678,55.04494436)
\curveto(610.13295198,55.01493798)(610.2079519,54.98993801)(610.27794678,54.96994436)
\curveto(610.33795177,54.94993805)(610.40295171,54.92493807)(610.47294678,54.89494436)
\curveto(610.67295144,54.81493818)(610.87795123,54.73493826)(611.08794678,54.65494436)
\curveto(611.28795082,54.58493841)(611.48795062,54.50993849)(611.68794678,54.42994436)
\curveto(612.37794973,54.13993886)(613.07294904,53.86993913)(613.77294678,53.61994436)
\curveto(614.47294764,53.36993963)(615.16794694,53.0999399)(615.85794678,52.80994436)
\lineto(616.00794678,52.74994436)
\curveto(616.06794604,52.73994026)(616.12794598,52.72494027)(616.18794678,52.70494436)
\curveto(616.55794555,52.54494045)(616.92294519,52.37494062)(617.28294678,52.19494436)
\curveto(617.65294446,52.01494098)(617.93794417,51.76494123)(618.13794678,51.44494436)
\curveto(618.19794391,51.33494166)(618.24294387,51.22494177)(618.27294678,51.11494436)
\curveto(618.3129438,51.00494199)(618.34794376,50.87994212)(618.37794678,50.73994436)
\curveto(618.39794371,50.68994231)(618.40294371,50.63494236)(618.39294678,50.57494436)
\curveto(618.38294373,50.52494247)(618.38294373,50.46994253)(618.39294678,50.40994436)
\curveto(618.4129437,50.32994267)(618.4129437,50.24994275)(618.39294678,50.16994436)
\curveto(618.38294373,50.12994287)(618.37794373,50.07994292)(618.37794678,50.01994436)
\lineto(618.31794678,49.77994436)
\curveto(618.29794381,49.70994329)(618.25794385,49.65494334)(618.19794678,49.61494436)
\curveto(618.13794397,49.56494343)(618.06294405,49.53494346)(617.97294678,49.52494436)
\lineto(617.70294678,49.52494436)
\lineto(617.49294678,49.52494436)
\curveto(617.43294468,49.53494346)(617.38294473,49.55494344)(617.34294678,49.58494436)
\curveto(617.23294488,49.65494334)(617.20294491,49.77494322)(617.25294678,49.94494436)
\curveto(617.27294484,50.05494294)(617.28294483,50.17494282)(617.28294678,50.30494436)
\curveto(617.28294483,50.43494256)(617.26294485,50.54994245)(617.22294678,50.64994436)
\curveto(617.17294494,50.7999422)(617.09794501,50.91994208)(616.99794678,51.00994436)
\curveto(616.89794521,51.10994189)(616.78294533,51.1949418)(616.65294678,51.26494436)
\curveto(616.53294558,51.33494166)(616.40294571,51.3949416)(616.26294678,51.44494436)
\lineto(615.84294678,51.62494436)
\curveto(615.75294636,51.66494133)(615.64294647,51.70494129)(615.51294678,51.74494436)
\curveto(615.38294673,51.7949412)(615.24794686,51.7999412)(615.10794678,51.75994436)
\curveto(614.94794716,51.70994129)(614.79794731,51.65494134)(614.65794678,51.59494436)
\curveto(614.51794759,51.54494145)(614.37794773,51.48994151)(614.23794678,51.42994436)
\curveto(614.02794808,51.33994166)(613.81794829,51.25494174)(613.60794678,51.17494436)
\curveto(613.39794871,51.0949419)(613.19294892,51.01494198)(612.99294678,50.93494436)
\curveto(612.85294926,50.87494212)(612.71794939,50.81994218)(612.58794678,50.76994436)
\curveto(612.45794965,50.71994228)(612.32294979,50.66994233)(612.18294678,50.61994436)
\lineto(610.86294678,50.07994436)
\curveto(610.42295169,49.90994309)(609.98295213,49.73494326)(609.54294678,49.55494436)
\curveto(609.3129528,49.45494354)(609.09295302,49.36494363)(608.88294678,49.28494436)
\curveto(608.66295345,49.20494379)(608.44295367,49.11994388)(608.22294678,49.02994436)
\curveto(608.16295395,49.00994399)(608.08295403,48.97994402)(607.98294678,48.93994436)
\curveto(607.87295424,48.8999441)(607.78295433,48.90494409)(607.71294678,48.95494436)
\curveto(607.66295445,48.98494401)(607.62795448,49.04494395)(607.60794678,49.13494436)
\curveto(607.59795451,49.15494384)(607.59795451,49.17494382)(607.60794678,49.19494436)
\curveto(607.6079545,49.22494377)(607.60295451,49.24994375)(607.59294678,49.26994436)
}
}
{
\newrgbcolor{curcolor}{0 0 0}
\pscustom[linestyle=none,fillstyle=solid,fillcolor=curcolor]
{
}
}
{
\newrgbcolor{curcolor}{0 0 0}
\pscustom[linestyle=none,fillstyle=solid,fillcolor=curcolor]
{
\newpath
\moveto(604.71294678,64.24510061)
\curveto(604.70295741,64.93509597)(604.82295729,65.53509537)(605.07294678,66.04510061)
\curveto(605.32295679,66.56509434)(605.65795645,66.96009395)(606.07794678,67.23010061)
\curveto(606.15795595,67.28009363)(606.24795586,67.32509358)(606.34794678,67.36510061)
\curveto(606.43795567,67.4050935)(606.53295558,67.45009346)(606.63294678,67.50010061)
\curveto(606.73295538,67.54009337)(606.83295528,67.57009334)(606.93294678,67.59010061)
\curveto(607.03295508,67.6100933)(607.13795497,67.63009328)(607.24794678,67.65010061)
\curveto(607.29795481,67.67009324)(607.34295477,67.67509323)(607.38294678,67.66510061)
\curveto(607.42295469,67.65509325)(607.46795464,67.66009325)(607.51794678,67.68010061)
\curveto(607.56795454,67.69009322)(607.65295446,67.69509321)(607.77294678,67.69510061)
\curveto(607.88295423,67.69509321)(607.96795414,67.69009322)(608.02794678,67.68010061)
\curveto(608.08795402,67.66009325)(608.14795396,67.65009326)(608.20794678,67.65010061)
\curveto(608.26795384,67.66009325)(608.32795378,67.65509325)(608.38794678,67.63510061)
\curveto(608.52795358,67.59509331)(608.66295345,67.56009335)(608.79294678,67.53010061)
\curveto(608.92295319,67.50009341)(609.04795306,67.46009345)(609.16794678,67.41010061)
\curveto(609.3079528,67.35009356)(609.43295268,67.28009363)(609.54294678,67.20010061)
\curveto(609.65295246,67.13009378)(609.76295235,67.05509385)(609.87294678,66.97510061)
\lineto(609.93294678,66.91510061)
\curveto(609.95295216,66.905094)(609.97295214,66.89009402)(609.99294678,66.87010061)
\curveto(610.15295196,66.75009416)(610.29795181,66.61509429)(610.42794678,66.46510061)
\curveto(610.55795155,66.31509459)(610.68295143,66.15509475)(610.80294678,65.98510061)
\curveto(611.02295109,65.67509523)(611.22795088,65.38009553)(611.41794678,65.10010061)
\curveto(611.55795055,64.87009604)(611.69295042,64.64009627)(611.82294678,64.41010061)
\curveto(611.95295016,64.19009672)(612.08795002,63.97009694)(612.22794678,63.75010061)
\curveto(612.39794971,63.50009741)(612.57794953,63.26009765)(612.76794678,63.03010061)
\curveto(612.95794915,62.8100981)(613.18294893,62.62009829)(613.44294678,62.46010061)
\curveto(613.50294861,62.42009849)(613.56294855,62.38509852)(613.62294678,62.35510061)
\curveto(613.67294844,62.32509858)(613.73794837,62.29509861)(613.81794678,62.26510061)
\curveto(613.88794822,62.24509866)(613.94794816,62.24009867)(613.99794678,62.25010061)
\curveto(614.06794804,62.27009864)(614.12294799,62.3050986)(614.16294678,62.35510061)
\curveto(614.19294792,62.4050985)(614.2129479,62.46509844)(614.22294678,62.53510061)
\lineto(614.22294678,62.77510061)
\lineto(614.22294678,63.52510061)
\lineto(614.22294678,66.33010061)
\lineto(614.22294678,66.99010061)
\curveto(614.22294789,67.08009383)(614.22794788,67.16509374)(614.23794678,67.24510061)
\curveto(614.23794787,67.32509358)(614.25794785,67.39009352)(614.29794678,67.44010061)
\curveto(614.33794777,67.49009342)(614.4129477,67.53009338)(614.52294678,67.56010061)
\curveto(614.62294749,67.60009331)(614.72294739,67.6100933)(614.82294678,67.59010061)
\lineto(614.95794678,67.59010061)
\curveto(615.02794708,67.57009334)(615.08794702,67.55009336)(615.13794678,67.53010061)
\curveto(615.18794692,67.5100934)(615.22794688,67.47509343)(615.25794678,67.42510061)
\curveto(615.29794681,67.37509353)(615.31794679,67.3050936)(615.31794678,67.21510061)
\lineto(615.31794678,66.94510061)
\lineto(615.31794678,66.04510061)
\lineto(615.31794678,62.53510061)
\lineto(615.31794678,61.47010061)
\curveto(615.31794679,61.39009952)(615.32294679,61.30009961)(615.33294678,61.20010061)
\curveto(615.33294678,61.10009981)(615.32294679,61.01509989)(615.30294678,60.94510061)
\curveto(615.23294688,60.73510017)(615.05294706,60.67010024)(614.76294678,60.75010061)
\curveto(614.72294739,60.76010015)(614.68794742,60.76010015)(614.65794678,60.75010061)
\curveto(614.61794749,60.75010016)(614.57294754,60.76010015)(614.52294678,60.78010061)
\curveto(614.44294767,60.80010011)(614.35794775,60.82010009)(614.26794678,60.84010061)
\curveto(614.17794793,60.86010005)(614.09294802,60.88510002)(614.01294678,60.91510061)
\curveto(613.52294859,61.07509983)(613.107949,61.27509963)(612.76794678,61.51510061)
\curveto(612.51794959,61.69509921)(612.29294982,61.90009901)(612.09294678,62.13010061)
\curveto(611.88295023,62.36009855)(611.68795042,62.60009831)(611.50794678,62.85010061)
\curveto(611.32795078,63.1100978)(611.15795095,63.37509753)(610.99794678,63.64510061)
\curveto(610.82795128,63.92509698)(610.65295146,64.19509671)(610.47294678,64.45510061)
\curveto(610.39295172,64.56509634)(610.31795179,64.67009624)(610.24794678,64.77010061)
\curveto(610.17795193,64.88009603)(610.10295201,64.99009592)(610.02294678,65.10010061)
\curveto(609.99295212,65.14009577)(609.96295215,65.17509573)(609.93294678,65.20510061)
\curveto(609.89295222,65.24509566)(609.86295225,65.28509562)(609.84294678,65.32510061)
\curveto(609.73295238,65.46509544)(609.6079525,65.59009532)(609.46794678,65.70010061)
\curveto(609.43795267,65.72009519)(609.4129527,65.74509516)(609.39294678,65.77510061)
\curveto(609.36295275,65.8050951)(609.33295278,65.83009508)(609.30294678,65.85010061)
\curveto(609.20295291,65.93009498)(609.10295301,65.99509491)(609.00294678,66.04510061)
\curveto(608.90295321,66.1050948)(608.79295332,66.16009475)(608.67294678,66.21010061)
\curveto(608.60295351,66.24009467)(608.52795358,66.26009465)(608.44794678,66.27010061)
\lineto(608.20794678,66.33010061)
\lineto(608.11794678,66.33010061)
\curveto(608.08795402,66.34009457)(608.05795405,66.34509456)(608.02794678,66.34510061)
\curveto(607.95795415,66.36509454)(607.86295425,66.37009454)(607.74294678,66.36010061)
\curveto(607.6129545,66.36009455)(607.5129546,66.35009456)(607.44294678,66.33010061)
\curveto(607.36295475,66.3100946)(607.28795482,66.29009462)(607.21794678,66.27010061)
\curveto(607.13795497,66.26009465)(607.05795505,66.24009467)(606.97794678,66.21010061)
\curveto(606.73795537,66.10009481)(606.53795557,65.95009496)(606.37794678,65.76010061)
\curveto(606.2079559,65.58009533)(606.06795604,65.36009555)(605.95794678,65.10010061)
\curveto(605.93795617,65.03009588)(605.92295619,64.96009595)(605.91294678,64.89010061)
\curveto(605.89295622,64.82009609)(605.87295624,64.74509616)(605.85294678,64.66510061)
\curveto(605.83295628,64.58509632)(605.82295629,64.47509643)(605.82294678,64.33510061)
\curveto(605.82295629,64.2050967)(605.83295628,64.10009681)(605.85294678,64.02010061)
\curveto(605.86295625,63.96009695)(605.86795624,63.905097)(605.86794678,63.85510061)
\curveto(605.86795624,63.8050971)(605.87795623,63.75509715)(605.89794678,63.70510061)
\curveto(605.93795617,63.6050973)(605.97795613,63.5100974)(606.01794678,63.42010061)
\curveto(606.05795605,63.34009757)(606.10295601,63.26009765)(606.15294678,63.18010061)
\curveto(606.17295594,63.15009776)(606.19795591,63.12009779)(606.22794678,63.09010061)
\curveto(606.25795585,63.07009784)(606.28295583,63.04509786)(606.30294678,63.01510061)
\lineto(606.37794678,62.94010061)
\curveto(606.39795571,62.910098)(606.41795569,62.88509802)(606.43794678,62.86510061)
\lineto(606.64794678,62.71510061)
\curveto(606.7079554,62.67509823)(606.77295534,62.63009828)(606.84294678,62.58010061)
\curveto(606.93295518,62.52009839)(607.03795507,62.47009844)(607.15794678,62.43010061)
\curveto(607.26795484,62.40009851)(607.37795473,62.36509854)(607.48794678,62.32510061)
\curveto(607.59795451,62.28509862)(607.74295437,62.26009865)(607.92294678,62.25010061)
\curveto(608.09295402,62.24009867)(608.21795389,62.2100987)(608.29794678,62.16010061)
\curveto(608.37795373,62.1100988)(608.42295369,62.03509887)(608.43294678,61.93510061)
\curveto(608.44295367,61.83509907)(608.44795366,61.72509918)(608.44794678,61.60510061)
\curveto(608.44795366,61.56509934)(608.45295366,61.52509938)(608.46294678,61.48510061)
\curveto(608.46295365,61.44509946)(608.45795365,61.4100995)(608.44794678,61.38010061)
\curveto(608.42795368,61.33009958)(608.41795369,61.28009963)(608.41794678,61.23010061)
\curveto(608.41795369,61.19009972)(608.4079537,61.15009976)(608.38794678,61.11010061)
\curveto(608.32795378,61.02009989)(608.19295392,60.97509993)(607.98294678,60.97510061)
\lineto(607.86294678,60.97510061)
\curveto(607.80295431,60.98509992)(607.74295437,60.99009992)(607.68294678,60.99010061)
\curveto(607.6129545,61.00009991)(607.54795456,61.0100999)(607.48794678,61.02010061)
\curveto(607.37795473,61.04009987)(607.27795483,61.06009985)(607.18794678,61.08010061)
\curveto(607.08795502,61.10009981)(606.99295512,61.13009978)(606.90294678,61.17010061)
\curveto(606.83295528,61.19009972)(606.77295534,61.2100997)(606.72294678,61.23010061)
\lineto(606.54294678,61.29010061)
\curveto(606.28295583,61.4100995)(606.03795607,61.56509934)(605.80794678,61.75510061)
\curveto(605.57795653,61.95509895)(605.39295672,62.17009874)(605.25294678,62.40010061)
\curveto(605.17295694,62.5100984)(605.107957,62.62509828)(605.05794678,62.74510061)
\lineto(604.90794678,63.13510061)
\curveto(604.85795725,63.24509766)(604.82795728,63.36009755)(604.81794678,63.48010061)
\curveto(604.79795731,63.60009731)(604.77295734,63.72509718)(604.74294678,63.85510061)
\curveto(604.74295737,63.92509698)(604.74295737,63.99009692)(604.74294678,64.05010061)
\curveto(604.73295738,64.1100968)(604.72295739,64.17509673)(604.71294678,64.24510061)
}
}
{
\newrgbcolor{curcolor}{0 0 0}
\pscustom[linestyle=none,fillstyle=solid,fillcolor=curcolor]
{
\newpath
\moveto(604.90794678,70.85470998)
\lineto(604.90794678,74.45470998)
\lineto(604.90794678,75.09970998)
\curveto(604.9079572,75.17970345)(604.9129572,75.25470338)(604.92294678,75.32470998)
\curveto(604.92295719,75.39470324)(604.93295718,75.45470318)(604.95294678,75.50470998)
\curveto(604.98295713,75.57470306)(605.04295707,75.629703)(605.13294678,75.66970998)
\curveto(605.16295695,75.68970294)(605.20295691,75.69970293)(605.25294678,75.69970998)
\lineto(605.38794678,75.69970998)
\curveto(605.49795661,75.70970292)(605.60295651,75.70470293)(605.70294678,75.68470998)
\curveto(605.80295631,75.67470296)(605.87295624,75.63970299)(605.91294678,75.57970998)
\curveto(605.98295613,75.48970314)(606.01795609,75.35470328)(606.01794678,75.17470998)
\curveto(606.0079561,74.99470364)(606.00295611,74.8297038)(606.00294678,74.67970998)
\lineto(606.00294678,72.68470998)
\lineto(606.00294678,72.18970998)
\lineto(606.00294678,72.05470998)
\curveto(606.00295611,72.01470662)(606.0079561,71.97470666)(606.01794678,71.93470998)
\lineto(606.01794678,71.72470998)
\curveto(606.04795606,71.61470702)(606.08795602,71.5347071)(606.13794678,71.48470998)
\curveto(606.17795593,71.4347072)(606.23295588,71.39970723)(606.30294678,71.37970998)
\curveto(606.36295575,71.35970727)(606.43295568,71.34470729)(606.51294678,71.33470998)
\curveto(606.59295552,71.32470731)(606.68295543,71.30470733)(606.78294678,71.27470998)
\curveto(606.98295513,71.22470741)(607.18795492,71.18470745)(607.39794678,71.15470998)
\curveto(607.6079545,71.12470751)(607.8129543,71.08470755)(608.01294678,71.03470998)
\curveto(608.08295403,71.01470762)(608.15295396,71.00470763)(608.22294678,71.00470998)
\curveto(608.28295383,71.00470763)(608.34795376,70.99470764)(608.41794678,70.97470998)
\curveto(608.44795366,70.96470767)(608.48795362,70.95470768)(608.53794678,70.94470998)
\curveto(608.57795353,70.94470769)(608.61795349,70.94970768)(608.65794678,70.95970998)
\curveto(608.7079534,70.97970765)(608.75295336,71.00470763)(608.79294678,71.03470998)
\curveto(608.82295329,71.07470756)(608.82795328,71.1347075)(608.80794678,71.21470998)
\curveto(608.78795332,71.27470736)(608.76295335,71.3347073)(608.73294678,71.39470998)
\curveto(608.69295342,71.45470718)(608.65795345,71.51470712)(608.62794678,71.57470998)
\curveto(608.6079535,71.634707)(608.59295352,71.68470695)(608.58294678,71.72470998)
\curveto(608.50295361,71.91470672)(608.44795366,72.11970651)(608.41794678,72.33970998)
\curveto(608.38795372,72.56970606)(608.37795373,72.79970583)(608.38794678,73.02970998)
\curveto(608.38795372,73.26970536)(608.4129537,73.49970513)(608.46294678,73.71970998)
\curveto(608.50295361,73.93970469)(608.56295355,74.13970449)(608.64294678,74.31970998)
\curveto(608.66295345,74.36970426)(608.68295343,74.41470422)(608.70294678,74.45470998)
\curveto(608.72295339,74.50470413)(608.74795336,74.55470408)(608.77794678,74.60470998)
\curveto(608.98795312,74.95470368)(609.21795289,75.2347034)(609.46794678,75.44470998)
\curveto(609.71795239,75.66470297)(610.04295207,75.85970277)(610.44294678,76.02970998)
\curveto(610.55295156,76.07970255)(610.66295145,76.11470252)(610.77294678,76.13470998)
\curveto(610.88295123,76.15470248)(610.99795111,76.17970245)(611.11794678,76.20970998)
\curveto(611.14795096,76.21970241)(611.19295092,76.22470241)(611.25294678,76.22470998)
\curveto(611.3129508,76.24470239)(611.38295073,76.25470238)(611.46294678,76.25470998)
\curveto(611.53295058,76.25470238)(611.59795051,76.26470237)(611.65794678,76.28470998)
\lineto(611.82294678,76.28470998)
\curveto(611.87295024,76.29470234)(611.94295017,76.29970233)(612.03294678,76.29970998)
\curveto(612.12294999,76.29970233)(612.19294992,76.28970234)(612.24294678,76.26970998)
\curveto(612.30294981,76.24970238)(612.36294975,76.24470239)(612.42294678,76.25470998)
\curveto(612.47294964,76.26470237)(612.52294959,76.25970237)(612.57294678,76.23970998)
\curveto(612.73294938,76.19970243)(612.88294923,76.16470247)(613.02294678,76.13470998)
\curveto(613.16294895,76.10470253)(613.29794881,76.05970257)(613.42794678,75.99970998)
\curveto(613.79794831,75.83970279)(614.13294798,75.61970301)(614.43294678,75.33970998)
\curveto(614.73294738,75.05970357)(614.96294715,74.73970389)(615.12294678,74.37970998)
\curveto(615.20294691,74.20970442)(615.27794683,74.00970462)(615.34794678,73.77970998)
\curveto(615.38794672,73.66970496)(615.4129467,73.55470508)(615.42294678,73.43470998)
\curveto(615.43294668,73.31470532)(615.45294666,73.19470544)(615.48294678,73.07470998)
\curveto(615.50294661,73.02470561)(615.50294661,72.96970566)(615.48294678,72.90970998)
\curveto(615.47294664,72.84970578)(615.47794663,72.78970584)(615.49794678,72.72970998)
\curveto(615.51794659,72.629706)(615.51794659,72.5297061)(615.49794678,72.42970998)
\lineto(615.49794678,72.29470998)
\curveto(615.47794663,72.24470639)(615.46794664,72.18470645)(615.46794678,72.11470998)
\curveto(615.47794663,72.05470658)(615.47294664,71.99970663)(615.45294678,71.94970998)
\curveto(615.44294667,71.90970672)(615.43794667,71.87470676)(615.43794678,71.84470998)
\curveto(615.43794667,71.81470682)(615.43294668,71.77970685)(615.42294678,71.73970998)
\lineto(615.36294678,71.46970998)
\curveto(615.34294677,71.37970725)(615.3129468,71.29470734)(615.27294678,71.21470998)
\curveto(615.13294698,70.87470776)(614.97794713,70.58470805)(614.80794678,70.34470998)
\curveto(614.62794748,70.10470853)(614.39794771,69.88470875)(614.11794678,69.68470998)
\curveto(613.88794822,69.5347091)(613.64794846,69.41970921)(613.39794678,69.33970998)
\curveto(613.34794876,69.31970931)(613.30294881,69.30970932)(613.26294678,69.30970998)
\curveto(613.2129489,69.30970932)(613.16294895,69.29970933)(613.11294678,69.27970998)
\curveto(613.05294906,69.25970937)(612.97294914,69.24470939)(612.87294678,69.23470998)
\curveto(612.77294934,69.2347094)(612.69794941,69.25470938)(612.64794678,69.29470998)
\curveto(612.56794954,69.34470929)(612.52294959,69.42470921)(612.51294678,69.53470998)
\curveto(612.50294961,69.64470899)(612.49794961,69.75970887)(612.49794678,69.87970998)
\lineto(612.49794678,70.04470998)
\curveto(612.49794961,70.10470853)(612.5079496,70.15970847)(612.52794678,70.20970998)
\curveto(612.54794956,70.29970833)(612.58794952,70.36970826)(612.64794678,70.41970998)
\curveto(612.73794937,70.48970814)(612.84794926,70.5347081)(612.97794678,70.55470998)
\curveto(613.09794901,70.58470805)(613.20294891,70.629708)(613.29294678,70.68970998)
\curveto(613.63294848,70.87970775)(613.90294821,71.13970749)(614.10294678,71.46970998)
\curveto(614.16294795,71.56970706)(614.2129479,71.67470696)(614.25294678,71.78470998)
\curveto(614.28294783,71.90470673)(614.31794779,72.02470661)(614.35794678,72.14470998)
\curveto(614.4079477,72.31470632)(614.42794768,72.51970611)(614.41794678,72.75970998)
\curveto(614.39794771,73.00970562)(614.36294775,73.20970542)(614.31294678,73.35970998)
\curveto(614.19294792,73.7297049)(614.03294808,74.01970461)(613.83294678,74.22970998)
\curveto(613.62294849,74.44970418)(613.34294877,74.629704)(612.99294678,74.76970998)
\curveto(612.89294922,74.81970381)(612.78794932,74.84970378)(612.67794678,74.85970998)
\curveto(612.56794954,74.87970375)(612.45294966,74.90470373)(612.33294678,74.93470998)
\lineto(612.22794678,74.93470998)
\curveto(612.18794992,74.94470369)(612.14794996,74.94970368)(612.10794678,74.94970998)
\curveto(612.07795003,74.95970367)(612.04295007,74.95970367)(612.00294678,74.94970998)
\lineto(611.88294678,74.94970998)
\curveto(611.62295049,74.94970368)(611.37795073,74.91970371)(611.14794678,74.85970998)
\curveto(610.79795131,74.74970388)(610.50295161,74.59470404)(610.26294678,74.39470998)
\curveto(610.0129521,74.19470444)(609.81795229,73.9347047)(609.67794678,73.61470998)
\lineto(609.61794678,73.43470998)
\curveto(609.59795251,73.38470525)(609.57795253,73.32470531)(609.55794678,73.25470998)
\curveto(609.53795257,73.20470543)(609.52795258,73.14470549)(609.52794678,73.07470998)
\curveto(609.51795259,73.01470562)(609.50295261,72.94970568)(609.48294678,72.87970998)
\lineto(609.48294678,72.72970998)
\curveto(609.46295265,72.68970594)(609.45295266,72.634706)(609.45294678,72.56470998)
\curveto(609.45295266,72.50470613)(609.46295265,72.44970618)(609.48294678,72.39970998)
\lineto(609.48294678,72.29470998)
\curveto(609.48295263,72.26470637)(609.48795262,72.2297064)(609.49794678,72.18970998)
\lineto(609.55794678,71.94970998)
\curveto(609.56795254,71.86970676)(609.58795252,71.78970684)(609.61794678,71.70970998)
\curveto(609.71795239,71.46970716)(609.85295226,71.23970739)(610.02294678,71.01970998)
\curveto(610.09295202,70.9297077)(610.16795194,70.84470779)(610.24794678,70.76470998)
\curveto(610.31795179,70.68470795)(610.37295174,70.58470805)(610.41294678,70.46470998)
\curveto(610.44295167,70.37470826)(610.45295166,70.2347084)(610.44294678,70.04470998)
\curveto(610.43295168,69.86470877)(610.4079517,69.74470889)(610.36794678,69.68470998)
\curveto(610.32795178,69.634709)(610.26795184,69.59470904)(610.18794678,69.56470998)
\curveto(610.107952,69.54470909)(610.02295209,69.54470909)(609.93294678,69.56470998)
\curveto(609.8129523,69.59470904)(609.69295242,69.61470902)(609.57294678,69.62470998)
\curveto(609.44295267,69.64470899)(609.31795279,69.66970896)(609.19794678,69.69970998)
\curveto(609.15795295,69.71970891)(609.12295299,69.72470891)(609.09294678,69.71470998)
\curveto(609.05295306,69.71470892)(609.0079531,69.72470891)(608.95794678,69.74470998)
\curveto(608.86795324,69.76470887)(608.77795333,69.77970885)(608.68794678,69.78970998)
\curveto(608.58795352,69.79970883)(608.49295362,69.81970881)(608.40294678,69.84970998)
\curveto(608.34295377,69.85970877)(608.28295383,69.86470877)(608.22294678,69.86470998)
\curveto(608.16295395,69.87470876)(608.10295401,69.88970874)(608.04294678,69.90970998)
\curveto(607.84295427,69.95970867)(607.63795447,69.99470864)(607.42794678,70.01470998)
\curveto(607.2079549,70.04470859)(606.99795511,70.08470855)(606.79794678,70.13470998)
\curveto(606.69795541,70.16470847)(606.59795551,70.18470845)(606.49794678,70.19470998)
\curveto(606.39795571,70.20470843)(606.29795581,70.21970841)(606.19794678,70.23970998)
\curveto(606.16795594,70.24970838)(606.12795598,70.25470838)(606.07794678,70.25470998)
\curveto(605.96795614,70.28470835)(605.86295625,70.30470833)(605.76294678,70.31470998)
\curveto(605.65295646,70.3347083)(605.54295657,70.35970827)(605.43294678,70.38970998)
\curveto(605.35295676,70.40970822)(605.28295683,70.42470821)(605.22294678,70.43470998)
\curveto(605.15295696,70.44470819)(605.09295702,70.46970816)(605.04294678,70.50970998)
\curveto(605.0129571,70.5297081)(604.99295712,70.55970807)(604.98294678,70.59970998)
\curveto(604.96295715,70.63970799)(604.94295717,70.68470795)(604.92294678,70.73470998)
\curveto(604.92295719,70.79470784)(604.91795719,70.8347078)(604.90794678,70.85470998)
}
}
{
\newrgbcolor{curcolor}{0 0 0}
\pscustom[linestyle=none,fillstyle=solid,fillcolor=curcolor]
{
\newpath
\moveto(613.68294678,78.63431936)
\lineto(613.68294678,79.26431936)
\lineto(613.68294678,79.45931936)
\curveto(613.68294843,79.52931683)(613.69294842,79.58931677)(613.71294678,79.63931936)
\curveto(613.75294836,79.70931665)(613.79294832,79.7593166)(613.83294678,79.78931936)
\curveto(613.88294823,79.82931653)(613.94794816,79.84931651)(614.02794678,79.84931936)
\curveto(614.107948,79.8593165)(614.19294792,79.86431649)(614.28294678,79.86431936)
\lineto(615.00294678,79.86431936)
\curveto(615.48294663,79.86431649)(615.89294622,79.80431655)(616.23294678,79.68431936)
\curveto(616.57294554,79.56431679)(616.84794526,79.36931699)(617.05794678,79.09931936)
\curveto(617.107945,79.02931733)(617.15294496,78.9593174)(617.19294678,78.88931936)
\curveto(617.24294487,78.82931753)(617.28794482,78.7543176)(617.32794678,78.66431936)
\curveto(617.33794477,78.64431771)(617.34794476,78.61431774)(617.35794678,78.57431936)
\curveto(617.37794473,78.53431782)(617.38294473,78.48931787)(617.37294678,78.43931936)
\curveto(617.34294477,78.34931801)(617.26794484,78.29431806)(617.14794678,78.27431936)
\curveto(617.03794507,78.2543181)(616.94294517,78.26931809)(616.86294678,78.31931936)
\curveto(616.79294532,78.34931801)(616.72794538,78.39431796)(616.66794678,78.45431936)
\curveto(616.61794549,78.52431783)(616.56794554,78.58931777)(616.51794678,78.64931936)
\curveto(616.46794564,78.71931764)(616.39294572,78.77931758)(616.29294678,78.82931936)
\curveto(616.20294591,78.88931747)(616.112946,78.93931742)(616.02294678,78.97931936)
\curveto(615.99294612,78.99931736)(615.93294618,79.02431733)(615.84294678,79.05431936)
\curveto(615.76294635,79.08431727)(615.69294642,79.08931727)(615.63294678,79.06931936)
\curveto(615.49294662,79.03931732)(615.40294671,78.97931738)(615.36294678,78.88931936)
\curveto(615.33294678,78.80931755)(615.31794679,78.71931764)(615.31794678,78.61931936)
\curveto(615.31794679,78.51931784)(615.29294682,78.43431792)(615.24294678,78.36431936)
\curveto(615.17294694,78.27431808)(615.03294708,78.22931813)(614.82294678,78.22931936)
\lineto(614.26794678,78.22931936)
\lineto(614.04294678,78.22931936)
\curveto(613.96294815,78.23931812)(613.89794821,78.2593181)(613.84794678,78.28931936)
\curveto(613.76794834,78.34931801)(613.72294839,78.41931794)(613.71294678,78.49931936)
\curveto(613.70294841,78.51931784)(613.69794841,78.53931782)(613.69794678,78.55931936)
\curveto(613.69794841,78.58931777)(613.69294842,78.61431774)(613.68294678,78.63431936)
}
}
{
\newrgbcolor{curcolor}{0 0 0}
\pscustom[linestyle=none,fillstyle=solid,fillcolor=curcolor]
{
}
}
{
\newrgbcolor{curcolor}{0 0 0}
\pscustom[linestyle=none,fillstyle=solid,fillcolor=curcolor]
{
\newpath
\moveto(604.71294678,89.26463186)
\curveto(604.70295741,89.95462722)(604.82295729,90.55462662)(605.07294678,91.06463186)
\curveto(605.32295679,91.58462559)(605.65795645,91.9796252)(606.07794678,92.24963186)
\curveto(606.15795595,92.29962488)(606.24795586,92.34462483)(606.34794678,92.38463186)
\curveto(606.43795567,92.42462475)(606.53295558,92.46962471)(606.63294678,92.51963186)
\curveto(606.73295538,92.55962462)(606.83295528,92.58962459)(606.93294678,92.60963186)
\curveto(607.03295508,92.62962455)(607.13795497,92.64962453)(607.24794678,92.66963186)
\curveto(607.29795481,92.68962449)(607.34295477,92.69462448)(607.38294678,92.68463186)
\curveto(607.42295469,92.6746245)(607.46795464,92.6796245)(607.51794678,92.69963186)
\curveto(607.56795454,92.70962447)(607.65295446,92.71462446)(607.77294678,92.71463186)
\curveto(607.88295423,92.71462446)(607.96795414,92.70962447)(608.02794678,92.69963186)
\curveto(608.08795402,92.6796245)(608.14795396,92.66962451)(608.20794678,92.66963186)
\curveto(608.26795384,92.6796245)(608.32795378,92.6746245)(608.38794678,92.65463186)
\curveto(608.52795358,92.61462456)(608.66295345,92.5796246)(608.79294678,92.54963186)
\curveto(608.92295319,92.51962466)(609.04795306,92.4796247)(609.16794678,92.42963186)
\curveto(609.3079528,92.36962481)(609.43295268,92.29962488)(609.54294678,92.21963186)
\curveto(609.65295246,92.14962503)(609.76295235,92.0746251)(609.87294678,91.99463186)
\lineto(609.93294678,91.93463186)
\curveto(609.95295216,91.92462525)(609.97295214,91.90962527)(609.99294678,91.88963186)
\curveto(610.15295196,91.76962541)(610.29795181,91.63462554)(610.42794678,91.48463186)
\curveto(610.55795155,91.33462584)(610.68295143,91.174626)(610.80294678,91.00463186)
\curveto(611.02295109,90.69462648)(611.22795088,90.39962678)(611.41794678,90.11963186)
\curveto(611.55795055,89.88962729)(611.69295042,89.65962752)(611.82294678,89.42963186)
\curveto(611.95295016,89.20962797)(612.08795002,88.98962819)(612.22794678,88.76963186)
\curveto(612.39794971,88.51962866)(612.57794953,88.2796289)(612.76794678,88.04963186)
\curveto(612.95794915,87.82962935)(613.18294893,87.63962954)(613.44294678,87.47963186)
\curveto(613.50294861,87.43962974)(613.56294855,87.40462977)(613.62294678,87.37463186)
\curveto(613.67294844,87.34462983)(613.73794837,87.31462986)(613.81794678,87.28463186)
\curveto(613.88794822,87.26462991)(613.94794816,87.25962992)(613.99794678,87.26963186)
\curveto(614.06794804,87.28962989)(614.12294799,87.32462985)(614.16294678,87.37463186)
\curveto(614.19294792,87.42462975)(614.2129479,87.48462969)(614.22294678,87.55463186)
\lineto(614.22294678,87.79463186)
\lineto(614.22294678,88.54463186)
\lineto(614.22294678,91.34963186)
\lineto(614.22294678,92.00963186)
\curveto(614.22294789,92.09962508)(614.22794788,92.18462499)(614.23794678,92.26463186)
\curveto(614.23794787,92.34462483)(614.25794785,92.40962477)(614.29794678,92.45963186)
\curveto(614.33794777,92.50962467)(614.4129477,92.54962463)(614.52294678,92.57963186)
\curveto(614.62294749,92.61962456)(614.72294739,92.62962455)(614.82294678,92.60963186)
\lineto(614.95794678,92.60963186)
\curveto(615.02794708,92.58962459)(615.08794702,92.56962461)(615.13794678,92.54963186)
\curveto(615.18794692,92.52962465)(615.22794688,92.49462468)(615.25794678,92.44463186)
\curveto(615.29794681,92.39462478)(615.31794679,92.32462485)(615.31794678,92.23463186)
\lineto(615.31794678,91.96463186)
\lineto(615.31794678,91.06463186)
\lineto(615.31794678,87.55463186)
\lineto(615.31794678,86.48963186)
\curveto(615.31794679,86.40963077)(615.32294679,86.31963086)(615.33294678,86.21963186)
\curveto(615.33294678,86.11963106)(615.32294679,86.03463114)(615.30294678,85.96463186)
\curveto(615.23294688,85.75463142)(615.05294706,85.68963149)(614.76294678,85.76963186)
\curveto(614.72294739,85.7796314)(614.68794742,85.7796314)(614.65794678,85.76963186)
\curveto(614.61794749,85.76963141)(614.57294754,85.7796314)(614.52294678,85.79963186)
\curveto(614.44294767,85.81963136)(614.35794775,85.83963134)(614.26794678,85.85963186)
\curveto(614.17794793,85.8796313)(614.09294802,85.90463127)(614.01294678,85.93463186)
\curveto(613.52294859,86.09463108)(613.107949,86.29463088)(612.76794678,86.53463186)
\curveto(612.51794959,86.71463046)(612.29294982,86.91963026)(612.09294678,87.14963186)
\curveto(611.88295023,87.3796298)(611.68795042,87.61962956)(611.50794678,87.86963186)
\curveto(611.32795078,88.12962905)(611.15795095,88.39462878)(610.99794678,88.66463186)
\curveto(610.82795128,88.94462823)(610.65295146,89.21462796)(610.47294678,89.47463186)
\curveto(610.39295172,89.58462759)(610.31795179,89.68962749)(610.24794678,89.78963186)
\curveto(610.17795193,89.89962728)(610.10295201,90.00962717)(610.02294678,90.11963186)
\curveto(609.99295212,90.15962702)(609.96295215,90.19462698)(609.93294678,90.22463186)
\curveto(609.89295222,90.26462691)(609.86295225,90.30462687)(609.84294678,90.34463186)
\curveto(609.73295238,90.48462669)(609.6079525,90.60962657)(609.46794678,90.71963186)
\curveto(609.43795267,90.73962644)(609.4129527,90.76462641)(609.39294678,90.79463186)
\curveto(609.36295275,90.82462635)(609.33295278,90.84962633)(609.30294678,90.86963186)
\curveto(609.20295291,90.94962623)(609.10295301,91.01462616)(609.00294678,91.06463186)
\curveto(608.90295321,91.12462605)(608.79295332,91.179626)(608.67294678,91.22963186)
\curveto(608.60295351,91.25962592)(608.52795358,91.2796259)(608.44794678,91.28963186)
\lineto(608.20794678,91.34963186)
\lineto(608.11794678,91.34963186)
\curveto(608.08795402,91.35962582)(608.05795405,91.36462581)(608.02794678,91.36463186)
\curveto(607.95795415,91.38462579)(607.86295425,91.38962579)(607.74294678,91.37963186)
\curveto(607.6129545,91.3796258)(607.5129546,91.36962581)(607.44294678,91.34963186)
\curveto(607.36295475,91.32962585)(607.28795482,91.30962587)(607.21794678,91.28963186)
\curveto(607.13795497,91.2796259)(607.05795505,91.25962592)(606.97794678,91.22963186)
\curveto(606.73795537,91.11962606)(606.53795557,90.96962621)(606.37794678,90.77963186)
\curveto(606.2079559,90.59962658)(606.06795604,90.3796268)(605.95794678,90.11963186)
\curveto(605.93795617,90.04962713)(605.92295619,89.9796272)(605.91294678,89.90963186)
\curveto(605.89295622,89.83962734)(605.87295624,89.76462741)(605.85294678,89.68463186)
\curveto(605.83295628,89.60462757)(605.82295629,89.49462768)(605.82294678,89.35463186)
\curveto(605.82295629,89.22462795)(605.83295628,89.11962806)(605.85294678,89.03963186)
\curveto(605.86295625,88.9796282)(605.86795624,88.92462825)(605.86794678,88.87463186)
\curveto(605.86795624,88.82462835)(605.87795623,88.7746284)(605.89794678,88.72463186)
\curveto(605.93795617,88.62462855)(605.97795613,88.52962865)(606.01794678,88.43963186)
\curveto(606.05795605,88.35962882)(606.10295601,88.2796289)(606.15294678,88.19963186)
\curveto(606.17295594,88.16962901)(606.19795591,88.13962904)(606.22794678,88.10963186)
\curveto(606.25795585,88.08962909)(606.28295583,88.06462911)(606.30294678,88.03463186)
\lineto(606.37794678,87.95963186)
\curveto(606.39795571,87.92962925)(606.41795569,87.90462927)(606.43794678,87.88463186)
\lineto(606.64794678,87.73463186)
\curveto(606.7079554,87.69462948)(606.77295534,87.64962953)(606.84294678,87.59963186)
\curveto(606.93295518,87.53962964)(607.03795507,87.48962969)(607.15794678,87.44963186)
\curveto(607.26795484,87.41962976)(607.37795473,87.38462979)(607.48794678,87.34463186)
\curveto(607.59795451,87.30462987)(607.74295437,87.2796299)(607.92294678,87.26963186)
\curveto(608.09295402,87.25962992)(608.21795389,87.22962995)(608.29794678,87.17963186)
\curveto(608.37795373,87.12963005)(608.42295369,87.05463012)(608.43294678,86.95463186)
\curveto(608.44295367,86.85463032)(608.44795366,86.74463043)(608.44794678,86.62463186)
\curveto(608.44795366,86.58463059)(608.45295366,86.54463063)(608.46294678,86.50463186)
\curveto(608.46295365,86.46463071)(608.45795365,86.42963075)(608.44794678,86.39963186)
\curveto(608.42795368,86.34963083)(608.41795369,86.29963088)(608.41794678,86.24963186)
\curveto(608.41795369,86.20963097)(608.4079537,86.16963101)(608.38794678,86.12963186)
\curveto(608.32795378,86.03963114)(608.19295392,85.99463118)(607.98294678,85.99463186)
\lineto(607.86294678,85.99463186)
\curveto(607.80295431,86.00463117)(607.74295437,86.00963117)(607.68294678,86.00963186)
\curveto(607.6129545,86.01963116)(607.54795456,86.02963115)(607.48794678,86.03963186)
\curveto(607.37795473,86.05963112)(607.27795483,86.0796311)(607.18794678,86.09963186)
\curveto(607.08795502,86.11963106)(606.99295512,86.14963103)(606.90294678,86.18963186)
\curveto(606.83295528,86.20963097)(606.77295534,86.22963095)(606.72294678,86.24963186)
\lineto(606.54294678,86.30963186)
\curveto(606.28295583,86.42963075)(606.03795607,86.58463059)(605.80794678,86.77463186)
\curveto(605.57795653,86.9746302)(605.39295672,87.18962999)(605.25294678,87.41963186)
\curveto(605.17295694,87.52962965)(605.107957,87.64462953)(605.05794678,87.76463186)
\lineto(604.90794678,88.15463186)
\curveto(604.85795725,88.26462891)(604.82795728,88.3796288)(604.81794678,88.49963186)
\curveto(604.79795731,88.61962856)(604.77295734,88.74462843)(604.74294678,88.87463186)
\curveto(604.74295737,88.94462823)(604.74295737,89.00962817)(604.74294678,89.06963186)
\curveto(604.73295738,89.12962805)(604.72295739,89.19462798)(604.71294678,89.26463186)
}
}
{
\newrgbcolor{curcolor}{0 0 0}
\pscustom[linestyle=none,fillstyle=solid,fillcolor=curcolor]
{
\newpath
\moveto(610.23294678,101.36424123)
\lineto(610.48794678,101.36424123)
\curveto(610.56795154,101.37423353)(610.64295147,101.36923353)(610.71294678,101.34924123)
\lineto(610.95294678,101.34924123)
\lineto(611.11794678,101.34924123)
\curveto(611.21795089,101.32923357)(611.32295079,101.31923358)(611.43294678,101.31924123)
\curveto(611.53295058,101.31923358)(611.63295048,101.30923359)(611.73294678,101.28924123)
\lineto(611.88294678,101.28924123)
\curveto(612.02295009,101.25923364)(612.16294995,101.23923366)(612.30294678,101.22924123)
\curveto(612.43294968,101.21923368)(612.56294955,101.19423371)(612.69294678,101.15424123)
\curveto(612.77294934,101.13423377)(612.85794925,101.11423379)(612.94794678,101.09424123)
\lineto(613.18794678,101.03424123)
\lineto(613.48794678,100.91424123)
\curveto(613.57794853,100.88423402)(613.66794844,100.84923405)(613.75794678,100.80924123)
\curveto(613.97794813,100.70923419)(614.19294792,100.57423433)(614.40294678,100.40424123)
\curveto(614.6129475,100.24423466)(614.78294733,100.06923483)(614.91294678,99.87924123)
\curveto(614.95294716,99.82923507)(614.99294712,99.76923513)(615.03294678,99.69924123)
\curveto(615.06294705,99.63923526)(615.09794701,99.57923532)(615.13794678,99.51924123)
\curveto(615.18794692,99.43923546)(615.22794688,99.34423556)(615.25794678,99.23424123)
\curveto(615.28794682,99.12423578)(615.31794679,99.01923588)(615.34794678,98.91924123)
\curveto(615.38794672,98.80923609)(615.4129467,98.6992362)(615.42294678,98.58924123)
\curveto(615.43294668,98.47923642)(615.44794666,98.36423654)(615.46794678,98.24424123)
\curveto(615.47794663,98.2042367)(615.47794663,98.15923674)(615.46794678,98.10924123)
\curveto(615.46794664,98.06923683)(615.47294664,98.02923687)(615.48294678,97.98924123)
\curveto(615.49294662,97.94923695)(615.49794661,97.89423701)(615.49794678,97.82424123)
\curveto(615.49794661,97.75423715)(615.49294662,97.7042372)(615.48294678,97.67424123)
\curveto(615.46294665,97.62423728)(615.45794665,97.57923732)(615.46794678,97.53924123)
\curveto(615.47794663,97.4992374)(615.47794663,97.46423744)(615.46794678,97.43424123)
\lineto(615.46794678,97.34424123)
\curveto(615.44794666,97.28423762)(615.43294668,97.21923768)(615.42294678,97.14924123)
\curveto(615.42294669,97.08923781)(615.41794669,97.02423788)(615.40794678,96.95424123)
\curveto(615.35794675,96.78423812)(615.3079468,96.62423828)(615.25794678,96.47424123)
\curveto(615.2079469,96.32423858)(615.14294697,96.17923872)(615.06294678,96.03924123)
\curveto(615.02294709,95.98923891)(614.99294712,95.93423897)(614.97294678,95.87424123)
\curveto(614.94294717,95.82423908)(614.9079472,95.77423913)(614.86794678,95.72424123)
\curveto(614.68794742,95.48423942)(614.46794764,95.28423962)(614.20794678,95.12424123)
\curveto(613.94794816,94.96423994)(613.66294845,94.82424008)(613.35294678,94.70424123)
\curveto(613.2129489,94.64424026)(613.07294904,94.5992403)(612.93294678,94.56924123)
\curveto(612.78294933,94.53924036)(612.62794948,94.5042404)(612.46794678,94.46424123)
\curveto(612.35794975,94.44424046)(612.24794986,94.42924047)(612.13794678,94.41924123)
\curveto(612.02795008,94.40924049)(611.91795019,94.39424051)(611.80794678,94.37424123)
\curveto(611.76795034,94.36424054)(611.72795038,94.35924054)(611.68794678,94.35924123)
\curveto(611.64795046,94.36924053)(611.6079505,94.36924053)(611.56794678,94.35924123)
\curveto(611.51795059,94.34924055)(611.46795064,94.34424056)(611.41794678,94.34424123)
\lineto(611.25294678,94.34424123)
\curveto(611.20295091,94.32424058)(611.15295096,94.31924058)(611.10294678,94.32924123)
\curveto(611.04295107,94.33924056)(610.98795112,94.33924056)(610.93794678,94.32924123)
\curveto(610.89795121,94.31924058)(610.85295126,94.31924058)(610.80294678,94.32924123)
\curveto(610.75295136,94.33924056)(610.70295141,94.33424057)(610.65294678,94.31424123)
\curveto(610.58295153,94.29424061)(610.5079516,94.28924061)(610.42794678,94.29924123)
\curveto(610.33795177,94.30924059)(610.25295186,94.31424059)(610.17294678,94.31424123)
\curveto(610.08295203,94.31424059)(609.98295213,94.30924059)(609.87294678,94.29924123)
\curveto(609.75295236,94.28924061)(609.65295246,94.29424061)(609.57294678,94.31424123)
\lineto(609.28794678,94.31424123)
\lineto(608.65794678,94.35924123)
\curveto(608.55795355,94.36924053)(608.46295365,94.37924052)(608.37294678,94.38924123)
\lineto(608.07294678,94.41924123)
\curveto(608.02295409,94.43924046)(607.97295414,94.44424046)(607.92294678,94.43424123)
\curveto(607.86295425,94.43424047)(607.8079543,94.44424046)(607.75794678,94.46424123)
\curveto(607.58795452,94.51424039)(607.42295469,94.55424035)(607.26294678,94.58424123)
\curveto(607.09295502,94.61424029)(606.93295518,94.66424024)(606.78294678,94.73424123)
\curveto(606.32295579,94.92423998)(605.94795616,95.14423976)(605.65794678,95.39424123)
\curveto(605.36795674,95.65423925)(605.12295699,96.01423889)(604.92294678,96.47424123)
\curveto(604.87295724,96.6042383)(604.83795727,96.73423817)(604.81794678,96.86424123)
\curveto(604.79795731,97.0042379)(604.77295734,97.14423776)(604.74294678,97.28424123)
\curveto(604.73295738,97.35423755)(604.72795738,97.41923748)(604.72794678,97.47924123)
\curveto(604.72795738,97.53923736)(604.72295739,97.6042373)(604.71294678,97.67424123)
\curveto(604.69295742,98.5042364)(604.84295727,99.17423573)(605.16294678,99.68424123)
\curveto(605.47295664,100.19423471)(605.9129562,100.57423433)(606.48294678,100.82424123)
\curveto(606.60295551,100.87423403)(606.72795538,100.91923398)(606.85794678,100.95924123)
\curveto(606.98795512,100.9992339)(607.12295499,101.04423386)(607.26294678,101.09424123)
\curveto(607.34295477,101.11423379)(607.42795468,101.12923377)(607.51794678,101.13924123)
\lineto(607.75794678,101.19924123)
\curveto(607.86795424,101.22923367)(607.97795413,101.24423366)(608.08794678,101.24424123)
\curveto(608.19795391,101.25423365)(608.3079538,101.26923363)(608.41794678,101.28924123)
\curveto(608.46795364,101.30923359)(608.5129536,101.31423359)(608.55294678,101.30424123)
\curveto(608.59295352,101.3042336)(608.63295348,101.30923359)(608.67294678,101.31924123)
\curveto(608.72295339,101.32923357)(608.77795333,101.32923357)(608.83794678,101.31924123)
\curveto(608.88795322,101.31923358)(608.93795317,101.32423358)(608.98794678,101.33424123)
\lineto(609.12294678,101.33424123)
\curveto(609.18295293,101.35423355)(609.25295286,101.35423355)(609.33294678,101.33424123)
\curveto(609.40295271,101.32423358)(609.46795264,101.32923357)(609.52794678,101.34924123)
\curveto(609.55795255,101.35923354)(609.59795251,101.36423354)(609.64794678,101.36424123)
\lineto(609.76794678,101.36424123)
\lineto(610.23294678,101.36424123)
\moveto(612.55794678,99.81924123)
\curveto(612.23794987,99.91923498)(611.87295024,99.97923492)(611.46294678,99.99924123)
\curveto(611.05295106,100.01923488)(610.64295147,100.02923487)(610.23294678,100.02924123)
\curveto(609.80295231,100.02923487)(609.38295273,100.01923488)(608.97294678,99.99924123)
\curveto(608.56295355,99.97923492)(608.17795393,99.93423497)(607.81794678,99.86424123)
\curveto(607.45795465,99.79423511)(607.13795497,99.68423522)(606.85794678,99.53424123)
\curveto(606.56795554,99.39423551)(606.33295578,99.1992357)(606.15294678,98.94924123)
\curveto(606.04295607,98.78923611)(605.96295615,98.60923629)(605.91294678,98.40924123)
\curveto(605.85295626,98.20923669)(605.82295629,97.96423694)(605.82294678,97.67424123)
\curveto(605.84295627,97.65423725)(605.85295626,97.61923728)(605.85294678,97.56924123)
\curveto(605.84295627,97.51923738)(605.84295627,97.47923742)(605.85294678,97.44924123)
\curveto(605.87295624,97.36923753)(605.89295622,97.29423761)(605.91294678,97.22424123)
\curveto(605.92295619,97.16423774)(605.94295617,97.0992378)(605.97294678,97.02924123)
\curveto(606.09295602,96.75923814)(606.26295585,96.53923836)(606.48294678,96.36924123)
\curveto(606.69295542,96.20923869)(606.93795517,96.07423883)(607.21794678,95.96424123)
\curveto(607.32795478,95.91423899)(607.44795466,95.87423903)(607.57794678,95.84424123)
\curveto(607.69795441,95.82423908)(607.82295429,95.7992391)(607.95294678,95.76924123)
\curveto(608.00295411,95.74923915)(608.05795405,95.73923916)(608.11794678,95.73924123)
\curveto(608.16795394,95.73923916)(608.21795389,95.73423917)(608.26794678,95.72424123)
\curveto(608.35795375,95.71423919)(608.45295366,95.7042392)(608.55294678,95.69424123)
\curveto(608.64295347,95.68423922)(608.73795337,95.67423923)(608.83794678,95.66424123)
\curveto(608.91795319,95.66423924)(609.00295311,95.65923924)(609.09294678,95.64924123)
\lineto(609.33294678,95.64924123)
\lineto(609.51294678,95.64924123)
\curveto(609.54295257,95.63923926)(609.57795253,95.63423927)(609.61794678,95.63424123)
\lineto(609.75294678,95.63424123)
\lineto(610.20294678,95.63424123)
\curveto(610.28295183,95.63423927)(610.36795174,95.62923927)(610.45794678,95.61924123)
\curveto(610.53795157,95.61923928)(610.6129515,95.62923927)(610.68294678,95.64924123)
\lineto(610.95294678,95.64924123)
\curveto(610.97295114,95.64923925)(611.00295111,95.64423926)(611.04294678,95.63424123)
\curveto(611.07295104,95.63423927)(611.09795101,95.63923926)(611.11794678,95.64924123)
\curveto(611.21795089,95.65923924)(611.31795079,95.66423924)(611.41794678,95.66424123)
\curveto(611.5079506,95.67423923)(611.6079505,95.68423922)(611.71794678,95.69424123)
\curveto(611.83795027,95.72423918)(611.96295015,95.73923916)(612.09294678,95.73924123)
\curveto(612.2129499,95.74923915)(612.32794978,95.77423913)(612.43794678,95.81424123)
\curveto(612.73794937,95.89423901)(613.00294911,95.97923892)(613.23294678,96.06924123)
\curveto(613.46294865,96.16923873)(613.67794843,96.31423859)(613.87794678,96.50424123)
\curveto(614.07794803,96.71423819)(614.22794788,96.97923792)(614.32794678,97.29924123)
\curveto(614.34794776,97.33923756)(614.35794775,97.37423753)(614.35794678,97.40424123)
\curveto(614.34794776,97.44423746)(614.35294776,97.48923741)(614.37294678,97.53924123)
\curveto(614.38294773,97.57923732)(614.39294772,97.64923725)(614.40294678,97.74924123)
\curveto(614.4129477,97.85923704)(614.4079477,97.94423696)(614.38794678,98.00424123)
\curveto(614.36794774,98.07423683)(614.35794775,98.14423676)(614.35794678,98.21424123)
\curveto(614.34794776,98.28423662)(614.33294778,98.34923655)(614.31294678,98.40924123)
\curveto(614.25294786,98.60923629)(614.16794794,98.78923611)(614.05794678,98.94924123)
\curveto(614.03794807,98.97923592)(614.01794809,99.0042359)(613.99794678,99.02424123)
\lineto(613.93794678,99.08424123)
\curveto(613.91794819,99.12423578)(613.87794823,99.17423573)(613.81794678,99.23424123)
\curveto(613.67794843,99.33423557)(613.54794856,99.41923548)(613.42794678,99.48924123)
\curveto(613.3079488,99.55923534)(613.16294895,99.62923527)(612.99294678,99.69924123)
\curveto(612.92294919,99.72923517)(612.85294926,99.74923515)(612.78294678,99.75924123)
\curveto(612.7129494,99.77923512)(612.63794947,99.7992351)(612.55794678,99.81924123)
}
}
{
\newrgbcolor{curcolor}{0 0 0}
\pscustom[linestyle=none,fillstyle=solid,fillcolor=curcolor]
{
\newpath
\moveto(604.71294678,106.77385061)
\curveto(604.7129574,106.87384575)(604.72295739,106.96884566)(604.74294678,107.05885061)
\curveto(604.75295736,107.14884548)(604.78295733,107.21384541)(604.83294678,107.25385061)
\curveto(604.9129572,107.31384531)(605.01795709,107.34384528)(605.14794678,107.34385061)
\lineto(605.53794678,107.34385061)
\lineto(607.03794678,107.34385061)
\lineto(613.42794678,107.34385061)
\lineto(614.59794678,107.34385061)
\lineto(614.91294678,107.34385061)
\curveto(615.0129471,107.35384527)(615.09294702,107.33884529)(615.15294678,107.29885061)
\curveto(615.23294688,107.24884538)(615.28294683,107.17384545)(615.30294678,107.07385061)
\curveto(615.3129468,106.98384564)(615.31794679,106.87384575)(615.31794678,106.74385061)
\lineto(615.31794678,106.51885061)
\curveto(615.29794681,106.43884619)(615.28294683,106.36884626)(615.27294678,106.30885061)
\curveto(615.25294686,106.24884638)(615.2129469,106.19884643)(615.15294678,106.15885061)
\curveto(615.09294702,106.11884651)(615.01794709,106.09884653)(614.92794678,106.09885061)
\lineto(614.62794678,106.09885061)
\lineto(613.53294678,106.09885061)
\lineto(608.19294678,106.09885061)
\curveto(608.10295401,106.07884655)(608.02795408,106.06384656)(607.96794678,106.05385061)
\curveto(607.89795421,106.05384657)(607.83795427,106.0238466)(607.78794678,105.96385061)
\curveto(607.73795437,105.89384673)(607.7129544,105.80384682)(607.71294678,105.69385061)
\curveto(607.70295441,105.59384703)(607.69795441,105.48384714)(607.69794678,105.36385061)
\lineto(607.69794678,104.22385061)
\lineto(607.69794678,103.72885061)
\curveto(607.68795442,103.56884906)(607.62795448,103.45884917)(607.51794678,103.39885061)
\curveto(607.48795462,103.37884925)(607.45795465,103.36884926)(607.42794678,103.36885061)
\curveto(607.38795472,103.36884926)(607.34295477,103.36384926)(607.29294678,103.35385061)
\curveto(607.17295494,103.33384929)(607.06295505,103.33884929)(606.96294678,103.36885061)
\curveto(606.86295525,103.40884922)(606.79295532,103.46384916)(606.75294678,103.53385061)
\curveto(606.70295541,103.61384901)(606.67795543,103.73384889)(606.67794678,103.89385061)
\curveto(606.67795543,104.05384857)(606.66295545,104.18884844)(606.63294678,104.29885061)
\curveto(606.62295549,104.34884828)(606.61795549,104.40384822)(606.61794678,104.46385061)
\curveto(606.6079555,104.5238481)(606.59295552,104.58384804)(606.57294678,104.64385061)
\curveto(606.52295559,104.79384783)(606.47295564,104.93884769)(606.42294678,105.07885061)
\curveto(606.36295575,105.21884741)(606.29295582,105.35384727)(606.21294678,105.48385061)
\curveto(606.12295599,105.623847)(606.01795609,105.74384688)(605.89794678,105.84385061)
\curveto(605.77795633,105.94384668)(605.64795646,106.03884659)(605.50794678,106.12885061)
\curveto(605.4079567,106.18884644)(605.29795681,106.23384639)(605.17794678,106.26385061)
\curveto(605.05795705,106.30384632)(604.95295716,106.35384627)(604.86294678,106.41385061)
\curveto(604.80295731,106.46384616)(604.76295735,106.53384609)(604.74294678,106.62385061)
\curveto(604.73295738,106.64384598)(604.72795738,106.66884596)(604.72794678,106.69885061)
\curveto(604.72795738,106.7288459)(604.72295739,106.75384587)(604.71294678,106.77385061)
}
}
{
\newrgbcolor{curcolor}{0 0 0}
\pscustom[linestyle=none,fillstyle=solid,fillcolor=curcolor]
{
\newpath
\moveto(604.71294678,115.12345998)
\curveto(604.7129574,115.22345513)(604.72295739,115.31845503)(604.74294678,115.40845998)
\curveto(604.75295736,115.49845485)(604.78295733,115.56345479)(604.83294678,115.60345998)
\curveto(604.9129572,115.66345469)(605.01795709,115.69345466)(605.14794678,115.69345998)
\lineto(605.53794678,115.69345998)
\lineto(607.03794678,115.69345998)
\lineto(613.42794678,115.69345998)
\lineto(614.59794678,115.69345998)
\lineto(614.91294678,115.69345998)
\curveto(615.0129471,115.70345465)(615.09294702,115.68845466)(615.15294678,115.64845998)
\curveto(615.23294688,115.59845475)(615.28294683,115.52345483)(615.30294678,115.42345998)
\curveto(615.3129468,115.33345502)(615.31794679,115.22345513)(615.31794678,115.09345998)
\lineto(615.31794678,114.86845998)
\curveto(615.29794681,114.78845556)(615.28294683,114.71845563)(615.27294678,114.65845998)
\curveto(615.25294686,114.59845575)(615.2129469,114.5484558)(615.15294678,114.50845998)
\curveto(615.09294702,114.46845588)(615.01794709,114.4484559)(614.92794678,114.44845998)
\lineto(614.62794678,114.44845998)
\lineto(613.53294678,114.44845998)
\lineto(608.19294678,114.44845998)
\curveto(608.10295401,114.42845592)(608.02795408,114.41345594)(607.96794678,114.40345998)
\curveto(607.89795421,114.40345595)(607.83795427,114.37345598)(607.78794678,114.31345998)
\curveto(607.73795437,114.24345611)(607.7129544,114.1534562)(607.71294678,114.04345998)
\curveto(607.70295441,113.94345641)(607.69795441,113.83345652)(607.69794678,113.71345998)
\lineto(607.69794678,112.57345998)
\lineto(607.69794678,112.07845998)
\curveto(607.68795442,111.91845843)(607.62795448,111.80845854)(607.51794678,111.74845998)
\curveto(607.48795462,111.72845862)(607.45795465,111.71845863)(607.42794678,111.71845998)
\curveto(607.38795472,111.71845863)(607.34295477,111.71345864)(607.29294678,111.70345998)
\curveto(607.17295494,111.68345867)(607.06295505,111.68845866)(606.96294678,111.71845998)
\curveto(606.86295525,111.75845859)(606.79295532,111.81345854)(606.75294678,111.88345998)
\curveto(606.70295541,111.96345839)(606.67795543,112.08345827)(606.67794678,112.24345998)
\curveto(606.67795543,112.40345795)(606.66295545,112.53845781)(606.63294678,112.64845998)
\curveto(606.62295549,112.69845765)(606.61795549,112.7534576)(606.61794678,112.81345998)
\curveto(606.6079555,112.87345748)(606.59295552,112.93345742)(606.57294678,112.99345998)
\curveto(606.52295559,113.14345721)(606.47295564,113.28845706)(606.42294678,113.42845998)
\curveto(606.36295575,113.56845678)(606.29295582,113.70345665)(606.21294678,113.83345998)
\curveto(606.12295599,113.97345638)(606.01795609,114.09345626)(605.89794678,114.19345998)
\curveto(605.77795633,114.29345606)(605.64795646,114.38845596)(605.50794678,114.47845998)
\curveto(605.4079567,114.53845581)(605.29795681,114.58345577)(605.17794678,114.61345998)
\curveto(605.05795705,114.6534557)(604.95295716,114.70345565)(604.86294678,114.76345998)
\curveto(604.80295731,114.81345554)(604.76295735,114.88345547)(604.74294678,114.97345998)
\curveto(604.73295738,114.99345536)(604.72795738,115.01845533)(604.72794678,115.04845998)
\curveto(604.72795738,115.07845527)(604.72295739,115.10345525)(604.71294678,115.12345998)
}
}
{
\newrgbcolor{curcolor}{0 0 0}
\pscustom[linestyle=none,fillstyle=solid,fillcolor=curcolor]
{
\newpath
\moveto(626.58423584,29.18119436)
\lineto(626.58423584,30.09619436)
\curveto(626.58424653,30.19619171)(626.58424653,30.29119161)(626.58423584,30.38119436)
\curveto(626.58424653,30.47119143)(626.60424651,30.54619136)(626.64423584,30.60619436)
\curveto(626.70424641,30.69619121)(626.78424633,30.75619115)(626.88423584,30.78619436)
\curveto(626.98424613,30.82619108)(627.08924603,30.87119103)(627.19923584,30.92119436)
\curveto(627.38924573,31.0011909)(627.57924554,31.07119083)(627.76923584,31.13119436)
\curveto(627.95924516,31.2011907)(628.14924497,31.27619063)(628.33923584,31.35619436)
\curveto(628.5192446,31.42619048)(628.70424441,31.49119041)(628.89423584,31.55119436)
\curveto(629.07424404,31.61119029)(629.25424386,31.68119022)(629.43423584,31.76119436)
\curveto(629.57424354,31.82119008)(629.7192434,31.87619003)(629.86923584,31.92619436)
\curveto(630.0192431,31.97618993)(630.16424295,32.03118987)(630.30423584,32.09119436)
\curveto(630.75424236,32.27118963)(631.20924191,32.44118946)(631.66923584,32.60119436)
\curveto(632.119241,32.76118914)(632.56924055,32.93118897)(633.01923584,33.11119436)
\curveto(633.06924005,33.13118877)(633.11924,33.14618876)(633.16923584,33.15619436)
\lineto(633.31923584,33.21619436)
\curveto(633.53923958,33.3061886)(633.76423935,33.39118851)(633.99423584,33.47119436)
\curveto(634.2142389,33.55118835)(634.43423868,33.63618827)(634.65423584,33.72619436)
\curveto(634.74423837,33.76618814)(634.85423826,33.8061881)(634.98423584,33.84619436)
\curveto(635.10423801,33.88618802)(635.17423794,33.95118795)(635.19423584,34.04119436)
\curveto(635.20423791,34.08118782)(635.20423791,34.11118779)(635.19423584,34.13119436)
\lineto(635.13423584,34.19119436)
\curveto(635.08423803,34.24118766)(635.02923809,34.27618763)(634.96923584,34.29619436)
\curveto(634.90923821,34.32618758)(634.84423827,34.35618755)(634.77423584,34.38619436)
\lineto(634.14423584,34.62619436)
\curveto(633.92423919,34.7061872)(633.70923941,34.78618712)(633.49923584,34.86619436)
\lineto(633.34923584,34.92619436)
\lineto(633.16923584,34.98619436)
\curveto(632.97924014,35.06618684)(632.78924033,35.13618677)(632.59923584,35.19619436)
\curveto(632.39924072,35.26618664)(632.19924092,35.34118656)(631.99923584,35.42119436)
\curveto(631.4192417,35.66118624)(630.83424228,35.88118602)(630.24423584,36.08119436)
\curveto(629.65424346,36.29118561)(629.06924405,36.51618539)(628.48923584,36.75619436)
\curveto(628.28924483,36.83618507)(628.08424503,36.91118499)(627.87423584,36.98119436)
\curveto(627.66424545,37.06118484)(627.45924566,37.14118476)(627.25923584,37.22119436)
\curveto(627.17924594,37.26118464)(627.07924604,37.29618461)(626.95923584,37.32619436)
\curveto(626.83924628,37.36618454)(626.75424636,37.42118448)(626.70423584,37.49119436)
\curveto(626.66424645,37.55118435)(626.63424648,37.62618428)(626.61423584,37.71619436)
\curveto(626.59424652,37.81618409)(626.58424653,37.92618398)(626.58423584,38.04619436)
\curveto(626.57424654,38.16618374)(626.57424654,38.28618362)(626.58423584,38.40619436)
\curveto(626.58424653,38.52618338)(626.58424653,38.63618327)(626.58423584,38.73619436)
\curveto(626.58424653,38.82618308)(626.58424653,38.91618299)(626.58423584,39.00619436)
\curveto(626.58424653,39.1061828)(626.60424651,39.18118272)(626.64423584,39.23119436)
\curveto(626.69424642,39.32118258)(626.78424633,39.37118253)(626.91423584,39.38119436)
\curveto(627.04424607,39.39118251)(627.18424593,39.39618251)(627.33423584,39.39619436)
\lineto(628.98423584,39.39619436)
\lineto(635.25423584,39.39619436)
\lineto(636.51423584,39.39619436)
\curveto(636.62423649,39.39618251)(636.73423638,39.39618251)(636.84423584,39.39619436)
\curveto(636.95423616,39.4061825)(637.03923608,39.38618252)(637.09923584,39.33619436)
\curveto(637.15923596,39.3061826)(637.19923592,39.26118264)(637.21923584,39.20119436)
\curveto(637.22923589,39.14118276)(637.24423587,39.07118283)(637.26423584,38.99119436)
\lineto(637.26423584,38.75119436)
\lineto(637.26423584,38.39119436)
\curveto(637.25423586,38.28118362)(637.20923591,38.2011837)(637.12923584,38.15119436)
\curveto(637.09923602,38.13118377)(637.06923605,38.11618379)(637.03923584,38.10619436)
\curveto(636.99923612,38.1061838)(636.95423616,38.09618381)(636.90423584,38.07619436)
\lineto(636.73923584,38.07619436)
\curveto(636.67923644,38.06618384)(636.60923651,38.06118384)(636.52923584,38.06119436)
\curveto(636.44923667,38.07118383)(636.37423674,38.07618383)(636.30423584,38.07619436)
\lineto(635.46423584,38.07619436)
\lineto(631.03923584,38.07619436)
\curveto(630.78924233,38.07618383)(630.53924258,38.07618383)(630.28923584,38.07619436)
\curveto(630.02924309,38.07618383)(629.77924334,38.07118383)(629.53923584,38.06119436)
\curveto(629.43924368,38.06118384)(629.32924379,38.05618385)(629.20923584,38.04619436)
\curveto(629.08924403,38.03618387)(629.02924409,37.98118392)(629.02923584,37.88119436)
\lineto(629.04423584,37.88119436)
\curveto(629.06424405,37.81118409)(629.12924399,37.75118415)(629.23923584,37.70119436)
\curveto(629.34924377,37.66118424)(629.44424367,37.62618428)(629.52423584,37.59619436)
\curveto(629.69424342,37.52618438)(629.86924325,37.46118444)(630.04923584,37.40119436)
\curveto(630.2192429,37.34118456)(630.38924273,37.27118463)(630.55923584,37.19119436)
\curveto(630.60924251,37.17118473)(630.65424246,37.15618475)(630.69423584,37.14619436)
\curveto(630.73424238,37.13618477)(630.77924234,37.12118478)(630.82923584,37.10119436)
\curveto(631.00924211,37.02118488)(631.19424192,36.95118495)(631.38423584,36.89119436)
\curveto(631.56424155,36.84118506)(631.74424137,36.77618513)(631.92423584,36.69619436)
\curveto(632.07424104,36.62618528)(632.22924089,36.56618534)(632.38923584,36.51619436)
\curveto(632.53924058,36.46618544)(632.68924043,36.41118549)(632.83923584,36.35119436)
\curveto(633.30923981,36.15118575)(633.78423933,35.97118593)(634.26423584,35.81119436)
\curveto(634.73423838,35.65118625)(635.19923792,35.47618643)(635.65923584,35.28619436)
\curveto(635.83923728,35.2061867)(636.0192371,35.13618677)(636.19923584,35.07619436)
\curveto(636.37923674,35.01618689)(636.55923656,34.95118695)(636.73923584,34.88119436)
\curveto(636.84923627,34.83118707)(636.95423616,34.78118712)(637.05423584,34.73119436)
\curveto(637.14423597,34.69118721)(637.20923591,34.6061873)(637.24923584,34.47619436)
\curveto(637.25923586,34.45618745)(637.26423585,34.43118747)(637.26423584,34.40119436)
\curveto(637.25423586,34.38118752)(637.25423586,34.35618755)(637.26423584,34.32619436)
\curveto(637.27423584,34.29618761)(637.27923584,34.26118764)(637.27923584,34.22119436)
\curveto(637.26923585,34.18118772)(637.26423585,34.14118776)(637.26423584,34.10119436)
\lineto(637.26423584,33.80119436)
\curveto(637.26423585,33.7011882)(637.23923588,33.62118828)(637.18923584,33.56119436)
\curveto(637.13923598,33.48118842)(637.06923605,33.42118848)(636.97923584,33.38119436)
\curveto(636.87923624,33.35118855)(636.77923634,33.31118859)(636.67923584,33.26119436)
\curveto(636.47923664,33.18118872)(636.27423684,33.1011888)(636.06423584,33.02119436)
\curveto(635.84423727,32.95118895)(635.63423748,32.87618903)(635.43423584,32.79619436)
\curveto(635.25423786,32.71618919)(635.07423804,32.64618926)(634.89423584,32.58619436)
\curveto(634.70423841,32.53618937)(634.5192386,32.47118943)(634.33923584,32.39119436)
\curveto(633.77923934,32.16118974)(633.2142399,31.94618996)(632.64423584,31.74619436)
\curveto(632.07424104,31.54619036)(631.50924161,31.33119057)(630.94923584,31.10119436)
\lineto(630.31923584,30.86119436)
\curveto(630.09924302,30.79119111)(629.88924323,30.71619119)(629.68923584,30.63619436)
\curveto(629.57924354,30.58619132)(629.47424364,30.54119136)(629.37423584,30.50119436)
\curveto(629.26424385,30.47119143)(629.16924395,30.42119148)(629.08923584,30.35119436)
\curveto(629.06924405,30.34119156)(629.05924406,30.33119157)(629.05923584,30.32119436)
\lineto(629.02923584,30.29119436)
\lineto(629.02923584,30.21619436)
\lineto(629.05923584,30.18619436)
\curveto(629.05924406,30.17619173)(629.06424405,30.16619174)(629.07423584,30.15619436)
\curveto(629.12424399,30.13619177)(629.17924394,30.12619178)(629.23923584,30.12619436)
\curveto(629.29924382,30.12619178)(629.35924376,30.11619179)(629.41923584,30.09619436)
\lineto(629.58423584,30.09619436)
\curveto(629.64424347,30.07619183)(629.70924341,30.07119183)(629.77923584,30.08119436)
\curveto(629.84924327,30.09119181)(629.9192432,30.09619181)(629.98923584,30.09619436)
\lineto(630.79923584,30.09619436)
\lineto(635.35923584,30.09619436)
\lineto(636.54423584,30.09619436)
\curveto(636.65423646,30.09619181)(636.76423635,30.09119181)(636.87423584,30.08119436)
\curveto(636.98423613,30.08119182)(637.06923605,30.05619185)(637.12923584,30.00619436)
\curveto(637.20923591,29.95619195)(637.25423586,29.86619204)(637.26423584,29.73619436)
\lineto(637.26423584,29.34619436)
\lineto(637.26423584,29.15119436)
\curveto(637.26423585,29.1011928)(637.25423586,29.05119285)(637.23423584,29.00119436)
\curveto(637.19423592,28.87119303)(637.10923601,28.79619311)(636.97923584,28.77619436)
\curveto(636.84923627,28.76619314)(636.69923642,28.76119314)(636.52923584,28.76119436)
\lineto(634.78923584,28.76119436)
\lineto(628.78923584,28.76119436)
\lineto(627.37923584,28.76119436)
\curveto(627.26924585,28.76119314)(627.15424596,28.75619315)(627.03423584,28.74619436)
\curveto(626.9142462,28.74619316)(626.8192463,28.77119313)(626.74923584,28.82119436)
\curveto(626.68924643,28.86119304)(626.63924648,28.93619297)(626.59923584,29.04619436)
\curveto(626.58924653,29.06619284)(626.58924653,29.08619282)(626.59923584,29.10619436)
\curveto(626.59924652,29.13619277)(626.59424652,29.16119274)(626.58423584,29.18119436)
}
}
{
\newrgbcolor{curcolor}{0 0 0}
\pscustom[linestyle=none,fillstyle=solid,fillcolor=curcolor]
{
\newpath
\moveto(636.70923584,48.38330373)
\curveto(636.86923625,48.4132959)(637.00423611,48.39829592)(637.11423584,48.33830373)
\curveto(637.2142359,48.27829604)(637.28923583,48.19829612)(637.33923584,48.09830373)
\curveto(637.35923576,48.04829627)(637.36923575,47.99329632)(637.36923584,47.93330373)
\curveto(637.36923575,47.88329643)(637.37923574,47.82829649)(637.39923584,47.76830373)
\curveto(637.44923567,47.54829677)(637.43423568,47.32829699)(637.35423584,47.10830373)
\curveto(637.28423583,46.89829742)(637.19423592,46.75329756)(637.08423584,46.67330373)
\curveto(637.0142361,46.62329769)(636.93423618,46.57829774)(636.84423584,46.53830373)
\curveto(636.74423637,46.49829782)(636.66423645,46.44829787)(636.60423584,46.38830373)
\curveto(636.58423653,46.36829795)(636.56423655,46.34329797)(636.54423584,46.31330373)
\curveto(636.52423659,46.29329802)(636.5192366,46.26329805)(636.52923584,46.22330373)
\curveto(636.55923656,46.1132982)(636.6142365,46.00829831)(636.69423584,45.90830373)
\curveto(636.77423634,45.8182985)(636.84423627,45.72829859)(636.90423584,45.63830373)
\curveto(636.98423613,45.50829881)(637.05923606,45.36829895)(637.12923584,45.21830373)
\curveto(637.18923593,45.06829925)(637.24423587,44.90829941)(637.29423584,44.73830373)
\curveto(637.32423579,44.63829968)(637.34423577,44.52829979)(637.35423584,44.40830373)
\curveto(637.36423575,44.29830002)(637.37923574,44.18830013)(637.39923584,44.07830373)
\curveto(637.40923571,44.02830029)(637.4142357,43.98330033)(637.41423584,43.94330373)
\lineto(637.41423584,43.83830373)
\curveto(637.43423568,43.72830059)(637.43423568,43.62330069)(637.41423584,43.52330373)
\lineto(637.41423584,43.38830373)
\curveto(637.40423571,43.33830098)(637.39923572,43.28830103)(637.39923584,43.23830373)
\curveto(637.39923572,43.18830113)(637.38923573,43.14330117)(637.36923584,43.10330373)
\curveto(637.35923576,43.06330125)(637.35423576,43.02830129)(637.35423584,42.99830373)
\curveto(637.36423575,42.97830134)(637.36423575,42.95330136)(637.35423584,42.92330373)
\lineto(637.29423584,42.68330373)
\curveto(637.28423583,42.60330171)(637.26423585,42.52830179)(637.23423584,42.45830373)
\curveto(637.10423601,42.15830216)(636.95923616,41.9133024)(636.79923584,41.72330373)
\curveto(636.62923649,41.54330277)(636.39423672,41.39330292)(636.09423584,41.27330373)
\curveto(635.87423724,41.18330313)(635.60923751,41.13830318)(635.29923584,41.13830373)
\lineto(634.98423584,41.13830373)
\curveto(634.93423818,41.14830317)(634.88423823,41.15330316)(634.83423584,41.15330373)
\lineto(634.65423584,41.18330373)
\lineto(634.32423584,41.30330373)
\curveto(634.2142389,41.34330297)(634.114239,41.39330292)(634.02423584,41.45330373)
\curveto(633.73423938,41.63330268)(633.5192396,41.87830244)(633.37923584,42.18830373)
\curveto(633.23923988,42.49830182)(633.11424,42.83830148)(633.00423584,43.20830373)
\curveto(632.96424015,43.34830097)(632.93424018,43.49330082)(632.91423584,43.64330373)
\curveto(632.89424022,43.79330052)(632.86924025,43.94330037)(632.83923584,44.09330373)
\curveto(632.8192403,44.16330015)(632.80924031,44.22830009)(632.80923584,44.28830373)
\curveto(632.80924031,44.35829996)(632.79924032,44.43329988)(632.77923584,44.51330373)
\curveto(632.75924036,44.58329973)(632.74924037,44.65329966)(632.74923584,44.72330373)
\curveto(632.73924038,44.79329952)(632.72424039,44.86829945)(632.70423584,44.94830373)
\curveto(632.64424047,45.19829912)(632.59424052,45.43329888)(632.55423584,45.65330373)
\curveto(632.50424061,45.87329844)(632.38924073,46.04829827)(632.20923584,46.17830373)
\curveto(632.12924099,46.23829808)(632.02924109,46.28829803)(631.90923584,46.32830373)
\curveto(631.77924134,46.36829795)(631.63924148,46.36829795)(631.48923584,46.32830373)
\curveto(631.24924187,46.26829805)(631.05924206,46.17829814)(630.91923584,46.05830373)
\curveto(630.77924234,45.94829837)(630.66924245,45.78829853)(630.58923584,45.57830373)
\curveto(630.53924258,45.45829886)(630.50424261,45.313299)(630.48423584,45.14330373)
\curveto(630.46424265,44.98329933)(630.45424266,44.8132995)(630.45423584,44.63330373)
\curveto(630.45424266,44.45329986)(630.46424265,44.27830004)(630.48423584,44.10830373)
\curveto(630.50424261,43.93830038)(630.53424258,43.79330052)(630.57423584,43.67330373)
\curveto(630.63424248,43.50330081)(630.7192424,43.33830098)(630.82923584,43.17830373)
\curveto(630.88924223,43.09830122)(630.96924215,43.02330129)(631.06923584,42.95330373)
\curveto(631.15924196,42.89330142)(631.25924186,42.83830148)(631.36923584,42.78830373)
\curveto(631.44924167,42.75830156)(631.53424158,42.72830159)(631.62423584,42.69830373)
\curveto(631.7142414,42.67830164)(631.78424133,42.63330168)(631.83423584,42.56330373)
\curveto(631.86424125,42.52330179)(631.88924123,42.45330186)(631.90923584,42.35330373)
\curveto(631.9192412,42.26330205)(631.92424119,42.16830215)(631.92423584,42.06830373)
\curveto(631.92424119,41.96830235)(631.9192412,41.86830245)(631.90923584,41.76830373)
\curveto(631.88924123,41.67830264)(631.86424125,41.6133027)(631.83423584,41.57330373)
\curveto(631.80424131,41.53330278)(631.75424136,41.50330281)(631.68423584,41.48330373)
\curveto(631.6142415,41.46330285)(631.53924158,41.46330285)(631.45923584,41.48330373)
\curveto(631.32924179,41.5133028)(631.20924191,41.54330277)(631.09923584,41.57330373)
\curveto(630.97924214,41.6133027)(630.86424225,41.65830266)(630.75423584,41.70830373)
\curveto(630.40424271,41.89830242)(630.13424298,42.13830218)(629.94423584,42.42830373)
\curveto(629.74424337,42.7183016)(629.58424353,43.07830124)(629.46423584,43.50830373)
\curveto(629.44424367,43.60830071)(629.42924369,43.70830061)(629.41923584,43.80830373)
\curveto(629.40924371,43.9183004)(629.39424372,44.02830029)(629.37423584,44.13830373)
\curveto(629.36424375,44.17830014)(629.36424375,44.24330007)(629.37423584,44.33330373)
\curveto(629.37424374,44.42329989)(629.36424375,44.47829984)(629.34423584,44.49830373)
\curveto(629.33424378,45.19829912)(629.4142437,45.80829851)(629.58423584,46.32830373)
\curveto(629.75424336,46.84829747)(630.07924304,47.2132971)(630.55923584,47.42330373)
\curveto(630.75924236,47.5132968)(630.99424212,47.56329675)(631.26423584,47.57330373)
\curveto(631.52424159,47.59329672)(631.79924132,47.60329671)(632.08923584,47.60330373)
\lineto(635.40423584,47.60330373)
\curveto(635.54423757,47.60329671)(635.67923744,47.60829671)(635.80923584,47.61830373)
\curveto(635.93923718,47.62829669)(636.04423707,47.65829666)(636.12423584,47.70830373)
\curveto(636.19423692,47.75829656)(636.24423687,47.82329649)(636.27423584,47.90330373)
\curveto(636.3142368,47.99329632)(636.34423677,48.07829624)(636.36423584,48.15830373)
\curveto(636.37423674,48.23829608)(636.4192367,48.29829602)(636.49923584,48.33830373)
\curveto(636.52923659,48.35829596)(636.55923656,48.36829595)(636.58923584,48.36830373)
\curveto(636.6192365,48.36829595)(636.65923646,48.37329594)(636.70923584,48.38330373)
\moveto(635.04423584,46.23830373)
\curveto(634.90423821,46.29829802)(634.74423837,46.32829799)(634.56423584,46.32830373)
\curveto(634.37423874,46.33829798)(634.17923894,46.34329797)(633.97923584,46.34330373)
\curveto(633.86923925,46.34329797)(633.76923935,46.33829798)(633.67923584,46.32830373)
\curveto(633.58923953,46.318298)(633.5192396,46.27829804)(633.46923584,46.20830373)
\curveto(633.44923967,46.17829814)(633.43923968,46.10829821)(633.43923584,45.99830373)
\curveto(633.45923966,45.97829834)(633.46923965,45.94329837)(633.46923584,45.89330373)
\curveto(633.46923965,45.84329847)(633.47923964,45.79829852)(633.49923584,45.75830373)
\curveto(633.5192396,45.67829864)(633.53923958,45.58829873)(633.55923584,45.48830373)
\lineto(633.61923584,45.18830373)
\curveto(633.6192395,45.15829916)(633.62423949,45.12329919)(633.63423584,45.08330373)
\lineto(633.63423584,44.97830373)
\curveto(633.67423944,44.82829949)(633.69923942,44.66329965)(633.70923584,44.48330373)
\curveto(633.70923941,44.3133)(633.72923939,44.15330016)(633.76923584,44.00330373)
\curveto(633.78923933,43.92330039)(633.80923931,43.84830047)(633.82923584,43.77830373)
\curveto(633.83923928,43.7183006)(633.85423926,43.64830067)(633.87423584,43.56830373)
\curveto(633.92423919,43.40830091)(633.98923913,43.25830106)(634.06923584,43.11830373)
\curveto(634.13923898,42.97830134)(634.22923889,42.85830146)(634.33923584,42.75830373)
\curveto(634.44923867,42.65830166)(634.58423853,42.58330173)(634.74423584,42.53330373)
\curveto(634.89423822,42.48330183)(635.07923804,42.46330185)(635.29923584,42.47330373)
\curveto(635.39923772,42.47330184)(635.49423762,42.48830183)(635.58423584,42.51830373)
\curveto(635.66423745,42.55830176)(635.73923738,42.60330171)(635.80923584,42.65330373)
\curveto(635.9192372,42.73330158)(636.0142371,42.83830148)(636.09423584,42.96830373)
\curveto(636.16423695,43.09830122)(636.22423689,43.23830108)(636.27423584,43.38830373)
\curveto(636.28423683,43.43830088)(636.28923683,43.48830083)(636.28923584,43.53830373)
\curveto(636.28923683,43.58830073)(636.29423682,43.63830068)(636.30423584,43.68830373)
\curveto(636.32423679,43.75830056)(636.33923678,43.84330047)(636.34923584,43.94330373)
\curveto(636.34923677,44.05330026)(636.33923678,44.14330017)(636.31923584,44.21330373)
\curveto(636.29923682,44.27330004)(636.29423682,44.33329998)(636.30423584,44.39330373)
\curveto(636.30423681,44.45329986)(636.29423682,44.5132998)(636.27423584,44.57330373)
\curveto(636.25423686,44.65329966)(636.23923688,44.72829959)(636.22923584,44.79830373)
\curveto(636.2192369,44.87829944)(636.19923692,44.95329936)(636.16923584,45.02330373)
\curveto(636.04923707,45.313299)(635.90423721,45.55829876)(635.73423584,45.75830373)
\curveto(635.56423755,45.96829835)(635.33423778,46.12829819)(635.04423584,46.23830373)
}
}
{
\newrgbcolor{curcolor}{0 0 0}
\pscustom[linestyle=none,fillstyle=solid,fillcolor=curcolor]
{
\newpath
\moveto(629.53923584,49.26994436)
\lineto(629.53923584,49.71994436)
\curveto(629.52924359,49.88994311)(629.54924357,50.01494298)(629.59923584,50.09494436)
\curveto(629.64924347,50.17494282)(629.7142434,50.22994277)(629.79423584,50.25994436)
\curveto(629.87424324,50.2999427)(629.95924316,50.33994266)(630.04923584,50.37994436)
\curveto(630.17924294,50.42994257)(630.30924281,50.47494252)(630.43923584,50.51494436)
\curveto(630.56924255,50.55494244)(630.69924242,50.5999424)(630.82923584,50.64994436)
\curveto(630.94924217,50.6999423)(631.07424204,50.74494225)(631.20423584,50.78494436)
\curveto(631.32424179,50.82494217)(631.44424167,50.86994213)(631.56423584,50.91994436)
\curveto(631.67424144,50.96994203)(631.78924133,51.00994199)(631.90923584,51.03994436)
\curveto(632.02924109,51.06994193)(632.14924097,51.10994189)(632.26923584,51.15994436)
\curveto(632.55924056,51.27994172)(632.85924026,51.38994161)(633.16923584,51.48994436)
\curveto(633.47923964,51.58994141)(633.77923934,51.6999413)(634.06923584,51.81994436)
\curveto(634.10923901,51.83994116)(634.14923897,51.84994115)(634.18923584,51.84994436)
\curveto(634.2192389,51.84994115)(634.24923887,51.85994114)(634.27923584,51.87994436)
\curveto(634.4192387,51.93994106)(634.56423855,51.994941)(634.71423584,52.04494436)
\lineto(635.13423584,52.19494436)
\curveto(635.20423791,52.22494077)(635.27923784,52.25494074)(635.35923584,52.28494436)
\curveto(635.42923769,52.31494068)(635.47423764,52.36494063)(635.49423584,52.43494436)
\curveto(635.52423759,52.51494048)(635.49923762,52.57494042)(635.41923584,52.61494436)
\curveto(635.32923779,52.66494033)(635.25923786,52.6999403)(635.20923584,52.71994436)
\curveto(635.03923808,52.7999402)(634.85923826,52.86494013)(634.66923584,52.91494436)
\curveto(634.47923864,52.96494003)(634.29423882,53.02493997)(634.11423584,53.09494436)
\curveto(633.88423923,53.18493981)(633.65423946,53.26493973)(633.42423584,53.33494436)
\curveto(633.18423993,53.40493959)(632.95424016,53.48993951)(632.73423584,53.58994436)
\curveto(632.68424043,53.5999394)(632.6192405,53.61493938)(632.53923584,53.63494436)
\curveto(632.44924067,53.67493932)(632.35924076,53.70993929)(632.26923584,53.73994436)
\curveto(632.16924095,53.76993923)(632.07924104,53.7999392)(631.99923584,53.82994436)
\curveto(631.94924117,53.84993915)(631.90424121,53.86493913)(631.86423584,53.87494436)
\curveto(631.82424129,53.88493911)(631.77924134,53.8999391)(631.72923584,53.91994436)
\curveto(631.60924151,53.96993903)(631.48924163,54.00993899)(631.36923584,54.03994436)
\curveto(631.23924188,54.07993892)(631.114242,54.12493887)(630.99423584,54.17494436)
\curveto(630.94424217,54.1949388)(630.89924222,54.20993879)(630.85923584,54.21994436)
\curveto(630.8192423,54.22993877)(630.77424234,54.24493875)(630.72423584,54.26494436)
\curveto(630.63424248,54.30493869)(630.54424257,54.33993866)(630.45423584,54.36994436)
\curveto(630.35424276,54.3999386)(630.25924286,54.42993857)(630.16923584,54.45994436)
\curveto(630.08924303,54.48993851)(630.00924311,54.51493848)(629.92923584,54.53494436)
\curveto(629.83924328,54.56493843)(629.76424335,54.60493839)(629.70423584,54.65494436)
\curveto(629.6142435,54.72493827)(629.56424355,54.81993818)(629.55423584,54.93994436)
\curveto(629.54424357,55.06993793)(629.53924358,55.20993779)(629.53923584,55.35994436)
\curveto(629.53924358,55.43993756)(629.54424357,55.51493748)(629.55423584,55.58494436)
\curveto(629.55424356,55.66493733)(629.56924355,55.72993727)(629.59923584,55.77994436)
\curveto(629.65924346,55.86993713)(629.75424336,55.8949371)(629.88423584,55.85494436)
\curveto(630.0142431,55.81493718)(630.114243,55.77993722)(630.18423584,55.74994436)
\lineto(630.24423584,55.71994436)
\curveto(630.26424285,55.71993728)(630.28424283,55.71493728)(630.30423584,55.70494436)
\curveto(630.58424253,55.5949374)(630.86924225,55.48493751)(631.15923584,55.37494436)
\lineto(631.99923584,55.04494436)
\curveto(632.07924104,55.01493798)(632.15424096,54.98993801)(632.22423584,54.96994436)
\curveto(632.28424083,54.94993805)(632.34924077,54.92493807)(632.41923584,54.89494436)
\curveto(632.6192405,54.81493818)(632.82424029,54.73493826)(633.03423584,54.65494436)
\curveto(633.23423988,54.58493841)(633.43423968,54.50993849)(633.63423584,54.42994436)
\curveto(634.32423879,54.13993886)(635.0192381,53.86993913)(635.71923584,53.61994436)
\curveto(636.4192367,53.36993963)(637.114236,53.0999399)(637.80423584,52.80994436)
\lineto(637.95423584,52.74994436)
\curveto(638.0142351,52.73994026)(638.07423504,52.72494027)(638.13423584,52.70494436)
\curveto(638.50423461,52.54494045)(638.86923425,52.37494062)(639.22923584,52.19494436)
\curveto(639.59923352,52.01494098)(639.88423323,51.76494123)(640.08423584,51.44494436)
\curveto(640.14423297,51.33494166)(640.18923293,51.22494177)(640.21923584,51.11494436)
\curveto(640.25923286,51.00494199)(640.29423282,50.87994212)(640.32423584,50.73994436)
\curveto(640.34423277,50.68994231)(640.34923277,50.63494236)(640.33923584,50.57494436)
\curveto(640.32923279,50.52494247)(640.32923279,50.46994253)(640.33923584,50.40994436)
\curveto(640.35923276,50.32994267)(640.35923276,50.24994275)(640.33923584,50.16994436)
\curveto(640.32923279,50.12994287)(640.32423279,50.07994292)(640.32423584,50.01994436)
\lineto(640.26423584,49.77994436)
\curveto(640.24423287,49.70994329)(640.20423291,49.65494334)(640.14423584,49.61494436)
\curveto(640.08423303,49.56494343)(640.00923311,49.53494346)(639.91923584,49.52494436)
\lineto(639.64923584,49.52494436)
\lineto(639.43923584,49.52494436)
\curveto(639.37923374,49.53494346)(639.32923379,49.55494344)(639.28923584,49.58494436)
\curveto(639.17923394,49.65494334)(639.14923397,49.77494322)(639.19923584,49.94494436)
\curveto(639.2192339,50.05494294)(639.22923389,50.17494282)(639.22923584,50.30494436)
\curveto(639.22923389,50.43494256)(639.20923391,50.54994245)(639.16923584,50.64994436)
\curveto(639.119234,50.7999422)(639.04423407,50.91994208)(638.94423584,51.00994436)
\curveto(638.84423427,51.10994189)(638.72923439,51.1949418)(638.59923584,51.26494436)
\curveto(638.47923464,51.33494166)(638.34923477,51.3949416)(638.20923584,51.44494436)
\lineto(637.78923584,51.62494436)
\curveto(637.69923542,51.66494133)(637.58923553,51.70494129)(637.45923584,51.74494436)
\curveto(637.32923579,51.7949412)(637.19423592,51.7999412)(637.05423584,51.75994436)
\curveto(636.89423622,51.70994129)(636.74423637,51.65494134)(636.60423584,51.59494436)
\curveto(636.46423665,51.54494145)(636.32423679,51.48994151)(636.18423584,51.42994436)
\curveto(635.97423714,51.33994166)(635.76423735,51.25494174)(635.55423584,51.17494436)
\curveto(635.34423777,51.0949419)(635.13923798,51.01494198)(634.93923584,50.93494436)
\curveto(634.79923832,50.87494212)(634.66423845,50.81994218)(634.53423584,50.76994436)
\curveto(634.40423871,50.71994228)(634.26923885,50.66994233)(634.12923584,50.61994436)
\lineto(632.80923584,50.07994436)
\curveto(632.36924075,49.90994309)(631.92924119,49.73494326)(631.48923584,49.55494436)
\curveto(631.25924186,49.45494354)(631.03924208,49.36494363)(630.82923584,49.28494436)
\curveto(630.60924251,49.20494379)(630.38924273,49.11994388)(630.16923584,49.02994436)
\curveto(630.10924301,49.00994399)(630.02924309,48.97994402)(629.92923584,48.93994436)
\curveto(629.8192433,48.8999441)(629.72924339,48.90494409)(629.65923584,48.95494436)
\curveto(629.60924351,48.98494401)(629.57424354,49.04494395)(629.55423584,49.13494436)
\curveto(629.54424357,49.15494384)(629.54424357,49.17494382)(629.55423584,49.19494436)
\curveto(629.55424356,49.22494377)(629.54924357,49.24994375)(629.53923584,49.26994436)
}
}
{
\newrgbcolor{curcolor}{0 0 0}
\pscustom[linestyle=none,fillstyle=solid,fillcolor=curcolor]
{
}
}
{
\newrgbcolor{curcolor}{0 0 0}
\pscustom[linestyle=none,fillstyle=solid,fillcolor=curcolor]
{
\newpath
\moveto(626.65923584,64.24510061)
\curveto(626.64924647,64.93509597)(626.76924635,65.53509537)(627.01923584,66.04510061)
\curveto(627.26924585,66.56509434)(627.60424551,66.96009395)(628.02423584,67.23010061)
\curveto(628.10424501,67.28009363)(628.19424492,67.32509358)(628.29423584,67.36510061)
\curveto(628.38424473,67.4050935)(628.47924464,67.45009346)(628.57923584,67.50010061)
\curveto(628.67924444,67.54009337)(628.77924434,67.57009334)(628.87923584,67.59010061)
\curveto(628.97924414,67.6100933)(629.08424403,67.63009328)(629.19423584,67.65010061)
\curveto(629.24424387,67.67009324)(629.28924383,67.67509323)(629.32923584,67.66510061)
\curveto(629.36924375,67.65509325)(629.4142437,67.66009325)(629.46423584,67.68010061)
\curveto(629.5142436,67.69009322)(629.59924352,67.69509321)(629.71923584,67.69510061)
\curveto(629.82924329,67.69509321)(629.9142432,67.69009322)(629.97423584,67.68010061)
\curveto(630.03424308,67.66009325)(630.09424302,67.65009326)(630.15423584,67.65010061)
\curveto(630.2142429,67.66009325)(630.27424284,67.65509325)(630.33423584,67.63510061)
\curveto(630.47424264,67.59509331)(630.60924251,67.56009335)(630.73923584,67.53010061)
\curveto(630.86924225,67.50009341)(630.99424212,67.46009345)(631.11423584,67.41010061)
\curveto(631.25424186,67.35009356)(631.37924174,67.28009363)(631.48923584,67.20010061)
\curveto(631.59924152,67.13009378)(631.70924141,67.05509385)(631.81923584,66.97510061)
\lineto(631.87923584,66.91510061)
\curveto(631.89924122,66.905094)(631.9192412,66.89009402)(631.93923584,66.87010061)
\curveto(632.09924102,66.75009416)(632.24424087,66.61509429)(632.37423584,66.46510061)
\curveto(632.50424061,66.31509459)(632.62924049,66.15509475)(632.74923584,65.98510061)
\curveto(632.96924015,65.67509523)(633.17423994,65.38009553)(633.36423584,65.10010061)
\curveto(633.50423961,64.87009604)(633.63923948,64.64009627)(633.76923584,64.41010061)
\curveto(633.89923922,64.19009672)(634.03423908,63.97009694)(634.17423584,63.75010061)
\curveto(634.34423877,63.50009741)(634.52423859,63.26009765)(634.71423584,63.03010061)
\curveto(634.90423821,62.8100981)(635.12923799,62.62009829)(635.38923584,62.46010061)
\curveto(635.44923767,62.42009849)(635.50923761,62.38509852)(635.56923584,62.35510061)
\curveto(635.6192375,62.32509858)(635.68423743,62.29509861)(635.76423584,62.26510061)
\curveto(635.83423728,62.24509866)(635.89423722,62.24009867)(635.94423584,62.25010061)
\curveto(636.0142371,62.27009864)(636.06923705,62.3050986)(636.10923584,62.35510061)
\curveto(636.13923698,62.4050985)(636.15923696,62.46509844)(636.16923584,62.53510061)
\lineto(636.16923584,62.77510061)
\lineto(636.16923584,63.52510061)
\lineto(636.16923584,66.33010061)
\lineto(636.16923584,66.99010061)
\curveto(636.16923695,67.08009383)(636.17423694,67.16509374)(636.18423584,67.24510061)
\curveto(636.18423693,67.32509358)(636.20423691,67.39009352)(636.24423584,67.44010061)
\curveto(636.28423683,67.49009342)(636.35923676,67.53009338)(636.46923584,67.56010061)
\curveto(636.56923655,67.60009331)(636.66923645,67.6100933)(636.76923584,67.59010061)
\lineto(636.90423584,67.59010061)
\curveto(636.97423614,67.57009334)(637.03423608,67.55009336)(637.08423584,67.53010061)
\curveto(637.13423598,67.5100934)(637.17423594,67.47509343)(637.20423584,67.42510061)
\curveto(637.24423587,67.37509353)(637.26423585,67.3050936)(637.26423584,67.21510061)
\lineto(637.26423584,66.94510061)
\lineto(637.26423584,66.04510061)
\lineto(637.26423584,62.53510061)
\lineto(637.26423584,61.47010061)
\curveto(637.26423585,61.39009952)(637.26923585,61.30009961)(637.27923584,61.20010061)
\curveto(637.27923584,61.10009981)(637.26923585,61.01509989)(637.24923584,60.94510061)
\curveto(637.17923594,60.73510017)(636.99923612,60.67010024)(636.70923584,60.75010061)
\curveto(636.66923645,60.76010015)(636.63423648,60.76010015)(636.60423584,60.75010061)
\curveto(636.56423655,60.75010016)(636.5192366,60.76010015)(636.46923584,60.78010061)
\curveto(636.38923673,60.80010011)(636.30423681,60.82010009)(636.21423584,60.84010061)
\curveto(636.12423699,60.86010005)(636.03923708,60.88510002)(635.95923584,60.91510061)
\curveto(635.46923765,61.07509983)(635.05423806,61.27509963)(634.71423584,61.51510061)
\curveto(634.46423865,61.69509921)(634.23923888,61.90009901)(634.03923584,62.13010061)
\curveto(633.82923929,62.36009855)(633.63423948,62.60009831)(633.45423584,62.85010061)
\curveto(633.27423984,63.1100978)(633.10424001,63.37509753)(632.94423584,63.64510061)
\curveto(632.77424034,63.92509698)(632.59924052,64.19509671)(632.41923584,64.45510061)
\curveto(632.33924078,64.56509634)(632.26424085,64.67009624)(632.19423584,64.77010061)
\curveto(632.12424099,64.88009603)(632.04924107,64.99009592)(631.96923584,65.10010061)
\curveto(631.93924118,65.14009577)(631.90924121,65.17509573)(631.87923584,65.20510061)
\curveto(631.83924128,65.24509566)(631.80924131,65.28509562)(631.78923584,65.32510061)
\curveto(631.67924144,65.46509544)(631.55424156,65.59009532)(631.41423584,65.70010061)
\curveto(631.38424173,65.72009519)(631.35924176,65.74509516)(631.33923584,65.77510061)
\curveto(631.30924181,65.8050951)(631.27924184,65.83009508)(631.24923584,65.85010061)
\curveto(631.14924197,65.93009498)(631.04924207,65.99509491)(630.94923584,66.04510061)
\curveto(630.84924227,66.1050948)(630.73924238,66.16009475)(630.61923584,66.21010061)
\curveto(630.54924257,66.24009467)(630.47424264,66.26009465)(630.39423584,66.27010061)
\lineto(630.15423584,66.33010061)
\lineto(630.06423584,66.33010061)
\curveto(630.03424308,66.34009457)(630.00424311,66.34509456)(629.97423584,66.34510061)
\curveto(629.90424321,66.36509454)(629.80924331,66.37009454)(629.68923584,66.36010061)
\curveto(629.55924356,66.36009455)(629.45924366,66.35009456)(629.38923584,66.33010061)
\curveto(629.30924381,66.3100946)(629.23424388,66.29009462)(629.16423584,66.27010061)
\curveto(629.08424403,66.26009465)(629.00424411,66.24009467)(628.92423584,66.21010061)
\curveto(628.68424443,66.10009481)(628.48424463,65.95009496)(628.32423584,65.76010061)
\curveto(628.15424496,65.58009533)(628.0142451,65.36009555)(627.90423584,65.10010061)
\curveto(627.88424523,65.03009588)(627.86924525,64.96009595)(627.85923584,64.89010061)
\curveto(627.83924528,64.82009609)(627.8192453,64.74509616)(627.79923584,64.66510061)
\curveto(627.77924534,64.58509632)(627.76924535,64.47509643)(627.76923584,64.33510061)
\curveto(627.76924535,64.2050967)(627.77924534,64.10009681)(627.79923584,64.02010061)
\curveto(627.80924531,63.96009695)(627.8142453,63.905097)(627.81423584,63.85510061)
\curveto(627.8142453,63.8050971)(627.82424529,63.75509715)(627.84423584,63.70510061)
\curveto(627.88424523,63.6050973)(627.92424519,63.5100974)(627.96423584,63.42010061)
\curveto(628.00424511,63.34009757)(628.04924507,63.26009765)(628.09923584,63.18010061)
\curveto(628.119245,63.15009776)(628.14424497,63.12009779)(628.17423584,63.09010061)
\curveto(628.20424491,63.07009784)(628.22924489,63.04509786)(628.24923584,63.01510061)
\lineto(628.32423584,62.94010061)
\curveto(628.34424477,62.910098)(628.36424475,62.88509802)(628.38423584,62.86510061)
\lineto(628.59423584,62.71510061)
\curveto(628.65424446,62.67509823)(628.7192444,62.63009828)(628.78923584,62.58010061)
\curveto(628.87924424,62.52009839)(628.98424413,62.47009844)(629.10423584,62.43010061)
\curveto(629.2142439,62.40009851)(629.32424379,62.36509854)(629.43423584,62.32510061)
\curveto(629.54424357,62.28509862)(629.68924343,62.26009865)(629.86923584,62.25010061)
\curveto(630.03924308,62.24009867)(630.16424295,62.2100987)(630.24423584,62.16010061)
\curveto(630.32424279,62.1100988)(630.36924275,62.03509887)(630.37923584,61.93510061)
\curveto(630.38924273,61.83509907)(630.39424272,61.72509918)(630.39423584,61.60510061)
\curveto(630.39424272,61.56509934)(630.39924272,61.52509938)(630.40923584,61.48510061)
\curveto(630.40924271,61.44509946)(630.40424271,61.4100995)(630.39423584,61.38010061)
\curveto(630.37424274,61.33009958)(630.36424275,61.28009963)(630.36423584,61.23010061)
\curveto(630.36424275,61.19009972)(630.35424276,61.15009976)(630.33423584,61.11010061)
\curveto(630.27424284,61.02009989)(630.13924298,60.97509993)(629.92923584,60.97510061)
\lineto(629.80923584,60.97510061)
\curveto(629.74924337,60.98509992)(629.68924343,60.99009992)(629.62923584,60.99010061)
\curveto(629.55924356,61.00009991)(629.49424362,61.0100999)(629.43423584,61.02010061)
\curveto(629.32424379,61.04009987)(629.22424389,61.06009985)(629.13423584,61.08010061)
\curveto(629.03424408,61.10009981)(628.93924418,61.13009978)(628.84923584,61.17010061)
\curveto(628.77924434,61.19009972)(628.7192444,61.2100997)(628.66923584,61.23010061)
\lineto(628.48923584,61.29010061)
\curveto(628.22924489,61.4100995)(627.98424513,61.56509934)(627.75423584,61.75510061)
\curveto(627.52424559,61.95509895)(627.33924578,62.17009874)(627.19923584,62.40010061)
\curveto(627.119246,62.5100984)(627.05424606,62.62509828)(627.00423584,62.74510061)
\lineto(626.85423584,63.13510061)
\curveto(626.80424631,63.24509766)(626.77424634,63.36009755)(626.76423584,63.48010061)
\curveto(626.74424637,63.60009731)(626.7192464,63.72509718)(626.68923584,63.85510061)
\curveto(626.68924643,63.92509698)(626.68924643,63.99009692)(626.68923584,64.05010061)
\curveto(626.67924644,64.1100968)(626.66924645,64.17509673)(626.65923584,64.24510061)
}
}
{
\newrgbcolor{curcolor}{0 0 0}
\pscustom[linestyle=none,fillstyle=solid,fillcolor=curcolor]
{
\newpath
\moveto(631.66923584,76.28470998)
\curveto(631.74924137,76.28470235)(631.82924129,76.28970234)(631.90923584,76.29970998)
\curveto(631.98924113,76.30970232)(632.06424105,76.30470233)(632.13423584,76.28470998)
\curveto(632.17424094,76.26470237)(632.2192409,76.25970237)(632.26923584,76.26970998)
\curveto(632.30924081,76.27970235)(632.34924077,76.27970235)(632.38923584,76.26970998)
\lineto(632.53923584,76.26970998)
\curveto(632.62924049,76.25970237)(632.7192404,76.25470238)(632.80923584,76.25470998)
\curveto(632.88924023,76.25470238)(632.96924015,76.24970238)(633.04923584,76.23970998)
\lineto(633.28923584,76.20970998)
\curveto(633.35923976,76.19970243)(633.43423968,76.18970244)(633.51423584,76.17970998)
\curveto(633.55423956,76.16970246)(633.59423952,76.16470247)(633.63423584,76.16470998)
\curveto(633.67423944,76.16470247)(633.7192394,76.15970247)(633.76923584,76.14970998)
\curveto(633.90923921,76.10970252)(634.04923907,76.07970255)(634.18923584,76.05970998)
\curveto(634.32923879,76.04970258)(634.46423865,76.01970261)(634.59423584,75.96970998)
\curveto(634.76423835,75.91970271)(634.92923819,75.86470277)(635.08923584,75.80470998)
\curveto(635.24923787,75.75470288)(635.40423771,75.69470294)(635.55423584,75.62470998)
\curveto(635.6142375,75.60470303)(635.67423744,75.57470306)(635.73423584,75.53470998)
\lineto(635.88423584,75.44470998)
\curveto(636.20423691,75.24470339)(636.46923665,75.0297036)(636.67923584,74.79970998)
\curveto(636.88923623,74.56970406)(637.06923605,74.27470436)(637.21923584,73.91470998)
\curveto(637.26923585,73.79470484)(637.30423581,73.66470497)(637.32423584,73.52470998)
\curveto(637.34423577,73.39470524)(637.36923575,73.25970537)(637.39923584,73.11970998)
\curveto(637.40923571,73.05970557)(637.4142357,72.99970563)(637.41423584,72.93970998)
\curveto(637.4142357,72.87970575)(637.4192357,72.81470582)(637.42923584,72.74470998)
\curveto(637.43923568,72.71470592)(637.43923568,72.66470597)(637.42923584,72.59470998)
\lineto(637.42923584,72.44470998)
\lineto(637.42923584,72.29470998)
\curveto(637.40923571,72.21470642)(637.39423572,72.1297065)(637.38423584,72.03970998)
\curveto(637.38423573,71.95970667)(637.37423574,71.88470675)(637.35423584,71.81470998)
\curveto(637.34423577,71.77470686)(637.33923578,71.73970689)(637.33923584,71.70970998)
\curveto(637.34923577,71.68970694)(637.34423577,71.66470697)(637.32423584,71.63470998)
\lineto(637.26423584,71.36470998)
\curveto(637.23423588,71.27470736)(637.20423591,71.18970744)(637.17423584,71.10970998)
\curveto(636.93423618,70.5297081)(636.56423655,70.09470854)(636.06423584,69.80470998)
\curveto(635.93423718,69.72470891)(635.79923732,69.65970897)(635.65923584,69.60970998)
\curveto(635.5192376,69.56970906)(635.36923775,69.52470911)(635.20923584,69.47470998)
\curveto(635.12923799,69.45470918)(635.04923807,69.44970918)(634.96923584,69.45970998)
\curveto(634.88923823,69.47970915)(634.83423828,69.51470912)(634.80423584,69.56470998)
\curveto(634.78423833,69.59470904)(634.76923835,69.64970898)(634.75923584,69.72970998)
\curveto(634.73923838,69.80970882)(634.72923839,69.89470874)(634.72923584,69.98470998)
\curveto(634.7192384,70.07470856)(634.7192384,70.15970847)(634.72923584,70.23970998)
\curveto(634.73923838,70.3297083)(634.74923837,70.39970823)(634.75923584,70.44970998)
\curveto(634.76923835,70.46970816)(634.78423833,70.49470814)(634.80423584,70.52470998)
\curveto(634.82423829,70.56470807)(634.84423827,70.59470804)(634.86423584,70.61470998)
\curveto(634.94423817,70.67470796)(635.03923808,70.71970791)(635.14923584,70.74970998)
\curveto(635.25923786,70.78970784)(635.35923776,70.8347078)(635.44923584,70.88470998)
\curveto(635.83923728,71.1347075)(636.10923701,71.50470713)(636.25923584,71.99470998)
\curveto(636.27923684,72.06470657)(636.29423682,72.1347065)(636.30423584,72.20470998)
\curveto(636.30423681,72.28470635)(636.3142368,72.36470627)(636.33423584,72.44470998)
\curveto(636.34423677,72.48470615)(636.34923677,72.53970609)(636.34923584,72.60970998)
\curveto(636.34923677,72.68970594)(636.34423677,72.74470589)(636.33423584,72.77470998)
\curveto(636.32423679,72.80470583)(636.3192368,72.8347058)(636.31923584,72.86470998)
\lineto(636.31923584,72.96970998)
\curveto(636.29923682,73.04970558)(636.27923684,73.12470551)(636.25923584,73.19470998)
\curveto(636.23923688,73.27470536)(636.2142369,73.34970528)(636.18423584,73.41970998)
\curveto(636.03423708,73.76970486)(635.8192373,74.03970459)(635.53923584,74.22970998)
\curveto(635.25923786,74.41970421)(634.93423818,74.57470406)(634.56423584,74.69470998)
\curveto(634.48423863,74.72470391)(634.40923871,74.74470389)(634.33923584,74.75470998)
\curveto(634.26923885,74.77470386)(634.19423892,74.79470384)(634.11423584,74.81470998)
\curveto(634.02423909,74.8347038)(633.92923919,74.84970378)(633.82923584,74.85970998)
\curveto(633.7192394,74.87970375)(633.6142395,74.89970373)(633.51423584,74.91970998)
\curveto(633.46423965,74.9297037)(633.4142397,74.9347037)(633.36423584,74.93470998)
\curveto(633.30423981,74.94470369)(633.24923987,74.94970368)(633.19923584,74.94970998)
\curveto(633.13923998,74.96970366)(633.06424005,74.97970365)(632.97423584,74.97970998)
\curveto(632.87424024,74.97970365)(632.79424032,74.96970366)(632.73423584,74.94970998)
\curveto(632.64424047,74.91970371)(632.60424051,74.86970376)(632.61423584,74.79970998)
\curveto(632.62424049,74.73970389)(632.65424046,74.68470395)(632.70423584,74.63470998)
\curveto(632.75424036,74.55470408)(632.8142403,74.48470415)(632.88423584,74.42470998)
\curveto(632.95424016,74.37470426)(633.0142401,74.30970432)(633.06423584,74.22970998)
\curveto(633.17423994,74.06970456)(633.27423984,73.90470473)(633.36423584,73.73470998)
\curveto(633.44423967,73.56470507)(633.5142396,73.36970526)(633.57423584,73.14970998)
\curveto(633.60423951,73.04970558)(633.6192395,72.94970568)(633.61923584,72.84970998)
\curveto(633.6192395,72.75970587)(633.62923949,72.65970597)(633.64923584,72.54970998)
\lineto(633.64923584,72.39970998)
\curveto(633.62923949,72.34970628)(633.62423949,72.29970633)(633.63423584,72.24970998)
\curveto(633.64423947,72.20970642)(633.64423947,72.16970646)(633.63423584,72.12970998)
\curveto(633.62423949,72.09970653)(633.6192395,72.05470658)(633.61923584,71.99470998)
\curveto(633.60923951,71.9347067)(633.59923952,71.86970676)(633.58923584,71.79970998)
\lineto(633.55923584,71.61970998)
\curveto(633.43923968,71.16970746)(633.27423984,70.78970784)(633.06423584,70.47970998)
\curveto(632.87424024,70.20970842)(632.64424047,69.97970865)(632.37423584,69.78970998)
\curveto(632.09424102,69.60970902)(631.77924134,69.46470917)(631.42923584,69.35470998)
\lineto(631.21923584,69.29470998)
\curveto(631.13924198,69.28470935)(631.05924206,69.26970936)(630.97923584,69.24970998)
\curveto(630.94924217,69.23970939)(630.9192422,69.2347094)(630.88923584,69.23470998)
\curveto(630.85924226,69.2347094)(630.82924229,69.2297094)(630.79923584,69.21970998)
\curveto(630.73924238,69.20970942)(630.67924244,69.20470943)(630.61923584,69.20470998)
\curveto(630.54924257,69.20470943)(630.48924263,69.19470944)(630.43923584,69.17470998)
\lineto(630.25923584,69.17470998)
\curveto(630.20924291,69.16470947)(630.13924298,69.15970947)(630.04923584,69.15970998)
\curveto(629.95924316,69.15970947)(629.88924323,69.16970946)(629.83923584,69.18970998)
\lineto(629.67423584,69.18970998)
\curveto(629.59424352,69.20970942)(629.5192436,69.21970941)(629.44923584,69.21970998)
\curveto(629.37924374,69.2297094)(629.30924381,69.24470939)(629.23923584,69.26470998)
\curveto(629.03924408,69.32470931)(628.84924427,69.38470925)(628.66923584,69.44470998)
\curveto(628.48924463,69.51470912)(628.3192448,69.60470903)(628.15923584,69.71470998)
\curveto(628.08924503,69.75470888)(628.02424509,69.79470884)(627.96423584,69.83470998)
\lineto(627.78423584,69.98470998)
\curveto(627.77424534,70.00470863)(627.75924536,70.02470861)(627.73923584,70.04470998)
\curveto(627.60924551,70.1347085)(627.49924562,70.24470839)(627.40923584,70.37470998)
\curveto(627.20924591,70.634708)(627.05424606,70.89970773)(626.94423584,71.16970998)
\curveto(626.90424621,71.24970738)(626.87424624,71.3297073)(626.85423584,71.40970998)
\curveto(626.82424629,71.49970713)(626.79924632,71.58970704)(626.77923584,71.67970998)
\curveto(626.74924637,71.77970685)(626.72924639,71.87970675)(626.71923584,71.97970998)
\curveto(626.70924641,72.07970655)(626.69424642,72.18470645)(626.67423584,72.29470998)
\curveto(626.66424645,72.32470631)(626.66424645,72.36470627)(626.67423584,72.41470998)
\curveto(626.68424643,72.47470616)(626.67924644,72.51470612)(626.65923584,72.53470998)
\curveto(626.63924648,73.25470538)(626.75424636,73.85470478)(627.00423584,74.33470998)
\curveto(627.25424586,74.81470382)(627.59424552,75.18970344)(628.02423584,75.45970998)
\curveto(628.16424495,75.54970308)(628.30924481,75.629703)(628.45923584,75.69970998)
\curveto(628.60924451,75.76970286)(628.76924435,75.83970279)(628.93923584,75.90970998)
\curveto(629.07924404,75.95970267)(629.22924389,75.99970263)(629.38923584,76.02970998)
\curveto(629.54924357,76.05970257)(629.70924341,76.09470254)(629.86923584,76.13470998)
\curveto(629.9192432,76.15470248)(629.97424314,76.16470247)(630.03423584,76.16470998)
\curveto(630.08424303,76.16470247)(630.13424298,76.16970246)(630.18423584,76.17970998)
\curveto(630.24424287,76.19970243)(630.30924281,76.20970242)(630.37923584,76.20970998)
\curveto(630.43924268,76.20970242)(630.49424262,76.21970241)(630.54423584,76.23970998)
\lineto(630.70923584,76.23970998)
\curveto(630.75924236,76.25970237)(630.80924231,76.26470237)(630.85923584,76.25470998)
\curveto(630.90924221,76.24470239)(630.95924216,76.24970238)(631.00923584,76.26970998)
\curveto(631.02924209,76.26970236)(631.05424206,76.26470237)(631.08423584,76.25470998)
\curveto(631.114242,76.25470238)(631.13924198,76.25970237)(631.15923584,76.26970998)
\curveto(631.18924193,76.27970235)(631.22424189,76.27970235)(631.26423584,76.26970998)
\curveto(631.30424181,76.26970236)(631.34424177,76.27470236)(631.38423584,76.28470998)
\curveto(631.42424169,76.29470234)(631.46924165,76.29470234)(631.51923584,76.28470998)
\lineto(631.66923584,76.28470998)
\moveto(630.36423584,74.78470998)
\curveto(630.3142428,74.79470384)(630.25424286,74.79970383)(630.18423584,74.79970998)
\curveto(630.114243,74.79970383)(630.05424306,74.79470384)(630.00423584,74.78470998)
\curveto(629.95424316,74.77470386)(629.87924324,74.76970386)(629.77923584,74.76970998)
\curveto(629.69924342,74.74970388)(629.62424349,74.7297039)(629.55423584,74.70970998)
\curveto(629.48424363,74.69970393)(629.4142437,74.68470395)(629.34423584,74.66470998)
\curveto(628.9142442,74.52470411)(628.57924454,74.3297043)(628.33923584,74.07970998)
\curveto(628.09924502,73.83970479)(627.9192452,73.49470514)(627.79923584,73.04470998)
\curveto(627.77924534,72.95470568)(627.76924535,72.85470578)(627.76923584,72.74470998)
\lineto(627.76923584,72.41470998)
\curveto(627.78924533,72.39470624)(627.79924532,72.35970627)(627.79923584,72.30970998)
\curveto(627.78924533,72.25970637)(627.78924533,72.21470642)(627.79923584,72.17470998)
\curveto(627.8192453,72.09470654)(627.83924528,72.01970661)(627.85923584,71.94970998)
\lineto(627.91923584,71.73970998)
\curveto(628.04924507,71.44970718)(628.22924489,71.21970741)(628.45923584,71.04970998)
\curveto(628.67924444,70.87970775)(628.93924418,70.74470789)(629.23923584,70.64470998)
\curveto(629.32924379,70.61470802)(629.42424369,70.58970804)(629.52423584,70.56970998)
\curveto(629.6142435,70.55970807)(629.70924341,70.54470809)(629.80923584,70.52470998)
\lineto(629.94423584,70.52470998)
\curveto(630.05424306,70.49470814)(630.19424292,70.48470815)(630.36423584,70.49470998)
\curveto(630.52424259,70.51470812)(630.65424246,70.5347081)(630.75423584,70.55470998)
\curveto(630.8142423,70.57470806)(630.87424224,70.58970804)(630.93423584,70.59970998)
\curveto(630.98424213,70.60970802)(631.03424208,70.62470801)(631.08423584,70.64470998)
\curveto(631.28424183,70.72470791)(631.47424164,70.81970781)(631.65423584,70.92970998)
\curveto(631.83424128,71.04970758)(631.97924114,71.18970744)(632.08923584,71.34970998)
\curveto(632.13924098,71.39970723)(632.17924094,71.45470718)(632.20923584,71.51470998)
\curveto(632.23924088,71.57470706)(632.27424084,71.634707)(632.31423584,71.69470998)
\curveto(632.39424072,71.84470679)(632.45924066,72.0297066)(632.50923584,72.24970998)
\curveto(632.52924059,72.29970633)(632.53424058,72.33970629)(632.52423584,72.36970998)
\curveto(632.5142406,72.40970622)(632.5192406,72.45470618)(632.53923584,72.50470998)
\curveto(632.54924057,72.54470609)(632.55424056,72.59970603)(632.55423584,72.66970998)
\curveto(632.55424056,72.73970589)(632.54924057,72.79970583)(632.53923584,72.84970998)
\curveto(632.5192406,72.94970568)(632.50424061,73.04470559)(632.49423584,73.13470998)
\curveto(632.47424064,73.22470541)(632.44424067,73.31470532)(632.40423584,73.40470998)
\curveto(632.18424093,73.94470469)(631.78924133,74.33970429)(631.21923584,74.58970998)
\curveto(631.119242,74.63970399)(631.0192421,74.67470396)(630.91923584,74.69470998)
\curveto(630.80924231,74.71470392)(630.69924242,74.73970389)(630.58923584,74.76970998)
\curveto(630.48924263,74.76970386)(630.4142427,74.77470386)(630.36423584,74.78470998)
}
}
{
\newrgbcolor{curcolor}{0 0 0}
\pscustom[linestyle=none,fillstyle=solid,fillcolor=curcolor]
{
\newpath
\moveto(635.62923584,78.63431936)
\lineto(635.62923584,79.26431936)
\lineto(635.62923584,79.45931936)
\curveto(635.62923749,79.52931683)(635.63923748,79.58931677)(635.65923584,79.63931936)
\curveto(635.69923742,79.70931665)(635.73923738,79.7593166)(635.77923584,79.78931936)
\curveto(635.82923729,79.82931653)(635.89423722,79.84931651)(635.97423584,79.84931936)
\curveto(636.05423706,79.8593165)(636.13923698,79.86431649)(636.22923584,79.86431936)
\lineto(636.94923584,79.86431936)
\curveto(637.42923569,79.86431649)(637.83923528,79.80431655)(638.17923584,79.68431936)
\curveto(638.5192346,79.56431679)(638.79423432,79.36931699)(639.00423584,79.09931936)
\curveto(639.05423406,79.02931733)(639.09923402,78.9593174)(639.13923584,78.88931936)
\curveto(639.18923393,78.82931753)(639.23423388,78.7543176)(639.27423584,78.66431936)
\curveto(639.28423383,78.64431771)(639.29423382,78.61431774)(639.30423584,78.57431936)
\curveto(639.32423379,78.53431782)(639.32923379,78.48931787)(639.31923584,78.43931936)
\curveto(639.28923383,78.34931801)(639.2142339,78.29431806)(639.09423584,78.27431936)
\curveto(638.98423413,78.2543181)(638.88923423,78.26931809)(638.80923584,78.31931936)
\curveto(638.73923438,78.34931801)(638.67423444,78.39431796)(638.61423584,78.45431936)
\curveto(638.56423455,78.52431783)(638.5142346,78.58931777)(638.46423584,78.64931936)
\curveto(638.4142347,78.71931764)(638.33923478,78.77931758)(638.23923584,78.82931936)
\curveto(638.14923497,78.88931747)(638.05923506,78.93931742)(637.96923584,78.97931936)
\curveto(637.93923518,78.99931736)(637.87923524,79.02431733)(637.78923584,79.05431936)
\curveto(637.70923541,79.08431727)(637.63923548,79.08931727)(637.57923584,79.06931936)
\curveto(637.43923568,79.03931732)(637.34923577,78.97931738)(637.30923584,78.88931936)
\curveto(637.27923584,78.80931755)(637.26423585,78.71931764)(637.26423584,78.61931936)
\curveto(637.26423585,78.51931784)(637.23923588,78.43431792)(637.18923584,78.36431936)
\curveto(637.119236,78.27431808)(636.97923614,78.22931813)(636.76923584,78.22931936)
\lineto(636.21423584,78.22931936)
\lineto(635.98923584,78.22931936)
\curveto(635.90923721,78.23931812)(635.84423727,78.2593181)(635.79423584,78.28931936)
\curveto(635.7142374,78.34931801)(635.66923745,78.41931794)(635.65923584,78.49931936)
\curveto(635.64923747,78.51931784)(635.64423747,78.53931782)(635.64423584,78.55931936)
\curveto(635.64423747,78.58931777)(635.63923748,78.61431774)(635.62923584,78.63431936)
}
}
{
\newrgbcolor{curcolor}{0 0 0}
\pscustom[linestyle=none,fillstyle=solid,fillcolor=curcolor]
{
}
}
{
\newrgbcolor{curcolor}{0 0 0}
\pscustom[linestyle=none,fillstyle=solid,fillcolor=curcolor]
{
\newpath
\moveto(626.65923584,89.26463186)
\curveto(626.64924647,89.95462722)(626.76924635,90.55462662)(627.01923584,91.06463186)
\curveto(627.26924585,91.58462559)(627.60424551,91.9796252)(628.02423584,92.24963186)
\curveto(628.10424501,92.29962488)(628.19424492,92.34462483)(628.29423584,92.38463186)
\curveto(628.38424473,92.42462475)(628.47924464,92.46962471)(628.57923584,92.51963186)
\curveto(628.67924444,92.55962462)(628.77924434,92.58962459)(628.87923584,92.60963186)
\curveto(628.97924414,92.62962455)(629.08424403,92.64962453)(629.19423584,92.66963186)
\curveto(629.24424387,92.68962449)(629.28924383,92.69462448)(629.32923584,92.68463186)
\curveto(629.36924375,92.6746245)(629.4142437,92.6796245)(629.46423584,92.69963186)
\curveto(629.5142436,92.70962447)(629.59924352,92.71462446)(629.71923584,92.71463186)
\curveto(629.82924329,92.71462446)(629.9142432,92.70962447)(629.97423584,92.69963186)
\curveto(630.03424308,92.6796245)(630.09424302,92.66962451)(630.15423584,92.66963186)
\curveto(630.2142429,92.6796245)(630.27424284,92.6746245)(630.33423584,92.65463186)
\curveto(630.47424264,92.61462456)(630.60924251,92.5796246)(630.73923584,92.54963186)
\curveto(630.86924225,92.51962466)(630.99424212,92.4796247)(631.11423584,92.42963186)
\curveto(631.25424186,92.36962481)(631.37924174,92.29962488)(631.48923584,92.21963186)
\curveto(631.59924152,92.14962503)(631.70924141,92.0746251)(631.81923584,91.99463186)
\lineto(631.87923584,91.93463186)
\curveto(631.89924122,91.92462525)(631.9192412,91.90962527)(631.93923584,91.88963186)
\curveto(632.09924102,91.76962541)(632.24424087,91.63462554)(632.37423584,91.48463186)
\curveto(632.50424061,91.33462584)(632.62924049,91.174626)(632.74923584,91.00463186)
\curveto(632.96924015,90.69462648)(633.17423994,90.39962678)(633.36423584,90.11963186)
\curveto(633.50423961,89.88962729)(633.63923948,89.65962752)(633.76923584,89.42963186)
\curveto(633.89923922,89.20962797)(634.03423908,88.98962819)(634.17423584,88.76963186)
\curveto(634.34423877,88.51962866)(634.52423859,88.2796289)(634.71423584,88.04963186)
\curveto(634.90423821,87.82962935)(635.12923799,87.63962954)(635.38923584,87.47963186)
\curveto(635.44923767,87.43962974)(635.50923761,87.40462977)(635.56923584,87.37463186)
\curveto(635.6192375,87.34462983)(635.68423743,87.31462986)(635.76423584,87.28463186)
\curveto(635.83423728,87.26462991)(635.89423722,87.25962992)(635.94423584,87.26963186)
\curveto(636.0142371,87.28962989)(636.06923705,87.32462985)(636.10923584,87.37463186)
\curveto(636.13923698,87.42462975)(636.15923696,87.48462969)(636.16923584,87.55463186)
\lineto(636.16923584,87.79463186)
\lineto(636.16923584,88.54463186)
\lineto(636.16923584,91.34963186)
\lineto(636.16923584,92.00963186)
\curveto(636.16923695,92.09962508)(636.17423694,92.18462499)(636.18423584,92.26463186)
\curveto(636.18423693,92.34462483)(636.20423691,92.40962477)(636.24423584,92.45963186)
\curveto(636.28423683,92.50962467)(636.35923676,92.54962463)(636.46923584,92.57963186)
\curveto(636.56923655,92.61962456)(636.66923645,92.62962455)(636.76923584,92.60963186)
\lineto(636.90423584,92.60963186)
\curveto(636.97423614,92.58962459)(637.03423608,92.56962461)(637.08423584,92.54963186)
\curveto(637.13423598,92.52962465)(637.17423594,92.49462468)(637.20423584,92.44463186)
\curveto(637.24423587,92.39462478)(637.26423585,92.32462485)(637.26423584,92.23463186)
\lineto(637.26423584,91.96463186)
\lineto(637.26423584,91.06463186)
\lineto(637.26423584,87.55463186)
\lineto(637.26423584,86.48963186)
\curveto(637.26423585,86.40963077)(637.26923585,86.31963086)(637.27923584,86.21963186)
\curveto(637.27923584,86.11963106)(637.26923585,86.03463114)(637.24923584,85.96463186)
\curveto(637.17923594,85.75463142)(636.99923612,85.68963149)(636.70923584,85.76963186)
\curveto(636.66923645,85.7796314)(636.63423648,85.7796314)(636.60423584,85.76963186)
\curveto(636.56423655,85.76963141)(636.5192366,85.7796314)(636.46923584,85.79963186)
\curveto(636.38923673,85.81963136)(636.30423681,85.83963134)(636.21423584,85.85963186)
\curveto(636.12423699,85.8796313)(636.03923708,85.90463127)(635.95923584,85.93463186)
\curveto(635.46923765,86.09463108)(635.05423806,86.29463088)(634.71423584,86.53463186)
\curveto(634.46423865,86.71463046)(634.23923888,86.91963026)(634.03923584,87.14963186)
\curveto(633.82923929,87.3796298)(633.63423948,87.61962956)(633.45423584,87.86963186)
\curveto(633.27423984,88.12962905)(633.10424001,88.39462878)(632.94423584,88.66463186)
\curveto(632.77424034,88.94462823)(632.59924052,89.21462796)(632.41923584,89.47463186)
\curveto(632.33924078,89.58462759)(632.26424085,89.68962749)(632.19423584,89.78963186)
\curveto(632.12424099,89.89962728)(632.04924107,90.00962717)(631.96923584,90.11963186)
\curveto(631.93924118,90.15962702)(631.90924121,90.19462698)(631.87923584,90.22463186)
\curveto(631.83924128,90.26462691)(631.80924131,90.30462687)(631.78923584,90.34463186)
\curveto(631.67924144,90.48462669)(631.55424156,90.60962657)(631.41423584,90.71963186)
\curveto(631.38424173,90.73962644)(631.35924176,90.76462641)(631.33923584,90.79463186)
\curveto(631.30924181,90.82462635)(631.27924184,90.84962633)(631.24923584,90.86963186)
\curveto(631.14924197,90.94962623)(631.04924207,91.01462616)(630.94923584,91.06463186)
\curveto(630.84924227,91.12462605)(630.73924238,91.179626)(630.61923584,91.22963186)
\curveto(630.54924257,91.25962592)(630.47424264,91.2796259)(630.39423584,91.28963186)
\lineto(630.15423584,91.34963186)
\lineto(630.06423584,91.34963186)
\curveto(630.03424308,91.35962582)(630.00424311,91.36462581)(629.97423584,91.36463186)
\curveto(629.90424321,91.38462579)(629.80924331,91.38962579)(629.68923584,91.37963186)
\curveto(629.55924356,91.3796258)(629.45924366,91.36962581)(629.38923584,91.34963186)
\curveto(629.30924381,91.32962585)(629.23424388,91.30962587)(629.16423584,91.28963186)
\curveto(629.08424403,91.2796259)(629.00424411,91.25962592)(628.92423584,91.22963186)
\curveto(628.68424443,91.11962606)(628.48424463,90.96962621)(628.32423584,90.77963186)
\curveto(628.15424496,90.59962658)(628.0142451,90.3796268)(627.90423584,90.11963186)
\curveto(627.88424523,90.04962713)(627.86924525,89.9796272)(627.85923584,89.90963186)
\curveto(627.83924528,89.83962734)(627.8192453,89.76462741)(627.79923584,89.68463186)
\curveto(627.77924534,89.60462757)(627.76924535,89.49462768)(627.76923584,89.35463186)
\curveto(627.76924535,89.22462795)(627.77924534,89.11962806)(627.79923584,89.03963186)
\curveto(627.80924531,88.9796282)(627.8142453,88.92462825)(627.81423584,88.87463186)
\curveto(627.8142453,88.82462835)(627.82424529,88.7746284)(627.84423584,88.72463186)
\curveto(627.88424523,88.62462855)(627.92424519,88.52962865)(627.96423584,88.43963186)
\curveto(628.00424511,88.35962882)(628.04924507,88.2796289)(628.09923584,88.19963186)
\curveto(628.119245,88.16962901)(628.14424497,88.13962904)(628.17423584,88.10963186)
\curveto(628.20424491,88.08962909)(628.22924489,88.06462911)(628.24923584,88.03463186)
\lineto(628.32423584,87.95963186)
\curveto(628.34424477,87.92962925)(628.36424475,87.90462927)(628.38423584,87.88463186)
\lineto(628.59423584,87.73463186)
\curveto(628.65424446,87.69462948)(628.7192444,87.64962953)(628.78923584,87.59963186)
\curveto(628.87924424,87.53962964)(628.98424413,87.48962969)(629.10423584,87.44963186)
\curveto(629.2142439,87.41962976)(629.32424379,87.38462979)(629.43423584,87.34463186)
\curveto(629.54424357,87.30462987)(629.68924343,87.2796299)(629.86923584,87.26963186)
\curveto(630.03924308,87.25962992)(630.16424295,87.22962995)(630.24423584,87.17963186)
\curveto(630.32424279,87.12963005)(630.36924275,87.05463012)(630.37923584,86.95463186)
\curveto(630.38924273,86.85463032)(630.39424272,86.74463043)(630.39423584,86.62463186)
\curveto(630.39424272,86.58463059)(630.39924272,86.54463063)(630.40923584,86.50463186)
\curveto(630.40924271,86.46463071)(630.40424271,86.42963075)(630.39423584,86.39963186)
\curveto(630.37424274,86.34963083)(630.36424275,86.29963088)(630.36423584,86.24963186)
\curveto(630.36424275,86.20963097)(630.35424276,86.16963101)(630.33423584,86.12963186)
\curveto(630.27424284,86.03963114)(630.13924298,85.99463118)(629.92923584,85.99463186)
\lineto(629.80923584,85.99463186)
\curveto(629.74924337,86.00463117)(629.68924343,86.00963117)(629.62923584,86.00963186)
\curveto(629.55924356,86.01963116)(629.49424362,86.02963115)(629.43423584,86.03963186)
\curveto(629.32424379,86.05963112)(629.22424389,86.0796311)(629.13423584,86.09963186)
\curveto(629.03424408,86.11963106)(628.93924418,86.14963103)(628.84923584,86.18963186)
\curveto(628.77924434,86.20963097)(628.7192444,86.22963095)(628.66923584,86.24963186)
\lineto(628.48923584,86.30963186)
\curveto(628.22924489,86.42963075)(627.98424513,86.58463059)(627.75423584,86.77463186)
\curveto(627.52424559,86.9746302)(627.33924578,87.18962999)(627.19923584,87.41963186)
\curveto(627.119246,87.52962965)(627.05424606,87.64462953)(627.00423584,87.76463186)
\lineto(626.85423584,88.15463186)
\curveto(626.80424631,88.26462891)(626.77424634,88.3796288)(626.76423584,88.49963186)
\curveto(626.74424637,88.61962856)(626.7192464,88.74462843)(626.68923584,88.87463186)
\curveto(626.68924643,88.94462823)(626.68924643,89.00962817)(626.68923584,89.06963186)
\curveto(626.67924644,89.12962805)(626.66924645,89.19462798)(626.65923584,89.26463186)
}
}
{
\newrgbcolor{curcolor}{0 0 0}
\pscustom[linestyle=none,fillstyle=solid,fillcolor=curcolor]
{
\newpath
\moveto(632.17923584,101.36424123)
\lineto(632.43423584,101.36424123)
\curveto(632.5142406,101.37423353)(632.58924053,101.36923353)(632.65923584,101.34924123)
\lineto(632.89923584,101.34924123)
\lineto(633.06423584,101.34924123)
\curveto(633.16423995,101.32923357)(633.26923985,101.31923358)(633.37923584,101.31924123)
\curveto(633.47923964,101.31923358)(633.57923954,101.30923359)(633.67923584,101.28924123)
\lineto(633.82923584,101.28924123)
\curveto(633.96923915,101.25923364)(634.10923901,101.23923366)(634.24923584,101.22924123)
\curveto(634.37923874,101.21923368)(634.50923861,101.19423371)(634.63923584,101.15424123)
\curveto(634.7192384,101.13423377)(634.80423831,101.11423379)(634.89423584,101.09424123)
\lineto(635.13423584,101.03424123)
\lineto(635.43423584,100.91424123)
\curveto(635.52423759,100.88423402)(635.6142375,100.84923405)(635.70423584,100.80924123)
\curveto(635.92423719,100.70923419)(636.13923698,100.57423433)(636.34923584,100.40424123)
\curveto(636.55923656,100.24423466)(636.72923639,100.06923483)(636.85923584,99.87924123)
\curveto(636.89923622,99.82923507)(636.93923618,99.76923513)(636.97923584,99.69924123)
\curveto(637.00923611,99.63923526)(637.04423607,99.57923532)(637.08423584,99.51924123)
\curveto(637.13423598,99.43923546)(637.17423594,99.34423556)(637.20423584,99.23424123)
\curveto(637.23423588,99.12423578)(637.26423585,99.01923588)(637.29423584,98.91924123)
\curveto(637.33423578,98.80923609)(637.35923576,98.6992362)(637.36923584,98.58924123)
\curveto(637.37923574,98.47923642)(637.39423572,98.36423654)(637.41423584,98.24424123)
\curveto(637.42423569,98.2042367)(637.42423569,98.15923674)(637.41423584,98.10924123)
\curveto(637.4142357,98.06923683)(637.4192357,98.02923687)(637.42923584,97.98924123)
\curveto(637.43923568,97.94923695)(637.44423567,97.89423701)(637.44423584,97.82424123)
\curveto(637.44423567,97.75423715)(637.43923568,97.7042372)(637.42923584,97.67424123)
\curveto(637.40923571,97.62423728)(637.40423571,97.57923732)(637.41423584,97.53924123)
\curveto(637.42423569,97.4992374)(637.42423569,97.46423744)(637.41423584,97.43424123)
\lineto(637.41423584,97.34424123)
\curveto(637.39423572,97.28423762)(637.37923574,97.21923768)(637.36923584,97.14924123)
\curveto(637.36923575,97.08923781)(637.36423575,97.02423788)(637.35423584,96.95424123)
\curveto(637.30423581,96.78423812)(637.25423586,96.62423828)(637.20423584,96.47424123)
\curveto(637.15423596,96.32423858)(637.08923603,96.17923872)(637.00923584,96.03924123)
\curveto(636.96923615,95.98923891)(636.93923618,95.93423897)(636.91923584,95.87424123)
\curveto(636.88923623,95.82423908)(636.85423626,95.77423913)(636.81423584,95.72424123)
\curveto(636.63423648,95.48423942)(636.4142367,95.28423962)(636.15423584,95.12424123)
\curveto(635.89423722,94.96423994)(635.60923751,94.82424008)(635.29923584,94.70424123)
\curveto(635.15923796,94.64424026)(635.0192381,94.5992403)(634.87923584,94.56924123)
\curveto(634.72923839,94.53924036)(634.57423854,94.5042404)(634.41423584,94.46424123)
\curveto(634.30423881,94.44424046)(634.19423892,94.42924047)(634.08423584,94.41924123)
\curveto(633.97423914,94.40924049)(633.86423925,94.39424051)(633.75423584,94.37424123)
\curveto(633.7142394,94.36424054)(633.67423944,94.35924054)(633.63423584,94.35924123)
\curveto(633.59423952,94.36924053)(633.55423956,94.36924053)(633.51423584,94.35924123)
\curveto(633.46423965,94.34924055)(633.4142397,94.34424056)(633.36423584,94.34424123)
\lineto(633.19923584,94.34424123)
\curveto(633.14923997,94.32424058)(633.09924002,94.31924058)(633.04923584,94.32924123)
\curveto(632.98924013,94.33924056)(632.93424018,94.33924056)(632.88423584,94.32924123)
\curveto(632.84424027,94.31924058)(632.79924032,94.31924058)(632.74923584,94.32924123)
\curveto(632.69924042,94.33924056)(632.64924047,94.33424057)(632.59923584,94.31424123)
\curveto(632.52924059,94.29424061)(632.45424066,94.28924061)(632.37423584,94.29924123)
\curveto(632.28424083,94.30924059)(632.19924092,94.31424059)(632.11923584,94.31424123)
\curveto(632.02924109,94.31424059)(631.92924119,94.30924059)(631.81923584,94.29924123)
\curveto(631.69924142,94.28924061)(631.59924152,94.29424061)(631.51923584,94.31424123)
\lineto(631.23423584,94.31424123)
\lineto(630.60423584,94.35924123)
\curveto(630.50424261,94.36924053)(630.40924271,94.37924052)(630.31923584,94.38924123)
\lineto(630.01923584,94.41924123)
\curveto(629.96924315,94.43924046)(629.9192432,94.44424046)(629.86923584,94.43424123)
\curveto(629.80924331,94.43424047)(629.75424336,94.44424046)(629.70423584,94.46424123)
\curveto(629.53424358,94.51424039)(629.36924375,94.55424035)(629.20923584,94.58424123)
\curveto(629.03924408,94.61424029)(628.87924424,94.66424024)(628.72923584,94.73424123)
\curveto(628.26924485,94.92423998)(627.89424522,95.14423976)(627.60423584,95.39424123)
\curveto(627.3142458,95.65423925)(627.06924605,96.01423889)(626.86923584,96.47424123)
\curveto(626.8192463,96.6042383)(626.78424633,96.73423817)(626.76423584,96.86424123)
\curveto(626.74424637,97.0042379)(626.7192464,97.14423776)(626.68923584,97.28424123)
\curveto(626.67924644,97.35423755)(626.67424644,97.41923748)(626.67423584,97.47924123)
\curveto(626.67424644,97.53923736)(626.66924645,97.6042373)(626.65923584,97.67424123)
\curveto(626.63924648,98.5042364)(626.78924633,99.17423573)(627.10923584,99.68424123)
\curveto(627.4192457,100.19423471)(627.85924526,100.57423433)(628.42923584,100.82424123)
\curveto(628.54924457,100.87423403)(628.67424444,100.91923398)(628.80423584,100.95924123)
\curveto(628.93424418,100.9992339)(629.06924405,101.04423386)(629.20923584,101.09424123)
\curveto(629.28924383,101.11423379)(629.37424374,101.12923377)(629.46423584,101.13924123)
\lineto(629.70423584,101.19924123)
\curveto(629.8142433,101.22923367)(629.92424319,101.24423366)(630.03423584,101.24424123)
\curveto(630.14424297,101.25423365)(630.25424286,101.26923363)(630.36423584,101.28924123)
\curveto(630.4142427,101.30923359)(630.45924266,101.31423359)(630.49923584,101.30424123)
\curveto(630.53924258,101.3042336)(630.57924254,101.30923359)(630.61923584,101.31924123)
\curveto(630.66924245,101.32923357)(630.72424239,101.32923357)(630.78423584,101.31924123)
\curveto(630.83424228,101.31923358)(630.88424223,101.32423358)(630.93423584,101.33424123)
\lineto(631.06923584,101.33424123)
\curveto(631.12924199,101.35423355)(631.19924192,101.35423355)(631.27923584,101.33424123)
\curveto(631.34924177,101.32423358)(631.4142417,101.32923357)(631.47423584,101.34924123)
\curveto(631.50424161,101.35923354)(631.54424157,101.36423354)(631.59423584,101.36424123)
\lineto(631.71423584,101.36424123)
\lineto(632.17923584,101.36424123)
\moveto(634.50423584,99.81924123)
\curveto(634.18423893,99.91923498)(633.8192393,99.97923492)(633.40923584,99.99924123)
\curveto(632.99924012,100.01923488)(632.58924053,100.02923487)(632.17923584,100.02924123)
\curveto(631.74924137,100.02923487)(631.32924179,100.01923488)(630.91923584,99.99924123)
\curveto(630.50924261,99.97923492)(630.12424299,99.93423497)(629.76423584,99.86424123)
\curveto(629.40424371,99.79423511)(629.08424403,99.68423522)(628.80423584,99.53424123)
\curveto(628.5142446,99.39423551)(628.27924484,99.1992357)(628.09923584,98.94924123)
\curveto(627.98924513,98.78923611)(627.90924521,98.60923629)(627.85923584,98.40924123)
\curveto(627.79924532,98.20923669)(627.76924535,97.96423694)(627.76923584,97.67424123)
\curveto(627.78924533,97.65423725)(627.79924532,97.61923728)(627.79923584,97.56924123)
\curveto(627.78924533,97.51923738)(627.78924533,97.47923742)(627.79923584,97.44924123)
\curveto(627.8192453,97.36923753)(627.83924528,97.29423761)(627.85923584,97.22424123)
\curveto(627.86924525,97.16423774)(627.88924523,97.0992378)(627.91923584,97.02924123)
\curveto(628.03924508,96.75923814)(628.20924491,96.53923836)(628.42923584,96.36924123)
\curveto(628.63924448,96.20923869)(628.88424423,96.07423883)(629.16423584,95.96424123)
\curveto(629.27424384,95.91423899)(629.39424372,95.87423903)(629.52423584,95.84424123)
\curveto(629.64424347,95.82423908)(629.76924335,95.7992391)(629.89923584,95.76924123)
\curveto(629.94924317,95.74923915)(630.00424311,95.73923916)(630.06423584,95.73924123)
\curveto(630.114243,95.73923916)(630.16424295,95.73423917)(630.21423584,95.72424123)
\curveto(630.30424281,95.71423919)(630.39924272,95.7042392)(630.49923584,95.69424123)
\curveto(630.58924253,95.68423922)(630.68424243,95.67423923)(630.78423584,95.66424123)
\curveto(630.86424225,95.66423924)(630.94924217,95.65923924)(631.03923584,95.64924123)
\lineto(631.27923584,95.64924123)
\lineto(631.45923584,95.64924123)
\curveto(631.48924163,95.63923926)(631.52424159,95.63423927)(631.56423584,95.63424123)
\lineto(631.69923584,95.63424123)
\lineto(632.14923584,95.63424123)
\curveto(632.22924089,95.63423927)(632.3142408,95.62923927)(632.40423584,95.61924123)
\curveto(632.48424063,95.61923928)(632.55924056,95.62923927)(632.62923584,95.64924123)
\lineto(632.89923584,95.64924123)
\curveto(632.9192402,95.64923925)(632.94924017,95.64423926)(632.98923584,95.63424123)
\curveto(633.0192401,95.63423927)(633.04424007,95.63923926)(633.06423584,95.64924123)
\curveto(633.16423995,95.65923924)(633.26423985,95.66423924)(633.36423584,95.66424123)
\curveto(633.45423966,95.67423923)(633.55423956,95.68423922)(633.66423584,95.69424123)
\curveto(633.78423933,95.72423918)(633.90923921,95.73923916)(634.03923584,95.73924123)
\curveto(634.15923896,95.74923915)(634.27423884,95.77423913)(634.38423584,95.81424123)
\curveto(634.68423843,95.89423901)(634.94923817,95.97923892)(635.17923584,96.06924123)
\curveto(635.40923771,96.16923873)(635.62423749,96.31423859)(635.82423584,96.50424123)
\curveto(636.02423709,96.71423819)(636.17423694,96.97923792)(636.27423584,97.29924123)
\curveto(636.29423682,97.33923756)(636.30423681,97.37423753)(636.30423584,97.40424123)
\curveto(636.29423682,97.44423746)(636.29923682,97.48923741)(636.31923584,97.53924123)
\curveto(636.32923679,97.57923732)(636.33923678,97.64923725)(636.34923584,97.74924123)
\curveto(636.35923676,97.85923704)(636.35423676,97.94423696)(636.33423584,98.00424123)
\curveto(636.3142368,98.07423683)(636.30423681,98.14423676)(636.30423584,98.21424123)
\curveto(636.29423682,98.28423662)(636.27923684,98.34923655)(636.25923584,98.40924123)
\curveto(636.19923692,98.60923629)(636.114237,98.78923611)(636.00423584,98.94924123)
\curveto(635.98423713,98.97923592)(635.96423715,99.0042359)(635.94423584,99.02424123)
\lineto(635.88423584,99.08424123)
\curveto(635.86423725,99.12423578)(635.82423729,99.17423573)(635.76423584,99.23424123)
\curveto(635.62423749,99.33423557)(635.49423762,99.41923548)(635.37423584,99.48924123)
\curveto(635.25423786,99.55923534)(635.10923801,99.62923527)(634.93923584,99.69924123)
\curveto(634.86923825,99.72923517)(634.79923832,99.74923515)(634.72923584,99.75924123)
\curveto(634.65923846,99.77923512)(634.58423853,99.7992351)(634.50423584,99.81924123)
}
}
{
\newrgbcolor{curcolor}{0 0 0}
\pscustom[linestyle=none,fillstyle=solid,fillcolor=curcolor]
{
\newpath
\moveto(626.65923584,106.77385061)
\curveto(626.65924646,106.87384575)(626.66924645,106.96884566)(626.68923584,107.05885061)
\curveto(626.69924642,107.14884548)(626.72924639,107.21384541)(626.77923584,107.25385061)
\curveto(626.85924626,107.31384531)(626.96424615,107.34384528)(627.09423584,107.34385061)
\lineto(627.48423584,107.34385061)
\lineto(628.98423584,107.34385061)
\lineto(635.37423584,107.34385061)
\lineto(636.54423584,107.34385061)
\lineto(636.85923584,107.34385061)
\curveto(636.95923616,107.35384527)(637.03923608,107.33884529)(637.09923584,107.29885061)
\curveto(637.17923594,107.24884538)(637.22923589,107.17384545)(637.24923584,107.07385061)
\curveto(637.25923586,106.98384564)(637.26423585,106.87384575)(637.26423584,106.74385061)
\lineto(637.26423584,106.51885061)
\curveto(637.24423587,106.43884619)(637.22923589,106.36884626)(637.21923584,106.30885061)
\curveto(637.19923592,106.24884638)(637.15923596,106.19884643)(637.09923584,106.15885061)
\curveto(637.03923608,106.11884651)(636.96423615,106.09884653)(636.87423584,106.09885061)
\lineto(636.57423584,106.09885061)
\lineto(635.47923584,106.09885061)
\lineto(630.13923584,106.09885061)
\curveto(630.04924307,106.07884655)(629.97424314,106.06384656)(629.91423584,106.05385061)
\curveto(629.84424327,106.05384657)(629.78424333,106.0238466)(629.73423584,105.96385061)
\curveto(629.68424343,105.89384673)(629.65924346,105.80384682)(629.65923584,105.69385061)
\curveto(629.64924347,105.59384703)(629.64424347,105.48384714)(629.64423584,105.36385061)
\lineto(629.64423584,104.22385061)
\lineto(629.64423584,103.72885061)
\curveto(629.63424348,103.56884906)(629.57424354,103.45884917)(629.46423584,103.39885061)
\curveto(629.43424368,103.37884925)(629.40424371,103.36884926)(629.37423584,103.36885061)
\curveto(629.33424378,103.36884926)(629.28924383,103.36384926)(629.23923584,103.35385061)
\curveto(629.119244,103.33384929)(629.00924411,103.33884929)(628.90923584,103.36885061)
\curveto(628.80924431,103.40884922)(628.73924438,103.46384916)(628.69923584,103.53385061)
\curveto(628.64924447,103.61384901)(628.62424449,103.73384889)(628.62423584,103.89385061)
\curveto(628.62424449,104.05384857)(628.60924451,104.18884844)(628.57923584,104.29885061)
\curveto(628.56924455,104.34884828)(628.56424455,104.40384822)(628.56423584,104.46385061)
\curveto(628.55424456,104.5238481)(628.53924458,104.58384804)(628.51923584,104.64385061)
\curveto(628.46924465,104.79384783)(628.4192447,104.93884769)(628.36923584,105.07885061)
\curveto(628.30924481,105.21884741)(628.23924488,105.35384727)(628.15923584,105.48385061)
\curveto(628.06924505,105.623847)(627.96424515,105.74384688)(627.84423584,105.84385061)
\curveto(627.72424539,105.94384668)(627.59424552,106.03884659)(627.45423584,106.12885061)
\curveto(627.35424576,106.18884644)(627.24424587,106.23384639)(627.12423584,106.26385061)
\curveto(627.00424611,106.30384632)(626.89924622,106.35384627)(626.80923584,106.41385061)
\curveto(626.74924637,106.46384616)(626.70924641,106.53384609)(626.68923584,106.62385061)
\curveto(626.67924644,106.64384598)(626.67424644,106.66884596)(626.67423584,106.69885061)
\curveto(626.67424644,106.7288459)(626.66924645,106.75384587)(626.65923584,106.77385061)
}
}
{
\newrgbcolor{curcolor}{0 0 0}
\pscustom[linestyle=none,fillstyle=solid,fillcolor=curcolor]
{
\newpath
\moveto(626.65923584,115.12345998)
\curveto(626.65924646,115.22345513)(626.66924645,115.31845503)(626.68923584,115.40845998)
\curveto(626.69924642,115.49845485)(626.72924639,115.56345479)(626.77923584,115.60345998)
\curveto(626.85924626,115.66345469)(626.96424615,115.69345466)(627.09423584,115.69345998)
\lineto(627.48423584,115.69345998)
\lineto(628.98423584,115.69345998)
\lineto(635.37423584,115.69345998)
\lineto(636.54423584,115.69345998)
\lineto(636.85923584,115.69345998)
\curveto(636.95923616,115.70345465)(637.03923608,115.68845466)(637.09923584,115.64845998)
\curveto(637.17923594,115.59845475)(637.22923589,115.52345483)(637.24923584,115.42345998)
\curveto(637.25923586,115.33345502)(637.26423585,115.22345513)(637.26423584,115.09345998)
\lineto(637.26423584,114.86845998)
\curveto(637.24423587,114.78845556)(637.22923589,114.71845563)(637.21923584,114.65845998)
\curveto(637.19923592,114.59845575)(637.15923596,114.5484558)(637.09923584,114.50845998)
\curveto(637.03923608,114.46845588)(636.96423615,114.4484559)(636.87423584,114.44845998)
\lineto(636.57423584,114.44845998)
\lineto(635.47923584,114.44845998)
\lineto(630.13923584,114.44845998)
\curveto(630.04924307,114.42845592)(629.97424314,114.41345594)(629.91423584,114.40345998)
\curveto(629.84424327,114.40345595)(629.78424333,114.37345598)(629.73423584,114.31345998)
\curveto(629.68424343,114.24345611)(629.65924346,114.1534562)(629.65923584,114.04345998)
\curveto(629.64924347,113.94345641)(629.64424347,113.83345652)(629.64423584,113.71345998)
\lineto(629.64423584,112.57345998)
\lineto(629.64423584,112.07845998)
\curveto(629.63424348,111.91845843)(629.57424354,111.80845854)(629.46423584,111.74845998)
\curveto(629.43424368,111.72845862)(629.40424371,111.71845863)(629.37423584,111.71845998)
\curveto(629.33424378,111.71845863)(629.28924383,111.71345864)(629.23923584,111.70345998)
\curveto(629.119244,111.68345867)(629.00924411,111.68845866)(628.90923584,111.71845998)
\curveto(628.80924431,111.75845859)(628.73924438,111.81345854)(628.69923584,111.88345998)
\curveto(628.64924447,111.96345839)(628.62424449,112.08345827)(628.62423584,112.24345998)
\curveto(628.62424449,112.40345795)(628.60924451,112.53845781)(628.57923584,112.64845998)
\curveto(628.56924455,112.69845765)(628.56424455,112.7534576)(628.56423584,112.81345998)
\curveto(628.55424456,112.87345748)(628.53924458,112.93345742)(628.51923584,112.99345998)
\curveto(628.46924465,113.14345721)(628.4192447,113.28845706)(628.36923584,113.42845998)
\curveto(628.30924481,113.56845678)(628.23924488,113.70345665)(628.15923584,113.83345998)
\curveto(628.06924505,113.97345638)(627.96424515,114.09345626)(627.84423584,114.19345998)
\curveto(627.72424539,114.29345606)(627.59424552,114.38845596)(627.45423584,114.47845998)
\curveto(627.35424576,114.53845581)(627.24424587,114.58345577)(627.12423584,114.61345998)
\curveto(627.00424611,114.6534557)(626.89924622,114.70345565)(626.80923584,114.76345998)
\curveto(626.74924637,114.81345554)(626.70924641,114.88345547)(626.68923584,114.97345998)
\curveto(626.67924644,114.99345536)(626.67424644,115.01845533)(626.67423584,115.04845998)
\curveto(626.67424644,115.07845527)(626.66924645,115.10345525)(626.65923584,115.12345998)
}
}
{
\newrgbcolor{curcolor}{0 0 0}
\pscustom[linestyle=none,fillstyle=solid,fillcolor=curcolor]
{
\newpath
\moveto(648.5305249,37.28705373)
\curveto(648.5305356,37.35704805)(648.5305356,37.43704797)(648.5305249,37.52705373)
\curveto(648.52053561,37.61704779)(648.52053561,37.70204771)(648.5305249,37.78205373)
\curveto(648.5305356,37.87204754)(648.54053559,37.95204746)(648.5605249,38.02205373)
\curveto(648.58053555,38.10204731)(648.61053552,38.15704725)(648.6505249,38.18705373)
\curveto(648.70053543,38.21704719)(648.77553535,38.23704717)(648.8755249,38.24705373)
\curveto(648.96553516,38.26704714)(649.07053506,38.27704713)(649.1905249,38.27705373)
\curveto(649.30053483,38.28704712)(649.41553471,38.28704712)(649.5355249,38.27705373)
\lineto(649.8355249,38.27705373)
\lineto(652.8505249,38.27705373)
\lineto(655.7455249,38.27705373)
\curveto(656.07552805,38.27704713)(656.40052773,38.27204714)(656.7205249,38.26205373)
\curveto(657.0305271,38.26204715)(657.31052682,38.22204719)(657.5605249,38.14205373)
\curveto(657.91052622,38.02204739)(658.20552592,37.86704754)(658.4455249,37.67705373)
\curveto(658.67552545,37.48704792)(658.87552525,37.24704816)(659.0455249,36.95705373)
\curveto(659.09552503,36.89704851)(659.130525,36.83204858)(659.1505249,36.76205373)
\curveto(659.17052496,36.70204871)(659.19552493,36.63204878)(659.2255249,36.55205373)
\curveto(659.27552485,36.43204898)(659.31052482,36.30204911)(659.3305249,36.16205373)
\curveto(659.36052477,36.03204938)(659.39052474,35.89704951)(659.4205249,35.75705373)
\curveto(659.44052469,35.7070497)(659.44552468,35.65704975)(659.4355249,35.60705373)
\curveto(659.4255247,35.55704985)(659.4255247,35.50204991)(659.4355249,35.44205373)
\curveto(659.44552468,35.42204999)(659.44552468,35.39705001)(659.4355249,35.36705373)
\curveto(659.43552469,35.33705007)(659.44052469,35.3120501)(659.4505249,35.29205373)
\curveto(659.46052467,35.25205016)(659.46552466,35.19705021)(659.4655249,35.12705373)
\curveto(659.46552466,35.05705035)(659.46052467,35.00205041)(659.4505249,34.96205373)
\curveto(659.44052469,34.9120505)(659.44052469,34.85705055)(659.4505249,34.79705373)
\curveto(659.46052467,34.73705067)(659.45552467,34.68205073)(659.4355249,34.63205373)
\curveto(659.40552472,34.50205091)(659.38552474,34.37705103)(659.3755249,34.25705373)
\curveto(659.36552476,34.13705127)(659.34052479,34.02205139)(659.3005249,33.91205373)
\curveto(659.18052495,33.54205187)(659.01052512,33.22205219)(658.7905249,32.95205373)
\curveto(658.57052556,32.68205273)(658.29052584,32.47205294)(657.9505249,32.32205373)
\curveto(657.8305263,32.27205314)(657.70552642,32.22705318)(657.5755249,32.18705373)
\curveto(657.44552668,32.15705325)(657.31052682,32.12205329)(657.1705249,32.08205373)
\curveto(657.12052701,32.07205334)(657.08052705,32.06705334)(657.0505249,32.06705373)
\curveto(657.01052712,32.06705334)(656.96552716,32.06205335)(656.9155249,32.05205373)
\curveto(656.88552724,32.04205337)(656.85052728,32.03705337)(656.8105249,32.03705373)
\curveto(656.76052737,32.03705337)(656.72052741,32.03205338)(656.6905249,32.02205373)
\lineto(656.5255249,32.02205373)
\curveto(656.44552768,32.00205341)(656.34552778,31.99705341)(656.2255249,32.00705373)
\curveto(656.09552803,32.01705339)(656.00552812,32.03205338)(655.9555249,32.05205373)
\curveto(655.86552826,32.07205334)(655.80052833,32.12705328)(655.7605249,32.21705373)
\curveto(655.74052839,32.24705316)(655.73552839,32.27705313)(655.7455249,32.30705373)
\curveto(655.74552838,32.33705307)(655.74052839,32.37705303)(655.7305249,32.42705373)
\curveto(655.72052841,32.46705294)(655.71552841,32.5070529)(655.7155249,32.54705373)
\lineto(655.7155249,32.69705373)
\curveto(655.71552841,32.81705259)(655.72052841,32.93705247)(655.7305249,33.05705373)
\curveto(655.7305284,33.18705222)(655.76552836,33.27705213)(655.8355249,33.32705373)
\curveto(655.89552823,33.36705204)(655.95552817,33.38705202)(656.0155249,33.38705373)
\curveto(656.07552805,33.38705202)(656.14552798,33.39705201)(656.2255249,33.41705373)
\curveto(656.25552787,33.42705198)(656.29052784,33.42705198)(656.3305249,33.41705373)
\curveto(656.36052777,33.41705199)(656.38552774,33.42205199)(656.4055249,33.43205373)
\lineto(656.6155249,33.43205373)
\curveto(656.66552746,33.45205196)(656.71552741,33.45705195)(656.7655249,33.44705373)
\curveto(656.80552732,33.44705196)(656.85052728,33.45705195)(656.9005249,33.47705373)
\curveto(657.0305271,33.5070519)(657.15552697,33.53705187)(657.2755249,33.56705373)
\curveto(657.38552674,33.59705181)(657.49052664,33.64205177)(657.5905249,33.70205373)
\curveto(657.88052625,33.87205154)(658.08552604,34.14205127)(658.2055249,34.51205373)
\curveto(658.2255259,34.56205085)(658.24052589,34.6120508)(658.2505249,34.66205373)
\curveto(658.25052588,34.72205069)(658.26052587,34.77705063)(658.2805249,34.82705373)
\lineto(658.2805249,34.90205373)
\curveto(658.29052584,34.97205044)(658.30052583,35.06705034)(658.3105249,35.18705373)
\curveto(658.31052582,35.31705009)(658.30052583,35.41704999)(658.2805249,35.48705373)
\curveto(658.26052587,35.55704985)(658.24552588,35.62704978)(658.2355249,35.69705373)
\curveto(658.21552591,35.77704963)(658.19552593,35.84704956)(658.1755249,35.90705373)
\curveto(658.01552611,36.28704912)(657.74052639,36.56204885)(657.3505249,36.73205373)
\curveto(657.22052691,36.78204863)(657.06552706,36.81704859)(656.8855249,36.83705373)
\curveto(656.70552742,36.86704854)(656.52052761,36.88204853)(656.3305249,36.88205373)
\curveto(656.130528,36.89204852)(655.9305282,36.89204852)(655.7305249,36.88205373)
\lineto(655.1605249,36.88205373)
\lineto(650.9155249,36.88205373)
\lineto(649.3705249,36.88205373)
\curveto(649.26053487,36.88204853)(649.14053499,36.87704853)(649.0105249,36.86705373)
\curveto(648.88053525,36.85704855)(648.77553535,36.87704853)(648.6955249,36.92705373)
\curveto(648.6255355,36.98704842)(648.57553555,37.06704834)(648.5455249,37.16705373)
\curveto(648.54553558,37.18704822)(648.54553558,37.2070482)(648.5455249,37.22705373)
\curveto(648.54553558,37.24704816)(648.54053559,37.26704814)(648.5305249,37.28705373)
}
}
{
\newrgbcolor{curcolor}{0 0 0}
\pscustom[linestyle=none,fillstyle=solid,fillcolor=curcolor]
{
\newpath
\moveto(651.4855249,40.82072561)
\lineto(651.4855249,41.25572561)
\curveto(651.48553264,41.40572364)(651.5255326,41.51072354)(651.6055249,41.57072561)
\curveto(651.68553244,41.62072343)(651.78553234,41.6457234)(651.9055249,41.64572561)
\curveto(652.0255321,41.65572339)(652.14553198,41.66072339)(652.2655249,41.66072561)
\lineto(653.6905249,41.66072561)
\lineto(655.9555249,41.66072561)
\lineto(656.6455249,41.66072561)
\curveto(656.87552725,41.66072339)(657.07552705,41.68572336)(657.2455249,41.73572561)
\curveto(657.69552643,41.89572315)(658.01052612,42.19572285)(658.1905249,42.63572561)
\curveto(658.28052585,42.85572219)(658.31552581,43.12072193)(658.2955249,43.43072561)
\curveto(658.26552586,43.74072131)(658.21052592,43.99072106)(658.1305249,44.18072561)
\curveto(657.99052614,44.51072054)(657.81552631,44.77072028)(657.6055249,44.96072561)
\curveto(657.38552674,45.16071989)(657.10052703,45.31571973)(656.7505249,45.42572561)
\curveto(656.67052746,45.45571959)(656.59052754,45.47571957)(656.5105249,45.48572561)
\curveto(656.4305277,45.49571955)(656.34552778,45.51071954)(656.2555249,45.53072561)
\curveto(656.20552792,45.54071951)(656.16052797,45.54071951)(656.1205249,45.53072561)
\curveto(656.08052805,45.53071952)(656.03552809,45.54071951)(655.9855249,45.56072561)
\lineto(655.6705249,45.56072561)
\curveto(655.59052854,45.58071947)(655.50052863,45.58571946)(655.4005249,45.57572561)
\curveto(655.29052884,45.56571948)(655.19052894,45.56071949)(655.1005249,45.56072561)
\lineto(653.9305249,45.56072561)
\lineto(652.3405249,45.56072561)
\curveto(652.22053191,45.56071949)(652.09553203,45.55571949)(651.9655249,45.54572561)
\curveto(651.8255323,45.5457195)(651.71553241,45.57071948)(651.6355249,45.62072561)
\curveto(651.58553254,45.66071939)(651.55553257,45.70571934)(651.5455249,45.75572561)
\curveto(651.5255326,45.81571923)(651.50553262,45.88571916)(651.4855249,45.96572561)
\lineto(651.4855249,46.19072561)
\curveto(651.48553264,46.31071874)(651.49053264,46.41571863)(651.5005249,46.50572561)
\curveto(651.51053262,46.60571844)(651.55553257,46.68071837)(651.6355249,46.73072561)
\curveto(651.68553244,46.78071827)(651.76053237,46.80571824)(651.8605249,46.80572561)
\lineto(652.1455249,46.80572561)
\lineto(653.1655249,46.80572561)
\lineto(657.2005249,46.80572561)
\lineto(658.5505249,46.80572561)
\curveto(658.67052546,46.80571824)(658.78552534,46.80071825)(658.8955249,46.79072561)
\curveto(658.99552513,46.79071826)(659.07052506,46.75571829)(659.1205249,46.68572561)
\curveto(659.15052498,46.6457184)(659.17552495,46.58571846)(659.1955249,46.50572561)
\curveto(659.20552492,46.42571862)(659.21552491,46.33571871)(659.2255249,46.23572561)
\curveto(659.2255249,46.1457189)(659.22052491,46.05571899)(659.2105249,45.96572561)
\curveto(659.20052493,45.88571916)(659.18052495,45.82571922)(659.1505249,45.78572561)
\curveto(659.11052502,45.73571931)(659.04552508,45.69071936)(658.9555249,45.65072561)
\curveto(658.91552521,45.64071941)(658.86052527,45.63071942)(658.7905249,45.62072561)
\curveto(658.72052541,45.62071943)(658.65552547,45.61571943)(658.5955249,45.60572561)
\curveto(658.5255256,45.59571945)(658.47052566,45.57571947)(658.4305249,45.54572561)
\curveto(658.39052574,45.51571953)(658.37552575,45.47071958)(658.3855249,45.41072561)
\curveto(658.40552572,45.33071972)(658.46552566,45.2507198)(658.5655249,45.17072561)
\curveto(658.65552547,45.09071996)(658.7255254,45.01572003)(658.7755249,44.94572561)
\curveto(658.93552519,44.72572032)(659.07552505,44.47572057)(659.1955249,44.19572561)
\curveto(659.24552488,44.08572096)(659.27552485,43.97072108)(659.2855249,43.85072561)
\curveto(659.30552482,43.74072131)(659.3305248,43.62572142)(659.3605249,43.50572561)
\curveto(659.37052476,43.45572159)(659.37052476,43.40072165)(659.3605249,43.34072561)
\curveto(659.35052478,43.29072176)(659.35552477,43.24072181)(659.3755249,43.19072561)
\curveto(659.39552473,43.09072196)(659.39552473,43.00072205)(659.3755249,42.92072561)
\lineto(659.3755249,42.77072561)
\curveto(659.35552477,42.72072233)(659.34552478,42.66072239)(659.3455249,42.59072561)
\curveto(659.34552478,42.53072252)(659.34052479,42.47572257)(659.3305249,42.42572561)
\curveto(659.31052482,42.38572266)(659.30052483,42.3457227)(659.3005249,42.30572561)
\curveto(659.31052482,42.27572277)(659.30552482,42.23572281)(659.2855249,42.18572561)
\lineto(659.2255249,41.94572561)
\curveto(659.20552492,41.87572317)(659.17552495,41.80072325)(659.1355249,41.72072561)
\curveto(659.0255251,41.46072359)(658.88052525,41.24072381)(658.7005249,41.06072561)
\curveto(658.51052562,40.89072416)(658.28552584,40.7507243)(658.0255249,40.64072561)
\curveto(657.93552619,40.60072445)(657.84552628,40.57072448)(657.7555249,40.55072561)
\lineto(657.4555249,40.49072561)
\curveto(657.39552673,40.47072458)(657.34052679,40.46072459)(657.2905249,40.46072561)
\curveto(657.2305269,40.47072458)(657.16552696,40.46572458)(657.0955249,40.44572561)
\curveto(657.07552705,40.43572461)(657.05052708,40.43072462)(657.0205249,40.43072561)
\curveto(656.98052715,40.43072462)(656.94552718,40.42572462)(656.9155249,40.41572561)
\lineto(656.7655249,40.41572561)
\curveto(656.7255274,40.40572464)(656.68052745,40.40072465)(656.6305249,40.40072561)
\curveto(656.57052756,40.41072464)(656.51552761,40.41572463)(656.4655249,40.41572561)
\lineto(655.8655249,40.41572561)
\lineto(653.1055249,40.41572561)
\lineto(652.1455249,40.41572561)
\lineto(651.8755249,40.41572561)
\curveto(651.78553234,40.41572463)(651.71053242,40.43572461)(651.6505249,40.47572561)
\curveto(651.58053255,40.51572453)(651.5305326,40.59072446)(651.5005249,40.70072561)
\curveto(651.49053264,40.72072433)(651.49053264,40.74072431)(651.5005249,40.76072561)
\curveto(651.50053263,40.78072427)(651.49553263,40.80072425)(651.4855249,40.82072561)
}
}
{
\newrgbcolor{curcolor}{0 0 0}
\pscustom[linestyle=none,fillstyle=solid,fillcolor=curcolor]
{
\newpath
\moveto(651.3355249,52.39533498)
\curveto(651.31553281,53.02532975)(651.40053273,53.53032924)(651.5905249,53.91033498)
\curveto(651.78053235,54.29032848)(652.06553206,54.59532818)(652.4455249,54.82533498)
\curveto(652.54553158,54.88532789)(652.65553147,54.93032784)(652.7755249,54.96033498)
\curveto(652.88553124,55.00032777)(653.00053113,55.03532774)(653.1205249,55.06533498)
\curveto(653.31053082,55.11532766)(653.51553061,55.14532763)(653.7355249,55.15533498)
\curveto(653.95553017,55.16532761)(654.18052995,55.1703276)(654.4105249,55.17033498)
\lineto(656.0155249,55.17033498)
\lineto(658.3555249,55.17033498)
\curveto(658.5255256,55.1703276)(658.69552543,55.16532761)(658.8655249,55.15533498)
\curveto(659.03552509,55.15532762)(659.14552498,55.09032768)(659.1955249,54.96033498)
\curveto(659.21552491,54.91032786)(659.2255249,54.85532792)(659.2255249,54.79533498)
\curveto(659.23552489,54.74532803)(659.24052489,54.69032808)(659.2405249,54.63033498)
\curveto(659.24052489,54.50032827)(659.23552489,54.3753284)(659.2255249,54.25533498)
\curveto(659.2255249,54.13532864)(659.18552494,54.05032872)(659.1055249,54.00033498)
\curveto(659.03552509,53.95032882)(658.94552518,53.92532885)(658.8355249,53.92533498)
\lineto(658.5055249,53.92533498)
\lineto(657.2155249,53.92533498)
\lineto(654.7705249,53.92533498)
\curveto(654.50052963,53.92532885)(654.23552989,53.92032885)(653.9755249,53.91033498)
\curveto(653.70553042,53.90032887)(653.47553065,53.85532892)(653.2855249,53.77533498)
\curveto(653.08553104,53.69532908)(652.9255312,53.5753292)(652.8055249,53.41533498)
\curveto(652.67553145,53.25532952)(652.57553155,53.0703297)(652.5055249,52.86033498)
\curveto(652.48553164,52.80032997)(652.47553165,52.73533004)(652.4755249,52.66533498)
\curveto(652.46553166,52.60533017)(652.45053168,52.54533023)(652.4305249,52.48533498)
\curveto(652.42053171,52.43533034)(652.42053171,52.35533042)(652.4305249,52.24533498)
\curveto(652.4305317,52.14533063)(652.43553169,52.0753307)(652.4455249,52.03533498)
\curveto(652.46553166,51.99533078)(652.47553165,51.96033081)(652.4755249,51.93033498)
\curveto(652.46553166,51.90033087)(652.46553166,51.86533091)(652.4755249,51.82533498)
\curveto(652.50553162,51.69533108)(652.54053159,51.5703312)(652.5805249,51.45033498)
\curveto(652.61053152,51.34033143)(652.65553147,51.23533154)(652.7155249,51.13533498)
\curveto(652.73553139,51.09533168)(652.75553137,51.06033171)(652.7755249,51.03033498)
\curveto(652.79553133,51.00033177)(652.81553131,50.96533181)(652.8355249,50.92533498)
\curveto(653.08553104,50.5753322)(653.46053067,50.32033245)(653.9605249,50.16033498)
\curveto(654.04053009,50.13033264)(654.12553,50.11033266)(654.2155249,50.10033498)
\curveto(654.29552983,50.09033268)(654.37552975,50.0753327)(654.4555249,50.05533498)
\curveto(654.50552962,50.03533274)(654.55552957,50.03033274)(654.6055249,50.04033498)
\curveto(654.64552948,50.05033272)(654.68552944,50.04533273)(654.7255249,50.02533498)
\lineto(655.0405249,50.02533498)
\curveto(655.07052906,50.01533276)(655.10552902,50.01033276)(655.1455249,50.01033498)
\curveto(655.18552894,50.02033275)(655.2305289,50.02533275)(655.2805249,50.02533498)
\lineto(655.7305249,50.02533498)
\lineto(657.1705249,50.02533498)
\lineto(658.4905249,50.02533498)
\lineto(658.8355249,50.02533498)
\curveto(658.94552518,50.02533275)(659.03552509,50.00033277)(659.1055249,49.95033498)
\curveto(659.18552494,49.90033287)(659.2255249,49.81033296)(659.2255249,49.68033498)
\curveto(659.23552489,49.56033321)(659.24052489,49.43533334)(659.2405249,49.30533498)
\curveto(659.24052489,49.22533355)(659.23552489,49.15033362)(659.2255249,49.08033498)
\curveto(659.21552491,49.01033376)(659.19052494,48.95033382)(659.1505249,48.90033498)
\curveto(659.10052503,48.82033395)(659.00552512,48.78033399)(658.8655249,48.78033498)
\lineto(658.4605249,48.78033498)
\lineto(656.6905249,48.78033498)
\lineto(653.0605249,48.78033498)
\lineto(652.1455249,48.78033498)
\lineto(651.8755249,48.78033498)
\curveto(651.78553234,48.78033399)(651.71553241,48.80033397)(651.6655249,48.84033498)
\curveto(651.60553252,48.8703339)(651.56553256,48.92033385)(651.5455249,48.99033498)
\curveto(651.53553259,49.03033374)(651.5255326,49.08533369)(651.5155249,49.15533498)
\curveto(651.50553262,49.23533354)(651.50053263,49.31533346)(651.5005249,49.39533498)
\curveto(651.50053263,49.4753333)(651.50553262,49.55033322)(651.5155249,49.62033498)
\curveto(651.5255326,49.70033307)(651.54053259,49.75533302)(651.5605249,49.78533498)
\curveto(651.6305325,49.89533288)(651.72053241,49.94533283)(651.8305249,49.93533498)
\curveto(651.9305322,49.92533285)(652.04553208,49.94033283)(652.1755249,49.98033498)
\curveto(652.23553189,50.00033277)(652.28553184,50.04033273)(652.3255249,50.10033498)
\curveto(652.33553179,50.22033255)(652.29053184,50.31533246)(652.1905249,50.38533498)
\curveto(652.09053204,50.46533231)(652.01053212,50.54533223)(651.9505249,50.62533498)
\curveto(651.85053228,50.76533201)(651.76053237,50.90533187)(651.6805249,51.04533498)
\curveto(651.59053254,51.19533158)(651.51553261,51.36533141)(651.4555249,51.55533498)
\curveto(651.4255327,51.63533114)(651.40553272,51.72033105)(651.3955249,51.81033498)
\curveto(651.38553274,51.91033086)(651.37053276,52.00533077)(651.3505249,52.09533498)
\curveto(651.34053279,52.14533063)(651.33553279,52.19533058)(651.3355249,52.24533498)
\lineto(651.3355249,52.39533498)
}
}
{
\newrgbcolor{curcolor}{0 0 0}
\pscustom[linestyle=none,fillstyle=solid,fillcolor=curcolor]
{
}
}
{
\newrgbcolor{curcolor}{0 0 0}
\pscustom[linestyle=none,fillstyle=solid,fillcolor=curcolor]
{
\newpath
\moveto(654.1255249,67.99510061)
\lineto(654.3805249,67.99510061)
\curveto(654.46052967,68.0050929)(654.53552959,68.00009291)(654.6055249,67.98010061)
\lineto(654.8455249,67.98010061)
\lineto(655.0105249,67.98010061)
\curveto(655.11052902,67.96009295)(655.21552891,67.95009296)(655.3255249,67.95010061)
\curveto(655.4255287,67.95009296)(655.5255286,67.94009297)(655.6255249,67.92010061)
\lineto(655.7755249,67.92010061)
\curveto(655.91552821,67.89009302)(656.05552807,67.87009304)(656.1955249,67.86010061)
\curveto(656.3255278,67.85009306)(656.45552767,67.82509308)(656.5855249,67.78510061)
\curveto(656.66552746,67.76509314)(656.75052738,67.74509316)(656.8405249,67.72510061)
\lineto(657.0805249,67.66510061)
\lineto(657.3805249,67.54510061)
\curveto(657.47052666,67.51509339)(657.56052657,67.48009343)(657.6505249,67.44010061)
\curveto(657.87052626,67.34009357)(658.08552604,67.2050937)(658.2955249,67.03510061)
\curveto(658.50552562,66.87509403)(658.67552545,66.70009421)(658.8055249,66.51010061)
\curveto(658.84552528,66.46009445)(658.88552524,66.40009451)(658.9255249,66.33010061)
\curveto(658.95552517,66.27009464)(658.99052514,66.2100947)(659.0305249,66.15010061)
\curveto(659.08052505,66.07009484)(659.12052501,65.97509493)(659.1505249,65.86510061)
\curveto(659.18052495,65.75509515)(659.21052492,65.65009526)(659.2405249,65.55010061)
\curveto(659.28052485,65.44009547)(659.30552482,65.33009558)(659.3155249,65.22010061)
\curveto(659.3255248,65.1100958)(659.34052479,64.99509591)(659.3605249,64.87510061)
\curveto(659.37052476,64.83509607)(659.37052476,64.79009612)(659.3605249,64.74010061)
\curveto(659.36052477,64.70009621)(659.36552476,64.66009625)(659.3755249,64.62010061)
\curveto(659.38552474,64.58009633)(659.39052474,64.52509638)(659.3905249,64.45510061)
\curveto(659.39052474,64.38509652)(659.38552474,64.33509657)(659.3755249,64.30510061)
\curveto(659.35552477,64.25509665)(659.35052478,64.2100967)(659.3605249,64.17010061)
\curveto(659.37052476,64.13009678)(659.37052476,64.09509681)(659.3605249,64.06510061)
\lineto(659.3605249,63.97510061)
\curveto(659.34052479,63.91509699)(659.3255248,63.85009706)(659.3155249,63.78010061)
\curveto(659.31552481,63.72009719)(659.31052482,63.65509725)(659.3005249,63.58510061)
\curveto(659.25052488,63.41509749)(659.20052493,63.25509765)(659.1505249,63.10510061)
\curveto(659.10052503,62.95509795)(659.03552509,62.8100981)(658.9555249,62.67010061)
\curveto(658.91552521,62.62009829)(658.88552524,62.56509834)(658.8655249,62.50510061)
\curveto(658.83552529,62.45509845)(658.80052533,62.4050985)(658.7605249,62.35510061)
\curveto(658.58052555,62.11509879)(658.36052577,61.91509899)(658.1005249,61.75510061)
\curveto(657.84052629,61.59509931)(657.55552657,61.45509945)(657.2455249,61.33510061)
\curveto(657.10552702,61.27509963)(656.96552716,61.23009968)(656.8255249,61.20010061)
\curveto(656.67552745,61.17009974)(656.52052761,61.13509977)(656.3605249,61.09510061)
\curveto(656.25052788,61.07509983)(656.14052799,61.06009985)(656.0305249,61.05010061)
\curveto(655.92052821,61.04009987)(655.81052832,61.02509988)(655.7005249,61.00510061)
\curveto(655.66052847,60.99509991)(655.62052851,60.99009992)(655.5805249,60.99010061)
\curveto(655.54052859,61.00009991)(655.50052863,61.00009991)(655.4605249,60.99010061)
\curveto(655.41052872,60.98009993)(655.36052877,60.97509993)(655.3105249,60.97510061)
\lineto(655.1455249,60.97510061)
\curveto(655.09552903,60.95509995)(655.04552908,60.95009996)(654.9955249,60.96010061)
\curveto(654.93552919,60.97009994)(654.88052925,60.97009994)(654.8305249,60.96010061)
\curveto(654.79052934,60.95009996)(654.74552938,60.95009996)(654.6955249,60.96010061)
\curveto(654.64552948,60.97009994)(654.59552953,60.96509994)(654.5455249,60.94510061)
\curveto(654.47552965,60.92509998)(654.40052973,60.92009999)(654.3205249,60.93010061)
\curveto(654.2305299,60.94009997)(654.14552998,60.94509996)(654.0655249,60.94510061)
\curveto(653.97553015,60.94509996)(653.87553025,60.94009997)(653.7655249,60.93010061)
\curveto(653.64553048,60.92009999)(653.54553058,60.92509998)(653.4655249,60.94510061)
\lineto(653.1805249,60.94510061)
\lineto(652.5505249,60.99010061)
\curveto(652.45053168,61.00009991)(652.35553177,61.0100999)(652.2655249,61.02010061)
\lineto(651.9655249,61.05010061)
\curveto(651.91553221,61.07009984)(651.86553226,61.07509983)(651.8155249,61.06510061)
\curveto(651.75553237,61.06509984)(651.70053243,61.07509983)(651.6505249,61.09510061)
\curveto(651.48053265,61.14509976)(651.31553281,61.18509972)(651.1555249,61.21510061)
\curveto(650.98553314,61.24509966)(650.8255333,61.29509961)(650.6755249,61.36510061)
\curveto(650.21553391,61.55509935)(649.84053429,61.77509913)(649.5505249,62.02510061)
\curveto(649.26053487,62.28509862)(649.01553511,62.64509826)(648.8155249,63.10510061)
\curveto(648.76553536,63.23509767)(648.7305354,63.36509754)(648.7105249,63.49510061)
\curveto(648.69053544,63.63509727)(648.66553546,63.77509713)(648.6355249,63.91510061)
\curveto(648.6255355,63.98509692)(648.62053551,64.05009686)(648.6205249,64.11010061)
\curveto(648.62053551,64.17009674)(648.61553551,64.23509667)(648.6055249,64.30510061)
\curveto(648.58553554,65.13509577)(648.73553539,65.8050951)(649.0555249,66.31510061)
\curveto(649.36553476,66.82509408)(649.80553432,67.2050937)(650.3755249,67.45510061)
\curveto(650.49553363,67.5050934)(650.62053351,67.55009336)(650.7505249,67.59010061)
\curveto(650.88053325,67.63009328)(651.01553311,67.67509323)(651.1555249,67.72510061)
\curveto(651.23553289,67.74509316)(651.32053281,67.76009315)(651.4105249,67.77010061)
\lineto(651.6505249,67.83010061)
\curveto(651.76053237,67.86009305)(651.87053226,67.87509303)(651.9805249,67.87510061)
\curveto(652.09053204,67.88509302)(652.20053193,67.90009301)(652.3105249,67.92010061)
\curveto(652.36053177,67.94009297)(652.40553172,67.94509296)(652.4455249,67.93510061)
\curveto(652.48553164,67.93509297)(652.5255316,67.94009297)(652.5655249,67.95010061)
\curveto(652.61553151,67.96009295)(652.67053146,67.96009295)(652.7305249,67.95010061)
\curveto(652.78053135,67.95009296)(652.8305313,67.95509295)(652.8805249,67.96510061)
\lineto(653.0155249,67.96510061)
\curveto(653.07553105,67.98509292)(653.14553098,67.98509292)(653.2255249,67.96510061)
\curveto(653.29553083,67.95509295)(653.36053077,67.96009295)(653.4205249,67.98010061)
\curveto(653.45053068,67.99009292)(653.49053064,67.99509291)(653.5405249,67.99510061)
\lineto(653.6605249,67.99510061)
\lineto(654.1255249,67.99510061)
\moveto(656.4505249,66.45010061)
\curveto(656.130528,66.55009436)(655.76552836,66.6100943)(655.3555249,66.63010061)
\curveto(654.94552918,66.65009426)(654.53552959,66.66009425)(654.1255249,66.66010061)
\curveto(653.69553043,66.66009425)(653.27553085,66.65009426)(652.8655249,66.63010061)
\curveto(652.45553167,66.6100943)(652.07053206,66.56509434)(651.7105249,66.49510061)
\curveto(651.35053278,66.42509448)(651.0305331,66.31509459)(650.7505249,66.16510061)
\curveto(650.46053367,66.02509488)(650.2255339,65.83009508)(650.0455249,65.58010061)
\curveto(649.93553419,65.42009549)(649.85553427,65.24009567)(649.8055249,65.04010061)
\curveto(649.74553438,64.84009607)(649.71553441,64.59509631)(649.7155249,64.30510061)
\curveto(649.73553439,64.28509662)(649.74553438,64.25009666)(649.7455249,64.20010061)
\curveto(649.73553439,64.15009676)(649.73553439,64.1100968)(649.7455249,64.08010061)
\curveto(649.76553436,64.00009691)(649.78553434,63.92509698)(649.8055249,63.85510061)
\curveto(649.81553431,63.79509711)(649.83553429,63.73009718)(649.8655249,63.66010061)
\curveto(649.98553414,63.39009752)(650.15553397,63.17009774)(650.3755249,63.00010061)
\curveto(650.58553354,62.84009807)(650.8305333,62.7050982)(651.1105249,62.59510061)
\curveto(651.22053291,62.54509836)(651.34053279,62.5050984)(651.4705249,62.47510061)
\curveto(651.59053254,62.45509845)(651.71553241,62.43009848)(651.8455249,62.40010061)
\curveto(651.89553223,62.38009853)(651.95053218,62.37009854)(652.0105249,62.37010061)
\curveto(652.06053207,62.37009854)(652.11053202,62.36509854)(652.1605249,62.35510061)
\curveto(652.25053188,62.34509856)(652.34553178,62.33509857)(652.4455249,62.32510061)
\curveto(652.53553159,62.31509859)(652.6305315,62.3050986)(652.7305249,62.29510061)
\curveto(652.81053132,62.29509861)(652.89553123,62.29009862)(652.9855249,62.28010061)
\lineto(653.2255249,62.28010061)
\lineto(653.4055249,62.28010061)
\curveto(653.43553069,62.27009864)(653.47053066,62.26509864)(653.5105249,62.26510061)
\lineto(653.6455249,62.26510061)
\lineto(654.0955249,62.26510061)
\curveto(654.17552995,62.26509864)(654.26052987,62.26009865)(654.3505249,62.25010061)
\curveto(654.4305297,62.25009866)(654.50552962,62.26009865)(654.5755249,62.28010061)
\lineto(654.8455249,62.28010061)
\curveto(654.86552926,62.28009863)(654.89552923,62.27509863)(654.9355249,62.26510061)
\curveto(654.96552916,62.26509864)(654.99052914,62.27009864)(655.0105249,62.28010061)
\curveto(655.11052902,62.29009862)(655.21052892,62.29509861)(655.3105249,62.29510061)
\curveto(655.40052873,62.3050986)(655.50052863,62.31509859)(655.6105249,62.32510061)
\curveto(655.7305284,62.35509855)(655.85552827,62.37009854)(655.9855249,62.37010061)
\curveto(656.10552802,62.38009853)(656.22052791,62.4050985)(656.3305249,62.44510061)
\curveto(656.6305275,62.52509838)(656.89552723,62.6100983)(657.1255249,62.70010061)
\curveto(657.35552677,62.80009811)(657.57052656,62.94509796)(657.7705249,63.13510061)
\curveto(657.97052616,63.34509756)(658.12052601,63.6100973)(658.2205249,63.93010061)
\curveto(658.24052589,63.97009694)(658.25052588,64.0050969)(658.2505249,64.03510061)
\curveto(658.24052589,64.07509683)(658.24552588,64.12009679)(658.2655249,64.17010061)
\curveto(658.27552585,64.2100967)(658.28552584,64.28009663)(658.2955249,64.38010061)
\curveto(658.30552582,64.49009642)(658.30052583,64.57509633)(658.2805249,64.63510061)
\curveto(658.26052587,64.7050962)(658.25052588,64.77509613)(658.2505249,64.84510061)
\curveto(658.24052589,64.91509599)(658.2255259,64.98009593)(658.2055249,65.04010061)
\curveto(658.14552598,65.24009567)(658.06052607,65.42009549)(657.9505249,65.58010061)
\curveto(657.9305262,65.6100953)(657.91052622,65.63509527)(657.8905249,65.65510061)
\lineto(657.8305249,65.71510061)
\curveto(657.81052632,65.75509515)(657.77052636,65.8050951)(657.7105249,65.86510061)
\curveto(657.57052656,65.96509494)(657.44052669,66.05009486)(657.3205249,66.12010061)
\curveto(657.20052693,66.19009472)(657.05552707,66.26009465)(656.8855249,66.33010061)
\curveto(656.81552731,66.36009455)(656.74552738,66.38009453)(656.6755249,66.39010061)
\curveto(656.60552752,66.4100945)(656.5305276,66.43009448)(656.4505249,66.45010061)
}
}
{
\newrgbcolor{curcolor}{0 0 0}
\pscustom[linestyle=none,fillstyle=solid,fillcolor=curcolor]
{
\newpath
\moveto(648.6055249,72.59470998)
\curveto(648.59553553,73.28470535)(648.71553541,73.88470475)(648.9655249,74.39470998)
\curveto(649.21553491,74.91470372)(649.55053458,75.30970332)(649.9705249,75.57970998)
\curveto(650.05053408,75.629703)(650.14053399,75.67470296)(650.2405249,75.71470998)
\curveto(650.3305338,75.75470288)(650.4255337,75.79970283)(650.5255249,75.84970998)
\curveto(650.6255335,75.88970274)(650.7255334,75.91970271)(650.8255249,75.93970998)
\curveto(650.9255332,75.95970267)(651.0305331,75.97970265)(651.1405249,75.99970998)
\curveto(651.19053294,76.01970261)(651.23553289,76.02470261)(651.2755249,76.01470998)
\curveto(651.31553281,76.00470263)(651.36053277,76.00970262)(651.4105249,76.02970998)
\curveto(651.46053267,76.03970259)(651.54553258,76.04470259)(651.6655249,76.04470998)
\curveto(651.77553235,76.04470259)(651.86053227,76.03970259)(651.9205249,76.02970998)
\curveto(651.98053215,76.00970262)(652.04053209,75.99970263)(652.1005249,75.99970998)
\curveto(652.16053197,76.00970262)(652.22053191,76.00470263)(652.2805249,75.98470998)
\curveto(652.42053171,75.94470269)(652.55553157,75.90970272)(652.6855249,75.87970998)
\curveto(652.81553131,75.84970278)(652.94053119,75.80970282)(653.0605249,75.75970998)
\curveto(653.20053093,75.69970293)(653.3255308,75.629703)(653.4355249,75.54970998)
\curveto(653.54553058,75.47970315)(653.65553047,75.40470323)(653.7655249,75.32470998)
\lineto(653.8255249,75.26470998)
\curveto(653.84553028,75.25470338)(653.86553026,75.23970339)(653.8855249,75.21970998)
\curveto(654.04553008,75.09970353)(654.19052994,74.96470367)(654.3205249,74.81470998)
\curveto(654.45052968,74.66470397)(654.57552955,74.50470413)(654.6955249,74.33470998)
\curveto(654.91552921,74.02470461)(655.12052901,73.7297049)(655.3105249,73.44970998)
\curveto(655.45052868,73.21970541)(655.58552854,72.98970564)(655.7155249,72.75970998)
\curveto(655.84552828,72.53970609)(655.98052815,72.31970631)(656.1205249,72.09970998)
\curveto(656.29052784,71.84970678)(656.47052766,71.60970702)(656.6605249,71.37970998)
\curveto(656.85052728,71.15970747)(657.07552705,70.96970766)(657.3355249,70.80970998)
\curveto(657.39552673,70.76970786)(657.45552667,70.7347079)(657.5155249,70.70470998)
\curveto(657.56552656,70.67470796)(657.6305265,70.64470799)(657.7105249,70.61470998)
\curveto(657.78052635,70.59470804)(657.84052629,70.58970804)(657.8905249,70.59970998)
\curveto(657.96052617,70.61970801)(658.01552611,70.65470798)(658.0555249,70.70470998)
\curveto(658.08552604,70.75470788)(658.10552602,70.81470782)(658.1155249,70.88470998)
\lineto(658.1155249,71.12470998)
\lineto(658.1155249,71.87470998)
\lineto(658.1155249,74.67970998)
\lineto(658.1155249,75.33970998)
\curveto(658.11552601,75.4297032)(658.12052601,75.51470312)(658.1305249,75.59470998)
\curveto(658.130526,75.67470296)(658.15052598,75.73970289)(658.1905249,75.78970998)
\curveto(658.2305259,75.83970279)(658.30552582,75.87970275)(658.4155249,75.90970998)
\curveto(658.51552561,75.94970268)(658.61552551,75.95970267)(658.7155249,75.93970998)
\lineto(658.8505249,75.93970998)
\curveto(658.92052521,75.91970271)(658.98052515,75.89970273)(659.0305249,75.87970998)
\curveto(659.08052505,75.85970277)(659.12052501,75.82470281)(659.1505249,75.77470998)
\curveto(659.19052494,75.72470291)(659.21052492,75.65470298)(659.2105249,75.56470998)
\lineto(659.2105249,75.29470998)
\lineto(659.2105249,74.39470998)
\lineto(659.2105249,70.88470998)
\lineto(659.2105249,69.81970998)
\curveto(659.21052492,69.73970889)(659.21552491,69.64970898)(659.2255249,69.54970998)
\curveto(659.2255249,69.44970918)(659.21552491,69.36470927)(659.1955249,69.29470998)
\curveto(659.125525,69.08470955)(658.94552518,69.01970961)(658.6555249,69.09970998)
\curveto(658.61552551,69.10970952)(658.58052555,69.10970952)(658.5505249,69.09970998)
\curveto(658.51052562,69.09970953)(658.46552566,69.10970952)(658.4155249,69.12970998)
\curveto(658.33552579,69.14970948)(658.25052588,69.16970946)(658.1605249,69.18970998)
\curveto(658.07052606,69.20970942)(657.98552614,69.2347094)(657.9055249,69.26470998)
\curveto(657.41552671,69.42470921)(657.00052713,69.62470901)(656.6605249,69.86470998)
\curveto(656.41052772,70.04470859)(656.18552794,70.24970838)(655.9855249,70.47970998)
\curveto(655.77552835,70.70970792)(655.58052855,70.94970768)(655.4005249,71.19970998)
\curveto(655.22052891,71.45970717)(655.05052908,71.72470691)(654.8905249,71.99470998)
\curveto(654.72052941,72.27470636)(654.54552958,72.54470609)(654.3655249,72.80470998)
\curveto(654.28552984,72.91470572)(654.21052992,73.01970561)(654.1405249,73.11970998)
\curveto(654.07053006,73.2297054)(653.99553013,73.33970529)(653.9155249,73.44970998)
\curveto(653.88553024,73.48970514)(653.85553027,73.52470511)(653.8255249,73.55470998)
\curveto(653.78553034,73.59470504)(653.75553037,73.634705)(653.7355249,73.67470998)
\curveto(653.6255305,73.81470482)(653.50053063,73.93970469)(653.3605249,74.04970998)
\curveto(653.3305308,74.06970456)(653.30553082,74.09470454)(653.2855249,74.12470998)
\curveto(653.25553087,74.15470448)(653.2255309,74.17970445)(653.1955249,74.19970998)
\curveto(653.09553103,74.27970435)(652.99553113,74.34470429)(652.8955249,74.39470998)
\curveto(652.79553133,74.45470418)(652.68553144,74.50970412)(652.5655249,74.55970998)
\curveto(652.49553163,74.58970404)(652.42053171,74.60970402)(652.3405249,74.61970998)
\lineto(652.1005249,74.67970998)
\lineto(652.0105249,74.67970998)
\curveto(651.98053215,74.68970394)(651.95053218,74.69470394)(651.9205249,74.69470998)
\curveto(651.85053228,74.71470392)(651.75553237,74.71970391)(651.6355249,74.70970998)
\curveto(651.50553262,74.70970392)(651.40553272,74.69970393)(651.3355249,74.67970998)
\curveto(651.25553287,74.65970397)(651.18053295,74.63970399)(651.1105249,74.61970998)
\curveto(651.0305331,74.60970402)(650.95053318,74.58970404)(650.8705249,74.55970998)
\curveto(650.6305335,74.44970418)(650.4305337,74.29970433)(650.2705249,74.10970998)
\curveto(650.10053403,73.9297047)(649.96053417,73.70970492)(649.8505249,73.44970998)
\curveto(649.8305343,73.37970525)(649.81553431,73.30970532)(649.8055249,73.23970998)
\curveto(649.78553434,73.16970546)(649.76553436,73.09470554)(649.7455249,73.01470998)
\curveto(649.7255344,72.9347057)(649.71553441,72.82470581)(649.7155249,72.68470998)
\curveto(649.71553441,72.55470608)(649.7255344,72.44970618)(649.7455249,72.36970998)
\curveto(649.75553437,72.30970632)(649.76053437,72.25470638)(649.7605249,72.20470998)
\curveto(649.76053437,72.15470648)(649.77053436,72.10470653)(649.7905249,72.05470998)
\curveto(649.8305343,71.95470668)(649.87053426,71.85970677)(649.9105249,71.76970998)
\curveto(649.95053418,71.68970694)(649.99553413,71.60970702)(650.0455249,71.52970998)
\curveto(650.06553406,71.49970713)(650.09053404,71.46970716)(650.1205249,71.43970998)
\curveto(650.15053398,71.41970721)(650.17553395,71.39470724)(650.1955249,71.36470998)
\lineto(650.2705249,71.28970998)
\curveto(650.29053384,71.25970737)(650.31053382,71.2347074)(650.3305249,71.21470998)
\lineto(650.5405249,71.06470998)
\curveto(650.60053353,71.02470761)(650.66553346,70.97970765)(650.7355249,70.92970998)
\curveto(650.8255333,70.86970776)(650.9305332,70.81970781)(651.0505249,70.77970998)
\curveto(651.16053297,70.74970788)(651.27053286,70.71470792)(651.3805249,70.67470998)
\curveto(651.49053264,70.634708)(651.63553249,70.60970802)(651.8155249,70.59970998)
\curveto(651.98553214,70.58970804)(652.11053202,70.55970807)(652.1905249,70.50970998)
\curveto(652.27053186,70.45970817)(652.31553181,70.38470825)(652.3255249,70.28470998)
\curveto(652.33553179,70.18470845)(652.34053179,70.07470856)(652.3405249,69.95470998)
\curveto(652.34053179,69.91470872)(652.34553178,69.87470876)(652.3555249,69.83470998)
\curveto(652.35553177,69.79470884)(652.35053178,69.75970887)(652.3405249,69.72970998)
\curveto(652.32053181,69.67970895)(652.31053182,69.629709)(652.3105249,69.57970998)
\curveto(652.31053182,69.53970909)(652.30053183,69.49970913)(652.2805249,69.45970998)
\curveto(652.22053191,69.36970926)(652.08553204,69.32470931)(651.8755249,69.32470998)
\lineto(651.7555249,69.32470998)
\curveto(651.69553243,69.3347093)(651.63553249,69.33970929)(651.5755249,69.33970998)
\curveto(651.50553262,69.34970928)(651.44053269,69.35970927)(651.3805249,69.36970998)
\curveto(651.27053286,69.38970924)(651.17053296,69.40970922)(651.0805249,69.42970998)
\curveto(650.98053315,69.44970918)(650.88553324,69.47970915)(650.7955249,69.51970998)
\curveto(650.7255334,69.53970909)(650.66553346,69.55970907)(650.6155249,69.57970998)
\lineto(650.4355249,69.63970998)
\curveto(650.17553395,69.75970887)(649.9305342,69.91470872)(649.7005249,70.10470998)
\curveto(649.47053466,70.30470833)(649.28553484,70.51970811)(649.1455249,70.74970998)
\curveto(649.06553506,70.85970777)(649.00053513,70.97470766)(648.9505249,71.09470998)
\lineto(648.8005249,71.48470998)
\curveto(648.75053538,71.59470704)(648.72053541,71.70970692)(648.7105249,71.82970998)
\curveto(648.69053544,71.94970668)(648.66553546,72.07470656)(648.6355249,72.20470998)
\curveto(648.63553549,72.27470636)(648.63553549,72.33970629)(648.6355249,72.39970998)
\curveto(648.6255355,72.45970617)(648.61553551,72.52470611)(648.6055249,72.59470998)
}
}
{
\newrgbcolor{curcolor}{0 0 0}
\pscustom[linestyle=none,fillstyle=solid,fillcolor=curcolor]
{
\newpath
\moveto(657.5755249,78.63431936)
\lineto(657.5755249,79.26431936)
\lineto(657.5755249,79.45931936)
\curveto(657.57552655,79.52931683)(657.58552654,79.58931677)(657.6055249,79.63931936)
\curveto(657.64552648,79.70931665)(657.68552644,79.7593166)(657.7255249,79.78931936)
\curveto(657.77552635,79.82931653)(657.84052629,79.84931651)(657.9205249,79.84931936)
\curveto(658.00052613,79.8593165)(658.08552604,79.86431649)(658.1755249,79.86431936)
\lineto(658.8955249,79.86431936)
\curveto(659.37552475,79.86431649)(659.78552434,79.80431655)(660.1255249,79.68431936)
\curveto(660.46552366,79.56431679)(660.74052339,79.36931699)(660.9505249,79.09931936)
\curveto(661.00052313,79.02931733)(661.04552308,78.9593174)(661.0855249,78.88931936)
\curveto(661.13552299,78.82931753)(661.18052295,78.7543176)(661.2205249,78.66431936)
\curveto(661.2305229,78.64431771)(661.24052289,78.61431774)(661.2505249,78.57431936)
\curveto(661.27052286,78.53431782)(661.27552285,78.48931787)(661.2655249,78.43931936)
\curveto(661.23552289,78.34931801)(661.16052297,78.29431806)(661.0405249,78.27431936)
\curveto(660.9305232,78.2543181)(660.83552329,78.26931809)(660.7555249,78.31931936)
\curveto(660.68552344,78.34931801)(660.62052351,78.39431796)(660.5605249,78.45431936)
\curveto(660.51052362,78.52431783)(660.46052367,78.58931777)(660.4105249,78.64931936)
\curveto(660.36052377,78.71931764)(660.28552384,78.77931758)(660.1855249,78.82931936)
\curveto(660.09552403,78.88931747)(660.00552412,78.93931742)(659.9155249,78.97931936)
\curveto(659.88552424,78.99931736)(659.8255243,79.02431733)(659.7355249,79.05431936)
\curveto(659.65552447,79.08431727)(659.58552454,79.08931727)(659.5255249,79.06931936)
\curveto(659.38552474,79.03931732)(659.29552483,78.97931738)(659.2555249,78.88931936)
\curveto(659.2255249,78.80931755)(659.21052492,78.71931764)(659.2105249,78.61931936)
\curveto(659.21052492,78.51931784)(659.18552494,78.43431792)(659.1355249,78.36431936)
\curveto(659.06552506,78.27431808)(658.9255252,78.22931813)(658.7155249,78.22931936)
\lineto(658.1605249,78.22931936)
\lineto(657.9355249,78.22931936)
\curveto(657.85552627,78.23931812)(657.79052634,78.2593181)(657.7405249,78.28931936)
\curveto(657.66052647,78.34931801)(657.61552651,78.41931794)(657.6055249,78.49931936)
\curveto(657.59552653,78.51931784)(657.59052654,78.53931782)(657.5905249,78.55931936)
\curveto(657.59052654,78.58931777)(657.58552654,78.61431774)(657.5755249,78.63431936)
}
}
{
\newrgbcolor{curcolor}{0 0 0}
\pscustom[linestyle=none,fillstyle=solid,fillcolor=curcolor]
{
}
}
{
\newrgbcolor{curcolor}{0 0 0}
\pscustom[linestyle=none,fillstyle=solid,fillcolor=curcolor]
{
\newpath
\moveto(648.6055249,89.26463186)
\curveto(648.59553553,89.95462722)(648.71553541,90.55462662)(648.9655249,91.06463186)
\curveto(649.21553491,91.58462559)(649.55053458,91.9796252)(649.9705249,92.24963186)
\curveto(650.05053408,92.29962488)(650.14053399,92.34462483)(650.2405249,92.38463186)
\curveto(650.3305338,92.42462475)(650.4255337,92.46962471)(650.5255249,92.51963186)
\curveto(650.6255335,92.55962462)(650.7255334,92.58962459)(650.8255249,92.60963186)
\curveto(650.9255332,92.62962455)(651.0305331,92.64962453)(651.1405249,92.66963186)
\curveto(651.19053294,92.68962449)(651.23553289,92.69462448)(651.2755249,92.68463186)
\curveto(651.31553281,92.6746245)(651.36053277,92.6796245)(651.4105249,92.69963186)
\curveto(651.46053267,92.70962447)(651.54553258,92.71462446)(651.6655249,92.71463186)
\curveto(651.77553235,92.71462446)(651.86053227,92.70962447)(651.9205249,92.69963186)
\curveto(651.98053215,92.6796245)(652.04053209,92.66962451)(652.1005249,92.66963186)
\curveto(652.16053197,92.6796245)(652.22053191,92.6746245)(652.2805249,92.65463186)
\curveto(652.42053171,92.61462456)(652.55553157,92.5796246)(652.6855249,92.54963186)
\curveto(652.81553131,92.51962466)(652.94053119,92.4796247)(653.0605249,92.42963186)
\curveto(653.20053093,92.36962481)(653.3255308,92.29962488)(653.4355249,92.21963186)
\curveto(653.54553058,92.14962503)(653.65553047,92.0746251)(653.7655249,91.99463186)
\lineto(653.8255249,91.93463186)
\curveto(653.84553028,91.92462525)(653.86553026,91.90962527)(653.8855249,91.88963186)
\curveto(654.04553008,91.76962541)(654.19052994,91.63462554)(654.3205249,91.48463186)
\curveto(654.45052968,91.33462584)(654.57552955,91.174626)(654.6955249,91.00463186)
\curveto(654.91552921,90.69462648)(655.12052901,90.39962678)(655.3105249,90.11963186)
\curveto(655.45052868,89.88962729)(655.58552854,89.65962752)(655.7155249,89.42963186)
\curveto(655.84552828,89.20962797)(655.98052815,88.98962819)(656.1205249,88.76963186)
\curveto(656.29052784,88.51962866)(656.47052766,88.2796289)(656.6605249,88.04963186)
\curveto(656.85052728,87.82962935)(657.07552705,87.63962954)(657.3355249,87.47963186)
\curveto(657.39552673,87.43962974)(657.45552667,87.40462977)(657.5155249,87.37463186)
\curveto(657.56552656,87.34462983)(657.6305265,87.31462986)(657.7105249,87.28463186)
\curveto(657.78052635,87.26462991)(657.84052629,87.25962992)(657.8905249,87.26963186)
\curveto(657.96052617,87.28962989)(658.01552611,87.32462985)(658.0555249,87.37463186)
\curveto(658.08552604,87.42462975)(658.10552602,87.48462969)(658.1155249,87.55463186)
\lineto(658.1155249,87.79463186)
\lineto(658.1155249,88.54463186)
\lineto(658.1155249,91.34963186)
\lineto(658.1155249,92.00963186)
\curveto(658.11552601,92.09962508)(658.12052601,92.18462499)(658.1305249,92.26463186)
\curveto(658.130526,92.34462483)(658.15052598,92.40962477)(658.1905249,92.45963186)
\curveto(658.2305259,92.50962467)(658.30552582,92.54962463)(658.4155249,92.57963186)
\curveto(658.51552561,92.61962456)(658.61552551,92.62962455)(658.7155249,92.60963186)
\lineto(658.8505249,92.60963186)
\curveto(658.92052521,92.58962459)(658.98052515,92.56962461)(659.0305249,92.54963186)
\curveto(659.08052505,92.52962465)(659.12052501,92.49462468)(659.1505249,92.44463186)
\curveto(659.19052494,92.39462478)(659.21052492,92.32462485)(659.2105249,92.23463186)
\lineto(659.2105249,91.96463186)
\lineto(659.2105249,91.06463186)
\lineto(659.2105249,87.55463186)
\lineto(659.2105249,86.48963186)
\curveto(659.21052492,86.40963077)(659.21552491,86.31963086)(659.2255249,86.21963186)
\curveto(659.2255249,86.11963106)(659.21552491,86.03463114)(659.1955249,85.96463186)
\curveto(659.125525,85.75463142)(658.94552518,85.68963149)(658.6555249,85.76963186)
\curveto(658.61552551,85.7796314)(658.58052555,85.7796314)(658.5505249,85.76963186)
\curveto(658.51052562,85.76963141)(658.46552566,85.7796314)(658.4155249,85.79963186)
\curveto(658.33552579,85.81963136)(658.25052588,85.83963134)(658.1605249,85.85963186)
\curveto(658.07052606,85.8796313)(657.98552614,85.90463127)(657.9055249,85.93463186)
\curveto(657.41552671,86.09463108)(657.00052713,86.29463088)(656.6605249,86.53463186)
\curveto(656.41052772,86.71463046)(656.18552794,86.91963026)(655.9855249,87.14963186)
\curveto(655.77552835,87.3796298)(655.58052855,87.61962956)(655.4005249,87.86963186)
\curveto(655.22052891,88.12962905)(655.05052908,88.39462878)(654.8905249,88.66463186)
\curveto(654.72052941,88.94462823)(654.54552958,89.21462796)(654.3655249,89.47463186)
\curveto(654.28552984,89.58462759)(654.21052992,89.68962749)(654.1405249,89.78963186)
\curveto(654.07053006,89.89962728)(653.99553013,90.00962717)(653.9155249,90.11963186)
\curveto(653.88553024,90.15962702)(653.85553027,90.19462698)(653.8255249,90.22463186)
\curveto(653.78553034,90.26462691)(653.75553037,90.30462687)(653.7355249,90.34463186)
\curveto(653.6255305,90.48462669)(653.50053063,90.60962657)(653.3605249,90.71963186)
\curveto(653.3305308,90.73962644)(653.30553082,90.76462641)(653.2855249,90.79463186)
\curveto(653.25553087,90.82462635)(653.2255309,90.84962633)(653.1955249,90.86963186)
\curveto(653.09553103,90.94962623)(652.99553113,91.01462616)(652.8955249,91.06463186)
\curveto(652.79553133,91.12462605)(652.68553144,91.179626)(652.5655249,91.22963186)
\curveto(652.49553163,91.25962592)(652.42053171,91.2796259)(652.3405249,91.28963186)
\lineto(652.1005249,91.34963186)
\lineto(652.0105249,91.34963186)
\curveto(651.98053215,91.35962582)(651.95053218,91.36462581)(651.9205249,91.36463186)
\curveto(651.85053228,91.38462579)(651.75553237,91.38962579)(651.6355249,91.37963186)
\curveto(651.50553262,91.3796258)(651.40553272,91.36962581)(651.3355249,91.34963186)
\curveto(651.25553287,91.32962585)(651.18053295,91.30962587)(651.1105249,91.28963186)
\curveto(651.0305331,91.2796259)(650.95053318,91.25962592)(650.8705249,91.22963186)
\curveto(650.6305335,91.11962606)(650.4305337,90.96962621)(650.2705249,90.77963186)
\curveto(650.10053403,90.59962658)(649.96053417,90.3796268)(649.8505249,90.11963186)
\curveto(649.8305343,90.04962713)(649.81553431,89.9796272)(649.8055249,89.90963186)
\curveto(649.78553434,89.83962734)(649.76553436,89.76462741)(649.7455249,89.68463186)
\curveto(649.7255344,89.60462757)(649.71553441,89.49462768)(649.7155249,89.35463186)
\curveto(649.71553441,89.22462795)(649.7255344,89.11962806)(649.7455249,89.03963186)
\curveto(649.75553437,88.9796282)(649.76053437,88.92462825)(649.7605249,88.87463186)
\curveto(649.76053437,88.82462835)(649.77053436,88.7746284)(649.7905249,88.72463186)
\curveto(649.8305343,88.62462855)(649.87053426,88.52962865)(649.9105249,88.43963186)
\curveto(649.95053418,88.35962882)(649.99553413,88.2796289)(650.0455249,88.19963186)
\curveto(650.06553406,88.16962901)(650.09053404,88.13962904)(650.1205249,88.10963186)
\curveto(650.15053398,88.08962909)(650.17553395,88.06462911)(650.1955249,88.03463186)
\lineto(650.2705249,87.95963186)
\curveto(650.29053384,87.92962925)(650.31053382,87.90462927)(650.3305249,87.88463186)
\lineto(650.5405249,87.73463186)
\curveto(650.60053353,87.69462948)(650.66553346,87.64962953)(650.7355249,87.59963186)
\curveto(650.8255333,87.53962964)(650.9305332,87.48962969)(651.0505249,87.44963186)
\curveto(651.16053297,87.41962976)(651.27053286,87.38462979)(651.3805249,87.34463186)
\curveto(651.49053264,87.30462987)(651.63553249,87.2796299)(651.8155249,87.26963186)
\curveto(651.98553214,87.25962992)(652.11053202,87.22962995)(652.1905249,87.17963186)
\curveto(652.27053186,87.12963005)(652.31553181,87.05463012)(652.3255249,86.95463186)
\curveto(652.33553179,86.85463032)(652.34053179,86.74463043)(652.3405249,86.62463186)
\curveto(652.34053179,86.58463059)(652.34553178,86.54463063)(652.3555249,86.50463186)
\curveto(652.35553177,86.46463071)(652.35053178,86.42963075)(652.3405249,86.39963186)
\curveto(652.32053181,86.34963083)(652.31053182,86.29963088)(652.3105249,86.24963186)
\curveto(652.31053182,86.20963097)(652.30053183,86.16963101)(652.2805249,86.12963186)
\curveto(652.22053191,86.03963114)(652.08553204,85.99463118)(651.8755249,85.99463186)
\lineto(651.7555249,85.99463186)
\curveto(651.69553243,86.00463117)(651.63553249,86.00963117)(651.5755249,86.00963186)
\curveto(651.50553262,86.01963116)(651.44053269,86.02963115)(651.3805249,86.03963186)
\curveto(651.27053286,86.05963112)(651.17053296,86.0796311)(651.0805249,86.09963186)
\curveto(650.98053315,86.11963106)(650.88553324,86.14963103)(650.7955249,86.18963186)
\curveto(650.7255334,86.20963097)(650.66553346,86.22963095)(650.6155249,86.24963186)
\lineto(650.4355249,86.30963186)
\curveto(650.17553395,86.42963075)(649.9305342,86.58463059)(649.7005249,86.77463186)
\curveto(649.47053466,86.9746302)(649.28553484,87.18962999)(649.1455249,87.41963186)
\curveto(649.06553506,87.52962965)(649.00053513,87.64462953)(648.9505249,87.76463186)
\lineto(648.8005249,88.15463186)
\curveto(648.75053538,88.26462891)(648.72053541,88.3796288)(648.7105249,88.49963186)
\curveto(648.69053544,88.61962856)(648.66553546,88.74462843)(648.6355249,88.87463186)
\curveto(648.63553549,88.94462823)(648.63553549,89.00962817)(648.6355249,89.06963186)
\curveto(648.6255355,89.12962805)(648.61553551,89.19462798)(648.6055249,89.26463186)
}
}
{
\newrgbcolor{curcolor}{0 0 0}
\pscustom[linestyle=none,fillstyle=solid,fillcolor=curcolor]
{
\newpath
\moveto(654.1255249,101.36424123)
\lineto(654.3805249,101.36424123)
\curveto(654.46052967,101.37423353)(654.53552959,101.36923353)(654.6055249,101.34924123)
\lineto(654.8455249,101.34924123)
\lineto(655.0105249,101.34924123)
\curveto(655.11052902,101.32923357)(655.21552891,101.31923358)(655.3255249,101.31924123)
\curveto(655.4255287,101.31923358)(655.5255286,101.30923359)(655.6255249,101.28924123)
\lineto(655.7755249,101.28924123)
\curveto(655.91552821,101.25923364)(656.05552807,101.23923366)(656.1955249,101.22924123)
\curveto(656.3255278,101.21923368)(656.45552767,101.19423371)(656.5855249,101.15424123)
\curveto(656.66552746,101.13423377)(656.75052738,101.11423379)(656.8405249,101.09424123)
\lineto(657.0805249,101.03424123)
\lineto(657.3805249,100.91424123)
\curveto(657.47052666,100.88423402)(657.56052657,100.84923405)(657.6505249,100.80924123)
\curveto(657.87052626,100.70923419)(658.08552604,100.57423433)(658.2955249,100.40424123)
\curveto(658.50552562,100.24423466)(658.67552545,100.06923483)(658.8055249,99.87924123)
\curveto(658.84552528,99.82923507)(658.88552524,99.76923513)(658.9255249,99.69924123)
\curveto(658.95552517,99.63923526)(658.99052514,99.57923532)(659.0305249,99.51924123)
\curveto(659.08052505,99.43923546)(659.12052501,99.34423556)(659.1505249,99.23424123)
\curveto(659.18052495,99.12423578)(659.21052492,99.01923588)(659.2405249,98.91924123)
\curveto(659.28052485,98.80923609)(659.30552482,98.6992362)(659.3155249,98.58924123)
\curveto(659.3255248,98.47923642)(659.34052479,98.36423654)(659.3605249,98.24424123)
\curveto(659.37052476,98.2042367)(659.37052476,98.15923674)(659.3605249,98.10924123)
\curveto(659.36052477,98.06923683)(659.36552476,98.02923687)(659.3755249,97.98924123)
\curveto(659.38552474,97.94923695)(659.39052474,97.89423701)(659.3905249,97.82424123)
\curveto(659.39052474,97.75423715)(659.38552474,97.7042372)(659.3755249,97.67424123)
\curveto(659.35552477,97.62423728)(659.35052478,97.57923732)(659.3605249,97.53924123)
\curveto(659.37052476,97.4992374)(659.37052476,97.46423744)(659.3605249,97.43424123)
\lineto(659.3605249,97.34424123)
\curveto(659.34052479,97.28423762)(659.3255248,97.21923768)(659.3155249,97.14924123)
\curveto(659.31552481,97.08923781)(659.31052482,97.02423788)(659.3005249,96.95424123)
\curveto(659.25052488,96.78423812)(659.20052493,96.62423828)(659.1505249,96.47424123)
\curveto(659.10052503,96.32423858)(659.03552509,96.17923872)(658.9555249,96.03924123)
\curveto(658.91552521,95.98923891)(658.88552524,95.93423897)(658.8655249,95.87424123)
\curveto(658.83552529,95.82423908)(658.80052533,95.77423913)(658.7605249,95.72424123)
\curveto(658.58052555,95.48423942)(658.36052577,95.28423962)(658.1005249,95.12424123)
\curveto(657.84052629,94.96423994)(657.55552657,94.82424008)(657.2455249,94.70424123)
\curveto(657.10552702,94.64424026)(656.96552716,94.5992403)(656.8255249,94.56924123)
\curveto(656.67552745,94.53924036)(656.52052761,94.5042404)(656.3605249,94.46424123)
\curveto(656.25052788,94.44424046)(656.14052799,94.42924047)(656.0305249,94.41924123)
\curveto(655.92052821,94.40924049)(655.81052832,94.39424051)(655.7005249,94.37424123)
\curveto(655.66052847,94.36424054)(655.62052851,94.35924054)(655.5805249,94.35924123)
\curveto(655.54052859,94.36924053)(655.50052863,94.36924053)(655.4605249,94.35924123)
\curveto(655.41052872,94.34924055)(655.36052877,94.34424056)(655.3105249,94.34424123)
\lineto(655.1455249,94.34424123)
\curveto(655.09552903,94.32424058)(655.04552908,94.31924058)(654.9955249,94.32924123)
\curveto(654.93552919,94.33924056)(654.88052925,94.33924056)(654.8305249,94.32924123)
\curveto(654.79052934,94.31924058)(654.74552938,94.31924058)(654.6955249,94.32924123)
\curveto(654.64552948,94.33924056)(654.59552953,94.33424057)(654.5455249,94.31424123)
\curveto(654.47552965,94.29424061)(654.40052973,94.28924061)(654.3205249,94.29924123)
\curveto(654.2305299,94.30924059)(654.14552998,94.31424059)(654.0655249,94.31424123)
\curveto(653.97553015,94.31424059)(653.87553025,94.30924059)(653.7655249,94.29924123)
\curveto(653.64553048,94.28924061)(653.54553058,94.29424061)(653.4655249,94.31424123)
\lineto(653.1805249,94.31424123)
\lineto(652.5505249,94.35924123)
\curveto(652.45053168,94.36924053)(652.35553177,94.37924052)(652.2655249,94.38924123)
\lineto(651.9655249,94.41924123)
\curveto(651.91553221,94.43924046)(651.86553226,94.44424046)(651.8155249,94.43424123)
\curveto(651.75553237,94.43424047)(651.70053243,94.44424046)(651.6505249,94.46424123)
\curveto(651.48053265,94.51424039)(651.31553281,94.55424035)(651.1555249,94.58424123)
\curveto(650.98553314,94.61424029)(650.8255333,94.66424024)(650.6755249,94.73424123)
\curveto(650.21553391,94.92423998)(649.84053429,95.14423976)(649.5505249,95.39424123)
\curveto(649.26053487,95.65423925)(649.01553511,96.01423889)(648.8155249,96.47424123)
\curveto(648.76553536,96.6042383)(648.7305354,96.73423817)(648.7105249,96.86424123)
\curveto(648.69053544,97.0042379)(648.66553546,97.14423776)(648.6355249,97.28424123)
\curveto(648.6255355,97.35423755)(648.62053551,97.41923748)(648.6205249,97.47924123)
\curveto(648.62053551,97.53923736)(648.61553551,97.6042373)(648.6055249,97.67424123)
\curveto(648.58553554,98.5042364)(648.73553539,99.17423573)(649.0555249,99.68424123)
\curveto(649.36553476,100.19423471)(649.80553432,100.57423433)(650.3755249,100.82424123)
\curveto(650.49553363,100.87423403)(650.62053351,100.91923398)(650.7505249,100.95924123)
\curveto(650.88053325,100.9992339)(651.01553311,101.04423386)(651.1555249,101.09424123)
\curveto(651.23553289,101.11423379)(651.32053281,101.12923377)(651.4105249,101.13924123)
\lineto(651.6505249,101.19924123)
\curveto(651.76053237,101.22923367)(651.87053226,101.24423366)(651.9805249,101.24424123)
\curveto(652.09053204,101.25423365)(652.20053193,101.26923363)(652.3105249,101.28924123)
\curveto(652.36053177,101.30923359)(652.40553172,101.31423359)(652.4455249,101.30424123)
\curveto(652.48553164,101.3042336)(652.5255316,101.30923359)(652.5655249,101.31924123)
\curveto(652.61553151,101.32923357)(652.67053146,101.32923357)(652.7305249,101.31924123)
\curveto(652.78053135,101.31923358)(652.8305313,101.32423358)(652.8805249,101.33424123)
\lineto(653.0155249,101.33424123)
\curveto(653.07553105,101.35423355)(653.14553098,101.35423355)(653.2255249,101.33424123)
\curveto(653.29553083,101.32423358)(653.36053077,101.32923357)(653.4205249,101.34924123)
\curveto(653.45053068,101.35923354)(653.49053064,101.36423354)(653.5405249,101.36424123)
\lineto(653.6605249,101.36424123)
\lineto(654.1255249,101.36424123)
\moveto(656.4505249,99.81924123)
\curveto(656.130528,99.91923498)(655.76552836,99.97923492)(655.3555249,99.99924123)
\curveto(654.94552918,100.01923488)(654.53552959,100.02923487)(654.1255249,100.02924123)
\curveto(653.69553043,100.02923487)(653.27553085,100.01923488)(652.8655249,99.99924123)
\curveto(652.45553167,99.97923492)(652.07053206,99.93423497)(651.7105249,99.86424123)
\curveto(651.35053278,99.79423511)(651.0305331,99.68423522)(650.7505249,99.53424123)
\curveto(650.46053367,99.39423551)(650.2255339,99.1992357)(650.0455249,98.94924123)
\curveto(649.93553419,98.78923611)(649.85553427,98.60923629)(649.8055249,98.40924123)
\curveto(649.74553438,98.20923669)(649.71553441,97.96423694)(649.7155249,97.67424123)
\curveto(649.73553439,97.65423725)(649.74553438,97.61923728)(649.7455249,97.56924123)
\curveto(649.73553439,97.51923738)(649.73553439,97.47923742)(649.7455249,97.44924123)
\curveto(649.76553436,97.36923753)(649.78553434,97.29423761)(649.8055249,97.22424123)
\curveto(649.81553431,97.16423774)(649.83553429,97.0992378)(649.8655249,97.02924123)
\curveto(649.98553414,96.75923814)(650.15553397,96.53923836)(650.3755249,96.36924123)
\curveto(650.58553354,96.20923869)(650.8305333,96.07423883)(651.1105249,95.96424123)
\curveto(651.22053291,95.91423899)(651.34053279,95.87423903)(651.4705249,95.84424123)
\curveto(651.59053254,95.82423908)(651.71553241,95.7992391)(651.8455249,95.76924123)
\curveto(651.89553223,95.74923915)(651.95053218,95.73923916)(652.0105249,95.73924123)
\curveto(652.06053207,95.73923916)(652.11053202,95.73423917)(652.1605249,95.72424123)
\curveto(652.25053188,95.71423919)(652.34553178,95.7042392)(652.4455249,95.69424123)
\curveto(652.53553159,95.68423922)(652.6305315,95.67423923)(652.7305249,95.66424123)
\curveto(652.81053132,95.66423924)(652.89553123,95.65923924)(652.9855249,95.64924123)
\lineto(653.2255249,95.64924123)
\lineto(653.4055249,95.64924123)
\curveto(653.43553069,95.63923926)(653.47053066,95.63423927)(653.5105249,95.63424123)
\lineto(653.6455249,95.63424123)
\lineto(654.0955249,95.63424123)
\curveto(654.17552995,95.63423927)(654.26052987,95.62923927)(654.3505249,95.61924123)
\curveto(654.4305297,95.61923928)(654.50552962,95.62923927)(654.5755249,95.64924123)
\lineto(654.8455249,95.64924123)
\curveto(654.86552926,95.64923925)(654.89552923,95.64423926)(654.9355249,95.63424123)
\curveto(654.96552916,95.63423927)(654.99052914,95.63923926)(655.0105249,95.64924123)
\curveto(655.11052902,95.65923924)(655.21052892,95.66423924)(655.3105249,95.66424123)
\curveto(655.40052873,95.67423923)(655.50052863,95.68423922)(655.6105249,95.69424123)
\curveto(655.7305284,95.72423918)(655.85552827,95.73923916)(655.9855249,95.73924123)
\curveto(656.10552802,95.74923915)(656.22052791,95.77423913)(656.3305249,95.81424123)
\curveto(656.6305275,95.89423901)(656.89552723,95.97923892)(657.1255249,96.06924123)
\curveto(657.35552677,96.16923873)(657.57052656,96.31423859)(657.7705249,96.50424123)
\curveto(657.97052616,96.71423819)(658.12052601,96.97923792)(658.2205249,97.29924123)
\curveto(658.24052589,97.33923756)(658.25052588,97.37423753)(658.2505249,97.40424123)
\curveto(658.24052589,97.44423746)(658.24552588,97.48923741)(658.2655249,97.53924123)
\curveto(658.27552585,97.57923732)(658.28552584,97.64923725)(658.2955249,97.74924123)
\curveto(658.30552582,97.85923704)(658.30052583,97.94423696)(658.2805249,98.00424123)
\curveto(658.26052587,98.07423683)(658.25052588,98.14423676)(658.2505249,98.21424123)
\curveto(658.24052589,98.28423662)(658.2255259,98.34923655)(658.2055249,98.40924123)
\curveto(658.14552598,98.60923629)(658.06052607,98.78923611)(657.9505249,98.94924123)
\curveto(657.9305262,98.97923592)(657.91052622,99.0042359)(657.8905249,99.02424123)
\lineto(657.8305249,99.08424123)
\curveto(657.81052632,99.12423578)(657.77052636,99.17423573)(657.7105249,99.23424123)
\curveto(657.57052656,99.33423557)(657.44052669,99.41923548)(657.3205249,99.48924123)
\curveto(657.20052693,99.55923534)(657.05552707,99.62923527)(656.8855249,99.69924123)
\curveto(656.81552731,99.72923517)(656.74552738,99.74923515)(656.6755249,99.75924123)
\curveto(656.60552752,99.77923512)(656.5305276,99.7992351)(656.4505249,99.81924123)
}
}
{
\newrgbcolor{curcolor}{0 0 0}
\pscustom[linestyle=none,fillstyle=solid,fillcolor=curcolor]
{
\newpath
\moveto(648.6055249,106.77385061)
\curveto(648.60553552,106.87384575)(648.61553551,106.96884566)(648.6355249,107.05885061)
\curveto(648.64553548,107.14884548)(648.67553545,107.21384541)(648.7255249,107.25385061)
\curveto(648.80553532,107.31384531)(648.91053522,107.34384528)(649.0405249,107.34385061)
\lineto(649.4305249,107.34385061)
\lineto(650.9305249,107.34385061)
\lineto(657.3205249,107.34385061)
\lineto(658.4905249,107.34385061)
\lineto(658.8055249,107.34385061)
\curveto(658.90552522,107.35384527)(658.98552514,107.33884529)(659.0455249,107.29885061)
\curveto(659.125525,107.24884538)(659.17552495,107.17384545)(659.1955249,107.07385061)
\curveto(659.20552492,106.98384564)(659.21052492,106.87384575)(659.2105249,106.74385061)
\lineto(659.2105249,106.51885061)
\curveto(659.19052494,106.43884619)(659.17552495,106.36884626)(659.1655249,106.30885061)
\curveto(659.14552498,106.24884638)(659.10552502,106.19884643)(659.0455249,106.15885061)
\curveto(658.98552514,106.11884651)(658.91052522,106.09884653)(658.8205249,106.09885061)
\lineto(658.5205249,106.09885061)
\lineto(657.4255249,106.09885061)
\lineto(652.0855249,106.09885061)
\curveto(651.99553213,106.07884655)(651.92053221,106.06384656)(651.8605249,106.05385061)
\curveto(651.79053234,106.05384657)(651.7305324,106.0238466)(651.6805249,105.96385061)
\curveto(651.6305325,105.89384673)(651.60553252,105.80384682)(651.6055249,105.69385061)
\curveto(651.59553253,105.59384703)(651.59053254,105.48384714)(651.5905249,105.36385061)
\lineto(651.5905249,104.22385061)
\lineto(651.5905249,103.72885061)
\curveto(651.58053255,103.56884906)(651.52053261,103.45884917)(651.4105249,103.39885061)
\curveto(651.38053275,103.37884925)(651.35053278,103.36884926)(651.3205249,103.36885061)
\curveto(651.28053285,103.36884926)(651.23553289,103.36384926)(651.1855249,103.35385061)
\curveto(651.06553306,103.33384929)(650.95553317,103.33884929)(650.8555249,103.36885061)
\curveto(650.75553337,103.40884922)(650.68553344,103.46384916)(650.6455249,103.53385061)
\curveto(650.59553353,103.61384901)(650.57053356,103.73384889)(650.5705249,103.89385061)
\curveto(650.57053356,104.05384857)(650.55553357,104.18884844)(650.5255249,104.29885061)
\curveto(650.51553361,104.34884828)(650.51053362,104.40384822)(650.5105249,104.46385061)
\curveto(650.50053363,104.5238481)(650.48553364,104.58384804)(650.4655249,104.64385061)
\curveto(650.41553371,104.79384783)(650.36553376,104.93884769)(650.3155249,105.07885061)
\curveto(650.25553387,105.21884741)(650.18553394,105.35384727)(650.1055249,105.48385061)
\curveto(650.01553411,105.623847)(649.91053422,105.74384688)(649.7905249,105.84385061)
\curveto(649.67053446,105.94384668)(649.54053459,106.03884659)(649.4005249,106.12885061)
\curveto(649.30053483,106.18884644)(649.19053494,106.23384639)(649.0705249,106.26385061)
\curveto(648.95053518,106.30384632)(648.84553528,106.35384627)(648.7555249,106.41385061)
\curveto(648.69553543,106.46384616)(648.65553547,106.53384609)(648.6355249,106.62385061)
\curveto(648.6255355,106.64384598)(648.62053551,106.66884596)(648.6205249,106.69885061)
\curveto(648.62053551,106.7288459)(648.61553551,106.75384587)(648.6055249,106.77385061)
}
}
{
\newrgbcolor{curcolor}{0 0 0}
\pscustom[linestyle=none,fillstyle=solid,fillcolor=curcolor]
{
\newpath
\moveto(648.6055249,115.12345998)
\curveto(648.60553552,115.22345513)(648.61553551,115.31845503)(648.6355249,115.40845998)
\curveto(648.64553548,115.49845485)(648.67553545,115.56345479)(648.7255249,115.60345998)
\curveto(648.80553532,115.66345469)(648.91053522,115.69345466)(649.0405249,115.69345998)
\lineto(649.4305249,115.69345998)
\lineto(650.9305249,115.69345998)
\lineto(657.3205249,115.69345998)
\lineto(658.4905249,115.69345998)
\lineto(658.8055249,115.69345998)
\curveto(658.90552522,115.70345465)(658.98552514,115.68845466)(659.0455249,115.64845998)
\curveto(659.125525,115.59845475)(659.17552495,115.52345483)(659.1955249,115.42345998)
\curveto(659.20552492,115.33345502)(659.21052492,115.22345513)(659.2105249,115.09345998)
\lineto(659.2105249,114.86845998)
\curveto(659.19052494,114.78845556)(659.17552495,114.71845563)(659.1655249,114.65845998)
\curveto(659.14552498,114.59845575)(659.10552502,114.5484558)(659.0455249,114.50845998)
\curveto(658.98552514,114.46845588)(658.91052522,114.4484559)(658.8205249,114.44845998)
\lineto(658.5205249,114.44845998)
\lineto(657.4255249,114.44845998)
\lineto(652.0855249,114.44845998)
\curveto(651.99553213,114.42845592)(651.92053221,114.41345594)(651.8605249,114.40345998)
\curveto(651.79053234,114.40345595)(651.7305324,114.37345598)(651.6805249,114.31345998)
\curveto(651.6305325,114.24345611)(651.60553252,114.1534562)(651.6055249,114.04345998)
\curveto(651.59553253,113.94345641)(651.59053254,113.83345652)(651.5905249,113.71345998)
\lineto(651.5905249,112.57345998)
\lineto(651.5905249,112.07845998)
\curveto(651.58053255,111.91845843)(651.52053261,111.80845854)(651.4105249,111.74845998)
\curveto(651.38053275,111.72845862)(651.35053278,111.71845863)(651.3205249,111.71845998)
\curveto(651.28053285,111.71845863)(651.23553289,111.71345864)(651.1855249,111.70345998)
\curveto(651.06553306,111.68345867)(650.95553317,111.68845866)(650.8555249,111.71845998)
\curveto(650.75553337,111.75845859)(650.68553344,111.81345854)(650.6455249,111.88345998)
\curveto(650.59553353,111.96345839)(650.57053356,112.08345827)(650.5705249,112.24345998)
\curveto(650.57053356,112.40345795)(650.55553357,112.53845781)(650.5255249,112.64845998)
\curveto(650.51553361,112.69845765)(650.51053362,112.7534576)(650.5105249,112.81345998)
\curveto(650.50053363,112.87345748)(650.48553364,112.93345742)(650.4655249,112.99345998)
\curveto(650.41553371,113.14345721)(650.36553376,113.28845706)(650.3155249,113.42845998)
\curveto(650.25553387,113.56845678)(650.18553394,113.70345665)(650.1055249,113.83345998)
\curveto(650.01553411,113.97345638)(649.91053422,114.09345626)(649.7905249,114.19345998)
\curveto(649.67053446,114.29345606)(649.54053459,114.38845596)(649.4005249,114.47845998)
\curveto(649.30053483,114.53845581)(649.19053494,114.58345577)(649.0705249,114.61345998)
\curveto(648.95053518,114.6534557)(648.84553528,114.70345565)(648.7555249,114.76345998)
\curveto(648.69553543,114.81345554)(648.65553547,114.88345547)(648.6355249,114.97345998)
\curveto(648.6255355,114.99345536)(648.62053551,115.01845533)(648.6205249,115.04845998)
\curveto(648.62053551,115.07845527)(648.61553551,115.10345525)(648.6055249,115.12345998)
}
}
{
\newrgbcolor{curcolor}{0 0 0}
\pscustom[linestyle=none,fillstyle=solid,fillcolor=curcolor]
{
\newpath
\moveto(669.44184082,37.28705373)
\curveto(669.44185152,37.35704805)(669.44185152,37.43704797)(669.44184082,37.52705373)
\curveto(669.43185153,37.61704779)(669.43185153,37.70204771)(669.44184082,37.78205373)
\curveto(669.44185152,37.87204754)(669.45185151,37.95204746)(669.47184082,38.02205373)
\curveto(669.49185147,38.10204731)(669.52185144,38.15704725)(669.56184082,38.18705373)
\curveto(669.61185135,38.21704719)(669.68685127,38.23704717)(669.78684082,38.24705373)
\curveto(669.87685108,38.26704714)(669.98185098,38.27704713)(670.10184082,38.27705373)
\curveto(670.21185075,38.28704712)(670.32685063,38.28704712)(670.44684082,38.27705373)
\lineto(670.74684082,38.27705373)
\lineto(673.76184082,38.27705373)
\lineto(676.65684082,38.27705373)
\curveto(676.98684397,38.27704713)(677.31184365,38.27204714)(677.63184082,38.26205373)
\curveto(677.94184302,38.26204715)(678.22184274,38.22204719)(678.47184082,38.14205373)
\curveto(678.82184214,38.02204739)(679.11684184,37.86704754)(679.35684082,37.67705373)
\curveto(679.58684137,37.48704792)(679.78684117,37.24704816)(679.95684082,36.95705373)
\curveto(680.00684095,36.89704851)(680.04184092,36.83204858)(680.06184082,36.76205373)
\curveto(680.08184088,36.70204871)(680.10684085,36.63204878)(680.13684082,36.55205373)
\curveto(680.18684077,36.43204898)(680.22184074,36.30204911)(680.24184082,36.16205373)
\curveto(680.27184069,36.03204938)(680.30184066,35.89704951)(680.33184082,35.75705373)
\curveto(680.35184061,35.7070497)(680.3568406,35.65704975)(680.34684082,35.60705373)
\curveto(680.33684062,35.55704985)(680.33684062,35.50204991)(680.34684082,35.44205373)
\curveto(680.3568406,35.42204999)(680.3568406,35.39705001)(680.34684082,35.36705373)
\curveto(680.34684061,35.33705007)(680.35184061,35.3120501)(680.36184082,35.29205373)
\curveto(680.37184059,35.25205016)(680.37684058,35.19705021)(680.37684082,35.12705373)
\curveto(680.37684058,35.05705035)(680.37184059,35.00205041)(680.36184082,34.96205373)
\curveto(680.35184061,34.9120505)(680.35184061,34.85705055)(680.36184082,34.79705373)
\curveto(680.37184059,34.73705067)(680.36684059,34.68205073)(680.34684082,34.63205373)
\curveto(680.31684064,34.50205091)(680.29684066,34.37705103)(680.28684082,34.25705373)
\curveto(680.27684068,34.13705127)(680.25184071,34.02205139)(680.21184082,33.91205373)
\curveto(680.09184087,33.54205187)(679.92184104,33.22205219)(679.70184082,32.95205373)
\curveto(679.48184148,32.68205273)(679.20184176,32.47205294)(678.86184082,32.32205373)
\curveto(678.74184222,32.27205314)(678.61684234,32.22705318)(678.48684082,32.18705373)
\curveto(678.3568426,32.15705325)(678.22184274,32.12205329)(678.08184082,32.08205373)
\curveto(678.03184293,32.07205334)(677.99184297,32.06705334)(677.96184082,32.06705373)
\curveto(677.92184304,32.06705334)(677.87684308,32.06205335)(677.82684082,32.05205373)
\curveto(677.79684316,32.04205337)(677.7618432,32.03705337)(677.72184082,32.03705373)
\curveto(677.67184329,32.03705337)(677.63184333,32.03205338)(677.60184082,32.02205373)
\lineto(677.43684082,32.02205373)
\curveto(677.3568436,32.00205341)(677.2568437,31.99705341)(677.13684082,32.00705373)
\curveto(677.00684395,32.01705339)(676.91684404,32.03205338)(676.86684082,32.05205373)
\curveto(676.77684418,32.07205334)(676.71184425,32.12705328)(676.67184082,32.21705373)
\curveto(676.65184431,32.24705316)(676.64684431,32.27705313)(676.65684082,32.30705373)
\curveto(676.6568443,32.33705307)(676.65184431,32.37705303)(676.64184082,32.42705373)
\curveto(676.63184433,32.46705294)(676.62684433,32.5070529)(676.62684082,32.54705373)
\lineto(676.62684082,32.69705373)
\curveto(676.62684433,32.81705259)(676.63184433,32.93705247)(676.64184082,33.05705373)
\curveto(676.64184432,33.18705222)(676.67684428,33.27705213)(676.74684082,33.32705373)
\curveto(676.80684415,33.36705204)(676.86684409,33.38705202)(676.92684082,33.38705373)
\curveto(676.98684397,33.38705202)(677.0568439,33.39705201)(677.13684082,33.41705373)
\curveto(677.16684379,33.42705198)(677.20184376,33.42705198)(677.24184082,33.41705373)
\curveto(677.27184369,33.41705199)(677.29684366,33.42205199)(677.31684082,33.43205373)
\lineto(677.52684082,33.43205373)
\curveto(677.57684338,33.45205196)(677.62684333,33.45705195)(677.67684082,33.44705373)
\curveto(677.71684324,33.44705196)(677.7618432,33.45705195)(677.81184082,33.47705373)
\curveto(677.94184302,33.5070519)(678.06684289,33.53705187)(678.18684082,33.56705373)
\curveto(678.29684266,33.59705181)(678.40184256,33.64205177)(678.50184082,33.70205373)
\curveto(678.79184217,33.87205154)(678.99684196,34.14205127)(679.11684082,34.51205373)
\curveto(679.13684182,34.56205085)(679.15184181,34.6120508)(679.16184082,34.66205373)
\curveto(679.1618418,34.72205069)(679.17184179,34.77705063)(679.19184082,34.82705373)
\lineto(679.19184082,34.90205373)
\curveto(679.20184176,34.97205044)(679.21184175,35.06705034)(679.22184082,35.18705373)
\curveto(679.22184174,35.31705009)(679.21184175,35.41704999)(679.19184082,35.48705373)
\curveto(679.17184179,35.55704985)(679.1568418,35.62704978)(679.14684082,35.69705373)
\curveto(679.12684183,35.77704963)(679.10684185,35.84704956)(679.08684082,35.90705373)
\curveto(678.92684203,36.28704912)(678.65184231,36.56204885)(678.26184082,36.73205373)
\curveto(678.13184283,36.78204863)(677.97684298,36.81704859)(677.79684082,36.83705373)
\curveto(677.61684334,36.86704854)(677.43184353,36.88204853)(677.24184082,36.88205373)
\curveto(677.04184392,36.89204852)(676.84184412,36.89204852)(676.64184082,36.88205373)
\lineto(676.07184082,36.88205373)
\lineto(671.82684082,36.88205373)
\lineto(670.28184082,36.88205373)
\curveto(670.17185079,36.88204853)(670.05185091,36.87704853)(669.92184082,36.86705373)
\curveto(669.79185117,36.85704855)(669.68685127,36.87704853)(669.60684082,36.92705373)
\curveto(669.53685142,36.98704842)(669.48685147,37.06704834)(669.45684082,37.16705373)
\curveto(669.4568515,37.18704822)(669.4568515,37.2070482)(669.45684082,37.22705373)
\curveto(669.4568515,37.24704816)(669.45185151,37.26704814)(669.44184082,37.28705373)
}
}
{
\newrgbcolor{curcolor}{0 0 0}
\pscustom[linestyle=none,fillstyle=solid,fillcolor=curcolor]
{
\newpath
\moveto(672.39684082,40.82072561)
\lineto(672.39684082,41.25572561)
\curveto(672.39684856,41.40572364)(672.43684852,41.51072354)(672.51684082,41.57072561)
\curveto(672.59684836,41.62072343)(672.69684826,41.6457234)(672.81684082,41.64572561)
\curveto(672.93684802,41.65572339)(673.0568479,41.66072339)(673.17684082,41.66072561)
\lineto(674.60184082,41.66072561)
\lineto(676.86684082,41.66072561)
\lineto(677.55684082,41.66072561)
\curveto(677.78684317,41.66072339)(677.98684297,41.68572336)(678.15684082,41.73572561)
\curveto(678.60684235,41.89572315)(678.92184204,42.19572285)(679.10184082,42.63572561)
\curveto(679.19184177,42.85572219)(679.22684173,43.12072193)(679.20684082,43.43072561)
\curveto(679.17684178,43.74072131)(679.12184184,43.99072106)(679.04184082,44.18072561)
\curveto(678.90184206,44.51072054)(678.72684223,44.77072028)(678.51684082,44.96072561)
\curveto(678.29684266,45.16071989)(678.01184295,45.31571973)(677.66184082,45.42572561)
\curveto(677.58184338,45.45571959)(677.50184346,45.47571957)(677.42184082,45.48572561)
\curveto(677.34184362,45.49571955)(677.2568437,45.51071954)(677.16684082,45.53072561)
\curveto(677.11684384,45.54071951)(677.07184389,45.54071951)(677.03184082,45.53072561)
\curveto(676.99184397,45.53071952)(676.94684401,45.54071951)(676.89684082,45.56072561)
\lineto(676.58184082,45.56072561)
\curveto(676.50184446,45.58071947)(676.41184455,45.58571946)(676.31184082,45.57572561)
\curveto(676.20184476,45.56571948)(676.10184486,45.56071949)(676.01184082,45.56072561)
\lineto(674.84184082,45.56072561)
\lineto(673.25184082,45.56072561)
\curveto(673.13184783,45.56071949)(673.00684795,45.55571949)(672.87684082,45.54572561)
\curveto(672.73684822,45.5457195)(672.62684833,45.57071948)(672.54684082,45.62072561)
\curveto(672.49684846,45.66071939)(672.46684849,45.70571934)(672.45684082,45.75572561)
\curveto(672.43684852,45.81571923)(672.41684854,45.88571916)(672.39684082,45.96572561)
\lineto(672.39684082,46.19072561)
\curveto(672.39684856,46.31071874)(672.40184856,46.41571863)(672.41184082,46.50572561)
\curveto(672.42184854,46.60571844)(672.46684849,46.68071837)(672.54684082,46.73072561)
\curveto(672.59684836,46.78071827)(672.67184829,46.80571824)(672.77184082,46.80572561)
\lineto(673.05684082,46.80572561)
\lineto(674.07684082,46.80572561)
\lineto(678.11184082,46.80572561)
\lineto(679.46184082,46.80572561)
\curveto(679.58184138,46.80571824)(679.69684126,46.80071825)(679.80684082,46.79072561)
\curveto(679.90684105,46.79071826)(679.98184098,46.75571829)(680.03184082,46.68572561)
\curveto(680.0618409,46.6457184)(680.08684087,46.58571846)(680.10684082,46.50572561)
\curveto(680.11684084,46.42571862)(680.12684083,46.33571871)(680.13684082,46.23572561)
\curveto(680.13684082,46.1457189)(680.13184083,46.05571899)(680.12184082,45.96572561)
\curveto(680.11184085,45.88571916)(680.09184087,45.82571922)(680.06184082,45.78572561)
\curveto(680.02184094,45.73571931)(679.956841,45.69071936)(679.86684082,45.65072561)
\curveto(679.82684113,45.64071941)(679.77184119,45.63071942)(679.70184082,45.62072561)
\curveto(679.63184133,45.62071943)(679.56684139,45.61571943)(679.50684082,45.60572561)
\curveto(679.43684152,45.59571945)(679.38184158,45.57571947)(679.34184082,45.54572561)
\curveto(679.30184166,45.51571953)(679.28684167,45.47071958)(679.29684082,45.41072561)
\curveto(679.31684164,45.33071972)(679.37684158,45.2507198)(679.47684082,45.17072561)
\curveto(679.56684139,45.09071996)(679.63684132,45.01572003)(679.68684082,44.94572561)
\curveto(679.84684111,44.72572032)(679.98684097,44.47572057)(680.10684082,44.19572561)
\curveto(680.1568408,44.08572096)(680.18684077,43.97072108)(680.19684082,43.85072561)
\curveto(680.21684074,43.74072131)(680.24184072,43.62572142)(680.27184082,43.50572561)
\curveto(680.28184068,43.45572159)(680.28184068,43.40072165)(680.27184082,43.34072561)
\curveto(680.2618407,43.29072176)(680.26684069,43.24072181)(680.28684082,43.19072561)
\curveto(680.30684065,43.09072196)(680.30684065,43.00072205)(680.28684082,42.92072561)
\lineto(680.28684082,42.77072561)
\curveto(680.26684069,42.72072233)(680.2568407,42.66072239)(680.25684082,42.59072561)
\curveto(680.2568407,42.53072252)(680.25184071,42.47572257)(680.24184082,42.42572561)
\curveto(680.22184074,42.38572266)(680.21184075,42.3457227)(680.21184082,42.30572561)
\curveto(680.22184074,42.27572277)(680.21684074,42.23572281)(680.19684082,42.18572561)
\lineto(680.13684082,41.94572561)
\curveto(680.11684084,41.87572317)(680.08684087,41.80072325)(680.04684082,41.72072561)
\curveto(679.93684102,41.46072359)(679.79184117,41.24072381)(679.61184082,41.06072561)
\curveto(679.42184154,40.89072416)(679.19684176,40.7507243)(678.93684082,40.64072561)
\curveto(678.84684211,40.60072445)(678.7568422,40.57072448)(678.66684082,40.55072561)
\lineto(678.36684082,40.49072561)
\curveto(678.30684265,40.47072458)(678.25184271,40.46072459)(678.20184082,40.46072561)
\curveto(678.14184282,40.47072458)(678.07684288,40.46572458)(678.00684082,40.44572561)
\curveto(677.98684297,40.43572461)(677.961843,40.43072462)(677.93184082,40.43072561)
\curveto(677.89184307,40.43072462)(677.8568431,40.42572462)(677.82684082,40.41572561)
\lineto(677.67684082,40.41572561)
\curveto(677.63684332,40.40572464)(677.59184337,40.40072465)(677.54184082,40.40072561)
\curveto(677.48184348,40.41072464)(677.42684353,40.41572463)(677.37684082,40.41572561)
\lineto(676.77684082,40.41572561)
\lineto(674.01684082,40.41572561)
\lineto(673.05684082,40.41572561)
\lineto(672.78684082,40.41572561)
\curveto(672.69684826,40.41572463)(672.62184834,40.43572461)(672.56184082,40.47572561)
\curveto(672.49184847,40.51572453)(672.44184852,40.59072446)(672.41184082,40.70072561)
\curveto(672.40184856,40.72072433)(672.40184856,40.74072431)(672.41184082,40.76072561)
\curveto(672.41184855,40.78072427)(672.40684855,40.80072425)(672.39684082,40.82072561)
}
}
{
\newrgbcolor{curcolor}{0 0 0}
\pscustom[linestyle=none,fillstyle=solid,fillcolor=curcolor]
{
\newpath
\moveto(672.24684082,52.39533498)
\curveto(672.22684873,53.02532975)(672.31184865,53.53032924)(672.50184082,53.91033498)
\curveto(672.69184827,54.29032848)(672.97684798,54.59532818)(673.35684082,54.82533498)
\curveto(673.4568475,54.88532789)(673.56684739,54.93032784)(673.68684082,54.96033498)
\curveto(673.79684716,55.00032777)(673.91184705,55.03532774)(674.03184082,55.06533498)
\curveto(674.22184674,55.11532766)(674.42684653,55.14532763)(674.64684082,55.15533498)
\curveto(674.86684609,55.16532761)(675.09184587,55.1703276)(675.32184082,55.17033498)
\lineto(676.92684082,55.17033498)
\lineto(679.26684082,55.17033498)
\curveto(679.43684152,55.1703276)(679.60684135,55.16532761)(679.77684082,55.15533498)
\curveto(679.94684101,55.15532762)(680.0568409,55.09032768)(680.10684082,54.96033498)
\curveto(680.12684083,54.91032786)(680.13684082,54.85532792)(680.13684082,54.79533498)
\curveto(680.14684081,54.74532803)(680.15184081,54.69032808)(680.15184082,54.63033498)
\curveto(680.15184081,54.50032827)(680.14684081,54.3753284)(680.13684082,54.25533498)
\curveto(680.13684082,54.13532864)(680.09684086,54.05032872)(680.01684082,54.00033498)
\curveto(679.94684101,53.95032882)(679.8568411,53.92532885)(679.74684082,53.92533498)
\lineto(679.41684082,53.92533498)
\lineto(678.12684082,53.92533498)
\lineto(675.68184082,53.92533498)
\curveto(675.41184555,53.92532885)(675.14684581,53.92032885)(674.88684082,53.91033498)
\curveto(674.61684634,53.90032887)(674.38684657,53.85532892)(674.19684082,53.77533498)
\curveto(673.99684696,53.69532908)(673.83684712,53.5753292)(673.71684082,53.41533498)
\curveto(673.58684737,53.25532952)(673.48684747,53.0703297)(673.41684082,52.86033498)
\curveto(673.39684756,52.80032997)(673.38684757,52.73533004)(673.38684082,52.66533498)
\curveto(673.37684758,52.60533017)(673.3618476,52.54533023)(673.34184082,52.48533498)
\curveto(673.33184763,52.43533034)(673.33184763,52.35533042)(673.34184082,52.24533498)
\curveto(673.34184762,52.14533063)(673.34684761,52.0753307)(673.35684082,52.03533498)
\curveto(673.37684758,51.99533078)(673.38684757,51.96033081)(673.38684082,51.93033498)
\curveto(673.37684758,51.90033087)(673.37684758,51.86533091)(673.38684082,51.82533498)
\curveto(673.41684754,51.69533108)(673.45184751,51.5703312)(673.49184082,51.45033498)
\curveto(673.52184744,51.34033143)(673.56684739,51.23533154)(673.62684082,51.13533498)
\curveto(673.64684731,51.09533168)(673.66684729,51.06033171)(673.68684082,51.03033498)
\curveto(673.70684725,51.00033177)(673.72684723,50.96533181)(673.74684082,50.92533498)
\curveto(673.99684696,50.5753322)(674.37184659,50.32033245)(674.87184082,50.16033498)
\curveto(674.95184601,50.13033264)(675.03684592,50.11033266)(675.12684082,50.10033498)
\curveto(675.20684575,50.09033268)(675.28684567,50.0753327)(675.36684082,50.05533498)
\curveto(675.41684554,50.03533274)(675.46684549,50.03033274)(675.51684082,50.04033498)
\curveto(675.5568454,50.05033272)(675.59684536,50.04533273)(675.63684082,50.02533498)
\lineto(675.95184082,50.02533498)
\curveto(675.98184498,50.01533276)(676.01684494,50.01033276)(676.05684082,50.01033498)
\curveto(676.09684486,50.02033275)(676.14184482,50.02533275)(676.19184082,50.02533498)
\lineto(676.64184082,50.02533498)
\lineto(678.08184082,50.02533498)
\lineto(679.40184082,50.02533498)
\lineto(679.74684082,50.02533498)
\curveto(679.8568411,50.02533275)(679.94684101,50.00033277)(680.01684082,49.95033498)
\curveto(680.09684086,49.90033287)(680.13684082,49.81033296)(680.13684082,49.68033498)
\curveto(680.14684081,49.56033321)(680.15184081,49.43533334)(680.15184082,49.30533498)
\curveto(680.15184081,49.22533355)(680.14684081,49.15033362)(680.13684082,49.08033498)
\curveto(680.12684083,49.01033376)(680.10184086,48.95033382)(680.06184082,48.90033498)
\curveto(680.01184095,48.82033395)(679.91684104,48.78033399)(679.77684082,48.78033498)
\lineto(679.37184082,48.78033498)
\lineto(677.60184082,48.78033498)
\lineto(673.97184082,48.78033498)
\lineto(673.05684082,48.78033498)
\lineto(672.78684082,48.78033498)
\curveto(672.69684826,48.78033399)(672.62684833,48.80033397)(672.57684082,48.84033498)
\curveto(672.51684844,48.8703339)(672.47684848,48.92033385)(672.45684082,48.99033498)
\curveto(672.44684851,49.03033374)(672.43684852,49.08533369)(672.42684082,49.15533498)
\curveto(672.41684854,49.23533354)(672.41184855,49.31533346)(672.41184082,49.39533498)
\curveto(672.41184855,49.4753333)(672.41684854,49.55033322)(672.42684082,49.62033498)
\curveto(672.43684852,49.70033307)(672.45184851,49.75533302)(672.47184082,49.78533498)
\curveto(672.54184842,49.89533288)(672.63184833,49.94533283)(672.74184082,49.93533498)
\curveto(672.84184812,49.92533285)(672.956848,49.94033283)(673.08684082,49.98033498)
\curveto(673.14684781,50.00033277)(673.19684776,50.04033273)(673.23684082,50.10033498)
\curveto(673.24684771,50.22033255)(673.20184776,50.31533246)(673.10184082,50.38533498)
\curveto(673.00184796,50.46533231)(672.92184804,50.54533223)(672.86184082,50.62533498)
\curveto(672.7618482,50.76533201)(672.67184829,50.90533187)(672.59184082,51.04533498)
\curveto(672.50184846,51.19533158)(672.42684853,51.36533141)(672.36684082,51.55533498)
\curveto(672.33684862,51.63533114)(672.31684864,51.72033105)(672.30684082,51.81033498)
\curveto(672.29684866,51.91033086)(672.28184868,52.00533077)(672.26184082,52.09533498)
\curveto(672.25184871,52.14533063)(672.24684871,52.19533058)(672.24684082,52.24533498)
\lineto(672.24684082,52.39533498)
}
}
{
\newrgbcolor{curcolor}{0 0 0}
\pscustom[linestyle=none,fillstyle=solid,fillcolor=curcolor]
{
}
}
{
\newrgbcolor{curcolor}{0 0 0}
\pscustom[linestyle=none,fillstyle=solid,fillcolor=curcolor]
{
\newpath
\moveto(675.03684082,67.99510061)
\lineto(675.29184082,67.99510061)
\curveto(675.37184559,68.0050929)(675.44684551,68.00009291)(675.51684082,67.98010061)
\lineto(675.75684082,67.98010061)
\lineto(675.92184082,67.98010061)
\curveto(676.02184494,67.96009295)(676.12684483,67.95009296)(676.23684082,67.95010061)
\curveto(676.33684462,67.95009296)(676.43684452,67.94009297)(676.53684082,67.92010061)
\lineto(676.68684082,67.92010061)
\curveto(676.82684413,67.89009302)(676.96684399,67.87009304)(677.10684082,67.86010061)
\curveto(677.23684372,67.85009306)(677.36684359,67.82509308)(677.49684082,67.78510061)
\curveto(677.57684338,67.76509314)(677.6618433,67.74509316)(677.75184082,67.72510061)
\lineto(677.99184082,67.66510061)
\lineto(678.29184082,67.54510061)
\curveto(678.38184258,67.51509339)(678.47184249,67.48009343)(678.56184082,67.44010061)
\curveto(678.78184218,67.34009357)(678.99684196,67.2050937)(679.20684082,67.03510061)
\curveto(679.41684154,66.87509403)(679.58684137,66.70009421)(679.71684082,66.51010061)
\curveto(679.7568412,66.46009445)(679.79684116,66.40009451)(679.83684082,66.33010061)
\curveto(679.86684109,66.27009464)(679.90184106,66.2100947)(679.94184082,66.15010061)
\curveto(679.99184097,66.07009484)(680.03184093,65.97509493)(680.06184082,65.86510061)
\curveto(680.09184087,65.75509515)(680.12184084,65.65009526)(680.15184082,65.55010061)
\curveto(680.19184077,65.44009547)(680.21684074,65.33009558)(680.22684082,65.22010061)
\curveto(680.23684072,65.1100958)(680.25184071,64.99509591)(680.27184082,64.87510061)
\curveto(680.28184068,64.83509607)(680.28184068,64.79009612)(680.27184082,64.74010061)
\curveto(680.27184069,64.70009621)(680.27684068,64.66009625)(680.28684082,64.62010061)
\curveto(680.29684066,64.58009633)(680.30184066,64.52509638)(680.30184082,64.45510061)
\curveto(680.30184066,64.38509652)(680.29684066,64.33509657)(680.28684082,64.30510061)
\curveto(680.26684069,64.25509665)(680.2618407,64.2100967)(680.27184082,64.17010061)
\curveto(680.28184068,64.13009678)(680.28184068,64.09509681)(680.27184082,64.06510061)
\lineto(680.27184082,63.97510061)
\curveto(680.25184071,63.91509699)(680.23684072,63.85009706)(680.22684082,63.78010061)
\curveto(680.22684073,63.72009719)(680.22184074,63.65509725)(680.21184082,63.58510061)
\curveto(680.1618408,63.41509749)(680.11184085,63.25509765)(680.06184082,63.10510061)
\curveto(680.01184095,62.95509795)(679.94684101,62.8100981)(679.86684082,62.67010061)
\curveto(679.82684113,62.62009829)(679.79684116,62.56509834)(679.77684082,62.50510061)
\curveto(679.74684121,62.45509845)(679.71184125,62.4050985)(679.67184082,62.35510061)
\curveto(679.49184147,62.11509879)(679.27184169,61.91509899)(679.01184082,61.75510061)
\curveto(678.75184221,61.59509931)(678.46684249,61.45509945)(678.15684082,61.33510061)
\curveto(678.01684294,61.27509963)(677.87684308,61.23009968)(677.73684082,61.20010061)
\curveto(677.58684337,61.17009974)(677.43184353,61.13509977)(677.27184082,61.09510061)
\curveto(677.1618438,61.07509983)(677.05184391,61.06009985)(676.94184082,61.05010061)
\curveto(676.83184413,61.04009987)(676.72184424,61.02509988)(676.61184082,61.00510061)
\curveto(676.57184439,60.99509991)(676.53184443,60.99009992)(676.49184082,60.99010061)
\curveto(676.45184451,61.00009991)(676.41184455,61.00009991)(676.37184082,60.99010061)
\curveto(676.32184464,60.98009993)(676.27184469,60.97509993)(676.22184082,60.97510061)
\lineto(676.05684082,60.97510061)
\curveto(676.00684495,60.95509995)(675.956845,60.95009996)(675.90684082,60.96010061)
\curveto(675.84684511,60.97009994)(675.79184517,60.97009994)(675.74184082,60.96010061)
\curveto(675.70184526,60.95009996)(675.6568453,60.95009996)(675.60684082,60.96010061)
\curveto(675.5568454,60.97009994)(675.50684545,60.96509994)(675.45684082,60.94510061)
\curveto(675.38684557,60.92509998)(675.31184565,60.92009999)(675.23184082,60.93010061)
\curveto(675.14184582,60.94009997)(675.0568459,60.94509996)(674.97684082,60.94510061)
\curveto(674.88684607,60.94509996)(674.78684617,60.94009997)(674.67684082,60.93010061)
\curveto(674.5568464,60.92009999)(674.4568465,60.92509998)(674.37684082,60.94510061)
\lineto(674.09184082,60.94510061)
\lineto(673.46184082,60.99010061)
\curveto(673.3618476,61.00009991)(673.26684769,61.0100999)(673.17684082,61.02010061)
\lineto(672.87684082,61.05010061)
\curveto(672.82684813,61.07009984)(672.77684818,61.07509983)(672.72684082,61.06510061)
\curveto(672.66684829,61.06509984)(672.61184835,61.07509983)(672.56184082,61.09510061)
\curveto(672.39184857,61.14509976)(672.22684873,61.18509972)(672.06684082,61.21510061)
\curveto(671.89684906,61.24509966)(671.73684922,61.29509961)(671.58684082,61.36510061)
\curveto(671.12684983,61.55509935)(670.75185021,61.77509913)(670.46184082,62.02510061)
\curveto(670.17185079,62.28509862)(669.92685103,62.64509826)(669.72684082,63.10510061)
\curveto(669.67685128,63.23509767)(669.64185132,63.36509754)(669.62184082,63.49510061)
\curveto(669.60185136,63.63509727)(669.57685138,63.77509713)(669.54684082,63.91510061)
\curveto(669.53685142,63.98509692)(669.53185143,64.05009686)(669.53184082,64.11010061)
\curveto(669.53185143,64.17009674)(669.52685143,64.23509667)(669.51684082,64.30510061)
\curveto(669.49685146,65.13509577)(669.64685131,65.8050951)(669.96684082,66.31510061)
\curveto(670.27685068,66.82509408)(670.71685024,67.2050937)(671.28684082,67.45510061)
\curveto(671.40684955,67.5050934)(671.53184943,67.55009336)(671.66184082,67.59010061)
\curveto(671.79184917,67.63009328)(671.92684903,67.67509323)(672.06684082,67.72510061)
\curveto(672.14684881,67.74509316)(672.23184873,67.76009315)(672.32184082,67.77010061)
\lineto(672.56184082,67.83010061)
\curveto(672.67184829,67.86009305)(672.78184818,67.87509303)(672.89184082,67.87510061)
\curveto(673.00184796,67.88509302)(673.11184785,67.90009301)(673.22184082,67.92010061)
\curveto(673.27184769,67.94009297)(673.31684764,67.94509296)(673.35684082,67.93510061)
\curveto(673.39684756,67.93509297)(673.43684752,67.94009297)(673.47684082,67.95010061)
\curveto(673.52684743,67.96009295)(673.58184738,67.96009295)(673.64184082,67.95010061)
\curveto(673.69184727,67.95009296)(673.74184722,67.95509295)(673.79184082,67.96510061)
\lineto(673.92684082,67.96510061)
\curveto(673.98684697,67.98509292)(674.0568469,67.98509292)(674.13684082,67.96510061)
\curveto(674.20684675,67.95509295)(674.27184669,67.96009295)(674.33184082,67.98010061)
\curveto(674.3618466,67.99009292)(674.40184656,67.99509291)(674.45184082,67.99510061)
\lineto(674.57184082,67.99510061)
\lineto(675.03684082,67.99510061)
\moveto(677.36184082,66.45010061)
\curveto(677.04184392,66.55009436)(676.67684428,66.6100943)(676.26684082,66.63010061)
\curveto(675.8568451,66.65009426)(675.44684551,66.66009425)(675.03684082,66.66010061)
\curveto(674.60684635,66.66009425)(674.18684677,66.65009426)(673.77684082,66.63010061)
\curveto(673.36684759,66.6100943)(672.98184798,66.56509434)(672.62184082,66.49510061)
\curveto(672.2618487,66.42509448)(671.94184902,66.31509459)(671.66184082,66.16510061)
\curveto(671.37184959,66.02509488)(671.13684982,65.83009508)(670.95684082,65.58010061)
\curveto(670.84685011,65.42009549)(670.76685019,65.24009567)(670.71684082,65.04010061)
\curveto(670.6568503,64.84009607)(670.62685033,64.59509631)(670.62684082,64.30510061)
\curveto(670.64685031,64.28509662)(670.6568503,64.25009666)(670.65684082,64.20010061)
\curveto(670.64685031,64.15009676)(670.64685031,64.1100968)(670.65684082,64.08010061)
\curveto(670.67685028,64.00009691)(670.69685026,63.92509698)(670.71684082,63.85510061)
\curveto(670.72685023,63.79509711)(670.74685021,63.73009718)(670.77684082,63.66010061)
\curveto(670.89685006,63.39009752)(671.06684989,63.17009774)(671.28684082,63.00010061)
\curveto(671.49684946,62.84009807)(671.74184922,62.7050982)(672.02184082,62.59510061)
\curveto(672.13184883,62.54509836)(672.25184871,62.5050984)(672.38184082,62.47510061)
\curveto(672.50184846,62.45509845)(672.62684833,62.43009848)(672.75684082,62.40010061)
\curveto(672.80684815,62.38009853)(672.8618481,62.37009854)(672.92184082,62.37010061)
\curveto(672.97184799,62.37009854)(673.02184794,62.36509854)(673.07184082,62.35510061)
\curveto(673.1618478,62.34509856)(673.2568477,62.33509857)(673.35684082,62.32510061)
\curveto(673.44684751,62.31509859)(673.54184742,62.3050986)(673.64184082,62.29510061)
\curveto(673.72184724,62.29509861)(673.80684715,62.29009862)(673.89684082,62.28010061)
\lineto(674.13684082,62.28010061)
\lineto(674.31684082,62.28010061)
\curveto(674.34684661,62.27009864)(674.38184658,62.26509864)(674.42184082,62.26510061)
\lineto(674.55684082,62.26510061)
\lineto(675.00684082,62.26510061)
\curveto(675.08684587,62.26509864)(675.17184579,62.26009865)(675.26184082,62.25010061)
\curveto(675.34184562,62.25009866)(675.41684554,62.26009865)(675.48684082,62.28010061)
\lineto(675.75684082,62.28010061)
\curveto(675.77684518,62.28009863)(675.80684515,62.27509863)(675.84684082,62.26510061)
\curveto(675.87684508,62.26509864)(675.90184506,62.27009864)(675.92184082,62.28010061)
\curveto(676.02184494,62.29009862)(676.12184484,62.29509861)(676.22184082,62.29510061)
\curveto(676.31184465,62.3050986)(676.41184455,62.31509859)(676.52184082,62.32510061)
\curveto(676.64184432,62.35509855)(676.76684419,62.37009854)(676.89684082,62.37010061)
\curveto(677.01684394,62.38009853)(677.13184383,62.4050985)(677.24184082,62.44510061)
\curveto(677.54184342,62.52509838)(677.80684315,62.6100983)(678.03684082,62.70010061)
\curveto(678.26684269,62.80009811)(678.48184248,62.94509796)(678.68184082,63.13510061)
\curveto(678.88184208,63.34509756)(679.03184193,63.6100973)(679.13184082,63.93010061)
\curveto(679.15184181,63.97009694)(679.1618418,64.0050969)(679.16184082,64.03510061)
\curveto(679.15184181,64.07509683)(679.1568418,64.12009679)(679.17684082,64.17010061)
\curveto(679.18684177,64.2100967)(679.19684176,64.28009663)(679.20684082,64.38010061)
\curveto(679.21684174,64.49009642)(679.21184175,64.57509633)(679.19184082,64.63510061)
\curveto(679.17184179,64.7050962)(679.1618418,64.77509613)(679.16184082,64.84510061)
\curveto(679.15184181,64.91509599)(679.13684182,64.98009593)(679.11684082,65.04010061)
\curveto(679.0568419,65.24009567)(678.97184199,65.42009549)(678.86184082,65.58010061)
\curveto(678.84184212,65.6100953)(678.82184214,65.63509527)(678.80184082,65.65510061)
\lineto(678.74184082,65.71510061)
\curveto(678.72184224,65.75509515)(678.68184228,65.8050951)(678.62184082,65.86510061)
\curveto(678.48184248,65.96509494)(678.35184261,66.05009486)(678.23184082,66.12010061)
\curveto(678.11184285,66.19009472)(677.96684299,66.26009465)(677.79684082,66.33010061)
\curveto(677.72684323,66.36009455)(677.6568433,66.38009453)(677.58684082,66.39010061)
\curveto(677.51684344,66.4100945)(677.44184352,66.43009448)(677.36184082,66.45010061)
}
}
{
\newrgbcolor{curcolor}{0 0 0}
\pscustom[linestyle=none,fillstyle=solid,fillcolor=curcolor]
{
\newpath
\moveto(675.80184082,76.34470998)
\curveto(675.92184504,76.37470226)(676.0618449,76.39970223)(676.22184082,76.41970998)
\curveto(676.38184458,76.43970219)(676.54684441,76.44970218)(676.71684082,76.44970998)
\curveto(676.88684407,76.44970218)(677.05184391,76.43970219)(677.21184082,76.41970998)
\curveto(677.37184359,76.39970223)(677.51184345,76.37470226)(677.63184082,76.34470998)
\curveto(677.77184319,76.30470233)(677.89684306,76.26970236)(678.00684082,76.23970998)
\curveto(678.11684284,76.20970242)(678.22684273,76.16970246)(678.33684082,76.11970998)
\curveto(678.97684198,75.84970278)(679.4618415,75.4347032)(679.79184082,74.87470998)
\curveto(679.85184111,74.79470384)(679.90184106,74.70970392)(679.94184082,74.61970998)
\curveto(679.97184099,74.5297041)(680.00684095,74.4297042)(680.04684082,74.31970998)
\curveto(680.09684086,74.20970442)(680.13184083,74.08970454)(680.15184082,73.95970998)
\curveto(680.18184078,73.83970479)(680.21184075,73.70970492)(680.24184082,73.56970998)
\curveto(680.2618407,73.50970512)(680.26684069,73.44970518)(680.25684082,73.38970998)
\curveto(680.24684071,73.33970529)(680.25184071,73.27970535)(680.27184082,73.20970998)
\curveto(680.28184068,73.18970544)(680.28184068,73.16470547)(680.27184082,73.13470998)
\curveto(680.27184069,73.10470553)(680.27684068,73.07970555)(680.28684082,73.05970998)
\lineto(680.28684082,72.90970998)
\curveto(680.29684066,72.83970579)(680.29684066,72.78970584)(680.28684082,72.75970998)
\curveto(680.27684068,72.71970591)(680.27184069,72.67470596)(680.27184082,72.62470998)
\curveto(680.28184068,72.58470605)(680.28184068,72.54470609)(680.27184082,72.50470998)
\curveto(680.25184071,72.41470622)(680.23684072,72.32470631)(680.22684082,72.23470998)
\curveto(680.22684073,72.14470649)(680.21684074,72.05470658)(680.19684082,71.96470998)
\curveto(680.16684079,71.87470676)(680.14184082,71.78470685)(680.12184082,71.69470998)
\curveto(680.10184086,71.60470703)(680.07184089,71.51970711)(680.03184082,71.43970998)
\curveto(679.92184104,71.19970743)(679.79184117,70.97470766)(679.64184082,70.76470998)
\curveto(679.48184148,70.55470808)(679.30184166,70.37470826)(679.10184082,70.22470998)
\curveto(678.93184203,70.10470853)(678.7568422,69.99970863)(678.57684082,69.90970998)
\curveto(678.39684256,69.81970881)(678.20684275,69.7297089)(678.00684082,69.63970998)
\curveto(677.90684305,69.59970903)(677.80684315,69.56470907)(677.70684082,69.53470998)
\curveto(677.59684336,69.51470912)(677.48684347,69.48970914)(677.37684082,69.45970998)
\curveto(677.23684372,69.41970921)(677.09684386,69.39470924)(676.95684082,69.38470998)
\curveto(676.81684414,69.37470926)(676.67684428,69.35470928)(676.53684082,69.32470998)
\curveto(676.42684453,69.31470932)(676.32684463,69.30470933)(676.23684082,69.29470998)
\curveto(676.13684482,69.29470934)(676.03684492,69.28470935)(675.93684082,69.26470998)
\lineto(675.84684082,69.26470998)
\curveto(675.81684514,69.27470936)(675.79184517,69.27470936)(675.77184082,69.26470998)
\lineto(675.56184082,69.26470998)
\curveto(675.50184546,69.24470939)(675.43684552,69.2347094)(675.36684082,69.23470998)
\curveto(675.28684567,69.24470939)(675.21184575,69.24970938)(675.14184082,69.24970998)
\lineto(674.99184082,69.24970998)
\curveto(674.94184602,69.24970938)(674.89184607,69.25470938)(674.84184082,69.26470998)
\lineto(674.46684082,69.26470998)
\curveto(674.43684652,69.27470936)(674.40184656,69.27470936)(674.36184082,69.26470998)
\curveto(674.32184664,69.26470937)(674.28184668,69.26970936)(674.24184082,69.27970998)
\curveto(674.13184683,69.29970933)(674.02184694,69.31470932)(673.91184082,69.32470998)
\curveto(673.79184717,69.3347093)(673.67684728,69.34470929)(673.56684082,69.35470998)
\curveto(673.41684754,69.39470924)(673.27184769,69.41970921)(673.13184082,69.42970998)
\curveto(672.98184798,69.44970918)(672.83684812,69.47970915)(672.69684082,69.51970998)
\curveto(672.39684856,69.60970902)(672.11184885,69.70470893)(671.84184082,69.80470998)
\curveto(671.57184939,69.90470873)(671.32184964,70.0297086)(671.09184082,70.17970998)
\curveto(670.77185019,70.37970825)(670.49185047,70.62470801)(670.25184082,70.91470998)
\curveto(670.01185095,71.20470743)(669.82685113,71.54470709)(669.69684082,71.93470998)
\curveto(669.6568513,72.04470659)(669.63185133,72.15470648)(669.62184082,72.26470998)
\curveto(669.60185136,72.38470625)(669.57685138,72.50470613)(669.54684082,72.62470998)
\curveto(669.53685142,72.69470594)(669.53185143,72.75970587)(669.53184082,72.81970998)
\curveto(669.53185143,72.87970575)(669.52685143,72.94470569)(669.51684082,73.01470998)
\curveto(669.49685146,73.71470492)(669.61185135,74.28970434)(669.86184082,74.73970998)
\curveto(670.11185085,75.18970344)(670.4618505,75.5347031)(670.91184082,75.77470998)
\curveto(671.14184982,75.88470275)(671.41684954,75.98470265)(671.73684082,76.07470998)
\curveto(671.80684915,76.09470254)(671.88184908,76.09470254)(671.96184082,76.07470998)
\curveto(672.03184893,76.06470257)(672.08184888,76.03970259)(672.11184082,75.99970998)
\curveto(672.14184882,75.96970266)(672.16684879,75.90970272)(672.18684082,75.81970998)
\curveto(672.19684876,75.7297029)(672.20684875,75.629703)(672.21684082,75.51970998)
\curveto(672.21684874,75.41970321)(672.21184875,75.31970331)(672.20184082,75.21970998)
\curveto(672.19184877,75.1297035)(672.17184879,75.06470357)(672.14184082,75.02470998)
\curveto(672.07184889,74.91470372)(671.961849,74.8347038)(671.81184082,74.78470998)
\curveto(671.6618493,74.74470389)(671.53184943,74.68970394)(671.42184082,74.61970998)
\curveto(671.11184985,74.4297042)(670.88185008,74.14970448)(670.73184082,73.77970998)
\curveto(670.70185026,73.70970492)(670.68185028,73.634705)(670.67184082,73.55470998)
\curveto(670.6618503,73.48470515)(670.64685031,73.40970522)(670.62684082,73.32970998)
\curveto(670.61685034,73.27970535)(670.61185035,73.20970542)(670.61184082,73.11970998)
\curveto(670.61185035,73.03970559)(670.61685034,72.97470566)(670.62684082,72.92470998)
\curveto(670.64685031,72.88470575)(670.65185031,72.84970578)(670.64184082,72.81970998)
\curveto(670.63185033,72.78970584)(670.63185033,72.75470588)(670.64184082,72.71470998)
\lineto(670.70184082,72.47470998)
\curveto(670.72185024,72.40470623)(670.74685021,72.3347063)(670.77684082,72.26470998)
\curveto(670.93685002,71.88470675)(671.14684981,71.59470704)(671.40684082,71.39470998)
\curveto(671.66684929,71.20470743)(671.98184898,71.0297076)(672.35184082,70.86970998)
\curveto(672.43184853,70.83970779)(672.51184845,70.81470782)(672.59184082,70.79470998)
\curveto(672.67184829,70.78470785)(672.75184821,70.76470787)(672.83184082,70.73470998)
\curveto(672.94184802,70.70470793)(673.0568479,70.67970795)(673.17684082,70.65970998)
\curveto(673.29684766,70.64970798)(673.41684754,70.629708)(673.53684082,70.59970998)
\curveto(673.58684737,70.57970805)(673.63684732,70.56970806)(673.68684082,70.56970998)
\curveto(673.73684722,70.57970805)(673.78684717,70.57470806)(673.83684082,70.55470998)
\curveto(673.89684706,70.54470809)(673.97684698,70.54470809)(674.07684082,70.55470998)
\curveto(674.16684679,70.56470807)(674.22184674,70.57970805)(674.24184082,70.59970998)
\curveto(674.28184668,70.61970801)(674.30184666,70.64970798)(674.30184082,70.68970998)
\curveto(674.30184666,70.73970789)(674.29184667,70.78470785)(674.27184082,70.82470998)
\curveto(674.23184673,70.89470774)(674.18684677,70.95470768)(674.13684082,71.00470998)
\curveto(674.08684687,71.05470758)(674.03684692,71.11470752)(673.98684082,71.18470998)
\lineto(673.92684082,71.24470998)
\curveto(673.89684706,71.27470736)(673.87184709,71.30470733)(673.85184082,71.33470998)
\curveto(673.69184727,71.56470707)(673.5568474,71.83970679)(673.44684082,72.15970998)
\curveto(673.42684753,72.2297064)(673.41184755,72.29970633)(673.40184082,72.36970998)
\curveto(673.39184757,72.43970619)(673.37684758,72.51470612)(673.35684082,72.59470998)
\curveto(673.3568476,72.634706)(673.35184761,72.66970596)(673.34184082,72.69970998)
\curveto(673.33184763,72.7297059)(673.33184763,72.76470587)(673.34184082,72.80470998)
\curveto(673.34184762,72.85470578)(673.33184763,72.89470574)(673.31184082,72.92470998)
\lineto(673.31184082,73.08970998)
\lineto(673.31184082,73.17970998)
\curveto(673.30184766,73.2297054)(673.30184766,73.26970536)(673.31184082,73.29970998)
\curveto(673.32184764,73.34970528)(673.32684763,73.39970523)(673.32684082,73.44970998)
\curveto(673.31684764,73.50970512)(673.31684764,73.56470507)(673.32684082,73.61470998)
\curveto(673.3568476,73.72470491)(673.37684758,73.8297048)(673.38684082,73.92970998)
\curveto(673.39684756,74.03970459)(673.42184754,74.14470449)(673.46184082,74.24470998)
\curveto(673.60184736,74.66470397)(673.78684717,75.00970362)(674.01684082,75.27970998)
\curveto(674.23684672,75.54970308)(674.52184644,75.78970284)(674.87184082,75.99970998)
\curveto(675.01184595,76.07970255)(675.1618458,76.14470249)(675.32184082,76.19470998)
\curveto(675.47184549,76.24470239)(675.63184533,76.29470234)(675.80184082,76.34470998)
\moveto(677.10684082,75.09970998)
\curveto(677.0568439,75.10970352)(677.01184395,75.11470352)(676.97184082,75.11470998)
\lineto(676.82184082,75.11470998)
\curveto(676.51184445,75.11470352)(676.22684473,75.07470356)(675.96684082,74.99470998)
\curveto(675.90684505,74.97470366)(675.85184511,74.95470368)(675.80184082,74.93470998)
\curveto(675.74184522,74.92470371)(675.68684527,74.90970372)(675.63684082,74.88970998)
\curveto(675.14684581,74.66970396)(674.79684616,74.32470431)(674.58684082,73.85470998)
\curveto(674.5568464,73.77470486)(674.53184643,73.69470494)(674.51184082,73.61470998)
\lineto(674.45184082,73.37470998)
\curveto(674.43184653,73.29470534)(674.42184654,73.20470543)(674.42184082,73.10470998)
\lineto(674.42184082,72.78970998)
\curveto(674.44184652,72.76970586)(674.45184651,72.7297059)(674.45184082,72.66970998)
\curveto(674.44184652,72.61970601)(674.44184652,72.57470606)(674.45184082,72.53470998)
\lineto(674.51184082,72.29470998)
\curveto(674.52184644,72.22470641)(674.54184642,72.15470648)(674.57184082,72.08470998)
\curveto(674.83184613,71.48470715)(675.29684566,71.07970755)(675.96684082,70.86970998)
\curveto(676.04684491,70.83970779)(676.12684483,70.81970781)(676.20684082,70.80970998)
\curveto(676.28684467,70.79970783)(676.37184459,70.78470785)(676.46184082,70.76470998)
\lineto(676.61184082,70.76470998)
\curveto(676.65184431,70.75470788)(676.72184424,70.74970788)(676.82184082,70.74970998)
\curveto(677.05184391,70.74970788)(677.24684371,70.76970786)(677.40684082,70.80970998)
\curveto(677.47684348,70.8297078)(677.54184342,70.84470779)(677.60184082,70.85470998)
\curveto(677.6618433,70.86470777)(677.72684323,70.88470775)(677.79684082,70.91470998)
\curveto(678.07684288,71.02470761)(678.32184264,71.16970746)(678.53184082,71.34970998)
\curveto(678.73184223,71.5297071)(678.89184207,71.76470687)(679.01184082,72.05470998)
\lineto(679.10184082,72.29470998)
\lineto(679.16184082,72.53470998)
\curveto(679.18184178,72.58470605)(679.18684177,72.62470601)(679.17684082,72.65470998)
\curveto(679.16684179,72.69470594)(679.17184179,72.73970589)(679.19184082,72.78970998)
\curveto(679.20184176,72.81970581)(679.20684175,72.87470576)(679.20684082,72.95470998)
\curveto(679.20684175,73.0347056)(679.20184176,73.09470554)(679.19184082,73.13470998)
\curveto(679.17184179,73.24470539)(679.1568418,73.34970528)(679.14684082,73.44970998)
\curveto(679.13684182,73.54970508)(679.10684185,73.64470499)(679.05684082,73.73470998)
\curveto(678.8568421,74.26470437)(678.48184248,74.65470398)(677.93184082,74.90470998)
\curveto(677.83184313,74.94470369)(677.72684323,74.97470366)(677.61684082,74.99470998)
\lineto(677.28684082,75.08470998)
\curveto(677.20684375,75.08470355)(677.14684381,75.08970354)(677.10684082,75.09970998)
}
}
{
\newrgbcolor{curcolor}{0 0 0}
\pscustom[linestyle=none,fillstyle=solid,fillcolor=curcolor]
{
\newpath
\moveto(678.48684082,78.63431936)
\lineto(678.48684082,79.26431936)
\lineto(678.48684082,79.45931936)
\curveto(678.48684247,79.52931683)(678.49684246,79.58931677)(678.51684082,79.63931936)
\curveto(678.5568424,79.70931665)(678.59684236,79.7593166)(678.63684082,79.78931936)
\curveto(678.68684227,79.82931653)(678.75184221,79.84931651)(678.83184082,79.84931936)
\curveto(678.91184205,79.8593165)(678.99684196,79.86431649)(679.08684082,79.86431936)
\lineto(679.80684082,79.86431936)
\curveto(680.28684067,79.86431649)(680.69684026,79.80431655)(681.03684082,79.68431936)
\curveto(681.37683958,79.56431679)(681.65183931,79.36931699)(681.86184082,79.09931936)
\curveto(681.91183905,79.02931733)(681.956839,78.9593174)(681.99684082,78.88931936)
\curveto(682.04683891,78.82931753)(682.09183887,78.7543176)(682.13184082,78.66431936)
\curveto(682.14183882,78.64431771)(682.15183881,78.61431774)(682.16184082,78.57431936)
\curveto(682.18183878,78.53431782)(682.18683877,78.48931787)(682.17684082,78.43931936)
\curveto(682.14683881,78.34931801)(682.07183889,78.29431806)(681.95184082,78.27431936)
\curveto(681.84183912,78.2543181)(681.74683921,78.26931809)(681.66684082,78.31931936)
\curveto(681.59683936,78.34931801)(681.53183943,78.39431796)(681.47184082,78.45431936)
\curveto(681.42183954,78.52431783)(681.37183959,78.58931777)(681.32184082,78.64931936)
\curveto(681.27183969,78.71931764)(681.19683976,78.77931758)(681.09684082,78.82931936)
\curveto(681.00683995,78.88931747)(680.91684004,78.93931742)(680.82684082,78.97931936)
\curveto(680.79684016,78.99931736)(680.73684022,79.02431733)(680.64684082,79.05431936)
\curveto(680.56684039,79.08431727)(680.49684046,79.08931727)(680.43684082,79.06931936)
\curveto(680.29684066,79.03931732)(680.20684075,78.97931738)(680.16684082,78.88931936)
\curveto(680.13684082,78.80931755)(680.12184084,78.71931764)(680.12184082,78.61931936)
\curveto(680.12184084,78.51931784)(680.09684086,78.43431792)(680.04684082,78.36431936)
\curveto(679.97684098,78.27431808)(679.83684112,78.22931813)(679.62684082,78.22931936)
\lineto(679.07184082,78.22931936)
\lineto(678.84684082,78.22931936)
\curveto(678.76684219,78.23931812)(678.70184226,78.2593181)(678.65184082,78.28931936)
\curveto(678.57184239,78.34931801)(678.52684243,78.41931794)(678.51684082,78.49931936)
\curveto(678.50684245,78.51931784)(678.50184246,78.53931782)(678.50184082,78.55931936)
\curveto(678.50184246,78.58931777)(678.49684246,78.61431774)(678.48684082,78.63431936)
}
}
{
\newrgbcolor{curcolor}{0 0 0}
\pscustom[linestyle=none,fillstyle=solid,fillcolor=curcolor]
{
}
}
{
\newrgbcolor{curcolor}{0 0 0}
\pscustom[linestyle=none,fillstyle=solid,fillcolor=curcolor]
{
\newpath
\moveto(669.51684082,89.26463186)
\curveto(669.50685145,89.95462722)(669.62685133,90.55462662)(669.87684082,91.06463186)
\curveto(670.12685083,91.58462559)(670.4618505,91.9796252)(670.88184082,92.24963186)
\curveto(670.96185,92.29962488)(671.05184991,92.34462483)(671.15184082,92.38463186)
\curveto(671.24184972,92.42462475)(671.33684962,92.46962471)(671.43684082,92.51963186)
\curveto(671.53684942,92.55962462)(671.63684932,92.58962459)(671.73684082,92.60963186)
\curveto(671.83684912,92.62962455)(671.94184902,92.64962453)(672.05184082,92.66963186)
\curveto(672.10184886,92.68962449)(672.14684881,92.69462448)(672.18684082,92.68463186)
\curveto(672.22684873,92.6746245)(672.27184869,92.6796245)(672.32184082,92.69963186)
\curveto(672.37184859,92.70962447)(672.4568485,92.71462446)(672.57684082,92.71463186)
\curveto(672.68684827,92.71462446)(672.77184819,92.70962447)(672.83184082,92.69963186)
\curveto(672.89184807,92.6796245)(672.95184801,92.66962451)(673.01184082,92.66963186)
\curveto(673.07184789,92.6796245)(673.13184783,92.6746245)(673.19184082,92.65463186)
\curveto(673.33184763,92.61462456)(673.46684749,92.5796246)(673.59684082,92.54963186)
\curveto(673.72684723,92.51962466)(673.85184711,92.4796247)(673.97184082,92.42963186)
\curveto(674.11184685,92.36962481)(674.23684672,92.29962488)(674.34684082,92.21963186)
\curveto(674.4568465,92.14962503)(674.56684639,92.0746251)(674.67684082,91.99463186)
\lineto(674.73684082,91.93463186)
\curveto(674.7568462,91.92462525)(674.77684618,91.90962527)(674.79684082,91.88963186)
\curveto(674.956846,91.76962541)(675.10184586,91.63462554)(675.23184082,91.48463186)
\curveto(675.3618456,91.33462584)(675.48684547,91.174626)(675.60684082,91.00463186)
\curveto(675.82684513,90.69462648)(676.03184493,90.39962678)(676.22184082,90.11963186)
\curveto(676.3618446,89.88962729)(676.49684446,89.65962752)(676.62684082,89.42963186)
\curveto(676.7568442,89.20962797)(676.89184407,88.98962819)(677.03184082,88.76963186)
\curveto(677.20184376,88.51962866)(677.38184358,88.2796289)(677.57184082,88.04963186)
\curveto(677.7618432,87.82962935)(677.98684297,87.63962954)(678.24684082,87.47963186)
\curveto(678.30684265,87.43962974)(678.36684259,87.40462977)(678.42684082,87.37463186)
\curveto(678.47684248,87.34462983)(678.54184242,87.31462986)(678.62184082,87.28463186)
\curveto(678.69184227,87.26462991)(678.75184221,87.25962992)(678.80184082,87.26963186)
\curveto(678.87184209,87.28962989)(678.92684203,87.32462985)(678.96684082,87.37463186)
\curveto(678.99684196,87.42462975)(679.01684194,87.48462969)(679.02684082,87.55463186)
\lineto(679.02684082,87.79463186)
\lineto(679.02684082,88.54463186)
\lineto(679.02684082,91.34963186)
\lineto(679.02684082,92.00963186)
\curveto(679.02684193,92.09962508)(679.03184193,92.18462499)(679.04184082,92.26463186)
\curveto(679.04184192,92.34462483)(679.0618419,92.40962477)(679.10184082,92.45963186)
\curveto(679.14184182,92.50962467)(679.21684174,92.54962463)(679.32684082,92.57963186)
\curveto(679.42684153,92.61962456)(679.52684143,92.62962455)(679.62684082,92.60963186)
\lineto(679.76184082,92.60963186)
\curveto(679.83184113,92.58962459)(679.89184107,92.56962461)(679.94184082,92.54963186)
\curveto(679.99184097,92.52962465)(680.03184093,92.49462468)(680.06184082,92.44463186)
\curveto(680.10184086,92.39462478)(680.12184084,92.32462485)(680.12184082,92.23463186)
\lineto(680.12184082,91.96463186)
\lineto(680.12184082,91.06463186)
\lineto(680.12184082,87.55463186)
\lineto(680.12184082,86.48963186)
\curveto(680.12184084,86.40963077)(680.12684083,86.31963086)(680.13684082,86.21963186)
\curveto(680.13684082,86.11963106)(680.12684083,86.03463114)(680.10684082,85.96463186)
\curveto(680.03684092,85.75463142)(679.8568411,85.68963149)(679.56684082,85.76963186)
\curveto(679.52684143,85.7796314)(679.49184147,85.7796314)(679.46184082,85.76963186)
\curveto(679.42184154,85.76963141)(679.37684158,85.7796314)(679.32684082,85.79963186)
\curveto(679.24684171,85.81963136)(679.1618418,85.83963134)(679.07184082,85.85963186)
\curveto(678.98184198,85.8796313)(678.89684206,85.90463127)(678.81684082,85.93463186)
\curveto(678.32684263,86.09463108)(677.91184305,86.29463088)(677.57184082,86.53463186)
\curveto(677.32184364,86.71463046)(677.09684386,86.91963026)(676.89684082,87.14963186)
\curveto(676.68684427,87.3796298)(676.49184447,87.61962956)(676.31184082,87.86963186)
\curveto(676.13184483,88.12962905)(675.961845,88.39462878)(675.80184082,88.66463186)
\curveto(675.63184533,88.94462823)(675.4568455,89.21462796)(675.27684082,89.47463186)
\curveto(675.19684576,89.58462759)(675.12184584,89.68962749)(675.05184082,89.78963186)
\curveto(674.98184598,89.89962728)(674.90684605,90.00962717)(674.82684082,90.11963186)
\curveto(674.79684616,90.15962702)(674.76684619,90.19462698)(674.73684082,90.22463186)
\curveto(674.69684626,90.26462691)(674.66684629,90.30462687)(674.64684082,90.34463186)
\curveto(674.53684642,90.48462669)(674.41184655,90.60962657)(674.27184082,90.71963186)
\curveto(674.24184672,90.73962644)(674.21684674,90.76462641)(674.19684082,90.79463186)
\curveto(674.16684679,90.82462635)(674.13684682,90.84962633)(674.10684082,90.86963186)
\curveto(674.00684695,90.94962623)(673.90684705,91.01462616)(673.80684082,91.06463186)
\curveto(673.70684725,91.12462605)(673.59684736,91.179626)(673.47684082,91.22963186)
\curveto(673.40684755,91.25962592)(673.33184763,91.2796259)(673.25184082,91.28963186)
\lineto(673.01184082,91.34963186)
\lineto(672.92184082,91.34963186)
\curveto(672.89184807,91.35962582)(672.8618481,91.36462581)(672.83184082,91.36463186)
\curveto(672.7618482,91.38462579)(672.66684829,91.38962579)(672.54684082,91.37963186)
\curveto(672.41684854,91.3796258)(672.31684864,91.36962581)(672.24684082,91.34963186)
\curveto(672.16684879,91.32962585)(672.09184887,91.30962587)(672.02184082,91.28963186)
\curveto(671.94184902,91.2796259)(671.8618491,91.25962592)(671.78184082,91.22963186)
\curveto(671.54184942,91.11962606)(671.34184962,90.96962621)(671.18184082,90.77963186)
\curveto(671.01184995,90.59962658)(670.87185009,90.3796268)(670.76184082,90.11963186)
\curveto(670.74185022,90.04962713)(670.72685023,89.9796272)(670.71684082,89.90963186)
\curveto(670.69685026,89.83962734)(670.67685028,89.76462741)(670.65684082,89.68463186)
\curveto(670.63685032,89.60462757)(670.62685033,89.49462768)(670.62684082,89.35463186)
\curveto(670.62685033,89.22462795)(670.63685032,89.11962806)(670.65684082,89.03963186)
\curveto(670.66685029,88.9796282)(670.67185029,88.92462825)(670.67184082,88.87463186)
\curveto(670.67185029,88.82462835)(670.68185028,88.7746284)(670.70184082,88.72463186)
\curveto(670.74185022,88.62462855)(670.78185018,88.52962865)(670.82184082,88.43963186)
\curveto(670.8618501,88.35962882)(670.90685005,88.2796289)(670.95684082,88.19963186)
\curveto(670.97684998,88.16962901)(671.00184996,88.13962904)(671.03184082,88.10963186)
\curveto(671.0618499,88.08962909)(671.08684987,88.06462911)(671.10684082,88.03463186)
\lineto(671.18184082,87.95963186)
\curveto(671.20184976,87.92962925)(671.22184974,87.90462927)(671.24184082,87.88463186)
\lineto(671.45184082,87.73463186)
\curveto(671.51184945,87.69462948)(671.57684938,87.64962953)(671.64684082,87.59963186)
\curveto(671.73684922,87.53962964)(671.84184912,87.48962969)(671.96184082,87.44963186)
\curveto(672.07184889,87.41962976)(672.18184878,87.38462979)(672.29184082,87.34463186)
\curveto(672.40184856,87.30462987)(672.54684841,87.2796299)(672.72684082,87.26963186)
\curveto(672.89684806,87.25962992)(673.02184794,87.22962995)(673.10184082,87.17963186)
\curveto(673.18184778,87.12963005)(673.22684773,87.05463012)(673.23684082,86.95463186)
\curveto(673.24684771,86.85463032)(673.25184771,86.74463043)(673.25184082,86.62463186)
\curveto(673.25184771,86.58463059)(673.2568477,86.54463063)(673.26684082,86.50463186)
\curveto(673.26684769,86.46463071)(673.2618477,86.42963075)(673.25184082,86.39963186)
\curveto(673.23184773,86.34963083)(673.22184774,86.29963088)(673.22184082,86.24963186)
\curveto(673.22184774,86.20963097)(673.21184775,86.16963101)(673.19184082,86.12963186)
\curveto(673.13184783,86.03963114)(672.99684796,85.99463118)(672.78684082,85.99463186)
\lineto(672.66684082,85.99463186)
\curveto(672.60684835,86.00463117)(672.54684841,86.00963117)(672.48684082,86.00963186)
\curveto(672.41684854,86.01963116)(672.35184861,86.02963115)(672.29184082,86.03963186)
\curveto(672.18184878,86.05963112)(672.08184888,86.0796311)(671.99184082,86.09963186)
\curveto(671.89184907,86.11963106)(671.79684916,86.14963103)(671.70684082,86.18963186)
\curveto(671.63684932,86.20963097)(671.57684938,86.22963095)(671.52684082,86.24963186)
\lineto(671.34684082,86.30963186)
\curveto(671.08684987,86.42963075)(670.84185012,86.58463059)(670.61184082,86.77463186)
\curveto(670.38185058,86.9746302)(670.19685076,87.18962999)(670.05684082,87.41963186)
\curveto(669.97685098,87.52962965)(669.91185105,87.64462953)(669.86184082,87.76463186)
\lineto(669.71184082,88.15463186)
\curveto(669.6618513,88.26462891)(669.63185133,88.3796288)(669.62184082,88.49963186)
\curveto(669.60185136,88.61962856)(669.57685138,88.74462843)(669.54684082,88.87463186)
\curveto(669.54685141,88.94462823)(669.54685141,89.00962817)(669.54684082,89.06963186)
\curveto(669.53685142,89.12962805)(669.52685143,89.19462798)(669.51684082,89.26463186)
}
}
{
\newrgbcolor{curcolor}{0 0 0}
\pscustom[linestyle=none,fillstyle=solid,fillcolor=curcolor]
{
\newpath
\moveto(675.03684082,101.36424123)
\lineto(675.29184082,101.36424123)
\curveto(675.37184559,101.37423353)(675.44684551,101.36923353)(675.51684082,101.34924123)
\lineto(675.75684082,101.34924123)
\lineto(675.92184082,101.34924123)
\curveto(676.02184494,101.32923357)(676.12684483,101.31923358)(676.23684082,101.31924123)
\curveto(676.33684462,101.31923358)(676.43684452,101.30923359)(676.53684082,101.28924123)
\lineto(676.68684082,101.28924123)
\curveto(676.82684413,101.25923364)(676.96684399,101.23923366)(677.10684082,101.22924123)
\curveto(677.23684372,101.21923368)(677.36684359,101.19423371)(677.49684082,101.15424123)
\curveto(677.57684338,101.13423377)(677.6618433,101.11423379)(677.75184082,101.09424123)
\lineto(677.99184082,101.03424123)
\lineto(678.29184082,100.91424123)
\curveto(678.38184258,100.88423402)(678.47184249,100.84923405)(678.56184082,100.80924123)
\curveto(678.78184218,100.70923419)(678.99684196,100.57423433)(679.20684082,100.40424123)
\curveto(679.41684154,100.24423466)(679.58684137,100.06923483)(679.71684082,99.87924123)
\curveto(679.7568412,99.82923507)(679.79684116,99.76923513)(679.83684082,99.69924123)
\curveto(679.86684109,99.63923526)(679.90184106,99.57923532)(679.94184082,99.51924123)
\curveto(679.99184097,99.43923546)(680.03184093,99.34423556)(680.06184082,99.23424123)
\curveto(680.09184087,99.12423578)(680.12184084,99.01923588)(680.15184082,98.91924123)
\curveto(680.19184077,98.80923609)(680.21684074,98.6992362)(680.22684082,98.58924123)
\curveto(680.23684072,98.47923642)(680.25184071,98.36423654)(680.27184082,98.24424123)
\curveto(680.28184068,98.2042367)(680.28184068,98.15923674)(680.27184082,98.10924123)
\curveto(680.27184069,98.06923683)(680.27684068,98.02923687)(680.28684082,97.98924123)
\curveto(680.29684066,97.94923695)(680.30184066,97.89423701)(680.30184082,97.82424123)
\curveto(680.30184066,97.75423715)(680.29684066,97.7042372)(680.28684082,97.67424123)
\curveto(680.26684069,97.62423728)(680.2618407,97.57923732)(680.27184082,97.53924123)
\curveto(680.28184068,97.4992374)(680.28184068,97.46423744)(680.27184082,97.43424123)
\lineto(680.27184082,97.34424123)
\curveto(680.25184071,97.28423762)(680.23684072,97.21923768)(680.22684082,97.14924123)
\curveto(680.22684073,97.08923781)(680.22184074,97.02423788)(680.21184082,96.95424123)
\curveto(680.1618408,96.78423812)(680.11184085,96.62423828)(680.06184082,96.47424123)
\curveto(680.01184095,96.32423858)(679.94684101,96.17923872)(679.86684082,96.03924123)
\curveto(679.82684113,95.98923891)(679.79684116,95.93423897)(679.77684082,95.87424123)
\curveto(679.74684121,95.82423908)(679.71184125,95.77423913)(679.67184082,95.72424123)
\curveto(679.49184147,95.48423942)(679.27184169,95.28423962)(679.01184082,95.12424123)
\curveto(678.75184221,94.96423994)(678.46684249,94.82424008)(678.15684082,94.70424123)
\curveto(678.01684294,94.64424026)(677.87684308,94.5992403)(677.73684082,94.56924123)
\curveto(677.58684337,94.53924036)(677.43184353,94.5042404)(677.27184082,94.46424123)
\curveto(677.1618438,94.44424046)(677.05184391,94.42924047)(676.94184082,94.41924123)
\curveto(676.83184413,94.40924049)(676.72184424,94.39424051)(676.61184082,94.37424123)
\curveto(676.57184439,94.36424054)(676.53184443,94.35924054)(676.49184082,94.35924123)
\curveto(676.45184451,94.36924053)(676.41184455,94.36924053)(676.37184082,94.35924123)
\curveto(676.32184464,94.34924055)(676.27184469,94.34424056)(676.22184082,94.34424123)
\lineto(676.05684082,94.34424123)
\curveto(676.00684495,94.32424058)(675.956845,94.31924058)(675.90684082,94.32924123)
\curveto(675.84684511,94.33924056)(675.79184517,94.33924056)(675.74184082,94.32924123)
\curveto(675.70184526,94.31924058)(675.6568453,94.31924058)(675.60684082,94.32924123)
\curveto(675.5568454,94.33924056)(675.50684545,94.33424057)(675.45684082,94.31424123)
\curveto(675.38684557,94.29424061)(675.31184565,94.28924061)(675.23184082,94.29924123)
\curveto(675.14184582,94.30924059)(675.0568459,94.31424059)(674.97684082,94.31424123)
\curveto(674.88684607,94.31424059)(674.78684617,94.30924059)(674.67684082,94.29924123)
\curveto(674.5568464,94.28924061)(674.4568465,94.29424061)(674.37684082,94.31424123)
\lineto(674.09184082,94.31424123)
\lineto(673.46184082,94.35924123)
\curveto(673.3618476,94.36924053)(673.26684769,94.37924052)(673.17684082,94.38924123)
\lineto(672.87684082,94.41924123)
\curveto(672.82684813,94.43924046)(672.77684818,94.44424046)(672.72684082,94.43424123)
\curveto(672.66684829,94.43424047)(672.61184835,94.44424046)(672.56184082,94.46424123)
\curveto(672.39184857,94.51424039)(672.22684873,94.55424035)(672.06684082,94.58424123)
\curveto(671.89684906,94.61424029)(671.73684922,94.66424024)(671.58684082,94.73424123)
\curveto(671.12684983,94.92423998)(670.75185021,95.14423976)(670.46184082,95.39424123)
\curveto(670.17185079,95.65423925)(669.92685103,96.01423889)(669.72684082,96.47424123)
\curveto(669.67685128,96.6042383)(669.64185132,96.73423817)(669.62184082,96.86424123)
\curveto(669.60185136,97.0042379)(669.57685138,97.14423776)(669.54684082,97.28424123)
\curveto(669.53685142,97.35423755)(669.53185143,97.41923748)(669.53184082,97.47924123)
\curveto(669.53185143,97.53923736)(669.52685143,97.6042373)(669.51684082,97.67424123)
\curveto(669.49685146,98.5042364)(669.64685131,99.17423573)(669.96684082,99.68424123)
\curveto(670.27685068,100.19423471)(670.71685024,100.57423433)(671.28684082,100.82424123)
\curveto(671.40684955,100.87423403)(671.53184943,100.91923398)(671.66184082,100.95924123)
\curveto(671.79184917,100.9992339)(671.92684903,101.04423386)(672.06684082,101.09424123)
\curveto(672.14684881,101.11423379)(672.23184873,101.12923377)(672.32184082,101.13924123)
\lineto(672.56184082,101.19924123)
\curveto(672.67184829,101.22923367)(672.78184818,101.24423366)(672.89184082,101.24424123)
\curveto(673.00184796,101.25423365)(673.11184785,101.26923363)(673.22184082,101.28924123)
\curveto(673.27184769,101.30923359)(673.31684764,101.31423359)(673.35684082,101.30424123)
\curveto(673.39684756,101.3042336)(673.43684752,101.30923359)(673.47684082,101.31924123)
\curveto(673.52684743,101.32923357)(673.58184738,101.32923357)(673.64184082,101.31924123)
\curveto(673.69184727,101.31923358)(673.74184722,101.32423358)(673.79184082,101.33424123)
\lineto(673.92684082,101.33424123)
\curveto(673.98684697,101.35423355)(674.0568469,101.35423355)(674.13684082,101.33424123)
\curveto(674.20684675,101.32423358)(674.27184669,101.32923357)(674.33184082,101.34924123)
\curveto(674.3618466,101.35923354)(674.40184656,101.36423354)(674.45184082,101.36424123)
\lineto(674.57184082,101.36424123)
\lineto(675.03684082,101.36424123)
\moveto(677.36184082,99.81924123)
\curveto(677.04184392,99.91923498)(676.67684428,99.97923492)(676.26684082,99.99924123)
\curveto(675.8568451,100.01923488)(675.44684551,100.02923487)(675.03684082,100.02924123)
\curveto(674.60684635,100.02923487)(674.18684677,100.01923488)(673.77684082,99.99924123)
\curveto(673.36684759,99.97923492)(672.98184798,99.93423497)(672.62184082,99.86424123)
\curveto(672.2618487,99.79423511)(671.94184902,99.68423522)(671.66184082,99.53424123)
\curveto(671.37184959,99.39423551)(671.13684982,99.1992357)(670.95684082,98.94924123)
\curveto(670.84685011,98.78923611)(670.76685019,98.60923629)(670.71684082,98.40924123)
\curveto(670.6568503,98.20923669)(670.62685033,97.96423694)(670.62684082,97.67424123)
\curveto(670.64685031,97.65423725)(670.6568503,97.61923728)(670.65684082,97.56924123)
\curveto(670.64685031,97.51923738)(670.64685031,97.47923742)(670.65684082,97.44924123)
\curveto(670.67685028,97.36923753)(670.69685026,97.29423761)(670.71684082,97.22424123)
\curveto(670.72685023,97.16423774)(670.74685021,97.0992378)(670.77684082,97.02924123)
\curveto(670.89685006,96.75923814)(671.06684989,96.53923836)(671.28684082,96.36924123)
\curveto(671.49684946,96.20923869)(671.74184922,96.07423883)(672.02184082,95.96424123)
\curveto(672.13184883,95.91423899)(672.25184871,95.87423903)(672.38184082,95.84424123)
\curveto(672.50184846,95.82423908)(672.62684833,95.7992391)(672.75684082,95.76924123)
\curveto(672.80684815,95.74923915)(672.8618481,95.73923916)(672.92184082,95.73924123)
\curveto(672.97184799,95.73923916)(673.02184794,95.73423917)(673.07184082,95.72424123)
\curveto(673.1618478,95.71423919)(673.2568477,95.7042392)(673.35684082,95.69424123)
\curveto(673.44684751,95.68423922)(673.54184742,95.67423923)(673.64184082,95.66424123)
\curveto(673.72184724,95.66423924)(673.80684715,95.65923924)(673.89684082,95.64924123)
\lineto(674.13684082,95.64924123)
\lineto(674.31684082,95.64924123)
\curveto(674.34684661,95.63923926)(674.38184658,95.63423927)(674.42184082,95.63424123)
\lineto(674.55684082,95.63424123)
\lineto(675.00684082,95.63424123)
\curveto(675.08684587,95.63423927)(675.17184579,95.62923927)(675.26184082,95.61924123)
\curveto(675.34184562,95.61923928)(675.41684554,95.62923927)(675.48684082,95.64924123)
\lineto(675.75684082,95.64924123)
\curveto(675.77684518,95.64923925)(675.80684515,95.64423926)(675.84684082,95.63424123)
\curveto(675.87684508,95.63423927)(675.90184506,95.63923926)(675.92184082,95.64924123)
\curveto(676.02184494,95.65923924)(676.12184484,95.66423924)(676.22184082,95.66424123)
\curveto(676.31184465,95.67423923)(676.41184455,95.68423922)(676.52184082,95.69424123)
\curveto(676.64184432,95.72423918)(676.76684419,95.73923916)(676.89684082,95.73924123)
\curveto(677.01684394,95.74923915)(677.13184383,95.77423913)(677.24184082,95.81424123)
\curveto(677.54184342,95.89423901)(677.80684315,95.97923892)(678.03684082,96.06924123)
\curveto(678.26684269,96.16923873)(678.48184248,96.31423859)(678.68184082,96.50424123)
\curveto(678.88184208,96.71423819)(679.03184193,96.97923792)(679.13184082,97.29924123)
\curveto(679.15184181,97.33923756)(679.1618418,97.37423753)(679.16184082,97.40424123)
\curveto(679.15184181,97.44423746)(679.1568418,97.48923741)(679.17684082,97.53924123)
\curveto(679.18684177,97.57923732)(679.19684176,97.64923725)(679.20684082,97.74924123)
\curveto(679.21684174,97.85923704)(679.21184175,97.94423696)(679.19184082,98.00424123)
\curveto(679.17184179,98.07423683)(679.1618418,98.14423676)(679.16184082,98.21424123)
\curveto(679.15184181,98.28423662)(679.13684182,98.34923655)(679.11684082,98.40924123)
\curveto(679.0568419,98.60923629)(678.97184199,98.78923611)(678.86184082,98.94924123)
\curveto(678.84184212,98.97923592)(678.82184214,99.0042359)(678.80184082,99.02424123)
\lineto(678.74184082,99.08424123)
\curveto(678.72184224,99.12423578)(678.68184228,99.17423573)(678.62184082,99.23424123)
\curveto(678.48184248,99.33423557)(678.35184261,99.41923548)(678.23184082,99.48924123)
\curveto(678.11184285,99.55923534)(677.96684299,99.62923527)(677.79684082,99.69924123)
\curveto(677.72684323,99.72923517)(677.6568433,99.74923515)(677.58684082,99.75924123)
\curveto(677.51684344,99.77923512)(677.44184352,99.7992351)(677.36184082,99.81924123)
}
}
{
\newrgbcolor{curcolor}{0 0 0}
\pscustom[linestyle=none,fillstyle=solid,fillcolor=curcolor]
{
\newpath
\moveto(669.51684082,106.77385061)
\curveto(669.51685144,106.87384575)(669.52685143,106.96884566)(669.54684082,107.05885061)
\curveto(669.5568514,107.14884548)(669.58685137,107.21384541)(669.63684082,107.25385061)
\curveto(669.71685124,107.31384531)(669.82185114,107.34384528)(669.95184082,107.34385061)
\lineto(670.34184082,107.34385061)
\lineto(671.84184082,107.34385061)
\lineto(678.23184082,107.34385061)
\lineto(679.40184082,107.34385061)
\lineto(679.71684082,107.34385061)
\curveto(679.81684114,107.35384527)(679.89684106,107.33884529)(679.95684082,107.29885061)
\curveto(680.03684092,107.24884538)(680.08684087,107.17384545)(680.10684082,107.07385061)
\curveto(680.11684084,106.98384564)(680.12184084,106.87384575)(680.12184082,106.74385061)
\lineto(680.12184082,106.51885061)
\curveto(680.10184086,106.43884619)(680.08684087,106.36884626)(680.07684082,106.30885061)
\curveto(680.0568409,106.24884638)(680.01684094,106.19884643)(679.95684082,106.15885061)
\curveto(679.89684106,106.11884651)(679.82184114,106.09884653)(679.73184082,106.09885061)
\lineto(679.43184082,106.09885061)
\lineto(678.33684082,106.09885061)
\lineto(672.99684082,106.09885061)
\curveto(672.90684805,106.07884655)(672.83184813,106.06384656)(672.77184082,106.05385061)
\curveto(672.70184826,106.05384657)(672.64184832,106.0238466)(672.59184082,105.96385061)
\curveto(672.54184842,105.89384673)(672.51684844,105.80384682)(672.51684082,105.69385061)
\curveto(672.50684845,105.59384703)(672.50184846,105.48384714)(672.50184082,105.36385061)
\lineto(672.50184082,104.22385061)
\lineto(672.50184082,103.72885061)
\curveto(672.49184847,103.56884906)(672.43184853,103.45884917)(672.32184082,103.39885061)
\curveto(672.29184867,103.37884925)(672.2618487,103.36884926)(672.23184082,103.36885061)
\curveto(672.19184877,103.36884926)(672.14684881,103.36384926)(672.09684082,103.35385061)
\curveto(671.97684898,103.33384929)(671.86684909,103.33884929)(671.76684082,103.36885061)
\curveto(671.66684929,103.40884922)(671.59684936,103.46384916)(671.55684082,103.53385061)
\curveto(671.50684945,103.61384901)(671.48184948,103.73384889)(671.48184082,103.89385061)
\curveto(671.48184948,104.05384857)(671.46684949,104.18884844)(671.43684082,104.29885061)
\curveto(671.42684953,104.34884828)(671.42184954,104.40384822)(671.42184082,104.46385061)
\curveto(671.41184955,104.5238481)(671.39684956,104.58384804)(671.37684082,104.64385061)
\curveto(671.32684963,104.79384783)(671.27684968,104.93884769)(671.22684082,105.07885061)
\curveto(671.16684979,105.21884741)(671.09684986,105.35384727)(671.01684082,105.48385061)
\curveto(670.92685003,105.623847)(670.82185014,105.74384688)(670.70184082,105.84385061)
\curveto(670.58185038,105.94384668)(670.45185051,106.03884659)(670.31184082,106.12885061)
\curveto(670.21185075,106.18884644)(670.10185086,106.23384639)(669.98184082,106.26385061)
\curveto(669.8618511,106.30384632)(669.7568512,106.35384627)(669.66684082,106.41385061)
\curveto(669.60685135,106.46384616)(669.56685139,106.53384609)(669.54684082,106.62385061)
\curveto(669.53685142,106.64384598)(669.53185143,106.66884596)(669.53184082,106.69885061)
\curveto(669.53185143,106.7288459)(669.52685143,106.75384587)(669.51684082,106.77385061)
}
}
{
\newrgbcolor{curcolor}{0 0 0}
\pscustom[linestyle=none,fillstyle=solid,fillcolor=curcolor]
{
\newpath
\moveto(669.51684082,115.12345998)
\curveto(669.51685144,115.22345513)(669.52685143,115.31845503)(669.54684082,115.40845998)
\curveto(669.5568514,115.49845485)(669.58685137,115.56345479)(669.63684082,115.60345998)
\curveto(669.71685124,115.66345469)(669.82185114,115.69345466)(669.95184082,115.69345998)
\lineto(670.34184082,115.69345998)
\lineto(671.84184082,115.69345998)
\lineto(678.23184082,115.69345998)
\lineto(679.40184082,115.69345998)
\lineto(679.71684082,115.69345998)
\curveto(679.81684114,115.70345465)(679.89684106,115.68845466)(679.95684082,115.64845998)
\curveto(680.03684092,115.59845475)(680.08684087,115.52345483)(680.10684082,115.42345998)
\curveto(680.11684084,115.33345502)(680.12184084,115.22345513)(680.12184082,115.09345998)
\lineto(680.12184082,114.86845998)
\curveto(680.10184086,114.78845556)(680.08684087,114.71845563)(680.07684082,114.65845998)
\curveto(680.0568409,114.59845575)(680.01684094,114.5484558)(679.95684082,114.50845998)
\curveto(679.89684106,114.46845588)(679.82184114,114.4484559)(679.73184082,114.44845998)
\lineto(679.43184082,114.44845998)
\lineto(678.33684082,114.44845998)
\lineto(672.99684082,114.44845998)
\curveto(672.90684805,114.42845592)(672.83184813,114.41345594)(672.77184082,114.40345998)
\curveto(672.70184826,114.40345595)(672.64184832,114.37345598)(672.59184082,114.31345998)
\curveto(672.54184842,114.24345611)(672.51684844,114.1534562)(672.51684082,114.04345998)
\curveto(672.50684845,113.94345641)(672.50184846,113.83345652)(672.50184082,113.71345998)
\lineto(672.50184082,112.57345998)
\lineto(672.50184082,112.07845998)
\curveto(672.49184847,111.91845843)(672.43184853,111.80845854)(672.32184082,111.74845998)
\curveto(672.29184867,111.72845862)(672.2618487,111.71845863)(672.23184082,111.71845998)
\curveto(672.19184877,111.71845863)(672.14684881,111.71345864)(672.09684082,111.70345998)
\curveto(671.97684898,111.68345867)(671.86684909,111.68845866)(671.76684082,111.71845998)
\curveto(671.66684929,111.75845859)(671.59684936,111.81345854)(671.55684082,111.88345998)
\curveto(671.50684945,111.96345839)(671.48184948,112.08345827)(671.48184082,112.24345998)
\curveto(671.48184948,112.40345795)(671.46684949,112.53845781)(671.43684082,112.64845998)
\curveto(671.42684953,112.69845765)(671.42184954,112.7534576)(671.42184082,112.81345998)
\curveto(671.41184955,112.87345748)(671.39684956,112.93345742)(671.37684082,112.99345998)
\curveto(671.32684963,113.14345721)(671.27684968,113.28845706)(671.22684082,113.42845998)
\curveto(671.16684979,113.56845678)(671.09684986,113.70345665)(671.01684082,113.83345998)
\curveto(670.92685003,113.97345638)(670.82185014,114.09345626)(670.70184082,114.19345998)
\curveto(670.58185038,114.29345606)(670.45185051,114.38845596)(670.31184082,114.47845998)
\curveto(670.21185075,114.53845581)(670.10185086,114.58345577)(669.98184082,114.61345998)
\curveto(669.8618511,114.6534557)(669.7568512,114.70345565)(669.66684082,114.76345998)
\curveto(669.60685135,114.81345554)(669.56685139,114.88345547)(669.54684082,114.97345998)
\curveto(669.53685142,114.99345536)(669.53185143,115.01845533)(669.53184082,115.04845998)
\curveto(669.53185143,115.07845527)(669.52685143,115.10345525)(669.51684082,115.12345998)
}
}
{
\newrgbcolor{curcolor}{0 0 0}
\pscustom[linestyle=none,fillstyle=solid,fillcolor=curcolor]
{
\newpath
\moveto(690.35315674,37.28705373)
\curveto(690.35316743,37.35704805)(690.35316743,37.43704797)(690.35315674,37.52705373)
\curveto(690.34316744,37.61704779)(690.34316744,37.70204771)(690.35315674,37.78205373)
\curveto(690.35316743,37.87204754)(690.36316742,37.95204746)(690.38315674,38.02205373)
\curveto(690.40316738,38.10204731)(690.43316735,38.15704725)(690.47315674,38.18705373)
\curveto(690.52316726,38.21704719)(690.59816719,38.23704717)(690.69815674,38.24705373)
\curveto(690.788167,38.26704714)(690.89316689,38.27704713)(691.01315674,38.27705373)
\curveto(691.12316666,38.28704712)(691.23816655,38.28704712)(691.35815674,38.27705373)
\lineto(691.65815674,38.27705373)
\lineto(694.67315674,38.27705373)
\lineto(697.56815674,38.27705373)
\curveto(697.89815989,38.27704713)(698.22315956,38.27204714)(698.54315674,38.26205373)
\curveto(698.85315893,38.26204715)(699.13315865,38.22204719)(699.38315674,38.14205373)
\curveto(699.73315805,38.02204739)(700.02815776,37.86704754)(700.26815674,37.67705373)
\curveto(700.49815729,37.48704792)(700.69815709,37.24704816)(700.86815674,36.95705373)
\curveto(700.91815687,36.89704851)(700.95315683,36.83204858)(700.97315674,36.76205373)
\curveto(700.99315679,36.70204871)(701.01815677,36.63204878)(701.04815674,36.55205373)
\curveto(701.09815669,36.43204898)(701.13315665,36.30204911)(701.15315674,36.16205373)
\curveto(701.1831566,36.03204938)(701.21315657,35.89704951)(701.24315674,35.75705373)
\curveto(701.26315652,35.7070497)(701.26815652,35.65704975)(701.25815674,35.60705373)
\curveto(701.24815654,35.55704985)(701.24815654,35.50204991)(701.25815674,35.44205373)
\curveto(701.26815652,35.42204999)(701.26815652,35.39705001)(701.25815674,35.36705373)
\curveto(701.25815653,35.33705007)(701.26315652,35.3120501)(701.27315674,35.29205373)
\curveto(701.2831565,35.25205016)(701.2881565,35.19705021)(701.28815674,35.12705373)
\curveto(701.2881565,35.05705035)(701.2831565,35.00205041)(701.27315674,34.96205373)
\curveto(701.26315652,34.9120505)(701.26315652,34.85705055)(701.27315674,34.79705373)
\curveto(701.2831565,34.73705067)(701.27815651,34.68205073)(701.25815674,34.63205373)
\curveto(701.22815656,34.50205091)(701.20815658,34.37705103)(701.19815674,34.25705373)
\curveto(701.1881566,34.13705127)(701.16315662,34.02205139)(701.12315674,33.91205373)
\curveto(701.00315678,33.54205187)(700.83315695,33.22205219)(700.61315674,32.95205373)
\curveto(700.39315739,32.68205273)(700.11315767,32.47205294)(699.77315674,32.32205373)
\curveto(699.65315813,32.27205314)(699.52815826,32.22705318)(699.39815674,32.18705373)
\curveto(699.26815852,32.15705325)(699.13315865,32.12205329)(698.99315674,32.08205373)
\curveto(698.94315884,32.07205334)(698.90315888,32.06705334)(698.87315674,32.06705373)
\curveto(698.83315895,32.06705334)(698.788159,32.06205335)(698.73815674,32.05205373)
\curveto(698.70815908,32.04205337)(698.67315911,32.03705337)(698.63315674,32.03705373)
\curveto(698.5831592,32.03705337)(698.54315924,32.03205338)(698.51315674,32.02205373)
\lineto(698.34815674,32.02205373)
\curveto(698.26815952,32.00205341)(698.16815962,31.99705341)(698.04815674,32.00705373)
\curveto(697.91815987,32.01705339)(697.82815996,32.03205338)(697.77815674,32.05205373)
\curveto(697.6881601,32.07205334)(697.62316016,32.12705328)(697.58315674,32.21705373)
\curveto(697.56316022,32.24705316)(697.55816023,32.27705313)(697.56815674,32.30705373)
\curveto(697.56816022,32.33705307)(697.56316022,32.37705303)(697.55315674,32.42705373)
\curveto(697.54316024,32.46705294)(697.53816025,32.5070529)(697.53815674,32.54705373)
\lineto(697.53815674,32.69705373)
\curveto(697.53816025,32.81705259)(697.54316024,32.93705247)(697.55315674,33.05705373)
\curveto(697.55316023,33.18705222)(697.5881602,33.27705213)(697.65815674,33.32705373)
\curveto(697.71816007,33.36705204)(697.77816001,33.38705202)(697.83815674,33.38705373)
\curveto(697.89815989,33.38705202)(697.96815982,33.39705201)(698.04815674,33.41705373)
\curveto(698.07815971,33.42705198)(698.11315967,33.42705198)(698.15315674,33.41705373)
\curveto(698.1831596,33.41705199)(698.20815958,33.42205199)(698.22815674,33.43205373)
\lineto(698.43815674,33.43205373)
\curveto(698.4881593,33.45205196)(698.53815925,33.45705195)(698.58815674,33.44705373)
\curveto(698.62815916,33.44705196)(698.67315911,33.45705195)(698.72315674,33.47705373)
\curveto(698.85315893,33.5070519)(698.97815881,33.53705187)(699.09815674,33.56705373)
\curveto(699.20815858,33.59705181)(699.31315847,33.64205177)(699.41315674,33.70205373)
\curveto(699.70315808,33.87205154)(699.90815788,34.14205127)(700.02815674,34.51205373)
\curveto(700.04815774,34.56205085)(700.06315772,34.6120508)(700.07315674,34.66205373)
\curveto(700.07315771,34.72205069)(700.0831577,34.77705063)(700.10315674,34.82705373)
\lineto(700.10315674,34.90205373)
\curveto(700.11315767,34.97205044)(700.12315766,35.06705034)(700.13315674,35.18705373)
\curveto(700.13315765,35.31705009)(700.12315766,35.41704999)(700.10315674,35.48705373)
\curveto(700.0831577,35.55704985)(700.06815772,35.62704978)(700.05815674,35.69705373)
\curveto(700.03815775,35.77704963)(700.01815777,35.84704956)(699.99815674,35.90705373)
\curveto(699.83815795,36.28704912)(699.56315822,36.56204885)(699.17315674,36.73205373)
\curveto(699.04315874,36.78204863)(698.8881589,36.81704859)(698.70815674,36.83705373)
\curveto(698.52815926,36.86704854)(698.34315944,36.88204853)(698.15315674,36.88205373)
\curveto(697.95315983,36.89204852)(697.75316003,36.89204852)(697.55315674,36.88205373)
\lineto(696.98315674,36.88205373)
\lineto(692.73815674,36.88205373)
\lineto(691.19315674,36.88205373)
\curveto(691.0831667,36.88204853)(690.96316682,36.87704853)(690.83315674,36.86705373)
\curveto(690.70316708,36.85704855)(690.59816719,36.87704853)(690.51815674,36.92705373)
\curveto(690.44816734,36.98704842)(690.39816739,37.06704834)(690.36815674,37.16705373)
\curveto(690.36816742,37.18704822)(690.36816742,37.2070482)(690.36815674,37.22705373)
\curveto(690.36816742,37.24704816)(690.36316742,37.26704814)(690.35315674,37.28705373)
}
}
{
\newrgbcolor{curcolor}{0 0 0}
\pscustom[linestyle=none,fillstyle=solid,fillcolor=curcolor]
{
\newpath
\moveto(693.30815674,40.82072561)
\lineto(693.30815674,41.25572561)
\curveto(693.30816448,41.40572364)(693.34816444,41.51072354)(693.42815674,41.57072561)
\curveto(693.50816428,41.62072343)(693.60816418,41.6457234)(693.72815674,41.64572561)
\curveto(693.84816394,41.65572339)(693.96816382,41.66072339)(694.08815674,41.66072561)
\lineto(695.51315674,41.66072561)
\lineto(697.77815674,41.66072561)
\lineto(698.46815674,41.66072561)
\curveto(698.69815909,41.66072339)(698.89815889,41.68572336)(699.06815674,41.73572561)
\curveto(699.51815827,41.89572315)(699.83315795,42.19572285)(700.01315674,42.63572561)
\curveto(700.10315768,42.85572219)(700.13815765,43.12072193)(700.11815674,43.43072561)
\curveto(700.0881577,43.74072131)(700.03315775,43.99072106)(699.95315674,44.18072561)
\curveto(699.81315797,44.51072054)(699.63815815,44.77072028)(699.42815674,44.96072561)
\curveto(699.20815858,45.16071989)(698.92315886,45.31571973)(698.57315674,45.42572561)
\curveto(698.49315929,45.45571959)(698.41315937,45.47571957)(698.33315674,45.48572561)
\curveto(698.25315953,45.49571955)(698.16815962,45.51071954)(698.07815674,45.53072561)
\curveto(698.02815976,45.54071951)(697.9831598,45.54071951)(697.94315674,45.53072561)
\curveto(697.90315988,45.53071952)(697.85815993,45.54071951)(697.80815674,45.56072561)
\lineto(697.49315674,45.56072561)
\curveto(697.41316037,45.58071947)(697.32316046,45.58571946)(697.22315674,45.57572561)
\curveto(697.11316067,45.56571948)(697.01316077,45.56071949)(696.92315674,45.56072561)
\lineto(695.75315674,45.56072561)
\lineto(694.16315674,45.56072561)
\curveto(694.04316374,45.56071949)(693.91816387,45.55571949)(693.78815674,45.54572561)
\curveto(693.64816414,45.5457195)(693.53816425,45.57071948)(693.45815674,45.62072561)
\curveto(693.40816438,45.66071939)(693.37816441,45.70571934)(693.36815674,45.75572561)
\curveto(693.34816444,45.81571923)(693.32816446,45.88571916)(693.30815674,45.96572561)
\lineto(693.30815674,46.19072561)
\curveto(693.30816448,46.31071874)(693.31316447,46.41571863)(693.32315674,46.50572561)
\curveto(693.33316445,46.60571844)(693.37816441,46.68071837)(693.45815674,46.73072561)
\curveto(693.50816428,46.78071827)(693.5831642,46.80571824)(693.68315674,46.80572561)
\lineto(693.96815674,46.80572561)
\lineto(694.98815674,46.80572561)
\lineto(699.02315674,46.80572561)
\lineto(700.37315674,46.80572561)
\curveto(700.49315729,46.80571824)(700.60815718,46.80071825)(700.71815674,46.79072561)
\curveto(700.81815697,46.79071826)(700.89315689,46.75571829)(700.94315674,46.68572561)
\curveto(700.97315681,46.6457184)(700.99815679,46.58571846)(701.01815674,46.50572561)
\curveto(701.02815676,46.42571862)(701.03815675,46.33571871)(701.04815674,46.23572561)
\curveto(701.04815674,46.1457189)(701.04315674,46.05571899)(701.03315674,45.96572561)
\curveto(701.02315676,45.88571916)(701.00315678,45.82571922)(700.97315674,45.78572561)
\curveto(700.93315685,45.73571931)(700.86815692,45.69071936)(700.77815674,45.65072561)
\curveto(700.73815705,45.64071941)(700.6831571,45.63071942)(700.61315674,45.62072561)
\curveto(700.54315724,45.62071943)(700.47815731,45.61571943)(700.41815674,45.60572561)
\curveto(700.34815744,45.59571945)(700.29315749,45.57571947)(700.25315674,45.54572561)
\curveto(700.21315757,45.51571953)(700.19815759,45.47071958)(700.20815674,45.41072561)
\curveto(700.22815756,45.33071972)(700.2881575,45.2507198)(700.38815674,45.17072561)
\curveto(700.47815731,45.09071996)(700.54815724,45.01572003)(700.59815674,44.94572561)
\curveto(700.75815703,44.72572032)(700.89815689,44.47572057)(701.01815674,44.19572561)
\curveto(701.06815672,44.08572096)(701.09815669,43.97072108)(701.10815674,43.85072561)
\curveto(701.12815666,43.74072131)(701.15315663,43.62572142)(701.18315674,43.50572561)
\curveto(701.19315659,43.45572159)(701.19315659,43.40072165)(701.18315674,43.34072561)
\curveto(701.17315661,43.29072176)(701.17815661,43.24072181)(701.19815674,43.19072561)
\curveto(701.21815657,43.09072196)(701.21815657,43.00072205)(701.19815674,42.92072561)
\lineto(701.19815674,42.77072561)
\curveto(701.17815661,42.72072233)(701.16815662,42.66072239)(701.16815674,42.59072561)
\curveto(701.16815662,42.53072252)(701.16315662,42.47572257)(701.15315674,42.42572561)
\curveto(701.13315665,42.38572266)(701.12315666,42.3457227)(701.12315674,42.30572561)
\curveto(701.13315665,42.27572277)(701.12815666,42.23572281)(701.10815674,42.18572561)
\lineto(701.04815674,41.94572561)
\curveto(701.02815676,41.87572317)(700.99815679,41.80072325)(700.95815674,41.72072561)
\curveto(700.84815694,41.46072359)(700.70315708,41.24072381)(700.52315674,41.06072561)
\curveto(700.33315745,40.89072416)(700.10815768,40.7507243)(699.84815674,40.64072561)
\curveto(699.75815803,40.60072445)(699.66815812,40.57072448)(699.57815674,40.55072561)
\lineto(699.27815674,40.49072561)
\curveto(699.21815857,40.47072458)(699.16315862,40.46072459)(699.11315674,40.46072561)
\curveto(699.05315873,40.47072458)(698.9881588,40.46572458)(698.91815674,40.44572561)
\curveto(698.89815889,40.43572461)(698.87315891,40.43072462)(698.84315674,40.43072561)
\curveto(698.80315898,40.43072462)(698.76815902,40.42572462)(698.73815674,40.41572561)
\lineto(698.58815674,40.41572561)
\curveto(698.54815924,40.40572464)(698.50315928,40.40072465)(698.45315674,40.40072561)
\curveto(698.39315939,40.41072464)(698.33815945,40.41572463)(698.28815674,40.41572561)
\lineto(697.68815674,40.41572561)
\lineto(694.92815674,40.41572561)
\lineto(693.96815674,40.41572561)
\lineto(693.69815674,40.41572561)
\curveto(693.60816418,40.41572463)(693.53316425,40.43572461)(693.47315674,40.47572561)
\curveto(693.40316438,40.51572453)(693.35316443,40.59072446)(693.32315674,40.70072561)
\curveto(693.31316447,40.72072433)(693.31316447,40.74072431)(693.32315674,40.76072561)
\curveto(693.32316446,40.78072427)(693.31816447,40.80072425)(693.30815674,40.82072561)
}
}
{
\newrgbcolor{curcolor}{0 0 0}
\pscustom[linestyle=none,fillstyle=solid,fillcolor=curcolor]
{
\newpath
\moveto(693.15815674,52.39533498)
\curveto(693.13816465,53.02532975)(693.22316456,53.53032924)(693.41315674,53.91033498)
\curveto(693.60316418,54.29032848)(693.8881639,54.59532818)(694.26815674,54.82533498)
\curveto(694.36816342,54.88532789)(694.47816331,54.93032784)(694.59815674,54.96033498)
\curveto(694.70816308,55.00032777)(694.82316296,55.03532774)(694.94315674,55.06533498)
\curveto(695.13316265,55.11532766)(695.33816245,55.14532763)(695.55815674,55.15533498)
\curveto(695.77816201,55.16532761)(696.00316178,55.1703276)(696.23315674,55.17033498)
\lineto(697.83815674,55.17033498)
\lineto(700.17815674,55.17033498)
\curveto(700.34815744,55.1703276)(700.51815727,55.16532761)(700.68815674,55.15533498)
\curveto(700.85815693,55.15532762)(700.96815682,55.09032768)(701.01815674,54.96033498)
\curveto(701.03815675,54.91032786)(701.04815674,54.85532792)(701.04815674,54.79533498)
\curveto(701.05815673,54.74532803)(701.06315672,54.69032808)(701.06315674,54.63033498)
\curveto(701.06315672,54.50032827)(701.05815673,54.3753284)(701.04815674,54.25533498)
\curveto(701.04815674,54.13532864)(701.00815678,54.05032872)(700.92815674,54.00033498)
\curveto(700.85815693,53.95032882)(700.76815702,53.92532885)(700.65815674,53.92533498)
\lineto(700.32815674,53.92533498)
\lineto(699.03815674,53.92533498)
\lineto(696.59315674,53.92533498)
\curveto(696.32316146,53.92532885)(696.05816173,53.92032885)(695.79815674,53.91033498)
\curveto(695.52816226,53.90032887)(695.29816249,53.85532892)(695.10815674,53.77533498)
\curveto(694.90816288,53.69532908)(694.74816304,53.5753292)(694.62815674,53.41533498)
\curveto(694.49816329,53.25532952)(694.39816339,53.0703297)(694.32815674,52.86033498)
\curveto(694.30816348,52.80032997)(694.29816349,52.73533004)(694.29815674,52.66533498)
\curveto(694.2881635,52.60533017)(694.27316351,52.54533023)(694.25315674,52.48533498)
\curveto(694.24316354,52.43533034)(694.24316354,52.35533042)(694.25315674,52.24533498)
\curveto(694.25316353,52.14533063)(694.25816353,52.0753307)(694.26815674,52.03533498)
\curveto(694.2881635,51.99533078)(694.29816349,51.96033081)(694.29815674,51.93033498)
\curveto(694.2881635,51.90033087)(694.2881635,51.86533091)(694.29815674,51.82533498)
\curveto(694.32816346,51.69533108)(694.36316342,51.5703312)(694.40315674,51.45033498)
\curveto(694.43316335,51.34033143)(694.47816331,51.23533154)(694.53815674,51.13533498)
\curveto(694.55816323,51.09533168)(694.57816321,51.06033171)(694.59815674,51.03033498)
\curveto(694.61816317,51.00033177)(694.63816315,50.96533181)(694.65815674,50.92533498)
\curveto(694.90816288,50.5753322)(695.2831625,50.32033245)(695.78315674,50.16033498)
\curveto(695.86316192,50.13033264)(695.94816184,50.11033266)(696.03815674,50.10033498)
\curveto(696.11816167,50.09033268)(696.19816159,50.0753327)(696.27815674,50.05533498)
\curveto(696.32816146,50.03533274)(696.37816141,50.03033274)(696.42815674,50.04033498)
\curveto(696.46816132,50.05033272)(696.50816128,50.04533273)(696.54815674,50.02533498)
\lineto(696.86315674,50.02533498)
\curveto(696.89316089,50.01533276)(696.92816086,50.01033276)(696.96815674,50.01033498)
\curveto(697.00816078,50.02033275)(697.05316073,50.02533275)(697.10315674,50.02533498)
\lineto(697.55315674,50.02533498)
\lineto(698.99315674,50.02533498)
\lineto(700.31315674,50.02533498)
\lineto(700.65815674,50.02533498)
\curveto(700.76815702,50.02533275)(700.85815693,50.00033277)(700.92815674,49.95033498)
\curveto(701.00815678,49.90033287)(701.04815674,49.81033296)(701.04815674,49.68033498)
\curveto(701.05815673,49.56033321)(701.06315672,49.43533334)(701.06315674,49.30533498)
\curveto(701.06315672,49.22533355)(701.05815673,49.15033362)(701.04815674,49.08033498)
\curveto(701.03815675,49.01033376)(701.01315677,48.95033382)(700.97315674,48.90033498)
\curveto(700.92315686,48.82033395)(700.82815696,48.78033399)(700.68815674,48.78033498)
\lineto(700.28315674,48.78033498)
\lineto(698.51315674,48.78033498)
\lineto(694.88315674,48.78033498)
\lineto(693.96815674,48.78033498)
\lineto(693.69815674,48.78033498)
\curveto(693.60816418,48.78033399)(693.53816425,48.80033397)(693.48815674,48.84033498)
\curveto(693.42816436,48.8703339)(693.3881644,48.92033385)(693.36815674,48.99033498)
\curveto(693.35816443,49.03033374)(693.34816444,49.08533369)(693.33815674,49.15533498)
\curveto(693.32816446,49.23533354)(693.32316446,49.31533346)(693.32315674,49.39533498)
\curveto(693.32316446,49.4753333)(693.32816446,49.55033322)(693.33815674,49.62033498)
\curveto(693.34816444,49.70033307)(693.36316442,49.75533302)(693.38315674,49.78533498)
\curveto(693.45316433,49.89533288)(693.54316424,49.94533283)(693.65315674,49.93533498)
\curveto(693.75316403,49.92533285)(693.86816392,49.94033283)(693.99815674,49.98033498)
\curveto(694.05816373,50.00033277)(694.10816368,50.04033273)(694.14815674,50.10033498)
\curveto(694.15816363,50.22033255)(694.11316367,50.31533246)(694.01315674,50.38533498)
\curveto(693.91316387,50.46533231)(693.83316395,50.54533223)(693.77315674,50.62533498)
\curveto(693.67316411,50.76533201)(693.5831642,50.90533187)(693.50315674,51.04533498)
\curveto(693.41316437,51.19533158)(693.33816445,51.36533141)(693.27815674,51.55533498)
\curveto(693.24816454,51.63533114)(693.22816456,51.72033105)(693.21815674,51.81033498)
\curveto(693.20816458,51.91033086)(693.19316459,52.00533077)(693.17315674,52.09533498)
\curveto(693.16316462,52.14533063)(693.15816463,52.19533058)(693.15815674,52.24533498)
\lineto(693.15815674,52.39533498)
}
}
{
\newrgbcolor{curcolor}{0 0 0}
\pscustom[linestyle=none,fillstyle=solid,fillcolor=curcolor]
{
}
}
{
\newrgbcolor{curcolor}{0 0 0}
\pscustom[linestyle=none,fillstyle=solid,fillcolor=curcolor]
{
\newpath
\moveto(690.42815674,65.05510061)
\curveto(690.42816736,65.15509575)(690.43816735,65.25009566)(690.45815674,65.34010061)
\curveto(690.46816732,65.43009548)(690.49816729,65.49509541)(690.54815674,65.53510061)
\curveto(690.62816716,65.59509531)(690.73316705,65.62509528)(690.86315674,65.62510061)
\lineto(691.25315674,65.62510061)
\lineto(692.75315674,65.62510061)
\lineto(699.14315674,65.62510061)
\lineto(700.31315674,65.62510061)
\lineto(700.62815674,65.62510061)
\curveto(700.72815706,65.63509527)(700.80815698,65.62009529)(700.86815674,65.58010061)
\curveto(700.94815684,65.53009538)(700.99815679,65.45509545)(701.01815674,65.35510061)
\curveto(701.02815676,65.26509564)(701.03315675,65.15509575)(701.03315674,65.02510061)
\lineto(701.03315674,64.80010061)
\curveto(701.01315677,64.72009619)(700.99815679,64.65009626)(700.98815674,64.59010061)
\curveto(700.96815682,64.53009638)(700.92815686,64.48009643)(700.86815674,64.44010061)
\curveto(700.80815698,64.40009651)(700.73315705,64.38009653)(700.64315674,64.38010061)
\lineto(700.34315674,64.38010061)
\lineto(699.24815674,64.38010061)
\lineto(693.90815674,64.38010061)
\curveto(693.81816397,64.36009655)(693.74316404,64.34509656)(693.68315674,64.33510061)
\curveto(693.61316417,64.33509657)(693.55316423,64.3050966)(693.50315674,64.24510061)
\curveto(693.45316433,64.17509673)(693.42816436,64.08509682)(693.42815674,63.97510061)
\curveto(693.41816437,63.87509703)(693.41316437,63.76509714)(693.41315674,63.64510061)
\lineto(693.41315674,62.50510061)
\lineto(693.41315674,62.01010061)
\curveto(693.40316438,61.85009906)(693.34316444,61.74009917)(693.23315674,61.68010061)
\curveto(693.20316458,61.66009925)(693.17316461,61.65009926)(693.14315674,61.65010061)
\curveto(693.10316468,61.65009926)(693.05816473,61.64509926)(693.00815674,61.63510061)
\curveto(692.8881649,61.61509929)(692.77816501,61.62009929)(692.67815674,61.65010061)
\curveto(692.57816521,61.69009922)(692.50816528,61.74509916)(692.46815674,61.81510061)
\curveto(692.41816537,61.89509901)(692.39316539,62.01509889)(692.39315674,62.17510061)
\curveto(692.39316539,62.33509857)(692.37816541,62.47009844)(692.34815674,62.58010061)
\curveto(692.33816545,62.63009828)(692.33316545,62.68509822)(692.33315674,62.74510061)
\curveto(692.32316546,62.8050981)(692.30816548,62.86509804)(692.28815674,62.92510061)
\curveto(692.23816555,63.07509783)(692.1881656,63.22009769)(692.13815674,63.36010061)
\curveto(692.07816571,63.50009741)(692.00816578,63.63509727)(691.92815674,63.76510061)
\curveto(691.83816595,63.905097)(691.73316605,64.02509688)(691.61315674,64.12510061)
\curveto(691.49316629,64.22509668)(691.36316642,64.32009659)(691.22315674,64.41010061)
\curveto(691.12316666,64.47009644)(691.01316677,64.51509639)(690.89315674,64.54510061)
\curveto(690.77316701,64.58509632)(690.66816712,64.63509627)(690.57815674,64.69510061)
\curveto(690.51816727,64.74509616)(690.47816731,64.81509609)(690.45815674,64.90510061)
\curveto(690.44816734,64.92509598)(690.44316734,64.95009596)(690.44315674,64.98010061)
\curveto(690.44316734,65.0100959)(690.43816735,65.03509587)(690.42815674,65.05510061)
}
}
{
\newrgbcolor{curcolor}{0 0 0}
\pscustom[linestyle=none,fillstyle=solid,fillcolor=curcolor]
{
\newpath
\moveto(695.94815674,76.34470998)
\lineto(696.20315674,76.34470998)
\curveto(696.2831615,76.35470228)(696.35816143,76.34970228)(696.42815674,76.32970998)
\lineto(696.66815674,76.32970998)
\lineto(696.83315674,76.32970998)
\curveto(696.93316085,76.30970232)(697.03816075,76.29970233)(697.14815674,76.29970998)
\curveto(697.24816054,76.29970233)(697.34816044,76.28970234)(697.44815674,76.26970998)
\lineto(697.59815674,76.26970998)
\curveto(697.73816005,76.23970239)(697.87815991,76.21970241)(698.01815674,76.20970998)
\curveto(698.14815964,76.19970243)(698.27815951,76.17470246)(698.40815674,76.13470998)
\curveto(698.4881593,76.11470252)(698.57315921,76.09470254)(698.66315674,76.07470998)
\lineto(698.90315674,76.01470998)
\lineto(699.20315674,75.89470998)
\curveto(699.29315849,75.86470277)(699.3831584,75.8297028)(699.47315674,75.78970998)
\curveto(699.69315809,75.68970294)(699.90815788,75.55470308)(700.11815674,75.38470998)
\curveto(700.32815746,75.22470341)(700.49815729,75.04970358)(700.62815674,74.85970998)
\curveto(700.66815712,74.80970382)(700.70815708,74.74970388)(700.74815674,74.67970998)
\curveto(700.77815701,74.61970401)(700.81315697,74.55970407)(700.85315674,74.49970998)
\curveto(700.90315688,74.41970421)(700.94315684,74.32470431)(700.97315674,74.21470998)
\curveto(701.00315678,74.10470453)(701.03315675,73.99970463)(701.06315674,73.89970998)
\curveto(701.10315668,73.78970484)(701.12815666,73.67970495)(701.13815674,73.56970998)
\curveto(701.14815664,73.45970517)(701.16315662,73.34470529)(701.18315674,73.22470998)
\curveto(701.19315659,73.18470545)(701.19315659,73.13970549)(701.18315674,73.08970998)
\curveto(701.1831566,73.04970558)(701.1881566,73.00970562)(701.19815674,72.96970998)
\curveto(701.20815658,72.9297057)(701.21315657,72.87470576)(701.21315674,72.80470998)
\curveto(701.21315657,72.7347059)(701.20815658,72.68470595)(701.19815674,72.65470998)
\curveto(701.17815661,72.60470603)(701.17315661,72.55970607)(701.18315674,72.51970998)
\curveto(701.19315659,72.47970615)(701.19315659,72.44470619)(701.18315674,72.41470998)
\lineto(701.18315674,72.32470998)
\curveto(701.16315662,72.26470637)(701.14815664,72.19970643)(701.13815674,72.12970998)
\curveto(701.13815665,72.06970656)(701.13315665,72.00470663)(701.12315674,71.93470998)
\curveto(701.07315671,71.76470687)(701.02315676,71.60470703)(700.97315674,71.45470998)
\curveto(700.92315686,71.30470733)(700.85815693,71.15970747)(700.77815674,71.01970998)
\curveto(700.73815705,70.96970766)(700.70815708,70.91470772)(700.68815674,70.85470998)
\curveto(700.65815713,70.80470783)(700.62315716,70.75470788)(700.58315674,70.70470998)
\curveto(700.40315738,70.46470817)(700.1831576,70.26470837)(699.92315674,70.10470998)
\curveto(699.66315812,69.94470869)(699.37815841,69.80470883)(699.06815674,69.68470998)
\curveto(698.92815886,69.62470901)(698.788159,69.57970905)(698.64815674,69.54970998)
\curveto(698.49815929,69.51970911)(698.34315944,69.48470915)(698.18315674,69.44470998)
\curveto(698.07315971,69.42470921)(697.96315982,69.40970922)(697.85315674,69.39970998)
\curveto(697.74316004,69.38970924)(697.63316015,69.37470926)(697.52315674,69.35470998)
\curveto(697.4831603,69.34470929)(697.44316034,69.33970929)(697.40315674,69.33970998)
\curveto(697.36316042,69.34970928)(697.32316046,69.34970928)(697.28315674,69.33970998)
\curveto(697.23316055,69.3297093)(697.1831606,69.32470931)(697.13315674,69.32470998)
\lineto(696.96815674,69.32470998)
\curveto(696.91816087,69.30470933)(696.86816092,69.29970933)(696.81815674,69.30970998)
\curveto(696.75816103,69.31970931)(696.70316108,69.31970931)(696.65315674,69.30970998)
\curveto(696.61316117,69.29970933)(696.56816122,69.29970933)(696.51815674,69.30970998)
\curveto(696.46816132,69.31970931)(696.41816137,69.31470932)(696.36815674,69.29470998)
\curveto(696.29816149,69.27470936)(696.22316156,69.26970936)(696.14315674,69.27970998)
\curveto(696.05316173,69.28970934)(695.96816182,69.29470934)(695.88815674,69.29470998)
\curveto(695.79816199,69.29470934)(695.69816209,69.28970934)(695.58815674,69.27970998)
\curveto(695.46816232,69.26970936)(695.36816242,69.27470936)(695.28815674,69.29470998)
\lineto(695.00315674,69.29470998)
\lineto(694.37315674,69.33970998)
\curveto(694.27316351,69.34970928)(694.17816361,69.35970927)(694.08815674,69.36970998)
\lineto(693.78815674,69.39970998)
\curveto(693.73816405,69.41970921)(693.6881641,69.42470921)(693.63815674,69.41470998)
\curveto(693.57816421,69.41470922)(693.52316426,69.42470921)(693.47315674,69.44470998)
\curveto(693.30316448,69.49470914)(693.13816465,69.5347091)(692.97815674,69.56470998)
\curveto(692.80816498,69.59470904)(692.64816514,69.64470899)(692.49815674,69.71470998)
\curveto(692.03816575,69.90470873)(691.66316612,70.12470851)(691.37315674,70.37470998)
\curveto(691.0831667,70.634708)(690.83816695,70.99470764)(690.63815674,71.45470998)
\curveto(690.5881672,71.58470705)(690.55316723,71.71470692)(690.53315674,71.84470998)
\curveto(690.51316727,71.98470665)(690.4881673,72.12470651)(690.45815674,72.26470998)
\curveto(690.44816734,72.3347063)(690.44316734,72.39970623)(690.44315674,72.45970998)
\curveto(690.44316734,72.51970611)(690.43816735,72.58470605)(690.42815674,72.65470998)
\curveto(690.40816738,73.48470515)(690.55816723,74.15470448)(690.87815674,74.66470998)
\curveto(691.1881666,75.17470346)(691.62816616,75.55470308)(692.19815674,75.80470998)
\curveto(692.31816547,75.85470278)(692.44316534,75.89970273)(692.57315674,75.93970998)
\curveto(692.70316508,75.97970265)(692.83816495,76.02470261)(692.97815674,76.07470998)
\curveto(693.05816473,76.09470254)(693.14316464,76.10970252)(693.23315674,76.11970998)
\lineto(693.47315674,76.17970998)
\curveto(693.5831642,76.20970242)(693.69316409,76.22470241)(693.80315674,76.22470998)
\curveto(693.91316387,76.2347024)(694.02316376,76.24970238)(694.13315674,76.26970998)
\curveto(694.1831636,76.28970234)(694.22816356,76.29470234)(694.26815674,76.28470998)
\curveto(694.30816348,76.28470235)(694.34816344,76.28970234)(694.38815674,76.29970998)
\curveto(694.43816335,76.30970232)(694.49316329,76.30970232)(694.55315674,76.29970998)
\curveto(694.60316318,76.29970233)(694.65316313,76.30470233)(694.70315674,76.31470998)
\lineto(694.83815674,76.31470998)
\curveto(694.89816289,76.3347023)(694.96816282,76.3347023)(695.04815674,76.31470998)
\curveto(695.11816267,76.30470233)(695.1831626,76.30970232)(695.24315674,76.32970998)
\curveto(695.27316251,76.33970229)(695.31316247,76.34470229)(695.36315674,76.34470998)
\lineto(695.48315674,76.34470998)
\lineto(695.94815674,76.34470998)
\moveto(698.27315674,74.79970998)
\curveto(697.95315983,74.89970373)(697.5881602,74.95970367)(697.17815674,74.97970998)
\curveto(696.76816102,74.99970363)(696.35816143,75.00970362)(695.94815674,75.00970998)
\curveto(695.51816227,75.00970362)(695.09816269,74.99970363)(694.68815674,74.97970998)
\curveto(694.27816351,74.95970367)(693.89316389,74.91470372)(693.53315674,74.84470998)
\curveto(693.17316461,74.77470386)(692.85316493,74.66470397)(692.57315674,74.51470998)
\curveto(692.2831655,74.37470426)(692.04816574,74.17970445)(691.86815674,73.92970998)
\curveto(691.75816603,73.76970486)(691.67816611,73.58970504)(691.62815674,73.38970998)
\curveto(691.56816622,73.18970544)(691.53816625,72.94470569)(691.53815674,72.65470998)
\curveto(691.55816623,72.634706)(691.56816622,72.59970603)(691.56815674,72.54970998)
\curveto(691.55816623,72.49970613)(691.55816623,72.45970617)(691.56815674,72.42970998)
\curveto(691.5881662,72.34970628)(691.60816618,72.27470636)(691.62815674,72.20470998)
\curveto(691.63816615,72.14470649)(691.65816613,72.07970655)(691.68815674,72.00970998)
\curveto(691.80816598,71.73970689)(691.97816581,71.51970711)(692.19815674,71.34970998)
\curveto(692.40816538,71.18970744)(692.65316513,71.05470758)(692.93315674,70.94470998)
\curveto(693.04316474,70.89470774)(693.16316462,70.85470778)(693.29315674,70.82470998)
\curveto(693.41316437,70.80470783)(693.53816425,70.77970785)(693.66815674,70.74970998)
\curveto(693.71816407,70.7297079)(693.77316401,70.71970791)(693.83315674,70.71970998)
\curveto(693.8831639,70.71970791)(693.93316385,70.71470792)(693.98315674,70.70470998)
\curveto(694.07316371,70.69470794)(694.16816362,70.68470795)(694.26815674,70.67470998)
\curveto(694.35816343,70.66470797)(694.45316333,70.65470798)(694.55315674,70.64470998)
\curveto(694.63316315,70.64470799)(694.71816307,70.63970799)(694.80815674,70.62970998)
\lineto(695.04815674,70.62970998)
\lineto(695.22815674,70.62970998)
\curveto(695.25816253,70.61970801)(695.29316249,70.61470802)(695.33315674,70.61470998)
\lineto(695.46815674,70.61470998)
\lineto(695.91815674,70.61470998)
\curveto(695.99816179,70.61470802)(696.0831617,70.60970802)(696.17315674,70.59970998)
\curveto(696.25316153,70.59970803)(696.32816146,70.60970802)(696.39815674,70.62970998)
\lineto(696.66815674,70.62970998)
\curveto(696.6881611,70.629708)(696.71816107,70.62470801)(696.75815674,70.61470998)
\curveto(696.788161,70.61470802)(696.81316097,70.61970801)(696.83315674,70.62970998)
\curveto(696.93316085,70.63970799)(697.03316075,70.64470799)(697.13315674,70.64470998)
\curveto(697.22316056,70.65470798)(697.32316046,70.66470797)(697.43315674,70.67470998)
\curveto(697.55316023,70.70470793)(697.67816011,70.71970791)(697.80815674,70.71970998)
\curveto(697.92815986,70.7297079)(698.04315974,70.75470788)(698.15315674,70.79470998)
\curveto(698.45315933,70.87470776)(698.71815907,70.95970767)(698.94815674,71.04970998)
\curveto(699.17815861,71.14970748)(699.39315839,71.29470734)(699.59315674,71.48470998)
\curveto(699.79315799,71.69470694)(699.94315784,71.95970667)(700.04315674,72.27970998)
\curveto(700.06315772,72.31970631)(700.07315771,72.35470628)(700.07315674,72.38470998)
\curveto(700.06315772,72.42470621)(700.06815772,72.46970616)(700.08815674,72.51970998)
\curveto(700.09815769,72.55970607)(700.10815768,72.629706)(700.11815674,72.72970998)
\curveto(700.12815766,72.83970579)(700.12315766,72.92470571)(700.10315674,72.98470998)
\curveto(700.0831577,73.05470558)(700.07315771,73.12470551)(700.07315674,73.19470998)
\curveto(700.06315772,73.26470537)(700.04815774,73.3297053)(700.02815674,73.38970998)
\curveto(699.96815782,73.58970504)(699.8831579,73.76970486)(699.77315674,73.92970998)
\curveto(699.75315803,73.95970467)(699.73315805,73.98470465)(699.71315674,74.00470998)
\lineto(699.65315674,74.06470998)
\curveto(699.63315815,74.10470453)(699.59315819,74.15470448)(699.53315674,74.21470998)
\curveto(699.39315839,74.31470432)(699.26315852,74.39970423)(699.14315674,74.46970998)
\curveto(699.02315876,74.53970409)(698.87815891,74.60970402)(698.70815674,74.67970998)
\curveto(698.63815915,74.70970392)(698.56815922,74.7297039)(698.49815674,74.73970998)
\curveto(698.42815936,74.75970387)(698.35315943,74.77970385)(698.27315674,74.79970998)
}
}
{
\newrgbcolor{curcolor}{0 0 0}
\pscustom[linestyle=none,fillstyle=solid,fillcolor=curcolor]
{
\newpath
\moveto(699.39815674,78.63431936)
\lineto(699.39815674,79.26431936)
\lineto(699.39815674,79.45931936)
\curveto(699.39815839,79.52931683)(699.40815838,79.58931677)(699.42815674,79.63931936)
\curveto(699.46815832,79.70931665)(699.50815828,79.7593166)(699.54815674,79.78931936)
\curveto(699.59815819,79.82931653)(699.66315812,79.84931651)(699.74315674,79.84931936)
\curveto(699.82315796,79.8593165)(699.90815788,79.86431649)(699.99815674,79.86431936)
\lineto(700.71815674,79.86431936)
\curveto(701.19815659,79.86431649)(701.60815618,79.80431655)(701.94815674,79.68431936)
\curveto(702.2881555,79.56431679)(702.56315522,79.36931699)(702.77315674,79.09931936)
\curveto(702.82315496,79.02931733)(702.86815492,78.9593174)(702.90815674,78.88931936)
\curveto(702.95815483,78.82931753)(703.00315478,78.7543176)(703.04315674,78.66431936)
\curveto(703.05315473,78.64431771)(703.06315472,78.61431774)(703.07315674,78.57431936)
\curveto(703.09315469,78.53431782)(703.09815469,78.48931787)(703.08815674,78.43931936)
\curveto(703.05815473,78.34931801)(702.9831548,78.29431806)(702.86315674,78.27431936)
\curveto(702.75315503,78.2543181)(702.65815513,78.26931809)(702.57815674,78.31931936)
\curveto(702.50815528,78.34931801)(702.44315534,78.39431796)(702.38315674,78.45431936)
\curveto(702.33315545,78.52431783)(702.2831555,78.58931777)(702.23315674,78.64931936)
\curveto(702.1831556,78.71931764)(702.10815568,78.77931758)(702.00815674,78.82931936)
\curveto(701.91815587,78.88931747)(701.82815596,78.93931742)(701.73815674,78.97931936)
\curveto(701.70815608,78.99931736)(701.64815614,79.02431733)(701.55815674,79.05431936)
\curveto(701.47815631,79.08431727)(701.40815638,79.08931727)(701.34815674,79.06931936)
\curveto(701.20815658,79.03931732)(701.11815667,78.97931738)(701.07815674,78.88931936)
\curveto(701.04815674,78.80931755)(701.03315675,78.71931764)(701.03315674,78.61931936)
\curveto(701.03315675,78.51931784)(701.00815678,78.43431792)(700.95815674,78.36431936)
\curveto(700.8881569,78.27431808)(700.74815704,78.22931813)(700.53815674,78.22931936)
\lineto(699.98315674,78.22931936)
\lineto(699.75815674,78.22931936)
\curveto(699.67815811,78.23931812)(699.61315817,78.2593181)(699.56315674,78.28931936)
\curveto(699.4831583,78.34931801)(699.43815835,78.41931794)(699.42815674,78.49931936)
\curveto(699.41815837,78.51931784)(699.41315837,78.53931782)(699.41315674,78.55931936)
\curveto(699.41315837,78.58931777)(699.40815838,78.61431774)(699.39815674,78.63431936)
}
}
{
\newrgbcolor{curcolor}{0 0 0}
\pscustom[linestyle=none,fillstyle=solid,fillcolor=curcolor]
{
}
}
{
\newrgbcolor{curcolor}{0 0 0}
\pscustom[linestyle=none,fillstyle=solid,fillcolor=curcolor]
{
\newpath
\moveto(690.42815674,89.26463186)
\curveto(690.41816737,89.95462722)(690.53816725,90.55462662)(690.78815674,91.06463186)
\curveto(691.03816675,91.58462559)(691.37316641,91.9796252)(691.79315674,92.24963186)
\curveto(691.87316591,92.29962488)(691.96316582,92.34462483)(692.06315674,92.38463186)
\curveto(692.15316563,92.42462475)(692.24816554,92.46962471)(692.34815674,92.51963186)
\curveto(692.44816534,92.55962462)(692.54816524,92.58962459)(692.64815674,92.60963186)
\curveto(692.74816504,92.62962455)(692.85316493,92.64962453)(692.96315674,92.66963186)
\curveto(693.01316477,92.68962449)(693.05816473,92.69462448)(693.09815674,92.68463186)
\curveto(693.13816465,92.6746245)(693.1831646,92.6796245)(693.23315674,92.69963186)
\curveto(693.2831645,92.70962447)(693.36816442,92.71462446)(693.48815674,92.71463186)
\curveto(693.59816419,92.71462446)(693.6831641,92.70962447)(693.74315674,92.69963186)
\curveto(693.80316398,92.6796245)(693.86316392,92.66962451)(693.92315674,92.66963186)
\curveto(693.9831638,92.6796245)(694.04316374,92.6746245)(694.10315674,92.65463186)
\curveto(694.24316354,92.61462456)(694.37816341,92.5796246)(694.50815674,92.54963186)
\curveto(694.63816315,92.51962466)(694.76316302,92.4796247)(694.88315674,92.42963186)
\curveto(695.02316276,92.36962481)(695.14816264,92.29962488)(695.25815674,92.21963186)
\curveto(695.36816242,92.14962503)(695.47816231,92.0746251)(695.58815674,91.99463186)
\lineto(695.64815674,91.93463186)
\curveto(695.66816212,91.92462525)(695.6881621,91.90962527)(695.70815674,91.88963186)
\curveto(695.86816192,91.76962541)(696.01316177,91.63462554)(696.14315674,91.48463186)
\curveto(696.27316151,91.33462584)(696.39816139,91.174626)(696.51815674,91.00463186)
\curveto(696.73816105,90.69462648)(696.94316084,90.39962678)(697.13315674,90.11963186)
\curveto(697.27316051,89.88962729)(697.40816038,89.65962752)(697.53815674,89.42963186)
\curveto(697.66816012,89.20962797)(697.80315998,88.98962819)(697.94315674,88.76963186)
\curveto(698.11315967,88.51962866)(698.29315949,88.2796289)(698.48315674,88.04963186)
\curveto(698.67315911,87.82962935)(698.89815889,87.63962954)(699.15815674,87.47963186)
\curveto(699.21815857,87.43962974)(699.27815851,87.40462977)(699.33815674,87.37463186)
\curveto(699.3881584,87.34462983)(699.45315833,87.31462986)(699.53315674,87.28463186)
\curveto(699.60315818,87.26462991)(699.66315812,87.25962992)(699.71315674,87.26963186)
\curveto(699.783158,87.28962989)(699.83815795,87.32462985)(699.87815674,87.37463186)
\curveto(699.90815788,87.42462975)(699.92815786,87.48462969)(699.93815674,87.55463186)
\lineto(699.93815674,87.79463186)
\lineto(699.93815674,88.54463186)
\lineto(699.93815674,91.34963186)
\lineto(699.93815674,92.00963186)
\curveto(699.93815785,92.09962508)(699.94315784,92.18462499)(699.95315674,92.26463186)
\curveto(699.95315783,92.34462483)(699.97315781,92.40962477)(700.01315674,92.45963186)
\curveto(700.05315773,92.50962467)(700.12815766,92.54962463)(700.23815674,92.57963186)
\curveto(700.33815745,92.61962456)(700.43815735,92.62962455)(700.53815674,92.60963186)
\lineto(700.67315674,92.60963186)
\curveto(700.74315704,92.58962459)(700.80315698,92.56962461)(700.85315674,92.54963186)
\curveto(700.90315688,92.52962465)(700.94315684,92.49462468)(700.97315674,92.44463186)
\curveto(701.01315677,92.39462478)(701.03315675,92.32462485)(701.03315674,92.23463186)
\lineto(701.03315674,91.96463186)
\lineto(701.03315674,91.06463186)
\lineto(701.03315674,87.55463186)
\lineto(701.03315674,86.48963186)
\curveto(701.03315675,86.40963077)(701.03815675,86.31963086)(701.04815674,86.21963186)
\curveto(701.04815674,86.11963106)(701.03815675,86.03463114)(701.01815674,85.96463186)
\curveto(700.94815684,85.75463142)(700.76815702,85.68963149)(700.47815674,85.76963186)
\curveto(700.43815735,85.7796314)(700.40315738,85.7796314)(700.37315674,85.76963186)
\curveto(700.33315745,85.76963141)(700.2881575,85.7796314)(700.23815674,85.79963186)
\curveto(700.15815763,85.81963136)(700.07315771,85.83963134)(699.98315674,85.85963186)
\curveto(699.89315789,85.8796313)(699.80815798,85.90463127)(699.72815674,85.93463186)
\curveto(699.23815855,86.09463108)(698.82315896,86.29463088)(698.48315674,86.53463186)
\curveto(698.23315955,86.71463046)(698.00815978,86.91963026)(697.80815674,87.14963186)
\curveto(697.59816019,87.3796298)(697.40316038,87.61962956)(697.22315674,87.86963186)
\curveto(697.04316074,88.12962905)(696.87316091,88.39462878)(696.71315674,88.66463186)
\curveto(696.54316124,88.94462823)(696.36816142,89.21462796)(696.18815674,89.47463186)
\curveto(696.10816168,89.58462759)(696.03316175,89.68962749)(695.96315674,89.78963186)
\curveto(695.89316189,89.89962728)(695.81816197,90.00962717)(695.73815674,90.11963186)
\curveto(695.70816208,90.15962702)(695.67816211,90.19462698)(695.64815674,90.22463186)
\curveto(695.60816218,90.26462691)(695.57816221,90.30462687)(695.55815674,90.34463186)
\curveto(695.44816234,90.48462669)(695.32316246,90.60962657)(695.18315674,90.71963186)
\curveto(695.15316263,90.73962644)(695.12816266,90.76462641)(695.10815674,90.79463186)
\curveto(695.07816271,90.82462635)(695.04816274,90.84962633)(695.01815674,90.86963186)
\curveto(694.91816287,90.94962623)(694.81816297,91.01462616)(694.71815674,91.06463186)
\curveto(694.61816317,91.12462605)(694.50816328,91.179626)(694.38815674,91.22963186)
\curveto(694.31816347,91.25962592)(694.24316354,91.2796259)(694.16315674,91.28963186)
\lineto(693.92315674,91.34963186)
\lineto(693.83315674,91.34963186)
\curveto(693.80316398,91.35962582)(693.77316401,91.36462581)(693.74315674,91.36463186)
\curveto(693.67316411,91.38462579)(693.57816421,91.38962579)(693.45815674,91.37963186)
\curveto(693.32816446,91.3796258)(693.22816456,91.36962581)(693.15815674,91.34963186)
\curveto(693.07816471,91.32962585)(693.00316478,91.30962587)(692.93315674,91.28963186)
\curveto(692.85316493,91.2796259)(692.77316501,91.25962592)(692.69315674,91.22963186)
\curveto(692.45316533,91.11962606)(692.25316553,90.96962621)(692.09315674,90.77963186)
\curveto(691.92316586,90.59962658)(691.783166,90.3796268)(691.67315674,90.11963186)
\curveto(691.65316613,90.04962713)(691.63816615,89.9796272)(691.62815674,89.90963186)
\curveto(691.60816618,89.83962734)(691.5881662,89.76462741)(691.56815674,89.68463186)
\curveto(691.54816624,89.60462757)(691.53816625,89.49462768)(691.53815674,89.35463186)
\curveto(691.53816625,89.22462795)(691.54816624,89.11962806)(691.56815674,89.03963186)
\curveto(691.57816621,88.9796282)(691.5831662,88.92462825)(691.58315674,88.87463186)
\curveto(691.5831662,88.82462835)(691.59316619,88.7746284)(691.61315674,88.72463186)
\curveto(691.65316613,88.62462855)(691.69316609,88.52962865)(691.73315674,88.43963186)
\curveto(691.77316601,88.35962882)(691.81816597,88.2796289)(691.86815674,88.19963186)
\curveto(691.8881659,88.16962901)(691.91316587,88.13962904)(691.94315674,88.10963186)
\curveto(691.97316581,88.08962909)(691.99816579,88.06462911)(692.01815674,88.03463186)
\lineto(692.09315674,87.95963186)
\curveto(692.11316567,87.92962925)(692.13316565,87.90462927)(692.15315674,87.88463186)
\lineto(692.36315674,87.73463186)
\curveto(692.42316536,87.69462948)(692.4881653,87.64962953)(692.55815674,87.59963186)
\curveto(692.64816514,87.53962964)(692.75316503,87.48962969)(692.87315674,87.44963186)
\curveto(692.9831648,87.41962976)(693.09316469,87.38462979)(693.20315674,87.34463186)
\curveto(693.31316447,87.30462987)(693.45816433,87.2796299)(693.63815674,87.26963186)
\curveto(693.80816398,87.25962992)(693.93316385,87.22962995)(694.01315674,87.17963186)
\curveto(694.09316369,87.12963005)(694.13816365,87.05463012)(694.14815674,86.95463186)
\curveto(694.15816363,86.85463032)(694.16316362,86.74463043)(694.16315674,86.62463186)
\curveto(694.16316362,86.58463059)(694.16816362,86.54463063)(694.17815674,86.50463186)
\curveto(694.17816361,86.46463071)(694.17316361,86.42963075)(694.16315674,86.39963186)
\curveto(694.14316364,86.34963083)(694.13316365,86.29963088)(694.13315674,86.24963186)
\curveto(694.13316365,86.20963097)(694.12316366,86.16963101)(694.10315674,86.12963186)
\curveto(694.04316374,86.03963114)(693.90816388,85.99463118)(693.69815674,85.99463186)
\lineto(693.57815674,85.99463186)
\curveto(693.51816427,86.00463117)(693.45816433,86.00963117)(693.39815674,86.00963186)
\curveto(693.32816446,86.01963116)(693.26316452,86.02963115)(693.20315674,86.03963186)
\curveto(693.09316469,86.05963112)(692.99316479,86.0796311)(692.90315674,86.09963186)
\curveto(692.80316498,86.11963106)(692.70816508,86.14963103)(692.61815674,86.18963186)
\curveto(692.54816524,86.20963097)(692.4881653,86.22963095)(692.43815674,86.24963186)
\lineto(692.25815674,86.30963186)
\curveto(691.99816579,86.42963075)(691.75316603,86.58463059)(691.52315674,86.77463186)
\curveto(691.29316649,86.9746302)(691.10816668,87.18962999)(690.96815674,87.41963186)
\curveto(690.8881669,87.52962965)(690.82316696,87.64462953)(690.77315674,87.76463186)
\lineto(690.62315674,88.15463186)
\curveto(690.57316721,88.26462891)(690.54316724,88.3796288)(690.53315674,88.49963186)
\curveto(690.51316727,88.61962856)(690.4881673,88.74462843)(690.45815674,88.87463186)
\curveto(690.45816733,88.94462823)(690.45816733,89.00962817)(690.45815674,89.06963186)
\curveto(690.44816734,89.12962805)(690.43816735,89.19462798)(690.42815674,89.26463186)
}
}
{
\newrgbcolor{curcolor}{0 0 0}
\pscustom[linestyle=none,fillstyle=solid,fillcolor=curcolor]
{
\newpath
\moveto(695.94815674,101.36424123)
\lineto(696.20315674,101.36424123)
\curveto(696.2831615,101.37423353)(696.35816143,101.36923353)(696.42815674,101.34924123)
\lineto(696.66815674,101.34924123)
\lineto(696.83315674,101.34924123)
\curveto(696.93316085,101.32923357)(697.03816075,101.31923358)(697.14815674,101.31924123)
\curveto(697.24816054,101.31923358)(697.34816044,101.30923359)(697.44815674,101.28924123)
\lineto(697.59815674,101.28924123)
\curveto(697.73816005,101.25923364)(697.87815991,101.23923366)(698.01815674,101.22924123)
\curveto(698.14815964,101.21923368)(698.27815951,101.19423371)(698.40815674,101.15424123)
\curveto(698.4881593,101.13423377)(698.57315921,101.11423379)(698.66315674,101.09424123)
\lineto(698.90315674,101.03424123)
\lineto(699.20315674,100.91424123)
\curveto(699.29315849,100.88423402)(699.3831584,100.84923405)(699.47315674,100.80924123)
\curveto(699.69315809,100.70923419)(699.90815788,100.57423433)(700.11815674,100.40424123)
\curveto(700.32815746,100.24423466)(700.49815729,100.06923483)(700.62815674,99.87924123)
\curveto(700.66815712,99.82923507)(700.70815708,99.76923513)(700.74815674,99.69924123)
\curveto(700.77815701,99.63923526)(700.81315697,99.57923532)(700.85315674,99.51924123)
\curveto(700.90315688,99.43923546)(700.94315684,99.34423556)(700.97315674,99.23424123)
\curveto(701.00315678,99.12423578)(701.03315675,99.01923588)(701.06315674,98.91924123)
\curveto(701.10315668,98.80923609)(701.12815666,98.6992362)(701.13815674,98.58924123)
\curveto(701.14815664,98.47923642)(701.16315662,98.36423654)(701.18315674,98.24424123)
\curveto(701.19315659,98.2042367)(701.19315659,98.15923674)(701.18315674,98.10924123)
\curveto(701.1831566,98.06923683)(701.1881566,98.02923687)(701.19815674,97.98924123)
\curveto(701.20815658,97.94923695)(701.21315657,97.89423701)(701.21315674,97.82424123)
\curveto(701.21315657,97.75423715)(701.20815658,97.7042372)(701.19815674,97.67424123)
\curveto(701.17815661,97.62423728)(701.17315661,97.57923732)(701.18315674,97.53924123)
\curveto(701.19315659,97.4992374)(701.19315659,97.46423744)(701.18315674,97.43424123)
\lineto(701.18315674,97.34424123)
\curveto(701.16315662,97.28423762)(701.14815664,97.21923768)(701.13815674,97.14924123)
\curveto(701.13815665,97.08923781)(701.13315665,97.02423788)(701.12315674,96.95424123)
\curveto(701.07315671,96.78423812)(701.02315676,96.62423828)(700.97315674,96.47424123)
\curveto(700.92315686,96.32423858)(700.85815693,96.17923872)(700.77815674,96.03924123)
\curveto(700.73815705,95.98923891)(700.70815708,95.93423897)(700.68815674,95.87424123)
\curveto(700.65815713,95.82423908)(700.62315716,95.77423913)(700.58315674,95.72424123)
\curveto(700.40315738,95.48423942)(700.1831576,95.28423962)(699.92315674,95.12424123)
\curveto(699.66315812,94.96423994)(699.37815841,94.82424008)(699.06815674,94.70424123)
\curveto(698.92815886,94.64424026)(698.788159,94.5992403)(698.64815674,94.56924123)
\curveto(698.49815929,94.53924036)(698.34315944,94.5042404)(698.18315674,94.46424123)
\curveto(698.07315971,94.44424046)(697.96315982,94.42924047)(697.85315674,94.41924123)
\curveto(697.74316004,94.40924049)(697.63316015,94.39424051)(697.52315674,94.37424123)
\curveto(697.4831603,94.36424054)(697.44316034,94.35924054)(697.40315674,94.35924123)
\curveto(697.36316042,94.36924053)(697.32316046,94.36924053)(697.28315674,94.35924123)
\curveto(697.23316055,94.34924055)(697.1831606,94.34424056)(697.13315674,94.34424123)
\lineto(696.96815674,94.34424123)
\curveto(696.91816087,94.32424058)(696.86816092,94.31924058)(696.81815674,94.32924123)
\curveto(696.75816103,94.33924056)(696.70316108,94.33924056)(696.65315674,94.32924123)
\curveto(696.61316117,94.31924058)(696.56816122,94.31924058)(696.51815674,94.32924123)
\curveto(696.46816132,94.33924056)(696.41816137,94.33424057)(696.36815674,94.31424123)
\curveto(696.29816149,94.29424061)(696.22316156,94.28924061)(696.14315674,94.29924123)
\curveto(696.05316173,94.30924059)(695.96816182,94.31424059)(695.88815674,94.31424123)
\curveto(695.79816199,94.31424059)(695.69816209,94.30924059)(695.58815674,94.29924123)
\curveto(695.46816232,94.28924061)(695.36816242,94.29424061)(695.28815674,94.31424123)
\lineto(695.00315674,94.31424123)
\lineto(694.37315674,94.35924123)
\curveto(694.27316351,94.36924053)(694.17816361,94.37924052)(694.08815674,94.38924123)
\lineto(693.78815674,94.41924123)
\curveto(693.73816405,94.43924046)(693.6881641,94.44424046)(693.63815674,94.43424123)
\curveto(693.57816421,94.43424047)(693.52316426,94.44424046)(693.47315674,94.46424123)
\curveto(693.30316448,94.51424039)(693.13816465,94.55424035)(692.97815674,94.58424123)
\curveto(692.80816498,94.61424029)(692.64816514,94.66424024)(692.49815674,94.73424123)
\curveto(692.03816575,94.92423998)(691.66316612,95.14423976)(691.37315674,95.39424123)
\curveto(691.0831667,95.65423925)(690.83816695,96.01423889)(690.63815674,96.47424123)
\curveto(690.5881672,96.6042383)(690.55316723,96.73423817)(690.53315674,96.86424123)
\curveto(690.51316727,97.0042379)(690.4881673,97.14423776)(690.45815674,97.28424123)
\curveto(690.44816734,97.35423755)(690.44316734,97.41923748)(690.44315674,97.47924123)
\curveto(690.44316734,97.53923736)(690.43816735,97.6042373)(690.42815674,97.67424123)
\curveto(690.40816738,98.5042364)(690.55816723,99.17423573)(690.87815674,99.68424123)
\curveto(691.1881666,100.19423471)(691.62816616,100.57423433)(692.19815674,100.82424123)
\curveto(692.31816547,100.87423403)(692.44316534,100.91923398)(692.57315674,100.95924123)
\curveto(692.70316508,100.9992339)(692.83816495,101.04423386)(692.97815674,101.09424123)
\curveto(693.05816473,101.11423379)(693.14316464,101.12923377)(693.23315674,101.13924123)
\lineto(693.47315674,101.19924123)
\curveto(693.5831642,101.22923367)(693.69316409,101.24423366)(693.80315674,101.24424123)
\curveto(693.91316387,101.25423365)(694.02316376,101.26923363)(694.13315674,101.28924123)
\curveto(694.1831636,101.30923359)(694.22816356,101.31423359)(694.26815674,101.30424123)
\curveto(694.30816348,101.3042336)(694.34816344,101.30923359)(694.38815674,101.31924123)
\curveto(694.43816335,101.32923357)(694.49316329,101.32923357)(694.55315674,101.31924123)
\curveto(694.60316318,101.31923358)(694.65316313,101.32423358)(694.70315674,101.33424123)
\lineto(694.83815674,101.33424123)
\curveto(694.89816289,101.35423355)(694.96816282,101.35423355)(695.04815674,101.33424123)
\curveto(695.11816267,101.32423358)(695.1831626,101.32923357)(695.24315674,101.34924123)
\curveto(695.27316251,101.35923354)(695.31316247,101.36423354)(695.36315674,101.36424123)
\lineto(695.48315674,101.36424123)
\lineto(695.94815674,101.36424123)
\moveto(698.27315674,99.81924123)
\curveto(697.95315983,99.91923498)(697.5881602,99.97923492)(697.17815674,99.99924123)
\curveto(696.76816102,100.01923488)(696.35816143,100.02923487)(695.94815674,100.02924123)
\curveto(695.51816227,100.02923487)(695.09816269,100.01923488)(694.68815674,99.99924123)
\curveto(694.27816351,99.97923492)(693.89316389,99.93423497)(693.53315674,99.86424123)
\curveto(693.17316461,99.79423511)(692.85316493,99.68423522)(692.57315674,99.53424123)
\curveto(692.2831655,99.39423551)(692.04816574,99.1992357)(691.86815674,98.94924123)
\curveto(691.75816603,98.78923611)(691.67816611,98.60923629)(691.62815674,98.40924123)
\curveto(691.56816622,98.20923669)(691.53816625,97.96423694)(691.53815674,97.67424123)
\curveto(691.55816623,97.65423725)(691.56816622,97.61923728)(691.56815674,97.56924123)
\curveto(691.55816623,97.51923738)(691.55816623,97.47923742)(691.56815674,97.44924123)
\curveto(691.5881662,97.36923753)(691.60816618,97.29423761)(691.62815674,97.22424123)
\curveto(691.63816615,97.16423774)(691.65816613,97.0992378)(691.68815674,97.02924123)
\curveto(691.80816598,96.75923814)(691.97816581,96.53923836)(692.19815674,96.36924123)
\curveto(692.40816538,96.20923869)(692.65316513,96.07423883)(692.93315674,95.96424123)
\curveto(693.04316474,95.91423899)(693.16316462,95.87423903)(693.29315674,95.84424123)
\curveto(693.41316437,95.82423908)(693.53816425,95.7992391)(693.66815674,95.76924123)
\curveto(693.71816407,95.74923915)(693.77316401,95.73923916)(693.83315674,95.73924123)
\curveto(693.8831639,95.73923916)(693.93316385,95.73423917)(693.98315674,95.72424123)
\curveto(694.07316371,95.71423919)(694.16816362,95.7042392)(694.26815674,95.69424123)
\curveto(694.35816343,95.68423922)(694.45316333,95.67423923)(694.55315674,95.66424123)
\curveto(694.63316315,95.66423924)(694.71816307,95.65923924)(694.80815674,95.64924123)
\lineto(695.04815674,95.64924123)
\lineto(695.22815674,95.64924123)
\curveto(695.25816253,95.63923926)(695.29316249,95.63423927)(695.33315674,95.63424123)
\lineto(695.46815674,95.63424123)
\lineto(695.91815674,95.63424123)
\curveto(695.99816179,95.63423927)(696.0831617,95.62923927)(696.17315674,95.61924123)
\curveto(696.25316153,95.61923928)(696.32816146,95.62923927)(696.39815674,95.64924123)
\lineto(696.66815674,95.64924123)
\curveto(696.6881611,95.64923925)(696.71816107,95.64423926)(696.75815674,95.63424123)
\curveto(696.788161,95.63423927)(696.81316097,95.63923926)(696.83315674,95.64924123)
\curveto(696.93316085,95.65923924)(697.03316075,95.66423924)(697.13315674,95.66424123)
\curveto(697.22316056,95.67423923)(697.32316046,95.68423922)(697.43315674,95.69424123)
\curveto(697.55316023,95.72423918)(697.67816011,95.73923916)(697.80815674,95.73924123)
\curveto(697.92815986,95.74923915)(698.04315974,95.77423913)(698.15315674,95.81424123)
\curveto(698.45315933,95.89423901)(698.71815907,95.97923892)(698.94815674,96.06924123)
\curveto(699.17815861,96.16923873)(699.39315839,96.31423859)(699.59315674,96.50424123)
\curveto(699.79315799,96.71423819)(699.94315784,96.97923792)(700.04315674,97.29924123)
\curveto(700.06315772,97.33923756)(700.07315771,97.37423753)(700.07315674,97.40424123)
\curveto(700.06315772,97.44423746)(700.06815772,97.48923741)(700.08815674,97.53924123)
\curveto(700.09815769,97.57923732)(700.10815768,97.64923725)(700.11815674,97.74924123)
\curveto(700.12815766,97.85923704)(700.12315766,97.94423696)(700.10315674,98.00424123)
\curveto(700.0831577,98.07423683)(700.07315771,98.14423676)(700.07315674,98.21424123)
\curveto(700.06315772,98.28423662)(700.04815774,98.34923655)(700.02815674,98.40924123)
\curveto(699.96815782,98.60923629)(699.8831579,98.78923611)(699.77315674,98.94924123)
\curveto(699.75315803,98.97923592)(699.73315805,99.0042359)(699.71315674,99.02424123)
\lineto(699.65315674,99.08424123)
\curveto(699.63315815,99.12423578)(699.59315819,99.17423573)(699.53315674,99.23424123)
\curveto(699.39315839,99.33423557)(699.26315852,99.41923548)(699.14315674,99.48924123)
\curveto(699.02315876,99.55923534)(698.87815891,99.62923527)(698.70815674,99.69924123)
\curveto(698.63815915,99.72923517)(698.56815922,99.74923515)(698.49815674,99.75924123)
\curveto(698.42815936,99.77923512)(698.35315943,99.7992351)(698.27315674,99.81924123)
}
}
{
\newrgbcolor{curcolor}{0 0 0}
\pscustom[linestyle=none,fillstyle=solid,fillcolor=curcolor]
{
\newpath
\moveto(690.42815674,106.77385061)
\curveto(690.42816736,106.87384575)(690.43816735,106.96884566)(690.45815674,107.05885061)
\curveto(690.46816732,107.14884548)(690.49816729,107.21384541)(690.54815674,107.25385061)
\curveto(690.62816716,107.31384531)(690.73316705,107.34384528)(690.86315674,107.34385061)
\lineto(691.25315674,107.34385061)
\lineto(692.75315674,107.34385061)
\lineto(699.14315674,107.34385061)
\lineto(700.31315674,107.34385061)
\lineto(700.62815674,107.34385061)
\curveto(700.72815706,107.35384527)(700.80815698,107.33884529)(700.86815674,107.29885061)
\curveto(700.94815684,107.24884538)(700.99815679,107.17384545)(701.01815674,107.07385061)
\curveto(701.02815676,106.98384564)(701.03315675,106.87384575)(701.03315674,106.74385061)
\lineto(701.03315674,106.51885061)
\curveto(701.01315677,106.43884619)(700.99815679,106.36884626)(700.98815674,106.30885061)
\curveto(700.96815682,106.24884638)(700.92815686,106.19884643)(700.86815674,106.15885061)
\curveto(700.80815698,106.11884651)(700.73315705,106.09884653)(700.64315674,106.09885061)
\lineto(700.34315674,106.09885061)
\lineto(699.24815674,106.09885061)
\lineto(693.90815674,106.09885061)
\curveto(693.81816397,106.07884655)(693.74316404,106.06384656)(693.68315674,106.05385061)
\curveto(693.61316417,106.05384657)(693.55316423,106.0238466)(693.50315674,105.96385061)
\curveto(693.45316433,105.89384673)(693.42816436,105.80384682)(693.42815674,105.69385061)
\curveto(693.41816437,105.59384703)(693.41316437,105.48384714)(693.41315674,105.36385061)
\lineto(693.41315674,104.22385061)
\lineto(693.41315674,103.72885061)
\curveto(693.40316438,103.56884906)(693.34316444,103.45884917)(693.23315674,103.39885061)
\curveto(693.20316458,103.37884925)(693.17316461,103.36884926)(693.14315674,103.36885061)
\curveto(693.10316468,103.36884926)(693.05816473,103.36384926)(693.00815674,103.35385061)
\curveto(692.8881649,103.33384929)(692.77816501,103.33884929)(692.67815674,103.36885061)
\curveto(692.57816521,103.40884922)(692.50816528,103.46384916)(692.46815674,103.53385061)
\curveto(692.41816537,103.61384901)(692.39316539,103.73384889)(692.39315674,103.89385061)
\curveto(692.39316539,104.05384857)(692.37816541,104.18884844)(692.34815674,104.29885061)
\curveto(692.33816545,104.34884828)(692.33316545,104.40384822)(692.33315674,104.46385061)
\curveto(692.32316546,104.5238481)(692.30816548,104.58384804)(692.28815674,104.64385061)
\curveto(692.23816555,104.79384783)(692.1881656,104.93884769)(692.13815674,105.07885061)
\curveto(692.07816571,105.21884741)(692.00816578,105.35384727)(691.92815674,105.48385061)
\curveto(691.83816595,105.623847)(691.73316605,105.74384688)(691.61315674,105.84385061)
\curveto(691.49316629,105.94384668)(691.36316642,106.03884659)(691.22315674,106.12885061)
\curveto(691.12316666,106.18884644)(691.01316677,106.23384639)(690.89315674,106.26385061)
\curveto(690.77316701,106.30384632)(690.66816712,106.35384627)(690.57815674,106.41385061)
\curveto(690.51816727,106.46384616)(690.47816731,106.53384609)(690.45815674,106.62385061)
\curveto(690.44816734,106.64384598)(690.44316734,106.66884596)(690.44315674,106.69885061)
\curveto(690.44316734,106.7288459)(690.43816735,106.75384587)(690.42815674,106.77385061)
}
}
{
\newrgbcolor{curcolor}{0 0 0}
\pscustom[linestyle=none,fillstyle=solid,fillcolor=curcolor]
{
\newpath
\moveto(690.42815674,115.12345998)
\curveto(690.42816736,115.22345513)(690.43816735,115.31845503)(690.45815674,115.40845998)
\curveto(690.46816732,115.49845485)(690.49816729,115.56345479)(690.54815674,115.60345998)
\curveto(690.62816716,115.66345469)(690.73316705,115.69345466)(690.86315674,115.69345998)
\lineto(691.25315674,115.69345998)
\lineto(692.75315674,115.69345998)
\lineto(699.14315674,115.69345998)
\lineto(700.31315674,115.69345998)
\lineto(700.62815674,115.69345998)
\curveto(700.72815706,115.70345465)(700.80815698,115.68845466)(700.86815674,115.64845998)
\curveto(700.94815684,115.59845475)(700.99815679,115.52345483)(701.01815674,115.42345998)
\curveto(701.02815676,115.33345502)(701.03315675,115.22345513)(701.03315674,115.09345998)
\lineto(701.03315674,114.86845998)
\curveto(701.01315677,114.78845556)(700.99815679,114.71845563)(700.98815674,114.65845998)
\curveto(700.96815682,114.59845575)(700.92815686,114.5484558)(700.86815674,114.50845998)
\curveto(700.80815698,114.46845588)(700.73315705,114.4484559)(700.64315674,114.44845998)
\lineto(700.34315674,114.44845998)
\lineto(699.24815674,114.44845998)
\lineto(693.90815674,114.44845998)
\curveto(693.81816397,114.42845592)(693.74316404,114.41345594)(693.68315674,114.40345998)
\curveto(693.61316417,114.40345595)(693.55316423,114.37345598)(693.50315674,114.31345998)
\curveto(693.45316433,114.24345611)(693.42816436,114.1534562)(693.42815674,114.04345998)
\curveto(693.41816437,113.94345641)(693.41316437,113.83345652)(693.41315674,113.71345998)
\lineto(693.41315674,112.57345998)
\lineto(693.41315674,112.07845998)
\curveto(693.40316438,111.91845843)(693.34316444,111.80845854)(693.23315674,111.74845998)
\curveto(693.20316458,111.72845862)(693.17316461,111.71845863)(693.14315674,111.71845998)
\curveto(693.10316468,111.71845863)(693.05816473,111.71345864)(693.00815674,111.70345998)
\curveto(692.8881649,111.68345867)(692.77816501,111.68845866)(692.67815674,111.71845998)
\curveto(692.57816521,111.75845859)(692.50816528,111.81345854)(692.46815674,111.88345998)
\curveto(692.41816537,111.96345839)(692.39316539,112.08345827)(692.39315674,112.24345998)
\curveto(692.39316539,112.40345795)(692.37816541,112.53845781)(692.34815674,112.64845998)
\curveto(692.33816545,112.69845765)(692.33316545,112.7534576)(692.33315674,112.81345998)
\curveto(692.32316546,112.87345748)(692.30816548,112.93345742)(692.28815674,112.99345998)
\curveto(692.23816555,113.14345721)(692.1881656,113.28845706)(692.13815674,113.42845998)
\curveto(692.07816571,113.56845678)(692.00816578,113.70345665)(691.92815674,113.83345998)
\curveto(691.83816595,113.97345638)(691.73316605,114.09345626)(691.61315674,114.19345998)
\curveto(691.49316629,114.29345606)(691.36316642,114.38845596)(691.22315674,114.47845998)
\curveto(691.12316666,114.53845581)(691.01316677,114.58345577)(690.89315674,114.61345998)
\curveto(690.77316701,114.6534557)(690.66816712,114.70345565)(690.57815674,114.76345998)
\curveto(690.51816727,114.81345554)(690.47816731,114.88345547)(690.45815674,114.97345998)
\curveto(690.44816734,114.99345536)(690.44316734,115.01845533)(690.44315674,115.04845998)
\curveto(690.44316734,115.07845527)(690.43816735,115.10345525)(690.42815674,115.12345998)
}
}
{
\newrgbcolor{curcolor}{0 0 0}
\pscustom[linestyle=none,fillstyle=solid,fillcolor=curcolor]
{
\newpath
\moveto(711.26447266,37.28705373)
\curveto(711.26448335,37.35704805)(711.26448335,37.43704797)(711.26447266,37.52705373)
\curveto(711.25448336,37.61704779)(711.25448336,37.70204771)(711.26447266,37.78205373)
\curveto(711.26448335,37.87204754)(711.27448334,37.95204746)(711.29447266,38.02205373)
\curveto(711.3144833,38.10204731)(711.34448327,38.15704725)(711.38447266,38.18705373)
\curveto(711.43448318,38.21704719)(711.50948311,38.23704717)(711.60947266,38.24705373)
\curveto(711.69948292,38.26704714)(711.80448281,38.27704713)(711.92447266,38.27705373)
\curveto(712.03448258,38.28704712)(712.14948247,38.28704712)(712.26947266,38.27705373)
\lineto(712.56947266,38.27705373)
\lineto(715.58447266,38.27705373)
\lineto(718.47947266,38.27705373)
\curveto(718.80947581,38.27704713)(719.13447548,38.27204714)(719.45447266,38.26205373)
\curveto(719.76447485,38.26204715)(720.04447457,38.22204719)(720.29447266,38.14205373)
\curveto(720.64447397,38.02204739)(720.93947368,37.86704754)(721.17947266,37.67705373)
\curveto(721.40947321,37.48704792)(721.60947301,37.24704816)(721.77947266,36.95705373)
\curveto(721.82947279,36.89704851)(721.86447275,36.83204858)(721.88447266,36.76205373)
\curveto(721.90447271,36.70204871)(721.92947269,36.63204878)(721.95947266,36.55205373)
\curveto(722.00947261,36.43204898)(722.04447257,36.30204911)(722.06447266,36.16205373)
\curveto(722.09447252,36.03204938)(722.12447249,35.89704951)(722.15447266,35.75705373)
\curveto(722.17447244,35.7070497)(722.17947244,35.65704975)(722.16947266,35.60705373)
\curveto(722.15947246,35.55704985)(722.15947246,35.50204991)(722.16947266,35.44205373)
\curveto(722.17947244,35.42204999)(722.17947244,35.39705001)(722.16947266,35.36705373)
\curveto(722.16947245,35.33705007)(722.17447244,35.3120501)(722.18447266,35.29205373)
\curveto(722.19447242,35.25205016)(722.19947242,35.19705021)(722.19947266,35.12705373)
\curveto(722.19947242,35.05705035)(722.19447242,35.00205041)(722.18447266,34.96205373)
\curveto(722.17447244,34.9120505)(722.17447244,34.85705055)(722.18447266,34.79705373)
\curveto(722.19447242,34.73705067)(722.18947243,34.68205073)(722.16947266,34.63205373)
\curveto(722.13947248,34.50205091)(722.1194725,34.37705103)(722.10947266,34.25705373)
\curveto(722.09947252,34.13705127)(722.07447254,34.02205139)(722.03447266,33.91205373)
\curveto(721.9144727,33.54205187)(721.74447287,33.22205219)(721.52447266,32.95205373)
\curveto(721.30447331,32.68205273)(721.02447359,32.47205294)(720.68447266,32.32205373)
\curveto(720.56447405,32.27205314)(720.43947418,32.22705318)(720.30947266,32.18705373)
\curveto(720.17947444,32.15705325)(720.04447457,32.12205329)(719.90447266,32.08205373)
\curveto(719.85447476,32.07205334)(719.8144748,32.06705334)(719.78447266,32.06705373)
\curveto(719.74447487,32.06705334)(719.69947492,32.06205335)(719.64947266,32.05205373)
\curveto(719.619475,32.04205337)(719.58447503,32.03705337)(719.54447266,32.03705373)
\curveto(719.49447512,32.03705337)(719.45447516,32.03205338)(719.42447266,32.02205373)
\lineto(719.25947266,32.02205373)
\curveto(719.17947544,32.00205341)(719.07947554,31.99705341)(718.95947266,32.00705373)
\curveto(718.82947579,32.01705339)(718.73947588,32.03205338)(718.68947266,32.05205373)
\curveto(718.59947602,32.07205334)(718.53447608,32.12705328)(718.49447266,32.21705373)
\curveto(718.47447614,32.24705316)(718.46947615,32.27705313)(718.47947266,32.30705373)
\curveto(718.47947614,32.33705307)(718.47447614,32.37705303)(718.46447266,32.42705373)
\curveto(718.45447616,32.46705294)(718.44947617,32.5070529)(718.44947266,32.54705373)
\lineto(718.44947266,32.69705373)
\curveto(718.44947617,32.81705259)(718.45447616,32.93705247)(718.46447266,33.05705373)
\curveto(718.46447615,33.18705222)(718.49947612,33.27705213)(718.56947266,33.32705373)
\curveto(718.62947599,33.36705204)(718.68947593,33.38705202)(718.74947266,33.38705373)
\curveto(718.80947581,33.38705202)(718.87947574,33.39705201)(718.95947266,33.41705373)
\curveto(718.98947563,33.42705198)(719.02447559,33.42705198)(719.06447266,33.41705373)
\curveto(719.09447552,33.41705199)(719.1194755,33.42205199)(719.13947266,33.43205373)
\lineto(719.34947266,33.43205373)
\curveto(719.39947522,33.45205196)(719.44947517,33.45705195)(719.49947266,33.44705373)
\curveto(719.53947508,33.44705196)(719.58447503,33.45705195)(719.63447266,33.47705373)
\curveto(719.76447485,33.5070519)(719.88947473,33.53705187)(720.00947266,33.56705373)
\curveto(720.1194745,33.59705181)(720.22447439,33.64205177)(720.32447266,33.70205373)
\curveto(720.614474,33.87205154)(720.8194738,34.14205127)(720.93947266,34.51205373)
\curveto(720.95947366,34.56205085)(720.97447364,34.6120508)(720.98447266,34.66205373)
\curveto(720.98447363,34.72205069)(720.99447362,34.77705063)(721.01447266,34.82705373)
\lineto(721.01447266,34.90205373)
\curveto(721.02447359,34.97205044)(721.03447358,35.06705034)(721.04447266,35.18705373)
\curveto(721.04447357,35.31705009)(721.03447358,35.41704999)(721.01447266,35.48705373)
\curveto(720.99447362,35.55704985)(720.97947364,35.62704978)(720.96947266,35.69705373)
\curveto(720.94947367,35.77704963)(720.92947369,35.84704956)(720.90947266,35.90705373)
\curveto(720.74947387,36.28704912)(720.47447414,36.56204885)(720.08447266,36.73205373)
\curveto(719.95447466,36.78204863)(719.79947482,36.81704859)(719.61947266,36.83705373)
\curveto(719.43947518,36.86704854)(719.25447536,36.88204853)(719.06447266,36.88205373)
\curveto(718.86447575,36.89204852)(718.66447595,36.89204852)(718.46447266,36.88205373)
\lineto(717.89447266,36.88205373)
\lineto(713.64947266,36.88205373)
\lineto(712.10447266,36.88205373)
\curveto(711.99448262,36.88204853)(711.87448274,36.87704853)(711.74447266,36.86705373)
\curveto(711.614483,36.85704855)(711.50948311,36.87704853)(711.42947266,36.92705373)
\curveto(711.35948326,36.98704842)(711.30948331,37.06704834)(711.27947266,37.16705373)
\curveto(711.27948334,37.18704822)(711.27948334,37.2070482)(711.27947266,37.22705373)
\curveto(711.27948334,37.24704816)(711.27448334,37.26704814)(711.26447266,37.28705373)
}
}
{
\newrgbcolor{curcolor}{0 0 0}
\pscustom[linestyle=none,fillstyle=solid,fillcolor=curcolor]
{
\newpath
\moveto(714.21947266,40.82072561)
\lineto(714.21947266,41.25572561)
\curveto(714.2194804,41.40572364)(714.25948036,41.51072354)(714.33947266,41.57072561)
\curveto(714.4194802,41.62072343)(714.5194801,41.6457234)(714.63947266,41.64572561)
\curveto(714.75947986,41.65572339)(714.87947974,41.66072339)(714.99947266,41.66072561)
\lineto(716.42447266,41.66072561)
\lineto(718.68947266,41.66072561)
\lineto(719.37947266,41.66072561)
\curveto(719.60947501,41.66072339)(719.80947481,41.68572336)(719.97947266,41.73572561)
\curveto(720.42947419,41.89572315)(720.74447387,42.19572285)(720.92447266,42.63572561)
\curveto(721.0144736,42.85572219)(721.04947357,43.12072193)(721.02947266,43.43072561)
\curveto(720.99947362,43.74072131)(720.94447367,43.99072106)(720.86447266,44.18072561)
\curveto(720.72447389,44.51072054)(720.54947407,44.77072028)(720.33947266,44.96072561)
\curveto(720.1194745,45.16071989)(719.83447478,45.31571973)(719.48447266,45.42572561)
\curveto(719.40447521,45.45571959)(719.32447529,45.47571957)(719.24447266,45.48572561)
\curveto(719.16447545,45.49571955)(719.07947554,45.51071954)(718.98947266,45.53072561)
\curveto(718.93947568,45.54071951)(718.89447572,45.54071951)(718.85447266,45.53072561)
\curveto(718.8144758,45.53071952)(718.76947585,45.54071951)(718.71947266,45.56072561)
\lineto(718.40447266,45.56072561)
\curveto(718.32447629,45.58071947)(718.23447638,45.58571946)(718.13447266,45.57572561)
\curveto(718.02447659,45.56571948)(717.92447669,45.56071949)(717.83447266,45.56072561)
\lineto(716.66447266,45.56072561)
\lineto(715.07447266,45.56072561)
\curveto(714.95447966,45.56071949)(714.82947979,45.55571949)(714.69947266,45.54572561)
\curveto(714.55948006,45.5457195)(714.44948017,45.57071948)(714.36947266,45.62072561)
\curveto(714.3194803,45.66071939)(714.28948033,45.70571934)(714.27947266,45.75572561)
\curveto(714.25948036,45.81571923)(714.23948038,45.88571916)(714.21947266,45.96572561)
\lineto(714.21947266,46.19072561)
\curveto(714.2194804,46.31071874)(714.22448039,46.41571863)(714.23447266,46.50572561)
\curveto(714.24448037,46.60571844)(714.28948033,46.68071837)(714.36947266,46.73072561)
\curveto(714.4194802,46.78071827)(714.49448012,46.80571824)(714.59447266,46.80572561)
\lineto(714.87947266,46.80572561)
\lineto(715.89947266,46.80572561)
\lineto(719.93447266,46.80572561)
\lineto(721.28447266,46.80572561)
\curveto(721.40447321,46.80571824)(721.5194731,46.80071825)(721.62947266,46.79072561)
\curveto(721.72947289,46.79071826)(721.80447281,46.75571829)(721.85447266,46.68572561)
\curveto(721.88447273,46.6457184)(721.90947271,46.58571846)(721.92947266,46.50572561)
\curveto(721.93947268,46.42571862)(721.94947267,46.33571871)(721.95947266,46.23572561)
\curveto(721.95947266,46.1457189)(721.95447266,46.05571899)(721.94447266,45.96572561)
\curveto(721.93447268,45.88571916)(721.9144727,45.82571922)(721.88447266,45.78572561)
\curveto(721.84447277,45.73571931)(721.77947284,45.69071936)(721.68947266,45.65072561)
\curveto(721.64947297,45.64071941)(721.59447302,45.63071942)(721.52447266,45.62072561)
\curveto(721.45447316,45.62071943)(721.38947323,45.61571943)(721.32947266,45.60572561)
\curveto(721.25947336,45.59571945)(721.20447341,45.57571947)(721.16447266,45.54572561)
\curveto(721.12447349,45.51571953)(721.10947351,45.47071958)(721.11947266,45.41072561)
\curveto(721.13947348,45.33071972)(721.19947342,45.2507198)(721.29947266,45.17072561)
\curveto(721.38947323,45.09071996)(721.45947316,45.01572003)(721.50947266,44.94572561)
\curveto(721.66947295,44.72572032)(721.80947281,44.47572057)(721.92947266,44.19572561)
\curveto(721.97947264,44.08572096)(722.00947261,43.97072108)(722.01947266,43.85072561)
\curveto(722.03947258,43.74072131)(722.06447255,43.62572142)(722.09447266,43.50572561)
\curveto(722.10447251,43.45572159)(722.10447251,43.40072165)(722.09447266,43.34072561)
\curveto(722.08447253,43.29072176)(722.08947253,43.24072181)(722.10947266,43.19072561)
\curveto(722.12947249,43.09072196)(722.12947249,43.00072205)(722.10947266,42.92072561)
\lineto(722.10947266,42.77072561)
\curveto(722.08947253,42.72072233)(722.07947254,42.66072239)(722.07947266,42.59072561)
\curveto(722.07947254,42.53072252)(722.07447254,42.47572257)(722.06447266,42.42572561)
\curveto(722.04447257,42.38572266)(722.03447258,42.3457227)(722.03447266,42.30572561)
\curveto(722.04447257,42.27572277)(722.03947258,42.23572281)(722.01947266,42.18572561)
\lineto(721.95947266,41.94572561)
\curveto(721.93947268,41.87572317)(721.90947271,41.80072325)(721.86947266,41.72072561)
\curveto(721.75947286,41.46072359)(721.614473,41.24072381)(721.43447266,41.06072561)
\curveto(721.24447337,40.89072416)(721.0194736,40.7507243)(720.75947266,40.64072561)
\curveto(720.66947395,40.60072445)(720.57947404,40.57072448)(720.48947266,40.55072561)
\lineto(720.18947266,40.49072561)
\curveto(720.12947449,40.47072458)(720.07447454,40.46072459)(720.02447266,40.46072561)
\curveto(719.96447465,40.47072458)(719.89947472,40.46572458)(719.82947266,40.44572561)
\curveto(719.80947481,40.43572461)(719.78447483,40.43072462)(719.75447266,40.43072561)
\curveto(719.7144749,40.43072462)(719.67947494,40.42572462)(719.64947266,40.41572561)
\lineto(719.49947266,40.41572561)
\curveto(719.45947516,40.40572464)(719.4144752,40.40072465)(719.36447266,40.40072561)
\curveto(719.30447531,40.41072464)(719.24947537,40.41572463)(719.19947266,40.41572561)
\lineto(718.59947266,40.41572561)
\lineto(715.83947266,40.41572561)
\lineto(714.87947266,40.41572561)
\lineto(714.60947266,40.41572561)
\curveto(714.5194801,40.41572463)(714.44448017,40.43572461)(714.38447266,40.47572561)
\curveto(714.3144803,40.51572453)(714.26448035,40.59072446)(714.23447266,40.70072561)
\curveto(714.22448039,40.72072433)(714.22448039,40.74072431)(714.23447266,40.76072561)
\curveto(714.23448038,40.78072427)(714.22948039,40.80072425)(714.21947266,40.82072561)
}
}
{
\newrgbcolor{curcolor}{0 0 0}
\pscustom[linestyle=none,fillstyle=solid,fillcolor=curcolor]
{
\newpath
\moveto(714.06947266,52.39533498)
\curveto(714.04948057,53.02532975)(714.13448048,53.53032924)(714.32447266,53.91033498)
\curveto(714.5144801,54.29032848)(714.79947982,54.59532818)(715.17947266,54.82533498)
\curveto(715.27947934,54.88532789)(715.38947923,54.93032784)(715.50947266,54.96033498)
\curveto(715.619479,55.00032777)(715.73447888,55.03532774)(715.85447266,55.06533498)
\curveto(716.04447857,55.11532766)(716.24947837,55.14532763)(716.46947266,55.15533498)
\curveto(716.68947793,55.16532761)(716.9144777,55.1703276)(717.14447266,55.17033498)
\lineto(718.74947266,55.17033498)
\lineto(721.08947266,55.17033498)
\curveto(721.25947336,55.1703276)(721.42947319,55.16532761)(721.59947266,55.15533498)
\curveto(721.76947285,55.15532762)(721.87947274,55.09032768)(721.92947266,54.96033498)
\curveto(721.94947267,54.91032786)(721.95947266,54.85532792)(721.95947266,54.79533498)
\curveto(721.96947265,54.74532803)(721.97447264,54.69032808)(721.97447266,54.63033498)
\curveto(721.97447264,54.50032827)(721.96947265,54.3753284)(721.95947266,54.25533498)
\curveto(721.95947266,54.13532864)(721.9194727,54.05032872)(721.83947266,54.00033498)
\curveto(721.76947285,53.95032882)(721.67947294,53.92532885)(721.56947266,53.92533498)
\lineto(721.23947266,53.92533498)
\lineto(719.94947266,53.92533498)
\lineto(717.50447266,53.92533498)
\curveto(717.23447738,53.92532885)(716.96947765,53.92032885)(716.70947266,53.91033498)
\curveto(716.43947818,53.90032887)(716.20947841,53.85532892)(716.01947266,53.77533498)
\curveto(715.8194788,53.69532908)(715.65947896,53.5753292)(715.53947266,53.41533498)
\curveto(715.40947921,53.25532952)(715.30947931,53.0703297)(715.23947266,52.86033498)
\curveto(715.2194794,52.80032997)(715.20947941,52.73533004)(715.20947266,52.66533498)
\curveto(715.19947942,52.60533017)(715.18447943,52.54533023)(715.16447266,52.48533498)
\curveto(715.15447946,52.43533034)(715.15447946,52.35533042)(715.16447266,52.24533498)
\curveto(715.16447945,52.14533063)(715.16947945,52.0753307)(715.17947266,52.03533498)
\curveto(715.19947942,51.99533078)(715.20947941,51.96033081)(715.20947266,51.93033498)
\curveto(715.19947942,51.90033087)(715.19947942,51.86533091)(715.20947266,51.82533498)
\curveto(715.23947938,51.69533108)(715.27447934,51.5703312)(715.31447266,51.45033498)
\curveto(715.34447927,51.34033143)(715.38947923,51.23533154)(715.44947266,51.13533498)
\curveto(715.46947915,51.09533168)(715.48947913,51.06033171)(715.50947266,51.03033498)
\curveto(715.52947909,51.00033177)(715.54947907,50.96533181)(715.56947266,50.92533498)
\curveto(715.8194788,50.5753322)(716.19447842,50.32033245)(716.69447266,50.16033498)
\curveto(716.77447784,50.13033264)(716.85947776,50.11033266)(716.94947266,50.10033498)
\curveto(717.02947759,50.09033268)(717.10947751,50.0753327)(717.18947266,50.05533498)
\curveto(717.23947738,50.03533274)(717.28947733,50.03033274)(717.33947266,50.04033498)
\curveto(717.37947724,50.05033272)(717.4194772,50.04533273)(717.45947266,50.02533498)
\lineto(717.77447266,50.02533498)
\curveto(717.80447681,50.01533276)(717.83947678,50.01033276)(717.87947266,50.01033498)
\curveto(717.9194767,50.02033275)(717.96447665,50.02533275)(718.01447266,50.02533498)
\lineto(718.46447266,50.02533498)
\lineto(719.90447266,50.02533498)
\lineto(721.22447266,50.02533498)
\lineto(721.56947266,50.02533498)
\curveto(721.67947294,50.02533275)(721.76947285,50.00033277)(721.83947266,49.95033498)
\curveto(721.9194727,49.90033287)(721.95947266,49.81033296)(721.95947266,49.68033498)
\curveto(721.96947265,49.56033321)(721.97447264,49.43533334)(721.97447266,49.30533498)
\curveto(721.97447264,49.22533355)(721.96947265,49.15033362)(721.95947266,49.08033498)
\curveto(721.94947267,49.01033376)(721.92447269,48.95033382)(721.88447266,48.90033498)
\curveto(721.83447278,48.82033395)(721.73947288,48.78033399)(721.59947266,48.78033498)
\lineto(721.19447266,48.78033498)
\lineto(719.42447266,48.78033498)
\lineto(715.79447266,48.78033498)
\lineto(714.87947266,48.78033498)
\lineto(714.60947266,48.78033498)
\curveto(714.5194801,48.78033399)(714.44948017,48.80033397)(714.39947266,48.84033498)
\curveto(714.33948028,48.8703339)(714.29948032,48.92033385)(714.27947266,48.99033498)
\curveto(714.26948035,49.03033374)(714.25948036,49.08533369)(714.24947266,49.15533498)
\curveto(714.23948038,49.23533354)(714.23448038,49.31533346)(714.23447266,49.39533498)
\curveto(714.23448038,49.4753333)(714.23948038,49.55033322)(714.24947266,49.62033498)
\curveto(714.25948036,49.70033307)(714.27448034,49.75533302)(714.29447266,49.78533498)
\curveto(714.36448025,49.89533288)(714.45448016,49.94533283)(714.56447266,49.93533498)
\curveto(714.66447995,49.92533285)(714.77947984,49.94033283)(714.90947266,49.98033498)
\curveto(714.96947965,50.00033277)(715.0194796,50.04033273)(715.05947266,50.10033498)
\curveto(715.06947955,50.22033255)(715.02447959,50.31533246)(714.92447266,50.38533498)
\curveto(714.82447979,50.46533231)(714.74447987,50.54533223)(714.68447266,50.62533498)
\curveto(714.58448003,50.76533201)(714.49448012,50.90533187)(714.41447266,51.04533498)
\curveto(714.32448029,51.19533158)(714.24948037,51.36533141)(714.18947266,51.55533498)
\curveto(714.15948046,51.63533114)(714.13948048,51.72033105)(714.12947266,51.81033498)
\curveto(714.1194805,51.91033086)(714.10448051,52.00533077)(714.08447266,52.09533498)
\curveto(714.07448054,52.14533063)(714.06948055,52.19533058)(714.06947266,52.24533498)
\lineto(714.06947266,52.39533498)
}
}
{
\newrgbcolor{curcolor}{0 0 0}
\pscustom[linestyle=none,fillstyle=solid,fillcolor=curcolor]
{
}
}
{
\newrgbcolor{curcolor}{0 0 0}
\pscustom[linestyle=none,fillstyle=solid,fillcolor=curcolor]
{
\newpath
\moveto(711.33947266,65.05510061)
\curveto(711.33948328,65.15509575)(711.34948327,65.25009566)(711.36947266,65.34010061)
\curveto(711.37948324,65.43009548)(711.40948321,65.49509541)(711.45947266,65.53510061)
\curveto(711.53948308,65.59509531)(711.64448297,65.62509528)(711.77447266,65.62510061)
\lineto(712.16447266,65.62510061)
\lineto(713.66447266,65.62510061)
\lineto(720.05447266,65.62510061)
\lineto(721.22447266,65.62510061)
\lineto(721.53947266,65.62510061)
\curveto(721.63947298,65.63509527)(721.7194729,65.62009529)(721.77947266,65.58010061)
\curveto(721.85947276,65.53009538)(721.90947271,65.45509545)(721.92947266,65.35510061)
\curveto(721.93947268,65.26509564)(721.94447267,65.15509575)(721.94447266,65.02510061)
\lineto(721.94447266,64.80010061)
\curveto(721.92447269,64.72009619)(721.90947271,64.65009626)(721.89947266,64.59010061)
\curveto(721.87947274,64.53009638)(721.83947278,64.48009643)(721.77947266,64.44010061)
\curveto(721.7194729,64.40009651)(721.64447297,64.38009653)(721.55447266,64.38010061)
\lineto(721.25447266,64.38010061)
\lineto(720.15947266,64.38010061)
\lineto(714.81947266,64.38010061)
\curveto(714.72947989,64.36009655)(714.65447996,64.34509656)(714.59447266,64.33510061)
\curveto(714.52448009,64.33509657)(714.46448015,64.3050966)(714.41447266,64.24510061)
\curveto(714.36448025,64.17509673)(714.33948028,64.08509682)(714.33947266,63.97510061)
\curveto(714.32948029,63.87509703)(714.32448029,63.76509714)(714.32447266,63.64510061)
\lineto(714.32447266,62.50510061)
\lineto(714.32447266,62.01010061)
\curveto(714.3144803,61.85009906)(714.25448036,61.74009917)(714.14447266,61.68010061)
\curveto(714.1144805,61.66009925)(714.08448053,61.65009926)(714.05447266,61.65010061)
\curveto(714.0144806,61.65009926)(713.96948065,61.64509926)(713.91947266,61.63510061)
\curveto(713.79948082,61.61509929)(713.68948093,61.62009929)(713.58947266,61.65010061)
\curveto(713.48948113,61.69009922)(713.4194812,61.74509916)(713.37947266,61.81510061)
\curveto(713.32948129,61.89509901)(713.30448131,62.01509889)(713.30447266,62.17510061)
\curveto(713.30448131,62.33509857)(713.28948133,62.47009844)(713.25947266,62.58010061)
\curveto(713.24948137,62.63009828)(713.24448137,62.68509822)(713.24447266,62.74510061)
\curveto(713.23448138,62.8050981)(713.2194814,62.86509804)(713.19947266,62.92510061)
\curveto(713.14948147,63.07509783)(713.09948152,63.22009769)(713.04947266,63.36010061)
\curveto(712.98948163,63.50009741)(712.9194817,63.63509727)(712.83947266,63.76510061)
\curveto(712.74948187,63.905097)(712.64448197,64.02509688)(712.52447266,64.12510061)
\curveto(712.40448221,64.22509668)(712.27448234,64.32009659)(712.13447266,64.41010061)
\curveto(712.03448258,64.47009644)(711.92448269,64.51509639)(711.80447266,64.54510061)
\curveto(711.68448293,64.58509632)(711.57948304,64.63509627)(711.48947266,64.69510061)
\curveto(711.42948319,64.74509616)(711.38948323,64.81509609)(711.36947266,64.90510061)
\curveto(711.35948326,64.92509598)(711.35448326,64.95009596)(711.35447266,64.98010061)
\curveto(711.35448326,65.0100959)(711.34948327,65.03509587)(711.33947266,65.05510061)
}
}
{
\newrgbcolor{curcolor}{0 0 0}
\pscustom[linestyle=none,fillstyle=solid,fillcolor=curcolor]
{
\newpath
\moveto(718.44947266,76.22470998)
\curveto(718.49947612,76.29470234)(718.56947605,76.3347023)(718.65947266,76.34470998)
\curveto(718.74947587,76.36470227)(718.85447576,76.37470226)(718.97447266,76.37470998)
\curveto(719.02447559,76.37470226)(719.07447554,76.36970226)(719.12447266,76.35970998)
\curveto(719.17447544,76.35970227)(719.2194754,76.34970228)(719.25947266,76.32970998)
\curveto(719.34947527,76.29970233)(719.40947521,76.23970239)(719.43947266,76.14970998)
\curveto(719.45947516,76.06970256)(719.46947515,75.97470266)(719.46947266,75.86470998)
\lineto(719.46947266,75.54970998)
\curveto(719.45947516,75.43970319)(719.46947515,75.3347033)(719.49947266,75.23470998)
\curveto(719.52947509,75.09470354)(719.60947501,75.00470363)(719.73947266,74.96470998)
\curveto(719.80947481,74.94470369)(719.89447472,74.9347037)(719.99447266,74.93470998)
\lineto(720.26447266,74.93470998)
\lineto(721.20947266,74.93470998)
\lineto(721.53947266,74.93470998)
\curveto(721.64947297,74.9347037)(721.73447288,74.91470372)(721.79447266,74.87470998)
\curveto(721.85447276,74.8347038)(721.89447272,74.78470385)(721.91447266,74.72470998)
\curveto(721.92447269,74.67470396)(721.93947268,74.60970402)(721.95947266,74.52970998)
\lineto(721.95947266,74.33470998)
\curveto(721.95947266,74.21470442)(721.95447266,74.10970452)(721.94447266,74.01970998)
\curveto(721.92447269,73.9297047)(721.87447274,73.85970477)(721.79447266,73.80970998)
\curveto(721.74447287,73.77970485)(721.67447294,73.76470487)(721.58447266,73.76470998)
\lineto(721.28447266,73.76470998)
\lineto(720.24947266,73.76470998)
\curveto(720.08947453,73.76470487)(719.94447467,73.75470488)(719.81447266,73.73470998)
\curveto(719.67447494,73.72470491)(719.57947504,73.66970496)(719.52947266,73.56970998)
\curveto(719.50947511,73.51970511)(719.49447512,73.44970518)(719.48447266,73.35970998)
\curveto(719.47447514,73.27970535)(719.46947515,73.18970544)(719.46947266,73.08970998)
\lineto(719.46947266,72.80470998)
\lineto(719.46947266,72.56470998)
\lineto(719.46947266,70.29970998)
\curveto(719.46947515,70.20970842)(719.47447514,70.10470853)(719.48447266,69.98470998)
\lineto(719.48447266,69.65470998)
\curveto(719.48447513,69.54470909)(719.47447514,69.44470919)(719.45447266,69.35470998)
\curveto(719.43447518,69.26470937)(719.39947522,69.20470943)(719.34947266,69.17470998)
\curveto(719.27947534,69.12470951)(719.18447543,69.09970953)(719.06447266,69.09970998)
\lineto(718.71947266,69.09970998)
\lineto(718.44947266,69.09970998)
\curveto(718.27947634,69.13970949)(718.13947648,69.19470944)(718.02947266,69.26470998)
\curveto(717.9194767,69.3347093)(717.80447681,69.41470922)(717.68447266,69.50470998)
\lineto(717.14447266,69.86470998)
\curveto(716.5144781,70.30470833)(715.89447872,70.73970789)(715.28447266,71.16970998)
\lineto(713.42447266,72.48970998)
\curveto(713.19448142,72.64970598)(712.97448164,72.80470583)(712.76447266,72.95470998)
\curveto(712.54448207,73.10470553)(712.3194823,73.25970537)(712.08947266,73.41970998)
\curveto(712.0194826,73.46970516)(711.95448266,73.51970511)(711.89447266,73.56970998)
\curveto(711.82448279,73.61970501)(711.74948287,73.66970496)(711.66947266,73.71970998)
\lineto(711.57947266,73.77970998)
\curveto(711.53948308,73.80970482)(711.50948311,73.83970479)(711.48947266,73.86970998)
\curveto(711.45948316,73.90970472)(711.43948318,73.94970468)(711.42947266,73.98970998)
\curveto(711.40948321,74.0297046)(711.38948323,74.07470456)(711.36947266,74.12470998)
\curveto(711.36948325,74.14470449)(711.37448324,74.16470447)(711.38447266,74.18470998)
\curveto(711.38448323,74.21470442)(711.37448324,74.23970439)(711.35447266,74.25970998)
\curveto(711.35448326,74.38970424)(711.35948326,74.50970412)(711.36947266,74.61970998)
\curveto(711.37948324,74.7297039)(711.42448319,74.80970382)(711.50447266,74.85970998)
\curveto(711.55448306,74.89970373)(711.62448299,74.91970371)(711.71447266,74.91970998)
\curveto(711.80448281,74.9297037)(711.89948272,74.9347037)(711.99947266,74.93470998)
\lineto(717.45947266,74.93470998)
\curveto(717.52947709,74.9347037)(717.60447701,74.9297037)(717.68447266,74.91970998)
\curveto(717.75447686,74.91970371)(717.82447679,74.92470371)(717.89447266,74.93470998)
\lineto(717.99947266,74.93470998)
\curveto(718.04947657,74.95470368)(718.10447651,74.96970366)(718.16447266,74.97970998)
\curveto(718.2144764,74.98970364)(718.25447636,75.01470362)(718.28447266,75.05470998)
\curveto(718.33447628,75.12470351)(718.36447625,75.20970342)(718.37447266,75.30970998)
\lineto(718.37447266,75.63970998)
\curveto(718.37447624,75.74970288)(718.37947624,75.85470278)(718.38947266,75.95470998)
\curveto(718.38947623,76.06470257)(718.40947621,76.15470248)(718.44947266,76.22470998)
\moveto(718.25447266,73.65970998)
\curveto(718.14447647,73.73970489)(717.97447664,73.77470486)(717.74447266,73.76470998)
\lineto(717.12947266,73.76470998)
\lineto(714.65447266,73.76470998)
\lineto(714.33947266,73.76470998)
\curveto(714.2194804,73.77470486)(714.1194805,73.76970486)(714.03947266,73.74970998)
\lineto(713.88947266,73.74970998)
\curveto(713.79948082,73.74970488)(713.7144809,73.7347049)(713.63447266,73.70470998)
\curveto(713.614481,73.69470494)(713.60448101,73.68470495)(713.60447266,73.67470998)
\lineto(713.55947266,73.62970998)
\curveto(713.54948107,73.60970502)(713.54448107,73.57970505)(713.54447266,73.53970998)
\curveto(713.56448105,73.51970511)(713.57948104,73.49970513)(713.58947266,73.47970998)
\curveto(713.58948103,73.46970516)(713.59448102,73.45470518)(713.60447266,73.43470998)
\curveto(713.65448096,73.37470526)(713.72448089,73.31470532)(713.81447266,73.25470998)
\curveto(713.90448071,73.19470544)(713.98448063,73.13970549)(714.05447266,73.08970998)
\curveto(714.19448042,72.98970564)(714.33948028,72.89470574)(714.48947266,72.80470998)
\curveto(714.62947999,72.71470592)(714.76947985,72.61970601)(714.90947266,72.51970998)
\lineto(715.68947266,71.97970998)
\curveto(715.94947867,71.80970682)(716.20947841,71.634707)(716.46947266,71.45470998)
\curveto(716.57947804,71.37470726)(716.68447793,71.29970733)(716.78447266,71.22970998)
\lineto(717.08447266,71.01970998)
\curveto(717.16447745,70.96970766)(717.23947738,70.91970771)(717.30947266,70.86970998)
\curveto(717.37947724,70.8297078)(717.45447716,70.78470785)(717.53447266,70.73470998)
\curveto(717.59447702,70.68470795)(717.65947696,70.634708)(717.72947266,70.58470998)
\curveto(717.78947683,70.54470809)(717.85947676,70.50470813)(717.93947266,70.46470998)
\curveto(717.99947662,70.42470821)(718.06947655,70.39970823)(718.14947266,70.38970998)
\curveto(718.2194764,70.37970825)(718.27447634,70.41470822)(718.31447266,70.49470998)
\curveto(718.36447625,70.56470807)(718.38947623,70.67470796)(718.38947266,70.82470998)
\curveto(718.37947624,70.98470765)(718.37447624,71.11970751)(718.37447266,71.22970998)
\lineto(718.37447266,72.90970998)
\lineto(718.37447266,73.34470998)
\curveto(718.37447624,73.49470514)(718.33447628,73.59970503)(718.25447266,73.65970998)
}
}
{
\newrgbcolor{curcolor}{0 0 0}
\pscustom[linestyle=none,fillstyle=solid,fillcolor=curcolor]
{
\newpath
\moveto(720.30947266,78.63431936)
\lineto(720.30947266,79.26431936)
\lineto(720.30947266,79.45931936)
\curveto(720.30947431,79.52931683)(720.3194743,79.58931677)(720.33947266,79.63931936)
\curveto(720.37947424,79.70931665)(720.4194742,79.7593166)(720.45947266,79.78931936)
\curveto(720.50947411,79.82931653)(720.57447404,79.84931651)(720.65447266,79.84931936)
\curveto(720.73447388,79.8593165)(720.8194738,79.86431649)(720.90947266,79.86431936)
\lineto(721.62947266,79.86431936)
\curveto(722.10947251,79.86431649)(722.5194721,79.80431655)(722.85947266,79.68431936)
\curveto(723.19947142,79.56431679)(723.47447114,79.36931699)(723.68447266,79.09931936)
\curveto(723.73447088,79.02931733)(723.77947084,78.9593174)(723.81947266,78.88931936)
\curveto(723.86947075,78.82931753)(723.9144707,78.7543176)(723.95447266,78.66431936)
\curveto(723.96447065,78.64431771)(723.97447064,78.61431774)(723.98447266,78.57431936)
\curveto(724.00447061,78.53431782)(724.00947061,78.48931787)(723.99947266,78.43931936)
\curveto(723.96947065,78.34931801)(723.89447072,78.29431806)(723.77447266,78.27431936)
\curveto(723.66447095,78.2543181)(723.56947105,78.26931809)(723.48947266,78.31931936)
\curveto(723.4194712,78.34931801)(723.35447126,78.39431796)(723.29447266,78.45431936)
\curveto(723.24447137,78.52431783)(723.19447142,78.58931777)(723.14447266,78.64931936)
\curveto(723.09447152,78.71931764)(723.0194716,78.77931758)(722.91947266,78.82931936)
\curveto(722.82947179,78.88931747)(722.73947188,78.93931742)(722.64947266,78.97931936)
\curveto(722.619472,78.99931736)(722.55947206,79.02431733)(722.46947266,79.05431936)
\curveto(722.38947223,79.08431727)(722.3194723,79.08931727)(722.25947266,79.06931936)
\curveto(722.1194725,79.03931732)(722.02947259,78.97931738)(721.98947266,78.88931936)
\curveto(721.95947266,78.80931755)(721.94447267,78.71931764)(721.94447266,78.61931936)
\curveto(721.94447267,78.51931784)(721.9194727,78.43431792)(721.86947266,78.36431936)
\curveto(721.79947282,78.27431808)(721.65947296,78.22931813)(721.44947266,78.22931936)
\lineto(720.89447266,78.22931936)
\lineto(720.66947266,78.22931936)
\curveto(720.58947403,78.23931812)(720.52447409,78.2593181)(720.47447266,78.28931936)
\curveto(720.39447422,78.34931801)(720.34947427,78.41931794)(720.33947266,78.49931936)
\curveto(720.32947429,78.51931784)(720.32447429,78.53931782)(720.32447266,78.55931936)
\curveto(720.32447429,78.58931777)(720.3194743,78.61431774)(720.30947266,78.63431936)
}
}
{
\newrgbcolor{curcolor}{0 0 0}
\pscustom[linestyle=none,fillstyle=solid,fillcolor=curcolor]
{
}
}
{
\newrgbcolor{curcolor}{0 0 0}
\pscustom[linestyle=none,fillstyle=solid,fillcolor=curcolor]
{
\newpath
\moveto(711.33947266,89.26463186)
\curveto(711.32948329,89.95462722)(711.44948317,90.55462662)(711.69947266,91.06463186)
\curveto(711.94948267,91.58462559)(712.28448233,91.9796252)(712.70447266,92.24963186)
\curveto(712.78448183,92.29962488)(712.87448174,92.34462483)(712.97447266,92.38463186)
\curveto(713.06448155,92.42462475)(713.15948146,92.46962471)(713.25947266,92.51963186)
\curveto(713.35948126,92.55962462)(713.45948116,92.58962459)(713.55947266,92.60963186)
\curveto(713.65948096,92.62962455)(713.76448085,92.64962453)(713.87447266,92.66963186)
\curveto(713.92448069,92.68962449)(713.96948065,92.69462448)(714.00947266,92.68463186)
\curveto(714.04948057,92.6746245)(714.09448052,92.6796245)(714.14447266,92.69963186)
\curveto(714.19448042,92.70962447)(714.27948034,92.71462446)(714.39947266,92.71463186)
\curveto(714.50948011,92.71462446)(714.59448002,92.70962447)(714.65447266,92.69963186)
\curveto(714.7144799,92.6796245)(714.77447984,92.66962451)(714.83447266,92.66963186)
\curveto(714.89447972,92.6796245)(714.95447966,92.6746245)(715.01447266,92.65463186)
\curveto(715.15447946,92.61462456)(715.28947933,92.5796246)(715.41947266,92.54963186)
\curveto(715.54947907,92.51962466)(715.67447894,92.4796247)(715.79447266,92.42963186)
\curveto(715.93447868,92.36962481)(716.05947856,92.29962488)(716.16947266,92.21963186)
\curveto(716.27947834,92.14962503)(716.38947823,92.0746251)(716.49947266,91.99463186)
\lineto(716.55947266,91.93463186)
\curveto(716.57947804,91.92462525)(716.59947802,91.90962527)(716.61947266,91.88963186)
\curveto(716.77947784,91.76962541)(716.92447769,91.63462554)(717.05447266,91.48463186)
\curveto(717.18447743,91.33462584)(717.30947731,91.174626)(717.42947266,91.00463186)
\curveto(717.64947697,90.69462648)(717.85447676,90.39962678)(718.04447266,90.11963186)
\curveto(718.18447643,89.88962729)(718.3194763,89.65962752)(718.44947266,89.42963186)
\curveto(718.57947604,89.20962797)(718.7144759,88.98962819)(718.85447266,88.76963186)
\curveto(719.02447559,88.51962866)(719.20447541,88.2796289)(719.39447266,88.04963186)
\curveto(719.58447503,87.82962935)(719.80947481,87.63962954)(720.06947266,87.47963186)
\curveto(720.12947449,87.43962974)(720.18947443,87.40462977)(720.24947266,87.37463186)
\curveto(720.29947432,87.34462983)(720.36447425,87.31462986)(720.44447266,87.28463186)
\curveto(720.5144741,87.26462991)(720.57447404,87.25962992)(720.62447266,87.26963186)
\curveto(720.69447392,87.28962989)(720.74947387,87.32462985)(720.78947266,87.37463186)
\curveto(720.8194738,87.42462975)(720.83947378,87.48462969)(720.84947266,87.55463186)
\lineto(720.84947266,87.79463186)
\lineto(720.84947266,88.54463186)
\lineto(720.84947266,91.34963186)
\lineto(720.84947266,92.00963186)
\curveto(720.84947377,92.09962508)(720.85447376,92.18462499)(720.86447266,92.26463186)
\curveto(720.86447375,92.34462483)(720.88447373,92.40962477)(720.92447266,92.45963186)
\curveto(720.96447365,92.50962467)(721.03947358,92.54962463)(721.14947266,92.57963186)
\curveto(721.24947337,92.61962456)(721.34947327,92.62962455)(721.44947266,92.60963186)
\lineto(721.58447266,92.60963186)
\curveto(721.65447296,92.58962459)(721.7144729,92.56962461)(721.76447266,92.54963186)
\curveto(721.8144728,92.52962465)(721.85447276,92.49462468)(721.88447266,92.44463186)
\curveto(721.92447269,92.39462478)(721.94447267,92.32462485)(721.94447266,92.23463186)
\lineto(721.94447266,91.96463186)
\lineto(721.94447266,91.06463186)
\lineto(721.94447266,87.55463186)
\lineto(721.94447266,86.48963186)
\curveto(721.94447267,86.40963077)(721.94947267,86.31963086)(721.95947266,86.21963186)
\curveto(721.95947266,86.11963106)(721.94947267,86.03463114)(721.92947266,85.96463186)
\curveto(721.85947276,85.75463142)(721.67947294,85.68963149)(721.38947266,85.76963186)
\curveto(721.34947327,85.7796314)(721.3144733,85.7796314)(721.28447266,85.76963186)
\curveto(721.24447337,85.76963141)(721.19947342,85.7796314)(721.14947266,85.79963186)
\curveto(721.06947355,85.81963136)(720.98447363,85.83963134)(720.89447266,85.85963186)
\curveto(720.80447381,85.8796313)(720.7194739,85.90463127)(720.63947266,85.93463186)
\curveto(720.14947447,86.09463108)(719.73447488,86.29463088)(719.39447266,86.53463186)
\curveto(719.14447547,86.71463046)(718.9194757,86.91963026)(718.71947266,87.14963186)
\curveto(718.50947611,87.3796298)(718.3144763,87.61962956)(718.13447266,87.86963186)
\curveto(717.95447666,88.12962905)(717.78447683,88.39462878)(717.62447266,88.66463186)
\curveto(717.45447716,88.94462823)(717.27947734,89.21462796)(717.09947266,89.47463186)
\curveto(717.0194776,89.58462759)(716.94447767,89.68962749)(716.87447266,89.78963186)
\curveto(716.80447781,89.89962728)(716.72947789,90.00962717)(716.64947266,90.11963186)
\curveto(716.619478,90.15962702)(716.58947803,90.19462698)(716.55947266,90.22463186)
\curveto(716.5194781,90.26462691)(716.48947813,90.30462687)(716.46947266,90.34463186)
\curveto(716.35947826,90.48462669)(716.23447838,90.60962657)(716.09447266,90.71963186)
\curveto(716.06447855,90.73962644)(716.03947858,90.76462641)(716.01947266,90.79463186)
\curveto(715.98947863,90.82462635)(715.95947866,90.84962633)(715.92947266,90.86963186)
\curveto(715.82947879,90.94962623)(715.72947889,91.01462616)(715.62947266,91.06463186)
\curveto(715.52947909,91.12462605)(715.4194792,91.179626)(715.29947266,91.22963186)
\curveto(715.22947939,91.25962592)(715.15447946,91.2796259)(715.07447266,91.28963186)
\lineto(714.83447266,91.34963186)
\lineto(714.74447266,91.34963186)
\curveto(714.7144799,91.35962582)(714.68447993,91.36462581)(714.65447266,91.36463186)
\curveto(714.58448003,91.38462579)(714.48948013,91.38962579)(714.36947266,91.37963186)
\curveto(714.23948038,91.3796258)(714.13948048,91.36962581)(714.06947266,91.34963186)
\curveto(713.98948063,91.32962585)(713.9144807,91.30962587)(713.84447266,91.28963186)
\curveto(713.76448085,91.2796259)(713.68448093,91.25962592)(713.60447266,91.22963186)
\curveto(713.36448125,91.11962606)(713.16448145,90.96962621)(713.00447266,90.77963186)
\curveto(712.83448178,90.59962658)(712.69448192,90.3796268)(712.58447266,90.11963186)
\curveto(712.56448205,90.04962713)(712.54948207,89.9796272)(712.53947266,89.90963186)
\curveto(712.5194821,89.83962734)(712.49948212,89.76462741)(712.47947266,89.68463186)
\curveto(712.45948216,89.60462757)(712.44948217,89.49462768)(712.44947266,89.35463186)
\curveto(712.44948217,89.22462795)(712.45948216,89.11962806)(712.47947266,89.03963186)
\curveto(712.48948213,88.9796282)(712.49448212,88.92462825)(712.49447266,88.87463186)
\curveto(712.49448212,88.82462835)(712.50448211,88.7746284)(712.52447266,88.72463186)
\curveto(712.56448205,88.62462855)(712.60448201,88.52962865)(712.64447266,88.43963186)
\curveto(712.68448193,88.35962882)(712.72948189,88.2796289)(712.77947266,88.19963186)
\curveto(712.79948182,88.16962901)(712.82448179,88.13962904)(712.85447266,88.10963186)
\curveto(712.88448173,88.08962909)(712.90948171,88.06462911)(712.92947266,88.03463186)
\lineto(713.00447266,87.95963186)
\curveto(713.02448159,87.92962925)(713.04448157,87.90462927)(713.06447266,87.88463186)
\lineto(713.27447266,87.73463186)
\curveto(713.33448128,87.69462948)(713.39948122,87.64962953)(713.46947266,87.59963186)
\curveto(713.55948106,87.53962964)(713.66448095,87.48962969)(713.78447266,87.44963186)
\curveto(713.89448072,87.41962976)(714.00448061,87.38462979)(714.11447266,87.34463186)
\curveto(714.22448039,87.30462987)(714.36948025,87.2796299)(714.54947266,87.26963186)
\curveto(714.7194799,87.25962992)(714.84447977,87.22962995)(714.92447266,87.17963186)
\curveto(715.00447961,87.12963005)(715.04947957,87.05463012)(715.05947266,86.95463186)
\curveto(715.06947955,86.85463032)(715.07447954,86.74463043)(715.07447266,86.62463186)
\curveto(715.07447954,86.58463059)(715.07947954,86.54463063)(715.08947266,86.50463186)
\curveto(715.08947953,86.46463071)(715.08447953,86.42963075)(715.07447266,86.39963186)
\curveto(715.05447956,86.34963083)(715.04447957,86.29963088)(715.04447266,86.24963186)
\curveto(715.04447957,86.20963097)(715.03447958,86.16963101)(715.01447266,86.12963186)
\curveto(714.95447966,86.03963114)(714.8194798,85.99463118)(714.60947266,85.99463186)
\lineto(714.48947266,85.99463186)
\curveto(714.42948019,86.00463117)(714.36948025,86.00963117)(714.30947266,86.00963186)
\curveto(714.23948038,86.01963116)(714.17448044,86.02963115)(714.11447266,86.03963186)
\curveto(714.00448061,86.05963112)(713.90448071,86.0796311)(713.81447266,86.09963186)
\curveto(713.7144809,86.11963106)(713.619481,86.14963103)(713.52947266,86.18963186)
\curveto(713.45948116,86.20963097)(713.39948122,86.22963095)(713.34947266,86.24963186)
\lineto(713.16947266,86.30963186)
\curveto(712.90948171,86.42963075)(712.66448195,86.58463059)(712.43447266,86.77463186)
\curveto(712.20448241,86.9746302)(712.0194826,87.18962999)(711.87947266,87.41963186)
\curveto(711.79948282,87.52962965)(711.73448288,87.64462953)(711.68447266,87.76463186)
\lineto(711.53447266,88.15463186)
\curveto(711.48448313,88.26462891)(711.45448316,88.3796288)(711.44447266,88.49963186)
\curveto(711.42448319,88.61962856)(711.39948322,88.74462843)(711.36947266,88.87463186)
\curveto(711.36948325,88.94462823)(711.36948325,89.00962817)(711.36947266,89.06963186)
\curveto(711.35948326,89.12962805)(711.34948327,89.19462798)(711.33947266,89.26463186)
}
}
{
\newrgbcolor{curcolor}{0 0 0}
\pscustom[linestyle=none,fillstyle=solid,fillcolor=curcolor]
{
\newpath
\moveto(716.85947266,101.36424123)
\lineto(717.11447266,101.36424123)
\curveto(717.19447742,101.37423353)(717.26947735,101.36923353)(717.33947266,101.34924123)
\lineto(717.57947266,101.34924123)
\lineto(717.74447266,101.34924123)
\curveto(717.84447677,101.32923357)(717.94947667,101.31923358)(718.05947266,101.31924123)
\curveto(718.15947646,101.31923358)(718.25947636,101.30923359)(718.35947266,101.28924123)
\lineto(718.50947266,101.28924123)
\curveto(718.64947597,101.25923364)(718.78947583,101.23923366)(718.92947266,101.22924123)
\curveto(719.05947556,101.21923368)(719.18947543,101.19423371)(719.31947266,101.15424123)
\curveto(719.39947522,101.13423377)(719.48447513,101.11423379)(719.57447266,101.09424123)
\lineto(719.81447266,101.03424123)
\lineto(720.11447266,100.91424123)
\curveto(720.20447441,100.88423402)(720.29447432,100.84923405)(720.38447266,100.80924123)
\curveto(720.60447401,100.70923419)(720.8194738,100.57423433)(721.02947266,100.40424123)
\curveto(721.23947338,100.24423466)(721.40947321,100.06923483)(721.53947266,99.87924123)
\curveto(721.57947304,99.82923507)(721.619473,99.76923513)(721.65947266,99.69924123)
\curveto(721.68947293,99.63923526)(721.72447289,99.57923532)(721.76447266,99.51924123)
\curveto(721.8144728,99.43923546)(721.85447276,99.34423556)(721.88447266,99.23424123)
\curveto(721.9144727,99.12423578)(721.94447267,99.01923588)(721.97447266,98.91924123)
\curveto(722.0144726,98.80923609)(722.03947258,98.6992362)(722.04947266,98.58924123)
\curveto(722.05947256,98.47923642)(722.07447254,98.36423654)(722.09447266,98.24424123)
\curveto(722.10447251,98.2042367)(722.10447251,98.15923674)(722.09447266,98.10924123)
\curveto(722.09447252,98.06923683)(722.09947252,98.02923687)(722.10947266,97.98924123)
\curveto(722.1194725,97.94923695)(722.12447249,97.89423701)(722.12447266,97.82424123)
\curveto(722.12447249,97.75423715)(722.1194725,97.7042372)(722.10947266,97.67424123)
\curveto(722.08947253,97.62423728)(722.08447253,97.57923732)(722.09447266,97.53924123)
\curveto(722.10447251,97.4992374)(722.10447251,97.46423744)(722.09447266,97.43424123)
\lineto(722.09447266,97.34424123)
\curveto(722.07447254,97.28423762)(722.05947256,97.21923768)(722.04947266,97.14924123)
\curveto(722.04947257,97.08923781)(722.04447257,97.02423788)(722.03447266,96.95424123)
\curveto(721.98447263,96.78423812)(721.93447268,96.62423828)(721.88447266,96.47424123)
\curveto(721.83447278,96.32423858)(721.76947285,96.17923872)(721.68947266,96.03924123)
\curveto(721.64947297,95.98923891)(721.619473,95.93423897)(721.59947266,95.87424123)
\curveto(721.56947305,95.82423908)(721.53447308,95.77423913)(721.49447266,95.72424123)
\curveto(721.3144733,95.48423942)(721.09447352,95.28423962)(720.83447266,95.12424123)
\curveto(720.57447404,94.96423994)(720.28947433,94.82424008)(719.97947266,94.70424123)
\curveto(719.83947478,94.64424026)(719.69947492,94.5992403)(719.55947266,94.56924123)
\curveto(719.40947521,94.53924036)(719.25447536,94.5042404)(719.09447266,94.46424123)
\curveto(718.98447563,94.44424046)(718.87447574,94.42924047)(718.76447266,94.41924123)
\curveto(718.65447596,94.40924049)(718.54447607,94.39424051)(718.43447266,94.37424123)
\curveto(718.39447622,94.36424054)(718.35447626,94.35924054)(718.31447266,94.35924123)
\curveto(718.27447634,94.36924053)(718.23447638,94.36924053)(718.19447266,94.35924123)
\curveto(718.14447647,94.34924055)(718.09447652,94.34424056)(718.04447266,94.34424123)
\lineto(717.87947266,94.34424123)
\curveto(717.82947679,94.32424058)(717.77947684,94.31924058)(717.72947266,94.32924123)
\curveto(717.66947695,94.33924056)(717.614477,94.33924056)(717.56447266,94.32924123)
\curveto(717.52447709,94.31924058)(717.47947714,94.31924058)(717.42947266,94.32924123)
\curveto(717.37947724,94.33924056)(717.32947729,94.33424057)(717.27947266,94.31424123)
\curveto(717.20947741,94.29424061)(717.13447748,94.28924061)(717.05447266,94.29924123)
\curveto(716.96447765,94.30924059)(716.87947774,94.31424059)(716.79947266,94.31424123)
\curveto(716.70947791,94.31424059)(716.60947801,94.30924059)(716.49947266,94.29924123)
\curveto(716.37947824,94.28924061)(716.27947834,94.29424061)(716.19947266,94.31424123)
\lineto(715.91447266,94.31424123)
\lineto(715.28447266,94.35924123)
\curveto(715.18447943,94.36924053)(715.08947953,94.37924052)(714.99947266,94.38924123)
\lineto(714.69947266,94.41924123)
\curveto(714.64947997,94.43924046)(714.59948002,94.44424046)(714.54947266,94.43424123)
\curveto(714.48948013,94.43424047)(714.43448018,94.44424046)(714.38447266,94.46424123)
\curveto(714.2144804,94.51424039)(714.04948057,94.55424035)(713.88947266,94.58424123)
\curveto(713.7194809,94.61424029)(713.55948106,94.66424024)(713.40947266,94.73424123)
\curveto(712.94948167,94.92423998)(712.57448204,95.14423976)(712.28447266,95.39424123)
\curveto(711.99448262,95.65423925)(711.74948287,96.01423889)(711.54947266,96.47424123)
\curveto(711.49948312,96.6042383)(711.46448315,96.73423817)(711.44447266,96.86424123)
\curveto(711.42448319,97.0042379)(711.39948322,97.14423776)(711.36947266,97.28424123)
\curveto(711.35948326,97.35423755)(711.35448326,97.41923748)(711.35447266,97.47924123)
\curveto(711.35448326,97.53923736)(711.34948327,97.6042373)(711.33947266,97.67424123)
\curveto(711.3194833,98.5042364)(711.46948315,99.17423573)(711.78947266,99.68424123)
\curveto(712.09948252,100.19423471)(712.53948208,100.57423433)(713.10947266,100.82424123)
\curveto(713.22948139,100.87423403)(713.35448126,100.91923398)(713.48447266,100.95924123)
\curveto(713.614481,100.9992339)(713.74948087,101.04423386)(713.88947266,101.09424123)
\curveto(713.96948065,101.11423379)(714.05448056,101.12923377)(714.14447266,101.13924123)
\lineto(714.38447266,101.19924123)
\curveto(714.49448012,101.22923367)(714.60448001,101.24423366)(714.71447266,101.24424123)
\curveto(714.82447979,101.25423365)(714.93447968,101.26923363)(715.04447266,101.28924123)
\curveto(715.09447952,101.30923359)(715.13947948,101.31423359)(715.17947266,101.30424123)
\curveto(715.2194794,101.3042336)(715.25947936,101.30923359)(715.29947266,101.31924123)
\curveto(715.34947927,101.32923357)(715.40447921,101.32923357)(715.46447266,101.31924123)
\curveto(715.5144791,101.31923358)(715.56447905,101.32423358)(715.61447266,101.33424123)
\lineto(715.74947266,101.33424123)
\curveto(715.80947881,101.35423355)(715.87947874,101.35423355)(715.95947266,101.33424123)
\curveto(716.02947859,101.32423358)(716.09447852,101.32923357)(716.15447266,101.34924123)
\curveto(716.18447843,101.35923354)(716.22447839,101.36423354)(716.27447266,101.36424123)
\lineto(716.39447266,101.36424123)
\lineto(716.85947266,101.36424123)
\moveto(719.18447266,99.81924123)
\curveto(718.86447575,99.91923498)(718.49947612,99.97923492)(718.08947266,99.99924123)
\curveto(717.67947694,100.01923488)(717.26947735,100.02923487)(716.85947266,100.02924123)
\curveto(716.42947819,100.02923487)(716.00947861,100.01923488)(715.59947266,99.99924123)
\curveto(715.18947943,99.97923492)(714.80447981,99.93423497)(714.44447266,99.86424123)
\curveto(714.08448053,99.79423511)(713.76448085,99.68423522)(713.48447266,99.53424123)
\curveto(713.19448142,99.39423551)(712.95948166,99.1992357)(712.77947266,98.94924123)
\curveto(712.66948195,98.78923611)(712.58948203,98.60923629)(712.53947266,98.40924123)
\curveto(712.47948214,98.20923669)(712.44948217,97.96423694)(712.44947266,97.67424123)
\curveto(712.46948215,97.65423725)(712.47948214,97.61923728)(712.47947266,97.56924123)
\curveto(712.46948215,97.51923738)(712.46948215,97.47923742)(712.47947266,97.44924123)
\curveto(712.49948212,97.36923753)(712.5194821,97.29423761)(712.53947266,97.22424123)
\curveto(712.54948207,97.16423774)(712.56948205,97.0992378)(712.59947266,97.02924123)
\curveto(712.7194819,96.75923814)(712.88948173,96.53923836)(713.10947266,96.36924123)
\curveto(713.3194813,96.20923869)(713.56448105,96.07423883)(713.84447266,95.96424123)
\curveto(713.95448066,95.91423899)(714.07448054,95.87423903)(714.20447266,95.84424123)
\curveto(714.32448029,95.82423908)(714.44948017,95.7992391)(714.57947266,95.76924123)
\curveto(714.62947999,95.74923915)(714.68447993,95.73923916)(714.74447266,95.73924123)
\curveto(714.79447982,95.73923916)(714.84447977,95.73423917)(714.89447266,95.72424123)
\curveto(714.98447963,95.71423919)(715.07947954,95.7042392)(715.17947266,95.69424123)
\curveto(715.26947935,95.68423922)(715.36447925,95.67423923)(715.46447266,95.66424123)
\curveto(715.54447907,95.66423924)(715.62947899,95.65923924)(715.71947266,95.64924123)
\lineto(715.95947266,95.64924123)
\lineto(716.13947266,95.64924123)
\curveto(716.16947845,95.63923926)(716.20447841,95.63423927)(716.24447266,95.63424123)
\lineto(716.37947266,95.63424123)
\lineto(716.82947266,95.63424123)
\curveto(716.90947771,95.63423927)(716.99447762,95.62923927)(717.08447266,95.61924123)
\curveto(717.16447745,95.61923928)(717.23947738,95.62923927)(717.30947266,95.64924123)
\lineto(717.57947266,95.64924123)
\curveto(717.59947702,95.64923925)(717.62947699,95.64423926)(717.66947266,95.63424123)
\curveto(717.69947692,95.63423927)(717.72447689,95.63923926)(717.74447266,95.64924123)
\curveto(717.84447677,95.65923924)(717.94447667,95.66423924)(718.04447266,95.66424123)
\curveto(718.13447648,95.67423923)(718.23447638,95.68423922)(718.34447266,95.69424123)
\curveto(718.46447615,95.72423918)(718.58947603,95.73923916)(718.71947266,95.73924123)
\curveto(718.83947578,95.74923915)(718.95447566,95.77423913)(719.06447266,95.81424123)
\curveto(719.36447525,95.89423901)(719.62947499,95.97923892)(719.85947266,96.06924123)
\curveto(720.08947453,96.16923873)(720.30447431,96.31423859)(720.50447266,96.50424123)
\curveto(720.70447391,96.71423819)(720.85447376,96.97923792)(720.95447266,97.29924123)
\curveto(720.97447364,97.33923756)(720.98447363,97.37423753)(720.98447266,97.40424123)
\curveto(720.97447364,97.44423746)(720.97947364,97.48923741)(720.99947266,97.53924123)
\curveto(721.00947361,97.57923732)(721.0194736,97.64923725)(721.02947266,97.74924123)
\curveto(721.03947358,97.85923704)(721.03447358,97.94423696)(721.01447266,98.00424123)
\curveto(720.99447362,98.07423683)(720.98447363,98.14423676)(720.98447266,98.21424123)
\curveto(720.97447364,98.28423662)(720.95947366,98.34923655)(720.93947266,98.40924123)
\curveto(720.87947374,98.60923629)(720.79447382,98.78923611)(720.68447266,98.94924123)
\curveto(720.66447395,98.97923592)(720.64447397,99.0042359)(720.62447266,99.02424123)
\lineto(720.56447266,99.08424123)
\curveto(720.54447407,99.12423578)(720.50447411,99.17423573)(720.44447266,99.23424123)
\curveto(720.30447431,99.33423557)(720.17447444,99.41923548)(720.05447266,99.48924123)
\curveto(719.93447468,99.55923534)(719.78947483,99.62923527)(719.61947266,99.69924123)
\curveto(719.54947507,99.72923517)(719.47947514,99.74923515)(719.40947266,99.75924123)
\curveto(719.33947528,99.77923512)(719.26447535,99.7992351)(719.18447266,99.81924123)
}
}
{
\newrgbcolor{curcolor}{0 0 0}
\pscustom[linestyle=none,fillstyle=solid,fillcolor=curcolor]
{
\newpath
\moveto(711.33947266,106.77385061)
\curveto(711.33948328,106.87384575)(711.34948327,106.96884566)(711.36947266,107.05885061)
\curveto(711.37948324,107.14884548)(711.40948321,107.21384541)(711.45947266,107.25385061)
\curveto(711.53948308,107.31384531)(711.64448297,107.34384528)(711.77447266,107.34385061)
\lineto(712.16447266,107.34385061)
\lineto(713.66447266,107.34385061)
\lineto(720.05447266,107.34385061)
\lineto(721.22447266,107.34385061)
\lineto(721.53947266,107.34385061)
\curveto(721.63947298,107.35384527)(721.7194729,107.33884529)(721.77947266,107.29885061)
\curveto(721.85947276,107.24884538)(721.90947271,107.17384545)(721.92947266,107.07385061)
\curveto(721.93947268,106.98384564)(721.94447267,106.87384575)(721.94447266,106.74385061)
\lineto(721.94447266,106.51885061)
\curveto(721.92447269,106.43884619)(721.90947271,106.36884626)(721.89947266,106.30885061)
\curveto(721.87947274,106.24884638)(721.83947278,106.19884643)(721.77947266,106.15885061)
\curveto(721.7194729,106.11884651)(721.64447297,106.09884653)(721.55447266,106.09885061)
\lineto(721.25447266,106.09885061)
\lineto(720.15947266,106.09885061)
\lineto(714.81947266,106.09885061)
\curveto(714.72947989,106.07884655)(714.65447996,106.06384656)(714.59447266,106.05385061)
\curveto(714.52448009,106.05384657)(714.46448015,106.0238466)(714.41447266,105.96385061)
\curveto(714.36448025,105.89384673)(714.33948028,105.80384682)(714.33947266,105.69385061)
\curveto(714.32948029,105.59384703)(714.32448029,105.48384714)(714.32447266,105.36385061)
\lineto(714.32447266,104.22385061)
\lineto(714.32447266,103.72885061)
\curveto(714.3144803,103.56884906)(714.25448036,103.45884917)(714.14447266,103.39885061)
\curveto(714.1144805,103.37884925)(714.08448053,103.36884926)(714.05447266,103.36885061)
\curveto(714.0144806,103.36884926)(713.96948065,103.36384926)(713.91947266,103.35385061)
\curveto(713.79948082,103.33384929)(713.68948093,103.33884929)(713.58947266,103.36885061)
\curveto(713.48948113,103.40884922)(713.4194812,103.46384916)(713.37947266,103.53385061)
\curveto(713.32948129,103.61384901)(713.30448131,103.73384889)(713.30447266,103.89385061)
\curveto(713.30448131,104.05384857)(713.28948133,104.18884844)(713.25947266,104.29885061)
\curveto(713.24948137,104.34884828)(713.24448137,104.40384822)(713.24447266,104.46385061)
\curveto(713.23448138,104.5238481)(713.2194814,104.58384804)(713.19947266,104.64385061)
\curveto(713.14948147,104.79384783)(713.09948152,104.93884769)(713.04947266,105.07885061)
\curveto(712.98948163,105.21884741)(712.9194817,105.35384727)(712.83947266,105.48385061)
\curveto(712.74948187,105.623847)(712.64448197,105.74384688)(712.52447266,105.84385061)
\curveto(712.40448221,105.94384668)(712.27448234,106.03884659)(712.13447266,106.12885061)
\curveto(712.03448258,106.18884644)(711.92448269,106.23384639)(711.80447266,106.26385061)
\curveto(711.68448293,106.30384632)(711.57948304,106.35384627)(711.48947266,106.41385061)
\curveto(711.42948319,106.46384616)(711.38948323,106.53384609)(711.36947266,106.62385061)
\curveto(711.35948326,106.64384598)(711.35448326,106.66884596)(711.35447266,106.69885061)
\curveto(711.35448326,106.7288459)(711.34948327,106.75384587)(711.33947266,106.77385061)
}
}
{
\newrgbcolor{curcolor}{0 0 0}
\pscustom[linestyle=none,fillstyle=solid,fillcolor=curcolor]
{
\newpath
\moveto(711.33947266,115.12345998)
\curveto(711.33948328,115.22345513)(711.34948327,115.31845503)(711.36947266,115.40845998)
\curveto(711.37948324,115.49845485)(711.40948321,115.56345479)(711.45947266,115.60345998)
\curveto(711.53948308,115.66345469)(711.64448297,115.69345466)(711.77447266,115.69345998)
\lineto(712.16447266,115.69345998)
\lineto(713.66447266,115.69345998)
\lineto(720.05447266,115.69345998)
\lineto(721.22447266,115.69345998)
\lineto(721.53947266,115.69345998)
\curveto(721.63947298,115.70345465)(721.7194729,115.68845466)(721.77947266,115.64845998)
\curveto(721.85947276,115.59845475)(721.90947271,115.52345483)(721.92947266,115.42345998)
\curveto(721.93947268,115.33345502)(721.94447267,115.22345513)(721.94447266,115.09345998)
\lineto(721.94447266,114.86845998)
\curveto(721.92447269,114.78845556)(721.90947271,114.71845563)(721.89947266,114.65845998)
\curveto(721.87947274,114.59845575)(721.83947278,114.5484558)(721.77947266,114.50845998)
\curveto(721.7194729,114.46845588)(721.64447297,114.4484559)(721.55447266,114.44845998)
\lineto(721.25447266,114.44845998)
\lineto(720.15947266,114.44845998)
\lineto(714.81947266,114.44845998)
\curveto(714.72947989,114.42845592)(714.65447996,114.41345594)(714.59447266,114.40345998)
\curveto(714.52448009,114.40345595)(714.46448015,114.37345598)(714.41447266,114.31345998)
\curveto(714.36448025,114.24345611)(714.33948028,114.1534562)(714.33947266,114.04345998)
\curveto(714.32948029,113.94345641)(714.32448029,113.83345652)(714.32447266,113.71345998)
\lineto(714.32447266,112.57345998)
\lineto(714.32447266,112.07845998)
\curveto(714.3144803,111.91845843)(714.25448036,111.80845854)(714.14447266,111.74845998)
\curveto(714.1144805,111.72845862)(714.08448053,111.71845863)(714.05447266,111.71845998)
\curveto(714.0144806,111.71845863)(713.96948065,111.71345864)(713.91947266,111.70345998)
\curveto(713.79948082,111.68345867)(713.68948093,111.68845866)(713.58947266,111.71845998)
\curveto(713.48948113,111.75845859)(713.4194812,111.81345854)(713.37947266,111.88345998)
\curveto(713.32948129,111.96345839)(713.30448131,112.08345827)(713.30447266,112.24345998)
\curveto(713.30448131,112.40345795)(713.28948133,112.53845781)(713.25947266,112.64845998)
\curveto(713.24948137,112.69845765)(713.24448137,112.7534576)(713.24447266,112.81345998)
\curveto(713.23448138,112.87345748)(713.2194814,112.93345742)(713.19947266,112.99345998)
\curveto(713.14948147,113.14345721)(713.09948152,113.28845706)(713.04947266,113.42845998)
\curveto(712.98948163,113.56845678)(712.9194817,113.70345665)(712.83947266,113.83345998)
\curveto(712.74948187,113.97345638)(712.64448197,114.09345626)(712.52447266,114.19345998)
\curveto(712.40448221,114.29345606)(712.27448234,114.38845596)(712.13447266,114.47845998)
\curveto(712.03448258,114.53845581)(711.92448269,114.58345577)(711.80447266,114.61345998)
\curveto(711.68448293,114.6534557)(711.57948304,114.70345565)(711.48947266,114.76345998)
\curveto(711.42948319,114.81345554)(711.38948323,114.88345547)(711.36947266,114.97345998)
\curveto(711.35948326,114.99345536)(711.35448326,115.01845533)(711.35447266,115.04845998)
\curveto(711.35448326,115.07845527)(711.34948327,115.10345525)(711.33947266,115.12345998)
}
}
{
\newrgbcolor{curcolor}{0 0 0}
\pscustom[linestyle=none,fillstyle=solid,fillcolor=curcolor]
{
\newpath
\moveto(732.17578857,37.28705373)
\curveto(732.17579927,37.35704805)(732.17579927,37.43704797)(732.17578857,37.52705373)
\curveto(732.16579928,37.61704779)(732.16579928,37.70204771)(732.17578857,37.78205373)
\curveto(732.17579927,37.87204754)(732.18579926,37.95204746)(732.20578857,38.02205373)
\curveto(732.22579922,38.10204731)(732.25579919,38.15704725)(732.29578857,38.18705373)
\curveto(732.3457991,38.21704719)(732.42079902,38.23704717)(732.52078857,38.24705373)
\curveto(732.61079883,38.26704714)(732.71579873,38.27704713)(732.83578857,38.27705373)
\curveto(732.9457985,38.28704712)(733.06079838,38.28704712)(733.18078857,38.27705373)
\lineto(733.48078857,38.27705373)
\lineto(736.49578857,38.27705373)
\lineto(739.39078857,38.27705373)
\curveto(739.72079172,38.27704713)(740.0457914,38.27204714)(740.36578857,38.26205373)
\curveto(740.67579077,38.26204715)(740.95579049,38.22204719)(741.20578857,38.14205373)
\curveto(741.55578989,38.02204739)(741.85078959,37.86704754)(742.09078857,37.67705373)
\curveto(742.32078912,37.48704792)(742.52078892,37.24704816)(742.69078857,36.95705373)
\curveto(742.7407887,36.89704851)(742.77578867,36.83204858)(742.79578857,36.76205373)
\curveto(742.81578863,36.70204871)(742.8407886,36.63204878)(742.87078857,36.55205373)
\curveto(742.92078852,36.43204898)(742.95578849,36.30204911)(742.97578857,36.16205373)
\curveto(743.00578844,36.03204938)(743.03578841,35.89704951)(743.06578857,35.75705373)
\curveto(743.08578836,35.7070497)(743.09078835,35.65704975)(743.08078857,35.60705373)
\curveto(743.07078837,35.55704985)(743.07078837,35.50204991)(743.08078857,35.44205373)
\curveto(743.09078835,35.42204999)(743.09078835,35.39705001)(743.08078857,35.36705373)
\curveto(743.08078836,35.33705007)(743.08578836,35.3120501)(743.09578857,35.29205373)
\curveto(743.10578834,35.25205016)(743.11078833,35.19705021)(743.11078857,35.12705373)
\curveto(743.11078833,35.05705035)(743.10578834,35.00205041)(743.09578857,34.96205373)
\curveto(743.08578836,34.9120505)(743.08578836,34.85705055)(743.09578857,34.79705373)
\curveto(743.10578834,34.73705067)(743.10078834,34.68205073)(743.08078857,34.63205373)
\curveto(743.05078839,34.50205091)(743.03078841,34.37705103)(743.02078857,34.25705373)
\curveto(743.01078843,34.13705127)(742.98578846,34.02205139)(742.94578857,33.91205373)
\curveto(742.82578862,33.54205187)(742.65578879,33.22205219)(742.43578857,32.95205373)
\curveto(742.21578923,32.68205273)(741.93578951,32.47205294)(741.59578857,32.32205373)
\curveto(741.47578997,32.27205314)(741.35079009,32.22705318)(741.22078857,32.18705373)
\curveto(741.09079035,32.15705325)(740.95579049,32.12205329)(740.81578857,32.08205373)
\curveto(740.76579068,32.07205334)(740.72579072,32.06705334)(740.69578857,32.06705373)
\curveto(740.65579079,32.06705334)(740.61079083,32.06205335)(740.56078857,32.05205373)
\curveto(740.53079091,32.04205337)(740.49579095,32.03705337)(740.45578857,32.03705373)
\curveto(740.40579104,32.03705337)(740.36579108,32.03205338)(740.33578857,32.02205373)
\lineto(740.17078857,32.02205373)
\curveto(740.09079135,32.00205341)(739.99079145,31.99705341)(739.87078857,32.00705373)
\curveto(739.7407917,32.01705339)(739.65079179,32.03205338)(739.60078857,32.05205373)
\curveto(739.51079193,32.07205334)(739.445792,32.12705328)(739.40578857,32.21705373)
\curveto(739.38579206,32.24705316)(739.38079206,32.27705313)(739.39078857,32.30705373)
\curveto(739.39079205,32.33705307)(739.38579206,32.37705303)(739.37578857,32.42705373)
\curveto(739.36579208,32.46705294)(739.36079208,32.5070529)(739.36078857,32.54705373)
\lineto(739.36078857,32.69705373)
\curveto(739.36079208,32.81705259)(739.36579208,32.93705247)(739.37578857,33.05705373)
\curveto(739.37579207,33.18705222)(739.41079203,33.27705213)(739.48078857,33.32705373)
\curveto(739.5407919,33.36705204)(739.60079184,33.38705202)(739.66078857,33.38705373)
\curveto(739.72079172,33.38705202)(739.79079165,33.39705201)(739.87078857,33.41705373)
\curveto(739.90079154,33.42705198)(739.93579151,33.42705198)(739.97578857,33.41705373)
\curveto(740.00579144,33.41705199)(740.03079141,33.42205199)(740.05078857,33.43205373)
\lineto(740.26078857,33.43205373)
\curveto(740.31079113,33.45205196)(740.36079108,33.45705195)(740.41078857,33.44705373)
\curveto(740.45079099,33.44705196)(740.49579095,33.45705195)(740.54578857,33.47705373)
\curveto(740.67579077,33.5070519)(740.80079064,33.53705187)(740.92078857,33.56705373)
\curveto(741.03079041,33.59705181)(741.13579031,33.64205177)(741.23578857,33.70205373)
\curveto(741.52578992,33.87205154)(741.73078971,34.14205127)(741.85078857,34.51205373)
\curveto(741.87078957,34.56205085)(741.88578956,34.6120508)(741.89578857,34.66205373)
\curveto(741.89578955,34.72205069)(741.90578954,34.77705063)(741.92578857,34.82705373)
\lineto(741.92578857,34.90205373)
\curveto(741.93578951,34.97205044)(741.9457895,35.06705034)(741.95578857,35.18705373)
\curveto(741.95578949,35.31705009)(741.9457895,35.41704999)(741.92578857,35.48705373)
\curveto(741.90578954,35.55704985)(741.89078955,35.62704978)(741.88078857,35.69705373)
\curveto(741.86078958,35.77704963)(741.8407896,35.84704956)(741.82078857,35.90705373)
\curveto(741.66078978,36.28704912)(741.38579006,36.56204885)(740.99578857,36.73205373)
\curveto(740.86579058,36.78204863)(740.71079073,36.81704859)(740.53078857,36.83705373)
\curveto(740.35079109,36.86704854)(740.16579128,36.88204853)(739.97578857,36.88205373)
\curveto(739.77579167,36.89204852)(739.57579187,36.89204852)(739.37578857,36.88205373)
\lineto(738.80578857,36.88205373)
\lineto(734.56078857,36.88205373)
\lineto(733.01578857,36.88205373)
\curveto(732.90579854,36.88204853)(732.78579866,36.87704853)(732.65578857,36.86705373)
\curveto(732.52579892,36.85704855)(732.42079902,36.87704853)(732.34078857,36.92705373)
\curveto(732.27079917,36.98704842)(732.22079922,37.06704834)(732.19078857,37.16705373)
\curveto(732.19079925,37.18704822)(732.19079925,37.2070482)(732.19078857,37.22705373)
\curveto(732.19079925,37.24704816)(732.18579926,37.26704814)(732.17578857,37.28705373)
}
}
{
\newrgbcolor{curcolor}{0 0 0}
\pscustom[linestyle=none,fillstyle=solid,fillcolor=curcolor]
{
\newpath
\moveto(735.13078857,40.82072561)
\lineto(735.13078857,41.25572561)
\curveto(735.13079631,41.40572364)(735.17079627,41.51072354)(735.25078857,41.57072561)
\curveto(735.33079611,41.62072343)(735.43079601,41.6457234)(735.55078857,41.64572561)
\curveto(735.67079577,41.65572339)(735.79079565,41.66072339)(735.91078857,41.66072561)
\lineto(737.33578857,41.66072561)
\lineto(739.60078857,41.66072561)
\lineto(740.29078857,41.66072561)
\curveto(740.52079092,41.66072339)(740.72079072,41.68572336)(740.89078857,41.73572561)
\curveto(741.3407901,41.89572315)(741.65578979,42.19572285)(741.83578857,42.63572561)
\curveto(741.92578952,42.85572219)(741.96078948,43.12072193)(741.94078857,43.43072561)
\curveto(741.91078953,43.74072131)(741.85578959,43.99072106)(741.77578857,44.18072561)
\curveto(741.63578981,44.51072054)(741.46078998,44.77072028)(741.25078857,44.96072561)
\curveto(741.03079041,45.16071989)(740.7457907,45.31571973)(740.39578857,45.42572561)
\curveto(740.31579113,45.45571959)(740.23579121,45.47571957)(740.15578857,45.48572561)
\curveto(740.07579137,45.49571955)(739.99079145,45.51071954)(739.90078857,45.53072561)
\curveto(739.85079159,45.54071951)(739.80579164,45.54071951)(739.76578857,45.53072561)
\curveto(739.72579172,45.53071952)(739.68079176,45.54071951)(739.63078857,45.56072561)
\lineto(739.31578857,45.56072561)
\curveto(739.23579221,45.58071947)(739.1457923,45.58571946)(739.04578857,45.57572561)
\curveto(738.93579251,45.56571948)(738.83579261,45.56071949)(738.74578857,45.56072561)
\lineto(737.57578857,45.56072561)
\lineto(735.98578857,45.56072561)
\curveto(735.86579558,45.56071949)(735.7407957,45.55571949)(735.61078857,45.54572561)
\curveto(735.47079597,45.5457195)(735.36079608,45.57071948)(735.28078857,45.62072561)
\curveto(735.23079621,45.66071939)(735.20079624,45.70571934)(735.19078857,45.75572561)
\curveto(735.17079627,45.81571923)(735.15079629,45.88571916)(735.13078857,45.96572561)
\lineto(735.13078857,46.19072561)
\curveto(735.13079631,46.31071874)(735.13579631,46.41571863)(735.14578857,46.50572561)
\curveto(735.15579629,46.60571844)(735.20079624,46.68071837)(735.28078857,46.73072561)
\curveto(735.33079611,46.78071827)(735.40579604,46.80571824)(735.50578857,46.80572561)
\lineto(735.79078857,46.80572561)
\lineto(736.81078857,46.80572561)
\lineto(740.84578857,46.80572561)
\lineto(742.19578857,46.80572561)
\curveto(742.31578913,46.80571824)(742.43078901,46.80071825)(742.54078857,46.79072561)
\curveto(742.6407888,46.79071826)(742.71578873,46.75571829)(742.76578857,46.68572561)
\curveto(742.79578865,46.6457184)(742.82078862,46.58571846)(742.84078857,46.50572561)
\curveto(742.85078859,46.42571862)(742.86078858,46.33571871)(742.87078857,46.23572561)
\curveto(742.87078857,46.1457189)(742.86578858,46.05571899)(742.85578857,45.96572561)
\curveto(742.8457886,45.88571916)(742.82578862,45.82571922)(742.79578857,45.78572561)
\curveto(742.75578869,45.73571931)(742.69078875,45.69071936)(742.60078857,45.65072561)
\curveto(742.56078888,45.64071941)(742.50578894,45.63071942)(742.43578857,45.62072561)
\curveto(742.36578908,45.62071943)(742.30078914,45.61571943)(742.24078857,45.60572561)
\curveto(742.17078927,45.59571945)(742.11578933,45.57571947)(742.07578857,45.54572561)
\curveto(742.03578941,45.51571953)(742.02078942,45.47071958)(742.03078857,45.41072561)
\curveto(742.05078939,45.33071972)(742.11078933,45.2507198)(742.21078857,45.17072561)
\curveto(742.30078914,45.09071996)(742.37078907,45.01572003)(742.42078857,44.94572561)
\curveto(742.58078886,44.72572032)(742.72078872,44.47572057)(742.84078857,44.19572561)
\curveto(742.89078855,44.08572096)(742.92078852,43.97072108)(742.93078857,43.85072561)
\curveto(742.95078849,43.74072131)(742.97578847,43.62572142)(743.00578857,43.50572561)
\curveto(743.01578843,43.45572159)(743.01578843,43.40072165)(743.00578857,43.34072561)
\curveto(742.99578845,43.29072176)(743.00078844,43.24072181)(743.02078857,43.19072561)
\curveto(743.0407884,43.09072196)(743.0407884,43.00072205)(743.02078857,42.92072561)
\lineto(743.02078857,42.77072561)
\curveto(743.00078844,42.72072233)(742.99078845,42.66072239)(742.99078857,42.59072561)
\curveto(742.99078845,42.53072252)(742.98578846,42.47572257)(742.97578857,42.42572561)
\curveto(742.95578849,42.38572266)(742.9457885,42.3457227)(742.94578857,42.30572561)
\curveto(742.95578849,42.27572277)(742.95078849,42.23572281)(742.93078857,42.18572561)
\lineto(742.87078857,41.94572561)
\curveto(742.85078859,41.87572317)(742.82078862,41.80072325)(742.78078857,41.72072561)
\curveto(742.67078877,41.46072359)(742.52578892,41.24072381)(742.34578857,41.06072561)
\curveto(742.15578929,40.89072416)(741.93078951,40.7507243)(741.67078857,40.64072561)
\curveto(741.58078986,40.60072445)(741.49078995,40.57072448)(741.40078857,40.55072561)
\lineto(741.10078857,40.49072561)
\curveto(741.0407904,40.47072458)(740.98579046,40.46072459)(740.93578857,40.46072561)
\curveto(740.87579057,40.47072458)(740.81079063,40.46572458)(740.74078857,40.44572561)
\curveto(740.72079072,40.43572461)(740.69579075,40.43072462)(740.66578857,40.43072561)
\curveto(740.62579082,40.43072462)(740.59079085,40.42572462)(740.56078857,40.41572561)
\lineto(740.41078857,40.41572561)
\curveto(740.37079107,40.40572464)(740.32579112,40.40072465)(740.27578857,40.40072561)
\curveto(740.21579123,40.41072464)(740.16079128,40.41572463)(740.11078857,40.41572561)
\lineto(739.51078857,40.41572561)
\lineto(736.75078857,40.41572561)
\lineto(735.79078857,40.41572561)
\lineto(735.52078857,40.41572561)
\curveto(735.43079601,40.41572463)(735.35579609,40.43572461)(735.29578857,40.47572561)
\curveto(735.22579622,40.51572453)(735.17579627,40.59072446)(735.14578857,40.70072561)
\curveto(735.13579631,40.72072433)(735.13579631,40.74072431)(735.14578857,40.76072561)
\curveto(735.1457963,40.78072427)(735.1407963,40.80072425)(735.13078857,40.82072561)
}
}
{
\newrgbcolor{curcolor}{0 0 0}
\pscustom[linestyle=none,fillstyle=solid,fillcolor=curcolor]
{
\newpath
\moveto(734.98078857,52.39533498)
\curveto(734.96079648,53.02532975)(735.0457964,53.53032924)(735.23578857,53.91033498)
\curveto(735.42579602,54.29032848)(735.71079573,54.59532818)(736.09078857,54.82533498)
\curveto(736.19079525,54.88532789)(736.30079514,54.93032784)(736.42078857,54.96033498)
\curveto(736.53079491,55.00032777)(736.6457948,55.03532774)(736.76578857,55.06533498)
\curveto(736.95579449,55.11532766)(737.16079428,55.14532763)(737.38078857,55.15533498)
\curveto(737.60079384,55.16532761)(737.82579362,55.1703276)(738.05578857,55.17033498)
\lineto(739.66078857,55.17033498)
\lineto(742.00078857,55.17033498)
\curveto(742.17078927,55.1703276)(742.3407891,55.16532761)(742.51078857,55.15533498)
\curveto(742.68078876,55.15532762)(742.79078865,55.09032768)(742.84078857,54.96033498)
\curveto(742.86078858,54.91032786)(742.87078857,54.85532792)(742.87078857,54.79533498)
\curveto(742.88078856,54.74532803)(742.88578856,54.69032808)(742.88578857,54.63033498)
\curveto(742.88578856,54.50032827)(742.88078856,54.3753284)(742.87078857,54.25533498)
\curveto(742.87078857,54.13532864)(742.83078861,54.05032872)(742.75078857,54.00033498)
\curveto(742.68078876,53.95032882)(742.59078885,53.92532885)(742.48078857,53.92533498)
\lineto(742.15078857,53.92533498)
\lineto(740.86078857,53.92533498)
\lineto(738.41578857,53.92533498)
\curveto(738.1457933,53.92532885)(737.88079356,53.92032885)(737.62078857,53.91033498)
\curveto(737.35079409,53.90032887)(737.12079432,53.85532892)(736.93078857,53.77533498)
\curveto(736.73079471,53.69532908)(736.57079487,53.5753292)(736.45078857,53.41533498)
\curveto(736.32079512,53.25532952)(736.22079522,53.0703297)(736.15078857,52.86033498)
\curveto(736.13079531,52.80032997)(736.12079532,52.73533004)(736.12078857,52.66533498)
\curveto(736.11079533,52.60533017)(736.09579535,52.54533023)(736.07578857,52.48533498)
\curveto(736.06579538,52.43533034)(736.06579538,52.35533042)(736.07578857,52.24533498)
\curveto(736.07579537,52.14533063)(736.08079536,52.0753307)(736.09078857,52.03533498)
\curveto(736.11079533,51.99533078)(736.12079532,51.96033081)(736.12078857,51.93033498)
\curveto(736.11079533,51.90033087)(736.11079533,51.86533091)(736.12078857,51.82533498)
\curveto(736.15079529,51.69533108)(736.18579526,51.5703312)(736.22578857,51.45033498)
\curveto(736.25579519,51.34033143)(736.30079514,51.23533154)(736.36078857,51.13533498)
\curveto(736.38079506,51.09533168)(736.40079504,51.06033171)(736.42078857,51.03033498)
\curveto(736.440795,51.00033177)(736.46079498,50.96533181)(736.48078857,50.92533498)
\curveto(736.73079471,50.5753322)(737.10579434,50.32033245)(737.60578857,50.16033498)
\curveto(737.68579376,50.13033264)(737.77079367,50.11033266)(737.86078857,50.10033498)
\curveto(737.9407935,50.09033268)(738.02079342,50.0753327)(738.10078857,50.05533498)
\curveto(738.15079329,50.03533274)(738.20079324,50.03033274)(738.25078857,50.04033498)
\curveto(738.29079315,50.05033272)(738.33079311,50.04533273)(738.37078857,50.02533498)
\lineto(738.68578857,50.02533498)
\curveto(738.71579273,50.01533276)(738.75079269,50.01033276)(738.79078857,50.01033498)
\curveto(738.83079261,50.02033275)(738.87579257,50.02533275)(738.92578857,50.02533498)
\lineto(739.37578857,50.02533498)
\lineto(740.81578857,50.02533498)
\lineto(742.13578857,50.02533498)
\lineto(742.48078857,50.02533498)
\curveto(742.59078885,50.02533275)(742.68078876,50.00033277)(742.75078857,49.95033498)
\curveto(742.83078861,49.90033287)(742.87078857,49.81033296)(742.87078857,49.68033498)
\curveto(742.88078856,49.56033321)(742.88578856,49.43533334)(742.88578857,49.30533498)
\curveto(742.88578856,49.22533355)(742.88078856,49.15033362)(742.87078857,49.08033498)
\curveto(742.86078858,49.01033376)(742.83578861,48.95033382)(742.79578857,48.90033498)
\curveto(742.7457887,48.82033395)(742.65078879,48.78033399)(742.51078857,48.78033498)
\lineto(742.10578857,48.78033498)
\lineto(740.33578857,48.78033498)
\lineto(736.70578857,48.78033498)
\lineto(735.79078857,48.78033498)
\lineto(735.52078857,48.78033498)
\curveto(735.43079601,48.78033399)(735.36079608,48.80033397)(735.31078857,48.84033498)
\curveto(735.25079619,48.8703339)(735.21079623,48.92033385)(735.19078857,48.99033498)
\curveto(735.18079626,49.03033374)(735.17079627,49.08533369)(735.16078857,49.15533498)
\curveto(735.15079629,49.23533354)(735.1457963,49.31533346)(735.14578857,49.39533498)
\curveto(735.1457963,49.4753333)(735.15079629,49.55033322)(735.16078857,49.62033498)
\curveto(735.17079627,49.70033307)(735.18579626,49.75533302)(735.20578857,49.78533498)
\curveto(735.27579617,49.89533288)(735.36579608,49.94533283)(735.47578857,49.93533498)
\curveto(735.57579587,49.92533285)(735.69079575,49.94033283)(735.82078857,49.98033498)
\curveto(735.88079556,50.00033277)(735.93079551,50.04033273)(735.97078857,50.10033498)
\curveto(735.98079546,50.22033255)(735.93579551,50.31533246)(735.83578857,50.38533498)
\curveto(735.73579571,50.46533231)(735.65579579,50.54533223)(735.59578857,50.62533498)
\curveto(735.49579595,50.76533201)(735.40579604,50.90533187)(735.32578857,51.04533498)
\curveto(735.23579621,51.19533158)(735.16079628,51.36533141)(735.10078857,51.55533498)
\curveto(735.07079637,51.63533114)(735.05079639,51.72033105)(735.04078857,51.81033498)
\curveto(735.03079641,51.91033086)(735.01579643,52.00533077)(734.99578857,52.09533498)
\curveto(734.98579646,52.14533063)(734.98079646,52.19533058)(734.98078857,52.24533498)
\lineto(734.98078857,52.39533498)
}
}
{
\newrgbcolor{curcolor}{0 0 0}
\pscustom[linestyle=none,fillstyle=solid,fillcolor=curcolor]
{
}
}
{
\newrgbcolor{curcolor}{0 0 0}
\pscustom[linestyle=none,fillstyle=solid,fillcolor=curcolor]
{
\newpath
\moveto(732.25078857,65.05510061)
\curveto(732.25079919,65.15509575)(732.26079918,65.25009566)(732.28078857,65.34010061)
\curveto(732.29079915,65.43009548)(732.32079912,65.49509541)(732.37078857,65.53510061)
\curveto(732.45079899,65.59509531)(732.55579889,65.62509528)(732.68578857,65.62510061)
\lineto(733.07578857,65.62510061)
\lineto(734.57578857,65.62510061)
\lineto(740.96578857,65.62510061)
\lineto(742.13578857,65.62510061)
\lineto(742.45078857,65.62510061)
\curveto(742.55078889,65.63509527)(742.63078881,65.62009529)(742.69078857,65.58010061)
\curveto(742.77078867,65.53009538)(742.82078862,65.45509545)(742.84078857,65.35510061)
\curveto(742.85078859,65.26509564)(742.85578859,65.15509575)(742.85578857,65.02510061)
\lineto(742.85578857,64.80010061)
\curveto(742.83578861,64.72009619)(742.82078862,64.65009626)(742.81078857,64.59010061)
\curveto(742.79078865,64.53009638)(742.75078869,64.48009643)(742.69078857,64.44010061)
\curveto(742.63078881,64.40009651)(742.55578889,64.38009653)(742.46578857,64.38010061)
\lineto(742.16578857,64.38010061)
\lineto(741.07078857,64.38010061)
\lineto(735.73078857,64.38010061)
\curveto(735.6407958,64.36009655)(735.56579588,64.34509656)(735.50578857,64.33510061)
\curveto(735.43579601,64.33509657)(735.37579607,64.3050966)(735.32578857,64.24510061)
\curveto(735.27579617,64.17509673)(735.25079619,64.08509682)(735.25078857,63.97510061)
\curveto(735.2407962,63.87509703)(735.23579621,63.76509714)(735.23578857,63.64510061)
\lineto(735.23578857,62.50510061)
\lineto(735.23578857,62.01010061)
\curveto(735.22579622,61.85009906)(735.16579628,61.74009917)(735.05578857,61.68010061)
\curveto(735.02579642,61.66009925)(734.99579645,61.65009926)(734.96578857,61.65010061)
\curveto(734.92579652,61.65009926)(734.88079656,61.64509926)(734.83078857,61.63510061)
\curveto(734.71079673,61.61509929)(734.60079684,61.62009929)(734.50078857,61.65010061)
\curveto(734.40079704,61.69009922)(734.33079711,61.74509916)(734.29078857,61.81510061)
\curveto(734.2407972,61.89509901)(734.21579723,62.01509889)(734.21578857,62.17510061)
\curveto(734.21579723,62.33509857)(734.20079724,62.47009844)(734.17078857,62.58010061)
\curveto(734.16079728,62.63009828)(734.15579729,62.68509822)(734.15578857,62.74510061)
\curveto(734.1457973,62.8050981)(734.13079731,62.86509804)(734.11078857,62.92510061)
\curveto(734.06079738,63.07509783)(734.01079743,63.22009769)(733.96078857,63.36010061)
\curveto(733.90079754,63.50009741)(733.83079761,63.63509727)(733.75078857,63.76510061)
\curveto(733.66079778,63.905097)(733.55579789,64.02509688)(733.43578857,64.12510061)
\curveto(733.31579813,64.22509668)(733.18579826,64.32009659)(733.04578857,64.41010061)
\curveto(732.9457985,64.47009644)(732.83579861,64.51509639)(732.71578857,64.54510061)
\curveto(732.59579885,64.58509632)(732.49079895,64.63509627)(732.40078857,64.69510061)
\curveto(732.3407991,64.74509616)(732.30079914,64.81509609)(732.28078857,64.90510061)
\curveto(732.27079917,64.92509598)(732.26579918,64.95009596)(732.26578857,64.98010061)
\curveto(732.26579918,65.0100959)(732.26079918,65.03509587)(732.25078857,65.05510061)
}
}
{
\newrgbcolor{curcolor}{0 0 0}
\pscustom[linestyle=none,fillstyle=solid,fillcolor=curcolor]
{
\newpath
\moveto(739.78078857,76.35970998)
\curveto(739.82079162,76.36970226)(739.87079157,76.36970226)(739.93078857,76.35970998)
\curveto(739.99079145,76.35970227)(740.0407914,76.35470228)(740.08078857,76.34470998)
\curveto(740.12079132,76.34470229)(740.16079128,76.33970229)(740.20078857,76.32970998)
\lineto(740.30578857,76.32970998)
\curveto(740.38579106,76.30970232)(740.46579098,76.29470234)(740.54578857,76.28470998)
\curveto(740.62579082,76.27470236)(740.70079074,76.25470238)(740.77078857,76.22470998)
\curveto(740.85079059,76.20470243)(740.92579052,76.18470245)(740.99578857,76.16470998)
\curveto(741.06579038,76.14470249)(741.1407903,76.11470252)(741.22078857,76.07470998)
\curveto(741.6407898,75.89470274)(741.98078946,75.63970299)(742.24078857,75.30970998)
\curveto(742.50078894,74.97970365)(742.70578874,74.58970404)(742.85578857,74.13970998)
\curveto(742.89578855,74.01970461)(742.92078852,73.89470474)(742.93078857,73.76470998)
\curveto(742.95078849,73.64470499)(742.97578847,73.51970511)(743.00578857,73.38970998)
\curveto(743.01578843,73.3297053)(743.02078842,73.26470537)(743.02078857,73.19470998)
\curveto(743.02078842,73.1347055)(743.02578842,73.06970556)(743.03578857,72.99970998)
\lineto(743.03578857,72.87970998)
\lineto(743.03578857,72.68470998)
\curveto(743.0457884,72.62470601)(743.0407884,72.56970606)(743.02078857,72.51970998)
\curveto(743.00078844,72.44970618)(742.99578845,72.38470625)(743.00578857,72.32470998)
\curveto(743.01578843,72.26470637)(743.01078843,72.20470643)(742.99078857,72.14470998)
\curveto(742.98078846,72.09470654)(742.97578847,72.04970658)(742.97578857,72.00970998)
\curveto(742.97578847,71.96970666)(742.96578848,71.92470671)(742.94578857,71.87470998)
\curveto(742.92578852,71.79470684)(742.90578854,71.71970691)(742.88578857,71.64970998)
\curveto(742.87578857,71.57970705)(742.86078858,71.50970712)(742.84078857,71.43970998)
\curveto(742.67078877,70.95970767)(742.46078898,70.55970807)(742.21078857,70.23970998)
\curveto(741.95078949,69.9297087)(741.59578985,69.67970895)(741.14578857,69.48970998)
\curveto(741.08579036,69.45970917)(741.02579042,69.4347092)(740.96578857,69.41470998)
\curveto(740.89579055,69.40470923)(740.82079062,69.38970924)(740.74078857,69.36970998)
\curveto(740.68079076,69.34970928)(740.61579083,69.3347093)(740.54578857,69.32470998)
\curveto(740.47579097,69.31470932)(740.40579104,69.29970933)(740.33578857,69.27970998)
\curveto(740.28579116,69.26970936)(740.2457912,69.26470937)(740.21578857,69.26470998)
\lineto(740.09578857,69.26470998)
\curveto(740.05579139,69.25470938)(740.00579144,69.24470939)(739.94578857,69.23470998)
\curveto(739.88579156,69.2347094)(739.83579161,69.23970939)(739.79578857,69.24970998)
\lineto(739.66078857,69.24970998)
\curveto(739.61079183,69.25970937)(739.56079188,69.26470937)(739.51078857,69.26470998)
\curveto(739.41079203,69.28470935)(739.31579213,69.29970933)(739.22578857,69.30970998)
\curveto(739.12579232,69.31970931)(739.03079241,69.33970929)(738.94078857,69.36970998)
\curveto(738.79079265,69.41970921)(738.65079279,69.47470916)(738.52078857,69.53470998)
\curveto(738.39079305,69.59470904)(738.27079317,69.66470897)(738.16078857,69.74470998)
\curveto(738.11079333,69.77470886)(738.07079337,69.80470883)(738.04078857,69.83470998)
\curveto(738.01079343,69.87470876)(737.97579347,69.90970872)(737.93578857,69.93970998)
\curveto(737.85579359,69.99970863)(737.78579366,70.06970856)(737.72578857,70.14970998)
\curveto(737.67579377,70.20970842)(737.63079381,70.26970836)(737.59078857,70.32970998)
\lineto(737.44078857,70.53970998)
\curveto(737.40079404,70.58970804)(737.36579408,70.63970799)(737.33578857,70.68970998)
\curveto(737.29579415,70.73970789)(737.2407942,70.77470786)(737.17078857,70.79470998)
\curveto(737.1407943,70.79470784)(737.11579433,70.78470785)(737.09578857,70.76470998)
\curveto(737.06579438,70.75470788)(737.0407944,70.74470789)(737.02078857,70.73470998)
\curveto(736.97079447,70.69470794)(736.92579452,70.64470799)(736.88578857,70.58470998)
\curveto(736.83579461,70.5347081)(736.79079465,70.48470815)(736.75078857,70.43470998)
\curveto(736.72079472,70.39470824)(736.66579478,70.34470829)(736.58578857,70.28470998)
\curveto(736.55579489,70.26470837)(736.53079491,70.2347084)(736.51078857,70.19470998)
\curveto(736.48079496,70.16470847)(736.445795,70.13970849)(736.40578857,70.11970998)
\curveto(736.19579525,69.94970868)(735.95079549,69.81970881)(735.67078857,69.72970998)
\curveto(735.59079585,69.70970892)(735.51079593,69.69470894)(735.43078857,69.68470998)
\curveto(735.35079609,69.67470896)(735.27079617,69.65970897)(735.19078857,69.63970998)
\curveto(735.1407963,69.61970901)(735.07579637,69.60970902)(734.99578857,69.60970998)
\curveto(734.90579654,69.60970902)(734.83579661,69.61970901)(734.78578857,69.63970998)
\curveto(734.68579676,69.63970899)(734.61579683,69.64470899)(734.57578857,69.65470998)
\curveto(734.49579695,69.67470896)(734.42579702,69.68970894)(734.36578857,69.69970998)
\curveto(734.29579715,69.70970892)(734.22579722,69.72470891)(734.15578857,69.74470998)
\curveto(733.72579772,69.89470874)(733.38079806,70.10970852)(733.12078857,70.38970998)
\curveto(732.86079858,70.67970795)(732.6457988,71.0297076)(732.47578857,71.43970998)
\curveto(732.42579902,71.54970708)(732.39579905,71.66470697)(732.38578857,71.78470998)
\curveto(732.36579908,71.91470672)(732.33579911,72.04470659)(732.29578857,72.17470998)
\curveto(732.29579915,72.25470638)(732.29579915,72.32470631)(732.29578857,72.38470998)
\curveto(732.28579916,72.45470618)(732.27579917,72.5297061)(732.26578857,72.60970998)
\curveto(732.2457992,73.39970523)(732.37579907,74.05470458)(732.65578857,74.57470998)
\curveto(732.93579851,75.10470353)(733.3457981,75.48470315)(733.88578857,75.71470998)
\curveto(734.11579733,75.82470281)(734.40079704,75.89470274)(734.74078857,75.92470998)
\curveto(735.07079637,75.96470267)(735.37579607,75.9347027)(735.65578857,75.83470998)
\curveto(735.78579566,75.79470284)(735.90579554,75.74470289)(736.01578857,75.68470998)
\curveto(736.12579532,75.634703)(736.23079521,75.57470306)(736.33078857,75.50470998)
\curveto(736.37079507,75.48470315)(736.40579504,75.45470318)(736.43578857,75.41470998)
\lineto(736.52578857,75.32470998)
\curveto(736.61579483,75.27470336)(736.68079476,75.21470342)(736.72078857,75.14470998)
\curveto(736.77079467,75.09470354)(736.82079462,75.03970359)(736.87078857,74.97970998)
\curveto(736.91079453,74.9297037)(736.95579449,74.88470375)(737.00578857,74.84470998)
\curveto(737.02579442,74.82470381)(737.05079439,74.80470383)(737.08078857,74.78470998)
\curveto(737.10079434,74.77470386)(737.12579432,74.77470386)(737.15578857,74.78470998)
\curveto(737.20579424,74.79470384)(737.25579419,74.82470381)(737.30578857,74.87470998)
\curveto(737.3457941,74.92470371)(737.38579406,74.97970365)(737.42578857,75.03970998)
\lineto(737.54578857,75.21970998)
\curveto(737.57579387,75.27970335)(737.60579384,75.3297033)(737.63578857,75.36970998)
\curveto(737.87579357,75.69970293)(738.18579326,75.94970268)(738.56578857,76.11970998)
\curveto(738.6457928,76.15970247)(738.73079271,76.18970244)(738.82078857,76.20970998)
\curveto(738.91079253,76.23970239)(739.00079244,76.26470237)(739.09078857,76.28470998)
\curveto(739.1407923,76.29470234)(739.19579225,76.30470233)(739.25578857,76.31470998)
\lineto(739.40578857,76.34470998)
\curveto(739.46579198,76.35470228)(739.53079191,76.35470228)(739.60078857,76.34470998)
\curveto(739.66079178,76.3347023)(739.72079172,76.33970229)(739.78078857,76.35970998)
\moveto(734.74078857,70.97470998)
\curveto(734.85079659,70.94470769)(734.99079645,70.93970769)(735.16078857,70.95970998)
\curveto(735.32079612,70.97970765)(735.445796,71.00470763)(735.53578857,71.03470998)
\curveto(735.85579559,71.14470749)(736.10079534,71.29470734)(736.27078857,71.48470998)
\curveto(736.43079501,71.67470696)(736.56079488,71.93970669)(736.66078857,72.27970998)
\curveto(736.69079475,72.40970622)(736.71579473,72.57470606)(736.73578857,72.77470998)
\curveto(736.7457947,72.97470566)(736.73079471,73.14470549)(736.69078857,73.28470998)
\curveto(736.61079483,73.57470506)(736.50079494,73.81470482)(736.36078857,74.00470998)
\curveto(736.21079523,74.20470443)(736.01079543,74.35970427)(735.76078857,74.46970998)
\curveto(735.71079573,74.48970414)(735.66579578,74.49970413)(735.62578857,74.49970998)
\curveto(735.58579586,74.50970412)(735.5407959,74.52470411)(735.49078857,74.54470998)
\curveto(735.38079606,74.57470406)(735.2407962,74.59470404)(735.07078857,74.60470998)
\curveto(734.90079654,74.61470402)(734.75579669,74.60470403)(734.63578857,74.57470998)
\curveto(734.5457969,74.55470408)(734.46079698,74.5297041)(734.38078857,74.49970998)
\curveto(734.30079714,74.47970415)(734.22079722,74.44470419)(734.14078857,74.39470998)
\curveto(733.87079757,74.22470441)(733.67579777,73.99970463)(733.55578857,73.71970998)
\curveto(733.43579801,73.44970518)(733.37579807,73.08970554)(733.37578857,72.63970998)
\curveto(733.39579805,72.61970601)(733.40079804,72.58970604)(733.39078857,72.54970998)
\curveto(733.38079806,72.50970612)(733.38079806,72.47470616)(733.39078857,72.44470998)
\curveto(733.41079803,72.39470624)(733.42579802,72.33970629)(733.43578857,72.27970998)
\curveto(733.43579801,72.2297064)(733.445798,72.17970645)(733.46578857,72.12970998)
\curveto(733.55579789,71.88970674)(733.67079777,71.67970695)(733.81078857,71.49970998)
\curveto(733.9407975,71.31970731)(734.12079732,71.17970745)(734.35078857,71.07970998)
\curveto(734.41079703,71.05970757)(734.47579697,71.03970759)(734.54578857,71.01970998)
\curveto(734.60579684,71.00970762)(734.67079677,70.99470764)(734.74078857,70.97470998)
\moveto(740.27578857,74.99470998)
\curveto(740.08579136,75.04470359)(739.88079156,75.04970358)(739.66078857,75.00970998)
\curveto(739.440792,74.97970365)(739.26079218,74.9347037)(739.12078857,74.87470998)
\curveto(738.75079269,74.70470393)(738.445793,74.44470419)(738.20578857,74.09470998)
\curveto(737.96579348,73.75470488)(737.8457936,73.31970531)(737.84578857,72.78970998)
\curveto(737.86579358,72.75970587)(737.87079357,72.71970591)(737.86078857,72.66970998)
\curveto(737.8407936,72.61970601)(737.83579361,72.57970605)(737.84578857,72.54970998)
\lineto(737.90578857,72.27970998)
\curveto(737.91579353,72.19970643)(737.93079351,72.11970651)(737.95078857,72.03970998)
\curveto(738.06079338,71.73970689)(738.20579324,71.47470716)(738.38578857,71.24470998)
\curveto(738.56579288,71.02470761)(738.79579265,70.85470778)(739.07578857,70.73470998)
\curveto(739.15579229,70.70470793)(739.23579221,70.67970795)(739.31578857,70.65970998)
\curveto(739.39579205,70.63970799)(739.48079196,70.61970801)(739.57078857,70.59970998)
\curveto(739.69079175,70.56970806)(739.8407916,70.55970807)(740.02078857,70.56970998)
\curveto(740.20079124,70.58970804)(740.3407911,70.61470802)(740.44078857,70.64470998)
\curveto(740.49079095,70.66470797)(740.53579091,70.67470796)(740.57578857,70.67470998)
\curveto(740.60579084,70.68470795)(740.6457908,70.69970793)(740.69578857,70.71970998)
\curveto(740.91579053,70.81970781)(741.11579033,70.94970768)(741.29578857,71.10970998)
\curveto(741.47578997,71.27970735)(741.61078983,71.47470716)(741.70078857,71.69470998)
\curveto(741.7407897,71.76470687)(741.77578967,71.85970677)(741.80578857,71.97970998)
\curveto(741.89578955,72.19970643)(741.9407895,72.45470618)(741.94078857,72.74470998)
\lineto(741.94078857,73.02970998)
\curveto(741.92078952,73.1297055)(741.90578954,73.22470541)(741.89578857,73.31470998)
\curveto(741.88578956,73.40470523)(741.86578958,73.49470514)(741.83578857,73.58470998)
\curveto(741.75578969,73.84470479)(741.62578982,74.08470455)(741.44578857,74.30470998)
\curveto(741.25579019,74.5347041)(741.0407904,74.70470393)(740.80078857,74.81470998)
\curveto(740.72079072,74.85470378)(740.6407908,74.88470375)(740.56078857,74.90470998)
\curveto(740.47079097,74.9347037)(740.37579107,74.96470367)(740.27578857,74.99470998)
}
}
{
\newrgbcolor{curcolor}{0 0 0}
\pscustom[linestyle=none,fillstyle=solid,fillcolor=curcolor]
{
\newpath
\moveto(741.22078857,78.63431936)
\lineto(741.22078857,79.26431936)
\lineto(741.22078857,79.45931936)
\curveto(741.22079022,79.52931683)(741.23079021,79.58931677)(741.25078857,79.63931936)
\curveto(741.29079015,79.70931665)(741.33079011,79.7593166)(741.37078857,79.78931936)
\curveto(741.42079002,79.82931653)(741.48578996,79.84931651)(741.56578857,79.84931936)
\curveto(741.6457898,79.8593165)(741.73078971,79.86431649)(741.82078857,79.86431936)
\lineto(742.54078857,79.86431936)
\curveto(743.02078842,79.86431649)(743.43078801,79.80431655)(743.77078857,79.68431936)
\curveto(744.11078733,79.56431679)(744.38578706,79.36931699)(744.59578857,79.09931936)
\curveto(744.6457868,79.02931733)(744.69078675,78.9593174)(744.73078857,78.88931936)
\curveto(744.78078666,78.82931753)(744.82578662,78.7543176)(744.86578857,78.66431936)
\curveto(744.87578657,78.64431771)(744.88578656,78.61431774)(744.89578857,78.57431936)
\curveto(744.91578653,78.53431782)(744.92078652,78.48931787)(744.91078857,78.43931936)
\curveto(744.88078656,78.34931801)(744.80578664,78.29431806)(744.68578857,78.27431936)
\curveto(744.57578687,78.2543181)(744.48078696,78.26931809)(744.40078857,78.31931936)
\curveto(744.33078711,78.34931801)(744.26578718,78.39431796)(744.20578857,78.45431936)
\curveto(744.15578729,78.52431783)(744.10578734,78.58931777)(744.05578857,78.64931936)
\curveto(744.00578744,78.71931764)(743.93078751,78.77931758)(743.83078857,78.82931936)
\curveto(743.7407877,78.88931747)(743.65078779,78.93931742)(743.56078857,78.97931936)
\curveto(743.53078791,78.99931736)(743.47078797,79.02431733)(743.38078857,79.05431936)
\curveto(743.30078814,79.08431727)(743.23078821,79.08931727)(743.17078857,79.06931936)
\curveto(743.03078841,79.03931732)(742.9407885,78.97931738)(742.90078857,78.88931936)
\curveto(742.87078857,78.80931755)(742.85578859,78.71931764)(742.85578857,78.61931936)
\curveto(742.85578859,78.51931784)(742.83078861,78.43431792)(742.78078857,78.36431936)
\curveto(742.71078873,78.27431808)(742.57078887,78.22931813)(742.36078857,78.22931936)
\lineto(741.80578857,78.22931936)
\lineto(741.58078857,78.22931936)
\curveto(741.50078994,78.23931812)(741.43579001,78.2593181)(741.38578857,78.28931936)
\curveto(741.30579014,78.34931801)(741.26079018,78.41931794)(741.25078857,78.49931936)
\curveto(741.2407902,78.51931784)(741.23579021,78.53931782)(741.23578857,78.55931936)
\curveto(741.23579021,78.58931777)(741.23079021,78.61431774)(741.22078857,78.63431936)
}
}
{
\newrgbcolor{curcolor}{0 0 0}
\pscustom[linestyle=none,fillstyle=solid,fillcolor=curcolor]
{
}
}
{
\newrgbcolor{curcolor}{0 0 0}
\pscustom[linestyle=none,fillstyle=solid,fillcolor=curcolor]
{
\newpath
\moveto(732.25078857,89.26463186)
\curveto(732.2407992,89.95462722)(732.36079908,90.55462662)(732.61078857,91.06463186)
\curveto(732.86079858,91.58462559)(733.19579825,91.9796252)(733.61578857,92.24963186)
\curveto(733.69579775,92.29962488)(733.78579766,92.34462483)(733.88578857,92.38463186)
\curveto(733.97579747,92.42462475)(734.07079737,92.46962471)(734.17078857,92.51963186)
\curveto(734.27079717,92.55962462)(734.37079707,92.58962459)(734.47078857,92.60963186)
\curveto(734.57079687,92.62962455)(734.67579677,92.64962453)(734.78578857,92.66963186)
\curveto(734.83579661,92.68962449)(734.88079656,92.69462448)(734.92078857,92.68463186)
\curveto(734.96079648,92.6746245)(735.00579644,92.6796245)(735.05578857,92.69963186)
\curveto(735.10579634,92.70962447)(735.19079625,92.71462446)(735.31078857,92.71463186)
\curveto(735.42079602,92.71462446)(735.50579594,92.70962447)(735.56578857,92.69963186)
\curveto(735.62579582,92.6796245)(735.68579576,92.66962451)(735.74578857,92.66963186)
\curveto(735.80579564,92.6796245)(735.86579558,92.6746245)(735.92578857,92.65463186)
\curveto(736.06579538,92.61462456)(736.20079524,92.5796246)(736.33078857,92.54963186)
\curveto(736.46079498,92.51962466)(736.58579486,92.4796247)(736.70578857,92.42963186)
\curveto(736.8457946,92.36962481)(736.97079447,92.29962488)(737.08078857,92.21963186)
\curveto(737.19079425,92.14962503)(737.30079414,92.0746251)(737.41078857,91.99463186)
\lineto(737.47078857,91.93463186)
\curveto(737.49079395,91.92462525)(737.51079393,91.90962527)(737.53078857,91.88963186)
\curveto(737.69079375,91.76962541)(737.83579361,91.63462554)(737.96578857,91.48463186)
\curveto(738.09579335,91.33462584)(738.22079322,91.174626)(738.34078857,91.00463186)
\curveto(738.56079288,90.69462648)(738.76579268,90.39962678)(738.95578857,90.11963186)
\curveto(739.09579235,89.88962729)(739.23079221,89.65962752)(739.36078857,89.42963186)
\curveto(739.49079195,89.20962797)(739.62579182,88.98962819)(739.76578857,88.76963186)
\curveto(739.93579151,88.51962866)(740.11579133,88.2796289)(740.30578857,88.04963186)
\curveto(740.49579095,87.82962935)(740.72079072,87.63962954)(740.98078857,87.47963186)
\curveto(741.0407904,87.43962974)(741.10079034,87.40462977)(741.16078857,87.37463186)
\curveto(741.21079023,87.34462983)(741.27579017,87.31462986)(741.35578857,87.28463186)
\curveto(741.42579002,87.26462991)(741.48578996,87.25962992)(741.53578857,87.26963186)
\curveto(741.60578984,87.28962989)(741.66078978,87.32462985)(741.70078857,87.37463186)
\curveto(741.73078971,87.42462975)(741.75078969,87.48462969)(741.76078857,87.55463186)
\lineto(741.76078857,87.79463186)
\lineto(741.76078857,88.54463186)
\lineto(741.76078857,91.34963186)
\lineto(741.76078857,92.00963186)
\curveto(741.76078968,92.09962508)(741.76578968,92.18462499)(741.77578857,92.26463186)
\curveto(741.77578967,92.34462483)(741.79578965,92.40962477)(741.83578857,92.45963186)
\curveto(741.87578957,92.50962467)(741.95078949,92.54962463)(742.06078857,92.57963186)
\curveto(742.16078928,92.61962456)(742.26078918,92.62962455)(742.36078857,92.60963186)
\lineto(742.49578857,92.60963186)
\curveto(742.56578888,92.58962459)(742.62578882,92.56962461)(742.67578857,92.54963186)
\curveto(742.72578872,92.52962465)(742.76578868,92.49462468)(742.79578857,92.44463186)
\curveto(742.83578861,92.39462478)(742.85578859,92.32462485)(742.85578857,92.23463186)
\lineto(742.85578857,91.96463186)
\lineto(742.85578857,91.06463186)
\lineto(742.85578857,87.55463186)
\lineto(742.85578857,86.48963186)
\curveto(742.85578859,86.40963077)(742.86078858,86.31963086)(742.87078857,86.21963186)
\curveto(742.87078857,86.11963106)(742.86078858,86.03463114)(742.84078857,85.96463186)
\curveto(742.77078867,85.75463142)(742.59078885,85.68963149)(742.30078857,85.76963186)
\curveto(742.26078918,85.7796314)(742.22578922,85.7796314)(742.19578857,85.76963186)
\curveto(742.15578929,85.76963141)(742.11078933,85.7796314)(742.06078857,85.79963186)
\curveto(741.98078946,85.81963136)(741.89578955,85.83963134)(741.80578857,85.85963186)
\curveto(741.71578973,85.8796313)(741.63078981,85.90463127)(741.55078857,85.93463186)
\curveto(741.06079038,86.09463108)(740.6457908,86.29463088)(740.30578857,86.53463186)
\curveto(740.05579139,86.71463046)(739.83079161,86.91963026)(739.63078857,87.14963186)
\curveto(739.42079202,87.3796298)(739.22579222,87.61962956)(739.04578857,87.86963186)
\curveto(738.86579258,88.12962905)(738.69579275,88.39462878)(738.53578857,88.66463186)
\curveto(738.36579308,88.94462823)(738.19079325,89.21462796)(738.01078857,89.47463186)
\curveto(737.93079351,89.58462759)(737.85579359,89.68962749)(737.78578857,89.78963186)
\curveto(737.71579373,89.89962728)(737.6407938,90.00962717)(737.56078857,90.11963186)
\curveto(737.53079391,90.15962702)(737.50079394,90.19462698)(737.47078857,90.22463186)
\curveto(737.43079401,90.26462691)(737.40079404,90.30462687)(737.38078857,90.34463186)
\curveto(737.27079417,90.48462669)(737.1457943,90.60962657)(737.00578857,90.71963186)
\curveto(736.97579447,90.73962644)(736.95079449,90.76462641)(736.93078857,90.79463186)
\curveto(736.90079454,90.82462635)(736.87079457,90.84962633)(736.84078857,90.86963186)
\curveto(736.7407947,90.94962623)(736.6407948,91.01462616)(736.54078857,91.06463186)
\curveto(736.440795,91.12462605)(736.33079511,91.179626)(736.21078857,91.22963186)
\curveto(736.1407953,91.25962592)(736.06579538,91.2796259)(735.98578857,91.28963186)
\lineto(735.74578857,91.34963186)
\lineto(735.65578857,91.34963186)
\curveto(735.62579582,91.35962582)(735.59579585,91.36462581)(735.56578857,91.36463186)
\curveto(735.49579595,91.38462579)(735.40079604,91.38962579)(735.28078857,91.37963186)
\curveto(735.15079629,91.3796258)(735.05079639,91.36962581)(734.98078857,91.34963186)
\curveto(734.90079654,91.32962585)(734.82579662,91.30962587)(734.75578857,91.28963186)
\curveto(734.67579677,91.2796259)(734.59579685,91.25962592)(734.51578857,91.22963186)
\curveto(734.27579717,91.11962606)(734.07579737,90.96962621)(733.91578857,90.77963186)
\curveto(733.7457977,90.59962658)(733.60579784,90.3796268)(733.49578857,90.11963186)
\curveto(733.47579797,90.04962713)(733.46079798,89.9796272)(733.45078857,89.90963186)
\curveto(733.43079801,89.83962734)(733.41079803,89.76462741)(733.39078857,89.68463186)
\curveto(733.37079807,89.60462757)(733.36079808,89.49462768)(733.36078857,89.35463186)
\curveto(733.36079808,89.22462795)(733.37079807,89.11962806)(733.39078857,89.03963186)
\curveto(733.40079804,88.9796282)(733.40579804,88.92462825)(733.40578857,88.87463186)
\curveto(733.40579804,88.82462835)(733.41579803,88.7746284)(733.43578857,88.72463186)
\curveto(733.47579797,88.62462855)(733.51579793,88.52962865)(733.55578857,88.43963186)
\curveto(733.59579785,88.35962882)(733.6407978,88.2796289)(733.69078857,88.19963186)
\curveto(733.71079773,88.16962901)(733.73579771,88.13962904)(733.76578857,88.10963186)
\curveto(733.79579765,88.08962909)(733.82079762,88.06462911)(733.84078857,88.03463186)
\lineto(733.91578857,87.95963186)
\curveto(733.93579751,87.92962925)(733.95579749,87.90462927)(733.97578857,87.88463186)
\lineto(734.18578857,87.73463186)
\curveto(734.2457972,87.69462948)(734.31079713,87.64962953)(734.38078857,87.59963186)
\curveto(734.47079697,87.53962964)(734.57579687,87.48962969)(734.69578857,87.44963186)
\curveto(734.80579664,87.41962976)(734.91579653,87.38462979)(735.02578857,87.34463186)
\curveto(735.13579631,87.30462987)(735.28079616,87.2796299)(735.46078857,87.26963186)
\curveto(735.63079581,87.25962992)(735.75579569,87.22962995)(735.83578857,87.17963186)
\curveto(735.91579553,87.12963005)(735.96079548,87.05463012)(735.97078857,86.95463186)
\curveto(735.98079546,86.85463032)(735.98579546,86.74463043)(735.98578857,86.62463186)
\curveto(735.98579546,86.58463059)(735.99079545,86.54463063)(736.00078857,86.50463186)
\curveto(736.00079544,86.46463071)(735.99579545,86.42963075)(735.98578857,86.39963186)
\curveto(735.96579548,86.34963083)(735.95579549,86.29963088)(735.95578857,86.24963186)
\curveto(735.95579549,86.20963097)(735.9457955,86.16963101)(735.92578857,86.12963186)
\curveto(735.86579558,86.03963114)(735.73079571,85.99463118)(735.52078857,85.99463186)
\lineto(735.40078857,85.99463186)
\curveto(735.3407961,86.00463117)(735.28079616,86.00963117)(735.22078857,86.00963186)
\curveto(735.15079629,86.01963116)(735.08579636,86.02963115)(735.02578857,86.03963186)
\curveto(734.91579653,86.05963112)(734.81579663,86.0796311)(734.72578857,86.09963186)
\curveto(734.62579682,86.11963106)(734.53079691,86.14963103)(734.44078857,86.18963186)
\curveto(734.37079707,86.20963097)(734.31079713,86.22963095)(734.26078857,86.24963186)
\lineto(734.08078857,86.30963186)
\curveto(733.82079762,86.42963075)(733.57579787,86.58463059)(733.34578857,86.77463186)
\curveto(733.11579833,86.9746302)(732.93079851,87.18962999)(732.79078857,87.41963186)
\curveto(732.71079873,87.52962965)(732.6457988,87.64462953)(732.59578857,87.76463186)
\lineto(732.44578857,88.15463186)
\curveto(732.39579905,88.26462891)(732.36579908,88.3796288)(732.35578857,88.49963186)
\curveto(732.33579911,88.61962856)(732.31079913,88.74462843)(732.28078857,88.87463186)
\curveto(732.28079916,88.94462823)(732.28079916,89.00962817)(732.28078857,89.06963186)
\curveto(732.27079917,89.12962805)(732.26079918,89.19462798)(732.25078857,89.26463186)
}
}
{
\newrgbcolor{curcolor}{0 0 0}
\pscustom[linestyle=none,fillstyle=solid,fillcolor=curcolor]
{
\newpath
\moveto(737.77078857,101.36424123)
\lineto(738.02578857,101.36424123)
\curveto(738.10579334,101.37423353)(738.18079326,101.36923353)(738.25078857,101.34924123)
\lineto(738.49078857,101.34924123)
\lineto(738.65578857,101.34924123)
\curveto(738.75579269,101.32923357)(738.86079258,101.31923358)(738.97078857,101.31924123)
\curveto(739.07079237,101.31923358)(739.17079227,101.30923359)(739.27078857,101.28924123)
\lineto(739.42078857,101.28924123)
\curveto(739.56079188,101.25923364)(739.70079174,101.23923366)(739.84078857,101.22924123)
\curveto(739.97079147,101.21923368)(740.10079134,101.19423371)(740.23078857,101.15424123)
\curveto(740.31079113,101.13423377)(740.39579105,101.11423379)(740.48578857,101.09424123)
\lineto(740.72578857,101.03424123)
\lineto(741.02578857,100.91424123)
\curveto(741.11579033,100.88423402)(741.20579024,100.84923405)(741.29578857,100.80924123)
\curveto(741.51578993,100.70923419)(741.73078971,100.57423433)(741.94078857,100.40424123)
\curveto(742.15078929,100.24423466)(742.32078912,100.06923483)(742.45078857,99.87924123)
\curveto(742.49078895,99.82923507)(742.53078891,99.76923513)(742.57078857,99.69924123)
\curveto(742.60078884,99.63923526)(742.63578881,99.57923532)(742.67578857,99.51924123)
\curveto(742.72578872,99.43923546)(742.76578868,99.34423556)(742.79578857,99.23424123)
\curveto(742.82578862,99.12423578)(742.85578859,99.01923588)(742.88578857,98.91924123)
\curveto(742.92578852,98.80923609)(742.95078849,98.6992362)(742.96078857,98.58924123)
\curveto(742.97078847,98.47923642)(742.98578846,98.36423654)(743.00578857,98.24424123)
\curveto(743.01578843,98.2042367)(743.01578843,98.15923674)(743.00578857,98.10924123)
\curveto(743.00578844,98.06923683)(743.01078843,98.02923687)(743.02078857,97.98924123)
\curveto(743.03078841,97.94923695)(743.03578841,97.89423701)(743.03578857,97.82424123)
\curveto(743.03578841,97.75423715)(743.03078841,97.7042372)(743.02078857,97.67424123)
\curveto(743.00078844,97.62423728)(742.99578845,97.57923732)(743.00578857,97.53924123)
\curveto(743.01578843,97.4992374)(743.01578843,97.46423744)(743.00578857,97.43424123)
\lineto(743.00578857,97.34424123)
\curveto(742.98578846,97.28423762)(742.97078847,97.21923768)(742.96078857,97.14924123)
\curveto(742.96078848,97.08923781)(742.95578849,97.02423788)(742.94578857,96.95424123)
\curveto(742.89578855,96.78423812)(742.8457886,96.62423828)(742.79578857,96.47424123)
\curveto(742.7457887,96.32423858)(742.68078876,96.17923872)(742.60078857,96.03924123)
\curveto(742.56078888,95.98923891)(742.53078891,95.93423897)(742.51078857,95.87424123)
\curveto(742.48078896,95.82423908)(742.445789,95.77423913)(742.40578857,95.72424123)
\curveto(742.22578922,95.48423942)(742.00578944,95.28423962)(741.74578857,95.12424123)
\curveto(741.48578996,94.96423994)(741.20079024,94.82424008)(740.89078857,94.70424123)
\curveto(740.75079069,94.64424026)(740.61079083,94.5992403)(740.47078857,94.56924123)
\curveto(740.32079112,94.53924036)(740.16579128,94.5042404)(740.00578857,94.46424123)
\curveto(739.89579155,94.44424046)(739.78579166,94.42924047)(739.67578857,94.41924123)
\curveto(739.56579188,94.40924049)(739.45579199,94.39424051)(739.34578857,94.37424123)
\curveto(739.30579214,94.36424054)(739.26579218,94.35924054)(739.22578857,94.35924123)
\curveto(739.18579226,94.36924053)(739.1457923,94.36924053)(739.10578857,94.35924123)
\curveto(739.05579239,94.34924055)(739.00579244,94.34424056)(738.95578857,94.34424123)
\lineto(738.79078857,94.34424123)
\curveto(738.7407927,94.32424058)(738.69079275,94.31924058)(738.64078857,94.32924123)
\curveto(738.58079286,94.33924056)(738.52579292,94.33924056)(738.47578857,94.32924123)
\curveto(738.43579301,94.31924058)(738.39079305,94.31924058)(738.34078857,94.32924123)
\curveto(738.29079315,94.33924056)(738.2407932,94.33424057)(738.19078857,94.31424123)
\curveto(738.12079332,94.29424061)(738.0457934,94.28924061)(737.96578857,94.29924123)
\curveto(737.87579357,94.30924059)(737.79079365,94.31424059)(737.71078857,94.31424123)
\curveto(737.62079382,94.31424059)(737.52079392,94.30924059)(737.41078857,94.29924123)
\curveto(737.29079415,94.28924061)(737.19079425,94.29424061)(737.11078857,94.31424123)
\lineto(736.82578857,94.31424123)
\lineto(736.19578857,94.35924123)
\curveto(736.09579535,94.36924053)(736.00079544,94.37924052)(735.91078857,94.38924123)
\lineto(735.61078857,94.41924123)
\curveto(735.56079588,94.43924046)(735.51079593,94.44424046)(735.46078857,94.43424123)
\curveto(735.40079604,94.43424047)(735.3457961,94.44424046)(735.29578857,94.46424123)
\curveto(735.12579632,94.51424039)(734.96079648,94.55424035)(734.80078857,94.58424123)
\curveto(734.63079681,94.61424029)(734.47079697,94.66424024)(734.32078857,94.73424123)
\curveto(733.86079758,94.92423998)(733.48579796,95.14423976)(733.19578857,95.39424123)
\curveto(732.90579854,95.65423925)(732.66079878,96.01423889)(732.46078857,96.47424123)
\curveto(732.41079903,96.6042383)(732.37579907,96.73423817)(732.35578857,96.86424123)
\curveto(732.33579911,97.0042379)(732.31079913,97.14423776)(732.28078857,97.28424123)
\curveto(732.27079917,97.35423755)(732.26579918,97.41923748)(732.26578857,97.47924123)
\curveto(732.26579918,97.53923736)(732.26079918,97.6042373)(732.25078857,97.67424123)
\curveto(732.23079921,98.5042364)(732.38079906,99.17423573)(732.70078857,99.68424123)
\curveto(733.01079843,100.19423471)(733.45079799,100.57423433)(734.02078857,100.82424123)
\curveto(734.1407973,100.87423403)(734.26579718,100.91923398)(734.39578857,100.95924123)
\curveto(734.52579692,100.9992339)(734.66079678,101.04423386)(734.80078857,101.09424123)
\curveto(734.88079656,101.11423379)(734.96579648,101.12923377)(735.05578857,101.13924123)
\lineto(735.29578857,101.19924123)
\curveto(735.40579604,101.22923367)(735.51579593,101.24423366)(735.62578857,101.24424123)
\curveto(735.73579571,101.25423365)(735.8457956,101.26923363)(735.95578857,101.28924123)
\curveto(736.00579544,101.30923359)(736.05079539,101.31423359)(736.09078857,101.30424123)
\curveto(736.13079531,101.3042336)(736.17079527,101.30923359)(736.21078857,101.31924123)
\curveto(736.26079518,101.32923357)(736.31579513,101.32923357)(736.37578857,101.31924123)
\curveto(736.42579502,101.31923358)(736.47579497,101.32423358)(736.52578857,101.33424123)
\lineto(736.66078857,101.33424123)
\curveto(736.72079472,101.35423355)(736.79079465,101.35423355)(736.87078857,101.33424123)
\curveto(736.9407945,101.32423358)(737.00579444,101.32923357)(737.06578857,101.34924123)
\curveto(737.09579435,101.35923354)(737.13579431,101.36423354)(737.18578857,101.36424123)
\lineto(737.30578857,101.36424123)
\lineto(737.77078857,101.36424123)
\moveto(740.09578857,99.81924123)
\curveto(739.77579167,99.91923498)(739.41079203,99.97923492)(739.00078857,99.99924123)
\curveto(738.59079285,100.01923488)(738.18079326,100.02923487)(737.77078857,100.02924123)
\curveto(737.3407941,100.02923487)(736.92079452,100.01923488)(736.51078857,99.99924123)
\curveto(736.10079534,99.97923492)(735.71579573,99.93423497)(735.35578857,99.86424123)
\curveto(734.99579645,99.79423511)(734.67579677,99.68423522)(734.39578857,99.53424123)
\curveto(734.10579734,99.39423551)(733.87079757,99.1992357)(733.69078857,98.94924123)
\curveto(733.58079786,98.78923611)(733.50079794,98.60923629)(733.45078857,98.40924123)
\curveto(733.39079805,98.20923669)(733.36079808,97.96423694)(733.36078857,97.67424123)
\curveto(733.38079806,97.65423725)(733.39079805,97.61923728)(733.39078857,97.56924123)
\curveto(733.38079806,97.51923738)(733.38079806,97.47923742)(733.39078857,97.44924123)
\curveto(733.41079803,97.36923753)(733.43079801,97.29423761)(733.45078857,97.22424123)
\curveto(733.46079798,97.16423774)(733.48079796,97.0992378)(733.51078857,97.02924123)
\curveto(733.63079781,96.75923814)(733.80079764,96.53923836)(734.02078857,96.36924123)
\curveto(734.23079721,96.20923869)(734.47579697,96.07423883)(734.75578857,95.96424123)
\curveto(734.86579658,95.91423899)(734.98579646,95.87423903)(735.11578857,95.84424123)
\curveto(735.23579621,95.82423908)(735.36079608,95.7992391)(735.49078857,95.76924123)
\curveto(735.5407959,95.74923915)(735.59579585,95.73923916)(735.65578857,95.73924123)
\curveto(735.70579574,95.73923916)(735.75579569,95.73423917)(735.80578857,95.72424123)
\curveto(735.89579555,95.71423919)(735.99079545,95.7042392)(736.09078857,95.69424123)
\curveto(736.18079526,95.68423922)(736.27579517,95.67423923)(736.37578857,95.66424123)
\curveto(736.45579499,95.66423924)(736.5407949,95.65923924)(736.63078857,95.64924123)
\lineto(736.87078857,95.64924123)
\lineto(737.05078857,95.64924123)
\curveto(737.08079436,95.63923926)(737.11579433,95.63423927)(737.15578857,95.63424123)
\lineto(737.29078857,95.63424123)
\lineto(737.74078857,95.63424123)
\curveto(737.82079362,95.63423927)(737.90579354,95.62923927)(737.99578857,95.61924123)
\curveto(738.07579337,95.61923928)(738.15079329,95.62923927)(738.22078857,95.64924123)
\lineto(738.49078857,95.64924123)
\curveto(738.51079293,95.64923925)(738.5407929,95.64423926)(738.58078857,95.63424123)
\curveto(738.61079283,95.63423927)(738.63579281,95.63923926)(738.65578857,95.64924123)
\curveto(738.75579269,95.65923924)(738.85579259,95.66423924)(738.95578857,95.66424123)
\curveto(739.0457924,95.67423923)(739.1457923,95.68423922)(739.25578857,95.69424123)
\curveto(739.37579207,95.72423918)(739.50079194,95.73923916)(739.63078857,95.73924123)
\curveto(739.75079169,95.74923915)(739.86579158,95.77423913)(739.97578857,95.81424123)
\curveto(740.27579117,95.89423901)(740.5407909,95.97923892)(740.77078857,96.06924123)
\curveto(741.00079044,96.16923873)(741.21579023,96.31423859)(741.41578857,96.50424123)
\curveto(741.61578983,96.71423819)(741.76578968,96.97923792)(741.86578857,97.29924123)
\curveto(741.88578956,97.33923756)(741.89578955,97.37423753)(741.89578857,97.40424123)
\curveto(741.88578956,97.44423746)(741.89078955,97.48923741)(741.91078857,97.53924123)
\curveto(741.92078952,97.57923732)(741.93078951,97.64923725)(741.94078857,97.74924123)
\curveto(741.95078949,97.85923704)(741.9457895,97.94423696)(741.92578857,98.00424123)
\curveto(741.90578954,98.07423683)(741.89578955,98.14423676)(741.89578857,98.21424123)
\curveto(741.88578956,98.28423662)(741.87078957,98.34923655)(741.85078857,98.40924123)
\curveto(741.79078965,98.60923629)(741.70578974,98.78923611)(741.59578857,98.94924123)
\curveto(741.57578987,98.97923592)(741.55578989,99.0042359)(741.53578857,99.02424123)
\lineto(741.47578857,99.08424123)
\curveto(741.45578999,99.12423578)(741.41579003,99.17423573)(741.35578857,99.23424123)
\curveto(741.21579023,99.33423557)(741.08579036,99.41923548)(740.96578857,99.48924123)
\curveto(740.8457906,99.55923534)(740.70079074,99.62923527)(740.53078857,99.69924123)
\curveto(740.46079098,99.72923517)(740.39079105,99.74923515)(740.32078857,99.75924123)
\curveto(740.25079119,99.77923512)(740.17579127,99.7992351)(740.09578857,99.81924123)
}
}
{
\newrgbcolor{curcolor}{0 0 0}
\pscustom[linestyle=none,fillstyle=solid,fillcolor=curcolor]
{
\newpath
\moveto(732.25078857,106.77385061)
\curveto(732.25079919,106.87384575)(732.26079918,106.96884566)(732.28078857,107.05885061)
\curveto(732.29079915,107.14884548)(732.32079912,107.21384541)(732.37078857,107.25385061)
\curveto(732.45079899,107.31384531)(732.55579889,107.34384528)(732.68578857,107.34385061)
\lineto(733.07578857,107.34385061)
\lineto(734.57578857,107.34385061)
\lineto(740.96578857,107.34385061)
\lineto(742.13578857,107.34385061)
\lineto(742.45078857,107.34385061)
\curveto(742.55078889,107.35384527)(742.63078881,107.33884529)(742.69078857,107.29885061)
\curveto(742.77078867,107.24884538)(742.82078862,107.17384545)(742.84078857,107.07385061)
\curveto(742.85078859,106.98384564)(742.85578859,106.87384575)(742.85578857,106.74385061)
\lineto(742.85578857,106.51885061)
\curveto(742.83578861,106.43884619)(742.82078862,106.36884626)(742.81078857,106.30885061)
\curveto(742.79078865,106.24884638)(742.75078869,106.19884643)(742.69078857,106.15885061)
\curveto(742.63078881,106.11884651)(742.55578889,106.09884653)(742.46578857,106.09885061)
\lineto(742.16578857,106.09885061)
\lineto(741.07078857,106.09885061)
\lineto(735.73078857,106.09885061)
\curveto(735.6407958,106.07884655)(735.56579588,106.06384656)(735.50578857,106.05385061)
\curveto(735.43579601,106.05384657)(735.37579607,106.0238466)(735.32578857,105.96385061)
\curveto(735.27579617,105.89384673)(735.25079619,105.80384682)(735.25078857,105.69385061)
\curveto(735.2407962,105.59384703)(735.23579621,105.48384714)(735.23578857,105.36385061)
\lineto(735.23578857,104.22385061)
\lineto(735.23578857,103.72885061)
\curveto(735.22579622,103.56884906)(735.16579628,103.45884917)(735.05578857,103.39885061)
\curveto(735.02579642,103.37884925)(734.99579645,103.36884926)(734.96578857,103.36885061)
\curveto(734.92579652,103.36884926)(734.88079656,103.36384926)(734.83078857,103.35385061)
\curveto(734.71079673,103.33384929)(734.60079684,103.33884929)(734.50078857,103.36885061)
\curveto(734.40079704,103.40884922)(734.33079711,103.46384916)(734.29078857,103.53385061)
\curveto(734.2407972,103.61384901)(734.21579723,103.73384889)(734.21578857,103.89385061)
\curveto(734.21579723,104.05384857)(734.20079724,104.18884844)(734.17078857,104.29885061)
\curveto(734.16079728,104.34884828)(734.15579729,104.40384822)(734.15578857,104.46385061)
\curveto(734.1457973,104.5238481)(734.13079731,104.58384804)(734.11078857,104.64385061)
\curveto(734.06079738,104.79384783)(734.01079743,104.93884769)(733.96078857,105.07885061)
\curveto(733.90079754,105.21884741)(733.83079761,105.35384727)(733.75078857,105.48385061)
\curveto(733.66079778,105.623847)(733.55579789,105.74384688)(733.43578857,105.84385061)
\curveto(733.31579813,105.94384668)(733.18579826,106.03884659)(733.04578857,106.12885061)
\curveto(732.9457985,106.18884644)(732.83579861,106.23384639)(732.71578857,106.26385061)
\curveto(732.59579885,106.30384632)(732.49079895,106.35384627)(732.40078857,106.41385061)
\curveto(732.3407991,106.46384616)(732.30079914,106.53384609)(732.28078857,106.62385061)
\curveto(732.27079917,106.64384598)(732.26579918,106.66884596)(732.26578857,106.69885061)
\curveto(732.26579918,106.7288459)(732.26079918,106.75384587)(732.25078857,106.77385061)
}
}
{
\newrgbcolor{curcolor}{0 0 0}
\pscustom[linestyle=none,fillstyle=solid,fillcolor=curcolor]
{
\newpath
\moveto(732.25078857,115.12345998)
\curveto(732.25079919,115.22345513)(732.26079918,115.31845503)(732.28078857,115.40845998)
\curveto(732.29079915,115.49845485)(732.32079912,115.56345479)(732.37078857,115.60345998)
\curveto(732.45079899,115.66345469)(732.55579889,115.69345466)(732.68578857,115.69345998)
\lineto(733.07578857,115.69345998)
\lineto(734.57578857,115.69345998)
\lineto(740.96578857,115.69345998)
\lineto(742.13578857,115.69345998)
\lineto(742.45078857,115.69345998)
\curveto(742.55078889,115.70345465)(742.63078881,115.68845466)(742.69078857,115.64845998)
\curveto(742.77078867,115.59845475)(742.82078862,115.52345483)(742.84078857,115.42345998)
\curveto(742.85078859,115.33345502)(742.85578859,115.22345513)(742.85578857,115.09345998)
\lineto(742.85578857,114.86845998)
\curveto(742.83578861,114.78845556)(742.82078862,114.71845563)(742.81078857,114.65845998)
\curveto(742.79078865,114.59845575)(742.75078869,114.5484558)(742.69078857,114.50845998)
\curveto(742.63078881,114.46845588)(742.55578889,114.4484559)(742.46578857,114.44845998)
\lineto(742.16578857,114.44845998)
\lineto(741.07078857,114.44845998)
\lineto(735.73078857,114.44845998)
\curveto(735.6407958,114.42845592)(735.56579588,114.41345594)(735.50578857,114.40345998)
\curveto(735.43579601,114.40345595)(735.37579607,114.37345598)(735.32578857,114.31345998)
\curveto(735.27579617,114.24345611)(735.25079619,114.1534562)(735.25078857,114.04345998)
\curveto(735.2407962,113.94345641)(735.23579621,113.83345652)(735.23578857,113.71345998)
\lineto(735.23578857,112.57345998)
\lineto(735.23578857,112.07845998)
\curveto(735.22579622,111.91845843)(735.16579628,111.80845854)(735.05578857,111.74845998)
\curveto(735.02579642,111.72845862)(734.99579645,111.71845863)(734.96578857,111.71845998)
\curveto(734.92579652,111.71845863)(734.88079656,111.71345864)(734.83078857,111.70345998)
\curveto(734.71079673,111.68345867)(734.60079684,111.68845866)(734.50078857,111.71845998)
\curveto(734.40079704,111.75845859)(734.33079711,111.81345854)(734.29078857,111.88345998)
\curveto(734.2407972,111.96345839)(734.21579723,112.08345827)(734.21578857,112.24345998)
\curveto(734.21579723,112.40345795)(734.20079724,112.53845781)(734.17078857,112.64845998)
\curveto(734.16079728,112.69845765)(734.15579729,112.7534576)(734.15578857,112.81345998)
\curveto(734.1457973,112.87345748)(734.13079731,112.93345742)(734.11078857,112.99345998)
\curveto(734.06079738,113.14345721)(734.01079743,113.28845706)(733.96078857,113.42845998)
\curveto(733.90079754,113.56845678)(733.83079761,113.70345665)(733.75078857,113.83345998)
\curveto(733.66079778,113.97345638)(733.55579789,114.09345626)(733.43578857,114.19345998)
\curveto(733.31579813,114.29345606)(733.18579826,114.38845596)(733.04578857,114.47845998)
\curveto(732.9457985,114.53845581)(732.83579861,114.58345577)(732.71578857,114.61345998)
\curveto(732.59579885,114.6534557)(732.49079895,114.70345565)(732.40078857,114.76345998)
\curveto(732.3407991,114.81345554)(732.30079914,114.88345547)(732.28078857,114.97345998)
\curveto(732.27079917,114.99345536)(732.26579918,115.01845533)(732.26578857,115.04845998)
\curveto(732.26579918,115.07845527)(732.26079918,115.10345525)(732.25078857,115.12345998)
}
}
{
\newrgbcolor{curcolor}{0 0 0}
\pscustom[linestyle=none,fillstyle=solid,fillcolor=curcolor]
{
\newpath
\moveto(753.08710449,37.28705373)
\curveto(753.08711519,37.35704805)(753.08711519,37.43704797)(753.08710449,37.52705373)
\curveto(753.0771152,37.61704779)(753.0771152,37.70204771)(753.08710449,37.78205373)
\curveto(753.08711519,37.87204754)(753.09711518,37.95204746)(753.11710449,38.02205373)
\curveto(753.13711514,38.10204731)(753.16711511,38.15704725)(753.20710449,38.18705373)
\curveto(753.25711502,38.21704719)(753.33211494,38.23704717)(753.43210449,38.24705373)
\curveto(753.52211475,38.26704714)(753.62711465,38.27704713)(753.74710449,38.27705373)
\curveto(753.85711442,38.28704712)(753.9721143,38.28704712)(754.09210449,38.27705373)
\lineto(754.39210449,38.27705373)
\lineto(757.40710449,38.27705373)
\lineto(760.30210449,38.27705373)
\curveto(760.63210764,38.27704713)(760.95710732,38.27204714)(761.27710449,38.26205373)
\curveto(761.58710669,38.26204715)(761.86710641,38.22204719)(762.11710449,38.14205373)
\curveto(762.46710581,38.02204739)(762.76210551,37.86704754)(763.00210449,37.67705373)
\curveto(763.23210504,37.48704792)(763.43210484,37.24704816)(763.60210449,36.95705373)
\curveto(763.65210462,36.89704851)(763.68710459,36.83204858)(763.70710449,36.76205373)
\curveto(763.72710455,36.70204871)(763.75210452,36.63204878)(763.78210449,36.55205373)
\curveto(763.83210444,36.43204898)(763.86710441,36.30204911)(763.88710449,36.16205373)
\curveto(763.91710436,36.03204938)(763.94710433,35.89704951)(763.97710449,35.75705373)
\curveto(763.99710428,35.7070497)(764.00210427,35.65704975)(763.99210449,35.60705373)
\curveto(763.98210429,35.55704985)(763.98210429,35.50204991)(763.99210449,35.44205373)
\curveto(764.00210427,35.42204999)(764.00210427,35.39705001)(763.99210449,35.36705373)
\curveto(763.99210428,35.33705007)(763.99710428,35.3120501)(764.00710449,35.29205373)
\curveto(764.01710426,35.25205016)(764.02210425,35.19705021)(764.02210449,35.12705373)
\curveto(764.02210425,35.05705035)(764.01710426,35.00205041)(764.00710449,34.96205373)
\curveto(763.99710428,34.9120505)(763.99710428,34.85705055)(764.00710449,34.79705373)
\curveto(764.01710426,34.73705067)(764.01210426,34.68205073)(763.99210449,34.63205373)
\curveto(763.96210431,34.50205091)(763.94210433,34.37705103)(763.93210449,34.25705373)
\curveto(763.92210435,34.13705127)(763.89710438,34.02205139)(763.85710449,33.91205373)
\curveto(763.73710454,33.54205187)(763.56710471,33.22205219)(763.34710449,32.95205373)
\curveto(763.12710515,32.68205273)(762.84710543,32.47205294)(762.50710449,32.32205373)
\curveto(762.38710589,32.27205314)(762.26210601,32.22705318)(762.13210449,32.18705373)
\curveto(762.00210627,32.15705325)(761.86710641,32.12205329)(761.72710449,32.08205373)
\curveto(761.6771066,32.07205334)(761.63710664,32.06705334)(761.60710449,32.06705373)
\curveto(761.56710671,32.06705334)(761.52210675,32.06205335)(761.47210449,32.05205373)
\curveto(761.44210683,32.04205337)(761.40710687,32.03705337)(761.36710449,32.03705373)
\curveto(761.31710696,32.03705337)(761.277107,32.03205338)(761.24710449,32.02205373)
\lineto(761.08210449,32.02205373)
\curveto(761.00210727,32.00205341)(760.90210737,31.99705341)(760.78210449,32.00705373)
\curveto(760.65210762,32.01705339)(760.56210771,32.03205338)(760.51210449,32.05205373)
\curveto(760.42210785,32.07205334)(760.35710792,32.12705328)(760.31710449,32.21705373)
\curveto(760.29710798,32.24705316)(760.29210798,32.27705313)(760.30210449,32.30705373)
\curveto(760.30210797,32.33705307)(760.29710798,32.37705303)(760.28710449,32.42705373)
\curveto(760.277108,32.46705294)(760.272108,32.5070529)(760.27210449,32.54705373)
\lineto(760.27210449,32.69705373)
\curveto(760.272108,32.81705259)(760.277108,32.93705247)(760.28710449,33.05705373)
\curveto(760.28710799,33.18705222)(760.32210795,33.27705213)(760.39210449,33.32705373)
\curveto(760.45210782,33.36705204)(760.51210776,33.38705202)(760.57210449,33.38705373)
\curveto(760.63210764,33.38705202)(760.70210757,33.39705201)(760.78210449,33.41705373)
\curveto(760.81210746,33.42705198)(760.84710743,33.42705198)(760.88710449,33.41705373)
\curveto(760.91710736,33.41705199)(760.94210733,33.42205199)(760.96210449,33.43205373)
\lineto(761.17210449,33.43205373)
\curveto(761.22210705,33.45205196)(761.272107,33.45705195)(761.32210449,33.44705373)
\curveto(761.36210691,33.44705196)(761.40710687,33.45705195)(761.45710449,33.47705373)
\curveto(761.58710669,33.5070519)(761.71210656,33.53705187)(761.83210449,33.56705373)
\curveto(761.94210633,33.59705181)(762.04710623,33.64205177)(762.14710449,33.70205373)
\curveto(762.43710584,33.87205154)(762.64210563,34.14205127)(762.76210449,34.51205373)
\curveto(762.78210549,34.56205085)(762.79710548,34.6120508)(762.80710449,34.66205373)
\curveto(762.80710547,34.72205069)(762.81710546,34.77705063)(762.83710449,34.82705373)
\lineto(762.83710449,34.90205373)
\curveto(762.84710543,34.97205044)(762.85710542,35.06705034)(762.86710449,35.18705373)
\curveto(762.86710541,35.31705009)(762.85710542,35.41704999)(762.83710449,35.48705373)
\curveto(762.81710546,35.55704985)(762.80210547,35.62704978)(762.79210449,35.69705373)
\curveto(762.7721055,35.77704963)(762.75210552,35.84704956)(762.73210449,35.90705373)
\curveto(762.5721057,36.28704912)(762.29710598,36.56204885)(761.90710449,36.73205373)
\curveto(761.7771065,36.78204863)(761.62210665,36.81704859)(761.44210449,36.83705373)
\curveto(761.26210701,36.86704854)(761.0771072,36.88204853)(760.88710449,36.88205373)
\curveto(760.68710759,36.89204852)(760.48710779,36.89204852)(760.28710449,36.88205373)
\lineto(759.71710449,36.88205373)
\lineto(755.47210449,36.88205373)
\lineto(753.92710449,36.88205373)
\curveto(753.81711446,36.88204853)(753.69711458,36.87704853)(753.56710449,36.86705373)
\curveto(753.43711484,36.85704855)(753.33211494,36.87704853)(753.25210449,36.92705373)
\curveto(753.18211509,36.98704842)(753.13211514,37.06704834)(753.10210449,37.16705373)
\curveto(753.10211517,37.18704822)(753.10211517,37.2070482)(753.10210449,37.22705373)
\curveto(753.10211517,37.24704816)(753.09711518,37.26704814)(753.08710449,37.28705373)
}
}
{
\newrgbcolor{curcolor}{0 0 0}
\pscustom[linestyle=none,fillstyle=solid,fillcolor=curcolor]
{
\newpath
\moveto(756.04210449,40.82072561)
\lineto(756.04210449,41.25572561)
\curveto(756.04211223,41.40572364)(756.08211219,41.51072354)(756.16210449,41.57072561)
\curveto(756.24211203,41.62072343)(756.34211193,41.6457234)(756.46210449,41.64572561)
\curveto(756.58211169,41.65572339)(756.70211157,41.66072339)(756.82210449,41.66072561)
\lineto(758.24710449,41.66072561)
\lineto(760.51210449,41.66072561)
\lineto(761.20210449,41.66072561)
\curveto(761.43210684,41.66072339)(761.63210664,41.68572336)(761.80210449,41.73572561)
\curveto(762.25210602,41.89572315)(762.56710571,42.19572285)(762.74710449,42.63572561)
\curveto(762.83710544,42.85572219)(762.8721054,43.12072193)(762.85210449,43.43072561)
\curveto(762.82210545,43.74072131)(762.76710551,43.99072106)(762.68710449,44.18072561)
\curveto(762.54710573,44.51072054)(762.3721059,44.77072028)(762.16210449,44.96072561)
\curveto(761.94210633,45.16071989)(761.65710662,45.31571973)(761.30710449,45.42572561)
\curveto(761.22710705,45.45571959)(761.14710713,45.47571957)(761.06710449,45.48572561)
\curveto(760.98710729,45.49571955)(760.90210737,45.51071954)(760.81210449,45.53072561)
\curveto(760.76210751,45.54071951)(760.71710756,45.54071951)(760.67710449,45.53072561)
\curveto(760.63710764,45.53071952)(760.59210768,45.54071951)(760.54210449,45.56072561)
\lineto(760.22710449,45.56072561)
\curveto(760.14710813,45.58071947)(760.05710822,45.58571946)(759.95710449,45.57572561)
\curveto(759.84710843,45.56571948)(759.74710853,45.56071949)(759.65710449,45.56072561)
\lineto(758.48710449,45.56072561)
\lineto(756.89710449,45.56072561)
\curveto(756.7771115,45.56071949)(756.65211162,45.55571949)(756.52210449,45.54572561)
\curveto(756.38211189,45.5457195)(756.272112,45.57071948)(756.19210449,45.62072561)
\curveto(756.14211213,45.66071939)(756.11211216,45.70571934)(756.10210449,45.75572561)
\curveto(756.08211219,45.81571923)(756.06211221,45.88571916)(756.04210449,45.96572561)
\lineto(756.04210449,46.19072561)
\curveto(756.04211223,46.31071874)(756.04711223,46.41571863)(756.05710449,46.50572561)
\curveto(756.06711221,46.60571844)(756.11211216,46.68071837)(756.19210449,46.73072561)
\curveto(756.24211203,46.78071827)(756.31711196,46.80571824)(756.41710449,46.80572561)
\lineto(756.70210449,46.80572561)
\lineto(757.72210449,46.80572561)
\lineto(761.75710449,46.80572561)
\lineto(763.10710449,46.80572561)
\curveto(763.22710505,46.80571824)(763.34210493,46.80071825)(763.45210449,46.79072561)
\curveto(763.55210472,46.79071826)(763.62710465,46.75571829)(763.67710449,46.68572561)
\curveto(763.70710457,46.6457184)(763.73210454,46.58571846)(763.75210449,46.50572561)
\curveto(763.76210451,46.42571862)(763.7721045,46.33571871)(763.78210449,46.23572561)
\curveto(763.78210449,46.1457189)(763.7771045,46.05571899)(763.76710449,45.96572561)
\curveto(763.75710452,45.88571916)(763.73710454,45.82571922)(763.70710449,45.78572561)
\curveto(763.66710461,45.73571931)(763.60210467,45.69071936)(763.51210449,45.65072561)
\curveto(763.4721048,45.64071941)(763.41710486,45.63071942)(763.34710449,45.62072561)
\curveto(763.277105,45.62071943)(763.21210506,45.61571943)(763.15210449,45.60572561)
\curveto(763.08210519,45.59571945)(763.02710525,45.57571947)(762.98710449,45.54572561)
\curveto(762.94710533,45.51571953)(762.93210534,45.47071958)(762.94210449,45.41072561)
\curveto(762.96210531,45.33071972)(763.02210525,45.2507198)(763.12210449,45.17072561)
\curveto(763.21210506,45.09071996)(763.28210499,45.01572003)(763.33210449,44.94572561)
\curveto(763.49210478,44.72572032)(763.63210464,44.47572057)(763.75210449,44.19572561)
\curveto(763.80210447,44.08572096)(763.83210444,43.97072108)(763.84210449,43.85072561)
\curveto(763.86210441,43.74072131)(763.88710439,43.62572142)(763.91710449,43.50572561)
\curveto(763.92710435,43.45572159)(763.92710435,43.40072165)(763.91710449,43.34072561)
\curveto(763.90710437,43.29072176)(763.91210436,43.24072181)(763.93210449,43.19072561)
\curveto(763.95210432,43.09072196)(763.95210432,43.00072205)(763.93210449,42.92072561)
\lineto(763.93210449,42.77072561)
\curveto(763.91210436,42.72072233)(763.90210437,42.66072239)(763.90210449,42.59072561)
\curveto(763.90210437,42.53072252)(763.89710438,42.47572257)(763.88710449,42.42572561)
\curveto(763.86710441,42.38572266)(763.85710442,42.3457227)(763.85710449,42.30572561)
\curveto(763.86710441,42.27572277)(763.86210441,42.23572281)(763.84210449,42.18572561)
\lineto(763.78210449,41.94572561)
\curveto(763.76210451,41.87572317)(763.73210454,41.80072325)(763.69210449,41.72072561)
\curveto(763.58210469,41.46072359)(763.43710484,41.24072381)(763.25710449,41.06072561)
\curveto(763.06710521,40.89072416)(762.84210543,40.7507243)(762.58210449,40.64072561)
\curveto(762.49210578,40.60072445)(762.40210587,40.57072448)(762.31210449,40.55072561)
\lineto(762.01210449,40.49072561)
\curveto(761.95210632,40.47072458)(761.89710638,40.46072459)(761.84710449,40.46072561)
\curveto(761.78710649,40.47072458)(761.72210655,40.46572458)(761.65210449,40.44572561)
\curveto(761.63210664,40.43572461)(761.60710667,40.43072462)(761.57710449,40.43072561)
\curveto(761.53710674,40.43072462)(761.50210677,40.42572462)(761.47210449,40.41572561)
\lineto(761.32210449,40.41572561)
\curveto(761.28210699,40.40572464)(761.23710704,40.40072465)(761.18710449,40.40072561)
\curveto(761.12710715,40.41072464)(761.0721072,40.41572463)(761.02210449,40.41572561)
\lineto(760.42210449,40.41572561)
\lineto(757.66210449,40.41572561)
\lineto(756.70210449,40.41572561)
\lineto(756.43210449,40.41572561)
\curveto(756.34211193,40.41572463)(756.26711201,40.43572461)(756.20710449,40.47572561)
\curveto(756.13711214,40.51572453)(756.08711219,40.59072446)(756.05710449,40.70072561)
\curveto(756.04711223,40.72072433)(756.04711223,40.74072431)(756.05710449,40.76072561)
\curveto(756.05711222,40.78072427)(756.05211222,40.80072425)(756.04210449,40.82072561)
}
}
{
\newrgbcolor{curcolor}{0 0 0}
\pscustom[linestyle=none,fillstyle=solid,fillcolor=curcolor]
{
\newpath
\moveto(755.89210449,52.39533498)
\curveto(755.8721124,53.02532975)(755.95711232,53.53032924)(756.14710449,53.91033498)
\curveto(756.33711194,54.29032848)(756.62211165,54.59532818)(757.00210449,54.82533498)
\curveto(757.10211117,54.88532789)(757.21211106,54.93032784)(757.33210449,54.96033498)
\curveto(757.44211083,55.00032777)(757.55711072,55.03532774)(757.67710449,55.06533498)
\curveto(757.86711041,55.11532766)(758.0721102,55.14532763)(758.29210449,55.15533498)
\curveto(758.51210976,55.16532761)(758.73710954,55.1703276)(758.96710449,55.17033498)
\lineto(760.57210449,55.17033498)
\lineto(762.91210449,55.17033498)
\curveto(763.08210519,55.1703276)(763.25210502,55.16532761)(763.42210449,55.15533498)
\curveto(763.59210468,55.15532762)(763.70210457,55.09032768)(763.75210449,54.96033498)
\curveto(763.7721045,54.91032786)(763.78210449,54.85532792)(763.78210449,54.79533498)
\curveto(763.79210448,54.74532803)(763.79710448,54.69032808)(763.79710449,54.63033498)
\curveto(763.79710448,54.50032827)(763.79210448,54.3753284)(763.78210449,54.25533498)
\curveto(763.78210449,54.13532864)(763.74210453,54.05032872)(763.66210449,54.00033498)
\curveto(763.59210468,53.95032882)(763.50210477,53.92532885)(763.39210449,53.92533498)
\lineto(763.06210449,53.92533498)
\lineto(761.77210449,53.92533498)
\lineto(759.32710449,53.92533498)
\curveto(759.05710922,53.92532885)(758.79210948,53.92032885)(758.53210449,53.91033498)
\curveto(758.26211001,53.90032887)(758.03211024,53.85532892)(757.84210449,53.77533498)
\curveto(757.64211063,53.69532908)(757.48211079,53.5753292)(757.36210449,53.41533498)
\curveto(757.23211104,53.25532952)(757.13211114,53.0703297)(757.06210449,52.86033498)
\curveto(757.04211123,52.80032997)(757.03211124,52.73533004)(757.03210449,52.66533498)
\curveto(757.02211125,52.60533017)(757.00711127,52.54533023)(756.98710449,52.48533498)
\curveto(756.9771113,52.43533034)(756.9771113,52.35533042)(756.98710449,52.24533498)
\curveto(756.98711129,52.14533063)(756.99211128,52.0753307)(757.00210449,52.03533498)
\curveto(757.02211125,51.99533078)(757.03211124,51.96033081)(757.03210449,51.93033498)
\curveto(757.02211125,51.90033087)(757.02211125,51.86533091)(757.03210449,51.82533498)
\curveto(757.06211121,51.69533108)(757.09711118,51.5703312)(757.13710449,51.45033498)
\curveto(757.16711111,51.34033143)(757.21211106,51.23533154)(757.27210449,51.13533498)
\curveto(757.29211098,51.09533168)(757.31211096,51.06033171)(757.33210449,51.03033498)
\curveto(757.35211092,51.00033177)(757.3721109,50.96533181)(757.39210449,50.92533498)
\curveto(757.64211063,50.5753322)(758.01711026,50.32033245)(758.51710449,50.16033498)
\curveto(758.59710968,50.13033264)(758.68210959,50.11033266)(758.77210449,50.10033498)
\curveto(758.85210942,50.09033268)(758.93210934,50.0753327)(759.01210449,50.05533498)
\curveto(759.06210921,50.03533274)(759.11210916,50.03033274)(759.16210449,50.04033498)
\curveto(759.20210907,50.05033272)(759.24210903,50.04533273)(759.28210449,50.02533498)
\lineto(759.59710449,50.02533498)
\curveto(759.62710865,50.01533276)(759.66210861,50.01033276)(759.70210449,50.01033498)
\curveto(759.74210853,50.02033275)(759.78710849,50.02533275)(759.83710449,50.02533498)
\lineto(760.28710449,50.02533498)
\lineto(761.72710449,50.02533498)
\lineto(763.04710449,50.02533498)
\lineto(763.39210449,50.02533498)
\curveto(763.50210477,50.02533275)(763.59210468,50.00033277)(763.66210449,49.95033498)
\curveto(763.74210453,49.90033287)(763.78210449,49.81033296)(763.78210449,49.68033498)
\curveto(763.79210448,49.56033321)(763.79710448,49.43533334)(763.79710449,49.30533498)
\curveto(763.79710448,49.22533355)(763.79210448,49.15033362)(763.78210449,49.08033498)
\curveto(763.7721045,49.01033376)(763.74710453,48.95033382)(763.70710449,48.90033498)
\curveto(763.65710462,48.82033395)(763.56210471,48.78033399)(763.42210449,48.78033498)
\lineto(763.01710449,48.78033498)
\lineto(761.24710449,48.78033498)
\lineto(757.61710449,48.78033498)
\lineto(756.70210449,48.78033498)
\lineto(756.43210449,48.78033498)
\curveto(756.34211193,48.78033399)(756.272112,48.80033397)(756.22210449,48.84033498)
\curveto(756.16211211,48.8703339)(756.12211215,48.92033385)(756.10210449,48.99033498)
\curveto(756.09211218,49.03033374)(756.08211219,49.08533369)(756.07210449,49.15533498)
\curveto(756.06211221,49.23533354)(756.05711222,49.31533346)(756.05710449,49.39533498)
\curveto(756.05711222,49.4753333)(756.06211221,49.55033322)(756.07210449,49.62033498)
\curveto(756.08211219,49.70033307)(756.09711218,49.75533302)(756.11710449,49.78533498)
\curveto(756.18711209,49.89533288)(756.277112,49.94533283)(756.38710449,49.93533498)
\curveto(756.48711179,49.92533285)(756.60211167,49.94033283)(756.73210449,49.98033498)
\curveto(756.79211148,50.00033277)(756.84211143,50.04033273)(756.88210449,50.10033498)
\curveto(756.89211138,50.22033255)(756.84711143,50.31533246)(756.74710449,50.38533498)
\curveto(756.64711163,50.46533231)(756.56711171,50.54533223)(756.50710449,50.62533498)
\curveto(756.40711187,50.76533201)(756.31711196,50.90533187)(756.23710449,51.04533498)
\curveto(756.14711213,51.19533158)(756.0721122,51.36533141)(756.01210449,51.55533498)
\curveto(755.98211229,51.63533114)(755.96211231,51.72033105)(755.95210449,51.81033498)
\curveto(755.94211233,51.91033086)(755.92711235,52.00533077)(755.90710449,52.09533498)
\curveto(755.89711238,52.14533063)(755.89211238,52.19533058)(755.89210449,52.24533498)
\lineto(755.89210449,52.39533498)
}
}
{
\newrgbcolor{curcolor}{0 0 0}
\pscustom[linestyle=none,fillstyle=solid,fillcolor=curcolor]
{
}
}
{
\newrgbcolor{curcolor}{0 0 0}
\pscustom[linestyle=none,fillstyle=solid,fillcolor=curcolor]
{
\newpath
\moveto(753.16210449,64.24510061)
\curveto(753.15211512,64.93509597)(753.272115,65.53509537)(753.52210449,66.04510061)
\curveto(753.7721145,66.56509434)(754.10711417,66.96009395)(754.52710449,67.23010061)
\curveto(754.60711367,67.28009363)(754.69711358,67.32509358)(754.79710449,67.36510061)
\curveto(754.88711339,67.4050935)(754.98211329,67.45009346)(755.08210449,67.50010061)
\curveto(755.18211309,67.54009337)(755.28211299,67.57009334)(755.38210449,67.59010061)
\curveto(755.48211279,67.6100933)(755.58711269,67.63009328)(755.69710449,67.65010061)
\curveto(755.74711253,67.67009324)(755.79211248,67.67509323)(755.83210449,67.66510061)
\curveto(755.8721124,67.65509325)(755.91711236,67.66009325)(755.96710449,67.68010061)
\curveto(756.01711226,67.69009322)(756.10211217,67.69509321)(756.22210449,67.69510061)
\curveto(756.33211194,67.69509321)(756.41711186,67.69009322)(756.47710449,67.68010061)
\curveto(756.53711174,67.66009325)(756.59711168,67.65009326)(756.65710449,67.65010061)
\curveto(756.71711156,67.66009325)(756.7771115,67.65509325)(756.83710449,67.63510061)
\curveto(756.9771113,67.59509331)(757.11211116,67.56009335)(757.24210449,67.53010061)
\curveto(757.3721109,67.50009341)(757.49711078,67.46009345)(757.61710449,67.41010061)
\curveto(757.75711052,67.35009356)(757.88211039,67.28009363)(757.99210449,67.20010061)
\curveto(758.10211017,67.13009378)(758.21211006,67.05509385)(758.32210449,66.97510061)
\lineto(758.38210449,66.91510061)
\curveto(758.40210987,66.905094)(758.42210985,66.89009402)(758.44210449,66.87010061)
\curveto(758.60210967,66.75009416)(758.74710953,66.61509429)(758.87710449,66.46510061)
\curveto(759.00710927,66.31509459)(759.13210914,66.15509475)(759.25210449,65.98510061)
\curveto(759.4721088,65.67509523)(759.6771086,65.38009553)(759.86710449,65.10010061)
\curveto(760.00710827,64.87009604)(760.14210813,64.64009627)(760.27210449,64.41010061)
\curveto(760.40210787,64.19009672)(760.53710774,63.97009694)(760.67710449,63.75010061)
\curveto(760.84710743,63.50009741)(761.02710725,63.26009765)(761.21710449,63.03010061)
\curveto(761.40710687,62.8100981)(761.63210664,62.62009829)(761.89210449,62.46010061)
\curveto(761.95210632,62.42009849)(762.01210626,62.38509852)(762.07210449,62.35510061)
\curveto(762.12210615,62.32509858)(762.18710609,62.29509861)(762.26710449,62.26510061)
\curveto(762.33710594,62.24509866)(762.39710588,62.24009867)(762.44710449,62.25010061)
\curveto(762.51710576,62.27009864)(762.5721057,62.3050986)(762.61210449,62.35510061)
\curveto(762.64210563,62.4050985)(762.66210561,62.46509844)(762.67210449,62.53510061)
\lineto(762.67210449,62.77510061)
\lineto(762.67210449,63.52510061)
\lineto(762.67210449,66.33010061)
\lineto(762.67210449,66.99010061)
\curveto(762.6721056,67.08009383)(762.6771056,67.16509374)(762.68710449,67.24510061)
\curveto(762.68710559,67.32509358)(762.70710557,67.39009352)(762.74710449,67.44010061)
\curveto(762.78710549,67.49009342)(762.86210541,67.53009338)(762.97210449,67.56010061)
\curveto(763.0721052,67.60009331)(763.1721051,67.6100933)(763.27210449,67.59010061)
\lineto(763.40710449,67.59010061)
\curveto(763.4771048,67.57009334)(763.53710474,67.55009336)(763.58710449,67.53010061)
\curveto(763.63710464,67.5100934)(763.6771046,67.47509343)(763.70710449,67.42510061)
\curveto(763.74710453,67.37509353)(763.76710451,67.3050936)(763.76710449,67.21510061)
\lineto(763.76710449,66.94510061)
\lineto(763.76710449,66.04510061)
\lineto(763.76710449,62.53510061)
\lineto(763.76710449,61.47010061)
\curveto(763.76710451,61.39009952)(763.7721045,61.30009961)(763.78210449,61.20010061)
\curveto(763.78210449,61.10009981)(763.7721045,61.01509989)(763.75210449,60.94510061)
\curveto(763.68210459,60.73510017)(763.50210477,60.67010024)(763.21210449,60.75010061)
\curveto(763.1721051,60.76010015)(763.13710514,60.76010015)(763.10710449,60.75010061)
\curveto(763.06710521,60.75010016)(763.02210525,60.76010015)(762.97210449,60.78010061)
\curveto(762.89210538,60.80010011)(762.80710547,60.82010009)(762.71710449,60.84010061)
\curveto(762.62710565,60.86010005)(762.54210573,60.88510002)(762.46210449,60.91510061)
\curveto(761.9721063,61.07509983)(761.55710672,61.27509963)(761.21710449,61.51510061)
\curveto(760.96710731,61.69509921)(760.74210753,61.90009901)(760.54210449,62.13010061)
\curveto(760.33210794,62.36009855)(760.13710814,62.60009831)(759.95710449,62.85010061)
\curveto(759.7771085,63.1100978)(759.60710867,63.37509753)(759.44710449,63.64510061)
\curveto(759.277109,63.92509698)(759.10210917,64.19509671)(758.92210449,64.45510061)
\curveto(758.84210943,64.56509634)(758.76710951,64.67009624)(758.69710449,64.77010061)
\curveto(758.62710965,64.88009603)(758.55210972,64.99009592)(758.47210449,65.10010061)
\curveto(758.44210983,65.14009577)(758.41210986,65.17509573)(758.38210449,65.20510061)
\curveto(758.34210993,65.24509566)(758.31210996,65.28509562)(758.29210449,65.32510061)
\curveto(758.18211009,65.46509544)(758.05711022,65.59009532)(757.91710449,65.70010061)
\curveto(757.88711039,65.72009519)(757.86211041,65.74509516)(757.84210449,65.77510061)
\curveto(757.81211046,65.8050951)(757.78211049,65.83009508)(757.75210449,65.85010061)
\curveto(757.65211062,65.93009498)(757.55211072,65.99509491)(757.45210449,66.04510061)
\curveto(757.35211092,66.1050948)(757.24211103,66.16009475)(757.12210449,66.21010061)
\curveto(757.05211122,66.24009467)(756.9771113,66.26009465)(756.89710449,66.27010061)
\lineto(756.65710449,66.33010061)
\lineto(756.56710449,66.33010061)
\curveto(756.53711174,66.34009457)(756.50711177,66.34509456)(756.47710449,66.34510061)
\curveto(756.40711187,66.36509454)(756.31211196,66.37009454)(756.19210449,66.36010061)
\curveto(756.06211221,66.36009455)(755.96211231,66.35009456)(755.89210449,66.33010061)
\curveto(755.81211246,66.3100946)(755.73711254,66.29009462)(755.66710449,66.27010061)
\curveto(755.58711269,66.26009465)(755.50711277,66.24009467)(755.42710449,66.21010061)
\curveto(755.18711309,66.10009481)(754.98711329,65.95009496)(754.82710449,65.76010061)
\curveto(754.65711362,65.58009533)(754.51711376,65.36009555)(754.40710449,65.10010061)
\curveto(754.38711389,65.03009588)(754.3721139,64.96009595)(754.36210449,64.89010061)
\curveto(754.34211393,64.82009609)(754.32211395,64.74509616)(754.30210449,64.66510061)
\curveto(754.28211399,64.58509632)(754.272114,64.47509643)(754.27210449,64.33510061)
\curveto(754.272114,64.2050967)(754.28211399,64.10009681)(754.30210449,64.02010061)
\curveto(754.31211396,63.96009695)(754.31711396,63.905097)(754.31710449,63.85510061)
\curveto(754.31711396,63.8050971)(754.32711395,63.75509715)(754.34710449,63.70510061)
\curveto(754.38711389,63.6050973)(754.42711385,63.5100974)(754.46710449,63.42010061)
\curveto(754.50711377,63.34009757)(754.55211372,63.26009765)(754.60210449,63.18010061)
\curveto(754.62211365,63.15009776)(754.64711363,63.12009779)(754.67710449,63.09010061)
\curveto(754.70711357,63.07009784)(754.73211354,63.04509786)(754.75210449,63.01510061)
\lineto(754.82710449,62.94010061)
\curveto(754.84711343,62.910098)(754.86711341,62.88509802)(754.88710449,62.86510061)
\lineto(755.09710449,62.71510061)
\curveto(755.15711312,62.67509823)(755.22211305,62.63009828)(755.29210449,62.58010061)
\curveto(755.38211289,62.52009839)(755.48711279,62.47009844)(755.60710449,62.43010061)
\curveto(755.71711256,62.40009851)(755.82711245,62.36509854)(755.93710449,62.32510061)
\curveto(756.04711223,62.28509862)(756.19211208,62.26009865)(756.37210449,62.25010061)
\curveto(756.54211173,62.24009867)(756.66711161,62.2100987)(756.74710449,62.16010061)
\curveto(756.82711145,62.1100988)(756.8721114,62.03509887)(756.88210449,61.93510061)
\curveto(756.89211138,61.83509907)(756.89711138,61.72509918)(756.89710449,61.60510061)
\curveto(756.89711138,61.56509934)(756.90211137,61.52509938)(756.91210449,61.48510061)
\curveto(756.91211136,61.44509946)(756.90711137,61.4100995)(756.89710449,61.38010061)
\curveto(756.8771114,61.33009958)(756.86711141,61.28009963)(756.86710449,61.23010061)
\curveto(756.86711141,61.19009972)(756.85711142,61.15009976)(756.83710449,61.11010061)
\curveto(756.7771115,61.02009989)(756.64211163,60.97509993)(756.43210449,60.97510061)
\lineto(756.31210449,60.97510061)
\curveto(756.25211202,60.98509992)(756.19211208,60.99009992)(756.13210449,60.99010061)
\curveto(756.06211221,61.00009991)(755.99711228,61.0100999)(755.93710449,61.02010061)
\curveto(755.82711245,61.04009987)(755.72711255,61.06009985)(755.63710449,61.08010061)
\curveto(755.53711274,61.10009981)(755.44211283,61.13009978)(755.35210449,61.17010061)
\curveto(755.28211299,61.19009972)(755.22211305,61.2100997)(755.17210449,61.23010061)
\lineto(754.99210449,61.29010061)
\curveto(754.73211354,61.4100995)(754.48711379,61.56509934)(754.25710449,61.75510061)
\curveto(754.02711425,61.95509895)(753.84211443,62.17009874)(753.70210449,62.40010061)
\curveto(753.62211465,62.5100984)(753.55711472,62.62509828)(753.50710449,62.74510061)
\lineto(753.35710449,63.13510061)
\curveto(753.30711497,63.24509766)(753.277115,63.36009755)(753.26710449,63.48010061)
\curveto(753.24711503,63.60009731)(753.22211505,63.72509718)(753.19210449,63.85510061)
\curveto(753.19211508,63.92509698)(753.19211508,63.99009692)(753.19210449,64.05010061)
\curveto(753.18211509,64.1100968)(753.1721151,64.17509673)(753.16210449,64.24510061)
}
}
{
\newrgbcolor{curcolor}{0 0 0}
\pscustom[linestyle=none,fillstyle=solid,fillcolor=curcolor]
{
\newpath
\moveto(753.16210449,72.59470998)
\curveto(753.15211512,73.28470535)(753.272115,73.88470475)(753.52210449,74.39470998)
\curveto(753.7721145,74.91470372)(754.10711417,75.30970332)(754.52710449,75.57970998)
\curveto(754.60711367,75.629703)(754.69711358,75.67470296)(754.79710449,75.71470998)
\curveto(754.88711339,75.75470288)(754.98211329,75.79970283)(755.08210449,75.84970998)
\curveto(755.18211309,75.88970274)(755.28211299,75.91970271)(755.38210449,75.93970998)
\curveto(755.48211279,75.95970267)(755.58711269,75.97970265)(755.69710449,75.99970998)
\curveto(755.74711253,76.01970261)(755.79211248,76.02470261)(755.83210449,76.01470998)
\curveto(755.8721124,76.00470263)(755.91711236,76.00970262)(755.96710449,76.02970998)
\curveto(756.01711226,76.03970259)(756.10211217,76.04470259)(756.22210449,76.04470998)
\curveto(756.33211194,76.04470259)(756.41711186,76.03970259)(756.47710449,76.02970998)
\curveto(756.53711174,76.00970262)(756.59711168,75.99970263)(756.65710449,75.99970998)
\curveto(756.71711156,76.00970262)(756.7771115,76.00470263)(756.83710449,75.98470998)
\curveto(756.9771113,75.94470269)(757.11211116,75.90970272)(757.24210449,75.87970998)
\curveto(757.3721109,75.84970278)(757.49711078,75.80970282)(757.61710449,75.75970998)
\curveto(757.75711052,75.69970293)(757.88211039,75.629703)(757.99210449,75.54970998)
\curveto(758.10211017,75.47970315)(758.21211006,75.40470323)(758.32210449,75.32470998)
\lineto(758.38210449,75.26470998)
\curveto(758.40210987,75.25470338)(758.42210985,75.23970339)(758.44210449,75.21970998)
\curveto(758.60210967,75.09970353)(758.74710953,74.96470367)(758.87710449,74.81470998)
\curveto(759.00710927,74.66470397)(759.13210914,74.50470413)(759.25210449,74.33470998)
\curveto(759.4721088,74.02470461)(759.6771086,73.7297049)(759.86710449,73.44970998)
\curveto(760.00710827,73.21970541)(760.14210813,72.98970564)(760.27210449,72.75970998)
\curveto(760.40210787,72.53970609)(760.53710774,72.31970631)(760.67710449,72.09970998)
\curveto(760.84710743,71.84970678)(761.02710725,71.60970702)(761.21710449,71.37970998)
\curveto(761.40710687,71.15970747)(761.63210664,70.96970766)(761.89210449,70.80970998)
\curveto(761.95210632,70.76970786)(762.01210626,70.7347079)(762.07210449,70.70470998)
\curveto(762.12210615,70.67470796)(762.18710609,70.64470799)(762.26710449,70.61470998)
\curveto(762.33710594,70.59470804)(762.39710588,70.58970804)(762.44710449,70.59970998)
\curveto(762.51710576,70.61970801)(762.5721057,70.65470798)(762.61210449,70.70470998)
\curveto(762.64210563,70.75470788)(762.66210561,70.81470782)(762.67210449,70.88470998)
\lineto(762.67210449,71.12470998)
\lineto(762.67210449,71.87470998)
\lineto(762.67210449,74.67970998)
\lineto(762.67210449,75.33970998)
\curveto(762.6721056,75.4297032)(762.6771056,75.51470312)(762.68710449,75.59470998)
\curveto(762.68710559,75.67470296)(762.70710557,75.73970289)(762.74710449,75.78970998)
\curveto(762.78710549,75.83970279)(762.86210541,75.87970275)(762.97210449,75.90970998)
\curveto(763.0721052,75.94970268)(763.1721051,75.95970267)(763.27210449,75.93970998)
\lineto(763.40710449,75.93970998)
\curveto(763.4771048,75.91970271)(763.53710474,75.89970273)(763.58710449,75.87970998)
\curveto(763.63710464,75.85970277)(763.6771046,75.82470281)(763.70710449,75.77470998)
\curveto(763.74710453,75.72470291)(763.76710451,75.65470298)(763.76710449,75.56470998)
\lineto(763.76710449,75.29470998)
\lineto(763.76710449,74.39470998)
\lineto(763.76710449,70.88470998)
\lineto(763.76710449,69.81970998)
\curveto(763.76710451,69.73970889)(763.7721045,69.64970898)(763.78210449,69.54970998)
\curveto(763.78210449,69.44970918)(763.7721045,69.36470927)(763.75210449,69.29470998)
\curveto(763.68210459,69.08470955)(763.50210477,69.01970961)(763.21210449,69.09970998)
\curveto(763.1721051,69.10970952)(763.13710514,69.10970952)(763.10710449,69.09970998)
\curveto(763.06710521,69.09970953)(763.02210525,69.10970952)(762.97210449,69.12970998)
\curveto(762.89210538,69.14970948)(762.80710547,69.16970946)(762.71710449,69.18970998)
\curveto(762.62710565,69.20970942)(762.54210573,69.2347094)(762.46210449,69.26470998)
\curveto(761.9721063,69.42470921)(761.55710672,69.62470901)(761.21710449,69.86470998)
\curveto(760.96710731,70.04470859)(760.74210753,70.24970838)(760.54210449,70.47970998)
\curveto(760.33210794,70.70970792)(760.13710814,70.94970768)(759.95710449,71.19970998)
\curveto(759.7771085,71.45970717)(759.60710867,71.72470691)(759.44710449,71.99470998)
\curveto(759.277109,72.27470636)(759.10210917,72.54470609)(758.92210449,72.80470998)
\curveto(758.84210943,72.91470572)(758.76710951,73.01970561)(758.69710449,73.11970998)
\curveto(758.62710965,73.2297054)(758.55210972,73.33970529)(758.47210449,73.44970998)
\curveto(758.44210983,73.48970514)(758.41210986,73.52470511)(758.38210449,73.55470998)
\curveto(758.34210993,73.59470504)(758.31210996,73.634705)(758.29210449,73.67470998)
\curveto(758.18211009,73.81470482)(758.05711022,73.93970469)(757.91710449,74.04970998)
\curveto(757.88711039,74.06970456)(757.86211041,74.09470454)(757.84210449,74.12470998)
\curveto(757.81211046,74.15470448)(757.78211049,74.17970445)(757.75210449,74.19970998)
\curveto(757.65211062,74.27970435)(757.55211072,74.34470429)(757.45210449,74.39470998)
\curveto(757.35211092,74.45470418)(757.24211103,74.50970412)(757.12210449,74.55970998)
\curveto(757.05211122,74.58970404)(756.9771113,74.60970402)(756.89710449,74.61970998)
\lineto(756.65710449,74.67970998)
\lineto(756.56710449,74.67970998)
\curveto(756.53711174,74.68970394)(756.50711177,74.69470394)(756.47710449,74.69470998)
\curveto(756.40711187,74.71470392)(756.31211196,74.71970391)(756.19210449,74.70970998)
\curveto(756.06211221,74.70970392)(755.96211231,74.69970393)(755.89210449,74.67970998)
\curveto(755.81211246,74.65970397)(755.73711254,74.63970399)(755.66710449,74.61970998)
\curveto(755.58711269,74.60970402)(755.50711277,74.58970404)(755.42710449,74.55970998)
\curveto(755.18711309,74.44970418)(754.98711329,74.29970433)(754.82710449,74.10970998)
\curveto(754.65711362,73.9297047)(754.51711376,73.70970492)(754.40710449,73.44970998)
\curveto(754.38711389,73.37970525)(754.3721139,73.30970532)(754.36210449,73.23970998)
\curveto(754.34211393,73.16970546)(754.32211395,73.09470554)(754.30210449,73.01470998)
\curveto(754.28211399,72.9347057)(754.272114,72.82470581)(754.27210449,72.68470998)
\curveto(754.272114,72.55470608)(754.28211399,72.44970618)(754.30210449,72.36970998)
\curveto(754.31211396,72.30970632)(754.31711396,72.25470638)(754.31710449,72.20470998)
\curveto(754.31711396,72.15470648)(754.32711395,72.10470653)(754.34710449,72.05470998)
\curveto(754.38711389,71.95470668)(754.42711385,71.85970677)(754.46710449,71.76970998)
\curveto(754.50711377,71.68970694)(754.55211372,71.60970702)(754.60210449,71.52970998)
\curveto(754.62211365,71.49970713)(754.64711363,71.46970716)(754.67710449,71.43970998)
\curveto(754.70711357,71.41970721)(754.73211354,71.39470724)(754.75210449,71.36470998)
\lineto(754.82710449,71.28970998)
\curveto(754.84711343,71.25970737)(754.86711341,71.2347074)(754.88710449,71.21470998)
\lineto(755.09710449,71.06470998)
\curveto(755.15711312,71.02470761)(755.22211305,70.97970765)(755.29210449,70.92970998)
\curveto(755.38211289,70.86970776)(755.48711279,70.81970781)(755.60710449,70.77970998)
\curveto(755.71711256,70.74970788)(755.82711245,70.71470792)(755.93710449,70.67470998)
\curveto(756.04711223,70.634708)(756.19211208,70.60970802)(756.37210449,70.59970998)
\curveto(756.54211173,70.58970804)(756.66711161,70.55970807)(756.74710449,70.50970998)
\curveto(756.82711145,70.45970817)(756.8721114,70.38470825)(756.88210449,70.28470998)
\curveto(756.89211138,70.18470845)(756.89711138,70.07470856)(756.89710449,69.95470998)
\curveto(756.89711138,69.91470872)(756.90211137,69.87470876)(756.91210449,69.83470998)
\curveto(756.91211136,69.79470884)(756.90711137,69.75970887)(756.89710449,69.72970998)
\curveto(756.8771114,69.67970895)(756.86711141,69.629709)(756.86710449,69.57970998)
\curveto(756.86711141,69.53970909)(756.85711142,69.49970913)(756.83710449,69.45970998)
\curveto(756.7771115,69.36970926)(756.64211163,69.32470931)(756.43210449,69.32470998)
\lineto(756.31210449,69.32470998)
\curveto(756.25211202,69.3347093)(756.19211208,69.33970929)(756.13210449,69.33970998)
\curveto(756.06211221,69.34970928)(755.99711228,69.35970927)(755.93710449,69.36970998)
\curveto(755.82711245,69.38970924)(755.72711255,69.40970922)(755.63710449,69.42970998)
\curveto(755.53711274,69.44970918)(755.44211283,69.47970915)(755.35210449,69.51970998)
\curveto(755.28211299,69.53970909)(755.22211305,69.55970907)(755.17210449,69.57970998)
\lineto(754.99210449,69.63970998)
\curveto(754.73211354,69.75970887)(754.48711379,69.91470872)(754.25710449,70.10470998)
\curveto(754.02711425,70.30470833)(753.84211443,70.51970811)(753.70210449,70.74970998)
\curveto(753.62211465,70.85970777)(753.55711472,70.97470766)(753.50710449,71.09470998)
\lineto(753.35710449,71.48470998)
\curveto(753.30711497,71.59470704)(753.277115,71.70970692)(753.26710449,71.82970998)
\curveto(753.24711503,71.94970668)(753.22211505,72.07470656)(753.19210449,72.20470998)
\curveto(753.19211508,72.27470636)(753.19211508,72.33970629)(753.19210449,72.39970998)
\curveto(753.18211509,72.45970617)(753.1721151,72.52470611)(753.16210449,72.59470998)
}
}
{
\newrgbcolor{curcolor}{0 0 0}
\pscustom[linestyle=none,fillstyle=solid,fillcolor=curcolor]
{
\newpath
\moveto(762.13210449,78.63431936)
\lineto(762.13210449,79.26431936)
\lineto(762.13210449,79.45931936)
\curveto(762.13210614,79.52931683)(762.14210613,79.58931677)(762.16210449,79.63931936)
\curveto(762.20210607,79.70931665)(762.24210603,79.7593166)(762.28210449,79.78931936)
\curveto(762.33210594,79.82931653)(762.39710588,79.84931651)(762.47710449,79.84931936)
\curveto(762.55710572,79.8593165)(762.64210563,79.86431649)(762.73210449,79.86431936)
\lineto(763.45210449,79.86431936)
\curveto(763.93210434,79.86431649)(764.34210393,79.80431655)(764.68210449,79.68431936)
\curveto(765.02210325,79.56431679)(765.29710298,79.36931699)(765.50710449,79.09931936)
\curveto(765.55710272,79.02931733)(765.60210267,78.9593174)(765.64210449,78.88931936)
\curveto(765.69210258,78.82931753)(765.73710254,78.7543176)(765.77710449,78.66431936)
\curveto(765.78710249,78.64431771)(765.79710248,78.61431774)(765.80710449,78.57431936)
\curveto(765.82710245,78.53431782)(765.83210244,78.48931787)(765.82210449,78.43931936)
\curveto(765.79210248,78.34931801)(765.71710256,78.29431806)(765.59710449,78.27431936)
\curveto(765.48710279,78.2543181)(765.39210288,78.26931809)(765.31210449,78.31931936)
\curveto(765.24210303,78.34931801)(765.1771031,78.39431796)(765.11710449,78.45431936)
\curveto(765.06710321,78.52431783)(765.01710326,78.58931777)(764.96710449,78.64931936)
\curveto(764.91710336,78.71931764)(764.84210343,78.77931758)(764.74210449,78.82931936)
\curveto(764.65210362,78.88931747)(764.56210371,78.93931742)(764.47210449,78.97931936)
\curveto(764.44210383,78.99931736)(764.38210389,79.02431733)(764.29210449,79.05431936)
\curveto(764.21210406,79.08431727)(764.14210413,79.08931727)(764.08210449,79.06931936)
\curveto(763.94210433,79.03931732)(763.85210442,78.97931738)(763.81210449,78.88931936)
\curveto(763.78210449,78.80931755)(763.76710451,78.71931764)(763.76710449,78.61931936)
\curveto(763.76710451,78.51931784)(763.74210453,78.43431792)(763.69210449,78.36431936)
\curveto(763.62210465,78.27431808)(763.48210479,78.22931813)(763.27210449,78.22931936)
\lineto(762.71710449,78.22931936)
\lineto(762.49210449,78.22931936)
\curveto(762.41210586,78.23931812)(762.34710593,78.2593181)(762.29710449,78.28931936)
\curveto(762.21710606,78.34931801)(762.1721061,78.41931794)(762.16210449,78.49931936)
\curveto(762.15210612,78.51931784)(762.14710613,78.53931782)(762.14710449,78.55931936)
\curveto(762.14710613,78.58931777)(762.14210613,78.61431774)(762.13210449,78.63431936)
}
}
{
\newrgbcolor{curcolor}{0 0 0}
\pscustom[linestyle=none,fillstyle=solid,fillcolor=curcolor]
{
}
}
{
\newrgbcolor{curcolor}{0 0 0}
\pscustom[linestyle=none,fillstyle=solid,fillcolor=curcolor]
{
\newpath
\moveto(753.16210449,89.26463186)
\curveto(753.15211512,89.95462722)(753.272115,90.55462662)(753.52210449,91.06463186)
\curveto(753.7721145,91.58462559)(754.10711417,91.9796252)(754.52710449,92.24963186)
\curveto(754.60711367,92.29962488)(754.69711358,92.34462483)(754.79710449,92.38463186)
\curveto(754.88711339,92.42462475)(754.98211329,92.46962471)(755.08210449,92.51963186)
\curveto(755.18211309,92.55962462)(755.28211299,92.58962459)(755.38210449,92.60963186)
\curveto(755.48211279,92.62962455)(755.58711269,92.64962453)(755.69710449,92.66963186)
\curveto(755.74711253,92.68962449)(755.79211248,92.69462448)(755.83210449,92.68463186)
\curveto(755.8721124,92.6746245)(755.91711236,92.6796245)(755.96710449,92.69963186)
\curveto(756.01711226,92.70962447)(756.10211217,92.71462446)(756.22210449,92.71463186)
\curveto(756.33211194,92.71462446)(756.41711186,92.70962447)(756.47710449,92.69963186)
\curveto(756.53711174,92.6796245)(756.59711168,92.66962451)(756.65710449,92.66963186)
\curveto(756.71711156,92.6796245)(756.7771115,92.6746245)(756.83710449,92.65463186)
\curveto(756.9771113,92.61462456)(757.11211116,92.5796246)(757.24210449,92.54963186)
\curveto(757.3721109,92.51962466)(757.49711078,92.4796247)(757.61710449,92.42963186)
\curveto(757.75711052,92.36962481)(757.88211039,92.29962488)(757.99210449,92.21963186)
\curveto(758.10211017,92.14962503)(758.21211006,92.0746251)(758.32210449,91.99463186)
\lineto(758.38210449,91.93463186)
\curveto(758.40210987,91.92462525)(758.42210985,91.90962527)(758.44210449,91.88963186)
\curveto(758.60210967,91.76962541)(758.74710953,91.63462554)(758.87710449,91.48463186)
\curveto(759.00710927,91.33462584)(759.13210914,91.174626)(759.25210449,91.00463186)
\curveto(759.4721088,90.69462648)(759.6771086,90.39962678)(759.86710449,90.11963186)
\curveto(760.00710827,89.88962729)(760.14210813,89.65962752)(760.27210449,89.42963186)
\curveto(760.40210787,89.20962797)(760.53710774,88.98962819)(760.67710449,88.76963186)
\curveto(760.84710743,88.51962866)(761.02710725,88.2796289)(761.21710449,88.04963186)
\curveto(761.40710687,87.82962935)(761.63210664,87.63962954)(761.89210449,87.47963186)
\curveto(761.95210632,87.43962974)(762.01210626,87.40462977)(762.07210449,87.37463186)
\curveto(762.12210615,87.34462983)(762.18710609,87.31462986)(762.26710449,87.28463186)
\curveto(762.33710594,87.26462991)(762.39710588,87.25962992)(762.44710449,87.26963186)
\curveto(762.51710576,87.28962989)(762.5721057,87.32462985)(762.61210449,87.37463186)
\curveto(762.64210563,87.42462975)(762.66210561,87.48462969)(762.67210449,87.55463186)
\lineto(762.67210449,87.79463186)
\lineto(762.67210449,88.54463186)
\lineto(762.67210449,91.34963186)
\lineto(762.67210449,92.00963186)
\curveto(762.6721056,92.09962508)(762.6771056,92.18462499)(762.68710449,92.26463186)
\curveto(762.68710559,92.34462483)(762.70710557,92.40962477)(762.74710449,92.45963186)
\curveto(762.78710549,92.50962467)(762.86210541,92.54962463)(762.97210449,92.57963186)
\curveto(763.0721052,92.61962456)(763.1721051,92.62962455)(763.27210449,92.60963186)
\lineto(763.40710449,92.60963186)
\curveto(763.4771048,92.58962459)(763.53710474,92.56962461)(763.58710449,92.54963186)
\curveto(763.63710464,92.52962465)(763.6771046,92.49462468)(763.70710449,92.44463186)
\curveto(763.74710453,92.39462478)(763.76710451,92.32462485)(763.76710449,92.23463186)
\lineto(763.76710449,91.96463186)
\lineto(763.76710449,91.06463186)
\lineto(763.76710449,87.55463186)
\lineto(763.76710449,86.48963186)
\curveto(763.76710451,86.40963077)(763.7721045,86.31963086)(763.78210449,86.21963186)
\curveto(763.78210449,86.11963106)(763.7721045,86.03463114)(763.75210449,85.96463186)
\curveto(763.68210459,85.75463142)(763.50210477,85.68963149)(763.21210449,85.76963186)
\curveto(763.1721051,85.7796314)(763.13710514,85.7796314)(763.10710449,85.76963186)
\curveto(763.06710521,85.76963141)(763.02210525,85.7796314)(762.97210449,85.79963186)
\curveto(762.89210538,85.81963136)(762.80710547,85.83963134)(762.71710449,85.85963186)
\curveto(762.62710565,85.8796313)(762.54210573,85.90463127)(762.46210449,85.93463186)
\curveto(761.9721063,86.09463108)(761.55710672,86.29463088)(761.21710449,86.53463186)
\curveto(760.96710731,86.71463046)(760.74210753,86.91963026)(760.54210449,87.14963186)
\curveto(760.33210794,87.3796298)(760.13710814,87.61962956)(759.95710449,87.86963186)
\curveto(759.7771085,88.12962905)(759.60710867,88.39462878)(759.44710449,88.66463186)
\curveto(759.277109,88.94462823)(759.10210917,89.21462796)(758.92210449,89.47463186)
\curveto(758.84210943,89.58462759)(758.76710951,89.68962749)(758.69710449,89.78963186)
\curveto(758.62710965,89.89962728)(758.55210972,90.00962717)(758.47210449,90.11963186)
\curveto(758.44210983,90.15962702)(758.41210986,90.19462698)(758.38210449,90.22463186)
\curveto(758.34210993,90.26462691)(758.31210996,90.30462687)(758.29210449,90.34463186)
\curveto(758.18211009,90.48462669)(758.05711022,90.60962657)(757.91710449,90.71963186)
\curveto(757.88711039,90.73962644)(757.86211041,90.76462641)(757.84210449,90.79463186)
\curveto(757.81211046,90.82462635)(757.78211049,90.84962633)(757.75210449,90.86963186)
\curveto(757.65211062,90.94962623)(757.55211072,91.01462616)(757.45210449,91.06463186)
\curveto(757.35211092,91.12462605)(757.24211103,91.179626)(757.12210449,91.22963186)
\curveto(757.05211122,91.25962592)(756.9771113,91.2796259)(756.89710449,91.28963186)
\lineto(756.65710449,91.34963186)
\lineto(756.56710449,91.34963186)
\curveto(756.53711174,91.35962582)(756.50711177,91.36462581)(756.47710449,91.36463186)
\curveto(756.40711187,91.38462579)(756.31211196,91.38962579)(756.19210449,91.37963186)
\curveto(756.06211221,91.3796258)(755.96211231,91.36962581)(755.89210449,91.34963186)
\curveto(755.81211246,91.32962585)(755.73711254,91.30962587)(755.66710449,91.28963186)
\curveto(755.58711269,91.2796259)(755.50711277,91.25962592)(755.42710449,91.22963186)
\curveto(755.18711309,91.11962606)(754.98711329,90.96962621)(754.82710449,90.77963186)
\curveto(754.65711362,90.59962658)(754.51711376,90.3796268)(754.40710449,90.11963186)
\curveto(754.38711389,90.04962713)(754.3721139,89.9796272)(754.36210449,89.90963186)
\curveto(754.34211393,89.83962734)(754.32211395,89.76462741)(754.30210449,89.68463186)
\curveto(754.28211399,89.60462757)(754.272114,89.49462768)(754.27210449,89.35463186)
\curveto(754.272114,89.22462795)(754.28211399,89.11962806)(754.30210449,89.03963186)
\curveto(754.31211396,88.9796282)(754.31711396,88.92462825)(754.31710449,88.87463186)
\curveto(754.31711396,88.82462835)(754.32711395,88.7746284)(754.34710449,88.72463186)
\curveto(754.38711389,88.62462855)(754.42711385,88.52962865)(754.46710449,88.43963186)
\curveto(754.50711377,88.35962882)(754.55211372,88.2796289)(754.60210449,88.19963186)
\curveto(754.62211365,88.16962901)(754.64711363,88.13962904)(754.67710449,88.10963186)
\curveto(754.70711357,88.08962909)(754.73211354,88.06462911)(754.75210449,88.03463186)
\lineto(754.82710449,87.95963186)
\curveto(754.84711343,87.92962925)(754.86711341,87.90462927)(754.88710449,87.88463186)
\lineto(755.09710449,87.73463186)
\curveto(755.15711312,87.69462948)(755.22211305,87.64962953)(755.29210449,87.59963186)
\curveto(755.38211289,87.53962964)(755.48711279,87.48962969)(755.60710449,87.44963186)
\curveto(755.71711256,87.41962976)(755.82711245,87.38462979)(755.93710449,87.34463186)
\curveto(756.04711223,87.30462987)(756.19211208,87.2796299)(756.37210449,87.26963186)
\curveto(756.54211173,87.25962992)(756.66711161,87.22962995)(756.74710449,87.17963186)
\curveto(756.82711145,87.12963005)(756.8721114,87.05463012)(756.88210449,86.95463186)
\curveto(756.89211138,86.85463032)(756.89711138,86.74463043)(756.89710449,86.62463186)
\curveto(756.89711138,86.58463059)(756.90211137,86.54463063)(756.91210449,86.50463186)
\curveto(756.91211136,86.46463071)(756.90711137,86.42963075)(756.89710449,86.39963186)
\curveto(756.8771114,86.34963083)(756.86711141,86.29963088)(756.86710449,86.24963186)
\curveto(756.86711141,86.20963097)(756.85711142,86.16963101)(756.83710449,86.12963186)
\curveto(756.7771115,86.03963114)(756.64211163,85.99463118)(756.43210449,85.99463186)
\lineto(756.31210449,85.99463186)
\curveto(756.25211202,86.00463117)(756.19211208,86.00963117)(756.13210449,86.00963186)
\curveto(756.06211221,86.01963116)(755.99711228,86.02963115)(755.93710449,86.03963186)
\curveto(755.82711245,86.05963112)(755.72711255,86.0796311)(755.63710449,86.09963186)
\curveto(755.53711274,86.11963106)(755.44211283,86.14963103)(755.35210449,86.18963186)
\curveto(755.28211299,86.20963097)(755.22211305,86.22963095)(755.17210449,86.24963186)
\lineto(754.99210449,86.30963186)
\curveto(754.73211354,86.42963075)(754.48711379,86.58463059)(754.25710449,86.77463186)
\curveto(754.02711425,86.9746302)(753.84211443,87.18962999)(753.70210449,87.41963186)
\curveto(753.62211465,87.52962965)(753.55711472,87.64462953)(753.50710449,87.76463186)
\lineto(753.35710449,88.15463186)
\curveto(753.30711497,88.26462891)(753.277115,88.3796288)(753.26710449,88.49963186)
\curveto(753.24711503,88.61962856)(753.22211505,88.74462843)(753.19210449,88.87463186)
\curveto(753.19211508,88.94462823)(753.19211508,89.00962817)(753.19210449,89.06963186)
\curveto(753.18211509,89.12962805)(753.1721151,89.19462798)(753.16210449,89.26463186)
}
}
{
\newrgbcolor{curcolor}{0 0 0}
\pscustom[linestyle=none,fillstyle=solid,fillcolor=curcolor]
{
\newpath
\moveto(758.68210449,101.36424123)
\lineto(758.93710449,101.36424123)
\curveto(759.01710926,101.37423353)(759.09210918,101.36923353)(759.16210449,101.34924123)
\lineto(759.40210449,101.34924123)
\lineto(759.56710449,101.34924123)
\curveto(759.66710861,101.32923357)(759.7721085,101.31923358)(759.88210449,101.31924123)
\curveto(759.98210829,101.31923358)(760.08210819,101.30923359)(760.18210449,101.28924123)
\lineto(760.33210449,101.28924123)
\curveto(760.4721078,101.25923364)(760.61210766,101.23923366)(760.75210449,101.22924123)
\curveto(760.88210739,101.21923368)(761.01210726,101.19423371)(761.14210449,101.15424123)
\curveto(761.22210705,101.13423377)(761.30710697,101.11423379)(761.39710449,101.09424123)
\lineto(761.63710449,101.03424123)
\lineto(761.93710449,100.91424123)
\curveto(762.02710625,100.88423402)(762.11710616,100.84923405)(762.20710449,100.80924123)
\curveto(762.42710585,100.70923419)(762.64210563,100.57423433)(762.85210449,100.40424123)
\curveto(763.06210521,100.24423466)(763.23210504,100.06923483)(763.36210449,99.87924123)
\curveto(763.40210487,99.82923507)(763.44210483,99.76923513)(763.48210449,99.69924123)
\curveto(763.51210476,99.63923526)(763.54710473,99.57923532)(763.58710449,99.51924123)
\curveto(763.63710464,99.43923546)(763.6771046,99.34423556)(763.70710449,99.23424123)
\curveto(763.73710454,99.12423578)(763.76710451,99.01923588)(763.79710449,98.91924123)
\curveto(763.83710444,98.80923609)(763.86210441,98.6992362)(763.87210449,98.58924123)
\curveto(763.88210439,98.47923642)(763.89710438,98.36423654)(763.91710449,98.24424123)
\curveto(763.92710435,98.2042367)(763.92710435,98.15923674)(763.91710449,98.10924123)
\curveto(763.91710436,98.06923683)(763.92210435,98.02923687)(763.93210449,97.98924123)
\curveto(763.94210433,97.94923695)(763.94710433,97.89423701)(763.94710449,97.82424123)
\curveto(763.94710433,97.75423715)(763.94210433,97.7042372)(763.93210449,97.67424123)
\curveto(763.91210436,97.62423728)(763.90710437,97.57923732)(763.91710449,97.53924123)
\curveto(763.92710435,97.4992374)(763.92710435,97.46423744)(763.91710449,97.43424123)
\lineto(763.91710449,97.34424123)
\curveto(763.89710438,97.28423762)(763.88210439,97.21923768)(763.87210449,97.14924123)
\curveto(763.8721044,97.08923781)(763.86710441,97.02423788)(763.85710449,96.95424123)
\curveto(763.80710447,96.78423812)(763.75710452,96.62423828)(763.70710449,96.47424123)
\curveto(763.65710462,96.32423858)(763.59210468,96.17923872)(763.51210449,96.03924123)
\curveto(763.4721048,95.98923891)(763.44210483,95.93423897)(763.42210449,95.87424123)
\curveto(763.39210488,95.82423908)(763.35710492,95.77423913)(763.31710449,95.72424123)
\curveto(763.13710514,95.48423942)(762.91710536,95.28423962)(762.65710449,95.12424123)
\curveto(762.39710588,94.96423994)(762.11210616,94.82424008)(761.80210449,94.70424123)
\curveto(761.66210661,94.64424026)(761.52210675,94.5992403)(761.38210449,94.56924123)
\curveto(761.23210704,94.53924036)(761.0771072,94.5042404)(760.91710449,94.46424123)
\curveto(760.80710747,94.44424046)(760.69710758,94.42924047)(760.58710449,94.41924123)
\curveto(760.4771078,94.40924049)(760.36710791,94.39424051)(760.25710449,94.37424123)
\curveto(760.21710806,94.36424054)(760.1771081,94.35924054)(760.13710449,94.35924123)
\curveto(760.09710818,94.36924053)(760.05710822,94.36924053)(760.01710449,94.35924123)
\curveto(759.96710831,94.34924055)(759.91710836,94.34424056)(759.86710449,94.34424123)
\lineto(759.70210449,94.34424123)
\curveto(759.65210862,94.32424058)(759.60210867,94.31924058)(759.55210449,94.32924123)
\curveto(759.49210878,94.33924056)(759.43710884,94.33924056)(759.38710449,94.32924123)
\curveto(759.34710893,94.31924058)(759.30210897,94.31924058)(759.25210449,94.32924123)
\curveto(759.20210907,94.33924056)(759.15210912,94.33424057)(759.10210449,94.31424123)
\curveto(759.03210924,94.29424061)(758.95710932,94.28924061)(758.87710449,94.29924123)
\curveto(758.78710949,94.30924059)(758.70210957,94.31424059)(758.62210449,94.31424123)
\curveto(758.53210974,94.31424059)(758.43210984,94.30924059)(758.32210449,94.29924123)
\curveto(758.20211007,94.28924061)(758.10211017,94.29424061)(758.02210449,94.31424123)
\lineto(757.73710449,94.31424123)
\lineto(757.10710449,94.35924123)
\curveto(757.00711127,94.36924053)(756.91211136,94.37924052)(756.82210449,94.38924123)
\lineto(756.52210449,94.41924123)
\curveto(756.4721118,94.43924046)(756.42211185,94.44424046)(756.37210449,94.43424123)
\curveto(756.31211196,94.43424047)(756.25711202,94.44424046)(756.20710449,94.46424123)
\curveto(756.03711224,94.51424039)(755.8721124,94.55424035)(755.71210449,94.58424123)
\curveto(755.54211273,94.61424029)(755.38211289,94.66424024)(755.23210449,94.73424123)
\curveto(754.7721135,94.92423998)(754.39711388,95.14423976)(754.10710449,95.39424123)
\curveto(753.81711446,95.65423925)(753.5721147,96.01423889)(753.37210449,96.47424123)
\curveto(753.32211495,96.6042383)(753.28711499,96.73423817)(753.26710449,96.86424123)
\curveto(753.24711503,97.0042379)(753.22211505,97.14423776)(753.19210449,97.28424123)
\curveto(753.18211509,97.35423755)(753.1771151,97.41923748)(753.17710449,97.47924123)
\curveto(753.1771151,97.53923736)(753.1721151,97.6042373)(753.16210449,97.67424123)
\curveto(753.14211513,98.5042364)(753.29211498,99.17423573)(753.61210449,99.68424123)
\curveto(753.92211435,100.19423471)(754.36211391,100.57423433)(754.93210449,100.82424123)
\curveto(755.05211322,100.87423403)(755.1771131,100.91923398)(755.30710449,100.95924123)
\curveto(755.43711284,100.9992339)(755.5721127,101.04423386)(755.71210449,101.09424123)
\curveto(755.79211248,101.11423379)(755.8771124,101.12923377)(755.96710449,101.13924123)
\lineto(756.20710449,101.19924123)
\curveto(756.31711196,101.22923367)(756.42711185,101.24423366)(756.53710449,101.24424123)
\curveto(756.64711163,101.25423365)(756.75711152,101.26923363)(756.86710449,101.28924123)
\curveto(756.91711136,101.30923359)(756.96211131,101.31423359)(757.00210449,101.30424123)
\curveto(757.04211123,101.3042336)(757.08211119,101.30923359)(757.12210449,101.31924123)
\curveto(757.1721111,101.32923357)(757.22711105,101.32923357)(757.28710449,101.31924123)
\curveto(757.33711094,101.31923358)(757.38711089,101.32423358)(757.43710449,101.33424123)
\lineto(757.57210449,101.33424123)
\curveto(757.63211064,101.35423355)(757.70211057,101.35423355)(757.78210449,101.33424123)
\curveto(757.85211042,101.32423358)(757.91711036,101.32923357)(757.97710449,101.34924123)
\curveto(758.00711027,101.35923354)(758.04711023,101.36423354)(758.09710449,101.36424123)
\lineto(758.21710449,101.36424123)
\lineto(758.68210449,101.36424123)
\moveto(761.00710449,99.81924123)
\curveto(760.68710759,99.91923498)(760.32210795,99.97923492)(759.91210449,99.99924123)
\curveto(759.50210877,100.01923488)(759.09210918,100.02923487)(758.68210449,100.02924123)
\curveto(758.25211002,100.02923487)(757.83211044,100.01923488)(757.42210449,99.99924123)
\curveto(757.01211126,99.97923492)(756.62711165,99.93423497)(756.26710449,99.86424123)
\curveto(755.90711237,99.79423511)(755.58711269,99.68423522)(755.30710449,99.53424123)
\curveto(755.01711326,99.39423551)(754.78211349,99.1992357)(754.60210449,98.94924123)
\curveto(754.49211378,98.78923611)(754.41211386,98.60923629)(754.36210449,98.40924123)
\curveto(754.30211397,98.20923669)(754.272114,97.96423694)(754.27210449,97.67424123)
\curveto(754.29211398,97.65423725)(754.30211397,97.61923728)(754.30210449,97.56924123)
\curveto(754.29211398,97.51923738)(754.29211398,97.47923742)(754.30210449,97.44924123)
\curveto(754.32211395,97.36923753)(754.34211393,97.29423761)(754.36210449,97.22424123)
\curveto(754.3721139,97.16423774)(754.39211388,97.0992378)(754.42210449,97.02924123)
\curveto(754.54211373,96.75923814)(754.71211356,96.53923836)(754.93210449,96.36924123)
\curveto(755.14211313,96.20923869)(755.38711289,96.07423883)(755.66710449,95.96424123)
\curveto(755.7771125,95.91423899)(755.89711238,95.87423903)(756.02710449,95.84424123)
\curveto(756.14711213,95.82423908)(756.272112,95.7992391)(756.40210449,95.76924123)
\curveto(756.45211182,95.74923915)(756.50711177,95.73923916)(756.56710449,95.73924123)
\curveto(756.61711166,95.73923916)(756.66711161,95.73423917)(756.71710449,95.72424123)
\curveto(756.80711147,95.71423919)(756.90211137,95.7042392)(757.00210449,95.69424123)
\curveto(757.09211118,95.68423922)(757.18711109,95.67423923)(757.28710449,95.66424123)
\curveto(757.36711091,95.66423924)(757.45211082,95.65923924)(757.54210449,95.64924123)
\lineto(757.78210449,95.64924123)
\lineto(757.96210449,95.64924123)
\curveto(757.99211028,95.63923926)(758.02711025,95.63423927)(758.06710449,95.63424123)
\lineto(758.20210449,95.63424123)
\lineto(758.65210449,95.63424123)
\curveto(758.73210954,95.63423927)(758.81710946,95.62923927)(758.90710449,95.61924123)
\curveto(758.98710929,95.61923928)(759.06210921,95.62923927)(759.13210449,95.64924123)
\lineto(759.40210449,95.64924123)
\curveto(759.42210885,95.64923925)(759.45210882,95.64423926)(759.49210449,95.63424123)
\curveto(759.52210875,95.63423927)(759.54710873,95.63923926)(759.56710449,95.64924123)
\curveto(759.66710861,95.65923924)(759.76710851,95.66423924)(759.86710449,95.66424123)
\curveto(759.95710832,95.67423923)(760.05710822,95.68423922)(760.16710449,95.69424123)
\curveto(760.28710799,95.72423918)(760.41210786,95.73923916)(760.54210449,95.73924123)
\curveto(760.66210761,95.74923915)(760.7771075,95.77423913)(760.88710449,95.81424123)
\curveto(761.18710709,95.89423901)(761.45210682,95.97923892)(761.68210449,96.06924123)
\curveto(761.91210636,96.16923873)(762.12710615,96.31423859)(762.32710449,96.50424123)
\curveto(762.52710575,96.71423819)(762.6771056,96.97923792)(762.77710449,97.29924123)
\curveto(762.79710548,97.33923756)(762.80710547,97.37423753)(762.80710449,97.40424123)
\curveto(762.79710548,97.44423746)(762.80210547,97.48923741)(762.82210449,97.53924123)
\curveto(762.83210544,97.57923732)(762.84210543,97.64923725)(762.85210449,97.74924123)
\curveto(762.86210541,97.85923704)(762.85710542,97.94423696)(762.83710449,98.00424123)
\curveto(762.81710546,98.07423683)(762.80710547,98.14423676)(762.80710449,98.21424123)
\curveto(762.79710548,98.28423662)(762.78210549,98.34923655)(762.76210449,98.40924123)
\curveto(762.70210557,98.60923629)(762.61710566,98.78923611)(762.50710449,98.94924123)
\curveto(762.48710579,98.97923592)(762.46710581,99.0042359)(762.44710449,99.02424123)
\lineto(762.38710449,99.08424123)
\curveto(762.36710591,99.12423578)(762.32710595,99.17423573)(762.26710449,99.23424123)
\curveto(762.12710615,99.33423557)(761.99710628,99.41923548)(761.87710449,99.48924123)
\curveto(761.75710652,99.55923534)(761.61210666,99.62923527)(761.44210449,99.69924123)
\curveto(761.3721069,99.72923517)(761.30210697,99.74923515)(761.23210449,99.75924123)
\curveto(761.16210711,99.77923512)(761.08710719,99.7992351)(761.00710449,99.81924123)
}
}
{
\newrgbcolor{curcolor}{0 0 0}
\pscustom[linestyle=none,fillstyle=solid,fillcolor=curcolor]
{
\newpath
\moveto(753.16210449,106.77385061)
\curveto(753.16211511,106.87384575)(753.1721151,106.96884566)(753.19210449,107.05885061)
\curveto(753.20211507,107.14884548)(753.23211504,107.21384541)(753.28210449,107.25385061)
\curveto(753.36211491,107.31384531)(753.46711481,107.34384528)(753.59710449,107.34385061)
\lineto(753.98710449,107.34385061)
\lineto(755.48710449,107.34385061)
\lineto(761.87710449,107.34385061)
\lineto(763.04710449,107.34385061)
\lineto(763.36210449,107.34385061)
\curveto(763.46210481,107.35384527)(763.54210473,107.33884529)(763.60210449,107.29885061)
\curveto(763.68210459,107.24884538)(763.73210454,107.17384545)(763.75210449,107.07385061)
\curveto(763.76210451,106.98384564)(763.76710451,106.87384575)(763.76710449,106.74385061)
\lineto(763.76710449,106.51885061)
\curveto(763.74710453,106.43884619)(763.73210454,106.36884626)(763.72210449,106.30885061)
\curveto(763.70210457,106.24884638)(763.66210461,106.19884643)(763.60210449,106.15885061)
\curveto(763.54210473,106.11884651)(763.46710481,106.09884653)(763.37710449,106.09885061)
\lineto(763.07710449,106.09885061)
\lineto(761.98210449,106.09885061)
\lineto(756.64210449,106.09885061)
\curveto(756.55211172,106.07884655)(756.4771118,106.06384656)(756.41710449,106.05385061)
\curveto(756.34711193,106.05384657)(756.28711199,106.0238466)(756.23710449,105.96385061)
\curveto(756.18711209,105.89384673)(756.16211211,105.80384682)(756.16210449,105.69385061)
\curveto(756.15211212,105.59384703)(756.14711213,105.48384714)(756.14710449,105.36385061)
\lineto(756.14710449,104.22385061)
\lineto(756.14710449,103.72885061)
\curveto(756.13711214,103.56884906)(756.0771122,103.45884917)(755.96710449,103.39885061)
\curveto(755.93711234,103.37884925)(755.90711237,103.36884926)(755.87710449,103.36885061)
\curveto(755.83711244,103.36884926)(755.79211248,103.36384926)(755.74210449,103.35385061)
\curveto(755.62211265,103.33384929)(755.51211276,103.33884929)(755.41210449,103.36885061)
\curveto(755.31211296,103.40884922)(755.24211303,103.46384916)(755.20210449,103.53385061)
\curveto(755.15211312,103.61384901)(755.12711315,103.73384889)(755.12710449,103.89385061)
\curveto(755.12711315,104.05384857)(755.11211316,104.18884844)(755.08210449,104.29885061)
\curveto(755.0721132,104.34884828)(755.06711321,104.40384822)(755.06710449,104.46385061)
\curveto(755.05711322,104.5238481)(755.04211323,104.58384804)(755.02210449,104.64385061)
\curveto(754.9721133,104.79384783)(754.92211335,104.93884769)(754.87210449,105.07885061)
\curveto(754.81211346,105.21884741)(754.74211353,105.35384727)(754.66210449,105.48385061)
\curveto(754.5721137,105.623847)(754.46711381,105.74384688)(754.34710449,105.84385061)
\curveto(754.22711405,105.94384668)(754.09711418,106.03884659)(753.95710449,106.12885061)
\curveto(753.85711442,106.18884644)(753.74711453,106.23384639)(753.62710449,106.26385061)
\curveto(753.50711477,106.30384632)(753.40211487,106.35384627)(753.31210449,106.41385061)
\curveto(753.25211502,106.46384616)(753.21211506,106.53384609)(753.19210449,106.62385061)
\curveto(753.18211509,106.64384598)(753.1771151,106.66884596)(753.17710449,106.69885061)
\curveto(753.1771151,106.7288459)(753.1721151,106.75384587)(753.16210449,106.77385061)
}
}
{
\newrgbcolor{curcolor}{0 0 0}
\pscustom[linestyle=none,fillstyle=solid,fillcolor=curcolor]
{
\newpath
\moveto(753.16210449,115.12345998)
\curveto(753.16211511,115.22345513)(753.1721151,115.31845503)(753.19210449,115.40845998)
\curveto(753.20211507,115.49845485)(753.23211504,115.56345479)(753.28210449,115.60345998)
\curveto(753.36211491,115.66345469)(753.46711481,115.69345466)(753.59710449,115.69345998)
\lineto(753.98710449,115.69345998)
\lineto(755.48710449,115.69345998)
\lineto(761.87710449,115.69345998)
\lineto(763.04710449,115.69345998)
\lineto(763.36210449,115.69345998)
\curveto(763.46210481,115.70345465)(763.54210473,115.68845466)(763.60210449,115.64845998)
\curveto(763.68210459,115.59845475)(763.73210454,115.52345483)(763.75210449,115.42345998)
\curveto(763.76210451,115.33345502)(763.76710451,115.22345513)(763.76710449,115.09345998)
\lineto(763.76710449,114.86845998)
\curveto(763.74710453,114.78845556)(763.73210454,114.71845563)(763.72210449,114.65845998)
\curveto(763.70210457,114.59845575)(763.66210461,114.5484558)(763.60210449,114.50845998)
\curveto(763.54210473,114.46845588)(763.46710481,114.4484559)(763.37710449,114.44845998)
\lineto(763.07710449,114.44845998)
\lineto(761.98210449,114.44845998)
\lineto(756.64210449,114.44845998)
\curveto(756.55211172,114.42845592)(756.4771118,114.41345594)(756.41710449,114.40345998)
\curveto(756.34711193,114.40345595)(756.28711199,114.37345598)(756.23710449,114.31345998)
\curveto(756.18711209,114.24345611)(756.16211211,114.1534562)(756.16210449,114.04345998)
\curveto(756.15211212,113.94345641)(756.14711213,113.83345652)(756.14710449,113.71345998)
\lineto(756.14710449,112.57345998)
\lineto(756.14710449,112.07845998)
\curveto(756.13711214,111.91845843)(756.0771122,111.80845854)(755.96710449,111.74845998)
\curveto(755.93711234,111.72845862)(755.90711237,111.71845863)(755.87710449,111.71845998)
\curveto(755.83711244,111.71845863)(755.79211248,111.71345864)(755.74210449,111.70345998)
\curveto(755.62211265,111.68345867)(755.51211276,111.68845866)(755.41210449,111.71845998)
\curveto(755.31211296,111.75845859)(755.24211303,111.81345854)(755.20210449,111.88345998)
\curveto(755.15211312,111.96345839)(755.12711315,112.08345827)(755.12710449,112.24345998)
\curveto(755.12711315,112.40345795)(755.11211316,112.53845781)(755.08210449,112.64845998)
\curveto(755.0721132,112.69845765)(755.06711321,112.7534576)(755.06710449,112.81345998)
\curveto(755.05711322,112.87345748)(755.04211323,112.93345742)(755.02210449,112.99345998)
\curveto(754.9721133,113.14345721)(754.92211335,113.28845706)(754.87210449,113.42845998)
\curveto(754.81211346,113.56845678)(754.74211353,113.70345665)(754.66210449,113.83345998)
\curveto(754.5721137,113.97345638)(754.46711381,114.09345626)(754.34710449,114.19345998)
\curveto(754.22711405,114.29345606)(754.09711418,114.38845596)(753.95710449,114.47845998)
\curveto(753.85711442,114.53845581)(753.74711453,114.58345577)(753.62710449,114.61345998)
\curveto(753.50711477,114.6534557)(753.40211487,114.70345565)(753.31210449,114.76345998)
\curveto(753.25211502,114.81345554)(753.21211506,114.88345547)(753.19210449,114.97345998)
\curveto(753.18211509,114.99345536)(753.1771151,115.01845533)(753.17710449,115.04845998)
\curveto(753.1771151,115.07845527)(753.1721151,115.10345525)(753.16210449,115.12345998)
}
}
{
\newrgbcolor{curcolor}{0 0 0}
\pscustom[linestyle=none,fillstyle=solid,fillcolor=curcolor]
{
\newpath
\moveto(773.99842041,37.28705373)
\curveto(773.99843111,37.35704805)(773.99843111,37.43704797)(773.99842041,37.52705373)
\curveto(773.98843112,37.61704779)(773.98843112,37.70204771)(773.99842041,37.78205373)
\curveto(773.99843111,37.87204754)(774.0084311,37.95204746)(774.02842041,38.02205373)
\curveto(774.04843106,38.10204731)(774.07843103,38.15704725)(774.11842041,38.18705373)
\curveto(774.16843094,38.21704719)(774.24343086,38.23704717)(774.34342041,38.24705373)
\curveto(774.43343067,38.26704714)(774.53843057,38.27704713)(774.65842041,38.27705373)
\curveto(774.76843034,38.28704712)(774.88343022,38.28704712)(775.00342041,38.27705373)
\lineto(775.30342041,38.27705373)
\lineto(778.31842041,38.27705373)
\lineto(781.21342041,38.27705373)
\curveto(781.54342356,38.27704713)(781.86842324,38.27204714)(782.18842041,38.26205373)
\curveto(782.49842261,38.26204715)(782.77842233,38.22204719)(783.02842041,38.14205373)
\curveto(783.37842173,38.02204739)(783.67342143,37.86704754)(783.91342041,37.67705373)
\curveto(784.14342096,37.48704792)(784.34342076,37.24704816)(784.51342041,36.95705373)
\curveto(784.56342054,36.89704851)(784.59842051,36.83204858)(784.61842041,36.76205373)
\curveto(784.63842047,36.70204871)(784.66342044,36.63204878)(784.69342041,36.55205373)
\curveto(784.74342036,36.43204898)(784.77842033,36.30204911)(784.79842041,36.16205373)
\curveto(784.82842028,36.03204938)(784.85842025,35.89704951)(784.88842041,35.75705373)
\curveto(784.9084202,35.7070497)(784.91342019,35.65704975)(784.90342041,35.60705373)
\curveto(784.89342021,35.55704985)(784.89342021,35.50204991)(784.90342041,35.44205373)
\curveto(784.91342019,35.42204999)(784.91342019,35.39705001)(784.90342041,35.36705373)
\curveto(784.9034202,35.33705007)(784.9084202,35.3120501)(784.91842041,35.29205373)
\curveto(784.92842018,35.25205016)(784.93342017,35.19705021)(784.93342041,35.12705373)
\curveto(784.93342017,35.05705035)(784.92842018,35.00205041)(784.91842041,34.96205373)
\curveto(784.9084202,34.9120505)(784.9084202,34.85705055)(784.91842041,34.79705373)
\curveto(784.92842018,34.73705067)(784.92342018,34.68205073)(784.90342041,34.63205373)
\curveto(784.87342023,34.50205091)(784.85342025,34.37705103)(784.84342041,34.25705373)
\curveto(784.83342027,34.13705127)(784.8084203,34.02205139)(784.76842041,33.91205373)
\curveto(784.64842046,33.54205187)(784.47842063,33.22205219)(784.25842041,32.95205373)
\curveto(784.03842107,32.68205273)(783.75842135,32.47205294)(783.41842041,32.32205373)
\curveto(783.29842181,32.27205314)(783.17342193,32.22705318)(783.04342041,32.18705373)
\curveto(782.91342219,32.15705325)(782.77842233,32.12205329)(782.63842041,32.08205373)
\curveto(782.58842252,32.07205334)(782.54842256,32.06705334)(782.51842041,32.06705373)
\curveto(782.47842263,32.06705334)(782.43342267,32.06205335)(782.38342041,32.05205373)
\curveto(782.35342275,32.04205337)(782.31842279,32.03705337)(782.27842041,32.03705373)
\curveto(782.22842288,32.03705337)(782.18842292,32.03205338)(782.15842041,32.02205373)
\lineto(781.99342041,32.02205373)
\curveto(781.91342319,32.00205341)(781.81342329,31.99705341)(781.69342041,32.00705373)
\curveto(781.56342354,32.01705339)(781.47342363,32.03205338)(781.42342041,32.05205373)
\curveto(781.33342377,32.07205334)(781.26842384,32.12705328)(781.22842041,32.21705373)
\curveto(781.2084239,32.24705316)(781.2034239,32.27705313)(781.21342041,32.30705373)
\curveto(781.21342389,32.33705307)(781.2084239,32.37705303)(781.19842041,32.42705373)
\curveto(781.18842392,32.46705294)(781.18342392,32.5070529)(781.18342041,32.54705373)
\lineto(781.18342041,32.69705373)
\curveto(781.18342392,32.81705259)(781.18842392,32.93705247)(781.19842041,33.05705373)
\curveto(781.19842391,33.18705222)(781.23342387,33.27705213)(781.30342041,33.32705373)
\curveto(781.36342374,33.36705204)(781.42342368,33.38705202)(781.48342041,33.38705373)
\curveto(781.54342356,33.38705202)(781.61342349,33.39705201)(781.69342041,33.41705373)
\curveto(781.72342338,33.42705198)(781.75842335,33.42705198)(781.79842041,33.41705373)
\curveto(781.82842328,33.41705199)(781.85342325,33.42205199)(781.87342041,33.43205373)
\lineto(782.08342041,33.43205373)
\curveto(782.13342297,33.45205196)(782.18342292,33.45705195)(782.23342041,33.44705373)
\curveto(782.27342283,33.44705196)(782.31842279,33.45705195)(782.36842041,33.47705373)
\curveto(782.49842261,33.5070519)(782.62342248,33.53705187)(782.74342041,33.56705373)
\curveto(782.85342225,33.59705181)(782.95842215,33.64205177)(783.05842041,33.70205373)
\curveto(783.34842176,33.87205154)(783.55342155,34.14205127)(783.67342041,34.51205373)
\curveto(783.69342141,34.56205085)(783.7084214,34.6120508)(783.71842041,34.66205373)
\curveto(783.71842139,34.72205069)(783.72842138,34.77705063)(783.74842041,34.82705373)
\lineto(783.74842041,34.90205373)
\curveto(783.75842135,34.97205044)(783.76842134,35.06705034)(783.77842041,35.18705373)
\curveto(783.77842133,35.31705009)(783.76842134,35.41704999)(783.74842041,35.48705373)
\curveto(783.72842138,35.55704985)(783.71342139,35.62704978)(783.70342041,35.69705373)
\curveto(783.68342142,35.77704963)(783.66342144,35.84704956)(783.64342041,35.90705373)
\curveto(783.48342162,36.28704912)(783.2084219,36.56204885)(782.81842041,36.73205373)
\curveto(782.68842242,36.78204863)(782.53342257,36.81704859)(782.35342041,36.83705373)
\curveto(782.17342293,36.86704854)(781.98842312,36.88204853)(781.79842041,36.88205373)
\curveto(781.59842351,36.89204852)(781.39842371,36.89204852)(781.19842041,36.88205373)
\lineto(780.62842041,36.88205373)
\lineto(776.38342041,36.88205373)
\lineto(774.83842041,36.88205373)
\curveto(774.72843038,36.88204853)(774.6084305,36.87704853)(774.47842041,36.86705373)
\curveto(774.34843076,36.85704855)(774.24343086,36.87704853)(774.16342041,36.92705373)
\curveto(774.09343101,36.98704842)(774.04343106,37.06704834)(774.01342041,37.16705373)
\curveto(774.01343109,37.18704822)(774.01343109,37.2070482)(774.01342041,37.22705373)
\curveto(774.01343109,37.24704816)(774.0084311,37.26704814)(773.99842041,37.28705373)
}
}
{
\newrgbcolor{curcolor}{0 0 0}
\pscustom[linestyle=none,fillstyle=solid,fillcolor=curcolor]
{
\newpath
\moveto(776.95342041,40.82072561)
\lineto(776.95342041,41.25572561)
\curveto(776.95342815,41.40572364)(776.99342811,41.51072354)(777.07342041,41.57072561)
\curveto(777.15342795,41.62072343)(777.25342785,41.6457234)(777.37342041,41.64572561)
\curveto(777.49342761,41.65572339)(777.61342749,41.66072339)(777.73342041,41.66072561)
\lineto(779.15842041,41.66072561)
\lineto(781.42342041,41.66072561)
\lineto(782.11342041,41.66072561)
\curveto(782.34342276,41.66072339)(782.54342256,41.68572336)(782.71342041,41.73572561)
\curveto(783.16342194,41.89572315)(783.47842163,42.19572285)(783.65842041,42.63572561)
\curveto(783.74842136,42.85572219)(783.78342132,43.12072193)(783.76342041,43.43072561)
\curveto(783.73342137,43.74072131)(783.67842143,43.99072106)(783.59842041,44.18072561)
\curveto(783.45842165,44.51072054)(783.28342182,44.77072028)(783.07342041,44.96072561)
\curveto(782.85342225,45.16071989)(782.56842254,45.31571973)(782.21842041,45.42572561)
\curveto(782.13842297,45.45571959)(782.05842305,45.47571957)(781.97842041,45.48572561)
\curveto(781.89842321,45.49571955)(781.81342329,45.51071954)(781.72342041,45.53072561)
\curveto(781.67342343,45.54071951)(781.62842348,45.54071951)(781.58842041,45.53072561)
\curveto(781.54842356,45.53071952)(781.5034236,45.54071951)(781.45342041,45.56072561)
\lineto(781.13842041,45.56072561)
\curveto(781.05842405,45.58071947)(780.96842414,45.58571946)(780.86842041,45.57572561)
\curveto(780.75842435,45.56571948)(780.65842445,45.56071949)(780.56842041,45.56072561)
\lineto(779.39842041,45.56072561)
\lineto(777.80842041,45.56072561)
\curveto(777.68842742,45.56071949)(777.56342754,45.55571949)(777.43342041,45.54572561)
\curveto(777.29342781,45.5457195)(777.18342792,45.57071948)(777.10342041,45.62072561)
\curveto(777.05342805,45.66071939)(777.02342808,45.70571934)(777.01342041,45.75572561)
\curveto(776.99342811,45.81571923)(776.97342813,45.88571916)(776.95342041,45.96572561)
\lineto(776.95342041,46.19072561)
\curveto(776.95342815,46.31071874)(776.95842815,46.41571863)(776.96842041,46.50572561)
\curveto(776.97842813,46.60571844)(777.02342808,46.68071837)(777.10342041,46.73072561)
\curveto(777.15342795,46.78071827)(777.22842788,46.80571824)(777.32842041,46.80572561)
\lineto(777.61342041,46.80572561)
\lineto(778.63342041,46.80572561)
\lineto(782.66842041,46.80572561)
\lineto(784.01842041,46.80572561)
\curveto(784.13842097,46.80571824)(784.25342085,46.80071825)(784.36342041,46.79072561)
\curveto(784.46342064,46.79071826)(784.53842057,46.75571829)(784.58842041,46.68572561)
\curveto(784.61842049,46.6457184)(784.64342046,46.58571846)(784.66342041,46.50572561)
\curveto(784.67342043,46.42571862)(784.68342042,46.33571871)(784.69342041,46.23572561)
\curveto(784.69342041,46.1457189)(784.68842042,46.05571899)(784.67842041,45.96572561)
\curveto(784.66842044,45.88571916)(784.64842046,45.82571922)(784.61842041,45.78572561)
\curveto(784.57842053,45.73571931)(784.51342059,45.69071936)(784.42342041,45.65072561)
\curveto(784.38342072,45.64071941)(784.32842078,45.63071942)(784.25842041,45.62072561)
\curveto(784.18842092,45.62071943)(784.12342098,45.61571943)(784.06342041,45.60572561)
\curveto(783.99342111,45.59571945)(783.93842117,45.57571947)(783.89842041,45.54572561)
\curveto(783.85842125,45.51571953)(783.84342126,45.47071958)(783.85342041,45.41072561)
\curveto(783.87342123,45.33071972)(783.93342117,45.2507198)(784.03342041,45.17072561)
\curveto(784.12342098,45.09071996)(784.19342091,45.01572003)(784.24342041,44.94572561)
\curveto(784.4034207,44.72572032)(784.54342056,44.47572057)(784.66342041,44.19572561)
\curveto(784.71342039,44.08572096)(784.74342036,43.97072108)(784.75342041,43.85072561)
\curveto(784.77342033,43.74072131)(784.79842031,43.62572142)(784.82842041,43.50572561)
\curveto(784.83842027,43.45572159)(784.83842027,43.40072165)(784.82842041,43.34072561)
\curveto(784.81842029,43.29072176)(784.82342028,43.24072181)(784.84342041,43.19072561)
\curveto(784.86342024,43.09072196)(784.86342024,43.00072205)(784.84342041,42.92072561)
\lineto(784.84342041,42.77072561)
\curveto(784.82342028,42.72072233)(784.81342029,42.66072239)(784.81342041,42.59072561)
\curveto(784.81342029,42.53072252)(784.8084203,42.47572257)(784.79842041,42.42572561)
\curveto(784.77842033,42.38572266)(784.76842034,42.3457227)(784.76842041,42.30572561)
\curveto(784.77842033,42.27572277)(784.77342033,42.23572281)(784.75342041,42.18572561)
\lineto(784.69342041,41.94572561)
\curveto(784.67342043,41.87572317)(784.64342046,41.80072325)(784.60342041,41.72072561)
\curveto(784.49342061,41.46072359)(784.34842076,41.24072381)(784.16842041,41.06072561)
\curveto(783.97842113,40.89072416)(783.75342135,40.7507243)(783.49342041,40.64072561)
\curveto(783.4034217,40.60072445)(783.31342179,40.57072448)(783.22342041,40.55072561)
\lineto(782.92342041,40.49072561)
\curveto(782.86342224,40.47072458)(782.8084223,40.46072459)(782.75842041,40.46072561)
\curveto(782.69842241,40.47072458)(782.63342247,40.46572458)(782.56342041,40.44572561)
\curveto(782.54342256,40.43572461)(782.51842259,40.43072462)(782.48842041,40.43072561)
\curveto(782.44842266,40.43072462)(782.41342269,40.42572462)(782.38342041,40.41572561)
\lineto(782.23342041,40.41572561)
\curveto(782.19342291,40.40572464)(782.14842296,40.40072465)(782.09842041,40.40072561)
\curveto(782.03842307,40.41072464)(781.98342312,40.41572463)(781.93342041,40.41572561)
\lineto(781.33342041,40.41572561)
\lineto(778.57342041,40.41572561)
\lineto(777.61342041,40.41572561)
\lineto(777.34342041,40.41572561)
\curveto(777.25342785,40.41572463)(777.17842793,40.43572461)(777.11842041,40.47572561)
\curveto(777.04842806,40.51572453)(776.99842811,40.59072446)(776.96842041,40.70072561)
\curveto(776.95842815,40.72072433)(776.95842815,40.74072431)(776.96842041,40.76072561)
\curveto(776.96842814,40.78072427)(776.96342814,40.80072425)(776.95342041,40.82072561)
}
}
{
\newrgbcolor{curcolor}{0 0 0}
\pscustom[linestyle=none,fillstyle=solid,fillcolor=curcolor]
{
\newpath
\moveto(776.80342041,52.39533498)
\curveto(776.78342832,53.02532975)(776.86842824,53.53032924)(777.05842041,53.91033498)
\curveto(777.24842786,54.29032848)(777.53342757,54.59532818)(777.91342041,54.82533498)
\curveto(778.01342709,54.88532789)(778.12342698,54.93032784)(778.24342041,54.96033498)
\curveto(778.35342675,55.00032777)(778.46842664,55.03532774)(778.58842041,55.06533498)
\curveto(778.77842633,55.11532766)(778.98342612,55.14532763)(779.20342041,55.15533498)
\curveto(779.42342568,55.16532761)(779.64842546,55.1703276)(779.87842041,55.17033498)
\lineto(781.48342041,55.17033498)
\lineto(783.82342041,55.17033498)
\curveto(783.99342111,55.1703276)(784.16342094,55.16532761)(784.33342041,55.15533498)
\curveto(784.5034206,55.15532762)(784.61342049,55.09032768)(784.66342041,54.96033498)
\curveto(784.68342042,54.91032786)(784.69342041,54.85532792)(784.69342041,54.79533498)
\curveto(784.7034204,54.74532803)(784.7084204,54.69032808)(784.70842041,54.63033498)
\curveto(784.7084204,54.50032827)(784.7034204,54.3753284)(784.69342041,54.25533498)
\curveto(784.69342041,54.13532864)(784.65342045,54.05032872)(784.57342041,54.00033498)
\curveto(784.5034206,53.95032882)(784.41342069,53.92532885)(784.30342041,53.92533498)
\lineto(783.97342041,53.92533498)
\lineto(782.68342041,53.92533498)
\lineto(780.23842041,53.92533498)
\curveto(779.96842514,53.92532885)(779.7034254,53.92032885)(779.44342041,53.91033498)
\curveto(779.17342593,53.90032887)(778.94342616,53.85532892)(778.75342041,53.77533498)
\curveto(778.55342655,53.69532908)(778.39342671,53.5753292)(778.27342041,53.41533498)
\curveto(778.14342696,53.25532952)(778.04342706,53.0703297)(777.97342041,52.86033498)
\curveto(777.95342715,52.80032997)(777.94342716,52.73533004)(777.94342041,52.66533498)
\curveto(777.93342717,52.60533017)(777.91842719,52.54533023)(777.89842041,52.48533498)
\curveto(777.88842722,52.43533034)(777.88842722,52.35533042)(777.89842041,52.24533498)
\curveto(777.89842721,52.14533063)(777.9034272,52.0753307)(777.91342041,52.03533498)
\curveto(777.93342717,51.99533078)(777.94342716,51.96033081)(777.94342041,51.93033498)
\curveto(777.93342717,51.90033087)(777.93342717,51.86533091)(777.94342041,51.82533498)
\curveto(777.97342713,51.69533108)(778.0084271,51.5703312)(778.04842041,51.45033498)
\curveto(778.07842703,51.34033143)(778.12342698,51.23533154)(778.18342041,51.13533498)
\curveto(778.2034269,51.09533168)(778.22342688,51.06033171)(778.24342041,51.03033498)
\curveto(778.26342684,51.00033177)(778.28342682,50.96533181)(778.30342041,50.92533498)
\curveto(778.55342655,50.5753322)(778.92842618,50.32033245)(779.42842041,50.16033498)
\curveto(779.5084256,50.13033264)(779.59342551,50.11033266)(779.68342041,50.10033498)
\curveto(779.76342534,50.09033268)(779.84342526,50.0753327)(779.92342041,50.05533498)
\curveto(779.97342513,50.03533274)(780.02342508,50.03033274)(780.07342041,50.04033498)
\curveto(780.11342499,50.05033272)(780.15342495,50.04533273)(780.19342041,50.02533498)
\lineto(780.50842041,50.02533498)
\curveto(780.53842457,50.01533276)(780.57342453,50.01033276)(780.61342041,50.01033498)
\curveto(780.65342445,50.02033275)(780.69842441,50.02533275)(780.74842041,50.02533498)
\lineto(781.19842041,50.02533498)
\lineto(782.63842041,50.02533498)
\lineto(783.95842041,50.02533498)
\lineto(784.30342041,50.02533498)
\curveto(784.41342069,50.02533275)(784.5034206,50.00033277)(784.57342041,49.95033498)
\curveto(784.65342045,49.90033287)(784.69342041,49.81033296)(784.69342041,49.68033498)
\curveto(784.7034204,49.56033321)(784.7084204,49.43533334)(784.70842041,49.30533498)
\curveto(784.7084204,49.22533355)(784.7034204,49.15033362)(784.69342041,49.08033498)
\curveto(784.68342042,49.01033376)(784.65842045,48.95033382)(784.61842041,48.90033498)
\curveto(784.56842054,48.82033395)(784.47342063,48.78033399)(784.33342041,48.78033498)
\lineto(783.92842041,48.78033498)
\lineto(782.15842041,48.78033498)
\lineto(778.52842041,48.78033498)
\lineto(777.61342041,48.78033498)
\lineto(777.34342041,48.78033498)
\curveto(777.25342785,48.78033399)(777.18342792,48.80033397)(777.13342041,48.84033498)
\curveto(777.07342803,48.8703339)(777.03342807,48.92033385)(777.01342041,48.99033498)
\curveto(777.0034281,49.03033374)(776.99342811,49.08533369)(776.98342041,49.15533498)
\curveto(776.97342813,49.23533354)(776.96842814,49.31533346)(776.96842041,49.39533498)
\curveto(776.96842814,49.4753333)(776.97342813,49.55033322)(776.98342041,49.62033498)
\curveto(776.99342811,49.70033307)(777.0084281,49.75533302)(777.02842041,49.78533498)
\curveto(777.09842801,49.89533288)(777.18842792,49.94533283)(777.29842041,49.93533498)
\curveto(777.39842771,49.92533285)(777.51342759,49.94033283)(777.64342041,49.98033498)
\curveto(777.7034274,50.00033277)(777.75342735,50.04033273)(777.79342041,50.10033498)
\curveto(777.8034273,50.22033255)(777.75842735,50.31533246)(777.65842041,50.38533498)
\curveto(777.55842755,50.46533231)(777.47842763,50.54533223)(777.41842041,50.62533498)
\curveto(777.31842779,50.76533201)(777.22842788,50.90533187)(777.14842041,51.04533498)
\curveto(777.05842805,51.19533158)(776.98342812,51.36533141)(776.92342041,51.55533498)
\curveto(776.89342821,51.63533114)(776.87342823,51.72033105)(776.86342041,51.81033498)
\curveto(776.85342825,51.91033086)(776.83842827,52.00533077)(776.81842041,52.09533498)
\curveto(776.8084283,52.14533063)(776.8034283,52.19533058)(776.80342041,52.24533498)
\lineto(776.80342041,52.39533498)
}
}
{
\newrgbcolor{curcolor}{0 0 0}
\pscustom[linestyle=none,fillstyle=solid,fillcolor=curcolor]
{
}
}
{
\newrgbcolor{curcolor}{0 0 0}
\pscustom[linestyle=none,fillstyle=solid,fillcolor=curcolor]
{
\newpath
\moveto(774.07342041,64.24510061)
\curveto(774.06343104,64.93509597)(774.18343092,65.53509537)(774.43342041,66.04510061)
\curveto(774.68343042,66.56509434)(775.01843009,66.96009395)(775.43842041,67.23010061)
\curveto(775.51842959,67.28009363)(775.6084295,67.32509358)(775.70842041,67.36510061)
\curveto(775.79842931,67.4050935)(775.89342921,67.45009346)(775.99342041,67.50010061)
\curveto(776.09342901,67.54009337)(776.19342891,67.57009334)(776.29342041,67.59010061)
\curveto(776.39342871,67.6100933)(776.49842861,67.63009328)(776.60842041,67.65010061)
\curveto(776.65842845,67.67009324)(776.7034284,67.67509323)(776.74342041,67.66510061)
\curveto(776.78342832,67.65509325)(776.82842828,67.66009325)(776.87842041,67.68010061)
\curveto(776.92842818,67.69009322)(777.01342809,67.69509321)(777.13342041,67.69510061)
\curveto(777.24342786,67.69509321)(777.32842778,67.69009322)(777.38842041,67.68010061)
\curveto(777.44842766,67.66009325)(777.5084276,67.65009326)(777.56842041,67.65010061)
\curveto(777.62842748,67.66009325)(777.68842742,67.65509325)(777.74842041,67.63510061)
\curveto(777.88842722,67.59509331)(778.02342708,67.56009335)(778.15342041,67.53010061)
\curveto(778.28342682,67.50009341)(778.4084267,67.46009345)(778.52842041,67.41010061)
\curveto(778.66842644,67.35009356)(778.79342631,67.28009363)(778.90342041,67.20010061)
\curveto(779.01342609,67.13009378)(779.12342598,67.05509385)(779.23342041,66.97510061)
\lineto(779.29342041,66.91510061)
\curveto(779.31342579,66.905094)(779.33342577,66.89009402)(779.35342041,66.87010061)
\curveto(779.51342559,66.75009416)(779.65842545,66.61509429)(779.78842041,66.46510061)
\curveto(779.91842519,66.31509459)(780.04342506,66.15509475)(780.16342041,65.98510061)
\curveto(780.38342472,65.67509523)(780.58842452,65.38009553)(780.77842041,65.10010061)
\curveto(780.91842419,64.87009604)(781.05342405,64.64009627)(781.18342041,64.41010061)
\curveto(781.31342379,64.19009672)(781.44842366,63.97009694)(781.58842041,63.75010061)
\curveto(781.75842335,63.50009741)(781.93842317,63.26009765)(782.12842041,63.03010061)
\curveto(782.31842279,62.8100981)(782.54342256,62.62009829)(782.80342041,62.46010061)
\curveto(782.86342224,62.42009849)(782.92342218,62.38509852)(782.98342041,62.35510061)
\curveto(783.03342207,62.32509858)(783.09842201,62.29509861)(783.17842041,62.26510061)
\curveto(783.24842186,62.24509866)(783.3084218,62.24009867)(783.35842041,62.25010061)
\curveto(783.42842168,62.27009864)(783.48342162,62.3050986)(783.52342041,62.35510061)
\curveto(783.55342155,62.4050985)(783.57342153,62.46509844)(783.58342041,62.53510061)
\lineto(783.58342041,62.77510061)
\lineto(783.58342041,63.52510061)
\lineto(783.58342041,66.33010061)
\lineto(783.58342041,66.99010061)
\curveto(783.58342152,67.08009383)(783.58842152,67.16509374)(783.59842041,67.24510061)
\curveto(783.59842151,67.32509358)(783.61842149,67.39009352)(783.65842041,67.44010061)
\curveto(783.69842141,67.49009342)(783.77342133,67.53009338)(783.88342041,67.56010061)
\curveto(783.98342112,67.60009331)(784.08342102,67.6100933)(784.18342041,67.59010061)
\lineto(784.31842041,67.59010061)
\curveto(784.38842072,67.57009334)(784.44842066,67.55009336)(784.49842041,67.53010061)
\curveto(784.54842056,67.5100934)(784.58842052,67.47509343)(784.61842041,67.42510061)
\curveto(784.65842045,67.37509353)(784.67842043,67.3050936)(784.67842041,67.21510061)
\lineto(784.67842041,66.94510061)
\lineto(784.67842041,66.04510061)
\lineto(784.67842041,62.53510061)
\lineto(784.67842041,61.47010061)
\curveto(784.67842043,61.39009952)(784.68342042,61.30009961)(784.69342041,61.20010061)
\curveto(784.69342041,61.10009981)(784.68342042,61.01509989)(784.66342041,60.94510061)
\curveto(784.59342051,60.73510017)(784.41342069,60.67010024)(784.12342041,60.75010061)
\curveto(784.08342102,60.76010015)(784.04842106,60.76010015)(784.01842041,60.75010061)
\curveto(783.97842113,60.75010016)(783.93342117,60.76010015)(783.88342041,60.78010061)
\curveto(783.8034213,60.80010011)(783.71842139,60.82010009)(783.62842041,60.84010061)
\curveto(783.53842157,60.86010005)(783.45342165,60.88510002)(783.37342041,60.91510061)
\curveto(782.88342222,61.07509983)(782.46842264,61.27509963)(782.12842041,61.51510061)
\curveto(781.87842323,61.69509921)(781.65342345,61.90009901)(781.45342041,62.13010061)
\curveto(781.24342386,62.36009855)(781.04842406,62.60009831)(780.86842041,62.85010061)
\curveto(780.68842442,63.1100978)(780.51842459,63.37509753)(780.35842041,63.64510061)
\curveto(780.18842492,63.92509698)(780.01342509,64.19509671)(779.83342041,64.45510061)
\curveto(779.75342535,64.56509634)(779.67842543,64.67009624)(779.60842041,64.77010061)
\curveto(779.53842557,64.88009603)(779.46342564,64.99009592)(779.38342041,65.10010061)
\curveto(779.35342575,65.14009577)(779.32342578,65.17509573)(779.29342041,65.20510061)
\curveto(779.25342585,65.24509566)(779.22342588,65.28509562)(779.20342041,65.32510061)
\curveto(779.09342601,65.46509544)(778.96842614,65.59009532)(778.82842041,65.70010061)
\curveto(778.79842631,65.72009519)(778.77342633,65.74509516)(778.75342041,65.77510061)
\curveto(778.72342638,65.8050951)(778.69342641,65.83009508)(778.66342041,65.85010061)
\curveto(778.56342654,65.93009498)(778.46342664,65.99509491)(778.36342041,66.04510061)
\curveto(778.26342684,66.1050948)(778.15342695,66.16009475)(778.03342041,66.21010061)
\curveto(777.96342714,66.24009467)(777.88842722,66.26009465)(777.80842041,66.27010061)
\lineto(777.56842041,66.33010061)
\lineto(777.47842041,66.33010061)
\curveto(777.44842766,66.34009457)(777.41842769,66.34509456)(777.38842041,66.34510061)
\curveto(777.31842779,66.36509454)(777.22342788,66.37009454)(777.10342041,66.36010061)
\curveto(776.97342813,66.36009455)(776.87342823,66.35009456)(776.80342041,66.33010061)
\curveto(776.72342838,66.3100946)(776.64842846,66.29009462)(776.57842041,66.27010061)
\curveto(776.49842861,66.26009465)(776.41842869,66.24009467)(776.33842041,66.21010061)
\curveto(776.09842901,66.10009481)(775.89842921,65.95009496)(775.73842041,65.76010061)
\curveto(775.56842954,65.58009533)(775.42842968,65.36009555)(775.31842041,65.10010061)
\curveto(775.29842981,65.03009588)(775.28342982,64.96009595)(775.27342041,64.89010061)
\curveto(775.25342985,64.82009609)(775.23342987,64.74509616)(775.21342041,64.66510061)
\curveto(775.19342991,64.58509632)(775.18342992,64.47509643)(775.18342041,64.33510061)
\curveto(775.18342992,64.2050967)(775.19342991,64.10009681)(775.21342041,64.02010061)
\curveto(775.22342988,63.96009695)(775.22842988,63.905097)(775.22842041,63.85510061)
\curveto(775.22842988,63.8050971)(775.23842987,63.75509715)(775.25842041,63.70510061)
\curveto(775.29842981,63.6050973)(775.33842977,63.5100974)(775.37842041,63.42010061)
\curveto(775.41842969,63.34009757)(775.46342964,63.26009765)(775.51342041,63.18010061)
\curveto(775.53342957,63.15009776)(775.55842955,63.12009779)(775.58842041,63.09010061)
\curveto(775.61842949,63.07009784)(775.64342946,63.04509786)(775.66342041,63.01510061)
\lineto(775.73842041,62.94010061)
\curveto(775.75842935,62.910098)(775.77842933,62.88509802)(775.79842041,62.86510061)
\lineto(776.00842041,62.71510061)
\curveto(776.06842904,62.67509823)(776.13342897,62.63009828)(776.20342041,62.58010061)
\curveto(776.29342881,62.52009839)(776.39842871,62.47009844)(776.51842041,62.43010061)
\curveto(776.62842848,62.40009851)(776.73842837,62.36509854)(776.84842041,62.32510061)
\curveto(776.95842815,62.28509862)(777.103428,62.26009865)(777.28342041,62.25010061)
\curveto(777.45342765,62.24009867)(777.57842753,62.2100987)(777.65842041,62.16010061)
\curveto(777.73842737,62.1100988)(777.78342732,62.03509887)(777.79342041,61.93510061)
\curveto(777.8034273,61.83509907)(777.8084273,61.72509918)(777.80842041,61.60510061)
\curveto(777.8084273,61.56509934)(777.81342729,61.52509938)(777.82342041,61.48510061)
\curveto(777.82342728,61.44509946)(777.81842729,61.4100995)(777.80842041,61.38010061)
\curveto(777.78842732,61.33009958)(777.77842733,61.28009963)(777.77842041,61.23010061)
\curveto(777.77842733,61.19009972)(777.76842734,61.15009976)(777.74842041,61.11010061)
\curveto(777.68842742,61.02009989)(777.55342755,60.97509993)(777.34342041,60.97510061)
\lineto(777.22342041,60.97510061)
\curveto(777.16342794,60.98509992)(777.103428,60.99009992)(777.04342041,60.99010061)
\curveto(776.97342813,61.00009991)(776.9084282,61.0100999)(776.84842041,61.02010061)
\curveto(776.73842837,61.04009987)(776.63842847,61.06009985)(776.54842041,61.08010061)
\curveto(776.44842866,61.10009981)(776.35342875,61.13009978)(776.26342041,61.17010061)
\curveto(776.19342891,61.19009972)(776.13342897,61.2100997)(776.08342041,61.23010061)
\lineto(775.90342041,61.29010061)
\curveto(775.64342946,61.4100995)(775.39842971,61.56509934)(775.16842041,61.75510061)
\curveto(774.93843017,61.95509895)(774.75343035,62.17009874)(774.61342041,62.40010061)
\curveto(774.53343057,62.5100984)(774.46843064,62.62509828)(774.41842041,62.74510061)
\lineto(774.26842041,63.13510061)
\curveto(774.21843089,63.24509766)(774.18843092,63.36009755)(774.17842041,63.48010061)
\curveto(774.15843095,63.60009731)(774.13343097,63.72509718)(774.10342041,63.85510061)
\curveto(774.103431,63.92509698)(774.103431,63.99009692)(774.10342041,64.05010061)
\curveto(774.09343101,64.1100968)(774.08343102,64.17509673)(774.07342041,64.24510061)
}
}
{
\newrgbcolor{curcolor}{0 0 0}
\pscustom[linestyle=none,fillstyle=solid,fillcolor=curcolor]
{
\newpath
\moveto(780.35842041,76.34470998)
\curveto(780.47842463,76.37470226)(780.61842449,76.39970223)(780.77842041,76.41970998)
\curveto(780.93842417,76.43970219)(781.103424,76.44970218)(781.27342041,76.44970998)
\curveto(781.44342366,76.44970218)(781.6084235,76.43970219)(781.76842041,76.41970998)
\curveto(781.92842318,76.39970223)(782.06842304,76.37470226)(782.18842041,76.34470998)
\curveto(782.32842278,76.30470233)(782.45342265,76.26970236)(782.56342041,76.23970998)
\curveto(782.67342243,76.20970242)(782.78342232,76.16970246)(782.89342041,76.11970998)
\curveto(783.53342157,75.84970278)(784.01842109,75.4347032)(784.34842041,74.87470998)
\curveto(784.4084207,74.79470384)(784.45842065,74.70970392)(784.49842041,74.61970998)
\curveto(784.52842058,74.5297041)(784.56342054,74.4297042)(784.60342041,74.31970998)
\curveto(784.65342045,74.20970442)(784.68842042,74.08970454)(784.70842041,73.95970998)
\curveto(784.73842037,73.83970479)(784.76842034,73.70970492)(784.79842041,73.56970998)
\curveto(784.81842029,73.50970512)(784.82342028,73.44970518)(784.81342041,73.38970998)
\curveto(784.8034203,73.33970529)(784.8084203,73.27970535)(784.82842041,73.20970998)
\curveto(784.83842027,73.18970544)(784.83842027,73.16470547)(784.82842041,73.13470998)
\curveto(784.82842028,73.10470553)(784.83342027,73.07970555)(784.84342041,73.05970998)
\lineto(784.84342041,72.90970998)
\curveto(784.85342025,72.83970579)(784.85342025,72.78970584)(784.84342041,72.75970998)
\curveto(784.83342027,72.71970591)(784.82842028,72.67470596)(784.82842041,72.62470998)
\curveto(784.83842027,72.58470605)(784.83842027,72.54470609)(784.82842041,72.50470998)
\curveto(784.8084203,72.41470622)(784.79342031,72.32470631)(784.78342041,72.23470998)
\curveto(784.78342032,72.14470649)(784.77342033,72.05470658)(784.75342041,71.96470998)
\curveto(784.72342038,71.87470676)(784.69842041,71.78470685)(784.67842041,71.69470998)
\curveto(784.65842045,71.60470703)(784.62842048,71.51970711)(784.58842041,71.43970998)
\curveto(784.47842063,71.19970743)(784.34842076,70.97470766)(784.19842041,70.76470998)
\curveto(784.03842107,70.55470808)(783.85842125,70.37470826)(783.65842041,70.22470998)
\curveto(783.48842162,70.10470853)(783.31342179,69.99970863)(783.13342041,69.90970998)
\curveto(782.95342215,69.81970881)(782.76342234,69.7297089)(782.56342041,69.63970998)
\curveto(782.46342264,69.59970903)(782.36342274,69.56470907)(782.26342041,69.53470998)
\curveto(782.15342295,69.51470912)(782.04342306,69.48970914)(781.93342041,69.45970998)
\curveto(781.79342331,69.41970921)(781.65342345,69.39470924)(781.51342041,69.38470998)
\curveto(781.37342373,69.37470926)(781.23342387,69.35470928)(781.09342041,69.32470998)
\curveto(780.98342412,69.31470932)(780.88342422,69.30470933)(780.79342041,69.29470998)
\curveto(780.69342441,69.29470934)(780.59342451,69.28470935)(780.49342041,69.26470998)
\lineto(780.40342041,69.26470998)
\curveto(780.37342473,69.27470936)(780.34842476,69.27470936)(780.32842041,69.26470998)
\lineto(780.11842041,69.26470998)
\curveto(780.05842505,69.24470939)(779.99342511,69.2347094)(779.92342041,69.23470998)
\curveto(779.84342526,69.24470939)(779.76842534,69.24970938)(779.69842041,69.24970998)
\lineto(779.54842041,69.24970998)
\curveto(779.49842561,69.24970938)(779.44842566,69.25470938)(779.39842041,69.26470998)
\lineto(779.02342041,69.26470998)
\curveto(778.99342611,69.27470936)(778.95842615,69.27470936)(778.91842041,69.26470998)
\curveto(778.87842623,69.26470937)(778.83842627,69.26970936)(778.79842041,69.27970998)
\curveto(778.68842642,69.29970933)(778.57842653,69.31470932)(778.46842041,69.32470998)
\curveto(778.34842676,69.3347093)(778.23342687,69.34470929)(778.12342041,69.35470998)
\curveto(777.97342713,69.39470924)(777.82842728,69.41970921)(777.68842041,69.42970998)
\curveto(777.53842757,69.44970918)(777.39342771,69.47970915)(777.25342041,69.51970998)
\curveto(776.95342815,69.60970902)(776.66842844,69.70470893)(776.39842041,69.80470998)
\curveto(776.12842898,69.90470873)(775.87842923,70.0297086)(775.64842041,70.17970998)
\curveto(775.32842978,70.37970825)(775.04843006,70.62470801)(774.80842041,70.91470998)
\curveto(774.56843054,71.20470743)(774.38343072,71.54470709)(774.25342041,71.93470998)
\curveto(774.21343089,72.04470659)(774.18843092,72.15470648)(774.17842041,72.26470998)
\curveto(774.15843095,72.38470625)(774.13343097,72.50470613)(774.10342041,72.62470998)
\curveto(774.09343101,72.69470594)(774.08843102,72.75970587)(774.08842041,72.81970998)
\curveto(774.08843102,72.87970575)(774.08343102,72.94470569)(774.07342041,73.01470998)
\curveto(774.05343105,73.71470492)(774.16843094,74.28970434)(774.41842041,74.73970998)
\curveto(774.66843044,75.18970344)(775.01843009,75.5347031)(775.46842041,75.77470998)
\curveto(775.69842941,75.88470275)(775.97342913,75.98470265)(776.29342041,76.07470998)
\curveto(776.36342874,76.09470254)(776.43842867,76.09470254)(776.51842041,76.07470998)
\curveto(776.58842852,76.06470257)(776.63842847,76.03970259)(776.66842041,75.99970998)
\curveto(776.69842841,75.96970266)(776.72342838,75.90970272)(776.74342041,75.81970998)
\curveto(776.75342835,75.7297029)(776.76342834,75.629703)(776.77342041,75.51970998)
\curveto(776.77342833,75.41970321)(776.76842834,75.31970331)(776.75842041,75.21970998)
\curveto(776.74842836,75.1297035)(776.72842838,75.06470357)(776.69842041,75.02470998)
\curveto(776.62842848,74.91470372)(776.51842859,74.8347038)(776.36842041,74.78470998)
\curveto(776.21842889,74.74470389)(776.08842902,74.68970394)(775.97842041,74.61970998)
\curveto(775.66842944,74.4297042)(775.43842967,74.14970448)(775.28842041,73.77970998)
\curveto(775.25842985,73.70970492)(775.23842987,73.634705)(775.22842041,73.55470998)
\curveto(775.21842989,73.48470515)(775.2034299,73.40970522)(775.18342041,73.32970998)
\curveto(775.17342993,73.27970535)(775.16842994,73.20970542)(775.16842041,73.11970998)
\curveto(775.16842994,73.03970559)(775.17342993,72.97470566)(775.18342041,72.92470998)
\curveto(775.2034299,72.88470575)(775.2084299,72.84970578)(775.19842041,72.81970998)
\curveto(775.18842992,72.78970584)(775.18842992,72.75470588)(775.19842041,72.71470998)
\lineto(775.25842041,72.47470998)
\curveto(775.27842983,72.40470623)(775.3034298,72.3347063)(775.33342041,72.26470998)
\curveto(775.49342961,71.88470675)(775.7034294,71.59470704)(775.96342041,71.39470998)
\curveto(776.22342888,71.20470743)(776.53842857,71.0297076)(776.90842041,70.86970998)
\curveto(776.98842812,70.83970779)(777.06842804,70.81470782)(777.14842041,70.79470998)
\curveto(777.22842788,70.78470785)(777.3084278,70.76470787)(777.38842041,70.73470998)
\curveto(777.49842761,70.70470793)(777.61342749,70.67970795)(777.73342041,70.65970998)
\curveto(777.85342725,70.64970798)(777.97342713,70.629708)(778.09342041,70.59970998)
\curveto(778.14342696,70.57970805)(778.19342691,70.56970806)(778.24342041,70.56970998)
\curveto(778.29342681,70.57970805)(778.34342676,70.57470806)(778.39342041,70.55470998)
\curveto(778.45342665,70.54470809)(778.53342657,70.54470809)(778.63342041,70.55470998)
\curveto(778.72342638,70.56470807)(778.77842633,70.57970805)(778.79842041,70.59970998)
\curveto(778.83842627,70.61970801)(778.85842625,70.64970798)(778.85842041,70.68970998)
\curveto(778.85842625,70.73970789)(778.84842626,70.78470785)(778.82842041,70.82470998)
\curveto(778.78842632,70.89470774)(778.74342636,70.95470768)(778.69342041,71.00470998)
\curveto(778.64342646,71.05470758)(778.59342651,71.11470752)(778.54342041,71.18470998)
\lineto(778.48342041,71.24470998)
\curveto(778.45342665,71.27470736)(778.42842668,71.30470733)(778.40842041,71.33470998)
\curveto(778.24842686,71.56470707)(778.11342699,71.83970679)(778.00342041,72.15970998)
\curveto(777.98342712,72.2297064)(777.96842714,72.29970633)(777.95842041,72.36970998)
\curveto(777.94842716,72.43970619)(777.93342717,72.51470612)(777.91342041,72.59470998)
\curveto(777.91342719,72.634706)(777.9084272,72.66970596)(777.89842041,72.69970998)
\curveto(777.88842722,72.7297059)(777.88842722,72.76470587)(777.89842041,72.80470998)
\curveto(777.89842721,72.85470578)(777.88842722,72.89470574)(777.86842041,72.92470998)
\lineto(777.86842041,73.08970998)
\lineto(777.86842041,73.17970998)
\curveto(777.85842725,73.2297054)(777.85842725,73.26970536)(777.86842041,73.29970998)
\curveto(777.87842723,73.34970528)(777.88342722,73.39970523)(777.88342041,73.44970998)
\curveto(777.87342723,73.50970512)(777.87342723,73.56470507)(777.88342041,73.61470998)
\curveto(777.91342719,73.72470491)(777.93342717,73.8297048)(777.94342041,73.92970998)
\curveto(777.95342715,74.03970459)(777.97842713,74.14470449)(778.01842041,74.24470998)
\curveto(778.15842695,74.66470397)(778.34342676,75.00970362)(778.57342041,75.27970998)
\curveto(778.79342631,75.54970308)(779.07842603,75.78970284)(779.42842041,75.99970998)
\curveto(779.56842554,76.07970255)(779.71842539,76.14470249)(779.87842041,76.19470998)
\curveto(780.02842508,76.24470239)(780.18842492,76.29470234)(780.35842041,76.34470998)
\moveto(781.66342041,75.09970998)
\curveto(781.61342349,75.10970352)(781.56842354,75.11470352)(781.52842041,75.11470998)
\lineto(781.37842041,75.11470998)
\curveto(781.06842404,75.11470352)(780.78342432,75.07470356)(780.52342041,74.99470998)
\curveto(780.46342464,74.97470366)(780.4084247,74.95470368)(780.35842041,74.93470998)
\curveto(780.29842481,74.92470371)(780.24342486,74.90970372)(780.19342041,74.88970998)
\curveto(779.7034254,74.66970396)(779.35342575,74.32470431)(779.14342041,73.85470998)
\curveto(779.11342599,73.77470486)(779.08842602,73.69470494)(779.06842041,73.61470998)
\lineto(779.00842041,73.37470998)
\curveto(778.98842612,73.29470534)(778.97842613,73.20470543)(778.97842041,73.10470998)
\lineto(778.97842041,72.78970998)
\curveto(778.99842611,72.76970586)(779.0084261,72.7297059)(779.00842041,72.66970998)
\curveto(778.99842611,72.61970601)(778.99842611,72.57470606)(779.00842041,72.53470998)
\lineto(779.06842041,72.29470998)
\curveto(779.07842603,72.22470641)(779.09842601,72.15470648)(779.12842041,72.08470998)
\curveto(779.38842572,71.48470715)(779.85342525,71.07970755)(780.52342041,70.86970998)
\curveto(780.6034245,70.83970779)(780.68342442,70.81970781)(780.76342041,70.80970998)
\curveto(780.84342426,70.79970783)(780.92842418,70.78470785)(781.01842041,70.76470998)
\lineto(781.16842041,70.76470998)
\curveto(781.2084239,70.75470788)(781.27842383,70.74970788)(781.37842041,70.74970998)
\curveto(781.6084235,70.74970788)(781.8034233,70.76970786)(781.96342041,70.80970998)
\curveto(782.03342307,70.8297078)(782.09842301,70.84470779)(782.15842041,70.85470998)
\curveto(782.21842289,70.86470777)(782.28342282,70.88470775)(782.35342041,70.91470998)
\curveto(782.63342247,71.02470761)(782.87842223,71.16970746)(783.08842041,71.34970998)
\curveto(783.28842182,71.5297071)(783.44842166,71.76470687)(783.56842041,72.05470998)
\lineto(783.65842041,72.29470998)
\lineto(783.71842041,72.53470998)
\curveto(783.73842137,72.58470605)(783.74342136,72.62470601)(783.73342041,72.65470998)
\curveto(783.72342138,72.69470594)(783.72842138,72.73970589)(783.74842041,72.78970998)
\curveto(783.75842135,72.81970581)(783.76342134,72.87470576)(783.76342041,72.95470998)
\curveto(783.76342134,73.0347056)(783.75842135,73.09470554)(783.74842041,73.13470998)
\curveto(783.72842138,73.24470539)(783.71342139,73.34970528)(783.70342041,73.44970998)
\curveto(783.69342141,73.54970508)(783.66342144,73.64470499)(783.61342041,73.73470998)
\curveto(783.41342169,74.26470437)(783.03842207,74.65470398)(782.48842041,74.90470998)
\curveto(782.38842272,74.94470369)(782.28342282,74.97470366)(782.17342041,74.99470998)
\lineto(781.84342041,75.08470998)
\curveto(781.76342334,75.08470355)(781.7034234,75.08970354)(781.66342041,75.09970998)
}
}
{
\newrgbcolor{curcolor}{0 0 0}
\pscustom[linestyle=none,fillstyle=solid,fillcolor=curcolor]
{
\newpath
\moveto(783.04342041,78.63431936)
\lineto(783.04342041,79.26431936)
\lineto(783.04342041,79.45931936)
\curveto(783.04342206,79.52931683)(783.05342205,79.58931677)(783.07342041,79.63931936)
\curveto(783.11342199,79.70931665)(783.15342195,79.7593166)(783.19342041,79.78931936)
\curveto(783.24342186,79.82931653)(783.3084218,79.84931651)(783.38842041,79.84931936)
\curveto(783.46842164,79.8593165)(783.55342155,79.86431649)(783.64342041,79.86431936)
\lineto(784.36342041,79.86431936)
\curveto(784.84342026,79.86431649)(785.25341985,79.80431655)(785.59342041,79.68431936)
\curveto(785.93341917,79.56431679)(786.2084189,79.36931699)(786.41842041,79.09931936)
\curveto(786.46841864,79.02931733)(786.51341859,78.9593174)(786.55342041,78.88931936)
\curveto(786.6034185,78.82931753)(786.64841846,78.7543176)(786.68842041,78.66431936)
\curveto(786.69841841,78.64431771)(786.7084184,78.61431774)(786.71842041,78.57431936)
\curveto(786.73841837,78.53431782)(786.74341836,78.48931787)(786.73342041,78.43931936)
\curveto(786.7034184,78.34931801)(786.62841848,78.29431806)(786.50842041,78.27431936)
\curveto(786.39841871,78.2543181)(786.3034188,78.26931809)(786.22342041,78.31931936)
\curveto(786.15341895,78.34931801)(786.08841902,78.39431796)(786.02842041,78.45431936)
\curveto(785.97841913,78.52431783)(785.92841918,78.58931777)(785.87842041,78.64931936)
\curveto(785.82841928,78.71931764)(785.75341935,78.77931758)(785.65342041,78.82931936)
\curveto(785.56341954,78.88931747)(785.47341963,78.93931742)(785.38342041,78.97931936)
\curveto(785.35341975,78.99931736)(785.29341981,79.02431733)(785.20342041,79.05431936)
\curveto(785.12341998,79.08431727)(785.05342005,79.08931727)(784.99342041,79.06931936)
\curveto(784.85342025,79.03931732)(784.76342034,78.97931738)(784.72342041,78.88931936)
\curveto(784.69342041,78.80931755)(784.67842043,78.71931764)(784.67842041,78.61931936)
\curveto(784.67842043,78.51931784)(784.65342045,78.43431792)(784.60342041,78.36431936)
\curveto(784.53342057,78.27431808)(784.39342071,78.22931813)(784.18342041,78.22931936)
\lineto(783.62842041,78.22931936)
\lineto(783.40342041,78.22931936)
\curveto(783.32342178,78.23931812)(783.25842185,78.2593181)(783.20842041,78.28931936)
\curveto(783.12842198,78.34931801)(783.08342202,78.41931794)(783.07342041,78.49931936)
\curveto(783.06342204,78.51931784)(783.05842205,78.53931782)(783.05842041,78.55931936)
\curveto(783.05842205,78.58931777)(783.05342205,78.61431774)(783.04342041,78.63431936)
}
}
{
\newrgbcolor{curcolor}{0 0 0}
\pscustom[linestyle=none,fillstyle=solid,fillcolor=curcolor]
{
}
}
{
\newrgbcolor{curcolor}{0 0 0}
\pscustom[linestyle=none,fillstyle=solid,fillcolor=curcolor]
{
\newpath
\moveto(774.07342041,89.26463186)
\curveto(774.06343104,89.95462722)(774.18343092,90.55462662)(774.43342041,91.06463186)
\curveto(774.68343042,91.58462559)(775.01843009,91.9796252)(775.43842041,92.24963186)
\curveto(775.51842959,92.29962488)(775.6084295,92.34462483)(775.70842041,92.38463186)
\curveto(775.79842931,92.42462475)(775.89342921,92.46962471)(775.99342041,92.51963186)
\curveto(776.09342901,92.55962462)(776.19342891,92.58962459)(776.29342041,92.60963186)
\curveto(776.39342871,92.62962455)(776.49842861,92.64962453)(776.60842041,92.66963186)
\curveto(776.65842845,92.68962449)(776.7034284,92.69462448)(776.74342041,92.68463186)
\curveto(776.78342832,92.6746245)(776.82842828,92.6796245)(776.87842041,92.69963186)
\curveto(776.92842818,92.70962447)(777.01342809,92.71462446)(777.13342041,92.71463186)
\curveto(777.24342786,92.71462446)(777.32842778,92.70962447)(777.38842041,92.69963186)
\curveto(777.44842766,92.6796245)(777.5084276,92.66962451)(777.56842041,92.66963186)
\curveto(777.62842748,92.6796245)(777.68842742,92.6746245)(777.74842041,92.65463186)
\curveto(777.88842722,92.61462456)(778.02342708,92.5796246)(778.15342041,92.54963186)
\curveto(778.28342682,92.51962466)(778.4084267,92.4796247)(778.52842041,92.42963186)
\curveto(778.66842644,92.36962481)(778.79342631,92.29962488)(778.90342041,92.21963186)
\curveto(779.01342609,92.14962503)(779.12342598,92.0746251)(779.23342041,91.99463186)
\lineto(779.29342041,91.93463186)
\curveto(779.31342579,91.92462525)(779.33342577,91.90962527)(779.35342041,91.88963186)
\curveto(779.51342559,91.76962541)(779.65842545,91.63462554)(779.78842041,91.48463186)
\curveto(779.91842519,91.33462584)(780.04342506,91.174626)(780.16342041,91.00463186)
\curveto(780.38342472,90.69462648)(780.58842452,90.39962678)(780.77842041,90.11963186)
\curveto(780.91842419,89.88962729)(781.05342405,89.65962752)(781.18342041,89.42963186)
\curveto(781.31342379,89.20962797)(781.44842366,88.98962819)(781.58842041,88.76963186)
\curveto(781.75842335,88.51962866)(781.93842317,88.2796289)(782.12842041,88.04963186)
\curveto(782.31842279,87.82962935)(782.54342256,87.63962954)(782.80342041,87.47963186)
\curveto(782.86342224,87.43962974)(782.92342218,87.40462977)(782.98342041,87.37463186)
\curveto(783.03342207,87.34462983)(783.09842201,87.31462986)(783.17842041,87.28463186)
\curveto(783.24842186,87.26462991)(783.3084218,87.25962992)(783.35842041,87.26963186)
\curveto(783.42842168,87.28962989)(783.48342162,87.32462985)(783.52342041,87.37463186)
\curveto(783.55342155,87.42462975)(783.57342153,87.48462969)(783.58342041,87.55463186)
\lineto(783.58342041,87.79463186)
\lineto(783.58342041,88.54463186)
\lineto(783.58342041,91.34963186)
\lineto(783.58342041,92.00963186)
\curveto(783.58342152,92.09962508)(783.58842152,92.18462499)(783.59842041,92.26463186)
\curveto(783.59842151,92.34462483)(783.61842149,92.40962477)(783.65842041,92.45963186)
\curveto(783.69842141,92.50962467)(783.77342133,92.54962463)(783.88342041,92.57963186)
\curveto(783.98342112,92.61962456)(784.08342102,92.62962455)(784.18342041,92.60963186)
\lineto(784.31842041,92.60963186)
\curveto(784.38842072,92.58962459)(784.44842066,92.56962461)(784.49842041,92.54963186)
\curveto(784.54842056,92.52962465)(784.58842052,92.49462468)(784.61842041,92.44463186)
\curveto(784.65842045,92.39462478)(784.67842043,92.32462485)(784.67842041,92.23463186)
\lineto(784.67842041,91.96463186)
\lineto(784.67842041,91.06463186)
\lineto(784.67842041,87.55463186)
\lineto(784.67842041,86.48963186)
\curveto(784.67842043,86.40963077)(784.68342042,86.31963086)(784.69342041,86.21963186)
\curveto(784.69342041,86.11963106)(784.68342042,86.03463114)(784.66342041,85.96463186)
\curveto(784.59342051,85.75463142)(784.41342069,85.68963149)(784.12342041,85.76963186)
\curveto(784.08342102,85.7796314)(784.04842106,85.7796314)(784.01842041,85.76963186)
\curveto(783.97842113,85.76963141)(783.93342117,85.7796314)(783.88342041,85.79963186)
\curveto(783.8034213,85.81963136)(783.71842139,85.83963134)(783.62842041,85.85963186)
\curveto(783.53842157,85.8796313)(783.45342165,85.90463127)(783.37342041,85.93463186)
\curveto(782.88342222,86.09463108)(782.46842264,86.29463088)(782.12842041,86.53463186)
\curveto(781.87842323,86.71463046)(781.65342345,86.91963026)(781.45342041,87.14963186)
\curveto(781.24342386,87.3796298)(781.04842406,87.61962956)(780.86842041,87.86963186)
\curveto(780.68842442,88.12962905)(780.51842459,88.39462878)(780.35842041,88.66463186)
\curveto(780.18842492,88.94462823)(780.01342509,89.21462796)(779.83342041,89.47463186)
\curveto(779.75342535,89.58462759)(779.67842543,89.68962749)(779.60842041,89.78963186)
\curveto(779.53842557,89.89962728)(779.46342564,90.00962717)(779.38342041,90.11963186)
\curveto(779.35342575,90.15962702)(779.32342578,90.19462698)(779.29342041,90.22463186)
\curveto(779.25342585,90.26462691)(779.22342588,90.30462687)(779.20342041,90.34463186)
\curveto(779.09342601,90.48462669)(778.96842614,90.60962657)(778.82842041,90.71963186)
\curveto(778.79842631,90.73962644)(778.77342633,90.76462641)(778.75342041,90.79463186)
\curveto(778.72342638,90.82462635)(778.69342641,90.84962633)(778.66342041,90.86963186)
\curveto(778.56342654,90.94962623)(778.46342664,91.01462616)(778.36342041,91.06463186)
\curveto(778.26342684,91.12462605)(778.15342695,91.179626)(778.03342041,91.22963186)
\curveto(777.96342714,91.25962592)(777.88842722,91.2796259)(777.80842041,91.28963186)
\lineto(777.56842041,91.34963186)
\lineto(777.47842041,91.34963186)
\curveto(777.44842766,91.35962582)(777.41842769,91.36462581)(777.38842041,91.36463186)
\curveto(777.31842779,91.38462579)(777.22342788,91.38962579)(777.10342041,91.37963186)
\curveto(776.97342813,91.3796258)(776.87342823,91.36962581)(776.80342041,91.34963186)
\curveto(776.72342838,91.32962585)(776.64842846,91.30962587)(776.57842041,91.28963186)
\curveto(776.49842861,91.2796259)(776.41842869,91.25962592)(776.33842041,91.22963186)
\curveto(776.09842901,91.11962606)(775.89842921,90.96962621)(775.73842041,90.77963186)
\curveto(775.56842954,90.59962658)(775.42842968,90.3796268)(775.31842041,90.11963186)
\curveto(775.29842981,90.04962713)(775.28342982,89.9796272)(775.27342041,89.90963186)
\curveto(775.25342985,89.83962734)(775.23342987,89.76462741)(775.21342041,89.68463186)
\curveto(775.19342991,89.60462757)(775.18342992,89.49462768)(775.18342041,89.35463186)
\curveto(775.18342992,89.22462795)(775.19342991,89.11962806)(775.21342041,89.03963186)
\curveto(775.22342988,88.9796282)(775.22842988,88.92462825)(775.22842041,88.87463186)
\curveto(775.22842988,88.82462835)(775.23842987,88.7746284)(775.25842041,88.72463186)
\curveto(775.29842981,88.62462855)(775.33842977,88.52962865)(775.37842041,88.43963186)
\curveto(775.41842969,88.35962882)(775.46342964,88.2796289)(775.51342041,88.19963186)
\curveto(775.53342957,88.16962901)(775.55842955,88.13962904)(775.58842041,88.10963186)
\curveto(775.61842949,88.08962909)(775.64342946,88.06462911)(775.66342041,88.03463186)
\lineto(775.73842041,87.95963186)
\curveto(775.75842935,87.92962925)(775.77842933,87.90462927)(775.79842041,87.88463186)
\lineto(776.00842041,87.73463186)
\curveto(776.06842904,87.69462948)(776.13342897,87.64962953)(776.20342041,87.59963186)
\curveto(776.29342881,87.53962964)(776.39842871,87.48962969)(776.51842041,87.44963186)
\curveto(776.62842848,87.41962976)(776.73842837,87.38462979)(776.84842041,87.34463186)
\curveto(776.95842815,87.30462987)(777.103428,87.2796299)(777.28342041,87.26963186)
\curveto(777.45342765,87.25962992)(777.57842753,87.22962995)(777.65842041,87.17963186)
\curveto(777.73842737,87.12963005)(777.78342732,87.05463012)(777.79342041,86.95463186)
\curveto(777.8034273,86.85463032)(777.8084273,86.74463043)(777.80842041,86.62463186)
\curveto(777.8084273,86.58463059)(777.81342729,86.54463063)(777.82342041,86.50463186)
\curveto(777.82342728,86.46463071)(777.81842729,86.42963075)(777.80842041,86.39963186)
\curveto(777.78842732,86.34963083)(777.77842733,86.29963088)(777.77842041,86.24963186)
\curveto(777.77842733,86.20963097)(777.76842734,86.16963101)(777.74842041,86.12963186)
\curveto(777.68842742,86.03963114)(777.55342755,85.99463118)(777.34342041,85.99463186)
\lineto(777.22342041,85.99463186)
\curveto(777.16342794,86.00463117)(777.103428,86.00963117)(777.04342041,86.00963186)
\curveto(776.97342813,86.01963116)(776.9084282,86.02963115)(776.84842041,86.03963186)
\curveto(776.73842837,86.05963112)(776.63842847,86.0796311)(776.54842041,86.09963186)
\curveto(776.44842866,86.11963106)(776.35342875,86.14963103)(776.26342041,86.18963186)
\curveto(776.19342891,86.20963097)(776.13342897,86.22963095)(776.08342041,86.24963186)
\lineto(775.90342041,86.30963186)
\curveto(775.64342946,86.42963075)(775.39842971,86.58463059)(775.16842041,86.77463186)
\curveto(774.93843017,86.9746302)(774.75343035,87.18962999)(774.61342041,87.41963186)
\curveto(774.53343057,87.52962965)(774.46843064,87.64462953)(774.41842041,87.76463186)
\lineto(774.26842041,88.15463186)
\curveto(774.21843089,88.26462891)(774.18843092,88.3796288)(774.17842041,88.49963186)
\curveto(774.15843095,88.61962856)(774.13343097,88.74462843)(774.10342041,88.87463186)
\curveto(774.103431,88.94462823)(774.103431,89.00962817)(774.10342041,89.06963186)
\curveto(774.09343101,89.12962805)(774.08343102,89.19462798)(774.07342041,89.26463186)
}
}
{
\newrgbcolor{curcolor}{0 0 0}
\pscustom[linestyle=none,fillstyle=solid,fillcolor=curcolor]
{
\newpath
\moveto(779.59342041,101.36424123)
\lineto(779.84842041,101.36424123)
\curveto(779.92842518,101.37423353)(780.0034251,101.36923353)(780.07342041,101.34924123)
\lineto(780.31342041,101.34924123)
\lineto(780.47842041,101.34924123)
\curveto(780.57842453,101.32923357)(780.68342442,101.31923358)(780.79342041,101.31924123)
\curveto(780.89342421,101.31923358)(780.99342411,101.30923359)(781.09342041,101.28924123)
\lineto(781.24342041,101.28924123)
\curveto(781.38342372,101.25923364)(781.52342358,101.23923366)(781.66342041,101.22924123)
\curveto(781.79342331,101.21923368)(781.92342318,101.19423371)(782.05342041,101.15424123)
\curveto(782.13342297,101.13423377)(782.21842289,101.11423379)(782.30842041,101.09424123)
\lineto(782.54842041,101.03424123)
\lineto(782.84842041,100.91424123)
\curveto(782.93842217,100.88423402)(783.02842208,100.84923405)(783.11842041,100.80924123)
\curveto(783.33842177,100.70923419)(783.55342155,100.57423433)(783.76342041,100.40424123)
\curveto(783.97342113,100.24423466)(784.14342096,100.06923483)(784.27342041,99.87924123)
\curveto(784.31342079,99.82923507)(784.35342075,99.76923513)(784.39342041,99.69924123)
\curveto(784.42342068,99.63923526)(784.45842065,99.57923532)(784.49842041,99.51924123)
\curveto(784.54842056,99.43923546)(784.58842052,99.34423556)(784.61842041,99.23424123)
\curveto(784.64842046,99.12423578)(784.67842043,99.01923588)(784.70842041,98.91924123)
\curveto(784.74842036,98.80923609)(784.77342033,98.6992362)(784.78342041,98.58924123)
\curveto(784.79342031,98.47923642)(784.8084203,98.36423654)(784.82842041,98.24424123)
\curveto(784.83842027,98.2042367)(784.83842027,98.15923674)(784.82842041,98.10924123)
\curveto(784.82842028,98.06923683)(784.83342027,98.02923687)(784.84342041,97.98924123)
\curveto(784.85342025,97.94923695)(784.85842025,97.89423701)(784.85842041,97.82424123)
\curveto(784.85842025,97.75423715)(784.85342025,97.7042372)(784.84342041,97.67424123)
\curveto(784.82342028,97.62423728)(784.81842029,97.57923732)(784.82842041,97.53924123)
\curveto(784.83842027,97.4992374)(784.83842027,97.46423744)(784.82842041,97.43424123)
\lineto(784.82842041,97.34424123)
\curveto(784.8084203,97.28423762)(784.79342031,97.21923768)(784.78342041,97.14924123)
\curveto(784.78342032,97.08923781)(784.77842033,97.02423788)(784.76842041,96.95424123)
\curveto(784.71842039,96.78423812)(784.66842044,96.62423828)(784.61842041,96.47424123)
\curveto(784.56842054,96.32423858)(784.5034206,96.17923872)(784.42342041,96.03924123)
\curveto(784.38342072,95.98923891)(784.35342075,95.93423897)(784.33342041,95.87424123)
\curveto(784.3034208,95.82423908)(784.26842084,95.77423913)(784.22842041,95.72424123)
\curveto(784.04842106,95.48423942)(783.82842128,95.28423962)(783.56842041,95.12424123)
\curveto(783.3084218,94.96423994)(783.02342208,94.82424008)(782.71342041,94.70424123)
\curveto(782.57342253,94.64424026)(782.43342267,94.5992403)(782.29342041,94.56924123)
\curveto(782.14342296,94.53924036)(781.98842312,94.5042404)(781.82842041,94.46424123)
\curveto(781.71842339,94.44424046)(781.6084235,94.42924047)(781.49842041,94.41924123)
\curveto(781.38842372,94.40924049)(781.27842383,94.39424051)(781.16842041,94.37424123)
\curveto(781.12842398,94.36424054)(781.08842402,94.35924054)(781.04842041,94.35924123)
\curveto(781.0084241,94.36924053)(780.96842414,94.36924053)(780.92842041,94.35924123)
\curveto(780.87842423,94.34924055)(780.82842428,94.34424056)(780.77842041,94.34424123)
\lineto(780.61342041,94.34424123)
\curveto(780.56342454,94.32424058)(780.51342459,94.31924058)(780.46342041,94.32924123)
\curveto(780.4034247,94.33924056)(780.34842476,94.33924056)(780.29842041,94.32924123)
\curveto(780.25842485,94.31924058)(780.21342489,94.31924058)(780.16342041,94.32924123)
\curveto(780.11342499,94.33924056)(780.06342504,94.33424057)(780.01342041,94.31424123)
\curveto(779.94342516,94.29424061)(779.86842524,94.28924061)(779.78842041,94.29924123)
\curveto(779.69842541,94.30924059)(779.61342549,94.31424059)(779.53342041,94.31424123)
\curveto(779.44342566,94.31424059)(779.34342576,94.30924059)(779.23342041,94.29924123)
\curveto(779.11342599,94.28924061)(779.01342609,94.29424061)(778.93342041,94.31424123)
\lineto(778.64842041,94.31424123)
\lineto(778.01842041,94.35924123)
\curveto(777.91842719,94.36924053)(777.82342728,94.37924052)(777.73342041,94.38924123)
\lineto(777.43342041,94.41924123)
\curveto(777.38342772,94.43924046)(777.33342777,94.44424046)(777.28342041,94.43424123)
\curveto(777.22342788,94.43424047)(777.16842794,94.44424046)(777.11842041,94.46424123)
\curveto(776.94842816,94.51424039)(776.78342832,94.55424035)(776.62342041,94.58424123)
\curveto(776.45342865,94.61424029)(776.29342881,94.66424024)(776.14342041,94.73424123)
\curveto(775.68342942,94.92423998)(775.3084298,95.14423976)(775.01842041,95.39424123)
\curveto(774.72843038,95.65423925)(774.48343062,96.01423889)(774.28342041,96.47424123)
\curveto(774.23343087,96.6042383)(774.19843091,96.73423817)(774.17842041,96.86424123)
\curveto(774.15843095,97.0042379)(774.13343097,97.14423776)(774.10342041,97.28424123)
\curveto(774.09343101,97.35423755)(774.08843102,97.41923748)(774.08842041,97.47924123)
\curveto(774.08843102,97.53923736)(774.08343102,97.6042373)(774.07342041,97.67424123)
\curveto(774.05343105,98.5042364)(774.2034309,99.17423573)(774.52342041,99.68424123)
\curveto(774.83343027,100.19423471)(775.27342983,100.57423433)(775.84342041,100.82424123)
\curveto(775.96342914,100.87423403)(776.08842902,100.91923398)(776.21842041,100.95924123)
\curveto(776.34842876,100.9992339)(776.48342862,101.04423386)(776.62342041,101.09424123)
\curveto(776.7034284,101.11423379)(776.78842832,101.12923377)(776.87842041,101.13924123)
\lineto(777.11842041,101.19924123)
\curveto(777.22842788,101.22923367)(777.33842777,101.24423366)(777.44842041,101.24424123)
\curveto(777.55842755,101.25423365)(777.66842744,101.26923363)(777.77842041,101.28924123)
\curveto(777.82842728,101.30923359)(777.87342723,101.31423359)(777.91342041,101.30424123)
\curveto(777.95342715,101.3042336)(777.99342711,101.30923359)(778.03342041,101.31924123)
\curveto(778.08342702,101.32923357)(778.13842697,101.32923357)(778.19842041,101.31924123)
\curveto(778.24842686,101.31923358)(778.29842681,101.32423358)(778.34842041,101.33424123)
\lineto(778.48342041,101.33424123)
\curveto(778.54342656,101.35423355)(778.61342649,101.35423355)(778.69342041,101.33424123)
\curveto(778.76342634,101.32423358)(778.82842628,101.32923357)(778.88842041,101.34924123)
\curveto(778.91842619,101.35923354)(778.95842615,101.36423354)(779.00842041,101.36424123)
\lineto(779.12842041,101.36424123)
\lineto(779.59342041,101.36424123)
\moveto(781.91842041,99.81924123)
\curveto(781.59842351,99.91923498)(781.23342387,99.97923492)(780.82342041,99.99924123)
\curveto(780.41342469,100.01923488)(780.0034251,100.02923487)(779.59342041,100.02924123)
\curveto(779.16342594,100.02923487)(778.74342636,100.01923488)(778.33342041,99.99924123)
\curveto(777.92342718,99.97923492)(777.53842757,99.93423497)(777.17842041,99.86424123)
\curveto(776.81842829,99.79423511)(776.49842861,99.68423522)(776.21842041,99.53424123)
\curveto(775.92842918,99.39423551)(775.69342941,99.1992357)(775.51342041,98.94924123)
\curveto(775.4034297,98.78923611)(775.32342978,98.60923629)(775.27342041,98.40924123)
\curveto(775.21342989,98.20923669)(775.18342992,97.96423694)(775.18342041,97.67424123)
\curveto(775.2034299,97.65423725)(775.21342989,97.61923728)(775.21342041,97.56924123)
\curveto(775.2034299,97.51923738)(775.2034299,97.47923742)(775.21342041,97.44924123)
\curveto(775.23342987,97.36923753)(775.25342985,97.29423761)(775.27342041,97.22424123)
\curveto(775.28342982,97.16423774)(775.3034298,97.0992378)(775.33342041,97.02924123)
\curveto(775.45342965,96.75923814)(775.62342948,96.53923836)(775.84342041,96.36924123)
\curveto(776.05342905,96.20923869)(776.29842881,96.07423883)(776.57842041,95.96424123)
\curveto(776.68842842,95.91423899)(776.8084283,95.87423903)(776.93842041,95.84424123)
\curveto(777.05842805,95.82423908)(777.18342792,95.7992391)(777.31342041,95.76924123)
\curveto(777.36342774,95.74923915)(777.41842769,95.73923916)(777.47842041,95.73924123)
\curveto(777.52842758,95.73923916)(777.57842753,95.73423917)(777.62842041,95.72424123)
\curveto(777.71842739,95.71423919)(777.81342729,95.7042392)(777.91342041,95.69424123)
\curveto(778.0034271,95.68423922)(778.09842701,95.67423923)(778.19842041,95.66424123)
\curveto(778.27842683,95.66423924)(778.36342674,95.65923924)(778.45342041,95.64924123)
\lineto(778.69342041,95.64924123)
\lineto(778.87342041,95.64924123)
\curveto(778.9034262,95.63923926)(778.93842617,95.63423927)(778.97842041,95.63424123)
\lineto(779.11342041,95.63424123)
\lineto(779.56342041,95.63424123)
\curveto(779.64342546,95.63423927)(779.72842538,95.62923927)(779.81842041,95.61924123)
\curveto(779.89842521,95.61923928)(779.97342513,95.62923927)(780.04342041,95.64924123)
\lineto(780.31342041,95.64924123)
\curveto(780.33342477,95.64923925)(780.36342474,95.64423926)(780.40342041,95.63424123)
\curveto(780.43342467,95.63423927)(780.45842465,95.63923926)(780.47842041,95.64924123)
\curveto(780.57842453,95.65923924)(780.67842443,95.66423924)(780.77842041,95.66424123)
\curveto(780.86842424,95.67423923)(780.96842414,95.68423922)(781.07842041,95.69424123)
\curveto(781.19842391,95.72423918)(781.32342378,95.73923916)(781.45342041,95.73924123)
\curveto(781.57342353,95.74923915)(781.68842342,95.77423913)(781.79842041,95.81424123)
\curveto(782.09842301,95.89423901)(782.36342274,95.97923892)(782.59342041,96.06924123)
\curveto(782.82342228,96.16923873)(783.03842207,96.31423859)(783.23842041,96.50424123)
\curveto(783.43842167,96.71423819)(783.58842152,96.97923792)(783.68842041,97.29924123)
\curveto(783.7084214,97.33923756)(783.71842139,97.37423753)(783.71842041,97.40424123)
\curveto(783.7084214,97.44423746)(783.71342139,97.48923741)(783.73342041,97.53924123)
\curveto(783.74342136,97.57923732)(783.75342135,97.64923725)(783.76342041,97.74924123)
\curveto(783.77342133,97.85923704)(783.76842134,97.94423696)(783.74842041,98.00424123)
\curveto(783.72842138,98.07423683)(783.71842139,98.14423676)(783.71842041,98.21424123)
\curveto(783.7084214,98.28423662)(783.69342141,98.34923655)(783.67342041,98.40924123)
\curveto(783.61342149,98.60923629)(783.52842158,98.78923611)(783.41842041,98.94924123)
\curveto(783.39842171,98.97923592)(783.37842173,99.0042359)(783.35842041,99.02424123)
\lineto(783.29842041,99.08424123)
\curveto(783.27842183,99.12423578)(783.23842187,99.17423573)(783.17842041,99.23424123)
\curveto(783.03842207,99.33423557)(782.9084222,99.41923548)(782.78842041,99.48924123)
\curveto(782.66842244,99.55923534)(782.52342258,99.62923527)(782.35342041,99.69924123)
\curveto(782.28342282,99.72923517)(782.21342289,99.74923515)(782.14342041,99.75924123)
\curveto(782.07342303,99.77923512)(781.99842311,99.7992351)(781.91842041,99.81924123)
}
}
{
\newrgbcolor{curcolor}{0 0 0}
\pscustom[linestyle=none,fillstyle=solid,fillcolor=curcolor]
{
\newpath
\moveto(774.07342041,106.77385061)
\curveto(774.07343103,106.87384575)(774.08343102,106.96884566)(774.10342041,107.05885061)
\curveto(774.11343099,107.14884548)(774.14343096,107.21384541)(774.19342041,107.25385061)
\curveto(774.27343083,107.31384531)(774.37843073,107.34384528)(774.50842041,107.34385061)
\lineto(774.89842041,107.34385061)
\lineto(776.39842041,107.34385061)
\lineto(782.78842041,107.34385061)
\lineto(783.95842041,107.34385061)
\lineto(784.27342041,107.34385061)
\curveto(784.37342073,107.35384527)(784.45342065,107.33884529)(784.51342041,107.29885061)
\curveto(784.59342051,107.24884538)(784.64342046,107.17384545)(784.66342041,107.07385061)
\curveto(784.67342043,106.98384564)(784.67842043,106.87384575)(784.67842041,106.74385061)
\lineto(784.67842041,106.51885061)
\curveto(784.65842045,106.43884619)(784.64342046,106.36884626)(784.63342041,106.30885061)
\curveto(784.61342049,106.24884638)(784.57342053,106.19884643)(784.51342041,106.15885061)
\curveto(784.45342065,106.11884651)(784.37842073,106.09884653)(784.28842041,106.09885061)
\lineto(783.98842041,106.09885061)
\lineto(782.89342041,106.09885061)
\lineto(777.55342041,106.09885061)
\curveto(777.46342764,106.07884655)(777.38842772,106.06384656)(777.32842041,106.05385061)
\curveto(777.25842785,106.05384657)(777.19842791,106.0238466)(777.14842041,105.96385061)
\curveto(777.09842801,105.89384673)(777.07342803,105.80384682)(777.07342041,105.69385061)
\curveto(777.06342804,105.59384703)(777.05842805,105.48384714)(777.05842041,105.36385061)
\lineto(777.05842041,104.22385061)
\lineto(777.05842041,103.72885061)
\curveto(777.04842806,103.56884906)(776.98842812,103.45884917)(776.87842041,103.39885061)
\curveto(776.84842826,103.37884925)(776.81842829,103.36884926)(776.78842041,103.36885061)
\curveto(776.74842836,103.36884926)(776.7034284,103.36384926)(776.65342041,103.35385061)
\curveto(776.53342857,103.33384929)(776.42342868,103.33884929)(776.32342041,103.36885061)
\curveto(776.22342888,103.40884922)(776.15342895,103.46384916)(776.11342041,103.53385061)
\curveto(776.06342904,103.61384901)(776.03842907,103.73384889)(776.03842041,103.89385061)
\curveto(776.03842907,104.05384857)(776.02342908,104.18884844)(775.99342041,104.29885061)
\curveto(775.98342912,104.34884828)(775.97842913,104.40384822)(775.97842041,104.46385061)
\curveto(775.96842914,104.5238481)(775.95342915,104.58384804)(775.93342041,104.64385061)
\curveto(775.88342922,104.79384783)(775.83342927,104.93884769)(775.78342041,105.07885061)
\curveto(775.72342938,105.21884741)(775.65342945,105.35384727)(775.57342041,105.48385061)
\curveto(775.48342962,105.623847)(775.37842973,105.74384688)(775.25842041,105.84385061)
\curveto(775.13842997,105.94384668)(775.0084301,106.03884659)(774.86842041,106.12885061)
\curveto(774.76843034,106.18884644)(774.65843045,106.23384639)(774.53842041,106.26385061)
\curveto(774.41843069,106.30384632)(774.31343079,106.35384627)(774.22342041,106.41385061)
\curveto(774.16343094,106.46384616)(774.12343098,106.53384609)(774.10342041,106.62385061)
\curveto(774.09343101,106.64384598)(774.08843102,106.66884596)(774.08842041,106.69885061)
\curveto(774.08843102,106.7288459)(774.08343102,106.75384587)(774.07342041,106.77385061)
}
}
{
\newrgbcolor{curcolor}{0 0 0}
\pscustom[linestyle=none,fillstyle=solid,fillcolor=curcolor]
{
\newpath
\moveto(774.07342041,115.12345998)
\curveto(774.07343103,115.22345513)(774.08343102,115.31845503)(774.10342041,115.40845998)
\curveto(774.11343099,115.49845485)(774.14343096,115.56345479)(774.19342041,115.60345998)
\curveto(774.27343083,115.66345469)(774.37843073,115.69345466)(774.50842041,115.69345998)
\lineto(774.89842041,115.69345998)
\lineto(776.39842041,115.69345998)
\lineto(782.78842041,115.69345998)
\lineto(783.95842041,115.69345998)
\lineto(784.27342041,115.69345998)
\curveto(784.37342073,115.70345465)(784.45342065,115.68845466)(784.51342041,115.64845998)
\curveto(784.59342051,115.59845475)(784.64342046,115.52345483)(784.66342041,115.42345998)
\curveto(784.67342043,115.33345502)(784.67842043,115.22345513)(784.67842041,115.09345998)
\lineto(784.67842041,114.86845998)
\curveto(784.65842045,114.78845556)(784.64342046,114.71845563)(784.63342041,114.65845998)
\curveto(784.61342049,114.59845575)(784.57342053,114.5484558)(784.51342041,114.50845998)
\curveto(784.45342065,114.46845588)(784.37842073,114.4484559)(784.28842041,114.44845998)
\lineto(783.98842041,114.44845998)
\lineto(782.89342041,114.44845998)
\lineto(777.55342041,114.44845998)
\curveto(777.46342764,114.42845592)(777.38842772,114.41345594)(777.32842041,114.40345998)
\curveto(777.25842785,114.40345595)(777.19842791,114.37345598)(777.14842041,114.31345998)
\curveto(777.09842801,114.24345611)(777.07342803,114.1534562)(777.07342041,114.04345998)
\curveto(777.06342804,113.94345641)(777.05842805,113.83345652)(777.05842041,113.71345998)
\lineto(777.05842041,112.57345998)
\lineto(777.05842041,112.07845998)
\curveto(777.04842806,111.91845843)(776.98842812,111.80845854)(776.87842041,111.74845998)
\curveto(776.84842826,111.72845862)(776.81842829,111.71845863)(776.78842041,111.71845998)
\curveto(776.74842836,111.71845863)(776.7034284,111.71345864)(776.65342041,111.70345998)
\curveto(776.53342857,111.68345867)(776.42342868,111.68845866)(776.32342041,111.71845998)
\curveto(776.22342888,111.75845859)(776.15342895,111.81345854)(776.11342041,111.88345998)
\curveto(776.06342904,111.96345839)(776.03842907,112.08345827)(776.03842041,112.24345998)
\curveto(776.03842907,112.40345795)(776.02342908,112.53845781)(775.99342041,112.64845998)
\curveto(775.98342912,112.69845765)(775.97842913,112.7534576)(775.97842041,112.81345998)
\curveto(775.96842914,112.87345748)(775.95342915,112.93345742)(775.93342041,112.99345998)
\curveto(775.88342922,113.14345721)(775.83342927,113.28845706)(775.78342041,113.42845998)
\curveto(775.72342938,113.56845678)(775.65342945,113.70345665)(775.57342041,113.83345998)
\curveto(775.48342962,113.97345638)(775.37842973,114.09345626)(775.25842041,114.19345998)
\curveto(775.13842997,114.29345606)(775.0084301,114.38845596)(774.86842041,114.47845998)
\curveto(774.76843034,114.53845581)(774.65843045,114.58345577)(774.53842041,114.61345998)
\curveto(774.41843069,114.6534557)(774.31343079,114.70345565)(774.22342041,114.76345998)
\curveto(774.16343094,114.81345554)(774.12343098,114.88345547)(774.10342041,114.97345998)
\curveto(774.09343101,114.99345536)(774.08843102,115.01845533)(774.08842041,115.04845998)
\curveto(774.08843102,115.07845527)(774.08343102,115.10345525)(774.07342041,115.12345998)
}
}
{
\newrgbcolor{curcolor}{0 0 0}
\pscustom[linestyle=none,fillstyle=solid,fillcolor=curcolor]
{
\newpath
\moveto(794.90973633,37.28705373)
\curveto(794.90974702,37.35704805)(794.90974702,37.43704797)(794.90973633,37.52705373)
\curveto(794.89974703,37.61704779)(794.89974703,37.70204771)(794.90973633,37.78205373)
\curveto(794.90974702,37.87204754)(794.91974701,37.95204746)(794.93973633,38.02205373)
\curveto(794.95974697,38.10204731)(794.98974694,38.15704725)(795.02973633,38.18705373)
\curveto(795.07974685,38.21704719)(795.15474678,38.23704717)(795.25473633,38.24705373)
\curveto(795.34474659,38.26704714)(795.44974648,38.27704713)(795.56973633,38.27705373)
\curveto(795.67974625,38.28704712)(795.79474614,38.28704712)(795.91473633,38.27705373)
\lineto(796.21473633,38.27705373)
\lineto(799.22973633,38.27705373)
\lineto(802.12473633,38.27705373)
\curveto(802.45473948,38.27704713)(802.77973915,38.27204714)(803.09973633,38.26205373)
\curveto(803.40973852,38.26204715)(803.68973824,38.22204719)(803.93973633,38.14205373)
\curveto(804.28973764,38.02204739)(804.58473735,37.86704754)(804.82473633,37.67705373)
\curveto(805.05473688,37.48704792)(805.25473668,37.24704816)(805.42473633,36.95705373)
\curveto(805.47473646,36.89704851)(805.50973642,36.83204858)(805.52973633,36.76205373)
\curveto(805.54973638,36.70204871)(805.57473636,36.63204878)(805.60473633,36.55205373)
\curveto(805.65473628,36.43204898)(805.68973624,36.30204911)(805.70973633,36.16205373)
\curveto(805.73973619,36.03204938)(805.76973616,35.89704951)(805.79973633,35.75705373)
\curveto(805.81973611,35.7070497)(805.82473611,35.65704975)(805.81473633,35.60705373)
\curveto(805.80473613,35.55704985)(805.80473613,35.50204991)(805.81473633,35.44205373)
\curveto(805.82473611,35.42204999)(805.82473611,35.39705001)(805.81473633,35.36705373)
\curveto(805.81473612,35.33705007)(805.81973611,35.3120501)(805.82973633,35.29205373)
\curveto(805.83973609,35.25205016)(805.84473609,35.19705021)(805.84473633,35.12705373)
\curveto(805.84473609,35.05705035)(805.83973609,35.00205041)(805.82973633,34.96205373)
\curveto(805.81973611,34.9120505)(805.81973611,34.85705055)(805.82973633,34.79705373)
\curveto(805.83973609,34.73705067)(805.8347361,34.68205073)(805.81473633,34.63205373)
\curveto(805.78473615,34.50205091)(805.76473617,34.37705103)(805.75473633,34.25705373)
\curveto(805.74473619,34.13705127)(805.71973621,34.02205139)(805.67973633,33.91205373)
\curveto(805.55973637,33.54205187)(805.38973654,33.22205219)(805.16973633,32.95205373)
\curveto(804.94973698,32.68205273)(804.66973726,32.47205294)(804.32973633,32.32205373)
\curveto(804.20973772,32.27205314)(804.08473785,32.22705318)(803.95473633,32.18705373)
\curveto(803.82473811,32.15705325)(803.68973824,32.12205329)(803.54973633,32.08205373)
\curveto(803.49973843,32.07205334)(803.45973847,32.06705334)(803.42973633,32.06705373)
\curveto(803.38973854,32.06705334)(803.34473859,32.06205335)(803.29473633,32.05205373)
\curveto(803.26473867,32.04205337)(803.2297387,32.03705337)(803.18973633,32.03705373)
\curveto(803.13973879,32.03705337)(803.09973883,32.03205338)(803.06973633,32.02205373)
\lineto(802.90473633,32.02205373)
\curveto(802.82473911,32.00205341)(802.72473921,31.99705341)(802.60473633,32.00705373)
\curveto(802.47473946,32.01705339)(802.38473955,32.03205338)(802.33473633,32.05205373)
\curveto(802.24473969,32.07205334)(802.17973975,32.12705328)(802.13973633,32.21705373)
\curveto(802.11973981,32.24705316)(802.11473982,32.27705313)(802.12473633,32.30705373)
\curveto(802.12473981,32.33705307)(802.11973981,32.37705303)(802.10973633,32.42705373)
\curveto(802.09973983,32.46705294)(802.09473984,32.5070529)(802.09473633,32.54705373)
\lineto(802.09473633,32.69705373)
\curveto(802.09473984,32.81705259)(802.09973983,32.93705247)(802.10973633,33.05705373)
\curveto(802.10973982,33.18705222)(802.14473979,33.27705213)(802.21473633,33.32705373)
\curveto(802.27473966,33.36705204)(802.3347396,33.38705202)(802.39473633,33.38705373)
\curveto(802.45473948,33.38705202)(802.52473941,33.39705201)(802.60473633,33.41705373)
\curveto(802.6347393,33.42705198)(802.66973926,33.42705198)(802.70973633,33.41705373)
\curveto(802.73973919,33.41705199)(802.76473917,33.42205199)(802.78473633,33.43205373)
\lineto(802.99473633,33.43205373)
\curveto(803.04473889,33.45205196)(803.09473884,33.45705195)(803.14473633,33.44705373)
\curveto(803.18473875,33.44705196)(803.2297387,33.45705195)(803.27973633,33.47705373)
\curveto(803.40973852,33.5070519)(803.5347384,33.53705187)(803.65473633,33.56705373)
\curveto(803.76473817,33.59705181)(803.86973806,33.64205177)(803.96973633,33.70205373)
\curveto(804.25973767,33.87205154)(804.46473747,34.14205127)(804.58473633,34.51205373)
\curveto(804.60473733,34.56205085)(804.61973731,34.6120508)(804.62973633,34.66205373)
\curveto(804.6297373,34.72205069)(804.63973729,34.77705063)(804.65973633,34.82705373)
\lineto(804.65973633,34.90205373)
\curveto(804.66973726,34.97205044)(804.67973725,35.06705034)(804.68973633,35.18705373)
\curveto(804.68973724,35.31705009)(804.67973725,35.41704999)(804.65973633,35.48705373)
\curveto(804.63973729,35.55704985)(804.62473731,35.62704978)(804.61473633,35.69705373)
\curveto(804.59473734,35.77704963)(804.57473736,35.84704956)(804.55473633,35.90705373)
\curveto(804.39473754,36.28704912)(804.11973781,36.56204885)(803.72973633,36.73205373)
\curveto(803.59973833,36.78204863)(803.44473849,36.81704859)(803.26473633,36.83705373)
\curveto(803.08473885,36.86704854)(802.89973903,36.88204853)(802.70973633,36.88205373)
\curveto(802.50973942,36.89204852)(802.30973962,36.89204852)(802.10973633,36.88205373)
\lineto(801.53973633,36.88205373)
\lineto(797.29473633,36.88205373)
\lineto(795.74973633,36.88205373)
\curveto(795.63974629,36.88204853)(795.51974641,36.87704853)(795.38973633,36.86705373)
\curveto(795.25974667,36.85704855)(795.15474678,36.87704853)(795.07473633,36.92705373)
\curveto(795.00474693,36.98704842)(794.95474698,37.06704834)(794.92473633,37.16705373)
\curveto(794.92474701,37.18704822)(794.92474701,37.2070482)(794.92473633,37.22705373)
\curveto(794.92474701,37.24704816)(794.91974701,37.26704814)(794.90973633,37.28705373)
}
}
{
\newrgbcolor{curcolor}{0 0 0}
\pscustom[linestyle=none,fillstyle=solid,fillcolor=curcolor]
{
\newpath
\moveto(797.86473633,40.82072561)
\lineto(797.86473633,41.25572561)
\curveto(797.86474407,41.40572364)(797.90474403,41.51072354)(797.98473633,41.57072561)
\curveto(798.06474387,41.62072343)(798.16474377,41.6457234)(798.28473633,41.64572561)
\curveto(798.40474353,41.65572339)(798.52474341,41.66072339)(798.64473633,41.66072561)
\lineto(800.06973633,41.66072561)
\lineto(802.33473633,41.66072561)
\lineto(803.02473633,41.66072561)
\curveto(803.25473868,41.66072339)(803.45473848,41.68572336)(803.62473633,41.73572561)
\curveto(804.07473786,41.89572315)(804.38973754,42.19572285)(804.56973633,42.63572561)
\curveto(804.65973727,42.85572219)(804.69473724,43.12072193)(804.67473633,43.43072561)
\curveto(804.64473729,43.74072131)(804.58973734,43.99072106)(804.50973633,44.18072561)
\curveto(804.36973756,44.51072054)(804.19473774,44.77072028)(803.98473633,44.96072561)
\curveto(803.76473817,45.16071989)(803.47973845,45.31571973)(803.12973633,45.42572561)
\curveto(803.04973888,45.45571959)(802.96973896,45.47571957)(802.88973633,45.48572561)
\curveto(802.80973912,45.49571955)(802.72473921,45.51071954)(802.63473633,45.53072561)
\curveto(802.58473935,45.54071951)(802.53973939,45.54071951)(802.49973633,45.53072561)
\curveto(802.45973947,45.53071952)(802.41473952,45.54071951)(802.36473633,45.56072561)
\lineto(802.04973633,45.56072561)
\curveto(801.96973996,45.58071947)(801.87974005,45.58571946)(801.77973633,45.57572561)
\curveto(801.66974026,45.56571948)(801.56974036,45.56071949)(801.47973633,45.56072561)
\lineto(800.30973633,45.56072561)
\lineto(798.71973633,45.56072561)
\curveto(798.59974333,45.56071949)(798.47474346,45.55571949)(798.34473633,45.54572561)
\curveto(798.20474373,45.5457195)(798.09474384,45.57071948)(798.01473633,45.62072561)
\curveto(797.96474397,45.66071939)(797.934744,45.70571934)(797.92473633,45.75572561)
\curveto(797.90474403,45.81571923)(797.88474405,45.88571916)(797.86473633,45.96572561)
\lineto(797.86473633,46.19072561)
\curveto(797.86474407,46.31071874)(797.86974406,46.41571863)(797.87973633,46.50572561)
\curveto(797.88974404,46.60571844)(797.934744,46.68071837)(798.01473633,46.73072561)
\curveto(798.06474387,46.78071827)(798.13974379,46.80571824)(798.23973633,46.80572561)
\lineto(798.52473633,46.80572561)
\lineto(799.54473633,46.80572561)
\lineto(803.57973633,46.80572561)
\lineto(804.92973633,46.80572561)
\curveto(805.04973688,46.80571824)(805.16473677,46.80071825)(805.27473633,46.79072561)
\curveto(805.37473656,46.79071826)(805.44973648,46.75571829)(805.49973633,46.68572561)
\curveto(805.5297364,46.6457184)(805.55473638,46.58571846)(805.57473633,46.50572561)
\curveto(805.58473635,46.42571862)(805.59473634,46.33571871)(805.60473633,46.23572561)
\curveto(805.60473633,46.1457189)(805.59973633,46.05571899)(805.58973633,45.96572561)
\curveto(805.57973635,45.88571916)(805.55973637,45.82571922)(805.52973633,45.78572561)
\curveto(805.48973644,45.73571931)(805.42473651,45.69071936)(805.33473633,45.65072561)
\curveto(805.29473664,45.64071941)(805.23973669,45.63071942)(805.16973633,45.62072561)
\curveto(805.09973683,45.62071943)(805.0347369,45.61571943)(804.97473633,45.60572561)
\curveto(804.90473703,45.59571945)(804.84973708,45.57571947)(804.80973633,45.54572561)
\curveto(804.76973716,45.51571953)(804.75473718,45.47071958)(804.76473633,45.41072561)
\curveto(804.78473715,45.33071972)(804.84473709,45.2507198)(804.94473633,45.17072561)
\curveto(805.0347369,45.09071996)(805.10473683,45.01572003)(805.15473633,44.94572561)
\curveto(805.31473662,44.72572032)(805.45473648,44.47572057)(805.57473633,44.19572561)
\curveto(805.62473631,44.08572096)(805.65473628,43.97072108)(805.66473633,43.85072561)
\curveto(805.68473625,43.74072131)(805.70973622,43.62572142)(805.73973633,43.50572561)
\curveto(805.74973618,43.45572159)(805.74973618,43.40072165)(805.73973633,43.34072561)
\curveto(805.7297362,43.29072176)(805.7347362,43.24072181)(805.75473633,43.19072561)
\curveto(805.77473616,43.09072196)(805.77473616,43.00072205)(805.75473633,42.92072561)
\lineto(805.75473633,42.77072561)
\curveto(805.7347362,42.72072233)(805.72473621,42.66072239)(805.72473633,42.59072561)
\curveto(805.72473621,42.53072252)(805.71973621,42.47572257)(805.70973633,42.42572561)
\curveto(805.68973624,42.38572266)(805.67973625,42.3457227)(805.67973633,42.30572561)
\curveto(805.68973624,42.27572277)(805.68473625,42.23572281)(805.66473633,42.18572561)
\lineto(805.60473633,41.94572561)
\curveto(805.58473635,41.87572317)(805.55473638,41.80072325)(805.51473633,41.72072561)
\curveto(805.40473653,41.46072359)(805.25973667,41.24072381)(805.07973633,41.06072561)
\curveto(804.88973704,40.89072416)(804.66473727,40.7507243)(804.40473633,40.64072561)
\curveto(804.31473762,40.60072445)(804.22473771,40.57072448)(804.13473633,40.55072561)
\lineto(803.83473633,40.49072561)
\curveto(803.77473816,40.47072458)(803.71973821,40.46072459)(803.66973633,40.46072561)
\curveto(803.60973832,40.47072458)(803.54473839,40.46572458)(803.47473633,40.44572561)
\curveto(803.45473848,40.43572461)(803.4297385,40.43072462)(803.39973633,40.43072561)
\curveto(803.35973857,40.43072462)(803.32473861,40.42572462)(803.29473633,40.41572561)
\lineto(803.14473633,40.41572561)
\curveto(803.10473883,40.40572464)(803.05973887,40.40072465)(803.00973633,40.40072561)
\curveto(802.94973898,40.41072464)(802.89473904,40.41572463)(802.84473633,40.41572561)
\lineto(802.24473633,40.41572561)
\lineto(799.48473633,40.41572561)
\lineto(798.52473633,40.41572561)
\lineto(798.25473633,40.41572561)
\curveto(798.16474377,40.41572463)(798.08974384,40.43572461)(798.02973633,40.47572561)
\curveto(797.95974397,40.51572453)(797.90974402,40.59072446)(797.87973633,40.70072561)
\curveto(797.86974406,40.72072433)(797.86974406,40.74072431)(797.87973633,40.76072561)
\curveto(797.87974405,40.78072427)(797.87474406,40.80072425)(797.86473633,40.82072561)
}
}
{
\newrgbcolor{curcolor}{0 0 0}
\pscustom[linestyle=none,fillstyle=solid,fillcolor=curcolor]
{
\newpath
\moveto(797.71473633,52.39533498)
\curveto(797.69474424,53.02532975)(797.77974415,53.53032924)(797.96973633,53.91033498)
\curveto(798.15974377,54.29032848)(798.44474349,54.59532818)(798.82473633,54.82533498)
\curveto(798.92474301,54.88532789)(799.0347429,54.93032784)(799.15473633,54.96033498)
\curveto(799.26474267,55.00032777)(799.37974255,55.03532774)(799.49973633,55.06533498)
\curveto(799.68974224,55.11532766)(799.89474204,55.14532763)(800.11473633,55.15533498)
\curveto(800.3347416,55.16532761)(800.55974137,55.1703276)(800.78973633,55.17033498)
\lineto(802.39473633,55.17033498)
\lineto(804.73473633,55.17033498)
\curveto(804.90473703,55.1703276)(805.07473686,55.16532761)(805.24473633,55.15533498)
\curveto(805.41473652,55.15532762)(805.52473641,55.09032768)(805.57473633,54.96033498)
\curveto(805.59473634,54.91032786)(805.60473633,54.85532792)(805.60473633,54.79533498)
\curveto(805.61473632,54.74532803)(805.61973631,54.69032808)(805.61973633,54.63033498)
\curveto(805.61973631,54.50032827)(805.61473632,54.3753284)(805.60473633,54.25533498)
\curveto(805.60473633,54.13532864)(805.56473637,54.05032872)(805.48473633,54.00033498)
\curveto(805.41473652,53.95032882)(805.32473661,53.92532885)(805.21473633,53.92533498)
\lineto(804.88473633,53.92533498)
\lineto(803.59473633,53.92533498)
\lineto(801.14973633,53.92533498)
\curveto(800.87974105,53.92532885)(800.61474132,53.92032885)(800.35473633,53.91033498)
\curveto(800.08474185,53.90032887)(799.85474208,53.85532892)(799.66473633,53.77533498)
\curveto(799.46474247,53.69532908)(799.30474263,53.5753292)(799.18473633,53.41533498)
\curveto(799.05474288,53.25532952)(798.95474298,53.0703297)(798.88473633,52.86033498)
\curveto(798.86474307,52.80032997)(798.85474308,52.73533004)(798.85473633,52.66533498)
\curveto(798.84474309,52.60533017)(798.8297431,52.54533023)(798.80973633,52.48533498)
\curveto(798.79974313,52.43533034)(798.79974313,52.35533042)(798.80973633,52.24533498)
\curveto(798.80974312,52.14533063)(798.81474312,52.0753307)(798.82473633,52.03533498)
\curveto(798.84474309,51.99533078)(798.85474308,51.96033081)(798.85473633,51.93033498)
\curveto(798.84474309,51.90033087)(798.84474309,51.86533091)(798.85473633,51.82533498)
\curveto(798.88474305,51.69533108)(798.91974301,51.5703312)(798.95973633,51.45033498)
\curveto(798.98974294,51.34033143)(799.0347429,51.23533154)(799.09473633,51.13533498)
\curveto(799.11474282,51.09533168)(799.1347428,51.06033171)(799.15473633,51.03033498)
\curveto(799.17474276,51.00033177)(799.19474274,50.96533181)(799.21473633,50.92533498)
\curveto(799.46474247,50.5753322)(799.83974209,50.32033245)(800.33973633,50.16033498)
\curveto(800.41974151,50.13033264)(800.50474143,50.11033266)(800.59473633,50.10033498)
\curveto(800.67474126,50.09033268)(800.75474118,50.0753327)(800.83473633,50.05533498)
\curveto(800.88474105,50.03533274)(800.934741,50.03033274)(800.98473633,50.04033498)
\curveto(801.02474091,50.05033272)(801.06474087,50.04533273)(801.10473633,50.02533498)
\lineto(801.41973633,50.02533498)
\curveto(801.44974048,50.01533276)(801.48474045,50.01033276)(801.52473633,50.01033498)
\curveto(801.56474037,50.02033275)(801.60974032,50.02533275)(801.65973633,50.02533498)
\lineto(802.10973633,50.02533498)
\lineto(803.54973633,50.02533498)
\lineto(804.86973633,50.02533498)
\lineto(805.21473633,50.02533498)
\curveto(805.32473661,50.02533275)(805.41473652,50.00033277)(805.48473633,49.95033498)
\curveto(805.56473637,49.90033287)(805.60473633,49.81033296)(805.60473633,49.68033498)
\curveto(805.61473632,49.56033321)(805.61973631,49.43533334)(805.61973633,49.30533498)
\curveto(805.61973631,49.22533355)(805.61473632,49.15033362)(805.60473633,49.08033498)
\curveto(805.59473634,49.01033376)(805.56973636,48.95033382)(805.52973633,48.90033498)
\curveto(805.47973645,48.82033395)(805.38473655,48.78033399)(805.24473633,48.78033498)
\lineto(804.83973633,48.78033498)
\lineto(803.06973633,48.78033498)
\lineto(799.43973633,48.78033498)
\lineto(798.52473633,48.78033498)
\lineto(798.25473633,48.78033498)
\curveto(798.16474377,48.78033399)(798.09474384,48.80033397)(798.04473633,48.84033498)
\curveto(797.98474395,48.8703339)(797.94474399,48.92033385)(797.92473633,48.99033498)
\curveto(797.91474402,49.03033374)(797.90474403,49.08533369)(797.89473633,49.15533498)
\curveto(797.88474405,49.23533354)(797.87974405,49.31533346)(797.87973633,49.39533498)
\curveto(797.87974405,49.4753333)(797.88474405,49.55033322)(797.89473633,49.62033498)
\curveto(797.90474403,49.70033307)(797.91974401,49.75533302)(797.93973633,49.78533498)
\curveto(798.00974392,49.89533288)(798.09974383,49.94533283)(798.20973633,49.93533498)
\curveto(798.30974362,49.92533285)(798.42474351,49.94033283)(798.55473633,49.98033498)
\curveto(798.61474332,50.00033277)(798.66474327,50.04033273)(798.70473633,50.10033498)
\curveto(798.71474322,50.22033255)(798.66974326,50.31533246)(798.56973633,50.38533498)
\curveto(798.46974346,50.46533231)(798.38974354,50.54533223)(798.32973633,50.62533498)
\curveto(798.2297437,50.76533201)(798.13974379,50.90533187)(798.05973633,51.04533498)
\curveto(797.96974396,51.19533158)(797.89474404,51.36533141)(797.83473633,51.55533498)
\curveto(797.80474413,51.63533114)(797.78474415,51.72033105)(797.77473633,51.81033498)
\curveto(797.76474417,51.91033086)(797.74974418,52.00533077)(797.72973633,52.09533498)
\curveto(797.71974421,52.14533063)(797.71474422,52.19533058)(797.71473633,52.24533498)
\lineto(797.71473633,52.39533498)
}
}
{
\newrgbcolor{curcolor}{0 0 0}
\pscustom[linestyle=none,fillstyle=solid,fillcolor=curcolor]
{
}
}
{
\newrgbcolor{curcolor}{0 0 0}
\pscustom[linestyle=none,fillstyle=solid,fillcolor=curcolor]
{
\newpath
\moveto(794.98473633,64.11010061)
\curveto(794.95474698,65.74009517)(795.50974642,66.79009412)(796.64973633,67.26010061)
\curveto(796.87974505,67.36009355)(797.16974476,67.42509348)(797.51973633,67.45510061)
\curveto(797.85974407,67.49509341)(798.16974376,67.47009344)(798.44973633,67.38010061)
\curveto(798.70974322,67.29009362)(798.934743,67.17009374)(799.12473633,67.02010061)
\curveto(799.16474277,67.00009391)(799.19974273,66.97509393)(799.22973633,66.94510061)
\curveto(799.24974268,66.91509399)(799.27474266,66.89009402)(799.30473633,66.87010061)
\lineto(799.42473633,66.78010061)
\curveto(799.45474248,66.75009416)(799.47974245,66.71509419)(799.49973633,66.67510061)
\curveto(799.54974238,66.62509428)(799.59474234,66.57009434)(799.63473633,66.51010061)
\curveto(799.67474226,66.46009445)(799.72474221,66.41509449)(799.78473633,66.37510061)
\curveto(799.82474211,66.33509457)(799.87474206,66.32009459)(799.93473633,66.33010061)
\curveto(799.98474195,66.34009457)(800.0297419,66.37009454)(800.06973633,66.42010061)
\curveto(800.10974182,66.47009444)(800.14974178,66.52509438)(800.18973633,66.58510061)
\curveto(800.21974171,66.65509425)(800.24974168,66.72009419)(800.27973633,66.78010061)
\curveto(800.30974162,66.84009407)(800.33974159,66.89009402)(800.36973633,66.93010061)
\curveto(800.58974134,67.25009366)(800.89974103,67.5050934)(801.29973633,67.69510061)
\curveto(801.38974054,67.73509317)(801.48474045,67.76509314)(801.58473633,67.78510061)
\curveto(801.67474026,67.81509309)(801.76474017,67.84009307)(801.85473633,67.86010061)
\curveto(801.90474003,67.87009304)(801.95473998,67.87509303)(802.00473633,67.87510061)
\curveto(802.04473989,67.88509302)(802.08973984,67.89509301)(802.13973633,67.90510061)
\curveto(802.18973974,67.91509299)(802.23973969,67.91509299)(802.28973633,67.90510061)
\curveto(802.33973959,67.89509301)(802.38973954,67.90009301)(802.43973633,67.92010061)
\curveto(802.48973944,67.93009298)(802.54973938,67.93509297)(802.61973633,67.93510061)
\curveto(802.68973924,67.93509297)(802.74973918,67.92509298)(802.79973633,67.90510061)
\lineto(803.02473633,67.90510061)
\lineto(803.26473633,67.84510061)
\curveto(803.3347386,67.83509307)(803.40473853,67.82009309)(803.47473633,67.80010061)
\curveto(803.56473837,67.77009314)(803.64973828,67.74009317)(803.72973633,67.71010061)
\curveto(803.80973812,67.69009322)(803.88973804,67.66009325)(803.96973633,67.62010061)
\curveto(804.0297379,67.60009331)(804.08973784,67.57009334)(804.14973633,67.53010061)
\curveto(804.19973773,67.50009341)(804.24973768,67.46509344)(804.29973633,67.42510061)
\curveto(804.60973732,67.22509368)(804.86973706,66.97509393)(805.07973633,66.67510061)
\curveto(805.27973665,66.37509453)(805.44473649,66.03009488)(805.57473633,65.64010061)
\curveto(805.61473632,65.52009539)(805.63973629,65.39009552)(805.64973633,65.25010061)
\curveto(805.66973626,65.12009579)(805.69473624,64.98509592)(805.72473633,64.84510061)
\curveto(805.7347362,64.77509613)(805.73973619,64.7050962)(805.73973633,64.63510061)
\curveto(805.73973619,64.57509633)(805.74473619,64.5100964)(805.75473633,64.44010061)
\curveto(805.76473617,64.40009651)(805.76973616,64.34009657)(805.76973633,64.26010061)
\curveto(805.76973616,64.19009672)(805.76473617,64.14009677)(805.75473633,64.11010061)
\curveto(805.74473619,64.06009685)(805.73973619,64.01509689)(805.73973633,63.97510061)
\lineto(805.73973633,63.85510061)
\curveto(805.71973621,63.75509715)(805.70473623,63.65509725)(805.69473633,63.55510061)
\curveto(805.68473625,63.45509745)(805.66973626,63.36009755)(805.64973633,63.27010061)
\curveto(805.61973631,63.16009775)(805.59473634,63.05009786)(805.57473633,62.94010061)
\curveto(805.54473639,62.84009807)(805.50473643,62.73509817)(805.45473633,62.62510061)
\curveto(805.29473664,62.25509865)(805.09473684,61.94009897)(804.85473633,61.68010061)
\curveto(804.60473733,61.42009949)(804.29473764,61.2100997)(803.92473633,61.05010061)
\curveto(803.8347381,61.0100999)(803.73973819,60.97509993)(803.63973633,60.94510061)
\curveto(803.53973839,60.91509999)(803.4347385,60.88510002)(803.32473633,60.85510061)
\curveto(803.27473866,60.83510007)(803.22473871,60.82510008)(803.17473633,60.82510061)
\curveto(803.11473882,60.82510008)(803.05473888,60.81510009)(802.99473633,60.79510061)
\curveto(802.934739,60.77510013)(802.85473908,60.76510014)(802.75473633,60.76510061)
\curveto(802.65473928,60.76510014)(802.57973935,60.78010013)(802.52973633,60.81010061)
\curveto(802.49973943,60.82010009)(802.47473946,60.83510007)(802.45473633,60.85510061)
\lineto(802.39473633,60.91510061)
\curveto(802.37473956,60.95509995)(802.35973957,61.01509989)(802.34973633,61.09510061)
\curveto(802.33973959,61.18509972)(802.3347396,61.27509963)(802.33473633,61.36510061)
\curveto(802.3347396,61.45509945)(802.33973959,61.54009937)(802.34973633,61.62010061)
\curveto(802.35973957,61.7100992)(802.36973956,61.77509913)(802.37973633,61.81510061)
\curveto(802.39973953,61.83509907)(802.41473952,61.85509905)(802.42473633,61.87510061)
\curveto(802.42473951,61.89509901)(802.4347395,61.91509899)(802.45473633,61.93510061)
\curveto(802.54473939,62.0050989)(802.65973927,62.04509886)(802.79973633,62.05510061)
\curveto(802.93973899,62.07509883)(803.06473887,62.1050988)(803.17473633,62.14510061)
\lineto(803.53473633,62.29510061)
\curveto(803.64473829,62.34509856)(803.74973818,62.4100985)(803.84973633,62.49010061)
\curveto(803.87973805,62.5100984)(803.90473803,62.53009838)(803.92473633,62.55010061)
\curveto(803.94473799,62.58009833)(803.96973796,62.6050983)(803.99973633,62.62510061)
\curveto(804.05973787,62.66509824)(804.10473783,62.70009821)(804.13473633,62.73010061)
\curveto(804.16473777,62.77009814)(804.19473774,62.8050981)(804.22473633,62.83510061)
\curveto(804.25473768,62.87509803)(804.28473765,62.92009799)(804.31473633,62.97010061)
\curveto(804.37473756,63.06009785)(804.42473751,63.15509775)(804.46473633,63.25510061)
\lineto(804.58473633,63.58510061)
\curveto(804.6347373,63.73509717)(804.66473727,63.93509697)(804.67473633,64.18510061)
\curveto(804.68473725,64.43509647)(804.66473727,64.64509626)(804.61473633,64.81510061)
\curveto(804.59473734,64.89509601)(804.57973735,64.96509594)(804.56973633,65.02510061)
\lineto(804.50973633,65.23510061)
\curveto(804.38973754,65.51509539)(804.23973769,65.75509515)(804.05973633,65.95510061)
\curveto(803.87973805,66.16509474)(803.64973828,66.33009458)(803.36973633,66.45010061)
\curveto(803.29973863,66.48009443)(803.2297387,66.50009441)(803.15973633,66.51010061)
\lineto(802.91973633,66.57010061)
\curveto(802.77973915,66.6100943)(802.61973931,66.62009429)(802.43973633,66.60010061)
\curveto(802.24973968,66.58009433)(802.09973983,66.55009436)(801.98973633,66.51010061)
\curveto(801.60974032,66.38009453)(801.31974061,66.19509471)(801.11973633,65.95510061)
\curveto(800.91974101,65.72509518)(800.75974117,65.41509549)(800.63973633,65.02510061)
\curveto(800.60974132,64.91509599)(800.58974134,64.79509611)(800.57973633,64.66510061)
\curveto(800.56974136,64.54509636)(800.56474137,64.42009649)(800.56473633,64.29010061)
\curveto(800.56474137,64.13009678)(800.55974137,63.99009692)(800.54973633,63.87010061)
\curveto(800.53974139,63.75009716)(800.47974145,63.66509724)(800.36973633,63.61510061)
\curveto(800.33974159,63.59509731)(800.30474163,63.58509732)(800.26473633,63.58510061)
\lineto(800.12973633,63.58510061)
\curveto(800.0297419,63.57509733)(799.934742,63.57509733)(799.84473633,63.58510061)
\curveto(799.75474218,63.6050973)(799.68974224,63.64509726)(799.64973633,63.70510061)
\curveto(799.61974231,63.74509716)(799.59974233,63.78509712)(799.58973633,63.82510061)
\curveto(799.57974235,63.87509703)(799.56974236,63.93009698)(799.55973633,63.99010061)
\curveto(799.54974238,64.0100969)(799.54974238,64.03509687)(799.55973633,64.06510061)
\curveto(799.55974237,64.09509681)(799.55474238,64.12009679)(799.54473633,64.14010061)
\lineto(799.54473633,64.27510061)
\curveto(799.52474241,64.38509652)(799.51474242,64.48509642)(799.51473633,64.57510061)
\curveto(799.50474243,64.67509623)(799.48474245,64.77009614)(799.45473633,64.86010061)
\curveto(799.34474259,65.18009573)(799.19974273,65.43509547)(799.01973633,65.62510061)
\curveto(798.83974309,65.81509509)(798.58974334,65.96509494)(798.26973633,66.07510061)
\curveto(798.16974376,66.1050948)(798.04474389,66.12509478)(797.89473633,66.13510061)
\curveto(797.7347442,66.15509475)(797.58974434,66.15009476)(797.45973633,66.12010061)
\curveto(797.38974454,66.10009481)(797.32474461,66.08009483)(797.26473633,66.06010061)
\curveto(797.19474474,66.05009486)(797.1297448,66.03009488)(797.06973633,66.00010061)
\curveto(796.8297451,65.90009501)(796.63974529,65.75509515)(796.49973633,65.56510061)
\curveto(796.35974557,65.37509553)(796.24974568,65.15009576)(796.16973633,64.89010061)
\curveto(796.14974578,64.83009608)(796.13974579,64.77009614)(796.13973633,64.71010061)
\curveto(796.13974579,64.65009626)(796.1297458,64.58509632)(796.10973633,64.51510061)
\curveto(796.08974584,64.43509647)(796.07974585,64.34009657)(796.07973633,64.23010061)
\curveto(796.07974585,64.12009679)(796.08974584,64.02509688)(796.10973633,63.94510061)
\curveto(796.1297458,63.89509701)(796.13974579,63.84509706)(796.13973633,63.79510061)
\curveto(796.13974579,63.75509715)(796.14974578,63.7100972)(796.16973633,63.66010061)
\curveto(796.21974571,63.48009743)(796.29474564,63.3100976)(796.39473633,63.15010061)
\curveto(796.48474545,63.00009791)(796.59974533,62.87009804)(796.73973633,62.76010061)
\curveto(796.85974507,62.67009824)(796.98974494,62.59009832)(797.12973633,62.52010061)
\curveto(797.26974466,62.45009846)(797.42474451,62.38509852)(797.59473633,62.32510061)
\curveto(797.70474423,62.29509861)(797.82474411,62.27509863)(797.95473633,62.26510061)
\curveto(798.07474386,62.25509865)(798.17474376,62.22009869)(798.25473633,62.16010061)
\curveto(798.29474364,62.14009877)(798.3347436,62.08009883)(798.37473633,61.98010061)
\curveto(798.38474355,61.94009897)(798.39474354,61.88009903)(798.40473633,61.80010061)
\lineto(798.40473633,61.54510061)
\curveto(798.39474354,61.45509945)(798.38474355,61.37009954)(798.37473633,61.29010061)
\curveto(798.36474357,61.22009969)(798.34974358,61.17009974)(798.32973633,61.14010061)
\curveto(798.29974363,61.10009981)(798.24474369,61.06509984)(798.16473633,61.03510061)
\curveto(798.08474385,61.0050999)(797.99974393,61.00009991)(797.90973633,61.02010061)
\curveto(797.85974407,61.03009988)(797.80974412,61.03509987)(797.75973633,61.03510061)
\lineto(797.57973633,61.06510061)
\curveto(797.47974445,61.09509981)(797.37974455,61.12009979)(797.27973633,61.14010061)
\curveto(797.17974475,61.17009974)(797.08974484,61.2050997)(797.00973633,61.24510061)
\curveto(796.89974503,61.29509961)(796.79474514,61.34009957)(796.69473633,61.38010061)
\curveto(796.58474535,61.42009949)(796.47974545,61.47009944)(796.37973633,61.53010061)
\curveto(795.83974609,61.86009905)(795.44474649,62.33009858)(795.19473633,62.94010061)
\curveto(795.14474679,63.06009785)(795.10974682,63.18509772)(795.08973633,63.31510061)
\curveto(795.06974686,63.45509745)(795.04474689,63.59509731)(795.01473633,63.73510061)
\curveto(795.00474693,63.79509711)(794.99974693,63.85509705)(794.99973633,63.91510061)
\curveto(794.99974693,63.98509692)(794.99474694,64.05009686)(794.98473633,64.11010061)
}
}
{
\newrgbcolor{curcolor}{0 0 0}
\pscustom[linestyle=none,fillstyle=solid,fillcolor=curcolor]
{
\newpath
\moveto(800.50473633,76.34470998)
\lineto(800.75973633,76.34470998)
\curveto(800.83974109,76.35470228)(800.91474102,76.34970228)(800.98473633,76.32970998)
\lineto(801.22473633,76.32970998)
\lineto(801.38973633,76.32970998)
\curveto(801.48974044,76.30970232)(801.59474034,76.29970233)(801.70473633,76.29970998)
\curveto(801.80474013,76.29970233)(801.90474003,76.28970234)(802.00473633,76.26970998)
\lineto(802.15473633,76.26970998)
\curveto(802.29473964,76.23970239)(802.4347395,76.21970241)(802.57473633,76.20970998)
\curveto(802.70473923,76.19970243)(802.8347391,76.17470246)(802.96473633,76.13470998)
\curveto(803.04473889,76.11470252)(803.1297388,76.09470254)(803.21973633,76.07470998)
\lineto(803.45973633,76.01470998)
\lineto(803.75973633,75.89470998)
\curveto(803.84973808,75.86470277)(803.93973799,75.8297028)(804.02973633,75.78970998)
\curveto(804.24973768,75.68970294)(804.46473747,75.55470308)(804.67473633,75.38470998)
\curveto(804.88473705,75.22470341)(805.05473688,75.04970358)(805.18473633,74.85970998)
\curveto(805.22473671,74.80970382)(805.26473667,74.74970388)(805.30473633,74.67970998)
\curveto(805.3347366,74.61970401)(805.36973656,74.55970407)(805.40973633,74.49970998)
\curveto(805.45973647,74.41970421)(805.49973643,74.32470431)(805.52973633,74.21470998)
\curveto(805.55973637,74.10470453)(805.58973634,73.99970463)(805.61973633,73.89970998)
\curveto(805.65973627,73.78970484)(805.68473625,73.67970495)(805.69473633,73.56970998)
\curveto(805.70473623,73.45970517)(805.71973621,73.34470529)(805.73973633,73.22470998)
\curveto(805.74973618,73.18470545)(805.74973618,73.13970549)(805.73973633,73.08970998)
\curveto(805.73973619,73.04970558)(805.74473619,73.00970562)(805.75473633,72.96970998)
\curveto(805.76473617,72.9297057)(805.76973616,72.87470576)(805.76973633,72.80470998)
\curveto(805.76973616,72.7347059)(805.76473617,72.68470595)(805.75473633,72.65470998)
\curveto(805.7347362,72.60470603)(805.7297362,72.55970607)(805.73973633,72.51970998)
\curveto(805.74973618,72.47970615)(805.74973618,72.44470619)(805.73973633,72.41470998)
\lineto(805.73973633,72.32470998)
\curveto(805.71973621,72.26470637)(805.70473623,72.19970643)(805.69473633,72.12970998)
\curveto(805.69473624,72.06970656)(805.68973624,72.00470663)(805.67973633,71.93470998)
\curveto(805.6297363,71.76470687)(805.57973635,71.60470703)(805.52973633,71.45470998)
\curveto(805.47973645,71.30470733)(805.41473652,71.15970747)(805.33473633,71.01970998)
\curveto(805.29473664,70.96970766)(805.26473667,70.91470772)(805.24473633,70.85470998)
\curveto(805.21473672,70.80470783)(805.17973675,70.75470788)(805.13973633,70.70470998)
\curveto(804.95973697,70.46470817)(804.73973719,70.26470837)(804.47973633,70.10470998)
\curveto(804.21973771,69.94470869)(803.934738,69.80470883)(803.62473633,69.68470998)
\curveto(803.48473845,69.62470901)(803.34473859,69.57970905)(803.20473633,69.54970998)
\curveto(803.05473888,69.51970911)(802.89973903,69.48470915)(802.73973633,69.44470998)
\curveto(802.6297393,69.42470921)(802.51973941,69.40970922)(802.40973633,69.39970998)
\curveto(802.29973963,69.38970924)(802.18973974,69.37470926)(802.07973633,69.35470998)
\curveto(802.03973989,69.34470929)(801.99973993,69.33970929)(801.95973633,69.33970998)
\curveto(801.91974001,69.34970928)(801.87974005,69.34970928)(801.83973633,69.33970998)
\curveto(801.78974014,69.3297093)(801.73974019,69.32470931)(801.68973633,69.32470998)
\lineto(801.52473633,69.32470998)
\curveto(801.47474046,69.30470933)(801.42474051,69.29970933)(801.37473633,69.30970998)
\curveto(801.31474062,69.31970931)(801.25974067,69.31970931)(801.20973633,69.30970998)
\curveto(801.16974076,69.29970933)(801.12474081,69.29970933)(801.07473633,69.30970998)
\curveto(801.02474091,69.31970931)(800.97474096,69.31470932)(800.92473633,69.29470998)
\curveto(800.85474108,69.27470936)(800.77974115,69.26970936)(800.69973633,69.27970998)
\curveto(800.60974132,69.28970934)(800.52474141,69.29470934)(800.44473633,69.29470998)
\curveto(800.35474158,69.29470934)(800.25474168,69.28970934)(800.14473633,69.27970998)
\curveto(800.02474191,69.26970936)(799.92474201,69.27470936)(799.84473633,69.29470998)
\lineto(799.55973633,69.29470998)
\lineto(798.92973633,69.33970998)
\curveto(798.8297431,69.34970928)(798.7347432,69.35970927)(798.64473633,69.36970998)
\lineto(798.34473633,69.39970998)
\curveto(798.29474364,69.41970921)(798.24474369,69.42470921)(798.19473633,69.41470998)
\curveto(798.1347438,69.41470922)(798.07974385,69.42470921)(798.02973633,69.44470998)
\curveto(797.85974407,69.49470914)(797.69474424,69.5347091)(797.53473633,69.56470998)
\curveto(797.36474457,69.59470904)(797.20474473,69.64470899)(797.05473633,69.71470998)
\curveto(796.59474534,69.90470873)(796.21974571,70.12470851)(795.92973633,70.37470998)
\curveto(795.63974629,70.634708)(795.39474654,70.99470764)(795.19473633,71.45470998)
\curveto(795.14474679,71.58470705)(795.10974682,71.71470692)(795.08973633,71.84470998)
\curveto(795.06974686,71.98470665)(795.04474689,72.12470651)(795.01473633,72.26470998)
\curveto(795.00474693,72.3347063)(794.99974693,72.39970623)(794.99973633,72.45970998)
\curveto(794.99974693,72.51970611)(794.99474694,72.58470605)(794.98473633,72.65470998)
\curveto(794.96474697,73.48470515)(795.11474682,74.15470448)(795.43473633,74.66470998)
\curveto(795.74474619,75.17470346)(796.18474575,75.55470308)(796.75473633,75.80470998)
\curveto(796.87474506,75.85470278)(796.99974493,75.89970273)(797.12973633,75.93970998)
\curveto(797.25974467,75.97970265)(797.39474454,76.02470261)(797.53473633,76.07470998)
\curveto(797.61474432,76.09470254)(797.69974423,76.10970252)(797.78973633,76.11970998)
\lineto(798.02973633,76.17970998)
\curveto(798.13974379,76.20970242)(798.24974368,76.22470241)(798.35973633,76.22470998)
\curveto(798.46974346,76.2347024)(798.57974335,76.24970238)(798.68973633,76.26970998)
\curveto(798.73974319,76.28970234)(798.78474315,76.29470234)(798.82473633,76.28470998)
\curveto(798.86474307,76.28470235)(798.90474303,76.28970234)(798.94473633,76.29970998)
\curveto(798.99474294,76.30970232)(799.04974288,76.30970232)(799.10973633,76.29970998)
\curveto(799.15974277,76.29970233)(799.20974272,76.30470233)(799.25973633,76.31470998)
\lineto(799.39473633,76.31470998)
\curveto(799.45474248,76.3347023)(799.52474241,76.3347023)(799.60473633,76.31470998)
\curveto(799.67474226,76.30470233)(799.73974219,76.30970232)(799.79973633,76.32970998)
\curveto(799.8297421,76.33970229)(799.86974206,76.34470229)(799.91973633,76.34470998)
\lineto(800.03973633,76.34470998)
\lineto(800.50473633,76.34470998)
\moveto(802.82973633,74.79970998)
\curveto(802.50973942,74.89970373)(802.14473979,74.95970367)(801.73473633,74.97970998)
\curveto(801.32474061,74.99970363)(800.91474102,75.00970362)(800.50473633,75.00970998)
\curveto(800.07474186,75.00970362)(799.65474228,74.99970363)(799.24473633,74.97970998)
\curveto(798.8347431,74.95970367)(798.44974348,74.91470372)(798.08973633,74.84470998)
\curveto(797.7297442,74.77470386)(797.40974452,74.66470397)(797.12973633,74.51470998)
\curveto(796.83974509,74.37470426)(796.60474533,74.17970445)(796.42473633,73.92970998)
\curveto(796.31474562,73.76970486)(796.2347457,73.58970504)(796.18473633,73.38970998)
\curveto(796.12474581,73.18970544)(796.09474584,72.94470569)(796.09473633,72.65470998)
\curveto(796.11474582,72.634706)(796.12474581,72.59970603)(796.12473633,72.54970998)
\curveto(796.11474582,72.49970613)(796.11474582,72.45970617)(796.12473633,72.42970998)
\curveto(796.14474579,72.34970628)(796.16474577,72.27470636)(796.18473633,72.20470998)
\curveto(796.19474574,72.14470649)(796.21474572,72.07970655)(796.24473633,72.00970998)
\curveto(796.36474557,71.73970689)(796.5347454,71.51970711)(796.75473633,71.34970998)
\curveto(796.96474497,71.18970744)(797.20974472,71.05470758)(797.48973633,70.94470998)
\curveto(797.59974433,70.89470774)(797.71974421,70.85470778)(797.84973633,70.82470998)
\curveto(797.96974396,70.80470783)(798.09474384,70.77970785)(798.22473633,70.74970998)
\curveto(798.27474366,70.7297079)(798.3297436,70.71970791)(798.38973633,70.71970998)
\curveto(798.43974349,70.71970791)(798.48974344,70.71470792)(798.53973633,70.70470998)
\curveto(798.6297433,70.69470794)(798.72474321,70.68470795)(798.82473633,70.67470998)
\curveto(798.91474302,70.66470797)(799.00974292,70.65470798)(799.10973633,70.64470998)
\curveto(799.18974274,70.64470799)(799.27474266,70.63970799)(799.36473633,70.62970998)
\lineto(799.60473633,70.62970998)
\lineto(799.78473633,70.62970998)
\curveto(799.81474212,70.61970801)(799.84974208,70.61470802)(799.88973633,70.61470998)
\lineto(800.02473633,70.61470998)
\lineto(800.47473633,70.61470998)
\curveto(800.55474138,70.61470802)(800.63974129,70.60970802)(800.72973633,70.59970998)
\curveto(800.80974112,70.59970803)(800.88474105,70.60970802)(800.95473633,70.62970998)
\lineto(801.22473633,70.62970998)
\curveto(801.24474069,70.629708)(801.27474066,70.62470801)(801.31473633,70.61470998)
\curveto(801.34474059,70.61470802)(801.36974056,70.61970801)(801.38973633,70.62970998)
\curveto(801.48974044,70.63970799)(801.58974034,70.64470799)(801.68973633,70.64470998)
\curveto(801.77974015,70.65470798)(801.87974005,70.66470797)(801.98973633,70.67470998)
\curveto(802.10973982,70.70470793)(802.2347397,70.71970791)(802.36473633,70.71970998)
\curveto(802.48473945,70.7297079)(802.59973933,70.75470788)(802.70973633,70.79470998)
\curveto(803.00973892,70.87470776)(803.27473866,70.95970767)(803.50473633,71.04970998)
\curveto(803.7347382,71.14970748)(803.94973798,71.29470734)(804.14973633,71.48470998)
\curveto(804.34973758,71.69470694)(804.49973743,71.95970667)(804.59973633,72.27970998)
\curveto(804.61973731,72.31970631)(804.6297373,72.35470628)(804.62973633,72.38470998)
\curveto(804.61973731,72.42470621)(804.62473731,72.46970616)(804.64473633,72.51970998)
\curveto(804.65473728,72.55970607)(804.66473727,72.629706)(804.67473633,72.72970998)
\curveto(804.68473725,72.83970579)(804.67973725,72.92470571)(804.65973633,72.98470998)
\curveto(804.63973729,73.05470558)(804.6297373,73.12470551)(804.62973633,73.19470998)
\curveto(804.61973731,73.26470537)(804.60473733,73.3297053)(804.58473633,73.38970998)
\curveto(804.52473741,73.58970504)(804.43973749,73.76970486)(804.32973633,73.92970998)
\curveto(804.30973762,73.95970467)(804.28973764,73.98470465)(804.26973633,74.00470998)
\lineto(804.20973633,74.06470998)
\curveto(804.18973774,74.10470453)(804.14973778,74.15470448)(804.08973633,74.21470998)
\curveto(803.94973798,74.31470432)(803.81973811,74.39970423)(803.69973633,74.46970998)
\curveto(803.57973835,74.53970409)(803.4347385,74.60970402)(803.26473633,74.67970998)
\curveto(803.19473874,74.70970392)(803.12473881,74.7297039)(803.05473633,74.73970998)
\curveto(802.98473895,74.75970387)(802.90973902,74.77970385)(802.82973633,74.79970998)
}
}
{
\newrgbcolor{curcolor}{0 0 0}
\pscustom[linestyle=none,fillstyle=solid,fillcolor=curcolor]
{
\newpath
\moveto(803.95473633,78.63431936)
\lineto(803.95473633,79.26431936)
\lineto(803.95473633,79.45931936)
\curveto(803.95473798,79.52931683)(803.96473797,79.58931677)(803.98473633,79.63931936)
\curveto(804.02473791,79.70931665)(804.06473787,79.7593166)(804.10473633,79.78931936)
\curveto(804.15473778,79.82931653)(804.21973771,79.84931651)(804.29973633,79.84931936)
\curveto(804.37973755,79.8593165)(804.46473747,79.86431649)(804.55473633,79.86431936)
\lineto(805.27473633,79.86431936)
\curveto(805.75473618,79.86431649)(806.16473577,79.80431655)(806.50473633,79.68431936)
\curveto(806.84473509,79.56431679)(807.11973481,79.36931699)(807.32973633,79.09931936)
\curveto(807.37973455,79.02931733)(807.42473451,78.9593174)(807.46473633,78.88931936)
\curveto(807.51473442,78.82931753)(807.55973437,78.7543176)(807.59973633,78.66431936)
\curveto(807.60973432,78.64431771)(807.61973431,78.61431774)(807.62973633,78.57431936)
\curveto(807.64973428,78.53431782)(807.65473428,78.48931787)(807.64473633,78.43931936)
\curveto(807.61473432,78.34931801)(807.53973439,78.29431806)(807.41973633,78.27431936)
\curveto(807.30973462,78.2543181)(807.21473472,78.26931809)(807.13473633,78.31931936)
\curveto(807.06473487,78.34931801)(806.99973493,78.39431796)(806.93973633,78.45431936)
\curveto(806.88973504,78.52431783)(806.83973509,78.58931777)(806.78973633,78.64931936)
\curveto(806.73973519,78.71931764)(806.66473527,78.77931758)(806.56473633,78.82931936)
\curveto(806.47473546,78.88931747)(806.38473555,78.93931742)(806.29473633,78.97931936)
\curveto(806.26473567,78.99931736)(806.20473573,79.02431733)(806.11473633,79.05431936)
\curveto(806.0347359,79.08431727)(805.96473597,79.08931727)(805.90473633,79.06931936)
\curveto(805.76473617,79.03931732)(805.67473626,78.97931738)(805.63473633,78.88931936)
\curveto(805.60473633,78.80931755)(805.58973634,78.71931764)(805.58973633,78.61931936)
\curveto(805.58973634,78.51931784)(805.56473637,78.43431792)(805.51473633,78.36431936)
\curveto(805.44473649,78.27431808)(805.30473663,78.22931813)(805.09473633,78.22931936)
\lineto(804.53973633,78.22931936)
\lineto(804.31473633,78.22931936)
\curveto(804.2347377,78.23931812)(804.16973776,78.2593181)(804.11973633,78.28931936)
\curveto(804.03973789,78.34931801)(803.99473794,78.41931794)(803.98473633,78.49931936)
\curveto(803.97473796,78.51931784)(803.96973796,78.53931782)(803.96973633,78.55931936)
\curveto(803.96973796,78.58931777)(803.96473797,78.61431774)(803.95473633,78.63431936)
}
}
{
\newrgbcolor{curcolor}{0 0 0}
\pscustom[linestyle=none,fillstyle=solid,fillcolor=curcolor]
{
}
}
{
\newrgbcolor{curcolor}{0 0 0}
\pscustom[linestyle=none,fillstyle=solid,fillcolor=curcolor]
{
\newpath
\moveto(794.98473633,89.26463186)
\curveto(794.97474696,89.95462722)(795.09474684,90.55462662)(795.34473633,91.06463186)
\curveto(795.59474634,91.58462559)(795.929746,91.9796252)(796.34973633,92.24963186)
\curveto(796.4297455,92.29962488)(796.51974541,92.34462483)(796.61973633,92.38463186)
\curveto(796.70974522,92.42462475)(796.80474513,92.46962471)(796.90473633,92.51963186)
\curveto(797.00474493,92.55962462)(797.10474483,92.58962459)(797.20473633,92.60963186)
\curveto(797.30474463,92.62962455)(797.40974452,92.64962453)(797.51973633,92.66963186)
\curveto(797.56974436,92.68962449)(797.61474432,92.69462448)(797.65473633,92.68463186)
\curveto(797.69474424,92.6746245)(797.73974419,92.6796245)(797.78973633,92.69963186)
\curveto(797.83974409,92.70962447)(797.92474401,92.71462446)(798.04473633,92.71463186)
\curveto(798.15474378,92.71462446)(798.23974369,92.70962447)(798.29973633,92.69963186)
\curveto(798.35974357,92.6796245)(798.41974351,92.66962451)(798.47973633,92.66963186)
\curveto(798.53974339,92.6796245)(798.59974333,92.6746245)(798.65973633,92.65463186)
\curveto(798.79974313,92.61462456)(798.934743,92.5796246)(799.06473633,92.54963186)
\curveto(799.19474274,92.51962466)(799.31974261,92.4796247)(799.43973633,92.42963186)
\curveto(799.57974235,92.36962481)(799.70474223,92.29962488)(799.81473633,92.21963186)
\curveto(799.92474201,92.14962503)(800.0347419,92.0746251)(800.14473633,91.99463186)
\lineto(800.20473633,91.93463186)
\curveto(800.22474171,91.92462525)(800.24474169,91.90962527)(800.26473633,91.88963186)
\curveto(800.42474151,91.76962541)(800.56974136,91.63462554)(800.69973633,91.48463186)
\curveto(800.8297411,91.33462584)(800.95474098,91.174626)(801.07473633,91.00463186)
\curveto(801.29474064,90.69462648)(801.49974043,90.39962678)(801.68973633,90.11963186)
\curveto(801.8297401,89.88962729)(801.96473997,89.65962752)(802.09473633,89.42963186)
\curveto(802.22473971,89.20962797)(802.35973957,88.98962819)(802.49973633,88.76963186)
\curveto(802.66973926,88.51962866)(802.84973908,88.2796289)(803.03973633,88.04963186)
\curveto(803.2297387,87.82962935)(803.45473848,87.63962954)(803.71473633,87.47963186)
\curveto(803.77473816,87.43962974)(803.8347381,87.40462977)(803.89473633,87.37463186)
\curveto(803.94473799,87.34462983)(804.00973792,87.31462986)(804.08973633,87.28463186)
\curveto(804.15973777,87.26462991)(804.21973771,87.25962992)(804.26973633,87.26963186)
\curveto(804.33973759,87.28962989)(804.39473754,87.32462985)(804.43473633,87.37463186)
\curveto(804.46473747,87.42462975)(804.48473745,87.48462969)(804.49473633,87.55463186)
\lineto(804.49473633,87.79463186)
\lineto(804.49473633,88.54463186)
\lineto(804.49473633,91.34963186)
\lineto(804.49473633,92.00963186)
\curveto(804.49473744,92.09962508)(804.49973743,92.18462499)(804.50973633,92.26463186)
\curveto(804.50973742,92.34462483)(804.5297374,92.40962477)(804.56973633,92.45963186)
\curveto(804.60973732,92.50962467)(804.68473725,92.54962463)(804.79473633,92.57963186)
\curveto(804.89473704,92.61962456)(804.99473694,92.62962455)(805.09473633,92.60963186)
\lineto(805.22973633,92.60963186)
\curveto(805.29973663,92.58962459)(805.35973657,92.56962461)(805.40973633,92.54963186)
\curveto(805.45973647,92.52962465)(805.49973643,92.49462468)(805.52973633,92.44463186)
\curveto(805.56973636,92.39462478)(805.58973634,92.32462485)(805.58973633,92.23463186)
\lineto(805.58973633,91.96463186)
\lineto(805.58973633,91.06463186)
\lineto(805.58973633,87.55463186)
\lineto(805.58973633,86.48963186)
\curveto(805.58973634,86.40963077)(805.59473634,86.31963086)(805.60473633,86.21963186)
\curveto(805.60473633,86.11963106)(805.59473634,86.03463114)(805.57473633,85.96463186)
\curveto(805.50473643,85.75463142)(805.32473661,85.68963149)(805.03473633,85.76963186)
\curveto(804.99473694,85.7796314)(804.95973697,85.7796314)(804.92973633,85.76963186)
\curveto(804.88973704,85.76963141)(804.84473709,85.7796314)(804.79473633,85.79963186)
\curveto(804.71473722,85.81963136)(804.6297373,85.83963134)(804.53973633,85.85963186)
\curveto(804.44973748,85.8796313)(804.36473757,85.90463127)(804.28473633,85.93463186)
\curveto(803.79473814,86.09463108)(803.37973855,86.29463088)(803.03973633,86.53463186)
\curveto(802.78973914,86.71463046)(802.56473937,86.91963026)(802.36473633,87.14963186)
\curveto(802.15473978,87.3796298)(801.95973997,87.61962956)(801.77973633,87.86963186)
\curveto(801.59974033,88.12962905)(801.4297405,88.39462878)(801.26973633,88.66463186)
\curveto(801.09974083,88.94462823)(800.92474101,89.21462796)(800.74473633,89.47463186)
\curveto(800.66474127,89.58462759)(800.58974134,89.68962749)(800.51973633,89.78963186)
\curveto(800.44974148,89.89962728)(800.37474156,90.00962717)(800.29473633,90.11963186)
\curveto(800.26474167,90.15962702)(800.2347417,90.19462698)(800.20473633,90.22463186)
\curveto(800.16474177,90.26462691)(800.1347418,90.30462687)(800.11473633,90.34463186)
\curveto(800.00474193,90.48462669)(799.87974205,90.60962657)(799.73973633,90.71963186)
\curveto(799.70974222,90.73962644)(799.68474225,90.76462641)(799.66473633,90.79463186)
\curveto(799.6347423,90.82462635)(799.60474233,90.84962633)(799.57473633,90.86963186)
\curveto(799.47474246,90.94962623)(799.37474256,91.01462616)(799.27473633,91.06463186)
\curveto(799.17474276,91.12462605)(799.06474287,91.179626)(798.94473633,91.22963186)
\curveto(798.87474306,91.25962592)(798.79974313,91.2796259)(798.71973633,91.28963186)
\lineto(798.47973633,91.34963186)
\lineto(798.38973633,91.34963186)
\curveto(798.35974357,91.35962582)(798.3297436,91.36462581)(798.29973633,91.36463186)
\curveto(798.2297437,91.38462579)(798.1347438,91.38962579)(798.01473633,91.37963186)
\curveto(797.88474405,91.3796258)(797.78474415,91.36962581)(797.71473633,91.34963186)
\curveto(797.6347443,91.32962585)(797.55974437,91.30962587)(797.48973633,91.28963186)
\curveto(797.40974452,91.2796259)(797.3297446,91.25962592)(797.24973633,91.22963186)
\curveto(797.00974492,91.11962606)(796.80974512,90.96962621)(796.64973633,90.77963186)
\curveto(796.47974545,90.59962658)(796.33974559,90.3796268)(796.22973633,90.11963186)
\curveto(796.20974572,90.04962713)(796.19474574,89.9796272)(796.18473633,89.90963186)
\curveto(796.16474577,89.83962734)(796.14474579,89.76462741)(796.12473633,89.68463186)
\curveto(796.10474583,89.60462757)(796.09474584,89.49462768)(796.09473633,89.35463186)
\curveto(796.09474584,89.22462795)(796.10474583,89.11962806)(796.12473633,89.03963186)
\curveto(796.1347458,88.9796282)(796.13974579,88.92462825)(796.13973633,88.87463186)
\curveto(796.13974579,88.82462835)(796.14974578,88.7746284)(796.16973633,88.72463186)
\curveto(796.20974572,88.62462855)(796.24974568,88.52962865)(796.28973633,88.43963186)
\curveto(796.3297456,88.35962882)(796.37474556,88.2796289)(796.42473633,88.19963186)
\curveto(796.44474549,88.16962901)(796.46974546,88.13962904)(796.49973633,88.10963186)
\curveto(796.5297454,88.08962909)(796.55474538,88.06462911)(796.57473633,88.03463186)
\lineto(796.64973633,87.95963186)
\curveto(796.66974526,87.92962925)(796.68974524,87.90462927)(796.70973633,87.88463186)
\lineto(796.91973633,87.73463186)
\curveto(796.97974495,87.69462948)(797.04474489,87.64962953)(797.11473633,87.59963186)
\curveto(797.20474473,87.53962964)(797.30974462,87.48962969)(797.42973633,87.44963186)
\curveto(797.53974439,87.41962976)(797.64974428,87.38462979)(797.75973633,87.34463186)
\curveto(797.86974406,87.30462987)(798.01474392,87.2796299)(798.19473633,87.26963186)
\curveto(798.36474357,87.25962992)(798.48974344,87.22962995)(798.56973633,87.17963186)
\curveto(798.64974328,87.12963005)(798.69474324,87.05463012)(798.70473633,86.95463186)
\curveto(798.71474322,86.85463032)(798.71974321,86.74463043)(798.71973633,86.62463186)
\curveto(798.71974321,86.58463059)(798.72474321,86.54463063)(798.73473633,86.50463186)
\curveto(798.7347432,86.46463071)(798.7297432,86.42963075)(798.71973633,86.39963186)
\curveto(798.69974323,86.34963083)(798.68974324,86.29963088)(798.68973633,86.24963186)
\curveto(798.68974324,86.20963097)(798.67974325,86.16963101)(798.65973633,86.12963186)
\curveto(798.59974333,86.03963114)(798.46474347,85.99463118)(798.25473633,85.99463186)
\lineto(798.13473633,85.99463186)
\curveto(798.07474386,86.00463117)(798.01474392,86.00963117)(797.95473633,86.00963186)
\curveto(797.88474405,86.01963116)(797.81974411,86.02963115)(797.75973633,86.03963186)
\curveto(797.64974428,86.05963112)(797.54974438,86.0796311)(797.45973633,86.09963186)
\curveto(797.35974457,86.11963106)(797.26474467,86.14963103)(797.17473633,86.18963186)
\curveto(797.10474483,86.20963097)(797.04474489,86.22963095)(796.99473633,86.24963186)
\lineto(796.81473633,86.30963186)
\curveto(796.55474538,86.42963075)(796.30974562,86.58463059)(796.07973633,86.77463186)
\curveto(795.84974608,86.9746302)(795.66474627,87.18962999)(795.52473633,87.41963186)
\curveto(795.44474649,87.52962965)(795.37974655,87.64462953)(795.32973633,87.76463186)
\lineto(795.17973633,88.15463186)
\curveto(795.1297468,88.26462891)(795.09974683,88.3796288)(795.08973633,88.49963186)
\curveto(795.06974686,88.61962856)(795.04474689,88.74462843)(795.01473633,88.87463186)
\curveto(795.01474692,88.94462823)(795.01474692,89.00962817)(795.01473633,89.06963186)
\curveto(795.00474693,89.12962805)(794.99474694,89.19462798)(794.98473633,89.26463186)
}
}
{
\newrgbcolor{curcolor}{0 0 0}
\pscustom[linestyle=none,fillstyle=solid,fillcolor=curcolor]
{
\newpath
\moveto(800.50473633,101.36424123)
\lineto(800.75973633,101.36424123)
\curveto(800.83974109,101.37423353)(800.91474102,101.36923353)(800.98473633,101.34924123)
\lineto(801.22473633,101.34924123)
\lineto(801.38973633,101.34924123)
\curveto(801.48974044,101.32923357)(801.59474034,101.31923358)(801.70473633,101.31924123)
\curveto(801.80474013,101.31923358)(801.90474003,101.30923359)(802.00473633,101.28924123)
\lineto(802.15473633,101.28924123)
\curveto(802.29473964,101.25923364)(802.4347395,101.23923366)(802.57473633,101.22924123)
\curveto(802.70473923,101.21923368)(802.8347391,101.19423371)(802.96473633,101.15424123)
\curveto(803.04473889,101.13423377)(803.1297388,101.11423379)(803.21973633,101.09424123)
\lineto(803.45973633,101.03424123)
\lineto(803.75973633,100.91424123)
\curveto(803.84973808,100.88423402)(803.93973799,100.84923405)(804.02973633,100.80924123)
\curveto(804.24973768,100.70923419)(804.46473747,100.57423433)(804.67473633,100.40424123)
\curveto(804.88473705,100.24423466)(805.05473688,100.06923483)(805.18473633,99.87924123)
\curveto(805.22473671,99.82923507)(805.26473667,99.76923513)(805.30473633,99.69924123)
\curveto(805.3347366,99.63923526)(805.36973656,99.57923532)(805.40973633,99.51924123)
\curveto(805.45973647,99.43923546)(805.49973643,99.34423556)(805.52973633,99.23424123)
\curveto(805.55973637,99.12423578)(805.58973634,99.01923588)(805.61973633,98.91924123)
\curveto(805.65973627,98.80923609)(805.68473625,98.6992362)(805.69473633,98.58924123)
\curveto(805.70473623,98.47923642)(805.71973621,98.36423654)(805.73973633,98.24424123)
\curveto(805.74973618,98.2042367)(805.74973618,98.15923674)(805.73973633,98.10924123)
\curveto(805.73973619,98.06923683)(805.74473619,98.02923687)(805.75473633,97.98924123)
\curveto(805.76473617,97.94923695)(805.76973616,97.89423701)(805.76973633,97.82424123)
\curveto(805.76973616,97.75423715)(805.76473617,97.7042372)(805.75473633,97.67424123)
\curveto(805.7347362,97.62423728)(805.7297362,97.57923732)(805.73973633,97.53924123)
\curveto(805.74973618,97.4992374)(805.74973618,97.46423744)(805.73973633,97.43424123)
\lineto(805.73973633,97.34424123)
\curveto(805.71973621,97.28423762)(805.70473623,97.21923768)(805.69473633,97.14924123)
\curveto(805.69473624,97.08923781)(805.68973624,97.02423788)(805.67973633,96.95424123)
\curveto(805.6297363,96.78423812)(805.57973635,96.62423828)(805.52973633,96.47424123)
\curveto(805.47973645,96.32423858)(805.41473652,96.17923872)(805.33473633,96.03924123)
\curveto(805.29473664,95.98923891)(805.26473667,95.93423897)(805.24473633,95.87424123)
\curveto(805.21473672,95.82423908)(805.17973675,95.77423913)(805.13973633,95.72424123)
\curveto(804.95973697,95.48423942)(804.73973719,95.28423962)(804.47973633,95.12424123)
\curveto(804.21973771,94.96423994)(803.934738,94.82424008)(803.62473633,94.70424123)
\curveto(803.48473845,94.64424026)(803.34473859,94.5992403)(803.20473633,94.56924123)
\curveto(803.05473888,94.53924036)(802.89973903,94.5042404)(802.73973633,94.46424123)
\curveto(802.6297393,94.44424046)(802.51973941,94.42924047)(802.40973633,94.41924123)
\curveto(802.29973963,94.40924049)(802.18973974,94.39424051)(802.07973633,94.37424123)
\curveto(802.03973989,94.36424054)(801.99973993,94.35924054)(801.95973633,94.35924123)
\curveto(801.91974001,94.36924053)(801.87974005,94.36924053)(801.83973633,94.35924123)
\curveto(801.78974014,94.34924055)(801.73974019,94.34424056)(801.68973633,94.34424123)
\lineto(801.52473633,94.34424123)
\curveto(801.47474046,94.32424058)(801.42474051,94.31924058)(801.37473633,94.32924123)
\curveto(801.31474062,94.33924056)(801.25974067,94.33924056)(801.20973633,94.32924123)
\curveto(801.16974076,94.31924058)(801.12474081,94.31924058)(801.07473633,94.32924123)
\curveto(801.02474091,94.33924056)(800.97474096,94.33424057)(800.92473633,94.31424123)
\curveto(800.85474108,94.29424061)(800.77974115,94.28924061)(800.69973633,94.29924123)
\curveto(800.60974132,94.30924059)(800.52474141,94.31424059)(800.44473633,94.31424123)
\curveto(800.35474158,94.31424059)(800.25474168,94.30924059)(800.14473633,94.29924123)
\curveto(800.02474191,94.28924061)(799.92474201,94.29424061)(799.84473633,94.31424123)
\lineto(799.55973633,94.31424123)
\lineto(798.92973633,94.35924123)
\curveto(798.8297431,94.36924053)(798.7347432,94.37924052)(798.64473633,94.38924123)
\lineto(798.34473633,94.41924123)
\curveto(798.29474364,94.43924046)(798.24474369,94.44424046)(798.19473633,94.43424123)
\curveto(798.1347438,94.43424047)(798.07974385,94.44424046)(798.02973633,94.46424123)
\curveto(797.85974407,94.51424039)(797.69474424,94.55424035)(797.53473633,94.58424123)
\curveto(797.36474457,94.61424029)(797.20474473,94.66424024)(797.05473633,94.73424123)
\curveto(796.59474534,94.92423998)(796.21974571,95.14423976)(795.92973633,95.39424123)
\curveto(795.63974629,95.65423925)(795.39474654,96.01423889)(795.19473633,96.47424123)
\curveto(795.14474679,96.6042383)(795.10974682,96.73423817)(795.08973633,96.86424123)
\curveto(795.06974686,97.0042379)(795.04474689,97.14423776)(795.01473633,97.28424123)
\curveto(795.00474693,97.35423755)(794.99974693,97.41923748)(794.99973633,97.47924123)
\curveto(794.99974693,97.53923736)(794.99474694,97.6042373)(794.98473633,97.67424123)
\curveto(794.96474697,98.5042364)(795.11474682,99.17423573)(795.43473633,99.68424123)
\curveto(795.74474619,100.19423471)(796.18474575,100.57423433)(796.75473633,100.82424123)
\curveto(796.87474506,100.87423403)(796.99974493,100.91923398)(797.12973633,100.95924123)
\curveto(797.25974467,100.9992339)(797.39474454,101.04423386)(797.53473633,101.09424123)
\curveto(797.61474432,101.11423379)(797.69974423,101.12923377)(797.78973633,101.13924123)
\lineto(798.02973633,101.19924123)
\curveto(798.13974379,101.22923367)(798.24974368,101.24423366)(798.35973633,101.24424123)
\curveto(798.46974346,101.25423365)(798.57974335,101.26923363)(798.68973633,101.28924123)
\curveto(798.73974319,101.30923359)(798.78474315,101.31423359)(798.82473633,101.30424123)
\curveto(798.86474307,101.3042336)(798.90474303,101.30923359)(798.94473633,101.31924123)
\curveto(798.99474294,101.32923357)(799.04974288,101.32923357)(799.10973633,101.31924123)
\curveto(799.15974277,101.31923358)(799.20974272,101.32423358)(799.25973633,101.33424123)
\lineto(799.39473633,101.33424123)
\curveto(799.45474248,101.35423355)(799.52474241,101.35423355)(799.60473633,101.33424123)
\curveto(799.67474226,101.32423358)(799.73974219,101.32923357)(799.79973633,101.34924123)
\curveto(799.8297421,101.35923354)(799.86974206,101.36423354)(799.91973633,101.36424123)
\lineto(800.03973633,101.36424123)
\lineto(800.50473633,101.36424123)
\moveto(802.82973633,99.81924123)
\curveto(802.50973942,99.91923498)(802.14473979,99.97923492)(801.73473633,99.99924123)
\curveto(801.32474061,100.01923488)(800.91474102,100.02923487)(800.50473633,100.02924123)
\curveto(800.07474186,100.02923487)(799.65474228,100.01923488)(799.24473633,99.99924123)
\curveto(798.8347431,99.97923492)(798.44974348,99.93423497)(798.08973633,99.86424123)
\curveto(797.7297442,99.79423511)(797.40974452,99.68423522)(797.12973633,99.53424123)
\curveto(796.83974509,99.39423551)(796.60474533,99.1992357)(796.42473633,98.94924123)
\curveto(796.31474562,98.78923611)(796.2347457,98.60923629)(796.18473633,98.40924123)
\curveto(796.12474581,98.20923669)(796.09474584,97.96423694)(796.09473633,97.67424123)
\curveto(796.11474582,97.65423725)(796.12474581,97.61923728)(796.12473633,97.56924123)
\curveto(796.11474582,97.51923738)(796.11474582,97.47923742)(796.12473633,97.44924123)
\curveto(796.14474579,97.36923753)(796.16474577,97.29423761)(796.18473633,97.22424123)
\curveto(796.19474574,97.16423774)(796.21474572,97.0992378)(796.24473633,97.02924123)
\curveto(796.36474557,96.75923814)(796.5347454,96.53923836)(796.75473633,96.36924123)
\curveto(796.96474497,96.20923869)(797.20974472,96.07423883)(797.48973633,95.96424123)
\curveto(797.59974433,95.91423899)(797.71974421,95.87423903)(797.84973633,95.84424123)
\curveto(797.96974396,95.82423908)(798.09474384,95.7992391)(798.22473633,95.76924123)
\curveto(798.27474366,95.74923915)(798.3297436,95.73923916)(798.38973633,95.73924123)
\curveto(798.43974349,95.73923916)(798.48974344,95.73423917)(798.53973633,95.72424123)
\curveto(798.6297433,95.71423919)(798.72474321,95.7042392)(798.82473633,95.69424123)
\curveto(798.91474302,95.68423922)(799.00974292,95.67423923)(799.10973633,95.66424123)
\curveto(799.18974274,95.66423924)(799.27474266,95.65923924)(799.36473633,95.64924123)
\lineto(799.60473633,95.64924123)
\lineto(799.78473633,95.64924123)
\curveto(799.81474212,95.63923926)(799.84974208,95.63423927)(799.88973633,95.63424123)
\lineto(800.02473633,95.63424123)
\lineto(800.47473633,95.63424123)
\curveto(800.55474138,95.63423927)(800.63974129,95.62923927)(800.72973633,95.61924123)
\curveto(800.80974112,95.61923928)(800.88474105,95.62923927)(800.95473633,95.64924123)
\lineto(801.22473633,95.64924123)
\curveto(801.24474069,95.64923925)(801.27474066,95.64423926)(801.31473633,95.63424123)
\curveto(801.34474059,95.63423927)(801.36974056,95.63923926)(801.38973633,95.64924123)
\curveto(801.48974044,95.65923924)(801.58974034,95.66423924)(801.68973633,95.66424123)
\curveto(801.77974015,95.67423923)(801.87974005,95.68423922)(801.98973633,95.69424123)
\curveto(802.10973982,95.72423918)(802.2347397,95.73923916)(802.36473633,95.73924123)
\curveto(802.48473945,95.74923915)(802.59973933,95.77423913)(802.70973633,95.81424123)
\curveto(803.00973892,95.89423901)(803.27473866,95.97923892)(803.50473633,96.06924123)
\curveto(803.7347382,96.16923873)(803.94973798,96.31423859)(804.14973633,96.50424123)
\curveto(804.34973758,96.71423819)(804.49973743,96.97923792)(804.59973633,97.29924123)
\curveto(804.61973731,97.33923756)(804.6297373,97.37423753)(804.62973633,97.40424123)
\curveto(804.61973731,97.44423746)(804.62473731,97.48923741)(804.64473633,97.53924123)
\curveto(804.65473728,97.57923732)(804.66473727,97.64923725)(804.67473633,97.74924123)
\curveto(804.68473725,97.85923704)(804.67973725,97.94423696)(804.65973633,98.00424123)
\curveto(804.63973729,98.07423683)(804.6297373,98.14423676)(804.62973633,98.21424123)
\curveto(804.61973731,98.28423662)(804.60473733,98.34923655)(804.58473633,98.40924123)
\curveto(804.52473741,98.60923629)(804.43973749,98.78923611)(804.32973633,98.94924123)
\curveto(804.30973762,98.97923592)(804.28973764,99.0042359)(804.26973633,99.02424123)
\lineto(804.20973633,99.08424123)
\curveto(804.18973774,99.12423578)(804.14973778,99.17423573)(804.08973633,99.23424123)
\curveto(803.94973798,99.33423557)(803.81973811,99.41923548)(803.69973633,99.48924123)
\curveto(803.57973835,99.55923534)(803.4347385,99.62923527)(803.26473633,99.69924123)
\curveto(803.19473874,99.72923517)(803.12473881,99.74923515)(803.05473633,99.75924123)
\curveto(802.98473895,99.77923512)(802.90973902,99.7992351)(802.82973633,99.81924123)
}
}
{
\newrgbcolor{curcolor}{0 0 0}
\pscustom[linestyle=none,fillstyle=solid,fillcolor=curcolor]
{
\newpath
\moveto(794.98473633,106.77385061)
\curveto(794.98474695,106.87384575)(794.99474694,106.96884566)(795.01473633,107.05885061)
\curveto(795.02474691,107.14884548)(795.05474688,107.21384541)(795.10473633,107.25385061)
\curveto(795.18474675,107.31384531)(795.28974664,107.34384528)(795.41973633,107.34385061)
\lineto(795.80973633,107.34385061)
\lineto(797.30973633,107.34385061)
\lineto(803.69973633,107.34385061)
\lineto(804.86973633,107.34385061)
\lineto(805.18473633,107.34385061)
\curveto(805.28473665,107.35384527)(805.36473657,107.33884529)(805.42473633,107.29885061)
\curveto(805.50473643,107.24884538)(805.55473638,107.17384545)(805.57473633,107.07385061)
\curveto(805.58473635,106.98384564)(805.58973634,106.87384575)(805.58973633,106.74385061)
\lineto(805.58973633,106.51885061)
\curveto(805.56973636,106.43884619)(805.55473638,106.36884626)(805.54473633,106.30885061)
\curveto(805.52473641,106.24884638)(805.48473645,106.19884643)(805.42473633,106.15885061)
\curveto(805.36473657,106.11884651)(805.28973664,106.09884653)(805.19973633,106.09885061)
\lineto(804.89973633,106.09885061)
\lineto(803.80473633,106.09885061)
\lineto(798.46473633,106.09885061)
\curveto(798.37474356,106.07884655)(798.29974363,106.06384656)(798.23973633,106.05385061)
\curveto(798.16974376,106.05384657)(798.10974382,106.0238466)(798.05973633,105.96385061)
\curveto(798.00974392,105.89384673)(797.98474395,105.80384682)(797.98473633,105.69385061)
\curveto(797.97474396,105.59384703)(797.96974396,105.48384714)(797.96973633,105.36385061)
\lineto(797.96973633,104.22385061)
\lineto(797.96973633,103.72885061)
\curveto(797.95974397,103.56884906)(797.89974403,103.45884917)(797.78973633,103.39885061)
\curveto(797.75974417,103.37884925)(797.7297442,103.36884926)(797.69973633,103.36885061)
\curveto(797.65974427,103.36884926)(797.61474432,103.36384926)(797.56473633,103.35385061)
\curveto(797.44474449,103.33384929)(797.3347446,103.33884929)(797.23473633,103.36885061)
\curveto(797.1347448,103.40884922)(797.06474487,103.46384916)(797.02473633,103.53385061)
\curveto(796.97474496,103.61384901)(796.94974498,103.73384889)(796.94973633,103.89385061)
\curveto(796.94974498,104.05384857)(796.934745,104.18884844)(796.90473633,104.29885061)
\curveto(796.89474504,104.34884828)(796.88974504,104.40384822)(796.88973633,104.46385061)
\curveto(796.87974505,104.5238481)(796.86474507,104.58384804)(796.84473633,104.64385061)
\curveto(796.79474514,104.79384783)(796.74474519,104.93884769)(796.69473633,105.07885061)
\curveto(796.6347453,105.21884741)(796.56474537,105.35384727)(796.48473633,105.48385061)
\curveto(796.39474554,105.623847)(796.28974564,105.74384688)(796.16973633,105.84385061)
\curveto(796.04974588,105.94384668)(795.91974601,106.03884659)(795.77973633,106.12885061)
\curveto(795.67974625,106.18884644)(795.56974636,106.23384639)(795.44973633,106.26385061)
\curveto(795.3297466,106.30384632)(795.22474671,106.35384627)(795.13473633,106.41385061)
\curveto(795.07474686,106.46384616)(795.0347469,106.53384609)(795.01473633,106.62385061)
\curveto(795.00474693,106.64384598)(794.99974693,106.66884596)(794.99973633,106.69885061)
\curveto(794.99974693,106.7288459)(794.99474694,106.75384587)(794.98473633,106.77385061)
}
}
{
\newrgbcolor{curcolor}{0 0 0}
\pscustom[linestyle=none,fillstyle=solid,fillcolor=curcolor]
{
\newpath
\moveto(794.98473633,115.12345998)
\curveto(794.98474695,115.22345513)(794.99474694,115.31845503)(795.01473633,115.40845998)
\curveto(795.02474691,115.49845485)(795.05474688,115.56345479)(795.10473633,115.60345998)
\curveto(795.18474675,115.66345469)(795.28974664,115.69345466)(795.41973633,115.69345998)
\lineto(795.80973633,115.69345998)
\lineto(797.30973633,115.69345998)
\lineto(803.69973633,115.69345998)
\lineto(804.86973633,115.69345998)
\lineto(805.18473633,115.69345998)
\curveto(805.28473665,115.70345465)(805.36473657,115.68845466)(805.42473633,115.64845998)
\curveto(805.50473643,115.59845475)(805.55473638,115.52345483)(805.57473633,115.42345998)
\curveto(805.58473635,115.33345502)(805.58973634,115.22345513)(805.58973633,115.09345998)
\lineto(805.58973633,114.86845998)
\curveto(805.56973636,114.78845556)(805.55473638,114.71845563)(805.54473633,114.65845998)
\curveto(805.52473641,114.59845575)(805.48473645,114.5484558)(805.42473633,114.50845998)
\curveto(805.36473657,114.46845588)(805.28973664,114.4484559)(805.19973633,114.44845998)
\lineto(804.89973633,114.44845998)
\lineto(803.80473633,114.44845998)
\lineto(798.46473633,114.44845998)
\curveto(798.37474356,114.42845592)(798.29974363,114.41345594)(798.23973633,114.40345998)
\curveto(798.16974376,114.40345595)(798.10974382,114.37345598)(798.05973633,114.31345998)
\curveto(798.00974392,114.24345611)(797.98474395,114.1534562)(797.98473633,114.04345998)
\curveto(797.97474396,113.94345641)(797.96974396,113.83345652)(797.96973633,113.71345998)
\lineto(797.96973633,112.57345998)
\lineto(797.96973633,112.07845998)
\curveto(797.95974397,111.91845843)(797.89974403,111.80845854)(797.78973633,111.74845998)
\curveto(797.75974417,111.72845862)(797.7297442,111.71845863)(797.69973633,111.71845998)
\curveto(797.65974427,111.71845863)(797.61474432,111.71345864)(797.56473633,111.70345998)
\curveto(797.44474449,111.68345867)(797.3347446,111.68845866)(797.23473633,111.71845998)
\curveto(797.1347448,111.75845859)(797.06474487,111.81345854)(797.02473633,111.88345998)
\curveto(796.97474496,111.96345839)(796.94974498,112.08345827)(796.94973633,112.24345998)
\curveto(796.94974498,112.40345795)(796.934745,112.53845781)(796.90473633,112.64845998)
\curveto(796.89474504,112.69845765)(796.88974504,112.7534576)(796.88973633,112.81345998)
\curveto(796.87974505,112.87345748)(796.86474507,112.93345742)(796.84473633,112.99345998)
\curveto(796.79474514,113.14345721)(796.74474519,113.28845706)(796.69473633,113.42845998)
\curveto(796.6347453,113.56845678)(796.56474537,113.70345665)(796.48473633,113.83345998)
\curveto(796.39474554,113.97345638)(796.28974564,114.09345626)(796.16973633,114.19345998)
\curveto(796.04974588,114.29345606)(795.91974601,114.38845596)(795.77973633,114.47845998)
\curveto(795.67974625,114.53845581)(795.56974636,114.58345577)(795.44973633,114.61345998)
\curveto(795.3297466,114.6534557)(795.22474671,114.70345565)(795.13473633,114.76345998)
\curveto(795.07474686,114.81345554)(795.0347469,114.88345547)(795.01473633,114.97345998)
\curveto(795.00474693,114.99345536)(794.99974693,115.01845533)(794.99973633,115.04845998)
\curveto(794.99974693,115.07845527)(794.99474694,115.10345525)(794.98473633,115.12345998)
}
}
{
\newrgbcolor{curcolor}{0 0 0}
\pscustom[linestyle=none,fillstyle=solid,fillcolor=curcolor]
{
\newpath
\moveto(815.82105225,42.29681936)
\curveto(815.82106294,42.36681368)(815.82106294,42.4468136)(815.82105225,42.53681936)
\curveto(815.81106295,42.62681342)(815.81106295,42.71181333)(815.82105225,42.79181936)
\curveto(815.82106294,42.88181316)(815.83106293,42.96181308)(815.85105225,43.03181936)
\curveto(815.87106289,43.11181293)(815.90106286,43.16681288)(815.94105225,43.19681936)
\curveto(815.99106277,43.22681282)(816.0660627,43.2468128)(816.16605225,43.25681936)
\curveto(816.25606251,43.27681277)(816.3610624,43.28681276)(816.48105225,43.28681936)
\curveto(816.59106217,43.29681275)(816.70606206,43.29681275)(816.82605225,43.28681936)
\lineto(817.12605225,43.28681936)
\lineto(820.14105225,43.28681936)
\lineto(823.03605225,43.28681936)
\curveto(823.3660554,43.28681276)(823.69105507,43.28181276)(824.01105225,43.27181936)
\curveto(824.32105444,43.27181277)(824.60105416,43.23181281)(824.85105225,43.15181936)
\curveto(825.20105356,43.03181301)(825.49605327,42.87681317)(825.73605225,42.68681936)
\curveto(825.9660528,42.49681355)(826.1660526,42.25681379)(826.33605225,41.96681936)
\curveto(826.38605238,41.90681414)(826.42105234,41.8418142)(826.44105225,41.77181936)
\curveto(826.4610523,41.71181433)(826.48605228,41.6418144)(826.51605225,41.56181936)
\curveto(826.5660522,41.4418146)(826.60105216,41.31181473)(826.62105225,41.17181936)
\curveto(826.65105211,41.041815)(826.68105208,40.90681514)(826.71105225,40.76681936)
\curveto(826.73105203,40.71681533)(826.73605203,40.66681538)(826.72605225,40.61681936)
\curveto(826.71605205,40.56681548)(826.71605205,40.51181553)(826.72605225,40.45181936)
\curveto(826.73605203,40.43181561)(826.73605203,40.40681564)(826.72605225,40.37681936)
\curveto(826.72605204,40.3468157)(826.73105203,40.32181572)(826.74105225,40.30181936)
\curveto(826.75105201,40.26181578)(826.75605201,40.20681584)(826.75605225,40.13681936)
\curveto(826.75605201,40.06681598)(826.75105201,40.01181603)(826.74105225,39.97181936)
\curveto(826.73105203,39.92181612)(826.73105203,39.86681618)(826.74105225,39.80681936)
\curveto(826.75105201,39.7468163)(826.74605202,39.69181635)(826.72605225,39.64181936)
\curveto(826.69605207,39.51181653)(826.67605209,39.38681666)(826.66605225,39.26681936)
\curveto(826.65605211,39.1468169)(826.63105213,39.03181701)(826.59105225,38.92181936)
\curveto(826.47105229,38.55181749)(826.30105246,38.23181781)(826.08105225,37.96181936)
\curveto(825.8610529,37.69181835)(825.58105318,37.48181856)(825.24105225,37.33181936)
\curveto(825.12105364,37.28181876)(824.99605377,37.23681881)(824.86605225,37.19681936)
\curveto(824.73605403,37.16681888)(824.60105416,37.13181891)(824.46105225,37.09181936)
\curveto(824.41105435,37.08181896)(824.37105439,37.07681897)(824.34105225,37.07681936)
\curveto(824.30105446,37.07681897)(824.25605451,37.07181897)(824.20605225,37.06181936)
\curveto(824.17605459,37.05181899)(824.14105462,37.046819)(824.10105225,37.04681936)
\curveto(824.05105471,37.046819)(824.01105475,37.041819)(823.98105225,37.03181936)
\lineto(823.81605225,37.03181936)
\curveto(823.73605503,37.01181903)(823.63605513,37.00681904)(823.51605225,37.01681936)
\curveto(823.38605538,37.02681902)(823.29605547,37.041819)(823.24605225,37.06181936)
\curveto(823.15605561,37.08181896)(823.09105567,37.13681891)(823.05105225,37.22681936)
\curveto(823.03105573,37.25681879)(823.02605574,37.28681876)(823.03605225,37.31681936)
\curveto(823.03605573,37.3468187)(823.03105573,37.38681866)(823.02105225,37.43681936)
\curveto(823.01105575,37.47681857)(823.00605576,37.51681853)(823.00605225,37.55681936)
\lineto(823.00605225,37.70681936)
\curveto(823.00605576,37.82681822)(823.01105575,37.9468181)(823.02105225,38.06681936)
\curveto(823.02105574,38.19681785)(823.05605571,38.28681776)(823.12605225,38.33681936)
\curveto(823.18605558,38.37681767)(823.24605552,38.39681765)(823.30605225,38.39681936)
\curveto(823.3660554,38.39681765)(823.43605533,38.40681764)(823.51605225,38.42681936)
\curveto(823.54605522,38.43681761)(823.58105518,38.43681761)(823.62105225,38.42681936)
\curveto(823.65105511,38.42681762)(823.67605509,38.43181761)(823.69605225,38.44181936)
\lineto(823.90605225,38.44181936)
\curveto(823.95605481,38.46181758)(824.00605476,38.46681758)(824.05605225,38.45681936)
\curveto(824.09605467,38.45681759)(824.14105462,38.46681758)(824.19105225,38.48681936)
\curveto(824.32105444,38.51681753)(824.44605432,38.5468175)(824.56605225,38.57681936)
\curveto(824.67605409,38.60681744)(824.78105398,38.65181739)(824.88105225,38.71181936)
\curveto(825.17105359,38.88181716)(825.37605339,39.15181689)(825.49605225,39.52181936)
\curveto(825.51605325,39.57181647)(825.53105323,39.62181642)(825.54105225,39.67181936)
\curveto(825.54105322,39.73181631)(825.55105321,39.78681626)(825.57105225,39.83681936)
\lineto(825.57105225,39.91181936)
\curveto(825.58105318,39.98181606)(825.59105317,40.07681597)(825.60105225,40.19681936)
\curveto(825.60105316,40.32681572)(825.59105317,40.42681562)(825.57105225,40.49681936)
\curveto(825.55105321,40.56681548)(825.53605323,40.63681541)(825.52605225,40.70681936)
\curveto(825.50605326,40.78681526)(825.48605328,40.85681519)(825.46605225,40.91681936)
\curveto(825.30605346,41.29681475)(825.03105373,41.57181447)(824.64105225,41.74181936)
\curveto(824.51105425,41.79181425)(824.35605441,41.82681422)(824.17605225,41.84681936)
\curveto(823.99605477,41.87681417)(823.81105495,41.89181415)(823.62105225,41.89181936)
\curveto(823.42105534,41.90181414)(823.22105554,41.90181414)(823.02105225,41.89181936)
\lineto(822.45105225,41.89181936)
\lineto(818.20605225,41.89181936)
\lineto(816.66105225,41.89181936)
\curveto(816.55106221,41.89181415)(816.43106233,41.88681416)(816.30105225,41.87681936)
\curveto(816.17106259,41.86681418)(816.0660627,41.88681416)(815.98605225,41.93681936)
\curveto(815.91606285,41.99681405)(815.8660629,42.07681397)(815.83605225,42.17681936)
\curveto(815.83606293,42.19681385)(815.83606293,42.21681383)(815.83605225,42.23681936)
\curveto(815.83606293,42.25681379)(815.83106293,42.27681377)(815.82105225,42.29681936)
}
}
{
\newrgbcolor{curcolor}{0 0 0}
\pscustom[linestyle=none,fillstyle=solid,fillcolor=curcolor]
{
\newpath
\moveto(818.77605225,45.83049123)
\lineto(818.77605225,46.26549123)
\curveto(818.77605999,46.41548927)(818.81605995,46.52048916)(818.89605225,46.58049123)
\curveto(818.97605979,46.63048905)(819.07605969,46.65548903)(819.19605225,46.65549123)
\curveto(819.31605945,46.66548902)(819.43605933,46.67048901)(819.55605225,46.67049123)
\lineto(820.98105225,46.67049123)
\lineto(823.24605225,46.67049123)
\lineto(823.93605225,46.67049123)
\curveto(824.1660546,46.67048901)(824.3660544,46.69548899)(824.53605225,46.74549123)
\curveto(824.98605378,46.90548878)(825.30105346,47.20548848)(825.48105225,47.64549123)
\curveto(825.57105319,47.86548782)(825.60605316,48.13048755)(825.58605225,48.44049123)
\curveto(825.55605321,48.75048693)(825.50105326,49.00048668)(825.42105225,49.19049123)
\curveto(825.28105348,49.52048616)(825.10605366,49.7804859)(824.89605225,49.97049123)
\curveto(824.67605409,50.17048551)(824.39105437,50.32548536)(824.04105225,50.43549123)
\curveto(823.9610548,50.46548522)(823.88105488,50.4854852)(823.80105225,50.49549123)
\curveto(823.72105504,50.50548518)(823.63605513,50.52048516)(823.54605225,50.54049123)
\curveto(823.49605527,50.55048513)(823.45105531,50.55048513)(823.41105225,50.54049123)
\curveto(823.37105539,50.54048514)(823.32605544,50.55048513)(823.27605225,50.57049123)
\lineto(822.96105225,50.57049123)
\curveto(822.88105588,50.59048509)(822.79105597,50.59548509)(822.69105225,50.58549123)
\curveto(822.58105618,50.57548511)(822.48105628,50.57048511)(822.39105225,50.57049123)
\lineto(821.22105225,50.57049123)
\lineto(819.63105225,50.57049123)
\curveto(819.51105925,50.57048511)(819.38605938,50.56548512)(819.25605225,50.55549123)
\curveto(819.11605965,50.55548513)(819.00605976,50.5804851)(818.92605225,50.63049123)
\curveto(818.87605989,50.67048501)(818.84605992,50.71548497)(818.83605225,50.76549123)
\curveto(818.81605995,50.82548486)(818.79605997,50.89548479)(818.77605225,50.97549123)
\lineto(818.77605225,51.20049123)
\curveto(818.77605999,51.32048436)(818.78105998,51.42548426)(818.79105225,51.51549123)
\curveto(818.80105996,51.61548407)(818.84605992,51.69048399)(818.92605225,51.74049123)
\curveto(818.97605979,51.79048389)(819.05105971,51.81548387)(819.15105225,51.81549123)
\lineto(819.43605225,51.81549123)
\lineto(820.45605225,51.81549123)
\lineto(824.49105225,51.81549123)
\lineto(825.84105225,51.81549123)
\curveto(825.9610528,51.81548387)(826.07605269,51.81048387)(826.18605225,51.80049123)
\curveto(826.28605248,51.80048388)(826.3610524,51.76548392)(826.41105225,51.69549123)
\curveto(826.44105232,51.65548403)(826.4660523,51.59548409)(826.48605225,51.51549123)
\curveto(826.49605227,51.43548425)(826.50605226,51.34548434)(826.51605225,51.24549123)
\curveto(826.51605225,51.15548453)(826.51105225,51.06548462)(826.50105225,50.97549123)
\curveto(826.49105227,50.89548479)(826.47105229,50.83548485)(826.44105225,50.79549123)
\curveto(826.40105236,50.74548494)(826.33605243,50.70048498)(826.24605225,50.66049123)
\curveto(826.20605256,50.65048503)(826.15105261,50.64048504)(826.08105225,50.63049123)
\curveto(826.01105275,50.63048505)(825.94605282,50.62548506)(825.88605225,50.61549123)
\curveto(825.81605295,50.60548508)(825.761053,50.5854851)(825.72105225,50.55549123)
\curveto(825.68105308,50.52548516)(825.6660531,50.4804852)(825.67605225,50.42049123)
\curveto(825.69605307,50.34048534)(825.75605301,50.26048542)(825.85605225,50.18049123)
\curveto(825.94605282,50.10048558)(826.01605275,50.02548566)(826.06605225,49.95549123)
\curveto(826.22605254,49.73548595)(826.3660524,49.4854862)(826.48605225,49.20549123)
\curveto(826.53605223,49.09548659)(826.5660522,48.9804867)(826.57605225,48.86049123)
\curveto(826.59605217,48.75048693)(826.62105214,48.63548705)(826.65105225,48.51549123)
\curveto(826.6610521,48.46548722)(826.6610521,48.41048727)(826.65105225,48.35049123)
\curveto(826.64105212,48.30048738)(826.64605212,48.25048743)(826.66605225,48.20049123)
\curveto(826.68605208,48.10048758)(826.68605208,48.01048767)(826.66605225,47.93049123)
\lineto(826.66605225,47.78049123)
\curveto(826.64605212,47.73048795)(826.63605213,47.67048801)(826.63605225,47.60049123)
\curveto(826.63605213,47.54048814)(826.63105213,47.4854882)(826.62105225,47.43549123)
\curveto(826.60105216,47.39548829)(826.59105217,47.35548833)(826.59105225,47.31549123)
\curveto(826.60105216,47.2854884)(826.59605217,47.24548844)(826.57605225,47.19549123)
\lineto(826.51605225,46.95549123)
\curveto(826.49605227,46.8854888)(826.4660523,46.81048887)(826.42605225,46.73049123)
\curveto(826.31605245,46.47048921)(826.17105259,46.25048943)(825.99105225,46.07049123)
\curveto(825.80105296,45.90048978)(825.57605319,45.76048992)(825.31605225,45.65049123)
\curveto(825.22605354,45.61049007)(825.13605363,45.5804901)(825.04605225,45.56049123)
\lineto(824.74605225,45.50049123)
\curveto(824.68605408,45.4804902)(824.63105413,45.47049021)(824.58105225,45.47049123)
\curveto(824.52105424,45.4804902)(824.45605431,45.47549021)(824.38605225,45.45549123)
\curveto(824.3660544,45.44549024)(824.34105442,45.44049024)(824.31105225,45.44049123)
\curveto(824.27105449,45.44049024)(824.23605453,45.43549025)(824.20605225,45.42549123)
\lineto(824.05605225,45.42549123)
\curveto(824.01605475,45.41549027)(823.97105479,45.41049027)(823.92105225,45.41049123)
\curveto(823.8610549,45.42049026)(823.80605496,45.42549026)(823.75605225,45.42549123)
\lineto(823.15605225,45.42549123)
\lineto(820.39605225,45.42549123)
\lineto(819.43605225,45.42549123)
\lineto(819.16605225,45.42549123)
\curveto(819.07605969,45.42549026)(819.00105976,45.44549024)(818.94105225,45.48549123)
\curveto(818.87105989,45.52549016)(818.82105994,45.60049008)(818.79105225,45.71049123)
\curveto(818.78105998,45.73048995)(818.78105998,45.75048993)(818.79105225,45.77049123)
\curveto(818.79105997,45.79048989)(818.78605998,45.81048987)(818.77605225,45.83049123)
}
}
{
\newrgbcolor{curcolor}{0 0 0}
\pscustom[linestyle=none,fillstyle=solid,fillcolor=curcolor]
{
\newpath
\moveto(815.82105225,54.28510061)
\curveto(815.82106294,54.41509899)(815.82106294,54.55009886)(815.82105225,54.69010061)
\curveto(815.82106294,54.84009857)(815.85606291,54.95009846)(815.92605225,55.02010061)
\curveto(815.99606277,55.07009834)(816.09106267,55.09509831)(816.21105225,55.09510061)
\curveto(816.32106244,55.1050983)(816.43606233,55.1100983)(816.55605225,55.11010061)
\lineto(817.89105225,55.11010061)
\lineto(823.96605225,55.11010061)
\lineto(825.64605225,55.11010061)
\lineto(826.03605225,55.11010061)
\curveto(826.17605259,55.1100983)(826.28605248,55.08509832)(826.36605225,55.03510061)
\curveto(826.41605235,55.0050984)(826.44605232,54.96009845)(826.45605225,54.90010061)
\curveto(826.4660523,54.85009856)(826.48105228,54.78509862)(826.50105225,54.70510061)
\lineto(826.50105225,54.49510061)
\lineto(826.50105225,54.18010061)
\curveto(826.49105227,54.08009933)(826.45605231,54.0050994)(826.39605225,53.95510061)
\curveto(826.31605245,53.9050995)(826.21605255,53.87509953)(826.09605225,53.86510061)
\lineto(825.72105225,53.86510061)
\lineto(824.34105225,53.86510061)
\lineto(818.10105225,53.86510061)
\lineto(816.63105225,53.86510061)
\curveto(816.52106224,53.86509954)(816.40606236,53.86009955)(816.28605225,53.85010061)
\curveto(816.15606261,53.85009956)(816.05606271,53.87509953)(815.98605225,53.92510061)
\curveto(815.92606284,53.96509944)(815.87606289,54.04009937)(815.83605225,54.15010061)
\curveto(815.82606294,54.17009924)(815.82606294,54.19009922)(815.83605225,54.21010061)
\curveto(815.83606293,54.24009917)(815.83106293,54.26509914)(815.82105225,54.28510061)
}
}
{
\newrgbcolor{curcolor}{0 0 0}
\pscustom[linestyle=none,fillstyle=solid,fillcolor=curcolor]
{
}
}
{
\newrgbcolor{curcolor}{0 0 0}
\pscustom[linestyle=none,fillstyle=solid,fillcolor=curcolor]
{
\newpath
\moveto(821.41605225,67.99510061)
\lineto(821.67105225,67.99510061)
\curveto(821.75105701,68.0050929)(821.82605694,68.00009291)(821.89605225,67.98010061)
\lineto(822.13605225,67.98010061)
\lineto(822.30105225,67.98010061)
\curveto(822.40105636,67.96009295)(822.50605626,67.95009296)(822.61605225,67.95010061)
\curveto(822.71605605,67.95009296)(822.81605595,67.94009297)(822.91605225,67.92010061)
\lineto(823.06605225,67.92010061)
\curveto(823.20605556,67.89009302)(823.34605542,67.87009304)(823.48605225,67.86010061)
\curveto(823.61605515,67.85009306)(823.74605502,67.82509308)(823.87605225,67.78510061)
\curveto(823.95605481,67.76509314)(824.04105472,67.74509316)(824.13105225,67.72510061)
\lineto(824.37105225,67.66510061)
\lineto(824.67105225,67.54510061)
\curveto(824.761054,67.51509339)(824.85105391,67.48009343)(824.94105225,67.44010061)
\curveto(825.1610536,67.34009357)(825.37605339,67.2050937)(825.58605225,67.03510061)
\curveto(825.79605297,66.87509403)(825.9660528,66.70009421)(826.09605225,66.51010061)
\curveto(826.13605263,66.46009445)(826.17605259,66.40009451)(826.21605225,66.33010061)
\curveto(826.24605252,66.27009464)(826.28105248,66.2100947)(826.32105225,66.15010061)
\curveto(826.37105239,66.07009484)(826.41105235,65.97509493)(826.44105225,65.86510061)
\curveto(826.47105229,65.75509515)(826.50105226,65.65009526)(826.53105225,65.55010061)
\curveto(826.57105219,65.44009547)(826.59605217,65.33009558)(826.60605225,65.22010061)
\curveto(826.61605215,65.1100958)(826.63105213,64.99509591)(826.65105225,64.87510061)
\curveto(826.6610521,64.83509607)(826.6610521,64.79009612)(826.65105225,64.74010061)
\curveto(826.65105211,64.70009621)(826.65605211,64.66009625)(826.66605225,64.62010061)
\curveto(826.67605209,64.58009633)(826.68105208,64.52509638)(826.68105225,64.45510061)
\curveto(826.68105208,64.38509652)(826.67605209,64.33509657)(826.66605225,64.30510061)
\curveto(826.64605212,64.25509665)(826.64105212,64.2100967)(826.65105225,64.17010061)
\curveto(826.6610521,64.13009678)(826.6610521,64.09509681)(826.65105225,64.06510061)
\lineto(826.65105225,63.97510061)
\curveto(826.63105213,63.91509699)(826.61605215,63.85009706)(826.60605225,63.78010061)
\curveto(826.60605216,63.72009719)(826.60105216,63.65509725)(826.59105225,63.58510061)
\curveto(826.54105222,63.41509749)(826.49105227,63.25509765)(826.44105225,63.10510061)
\curveto(826.39105237,62.95509795)(826.32605244,62.8100981)(826.24605225,62.67010061)
\curveto(826.20605256,62.62009829)(826.17605259,62.56509834)(826.15605225,62.50510061)
\curveto(826.12605264,62.45509845)(826.09105267,62.4050985)(826.05105225,62.35510061)
\curveto(825.87105289,62.11509879)(825.65105311,61.91509899)(825.39105225,61.75510061)
\curveto(825.13105363,61.59509931)(824.84605392,61.45509945)(824.53605225,61.33510061)
\curveto(824.39605437,61.27509963)(824.25605451,61.23009968)(824.11605225,61.20010061)
\curveto(823.9660548,61.17009974)(823.81105495,61.13509977)(823.65105225,61.09510061)
\curveto(823.54105522,61.07509983)(823.43105533,61.06009985)(823.32105225,61.05010061)
\curveto(823.21105555,61.04009987)(823.10105566,61.02509988)(822.99105225,61.00510061)
\curveto(822.95105581,60.99509991)(822.91105585,60.99009992)(822.87105225,60.99010061)
\curveto(822.83105593,61.00009991)(822.79105597,61.00009991)(822.75105225,60.99010061)
\curveto(822.70105606,60.98009993)(822.65105611,60.97509993)(822.60105225,60.97510061)
\lineto(822.43605225,60.97510061)
\curveto(822.38605638,60.95509995)(822.33605643,60.95009996)(822.28605225,60.96010061)
\curveto(822.22605654,60.97009994)(822.17105659,60.97009994)(822.12105225,60.96010061)
\curveto(822.08105668,60.95009996)(822.03605673,60.95009996)(821.98605225,60.96010061)
\curveto(821.93605683,60.97009994)(821.88605688,60.96509994)(821.83605225,60.94510061)
\curveto(821.766057,60.92509998)(821.69105707,60.92009999)(821.61105225,60.93010061)
\curveto(821.52105724,60.94009997)(821.43605733,60.94509996)(821.35605225,60.94510061)
\curveto(821.2660575,60.94509996)(821.1660576,60.94009997)(821.05605225,60.93010061)
\curveto(820.93605783,60.92009999)(820.83605793,60.92509998)(820.75605225,60.94510061)
\lineto(820.47105225,60.94510061)
\lineto(819.84105225,60.99010061)
\curveto(819.74105902,61.00009991)(819.64605912,61.0100999)(819.55605225,61.02010061)
\lineto(819.25605225,61.05010061)
\curveto(819.20605956,61.07009984)(819.15605961,61.07509983)(819.10605225,61.06510061)
\curveto(819.04605972,61.06509984)(818.99105977,61.07509983)(818.94105225,61.09510061)
\curveto(818.77105999,61.14509976)(818.60606016,61.18509972)(818.44605225,61.21510061)
\curveto(818.27606049,61.24509966)(818.11606065,61.29509961)(817.96605225,61.36510061)
\curveto(817.50606126,61.55509935)(817.13106163,61.77509913)(816.84105225,62.02510061)
\curveto(816.55106221,62.28509862)(816.30606246,62.64509826)(816.10605225,63.10510061)
\curveto(816.05606271,63.23509767)(816.02106274,63.36509754)(816.00105225,63.49510061)
\curveto(815.98106278,63.63509727)(815.95606281,63.77509713)(815.92605225,63.91510061)
\curveto(815.91606285,63.98509692)(815.91106285,64.05009686)(815.91105225,64.11010061)
\curveto(815.91106285,64.17009674)(815.90606286,64.23509667)(815.89605225,64.30510061)
\curveto(815.87606289,65.13509577)(816.02606274,65.8050951)(816.34605225,66.31510061)
\curveto(816.65606211,66.82509408)(817.09606167,67.2050937)(817.66605225,67.45510061)
\curveto(817.78606098,67.5050934)(817.91106085,67.55009336)(818.04105225,67.59010061)
\curveto(818.17106059,67.63009328)(818.30606046,67.67509323)(818.44605225,67.72510061)
\curveto(818.52606024,67.74509316)(818.61106015,67.76009315)(818.70105225,67.77010061)
\lineto(818.94105225,67.83010061)
\curveto(819.05105971,67.86009305)(819.1610596,67.87509303)(819.27105225,67.87510061)
\curveto(819.38105938,67.88509302)(819.49105927,67.90009301)(819.60105225,67.92010061)
\curveto(819.65105911,67.94009297)(819.69605907,67.94509296)(819.73605225,67.93510061)
\curveto(819.77605899,67.93509297)(819.81605895,67.94009297)(819.85605225,67.95010061)
\curveto(819.90605886,67.96009295)(819.9610588,67.96009295)(820.02105225,67.95010061)
\curveto(820.07105869,67.95009296)(820.12105864,67.95509295)(820.17105225,67.96510061)
\lineto(820.30605225,67.96510061)
\curveto(820.3660584,67.98509292)(820.43605833,67.98509292)(820.51605225,67.96510061)
\curveto(820.58605818,67.95509295)(820.65105811,67.96009295)(820.71105225,67.98010061)
\curveto(820.74105802,67.99009292)(820.78105798,67.99509291)(820.83105225,67.99510061)
\lineto(820.95105225,67.99510061)
\lineto(821.41605225,67.99510061)
\moveto(823.74105225,66.45010061)
\curveto(823.42105534,66.55009436)(823.05605571,66.6100943)(822.64605225,66.63010061)
\curveto(822.23605653,66.65009426)(821.82605694,66.66009425)(821.41605225,66.66010061)
\curveto(820.98605778,66.66009425)(820.5660582,66.65009426)(820.15605225,66.63010061)
\curveto(819.74605902,66.6100943)(819.3610594,66.56509434)(819.00105225,66.49510061)
\curveto(818.64106012,66.42509448)(818.32106044,66.31509459)(818.04105225,66.16510061)
\curveto(817.75106101,66.02509488)(817.51606125,65.83009508)(817.33605225,65.58010061)
\curveto(817.22606154,65.42009549)(817.14606162,65.24009567)(817.09605225,65.04010061)
\curveto(817.03606173,64.84009607)(817.00606176,64.59509631)(817.00605225,64.30510061)
\curveto(817.02606174,64.28509662)(817.03606173,64.25009666)(817.03605225,64.20010061)
\curveto(817.02606174,64.15009676)(817.02606174,64.1100968)(817.03605225,64.08010061)
\curveto(817.05606171,64.00009691)(817.07606169,63.92509698)(817.09605225,63.85510061)
\curveto(817.10606166,63.79509711)(817.12606164,63.73009718)(817.15605225,63.66010061)
\curveto(817.27606149,63.39009752)(817.44606132,63.17009774)(817.66605225,63.00010061)
\curveto(817.87606089,62.84009807)(818.12106064,62.7050982)(818.40105225,62.59510061)
\curveto(818.51106025,62.54509836)(818.63106013,62.5050984)(818.76105225,62.47510061)
\curveto(818.88105988,62.45509845)(819.00605976,62.43009848)(819.13605225,62.40010061)
\curveto(819.18605958,62.38009853)(819.24105952,62.37009854)(819.30105225,62.37010061)
\curveto(819.35105941,62.37009854)(819.40105936,62.36509854)(819.45105225,62.35510061)
\curveto(819.54105922,62.34509856)(819.63605913,62.33509857)(819.73605225,62.32510061)
\curveto(819.82605894,62.31509859)(819.92105884,62.3050986)(820.02105225,62.29510061)
\curveto(820.10105866,62.29509861)(820.18605858,62.29009862)(820.27605225,62.28010061)
\lineto(820.51605225,62.28010061)
\lineto(820.69605225,62.28010061)
\curveto(820.72605804,62.27009864)(820.761058,62.26509864)(820.80105225,62.26510061)
\lineto(820.93605225,62.26510061)
\lineto(821.38605225,62.26510061)
\curveto(821.4660573,62.26509864)(821.55105721,62.26009865)(821.64105225,62.25010061)
\curveto(821.72105704,62.25009866)(821.79605697,62.26009865)(821.86605225,62.28010061)
\lineto(822.13605225,62.28010061)
\curveto(822.15605661,62.28009863)(822.18605658,62.27509863)(822.22605225,62.26510061)
\curveto(822.25605651,62.26509864)(822.28105648,62.27009864)(822.30105225,62.28010061)
\curveto(822.40105636,62.29009862)(822.50105626,62.29509861)(822.60105225,62.29510061)
\curveto(822.69105607,62.3050986)(822.79105597,62.31509859)(822.90105225,62.32510061)
\curveto(823.02105574,62.35509855)(823.14605562,62.37009854)(823.27605225,62.37010061)
\curveto(823.39605537,62.38009853)(823.51105525,62.4050985)(823.62105225,62.44510061)
\curveto(823.92105484,62.52509838)(824.18605458,62.6100983)(824.41605225,62.70010061)
\curveto(824.64605412,62.80009811)(824.8610539,62.94509796)(825.06105225,63.13510061)
\curveto(825.2610535,63.34509756)(825.41105335,63.6100973)(825.51105225,63.93010061)
\curveto(825.53105323,63.97009694)(825.54105322,64.0050969)(825.54105225,64.03510061)
\curveto(825.53105323,64.07509683)(825.53605323,64.12009679)(825.55605225,64.17010061)
\curveto(825.5660532,64.2100967)(825.57605319,64.28009663)(825.58605225,64.38010061)
\curveto(825.59605317,64.49009642)(825.59105317,64.57509633)(825.57105225,64.63510061)
\curveto(825.55105321,64.7050962)(825.54105322,64.77509613)(825.54105225,64.84510061)
\curveto(825.53105323,64.91509599)(825.51605325,64.98009593)(825.49605225,65.04010061)
\curveto(825.43605333,65.24009567)(825.35105341,65.42009549)(825.24105225,65.58010061)
\curveto(825.22105354,65.6100953)(825.20105356,65.63509527)(825.18105225,65.65510061)
\lineto(825.12105225,65.71510061)
\curveto(825.10105366,65.75509515)(825.0610537,65.8050951)(825.00105225,65.86510061)
\curveto(824.8610539,65.96509494)(824.73105403,66.05009486)(824.61105225,66.12010061)
\curveto(824.49105427,66.19009472)(824.34605442,66.26009465)(824.17605225,66.33010061)
\curveto(824.10605466,66.36009455)(824.03605473,66.38009453)(823.96605225,66.39010061)
\curveto(823.89605487,66.4100945)(823.82105494,66.43009448)(823.74105225,66.45010061)
}
}
{
\newrgbcolor{curcolor}{0 0 0}
\pscustom[linestyle=none,fillstyle=solid,fillcolor=curcolor]
{
\newpath
\moveto(823.00605225,76.22470998)
\curveto(823.05605571,76.29470234)(823.12605564,76.3347023)(823.21605225,76.34470998)
\curveto(823.30605546,76.36470227)(823.41105535,76.37470226)(823.53105225,76.37470998)
\curveto(823.58105518,76.37470226)(823.63105513,76.36970226)(823.68105225,76.35970998)
\curveto(823.73105503,76.35970227)(823.77605499,76.34970228)(823.81605225,76.32970998)
\curveto(823.90605486,76.29970233)(823.9660548,76.23970239)(823.99605225,76.14970998)
\curveto(824.01605475,76.06970256)(824.02605474,75.97470266)(824.02605225,75.86470998)
\lineto(824.02605225,75.54970998)
\curveto(824.01605475,75.43970319)(824.02605474,75.3347033)(824.05605225,75.23470998)
\curveto(824.08605468,75.09470354)(824.1660546,75.00470363)(824.29605225,74.96470998)
\curveto(824.3660544,74.94470369)(824.45105431,74.9347037)(824.55105225,74.93470998)
\lineto(824.82105225,74.93470998)
\lineto(825.76605225,74.93470998)
\lineto(826.09605225,74.93470998)
\curveto(826.20605256,74.9347037)(826.29105247,74.91470372)(826.35105225,74.87470998)
\curveto(826.41105235,74.8347038)(826.45105231,74.78470385)(826.47105225,74.72470998)
\curveto(826.48105228,74.67470396)(826.49605227,74.60970402)(826.51605225,74.52970998)
\lineto(826.51605225,74.33470998)
\curveto(826.51605225,74.21470442)(826.51105225,74.10970452)(826.50105225,74.01970998)
\curveto(826.48105228,73.9297047)(826.43105233,73.85970477)(826.35105225,73.80970998)
\curveto(826.30105246,73.77970485)(826.23105253,73.76470487)(826.14105225,73.76470998)
\lineto(825.84105225,73.76470998)
\lineto(824.80605225,73.76470998)
\curveto(824.64605412,73.76470487)(824.50105426,73.75470488)(824.37105225,73.73470998)
\curveto(824.23105453,73.72470491)(824.13605463,73.66970496)(824.08605225,73.56970998)
\curveto(824.0660547,73.51970511)(824.05105471,73.44970518)(824.04105225,73.35970998)
\curveto(824.03105473,73.27970535)(824.02605474,73.18970544)(824.02605225,73.08970998)
\lineto(824.02605225,72.80470998)
\lineto(824.02605225,72.56470998)
\lineto(824.02605225,70.29970998)
\curveto(824.02605474,70.20970842)(824.03105473,70.10470853)(824.04105225,69.98470998)
\lineto(824.04105225,69.65470998)
\curveto(824.04105472,69.54470909)(824.03105473,69.44470919)(824.01105225,69.35470998)
\curveto(823.99105477,69.26470937)(823.95605481,69.20470943)(823.90605225,69.17470998)
\curveto(823.83605493,69.12470951)(823.74105502,69.09970953)(823.62105225,69.09970998)
\lineto(823.27605225,69.09970998)
\lineto(823.00605225,69.09970998)
\curveto(822.83605593,69.13970949)(822.69605607,69.19470944)(822.58605225,69.26470998)
\curveto(822.47605629,69.3347093)(822.3610564,69.41470922)(822.24105225,69.50470998)
\lineto(821.70105225,69.86470998)
\curveto(821.07105769,70.30470833)(820.45105831,70.73970789)(819.84105225,71.16970998)
\lineto(817.98105225,72.48970998)
\curveto(817.75106101,72.64970598)(817.53106123,72.80470583)(817.32105225,72.95470998)
\curveto(817.10106166,73.10470553)(816.87606189,73.25970537)(816.64605225,73.41970998)
\curveto(816.57606219,73.46970516)(816.51106225,73.51970511)(816.45105225,73.56970998)
\curveto(816.38106238,73.61970501)(816.30606246,73.66970496)(816.22605225,73.71970998)
\lineto(816.13605225,73.77970998)
\curveto(816.09606267,73.80970482)(816.0660627,73.83970479)(816.04605225,73.86970998)
\curveto(816.01606275,73.90970472)(815.99606277,73.94970468)(815.98605225,73.98970998)
\curveto(815.9660628,74.0297046)(815.94606282,74.07470456)(815.92605225,74.12470998)
\curveto(815.92606284,74.14470449)(815.93106283,74.16470447)(815.94105225,74.18470998)
\curveto(815.94106282,74.21470442)(815.93106283,74.23970439)(815.91105225,74.25970998)
\curveto(815.91106285,74.38970424)(815.91606285,74.50970412)(815.92605225,74.61970998)
\curveto(815.93606283,74.7297039)(815.98106278,74.80970382)(816.06105225,74.85970998)
\curveto(816.11106265,74.89970373)(816.18106258,74.91970371)(816.27105225,74.91970998)
\curveto(816.3610624,74.9297037)(816.45606231,74.9347037)(816.55605225,74.93470998)
\lineto(822.01605225,74.93470998)
\curveto(822.08605668,74.9347037)(822.1610566,74.9297037)(822.24105225,74.91970998)
\curveto(822.31105645,74.91970371)(822.38105638,74.92470371)(822.45105225,74.93470998)
\lineto(822.55605225,74.93470998)
\curveto(822.60605616,74.95470368)(822.6610561,74.96970366)(822.72105225,74.97970998)
\curveto(822.77105599,74.98970364)(822.81105595,75.01470362)(822.84105225,75.05470998)
\curveto(822.89105587,75.12470351)(822.92105584,75.20970342)(822.93105225,75.30970998)
\lineto(822.93105225,75.63970998)
\curveto(822.93105583,75.74970288)(822.93605583,75.85470278)(822.94605225,75.95470998)
\curveto(822.94605582,76.06470257)(822.9660558,76.15470248)(823.00605225,76.22470998)
\moveto(822.81105225,73.65970998)
\curveto(822.70105606,73.73970489)(822.53105623,73.77470486)(822.30105225,73.76470998)
\lineto(821.68605225,73.76470998)
\lineto(819.21105225,73.76470998)
\lineto(818.89605225,73.76470998)
\curveto(818.77605999,73.77470486)(818.67606009,73.76970486)(818.59605225,73.74970998)
\lineto(818.44605225,73.74970998)
\curveto(818.35606041,73.74970488)(818.27106049,73.7347049)(818.19105225,73.70470998)
\curveto(818.17106059,73.69470494)(818.1610606,73.68470495)(818.16105225,73.67470998)
\lineto(818.11605225,73.62970998)
\curveto(818.10606066,73.60970502)(818.10106066,73.57970505)(818.10105225,73.53970998)
\curveto(818.12106064,73.51970511)(818.13606063,73.49970513)(818.14605225,73.47970998)
\curveto(818.14606062,73.46970516)(818.15106061,73.45470518)(818.16105225,73.43470998)
\curveto(818.21106055,73.37470526)(818.28106048,73.31470532)(818.37105225,73.25470998)
\curveto(818.4610603,73.19470544)(818.54106022,73.13970549)(818.61105225,73.08970998)
\curveto(818.75106001,72.98970564)(818.89605987,72.89470574)(819.04605225,72.80470998)
\curveto(819.18605958,72.71470592)(819.32605944,72.61970601)(819.46605225,72.51970998)
\lineto(820.24605225,71.97970998)
\curveto(820.50605826,71.80970682)(820.766058,71.634707)(821.02605225,71.45470998)
\curveto(821.13605763,71.37470726)(821.24105752,71.29970733)(821.34105225,71.22970998)
\lineto(821.64105225,71.01970998)
\curveto(821.72105704,70.96970766)(821.79605697,70.91970771)(821.86605225,70.86970998)
\curveto(821.93605683,70.8297078)(822.01105675,70.78470785)(822.09105225,70.73470998)
\curveto(822.15105661,70.68470795)(822.21605655,70.634708)(822.28605225,70.58470998)
\curveto(822.34605642,70.54470809)(822.41605635,70.50470813)(822.49605225,70.46470998)
\curveto(822.55605621,70.42470821)(822.62605614,70.39970823)(822.70605225,70.38970998)
\curveto(822.77605599,70.37970825)(822.83105593,70.41470822)(822.87105225,70.49470998)
\curveto(822.92105584,70.56470807)(822.94605582,70.67470796)(822.94605225,70.82470998)
\curveto(822.93605583,70.98470765)(822.93105583,71.11970751)(822.93105225,71.22970998)
\lineto(822.93105225,72.90970998)
\lineto(822.93105225,73.34470998)
\curveto(822.93105583,73.49470514)(822.89105587,73.59970503)(822.81105225,73.65970998)
}
}
{
\newrgbcolor{curcolor}{0 0 0}
\pscustom[linestyle=none,fillstyle=solid,fillcolor=curcolor]
{
\newpath
\moveto(824.86605225,78.63431936)
\lineto(824.86605225,79.26431936)
\lineto(824.86605225,79.45931936)
\curveto(824.8660539,79.52931683)(824.87605389,79.58931677)(824.89605225,79.63931936)
\curveto(824.93605383,79.70931665)(824.97605379,79.7593166)(825.01605225,79.78931936)
\curveto(825.0660537,79.82931653)(825.13105363,79.84931651)(825.21105225,79.84931936)
\curveto(825.29105347,79.8593165)(825.37605339,79.86431649)(825.46605225,79.86431936)
\lineto(826.18605225,79.86431936)
\curveto(826.6660521,79.86431649)(827.07605169,79.80431655)(827.41605225,79.68431936)
\curveto(827.75605101,79.56431679)(828.03105073,79.36931699)(828.24105225,79.09931936)
\curveto(828.29105047,79.02931733)(828.33605043,78.9593174)(828.37605225,78.88931936)
\curveto(828.42605034,78.82931753)(828.47105029,78.7543176)(828.51105225,78.66431936)
\curveto(828.52105024,78.64431771)(828.53105023,78.61431774)(828.54105225,78.57431936)
\curveto(828.5610502,78.53431782)(828.5660502,78.48931787)(828.55605225,78.43931936)
\curveto(828.52605024,78.34931801)(828.45105031,78.29431806)(828.33105225,78.27431936)
\curveto(828.22105054,78.2543181)(828.12605064,78.26931809)(828.04605225,78.31931936)
\curveto(827.97605079,78.34931801)(827.91105085,78.39431796)(827.85105225,78.45431936)
\curveto(827.80105096,78.52431783)(827.75105101,78.58931777)(827.70105225,78.64931936)
\curveto(827.65105111,78.71931764)(827.57605119,78.77931758)(827.47605225,78.82931936)
\curveto(827.38605138,78.88931747)(827.29605147,78.93931742)(827.20605225,78.97931936)
\curveto(827.17605159,78.99931736)(827.11605165,79.02431733)(827.02605225,79.05431936)
\curveto(826.94605182,79.08431727)(826.87605189,79.08931727)(826.81605225,79.06931936)
\curveto(826.67605209,79.03931732)(826.58605218,78.97931738)(826.54605225,78.88931936)
\curveto(826.51605225,78.80931755)(826.50105226,78.71931764)(826.50105225,78.61931936)
\curveto(826.50105226,78.51931784)(826.47605229,78.43431792)(826.42605225,78.36431936)
\curveto(826.35605241,78.27431808)(826.21605255,78.22931813)(826.00605225,78.22931936)
\lineto(825.45105225,78.22931936)
\lineto(825.22605225,78.22931936)
\curveto(825.14605362,78.23931812)(825.08105368,78.2593181)(825.03105225,78.28931936)
\curveto(824.95105381,78.34931801)(824.90605386,78.41931794)(824.89605225,78.49931936)
\curveto(824.88605388,78.51931784)(824.88105388,78.53931782)(824.88105225,78.55931936)
\curveto(824.88105388,78.58931777)(824.87605389,78.61431774)(824.86605225,78.63431936)
}
}
{
\newrgbcolor{curcolor}{0 0 0}
\pscustom[linestyle=none,fillstyle=solid,fillcolor=curcolor]
{
}
}
{
\newrgbcolor{curcolor}{0 0 0}
\pscustom[linestyle=none,fillstyle=solid,fillcolor=curcolor]
{
\newpath
\moveto(815.89605225,89.26463186)
\curveto(815.88606288,89.95462722)(816.00606276,90.55462662)(816.25605225,91.06463186)
\curveto(816.50606226,91.58462559)(816.84106192,91.9796252)(817.26105225,92.24963186)
\curveto(817.34106142,92.29962488)(817.43106133,92.34462483)(817.53105225,92.38463186)
\curveto(817.62106114,92.42462475)(817.71606105,92.46962471)(817.81605225,92.51963186)
\curveto(817.91606085,92.55962462)(818.01606075,92.58962459)(818.11605225,92.60963186)
\curveto(818.21606055,92.62962455)(818.32106044,92.64962453)(818.43105225,92.66963186)
\curveto(818.48106028,92.68962449)(818.52606024,92.69462448)(818.56605225,92.68463186)
\curveto(818.60606016,92.6746245)(818.65106011,92.6796245)(818.70105225,92.69963186)
\curveto(818.75106001,92.70962447)(818.83605993,92.71462446)(818.95605225,92.71463186)
\curveto(819.0660597,92.71462446)(819.15105961,92.70962447)(819.21105225,92.69963186)
\curveto(819.27105949,92.6796245)(819.33105943,92.66962451)(819.39105225,92.66963186)
\curveto(819.45105931,92.6796245)(819.51105925,92.6746245)(819.57105225,92.65463186)
\curveto(819.71105905,92.61462456)(819.84605892,92.5796246)(819.97605225,92.54963186)
\curveto(820.10605866,92.51962466)(820.23105853,92.4796247)(820.35105225,92.42963186)
\curveto(820.49105827,92.36962481)(820.61605815,92.29962488)(820.72605225,92.21963186)
\curveto(820.83605793,92.14962503)(820.94605782,92.0746251)(821.05605225,91.99463186)
\lineto(821.11605225,91.93463186)
\curveto(821.13605763,91.92462525)(821.15605761,91.90962527)(821.17605225,91.88963186)
\curveto(821.33605743,91.76962541)(821.48105728,91.63462554)(821.61105225,91.48463186)
\curveto(821.74105702,91.33462584)(821.8660569,91.174626)(821.98605225,91.00463186)
\curveto(822.20605656,90.69462648)(822.41105635,90.39962678)(822.60105225,90.11963186)
\curveto(822.74105602,89.88962729)(822.87605589,89.65962752)(823.00605225,89.42963186)
\curveto(823.13605563,89.20962797)(823.27105549,88.98962819)(823.41105225,88.76963186)
\curveto(823.58105518,88.51962866)(823.761055,88.2796289)(823.95105225,88.04963186)
\curveto(824.14105462,87.82962935)(824.3660544,87.63962954)(824.62605225,87.47963186)
\curveto(824.68605408,87.43962974)(824.74605402,87.40462977)(824.80605225,87.37463186)
\curveto(824.85605391,87.34462983)(824.92105384,87.31462986)(825.00105225,87.28463186)
\curveto(825.07105369,87.26462991)(825.13105363,87.25962992)(825.18105225,87.26963186)
\curveto(825.25105351,87.28962989)(825.30605346,87.32462985)(825.34605225,87.37463186)
\curveto(825.37605339,87.42462975)(825.39605337,87.48462969)(825.40605225,87.55463186)
\lineto(825.40605225,87.79463186)
\lineto(825.40605225,88.54463186)
\lineto(825.40605225,91.34963186)
\lineto(825.40605225,92.00963186)
\curveto(825.40605336,92.09962508)(825.41105335,92.18462499)(825.42105225,92.26463186)
\curveto(825.42105334,92.34462483)(825.44105332,92.40962477)(825.48105225,92.45963186)
\curveto(825.52105324,92.50962467)(825.59605317,92.54962463)(825.70605225,92.57963186)
\curveto(825.80605296,92.61962456)(825.90605286,92.62962455)(826.00605225,92.60963186)
\lineto(826.14105225,92.60963186)
\curveto(826.21105255,92.58962459)(826.27105249,92.56962461)(826.32105225,92.54963186)
\curveto(826.37105239,92.52962465)(826.41105235,92.49462468)(826.44105225,92.44463186)
\curveto(826.48105228,92.39462478)(826.50105226,92.32462485)(826.50105225,92.23463186)
\lineto(826.50105225,91.96463186)
\lineto(826.50105225,91.06463186)
\lineto(826.50105225,87.55463186)
\lineto(826.50105225,86.48963186)
\curveto(826.50105226,86.40963077)(826.50605226,86.31963086)(826.51605225,86.21963186)
\curveto(826.51605225,86.11963106)(826.50605226,86.03463114)(826.48605225,85.96463186)
\curveto(826.41605235,85.75463142)(826.23605253,85.68963149)(825.94605225,85.76963186)
\curveto(825.90605286,85.7796314)(825.87105289,85.7796314)(825.84105225,85.76963186)
\curveto(825.80105296,85.76963141)(825.75605301,85.7796314)(825.70605225,85.79963186)
\curveto(825.62605314,85.81963136)(825.54105322,85.83963134)(825.45105225,85.85963186)
\curveto(825.3610534,85.8796313)(825.27605349,85.90463127)(825.19605225,85.93463186)
\curveto(824.70605406,86.09463108)(824.29105447,86.29463088)(823.95105225,86.53463186)
\curveto(823.70105506,86.71463046)(823.47605529,86.91963026)(823.27605225,87.14963186)
\curveto(823.0660557,87.3796298)(822.87105589,87.61962956)(822.69105225,87.86963186)
\curveto(822.51105625,88.12962905)(822.34105642,88.39462878)(822.18105225,88.66463186)
\curveto(822.01105675,88.94462823)(821.83605693,89.21462796)(821.65605225,89.47463186)
\curveto(821.57605719,89.58462759)(821.50105726,89.68962749)(821.43105225,89.78963186)
\curveto(821.3610574,89.89962728)(821.28605748,90.00962717)(821.20605225,90.11963186)
\curveto(821.17605759,90.15962702)(821.14605762,90.19462698)(821.11605225,90.22463186)
\curveto(821.07605769,90.26462691)(821.04605772,90.30462687)(821.02605225,90.34463186)
\curveto(820.91605785,90.48462669)(820.79105797,90.60962657)(820.65105225,90.71963186)
\curveto(820.62105814,90.73962644)(820.59605817,90.76462641)(820.57605225,90.79463186)
\curveto(820.54605822,90.82462635)(820.51605825,90.84962633)(820.48605225,90.86963186)
\curveto(820.38605838,90.94962623)(820.28605848,91.01462616)(820.18605225,91.06463186)
\curveto(820.08605868,91.12462605)(819.97605879,91.179626)(819.85605225,91.22963186)
\curveto(819.78605898,91.25962592)(819.71105905,91.2796259)(819.63105225,91.28963186)
\lineto(819.39105225,91.34963186)
\lineto(819.30105225,91.34963186)
\curveto(819.27105949,91.35962582)(819.24105952,91.36462581)(819.21105225,91.36463186)
\curveto(819.14105962,91.38462579)(819.04605972,91.38962579)(818.92605225,91.37963186)
\curveto(818.79605997,91.3796258)(818.69606007,91.36962581)(818.62605225,91.34963186)
\curveto(818.54606022,91.32962585)(818.47106029,91.30962587)(818.40105225,91.28963186)
\curveto(818.32106044,91.2796259)(818.24106052,91.25962592)(818.16105225,91.22963186)
\curveto(817.92106084,91.11962606)(817.72106104,90.96962621)(817.56105225,90.77963186)
\curveto(817.39106137,90.59962658)(817.25106151,90.3796268)(817.14105225,90.11963186)
\curveto(817.12106164,90.04962713)(817.10606166,89.9796272)(817.09605225,89.90963186)
\curveto(817.07606169,89.83962734)(817.05606171,89.76462741)(817.03605225,89.68463186)
\curveto(817.01606175,89.60462757)(817.00606176,89.49462768)(817.00605225,89.35463186)
\curveto(817.00606176,89.22462795)(817.01606175,89.11962806)(817.03605225,89.03963186)
\curveto(817.04606172,88.9796282)(817.05106171,88.92462825)(817.05105225,88.87463186)
\curveto(817.05106171,88.82462835)(817.0610617,88.7746284)(817.08105225,88.72463186)
\curveto(817.12106164,88.62462855)(817.1610616,88.52962865)(817.20105225,88.43963186)
\curveto(817.24106152,88.35962882)(817.28606148,88.2796289)(817.33605225,88.19963186)
\curveto(817.35606141,88.16962901)(817.38106138,88.13962904)(817.41105225,88.10963186)
\curveto(817.44106132,88.08962909)(817.4660613,88.06462911)(817.48605225,88.03463186)
\lineto(817.56105225,87.95963186)
\curveto(817.58106118,87.92962925)(817.60106116,87.90462927)(817.62105225,87.88463186)
\lineto(817.83105225,87.73463186)
\curveto(817.89106087,87.69462948)(817.95606081,87.64962953)(818.02605225,87.59963186)
\curveto(818.11606065,87.53962964)(818.22106054,87.48962969)(818.34105225,87.44963186)
\curveto(818.45106031,87.41962976)(818.5610602,87.38462979)(818.67105225,87.34463186)
\curveto(818.78105998,87.30462987)(818.92605984,87.2796299)(819.10605225,87.26963186)
\curveto(819.27605949,87.25962992)(819.40105936,87.22962995)(819.48105225,87.17963186)
\curveto(819.5610592,87.12963005)(819.60605916,87.05463012)(819.61605225,86.95463186)
\curveto(819.62605914,86.85463032)(819.63105913,86.74463043)(819.63105225,86.62463186)
\curveto(819.63105913,86.58463059)(819.63605913,86.54463063)(819.64605225,86.50463186)
\curveto(819.64605912,86.46463071)(819.64105912,86.42963075)(819.63105225,86.39963186)
\curveto(819.61105915,86.34963083)(819.60105916,86.29963088)(819.60105225,86.24963186)
\curveto(819.60105916,86.20963097)(819.59105917,86.16963101)(819.57105225,86.12963186)
\curveto(819.51105925,86.03963114)(819.37605939,85.99463118)(819.16605225,85.99463186)
\lineto(819.04605225,85.99463186)
\curveto(818.98605978,86.00463117)(818.92605984,86.00963117)(818.86605225,86.00963186)
\curveto(818.79605997,86.01963116)(818.73106003,86.02963115)(818.67105225,86.03963186)
\curveto(818.5610602,86.05963112)(818.4610603,86.0796311)(818.37105225,86.09963186)
\curveto(818.27106049,86.11963106)(818.17606059,86.14963103)(818.08605225,86.18963186)
\curveto(818.01606075,86.20963097)(817.95606081,86.22963095)(817.90605225,86.24963186)
\lineto(817.72605225,86.30963186)
\curveto(817.4660613,86.42963075)(817.22106154,86.58463059)(816.99105225,86.77463186)
\curveto(816.761062,86.9746302)(816.57606219,87.18962999)(816.43605225,87.41963186)
\curveto(816.35606241,87.52962965)(816.29106247,87.64462953)(816.24105225,87.76463186)
\lineto(816.09105225,88.15463186)
\curveto(816.04106272,88.26462891)(816.01106275,88.3796288)(816.00105225,88.49963186)
\curveto(815.98106278,88.61962856)(815.95606281,88.74462843)(815.92605225,88.87463186)
\curveto(815.92606284,88.94462823)(815.92606284,89.00962817)(815.92605225,89.06963186)
\curveto(815.91606285,89.12962805)(815.90606286,89.19462798)(815.89605225,89.26463186)
}
}
{
\newrgbcolor{curcolor}{0 0 0}
\pscustom[linestyle=none,fillstyle=solid,fillcolor=curcolor]
{
\newpath
\moveto(821.41605225,101.36424123)
\lineto(821.67105225,101.36424123)
\curveto(821.75105701,101.37423353)(821.82605694,101.36923353)(821.89605225,101.34924123)
\lineto(822.13605225,101.34924123)
\lineto(822.30105225,101.34924123)
\curveto(822.40105636,101.32923357)(822.50605626,101.31923358)(822.61605225,101.31924123)
\curveto(822.71605605,101.31923358)(822.81605595,101.30923359)(822.91605225,101.28924123)
\lineto(823.06605225,101.28924123)
\curveto(823.20605556,101.25923364)(823.34605542,101.23923366)(823.48605225,101.22924123)
\curveto(823.61605515,101.21923368)(823.74605502,101.19423371)(823.87605225,101.15424123)
\curveto(823.95605481,101.13423377)(824.04105472,101.11423379)(824.13105225,101.09424123)
\lineto(824.37105225,101.03424123)
\lineto(824.67105225,100.91424123)
\curveto(824.761054,100.88423402)(824.85105391,100.84923405)(824.94105225,100.80924123)
\curveto(825.1610536,100.70923419)(825.37605339,100.57423433)(825.58605225,100.40424123)
\curveto(825.79605297,100.24423466)(825.9660528,100.06923483)(826.09605225,99.87924123)
\curveto(826.13605263,99.82923507)(826.17605259,99.76923513)(826.21605225,99.69924123)
\curveto(826.24605252,99.63923526)(826.28105248,99.57923532)(826.32105225,99.51924123)
\curveto(826.37105239,99.43923546)(826.41105235,99.34423556)(826.44105225,99.23424123)
\curveto(826.47105229,99.12423578)(826.50105226,99.01923588)(826.53105225,98.91924123)
\curveto(826.57105219,98.80923609)(826.59605217,98.6992362)(826.60605225,98.58924123)
\curveto(826.61605215,98.47923642)(826.63105213,98.36423654)(826.65105225,98.24424123)
\curveto(826.6610521,98.2042367)(826.6610521,98.15923674)(826.65105225,98.10924123)
\curveto(826.65105211,98.06923683)(826.65605211,98.02923687)(826.66605225,97.98924123)
\curveto(826.67605209,97.94923695)(826.68105208,97.89423701)(826.68105225,97.82424123)
\curveto(826.68105208,97.75423715)(826.67605209,97.7042372)(826.66605225,97.67424123)
\curveto(826.64605212,97.62423728)(826.64105212,97.57923732)(826.65105225,97.53924123)
\curveto(826.6610521,97.4992374)(826.6610521,97.46423744)(826.65105225,97.43424123)
\lineto(826.65105225,97.34424123)
\curveto(826.63105213,97.28423762)(826.61605215,97.21923768)(826.60605225,97.14924123)
\curveto(826.60605216,97.08923781)(826.60105216,97.02423788)(826.59105225,96.95424123)
\curveto(826.54105222,96.78423812)(826.49105227,96.62423828)(826.44105225,96.47424123)
\curveto(826.39105237,96.32423858)(826.32605244,96.17923872)(826.24605225,96.03924123)
\curveto(826.20605256,95.98923891)(826.17605259,95.93423897)(826.15605225,95.87424123)
\curveto(826.12605264,95.82423908)(826.09105267,95.77423913)(826.05105225,95.72424123)
\curveto(825.87105289,95.48423942)(825.65105311,95.28423962)(825.39105225,95.12424123)
\curveto(825.13105363,94.96423994)(824.84605392,94.82424008)(824.53605225,94.70424123)
\curveto(824.39605437,94.64424026)(824.25605451,94.5992403)(824.11605225,94.56924123)
\curveto(823.9660548,94.53924036)(823.81105495,94.5042404)(823.65105225,94.46424123)
\curveto(823.54105522,94.44424046)(823.43105533,94.42924047)(823.32105225,94.41924123)
\curveto(823.21105555,94.40924049)(823.10105566,94.39424051)(822.99105225,94.37424123)
\curveto(822.95105581,94.36424054)(822.91105585,94.35924054)(822.87105225,94.35924123)
\curveto(822.83105593,94.36924053)(822.79105597,94.36924053)(822.75105225,94.35924123)
\curveto(822.70105606,94.34924055)(822.65105611,94.34424056)(822.60105225,94.34424123)
\lineto(822.43605225,94.34424123)
\curveto(822.38605638,94.32424058)(822.33605643,94.31924058)(822.28605225,94.32924123)
\curveto(822.22605654,94.33924056)(822.17105659,94.33924056)(822.12105225,94.32924123)
\curveto(822.08105668,94.31924058)(822.03605673,94.31924058)(821.98605225,94.32924123)
\curveto(821.93605683,94.33924056)(821.88605688,94.33424057)(821.83605225,94.31424123)
\curveto(821.766057,94.29424061)(821.69105707,94.28924061)(821.61105225,94.29924123)
\curveto(821.52105724,94.30924059)(821.43605733,94.31424059)(821.35605225,94.31424123)
\curveto(821.2660575,94.31424059)(821.1660576,94.30924059)(821.05605225,94.29924123)
\curveto(820.93605783,94.28924061)(820.83605793,94.29424061)(820.75605225,94.31424123)
\lineto(820.47105225,94.31424123)
\lineto(819.84105225,94.35924123)
\curveto(819.74105902,94.36924053)(819.64605912,94.37924052)(819.55605225,94.38924123)
\lineto(819.25605225,94.41924123)
\curveto(819.20605956,94.43924046)(819.15605961,94.44424046)(819.10605225,94.43424123)
\curveto(819.04605972,94.43424047)(818.99105977,94.44424046)(818.94105225,94.46424123)
\curveto(818.77105999,94.51424039)(818.60606016,94.55424035)(818.44605225,94.58424123)
\curveto(818.27606049,94.61424029)(818.11606065,94.66424024)(817.96605225,94.73424123)
\curveto(817.50606126,94.92423998)(817.13106163,95.14423976)(816.84105225,95.39424123)
\curveto(816.55106221,95.65423925)(816.30606246,96.01423889)(816.10605225,96.47424123)
\curveto(816.05606271,96.6042383)(816.02106274,96.73423817)(816.00105225,96.86424123)
\curveto(815.98106278,97.0042379)(815.95606281,97.14423776)(815.92605225,97.28424123)
\curveto(815.91606285,97.35423755)(815.91106285,97.41923748)(815.91105225,97.47924123)
\curveto(815.91106285,97.53923736)(815.90606286,97.6042373)(815.89605225,97.67424123)
\curveto(815.87606289,98.5042364)(816.02606274,99.17423573)(816.34605225,99.68424123)
\curveto(816.65606211,100.19423471)(817.09606167,100.57423433)(817.66605225,100.82424123)
\curveto(817.78606098,100.87423403)(817.91106085,100.91923398)(818.04105225,100.95924123)
\curveto(818.17106059,100.9992339)(818.30606046,101.04423386)(818.44605225,101.09424123)
\curveto(818.52606024,101.11423379)(818.61106015,101.12923377)(818.70105225,101.13924123)
\lineto(818.94105225,101.19924123)
\curveto(819.05105971,101.22923367)(819.1610596,101.24423366)(819.27105225,101.24424123)
\curveto(819.38105938,101.25423365)(819.49105927,101.26923363)(819.60105225,101.28924123)
\curveto(819.65105911,101.30923359)(819.69605907,101.31423359)(819.73605225,101.30424123)
\curveto(819.77605899,101.3042336)(819.81605895,101.30923359)(819.85605225,101.31924123)
\curveto(819.90605886,101.32923357)(819.9610588,101.32923357)(820.02105225,101.31924123)
\curveto(820.07105869,101.31923358)(820.12105864,101.32423358)(820.17105225,101.33424123)
\lineto(820.30605225,101.33424123)
\curveto(820.3660584,101.35423355)(820.43605833,101.35423355)(820.51605225,101.33424123)
\curveto(820.58605818,101.32423358)(820.65105811,101.32923357)(820.71105225,101.34924123)
\curveto(820.74105802,101.35923354)(820.78105798,101.36423354)(820.83105225,101.36424123)
\lineto(820.95105225,101.36424123)
\lineto(821.41605225,101.36424123)
\moveto(823.74105225,99.81924123)
\curveto(823.42105534,99.91923498)(823.05605571,99.97923492)(822.64605225,99.99924123)
\curveto(822.23605653,100.01923488)(821.82605694,100.02923487)(821.41605225,100.02924123)
\curveto(820.98605778,100.02923487)(820.5660582,100.01923488)(820.15605225,99.99924123)
\curveto(819.74605902,99.97923492)(819.3610594,99.93423497)(819.00105225,99.86424123)
\curveto(818.64106012,99.79423511)(818.32106044,99.68423522)(818.04105225,99.53424123)
\curveto(817.75106101,99.39423551)(817.51606125,99.1992357)(817.33605225,98.94924123)
\curveto(817.22606154,98.78923611)(817.14606162,98.60923629)(817.09605225,98.40924123)
\curveto(817.03606173,98.20923669)(817.00606176,97.96423694)(817.00605225,97.67424123)
\curveto(817.02606174,97.65423725)(817.03606173,97.61923728)(817.03605225,97.56924123)
\curveto(817.02606174,97.51923738)(817.02606174,97.47923742)(817.03605225,97.44924123)
\curveto(817.05606171,97.36923753)(817.07606169,97.29423761)(817.09605225,97.22424123)
\curveto(817.10606166,97.16423774)(817.12606164,97.0992378)(817.15605225,97.02924123)
\curveto(817.27606149,96.75923814)(817.44606132,96.53923836)(817.66605225,96.36924123)
\curveto(817.87606089,96.20923869)(818.12106064,96.07423883)(818.40105225,95.96424123)
\curveto(818.51106025,95.91423899)(818.63106013,95.87423903)(818.76105225,95.84424123)
\curveto(818.88105988,95.82423908)(819.00605976,95.7992391)(819.13605225,95.76924123)
\curveto(819.18605958,95.74923915)(819.24105952,95.73923916)(819.30105225,95.73924123)
\curveto(819.35105941,95.73923916)(819.40105936,95.73423917)(819.45105225,95.72424123)
\curveto(819.54105922,95.71423919)(819.63605913,95.7042392)(819.73605225,95.69424123)
\curveto(819.82605894,95.68423922)(819.92105884,95.67423923)(820.02105225,95.66424123)
\curveto(820.10105866,95.66423924)(820.18605858,95.65923924)(820.27605225,95.64924123)
\lineto(820.51605225,95.64924123)
\lineto(820.69605225,95.64924123)
\curveto(820.72605804,95.63923926)(820.761058,95.63423927)(820.80105225,95.63424123)
\lineto(820.93605225,95.63424123)
\lineto(821.38605225,95.63424123)
\curveto(821.4660573,95.63423927)(821.55105721,95.62923927)(821.64105225,95.61924123)
\curveto(821.72105704,95.61923928)(821.79605697,95.62923927)(821.86605225,95.64924123)
\lineto(822.13605225,95.64924123)
\curveto(822.15605661,95.64923925)(822.18605658,95.64423926)(822.22605225,95.63424123)
\curveto(822.25605651,95.63423927)(822.28105648,95.63923926)(822.30105225,95.64924123)
\curveto(822.40105636,95.65923924)(822.50105626,95.66423924)(822.60105225,95.66424123)
\curveto(822.69105607,95.67423923)(822.79105597,95.68423922)(822.90105225,95.69424123)
\curveto(823.02105574,95.72423918)(823.14605562,95.73923916)(823.27605225,95.73924123)
\curveto(823.39605537,95.74923915)(823.51105525,95.77423913)(823.62105225,95.81424123)
\curveto(823.92105484,95.89423901)(824.18605458,95.97923892)(824.41605225,96.06924123)
\curveto(824.64605412,96.16923873)(824.8610539,96.31423859)(825.06105225,96.50424123)
\curveto(825.2610535,96.71423819)(825.41105335,96.97923792)(825.51105225,97.29924123)
\curveto(825.53105323,97.33923756)(825.54105322,97.37423753)(825.54105225,97.40424123)
\curveto(825.53105323,97.44423746)(825.53605323,97.48923741)(825.55605225,97.53924123)
\curveto(825.5660532,97.57923732)(825.57605319,97.64923725)(825.58605225,97.74924123)
\curveto(825.59605317,97.85923704)(825.59105317,97.94423696)(825.57105225,98.00424123)
\curveto(825.55105321,98.07423683)(825.54105322,98.14423676)(825.54105225,98.21424123)
\curveto(825.53105323,98.28423662)(825.51605325,98.34923655)(825.49605225,98.40924123)
\curveto(825.43605333,98.60923629)(825.35105341,98.78923611)(825.24105225,98.94924123)
\curveto(825.22105354,98.97923592)(825.20105356,99.0042359)(825.18105225,99.02424123)
\lineto(825.12105225,99.08424123)
\curveto(825.10105366,99.12423578)(825.0610537,99.17423573)(825.00105225,99.23424123)
\curveto(824.8610539,99.33423557)(824.73105403,99.41923548)(824.61105225,99.48924123)
\curveto(824.49105427,99.55923534)(824.34605442,99.62923527)(824.17605225,99.69924123)
\curveto(824.10605466,99.72923517)(824.03605473,99.74923515)(823.96605225,99.75924123)
\curveto(823.89605487,99.77923512)(823.82105494,99.7992351)(823.74105225,99.81924123)
}
}
{
\newrgbcolor{curcolor}{0 0 0}
\pscustom[linestyle=none,fillstyle=solid,fillcolor=curcolor]
{
\newpath
\moveto(815.89605225,106.77385061)
\curveto(815.89606287,106.87384575)(815.90606286,106.96884566)(815.92605225,107.05885061)
\curveto(815.93606283,107.14884548)(815.9660628,107.21384541)(816.01605225,107.25385061)
\curveto(816.09606267,107.31384531)(816.20106256,107.34384528)(816.33105225,107.34385061)
\lineto(816.72105225,107.34385061)
\lineto(818.22105225,107.34385061)
\lineto(824.61105225,107.34385061)
\lineto(825.78105225,107.34385061)
\lineto(826.09605225,107.34385061)
\curveto(826.19605257,107.35384527)(826.27605249,107.33884529)(826.33605225,107.29885061)
\curveto(826.41605235,107.24884538)(826.4660523,107.17384545)(826.48605225,107.07385061)
\curveto(826.49605227,106.98384564)(826.50105226,106.87384575)(826.50105225,106.74385061)
\lineto(826.50105225,106.51885061)
\curveto(826.48105228,106.43884619)(826.4660523,106.36884626)(826.45605225,106.30885061)
\curveto(826.43605233,106.24884638)(826.39605237,106.19884643)(826.33605225,106.15885061)
\curveto(826.27605249,106.11884651)(826.20105256,106.09884653)(826.11105225,106.09885061)
\lineto(825.81105225,106.09885061)
\lineto(824.71605225,106.09885061)
\lineto(819.37605225,106.09885061)
\curveto(819.28605948,106.07884655)(819.21105955,106.06384656)(819.15105225,106.05385061)
\curveto(819.08105968,106.05384657)(819.02105974,106.0238466)(818.97105225,105.96385061)
\curveto(818.92105984,105.89384673)(818.89605987,105.80384682)(818.89605225,105.69385061)
\curveto(818.88605988,105.59384703)(818.88105988,105.48384714)(818.88105225,105.36385061)
\lineto(818.88105225,104.22385061)
\lineto(818.88105225,103.72885061)
\curveto(818.87105989,103.56884906)(818.81105995,103.45884917)(818.70105225,103.39885061)
\curveto(818.67106009,103.37884925)(818.64106012,103.36884926)(818.61105225,103.36885061)
\curveto(818.57106019,103.36884926)(818.52606024,103.36384926)(818.47605225,103.35385061)
\curveto(818.35606041,103.33384929)(818.24606052,103.33884929)(818.14605225,103.36885061)
\curveto(818.04606072,103.40884922)(817.97606079,103.46384916)(817.93605225,103.53385061)
\curveto(817.88606088,103.61384901)(817.8610609,103.73384889)(817.86105225,103.89385061)
\curveto(817.8610609,104.05384857)(817.84606092,104.18884844)(817.81605225,104.29885061)
\curveto(817.80606096,104.34884828)(817.80106096,104.40384822)(817.80105225,104.46385061)
\curveto(817.79106097,104.5238481)(817.77606099,104.58384804)(817.75605225,104.64385061)
\curveto(817.70606106,104.79384783)(817.65606111,104.93884769)(817.60605225,105.07885061)
\curveto(817.54606122,105.21884741)(817.47606129,105.35384727)(817.39605225,105.48385061)
\curveto(817.30606146,105.623847)(817.20106156,105.74384688)(817.08105225,105.84385061)
\curveto(816.9610618,105.94384668)(816.83106193,106.03884659)(816.69105225,106.12885061)
\curveto(816.59106217,106.18884644)(816.48106228,106.23384639)(816.36105225,106.26385061)
\curveto(816.24106252,106.30384632)(816.13606263,106.35384627)(816.04605225,106.41385061)
\curveto(815.98606278,106.46384616)(815.94606282,106.53384609)(815.92605225,106.62385061)
\curveto(815.91606285,106.64384598)(815.91106285,106.66884596)(815.91105225,106.69885061)
\curveto(815.91106285,106.7288459)(815.90606286,106.75384587)(815.89605225,106.77385061)
}
}
{
\newrgbcolor{curcolor}{0 0 0}
\pscustom[linestyle=none,fillstyle=solid,fillcolor=curcolor]
{
\newpath
\moveto(815.89605225,115.12345998)
\curveto(815.89606287,115.22345513)(815.90606286,115.31845503)(815.92605225,115.40845998)
\curveto(815.93606283,115.49845485)(815.9660628,115.56345479)(816.01605225,115.60345998)
\curveto(816.09606267,115.66345469)(816.20106256,115.69345466)(816.33105225,115.69345998)
\lineto(816.72105225,115.69345998)
\lineto(818.22105225,115.69345998)
\lineto(824.61105225,115.69345998)
\lineto(825.78105225,115.69345998)
\lineto(826.09605225,115.69345998)
\curveto(826.19605257,115.70345465)(826.27605249,115.68845466)(826.33605225,115.64845998)
\curveto(826.41605235,115.59845475)(826.4660523,115.52345483)(826.48605225,115.42345998)
\curveto(826.49605227,115.33345502)(826.50105226,115.22345513)(826.50105225,115.09345998)
\lineto(826.50105225,114.86845998)
\curveto(826.48105228,114.78845556)(826.4660523,114.71845563)(826.45605225,114.65845998)
\curveto(826.43605233,114.59845575)(826.39605237,114.5484558)(826.33605225,114.50845998)
\curveto(826.27605249,114.46845588)(826.20105256,114.4484559)(826.11105225,114.44845998)
\lineto(825.81105225,114.44845998)
\lineto(824.71605225,114.44845998)
\lineto(819.37605225,114.44845998)
\curveto(819.28605948,114.42845592)(819.21105955,114.41345594)(819.15105225,114.40345998)
\curveto(819.08105968,114.40345595)(819.02105974,114.37345598)(818.97105225,114.31345998)
\curveto(818.92105984,114.24345611)(818.89605987,114.1534562)(818.89605225,114.04345998)
\curveto(818.88605988,113.94345641)(818.88105988,113.83345652)(818.88105225,113.71345998)
\lineto(818.88105225,112.57345998)
\lineto(818.88105225,112.07845998)
\curveto(818.87105989,111.91845843)(818.81105995,111.80845854)(818.70105225,111.74845998)
\curveto(818.67106009,111.72845862)(818.64106012,111.71845863)(818.61105225,111.71845998)
\curveto(818.57106019,111.71845863)(818.52606024,111.71345864)(818.47605225,111.70345998)
\curveto(818.35606041,111.68345867)(818.24606052,111.68845866)(818.14605225,111.71845998)
\curveto(818.04606072,111.75845859)(817.97606079,111.81345854)(817.93605225,111.88345998)
\curveto(817.88606088,111.96345839)(817.8610609,112.08345827)(817.86105225,112.24345998)
\curveto(817.8610609,112.40345795)(817.84606092,112.53845781)(817.81605225,112.64845998)
\curveto(817.80606096,112.69845765)(817.80106096,112.7534576)(817.80105225,112.81345998)
\curveto(817.79106097,112.87345748)(817.77606099,112.93345742)(817.75605225,112.99345998)
\curveto(817.70606106,113.14345721)(817.65606111,113.28845706)(817.60605225,113.42845998)
\curveto(817.54606122,113.56845678)(817.47606129,113.70345665)(817.39605225,113.83345998)
\curveto(817.30606146,113.97345638)(817.20106156,114.09345626)(817.08105225,114.19345998)
\curveto(816.9610618,114.29345606)(816.83106193,114.38845596)(816.69105225,114.47845998)
\curveto(816.59106217,114.53845581)(816.48106228,114.58345577)(816.36105225,114.61345998)
\curveto(816.24106252,114.6534557)(816.13606263,114.70345565)(816.04605225,114.76345998)
\curveto(815.98606278,114.81345554)(815.94606282,114.88345547)(815.92605225,114.97345998)
\curveto(815.91606285,114.99345536)(815.91106285,115.01845533)(815.91105225,115.04845998)
\curveto(815.91106285,115.07845527)(815.90606286,115.10345525)(815.89605225,115.12345998)
}
}
{
\newrgbcolor{curcolor}{0 0 0}
\pscustom[linestyle=none,fillstyle=solid,fillcolor=curcolor]
{
\newpath
\moveto(836.73236816,42.29681936)
\curveto(836.73237886,42.36681368)(836.73237886,42.4468136)(836.73236816,42.53681936)
\curveto(836.72237887,42.62681342)(836.72237887,42.71181333)(836.73236816,42.79181936)
\curveto(836.73237886,42.88181316)(836.74237885,42.96181308)(836.76236816,43.03181936)
\curveto(836.78237881,43.11181293)(836.81237878,43.16681288)(836.85236816,43.19681936)
\curveto(836.90237869,43.22681282)(836.97737861,43.2468128)(837.07736816,43.25681936)
\curveto(837.16737842,43.27681277)(837.27237832,43.28681276)(837.39236816,43.28681936)
\curveto(837.50237809,43.29681275)(837.61737797,43.29681275)(837.73736816,43.28681936)
\lineto(838.03736816,43.28681936)
\lineto(841.05236816,43.28681936)
\lineto(843.94736816,43.28681936)
\curveto(844.27737131,43.28681276)(844.60237099,43.28181276)(844.92236816,43.27181936)
\curveto(845.23237036,43.27181277)(845.51237008,43.23181281)(845.76236816,43.15181936)
\curveto(846.11236948,43.03181301)(846.40736918,42.87681317)(846.64736816,42.68681936)
\curveto(846.87736871,42.49681355)(847.07736851,42.25681379)(847.24736816,41.96681936)
\curveto(847.29736829,41.90681414)(847.33236826,41.8418142)(847.35236816,41.77181936)
\curveto(847.37236822,41.71181433)(847.39736819,41.6418144)(847.42736816,41.56181936)
\curveto(847.47736811,41.4418146)(847.51236808,41.31181473)(847.53236816,41.17181936)
\curveto(847.56236803,41.041815)(847.592368,40.90681514)(847.62236816,40.76681936)
\curveto(847.64236795,40.71681533)(847.64736794,40.66681538)(847.63736816,40.61681936)
\curveto(847.62736796,40.56681548)(847.62736796,40.51181553)(847.63736816,40.45181936)
\curveto(847.64736794,40.43181561)(847.64736794,40.40681564)(847.63736816,40.37681936)
\curveto(847.63736795,40.3468157)(847.64236795,40.32181572)(847.65236816,40.30181936)
\curveto(847.66236793,40.26181578)(847.66736792,40.20681584)(847.66736816,40.13681936)
\curveto(847.66736792,40.06681598)(847.66236793,40.01181603)(847.65236816,39.97181936)
\curveto(847.64236795,39.92181612)(847.64236795,39.86681618)(847.65236816,39.80681936)
\curveto(847.66236793,39.7468163)(847.65736793,39.69181635)(847.63736816,39.64181936)
\curveto(847.60736798,39.51181653)(847.587368,39.38681666)(847.57736816,39.26681936)
\curveto(847.56736802,39.1468169)(847.54236805,39.03181701)(847.50236816,38.92181936)
\curveto(847.38236821,38.55181749)(847.21236838,38.23181781)(846.99236816,37.96181936)
\curveto(846.77236882,37.69181835)(846.4923691,37.48181856)(846.15236816,37.33181936)
\curveto(846.03236956,37.28181876)(845.90736968,37.23681881)(845.77736816,37.19681936)
\curveto(845.64736994,37.16681888)(845.51237008,37.13181891)(845.37236816,37.09181936)
\curveto(845.32237027,37.08181896)(845.28237031,37.07681897)(845.25236816,37.07681936)
\curveto(845.21237038,37.07681897)(845.16737042,37.07181897)(845.11736816,37.06181936)
\curveto(845.0873705,37.05181899)(845.05237054,37.046819)(845.01236816,37.04681936)
\curveto(844.96237063,37.046819)(844.92237067,37.041819)(844.89236816,37.03181936)
\lineto(844.72736816,37.03181936)
\curveto(844.64737094,37.01181903)(844.54737104,37.00681904)(844.42736816,37.01681936)
\curveto(844.29737129,37.02681902)(844.20737138,37.041819)(844.15736816,37.06181936)
\curveto(844.06737152,37.08181896)(844.00237159,37.13681891)(843.96236816,37.22681936)
\curveto(843.94237165,37.25681879)(843.93737165,37.28681876)(843.94736816,37.31681936)
\curveto(843.94737164,37.3468187)(843.94237165,37.38681866)(843.93236816,37.43681936)
\curveto(843.92237167,37.47681857)(843.91737167,37.51681853)(843.91736816,37.55681936)
\lineto(843.91736816,37.70681936)
\curveto(843.91737167,37.82681822)(843.92237167,37.9468181)(843.93236816,38.06681936)
\curveto(843.93237166,38.19681785)(843.96737162,38.28681776)(844.03736816,38.33681936)
\curveto(844.09737149,38.37681767)(844.15737143,38.39681765)(844.21736816,38.39681936)
\curveto(844.27737131,38.39681765)(844.34737124,38.40681764)(844.42736816,38.42681936)
\curveto(844.45737113,38.43681761)(844.4923711,38.43681761)(844.53236816,38.42681936)
\curveto(844.56237103,38.42681762)(844.587371,38.43181761)(844.60736816,38.44181936)
\lineto(844.81736816,38.44181936)
\curveto(844.86737072,38.46181758)(844.91737067,38.46681758)(844.96736816,38.45681936)
\curveto(845.00737058,38.45681759)(845.05237054,38.46681758)(845.10236816,38.48681936)
\curveto(845.23237036,38.51681753)(845.35737023,38.5468175)(845.47736816,38.57681936)
\curveto(845.58737,38.60681744)(845.6923699,38.65181739)(845.79236816,38.71181936)
\curveto(846.08236951,38.88181716)(846.2873693,39.15181689)(846.40736816,39.52181936)
\curveto(846.42736916,39.57181647)(846.44236915,39.62181642)(846.45236816,39.67181936)
\curveto(846.45236914,39.73181631)(846.46236913,39.78681626)(846.48236816,39.83681936)
\lineto(846.48236816,39.91181936)
\curveto(846.4923691,39.98181606)(846.50236909,40.07681597)(846.51236816,40.19681936)
\curveto(846.51236908,40.32681572)(846.50236909,40.42681562)(846.48236816,40.49681936)
\curveto(846.46236913,40.56681548)(846.44736914,40.63681541)(846.43736816,40.70681936)
\curveto(846.41736917,40.78681526)(846.39736919,40.85681519)(846.37736816,40.91681936)
\curveto(846.21736937,41.29681475)(845.94236965,41.57181447)(845.55236816,41.74181936)
\curveto(845.42237017,41.79181425)(845.26737032,41.82681422)(845.08736816,41.84681936)
\curveto(844.90737068,41.87681417)(844.72237087,41.89181415)(844.53236816,41.89181936)
\curveto(844.33237126,41.90181414)(844.13237146,41.90181414)(843.93236816,41.89181936)
\lineto(843.36236816,41.89181936)
\lineto(839.11736816,41.89181936)
\lineto(837.57236816,41.89181936)
\curveto(837.46237813,41.89181415)(837.34237825,41.88681416)(837.21236816,41.87681936)
\curveto(837.08237851,41.86681418)(836.97737861,41.88681416)(836.89736816,41.93681936)
\curveto(836.82737876,41.99681405)(836.77737881,42.07681397)(836.74736816,42.17681936)
\curveto(836.74737884,42.19681385)(836.74737884,42.21681383)(836.74736816,42.23681936)
\curveto(836.74737884,42.25681379)(836.74237885,42.27681377)(836.73236816,42.29681936)
}
}
{
\newrgbcolor{curcolor}{0 0 0}
\pscustom[linestyle=none,fillstyle=solid,fillcolor=curcolor]
{
\newpath
\moveto(839.68736816,45.83049123)
\lineto(839.68736816,46.26549123)
\curveto(839.6873759,46.41548927)(839.72737586,46.52048916)(839.80736816,46.58049123)
\curveto(839.8873757,46.63048905)(839.9873756,46.65548903)(840.10736816,46.65549123)
\curveto(840.22737536,46.66548902)(840.34737524,46.67048901)(840.46736816,46.67049123)
\lineto(841.89236816,46.67049123)
\lineto(844.15736816,46.67049123)
\lineto(844.84736816,46.67049123)
\curveto(845.07737051,46.67048901)(845.27737031,46.69548899)(845.44736816,46.74549123)
\curveto(845.89736969,46.90548878)(846.21236938,47.20548848)(846.39236816,47.64549123)
\curveto(846.48236911,47.86548782)(846.51736907,48.13048755)(846.49736816,48.44049123)
\curveto(846.46736912,48.75048693)(846.41236918,49.00048668)(846.33236816,49.19049123)
\curveto(846.1923694,49.52048616)(846.01736957,49.7804859)(845.80736816,49.97049123)
\curveto(845.58737,50.17048551)(845.30237029,50.32548536)(844.95236816,50.43549123)
\curveto(844.87237072,50.46548522)(844.7923708,50.4854852)(844.71236816,50.49549123)
\curveto(844.63237096,50.50548518)(844.54737104,50.52048516)(844.45736816,50.54049123)
\curveto(844.40737118,50.55048513)(844.36237123,50.55048513)(844.32236816,50.54049123)
\curveto(844.28237131,50.54048514)(844.23737135,50.55048513)(844.18736816,50.57049123)
\lineto(843.87236816,50.57049123)
\curveto(843.7923718,50.59048509)(843.70237189,50.59548509)(843.60236816,50.58549123)
\curveto(843.4923721,50.57548511)(843.3923722,50.57048511)(843.30236816,50.57049123)
\lineto(842.13236816,50.57049123)
\lineto(840.54236816,50.57049123)
\curveto(840.42237517,50.57048511)(840.29737529,50.56548512)(840.16736816,50.55549123)
\curveto(840.02737556,50.55548513)(839.91737567,50.5804851)(839.83736816,50.63049123)
\curveto(839.7873758,50.67048501)(839.75737583,50.71548497)(839.74736816,50.76549123)
\curveto(839.72737586,50.82548486)(839.70737588,50.89548479)(839.68736816,50.97549123)
\lineto(839.68736816,51.20049123)
\curveto(839.6873759,51.32048436)(839.6923759,51.42548426)(839.70236816,51.51549123)
\curveto(839.71237588,51.61548407)(839.75737583,51.69048399)(839.83736816,51.74049123)
\curveto(839.8873757,51.79048389)(839.96237563,51.81548387)(840.06236816,51.81549123)
\lineto(840.34736816,51.81549123)
\lineto(841.36736816,51.81549123)
\lineto(845.40236816,51.81549123)
\lineto(846.75236816,51.81549123)
\curveto(846.87236872,51.81548387)(846.9873686,51.81048387)(847.09736816,51.80049123)
\curveto(847.19736839,51.80048388)(847.27236832,51.76548392)(847.32236816,51.69549123)
\curveto(847.35236824,51.65548403)(847.37736821,51.59548409)(847.39736816,51.51549123)
\curveto(847.40736818,51.43548425)(847.41736817,51.34548434)(847.42736816,51.24549123)
\curveto(847.42736816,51.15548453)(847.42236817,51.06548462)(847.41236816,50.97549123)
\curveto(847.40236819,50.89548479)(847.38236821,50.83548485)(847.35236816,50.79549123)
\curveto(847.31236828,50.74548494)(847.24736834,50.70048498)(847.15736816,50.66049123)
\curveto(847.11736847,50.65048503)(847.06236853,50.64048504)(846.99236816,50.63049123)
\curveto(846.92236867,50.63048505)(846.85736873,50.62548506)(846.79736816,50.61549123)
\curveto(846.72736886,50.60548508)(846.67236892,50.5854851)(846.63236816,50.55549123)
\curveto(846.592369,50.52548516)(846.57736901,50.4804852)(846.58736816,50.42049123)
\curveto(846.60736898,50.34048534)(846.66736892,50.26048542)(846.76736816,50.18049123)
\curveto(846.85736873,50.10048558)(846.92736866,50.02548566)(846.97736816,49.95549123)
\curveto(847.13736845,49.73548595)(847.27736831,49.4854862)(847.39736816,49.20549123)
\curveto(847.44736814,49.09548659)(847.47736811,48.9804867)(847.48736816,48.86049123)
\curveto(847.50736808,48.75048693)(847.53236806,48.63548705)(847.56236816,48.51549123)
\curveto(847.57236802,48.46548722)(847.57236802,48.41048727)(847.56236816,48.35049123)
\curveto(847.55236804,48.30048738)(847.55736803,48.25048743)(847.57736816,48.20049123)
\curveto(847.59736799,48.10048758)(847.59736799,48.01048767)(847.57736816,47.93049123)
\lineto(847.57736816,47.78049123)
\curveto(847.55736803,47.73048795)(847.54736804,47.67048801)(847.54736816,47.60049123)
\curveto(847.54736804,47.54048814)(847.54236805,47.4854882)(847.53236816,47.43549123)
\curveto(847.51236808,47.39548829)(847.50236809,47.35548833)(847.50236816,47.31549123)
\curveto(847.51236808,47.2854884)(847.50736808,47.24548844)(847.48736816,47.19549123)
\lineto(847.42736816,46.95549123)
\curveto(847.40736818,46.8854888)(847.37736821,46.81048887)(847.33736816,46.73049123)
\curveto(847.22736836,46.47048921)(847.08236851,46.25048943)(846.90236816,46.07049123)
\curveto(846.71236888,45.90048978)(846.4873691,45.76048992)(846.22736816,45.65049123)
\curveto(846.13736945,45.61049007)(846.04736954,45.5804901)(845.95736816,45.56049123)
\lineto(845.65736816,45.50049123)
\curveto(845.59736999,45.4804902)(845.54237005,45.47049021)(845.49236816,45.47049123)
\curveto(845.43237016,45.4804902)(845.36737022,45.47549021)(845.29736816,45.45549123)
\curveto(845.27737031,45.44549024)(845.25237034,45.44049024)(845.22236816,45.44049123)
\curveto(845.18237041,45.44049024)(845.14737044,45.43549025)(845.11736816,45.42549123)
\lineto(844.96736816,45.42549123)
\curveto(844.92737066,45.41549027)(844.88237071,45.41049027)(844.83236816,45.41049123)
\curveto(844.77237082,45.42049026)(844.71737087,45.42549026)(844.66736816,45.42549123)
\lineto(844.06736816,45.42549123)
\lineto(841.30736816,45.42549123)
\lineto(840.34736816,45.42549123)
\lineto(840.07736816,45.42549123)
\curveto(839.9873756,45.42549026)(839.91237568,45.44549024)(839.85236816,45.48549123)
\curveto(839.78237581,45.52549016)(839.73237586,45.60049008)(839.70236816,45.71049123)
\curveto(839.6923759,45.73048995)(839.6923759,45.75048993)(839.70236816,45.77049123)
\curveto(839.70237589,45.79048989)(839.69737589,45.81048987)(839.68736816,45.83049123)
}
}
{
\newrgbcolor{curcolor}{0 0 0}
\pscustom[linestyle=none,fillstyle=solid,fillcolor=curcolor]
{
\newpath
\moveto(836.73236816,54.28510061)
\curveto(836.73237886,54.41509899)(836.73237886,54.55009886)(836.73236816,54.69010061)
\curveto(836.73237886,54.84009857)(836.76737882,54.95009846)(836.83736816,55.02010061)
\curveto(836.90737868,55.07009834)(837.00237859,55.09509831)(837.12236816,55.09510061)
\curveto(837.23237836,55.1050983)(837.34737824,55.1100983)(837.46736816,55.11010061)
\lineto(838.80236816,55.11010061)
\lineto(844.87736816,55.11010061)
\lineto(846.55736816,55.11010061)
\lineto(846.94736816,55.11010061)
\curveto(847.0873685,55.1100983)(847.19736839,55.08509832)(847.27736816,55.03510061)
\curveto(847.32736826,55.0050984)(847.35736823,54.96009845)(847.36736816,54.90010061)
\curveto(847.37736821,54.85009856)(847.3923682,54.78509862)(847.41236816,54.70510061)
\lineto(847.41236816,54.49510061)
\lineto(847.41236816,54.18010061)
\curveto(847.40236819,54.08009933)(847.36736822,54.0050994)(847.30736816,53.95510061)
\curveto(847.22736836,53.9050995)(847.12736846,53.87509953)(847.00736816,53.86510061)
\lineto(846.63236816,53.86510061)
\lineto(845.25236816,53.86510061)
\lineto(839.01236816,53.86510061)
\lineto(837.54236816,53.86510061)
\curveto(837.43237816,53.86509954)(837.31737827,53.86009955)(837.19736816,53.85010061)
\curveto(837.06737852,53.85009956)(836.96737862,53.87509953)(836.89736816,53.92510061)
\curveto(836.83737875,53.96509944)(836.7873788,54.04009937)(836.74736816,54.15010061)
\curveto(836.73737885,54.17009924)(836.73737885,54.19009922)(836.74736816,54.21010061)
\curveto(836.74737884,54.24009917)(836.74237885,54.26509914)(836.73236816,54.28510061)
}
}
{
\newrgbcolor{curcolor}{0 0 0}
\pscustom[linestyle=none,fillstyle=solid,fillcolor=curcolor]
{
}
}
{
\newrgbcolor{curcolor}{0 0 0}
\pscustom[linestyle=none,fillstyle=solid,fillcolor=curcolor]
{
\newpath
\moveto(842.32736816,67.99510061)
\lineto(842.58236816,67.99510061)
\curveto(842.66237293,68.0050929)(842.73737285,68.00009291)(842.80736816,67.98010061)
\lineto(843.04736816,67.98010061)
\lineto(843.21236816,67.98010061)
\curveto(843.31237228,67.96009295)(843.41737217,67.95009296)(843.52736816,67.95010061)
\curveto(843.62737196,67.95009296)(843.72737186,67.94009297)(843.82736816,67.92010061)
\lineto(843.97736816,67.92010061)
\curveto(844.11737147,67.89009302)(844.25737133,67.87009304)(844.39736816,67.86010061)
\curveto(844.52737106,67.85009306)(844.65737093,67.82509308)(844.78736816,67.78510061)
\curveto(844.86737072,67.76509314)(844.95237064,67.74509316)(845.04236816,67.72510061)
\lineto(845.28236816,67.66510061)
\lineto(845.58236816,67.54510061)
\curveto(845.67236992,67.51509339)(845.76236983,67.48009343)(845.85236816,67.44010061)
\curveto(846.07236952,67.34009357)(846.2873693,67.2050937)(846.49736816,67.03510061)
\curveto(846.70736888,66.87509403)(846.87736871,66.70009421)(847.00736816,66.51010061)
\curveto(847.04736854,66.46009445)(847.0873685,66.40009451)(847.12736816,66.33010061)
\curveto(847.15736843,66.27009464)(847.1923684,66.2100947)(847.23236816,66.15010061)
\curveto(847.28236831,66.07009484)(847.32236827,65.97509493)(847.35236816,65.86510061)
\curveto(847.38236821,65.75509515)(847.41236818,65.65009526)(847.44236816,65.55010061)
\curveto(847.48236811,65.44009547)(847.50736808,65.33009558)(847.51736816,65.22010061)
\curveto(847.52736806,65.1100958)(847.54236805,64.99509591)(847.56236816,64.87510061)
\curveto(847.57236802,64.83509607)(847.57236802,64.79009612)(847.56236816,64.74010061)
\curveto(847.56236803,64.70009621)(847.56736802,64.66009625)(847.57736816,64.62010061)
\curveto(847.587368,64.58009633)(847.592368,64.52509638)(847.59236816,64.45510061)
\curveto(847.592368,64.38509652)(847.587368,64.33509657)(847.57736816,64.30510061)
\curveto(847.55736803,64.25509665)(847.55236804,64.2100967)(847.56236816,64.17010061)
\curveto(847.57236802,64.13009678)(847.57236802,64.09509681)(847.56236816,64.06510061)
\lineto(847.56236816,63.97510061)
\curveto(847.54236805,63.91509699)(847.52736806,63.85009706)(847.51736816,63.78010061)
\curveto(847.51736807,63.72009719)(847.51236808,63.65509725)(847.50236816,63.58510061)
\curveto(847.45236814,63.41509749)(847.40236819,63.25509765)(847.35236816,63.10510061)
\curveto(847.30236829,62.95509795)(847.23736835,62.8100981)(847.15736816,62.67010061)
\curveto(847.11736847,62.62009829)(847.0873685,62.56509834)(847.06736816,62.50510061)
\curveto(847.03736855,62.45509845)(847.00236859,62.4050985)(846.96236816,62.35510061)
\curveto(846.78236881,62.11509879)(846.56236903,61.91509899)(846.30236816,61.75510061)
\curveto(846.04236955,61.59509931)(845.75736983,61.45509945)(845.44736816,61.33510061)
\curveto(845.30737028,61.27509963)(845.16737042,61.23009968)(845.02736816,61.20010061)
\curveto(844.87737071,61.17009974)(844.72237087,61.13509977)(844.56236816,61.09510061)
\curveto(844.45237114,61.07509983)(844.34237125,61.06009985)(844.23236816,61.05010061)
\curveto(844.12237147,61.04009987)(844.01237158,61.02509988)(843.90236816,61.00510061)
\curveto(843.86237173,60.99509991)(843.82237177,60.99009992)(843.78236816,60.99010061)
\curveto(843.74237185,61.00009991)(843.70237189,61.00009991)(843.66236816,60.99010061)
\curveto(843.61237198,60.98009993)(843.56237203,60.97509993)(843.51236816,60.97510061)
\lineto(843.34736816,60.97510061)
\curveto(843.29737229,60.95509995)(843.24737234,60.95009996)(843.19736816,60.96010061)
\curveto(843.13737245,60.97009994)(843.08237251,60.97009994)(843.03236816,60.96010061)
\curveto(842.9923726,60.95009996)(842.94737264,60.95009996)(842.89736816,60.96010061)
\curveto(842.84737274,60.97009994)(842.79737279,60.96509994)(842.74736816,60.94510061)
\curveto(842.67737291,60.92509998)(842.60237299,60.92009999)(842.52236816,60.93010061)
\curveto(842.43237316,60.94009997)(842.34737324,60.94509996)(842.26736816,60.94510061)
\curveto(842.17737341,60.94509996)(842.07737351,60.94009997)(841.96736816,60.93010061)
\curveto(841.84737374,60.92009999)(841.74737384,60.92509998)(841.66736816,60.94510061)
\lineto(841.38236816,60.94510061)
\lineto(840.75236816,60.99010061)
\curveto(840.65237494,61.00009991)(840.55737503,61.0100999)(840.46736816,61.02010061)
\lineto(840.16736816,61.05010061)
\curveto(840.11737547,61.07009984)(840.06737552,61.07509983)(840.01736816,61.06510061)
\curveto(839.95737563,61.06509984)(839.90237569,61.07509983)(839.85236816,61.09510061)
\curveto(839.68237591,61.14509976)(839.51737607,61.18509972)(839.35736816,61.21510061)
\curveto(839.1873764,61.24509966)(839.02737656,61.29509961)(838.87736816,61.36510061)
\curveto(838.41737717,61.55509935)(838.04237755,61.77509913)(837.75236816,62.02510061)
\curveto(837.46237813,62.28509862)(837.21737837,62.64509826)(837.01736816,63.10510061)
\curveto(836.96737862,63.23509767)(836.93237866,63.36509754)(836.91236816,63.49510061)
\curveto(836.8923787,63.63509727)(836.86737872,63.77509713)(836.83736816,63.91510061)
\curveto(836.82737876,63.98509692)(836.82237877,64.05009686)(836.82236816,64.11010061)
\curveto(836.82237877,64.17009674)(836.81737877,64.23509667)(836.80736816,64.30510061)
\curveto(836.7873788,65.13509577)(836.93737865,65.8050951)(837.25736816,66.31510061)
\curveto(837.56737802,66.82509408)(838.00737758,67.2050937)(838.57736816,67.45510061)
\curveto(838.69737689,67.5050934)(838.82237677,67.55009336)(838.95236816,67.59010061)
\curveto(839.08237651,67.63009328)(839.21737637,67.67509323)(839.35736816,67.72510061)
\curveto(839.43737615,67.74509316)(839.52237607,67.76009315)(839.61236816,67.77010061)
\lineto(839.85236816,67.83010061)
\curveto(839.96237563,67.86009305)(840.07237552,67.87509303)(840.18236816,67.87510061)
\curveto(840.2923753,67.88509302)(840.40237519,67.90009301)(840.51236816,67.92010061)
\curveto(840.56237503,67.94009297)(840.60737498,67.94509296)(840.64736816,67.93510061)
\curveto(840.6873749,67.93509297)(840.72737486,67.94009297)(840.76736816,67.95010061)
\curveto(840.81737477,67.96009295)(840.87237472,67.96009295)(840.93236816,67.95010061)
\curveto(840.98237461,67.95009296)(841.03237456,67.95509295)(841.08236816,67.96510061)
\lineto(841.21736816,67.96510061)
\curveto(841.27737431,67.98509292)(841.34737424,67.98509292)(841.42736816,67.96510061)
\curveto(841.49737409,67.95509295)(841.56237403,67.96009295)(841.62236816,67.98010061)
\curveto(841.65237394,67.99009292)(841.6923739,67.99509291)(841.74236816,67.99510061)
\lineto(841.86236816,67.99510061)
\lineto(842.32736816,67.99510061)
\moveto(844.65236816,66.45010061)
\curveto(844.33237126,66.55009436)(843.96737162,66.6100943)(843.55736816,66.63010061)
\curveto(843.14737244,66.65009426)(842.73737285,66.66009425)(842.32736816,66.66010061)
\curveto(841.89737369,66.66009425)(841.47737411,66.65009426)(841.06736816,66.63010061)
\curveto(840.65737493,66.6100943)(840.27237532,66.56509434)(839.91236816,66.49510061)
\curveto(839.55237604,66.42509448)(839.23237636,66.31509459)(838.95236816,66.16510061)
\curveto(838.66237693,66.02509488)(838.42737716,65.83009508)(838.24736816,65.58010061)
\curveto(838.13737745,65.42009549)(838.05737753,65.24009567)(838.00736816,65.04010061)
\curveto(837.94737764,64.84009607)(837.91737767,64.59509631)(837.91736816,64.30510061)
\curveto(837.93737765,64.28509662)(837.94737764,64.25009666)(837.94736816,64.20010061)
\curveto(837.93737765,64.15009676)(837.93737765,64.1100968)(837.94736816,64.08010061)
\curveto(837.96737762,64.00009691)(837.9873776,63.92509698)(838.00736816,63.85510061)
\curveto(838.01737757,63.79509711)(838.03737755,63.73009718)(838.06736816,63.66010061)
\curveto(838.1873774,63.39009752)(838.35737723,63.17009774)(838.57736816,63.00010061)
\curveto(838.7873768,62.84009807)(839.03237656,62.7050982)(839.31236816,62.59510061)
\curveto(839.42237617,62.54509836)(839.54237605,62.5050984)(839.67236816,62.47510061)
\curveto(839.7923758,62.45509845)(839.91737567,62.43009848)(840.04736816,62.40010061)
\curveto(840.09737549,62.38009853)(840.15237544,62.37009854)(840.21236816,62.37010061)
\curveto(840.26237533,62.37009854)(840.31237528,62.36509854)(840.36236816,62.35510061)
\curveto(840.45237514,62.34509856)(840.54737504,62.33509857)(840.64736816,62.32510061)
\curveto(840.73737485,62.31509859)(840.83237476,62.3050986)(840.93236816,62.29510061)
\curveto(841.01237458,62.29509861)(841.09737449,62.29009862)(841.18736816,62.28010061)
\lineto(841.42736816,62.28010061)
\lineto(841.60736816,62.28010061)
\curveto(841.63737395,62.27009864)(841.67237392,62.26509864)(841.71236816,62.26510061)
\lineto(841.84736816,62.26510061)
\lineto(842.29736816,62.26510061)
\curveto(842.37737321,62.26509864)(842.46237313,62.26009865)(842.55236816,62.25010061)
\curveto(842.63237296,62.25009866)(842.70737288,62.26009865)(842.77736816,62.28010061)
\lineto(843.04736816,62.28010061)
\curveto(843.06737252,62.28009863)(843.09737249,62.27509863)(843.13736816,62.26510061)
\curveto(843.16737242,62.26509864)(843.1923724,62.27009864)(843.21236816,62.28010061)
\curveto(843.31237228,62.29009862)(843.41237218,62.29509861)(843.51236816,62.29510061)
\curveto(843.60237199,62.3050986)(843.70237189,62.31509859)(843.81236816,62.32510061)
\curveto(843.93237166,62.35509855)(844.05737153,62.37009854)(844.18736816,62.37010061)
\curveto(844.30737128,62.38009853)(844.42237117,62.4050985)(844.53236816,62.44510061)
\curveto(844.83237076,62.52509838)(845.09737049,62.6100983)(845.32736816,62.70010061)
\curveto(845.55737003,62.80009811)(845.77236982,62.94509796)(845.97236816,63.13510061)
\curveto(846.17236942,63.34509756)(846.32236927,63.6100973)(846.42236816,63.93010061)
\curveto(846.44236915,63.97009694)(846.45236914,64.0050969)(846.45236816,64.03510061)
\curveto(846.44236915,64.07509683)(846.44736914,64.12009679)(846.46736816,64.17010061)
\curveto(846.47736911,64.2100967)(846.4873691,64.28009663)(846.49736816,64.38010061)
\curveto(846.50736908,64.49009642)(846.50236909,64.57509633)(846.48236816,64.63510061)
\curveto(846.46236913,64.7050962)(846.45236914,64.77509613)(846.45236816,64.84510061)
\curveto(846.44236915,64.91509599)(846.42736916,64.98009593)(846.40736816,65.04010061)
\curveto(846.34736924,65.24009567)(846.26236933,65.42009549)(846.15236816,65.58010061)
\curveto(846.13236946,65.6100953)(846.11236948,65.63509527)(846.09236816,65.65510061)
\lineto(846.03236816,65.71510061)
\curveto(846.01236958,65.75509515)(845.97236962,65.8050951)(845.91236816,65.86510061)
\curveto(845.77236982,65.96509494)(845.64236995,66.05009486)(845.52236816,66.12010061)
\curveto(845.40237019,66.19009472)(845.25737033,66.26009465)(845.08736816,66.33010061)
\curveto(845.01737057,66.36009455)(844.94737064,66.38009453)(844.87736816,66.39010061)
\curveto(844.80737078,66.4100945)(844.73237086,66.43009448)(844.65236816,66.45010061)
}
}
{
\newrgbcolor{curcolor}{0 0 0}
\pscustom[linestyle=none,fillstyle=solid,fillcolor=curcolor]
{
\newpath
\moveto(844.33736816,76.35970998)
\curveto(844.37737121,76.36970226)(844.42737116,76.36970226)(844.48736816,76.35970998)
\curveto(844.54737104,76.35970227)(844.59737099,76.35470228)(844.63736816,76.34470998)
\curveto(844.67737091,76.34470229)(844.71737087,76.33970229)(844.75736816,76.32970998)
\lineto(844.86236816,76.32970998)
\curveto(844.94237065,76.30970232)(845.02237057,76.29470234)(845.10236816,76.28470998)
\curveto(845.18237041,76.27470236)(845.25737033,76.25470238)(845.32736816,76.22470998)
\curveto(845.40737018,76.20470243)(845.48237011,76.18470245)(845.55236816,76.16470998)
\curveto(845.62236997,76.14470249)(845.69736989,76.11470252)(845.77736816,76.07470998)
\curveto(846.19736939,75.89470274)(846.53736905,75.63970299)(846.79736816,75.30970998)
\curveto(847.05736853,74.97970365)(847.26236833,74.58970404)(847.41236816,74.13970998)
\curveto(847.45236814,74.01970461)(847.47736811,73.89470474)(847.48736816,73.76470998)
\curveto(847.50736808,73.64470499)(847.53236806,73.51970511)(847.56236816,73.38970998)
\curveto(847.57236802,73.3297053)(847.57736801,73.26470537)(847.57736816,73.19470998)
\curveto(847.57736801,73.1347055)(847.58236801,73.06970556)(847.59236816,72.99970998)
\lineto(847.59236816,72.87970998)
\lineto(847.59236816,72.68470998)
\curveto(847.60236799,72.62470601)(847.59736799,72.56970606)(847.57736816,72.51970998)
\curveto(847.55736803,72.44970618)(847.55236804,72.38470625)(847.56236816,72.32470998)
\curveto(847.57236802,72.26470637)(847.56736802,72.20470643)(847.54736816,72.14470998)
\curveto(847.53736805,72.09470654)(847.53236806,72.04970658)(847.53236816,72.00970998)
\curveto(847.53236806,71.96970666)(847.52236807,71.92470671)(847.50236816,71.87470998)
\curveto(847.48236811,71.79470684)(847.46236813,71.71970691)(847.44236816,71.64970998)
\curveto(847.43236816,71.57970705)(847.41736817,71.50970712)(847.39736816,71.43970998)
\curveto(847.22736836,70.95970767)(847.01736857,70.55970807)(846.76736816,70.23970998)
\curveto(846.50736908,69.9297087)(846.15236944,69.67970895)(845.70236816,69.48970998)
\curveto(845.64236995,69.45970917)(845.58237001,69.4347092)(845.52236816,69.41470998)
\curveto(845.45237014,69.40470923)(845.37737021,69.38970924)(845.29736816,69.36970998)
\curveto(845.23737035,69.34970928)(845.17237042,69.3347093)(845.10236816,69.32470998)
\curveto(845.03237056,69.31470932)(844.96237063,69.29970933)(844.89236816,69.27970998)
\curveto(844.84237075,69.26970936)(844.80237079,69.26470937)(844.77236816,69.26470998)
\lineto(844.65236816,69.26470998)
\curveto(844.61237098,69.25470938)(844.56237103,69.24470939)(844.50236816,69.23470998)
\curveto(844.44237115,69.2347094)(844.3923712,69.23970939)(844.35236816,69.24970998)
\lineto(844.21736816,69.24970998)
\curveto(844.16737142,69.25970937)(844.11737147,69.26470937)(844.06736816,69.26470998)
\curveto(843.96737162,69.28470935)(843.87237172,69.29970933)(843.78236816,69.30970998)
\curveto(843.68237191,69.31970931)(843.587372,69.33970929)(843.49736816,69.36970998)
\curveto(843.34737224,69.41970921)(843.20737238,69.47470916)(843.07736816,69.53470998)
\curveto(842.94737264,69.59470904)(842.82737276,69.66470897)(842.71736816,69.74470998)
\curveto(842.66737292,69.77470886)(842.62737296,69.80470883)(842.59736816,69.83470998)
\curveto(842.56737302,69.87470876)(842.53237306,69.90970872)(842.49236816,69.93970998)
\curveto(842.41237318,69.99970863)(842.34237325,70.06970856)(842.28236816,70.14970998)
\curveto(842.23237336,70.20970842)(842.1873734,70.26970836)(842.14736816,70.32970998)
\lineto(841.99736816,70.53970998)
\curveto(841.95737363,70.58970804)(841.92237367,70.63970799)(841.89236816,70.68970998)
\curveto(841.85237374,70.73970789)(841.79737379,70.77470786)(841.72736816,70.79470998)
\curveto(841.69737389,70.79470784)(841.67237392,70.78470785)(841.65236816,70.76470998)
\curveto(841.62237397,70.75470788)(841.59737399,70.74470789)(841.57736816,70.73470998)
\curveto(841.52737406,70.69470794)(841.48237411,70.64470799)(841.44236816,70.58470998)
\curveto(841.3923742,70.5347081)(841.34737424,70.48470815)(841.30736816,70.43470998)
\curveto(841.27737431,70.39470824)(841.22237437,70.34470829)(841.14236816,70.28470998)
\curveto(841.11237448,70.26470837)(841.0873745,70.2347084)(841.06736816,70.19470998)
\curveto(841.03737455,70.16470847)(841.00237459,70.13970849)(840.96236816,70.11970998)
\curveto(840.75237484,69.94970868)(840.50737508,69.81970881)(840.22736816,69.72970998)
\curveto(840.14737544,69.70970892)(840.06737552,69.69470894)(839.98736816,69.68470998)
\curveto(839.90737568,69.67470896)(839.82737576,69.65970897)(839.74736816,69.63970998)
\curveto(839.69737589,69.61970901)(839.63237596,69.60970902)(839.55236816,69.60970998)
\curveto(839.46237613,69.60970902)(839.3923762,69.61970901)(839.34236816,69.63970998)
\curveto(839.24237635,69.63970899)(839.17237642,69.64470899)(839.13236816,69.65470998)
\curveto(839.05237654,69.67470896)(838.98237661,69.68970894)(838.92236816,69.69970998)
\curveto(838.85237674,69.70970892)(838.78237681,69.72470891)(838.71236816,69.74470998)
\curveto(838.28237731,69.89470874)(837.93737765,70.10970852)(837.67736816,70.38970998)
\curveto(837.41737817,70.67970795)(837.20237839,71.0297076)(837.03236816,71.43970998)
\curveto(836.98237861,71.54970708)(836.95237864,71.66470697)(836.94236816,71.78470998)
\curveto(836.92237867,71.91470672)(836.8923787,72.04470659)(836.85236816,72.17470998)
\curveto(836.85237874,72.25470638)(836.85237874,72.32470631)(836.85236816,72.38470998)
\curveto(836.84237875,72.45470618)(836.83237876,72.5297061)(836.82236816,72.60970998)
\curveto(836.80237879,73.39970523)(836.93237866,74.05470458)(837.21236816,74.57470998)
\curveto(837.4923781,75.10470353)(837.90237769,75.48470315)(838.44236816,75.71470998)
\curveto(838.67237692,75.82470281)(838.95737663,75.89470274)(839.29736816,75.92470998)
\curveto(839.62737596,75.96470267)(839.93237566,75.9347027)(840.21236816,75.83470998)
\curveto(840.34237525,75.79470284)(840.46237513,75.74470289)(840.57236816,75.68470998)
\curveto(840.68237491,75.634703)(840.7873748,75.57470306)(840.88736816,75.50470998)
\curveto(840.92737466,75.48470315)(840.96237463,75.45470318)(840.99236816,75.41470998)
\lineto(841.08236816,75.32470998)
\curveto(841.17237442,75.27470336)(841.23737435,75.21470342)(841.27736816,75.14470998)
\curveto(841.32737426,75.09470354)(841.37737421,75.03970359)(841.42736816,74.97970998)
\curveto(841.46737412,74.9297037)(841.51237408,74.88470375)(841.56236816,74.84470998)
\curveto(841.58237401,74.82470381)(841.60737398,74.80470383)(841.63736816,74.78470998)
\curveto(841.65737393,74.77470386)(841.68237391,74.77470386)(841.71236816,74.78470998)
\curveto(841.76237383,74.79470384)(841.81237378,74.82470381)(841.86236816,74.87470998)
\curveto(841.90237369,74.92470371)(841.94237365,74.97970365)(841.98236816,75.03970998)
\lineto(842.10236816,75.21970998)
\curveto(842.13237346,75.27970335)(842.16237343,75.3297033)(842.19236816,75.36970998)
\curveto(842.43237316,75.69970293)(842.74237285,75.94970268)(843.12236816,76.11970998)
\curveto(843.20237239,76.15970247)(843.2873723,76.18970244)(843.37736816,76.20970998)
\curveto(843.46737212,76.23970239)(843.55737203,76.26470237)(843.64736816,76.28470998)
\curveto(843.69737189,76.29470234)(843.75237184,76.30470233)(843.81236816,76.31470998)
\lineto(843.96236816,76.34470998)
\curveto(844.02237157,76.35470228)(844.0873715,76.35470228)(844.15736816,76.34470998)
\curveto(844.21737137,76.3347023)(844.27737131,76.33970229)(844.33736816,76.35970998)
\moveto(839.29736816,70.97470998)
\curveto(839.40737618,70.94470769)(839.54737604,70.93970769)(839.71736816,70.95970998)
\curveto(839.87737571,70.97970765)(840.00237559,71.00470763)(840.09236816,71.03470998)
\curveto(840.41237518,71.14470749)(840.65737493,71.29470734)(840.82736816,71.48470998)
\curveto(840.9873746,71.67470696)(841.11737447,71.93970669)(841.21736816,72.27970998)
\curveto(841.24737434,72.40970622)(841.27237432,72.57470606)(841.29236816,72.77470998)
\curveto(841.30237429,72.97470566)(841.2873743,73.14470549)(841.24736816,73.28470998)
\curveto(841.16737442,73.57470506)(841.05737453,73.81470482)(840.91736816,74.00470998)
\curveto(840.76737482,74.20470443)(840.56737502,74.35970427)(840.31736816,74.46970998)
\curveto(840.26737532,74.48970414)(840.22237537,74.49970413)(840.18236816,74.49970998)
\curveto(840.14237545,74.50970412)(840.09737549,74.52470411)(840.04736816,74.54470998)
\curveto(839.93737565,74.57470406)(839.79737579,74.59470404)(839.62736816,74.60470998)
\curveto(839.45737613,74.61470402)(839.31237628,74.60470403)(839.19236816,74.57470998)
\curveto(839.10237649,74.55470408)(839.01737657,74.5297041)(838.93736816,74.49970998)
\curveto(838.85737673,74.47970415)(838.77737681,74.44470419)(838.69736816,74.39470998)
\curveto(838.42737716,74.22470441)(838.23237736,73.99970463)(838.11236816,73.71970998)
\curveto(837.9923776,73.44970518)(837.93237766,73.08970554)(837.93236816,72.63970998)
\curveto(837.95237764,72.61970601)(837.95737763,72.58970604)(837.94736816,72.54970998)
\curveto(837.93737765,72.50970612)(837.93737765,72.47470616)(837.94736816,72.44470998)
\curveto(837.96737762,72.39470624)(837.98237761,72.33970629)(837.99236816,72.27970998)
\curveto(837.9923776,72.2297064)(838.00237759,72.17970645)(838.02236816,72.12970998)
\curveto(838.11237748,71.88970674)(838.22737736,71.67970695)(838.36736816,71.49970998)
\curveto(838.49737709,71.31970731)(838.67737691,71.17970745)(838.90736816,71.07970998)
\curveto(838.96737662,71.05970757)(839.03237656,71.03970759)(839.10236816,71.01970998)
\curveto(839.16237643,71.00970762)(839.22737636,70.99470764)(839.29736816,70.97470998)
\moveto(844.83236816,74.99470998)
\curveto(844.64237095,75.04470359)(844.43737115,75.04970358)(844.21736816,75.00970998)
\curveto(843.99737159,74.97970365)(843.81737177,74.9347037)(843.67736816,74.87470998)
\curveto(843.30737228,74.70470393)(843.00237259,74.44470419)(842.76236816,74.09470998)
\curveto(842.52237307,73.75470488)(842.40237319,73.31970531)(842.40236816,72.78970998)
\curveto(842.42237317,72.75970587)(842.42737316,72.71970591)(842.41736816,72.66970998)
\curveto(842.39737319,72.61970601)(842.3923732,72.57970605)(842.40236816,72.54970998)
\lineto(842.46236816,72.27970998)
\curveto(842.47237312,72.19970643)(842.4873731,72.11970651)(842.50736816,72.03970998)
\curveto(842.61737297,71.73970689)(842.76237283,71.47470716)(842.94236816,71.24470998)
\curveto(843.12237247,71.02470761)(843.35237224,70.85470778)(843.63236816,70.73470998)
\curveto(843.71237188,70.70470793)(843.7923718,70.67970795)(843.87236816,70.65970998)
\curveto(843.95237164,70.63970799)(844.03737155,70.61970801)(844.12736816,70.59970998)
\curveto(844.24737134,70.56970806)(844.39737119,70.55970807)(844.57736816,70.56970998)
\curveto(844.75737083,70.58970804)(844.89737069,70.61470802)(844.99736816,70.64470998)
\curveto(845.04737054,70.66470797)(845.0923705,70.67470796)(845.13236816,70.67470998)
\curveto(845.16237043,70.68470795)(845.20237039,70.69970793)(845.25236816,70.71970998)
\curveto(845.47237012,70.81970781)(845.67236992,70.94970768)(845.85236816,71.10970998)
\curveto(846.03236956,71.27970735)(846.16736942,71.47470716)(846.25736816,71.69470998)
\curveto(846.29736929,71.76470687)(846.33236926,71.85970677)(846.36236816,71.97970998)
\curveto(846.45236914,72.19970643)(846.49736909,72.45470618)(846.49736816,72.74470998)
\lineto(846.49736816,73.02970998)
\curveto(846.47736911,73.1297055)(846.46236913,73.22470541)(846.45236816,73.31470998)
\curveto(846.44236915,73.40470523)(846.42236917,73.49470514)(846.39236816,73.58470998)
\curveto(846.31236928,73.84470479)(846.18236941,74.08470455)(846.00236816,74.30470998)
\curveto(845.81236978,74.5347041)(845.59736999,74.70470393)(845.35736816,74.81470998)
\curveto(845.27737031,74.85470378)(845.19737039,74.88470375)(845.11736816,74.90470998)
\curveto(845.02737056,74.9347037)(844.93237066,74.96470367)(844.83236816,74.99470998)
}
}
{
\newrgbcolor{curcolor}{0 0 0}
\pscustom[linestyle=none,fillstyle=solid,fillcolor=curcolor]
{
\newpath
\moveto(845.77736816,78.63431936)
\lineto(845.77736816,79.26431936)
\lineto(845.77736816,79.45931936)
\curveto(845.77736981,79.52931683)(845.7873698,79.58931677)(845.80736816,79.63931936)
\curveto(845.84736974,79.70931665)(845.8873697,79.7593166)(845.92736816,79.78931936)
\curveto(845.97736961,79.82931653)(846.04236955,79.84931651)(846.12236816,79.84931936)
\curveto(846.20236939,79.8593165)(846.2873693,79.86431649)(846.37736816,79.86431936)
\lineto(847.09736816,79.86431936)
\curveto(847.57736801,79.86431649)(847.9873676,79.80431655)(848.32736816,79.68431936)
\curveto(848.66736692,79.56431679)(848.94236665,79.36931699)(849.15236816,79.09931936)
\curveto(849.20236639,79.02931733)(849.24736634,78.9593174)(849.28736816,78.88931936)
\curveto(849.33736625,78.82931753)(849.38236621,78.7543176)(849.42236816,78.66431936)
\curveto(849.43236616,78.64431771)(849.44236615,78.61431774)(849.45236816,78.57431936)
\curveto(849.47236612,78.53431782)(849.47736611,78.48931787)(849.46736816,78.43931936)
\curveto(849.43736615,78.34931801)(849.36236623,78.29431806)(849.24236816,78.27431936)
\curveto(849.13236646,78.2543181)(849.03736655,78.26931809)(848.95736816,78.31931936)
\curveto(848.8873667,78.34931801)(848.82236677,78.39431796)(848.76236816,78.45431936)
\curveto(848.71236688,78.52431783)(848.66236693,78.58931777)(848.61236816,78.64931936)
\curveto(848.56236703,78.71931764)(848.4873671,78.77931758)(848.38736816,78.82931936)
\curveto(848.29736729,78.88931747)(848.20736738,78.93931742)(848.11736816,78.97931936)
\curveto(848.0873675,78.99931736)(848.02736756,79.02431733)(847.93736816,79.05431936)
\curveto(847.85736773,79.08431727)(847.7873678,79.08931727)(847.72736816,79.06931936)
\curveto(847.587368,79.03931732)(847.49736809,78.97931738)(847.45736816,78.88931936)
\curveto(847.42736816,78.80931755)(847.41236818,78.71931764)(847.41236816,78.61931936)
\curveto(847.41236818,78.51931784)(847.3873682,78.43431792)(847.33736816,78.36431936)
\curveto(847.26736832,78.27431808)(847.12736846,78.22931813)(846.91736816,78.22931936)
\lineto(846.36236816,78.22931936)
\lineto(846.13736816,78.22931936)
\curveto(846.05736953,78.23931812)(845.9923696,78.2593181)(845.94236816,78.28931936)
\curveto(845.86236973,78.34931801)(845.81736977,78.41931794)(845.80736816,78.49931936)
\curveto(845.79736979,78.51931784)(845.7923698,78.53931782)(845.79236816,78.55931936)
\curveto(845.7923698,78.58931777)(845.7873698,78.61431774)(845.77736816,78.63431936)
}
}
{
\newrgbcolor{curcolor}{0 0 0}
\pscustom[linestyle=none,fillstyle=solid,fillcolor=curcolor]
{
}
}
{
\newrgbcolor{curcolor}{0 0 0}
\pscustom[linestyle=none,fillstyle=solid,fillcolor=curcolor]
{
\newpath
\moveto(836.80736816,89.26463186)
\curveto(836.79737879,89.95462722)(836.91737867,90.55462662)(837.16736816,91.06463186)
\curveto(837.41737817,91.58462559)(837.75237784,91.9796252)(838.17236816,92.24963186)
\curveto(838.25237734,92.29962488)(838.34237725,92.34462483)(838.44236816,92.38463186)
\curveto(838.53237706,92.42462475)(838.62737696,92.46962471)(838.72736816,92.51963186)
\curveto(838.82737676,92.55962462)(838.92737666,92.58962459)(839.02736816,92.60963186)
\curveto(839.12737646,92.62962455)(839.23237636,92.64962453)(839.34236816,92.66963186)
\curveto(839.3923762,92.68962449)(839.43737615,92.69462448)(839.47736816,92.68463186)
\curveto(839.51737607,92.6746245)(839.56237603,92.6796245)(839.61236816,92.69963186)
\curveto(839.66237593,92.70962447)(839.74737584,92.71462446)(839.86736816,92.71463186)
\curveto(839.97737561,92.71462446)(840.06237553,92.70962447)(840.12236816,92.69963186)
\curveto(840.18237541,92.6796245)(840.24237535,92.66962451)(840.30236816,92.66963186)
\curveto(840.36237523,92.6796245)(840.42237517,92.6746245)(840.48236816,92.65463186)
\curveto(840.62237497,92.61462456)(840.75737483,92.5796246)(840.88736816,92.54963186)
\curveto(841.01737457,92.51962466)(841.14237445,92.4796247)(841.26236816,92.42963186)
\curveto(841.40237419,92.36962481)(841.52737406,92.29962488)(841.63736816,92.21963186)
\curveto(841.74737384,92.14962503)(841.85737373,92.0746251)(841.96736816,91.99463186)
\lineto(842.02736816,91.93463186)
\curveto(842.04737354,91.92462525)(842.06737352,91.90962527)(842.08736816,91.88963186)
\curveto(842.24737334,91.76962541)(842.3923732,91.63462554)(842.52236816,91.48463186)
\curveto(842.65237294,91.33462584)(842.77737281,91.174626)(842.89736816,91.00463186)
\curveto(843.11737247,90.69462648)(843.32237227,90.39962678)(843.51236816,90.11963186)
\curveto(843.65237194,89.88962729)(843.7873718,89.65962752)(843.91736816,89.42963186)
\curveto(844.04737154,89.20962797)(844.18237141,88.98962819)(844.32236816,88.76963186)
\curveto(844.4923711,88.51962866)(844.67237092,88.2796289)(844.86236816,88.04963186)
\curveto(845.05237054,87.82962935)(845.27737031,87.63962954)(845.53736816,87.47963186)
\curveto(845.59736999,87.43962974)(845.65736993,87.40462977)(845.71736816,87.37463186)
\curveto(845.76736982,87.34462983)(845.83236976,87.31462986)(845.91236816,87.28463186)
\curveto(845.98236961,87.26462991)(846.04236955,87.25962992)(846.09236816,87.26963186)
\curveto(846.16236943,87.28962989)(846.21736937,87.32462985)(846.25736816,87.37463186)
\curveto(846.2873693,87.42462975)(846.30736928,87.48462969)(846.31736816,87.55463186)
\lineto(846.31736816,87.79463186)
\lineto(846.31736816,88.54463186)
\lineto(846.31736816,91.34963186)
\lineto(846.31736816,92.00963186)
\curveto(846.31736927,92.09962508)(846.32236927,92.18462499)(846.33236816,92.26463186)
\curveto(846.33236926,92.34462483)(846.35236924,92.40962477)(846.39236816,92.45963186)
\curveto(846.43236916,92.50962467)(846.50736908,92.54962463)(846.61736816,92.57963186)
\curveto(846.71736887,92.61962456)(846.81736877,92.62962455)(846.91736816,92.60963186)
\lineto(847.05236816,92.60963186)
\curveto(847.12236847,92.58962459)(847.18236841,92.56962461)(847.23236816,92.54963186)
\curveto(847.28236831,92.52962465)(847.32236827,92.49462468)(847.35236816,92.44463186)
\curveto(847.3923682,92.39462478)(847.41236818,92.32462485)(847.41236816,92.23463186)
\lineto(847.41236816,91.96463186)
\lineto(847.41236816,91.06463186)
\lineto(847.41236816,87.55463186)
\lineto(847.41236816,86.48963186)
\curveto(847.41236818,86.40963077)(847.41736817,86.31963086)(847.42736816,86.21963186)
\curveto(847.42736816,86.11963106)(847.41736817,86.03463114)(847.39736816,85.96463186)
\curveto(847.32736826,85.75463142)(847.14736844,85.68963149)(846.85736816,85.76963186)
\curveto(846.81736877,85.7796314)(846.78236881,85.7796314)(846.75236816,85.76963186)
\curveto(846.71236888,85.76963141)(846.66736892,85.7796314)(846.61736816,85.79963186)
\curveto(846.53736905,85.81963136)(846.45236914,85.83963134)(846.36236816,85.85963186)
\curveto(846.27236932,85.8796313)(846.1873694,85.90463127)(846.10736816,85.93463186)
\curveto(845.61736997,86.09463108)(845.20237039,86.29463088)(844.86236816,86.53463186)
\curveto(844.61237098,86.71463046)(844.3873712,86.91963026)(844.18736816,87.14963186)
\curveto(843.97737161,87.3796298)(843.78237181,87.61962956)(843.60236816,87.86963186)
\curveto(843.42237217,88.12962905)(843.25237234,88.39462878)(843.09236816,88.66463186)
\curveto(842.92237267,88.94462823)(842.74737284,89.21462796)(842.56736816,89.47463186)
\curveto(842.4873731,89.58462759)(842.41237318,89.68962749)(842.34236816,89.78963186)
\curveto(842.27237332,89.89962728)(842.19737339,90.00962717)(842.11736816,90.11963186)
\curveto(842.0873735,90.15962702)(842.05737353,90.19462698)(842.02736816,90.22463186)
\curveto(841.9873736,90.26462691)(841.95737363,90.30462687)(841.93736816,90.34463186)
\curveto(841.82737376,90.48462669)(841.70237389,90.60962657)(841.56236816,90.71963186)
\curveto(841.53237406,90.73962644)(841.50737408,90.76462641)(841.48736816,90.79463186)
\curveto(841.45737413,90.82462635)(841.42737416,90.84962633)(841.39736816,90.86963186)
\curveto(841.29737429,90.94962623)(841.19737439,91.01462616)(841.09736816,91.06463186)
\curveto(840.99737459,91.12462605)(840.8873747,91.179626)(840.76736816,91.22963186)
\curveto(840.69737489,91.25962592)(840.62237497,91.2796259)(840.54236816,91.28963186)
\lineto(840.30236816,91.34963186)
\lineto(840.21236816,91.34963186)
\curveto(840.18237541,91.35962582)(840.15237544,91.36462581)(840.12236816,91.36463186)
\curveto(840.05237554,91.38462579)(839.95737563,91.38962579)(839.83736816,91.37963186)
\curveto(839.70737588,91.3796258)(839.60737598,91.36962581)(839.53736816,91.34963186)
\curveto(839.45737613,91.32962585)(839.38237621,91.30962587)(839.31236816,91.28963186)
\curveto(839.23237636,91.2796259)(839.15237644,91.25962592)(839.07236816,91.22963186)
\curveto(838.83237676,91.11962606)(838.63237696,90.96962621)(838.47236816,90.77963186)
\curveto(838.30237729,90.59962658)(838.16237743,90.3796268)(838.05236816,90.11963186)
\curveto(838.03237756,90.04962713)(838.01737757,89.9796272)(838.00736816,89.90963186)
\curveto(837.9873776,89.83962734)(837.96737762,89.76462741)(837.94736816,89.68463186)
\curveto(837.92737766,89.60462757)(837.91737767,89.49462768)(837.91736816,89.35463186)
\curveto(837.91737767,89.22462795)(837.92737766,89.11962806)(837.94736816,89.03963186)
\curveto(837.95737763,88.9796282)(837.96237763,88.92462825)(837.96236816,88.87463186)
\curveto(837.96237763,88.82462835)(837.97237762,88.7746284)(837.99236816,88.72463186)
\curveto(838.03237756,88.62462855)(838.07237752,88.52962865)(838.11236816,88.43963186)
\curveto(838.15237744,88.35962882)(838.19737739,88.2796289)(838.24736816,88.19963186)
\curveto(838.26737732,88.16962901)(838.2923773,88.13962904)(838.32236816,88.10963186)
\curveto(838.35237724,88.08962909)(838.37737721,88.06462911)(838.39736816,88.03463186)
\lineto(838.47236816,87.95963186)
\curveto(838.4923771,87.92962925)(838.51237708,87.90462927)(838.53236816,87.88463186)
\lineto(838.74236816,87.73463186)
\curveto(838.80237679,87.69462948)(838.86737672,87.64962953)(838.93736816,87.59963186)
\curveto(839.02737656,87.53962964)(839.13237646,87.48962969)(839.25236816,87.44963186)
\curveto(839.36237623,87.41962976)(839.47237612,87.38462979)(839.58236816,87.34463186)
\curveto(839.6923759,87.30462987)(839.83737575,87.2796299)(840.01736816,87.26963186)
\curveto(840.1873754,87.25962992)(840.31237528,87.22962995)(840.39236816,87.17963186)
\curveto(840.47237512,87.12963005)(840.51737507,87.05463012)(840.52736816,86.95463186)
\curveto(840.53737505,86.85463032)(840.54237505,86.74463043)(840.54236816,86.62463186)
\curveto(840.54237505,86.58463059)(840.54737504,86.54463063)(840.55736816,86.50463186)
\curveto(840.55737503,86.46463071)(840.55237504,86.42963075)(840.54236816,86.39963186)
\curveto(840.52237507,86.34963083)(840.51237508,86.29963088)(840.51236816,86.24963186)
\curveto(840.51237508,86.20963097)(840.50237509,86.16963101)(840.48236816,86.12963186)
\curveto(840.42237517,86.03963114)(840.2873753,85.99463118)(840.07736816,85.99463186)
\lineto(839.95736816,85.99463186)
\curveto(839.89737569,86.00463117)(839.83737575,86.00963117)(839.77736816,86.00963186)
\curveto(839.70737588,86.01963116)(839.64237595,86.02963115)(839.58236816,86.03963186)
\curveto(839.47237612,86.05963112)(839.37237622,86.0796311)(839.28236816,86.09963186)
\curveto(839.18237641,86.11963106)(839.0873765,86.14963103)(838.99736816,86.18963186)
\curveto(838.92737666,86.20963097)(838.86737672,86.22963095)(838.81736816,86.24963186)
\lineto(838.63736816,86.30963186)
\curveto(838.37737721,86.42963075)(838.13237746,86.58463059)(837.90236816,86.77463186)
\curveto(837.67237792,86.9746302)(837.4873781,87.18962999)(837.34736816,87.41963186)
\curveto(837.26737832,87.52962965)(837.20237839,87.64462953)(837.15236816,87.76463186)
\lineto(837.00236816,88.15463186)
\curveto(836.95237864,88.26462891)(836.92237867,88.3796288)(836.91236816,88.49963186)
\curveto(836.8923787,88.61962856)(836.86737872,88.74462843)(836.83736816,88.87463186)
\curveto(836.83737875,88.94462823)(836.83737875,89.00962817)(836.83736816,89.06963186)
\curveto(836.82737876,89.12962805)(836.81737877,89.19462798)(836.80736816,89.26463186)
}
}
{
\newrgbcolor{curcolor}{0 0 0}
\pscustom[linestyle=none,fillstyle=solid,fillcolor=curcolor]
{
\newpath
\moveto(842.32736816,101.36424123)
\lineto(842.58236816,101.36424123)
\curveto(842.66237293,101.37423353)(842.73737285,101.36923353)(842.80736816,101.34924123)
\lineto(843.04736816,101.34924123)
\lineto(843.21236816,101.34924123)
\curveto(843.31237228,101.32923357)(843.41737217,101.31923358)(843.52736816,101.31924123)
\curveto(843.62737196,101.31923358)(843.72737186,101.30923359)(843.82736816,101.28924123)
\lineto(843.97736816,101.28924123)
\curveto(844.11737147,101.25923364)(844.25737133,101.23923366)(844.39736816,101.22924123)
\curveto(844.52737106,101.21923368)(844.65737093,101.19423371)(844.78736816,101.15424123)
\curveto(844.86737072,101.13423377)(844.95237064,101.11423379)(845.04236816,101.09424123)
\lineto(845.28236816,101.03424123)
\lineto(845.58236816,100.91424123)
\curveto(845.67236992,100.88423402)(845.76236983,100.84923405)(845.85236816,100.80924123)
\curveto(846.07236952,100.70923419)(846.2873693,100.57423433)(846.49736816,100.40424123)
\curveto(846.70736888,100.24423466)(846.87736871,100.06923483)(847.00736816,99.87924123)
\curveto(847.04736854,99.82923507)(847.0873685,99.76923513)(847.12736816,99.69924123)
\curveto(847.15736843,99.63923526)(847.1923684,99.57923532)(847.23236816,99.51924123)
\curveto(847.28236831,99.43923546)(847.32236827,99.34423556)(847.35236816,99.23424123)
\curveto(847.38236821,99.12423578)(847.41236818,99.01923588)(847.44236816,98.91924123)
\curveto(847.48236811,98.80923609)(847.50736808,98.6992362)(847.51736816,98.58924123)
\curveto(847.52736806,98.47923642)(847.54236805,98.36423654)(847.56236816,98.24424123)
\curveto(847.57236802,98.2042367)(847.57236802,98.15923674)(847.56236816,98.10924123)
\curveto(847.56236803,98.06923683)(847.56736802,98.02923687)(847.57736816,97.98924123)
\curveto(847.587368,97.94923695)(847.592368,97.89423701)(847.59236816,97.82424123)
\curveto(847.592368,97.75423715)(847.587368,97.7042372)(847.57736816,97.67424123)
\curveto(847.55736803,97.62423728)(847.55236804,97.57923732)(847.56236816,97.53924123)
\curveto(847.57236802,97.4992374)(847.57236802,97.46423744)(847.56236816,97.43424123)
\lineto(847.56236816,97.34424123)
\curveto(847.54236805,97.28423762)(847.52736806,97.21923768)(847.51736816,97.14924123)
\curveto(847.51736807,97.08923781)(847.51236808,97.02423788)(847.50236816,96.95424123)
\curveto(847.45236814,96.78423812)(847.40236819,96.62423828)(847.35236816,96.47424123)
\curveto(847.30236829,96.32423858)(847.23736835,96.17923872)(847.15736816,96.03924123)
\curveto(847.11736847,95.98923891)(847.0873685,95.93423897)(847.06736816,95.87424123)
\curveto(847.03736855,95.82423908)(847.00236859,95.77423913)(846.96236816,95.72424123)
\curveto(846.78236881,95.48423942)(846.56236903,95.28423962)(846.30236816,95.12424123)
\curveto(846.04236955,94.96423994)(845.75736983,94.82424008)(845.44736816,94.70424123)
\curveto(845.30737028,94.64424026)(845.16737042,94.5992403)(845.02736816,94.56924123)
\curveto(844.87737071,94.53924036)(844.72237087,94.5042404)(844.56236816,94.46424123)
\curveto(844.45237114,94.44424046)(844.34237125,94.42924047)(844.23236816,94.41924123)
\curveto(844.12237147,94.40924049)(844.01237158,94.39424051)(843.90236816,94.37424123)
\curveto(843.86237173,94.36424054)(843.82237177,94.35924054)(843.78236816,94.35924123)
\curveto(843.74237185,94.36924053)(843.70237189,94.36924053)(843.66236816,94.35924123)
\curveto(843.61237198,94.34924055)(843.56237203,94.34424056)(843.51236816,94.34424123)
\lineto(843.34736816,94.34424123)
\curveto(843.29737229,94.32424058)(843.24737234,94.31924058)(843.19736816,94.32924123)
\curveto(843.13737245,94.33924056)(843.08237251,94.33924056)(843.03236816,94.32924123)
\curveto(842.9923726,94.31924058)(842.94737264,94.31924058)(842.89736816,94.32924123)
\curveto(842.84737274,94.33924056)(842.79737279,94.33424057)(842.74736816,94.31424123)
\curveto(842.67737291,94.29424061)(842.60237299,94.28924061)(842.52236816,94.29924123)
\curveto(842.43237316,94.30924059)(842.34737324,94.31424059)(842.26736816,94.31424123)
\curveto(842.17737341,94.31424059)(842.07737351,94.30924059)(841.96736816,94.29924123)
\curveto(841.84737374,94.28924061)(841.74737384,94.29424061)(841.66736816,94.31424123)
\lineto(841.38236816,94.31424123)
\lineto(840.75236816,94.35924123)
\curveto(840.65237494,94.36924053)(840.55737503,94.37924052)(840.46736816,94.38924123)
\lineto(840.16736816,94.41924123)
\curveto(840.11737547,94.43924046)(840.06737552,94.44424046)(840.01736816,94.43424123)
\curveto(839.95737563,94.43424047)(839.90237569,94.44424046)(839.85236816,94.46424123)
\curveto(839.68237591,94.51424039)(839.51737607,94.55424035)(839.35736816,94.58424123)
\curveto(839.1873764,94.61424029)(839.02737656,94.66424024)(838.87736816,94.73424123)
\curveto(838.41737717,94.92423998)(838.04237755,95.14423976)(837.75236816,95.39424123)
\curveto(837.46237813,95.65423925)(837.21737837,96.01423889)(837.01736816,96.47424123)
\curveto(836.96737862,96.6042383)(836.93237866,96.73423817)(836.91236816,96.86424123)
\curveto(836.8923787,97.0042379)(836.86737872,97.14423776)(836.83736816,97.28424123)
\curveto(836.82737876,97.35423755)(836.82237877,97.41923748)(836.82236816,97.47924123)
\curveto(836.82237877,97.53923736)(836.81737877,97.6042373)(836.80736816,97.67424123)
\curveto(836.7873788,98.5042364)(836.93737865,99.17423573)(837.25736816,99.68424123)
\curveto(837.56737802,100.19423471)(838.00737758,100.57423433)(838.57736816,100.82424123)
\curveto(838.69737689,100.87423403)(838.82237677,100.91923398)(838.95236816,100.95924123)
\curveto(839.08237651,100.9992339)(839.21737637,101.04423386)(839.35736816,101.09424123)
\curveto(839.43737615,101.11423379)(839.52237607,101.12923377)(839.61236816,101.13924123)
\lineto(839.85236816,101.19924123)
\curveto(839.96237563,101.22923367)(840.07237552,101.24423366)(840.18236816,101.24424123)
\curveto(840.2923753,101.25423365)(840.40237519,101.26923363)(840.51236816,101.28924123)
\curveto(840.56237503,101.30923359)(840.60737498,101.31423359)(840.64736816,101.30424123)
\curveto(840.6873749,101.3042336)(840.72737486,101.30923359)(840.76736816,101.31924123)
\curveto(840.81737477,101.32923357)(840.87237472,101.32923357)(840.93236816,101.31924123)
\curveto(840.98237461,101.31923358)(841.03237456,101.32423358)(841.08236816,101.33424123)
\lineto(841.21736816,101.33424123)
\curveto(841.27737431,101.35423355)(841.34737424,101.35423355)(841.42736816,101.33424123)
\curveto(841.49737409,101.32423358)(841.56237403,101.32923357)(841.62236816,101.34924123)
\curveto(841.65237394,101.35923354)(841.6923739,101.36423354)(841.74236816,101.36424123)
\lineto(841.86236816,101.36424123)
\lineto(842.32736816,101.36424123)
\moveto(844.65236816,99.81924123)
\curveto(844.33237126,99.91923498)(843.96737162,99.97923492)(843.55736816,99.99924123)
\curveto(843.14737244,100.01923488)(842.73737285,100.02923487)(842.32736816,100.02924123)
\curveto(841.89737369,100.02923487)(841.47737411,100.01923488)(841.06736816,99.99924123)
\curveto(840.65737493,99.97923492)(840.27237532,99.93423497)(839.91236816,99.86424123)
\curveto(839.55237604,99.79423511)(839.23237636,99.68423522)(838.95236816,99.53424123)
\curveto(838.66237693,99.39423551)(838.42737716,99.1992357)(838.24736816,98.94924123)
\curveto(838.13737745,98.78923611)(838.05737753,98.60923629)(838.00736816,98.40924123)
\curveto(837.94737764,98.20923669)(837.91737767,97.96423694)(837.91736816,97.67424123)
\curveto(837.93737765,97.65423725)(837.94737764,97.61923728)(837.94736816,97.56924123)
\curveto(837.93737765,97.51923738)(837.93737765,97.47923742)(837.94736816,97.44924123)
\curveto(837.96737762,97.36923753)(837.9873776,97.29423761)(838.00736816,97.22424123)
\curveto(838.01737757,97.16423774)(838.03737755,97.0992378)(838.06736816,97.02924123)
\curveto(838.1873774,96.75923814)(838.35737723,96.53923836)(838.57736816,96.36924123)
\curveto(838.7873768,96.20923869)(839.03237656,96.07423883)(839.31236816,95.96424123)
\curveto(839.42237617,95.91423899)(839.54237605,95.87423903)(839.67236816,95.84424123)
\curveto(839.7923758,95.82423908)(839.91737567,95.7992391)(840.04736816,95.76924123)
\curveto(840.09737549,95.74923915)(840.15237544,95.73923916)(840.21236816,95.73924123)
\curveto(840.26237533,95.73923916)(840.31237528,95.73423917)(840.36236816,95.72424123)
\curveto(840.45237514,95.71423919)(840.54737504,95.7042392)(840.64736816,95.69424123)
\curveto(840.73737485,95.68423922)(840.83237476,95.67423923)(840.93236816,95.66424123)
\curveto(841.01237458,95.66423924)(841.09737449,95.65923924)(841.18736816,95.64924123)
\lineto(841.42736816,95.64924123)
\lineto(841.60736816,95.64924123)
\curveto(841.63737395,95.63923926)(841.67237392,95.63423927)(841.71236816,95.63424123)
\lineto(841.84736816,95.63424123)
\lineto(842.29736816,95.63424123)
\curveto(842.37737321,95.63423927)(842.46237313,95.62923927)(842.55236816,95.61924123)
\curveto(842.63237296,95.61923928)(842.70737288,95.62923927)(842.77736816,95.64924123)
\lineto(843.04736816,95.64924123)
\curveto(843.06737252,95.64923925)(843.09737249,95.64423926)(843.13736816,95.63424123)
\curveto(843.16737242,95.63423927)(843.1923724,95.63923926)(843.21236816,95.64924123)
\curveto(843.31237228,95.65923924)(843.41237218,95.66423924)(843.51236816,95.66424123)
\curveto(843.60237199,95.67423923)(843.70237189,95.68423922)(843.81236816,95.69424123)
\curveto(843.93237166,95.72423918)(844.05737153,95.73923916)(844.18736816,95.73924123)
\curveto(844.30737128,95.74923915)(844.42237117,95.77423913)(844.53236816,95.81424123)
\curveto(844.83237076,95.89423901)(845.09737049,95.97923892)(845.32736816,96.06924123)
\curveto(845.55737003,96.16923873)(845.77236982,96.31423859)(845.97236816,96.50424123)
\curveto(846.17236942,96.71423819)(846.32236927,96.97923792)(846.42236816,97.29924123)
\curveto(846.44236915,97.33923756)(846.45236914,97.37423753)(846.45236816,97.40424123)
\curveto(846.44236915,97.44423746)(846.44736914,97.48923741)(846.46736816,97.53924123)
\curveto(846.47736911,97.57923732)(846.4873691,97.64923725)(846.49736816,97.74924123)
\curveto(846.50736908,97.85923704)(846.50236909,97.94423696)(846.48236816,98.00424123)
\curveto(846.46236913,98.07423683)(846.45236914,98.14423676)(846.45236816,98.21424123)
\curveto(846.44236915,98.28423662)(846.42736916,98.34923655)(846.40736816,98.40924123)
\curveto(846.34736924,98.60923629)(846.26236933,98.78923611)(846.15236816,98.94924123)
\curveto(846.13236946,98.97923592)(846.11236948,99.0042359)(846.09236816,99.02424123)
\lineto(846.03236816,99.08424123)
\curveto(846.01236958,99.12423578)(845.97236962,99.17423573)(845.91236816,99.23424123)
\curveto(845.77236982,99.33423557)(845.64236995,99.41923548)(845.52236816,99.48924123)
\curveto(845.40237019,99.55923534)(845.25737033,99.62923527)(845.08736816,99.69924123)
\curveto(845.01737057,99.72923517)(844.94737064,99.74923515)(844.87736816,99.75924123)
\curveto(844.80737078,99.77923512)(844.73237086,99.7992351)(844.65236816,99.81924123)
}
}
{
\newrgbcolor{curcolor}{0 0 0}
\pscustom[linestyle=none,fillstyle=solid,fillcolor=curcolor]
{
\newpath
\moveto(836.80736816,106.77385061)
\curveto(836.80737878,106.87384575)(836.81737877,106.96884566)(836.83736816,107.05885061)
\curveto(836.84737874,107.14884548)(836.87737871,107.21384541)(836.92736816,107.25385061)
\curveto(837.00737858,107.31384531)(837.11237848,107.34384528)(837.24236816,107.34385061)
\lineto(837.63236816,107.34385061)
\lineto(839.13236816,107.34385061)
\lineto(845.52236816,107.34385061)
\lineto(846.69236816,107.34385061)
\lineto(847.00736816,107.34385061)
\curveto(847.10736848,107.35384527)(847.1873684,107.33884529)(847.24736816,107.29885061)
\curveto(847.32736826,107.24884538)(847.37736821,107.17384545)(847.39736816,107.07385061)
\curveto(847.40736818,106.98384564)(847.41236818,106.87384575)(847.41236816,106.74385061)
\lineto(847.41236816,106.51885061)
\curveto(847.3923682,106.43884619)(847.37736821,106.36884626)(847.36736816,106.30885061)
\curveto(847.34736824,106.24884638)(847.30736828,106.19884643)(847.24736816,106.15885061)
\curveto(847.1873684,106.11884651)(847.11236848,106.09884653)(847.02236816,106.09885061)
\lineto(846.72236816,106.09885061)
\lineto(845.62736816,106.09885061)
\lineto(840.28736816,106.09885061)
\curveto(840.19737539,106.07884655)(840.12237547,106.06384656)(840.06236816,106.05385061)
\curveto(839.9923756,106.05384657)(839.93237566,106.0238466)(839.88236816,105.96385061)
\curveto(839.83237576,105.89384673)(839.80737578,105.80384682)(839.80736816,105.69385061)
\curveto(839.79737579,105.59384703)(839.7923758,105.48384714)(839.79236816,105.36385061)
\lineto(839.79236816,104.22385061)
\lineto(839.79236816,103.72885061)
\curveto(839.78237581,103.56884906)(839.72237587,103.45884917)(839.61236816,103.39885061)
\curveto(839.58237601,103.37884925)(839.55237604,103.36884926)(839.52236816,103.36885061)
\curveto(839.48237611,103.36884926)(839.43737615,103.36384926)(839.38736816,103.35385061)
\curveto(839.26737632,103.33384929)(839.15737643,103.33884929)(839.05736816,103.36885061)
\curveto(838.95737663,103.40884922)(838.8873767,103.46384916)(838.84736816,103.53385061)
\curveto(838.79737679,103.61384901)(838.77237682,103.73384889)(838.77236816,103.89385061)
\curveto(838.77237682,104.05384857)(838.75737683,104.18884844)(838.72736816,104.29885061)
\curveto(838.71737687,104.34884828)(838.71237688,104.40384822)(838.71236816,104.46385061)
\curveto(838.70237689,104.5238481)(838.6873769,104.58384804)(838.66736816,104.64385061)
\curveto(838.61737697,104.79384783)(838.56737702,104.93884769)(838.51736816,105.07885061)
\curveto(838.45737713,105.21884741)(838.3873772,105.35384727)(838.30736816,105.48385061)
\curveto(838.21737737,105.623847)(838.11237748,105.74384688)(837.99236816,105.84385061)
\curveto(837.87237772,105.94384668)(837.74237785,106.03884659)(837.60236816,106.12885061)
\curveto(837.50237809,106.18884644)(837.3923782,106.23384639)(837.27236816,106.26385061)
\curveto(837.15237844,106.30384632)(837.04737854,106.35384627)(836.95736816,106.41385061)
\curveto(836.89737869,106.46384616)(836.85737873,106.53384609)(836.83736816,106.62385061)
\curveto(836.82737876,106.64384598)(836.82237877,106.66884596)(836.82236816,106.69885061)
\curveto(836.82237877,106.7288459)(836.81737877,106.75384587)(836.80736816,106.77385061)
}
}
{
\newrgbcolor{curcolor}{0 0 0}
\pscustom[linestyle=none,fillstyle=solid,fillcolor=curcolor]
{
\newpath
\moveto(836.80736816,115.12345998)
\curveto(836.80737878,115.22345513)(836.81737877,115.31845503)(836.83736816,115.40845998)
\curveto(836.84737874,115.49845485)(836.87737871,115.56345479)(836.92736816,115.60345998)
\curveto(837.00737858,115.66345469)(837.11237848,115.69345466)(837.24236816,115.69345998)
\lineto(837.63236816,115.69345998)
\lineto(839.13236816,115.69345998)
\lineto(845.52236816,115.69345998)
\lineto(846.69236816,115.69345998)
\lineto(847.00736816,115.69345998)
\curveto(847.10736848,115.70345465)(847.1873684,115.68845466)(847.24736816,115.64845998)
\curveto(847.32736826,115.59845475)(847.37736821,115.52345483)(847.39736816,115.42345998)
\curveto(847.40736818,115.33345502)(847.41236818,115.22345513)(847.41236816,115.09345998)
\lineto(847.41236816,114.86845998)
\curveto(847.3923682,114.78845556)(847.37736821,114.71845563)(847.36736816,114.65845998)
\curveto(847.34736824,114.59845575)(847.30736828,114.5484558)(847.24736816,114.50845998)
\curveto(847.1873684,114.46845588)(847.11236848,114.4484559)(847.02236816,114.44845998)
\lineto(846.72236816,114.44845998)
\lineto(845.62736816,114.44845998)
\lineto(840.28736816,114.44845998)
\curveto(840.19737539,114.42845592)(840.12237547,114.41345594)(840.06236816,114.40345998)
\curveto(839.9923756,114.40345595)(839.93237566,114.37345598)(839.88236816,114.31345998)
\curveto(839.83237576,114.24345611)(839.80737578,114.1534562)(839.80736816,114.04345998)
\curveto(839.79737579,113.94345641)(839.7923758,113.83345652)(839.79236816,113.71345998)
\lineto(839.79236816,112.57345998)
\lineto(839.79236816,112.07845998)
\curveto(839.78237581,111.91845843)(839.72237587,111.80845854)(839.61236816,111.74845998)
\curveto(839.58237601,111.72845862)(839.55237604,111.71845863)(839.52236816,111.71845998)
\curveto(839.48237611,111.71845863)(839.43737615,111.71345864)(839.38736816,111.70345998)
\curveto(839.26737632,111.68345867)(839.15737643,111.68845866)(839.05736816,111.71845998)
\curveto(838.95737663,111.75845859)(838.8873767,111.81345854)(838.84736816,111.88345998)
\curveto(838.79737679,111.96345839)(838.77237682,112.08345827)(838.77236816,112.24345998)
\curveto(838.77237682,112.40345795)(838.75737683,112.53845781)(838.72736816,112.64845998)
\curveto(838.71737687,112.69845765)(838.71237688,112.7534576)(838.71236816,112.81345998)
\curveto(838.70237689,112.87345748)(838.6873769,112.93345742)(838.66736816,112.99345998)
\curveto(838.61737697,113.14345721)(838.56737702,113.28845706)(838.51736816,113.42845998)
\curveto(838.45737713,113.56845678)(838.3873772,113.70345665)(838.30736816,113.83345998)
\curveto(838.21737737,113.97345638)(838.11237748,114.09345626)(837.99236816,114.19345998)
\curveto(837.87237772,114.29345606)(837.74237785,114.38845596)(837.60236816,114.47845998)
\curveto(837.50237809,114.53845581)(837.3923782,114.58345577)(837.27236816,114.61345998)
\curveto(837.15237844,114.6534557)(837.04737854,114.70345565)(836.95736816,114.76345998)
\curveto(836.89737869,114.81345554)(836.85737873,114.88345547)(836.83736816,114.97345998)
\curveto(836.82737876,114.99345536)(836.82237877,115.01845533)(836.82236816,115.04845998)
\curveto(836.82237877,115.07845527)(836.81737877,115.10345525)(836.80736816,115.12345998)
}
}
{
\newrgbcolor{curcolor}{0 0 0}
\pscustom[linestyle=none,fillstyle=solid,fillcolor=curcolor]
{
\newpath
\moveto(857.64368408,42.29681936)
\curveto(857.64369478,42.36681368)(857.64369478,42.4468136)(857.64368408,42.53681936)
\curveto(857.63369479,42.62681342)(857.63369479,42.71181333)(857.64368408,42.79181936)
\curveto(857.64369478,42.88181316)(857.65369477,42.96181308)(857.67368408,43.03181936)
\curveto(857.69369473,43.11181293)(857.7236947,43.16681288)(857.76368408,43.19681936)
\curveto(857.81369461,43.22681282)(857.88869453,43.2468128)(857.98868408,43.25681936)
\curveto(858.07869434,43.27681277)(858.18369424,43.28681276)(858.30368408,43.28681936)
\curveto(858.41369401,43.29681275)(858.52869389,43.29681275)(858.64868408,43.28681936)
\lineto(858.94868408,43.28681936)
\lineto(861.96368408,43.28681936)
\lineto(864.85868408,43.28681936)
\curveto(865.18868723,43.28681276)(865.51368691,43.28181276)(865.83368408,43.27181936)
\curveto(866.14368628,43.27181277)(866.423686,43.23181281)(866.67368408,43.15181936)
\curveto(867.0236854,43.03181301)(867.3186851,42.87681317)(867.55868408,42.68681936)
\curveto(867.78868463,42.49681355)(867.98868443,42.25681379)(868.15868408,41.96681936)
\curveto(868.20868421,41.90681414)(868.24368418,41.8418142)(868.26368408,41.77181936)
\curveto(868.28368414,41.71181433)(868.30868411,41.6418144)(868.33868408,41.56181936)
\curveto(868.38868403,41.4418146)(868.423684,41.31181473)(868.44368408,41.17181936)
\curveto(868.47368395,41.041815)(868.50368392,40.90681514)(868.53368408,40.76681936)
\curveto(868.55368387,40.71681533)(868.55868386,40.66681538)(868.54868408,40.61681936)
\curveto(868.53868388,40.56681548)(868.53868388,40.51181553)(868.54868408,40.45181936)
\curveto(868.55868386,40.43181561)(868.55868386,40.40681564)(868.54868408,40.37681936)
\curveto(868.54868387,40.3468157)(868.55368387,40.32181572)(868.56368408,40.30181936)
\curveto(868.57368385,40.26181578)(868.57868384,40.20681584)(868.57868408,40.13681936)
\curveto(868.57868384,40.06681598)(868.57368385,40.01181603)(868.56368408,39.97181936)
\curveto(868.55368387,39.92181612)(868.55368387,39.86681618)(868.56368408,39.80681936)
\curveto(868.57368385,39.7468163)(868.56868385,39.69181635)(868.54868408,39.64181936)
\curveto(868.5186839,39.51181653)(868.49868392,39.38681666)(868.48868408,39.26681936)
\curveto(868.47868394,39.1468169)(868.45368397,39.03181701)(868.41368408,38.92181936)
\curveto(868.29368413,38.55181749)(868.1236843,38.23181781)(867.90368408,37.96181936)
\curveto(867.68368474,37.69181835)(867.40368502,37.48181856)(867.06368408,37.33181936)
\curveto(866.94368548,37.28181876)(866.8186856,37.23681881)(866.68868408,37.19681936)
\curveto(866.55868586,37.16681888)(866.423686,37.13181891)(866.28368408,37.09181936)
\curveto(866.23368619,37.08181896)(866.19368623,37.07681897)(866.16368408,37.07681936)
\curveto(866.1236863,37.07681897)(866.07868634,37.07181897)(866.02868408,37.06181936)
\curveto(865.99868642,37.05181899)(865.96368646,37.046819)(865.92368408,37.04681936)
\curveto(865.87368655,37.046819)(865.83368659,37.041819)(865.80368408,37.03181936)
\lineto(865.63868408,37.03181936)
\curveto(865.55868686,37.01181903)(865.45868696,37.00681904)(865.33868408,37.01681936)
\curveto(865.20868721,37.02681902)(865.1186873,37.041819)(865.06868408,37.06181936)
\curveto(864.97868744,37.08181896)(864.91368751,37.13681891)(864.87368408,37.22681936)
\curveto(864.85368757,37.25681879)(864.84868757,37.28681876)(864.85868408,37.31681936)
\curveto(864.85868756,37.3468187)(864.85368757,37.38681866)(864.84368408,37.43681936)
\curveto(864.83368759,37.47681857)(864.82868759,37.51681853)(864.82868408,37.55681936)
\lineto(864.82868408,37.70681936)
\curveto(864.82868759,37.82681822)(864.83368759,37.9468181)(864.84368408,38.06681936)
\curveto(864.84368758,38.19681785)(864.87868754,38.28681776)(864.94868408,38.33681936)
\curveto(865.00868741,38.37681767)(865.06868735,38.39681765)(865.12868408,38.39681936)
\curveto(865.18868723,38.39681765)(865.25868716,38.40681764)(865.33868408,38.42681936)
\curveto(865.36868705,38.43681761)(865.40368702,38.43681761)(865.44368408,38.42681936)
\curveto(865.47368695,38.42681762)(865.49868692,38.43181761)(865.51868408,38.44181936)
\lineto(865.72868408,38.44181936)
\curveto(865.77868664,38.46181758)(865.82868659,38.46681758)(865.87868408,38.45681936)
\curveto(865.9186865,38.45681759)(865.96368646,38.46681758)(866.01368408,38.48681936)
\curveto(866.14368628,38.51681753)(866.26868615,38.5468175)(866.38868408,38.57681936)
\curveto(866.49868592,38.60681744)(866.60368582,38.65181739)(866.70368408,38.71181936)
\curveto(866.99368543,38.88181716)(867.19868522,39.15181689)(867.31868408,39.52181936)
\curveto(867.33868508,39.57181647)(867.35368507,39.62181642)(867.36368408,39.67181936)
\curveto(867.36368506,39.73181631)(867.37368505,39.78681626)(867.39368408,39.83681936)
\lineto(867.39368408,39.91181936)
\curveto(867.40368502,39.98181606)(867.41368501,40.07681597)(867.42368408,40.19681936)
\curveto(867.423685,40.32681572)(867.41368501,40.42681562)(867.39368408,40.49681936)
\curveto(867.37368505,40.56681548)(867.35868506,40.63681541)(867.34868408,40.70681936)
\curveto(867.32868509,40.78681526)(867.30868511,40.85681519)(867.28868408,40.91681936)
\curveto(867.12868529,41.29681475)(866.85368557,41.57181447)(866.46368408,41.74181936)
\curveto(866.33368609,41.79181425)(866.17868624,41.82681422)(865.99868408,41.84681936)
\curveto(865.8186866,41.87681417)(865.63368679,41.89181415)(865.44368408,41.89181936)
\curveto(865.24368718,41.90181414)(865.04368738,41.90181414)(864.84368408,41.89181936)
\lineto(864.27368408,41.89181936)
\lineto(860.02868408,41.89181936)
\lineto(858.48368408,41.89181936)
\curveto(858.37369405,41.89181415)(858.25369417,41.88681416)(858.12368408,41.87681936)
\curveto(857.99369443,41.86681418)(857.88869453,41.88681416)(857.80868408,41.93681936)
\curveto(857.73869468,41.99681405)(857.68869473,42.07681397)(857.65868408,42.17681936)
\curveto(857.65869476,42.19681385)(857.65869476,42.21681383)(857.65868408,42.23681936)
\curveto(857.65869476,42.25681379)(857.65369477,42.27681377)(857.64368408,42.29681936)
}
}
{
\newrgbcolor{curcolor}{0 0 0}
\pscustom[linestyle=none,fillstyle=solid,fillcolor=curcolor]
{
\newpath
\moveto(860.59868408,45.83049123)
\lineto(860.59868408,46.26549123)
\curveto(860.59869182,46.41548927)(860.63869178,46.52048916)(860.71868408,46.58049123)
\curveto(860.79869162,46.63048905)(860.89869152,46.65548903)(861.01868408,46.65549123)
\curveto(861.13869128,46.66548902)(861.25869116,46.67048901)(861.37868408,46.67049123)
\lineto(862.80368408,46.67049123)
\lineto(865.06868408,46.67049123)
\lineto(865.75868408,46.67049123)
\curveto(865.98868643,46.67048901)(866.18868623,46.69548899)(866.35868408,46.74549123)
\curveto(866.80868561,46.90548878)(867.1236853,47.20548848)(867.30368408,47.64549123)
\curveto(867.39368503,47.86548782)(867.42868499,48.13048755)(867.40868408,48.44049123)
\curveto(867.37868504,48.75048693)(867.3236851,49.00048668)(867.24368408,49.19049123)
\curveto(867.10368532,49.52048616)(866.92868549,49.7804859)(866.71868408,49.97049123)
\curveto(866.49868592,50.17048551)(866.21368621,50.32548536)(865.86368408,50.43549123)
\curveto(865.78368664,50.46548522)(865.70368672,50.4854852)(865.62368408,50.49549123)
\curveto(865.54368688,50.50548518)(865.45868696,50.52048516)(865.36868408,50.54049123)
\curveto(865.3186871,50.55048513)(865.27368715,50.55048513)(865.23368408,50.54049123)
\curveto(865.19368723,50.54048514)(865.14868727,50.55048513)(865.09868408,50.57049123)
\lineto(864.78368408,50.57049123)
\curveto(864.70368772,50.59048509)(864.61368781,50.59548509)(864.51368408,50.58549123)
\curveto(864.40368802,50.57548511)(864.30368812,50.57048511)(864.21368408,50.57049123)
\lineto(863.04368408,50.57049123)
\lineto(861.45368408,50.57049123)
\curveto(861.33369109,50.57048511)(861.20869121,50.56548512)(861.07868408,50.55549123)
\curveto(860.93869148,50.55548513)(860.82869159,50.5804851)(860.74868408,50.63049123)
\curveto(860.69869172,50.67048501)(860.66869175,50.71548497)(860.65868408,50.76549123)
\curveto(860.63869178,50.82548486)(860.6186918,50.89548479)(860.59868408,50.97549123)
\lineto(860.59868408,51.20049123)
\curveto(860.59869182,51.32048436)(860.60369182,51.42548426)(860.61368408,51.51549123)
\curveto(860.6236918,51.61548407)(860.66869175,51.69048399)(860.74868408,51.74049123)
\curveto(860.79869162,51.79048389)(860.87369155,51.81548387)(860.97368408,51.81549123)
\lineto(861.25868408,51.81549123)
\lineto(862.27868408,51.81549123)
\lineto(866.31368408,51.81549123)
\lineto(867.66368408,51.81549123)
\curveto(867.78368464,51.81548387)(867.89868452,51.81048387)(868.00868408,51.80049123)
\curveto(868.10868431,51.80048388)(868.18368424,51.76548392)(868.23368408,51.69549123)
\curveto(868.26368416,51.65548403)(868.28868413,51.59548409)(868.30868408,51.51549123)
\curveto(868.3186841,51.43548425)(868.32868409,51.34548434)(868.33868408,51.24549123)
\curveto(868.33868408,51.15548453)(868.33368409,51.06548462)(868.32368408,50.97549123)
\curveto(868.31368411,50.89548479)(868.29368413,50.83548485)(868.26368408,50.79549123)
\curveto(868.2236842,50.74548494)(868.15868426,50.70048498)(868.06868408,50.66049123)
\curveto(868.02868439,50.65048503)(867.97368445,50.64048504)(867.90368408,50.63049123)
\curveto(867.83368459,50.63048505)(867.76868465,50.62548506)(867.70868408,50.61549123)
\curveto(867.63868478,50.60548508)(867.58368484,50.5854851)(867.54368408,50.55549123)
\curveto(867.50368492,50.52548516)(867.48868493,50.4804852)(867.49868408,50.42049123)
\curveto(867.5186849,50.34048534)(867.57868484,50.26048542)(867.67868408,50.18049123)
\curveto(867.76868465,50.10048558)(867.83868458,50.02548566)(867.88868408,49.95549123)
\curveto(868.04868437,49.73548595)(868.18868423,49.4854862)(868.30868408,49.20549123)
\curveto(868.35868406,49.09548659)(868.38868403,48.9804867)(868.39868408,48.86049123)
\curveto(868.418684,48.75048693)(868.44368398,48.63548705)(868.47368408,48.51549123)
\curveto(868.48368394,48.46548722)(868.48368394,48.41048727)(868.47368408,48.35049123)
\curveto(868.46368396,48.30048738)(868.46868395,48.25048743)(868.48868408,48.20049123)
\curveto(868.50868391,48.10048758)(868.50868391,48.01048767)(868.48868408,47.93049123)
\lineto(868.48868408,47.78049123)
\curveto(868.46868395,47.73048795)(868.45868396,47.67048801)(868.45868408,47.60049123)
\curveto(868.45868396,47.54048814)(868.45368397,47.4854882)(868.44368408,47.43549123)
\curveto(868.423684,47.39548829)(868.41368401,47.35548833)(868.41368408,47.31549123)
\curveto(868.423684,47.2854884)(868.418684,47.24548844)(868.39868408,47.19549123)
\lineto(868.33868408,46.95549123)
\curveto(868.3186841,46.8854888)(868.28868413,46.81048887)(868.24868408,46.73049123)
\curveto(868.13868428,46.47048921)(867.99368443,46.25048943)(867.81368408,46.07049123)
\curveto(867.6236848,45.90048978)(867.39868502,45.76048992)(867.13868408,45.65049123)
\curveto(867.04868537,45.61049007)(866.95868546,45.5804901)(866.86868408,45.56049123)
\lineto(866.56868408,45.50049123)
\curveto(866.50868591,45.4804902)(866.45368597,45.47049021)(866.40368408,45.47049123)
\curveto(866.34368608,45.4804902)(866.27868614,45.47549021)(866.20868408,45.45549123)
\curveto(866.18868623,45.44549024)(866.16368626,45.44049024)(866.13368408,45.44049123)
\curveto(866.09368633,45.44049024)(866.05868636,45.43549025)(866.02868408,45.42549123)
\lineto(865.87868408,45.42549123)
\curveto(865.83868658,45.41549027)(865.79368663,45.41049027)(865.74368408,45.41049123)
\curveto(865.68368674,45.42049026)(865.62868679,45.42549026)(865.57868408,45.42549123)
\lineto(864.97868408,45.42549123)
\lineto(862.21868408,45.42549123)
\lineto(861.25868408,45.42549123)
\lineto(860.98868408,45.42549123)
\curveto(860.89869152,45.42549026)(860.8236916,45.44549024)(860.76368408,45.48549123)
\curveto(860.69369173,45.52549016)(860.64369178,45.60049008)(860.61368408,45.71049123)
\curveto(860.60369182,45.73048995)(860.60369182,45.75048993)(860.61368408,45.77049123)
\curveto(860.61369181,45.79048989)(860.60869181,45.81048987)(860.59868408,45.83049123)
}
}
{
\newrgbcolor{curcolor}{0 0 0}
\pscustom[linestyle=none,fillstyle=solid,fillcolor=curcolor]
{
\newpath
\moveto(857.64368408,54.28510061)
\curveto(857.64369478,54.41509899)(857.64369478,54.55009886)(857.64368408,54.69010061)
\curveto(857.64369478,54.84009857)(857.67869474,54.95009846)(857.74868408,55.02010061)
\curveto(857.8186946,55.07009834)(857.91369451,55.09509831)(858.03368408,55.09510061)
\curveto(858.14369428,55.1050983)(858.25869416,55.1100983)(858.37868408,55.11010061)
\lineto(859.71368408,55.11010061)
\lineto(865.78868408,55.11010061)
\lineto(867.46868408,55.11010061)
\lineto(867.85868408,55.11010061)
\curveto(867.99868442,55.1100983)(868.10868431,55.08509832)(868.18868408,55.03510061)
\curveto(868.23868418,55.0050984)(868.26868415,54.96009845)(868.27868408,54.90010061)
\curveto(868.28868413,54.85009856)(868.30368412,54.78509862)(868.32368408,54.70510061)
\lineto(868.32368408,54.49510061)
\lineto(868.32368408,54.18010061)
\curveto(868.31368411,54.08009933)(868.27868414,54.0050994)(868.21868408,53.95510061)
\curveto(868.13868428,53.9050995)(868.03868438,53.87509953)(867.91868408,53.86510061)
\lineto(867.54368408,53.86510061)
\lineto(866.16368408,53.86510061)
\lineto(859.92368408,53.86510061)
\lineto(858.45368408,53.86510061)
\curveto(858.34369408,53.86509954)(858.22869419,53.86009955)(858.10868408,53.85010061)
\curveto(857.97869444,53.85009956)(857.87869454,53.87509953)(857.80868408,53.92510061)
\curveto(857.74869467,53.96509944)(857.69869472,54.04009937)(857.65868408,54.15010061)
\curveto(857.64869477,54.17009924)(857.64869477,54.19009922)(857.65868408,54.21010061)
\curveto(857.65869476,54.24009917)(857.65369477,54.26509914)(857.64368408,54.28510061)
}
}
{
\newrgbcolor{curcolor}{0 0 0}
\pscustom[linestyle=none,fillstyle=solid,fillcolor=curcolor]
{
}
}
{
\newrgbcolor{curcolor}{0 0 0}
\pscustom[linestyle=none,fillstyle=solid,fillcolor=curcolor]
{
\newpath
\moveto(857.71868408,65.05510061)
\curveto(857.7186947,65.15509575)(857.72869469,65.25009566)(857.74868408,65.34010061)
\curveto(857.75869466,65.43009548)(857.78869463,65.49509541)(857.83868408,65.53510061)
\curveto(857.9186945,65.59509531)(858.0236944,65.62509528)(858.15368408,65.62510061)
\lineto(858.54368408,65.62510061)
\lineto(860.04368408,65.62510061)
\lineto(866.43368408,65.62510061)
\lineto(867.60368408,65.62510061)
\lineto(867.91868408,65.62510061)
\curveto(868.0186844,65.63509527)(868.09868432,65.62009529)(868.15868408,65.58010061)
\curveto(868.23868418,65.53009538)(868.28868413,65.45509545)(868.30868408,65.35510061)
\curveto(868.3186841,65.26509564)(868.3236841,65.15509575)(868.32368408,65.02510061)
\lineto(868.32368408,64.80010061)
\curveto(868.30368412,64.72009619)(868.28868413,64.65009626)(868.27868408,64.59010061)
\curveto(868.25868416,64.53009638)(868.2186842,64.48009643)(868.15868408,64.44010061)
\curveto(868.09868432,64.40009651)(868.0236844,64.38009653)(867.93368408,64.38010061)
\lineto(867.63368408,64.38010061)
\lineto(866.53868408,64.38010061)
\lineto(861.19868408,64.38010061)
\curveto(861.10869131,64.36009655)(861.03369139,64.34509656)(860.97368408,64.33510061)
\curveto(860.90369152,64.33509657)(860.84369158,64.3050966)(860.79368408,64.24510061)
\curveto(860.74369168,64.17509673)(860.7186917,64.08509682)(860.71868408,63.97510061)
\curveto(860.70869171,63.87509703)(860.70369172,63.76509714)(860.70368408,63.64510061)
\lineto(860.70368408,62.50510061)
\lineto(860.70368408,62.01010061)
\curveto(860.69369173,61.85009906)(860.63369179,61.74009917)(860.52368408,61.68010061)
\curveto(860.49369193,61.66009925)(860.46369196,61.65009926)(860.43368408,61.65010061)
\curveto(860.39369203,61.65009926)(860.34869207,61.64509926)(860.29868408,61.63510061)
\curveto(860.17869224,61.61509929)(860.06869235,61.62009929)(859.96868408,61.65010061)
\curveto(859.86869255,61.69009922)(859.79869262,61.74509916)(859.75868408,61.81510061)
\curveto(859.70869271,61.89509901)(859.68369274,62.01509889)(859.68368408,62.17510061)
\curveto(859.68369274,62.33509857)(859.66869275,62.47009844)(859.63868408,62.58010061)
\curveto(859.62869279,62.63009828)(859.6236928,62.68509822)(859.62368408,62.74510061)
\curveto(859.61369281,62.8050981)(859.59869282,62.86509804)(859.57868408,62.92510061)
\curveto(859.52869289,63.07509783)(859.47869294,63.22009769)(859.42868408,63.36010061)
\curveto(859.36869305,63.50009741)(859.29869312,63.63509727)(859.21868408,63.76510061)
\curveto(859.12869329,63.905097)(859.0236934,64.02509688)(858.90368408,64.12510061)
\curveto(858.78369364,64.22509668)(858.65369377,64.32009659)(858.51368408,64.41010061)
\curveto(858.41369401,64.47009644)(858.30369412,64.51509639)(858.18368408,64.54510061)
\curveto(858.06369436,64.58509632)(857.95869446,64.63509627)(857.86868408,64.69510061)
\curveto(857.80869461,64.74509616)(857.76869465,64.81509609)(857.74868408,64.90510061)
\curveto(857.73869468,64.92509598)(857.73369469,64.95009596)(857.73368408,64.98010061)
\curveto(857.73369469,65.0100959)(857.72869469,65.03509587)(857.71868408,65.05510061)
}
}
{
\newrgbcolor{curcolor}{0 0 0}
\pscustom[linestyle=none,fillstyle=solid,fillcolor=curcolor]
{
\newpath
\moveto(857.71868408,72.59470998)
\curveto(857.70869471,73.28470535)(857.82869459,73.88470475)(858.07868408,74.39470998)
\curveto(858.32869409,74.91470372)(858.66369376,75.30970332)(859.08368408,75.57970998)
\curveto(859.16369326,75.629703)(859.25369317,75.67470296)(859.35368408,75.71470998)
\curveto(859.44369298,75.75470288)(859.53869288,75.79970283)(859.63868408,75.84970998)
\curveto(859.73869268,75.88970274)(859.83869258,75.91970271)(859.93868408,75.93970998)
\curveto(860.03869238,75.95970267)(860.14369228,75.97970265)(860.25368408,75.99970998)
\curveto(860.30369212,76.01970261)(860.34869207,76.02470261)(860.38868408,76.01470998)
\curveto(860.42869199,76.00470263)(860.47369195,76.00970262)(860.52368408,76.02970998)
\curveto(860.57369185,76.03970259)(860.65869176,76.04470259)(860.77868408,76.04470998)
\curveto(860.88869153,76.04470259)(860.97369145,76.03970259)(861.03368408,76.02970998)
\curveto(861.09369133,76.00970262)(861.15369127,75.99970263)(861.21368408,75.99970998)
\curveto(861.27369115,76.00970262)(861.33369109,76.00470263)(861.39368408,75.98470998)
\curveto(861.53369089,75.94470269)(861.66869075,75.90970272)(861.79868408,75.87970998)
\curveto(861.92869049,75.84970278)(862.05369037,75.80970282)(862.17368408,75.75970998)
\curveto(862.31369011,75.69970293)(862.43868998,75.629703)(862.54868408,75.54970998)
\curveto(862.65868976,75.47970315)(862.76868965,75.40470323)(862.87868408,75.32470998)
\lineto(862.93868408,75.26470998)
\curveto(862.95868946,75.25470338)(862.97868944,75.23970339)(862.99868408,75.21970998)
\curveto(863.15868926,75.09970353)(863.30368912,74.96470367)(863.43368408,74.81470998)
\curveto(863.56368886,74.66470397)(863.68868873,74.50470413)(863.80868408,74.33470998)
\curveto(864.02868839,74.02470461)(864.23368819,73.7297049)(864.42368408,73.44970998)
\curveto(864.56368786,73.21970541)(864.69868772,72.98970564)(864.82868408,72.75970998)
\curveto(864.95868746,72.53970609)(865.09368733,72.31970631)(865.23368408,72.09970998)
\curveto(865.40368702,71.84970678)(865.58368684,71.60970702)(865.77368408,71.37970998)
\curveto(865.96368646,71.15970747)(866.18868623,70.96970766)(866.44868408,70.80970998)
\curveto(866.50868591,70.76970786)(866.56868585,70.7347079)(866.62868408,70.70470998)
\curveto(866.67868574,70.67470796)(866.74368568,70.64470799)(866.82368408,70.61470998)
\curveto(866.89368553,70.59470804)(866.95368547,70.58970804)(867.00368408,70.59970998)
\curveto(867.07368535,70.61970801)(867.12868529,70.65470798)(867.16868408,70.70470998)
\curveto(867.19868522,70.75470788)(867.2186852,70.81470782)(867.22868408,70.88470998)
\lineto(867.22868408,71.12470998)
\lineto(867.22868408,71.87470998)
\lineto(867.22868408,74.67970998)
\lineto(867.22868408,75.33970998)
\curveto(867.22868519,75.4297032)(867.23368519,75.51470312)(867.24368408,75.59470998)
\curveto(867.24368518,75.67470296)(867.26368516,75.73970289)(867.30368408,75.78970998)
\curveto(867.34368508,75.83970279)(867.418685,75.87970275)(867.52868408,75.90970998)
\curveto(867.62868479,75.94970268)(867.72868469,75.95970267)(867.82868408,75.93970998)
\lineto(867.96368408,75.93970998)
\curveto(868.03368439,75.91970271)(868.09368433,75.89970273)(868.14368408,75.87970998)
\curveto(868.19368423,75.85970277)(868.23368419,75.82470281)(868.26368408,75.77470998)
\curveto(868.30368412,75.72470291)(868.3236841,75.65470298)(868.32368408,75.56470998)
\lineto(868.32368408,75.29470998)
\lineto(868.32368408,74.39470998)
\lineto(868.32368408,70.88470998)
\lineto(868.32368408,69.81970998)
\curveto(868.3236841,69.73970889)(868.32868409,69.64970898)(868.33868408,69.54970998)
\curveto(868.33868408,69.44970918)(868.32868409,69.36470927)(868.30868408,69.29470998)
\curveto(868.23868418,69.08470955)(868.05868436,69.01970961)(867.76868408,69.09970998)
\curveto(867.72868469,69.10970952)(867.69368473,69.10970952)(867.66368408,69.09970998)
\curveto(867.6236848,69.09970953)(867.57868484,69.10970952)(867.52868408,69.12970998)
\curveto(867.44868497,69.14970948)(867.36368506,69.16970946)(867.27368408,69.18970998)
\curveto(867.18368524,69.20970942)(867.09868532,69.2347094)(867.01868408,69.26470998)
\curveto(866.52868589,69.42470921)(866.11368631,69.62470901)(865.77368408,69.86470998)
\curveto(865.5236869,70.04470859)(865.29868712,70.24970838)(865.09868408,70.47970998)
\curveto(864.88868753,70.70970792)(864.69368773,70.94970768)(864.51368408,71.19970998)
\curveto(864.33368809,71.45970717)(864.16368826,71.72470691)(864.00368408,71.99470998)
\curveto(863.83368859,72.27470636)(863.65868876,72.54470609)(863.47868408,72.80470998)
\curveto(863.39868902,72.91470572)(863.3236891,73.01970561)(863.25368408,73.11970998)
\curveto(863.18368924,73.2297054)(863.10868931,73.33970529)(863.02868408,73.44970998)
\curveto(862.99868942,73.48970514)(862.96868945,73.52470511)(862.93868408,73.55470998)
\curveto(862.89868952,73.59470504)(862.86868955,73.634705)(862.84868408,73.67470998)
\curveto(862.73868968,73.81470482)(862.61368981,73.93970469)(862.47368408,74.04970998)
\curveto(862.44368998,74.06970456)(862.41869,74.09470454)(862.39868408,74.12470998)
\curveto(862.36869005,74.15470448)(862.33869008,74.17970445)(862.30868408,74.19970998)
\curveto(862.20869021,74.27970435)(862.10869031,74.34470429)(862.00868408,74.39470998)
\curveto(861.90869051,74.45470418)(861.79869062,74.50970412)(861.67868408,74.55970998)
\curveto(861.60869081,74.58970404)(861.53369089,74.60970402)(861.45368408,74.61970998)
\lineto(861.21368408,74.67970998)
\lineto(861.12368408,74.67970998)
\curveto(861.09369133,74.68970394)(861.06369136,74.69470394)(861.03368408,74.69470998)
\curveto(860.96369146,74.71470392)(860.86869155,74.71970391)(860.74868408,74.70970998)
\curveto(860.6186918,74.70970392)(860.5186919,74.69970393)(860.44868408,74.67970998)
\curveto(860.36869205,74.65970397)(860.29369213,74.63970399)(860.22368408,74.61970998)
\curveto(860.14369228,74.60970402)(860.06369236,74.58970404)(859.98368408,74.55970998)
\curveto(859.74369268,74.44970418)(859.54369288,74.29970433)(859.38368408,74.10970998)
\curveto(859.21369321,73.9297047)(859.07369335,73.70970492)(858.96368408,73.44970998)
\curveto(858.94369348,73.37970525)(858.92869349,73.30970532)(858.91868408,73.23970998)
\curveto(858.89869352,73.16970546)(858.87869354,73.09470554)(858.85868408,73.01470998)
\curveto(858.83869358,72.9347057)(858.82869359,72.82470581)(858.82868408,72.68470998)
\curveto(858.82869359,72.55470608)(858.83869358,72.44970618)(858.85868408,72.36970998)
\curveto(858.86869355,72.30970632)(858.87369355,72.25470638)(858.87368408,72.20470998)
\curveto(858.87369355,72.15470648)(858.88369354,72.10470653)(858.90368408,72.05470998)
\curveto(858.94369348,71.95470668)(858.98369344,71.85970677)(859.02368408,71.76970998)
\curveto(859.06369336,71.68970694)(859.10869331,71.60970702)(859.15868408,71.52970998)
\curveto(859.17869324,71.49970713)(859.20369322,71.46970716)(859.23368408,71.43970998)
\curveto(859.26369316,71.41970721)(859.28869313,71.39470724)(859.30868408,71.36470998)
\lineto(859.38368408,71.28970998)
\curveto(859.40369302,71.25970737)(859.423693,71.2347074)(859.44368408,71.21470998)
\lineto(859.65368408,71.06470998)
\curveto(859.71369271,71.02470761)(859.77869264,70.97970765)(859.84868408,70.92970998)
\curveto(859.93869248,70.86970776)(860.04369238,70.81970781)(860.16368408,70.77970998)
\curveto(860.27369215,70.74970788)(860.38369204,70.71470792)(860.49368408,70.67470998)
\curveto(860.60369182,70.634708)(860.74869167,70.60970802)(860.92868408,70.59970998)
\curveto(861.09869132,70.58970804)(861.2236912,70.55970807)(861.30368408,70.50970998)
\curveto(861.38369104,70.45970817)(861.42869099,70.38470825)(861.43868408,70.28470998)
\curveto(861.44869097,70.18470845)(861.45369097,70.07470856)(861.45368408,69.95470998)
\curveto(861.45369097,69.91470872)(861.45869096,69.87470876)(861.46868408,69.83470998)
\curveto(861.46869095,69.79470884)(861.46369096,69.75970887)(861.45368408,69.72970998)
\curveto(861.43369099,69.67970895)(861.423691,69.629709)(861.42368408,69.57970998)
\curveto(861.423691,69.53970909)(861.41369101,69.49970913)(861.39368408,69.45970998)
\curveto(861.33369109,69.36970926)(861.19869122,69.32470931)(860.98868408,69.32470998)
\lineto(860.86868408,69.32470998)
\curveto(860.80869161,69.3347093)(860.74869167,69.33970929)(860.68868408,69.33970998)
\curveto(860.6186918,69.34970928)(860.55369187,69.35970927)(860.49368408,69.36970998)
\curveto(860.38369204,69.38970924)(860.28369214,69.40970922)(860.19368408,69.42970998)
\curveto(860.09369233,69.44970918)(859.99869242,69.47970915)(859.90868408,69.51970998)
\curveto(859.83869258,69.53970909)(859.77869264,69.55970907)(859.72868408,69.57970998)
\lineto(859.54868408,69.63970998)
\curveto(859.28869313,69.75970887)(859.04369338,69.91470872)(858.81368408,70.10470998)
\curveto(858.58369384,70.30470833)(858.39869402,70.51970811)(858.25868408,70.74970998)
\curveto(858.17869424,70.85970777)(858.11369431,70.97470766)(858.06368408,71.09470998)
\lineto(857.91368408,71.48470998)
\curveto(857.86369456,71.59470704)(857.83369459,71.70970692)(857.82368408,71.82970998)
\curveto(857.80369462,71.94970668)(857.77869464,72.07470656)(857.74868408,72.20470998)
\curveto(857.74869467,72.27470636)(857.74869467,72.33970629)(857.74868408,72.39970998)
\curveto(857.73869468,72.45970617)(857.72869469,72.52470611)(857.71868408,72.59470998)
}
}
{
\newrgbcolor{curcolor}{0 0 0}
\pscustom[linestyle=none,fillstyle=solid,fillcolor=curcolor]
{
\newpath
\moveto(866.68868408,78.63431936)
\lineto(866.68868408,79.26431936)
\lineto(866.68868408,79.45931936)
\curveto(866.68868573,79.52931683)(866.69868572,79.58931677)(866.71868408,79.63931936)
\curveto(866.75868566,79.70931665)(866.79868562,79.7593166)(866.83868408,79.78931936)
\curveto(866.88868553,79.82931653)(866.95368547,79.84931651)(867.03368408,79.84931936)
\curveto(867.11368531,79.8593165)(867.19868522,79.86431649)(867.28868408,79.86431936)
\lineto(868.00868408,79.86431936)
\curveto(868.48868393,79.86431649)(868.89868352,79.80431655)(869.23868408,79.68431936)
\curveto(869.57868284,79.56431679)(869.85368257,79.36931699)(870.06368408,79.09931936)
\curveto(870.11368231,79.02931733)(870.15868226,78.9593174)(870.19868408,78.88931936)
\curveto(870.24868217,78.82931753)(870.29368213,78.7543176)(870.33368408,78.66431936)
\curveto(870.34368208,78.64431771)(870.35368207,78.61431774)(870.36368408,78.57431936)
\curveto(870.38368204,78.53431782)(870.38868203,78.48931787)(870.37868408,78.43931936)
\curveto(870.34868207,78.34931801)(870.27368215,78.29431806)(870.15368408,78.27431936)
\curveto(870.04368238,78.2543181)(869.94868247,78.26931809)(869.86868408,78.31931936)
\curveto(869.79868262,78.34931801)(869.73368269,78.39431796)(869.67368408,78.45431936)
\curveto(869.6236828,78.52431783)(869.57368285,78.58931777)(869.52368408,78.64931936)
\curveto(869.47368295,78.71931764)(869.39868302,78.77931758)(869.29868408,78.82931936)
\curveto(869.20868321,78.88931747)(869.1186833,78.93931742)(869.02868408,78.97931936)
\curveto(868.99868342,78.99931736)(868.93868348,79.02431733)(868.84868408,79.05431936)
\curveto(868.76868365,79.08431727)(868.69868372,79.08931727)(868.63868408,79.06931936)
\curveto(868.49868392,79.03931732)(868.40868401,78.97931738)(868.36868408,78.88931936)
\curveto(868.33868408,78.80931755)(868.3236841,78.71931764)(868.32368408,78.61931936)
\curveto(868.3236841,78.51931784)(868.29868412,78.43431792)(868.24868408,78.36431936)
\curveto(868.17868424,78.27431808)(868.03868438,78.22931813)(867.82868408,78.22931936)
\lineto(867.27368408,78.22931936)
\lineto(867.04868408,78.22931936)
\curveto(866.96868545,78.23931812)(866.90368552,78.2593181)(866.85368408,78.28931936)
\curveto(866.77368565,78.34931801)(866.72868569,78.41931794)(866.71868408,78.49931936)
\curveto(866.70868571,78.51931784)(866.70368572,78.53931782)(866.70368408,78.55931936)
\curveto(866.70368572,78.58931777)(866.69868572,78.61431774)(866.68868408,78.63431936)
}
}
{
\newrgbcolor{curcolor}{0 0 0}
\pscustom[linestyle=none,fillstyle=solid,fillcolor=curcolor]
{
}
}
{
\newrgbcolor{curcolor}{0 0 0}
\pscustom[linestyle=none,fillstyle=solid,fillcolor=curcolor]
{
\newpath
\moveto(857.71868408,89.26463186)
\curveto(857.70869471,89.95462722)(857.82869459,90.55462662)(858.07868408,91.06463186)
\curveto(858.32869409,91.58462559)(858.66369376,91.9796252)(859.08368408,92.24963186)
\curveto(859.16369326,92.29962488)(859.25369317,92.34462483)(859.35368408,92.38463186)
\curveto(859.44369298,92.42462475)(859.53869288,92.46962471)(859.63868408,92.51963186)
\curveto(859.73869268,92.55962462)(859.83869258,92.58962459)(859.93868408,92.60963186)
\curveto(860.03869238,92.62962455)(860.14369228,92.64962453)(860.25368408,92.66963186)
\curveto(860.30369212,92.68962449)(860.34869207,92.69462448)(860.38868408,92.68463186)
\curveto(860.42869199,92.6746245)(860.47369195,92.6796245)(860.52368408,92.69963186)
\curveto(860.57369185,92.70962447)(860.65869176,92.71462446)(860.77868408,92.71463186)
\curveto(860.88869153,92.71462446)(860.97369145,92.70962447)(861.03368408,92.69963186)
\curveto(861.09369133,92.6796245)(861.15369127,92.66962451)(861.21368408,92.66963186)
\curveto(861.27369115,92.6796245)(861.33369109,92.6746245)(861.39368408,92.65463186)
\curveto(861.53369089,92.61462456)(861.66869075,92.5796246)(861.79868408,92.54963186)
\curveto(861.92869049,92.51962466)(862.05369037,92.4796247)(862.17368408,92.42963186)
\curveto(862.31369011,92.36962481)(862.43868998,92.29962488)(862.54868408,92.21963186)
\curveto(862.65868976,92.14962503)(862.76868965,92.0746251)(862.87868408,91.99463186)
\lineto(862.93868408,91.93463186)
\curveto(862.95868946,91.92462525)(862.97868944,91.90962527)(862.99868408,91.88963186)
\curveto(863.15868926,91.76962541)(863.30368912,91.63462554)(863.43368408,91.48463186)
\curveto(863.56368886,91.33462584)(863.68868873,91.174626)(863.80868408,91.00463186)
\curveto(864.02868839,90.69462648)(864.23368819,90.39962678)(864.42368408,90.11963186)
\curveto(864.56368786,89.88962729)(864.69868772,89.65962752)(864.82868408,89.42963186)
\curveto(864.95868746,89.20962797)(865.09368733,88.98962819)(865.23368408,88.76963186)
\curveto(865.40368702,88.51962866)(865.58368684,88.2796289)(865.77368408,88.04963186)
\curveto(865.96368646,87.82962935)(866.18868623,87.63962954)(866.44868408,87.47963186)
\curveto(866.50868591,87.43962974)(866.56868585,87.40462977)(866.62868408,87.37463186)
\curveto(866.67868574,87.34462983)(866.74368568,87.31462986)(866.82368408,87.28463186)
\curveto(866.89368553,87.26462991)(866.95368547,87.25962992)(867.00368408,87.26963186)
\curveto(867.07368535,87.28962989)(867.12868529,87.32462985)(867.16868408,87.37463186)
\curveto(867.19868522,87.42462975)(867.2186852,87.48462969)(867.22868408,87.55463186)
\lineto(867.22868408,87.79463186)
\lineto(867.22868408,88.54463186)
\lineto(867.22868408,91.34963186)
\lineto(867.22868408,92.00963186)
\curveto(867.22868519,92.09962508)(867.23368519,92.18462499)(867.24368408,92.26463186)
\curveto(867.24368518,92.34462483)(867.26368516,92.40962477)(867.30368408,92.45963186)
\curveto(867.34368508,92.50962467)(867.418685,92.54962463)(867.52868408,92.57963186)
\curveto(867.62868479,92.61962456)(867.72868469,92.62962455)(867.82868408,92.60963186)
\lineto(867.96368408,92.60963186)
\curveto(868.03368439,92.58962459)(868.09368433,92.56962461)(868.14368408,92.54963186)
\curveto(868.19368423,92.52962465)(868.23368419,92.49462468)(868.26368408,92.44463186)
\curveto(868.30368412,92.39462478)(868.3236841,92.32462485)(868.32368408,92.23463186)
\lineto(868.32368408,91.96463186)
\lineto(868.32368408,91.06463186)
\lineto(868.32368408,87.55463186)
\lineto(868.32368408,86.48963186)
\curveto(868.3236841,86.40963077)(868.32868409,86.31963086)(868.33868408,86.21963186)
\curveto(868.33868408,86.11963106)(868.32868409,86.03463114)(868.30868408,85.96463186)
\curveto(868.23868418,85.75463142)(868.05868436,85.68963149)(867.76868408,85.76963186)
\curveto(867.72868469,85.7796314)(867.69368473,85.7796314)(867.66368408,85.76963186)
\curveto(867.6236848,85.76963141)(867.57868484,85.7796314)(867.52868408,85.79963186)
\curveto(867.44868497,85.81963136)(867.36368506,85.83963134)(867.27368408,85.85963186)
\curveto(867.18368524,85.8796313)(867.09868532,85.90463127)(867.01868408,85.93463186)
\curveto(866.52868589,86.09463108)(866.11368631,86.29463088)(865.77368408,86.53463186)
\curveto(865.5236869,86.71463046)(865.29868712,86.91963026)(865.09868408,87.14963186)
\curveto(864.88868753,87.3796298)(864.69368773,87.61962956)(864.51368408,87.86963186)
\curveto(864.33368809,88.12962905)(864.16368826,88.39462878)(864.00368408,88.66463186)
\curveto(863.83368859,88.94462823)(863.65868876,89.21462796)(863.47868408,89.47463186)
\curveto(863.39868902,89.58462759)(863.3236891,89.68962749)(863.25368408,89.78963186)
\curveto(863.18368924,89.89962728)(863.10868931,90.00962717)(863.02868408,90.11963186)
\curveto(862.99868942,90.15962702)(862.96868945,90.19462698)(862.93868408,90.22463186)
\curveto(862.89868952,90.26462691)(862.86868955,90.30462687)(862.84868408,90.34463186)
\curveto(862.73868968,90.48462669)(862.61368981,90.60962657)(862.47368408,90.71963186)
\curveto(862.44368998,90.73962644)(862.41869,90.76462641)(862.39868408,90.79463186)
\curveto(862.36869005,90.82462635)(862.33869008,90.84962633)(862.30868408,90.86963186)
\curveto(862.20869021,90.94962623)(862.10869031,91.01462616)(862.00868408,91.06463186)
\curveto(861.90869051,91.12462605)(861.79869062,91.179626)(861.67868408,91.22963186)
\curveto(861.60869081,91.25962592)(861.53369089,91.2796259)(861.45368408,91.28963186)
\lineto(861.21368408,91.34963186)
\lineto(861.12368408,91.34963186)
\curveto(861.09369133,91.35962582)(861.06369136,91.36462581)(861.03368408,91.36463186)
\curveto(860.96369146,91.38462579)(860.86869155,91.38962579)(860.74868408,91.37963186)
\curveto(860.6186918,91.3796258)(860.5186919,91.36962581)(860.44868408,91.34963186)
\curveto(860.36869205,91.32962585)(860.29369213,91.30962587)(860.22368408,91.28963186)
\curveto(860.14369228,91.2796259)(860.06369236,91.25962592)(859.98368408,91.22963186)
\curveto(859.74369268,91.11962606)(859.54369288,90.96962621)(859.38368408,90.77963186)
\curveto(859.21369321,90.59962658)(859.07369335,90.3796268)(858.96368408,90.11963186)
\curveto(858.94369348,90.04962713)(858.92869349,89.9796272)(858.91868408,89.90963186)
\curveto(858.89869352,89.83962734)(858.87869354,89.76462741)(858.85868408,89.68463186)
\curveto(858.83869358,89.60462757)(858.82869359,89.49462768)(858.82868408,89.35463186)
\curveto(858.82869359,89.22462795)(858.83869358,89.11962806)(858.85868408,89.03963186)
\curveto(858.86869355,88.9796282)(858.87369355,88.92462825)(858.87368408,88.87463186)
\curveto(858.87369355,88.82462835)(858.88369354,88.7746284)(858.90368408,88.72463186)
\curveto(858.94369348,88.62462855)(858.98369344,88.52962865)(859.02368408,88.43963186)
\curveto(859.06369336,88.35962882)(859.10869331,88.2796289)(859.15868408,88.19963186)
\curveto(859.17869324,88.16962901)(859.20369322,88.13962904)(859.23368408,88.10963186)
\curveto(859.26369316,88.08962909)(859.28869313,88.06462911)(859.30868408,88.03463186)
\lineto(859.38368408,87.95963186)
\curveto(859.40369302,87.92962925)(859.423693,87.90462927)(859.44368408,87.88463186)
\lineto(859.65368408,87.73463186)
\curveto(859.71369271,87.69462948)(859.77869264,87.64962953)(859.84868408,87.59963186)
\curveto(859.93869248,87.53962964)(860.04369238,87.48962969)(860.16368408,87.44963186)
\curveto(860.27369215,87.41962976)(860.38369204,87.38462979)(860.49368408,87.34463186)
\curveto(860.60369182,87.30462987)(860.74869167,87.2796299)(860.92868408,87.26963186)
\curveto(861.09869132,87.25962992)(861.2236912,87.22962995)(861.30368408,87.17963186)
\curveto(861.38369104,87.12963005)(861.42869099,87.05463012)(861.43868408,86.95463186)
\curveto(861.44869097,86.85463032)(861.45369097,86.74463043)(861.45368408,86.62463186)
\curveto(861.45369097,86.58463059)(861.45869096,86.54463063)(861.46868408,86.50463186)
\curveto(861.46869095,86.46463071)(861.46369096,86.42963075)(861.45368408,86.39963186)
\curveto(861.43369099,86.34963083)(861.423691,86.29963088)(861.42368408,86.24963186)
\curveto(861.423691,86.20963097)(861.41369101,86.16963101)(861.39368408,86.12963186)
\curveto(861.33369109,86.03963114)(861.19869122,85.99463118)(860.98868408,85.99463186)
\lineto(860.86868408,85.99463186)
\curveto(860.80869161,86.00463117)(860.74869167,86.00963117)(860.68868408,86.00963186)
\curveto(860.6186918,86.01963116)(860.55369187,86.02963115)(860.49368408,86.03963186)
\curveto(860.38369204,86.05963112)(860.28369214,86.0796311)(860.19368408,86.09963186)
\curveto(860.09369233,86.11963106)(859.99869242,86.14963103)(859.90868408,86.18963186)
\curveto(859.83869258,86.20963097)(859.77869264,86.22963095)(859.72868408,86.24963186)
\lineto(859.54868408,86.30963186)
\curveto(859.28869313,86.42963075)(859.04369338,86.58463059)(858.81368408,86.77463186)
\curveto(858.58369384,86.9746302)(858.39869402,87.18962999)(858.25868408,87.41963186)
\curveto(858.17869424,87.52962965)(858.11369431,87.64462953)(858.06368408,87.76463186)
\lineto(857.91368408,88.15463186)
\curveto(857.86369456,88.26462891)(857.83369459,88.3796288)(857.82368408,88.49963186)
\curveto(857.80369462,88.61962856)(857.77869464,88.74462843)(857.74868408,88.87463186)
\curveto(857.74869467,88.94462823)(857.74869467,89.00962817)(857.74868408,89.06963186)
\curveto(857.73869468,89.12962805)(857.72869469,89.19462798)(857.71868408,89.26463186)
}
}
{
\newrgbcolor{curcolor}{0 0 0}
\pscustom[linestyle=none,fillstyle=solid,fillcolor=curcolor]
{
\newpath
\moveto(863.23868408,101.36424123)
\lineto(863.49368408,101.36424123)
\curveto(863.57368885,101.37423353)(863.64868877,101.36923353)(863.71868408,101.34924123)
\lineto(863.95868408,101.34924123)
\lineto(864.12368408,101.34924123)
\curveto(864.2236882,101.32923357)(864.32868809,101.31923358)(864.43868408,101.31924123)
\curveto(864.53868788,101.31923358)(864.63868778,101.30923359)(864.73868408,101.28924123)
\lineto(864.88868408,101.28924123)
\curveto(865.02868739,101.25923364)(865.16868725,101.23923366)(865.30868408,101.22924123)
\curveto(865.43868698,101.21923368)(865.56868685,101.19423371)(865.69868408,101.15424123)
\curveto(865.77868664,101.13423377)(865.86368656,101.11423379)(865.95368408,101.09424123)
\lineto(866.19368408,101.03424123)
\lineto(866.49368408,100.91424123)
\curveto(866.58368584,100.88423402)(866.67368575,100.84923405)(866.76368408,100.80924123)
\curveto(866.98368544,100.70923419)(867.19868522,100.57423433)(867.40868408,100.40424123)
\curveto(867.6186848,100.24423466)(867.78868463,100.06923483)(867.91868408,99.87924123)
\curveto(867.95868446,99.82923507)(867.99868442,99.76923513)(868.03868408,99.69924123)
\curveto(868.06868435,99.63923526)(868.10368432,99.57923532)(868.14368408,99.51924123)
\curveto(868.19368423,99.43923546)(868.23368419,99.34423556)(868.26368408,99.23424123)
\curveto(868.29368413,99.12423578)(868.3236841,99.01923588)(868.35368408,98.91924123)
\curveto(868.39368403,98.80923609)(868.418684,98.6992362)(868.42868408,98.58924123)
\curveto(868.43868398,98.47923642)(868.45368397,98.36423654)(868.47368408,98.24424123)
\curveto(868.48368394,98.2042367)(868.48368394,98.15923674)(868.47368408,98.10924123)
\curveto(868.47368395,98.06923683)(868.47868394,98.02923687)(868.48868408,97.98924123)
\curveto(868.49868392,97.94923695)(868.50368392,97.89423701)(868.50368408,97.82424123)
\curveto(868.50368392,97.75423715)(868.49868392,97.7042372)(868.48868408,97.67424123)
\curveto(868.46868395,97.62423728)(868.46368396,97.57923732)(868.47368408,97.53924123)
\curveto(868.48368394,97.4992374)(868.48368394,97.46423744)(868.47368408,97.43424123)
\lineto(868.47368408,97.34424123)
\curveto(868.45368397,97.28423762)(868.43868398,97.21923768)(868.42868408,97.14924123)
\curveto(868.42868399,97.08923781)(868.423684,97.02423788)(868.41368408,96.95424123)
\curveto(868.36368406,96.78423812)(868.31368411,96.62423828)(868.26368408,96.47424123)
\curveto(868.21368421,96.32423858)(868.14868427,96.17923872)(868.06868408,96.03924123)
\curveto(868.02868439,95.98923891)(867.99868442,95.93423897)(867.97868408,95.87424123)
\curveto(867.94868447,95.82423908)(867.91368451,95.77423913)(867.87368408,95.72424123)
\curveto(867.69368473,95.48423942)(867.47368495,95.28423962)(867.21368408,95.12424123)
\curveto(866.95368547,94.96423994)(866.66868575,94.82424008)(866.35868408,94.70424123)
\curveto(866.2186862,94.64424026)(866.07868634,94.5992403)(865.93868408,94.56924123)
\curveto(865.78868663,94.53924036)(865.63368679,94.5042404)(865.47368408,94.46424123)
\curveto(865.36368706,94.44424046)(865.25368717,94.42924047)(865.14368408,94.41924123)
\curveto(865.03368739,94.40924049)(864.9236875,94.39424051)(864.81368408,94.37424123)
\curveto(864.77368765,94.36424054)(864.73368769,94.35924054)(864.69368408,94.35924123)
\curveto(864.65368777,94.36924053)(864.61368781,94.36924053)(864.57368408,94.35924123)
\curveto(864.5236879,94.34924055)(864.47368795,94.34424056)(864.42368408,94.34424123)
\lineto(864.25868408,94.34424123)
\curveto(864.20868821,94.32424058)(864.15868826,94.31924058)(864.10868408,94.32924123)
\curveto(864.04868837,94.33924056)(863.99368843,94.33924056)(863.94368408,94.32924123)
\curveto(863.90368852,94.31924058)(863.85868856,94.31924058)(863.80868408,94.32924123)
\curveto(863.75868866,94.33924056)(863.70868871,94.33424057)(863.65868408,94.31424123)
\curveto(863.58868883,94.29424061)(863.51368891,94.28924061)(863.43368408,94.29924123)
\curveto(863.34368908,94.30924059)(863.25868916,94.31424059)(863.17868408,94.31424123)
\curveto(863.08868933,94.31424059)(862.98868943,94.30924059)(862.87868408,94.29924123)
\curveto(862.75868966,94.28924061)(862.65868976,94.29424061)(862.57868408,94.31424123)
\lineto(862.29368408,94.31424123)
\lineto(861.66368408,94.35924123)
\curveto(861.56369086,94.36924053)(861.46869095,94.37924052)(861.37868408,94.38924123)
\lineto(861.07868408,94.41924123)
\curveto(861.02869139,94.43924046)(860.97869144,94.44424046)(860.92868408,94.43424123)
\curveto(860.86869155,94.43424047)(860.81369161,94.44424046)(860.76368408,94.46424123)
\curveto(860.59369183,94.51424039)(860.42869199,94.55424035)(860.26868408,94.58424123)
\curveto(860.09869232,94.61424029)(859.93869248,94.66424024)(859.78868408,94.73424123)
\curveto(859.32869309,94.92423998)(858.95369347,95.14423976)(858.66368408,95.39424123)
\curveto(858.37369405,95.65423925)(858.12869429,96.01423889)(857.92868408,96.47424123)
\curveto(857.87869454,96.6042383)(857.84369458,96.73423817)(857.82368408,96.86424123)
\curveto(857.80369462,97.0042379)(857.77869464,97.14423776)(857.74868408,97.28424123)
\curveto(857.73869468,97.35423755)(857.73369469,97.41923748)(857.73368408,97.47924123)
\curveto(857.73369469,97.53923736)(857.72869469,97.6042373)(857.71868408,97.67424123)
\curveto(857.69869472,98.5042364)(857.84869457,99.17423573)(858.16868408,99.68424123)
\curveto(858.47869394,100.19423471)(858.9186935,100.57423433)(859.48868408,100.82424123)
\curveto(859.60869281,100.87423403)(859.73369269,100.91923398)(859.86368408,100.95924123)
\curveto(859.99369243,100.9992339)(860.12869229,101.04423386)(860.26868408,101.09424123)
\curveto(860.34869207,101.11423379)(860.43369199,101.12923377)(860.52368408,101.13924123)
\lineto(860.76368408,101.19924123)
\curveto(860.87369155,101.22923367)(860.98369144,101.24423366)(861.09368408,101.24424123)
\curveto(861.20369122,101.25423365)(861.31369111,101.26923363)(861.42368408,101.28924123)
\curveto(861.47369095,101.30923359)(861.5186909,101.31423359)(861.55868408,101.30424123)
\curveto(861.59869082,101.3042336)(861.63869078,101.30923359)(861.67868408,101.31924123)
\curveto(861.72869069,101.32923357)(861.78369064,101.32923357)(861.84368408,101.31924123)
\curveto(861.89369053,101.31923358)(861.94369048,101.32423358)(861.99368408,101.33424123)
\lineto(862.12868408,101.33424123)
\curveto(862.18869023,101.35423355)(862.25869016,101.35423355)(862.33868408,101.33424123)
\curveto(862.40869001,101.32423358)(862.47368995,101.32923357)(862.53368408,101.34924123)
\curveto(862.56368986,101.35923354)(862.60368982,101.36423354)(862.65368408,101.36424123)
\lineto(862.77368408,101.36424123)
\lineto(863.23868408,101.36424123)
\moveto(865.56368408,99.81924123)
\curveto(865.24368718,99.91923498)(864.87868754,99.97923492)(864.46868408,99.99924123)
\curveto(864.05868836,100.01923488)(863.64868877,100.02923487)(863.23868408,100.02924123)
\curveto(862.80868961,100.02923487)(862.38869003,100.01923488)(861.97868408,99.99924123)
\curveto(861.56869085,99.97923492)(861.18369124,99.93423497)(860.82368408,99.86424123)
\curveto(860.46369196,99.79423511)(860.14369228,99.68423522)(859.86368408,99.53424123)
\curveto(859.57369285,99.39423551)(859.33869308,99.1992357)(859.15868408,98.94924123)
\curveto(859.04869337,98.78923611)(858.96869345,98.60923629)(858.91868408,98.40924123)
\curveto(858.85869356,98.20923669)(858.82869359,97.96423694)(858.82868408,97.67424123)
\curveto(858.84869357,97.65423725)(858.85869356,97.61923728)(858.85868408,97.56924123)
\curveto(858.84869357,97.51923738)(858.84869357,97.47923742)(858.85868408,97.44924123)
\curveto(858.87869354,97.36923753)(858.89869352,97.29423761)(858.91868408,97.22424123)
\curveto(858.92869349,97.16423774)(858.94869347,97.0992378)(858.97868408,97.02924123)
\curveto(859.09869332,96.75923814)(859.26869315,96.53923836)(859.48868408,96.36924123)
\curveto(859.69869272,96.20923869)(859.94369248,96.07423883)(860.22368408,95.96424123)
\curveto(860.33369209,95.91423899)(860.45369197,95.87423903)(860.58368408,95.84424123)
\curveto(860.70369172,95.82423908)(860.82869159,95.7992391)(860.95868408,95.76924123)
\curveto(861.00869141,95.74923915)(861.06369136,95.73923916)(861.12368408,95.73924123)
\curveto(861.17369125,95.73923916)(861.2236912,95.73423917)(861.27368408,95.72424123)
\curveto(861.36369106,95.71423919)(861.45869096,95.7042392)(861.55868408,95.69424123)
\curveto(861.64869077,95.68423922)(861.74369068,95.67423923)(861.84368408,95.66424123)
\curveto(861.9236905,95.66423924)(862.00869041,95.65923924)(862.09868408,95.64924123)
\lineto(862.33868408,95.64924123)
\lineto(862.51868408,95.64924123)
\curveto(862.54868987,95.63923926)(862.58368984,95.63423927)(862.62368408,95.63424123)
\lineto(862.75868408,95.63424123)
\lineto(863.20868408,95.63424123)
\curveto(863.28868913,95.63423927)(863.37368905,95.62923927)(863.46368408,95.61924123)
\curveto(863.54368888,95.61923928)(863.6186888,95.62923927)(863.68868408,95.64924123)
\lineto(863.95868408,95.64924123)
\curveto(863.97868844,95.64923925)(864.00868841,95.64423926)(864.04868408,95.63424123)
\curveto(864.07868834,95.63423927)(864.10368832,95.63923926)(864.12368408,95.64924123)
\curveto(864.2236882,95.65923924)(864.3236881,95.66423924)(864.42368408,95.66424123)
\curveto(864.51368791,95.67423923)(864.61368781,95.68423922)(864.72368408,95.69424123)
\curveto(864.84368758,95.72423918)(864.96868745,95.73923916)(865.09868408,95.73924123)
\curveto(865.2186872,95.74923915)(865.33368709,95.77423913)(865.44368408,95.81424123)
\curveto(865.74368668,95.89423901)(866.00868641,95.97923892)(866.23868408,96.06924123)
\curveto(866.46868595,96.16923873)(866.68368574,96.31423859)(866.88368408,96.50424123)
\curveto(867.08368534,96.71423819)(867.23368519,96.97923792)(867.33368408,97.29924123)
\curveto(867.35368507,97.33923756)(867.36368506,97.37423753)(867.36368408,97.40424123)
\curveto(867.35368507,97.44423746)(867.35868506,97.48923741)(867.37868408,97.53924123)
\curveto(867.38868503,97.57923732)(867.39868502,97.64923725)(867.40868408,97.74924123)
\curveto(867.418685,97.85923704)(867.41368501,97.94423696)(867.39368408,98.00424123)
\curveto(867.37368505,98.07423683)(867.36368506,98.14423676)(867.36368408,98.21424123)
\curveto(867.35368507,98.28423662)(867.33868508,98.34923655)(867.31868408,98.40924123)
\curveto(867.25868516,98.60923629)(867.17368525,98.78923611)(867.06368408,98.94924123)
\curveto(867.04368538,98.97923592)(867.0236854,99.0042359)(867.00368408,99.02424123)
\lineto(866.94368408,99.08424123)
\curveto(866.9236855,99.12423578)(866.88368554,99.17423573)(866.82368408,99.23424123)
\curveto(866.68368574,99.33423557)(866.55368587,99.41923548)(866.43368408,99.48924123)
\curveto(866.31368611,99.55923534)(866.16868625,99.62923527)(865.99868408,99.69924123)
\curveto(865.92868649,99.72923517)(865.85868656,99.74923515)(865.78868408,99.75924123)
\curveto(865.7186867,99.77923512)(865.64368678,99.7992351)(865.56368408,99.81924123)
}
}
{
\newrgbcolor{curcolor}{0 0 0}
\pscustom[linestyle=none,fillstyle=solid,fillcolor=curcolor]
{
\newpath
\moveto(857.71868408,106.77385061)
\curveto(857.7186947,106.87384575)(857.72869469,106.96884566)(857.74868408,107.05885061)
\curveto(857.75869466,107.14884548)(857.78869463,107.21384541)(857.83868408,107.25385061)
\curveto(857.9186945,107.31384531)(858.0236944,107.34384528)(858.15368408,107.34385061)
\lineto(858.54368408,107.34385061)
\lineto(860.04368408,107.34385061)
\lineto(866.43368408,107.34385061)
\lineto(867.60368408,107.34385061)
\lineto(867.91868408,107.34385061)
\curveto(868.0186844,107.35384527)(868.09868432,107.33884529)(868.15868408,107.29885061)
\curveto(868.23868418,107.24884538)(868.28868413,107.17384545)(868.30868408,107.07385061)
\curveto(868.3186841,106.98384564)(868.3236841,106.87384575)(868.32368408,106.74385061)
\lineto(868.32368408,106.51885061)
\curveto(868.30368412,106.43884619)(868.28868413,106.36884626)(868.27868408,106.30885061)
\curveto(868.25868416,106.24884638)(868.2186842,106.19884643)(868.15868408,106.15885061)
\curveto(868.09868432,106.11884651)(868.0236844,106.09884653)(867.93368408,106.09885061)
\lineto(867.63368408,106.09885061)
\lineto(866.53868408,106.09885061)
\lineto(861.19868408,106.09885061)
\curveto(861.10869131,106.07884655)(861.03369139,106.06384656)(860.97368408,106.05385061)
\curveto(860.90369152,106.05384657)(860.84369158,106.0238466)(860.79368408,105.96385061)
\curveto(860.74369168,105.89384673)(860.7186917,105.80384682)(860.71868408,105.69385061)
\curveto(860.70869171,105.59384703)(860.70369172,105.48384714)(860.70368408,105.36385061)
\lineto(860.70368408,104.22385061)
\lineto(860.70368408,103.72885061)
\curveto(860.69369173,103.56884906)(860.63369179,103.45884917)(860.52368408,103.39885061)
\curveto(860.49369193,103.37884925)(860.46369196,103.36884926)(860.43368408,103.36885061)
\curveto(860.39369203,103.36884926)(860.34869207,103.36384926)(860.29868408,103.35385061)
\curveto(860.17869224,103.33384929)(860.06869235,103.33884929)(859.96868408,103.36885061)
\curveto(859.86869255,103.40884922)(859.79869262,103.46384916)(859.75868408,103.53385061)
\curveto(859.70869271,103.61384901)(859.68369274,103.73384889)(859.68368408,103.89385061)
\curveto(859.68369274,104.05384857)(859.66869275,104.18884844)(859.63868408,104.29885061)
\curveto(859.62869279,104.34884828)(859.6236928,104.40384822)(859.62368408,104.46385061)
\curveto(859.61369281,104.5238481)(859.59869282,104.58384804)(859.57868408,104.64385061)
\curveto(859.52869289,104.79384783)(859.47869294,104.93884769)(859.42868408,105.07885061)
\curveto(859.36869305,105.21884741)(859.29869312,105.35384727)(859.21868408,105.48385061)
\curveto(859.12869329,105.623847)(859.0236934,105.74384688)(858.90368408,105.84385061)
\curveto(858.78369364,105.94384668)(858.65369377,106.03884659)(858.51368408,106.12885061)
\curveto(858.41369401,106.18884644)(858.30369412,106.23384639)(858.18368408,106.26385061)
\curveto(858.06369436,106.30384632)(857.95869446,106.35384627)(857.86868408,106.41385061)
\curveto(857.80869461,106.46384616)(857.76869465,106.53384609)(857.74868408,106.62385061)
\curveto(857.73869468,106.64384598)(857.73369469,106.66884596)(857.73368408,106.69885061)
\curveto(857.73369469,106.7288459)(857.72869469,106.75384587)(857.71868408,106.77385061)
}
}
{
\newrgbcolor{curcolor}{0 0 0}
\pscustom[linestyle=none,fillstyle=solid,fillcolor=curcolor]
{
\newpath
\moveto(857.71868408,115.12345998)
\curveto(857.7186947,115.22345513)(857.72869469,115.31845503)(857.74868408,115.40845998)
\curveto(857.75869466,115.49845485)(857.78869463,115.56345479)(857.83868408,115.60345998)
\curveto(857.9186945,115.66345469)(858.0236944,115.69345466)(858.15368408,115.69345998)
\lineto(858.54368408,115.69345998)
\lineto(860.04368408,115.69345998)
\lineto(866.43368408,115.69345998)
\lineto(867.60368408,115.69345998)
\lineto(867.91868408,115.69345998)
\curveto(868.0186844,115.70345465)(868.09868432,115.68845466)(868.15868408,115.64845998)
\curveto(868.23868418,115.59845475)(868.28868413,115.52345483)(868.30868408,115.42345998)
\curveto(868.3186841,115.33345502)(868.3236841,115.22345513)(868.32368408,115.09345998)
\lineto(868.32368408,114.86845998)
\curveto(868.30368412,114.78845556)(868.28868413,114.71845563)(868.27868408,114.65845998)
\curveto(868.25868416,114.59845575)(868.2186842,114.5484558)(868.15868408,114.50845998)
\curveto(868.09868432,114.46845588)(868.0236844,114.4484559)(867.93368408,114.44845998)
\lineto(867.63368408,114.44845998)
\lineto(866.53868408,114.44845998)
\lineto(861.19868408,114.44845998)
\curveto(861.10869131,114.42845592)(861.03369139,114.41345594)(860.97368408,114.40345998)
\curveto(860.90369152,114.40345595)(860.84369158,114.37345598)(860.79368408,114.31345998)
\curveto(860.74369168,114.24345611)(860.7186917,114.1534562)(860.71868408,114.04345998)
\curveto(860.70869171,113.94345641)(860.70369172,113.83345652)(860.70368408,113.71345998)
\lineto(860.70368408,112.57345998)
\lineto(860.70368408,112.07845998)
\curveto(860.69369173,111.91845843)(860.63369179,111.80845854)(860.52368408,111.74845998)
\curveto(860.49369193,111.72845862)(860.46369196,111.71845863)(860.43368408,111.71845998)
\curveto(860.39369203,111.71845863)(860.34869207,111.71345864)(860.29868408,111.70345998)
\curveto(860.17869224,111.68345867)(860.06869235,111.68845866)(859.96868408,111.71845998)
\curveto(859.86869255,111.75845859)(859.79869262,111.81345854)(859.75868408,111.88345998)
\curveto(859.70869271,111.96345839)(859.68369274,112.08345827)(859.68368408,112.24345998)
\curveto(859.68369274,112.40345795)(859.66869275,112.53845781)(859.63868408,112.64845998)
\curveto(859.62869279,112.69845765)(859.6236928,112.7534576)(859.62368408,112.81345998)
\curveto(859.61369281,112.87345748)(859.59869282,112.93345742)(859.57868408,112.99345998)
\curveto(859.52869289,113.14345721)(859.47869294,113.28845706)(859.42868408,113.42845998)
\curveto(859.36869305,113.56845678)(859.29869312,113.70345665)(859.21868408,113.83345998)
\curveto(859.12869329,113.97345638)(859.0236934,114.09345626)(858.90368408,114.19345998)
\curveto(858.78369364,114.29345606)(858.65369377,114.38845596)(858.51368408,114.47845998)
\curveto(858.41369401,114.53845581)(858.30369412,114.58345577)(858.18368408,114.61345998)
\curveto(858.06369436,114.6534557)(857.95869446,114.70345565)(857.86868408,114.76345998)
\curveto(857.80869461,114.81345554)(857.76869465,114.88345547)(857.74868408,114.97345998)
\curveto(857.73869468,114.99345536)(857.73369469,115.01845533)(857.73368408,115.04845998)
\curveto(857.73369469,115.07845527)(857.72869469,115.10345525)(857.71868408,115.12345998)
}
}
{
\newrgbcolor{curcolor}{0 0 0}
\pscustom[linestyle=none,fillstyle=solid,fillcolor=curcolor]
{
\newpath
\moveto(878.555,42.29681936)
\curveto(878.55501069,42.36681368)(878.55501069,42.4468136)(878.555,42.53681936)
\curveto(878.5450107,42.62681342)(878.5450107,42.71181333)(878.555,42.79181936)
\curveto(878.55501069,42.88181316)(878.56501068,42.96181308)(878.585,43.03181936)
\curveto(878.60501064,43.11181293)(878.63501062,43.16681288)(878.675,43.19681936)
\curveto(878.72501053,43.22681282)(878.80001045,43.2468128)(878.9,43.25681936)
\curveto(878.99001026,43.27681277)(879.09501015,43.28681276)(879.215,43.28681936)
\curveto(879.32500993,43.29681275)(879.44000981,43.29681275)(879.56,43.28681936)
\lineto(879.86,43.28681936)
\lineto(882.875,43.28681936)
\lineto(885.77,43.28681936)
\curveto(886.10000315,43.28681276)(886.42500282,43.28181276)(886.745,43.27181936)
\curveto(887.0550022,43.27181277)(887.33500192,43.23181281)(887.585,43.15181936)
\curveto(887.93500132,43.03181301)(888.23000102,42.87681317)(888.47,42.68681936)
\curveto(888.70000055,42.49681355)(888.90000035,42.25681379)(889.07,41.96681936)
\curveto(889.12000013,41.90681414)(889.15500009,41.8418142)(889.175,41.77181936)
\curveto(889.19500006,41.71181433)(889.22000003,41.6418144)(889.25,41.56181936)
\curveto(889.29999995,41.4418146)(889.33499992,41.31181473)(889.355,41.17181936)
\curveto(889.38499987,41.041815)(889.41499984,40.90681514)(889.445,40.76681936)
\curveto(889.46499979,40.71681533)(889.46999978,40.66681538)(889.46,40.61681936)
\curveto(889.4499998,40.56681548)(889.4499998,40.51181553)(889.46,40.45181936)
\curveto(889.46999978,40.43181561)(889.46999978,40.40681564)(889.46,40.37681936)
\curveto(889.45999979,40.3468157)(889.46499979,40.32181572)(889.475,40.30181936)
\curveto(889.48499976,40.26181578)(889.48999976,40.20681584)(889.49,40.13681936)
\curveto(889.48999976,40.06681598)(889.48499976,40.01181603)(889.475,39.97181936)
\curveto(889.46499979,39.92181612)(889.46499979,39.86681618)(889.475,39.80681936)
\curveto(889.48499976,39.7468163)(889.47999977,39.69181635)(889.46,39.64181936)
\curveto(889.42999982,39.51181653)(889.40999984,39.38681666)(889.4,39.26681936)
\curveto(889.38999986,39.1468169)(889.36499988,39.03181701)(889.325,38.92181936)
\curveto(889.20500005,38.55181749)(889.03500021,38.23181781)(888.815,37.96181936)
\curveto(888.59500066,37.69181835)(888.31500093,37.48181856)(887.975,37.33181936)
\curveto(887.8550014,37.28181876)(887.73000152,37.23681881)(887.6,37.19681936)
\curveto(887.47000178,37.16681888)(887.33500192,37.13181891)(887.195,37.09181936)
\curveto(887.14500211,37.08181896)(887.10500214,37.07681897)(887.075,37.07681936)
\curveto(887.03500221,37.07681897)(886.99000226,37.07181897)(886.94,37.06181936)
\curveto(886.91000234,37.05181899)(886.87500238,37.046819)(886.835,37.04681936)
\curveto(886.78500247,37.046819)(886.74500251,37.041819)(886.715,37.03181936)
\lineto(886.55,37.03181936)
\curveto(886.47000278,37.01181903)(886.37000288,37.00681904)(886.25,37.01681936)
\curveto(886.12000313,37.02681902)(886.03000322,37.041819)(885.98,37.06181936)
\curveto(885.89000336,37.08181896)(885.82500342,37.13681891)(885.785,37.22681936)
\curveto(885.76500348,37.25681879)(885.76000349,37.28681876)(885.77,37.31681936)
\curveto(885.77000348,37.3468187)(885.76500348,37.38681866)(885.755,37.43681936)
\curveto(885.7450035,37.47681857)(885.74000351,37.51681853)(885.74,37.55681936)
\lineto(885.74,37.70681936)
\curveto(885.74000351,37.82681822)(885.7450035,37.9468181)(885.755,38.06681936)
\curveto(885.75500349,38.19681785)(885.79000346,38.28681776)(885.86,38.33681936)
\curveto(885.92000333,38.37681767)(885.98000327,38.39681765)(886.04,38.39681936)
\curveto(886.10000315,38.39681765)(886.17000308,38.40681764)(886.25,38.42681936)
\curveto(886.28000297,38.43681761)(886.31500294,38.43681761)(886.355,38.42681936)
\curveto(886.38500287,38.42681762)(886.41000284,38.43181761)(886.43,38.44181936)
\lineto(886.64,38.44181936)
\curveto(886.69000256,38.46181758)(886.74000251,38.46681758)(886.79,38.45681936)
\curveto(886.83000242,38.45681759)(886.87500238,38.46681758)(886.925,38.48681936)
\curveto(887.0550022,38.51681753)(887.18000207,38.5468175)(887.3,38.57681936)
\curveto(887.41000184,38.60681744)(887.51500174,38.65181739)(887.615,38.71181936)
\curveto(887.90500134,38.88181716)(888.11000114,39.15181689)(888.23,39.52181936)
\curveto(888.250001,39.57181647)(888.26500099,39.62181642)(888.275,39.67181936)
\curveto(888.27500098,39.73181631)(888.28500096,39.78681626)(888.305,39.83681936)
\lineto(888.305,39.91181936)
\curveto(888.31500093,39.98181606)(888.32500093,40.07681597)(888.335,40.19681936)
\curveto(888.33500092,40.32681572)(888.32500093,40.42681562)(888.305,40.49681936)
\curveto(888.28500096,40.56681548)(888.27000098,40.63681541)(888.26,40.70681936)
\curveto(888.24000101,40.78681526)(888.22000103,40.85681519)(888.2,40.91681936)
\curveto(888.04000121,41.29681475)(887.76500148,41.57181447)(887.375,41.74181936)
\curveto(887.245002,41.79181425)(887.09000216,41.82681422)(886.91,41.84681936)
\curveto(886.73000252,41.87681417)(886.54500271,41.89181415)(886.355,41.89181936)
\curveto(886.15500309,41.90181414)(885.95500329,41.90181414)(885.755,41.89181936)
\lineto(885.185,41.89181936)
\lineto(880.94,41.89181936)
\lineto(879.395,41.89181936)
\curveto(879.28500996,41.89181415)(879.16501008,41.88681416)(879.035,41.87681936)
\curveto(878.90501035,41.86681418)(878.80001045,41.88681416)(878.72,41.93681936)
\curveto(878.6500106,41.99681405)(878.60001065,42.07681397)(878.57,42.17681936)
\curveto(878.57001068,42.19681385)(878.57001068,42.21681383)(878.57,42.23681936)
\curveto(878.57001068,42.25681379)(878.56501068,42.27681377)(878.555,42.29681936)
}
}
{
\newrgbcolor{curcolor}{0 0 0}
\pscustom[linestyle=none,fillstyle=solid,fillcolor=curcolor]
{
\newpath
\moveto(881.51,45.83049123)
\lineto(881.51,46.26549123)
\curveto(881.51000774,46.41548927)(881.5500077,46.52048916)(881.63,46.58049123)
\curveto(881.71000754,46.63048905)(881.81000744,46.65548903)(881.93,46.65549123)
\curveto(882.0500072,46.66548902)(882.17000708,46.67048901)(882.29,46.67049123)
\lineto(883.715,46.67049123)
\lineto(885.98,46.67049123)
\lineto(886.67,46.67049123)
\curveto(886.90000235,46.67048901)(887.10000215,46.69548899)(887.27,46.74549123)
\curveto(887.72000153,46.90548878)(888.03500121,47.20548848)(888.215,47.64549123)
\curveto(888.30500094,47.86548782)(888.34000091,48.13048755)(888.32,48.44049123)
\curveto(888.29000096,48.75048693)(888.23500101,49.00048668)(888.155,49.19049123)
\curveto(888.01500123,49.52048616)(887.84000141,49.7804859)(887.63,49.97049123)
\curveto(887.41000184,50.17048551)(887.12500213,50.32548536)(886.775,50.43549123)
\curveto(886.69500255,50.46548522)(886.61500263,50.4854852)(886.535,50.49549123)
\curveto(886.4550028,50.50548518)(886.37000288,50.52048516)(886.28,50.54049123)
\curveto(886.23000302,50.55048513)(886.18500307,50.55048513)(886.145,50.54049123)
\curveto(886.10500314,50.54048514)(886.06000319,50.55048513)(886.01,50.57049123)
\lineto(885.695,50.57049123)
\curveto(885.61500363,50.59048509)(885.52500373,50.59548509)(885.425,50.58549123)
\curveto(885.31500394,50.57548511)(885.21500403,50.57048511)(885.125,50.57049123)
\lineto(883.955,50.57049123)
\lineto(882.365,50.57049123)
\curveto(882.24500701,50.57048511)(882.12000713,50.56548512)(881.99,50.55549123)
\curveto(881.8500074,50.55548513)(881.74000751,50.5804851)(881.66,50.63049123)
\curveto(881.61000764,50.67048501)(881.58000767,50.71548497)(881.57,50.76549123)
\curveto(881.5500077,50.82548486)(881.53000772,50.89548479)(881.51,50.97549123)
\lineto(881.51,51.20049123)
\curveto(881.51000774,51.32048436)(881.51500774,51.42548426)(881.525,51.51549123)
\curveto(881.53500772,51.61548407)(881.58000767,51.69048399)(881.66,51.74049123)
\curveto(881.71000754,51.79048389)(881.78500747,51.81548387)(881.885,51.81549123)
\lineto(882.17,51.81549123)
\lineto(883.19,51.81549123)
\lineto(887.225,51.81549123)
\lineto(888.575,51.81549123)
\curveto(888.69500055,51.81548387)(888.81000044,51.81048387)(888.92,51.80049123)
\curveto(889.02000023,51.80048388)(889.09500015,51.76548392)(889.145,51.69549123)
\curveto(889.17500007,51.65548403)(889.20000005,51.59548409)(889.22,51.51549123)
\curveto(889.23000002,51.43548425)(889.24000001,51.34548434)(889.25,51.24549123)
\curveto(889.25,51.15548453)(889.245,51.06548462)(889.235,50.97549123)
\curveto(889.22500002,50.89548479)(889.20500005,50.83548485)(889.175,50.79549123)
\curveto(889.13500012,50.74548494)(889.07000018,50.70048498)(888.98,50.66049123)
\curveto(888.94000031,50.65048503)(888.88500036,50.64048504)(888.815,50.63049123)
\curveto(888.74500051,50.63048505)(888.68000057,50.62548506)(888.62,50.61549123)
\curveto(888.5500007,50.60548508)(888.49500075,50.5854851)(888.455,50.55549123)
\curveto(888.41500084,50.52548516)(888.40000085,50.4804852)(888.41,50.42049123)
\curveto(888.43000082,50.34048534)(888.49000076,50.26048542)(888.59,50.18049123)
\curveto(888.68000057,50.10048558)(888.7500005,50.02548566)(888.8,49.95549123)
\curveto(888.96000029,49.73548595)(889.10000015,49.4854862)(889.22,49.20549123)
\curveto(889.26999998,49.09548659)(889.29999995,48.9804867)(889.31,48.86049123)
\curveto(889.32999992,48.75048693)(889.35499989,48.63548705)(889.385,48.51549123)
\curveto(889.39499986,48.46548722)(889.39499986,48.41048727)(889.385,48.35049123)
\curveto(889.37499987,48.30048738)(889.37999987,48.25048743)(889.4,48.20049123)
\curveto(889.41999983,48.10048758)(889.41999983,48.01048767)(889.4,47.93049123)
\lineto(889.4,47.78049123)
\curveto(889.37999987,47.73048795)(889.36999988,47.67048801)(889.37,47.60049123)
\curveto(889.36999988,47.54048814)(889.36499988,47.4854882)(889.355,47.43549123)
\curveto(889.33499992,47.39548829)(889.32499993,47.35548833)(889.325,47.31549123)
\curveto(889.33499992,47.2854884)(889.32999992,47.24548844)(889.31,47.19549123)
\lineto(889.25,46.95549123)
\curveto(889.23000002,46.8854888)(889.20000005,46.81048887)(889.16,46.73049123)
\curveto(889.0500002,46.47048921)(888.90500034,46.25048943)(888.725,46.07049123)
\curveto(888.53500072,45.90048978)(888.31000094,45.76048992)(888.05,45.65049123)
\curveto(887.96000129,45.61049007)(887.87000138,45.5804901)(887.78,45.56049123)
\lineto(887.48,45.50049123)
\curveto(887.42000183,45.4804902)(887.36500188,45.47049021)(887.315,45.47049123)
\curveto(887.255002,45.4804902)(887.19000206,45.47549021)(887.12,45.45549123)
\curveto(887.10000215,45.44549024)(887.07500218,45.44049024)(887.045,45.44049123)
\curveto(887.00500225,45.44049024)(886.97000228,45.43549025)(886.94,45.42549123)
\lineto(886.79,45.42549123)
\curveto(886.7500025,45.41549027)(886.70500254,45.41049027)(886.655,45.41049123)
\curveto(886.59500266,45.42049026)(886.54000271,45.42549026)(886.49,45.42549123)
\lineto(885.89,45.42549123)
\lineto(883.13,45.42549123)
\lineto(882.17,45.42549123)
\lineto(881.9,45.42549123)
\curveto(881.81000744,45.42549026)(881.73500752,45.44549024)(881.675,45.48549123)
\curveto(881.60500765,45.52549016)(881.55500769,45.60049008)(881.525,45.71049123)
\curveto(881.51500774,45.73048995)(881.51500774,45.75048993)(881.525,45.77049123)
\curveto(881.52500773,45.79048989)(881.52000773,45.81048987)(881.51,45.83049123)
}
}
{
\newrgbcolor{curcolor}{0 0 0}
\pscustom[linestyle=none,fillstyle=solid,fillcolor=curcolor]
{
\newpath
\moveto(878.555,54.28510061)
\curveto(878.55501069,54.41509899)(878.55501069,54.55009886)(878.555,54.69010061)
\curveto(878.55501069,54.84009857)(878.59001066,54.95009846)(878.66,55.02010061)
\curveto(878.73001052,55.07009834)(878.82501042,55.09509831)(878.945,55.09510061)
\curveto(879.0550102,55.1050983)(879.17001008,55.1100983)(879.29,55.11010061)
\lineto(880.625,55.11010061)
\lineto(886.7,55.11010061)
\lineto(888.38,55.11010061)
\lineto(888.77,55.11010061)
\curveto(888.91000034,55.1100983)(889.02000023,55.08509832)(889.1,55.03510061)
\curveto(889.1500001,55.0050984)(889.18000007,54.96009845)(889.19,54.90010061)
\curveto(889.20000005,54.85009856)(889.21500004,54.78509862)(889.235,54.70510061)
\lineto(889.235,54.49510061)
\lineto(889.235,54.18010061)
\curveto(889.22500002,54.08009933)(889.19000006,54.0050994)(889.13,53.95510061)
\curveto(889.0500002,53.9050995)(888.9500003,53.87509953)(888.83,53.86510061)
\lineto(888.455,53.86510061)
\lineto(887.075,53.86510061)
\lineto(880.835,53.86510061)
\lineto(879.365,53.86510061)
\curveto(879.25501,53.86509954)(879.14001011,53.86009955)(879.02,53.85010061)
\curveto(878.89001036,53.85009956)(878.79001046,53.87509953)(878.72,53.92510061)
\curveto(878.66001059,53.96509944)(878.61001064,54.04009937)(878.57,54.15010061)
\curveto(878.56001069,54.17009924)(878.56001069,54.19009922)(878.57,54.21010061)
\curveto(878.57001068,54.24009917)(878.56501068,54.26509914)(878.555,54.28510061)
}
}
{
\newrgbcolor{curcolor}{0 0 0}
\pscustom[linestyle=none,fillstyle=solid,fillcolor=curcolor]
{
}
}
{
\newrgbcolor{curcolor}{0 0 0}
\pscustom[linestyle=none,fillstyle=solid,fillcolor=curcolor]
{
\newpath
\moveto(878.63,65.05510061)
\curveto(878.63001062,65.15509575)(878.64001061,65.25009566)(878.66,65.34010061)
\curveto(878.67001058,65.43009548)(878.70001055,65.49509541)(878.75,65.53510061)
\curveto(878.83001042,65.59509531)(878.93501031,65.62509528)(879.065,65.62510061)
\lineto(879.455,65.62510061)
\lineto(880.955,65.62510061)
\lineto(887.345,65.62510061)
\lineto(888.515,65.62510061)
\lineto(888.83,65.62510061)
\curveto(888.93000032,65.63509527)(889.01000024,65.62009529)(889.07,65.58010061)
\curveto(889.1500001,65.53009538)(889.20000005,65.45509545)(889.22,65.35510061)
\curveto(889.23000002,65.26509564)(889.23500001,65.15509575)(889.235,65.02510061)
\lineto(889.235,64.80010061)
\curveto(889.21500004,64.72009619)(889.20000005,64.65009626)(889.19,64.59010061)
\curveto(889.17000008,64.53009638)(889.13000012,64.48009643)(889.07,64.44010061)
\curveto(889.01000024,64.40009651)(888.93500032,64.38009653)(888.845,64.38010061)
\lineto(888.545,64.38010061)
\lineto(887.45,64.38010061)
\lineto(882.11,64.38010061)
\curveto(882.02000723,64.36009655)(881.9450073,64.34509656)(881.885,64.33510061)
\curveto(881.81500743,64.33509657)(881.75500749,64.3050966)(881.705,64.24510061)
\curveto(881.6550076,64.17509673)(881.63000762,64.08509682)(881.63,63.97510061)
\curveto(881.62000763,63.87509703)(881.61500763,63.76509714)(881.615,63.64510061)
\lineto(881.615,62.50510061)
\lineto(881.615,62.01010061)
\curveto(881.60500765,61.85009906)(881.5450077,61.74009917)(881.435,61.68010061)
\curveto(881.40500785,61.66009925)(881.37500788,61.65009926)(881.345,61.65010061)
\curveto(881.30500794,61.65009926)(881.26000799,61.64509926)(881.21,61.63510061)
\curveto(881.09000816,61.61509929)(880.98000827,61.62009929)(880.88,61.65010061)
\curveto(880.78000847,61.69009922)(880.71000854,61.74509916)(880.67,61.81510061)
\curveto(880.62000863,61.89509901)(880.59500866,62.01509889)(880.595,62.17510061)
\curveto(880.59500866,62.33509857)(880.58000867,62.47009844)(880.55,62.58010061)
\curveto(880.54000871,62.63009828)(880.53500872,62.68509822)(880.535,62.74510061)
\curveto(880.52500873,62.8050981)(880.51000874,62.86509804)(880.49,62.92510061)
\curveto(880.44000881,63.07509783)(880.39000886,63.22009769)(880.34,63.36010061)
\curveto(880.28000897,63.50009741)(880.21000904,63.63509727)(880.13,63.76510061)
\curveto(880.04000921,63.905097)(879.93500932,64.02509688)(879.815,64.12510061)
\curveto(879.69500955,64.22509668)(879.56500968,64.32009659)(879.425,64.41010061)
\curveto(879.32500993,64.47009644)(879.21501003,64.51509639)(879.095,64.54510061)
\curveto(878.97501028,64.58509632)(878.87001038,64.63509627)(878.78,64.69510061)
\curveto(878.72001053,64.74509616)(878.68001057,64.81509609)(878.66,64.90510061)
\curveto(878.6500106,64.92509598)(878.64501061,64.95009596)(878.645,64.98010061)
\curveto(878.64501061,65.0100959)(878.64001061,65.03509587)(878.63,65.05510061)
}
}
{
\newrgbcolor{curcolor}{0 0 0}
\pscustom[linestyle=none,fillstyle=solid,fillcolor=curcolor]
{
\newpath
\moveto(884.915,76.34470998)
\curveto(885.03500421,76.37470226)(885.17500407,76.39970223)(885.335,76.41970998)
\curveto(885.49500375,76.43970219)(885.66000359,76.44970218)(885.83,76.44970998)
\curveto(886.00000325,76.44970218)(886.16500308,76.43970219)(886.325,76.41970998)
\curveto(886.48500276,76.39970223)(886.62500262,76.37470226)(886.745,76.34470998)
\curveto(886.88500236,76.30470233)(887.01000224,76.26970236)(887.12,76.23970998)
\curveto(887.23000202,76.20970242)(887.34000191,76.16970246)(887.45,76.11970998)
\curveto(888.09000116,75.84970278)(888.57500067,75.4347032)(888.905,74.87470998)
\curveto(888.96500028,74.79470384)(889.01500024,74.70970392)(889.055,74.61970998)
\curveto(889.08500016,74.5297041)(889.12000013,74.4297042)(889.16,74.31970998)
\curveto(889.21000004,74.20970442)(889.245,74.08970454)(889.265,73.95970998)
\curveto(889.29499995,73.83970479)(889.32499993,73.70970492)(889.355,73.56970998)
\curveto(889.37499987,73.50970512)(889.37999987,73.44970518)(889.37,73.38970998)
\curveto(889.35999989,73.33970529)(889.36499988,73.27970535)(889.385,73.20970998)
\curveto(889.39499986,73.18970544)(889.39499986,73.16470547)(889.385,73.13470998)
\curveto(889.38499987,73.10470553)(889.38999986,73.07970555)(889.4,73.05970998)
\lineto(889.4,72.90970998)
\curveto(889.40999984,72.83970579)(889.40999984,72.78970584)(889.4,72.75970998)
\curveto(889.38999986,72.71970591)(889.38499987,72.67470596)(889.385,72.62470998)
\curveto(889.39499986,72.58470605)(889.39499986,72.54470609)(889.385,72.50470998)
\curveto(889.36499988,72.41470622)(889.3499999,72.32470631)(889.34,72.23470998)
\curveto(889.33999991,72.14470649)(889.32999992,72.05470658)(889.31,71.96470998)
\curveto(889.27999997,71.87470676)(889.255,71.78470685)(889.235,71.69470998)
\curveto(889.21500004,71.60470703)(889.18500007,71.51970711)(889.145,71.43970998)
\curveto(889.03500021,71.19970743)(888.90500034,70.97470766)(888.755,70.76470998)
\curveto(888.59500066,70.55470808)(888.41500084,70.37470826)(888.215,70.22470998)
\curveto(888.0450012,70.10470853)(887.87000138,69.99970863)(887.69,69.90970998)
\curveto(887.51000174,69.81970881)(887.32000193,69.7297089)(887.12,69.63970998)
\curveto(887.02000223,69.59970903)(886.92000233,69.56470907)(886.82,69.53470998)
\curveto(886.71000254,69.51470912)(886.60000265,69.48970914)(886.49,69.45970998)
\curveto(886.3500029,69.41970921)(886.21000304,69.39470924)(886.07,69.38470998)
\curveto(885.93000332,69.37470926)(885.79000346,69.35470928)(885.65,69.32470998)
\curveto(885.54000371,69.31470932)(885.44000381,69.30470933)(885.35,69.29470998)
\curveto(885.250004,69.29470934)(885.1500041,69.28470935)(885.05,69.26470998)
\lineto(884.96,69.26470998)
\curveto(884.93000432,69.27470936)(884.90500434,69.27470936)(884.885,69.26470998)
\lineto(884.675,69.26470998)
\curveto(884.61500463,69.24470939)(884.5500047,69.2347094)(884.48,69.23470998)
\curveto(884.40000485,69.24470939)(884.32500493,69.24970938)(884.255,69.24970998)
\lineto(884.105,69.24970998)
\curveto(884.0550052,69.24970938)(884.00500525,69.25470938)(883.955,69.26470998)
\lineto(883.58,69.26470998)
\curveto(883.5500057,69.27470936)(883.51500574,69.27470936)(883.475,69.26470998)
\curveto(883.43500581,69.26470937)(883.39500586,69.26970936)(883.355,69.27970998)
\curveto(883.24500601,69.29970933)(883.13500612,69.31470932)(883.025,69.32470998)
\curveto(882.90500634,69.3347093)(882.79000646,69.34470929)(882.68,69.35470998)
\curveto(882.53000672,69.39470924)(882.38500687,69.41970921)(882.245,69.42970998)
\curveto(882.09500715,69.44970918)(881.9500073,69.47970915)(881.81,69.51970998)
\curveto(881.51000774,69.60970902)(881.22500802,69.70470893)(880.955,69.80470998)
\curveto(880.68500856,69.90470873)(880.43500881,70.0297086)(880.205,70.17970998)
\curveto(879.88500936,70.37970825)(879.60500964,70.62470801)(879.365,70.91470998)
\curveto(879.12501013,71.20470743)(878.94001031,71.54470709)(878.81,71.93470998)
\curveto(878.77001048,72.04470659)(878.7450105,72.15470648)(878.735,72.26470998)
\curveto(878.71501054,72.38470625)(878.69001056,72.50470613)(878.66,72.62470998)
\curveto(878.6500106,72.69470594)(878.64501061,72.75970587)(878.645,72.81970998)
\curveto(878.64501061,72.87970575)(878.64001061,72.94470569)(878.63,73.01470998)
\curveto(878.61001064,73.71470492)(878.72501053,74.28970434)(878.975,74.73970998)
\curveto(879.22501002,75.18970344)(879.57500968,75.5347031)(880.025,75.77470998)
\curveto(880.255009,75.88470275)(880.53000872,75.98470265)(880.85,76.07470998)
\curveto(880.92000833,76.09470254)(880.99500826,76.09470254)(881.075,76.07470998)
\curveto(881.1450081,76.06470257)(881.19500806,76.03970259)(881.225,75.99970998)
\curveto(881.255008,75.96970266)(881.28000797,75.90970272)(881.3,75.81970998)
\curveto(881.31000794,75.7297029)(881.32000793,75.629703)(881.33,75.51970998)
\curveto(881.33000792,75.41970321)(881.32500793,75.31970331)(881.315,75.21970998)
\curveto(881.30500794,75.1297035)(881.28500796,75.06470357)(881.255,75.02470998)
\curveto(881.18500807,74.91470372)(881.07500817,74.8347038)(880.925,74.78470998)
\curveto(880.77500848,74.74470389)(880.64500861,74.68970394)(880.535,74.61970998)
\curveto(880.22500902,74.4297042)(879.99500926,74.14970448)(879.845,73.77970998)
\curveto(879.81500943,73.70970492)(879.79500946,73.634705)(879.785,73.55470998)
\curveto(879.77500948,73.48470515)(879.76000949,73.40970522)(879.74,73.32970998)
\curveto(879.73000952,73.27970535)(879.72500953,73.20970542)(879.725,73.11970998)
\curveto(879.72500953,73.03970559)(879.73000952,72.97470566)(879.74,72.92470998)
\curveto(879.76000949,72.88470575)(879.76500948,72.84970578)(879.755,72.81970998)
\curveto(879.7450095,72.78970584)(879.7450095,72.75470588)(879.755,72.71470998)
\lineto(879.815,72.47470998)
\curveto(879.83500941,72.40470623)(879.86000939,72.3347063)(879.89,72.26470998)
\curveto(880.0500092,71.88470675)(880.26000899,71.59470704)(880.52,71.39470998)
\curveto(880.78000847,71.20470743)(881.09500815,71.0297076)(881.465,70.86970998)
\curveto(881.5450077,70.83970779)(881.62500762,70.81470782)(881.705,70.79470998)
\curveto(881.78500747,70.78470785)(881.86500739,70.76470787)(881.945,70.73470998)
\curveto(882.0550072,70.70470793)(882.17000708,70.67970795)(882.29,70.65970998)
\curveto(882.41000684,70.64970798)(882.53000672,70.629708)(882.65,70.59970998)
\curveto(882.70000655,70.57970805)(882.7500065,70.56970806)(882.8,70.56970998)
\curveto(882.8500064,70.57970805)(882.90000635,70.57470806)(882.95,70.55470998)
\curveto(883.01000624,70.54470809)(883.09000616,70.54470809)(883.19,70.55470998)
\curveto(883.28000597,70.56470807)(883.33500592,70.57970805)(883.355,70.59970998)
\curveto(883.39500586,70.61970801)(883.41500583,70.64970798)(883.415,70.68970998)
\curveto(883.41500583,70.73970789)(883.40500585,70.78470785)(883.385,70.82470998)
\curveto(883.3450059,70.89470774)(883.30000595,70.95470768)(883.25,71.00470998)
\curveto(883.20000605,71.05470758)(883.1500061,71.11470752)(883.1,71.18470998)
\lineto(883.04,71.24470998)
\curveto(883.01000624,71.27470736)(882.98500627,71.30470733)(882.965,71.33470998)
\curveto(882.80500645,71.56470707)(882.67000658,71.83970679)(882.56,72.15970998)
\curveto(882.54000671,72.2297064)(882.52500673,72.29970633)(882.515,72.36970998)
\curveto(882.50500674,72.43970619)(882.49000676,72.51470612)(882.47,72.59470998)
\curveto(882.47000678,72.634706)(882.46500679,72.66970596)(882.455,72.69970998)
\curveto(882.44500681,72.7297059)(882.44500681,72.76470587)(882.455,72.80470998)
\curveto(882.4550068,72.85470578)(882.44500681,72.89470574)(882.425,72.92470998)
\lineto(882.425,73.08970998)
\lineto(882.425,73.17970998)
\curveto(882.41500683,73.2297054)(882.41500683,73.26970536)(882.425,73.29970998)
\curveto(882.43500681,73.34970528)(882.44000681,73.39970523)(882.44,73.44970998)
\curveto(882.43000682,73.50970512)(882.43000682,73.56470507)(882.44,73.61470998)
\curveto(882.47000678,73.72470491)(882.49000676,73.8297048)(882.5,73.92970998)
\curveto(882.51000674,74.03970459)(882.53500672,74.14470449)(882.575,74.24470998)
\curveto(882.71500654,74.66470397)(882.90000635,75.00970362)(883.13,75.27970998)
\curveto(883.3500059,75.54970308)(883.63500561,75.78970284)(883.985,75.99970998)
\curveto(884.12500513,76.07970255)(884.27500498,76.14470249)(884.435,76.19470998)
\curveto(884.58500467,76.24470239)(884.7450045,76.29470234)(884.915,76.34470998)
\moveto(886.22,75.09970998)
\curveto(886.17000308,75.10970352)(886.12500313,75.11470352)(886.085,75.11470998)
\lineto(885.935,75.11470998)
\curveto(885.62500362,75.11470352)(885.34000391,75.07470356)(885.08,74.99470998)
\curveto(885.02000423,74.97470366)(884.96500428,74.95470368)(884.915,74.93470998)
\curveto(884.8550044,74.92470371)(884.80000445,74.90970372)(884.75,74.88970998)
\curveto(884.26000499,74.66970396)(883.91000534,74.32470431)(883.7,73.85470998)
\curveto(883.67000558,73.77470486)(883.64500561,73.69470494)(883.625,73.61470998)
\lineto(883.565,73.37470998)
\curveto(883.5450057,73.29470534)(883.53500572,73.20470543)(883.535,73.10470998)
\lineto(883.535,72.78970998)
\curveto(883.55500569,72.76970586)(883.56500568,72.7297059)(883.565,72.66970998)
\curveto(883.55500569,72.61970601)(883.55500569,72.57470606)(883.565,72.53470998)
\lineto(883.625,72.29470998)
\curveto(883.63500561,72.22470641)(883.6550056,72.15470648)(883.685,72.08470998)
\curveto(883.9450053,71.48470715)(884.41000484,71.07970755)(885.08,70.86970998)
\curveto(885.16000409,70.83970779)(885.24000401,70.81970781)(885.32,70.80970998)
\curveto(885.40000385,70.79970783)(885.48500376,70.78470785)(885.575,70.76470998)
\lineto(885.725,70.76470998)
\curveto(885.76500348,70.75470788)(885.83500341,70.74970788)(885.935,70.74970998)
\curveto(886.16500308,70.74970788)(886.36000289,70.76970786)(886.52,70.80970998)
\curveto(886.59000266,70.8297078)(886.6550026,70.84470779)(886.715,70.85470998)
\curveto(886.77500247,70.86470777)(886.84000241,70.88470775)(886.91,70.91470998)
\curveto(887.19000206,71.02470761)(887.43500181,71.16970746)(887.645,71.34970998)
\curveto(887.8450014,71.5297071)(888.00500125,71.76470687)(888.125,72.05470998)
\lineto(888.215,72.29470998)
\lineto(888.275,72.53470998)
\curveto(888.29500095,72.58470605)(888.30000095,72.62470601)(888.29,72.65470998)
\curveto(888.28000097,72.69470594)(888.28500096,72.73970589)(888.305,72.78970998)
\curveto(888.31500093,72.81970581)(888.32000093,72.87470576)(888.32,72.95470998)
\curveto(888.32000093,73.0347056)(888.31500093,73.09470554)(888.305,73.13470998)
\curveto(888.28500096,73.24470539)(888.27000098,73.34970528)(888.26,73.44970998)
\curveto(888.250001,73.54970508)(888.22000103,73.64470499)(888.17,73.73470998)
\curveto(887.97000128,74.26470437)(887.59500166,74.65470398)(887.045,74.90470998)
\curveto(886.94500231,74.94470369)(886.84000241,74.97470366)(886.73,74.99470998)
\lineto(886.4,75.08470998)
\curveto(886.32000293,75.08470355)(886.26000299,75.08970354)(886.22,75.09970998)
}
}
{
\newrgbcolor{curcolor}{0 0 0}
\pscustom[linestyle=none,fillstyle=solid,fillcolor=curcolor]
{
\newpath
\moveto(887.6,78.63431936)
\lineto(887.6,79.26431936)
\lineto(887.6,79.45931936)
\curveto(887.60000165,79.52931683)(887.61000164,79.58931677)(887.63,79.63931936)
\curveto(887.67000158,79.70931665)(887.71000154,79.7593166)(887.75,79.78931936)
\curveto(887.80000145,79.82931653)(887.86500139,79.84931651)(887.945,79.84931936)
\curveto(888.02500122,79.8593165)(888.11000114,79.86431649)(888.2,79.86431936)
\lineto(888.92,79.86431936)
\curveto(889.39999985,79.86431649)(889.80999944,79.80431655)(890.15,79.68431936)
\curveto(890.48999876,79.56431679)(890.76499848,79.36931699)(890.975,79.09931936)
\curveto(891.02499822,79.02931733)(891.06999818,78.9593174)(891.11,78.88931936)
\curveto(891.15999809,78.82931753)(891.20499805,78.7543176)(891.245,78.66431936)
\curveto(891.254998,78.64431771)(891.26499799,78.61431774)(891.275,78.57431936)
\curveto(891.29499795,78.53431782)(891.29999795,78.48931787)(891.29,78.43931936)
\curveto(891.25999799,78.34931801)(891.18499806,78.29431806)(891.065,78.27431936)
\curveto(890.95499829,78.2543181)(890.85999839,78.26931809)(890.78,78.31931936)
\curveto(890.70999854,78.34931801)(890.6449986,78.39431796)(890.585,78.45431936)
\curveto(890.53499872,78.52431783)(890.48499877,78.58931777)(890.435,78.64931936)
\curveto(890.38499886,78.71931764)(890.30999894,78.77931758)(890.21,78.82931936)
\curveto(890.11999913,78.88931747)(890.02999922,78.93931742)(889.94,78.97931936)
\curveto(889.90999934,78.99931736)(889.8499994,79.02431733)(889.76,79.05431936)
\curveto(889.67999957,79.08431727)(889.60999964,79.08931727)(889.55,79.06931936)
\curveto(889.40999984,79.03931732)(889.31999993,78.97931738)(889.28,78.88931936)
\curveto(889.25,78.80931755)(889.23500001,78.71931764)(889.235,78.61931936)
\curveto(889.23500001,78.51931784)(889.21000004,78.43431792)(889.16,78.36431936)
\curveto(889.09000016,78.27431808)(888.9500003,78.22931813)(888.74,78.22931936)
\lineto(888.185,78.22931936)
\lineto(887.96,78.22931936)
\curveto(887.88000137,78.23931812)(887.81500143,78.2593181)(887.765,78.28931936)
\curveto(887.68500156,78.34931801)(887.64000161,78.41931794)(887.63,78.49931936)
\curveto(887.62000163,78.51931784)(887.61500163,78.53931782)(887.615,78.55931936)
\curveto(887.61500163,78.58931777)(887.61000164,78.61431774)(887.6,78.63431936)
}
}
{
\newrgbcolor{curcolor}{0 0 0}
\pscustom[linestyle=none,fillstyle=solid,fillcolor=curcolor]
{
}
}
{
\newrgbcolor{curcolor}{0 0 0}
\pscustom[linestyle=none,fillstyle=solid,fillcolor=curcolor]
{
\newpath
\moveto(878.63,89.26463186)
\curveto(878.62001063,89.95462722)(878.74001051,90.55462662)(878.99,91.06463186)
\curveto(879.24001001,91.58462559)(879.57500968,91.9796252)(879.995,92.24963186)
\curveto(880.07500917,92.29962488)(880.16500908,92.34462483)(880.265,92.38463186)
\curveto(880.35500889,92.42462475)(880.4500088,92.46962471)(880.55,92.51963186)
\curveto(880.6500086,92.55962462)(880.7500085,92.58962459)(880.85,92.60963186)
\curveto(880.9500083,92.62962455)(881.0550082,92.64962453)(881.165,92.66963186)
\curveto(881.21500803,92.68962449)(881.26000799,92.69462448)(881.3,92.68463186)
\curveto(881.34000791,92.6746245)(881.38500787,92.6796245)(881.435,92.69963186)
\curveto(881.48500776,92.70962447)(881.57000768,92.71462446)(881.69,92.71463186)
\curveto(881.80000745,92.71462446)(881.88500736,92.70962447)(881.945,92.69963186)
\curveto(882.00500725,92.6796245)(882.06500719,92.66962451)(882.125,92.66963186)
\curveto(882.18500707,92.6796245)(882.24500701,92.6746245)(882.305,92.65463186)
\curveto(882.44500681,92.61462456)(882.58000667,92.5796246)(882.71,92.54963186)
\curveto(882.84000641,92.51962466)(882.96500628,92.4796247)(883.085,92.42963186)
\curveto(883.22500602,92.36962481)(883.3500059,92.29962488)(883.46,92.21963186)
\curveto(883.57000568,92.14962503)(883.68000557,92.0746251)(883.79,91.99463186)
\lineto(883.85,91.93463186)
\curveto(883.87000538,91.92462525)(883.89000536,91.90962527)(883.91,91.88963186)
\curveto(884.07000518,91.76962541)(884.21500503,91.63462554)(884.345,91.48463186)
\curveto(884.47500478,91.33462584)(884.60000465,91.174626)(884.72,91.00463186)
\curveto(884.94000431,90.69462648)(885.1450041,90.39962678)(885.335,90.11963186)
\curveto(885.47500378,89.88962729)(885.61000364,89.65962752)(885.74,89.42963186)
\curveto(885.87000338,89.20962797)(886.00500325,88.98962819)(886.145,88.76963186)
\curveto(886.31500294,88.51962866)(886.49500275,88.2796289)(886.685,88.04963186)
\curveto(886.87500238,87.82962935)(887.10000215,87.63962954)(887.36,87.47963186)
\curveto(887.42000183,87.43962974)(887.48000177,87.40462977)(887.54,87.37463186)
\curveto(887.59000166,87.34462983)(887.6550016,87.31462986)(887.735,87.28463186)
\curveto(887.80500145,87.26462991)(887.86500139,87.25962992)(887.915,87.26963186)
\curveto(887.98500127,87.28962989)(888.04000121,87.32462985)(888.08,87.37463186)
\curveto(888.11000114,87.42462975)(888.13000112,87.48462969)(888.14,87.55463186)
\lineto(888.14,87.79463186)
\lineto(888.14,88.54463186)
\lineto(888.14,91.34963186)
\lineto(888.14,92.00963186)
\curveto(888.14000111,92.09962508)(888.14500111,92.18462499)(888.155,92.26463186)
\curveto(888.15500109,92.34462483)(888.17500107,92.40962477)(888.215,92.45963186)
\curveto(888.255001,92.50962467)(888.33000092,92.54962463)(888.44,92.57963186)
\curveto(888.54000071,92.61962456)(888.64000061,92.62962455)(888.74,92.60963186)
\lineto(888.875,92.60963186)
\curveto(888.94500031,92.58962459)(889.00500025,92.56962461)(889.055,92.54963186)
\curveto(889.10500014,92.52962465)(889.14500011,92.49462468)(889.175,92.44463186)
\curveto(889.21500004,92.39462478)(889.23500001,92.32462485)(889.235,92.23463186)
\lineto(889.235,91.96463186)
\lineto(889.235,91.06463186)
\lineto(889.235,87.55463186)
\lineto(889.235,86.48963186)
\curveto(889.23500001,86.40963077)(889.24000001,86.31963086)(889.25,86.21963186)
\curveto(889.25,86.11963106)(889.24000001,86.03463114)(889.22,85.96463186)
\curveto(889.1500001,85.75463142)(888.97000028,85.68963149)(888.68,85.76963186)
\curveto(888.64000061,85.7796314)(888.60500065,85.7796314)(888.575,85.76963186)
\curveto(888.53500072,85.76963141)(888.49000076,85.7796314)(888.44,85.79963186)
\curveto(888.36000089,85.81963136)(888.27500098,85.83963134)(888.185,85.85963186)
\curveto(888.09500115,85.8796313)(888.01000124,85.90463127)(887.93,85.93463186)
\curveto(887.44000181,86.09463108)(887.02500222,86.29463088)(886.685,86.53463186)
\curveto(886.43500281,86.71463046)(886.21000304,86.91963026)(886.01,87.14963186)
\curveto(885.80000345,87.3796298)(885.60500365,87.61962956)(885.425,87.86963186)
\curveto(885.24500401,88.12962905)(885.07500418,88.39462878)(884.915,88.66463186)
\curveto(884.7450045,88.94462823)(884.57000468,89.21462796)(884.39,89.47463186)
\curveto(884.31000494,89.58462759)(884.23500501,89.68962749)(884.165,89.78963186)
\curveto(884.09500515,89.89962728)(884.02000523,90.00962717)(883.94,90.11963186)
\curveto(883.91000534,90.15962702)(883.88000537,90.19462698)(883.85,90.22463186)
\curveto(883.81000544,90.26462691)(883.78000547,90.30462687)(883.76,90.34463186)
\curveto(883.6500056,90.48462669)(883.52500573,90.60962657)(883.385,90.71963186)
\curveto(883.35500589,90.73962644)(883.33000592,90.76462641)(883.31,90.79463186)
\curveto(883.28000597,90.82462635)(883.250006,90.84962633)(883.22,90.86963186)
\curveto(883.12000613,90.94962623)(883.02000623,91.01462616)(882.92,91.06463186)
\curveto(882.82000643,91.12462605)(882.71000654,91.179626)(882.59,91.22963186)
\curveto(882.52000673,91.25962592)(882.44500681,91.2796259)(882.365,91.28963186)
\lineto(882.125,91.34963186)
\lineto(882.035,91.34963186)
\curveto(882.00500725,91.35962582)(881.97500728,91.36462581)(881.945,91.36463186)
\curveto(881.87500737,91.38462579)(881.78000747,91.38962579)(881.66,91.37963186)
\curveto(881.53000772,91.3796258)(881.43000782,91.36962581)(881.36,91.34963186)
\curveto(881.28000797,91.32962585)(881.20500804,91.30962587)(881.135,91.28963186)
\curveto(881.0550082,91.2796259)(880.97500828,91.25962592)(880.895,91.22963186)
\curveto(880.6550086,91.11962606)(880.4550088,90.96962621)(880.295,90.77963186)
\curveto(880.12500913,90.59962658)(879.98500927,90.3796268)(879.875,90.11963186)
\curveto(879.8550094,90.04962713)(879.84000941,89.9796272)(879.83,89.90963186)
\curveto(879.81000944,89.83962734)(879.79000946,89.76462741)(879.77,89.68463186)
\curveto(879.7500095,89.60462757)(879.74000951,89.49462768)(879.74,89.35463186)
\curveto(879.74000951,89.22462795)(879.7500095,89.11962806)(879.77,89.03963186)
\curveto(879.78000947,88.9796282)(879.78500947,88.92462825)(879.785,88.87463186)
\curveto(879.78500947,88.82462835)(879.79500946,88.7746284)(879.815,88.72463186)
\curveto(879.8550094,88.62462855)(879.89500935,88.52962865)(879.935,88.43963186)
\curveto(879.97500928,88.35962882)(880.02000923,88.2796289)(880.07,88.19963186)
\curveto(880.09000916,88.16962901)(880.11500914,88.13962904)(880.145,88.10963186)
\curveto(880.17500908,88.08962909)(880.20000905,88.06462911)(880.22,88.03463186)
\lineto(880.295,87.95963186)
\curveto(880.31500894,87.92962925)(880.33500892,87.90462927)(880.355,87.88463186)
\lineto(880.565,87.73463186)
\curveto(880.62500862,87.69462948)(880.69000856,87.64962953)(880.76,87.59963186)
\curveto(880.8500084,87.53962964)(880.95500829,87.48962969)(881.075,87.44963186)
\curveto(881.18500807,87.41962976)(881.29500795,87.38462979)(881.405,87.34463186)
\curveto(881.51500774,87.30462987)(881.66000759,87.2796299)(881.84,87.26963186)
\curveto(882.01000724,87.25962992)(882.13500712,87.22962995)(882.215,87.17963186)
\curveto(882.29500695,87.12963005)(882.34000691,87.05463012)(882.35,86.95463186)
\curveto(882.36000689,86.85463032)(882.36500688,86.74463043)(882.365,86.62463186)
\curveto(882.36500688,86.58463059)(882.37000688,86.54463063)(882.38,86.50463186)
\curveto(882.38000687,86.46463071)(882.37500688,86.42963075)(882.365,86.39963186)
\curveto(882.3450069,86.34963083)(882.33500692,86.29963088)(882.335,86.24963186)
\curveto(882.33500692,86.20963097)(882.32500693,86.16963101)(882.305,86.12963186)
\curveto(882.24500701,86.03963114)(882.11000714,85.99463118)(881.9,85.99463186)
\lineto(881.78,85.99463186)
\curveto(881.72000753,86.00463117)(881.66000759,86.00963117)(881.6,86.00963186)
\curveto(881.53000772,86.01963116)(881.46500779,86.02963115)(881.405,86.03963186)
\curveto(881.29500795,86.05963112)(881.19500806,86.0796311)(881.105,86.09963186)
\curveto(881.00500824,86.11963106)(880.91000834,86.14963103)(880.82,86.18963186)
\curveto(880.7500085,86.20963097)(880.69000856,86.22963095)(880.64,86.24963186)
\lineto(880.46,86.30963186)
\curveto(880.20000905,86.42963075)(879.95500929,86.58463059)(879.725,86.77463186)
\curveto(879.49500975,86.9746302)(879.31000994,87.18962999)(879.17,87.41963186)
\curveto(879.09001016,87.52962965)(879.02501022,87.64462953)(878.975,87.76463186)
\lineto(878.825,88.15463186)
\curveto(878.77501048,88.26462891)(878.7450105,88.3796288)(878.735,88.49963186)
\curveto(878.71501054,88.61962856)(878.69001056,88.74462843)(878.66,88.87463186)
\curveto(878.66001059,88.94462823)(878.66001059,89.00962817)(878.66,89.06963186)
\curveto(878.6500106,89.12962805)(878.64001061,89.19462798)(878.63,89.26463186)
}
}
{
\newrgbcolor{curcolor}{0 0 0}
\pscustom[linestyle=none,fillstyle=solid,fillcolor=curcolor]
{
\newpath
\moveto(884.15,101.36424123)
\lineto(884.405,101.36424123)
\curveto(884.48500476,101.37423353)(884.56000469,101.36923353)(884.63,101.34924123)
\lineto(884.87,101.34924123)
\lineto(885.035,101.34924123)
\curveto(885.13500412,101.32923357)(885.24000401,101.31923358)(885.35,101.31924123)
\curveto(885.4500038,101.31923358)(885.5500037,101.30923359)(885.65,101.28924123)
\lineto(885.8,101.28924123)
\curveto(885.94000331,101.25923364)(886.08000317,101.23923366)(886.22,101.22924123)
\curveto(886.3500029,101.21923368)(886.48000277,101.19423371)(886.61,101.15424123)
\curveto(886.69000256,101.13423377)(886.77500247,101.11423379)(886.865,101.09424123)
\lineto(887.105,101.03424123)
\lineto(887.405,100.91424123)
\curveto(887.49500175,100.88423402)(887.58500167,100.84923405)(887.675,100.80924123)
\curveto(887.89500135,100.70923419)(888.11000114,100.57423433)(888.32,100.40424123)
\curveto(888.53000072,100.24423466)(888.70000055,100.06923483)(888.83,99.87924123)
\curveto(888.87000038,99.82923507)(888.91000034,99.76923513)(888.95,99.69924123)
\curveto(888.98000027,99.63923526)(889.01500024,99.57923532)(889.055,99.51924123)
\curveto(889.10500014,99.43923546)(889.14500011,99.34423556)(889.175,99.23424123)
\curveto(889.20500005,99.12423578)(889.23500001,99.01923588)(889.265,98.91924123)
\curveto(889.30499994,98.80923609)(889.32999992,98.6992362)(889.34,98.58924123)
\curveto(889.3499999,98.47923642)(889.36499988,98.36423654)(889.385,98.24424123)
\curveto(889.39499986,98.2042367)(889.39499986,98.15923674)(889.385,98.10924123)
\curveto(889.38499987,98.06923683)(889.38999986,98.02923687)(889.4,97.98924123)
\curveto(889.40999984,97.94923695)(889.41499984,97.89423701)(889.415,97.82424123)
\curveto(889.41499984,97.75423715)(889.40999984,97.7042372)(889.4,97.67424123)
\curveto(889.37999987,97.62423728)(889.37499987,97.57923732)(889.385,97.53924123)
\curveto(889.39499986,97.4992374)(889.39499986,97.46423744)(889.385,97.43424123)
\lineto(889.385,97.34424123)
\curveto(889.36499988,97.28423762)(889.3499999,97.21923768)(889.34,97.14924123)
\curveto(889.33999991,97.08923781)(889.33499992,97.02423788)(889.325,96.95424123)
\curveto(889.27499998,96.78423812)(889.22500002,96.62423828)(889.175,96.47424123)
\curveto(889.12500013,96.32423858)(889.06000019,96.17923872)(888.98,96.03924123)
\curveto(888.94000031,95.98923891)(888.91000034,95.93423897)(888.89,95.87424123)
\curveto(888.86000039,95.82423908)(888.82500042,95.77423913)(888.785,95.72424123)
\curveto(888.60500065,95.48423942)(888.38500087,95.28423962)(888.125,95.12424123)
\curveto(887.86500139,94.96423994)(887.58000167,94.82424008)(887.27,94.70424123)
\curveto(887.13000212,94.64424026)(886.99000226,94.5992403)(886.85,94.56924123)
\curveto(886.70000255,94.53924036)(886.54500271,94.5042404)(886.385,94.46424123)
\curveto(886.27500298,94.44424046)(886.16500308,94.42924047)(886.055,94.41924123)
\curveto(885.94500331,94.40924049)(885.83500341,94.39424051)(885.725,94.37424123)
\curveto(885.68500356,94.36424054)(885.64500361,94.35924054)(885.605,94.35924123)
\curveto(885.56500368,94.36924053)(885.52500373,94.36924053)(885.485,94.35924123)
\curveto(885.43500381,94.34924055)(885.38500387,94.34424056)(885.335,94.34424123)
\lineto(885.17,94.34424123)
\curveto(885.12000413,94.32424058)(885.07000418,94.31924058)(885.02,94.32924123)
\curveto(884.96000429,94.33924056)(884.90500434,94.33924056)(884.855,94.32924123)
\curveto(884.81500443,94.31924058)(884.77000448,94.31924058)(884.72,94.32924123)
\curveto(884.67000458,94.33924056)(884.62000463,94.33424057)(884.57,94.31424123)
\curveto(884.50000475,94.29424061)(884.42500482,94.28924061)(884.345,94.29924123)
\curveto(884.255005,94.30924059)(884.17000508,94.31424059)(884.09,94.31424123)
\curveto(884.00000525,94.31424059)(883.90000535,94.30924059)(883.79,94.29924123)
\curveto(883.67000558,94.28924061)(883.57000568,94.29424061)(883.49,94.31424123)
\lineto(883.205,94.31424123)
\lineto(882.575,94.35924123)
\curveto(882.47500677,94.36924053)(882.38000687,94.37924052)(882.29,94.38924123)
\lineto(881.99,94.41924123)
\curveto(881.94000731,94.43924046)(881.89000736,94.44424046)(881.84,94.43424123)
\curveto(881.78000747,94.43424047)(881.72500753,94.44424046)(881.675,94.46424123)
\curveto(881.50500775,94.51424039)(881.34000791,94.55424035)(881.18,94.58424123)
\curveto(881.01000824,94.61424029)(880.8500084,94.66424024)(880.7,94.73424123)
\curveto(880.24000901,94.92423998)(879.86500939,95.14423976)(879.575,95.39424123)
\curveto(879.28500996,95.65423925)(879.04001021,96.01423889)(878.84,96.47424123)
\curveto(878.79001046,96.6042383)(878.75501049,96.73423817)(878.735,96.86424123)
\curveto(878.71501054,97.0042379)(878.69001056,97.14423776)(878.66,97.28424123)
\curveto(878.6500106,97.35423755)(878.64501061,97.41923748)(878.645,97.47924123)
\curveto(878.64501061,97.53923736)(878.64001061,97.6042373)(878.63,97.67424123)
\curveto(878.61001064,98.5042364)(878.76001049,99.17423573)(879.08,99.68424123)
\curveto(879.39000986,100.19423471)(879.83000942,100.57423433)(880.4,100.82424123)
\curveto(880.52000873,100.87423403)(880.64500861,100.91923398)(880.775,100.95924123)
\curveto(880.90500835,100.9992339)(881.04000821,101.04423386)(881.18,101.09424123)
\curveto(881.26000799,101.11423379)(881.3450079,101.12923377)(881.435,101.13924123)
\lineto(881.675,101.19924123)
\curveto(881.78500747,101.22923367)(881.89500735,101.24423366)(882.005,101.24424123)
\curveto(882.11500714,101.25423365)(882.22500702,101.26923363)(882.335,101.28924123)
\curveto(882.38500687,101.30923359)(882.43000682,101.31423359)(882.47,101.30424123)
\curveto(882.51000674,101.3042336)(882.5500067,101.30923359)(882.59,101.31924123)
\curveto(882.64000661,101.32923357)(882.69500655,101.32923357)(882.755,101.31924123)
\curveto(882.80500645,101.31923358)(882.8550064,101.32423358)(882.905,101.33424123)
\lineto(883.04,101.33424123)
\curveto(883.10000615,101.35423355)(883.17000608,101.35423355)(883.25,101.33424123)
\curveto(883.32000593,101.32423358)(883.38500587,101.32923357)(883.445,101.34924123)
\curveto(883.47500577,101.35923354)(883.51500574,101.36423354)(883.565,101.36424123)
\lineto(883.685,101.36424123)
\lineto(884.15,101.36424123)
\moveto(886.475,99.81924123)
\curveto(886.15500309,99.91923498)(885.79000346,99.97923492)(885.38,99.99924123)
\curveto(884.97000428,100.01923488)(884.56000469,100.02923487)(884.15,100.02924123)
\curveto(883.72000553,100.02923487)(883.30000595,100.01923488)(882.89,99.99924123)
\curveto(882.48000677,99.97923492)(882.09500715,99.93423497)(881.735,99.86424123)
\curveto(881.37500788,99.79423511)(881.0550082,99.68423522)(880.775,99.53424123)
\curveto(880.48500876,99.39423551)(880.250009,99.1992357)(880.07,98.94924123)
\curveto(879.96000929,98.78923611)(879.88000937,98.60923629)(879.83,98.40924123)
\curveto(879.77000948,98.20923669)(879.74000951,97.96423694)(879.74,97.67424123)
\curveto(879.76000949,97.65423725)(879.77000948,97.61923728)(879.77,97.56924123)
\curveto(879.76000949,97.51923738)(879.76000949,97.47923742)(879.77,97.44924123)
\curveto(879.79000946,97.36923753)(879.81000944,97.29423761)(879.83,97.22424123)
\curveto(879.84000941,97.16423774)(879.86000939,97.0992378)(879.89,97.02924123)
\curveto(880.01000924,96.75923814)(880.18000907,96.53923836)(880.4,96.36924123)
\curveto(880.61000864,96.20923869)(880.8550084,96.07423883)(881.135,95.96424123)
\curveto(881.24500801,95.91423899)(881.36500788,95.87423903)(881.495,95.84424123)
\curveto(881.61500763,95.82423908)(881.74000751,95.7992391)(881.87,95.76924123)
\curveto(881.92000733,95.74923915)(881.97500728,95.73923916)(882.035,95.73924123)
\curveto(882.08500716,95.73923916)(882.13500712,95.73423917)(882.185,95.72424123)
\curveto(882.27500697,95.71423919)(882.37000688,95.7042392)(882.47,95.69424123)
\curveto(882.56000669,95.68423922)(882.6550066,95.67423923)(882.755,95.66424123)
\curveto(882.83500641,95.66423924)(882.92000633,95.65923924)(883.01,95.64924123)
\lineto(883.25,95.64924123)
\lineto(883.43,95.64924123)
\curveto(883.46000579,95.63923926)(883.49500575,95.63423927)(883.535,95.63424123)
\lineto(883.67,95.63424123)
\lineto(884.12,95.63424123)
\curveto(884.20000505,95.63423927)(884.28500496,95.62923927)(884.375,95.61924123)
\curveto(884.4550048,95.61923928)(884.53000472,95.62923927)(884.6,95.64924123)
\lineto(884.87,95.64924123)
\curveto(884.89000436,95.64923925)(884.92000433,95.64423926)(884.96,95.63424123)
\curveto(884.99000426,95.63423927)(885.01500423,95.63923926)(885.035,95.64924123)
\curveto(885.13500412,95.65923924)(885.23500401,95.66423924)(885.335,95.66424123)
\curveto(885.42500382,95.67423923)(885.52500373,95.68423922)(885.635,95.69424123)
\curveto(885.75500349,95.72423918)(885.88000337,95.73923916)(886.01,95.73924123)
\curveto(886.13000312,95.74923915)(886.245003,95.77423913)(886.355,95.81424123)
\curveto(886.6550026,95.89423901)(886.92000233,95.97923892)(887.15,96.06924123)
\curveto(887.38000187,96.16923873)(887.59500166,96.31423859)(887.795,96.50424123)
\curveto(887.99500126,96.71423819)(888.14500111,96.97923792)(888.245,97.29924123)
\curveto(888.26500099,97.33923756)(888.27500098,97.37423753)(888.275,97.40424123)
\curveto(888.26500099,97.44423746)(888.27000098,97.48923741)(888.29,97.53924123)
\curveto(888.30000095,97.57923732)(888.31000094,97.64923725)(888.32,97.74924123)
\curveto(888.33000092,97.85923704)(888.32500093,97.94423696)(888.305,98.00424123)
\curveto(888.28500096,98.07423683)(888.27500098,98.14423676)(888.275,98.21424123)
\curveto(888.26500099,98.28423662)(888.250001,98.34923655)(888.23,98.40924123)
\curveto(888.17000108,98.60923629)(888.08500116,98.78923611)(887.975,98.94924123)
\curveto(887.95500129,98.97923592)(887.93500132,99.0042359)(887.915,99.02424123)
\lineto(887.855,99.08424123)
\curveto(887.83500141,99.12423578)(887.79500146,99.17423573)(887.735,99.23424123)
\curveto(887.59500166,99.33423557)(887.46500179,99.41923548)(887.345,99.48924123)
\curveto(887.22500202,99.55923534)(887.08000217,99.62923527)(886.91,99.69924123)
\curveto(886.84000241,99.72923517)(886.77000248,99.74923515)(886.7,99.75924123)
\curveto(886.63000262,99.77923512)(886.55500269,99.7992351)(886.475,99.81924123)
}
}
{
\newrgbcolor{curcolor}{0 0 0}
\pscustom[linestyle=none,fillstyle=solid,fillcolor=curcolor]
{
\newpath
\moveto(878.63,106.77385061)
\curveto(878.63001062,106.87384575)(878.64001061,106.96884566)(878.66,107.05885061)
\curveto(878.67001058,107.14884548)(878.70001055,107.21384541)(878.75,107.25385061)
\curveto(878.83001042,107.31384531)(878.93501031,107.34384528)(879.065,107.34385061)
\lineto(879.455,107.34385061)
\lineto(880.955,107.34385061)
\lineto(887.345,107.34385061)
\lineto(888.515,107.34385061)
\lineto(888.83,107.34385061)
\curveto(888.93000032,107.35384527)(889.01000024,107.33884529)(889.07,107.29885061)
\curveto(889.1500001,107.24884538)(889.20000005,107.17384545)(889.22,107.07385061)
\curveto(889.23000002,106.98384564)(889.23500001,106.87384575)(889.235,106.74385061)
\lineto(889.235,106.51885061)
\curveto(889.21500004,106.43884619)(889.20000005,106.36884626)(889.19,106.30885061)
\curveto(889.17000008,106.24884638)(889.13000012,106.19884643)(889.07,106.15885061)
\curveto(889.01000024,106.11884651)(888.93500032,106.09884653)(888.845,106.09885061)
\lineto(888.545,106.09885061)
\lineto(887.45,106.09885061)
\lineto(882.11,106.09885061)
\curveto(882.02000723,106.07884655)(881.9450073,106.06384656)(881.885,106.05385061)
\curveto(881.81500743,106.05384657)(881.75500749,106.0238466)(881.705,105.96385061)
\curveto(881.6550076,105.89384673)(881.63000762,105.80384682)(881.63,105.69385061)
\curveto(881.62000763,105.59384703)(881.61500763,105.48384714)(881.615,105.36385061)
\lineto(881.615,104.22385061)
\lineto(881.615,103.72885061)
\curveto(881.60500765,103.56884906)(881.5450077,103.45884917)(881.435,103.39885061)
\curveto(881.40500785,103.37884925)(881.37500788,103.36884926)(881.345,103.36885061)
\curveto(881.30500794,103.36884926)(881.26000799,103.36384926)(881.21,103.35385061)
\curveto(881.09000816,103.33384929)(880.98000827,103.33884929)(880.88,103.36885061)
\curveto(880.78000847,103.40884922)(880.71000854,103.46384916)(880.67,103.53385061)
\curveto(880.62000863,103.61384901)(880.59500866,103.73384889)(880.595,103.89385061)
\curveto(880.59500866,104.05384857)(880.58000867,104.18884844)(880.55,104.29885061)
\curveto(880.54000871,104.34884828)(880.53500872,104.40384822)(880.535,104.46385061)
\curveto(880.52500873,104.5238481)(880.51000874,104.58384804)(880.49,104.64385061)
\curveto(880.44000881,104.79384783)(880.39000886,104.93884769)(880.34,105.07885061)
\curveto(880.28000897,105.21884741)(880.21000904,105.35384727)(880.13,105.48385061)
\curveto(880.04000921,105.623847)(879.93500932,105.74384688)(879.815,105.84385061)
\curveto(879.69500955,105.94384668)(879.56500968,106.03884659)(879.425,106.12885061)
\curveto(879.32500993,106.18884644)(879.21501003,106.23384639)(879.095,106.26385061)
\curveto(878.97501028,106.30384632)(878.87001038,106.35384627)(878.78,106.41385061)
\curveto(878.72001053,106.46384616)(878.68001057,106.53384609)(878.66,106.62385061)
\curveto(878.6500106,106.64384598)(878.64501061,106.66884596)(878.645,106.69885061)
\curveto(878.64501061,106.7288459)(878.64001061,106.75384587)(878.63,106.77385061)
}
}
{
\newrgbcolor{curcolor}{0 0 0}
\pscustom[linestyle=none,fillstyle=solid,fillcolor=curcolor]
{
\newpath
\moveto(878.63,115.12345998)
\curveto(878.63001062,115.22345513)(878.64001061,115.31845503)(878.66,115.40845998)
\curveto(878.67001058,115.49845485)(878.70001055,115.56345479)(878.75,115.60345998)
\curveto(878.83001042,115.66345469)(878.93501031,115.69345466)(879.065,115.69345998)
\lineto(879.455,115.69345998)
\lineto(880.955,115.69345998)
\lineto(887.345,115.69345998)
\lineto(888.515,115.69345998)
\lineto(888.83,115.69345998)
\curveto(888.93000032,115.70345465)(889.01000024,115.68845466)(889.07,115.64845998)
\curveto(889.1500001,115.59845475)(889.20000005,115.52345483)(889.22,115.42345998)
\curveto(889.23000002,115.33345502)(889.23500001,115.22345513)(889.235,115.09345998)
\lineto(889.235,114.86845998)
\curveto(889.21500004,114.78845556)(889.20000005,114.71845563)(889.19,114.65845998)
\curveto(889.17000008,114.59845575)(889.13000012,114.5484558)(889.07,114.50845998)
\curveto(889.01000024,114.46845588)(888.93500032,114.4484559)(888.845,114.44845998)
\lineto(888.545,114.44845998)
\lineto(887.45,114.44845998)
\lineto(882.11,114.44845998)
\curveto(882.02000723,114.42845592)(881.9450073,114.41345594)(881.885,114.40345998)
\curveto(881.81500743,114.40345595)(881.75500749,114.37345598)(881.705,114.31345998)
\curveto(881.6550076,114.24345611)(881.63000762,114.1534562)(881.63,114.04345998)
\curveto(881.62000763,113.94345641)(881.61500763,113.83345652)(881.615,113.71345998)
\lineto(881.615,112.57345998)
\lineto(881.615,112.07845998)
\curveto(881.60500765,111.91845843)(881.5450077,111.80845854)(881.435,111.74845998)
\curveto(881.40500785,111.72845862)(881.37500788,111.71845863)(881.345,111.71845998)
\curveto(881.30500794,111.71845863)(881.26000799,111.71345864)(881.21,111.70345998)
\curveto(881.09000816,111.68345867)(880.98000827,111.68845866)(880.88,111.71845998)
\curveto(880.78000847,111.75845859)(880.71000854,111.81345854)(880.67,111.88345998)
\curveto(880.62000863,111.96345839)(880.59500866,112.08345827)(880.595,112.24345998)
\curveto(880.59500866,112.40345795)(880.58000867,112.53845781)(880.55,112.64845998)
\curveto(880.54000871,112.69845765)(880.53500872,112.7534576)(880.535,112.81345998)
\curveto(880.52500873,112.87345748)(880.51000874,112.93345742)(880.49,112.99345998)
\curveto(880.44000881,113.14345721)(880.39000886,113.28845706)(880.34,113.42845998)
\curveto(880.28000897,113.56845678)(880.21000904,113.70345665)(880.13,113.83345998)
\curveto(880.04000921,113.97345638)(879.93500932,114.09345626)(879.815,114.19345998)
\curveto(879.69500955,114.29345606)(879.56500968,114.38845596)(879.425,114.47845998)
\curveto(879.32500993,114.53845581)(879.21501003,114.58345577)(879.095,114.61345998)
\curveto(878.97501028,114.6534557)(878.87001038,114.70345565)(878.78,114.76345998)
\curveto(878.72001053,114.81345554)(878.68001057,114.88345547)(878.66,114.97345998)
\curveto(878.6500106,114.99345536)(878.64501061,115.01845533)(878.645,115.04845998)
\curveto(878.64501061,115.07845527)(878.64001061,115.10345525)(878.63,115.12345998)
}
}
{
\newrgbcolor{curcolor}{0 0 0}
\pscustom[linestyle=none,fillstyle=solid,fillcolor=curcolor]
{
\newpath
\moveto(899.46631592,42.29681936)
\curveto(899.46632661,42.36681368)(899.46632661,42.4468136)(899.46631592,42.53681936)
\curveto(899.45632662,42.62681342)(899.45632662,42.71181333)(899.46631592,42.79181936)
\curveto(899.46632661,42.88181316)(899.4763266,42.96181308)(899.49631592,43.03181936)
\curveto(899.51632656,43.11181293)(899.54632653,43.16681288)(899.58631592,43.19681936)
\curveto(899.63632644,43.22681282)(899.71132637,43.2468128)(899.81131592,43.25681936)
\curveto(899.90132618,43.27681277)(900.00632607,43.28681276)(900.12631592,43.28681936)
\curveto(900.23632584,43.29681275)(900.35132573,43.29681275)(900.47131592,43.28681936)
\lineto(900.77131592,43.28681936)
\lineto(903.78631592,43.28681936)
\lineto(906.68131592,43.28681936)
\curveto(907.01131907,43.28681276)(907.33631874,43.28181276)(907.65631592,43.27181936)
\curveto(907.96631811,43.27181277)(908.24631783,43.23181281)(908.49631592,43.15181936)
\curveto(908.84631723,43.03181301)(909.14131694,42.87681317)(909.38131592,42.68681936)
\curveto(909.61131647,42.49681355)(909.81131627,42.25681379)(909.98131592,41.96681936)
\curveto(910.03131605,41.90681414)(910.06631601,41.8418142)(910.08631592,41.77181936)
\curveto(910.10631597,41.71181433)(910.13131595,41.6418144)(910.16131592,41.56181936)
\curveto(910.21131587,41.4418146)(910.24631583,41.31181473)(910.26631592,41.17181936)
\curveto(910.29631578,41.041815)(910.32631575,40.90681514)(910.35631592,40.76681936)
\curveto(910.3763157,40.71681533)(910.3813157,40.66681538)(910.37131592,40.61681936)
\curveto(910.36131572,40.56681548)(910.36131572,40.51181553)(910.37131592,40.45181936)
\curveto(910.3813157,40.43181561)(910.3813157,40.40681564)(910.37131592,40.37681936)
\curveto(910.37131571,40.3468157)(910.3763157,40.32181572)(910.38631592,40.30181936)
\curveto(910.39631568,40.26181578)(910.40131568,40.20681584)(910.40131592,40.13681936)
\curveto(910.40131568,40.06681598)(910.39631568,40.01181603)(910.38631592,39.97181936)
\curveto(910.3763157,39.92181612)(910.3763157,39.86681618)(910.38631592,39.80681936)
\curveto(910.39631568,39.7468163)(910.39131569,39.69181635)(910.37131592,39.64181936)
\curveto(910.34131574,39.51181653)(910.32131576,39.38681666)(910.31131592,39.26681936)
\curveto(910.30131578,39.1468169)(910.2763158,39.03181701)(910.23631592,38.92181936)
\curveto(910.11631596,38.55181749)(909.94631613,38.23181781)(909.72631592,37.96181936)
\curveto(909.50631657,37.69181835)(909.22631685,37.48181856)(908.88631592,37.33181936)
\curveto(908.76631731,37.28181876)(908.64131744,37.23681881)(908.51131592,37.19681936)
\curveto(908.3813177,37.16681888)(908.24631783,37.13181891)(908.10631592,37.09181936)
\curveto(908.05631802,37.08181896)(908.01631806,37.07681897)(907.98631592,37.07681936)
\curveto(907.94631813,37.07681897)(907.90131818,37.07181897)(907.85131592,37.06181936)
\curveto(907.82131826,37.05181899)(907.78631829,37.046819)(907.74631592,37.04681936)
\curveto(907.69631838,37.046819)(907.65631842,37.041819)(907.62631592,37.03181936)
\lineto(907.46131592,37.03181936)
\curveto(907.3813187,37.01181903)(907.2813188,37.00681904)(907.16131592,37.01681936)
\curveto(907.03131905,37.02681902)(906.94131914,37.041819)(906.89131592,37.06181936)
\curveto(906.80131928,37.08181896)(906.73631934,37.13681891)(906.69631592,37.22681936)
\curveto(906.6763194,37.25681879)(906.67131941,37.28681876)(906.68131592,37.31681936)
\curveto(906.6813194,37.3468187)(906.6763194,37.38681866)(906.66631592,37.43681936)
\curveto(906.65631942,37.47681857)(906.65131943,37.51681853)(906.65131592,37.55681936)
\lineto(906.65131592,37.70681936)
\curveto(906.65131943,37.82681822)(906.65631942,37.9468181)(906.66631592,38.06681936)
\curveto(906.66631941,38.19681785)(906.70131938,38.28681776)(906.77131592,38.33681936)
\curveto(906.83131925,38.37681767)(906.89131919,38.39681765)(906.95131592,38.39681936)
\curveto(907.01131907,38.39681765)(907.081319,38.40681764)(907.16131592,38.42681936)
\curveto(907.19131889,38.43681761)(907.22631885,38.43681761)(907.26631592,38.42681936)
\curveto(907.29631878,38.42681762)(907.32131876,38.43181761)(907.34131592,38.44181936)
\lineto(907.55131592,38.44181936)
\curveto(907.60131848,38.46181758)(907.65131843,38.46681758)(907.70131592,38.45681936)
\curveto(907.74131834,38.45681759)(907.78631829,38.46681758)(907.83631592,38.48681936)
\curveto(907.96631811,38.51681753)(908.09131799,38.5468175)(908.21131592,38.57681936)
\curveto(908.32131776,38.60681744)(908.42631765,38.65181739)(908.52631592,38.71181936)
\curveto(908.81631726,38.88181716)(909.02131706,39.15181689)(909.14131592,39.52181936)
\curveto(909.16131692,39.57181647)(909.1763169,39.62181642)(909.18631592,39.67181936)
\curveto(909.18631689,39.73181631)(909.19631688,39.78681626)(909.21631592,39.83681936)
\lineto(909.21631592,39.91181936)
\curveto(909.22631685,39.98181606)(909.23631684,40.07681597)(909.24631592,40.19681936)
\curveto(909.24631683,40.32681572)(909.23631684,40.42681562)(909.21631592,40.49681936)
\curveto(909.19631688,40.56681548)(909.1813169,40.63681541)(909.17131592,40.70681936)
\curveto(909.15131693,40.78681526)(909.13131695,40.85681519)(909.11131592,40.91681936)
\curveto(908.95131713,41.29681475)(908.6763174,41.57181447)(908.28631592,41.74181936)
\curveto(908.15631792,41.79181425)(908.00131808,41.82681422)(907.82131592,41.84681936)
\curveto(907.64131844,41.87681417)(907.45631862,41.89181415)(907.26631592,41.89181936)
\curveto(907.06631901,41.90181414)(906.86631921,41.90181414)(906.66631592,41.89181936)
\lineto(906.09631592,41.89181936)
\lineto(901.85131592,41.89181936)
\lineto(900.30631592,41.89181936)
\curveto(900.19632588,41.89181415)(900.076326,41.88681416)(899.94631592,41.87681936)
\curveto(899.81632626,41.86681418)(899.71132637,41.88681416)(899.63131592,41.93681936)
\curveto(899.56132652,41.99681405)(899.51132657,42.07681397)(899.48131592,42.17681936)
\curveto(899.4813266,42.19681385)(899.4813266,42.21681383)(899.48131592,42.23681936)
\curveto(899.4813266,42.25681379)(899.4763266,42.27681377)(899.46631592,42.29681936)
}
}
{
\newrgbcolor{curcolor}{0 0 0}
\pscustom[linestyle=none,fillstyle=solid,fillcolor=curcolor]
{
\newpath
\moveto(902.42131592,45.83049123)
\lineto(902.42131592,46.26549123)
\curveto(902.42132366,46.41548927)(902.46132362,46.52048916)(902.54131592,46.58049123)
\curveto(902.62132346,46.63048905)(902.72132336,46.65548903)(902.84131592,46.65549123)
\curveto(902.96132312,46.66548902)(903.081323,46.67048901)(903.20131592,46.67049123)
\lineto(904.62631592,46.67049123)
\lineto(906.89131592,46.67049123)
\lineto(907.58131592,46.67049123)
\curveto(907.81131827,46.67048901)(908.01131807,46.69548899)(908.18131592,46.74549123)
\curveto(908.63131745,46.90548878)(908.94631713,47.20548848)(909.12631592,47.64549123)
\curveto(909.21631686,47.86548782)(909.25131683,48.13048755)(909.23131592,48.44049123)
\curveto(909.20131688,48.75048693)(909.14631693,49.00048668)(909.06631592,49.19049123)
\curveto(908.92631715,49.52048616)(908.75131733,49.7804859)(908.54131592,49.97049123)
\curveto(908.32131776,50.17048551)(908.03631804,50.32548536)(907.68631592,50.43549123)
\curveto(907.60631847,50.46548522)(907.52631855,50.4854852)(907.44631592,50.49549123)
\curveto(907.36631871,50.50548518)(907.2813188,50.52048516)(907.19131592,50.54049123)
\curveto(907.14131894,50.55048513)(907.09631898,50.55048513)(907.05631592,50.54049123)
\curveto(907.01631906,50.54048514)(906.97131911,50.55048513)(906.92131592,50.57049123)
\lineto(906.60631592,50.57049123)
\curveto(906.52631955,50.59048509)(906.43631964,50.59548509)(906.33631592,50.58549123)
\curveto(906.22631985,50.57548511)(906.12631995,50.57048511)(906.03631592,50.57049123)
\lineto(904.86631592,50.57049123)
\lineto(903.27631592,50.57049123)
\curveto(903.15632292,50.57048511)(903.03132305,50.56548512)(902.90131592,50.55549123)
\curveto(902.76132332,50.55548513)(902.65132343,50.5804851)(902.57131592,50.63049123)
\curveto(902.52132356,50.67048501)(902.49132359,50.71548497)(902.48131592,50.76549123)
\curveto(902.46132362,50.82548486)(902.44132364,50.89548479)(902.42131592,50.97549123)
\lineto(902.42131592,51.20049123)
\curveto(902.42132366,51.32048436)(902.42632365,51.42548426)(902.43631592,51.51549123)
\curveto(902.44632363,51.61548407)(902.49132359,51.69048399)(902.57131592,51.74049123)
\curveto(902.62132346,51.79048389)(902.69632338,51.81548387)(902.79631592,51.81549123)
\lineto(903.08131592,51.81549123)
\lineto(904.10131592,51.81549123)
\lineto(908.13631592,51.81549123)
\lineto(909.48631592,51.81549123)
\curveto(909.60631647,51.81548387)(909.72131636,51.81048387)(909.83131592,51.80049123)
\curveto(909.93131615,51.80048388)(910.00631607,51.76548392)(910.05631592,51.69549123)
\curveto(910.08631599,51.65548403)(910.11131597,51.59548409)(910.13131592,51.51549123)
\curveto(910.14131594,51.43548425)(910.15131593,51.34548434)(910.16131592,51.24549123)
\curveto(910.16131592,51.15548453)(910.15631592,51.06548462)(910.14631592,50.97549123)
\curveto(910.13631594,50.89548479)(910.11631596,50.83548485)(910.08631592,50.79549123)
\curveto(910.04631603,50.74548494)(909.9813161,50.70048498)(909.89131592,50.66049123)
\curveto(909.85131623,50.65048503)(909.79631628,50.64048504)(909.72631592,50.63049123)
\curveto(909.65631642,50.63048505)(909.59131649,50.62548506)(909.53131592,50.61549123)
\curveto(909.46131662,50.60548508)(909.40631667,50.5854851)(909.36631592,50.55549123)
\curveto(909.32631675,50.52548516)(909.31131677,50.4804852)(909.32131592,50.42049123)
\curveto(909.34131674,50.34048534)(909.40131668,50.26048542)(909.50131592,50.18049123)
\curveto(909.59131649,50.10048558)(909.66131642,50.02548566)(909.71131592,49.95549123)
\curveto(909.87131621,49.73548595)(910.01131607,49.4854862)(910.13131592,49.20549123)
\curveto(910.1813159,49.09548659)(910.21131587,48.9804867)(910.22131592,48.86049123)
\curveto(910.24131584,48.75048693)(910.26631581,48.63548705)(910.29631592,48.51549123)
\curveto(910.30631577,48.46548722)(910.30631577,48.41048727)(910.29631592,48.35049123)
\curveto(910.28631579,48.30048738)(910.29131579,48.25048743)(910.31131592,48.20049123)
\curveto(910.33131575,48.10048758)(910.33131575,48.01048767)(910.31131592,47.93049123)
\lineto(910.31131592,47.78049123)
\curveto(910.29131579,47.73048795)(910.2813158,47.67048801)(910.28131592,47.60049123)
\curveto(910.2813158,47.54048814)(910.2763158,47.4854882)(910.26631592,47.43549123)
\curveto(910.24631583,47.39548829)(910.23631584,47.35548833)(910.23631592,47.31549123)
\curveto(910.24631583,47.2854884)(910.24131584,47.24548844)(910.22131592,47.19549123)
\lineto(910.16131592,46.95549123)
\curveto(910.14131594,46.8854888)(910.11131597,46.81048887)(910.07131592,46.73049123)
\curveto(909.96131612,46.47048921)(909.81631626,46.25048943)(909.63631592,46.07049123)
\curveto(909.44631663,45.90048978)(909.22131686,45.76048992)(908.96131592,45.65049123)
\curveto(908.87131721,45.61049007)(908.7813173,45.5804901)(908.69131592,45.56049123)
\lineto(908.39131592,45.50049123)
\curveto(908.33131775,45.4804902)(908.2763178,45.47049021)(908.22631592,45.47049123)
\curveto(908.16631791,45.4804902)(908.10131798,45.47549021)(908.03131592,45.45549123)
\curveto(908.01131807,45.44549024)(907.98631809,45.44049024)(907.95631592,45.44049123)
\curveto(907.91631816,45.44049024)(907.8813182,45.43549025)(907.85131592,45.42549123)
\lineto(907.70131592,45.42549123)
\curveto(907.66131842,45.41549027)(907.61631846,45.41049027)(907.56631592,45.41049123)
\curveto(907.50631857,45.42049026)(907.45131863,45.42549026)(907.40131592,45.42549123)
\lineto(906.80131592,45.42549123)
\lineto(904.04131592,45.42549123)
\lineto(903.08131592,45.42549123)
\lineto(902.81131592,45.42549123)
\curveto(902.72132336,45.42549026)(902.64632343,45.44549024)(902.58631592,45.48549123)
\curveto(902.51632356,45.52549016)(902.46632361,45.60049008)(902.43631592,45.71049123)
\curveto(902.42632365,45.73048995)(902.42632365,45.75048993)(902.43631592,45.77049123)
\curveto(902.43632364,45.79048989)(902.43132365,45.81048987)(902.42131592,45.83049123)
}
}
{
\newrgbcolor{curcolor}{0 0 0}
\pscustom[linestyle=none,fillstyle=solid,fillcolor=curcolor]
{
\newpath
\moveto(899.46631592,54.28510061)
\curveto(899.46632661,54.41509899)(899.46632661,54.55009886)(899.46631592,54.69010061)
\curveto(899.46632661,54.84009857)(899.50132658,54.95009846)(899.57131592,55.02010061)
\curveto(899.64132644,55.07009834)(899.73632634,55.09509831)(899.85631592,55.09510061)
\curveto(899.96632611,55.1050983)(900.081326,55.1100983)(900.20131592,55.11010061)
\lineto(901.53631592,55.11010061)
\lineto(907.61131592,55.11010061)
\lineto(909.29131592,55.11010061)
\lineto(909.68131592,55.11010061)
\curveto(909.82131626,55.1100983)(909.93131615,55.08509832)(910.01131592,55.03510061)
\curveto(910.06131602,55.0050984)(910.09131599,54.96009845)(910.10131592,54.90010061)
\curveto(910.11131597,54.85009856)(910.12631595,54.78509862)(910.14631592,54.70510061)
\lineto(910.14631592,54.49510061)
\lineto(910.14631592,54.18010061)
\curveto(910.13631594,54.08009933)(910.10131598,54.0050994)(910.04131592,53.95510061)
\curveto(909.96131612,53.9050995)(909.86131622,53.87509953)(909.74131592,53.86510061)
\lineto(909.36631592,53.86510061)
\lineto(907.98631592,53.86510061)
\lineto(901.74631592,53.86510061)
\lineto(900.27631592,53.86510061)
\curveto(900.16632591,53.86509954)(900.05132603,53.86009955)(899.93131592,53.85010061)
\curveto(899.80132628,53.85009956)(899.70132638,53.87509953)(899.63131592,53.92510061)
\curveto(899.57132651,53.96509944)(899.52132656,54.04009937)(899.48131592,54.15010061)
\curveto(899.47132661,54.17009924)(899.47132661,54.19009922)(899.48131592,54.21010061)
\curveto(899.4813266,54.24009917)(899.4763266,54.26509914)(899.46631592,54.28510061)
}
}
{
\newrgbcolor{curcolor}{0 0 0}
\pscustom[linestyle=none,fillstyle=solid,fillcolor=curcolor]
{
}
}
{
\newrgbcolor{curcolor}{0 0 0}
\pscustom[linestyle=none,fillstyle=solid,fillcolor=curcolor]
{
\newpath
\moveto(899.54131592,64.24510061)
\curveto(899.53132655,64.93509597)(899.65132643,65.53509537)(899.90131592,66.04510061)
\curveto(900.15132593,66.56509434)(900.48632559,66.96009395)(900.90631592,67.23010061)
\curveto(900.98632509,67.28009363)(901.076325,67.32509358)(901.17631592,67.36510061)
\curveto(901.26632481,67.4050935)(901.36132472,67.45009346)(901.46131592,67.50010061)
\curveto(901.56132452,67.54009337)(901.66132442,67.57009334)(901.76131592,67.59010061)
\curveto(901.86132422,67.6100933)(901.96632411,67.63009328)(902.07631592,67.65010061)
\curveto(902.12632395,67.67009324)(902.17132391,67.67509323)(902.21131592,67.66510061)
\curveto(902.25132383,67.65509325)(902.29632378,67.66009325)(902.34631592,67.68010061)
\curveto(902.39632368,67.69009322)(902.4813236,67.69509321)(902.60131592,67.69510061)
\curveto(902.71132337,67.69509321)(902.79632328,67.69009322)(902.85631592,67.68010061)
\curveto(902.91632316,67.66009325)(902.9763231,67.65009326)(903.03631592,67.65010061)
\curveto(903.09632298,67.66009325)(903.15632292,67.65509325)(903.21631592,67.63510061)
\curveto(903.35632272,67.59509331)(903.49132259,67.56009335)(903.62131592,67.53010061)
\curveto(903.75132233,67.50009341)(903.8763222,67.46009345)(903.99631592,67.41010061)
\curveto(904.13632194,67.35009356)(904.26132182,67.28009363)(904.37131592,67.20010061)
\curveto(904.4813216,67.13009378)(904.59132149,67.05509385)(904.70131592,66.97510061)
\lineto(904.76131592,66.91510061)
\curveto(904.7813213,66.905094)(904.80132128,66.89009402)(904.82131592,66.87010061)
\curveto(904.9813211,66.75009416)(905.12632095,66.61509429)(905.25631592,66.46510061)
\curveto(905.38632069,66.31509459)(905.51132057,66.15509475)(905.63131592,65.98510061)
\curveto(905.85132023,65.67509523)(906.05632002,65.38009553)(906.24631592,65.10010061)
\curveto(906.38631969,64.87009604)(906.52131956,64.64009627)(906.65131592,64.41010061)
\curveto(906.7813193,64.19009672)(906.91631916,63.97009694)(907.05631592,63.75010061)
\curveto(907.22631885,63.50009741)(907.40631867,63.26009765)(907.59631592,63.03010061)
\curveto(907.78631829,62.8100981)(908.01131807,62.62009829)(908.27131592,62.46010061)
\curveto(908.33131775,62.42009849)(908.39131769,62.38509852)(908.45131592,62.35510061)
\curveto(908.50131758,62.32509858)(908.56631751,62.29509861)(908.64631592,62.26510061)
\curveto(908.71631736,62.24509866)(908.7763173,62.24009867)(908.82631592,62.25010061)
\curveto(908.89631718,62.27009864)(908.95131713,62.3050986)(908.99131592,62.35510061)
\curveto(909.02131706,62.4050985)(909.04131704,62.46509844)(909.05131592,62.53510061)
\lineto(909.05131592,62.77510061)
\lineto(909.05131592,63.52510061)
\lineto(909.05131592,66.33010061)
\lineto(909.05131592,66.99010061)
\curveto(909.05131703,67.08009383)(909.05631702,67.16509374)(909.06631592,67.24510061)
\curveto(909.06631701,67.32509358)(909.08631699,67.39009352)(909.12631592,67.44010061)
\curveto(909.16631691,67.49009342)(909.24131684,67.53009338)(909.35131592,67.56010061)
\curveto(909.45131663,67.60009331)(909.55131653,67.6100933)(909.65131592,67.59010061)
\lineto(909.78631592,67.59010061)
\curveto(909.85631622,67.57009334)(909.91631616,67.55009336)(909.96631592,67.53010061)
\curveto(910.01631606,67.5100934)(910.05631602,67.47509343)(910.08631592,67.42510061)
\curveto(910.12631595,67.37509353)(910.14631593,67.3050936)(910.14631592,67.21510061)
\lineto(910.14631592,66.94510061)
\lineto(910.14631592,66.04510061)
\lineto(910.14631592,62.53510061)
\lineto(910.14631592,61.47010061)
\curveto(910.14631593,61.39009952)(910.15131593,61.30009961)(910.16131592,61.20010061)
\curveto(910.16131592,61.10009981)(910.15131593,61.01509989)(910.13131592,60.94510061)
\curveto(910.06131602,60.73510017)(909.8813162,60.67010024)(909.59131592,60.75010061)
\curveto(909.55131653,60.76010015)(909.51631656,60.76010015)(909.48631592,60.75010061)
\curveto(909.44631663,60.75010016)(909.40131668,60.76010015)(909.35131592,60.78010061)
\curveto(909.27131681,60.80010011)(909.18631689,60.82010009)(909.09631592,60.84010061)
\curveto(909.00631707,60.86010005)(908.92131716,60.88510002)(908.84131592,60.91510061)
\curveto(908.35131773,61.07509983)(907.93631814,61.27509963)(907.59631592,61.51510061)
\curveto(907.34631873,61.69509921)(907.12131896,61.90009901)(906.92131592,62.13010061)
\curveto(906.71131937,62.36009855)(906.51631956,62.60009831)(906.33631592,62.85010061)
\curveto(906.15631992,63.1100978)(905.98632009,63.37509753)(905.82631592,63.64510061)
\curveto(905.65632042,63.92509698)(905.4813206,64.19509671)(905.30131592,64.45510061)
\curveto(905.22132086,64.56509634)(905.14632093,64.67009624)(905.07631592,64.77010061)
\curveto(905.00632107,64.88009603)(904.93132115,64.99009592)(904.85131592,65.10010061)
\curveto(904.82132126,65.14009577)(904.79132129,65.17509573)(904.76131592,65.20510061)
\curveto(904.72132136,65.24509566)(904.69132139,65.28509562)(904.67131592,65.32510061)
\curveto(904.56132152,65.46509544)(904.43632164,65.59009532)(904.29631592,65.70010061)
\curveto(904.26632181,65.72009519)(904.24132184,65.74509516)(904.22131592,65.77510061)
\curveto(904.19132189,65.8050951)(904.16132192,65.83009508)(904.13131592,65.85010061)
\curveto(904.03132205,65.93009498)(903.93132215,65.99509491)(903.83131592,66.04510061)
\curveto(903.73132235,66.1050948)(903.62132246,66.16009475)(903.50131592,66.21010061)
\curveto(903.43132265,66.24009467)(903.35632272,66.26009465)(903.27631592,66.27010061)
\lineto(903.03631592,66.33010061)
\lineto(902.94631592,66.33010061)
\curveto(902.91632316,66.34009457)(902.88632319,66.34509456)(902.85631592,66.34510061)
\curveto(902.78632329,66.36509454)(902.69132339,66.37009454)(902.57131592,66.36010061)
\curveto(902.44132364,66.36009455)(902.34132374,66.35009456)(902.27131592,66.33010061)
\curveto(902.19132389,66.3100946)(902.11632396,66.29009462)(902.04631592,66.27010061)
\curveto(901.96632411,66.26009465)(901.88632419,66.24009467)(901.80631592,66.21010061)
\curveto(901.56632451,66.10009481)(901.36632471,65.95009496)(901.20631592,65.76010061)
\curveto(901.03632504,65.58009533)(900.89632518,65.36009555)(900.78631592,65.10010061)
\curveto(900.76632531,65.03009588)(900.75132533,64.96009595)(900.74131592,64.89010061)
\curveto(900.72132536,64.82009609)(900.70132538,64.74509616)(900.68131592,64.66510061)
\curveto(900.66132542,64.58509632)(900.65132543,64.47509643)(900.65131592,64.33510061)
\curveto(900.65132543,64.2050967)(900.66132542,64.10009681)(900.68131592,64.02010061)
\curveto(900.69132539,63.96009695)(900.69632538,63.905097)(900.69631592,63.85510061)
\curveto(900.69632538,63.8050971)(900.70632537,63.75509715)(900.72631592,63.70510061)
\curveto(900.76632531,63.6050973)(900.80632527,63.5100974)(900.84631592,63.42010061)
\curveto(900.88632519,63.34009757)(900.93132515,63.26009765)(900.98131592,63.18010061)
\curveto(901.00132508,63.15009776)(901.02632505,63.12009779)(901.05631592,63.09010061)
\curveto(901.08632499,63.07009784)(901.11132497,63.04509786)(901.13131592,63.01510061)
\lineto(901.20631592,62.94010061)
\curveto(901.22632485,62.910098)(901.24632483,62.88509802)(901.26631592,62.86510061)
\lineto(901.47631592,62.71510061)
\curveto(901.53632454,62.67509823)(901.60132448,62.63009828)(901.67131592,62.58010061)
\curveto(901.76132432,62.52009839)(901.86632421,62.47009844)(901.98631592,62.43010061)
\curveto(902.09632398,62.40009851)(902.20632387,62.36509854)(902.31631592,62.32510061)
\curveto(902.42632365,62.28509862)(902.57132351,62.26009865)(902.75131592,62.25010061)
\curveto(902.92132316,62.24009867)(903.04632303,62.2100987)(903.12631592,62.16010061)
\curveto(903.20632287,62.1100988)(903.25132283,62.03509887)(903.26131592,61.93510061)
\curveto(903.27132281,61.83509907)(903.2763228,61.72509918)(903.27631592,61.60510061)
\curveto(903.2763228,61.56509934)(903.2813228,61.52509938)(903.29131592,61.48510061)
\curveto(903.29132279,61.44509946)(903.28632279,61.4100995)(903.27631592,61.38010061)
\curveto(903.25632282,61.33009958)(903.24632283,61.28009963)(903.24631592,61.23010061)
\curveto(903.24632283,61.19009972)(903.23632284,61.15009976)(903.21631592,61.11010061)
\curveto(903.15632292,61.02009989)(903.02132306,60.97509993)(902.81131592,60.97510061)
\lineto(902.69131592,60.97510061)
\curveto(902.63132345,60.98509992)(902.57132351,60.99009992)(902.51131592,60.99010061)
\curveto(902.44132364,61.00009991)(902.3763237,61.0100999)(902.31631592,61.02010061)
\curveto(902.20632387,61.04009987)(902.10632397,61.06009985)(902.01631592,61.08010061)
\curveto(901.91632416,61.10009981)(901.82132426,61.13009978)(901.73131592,61.17010061)
\curveto(901.66132442,61.19009972)(901.60132448,61.2100997)(901.55131592,61.23010061)
\lineto(901.37131592,61.29010061)
\curveto(901.11132497,61.4100995)(900.86632521,61.56509934)(900.63631592,61.75510061)
\curveto(900.40632567,61.95509895)(900.22132586,62.17009874)(900.08131592,62.40010061)
\curveto(900.00132608,62.5100984)(899.93632614,62.62509828)(899.88631592,62.74510061)
\lineto(899.73631592,63.13510061)
\curveto(899.68632639,63.24509766)(899.65632642,63.36009755)(899.64631592,63.48010061)
\curveto(899.62632645,63.60009731)(899.60132648,63.72509718)(899.57131592,63.85510061)
\curveto(899.57132651,63.92509698)(899.57132651,63.99009692)(899.57131592,64.05010061)
\curveto(899.56132652,64.1100968)(899.55132653,64.17509673)(899.54131592,64.24510061)
}
}
{
\newrgbcolor{curcolor}{0 0 0}
\pscustom[linestyle=none,fillstyle=solid,fillcolor=curcolor]
{
\newpath
\moveto(905.06131592,76.34470998)
\lineto(905.31631592,76.34470998)
\curveto(905.39632068,76.35470228)(905.47132061,76.34970228)(905.54131592,76.32970998)
\lineto(905.78131592,76.32970998)
\lineto(905.94631592,76.32970998)
\curveto(906.04632003,76.30970232)(906.15131993,76.29970233)(906.26131592,76.29970998)
\curveto(906.36131972,76.29970233)(906.46131962,76.28970234)(906.56131592,76.26970998)
\lineto(906.71131592,76.26970998)
\curveto(906.85131923,76.23970239)(906.99131909,76.21970241)(907.13131592,76.20970998)
\curveto(907.26131882,76.19970243)(907.39131869,76.17470246)(907.52131592,76.13470998)
\curveto(907.60131848,76.11470252)(907.68631839,76.09470254)(907.77631592,76.07470998)
\lineto(908.01631592,76.01470998)
\lineto(908.31631592,75.89470998)
\curveto(908.40631767,75.86470277)(908.49631758,75.8297028)(908.58631592,75.78970998)
\curveto(908.80631727,75.68970294)(909.02131706,75.55470308)(909.23131592,75.38470998)
\curveto(909.44131664,75.22470341)(909.61131647,75.04970358)(909.74131592,74.85970998)
\curveto(909.7813163,74.80970382)(909.82131626,74.74970388)(909.86131592,74.67970998)
\curveto(909.89131619,74.61970401)(909.92631615,74.55970407)(909.96631592,74.49970998)
\curveto(910.01631606,74.41970421)(910.05631602,74.32470431)(910.08631592,74.21470998)
\curveto(910.11631596,74.10470453)(910.14631593,73.99970463)(910.17631592,73.89970998)
\curveto(910.21631586,73.78970484)(910.24131584,73.67970495)(910.25131592,73.56970998)
\curveto(910.26131582,73.45970517)(910.2763158,73.34470529)(910.29631592,73.22470998)
\curveto(910.30631577,73.18470545)(910.30631577,73.13970549)(910.29631592,73.08970998)
\curveto(910.29631578,73.04970558)(910.30131578,73.00970562)(910.31131592,72.96970998)
\curveto(910.32131576,72.9297057)(910.32631575,72.87470576)(910.32631592,72.80470998)
\curveto(910.32631575,72.7347059)(910.32131576,72.68470595)(910.31131592,72.65470998)
\curveto(910.29131579,72.60470603)(910.28631579,72.55970607)(910.29631592,72.51970998)
\curveto(910.30631577,72.47970615)(910.30631577,72.44470619)(910.29631592,72.41470998)
\lineto(910.29631592,72.32470998)
\curveto(910.2763158,72.26470637)(910.26131582,72.19970643)(910.25131592,72.12970998)
\curveto(910.25131583,72.06970656)(910.24631583,72.00470663)(910.23631592,71.93470998)
\curveto(910.18631589,71.76470687)(910.13631594,71.60470703)(910.08631592,71.45470998)
\curveto(910.03631604,71.30470733)(909.97131611,71.15970747)(909.89131592,71.01970998)
\curveto(909.85131623,70.96970766)(909.82131626,70.91470772)(909.80131592,70.85470998)
\curveto(909.77131631,70.80470783)(909.73631634,70.75470788)(909.69631592,70.70470998)
\curveto(909.51631656,70.46470817)(909.29631678,70.26470837)(909.03631592,70.10470998)
\curveto(908.7763173,69.94470869)(908.49131759,69.80470883)(908.18131592,69.68470998)
\curveto(908.04131804,69.62470901)(907.90131818,69.57970905)(907.76131592,69.54970998)
\curveto(907.61131847,69.51970911)(907.45631862,69.48470915)(907.29631592,69.44470998)
\curveto(907.18631889,69.42470921)(907.076319,69.40970922)(906.96631592,69.39970998)
\curveto(906.85631922,69.38970924)(906.74631933,69.37470926)(906.63631592,69.35470998)
\curveto(906.59631948,69.34470929)(906.55631952,69.33970929)(906.51631592,69.33970998)
\curveto(906.4763196,69.34970928)(906.43631964,69.34970928)(906.39631592,69.33970998)
\curveto(906.34631973,69.3297093)(906.29631978,69.32470931)(906.24631592,69.32470998)
\lineto(906.08131592,69.32470998)
\curveto(906.03132005,69.30470933)(905.9813201,69.29970933)(905.93131592,69.30970998)
\curveto(905.87132021,69.31970931)(905.81632026,69.31970931)(905.76631592,69.30970998)
\curveto(905.72632035,69.29970933)(905.6813204,69.29970933)(905.63131592,69.30970998)
\curveto(905.5813205,69.31970931)(905.53132055,69.31470932)(905.48131592,69.29470998)
\curveto(905.41132067,69.27470936)(905.33632074,69.26970936)(905.25631592,69.27970998)
\curveto(905.16632091,69.28970934)(905.081321,69.29470934)(905.00131592,69.29470998)
\curveto(904.91132117,69.29470934)(904.81132127,69.28970934)(904.70131592,69.27970998)
\curveto(904.5813215,69.26970936)(904.4813216,69.27470936)(904.40131592,69.29470998)
\lineto(904.11631592,69.29470998)
\lineto(903.48631592,69.33970998)
\curveto(903.38632269,69.34970928)(903.29132279,69.35970927)(903.20131592,69.36970998)
\lineto(902.90131592,69.39970998)
\curveto(902.85132323,69.41970921)(902.80132328,69.42470921)(902.75131592,69.41470998)
\curveto(902.69132339,69.41470922)(902.63632344,69.42470921)(902.58631592,69.44470998)
\curveto(902.41632366,69.49470914)(902.25132383,69.5347091)(902.09131592,69.56470998)
\curveto(901.92132416,69.59470904)(901.76132432,69.64470899)(901.61131592,69.71470998)
\curveto(901.15132493,69.90470873)(900.7763253,70.12470851)(900.48631592,70.37470998)
\curveto(900.19632588,70.634708)(899.95132613,70.99470764)(899.75131592,71.45470998)
\curveto(899.70132638,71.58470705)(899.66632641,71.71470692)(899.64631592,71.84470998)
\curveto(899.62632645,71.98470665)(899.60132648,72.12470651)(899.57131592,72.26470998)
\curveto(899.56132652,72.3347063)(899.55632652,72.39970623)(899.55631592,72.45970998)
\curveto(899.55632652,72.51970611)(899.55132653,72.58470605)(899.54131592,72.65470998)
\curveto(899.52132656,73.48470515)(899.67132641,74.15470448)(899.99131592,74.66470998)
\curveto(900.30132578,75.17470346)(900.74132534,75.55470308)(901.31131592,75.80470998)
\curveto(901.43132465,75.85470278)(901.55632452,75.89970273)(901.68631592,75.93970998)
\curveto(901.81632426,75.97970265)(901.95132413,76.02470261)(902.09131592,76.07470998)
\curveto(902.17132391,76.09470254)(902.25632382,76.10970252)(902.34631592,76.11970998)
\lineto(902.58631592,76.17970998)
\curveto(902.69632338,76.20970242)(902.80632327,76.22470241)(902.91631592,76.22470998)
\curveto(903.02632305,76.2347024)(903.13632294,76.24970238)(903.24631592,76.26970998)
\curveto(903.29632278,76.28970234)(903.34132274,76.29470234)(903.38131592,76.28470998)
\curveto(903.42132266,76.28470235)(903.46132262,76.28970234)(903.50131592,76.29970998)
\curveto(903.55132253,76.30970232)(903.60632247,76.30970232)(903.66631592,76.29970998)
\curveto(903.71632236,76.29970233)(903.76632231,76.30470233)(903.81631592,76.31470998)
\lineto(903.95131592,76.31470998)
\curveto(904.01132207,76.3347023)(904.081322,76.3347023)(904.16131592,76.31470998)
\curveto(904.23132185,76.30470233)(904.29632178,76.30970232)(904.35631592,76.32970998)
\curveto(904.38632169,76.33970229)(904.42632165,76.34470229)(904.47631592,76.34470998)
\lineto(904.59631592,76.34470998)
\lineto(905.06131592,76.34470998)
\moveto(907.38631592,74.79970998)
\curveto(907.06631901,74.89970373)(906.70131938,74.95970367)(906.29131592,74.97970998)
\curveto(905.8813202,74.99970363)(905.47132061,75.00970362)(905.06131592,75.00970998)
\curveto(904.63132145,75.00970362)(904.21132187,74.99970363)(903.80131592,74.97970998)
\curveto(903.39132269,74.95970367)(903.00632307,74.91470372)(902.64631592,74.84470998)
\curveto(902.28632379,74.77470386)(901.96632411,74.66470397)(901.68631592,74.51470998)
\curveto(901.39632468,74.37470426)(901.16132492,74.17970445)(900.98131592,73.92970998)
\curveto(900.87132521,73.76970486)(900.79132529,73.58970504)(900.74131592,73.38970998)
\curveto(900.6813254,73.18970544)(900.65132543,72.94470569)(900.65131592,72.65470998)
\curveto(900.67132541,72.634706)(900.6813254,72.59970603)(900.68131592,72.54970998)
\curveto(900.67132541,72.49970613)(900.67132541,72.45970617)(900.68131592,72.42970998)
\curveto(900.70132538,72.34970628)(900.72132536,72.27470636)(900.74131592,72.20470998)
\curveto(900.75132533,72.14470649)(900.77132531,72.07970655)(900.80131592,72.00970998)
\curveto(900.92132516,71.73970689)(901.09132499,71.51970711)(901.31131592,71.34970998)
\curveto(901.52132456,71.18970744)(901.76632431,71.05470758)(902.04631592,70.94470998)
\curveto(902.15632392,70.89470774)(902.2763238,70.85470778)(902.40631592,70.82470998)
\curveto(902.52632355,70.80470783)(902.65132343,70.77970785)(902.78131592,70.74970998)
\curveto(902.83132325,70.7297079)(902.88632319,70.71970791)(902.94631592,70.71970998)
\curveto(902.99632308,70.71970791)(903.04632303,70.71470792)(903.09631592,70.70470998)
\curveto(903.18632289,70.69470794)(903.2813228,70.68470795)(903.38131592,70.67470998)
\curveto(903.47132261,70.66470797)(903.56632251,70.65470798)(903.66631592,70.64470998)
\curveto(903.74632233,70.64470799)(903.83132225,70.63970799)(903.92131592,70.62970998)
\lineto(904.16131592,70.62970998)
\lineto(904.34131592,70.62970998)
\curveto(904.37132171,70.61970801)(904.40632167,70.61470802)(904.44631592,70.61470998)
\lineto(904.58131592,70.61470998)
\lineto(905.03131592,70.61470998)
\curveto(905.11132097,70.61470802)(905.19632088,70.60970802)(905.28631592,70.59970998)
\curveto(905.36632071,70.59970803)(905.44132064,70.60970802)(905.51131592,70.62970998)
\lineto(905.78131592,70.62970998)
\curveto(905.80132028,70.629708)(905.83132025,70.62470801)(905.87131592,70.61470998)
\curveto(905.90132018,70.61470802)(905.92632015,70.61970801)(905.94631592,70.62970998)
\curveto(906.04632003,70.63970799)(906.14631993,70.64470799)(906.24631592,70.64470998)
\curveto(906.33631974,70.65470798)(906.43631964,70.66470797)(906.54631592,70.67470998)
\curveto(906.66631941,70.70470793)(906.79131929,70.71970791)(906.92131592,70.71970998)
\curveto(907.04131904,70.7297079)(907.15631892,70.75470788)(907.26631592,70.79470998)
\curveto(907.56631851,70.87470776)(907.83131825,70.95970767)(908.06131592,71.04970998)
\curveto(908.29131779,71.14970748)(908.50631757,71.29470734)(908.70631592,71.48470998)
\curveto(908.90631717,71.69470694)(909.05631702,71.95970667)(909.15631592,72.27970998)
\curveto(909.1763169,72.31970631)(909.18631689,72.35470628)(909.18631592,72.38470998)
\curveto(909.1763169,72.42470621)(909.1813169,72.46970616)(909.20131592,72.51970998)
\curveto(909.21131687,72.55970607)(909.22131686,72.629706)(909.23131592,72.72970998)
\curveto(909.24131684,72.83970579)(909.23631684,72.92470571)(909.21631592,72.98470998)
\curveto(909.19631688,73.05470558)(909.18631689,73.12470551)(909.18631592,73.19470998)
\curveto(909.1763169,73.26470537)(909.16131692,73.3297053)(909.14131592,73.38970998)
\curveto(909.081317,73.58970504)(908.99631708,73.76970486)(908.88631592,73.92970998)
\curveto(908.86631721,73.95970467)(908.84631723,73.98470465)(908.82631592,74.00470998)
\lineto(908.76631592,74.06470998)
\curveto(908.74631733,74.10470453)(908.70631737,74.15470448)(908.64631592,74.21470998)
\curveto(908.50631757,74.31470432)(908.3763177,74.39970423)(908.25631592,74.46970998)
\curveto(908.13631794,74.53970409)(907.99131809,74.60970402)(907.82131592,74.67970998)
\curveto(907.75131833,74.70970392)(907.6813184,74.7297039)(907.61131592,74.73970998)
\curveto(907.54131854,74.75970387)(907.46631861,74.77970385)(907.38631592,74.79970998)
}
}
{
\newrgbcolor{curcolor}{0 0 0}
\pscustom[linestyle=none,fillstyle=solid,fillcolor=curcolor]
{
\newpath
\moveto(908.51131592,78.63431936)
\lineto(908.51131592,79.26431936)
\lineto(908.51131592,79.45931936)
\curveto(908.51131757,79.52931683)(908.52131756,79.58931677)(908.54131592,79.63931936)
\curveto(908.5813175,79.70931665)(908.62131746,79.7593166)(908.66131592,79.78931936)
\curveto(908.71131737,79.82931653)(908.7763173,79.84931651)(908.85631592,79.84931936)
\curveto(908.93631714,79.8593165)(909.02131706,79.86431649)(909.11131592,79.86431936)
\lineto(909.83131592,79.86431936)
\curveto(910.31131577,79.86431649)(910.72131536,79.80431655)(911.06131592,79.68431936)
\curveto(911.40131468,79.56431679)(911.6763144,79.36931699)(911.88631592,79.09931936)
\curveto(911.93631414,79.02931733)(911.9813141,78.9593174)(912.02131592,78.88931936)
\curveto(912.07131401,78.82931753)(912.11631396,78.7543176)(912.15631592,78.66431936)
\curveto(912.16631391,78.64431771)(912.1763139,78.61431774)(912.18631592,78.57431936)
\curveto(912.20631387,78.53431782)(912.21131387,78.48931787)(912.20131592,78.43931936)
\curveto(912.17131391,78.34931801)(912.09631398,78.29431806)(911.97631592,78.27431936)
\curveto(911.86631421,78.2543181)(911.77131431,78.26931809)(911.69131592,78.31931936)
\curveto(911.62131446,78.34931801)(911.55631452,78.39431796)(911.49631592,78.45431936)
\curveto(911.44631463,78.52431783)(911.39631468,78.58931777)(911.34631592,78.64931936)
\curveto(911.29631478,78.71931764)(911.22131486,78.77931758)(911.12131592,78.82931936)
\curveto(911.03131505,78.88931747)(910.94131514,78.93931742)(910.85131592,78.97931936)
\curveto(910.82131526,78.99931736)(910.76131532,79.02431733)(910.67131592,79.05431936)
\curveto(910.59131549,79.08431727)(910.52131556,79.08931727)(910.46131592,79.06931936)
\curveto(910.32131576,79.03931732)(910.23131585,78.97931738)(910.19131592,78.88931936)
\curveto(910.16131592,78.80931755)(910.14631593,78.71931764)(910.14631592,78.61931936)
\curveto(910.14631593,78.51931784)(910.12131596,78.43431792)(910.07131592,78.36431936)
\curveto(910.00131608,78.27431808)(909.86131622,78.22931813)(909.65131592,78.22931936)
\lineto(909.09631592,78.22931936)
\lineto(908.87131592,78.22931936)
\curveto(908.79131729,78.23931812)(908.72631735,78.2593181)(908.67631592,78.28931936)
\curveto(908.59631748,78.34931801)(908.55131753,78.41931794)(908.54131592,78.49931936)
\curveto(908.53131755,78.51931784)(908.52631755,78.53931782)(908.52631592,78.55931936)
\curveto(908.52631755,78.58931777)(908.52131756,78.61431774)(908.51131592,78.63431936)
}
}
{
\newrgbcolor{curcolor}{0 0 0}
\pscustom[linestyle=none,fillstyle=solid,fillcolor=curcolor]
{
}
}
{
\newrgbcolor{curcolor}{0 0 0}
\pscustom[linestyle=none,fillstyle=solid,fillcolor=curcolor]
{
\newpath
\moveto(899.54131592,89.26463186)
\curveto(899.53132655,89.95462722)(899.65132643,90.55462662)(899.90131592,91.06463186)
\curveto(900.15132593,91.58462559)(900.48632559,91.9796252)(900.90631592,92.24963186)
\curveto(900.98632509,92.29962488)(901.076325,92.34462483)(901.17631592,92.38463186)
\curveto(901.26632481,92.42462475)(901.36132472,92.46962471)(901.46131592,92.51963186)
\curveto(901.56132452,92.55962462)(901.66132442,92.58962459)(901.76131592,92.60963186)
\curveto(901.86132422,92.62962455)(901.96632411,92.64962453)(902.07631592,92.66963186)
\curveto(902.12632395,92.68962449)(902.17132391,92.69462448)(902.21131592,92.68463186)
\curveto(902.25132383,92.6746245)(902.29632378,92.6796245)(902.34631592,92.69963186)
\curveto(902.39632368,92.70962447)(902.4813236,92.71462446)(902.60131592,92.71463186)
\curveto(902.71132337,92.71462446)(902.79632328,92.70962447)(902.85631592,92.69963186)
\curveto(902.91632316,92.6796245)(902.9763231,92.66962451)(903.03631592,92.66963186)
\curveto(903.09632298,92.6796245)(903.15632292,92.6746245)(903.21631592,92.65463186)
\curveto(903.35632272,92.61462456)(903.49132259,92.5796246)(903.62131592,92.54963186)
\curveto(903.75132233,92.51962466)(903.8763222,92.4796247)(903.99631592,92.42963186)
\curveto(904.13632194,92.36962481)(904.26132182,92.29962488)(904.37131592,92.21963186)
\curveto(904.4813216,92.14962503)(904.59132149,92.0746251)(904.70131592,91.99463186)
\lineto(904.76131592,91.93463186)
\curveto(904.7813213,91.92462525)(904.80132128,91.90962527)(904.82131592,91.88963186)
\curveto(904.9813211,91.76962541)(905.12632095,91.63462554)(905.25631592,91.48463186)
\curveto(905.38632069,91.33462584)(905.51132057,91.174626)(905.63131592,91.00463186)
\curveto(905.85132023,90.69462648)(906.05632002,90.39962678)(906.24631592,90.11963186)
\curveto(906.38631969,89.88962729)(906.52131956,89.65962752)(906.65131592,89.42963186)
\curveto(906.7813193,89.20962797)(906.91631916,88.98962819)(907.05631592,88.76963186)
\curveto(907.22631885,88.51962866)(907.40631867,88.2796289)(907.59631592,88.04963186)
\curveto(907.78631829,87.82962935)(908.01131807,87.63962954)(908.27131592,87.47963186)
\curveto(908.33131775,87.43962974)(908.39131769,87.40462977)(908.45131592,87.37463186)
\curveto(908.50131758,87.34462983)(908.56631751,87.31462986)(908.64631592,87.28463186)
\curveto(908.71631736,87.26462991)(908.7763173,87.25962992)(908.82631592,87.26963186)
\curveto(908.89631718,87.28962989)(908.95131713,87.32462985)(908.99131592,87.37463186)
\curveto(909.02131706,87.42462975)(909.04131704,87.48462969)(909.05131592,87.55463186)
\lineto(909.05131592,87.79463186)
\lineto(909.05131592,88.54463186)
\lineto(909.05131592,91.34963186)
\lineto(909.05131592,92.00963186)
\curveto(909.05131703,92.09962508)(909.05631702,92.18462499)(909.06631592,92.26463186)
\curveto(909.06631701,92.34462483)(909.08631699,92.40962477)(909.12631592,92.45963186)
\curveto(909.16631691,92.50962467)(909.24131684,92.54962463)(909.35131592,92.57963186)
\curveto(909.45131663,92.61962456)(909.55131653,92.62962455)(909.65131592,92.60963186)
\lineto(909.78631592,92.60963186)
\curveto(909.85631622,92.58962459)(909.91631616,92.56962461)(909.96631592,92.54963186)
\curveto(910.01631606,92.52962465)(910.05631602,92.49462468)(910.08631592,92.44463186)
\curveto(910.12631595,92.39462478)(910.14631593,92.32462485)(910.14631592,92.23463186)
\lineto(910.14631592,91.96463186)
\lineto(910.14631592,91.06463186)
\lineto(910.14631592,87.55463186)
\lineto(910.14631592,86.48963186)
\curveto(910.14631593,86.40963077)(910.15131593,86.31963086)(910.16131592,86.21963186)
\curveto(910.16131592,86.11963106)(910.15131593,86.03463114)(910.13131592,85.96463186)
\curveto(910.06131602,85.75463142)(909.8813162,85.68963149)(909.59131592,85.76963186)
\curveto(909.55131653,85.7796314)(909.51631656,85.7796314)(909.48631592,85.76963186)
\curveto(909.44631663,85.76963141)(909.40131668,85.7796314)(909.35131592,85.79963186)
\curveto(909.27131681,85.81963136)(909.18631689,85.83963134)(909.09631592,85.85963186)
\curveto(909.00631707,85.8796313)(908.92131716,85.90463127)(908.84131592,85.93463186)
\curveto(908.35131773,86.09463108)(907.93631814,86.29463088)(907.59631592,86.53463186)
\curveto(907.34631873,86.71463046)(907.12131896,86.91963026)(906.92131592,87.14963186)
\curveto(906.71131937,87.3796298)(906.51631956,87.61962956)(906.33631592,87.86963186)
\curveto(906.15631992,88.12962905)(905.98632009,88.39462878)(905.82631592,88.66463186)
\curveto(905.65632042,88.94462823)(905.4813206,89.21462796)(905.30131592,89.47463186)
\curveto(905.22132086,89.58462759)(905.14632093,89.68962749)(905.07631592,89.78963186)
\curveto(905.00632107,89.89962728)(904.93132115,90.00962717)(904.85131592,90.11963186)
\curveto(904.82132126,90.15962702)(904.79132129,90.19462698)(904.76131592,90.22463186)
\curveto(904.72132136,90.26462691)(904.69132139,90.30462687)(904.67131592,90.34463186)
\curveto(904.56132152,90.48462669)(904.43632164,90.60962657)(904.29631592,90.71963186)
\curveto(904.26632181,90.73962644)(904.24132184,90.76462641)(904.22131592,90.79463186)
\curveto(904.19132189,90.82462635)(904.16132192,90.84962633)(904.13131592,90.86963186)
\curveto(904.03132205,90.94962623)(903.93132215,91.01462616)(903.83131592,91.06463186)
\curveto(903.73132235,91.12462605)(903.62132246,91.179626)(903.50131592,91.22963186)
\curveto(903.43132265,91.25962592)(903.35632272,91.2796259)(903.27631592,91.28963186)
\lineto(903.03631592,91.34963186)
\lineto(902.94631592,91.34963186)
\curveto(902.91632316,91.35962582)(902.88632319,91.36462581)(902.85631592,91.36463186)
\curveto(902.78632329,91.38462579)(902.69132339,91.38962579)(902.57131592,91.37963186)
\curveto(902.44132364,91.3796258)(902.34132374,91.36962581)(902.27131592,91.34963186)
\curveto(902.19132389,91.32962585)(902.11632396,91.30962587)(902.04631592,91.28963186)
\curveto(901.96632411,91.2796259)(901.88632419,91.25962592)(901.80631592,91.22963186)
\curveto(901.56632451,91.11962606)(901.36632471,90.96962621)(901.20631592,90.77963186)
\curveto(901.03632504,90.59962658)(900.89632518,90.3796268)(900.78631592,90.11963186)
\curveto(900.76632531,90.04962713)(900.75132533,89.9796272)(900.74131592,89.90963186)
\curveto(900.72132536,89.83962734)(900.70132538,89.76462741)(900.68131592,89.68463186)
\curveto(900.66132542,89.60462757)(900.65132543,89.49462768)(900.65131592,89.35463186)
\curveto(900.65132543,89.22462795)(900.66132542,89.11962806)(900.68131592,89.03963186)
\curveto(900.69132539,88.9796282)(900.69632538,88.92462825)(900.69631592,88.87463186)
\curveto(900.69632538,88.82462835)(900.70632537,88.7746284)(900.72631592,88.72463186)
\curveto(900.76632531,88.62462855)(900.80632527,88.52962865)(900.84631592,88.43963186)
\curveto(900.88632519,88.35962882)(900.93132515,88.2796289)(900.98131592,88.19963186)
\curveto(901.00132508,88.16962901)(901.02632505,88.13962904)(901.05631592,88.10963186)
\curveto(901.08632499,88.08962909)(901.11132497,88.06462911)(901.13131592,88.03463186)
\lineto(901.20631592,87.95963186)
\curveto(901.22632485,87.92962925)(901.24632483,87.90462927)(901.26631592,87.88463186)
\lineto(901.47631592,87.73463186)
\curveto(901.53632454,87.69462948)(901.60132448,87.64962953)(901.67131592,87.59963186)
\curveto(901.76132432,87.53962964)(901.86632421,87.48962969)(901.98631592,87.44963186)
\curveto(902.09632398,87.41962976)(902.20632387,87.38462979)(902.31631592,87.34463186)
\curveto(902.42632365,87.30462987)(902.57132351,87.2796299)(902.75131592,87.26963186)
\curveto(902.92132316,87.25962992)(903.04632303,87.22962995)(903.12631592,87.17963186)
\curveto(903.20632287,87.12963005)(903.25132283,87.05463012)(903.26131592,86.95463186)
\curveto(903.27132281,86.85463032)(903.2763228,86.74463043)(903.27631592,86.62463186)
\curveto(903.2763228,86.58463059)(903.2813228,86.54463063)(903.29131592,86.50463186)
\curveto(903.29132279,86.46463071)(903.28632279,86.42963075)(903.27631592,86.39963186)
\curveto(903.25632282,86.34963083)(903.24632283,86.29963088)(903.24631592,86.24963186)
\curveto(903.24632283,86.20963097)(903.23632284,86.16963101)(903.21631592,86.12963186)
\curveto(903.15632292,86.03963114)(903.02132306,85.99463118)(902.81131592,85.99463186)
\lineto(902.69131592,85.99463186)
\curveto(902.63132345,86.00463117)(902.57132351,86.00963117)(902.51131592,86.00963186)
\curveto(902.44132364,86.01963116)(902.3763237,86.02963115)(902.31631592,86.03963186)
\curveto(902.20632387,86.05963112)(902.10632397,86.0796311)(902.01631592,86.09963186)
\curveto(901.91632416,86.11963106)(901.82132426,86.14963103)(901.73131592,86.18963186)
\curveto(901.66132442,86.20963097)(901.60132448,86.22963095)(901.55131592,86.24963186)
\lineto(901.37131592,86.30963186)
\curveto(901.11132497,86.42963075)(900.86632521,86.58463059)(900.63631592,86.77463186)
\curveto(900.40632567,86.9746302)(900.22132586,87.18962999)(900.08131592,87.41963186)
\curveto(900.00132608,87.52962965)(899.93632614,87.64462953)(899.88631592,87.76463186)
\lineto(899.73631592,88.15463186)
\curveto(899.68632639,88.26462891)(899.65632642,88.3796288)(899.64631592,88.49963186)
\curveto(899.62632645,88.61962856)(899.60132648,88.74462843)(899.57131592,88.87463186)
\curveto(899.57132651,88.94462823)(899.57132651,89.00962817)(899.57131592,89.06963186)
\curveto(899.56132652,89.12962805)(899.55132653,89.19462798)(899.54131592,89.26463186)
}
}
{
\newrgbcolor{curcolor}{0 0 0}
\pscustom[linestyle=none,fillstyle=solid,fillcolor=curcolor]
{
\newpath
\moveto(905.06131592,101.36424123)
\lineto(905.31631592,101.36424123)
\curveto(905.39632068,101.37423353)(905.47132061,101.36923353)(905.54131592,101.34924123)
\lineto(905.78131592,101.34924123)
\lineto(905.94631592,101.34924123)
\curveto(906.04632003,101.32923357)(906.15131993,101.31923358)(906.26131592,101.31924123)
\curveto(906.36131972,101.31923358)(906.46131962,101.30923359)(906.56131592,101.28924123)
\lineto(906.71131592,101.28924123)
\curveto(906.85131923,101.25923364)(906.99131909,101.23923366)(907.13131592,101.22924123)
\curveto(907.26131882,101.21923368)(907.39131869,101.19423371)(907.52131592,101.15424123)
\curveto(907.60131848,101.13423377)(907.68631839,101.11423379)(907.77631592,101.09424123)
\lineto(908.01631592,101.03424123)
\lineto(908.31631592,100.91424123)
\curveto(908.40631767,100.88423402)(908.49631758,100.84923405)(908.58631592,100.80924123)
\curveto(908.80631727,100.70923419)(909.02131706,100.57423433)(909.23131592,100.40424123)
\curveto(909.44131664,100.24423466)(909.61131647,100.06923483)(909.74131592,99.87924123)
\curveto(909.7813163,99.82923507)(909.82131626,99.76923513)(909.86131592,99.69924123)
\curveto(909.89131619,99.63923526)(909.92631615,99.57923532)(909.96631592,99.51924123)
\curveto(910.01631606,99.43923546)(910.05631602,99.34423556)(910.08631592,99.23424123)
\curveto(910.11631596,99.12423578)(910.14631593,99.01923588)(910.17631592,98.91924123)
\curveto(910.21631586,98.80923609)(910.24131584,98.6992362)(910.25131592,98.58924123)
\curveto(910.26131582,98.47923642)(910.2763158,98.36423654)(910.29631592,98.24424123)
\curveto(910.30631577,98.2042367)(910.30631577,98.15923674)(910.29631592,98.10924123)
\curveto(910.29631578,98.06923683)(910.30131578,98.02923687)(910.31131592,97.98924123)
\curveto(910.32131576,97.94923695)(910.32631575,97.89423701)(910.32631592,97.82424123)
\curveto(910.32631575,97.75423715)(910.32131576,97.7042372)(910.31131592,97.67424123)
\curveto(910.29131579,97.62423728)(910.28631579,97.57923732)(910.29631592,97.53924123)
\curveto(910.30631577,97.4992374)(910.30631577,97.46423744)(910.29631592,97.43424123)
\lineto(910.29631592,97.34424123)
\curveto(910.2763158,97.28423762)(910.26131582,97.21923768)(910.25131592,97.14924123)
\curveto(910.25131583,97.08923781)(910.24631583,97.02423788)(910.23631592,96.95424123)
\curveto(910.18631589,96.78423812)(910.13631594,96.62423828)(910.08631592,96.47424123)
\curveto(910.03631604,96.32423858)(909.97131611,96.17923872)(909.89131592,96.03924123)
\curveto(909.85131623,95.98923891)(909.82131626,95.93423897)(909.80131592,95.87424123)
\curveto(909.77131631,95.82423908)(909.73631634,95.77423913)(909.69631592,95.72424123)
\curveto(909.51631656,95.48423942)(909.29631678,95.28423962)(909.03631592,95.12424123)
\curveto(908.7763173,94.96423994)(908.49131759,94.82424008)(908.18131592,94.70424123)
\curveto(908.04131804,94.64424026)(907.90131818,94.5992403)(907.76131592,94.56924123)
\curveto(907.61131847,94.53924036)(907.45631862,94.5042404)(907.29631592,94.46424123)
\curveto(907.18631889,94.44424046)(907.076319,94.42924047)(906.96631592,94.41924123)
\curveto(906.85631922,94.40924049)(906.74631933,94.39424051)(906.63631592,94.37424123)
\curveto(906.59631948,94.36424054)(906.55631952,94.35924054)(906.51631592,94.35924123)
\curveto(906.4763196,94.36924053)(906.43631964,94.36924053)(906.39631592,94.35924123)
\curveto(906.34631973,94.34924055)(906.29631978,94.34424056)(906.24631592,94.34424123)
\lineto(906.08131592,94.34424123)
\curveto(906.03132005,94.32424058)(905.9813201,94.31924058)(905.93131592,94.32924123)
\curveto(905.87132021,94.33924056)(905.81632026,94.33924056)(905.76631592,94.32924123)
\curveto(905.72632035,94.31924058)(905.6813204,94.31924058)(905.63131592,94.32924123)
\curveto(905.5813205,94.33924056)(905.53132055,94.33424057)(905.48131592,94.31424123)
\curveto(905.41132067,94.29424061)(905.33632074,94.28924061)(905.25631592,94.29924123)
\curveto(905.16632091,94.30924059)(905.081321,94.31424059)(905.00131592,94.31424123)
\curveto(904.91132117,94.31424059)(904.81132127,94.30924059)(904.70131592,94.29924123)
\curveto(904.5813215,94.28924061)(904.4813216,94.29424061)(904.40131592,94.31424123)
\lineto(904.11631592,94.31424123)
\lineto(903.48631592,94.35924123)
\curveto(903.38632269,94.36924053)(903.29132279,94.37924052)(903.20131592,94.38924123)
\lineto(902.90131592,94.41924123)
\curveto(902.85132323,94.43924046)(902.80132328,94.44424046)(902.75131592,94.43424123)
\curveto(902.69132339,94.43424047)(902.63632344,94.44424046)(902.58631592,94.46424123)
\curveto(902.41632366,94.51424039)(902.25132383,94.55424035)(902.09131592,94.58424123)
\curveto(901.92132416,94.61424029)(901.76132432,94.66424024)(901.61131592,94.73424123)
\curveto(901.15132493,94.92423998)(900.7763253,95.14423976)(900.48631592,95.39424123)
\curveto(900.19632588,95.65423925)(899.95132613,96.01423889)(899.75131592,96.47424123)
\curveto(899.70132638,96.6042383)(899.66632641,96.73423817)(899.64631592,96.86424123)
\curveto(899.62632645,97.0042379)(899.60132648,97.14423776)(899.57131592,97.28424123)
\curveto(899.56132652,97.35423755)(899.55632652,97.41923748)(899.55631592,97.47924123)
\curveto(899.55632652,97.53923736)(899.55132653,97.6042373)(899.54131592,97.67424123)
\curveto(899.52132656,98.5042364)(899.67132641,99.17423573)(899.99131592,99.68424123)
\curveto(900.30132578,100.19423471)(900.74132534,100.57423433)(901.31131592,100.82424123)
\curveto(901.43132465,100.87423403)(901.55632452,100.91923398)(901.68631592,100.95924123)
\curveto(901.81632426,100.9992339)(901.95132413,101.04423386)(902.09131592,101.09424123)
\curveto(902.17132391,101.11423379)(902.25632382,101.12923377)(902.34631592,101.13924123)
\lineto(902.58631592,101.19924123)
\curveto(902.69632338,101.22923367)(902.80632327,101.24423366)(902.91631592,101.24424123)
\curveto(903.02632305,101.25423365)(903.13632294,101.26923363)(903.24631592,101.28924123)
\curveto(903.29632278,101.30923359)(903.34132274,101.31423359)(903.38131592,101.30424123)
\curveto(903.42132266,101.3042336)(903.46132262,101.30923359)(903.50131592,101.31924123)
\curveto(903.55132253,101.32923357)(903.60632247,101.32923357)(903.66631592,101.31924123)
\curveto(903.71632236,101.31923358)(903.76632231,101.32423358)(903.81631592,101.33424123)
\lineto(903.95131592,101.33424123)
\curveto(904.01132207,101.35423355)(904.081322,101.35423355)(904.16131592,101.33424123)
\curveto(904.23132185,101.32423358)(904.29632178,101.32923357)(904.35631592,101.34924123)
\curveto(904.38632169,101.35923354)(904.42632165,101.36423354)(904.47631592,101.36424123)
\lineto(904.59631592,101.36424123)
\lineto(905.06131592,101.36424123)
\moveto(907.38631592,99.81924123)
\curveto(907.06631901,99.91923498)(906.70131938,99.97923492)(906.29131592,99.99924123)
\curveto(905.8813202,100.01923488)(905.47132061,100.02923487)(905.06131592,100.02924123)
\curveto(904.63132145,100.02923487)(904.21132187,100.01923488)(903.80131592,99.99924123)
\curveto(903.39132269,99.97923492)(903.00632307,99.93423497)(902.64631592,99.86424123)
\curveto(902.28632379,99.79423511)(901.96632411,99.68423522)(901.68631592,99.53424123)
\curveto(901.39632468,99.39423551)(901.16132492,99.1992357)(900.98131592,98.94924123)
\curveto(900.87132521,98.78923611)(900.79132529,98.60923629)(900.74131592,98.40924123)
\curveto(900.6813254,98.20923669)(900.65132543,97.96423694)(900.65131592,97.67424123)
\curveto(900.67132541,97.65423725)(900.6813254,97.61923728)(900.68131592,97.56924123)
\curveto(900.67132541,97.51923738)(900.67132541,97.47923742)(900.68131592,97.44924123)
\curveto(900.70132538,97.36923753)(900.72132536,97.29423761)(900.74131592,97.22424123)
\curveto(900.75132533,97.16423774)(900.77132531,97.0992378)(900.80131592,97.02924123)
\curveto(900.92132516,96.75923814)(901.09132499,96.53923836)(901.31131592,96.36924123)
\curveto(901.52132456,96.20923869)(901.76632431,96.07423883)(902.04631592,95.96424123)
\curveto(902.15632392,95.91423899)(902.2763238,95.87423903)(902.40631592,95.84424123)
\curveto(902.52632355,95.82423908)(902.65132343,95.7992391)(902.78131592,95.76924123)
\curveto(902.83132325,95.74923915)(902.88632319,95.73923916)(902.94631592,95.73924123)
\curveto(902.99632308,95.73923916)(903.04632303,95.73423917)(903.09631592,95.72424123)
\curveto(903.18632289,95.71423919)(903.2813228,95.7042392)(903.38131592,95.69424123)
\curveto(903.47132261,95.68423922)(903.56632251,95.67423923)(903.66631592,95.66424123)
\curveto(903.74632233,95.66423924)(903.83132225,95.65923924)(903.92131592,95.64924123)
\lineto(904.16131592,95.64924123)
\lineto(904.34131592,95.64924123)
\curveto(904.37132171,95.63923926)(904.40632167,95.63423927)(904.44631592,95.63424123)
\lineto(904.58131592,95.63424123)
\lineto(905.03131592,95.63424123)
\curveto(905.11132097,95.63423927)(905.19632088,95.62923927)(905.28631592,95.61924123)
\curveto(905.36632071,95.61923928)(905.44132064,95.62923927)(905.51131592,95.64924123)
\lineto(905.78131592,95.64924123)
\curveto(905.80132028,95.64923925)(905.83132025,95.64423926)(905.87131592,95.63424123)
\curveto(905.90132018,95.63423927)(905.92632015,95.63923926)(905.94631592,95.64924123)
\curveto(906.04632003,95.65923924)(906.14631993,95.66423924)(906.24631592,95.66424123)
\curveto(906.33631974,95.67423923)(906.43631964,95.68423922)(906.54631592,95.69424123)
\curveto(906.66631941,95.72423918)(906.79131929,95.73923916)(906.92131592,95.73924123)
\curveto(907.04131904,95.74923915)(907.15631892,95.77423913)(907.26631592,95.81424123)
\curveto(907.56631851,95.89423901)(907.83131825,95.97923892)(908.06131592,96.06924123)
\curveto(908.29131779,96.16923873)(908.50631757,96.31423859)(908.70631592,96.50424123)
\curveto(908.90631717,96.71423819)(909.05631702,96.97923792)(909.15631592,97.29924123)
\curveto(909.1763169,97.33923756)(909.18631689,97.37423753)(909.18631592,97.40424123)
\curveto(909.1763169,97.44423746)(909.1813169,97.48923741)(909.20131592,97.53924123)
\curveto(909.21131687,97.57923732)(909.22131686,97.64923725)(909.23131592,97.74924123)
\curveto(909.24131684,97.85923704)(909.23631684,97.94423696)(909.21631592,98.00424123)
\curveto(909.19631688,98.07423683)(909.18631689,98.14423676)(909.18631592,98.21424123)
\curveto(909.1763169,98.28423662)(909.16131692,98.34923655)(909.14131592,98.40924123)
\curveto(909.081317,98.60923629)(908.99631708,98.78923611)(908.88631592,98.94924123)
\curveto(908.86631721,98.97923592)(908.84631723,99.0042359)(908.82631592,99.02424123)
\lineto(908.76631592,99.08424123)
\curveto(908.74631733,99.12423578)(908.70631737,99.17423573)(908.64631592,99.23424123)
\curveto(908.50631757,99.33423557)(908.3763177,99.41923548)(908.25631592,99.48924123)
\curveto(908.13631794,99.55923534)(907.99131809,99.62923527)(907.82131592,99.69924123)
\curveto(907.75131833,99.72923517)(907.6813184,99.74923515)(907.61131592,99.75924123)
\curveto(907.54131854,99.77923512)(907.46631861,99.7992351)(907.38631592,99.81924123)
}
}
{
\newrgbcolor{curcolor}{0 0 0}
\pscustom[linestyle=none,fillstyle=solid,fillcolor=curcolor]
{
\newpath
\moveto(899.54131592,106.77385061)
\curveto(899.54132654,106.87384575)(899.55132653,106.96884566)(899.57131592,107.05885061)
\curveto(899.5813265,107.14884548)(899.61132647,107.21384541)(899.66131592,107.25385061)
\curveto(899.74132634,107.31384531)(899.84632623,107.34384528)(899.97631592,107.34385061)
\lineto(900.36631592,107.34385061)
\lineto(901.86631592,107.34385061)
\lineto(908.25631592,107.34385061)
\lineto(909.42631592,107.34385061)
\lineto(909.74131592,107.34385061)
\curveto(909.84131624,107.35384527)(909.92131616,107.33884529)(909.98131592,107.29885061)
\curveto(910.06131602,107.24884538)(910.11131597,107.17384545)(910.13131592,107.07385061)
\curveto(910.14131594,106.98384564)(910.14631593,106.87384575)(910.14631592,106.74385061)
\lineto(910.14631592,106.51885061)
\curveto(910.12631595,106.43884619)(910.11131597,106.36884626)(910.10131592,106.30885061)
\curveto(910.081316,106.24884638)(910.04131604,106.19884643)(909.98131592,106.15885061)
\curveto(909.92131616,106.11884651)(909.84631623,106.09884653)(909.75631592,106.09885061)
\lineto(909.45631592,106.09885061)
\lineto(908.36131592,106.09885061)
\lineto(903.02131592,106.09885061)
\curveto(902.93132315,106.07884655)(902.85632322,106.06384656)(902.79631592,106.05385061)
\curveto(902.72632335,106.05384657)(902.66632341,106.0238466)(902.61631592,105.96385061)
\curveto(902.56632351,105.89384673)(902.54132354,105.80384682)(902.54131592,105.69385061)
\curveto(902.53132355,105.59384703)(902.52632355,105.48384714)(902.52631592,105.36385061)
\lineto(902.52631592,104.22385061)
\lineto(902.52631592,103.72885061)
\curveto(902.51632356,103.56884906)(902.45632362,103.45884917)(902.34631592,103.39885061)
\curveto(902.31632376,103.37884925)(902.28632379,103.36884926)(902.25631592,103.36885061)
\curveto(902.21632386,103.36884926)(902.17132391,103.36384926)(902.12131592,103.35385061)
\curveto(902.00132408,103.33384929)(901.89132419,103.33884929)(901.79131592,103.36885061)
\curveto(901.69132439,103.40884922)(901.62132446,103.46384916)(901.58131592,103.53385061)
\curveto(901.53132455,103.61384901)(901.50632457,103.73384889)(901.50631592,103.89385061)
\curveto(901.50632457,104.05384857)(901.49132459,104.18884844)(901.46131592,104.29885061)
\curveto(901.45132463,104.34884828)(901.44632463,104.40384822)(901.44631592,104.46385061)
\curveto(901.43632464,104.5238481)(901.42132466,104.58384804)(901.40131592,104.64385061)
\curveto(901.35132473,104.79384783)(901.30132478,104.93884769)(901.25131592,105.07885061)
\curveto(901.19132489,105.21884741)(901.12132496,105.35384727)(901.04131592,105.48385061)
\curveto(900.95132513,105.623847)(900.84632523,105.74384688)(900.72631592,105.84385061)
\curveto(900.60632547,105.94384668)(900.4763256,106.03884659)(900.33631592,106.12885061)
\curveto(900.23632584,106.18884644)(900.12632595,106.23384639)(900.00631592,106.26385061)
\curveto(899.88632619,106.30384632)(899.7813263,106.35384627)(899.69131592,106.41385061)
\curveto(899.63132645,106.46384616)(899.59132649,106.53384609)(899.57131592,106.62385061)
\curveto(899.56132652,106.64384598)(899.55632652,106.66884596)(899.55631592,106.69885061)
\curveto(899.55632652,106.7288459)(899.55132653,106.75384587)(899.54131592,106.77385061)
}
}
{
\newrgbcolor{curcolor}{0 0 0}
\pscustom[linestyle=none,fillstyle=solid,fillcolor=curcolor]
{
\newpath
\moveto(899.54131592,115.12345998)
\curveto(899.54132654,115.22345513)(899.55132653,115.31845503)(899.57131592,115.40845998)
\curveto(899.5813265,115.49845485)(899.61132647,115.56345479)(899.66131592,115.60345998)
\curveto(899.74132634,115.66345469)(899.84632623,115.69345466)(899.97631592,115.69345998)
\lineto(900.36631592,115.69345998)
\lineto(901.86631592,115.69345998)
\lineto(908.25631592,115.69345998)
\lineto(909.42631592,115.69345998)
\lineto(909.74131592,115.69345998)
\curveto(909.84131624,115.70345465)(909.92131616,115.68845466)(909.98131592,115.64845998)
\curveto(910.06131602,115.59845475)(910.11131597,115.52345483)(910.13131592,115.42345998)
\curveto(910.14131594,115.33345502)(910.14631593,115.22345513)(910.14631592,115.09345998)
\lineto(910.14631592,114.86845998)
\curveto(910.12631595,114.78845556)(910.11131597,114.71845563)(910.10131592,114.65845998)
\curveto(910.081316,114.59845575)(910.04131604,114.5484558)(909.98131592,114.50845998)
\curveto(909.92131616,114.46845588)(909.84631623,114.4484559)(909.75631592,114.44845998)
\lineto(909.45631592,114.44845998)
\lineto(908.36131592,114.44845998)
\lineto(903.02131592,114.44845998)
\curveto(902.93132315,114.42845592)(902.85632322,114.41345594)(902.79631592,114.40345998)
\curveto(902.72632335,114.40345595)(902.66632341,114.37345598)(902.61631592,114.31345998)
\curveto(902.56632351,114.24345611)(902.54132354,114.1534562)(902.54131592,114.04345998)
\curveto(902.53132355,113.94345641)(902.52632355,113.83345652)(902.52631592,113.71345998)
\lineto(902.52631592,112.57345998)
\lineto(902.52631592,112.07845998)
\curveto(902.51632356,111.91845843)(902.45632362,111.80845854)(902.34631592,111.74845998)
\curveto(902.31632376,111.72845862)(902.28632379,111.71845863)(902.25631592,111.71845998)
\curveto(902.21632386,111.71845863)(902.17132391,111.71345864)(902.12131592,111.70345998)
\curveto(902.00132408,111.68345867)(901.89132419,111.68845866)(901.79131592,111.71845998)
\curveto(901.69132439,111.75845859)(901.62132446,111.81345854)(901.58131592,111.88345998)
\curveto(901.53132455,111.96345839)(901.50632457,112.08345827)(901.50631592,112.24345998)
\curveto(901.50632457,112.40345795)(901.49132459,112.53845781)(901.46131592,112.64845998)
\curveto(901.45132463,112.69845765)(901.44632463,112.7534576)(901.44631592,112.81345998)
\curveto(901.43632464,112.87345748)(901.42132466,112.93345742)(901.40131592,112.99345998)
\curveto(901.35132473,113.14345721)(901.30132478,113.28845706)(901.25131592,113.42845998)
\curveto(901.19132489,113.56845678)(901.12132496,113.70345665)(901.04131592,113.83345998)
\curveto(900.95132513,113.97345638)(900.84632523,114.09345626)(900.72631592,114.19345998)
\curveto(900.60632547,114.29345606)(900.4763256,114.38845596)(900.33631592,114.47845998)
\curveto(900.23632584,114.53845581)(900.12632595,114.58345577)(900.00631592,114.61345998)
\curveto(899.88632619,114.6534557)(899.7813263,114.70345565)(899.69131592,114.76345998)
\curveto(899.63132645,114.81345554)(899.59132649,114.88345547)(899.57131592,114.97345998)
\curveto(899.56132652,114.99345536)(899.55632652,115.01845533)(899.55631592,115.04845998)
\curveto(899.55632652,115.07845527)(899.55132653,115.10345525)(899.54131592,115.12345998)
}
}
{
\newrgbcolor{curcolor}{0 0 0}
\pscustom[linestyle=none,fillstyle=solid,fillcolor=curcolor]
{
\newpath
\moveto(920.37763184,42.29681936)
\curveto(920.37764253,42.36681368)(920.37764253,42.4468136)(920.37763184,42.53681936)
\curveto(920.36764254,42.62681342)(920.36764254,42.71181333)(920.37763184,42.79181936)
\curveto(920.37764253,42.88181316)(920.38764252,42.96181308)(920.40763184,43.03181936)
\curveto(920.42764248,43.11181293)(920.45764245,43.16681288)(920.49763184,43.19681936)
\curveto(920.54764236,43.22681282)(920.62264229,43.2468128)(920.72263184,43.25681936)
\curveto(920.8126421,43.27681277)(920.91764199,43.28681276)(921.03763184,43.28681936)
\curveto(921.14764176,43.29681275)(921.26264165,43.29681275)(921.38263184,43.28681936)
\lineto(921.68263184,43.28681936)
\lineto(924.69763184,43.28681936)
\lineto(927.59263184,43.28681936)
\curveto(927.92263499,43.28681276)(928.24763466,43.28181276)(928.56763184,43.27181936)
\curveto(928.87763403,43.27181277)(929.15763375,43.23181281)(929.40763184,43.15181936)
\curveto(929.75763315,43.03181301)(930.05263286,42.87681317)(930.29263184,42.68681936)
\curveto(930.52263239,42.49681355)(930.72263219,42.25681379)(930.89263184,41.96681936)
\curveto(930.94263197,41.90681414)(930.97763193,41.8418142)(930.99763184,41.77181936)
\curveto(931.01763189,41.71181433)(931.04263187,41.6418144)(931.07263184,41.56181936)
\curveto(931.12263179,41.4418146)(931.15763175,41.31181473)(931.17763184,41.17181936)
\curveto(931.2076317,41.041815)(931.23763167,40.90681514)(931.26763184,40.76681936)
\curveto(931.28763162,40.71681533)(931.29263162,40.66681538)(931.28263184,40.61681936)
\curveto(931.27263164,40.56681548)(931.27263164,40.51181553)(931.28263184,40.45181936)
\curveto(931.29263162,40.43181561)(931.29263162,40.40681564)(931.28263184,40.37681936)
\curveto(931.28263163,40.3468157)(931.28763162,40.32181572)(931.29763184,40.30181936)
\curveto(931.3076316,40.26181578)(931.3126316,40.20681584)(931.31263184,40.13681936)
\curveto(931.3126316,40.06681598)(931.3076316,40.01181603)(931.29763184,39.97181936)
\curveto(931.28763162,39.92181612)(931.28763162,39.86681618)(931.29763184,39.80681936)
\curveto(931.3076316,39.7468163)(931.30263161,39.69181635)(931.28263184,39.64181936)
\curveto(931.25263166,39.51181653)(931.23263168,39.38681666)(931.22263184,39.26681936)
\curveto(931.2126317,39.1468169)(931.18763172,39.03181701)(931.14763184,38.92181936)
\curveto(931.02763188,38.55181749)(930.85763205,38.23181781)(930.63763184,37.96181936)
\curveto(930.41763249,37.69181835)(930.13763277,37.48181856)(929.79763184,37.33181936)
\curveto(929.67763323,37.28181876)(929.55263336,37.23681881)(929.42263184,37.19681936)
\curveto(929.29263362,37.16681888)(929.15763375,37.13181891)(929.01763184,37.09181936)
\curveto(928.96763394,37.08181896)(928.92763398,37.07681897)(928.89763184,37.07681936)
\curveto(928.85763405,37.07681897)(928.8126341,37.07181897)(928.76263184,37.06181936)
\curveto(928.73263418,37.05181899)(928.69763421,37.046819)(928.65763184,37.04681936)
\curveto(928.6076343,37.046819)(928.56763434,37.041819)(928.53763184,37.03181936)
\lineto(928.37263184,37.03181936)
\curveto(928.29263462,37.01181903)(928.19263472,37.00681904)(928.07263184,37.01681936)
\curveto(927.94263497,37.02681902)(927.85263506,37.041819)(927.80263184,37.06181936)
\curveto(927.7126352,37.08181896)(927.64763526,37.13681891)(927.60763184,37.22681936)
\curveto(927.58763532,37.25681879)(927.58263533,37.28681876)(927.59263184,37.31681936)
\curveto(927.59263532,37.3468187)(927.58763532,37.38681866)(927.57763184,37.43681936)
\curveto(927.56763534,37.47681857)(927.56263535,37.51681853)(927.56263184,37.55681936)
\lineto(927.56263184,37.70681936)
\curveto(927.56263535,37.82681822)(927.56763534,37.9468181)(927.57763184,38.06681936)
\curveto(927.57763533,38.19681785)(927.6126353,38.28681776)(927.68263184,38.33681936)
\curveto(927.74263517,38.37681767)(927.80263511,38.39681765)(927.86263184,38.39681936)
\curveto(927.92263499,38.39681765)(927.99263492,38.40681764)(928.07263184,38.42681936)
\curveto(928.10263481,38.43681761)(928.13763477,38.43681761)(928.17763184,38.42681936)
\curveto(928.2076347,38.42681762)(928.23263468,38.43181761)(928.25263184,38.44181936)
\lineto(928.46263184,38.44181936)
\curveto(928.5126344,38.46181758)(928.56263435,38.46681758)(928.61263184,38.45681936)
\curveto(928.65263426,38.45681759)(928.69763421,38.46681758)(928.74763184,38.48681936)
\curveto(928.87763403,38.51681753)(929.00263391,38.5468175)(929.12263184,38.57681936)
\curveto(929.23263368,38.60681744)(929.33763357,38.65181739)(929.43763184,38.71181936)
\curveto(929.72763318,38.88181716)(929.93263298,39.15181689)(930.05263184,39.52181936)
\curveto(930.07263284,39.57181647)(930.08763282,39.62181642)(930.09763184,39.67181936)
\curveto(930.09763281,39.73181631)(930.1076328,39.78681626)(930.12763184,39.83681936)
\lineto(930.12763184,39.91181936)
\curveto(930.13763277,39.98181606)(930.14763276,40.07681597)(930.15763184,40.19681936)
\curveto(930.15763275,40.32681572)(930.14763276,40.42681562)(930.12763184,40.49681936)
\curveto(930.1076328,40.56681548)(930.09263282,40.63681541)(930.08263184,40.70681936)
\curveto(930.06263285,40.78681526)(930.04263287,40.85681519)(930.02263184,40.91681936)
\curveto(929.86263305,41.29681475)(929.58763332,41.57181447)(929.19763184,41.74181936)
\curveto(929.06763384,41.79181425)(928.912634,41.82681422)(928.73263184,41.84681936)
\curveto(928.55263436,41.87681417)(928.36763454,41.89181415)(928.17763184,41.89181936)
\curveto(927.97763493,41.90181414)(927.77763513,41.90181414)(927.57763184,41.89181936)
\lineto(927.00763184,41.89181936)
\lineto(922.76263184,41.89181936)
\lineto(921.21763184,41.89181936)
\curveto(921.1076418,41.89181415)(920.98764192,41.88681416)(920.85763184,41.87681936)
\curveto(920.72764218,41.86681418)(920.62264229,41.88681416)(920.54263184,41.93681936)
\curveto(920.47264244,41.99681405)(920.42264249,42.07681397)(920.39263184,42.17681936)
\curveto(920.39264252,42.19681385)(920.39264252,42.21681383)(920.39263184,42.23681936)
\curveto(920.39264252,42.25681379)(920.38764252,42.27681377)(920.37763184,42.29681936)
}
}
{
\newrgbcolor{curcolor}{0 0 0}
\pscustom[linestyle=none,fillstyle=solid,fillcolor=curcolor]
{
\newpath
\moveto(923.33263184,45.83049123)
\lineto(923.33263184,46.26549123)
\curveto(923.33263958,46.41548927)(923.37263954,46.52048916)(923.45263184,46.58049123)
\curveto(923.53263938,46.63048905)(923.63263928,46.65548903)(923.75263184,46.65549123)
\curveto(923.87263904,46.66548902)(923.99263892,46.67048901)(924.11263184,46.67049123)
\lineto(925.53763184,46.67049123)
\lineto(927.80263184,46.67049123)
\lineto(928.49263184,46.67049123)
\curveto(928.72263419,46.67048901)(928.92263399,46.69548899)(929.09263184,46.74549123)
\curveto(929.54263337,46.90548878)(929.85763305,47.20548848)(930.03763184,47.64549123)
\curveto(930.12763278,47.86548782)(930.16263275,48.13048755)(930.14263184,48.44049123)
\curveto(930.1126328,48.75048693)(930.05763285,49.00048668)(929.97763184,49.19049123)
\curveto(929.83763307,49.52048616)(929.66263325,49.7804859)(929.45263184,49.97049123)
\curveto(929.23263368,50.17048551)(928.94763396,50.32548536)(928.59763184,50.43549123)
\curveto(928.51763439,50.46548522)(928.43763447,50.4854852)(928.35763184,50.49549123)
\curveto(928.27763463,50.50548518)(928.19263472,50.52048516)(928.10263184,50.54049123)
\curveto(928.05263486,50.55048513)(928.0076349,50.55048513)(927.96763184,50.54049123)
\curveto(927.92763498,50.54048514)(927.88263503,50.55048513)(927.83263184,50.57049123)
\lineto(927.51763184,50.57049123)
\curveto(927.43763547,50.59048509)(927.34763556,50.59548509)(927.24763184,50.58549123)
\curveto(927.13763577,50.57548511)(927.03763587,50.57048511)(926.94763184,50.57049123)
\lineto(925.77763184,50.57049123)
\lineto(924.18763184,50.57049123)
\curveto(924.06763884,50.57048511)(923.94263897,50.56548512)(923.81263184,50.55549123)
\curveto(923.67263924,50.55548513)(923.56263935,50.5804851)(923.48263184,50.63049123)
\curveto(923.43263948,50.67048501)(923.40263951,50.71548497)(923.39263184,50.76549123)
\curveto(923.37263954,50.82548486)(923.35263956,50.89548479)(923.33263184,50.97549123)
\lineto(923.33263184,51.20049123)
\curveto(923.33263958,51.32048436)(923.33763957,51.42548426)(923.34763184,51.51549123)
\curveto(923.35763955,51.61548407)(923.40263951,51.69048399)(923.48263184,51.74049123)
\curveto(923.53263938,51.79048389)(923.6076393,51.81548387)(923.70763184,51.81549123)
\lineto(923.99263184,51.81549123)
\lineto(925.01263184,51.81549123)
\lineto(929.04763184,51.81549123)
\lineto(930.39763184,51.81549123)
\curveto(930.51763239,51.81548387)(930.63263228,51.81048387)(930.74263184,51.80049123)
\curveto(930.84263207,51.80048388)(930.91763199,51.76548392)(930.96763184,51.69549123)
\curveto(930.99763191,51.65548403)(931.02263189,51.59548409)(931.04263184,51.51549123)
\curveto(931.05263186,51.43548425)(931.06263185,51.34548434)(931.07263184,51.24549123)
\curveto(931.07263184,51.15548453)(931.06763184,51.06548462)(931.05763184,50.97549123)
\curveto(931.04763186,50.89548479)(931.02763188,50.83548485)(930.99763184,50.79549123)
\curveto(930.95763195,50.74548494)(930.89263202,50.70048498)(930.80263184,50.66049123)
\curveto(930.76263215,50.65048503)(930.7076322,50.64048504)(930.63763184,50.63049123)
\curveto(930.56763234,50.63048505)(930.50263241,50.62548506)(930.44263184,50.61549123)
\curveto(930.37263254,50.60548508)(930.31763259,50.5854851)(930.27763184,50.55549123)
\curveto(930.23763267,50.52548516)(930.22263269,50.4804852)(930.23263184,50.42049123)
\curveto(930.25263266,50.34048534)(930.3126326,50.26048542)(930.41263184,50.18049123)
\curveto(930.50263241,50.10048558)(930.57263234,50.02548566)(930.62263184,49.95549123)
\curveto(930.78263213,49.73548595)(930.92263199,49.4854862)(931.04263184,49.20549123)
\curveto(931.09263182,49.09548659)(931.12263179,48.9804867)(931.13263184,48.86049123)
\curveto(931.15263176,48.75048693)(931.17763173,48.63548705)(931.20763184,48.51549123)
\curveto(931.21763169,48.46548722)(931.21763169,48.41048727)(931.20763184,48.35049123)
\curveto(931.19763171,48.30048738)(931.20263171,48.25048743)(931.22263184,48.20049123)
\curveto(931.24263167,48.10048758)(931.24263167,48.01048767)(931.22263184,47.93049123)
\lineto(931.22263184,47.78049123)
\curveto(931.20263171,47.73048795)(931.19263172,47.67048801)(931.19263184,47.60049123)
\curveto(931.19263172,47.54048814)(931.18763172,47.4854882)(931.17763184,47.43549123)
\curveto(931.15763175,47.39548829)(931.14763176,47.35548833)(931.14763184,47.31549123)
\curveto(931.15763175,47.2854884)(931.15263176,47.24548844)(931.13263184,47.19549123)
\lineto(931.07263184,46.95549123)
\curveto(931.05263186,46.8854888)(931.02263189,46.81048887)(930.98263184,46.73049123)
\curveto(930.87263204,46.47048921)(930.72763218,46.25048943)(930.54763184,46.07049123)
\curveto(930.35763255,45.90048978)(930.13263278,45.76048992)(929.87263184,45.65049123)
\curveto(929.78263313,45.61049007)(929.69263322,45.5804901)(929.60263184,45.56049123)
\lineto(929.30263184,45.50049123)
\curveto(929.24263367,45.4804902)(929.18763372,45.47049021)(929.13763184,45.47049123)
\curveto(929.07763383,45.4804902)(929.0126339,45.47549021)(928.94263184,45.45549123)
\curveto(928.92263399,45.44549024)(928.89763401,45.44049024)(928.86763184,45.44049123)
\curveto(928.82763408,45.44049024)(928.79263412,45.43549025)(928.76263184,45.42549123)
\lineto(928.61263184,45.42549123)
\curveto(928.57263434,45.41549027)(928.52763438,45.41049027)(928.47763184,45.41049123)
\curveto(928.41763449,45.42049026)(928.36263455,45.42549026)(928.31263184,45.42549123)
\lineto(927.71263184,45.42549123)
\lineto(924.95263184,45.42549123)
\lineto(923.99263184,45.42549123)
\lineto(923.72263184,45.42549123)
\curveto(923.63263928,45.42549026)(923.55763935,45.44549024)(923.49763184,45.48549123)
\curveto(923.42763948,45.52549016)(923.37763953,45.60049008)(923.34763184,45.71049123)
\curveto(923.33763957,45.73048995)(923.33763957,45.75048993)(923.34763184,45.77049123)
\curveto(923.34763956,45.79048989)(923.34263957,45.81048987)(923.33263184,45.83049123)
}
}
{
\newrgbcolor{curcolor}{0 0 0}
\pscustom[linestyle=none,fillstyle=solid,fillcolor=curcolor]
{
\newpath
\moveto(920.37763184,54.28510061)
\curveto(920.37764253,54.41509899)(920.37764253,54.55009886)(920.37763184,54.69010061)
\curveto(920.37764253,54.84009857)(920.4126425,54.95009846)(920.48263184,55.02010061)
\curveto(920.55264236,55.07009834)(920.64764226,55.09509831)(920.76763184,55.09510061)
\curveto(920.87764203,55.1050983)(920.99264192,55.1100983)(921.11263184,55.11010061)
\lineto(922.44763184,55.11010061)
\lineto(928.52263184,55.11010061)
\lineto(930.20263184,55.11010061)
\lineto(930.59263184,55.11010061)
\curveto(930.73263218,55.1100983)(930.84263207,55.08509832)(930.92263184,55.03510061)
\curveto(930.97263194,55.0050984)(931.00263191,54.96009845)(931.01263184,54.90010061)
\curveto(931.02263189,54.85009856)(931.03763187,54.78509862)(931.05763184,54.70510061)
\lineto(931.05763184,54.49510061)
\lineto(931.05763184,54.18010061)
\curveto(931.04763186,54.08009933)(931.0126319,54.0050994)(930.95263184,53.95510061)
\curveto(930.87263204,53.9050995)(930.77263214,53.87509953)(930.65263184,53.86510061)
\lineto(930.27763184,53.86510061)
\lineto(928.89763184,53.86510061)
\lineto(922.65763184,53.86510061)
\lineto(921.18763184,53.86510061)
\curveto(921.07764183,53.86509954)(920.96264195,53.86009955)(920.84263184,53.85010061)
\curveto(920.7126422,53.85009956)(920.6126423,53.87509953)(920.54263184,53.92510061)
\curveto(920.48264243,53.96509944)(920.43264248,54.04009937)(920.39263184,54.15010061)
\curveto(920.38264253,54.17009924)(920.38264253,54.19009922)(920.39263184,54.21010061)
\curveto(920.39264252,54.24009917)(920.38764252,54.26509914)(920.37763184,54.28510061)
}
}
{
\newrgbcolor{curcolor}{0 0 0}
\pscustom[linestyle=none,fillstyle=solid,fillcolor=curcolor]
{
}
}
{
\newrgbcolor{curcolor}{0 0 0}
\pscustom[linestyle=none,fillstyle=solid,fillcolor=curcolor]
{
\newpath
\moveto(920.45263184,64.24510061)
\curveto(920.44264247,64.93509597)(920.56264235,65.53509537)(920.81263184,66.04510061)
\curveto(921.06264185,66.56509434)(921.39764151,66.96009395)(921.81763184,67.23010061)
\curveto(921.89764101,67.28009363)(921.98764092,67.32509358)(922.08763184,67.36510061)
\curveto(922.17764073,67.4050935)(922.27264064,67.45009346)(922.37263184,67.50010061)
\curveto(922.47264044,67.54009337)(922.57264034,67.57009334)(922.67263184,67.59010061)
\curveto(922.77264014,67.6100933)(922.87764003,67.63009328)(922.98763184,67.65010061)
\curveto(923.03763987,67.67009324)(923.08263983,67.67509323)(923.12263184,67.66510061)
\curveto(923.16263975,67.65509325)(923.2076397,67.66009325)(923.25763184,67.68010061)
\curveto(923.3076396,67.69009322)(923.39263952,67.69509321)(923.51263184,67.69510061)
\curveto(923.62263929,67.69509321)(923.7076392,67.69009322)(923.76763184,67.68010061)
\curveto(923.82763908,67.66009325)(923.88763902,67.65009326)(923.94763184,67.65010061)
\curveto(924.0076389,67.66009325)(924.06763884,67.65509325)(924.12763184,67.63510061)
\curveto(924.26763864,67.59509331)(924.40263851,67.56009335)(924.53263184,67.53010061)
\curveto(924.66263825,67.50009341)(924.78763812,67.46009345)(924.90763184,67.41010061)
\curveto(925.04763786,67.35009356)(925.17263774,67.28009363)(925.28263184,67.20010061)
\curveto(925.39263752,67.13009378)(925.50263741,67.05509385)(925.61263184,66.97510061)
\lineto(925.67263184,66.91510061)
\curveto(925.69263722,66.905094)(925.7126372,66.89009402)(925.73263184,66.87010061)
\curveto(925.89263702,66.75009416)(926.03763687,66.61509429)(926.16763184,66.46510061)
\curveto(926.29763661,66.31509459)(926.42263649,66.15509475)(926.54263184,65.98510061)
\curveto(926.76263615,65.67509523)(926.96763594,65.38009553)(927.15763184,65.10010061)
\curveto(927.29763561,64.87009604)(927.43263548,64.64009627)(927.56263184,64.41010061)
\curveto(927.69263522,64.19009672)(927.82763508,63.97009694)(927.96763184,63.75010061)
\curveto(928.13763477,63.50009741)(928.31763459,63.26009765)(928.50763184,63.03010061)
\curveto(928.69763421,62.8100981)(928.92263399,62.62009829)(929.18263184,62.46010061)
\curveto(929.24263367,62.42009849)(929.30263361,62.38509852)(929.36263184,62.35510061)
\curveto(929.4126335,62.32509858)(929.47763343,62.29509861)(929.55763184,62.26510061)
\curveto(929.62763328,62.24509866)(929.68763322,62.24009867)(929.73763184,62.25010061)
\curveto(929.8076331,62.27009864)(929.86263305,62.3050986)(929.90263184,62.35510061)
\curveto(929.93263298,62.4050985)(929.95263296,62.46509844)(929.96263184,62.53510061)
\lineto(929.96263184,62.77510061)
\lineto(929.96263184,63.52510061)
\lineto(929.96263184,66.33010061)
\lineto(929.96263184,66.99010061)
\curveto(929.96263295,67.08009383)(929.96763294,67.16509374)(929.97763184,67.24510061)
\curveto(929.97763293,67.32509358)(929.99763291,67.39009352)(930.03763184,67.44010061)
\curveto(930.07763283,67.49009342)(930.15263276,67.53009338)(930.26263184,67.56010061)
\curveto(930.36263255,67.60009331)(930.46263245,67.6100933)(930.56263184,67.59010061)
\lineto(930.69763184,67.59010061)
\curveto(930.76763214,67.57009334)(930.82763208,67.55009336)(930.87763184,67.53010061)
\curveto(930.92763198,67.5100934)(930.96763194,67.47509343)(930.99763184,67.42510061)
\curveto(931.03763187,67.37509353)(931.05763185,67.3050936)(931.05763184,67.21510061)
\lineto(931.05763184,66.94510061)
\lineto(931.05763184,66.04510061)
\lineto(931.05763184,62.53510061)
\lineto(931.05763184,61.47010061)
\curveto(931.05763185,61.39009952)(931.06263185,61.30009961)(931.07263184,61.20010061)
\curveto(931.07263184,61.10009981)(931.06263185,61.01509989)(931.04263184,60.94510061)
\curveto(930.97263194,60.73510017)(930.79263212,60.67010024)(930.50263184,60.75010061)
\curveto(930.46263245,60.76010015)(930.42763248,60.76010015)(930.39763184,60.75010061)
\curveto(930.35763255,60.75010016)(930.3126326,60.76010015)(930.26263184,60.78010061)
\curveto(930.18263273,60.80010011)(930.09763281,60.82010009)(930.00763184,60.84010061)
\curveto(929.91763299,60.86010005)(929.83263308,60.88510002)(929.75263184,60.91510061)
\curveto(929.26263365,61.07509983)(928.84763406,61.27509963)(928.50763184,61.51510061)
\curveto(928.25763465,61.69509921)(928.03263488,61.90009901)(927.83263184,62.13010061)
\curveto(927.62263529,62.36009855)(927.42763548,62.60009831)(927.24763184,62.85010061)
\curveto(927.06763584,63.1100978)(926.89763601,63.37509753)(926.73763184,63.64510061)
\curveto(926.56763634,63.92509698)(926.39263652,64.19509671)(926.21263184,64.45510061)
\curveto(926.13263678,64.56509634)(926.05763685,64.67009624)(925.98763184,64.77010061)
\curveto(925.91763699,64.88009603)(925.84263707,64.99009592)(925.76263184,65.10010061)
\curveto(925.73263718,65.14009577)(925.70263721,65.17509573)(925.67263184,65.20510061)
\curveto(925.63263728,65.24509566)(925.60263731,65.28509562)(925.58263184,65.32510061)
\curveto(925.47263744,65.46509544)(925.34763756,65.59009532)(925.20763184,65.70010061)
\curveto(925.17763773,65.72009519)(925.15263776,65.74509516)(925.13263184,65.77510061)
\curveto(925.10263781,65.8050951)(925.07263784,65.83009508)(925.04263184,65.85010061)
\curveto(924.94263797,65.93009498)(924.84263807,65.99509491)(924.74263184,66.04510061)
\curveto(924.64263827,66.1050948)(924.53263838,66.16009475)(924.41263184,66.21010061)
\curveto(924.34263857,66.24009467)(924.26763864,66.26009465)(924.18763184,66.27010061)
\lineto(923.94763184,66.33010061)
\lineto(923.85763184,66.33010061)
\curveto(923.82763908,66.34009457)(923.79763911,66.34509456)(923.76763184,66.34510061)
\curveto(923.69763921,66.36509454)(923.60263931,66.37009454)(923.48263184,66.36010061)
\curveto(923.35263956,66.36009455)(923.25263966,66.35009456)(923.18263184,66.33010061)
\curveto(923.10263981,66.3100946)(923.02763988,66.29009462)(922.95763184,66.27010061)
\curveto(922.87764003,66.26009465)(922.79764011,66.24009467)(922.71763184,66.21010061)
\curveto(922.47764043,66.10009481)(922.27764063,65.95009496)(922.11763184,65.76010061)
\curveto(921.94764096,65.58009533)(921.8076411,65.36009555)(921.69763184,65.10010061)
\curveto(921.67764123,65.03009588)(921.66264125,64.96009595)(921.65263184,64.89010061)
\curveto(921.63264128,64.82009609)(921.6126413,64.74509616)(921.59263184,64.66510061)
\curveto(921.57264134,64.58509632)(921.56264135,64.47509643)(921.56263184,64.33510061)
\curveto(921.56264135,64.2050967)(921.57264134,64.10009681)(921.59263184,64.02010061)
\curveto(921.60264131,63.96009695)(921.6076413,63.905097)(921.60763184,63.85510061)
\curveto(921.6076413,63.8050971)(921.61764129,63.75509715)(921.63763184,63.70510061)
\curveto(921.67764123,63.6050973)(921.71764119,63.5100974)(921.75763184,63.42010061)
\curveto(921.79764111,63.34009757)(921.84264107,63.26009765)(921.89263184,63.18010061)
\curveto(921.912641,63.15009776)(921.93764097,63.12009779)(921.96763184,63.09010061)
\curveto(921.99764091,63.07009784)(922.02264089,63.04509786)(922.04263184,63.01510061)
\lineto(922.11763184,62.94010061)
\curveto(922.13764077,62.910098)(922.15764075,62.88509802)(922.17763184,62.86510061)
\lineto(922.38763184,62.71510061)
\curveto(922.44764046,62.67509823)(922.5126404,62.63009828)(922.58263184,62.58010061)
\curveto(922.67264024,62.52009839)(922.77764013,62.47009844)(922.89763184,62.43010061)
\curveto(923.0076399,62.40009851)(923.11763979,62.36509854)(923.22763184,62.32510061)
\curveto(923.33763957,62.28509862)(923.48263943,62.26009865)(923.66263184,62.25010061)
\curveto(923.83263908,62.24009867)(923.95763895,62.2100987)(924.03763184,62.16010061)
\curveto(924.11763879,62.1100988)(924.16263875,62.03509887)(924.17263184,61.93510061)
\curveto(924.18263873,61.83509907)(924.18763872,61.72509918)(924.18763184,61.60510061)
\curveto(924.18763872,61.56509934)(924.19263872,61.52509938)(924.20263184,61.48510061)
\curveto(924.20263871,61.44509946)(924.19763871,61.4100995)(924.18763184,61.38010061)
\curveto(924.16763874,61.33009958)(924.15763875,61.28009963)(924.15763184,61.23010061)
\curveto(924.15763875,61.19009972)(924.14763876,61.15009976)(924.12763184,61.11010061)
\curveto(924.06763884,61.02009989)(923.93263898,60.97509993)(923.72263184,60.97510061)
\lineto(923.60263184,60.97510061)
\curveto(923.54263937,60.98509992)(923.48263943,60.99009992)(923.42263184,60.99010061)
\curveto(923.35263956,61.00009991)(923.28763962,61.0100999)(923.22763184,61.02010061)
\curveto(923.11763979,61.04009987)(923.01763989,61.06009985)(922.92763184,61.08010061)
\curveto(922.82764008,61.10009981)(922.73264018,61.13009978)(922.64263184,61.17010061)
\curveto(922.57264034,61.19009972)(922.5126404,61.2100997)(922.46263184,61.23010061)
\lineto(922.28263184,61.29010061)
\curveto(922.02264089,61.4100995)(921.77764113,61.56509934)(921.54763184,61.75510061)
\curveto(921.31764159,61.95509895)(921.13264178,62.17009874)(920.99263184,62.40010061)
\curveto(920.912642,62.5100984)(920.84764206,62.62509828)(920.79763184,62.74510061)
\lineto(920.64763184,63.13510061)
\curveto(920.59764231,63.24509766)(920.56764234,63.36009755)(920.55763184,63.48010061)
\curveto(920.53764237,63.60009731)(920.5126424,63.72509718)(920.48263184,63.85510061)
\curveto(920.48264243,63.92509698)(920.48264243,63.99009692)(920.48263184,64.05010061)
\curveto(920.47264244,64.1100968)(920.46264245,64.17509673)(920.45263184,64.24510061)
}
}
{
\newrgbcolor{curcolor}{0 0 0}
\pscustom[linestyle=none,fillstyle=solid,fillcolor=curcolor]
{
\newpath
\moveto(927.56263184,76.22470998)
\curveto(927.6126353,76.29470234)(927.68263523,76.3347023)(927.77263184,76.34470998)
\curveto(927.86263505,76.36470227)(927.96763494,76.37470226)(928.08763184,76.37470998)
\curveto(928.13763477,76.37470226)(928.18763472,76.36970226)(928.23763184,76.35970998)
\curveto(928.28763462,76.35970227)(928.33263458,76.34970228)(928.37263184,76.32970998)
\curveto(928.46263445,76.29970233)(928.52263439,76.23970239)(928.55263184,76.14970998)
\curveto(928.57263434,76.06970256)(928.58263433,75.97470266)(928.58263184,75.86470998)
\lineto(928.58263184,75.54970998)
\curveto(928.57263434,75.43970319)(928.58263433,75.3347033)(928.61263184,75.23470998)
\curveto(928.64263427,75.09470354)(928.72263419,75.00470363)(928.85263184,74.96470998)
\curveto(928.92263399,74.94470369)(929.0076339,74.9347037)(929.10763184,74.93470998)
\lineto(929.37763184,74.93470998)
\lineto(930.32263184,74.93470998)
\lineto(930.65263184,74.93470998)
\curveto(930.76263215,74.9347037)(930.84763206,74.91470372)(930.90763184,74.87470998)
\curveto(930.96763194,74.8347038)(931.0076319,74.78470385)(931.02763184,74.72470998)
\curveto(931.03763187,74.67470396)(931.05263186,74.60970402)(931.07263184,74.52970998)
\lineto(931.07263184,74.33470998)
\curveto(931.07263184,74.21470442)(931.06763184,74.10970452)(931.05763184,74.01970998)
\curveto(931.03763187,73.9297047)(930.98763192,73.85970477)(930.90763184,73.80970998)
\curveto(930.85763205,73.77970485)(930.78763212,73.76470487)(930.69763184,73.76470998)
\lineto(930.39763184,73.76470998)
\lineto(929.36263184,73.76470998)
\curveto(929.20263371,73.76470487)(929.05763385,73.75470488)(928.92763184,73.73470998)
\curveto(928.78763412,73.72470491)(928.69263422,73.66970496)(928.64263184,73.56970998)
\curveto(928.62263429,73.51970511)(928.6076343,73.44970518)(928.59763184,73.35970998)
\curveto(928.58763432,73.27970535)(928.58263433,73.18970544)(928.58263184,73.08970998)
\lineto(928.58263184,72.80470998)
\lineto(928.58263184,72.56470998)
\lineto(928.58263184,70.29970998)
\curveto(928.58263433,70.20970842)(928.58763432,70.10470853)(928.59763184,69.98470998)
\lineto(928.59763184,69.65470998)
\curveto(928.59763431,69.54470909)(928.58763432,69.44470919)(928.56763184,69.35470998)
\curveto(928.54763436,69.26470937)(928.5126344,69.20470943)(928.46263184,69.17470998)
\curveto(928.39263452,69.12470951)(928.29763461,69.09970953)(928.17763184,69.09970998)
\lineto(927.83263184,69.09970998)
\lineto(927.56263184,69.09970998)
\curveto(927.39263552,69.13970949)(927.25263566,69.19470944)(927.14263184,69.26470998)
\curveto(927.03263588,69.3347093)(926.91763599,69.41470922)(926.79763184,69.50470998)
\lineto(926.25763184,69.86470998)
\curveto(925.62763728,70.30470833)(925.0076379,70.73970789)(924.39763184,71.16970998)
\lineto(922.53763184,72.48970998)
\curveto(922.3076406,72.64970598)(922.08764082,72.80470583)(921.87763184,72.95470998)
\curveto(921.65764125,73.10470553)(921.43264148,73.25970537)(921.20263184,73.41970998)
\curveto(921.13264178,73.46970516)(921.06764184,73.51970511)(921.00763184,73.56970998)
\curveto(920.93764197,73.61970501)(920.86264205,73.66970496)(920.78263184,73.71970998)
\lineto(920.69263184,73.77970998)
\curveto(920.65264226,73.80970482)(920.62264229,73.83970479)(920.60263184,73.86970998)
\curveto(920.57264234,73.90970472)(920.55264236,73.94970468)(920.54263184,73.98970998)
\curveto(920.52264239,74.0297046)(920.50264241,74.07470456)(920.48263184,74.12470998)
\curveto(920.48264243,74.14470449)(920.48764242,74.16470447)(920.49763184,74.18470998)
\curveto(920.49764241,74.21470442)(920.48764242,74.23970439)(920.46763184,74.25970998)
\curveto(920.46764244,74.38970424)(920.47264244,74.50970412)(920.48263184,74.61970998)
\curveto(920.49264242,74.7297039)(920.53764237,74.80970382)(920.61763184,74.85970998)
\curveto(920.66764224,74.89970373)(920.73764217,74.91970371)(920.82763184,74.91970998)
\curveto(920.91764199,74.9297037)(921.0126419,74.9347037)(921.11263184,74.93470998)
\lineto(926.57263184,74.93470998)
\curveto(926.64263627,74.9347037)(926.71763619,74.9297037)(926.79763184,74.91970998)
\curveto(926.86763604,74.91970371)(926.93763597,74.92470371)(927.00763184,74.93470998)
\lineto(927.11263184,74.93470998)
\curveto(927.16263575,74.95470368)(927.21763569,74.96970366)(927.27763184,74.97970998)
\curveto(927.32763558,74.98970364)(927.36763554,75.01470362)(927.39763184,75.05470998)
\curveto(927.44763546,75.12470351)(927.47763543,75.20970342)(927.48763184,75.30970998)
\lineto(927.48763184,75.63970998)
\curveto(927.48763542,75.74970288)(927.49263542,75.85470278)(927.50263184,75.95470998)
\curveto(927.50263541,76.06470257)(927.52263539,76.15470248)(927.56263184,76.22470998)
\moveto(927.36763184,73.65970998)
\curveto(927.25763565,73.73970489)(927.08763582,73.77470486)(926.85763184,73.76470998)
\lineto(926.24263184,73.76470998)
\lineto(923.76763184,73.76470998)
\lineto(923.45263184,73.76470998)
\curveto(923.33263958,73.77470486)(923.23263968,73.76970486)(923.15263184,73.74970998)
\lineto(923.00263184,73.74970998)
\curveto(922.91264,73.74970488)(922.82764008,73.7347049)(922.74763184,73.70470998)
\curveto(922.72764018,73.69470494)(922.71764019,73.68470495)(922.71763184,73.67470998)
\lineto(922.67263184,73.62970998)
\curveto(922.66264025,73.60970502)(922.65764025,73.57970505)(922.65763184,73.53970998)
\curveto(922.67764023,73.51970511)(922.69264022,73.49970513)(922.70263184,73.47970998)
\curveto(922.70264021,73.46970516)(922.7076402,73.45470518)(922.71763184,73.43470998)
\curveto(922.76764014,73.37470526)(922.83764007,73.31470532)(922.92763184,73.25470998)
\curveto(923.01763989,73.19470544)(923.09763981,73.13970549)(923.16763184,73.08970998)
\curveto(923.3076396,72.98970564)(923.45263946,72.89470574)(923.60263184,72.80470998)
\curveto(923.74263917,72.71470592)(923.88263903,72.61970601)(924.02263184,72.51970998)
\lineto(924.80263184,71.97970998)
\curveto(925.06263785,71.80970682)(925.32263759,71.634707)(925.58263184,71.45470998)
\curveto(925.69263722,71.37470726)(925.79763711,71.29970733)(925.89763184,71.22970998)
\lineto(926.19763184,71.01970998)
\curveto(926.27763663,70.96970766)(926.35263656,70.91970771)(926.42263184,70.86970998)
\curveto(926.49263642,70.8297078)(926.56763634,70.78470785)(926.64763184,70.73470998)
\curveto(926.7076362,70.68470795)(926.77263614,70.634708)(926.84263184,70.58470998)
\curveto(926.90263601,70.54470809)(926.97263594,70.50470813)(927.05263184,70.46470998)
\curveto(927.1126358,70.42470821)(927.18263573,70.39970823)(927.26263184,70.38970998)
\curveto(927.33263558,70.37970825)(927.38763552,70.41470822)(927.42763184,70.49470998)
\curveto(927.47763543,70.56470807)(927.50263541,70.67470796)(927.50263184,70.82470998)
\curveto(927.49263542,70.98470765)(927.48763542,71.11970751)(927.48763184,71.22970998)
\lineto(927.48763184,72.90970998)
\lineto(927.48763184,73.34470998)
\curveto(927.48763542,73.49470514)(927.44763546,73.59970503)(927.36763184,73.65970998)
}
}
{
\newrgbcolor{curcolor}{0 0 0}
\pscustom[linestyle=none,fillstyle=solid,fillcolor=curcolor]
{
\newpath
\moveto(929.42263184,78.63431936)
\lineto(929.42263184,79.26431936)
\lineto(929.42263184,79.45931936)
\curveto(929.42263349,79.52931683)(929.43263348,79.58931677)(929.45263184,79.63931936)
\curveto(929.49263342,79.70931665)(929.53263338,79.7593166)(929.57263184,79.78931936)
\curveto(929.62263329,79.82931653)(929.68763322,79.84931651)(929.76763184,79.84931936)
\curveto(929.84763306,79.8593165)(929.93263298,79.86431649)(930.02263184,79.86431936)
\lineto(930.74263184,79.86431936)
\curveto(931.22263169,79.86431649)(931.63263128,79.80431655)(931.97263184,79.68431936)
\curveto(932.3126306,79.56431679)(932.58763032,79.36931699)(932.79763184,79.09931936)
\curveto(932.84763006,79.02931733)(932.89263002,78.9593174)(932.93263184,78.88931936)
\curveto(932.98262993,78.82931753)(933.02762988,78.7543176)(933.06763184,78.66431936)
\curveto(933.07762983,78.64431771)(933.08762982,78.61431774)(933.09763184,78.57431936)
\curveto(933.11762979,78.53431782)(933.12262979,78.48931787)(933.11263184,78.43931936)
\curveto(933.08262983,78.34931801)(933.0076299,78.29431806)(932.88763184,78.27431936)
\curveto(932.77763013,78.2543181)(932.68263023,78.26931809)(932.60263184,78.31931936)
\curveto(932.53263038,78.34931801)(932.46763044,78.39431796)(932.40763184,78.45431936)
\curveto(932.35763055,78.52431783)(932.3076306,78.58931777)(932.25763184,78.64931936)
\curveto(932.2076307,78.71931764)(932.13263078,78.77931758)(932.03263184,78.82931936)
\curveto(931.94263097,78.88931747)(931.85263106,78.93931742)(931.76263184,78.97931936)
\curveto(931.73263118,78.99931736)(931.67263124,79.02431733)(931.58263184,79.05431936)
\curveto(931.50263141,79.08431727)(931.43263148,79.08931727)(931.37263184,79.06931936)
\curveto(931.23263168,79.03931732)(931.14263177,78.97931738)(931.10263184,78.88931936)
\curveto(931.07263184,78.80931755)(931.05763185,78.71931764)(931.05763184,78.61931936)
\curveto(931.05763185,78.51931784)(931.03263188,78.43431792)(930.98263184,78.36431936)
\curveto(930.912632,78.27431808)(930.77263214,78.22931813)(930.56263184,78.22931936)
\lineto(930.00763184,78.22931936)
\lineto(929.78263184,78.22931936)
\curveto(929.70263321,78.23931812)(929.63763327,78.2593181)(929.58763184,78.28931936)
\curveto(929.5076334,78.34931801)(929.46263345,78.41931794)(929.45263184,78.49931936)
\curveto(929.44263347,78.51931784)(929.43763347,78.53931782)(929.43763184,78.55931936)
\curveto(929.43763347,78.58931777)(929.43263348,78.61431774)(929.42263184,78.63431936)
}
}
{
\newrgbcolor{curcolor}{0 0 0}
\pscustom[linestyle=none,fillstyle=solid,fillcolor=curcolor]
{
}
}
{
\newrgbcolor{curcolor}{0 0 0}
\pscustom[linestyle=none,fillstyle=solid,fillcolor=curcolor]
{
\newpath
\moveto(920.45263184,89.26463186)
\curveto(920.44264247,89.95462722)(920.56264235,90.55462662)(920.81263184,91.06463186)
\curveto(921.06264185,91.58462559)(921.39764151,91.9796252)(921.81763184,92.24963186)
\curveto(921.89764101,92.29962488)(921.98764092,92.34462483)(922.08763184,92.38463186)
\curveto(922.17764073,92.42462475)(922.27264064,92.46962471)(922.37263184,92.51963186)
\curveto(922.47264044,92.55962462)(922.57264034,92.58962459)(922.67263184,92.60963186)
\curveto(922.77264014,92.62962455)(922.87764003,92.64962453)(922.98763184,92.66963186)
\curveto(923.03763987,92.68962449)(923.08263983,92.69462448)(923.12263184,92.68463186)
\curveto(923.16263975,92.6746245)(923.2076397,92.6796245)(923.25763184,92.69963186)
\curveto(923.3076396,92.70962447)(923.39263952,92.71462446)(923.51263184,92.71463186)
\curveto(923.62263929,92.71462446)(923.7076392,92.70962447)(923.76763184,92.69963186)
\curveto(923.82763908,92.6796245)(923.88763902,92.66962451)(923.94763184,92.66963186)
\curveto(924.0076389,92.6796245)(924.06763884,92.6746245)(924.12763184,92.65463186)
\curveto(924.26763864,92.61462456)(924.40263851,92.5796246)(924.53263184,92.54963186)
\curveto(924.66263825,92.51962466)(924.78763812,92.4796247)(924.90763184,92.42963186)
\curveto(925.04763786,92.36962481)(925.17263774,92.29962488)(925.28263184,92.21963186)
\curveto(925.39263752,92.14962503)(925.50263741,92.0746251)(925.61263184,91.99463186)
\lineto(925.67263184,91.93463186)
\curveto(925.69263722,91.92462525)(925.7126372,91.90962527)(925.73263184,91.88963186)
\curveto(925.89263702,91.76962541)(926.03763687,91.63462554)(926.16763184,91.48463186)
\curveto(926.29763661,91.33462584)(926.42263649,91.174626)(926.54263184,91.00463186)
\curveto(926.76263615,90.69462648)(926.96763594,90.39962678)(927.15763184,90.11963186)
\curveto(927.29763561,89.88962729)(927.43263548,89.65962752)(927.56263184,89.42963186)
\curveto(927.69263522,89.20962797)(927.82763508,88.98962819)(927.96763184,88.76963186)
\curveto(928.13763477,88.51962866)(928.31763459,88.2796289)(928.50763184,88.04963186)
\curveto(928.69763421,87.82962935)(928.92263399,87.63962954)(929.18263184,87.47963186)
\curveto(929.24263367,87.43962974)(929.30263361,87.40462977)(929.36263184,87.37463186)
\curveto(929.4126335,87.34462983)(929.47763343,87.31462986)(929.55763184,87.28463186)
\curveto(929.62763328,87.26462991)(929.68763322,87.25962992)(929.73763184,87.26963186)
\curveto(929.8076331,87.28962989)(929.86263305,87.32462985)(929.90263184,87.37463186)
\curveto(929.93263298,87.42462975)(929.95263296,87.48462969)(929.96263184,87.55463186)
\lineto(929.96263184,87.79463186)
\lineto(929.96263184,88.54463186)
\lineto(929.96263184,91.34963186)
\lineto(929.96263184,92.00963186)
\curveto(929.96263295,92.09962508)(929.96763294,92.18462499)(929.97763184,92.26463186)
\curveto(929.97763293,92.34462483)(929.99763291,92.40962477)(930.03763184,92.45963186)
\curveto(930.07763283,92.50962467)(930.15263276,92.54962463)(930.26263184,92.57963186)
\curveto(930.36263255,92.61962456)(930.46263245,92.62962455)(930.56263184,92.60963186)
\lineto(930.69763184,92.60963186)
\curveto(930.76763214,92.58962459)(930.82763208,92.56962461)(930.87763184,92.54963186)
\curveto(930.92763198,92.52962465)(930.96763194,92.49462468)(930.99763184,92.44463186)
\curveto(931.03763187,92.39462478)(931.05763185,92.32462485)(931.05763184,92.23463186)
\lineto(931.05763184,91.96463186)
\lineto(931.05763184,91.06463186)
\lineto(931.05763184,87.55463186)
\lineto(931.05763184,86.48963186)
\curveto(931.05763185,86.40963077)(931.06263185,86.31963086)(931.07263184,86.21963186)
\curveto(931.07263184,86.11963106)(931.06263185,86.03463114)(931.04263184,85.96463186)
\curveto(930.97263194,85.75463142)(930.79263212,85.68963149)(930.50263184,85.76963186)
\curveto(930.46263245,85.7796314)(930.42763248,85.7796314)(930.39763184,85.76963186)
\curveto(930.35763255,85.76963141)(930.3126326,85.7796314)(930.26263184,85.79963186)
\curveto(930.18263273,85.81963136)(930.09763281,85.83963134)(930.00763184,85.85963186)
\curveto(929.91763299,85.8796313)(929.83263308,85.90463127)(929.75263184,85.93463186)
\curveto(929.26263365,86.09463108)(928.84763406,86.29463088)(928.50763184,86.53463186)
\curveto(928.25763465,86.71463046)(928.03263488,86.91963026)(927.83263184,87.14963186)
\curveto(927.62263529,87.3796298)(927.42763548,87.61962956)(927.24763184,87.86963186)
\curveto(927.06763584,88.12962905)(926.89763601,88.39462878)(926.73763184,88.66463186)
\curveto(926.56763634,88.94462823)(926.39263652,89.21462796)(926.21263184,89.47463186)
\curveto(926.13263678,89.58462759)(926.05763685,89.68962749)(925.98763184,89.78963186)
\curveto(925.91763699,89.89962728)(925.84263707,90.00962717)(925.76263184,90.11963186)
\curveto(925.73263718,90.15962702)(925.70263721,90.19462698)(925.67263184,90.22463186)
\curveto(925.63263728,90.26462691)(925.60263731,90.30462687)(925.58263184,90.34463186)
\curveto(925.47263744,90.48462669)(925.34763756,90.60962657)(925.20763184,90.71963186)
\curveto(925.17763773,90.73962644)(925.15263776,90.76462641)(925.13263184,90.79463186)
\curveto(925.10263781,90.82462635)(925.07263784,90.84962633)(925.04263184,90.86963186)
\curveto(924.94263797,90.94962623)(924.84263807,91.01462616)(924.74263184,91.06463186)
\curveto(924.64263827,91.12462605)(924.53263838,91.179626)(924.41263184,91.22963186)
\curveto(924.34263857,91.25962592)(924.26763864,91.2796259)(924.18763184,91.28963186)
\lineto(923.94763184,91.34963186)
\lineto(923.85763184,91.34963186)
\curveto(923.82763908,91.35962582)(923.79763911,91.36462581)(923.76763184,91.36463186)
\curveto(923.69763921,91.38462579)(923.60263931,91.38962579)(923.48263184,91.37963186)
\curveto(923.35263956,91.3796258)(923.25263966,91.36962581)(923.18263184,91.34963186)
\curveto(923.10263981,91.32962585)(923.02763988,91.30962587)(922.95763184,91.28963186)
\curveto(922.87764003,91.2796259)(922.79764011,91.25962592)(922.71763184,91.22963186)
\curveto(922.47764043,91.11962606)(922.27764063,90.96962621)(922.11763184,90.77963186)
\curveto(921.94764096,90.59962658)(921.8076411,90.3796268)(921.69763184,90.11963186)
\curveto(921.67764123,90.04962713)(921.66264125,89.9796272)(921.65263184,89.90963186)
\curveto(921.63264128,89.83962734)(921.6126413,89.76462741)(921.59263184,89.68463186)
\curveto(921.57264134,89.60462757)(921.56264135,89.49462768)(921.56263184,89.35463186)
\curveto(921.56264135,89.22462795)(921.57264134,89.11962806)(921.59263184,89.03963186)
\curveto(921.60264131,88.9796282)(921.6076413,88.92462825)(921.60763184,88.87463186)
\curveto(921.6076413,88.82462835)(921.61764129,88.7746284)(921.63763184,88.72463186)
\curveto(921.67764123,88.62462855)(921.71764119,88.52962865)(921.75763184,88.43963186)
\curveto(921.79764111,88.35962882)(921.84264107,88.2796289)(921.89263184,88.19963186)
\curveto(921.912641,88.16962901)(921.93764097,88.13962904)(921.96763184,88.10963186)
\curveto(921.99764091,88.08962909)(922.02264089,88.06462911)(922.04263184,88.03463186)
\lineto(922.11763184,87.95963186)
\curveto(922.13764077,87.92962925)(922.15764075,87.90462927)(922.17763184,87.88463186)
\lineto(922.38763184,87.73463186)
\curveto(922.44764046,87.69462948)(922.5126404,87.64962953)(922.58263184,87.59963186)
\curveto(922.67264024,87.53962964)(922.77764013,87.48962969)(922.89763184,87.44963186)
\curveto(923.0076399,87.41962976)(923.11763979,87.38462979)(923.22763184,87.34463186)
\curveto(923.33763957,87.30462987)(923.48263943,87.2796299)(923.66263184,87.26963186)
\curveto(923.83263908,87.25962992)(923.95763895,87.22962995)(924.03763184,87.17963186)
\curveto(924.11763879,87.12963005)(924.16263875,87.05463012)(924.17263184,86.95463186)
\curveto(924.18263873,86.85463032)(924.18763872,86.74463043)(924.18763184,86.62463186)
\curveto(924.18763872,86.58463059)(924.19263872,86.54463063)(924.20263184,86.50463186)
\curveto(924.20263871,86.46463071)(924.19763871,86.42963075)(924.18763184,86.39963186)
\curveto(924.16763874,86.34963083)(924.15763875,86.29963088)(924.15763184,86.24963186)
\curveto(924.15763875,86.20963097)(924.14763876,86.16963101)(924.12763184,86.12963186)
\curveto(924.06763884,86.03963114)(923.93263898,85.99463118)(923.72263184,85.99463186)
\lineto(923.60263184,85.99463186)
\curveto(923.54263937,86.00463117)(923.48263943,86.00963117)(923.42263184,86.00963186)
\curveto(923.35263956,86.01963116)(923.28763962,86.02963115)(923.22763184,86.03963186)
\curveto(923.11763979,86.05963112)(923.01763989,86.0796311)(922.92763184,86.09963186)
\curveto(922.82764008,86.11963106)(922.73264018,86.14963103)(922.64263184,86.18963186)
\curveto(922.57264034,86.20963097)(922.5126404,86.22963095)(922.46263184,86.24963186)
\lineto(922.28263184,86.30963186)
\curveto(922.02264089,86.42963075)(921.77764113,86.58463059)(921.54763184,86.77463186)
\curveto(921.31764159,86.9746302)(921.13264178,87.18962999)(920.99263184,87.41963186)
\curveto(920.912642,87.52962965)(920.84764206,87.64462953)(920.79763184,87.76463186)
\lineto(920.64763184,88.15463186)
\curveto(920.59764231,88.26462891)(920.56764234,88.3796288)(920.55763184,88.49963186)
\curveto(920.53764237,88.61962856)(920.5126424,88.74462843)(920.48263184,88.87463186)
\curveto(920.48264243,88.94462823)(920.48264243,89.00962817)(920.48263184,89.06963186)
\curveto(920.47264244,89.12962805)(920.46264245,89.19462798)(920.45263184,89.26463186)
}
}
{
\newrgbcolor{curcolor}{0 0 0}
\pscustom[linestyle=none,fillstyle=solid,fillcolor=curcolor]
{
\newpath
\moveto(925.97263184,101.36424123)
\lineto(926.22763184,101.36424123)
\curveto(926.3076366,101.37423353)(926.38263653,101.36923353)(926.45263184,101.34924123)
\lineto(926.69263184,101.34924123)
\lineto(926.85763184,101.34924123)
\curveto(926.95763595,101.32923357)(927.06263585,101.31923358)(927.17263184,101.31924123)
\curveto(927.27263564,101.31923358)(927.37263554,101.30923359)(927.47263184,101.28924123)
\lineto(927.62263184,101.28924123)
\curveto(927.76263515,101.25923364)(927.90263501,101.23923366)(928.04263184,101.22924123)
\curveto(928.17263474,101.21923368)(928.30263461,101.19423371)(928.43263184,101.15424123)
\curveto(928.5126344,101.13423377)(928.59763431,101.11423379)(928.68763184,101.09424123)
\lineto(928.92763184,101.03424123)
\lineto(929.22763184,100.91424123)
\curveto(929.31763359,100.88423402)(929.4076335,100.84923405)(929.49763184,100.80924123)
\curveto(929.71763319,100.70923419)(929.93263298,100.57423433)(930.14263184,100.40424123)
\curveto(930.35263256,100.24423466)(930.52263239,100.06923483)(930.65263184,99.87924123)
\curveto(930.69263222,99.82923507)(930.73263218,99.76923513)(930.77263184,99.69924123)
\curveto(930.80263211,99.63923526)(930.83763207,99.57923532)(930.87763184,99.51924123)
\curveto(930.92763198,99.43923546)(930.96763194,99.34423556)(930.99763184,99.23424123)
\curveto(931.02763188,99.12423578)(931.05763185,99.01923588)(931.08763184,98.91924123)
\curveto(931.12763178,98.80923609)(931.15263176,98.6992362)(931.16263184,98.58924123)
\curveto(931.17263174,98.47923642)(931.18763172,98.36423654)(931.20763184,98.24424123)
\curveto(931.21763169,98.2042367)(931.21763169,98.15923674)(931.20763184,98.10924123)
\curveto(931.2076317,98.06923683)(931.2126317,98.02923687)(931.22263184,97.98924123)
\curveto(931.23263168,97.94923695)(931.23763167,97.89423701)(931.23763184,97.82424123)
\curveto(931.23763167,97.75423715)(931.23263168,97.7042372)(931.22263184,97.67424123)
\curveto(931.20263171,97.62423728)(931.19763171,97.57923732)(931.20763184,97.53924123)
\curveto(931.21763169,97.4992374)(931.21763169,97.46423744)(931.20763184,97.43424123)
\lineto(931.20763184,97.34424123)
\curveto(931.18763172,97.28423762)(931.17263174,97.21923768)(931.16263184,97.14924123)
\curveto(931.16263175,97.08923781)(931.15763175,97.02423788)(931.14763184,96.95424123)
\curveto(931.09763181,96.78423812)(931.04763186,96.62423828)(930.99763184,96.47424123)
\curveto(930.94763196,96.32423858)(930.88263203,96.17923872)(930.80263184,96.03924123)
\curveto(930.76263215,95.98923891)(930.73263218,95.93423897)(930.71263184,95.87424123)
\curveto(930.68263223,95.82423908)(930.64763226,95.77423913)(930.60763184,95.72424123)
\curveto(930.42763248,95.48423942)(930.2076327,95.28423962)(929.94763184,95.12424123)
\curveto(929.68763322,94.96423994)(929.40263351,94.82424008)(929.09263184,94.70424123)
\curveto(928.95263396,94.64424026)(928.8126341,94.5992403)(928.67263184,94.56924123)
\curveto(928.52263439,94.53924036)(928.36763454,94.5042404)(928.20763184,94.46424123)
\curveto(928.09763481,94.44424046)(927.98763492,94.42924047)(927.87763184,94.41924123)
\curveto(927.76763514,94.40924049)(927.65763525,94.39424051)(927.54763184,94.37424123)
\curveto(927.5076354,94.36424054)(927.46763544,94.35924054)(927.42763184,94.35924123)
\curveto(927.38763552,94.36924053)(927.34763556,94.36924053)(927.30763184,94.35924123)
\curveto(927.25763565,94.34924055)(927.2076357,94.34424056)(927.15763184,94.34424123)
\lineto(926.99263184,94.34424123)
\curveto(926.94263597,94.32424058)(926.89263602,94.31924058)(926.84263184,94.32924123)
\curveto(926.78263613,94.33924056)(926.72763618,94.33924056)(926.67763184,94.32924123)
\curveto(926.63763627,94.31924058)(926.59263632,94.31924058)(926.54263184,94.32924123)
\curveto(926.49263642,94.33924056)(926.44263647,94.33424057)(926.39263184,94.31424123)
\curveto(926.32263659,94.29424061)(926.24763666,94.28924061)(926.16763184,94.29924123)
\curveto(926.07763683,94.30924059)(925.99263692,94.31424059)(925.91263184,94.31424123)
\curveto(925.82263709,94.31424059)(925.72263719,94.30924059)(925.61263184,94.29924123)
\curveto(925.49263742,94.28924061)(925.39263752,94.29424061)(925.31263184,94.31424123)
\lineto(925.02763184,94.31424123)
\lineto(924.39763184,94.35924123)
\curveto(924.29763861,94.36924053)(924.20263871,94.37924052)(924.11263184,94.38924123)
\lineto(923.81263184,94.41924123)
\curveto(923.76263915,94.43924046)(923.7126392,94.44424046)(923.66263184,94.43424123)
\curveto(923.60263931,94.43424047)(923.54763936,94.44424046)(923.49763184,94.46424123)
\curveto(923.32763958,94.51424039)(923.16263975,94.55424035)(923.00263184,94.58424123)
\curveto(922.83264008,94.61424029)(922.67264024,94.66424024)(922.52263184,94.73424123)
\curveto(922.06264085,94.92423998)(921.68764122,95.14423976)(921.39763184,95.39424123)
\curveto(921.1076418,95.65423925)(920.86264205,96.01423889)(920.66263184,96.47424123)
\curveto(920.6126423,96.6042383)(920.57764233,96.73423817)(920.55763184,96.86424123)
\curveto(920.53764237,97.0042379)(920.5126424,97.14423776)(920.48263184,97.28424123)
\curveto(920.47264244,97.35423755)(920.46764244,97.41923748)(920.46763184,97.47924123)
\curveto(920.46764244,97.53923736)(920.46264245,97.6042373)(920.45263184,97.67424123)
\curveto(920.43264248,98.5042364)(920.58264233,99.17423573)(920.90263184,99.68424123)
\curveto(921.2126417,100.19423471)(921.65264126,100.57423433)(922.22263184,100.82424123)
\curveto(922.34264057,100.87423403)(922.46764044,100.91923398)(922.59763184,100.95924123)
\curveto(922.72764018,100.9992339)(922.86264005,101.04423386)(923.00263184,101.09424123)
\curveto(923.08263983,101.11423379)(923.16763974,101.12923377)(923.25763184,101.13924123)
\lineto(923.49763184,101.19924123)
\curveto(923.6076393,101.22923367)(923.71763919,101.24423366)(923.82763184,101.24424123)
\curveto(923.93763897,101.25423365)(924.04763886,101.26923363)(924.15763184,101.28924123)
\curveto(924.2076387,101.30923359)(924.25263866,101.31423359)(924.29263184,101.30424123)
\curveto(924.33263858,101.3042336)(924.37263854,101.30923359)(924.41263184,101.31924123)
\curveto(924.46263845,101.32923357)(924.51763839,101.32923357)(924.57763184,101.31924123)
\curveto(924.62763828,101.31923358)(924.67763823,101.32423358)(924.72763184,101.33424123)
\lineto(924.86263184,101.33424123)
\curveto(924.92263799,101.35423355)(924.99263792,101.35423355)(925.07263184,101.33424123)
\curveto(925.14263777,101.32423358)(925.2076377,101.32923357)(925.26763184,101.34924123)
\curveto(925.29763761,101.35923354)(925.33763757,101.36423354)(925.38763184,101.36424123)
\lineto(925.50763184,101.36424123)
\lineto(925.97263184,101.36424123)
\moveto(928.29763184,99.81924123)
\curveto(927.97763493,99.91923498)(927.6126353,99.97923492)(927.20263184,99.99924123)
\curveto(926.79263612,100.01923488)(926.38263653,100.02923487)(925.97263184,100.02924123)
\curveto(925.54263737,100.02923487)(925.12263779,100.01923488)(924.71263184,99.99924123)
\curveto(924.30263861,99.97923492)(923.91763899,99.93423497)(923.55763184,99.86424123)
\curveto(923.19763971,99.79423511)(922.87764003,99.68423522)(922.59763184,99.53424123)
\curveto(922.3076406,99.39423551)(922.07264084,99.1992357)(921.89263184,98.94924123)
\curveto(921.78264113,98.78923611)(921.70264121,98.60923629)(921.65263184,98.40924123)
\curveto(921.59264132,98.20923669)(921.56264135,97.96423694)(921.56263184,97.67424123)
\curveto(921.58264133,97.65423725)(921.59264132,97.61923728)(921.59263184,97.56924123)
\curveto(921.58264133,97.51923738)(921.58264133,97.47923742)(921.59263184,97.44924123)
\curveto(921.6126413,97.36923753)(921.63264128,97.29423761)(921.65263184,97.22424123)
\curveto(921.66264125,97.16423774)(921.68264123,97.0992378)(921.71263184,97.02924123)
\curveto(921.83264108,96.75923814)(922.00264091,96.53923836)(922.22263184,96.36924123)
\curveto(922.43264048,96.20923869)(922.67764023,96.07423883)(922.95763184,95.96424123)
\curveto(923.06763984,95.91423899)(923.18763972,95.87423903)(923.31763184,95.84424123)
\curveto(923.43763947,95.82423908)(923.56263935,95.7992391)(923.69263184,95.76924123)
\curveto(923.74263917,95.74923915)(923.79763911,95.73923916)(923.85763184,95.73924123)
\curveto(923.907639,95.73923916)(923.95763895,95.73423917)(924.00763184,95.72424123)
\curveto(924.09763881,95.71423919)(924.19263872,95.7042392)(924.29263184,95.69424123)
\curveto(924.38263853,95.68423922)(924.47763843,95.67423923)(924.57763184,95.66424123)
\curveto(924.65763825,95.66423924)(924.74263817,95.65923924)(924.83263184,95.64924123)
\lineto(925.07263184,95.64924123)
\lineto(925.25263184,95.64924123)
\curveto(925.28263763,95.63923926)(925.31763759,95.63423927)(925.35763184,95.63424123)
\lineto(925.49263184,95.63424123)
\lineto(925.94263184,95.63424123)
\curveto(926.02263689,95.63423927)(926.1076368,95.62923927)(926.19763184,95.61924123)
\curveto(926.27763663,95.61923928)(926.35263656,95.62923927)(926.42263184,95.64924123)
\lineto(926.69263184,95.64924123)
\curveto(926.7126362,95.64923925)(926.74263617,95.64423926)(926.78263184,95.63424123)
\curveto(926.8126361,95.63423927)(926.83763607,95.63923926)(926.85763184,95.64924123)
\curveto(926.95763595,95.65923924)(927.05763585,95.66423924)(927.15763184,95.66424123)
\curveto(927.24763566,95.67423923)(927.34763556,95.68423922)(927.45763184,95.69424123)
\curveto(927.57763533,95.72423918)(927.70263521,95.73923916)(927.83263184,95.73924123)
\curveto(927.95263496,95.74923915)(928.06763484,95.77423913)(928.17763184,95.81424123)
\curveto(928.47763443,95.89423901)(928.74263417,95.97923892)(928.97263184,96.06924123)
\curveto(929.20263371,96.16923873)(929.41763349,96.31423859)(929.61763184,96.50424123)
\curveto(929.81763309,96.71423819)(929.96763294,96.97923792)(930.06763184,97.29924123)
\curveto(930.08763282,97.33923756)(930.09763281,97.37423753)(930.09763184,97.40424123)
\curveto(930.08763282,97.44423746)(930.09263282,97.48923741)(930.11263184,97.53924123)
\curveto(930.12263279,97.57923732)(930.13263278,97.64923725)(930.14263184,97.74924123)
\curveto(930.15263276,97.85923704)(930.14763276,97.94423696)(930.12763184,98.00424123)
\curveto(930.1076328,98.07423683)(930.09763281,98.14423676)(930.09763184,98.21424123)
\curveto(930.08763282,98.28423662)(930.07263284,98.34923655)(930.05263184,98.40924123)
\curveto(929.99263292,98.60923629)(929.907633,98.78923611)(929.79763184,98.94924123)
\curveto(929.77763313,98.97923592)(929.75763315,99.0042359)(929.73763184,99.02424123)
\lineto(929.67763184,99.08424123)
\curveto(929.65763325,99.12423578)(929.61763329,99.17423573)(929.55763184,99.23424123)
\curveto(929.41763349,99.33423557)(929.28763362,99.41923548)(929.16763184,99.48924123)
\curveto(929.04763386,99.55923534)(928.90263401,99.62923527)(928.73263184,99.69924123)
\curveto(928.66263425,99.72923517)(928.59263432,99.74923515)(928.52263184,99.75924123)
\curveto(928.45263446,99.77923512)(928.37763453,99.7992351)(928.29763184,99.81924123)
}
}
{
\newrgbcolor{curcolor}{0 0 0}
\pscustom[linestyle=none,fillstyle=solid,fillcolor=curcolor]
{
\newpath
\moveto(920.45263184,106.77385061)
\curveto(920.45264246,106.87384575)(920.46264245,106.96884566)(920.48263184,107.05885061)
\curveto(920.49264242,107.14884548)(920.52264239,107.21384541)(920.57263184,107.25385061)
\curveto(920.65264226,107.31384531)(920.75764215,107.34384528)(920.88763184,107.34385061)
\lineto(921.27763184,107.34385061)
\lineto(922.77763184,107.34385061)
\lineto(929.16763184,107.34385061)
\lineto(930.33763184,107.34385061)
\lineto(930.65263184,107.34385061)
\curveto(930.75263216,107.35384527)(930.83263208,107.33884529)(930.89263184,107.29885061)
\curveto(930.97263194,107.24884538)(931.02263189,107.17384545)(931.04263184,107.07385061)
\curveto(931.05263186,106.98384564)(931.05763185,106.87384575)(931.05763184,106.74385061)
\lineto(931.05763184,106.51885061)
\curveto(931.03763187,106.43884619)(931.02263189,106.36884626)(931.01263184,106.30885061)
\curveto(930.99263192,106.24884638)(930.95263196,106.19884643)(930.89263184,106.15885061)
\curveto(930.83263208,106.11884651)(930.75763215,106.09884653)(930.66763184,106.09885061)
\lineto(930.36763184,106.09885061)
\lineto(929.27263184,106.09885061)
\lineto(923.93263184,106.09885061)
\curveto(923.84263907,106.07884655)(923.76763914,106.06384656)(923.70763184,106.05385061)
\curveto(923.63763927,106.05384657)(923.57763933,106.0238466)(923.52763184,105.96385061)
\curveto(923.47763943,105.89384673)(923.45263946,105.80384682)(923.45263184,105.69385061)
\curveto(923.44263947,105.59384703)(923.43763947,105.48384714)(923.43763184,105.36385061)
\lineto(923.43763184,104.22385061)
\lineto(923.43763184,103.72885061)
\curveto(923.42763948,103.56884906)(923.36763954,103.45884917)(923.25763184,103.39885061)
\curveto(923.22763968,103.37884925)(923.19763971,103.36884926)(923.16763184,103.36885061)
\curveto(923.12763978,103.36884926)(923.08263983,103.36384926)(923.03263184,103.35385061)
\curveto(922.91264,103.33384929)(922.80264011,103.33884929)(922.70263184,103.36885061)
\curveto(922.60264031,103.40884922)(922.53264038,103.46384916)(922.49263184,103.53385061)
\curveto(922.44264047,103.61384901)(922.41764049,103.73384889)(922.41763184,103.89385061)
\curveto(922.41764049,104.05384857)(922.40264051,104.18884844)(922.37263184,104.29885061)
\curveto(922.36264055,104.34884828)(922.35764055,104.40384822)(922.35763184,104.46385061)
\curveto(922.34764056,104.5238481)(922.33264058,104.58384804)(922.31263184,104.64385061)
\curveto(922.26264065,104.79384783)(922.2126407,104.93884769)(922.16263184,105.07885061)
\curveto(922.10264081,105.21884741)(922.03264088,105.35384727)(921.95263184,105.48385061)
\curveto(921.86264105,105.623847)(921.75764115,105.74384688)(921.63763184,105.84385061)
\curveto(921.51764139,105.94384668)(921.38764152,106.03884659)(921.24763184,106.12885061)
\curveto(921.14764176,106.18884644)(921.03764187,106.23384639)(920.91763184,106.26385061)
\curveto(920.79764211,106.30384632)(920.69264222,106.35384627)(920.60263184,106.41385061)
\curveto(920.54264237,106.46384616)(920.50264241,106.53384609)(920.48263184,106.62385061)
\curveto(920.47264244,106.64384598)(920.46764244,106.66884596)(920.46763184,106.69885061)
\curveto(920.46764244,106.7288459)(920.46264245,106.75384587)(920.45263184,106.77385061)
}
}
{
\newrgbcolor{curcolor}{0 0 0}
\pscustom[linestyle=none,fillstyle=solid,fillcolor=curcolor]
{
\newpath
\moveto(920.45263184,115.12345998)
\curveto(920.45264246,115.22345513)(920.46264245,115.31845503)(920.48263184,115.40845998)
\curveto(920.49264242,115.49845485)(920.52264239,115.56345479)(920.57263184,115.60345998)
\curveto(920.65264226,115.66345469)(920.75764215,115.69345466)(920.88763184,115.69345998)
\lineto(921.27763184,115.69345998)
\lineto(922.77763184,115.69345998)
\lineto(929.16763184,115.69345998)
\lineto(930.33763184,115.69345998)
\lineto(930.65263184,115.69345998)
\curveto(930.75263216,115.70345465)(930.83263208,115.68845466)(930.89263184,115.64845998)
\curveto(930.97263194,115.59845475)(931.02263189,115.52345483)(931.04263184,115.42345998)
\curveto(931.05263186,115.33345502)(931.05763185,115.22345513)(931.05763184,115.09345998)
\lineto(931.05763184,114.86845998)
\curveto(931.03763187,114.78845556)(931.02263189,114.71845563)(931.01263184,114.65845998)
\curveto(930.99263192,114.59845575)(930.95263196,114.5484558)(930.89263184,114.50845998)
\curveto(930.83263208,114.46845588)(930.75763215,114.4484559)(930.66763184,114.44845998)
\lineto(930.36763184,114.44845998)
\lineto(929.27263184,114.44845998)
\lineto(923.93263184,114.44845998)
\curveto(923.84263907,114.42845592)(923.76763914,114.41345594)(923.70763184,114.40345998)
\curveto(923.63763927,114.40345595)(923.57763933,114.37345598)(923.52763184,114.31345998)
\curveto(923.47763943,114.24345611)(923.45263946,114.1534562)(923.45263184,114.04345998)
\curveto(923.44263947,113.94345641)(923.43763947,113.83345652)(923.43763184,113.71345998)
\lineto(923.43763184,112.57345998)
\lineto(923.43763184,112.07845998)
\curveto(923.42763948,111.91845843)(923.36763954,111.80845854)(923.25763184,111.74845998)
\curveto(923.22763968,111.72845862)(923.19763971,111.71845863)(923.16763184,111.71845998)
\curveto(923.12763978,111.71845863)(923.08263983,111.71345864)(923.03263184,111.70345998)
\curveto(922.91264,111.68345867)(922.80264011,111.68845866)(922.70263184,111.71845998)
\curveto(922.60264031,111.75845859)(922.53264038,111.81345854)(922.49263184,111.88345998)
\curveto(922.44264047,111.96345839)(922.41764049,112.08345827)(922.41763184,112.24345998)
\curveto(922.41764049,112.40345795)(922.40264051,112.53845781)(922.37263184,112.64845998)
\curveto(922.36264055,112.69845765)(922.35764055,112.7534576)(922.35763184,112.81345998)
\curveto(922.34764056,112.87345748)(922.33264058,112.93345742)(922.31263184,112.99345998)
\curveto(922.26264065,113.14345721)(922.2126407,113.28845706)(922.16263184,113.42845998)
\curveto(922.10264081,113.56845678)(922.03264088,113.70345665)(921.95263184,113.83345998)
\curveto(921.86264105,113.97345638)(921.75764115,114.09345626)(921.63763184,114.19345998)
\curveto(921.51764139,114.29345606)(921.38764152,114.38845596)(921.24763184,114.47845998)
\curveto(921.14764176,114.53845581)(921.03764187,114.58345577)(920.91763184,114.61345998)
\curveto(920.79764211,114.6534557)(920.69264222,114.70345565)(920.60263184,114.76345998)
\curveto(920.54264237,114.81345554)(920.50264241,114.88345547)(920.48263184,114.97345998)
\curveto(920.47264244,114.99345536)(920.46764244,115.01845533)(920.46763184,115.04845998)
\curveto(920.46764244,115.07845527)(920.46264245,115.10345525)(920.45263184,115.12345998)
}
}
{
\newrgbcolor{curcolor}{0 0 0}
\pscustom[linestyle=none,fillstyle=solid,fillcolor=curcolor]
{
\newpath
\moveto(941.28894775,42.29681936)
\curveto(941.28895845,42.36681368)(941.28895845,42.4468136)(941.28894775,42.53681936)
\curveto(941.27895846,42.62681342)(941.27895846,42.71181333)(941.28894775,42.79181936)
\curveto(941.28895845,42.88181316)(941.29895844,42.96181308)(941.31894775,43.03181936)
\curveto(941.3389584,43.11181293)(941.36895837,43.16681288)(941.40894775,43.19681936)
\curveto(941.45895828,43.22681282)(941.5339582,43.2468128)(941.63394775,43.25681936)
\curveto(941.72395801,43.27681277)(941.82895791,43.28681276)(941.94894775,43.28681936)
\curveto(942.05895768,43.29681275)(942.17395756,43.29681275)(942.29394775,43.28681936)
\lineto(942.59394775,43.28681936)
\lineto(945.60894775,43.28681936)
\lineto(948.50394775,43.28681936)
\curveto(948.8339509,43.28681276)(949.15895058,43.28181276)(949.47894775,43.27181936)
\curveto(949.78894995,43.27181277)(950.06894967,43.23181281)(950.31894775,43.15181936)
\curveto(950.66894907,43.03181301)(950.96394877,42.87681317)(951.20394775,42.68681936)
\curveto(951.4339483,42.49681355)(951.6339481,42.25681379)(951.80394775,41.96681936)
\curveto(951.85394788,41.90681414)(951.88894785,41.8418142)(951.90894775,41.77181936)
\curveto(951.92894781,41.71181433)(951.95394778,41.6418144)(951.98394775,41.56181936)
\curveto(952.0339477,41.4418146)(952.06894767,41.31181473)(952.08894775,41.17181936)
\curveto(952.11894762,41.041815)(952.14894759,40.90681514)(952.17894775,40.76681936)
\curveto(952.19894754,40.71681533)(952.20394753,40.66681538)(952.19394775,40.61681936)
\curveto(952.18394755,40.56681548)(952.18394755,40.51181553)(952.19394775,40.45181936)
\curveto(952.20394753,40.43181561)(952.20394753,40.40681564)(952.19394775,40.37681936)
\curveto(952.19394754,40.3468157)(952.19894754,40.32181572)(952.20894775,40.30181936)
\curveto(952.21894752,40.26181578)(952.22394751,40.20681584)(952.22394775,40.13681936)
\curveto(952.22394751,40.06681598)(952.21894752,40.01181603)(952.20894775,39.97181936)
\curveto(952.19894754,39.92181612)(952.19894754,39.86681618)(952.20894775,39.80681936)
\curveto(952.21894752,39.7468163)(952.21394752,39.69181635)(952.19394775,39.64181936)
\curveto(952.16394757,39.51181653)(952.14394759,39.38681666)(952.13394775,39.26681936)
\curveto(952.12394761,39.1468169)(952.09894764,39.03181701)(952.05894775,38.92181936)
\curveto(951.9389478,38.55181749)(951.76894797,38.23181781)(951.54894775,37.96181936)
\curveto(951.32894841,37.69181835)(951.04894869,37.48181856)(950.70894775,37.33181936)
\curveto(950.58894915,37.28181876)(950.46394927,37.23681881)(950.33394775,37.19681936)
\curveto(950.20394953,37.16681888)(950.06894967,37.13181891)(949.92894775,37.09181936)
\curveto(949.87894986,37.08181896)(949.8389499,37.07681897)(949.80894775,37.07681936)
\curveto(949.76894997,37.07681897)(949.72395001,37.07181897)(949.67394775,37.06181936)
\curveto(949.64395009,37.05181899)(949.60895013,37.046819)(949.56894775,37.04681936)
\curveto(949.51895022,37.046819)(949.47895026,37.041819)(949.44894775,37.03181936)
\lineto(949.28394775,37.03181936)
\curveto(949.20395053,37.01181903)(949.10395063,37.00681904)(948.98394775,37.01681936)
\curveto(948.85395088,37.02681902)(948.76395097,37.041819)(948.71394775,37.06181936)
\curveto(948.62395111,37.08181896)(948.55895118,37.13681891)(948.51894775,37.22681936)
\curveto(948.49895124,37.25681879)(948.49395124,37.28681876)(948.50394775,37.31681936)
\curveto(948.50395123,37.3468187)(948.49895124,37.38681866)(948.48894775,37.43681936)
\curveto(948.47895126,37.47681857)(948.47395126,37.51681853)(948.47394775,37.55681936)
\lineto(948.47394775,37.70681936)
\curveto(948.47395126,37.82681822)(948.47895126,37.9468181)(948.48894775,38.06681936)
\curveto(948.48895125,38.19681785)(948.52395121,38.28681776)(948.59394775,38.33681936)
\curveto(948.65395108,38.37681767)(948.71395102,38.39681765)(948.77394775,38.39681936)
\curveto(948.8339509,38.39681765)(948.90395083,38.40681764)(948.98394775,38.42681936)
\curveto(949.01395072,38.43681761)(949.04895069,38.43681761)(949.08894775,38.42681936)
\curveto(949.11895062,38.42681762)(949.14395059,38.43181761)(949.16394775,38.44181936)
\lineto(949.37394775,38.44181936)
\curveto(949.42395031,38.46181758)(949.47395026,38.46681758)(949.52394775,38.45681936)
\curveto(949.56395017,38.45681759)(949.60895013,38.46681758)(949.65894775,38.48681936)
\curveto(949.78894995,38.51681753)(949.91394982,38.5468175)(950.03394775,38.57681936)
\curveto(950.14394959,38.60681744)(950.24894949,38.65181739)(950.34894775,38.71181936)
\curveto(950.6389491,38.88181716)(950.84394889,39.15181689)(950.96394775,39.52181936)
\curveto(950.98394875,39.57181647)(950.99894874,39.62181642)(951.00894775,39.67181936)
\curveto(951.00894873,39.73181631)(951.01894872,39.78681626)(951.03894775,39.83681936)
\lineto(951.03894775,39.91181936)
\curveto(951.04894869,39.98181606)(951.05894868,40.07681597)(951.06894775,40.19681936)
\curveto(951.06894867,40.32681572)(951.05894868,40.42681562)(951.03894775,40.49681936)
\curveto(951.01894872,40.56681548)(951.00394873,40.63681541)(950.99394775,40.70681936)
\curveto(950.97394876,40.78681526)(950.95394878,40.85681519)(950.93394775,40.91681936)
\curveto(950.77394896,41.29681475)(950.49894924,41.57181447)(950.10894775,41.74181936)
\curveto(949.97894976,41.79181425)(949.82394991,41.82681422)(949.64394775,41.84681936)
\curveto(949.46395027,41.87681417)(949.27895046,41.89181415)(949.08894775,41.89181936)
\curveto(948.88895085,41.90181414)(948.68895105,41.90181414)(948.48894775,41.89181936)
\lineto(947.91894775,41.89181936)
\lineto(943.67394775,41.89181936)
\lineto(942.12894775,41.89181936)
\curveto(942.01895772,41.89181415)(941.89895784,41.88681416)(941.76894775,41.87681936)
\curveto(941.6389581,41.86681418)(941.5339582,41.88681416)(941.45394775,41.93681936)
\curveto(941.38395835,41.99681405)(941.3339584,42.07681397)(941.30394775,42.17681936)
\curveto(941.30395843,42.19681385)(941.30395843,42.21681383)(941.30394775,42.23681936)
\curveto(941.30395843,42.25681379)(941.29895844,42.27681377)(941.28894775,42.29681936)
}
}
{
\newrgbcolor{curcolor}{0 0 0}
\pscustom[linestyle=none,fillstyle=solid,fillcolor=curcolor]
{
\newpath
\moveto(944.24394775,45.83049123)
\lineto(944.24394775,46.26549123)
\curveto(944.24395549,46.41548927)(944.28395545,46.52048916)(944.36394775,46.58049123)
\curveto(944.44395529,46.63048905)(944.54395519,46.65548903)(944.66394775,46.65549123)
\curveto(944.78395495,46.66548902)(944.90395483,46.67048901)(945.02394775,46.67049123)
\lineto(946.44894775,46.67049123)
\lineto(948.71394775,46.67049123)
\lineto(949.40394775,46.67049123)
\curveto(949.6339501,46.67048901)(949.8339499,46.69548899)(950.00394775,46.74549123)
\curveto(950.45394928,46.90548878)(950.76894897,47.20548848)(950.94894775,47.64549123)
\curveto(951.0389487,47.86548782)(951.07394866,48.13048755)(951.05394775,48.44049123)
\curveto(951.02394871,48.75048693)(950.96894877,49.00048668)(950.88894775,49.19049123)
\curveto(950.74894899,49.52048616)(950.57394916,49.7804859)(950.36394775,49.97049123)
\curveto(950.14394959,50.17048551)(949.85894988,50.32548536)(949.50894775,50.43549123)
\curveto(949.42895031,50.46548522)(949.34895039,50.4854852)(949.26894775,50.49549123)
\curveto(949.18895055,50.50548518)(949.10395063,50.52048516)(949.01394775,50.54049123)
\curveto(948.96395077,50.55048513)(948.91895082,50.55048513)(948.87894775,50.54049123)
\curveto(948.8389509,50.54048514)(948.79395094,50.55048513)(948.74394775,50.57049123)
\lineto(948.42894775,50.57049123)
\curveto(948.34895139,50.59048509)(948.25895148,50.59548509)(948.15894775,50.58549123)
\curveto(948.04895169,50.57548511)(947.94895179,50.57048511)(947.85894775,50.57049123)
\lineto(946.68894775,50.57049123)
\lineto(945.09894775,50.57049123)
\curveto(944.97895476,50.57048511)(944.85395488,50.56548512)(944.72394775,50.55549123)
\curveto(944.58395515,50.55548513)(944.47395526,50.5804851)(944.39394775,50.63049123)
\curveto(944.34395539,50.67048501)(944.31395542,50.71548497)(944.30394775,50.76549123)
\curveto(944.28395545,50.82548486)(944.26395547,50.89548479)(944.24394775,50.97549123)
\lineto(944.24394775,51.20049123)
\curveto(944.24395549,51.32048436)(944.24895549,51.42548426)(944.25894775,51.51549123)
\curveto(944.26895547,51.61548407)(944.31395542,51.69048399)(944.39394775,51.74049123)
\curveto(944.44395529,51.79048389)(944.51895522,51.81548387)(944.61894775,51.81549123)
\lineto(944.90394775,51.81549123)
\lineto(945.92394775,51.81549123)
\lineto(949.95894775,51.81549123)
\lineto(951.30894775,51.81549123)
\curveto(951.42894831,51.81548387)(951.54394819,51.81048387)(951.65394775,51.80049123)
\curveto(951.75394798,51.80048388)(951.82894791,51.76548392)(951.87894775,51.69549123)
\curveto(951.90894783,51.65548403)(951.9339478,51.59548409)(951.95394775,51.51549123)
\curveto(951.96394777,51.43548425)(951.97394776,51.34548434)(951.98394775,51.24549123)
\curveto(951.98394775,51.15548453)(951.97894776,51.06548462)(951.96894775,50.97549123)
\curveto(951.95894778,50.89548479)(951.9389478,50.83548485)(951.90894775,50.79549123)
\curveto(951.86894787,50.74548494)(951.80394793,50.70048498)(951.71394775,50.66049123)
\curveto(951.67394806,50.65048503)(951.61894812,50.64048504)(951.54894775,50.63049123)
\curveto(951.47894826,50.63048505)(951.41394832,50.62548506)(951.35394775,50.61549123)
\curveto(951.28394845,50.60548508)(951.22894851,50.5854851)(951.18894775,50.55549123)
\curveto(951.14894859,50.52548516)(951.1339486,50.4804852)(951.14394775,50.42049123)
\curveto(951.16394857,50.34048534)(951.22394851,50.26048542)(951.32394775,50.18049123)
\curveto(951.41394832,50.10048558)(951.48394825,50.02548566)(951.53394775,49.95549123)
\curveto(951.69394804,49.73548595)(951.8339479,49.4854862)(951.95394775,49.20549123)
\curveto(952.00394773,49.09548659)(952.0339477,48.9804867)(952.04394775,48.86049123)
\curveto(952.06394767,48.75048693)(952.08894765,48.63548705)(952.11894775,48.51549123)
\curveto(952.12894761,48.46548722)(952.12894761,48.41048727)(952.11894775,48.35049123)
\curveto(952.10894763,48.30048738)(952.11394762,48.25048743)(952.13394775,48.20049123)
\curveto(952.15394758,48.10048758)(952.15394758,48.01048767)(952.13394775,47.93049123)
\lineto(952.13394775,47.78049123)
\curveto(952.11394762,47.73048795)(952.10394763,47.67048801)(952.10394775,47.60049123)
\curveto(952.10394763,47.54048814)(952.09894764,47.4854882)(952.08894775,47.43549123)
\curveto(952.06894767,47.39548829)(952.05894768,47.35548833)(952.05894775,47.31549123)
\curveto(952.06894767,47.2854884)(952.06394767,47.24548844)(952.04394775,47.19549123)
\lineto(951.98394775,46.95549123)
\curveto(951.96394777,46.8854888)(951.9339478,46.81048887)(951.89394775,46.73049123)
\curveto(951.78394795,46.47048921)(951.6389481,46.25048943)(951.45894775,46.07049123)
\curveto(951.26894847,45.90048978)(951.04394869,45.76048992)(950.78394775,45.65049123)
\curveto(950.69394904,45.61049007)(950.60394913,45.5804901)(950.51394775,45.56049123)
\lineto(950.21394775,45.50049123)
\curveto(950.15394958,45.4804902)(950.09894964,45.47049021)(950.04894775,45.47049123)
\curveto(949.98894975,45.4804902)(949.92394981,45.47549021)(949.85394775,45.45549123)
\curveto(949.8339499,45.44549024)(949.80894993,45.44049024)(949.77894775,45.44049123)
\curveto(949.73895,45.44049024)(949.70395003,45.43549025)(949.67394775,45.42549123)
\lineto(949.52394775,45.42549123)
\curveto(949.48395025,45.41549027)(949.4389503,45.41049027)(949.38894775,45.41049123)
\curveto(949.32895041,45.42049026)(949.27395046,45.42549026)(949.22394775,45.42549123)
\lineto(948.62394775,45.42549123)
\lineto(945.86394775,45.42549123)
\lineto(944.90394775,45.42549123)
\lineto(944.63394775,45.42549123)
\curveto(944.54395519,45.42549026)(944.46895527,45.44549024)(944.40894775,45.48549123)
\curveto(944.3389554,45.52549016)(944.28895545,45.60049008)(944.25894775,45.71049123)
\curveto(944.24895549,45.73048995)(944.24895549,45.75048993)(944.25894775,45.77049123)
\curveto(944.25895548,45.79048989)(944.25395548,45.81048987)(944.24394775,45.83049123)
}
}
{
\newrgbcolor{curcolor}{0 0 0}
\pscustom[linestyle=none,fillstyle=solid,fillcolor=curcolor]
{
\newpath
\moveto(941.28894775,54.28510061)
\curveto(941.28895845,54.41509899)(941.28895845,54.55009886)(941.28894775,54.69010061)
\curveto(941.28895845,54.84009857)(941.32395841,54.95009846)(941.39394775,55.02010061)
\curveto(941.46395827,55.07009834)(941.55895818,55.09509831)(941.67894775,55.09510061)
\curveto(941.78895795,55.1050983)(941.90395783,55.1100983)(942.02394775,55.11010061)
\lineto(943.35894775,55.11010061)
\lineto(949.43394775,55.11010061)
\lineto(951.11394775,55.11010061)
\lineto(951.50394775,55.11010061)
\curveto(951.64394809,55.1100983)(951.75394798,55.08509832)(951.83394775,55.03510061)
\curveto(951.88394785,55.0050984)(951.91394782,54.96009845)(951.92394775,54.90010061)
\curveto(951.9339478,54.85009856)(951.94894779,54.78509862)(951.96894775,54.70510061)
\lineto(951.96894775,54.49510061)
\lineto(951.96894775,54.18010061)
\curveto(951.95894778,54.08009933)(951.92394781,54.0050994)(951.86394775,53.95510061)
\curveto(951.78394795,53.9050995)(951.68394805,53.87509953)(951.56394775,53.86510061)
\lineto(951.18894775,53.86510061)
\lineto(949.80894775,53.86510061)
\lineto(943.56894775,53.86510061)
\lineto(942.09894775,53.86510061)
\curveto(941.98895775,53.86509954)(941.87395786,53.86009955)(941.75394775,53.85010061)
\curveto(941.62395811,53.85009956)(941.52395821,53.87509953)(941.45394775,53.92510061)
\curveto(941.39395834,53.96509944)(941.34395839,54.04009937)(941.30394775,54.15010061)
\curveto(941.29395844,54.17009924)(941.29395844,54.19009922)(941.30394775,54.21010061)
\curveto(941.30395843,54.24009917)(941.29895844,54.26509914)(941.28894775,54.28510061)
}
}
{
\newrgbcolor{curcolor}{0 0 0}
\pscustom[linestyle=none,fillstyle=solid,fillcolor=curcolor]
{
}
}
{
\newrgbcolor{curcolor}{0 0 0}
\pscustom[linestyle=none,fillstyle=solid,fillcolor=curcolor]
{
\newpath
\moveto(941.36394775,64.24510061)
\curveto(941.35395838,64.93509597)(941.47395826,65.53509537)(941.72394775,66.04510061)
\curveto(941.97395776,66.56509434)(942.30895743,66.96009395)(942.72894775,67.23010061)
\curveto(942.80895693,67.28009363)(942.89895684,67.32509358)(942.99894775,67.36510061)
\curveto(943.08895665,67.4050935)(943.18395655,67.45009346)(943.28394775,67.50010061)
\curveto(943.38395635,67.54009337)(943.48395625,67.57009334)(943.58394775,67.59010061)
\curveto(943.68395605,67.6100933)(943.78895595,67.63009328)(943.89894775,67.65010061)
\curveto(943.94895579,67.67009324)(943.99395574,67.67509323)(944.03394775,67.66510061)
\curveto(944.07395566,67.65509325)(944.11895562,67.66009325)(944.16894775,67.68010061)
\curveto(944.21895552,67.69009322)(944.30395543,67.69509321)(944.42394775,67.69510061)
\curveto(944.5339552,67.69509321)(944.61895512,67.69009322)(944.67894775,67.68010061)
\curveto(944.738955,67.66009325)(944.79895494,67.65009326)(944.85894775,67.65010061)
\curveto(944.91895482,67.66009325)(944.97895476,67.65509325)(945.03894775,67.63510061)
\curveto(945.17895456,67.59509331)(945.31395442,67.56009335)(945.44394775,67.53010061)
\curveto(945.57395416,67.50009341)(945.69895404,67.46009345)(945.81894775,67.41010061)
\curveto(945.95895378,67.35009356)(946.08395365,67.28009363)(946.19394775,67.20010061)
\curveto(946.30395343,67.13009378)(946.41395332,67.05509385)(946.52394775,66.97510061)
\lineto(946.58394775,66.91510061)
\curveto(946.60395313,66.905094)(946.62395311,66.89009402)(946.64394775,66.87010061)
\curveto(946.80395293,66.75009416)(946.94895279,66.61509429)(947.07894775,66.46510061)
\curveto(947.20895253,66.31509459)(947.3339524,66.15509475)(947.45394775,65.98510061)
\curveto(947.67395206,65.67509523)(947.87895186,65.38009553)(948.06894775,65.10010061)
\curveto(948.20895153,64.87009604)(948.34395139,64.64009627)(948.47394775,64.41010061)
\curveto(948.60395113,64.19009672)(948.738951,63.97009694)(948.87894775,63.75010061)
\curveto(949.04895069,63.50009741)(949.22895051,63.26009765)(949.41894775,63.03010061)
\curveto(949.60895013,62.8100981)(949.8339499,62.62009829)(950.09394775,62.46010061)
\curveto(950.15394958,62.42009849)(950.21394952,62.38509852)(950.27394775,62.35510061)
\curveto(950.32394941,62.32509858)(950.38894935,62.29509861)(950.46894775,62.26510061)
\curveto(950.5389492,62.24509866)(950.59894914,62.24009867)(950.64894775,62.25010061)
\curveto(950.71894902,62.27009864)(950.77394896,62.3050986)(950.81394775,62.35510061)
\curveto(950.84394889,62.4050985)(950.86394887,62.46509844)(950.87394775,62.53510061)
\lineto(950.87394775,62.77510061)
\lineto(950.87394775,63.52510061)
\lineto(950.87394775,66.33010061)
\lineto(950.87394775,66.99010061)
\curveto(950.87394886,67.08009383)(950.87894886,67.16509374)(950.88894775,67.24510061)
\curveto(950.88894885,67.32509358)(950.90894883,67.39009352)(950.94894775,67.44010061)
\curveto(950.98894875,67.49009342)(951.06394867,67.53009338)(951.17394775,67.56010061)
\curveto(951.27394846,67.60009331)(951.37394836,67.6100933)(951.47394775,67.59010061)
\lineto(951.60894775,67.59010061)
\curveto(951.67894806,67.57009334)(951.738948,67.55009336)(951.78894775,67.53010061)
\curveto(951.8389479,67.5100934)(951.87894786,67.47509343)(951.90894775,67.42510061)
\curveto(951.94894779,67.37509353)(951.96894777,67.3050936)(951.96894775,67.21510061)
\lineto(951.96894775,66.94510061)
\lineto(951.96894775,66.04510061)
\lineto(951.96894775,62.53510061)
\lineto(951.96894775,61.47010061)
\curveto(951.96894777,61.39009952)(951.97394776,61.30009961)(951.98394775,61.20010061)
\curveto(951.98394775,61.10009981)(951.97394776,61.01509989)(951.95394775,60.94510061)
\curveto(951.88394785,60.73510017)(951.70394803,60.67010024)(951.41394775,60.75010061)
\curveto(951.37394836,60.76010015)(951.3389484,60.76010015)(951.30894775,60.75010061)
\curveto(951.26894847,60.75010016)(951.22394851,60.76010015)(951.17394775,60.78010061)
\curveto(951.09394864,60.80010011)(951.00894873,60.82010009)(950.91894775,60.84010061)
\curveto(950.82894891,60.86010005)(950.74394899,60.88510002)(950.66394775,60.91510061)
\curveto(950.17394956,61.07509983)(949.75894998,61.27509963)(949.41894775,61.51510061)
\curveto(949.16895057,61.69509921)(948.94395079,61.90009901)(948.74394775,62.13010061)
\curveto(948.5339512,62.36009855)(948.3389514,62.60009831)(948.15894775,62.85010061)
\curveto(947.97895176,63.1100978)(947.80895193,63.37509753)(947.64894775,63.64510061)
\curveto(947.47895226,63.92509698)(947.30395243,64.19509671)(947.12394775,64.45510061)
\curveto(947.04395269,64.56509634)(946.96895277,64.67009624)(946.89894775,64.77010061)
\curveto(946.82895291,64.88009603)(946.75395298,64.99009592)(946.67394775,65.10010061)
\curveto(946.64395309,65.14009577)(946.61395312,65.17509573)(946.58394775,65.20510061)
\curveto(946.54395319,65.24509566)(946.51395322,65.28509562)(946.49394775,65.32510061)
\curveto(946.38395335,65.46509544)(946.25895348,65.59009532)(946.11894775,65.70010061)
\curveto(946.08895365,65.72009519)(946.06395367,65.74509516)(946.04394775,65.77510061)
\curveto(946.01395372,65.8050951)(945.98395375,65.83009508)(945.95394775,65.85010061)
\curveto(945.85395388,65.93009498)(945.75395398,65.99509491)(945.65394775,66.04510061)
\curveto(945.55395418,66.1050948)(945.44395429,66.16009475)(945.32394775,66.21010061)
\curveto(945.25395448,66.24009467)(945.17895456,66.26009465)(945.09894775,66.27010061)
\lineto(944.85894775,66.33010061)
\lineto(944.76894775,66.33010061)
\curveto(944.738955,66.34009457)(944.70895503,66.34509456)(944.67894775,66.34510061)
\curveto(944.60895513,66.36509454)(944.51395522,66.37009454)(944.39394775,66.36010061)
\curveto(944.26395547,66.36009455)(944.16395557,66.35009456)(944.09394775,66.33010061)
\curveto(944.01395572,66.3100946)(943.9389558,66.29009462)(943.86894775,66.27010061)
\curveto(943.78895595,66.26009465)(943.70895603,66.24009467)(943.62894775,66.21010061)
\curveto(943.38895635,66.10009481)(943.18895655,65.95009496)(943.02894775,65.76010061)
\curveto(942.85895688,65.58009533)(942.71895702,65.36009555)(942.60894775,65.10010061)
\curveto(942.58895715,65.03009588)(942.57395716,64.96009595)(942.56394775,64.89010061)
\curveto(942.54395719,64.82009609)(942.52395721,64.74509616)(942.50394775,64.66510061)
\curveto(942.48395725,64.58509632)(942.47395726,64.47509643)(942.47394775,64.33510061)
\curveto(942.47395726,64.2050967)(942.48395725,64.10009681)(942.50394775,64.02010061)
\curveto(942.51395722,63.96009695)(942.51895722,63.905097)(942.51894775,63.85510061)
\curveto(942.51895722,63.8050971)(942.52895721,63.75509715)(942.54894775,63.70510061)
\curveto(942.58895715,63.6050973)(942.62895711,63.5100974)(942.66894775,63.42010061)
\curveto(942.70895703,63.34009757)(942.75395698,63.26009765)(942.80394775,63.18010061)
\curveto(942.82395691,63.15009776)(942.84895689,63.12009779)(942.87894775,63.09010061)
\curveto(942.90895683,63.07009784)(942.9339568,63.04509786)(942.95394775,63.01510061)
\lineto(943.02894775,62.94010061)
\curveto(943.04895669,62.910098)(943.06895667,62.88509802)(943.08894775,62.86510061)
\lineto(943.29894775,62.71510061)
\curveto(943.35895638,62.67509823)(943.42395631,62.63009828)(943.49394775,62.58010061)
\curveto(943.58395615,62.52009839)(943.68895605,62.47009844)(943.80894775,62.43010061)
\curveto(943.91895582,62.40009851)(944.02895571,62.36509854)(944.13894775,62.32510061)
\curveto(944.24895549,62.28509862)(944.39395534,62.26009865)(944.57394775,62.25010061)
\curveto(944.74395499,62.24009867)(944.86895487,62.2100987)(944.94894775,62.16010061)
\curveto(945.02895471,62.1100988)(945.07395466,62.03509887)(945.08394775,61.93510061)
\curveto(945.09395464,61.83509907)(945.09895464,61.72509918)(945.09894775,61.60510061)
\curveto(945.09895464,61.56509934)(945.10395463,61.52509938)(945.11394775,61.48510061)
\curveto(945.11395462,61.44509946)(945.10895463,61.4100995)(945.09894775,61.38010061)
\curveto(945.07895466,61.33009958)(945.06895467,61.28009963)(945.06894775,61.23010061)
\curveto(945.06895467,61.19009972)(945.05895468,61.15009976)(945.03894775,61.11010061)
\curveto(944.97895476,61.02009989)(944.84395489,60.97509993)(944.63394775,60.97510061)
\lineto(944.51394775,60.97510061)
\curveto(944.45395528,60.98509992)(944.39395534,60.99009992)(944.33394775,60.99010061)
\curveto(944.26395547,61.00009991)(944.19895554,61.0100999)(944.13894775,61.02010061)
\curveto(944.02895571,61.04009987)(943.92895581,61.06009985)(943.83894775,61.08010061)
\curveto(943.738956,61.10009981)(943.64395609,61.13009978)(943.55394775,61.17010061)
\curveto(943.48395625,61.19009972)(943.42395631,61.2100997)(943.37394775,61.23010061)
\lineto(943.19394775,61.29010061)
\curveto(942.9339568,61.4100995)(942.68895705,61.56509934)(942.45894775,61.75510061)
\curveto(942.22895751,61.95509895)(942.04395769,62.17009874)(941.90394775,62.40010061)
\curveto(941.82395791,62.5100984)(941.75895798,62.62509828)(941.70894775,62.74510061)
\lineto(941.55894775,63.13510061)
\curveto(941.50895823,63.24509766)(941.47895826,63.36009755)(941.46894775,63.48010061)
\curveto(941.44895829,63.60009731)(941.42395831,63.72509718)(941.39394775,63.85510061)
\curveto(941.39395834,63.92509698)(941.39395834,63.99009692)(941.39394775,64.05010061)
\curveto(941.38395835,64.1100968)(941.37395836,64.17509673)(941.36394775,64.24510061)
}
}
{
\newrgbcolor{curcolor}{0 0 0}
\pscustom[linestyle=none,fillstyle=solid,fillcolor=curcolor]
{
\newpath
\moveto(948.89394775,76.35970998)
\curveto(948.9339508,76.36970226)(948.98395075,76.36970226)(949.04394775,76.35970998)
\curveto(949.10395063,76.35970227)(949.15395058,76.35470228)(949.19394775,76.34470998)
\curveto(949.2339505,76.34470229)(949.27395046,76.33970229)(949.31394775,76.32970998)
\lineto(949.41894775,76.32970998)
\curveto(949.49895024,76.30970232)(949.57895016,76.29470234)(949.65894775,76.28470998)
\curveto(949.73895,76.27470236)(949.81394992,76.25470238)(949.88394775,76.22470998)
\curveto(949.96394977,76.20470243)(950.0389497,76.18470245)(950.10894775,76.16470998)
\curveto(950.17894956,76.14470249)(950.25394948,76.11470252)(950.33394775,76.07470998)
\curveto(950.75394898,75.89470274)(951.09394864,75.63970299)(951.35394775,75.30970998)
\curveto(951.61394812,74.97970365)(951.81894792,74.58970404)(951.96894775,74.13970998)
\curveto(952.00894773,74.01970461)(952.0339477,73.89470474)(952.04394775,73.76470998)
\curveto(952.06394767,73.64470499)(952.08894765,73.51970511)(952.11894775,73.38970998)
\curveto(952.12894761,73.3297053)(952.1339476,73.26470537)(952.13394775,73.19470998)
\curveto(952.1339476,73.1347055)(952.1389476,73.06970556)(952.14894775,72.99970998)
\lineto(952.14894775,72.87970998)
\lineto(952.14894775,72.68470998)
\curveto(952.15894758,72.62470601)(952.15394758,72.56970606)(952.13394775,72.51970998)
\curveto(952.11394762,72.44970618)(952.10894763,72.38470625)(952.11894775,72.32470998)
\curveto(952.12894761,72.26470637)(952.12394761,72.20470643)(952.10394775,72.14470998)
\curveto(952.09394764,72.09470654)(952.08894765,72.04970658)(952.08894775,72.00970998)
\curveto(952.08894765,71.96970666)(952.07894766,71.92470671)(952.05894775,71.87470998)
\curveto(952.0389477,71.79470684)(952.01894772,71.71970691)(951.99894775,71.64970998)
\curveto(951.98894775,71.57970705)(951.97394776,71.50970712)(951.95394775,71.43970998)
\curveto(951.78394795,70.95970767)(951.57394816,70.55970807)(951.32394775,70.23970998)
\curveto(951.06394867,69.9297087)(950.70894903,69.67970895)(950.25894775,69.48970998)
\curveto(950.19894954,69.45970917)(950.1389496,69.4347092)(950.07894775,69.41470998)
\curveto(950.00894973,69.40470923)(949.9339498,69.38970924)(949.85394775,69.36970998)
\curveto(949.79394994,69.34970928)(949.72895001,69.3347093)(949.65894775,69.32470998)
\curveto(949.58895015,69.31470932)(949.51895022,69.29970933)(949.44894775,69.27970998)
\curveto(949.39895034,69.26970936)(949.35895038,69.26470937)(949.32894775,69.26470998)
\lineto(949.20894775,69.26470998)
\curveto(949.16895057,69.25470938)(949.11895062,69.24470939)(949.05894775,69.23470998)
\curveto(948.99895074,69.2347094)(948.94895079,69.23970939)(948.90894775,69.24970998)
\lineto(948.77394775,69.24970998)
\curveto(948.72395101,69.25970937)(948.67395106,69.26470937)(948.62394775,69.26470998)
\curveto(948.52395121,69.28470935)(948.42895131,69.29970933)(948.33894775,69.30970998)
\curveto(948.2389515,69.31970931)(948.14395159,69.33970929)(948.05394775,69.36970998)
\curveto(947.90395183,69.41970921)(947.76395197,69.47470916)(947.63394775,69.53470998)
\curveto(947.50395223,69.59470904)(947.38395235,69.66470897)(947.27394775,69.74470998)
\curveto(947.22395251,69.77470886)(947.18395255,69.80470883)(947.15394775,69.83470998)
\curveto(947.12395261,69.87470876)(947.08895265,69.90970872)(947.04894775,69.93970998)
\curveto(946.96895277,69.99970863)(946.89895284,70.06970856)(946.83894775,70.14970998)
\curveto(946.78895295,70.20970842)(946.74395299,70.26970836)(946.70394775,70.32970998)
\lineto(946.55394775,70.53970998)
\curveto(946.51395322,70.58970804)(946.47895326,70.63970799)(946.44894775,70.68970998)
\curveto(946.40895333,70.73970789)(946.35395338,70.77470786)(946.28394775,70.79470998)
\curveto(946.25395348,70.79470784)(946.22895351,70.78470785)(946.20894775,70.76470998)
\curveto(946.17895356,70.75470788)(946.15395358,70.74470789)(946.13394775,70.73470998)
\curveto(946.08395365,70.69470794)(946.0389537,70.64470799)(945.99894775,70.58470998)
\curveto(945.94895379,70.5347081)(945.90395383,70.48470815)(945.86394775,70.43470998)
\curveto(945.8339539,70.39470824)(945.77895396,70.34470829)(945.69894775,70.28470998)
\curveto(945.66895407,70.26470837)(945.64395409,70.2347084)(945.62394775,70.19470998)
\curveto(945.59395414,70.16470847)(945.55895418,70.13970849)(945.51894775,70.11970998)
\curveto(945.30895443,69.94970868)(945.06395467,69.81970881)(944.78394775,69.72970998)
\curveto(944.70395503,69.70970892)(944.62395511,69.69470894)(944.54394775,69.68470998)
\curveto(944.46395527,69.67470896)(944.38395535,69.65970897)(944.30394775,69.63970998)
\curveto(944.25395548,69.61970901)(944.18895555,69.60970902)(944.10894775,69.60970998)
\curveto(944.01895572,69.60970902)(943.94895579,69.61970901)(943.89894775,69.63970998)
\curveto(943.79895594,69.63970899)(943.72895601,69.64470899)(943.68894775,69.65470998)
\curveto(943.60895613,69.67470896)(943.5389562,69.68970894)(943.47894775,69.69970998)
\curveto(943.40895633,69.70970892)(943.3389564,69.72470891)(943.26894775,69.74470998)
\curveto(942.8389569,69.89470874)(942.49395724,70.10970852)(942.23394775,70.38970998)
\curveto(941.97395776,70.67970795)(941.75895798,71.0297076)(941.58894775,71.43970998)
\curveto(941.5389582,71.54970708)(941.50895823,71.66470697)(941.49894775,71.78470998)
\curveto(941.47895826,71.91470672)(941.44895829,72.04470659)(941.40894775,72.17470998)
\curveto(941.40895833,72.25470638)(941.40895833,72.32470631)(941.40894775,72.38470998)
\curveto(941.39895834,72.45470618)(941.38895835,72.5297061)(941.37894775,72.60970998)
\curveto(941.35895838,73.39970523)(941.48895825,74.05470458)(941.76894775,74.57470998)
\curveto(942.04895769,75.10470353)(942.45895728,75.48470315)(942.99894775,75.71470998)
\curveto(943.22895651,75.82470281)(943.51395622,75.89470274)(943.85394775,75.92470998)
\curveto(944.18395555,75.96470267)(944.48895525,75.9347027)(944.76894775,75.83470998)
\curveto(944.89895484,75.79470284)(945.01895472,75.74470289)(945.12894775,75.68470998)
\curveto(945.2389545,75.634703)(945.34395439,75.57470306)(945.44394775,75.50470998)
\curveto(945.48395425,75.48470315)(945.51895422,75.45470318)(945.54894775,75.41470998)
\lineto(945.63894775,75.32470998)
\curveto(945.72895401,75.27470336)(945.79395394,75.21470342)(945.83394775,75.14470998)
\curveto(945.88395385,75.09470354)(945.9339538,75.03970359)(945.98394775,74.97970998)
\curveto(946.02395371,74.9297037)(946.06895367,74.88470375)(946.11894775,74.84470998)
\curveto(946.1389536,74.82470381)(946.16395357,74.80470383)(946.19394775,74.78470998)
\curveto(946.21395352,74.77470386)(946.2389535,74.77470386)(946.26894775,74.78470998)
\curveto(946.31895342,74.79470384)(946.36895337,74.82470381)(946.41894775,74.87470998)
\curveto(946.45895328,74.92470371)(946.49895324,74.97970365)(946.53894775,75.03970998)
\lineto(946.65894775,75.21970998)
\curveto(946.68895305,75.27970335)(946.71895302,75.3297033)(946.74894775,75.36970998)
\curveto(946.98895275,75.69970293)(947.29895244,75.94970268)(947.67894775,76.11970998)
\curveto(947.75895198,76.15970247)(947.84395189,76.18970244)(947.93394775,76.20970998)
\curveto(948.02395171,76.23970239)(948.11395162,76.26470237)(948.20394775,76.28470998)
\curveto(948.25395148,76.29470234)(948.30895143,76.30470233)(948.36894775,76.31470998)
\lineto(948.51894775,76.34470998)
\curveto(948.57895116,76.35470228)(948.64395109,76.35470228)(948.71394775,76.34470998)
\curveto(948.77395096,76.3347023)(948.8339509,76.33970229)(948.89394775,76.35970998)
\moveto(943.85394775,70.97470998)
\curveto(943.96395577,70.94470769)(944.10395563,70.93970769)(944.27394775,70.95970998)
\curveto(944.4339553,70.97970765)(944.55895518,71.00470763)(944.64894775,71.03470998)
\curveto(944.96895477,71.14470749)(945.21395452,71.29470734)(945.38394775,71.48470998)
\curveto(945.54395419,71.67470696)(945.67395406,71.93970669)(945.77394775,72.27970998)
\curveto(945.80395393,72.40970622)(945.82895391,72.57470606)(945.84894775,72.77470998)
\curveto(945.85895388,72.97470566)(945.84395389,73.14470549)(945.80394775,73.28470998)
\curveto(945.72395401,73.57470506)(945.61395412,73.81470482)(945.47394775,74.00470998)
\curveto(945.32395441,74.20470443)(945.12395461,74.35970427)(944.87394775,74.46970998)
\curveto(944.82395491,74.48970414)(944.77895496,74.49970413)(944.73894775,74.49970998)
\curveto(944.69895504,74.50970412)(944.65395508,74.52470411)(944.60394775,74.54470998)
\curveto(944.49395524,74.57470406)(944.35395538,74.59470404)(944.18394775,74.60470998)
\curveto(944.01395572,74.61470402)(943.86895587,74.60470403)(943.74894775,74.57470998)
\curveto(943.65895608,74.55470408)(943.57395616,74.5297041)(943.49394775,74.49970998)
\curveto(943.41395632,74.47970415)(943.3339564,74.44470419)(943.25394775,74.39470998)
\curveto(942.98395675,74.22470441)(942.78895695,73.99970463)(942.66894775,73.71970998)
\curveto(942.54895719,73.44970518)(942.48895725,73.08970554)(942.48894775,72.63970998)
\curveto(942.50895723,72.61970601)(942.51395722,72.58970604)(942.50394775,72.54970998)
\curveto(942.49395724,72.50970612)(942.49395724,72.47470616)(942.50394775,72.44470998)
\curveto(942.52395721,72.39470624)(942.5389572,72.33970629)(942.54894775,72.27970998)
\curveto(942.54895719,72.2297064)(942.55895718,72.17970645)(942.57894775,72.12970998)
\curveto(942.66895707,71.88970674)(942.78395695,71.67970695)(942.92394775,71.49970998)
\curveto(943.05395668,71.31970731)(943.2339565,71.17970745)(943.46394775,71.07970998)
\curveto(943.52395621,71.05970757)(943.58895615,71.03970759)(943.65894775,71.01970998)
\curveto(943.71895602,71.00970762)(943.78395595,70.99470764)(943.85394775,70.97470998)
\moveto(949.38894775,74.99470998)
\curveto(949.19895054,75.04470359)(948.99395074,75.04970358)(948.77394775,75.00970998)
\curveto(948.55395118,74.97970365)(948.37395136,74.9347037)(948.23394775,74.87470998)
\curveto(947.86395187,74.70470393)(947.55895218,74.44470419)(947.31894775,74.09470998)
\curveto(947.07895266,73.75470488)(946.95895278,73.31970531)(946.95894775,72.78970998)
\curveto(946.97895276,72.75970587)(946.98395275,72.71970591)(946.97394775,72.66970998)
\curveto(946.95395278,72.61970601)(946.94895279,72.57970605)(946.95894775,72.54970998)
\lineto(947.01894775,72.27970998)
\curveto(947.02895271,72.19970643)(947.04395269,72.11970651)(947.06394775,72.03970998)
\curveto(947.17395256,71.73970689)(947.31895242,71.47470716)(947.49894775,71.24470998)
\curveto(947.67895206,71.02470761)(947.90895183,70.85470778)(948.18894775,70.73470998)
\curveto(948.26895147,70.70470793)(948.34895139,70.67970795)(948.42894775,70.65970998)
\curveto(948.50895123,70.63970799)(948.59395114,70.61970801)(948.68394775,70.59970998)
\curveto(948.80395093,70.56970806)(948.95395078,70.55970807)(949.13394775,70.56970998)
\curveto(949.31395042,70.58970804)(949.45395028,70.61470802)(949.55394775,70.64470998)
\curveto(949.60395013,70.66470797)(949.64895009,70.67470796)(949.68894775,70.67470998)
\curveto(949.71895002,70.68470795)(949.75894998,70.69970793)(949.80894775,70.71970998)
\curveto(950.02894971,70.81970781)(950.22894951,70.94970768)(950.40894775,71.10970998)
\curveto(950.58894915,71.27970735)(950.72394901,71.47470716)(950.81394775,71.69470998)
\curveto(950.85394888,71.76470687)(950.88894885,71.85970677)(950.91894775,71.97970998)
\curveto(951.00894873,72.19970643)(951.05394868,72.45470618)(951.05394775,72.74470998)
\lineto(951.05394775,73.02970998)
\curveto(951.0339487,73.1297055)(951.01894872,73.22470541)(951.00894775,73.31470998)
\curveto(950.99894874,73.40470523)(950.97894876,73.49470514)(950.94894775,73.58470998)
\curveto(950.86894887,73.84470479)(950.738949,74.08470455)(950.55894775,74.30470998)
\curveto(950.36894937,74.5347041)(950.15394958,74.70470393)(949.91394775,74.81470998)
\curveto(949.8339499,74.85470378)(949.75394998,74.88470375)(949.67394775,74.90470998)
\curveto(949.58395015,74.9347037)(949.48895025,74.96470367)(949.38894775,74.99470998)
}
}
{
\newrgbcolor{curcolor}{0 0 0}
\pscustom[linestyle=none,fillstyle=solid,fillcolor=curcolor]
{
\newpath
\moveto(950.33394775,78.63431936)
\lineto(950.33394775,79.26431936)
\lineto(950.33394775,79.45931936)
\curveto(950.3339494,79.52931683)(950.34394939,79.58931677)(950.36394775,79.63931936)
\curveto(950.40394933,79.70931665)(950.44394929,79.7593166)(950.48394775,79.78931936)
\curveto(950.5339492,79.82931653)(950.59894914,79.84931651)(950.67894775,79.84931936)
\curveto(950.75894898,79.8593165)(950.84394889,79.86431649)(950.93394775,79.86431936)
\lineto(951.65394775,79.86431936)
\curveto(952.1339476,79.86431649)(952.54394719,79.80431655)(952.88394775,79.68431936)
\curveto(953.22394651,79.56431679)(953.49894624,79.36931699)(953.70894775,79.09931936)
\curveto(953.75894598,79.02931733)(953.80394593,78.9593174)(953.84394775,78.88931936)
\curveto(953.89394584,78.82931753)(953.9389458,78.7543176)(953.97894775,78.66431936)
\curveto(953.98894575,78.64431771)(953.99894574,78.61431774)(954.00894775,78.57431936)
\curveto(954.02894571,78.53431782)(954.0339457,78.48931787)(954.02394775,78.43931936)
\curveto(953.99394574,78.34931801)(953.91894582,78.29431806)(953.79894775,78.27431936)
\curveto(953.68894605,78.2543181)(953.59394614,78.26931809)(953.51394775,78.31931936)
\curveto(953.44394629,78.34931801)(953.37894636,78.39431796)(953.31894775,78.45431936)
\curveto(953.26894647,78.52431783)(953.21894652,78.58931777)(953.16894775,78.64931936)
\curveto(953.11894662,78.71931764)(953.04394669,78.77931758)(952.94394775,78.82931936)
\curveto(952.85394688,78.88931747)(952.76394697,78.93931742)(952.67394775,78.97931936)
\curveto(952.64394709,78.99931736)(952.58394715,79.02431733)(952.49394775,79.05431936)
\curveto(952.41394732,79.08431727)(952.34394739,79.08931727)(952.28394775,79.06931936)
\curveto(952.14394759,79.03931732)(952.05394768,78.97931738)(952.01394775,78.88931936)
\curveto(951.98394775,78.80931755)(951.96894777,78.71931764)(951.96894775,78.61931936)
\curveto(951.96894777,78.51931784)(951.94394779,78.43431792)(951.89394775,78.36431936)
\curveto(951.82394791,78.27431808)(951.68394805,78.22931813)(951.47394775,78.22931936)
\lineto(950.91894775,78.22931936)
\lineto(950.69394775,78.22931936)
\curveto(950.61394912,78.23931812)(950.54894919,78.2593181)(950.49894775,78.28931936)
\curveto(950.41894932,78.34931801)(950.37394936,78.41931794)(950.36394775,78.49931936)
\curveto(950.35394938,78.51931784)(950.34894939,78.53931782)(950.34894775,78.55931936)
\curveto(950.34894939,78.58931777)(950.34394939,78.61431774)(950.33394775,78.63431936)
}
}
{
\newrgbcolor{curcolor}{0 0 0}
\pscustom[linestyle=none,fillstyle=solid,fillcolor=curcolor]
{
}
}
{
\newrgbcolor{curcolor}{0 0 0}
\pscustom[linestyle=none,fillstyle=solid,fillcolor=curcolor]
{
\newpath
\moveto(941.36394775,89.26463186)
\curveto(941.35395838,89.95462722)(941.47395826,90.55462662)(941.72394775,91.06463186)
\curveto(941.97395776,91.58462559)(942.30895743,91.9796252)(942.72894775,92.24963186)
\curveto(942.80895693,92.29962488)(942.89895684,92.34462483)(942.99894775,92.38463186)
\curveto(943.08895665,92.42462475)(943.18395655,92.46962471)(943.28394775,92.51963186)
\curveto(943.38395635,92.55962462)(943.48395625,92.58962459)(943.58394775,92.60963186)
\curveto(943.68395605,92.62962455)(943.78895595,92.64962453)(943.89894775,92.66963186)
\curveto(943.94895579,92.68962449)(943.99395574,92.69462448)(944.03394775,92.68463186)
\curveto(944.07395566,92.6746245)(944.11895562,92.6796245)(944.16894775,92.69963186)
\curveto(944.21895552,92.70962447)(944.30395543,92.71462446)(944.42394775,92.71463186)
\curveto(944.5339552,92.71462446)(944.61895512,92.70962447)(944.67894775,92.69963186)
\curveto(944.738955,92.6796245)(944.79895494,92.66962451)(944.85894775,92.66963186)
\curveto(944.91895482,92.6796245)(944.97895476,92.6746245)(945.03894775,92.65463186)
\curveto(945.17895456,92.61462456)(945.31395442,92.5796246)(945.44394775,92.54963186)
\curveto(945.57395416,92.51962466)(945.69895404,92.4796247)(945.81894775,92.42963186)
\curveto(945.95895378,92.36962481)(946.08395365,92.29962488)(946.19394775,92.21963186)
\curveto(946.30395343,92.14962503)(946.41395332,92.0746251)(946.52394775,91.99463186)
\lineto(946.58394775,91.93463186)
\curveto(946.60395313,91.92462525)(946.62395311,91.90962527)(946.64394775,91.88963186)
\curveto(946.80395293,91.76962541)(946.94895279,91.63462554)(947.07894775,91.48463186)
\curveto(947.20895253,91.33462584)(947.3339524,91.174626)(947.45394775,91.00463186)
\curveto(947.67395206,90.69462648)(947.87895186,90.39962678)(948.06894775,90.11963186)
\curveto(948.20895153,89.88962729)(948.34395139,89.65962752)(948.47394775,89.42963186)
\curveto(948.60395113,89.20962797)(948.738951,88.98962819)(948.87894775,88.76963186)
\curveto(949.04895069,88.51962866)(949.22895051,88.2796289)(949.41894775,88.04963186)
\curveto(949.60895013,87.82962935)(949.8339499,87.63962954)(950.09394775,87.47963186)
\curveto(950.15394958,87.43962974)(950.21394952,87.40462977)(950.27394775,87.37463186)
\curveto(950.32394941,87.34462983)(950.38894935,87.31462986)(950.46894775,87.28463186)
\curveto(950.5389492,87.26462991)(950.59894914,87.25962992)(950.64894775,87.26963186)
\curveto(950.71894902,87.28962989)(950.77394896,87.32462985)(950.81394775,87.37463186)
\curveto(950.84394889,87.42462975)(950.86394887,87.48462969)(950.87394775,87.55463186)
\lineto(950.87394775,87.79463186)
\lineto(950.87394775,88.54463186)
\lineto(950.87394775,91.34963186)
\lineto(950.87394775,92.00963186)
\curveto(950.87394886,92.09962508)(950.87894886,92.18462499)(950.88894775,92.26463186)
\curveto(950.88894885,92.34462483)(950.90894883,92.40962477)(950.94894775,92.45963186)
\curveto(950.98894875,92.50962467)(951.06394867,92.54962463)(951.17394775,92.57963186)
\curveto(951.27394846,92.61962456)(951.37394836,92.62962455)(951.47394775,92.60963186)
\lineto(951.60894775,92.60963186)
\curveto(951.67894806,92.58962459)(951.738948,92.56962461)(951.78894775,92.54963186)
\curveto(951.8389479,92.52962465)(951.87894786,92.49462468)(951.90894775,92.44463186)
\curveto(951.94894779,92.39462478)(951.96894777,92.32462485)(951.96894775,92.23463186)
\lineto(951.96894775,91.96463186)
\lineto(951.96894775,91.06463186)
\lineto(951.96894775,87.55463186)
\lineto(951.96894775,86.48963186)
\curveto(951.96894777,86.40963077)(951.97394776,86.31963086)(951.98394775,86.21963186)
\curveto(951.98394775,86.11963106)(951.97394776,86.03463114)(951.95394775,85.96463186)
\curveto(951.88394785,85.75463142)(951.70394803,85.68963149)(951.41394775,85.76963186)
\curveto(951.37394836,85.7796314)(951.3389484,85.7796314)(951.30894775,85.76963186)
\curveto(951.26894847,85.76963141)(951.22394851,85.7796314)(951.17394775,85.79963186)
\curveto(951.09394864,85.81963136)(951.00894873,85.83963134)(950.91894775,85.85963186)
\curveto(950.82894891,85.8796313)(950.74394899,85.90463127)(950.66394775,85.93463186)
\curveto(950.17394956,86.09463108)(949.75894998,86.29463088)(949.41894775,86.53463186)
\curveto(949.16895057,86.71463046)(948.94395079,86.91963026)(948.74394775,87.14963186)
\curveto(948.5339512,87.3796298)(948.3389514,87.61962956)(948.15894775,87.86963186)
\curveto(947.97895176,88.12962905)(947.80895193,88.39462878)(947.64894775,88.66463186)
\curveto(947.47895226,88.94462823)(947.30395243,89.21462796)(947.12394775,89.47463186)
\curveto(947.04395269,89.58462759)(946.96895277,89.68962749)(946.89894775,89.78963186)
\curveto(946.82895291,89.89962728)(946.75395298,90.00962717)(946.67394775,90.11963186)
\curveto(946.64395309,90.15962702)(946.61395312,90.19462698)(946.58394775,90.22463186)
\curveto(946.54395319,90.26462691)(946.51395322,90.30462687)(946.49394775,90.34463186)
\curveto(946.38395335,90.48462669)(946.25895348,90.60962657)(946.11894775,90.71963186)
\curveto(946.08895365,90.73962644)(946.06395367,90.76462641)(946.04394775,90.79463186)
\curveto(946.01395372,90.82462635)(945.98395375,90.84962633)(945.95394775,90.86963186)
\curveto(945.85395388,90.94962623)(945.75395398,91.01462616)(945.65394775,91.06463186)
\curveto(945.55395418,91.12462605)(945.44395429,91.179626)(945.32394775,91.22963186)
\curveto(945.25395448,91.25962592)(945.17895456,91.2796259)(945.09894775,91.28963186)
\lineto(944.85894775,91.34963186)
\lineto(944.76894775,91.34963186)
\curveto(944.738955,91.35962582)(944.70895503,91.36462581)(944.67894775,91.36463186)
\curveto(944.60895513,91.38462579)(944.51395522,91.38962579)(944.39394775,91.37963186)
\curveto(944.26395547,91.3796258)(944.16395557,91.36962581)(944.09394775,91.34963186)
\curveto(944.01395572,91.32962585)(943.9389558,91.30962587)(943.86894775,91.28963186)
\curveto(943.78895595,91.2796259)(943.70895603,91.25962592)(943.62894775,91.22963186)
\curveto(943.38895635,91.11962606)(943.18895655,90.96962621)(943.02894775,90.77963186)
\curveto(942.85895688,90.59962658)(942.71895702,90.3796268)(942.60894775,90.11963186)
\curveto(942.58895715,90.04962713)(942.57395716,89.9796272)(942.56394775,89.90963186)
\curveto(942.54395719,89.83962734)(942.52395721,89.76462741)(942.50394775,89.68463186)
\curveto(942.48395725,89.60462757)(942.47395726,89.49462768)(942.47394775,89.35463186)
\curveto(942.47395726,89.22462795)(942.48395725,89.11962806)(942.50394775,89.03963186)
\curveto(942.51395722,88.9796282)(942.51895722,88.92462825)(942.51894775,88.87463186)
\curveto(942.51895722,88.82462835)(942.52895721,88.7746284)(942.54894775,88.72463186)
\curveto(942.58895715,88.62462855)(942.62895711,88.52962865)(942.66894775,88.43963186)
\curveto(942.70895703,88.35962882)(942.75395698,88.2796289)(942.80394775,88.19963186)
\curveto(942.82395691,88.16962901)(942.84895689,88.13962904)(942.87894775,88.10963186)
\curveto(942.90895683,88.08962909)(942.9339568,88.06462911)(942.95394775,88.03463186)
\lineto(943.02894775,87.95963186)
\curveto(943.04895669,87.92962925)(943.06895667,87.90462927)(943.08894775,87.88463186)
\lineto(943.29894775,87.73463186)
\curveto(943.35895638,87.69462948)(943.42395631,87.64962953)(943.49394775,87.59963186)
\curveto(943.58395615,87.53962964)(943.68895605,87.48962969)(943.80894775,87.44963186)
\curveto(943.91895582,87.41962976)(944.02895571,87.38462979)(944.13894775,87.34463186)
\curveto(944.24895549,87.30462987)(944.39395534,87.2796299)(944.57394775,87.26963186)
\curveto(944.74395499,87.25962992)(944.86895487,87.22962995)(944.94894775,87.17963186)
\curveto(945.02895471,87.12963005)(945.07395466,87.05463012)(945.08394775,86.95463186)
\curveto(945.09395464,86.85463032)(945.09895464,86.74463043)(945.09894775,86.62463186)
\curveto(945.09895464,86.58463059)(945.10395463,86.54463063)(945.11394775,86.50463186)
\curveto(945.11395462,86.46463071)(945.10895463,86.42963075)(945.09894775,86.39963186)
\curveto(945.07895466,86.34963083)(945.06895467,86.29963088)(945.06894775,86.24963186)
\curveto(945.06895467,86.20963097)(945.05895468,86.16963101)(945.03894775,86.12963186)
\curveto(944.97895476,86.03963114)(944.84395489,85.99463118)(944.63394775,85.99463186)
\lineto(944.51394775,85.99463186)
\curveto(944.45395528,86.00463117)(944.39395534,86.00963117)(944.33394775,86.00963186)
\curveto(944.26395547,86.01963116)(944.19895554,86.02963115)(944.13894775,86.03963186)
\curveto(944.02895571,86.05963112)(943.92895581,86.0796311)(943.83894775,86.09963186)
\curveto(943.738956,86.11963106)(943.64395609,86.14963103)(943.55394775,86.18963186)
\curveto(943.48395625,86.20963097)(943.42395631,86.22963095)(943.37394775,86.24963186)
\lineto(943.19394775,86.30963186)
\curveto(942.9339568,86.42963075)(942.68895705,86.58463059)(942.45894775,86.77463186)
\curveto(942.22895751,86.9746302)(942.04395769,87.18962999)(941.90394775,87.41963186)
\curveto(941.82395791,87.52962965)(941.75895798,87.64462953)(941.70894775,87.76463186)
\lineto(941.55894775,88.15463186)
\curveto(941.50895823,88.26462891)(941.47895826,88.3796288)(941.46894775,88.49963186)
\curveto(941.44895829,88.61962856)(941.42395831,88.74462843)(941.39394775,88.87463186)
\curveto(941.39395834,88.94462823)(941.39395834,89.00962817)(941.39394775,89.06963186)
\curveto(941.38395835,89.12962805)(941.37395836,89.19462798)(941.36394775,89.26463186)
}
}
{
\newrgbcolor{curcolor}{0 0 0}
\pscustom[linestyle=none,fillstyle=solid,fillcolor=curcolor]
{
\newpath
\moveto(946.88394775,101.36424123)
\lineto(947.13894775,101.36424123)
\curveto(947.21895252,101.37423353)(947.29395244,101.36923353)(947.36394775,101.34924123)
\lineto(947.60394775,101.34924123)
\lineto(947.76894775,101.34924123)
\curveto(947.86895187,101.32923357)(947.97395176,101.31923358)(948.08394775,101.31924123)
\curveto(948.18395155,101.31923358)(948.28395145,101.30923359)(948.38394775,101.28924123)
\lineto(948.53394775,101.28924123)
\curveto(948.67395106,101.25923364)(948.81395092,101.23923366)(948.95394775,101.22924123)
\curveto(949.08395065,101.21923368)(949.21395052,101.19423371)(949.34394775,101.15424123)
\curveto(949.42395031,101.13423377)(949.50895023,101.11423379)(949.59894775,101.09424123)
\lineto(949.83894775,101.03424123)
\lineto(950.13894775,100.91424123)
\curveto(950.22894951,100.88423402)(950.31894942,100.84923405)(950.40894775,100.80924123)
\curveto(950.62894911,100.70923419)(950.84394889,100.57423433)(951.05394775,100.40424123)
\curveto(951.26394847,100.24423466)(951.4339483,100.06923483)(951.56394775,99.87924123)
\curveto(951.60394813,99.82923507)(951.64394809,99.76923513)(951.68394775,99.69924123)
\curveto(951.71394802,99.63923526)(951.74894799,99.57923532)(951.78894775,99.51924123)
\curveto(951.8389479,99.43923546)(951.87894786,99.34423556)(951.90894775,99.23424123)
\curveto(951.9389478,99.12423578)(951.96894777,99.01923588)(951.99894775,98.91924123)
\curveto(952.0389477,98.80923609)(952.06394767,98.6992362)(952.07394775,98.58924123)
\curveto(952.08394765,98.47923642)(952.09894764,98.36423654)(952.11894775,98.24424123)
\curveto(952.12894761,98.2042367)(952.12894761,98.15923674)(952.11894775,98.10924123)
\curveto(952.11894762,98.06923683)(952.12394761,98.02923687)(952.13394775,97.98924123)
\curveto(952.14394759,97.94923695)(952.14894759,97.89423701)(952.14894775,97.82424123)
\curveto(952.14894759,97.75423715)(952.14394759,97.7042372)(952.13394775,97.67424123)
\curveto(952.11394762,97.62423728)(952.10894763,97.57923732)(952.11894775,97.53924123)
\curveto(952.12894761,97.4992374)(952.12894761,97.46423744)(952.11894775,97.43424123)
\lineto(952.11894775,97.34424123)
\curveto(952.09894764,97.28423762)(952.08394765,97.21923768)(952.07394775,97.14924123)
\curveto(952.07394766,97.08923781)(952.06894767,97.02423788)(952.05894775,96.95424123)
\curveto(952.00894773,96.78423812)(951.95894778,96.62423828)(951.90894775,96.47424123)
\curveto(951.85894788,96.32423858)(951.79394794,96.17923872)(951.71394775,96.03924123)
\curveto(951.67394806,95.98923891)(951.64394809,95.93423897)(951.62394775,95.87424123)
\curveto(951.59394814,95.82423908)(951.55894818,95.77423913)(951.51894775,95.72424123)
\curveto(951.3389484,95.48423942)(951.11894862,95.28423962)(950.85894775,95.12424123)
\curveto(950.59894914,94.96423994)(950.31394942,94.82424008)(950.00394775,94.70424123)
\curveto(949.86394987,94.64424026)(949.72395001,94.5992403)(949.58394775,94.56924123)
\curveto(949.4339503,94.53924036)(949.27895046,94.5042404)(949.11894775,94.46424123)
\curveto(949.00895073,94.44424046)(948.89895084,94.42924047)(948.78894775,94.41924123)
\curveto(948.67895106,94.40924049)(948.56895117,94.39424051)(948.45894775,94.37424123)
\curveto(948.41895132,94.36424054)(948.37895136,94.35924054)(948.33894775,94.35924123)
\curveto(948.29895144,94.36924053)(948.25895148,94.36924053)(948.21894775,94.35924123)
\curveto(948.16895157,94.34924055)(948.11895162,94.34424056)(948.06894775,94.34424123)
\lineto(947.90394775,94.34424123)
\curveto(947.85395188,94.32424058)(947.80395193,94.31924058)(947.75394775,94.32924123)
\curveto(947.69395204,94.33924056)(947.6389521,94.33924056)(947.58894775,94.32924123)
\curveto(947.54895219,94.31924058)(947.50395223,94.31924058)(947.45394775,94.32924123)
\curveto(947.40395233,94.33924056)(947.35395238,94.33424057)(947.30394775,94.31424123)
\curveto(947.2339525,94.29424061)(947.15895258,94.28924061)(947.07894775,94.29924123)
\curveto(946.98895275,94.30924059)(946.90395283,94.31424059)(946.82394775,94.31424123)
\curveto(946.733953,94.31424059)(946.6339531,94.30924059)(946.52394775,94.29924123)
\curveto(946.40395333,94.28924061)(946.30395343,94.29424061)(946.22394775,94.31424123)
\lineto(945.93894775,94.31424123)
\lineto(945.30894775,94.35924123)
\curveto(945.20895453,94.36924053)(945.11395462,94.37924052)(945.02394775,94.38924123)
\lineto(944.72394775,94.41924123)
\curveto(944.67395506,94.43924046)(944.62395511,94.44424046)(944.57394775,94.43424123)
\curveto(944.51395522,94.43424047)(944.45895528,94.44424046)(944.40894775,94.46424123)
\curveto(944.2389555,94.51424039)(944.07395566,94.55424035)(943.91394775,94.58424123)
\curveto(943.74395599,94.61424029)(943.58395615,94.66424024)(943.43394775,94.73424123)
\curveto(942.97395676,94.92423998)(942.59895714,95.14423976)(942.30894775,95.39424123)
\curveto(942.01895772,95.65423925)(941.77395796,96.01423889)(941.57394775,96.47424123)
\curveto(941.52395821,96.6042383)(941.48895825,96.73423817)(941.46894775,96.86424123)
\curveto(941.44895829,97.0042379)(941.42395831,97.14423776)(941.39394775,97.28424123)
\curveto(941.38395835,97.35423755)(941.37895836,97.41923748)(941.37894775,97.47924123)
\curveto(941.37895836,97.53923736)(941.37395836,97.6042373)(941.36394775,97.67424123)
\curveto(941.34395839,98.5042364)(941.49395824,99.17423573)(941.81394775,99.68424123)
\curveto(942.12395761,100.19423471)(942.56395717,100.57423433)(943.13394775,100.82424123)
\curveto(943.25395648,100.87423403)(943.37895636,100.91923398)(943.50894775,100.95924123)
\curveto(943.6389561,100.9992339)(943.77395596,101.04423386)(943.91394775,101.09424123)
\curveto(943.99395574,101.11423379)(944.07895566,101.12923377)(944.16894775,101.13924123)
\lineto(944.40894775,101.19924123)
\curveto(944.51895522,101.22923367)(944.62895511,101.24423366)(944.73894775,101.24424123)
\curveto(944.84895489,101.25423365)(944.95895478,101.26923363)(945.06894775,101.28924123)
\curveto(945.11895462,101.30923359)(945.16395457,101.31423359)(945.20394775,101.30424123)
\curveto(945.24395449,101.3042336)(945.28395445,101.30923359)(945.32394775,101.31924123)
\curveto(945.37395436,101.32923357)(945.42895431,101.32923357)(945.48894775,101.31924123)
\curveto(945.5389542,101.31923358)(945.58895415,101.32423358)(945.63894775,101.33424123)
\lineto(945.77394775,101.33424123)
\curveto(945.8339539,101.35423355)(945.90395383,101.35423355)(945.98394775,101.33424123)
\curveto(946.05395368,101.32423358)(946.11895362,101.32923357)(946.17894775,101.34924123)
\curveto(946.20895353,101.35923354)(946.24895349,101.36423354)(946.29894775,101.36424123)
\lineto(946.41894775,101.36424123)
\lineto(946.88394775,101.36424123)
\moveto(949.20894775,99.81924123)
\curveto(948.88895085,99.91923498)(948.52395121,99.97923492)(948.11394775,99.99924123)
\curveto(947.70395203,100.01923488)(947.29395244,100.02923487)(946.88394775,100.02924123)
\curveto(946.45395328,100.02923487)(946.0339537,100.01923488)(945.62394775,99.99924123)
\curveto(945.21395452,99.97923492)(944.82895491,99.93423497)(944.46894775,99.86424123)
\curveto(944.10895563,99.79423511)(943.78895595,99.68423522)(943.50894775,99.53424123)
\curveto(943.21895652,99.39423551)(942.98395675,99.1992357)(942.80394775,98.94924123)
\curveto(942.69395704,98.78923611)(942.61395712,98.60923629)(942.56394775,98.40924123)
\curveto(942.50395723,98.20923669)(942.47395726,97.96423694)(942.47394775,97.67424123)
\curveto(942.49395724,97.65423725)(942.50395723,97.61923728)(942.50394775,97.56924123)
\curveto(942.49395724,97.51923738)(942.49395724,97.47923742)(942.50394775,97.44924123)
\curveto(942.52395721,97.36923753)(942.54395719,97.29423761)(942.56394775,97.22424123)
\curveto(942.57395716,97.16423774)(942.59395714,97.0992378)(942.62394775,97.02924123)
\curveto(942.74395699,96.75923814)(942.91395682,96.53923836)(943.13394775,96.36924123)
\curveto(943.34395639,96.20923869)(943.58895615,96.07423883)(943.86894775,95.96424123)
\curveto(943.97895576,95.91423899)(944.09895564,95.87423903)(944.22894775,95.84424123)
\curveto(944.34895539,95.82423908)(944.47395526,95.7992391)(944.60394775,95.76924123)
\curveto(944.65395508,95.74923915)(944.70895503,95.73923916)(944.76894775,95.73924123)
\curveto(944.81895492,95.73923916)(944.86895487,95.73423917)(944.91894775,95.72424123)
\curveto(945.00895473,95.71423919)(945.10395463,95.7042392)(945.20394775,95.69424123)
\curveto(945.29395444,95.68423922)(945.38895435,95.67423923)(945.48894775,95.66424123)
\curveto(945.56895417,95.66423924)(945.65395408,95.65923924)(945.74394775,95.64924123)
\lineto(945.98394775,95.64924123)
\lineto(946.16394775,95.64924123)
\curveto(946.19395354,95.63923926)(946.22895351,95.63423927)(946.26894775,95.63424123)
\lineto(946.40394775,95.63424123)
\lineto(946.85394775,95.63424123)
\curveto(946.9339528,95.63423927)(947.01895272,95.62923927)(947.10894775,95.61924123)
\curveto(947.18895255,95.61923928)(947.26395247,95.62923927)(947.33394775,95.64924123)
\lineto(947.60394775,95.64924123)
\curveto(947.62395211,95.64923925)(947.65395208,95.64423926)(947.69394775,95.63424123)
\curveto(947.72395201,95.63423927)(947.74895199,95.63923926)(947.76894775,95.64924123)
\curveto(947.86895187,95.65923924)(947.96895177,95.66423924)(948.06894775,95.66424123)
\curveto(948.15895158,95.67423923)(948.25895148,95.68423922)(948.36894775,95.69424123)
\curveto(948.48895125,95.72423918)(948.61395112,95.73923916)(948.74394775,95.73924123)
\curveto(948.86395087,95.74923915)(948.97895076,95.77423913)(949.08894775,95.81424123)
\curveto(949.38895035,95.89423901)(949.65395008,95.97923892)(949.88394775,96.06924123)
\curveto(950.11394962,96.16923873)(950.32894941,96.31423859)(950.52894775,96.50424123)
\curveto(950.72894901,96.71423819)(950.87894886,96.97923792)(950.97894775,97.29924123)
\curveto(950.99894874,97.33923756)(951.00894873,97.37423753)(951.00894775,97.40424123)
\curveto(950.99894874,97.44423746)(951.00394873,97.48923741)(951.02394775,97.53924123)
\curveto(951.0339487,97.57923732)(951.04394869,97.64923725)(951.05394775,97.74924123)
\curveto(951.06394867,97.85923704)(951.05894868,97.94423696)(951.03894775,98.00424123)
\curveto(951.01894872,98.07423683)(951.00894873,98.14423676)(951.00894775,98.21424123)
\curveto(950.99894874,98.28423662)(950.98394875,98.34923655)(950.96394775,98.40924123)
\curveto(950.90394883,98.60923629)(950.81894892,98.78923611)(950.70894775,98.94924123)
\curveto(950.68894905,98.97923592)(950.66894907,99.0042359)(950.64894775,99.02424123)
\lineto(950.58894775,99.08424123)
\curveto(950.56894917,99.12423578)(950.52894921,99.17423573)(950.46894775,99.23424123)
\curveto(950.32894941,99.33423557)(950.19894954,99.41923548)(950.07894775,99.48924123)
\curveto(949.95894978,99.55923534)(949.81394992,99.62923527)(949.64394775,99.69924123)
\curveto(949.57395016,99.72923517)(949.50395023,99.74923515)(949.43394775,99.75924123)
\curveto(949.36395037,99.77923512)(949.28895045,99.7992351)(949.20894775,99.81924123)
}
}
{
\newrgbcolor{curcolor}{0 0 0}
\pscustom[linestyle=none,fillstyle=solid,fillcolor=curcolor]
{
\newpath
\moveto(941.36394775,106.77385061)
\curveto(941.36395837,106.87384575)(941.37395836,106.96884566)(941.39394775,107.05885061)
\curveto(941.40395833,107.14884548)(941.4339583,107.21384541)(941.48394775,107.25385061)
\curveto(941.56395817,107.31384531)(941.66895807,107.34384528)(941.79894775,107.34385061)
\lineto(942.18894775,107.34385061)
\lineto(943.68894775,107.34385061)
\lineto(950.07894775,107.34385061)
\lineto(951.24894775,107.34385061)
\lineto(951.56394775,107.34385061)
\curveto(951.66394807,107.35384527)(951.74394799,107.33884529)(951.80394775,107.29885061)
\curveto(951.88394785,107.24884538)(951.9339478,107.17384545)(951.95394775,107.07385061)
\curveto(951.96394777,106.98384564)(951.96894777,106.87384575)(951.96894775,106.74385061)
\lineto(951.96894775,106.51885061)
\curveto(951.94894779,106.43884619)(951.9339478,106.36884626)(951.92394775,106.30885061)
\curveto(951.90394783,106.24884638)(951.86394787,106.19884643)(951.80394775,106.15885061)
\curveto(951.74394799,106.11884651)(951.66894807,106.09884653)(951.57894775,106.09885061)
\lineto(951.27894775,106.09885061)
\lineto(950.18394775,106.09885061)
\lineto(944.84394775,106.09885061)
\curveto(944.75395498,106.07884655)(944.67895506,106.06384656)(944.61894775,106.05385061)
\curveto(944.54895519,106.05384657)(944.48895525,106.0238466)(944.43894775,105.96385061)
\curveto(944.38895535,105.89384673)(944.36395537,105.80384682)(944.36394775,105.69385061)
\curveto(944.35395538,105.59384703)(944.34895539,105.48384714)(944.34894775,105.36385061)
\lineto(944.34894775,104.22385061)
\lineto(944.34894775,103.72885061)
\curveto(944.3389554,103.56884906)(944.27895546,103.45884917)(944.16894775,103.39885061)
\curveto(944.1389556,103.37884925)(944.10895563,103.36884926)(944.07894775,103.36885061)
\curveto(944.0389557,103.36884926)(943.99395574,103.36384926)(943.94394775,103.35385061)
\curveto(943.82395591,103.33384929)(943.71395602,103.33884929)(943.61394775,103.36885061)
\curveto(943.51395622,103.40884922)(943.44395629,103.46384916)(943.40394775,103.53385061)
\curveto(943.35395638,103.61384901)(943.32895641,103.73384889)(943.32894775,103.89385061)
\curveto(943.32895641,104.05384857)(943.31395642,104.18884844)(943.28394775,104.29885061)
\curveto(943.27395646,104.34884828)(943.26895647,104.40384822)(943.26894775,104.46385061)
\curveto(943.25895648,104.5238481)(943.24395649,104.58384804)(943.22394775,104.64385061)
\curveto(943.17395656,104.79384783)(943.12395661,104.93884769)(943.07394775,105.07885061)
\curveto(943.01395672,105.21884741)(942.94395679,105.35384727)(942.86394775,105.48385061)
\curveto(942.77395696,105.623847)(942.66895707,105.74384688)(942.54894775,105.84385061)
\curveto(942.42895731,105.94384668)(942.29895744,106.03884659)(942.15894775,106.12885061)
\curveto(942.05895768,106.18884644)(941.94895779,106.23384639)(941.82894775,106.26385061)
\curveto(941.70895803,106.30384632)(941.60395813,106.35384627)(941.51394775,106.41385061)
\curveto(941.45395828,106.46384616)(941.41395832,106.53384609)(941.39394775,106.62385061)
\curveto(941.38395835,106.64384598)(941.37895836,106.66884596)(941.37894775,106.69885061)
\curveto(941.37895836,106.7288459)(941.37395836,106.75384587)(941.36394775,106.77385061)
}
}
{
\newrgbcolor{curcolor}{0 0 0}
\pscustom[linestyle=none,fillstyle=solid,fillcolor=curcolor]
{
\newpath
\moveto(941.36394775,115.12345998)
\curveto(941.36395837,115.22345513)(941.37395836,115.31845503)(941.39394775,115.40845998)
\curveto(941.40395833,115.49845485)(941.4339583,115.56345479)(941.48394775,115.60345998)
\curveto(941.56395817,115.66345469)(941.66895807,115.69345466)(941.79894775,115.69345998)
\lineto(942.18894775,115.69345998)
\lineto(943.68894775,115.69345998)
\lineto(950.07894775,115.69345998)
\lineto(951.24894775,115.69345998)
\lineto(951.56394775,115.69345998)
\curveto(951.66394807,115.70345465)(951.74394799,115.68845466)(951.80394775,115.64845998)
\curveto(951.88394785,115.59845475)(951.9339478,115.52345483)(951.95394775,115.42345998)
\curveto(951.96394777,115.33345502)(951.96894777,115.22345513)(951.96894775,115.09345998)
\lineto(951.96894775,114.86845998)
\curveto(951.94894779,114.78845556)(951.9339478,114.71845563)(951.92394775,114.65845998)
\curveto(951.90394783,114.59845575)(951.86394787,114.5484558)(951.80394775,114.50845998)
\curveto(951.74394799,114.46845588)(951.66894807,114.4484559)(951.57894775,114.44845998)
\lineto(951.27894775,114.44845998)
\lineto(950.18394775,114.44845998)
\lineto(944.84394775,114.44845998)
\curveto(944.75395498,114.42845592)(944.67895506,114.41345594)(944.61894775,114.40345998)
\curveto(944.54895519,114.40345595)(944.48895525,114.37345598)(944.43894775,114.31345998)
\curveto(944.38895535,114.24345611)(944.36395537,114.1534562)(944.36394775,114.04345998)
\curveto(944.35395538,113.94345641)(944.34895539,113.83345652)(944.34894775,113.71345998)
\lineto(944.34894775,112.57345998)
\lineto(944.34894775,112.07845998)
\curveto(944.3389554,111.91845843)(944.27895546,111.80845854)(944.16894775,111.74845998)
\curveto(944.1389556,111.72845862)(944.10895563,111.71845863)(944.07894775,111.71845998)
\curveto(944.0389557,111.71845863)(943.99395574,111.71345864)(943.94394775,111.70345998)
\curveto(943.82395591,111.68345867)(943.71395602,111.68845866)(943.61394775,111.71845998)
\curveto(943.51395622,111.75845859)(943.44395629,111.81345854)(943.40394775,111.88345998)
\curveto(943.35395638,111.96345839)(943.32895641,112.08345827)(943.32894775,112.24345998)
\curveto(943.32895641,112.40345795)(943.31395642,112.53845781)(943.28394775,112.64845998)
\curveto(943.27395646,112.69845765)(943.26895647,112.7534576)(943.26894775,112.81345998)
\curveto(943.25895648,112.87345748)(943.24395649,112.93345742)(943.22394775,112.99345998)
\curveto(943.17395656,113.14345721)(943.12395661,113.28845706)(943.07394775,113.42845998)
\curveto(943.01395672,113.56845678)(942.94395679,113.70345665)(942.86394775,113.83345998)
\curveto(942.77395696,113.97345638)(942.66895707,114.09345626)(942.54894775,114.19345998)
\curveto(942.42895731,114.29345606)(942.29895744,114.38845596)(942.15894775,114.47845998)
\curveto(942.05895768,114.53845581)(941.94895779,114.58345577)(941.82894775,114.61345998)
\curveto(941.70895803,114.6534557)(941.60395813,114.70345565)(941.51394775,114.76345998)
\curveto(941.45395828,114.81345554)(941.41395832,114.88345547)(941.39394775,114.97345998)
\curveto(941.38395835,114.99345536)(941.37895836,115.01845533)(941.37894775,115.04845998)
\curveto(941.37895836,115.07845527)(941.37395836,115.10345525)(941.36394775,115.12345998)
}
}
{
\newrgbcolor{curcolor}{0 0 0}
\pscustom[linestyle=none,fillstyle=solid,fillcolor=curcolor]
{
\newpath
\moveto(972.10026367,38.71181936)
\curveto(972.15026442,38.73180981)(972.21026436,38.75680979)(972.28026367,38.78681936)
\curveto(972.35026422,38.81680973)(972.42526414,38.83680971)(972.50526367,38.84681936)
\curveto(972.57526399,38.86680968)(972.64526392,38.86680968)(972.71526367,38.84681936)
\curveto(972.77526379,38.83680971)(972.82026375,38.79680975)(972.85026367,38.72681936)
\curveto(972.8702637,38.67680987)(972.88026369,38.61680993)(972.88026367,38.54681936)
\lineto(972.88026367,38.33681936)
\lineto(972.88026367,37.88681936)
\curveto(972.88026369,37.73681081)(972.85526371,37.61681093)(972.80526367,37.52681936)
\curveto(972.74526382,37.42681112)(972.64026393,37.35181119)(972.49026367,37.30181936)
\curveto(972.34026423,37.26181128)(972.20526436,37.21681133)(972.08526367,37.16681936)
\curveto(971.82526474,37.05681149)(971.55526501,36.95681159)(971.27526367,36.86681936)
\curveto(970.99526557,36.77681177)(970.72026585,36.67681187)(970.45026367,36.56681936)
\curveto(970.36026621,36.53681201)(970.27526629,36.50681204)(970.19526367,36.47681936)
\curveto(970.11526645,36.45681209)(970.04026653,36.42681212)(969.97026367,36.38681936)
\curveto(969.90026667,36.35681219)(969.84026673,36.31181223)(969.79026367,36.25181936)
\curveto(969.74026683,36.19181235)(969.70026687,36.11181243)(969.67026367,36.01181936)
\curveto(969.65026692,35.96181258)(969.64526692,35.90181264)(969.65526367,35.83181936)
\lineto(969.65526367,35.63681936)
\lineto(969.65526367,32.80181936)
\lineto(969.65526367,32.50181936)
\curveto(969.64526692,32.39181615)(969.64526692,32.28681626)(969.65526367,32.18681936)
\curveto(969.6652669,32.08681646)(969.68026689,31.99181655)(969.70026367,31.90181936)
\curveto(969.72026685,31.82181672)(969.76026681,31.76181678)(969.82026367,31.72181936)
\curveto(969.92026665,31.6418169)(970.03526653,31.58181696)(970.16526367,31.54181936)
\curveto(970.28526628,31.51181703)(970.41026616,31.47181707)(970.54026367,31.42181936)
\curveto(970.7702658,31.32181722)(971.01026556,31.22681732)(971.26026367,31.13681936)
\curveto(971.51026506,31.05681749)(971.75026482,30.96681758)(971.98026367,30.86681936)
\curveto(972.04026453,30.8468177)(972.11026446,30.82181772)(972.19026367,30.79181936)
\curveto(972.26026431,30.77181777)(972.33526423,30.7468178)(972.41526367,30.71681936)
\curveto(972.49526407,30.68681786)(972.570264,30.65181789)(972.64026367,30.61181936)
\curveto(972.70026387,30.58181796)(972.74526382,30.546818)(972.77526367,30.50681936)
\curveto(972.83526373,30.42681812)(972.8702637,30.31681823)(972.88026367,30.17681936)
\lineto(972.88026367,29.75681936)
\lineto(972.88026367,29.51681936)
\curveto(972.8702637,29.4468191)(972.84526372,29.38681916)(972.80526367,29.33681936)
\curveto(972.77526379,29.28681926)(972.73026384,29.25681929)(972.67026367,29.24681936)
\curveto(972.61026396,29.2468193)(972.55026402,29.25181929)(972.49026367,29.26181936)
\curveto(972.42026415,29.28181926)(972.35526421,29.30181924)(972.29526367,29.32181936)
\curveto(972.22526434,29.35181919)(972.17526439,29.37681917)(972.14526367,29.39681936)
\curveto(971.82526474,29.53681901)(971.51026506,29.66181888)(971.20026367,29.77181936)
\curveto(970.88026569,29.88181866)(970.56026601,30.00181854)(970.24026367,30.13181936)
\curveto(970.02026655,30.22181832)(969.80526676,30.30681824)(969.59526367,30.38681936)
\curveto(969.37526719,30.46681808)(969.15526741,30.55181799)(968.93526367,30.64181936)
\curveto(968.21526835,30.9418176)(967.49026908,31.22681732)(966.76026367,31.49681936)
\curveto(966.02027055,31.76681678)(965.28527128,32.05181649)(964.55526367,32.35181936)
\curveto(964.29527227,32.46181608)(964.03027254,32.56181598)(963.76026367,32.65181936)
\curveto(963.49027308,32.75181579)(963.22527334,32.85681569)(962.96526367,32.96681936)
\curveto(962.85527371,33.01681553)(962.73527383,33.06181548)(962.60526367,33.10181936)
\curveto(962.4652741,33.15181539)(962.3652742,33.22181532)(962.30526367,33.31181936)
\curveto(962.2652743,33.35181519)(962.23527433,33.41681513)(962.21526367,33.50681936)
\curveto(962.20527436,33.52681502)(962.20527436,33.546815)(962.21526367,33.56681936)
\curveto(962.21527435,33.59681495)(962.21027436,33.62181492)(962.20026367,33.64181936)
\curveto(962.20027437,33.82181472)(962.20027437,34.03181451)(962.20026367,34.27181936)
\curveto(962.19027438,34.51181403)(962.22527434,34.68681386)(962.30526367,34.79681936)
\curveto(962.3652742,34.87681367)(962.4652741,34.93681361)(962.60526367,34.97681936)
\curveto(962.73527383,35.02681352)(962.85527371,35.07681347)(962.96526367,35.12681936)
\curveto(963.19527337,35.22681332)(963.42527314,35.31681323)(963.65526367,35.39681936)
\curveto(963.88527268,35.47681307)(964.11527245,35.56681298)(964.34526367,35.66681936)
\curveto(964.54527202,35.7468128)(964.75027182,35.82181272)(964.96026367,35.89181936)
\curveto(965.1702714,35.97181257)(965.37527119,36.05681249)(965.57526367,36.14681936)
\curveto(966.30527026,36.4468121)(967.04526952,36.73181181)(967.79526367,37.00181936)
\curveto(968.53526803,37.28181126)(969.2702673,37.57681097)(970.00026367,37.88681936)
\curveto(970.09026648,37.92681062)(970.17526639,37.95681059)(970.25526367,37.97681936)
\curveto(970.33526623,38.00681054)(970.42026615,38.03681051)(970.51026367,38.06681936)
\curveto(970.7702658,38.17681037)(971.03526553,38.28181026)(971.30526367,38.38181936)
\curveto(971.57526499,38.49181005)(971.84026473,38.60180994)(972.10026367,38.71181936)
\moveto(968.45526367,35.50181936)
\curveto(968.42526814,35.59181295)(968.37526819,35.6468129)(968.30526367,35.66681936)
\curveto(968.23526833,35.69681285)(968.16026841,35.70181284)(968.08026367,35.68181936)
\curveto(967.99026858,35.67181287)(967.90526866,35.6468129)(967.82526367,35.60681936)
\curveto(967.73526883,35.57681297)(967.66026891,35.546813)(967.60026367,35.51681936)
\curveto(967.56026901,35.49681305)(967.52526904,35.48681306)(967.49526367,35.48681936)
\curveto(967.4652691,35.48681306)(967.43026914,35.47681307)(967.39026367,35.45681936)
\lineto(967.15026367,35.36681936)
\curveto(967.06026951,35.3468132)(966.9702696,35.31681323)(966.88026367,35.27681936)
\curveto(966.52027005,35.12681342)(966.15527041,34.99181355)(965.78526367,34.87181936)
\curveto(965.40527116,34.76181378)(965.03527153,34.63181391)(964.67526367,34.48181936)
\curveto(964.565272,34.43181411)(964.45527211,34.38681416)(964.34526367,34.34681936)
\curveto(964.23527233,34.31681423)(964.13027244,34.27681427)(964.03026367,34.22681936)
\curveto(963.98027259,34.20681434)(963.93527263,34.18181436)(963.89526367,34.15181936)
\curveto(963.84527272,34.13181441)(963.82027275,34.08181446)(963.82026367,34.00181936)
\curveto(963.84027273,33.98181456)(963.85527271,33.96181458)(963.86526367,33.94181936)
\curveto(963.87527269,33.92181462)(963.89027268,33.90181464)(963.91026367,33.88181936)
\curveto(963.96027261,33.8418147)(964.01527255,33.81181473)(964.07526367,33.79181936)
\curveto(964.12527244,33.77181477)(964.18027239,33.75181479)(964.24026367,33.73181936)
\curveto(964.35027222,33.68181486)(964.46027211,33.6418149)(964.57026367,33.61181936)
\curveto(964.68027189,33.58181496)(964.79027178,33.541815)(964.90026367,33.49181936)
\curveto(965.29027128,33.32181522)(965.68527088,33.17181537)(966.08526367,33.04181936)
\curveto(966.48527008,32.92181562)(966.87526969,32.78181576)(967.25526367,32.62181936)
\lineto(967.40526367,32.56181936)
\curveto(967.45526911,32.55181599)(967.50526906,32.53681601)(967.55526367,32.51681936)
\lineto(967.79526367,32.42681936)
\curveto(967.87526869,32.39681615)(967.95526861,32.37181617)(968.03526367,32.35181936)
\curveto(968.08526848,32.33181621)(968.14026843,32.32181622)(968.20026367,32.32181936)
\curveto(968.26026831,32.33181621)(968.31026826,32.3468162)(968.35026367,32.36681936)
\curveto(968.43026814,32.41681613)(968.47526809,32.52181602)(968.48526367,32.68181936)
\lineto(968.48526367,33.13181936)
\lineto(968.48526367,34.73681936)
\curveto(968.48526808,34.8468137)(968.49026808,34.98181356)(968.50026367,35.14181936)
\curveto(968.50026807,35.30181324)(968.48526808,35.42181312)(968.45526367,35.50181936)
}
}
{
\newrgbcolor{curcolor}{0 0 0}
\pscustom[linestyle=none,fillstyle=solid,fillcolor=curcolor]
{
\newpath
\moveto(965.26026367,46.44338186)
\curveto(965.31027126,46.51337426)(965.38527118,46.54837422)(965.48526367,46.54838186)
\curveto(965.58527098,46.55837421)(965.69027088,46.56337421)(965.80026367,46.56338186)
\lineto(972.07026367,46.56338186)
\lineto(972.67026367,46.56338186)
\curveto(972.72026385,46.54337423)(972.7702638,46.53837423)(972.82026367,46.54838186)
\curveto(972.86026371,46.55837421)(972.90526366,46.55337422)(972.95526367,46.53338186)
\curveto(973.05526351,46.51337426)(973.15526341,46.49837427)(973.25526367,46.48838186)
\curveto(973.3652632,46.48837428)(973.4702631,46.4733743)(973.57026367,46.44338186)
\curveto(973.68026289,46.41337436)(973.78526278,46.38337439)(973.88526367,46.35338186)
\curveto(973.98526258,46.33337444)(974.08526248,46.29837447)(974.18526367,46.24838186)
\curveto(974.44526212,46.14837462)(974.68026189,46.01837475)(974.89026367,45.85838186)
\curveto(975.10026147,45.70837506)(975.27526129,45.52837524)(975.41526367,45.31838186)
\curveto(975.53526103,45.14837562)(975.63026094,44.9683758)(975.70026367,44.77838186)
\curveto(975.78026079,44.58837618)(975.85526071,44.38337639)(975.92526367,44.16338186)
\curveto(975.94526062,44.0733767)(975.95526061,43.98337679)(975.95526367,43.89338186)
\curveto(975.9652606,43.80337697)(975.98026059,43.71337706)(976.00026367,43.62338186)
\lineto(976.00026367,43.53338186)
\curveto(976.01026056,43.51337726)(976.01526055,43.49337728)(976.01526367,43.47338186)
\curveto(976.02526054,43.42337735)(976.02526054,43.3733774)(976.01526367,43.32338186)
\curveto(976.00526056,43.28337749)(976.01026056,43.23837753)(976.03026367,43.18838186)
\curveto(976.05026052,43.11837765)(976.05526051,43.00837776)(976.04526367,42.85838186)
\curveto(976.04526052,42.71837805)(976.03526053,42.61837815)(976.01526367,42.55838186)
\curveto(976.01526055,42.52837824)(976.01026056,42.49837827)(976.00026367,42.46838186)
\lineto(976.00026367,42.40838186)
\curveto(975.98026059,42.31837845)(975.9652606,42.22837854)(975.95526367,42.13838186)
\curveto(975.95526061,42.04837872)(975.94526062,41.96337881)(975.92526367,41.88338186)
\curveto(975.90526066,41.80337897)(975.88026069,41.72337905)(975.85026367,41.64338186)
\curveto(975.83026074,41.56337921)(975.80526076,41.48337929)(975.77526367,41.40338186)
\curveto(975.64526092,41.08337969)(975.50026107,40.81337996)(975.34026367,40.59338186)
\curveto(975.18026139,40.38338039)(974.95526161,40.19338058)(974.66526367,40.02338186)
\curveto(974.64526192,40.00338077)(974.62026195,39.98838078)(974.59026367,39.97838186)
\curveto(974.570262,39.97838079)(974.54526202,39.9683808)(974.51526367,39.94838186)
\curveto(974.43526213,39.91838085)(974.32026225,39.88338089)(974.17026367,39.84338186)
\curveto(974.03026254,39.81338096)(973.92526264,39.84338093)(973.85526367,39.93338186)
\curveto(973.80526276,39.99338078)(973.78026279,40.0733807)(973.78026367,40.17338186)
\lineto(973.78026367,40.50338186)
\lineto(973.78026367,40.66838186)
\curveto(973.78026279,40.72838004)(973.79026278,40.78337999)(973.81026367,40.83338186)
\curveto(973.84026273,40.92337985)(973.89026268,40.98837978)(973.96026367,41.02838186)
\curveto(974.03026254,41.0683797)(974.10526246,41.11337966)(974.18526367,41.16338186)
\lineto(974.36526367,41.28338186)
\curveto(974.43526213,41.33337944)(974.49026208,41.38337939)(974.53026367,41.43338186)
\curveto(974.72026185,41.68337909)(974.86026171,41.98337879)(974.95026367,42.33338186)
\curveto(974.9702616,42.39337838)(974.98026159,42.45337832)(974.98026367,42.51338186)
\curveto(974.99026158,42.58337819)(975.00526156,42.64837812)(975.02526367,42.70838186)
\lineto(975.02526367,42.79838186)
\curveto(975.04526152,42.8683779)(975.05526151,42.95337782)(975.05526367,43.05338186)
\curveto(975.05526151,43.15337762)(975.04526152,43.24337753)(975.02526367,43.32338186)
\curveto(975.01526155,43.35337742)(975.01026156,43.39337738)(975.01026367,43.44338186)
\curveto(974.99026158,43.54337723)(974.9702616,43.63837713)(974.95026367,43.72838186)
\curveto(974.94026163,43.81837695)(974.91526165,43.90337687)(974.87526367,43.98338186)
\curveto(974.75526181,44.2733765)(974.59026198,44.50837626)(974.38026367,44.68838186)
\curveto(974.18026239,44.87837589)(973.93526263,45.03337574)(973.64526367,45.15338186)
\curveto(973.55526301,45.19337558)(973.46026311,45.21837555)(973.36026367,45.22838186)
\curveto(973.26026331,45.24837552)(973.15526341,45.2733755)(973.04526367,45.30338186)
\curveto(972.99526357,45.32337545)(972.94526362,45.33337544)(972.89526367,45.33338186)
\curveto(972.84526372,45.33337544)(972.79526377,45.33837543)(972.74526367,45.34838186)
\curveto(972.71526385,45.35837541)(972.6652639,45.36337541)(972.59526367,45.36338186)
\curveto(972.51526405,45.38337539)(972.43026414,45.38337539)(972.34026367,45.36338186)
\curveto(972.29026428,45.35337542)(972.24526432,45.34837542)(972.20526367,45.34838186)
\curveto(972.1652644,45.35837541)(972.13026444,45.35337542)(972.10026367,45.33338186)
\curveto(972.08026449,45.31337546)(972.0702645,45.29837547)(972.07026367,45.28838186)
\lineto(972.02526367,45.24338186)
\curveto(972.02526454,45.14337563)(972.05526451,45.0683757)(972.11526367,45.01838186)
\curveto(972.1652644,44.97837579)(972.21026436,44.92837584)(972.25026367,44.86838186)
\lineto(972.46026367,44.62838186)
\curveto(972.52026405,44.54837622)(972.57526399,44.45837631)(972.62526367,44.35838186)
\curveto(972.71526385,44.21837655)(972.79026378,44.04337673)(972.85026367,43.83338186)
\curveto(972.90026367,43.62337715)(972.93526363,43.40337737)(972.95526367,43.17338186)
\curveto(972.97526359,42.94337783)(972.9702636,42.71337806)(972.94026367,42.48338186)
\curveto(972.92026365,42.25337852)(972.88026369,42.04337873)(972.82026367,41.85338186)
\curveto(972.51026406,40.91337986)(971.91526465,40.25338052)(971.03526367,39.87338186)
\curveto(970.93526563,39.82338095)(970.84026573,39.78338099)(970.75026367,39.75338186)
\curveto(970.65026592,39.72338105)(970.54526602,39.68838108)(970.43526367,39.64838186)
\curveto(970.38526618,39.62838114)(970.34026623,39.61838115)(970.30026367,39.61838186)
\curveto(970.26026631,39.61838115)(970.21526635,39.60838116)(970.16526367,39.58838186)
\curveto(970.09526647,39.5683812)(970.02526654,39.55338122)(969.95526367,39.54338186)
\curveto(969.87526669,39.54338123)(969.80026677,39.53338124)(969.73026367,39.51338186)
\curveto(969.69026688,39.50338127)(969.65526691,39.49838127)(969.62526367,39.49838186)
\curveto(969.58526698,39.50838126)(969.54526702,39.50838126)(969.50526367,39.49838186)
\curveto(969.4652671,39.49838127)(969.42526714,39.49338128)(969.38526367,39.48338186)
\lineto(969.26526367,39.48338186)
\curveto(969.14526742,39.46338131)(969.02026755,39.46338131)(968.89026367,39.48338186)
\curveto(968.83026774,39.49338128)(968.7702678,39.49838127)(968.71026367,39.49838186)
\lineto(968.54526367,39.49838186)
\curveto(968.49526807,39.50838126)(968.45526811,39.51338126)(968.42526367,39.51338186)
\curveto(968.38526818,39.51338126)(968.34026823,39.51838125)(968.29026367,39.52838186)
\curveto(968.18026839,39.55838121)(968.07526849,39.57838119)(967.97526367,39.58838186)
\curveto(967.8652687,39.59838117)(967.75526881,39.62338115)(967.64526367,39.66338186)
\curveto(967.52526904,39.70338107)(967.41026916,39.73838103)(967.30026367,39.76838186)
\curveto(967.18026939,39.80838096)(967.0652695,39.85338092)(966.95526367,39.90338186)
\curveto(966.79526977,39.9733808)(966.65026992,40.05338072)(966.52026367,40.14338186)
\curveto(966.38027019,40.23338054)(966.24527032,40.32838044)(966.11526367,40.42838186)
\curveto(966.00527056,40.49838027)(965.91527065,40.58838018)(965.84526367,40.69838186)
\lineto(965.78526367,40.75838186)
\lineto(965.72526367,40.81838186)
\lineto(965.60526367,40.96838186)
\lineto(965.48526367,41.14838186)
\curveto(965.40527116,41.27837949)(965.33527123,41.41337936)(965.27526367,41.55338186)
\curveto(965.21527135,41.70337907)(965.16027141,41.86337891)(965.11026367,42.03338186)
\curveto(965.08027149,42.13337864)(965.06027151,42.23337854)(965.05026367,42.33338186)
\curveto(965.04027153,42.44337833)(965.02527154,42.55337822)(965.00526367,42.66338186)
\curveto(964.99527157,42.70337807)(964.99527157,42.75337802)(965.00526367,42.81338186)
\curveto(965.01527155,42.88337789)(965.01027156,42.93337784)(964.99026367,42.96338186)
\curveto(964.98027159,43.28337749)(965.01027156,43.5683772)(965.08026367,43.81838186)
\curveto(965.15027142,44.07837669)(965.25027132,44.30837646)(965.38026367,44.50838186)
\curveto(965.42027115,44.57837619)(965.4652711,44.64337613)(965.51526367,44.70338186)
\lineto(965.66526367,44.88338186)
\curveto(965.70527086,44.93337584)(965.75027082,44.97837579)(965.80026367,45.01838186)
\curveto(965.84027073,45.0683757)(965.86027071,45.14337563)(965.86026367,45.24338186)
\lineto(965.81526367,45.28838186)
\curveto(965.79527077,45.30837546)(965.7702708,45.32837544)(965.74026367,45.34838186)
\curveto(965.66027091,45.37837539)(965.58027099,45.39337538)(965.50026367,45.39338186)
\curveto(965.42027115,45.40337537)(965.35027122,45.43337534)(965.29026367,45.48338186)
\curveto(965.25027132,45.51337526)(965.22027135,45.5733752)(965.20026367,45.66338186)
\curveto(965.1702714,45.75337502)(965.15527141,45.84837492)(965.15526367,45.94838186)
\curveto(965.15527141,46.04837472)(965.1652714,46.14337463)(965.18526367,46.23338186)
\curveto(965.20527136,46.33337444)(965.23027134,46.40337437)(965.26026367,46.44338186)
\moveto(969.04026367,45.31838186)
\curveto(969.00026757,45.32837544)(968.95026762,45.33337544)(968.89026367,45.33338186)
\curveto(968.82026775,45.33337544)(968.7652678,45.32837544)(968.72526367,45.31838186)
\lineto(968.48526367,45.31838186)
\curveto(968.39526817,45.29837547)(968.31026826,45.28337549)(968.23026367,45.27338186)
\curveto(968.14026843,45.26337551)(968.05526851,45.24837552)(967.97526367,45.22838186)
\curveto(967.89526867,45.20837556)(967.82026875,45.18837558)(967.75026367,45.16838186)
\curveto(967.6702689,45.15837561)(967.59526897,45.13837563)(967.52526367,45.10838186)
\curveto(967.24526932,44.99837577)(966.99526957,44.85337592)(966.77526367,44.67338186)
\curveto(966.55527001,44.50337627)(966.39027018,44.28337649)(966.28026367,44.01338186)
\curveto(966.24027033,43.93337684)(966.21027036,43.84837692)(966.19026367,43.75838186)
\curveto(966.16027041,43.6683771)(966.13527043,43.5733772)(966.11526367,43.47338186)
\curveto(966.09527047,43.39337738)(966.09027048,43.30337747)(966.10026367,43.20338186)
\lineto(966.10026367,42.93338186)
\curveto(966.11027046,42.88337789)(966.11527045,42.83337794)(966.11526367,42.78338186)
\curveto(966.11527045,42.74337803)(966.12027045,42.69837807)(966.13026367,42.64838186)
\curveto(966.18027039,42.45837831)(966.23027034,42.29837847)(966.28026367,42.16838186)
\curveto(966.42027015,41.82837894)(966.63026994,41.56337921)(966.91026367,41.37338186)
\curveto(967.19026938,41.18337959)(967.51526905,41.03337974)(967.88526367,40.92338186)
\curveto(967.9652686,40.90337987)(968.04526852,40.88837988)(968.12526367,40.87838186)
\curveto(968.19526837,40.87837989)(968.2702683,40.8683799)(968.35026367,40.84838186)
\curveto(968.38026819,40.82837994)(968.41526815,40.81837995)(968.45526367,40.81838186)
\curveto(968.49526807,40.82837994)(968.53026804,40.82837994)(968.56026367,40.81838186)
\lineto(968.89026367,40.81838186)
\lineto(969.23526367,40.81838186)
\curveto(969.34526722,40.81837995)(969.45026712,40.82837994)(969.55026367,40.84838186)
\lineto(969.62526367,40.84838186)
\curveto(969.65526691,40.85837991)(969.68026689,40.86337991)(969.70026367,40.86338186)
\curveto(969.79026678,40.88337989)(969.88026669,40.89837987)(969.97026367,40.90838186)
\curveto(970.06026651,40.92837984)(970.14526642,40.95337982)(970.22526367,40.98338186)
\curveto(970.48526608,41.06337971)(970.72526584,41.16337961)(970.94526367,41.28338186)
\curveto(971.1652654,41.40337937)(971.34526522,41.56337921)(971.48526367,41.76338186)
\lineto(971.57526367,41.88338186)
\curveto(971.59526497,41.92337885)(971.61526495,41.9683788)(971.63526367,42.01838186)
\curveto(971.68526488,42.09837867)(971.72526484,42.18337859)(971.75526367,42.27338186)
\curveto(971.78526478,42.36337841)(971.81526475,42.46337831)(971.84526367,42.57338186)
\curveto(971.85526471,42.62337815)(971.86026471,42.6683781)(971.86026367,42.70838186)
\curveto(971.85026472,42.75837801)(971.85526471,42.80837796)(971.87526367,42.85838186)
\curveto(971.88526468,42.88837788)(971.89026468,42.93837783)(971.89026367,43.00838186)
\curveto(971.89026468,43.07837769)(971.88526468,43.12837764)(971.87526367,43.15838186)
\curveto(971.8652647,43.18837758)(971.8652647,43.21837755)(971.87526367,43.24838186)
\curveto(971.87526469,43.28837748)(971.8702647,43.32837744)(971.86026367,43.36838186)
\curveto(971.84026473,43.45837731)(971.82026475,43.54337723)(971.80026367,43.62338186)
\curveto(971.78026479,43.70337707)(971.75526481,43.78337699)(971.72526367,43.86338186)
\curveto(971.57526499,44.20337657)(971.3652652,44.4733763)(971.09526367,44.67338186)
\curveto(970.82526574,44.8733759)(970.51026606,45.03337574)(970.15026367,45.15338186)
\curveto(970.06026651,45.18337559)(969.9702666,45.20337557)(969.88026367,45.21338186)
\curveto(969.78026679,45.23337554)(969.68526688,45.25337552)(969.59526367,45.27338186)
\curveto(969.55526701,45.28337549)(969.52026705,45.28837548)(969.49026367,45.28838186)
\curveto(969.45026712,45.28837548)(969.41026716,45.29337548)(969.37026367,45.30338186)
\curveto(969.32026725,45.32337545)(969.2702673,45.32337545)(969.22026367,45.30338186)
\curveto(969.16026741,45.29337548)(969.10026747,45.29837547)(969.04026367,45.31838186)
}
}
{
\newrgbcolor{curcolor}{0 0 0}
\pscustom[linestyle=none,fillstyle=solid,fillcolor=curcolor]
{
\newpath
\moveto(968.68026367,55.56666311)
\curveto(968.74026783,55.58665505)(968.83526773,55.59665504)(968.96526367,55.59666311)
\curveto(969.08526748,55.59665504)(969.1702674,55.59165504)(969.22026367,55.58166311)
\lineto(969.37026367,55.58166311)
\curveto(969.45026712,55.57165506)(969.52526704,55.56165507)(969.59526367,55.55166311)
\curveto(969.65526691,55.55165508)(969.72526684,55.54665509)(969.80526367,55.53666311)
\curveto(969.8652667,55.51665512)(969.92526664,55.50165513)(969.98526367,55.49166311)
\curveto(970.04526652,55.49165514)(970.10526646,55.48165515)(970.16526367,55.46166311)
\curveto(970.29526627,55.42165521)(970.42526614,55.38665525)(970.55526367,55.35666311)
\curveto(970.68526588,55.32665531)(970.80526576,55.28665535)(970.91526367,55.23666311)
\curveto(971.39526517,55.02665561)(971.80026477,54.74665589)(972.13026367,54.39666311)
\curveto(972.45026412,54.04665659)(972.69526387,53.61665702)(972.86526367,53.10666311)
\curveto(972.90526366,52.99665764)(972.93526363,52.87665776)(972.95526367,52.74666311)
\curveto(972.97526359,52.62665801)(972.99526357,52.50165813)(973.01526367,52.37166311)
\curveto(973.02526354,52.31165832)(973.03026354,52.24665839)(973.03026367,52.17666311)
\curveto(973.04026353,52.11665852)(973.04526352,52.05665858)(973.04526367,51.99666311)
\curveto(973.05526351,51.95665868)(973.06026351,51.89665874)(973.06026367,51.81666311)
\curveto(973.06026351,51.74665889)(973.05526351,51.69665894)(973.04526367,51.66666311)
\curveto(973.03526353,51.62665901)(973.03026354,51.58665905)(973.03026367,51.54666311)
\curveto(973.04026353,51.50665913)(973.04026353,51.47165916)(973.03026367,51.44166311)
\lineto(973.03026367,51.35166311)
\lineto(972.98526367,50.99166311)
\curveto(972.94526362,50.85165978)(972.90526366,50.71665992)(972.86526367,50.58666311)
\curveto(972.82526374,50.45666018)(972.78026379,50.3316603)(972.73026367,50.21166311)
\curveto(972.53026404,49.76166087)(972.2702643,49.39166124)(971.95026367,49.10166311)
\curveto(971.63026494,48.81166182)(971.24026533,48.57166206)(970.78026367,48.38166311)
\curveto(970.68026589,48.3316623)(970.58026599,48.29166234)(970.48026367,48.26166311)
\curveto(970.38026619,48.24166239)(970.27526629,48.22166241)(970.16526367,48.20166311)
\curveto(970.12526644,48.18166245)(970.09526647,48.17166246)(970.07526367,48.17166311)
\curveto(970.04526652,48.18166245)(970.01026656,48.18166245)(969.97026367,48.17166311)
\curveto(969.89026668,48.15166248)(969.81026676,48.1366625)(969.73026367,48.12666311)
\curveto(969.64026693,48.12666251)(969.55526701,48.11666252)(969.47526367,48.09666311)
\lineto(969.35526367,48.09666311)
\curveto(969.31526725,48.09666254)(969.2702673,48.09166254)(969.22026367,48.08166311)
\curveto(969.1702674,48.07166256)(969.08526748,48.06666257)(968.96526367,48.06666311)
\curveto(968.83526773,48.06666257)(968.74026783,48.07666256)(968.68026367,48.09666311)
\curveto(968.61026796,48.11666252)(968.54026803,48.12166251)(968.47026367,48.11166311)
\curveto(968.40026817,48.10166253)(968.33026824,48.10666253)(968.26026367,48.12666311)
\curveto(968.21026836,48.1366625)(968.1702684,48.14166249)(968.14026367,48.14166311)
\curveto(968.10026847,48.15166248)(968.05526851,48.16166247)(968.00526367,48.17166311)
\curveto(967.88526868,48.20166243)(967.7652688,48.22666241)(967.64526367,48.24666311)
\curveto(967.52526904,48.27666236)(967.41026916,48.31666232)(967.30026367,48.36666311)
\curveto(966.93026964,48.51666212)(966.60026997,48.69666194)(966.31026367,48.90666311)
\curveto(966.01027056,49.12666151)(965.76027081,49.39166124)(965.56026367,49.70166311)
\curveto(965.48027109,49.82166081)(965.41527115,49.94666069)(965.36526367,50.07666311)
\curveto(965.30527126,50.20666043)(965.24527132,50.34166029)(965.18526367,50.48166311)
\curveto(965.13527143,50.60166003)(965.10527146,50.7316599)(965.09526367,50.87166311)
\curveto(965.07527149,51.01165962)(965.04527152,51.15165948)(965.00526367,51.29166311)
\lineto(965.00526367,51.48666311)
\curveto(964.99527157,51.55665908)(964.98527158,51.62165901)(964.97526367,51.68166311)
\curveto(964.9652716,52.57165806)(965.15027142,53.31165732)(965.53026367,53.90166311)
\curveto(965.91027066,54.49165614)(966.40527016,54.91665572)(967.01526367,55.17666311)
\curveto(967.11526945,55.22665541)(967.21526935,55.26665537)(967.31526367,55.29666311)
\curveto(967.41526915,55.32665531)(967.52026905,55.36165527)(967.63026367,55.40166311)
\curveto(967.74026883,55.4316552)(967.86026871,55.45665518)(967.99026367,55.47666311)
\curveto(968.11026846,55.49665514)(968.23526833,55.52165511)(968.36526367,55.55166311)
\curveto(968.41526815,55.56165507)(968.4702681,55.56165507)(968.53026367,55.55166311)
\curveto(968.58026799,55.55165508)(968.63026794,55.55665508)(968.68026367,55.56666311)
\moveto(969.53526367,54.23166311)
\curveto(969.4652671,54.25165638)(969.38526718,54.25665638)(969.29526367,54.24666311)
\lineto(969.04026367,54.24666311)
\curveto(968.65026792,54.24665639)(968.32026825,54.21165642)(968.05026367,54.14166311)
\curveto(967.9702686,54.11165652)(967.89026868,54.08665655)(967.81026367,54.06666311)
\curveto(967.73026884,54.04665659)(967.65526891,54.02165661)(967.58526367,53.99166311)
\curveto(966.93526963,53.71165692)(966.48527008,53.26665737)(966.23526367,52.65666311)
\curveto(966.20527036,52.58665805)(966.18527038,52.51165812)(966.17526367,52.43166311)
\lineto(966.11526367,52.19166311)
\curveto(966.09527047,52.11165852)(966.08527048,52.02665861)(966.08526367,51.93666311)
\lineto(966.08526367,51.66666311)
\lineto(966.13026367,51.39666311)
\curveto(966.15027042,51.29665934)(966.17527039,51.20165943)(966.20526367,51.11166311)
\curveto(966.22527034,51.0316596)(966.25527031,50.95165968)(966.29526367,50.87166311)
\curveto(966.31527025,50.80165983)(966.34527022,50.7366599)(966.38526367,50.67666311)
\curveto(966.42527014,50.61666002)(966.4652701,50.56166007)(966.50526367,50.51166311)
\curveto(966.67526989,50.27166036)(966.88026969,50.07666056)(967.12026367,49.92666311)
\curveto(967.36026921,49.77666086)(967.64026893,49.64666099)(967.96026367,49.53666311)
\curveto(968.06026851,49.50666113)(968.1652684,49.48666115)(968.27526367,49.47666311)
\curveto(968.37526819,49.46666117)(968.48026809,49.45166118)(968.59026367,49.43166311)
\curveto(968.63026794,49.42166121)(968.69526787,49.41666122)(968.78526367,49.41666311)
\curveto(968.81526775,49.40666123)(968.85026772,49.40166123)(968.89026367,49.40166311)
\curveto(968.93026764,49.41166122)(968.97526759,49.41666122)(969.02526367,49.41666311)
\lineto(969.32526367,49.41666311)
\curveto(969.42526714,49.41666122)(969.51526705,49.42666121)(969.59526367,49.44666311)
\lineto(969.77526367,49.47666311)
\curveto(969.87526669,49.49666114)(969.97526659,49.51166112)(970.07526367,49.52166311)
\curveto(970.1652664,49.54166109)(970.25026632,49.57166106)(970.33026367,49.61166311)
\curveto(970.570266,49.71166092)(970.79526577,49.82666081)(971.00526367,49.95666311)
\curveto(971.21526535,50.09666054)(971.39026518,50.26666037)(971.53026367,50.46666311)
\curveto(971.56026501,50.51666012)(971.58526498,50.56166007)(971.60526367,50.60166311)
\curveto(971.62526494,50.64165999)(971.65026492,50.68665995)(971.68026367,50.73666311)
\curveto(971.73026484,50.81665982)(971.77526479,50.90165973)(971.81526367,50.99166311)
\curveto(971.84526472,51.09165954)(971.87526469,51.19665944)(971.90526367,51.30666311)
\curveto(971.92526464,51.35665928)(971.93526463,51.40165923)(971.93526367,51.44166311)
\curveto(971.92526464,51.49165914)(971.92526464,51.54165909)(971.93526367,51.59166311)
\curveto(971.94526462,51.62165901)(971.95526461,51.68165895)(971.96526367,51.77166311)
\curveto(971.97526459,51.87165876)(971.9702646,51.94665869)(971.95026367,51.99666311)
\curveto(971.94026463,52.0366586)(971.94026463,52.07665856)(971.95026367,52.11666311)
\curveto(971.95026462,52.15665848)(971.94026463,52.19665844)(971.92026367,52.23666311)
\curveto(971.90026467,52.31665832)(971.88526468,52.39665824)(971.87526367,52.47666311)
\curveto(971.85526471,52.55665808)(971.83026474,52.631658)(971.80026367,52.70166311)
\curveto(971.66026491,53.04165759)(971.4652651,53.31665732)(971.21526367,53.52666311)
\curveto(970.9652656,53.7366569)(970.6702659,53.91165672)(970.33026367,54.05166311)
\curveto(970.21026636,54.10165653)(970.08526648,54.1316565)(969.95526367,54.14166311)
\curveto(969.81526675,54.16165647)(969.67526689,54.19165644)(969.53526367,54.23166311)
}
}
{
\newrgbcolor{curcolor}{0 0 0}
\pscustom[linestyle=none,fillstyle=solid,fillcolor=curcolor]
{
}
}
{
\newrgbcolor{curcolor}{0 0 0}
\pscustom[linestyle=none,fillstyle=solid,fillcolor=curcolor]
{
\newpath
\moveto(967.79526367,67.99510061)
\lineto(968.05026367,67.99510061)
\curveto(968.13026844,68.0050929)(968.20526836,68.00009291)(968.27526367,67.98010061)
\lineto(968.51526367,67.98010061)
\lineto(968.68026367,67.98010061)
\curveto(968.78026779,67.96009295)(968.88526768,67.95009296)(968.99526367,67.95010061)
\curveto(969.09526747,67.95009296)(969.19526737,67.94009297)(969.29526367,67.92010061)
\lineto(969.44526367,67.92010061)
\curveto(969.58526698,67.89009302)(969.72526684,67.87009304)(969.86526367,67.86010061)
\curveto(969.99526657,67.85009306)(970.12526644,67.82509308)(970.25526367,67.78510061)
\curveto(970.33526623,67.76509314)(970.42026615,67.74509316)(970.51026367,67.72510061)
\lineto(970.75026367,67.66510061)
\lineto(971.05026367,67.54510061)
\curveto(971.14026543,67.51509339)(971.23026534,67.48009343)(971.32026367,67.44010061)
\curveto(971.54026503,67.34009357)(971.75526481,67.2050937)(971.96526367,67.03510061)
\curveto(972.17526439,66.87509403)(972.34526422,66.70009421)(972.47526367,66.51010061)
\curveto(972.51526405,66.46009445)(972.55526401,66.40009451)(972.59526367,66.33010061)
\curveto(972.62526394,66.27009464)(972.66026391,66.2100947)(972.70026367,66.15010061)
\curveto(972.75026382,66.07009484)(972.79026378,65.97509493)(972.82026367,65.86510061)
\curveto(972.85026372,65.75509515)(972.88026369,65.65009526)(972.91026367,65.55010061)
\curveto(972.95026362,65.44009547)(972.97526359,65.33009558)(972.98526367,65.22010061)
\curveto(972.99526357,65.1100958)(973.01026356,64.99509591)(973.03026367,64.87510061)
\curveto(973.04026353,64.83509607)(973.04026353,64.79009612)(973.03026367,64.74010061)
\curveto(973.03026354,64.70009621)(973.03526353,64.66009625)(973.04526367,64.62010061)
\curveto(973.05526351,64.58009633)(973.06026351,64.52509638)(973.06026367,64.45510061)
\curveto(973.06026351,64.38509652)(973.05526351,64.33509657)(973.04526367,64.30510061)
\curveto(973.02526354,64.25509665)(973.02026355,64.2100967)(973.03026367,64.17010061)
\curveto(973.04026353,64.13009678)(973.04026353,64.09509681)(973.03026367,64.06510061)
\lineto(973.03026367,63.97510061)
\curveto(973.01026356,63.91509699)(972.99526357,63.85009706)(972.98526367,63.78010061)
\curveto(972.98526358,63.72009719)(972.98026359,63.65509725)(972.97026367,63.58510061)
\curveto(972.92026365,63.41509749)(972.8702637,63.25509765)(972.82026367,63.10510061)
\curveto(972.7702638,62.95509795)(972.70526386,62.8100981)(972.62526367,62.67010061)
\curveto(972.58526398,62.62009829)(972.55526401,62.56509834)(972.53526367,62.50510061)
\curveto(972.50526406,62.45509845)(972.4702641,62.4050985)(972.43026367,62.35510061)
\curveto(972.25026432,62.11509879)(972.03026454,61.91509899)(971.77026367,61.75510061)
\curveto(971.51026506,61.59509931)(971.22526534,61.45509945)(970.91526367,61.33510061)
\curveto(970.77526579,61.27509963)(970.63526593,61.23009968)(970.49526367,61.20010061)
\curveto(970.34526622,61.17009974)(970.19026638,61.13509977)(970.03026367,61.09510061)
\curveto(969.92026665,61.07509983)(969.81026676,61.06009985)(969.70026367,61.05010061)
\curveto(969.59026698,61.04009987)(969.48026709,61.02509988)(969.37026367,61.00510061)
\curveto(969.33026724,60.99509991)(969.29026728,60.99009992)(969.25026367,60.99010061)
\curveto(969.21026736,61.00009991)(969.1702674,61.00009991)(969.13026367,60.99010061)
\curveto(969.08026749,60.98009993)(969.03026754,60.97509993)(968.98026367,60.97510061)
\lineto(968.81526367,60.97510061)
\curveto(968.7652678,60.95509995)(968.71526785,60.95009996)(968.66526367,60.96010061)
\curveto(968.60526796,60.97009994)(968.55026802,60.97009994)(968.50026367,60.96010061)
\curveto(968.46026811,60.95009996)(968.41526815,60.95009996)(968.36526367,60.96010061)
\curveto(968.31526825,60.97009994)(968.2652683,60.96509994)(968.21526367,60.94510061)
\curveto(968.14526842,60.92509998)(968.0702685,60.92009999)(967.99026367,60.93010061)
\curveto(967.90026867,60.94009997)(967.81526875,60.94509996)(967.73526367,60.94510061)
\curveto(967.64526892,60.94509996)(967.54526902,60.94009997)(967.43526367,60.93010061)
\curveto(967.31526925,60.92009999)(967.21526935,60.92509998)(967.13526367,60.94510061)
\lineto(966.85026367,60.94510061)
\lineto(966.22026367,60.99010061)
\curveto(966.12027045,61.00009991)(966.02527054,61.0100999)(965.93526367,61.02010061)
\lineto(965.63526367,61.05010061)
\curveto(965.58527098,61.07009984)(965.53527103,61.07509983)(965.48526367,61.06510061)
\curveto(965.42527114,61.06509984)(965.3702712,61.07509983)(965.32026367,61.09510061)
\curveto(965.15027142,61.14509976)(964.98527158,61.18509972)(964.82526367,61.21510061)
\curveto(964.65527191,61.24509966)(964.49527207,61.29509961)(964.34526367,61.36510061)
\curveto(963.88527268,61.55509935)(963.51027306,61.77509913)(963.22026367,62.02510061)
\curveto(962.93027364,62.28509862)(962.68527388,62.64509826)(962.48526367,63.10510061)
\curveto(962.43527413,63.23509767)(962.40027417,63.36509754)(962.38026367,63.49510061)
\curveto(962.36027421,63.63509727)(962.33527423,63.77509713)(962.30526367,63.91510061)
\curveto(962.29527427,63.98509692)(962.29027428,64.05009686)(962.29026367,64.11010061)
\curveto(962.29027428,64.17009674)(962.28527428,64.23509667)(962.27526367,64.30510061)
\curveto(962.25527431,65.13509577)(962.40527416,65.8050951)(962.72526367,66.31510061)
\curveto(963.03527353,66.82509408)(963.47527309,67.2050937)(964.04526367,67.45510061)
\curveto(964.1652724,67.5050934)(964.29027228,67.55009336)(964.42026367,67.59010061)
\curveto(964.55027202,67.63009328)(964.68527188,67.67509323)(964.82526367,67.72510061)
\curveto(964.90527166,67.74509316)(964.99027158,67.76009315)(965.08026367,67.77010061)
\lineto(965.32026367,67.83010061)
\curveto(965.43027114,67.86009305)(965.54027103,67.87509303)(965.65026367,67.87510061)
\curveto(965.76027081,67.88509302)(965.8702707,67.90009301)(965.98026367,67.92010061)
\curveto(966.03027054,67.94009297)(966.07527049,67.94509296)(966.11526367,67.93510061)
\curveto(966.15527041,67.93509297)(966.19527037,67.94009297)(966.23526367,67.95010061)
\curveto(966.28527028,67.96009295)(966.34027023,67.96009295)(966.40026367,67.95010061)
\curveto(966.45027012,67.95009296)(966.50027007,67.95509295)(966.55026367,67.96510061)
\lineto(966.68526367,67.96510061)
\curveto(966.74526982,67.98509292)(966.81526975,67.98509292)(966.89526367,67.96510061)
\curveto(966.9652696,67.95509295)(967.03026954,67.96009295)(967.09026367,67.98010061)
\curveto(967.12026945,67.99009292)(967.16026941,67.99509291)(967.21026367,67.99510061)
\lineto(967.33026367,67.99510061)
\lineto(967.79526367,67.99510061)
\moveto(970.12026367,66.45010061)
\curveto(969.80026677,66.55009436)(969.43526713,66.6100943)(969.02526367,66.63010061)
\curveto(968.61526795,66.65009426)(968.20526836,66.66009425)(967.79526367,66.66010061)
\curveto(967.3652692,66.66009425)(966.94526962,66.65009426)(966.53526367,66.63010061)
\curveto(966.12527044,66.6100943)(965.74027083,66.56509434)(965.38026367,66.49510061)
\curveto(965.02027155,66.42509448)(964.70027187,66.31509459)(964.42026367,66.16510061)
\curveto(964.13027244,66.02509488)(963.89527267,65.83009508)(963.71526367,65.58010061)
\curveto(963.60527296,65.42009549)(963.52527304,65.24009567)(963.47526367,65.04010061)
\curveto(963.41527315,64.84009607)(963.38527318,64.59509631)(963.38526367,64.30510061)
\curveto(963.40527316,64.28509662)(963.41527315,64.25009666)(963.41526367,64.20010061)
\curveto(963.40527316,64.15009676)(963.40527316,64.1100968)(963.41526367,64.08010061)
\curveto(963.43527313,64.00009691)(963.45527311,63.92509698)(963.47526367,63.85510061)
\curveto(963.48527308,63.79509711)(963.50527306,63.73009718)(963.53526367,63.66010061)
\curveto(963.65527291,63.39009752)(963.82527274,63.17009774)(964.04526367,63.00010061)
\curveto(964.25527231,62.84009807)(964.50027207,62.7050982)(964.78026367,62.59510061)
\curveto(964.89027168,62.54509836)(965.01027156,62.5050984)(965.14026367,62.47510061)
\curveto(965.26027131,62.45509845)(965.38527118,62.43009848)(965.51526367,62.40010061)
\curveto(965.565271,62.38009853)(965.62027095,62.37009854)(965.68026367,62.37010061)
\curveto(965.73027084,62.37009854)(965.78027079,62.36509854)(965.83026367,62.35510061)
\curveto(965.92027065,62.34509856)(966.01527055,62.33509857)(966.11526367,62.32510061)
\curveto(966.20527036,62.31509859)(966.30027027,62.3050986)(966.40026367,62.29510061)
\curveto(966.48027009,62.29509861)(966.56527,62.29009862)(966.65526367,62.28010061)
\lineto(966.89526367,62.28010061)
\lineto(967.07526367,62.28010061)
\curveto(967.10526946,62.27009864)(967.14026943,62.26509864)(967.18026367,62.26510061)
\lineto(967.31526367,62.26510061)
\lineto(967.76526367,62.26510061)
\curveto(967.84526872,62.26509864)(967.93026864,62.26009865)(968.02026367,62.25010061)
\curveto(968.10026847,62.25009866)(968.17526839,62.26009865)(968.24526367,62.28010061)
\lineto(968.51526367,62.28010061)
\curveto(968.53526803,62.28009863)(968.565268,62.27509863)(968.60526367,62.26510061)
\curveto(968.63526793,62.26509864)(968.66026791,62.27009864)(968.68026367,62.28010061)
\curveto(968.78026779,62.29009862)(968.88026769,62.29509861)(968.98026367,62.29510061)
\curveto(969.0702675,62.3050986)(969.1702674,62.31509859)(969.28026367,62.32510061)
\curveto(969.40026717,62.35509855)(969.52526704,62.37009854)(969.65526367,62.37010061)
\curveto(969.77526679,62.38009853)(969.89026668,62.4050985)(970.00026367,62.44510061)
\curveto(970.30026627,62.52509838)(970.565266,62.6100983)(970.79526367,62.70010061)
\curveto(971.02526554,62.80009811)(971.24026533,62.94509796)(971.44026367,63.13510061)
\curveto(971.64026493,63.34509756)(971.79026478,63.6100973)(971.89026367,63.93010061)
\curveto(971.91026466,63.97009694)(971.92026465,64.0050969)(971.92026367,64.03510061)
\curveto(971.91026466,64.07509683)(971.91526465,64.12009679)(971.93526367,64.17010061)
\curveto(971.94526462,64.2100967)(971.95526461,64.28009663)(971.96526367,64.38010061)
\curveto(971.97526459,64.49009642)(971.9702646,64.57509633)(971.95026367,64.63510061)
\curveto(971.93026464,64.7050962)(971.92026465,64.77509613)(971.92026367,64.84510061)
\curveto(971.91026466,64.91509599)(971.89526467,64.98009593)(971.87526367,65.04010061)
\curveto(971.81526475,65.24009567)(971.73026484,65.42009549)(971.62026367,65.58010061)
\curveto(971.60026497,65.6100953)(971.58026499,65.63509527)(971.56026367,65.65510061)
\lineto(971.50026367,65.71510061)
\curveto(971.48026509,65.75509515)(971.44026513,65.8050951)(971.38026367,65.86510061)
\curveto(971.24026533,65.96509494)(971.11026546,66.05009486)(970.99026367,66.12010061)
\curveto(970.8702657,66.19009472)(970.72526584,66.26009465)(970.55526367,66.33010061)
\curveto(970.48526608,66.36009455)(970.41526615,66.38009453)(970.34526367,66.39010061)
\curveto(970.27526629,66.4100945)(970.20026637,66.43009448)(970.12026367,66.45010061)
}
}
{
\newrgbcolor{curcolor}{0 0 0}
\pscustom[linestyle=none,fillstyle=solid,fillcolor=curcolor]
{
\newpath
\moveto(962.27526367,73.40470998)
\curveto(962.27527429,73.50470513)(962.28527428,73.59970503)(962.30526367,73.68970998)
\curveto(962.31527425,73.77970485)(962.34527422,73.84470479)(962.39526367,73.88470998)
\curveto(962.47527409,73.94470469)(962.58027399,73.97470466)(962.71026367,73.97470998)
\lineto(963.10026367,73.97470998)
\lineto(964.60026367,73.97470998)
\lineto(970.99026367,73.97470998)
\lineto(972.16026367,73.97470998)
\lineto(972.47526367,73.97470998)
\curveto(972.57526399,73.98470465)(972.65526391,73.96970466)(972.71526367,73.92970998)
\curveto(972.79526377,73.87970475)(972.84526372,73.80470483)(972.86526367,73.70470998)
\curveto(972.87526369,73.61470502)(972.88026369,73.50470513)(972.88026367,73.37470998)
\lineto(972.88026367,73.14970998)
\curveto(972.86026371,73.06970556)(972.84526372,72.99970563)(972.83526367,72.93970998)
\curveto(972.81526375,72.87970575)(972.77526379,72.8297058)(972.71526367,72.78970998)
\curveto(972.65526391,72.74970588)(972.58026399,72.7297059)(972.49026367,72.72970998)
\lineto(972.19026367,72.72970998)
\lineto(971.09526367,72.72970998)
\lineto(965.75526367,72.72970998)
\curveto(965.6652709,72.70970592)(965.59027098,72.69470594)(965.53026367,72.68470998)
\curveto(965.46027111,72.68470595)(965.40027117,72.65470598)(965.35026367,72.59470998)
\curveto(965.30027127,72.52470611)(965.27527129,72.4347062)(965.27526367,72.32470998)
\curveto(965.2652713,72.22470641)(965.26027131,72.11470652)(965.26026367,71.99470998)
\lineto(965.26026367,70.85470998)
\lineto(965.26026367,70.35970998)
\curveto(965.25027132,70.19970843)(965.19027138,70.08970854)(965.08026367,70.02970998)
\curveto(965.05027152,70.00970862)(965.02027155,69.99970863)(964.99026367,69.99970998)
\curveto(964.95027162,69.99970863)(964.90527166,69.99470864)(964.85526367,69.98470998)
\curveto(964.73527183,69.96470867)(964.62527194,69.96970866)(964.52526367,69.99970998)
\curveto(964.42527214,70.03970859)(964.35527221,70.09470854)(964.31526367,70.16470998)
\curveto(964.2652723,70.24470839)(964.24027233,70.36470827)(964.24026367,70.52470998)
\curveto(964.24027233,70.68470795)(964.22527234,70.81970781)(964.19526367,70.92970998)
\curveto(964.18527238,70.97970765)(964.18027239,71.0347076)(964.18026367,71.09470998)
\curveto(964.1702724,71.15470748)(964.15527241,71.21470742)(964.13526367,71.27470998)
\curveto(964.08527248,71.42470721)(964.03527253,71.56970706)(963.98526367,71.70970998)
\curveto(963.92527264,71.84970678)(963.85527271,71.98470665)(963.77526367,72.11470998)
\curveto(963.68527288,72.25470638)(963.58027299,72.37470626)(963.46026367,72.47470998)
\curveto(963.34027323,72.57470606)(963.21027336,72.66970596)(963.07026367,72.75970998)
\curveto(962.9702736,72.81970581)(962.86027371,72.86470577)(962.74026367,72.89470998)
\curveto(962.62027395,72.9347057)(962.51527405,72.98470565)(962.42526367,73.04470998)
\curveto(962.3652742,73.09470554)(962.32527424,73.16470547)(962.30526367,73.25470998)
\curveto(962.29527427,73.27470536)(962.29027428,73.29970533)(962.29026367,73.32970998)
\curveto(962.29027428,73.35970527)(962.28527428,73.38470525)(962.27526367,73.40470998)
}
}
{
\newrgbcolor{curcolor}{0 0 0}
\pscustom[linestyle=none,fillstyle=solid,fillcolor=curcolor]
{
\newpath
\moveto(971.24526367,78.63431936)
\lineto(971.24526367,79.26431936)
\lineto(971.24526367,79.45931936)
\curveto(971.24526532,79.52931683)(971.25526531,79.58931677)(971.27526367,79.63931936)
\curveto(971.31526525,79.70931665)(971.35526521,79.7593166)(971.39526367,79.78931936)
\curveto(971.44526512,79.82931653)(971.51026506,79.84931651)(971.59026367,79.84931936)
\curveto(971.6702649,79.8593165)(971.75526481,79.86431649)(971.84526367,79.86431936)
\lineto(972.56526367,79.86431936)
\curveto(973.04526352,79.86431649)(973.45526311,79.80431655)(973.79526367,79.68431936)
\curveto(974.13526243,79.56431679)(974.41026216,79.36931699)(974.62026367,79.09931936)
\curveto(974.6702619,79.02931733)(974.71526185,78.9593174)(974.75526367,78.88931936)
\curveto(974.80526176,78.82931753)(974.85026172,78.7543176)(974.89026367,78.66431936)
\curveto(974.90026167,78.64431771)(974.91026166,78.61431774)(974.92026367,78.57431936)
\curveto(974.94026163,78.53431782)(974.94526162,78.48931787)(974.93526367,78.43931936)
\curveto(974.90526166,78.34931801)(974.83026174,78.29431806)(974.71026367,78.27431936)
\curveto(974.60026197,78.2543181)(974.50526206,78.26931809)(974.42526367,78.31931936)
\curveto(974.35526221,78.34931801)(974.29026228,78.39431796)(974.23026367,78.45431936)
\curveto(974.18026239,78.52431783)(974.13026244,78.58931777)(974.08026367,78.64931936)
\curveto(974.03026254,78.71931764)(973.95526261,78.77931758)(973.85526367,78.82931936)
\curveto(973.7652628,78.88931747)(973.67526289,78.93931742)(973.58526367,78.97931936)
\curveto(973.55526301,78.99931736)(973.49526307,79.02431733)(973.40526367,79.05431936)
\curveto(973.32526324,79.08431727)(973.25526331,79.08931727)(973.19526367,79.06931936)
\curveto(973.05526351,79.03931732)(972.9652636,78.97931738)(972.92526367,78.88931936)
\curveto(972.89526367,78.80931755)(972.88026369,78.71931764)(972.88026367,78.61931936)
\curveto(972.88026369,78.51931784)(972.85526371,78.43431792)(972.80526367,78.36431936)
\curveto(972.73526383,78.27431808)(972.59526397,78.22931813)(972.38526367,78.22931936)
\lineto(971.83026367,78.22931936)
\lineto(971.60526367,78.22931936)
\curveto(971.52526504,78.23931812)(971.46026511,78.2593181)(971.41026367,78.28931936)
\curveto(971.33026524,78.34931801)(971.28526528,78.41931794)(971.27526367,78.49931936)
\curveto(971.2652653,78.51931784)(971.26026531,78.53931782)(971.26026367,78.55931936)
\curveto(971.26026531,78.58931777)(971.25526531,78.61431774)(971.24526367,78.63431936)
}
}
{
\newrgbcolor{curcolor}{0 0 0}
\pscustom[linestyle=none,fillstyle=solid,fillcolor=curcolor]
{
}
}
{
\newrgbcolor{curcolor}{0 0 0}
\pscustom[linestyle=none,fillstyle=solid,fillcolor=curcolor]
{
\newpath
\moveto(962.27526367,89.26463186)
\curveto(962.2652743,89.95462722)(962.38527418,90.55462662)(962.63526367,91.06463186)
\curveto(962.88527368,91.58462559)(963.22027335,91.9796252)(963.64026367,92.24963186)
\curveto(963.72027285,92.29962488)(963.81027276,92.34462483)(963.91026367,92.38463186)
\curveto(964.00027257,92.42462475)(964.09527247,92.46962471)(964.19526367,92.51963186)
\curveto(964.29527227,92.55962462)(964.39527217,92.58962459)(964.49526367,92.60963186)
\curveto(964.59527197,92.62962455)(964.70027187,92.64962453)(964.81026367,92.66963186)
\curveto(964.86027171,92.68962449)(964.90527166,92.69462448)(964.94526367,92.68463186)
\curveto(964.98527158,92.6746245)(965.03027154,92.6796245)(965.08026367,92.69963186)
\curveto(965.13027144,92.70962447)(965.21527135,92.71462446)(965.33526367,92.71463186)
\curveto(965.44527112,92.71462446)(965.53027104,92.70962447)(965.59026367,92.69963186)
\curveto(965.65027092,92.6796245)(965.71027086,92.66962451)(965.77026367,92.66963186)
\curveto(965.83027074,92.6796245)(965.89027068,92.6746245)(965.95026367,92.65463186)
\curveto(966.09027048,92.61462456)(966.22527034,92.5796246)(966.35526367,92.54963186)
\curveto(966.48527008,92.51962466)(966.61026996,92.4796247)(966.73026367,92.42963186)
\curveto(966.8702697,92.36962481)(966.99526957,92.29962488)(967.10526367,92.21963186)
\curveto(967.21526935,92.14962503)(967.32526924,92.0746251)(967.43526367,91.99463186)
\lineto(967.49526367,91.93463186)
\curveto(967.51526905,91.92462525)(967.53526903,91.90962527)(967.55526367,91.88963186)
\curveto(967.71526885,91.76962541)(967.86026871,91.63462554)(967.99026367,91.48463186)
\curveto(968.12026845,91.33462584)(968.24526832,91.174626)(968.36526367,91.00463186)
\curveto(968.58526798,90.69462648)(968.79026778,90.39962678)(968.98026367,90.11963186)
\curveto(969.12026745,89.88962729)(969.25526731,89.65962752)(969.38526367,89.42963186)
\curveto(969.51526705,89.20962797)(969.65026692,88.98962819)(969.79026367,88.76963186)
\curveto(969.96026661,88.51962866)(970.14026643,88.2796289)(970.33026367,88.04963186)
\curveto(970.52026605,87.82962935)(970.74526582,87.63962954)(971.00526367,87.47963186)
\curveto(971.0652655,87.43962974)(971.12526544,87.40462977)(971.18526367,87.37463186)
\curveto(971.23526533,87.34462983)(971.30026527,87.31462986)(971.38026367,87.28463186)
\curveto(971.45026512,87.26462991)(971.51026506,87.25962992)(971.56026367,87.26963186)
\curveto(971.63026494,87.28962989)(971.68526488,87.32462985)(971.72526367,87.37463186)
\curveto(971.75526481,87.42462975)(971.77526479,87.48462969)(971.78526367,87.55463186)
\lineto(971.78526367,87.79463186)
\lineto(971.78526367,88.54463186)
\lineto(971.78526367,91.34963186)
\lineto(971.78526367,92.00963186)
\curveto(971.78526478,92.09962508)(971.79026478,92.18462499)(971.80026367,92.26463186)
\curveto(971.80026477,92.34462483)(971.82026475,92.40962477)(971.86026367,92.45963186)
\curveto(971.90026467,92.50962467)(971.97526459,92.54962463)(972.08526367,92.57963186)
\curveto(972.18526438,92.61962456)(972.28526428,92.62962455)(972.38526367,92.60963186)
\lineto(972.52026367,92.60963186)
\curveto(972.59026398,92.58962459)(972.65026392,92.56962461)(972.70026367,92.54963186)
\curveto(972.75026382,92.52962465)(972.79026378,92.49462468)(972.82026367,92.44463186)
\curveto(972.86026371,92.39462478)(972.88026369,92.32462485)(972.88026367,92.23463186)
\lineto(972.88026367,91.96463186)
\lineto(972.88026367,91.06463186)
\lineto(972.88026367,87.55463186)
\lineto(972.88026367,86.48963186)
\curveto(972.88026369,86.40963077)(972.88526368,86.31963086)(972.89526367,86.21963186)
\curveto(972.89526367,86.11963106)(972.88526368,86.03463114)(972.86526367,85.96463186)
\curveto(972.79526377,85.75463142)(972.61526395,85.68963149)(972.32526367,85.76963186)
\curveto(972.28526428,85.7796314)(972.25026432,85.7796314)(972.22026367,85.76963186)
\curveto(972.18026439,85.76963141)(972.13526443,85.7796314)(972.08526367,85.79963186)
\curveto(972.00526456,85.81963136)(971.92026465,85.83963134)(971.83026367,85.85963186)
\curveto(971.74026483,85.8796313)(971.65526491,85.90463127)(971.57526367,85.93463186)
\curveto(971.08526548,86.09463108)(970.6702659,86.29463088)(970.33026367,86.53463186)
\curveto(970.08026649,86.71463046)(969.85526671,86.91963026)(969.65526367,87.14963186)
\curveto(969.44526712,87.3796298)(969.25026732,87.61962956)(969.07026367,87.86963186)
\curveto(968.89026768,88.12962905)(968.72026785,88.39462878)(968.56026367,88.66463186)
\curveto(968.39026818,88.94462823)(968.21526835,89.21462796)(968.03526367,89.47463186)
\curveto(967.95526861,89.58462759)(967.88026869,89.68962749)(967.81026367,89.78963186)
\curveto(967.74026883,89.89962728)(967.6652689,90.00962717)(967.58526367,90.11963186)
\curveto(967.55526901,90.15962702)(967.52526904,90.19462698)(967.49526367,90.22463186)
\curveto(967.45526911,90.26462691)(967.42526914,90.30462687)(967.40526367,90.34463186)
\curveto(967.29526927,90.48462669)(967.1702694,90.60962657)(967.03026367,90.71963186)
\curveto(967.00026957,90.73962644)(966.97526959,90.76462641)(966.95526367,90.79463186)
\curveto(966.92526964,90.82462635)(966.89526967,90.84962633)(966.86526367,90.86963186)
\curveto(966.7652698,90.94962623)(966.6652699,91.01462616)(966.56526367,91.06463186)
\curveto(966.4652701,91.12462605)(966.35527021,91.179626)(966.23526367,91.22963186)
\curveto(966.1652704,91.25962592)(966.09027048,91.2796259)(966.01026367,91.28963186)
\lineto(965.77026367,91.34963186)
\lineto(965.68026367,91.34963186)
\curveto(965.65027092,91.35962582)(965.62027095,91.36462581)(965.59026367,91.36463186)
\curveto(965.52027105,91.38462579)(965.42527114,91.38962579)(965.30526367,91.37963186)
\curveto(965.17527139,91.3796258)(965.07527149,91.36962581)(965.00526367,91.34963186)
\curveto(964.92527164,91.32962585)(964.85027172,91.30962587)(964.78026367,91.28963186)
\curveto(964.70027187,91.2796259)(964.62027195,91.25962592)(964.54026367,91.22963186)
\curveto(964.30027227,91.11962606)(964.10027247,90.96962621)(963.94026367,90.77963186)
\curveto(963.7702728,90.59962658)(963.63027294,90.3796268)(963.52026367,90.11963186)
\curveto(963.50027307,90.04962713)(963.48527308,89.9796272)(963.47526367,89.90963186)
\curveto(963.45527311,89.83962734)(963.43527313,89.76462741)(963.41526367,89.68463186)
\curveto(963.39527317,89.60462757)(963.38527318,89.49462768)(963.38526367,89.35463186)
\curveto(963.38527318,89.22462795)(963.39527317,89.11962806)(963.41526367,89.03963186)
\curveto(963.42527314,88.9796282)(963.43027314,88.92462825)(963.43026367,88.87463186)
\curveto(963.43027314,88.82462835)(963.44027313,88.7746284)(963.46026367,88.72463186)
\curveto(963.50027307,88.62462855)(963.54027303,88.52962865)(963.58026367,88.43963186)
\curveto(963.62027295,88.35962882)(963.6652729,88.2796289)(963.71526367,88.19963186)
\curveto(963.73527283,88.16962901)(963.76027281,88.13962904)(963.79026367,88.10963186)
\curveto(963.82027275,88.08962909)(963.84527272,88.06462911)(963.86526367,88.03463186)
\lineto(963.94026367,87.95963186)
\curveto(963.96027261,87.92962925)(963.98027259,87.90462927)(964.00026367,87.88463186)
\lineto(964.21026367,87.73463186)
\curveto(964.2702723,87.69462948)(964.33527223,87.64962953)(964.40526367,87.59963186)
\curveto(964.49527207,87.53962964)(964.60027197,87.48962969)(964.72026367,87.44963186)
\curveto(964.83027174,87.41962976)(964.94027163,87.38462979)(965.05026367,87.34463186)
\curveto(965.16027141,87.30462987)(965.30527126,87.2796299)(965.48526367,87.26963186)
\curveto(965.65527091,87.25962992)(965.78027079,87.22962995)(965.86026367,87.17963186)
\curveto(965.94027063,87.12963005)(965.98527058,87.05463012)(965.99526367,86.95463186)
\curveto(966.00527056,86.85463032)(966.01027056,86.74463043)(966.01026367,86.62463186)
\curveto(966.01027056,86.58463059)(966.01527055,86.54463063)(966.02526367,86.50463186)
\curveto(966.02527054,86.46463071)(966.02027055,86.42963075)(966.01026367,86.39963186)
\curveto(965.99027058,86.34963083)(965.98027059,86.29963088)(965.98026367,86.24963186)
\curveto(965.98027059,86.20963097)(965.9702706,86.16963101)(965.95026367,86.12963186)
\curveto(965.89027068,86.03963114)(965.75527081,85.99463118)(965.54526367,85.99463186)
\lineto(965.42526367,85.99463186)
\curveto(965.3652712,86.00463117)(965.30527126,86.00963117)(965.24526367,86.00963186)
\curveto(965.17527139,86.01963116)(965.11027146,86.02963115)(965.05026367,86.03963186)
\curveto(964.94027163,86.05963112)(964.84027173,86.0796311)(964.75026367,86.09963186)
\curveto(964.65027192,86.11963106)(964.55527201,86.14963103)(964.46526367,86.18963186)
\curveto(964.39527217,86.20963097)(964.33527223,86.22963095)(964.28526367,86.24963186)
\lineto(964.10526367,86.30963186)
\curveto(963.84527272,86.42963075)(963.60027297,86.58463059)(963.37026367,86.77463186)
\curveto(963.14027343,86.9746302)(962.95527361,87.18962999)(962.81526367,87.41963186)
\curveto(962.73527383,87.52962965)(962.6702739,87.64462953)(962.62026367,87.76463186)
\lineto(962.47026367,88.15463186)
\curveto(962.42027415,88.26462891)(962.39027418,88.3796288)(962.38026367,88.49963186)
\curveto(962.36027421,88.61962856)(962.33527423,88.74462843)(962.30526367,88.87463186)
\curveto(962.30527426,88.94462823)(962.30527426,89.00962817)(962.30526367,89.06963186)
\curveto(962.29527427,89.12962805)(962.28527428,89.19462798)(962.27526367,89.26463186)
}
}
{
\newrgbcolor{curcolor}{0 0 0}
\pscustom[linestyle=none,fillstyle=solid,fillcolor=curcolor]
{
\newpath
\moveto(967.79526367,101.36424123)
\lineto(968.05026367,101.36424123)
\curveto(968.13026844,101.37423353)(968.20526836,101.36923353)(968.27526367,101.34924123)
\lineto(968.51526367,101.34924123)
\lineto(968.68026367,101.34924123)
\curveto(968.78026779,101.32923357)(968.88526768,101.31923358)(968.99526367,101.31924123)
\curveto(969.09526747,101.31923358)(969.19526737,101.30923359)(969.29526367,101.28924123)
\lineto(969.44526367,101.28924123)
\curveto(969.58526698,101.25923364)(969.72526684,101.23923366)(969.86526367,101.22924123)
\curveto(969.99526657,101.21923368)(970.12526644,101.19423371)(970.25526367,101.15424123)
\curveto(970.33526623,101.13423377)(970.42026615,101.11423379)(970.51026367,101.09424123)
\lineto(970.75026367,101.03424123)
\lineto(971.05026367,100.91424123)
\curveto(971.14026543,100.88423402)(971.23026534,100.84923405)(971.32026367,100.80924123)
\curveto(971.54026503,100.70923419)(971.75526481,100.57423433)(971.96526367,100.40424123)
\curveto(972.17526439,100.24423466)(972.34526422,100.06923483)(972.47526367,99.87924123)
\curveto(972.51526405,99.82923507)(972.55526401,99.76923513)(972.59526367,99.69924123)
\curveto(972.62526394,99.63923526)(972.66026391,99.57923532)(972.70026367,99.51924123)
\curveto(972.75026382,99.43923546)(972.79026378,99.34423556)(972.82026367,99.23424123)
\curveto(972.85026372,99.12423578)(972.88026369,99.01923588)(972.91026367,98.91924123)
\curveto(972.95026362,98.80923609)(972.97526359,98.6992362)(972.98526367,98.58924123)
\curveto(972.99526357,98.47923642)(973.01026356,98.36423654)(973.03026367,98.24424123)
\curveto(973.04026353,98.2042367)(973.04026353,98.15923674)(973.03026367,98.10924123)
\curveto(973.03026354,98.06923683)(973.03526353,98.02923687)(973.04526367,97.98924123)
\curveto(973.05526351,97.94923695)(973.06026351,97.89423701)(973.06026367,97.82424123)
\curveto(973.06026351,97.75423715)(973.05526351,97.7042372)(973.04526367,97.67424123)
\curveto(973.02526354,97.62423728)(973.02026355,97.57923732)(973.03026367,97.53924123)
\curveto(973.04026353,97.4992374)(973.04026353,97.46423744)(973.03026367,97.43424123)
\lineto(973.03026367,97.34424123)
\curveto(973.01026356,97.28423762)(972.99526357,97.21923768)(972.98526367,97.14924123)
\curveto(972.98526358,97.08923781)(972.98026359,97.02423788)(972.97026367,96.95424123)
\curveto(972.92026365,96.78423812)(972.8702637,96.62423828)(972.82026367,96.47424123)
\curveto(972.7702638,96.32423858)(972.70526386,96.17923872)(972.62526367,96.03924123)
\curveto(972.58526398,95.98923891)(972.55526401,95.93423897)(972.53526367,95.87424123)
\curveto(972.50526406,95.82423908)(972.4702641,95.77423913)(972.43026367,95.72424123)
\curveto(972.25026432,95.48423942)(972.03026454,95.28423962)(971.77026367,95.12424123)
\curveto(971.51026506,94.96423994)(971.22526534,94.82424008)(970.91526367,94.70424123)
\curveto(970.77526579,94.64424026)(970.63526593,94.5992403)(970.49526367,94.56924123)
\curveto(970.34526622,94.53924036)(970.19026638,94.5042404)(970.03026367,94.46424123)
\curveto(969.92026665,94.44424046)(969.81026676,94.42924047)(969.70026367,94.41924123)
\curveto(969.59026698,94.40924049)(969.48026709,94.39424051)(969.37026367,94.37424123)
\curveto(969.33026724,94.36424054)(969.29026728,94.35924054)(969.25026367,94.35924123)
\curveto(969.21026736,94.36924053)(969.1702674,94.36924053)(969.13026367,94.35924123)
\curveto(969.08026749,94.34924055)(969.03026754,94.34424056)(968.98026367,94.34424123)
\lineto(968.81526367,94.34424123)
\curveto(968.7652678,94.32424058)(968.71526785,94.31924058)(968.66526367,94.32924123)
\curveto(968.60526796,94.33924056)(968.55026802,94.33924056)(968.50026367,94.32924123)
\curveto(968.46026811,94.31924058)(968.41526815,94.31924058)(968.36526367,94.32924123)
\curveto(968.31526825,94.33924056)(968.2652683,94.33424057)(968.21526367,94.31424123)
\curveto(968.14526842,94.29424061)(968.0702685,94.28924061)(967.99026367,94.29924123)
\curveto(967.90026867,94.30924059)(967.81526875,94.31424059)(967.73526367,94.31424123)
\curveto(967.64526892,94.31424059)(967.54526902,94.30924059)(967.43526367,94.29924123)
\curveto(967.31526925,94.28924061)(967.21526935,94.29424061)(967.13526367,94.31424123)
\lineto(966.85026367,94.31424123)
\lineto(966.22026367,94.35924123)
\curveto(966.12027045,94.36924053)(966.02527054,94.37924052)(965.93526367,94.38924123)
\lineto(965.63526367,94.41924123)
\curveto(965.58527098,94.43924046)(965.53527103,94.44424046)(965.48526367,94.43424123)
\curveto(965.42527114,94.43424047)(965.3702712,94.44424046)(965.32026367,94.46424123)
\curveto(965.15027142,94.51424039)(964.98527158,94.55424035)(964.82526367,94.58424123)
\curveto(964.65527191,94.61424029)(964.49527207,94.66424024)(964.34526367,94.73424123)
\curveto(963.88527268,94.92423998)(963.51027306,95.14423976)(963.22026367,95.39424123)
\curveto(962.93027364,95.65423925)(962.68527388,96.01423889)(962.48526367,96.47424123)
\curveto(962.43527413,96.6042383)(962.40027417,96.73423817)(962.38026367,96.86424123)
\curveto(962.36027421,97.0042379)(962.33527423,97.14423776)(962.30526367,97.28424123)
\curveto(962.29527427,97.35423755)(962.29027428,97.41923748)(962.29026367,97.47924123)
\curveto(962.29027428,97.53923736)(962.28527428,97.6042373)(962.27526367,97.67424123)
\curveto(962.25527431,98.5042364)(962.40527416,99.17423573)(962.72526367,99.68424123)
\curveto(963.03527353,100.19423471)(963.47527309,100.57423433)(964.04526367,100.82424123)
\curveto(964.1652724,100.87423403)(964.29027228,100.91923398)(964.42026367,100.95924123)
\curveto(964.55027202,100.9992339)(964.68527188,101.04423386)(964.82526367,101.09424123)
\curveto(964.90527166,101.11423379)(964.99027158,101.12923377)(965.08026367,101.13924123)
\lineto(965.32026367,101.19924123)
\curveto(965.43027114,101.22923367)(965.54027103,101.24423366)(965.65026367,101.24424123)
\curveto(965.76027081,101.25423365)(965.8702707,101.26923363)(965.98026367,101.28924123)
\curveto(966.03027054,101.30923359)(966.07527049,101.31423359)(966.11526367,101.30424123)
\curveto(966.15527041,101.3042336)(966.19527037,101.30923359)(966.23526367,101.31924123)
\curveto(966.28527028,101.32923357)(966.34027023,101.32923357)(966.40026367,101.31924123)
\curveto(966.45027012,101.31923358)(966.50027007,101.32423358)(966.55026367,101.33424123)
\lineto(966.68526367,101.33424123)
\curveto(966.74526982,101.35423355)(966.81526975,101.35423355)(966.89526367,101.33424123)
\curveto(966.9652696,101.32423358)(967.03026954,101.32923357)(967.09026367,101.34924123)
\curveto(967.12026945,101.35923354)(967.16026941,101.36423354)(967.21026367,101.36424123)
\lineto(967.33026367,101.36424123)
\lineto(967.79526367,101.36424123)
\moveto(970.12026367,99.81924123)
\curveto(969.80026677,99.91923498)(969.43526713,99.97923492)(969.02526367,99.99924123)
\curveto(968.61526795,100.01923488)(968.20526836,100.02923487)(967.79526367,100.02924123)
\curveto(967.3652692,100.02923487)(966.94526962,100.01923488)(966.53526367,99.99924123)
\curveto(966.12527044,99.97923492)(965.74027083,99.93423497)(965.38026367,99.86424123)
\curveto(965.02027155,99.79423511)(964.70027187,99.68423522)(964.42026367,99.53424123)
\curveto(964.13027244,99.39423551)(963.89527267,99.1992357)(963.71526367,98.94924123)
\curveto(963.60527296,98.78923611)(963.52527304,98.60923629)(963.47526367,98.40924123)
\curveto(963.41527315,98.20923669)(963.38527318,97.96423694)(963.38526367,97.67424123)
\curveto(963.40527316,97.65423725)(963.41527315,97.61923728)(963.41526367,97.56924123)
\curveto(963.40527316,97.51923738)(963.40527316,97.47923742)(963.41526367,97.44924123)
\curveto(963.43527313,97.36923753)(963.45527311,97.29423761)(963.47526367,97.22424123)
\curveto(963.48527308,97.16423774)(963.50527306,97.0992378)(963.53526367,97.02924123)
\curveto(963.65527291,96.75923814)(963.82527274,96.53923836)(964.04526367,96.36924123)
\curveto(964.25527231,96.20923869)(964.50027207,96.07423883)(964.78026367,95.96424123)
\curveto(964.89027168,95.91423899)(965.01027156,95.87423903)(965.14026367,95.84424123)
\curveto(965.26027131,95.82423908)(965.38527118,95.7992391)(965.51526367,95.76924123)
\curveto(965.565271,95.74923915)(965.62027095,95.73923916)(965.68026367,95.73924123)
\curveto(965.73027084,95.73923916)(965.78027079,95.73423917)(965.83026367,95.72424123)
\curveto(965.92027065,95.71423919)(966.01527055,95.7042392)(966.11526367,95.69424123)
\curveto(966.20527036,95.68423922)(966.30027027,95.67423923)(966.40026367,95.66424123)
\curveto(966.48027009,95.66423924)(966.56527,95.65923924)(966.65526367,95.64924123)
\lineto(966.89526367,95.64924123)
\lineto(967.07526367,95.64924123)
\curveto(967.10526946,95.63923926)(967.14026943,95.63423927)(967.18026367,95.63424123)
\lineto(967.31526367,95.63424123)
\lineto(967.76526367,95.63424123)
\curveto(967.84526872,95.63423927)(967.93026864,95.62923927)(968.02026367,95.61924123)
\curveto(968.10026847,95.61923928)(968.17526839,95.62923927)(968.24526367,95.64924123)
\lineto(968.51526367,95.64924123)
\curveto(968.53526803,95.64923925)(968.565268,95.64423926)(968.60526367,95.63424123)
\curveto(968.63526793,95.63423927)(968.66026791,95.63923926)(968.68026367,95.64924123)
\curveto(968.78026779,95.65923924)(968.88026769,95.66423924)(968.98026367,95.66424123)
\curveto(969.0702675,95.67423923)(969.1702674,95.68423922)(969.28026367,95.69424123)
\curveto(969.40026717,95.72423918)(969.52526704,95.73923916)(969.65526367,95.73924123)
\curveto(969.77526679,95.74923915)(969.89026668,95.77423913)(970.00026367,95.81424123)
\curveto(970.30026627,95.89423901)(970.565266,95.97923892)(970.79526367,96.06924123)
\curveto(971.02526554,96.16923873)(971.24026533,96.31423859)(971.44026367,96.50424123)
\curveto(971.64026493,96.71423819)(971.79026478,96.97923792)(971.89026367,97.29924123)
\curveto(971.91026466,97.33923756)(971.92026465,97.37423753)(971.92026367,97.40424123)
\curveto(971.91026466,97.44423746)(971.91526465,97.48923741)(971.93526367,97.53924123)
\curveto(971.94526462,97.57923732)(971.95526461,97.64923725)(971.96526367,97.74924123)
\curveto(971.97526459,97.85923704)(971.9702646,97.94423696)(971.95026367,98.00424123)
\curveto(971.93026464,98.07423683)(971.92026465,98.14423676)(971.92026367,98.21424123)
\curveto(971.91026466,98.28423662)(971.89526467,98.34923655)(971.87526367,98.40924123)
\curveto(971.81526475,98.60923629)(971.73026484,98.78923611)(971.62026367,98.94924123)
\curveto(971.60026497,98.97923592)(971.58026499,99.0042359)(971.56026367,99.02424123)
\lineto(971.50026367,99.08424123)
\curveto(971.48026509,99.12423578)(971.44026513,99.17423573)(971.38026367,99.23424123)
\curveto(971.24026533,99.33423557)(971.11026546,99.41923548)(970.99026367,99.48924123)
\curveto(970.8702657,99.55923534)(970.72526584,99.62923527)(970.55526367,99.69924123)
\curveto(970.48526608,99.72923517)(970.41526615,99.74923515)(970.34526367,99.75924123)
\curveto(970.27526629,99.77923512)(970.20026637,99.7992351)(970.12026367,99.81924123)
}
}
{
\newrgbcolor{curcolor}{0 0 0}
\pscustom[linestyle=none,fillstyle=solid,fillcolor=curcolor]
{
\newpath
\moveto(962.27526367,106.77385061)
\curveto(962.27527429,106.87384575)(962.28527428,106.96884566)(962.30526367,107.05885061)
\curveto(962.31527425,107.14884548)(962.34527422,107.21384541)(962.39526367,107.25385061)
\curveto(962.47527409,107.31384531)(962.58027399,107.34384528)(962.71026367,107.34385061)
\lineto(963.10026367,107.34385061)
\lineto(964.60026367,107.34385061)
\lineto(970.99026367,107.34385061)
\lineto(972.16026367,107.34385061)
\lineto(972.47526367,107.34385061)
\curveto(972.57526399,107.35384527)(972.65526391,107.33884529)(972.71526367,107.29885061)
\curveto(972.79526377,107.24884538)(972.84526372,107.17384545)(972.86526367,107.07385061)
\curveto(972.87526369,106.98384564)(972.88026369,106.87384575)(972.88026367,106.74385061)
\lineto(972.88026367,106.51885061)
\curveto(972.86026371,106.43884619)(972.84526372,106.36884626)(972.83526367,106.30885061)
\curveto(972.81526375,106.24884638)(972.77526379,106.19884643)(972.71526367,106.15885061)
\curveto(972.65526391,106.11884651)(972.58026399,106.09884653)(972.49026367,106.09885061)
\lineto(972.19026367,106.09885061)
\lineto(971.09526367,106.09885061)
\lineto(965.75526367,106.09885061)
\curveto(965.6652709,106.07884655)(965.59027098,106.06384656)(965.53026367,106.05385061)
\curveto(965.46027111,106.05384657)(965.40027117,106.0238466)(965.35026367,105.96385061)
\curveto(965.30027127,105.89384673)(965.27527129,105.80384682)(965.27526367,105.69385061)
\curveto(965.2652713,105.59384703)(965.26027131,105.48384714)(965.26026367,105.36385061)
\lineto(965.26026367,104.22385061)
\lineto(965.26026367,103.72885061)
\curveto(965.25027132,103.56884906)(965.19027138,103.45884917)(965.08026367,103.39885061)
\curveto(965.05027152,103.37884925)(965.02027155,103.36884926)(964.99026367,103.36885061)
\curveto(964.95027162,103.36884926)(964.90527166,103.36384926)(964.85526367,103.35385061)
\curveto(964.73527183,103.33384929)(964.62527194,103.33884929)(964.52526367,103.36885061)
\curveto(964.42527214,103.40884922)(964.35527221,103.46384916)(964.31526367,103.53385061)
\curveto(964.2652723,103.61384901)(964.24027233,103.73384889)(964.24026367,103.89385061)
\curveto(964.24027233,104.05384857)(964.22527234,104.18884844)(964.19526367,104.29885061)
\curveto(964.18527238,104.34884828)(964.18027239,104.40384822)(964.18026367,104.46385061)
\curveto(964.1702724,104.5238481)(964.15527241,104.58384804)(964.13526367,104.64385061)
\curveto(964.08527248,104.79384783)(964.03527253,104.93884769)(963.98526367,105.07885061)
\curveto(963.92527264,105.21884741)(963.85527271,105.35384727)(963.77526367,105.48385061)
\curveto(963.68527288,105.623847)(963.58027299,105.74384688)(963.46026367,105.84385061)
\curveto(963.34027323,105.94384668)(963.21027336,106.03884659)(963.07026367,106.12885061)
\curveto(962.9702736,106.18884644)(962.86027371,106.23384639)(962.74026367,106.26385061)
\curveto(962.62027395,106.30384632)(962.51527405,106.35384627)(962.42526367,106.41385061)
\curveto(962.3652742,106.46384616)(962.32527424,106.53384609)(962.30526367,106.62385061)
\curveto(962.29527427,106.64384598)(962.29027428,106.66884596)(962.29026367,106.69885061)
\curveto(962.29027428,106.7288459)(962.28527428,106.75384587)(962.27526367,106.77385061)
}
}
{
\newrgbcolor{curcolor}{0 0 0}
\pscustom[linestyle=none,fillstyle=solid,fillcolor=curcolor]
{
\newpath
\moveto(962.27526367,115.12345998)
\curveto(962.27527429,115.22345513)(962.28527428,115.31845503)(962.30526367,115.40845998)
\curveto(962.31527425,115.49845485)(962.34527422,115.56345479)(962.39526367,115.60345998)
\curveto(962.47527409,115.66345469)(962.58027399,115.69345466)(962.71026367,115.69345998)
\lineto(963.10026367,115.69345998)
\lineto(964.60026367,115.69345998)
\lineto(970.99026367,115.69345998)
\lineto(972.16026367,115.69345998)
\lineto(972.47526367,115.69345998)
\curveto(972.57526399,115.70345465)(972.65526391,115.68845466)(972.71526367,115.64845998)
\curveto(972.79526377,115.59845475)(972.84526372,115.52345483)(972.86526367,115.42345998)
\curveto(972.87526369,115.33345502)(972.88026369,115.22345513)(972.88026367,115.09345998)
\lineto(972.88026367,114.86845998)
\curveto(972.86026371,114.78845556)(972.84526372,114.71845563)(972.83526367,114.65845998)
\curveto(972.81526375,114.59845575)(972.77526379,114.5484558)(972.71526367,114.50845998)
\curveto(972.65526391,114.46845588)(972.58026399,114.4484559)(972.49026367,114.44845998)
\lineto(972.19026367,114.44845998)
\lineto(971.09526367,114.44845998)
\lineto(965.75526367,114.44845998)
\curveto(965.6652709,114.42845592)(965.59027098,114.41345594)(965.53026367,114.40345998)
\curveto(965.46027111,114.40345595)(965.40027117,114.37345598)(965.35026367,114.31345998)
\curveto(965.30027127,114.24345611)(965.27527129,114.1534562)(965.27526367,114.04345998)
\curveto(965.2652713,113.94345641)(965.26027131,113.83345652)(965.26026367,113.71345998)
\lineto(965.26026367,112.57345998)
\lineto(965.26026367,112.07845998)
\curveto(965.25027132,111.91845843)(965.19027138,111.80845854)(965.08026367,111.74845998)
\curveto(965.05027152,111.72845862)(965.02027155,111.71845863)(964.99026367,111.71845998)
\curveto(964.95027162,111.71845863)(964.90527166,111.71345864)(964.85526367,111.70345998)
\curveto(964.73527183,111.68345867)(964.62527194,111.68845866)(964.52526367,111.71845998)
\curveto(964.42527214,111.75845859)(964.35527221,111.81345854)(964.31526367,111.88345998)
\curveto(964.2652723,111.96345839)(964.24027233,112.08345827)(964.24026367,112.24345998)
\curveto(964.24027233,112.40345795)(964.22527234,112.53845781)(964.19526367,112.64845998)
\curveto(964.18527238,112.69845765)(964.18027239,112.7534576)(964.18026367,112.81345998)
\curveto(964.1702724,112.87345748)(964.15527241,112.93345742)(964.13526367,112.99345998)
\curveto(964.08527248,113.14345721)(964.03527253,113.28845706)(963.98526367,113.42845998)
\curveto(963.92527264,113.56845678)(963.85527271,113.70345665)(963.77526367,113.83345998)
\curveto(963.68527288,113.97345638)(963.58027299,114.09345626)(963.46026367,114.19345998)
\curveto(963.34027323,114.29345606)(963.21027336,114.38845596)(963.07026367,114.47845998)
\curveto(962.9702736,114.53845581)(962.86027371,114.58345577)(962.74026367,114.61345998)
\curveto(962.62027395,114.6534557)(962.51527405,114.70345565)(962.42526367,114.76345998)
\curveto(962.3652742,114.81345554)(962.32527424,114.88345547)(962.30526367,114.97345998)
\curveto(962.29527427,114.99345536)(962.29027428,115.01845533)(962.29026367,115.04845998)
\curveto(962.29027428,115.07845527)(962.28527428,115.10345525)(962.27526367,115.12345998)
}
}
{
\newrgbcolor{curcolor}{0 0 0}
\pscustom[linestyle=none,fillstyle=solid,fillcolor=curcolor]
{
\newpath
\moveto(994.12156494,38.71181936)
\curveto(994.17156569,38.73180981)(994.23156563,38.75680979)(994.30156494,38.78681936)
\curveto(994.37156549,38.81680973)(994.44656541,38.83680971)(994.52656494,38.84681936)
\curveto(994.59656526,38.86680968)(994.66656519,38.86680968)(994.73656494,38.84681936)
\curveto(994.79656506,38.83680971)(994.84156502,38.79680975)(994.87156494,38.72681936)
\curveto(994.89156497,38.67680987)(994.90156496,38.61680993)(994.90156494,38.54681936)
\lineto(994.90156494,38.33681936)
\lineto(994.90156494,37.88681936)
\curveto(994.90156496,37.73681081)(994.87656498,37.61681093)(994.82656494,37.52681936)
\curveto(994.76656509,37.42681112)(994.6615652,37.35181119)(994.51156494,37.30181936)
\curveto(994.3615655,37.26181128)(994.22656563,37.21681133)(994.10656494,37.16681936)
\curveto(993.84656601,37.05681149)(993.57656628,36.95681159)(993.29656494,36.86681936)
\curveto(993.01656684,36.77681177)(992.74156712,36.67681187)(992.47156494,36.56681936)
\curveto(992.38156748,36.53681201)(992.29656756,36.50681204)(992.21656494,36.47681936)
\curveto(992.13656772,36.45681209)(992.0615678,36.42681212)(991.99156494,36.38681936)
\curveto(991.92156794,36.35681219)(991.861568,36.31181223)(991.81156494,36.25181936)
\curveto(991.7615681,36.19181235)(991.72156814,36.11181243)(991.69156494,36.01181936)
\curveto(991.67156819,35.96181258)(991.66656819,35.90181264)(991.67656494,35.83181936)
\lineto(991.67656494,35.63681936)
\lineto(991.67656494,32.80181936)
\lineto(991.67656494,32.50181936)
\curveto(991.66656819,32.39181615)(991.66656819,32.28681626)(991.67656494,32.18681936)
\curveto(991.68656817,32.08681646)(991.70156816,31.99181655)(991.72156494,31.90181936)
\curveto(991.74156812,31.82181672)(991.78156808,31.76181678)(991.84156494,31.72181936)
\curveto(991.94156792,31.6418169)(992.0565678,31.58181696)(992.18656494,31.54181936)
\curveto(992.30656755,31.51181703)(992.43156743,31.47181707)(992.56156494,31.42181936)
\curveto(992.79156707,31.32181722)(993.03156683,31.22681732)(993.28156494,31.13681936)
\curveto(993.53156633,31.05681749)(993.77156609,30.96681758)(994.00156494,30.86681936)
\curveto(994.0615658,30.8468177)(994.13156573,30.82181772)(994.21156494,30.79181936)
\curveto(994.28156558,30.77181777)(994.3565655,30.7468178)(994.43656494,30.71681936)
\curveto(994.51656534,30.68681786)(994.59156527,30.65181789)(994.66156494,30.61181936)
\curveto(994.72156514,30.58181796)(994.76656509,30.546818)(994.79656494,30.50681936)
\curveto(994.856565,30.42681812)(994.89156497,30.31681823)(994.90156494,30.17681936)
\lineto(994.90156494,29.75681936)
\lineto(994.90156494,29.51681936)
\curveto(994.89156497,29.4468191)(994.86656499,29.38681916)(994.82656494,29.33681936)
\curveto(994.79656506,29.28681926)(994.75156511,29.25681929)(994.69156494,29.24681936)
\curveto(994.63156523,29.2468193)(994.57156529,29.25181929)(994.51156494,29.26181936)
\curveto(994.44156542,29.28181926)(994.37656548,29.30181924)(994.31656494,29.32181936)
\curveto(994.24656561,29.35181919)(994.19656566,29.37681917)(994.16656494,29.39681936)
\curveto(993.84656601,29.53681901)(993.53156633,29.66181888)(993.22156494,29.77181936)
\curveto(992.90156696,29.88181866)(992.58156728,30.00181854)(992.26156494,30.13181936)
\curveto(992.04156782,30.22181832)(991.82656803,30.30681824)(991.61656494,30.38681936)
\curveto(991.39656846,30.46681808)(991.17656868,30.55181799)(990.95656494,30.64181936)
\curveto(990.23656962,30.9418176)(989.51157035,31.22681732)(988.78156494,31.49681936)
\curveto(988.04157182,31.76681678)(987.30657255,32.05181649)(986.57656494,32.35181936)
\curveto(986.31657354,32.46181608)(986.05157381,32.56181598)(985.78156494,32.65181936)
\curveto(985.51157435,32.75181579)(985.24657461,32.85681569)(984.98656494,32.96681936)
\curveto(984.87657498,33.01681553)(984.7565751,33.06181548)(984.62656494,33.10181936)
\curveto(984.48657537,33.15181539)(984.38657547,33.22181532)(984.32656494,33.31181936)
\curveto(984.28657557,33.35181519)(984.2565756,33.41681513)(984.23656494,33.50681936)
\curveto(984.22657563,33.52681502)(984.22657563,33.546815)(984.23656494,33.56681936)
\curveto(984.23657562,33.59681495)(984.23157563,33.62181492)(984.22156494,33.64181936)
\curveto(984.22157564,33.82181472)(984.22157564,34.03181451)(984.22156494,34.27181936)
\curveto(984.21157565,34.51181403)(984.24657561,34.68681386)(984.32656494,34.79681936)
\curveto(984.38657547,34.87681367)(984.48657537,34.93681361)(984.62656494,34.97681936)
\curveto(984.7565751,35.02681352)(984.87657498,35.07681347)(984.98656494,35.12681936)
\curveto(985.21657464,35.22681332)(985.44657441,35.31681323)(985.67656494,35.39681936)
\curveto(985.90657395,35.47681307)(986.13657372,35.56681298)(986.36656494,35.66681936)
\curveto(986.56657329,35.7468128)(986.77157309,35.82181272)(986.98156494,35.89181936)
\curveto(987.19157267,35.97181257)(987.39657246,36.05681249)(987.59656494,36.14681936)
\curveto(988.32657153,36.4468121)(989.06657079,36.73181181)(989.81656494,37.00181936)
\curveto(990.5565693,37.28181126)(991.29156857,37.57681097)(992.02156494,37.88681936)
\curveto(992.11156775,37.92681062)(992.19656766,37.95681059)(992.27656494,37.97681936)
\curveto(992.3565675,38.00681054)(992.44156742,38.03681051)(992.53156494,38.06681936)
\curveto(992.79156707,38.17681037)(993.0565668,38.28181026)(993.32656494,38.38181936)
\curveto(993.59656626,38.49181005)(993.861566,38.60180994)(994.12156494,38.71181936)
\moveto(990.47656494,35.50181936)
\curveto(990.44656941,35.59181295)(990.39656946,35.6468129)(990.32656494,35.66681936)
\curveto(990.2565696,35.69681285)(990.18156968,35.70181284)(990.10156494,35.68181936)
\curveto(990.01156985,35.67181287)(989.92656993,35.6468129)(989.84656494,35.60681936)
\curveto(989.7565701,35.57681297)(989.68157018,35.546813)(989.62156494,35.51681936)
\curveto(989.58157028,35.49681305)(989.54657031,35.48681306)(989.51656494,35.48681936)
\curveto(989.48657037,35.48681306)(989.45157041,35.47681307)(989.41156494,35.45681936)
\lineto(989.17156494,35.36681936)
\curveto(989.08157078,35.3468132)(988.99157087,35.31681323)(988.90156494,35.27681936)
\curveto(988.54157132,35.12681342)(988.17657168,34.99181355)(987.80656494,34.87181936)
\curveto(987.42657243,34.76181378)(987.0565728,34.63181391)(986.69656494,34.48181936)
\curveto(986.58657327,34.43181411)(986.47657338,34.38681416)(986.36656494,34.34681936)
\curveto(986.2565736,34.31681423)(986.15157371,34.27681427)(986.05156494,34.22681936)
\curveto(986.00157386,34.20681434)(985.9565739,34.18181436)(985.91656494,34.15181936)
\curveto(985.86657399,34.13181441)(985.84157402,34.08181446)(985.84156494,34.00181936)
\curveto(985.861574,33.98181456)(985.87657398,33.96181458)(985.88656494,33.94181936)
\curveto(985.89657396,33.92181462)(985.91157395,33.90181464)(985.93156494,33.88181936)
\curveto(985.98157388,33.8418147)(986.03657382,33.81181473)(986.09656494,33.79181936)
\curveto(986.14657371,33.77181477)(986.20157366,33.75181479)(986.26156494,33.73181936)
\curveto(986.37157349,33.68181486)(986.48157338,33.6418149)(986.59156494,33.61181936)
\curveto(986.70157316,33.58181496)(986.81157305,33.541815)(986.92156494,33.49181936)
\curveto(987.31157255,33.32181522)(987.70657215,33.17181537)(988.10656494,33.04181936)
\curveto(988.50657135,32.92181562)(988.89657096,32.78181576)(989.27656494,32.62181936)
\lineto(989.42656494,32.56181936)
\curveto(989.47657038,32.55181599)(989.52657033,32.53681601)(989.57656494,32.51681936)
\lineto(989.81656494,32.42681936)
\curveto(989.89656996,32.39681615)(989.97656988,32.37181617)(990.05656494,32.35181936)
\curveto(990.10656975,32.33181621)(990.1615697,32.32181622)(990.22156494,32.32181936)
\curveto(990.28156958,32.33181621)(990.33156953,32.3468162)(990.37156494,32.36681936)
\curveto(990.45156941,32.41681613)(990.49656936,32.52181602)(990.50656494,32.68181936)
\lineto(990.50656494,33.13181936)
\lineto(990.50656494,34.73681936)
\curveto(990.50656935,34.8468137)(990.51156935,34.98181356)(990.52156494,35.14181936)
\curveto(990.52156934,35.30181324)(990.50656935,35.42181312)(990.47656494,35.50181936)
}
}
{
\newrgbcolor{curcolor}{0 0 0}
\pscustom[linestyle=none,fillstyle=solid,fillcolor=curcolor]
{
\newpath
\moveto(987.28156494,46.44338186)
\curveto(987.33157253,46.51337426)(987.40657245,46.54837422)(987.50656494,46.54838186)
\curveto(987.60657225,46.55837421)(987.71157215,46.56337421)(987.82156494,46.56338186)
\lineto(994.09156494,46.56338186)
\lineto(994.69156494,46.56338186)
\curveto(994.74156512,46.54337423)(994.79156507,46.53837423)(994.84156494,46.54838186)
\curveto(994.88156498,46.55837421)(994.92656493,46.55337422)(994.97656494,46.53338186)
\curveto(995.07656478,46.51337426)(995.17656468,46.49837427)(995.27656494,46.48838186)
\curveto(995.38656447,46.48837428)(995.49156437,46.4733743)(995.59156494,46.44338186)
\curveto(995.70156416,46.41337436)(995.80656405,46.38337439)(995.90656494,46.35338186)
\curveto(996.00656385,46.33337444)(996.10656375,46.29837447)(996.20656494,46.24838186)
\curveto(996.46656339,46.14837462)(996.70156316,46.01837475)(996.91156494,45.85838186)
\curveto(997.12156274,45.70837506)(997.29656256,45.52837524)(997.43656494,45.31838186)
\curveto(997.5565623,45.14837562)(997.65156221,44.9683758)(997.72156494,44.77838186)
\curveto(997.80156206,44.58837618)(997.87656198,44.38337639)(997.94656494,44.16338186)
\curveto(997.96656189,44.0733767)(997.97656188,43.98337679)(997.97656494,43.89338186)
\curveto(997.98656187,43.80337697)(998.00156186,43.71337706)(998.02156494,43.62338186)
\lineto(998.02156494,43.53338186)
\curveto(998.03156183,43.51337726)(998.03656182,43.49337728)(998.03656494,43.47338186)
\curveto(998.04656181,43.42337735)(998.04656181,43.3733774)(998.03656494,43.32338186)
\curveto(998.02656183,43.28337749)(998.03156183,43.23837753)(998.05156494,43.18838186)
\curveto(998.07156179,43.11837765)(998.07656178,43.00837776)(998.06656494,42.85838186)
\curveto(998.06656179,42.71837805)(998.0565618,42.61837815)(998.03656494,42.55838186)
\curveto(998.03656182,42.52837824)(998.03156183,42.49837827)(998.02156494,42.46838186)
\lineto(998.02156494,42.40838186)
\curveto(998.00156186,42.31837845)(997.98656187,42.22837854)(997.97656494,42.13838186)
\curveto(997.97656188,42.04837872)(997.96656189,41.96337881)(997.94656494,41.88338186)
\curveto(997.92656193,41.80337897)(997.90156196,41.72337905)(997.87156494,41.64338186)
\curveto(997.85156201,41.56337921)(997.82656203,41.48337929)(997.79656494,41.40338186)
\curveto(997.66656219,41.08337969)(997.52156234,40.81337996)(997.36156494,40.59338186)
\curveto(997.20156266,40.38338039)(996.97656288,40.19338058)(996.68656494,40.02338186)
\curveto(996.66656319,40.00338077)(996.64156322,39.98838078)(996.61156494,39.97838186)
\curveto(996.59156327,39.97838079)(996.56656329,39.9683808)(996.53656494,39.94838186)
\curveto(996.4565634,39.91838085)(996.34156352,39.88338089)(996.19156494,39.84338186)
\curveto(996.05156381,39.81338096)(995.94656391,39.84338093)(995.87656494,39.93338186)
\curveto(995.82656403,39.99338078)(995.80156406,40.0733807)(995.80156494,40.17338186)
\lineto(995.80156494,40.50338186)
\lineto(995.80156494,40.66838186)
\curveto(995.80156406,40.72838004)(995.81156405,40.78337999)(995.83156494,40.83338186)
\curveto(995.861564,40.92337985)(995.91156395,40.98837978)(995.98156494,41.02838186)
\curveto(996.05156381,41.0683797)(996.12656373,41.11337966)(996.20656494,41.16338186)
\lineto(996.38656494,41.28338186)
\curveto(996.4565634,41.33337944)(996.51156335,41.38337939)(996.55156494,41.43338186)
\curveto(996.74156312,41.68337909)(996.88156298,41.98337879)(996.97156494,42.33338186)
\curveto(996.99156287,42.39337838)(997.00156286,42.45337832)(997.00156494,42.51338186)
\curveto(997.01156285,42.58337819)(997.02656283,42.64837812)(997.04656494,42.70838186)
\lineto(997.04656494,42.79838186)
\curveto(997.06656279,42.8683779)(997.07656278,42.95337782)(997.07656494,43.05338186)
\curveto(997.07656278,43.15337762)(997.06656279,43.24337753)(997.04656494,43.32338186)
\curveto(997.03656282,43.35337742)(997.03156283,43.39337738)(997.03156494,43.44338186)
\curveto(997.01156285,43.54337723)(996.99156287,43.63837713)(996.97156494,43.72838186)
\curveto(996.9615629,43.81837695)(996.93656292,43.90337687)(996.89656494,43.98338186)
\curveto(996.77656308,44.2733765)(996.61156325,44.50837626)(996.40156494,44.68838186)
\curveto(996.20156366,44.87837589)(995.9565639,45.03337574)(995.66656494,45.15338186)
\curveto(995.57656428,45.19337558)(995.48156438,45.21837555)(995.38156494,45.22838186)
\curveto(995.28156458,45.24837552)(995.17656468,45.2733755)(995.06656494,45.30338186)
\curveto(995.01656484,45.32337545)(994.96656489,45.33337544)(994.91656494,45.33338186)
\curveto(994.86656499,45.33337544)(994.81656504,45.33837543)(994.76656494,45.34838186)
\curveto(994.73656512,45.35837541)(994.68656517,45.36337541)(994.61656494,45.36338186)
\curveto(994.53656532,45.38337539)(994.45156541,45.38337539)(994.36156494,45.36338186)
\curveto(994.31156555,45.35337542)(994.26656559,45.34837542)(994.22656494,45.34838186)
\curveto(994.18656567,45.35837541)(994.15156571,45.35337542)(994.12156494,45.33338186)
\curveto(994.10156576,45.31337546)(994.09156577,45.29837547)(994.09156494,45.28838186)
\lineto(994.04656494,45.24338186)
\curveto(994.04656581,45.14337563)(994.07656578,45.0683757)(994.13656494,45.01838186)
\curveto(994.18656567,44.97837579)(994.23156563,44.92837584)(994.27156494,44.86838186)
\lineto(994.48156494,44.62838186)
\curveto(994.54156532,44.54837622)(994.59656526,44.45837631)(994.64656494,44.35838186)
\curveto(994.73656512,44.21837655)(994.81156505,44.04337673)(994.87156494,43.83338186)
\curveto(994.92156494,43.62337715)(994.9565649,43.40337737)(994.97656494,43.17338186)
\curveto(994.99656486,42.94337783)(994.99156487,42.71337806)(994.96156494,42.48338186)
\curveto(994.94156492,42.25337852)(994.90156496,42.04337873)(994.84156494,41.85338186)
\curveto(994.53156533,40.91337986)(993.93656592,40.25338052)(993.05656494,39.87338186)
\curveto(992.9565669,39.82338095)(992.861567,39.78338099)(992.77156494,39.75338186)
\curveto(992.67156719,39.72338105)(992.56656729,39.68838108)(992.45656494,39.64838186)
\curveto(992.40656745,39.62838114)(992.3615675,39.61838115)(992.32156494,39.61838186)
\curveto(992.28156758,39.61838115)(992.23656762,39.60838116)(992.18656494,39.58838186)
\curveto(992.11656774,39.5683812)(992.04656781,39.55338122)(991.97656494,39.54338186)
\curveto(991.89656796,39.54338123)(991.82156804,39.53338124)(991.75156494,39.51338186)
\curveto(991.71156815,39.50338127)(991.67656818,39.49838127)(991.64656494,39.49838186)
\curveto(991.60656825,39.50838126)(991.56656829,39.50838126)(991.52656494,39.49838186)
\curveto(991.48656837,39.49838127)(991.44656841,39.49338128)(991.40656494,39.48338186)
\lineto(991.28656494,39.48338186)
\curveto(991.16656869,39.46338131)(991.04156882,39.46338131)(990.91156494,39.48338186)
\curveto(990.85156901,39.49338128)(990.79156907,39.49838127)(990.73156494,39.49838186)
\lineto(990.56656494,39.49838186)
\curveto(990.51656934,39.50838126)(990.47656938,39.51338126)(990.44656494,39.51338186)
\curveto(990.40656945,39.51338126)(990.3615695,39.51838125)(990.31156494,39.52838186)
\curveto(990.20156966,39.55838121)(990.09656976,39.57838119)(989.99656494,39.58838186)
\curveto(989.88656997,39.59838117)(989.77657008,39.62338115)(989.66656494,39.66338186)
\curveto(989.54657031,39.70338107)(989.43157043,39.73838103)(989.32156494,39.76838186)
\curveto(989.20157066,39.80838096)(989.08657077,39.85338092)(988.97656494,39.90338186)
\curveto(988.81657104,39.9733808)(988.67157119,40.05338072)(988.54156494,40.14338186)
\curveto(988.40157146,40.23338054)(988.26657159,40.32838044)(988.13656494,40.42838186)
\curveto(988.02657183,40.49838027)(987.93657192,40.58838018)(987.86656494,40.69838186)
\lineto(987.80656494,40.75838186)
\lineto(987.74656494,40.81838186)
\lineto(987.62656494,40.96838186)
\lineto(987.50656494,41.14838186)
\curveto(987.42657243,41.27837949)(987.3565725,41.41337936)(987.29656494,41.55338186)
\curveto(987.23657262,41.70337907)(987.18157268,41.86337891)(987.13156494,42.03338186)
\curveto(987.10157276,42.13337864)(987.08157278,42.23337854)(987.07156494,42.33338186)
\curveto(987.0615728,42.44337833)(987.04657281,42.55337822)(987.02656494,42.66338186)
\curveto(987.01657284,42.70337807)(987.01657284,42.75337802)(987.02656494,42.81338186)
\curveto(987.03657282,42.88337789)(987.03157283,42.93337784)(987.01156494,42.96338186)
\curveto(987.00157286,43.28337749)(987.03157283,43.5683772)(987.10156494,43.81838186)
\curveto(987.17157269,44.07837669)(987.27157259,44.30837646)(987.40156494,44.50838186)
\curveto(987.44157242,44.57837619)(987.48657237,44.64337613)(987.53656494,44.70338186)
\lineto(987.68656494,44.88338186)
\curveto(987.72657213,44.93337584)(987.77157209,44.97837579)(987.82156494,45.01838186)
\curveto(987.861572,45.0683757)(987.88157198,45.14337563)(987.88156494,45.24338186)
\lineto(987.83656494,45.28838186)
\curveto(987.81657204,45.30837546)(987.79157207,45.32837544)(987.76156494,45.34838186)
\curveto(987.68157218,45.37837539)(987.60157226,45.39337538)(987.52156494,45.39338186)
\curveto(987.44157242,45.40337537)(987.37157249,45.43337534)(987.31156494,45.48338186)
\curveto(987.27157259,45.51337526)(987.24157262,45.5733752)(987.22156494,45.66338186)
\curveto(987.19157267,45.75337502)(987.17657268,45.84837492)(987.17656494,45.94838186)
\curveto(987.17657268,46.04837472)(987.18657267,46.14337463)(987.20656494,46.23338186)
\curveto(987.22657263,46.33337444)(987.25157261,46.40337437)(987.28156494,46.44338186)
\moveto(991.06156494,45.31838186)
\curveto(991.02156884,45.32837544)(990.97156889,45.33337544)(990.91156494,45.33338186)
\curveto(990.84156902,45.33337544)(990.78656907,45.32837544)(990.74656494,45.31838186)
\lineto(990.50656494,45.31838186)
\curveto(990.41656944,45.29837547)(990.33156953,45.28337549)(990.25156494,45.27338186)
\curveto(990.1615697,45.26337551)(990.07656978,45.24837552)(989.99656494,45.22838186)
\curveto(989.91656994,45.20837556)(989.84157002,45.18837558)(989.77156494,45.16838186)
\curveto(989.69157017,45.15837561)(989.61657024,45.13837563)(989.54656494,45.10838186)
\curveto(989.26657059,44.99837577)(989.01657084,44.85337592)(988.79656494,44.67338186)
\curveto(988.57657128,44.50337627)(988.41157145,44.28337649)(988.30156494,44.01338186)
\curveto(988.2615716,43.93337684)(988.23157163,43.84837692)(988.21156494,43.75838186)
\curveto(988.18157168,43.6683771)(988.1565717,43.5733772)(988.13656494,43.47338186)
\curveto(988.11657174,43.39337738)(988.11157175,43.30337747)(988.12156494,43.20338186)
\lineto(988.12156494,42.93338186)
\curveto(988.13157173,42.88337789)(988.13657172,42.83337794)(988.13656494,42.78338186)
\curveto(988.13657172,42.74337803)(988.14157172,42.69837807)(988.15156494,42.64838186)
\curveto(988.20157166,42.45837831)(988.25157161,42.29837847)(988.30156494,42.16838186)
\curveto(988.44157142,41.82837894)(988.65157121,41.56337921)(988.93156494,41.37338186)
\curveto(989.21157065,41.18337959)(989.53657032,41.03337974)(989.90656494,40.92338186)
\curveto(989.98656987,40.90337987)(990.06656979,40.88837988)(990.14656494,40.87838186)
\curveto(990.21656964,40.87837989)(990.29156957,40.8683799)(990.37156494,40.84838186)
\curveto(990.40156946,40.82837994)(990.43656942,40.81837995)(990.47656494,40.81838186)
\curveto(990.51656934,40.82837994)(990.55156931,40.82837994)(990.58156494,40.81838186)
\lineto(990.91156494,40.81838186)
\lineto(991.25656494,40.81838186)
\curveto(991.36656849,40.81837995)(991.47156839,40.82837994)(991.57156494,40.84838186)
\lineto(991.64656494,40.84838186)
\curveto(991.67656818,40.85837991)(991.70156816,40.86337991)(991.72156494,40.86338186)
\curveto(991.81156805,40.88337989)(991.90156796,40.89837987)(991.99156494,40.90838186)
\curveto(992.08156778,40.92837984)(992.16656769,40.95337982)(992.24656494,40.98338186)
\curveto(992.50656735,41.06337971)(992.74656711,41.16337961)(992.96656494,41.28338186)
\curveto(993.18656667,41.40337937)(993.36656649,41.56337921)(993.50656494,41.76338186)
\lineto(993.59656494,41.88338186)
\curveto(993.61656624,41.92337885)(993.63656622,41.9683788)(993.65656494,42.01838186)
\curveto(993.70656615,42.09837867)(993.74656611,42.18337859)(993.77656494,42.27338186)
\curveto(993.80656605,42.36337841)(993.83656602,42.46337831)(993.86656494,42.57338186)
\curveto(993.87656598,42.62337815)(993.88156598,42.6683781)(993.88156494,42.70838186)
\curveto(993.87156599,42.75837801)(993.87656598,42.80837796)(993.89656494,42.85838186)
\curveto(993.90656595,42.88837788)(993.91156595,42.93837783)(993.91156494,43.00838186)
\curveto(993.91156595,43.07837769)(993.90656595,43.12837764)(993.89656494,43.15838186)
\curveto(993.88656597,43.18837758)(993.88656597,43.21837755)(993.89656494,43.24838186)
\curveto(993.89656596,43.28837748)(993.89156597,43.32837744)(993.88156494,43.36838186)
\curveto(993.861566,43.45837731)(993.84156602,43.54337723)(993.82156494,43.62338186)
\curveto(993.80156606,43.70337707)(993.77656608,43.78337699)(993.74656494,43.86338186)
\curveto(993.59656626,44.20337657)(993.38656647,44.4733763)(993.11656494,44.67338186)
\curveto(992.84656701,44.8733759)(992.53156733,45.03337574)(992.17156494,45.15338186)
\curveto(992.08156778,45.18337559)(991.99156787,45.20337557)(991.90156494,45.21338186)
\curveto(991.80156806,45.23337554)(991.70656815,45.25337552)(991.61656494,45.27338186)
\curveto(991.57656828,45.28337549)(991.54156832,45.28837548)(991.51156494,45.28838186)
\curveto(991.47156839,45.28837548)(991.43156843,45.29337548)(991.39156494,45.30338186)
\curveto(991.34156852,45.32337545)(991.29156857,45.32337545)(991.24156494,45.30338186)
\curveto(991.18156868,45.29337548)(991.12156874,45.29837547)(991.06156494,45.31838186)
}
}
{
\newrgbcolor{curcolor}{0 0 0}
\pscustom[linestyle=none,fillstyle=solid,fillcolor=curcolor]
{
\newpath
\moveto(990.70156494,55.56666311)
\curveto(990.7615691,55.58665505)(990.856569,55.59665504)(990.98656494,55.59666311)
\curveto(991.10656875,55.59665504)(991.19156867,55.59165504)(991.24156494,55.58166311)
\lineto(991.39156494,55.58166311)
\curveto(991.47156839,55.57165506)(991.54656831,55.56165507)(991.61656494,55.55166311)
\curveto(991.67656818,55.55165508)(991.74656811,55.54665509)(991.82656494,55.53666311)
\curveto(991.88656797,55.51665512)(991.94656791,55.50165513)(992.00656494,55.49166311)
\curveto(992.06656779,55.49165514)(992.12656773,55.48165515)(992.18656494,55.46166311)
\curveto(992.31656754,55.42165521)(992.44656741,55.38665525)(992.57656494,55.35666311)
\curveto(992.70656715,55.32665531)(992.82656703,55.28665535)(992.93656494,55.23666311)
\curveto(993.41656644,55.02665561)(993.82156604,54.74665589)(994.15156494,54.39666311)
\curveto(994.47156539,54.04665659)(994.71656514,53.61665702)(994.88656494,53.10666311)
\curveto(994.92656493,52.99665764)(994.9565649,52.87665776)(994.97656494,52.74666311)
\curveto(994.99656486,52.62665801)(995.01656484,52.50165813)(995.03656494,52.37166311)
\curveto(995.04656481,52.31165832)(995.05156481,52.24665839)(995.05156494,52.17666311)
\curveto(995.0615648,52.11665852)(995.06656479,52.05665858)(995.06656494,51.99666311)
\curveto(995.07656478,51.95665868)(995.08156478,51.89665874)(995.08156494,51.81666311)
\curveto(995.08156478,51.74665889)(995.07656478,51.69665894)(995.06656494,51.66666311)
\curveto(995.0565648,51.62665901)(995.05156481,51.58665905)(995.05156494,51.54666311)
\curveto(995.0615648,51.50665913)(995.0615648,51.47165916)(995.05156494,51.44166311)
\lineto(995.05156494,51.35166311)
\lineto(995.00656494,50.99166311)
\curveto(994.96656489,50.85165978)(994.92656493,50.71665992)(994.88656494,50.58666311)
\curveto(994.84656501,50.45666018)(994.80156506,50.3316603)(994.75156494,50.21166311)
\curveto(994.55156531,49.76166087)(994.29156557,49.39166124)(993.97156494,49.10166311)
\curveto(993.65156621,48.81166182)(993.2615666,48.57166206)(992.80156494,48.38166311)
\curveto(992.70156716,48.3316623)(992.60156726,48.29166234)(992.50156494,48.26166311)
\curveto(992.40156746,48.24166239)(992.29656756,48.22166241)(992.18656494,48.20166311)
\curveto(992.14656771,48.18166245)(992.11656774,48.17166246)(992.09656494,48.17166311)
\curveto(992.06656779,48.18166245)(992.03156783,48.18166245)(991.99156494,48.17166311)
\curveto(991.91156795,48.15166248)(991.83156803,48.1366625)(991.75156494,48.12666311)
\curveto(991.6615682,48.12666251)(991.57656828,48.11666252)(991.49656494,48.09666311)
\lineto(991.37656494,48.09666311)
\curveto(991.33656852,48.09666254)(991.29156857,48.09166254)(991.24156494,48.08166311)
\curveto(991.19156867,48.07166256)(991.10656875,48.06666257)(990.98656494,48.06666311)
\curveto(990.856569,48.06666257)(990.7615691,48.07666256)(990.70156494,48.09666311)
\curveto(990.63156923,48.11666252)(990.5615693,48.12166251)(990.49156494,48.11166311)
\curveto(990.42156944,48.10166253)(990.35156951,48.10666253)(990.28156494,48.12666311)
\curveto(990.23156963,48.1366625)(990.19156967,48.14166249)(990.16156494,48.14166311)
\curveto(990.12156974,48.15166248)(990.07656978,48.16166247)(990.02656494,48.17166311)
\curveto(989.90656995,48.20166243)(989.78657007,48.22666241)(989.66656494,48.24666311)
\curveto(989.54657031,48.27666236)(989.43157043,48.31666232)(989.32156494,48.36666311)
\curveto(988.95157091,48.51666212)(988.62157124,48.69666194)(988.33156494,48.90666311)
\curveto(988.03157183,49.12666151)(987.78157208,49.39166124)(987.58156494,49.70166311)
\curveto(987.50157236,49.82166081)(987.43657242,49.94666069)(987.38656494,50.07666311)
\curveto(987.32657253,50.20666043)(987.26657259,50.34166029)(987.20656494,50.48166311)
\curveto(987.1565727,50.60166003)(987.12657273,50.7316599)(987.11656494,50.87166311)
\curveto(987.09657276,51.01165962)(987.06657279,51.15165948)(987.02656494,51.29166311)
\lineto(987.02656494,51.48666311)
\curveto(987.01657284,51.55665908)(987.00657285,51.62165901)(986.99656494,51.68166311)
\curveto(986.98657287,52.57165806)(987.17157269,53.31165732)(987.55156494,53.90166311)
\curveto(987.93157193,54.49165614)(988.42657143,54.91665572)(989.03656494,55.17666311)
\curveto(989.13657072,55.22665541)(989.23657062,55.26665537)(989.33656494,55.29666311)
\curveto(989.43657042,55.32665531)(989.54157032,55.36165527)(989.65156494,55.40166311)
\curveto(989.7615701,55.4316552)(989.88156998,55.45665518)(990.01156494,55.47666311)
\curveto(990.13156973,55.49665514)(990.2565696,55.52165511)(990.38656494,55.55166311)
\curveto(990.43656942,55.56165507)(990.49156937,55.56165507)(990.55156494,55.55166311)
\curveto(990.60156926,55.55165508)(990.65156921,55.55665508)(990.70156494,55.56666311)
\moveto(991.55656494,54.23166311)
\curveto(991.48656837,54.25165638)(991.40656845,54.25665638)(991.31656494,54.24666311)
\lineto(991.06156494,54.24666311)
\curveto(990.67156919,54.24665639)(990.34156952,54.21165642)(990.07156494,54.14166311)
\curveto(989.99156987,54.11165652)(989.91156995,54.08665655)(989.83156494,54.06666311)
\curveto(989.75157011,54.04665659)(989.67657018,54.02165661)(989.60656494,53.99166311)
\curveto(988.9565709,53.71165692)(988.50657135,53.26665737)(988.25656494,52.65666311)
\curveto(988.22657163,52.58665805)(988.20657165,52.51165812)(988.19656494,52.43166311)
\lineto(988.13656494,52.19166311)
\curveto(988.11657174,52.11165852)(988.10657175,52.02665861)(988.10656494,51.93666311)
\lineto(988.10656494,51.66666311)
\lineto(988.15156494,51.39666311)
\curveto(988.17157169,51.29665934)(988.19657166,51.20165943)(988.22656494,51.11166311)
\curveto(988.24657161,51.0316596)(988.27657158,50.95165968)(988.31656494,50.87166311)
\curveto(988.33657152,50.80165983)(988.36657149,50.7366599)(988.40656494,50.67666311)
\curveto(988.44657141,50.61666002)(988.48657137,50.56166007)(988.52656494,50.51166311)
\curveto(988.69657116,50.27166036)(988.90157096,50.07666056)(989.14156494,49.92666311)
\curveto(989.38157048,49.77666086)(989.6615702,49.64666099)(989.98156494,49.53666311)
\curveto(990.08156978,49.50666113)(990.18656967,49.48666115)(990.29656494,49.47666311)
\curveto(990.39656946,49.46666117)(990.50156936,49.45166118)(990.61156494,49.43166311)
\curveto(990.65156921,49.42166121)(990.71656914,49.41666122)(990.80656494,49.41666311)
\curveto(990.83656902,49.40666123)(990.87156899,49.40166123)(990.91156494,49.40166311)
\curveto(990.95156891,49.41166122)(990.99656886,49.41666122)(991.04656494,49.41666311)
\lineto(991.34656494,49.41666311)
\curveto(991.44656841,49.41666122)(991.53656832,49.42666121)(991.61656494,49.44666311)
\lineto(991.79656494,49.47666311)
\curveto(991.89656796,49.49666114)(991.99656786,49.51166112)(992.09656494,49.52166311)
\curveto(992.18656767,49.54166109)(992.27156759,49.57166106)(992.35156494,49.61166311)
\curveto(992.59156727,49.71166092)(992.81656704,49.82666081)(993.02656494,49.95666311)
\curveto(993.23656662,50.09666054)(993.41156645,50.26666037)(993.55156494,50.46666311)
\curveto(993.58156628,50.51666012)(993.60656625,50.56166007)(993.62656494,50.60166311)
\curveto(993.64656621,50.64165999)(993.67156619,50.68665995)(993.70156494,50.73666311)
\curveto(993.75156611,50.81665982)(993.79656606,50.90165973)(993.83656494,50.99166311)
\curveto(993.86656599,51.09165954)(993.89656596,51.19665944)(993.92656494,51.30666311)
\curveto(993.94656591,51.35665928)(993.9565659,51.40165923)(993.95656494,51.44166311)
\curveto(993.94656591,51.49165914)(993.94656591,51.54165909)(993.95656494,51.59166311)
\curveto(993.96656589,51.62165901)(993.97656588,51.68165895)(993.98656494,51.77166311)
\curveto(993.99656586,51.87165876)(993.99156587,51.94665869)(993.97156494,51.99666311)
\curveto(993.9615659,52.0366586)(993.9615659,52.07665856)(993.97156494,52.11666311)
\curveto(993.97156589,52.15665848)(993.9615659,52.19665844)(993.94156494,52.23666311)
\curveto(993.92156594,52.31665832)(993.90656595,52.39665824)(993.89656494,52.47666311)
\curveto(993.87656598,52.55665808)(993.85156601,52.631658)(993.82156494,52.70166311)
\curveto(993.68156618,53.04165759)(993.48656637,53.31665732)(993.23656494,53.52666311)
\curveto(992.98656687,53.7366569)(992.69156717,53.91165672)(992.35156494,54.05166311)
\curveto(992.23156763,54.10165653)(992.10656775,54.1316565)(991.97656494,54.14166311)
\curveto(991.83656802,54.16165647)(991.69656816,54.19165644)(991.55656494,54.23166311)
}
}
{
\newrgbcolor{curcolor}{0 0 0}
\pscustom[linestyle=none,fillstyle=solid,fillcolor=curcolor]
{
}
}
{
\newrgbcolor{curcolor}{0 0 0}
\pscustom[linestyle=none,fillstyle=solid,fillcolor=curcolor]
{
\newpath
\moveto(989.81656494,67.99510061)
\lineto(990.07156494,67.99510061)
\curveto(990.15156971,68.0050929)(990.22656963,68.00009291)(990.29656494,67.98010061)
\lineto(990.53656494,67.98010061)
\lineto(990.70156494,67.98010061)
\curveto(990.80156906,67.96009295)(990.90656895,67.95009296)(991.01656494,67.95010061)
\curveto(991.11656874,67.95009296)(991.21656864,67.94009297)(991.31656494,67.92010061)
\lineto(991.46656494,67.92010061)
\curveto(991.60656825,67.89009302)(991.74656811,67.87009304)(991.88656494,67.86010061)
\curveto(992.01656784,67.85009306)(992.14656771,67.82509308)(992.27656494,67.78510061)
\curveto(992.3565675,67.76509314)(992.44156742,67.74509316)(992.53156494,67.72510061)
\lineto(992.77156494,67.66510061)
\lineto(993.07156494,67.54510061)
\curveto(993.1615667,67.51509339)(993.25156661,67.48009343)(993.34156494,67.44010061)
\curveto(993.5615663,67.34009357)(993.77656608,67.2050937)(993.98656494,67.03510061)
\curveto(994.19656566,66.87509403)(994.36656549,66.70009421)(994.49656494,66.51010061)
\curveto(994.53656532,66.46009445)(994.57656528,66.40009451)(994.61656494,66.33010061)
\curveto(994.64656521,66.27009464)(994.68156518,66.2100947)(994.72156494,66.15010061)
\curveto(994.77156509,66.07009484)(994.81156505,65.97509493)(994.84156494,65.86510061)
\curveto(994.87156499,65.75509515)(994.90156496,65.65009526)(994.93156494,65.55010061)
\curveto(994.97156489,65.44009547)(994.99656486,65.33009558)(995.00656494,65.22010061)
\curveto(995.01656484,65.1100958)(995.03156483,64.99509591)(995.05156494,64.87510061)
\curveto(995.0615648,64.83509607)(995.0615648,64.79009612)(995.05156494,64.74010061)
\curveto(995.05156481,64.70009621)(995.0565648,64.66009625)(995.06656494,64.62010061)
\curveto(995.07656478,64.58009633)(995.08156478,64.52509638)(995.08156494,64.45510061)
\curveto(995.08156478,64.38509652)(995.07656478,64.33509657)(995.06656494,64.30510061)
\curveto(995.04656481,64.25509665)(995.04156482,64.2100967)(995.05156494,64.17010061)
\curveto(995.0615648,64.13009678)(995.0615648,64.09509681)(995.05156494,64.06510061)
\lineto(995.05156494,63.97510061)
\curveto(995.03156483,63.91509699)(995.01656484,63.85009706)(995.00656494,63.78010061)
\curveto(995.00656485,63.72009719)(995.00156486,63.65509725)(994.99156494,63.58510061)
\curveto(994.94156492,63.41509749)(994.89156497,63.25509765)(994.84156494,63.10510061)
\curveto(994.79156507,62.95509795)(994.72656513,62.8100981)(994.64656494,62.67010061)
\curveto(994.60656525,62.62009829)(994.57656528,62.56509834)(994.55656494,62.50510061)
\curveto(994.52656533,62.45509845)(994.49156537,62.4050985)(994.45156494,62.35510061)
\curveto(994.27156559,62.11509879)(994.05156581,61.91509899)(993.79156494,61.75510061)
\curveto(993.53156633,61.59509931)(993.24656661,61.45509945)(992.93656494,61.33510061)
\curveto(992.79656706,61.27509963)(992.6565672,61.23009968)(992.51656494,61.20010061)
\curveto(992.36656749,61.17009974)(992.21156765,61.13509977)(992.05156494,61.09510061)
\curveto(991.94156792,61.07509983)(991.83156803,61.06009985)(991.72156494,61.05010061)
\curveto(991.61156825,61.04009987)(991.50156836,61.02509988)(991.39156494,61.00510061)
\curveto(991.35156851,60.99509991)(991.31156855,60.99009992)(991.27156494,60.99010061)
\curveto(991.23156863,61.00009991)(991.19156867,61.00009991)(991.15156494,60.99010061)
\curveto(991.10156876,60.98009993)(991.05156881,60.97509993)(991.00156494,60.97510061)
\lineto(990.83656494,60.97510061)
\curveto(990.78656907,60.95509995)(990.73656912,60.95009996)(990.68656494,60.96010061)
\curveto(990.62656923,60.97009994)(990.57156929,60.97009994)(990.52156494,60.96010061)
\curveto(990.48156938,60.95009996)(990.43656942,60.95009996)(990.38656494,60.96010061)
\curveto(990.33656952,60.97009994)(990.28656957,60.96509994)(990.23656494,60.94510061)
\curveto(990.16656969,60.92509998)(990.09156977,60.92009999)(990.01156494,60.93010061)
\curveto(989.92156994,60.94009997)(989.83657002,60.94509996)(989.75656494,60.94510061)
\curveto(989.66657019,60.94509996)(989.56657029,60.94009997)(989.45656494,60.93010061)
\curveto(989.33657052,60.92009999)(989.23657062,60.92509998)(989.15656494,60.94510061)
\lineto(988.87156494,60.94510061)
\lineto(988.24156494,60.99010061)
\curveto(988.14157172,61.00009991)(988.04657181,61.0100999)(987.95656494,61.02010061)
\lineto(987.65656494,61.05010061)
\curveto(987.60657225,61.07009984)(987.5565723,61.07509983)(987.50656494,61.06510061)
\curveto(987.44657241,61.06509984)(987.39157247,61.07509983)(987.34156494,61.09510061)
\curveto(987.17157269,61.14509976)(987.00657285,61.18509972)(986.84656494,61.21510061)
\curveto(986.67657318,61.24509966)(986.51657334,61.29509961)(986.36656494,61.36510061)
\curveto(985.90657395,61.55509935)(985.53157433,61.77509913)(985.24156494,62.02510061)
\curveto(984.95157491,62.28509862)(984.70657515,62.64509826)(984.50656494,63.10510061)
\curveto(984.4565754,63.23509767)(984.42157544,63.36509754)(984.40156494,63.49510061)
\curveto(984.38157548,63.63509727)(984.3565755,63.77509713)(984.32656494,63.91510061)
\curveto(984.31657554,63.98509692)(984.31157555,64.05009686)(984.31156494,64.11010061)
\curveto(984.31157555,64.17009674)(984.30657555,64.23509667)(984.29656494,64.30510061)
\curveto(984.27657558,65.13509577)(984.42657543,65.8050951)(984.74656494,66.31510061)
\curveto(985.0565748,66.82509408)(985.49657436,67.2050937)(986.06656494,67.45510061)
\curveto(986.18657367,67.5050934)(986.31157355,67.55009336)(986.44156494,67.59010061)
\curveto(986.57157329,67.63009328)(986.70657315,67.67509323)(986.84656494,67.72510061)
\curveto(986.92657293,67.74509316)(987.01157285,67.76009315)(987.10156494,67.77010061)
\lineto(987.34156494,67.83010061)
\curveto(987.45157241,67.86009305)(987.5615723,67.87509303)(987.67156494,67.87510061)
\curveto(987.78157208,67.88509302)(987.89157197,67.90009301)(988.00156494,67.92010061)
\curveto(988.05157181,67.94009297)(988.09657176,67.94509296)(988.13656494,67.93510061)
\curveto(988.17657168,67.93509297)(988.21657164,67.94009297)(988.25656494,67.95010061)
\curveto(988.30657155,67.96009295)(988.3615715,67.96009295)(988.42156494,67.95010061)
\curveto(988.47157139,67.95009296)(988.52157134,67.95509295)(988.57156494,67.96510061)
\lineto(988.70656494,67.96510061)
\curveto(988.76657109,67.98509292)(988.83657102,67.98509292)(988.91656494,67.96510061)
\curveto(988.98657087,67.95509295)(989.05157081,67.96009295)(989.11156494,67.98010061)
\curveto(989.14157072,67.99009292)(989.18157068,67.99509291)(989.23156494,67.99510061)
\lineto(989.35156494,67.99510061)
\lineto(989.81656494,67.99510061)
\moveto(992.14156494,66.45010061)
\curveto(991.82156804,66.55009436)(991.4565684,66.6100943)(991.04656494,66.63010061)
\curveto(990.63656922,66.65009426)(990.22656963,66.66009425)(989.81656494,66.66010061)
\curveto(989.38657047,66.66009425)(988.96657089,66.65009426)(988.55656494,66.63010061)
\curveto(988.14657171,66.6100943)(987.7615721,66.56509434)(987.40156494,66.49510061)
\curveto(987.04157282,66.42509448)(986.72157314,66.31509459)(986.44156494,66.16510061)
\curveto(986.15157371,66.02509488)(985.91657394,65.83009508)(985.73656494,65.58010061)
\curveto(985.62657423,65.42009549)(985.54657431,65.24009567)(985.49656494,65.04010061)
\curveto(985.43657442,64.84009607)(985.40657445,64.59509631)(985.40656494,64.30510061)
\curveto(985.42657443,64.28509662)(985.43657442,64.25009666)(985.43656494,64.20010061)
\curveto(985.42657443,64.15009676)(985.42657443,64.1100968)(985.43656494,64.08010061)
\curveto(985.4565744,64.00009691)(985.47657438,63.92509698)(985.49656494,63.85510061)
\curveto(985.50657435,63.79509711)(985.52657433,63.73009718)(985.55656494,63.66010061)
\curveto(985.67657418,63.39009752)(985.84657401,63.17009774)(986.06656494,63.00010061)
\curveto(986.27657358,62.84009807)(986.52157334,62.7050982)(986.80156494,62.59510061)
\curveto(986.91157295,62.54509836)(987.03157283,62.5050984)(987.16156494,62.47510061)
\curveto(987.28157258,62.45509845)(987.40657245,62.43009848)(987.53656494,62.40010061)
\curveto(987.58657227,62.38009853)(987.64157222,62.37009854)(987.70156494,62.37010061)
\curveto(987.75157211,62.37009854)(987.80157206,62.36509854)(987.85156494,62.35510061)
\curveto(987.94157192,62.34509856)(988.03657182,62.33509857)(988.13656494,62.32510061)
\curveto(988.22657163,62.31509859)(988.32157154,62.3050986)(988.42156494,62.29510061)
\curveto(988.50157136,62.29509861)(988.58657127,62.29009862)(988.67656494,62.28010061)
\lineto(988.91656494,62.28010061)
\lineto(989.09656494,62.28010061)
\curveto(989.12657073,62.27009864)(989.1615707,62.26509864)(989.20156494,62.26510061)
\lineto(989.33656494,62.26510061)
\lineto(989.78656494,62.26510061)
\curveto(989.86656999,62.26509864)(989.95156991,62.26009865)(990.04156494,62.25010061)
\curveto(990.12156974,62.25009866)(990.19656966,62.26009865)(990.26656494,62.28010061)
\lineto(990.53656494,62.28010061)
\curveto(990.5565693,62.28009863)(990.58656927,62.27509863)(990.62656494,62.26510061)
\curveto(990.6565692,62.26509864)(990.68156918,62.27009864)(990.70156494,62.28010061)
\curveto(990.80156906,62.29009862)(990.90156896,62.29509861)(991.00156494,62.29510061)
\curveto(991.09156877,62.3050986)(991.19156867,62.31509859)(991.30156494,62.32510061)
\curveto(991.42156844,62.35509855)(991.54656831,62.37009854)(991.67656494,62.37010061)
\curveto(991.79656806,62.38009853)(991.91156795,62.4050985)(992.02156494,62.44510061)
\curveto(992.32156754,62.52509838)(992.58656727,62.6100983)(992.81656494,62.70010061)
\curveto(993.04656681,62.80009811)(993.2615666,62.94509796)(993.46156494,63.13510061)
\curveto(993.6615662,63.34509756)(993.81156605,63.6100973)(993.91156494,63.93010061)
\curveto(993.93156593,63.97009694)(993.94156592,64.0050969)(993.94156494,64.03510061)
\curveto(993.93156593,64.07509683)(993.93656592,64.12009679)(993.95656494,64.17010061)
\curveto(993.96656589,64.2100967)(993.97656588,64.28009663)(993.98656494,64.38010061)
\curveto(993.99656586,64.49009642)(993.99156587,64.57509633)(993.97156494,64.63510061)
\curveto(993.95156591,64.7050962)(993.94156592,64.77509613)(993.94156494,64.84510061)
\curveto(993.93156593,64.91509599)(993.91656594,64.98009593)(993.89656494,65.04010061)
\curveto(993.83656602,65.24009567)(993.75156611,65.42009549)(993.64156494,65.58010061)
\curveto(993.62156624,65.6100953)(993.60156626,65.63509527)(993.58156494,65.65510061)
\lineto(993.52156494,65.71510061)
\curveto(993.50156636,65.75509515)(993.4615664,65.8050951)(993.40156494,65.86510061)
\curveto(993.2615666,65.96509494)(993.13156673,66.05009486)(993.01156494,66.12010061)
\curveto(992.89156697,66.19009472)(992.74656711,66.26009465)(992.57656494,66.33010061)
\curveto(992.50656735,66.36009455)(992.43656742,66.38009453)(992.36656494,66.39010061)
\curveto(992.29656756,66.4100945)(992.22156764,66.43009448)(992.14156494,66.45010061)
}
}
{
\newrgbcolor{curcolor}{0 0 0}
\pscustom[linestyle=none,fillstyle=solid,fillcolor=curcolor]
{
\newpath
\moveto(984.49156494,70.85470998)
\lineto(984.49156494,74.45470998)
\lineto(984.49156494,75.09970998)
\curveto(984.49157537,75.17970345)(984.49657536,75.25470338)(984.50656494,75.32470998)
\curveto(984.50657535,75.39470324)(984.51657534,75.45470318)(984.53656494,75.50470998)
\curveto(984.56657529,75.57470306)(984.62657523,75.629703)(984.71656494,75.66970998)
\curveto(984.74657511,75.68970294)(984.78657507,75.69970293)(984.83656494,75.69970998)
\lineto(984.97156494,75.69970998)
\curveto(985.08157478,75.70970292)(985.18657467,75.70470293)(985.28656494,75.68470998)
\curveto(985.38657447,75.67470296)(985.4565744,75.63970299)(985.49656494,75.57970998)
\curveto(985.56657429,75.48970314)(985.60157426,75.35470328)(985.60156494,75.17470998)
\curveto(985.59157427,74.99470364)(985.58657427,74.8297038)(985.58656494,74.67970998)
\lineto(985.58656494,72.68470998)
\lineto(985.58656494,72.18970998)
\lineto(985.58656494,72.05470998)
\curveto(985.58657427,72.01470662)(985.59157427,71.97470666)(985.60156494,71.93470998)
\lineto(985.60156494,71.72470998)
\curveto(985.63157423,71.61470702)(985.67157419,71.5347071)(985.72156494,71.48470998)
\curveto(985.7615741,71.4347072)(985.81657404,71.39970723)(985.88656494,71.37970998)
\curveto(985.94657391,71.35970727)(986.01657384,71.34470729)(986.09656494,71.33470998)
\curveto(986.17657368,71.32470731)(986.26657359,71.30470733)(986.36656494,71.27470998)
\curveto(986.56657329,71.22470741)(986.77157309,71.18470745)(986.98156494,71.15470998)
\curveto(987.19157267,71.12470751)(987.39657246,71.08470755)(987.59656494,71.03470998)
\curveto(987.66657219,71.01470762)(987.73657212,71.00470763)(987.80656494,71.00470998)
\curveto(987.86657199,71.00470763)(987.93157193,70.99470764)(988.00156494,70.97470998)
\curveto(988.03157183,70.96470767)(988.07157179,70.95470768)(988.12156494,70.94470998)
\curveto(988.1615717,70.94470769)(988.20157166,70.94970768)(988.24156494,70.95970998)
\curveto(988.29157157,70.97970765)(988.33657152,71.00470763)(988.37656494,71.03470998)
\curveto(988.40657145,71.07470756)(988.41157145,71.1347075)(988.39156494,71.21470998)
\curveto(988.37157149,71.27470736)(988.34657151,71.3347073)(988.31656494,71.39470998)
\curveto(988.27657158,71.45470718)(988.24157162,71.51470712)(988.21156494,71.57470998)
\curveto(988.19157167,71.634707)(988.17657168,71.68470695)(988.16656494,71.72470998)
\curveto(988.08657177,71.91470672)(988.03157183,72.11970651)(988.00156494,72.33970998)
\curveto(987.97157189,72.56970606)(987.9615719,72.79970583)(987.97156494,73.02970998)
\curveto(987.97157189,73.26970536)(987.99657186,73.49970513)(988.04656494,73.71970998)
\curveto(988.08657177,73.93970469)(988.14657171,74.13970449)(988.22656494,74.31970998)
\curveto(988.24657161,74.36970426)(988.26657159,74.41470422)(988.28656494,74.45470998)
\curveto(988.30657155,74.50470413)(988.33157153,74.55470408)(988.36156494,74.60470998)
\curveto(988.57157129,74.95470368)(988.80157106,75.2347034)(989.05156494,75.44470998)
\curveto(989.30157056,75.66470297)(989.62657023,75.85970277)(990.02656494,76.02970998)
\curveto(990.13656972,76.07970255)(990.24656961,76.11470252)(990.35656494,76.13470998)
\curveto(990.46656939,76.15470248)(990.58156928,76.17970245)(990.70156494,76.20970998)
\curveto(990.73156913,76.21970241)(990.77656908,76.22470241)(990.83656494,76.22470998)
\curveto(990.89656896,76.24470239)(990.96656889,76.25470238)(991.04656494,76.25470998)
\curveto(991.11656874,76.25470238)(991.18156868,76.26470237)(991.24156494,76.28470998)
\lineto(991.40656494,76.28470998)
\curveto(991.4565684,76.29470234)(991.52656833,76.29970233)(991.61656494,76.29970998)
\curveto(991.70656815,76.29970233)(991.77656808,76.28970234)(991.82656494,76.26970998)
\curveto(991.88656797,76.24970238)(991.94656791,76.24470239)(992.00656494,76.25470998)
\curveto(992.0565678,76.26470237)(992.10656775,76.25970237)(992.15656494,76.23970998)
\curveto(992.31656754,76.19970243)(992.46656739,76.16470247)(992.60656494,76.13470998)
\curveto(992.74656711,76.10470253)(992.88156698,76.05970257)(993.01156494,75.99970998)
\curveto(993.38156648,75.83970279)(993.71656614,75.61970301)(994.01656494,75.33970998)
\curveto(994.31656554,75.05970357)(994.54656531,74.73970389)(994.70656494,74.37970998)
\curveto(994.78656507,74.20970442)(994.861565,74.00970462)(994.93156494,73.77970998)
\curveto(994.97156489,73.66970496)(994.99656486,73.55470508)(995.00656494,73.43470998)
\curveto(995.01656484,73.31470532)(995.03656482,73.19470544)(995.06656494,73.07470998)
\curveto(995.08656477,73.02470561)(995.08656477,72.96970566)(995.06656494,72.90970998)
\curveto(995.0565648,72.84970578)(995.0615648,72.78970584)(995.08156494,72.72970998)
\curveto(995.10156476,72.629706)(995.10156476,72.5297061)(995.08156494,72.42970998)
\lineto(995.08156494,72.29470998)
\curveto(995.0615648,72.24470639)(995.05156481,72.18470645)(995.05156494,72.11470998)
\curveto(995.0615648,72.05470658)(995.0565648,71.99970663)(995.03656494,71.94970998)
\curveto(995.02656483,71.90970672)(995.02156484,71.87470676)(995.02156494,71.84470998)
\curveto(995.02156484,71.81470682)(995.01656484,71.77970685)(995.00656494,71.73970998)
\lineto(994.94656494,71.46970998)
\curveto(994.92656493,71.37970725)(994.89656496,71.29470734)(994.85656494,71.21470998)
\curveto(994.71656514,70.87470776)(994.5615653,70.58470805)(994.39156494,70.34470998)
\curveto(994.21156565,70.10470853)(993.98156588,69.88470875)(993.70156494,69.68470998)
\curveto(993.47156639,69.5347091)(993.23156663,69.41970921)(992.98156494,69.33970998)
\curveto(992.93156693,69.31970931)(992.88656697,69.30970932)(992.84656494,69.30970998)
\curveto(992.79656706,69.30970932)(992.74656711,69.29970933)(992.69656494,69.27970998)
\curveto(992.63656722,69.25970937)(992.5565673,69.24470939)(992.45656494,69.23470998)
\curveto(992.3565675,69.2347094)(992.28156758,69.25470938)(992.23156494,69.29470998)
\curveto(992.15156771,69.34470929)(992.10656775,69.42470921)(992.09656494,69.53470998)
\curveto(992.08656777,69.64470899)(992.08156778,69.75970887)(992.08156494,69.87970998)
\lineto(992.08156494,70.04470998)
\curveto(992.08156778,70.10470853)(992.09156777,70.15970847)(992.11156494,70.20970998)
\curveto(992.13156773,70.29970833)(992.17156769,70.36970826)(992.23156494,70.41970998)
\curveto(992.32156754,70.48970814)(992.43156743,70.5347081)(992.56156494,70.55470998)
\curveto(992.68156718,70.58470805)(992.78656707,70.629708)(992.87656494,70.68970998)
\curveto(993.21656664,70.87970775)(993.48656637,71.13970749)(993.68656494,71.46970998)
\curveto(993.74656611,71.56970706)(993.79656606,71.67470696)(993.83656494,71.78470998)
\curveto(993.86656599,71.90470673)(993.90156596,72.02470661)(993.94156494,72.14470998)
\curveto(993.99156587,72.31470632)(994.01156585,72.51970611)(994.00156494,72.75970998)
\curveto(993.98156588,73.00970562)(993.94656591,73.20970542)(993.89656494,73.35970998)
\curveto(993.77656608,73.7297049)(993.61656624,74.01970461)(993.41656494,74.22970998)
\curveto(993.20656665,74.44970418)(992.92656693,74.629704)(992.57656494,74.76970998)
\curveto(992.47656738,74.81970381)(992.37156749,74.84970378)(992.26156494,74.85970998)
\curveto(992.15156771,74.87970375)(992.03656782,74.90470373)(991.91656494,74.93470998)
\lineto(991.81156494,74.93470998)
\curveto(991.77156809,74.94470369)(991.73156813,74.94970368)(991.69156494,74.94970998)
\curveto(991.6615682,74.95970367)(991.62656823,74.95970367)(991.58656494,74.94970998)
\lineto(991.46656494,74.94970998)
\curveto(991.20656865,74.94970368)(990.9615689,74.91970371)(990.73156494,74.85970998)
\curveto(990.38156948,74.74970388)(990.08656977,74.59470404)(989.84656494,74.39470998)
\curveto(989.59657026,74.19470444)(989.40157046,73.9347047)(989.26156494,73.61470998)
\lineto(989.20156494,73.43470998)
\curveto(989.18157068,73.38470525)(989.1615707,73.32470531)(989.14156494,73.25470998)
\curveto(989.12157074,73.20470543)(989.11157075,73.14470549)(989.11156494,73.07470998)
\curveto(989.10157076,73.01470562)(989.08657077,72.94970568)(989.06656494,72.87970998)
\lineto(989.06656494,72.72970998)
\curveto(989.04657081,72.68970594)(989.03657082,72.634706)(989.03656494,72.56470998)
\curveto(989.03657082,72.50470613)(989.04657081,72.44970618)(989.06656494,72.39970998)
\lineto(989.06656494,72.29470998)
\curveto(989.06657079,72.26470637)(989.07157079,72.2297064)(989.08156494,72.18970998)
\lineto(989.14156494,71.94970998)
\curveto(989.15157071,71.86970676)(989.17157069,71.78970684)(989.20156494,71.70970998)
\curveto(989.30157056,71.46970716)(989.43657042,71.23970739)(989.60656494,71.01970998)
\curveto(989.67657018,70.9297077)(989.75157011,70.84470779)(989.83156494,70.76470998)
\curveto(989.90156996,70.68470795)(989.9565699,70.58470805)(989.99656494,70.46470998)
\curveto(990.02656983,70.37470826)(990.03656982,70.2347084)(990.02656494,70.04470998)
\curveto(990.01656984,69.86470877)(989.99156987,69.74470889)(989.95156494,69.68470998)
\curveto(989.91156995,69.634709)(989.85157001,69.59470904)(989.77156494,69.56470998)
\curveto(989.69157017,69.54470909)(989.60657025,69.54470909)(989.51656494,69.56470998)
\curveto(989.39657046,69.59470904)(989.27657058,69.61470902)(989.15656494,69.62470998)
\curveto(989.02657083,69.64470899)(988.90157096,69.66970896)(988.78156494,69.69970998)
\curveto(988.74157112,69.71970891)(988.70657115,69.72470891)(988.67656494,69.71470998)
\curveto(988.63657122,69.71470892)(988.59157127,69.72470891)(988.54156494,69.74470998)
\curveto(988.45157141,69.76470887)(988.3615715,69.77970885)(988.27156494,69.78970998)
\curveto(988.17157169,69.79970883)(988.07657178,69.81970881)(987.98656494,69.84970998)
\curveto(987.92657193,69.85970877)(987.86657199,69.86470877)(987.80656494,69.86470998)
\curveto(987.74657211,69.87470876)(987.68657217,69.88970874)(987.62656494,69.90970998)
\curveto(987.42657243,69.95970867)(987.22157264,69.99470864)(987.01156494,70.01470998)
\curveto(986.79157307,70.04470859)(986.58157328,70.08470855)(986.38156494,70.13470998)
\curveto(986.28157358,70.16470847)(986.18157368,70.18470845)(986.08156494,70.19470998)
\curveto(985.98157388,70.20470843)(985.88157398,70.21970841)(985.78156494,70.23970998)
\curveto(985.75157411,70.24970838)(985.71157415,70.25470838)(985.66156494,70.25470998)
\curveto(985.55157431,70.28470835)(985.44657441,70.30470833)(985.34656494,70.31470998)
\curveto(985.23657462,70.3347083)(985.12657473,70.35970827)(985.01656494,70.38970998)
\curveto(984.93657492,70.40970822)(984.86657499,70.42470821)(984.80656494,70.43470998)
\curveto(984.73657512,70.44470819)(984.67657518,70.46970816)(984.62656494,70.50970998)
\curveto(984.59657526,70.5297081)(984.57657528,70.55970807)(984.56656494,70.59970998)
\curveto(984.54657531,70.63970799)(984.52657533,70.68470795)(984.50656494,70.73470998)
\curveto(984.50657535,70.79470784)(984.50157536,70.8347078)(984.49156494,70.85470998)
}
}
{
\newrgbcolor{curcolor}{0 0 0}
\pscustom[linestyle=none,fillstyle=solid,fillcolor=curcolor]
{
\newpath
\moveto(993.26656494,78.63431936)
\lineto(993.26656494,79.26431936)
\lineto(993.26656494,79.45931936)
\curveto(993.26656659,79.52931683)(993.27656658,79.58931677)(993.29656494,79.63931936)
\curveto(993.33656652,79.70931665)(993.37656648,79.7593166)(993.41656494,79.78931936)
\curveto(993.46656639,79.82931653)(993.53156633,79.84931651)(993.61156494,79.84931936)
\curveto(993.69156617,79.8593165)(993.77656608,79.86431649)(993.86656494,79.86431936)
\lineto(994.58656494,79.86431936)
\curveto(995.06656479,79.86431649)(995.47656438,79.80431655)(995.81656494,79.68431936)
\curveto(996.1565637,79.56431679)(996.43156343,79.36931699)(996.64156494,79.09931936)
\curveto(996.69156317,79.02931733)(996.73656312,78.9593174)(996.77656494,78.88931936)
\curveto(996.82656303,78.82931753)(996.87156299,78.7543176)(996.91156494,78.66431936)
\curveto(996.92156294,78.64431771)(996.93156293,78.61431774)(996.94156494,78.57431936)
\curveto(996.9615629,78.53431782)(996.96656289,78.48931787)(996.95656494,78.43931936)
\curveto(996.92656293,78.34931801)(996.85156301,78.29431806)(996.73156494,78.27431936)
\curveto(996.62156324,78.2543181)(996.52656333,78.26931809)(996.44656494,78.31931936)
\curveto(996.37656348,78.34931801)(996.31156355,78.39431796)(996.25156494,78.45431936)
\curveto(996.20156366,78.52431783)(996.15156371,78.58931777)(996.10156494,78.64931936)
\curveto(996.05156381,78.71931764)(995.97656388,78.77931758)(995.87656494,78.82931936)
\curveto(995.78656407,78.88931747)(995.69656416,78.93931742)(995.60656494,78.97931936)
\curveto(995.57656428,78.99931736)(995.51656434,79.02431733)(995.42656494,79.05431936)
\curveto(995.34656451,79.08431727)(995.27656458,79.08931727)(995.21656494,79.06931936)
\curveto(995.07656478,79.03931732)(994.98656487,78.97931738)(994.94656494,78.88931936)
\curveto(994.91656494,78.80931755)(994.90156496,78.71931764)(994.90156494,78.61931936)
\curveto(994.90156496,78.51931784)(994.87656498,78.43431792)(994.82656494,78.36431936)
\curveto(994.7565651,78.27431808)(994.61656524,78.22931813)(994.40656494,78.22931936)
\lineto(993.85156494,78.22931936)
\lineto(993.62656494,78.22931936)
\curveto(993.54656631,78.23931812)(993.48156638,78.2593181)(993.43156494,78.28931936)
\curveto(993.35156651,78.34931801)(993.30656655,78.41931794)(993.29656494,78.49931936)
\curveto(993.28656657,78.51931784)(993.28156658,78.53931782)(993.28156494,78.55931936)
\curveto(993.28156658,78.58931777)(993.27656658,78.61431774)(993.26656494,78.63431936)
}
}
{
\newrgbcolor{curcolor}{0 0 0}
\pscustom[linestyle=none,fillstyle=solid,fillcolor=curcolor]
{
}
}
{
\newrgbcolor{curcolor}{0 0 0}
\pscustom[linestyle=none,fillstyle=solid,fillcolor=curcolor]
{
\newpath
\moveto(984.29656494,89.26463186)
\curveto(984.28657557,89.95462722)(984.40657545,90.55462662)(984.65656494,91.06463186)
\curveto(984.90657495,91.58462559)(985.24157462,91.9796252)(985.66156494,92.24963186)
\curveto(985.74157412,92.29962488)(985.83157403,92.34462483)(985.93156494,92.38463186)
\curveto(986.02157384,92.42462475)(986.11657374,92.46962471)(986.21656494,92.51963186)
\curveto(986.31657354,92.55962462)(986.41657344,92.58962459)(986.51656494,92.60963186)
\curveto(986.61657324,92.62962455)(986.72157314,92.64962453)(986.83156494,92.66963186)
\curveto(986.88157298,92.68962449)(986.92657293,92.69462448)(986.96656494,92.68463186)
\curveto(987.00657285,92.6746245)(987.05157281,92.6796245)(987.10156494,92.69963186)
\curveto(987.15157271,92.70962447)(987.23657262,92.71462446)(987.35656494,92.71463186)
\curveto(987.46657239,92.71462446)(987.55157231,92.70962447)(987.61156494,92.69963186)
\curveto(987.67157219,92.6796245)(987.73157213,92.66962451)(987.79156494,92.66963186)
\curveto(987.85157201,92.6796245)(987.91157195,92.6746245)(987.97156494,92.65463186)
\curveto(988.11157175,92.61462456)(988.24657161,92.5796246)(988.37656494,92.54963186)
\curveto(988.50657135,92.51962466)(988.63157123,92.4796247)(988.75156494,92.42963186)
\curveto(988.89157097,92.36962481)(989.01657084,92.29962488)(989.12656494,92.21963186)
\curveto(989.23657062,92.14962503)(989.34657051,92.0746251)(989.45656494,91.99463186)
\lineto(989.51656494,91.93463186)
\curveto(989.53657032,91.92462525)(989.5565703,91.90962527)(989.57656494,91.88963186)
\curveto(989.73657012,91.76962541)(989.88156998,91.63462554)(990.01156494,91.48463186)
\curveto(990.14156972,91.33462584)(990.26656959,91.174626)(990.38656494,91.00463186)
\curveto(990.60656925,90.69462648)(990.81156905,90.39962678)(991.00156494,90.11963186)
\curveto(991.14156872,89.88962729)(991.27656858,89.65962752)(991.40656494,89.42963186)
\curveto(991.53656832,89.20962797)(991.67156819,88.98962819)(991.81156494,88.76963186)
\curveto(991.98156788,88.51962866)(992.1615677,88.2796289)(992.35156494,88.04963186)
\curveto(992.54156732,87.82962935)(992.76656709,87.63962954)(993.02656494,87.47963186)
\curveto(993.08656677,87.43962974)(993.14656671,87.40462977)(993.20656494,87.37463186)
\curveto(993.2565666,87.34462983)(993.32156654,87.31462986)(993.40156494,87.28463186)
\curveto(993.47156639,87.26462991)(993.53156633,87.25962992)(993.58156494,87.26963186)
\curveto(993.65156621,87.28962989)(993.70656615,87.32462985)(993.74656494,87.37463186)
\curveto(993.77656608,87.42462975)(993.79656606,87.48462969)(993.80656494,87.55463186)
\lineto(993.80656494,87.79463186)
\lineto(993.80656494,88.54463186)
\lineto(993.80656494,91.34963186)
\lineto(993.80656494,92.00963186)
\curveto(993.80656605,92.09962508)(993.81156605,92.18462499)(993.82156494,92.26463186)
\curveto(993.82156604,92.34462483)(993.84156602,92.40962477)(993.88156494,92.45963186)
\curveto(993.92156594,92.50962467)(993.99656586,92.54962463)(994.10656494,92.57963186)
\curveto(994.20656565,92.61962456)(994.30656555,92.62962455)(994.40656494,92.60963186)
\lineto(994.54156494,92.60963186)
\curveto(994.61156525,92.58962459)(994.67156519,92.56962461)(994.72156494,92.54963186)
\curveto(994.77156509,92.52962465)(994.81156505,92.49462468)(994.84156494,92.44463186)
\curveto(994.88156498,92.39462478)(994.90156496,92.32462485)(994.90156494,92.23463186)
\lineto(994.90156494,91.96463186)
\lineto(994.90156494,91.06463186)
\lineto(994.90156494,87.55463186)
\lineto(994.90156494,86.48963186)
\curveto(994.90156496,86.40963077)(994.90656495,86.31963086)(994.91656494,86.21963186)
\curveto(994.91656494,86.11963106)(994.90656495,86.03463114)(994.88656494,85.96463186)
\curveto(994.81656504,85.75463142)(994.63656522,85.68963149)(994.34656494,85.76963186)
\curveto(994.30656555,85.7796314)(994.27156559,85.7796314)(994.24156494,85.76963186)
\curveto(994.20156566,85.76963141)(994.1565657,85.7796314)(994.10656494,85.79963186)
\curveto(994.02656583,85.81963136)(993.94156592,85.83963134)(993.85156494,85.85963186)
\curveto(993.7615661,85.8796313)(993.67656618,85.90463127)(993.59656494,85.93463186)
\curveto(993.10656675,86.09463108)(992.69156717,86.29463088)(992.35156494,86.53463186)
\curveto(992.10156776,86.71463046)(991.87656798,86.91963026)(991.67656494,87.14963186)
\curveto(991.46656839,87.3796298)(991.27156859,87.61962956)(991.09156494,87.86963186)
\curveto(990.91156895,88.12962905)(990.74156912,88.39462878)(990.58156494,88.66463186)
\curveto(990.41156945,88.94462823)(990.23656962,89.21462796)(990.05656494,89.47463186)
\curveto(989.97656988,89.58462759)(989.90156996,89.68962749)(989.83156494,89.78963186)
\curveto(989.7615701,89.89962728)(989.68657017,90.00962717)(989.60656494,90.11963186)
\curveto(989.57657028,90.15962702)(989.54657031,90.19462698)(989.51656494,90.22463186)
\curveto(989.47657038,90.26462691)(989.44657041,90.30462687)(989.42656494,90.34463186)
\curveto(989.31657054,90.48462669)(989.19157067,90.60962657)(989.05156494,90.71963186)
\curveto(989.02157084,90.73962644)(988.99657086,90.76462641)(988.97656494,90.79463186)
\curveto(988.94657091,90.82462635)(988.91657094,90.84962633)(988.88656494,90.86963186)
\curveto(988.78657107,90.94962623)(988.68657117,91.01462616)(988.58656494,91.06463186)
\curveto(988.48657137,91.12462605)(988.37657148,91.179626)(988.25656494,91.22963186)
\curveto(988.18657167,91.25962592)(988.11157175,91.2796259)(988.03156494,91.28963186)
\lineto(987.79156494,91.34963186)
\lineto(987.70156494,91.34963186)
\curveto(987.67157219,91.35962582)(987.64157222,91.36462581)(987.61156494,91.36463186)
\curveto(987.54157232,91.38462579)(987.44657241,91.38962579)(987.32656494,91.37963186)
\curveto(987.19657266,91.3796258)(987.09657276,91.36962581)(987.02656494,91.34963186)
\curveto(986.94657291,91.32962585)(986.87157299,91.30962587)(986.80156494,91.28963186)
\curveto(986.72157314,91.2796259)(986.64157322,91.25962592)(986.56156494,91.22963186)
\curveto(986.32157354,91.11962606)(986.12157374,90.96962621)(985.96156494,90.77963186)
\curveto(985.79157407,90.59962658)(985.65157421,90.3796268)(985.54156494,90.11963186)
\curveto(985.52157434,90.04962713)(985.50657435,89.9796272)(985.49656494,89.90963186)
\curveto(985.47657438,89.83962734)(985.4565744,89.76462741)(985.43656494,89.68463186)
\curveto(985.41657444,89.60462757)(985.40657445,89.49462768)(985.40656494,89.35463186)
\curveto(985.40657445,89.22462795)(985.41657444,89.11962806)(985.43656494,89.03963186)
\curveto(985.44657441,88.9796282)(985.45157441,88.92462825)(985.45156494,88.87463186)
\curveto(985.45157441,88.82462835)(985.4615744,88.7746284)(985.48156494,88.72463186)
\curveto(985.52157434,88.62462855)(985.5615743,88.52962865)(985.60156494,88.43963186)
\curveto(985.64157422,88.35962882)(985.68657417,88.2796289)(985.73656494,88.19963186)
\curveto(985.7565741,88.16962901)(985.78157408,88.13962904)(985.81156494,88.10963186)
\curveto(985.84157402,88.08962909)(985.86657399,88.06462911)(985.88656494,88.03463186)
\lineto(985.96156494,87.95963186)
\curveto(985.98157388,87.92962925)(986.00157386,87.90462927)(986.02156494,87.88463186)
\lineto(986.23156494,87.73463186)
\curveto(986.29157357,87.69462948)(986.3565735,87.64962953)(986.42656494,87.59963186)
\curveto(986.51657334,87.53962964)(986.62157324,87.48962969)(986.74156494,87.44963186)
\curveto(986.85157301,87.41962976)(986.9615729,87.38462979)(987.07156494,87.34463186)
\curveto(987.18157268,87.30462987)(987.32657253,87.2796299)(987.50656494,87.26963186)
\curveto(987.67657218,87.25962992)(987.80157206,87.22962995)(987.88156494,87.17963186)
\curveto(987.9615719,87.12963005)(988.00657185,87.05463012)(988.01656494,86.95463186)
\curveto(988.02657183,86.85463032)(988.03157183,86.74463043)(988.03156494,86.62463186)
\curveto(988.03157183,86.58463059)(988.03657182,86.54463063)(988.04656494,86.50463186)
\curveto(988.04657181,86.46463071)(988.04157182,86.42963075)(988.03156494,86.39963186)
\curveto(988.01157185,86.34963083)(988.00157186,86.29963088)(988.00156494,86.24963186)
\curveto(988.00157186,86.20963097)(987.99157187,86.16963101)(987.97156494,86.12963186)
\curveto(987.91157195,86.03963114)(987.77657208,85.99463118)(987.56656494,85.99463186)
\lineto(987.44656494,85.99463186)
\curveto(987.38657247,86.00463117)(987.32657253,86.00963117)(987.26656494,86.00963186)
\curveto(987.19657266,86.01963116)(987.13157273,86.02963115)(987.07156494,86.03963186)
\curveto(986.9615729,86.05963112)(986.861573,86.0796311)(986.77156494,86.09963186)
\curveto(986.67157319,86.11963106)(986.57657328,86.14963103)(986.48656494,86.18963186)
\curveto(986.41657344,86.20963097)(986.3565735,86.22963095)(986.30656494,86.24963186)
\lineto(986.12656494,86.30963186)
\curveto(985.86657399,86.42963075)(985.62157424,86.58463059)(985.39156494,86.77463186)
\curveto(985.1615747,86.9746302)(984.97657488,87.18962999)(984.83656494,87.41963186)
\curveto(984.7565751,87.52962965)(984.69157517,87.64462953)(984.64156494,87.76463186)
\lineto(984.49156494,88.15463186)
\curveto(984.44157542,88.26462891)(984.41157545,88.3796288)(984.40156494,88.49963186)
\curveto(984.38157548,88.61962856)(984.3565755,88.74462843)(984.32656494,88.87463186)
\curveto(984.32657553,88.94462823)(984.32657553,89.00962817)(984.32656494,89.06963186)
\curveto(984.31657554,89.12962805)(984.30657555,89.19462798)(984.29656494,89.26463186)
}
}
{
\newrgbcolor{curcolor}{0 0 0}
\pscustom[linestyle=none,fillstyle=solid,fillcolor=curcolor]
{
\newpath
\moveto(989.81656494,101.36424123)
\lineto(990.07156494,101.36424123)
\curveto(990.15156971,101.37423353)(990.22656963,101.36923353)(990.29656494,101.34924123)
\lineto(990.53656494,101.34924123)
\lineto(990.70156494,101.34924123)
\curveto(990.80156906,101.32923357)(990.90656895,101.31923358)(991.01656494,101.31924123)
\curveto(991.11656874,101.31923358)(991.21656864,101.30923359)(991.31656494,101.28924123)
\lineto(991.46656494,101.28924123)
\curveto(991.60656825,101.25923364)(991.74656811,101.23923366)(991.88656494,101.22924123)
\curveto(992.01656784,101.21923368)(992.14656771,101.19423371)(992.27656494,101.15424123)
\curveto(992.3565675,101.13423377)(992.44156742,101.11423379)(992.53156494,101.09424123)
\lineto(992.77156494,101.03424123)
\lineto(993.07156494,100.91424123)
\curveto(993.1615667,100.88423402)(993.25156661,100.84923405)(993.34156494,100.80924123)
\curveto(993.5615663,100.70923419)(993.77656608,100.57423433)(993.98656494,100.40424123)
\curveto(994.19656566,100.24423466)(994.36656549,100.06923483)(994.49656494,99.87924123)
\curveto(994.53656532,99.82923507)(994.57656528,99.76923513)(994.61656494,99.69924123)
\curveto(994.64656521,99.63923526)(994.68156518,99.57923532)(994.72156494,99.51924123)
\curveto(994.77156509,99.43923546)(994.81156505,99.34423556)(994.84156494,99.23424123)
\curveto(994.87156499,99.12423578)(994.90156496,99.01923588)(994.93156494,98.91924123)
\curveto(994.97156489,98.80923609)(994.99656486,98.6992362)(995.00656494,98.58924123)
\curveto(995.01656484,98.47923642)(995.03156483,98.36423654)(995.05156494,98.24424123)
\curveto(995.0615648,98.2042367)(995.0615648,98.15923674)(995.05156494,98.10924123)
\curveto(995.05156481,98.06923683)(995.0565648,98.02923687)(995.06656494,97.98924123)
\curveto(995.07656478,97.94923695)(995.08156478,97.89423701)(995.08156494,97.82424123)
\curveto(995.08156478,97.75423715)(995.07656478,97.7042372)(995.06656494,97.67424123)
\curveto(995.04656481,97.62423728)(995.04156482,97.57923732)(995.05156494,97.53924123)
\curveto(995.0615648,97.4992374)(995.0615648,97.46423744)(995.05156494,97.43424123)
\lineto(995.05156494,97.34424123)
\curveto(995.03156483,97.28423762)(995.01656484,97.21923768)(995.00656494,97.14924123)
\curveto(995.00656485,97.08923781)(995.00156486,97.02423788)(994.99156494,96.95424123)
\curveto(994.94156492,96.78423812)(994.89156497,96.62423828)(994.84156494,96.47424123)
\curveto(994.79156507,96.32423858)(994.72656513,96.17923872)(994.64656494,96.03924123)
\curveto(994.60656525,95.98923891)(994.57656528,95.93423897)(994.55656494,95.87424123)
\curveto(994.52656533,95.82423908)(994.49156537,95.77423913)(994.45156494,95.72424123)
\curveto(994.27156559,95.48423942)(994.05156581,95.28423962)(993.79156494,95.12424123)
\curveto(993.53156633,94.96423994)(993.24656661,94.82424008)(992.93656494,94.70424123)
\curveto(992.79656706,94.64424026)(992.6565672,94.5992403)(992.51656494,94.56924123)
\curveto(992.36656749,94.53924036)(992.21156765,94.5042404)(992.05156494,94.46424123)
\curveto(991.94156792,94.44424046)(991.83156803,94.42924047)(991.72156494,94.41924123)
\curveto(991.61156825,94.40924049)(991.50156836,94.39424051)(991.39156494,94.37424123)
\curveto(991.35156851,94.36424054)(991.31156855,94.35924054)(991.27156494,94.35924123)
\curveto(991.23156863,94.36924053)(991.19156867,94.36924053)(991.15156494,94.35924123)
\curveto(991.10156876,94.34924055)(991.05156881,94.34424056)(991.00156494,94.34424123)
\lineto(990.83656494,94.34424123)
\curveto(990.78656907,94.32424058)(990.73656912,94.31924058)(990.68656494,94.32924123)
\curveto(990.62656923,94.33924056)(990.57156929,94.33924056)(990.52156494,94.32924123)
\curveto(990.48156938,94.31924058)(990.43656942,94.31924058)(990.38656494,94.32924123)
\curveto(990.33656952,94.33924056)(990.28656957,94.33424057)(990.23656494,94.31424123)
\curveto(990.16656969,94.29424061)(990.09156977,94.28924061)(990.01156494,94.29924123)
\curveto(989.92156994,94.30924059)(989.83657002,94.31424059)(989.75656494,94.31424123)
\curveto(989.66657019,94.31424059)(989.56657029,94.30924059)(989.45656494,94.29924123)
\curveto(989.33657052,94.28924061)(989.23657062,94.29424061)(989.15656494,94.31424123)
\lineto(988.87156494,94.31424123)
\lineto(988.24156494,94.35924123)
\curveto(988.14157172,94.36924053)(988.04657181,94.37924052)(987.95656494,94.38924123)
\lineto(987.65656494,94.41924123)
\curveto(987.60657225,94.43924046)(987.5565723,94.44424046)(987.50656494,94.43424123)
\curveto(987.44657241,94.43424047)(987.39157247,94.44424046)(987.34156494,94.46424123)
\curveto(987.17157269,94.51424039)(987.00657285,94.55424035)(986.84656494,94.58424123)
\curveto(986.67657318,94.61424029)(986.51657334,94.66424024)(986.36656494,94.73424123)
\curveto(985.90657395,94.92423998)(985.53157433,95.14423976)(985.24156494,95.39424123)
\curveto(984.95157491,95.65423925)(984.70657515,96.01423889)(984.50656494,96.47424123)
\curveto(984.4565754,96.6042383)(984.42157544,96.73423817)(984.40156494,96.86424123)
\curveto(984.38157548,97.0042379)(984.3565755,97.14423776)(984.32656494,97.28424123)
\curveto(984.31657554,97.35423755)(984.31157555,97.41923748)(984.31156494,97.47924123)
\curveto(984.31157555,97.53923736)(984.30657555,97.6042373)(984.29656494,97.67424123)
\curveto(984.27657558,98.5042364)(984.42657543,99.17423573)(984.74656494,99.68424123)
\curveto(985.0565748,100.19423471)(985.49657436,100.57423433)(986.06656494,100.82424123)
\curveto(986.18657367,100.87423403)(986.31157355,100.91923398)(986.44156494,100.95924123)
\curveto(986.57157329,100.9992339)(986.70657315,101.04423386)(986.84656494,101.09424123)
\curveto(986.92657293,101.11423379)(987.01157285,101.12923377)(987.10156494,101.13924123)
\lineto(987.34156494,101.19924123)
\curveto(987.45157241,101.22923367)(987.5615723,101.24423366)(987.67156494,101.24424123)
\curveto(987.78157208,101.25423365)(987.89157197,101.26923363)(988.00156494,101.28924123)
\curveto(988.05157181,101.30923359)(988.09657176,101.31423359)(988.13656494,101.30424123)
\curveto(988.17657168,101.3042336)(988.21657164,101.30923359)(988.25656494,101.31924123)
\curveto(988.30657155,101.32923357)(988.3615715,101.32923357)(988.42156494,101.31924123)
\curveto(988.47157139,101.31923358)(988.52157134,101.32423358)(988.57156494,101.33424123)
\lineto(988.70656494,101.33424123)
\curveto(988.76657109,101.35423355)(988.83657102,101.35423355)(988.91656494,101.33424123)
\curveto(988.98657087,101.32423358)(989.05157081,101.32923357)(989.11156494,101.34924123)
\curveto(989.14157072,101.35923354)(989.18157068,101.36423354)(989.23156494,101.36424123)
\lineto(989.35156494,101.36424123)
\lineto(989.81656494,101.36424123)
\moveto(992.14156494,99.81924123)
\curveto(991.82156804,99.91923498)(991.4565684,99.97923492)(991.04656494,99.99924123)
\curveto(990.63656922,100.01923488)(990.22656963,100.02923487)(989.81656494,100.02924123)
\curveto(989.38657047,100.02923487)(988.96657089,100.01923488)(988.55656494,99.99924123)
\curveto(988.14657171,99.97923492)(987.7615721,99.93423497)(987.40156494,99.86424123)
\curveto(987.04157282,99.79423511)(986.72157314,99.68423522)(986.44156494,99.53424123)
\curveto(986.15157371,99.39423551)(985.91657394,99.1992357)(985.73656494,98.94924123)
\curveto(985.62657423,98.78923611)(985.54657431,98.60923629)(985.49656494,98.40924123)
\curveto(985.43657442,98.20923669)(985.40657445,97.96423694)(985.40656494,97.67424123)
\curveto(985.42657443,97.65423725)(985.43657442,97.61923728)(985.43656494,97.56924123)
\curveto(985.42657443,97.51923738)(985.42657443,97.47923742)(985.43656494,97.44924123)
\curveto(985.4565744,97.36923753)(985.47657438,97.29423761)(985.49656494,97.22424123)
\curveto(985.50657435,97.16423774)(985.52657433,97.0992378)(985.55656494,97.02924123)
\curveto(985.67657418,96.75923814)(985.84657401,96.53923836)(986.06656494,96.36924123)
\curveto(986.27657358,96.20923869)(986.52157334,96.07423883)(986.80156494,95.96424123)
\curveto(986.91157295,95.91423899)(987.03157283,95.87423903)(987.16156494,95.84424123)
\curveto(987.28157258,95.82423908)(987.40657245,95.7992391)(987.53656494,95.76924123)
\curveto(987.58657227,95.74923915)(987.64157222,95.73923916)(987.70156494,95.73924123)
\curveto(987.75157211,95.73923916)(987.80157206,95.73423917)(987.85156494,95.72424123)
\curveto(987.94157192,95.71423919)(988.03657182,95.7042392)(988.13656494,95.69424123)
\curveto(988.22657163,95.68423922)(988.32157154,95.67423923)(988.42156494,95.66424123)
\curveto(988.50157136,95.66423924)(988.58657127,95.65923924)(988.67656494,95.64924123)
\lineto(988.91656494,95.64924123)
\lineto(989.09656494,95.64924123)
\curveto(989.12657073,95.63923926)(989.1615707,95.63423927)(989.20156494,95.63424123)
\lineto(989.33656494,95.63424123)
\lineto(989.78656494,95.63424123)
\curveto(989.86656999,95.63423927)(989.95156991,95.62923927)(990.04156494,95.61924123)
\curveto(990.12156974,95.61923928)(990.19656966,95.62923927)(990.26656494,95.64924123)
\lineto(990.53656494,95.64924123)
\curveto(990.5565693,95.64923925)(990.58656927,95.64423926)(990.62656494,95.63424123)
\curveto(990.6565692,95.63423927)(990.68156918,95.63923926)(990.70156494,95.64924123)
\curveto(990.80156906,95.65923924)(990.90156896,95.66423924)(991.00156494,95.66424123)
\curveto(991.09156877,95.67423923)(991.19156867,95.68423922)(991.30156494,95.69424123)
\curveto(991.42156844,95.72423918)(991.54656831,95.73923916)(991.67656494,95.73924123)
\curveto(991.79656806,95.74923915)(991.91156795,95.77423913)(992.02156494,95.81424123)
\curveto(992.32156754,95.89423901)(992.58656727,95.97923892)(992.81656494,96.06924123)
\curveto(993.04656681,96.16923873)(993.2615666,96.31423859)(993.46156494,96.50424123)
\curveto(993.6615662,96.71423819)(993.81156605,96.97923792)(993.91156494,97.29924123)
\curveto(993.93156593,97.33923756)(993.94156592,97.37423753)(993.94156494,97.40424123)
\curveto(993.93156593,97.44423746)(993.93656592,97.48923741)(993.95656494,97.53924123)
\curveto(993.96656589,97.57923732)(993.97656588,97.64923725)(993.98656494,97.74924123)
\curveto(993.99656586,97.85923704)(993.99156587,97.94423696)(993.97156494,98.00424123)
\curveto(993.95156591,98.07423683)(993.94156592,98.14423676)(993.94156494,98.21424123)
\curveto(993.93156593,98.28423662)(993.91656594,98.34923655)(993.89656494,98.40924123)
\curveto(993.83656602,98.60923629)(993.75156611,98.78923611)(993.64156494,98.94924123)
\curveto(993.62156624,98.97923592)(993.60156626,99.0042359)(993.58156494,99.02424123)
\lineto(993.52156494,99.08424123)
\curveto(993.50156636,99.12423578)(993.4615664,99.17423573)(993.40156494,99.23424123)
\curveto(993.2615666,99.33423557)(993.13156673,99.41923548)(993.01156494,99.48924123)
\curveto(992.89156697,99.55923534)(992.74656711,99.62923527)(992.57656494,99.69924123)
\curveto(992.50656735,99.72923517)(992.43656742,99.74923515)(992.36656494,99.75924123)
\curveto(992.29656756,99.77923512)(992.22156764,99.7992351)(992.14156494,99.81924123)
}
}
{
\newrgbcolor{curcolor}{0 0 0}
\pscustom[linestyle=none,fillstyle=solid,fillcolor=curcolor]
{
\newpath
\moveto(984.29656494,106.77385061)
\curveto(984.29657556,106.87384575)(984.30657555,106.96884566)(984.32656494,107.05885061)
\curveto(984.33657552,107.14884548)(984.36657549,107.21384541)(984.41656494,107.25385061)
\curveto(984.49657536,107.31384531)(984.60157526,107.34384528)(984.73156494,107.34385061)
\lineto(985.12156494,107.34385061)
\lineto(986.62156494,107.34385061)
\lineto(993.01156494,107.34385061)
\lineto(994.18156494,107.34385061)
\lineto(994.49656494,107.34385061)
\curveto(994.59656526,107.35384527)(994.67656518,107.33884529)(994.73656494,107.29885061)
\curveto(994.81656504,107.24884538)(994.86656499,107.17384545)(994.88656494,107.07385061)
\curveto(994.89656496,106.98384564)(994.90156496,106.87384575)(994.90156494,106.74385061)
\lineto(994.90156494,106.51885061)
\curveto(994.88156498,106.43884619)(994.86656499,106.36884626)(994.85656494,106.30885061)
\curveto(994.83656502,106.24884638)(994.79656506,106.19884643)(994.73656494,106.15885061)
\curveto(994.67656518,106.11884651)(994.60156526,106.09884653)(994.51156494,106.09885061)
\lineto(994.21156494,106.09885061)
\lineto(993.11656494,106.09885061)
\lineto(987.77656494,106.09885061)
\curveto(987.68657217,106.07884655)(987.61157225,106.06384656)(987.55156494,106.05385061)
\curveto(987.48157238,106.05384657)(987.42157244,106.0238466)(987.37156494,105.96385061)
\curveto(987.32157254,105.89384673)(987.29657256,105.80384682)(987.29656494,105.69385061)
\curveto(987.28657257,105.59384703)(987.28157258,105.48384714)(987.28156494,105.36385061)
\lineto(987.28156494,104.22385061)
\lineto(987.28156494,103.72885061)
\curveto(987.27157259,103.56884906)(987.21157265,103.45884917)(987.10156494,103.39885061)
\curveto(987.07157279,103.37884925)(987.04157282,103.36884926)(987.01156494,103.36885061)
\curveto(986.97157289,103.36884926)(986.92657293,103.36384926)(986.87656494,103.35385061)
\curveto(986.7565731,103.33384929)(986.64657321,103.33884929)(986.54656494,103.36885061)
\curveto(986.44657341,103.40884922)(986.37657348,103.46384916)(986.33656494,103.53385061)
\curveto(986.28657357,103.61384901)(986.2615736,103.73384889)(986.26156494,103.89385061)
\curveto(986.2615736,104.05384857)(986.24657361,104.18884844)(986.21656494,104.29885061)
\curveto(986.20657365,104.34884828)(986.20157366,104.40384822)(986.20156494,104.46385061)
\curveto(986.19157367,104.5238481)(986.17657368,104.58384804)(986.15656494,104.64385061)
\curveto(986.10657375,104.79384783)(986.0565738,104.93884769)(986.00656494,105.07885061)
\curveto(985.94657391,105.21884741)(985.87657398,105.35384727)(985.79656494,105.48385061)
\curveto(985.70657415,105.623847)(985.60157426,105.74384688)(985.48156494,105.84385061)
\curveto(985.3615745,105.94384668)(985.23157463,106.03884659)(985.09156494,106.12885061)
\curveto(984.99157487,106.18884644)(984.88157498,106.23384639)(984.76156494,106.26385061)
\curveto(984.64157522,106.30384632)(984.53657532,106.35384627)(984.44656494,106.41385061)
\curveto(984.38657547,106.46384616)(984.34657551,106.53384609)(984.32656494,106.62385061)
\curveto(984.31657554,106.64384598)(984.31157555,106.66884596)(984.31156494,106.69885061)
\curveto(984.31157555,106.7288459)(984.30657555,106.75384587)(984.29656494,106.77385061)
}
}
{
\newrgbcolor{curcolor}{0 0 0}
\pscustom[linestyle=none,fillstyle=solid,fillcolor=curcolor]
{
\newpath
\moveto(984.29656494,115.12345998)
\curveto(984.29657556,115.22345513)(984.30657555,115.31845503)(984.32656494,115.40845998)
\curveto(984.33657552,115.49845485)(984.36657549,115.56345479)(984.41656494,115.60345998)
\curveto(984.49657536,115.66345469)(984.60157526,115.69345466)(984.73156494,115.69345998)
\lineto(985.12156494,115.69345998)
\lineto(986.62156494,115.69345998)
\lineto(993.01156494,115.69345998)
\lineto(994.18156494,115.69345998)
\lineto(994.49656494,115.69345998)
\curveto(994.59656526,115.70345465)(994.67656518,115.68845466)(994.73656494,115.64845998)
\curveto(994.81656504,115.59845475)(994.86656499,115.52345483)(994.88656494,115.42345998)
\curveto(994.89656496,115.33345502)(994.90156496,115.22345513)(994.90156494,115.09345998)
\lineto(994.90156494,114.86845998)
\curveto(994.88156498,114.78845556)(994.86656499,114.71845563)(994.85656494,114.65845998)
\curveto(994.83656502,114.59845575)(994.79656506,114.5484558)(994.73656494,114.50845998)
\curveto(994.67656518,114.46845588)(994.60156526,114.4484559)(994.51156494,114.44845998)
\lineto(994.21156494,114.44845998)
\lineto(993.11656494,114.44845998)
\lineto(987.77656494,114.44845998)
\curveto(987.68657217,114.42845592)(987.61157225,114.41345594)(987.55156494,114.40345998)
\curveto(987.48157238,114.40345595)(987.42157244,114.37345598)(987.37156494,114.31345998)
\curveto(987.32157254,114.24345611)(987.29657256,114.1534562)(987.29656494,114.04345998)
\curveto(987.28657257,113.94345641)(987.28157258,113.83345652)(987.28156494,113.71345998)
\lineto(987.28156494,112.57345998)
\lineto(987.28156494,112.07845998)
\curveto(987.27157259,111.91845843)(987.21157265,111.80845854)(987.10156494,111.74845998)
\curveto(987.07157279,111.72845862)(987.04157282,111.71845863)(987.01156494,111.71845998)
\curveto(986.97157289,111.71845863)(986.92657293,111.71345864)(986.87656494,111.70345998)
\curveto(986.7565731,111.68345867)(986.64657321,111.68845866)(986.54656494,111.71845998)
\curveto(986.44657341,111.75845859)(986.37657348,111.81345854)(986.33656494,111.88345998)
\curveto(986.28657357,111.96345839)(986.2615736,112.08345827)(986.26156494,112.24345998)
\curveto(986.2615736,112.40345795)(986.24657361,112.53845781)(986.21656494,112.64845998)
\curveto(986.20657365,112.69845765)(986.20157366,112.7534576)(986.20156494,112.81345998)
\curveto(986.19157367,112.87345748)(986.17657368,112.93345742)(986.15656494,112.99345998)
\curveto(986.10657375,113.14345721)(986.0565738,113.28845706)(986.00656494,113.42845998)
\curveto(985.94657391,113.56845678)(985.87657398,113.70345665)(985.79656494,113.83345998)
\curveto(985.70657415,113.97345638)(985.60157426,114.09345626)(985.48156494,114.19345998)
\curveto(985.3615745,114.29345606)(985.23157463,114.38845596)(985.09156494,114.47845998)
\curveto(984.99157487,114.53845581)(984.88157498,114.58345577)(984.76156494,114.61345998)
\curveto(984.64157522,114.6534557)(984.53657532,114.70345565)(984.44656494,114.76345998)
\curveto(984.38657547,114.81345554)(984.34657551,114.88345547)(984.32656494,114.97345998)
\curveto(984.31657554,114.99345536)(984.31157555,115.01845533)(984.31156494,115.04845998)
\curveto(984.31157555,115.07845527)(984.30657555,115.10345525)(984.29656494,115.12345998)
}
}
{
\newrgbcolor{curcolor}{0 0 0}
\pscustom[linestyle=none,fillstyle=solid,fillcolor=curcolor]
{
\newpath
\moveto(1016.14292725,38.71181936)
\curveto(1016.19292799,38.73180981)(1016.25292793,38.75680979)(1016.32292725,38.78681936)
\curveto(1016.39292779,38.81680973)(1016.46792772,38.83680971)(1016.54792725,38.84681936)
\curveto(1016.61792757,38.86680968)(1016.6879275,38.86680968)(1016.75792725,38.84681936)
\curveto(1016.81792737,38.83680971)(1016.86292732,38.79680975)(1016.89292725,38.72681936)
\curveto(1016.91292727,38.67680987)(1016.92292726,38.61680993)(1016.92292725,38.54681936)
\lineto(1016.92292725,38.33681936)
\lineto(1016.92292725,37.88681936)
\curveto(1016.92292726,37.73681081)(1016.89792729,37.61681093)(1016.84792725,37.52681936)
\curveto(1016.7879274,37.42681112)(1016.6829275,37.35181119)(1016.53292725,37.30181936)
\curveto(1016.3829278,37.26181128)(1016.24792794,37.21681133)(1016.12792725,37.16681936)
\curveto(1015.86792832,37.05681149)(1015.59792859,36.95681159)(1015.31792725,36.86681936)
\curveto(1015.03792915,36.77681177)(1014.76292942,36.67681187)(1014.49292725,36.56681936)
\curveto(1014.40292978,36.53681201)(1014.31792987,36.50681204)(1014.23792725,36.47681936)
\curveto(1014.15793003,36.45681209)(1014.0829301,36.42681212)(1014.01292725,36.38681936)
\curveto(1013.94293024,36.35681219)(1013.8829303,36.31181223)(1013.83292725,36.25181936)
\curveto(1013.7829304,36.19181235)(1013.74293044,36.11181243)(1013.71292725,36.01181936)
\curveto(1013.69293049,35.96181258)(1013.6879305,35.90181264)(1013.69792725,35.83181936)
\lineto(1013.69792725,35.63681936)
\lineto(1013.69792725,32.80181936)
\lineto(1013.69792725,32.50181936)
\curveto(1013.6879305,32.39181615)(1013.6879305,32.28681626)(1013.69792725,32.18681936)
\curveto(1013.70793048,32.08681646)(1013.72293046,31.99181655)(1013.74292725,31.90181936)
\curveto(1013.76293042,31.82181672)(1013.80293038,31.76181678)(1013.86292725,31.72181936)
\curveto(1013.96293022,31.6418169)(1014.07793011,31.58181696)(1014.20792725,31.54181936)
\curveto(1014.32792986,31.51181703)(1014.45292973,31.47181707)(1014.58292725,31.42181936)
\curveto(1014.81292937,31.32181722)(1015.05292913,31.22681732)(1015.30292725,31.13681936)
\curveto(1015.55292863,31.05681749)(1015.79292839,30.96681758)(1016.02292725,30.86681936)
\curveto(1016.0829281,30.8468177)(1016.15292803,30.82181772)(1016.23292725,30.79181936)
\curveto(1016.30292788,30.77181777)(1016.37792781,30.7468178)(1016.45792725,30.71681936)
\curveto(1016.53792765,30.68681786)(1016.61292757,30.65181789)(1016.68292725,30.61181936)
\curveto(1016.74292744,30.58181796)(1016.7879274,30.546818)(1016.81792725,30.50681936)
\curveto(1016.87792731,30.42681812)(1016.91292727,30.31681823)(1016.92292725,30.17681936)
\lineto(1016.92292725,29.75681936)
\lineto(1016.92292725,29.51681936)
\curveto(1016.91292727,29.4468191)(1016.8879273,29.38681916)(1016.84792725,29.33681936)
\curveto(1016.81792737,29.28681926)(1016.77292741,29.25681929)(1016.71292725,29.24681936)
\curveto(1016.65292753,29.2468193)(1016.59292759,29.25181929)(1016.53292725,29.26181936)
\curveto(1016.46292772,29.28181926)(1016.39792779,29.30181924)(1016.33792725,29.32181936)
\curveto(1016.26792792,29.35181919)(1016.21792797,29.37681917)(1016.18792725,29.39681936)
\curveto(1015.86792832,29.53681901)(1015.55292863,29.66181888)(1015.24292725,29.77181936)
\curveto(1014.92292926,29.88181866)(1014.60292958,30.00181854)(1014.28292725,30.13181936)
\curveto(1014.06293012,30.22181832)(1013.84793034,30.30681824)(1013.63792725,30.38681936)
\curveto(1013.41793077,30.46681808)(1013.19793099,30.55181799)(1012.97792725,30.64181936)
\curveto(1012.25793193,30.9418176)(1011.53293265,31.22681732)(1010.80292725,31.49681936)
\curveto(1010.06293412,31.76681678)(1009.32793486,32.05181649)(1008.59792725,32.35181936)
\curveto(1008.33793585,32.46181608)(1008.07293611,32.56181598)(1007.80292725,32.65181936)
\curveto(1007.53293665,32.75181579)(1007.26793692,32.85681569)(1007.00792725,32.96681936)
\curveto(1006.89793729,33.01681553)(1006.77793741,33.06181548)(1006.64792725,33.10181936)
\curveto(1006.50793768,33.15181539)(1006.40793778,33.22181532)(1006.34792725,33.31181936)
\curveto(1006.30793788,33.35181519)(1006.27793791,33.41681513)(1006.25792725,33.50681936)
\curveto(1006.24793794,33.52681502)(1006.24793794,33.546815)(1006.25792725,33.56681936)
\curveto(1006.25793793,33.59681495)(1006.25293793,33.62181492)(1006.24292725,33.64181936)
\curveto(1006.24293794,33.82181472)(1006.24293794,34.03181451)(1006.24292725,34.27181936)
\curveto(1006.23293795,34.51181403)(1006.26793792,34.68681386)(1006.34792725,34.79681936)
\curveto(1006.40793778,34.87681367)(1006.50793768,34.93681361)(1006.64792725,34.97681936)
\curveto(1006.77793741,35.02681352)(1006.89793729,35.07681347)(1007.00792725,35.12681936)
\curveto(1007.23793695,35.22681332)(1007.46793672,35.31681323)(1007.69792725,35.39681936)
\curveto(1007.92793626,35.47681307)(1008.15793603,35.56681298)(1008.38792725,35.66681936)
\curveto(1008.5879356,35.7468128)(1008.79293539,35.82181272)(1009.00292725,35.89181936)
\curveto(1009.21293497,35.97181257)(1009.41793477,36.05681249)(1009.61792725,36.14681936)
\curveto(1010.34793384,36.4468121)(1011.0879331,36.73181181)(1011.83792725,37.00181936)
\curveto(1012.57793161,37.28181126)(1013.31293087,37.57681097)(1014.04292725,37.88681936)
\curveto(1014.13293005,37.92681062)(1014.21792997,37.95681059)(1014.29792725,37.97681936)
\curveto(1014.37792981,38.00681054)(1014.46292972,38.03681051)(1014.55292725,38.06681936)
\curveto(1014.81292937,38.17681037)(1015.07792911,38.28181026)(1015.34792725,38.38181936)
\curveto(1015.61792857,38.49181005)(1015.8829283,38.60180994)(1016.14292725,38.71181936)
\moveto(1012.49792725,35.50181936)
\curveto(1012.46793172,35.59181295)(1012.41793177,35.6468129)(1012.34792725,35.66681936)
\curveto(1012.27793191,35.69681285)(1012.20293198,35.70181284)(1012.12292725,35.68181936)
\curveto(1012.03293215,35.67181287)(1011.94793224,35.6468129)(1011.86792725,35.60681936)
\curveto(1011.77793241,35.57681297)(1011.70293248,35.546813)(1011.64292725,35.51681936)
\curveto(1011.60293258,35.49681305)(1011.56793262,35.48681306)(1011.53792725,35.48681936)
\curveto(1011.50793268,35.48681306)(1011.47293271,35.47681307)(1011.43292725,35.45681936)
\lineto(1011.19292725,35.36681936)
\curveto(1011.10293308,35.3468132)(1011.01293317,35.31681323)(1010.92292725,35.27681936)
\curveto(1010.56293362,35.12681342)(1010.19793399,34.99181355)(1009.82792725,34.87181936)
\curveto(1009.44793474,34.76181378)(1009.07793511,34.63181391)(1008.71792725,34.48181936)
\curveto(1008.60793558,34.43181411)(1008.49793569,34.38681416)(1008.38792725,34.34681936)
\curveto(1008.27793591,34.31681423)(1008.17293601,34.27681427)(1008.07292725,34.22681936)
\curveto(1008.02293616,34.20681434)(1007.97793621,34.18181436)(1007.93792725,34.15181936)
\curveto(1007.8879363,34.13181441)(1007.86293632,34.08181446)(1007.86292725,34.00181936)
\curveto(1007.8829363,33.98181456)(1007.89793629,33.96181458)(1007.90792725,33.94181936)
\curveto(1007.91793627,33.92181462)(1007.93293625,33.90181464)(1007.95292725,33.88181936)
\curveto(1008.00293618,33.8418147)(1008.05793613,33.81181473)(1008.11792725,33.79181936)
\curveto(1008.16793602,33.77181477)(1008.22293596,33.75181479)(1008.28292725,33.73181936)
\curveto(1008.39293579,33.68181486)(1008.50293568,33.6418149)(1008.61292725,33.61181936)
\curveto(1008.72293546,33.58181496)(1008.83293535,33.541815)(1008.94292725,33.49181936)
\curveto(1009.33293485,33.32181522)(1009.72793446,33.17181537)(1010.12792725,33.04181936)
\curveto(1010.52793366,32.92181562)(1010.91793327,32.78181576)(1011.29792725,32.62181936)
\lineto(1011.44792725,32.56181936)
\curveto(1011.49793269,32.55181599)(1011.54793264,32.53681601)(1011.59792725,32.51681936)
\lineto(1011.83792725,32.42681936)
\curveto(1011.91793227,32.39681615)(1011.99793219,32.37181617)(1012.07792725,32.35181936)
\curveto(1012.12793206,32.33181621)(1012.182932,32.32181622)(1012.24292725,32.32181936)
\curveto(1012.30293188,32.33181621)(1012.35293183,32.3468162)(1012.39292725,32.36681936)
\curveto(1012.47293171,32.41681613)(1012.51793167,32.52181602)(1012.52792725,32.68181936)
\lineto(1012.52792725,33.13181936)
\lineto(1012.52792725,34.73681936)
\curveto(1012.52793166,34.8468137)(1012.53293165,34.98181356)(1012.54292725,35.14181936)
\curveto(1012.54293164,35.30181324)(1012.52793166,35.42181312)(1012.49792725,35.50181936)
}
}
{
\newrgbcolor{curcolor}{0 0 0}
\pscustom[linestyle=none,fillstyle=solid,fillcolor=curcolor]
{
\newpath
\moveto(1009.30292725,46.44338186)
\curveto(1009.35293483,46.51337426)(1009.42793476,46.54837422)(1009.52792725,46.54838186)
\curveto(1009.62793456,46.55837421)(1009.73293445,46.56337421)(1009.84292725,46.56338186)
\lineto(1016.11292725,46.56338186)
\lineto(1016.71292725,46.56338186)
\curveto(1016.76292742,46.54337423)(1016.81292737,46.53837423)(1016.86292725,46.54838186)
\curveto(1016.90292728,46.55837421)(1016.94792724,46.55337422)(1016.99792725,46.53338186)
\curveto(1017.09792709,46.51337426)(1017.19792699,46.49837427)(1017.29792725,46.48838186)
\curveto(1017.40792678,46.48837428)(1017.51292667,46.4733743)(1017.61292725,46.44338186)
\curveto(1017.72292646,46.41337436)(1017.82792636,46.38337439)(1017.92792725,46.35338186)
\curveto(1018.02792616,46.33337444)(1018.12792606,46.29837447)(1018.22792725,46.24838186)
\curveto(1018.4879257,46.14837462)(1018.72292546,46.01837475)(1018.93292725,45.85838186)
\curveto(1019.14292504,45.70837506)(1019.31792487,45.52837524)(1019.45792725,45.31838186)
\curveto(1019.57792461,45.14837562)(1019.67292451,44.9683758)(1019.74292725,44.77838186)
\curveto(1019.82292436,44.58837618)(1019.89792429,44.38337639)(1019.96792725,44.16338186)
\curveto(1019.9879242,44.0733767)(1019.99792419,43.98337679)(1019.99792725,43.89338186)
\curveto(1020.00792418,43.80337697)(1020.02292416,43.71337706)(1020.04292725,43.62338186)
\lineto(1020.04292725,43.53338186)
\curveto(1020.05292413,43.51337726)(1020.05792413,43.49337728)(1020.05792725,43.47338186)
\curveto(1020.06792412,43.42337735)(1020.06792412,43.3733774)(1020.05792725,43.32338186)
\curveto(1020.04792414,43.28337749)(1020.05292413,43.23837753)(1020.07292725,43.18838186)
\curveto(1020.09292409,43.11837765)(1020.09792409,43.00837776)(1020.08792725,42.85838186)
\curveto(1020.0879241,42.71837805)(1020.07792411,42.61837815)(1020.05792725,42.55838186)
\curveto(1020.05792413,42.52837824)(1020.05292413,42.49837827)(1020.04292725,42.46838186)
\lineto(1020.04292725,42.40838186)
\curveto(1020.02292416,42.31837845)(1020.00792418,42.22837854)(1019.99792725,42.13838186)
\curveto(1019.99792419,42.04837872)(1019.9879242,41.96337881)(1019.96792725,41.88338186)
\curveto(1019.94792424,41.80337897)(1019.92292426,41.72337905)(1019.89292725,41.64338186)
\curveto(1019.87292431,41.56337921)(1019.84792434,41.48337929)(1019.81792725,41.40338186)
\curveto(1019.6879245,41.08337969)(1019.54292464,40.81337996)(1019.38292725,40.59338186)
\curveto(1019.22292496,40.38338039)(1018.99792519,40.19338058)(1018.70792725,40.02338186)
\curveto(1018.6879255,40.00338077)(1018.66292552,39.98838078)(1018.63292725,39.97838186)
\curveto(1018.61292557,39.97838079)(1018.5879256,39.9683808)(1018.55792725,39.94838186)
\curveto(1018.47792571,39.91838085)(1018.36292582,39.88338089)(1018.21292725,39.84338186)
\curveto(1018.07292611,39.81338096)(1017.96792622,39.84338093)(1017.89792725,39.93338186)
\curveto(1017.84792634,39.99338078)(1017.82292636,40.0733807)(1017.82292725,40.17338186)
\lineto(1017.82292725,40.50338186)
\lineto(1017.82292725,40.66838186)
\curveto(1017.82292636,40.72838004)(1017.83292635,40.78337999)(1017.85292725,40.83338186)
\curveto(1017.8829263,40.92337985)(1017.93292625,40.98837978)(1018.00292725,41.02838186)
\curveto(1018.07292611,41.0683797)(1018.14792604,41.11337966)(1018.22792725,41.16338186)
\lineto(1018.40792725,41.28338186)
\curveto(1018.47792571,41.33337944)(1018.53292565,41.38337939)(1018.57292725,41.43338186)
\curveto(1018.76292542,41.68337909)(1018.90292528,41.98337879)(1018.99292725,42.33338186)
\curveto(1019.01292517,42.39337838)(1019.02292516,42.45337832)(1019.02292725,42.51338186)
\curveto(1019.03292515,42.58337819)(1019.04792514,42.64837812)(1019.06792725,42.70838186)
\lineto(1019.06792725,42.79838186)
\curveto(1019.0879251,42.8683779)(1019.09792509,42.95337782)(1019.09792725,43.05338186)
\curveto(1019.09792509,43.15337762)(1019.0879251,43.24337753)(1019.06792725,43.32338186)
\curveto(1019.05792513,43.35337742)(1019.05292513,43.39337738)(1019.05292725,43.44338186)
\curveto(1019.03292515,43.54337723)(1019.01292517,43.63837713)(1018.99292725,43.72838186)
\curveto(1018.9829252,43.81837695)(1018.95792523,43.90337687)(1018.91792725,43.98338186)
\curveto(1018.79792539,44.2733765)(1018.63292555,44.50837626)(1018.42292725,44.68838186)
\curveto(1018.22292596,44.87837589)(1017.97792621,45.03337574)(1017.68792725,45.15338186)
\curveto(1017.59792659,45.19337558)(1017.50292668,45.21837555)(1017.40292725,45.22838186)
\curveto(1017.30292688,45.24837552)(1017.19792699,45.2733755)(1017.08792725,45.30338186)
\curveto(1017.03792715,45.32337545)(1016.9879272,45.33337544)(1016.93792725,45.33338186)
\curveto(1016.8879273,45.33337544)(1016.83792735,45.33837543)(1016.78792725,45.34838186)
\curveto(1016.75792743,45.35837541)(1016.70792748,45.36337541)(1016.63792725,45.36338186)
\curveto(1016.55792763,45.38337539)(1016.47292771,45.38337539)(1016.38292725,45.36338186)
\curveto(1016.33292785,45.35337542)(1016.2879279,45.34837542)(1016.24792725,45.34838186)
\curveto(1016.20792798,45.35837541)(1016.17292801,45.35337542)(1016.14292725,45.33338186)
\curveto(1016.12292806,45.31337546)(1016.11292807,45.29837547)(1016.11292725,45.28838186)
\lineto(1016.06792725,45.24338186)
\curveto(1016.06792812,45.14337563)(1016.09792809,45.0683757)(1016.15792725,45.01838186)
\curveto(1016.20792798,44.97837579)(1016.25292793,44.92837584)(1016.29292725,44.86838186)
\lineto(1016.50292725,44.62838186)
\curveto(1016.56292762,44.54837622)(1016.61792757,44.45837631)(1016.66792725,44.35838186)
\curveto(1016.75792743,44.21837655)(1016.83292735,44.04337673)(1016.89292725,43.83338186)
\curveto(1016.94292724,43.62337715)(1016.97792721,43.40337737)(1016.99792725,43.17338186)
\curveto(1017.01792717,42.94337783)(1017.01292717,42.71337806)(1016.98292725,42.48338186)
\curveto(1016.96292722,42.25337852)(1016.92292726,42.04337873)(1016.86292725,41.85338186)
\curveto(1016.55292763,40.91337986)(1015.95792823,40.25338052)(1015.07792725,39.87338186)
\curveto(1014.97792921,39.82338095)(1014.8829293,39.78338099)(1014.79292725,39.75338186)
\curveto(1014.69292949,39.72338105)(1014.5879296,39.68838108)(1014.47792725,39.64838186)
\curveto(1014.42792976,39.62838114)(1014.3829298,39.61838115)(1014.34292725,39.61838186)
\curveto(1014.30292988,39.61838115)(1014.25792993,39.60838116)(1014.20792725,39.58838186)
\curveto(1014.13793005,39.5683812)(1014.06793012,39.55338122)(1013.99792725,39.54338186)
\curveto(1013.91793027,39.54338123)(1013.84293034,39.53338124)(1013.77292725,39.51338186)
\curveto(1013.73293045,39.50338127)(1013.69793049,39.49838127)(1013.66792725,39.49838186)
\curveto(1013.62793056,39.50838126)(1013.5879306,39.50838126)(1013.54792725,39.49838186)
\curveto(1013.50793068,39.49838127)(1013.46793072,39.49338128)(1013.42792725,39.48338186)
\lineto(1013.30792725,39.48338186)
\curveto(1013.187931,39.46338131)(1013.06293112,39.46338131)(1012.93292725,39.48338186)
\curveto(1012.87293131,39.49338128)(1012.81293137,39.49838127)(1012.75292725,39.49838186)
\lineto(1012.58792725,39.49838186)
\curveto(1012.53793165,39.50838126)(1012.49793169,39.51338126)(1012.46792725,39.51338186)
\curveto(1012.42793176,39.51338126)(1012.3829318,39.51838125)(1012.33292725,39.52838186)
\curveto(1012.22293196,39.55838121)(1012.11793207,39.57838119)(1012.01792725,39.58838186)
\curveto(1011.90793228,39.59838117)(1011.79793239,39.62338115)(1011.68792725,39.66338186)
\curveto(1011.56793262,39.70338107)(1011.45293273,39.73838103)(1011.34292725,39.76838186)
\curveto(1011.22293296,39.80838096)(1011.10793308,39.85338092)(1010.99792725,39.90338186)
\curveto(1010.83793335,39.9733808)(1010.69293349,40.05338072)(1010.56292725,40.14338186)
\curveto(1010.42293376,40.23338054)(1010.2879339,40.32838044)(1010.15792725,40.42838186)
\curveto(1010.04793414,40.49838027)(1009.95793423,40.58838018)(1009.88792725,40.69838186)
\lineto(1009.82792725,40.75838186)
\lineto(1009.76792725,40.81838186)
\lineto(1009.64792725,40.96838186)
\lineto(1009.52792725,41.14838186)
\curveto(1009.44793474,41.27837949)(1009.37793481,41.41337936)(1009.31792725,41.55338186)
\curveto(1009.25793493,41.70337907)(1009.20293498,41.86337891)(1009.15292725,42.03338186)
\curveto(1009.12293506,42.13337864)(1009.10293508,42.23337854)(1009.09292725,42.33338186)
\curveto(1009.0829351,42.44337833)(1009.06793512,42.55337822)(1009.04792725,42.66338186)
\curveto(1009.03793515,42.70337807)(1009.03793515,42.75337802)(1009.04792725,42.81338186)
\curveto(1009.05793513,42.88337789)(1009.05293513,42.93337784)(1009.03292725,42.96338186)
\curveto(1009.02293516,43.28337749)(1009.05293513,43.5683772)(1009.12292725,43.81838186)
\curveto(1009.19293499,44.07837669)(1009.29293489,44.30837646)(1009.42292725,44.50838186)
\curveto(1009.46293472,44.57837619)(1009.50793468,44.64337613)(1009.55792725,44.70338186)
\lineto(1009.70792725,44.88338186)
\curveto(1009.74793444,44.93337584)(1009.79293439,44.97837579)(1009.84292725,45.01838186)
\curveto(1009.8829343,45.0683757)(1009.90293428,45.14337563)(1009.90292725,45.24338186)
\lineto(1009.85792725,45.28838186)
\curveto(1009.83793435,45.30837546)(1009.81293437,45.32837544)(1009.78292725,45.34838186)
\curveto(1009.70293448,45.37837539)(1009.62293456,45.39337538)(1009.54292725,45.39338186)
\curveto(1009.46293472,45.40337537)(1009.39293479,45.43337534)(1009.33292725,45.48338186)
\curveto(1009.29293489,45.51337526)(1009.26293492,45.5733752)(1009.24292725,45.66338186)
\curveto(1009.21293497,45.75337502)(1009.19793499,45.84837492)(1009.19792725,45.94838186)
\curveto(1009.19793499,46.04837472)(1009.20793498,46.14337463)(1009.22792725,46.23338186)
\curveto(1009.24793494,46.33337444)(1009.27293491,46.40337437)(1009.30292725,46.44338186)
\moveto(1013.08292725,45.31838186)
\curveto(1013.04293114,45.32837544)(1012.99293119,45.33337544)(1012.93292725,45.33338186)
\curveto(1012.86293132,45.33337544)(1012.80793138,45.32837544)(1012.76792725,45.31838186)
\lineto(1012.52792725,45.31838186)
\curveto(1012.43793175,45.29837547)(1012.35293183,45.28337549)(1012.27292725,45.27338186)
\curveto(1012.182932,45.26337551)(1012.09793209,45.24837552)(1012.01792725,45.22838186)
\curveto(1011.93793225,45.20837556)(1011.86293232,45.18837558)(1011.79292725,45.16838186)
\curveto(1011.71293247,45.15837561)(1011.63793255,45.13837563)(1011.56792725,45.10838186)
\curveto(1011.2879329,44.99837577)(1011.03793315,44.85337592)(1010.81792725,44.67338186)
\curveto(1010.59793359,44.50337627)(1010.43293375,44.28337649)(1010.32292725,44.01338186)
\curveto(1010.2829339,43.93337684)(1010.25293393,43.84837692)(1010.23292725,43.75838186)
\curveto(1010.20293398,43.6683771)(1010.17793401,43.5733772)(1010.15792725,43.47338186)
\curveto(1010.13793405,43.39337738)(1010.13293405,43.30337747)(1010.14292725,43.20338186)
\lineto(1010.14292725,42.93338186)
\curveto(1010.15293403,42.88337789)(1010.15793403,42.83337794)(1010.15792725,42.78338186)
\curveto(1010.15793403,42.74337803)(1010.16293402,42.69837807)(1010.17292725,42.64838186)
\curveto(1010.22293396,42.45837831)(1010.27293391,42.29837847)(1010.32292725,42.16838186)
\curveto(1010.46293372,41.82837894)(1010.67293351,41.56337921)(1010.95292725,41.37338186)
\curveto(1011.23293295,41.18337959)(1011.55793263,41.03337974)(1011.92792725,40.92338186)
\curveto(1012.00793218,40.90337987)(1012.0879321,40.88837988)(1012.16792725,40.87838186)
\curveto(1012.23793195,40.87837989)(1012.31293187,40.8683799)(1012.39292725,40.84838186)
\curveto(1012.42293176,40.82837994)(1012.45793173,40.81837995)(1012.49792725,40.81838186)
\curveto(1012.53793165,40.82837994)(1012.57293161,40.82837994)(1012.60292725,40.81838186)
\lineto(1012.93292725,40.81838186)
\lineto(1013.27792725,40.81838186)
\curveto(1013.3879308,40.81837995)(1013.49293069,40.82837994)(1013.59292725,40.84838186)
\lineto(1013.66792725,40.84838186)
\curveto(1013.69793049,40.85837991)(1013.72293046,40.86337991)(1013.74292725,40.86338186)
\curveto(1013.83293035,40.88337989)(1013.92293026,40.89837987)(1014.01292725,40.90838186)
\curveto(1014.10293008,40.92837984)(1014.18793,40.95337982)(1014.26792725,40.98338186)
\curveto(1014.52792966,41.06337971)(1014.76792942,41.16337961)(1014.98792725,41.28338186)
\curveto(1015.20792898,41.40337937)(1015.3879288,41.56337921)(1015.52792725,41.76338186)
\lineto(1015.61792725,41.88338186)
\curveto(1015.63792855,41.92337885)(1015.65792853,41.9683788)(1015.67792725,42.01838186)
\curveto(1015.72792846,42.09837867)(1015.76792842,42.18337859)(1015.79792725,42.27338186)
\curveto(1015.82792836,42.36337841)(1015.85792833,42.46337831)(1015.88792725,42.57338186)
\curveto(1015.89792829,42.62337815)(1015.90292828,42.6683781)(1015.90292725,42.70838186)
\curveto(1015.89292829,42.75837801)(1015.89792829,42.80837796)(1015.91792725,42.85838186)
\curveto(1015.92792826,42.88837788)(1015.93292825,42.93837783)(1015.93292725,43.00838186)
\curveto(1015.93292825,43.07837769)(1015.92792826,43.12837764)(1015.91792725,43.15838186)
\curveto(1015.90792828,43.18837758)(1015.90792828,43.21837755)(1015.91792725,43.24838186)
\curveto(1015.91792827,43.28837748)(1015.91292827,43.32837744)(1015.90292725,43.36838186)
\curveto(1015.8829283,43.45837731)(1015.86292832,43.54337723)(1015.84292725,43.62338186)
\curveto(1015.82292836,43.70337707)(1015.79792839,43.78337699)(1015.76792725,43.86338186)
\curveto(1015.61792857,44.20337657)(1015.40792878,44.4733763)(1015.13792725,44.67338186)
\curveto(1014.86792932,44.8733759)(1014.55292963,45.03337574)(1014.19292725,45.15338186)
\curveto(1014.10293008,45.18337559)(1014.01293017,45.20337557)(1013.92292725,45.21338186)
\curveto(1013.82293036,45.23337554)(1013.72793046,45.25337552)(1013.63792725,45.27338186)
\curveto(1013.59793059,45.28337549)(1013.56293062,45.28837548)(1013.53292725,45.28838186)
\curveto(1013.49293069,45.28837548)(1013.45293073,45.29337548)(1013.41292725,45.30338186)
\curveto(1013.36293082,45.32337545)(1013.31293087,45.32337545)(1013.26292725,45.30338186)
\curveto(1013.20293098,45.29337548)(1013.14293104,45.29837547)(1013.08292725,45.31838186)
}
}
{
\newrgbcolor{curcolor}{0 0 0}
\pscustom[linestyle=none,fillstyle=solid,fillcolor=curcolor]
{
\newpath
\moveto(1012.72292725,55.56666311)
\curveto(1012.7829314,55.58665505)(1012.87793131,55.59665504)(1013.00792725,55.59666311)
\curveto(1013.12793106,55.59665504)(1013.21293097,55.59165504)(1013.26292725,55.58166311)
\lineto(1013.41292725,55.58166311)
\curveto(1013.49293069,55.57165506)(1013.56793062,55.56165507)(1013.63792725,55.55166311)
\curveto(1013.69793049,55.55165508)(1013.76793042,55.54665509)(1013.84792725,55.53666311)
\curveto(1013.90793028,55.51665512)(1013.96793022,55.50165513)(1014.02792725,55.49166311)
\curveto(1014.0879301,55.49165514)(1014.14793004,55.48165515)(1014.20792725,55.46166311)
\curveto(1014.33792985,55.42165521)(1014.46792972,55.38665525)(1014.59792725,55.35666311)
\curveto(1014.72792946,55.32665531)(1014.84792934,55.28665535)(1014.95792725,55.23666311)
\curveto(1015.43792875,55.02665561)(1015.84292834,54.74665589)(1016.17292725,54.39666311)
\curveto(1016.49292769,54.04665659)(1016.73792745,53.61665702)(1016.90792725,53.10666311)
\curveto(1016.94792724,52.99665764)(1016.97792721,52.87665776)(1016.99792725,52.74666311)
\curveto(1017.01792717,52.62665801)(1017.03792715,52.50165813)(1017.05792725,52.37166311)
\curveto(1017.06792712,52.31165832)(1017.07292711,52.24665839)(1017.07292725,52.17666311)
\curveto(1017.0829271,52.11665852)(1017.0879271,52.05665858)(1017.08792725,51.99666311)
\curveto(1017.09792709,51.95665868)(1017.10292708,51.89665874)(1017.10292725,51.81666311)
\curveto(1017.10292708,51.74665889)(1017.09792709,51.69665894)(1017.08792725,51.66666311)
\curveto(1017.07792711,51.62665901)(1017.07292711,51.58665905)(1017.07292725,51.54666311)
\curveto(1017.0829271,51.50665913)(1017.0829271,51.47165916)(1017.07292725,51.44166311)
\lineto(1017.07292725,51.35166311)
\lineto(1017.02792725,50.99166311)
\curveto(1016.9879272,50.85165978)(1016.94792724,50.71665992)(1016.90792725,50.58666311)
\curveto(1016.86792732,50.45666018)(1016.82292736,50.3316603)(1016.77292725,50.21166311)
\curveto(1016.57292761,49.76166087)(1016.31292787,49.39166124)(1015.99292725,49.10166311)
\curveto(1015.67292851,48.81166182)(1015.2829289,48.57166206)(1014.82292725,48.38166311)
\curveto(1014.72292946,48.3316623)(1014.62292956,48.29166234)(1014.52292725,48.26166311)
\curveto(1014.42292976,48.24166239)(1014.31792987,48.22166241)(1014.20792725,48.20166311)
\curveto(1014.16793002,48.18166245)(1014.13793005,48.17166246)(1014.11792725,48.17166311)
\curveto(1014.0879301,48.18166245)(1014.05293013,48.18166245)(1014.01292725,48.17166311)
\curveto(1013.93293025,48.15166248)(1013.85293033,48.1366625)(1013.77292725,48.12666311)
\curveto(1013.6829305,48.12666251)(1013.59793059,48.11666252)(1013.51792725,48.09666311)
\lineto(1013.39792725,48.09666311)
\curveto(1013.35793083,48.09666254)(1013.31293087,48.09166254)(1013.26292725,48.08166311)
\curveto(1013.21293097,48.07166256)(1013.12793106,48.06666257)(1013.00792725,48.06666311)
\curveto(1012.87793131,48.06666257)(1012.7829314,48.07666256)(1012.72292725,48.09666311)
\curveto(1012.65293153,48.11666252)(1012.5829316,48.12166251)(1012.51292725,48.11166311)
\curveto(1012.44293174,48.10166253)(1012.37293181,48.10666253)(1012.30292725,48.12666311)
\curveto(1012.25293193,48.1366625)(1012.21293197,48.14166249)(1012.18292725,48.14166311)
\curveto(1012.14293204,48.15166248)(1012.09793209,48.16166247)(1012.04792725,48.17166311)
\curveto(1011.92793226,48.20166243)(1011.80793238,48.22666241)(1011.68792725,48.24666311)
\curveto(1011.56793262,48.27666236)(1011.45293273,48.31666232)(1011.34292725,48.36666311)
\curveto(1010.97293321,48.51666212)(1010.64293354,48.69666194)(1010.35292725,48.90666311)
\curveto(1010.05293413,49.12666151)(1009.80293438,49.39166124)(1009.60292725,49.70166311)
\curveto(1009.52293466,49.82166081)(1009.45793473,49.94666069)(1009.40792725,50.07666311)
\curveto(1009.34793484,50.20666043)(1009.2879349,50.34166029)(1009.22792725,50.48166311)
\curveto(1009.17793501,50.60166003)(1009.14793504,50.7316599)(1009.13792725,50.87166311)
\curveto(1009.11793507,51.01165962)(1009.0879351,51.15165948)(1009.04792725,51.29166311)
\lineto(1009.04792725,51.48666311)
\curveto(1009.03793515,51.55665908)(1009.02793516,51.62165901)(1009.01792725,51.68166311)
\curveto(1009.00793518,52.57165806)(1009.19293499,53.31165732)(1009.57292725,53.90166311)
\curveto(1009.95293423,54.49165614)(1010.44793374,54.91665572)(1011.05792725,55.17666311)
\curveto(1011.15793303,55.22665541)(1011.25793293,55.26665537)(1011.35792725,55.29666311)
\curveto(1011.45793273,55.32665531)(1011.56293262,55.36165527)(1011.67292725,55.40166311)
\curveto(1011.7829324,55.4316552)(1011.90293228,55.45665518)(1012.03292725,55.47666311)
\curveto(1012.15293203,55.49665514)(1012.27793191,55.52165511)(1012.40792725,55.55166311)
\curveto(1012.45793173,55.56165507)(1012.51293167,55.56165507)(1012.57292725,55.55166311)
\curveto(1012.62293156,55.55165508)(1012.67293151,55.55665508)(1012.72292725,55.56666311)
\moveto(1013.57792725,54.23166311)
\curveto(1013.50793068,54.25165638)(1013.42793076,54.25665638)(1013.33792725,54.24666311)
\lineto(1013.08292725,54.24666311)
\curveto(1012.69293149,54.24665639)(1012.36293182,54.21165642)(1012.09292725,54.14166311)
\curveto(1012.01293217,54.11165652)(1011.93293225,54.08665655)(1011.85292725,54.06666311)
\curveto(1011.77293241,54.04665659)(1011.69793249,54.02165661)(1011.62792725,53.99166311)
\curveto(1010.97793321,53.71165692)(1010.52793366,53.26665737)(1010.27792725,52.65666311)
\curveto(1010.24793394,52.58665805)(1010.22793396,52.51165812)(1010.21792725,52.43166311)
\lineto(1010.15792725,52.19166311)
\curveto(1010.13793405,52.11165852)(1010.12793406,52.02665861)(1010.12792725,51.93666311)
\lineto(1010.12792725,51.66666311)
\lineto(1010.17292725,51.39666311)
\curveto(1010.19293399,51.29665934)(1010.21793397,51.20165943)(1010.24792725,51.11166311)
\curveto(1010.26793392,51.0316596)(1010.29793389,50.95165968)(1010.33792725,50.87166311)
\curveto(1010.35793383,50.80165983)(1010.3879338,50.7366599)(1010.42792725,50.67666311)
\curveto(1010.46793372,50.61666002)(1010.50793368,50.56166007)(1010.54792725,50.51166311)
\curveto(1010.71793347,50.27166036)(1010.92293326,50.07666056)(1011.16292725,49.92666311)
\curveto(1011.40293278,49.77666086)(1011.6829325,49.64666099)(1012.00292725,49.53666311)
\curveto(1012.10293208,49.50666113)(1012.20793198,49.48666115)(1012.31792725,49.47666311)
\curveto(1012.41793177,49.46666117)(1012.52293166,49.45166118)(1012.63292725,49.43166311)
\curveto(1012.67293151,49.42166121)(1012.73793145,49.41666122)(1012.82792725,49.41666311)
\curveto(1012.85793133,49.40666123)(1012.89293129,49.40166123)(1012.93292725,49.40166311)
\curveto(1012.97293121,49.41166122)(1013.01793117,49.41666122)(1013.06792725,49.41666311)
\lineto(1013.36792725,49.41666311)
\curveto(1013.46793072,49.41666122)(1013.55793063,49.42666121)(1013.63792725,49.44666311)
\lineto(1013.81792725,49.47666311)
\curveto(1013.91793027,49.49666114)(1014.01793017,49.51166112)(1014.11792725,49.52166311)
\curveto(1014.20792998,49.54166109)(1014.29292989,49.57166106)(1014.37292725,49.61166311)
\curveto(1014.61292957,49.71166092)(1014.83792935,49.82666081)(1015.04792725,49.95666311)
\curveto(1015.25792893,50.09666054)(1015.43292875,50.26666037)(1015.57292725,50.46666311)
\curveto(1015.60292858,50.51666012)(1015.62792856,50.56166007)(1015.64792725,50.60166311)
\curveto(1015.66792852,50.64165999)(1015.69292849,50.68665995)(1015.72292725,50.73666311)
\curveto(1015.77292841,50.81665982)(1015.81792837,50.90165973)(1015.85792725,50.99166311)
\curveto(1015.8879283,51.09165954)(1015.91792827,51.19665944)(1015.94792725,51.30666311)
\curveto(1015.96792822,51.35665928)(1015.97792821,51.40165923)(1015.97792725,51.44166311)
\curveto(1015.96792822,51.49165914)(1015.96792822,51.54165909)(1015.97792725,51.59166311)
\curveto(1015.9879282,51.62165901)(1015.99792819,51.68165895)(1016.00792725,51.77166311)
\curveto(1016.01792817,51.87165876)(1016.01292817,51.94665869)(1015.99292725,51.99666311)
\curveto(1015.9829282,52.0366586)(1015.9829282,52.07665856)(1015.99292725,52.11666311)
\curveto(1015.99292819,52.15665848)(1015.9829282,52.19665844)(1015.96292725,52.23666311)
\curveto(1015.94292824,52.31665832)(1015.92792826,52.39665824)(1015.91792725,52.47666311)
\curveto(1015.89792829,52.55665808)(1015.87292831,52.631658)(1015.84292725,52.70166311)
\curveto(1015.70292848,53.04165759)(1015.50792868,53.31665732)(1015.25792725,53.52666311)
\curveto(1015.00792918,53.7366569)(1014.71292947,53.91165672)(1014.37292725,54.05166311)
\curveto(1014.25292993,54.10165653)(1014.12793006,54.1316565)(1013.99792725,54.14166311)
\curveto(1013.85793033,54.16165647)(1013.71793047,54.19165644)(1013.57792725,54.23166311)
}
}
{
\newrgbcolor{curcolor}{0 0 0}
\pscustom[linestyle=none,fillstyle=solid,fillcolor=curcolor]
{
}
}
{
\newrgbcolor{curcolor}{0 0 0}
\pscustom[linestyle=none,fillstyle=solid,fillcolor=curcolor]
{
\newpath
\moveto(1011.83792725,67.99510061)
\lineto(1012.09292725,67.99510061)
\curveto(1012.17293201,68.0050929)(1012.24793194,68.00009291)(1012.31792725,67.98010061)
\lineto(1012.55792725,67.98010061)
\lineto(1012.72292725,67.98010061)
\curveto(1012.82293136,67.96009295)(1012.92793126,67.95009296)(1013.03792725,67.95010061)
\curveto(1013.13793105,67.95009296)(1013.23793095,67.94009297)(1013.33792725,67.92010061)
\lineto(1013.48792725,67.92010061)
\curveto(1013.62793056,67.89009302)(1013.76793042,67.87009304)(1013.90792725,67.86010061)
\curveto(1014.03793015,67.85009306)(1014.16793002,67.82509308)(1014.29792725,67.78510061)
\curveto(1014.37792981,67.76509314)(1014.46292972,67.74509316)(1014.55292725,67.72510061)
\lineto(1014.79292725,67.66510061)
\lineto(1015.09292725,67.54510061)
\curveto(1015.182929,67.51509339)(1015.27292891,67.48009343)(1015.36292725,67.44010061)
\curveto(1015.5829286,67.34009357)(1015.79792839,67.2050937)(1016.00792725,67.03510061)
\curveto(1016.21792797,66.87509403)(1016.3879278,66.70009421)(1016.51792725,66.51010061)
\curveto(1016.55792763,66.46009445)(1016.59792759,66.40009451)(1016.63792725,66.33010061)
\curveto(1016.66792752,66.27009464)(1016.70292748,66.2100947)(1016.74292725,66.15010061)
\curveto(1016.79292739,66.07009484)(1016.83292735,65.97509493)(1016.86292725,65.86510061)
\curveto(1016.89292729,65.75509515)(1016.92292726,65.65009526)(1016.95292725,65.55010061)
\curveto(1016.99292719,65.44009547)(1017.01792717,65.33009558)(1017.02792725,65.22010061)
\curveto(1017.03792715,65.1100958)(1017.05292713,64.99509591)(1017.07292725,64.87510061)
\curveto(1017.0829271,64.83509607)(1017.0829271,64.79009612)(1017.07292725,64.74010061)
\curveto(1017.07292711,64.70009621)(1017.07792711,64.66009625)(1017.08792725,64.62010061)
\curveto(1017.09792709,64.58009633)(1017.10292708,64.52509638)(1017.10292725,64.45510061)
\curveto(1017.10292708,64.38509652)(1017.09792709,64.33509657)(1017.08792725,64.30510061)
\curveto(1017.06792712,64.25509665)(1017.06292712,64.2100967)(1017.07292725,64.17010061)
\curveto(1017.0829271,64.13009678)(1017.0829271,64.09509681)(1017.07292725,64.06510061)
\lineto(1017.07292725,63.97510061)
\curveto(1017.05292713,63.91509699)(1017.03792715,63.85009706)(1017.02792725,63.78010061)
\curveto(1017.02792716,63.72009719)(1017.02292716,63.65509725)(1017.01292725,63.58510061)
\curveto(1016.96292722,63.41509749)(1016.91292727,63.25509765)(1016.86292725,63.10510061)
\curveto(1016.81292737,62.95509795)(1016.74792744,62.8100981)(1016.66792725,62.67010061)
\curveto(1016.62792756,62.62009829)(1016.59792759,62.56509834)(1016.57792725,62.50510061)
\curveto(1016.54792764,62.45509845)(1016.51292767,62.4050985)(1016.47292725,62.35510061)
\curveto(1016.29292789,62.11509879)(1016.07292811,61.91509899)(1015.81292725,61.75510061)
\curveto(1015.55292863,61.59509931)(1015.26792892,61.45509945)(1014.95792725,61.33510061)
\curveto(1014.81792937,61.27509963)(1014.67792951,61.23009968)(1014.53792725,61.20010061)
\curveto(1014.3879298,61.17009974)(1014.23292995,61.13509977)(1014.07292725,61.09510061)
\curveto(1013.96293022,61.07509983)(1013.85293033,61.06009985)(1013.74292725,61.05010061)
\curveto(1013.63293055,61.04009987)(1013.52293066,61.02509988)(1013.41292725,61.00510061)
\curveto(1013.37293081,60.99509991)(1013.33293085,60.99009992)(1013.29292725,60.99010061)
\curveto(1013.25293093,61.00009991)(1013.21293097,61.00009991)(1013.17292725,60.99010061)
\curveto(1013.12293106,60.98009993)(1013.07293111,60.97509993)(1013.02292725,60.97510061)
\lineto(1012.85792725,60.97510061)
\curveto(1012.80793138,60.95509995)(1012.75793143,60.95009996)(1012.70792725,60.96010061)
\curveto(1012.64793154,60.97009994)(1012.59293159,60.97009994)(1012.54292725,60.96010061)
\curveto(1012.50293168,60.95009996)(1012.45793173,60.95009996)(1012.40792725,60.96010061)
\curveto(1012.35793183,60.97009994)(1012.30793188,60.96509994)(1012.25792725,60.94510061)
\curveto(1012.187932,60.92509998)(1012.11293207,60.92009999)(1012.03292725,60.93010061)
\curveto(1011.94293224,60.94009997)(1011.85793233,60.94509996)(1011.77792725,60.94510061)
\curveto(1011.6879325,60.94509996)(1011.5879326,60.94009997)(1011.47792725,60.93010061)
\curveto(1011.35793283,60.92009999)(1011.25793293,60.92509998)(1011.17792725,60.94510061)
\lineto(1010.89292725,60.94510061)
\lineto(1010.26292725,60.99010061)
\curveto(1010.16293402,61.00009991)(1010.06793412,61.0100999)(1009.97792725,61.02010061)
\lineto(1009.67792725,61.05010061)
\curveto(1009.62793456,61.07009984)(1009.57793461,61.07509983)(1009.52792725,61.06510061)
\curveto(1009.46793472,61.06509984)(1009.41293477,61.07509983)(1009.36292725,61.09510061)
\curveto(1009.19293499,61.14509976)(1009.02793516,61.18509972)(1008.86792725,61.21510061)
\curveto(1008.69793549,61.24509966)(1008.53793565,61.29509961)(1008.38792725,61.36510061)
\curveto(1007.92793626,61.55509935)(1007.55293663,61.77509913)(1007.26292725,62.02510061)
\curveto(1006.97293721,62.28509862)(1006.72793746,62.64509826)(1006.52792725,63.10510061)
\curveto(1006.47793771,63.23509767)(1006.44293774,63.36509754)(1006.42292725,63.49510061)
\curveto(1006.40293778,63.63509727)(1006.37793781,63.77509713)(1006.34792725,63.91510061)
\curveto(1006.33793785,63.98509692)(1006.33293785,64.05009686)(1006.33292725,64.11010061)
\curveto(1006.33293785,64.17009674)(1006.32793786,64.23509667)(1006.31792725,64.30510061)
\curveto(1006.29793789,65.13509577)(1006.44793774,65.8050951)(1006.76792725,66.31510061)
\curveto(1007.07793711,66.82509408)(1007.51793667,67.2050937)(1008.08792725,67.45510061)
\curveto(1008.20793598,67.5050934)(1008.33293585,67.55009336)(1008.46292725,67.59010061)
\curveto(1008.59293559,67.63009328)(1008.72793546,67.67509323)(1008.86792725,67.72510061)
\curveto(1008.94793524,67.74509316)(1009.03293515,67.76009315)(1009.12292725,67.77010061)
\lineto(1009.36292725,67.83010061)
\curveto(1009.47293471,67.86009305)(1009.5829346,67.87509303)(1009.69292725,67.87510061)
\curveto(1009.80293438,67.88509302)(1009.91293427,67.90009301)(1010.02292725,67.92010061)
\curveto(1010.07293411,67.94009297)(1010.11793407,67.94509296)(1010.15792725,67.93510061)
\curveto(1010.19793399,67.93509297)(1010.23793395,67.94009297)(1010.27792725,67.95010061)
\curveto(1010.32793386,67.96009295)(1010.3829338,67.96009295)(1010.44292725,67.95010061)
\curveto(1010.49293369,67.95009296)(1010.54293364,67.95509295)(1010.59292725,67.96510061)
\lineto(1010.72792725,67.96510061)
\curveto(1010.7879334,67.98509292)(1010.85793333,67.98509292)(1010.93792725,67.96510061)
\curveto(1011.00793318,67.95509295)(1011.07293311,67.96009295)(1011.13292725,67.98010061)
\curveto(1011.16293302,67.99009292)(1011.20293298,67.99509291)(1011.25292725,67.99510061)
\lineto(1011.37292725,67.99510061)
\lineto(1011.83792725,67.99510061)
\moveto(1014.16292725,66.45010061)
\curveto(1013.84293034,66.55009436)(1013.47793071,66.6100943)(1013.06792725,66.63010061)
\curveto(1012.65793153,66.65009426)(1012.24793194,66.66009425)(1011.83792725,66.66010061)
\curveto(1011.40793278,66.66009425)(1010.9879332,66.65009426)(1010.57792725,66.63010061)
\curveto(1010.16793402,66.6100943)(1009.7829344,66.56509434)(1009.42292725,66.49510061)
\curveto(1009.06293512,66.42509448)(1008.74293544,66.31509459)(1008.46292725,66.16510061)
\curveto(1008.17293601,66.02509488)(1007.93793625,65.83009508)(1007.75792725,65.58010061)
\curveto(1007.64793654,65.42009549)(1007.56793662,65.24009567)(1007.51792725,65.04010061)
\curveto(1007.45793673,64.84009607)(1007.42793676,64.59509631)(1007.42792725,64.30510061)
\curveto(1007.44793674,64.28509662)(1007.45793673,64.25009666)(1007.45792725,64.20010061)
\curveto(1007.44793674,64.15009676)(1007.44793674,64.1100968)(1007.45792725,64.08010061)
\curveto(1007.47793671,64.00009691)(1007.49793669,63.92509698)(1007.51792725,63.85510061)
\curveto(1007.52793666,63.79509711)(1007.54793664,63.73009718)(1007.57792725,63.66010061)
\curveto(1007.69793649,63.39009752)(1007.86793632,63.17009774)(1008.08792725,63.00010061)
\curveto(1008.29793589,62.84009807)(1008.54293564,62.7050982)(1008.82292725,62.59510061)
\curveto(1008.93293525,62.54509836)(1009.05293513,62.5050984)(1009.18292725,62.47510061)
\curveto(1009.30293488,62.45509845)(1009.42793476,62.43009848)(1009.55792725,62.40010061)
\curveto(1009.60793458,62.38009853)(1009.66293452,62.37009854)(1009.72292725,62.37010061)
\curveto(1009.77293441,62.37009854)(1009.82293436,62.36509854)(1009.87292725,62.35510061)
\curveto(1009.96293422,62.34509856)(1010.05793413,62.33509857)(1010.15792725,62.32510061)
\curveto(1010.24793394,62.31509859)(1010.34293384,62.3050986)(1010.44292725,62.29510061)
\curveto(1010.52293366,62.29509861)(1010.60793358,62.29009862)(1010.69792725,62.28010061)
\lineto(1010.93792725,62.28010061)
\lineto(1011.11792725,62.28010061)
\curveto(1011.14793304,62.27009864)(1011.182933,62.26509864)(1011.22292725,62.26510061)
\lineto(1011.35792725,62.26510061)
\lineto(1011.80792725,62.26510061)
\curveto(1011.8879323,62.26509864)(1011.97293221,62.26009865)(1012.06292725,62.25010061)
\curveto(1012.14293204,62.25009866)(1012.21793197,62.26009865)(1012.28792725,62.28010061)
\lineto(1012.55792725,62.28010061)
\curveto(1012.57793161,62.28009863)(1012.60793158,62.27509863)(1012.64792725,62.26510061)
\curveto(1012.67793151,62.26509864)(1012.70293148,62.27009864)(1012.72292725,62.28010061)
\curveto(1012.82293136,62.29009862)(1012.92293126,62.29509861)(1013.02292725,62.29510061)
\curveto(1013.11293107,62.3050986)(1013.21293097,62.31509859)(1013.32292725,62.32510061)
\curveto(1013.44293074,62.35509855)(1013.56793062,62.37009854)(1013.69792725,62.37010061)
\curveto(1013.81793037,62.38009853)(1013.93293025,62.4050985)(1014.04292725,62.44510061)
\curveto(1014.34292984,62.52509838)(1014.60792958,62.6100983)(1014.83792725,62.70010061)
\curveto(1015.06792912,62.80009811)(1015.2829289,62.94509796)(1015.48292725,63.13510061)
\curveto(1015.6829285,63.34509756)(1015.83292835,63.6100973)(1015.93292725,63.93010061)
\curveto(1015.95292823,63.97009694)(1015.96292822,64.0050969)(1015.96292725,64.03510061)
\curveto(1015.95292823,64.07509683)(1015.95792823,64.12009679)(1015.97792725,64.17010061)
\curveto(1015.9879282,64.2100967)(1015.99792819,64.28009663)(1016.00792725,64.38010061)
\curveto(1016.01792817,64.49009642)(1016.01292817,64.57509633)(1015.99292725,64.63510061)
\curveto(1015.97292821,64.7050962)(1015.96292822,64.77509613)(1015.96292725,64.84510061)
\curveto(1015.95292823,64.91509599)(1015.93792825,64.98009593)(1015.91792725,65.04010061)
\curveto(1015.85792833,65.24009567)(1015.77292841,65.42009549)(1015.66292725,65.58010061)
\curveto(1015.64292854,65.6100953)(1015.62292856,65.63509527)(1015.60292725,65.65510061)
\lineto(1015.54292725,65.71510061)
\curveto(1015.52292866,65.75509515)(1015.4829287,65.8050951)(1015.42292725,65.86510061)
\curveto(1015.2829289,65.96509494)(1015.15292903,66.05009486)(1015.03292725,66.12010061)
\curveto(1014.91292927,66.19009472)(1014.76792942,66.26009465)(1014.59792725,66.33010061)
\curveto(1014.52792966,66.36009455)(1014.45792973,66.38009453)(1014.38792725,66.39010061)
\curveto(1014.31792987,66.4100945)(1014.24292994,66.43009448)(1014.16292725,66.45010061)
}
}
{
\newrgbcolor{curcolor}{0 0 0}
\pscustom[linestyle=none,fillstyle=solid,fillcolor=curcolor]
{
\newpath
\moveto(1011.32792725,76.28470998)
\curveto(1011.40793278,76.28470235)(1011.4879327,76.28970234)(1011.56792725,76.29970998)
\curveto(1011.64793254,76.30970232)(1011.72293246,76.30470233)(1011.79292725,76.28470998)
\curveto(1011.83293235,76.26470237)(1011.87793231,76.25970237)(1011.92792725,76.26970998)
\curveto(1011.96793222,76.27970235)(1012.00793218,76.27970235)(1012.04792725,76.26970998)
\lineto(1012.19792725,76.26970998)
\curveto(1012.2879319,76.25970237)(1012.37793181,76.25470238)(1012.46792725,76.25470998)
\curveto(1012.54793164,76.25470238)(1012.62793156,76.24970238)(1012.70792725,76.23970998)
\lineto(1012.94792725,76.20970998)
\curveto(1013.01793117,76.19970243)(1013.09293109,76.18970244)(1013.17292725,76.17970998)
\curveto(1013.21293097,76.16970246)(1013.25293093,76.16470247)(1013.29292725,76.16470998)
\curveto(1013.33293085,76.16470247)(1013.37793081,76.15970247)(1013.42792725,76.14970998)
\curveto(1013.56793062,76.10970252)(1013.70793048,76.07970255)(1013.84792725,76.05970998)
\curveto(1013.9879302,76.04970258)(1014.12293006,76.01970261)(1014.25292725,75.96970998)
\curveto(1014.42292976,75.91970271)(1014.5879296,75.86470277)(1014.74792725,75.80470998)
\curveto(1014.90792928,75.75470288)(1015.06292912,75.69470294)(1015.21292725,75.62470998)
\curveto(1015.27292891,75.60470303)(1015.33292885,75.57470306)(1015.39292725,75.53470998)
\lineto(1015.54292725,75.44470998)
\curveto(1015.86292832,75.24470339)(1016.12792806,75.0297036)(1016.33792725,74.79970998)
\curveto(1016.54792764,74.56970406)(1016.72792746,74.27470436)(1016.87792725,73.91470998)
\curveto(1016.92792726,73.79470484)(1016.96292722,73.66470497)(1016.98292725,73.52470998)
\curveto(1017.00292718,73.39470524)(1017.02792716,73.25970537)(1017.05792725,73.11970998)
\curveto(1017.06792712,73.05970557)(1017.07292711,72.99970563)(1017.07292725,72.93970998)
\curveto(1017.07292711,72.87970575)(1017.07792711,72.81470582)(1017.08792725,72.74470998)
\curveto(1017.09792709,72.71470592)(1017.09792709,72.66470597)(1017.08792725,72.59470998)
\lineto(1017.08792725,72.44470998)
\lineto(1017.08792725,72.29470998)
\curveto(1017.06792712,72.21470642)(1017.05292713,72.1297065)(1017.04292725,72.03970998)
\curveto(1017.04292714,71.95970667)(1017.03292715,71.88470675)(1017.01292725,71.81470998)
\curveto(1017.00292718,71.77470686)(1016.99792719,71.73970689)(1016.99792725,71.70970998)
\curveto(1017.00792718,71.68970694)(1017.00292718,71.66470697)(1016.98292725,71.63470998)
\lineto(1016.92292725,71.36470998)
\curveto(1016.89292729,71.27470736)(1016.86292732,71.18970744)(1016.83292725,71.10970998)
\curveto(1016.59292759,70.5297081)(1016.22292796,70.09470854)(1015.72292725,69.80470998)
\curveto(1015.59292859,69.72470891)(1015.45792873,69.65970897)(1015.31792725,69.60970998)
\curveto(1015.17792901,69.56970906)(1015.02792916,69.52470911)(1014.86792725,69.47470998)
\curveto(1014.7879294,69.45470918)(1014.70792948,69.44970918)(1014.62792725,69.45970998)
\curveto(1014.54792964,69.47970915)(1014.49292969,69.51470912)(1014.46292725,69.56470998)
\curveto(1014.44292974,69.59470904)(1014.42792976,69.64970898)(1014.41792725,69.72970998)
\curveto(1014.39792979,69.80970882)(1014.3879298,69.89470874)(1014.38792725,69.98470998)
\curveto(1014.37792981,70.07470856)(1014.37792981,70.15970847)(1014.38792725,70.23970998)
\curveto(1014.39792979,70.3297083)(1014.40792978,70.39970823)(1014.41792725,70.44970998)
\curveto(1014.42792976,70.46970816)(1014.44292974,70.49470814)(1014.46292725,70.52470998)
\curveto(1014.4829297,70.56470807)(1014.50292968,70.59470804)(1014.52292725,70.61470998)
\curveto(1014.60292958,70.67470796)(1014.69792949,70.71970791)(1014.80792725,70.74970998)
\curveto(1014.91792927,70.78970784)(1015.01792917,70.8347078)(1015.10792725,70.88470998)
\curveto(1015.49792869,71.1347075)(1015.76792842,71.50470713)(1015.91792725,71.99470998)
\curveto(1015.93792825,72.06470657)(1015.95292823,72.1347065)(1015.96292725,72.20470998)
\curveto(1015.96292822,72.28470635)(1015.97292821,72.36470627)(1015.99292725,72.44470998)
\curveto(1016.00292818,72.48470615)(1016.00792818,72.53970609)(1016.00792725,72.60970998)
\curveto(1016.00792818,72.68970594)(1016.00292818,72.74470589)(1015.99292725,72.77470998)
\curveto(1015.9829282,72.80470583)(1015.97792821,72.8347058)(1015.97792725,72.86470998)
\lineto(1015.97792725,72.96970998)
\curveto(1015.95792823,73.04970558)(1015.93792825,73.12470551)(1015.91792725,73.19470998)
\curveto(1015.89792829,73.27470536)(1015.87292831,73.34970528)(1015.84292725,73.41970998)
\curveto(1015.69292849,73.76970486)(1015.47792871,74.03970459)(1015.19792725,74.22970998)
\curveto(1014.91792927,74.41970421)(1014.59292959,74.57470406)(1014.22292725,74.69470998)
\curveto(1014.14293004,74.72470391)(1014.06793012,74.74470389)(1013.99792725,74.75470998)
\curveto(1013.92793026,74.77470386)(1013.85293033,74.79470384)(1013.77292725,74.81470998)
\curveto(1013.6829305,74.8347038)(1013.5879306,74.84970378)(1013.48792725,74.85970998)
\curveto(1013.37793081,74.87970375)(1013.27293091,74.89970373)(1013.17292725,74.91970998)
\curveto(1013.12293106,74.9297037)(1013.07293111,74.9347037)(1013.02292725,74.93470998)
\curveto(1012.96293122,74.94470369)(1012.90793128,74.94970368)(1012.85792725,74.94970998)
\curveto(1012.79793139,74.96970366)(1012.72293146,74.97970365)(1012.63292725,74.97970998)
\curveto(1012.53293165,74.97970365)(1012.45293173,74.96970366)(1012.39292725,74.94970998)
\curveto(1012.30293188,74.91970371)(1012.26293192,74.86970376)(1012.27292725,74.79970998)
\curveto(1012.2829319,74.73970389)(1012.31293187,74.68470395)(1012.36292725,74.63470998)
\curveto(1012.41293177,74.55470408)(1012.47293171,74.48470415)(1012.54292725,74.42470998)
\curveto(1012.61293157,74.37470426)(1012.67293151,74.30970432)(1012.72292725,74.22970998)
\curveto(1012.83293135,74.06970456)(1012.93293125,73.90470473)(1013.02292725,73.73470998)
\curveto(1013.10293108,73.56470507)(1013.17293101,73.36970526)(1013.23292725,73.14970998)
\curveto(1013.26293092,73.04970558)(1013.27793091,72.94970568)(1013.27792725,72.84970998)
\curveto(1013.27793091,72.75970587)(1013.2879309,72.65970597)(1013.30792725,72.54970998)
\lineto(1013.30792725,72.39970998)
\curveto(1013.2879309,72.34970628)(1013.2829309,72.29970633)(1013.29292725,72.24970998)
\curveto(1013.30293088,72.20970642)(1013.30293088,72.16970646)(1013.29292725,72.12970998)
\curveto(1013.2829309,72.09970653)(1013.27793091,72.05470658)(1013.27792725,71.99470998)
\curveto(1013.26793092,71.9347067)(1013.25793093,71.86970676)(1013.24792725,71.79970998)
\lineto(1013.21792725,71.61970998)
\curveto(1013.09793109,71.16970746)(1012.93293125,70.78970784)(1012.72292725,70.47970998)
\curveto(1012.53293165,70.20970842)(1012.30293188,69.97970865)(1012.03292725,69.78970998)
\curveto(1011.75293243,69.60970902)(1011.43793275,69.46470917)(1011.08792725,69.35470998)
\lineto(1010.87792725,69.29470998)
\curveto(1010.79793339,69.28470935)(1010.71793347,69.26970936)(1010.63792725,69.24970998)
\curveto(1010.60793358,69.23970939)(1010.57793361,69.2347094)(1010.54792725,69.23470998)
\curveto(1010.51793367,69.2347094)(1010.4879337,69.2297094)(1010.45792725,69.21970998)
\curveto(1010.39793379,69.20970942)(1010.33793385,69.20470943)(1010.27792725,69.20470998)
\curveto(1010.20793398,69.20470943)(1010.14793404,69.19470944)(1010.09792725,69.17470998)
\lineto(1009.91792725,69.17470998)
\curveto(1009.86793432,69.16470947)(1009.79793439,69.15970947)(1009.70792725,69.15970998)
\curveto(1009.61793457,69.15970947)(1009.54793464,69.16970946)(1009.49792725,69.18970998)
\lineto(1009.33292725,69.18970998)
\curveto(1009.25293493,69.20970942)(1009.17793501,69.21970941)(1009.10792725,69.21970998)
\curveto(1009.03793515,69.2297094)(1008.96793522,69.24470939)(1008.89792725,69.26470998)
\curveto(1008.69793549,69.32470931)(1008.50793568,69.38470925)(1008.32792725,69.44470998)
\curveto(1008.14793604,69.51470912)(1007.97793621,69.60470903)(1007.81792725,69.71470998)
\curveto(1007.74793644,69.75470888)(1007.6829365,69.79470884)(1007.62292725,69.83470998)
\lineto(1007.44292725,69.98470998)
\curveto(1007.43293675,70.00470863)(1007.41793677,70.02470861)(1007.39792725,70.04470998)
\curveto(1007.26793692,70.1347085)(1007.15793703,70.24470839)(1007.06792725,70.37470998)
\curveto(1006.86793732,70.634708)(1006.71293747,70.89970773)(1006.60292725,71.16970998)
\curveto(1006.56293762,71.24970738)(1006.53293765,71.3297073)(1006.51292725,71.40970998)
\curveto(1006.4829377,71.49970713)(1006.45793773,71.58970704)(1006.43792725,71.67970998)
\curveto(1006.40793778,71.77970685)(1006.3879378,71.87970675)(1006.37792725,71.97970998)
\curveto(1006.36793782,72.07970655)(1006.35293783,72.18470645)(1006.33292725,72.29470998)
\curveto(1006.32293786,72.32470631)(1006.32293786,72.36470627)(1006.33292725,72.41470998)
\curveto(1006.34293784,72.47470616)(1006.33793785,72.51470612)(1006.31792725,72.53470998)
\curveto(1006.29793789,73.25470538)(1006.41293777,73.85470478)(1006.66292725,74.33470998)
\curveto(1006.91293727,74.81470382)(1007.25293693,75.18970344)(1007.68292725,75.45970998)
\curveto(1007.82293636,75.54970308)(1007.96793622,75.629703)(1008.11792725,75.69970998)
\curveto(1008.26793592,75.76970286)(1008.42793576,75.83970279)(1008.59792725,75.90970998)
\curveto(1008.73793545,75.95970267)(1008.8879353,75.99970263)(1009.04792725,76.02970998)
\curveto(1009.20793498,76.05970257)(1009.36793482,76.09470254)(1009.52792725,76.13470998)
\curveto(1009.57793461,76.15470248)(1009.63293455,76.16470247)(1009.69292725,76.16470998)
\curveto(1009.74293444,76.16470247)(1009.79293439,76.16970246)(1009.84292725,76.17970998)
\curveto(1009.90293428,76.19970243)(1009.96793422,76.20970242)(1010.03792725,76.20970998)
\curveto(1010.09793409,76.20970242)(1010.15293403,76.21970241)(1010.20292725,76.23970998)
\lineto(1010.36792725,76.23970998)
\curveto(1010.41793377,76.25970237)(1010.46793372,76.26470237)(1010.51792725,76.25470998)
\curveto(1010.56793362,76.24470239)(1010.61793357,76.24970238)(1010.66792725,76.26970998)
\curveto(1010.6879335,76.26970236)(1010.71293347,76.26470237)(1010.74292725,76.25470998)
\curveto(1010.77293341,76.25470238)(1010.79793339,76.25970237)(1010.81792725,76.26970998)
\curveto(1010.84793334,76.27970235)(1010.8829333,76.27970235)(1010.92292725,76.26970998)
\curveto(1010.96293322,76.26970236)(1011.00293318,76.27470236)(1011.04292725,76.28470998)
\curveto(1011.0829331,76.29470234)(1011.12793306,76.29470234)(1011.17792725,76.28470998)
\lineto(1011.32792725,76.28470998)
\moveto(1010.02292725,74.78470998)
\curveto(1009.97293421,74.79470384)(1009.91293427,74.79970383)(1009.84292725,74.79970998)
\curveto(1009.77293441,74.79970383)(1009.71293447,74.79470384)(1009.66292725,74.78470998)
\curveto(1009.61293457,74.77470386)(1009.53793465,74.76970386)(1009.43792725,74.76970998)
\curveto(1009.35793483,74.74970388)(1009.2829349,74.7297039)(1009.21292725,74.70970998)
\curveto(1009.14293504,74.69970393)(1009.07293511,74.68470395)(1009.00292725,74.66470998)
\curveto(1008.57293561,74.52470411)(1008.23793595,74.3297043)(1007.99792725,74.07970998)
\curveto(1007.75793643,73.83970479)(1007.57793661,73.49470514)(1007.45792725,73.04470998)
\curveto(1007.43793675,72.95470568)(1007.42793676,72.85470578)(1007.42792725,72.74470998)
\lineto(1007.42792725,72.41470998)
\curveto(1007.44793674,72.39470624)(1007.45793673,72.35970627)(1007.45792725,72.30970998)
\curveto(1007.44793674,72.25970637)(1007.44793674,72.21470642)(1007.45792725,72.17470998)
\curveto(1007.47793671,72.09470654)(1007.49793669,72.01970661)(1007.51792725,71.94970998)
\lineto(1007.57792725,71.73970998)
\curveto(1007.70793648,71.44970718)(1007.8879363,71.21970741)(1008.11792725,71.04970998)
\curveto(1008.33793585,70.87970775)(1008.59793559,70.74470789)(1008.89792725,70.64470998)
\curveto(1008.9879352,70.61470802)(1009.0829351,70.58970804)(1009.18292725,70.56970998)
\curveto(1009.27293491,70.55970807)(1009.36793482,70.54470809)(1009.46792725,70.52470998)
\lineto(1009.60292725,70.52470998)
\curveto(1009.71293447,70.49470814)(1009.85293433,70.48470815)(1010.02292725,70.49470998)
\curveto(1010.182934,70.51470812)(1010.31293387,70.5347081)(1010.41292725,70.55470998)
\curveto(1010.47293371,70.57470806)(1010.53293365,70.58970804)(1010.59292725,70.59970998)
\curveto(1010.64293354,70.60970802)(1010.69293349,70.62470801)(1010.74292725,70.64470998)
\curveto(1010.94293324,70.72470791)(1011.13293305,70.81970781)(1011.31292725,70.92970998)
\curveto(1011.49293269,71.04970758)(1011.63793255,71.18970744)(1011.74792725,71.34970998)
\curveto(1011.79793239,71.39970723)(1011.83793235,71.45470718)(1011.86792725,71.51470998)
\curveto(1011.89793229,71.57470706)(1011.93293225,71.634707)(1011.97292725,71.69470998)
\curveto(1012.05293213,71.84470679)(1012.11793207,72.0297066)(1012.16792725,72.24970998)
\curveto(1012.187932,72.29970633)(1012.19293199,72.33970629)(1012.18292725,72.36970998)
\curveto(1012.17293201,72.40970622)(1012.17793201,72.45470618)(1012.19792725,72.50470998)
\curveto(1012.20793198,72.54470609)(1012.21293197,72.59970603)(1012.21292725,72.66970998)
\curveto(1012.21293197,72.73970589)(1012.20793198,72.79970583)(1012.19792725,72.84970998)
\curveto(1012.17793201,72.94970568)(1012.16293202,73.04470559)(1012.15292725,73.13470998)
\curveto(1012.13293205,73.22470541)(1012.10293208,73.31470532)(1012.06292725,73.40470998)
\curveto(1011.84293234,73.94470469)(1011.44793274,74.33970429)(1010.87792725,74.58970998)
\curveto(1010.77793341,74.63970399)(1010.67793351,74.67470396)(1010.57792725,74.69470998)
\curveto(1010.46793372,74.71470392)(1010.35793383,74.73970389)(1010.24792725,74.76970998)
\curveto(1010.14793404,74.76970386)(1010.07293411,74.77470386)(1010.02292725,74.78470998)
}
}
{
\newrgbcolor{curcolor}{0 0 0}
\pscustom[linestyle=none,fillstyle=solid,fillcolor=curcolor]
{
\newpath
\moveto(1015.28792725,78.63431936)
\lineto(1015.28792725,79.26431936)
\lineto(1015.28792725,79.45931936)
\curveto(1015.2879289,79.52931683)(1015.29792889,79.58931677)(1015.31792725,79.63931936)
\curveto(1015.35792883,79.70931665)(1015.39792879,79.7593166)(1015.43792725,79.78931936)
\curveto(1015.4879287,79.82931653)(1015.55292863,79.84931651)(1015.63292725,79.84931936)
\curveto(1015.71292847,79.8593165)(1015.79792839,79.86431649)(1015.88792725,79.86431936)
\lineto(1016.60792725,79.86431936)
\curveto(1017.0879271,79.86431649)(1017.49792669,79.80431655)(1017.83792725,79.68431936)
\curveto(1018.17792601,79.56431679)(1018.45292573,79.36931699)(1018.66292725,79.09931936)
\curveto(1018.71292547,79.02931733)(1018.75792543,78.9593174)(1018.79792725,78.88931936)
\curveto(1018.84792534,78.82931753)(1018.89292529,78.7543176)(1018.93292725,78.66431936)
\curveto(1018.94292524,78.64431771)(1018.95292523,78.61431774)(1018.96292725,78.57431936)
\curveto(1018.9829252,78.53431782)(1018.9879252,78.48931787)(1018.97792725,78.43931936)
\curveto(1018.94792524,78.34931801)(1018.87292531,78.29431806)(1018.75292725,78.27431936)
\curveto(1018.64292554,78.2543181)(1018.54792564,78.26931809)(1018.46792725,78.31931936)
\curveto(1018.39792579,78.34931801)(1018.33292585,78.39431796)(1018.27292725,78.45431936)
\curveto(1018.22292596,78.52431783)(1018.17292601,78.58931777)(1018.12292725,78.64931936)
\curveto(1018.07292611,78.71931764)(1017.99792619,78.77931758)(1017.89792725,78.82931936)
\curveto(1017.80792638,78.88931747)(1017.71792647,78.93931742)(1017.62792725,78.97931936)
\curveto(1017.59792659,78.99931736)(1017.53792665,79.02431733)(1017.44792725,79.05431936)
\curveto(1017.36792682,79.08431727)(1017.29792689,79.08931727)(1017.23792725,79.06931936)
\curveto(1017.09792709,79.03931732)(1017.00792718,78.97931738)(1016.96792725,78.88931936)
\curveto(1016.93792725,78.80931755)(1016.92292726,78.71931764)(1016.92292725,78.61931936)
\curveto(1016.92292726,78.51931784)(1016.89792729,78.43431792)(1016.84792725,78.36431936)
\curveto(1016.77792741,78.27431808)(1016.63792755,78.22931813)(1016.42792725,78.22931936)
\lineto(1015.87292725,78.22931936)
\lineto(1015.64792725,78.22931936)
\curveto(1015.56792862,78.23931812)(1015.50292868,78.2593181)(1015.45292725,78.28931936)
\curveto(1015.37292881,78.34931801)(1015.32792886,78.41931794)(1015.31792725,78.49931936)
\curveto(1015.30792888,78.51931784)(1015.30292888,78.53931782)(1015.30292725,78.55931936)
\curveto(1015.30292888,78.58931777)(1015.29792889,78.61431774)(1015.28792725,78.63431936)
}
}
{
\newrgbcolor{curcolor}{0 0 0}
\pscustom[linestyle=none,fillstyle=solid,fillcolor=curcolor]
{
}
}
{
\newrgbcolor{curcolor}{0 0 0}
\pscustom[linestyle=none,fillstyle=solid,fillcolor=curcolor]
{
\newpath
\moveto(1006.31792725,89.26463186)
\curveto(1006.30793788,89.95462722)(1006.42793776,90.55462662)(1006.67792725,91.06463186)
\curveto(1006.92793726,91.58462559)(1007.26293692,91.9796252)(1007.68292725,92.24963186)
\curveto(1007.76293642,92.29962488)(1007.85293633,92.34462483)(1007.95292725,92.38463186)
\curveto(1008.04293614,92.42462475)(1008.13793605,92.46962471)(1008.23792725,92.51963186)
\curveto(1008.33793585,92.55962462)(1008.43793575,92.58962459)(1008.53792725,92.60963186)
\curveto(1008.63793555,92.62962455)(1008.74293544,92.64962453)(1008.85292725,92.66963186)
\curveto(1008.90293528,92.68962449)(1008.94793524,92.69462448)(1008.98792725,92.68463186)
\curveto(1009.02793516,92.6746245)(1009.07293511,92.6796245)(1009.12292725,92.69963186)
\curveto(1009.17293501,92.70962447)(1009.25793493,92.71462446)(1009.37792725,92.71463186)
\curveto(1009.4879347,92.71462446)(1009.57293461,92.70962447)(1009.63292725,92.69963186)
\curveto(1009.69293449,92.6796245)(1009.75293443,92.66962451)(1009.81292725,92.66963186)
\curveto(1009.87293431,92.6796245)(1009.93293425,92.6746245)(1009.99292725,92.65463186)
\curveto(1010.13293405,92.61462456)(1010.26793392,92.5796246)(1010.39792725,92.54963186)
\curveto(1010.52793366,92.51962466)(1010.65293353,92.4796247)(1010.77292725,92.42963186)
\curveto(1010.91293327,92.36962481)(1011.03793315,92.29962488)(1011.14792725,92.21963186)
\curveto(1011.25793293,92.14962503)(1011.36793282,92.0746251)(1011.47792725,91.99463186)
\lineto(1011.53792725,91.93463186)
\curveto(1011.55793263,91.92462525)(1011.57793261,91.90962527)(1011.59792725,91.88963186)
\curveto(1011.75793243,91.76962541)(1011.90293228,91.63462554)(1012.03292725,91.48463186)
\curveto(1012.16293202,91.33462584)(1012.2879319,91.174626)(1012.40792725,91.00463186)
\curveto(1012.62793156,90.69462648)(1012.83293135,90.39962678)(1013.02292725,90.11963186)
\curveto(1013.16293102,89.88962729)(1013.29793089,89.65962752)(1013.42792725,89.42963186)
\curveto(1013.55793063,89.20962797)(1013.69293049,88.98962819)(1013.83292725,88.76963186)
\curveto(1014.00293018,88.51962866)(1014.18293,88.2796289)(1014.37292725,88.04963186)
\curveto(1014.56292962,87.82962935)(1014.7879294,87.63962954)(1015.04792725,87.47963186)
\curveto(1015.10792908,87.43962974)(1015.16792902,87.40462977)(1015.22792725,87.37463186)
\curveto(1015.27792891,87.34462983)(1015.34292884,87.31462986)(1015.42292725,87.28463186)
\curveto(1015.49292869,87.26462991)(1015.55292863,87.25962992)(1015.60292725,87.26963186)
\curveto(1015.67292851,87.28962989)(1015.72792846,87.32462985)(1015.76792725,87.37463186)
\curveto(1015.79792839,87.42462975)(1015.81792837,87.48462969)(1015.82792725,87.55463186)
\lineto(1015.82792725,87.79463186)
\lineto(1015.82792725,88.54463186)
\lineto(1015.82792725,91.34963186)
\lineto(1015.82792725,92.00963186)
\curveto(1015.82792836,92.09962508)(1015.83292835,92.18462499)(1015.84292725,92.26463186)
\curveto(1015.84292834,92.34462483)(1015.86292832,92.40962477)(1015.90292725,92.45963186)
\curveto(1015.94292824,92.50962467)(1016.01792817,92.54962463)(1016.12792725,92.57963186)
\curveto(1016.22792796,92.61962456)(1016.32792786,92.62962455)(1016.42792725,92.60963186)
\lineto(1016.56292725,92.60963186)
\curveto(1016.63292755,92.58962459)(1016.69292749,92.56962461)(1016.74292725,92.54963186)
\curveto(1016.79292739,92.52962465)(1016.83292735,92.49462468)(1016.86292725,92.44463186)
\curveto(1016.90292728,92.39462478)(1016.92292726,92.32462485)(1016.92292725,92.23463186)
\lineto(1016.92292725,91.96463186)
\lineto(1016.92292725,91.06463186)
\lineto(1016.92292725,87.55463186)
\lineto(1016.92292725,86.48963186)
\curveto(1016.92292726,86.40963077)(1016.92792726,86.31963086)(1016.93792725,86.21963186)
\curveto(1016.93792725,86.11963106)(1016.92792726,86.03463114)(1016.90792725,85.96463186)
\curveto(1016.83792735,85.75463142)(1016.65792753,85.68963149)(1016.36792725,85.76963186)
\curveto(1016.32792786,85.7796314)(1016.29292789,85.7796314)(1016.26292725,85.76963186)
\curveto(1016.22292796,85.76963141)(1016.17792801,85.7796314)(1016.12792725,85.79963186)
\curveto(1016.04792814,85.81963136)(1015.96292822,85.83963134)(1015.87292725,85.85963186)
\curveto(1015.7829284,85.8796313)(1015.69792849,85.90463127)(1015.61792725,85.93463186)
\curveto(1015.12792906,86.09463108)(1014.71292947,86.29463088)(1014.37292725,86.53463186)
\curveto(1014.12293006,86.71463046)(1013.89793029,86.91963026)(1013.69792725,87.14963186)
\curveto(1013.4879307,87.3796298)(1013.29293089,87.61962956)(1013.11292725,87.86963186)
\curveto(1012.93293125,88.12962905)(1012.76293142,88.39462878)(1012.60292725,88.66463186)
\curveto(1012.43293175,88.94462823)(1012.25793193,89.21462796)(1012.07792725,89.47463186)
\curveto(1011.99793219,89.58462759)(1011.92293226,89.68962749)(1011.85292725,89.78963186)
\curveto(1011.7829324,89.89962728)(1011.70793248,90.00962717)(1011.62792725,90.11963186)
\curveto(1011.59793259,90.15962702)(1011.56793262,90.19462698)(1011.53792725,90.22463186)
\curveto(1011.49793269,90.26462691)(1011.46793272,90.30462687)(1011.44792725,90.34463186)
\curveto(1011.33793285,90.48462669)(1011.21293297,90.60962657)(1011.07292725,90.71963186)
\curveto(1011.04293314,90.73962644)(1011.01793317,90.76462641)(1010.99792725,90.79463186)
\curveto(1010.96793322,90.82462635)(1010.93793325,90.84962633)(1010.90792725,90.86963186)
\curveto(1010.80793338,90.94962623)(1010.70793348,91.01462616)(1010.60792725,91.06463186)
\curveto(1010.50793368,91.12462605)(1010.39793379,91.179626)(1010.27792725,91.22963186)
\curveto(1010.20793398,91.25962592)(1010.13293405,91.2796259)(1010.05292725,91.28963186)
\lineto(1009.81292725,91.34963186)
\lineto(1009.72292725,91.34963186)
\curveto(1009.69293449,91.35962582)(1009.66293452,91.36462581)(1009.63292725,91.36463186)
\curveto(1009.56293462,91.38462579)(1009.46793472,91.38962579)(1009.34792725,91.37963186)
\curveto(1009.21793497,91.3796258)(1009.11793507,91.36962581)(1009.04792725,91.34963186)
\curveto(1008.96793522,91.32962585)(1008.89293529,91.30962587)(1008.82292725,91.28963186)
\curveto(1008.74293544,91.2796259)(1008.66293552,91.25962592)(1008.58292725,91.22963186)
\curveto(1008.34293584,91.11962606)(1008.14293604,90.96962621)(1007.98292725,90.77963186)
\curveto(1007.81293637,90.59962658)(1007.67293651,90.3796268)(1007.56292725,90.11963186)
\curveto(1007.54293664,90.04962713)(1007.52793666,89.9796272)(1007.51792725,89.90963186)
\curveto(1007.49793669,89.83962734)(1007.47793671,89.76462741)(1007.45792725,89.68463186)
\curveto(1007.43793675,89.60462757)(1007.42793676,89.49462768)(1007.42792725,89.35463186)
\curveto(1007.42793676,89.22462795)(1007.43793675,89.11962806)(1007.45792725,89.03963186)
\curveto(1007.46793672,88.9796282)(1007.47293671,88.92462825)(1007.47292725,88.87463186)
\curveto(1007.47293671,88.82462835)(1007.4829367,88.7746284)(1007.50292725,88.72463186)
\curveto(1007.54293664,88.62462855)(1007.5829366,88.52962865)(1007.62292725,88.43963186)
\curveto(1007.66293652,88.35962882)(1007.70793648,88.2796289)(1007.75792725,88.19963186)
\curveto(1007.77793641,88.16962901)(1007.80293638,88.13962904)(1007.83292725,88.10963186)
\curveto(1007.86293632,88.08962909)(1007.8879363,88.06462911)(1007.90792725,88.03463186)
\lineto(1007.98292725,87.95963186)
\curveto(1008.00293618,87.92962925)(1008.02293616,87.90462927)(1008.04292725,87.88463186)
\lineto(1008.25292725,87.73463186)
\curveto(1008.31293587,87.69462948)(1008.37793581,87.64962953)(1008.44792725,87.59963186)
\curveto(1008.53793565,87.53962964)(1008.64293554,87.48962969)(1008.76292725,87.44963186)
\curveto(1008.87293531,87.41962976)(1008.9829352,87.38462979)(1009.09292725,87.34463186)
\curveto(1009.20293498,87.30462987)(1009.34793484,87.2796299)(1009.52792725,87.26963186)
\curveto(1009.69793449,87.25962992)(1009.82293436,87.22962995)(1009.90292725,87.17963186)
\curveto(1009.9829342,87.12963005)(1010.02793416,87.05463012)(1010.03792725,86.95463186)
\curveto(1010.04793414,86.85463032)(1010.05293413,86.74463043)(1010.05292725,86.62463186)
\curveto(1010.05293413,86.58463059)(1010.05793413,86.54463063)(1010.06792725,86.50463186)
\curveto(1010.06793412,86.46463071)(1010.06293412,86.42963075)(1010.05292725,86.39963186)
\curveto(1010.03293415,86.34963083)(1010.02293416,86.29963088)(1010.02292725,86.24963186)
\curveto(1010.02293416,86.20963097)(1010.01293417,86.16963101)(1009.99292725,86.12963186)
\curveto(1009.93293425,86.03963114)(1009.79793439,85.99463118)(1009.58792725,85.99463186)
\lineto(1009.46792725,85.99463186)
\curveto(1009.40793478,86.00463117)(1009.34793484,86.00963117)(1009.28792725,86.00963186)
\curveto(1009.21793497,86.01963116)(1009.15293503,86.02963115)(1009.09292725,86.03963186)
\curveto(1008.9829352,86.05963112)(1008.8829353,86.0796311)(1008.79292725,86.09963186)
\curveto(1008.69293549,86.11963106)(1008.59793559,86.14963103)(1008.50792725,86.18963186)
\curveto(1008.43793575,86.20963097)(1008.37793581,86.22963095)(1008.32792725,86.24963186)
\lineto(1008.14792725,86.30963186)
\curveto(1007.8879363,86.42963075)(1007.64293654,86.58463059)(1007.41292725,86.77463186)
\curveto(1007.182937,86.9746302)(1006.99793719,87.18962999)(1006.85792725,87.41963186)
\curveto(1006.77793741,87.52962965)(1006.71293747,87.64462953)(1006.66292725,87.76463186)
\lineto(1006.51292725,88.15463186)
\curveto(1006.46293772,88.26462891)(1006.43293775,88.3796288)(1006.42292725,88.49963186)
\curveto(1006.40293778,88.61962856)(1006.37793781,88.74462843)(1006.34792725,88.87463186)
\curveto(1006.34793784,88.94462823)(1006.34793784,89.00962817)(1006.34792725,89.06963186)
\curveto(1006.33793785,89.12962805)(1006.32793786,89.19462798)(1006.31792725,89.26463186)
}
}
{
\newrgbcolor{curcolor}{0 0 0}
\pscustom[linestyle=none,fillstyle=solid,fillcolor=curcolor]
{
\newpath
\moveto(1011.83792725,101.36424123)
\lineto(1012.09292725,101.36424123)
\curveto(1012.17293201,101.37423353)(1012.24793194,101.36923353)(1012.31792725,101.34924123)
\lineto(1012.55792725,101.34924123)
\lineto(1012.72292725,101.34924123)
\curveto(1012.82293136,101.32923357)(1012.92793126,101.31923358)(1013.03792725,101.31924123)
\curveto(1013.13793105,101.31923358)(1013.23793095,101.30923359)(1013.33792725,101.28924123)
\lineto(1013.48792725,101.28924123)
\curveto(1013.62793056,101.25923364)(1013.76793042,101.23923366)(1013.90792725,101.22924123)
\curveto(1014.03793015,101.21923368)(1014.16793002,101.19423371)(1014.29792725,101.15424123)
\curveto(1014.37792981,101.13423377)(1014.46292972,101.11423379)(1014.55292725,101.09424123)
\lineto(1014.79292725,101.03424123)
\lineto(1015.09292725,100.91424123)
\curveto(1015.182929,100.88423402)(1015.27292891,100.84923405)(1015.36292725,100.80924123)
\curveto(1015.5829286,100.70923419)(1015.79792839,100.57423433)(1016.00792725,100.40424123)
\curveto(1016.21792797,100.24423466)(1016.3879278,100.06923483)(1016.51792725,99.87924123)
\curveto(1016.55792763,99.82923507)(1016.59792759,99.76923513)(1016.63792725,99.69924123)
\curveto(1016.66792752,99.63923526)(1016.70292748,99.57923532)(1016.74292725,99.51924123)
\curveto(1016.79292739,99.43923546)(1016.83292735,99.34423556)(1016.86292725,99.23424123)
\curveto(1016.89292729,99.12423578)(1016.92292726,99.01923588)(1016.95292725,98.91924123)
\curveto(1016.99292719,98.80923609)(1017.01792717,98.6992362)(1017.02792725,98.58924123)
\curveto(1017.03792715,98.47923642)(1017.05292713,98.36423654)(1017.07292725,98.24424123)
\curveto(1017.0829271,98.2042367)(1017.0829271,98.15923674)(1017.07292725,98.10924123)
\curveto(1017.07292711,98.06923683)(1017.07792711,98.02923687)(1017.08792725,97.98924123)
\curveto(1017.09792709,97.94923695)(1017.10292708,97.89423701)(1017.10292725,97.82424123)
\curveto(1017.10292708,97.75423715)(1017.09792709,97.7042372)(1017.08792725,97.67424123)
\curveto(1017.06792712,97.62423728)(1017.06292712,97.57923732)(1017.07292725,97.53924123)
\curveto(1017.0829271,97.4992374)(1017.0829271,97.46423744)(1017.07292725,97.43424123)
\lineto(1017.07292725,97.34424123)
\curveto(1017.05292713,97.28423762)(1017.03792715,97.21923768)(1017.02792725,97.14924123)
\curveto(1017.02792716,97.08923781)(1017.02292716,97.02423788)(1017.01292725,96.95424123)
\curveto(1016.96292722,96.78423812)(1016.91292727,96.62423828)(1016.86292725,96.47424123)
\curveto(1016.81292737,96.32423858)(1016.74792744,96.17923872)(1016.66792725,96.03924123)
\curveto(1016.62792756,95.98923891)(1016.59792759,95.93423897)(1016.57792725,95.87424123)
\curveto(1016.54792764,95.82423908)(1016.51292767,95.77423913)(1016.47292725,95.72424123)
\curveto(1016.29292789,95.48423942)(1016.07292811,95.28423962)(1015.81292725,95.12424123)
\curveto(1015.55292863,94.96423994)(1015.26792892,94.82424008)(1014.95792725,94.70424123)
\curveto(1014.81792937,94.64424026)(1014.67792951,94.5992403)(1014.53792725,94.56924123)
\curveto(1014.3879298,94.53924036)(1014.23292995,94.5042404)(1014.07292725,94.46424123)
\curveto(1013.96293022,94.44424046)(1013.85293033,94.42924047)(1013.74292725,94.41924123)
\curveto(1013.63293055,94.40924049)(1013.52293066,94.39424051)(1013.41292725,94.37424123)
\curveto(1013.37293081,94.36424054)(1013.33293085,94.35924054)(1013.29292725,94.35924123)
\curveto(1013.25293093,94.36924053)(1013.21293097,94.36924053)(1013.17292725,94.35924123)
\curveto(1013.12293106,94.34924055)(1013.07293111,94.34424056)(1013.02292725,94.34424123)
\lineto(1012.85792725,94.34424123)
\curveto(1012.80793138,94.32424058)(1012.75793143,94.31924058)(1012.70792725,94.32924123)
\curveto(1012.64793154,94.33924056)(1012.59293159,94.33924056)(1012.54292725,94.32924123)
\curveto(1012.50293168,94.31924058)(1012.45793173,94.31924058)(1012.40792725,94.32924123)
\curveto(1012.35793183,94.33924056)(1012.30793188,94.33424057)(1012.25792725,94.31424123)
\curveto(1012.187932,94.29424061)(1012.11293207,94.28924061)(1012.03292725,94.29924123)
\curveto(1011.94293224,94.30924059)(1011.85793233,94.31424059)(1011.77792725,94.31424123)
\curveto(1011.6879325,94.31424059)(1011.5879326,94.30924059)(1011.47792725,94.29924123)
\curveto(1011.35793283,94.28924061)(1011.25793293,94.29424061)(1011.17792725,94.31424123)
\lineto(1010.89292725,94.31424123)
\lineto(1010.26292725,94.35924123)
\curveto(1010.16293402,94.36924053)(1010.06793412,94.37924052)(1009.97792725,94.38924123)
\lineto(1009.67792725,94.41924123)
\curveto(1009.62793456,94.43924046)(1009.57793461,94.44424046)(1009.52792725,94.43424123)
\curveto(1009.46793472,94.43424047)(1009.41293477,94.44424046)(1009.36292725,94.46424123)
\curveto(1009.19293499,94.51424039)(1009.02793516,94.55424035)(1008.86792725,94.58424123)
\curveto(1008.69793549,94.61424029)(1008.53793565,94.66424024)(1008.38792725,94.73424123)
\curveto(1007.92793626,94.92423998)(1007.55293663,95.14423976)(1007.26292725,95.39424123)
\curveto(1006.97293721,95.65423925)(1006.72793746,96.01423889)(1006.52792725,96.47424123)
\curveto(1006.47793771,96.6042383)(1006.44293774,96.73423817)(1006.42292725,96.86424123)
\curveto(1006.40293778,97.0042379)(1006.37793781,97.14423776)(1006.34792725,97.28424123)
\curveto(1006.33793785,97.35423755)(1006.33293785,97.41923748)(1006.33292725,97.47924123)
\curveto(1006.33293785,97.53923736)(1006.32793786,97.6042373)(1006.31792725,97.67424123)
\curveto(1006.29793789,98.5042364)(1006.44793774,99.17423573)(1006.76792725,99.68424123)
\curveto(1007.07793711,100.19423471)(1007.51793667,100.57423433)(1008.08792725,100.82424123)
\curveto(1008.20793598,100.87423403)(1008.33293585,100.91923398)(1008.46292725,100.95924123)
\curveto(1008.59293559,100.9992339)(1008.72793546,101.04423386)(1008.86792725,101.09424123)
\curveto(1008.94793524,101.11423379)(1009.03293515,101.12923377)(1009.12292725,101.13924123)
\lineto(1009.36292725,101.19924123)
\curveto(1009.47293471,101.22923367)(1009.5829346,101.24423366)(1009.69292725,101.24424123)
\curveto(1009.80293438,101.25423365)(1009.91293427,101.26923363)(1010.02292725,101.28924123)
\curveto(1010.07293411,101.30923359)(1010.11793407,101.31423359)(1010.15792725,101.30424123)
\curveto(1010.19793399,101.3042336)(1010.23793395,101.30923359)(1010.27792725,101.31924123)
\curveto(1010.32793386,101.32923357)(1010.3829338,101.32923357)(1010.44292725,101.31924123)
\curveto(1010.49293369,101.31923358)(1010.54293364,101.32423358)(1010.59292725,101.33424123)
\lineto(1010.72792725,101.33424123)
\curveto(1010.7879334,101.35423355)(1010.85793333,101.35423355)(1010.93792725,101.33424123)
\curveto(1011.00793318,101.32423358)(1011.07293311,101.32923357)(1011.13292725,101.34924123)
\curveto(1011.16293302,101.35923354)(1011.20293298,101.36423354)(1011.25292725,101.36424123)
\lineto(1011.37292725,101.36424123)
\lineto(1011.83792725,101.36424123)
\moveto(1014.16292725,99.81924123)
\curveto(1013.84293034,99.91923498)(1013.47793071,99.97923492)(1013.06792725,99.99924123)
\curveto(1012.65793153,100.01923488)(1012.24793194,100.02923487)(1011.83792725,100.02924123)
\curveto(1011.40793278,100.02923487)(1010.9879332,100.01923488)(1010.57792725,99.99924123)
\curveto(1010.16793402,99.97923492)(1009.7829344,99.93423497)(1009.42292725,99.86424123)
\curveto(1009.06293512,99.79423511)(1008.74293544,99.68423522)(1008.46292725,99.53424123)
\curveto(1008.17293601,99.39423551)(1007.93793625,99.1992357)(1007.75792725,98.94924123)
\curveto(1007.64793654,98.78923611)(1007.56793662,98.60923629)(1007.51792725,98.40924123)
\curveto(1007.45793673,98.20923669)(1007.42793676,97.96423694)(1007.42792725,97.67424123)
\curveto(1007.44793674,97.65423725)(1007.45793673,97.61923728)(1007.45792725,97.56924123)
\curveto(1007.44793674,97.51923738)(1007.44793674,97.47923742)(1007.45792725,97.44924123)
\curveto(1007.47793671,97.36923753)(1007.49793669,97.29423761)(1007.51792725,97.22424123)
\curveto(1007.52793666,97.16423774)(1007.54793664,97.0992378)(1007.57792725,97.02924123)
\curveto(1007.69793649,96.75923814)(1007.86793632,96.53923836)(1008.08792725,96.36924123)
\curveto(1008.29793589,96.20923869)(1008.54293564,96.07423883)(1008.82292725,95.96424123)
\curveto(1008.93293525,95.91423899)(1009.05293513,95.87423903)(1009.18292725,95.84424123)
\curveto(1009.30293488,95.82423908)(1009.42793476,95.7992391)(1009.55792725,95.76924123)
\curveto(1009.60793458,95.74923915)(1009.66293452,95.73923916)(1009.72292725,95.73924123)
\curveto(1009.77293441,95.73923916)(1009.82293436,95.73423917)(1009.87292725,95.72424123)
\curveto(1009.96293422,95.71423919)(1010.05793413,95.7042392)(1010.15792725,95.69424123)
\curveto(1010.24793394,95.68423922)(1010.34293384,95.67423923)(1010.44292725,95.66424123)
\curveto(1010.52293366,95.66423924)(1010.60793358,95.65923924)(1010.69792725,95.64924123)
\lineto(1010.93792725,95.64924123)
\lineto(1011.11792725,95.64924123)
\curveto(1011.14793304,95.63923926)(1011.182933,95.63423927)(1011.22292725,95.63424123)
\lineto(1011.35792725,95.63424123)
\lineto(1011.80792725,95.63424123)
\curveto(1011.8879323,95.63423927)(1011.97293221,95.62923927)(1012.06292725,95.61924123)
\curveto(1012.14293204,95.61923928)(1012.21793197,95.62923927)(1012.28792725,95.64924123)
\lineto(1012.55792725,95.64924123)
\curveto(1012.57793161,95.64923925)(1012.60793158,95.64423926)(1012.64792725,95.63424123)
\curveto(1012.67793151,95.63423927)(1012.70293148,95.63923926)(1012.72292725,95.64924123)
\curveto(1012.82293136,95.65923924)(1012.92293126,95.66423924)(1013.02292725,95.66424123)
\curveto(1013.11293107,95.67423923)(1013.21293097,95.68423922)(1013.32292725,95.69424123)
\curveto(1013.44293074,95.72423918)(1013.56793062,95.73923916)(1013.69792725,95.73924123)
\curveto(1013.81793037,95.74923915)(1013.93293025,95.77423913)(1014.04292725,95.81424123)
\curveto(1014.34292984,95.89423901)(1014.60792958,95.97923892)(1014.83792725,96.06924123)
\curveto(1015.06792912,96.16923873)(1015.2829289,96.31423859)(1015.48292725,96.50424123)
\curveto(1015.6829285,96.71423819)(1015.83292835,96.97923792)(1015.93292725,97.29924123)
\curveto(1015.95292823,97.33923756)(1015.96292822,97.37423753)(1015.96292725,97.40424123)
\curveto(1015.95292823,97.44423746)(1015.95792823,97.48923741)(1015.97792725,97.53924123)
\curveto(1015.9879282,97.57923732)(1015.99792819,97.64923725)(1016.00792725,97.74924123)
\curveto(1016.01792817,97.85923704)(1016.01292817,97.94423696)(1015.99292725,98.00424123)
\curveto(1015.97292821,98.07423683)(1015.96292822,98.14423676)(1015.96292725,98.21424123)
\curveto(1015.95292823,98.28423662)(1015.93792825,98.34923655)(1015.91792725,98.40924123)
\curveto(1015.85792833,98.60923629)(1015.77292841,98.78923611)(1015.66292725,98.94924123)
\curveto(1015.64292854,98.97923592)(1015.62292856,99.0042359)(1015.60292725,99.02424123)
\lineto(1015.54292725,99.08424123)
\curveto(1015.52292866,99.12423578)(1015.4829287,99.17423573)(1015.42292725,99.23424123)
\curveto(1015.2829289,99.33423557)(1015.15292903,99.41923548)(1015.03292725,99.48924123)
\curveto(1014.91292927,99.55923534)(1014.76792942,99.62923527)(1014.59792725,99.69924123)
\curveto(1014.52792966,99.72923517)(1014.45792973,99.74923515)(1014.38792725,99.75924123)
\curveto(1014.31792987,99.77923512)(1014.24292994,99.7992351)(1014.16292725,99.81924123)
}
}
{
\newrgbcolor{curcolor}{0 0 0}
\pscustom[linestyle=none,fillstyle=solid,fillcolor=curcolor]
{
\newpath
\moveto(1006.31792725,106.77385061)
\curveto(1006.31793787,106.87384575)(1006.32793786,106.96884566)(1006.34792725,107.05885061)
\curveto(1006.35793783,107.14884548)(1006.3879378,107.21384541)(1006.43792725,107.25385061)
\curveto(1006.51793767,107.31384531)(1006.62293756,107.34384528)(1006.75292725,107.34385061)
\lineto(1007.14292725,107.34385061)
\lineto(1008.64292725,107.34385061)
\lineto(1015.03292725,107.34385061)
\lineto(1016.20292725,107.34385061)
\lineto(1016.51792725,107.34385061)
\curveto(1016.61792757,107.35384527)(1016.69792749,107.33884529)(1016.75792725,107.29885061)
\curveto(1016.83792735,107.24884538)(1016.8879273,107.17384545)(1016.90792725,107.07385061)
\curveto(1016.91792727,106.98384564)(1016.92292726,106.87384575)(1016.92292725,106.74385061)
\lineto(1016.92292725,106.51885061)
\curveto(1016.90292728,106.43884619)(1016.8879273,106.36884626)(1016.87792725,106.30885061)
\curveto(1016.85792733,106.24884638)(1016.81792737,106.19884643)(1016.75792725,106.15885061)
\curveto(1016.69792749,106.11884651)(1016.62292756,106.09884653)(1016.53292725,106.09885061)
\lineto(1016.23292725,106.09885061)
\lineto(1015.13792725,106.09885061)
\lineto(1009.79792725,106.09885061)
\curveto(1009.70793448,106.07884655)(1009.63293455,106.06384656)(1009.57292725,106.05385061)
\curveto(1009.50293468,106.05384657)(1009.44293474,106.0238466)(1009.39292725,105.96385061)
\curveto(1009.34293484,105.89384673)(1009.31793487,105.80384682)(1009.31792725,105.69385061)
\curveto(1009.30793488,105.59384703)(1009.30293488,105.48384714)(1009.30292725,105.36385061)
\lineto(1009.30292725,104.22385061)
\lineto(1009.30292725,103.72885061)
\curveto(1009.29293489,103.56884906)(1009.23293495,103.45884917)(1009.12292725,103.39885061)
\curveto(1009.09293509,103.37884925)(1009.06293512,103.36884926)(1009.03292725,103.36885061)
\curveto(1008.99293519,103.36884926)(1008.94793524,103.36384926)(1008.89792725,103.35385061)
\curveto(1008.77793541,103.33384929)(1008.66793552,103.33884929)(1008.56792725,103.36885061)
\curveto(1008.46793572,103.40884922)(1008.39793579,103.46384916)(1008.35792725,103.53385061)
\curveto(1008.30793588,103.61384901)(1008.2829359,103.73384889)(1008.28292725,103.89385061)
\curveto(1008.2829359,104.05384857)(1008.26793592,104.18884844)(1008.23792725,104.29885061)
\curveto(1008.22793596,104.34884828)(1008.22293596,104.40384822)(1008.22292725,104.46385061)
\curveto(1008.21293597,104.5238481)(1008.19793599,104.58384804)(1008.17792725,104.64385061)
\curveto(1008.12793606,104.79384783)(1008.07793611,104.93884769)(1008.02792725,105.07885061)
\curveto(1007.96793622,105.21884741)(1007.89793629,105.35384727)(1007.81792725,105.48385061)
\curveto(1007.72793646,105.623847)(1007.62293656,105.74384688)(1007.50292725,105.84385061)
\curveto(1007.3829368,105.94384668)(1007.25293693,106.03884659)(1007.11292725,106.12885061)
\curveto(1007.01293717,106.18884644)(1006.90293728,106.23384639)(1006.78292725,106.26385061)
\curveto(1006.66293752,106.30384632)(1006.55793763,106.35384627)(1006.46792725,106.41385061)
\curveto(1006.40793778,106.46384616)(1006.36793782,106.53384609)(1006.34792725,106.62385061)
\curveto(1006.33793785,106.64384598)(1006.33293785,106.66884596)(1006.33292725,106.69885061)
\curveto(1006.33293785,106.7288459)(1006.32793786,106.75384587)(1006.31792725,106.77385061)
}
}
{
\newrgbcolor{curcolor}{0 0 0}
\pscustom[linestyle=none,fillstyle=solid,fillcolor=curcolor]
{
\newpath
\moveto(1006.31792725,115.12345998)
\curveto(1006.31793787,115.22345513)(1006.32793786,115.31845503)(1006.34792725,115.40845998)
\curveto(1006.35793783,115.49845485)(1006.3879378,115.56345479)(1006.43792725,115.60345998)
\curveto(1006.51793767,115.66345469)(1006.62293756,115.69345466)(1006.75292725,115.69345998)
\lineto(1007.14292725,115.69345998)
\lineto(1008.64292725,115.69345998)
\lineto(1015.03292725,115.69345998)
\lineto(1016.20292725,115.69345998)
\lineto(1016.51792725,115.69345998)
\curveto(1016.61792757,115.70345465)(1016.69792749,115.68845466)(1016.75792725,115.64845998)
\curveto(1016.83792735,115.59845475)(1016.8879273,115.52345483)(1016.90792725,115.42345998)
\curveto(1016.91792727,115.33345502)(1016.92292726,115.22345513)(1016.92292725,115.09345998)
\lineto(1016.92292725,114.86845998)
\curveto(1016.90292728,114.78845556)(1016.8879273,114.71845563)(1016.87792725,114.65845998)
\curveto(1016.85792733,114.59845575)(1016.81792737,114.5484558)(1016.75792725,114.50845998)
\curveto(1016.69792749,114.46845588)(1016.62292756,114.4484559)(1016.53292725,114.44845998)
\lineto(1016.23292725,114.44845998)
\lineto(1015.13792725,114.44845998)
\lineto(1009.79792725,114.44845998)
\curveto(1009.70793448,114.42845592)(1009.63293455,114.41345594)(1009.57292725,114.40345998)
\curveto(1009.50293468,114.40345595)(1009.44293474,114.37345598)(1009.39292725,114.31345998)
\curveto(1009.34293484,114.24345611)(1009.31793487,114.1534562)(1009.31792725,114.04345998)
\curveto(1009.30793488,113.94345641)(1009.30293488,113.83345652)(1009.30292725,113.71345998)
\lineto(1009.30292725,112.57345998)
\lineto(1009.30292725,112.07845998)
\curveto(1009.29293489,111.91845843)(1009.23293495,111.80845854)(1009.12292725,111.74845998)
\curveto(1009.09293509,111.72845862)(1009.06293512,111.71845863)(1009.03292725,111.71845998)
\curveto(1008.99293519,111.71845863)(1008.94793524,111.71345864)(1008.89792725,111.70345998)
\curveto(1008.77793541,111.68345867)(1008.66793552,111.68845866)(1008.56792725,111.71845998)
\curveto(1008.46793572,111.75845859)(1008.39793579,111.81345854)(1008.35792725,111.88345998)
\curveto(1008.30793588,111.96345839)(1008.2829359,112.08345827)(1008.28292725,112.24345998)
\curveto(1008.2829359,112.40345795)(1008.26793592,112.53845781)(1008.23792725,112.64845998)
\curveto(1008.22793596,112.69845765)(1008.22293596,112.7534576)(1008.22292725,112.81345998)
\curveto(1008.21293597,112.87345748)(1008.19793599,112.93345742)(1008.17792725,112.99345998)
\curveto(1008.12793606,113.14345721)(1008.07793611,113.28845706)(1008.02792725,113.42845998)
\curveto(1007.96793622,113.56845678)(1007.89793629,113.70345665)(1007.81792725,113.83345998)
\curveto(1007.72793646,113.97345638)(1007.62293656,114.09345626)(1007.50292725,114.19345998)
\curveto(1007.3829368,114.29345606)(1007.25293693,114.38845596)(1007.11292725,114.47845998)
\curveto(1007.01293717,114.53845581)(1006.90293728,114.58345577)(1006.78292725,114.61345998)
\curveto(1006.66293752,114.6534557)(1006.55793763,114.70345565)(1006.46792725,114.76345998)
\curveto(1006.40793778,114.81345554)(1006.36793782,114.88345547)(1006.34792725,114.97345998)
\curveto(1006.33793785,114.99345536)(1006.33293785,115.01845533)(1006.33292725,115.04845998)
\curveto(1006.33293785,115.07845527)(1006.32793786,115.10345525)(1006.31792725,115.12345998)
}
}
{
\newrgbcolor{curcolor}{0 0 0}
\pscustom[linestyle=none,fillstyle=solid,fillcolor=curcolor]
{
\newpath
\moveto(222.02405518,31.67142873)
\lineto(222.02405518,32.58642873)
\curveto(222.02406587,32.68642608)(222.02406587,32.78142599)(222.02405518,32.87142873)
\curveto(222.02406587,32.96142581)(222.04406585,33.03642573)(222.08405518,33.09642873)
\curveto(222.14406575,33.18642558)(222.22406567,33.24642552)(222.32405518,33.27642873)
\curveto(222.42406547,33.31642545)(222.52906537,33.36142541)(222.63905518,33.41142873)
\curveto(222.82906507,33.49142528)(223.01906488,33.56142521)(223.20905518,33.62142873)
\curveto(223.3990645,33.69142508)(223.58906431,33.766425)(223.77905518,33.84642873)
\curveto(223.95906394,33.91642485)(224.14406375,33.98142479)(224.33405518,34.04142873)
\curveto(224.51406338,34.10142467)(224.6940632,34.1714246)(224.87405518,34.25142873)
\curveto(225.01406288,34.31142446)(225.15906274,34.3664244)(225.30905518,34.41642873)
\curveto(225.45906244,34.4664243)(225.60406229,34.52142425)(225.74405518,34.58142873)
\curveto(226.1940617,34.76142401)(226.64906125,34.93142384)(227.10905518,35.09142873)
\curveto(227.55906034,35.25142352)(228.00905989,35.42142335)(228.45905518,35.60142873)
\curveto(228.50905939,35.62142315)(228.55905934,35.63642313)(228.60905518,35.64642873)
\lineto(228.75905518,35.70642873)
\curveto(228.97905892,35.79642297)(229.20405869,35.88142289)(229.43405518,35.96142873)
\curveto(229.65405824,36.04142273)(229.87405802,36.12642264)(230.09405518,36.21642873)
\curveto(230.18405771,36.25642251)(230.2940576,36.29642247)(230.42405518,36.33642873)
\curveto(230.54405735,36.37642239)(230.61405728,36.44142233)(230.63405518,36.53142873)
\curveto(230.64405725,36.5714222)(230.64405725,36.60142217)(230.63405518,36.62142873)
\lineto(230.57405518,36.68142873)
\curveto(230.52405737,36.73142204)(230.46905743,36.766422)(230.40905518,36.78642873)
\curveto(230.34905755,36.81642195)(230.28405761,36.84642192)(230.21405518,36.87642873)
\lineto(229.58405518,37.11642873)
\curveto(229.36405853,37.19642157)(229.14905875,37.27642149)(228.93905518,37.35642873)
\lineto(228.78905518,37.41642873)
\lineto(228.60905518,37.47642873)
\curveto(228.41905948,37.55642121)(228.22905967,37.62642114)(228.03905518,37.68642873)
\curveto(227.83906006,37.75642101)(227.63906026,37.83142094)(227.43905518,37.91142873)
\curveto(226.85906104,38.15142062)(226.27406162,38.3714204)(225.68405518,38.57142873)
\curveto(225.0940628,38.78141999)(224.50906339,39.00641976)(223.92905518,39.24642873)
\curveto(223.72906417,39.32641944)(223.52406437,39.40141937)(223.31405518,39.47142873)
\curveto(223.10406479,39.55141922)(222.899065,39.63141914)(222.69905518,39.71142873)
\curveto(222.61906528,39.75141902)(222.51906538,39.78641898)(222.39905518,39.81642873)
\curveto(222.27906562,39.85641891)(222.1940657,39.91141886)(222.14405518,39.98142873)
\curveto(222.10406579,40.04141873)(222.07406582,40.11641865)(222.05405518,40.20642873)
\curveto(222.03406586,40.30641846)(222.02406587,40.41641835)(222.02405518,40.53642873)
\curveto(222.01406588,40.65641811)(222.01406588,40.77641799)(222.02405518,40.89642873)
\curveto(222.02406587,41.01641775)(222.02406587,41.12641764)(222.02405518,41.22642873)
\curveto(222.02406587,41.31641745)(222.02406587,41.40641736)(222.02405518,41.49642873)
\curveto(222.02406587,41.59641717)(222.04406585,41.6714171)(222.08405518,41.72142873)
\curveto(222.13406576,41.81141696)(222.22406567,41.86141691)(222.35405518,41.87142873)
\curveto(222.48406541,41.88141689)(222.62406527,41.88641688)(222.77405518,41.88642873)
\lineto(224.42405518,41.88642873)
\lineto(230.69405518,41.88642873)
\lineto(231.95405518,41.88642873)
\curveto(232.06405583,41.88641688)(232.17405572,41.88641688)(232.28405518,41.88642873)
\curveto(232.3940555,41.89641687)(232.47905542,41.87641689)(232.53905518,41.82642873)
\curveto(232.5990553,41.79641697)(232.63905526,41.75141702)(232.65905518,41.69142873)
\curveto(232.66905523,41.63141714)(232.68405521,41.56141721)(232.70405518,41.48142873)
\lineto(232.70405518,41.24142873)
\lineto(232.70405518,40.88142873)
\curveto(232.6940552,40.771418)(232.64905525,40.69141808)(232.56905518,40.64142873)
\curveto(232.53905536,40.62141815)(232.50905539,40.60641816)(232.47905518,40.59642873)
\curveto(232.43905546,40.59641817)(232.3940555,40.58641818)(232.34405518,40.56642873)
\lineto(232.17905518,40.56642873)
\curveto(232.11905578,40.55641821)(232.04905585,40.55141822)(231.96905518,40.55142873)
\curveto(231.88905601,40.56141821)(231.81405608,40.5664182)(231.74405518,40.56642873)
\lineto(230.90405518,40.56642873)
\lineto(226.47905518,40.56642873)
\curveto(226.22906167,40.5664182)(225.97906192,40.5664182)(225.72905518,40.56642873)
\curveto(225.46906243,40.5664182)(225.21906268,40.56141821)(224.97905518,40.55142873)
\curveto(224.87906302,40.55141822)(224.76906313,40.54641822)(224.64905518,40.53642873)
\curveto(224.52906337,40.52641824)(224.46906343,40.4714183)(224.46905518,40.37142873)
\lineto(224.48405518,40.37142873)
\curveto(224.50406339,40.30141847)(224.56906333,40.24141853)(224.67905518,40.19142873)
\curveto(224.78906311,40.15141862)(224.88406301,40.11641865)(224.96405518,40.08642873)
\curveto(225.13406276,40.01641875)(225.30906259,39.95141882)(225.48905518,39.89142873)
\curveto(225.65906224,39.83141894)(225.82906207,39.76141901)(225.99905518,39.68142873)
\curveto(226.04906185,39.66141911)(226.0940618,39.64641912)(226.13405518,39.63642873)
\curveto(226.17406172,39.62641914)(226.21906168,39.61141916)(226.26905518,39.59142873)
\curveto(226.44906145,39.51141926)(226.63406126,39.44141933)(226.82405518,39.38142873)
\curveto(227.00406089,39.33141944)(227.18406071,39.2664195)(227.36405518,39.18642873)
\curveto(227.51406038,39.11641965)(227.66906023,39.05641971)(227.82905518,39.00642873)
\curveto(227.97905992,38.95641981)(228.12905977,38.90141987)(228.27905518,38.84142873)
\curveto(228.74905915,38.64142013)(229.22405867,38.46142031)(229.70405518,38.30142873)
\curveto(230.17405772,38.14142063)(230.63905726,37.9664208)(231.09905518,37.77642873)
\curveto(231.27905662,37.69642107)(231.45905644,37.62642114)(231.63905518,37.56642873)
\curveto(231.81905608,37.50642126)(231.9990559,37.44142133)(232.17905518,37.37142873)
\curveto(232.28905561,37.32142145)(232.3940555,37.2714215)(232.49405518,37.22142873)
\curveto(232.58405531,37.18142159)(232.64905525,37.09642167)(232.68905518,36.96642873)
\curveto(232.6990552,36.94642182)(232.70405519,36.92142185)(232.70405518,36.89142873)
\curveto(232.6940552,36.8714219)(232.6940552,36.84642192)(232.70405518,36.81642873)
\curveto(232.71405518,36.78642198)(232.71905518,36.75142202)(232.71905518,36.71142873)
\curveto(232.70905519,36.6714221)(232.70405519,36.63142214)(232.70405518,36.59142873)
\lineto(232.70405518,36.29142873)
\curveto(232.70405519,36.19142258)(232.67905522,36.11142266)(232.62905518,36.05142873)
\curveto(232.57905532,35.9714228)(232.50905539,35.91142286)(232.41905518,35.87142873)
\curveto(232.31905558,35.84142293)(232.21905568,35.80142297)(232.11905518,35.75142873)
\curveto(231.91905598,35.6714231)(231.71405618,35.59142318)(231.50405518,35.51142873)
\curveto(231.28405661,35.44142333)(231.07405682,35.3664234)(230.87405518,35.28642873)
\curveto(230.6940572,35.20642356)(230.51405738,35.13642363)(230.33405518,35.07642873)
\curveto(230.14405775,35.02642374)(229.95905794,34.96142381)(229.77905518,34.88142873)
\curveto(229.21905868,34.65142412)(228.65405924,34.43642433)(228.08405518,34.23642873)
\curveto(227.51406038,34.03642473)(226.94906095,33.82142495)(226.38905518,33.59142873)
\lineto(225.75905518,33.35142873)
\curveto(225.53906236,33.28142549)(225.32906257,33.20642556)(225.12905518,33.12642873)
\curveto(225.01906288,33.07642569)(224.91406298,33.03142574)(224.81405518,32.99142873)
\curveto(224.70406319,32.96142581)(224.60906329,32.91142586)(224.52905518,32.84142873)
\curveto(224.50906339,32.83142594)(224.4990634,32.82142595)(224.49905518,32.81142873)
\lineto(224.46905518,32.78142873)
\lineto(224.46905518,32.70642873)
\lineto(224.49905518,32.67642873)
\curveto(224.4990634,32.6664261)(224.50406339,32.65642611)(224.51405518,32.64642873)
\curveto(224.56406333,32.62642614)(224.61906328,32.61642615)(224.67905518,32.61642873)
\curveto(224.73906316,32.61642615)(224.7990631,32.60642616)(224.85905518,32.58642873)
\lineto(225.02405518,32.58642873)
\curveto(225.08406281,32.5664262)(225.14906275,32.56142621)(225.21905518,32.57142873)
\curveto(225.28906261,32.58142619)(225.35906254,32.58642618)(225.42905518,32.58642873)
\lineto(226.23905518,32.58642873)
\lineto(230.79905518,32.58642873)
\lineto(231.98405518,32.58642873)
\curveto(232.0940558,32.58642618)(232.20405569,32.58142619)(232.31405518,32.57142873)
\curveto(232.42405547,32.5714262)(232.50905539,32.54642622)(232.56905518,32.49642873)
\curveto(232.64905525,32.44642632)(232.6940552,32.35642641)(232.70405518,32.22642873)
\lineto(232.70405518,31.83642873)
\lineto(232.70405518,31.64142873)
\curveto(232.70405519,31.59142718)(232.6940552,31.54142723)(232.67405518,31.49142873)
\curveto(232.63405526,31.36142741)(232.54905535,31.28642748)(232.41905518,31.26642873)
\curveto(232.28905561,31.25642751)(232.13905576,31.25142752)(231.96905518,31.25142873)
\lineto(230.22905518,31.25142873)
\lineto(224.22905518,31.25142873)
\lineto(222.81905518,31.25142873)
\curveto(222.70906519,31.25142752)(222.5940653,31.24642752)(222.47405518,31.23642873)
\curveto(222.35406554,31.23642753)(222.25906564,31.26142751)(222.18905518,31.31142873)
\curveto(222.12906577,31.35142742)(222.07906582,31.42642734)(222.03905518,31.53642873)
\curveto(222.02906587,31.55642721)(222.02906587,31.57642719)(222.03905518,31.59642873)
\curveto(222.03906586,31.62642714)(222.03406586,31.65142712)(222.02405518,31.67142873)
}
}
{
\newrgbcolor{curcolor}{0 0 0}
\pscustom[linestyle=none,fillstyle=solid,fillcolor=curcolor]
{
\newpath
\moveto(232.14905518,50.87353811)
\curveto(232.30905559,50.90353028)(232.44405545,50.88853029)(232.55405518,50.82853811)
\curveto(232.65405524,50.76853041)(232.72905517,50.68853049)(232.77905518,50.58853811)
\curveto(232.7990551,50.53853064)(232.80905509,50.4835307)(232.80905518,50.42353811)
\curveto(232.80905509,50.37353081)(232.81905508,50.31853086)(232.83905518,50.25853811)
\curveto(232.88905501,50.03853114)(232.87405502,49.81853136)(232.79405518,49.59853811)
\curveto(232.72405517,49.38853179)(232.63405526,49.24353194)(232.52405518,49.16353811)
\curveto(232.45405544,49.11353207)(232.37405552,49.06853211)(232.28405518,49.02853811)
\curveto(232.18405571,48.98853219)(232.10405579,48.93853224)(232.04405518,48.87853811)
\curveto(232.02405587,48.85853232)(232.00405589,48.83353235)(231.98405518,48.80353811)
\curveto(231.96405593,48.7835324)(231.95905594,48.75353243)(231.96905518,48.71353811)
\curveto(231.9990559,48.60353258)(232.05405584,48.49853268)(232.13405518,48.39853811)
\curveto(232.21405568,48.30853287)(232.28405561,48.21853296)(232.34405518,48.12853811)
\curveto(232.42405547,47.99853318)(232.4990554,47.85853332)(232.56905518,47.70853811)
\curveto(232.62905527,47.55853362)(232.68405521,47.39853378)(232.73405518,47.22853811)
\curveto(232.76405513,47.12853405)(232.78405511,47.01853416)(232.79405518,46.89853811)
\curveto(232.80405509,46.78853439)(232.81905508,46.6785345)(232.83905518,46.56853811)
\curveto(232.84905505,46.51853466)(232.85405504,46.47353471)(232.85405518,46.43353811)
\lineto(232.85405518,46.32853811)
\curveto(232.87405502,46.21853496)(232.87405502,46.11353507)(232.85405518,46.01353811)
\lineto(232.85405518,45.87853811)
\curveto(232.84405505,45.82853535)(232.83905506,45.7785354)(232.83905518,45.72853811)
\curveto(232.83905506,45.6785355)(232.82905507,45.63353555)(232.80905518,45.59353811)
\curveto(232.7990551,45.55353563)(232.7940551,45.51853566)(232.79405518,45.48853811)
\curveto(232.80405509,45.46853571)(232.80405509,45.44353574)(232.79405518,45.41353811)
\lineto(232.73405518,45.17353811)
\curveto(232.72405517,45.09353609)(232.70405519,45.01853616)(232.67405518,44.94853811)
\curveto(232.54405535,44.64853653)(232.3990555,44.40353678)(232.23905518,44.21353811)
\curveto(232.06905583,44.03353715)(231.83405606,43.8835373)(231.53405518,43.76353811)
\curveto(231.31405658,43.67353751)(231.04905685,43.62853755)(230.73905518,43.62853811)
\lineto(230.42405518,43.62853811)
\curveto(230.37405752,43.63853754)(230.32405757,43.64353754)(230.27405518,43.64353811)
\lineto(230.09405518,43.67353811)
\lineto(229.76405518,43.79353811)
\curveto(229.65405824,43.83353735)(229.55405834,43.8835373)(229.46405518,43.94353811)
\curveto(229.17405872,44.12353706)(228.95905894,44.36853681)(228.81905518,44.67853811)
\curveto(228.67905922,44.98853619)(228.55405934,45.32853585)(228.44405518,45.69853811)
\curveto(228.40405949,45.83853534)(228.37405952,45.9835352)(228.35405518,46.13353811)
\curveto(228.33405956,46.2835349)(228.30905959,46.43353475)(228.27905518,46.58353811)
\curveto(228.25905964,46.65353453)(228.24905965,46.71853446)(228.24905518,46.77853811)
\curveto(228.24905965,46.84853433)(228.23905966,46.92353426)(228.21905518,47.00353811)
\curveto(228.1990597,47.07353411)(228.18905971,47.14353404)(228.18905518,47.21353811)
\curveto(228.17905972,47.2835339)(228.16405973,47.35853382)(228.14405518,47.43853811)
\curveto(228.08405981,47.68853349)(228.03405986,47.92353326)(227.99405518,48.14353811)
\curveto(227.94405995,48.36353282)(227.82906007,48.53853264)(227.64905518,48.66853811)
\curveto(227.56906033,48.72853245)(227.46906043,48.7785324)(227.34905518,48.81853811)
\curveto(227.21906068,48.85853232)(227.07906082,48.85853232)(226.92905518,48.81853811)
\curveto(226.68906121,48.75853242)(226.4990614,48.66853251)(226.35905518,48.54853811)
\curveto(226.21906168,48.43853274)(226.10906179,48.2785329)(226.02905518,48.06853811)
\curveto(225.97906192,47.94853323)(225.94406195,47.80353338)(225.92405518,47.63353811)
\curveto(225.90406199,47.47353371)(225.894062,47.30353388)(225.89405518,47.12353811)
\curveto(225.894062,46.94353424)(225.90406199,46.76853441)(225.92405518,46.59853811)
\curveto(225.94406195,46.42853475)(225.97406192,46.2835349)(226.01405518,46.16353811)
\curveto(226.07406182,45.99353519)(226.15906174,45.82853535)(226.26905518,45.66853811)
\curveto(226.32906157,45.58853559)(226.40906149,45.51353567)(226.50905518,45.44353811)
\curveto(226.5990613,45.3835358)(226.6990612,45.32853585)(226.80905518,45.27853811)
\curveto(226.88906101,45.24853593)(226.97406092,45.21853596)(227.06405518,45.18853811)
\curveto(227.15406074,45.16853601)(227.22406067,45.12353606)(227.27405518,45.05353811)
\curveto(227.30406059,45.01353617)(227.32906057,44.94353624)(227.34905518,44.84353811)
\curveto(227.35906054,44.75353643)(227.36406053,44.65853652)(227.36405518,44.55853811)
\curveto(227.36406053,44.45853672)(227.35906054,44.35853682)(227.34905518,44.25853811)
\curveto(227.32906057,44.16853701)(227.30406059,44.10353708)(227.27405518,44.06353811)
\curveto(227.24406065,44.02353716)(227.1940607,43.99353719)(227.12405518,43.97353811)
\curveto(227.05406084,43.95353723)(226.97906092,43.95353723)(226.89905518,43.97353811)
\curveto(226.76906113,44.00353718)(226.64906125,44.03353715)(226.53905518,44.06353811)
\curveto(226.41906148,44.10353708)(226.30406159,44.14853703)(226.19405518,44.19853811)
\curveto(225.84406205,44.38853679)(225.57406232,44.62853655)(225.38405518,44.91853811)
\curveto(225.18406271,45.20853597)(225.02406287,45.56853561)(224.90405518,45.99853811)
\curveto(224.88406301,46.09853508)(224.86906303,46.19853498)(224.85905518,46.29853811)
\curveto(224.84906305,46.40853477)(224.83406306,46.51853466)(224.81405518,46.62853811)
\curveto(224.80406309,46.66853451)(224.80406309,46.73353445)(224.81405518,46.82353811)
\curveto(224.81406308,46.91353427)(224.80406309,46.96853421)(224.78405518,46.98853811)
\curveto(224.77406312,47.68853349)(224.85406304,48.29853288)(225.02405518,48.81853811)
\curveto(225.1940627,49.33853184)(225.51906238,49.70353148)(225.99905518,49.91353811)
\curveto(226.1990617,50.00353118)(226.43406146,50.05353113)(226.70405518,50.06353811)
\curveto(226.96406093,50.0835311)(227.23906066,50.09353109)(227.52905518,50.09353811)
\lineto(230.84405518,50.09353811)
\curveto(230.98405691,50.09353109)(231.11905678,50.09853108)(231.24905518,50.10853811)
\curveto(231.37905652,50.11853106)(231.48405641,50.14853103)(231.56405518,50.19853811)
\curveto(231.63405626,50.24853093)(231.68405621,50.31353087)(231.71405518,50.39353811)
\curveto(231.75405614,50.4835307)(231.78405611,50.56853061)(231.80405518,50.64853811)
\curveto(231.81405608,50.72853045)(231.85905604,50.78853039)(231.93905518,50.82853811)
\curveto(231.96905593,50.84853033)(231.9990559,50.85853032)(232.02905518,50.85853811)
\curveto(232.05905584,50.85853032)(232.0990558,50.86353032)(232.14905518,50.87353811)
\moveto(230.48405518,48.72853811)
\curveto(230.34405755,48.78853239)(230.18405771,48.81853236)(230.00405518,48.81853811)
\curveto(229.81405808,48.82853235)(229.61905828,48.83353235)(229.41905518,48.83353811)
\curveto(229.30905859,48.83353235)(229.20905869,48.82853235)(229.11905518,48.81853811)
\curveto(229.02905887,48.80853237)(228.95905894,48.76853241)(228.90905518,48.69853811)
\curveto(228.88905901,48.66853251)(228.87905902,48.59853258)(228.87905518,48.48853811)
\curveto(228.899059,48.46853271)(228.90905899,48.43353275)(228.90905518,48.38353811)
\curveto(228.90905899,48.33353285)(228.91905898,48.28853289)(228.93905518,48.24853811)
\curveto(228.95905894,48.16853301)(228.97905892,48.0785331)(228.99905518,47.97853811)
\lineto(229.05905518,47.67853811)
\curveto(229.05905884,47.64853353)(229.06405883,47.61353357)(229.07405518,47.57353811)
\lineto(229.07405518,47.46853811)
\curveto(229.11405878,47.31853386)(229.13905876,47.15353403)(229.14905518,46.97353811)
\curveto(229.14905875,46.80353438)(229.16905873,46.64353454)(229.20905518,46.49353811)
\curveto(229.22905867,46.41353477)(229.24905865,46.33853484)(229.26905518,46.26853811)
\curveto(229.27905862,46.20853497)(229.2940586,46.13853504)(229.31405518,46.05853811)
\curveto(229.36405853,45.89853528)(229.42905847,45.74853543)(229.50905518,45.60853811)
\curveto(229.57905832,45.46853571)(229.66905823,45.34853583)(229.77905518,45.24853811)
\curveto(229.88905801,45.14853603)(230.02405787,45.07353611)(230.18405518,45.02353811)
\curveto(230.33405756,44.97353621)(230.51905738,44.95353623)(230.73905518,44.96353811)
\curveto(230.83905706,44.96353622)(230.93405696,44.9785362)(231.02405518,45.00853811)
\curveto(231.10405679,45.04853613)(231.17905672,45.09353609)(231.24905518,45.14353811)
\curveto(231.35905654,45.22353596)(231.45405644,45.32853585)(231.53405518,45.45853811)
\curveto(231.60405629,45.58853559)(231.66405623,45.72853545)(231.71405518,45.87853811)
\curveto(231.72405617,45.92853525)(231.72905617,45.9785352)(231.72905518,46.02853811)
\curveto(231.72905617,46.0785351)(231.73405616,46.12853505)(231.74405518,46.17853811)
\curveto(231.76405613,46.24853493)(231.77905612,46.33353485)(231.78905518,46.43353811)
\curveto(231.78905611,46.54353464)(231.77905612,46.63353455)(231.75905518,46.70353811)
\curveto(231.73905616,46.76353442)(231.73405616,46.82353436)(231.74405518,46.88353811)
\curveto(231.74405615,46.94353424)(231.73405616,47.00353418)(231.71405518,47.06353811)
\curveto(231.6940562,47.14353404)(231.67905622,47.21853396)(231.66905518,47.28853811)
\curveto(231.65905624,47.36853381)(231.63905626,47.44353374)(231.60905518,47.51353811)
\curveto(231.48905641,47.80353338)(231.34405655,48.04853313)(231.17405518,48.24853811)
\curveto(231.00405689,48.45853272)(230.77405712,48.61853256)(230.48405518,48.72853811)
}
}
{
\newrgbcolor{curcolor}{0 0 0}
\pscustom[linestyle=none,fillstyle=solid,fillcolor=curcolor]
{
\newpath
\moveto(224.79905518,55.69017873)
\curveto(224.7990631,55.92017394)(224.85906304,56.05017381)(224.97905518,56.08017873)
\curveto(225.08906281,56.11017375)(225.25406264,56.12517374)(225.47405518,56.12517873)
\lineto(225.75905518,56.12517873)
\curveto(225.84906205,56.12517374)(225.92406197,56.10017376)(225.98405518,56.05017873)
\curveto(226.06406183,55.99017387)(226.10906179,55.90517396)(226.11905518,55.79517873)
\curveto(226.11906178,55.68517418)(226.13406176,55.57517429)(226.16405518,55.46517873)
\curveto(226.1940617,55.32517454)(226.22406167,55.19017467)(226.25405518,55.06017873)
\curveto(226.28406161,54.94017492)(226.32406157,54.82517504)(226.37405518,54.71517873)
\curveto(226.50406139,54.42517544)(226.68406121,54.19017567)(226.91405518,54.01017873)
\curveto(227.13406076,53.83017603)(227.38906051,53.67517619)(227.67905518,53.54517873)
\curveto(227.78906011,53.50517636)(227.90405999,53.47517639)(228.02405518,53.45517873)
\curveto(228.13405976,53.43517643)(228.24905965,53.41017645)(228.36905518,53.38017873)
\curveto(228.41905948,53.37017649)(228.46905943,53.3651765)(228.51905518,53.36517873)
\curveto(228.56905933,53.37517649)(228.61905928,53.37517649)(228.66905518,53.36517873)
\curveto(228.78905911,53.33517653)(228.92905897,53.32017654)(229.08905518,53.32017873)
\curveto(229.23905866,53.33017653)(229.38405851,53.33517653)(229.52405518,53.33517873)
\lineto(231.36905518,53.33517873)
\lineto(231.71405518,53.33517873)
\curveto(231.83405606,53.33517653)(231.94905595,53.33017653)(232.05905518,53.32017873)
\curveto(232.16905573,53.31017655)(232.26405563,53.30517656)(232.34405518,53.30517873)
\curveto(232.42405547,53.31517655)(232.4940554,53.29517657)(232.55405518,53.24517873)
\curveto(232.62405527,53.19517667)(232.66405523,53.11517675)(232.67405518,53.00517873)
\curveto(232.68405521,52.90517696)(232.68905521,52.79517707)(232.68905518,52.67517873)
\lineto(232.68905518,52.40517873)
\curveto(232.66905523,52.35517751)(232.65405524,52.30517756)(232.64405518,52.25517873)
\curveto(232.62405527,52.21517765)(232.5990553,52.18517768)(232.56905518,52.16517873)
\curveto(232.4990554,52.11517775)(232.41405548,52.08517778)(232.31405518,52.07517873)
\lineto(231.98405518,52.07517873)
\lineto(230.82905518,52.07517873)
\lineto(226.67405518,52.07517873)
\lineto(225.63905518,52.07517873)
\lineto(225.33905518,52.07517873)
\curveto(225.23906266,52.08517778)(225.15406274,52.11517775)(225.08405518,52.16517873)
\curveto(225.04406285,52.19517767)(225.01406288,52.24517762)(224.99405518,52.31517873)
\curveto(224.97406292,52.39517747)(224.96406293,52.48017738)(224.96405518,52.57017873)
\curveto(224.95406294,52.6601772)(224.95406294,52.75017711)(224.96405518,52.84017873)
\curveto(224.97406292,52.93017693)(224.98906291,53.00017686)(225.00905518,53.05017873)
\curveto(225.03906286,53.13017673)(225.0990628,53.18017668)(225.18905518,53.20017873)
\curveto(225.26906263,53.23017663)(225.35906254,53.24517662)(225.45905518,53.24517873)
\lineto(225.75905518,53.24517873)
\curveto(225.85906204,53.24517662)(225.94906195,53.2651766)(226.02905518,53.30517873)
\curveto(226.04906185,53.31517655)(226.06406183,53.32517654)(226.07405518,53.33517873)
\lineto(226.11905518,53.38017873)
\curveto(226.11906178,53.49017637)(226.07406182,53.58017628)(225.98405518,53.65017873)
\curveto(225.88406201,53.72017614)(225.80406209,53.78017608)(225.74405518,53.83017873)
\lineto(225.65405518,53.92017873)
\curveto(225.54406235,54.01017585)(225.42906247,54.13517573)(225.30905518,54.29517873)
\curveto(225.18906271,54.45517541)(225.0990628,54.60517526)(225.03905518,54.74517873)
\curveto(224.98906291,54.83517503)(224.95406294,54.93017493)(224.93405518,55.03017873)
\curveto(224.90406299,55.13017473)(224.87406302,55.23517463)(224.84405518,55.34517873)
\curveto(224.83406306,55.40517446)(224.82906307,55.4651744)(224.82905518,55.52517873)
\curveto(224.81906308,55.58517428)(224.80906309,55.64017422)(224.79905518,55.69017873)
}
}
{
\newrgbcolor{curcolor}{0 0 0}
\pscustom[linestyle=none,fillstyle=solid,fillcolor=curcolor]
{
}
}
{
\newrgbcolor{curcolor}{0 0 0}
\pscustom[linestyle=none,fillstyle=solid,fillcolor=curcolor]
{
\newpath
\moveto(222.09905518,64.24510061)
\curveto(222.08906581,64.93509597)(222.20906569,65.53509537)(222.45905518,66.04510061)
\curveto(222.70906519,66.56509434)(223.04406485,66.96009395)(223.46405518,67.23010061)
\curveto(223.54406435,67.28009363)(223.63406426,67.32509358)(223.73405518,67.36510061)
\curveto(223.82406407,67.4050935)(223.91906398,67.45009346)(224.01905518,67.50010061)
\curveto(224.11906378,67.54009337)(224.21906368,67.57009334)(224.31905518,67.59010061)
\curveto(224.41906348,67.6100933)(224.52406337,67.63009328)(224.63405518,67.65010061)
\curveto(224.68406321,67.67009324)(224.72906317,67.67509323)(224.76905518,67.66510061)
\curveto(224.80906309,67.65509325)(224.85406304,67.66009325)(224.90405518,67.68010061)
\curveto(224.95406294,67.69009322)(225.03906286,67.69509321)(225.15905518,67.69510061)
\curveto(225.26906263,67.69509321)(225.35406254,67.69009322)(225.41405518,67.68010061)
\curveto(225.47406242,67.66009325)(225.53406236,67.65009326)(225.59405518,67.65010061)
\curveto(225.65406224,67.66009325)(225.71406218,67.65509325)(225.77405518,67.63510061)
\curveto(225.91406198,67.59509331)(226.04906185,67.56009335)(226.17905518,67.53010061)
\curveto(226.30906159,67.50009341)(226.43406146,67.46009345)(226.55405518,67.41010061)
\curveto(226.6940612,67.35009356)(226.81906108,67.28009363)(226.92905518,67.20010061)
\curveto(227.03906086,67.13009378)(227.14906075,67.05509385)(227.25905518,66.97510061)
\lineto(227.31905518,66.91510061)
\curveto(227.33906056,66.905094)(227.35906054,66.89009402)(227.37905518,66.87010061)
\curveto(227.53906036,66.75009416)(227.68406021,66.61509429)(227.81405518,66.46510061)
\curveto(227.94405995,66.31509459)(228.06905983,66.15509475)(228.18905518,65.98510061)
\curveto(228.40905949,65.67509523)(228.61405928,65.38009553)(228.80405518,65.10010061)
\curveto(228.94405895,64.87009604)(229.07905882,64.64009627)(229.20905518,64.41010061)
\curveto(229.33905856,64.19009672)(229.47405842,63.97009694)(229.61405518,63.75010061)
\curveto(229.78405811,63.50009741)(229.96405793,63.26009765)(230.15405518,63.03010061)
\curveto(230.34405755,62.8100981)(230.56905733,62.62009829)(230.82905518,62.46010061)
\curveto(230.88905701,62.42009849)(230.94905695,62.38509852)(231.00905518,62.35510061)
\curveto(231.05905684,62.32509858)(231.12405677,62.29509861)(231.20405518,62.26510061)
\curveto(231.27405662,62.24509866)(231.33405656,62.24009867)(231.38405518,62.25010061)
\curveto(231.45405644,62.27009864)(231.50905639,62.3050986)(231.54905518,62.35510061)
\curveto(231.57905632,62.4050985)(231.5990563,62.46509844)(231.60905518,62.53510061)
\lineto(231.60905518,62.77510061)
\lineto(231.60905518,63.52510061)
\lineto(231.60905518,66.33010061)
\lineto(231.60905518,66.99010061)
\curveto(231.60905629,67.08009383)(231.61405628,67.16509374)(231.62405518,67.24510061)
\curveto(231.62405627,67.32509358)(231.64405625,67.39009352)(231.68405518,67.44010061)
\curveto(231.72405617,67.49009342)(231.7990561,67.53009338)(231.90905518,67.56010061)
\curveto(232.00905589,67.60009331)(232.10905579,67.6100933)(232.20905518,67.59010061)
\lineto(232.34405518,67.59010061)
\curveto(232.41405548,67.57009334)(232.47405542,67.55009336)(232.52405518,67.53010061)
\curveto(232.57405532,67.5100934)(232.61405528,67.47509343)(232.64405518,67.42510061)
\curveto(232.68405521,67.37509353)(232.70405519,67.3050936)(232.70405518,67.21510061)
\lineto(232.70405518,66.94510061)
\lineto(232.70405518,66.04510061)
\lineto(232.70405518,62.53510061)
\lineto(232.70405518,61.47010061)
\curveto(232.70405519,61.39009952)(232.70905519,61.30009961)(232.71905518,61.20010061)
\curveto(232.71905518,61.10009981)(232.70905519,61.01509989)(232.68905518,60.94510061)
\curveto(232.61905528,60.73510017)(232.43905546,60.67010024)(232.14905518,60.75010061)
\curveto(232.10905579,60.76010015)(232.07405582,60.76010015)(232.04405518,60.75010061)
\curveto(232.00405589,60.75010016)(231.95905594,60.76010015)(231.90905518,60.78010061)
\curveto(231.82905607,60.80010011)(231.74405615,60.82010009)(231.65405518,60.84010061)
\curveto(231.56405633,60.86010005)(231.47905642,60.88510002)(231.39905518,60.91510061)
\curveto(230.90905699,61.07509983)(230.4940574,61.27509963)(230.15405518,61.51510061)
\curveto(229.90405799,61.69509921)(229.67905822,61.90009901)(229.47905518,62.13010061)
\curveto(229.26905863,62.36009855)(229.07405882,62.60009831)(228.89405518,62.85010061)
\curveto(228.71405918,63.1100978)(228.54405935,63.37509753)(228.38405518,63.64510061)
\curveto(228.21405968,63.92509698)(228.03905986,64.19509671)(227.85905518,64.45510061)
\curveto(227.77906012,64.56509634)(227.70406019,64.67009624)(227.63405518,64.77010061)
\curveto(227.56406033,64.88009603)(227.48906041,64.99009592)(227.40905518,65.10010061)
\curveto(227.37906052,65.14009577)(227.34906055,65.17509573)(227.31905518,65.20510061)
\curveto(227.27906062,65.24509566)(227.24906065,65.28509562)(227.22905518,65.32510061)
\curveto(227.11906078,65.46509544)(226.9940609,65.59009532)(226.85405518,65.70010061)
\curveto(226.82406107,65.72009519)(226.7990611,65.74509516)(226.77905518,65.77510061)
\curveto(226.74906115,65.8050951)(226.71906118,65.83009508)(226.68905518,65.85010061)
\curveto(226.58906131,65.93009498)(226.48906141,65.99509491)(226.38905518,66.04510061)
\curveto(226.28906161,66.1050948)(226.17906172,66.16009475)(226.05905518,66.21010061)
\curveto(225.98906191,66.24009467)(225.91406198,66.26009465)(225.83405518,66.27010061)
\lineto(225.59405518,66.33010061)
\lineto(225.50405518,66.33010061)
\curveto(225.47406242,66.34009457)(225.44406245,66.34509456)(225.41405518,66.34510061)
\curveto(225.34406255,66.36509454)(225.24906265,66.37009454)(225.12905518,66.36010061)
\curveto(224.9990629,66.36009455)(224.899063,66.35009456)(224.82905518,66.33010061)
\curveto(224.74906315,66.3100946)(224.67406322,66.29009462)(224.60405518,66.27010061)
\curveto(224.52406337,66.26009465)(224.44406345,66.24009467)(224.36405518,66.21010061)
\curveto(224.12406377,66.10009481)(223.92406397,65.95009496)(223.76405518,65.76010061)
\curveto(223.5940643,65.58009533)(223.45406444,65.36009555)(223.34405518,65.10010061)
\curveto(223.32406457,65.03009588)(223.30906459,64.96009595)(223.29905518,64.89010061)
\curveto(223.27906462,64.82009609)(223.25906464,64.74509616)(223.23905518,64.66510061)
\curveto(223.21906468,64.58509632)(223.20906469,64.47509643)(223.20905518,64.33510061)
\curveto(223.20906469,64.2050967)(223.21906468,64.10009681)(223.23905518,64.02010061)
\curveto(223.24906465,63.96009695)(223.25406464,63.905097)(223.25405518,63.85510061)
\curveto(223.25406464,63.8050971)(223.26406463,63.75509715)(223.28405518,63.70510061)
\curveto(223.32406457,63.6050973)(223.36406453,63.5100974)(223.40405518,63.42010061)
\curveto(223.44406445,63.34009757)(223.48906441,63.26009765)(223.53905518,63.18010061)
\curveto(223.55906434,63.15009776)(223.58406431,63.12009779)(223.61405518,63.09010061)
\curveto(223.64406425,63.07009784)(223.66906423,63.04509786)(223.68905518,63.01510061)
\lineto(223.76405518,62.94010061)
\curveto(223.78406411,62.910098)(223.80406409,62.88509802)(223.82405518,62.86510061)
\lineto(224.03405518,62.71510061)
\curveto(224.0940638,62.67509823)(224.15906374,62.63009828)(224.22905518,62.58010061)
\curveto(224.31906358,62.52009839)(224.42406347,62.47009844)(224.54405518,62.43010061)
\curveto(224.65406324,62.40009851)(224.76406313,62.36509854)(224.87405518,62.32510061)
\curveto(224.98406291,62.28509862)(225.12906277,62.26009865)(225.30905518,62.25010061)
\curveto(225.47906242,62.24009867)(225.60406229,62.2100987)(225.68405518,62.16010061)
\curveto(225.76406213,62.1100988)(225.80906209,62.03509887)(225.81905518,61.93510061)
\curveto(225.82906207,61.83509907)(225.83406206,61.72509918)(225.83405518,61.60510061)
\curveto(225.83406206,61.56509934)(225.83906206,61.52509938)(225.84905518,61.48510061)
\curveto(225.84906205,61.44509946)(225.84406205,61.4100995)(225.83405518,61.38010061)
\curveto(225.81406208,61.33009958)(225.80406209,61.28009963)(225.80405518,61.23010061)
\curveto(225.80406209,61.19009972)(225.7940621,61.15009976)(225.77405518,61.11010061)
\curveto(225.71406218,61.02009989)(225.57906232,60.97509993)(225.36905518,60.97510061)
\lineto(225.24905518,60.97510061)
\curveto(225.18906271,60.98509992)(225.12906277,60.99009992)(225.06905518,60.99010061)
\curveto(224.9990629,61.00009991)(224.93406296,61.0100999)(224.87405518,61.02010061)
\curveto(224.76406313,61.04009987)(224.66406323,61.06009985)(224.57405518,61.08010061)
\curveto(224.47406342,61.10009981)(224.37906352,61.13009978)(224.28905518,61.17010061)
\curveto(224.21906368,61.19009972)(224.15906374,61.2100997)(224.10905518,61.23010061)
\lineto(223.92905518,61.29010061)
\curveto(223.66906423,61.4100995)(223.42406447,61.56509934)(223.19405518,61.75510061)
\curveto(222.96406493,61.95509895)(222.77906512,62.17009874)(222.63905518,62.40010061)
\curveto(222.55906534,62.5100984)(222.4940654,62.62509828)(222.44405518,62.74510061)
\lineto(222.29405518,63.13510061)
\curveto(222.24406565,63.24509766)(222.21406568,63.36009755)(222.20405518,63.48010061)
\curveto(222.18406571,63.60009731)(222.15906574,63.72509718)(222.12905518,63.85510061)
\curveto(222.12906577,63.92509698)(222.12906577,63.99009692)(222.12905518,64.05010061)
\curveto(222.11906578,64.1100968)(222.10906579,64.17509673)(222.09905518,64.24510061)
}
}
{
\newrgbcolor{curcolor}{0 0 0}
\pscustom[linestyle=none,fillstyle=solid,fillcolor=curcolor]
{
\newpath
\moveto(222.09905518,72.59470998)
\curveto(222.08906581,73.28470535)(222.20906569,73.88470475)(222.45905518,74.39470998)
\curveto(222.70906519,74.91470372)(223.04406485,75.30970332)(223.46405518,75.57970998)
\curveto(223.54406435,75.629703)(223.63406426,75.67470296)(223.73405518,75.71470998)
\curveto(223.82406407,75.75470288)(223.91906398,75.79970283)(224.01905518,75.84970998)
\curveto(224.11906378,75.88970274)(224.21906368,75.91970271)(224.31905518,75.93970998)
\curveto(224.41906348,75.95970267)(224.52406337,75.97970265)(224.63405518,75.99970998)
\curveto(224.68406321,76.01970261)(224.72906317,76.02470261)(224.76905518,76.01470998)
\curveto(224.80906309,76.00470263)(224.85406304,76.00970262)(224.90405518,76.02970998)
\curveto(224.95406294,76.03970259)(225.03906286,76.04470259)(225.15905518,76.04470998)
\curveto(225.26906263,76.04470259)(225.35406254,76.03970259)(225.41405518,76.02970998)
\curveto(225.47406242,76.00970262)(225.53406236,75.99970263)(225.59405518,75.99970998)
\curveto(225.65406224,76.00970262)(225.71406218,76.00470263)(225.77405518,75.98470998)
\curveto(225.91406198,75.94470269)(226.04906185,75.90970272)(226.17905518,75.87970998)
\curveto(226.30906159,75.84970278)(226.43406146,75.80970282)(226.55405518,75.75970998)
\curveto(226.6940612,75.69970293)(226.81906108,75.629703)(226.92905518,75.54970998)
\curveto(227.03906086,75.47970315)(227.14906075,75.40470323)(227.25905518,75.32470998)
\lineto(227.31905518,75.26470998)
\curveto(227.33906056,75.25470338)(227.35906054,75.23970339)(227.37905518,75.21970998)
\curveto(227.53906036,75.09970353)(227.68406021,74.96470367)(227.81405518,74.81470998)
\curveto(227.94405995,74.66470397)(228.06905983,74.50470413)(228.18905518,74.33470998)
\curveto(228.40905949,74.02470461)(228.61405928,73.7297049)(228.80405518,73.44970998)
\curveto(228.94405895,73.21970541)(229.07905882,72.98970564)(229.20905518,72.75970998)
\curveto(229.33905856,72.53970609)(229.47405842,72.31970631)(229.61405518,72.09970998)
\curveto(229.78405811,71.84970678)(229.96405793,71.60970702)(230.15405518,71.37970998)
\curveto(230.34405755,71.15970747)(230.56905733,70.96970766)(230.82905518,70.80970998)
\curveto(230.88905701,70.76970786)(230.94905695,70.7347079)(231.00905518,70.70470998)
\curveto(231.05905684,70.67470796)(231.12405677,70.64470799)(231.20405518,70.61470998)
\curveto(231.27405662,70.59470804)(231.33405656,70.58970804)(231.38405518,70.59970998)
\curveto(231.45405644,70.61970801)(231.50905639,70.65470798)(231.54905518,70.70470998)
\curveto(231.57905632,70.75470788)(231.5990563,70.81470782)(231.60905518,70.88470998)
\lineto(231.60905518,71.12470998)
\lineto(231.60905518,71.87470998)
\lineto(231.60905518,74.67970998)
\lineto(231.60905518,75.33970998)
\curveto(231.60905629,75.4297032)(231.61405628,75.51470312)(231.62405518,75.59470998)
\curveto(231.62405627,75.67470296)(231.64405625,75.73970289)(231.68405518,75.78970998)
\curveto(231.72405617,75.83970279)(231.7990561,75.87970275)(231.90905518,75.90970998)
\curveto(232.00905589,75.94970268)(232.10905579,75.95970267)(232.20905518,75.93970998)
\lineto(232.34405518,75.93970998)
\curveto(232.41405548,75.91970271)(232.47405542,75.89970273)(232.52405518,75.87970998)
\curveto(232.57405532,75.85970277)(232.61405528,75.82470281)(232.64405518,75.77470998)
\curveto(232.68405521,75.72470291)(232.70405519,75.65470298)(232.70405518,75.56470998)
\lineto(232.70405518,75.29470998)
\lineto(232.70405518,74.39470998)
\lineto(232.70405518,70.88470998)
\lineto(232.70405518,69.81970998)
\curveto(232.70405519,69.73970889)(232.70905519,69.64970898)(232.71905518,69.54970998)
\curveto(232.71905518,69.44970918)(232.70905519,69.36470927)(232.68905518,69.29470998)
\curveto(232.61905528,69.08470955)(232.43905546,69.01970961)(232.14905518,69.09970998)
\curveto(232.10905579,69.10970952)(232.07405582,69.10970952)(232.04405518,69.09970998)
\curveto(232.00405589,69.09970953)(231.95905594,69.10970952)(231.90905518,69.12970998)
\curveto(231.82905607,69.14970948)(231.74405615,69.16970946)(231.65405518,69.18970998)
\curveto(231.56405633,69.20970942)(231.47905642,69.2347094)(231.39905518,69.26470998)
\curveto(230.90905699,69.42470921)(230.4940574,69.62470901)(230.15405518,69.86470998)
\curveto(229.90405799,70.04470859)(229.67905822,70.24970838)(229.47905518,70.47970998)
\curveto(229.26905863,70.70970792)(229.07405882,70.94970768)(228.89405518,71.19970998)
\curveto(228.71405918,71.45970717)(228.54405935,71.72470691)(228.38405518,71.99470998)
\curveto(228.21405968,72.27470636)(228.03905986,72.54470609)(227.85905518,72.80470998)
\curveto(227.77906012,72.91470572)(227.70406019,73.01970561)(227.63405518,73.11970998)
\curveto(227.56406033,73.2297054)(227.48906041,73.33970529)(227.40905518,73.44970998)
\curveto(227.37906052,73.48970514)(227.34906055,73.52470511)(227.31905518,73.55470998)
\curveto(227.27906062,73.59470504)(227.24906065,73.634705)(227.22905518,73.67470998)
\curveto(227.11906078,73.81470482)(226.9940609,73.93970469)(226.85405518,74.04970998)
\curveto(226.82406107,74.06970456)(226.7990611,74.09470454)(226.77905518,74.12470998)
\curveto(226.74906115,74.15470448)(226.71906118,74.17970445)(226.68905518,74.19970998)
\curveto(226.58906131,74.27970435)(226.48906141,74.34470429)(226.38905518,74.39470998)
\curveto(226.28906161,74.45470418)(226.17906172,74.50970412)(226.05905518,74.55970998)
\curveto(225.98906191,74.58970404)(225.91406198,74.60970402)(225.83405518,74.61970998)
\lineto(225.59405518,74.67970998)
\lineto(225.50405518,74.67970998)
\curveto(225.47406242,74.68970394)(225.44406245,74.69470394)(225.41405518,74.69470998)
\curveto(225.34406255,74.71470392)(225.24906265,74.71970391)(225.12905518,74.70970998)
\curveto(224.9990629,74.70970392)(224.899063,74.69970393)(224.82905518,74.67970998)
\curveto(224.74906315,74.65970397)(224.67406322,74.63970399)(224.60405518,74.61970998)
\curveto(224.52406337,74.60970402)(224.44406345,74.58970404)(224.36405518,74.55970998)
\curveto(224.12406377,74.44970418)(223.92406397,74.29970433)(223.76405518,74.10970998)
\curveto(223.5940643,73.9297047)(223.45406444,73.70970492)(223.34405518,73.44970998)
\curveto(223.32406457,73.37970525)(223.30906459,73.30970532)(223.29905518,73.23970998)
\curveto(223.27906462,73.16970546)(223.25906464,73.09470554)(223.23905518,73.01470998)
\curveto(223.21906468,72.9347057)(223.20906469,72.82470581)(223.20905518,72.68470998)
\curveto(223.20906469,72.55470608)(223.21906468,72.44970618)(223.23905518,72.36970998)
\curveto(223.24906465,72.30970632)(223.25406464,72.25470638)(223.25405518,72.20470998)
\curveto(223.25406464,72.15470648)(223.26406463,72.10470653)(223.28405518,72.05470998)
\curveto(223.32406457,71.95470668)(223.36406453,71.85970677)(223.40405518,71.76970998)
\curveto(223.44406445,71.68970694)(223.48906441,71.60970702)(223.53905518,71.52970998)
\curveto(223.55906434,71.49970713)(223.58406431,71.46970716)(223.61405518,71.43970998)
\curveto(223.64406425,71.41970721)(223.66906423,71.39470724)(223.68905518,71.36470998)
\lineto(223.76405518,71.28970998)
\curveto(223.78406411,71.25970737)(223.80406409,71.2347074)(223.82405518,71.21470998)
\lineto(224.03405518,71.06470998)
\curveto(224.0940638,71.02470761)(224.15906374,70.97970765)(224.22905518,70.92970998)
\curveto(224.31906358,70.86970776)(224.42406347,70.81970781)(224.54405518,70.77970998)
\curveto(224.65406324,70.74970788)(224.76406313,70.71470792)(224.87405518,70.67470998)
\curveto(224.98406291,70.634708)(225.12906277,70.60970802)(225.30905518,70.59970998)
\curveto(225.47906242,70.58970804)(225.60406229,70.55970807)(225.68405518,70.50970998)
\curveto(225.76406213,70.45970817)(225.80906209,70.38470825)(225.81905518,70.28470998)
\curveto(225.82906207,70.18470845)(225.83406206,70.07470856)(225.83405518,69.95470998)
\curveto(225.83406206,69.91470872)(225.83906206,69.87470876)(225.84905518,69.83470998)
\curveto(225.84906205,69.79470884)(225.84406205,69.75970887)(225.83405518,69.72970998)
\curveto(225.81406208,69.67970895)(225.80406209,69.629709)(225.80405518,69.57970998)
\curveto(225.80406209,69.53970909)(225.7940621,69.49970913)(225.77405518,69.45970998)
\curveto(225.71406218,69.36970926)(225.57906232,69.32470931)(225.36905518,69.32470998)
\lineto(225.24905518,69.32470998)
\curveto(225.18906271,69.3347093)(225.12906277,69.33970929)(225.06905518,69.33970998)
\curveto(224.9990629,69.34970928)(224.93406296,69.35970927)(224.87405518,69.36970998)
\curveto(224.76406313,69.38970924)(224.66406323,69.40970922)(224.57405518,69.42970998)
\curveto(224.47406342,69.44970918)(224.37906352,69.47970915)(224.28905518,69.51970998)
\curveto(224.21906368,69.53970909)(224.15906374,69.55970907)(224.10905518,69.57970998)
\lineto(223.92905518,69.63970998)
\curveto(223.66906423,69.75970887)(223.42406447,69.91470872)(223.19405518,70.10470998)
\curveto(222.96406493,70.30470833)(222.77906512,70.51970811)(222.63905518,70.74970998)
\curveto(222.55906534,70.85970777)(222.4940654,70.97470766)(222.44405518,71.09470998)
\lineto(222.29405518,71.48470998)
\curveto(222.24406565,71.59470704)(222.21406568,71.70970692)(222.20405518,71.82970998)
\curveto(222.18406571,71.94970668)(222.15906574,72.07470656)(222.12905518,72.20470998)
\curveto(222.12906577,72.27470636)(222.12906577,72.33970629)(222.12905518,72.39970998)
\curveto(222.11906578,72.45970617)(222.10906579,72.52470611)(222.09905518,72.59470998)
}
}
{
\newrgbcolor{curcolor}{0 0 0}
\pscustom[linestyle=none,fillstyle=solid,fillcolor=curcolor]
{
\newpath
\moveto(231.06905518,78.63431936)
\lineto(231.06905518,79.26431936)
\lineto(231.06905518,79.45931936)
\curveto(231.06905683,79.52931683)(231.07905682,79.58931677)(231.09905518,79.63931936)
\curveto(231.13905676,79.70931665)(231.17905672,79.7593166)(231.21905518,79.78931936)
\curveto(231.26905663,79.82931653)(231.33405656,79.84931651)(231.41405518,79.84931936)
\curveto(231.4940564,79.8593165)(231.57905632,79.86431649)(231.66905518,79.86431936)
\lineto(232.38905518,79.86431936)
\curveto(232.86905503,79.86431649)(233.27905462,79.80431655)(233.61905518,79.68431936)
\curveto(233.95905394,79.56431679)(234.23405366,79.36931699)(234.44405518,79.09931936)
\curveto(234.4940534,79.02931733)(234.53905336,78.9593174)(234.57905518,78.88931936)
\curveto(234.62905327,78.82931753)(234.67405322,78.7543176)(234.71405518,78.66431936)
\curveto(234.72405317,78.64431771)(234.73405316,78.61431774)(234.74405518,78.57431936)
\curveto(234.76405313,78.53431782)(234.76905313,78.48931787)(234.75905518,78.43931936)
\curveto(234.72905317,78.34931801)(234.65405324,78.29431806)(234.53405518,78.27431936)
\curveto(234.42405347,78.2543181)(234.32905357,78.26931809)(234.24905518,78.31931936)
\curveto(234.17905372,78.34931801)(234.11405378,78.39431796)(234.05405518,78.45431936)
\curveto(234.00405389,78.52431783)(233.95405394,78.58931777)(233.90405518,78.64931936)
\curveto(233.85405404,78.71931764)(233.77905412,78.77931758)(233.67905518,78.82931936)
\curveto(233.58905431,78.88931747)(233.4990544,78.93931742)(233.40905518,78.97931936)
\curveto(233.37905452,78.99931736)(233.31905458,79.02431733)(233.22905518,79.05431936)
\curveto(233.14905475,79.08431727)(233.07905482,79.08931727)(233.01905518,79.06931936)
\curveto(232.87905502,79.03931732)(232.78905511,78.97931738)(232.74905518,78.88931936)
\curveto(232.71905518,78.80931755)(232.70405519,78.71931764)(232.70405518,78.61931936)
\curveto(232.70405519,78.51931784)(232.67905522,78.43431792)(232.62905518,78.36431936)
\curveto(232.55905534,78.27431808)(232.41905548,78.22931813)(232.20905518,78.22931936)
\lineto(231.65405518,78.22931936)
\lineto(231.42905518,78.22931936)
\curveto(231.34905655,78.23931812)(231.28405661,78.2593181)(231.23405518,78.28931936)
\curveto(231.15405674,78.34931801)(231.10905679,78.41931794)(231.09905518,78.49931936)
\curveto(231.08905681,78.51931784)(231.08405681,78.53931782)(231.08405518,78.55931936)
\curveto(231.08405681,78.58931777)(231.07905682,78.61431774)(231.06905518,78.63431936)
}
}
{
\newrgbcolor{curcolor}{0 0 0}
\pscustom[linestyle=none,fillstyle=solid,fillcolor=curcolor]
{
}
}
{
\newrgbcolor{curcolor}{0 0 0}
\pscustom[linestyle=none,fillstyle=solid,fillcolor=curcolor]
{
\newpath
\moveto(222.09905518,89.26463186)
\curveto(222.08906581,89.95462722)(222.20906569,90.55462662)(222.45905518,91.06463186)
\curveto(222.70906519,91.58462559)(223.04406485,91.9796252)(223.46405518,92.24963186)
\curveto(223.54406435,92.29962488)(223.63406426,92.34462483)(223.73405518,92.38463186)
\curveto(223.82406407,92.42462475)(223.91906398,92.46962471)(224.01905518,92.51963186)
\curveto(224.11906378,92.55962462)(224.21906368,92.58962459)(224.31905518,92.60963186)
\curveto(224.41906348,92.62962455)(224.52406337,92.64962453)(224.63405518,92.66963186)
\curveto(224.68406321,92.68962449)(224.72906317,92.69462448)(224.76905518,92.68463186)
\curveto(224.80906309,92.6746245)(224.85406304,92.6796245)(224.90405518,92.69963186)
\curveto(224.95406294,92.70962447)(225.03906286,92.71462446)(225.15905518,92.71463186)
\curveto(225.26906263,92.71462446)(225.35406254,92.70962447)(225.41405518,92.69963186)
\curveto(225.47406242,92.6796245)(225.53406236,92.66962451)(225.59405518,92.66963186)
\curveto(225.65406224,92.6796245)(225.71406218,92.6746245)(225.77405518,92.65463186)
\curveto(225.91406198,92.61462456)(226.04906185,92.5796246)(226.17905518,92.54963186)
\curveto(226.30906159,92.51962466)(226.43406146,92.4796247)(226.55405518,92.42963186)
\curveto(226.6940612,92.36962481)(226.81906108,92.29962488)(226.92905518,92.21963186)
\curveto(227.03906086,92.14962503)(227.14906075,92.0746251)(227.25905518,91.99463186)
\lineto(227.31905518,91.93463186)
\curveto(227.33906056,91.92462525)(227.35906054,91.90962527)(227.37905518,91.88963186)
\curveto(227.53906036,91.76962541)(227.68406021,91.63462554)(227.81405518,91.48463186)
\curveto(227.94405995,91.33462584)(228.06905983,91.174626)(228.18905518,91.00463186)
\curveto(228.40905949,90.69462648)(228.61405928,90.39962678)(228.80405518,90.11963186)
\curveto(228.94405895,89.88962729)(229.07905882,89.65962752)(229.20905518,89.42963186)
\curveto(229.33905856,89.20962797)(229.47405842,88.98962819)(229.61405518,88.76963186)
\curveto(229.78405811,88.51962866)(229.96405793,88.2796289)(230.15405518,88.04963186)
\curveto(230.34405755,87.82962935)(230.56905733,87.63962954)(230.82905518,87.47963186)
\curveto(230.88905701,87.43962974)(230.94905695,87.40462977)(231.00905518,87.37463186)
\curveto(231.05905684,87.34462983)(231.12405677,87.31462986)(231.20405518,87.28463186)
\curveto(231.27405662,87.26462991)(231.33405656,87.25962992)(231.38405518,87.26963186)
\curveto(231.45405644,87.28962989)(231.50905639,87.32462985)(231.54905518,87.37463186)
\curveto(231.57905632,87.42462975)(231.5990563,87.48462969)(231.60905518,87.55463186)
\lineto(231.60905518,87.79463186)
\lineto(231.60905518,88.54463186)
\lineto(231.60905518,91.34963186)
\lineto(231.60905518,92.00963186)
\curveto(231.60905629,92.09962508)(231.61405628,92.18462499)(231.62405518,92.26463186)
\curveto(231.62405627,92.34462483)(231.64405625,92.40962477)(231.68405518,92.45963186)
\curveto(231.72405617,92.50962467)(231.7990561,92.54962463)(231.90905518,92.57963186)
\curveto(232.00905589,92.61962456)(232.10905579,92.62962455)(232.20905518,92.60963186)
\lineto(232.34405518,92.60963186)
\curveto(232.41405548,92.58962459)(232.47405542,92.56962461)(232.52405518,92.54963186)
\curveto(232.57405532,92.52962465)(232.61405528,92.49462468)(232.64405518,92.44463186)
\curveto(232.68405521,92.39462478)(232.70405519,92.32462485)(232.70405518,92.23463186)
\lineto(232.70405518,91.96463186)
\lineto(232.70405518,91.06463186)
\lineto(232.70405518,87.55463186)
\lineto(232.70405518,86.48963186)
\curveto(232.70405519,86.40963077)(232.70905519,86.31963086)(232.71905518,86.21963186)
\curveto(232.71905518,86.11963106)(232.70905519,86.03463114)(232.68905518,85.96463186)
\curveto(232.61905528,85.75463142)(232.43905546,85.68963149)(232.14905518,85.76963186)
\curveto(232.10905579,85.7796314)(232.07405582,85.7796314)(232.04405518,85.76963186)
\curveto(232.00405589,85.76963141)(231.95905594,85.7796314)(231.90905518,85.79963186)
\curveto(231.82905607,85.81963136)(231.74405615,85.83963134)(231.65405518,85.85963186)
\curveto(231.56405633,85.8796313)(231.47905642,85.90463127)(231.39905518,85.93463186)
\curveto(230.90905699,86.09463108)(230.4940574,86.29463088)(230.15405518,86.53463186)
\curveto(229.90405799,86.71463046)(229.67905822,86.91963026)(229.47905518,87.14963186)
\curveto(229.26905863,87.3796298)(229.07405882,87.61962956)(228.89405518,87.86963186)
\curveto(228.71405918,88.12962905)(228.54405935,88.39462878)(228.38405518,88.66463186)
\curveto(228.21405968,88.94462823)(228.03905986,89.21462796)(227.85905518,89.47463186)
\curveto(227.77906012,89.58462759)(227.70406019,89.68962749)(227.63405518,89.78963186)
\curveto(227.56406033,89.89962728)(227.48906041,90.00962717)(227.40905518,90.11963186)
\curveto(227.37906052,90.15962702)(227.34906055,90.19462698)(227.31905518,90.22463186)
\curveto(227.27906062,90.26462691)(227.24906065,90.30462687)(227.22905518,90.34463186)
\curveto(227.11906078,90.48462669)(226.9940609,90.60962657)(226.85405518,90.71963186)
\curveto(226.82406107,90.73962644)(226.7990611,90.76462641)(226.77905518,90.79463186)
\curveto(226.74906115,90.82462635)(226.71906118,90.84962633)(226.68905518,90.86963186)
\curveto(226.58906131,90.94962623)(226.48906141,91.01462616)(226.38905518,91.06463186)
\curveto(226.28906161,91.12462605)(226.17906172,91.179626)(226.05905518,91.22963186)
\curveto(225.98906191,91.25962592)(225.91406198,91.2796259)(225.83405518,91.28963186)
\lineto(225.59405518,91.34963186)
\lineto(225.50405518,91.34963186)
\curveto(225.47406242,91.35962582)(225.44406245,91.36462581)(225.41405518,91.36463186)
\curveto(225.34406255,91.38462579)(225.24906265,91.38962579)(225.12905518,91.37963186)
\curveto(224.9990629,91.3796258)(224.899063,91.36962581)(224.82905518,91.34963186)
\curveto(224.74906315,91.32962585)(224.67406322,91.30962587)(224.60405518,91.28963186)
\curveto(224.52406337,91.2796259)(224.44406345,91.25962592)(224.36405518,91.22963186)
\curveto(224.12406377,91.11962606)(223.92406397,90.96962621)(223.76405518,90.77963186)
\curveto(223.5940643,90.59962658)(223.45406444,90.3796268)(223.34405518,90.11963186)
\curveto(223.32406457,90.04962713)(223.30906459,89.9796272)(223.29905518,89.90963186)
\curveto(223.27906462,89.83962734)(223.25906464,89.76462741)(223.23905518,89.68463186)
\curveto(223.21906468,89.60462757)(223.20906469,89.49462768)(223.20905518,89.35463186)
\curveto(223.20906469,89.22462795)(223.21906468,89.11962806)(223.23905518,89.03963186)
\curveto(223.24906465,88.9796282)(223.25406464,88.92462825)(223.25405518,88.87463186)
\curveto(223.25406464,88.82462835)(223.26406463,88.7746284)(223.28405518,88.72463186)
\curveto(223.32406457,88.62462855)(223.36406453,88.52962865)(223.40405518,88.43963186)
\curveto(223.44406445,88.35962882)(223.48906441,88.2796289)(223.53905518,88.19963186)
\curveto(223.55906434,88.16962901)(223.58406431,88.13962904)(223.61405518,88.10963186)
\curveto(223.64406425,88.08962909)(223.66906423,88.06462911)(223.68905518,88.03463186)
\lineto(223.76405518,87.95963186)
\curveto(223.78406411,87.92962925)(223.80406409,87.90462927)(223.82405518,87.88463186)
\lineto(224.03405518,87.73463186)
\curveto(224.0940638,87.69462948)(224.15906374,87.64962953)(224.22905518,87.59963186)
\curveto(224.31906358,87.53962964)(224.42406347,87.48962969)(224.54405518,87.44963186)
\curveto(224.65406324,87.41962976)(224.76406313,87.38462979)(224.87405518,87.34463186)
\curveto(224.98406291,87.30462987)(225.12906277,87.2796299)(225.30905518,87.26963186)
\curveto(225.47906242,87.25962992)(225.60406229,87.22962995)(225.68405518,87.17963186)
\curveto(225.76406213,87.12963005)(225.80906209,87.05463012)(225.81905518,86.95463186)
\curveto(225.82906207,86.85463032)(225.83406206,86.74463043)(225.83405518,86.62463186)
\curveto(225.83406206,86.58463059)(225.83906206,86.54463063)(225.84905518,86.50463186)
\curveto(225.84906205,86.46463071)(225.84406205,86.42963075)(225.83405518,86.39963186)
\curveto(225.81406208,86.34963083)(225.80406209,86.29963088)(225.80405518,86.24963186)
\curveto(225.80406209,86.20963097)(225.7940621,86.16963101)(225.77405518,86.12963186)
\curveto(225.71406218,86.03963114)(225.57906232,85.99463118)(225.36905518,85.99463186)
\lineto(225.24905518,85.99463186)
\curveto(225.18906271,86.00463117)(225.12906277,86.00963117)(225.06905518,86.00963186)
\curveto(224.9990629,86.01963116)(224.93406296,86.02963115)(224.87405518,86.03963186)
\curveto(224.76406313,86.05963112)(224.66406323,86.0796311)(224.57405518,86.09963186)
\curveto(224.47406342,86.11963106)(224.37906352,86.14963103)(224.28905518,86.18963186)
\curveto(224.21906368,86.20963097)(224.15906374,86.22963095)(224.10905518,86.24963186)
\lineto(223.92905518,86.30963186)
\curveto(223.66906423,86.42963075)(223.42406447,86.58463059)(223.19405518,86.77463186)
\curveto(222.96406493,86.9746302)(222.77906512,87.18962999)(222.63905518,87.41963186)
\curveto(222.55906534,87.52962965)(222.4940654,87.64462953)(222.44405518,87.76463186)
\lineto(222.29405518,88.15463186)
\curveto(222.24406565,88.26462891)(222.21406568,88.3796288)(222.20405518,88.49963186)
\curveto(222.18406571,88.61962856)(222.15906574,88.74462843)(222.12905518,88.87463186)
\curveto(222.12906577,88.94462823)(222.12906577,89.00962817)(222.12905518,89.06963186)
\curveto(222.11906578,89.12962805)(222.10906579,89.19462798)(222.09905518,89.26463186)
}
}
{
\newrgbcolor{curcolor}{0 0 0}
\pscustom[linestyle=none,fillstyle=solid,fillcolor=curcolor]
{
\newpath
\moveto(227.61905518,101.36424123)
\lineto(227.87405518,101.36424123)
\curveto(227.95405994,101.37423353)(228.02905987,101.36923353)(228.09905518,101.34924123)
\lineto(228.33905518,101.34924123)
\lineto(228.50405518,101.34924123)
\curveto(228.60405929,101.32923357)(228.70905919,101.31923358)(228.81905518,101.31924123)
\curveto(228.91905898,101.31923358)(229.01905888,101.30923359)(229.11905518,101.28924123)
\lineto(229.26905518,101.28924123)
\curveto(229.40905849,101.25923364)(229.54905835,101.23923366)(229.68905518,101.22924123)
\curveto(229.81905808,101.21923368)(229.94905795,101.19423371)(230.07905518,101.15424123)
\curveto(230.15905774,101.13423377)(230.24405765,101.11423379)(230.33405518,101.09424123)
\lineto(230.57405518,101.03424123)
\lineto(230.87405518,100.91424123)
\curveto(230.96405693,100.88423402)(231.05405684,100.84923405)(231.14405518,100.80924123)
\curveto(231.36405653,100.70923419)(231.57905632,100.57423433)(231.78905518,100.40424123)
\curveto(231.9990559,100.24423466)(232.16905573,100.06923483)(232.29905518,99.87924123)
\curveto(232.33905556,99.82923507)(232.37905552,99.76923513)(232.41905518,99.69924123)
\curveto(232.44905545,99.63923526)(232.48405541,99.57923532)(232.52405518,99.51924123)
\curveto(232.57405532,99.43923546)(232.61405528,99.34423556)(232.64405518,99.23424123)
\curveto(232.67405522,99.12423578)(232.70405519,99.01923588)(232.73405518,98.91924123)
\curveto(232.77405512,98.80923609)(232.7990551,98.6992362)(232.80905518,98.58924123)
\curveto(232.81905508,98.47923642)(232.83405506,98.36423654)(232.85405518,98.24424123)
\curveto(232.86405503,98.2042367)(232.86405503,98.15923674)(232.85405518,98.10924123)
\curveto(232.85405504,98.06923683)(232.85905504,98.02923687)(232.86905518,97.98924123)
\curveto(232.87905502,97.94923695)(232.88405501,97.89423701)(232.88405518,97.82424123)
\curveto(232.88405501,97.75423715)(232.87905502,97.7042372)(232.86905518,97.67424123)
\curveto(232.84905505,97.62423728)(232.84405505,97.57923732)(232.85405518,97.53924123)
\curveto(232.86405503,97.4992374)(232.86405503,97.46423744)(232.85405518,97.43424123)
\lineto(232.85405518,97.34424123)
\curveto(232.83405506,97.28423762)(232.81905508,97.21923768)(232.80905518,97.14924123)
\curveto(232.80905509,97.08923781)(232.80405509,97.02423788)(232.79405518,96.95424123)
\curveto(232.74405515,96.78423812)(232.6940552,96.62423828)(232.64405518,96.47424123)
\curveto(232.5940553,96.32423858)(232.52905537,96.17923872)(232.44905518,96.03924123)
\curveto(232.40905549,95.98923891)(232.37905552,95.93423897)(232.35905518,95.87424123)
\curveto(232.32905557,95.82423908)(232.2940556,95.77423913)(232.25405518,95.72424123)
\curveto(232.07405582,95.48423942)(231.85405604,95.28423962)(231.59405518,95.12424123)
\curveto(231.33405656,94.96423994)(231.04905685,94.82424008)(230.73905518,94.70424123)
\curveto(230.5990573,94.64424026)(230.45905744,94.5992403)(230.31905518,94.56924123)
\curveto(230.16905773,94.53924036)(230.01405788,94.5042404)(229.85405518,94.46424123)
\curveto(229.74405815,94.44424046)(229.63405826,94.42924047)(229.52405518,94.41924123)
\curveto(229.41405848,94.40924049)(229.30405859,94.39424051)(229.19405518,94.37424123)
\curveto(229.15405874,94.36424054)(229.11405878,94.35924054)(229.07405518,94.35924123)
\curveto(229.03405886,94.36924053)(228.9940589,94.36924053)(228.95405518,94.35924123)
\curveto(228.90405899,94.34924055)(228.85405904,94.34424056)(228.80405518,94.34424123)
\lineto(228.63905518,94.34424123)
\curveto(228.58905931,94.32424058)(228.53905936,94.31924058)(228.48905518,94.32924123)
\curveto(228.42905947,94.33924056)(228.37405952,94.33924056)(228.32405518,94.32924123)
\curveto(228.28405961,94.31924058)(228.23905966,94.31924058)(228.18905518,94.32924123)
\curveto(228.13905976,94.33924056)(228.08905981,94.33424057)(228.03905518,94.31424123)
\curveto(227.96905993,94.29424061)(227.89406,94.28924061)(227.81405518,94.29924123)
\curveto(227.72406017,94.30924059)(227.63906026,94.31424059)(227.55905518,94.31424123)
\curveto(227.46906043,94.31424059)(227.36906053,94.30924059)(227.25905518,94.29924123)
\curveto(227.13906076,94.28924061)(227.03906086,94.29424061)(226.95905518,94.31424123)
\lineto(226.67405518,94.31424123)
\lineto(226.04405518,94.35924123)
\curveto(225.94406195,94.36924053)(225.84906205,94.37924052)(225.75905518,94.38924123)
\lineto(225.45905518,94.41924123)
\curveto(225.40906249,94.43924046)(225.35906254,94.44424046)(225.30905518,94.43424123)
\curveto(225.24906265,94.43424047)(225.1940627,94.44424046)(225.14405518,94.46424123)
\curveto(224.97406292,94.51424039)(224.80906309,94.55424035)(224.64905518,94.58424123)
\curveto(224.47906342,94.61424029)(224.31906358,94.66424024)(224.16905518,94.73424123)
\curveto(223.70906419,94.92423998)(223.33406456,95.14423976)(223.04405518,95.39424123)
\curveto(222.75406514,95.65423925)(222.50906539,96.01423889)(222.30905518,96.47424123)
\curveto(222.25906564,96.6042383)(222.22406567,96.73423817)(222.20405518,96.86424123)
\curveto(222.18406571,97.0042379)(222.15906574,97.14423776)(222.12905518,97.28424123)
\curveto(222.11906578,97.35423755)(222.11406578,97.41923748)(222.11405518,97.47924123)
\curveto(222.11406578,97.53923736)(222.10906579,97.6042373)(222.09905518,97.67424123)
\curveto(222.07906582,98.5042364)(222.22906567,99.17423573)(222.54905518,99.68424123)
\curveto(222.85906504,100.19423471)(223.2990646,100.57423433)(223.86905518,100.82424123)
\curveto(223.98906391,100.87423403)(224.11406378,100.91923398)(224.24405518,100.95924123)
\curveto(224.37406352,100.9992339)(224.50906339,101.04423386)(224.64905518,101.09424123)
\curveto(224.72906317,101.11423379)(224.81406308,101.12923377)(224.90405518,101.13924123)
\lineto(225.14405518,101.19924123)
\curveto(225.25406264,101.22923367)(225.36406253,101.24423366)(225.47405518,101.24424123)
\curveto(225.58406231,101.25423365)(225.6940622,101.26923363)(225.80405518,101.28924123)
\curveto(225.85406204,101.30923359)(225.899062,101.31423359)(225.93905518,101.30424123)
\curveto(225.97906192,101.3042336)(226.01906188,101.30923359)(226.05905518,101.31924123)
\curveto(226.10906179,101.32923357)(226.16406173,101.32923357)(226.22405518,101.31924123)
\curveto(226.27406162,101.31923358)(226.32406157,101.32423358)(226.37405518,101.33424123)
\lineto(226.50905518,101.33424123)
\curveto(226.56906133,101.35423355)(226.63906126,101.35423355)(226.71905518,101.33424123)
\curveto(226.78906111,101.32423358)(226.85406104,101.32923357)(226.91405518,101.34924123)
\curveto(226.94406095,101.35923354)(226.98406091,101.36423354)(227.03405518,101.36424123)
\lineto(227.15405518,101.36424123)
\lineto(227.61905518,101.36424123)
\moveto(229.94405518,99.81924123)
\curveto(229.62405827,99.91923498)(229.25905864,99.97923492)(228.84905518,99.99924123)
\curveto(228.43905946,100.01923488)(228.02905987,100.02923487)(227.61905518,100.02924123)
\curveto(227.18906071,100.02923487)(226.76906113,100.01923488)(226.35905518,99.99924123)
\curveto(225.94906195,99.97923492)(225.56406233,99.93423497)(225.20405518,99.86424123)
\curveto(224.84406305,99.79423511)(224.52406337,99.68423522)(224.24405518,99.53424123)
\curveto(223.95406394,99.39423551)(223.71906418,99.1992357)(223.53905518,98.94924123)
\curveto(223.42906447,98.78923611)(223.34906455,98.60923629)(223.29905518,98.40924123)
\curveto(223.23906466,98.20923669)(223.20906469,97.96423694)(223.20905518,97.67424123)
\curveto(223.22906467,97.65423725)(223.23906466,97.61923728)(223.23905518,97.56924123)
\curveto(223.22906467,97.51923738)(223.22906467,97.47923742)(223.23905518,97.44924123)
\curveto(223.25906464,97.36923753)(223.27906462,97.29423761)(223.29905518,97.22424123)
\curveto(223.30906459,97.16423774)(223.32906457,97.0992378)(223.35905518,97.02924123)
\curveto(223.47906442,96.75923814)(223.64906425,96.53923836)(223.86905518,96.36924123)
\curveto(224.07906382,96.20923869)(224.32406357,96.07423883)(224.60405518,95.96424123)
\curveto(224.71406318,95.91423899)(224.83406306,95.87423903)(224.96405518,95.84424123)
\curveto(225.08406281,95.82423908)(225.20906269,95.7992391)(225.33905518,95.76924123)
\curveto(225.38906251,95.74923915)(225.44406245,95.73923916)(225.50405518,95.73924123)
\curveto(225.55406234,95.73923916)(225.60406229,95.73423917)(225.65405518,95.72424123)
\curveto(225.74406215,95.71423919)(225.83906206,95.7042392)(225.93905518,95.69424123)
\curveto(226.02906187,95.68423922)(226.12406177,95.67423923)(226.22405518,95.66424123)
\curveto(226.30406159,95.66423924)(226.38906151,95.65923924)(226.47905518,95.64924123)
\lineto(226.71905518,95.64924123)
\lineto(226.89905518,95.64924123)
\curveto(226.92906097,95.63923926)(226.96406093,95.63423927)(227.00405518,95.63424123)
\lineto(227.13905518,95.63424123)
\lineto(227.58905518,95.63424123)
\curveto(227.66906023,95.63423927)(227.75406014,95.62923927)(227.84405518,95.61924123)
\curveto(227.92405997,95.61923928)(227.9990599,95.62923927)(228.06905518,95.64924123)
\lineto(228.33905518,95.64924123)
\curveto(228.35905954,95.64923925)(228.38905951,95.64423926)(228.42905518,95.63424123)
\curveto(228.45905944,95.63423927)(228.48405941,95.63923926)(228.50405518,95.64924123)
\curveto(228.60405929,95.65923924)(228.70405919,95.66423924)(228.80405518,95.66424123)
\curveto(228.894059,95.67423923)(228.9940589,95.68423922)(229.10405518,95.69424123)
\curveto(229.22405867,95.72423918)(229.34905855,95.73923916)(229.47905518,95.73924123)
\curveto(229.5990583,95.74923915)(229.71405818,95.77423913)(229.82405518,95.81424123)
\curveto(230.12405777,95.89423901)(230.38905751,95.97923892)(230.61905518,96.06924123)
\curveto(230.84905705,96.16923873)(231.06405683,96.31423859)(231.26405518,96.50424123)
\curveto(231.46405643,96.71423819)(231.61405628,96.97923792)(231.71405518,97.29924123)
\curveto(231.73405616,97.33923756)(231.74405615,97.37423753)(231.74405518,97.40424123)
\curveto(231.73405616,97.44423746)(231.73905616,97.48923741)(231.75905518,97.53924123)
\curveto(231.76905613,97.57923732)(231.77905612,97.64923725)(231.78905518,97.74924123)
\curveto(231.7990561,97.85923704)(231.7940561,97.94423696)(231.77405518,98.00424123)
\curveto(231.75405614,98.07423683)(231.74405615,98.14423676)(231.74405518,98.21424123)
\curveto(231.73405616,98.28423662)(231.71905618,98.34923655)(231.69905518,98.40924123)
\curveto(231.63905626,98.60923629)(231.55405634,98.78923611)(231.44405518,98.94924123)
\curveto(231.42405647,98.97923592)(231.40405649,99.0042359)(231.38405518,99.02424123)
\lineto(231.32405518,99.08424123)
\curveto(231.30405659,99.12423578)(231.26405663,99.17423573)(231.20405518,99.23424123)
\curveto(231.06405683,99.33423557)(230.93405696,99.41923548)(230.81405518,99.48924123)
\curveto(230.6940572,99.55923534)(230.54905735,99.62923527)(230.37905518,99.69924123)
\curveto(230.30905759,99.72923517)(230.23905766,99.74923515)(230.16905518,99.75924123)
\curveto(230.0990578,99.77923512)(230.02405787,99.7992351)(229.94405518,99.81924123)
}
}
{
\newrgbcolor{curcolor}{0 0 0}
\pscustom[linestyle=none,fillstyle=solid,fillcolor=curcolor]
{
\newpath
\moveto(222.09905518,106.77385061)
\curveto(222.0990658,106.87384575)(222.10906579,106.96884566)(222.12905518,107.05885061)
\curveto(222.13906576,107.14884548)(222.16906573,107.21384541)(222.21905518,107.25385061)
\curveto(222.2990656,107.31384531)(222.40406549,107.34384528)(222.53405518,107.34385061)
\lineto(222.92405518,107.34385061)
\lineto(224.42405518,107.34385061)
\lineto(230.81405518,107.34385061)
\lineto(231.98405518,107.34385061)
\lineto(232.29905518,107.34385061)
\curveto(232.3990555,107.35384527)(232.47905542,107.33884529)(232.53905518,107.29885061)
\curveto(232.61905528,107.24884538)(232.66905523,107.17384545)(232.68905518,107.07385061)
\curveto(232.6990552,106.98384564)(232.70405519,106.87384575)(232.70405518,106.74385061)
\lineto(232.70405518,106.51885061)
\curveto(232.68405521,106.43884619)(232.66905523,106.36884626)(232.65905518,106.30885061)
\curveto(232.63905526,106.24884638)(232.5990553,106.19884643)(232.53905518,106.15885061)
\curveto(232.47905542,106.11884651)(232.40405549,106.09884653)(232.31405518,106.09885061)
\lineto(232.01405518,106.09885061)
\lineto(230.91905518,106.09885061)
\lineto(225.57905518,106.09885061)
\curveto(225.48906241,106.07884655)(225.41406248,106.06384656)(225.35405518,106.05385061)
\curveto(225.28406261,106.05384657)(225.22406267,106.0238466)(225.17405518,105.96385061)
\curveto(225.12406277,105.89384673)(225.0990628,105.80384682)(225.09905518,105.69385061)
\curveto(225.08906281,105.59384703)(225.08406281,105.48384714)(225.08405518,105.36385061)
\lineto(225.08405518,104.22385061)
\lineto(225.08405518,103.72885061)
\curveto(225.07406282,103.56884906)(225.01406288,103.45884917)(224.90405518,103.39885061)
\curveto(224.87406302,103.37884925)(224.84406305,103.36884926)(224.81405518,103.36885061)
\curveto(224.77406312,103.36884926)(224.72906317,103.36384926)(224.67905518,103.35385061)
\curveto(224.55906334,103.33384929)(224.44906345,103.33884929)(224.34905518,103.36885061)
\curveto(224.24906365,103.40884922)(224.17906372,103.46384916)(224.13905518,103.53385061)
\curveto(224.08906381,103.61384901)(224.06406383,103.73384889)(224.06405518,103.89385061)
\curveto(224.06406383,104.05384857)(224.04906385,104.18884844)(224.01905518,104.29885061)
\curveto(224.00906389,104.34884828)(224.00406389,104.40384822)(224.00405518,104.46385061)
\curveto(223.9940639,104.5238481)(223.97906392,104.58384804)(223.95905518,104.64385061)
\curveto(223.90906399,104.79384783)(223.85906404,104.93884769)(223.80905518,105.07885061)
\curveto(223.74906415,105.21884741)(223.67906422,105.35384727)(223.59905518,105.48385061)
\curveto(223.50906439,105.623847)(223.40406449,105.74384688)(223.28405518,105.84385061)
\curveto(223.16406473,105.94384668)(223.03406486,106.03884659)(222.89405518,106.12885061)
\curveto(222.7940651,106.18884644)(222.68406521,106.23384639)(222.56405518,106.26385061)
\curveto(222.44406545,106.30384632)(222.33906556,106.35384627)(222.24905518,106.41385061)
\curveto(222.18906571,106.46384616)(222.14906575,106.53384609)(222.12905518,106.62385061)
\curveto(222.11906578,106.64384598)(222.11406578,106.66884596)(222.11405518,106.69885061)
\curveto(222.11406578,106.7288459)(222.10906579,106.75384587)(222.09905518,106.77385061)
}
}
{
\newrgbcolor{curcolor}{0 0 0}
\pscustom[linestyle=none,fillstyle=solid,fillcolor=curcolor]
{
\newpath
\moveto(222.09905518,115.12345998)
\curveto(222.0990658,115.22345513)(222.10906579,115.31845503)(222.12905518,115.40845998)
\curveto(222.13906576,115.49845485)(222.16906573,115.56345479)(222.21905518,115.60345998)
\curveto(222.2990656,115.66345469)(222.40406549,115.69345466)(222.53405518,115.69345998)
\lineto(222.92405518,115.69345998)
\lineto(224.42405518,115.69345998)
\lineto(230.81405518,115.69345998)
\lineto(231.98405518,115.69345998)
\lineto(232.29905518,115.69345998)
\curveto(232.3990555,115.70345465)(232.47905542,115.68845466)(232.53905518,115.64845998)
\curveto(232.61905528,115.59845475)(232.66905523,115.52345483)(232.68905518,115.42345998)
\curveto(232.6990552,115.33345502)(232.70405519,115.22345513)(232.70405518,115.09345998)
\lineto(232.70405518,114.86845998)
\curveto(232.68405521,114.78845556)(232.66905523,114.71845563)(232.65905518,114.65845998)
\curveto(232.63905526,114.59845575)(232.5990553,114.5484558)(232.53905518,114.50845998)
\curveto(232.47905542,114.46845588)(232.40405549,114.4484559)(232.31405518,114.44845998)
\lineto(232.01405518,114.44845998)
\lineto(230.91905518,114.44845998)
\lineto(225.57905518,114.44845998)
\curveto(225.48906241,114.42845592)(225.41406248,114.41345594)(225.35405518,114.40345998)
\curveto(225.28406261,114.40345595)(225.22406267,114.37345598)(225.17405518,114.31345998)
\curveto(225.12406277,114.24345611)(225.0990628,114.1534562)(225.09905518,114.04345998)
\curveto(225.08906281,113.94345641)(225.08406281,113.83345652)(225.08405518,113.71345998)
\lineto(225.08405518,112.57345998)
\lineto(225.08405518,112.07845998)
\curveto(225.07406282,111.91845843)(225.01406288,111.80845854)(224.90405518,111.74845998)
\curveto(224.87406302,111.72845862)(224.84406305,111.71845863)(224.81405518,111.71845998)
\curveto(224.77406312,111.71845863)(224.72906317,111.71345864)(224.67905518,111.70345998)
\curveto(224.55906334,111.68345867)(224.44906345,111.68845866)(224.34905518,111.71845998)
\curveto(224.24906365,111.75845859)(224.17906372,111.81345854)(224.13905518,111.88345998)
\curveto(224.08906381,111.96345839)(224.06406383,112.08345827)(224.06405518,112.24345998)
\curveto(224.06406383,112.40345795)(224.04906385,112.53845781)(224.01905518,112.64845998)
\curveto(224.00906389,112.69845765)(224.00406389,112.7534576)(224.00405518,112.81345998)
\curveto(223.9940639,112.87345748)(223.97906392,112.93345742)(223.95905518,112.99345998)
\curveto(223.90906399,113.14345721)(223.85906404,113.28845706)(223.80905518,113.42845998)
\curveto(223.74906415,113.56845678)(223.67906422,113.70345665)(223.59905518,113.83345998)
\curveto(223.50906439,113.97345638)(223.40406449,114.09345626)(223.28405518,114.19345998)
\curveto(223.16406473,114.29345606)(223.03406486,114.38845596)(222.89405518,114.47845998)
\curveto(222.7940651,114.53845581)(222.68406521,114.58345577)(222.56405518,114.61345998)
\curveto(222.44406545,114.6534557)(222.33906556,114.70345565)(222.24905518,114.76345998)
\curveto(222.18906571,114.81345554)(222.14906575,114.88345547)(222.12905518,114.97345998)
\curveto(222.11906578,114.99345536)(222.11406578,115.01845533)(222.11405518,115.04845998)
\curveto(222.11406578,115.07845527)(222.10906579,115.10345525)(222.09905518,115.12345998)
}
}
{
\newrgbcolor{curcolor}{0 0 0}
\pscustom[linestyle=none,fillstyle=solid,fillcolor=curcolor]
{
\newpath
\moveto(242.93538635,31.67142873)
\lineto(242.93538635,32.58642873)
\curveto(242.93539705,32.68642608)(242.93539705,32.78142599)(242.93538635,32.87142873)
\curveto(242.93539705,32.96142581)(242.95539703,33.03642573)(242.99538635,33.09642873)
\curveto(243.05539693,33.18642558)(243.13539685,33.24642552)(243.23538635,33.27642873)
\curveto(243.33539665,33.31642545)(243.44039654,33.36142541)(243.55038635,33.41142873)
\curveto(243.74039624,33.49142528)(243.93039605,33.56142521)(244.12038635,33.62142873)
\curveto(244.31039567,33.69142508)(244.50039548,33.766425)(244.69038635,33.84642873)
\curveto(244.87039511,33.91642485)(245.05539493,33.98142479)(245.24538635,34.04142873)
\curveto(245.42539456,34.10142467)(245.60539438,34.1714246)(245.78538635,34.25142873)
\curveto(245.92539406,34.31142446)(246.07039391,34.3664244)(246.22038635,34.41642873)
\curveto(246.37039361,34.4664243)(246.51539347,34.52142425)(246.65538635,34.58142873)
\curveto(247.10539288,34.76142401)(247.56039242,34.93142384)(248.02038635,35.09142873)
\curveto(248.47039151,35.25142352)(248.92039106,35.42142335)(249.37038635,35.60142873)
\curveto(249.42039056,35.62142315)(249.47039051,35.63642313)(249.52038635,35.64642873)
\lineto(249.67038635,35.70642873)
\curveto(249.89039009,35.79642297)(250.11538987,35.88142289)(250.34538635,35.96142873)
\curveto(250.56538942,36.04142273)(250.7853892,36.12642264)(251.00538635,36.21642873)
\curveto(251.09538889,36.25642251)(251.20538878,36.29642247)(251.33538635,36.33642873)
\curveto(251.45538853,36.37642239)(251.52538846,36.44142233)(251.54538635,36.53142873)
\curveto(251.55538843,36.5714222)(251.55538843,36.60142217)(251.54538635,36.62142873)
\lineto(251.48538635,36.68142873)
\curveto(251.43538855,36.73142204)(251.3803886,36.766422)(251.32038635,36.78642873)
\curveto(251.26038872,36.81642195)(251.19538879,36.84642192)(251.12538635,36.87642873)
\lineto(250.49538635,37.11642873)
\curveto(250.27538971,37.19642157)(250.06038992,37.27642149)(249.85038635,37.35642873)
\lineto(249.70038635,37.41642873)
\lineto(249.52038635,37.47642873)
\curveto(249.33039065,37.55642121)(249.14039084,37.62642114)(248.95038635,37.68642873)
\curveto(248.75039123,37.75642101)(248.55039143,37.83142094)(248.35038635,37.91142873)
\curveto(247.77039221,38.15142062)(247.1853928,38.3714204)(246.59538635,38.57142873)
\curveto(246.00539398,38.78141999)(245.42039456,39.00641976)(244.84038635,39.24642873)
\curveto(244.64039534,39.32641944)(244.43539555,39.40141937)(244.22538635,39.47142873)
\curveto(244.01539597,39.55141922)(243.81039617,39.63141914)(243.61038635,39.71142873)
\curveto(243.53039645,39.75141902)(243.43039655,39.78641898)(243.31038635,39.81642873)
\curveto(243.19039679,39.85641891)(243.10539688,39.91141886)(243.05538635,39.98142873)
\curveto(243.01539697,40.04141873)(242.985397,40.11641865)(242.96538635,40.20642873)
\curveto(242.94539704,40.30641846)(242.93539705,40.41641835)(242.93538635,40.53642873)
\curveto(242.92539706,40.65641811)(242.92539706,40.77641799)(242.93538635,40.89642873)
\curveto(242.93539705,41.01641775)(242.93539705,41.12641764)(242.93538635,41.22642873)
\curveto(242.93539705,41.31641745)(242.93539705,41.40641736)(242.93538635,41.49642873)
\curveto(242.93539705,41.59641717)(242.95539703,41.6714171)(242.99538635,41.72142873)
\curveto(243.04539694,41.81141696)(243.13539685,41.86141691)(243.26538635,41.87142873)
\curveto(243.39539659,41.88141689)(243.53539645,41.88641688)(243.68538635,41.88642873)
\lineto(245.33538635,41.88642873)
\lineto(251.60538635,41.88642873)
\lineto(252.86538635,41.88642873)
\curveto(252.97538701,41.88641688)(253.0853869,41.88641688)(253.19538635,41.88642873)
\curveto(253.30538668,41.89641687)(253.39038659,41.87641689)(253.45038635,41.82642873)
\curveto(253.51038647,41.79641697)(253.55038643,41.75141702)(253.57038635,41.69142873)
\curveto(253.5803864,41.63141714)(253.59538639,41.56141721)(253.61538635,41.48142873)
\lineto(253.61538635,41.24142873)
\lineto(253.61538635,40.88142873)
\curveto(253.60538638,40.771418)(253.56038642,40.69141808)(253.48038635,40.64142873)
\curveto(253.45038653,40.62141815)(253.42038656,40.60641816)(253.39038635,40.59642873)
\curveto(253.35038663,40.59641817)(253.30538668,40.58641818)(253.25538635,40.56642873)
\lineto(253.09038635,40.56642873)
\curveto(253.03038695,40.55641821)(252.96038702,40.55141822)(252.88038635,40.55142873)
\curveto(252.80038718,40.56141821)(252.72538726,40.5664182)(252.65538635,40.56642873)
\lineto(251.81538635,40.56642873)
\lineto(247.39038635,40.56642873)
\curveto(247.14039284,40.5664182)(246.89039309,40.5664182)(246.64038635,40.56642873)
\curveto(246.3803936,40.5664182)(246.13039385,40.56141821)(245.89038635,40.55142873)
\curveto(245.79039419,40.55141822)(245.6803943,40.54641822)(245.56038635,40.53642873)
\curveto(245.44039454,40.52641824)(245.3803946,40.4714183)(245.38038635,40.37142873)
\lineto(245.39538635,40.37142873)
\curveto(245.41539457,40.30141847)(245.4803945,40.24141853)(245.59038635,40.19142873)
\curveto(245.70039428,40.15141862)(245.79539419,40.11641865)(245.87538635,40.08642873)
\curveto(246.04539394,40.01641875)(246.22039376,39.95141882)(246.40038635,39.89142873)
\curveto(246.57039341,39.83141894)(246.74039324,39.76141901)(246.91038635,39.68142873)
\curveto(246.96039302,39.66141911)(247.00539298,39.64641912)(247.04538635,39.63642873)
\curveto(247.0853929,39.62641914)(247.13039285,39.61141916)(247.18038635,39.59142873)
\curveto(247.36039262,39.51141926)(247.54539244,39.44141933)(247.73538635,39.38142873)
\curveto(247.91539207,39.33141944)(248.09539189,39.2664195)(248.27538635,39.18642873)
\curveto(248.42539156,39.11641965)(248.5803914,39.05641971)(248.74038635,39.00642873)
\curveto(248.89039109,38.95641981)(249.04039094,38.90141987)(249.19038635,38.84142873)
\curveto(249.66039032,38.64142013)(250.13538985,38.46142031)(250.61538635,38.30142873)
\curveto(251.0853889,38.14142063)(251.55038843,37.9664208)(252.01038635,37.77642873)
\curveto(252.19038779,37.69642107)(252.37038761,37.62642114)(252.55038635,37.56642873)
\curveto(252.73038725,37.50642126)(252.91038707,37.44142133)(253.09038635,37.37142873)
\curveto(253.20038678,37.32142145)(253.30538668,37.2714215)(253.40538635,37.22142873)
\curveto(253.49538649,37.18142159)(253.56038642,37.09642167)(253.60038635,36.96642873)
\curveto(253.61038637,36.94642182)(253.61538637,36.92142185)(253.61538635,36.89142873)
\curveto(253.60538638,36.8714219)(253.60538638,36.84642192)(253.61538635,36.81642873)
\curveto(253.62538636,36.78642198)(253.63038635,36.75142202)(253.63038635,36.71142873)
\curveto(253.62038636,36.6714221)(253.61538637,36.63142214)(253.61538635,36.59142873)
\lineto(253.61538635,36.29142873)
\curveto(253.61538637,36.19142258)(253.59038639,36.11142266)(253.54038635,36.05142873)
\curveto(253.49038649,35.9714228)(253.42038656,35.91142286)(253.33038635,35.87142873)
\curveto(253.23038675,35.84142293)(253.13038685,35.80142297)(253.03038635,35.75142873)
\curveto(252.83038715,35.6714231)(252.62538736,35.59142318)(252.41538635,35.51142873)
\curveto(252.19538779,35.44142333)(251.985388,35.3664234)(251.78538635,35.28642873)
\curveto(251.60538838,35.20642356)(251.42538856,35.13642363)(251.24538635,35.07642873)
\curveto(251.05538893,35.02642374)(250.87038911,34.96142381)(250.69038635,34.88142873)
\curveto(250.13038985,34.65142412)(249.56539042,34.43642433)(248.99538635,34.23642873)
\curveto(248.42539156,34.03642473)(247.86039212,33.82142495)(247.30038635,33.59142873)
\lineto(246.67038635,33.35142873)
\curveto(246.45039353,33.28142549)(246.24039374,33.20642556)(246.04038635,33.12642873)
\curveto(245.93039405,33.07642569)(245.82539416,33.03142574)(245.72538635,32.99142873)
\curveto(245.61539437,32.96142581)(245.52039446,32.91142586)(245.44038635,32.84142873)
\curveto(245.42039456,32.83142594)(245.41039457,32.82142595)(245.41038635,32.81142873)
\lineto(245.38038635,32.78142873)
\lineto(245.38038635,32.70642873)
\lineto(245.41038635,32.67642873)
\curveto(245.41039457,32.6664261)(245.41539457,32.65642611)(245.42538635,32.64642873)
\curveto(245.47539451,32.62642614)(245.53039445,32.61642615)(245.59038635,32.61642873)
\curveto(245.65039433,32.61642615)(245.71039427,32.60642616)(245.77038635,32.58642873)
\lineto(245.93538635,32.58642873)
\curveto(245.99539399,32.5664262)(246.06039392,32.56142621)(246.13038635,32.57142873)
\curveto(246.20039378,32.58142619)(246.27039371,32.58642618)(246.34038635,32.58642873)
\lineto(247.15038635,32.58642873)
\lineto(251.71038635,32.58642873)
\lineto(252.89538635,32.58642873)
\curveto(253.00538698,32.58642618)(253.11538687,32.58142619)(253.22538635,32.57142873)
\curveto(253.33538665,32.5714262)(253.42038656,32.54642622)(253.48038635,32.49642873)
\curveto(253.56038642,32.44642632)(253.60538638,32.35642641)(253.61538635,32.22642873)
\lineto(253.61538635,31.83642873)
\lineto(253.61538635,31.64142873)
\curveto(253.61538637,31.59142718)(253.60538638,31.54142723)(253.58538635,31.49142873)
\curveto(253.54538644,31.36142741)(253.46038652,31.28642748)(253.33038635,31.26642873)
\curveto(253.20038678,31.25642751)(253.05038693,31.25142752)(252.88038635,31.25142873)
\lineto(251.14038635,31.25142873)
\lineto(245.14038635,31.25142873)
\lineto(243.73038635,31.25142873)
\curveto(243.62039636,31.25142752)(243.50539648,31.24642752)(243.38538635,31.23642873)
\curveto(243.26539672,31.23642753)(243.17039681,31.26142751)(243.10038635,31.31142873)
\curveto(243.04039694,31.35142742)(242.99039699,31.42642734)(242.95038635,31.53642873)
\curveto(242.94039704,31.55642721)(242.94039704,31.57642719)(242.95038635,31.59642873)
\curveto(242.95039703,31.62642714)(242.94539704,31.65142712)(242.93538635,31.67142873)
}
}
{
\newrgbcolor{curcolor}{0 0 0}
\pscustom[linestyle=none,fillstyle=solid,fillcolor=curcolor]
{
\newpath
\moveto(253.06038635,50.87353811)
\curveto(253.22038676,50.90353028)(253.35538663,50.88853029)(253.46538635,50.82853811)
\curveto(253.56538642,50.76853041)(253.64038634,50.68853049)(253.69038635,50.58853811)
\curveto(253.71038627,50.53853064)(253.72038626,50.4835307)(253.72038635,50.42353811)
\curveto(253.72038626,50.37353081)(253.73038625,50.31853086)(253.75038635,50.25853811)
\curveto(253.80038618,50.03853114)(253.7853862,49.81853136)(253.70538635,49.59853811)
\curveto(253.63538635,49.38853179)(253.54538644,49.24353194)(253.43538635,49.16353811)
\curveto(253.36538662,49.11353207)(253.2853867,49.06853211)(253.19538635,49.02853811)
\curveto(253.09538689,48.98853219)(253.01538697,48.93853224)(252.95538635,48.87853811)
\curveto(252.93538705,48.85853232)(252.91538707,48.83353235)(252.89538635,48.80353811)
\curveto(252.87538711,48.7835324)(252.87038711,48.75353243)(252.88038635,48.71353811)
\curveto(252.91038707,48.60353258)(252.96538702,48.49853268)(253.04538635,48.39853811)
\curveto(253.12538686,48.30853287)(253.19538679,48.21853296)(253.25538635,48.12853811)
\curveto(253.33538665,47.99853318)(253.41038657,47.85853332)(253.48038635,47.70853811)
\curveto(253.54038644,47.55853362)(253.59538639,47.39853378)(253.64538635,47.22853811)
\curveto(253.67538631,47.12853405)(253.69538629,47.01853416)(253.70538635,46.89853811)
\curveto(253.71538627,46.78853439)(253.73038625,46.6785345)(253.75038635,46.56853811)
\curveto(253.76038622,46.51853466)(253.76538622,46.47353471)(253.76538635,46.43353811)
\lineto(253.76538635,46.32853811)
\curveto(253.7853862,46.21853496)(253.7853862,46.11353507)(253.76538635,46.01353811)
\lineto(253.76538635,45.87853811)
\curveto(253.75538623,45.82853535)(253.75038623,45.7785354)(253.75038635,45.72853811)
\curveto(253.75038623,45.6785355)(253.74038624,45.63353555)(253.72038635,45.59353811)
\curveto(253.71038627,45.55353563)(253.70538628,45.51853566)(253.70538635,45.48853811)
\curveto(253.71538627,45.46853571)(253.71538627,45.44353574)(253.70538635,45.41353811)
\lineto(253.64538635,45.17353811)
\curveto(253.63538635,45.09353609)(253.61538637,45.01853616)(253.58538635,44.94853811)
\curveto(253.45538653,44.64853653)(253.31038667,44.40353678)(253.15038635,44.21353811)
\curveto(252.980387,44.03353715)(252.74538724,43.8835373)(252.44538635,43.76353811)
\curveto(252.22538776,43.67353751)(251.96038802,43.62853755)(251.65038635,43.62853811)
\lineto(251.33538635,43.62853811)
\curveto(251.2853887,43.63853754)(251.23538875,43.64353754)(251.18538635,43.64353811)
\lineto(251.00538635,43.67353811)
\lineto(250.67538635,43.79353811)
\curveto(250.56538942,43.83353735)(250.46538952,43.8835373)(250.37538635,43.94353811)
\curveto(250.0853899,44.12353706)(249.87039011,44.36853681)(249.73038635,44.67853811)
\curveto(249.59039039,44.98853619)(249.46539052,45.32853585)(249.35538635,45.69853811)
\curveto(249.31539067,45.83853534)(249.2853907,45.9835352)(249.26538635,46.13353811)
\curveto(249.24539074,46.2835349)(249.22039076,46.43353475)(249.19038635,46.58353811)
\curveto(249.17039081,46.65353453)(249.16039082,46.71853446)(249.16038635,46.77853811)
\curveto(249.16039082,46.84853433)(249.15039083,46.92353426)(249.13038635,47.00353811)
\curveto(249.11039087,47.07353411)(249.10039088,47.14353404)(249.10038635,47.21353811)
\curveto(249.09039089,47.2835339)(249.07539091,47.35853382)(249.05538635,47.43853811)
\curveto(248.99539099,47.68853349)(248.94539104,47.92353326)(248.90538635,48.14353811)
\curveto(248.85539113,48.36353282)(248.74039124,48.53853264)(248.56038635,48.66853811)
\curveto(248.4803915,48.72853245)(248.3803916,48.7785324)(248.26038635,48.81853811)
\curveto(248.13039185,48.85853232)(247.99039199,48.85853232)(247.84038635,48.81853811)
\curveto(247.60039238,48.75853242)(247.41039257,48.66853251)(247.27038635,48.54853811)
\curveto(247.13039285,48.43853274)(247.02039296,48.2785329)(246.94038635,48.06853811)
\curveto(246.89039309,47.94853323)(246.85539313,47.80353338)(246.83538635,47.63353811)
\curveto(246.81539317,47.47353371)(246.80539318,47.30353388)(246.80538635,47.12353811)
\curveto(246.80539318,46.94353424)(246.81539317,46.76853441)(246.83538635,46.59853811)
\curveto(246.85539313,46.42853475)(246.8853931,46.2835349)(246.92538635,46.16353811)
\curveto(246.985393,45.99353519)(247.07039291,45.82853535)(247.18038635,45.66853811)
\curveto(247.24039274,45.58853559)(247.32039266,45.51353567)(247.42038635,45.44353811)
\curveto(247.51039247,45.3835358)(247.61039237,45.32853585)(247.72038635,45.27853811)
\curveto(247.80039218,45.24853593)(247.8853921,45.21853596)(247.97538635,45.18853811)
\curveto(248.06539192,45.16853601)(248.13539185,45.12353606)(248.18538635,45.05353811)
\curveto(248.21539177,45.01353617)(248.24039174,44.94353624)(248.26038635,44.84353811)
\curveto(248.27039171,44.75353643)(248.27539171,44.65853652)(248.27538635,44.55853811)
\curveto(248.27539171,44.45853672)(248.27039171,44.35853682)(248.26038635,44.25853811)
\curveto(248.24039174,44.16853701)(248.21539177,44.10353708)(248.18538635,44.06353811)
\curveto(248.15539183,44.02353716)(248.10539188,43.99353719)(248.03538635,43.97353811)
\curveto(247.96539202,43.95353723)(247.89039209,43.95353723)(247.81038635,43.97353811)
\curveto(247.6803923,44.00353718)(247.56039242,44.03353715)(247.45038635,44.06353811)
\curveto(247.33039265,44.10353708)(247.21539277,44.14853703)(247.10538635,44.19853811)
\curveto(246.75539323,44.38853679)(246.4853935,44.62853655)(246.29538635,44.91853811)
\curveto(246.09539389,45.20853597)(245.93539405,45.56853561)(245.81538635,45.99853811)
\curveto(245.79539419,46.09853508)(245.7803942,46.19853498)(245.77038635,46.29853811)
\curveto(245.76039422,46.40853477)(245.74539424,46.51853466)(245.72538635,46.62853811)
\curveto(245.71539427,46.66853451)(245.71539427,46.73353445)(245.72538635,46.82353811)
\curveto(245.72539426,46.91353427)(245.71539427,46.96853421)(245.69538635,46.98853811)
\curveto(245.6853943,47.68853349)(245.76539422,48.29853288)(245.93538635,48.81853811)
\curveto(246.10539388,49.33853184)(246.43039355,49.70353148)(246.91038635,49.91353811)
\curveto(247.11039287,50.00353118)(247.34539264,50.05353113)(247.61538635,50.06353811)
\curveto(247.87539211,50.0835311)(248.15039183,50.09353109)(248.44038635,50.09353811)
\lineto(251.75538635,50.09353811)
\curveto(251.89538809,50.09353109)(252.03038795,50.09853108)(252.16038635,50.10853811)
\curveto(252.29038769,50.11853106)(252.39538759,50.14853103)(252.47538635,50.19853811)
\curveto(252.54538744,50.24853093)(252.59538739,50.31353087)(252.62538635,50.39353811)
\curveto(252.66538732,50.4835307)(252.69538729,50.56853061)(252.71538635,50.64853811)
\curveto(252.72538726,50.72853045)(252.77038721,50.78853039)(252.85038635,50.82853811)
\curveto(252.8803871,50.84853033)(252.91038707,50.85853032)(252.94038635,50.85853811)
\curveto(252.97038701,50.85853032)(253.01038697,50.86353032)(253.06038635,50.87353811)
\moveto(251.39538635,48.72853811)
\curveto(251.25538873,48.78853239)(251.09538889,48.81853236)(250.91538635,48.81853811)
\curveto(250.72538926,48.82853235)(250.53038945,48.83353235)(250.33038635,48.83353811)
\curveto(250.22038976,48.83353235)(250.12038986,48.82853235)(250.03038635,48.81853811)
\curveto(249.94039004,48.80853237)(249.87039011,48.76853241)(249.82038635,48.69853811)
\curveto(249.80039018,48.66853251)(249.79039019,48.59853258)(249.79038635,48.48853811)
\curveto(249.81039017,48.46853271)(249.82039016,48.43353275)(249.82038635,48.38353811)
\curveto(249.82039016,48.33353285)(249.83039015,48.28853289)(249.85038635,48.24853811)
\curveto(249.87039011,48.16853301)(249.89039009,48.0785331)(249.91038635,47.97853811)
\lineto(249.97038635,47.67853811)
\curveto(249.97039001,47.64853353)(249.97539001,47.61353357)(249.98538635,47.57353811)
\lineto(249.98538635,47.46853811)
\curveto(250.02538996,47.31853386)(250.05038993,47.15353403)(250.06038635,46.97353811)
\curveto(250.06038992,46.80353438)(250.0803899,46.64353454)(250.12038635,46.49353811)
\curveto(250.14038984,46.41353477)(250.16038982,46.33853484)(250.18038635,46.26853811)
\curveto(250.19038979,46.20853497)(250.20538978,46.13853504)(250.22538635,46.05853811)
\curveto(250.27538971,45.89853528)(250.34038964,45.74853543)(250.42038635,45.60853811)
\curveto(250.49038949,45.46853571)(250.5803894,45.34853583)(250.69038635,45.24853811)
\curveto(250.80038918,45.14853603)(250.93538905,45.07353611)(251.09538635,45.02353811)
\curveto(251.24538874,44.97353621)(251.43038855,44.95353623)(251.65038635,44.96353811)
\curveto(251.75038823,44.96353622)(251.84538814,44.9785362)(251.93538635,45.00853811)
\curveto(252.01538797,45.04853613)(252.09038789,45.09353609)(252.16038635,45.14353811)
\curveto(252.27038771,45.22353596)(252.36538762,45.32853585)(252.44538635,45.45853811)
\curveto(252.51538747,45.58853559)(252.57538741,45.72853545)(252.62538635,45.87853811)
\curveto(252.63538735,45.92853525)(252.64038734,45.9785352)(252.64038635,46.02853811)
\curveto(252.64038734,46.0785351)(252.64538734,46.12853505)(252.65538635,46.17853811)
\curveto(252.67538731,46.24853493)(252.69038729,46.33353485)(252.70038635,46.43353811)
\curveto(252.70038728,46.54353464)(252.69038729,46.63353455)(252.67038635,46.70353811)
\curveto(252.65038733,46.76353442)(252.64538734,46.82353436)(252.65538635,46.88353811)
\curveto(252.65538733,46.94353424)(252.64538734,47.00353418)(252.62538635,47.06353811)
\curveto(252.60538738,47.14353404)(252.59038739,47.21853396)(252.58038635,47.28853811)
\curveto(252.57038741,47.36853381)(252.55038743,47.44353374)(252.52038635,47.51353811)
\curveto(252.40038758,47.80353338)(252.25538773,48.04853313)(252.08538635,48.24853811)
\curveto(251.91538807,48.45853272)(251.6853883,48.61853256)(251.39538635,48.72853811)
}
}
{
\newrgbcolor{curcolor}{0 0 0}
\pscustom[linestyle=none,fillstyle=solid,fillcolor=curcolor]
{
\newpath
\moveto(245.71038635,55.69017873)
\curveto(245.71039427,55.92017394)(245.77039421,56.05017381)(245.89038635,56.08017873)
\curveto(246.00039398,56.11017375)(246.16539382,56.12517374)(246.38538635,56.12517873)
\lineto(246.67038635,56.12517873)
\curveto(246.76039322,56.12517374)(246.83539315,56.10017376)(246.89538635,56.05017873)
\curveto(246.97539301,55.99017387)(247.02039296,55.90517396)(247.03038635,55.79517873)
\curveto(247.03039295,55.68517418)(247.04539294,55.57517429)(247.07538635,55.46517873)
\curveto(247.10539288,55.32517454)(247.13539285,55.19017467)(247.16538635,55.06017873)
\curveto(247.19539279,54.94017492)(247.23539275,54.82517504)(247.28538635,54.71517873)
\curveto(247.41539257,54.42517544)(247.59539239,54.19017567)(247.82538635,54.01017873)
\curveto(248.04539194,53.83017603)(248.30039168,53.67517619)(248.59038635,53.54517873)
\curveto(248.70039128,53.50517636)(248.81539117,53.47517639)(248.93538635,53.45517873)
\curveto(249.04539094,53.43517643)(249.16039082,53.41017645)(249.28038635,53.38017873)
\curveto(249.33039065,53.37017649)(249.3803906,53.3651765)(249.43038635,53.36517873)
\curveto(249.4803905,53.37517649)(249.53039045,53.37517649)(249.58038635,53.36517873)
\curveto(249.70039028,53.33517653)(249.84039014,53.32017654)(250.00038635,53.32017873)
\curveto(250.15038983,53.33017653)(250.29538969,53.33517653)(250.43538635,53.33517873)
\lineto(252.28038635,53.33517873)
\lineto(252.62538635,53.33517873)
\curveto(252.74538724,53.33517653)(252.86038712,53.33017653)(252.97038635,53.32017873)
\curveto(253.0803869,53.31017655)(253.17538681,53.30517656)(253.25538635,53.30517873)
\curveto(253.33538665,53.31517655)(253.40538658,53.29517657)(253.46538635,53.24517873)
\curveto(253.53538645,53.19517667)(253.57538641,53.11517675)(253.58538635,53.00517873)
\curveto(253.59538639,52.90517696)(253.60038638,52.79517707)(253.60038635,52.67517873)
\lineto(253.60038635,52.40517873)
\curveto(253.5803864,52.35517751)(253.56538642,52.30517756)(253.55538635,52.25517873)
\curveto(253.53538645,52.21517765)(253.51038647,52.18517768)(253.48038635,52.16517873)
\curveto(253.41038657,52.11517775)(253.32538666,52.08517778)(253.22538635,52.07517873)
\lineto(252.89538635,52.07517873)
\lineto(251.74038635,52.07517873)
\lineto(247.58538635,52.07517873)
\lineto(246.55038635,52.07517873)
\lineto(246.25038635,52.07517873)
\curveto(246.15039383,52.08517778)(246.06539392,52.11517775)(245.99538635,52.16517873)
\curveto(245.95539403,52.19517767)(245.92539406,52.24517762)(245.90538635,52.31517873)
\curveto(245.8853941,52.39517747)(245.87539411,52.48017738)(245.87538635,52.57017873)
\curveto(245.86539412,52.6601772)(245.86539412,52.75017711)(245.87538635,52.84017873)
\curveto(245.8853941,52.93017693)(245.90039408,53.00017686)(245.92038635,53.05017873)
\curveto(245.95039403,53.13017673)(246.01039397,53.18017668)(246.10038635,53.20017873)
\curveto(246.1803938,53.23017663)(246.27039371,53.24517662)(246.37038635,53.24517873)
\lineto(246.67038635,53.24517873)
\curveto(246.77039321,53.24517662)(246.86039312,53.2651766)(246.94038635,53.30517873)
\curveto(246.96039302,53.31517655)(246.97539301,53.32517654)(246.98538635,53.33517873)
\lineto(247.03038635,53.38017873)
\curveto(247.03039295,53.49017637)(246.985393,53.58017628)(246.89538635,53.65017873)
\curveto(246.79539319,53.72017614)(246.71539327,53.78017608)(246.65538635,53.83017873)
\lineto(246.56538635,53.92017873)
\curveto(246.45539353,54.01017585)(246.34039364,54.13517573)(246.22038635,54.29517873)
\curveto(246.10039388,54.45517541)(246.01039397,54.60517526)(245.95038635,54.74517873)
\curveto(245.90039408,54.83517503)(245.86539412,54.93017493)(245.84538635,55.03017873)
\curveto(245.81539417,55.13017473)(245.7853942,55.23517463)(245.75538635,55.34517873)
\curveto(245.74539424,55.40517446)(245.74039424,55.4651744)(245.74038635,55.52517873)
\curveto(245.73039425,55.58517428)(245.72039426,55.64017422)(245.71038635,55.69017873)
}
}
{
\newrgbcolor{curcolor}{0 0 0}
\pscustom[linestyle=none,fillstyle=solid,fillcolor=curcolor]
{
}
}
{
\newrgbcolor{curcolor}{0 0 0}
\pscustom[linestyle=none,fillstyle=solid,fillcolor=curcolor]
{
\newpath
\moveto(243.01038635,64.24510061)
\curveto(243.00039698,64.93509597)(243.12039686,65.53509537)(243.37038635,66.04510061)
\curveto(243.62039636,66.56509434)(243.95539603,66.96009395)(244.37538635,67.23010061)
\curveto(244.45539553,67.28009363)(244.54539544,67.32509358)(244.64538635,67.36510061)
\curveto(244.73539525,67.4050935)(244.83039515,67.45009346)(244.93038635,67.50010061)
\curveto(245.03039495,67.54009337)(245.13039485,67.57009334)(245.23038635,67.59010061)
\curveto(245.33039465,67.6100933)(245.43539455,67.63009328)(245.54538635,67.65010061)
\curveto(245.59539439,67.67009324)(245.64039434,67.67509323)(245.68038635,67.66510061)
\curveto(245.72039426,67.65509325)(245.76539422,67.66009325)(245.81538635,67.68010061)
\curveto(245.86539412,67.69009322)(245.95039403,67.69509321)(246.07038635,67.69510061)
\curveto(246.1803938,67.69509321)(246.26539372,67.69009322)(246.32538635,67.68010061)
\curveto(246.3853936,67.66009325)(246.44539354,67.65009326)(246.50538635,67.65010061)
\curveto(246.56539342,67.66009325)(246.62539336,67.65509325)(246.68538635,67.63510061)
\curveto(246.82539316,67.59509331)(246.96039302,67.56009335)(247.09038635,67.53010061)
\curveto(247.22039276,67.50009341)(247.34539264,67.46009345)(247.46538635,67.41010061)
\curveto(247.60539238,67.35009356)(247.73039225,67.28009363)(247.84038635,67.20010061)
\curveto(247.95039203,67.13009378)(248.06039192,67.05509385)(248.17038635,66.97510061)
\lineto(248.23038635,66.91510061)
\curveto(248.25039173,66.905094)(248.27039171,66.89009402)(248.29038635,66.87010061)
\curveto(248.45039153,66.75009416)(248.59539139,66.61509429)(248.72538635,66.46510061)
\curveto(248.85539113,66.31509459)(248.980391,66.15509475)(249.10038635,65.98510061)
\curveto(249.32039066,65.67509523)(249.52539046,65.38009553)(249.71538635,65.10010061)
\curveto(249.85539013,64.87009604)(249.99038999,64.64009627)(250.12038635,64.41010061)
\curveto(250.25038973,64.19009672)(250.3853896,63.97009694)(250.52538635,63.75010061)
\curveto(250.69538929,63.50009741)(250.87538911,63.26009765)(251.06538635,63.03010061)
\curveto(251.25538873,62.8100981)(251.4803885,62.62009829)(251.74038635,62.46010061)
\curveto(251.80038818,62.42009849)(251.86038812,62.38509852)(251.92038635,62.35510061)
\curveto(251.97038801,62.32509858)(252.03538795,62.29509861)(252.11538635,62.26510061)
\curveto(252.1853878,62.24509866)(252.24538774,62.24009867)(252.29538635,62.25010061)
\curveto(252.36538762,62.27009864)(252.42038756,62.3050986)(252.46038635,62.35510061)
\curveto(252.49038749,62.4050985)(252.51038747,62.46509844)(252.52038635,62.53510061)
\lineto(252.52038635,62.77510061)
\lineto(252.52038635,63.52510061)
\lineto(252.52038635,66.33010061)
\lineto(252.52038635,66.99010061)
\curveto(252.52038746,67.08009383)(252.52538746,67.16509374)(252.53538635,67.24510061)
\curveto(252.53538745,67.32509358)(252.55538743,67.39009352)(252.59538635,67.44010061)
\curveto(252.63538735,67.49009342)(252.71038727,67.53009338)(252.82038635,67.56010061)
\curveto(252.92038706,67.60009331)(253.02038696,67.6100933)(253.12038635,67.59010061)
\lineto(253.25538635,67.59010061)
\curveto(253.32538666,67.57009334)(253.3853866,67.55009336)(253.43538635,67.53010061)
\curveto(253.4853865,67.5100934)(253.52538646,67.47509343)(253.55538635,67.42510061)
\curveto(253.59538639,67.37509353)(253.61538637,67.3050936)(253.61538635,67.21510061)
\lineto(253.61538635,66.94510061)
\lineto(253.61538635,66.04510061)
\lineto(253.61538635,62.53510061)
\lineto(253.61538635,61.47010061)
\curveto(253.61538637,61.39009952)(253.62038636,61.30009961)(253.63038635,61.20010061)
\curveto(253.63038635,61.10009981)(253.62038636,61.01509989)(253.60038635,60.94510061)
\curveto(253.53038645,60.73510017)(253.35038663,60.67010024)(253.06038635,60.75010061)
\curveto(253.02038696,60.76010015)(252.985387,60.76010015)(252.95538635,60.75010061)
\curveto(252.91538707,60.75010016)(252.87038711,60.76010015)(252.82038635,60.78010061)
\curveto(252.74038724,60.80010011)(252.65538733,60.82010009)(252.56538635,60.84010061)
\curveto(252.47538751,60.86010005)(252.39038759,60.88510002)(252.31038635,60.91510061)
\curveto(251.82038816,61.07509983)(251.40538858,61.27509963)(251.06538635,61.51510061)
\curveto(250.81538917,61.69509921)(250.59038939,61.90009901)(250.39038635,62.13010061)
\curveto(250.1803898,62.36009855)(249.98539,62.60009831)(249.80538635,62.85010061)
\curveto(249.62539036,63.1100978)(249.45539053,63.37509753)(249.29538635,63.64510061)
\curveto(249.12539086,63.92509698)(248.95039103,64.19509671)(248.77038635,64.45510061)
\curveto(248.69039129,64.56509634)(248.61539137,64.67009624)(248.54538635,64.77010061)
\curveto(248.47539151,64.88009603)(248.40039158,64.99009592)(248.32038635,65.10010061)
\curveto(248.29039169,65.14009577)(248.26039172,65.17509573)(248.23038635,65.20510061)
\curveto(248.19039179,65.24509566)(248.16039182,65.28509562)(248.14038635,65.32510061)
\curveto(248.03039195,65.46509544)(247.90539208,65.59009532)(247.76538635,65.70010061)
\curveto(247.73539225,65.72009519)(247.71039227,65.74509516)(247.69038635,65.77510061)
\curveto(247.66039232,65.8050951)(247.63039235,65.83009508)(247.60038635,65.85010061)
\curveto(247.50039248,65.93009498)(247.40039258,65.99509491)(247.30038635,66.04510061)
\curveto(247.20039278,66.1050948)(247.09039289,66.16009475)(246.97038635,66.21010061)
\curveto(246.90039308,66.24009467)(246.82539316,66.26009465)(246.74538635,66.27010061)
\lineto(246.50538635,66.33010061)
\lineto(246.41538635,66.33010061)
\curveto(246.3853936,66.34009457)(246.35539363,66.34509456)(246.32538635,66.34510061)
\curveto(246.25539373,66.36509454)(246.16039382,66.37009454)(246.04038635,66.36010061)
\curveto(245.91039407,66.36009455)(245.81039417,66.35009456)(245.74038635,66.33010061)
\curveto(245.66039432,66.3100946)(245.5853944,66.29009462)(245.51538635,66.27010061)
\curveto(245.43539455,66.26009465)(245.35539463,66.24009467)(245.27538635,66.21010061)
\curveto(245.03539495,66.10009481)(244.83539515,65.95009496)(244.67538635,65.76010061)
\curveto(244.50539548,65.58009533)(244.36539562,65.36009555)(244.25538635,65.10010061)
\curveto(244.23539575,65.03009588)(244.22039576,64.96009595)(244.21038635,64.89010061)
\curveto(244.19039579,64.82009609)(244.17039581,64.74509616)(244.15038635,64.66510061)
\curveto(244.13039585,64.58509632)(244.12039586,64.47509643)(244.12038635,64.33510061)
\curveto(244.12039586,64.2050967)(244.13039585,64.10009681)(244.15038635,64.02010061)
\curveto(244.16039582,63.96009695)(244.16539582,63.905097)(244.16538635,63.85510061)
\curveto(244.16539582,63.8050971)(244.17539581,63.75509715)(244.19538635,63.70510061)
\curveto(244.23539575,63.6050973)(244.27539571,63.5100974)(244.31538635,63.42010061)
\curveto(244.35539563,63.34009757)(244.40039558,63.26009765)(244.45038635,63.18010061)
\curveto(244.47039551,63.15009776)(244.49539549,63.12009779)(244.52538635,63.09010061)
\curveto(244.55539543,63.07009784)(244.5803954,63.04509786)(244.60038635,63.01510061)
\lineto(244.67538635,62.94010061)
\curveto(244.69539529,62.910098)(244.71539527,62.88509802)(244.73538635,62.86510061)
\lineto(244.94538635,62.71510061)
\curveto(245.00539498,62.67509823)(245.07039491,62.63009828)(245.14038635,62.58010061)
\curveto(245.23039475,62.52009839)(245.33539465,62.47009844)(245.45538635,62.43010061)
\curveto(245.56539442,62.40009851)(245.67539431,62.36509854)(245.78538635,62.32510061)
\curveto(245.89539409,62.28509862)(246.04039394,62.26009865)(246.22038635,62.25010061)
\curveto(246.39039359,62.24009867)(246.51539347,62.2100987)(246.59538635,62.16010061)
\curveto(246.67539331,62.1100988)(246.72039326,62.03509887)(246.73038635,61.93510061)
\curveto(246.74039324,61.83509907)(246.74539324,61.72509918)(246.74538635,61.60510061)
\curveto(246.74539324,61.56509934)(246.75039323,61.52509938)(246.76038635,61.48510061)
\curveto(246.76039322,61.44509946)(246.75539323,61.4100995)(246.74538635,61.38010061)
\curveto(246.72539326,61.33009958)(246.71539327,61.28009963)(246.71538635,61.23010061)
\curveto(246.71539327,61.19009972)(246.70539328,61.15009976)(246.68538635,61.11010061)
\curveto(246.62539336,61.02009989)(246.49039349,60.97509993)(246.28038635,60.97510061)
\lineto(246.16038635,60.97510061)
\curveto(246.10039388,60.98509992)(246.04039394,60.99009992)(245.98038635,60.99010061)
\curveto(245.91039407,61.00009991)(245.84539414,61.0100999)(245.78538635,61.02010061)
\curveto(245.67539431,61.04009987)(245.57539441,61.06009985)(245.48538635,61.08010061)
\curveto(245.3853946,61.10009981)(245.29039469,61.13009978)(245.20038635,61.17010061)
\curveto(245.13039485,61.19009972)(245.07039491,61.2100997)(245.02038635,61.23010061)
\lineto(244.84038635,61.29010061)
\curveto(244.5803954,61.4100995)(244.33539565,61.56509934)(244.10538635,61.75510061)
\curveto(243.87539611,61.95509895)(243.69039629,62.17009874)(243.55038635,62.40010061)
\curveto(243.47039651,62.5100984)(243.40539658,62.62509828)(243.35538635,62.74510061)
\lineto(243.20538635,63.13510061)
\curveto(243.15539683,63.24509766)(243.12539686,63.36009755)(243.11538635,63.48010061)
\curveto(243.09539689,63.60009731)(243.07039691,63.72509718)(243.04038635,63.85510061)
\curveto(243.04039694,63.92509698)(243.04039694,63.99009692)(243.04038635,64.05010061)
\curveto(243.03039695,64.1100968)(243.02039696,64.17509673)(243.01038635,64.24510061)
}
}
{
\newrgbcolor{curcolor}{0 0 0}
\pscustom[linestyle=none,fillstyle=solid,fillcolor=curcolor]
{
\newpath
\moveto(249.29538635,76.34470998)
\curveto(249.41539057,76.37470226)(249.55539043,76.39970223)(249.71538635,76.41970998)
\curveto(249.87539011,76.43970219)(250.04038994,76.44970218)(250.21038635,76.44970998)
\curveto(250.3803896,76.44970218)(250.54538944,76.43970219)(250.70538635,76.41970998)
\curveto(250.86538912,76.39970223)(251.00538898,76.37470226)(251.12538635,76.34470998)
\curveto(251.26538872,76.30470233)(251.39038859,76.26970236)(251.50038635,76.23970998)
\curveto(251.61038837,76.20970242)(251.72038826,76.16970246)(251.83038635,76.11970998)
\curveto(252.47038751,75.84970278)(252.95538703,75.4347032)(253.28538635,74.87470998)
\curveto(253.34538664,74.79470384)(253.39538659,74.70970392)(253.43538635,74.61970998)
\curveto(253.46538652,74.5297041)(253.50038648,74.4297042)(253.54038635,74.31970998)
\curveto(253.59038639,74.20970442)(253.62538636,74.08970454)(253.64538635,73.95970998)
\curveto(253.67538631,73.83970479)(253.70538628,73.70970492)(253.73538635,73.56970998)
\curveto(253.75538623,73.50970512)(253.76038622,73.44970518)(253.75038635,73.38970998)
\curveto(253.74038624,73.33970529)(253.74538624,73.27970535)(253.76538635,73.20970998)
\curveto(253.77538621,73.18970544)(253.77538621,73.16470547)(253.76538635,73.13470998)
\curveto(253.76538622,73.10470553)(253.77038621,73.07970555)(253.78038635,73.05970998)
\lineto(253.78038635,72.90970998)
\curveto(253.79038619,72.83970579)(253.79038619,72.78970584)(253.78038635,72.75970998)
\curveto(253.77038621,72.71970591)(253.76538622,72.67470596)(253.76538635,72.62470998)
\curveto(253.77538621,72.58470605)(253.77538621,72.54470609)(253.76538635,72.50470998)
\curveto(253.74538624,72.41470622)(253.73038625,72.32470631)(253.72038635,72.23470998)
\curveto(253.72038626,72.14470649)(253.71038627,72.05470658)(253.69038635,71.96470998)
\curveto(253.66038632,71.87470676)(253.63538635,71.78470685)(253.61538635,71.69470998)
\curveto(253.59538639,71.60470703)(253.56538642,71.51970711)(253.52538635,71.43970998)
\curveto(253.41538657,71.19970743)(253.2853867,70.97470766)(253.13538635,70.76470998)
\curveto(252.97538701,70.55470808)(252.79538719,70.37470826)(252.59538635,70.22470998)
\curveto(252.42538756,70.10470853)(252.25038773,69.99970863)(252.07038635,69.90970998)
\curveto(251.89038809,69.81970881)(251.70038828,69.7297089)(251.50038635,69.63970998)
\curveto(251.40038858,69.59970903)(251.30038868,69.56470907)(251.20038635,69.53470998)
\curveto(251.09038889,69.51470912)(250.980389,69.48970914)(250.87038635,69.45970998)
\curveto(250.73038925,69.41970921)(250.59038939,69.39470924)(250.45038635,69.38470998)
\curveto(250.31038967,69.37470926)(250.17038981,69.35470928)(250.03038635,69.32470998)
\curveto(249.92039006,69.31470932)(249.82039016,69.30470933)(249.73038635,69.29470998)
\curveto(249.63039035,69.29470934)(249.53039045,69.28470935)(249.43038635,69.26470998)
\lineto(249.34038635,69.26470998)
\curveto(249.31039067,69.27470936)(249.2853907,69.27470936)(249.26538635,69.26470998)
\lineto(249.05538635,69.26470998)
\curveto(248.99539099,69.24470939)(248.93039105,69.2347094)(248.86038635,69.23470998)
\curveto(248.7803912,69.24470939)(248.70539128,69.24970938)(248.63538635,69.24970998)
\lineto(248.48538635,69.24970998)
\curveto(248.43539155,69.24970938)(248.3853916,69.25470938)(248.33538635,69.26470998)
\lineto(247.96038635,69.26470998)
\curveto(247.93039205,69.27470936)(247.89539209,69.27470936)(247.85538635,69.26470998)
\curveto(247.81539217,69.26470937)(247.77539221,69.26970936)(247.73538635,69.27970998)
\curveto(247.62539236,69.29970933)(247.51539247,69.31470932)(247.40538635,69.32470998)
\curveto(247.2853927,69.3347093)(247.17039281,69.34470929)(247.06038635,69.35470998)
\curveto(246.91039307,69.39470924)(246.76539322,69.41970921)(246.62538635,69.42970998)
\curveto(246.47539351,69.44970918)(246.33039365,69.47970915)(246.19038635,69.51970998)
\curveto(245.89039409,69.60970902)(245.60539438,69.70470893)(245.33538635,69.80470998)
\curveto(245.06539492,69.90470873)(244.81539517,70.0297086)(244.58538635,70.17970998)
\curveto(244.26539572,70.37970825)(243.985396,70.62470801)(243.74538635,70.91470998)
\curveto(243.50539648,71.20470743)(243.32039666,71.54470709)(243.19038635,71.93470998)
\curveto(243.15039683,72.04470659)(243.12539686,72.15470648)(243.11538635,72.26470998)
\curveto(243.09539689,72.38470625)(243.07039691,72.50470613)(243.04038635,72.62470998)
\curveto(243.03039695,72.69470594)(243.02539696,72.75970587)(243.02538635,72.81970998)
\curveto(243.02539696,72.87970575)(243.02039696,72.94470569)(243.01038635,73.01470998)
\curveto(242.99039699,73.71470492)(243.10539688,74.28970434)(243.35538635,74.73970998)
\curveto(243.60539638,75.18970344)(243.95539603,75.5347031)(244.40538635,75.77470998)
\curveto(244.63539535,75.88470275)(244.91039507,75.98470265)(245.23038635,76.07470998)
\curveto(245.30039468,76.09470254)(245.37539461,76.09470254)(245.45538635,76.07470998)
\curveto(245.52539446,76.06470257)(245.57539441,76.03970259)(245.60538635,75.99970998)
\curveto(245.63539435,75.96970266)(245.66039432,75.90970272)(245.68038635,75.81970998)
\curveto(245.69039429,75.7297029)(245.70039428,75.629703)(245.71038635,75.51970998)
\curveto(245.71039427,75.41970321)(245.70539428,75.31970331)(245.69538635,75.21970998)
\curveto(245.6853943,75.1297035)(245.66539432,75.06470357)(245.63538635,75.02470998)
\curveto(245.56539442,74.91470372)(245.45539453,74.8347038)(245.30538635,74.78470998)
\curveto(245.15539483,74.74470389)(245.02539496,74.68970394)(244.91538635,74.61970998)
\curveto(244.60539538,74.4297042)(244.37539561,74.14970448)(244.22538635,73.77970998)
\curveto(244.19539579,73.70970492)(244.17539581,73.634705)(244.16538635,73.55470998)
\curveto(244.15539583,73.48470515)(244.14039584,73.40970522)(244.12038635,73.32970998)
\curveto(244.11039587,73.27970535)(244.10539588,73.20970542)(244.10538635,73.11970998)
\curveto(244.10539588,73.03970559)(244.11039587,72.97470566)(244.12038635,72.92470998)
\curveto(244.14039584,72.88470575)(244.14539584,72.84970578)(244.13538635,72.81970998)
\curveto(244.12539586,72.78970584)(244.12539586,72.75470588)(244.13538635,72.71470998)
\lineto(244.19538635,72.47470998)
\curveto(244.21539577,72.40470623)(244.24039574,72.3347063)(244.27038635,72.26470998)
\curveto(244.43039555,71.88470675)(244.64039534,71.59470704)(244.90038635,71.39470998)
\curveto(245.16039482,71.20470743)(245.47539451,71.0297076)(245.84538635,70.86970998)
\curveto(245.92539406,70.83970779)(246.00539398,70.81470782)(246.08538635,70.79470998)
\curveto(246.16539382,70.78470785)(246.24539374,70.76470787)(246.32538635,70.73470998)
\curveto(246.43539355,70.70470793)(246.55039343,70.67970795)(246.67038635,70.65970998)
\curveto(246.79039319,70.64970798)(246.91039307,70.629708)(247.03038635,70.59970998)
\curveto(247.0803929,70.57970805)(247.13039285,70.56970806)(247.18038635,70.56970998)
\curveto(247.23039275,70.57970805)(247.2803927,70.57470806)(247.33038635,70.55470998)
\curveto(247.39039259,70.54470809)(247.47039251,70.54470809)(247.57038635,70.55470998)
\curveto(247.66039232,70.56470807)(247.71539227,70.57970805)(247.73538635,70.59970998)
\curveto(247.77539221,70.61970801)(247.79539219,70.64970798)(247.79538635,70.68970998)
\curveto(247.79539219,70.73970789)(247.7853922,70.78470785)(247.76538635,70.82470998)
\curveto(247.72539226,70.89470774)(247.6803923,70.95470768)(247.63038635,71.00470998)
\curveto(247.5803924,71.05470758)(247.53039245,71.11470752)(247.48038635,71.18470998)
\lineto(247.42038635,71.24470998)
\curveto(247.39039259,71.27470736)(247.36539262,71.30470733)(247.34538635,71.33470998)
\curveto(247.1853928,71.56470707)(247.05039293,71.83970679)(246.94038635,72.15970998)
\curveto(246.92039306,72.2297064)(246.90539308,72.29970633)(246.89538635,72.36970998)
\curveto(246.8853931,72.43970619)(246.87039311,72.51470612)(246.85038635,72.59470998)
\curveto(246.85039313,72.634706)(246.84539314,72.66970596)(246.83538635,72.69970998)
\curveto(246.82539316,72.7297059)(246.82539316,72.76470587)(246.83538635,72.80470998)
\curveto(246.83539315,72.85470578)(246.82539316,72.89470574)(246.80538635,72.92470998)
\lineto(246.80538635,73.08970998)
\lineto(246.80538635,73.17970998)
\curveto(246.79539319,73.2297054)(246.79539319,73.26970536)(246.80538635,73.29970998)
\curveto(246.81539317,73.34970528)(246.82039316,73.39970523)(246.82038635,73.44970998)
\curveto(246.81039317,73.50970512)(246.81039317,73.56470507)(246.82038635,73.61470998)
\curveto(246.85039313,73.72470491)(246.87039311,73.8297048)(246.88038635,73.92970998)
\curveto(246.89039309,74.03970459)(246.91539307,74.14470449)(246.95538635,74.24470998)
\curveto(247.09539289,74.66470397)(247.2803927,75.00970362)(247.51038635,75.27970998)
\curveto(247.73039225,75.54970308)(248.01539197,75.78970284)(248.36538635,75.99970998)
\curveto(248.50539148,76.07970255)(248.65539133,76.14470249)(248.81538635,76.19470998)
\curveto(248.96539102,76.24470239)(249.12539086,76.29470234)(249.29538635,76.34470998)
\moveto(250.60038635,75.09970998)
\curveto(250.55038943,75.10970352)(250.50538948,75.11470352)(250.46538635,75.11470998)
\lineto(250.31538635,75.11470998)
\curveto(250.00538998,75.11470352)(249.72039026,75.07470356)(249.46038635,74.99470998)
\curveto(249.40039058,74.97470366)(249.34539064,74.95470368)(249.29538635,74.93470998)
\curveto(249.23539075,74.92470371)(249.1803908,74.90970372)(249.13038635,74.88970998)
\curveto(248.64039134,74.66970396)(248.29039169,74.32470431)(248.08038635,73.85470998)
\curveto(248.05039193,73.77470486)(248.02539196,73.69470494)(248.00538635,73.61470998)
\lineto(247.94538635,73.37470998)
\curveto(247.92539206,73.29470534)(247.91539207,73.20470543)(247.91538635,73.10470998)
\lineto(247.91538635,72.78970998)
\curveto(247.93539205,72.76970586)(247.94539204,72.7297059)(247.94538635,72.66970998)
\curveto(247.93539205,72.61970601)(247.93539205,72.57470606)(247.94538635,72.53470998)
\lineto(248.00538635,72.29470998)
\curveto(248.01539197,72.22470641)(248.03539195,72.15470648)(248.06538635,72.08470998)
\curveto(248.32539166,71.48470715)(248.79039119,71.07970755)(249.46038635,70.86970998)
\curveto(249.54039044,70.83970779)(249.62039036,70.81970781)(249.70038635,70.80970998)
\curveto(249.7803902,70.79970783)(249.86539012,70.78470785)(249.95538635,70.76470998)
\lineto(250.10538635,70.76470998)
\curveto(250.14538984,70.75470788)(250.21538977,70.74970788)(250.31538635,70.74970998)
\curveto(250.54538944,70.74970788)(250.74038924,70.76970786)(250.90038635,70.80970998)
\curveto(250.97038901,70.8297078)(251.03538895,70.84470779)(251.09538635,70.85470998)
\curveto(251.15538883,70.86470777)(251.22038876,70.88470775)(251.29038635,70.91470998)
\curveto(251.57038841,71.02470761)(251.81538817,71.16970746)(252.02538635,71.34970998)
\curveto(252.22538776,71.5297071)(252.3853876,71.76470687)(252.50538635,72.05470998)
\lineto(252.59538635,72.29470998)
\lineto(252.65538635,72.53470998)
\curveto(252.67538731,72.58470605)(252.6803873,72.62470601)(252.67038635,72.65470998)
\curveto(252.66038732,72.69470594)(252.66538732,72.73970589)(252.68538635,72.78970998)
\curveto(252.69538729,72.81970581)(252.70038728,72.87470576)(252.70038635,72.95470998)
\curveto(252.70038728,73.0347056)(252.69538729,73.09470554)(252.68538635,73.13470998)
\curveto(252.66538732,73.24470539)(252.65038733,73.34970528)(252.64038635,73.44970998)
\curveto(252.63038735,73.54970508)(252.60038738,73.64470499)(252.55038635,73.73470998)
\curveto(252.35038763,74.26470437)(251.97538801,74.65470398)(251.42538635,74.90470998)
\curveto(251.32538866,74.94470369)(251.22038876,74.97470366)(251.11038635,74.99470998)
\lineto(250.78038635,75.08470998)
\curveto(250.70038928,75.08470355)(250.64038934,75.08970354)(250.60038635,75.09970998)
}
}
{
\newrgbcolor{curcolor}{0 0 0}
\pscustom[linestyle=none,fillstyle=solid,fillcolor=curcolor]
{
\newpath
\moveto(251.98038635,78.63431936)
\lineto(251.98038635,79.26431936)
\lineto(251.98038635,79.45931936)
\curveto(251.980388,79.52931683)(251.99038799,79.58931677)(252.01038635,79.63931936)
\curveto(252.05038793,79.70931665)(252.09038789,79.7593166)(252.13038635,79.78931936)
\curveto(252.1803878,79.82931653)(252.24538774,79.84931651)(252.32538635,79.84931936)
\curveto(252.40538758,79.8593165)(252.49038749,79.86431649)(252.58038635,79.86431936)
\lineto(253.30038635,79.86431936)
\curveto(253.7803862,79.86431649)(254.19038579,79.80431655)(254.53038635,79.68431936)
\curveto(254.87038511,79.56431679)(255.14538484,79.36931699)(255.35538635,79.09931936)
\curveto(255.40538458,79.02931733)(255.45038453,78.9593174)(255.49038635,78.88931936)
\curveto(255.54038444,78.82931753)(255.5853844,78.7543176)(255.62538635,78.66431936)
\curveto(255.63538435,78.64431771)(255.64538434,78.61431774)(255.65538635,78.57431936)
\curveto(255.67538431,78.53431782)(255.6803843,78.48931787)(255.67038635,78.43931936)
\curveto(255.64038434,78.34931801)(255.56538442,78.29431806)(255.44538635,78.27431936)
\curveto(255.33538465,78.2543181)(255.24038474,78.26931809)(255.16038635,78.31931936)
\curveto(255.09038489,78.34931801)(255.02538496,78.39431796)(254.96538635,78.45431936)
\curveto(254.91538507,78.52431783)(254.86538512,78.58931777)(254.81538635,78.64931936)
\curveto(254.76538522,78.71931764)(254.69038529,78.77931758)(254.59038635,78.82931936)
\curveto(254.50038548,78.88931747)(254.41038557,78.93931742)(254.32038635,78.97931936)
\curveto(254.29038569,78.99931736)(254.23038575,79.02431733)(254.14038635,79.05431936)
\curveto(254.06038592,79.08431727)(253.99038599,79.08931727)(253.93038635,79.06931936)
\curveto(253.79038619,79.03931732)(253.70038628,78.97931738)(253.66038635,78.88931936)
\curveto(253.63038635,78.80931755)(253.61538637,78.71931764)(253.61538635,78.61931936)
\curveto(253.61538637,78.51931784)(253.59038639,78.43431792)(253.54038635,78.36431936)
\curveto(253.47038651,78.27431808)(253.33038665,78.22931813)(253.12038635,78.22931936)
\lineto(252.56538635,78.22931936)
\lineto(252.34038635,78.22931936)
\curveto(252.26038772,78.23931812)(252.19538779,78.2593181)(252.14538635,78.28931936)
\curveto(252.06538792,78.34931801)(252.02038796,78.41931794)(252.01038635,78.49931936)
\curveto(252.00038798,78.51931784)(251.99538799,78.53931782)(251.99538635,78.55931936)
\curveto(251.99538799,78.58931777)(251.99038799,78.61431774)(251.98038635,78.63431936)
}
}
{
\newrgbcolor{curcolor}{0 0 0}
\pscustom[linestyle=none,fillstyle=solid,fillcolor=curcolor]
{
}
}
{
\newrgbcolor{curcolor}{0 0 0}
\pscustom[linestyle=none,fillstyle=solid,fillcolor=curcolor]
{
\newpath
\moveto(243.01038635,89.26463186)
\curveto(243.00039698,89.95462722)(243.12039686,90.55462662)(243.37038635,91.06463186)
\curveto(243.62039636,91.58462559)(243.95539603,91.9796252)(244.37538635,92.24963186)
\curveto(244.45539553,92.29962488)(244.54539544,92.34462483)(244.64538635,92.38463186)
\curveto(244.73539525,92.42462475)(244.83039515,92.46962471)(244.93038635,92.51963186)
\curveto(245.03039495,92.55962462)(245.13039485,92.58962459)(245.23038635,92.60963186)
\curveto(245.33039465,92.62962455)(245.43539455,92.64962453)(245.54538635,92.66963186)
\curveto(245.59539439,92.68962449)(245.64039434,92.69462448)(245.68038635,92.68463186)
\curveto(245.72039426,92.6746245)(245.76539422,92.6796245)(245.81538635,92.69963186)
\curveto(245.86539412,92.70962447)(245.95039403,92.71462446)(246.07038635,92.71463186)
\curveto(246.1803938,92.71462446)(246.26539372,92.70962447)(246.32538635,92.69963186)
\curveto(246.3853936,92.6796245)(246.44539354,92.66962451)(246.50538635,92.66963186)
\curveto(246.56539342,92.6796245)(246.62539336,92.6746245)(246.68538635,92.65463186)
\curveto(246.82539316,92.61462456)(246.96039302,92.5796246)(247.09038635,92.54963186)
\curveto(247.22039276,92.51962466)(247.34539264,92.4796247)(247.46538635,92.42963186)
\curveto(247.60539238,92.36962481)(247.73039225,92.29962488)(247.84038635,92.21963186)
\curveto(247.95039203,92.14962503)(248.06039192,92.0746251)(248.17038635,91.99463186)
\lineto(248.23038635,91.93463186)
\curveto(248.25039173,91.92462525)(248.27039171,91.90962527)(248.29038635,91.88963186)
\curveto(248.45039153,91.76962541)(248.59539139,91.63462554)(248.72538635,91.48463186)
\curveto(248.85539113,91.33462584)(248.980391,91.174626)(249.10038635,91.00463186)
\curveto(249.32039066,90.69462648)(249.52539046,90.39962678)(249.71538635,90.11963186)
\curveto(249.85539013,89.88962729)(249.99038999,89.65962752)(250.12038635,89.42963186)
\curveto(250.25038973,89.20962797)(250.3853896,88.98962819)(250.52538635,88.76963186)
\curveto(250.69538929,88.51962866)(250.87538911,88.2796289)(251.06538635,88.04963186)
\curveto(251.25538873,87.82962935)(251.4803885,87.63962954)(251.74038635,87.47963186)
\curveto(251.80038818,87.43962974)(251.86038812,87.40462977)(251.92038635,87.37463186)
\curveto(251.97038801,87.34462983)(252.03538795,87.31462986)(252.11538635,87.28463186)
\curveto(252.1853878,87.26462991)(252.24538774,87.25962992)(252.29538635,87.26963186)
\curveto(252.36538762,87.28962989)(252.42038756,87.32462985)(252.46038635,87.37463186)
\curveto(252.49038749,87.42462975)(252.51038747,87.48462969)(252.52038635,87.55463186)
\lineto(252.52038635,87.79463186)
\lineto(252.52038635,88.54463186)
\lineto(252.52038635,91.34963186)
\lineto(252.52038635,92.00963186)
\curveto(252.52038746,92.09962508)(252.52538746,92.18462499)(252.53538635,92.26463186)
\curveto(252.53538745,92.34462483)(252.55538743,92.40962477)(252.59538635,92.45963186)
\curveto(252.63538735,92.50962467)(252.71038727,92.54962463)(252.82038635,92.57963186)
\curveto(252.92038706,92.61962456)(253.02038696,92.62962455)(253.12038635,92.60963186)
\lineto(253.25538635,92.60963186)
\curveto(253.32538666,92.58962459)(253.3853866,92.56962461)(253.43538635,92.54963186)
\curveto(253.4853865,92.52962465)(253.52538646,92.49462468)(253.55538635,92.44463186)
\curveto(253.59538639,92.39462478)(253.61538637,92.32462485)(253.61538635,92.23463186)
\lineto(253.61538635,91.96463186)
\lineto(253.61538635,91.06463186)
\lineto(253.61538635,87.55463186)
\lineto(253.61538635,86.48963186)
\curveto(253.61538637,86.40963077)(253.62038636,86.31963086)(253.63038635,86.21963186)
\curveto(253.63038635,86.11963106)(253.62038636,86.03463114)(253.60038635,85.96463186)
\curveto(253.53038645,85.75463142)(253.35038663,85.68963149)(253.06038635,85.76963186)
\curveto(253.02038696,85.7796314)(252.985387,85.7796314)(252.95538635,85.76963186)
\curveto(252.91538707,85.76963141)(252.87038711,85.7796314)(252.82038635,85.79963186)
\curveto(252.74038724,85.81963136)(252.65538733,85.83963134)(252.56538635,85.85963186)
\curveto(252.47538751,85.8796313)(252.39038759,85.90463127)(252.31038635,85.93463186)
\curveto(251.82038816,86.09463108)(251.40538858,86.29463088)(251.06538635,86.53463186)
\curveto(250.81538917,86.71463046)(250.59038939,86.91963026)(250.39038635,87.14963186)
\curveto(250.1803898,87.3796298)(249.98539,87.61962956)(249.80538635,87.86963186)
\curveto(249.62539036,88.12962905)(249.45539053,88.39462878)(249.29538635,88.66463186)
\curveto(249.12539086,88.94462823)(248.95039103,89.21462796)(248.77038635,89.47463186)
\curveto(248.69039129,89.58462759)(248.61539137,89.68962749)(248.54538635,89.78963186)
\curveto(248.47539151,89.89962728)(248.40039158,90.00962717)(248.32038635,90.11963186)
\curveto(248.29039169,90.15962702)(248.26039172,90.19462698)(248.23038635,90.22463186)
\curveto(248.19039179,90.26462691)(248.16039182,90.30462687)(248.14038635,90.34463186)
\curveto(248.03039195,90.48462669)(247.90539208,90.60962657)(247.76538635,90.71963186)
\curveto(247.73539225,90.73962644)(247.71039227,90.76462641)(247.69038635,90.79463186)
\curveto(247.66039232,90.82462635)(247.63039235,90.84962633)(247.60038635,90.86963186)
\curveto(247.50039248,90.94962623)(247.40039258,91.01462616)(247.30038635,91.06463186)
\curveto(247.20039278,91.12462605)(247.09039289,91.179626)(246.97038635,91.22963186)
\curveto(246.90039308,91.25962592)(246.82539316,91.2796259)(246.74538635,91.28963186)
\lineto(246.50538635,91.34963186)
\lineto(246.41538635,91.34963186)
\curveto(246.3853936,91.35962582)(246.35539363,91.36462581)(246.32538635,91.36463186)
\curveto(246.25539373,91.38462579)(246.16039382,91.38962579)(246.04038635,91.37963186)
\curveto(245.91039407,91.3796258)(245.81039417,91.36962581)(245.74038635,91.34963186)
\curveto(245.66039432,91.32962585)(245.5853944,91.30962587)(245.51538635,91.28963186)
\curveto(245.43539455,91.2796259)(245.35539463,91.25962592)(245.27538635,91.22963186)
\curveto(245.03539495,91.11962606)(244.83539515,90.96962621)(244.67538635,90.77963186)
\curveto(244.50539548,90.59962658)(244.36539562,90.3796268)(244.25538635,90.11963186)
\curveto(244.23539575,90.04962713)(244.22039576,89.9796272)(244.21038635,89.90963186)
\curveto(244.19039579,89.83962734)(244.17039581,89.76462741)(244.15038635,89.68463186)
\curveto(244.13039585,89.60462757)(244.12039586,89.49462768)(244.12038635,89.35463186)
\curveto(244.12039586,89.22462795)(244.13039585,89.11962806)(244.15038635,89.03963186)
\curveto(244.16039582,88.9796282)(244.16539582,88.92462825)(244.16538635,88.87463186)
\curveto(244.16539582,88.82462835)(244.17539581,88.7746284)(244.19538635,88.72463186)
\curveto(244.23539575,88.62462855)(244.27539571,88.52962865)(244.31538635,88.43963186)
\curveto(244.35539563,88.35962882)(244.40039558,88.2796289)(244.45038635,88.19963186)
\curveto(244.47039551,88.16962901)(244.49539549,88.13962904)(244.52538635,88.10963186)
\curveto(244.55539543,88.08962909)(244.5803954,88.06462911)(244.60038635,88.03463186)
\lineto(244.67538635,87.95963186)
\curveto(244.69539529,87.92962925)(244.71539527,87.90462927)(244.73538635,87.88463186)
\lineto(244.94538635,87.73463186)
\curveto(245.00539498,87.69462948)(245.07039491,87.64962953)(245.14038635,87.59963186)
\curveto(245.23039475,87.53962964)(245.33539465,87.48962969)(245.45538635,87.44963186)
\curveto(245.56539442,87.41962976)(245.67539431,87.38462979)(245.78538635,87.34463186)
\curveto(245.89539409,87.30462987)(246.04039394,87.2796299)(246.22038635,87.26963186)
\curveto(246.39039359,87.25962992)(246.51539347,87.22962995)(246.59538635,87.17963186)
\curveto(246.67539331,87.12963005)(246.72039326,87.05463012)(246.73038635,86.95463186)
\curveto(246.74039324,86.85463032)(246.74539324,86.74463043)(246.74538635,86.62463186)
\curveto(246.74539324,86.58463059)(246.75039323,86.54463063)(246.76038635,86.50463186)
\curveto(246.76039322,86.46463071)(246.75539323,86.42963075)(246.74538635,86.39963186)
\curveto(246.72539326,86.34963083)(246.71539327,86.29963088)(246.71538635,86.24963186)
\curveto(246.71539327,86.20963097)(246.70539328,86.16963101)(246.68538635,86.12963186)
\curveto(246.62539336,86.03963114)(246.49039349,85.99463118)(246.28038635,85.99463186)
\lineto(246.16038635,85.99463186)
\curveto(246.10039388,86.00463117)(246.04039394,86.00963117)(245.98038635,86.00963186)
\curveto(245.91039407,86.01963116)(245.84539414,86.02963115)(245.78538635,86.03963186)
\curveto(245.67539431,86.05963112)(245.57539441,86.0796311)(245.48538635,86.09963186)
\curveto(245.3853946,86.11963106)(245.29039469,86.14963103)(245.20038635,86.18963186)
\curveto(245.13039485,86.20963097)(245.07039491,86.22963095)(245.02038635,86.24963186)
\lineto(244.84038635,86.30963186)
\curveto(244.5803954,86.42963075)(244.33539565,86.58463059)(244.10538635,86.77463186)
\curveto(243.87539611,86.9746302)(243.69039629,87.18962999)(243.55038635,87.41963186)
\curveto(243.47039651,87.52962965)(243.40539658,87.64462953)(243.35538635,87.76463186)
\lineto(243.20538635,88.15463186)
\curveto(243.15539683,88.26462891)(243.12539686,88.3796288)(243.11538635,88.49963186)
\curveto(243.09539689,88.61962856)(243.07039691,88.74462843)(243.04038635,88.87463186)
\curveto(243.04039694,88.94462823)(243.04039694,89.00962817)(243.04038635,89.06963186)
\curveto(243.03039695,89.12962805)(243.02039696,89.19462798)(243.01038635,89.26463186)
}
}
{
\newrgbcolor{curcolor}{0 0 0}
\pscustom[linestyle=none,fillstyle=solid,fillcolor=curcolor]
{
\newpath
\moveto(248.53038635,101.36424123)
\lineto(248.78538635,101.36424123)
\curveto(248.86539112,101.37423353)(248.94039104,101.36923353)(249.01038635,101.34924123)
\lineto(249.25038635,101.34924123)
\lineto(249.41538635,101.34924123)
\curveto(249.51539047,101.32923357)(249.62039036,101.31923358)(249.73038635,101.31924123)
\curveto(249.83039015,101.31923358)(249.93039005,101.30923359)(250.03038635,101.28924123)
\lineto(250.18038635,101.28924123)
\curveto(250.32038966,101.25923364)(250.46038952,101.23923366)(250.60038635,101.22924123)
\curveto(250.73038925,101.21923368)(250.86038912,101.19423371)(250.99038635,101.15424123)
\curveto(251.07038891,101.13423377)(251.15538883,101.11423379)(251.24538635,101.09424123)
\lineto(251.48538635,101.03424123)
\lineto(251.78538635,100.91424123)
\curveto(251.87538811,100.88423402)(251.96538802,100.84923405)(252.05538635,100.80924123)
\curveto(252.27538771,100.70923419)(252.49038749,100.57423433)(252.70038635,100.40424123)
\curveto(252.91038707,100.24423466)(253.0803869,100.06923483)(253.21038635,99.87924123)
\curveto(253.25038673,99.82923507)(253.29038669,99.76923513)(253.33038635,99.69924123)
\curveto(253.36038662,99.63923526)(253.39538659,99.57923532)(253.43538635,99.51924123)
\curveto(253.4853865,99.43923546)(253.52538646,99.34423556)(253.55538635,99.23424123)
\curveto(253.5853864,99.12423578)(253.61538637,99.01923588)(253.64538635,98.91924123)
\curveto(253.6853863,98.80923609)(253.71038627,98.6992362)(253.72038635,98.58924123)
\curveto(253.73038625,98.47923642)(253.74538624,98.36423654)(253.76538635,98.24424123)
\curveto(253.77538621,98.2042367)(253.77538621,98.15923674)(253.76538635,98.10924123)
\curveto(253.76538622,98.06923683)(253.77038621,98.02923687)(253.78038635,97.98924123)
\curveto(253.79038619,97.94923695)(253.79538619,97.89423701)(253.79538635,97.82424123)
\curveto(253.79538619,97.75423715)(253.79038619,97.7042372)(253.78038635,97.67424123)
\curveto(253.76038622,97.62423728)(253.75538623,97.57923732)(253.76538635,97.53924123)
\curveto(253.77538621,97.4992374)(253.77538621,97.46423744)(253.76538635,97.43424123)
\lineto(253.76538635,97.34424123)
\curveto(253.74538624,97.28423762)(253.73038625,97.21923768)(253.72038635,97.14924123)
\curveto(253.72038626,97.08923781)(253.71538627,97.02423788)(253.70538635,96.95424123)
\curveto(253.65538633,96.78423812)(253.60538638,96.62423828)(253.55538635,96.47424123)
\curveto(253.50538648,96.32423858)(253.44038654,96.17923872)(253.36038635,96.03924123)
\curveto(253.32038666,95.98923891)(253.29038669,95.93423897)(253.27038635,95.87424123)
\curveto(253.24038674,95.82423908)(253.20538678,95.77423913)(253.16538635,95.72424123)
\curveto(252.985387,95.48423942)(252.76538722,95.28423962)(252.50538635,95.12424123)
\curveto(252.24538774,94.96423994)(251.96038802,94.82424008)(251.65038635,94.70424123)
\curveto(251.51038847,94.64424026)(251.37038861,94.5992403)(251.23038635,94.56924123)
\curveto(251.0803889,94.53924036)(250.92538906,94.5042404)(250.76538635,94.46424123)
\curveto(250.65538933,94.44424046)(250.54538944,94.42924047)(250.43538635,94.41924123)
\curveto(250.32538966,94.40924049)(250.21538977,94.39424051)(250.10538635,94.37424123)
\curveto(250.06538992,94.36424054)(250.02538996,94.35924054)(249.98538635,94.35924123)
\curveto(249.94539004,94.36924053)(249.90539008,94.36924053)(249.86538635,94.35924123)
\curveto(249.81539017,94.34924055)(249.76539022,94.34424056)(249.71538635,94.34424123)
\lineto(249.55038635,94.34424123)
\curveto(249.50039048,94.32424058)(249.45039053,94.31924058)(249.40038635,94.32924123)
\curveto(249.34039064,94.33924056)(249.2853907,94.33924056)(249.23538635,94.32924123)
\curveto(249.19539079,94.31924058)(249.15039083,94.31924058)(249.10038635,94.32924123)
\curveto(249.05039093,94.33924056)(249.00039098,94.33424057)(248.95038635,94.31424123)
\curveto(248.8803911,94.29424061)(248.80539118,94.28924061)(248.72538635,94.29924123)
\curveto(248.63539135,94.30924059)(248.55039143,94.31424059)(248.47038635,94.31424123)
\curveto(248.3803916,94.31424059)(248.2803917,94.30924059)(248.17038635,94.29924123)
\curveto(248.05039193,94.28924061)(247.95039203,94.29424061)(247.87038635,94.31424123)
\lineto(247.58538635,94.31424123)
\lineto(246.95538635,94.35924123)
\curveto(246.85539313,94.36924053)(246.76039322,94.37924052)(246.67038635,94.38924123)
\lineto(246.37038635,94.41924123)
\curveto(246.32039366,94.43924046)(246.27039371,94.44424046)(246.22038635,94.43424123)
\curveto(246.16039382,94.43424047)(246.10539388,94.44424046)(246.05538635,94.46424123)
\curveto(245.8853941,94.51424039)(245.72039426,94.55424035)(245.56038635,94.58424123)
\curveto(245.39039459,94.61424029)(245.23039475,94.66424024)(245.08038635,94.73424123)
\curveto(244.62039536,94.92423998)(244.24539574,95.14423976)(243.95538635,95.39424123)
\curveto(243.66539632,95.65423925)(243.42039656,96.01423889)(243.22038635,96.47424123)
\curveto(243.17039681,96.6042383)(243.13539685,96.73423817)(243.11538635,96.86424123)
\curveto(243.09539689,97.0042379)(243.07039691,97.14423776)(243.04038635,97.28424123)
\curveto(243.03039695,97.35423755)(243.02539696,97.41923748)(243.02538635,97.47924123)
\curveto(243.02539696,97.53923736)(243.02039696,97.6042373)(243.01038635,97.67424123)
\curveto(242.99039699,98.5042364)(243.14039684,99.17423573)(243.46038635,99.68424123)
\curveto(243.77039621,100.19423471)(244.21039577,100.57423433)(244.78038635,100.82424123)
\curveto(244.90039508,100.87423403)(245.02539496,100.91923398)(245.15538635,100.95924123)
\curveto(245.2853947,100.9992339)(245.42039456,101.04423386)(245.56038635,101.09424123)
\curveto(245.64039434,101.11423379)(245.72539426,101.12923377)(245.81538635,101.13924123)
\lineto(246.05538635,101.19924123)
\curveto(246.16539382,101.22923367)(246.27539371,101.24423366)(246.38538635,101.24424123)
\curveto(246.49539349,101.25423365)(246.60539338,101.26923363)(246.71538635,101.28924123)
\curveto(246.76539322,101.30923359)(246.81039317,101.31423359)(246.85038635,101.30424123)
\curveto(246.89039309,101.3042336)(246.93039305,101.30923359)(246.97038635,101.31924123)
\curveto(247.02039296,101.32923357)(247.07539291,101.32923357)(247.13538635,101.31924123)
\curveto(247.1853928,101.31923358)(247.23539275,101.32423358)(247.28538635,101.33424123)
\lineto(247.42038635,101.33424123)
\curveto(247.4803925,101.35423355)(247.55039243,101.35423355)(247.63038635,101.33424123)
\curveto(247.70039228,101.32423358)(247.76539222,101.32923357)(247.82538635,101.34924123)
\curveto(247.85539213,101.35923354)(247.89539209,101.36423354)(247.94538635,101.36424123)
\lineto(248.06538635,101.36424123)
\lineto(248.53038635,101.36424123)
\moveto(250.85538635,99.81924123)
\curveto(250.53538945,99.91923498)(250.17038981,99.97923492)(249.76038635,99.99924123)
\curveto(249.35039063,100.01923488)(248.94039104,100.02923487)(248.53038635,100.02924123)
\curveto(248.10039188,100.02923487)(247.6803923,100.01923488)(247.27038635,99.99924123)
\curveto(246.86039312,99.97923492)(246.47539351,99.93423497)(246.11538635,99.86424123)
\curveto(245.75539423,99.79423511)(245.43539455,99.68423522)(245.15538635,99.53424123)
\curveto(244.86539512,99.39423551)(244.63039535,99.1992357)(244.45038635,98.94924123)
\curveto(244.34039564,98.78923611)(244.26039572,98.60923629)(244.21038635,98.40924123)
\curveto(244.15039583,98.20923669)(244.12039586,97.96423694)(244.12038635,97.67424123)
\curveto(244.14039584,97.65423725)(244.15039583,97.61923728)(244.15038635,97.56924123)
\curveto(244.14039584,97.51923738)(244.14039584,97.47923742)(244.15038635,97.44924123)
\curveto(244.17039581,97.36923753)(244.19039579,97.29423761)(244.21038635,97.22424123)
\curveto(244.22039576,97.16423774)(244.24039574,97.0992378)(244.27038635,97.02924123)
\curveto(244.39039559,96.75923814)(244.56039542,96.53923836)(244.78038635,96.36924123)
\curveto(244.99039499,96.20923869)(245.23539475,96.07423883)(245.51538635,95.96424123)
\curveto(245.62539436,95.91423899)(245.74539424,95.87423903)(245.87538635,95.84424123)
\curveto(245.99539399,95.82423908)(246.12039386,95.7992391)(246.25038635,95.76924123)
\curveto(246.30039368,95.74923915)(246.35539363,95.73923916)(246.41538635,95.73924123)
\curveto(246.46539352,95.73923916)(246.51539347,95.73423917)(246.56538635,95.72424123)
\curveto(246.65539333,95.71423919)(246.75039323,95.7042392)(246.85038635,95.69424123)
\curveto(246.94039304,95.68423922)(247.03539295,95.67423923)(247.13538635,95.66424123)
\curveto(247.21539277,95.66423924)(247.30039268,95.65923924)(247.39038635,95.64924123)
\lineto(247.63038635,95.64924123)
\lineto(247.81038635,95.64924123)
\curveto(247.84039214,95.63923926)(247.87539211,95.63423927)(247.91538635,95.63424123)
\lineto(248.05038635,95.63424123)
\lineto(248.50038635,95.63424123)
\curveto(248.5803914,95.63423927)(248.66539132,95.62923927)(248.75538635,95.61924123)
\curveto(248.83539115,95.61923928)(248.91039107,95.62923927)(248.98038635,95.64924123)
\lineto(249.25038635,95.64924123)
\curveto(249.27039071,95.64923925)(249.30039068,95.64423926)(249.34038635,95.63424123)
\curveto(249.37039061,95.63423927)(249.39539059,95.63923926)(249.41538635,95.64924123)
\curveto(249.51539047,95.65923924)(249.61539037,95.66423924)(249.71538635,95.66424123)
\curveto(249.80539018,95.67423923)(249.90539008,95.68423922)(250.01538635,95.69424123)
\curveto(250.13538985,95.72423918)(250.26038972,95.73923916)(250.39038635,95.73924123)
\curveto(250.51038947,95.74923915)(250.62538936,95.77423913)(250.73538635,95.81424123)
\curveto(251.03538895,95.89423901)(251.30038868,95.97923892)(251.53038635,96.06924123)
\curveto(251.76038822,96.16923873)(251.97538801,96.31423859)(252.17538635,96.50424123)
\curveto(252.37538761,96.71423819)(252.52538746,96.97923792)(252.62538635,97.29924123)
\curveto(252.64538734,97.33923756)(252.65538733,97.37423753)(252.65538635,97.40424123)
\curveto(252.64538734,97.44423746)(252.65038733,97.48923741)(252.67038635,97.53924123)
\curveto(252.6803873,97.57923732)(252.69038729,97.64923725)(252.70038635,97.74924123)
\curveto(252.71038727,97.85923704)(252.70538728,97.94423696)(252.68538635,98.00424123)
\curveto(252.66538732,98.07423683)(252.65538733,98.14423676)(252.65538635,98.21424123)
\curveto(252.64538734,98.28423662)(252.63038735,98.34923655)(252.61038635,98.40924123)
\curveto(252.55038743,98.60923629)(252.46538752,98.78923611)(252.35538635,98.94924123)
\curveto(252.33538765,98.97923592)(252.31538767,99.0042359)(252.29538635,99.02424123)
\lineto(252.23538635,99.08424123)
\curveto(252.21538777,99.12423578)(252.17538781,99.17423573)(252.11538635,99.23424123)
\curveto(251.97538801,99.33423557)(251.84538814,99.41923548)(251.72538635,99.48924123)
\curveto(251.60538838,99.55923534)(251.46038852,99.62923527)(251.29038635,99.69924123)
\curveto(251.22038876,99.72923517)(251.15038883,99.74923515)(251.08038635,99.75924123)
\curveto(251.01038897,99.77923512)(250.93538905,99.7992351)(250.85538635,99.81924123)
}
}
{
\newrgbcolor{curcolor}{0 0 0}
\pscustom[linestyle=none,fillstyle=solid,fillcolor=curcolor]
{
\newpath
\moveto(243.01038635,106.77385061)
\curveto(243.01039697,106.87384575)(243.02039696,106.96884566)(243.04038635,107.05885061)
\curveto(243.05039693,107.14884548)(243.0803969,107.21384541)(243.13038635,107.25385061)
\curveto(243.21039677,107.31384531)(243.31539667,107.34384528)(243.44538635,107.34385061)
\lineto(243.83538635,107.34385061)
\lineto(245.33538635,107.34385061)
\lineto(251.72538635,107.34385061)
\lineto(252.89538635,107.34385061)
\lineto(253.21038635,107.34385061)
\curveto(253.31038667,107.35384527)(253.39038659,107.33884529)(253.45038635,107.29885061)
\curveto(253.53038645,107.24884538)(253.5803864,107.17384545)(253.60038635,107.07385061)
\curveto(253.61038637,106.98384564)(253.61538637,106.87384575)(253.61538635,106.74385061)
\lineto(253.61538635,106.51885061)
\curveto(253.59538639,106.43884619)(253.5803864,106.36884626)(253.57038635,106.30885061)
\curveto(253.55038643,106.24884638)(253.51038647,106.19884643)(253.45038635,106.15885061)
\curveto(253.39038659,106.11884651)(253.31538667,106.09884653)(253.22538635,106.09885061)
\lineto(252.92538635,106.09885061)
\lineto(251.83038635,106.09885061)
\lineto(246.49038635,106.09885061)
\curveto(246.40039358,106.07884655)(246.32539366,106.06384656)(246.26538635,106.05385061)
\curveto(246.19539379,106.05384657)(246.13539385,106.0238466)(246.08538635,105.96385061)
\curveto(246.03539395,105.89384673)(246.01039397,105.80384682)(246.01038635,105.69385061)
\curveto(246.00039398,105.59384703)(245.99539399,105.48384714)(245.99538635,105.36385061)
\lineto(245.99538635,104.22385061)
\lineto(245.99538635,103.72885061)
\curveto(245.985394,103.56884906)(245.92539406,103.45884917)(245.81538635,103.39885061)
\curveto(245.7853942,103.37884925)(245.75539423,103.36884926)(245.72538635,103.36885061)
\curveto(245.6853943,103.36884926)(245.64039434,103.36384926)(245.59038635,103.35385061)
\curveto(245.47039451,103.33384929)(245.36039462,103.33884929)(245.26038635,103.36885061)
\curveto(245.16039482,103.40884922)(245.09039489,103.46384916)(245.05038635,103.53385061)
\curveto(245.00039498,103.61384901)(244.97539501,103.73384889)(244.97538635,103.89385061)
\curveto(244.97539501,104.05384857)(244.96039502,104.18884844)(244.93038635,104.29885061)
\curveto(244.92039506,104.34884828)(244.91539507,104.40384822)(244.91538635,104.46385061)
\curveto(244.90539508,104.5238481)(244.89039509,104.58384804)(244.87038635,104.64385061)
\curveto(244.82039516,104.79384783)(244.77039521,104.93884769)(244.72038635,105.07885061)
\curveto(244.66039532,105.21884741)(244.59039539,105.35384727)(244.51038635,105.48385061)
\curveto(244.42039556,105.623847)(244.31539567,105.74384688)(244.19538635,105.84385061)
\curveto(244.07539591,105.94384668)(243.94539604,106.03884659)(243.80538635,106.12885061)
\curveto(243.70539628,106.18884644)(243.59539639,106.23384639)(243.47538635,106.26385061)
\curveto(243.35539663,106.30384632)(243.25039673,106.35384627)(243.16038635,106.41385061)
\curveto(243.10039688,106.46384616)(243.06039692,106.53384609)(243.04038635,106.62385061)
\curveto(243.03039695,106.64384598)(243.02539696,106.66884596)(243.02538635,106.69885061)
\curveto(243.02539696,106.7288459)(243.02039696,106.75384587)(243.01038635,106.77385061)
}
}
{
\newrgbcolor{curcolor}{0 0 0}
\pscustom[linestyle=none,fillstyle=solid,fillcolor=curcolor]
{
\newpath
\moveto(243.01038635,115.12345998)
\curveto(243.01039697,115.22345513)(243.02039696,115.31845503)(243.04038635,115.40845998)
\curveto(243.05039693,115.49845485)(243.0803969,115.56345479)(243.13038635,115.60345998)
\curveto(243.21039677,115.66345469)(243.31539667,115.69345466)(243.44538635,115.69345998)
\lineto(243.83538635,115.69345998)
\lineto(245.33538635,115.69345998)
\lineto(251.72538635,115.69345998)
\lineto(252.89538635,115.69345998)
\lineto(253.21038635,115.69345998)
\curveto(253.31038667,115.70345465)(253.39038659,115.68845466)(253.45038635,115.64845998)
\curveto(253.53038645,115.59845475)(253.5803864,115.52345483)(253.60038635,115.42345998)
\curveto(253.61038637,115.33345502)(253.61538637,115.22345513)(253.61538635,115.09345998)
\lineto(253.61538635,114.86845998)
\curveto(253.59538639,114.78845556)(253.5803864,114.71845563)(253.57038635,114.65845998)
\curveto(253.55038643,114.59845575)(253.51038647,114.5484558)(253.45038635,114.50845998)
\curveto(253.39038659,114.46845588)(253.31538667,114.4484559)(253.22538635,114.44845998)
\lineto(252.92538635,114.44845998)
\lineto(251.83038635,114.44845998)
\lineto(246.49038635,114.44845998)
\curveto(246.40039358,114.42845592)(246.32539366,114.41345594)(246.26538635,114.40345998)
\curveto(246.19539379,114.40345595)(246.13539385,114.37345598)(246.08538635,114.31345998)
\curveto(246.03539395,114.24345611)(246.01039397,114.1534562)(246.01038635,114.04345998)
\curveto(246.00039398,113.94345641)(245.99539399,113.83345652)(245.99538635,113.71345998)
\lineto(245.99538635,112.57345998)
\lineto(245.99538635,112.07845998)
\curveto(245.985394,111.91845843)(245.92539406,111.80845854)(245.81538635,111.74845998)
\curveto(245.7853942,111.72845862)(245.75539423,111.71845863)(245.72538635,111.71845998)
\curveto(245.6853943,111.71845863)(245.64039434,111.71345864)(245.59038635,111.70345998)
\curveto(245.47039451,111.68345867)(245.36039462,111.68845866)(245.26038635,111.71845998)
\curveto(245.16039482,111.75845859)(245.09039489,111.81345854)(245.05038635,111.88345998)
\curveto(245.00039498,111.96345839)(244.97539501,112.08345827)(244.97538635,112.24345998)
\curveto(244.97539501,112.40345795)(244.96039502,112.53845781)(244.93038635,112.64845998)
\curveto(244.92039506,112.69845765)(244.91539507,112.7534576)(244.91538635,112.81345998)
\curveto(244.90539508,112.87345748)(244.89039509,112.93345742)(244.87038635,112.99345998)
\curveto(244.82039516,113.14345721)(244.77039521,113.28845706)(244.72038635,113.42845998)
\curveto(244.66039532,113.56845678)(244.59039539,113.70345665)(244.51038635,113.83345998)
\curveto(244.42039556,113.97345638)(244.31539567,114.09345626)(244.19538635,114.19345998)
\curveto(244.07539591,114.29345606)(243.94539604,114.38845596)(243.80538635,114.47845998)
\curveto(243.70539628,114.53845581)(243.59539639,114.58345577)(243.47538635,114.61345998)
\curveto(243.35539663,114.6534557)(243.25039673,114.70345565)(243.16038635,114.76345998)
\curveto(243.10039688,114.81345554)(243.06039692,114.88345547)(243.04038635,114.97345998)
\curveto(243.03039695,114.99345536)(243.02539696,115.01845533)(243.02538635,115.04845998)
\curveto(243.02539696,115.07845527)(243.02039696,115.10345525)(243.01038635,115.12345998)
}
}
{
\newrgbcolor{curcolor}{0 0 0}
\pscustom[linestyle=none,fillstyle=solid,fillcolor=curcolor]
{
\newpath
\moveto(180.20139282,31.67142873)
\lineto(180.20139282,32.58642873)
\curveto(180.20140352,32.68642608)(180.20140352,32.78142599)(180.20139282,32.87142873)
\curveto(180.20140352,32.96142581)(180.2214035,33.03642573)(180.26139282,33.09642873)
\curveto(180.3214034,33.18642558)(180.40140332,33.24642552)(180.50139282,33.27642873)
\curveto(180.60140312,33.31642545)(180.70640301,33.36142541)(180.81639282,33.41142873)
\curveto(181.00640271,33.49142528)(181.19640252,33.56142521)(181.38639282,33.62142873)
\curveto(181.57640214,33.69142508)(181.76640195,33.766425)(181.95639282,33.84642873)
\curveto(182.13640158,33.91642485)(182.3214014,33.98142479)(182.51139282,34.04142873)
\curveto(182.69140103,34.10142467)(182.87140085,34.1714246)(183.05139282,34.25142873)
\curveto(183.19140053,34.31142446)(183.33640038,34.3664244)(183.48639282,34.41642873)
\curveto(183.63640008,34.4664243)(183.78139994,34.52142425)(183.92139282,34.58142873)
\curveto(184.37139935,34.76142401)(184.82639889,34.93142384)(185.28639282,35.09142873)
\curveto(185.73639798,35.25142352)(186.18639753,35.42142335)(186.63639282,35.60142873)
\curveto(186.68639703,35.62142315)(186.73639698,35.63642313)(186.78639282,35.64642873)
\lineto(186.93639282,35.70642873)
\curveto(187.15639656,35.79642297)(187.38139634,35.88142289)(187.61139282,35.96142873)
\curveto(187.83139589,36.04142273)(188.05139567,36.12642264)(188.27139282,36.21642873)
\curveto(188.36139536,36.25642251)(188.47139525,36.29642247)(188.60139282,36.33642873)
\curveto(188.721395,36.37642239)(188.79139493,36.44142233)(188.81139282,36.53142873)
\curveto(188.8213949,36.5714222)(188.8213949,36.60142217)(188.81139282,36.62142873)
\lineto(188.75139282,36.68142873)
\curveto(188.70139502,36.73142204)(188.64639507,36.766422)(188.58639282,36.78642873)
\curveto(188.52639519,36.81642195)(188.46139526,36.84642192)(188.39139282,36.87642873)
\lineto(187.76139282,37.11642873)
\curveto(187.54139618,37.19642157)(187.32639639,37.27642149)(187.11639282,37.35642873)
\lineto(186.96639282,37.41642873)
\lineto(186.78639282,37.47642873)
\curveto(186.59639712,37.55642121)(186.40639731,37.62642114)(186.21639282,37.68642873)
\curveto(186.0163977,37.75642101)(185.8163979,37.83142094)(185.61639282,37.91142873)
\curveto(185.03639868,38.15142062)(184.45139927,38.3714204)(183.86139282,38.57142873)
\curveto(183.27140045,38.78141999)(182.68640103,39.00641976)(182.10639282,39.24642873)
\curveto(181.90640181,39.32641944)(181.70140202,39.40141937)(181.49139282,39.47142873)
\curveto(181.28140244,39.55141922)(181.07640264,39.63141914)(180.87639282,39.71142873)
\curveto(180.79640292,39.75141902)(180.69640302,39.78641898)(180.57639282,39.81642873)
\curveto(180.45640326,39.85641891)(180.37140335,39.91141886)(180.32139282,39.98142873)
\curveto(180.28140344,40.04141873)(180.25140347,40.11641865)(180.23139282,40.20642873)
\curveto(180.21140351,40.30641846)(180.20140352,40.41641835)(180.20139282,40.53642873)
\curveto(180.19140353,40.65641811)(180.19140353,40.77641799)(180.20139282,40.89642873)
\curveto(180.20140352,41.01641775)(180.20140352,41.12641764)(180.20139282,41.22642873)
\curveto(180.20140352,41.31641745)(180.20140352,41.40641736)(180.20139282,41.49642873)
\curveto(180.20140352,41.59641717)(180.2214035,41.6714171)(180.26139282,41.72142873)
\curveto(180.31140341,41.81141696)(180.40140332,41.86141691)(180.53139282,41.87142873)
\curveto(180.66140306,41.88141689)(180.80140292,41.88641688)(180.95139282,41.88642873)
\lineto(182.60139282,41.88642873)
\lineto(188.87139282,41.88642873)
\lineto(190.13139282,41.88642873)
\curveto(190.24139348,41.88641688)(190.35139337,41.88641688)(190.46139282,41.88642873)
\curveto(190.57139315,41.89641687)(190.65639306,41.87641689)(190.71639282,41.82642873)
\curveto(190.77639294,41.79641697)(190.8163929,41.75141702)(190.83639282,41.69142873)
\curveto(190.84639287,41.63141714)(190.86139286,41.56141721)(190.88139282,41.48142873)
\lineto(190.88139282,41.24142873)
\lineto(190.88139282,40.88142873)
\curveto(190.87139285,40.771418)(190.82639289,40.69141808)(190.74639282,40.64142873)
\curveto(190.716393,40.62141815)(190.68639303,40.60641816)(190.65639282,40.59642873)
\curveto(190.6163931,40.59641817)(190.57139315,40.58641818)(190.52139282,40.56642873)
\lineto(190.35639282,40.56642873)
\curveto(190.29639342,40.55641821)(190.22639349,40.55141822)(190.14639282,40.55142873)
\curveto(190.06639365,40.56141821)(189.99139373,40.5664182)(189.92139282,40.56642873)
\lineto(189.08139282,40.56642873)
\lineto(184.65639282,40.56642873)
\curveto(184.40639931,40.5664182)(184.15639956,40.5664182)(183.90639282,40.56642873)
\curveto(183.64640007,40.5664182)(183.39640032,40.56141821)(183.15639282,40.55142873)
\curveto(183.05640066,40.55141822)(182.94640077,40.54641822)(182.82639282,40.53642873)
\curveto(182.70640101,40.52641824)(182.64640107,40.4714183)(182.64639282,40.37142873)
\lineto(182.66139282,40.37142873)
\curveto(182.68140104,40.30141847)(182.74640097,40.24141853)(182.85639282,40.19142873)
\curveto(182.96640075,40.15141862)(183.06140066,40.11641865)(183.14139282,40.08642873)
\curveto(183.31140041,40.01641875)(183.48640023,39.95141882)(183.66639282,39.89142873)
\curveto(183.83639988,39.83141894)(184.00639971,39.76141901)(184.17639282,39.68142873)
\curveto(184.22639949,39.66141911)(184.27139945,39.64641912)(184.31139282,39.63642873)
\curveto(184.35139937,39.62641914)(184.39639932,39.61141916)(184.44639282,39.59142873)
\curveto(184.62639909,39.51141926)(184.81139891,39.44141933)(185.00139282,39.38142873)
\curveto(185.18139854,39.33141944)(185.36139836,39.2664195)(185.54139282,39.18642873)
\curveto(185.69139803,39.11641965)(185.84639787,39.05641971)(186.00639282,39.00642873)
\curveto(186.15639756,38.95641981)(186.30639741,38.90141987)(186.45639282,38.84142873)
\curveto(186.92639679,38.64142013)(187.40139632,38.46142031)(187.88139282,38.30142873)
\curveto(188.35139537,38.14142063)(188.8163949,37.9664208)(189.27639282,37.77642873)
\curveto(189.45639426,37.69642107)(189.63639408,37.62642114)(189.81639282,37.56642873)
\curveto(189.99639372,37.50642126)(190.17639354,37.44142133)(190.35639282,37.37142873)
\curveto(190.46639325,37.32142145)(190.57139315,37.2714215)(190.67139282,37.22142873)
\curveto(190.76139296,37.18142159)(190.82639289,37.09642167)(190.86639282,36.96642873)
\curveto(190.87639284,36.94642182)(190.88139284,36.92142185)(190.88139282,36.89142873)
\curveto(190.87139285,36.8714219)(190.87139285,36.84642192)(190.88139282,36.81642873)
\curveto(190.89139283,36.78642198)(190.89639282,36.75142202)(190.89639282,36.71142873)
\curveto(190.88639283,36.6714221)(190.88139284,36.63142214)(190.88139282,36.59142873)
\lineto(190.88139282,36.29142873)
\curveto(190.88139284,36.19142258)(190.85639286,36.11142266)(190.80639282,36.05142873)
\curveto(190.75639296,35.9714228)(190.68639303,35.91142286)(190.59639282,35.87142873)
\curveto(190.49639322,35.84142293)(190.39639332,35.80142297)(190.29639282,35.75142873)
\curveto(190.09639362,35.6714231)(189.89139383,35.59142318)(189.68139282,35.51142873)
\curveto(189.46139426,35.44142333)(189.25139447,35.3664234)(189.05139282,35.28642873)
\curveto(188.87139485,35.20642356)(188.69139503,35.13642363)(188.51139282,35.07642873)
\curveto(188.3213954,35.02642374)(188.13639558,34.96142381)(187.95639282,34.88142873)
\curveto(187.39639632,34.65142412)(186.83139689,34.43642433)(186.26139282,34.23642873)
\curveto(185.69139803,34.03642473)(185.12639859,33.82142495)(184.56639282,33.59142873)
\lineto(183.93639282,33.35142873)
\curveto(183.7164,33.28142549)(183.50640021,33.20642556)(183.30639282,33.12642873)
\curveto(183.19640052,33.07642569)(183.09140063,33.03142574)(182.99139282,32.99142873)
\curveto(182.88140084,32.96142581)(182.78640093,32.91142586)(182.70639282,32.84142873)
\curveto(182.68640103,32.83142594)(182.67640104,32.82142595)(182.67639282,32.81142873)
\lineto(182.64639282,32.78142873)
\lineto(182.64639282,32.70642873)
\lineto(182.67639282,32.67642873)
\curveto(182.67640104,32.6664261)(182.68140104,32.65642611)(182.69139282,32.64642873)
\curveto(182.74140098,32.62642614)(182.79640092,32.61642615)(182.85639282,32.61642873)
\curveto(182.9164008,32.61642615)(182.97640074,32.60642616)(183.03639282,32.58642873)
\lineto(183.20139282,32.58642873)
\curveto(183.26140046,32.5664262)(183.32640039,32.56142621)(183.39639282,32.57142873)
\curveto(183.46640025,32.58142619)(183.53640018,32.58642618)(183.60639282,32.58642873)
\lineto(184.41639282,32.58642873)
\lineto(188.97639282,32.58642873)
\lineto(190.16139282,32.58642873)
\curveto(190.27139345,32.58642618)(190.38139334,32.58142619)(190.49139282,32.57142873)
\curveto(190.60139312,32.5714262)(190.68639303,32.54642622)(190.74639282,32.49642873)
\curveto(190.82639289,32.44642632)(190.87139285,32.35642641)(190.88139282,32.22642873)
\lineto(190.88139282,31.83642873)
\lineto(190.88139282,31.64142873)
\curveto(190.88139284,31.59142718)(190.87139285,31.54142723)(190.85139282,31.49142873)
\curveto(190.81139291,31.36142741)(190.72639299,31.28642748)(190.59639282,31.26642873)
\curveto(190.46639325,31.25642751)(190.3163934,31.25142752)(190.14639282,31.25142873)
\lineto(188.40639282,31.25142873)
\lineto(182.40639282,31.25142873)
\lineto(180.99639282,31.25142873)
\curveto(180.88640283,31.25142752)(180.77140295,31.24642752)(180.65139282,31.23642873)
\curveto(180.53140319,31.23642753)(180.43640328,31.26142751)(180.36639282,31.31142873)
\curveto(180.30640341,31.35142742)(180.25640346,31.42642734)(180.21639282,31.53642873)
\curveto(180.20640351,31.55642721)(180.20640351,31.57642719)(180.21639282,31.59642873)
\curveto(180.2164035,31.62642714)(180.21140351,31.65142712)(180.20139282,31.67142873)
}
}
{
\newrgbcolor{curcolor}{0 0 0}
\pscustom[linestyle=none,fillstyle=solid,fillcolor=curcolor]
{
\newpath
\moveto(190.32639282,50.87353811)
\curveto(190.48639323,50.90353028)(190.6213931,50.88853029)(190.73139282,50.82853811)
\curveto(190.83139289,50.76853041)(190.90639281,50.68853049)(190.95639282,50.58853811)
\curveto(190.97639274,50.53853064)(190.98639273,50.4835307)(190.98639282,50.42353811)
\curveto(190.98639273,50.37353081)(190.99639272,50.31853086)(191.01639282,50.25853811)
\curveto(191.06639265,50.03853114)(191.05139267,49.81853136)(190.97139282,49.59853811)
\curveto(190.90139282,49.38853179)(190.81139291,49.24353194)(190.70139282,49.16353811)
\curveto(190.63139309,49.11353207)(190.55139317,49.06853211)(190.46139282,49.02853811)
\curveto(190.36139336,48.98853219)(190.28139344,48.93853224)(190.22139282,48.87853811)
\curveto(190.20139352,48.85853232)(190.18139354,48.83353235)(190.16139282,48.80353811)
\curveto(190.14139358,48.7835324)(190.13639358,48.75353243)(190.14639282,48.71353811)
\curveto(190.17639354,48.60353258)(190.23139349,48.49853268)(190.31139282,48.39853811)
\curveto(190.39139333,48.30853287)(190.46139326,48.21853296)(190.52139282,48.12853811)
\curveto(190.60139312,47.99853318)(190.67639304,47.85853332)(190.74639282,47.70853811)
\curveto(190.80639291,47.55853362)(190.86139286,47.39853378)(190.91139282,47.22853811)
\curveto(190.94139278,47.12853405)(190.96139276,47.01853416)(190.97139282,46.89853811)
\curveto(190.98139274,46.78853439)(190.99639272,46.6785345)(191.01639282,46.56853811)
\curveto(191.02639269,46.51853466)(191.03139269,46.47353471)(191.03139282,46.43353811)
\lineto(191.03139282,46.32853811)
\curveto(191.05139267,46.21853496)(191.05139267,46.11353507)(191.03139282,46.01353811)
\lineto(191.03139282,45.87853811)
\curveto(191.0213927,45.82853535)(191.0163927,45.7785354)(191.01639282,45.72853811)
\curveto(191.0163927,45.6785355)(191.00639271,45.63353555)(190.98639282,45.59353811)
\curveto(190.97639274,45.55353563)(190.97139275,45.51853566)(190.97139282,45.48853811)
\curveto(190.98139274,45.46853571)(190.98139274,45.44353574)(190.97139282,45.41353811)
\lineto(190.91139282,45.17353811)
\curveto(190.90139282,45.09353609)(190.88139284,45.01853616)(190.85139282,44.94853811)
\curveto(190.721393,44.64853653)(190.57639314,44.40353678)(190.41639282,44.21353811)
\curveto(190.24639347,44.03353715)(190.01139371,43.8835373)(189.71139282,43.76353811)
\curveto(189.49139423,43.67353751)(189.22639449,43.62853755)(188.91639282,43.62853811)
\lineto(188.60139282,43.62853811)
\curveto(188.55139517,43.63853754)(188.50139522,43.64353754)(188.45139282,43.64353811)
\lineto(188.27139282,43.67353811)
\lineto(187.94139282,43.79353811)
\curveto(187.83139589,43.83353735)(187.73139599,43.8835373)(187.64139282,43.94353811)
\curveto(187.35139637,44.12353706)(187.13639658,44.36853681)(186.99639282,44.67853811)
\curveto(186.85639686,44.98853619)(186.73139699,45.32853585)(186.62139282,45.69853811)
\curveto(186.58139714,45.83853534)(186.55139717,45.9835352)(186.53139282,46.13353811)
\curveto(186.51139721,46.2835349)(186.48639723,46.43353475)(186.45639282,46.58353811)
\curveto(186.43639728,46.65353453)(186.42639729,46.71853446)(186.42639282,46.77853811)
\curveto(186.42639729,46.84853433)(186.4163973,46.92353426)(186.39639282,47.00353811)
\curveto(186.37639734,47.07353411)(186.36639735,47.14353404)(186.36639282,47.21353811)
\curveto(186.35639736,47.2835339)(186.34139738,47.35853382)(186.32139282,47.43853811)
\curveto(186.26139746,47.68853349)(186.21139751,47.92353326)(186.17139282,48.14353811)
\curveto(186.1213976,48.36353282)(186.00639771,48.53853264)(185.82639282,48.66853811)
\curveto(185.74639797,48.72853245)(185.64639807,48.7785324)(185.52639282,48.81853811)
\curveto(185.39639832,48.85853232)(185.25639846,48.85853232)(185.10639282,48.81853811)
\curveto(184.86639885,48.75853242)(184.67639904,48.66853251)(184.53639282,48.54853811)
\curveto(184.39639932,48.43853274)(184.28639943,48.2785329)(184.20639282,48.06853811)
\curveto(184.15639956,47.94853323)(184.1213996,47.80353338)(184.10139282,47.63353811)
\curveto(184.08139964,47.47353371)(184.07139965,47.30353388)(184.07139282,47.12353811)
\curveto(184.07139965,46.94353424)(184.08139964,46.76853441)(184.10139282,46.59853811)
\curveto(184.1213996,46.42853475)(184.15139957,46.2835349)(184.19139282,46.16353811)
\curveto(184.25139947,45.99353519)(184.33639938,45.82853535)(184.44639282,45.66853811)
\curveto(184.50639921,45.58853559)(184.58639913,45.51353567)(184.68639282,45.44353811)
\curveto(184.77639894,45.3835358)(184.87639884,45.32853585)(184.98639282,45.27853811)
\curveto(185.06639865,45.24853593)(185.15139857,45.21853596)(185.24139282,45.18853811)
\curveto(185.33139839,45.16853601)(185.40139832,45.12353606)(185.45139282,45.05353811)
\curveto(185.48139824,45.01353617)(185.50639821,44.94353624)(185.52639282,44.84353811)
\curveto(185.53639818,44.75353643)(185.54139818,44.65853652)(185.54139282,44.55853811)
\curveto(185.54139818,44.45853672)(185.53639818,44.35853682)(185.52639282,44.25853811)
\curveto(185.50639821,44.16853701)(185.48139824,44.10353708)(185.45139282,44.06353811)
\curveto(185.4213983,44.02353716)(185.37139835,43.99353719)(185.30139282,43.97353811)
\curveto(185.23139849,43.95353723)(185.15639856,43.95353723)(185.07639282,43.97353811)
\curveto(184.94639877,44.00353718)(184.82639889,44.03353715)(184.71639282,44.06353811)
\curveto(184.59639912,44.10353708)(184.48139924,44.14853703)(184.37139282,44.19853811)
\curveto(184.0213997,44.38853679)(183.75139997,44.62853655)(183.56139282,44.91853811)
\curveto(183.36140036,45.20853597)(183.20140052,45.56853561)(183.08139282,45.99853811)
\curveto(183.06140066,46.09853508)(183.04640067,46.19853498)(183.03639282,46.29853811)
\curveto(183.02640069,46.40853477)(183.01140071,46.51853466)(182.99139282,46.62853811)
\curveto(182.98140074,46.66853451)(182.98140074,46.73353445)(182.99139282,46.82353811)
\curveto(182.99140073,46.91353427)(182.98140074,46.96853421)(182.96139282,46.98853811)
\curveto(182.95140077,47.68853349)(183.03140069,48.29853288)(183.20139282,48.81853811)
\curveto(183.37140035,49.33853184)(183.69640002,49.70353148)(184.17639282,49.91353811)
\curveto(184.37639934,50.00353118)(184.61139911,50.05353113)(184.88139282,50.06353811)
\curveto(185.14139858,50.0835311)(185.4163983,50.09353109)(185.70639282,50.09353811)
\lineto(189.02139282,50.09353811)
\curveto(189.16139456,50.09353109)(189.29639442,50.09853108)(189.42639282,50.10853811)
\curveto(189.55639416,50.11853106)(189.66139406,50.14853103)(189.74139282,50.19853811)
\curveto(189.81139391,50.24853093)(189.86139386,50.31353087)(189.89139282,50.39353811)
\curveto(189.93139379,50.4835307)(189.96139376,50.56853061)(189.98139282,50.64853811)
\curveto(189.99139373,50.72853045)(190.03639368,50.78853039)(190.11639282,50.82853811)
\curveto(190.14639357,50.84853033)(190.17639354,50.85853032)(190.20639282,50.85853811)
\curveto(190.23639348,50.85853032)(190.27639344,50.86353032)(190.32639282,50.87353811)
\moveto(188.66139282,48.72853811)
\curveto(188.5213952,48.78853239)(188.36139536,48.81853236)(188.18139282,48.81853811)
\curveto(187.99139573,48.82853235)(187.79639592,48.83353235)(187.59639282,48.83353811)
\curveto(187.48639623,48.83353235)(187.38639633,48.82853235)(187.29639282,48.81853811)
\curveto(187.20639651,48.80853237)(187.13639658,48.76853241)(187.08639282,48.69853811)
\curveto(187.06639665,48.66853251)(187.05639666,48.59853258)(187.05639282,48.48853811)
\curveto(187.07639664,48.46853271)(187.08639663,48.43353275)(187.08639282,48.38353811)
\curveto(187.08639663,48.33353285)(187.09639662,48.28853289)(187.11639282,48.24853811)
\curveto(187.13639658,48.16853301)(187.15639656,48.0785331)(187.17639282,47.97853811)
\lineto(187.23639282,47.67853811)
\curveto(187.23639648,47.64853353)(187.24139648,47.61353357)(187.25139282,47.57353811)
\lineto(187.25139282,47.46853811)
\curveto(187.29139643,47.31853386)(187.3163964,47.15353403)(187.32639282,46.97353811)
\curveto(187.32639639,46.80353438)(187.34639637,46.64353454)(187.38639282,46.49353811)
\curveto(187.40639631,46.41353477)(187.42639629,46.33853484)(187.44639282,46.26853811)
\curveto(187.45639626,46.20853497)(187.47139625,46.13853504)(187.49139282,46.05853811)
\curveto(187.54139618,45.89853528)(187.60639611,45.74853543)(187.68639282,45.60853811)
\curveto(187.75639596,45.46853571)(187.84639587,45.34853583)(187.95639282,45.24853811)
\curveto(188.06639565,45.14853603)(188.20139552,45.07353611)(188.36139282,45.02353811)
\curveto(188.51139521,44.97353621)(188.69639502,44.95353623)(188.91639282,44.96353811)
\curveto(189.0163947,44.96353622)(189.11139461,44.9785362)(189.20139282,45.00853811)
\curveto(189.28139444,45.04853613)(189.35639436,45.09353609)(189.42639282,45.14353811)
\curveto(189.53639418,45.22353596)(189.63139409,45.32853585)(189.71139282,45.45853811)
\curveto(189.78139394,45.58853559)(189.84139388,45.72853545)(189.89139282,45.87853811)
\curveto(189.90139382,45.92853525)(189.90639381,45.9785352)(189.90639282,46.02853811)
\curveto(189.90639381,46.0785351)(189.91139381,46.12853505)(189.92139282,46.17853811)
\curveto(189.94139378,46.24853493)(189.95639376,46.33353485)(189.96639282,46.43353811)
\curveto(189.96639375,46.54353464)(189.95639376,46.63353455)(189.93639282,46.70353811)
\curveto(189.9163938,46.76353442)(189.91139381,46.82353436)(189.92139282,46.88353811)
\curveto(189.9213938,46.94353424)(189.91139381,47.00353418)(189.89139282,47.06353811)
\curveto(189.87139385,47.14353404)(189.85639386,47.21853396)(189.84639282,47.28853811)
\curveto(189.83639388,47.36853381)(189.8163939,47.44353374)(189.78639282,47.51353811)
\curveto(189.66639405,47.80353338)(189.5213942,48.04853313)(189.35139282,48.24853811)
\curveto(189.18139454,48.45853272)(188.95139477,48.61853256)(188.66139282,48.72853811)
}
}
{
\newrgbcolor{curcolor}{0 0 0}
\pscustom[linestyle=none,fillstyle=solid,fillcolor=curcolor]
{
\newpath
\moveto(182.97639282,55.69017873)
\curveto(182.97640074,55.92017394)(183.03640068,56.05017381)(183.15639282,56.08017873)
\curveto(183.26640045,56.11017375)(183.43140029,56.12517374)(183.65139282,56.12517873)
\lineto(183.93639282,56.12517873)
\curveto(184.02639969,56.12517374)(184.10139962,56.10017376)(184.16139282,56.05017873)
\curveto(184.24139948,55.99017387)(184.28639943,55.90517396)(184.29639282,55.79517873)
\curveto(184.29639942,55.68517418)(184.31139941,55.57517429)(184.34139282,55.46517873)
\curveto(184.37139935,55.32517454)(184.40139932,55.19017467)(184.43139282,55.06017873)
\curveto(184.46139926,54.94017492)(184.50139922,54.82517504)(184.55139282,54.71517873)
\curveto(184.68139904,54.42517544)(184.86139886,54.19017567)(185.09139282,54.01017873)
\curveto(185.31139841,53.83017603)(185.56639815,53.67517619)(185.85639282,53.54517873)
\curveto(185.96639775,53.50517636)(186.08139764,53.47517639)(186.20139282,53.45517873)
\curveto(186.31139741,53.43517643)(186.42639729,53.41017645)(186.54639282,53.38017873)
\curveto(186.59639712,53.37017649)(186.64639707,53.3651765)(186.69639282,53.36517873)
\curveto(186.74639697,53.37517649)(186.79639692,53.37517649)(186.84639282,53.36517873)
\curveto(186.96639675,53.33517653)(187.10639661,53.32017654)(187.26639282,53.32017873)
\curveto(187.4163963,53.33017653)(187.56139616,53.33517653)(187.70139282,53.33517873)
\lineto(189.54639282,53.33517873)
\lineto(189.89139282,53.33517873)
\curveto(190.01139371,53.33517653)(190.12639359,53.33017653)(190.23639282,53.32017873)
\curveto(190.34639337,53.31017655)(190.44139328,53.30517656)(190.52139282,53.30517873)
\curveto(190.60139312,53.31517655)(190.67139305,53.29517657)(190.73139282,53.24517873)
\curveto(190.80139292,53.19517667)(190.84139288,53.11517675)(190.85139282,53.00517873)
\curveto(190.86139286,52.90517696)(190.86639285,52.79517707)(190.86639282,52.67517873)
\lineto(190.86639282,52.40517873)
\curveto(190.84639287,52.35517751)(190.83139289,52.30517756)(190.82139282,52.25517873)
\curveto(190.80139292,52.21517765)(190.77639294,52.18517768)(190.74639282,52.16517873)
\curveto(190.67639304,52.11517775)(190.59139313,52.08517778)(190.49139282,52.07517873)
\lineto(190.16139282,52.07517873)
\lineto(189.00639282,52.07517873)
\lineto(184.85139282,52.07517873)
\lineto(183.81639282,52.07517873)
\lineto(183.51639282,52.07517873)
\curveto(183.4164003,52.08517778)(183.33140039,52.11517775)(183.26139282,52.16517873)
\curveto(183.2214005,52.19517767)(183.19140053,52.24517762)(183.17139282,52.31517873)
\curveto(183.15140057,52.39517747)(183.14140058,52.48017738)(183.14139282,52.57017873)
\curveto(183.13140059,52.6601772)(183.13140059,52.75017711)(183.14139282,52.84017873)
\curveto(183.15140057,52.93017693)(183.16640055,53.00017686)(183.18639282,53.05017873)
\curveto(183.2164005,53.13017673)(183.27640044,53.18017668)(183.36639282,53.20017873)
\curveto(183.44640027,53.23017663)(183.53640018,53.24517662)(183.63639282,53.24517873)
\lineto(183.93639282,53.24517873)
\curveto(184.03639968,53.24517662)(184.12639959,53.2651766)(184.20639282,53.30517873)
\curveto(184.22639949,53.31517655)(184.24139948,53.32517654)(184.25139282,53.33517873)
\lineto(184.29639282,53.38017873)
\curveto(184.29639942,53.49017637)(184.25139947,53.58017628)(184.16139282,53.65017873)
\curveto(184.06139966,53.72017614)(183.98139974,53.78017608)(183.92139282,53.83017873)
\lineto(183.83139282,53.92017873)
\curveto(183.7214,54.01017585)(183.60640011,54.13517573)(183.48639282,54.29517873)
\curveto(183.36640035,54.45517541)(183.27640044,54.60517526)(183.21639282,54.74517873)
\curveto(183.16640055,54.83517503)(183.13140059,54.93017493)(183.11139282,55.03017873)
\curveto(183.08140064,55.13017473)(183.05140067,55.23517463)(183.02139282,55.34517873)
\curveto(183.01140071,55.40517446)(183.00640071,55.4651744)(183.00639282,55.52517873)
\curveto(182.99640072,55.58517428)(182.98640073,55.64017422)(182.97639282,55.69017873)
}
}
{
\newrgbcolor{curcolor}{0 0 0}
\pscustom[linestyle=none,fillstyle=solid,fillcolor=curcolor]
{
}
}
{
\newrgbcolor{curcolor}{0 0 0}
\pscustom[linestyle=none,fillstyle=solid,fillcolor=curcolor]
{
\newpath
\moveto(180.27639282,65.05510061)
\curveto(180.27640344,65.15509575)(180.28640343,65.25009566)(180.30639282,65.34010061)
\curveto(180.3164034,65.43009548)(180.34640337,65.49509541)(180.39639282,65.53510061)
\curveto(180.47640324,65.59509531)(180.58140314,65.62509528)(180.71139282,65.62510061)
\lineto(181.10139282,65.62510061)
\lineto(182.60139282,65.62510061)
\lineto(188.99139282,65.62510061)
\lineto(190.16139282,65.62510061)
\lineto(190.47639282,65.62510061)
\curveto(190.57639314,65.63509527)(190.65639306,65.62009529)(190.71639282,65.58010061)
\curveto(190.79639292,65.53009538)(190.84639287,65.45509545)(190.86639282,65.35510061)
\curveto(190.87639284,65.26509564)(190.88139284,65.15509575)(190.88139282,65.02510061)
\lineto(190.88139282,64.80010061)
\curveto(190.86139286,64.72009619)(190.84639287,64.65009626)(190.83639282,64.59010061)
\curveto(190.8163929,64.53009638)(190.77639294,64.48009643)(190.71639282,64.44010061)
\curveto(190.65639306,64.40009651)(190.58139314,64.38009653)(190.49139282,64.38010061)
\lineto(190.19139282,64.38010061)
\lineto(189.09639282,64.38010061)
\lineto(183.75639282,64.38010061)
\curveto(183.66640005,64.36009655)(183.59140013,64.34509656)(183.53139282,64.33510061)
\curveto(183.46140026,64.33509657)(183.40140032,64.3050966)(183.35139282,64.24510061)
\curveto(183.30140042,64.17509673)(183.27640044,64.08509682)(183.27639282,63.97510061)
\curveto(183.26640045,63.87509703)(183.26140046,63.76509714)(183.26139282,63.64510061)
\lineto(183.26139282,62.50510061)
\lineto(183.26139282,62.01010061)
\curveto(183.25140047,61.85009906)(183.19140053,61.74009917)(183.08139282,61.68010061)
\curveto(183.05140067,61.66009925)(183.0214007,61.65009926)(182.99139282,61.65010061)
\curveto(182.95140077,61.65009926)(182.90640081,61.64509926)(182.85639282,61.63510061)
\curveto(182.73640098,61.61509929)(182.62640109,61.62009929)(182.52639282,61.65010061)
\curveto(182.42640129,61.69009922)(182.35640136,61.74509916)(182.31639282,61.81510061)
\curveto(182.26640145,61.89509901)(182.24140148,62.01509889)(182.24139282,62.17510061)
\curveto(182.24140148,62.33509857)(182.22640149,62.47009844)(182.19639282,62.58010061)
\curveto(182.18640153,62.63009828)(182.18140154,62.68509822)(182.18139282,62.74510061)
\curveto(182.17140155,62.8050981)(182.15640156,62.86509804)(182.13639282,62.92510061)
\curveto(182.08640163,63.07509783)(182.03640168,63.22009769)(181.98639282,63.36010061)
\curveto(181.92640179,63.50009741)(181.85640186,63.63509727)(181.77639282,63.76510061)
\curveto(181.68640203,63.905097)(181.58140214,64.02509688)(181.46139282,64.12510061)
\curveto(181.34140238,64.22509668)(181.21140251,64.32009659)(181.07139282,64.41010061)
\curveto(180.97140275,64.47009644)(180.86140286,64.51509639)(180.74139282,64.54510061)
\curveto(180.6214031,64.58509632)(180.5164032,64.63509627)(180.42639282,64.69510061)
\curveto(180.36640335,64.74509616)(180.32640339,64.81509609)(180.30639282,64.90510061)
\curveto(180.29640342,64.92509598)(180.29140343,64.95009596)(180.29139282,64.98010061)
\curveto(180.29140343,65.0100959)(180.28640343,65.03509587)(180.27639282,65.05510061)
}
}
{
\newrgbcolor{curcolor}{0 0 0}
\pscustom[linestyle=none,fillstyle=solid,fillcolor=curcolor]
{
\newpath
\moveto(187.38639282,76.22470998)
\curveto(187.43639628,76.29470234)(187.50639621,76.3347023)(187.59639282,76.34470998)
\curveto(187.68639603,76.36470227)(187.79139593,76.37470226)(187.91139282,76.37470998)
\curveto(187.96139576,76.37470226)(188.01139571,76.36970226)(188.06139282,76.35970998)
\curveto(188.11139561,76.35970227)(188.15639556,76.34970228)(188.19639282,76.32970998)
\curveto(188.28639543,76.29970233)(188.34639537,76.23970239)(188.37639282,76.14970998)
\curveto(188.39639532,76.06970256)(188.40639531,75.97470266)(188.40639282,75.86470998)
\lineto(188.40639282,75.54970998)
\curveto(188.39639532,75.43970319)(188.40639531,75.3347033)(188.43639282,75.23470998)
\curveto(188.46639525,75.09470354)(188.54639517,75.00470363)(188.67639282,74.96470998)
\curveto(188.74639497,74.94470369)(188.83139489,74.9347037)(188.93139282,74.93470998)
\lineto(189.20139282,74.93470998)
\lineto(190.14639282,74.93470998)
\lineto(190.47639282,74.93470998)
\curveto(190.58639313,74.9347037)(190.67139305,74.91470372)(190.73139282,74.87470998)
\curveto(190.79139293,74.8347038)(190.83139289,74.78470385)(190.85139282,74.72470998)
\curveto(190.86139286,74.67470396)(190.87639284,74.60970402)(190.89639282,74.52970998)
\lineto(190.89639282,74.33470998)
\curveto(190.89639282,74.21470442)(190.89139283,74.10970452)(190.88139282,74.01970998)
\curveto(190.86139286,73.9297047)(190.81139291,73.85970477)(190.73139282,73.80970998)
\curveto(190.68139304,73.77970485)(190.61139311,73.76470487)(190.52139282,73.76470998)
\lineto(190.22139282,73.76470998)
\lineto(189.18639282,73.76470998)
\curveto(189.02639469,73.76470487)(188.88139484,73.75470488)(188.75139282,73.73470998)
\curveto(188.61139511,73.72470491)(188.5163952,73.66970496)(188.46639282,73.56970998)
\curveto(188.44639527,73.51970511)(188.43139529,73.44970518)(188.42139282,73.35970998)
\curveto(188.41139531,73.27970535)(188.40639531,73.18970544)(188.40639282,73.08970998)
\lineto(188.40639282,72.80470998)
\lineto(188.40639282,72.56470998)
\lineto(188.40639282,70.29970998)
\curveto(188.40639531,70.20970842)(188.41139531,70.10470853)(188.42139282,69.98470998)
\lineto(188.42139282,69.65470998)
\curveto(188.4213953,69.54470909)(188.41139531,69.44470919)(188.39139282,69.35470998)
\curveto(188.37139535,69.26470937)(188.33639538,69.20470943)(188.28639282,69.17470998)
\curveto(188.2163955,69.12470951)(188.1213956,69.09970953)(188.00139282,69.09970998)
\lineto(187.65639282,69.09970998)
\lineto(187.38639282,69.09970998)
\curveto(187.2163965,69.13970949)(187.07639664,69.19470944)(186.96639282,69.26470998)
\curveto(186.85639686,69.3347093)(186.74139698,69.41470922)(186.62139282,69.50470998)
\lineto(186.08139282,69.86470998)
\curveto(185.45139827,70.30470833)(184.83139889,70.73970789)(184.22139282,71.16970998)
\lineto(182.36139282,72.48970998)
\curveto(182.13140159,72.64970598)(181.91140181,72.80470583)(181.70139282,72.95470998)
\curveto(181.48140224,73.10470553)(181.25640246,73.25970537)(181.02639282,73.41970998)
\curveto(180.95640276,73.46970516)(180.89140283,73.51970511)(180.83139282,73.56970998)
\curveto(180.76140296,73.61970501)(180.68640303,73.66970496)(180.60639282,73.71970998)
\lineto(180.51639282,73.77970998)
\curveto(180.47640324,73.80970482)(180.44640327,73.83970479)(180.42639282,73.86970998)
\curveto(180.39640332,73.90970472)(180.37640334,73.94970468)(180.36639282,73.98970998)
\curveto(180.34640337,74.0297046)(180.32640339,74.07470456)(180.30639282,74.12470998)
\curveto(180.30640341,74.14470449)(180.31140341,74.16470447)(180.32139282,74.18470998)
\curveto(180.3214034,74.21470442)(180.31140341,74.23970439)(180.29139282,74.25970998)
\curveto(180.29140343,74.38970424)(180.29640342,74.50970412)(180.30639282,74.61970998)
\curveto(180.3164034,74.7297039)(180.36140336,74.80970382)(180.44139282,74.85970998)
\curveto(180.49140323,74.89970373)(180.56140316,74.91970371)(180.65139282,74.91970998)
\curveto(180.74140298,74.9297037)(180.83640288,74.9347037)(180.93639282,74.93470998)
\lineto(186.39639282,74.93470998)
\curveto(186.46639725,74.9347037)(186.54139718,74.9297037)(186.62139282,74.91970998)
\curveto(186.69139703,74.91970371)(186.76139696,74.92470371)(186.83139282,74.93470998)
\lineto(186.93639282,74.93470998)
\curveto(186.98639673,74.95470368)(187.04139668,74.96970366)(187.10139282,74.97970998)
\curveto(187.15139657,74.98970364)(187.19139653,75.01470362)(187.22139282,75.05470998)
\curveto(187.27139645,75.12470351)(187.30139642,75.20970342)(187.31139282,75.30970998)
\lineto(187.31139282,75.63970998)
\curveto(187.31139641,75.74970288)(187.3163964,75.85470278)(187.32639282,75.95470998)
\curveto(187.32639639,76.06470257)(187.34639637,76.15470248)(187.38639282,76.22470998)
\moveto(187.19139282,73.65970998)
\curveto(187.08139664,73.73970489)(186.91139681,73.77470486)(186.68139282,73.76470998)
\lineto(186.06639282,73.76470998)
\lineto(183.59139282,73.76470998)
\lineto(183.27639282,73.76470998)
\curveto(183.15640056,73.77470486)(183.05640066,73.76970486)(182.97639282,73.74970998)
\lineto(182.82639282,73.74970998)
\curveto(182.73640098,73.74970488)(182.65140107,73.7347049)(182.57139282,73.70470998)
\curveto(182.55140117,73.69470494)(182.54140118,73.68470495)(182.54139282,73.67470998)
\lineto(182.49639282,73.62970998)
\curveto(182.48640123,73.60970502)(182.48140124,73.57970505)(182.48139282,73.53970998)
\curveto(182.50140122,73.51970511)(182.5164012,73.49970513)(182.52639282,73.47970998)
\curveto(182.52640119,73.46970516)(182.53140119,73.45470518)(182.54139282,73.43470998)
\curveto(182.59140113,73.37470526)(182.66140106,73.31470532)(182.75139282,73.25470998)
\curveto(182.84140088,73.19470544)(182.9214008,73.13970549)(182.99139282,73.08970998)
\curveto(183.13140059,72.98970564)(183.27640044,72.89470574)(183.42639282,72.80470998)
\curveto(183.56640015,72.71470592)(183.70640001,72.61970601)(183.84639282,72.51970998)
\lineto(184.62639282,71.97970998)
\curveto(184.88639883,71.80970682)(185.14639857,71.634707)(185.40639282,71.45470998)
\curveto(185.5163982,71.37470726)(185.6213981,71.29970733)(185.72139282,71.22970998)
\lineto(186.02139282,71.01970998)
\curveto(186.10139762,70.96970766)(186.17639754,70.91970771)(186.24639282,70.86970998)
\curveto(186.3163974,70.8297078)(186.39139733,70.78470785)(186.47139282,70.73470998)
\curveto(186.53139719,70.68470795)(186.59639712,70.634708)(186.66639282,70.58470998)
\curveto(186.72639699,70.54470809)(186.79639692,70.50470813)(186.87639282,70.46470998)
\curveto(186.93639678,70.42470821)(187.00639671,70.39970823)(187.08639282,70.38970998)
\curveto(187.15639656,70.37970825)(187.21139651,70.41470822)(187.25139282,70.49470998)
\curveto(187.30139642,70.56470807)(187.32639639,70.67470796)(187.32639282,70.82470998)
\curveto(187.3163964,70.98470765)(187.31139641,71.11970751)(187.31139282,71.22970998)
\lineto(187.31139282,72.90970998)
\lineto(187.31139282,73.34470998)
\curveto(187.31139641,73.49470514)(187.27139645,73.59970503)(187.19139282,73.65970998)
}
}
{
\newrgbcolor{curcolor}{0 0 0}
\pscustom[linestyle=none,fillstyle=solid,fillcolor=curcolor]
{
\newpath
\moveto(189.24639282,78.63431936)
\lineto(189.24639282,79.26431936)
\lineto(189.24639282,79.45931936)
\curveto(189.24639447,79.52931683)(189.25639446,79.58931677)(189.27639282,79.63931936)
\curveto(189.3163944,79.70931665)(189.35639436,79.7593166)(189.39639282,79.78931936)
\curveto(189.44639427,79.82931653)(189.51139421,79.84931651)(189.59139282,79.84931936)
\curveto(189.67139405,79.8593165)(189.75639396,79.86431649)(189.84639282,79.86431936)
\lineto(190.56639282,79.86431936)
\curveto(191.04639267,79.86431649)(191.45639226,79.80431655)(191.79639282,79.68431936)
\curveto(192.13639158,79.56431679)(192.41139131,79.36931699)(192.62139282,79.09931936)
\curveto(192.67139105,79.02931733)(192.716391,78.9593174)(192.75639282,78.88931936)
\curveto(192.80639091,78.82931753)(192.85139087,78.7543176)(192.89139282,78.66431936)
\curveto(192.90139082,78.64431771)(192.91139081,78.61431774)(192.92139282,78.57431936)
\curveto(192.94139078,78.53431782)(192.94639077,78.48931787)(192.93639282,78.43931936)
\curveto(192.90639081,78.34931801)(192.83139089,78.29431806)(192.71139282,78.27431936)
\curveto(192.60139112,78.2543181)(192.50639121,78.26931809)(192.42639282,78.31931936)
\curveto(192.35639136,78.34931801)(192.29139143,78.39431796)(192.23139282,78.45431936)
\curveto(192.18139154,78.52431783)(192.13139159,78.58931777)(192.08139282,78.64931936)
\curveto(192.03139169,78.71931764)(191.95639176,78.77931758)(191.85639282,78.82931936)
\curveto(191.76639195,78.88931747)(191.67639204,78.93931742)(191.58639282,78.97931936)
\curveto(191.55639216,78.99931736)(191.49639222,79.02431733)(191.40639282,79.05431936)
\curveto(191.32639239,79.08431727)(191.25639246,79.08931727)(191.19639282,79.06931936)
\curveto(191.05639266,79.03931732)(190.96639275,78.97931738)(190.92639282,78.88931936)
\curveto(190.89639282,78.80931755)(190.88139284,78.71931764)(190.88139282,78.61931936)
\curveto(190.88139284,78.51931784)(190.85639286,78.43431792)(190.80639282,78.36431936)
\curveto(190.73639298,78.27431808)(190.59639312,78.22931813)(190.38639282,78.22931936)
\lineto(189.83139282,78.22931936)
\lineto(189.60639282,78.22931936)
\curveto(189.52639419,78.23931812)(189.46139426,78.2593181)(189.41139282,78.28931936)
\curveto(189.33139439,78.34931801)(189.28639443,78.41931794)(189.27639282,78.49931936)
\curveto(189.26639445,78.51931784)(189.26139446,78.53931782)(189.26139282,78.55931936)
\curveto(189.26139446,78.58931777)(189.25639446,78.61431774)(189.24639282,78.63431936)
}
}
{
\newrgbcolor{curcolor}{0 0 0}
\pscustom[linestyle=none,fillstyle=solid,fillcolor=curcolor]
{
}
}
{
\newrgbcolor{curcolor}{0 0 0}
\pscustom[linestyle=none,fillstyle=solid,fillcolor=curcolor]
{
\newpath
\moveto(180.27639282,89.26463186)
\curveto(180.26640345,89.95462722)(180.38640333,90.55462662)(180.63639282,91.06463186)
\curveto(180.88640283,91.58462559)(181.2214025,91.9796252)(181.64139282,92.24963186)
\curveto(181.721402,92.29962488)(181.81140191,92.34462483)(181.91139282,92.38463186)
\curveto(182.00140172,92.42462475)(182.09640162,92.46962471)(182.19639282,92.51963186)
\curveto(182.29640142,92.55962462)(182.39640132,92.58962459)(182.49639282,92.60963186)
\curveto(182.59640112,92.62962455)(182.70140102,92.64962453)(182.81139282,92.66963186)
\curveto(182.86140086,92.68962449)(182.90640081,92.69462448)(182.94639282,92.68463186)
\curveto(182.98640073,92.6746245)(183.03140069,92.6796245)(183.08139282,92.69963186)
\curveto(183.13140059,92.70962447)(183.2164005,92.71462446)(183.33639282,92.71463186)
\curveto(183.44640027,92.71462446)(183.53140019,92.70962447)(183.59139282,92.69963186)
\curveto(183.65140007,92.6796245)(183.71140001,92.66962451)(183.77139282,92.66963186)
\curveto(183.83139989,92.6796245)(183.89139983,92.6746245)(183.95139282,92.65463186)
\curveto(184.09139963,92.61462456)(184.22639949,92.5796246)(184.35639282,92.54963186)
\curveto(184.48639923,92.51962466)(184.61139911,92.4796247)(184.73139282,92.42963186)
\curveto(184.87139885,92.36962481)(184.99639872,92.29962488)(185.10639282,92.21963186)
\curveto(185.2163985,92.14962503)(185.32639839,92.0746251)(185.43639282,91.99463186)
\lineto(185.49639282,91.93463186)
\curveto(185.5163982,91.92462525)(185.53639818,91.90962527)(185.55639282,91.88963186)
\curveto(185.716398,91.76962541)(185.86139786,91.63462554)(185.99139282,91.48463186)
\curveto(186.1213976,91.33462584)(186.24639747,91.174626)(186.36639282,91.00463186)
\curveto(186.58639713,90.69462648)(186.79139693,90.39962678)(186.98139282,90.11963186)
\curveto(187.1213966,89.88962729)(187.25639646,89.65962752)(187.38639282,89.42963186)
\curveto(187.5163962,89.20962797)(187.65139607,88.98962819)(187.79139282,88.76963186)
\curveto(187.96139576,88.51962866)(188.14139558,88.2796289)(188.33139282,88.04963186)
\curveto(188.5213952,87.82962935)(188.74639497,87.63962954)(189.00639282,87.47963186)
\curveto(189.06639465,87.43962974)(189.12639459,87.40462977)(189.18639282,87.37463186)
\curveto(189.23639448,87.34462983)(189.30139442,87.31462986)(189.38139282,87.28463186)
\curveto(189.45139427,87.26462991)(189.51139421,87.25962992)(189.56139282,87.26963186)
\curveto(189.63139409,87.28962989)(189.68639403,87.32462985)(189.72639282,87.37463186)
\curveto(189.75639396,87.42462975)(189.77639394,87.48462969)(189.78639282,87.55463186)
\lineto(189.78639282,87.79463186)
\lineto(189.78639282,88.54463186)
\lineto(189.78639282,91.34963186)
\lineto(189.78639282,92.00963186)
\curveto(189.78639393,92.09962508)(189.79139393,92.18462499)(189.80139282,92.26463186)
\curveto(189.80139392,92.34462483)(189.8213939,92.40962477)(189.86139282,92.45963186)
\curveto(189.90139382,92.50962467)(189.97639374,92.54962463)(190.08639282,92.57963186)
\curveto(190.18639353,92.61962456)(190.28639343,92.62962455)(190.38639282,92.60963186)
\lineto(190.52139282,92.60963186)
\curveto(190.59139313,92.58962459)(190.65139307,92.56962461)(190.70139282,92.54963186)
\curveto(190.75139297,92.52962465)(190.79139293,92.49462468)(190.82139282,92.44463186)
\curveto(190.86139286,92.39462478)(190.88139284,92.32462485)(190.88139282,92.23463186)
\lineto(190.88139282,91.96463186)
\lineto(190.88139282,91.06463186)
\lineto(190.88139282,87.55463186)
\lineto(190.88139282,86.48963186)
\curveto(190.88139284,86.40963077)(190.88639283,86.31963086)(190.89639282,86.21963186)
\curveto(190.89639282,86.11963106)(190.88639283,86.03463114)(190.86639282,85.96463186)
\curveto(190.79639292,85.75463142)(190.6163931,85.68963149)(190.32639282,85.76963186)
\curveto(190.28639343,85.7796314)(190.25139347,85.7796314)(190.22139282,85.76963186)
\curveto(190.18139354,85.76963141)(190.13639358,85.7796314)(190.08639282,85.79963186)
\curveto(190.00639371,85.81963136)(189.9213938,85.83963134)(189.83139282,85.85963186)
\curveto(189.74139398,85.8796313)(189.65639406,85.90463127)(189.57639282,85.93463186)
\curveto(189.08639463,86.09463108)(188.67139505,86.29463088)(188.33139282,86.53463186)
\curveto(188.08139564,86.71463046)(187.85639586,86.91963026)(187.65639282,87.14963186)
\curveto(187.44639627,87.3796298)(187.25139647,87.61962956)(187.07139282,87.86963186)
\curveto(186.89139683,88.12962905)(186.721397,88.39462878)(186.56139282,88.66463186)
\curveto(186.39139733,88.94462823)(186.2163975,89.21462796)(186.03639282,89.47463186)
\curveto(185.95639776,89.58462759)(185.88139784,89.68962749)(185.81139282,89.78963186)
\curveto(185.74139798,89.89962728)(185.66639805,90.00962717)(185.58639282,90.11963186)
\curveto(185.55639816,90.15962702)(185.52639819,90.19462698)(185.49639282,90.22463186)
\curveto(185.45639826,90.26462691)(185.42639829,90.30462687)(185.40639282,90.34463186)
\curveto(185.29639842,90.48462669)(185.17139855,90.60962657)(185.03139282,90.71963186)
\curveto(185.00139872,90.73962644)(184.97639874,90.76462641)(184.95639282,90.79463186)
\curveto(184.92639879,90.82462635)(184.89639882,90.84962633)(184.86639282,90.86963186)
\curveto(184.76639895,90.94962623)(184.66639905,91.01462616)(184.56639282,91.06463186)
\curveto(184.46639925,91.12462605)(184.35639936,91.179626)(184.23639282,91.22963186)
\curveto(184.16639955,91.25962592)(184.09139963,91.2796259)(184.01139282,91.28963186)
\lineto(183.77139282,91.34963186)
\lineto(183.68139282,91.34963186)
\curveto(183.65140007,91.35962582)(183.6214001,91.36462581)(183.59139282,91.36463186)
\curveto(183.5214002,91.38462579)(183.42640029,91.38962579)(183.30639282,91.37963186)
\curveto(183.17640054,91.3796258)(183.07640064,91.36962581)(183.00639282,91.34963186)
\curveto(182.92640079,91.32962585)(182.85140087,91.30962587)(182.78139282,91.28963186)
\curveto(182.70140102,91.2796259)(182.6214011,91.25962592)(182.54139282,91.22963186)
\curveto(182.30140142,91.11962606)(182.10140162,90.96962621)(181.94139282,90.77963186)
\curveto(181.77140195,90.59962658)(181.63140209,90.3796268)(181.52139282,90.11963186)
\curveto(181.50140222,90.04962713)(181.48640223,89.9796272)(181.47639282,89.90963186)
\curveto(181.45640226,89.83962734)(181.43640228,89.76462741)(181.41639282,89.68463186)
\curveto(181.39640232,89.60462757)(181.38640233,89.49462768)(181.38639282,89.35463186)
\curveto(181.38640233,89.22462795)(181.39640232,89.11962806)(181.41639282,89.03963186)
\curveto(181.42640229,88.9796282)(181.43140229,88.92462825)(181.43139282,88.87463186)
\curveto(181.43140229,88.82462835)(181.44140228,88.7746284)(181.46139282,88.72463186)
\curveto(181.50140222,88.62462855)(181.54140218,88.52962865)(181.58139282,88.43963186)
\curveto(181.6214021,88.35962882)(181.66640205,88.2796289)(181.71639282,88.19963186)
\curveto(181.73640198,88.16962901)(181.76140196,88.13962904)(181.79139282,88.10963186)
\curveto(181.8214019,88.08962909)(181.84640187,88.06462911)(181.86639282,88.03463186)
\lineto(181.94139282,87.95963186)
\curveto(181.96140176,87.92962925)(181.98140174,87.90462927)(182.00139282,87.88463186)
\lineto(182.21139282,87.73463186)
\curveto(182.27140145,87.69462948)(182.33640138,87.64962953)(182.40639282,87.59963186)
\curveto(182.49640122,87.53962964)(182.60140112,87.48962969)(182.72139282,87.44963186)
\curveto(182.83140089,87.41962976)(182.94140078,87.38462979)(183.05139282,87.34463186)
\curveto(183.16140056,87.30462987)(183.30640041,87.2796299)(183.48639282,87.26963186)
\curveto(183.65640006,87.25962992)(183.78139994,87.22962995)(183.86139282,87.17963186)
\curveto(183.94139978,87.12963005)(183.98639973,87.05463012)(183.99639282,86.95463186)
\curveto(184.00639971,86.85463032)(184.01139971,86.74463043)(184.01139282,86.62463186)
\curveto(184.01139971,86.58463059)(184.0163997,86.54463063)(184.02639282,86.50463186)
\curveto(184.02639969,86.46463071)(184.0213997,86.42963075)(184.01139282,86.39963186)
\curveto(183.99139973,86.34963083)(183.98139974,86.29963088)(183.98139282,86.24963186)
\curveto(183.98139974,86.20963097)(183.97139975,86.16963101)(183.95139282,86.12963186)
\curveto(183.89139983,86.03963114)(183.75639996,85.99463118)(183.54639282,85.99463186)
\lineto(183.42639282,85.99463186)
\curveto(183.36640035,86.00463117)(183.30640041,86.00963117)(183.24639282,86.00963186)
\curveto(183.17640054,86.01963116)(183.11140061,86.02963115)(183.05139282,86.03963186)
\curveto(182.94140078,86.05963112)(182.84140088,86.0796311)(182.75139282,86.09963186)
\curveto(182.65140107,86.11963106)(182.55640116,86.14963103)(182.46639282,86.18963186)
\curveto(182.39640132,86.20963097)(182.33640138,86.22963095)(182.28639282,86.24963186)
\lineto(182.10639282,86.30963186)
\curveto(181.84640187,86.42963075)(181.60140212,86.58463059)(181.37139282,86.77463186)
\curveto(181.14140258,86.9746302)(180.95640276,87.18962999)(180.81639282,87.41963186)
\curveto(180.73640298,87.52962965)(180.67140305,87.64462953)(180.62139282,87.76463186)
\lineto(180.47139282,88.15463186)
\curveto(180.4214033,88.26462891)(180.39140333,88.3796288)(180.38139282,88.49963186)
\curveto(180.36140336,88.61962856)(180.33640338,88.74462843)(180.30639282,88.87463186)
\curveto(180.30640341,88.94462823)(180.30640341,89.00962817)(180.30639282,89.06963186)
\curveto(180.29640342,89.12962805)(180.28640343,89.19462798)(180.27639282,89.26463186)
}
}
{
\newrgbcolor{curcolor}{0 0 0}
\pscustom[linestyle=none,fillstyle=solid,fillcolor=curcolor]
{
\newpath
\moveto(185.79639282,101.36424123)
\lineto(186.05139282,101.36424123)
\curveto(186.13139759,101.37423353)(186.20639751,101.36923353)(186.27639282,101.34924123)
\lineto(186.51639282,101.34924123)
\lineto(186.68139282,101.34924123)
\curveto(186.78139694,101.32923357)(186.88639683,101.31923358)(186.99639282,101.31924123)
\curveto(187.09639662,101.31923358)(187.19639652,101.30923359)(187.29639282,101.28924123)
\lineto(187.44639282,101.28924123)
\curveto(187.58639613,101.25923364)(187.72639599,101.23923366)(187.86639282,101.22924123)
\curveto(187.99639572,101.21923368)(188.12639559,101.19423371)(188.25639282,101.15424123)
\curveto(188.33639538,101.13423377)(188.4213953,101.11423379)(188.51139282,101.09424123)
\lineto(188.75139282,101.03424123)
\lineto(189.05139282,100.91424123)
\curveto(189.14139458,100.88423402)(189.23139449,100.84923405)(189.32139282,100.80924123)
\curveto(189.54139418,100.70923419)(189.75639396,100.57423433)(189.96639282,100.40424123)
\curveto(190.17639354,100.24423466)(190.34639337,100.06923483)(190.47639282,99.87924123)
\curveto(190.5163932,99.82923507)(190.55639316,99.76923513)(190.59639282,99.69924123)
\curveto(190.62639309,99.63923526)(190.66139306,99.57923532)(190.70139282,99.51924123)
\curveto(190.75139297,99.43923546)(190.79139293,99.34423556)(190.82139282,99.23424123)
\curveto(190.85139287,99.12423578)(190.88139284,99.01923588)(190.91139282,98.91924123)
\curveto(190.95139277,98.80923609)(190.97639274,98.6992362)(190.98639282,98.58924123)
\curveto(190.99639272,98.47923642)(191.01139271,98.36423654)(191.03139282,98.24424123)
\curveto(191.04139268,98.2042367)(191.04139268,98.15923674)(191.03139282,98.10924123)
\curveto(191.03139269,98.06923683)(191.03639268,98.02923687)(191.04639282,97.98924123)
\curveto(191.05639266,97.94923695)(191.06139266,97.89423701)(191.06139282,97.82424123)
\curveto(191.06139266,97.75423715)(191.05639266,97.7042372)(191.04639282,97.67424123)
\curveto(191.02639269,97.62423728)(191.0213927,97.57923732)(191.03139282,97.53924123)
\curveto(191.04139268,97.4992374)(191.04139268,97.46423744)(191.03139282,97.43424123)
\lineto(191.03139282,97.34424123)
\curveto(191.01139271,97.28423762)(190.99639272,97.21923768)(190.98639282,97.14924123)
\curveto(190.98639273,97.08923781)(190.98139274,97.02423788)(190.97139282,96.95424123)
\curveto(190.9213928,96.78423812)(190.87139285,96.62423828)(190.82139282,96.47424123)
\curveto(190.77139295,96.32423858)(190.70639301,96.17923872)(190.62639282,96.03924123)
\curveto(190.58639313,95.98923891)(190.55639316,95.93423897)(190.53639282,95.87424123)
\curveto(190.50639321,95.82423908)(190.47139325,95.77423913)(190.43139282,95.72424123)
\curveto(190.25139347,95.48423942)(190.03139369,95.28423962)(189.77139282,95.12424123)
\curveto(189.51139421,94.96423994)(189.22639449,94.82424008)(188.91639282,94.70424123)
\curveto(188.77639494,94.64424026)(188.63639508,94.5992403)(188.49639282,94.56924123)
\curveto(188.34639537,94.53924036)(188.19139553,94.5042404)(188.03139282,94.46424123)
\curveto(187.9213958,94.44424046)(187.81139591,94.42924047)(187.70139282,94.41924123)
\curveto(187.59139613,94.40924049)(187.48139624,94.39424051)(187.37139282,94.37424123)
\curveto(187.33139639,94.36424054)(187.29139643,94.35924054)(187.25139282,94.35924123)
\curveto(187.21139651,94.36924053)(187.17139655,94.36924053)(187.13139282,94.35924123)
\curveto(187.08139664,94.34924055)(187.03139669,94.34424056)(186.98139282,94.34424123)
\lineto(186.81639282,94.34424123)
\curveto(186.76639695,94.32424058)(186.716397,94.31924058)(186.66639282,94.32924123)
\curveto(186.60639711,94.33924056)(186.55139717,94.33924056)(186.50139282,94.32924123)
\curveto(186.46139726,94.31924058)(186.4163973,94.31924058)(186.36639282,94.32924123)
\curveto(186.3163974,94.33924056)(186.26639745,94.33424057)(186.21639282,94.31424123)
\curveto(186.14639757,94.29424061)(186.07139765,94.28924061)(185.99139282,94.29924123)
\curveto(185.90139782,94.30924059)(185.8163979,94.31424059)(185.73639282,94.31424123)
\curveto(185.64639807,94.31424059)(185.54639817,94.30924059)(185.43639282,94.29924123)
\curveto(185.3163984,94.28924061)(185.2163985,94.29424061)(185.13639282,94.31424123)
\lineto(184.85139282,94.31424123)
\lineto(184.22139282,94.35924123)
\curveto(184.1213996,94.36924053)(184.02639969,94.37924052)(183.93639282,94.38924123)
\lineto(183.63639282,94.41924123)
\curveto(183.58640013,94.43924046)(183.53640018,94.44424046)(183.48639282,94.43424123)
\curveto(183.42640029,94.43424047)(183.37140035,94.44424046)(183.32139282,94.46424123)
\curveto(183.15140057,94.51424039)(182.98640073,94.55424035)(182.82639282,94.58424123)
\curveto(182.65640106,94.61424029)(182.49640122,94.66424024)(182.34639282,94.73424123)
\curveto(181.88640183,94.92423998)(181.51140221,95.14423976)(181.22139282,95.39424123)
\curveto(180.93140279,95.65423925)(180.68640303,96.01423889)(180.48639282,96.47424123)
\curveto(180.43640328,96.6042383)(180.40140332,96.73423817)(180.38139282,96.86424123)
\curveto(180.36140336,97.0042379)(180.33640338,97.14423776)(180.30639282,97.28424123)
\curveto(180.29640342,97.35423755)(180.29140343,97.41923748)(180.29139282,97.47924123)
\curveto(180.29140343,97.53923736)(180.28640343,97.6042373)(180.27639282,97.67424123)
\curveto(180.25640346,98.5042364)(180.40640331,99.17423573)(180.72639282,99.68424123)
\curveto(181.03640268,100.19423471)(181.47640224,100.57423433)(182.04639282,100.82424123)
\curveto(182.16640155,100.87423403)(182.29140143,100.91923398)(182.42139282,100.95924123)
\curveto(182.55140117,100.9992339)(182.68640103,101.04423386)(182.82639282,101.09424123)
\curveto(182.90640081,101.11423379)(182.99140073,101.12923377)(183.08139282,101.13924123)
\lineto(183.32139282,101.19924123)
\curveto(183.43140029,101.22923367)(183.54140018,101.24423366)(183.65139282,101.24424123)
\curveto(183.76139996,101.25423365)(183.87139985,101.26923363)(183.98139282,101.28924123)
\curveto(184.03139969,101.30923359)(184.07639964,101.31423359)(184.11639282,101.30424123)
\curveto(184.15639956,101.3042336)(184.19639952,101.30923359)(184.23639282,101.31924123)
\curveto(184.28639943,101.32923357)(184.34139938,101.32923357)(184.40139282,101.31924123)
\curveto(184.45139927,101.31923358)(184.50139922,101.32423358)(184.55139282,101.33424123)
\lineto(184.68639282,101.33424123)
\curveto(184.74639897,101.35423355)(184.8163989,101.35423355)(184.89639282,101.33424123)
\curveto(184.96639875,101.32423358)(185.03139869,101.32923357)(185.09139282,101.34924123)
\curveto(185.1213986,101.35923354)(185.16139856,101.36423354)(185.21139282,101.36424123)
\lineto(185.33139282,101.36424123)
\lineto(185.79639282,101.36424123)
\moveto(188.12139282,99.81924123)
\curveto(187.80139592,99.91923498)(187.43639628,99.97923492)(187.02639282,99.99924123)
\curveto(186.6163971,100.01923488)(186.20639751,100.02923487)(185.79639282,100.02924123)
\curveto(185.36639835,100.02923487)(184.94639877,100.01923488)(184.53639282,99.99924123)
\curveto(184.12639959,99.97923492)(183.74139998,99.93423497)(183.38139282,99.86424123)
\curveto(183.0214007,99.79423511)(182.70140102,99.68423522)(182.42139282,99.53424123)
\curveto(182.13140159,99.39423551)(181.89640182,99.1992357)(181.71639282,98.94924123)
\curveto(181.60640211,98.78923611)(181.52640219,98.60923629)(181.47639282,98.40924123)
\curveto(181.4164023,98.20923669)(181.38640233,97.96423694)(181.38639282,97.67424123)
\curveto(181.40640231,97.65423725)(181.4164023,97.61923728)(181.41639282,97.56924123)
\curveto(181.40640231,97.51923738)(181.40640231,97.47923742)(181.41639282,97.44924123)
\curveto(181.43640228,97.36923753)(181.45640226,97.29423761)(181.47639282,97.22424123)
\curveto(181.48640223,97.16423774)(181.50640221,97.0992378)(181.53639282,97.02924123)
\curveto(181.65640206,96.75923814)(181.82640189,96.53923836)(182.04639282,96.36924123)
\curveto(182.25640146,96.20923869)(182.50140122,96.07423883)(182.78139282,95.96424123)
\curveto(182.89140083,95.91423899)(183.01140071,95.87423903)(183.14139282,95.84424123)
\curveto(183.26140046,95.82423908)(183.38640033,95.7992391)(183.51639282,95.76924123)
\curveto(183.56640015,95.74923915)(183.6214001,95.73923916)(183.68139282,95.73924123)
\curveto(183.73139999,95.73923916)(183.78139994,95.73423917)(183.83139282,95.72424123)
\curveto(183.9213998,95.71423919)(184.0163997,95.7042392)(184.11639282,95.69424123)
\curveto(184.20639951,95.68423922)(184.30139942,95.67423923)(184.40139282,95.66424123)
\curveto(184.48139924,95.66423924)(184.56639915,95.65923924)(184.65639282,95.64924123)
\lineto(184.89639282,95.64924123)
\lineto(185.07639282,95.64924123)
\curveto(185.10639861,95.63923926)(185.14139858,95.63423927)(185.18139282,95.63424123)
\lineto(185.31639282,95.63424123)
\lineto(185.76639282,95.63424123)
\curveto(185.84639787,95.63423927)(185.93139779,95.62923927)(186.02139282,95.61924123)
\curveto(186.10139762,95.61923928)(186.17639754,95.62923927)(186.24639282,95.64924123)
\lineto(186.51639282,95.64924123)
\curveto(186.53639718,95.64923925)(186.56639715,95.64423926)(186.60639282,95.63424123)
\curveto(186.63639708,95.63423927)(186.66139706,95.63923926)(186.68139282,95.64924123)
\curveto(186.78139694,95.65923924)(186.88139684,95.66423924)(186.98139282,95.66424123)
\curveto(187.07139665,95.67423923)(187.17139655,95.68423922)(187.28139282,95.69424123)
\curveto(187.40139632,95.72423918)(187.52639619,95.73923916)(187.65639282,95.73924123)
\curveto(187.77639594,95.74923915)(187.89139583,95.77423913)(188.00139282,95.81424123)
\curveto(188.30139542,95.89423901)(188.56639515,95.97923892)(188.79639282,96.06924123)
\curveto(189.02639469,96.16923873)(189.24139448,96.31423859)(189.44139282,96.50424123)
\curveto(189.64139408,96.71423819)(189.79139393,96.97923792)(189.89139282,97.29924123)
\curveto(189.91139381,97.33923756)(189.9213938,97.37423753)(189.92139282,97.40424123)
\curveto(189.91139381,97.44423746)(189.9163938,97.48923741)(189.93639282,97.53924123)
\curveto(189.94639377,97.57923732)(189.95639376,97.64923725)(189.96639282,97.74924123)
\curveto(189.97639374,97.85923704)(189.97139375,97.94423696)(189.95139282,98.00424123)
\curveto(189.93139379,98.07423683)(189.9213938,98.14423676)(189.92139282,98.21424123)
\curveto(189.91139381,98.28423662)(189.89639382,98.34923655)(189.87639282,98.40924123)
\curveto(189.8163939,98.60923629)(189.73139399,98.78923611)(189.62139282,98.94924123)
\curveto(189.60139412,98.97923592)(189.58139414,99.0042359)(189.56139282,99.02424123)
\lineto(189.50139282,99.08424123)
\curveto(189.48139424,99.12423578)(189.44139428,99.17423573)(189.38139282,99.23424123)
\curveto(189.24139448,99.33423557)(189.11139461,99.41923548)(188.99139282,99.48924123)
\curveto(188.87139485,99.55923534)(188.72639499,99.62923527)(188.55639282,99.69924123)
\curveto(188.48639523,99.72923517)(188.4163953,99.74923515)(188.34639282,99.75924123)
\curveto(188.27639544,99.77923512)(188.20139552,99.7992351)(188.12139282,99.81924123)
}
}
{
\newrgbcolor{curcolor}{0 0 0}
\pscustom[linestyle=none,fillstyle=solid,fillcolor=curcolor]
{
\newpath
\moveto(180.27639282,106.77385061)
\curveto(180.27640344,106.87384575)(180.28640343,106.96884566)(180.30639282,107.05885061)
\curveto(180.3164034,107.14884548)(180.34640337,107.21384541)(180.39639282,107.25385061)
\curveto(180.47640324,107.31384531)(180.58140314,107.34384528)(180.71139282,107.34385061)
\lineto(181.10139282,107.34385061)
\lineto(182.60139282,107.34385061)
\lineto(188.99139282,107.34385061)
\lineto(190.16139282,107.34385061)
\lineto(190.47639282,107.34385061)
\curveto(190.57639314,107.35384527)(190.65639306,107.33884529)(190.71639282,107.29885061)
\curveto(190.79639292,107.24884538)(190.84639287,107.17384545)(190.86639282,107.07385061)
\curveto(190.87639284,106.98384564)(190.88139284,106.87384575)(190.88139282,106.74385061)
\lineto(190.88139282,106.51885061)
\curveto(190.86139286,106.43884619)(190.84639287,106.36884626)(190.83639282,106.30885061)
\curveto(190.8163929,106.24884638)(190.77639294,106.19884643)(190.71639282,106.15885061)
\curveto(190.65639306,106.11884651)(190.58139314,106.09884653)(190.49139282,106.09885061)
\lineto(190.19139282,106.09885061)
\lineto(189.09639282,106.09885061)
\lineto(183.75639282,106.09885061)
\curveto(183.66640005,106.07884655)(183.59140013,106.06384656)(183.53139282,106.05385061)
\curveto(183.46140026,106.05384657)(183.40140032,106.0238466)(183.35139282,105.96385061)
\curveto(183.30140042,105.89384673)(183.27640044,105.80384682)(183.27639282,105.69385061)
\curveto(183.26640045,105.59384703)(183.26140046,105.48384714)(183.26139282,105.36385061)
\lineto(183.26139282,104.22385061)
\lineto(183.26139282,103.72885061)
\curveto(183.25140047,103.56884906)(183.19140053,103.45884917)(183.08139282,103.39885061)
\curveto(183.05140067,103.37884925)(183.0214007,103.36884926)(182.99139282,103.36885061)
\curveto(182.95140077,103.36884926)(182.90640081,103.36384926)(182.85639282,103.35385061)
\curveto(182.73640098,103.33384929)(182.62640109,103.33884929)(182.52639282,103.36885061)
\curveto(182.42640129,103.40884922)(182.35640136,103.46384916)(182.31639282,103.53385061)
\curveto(182.26640145,103.61384901)(182.24140148,103.73384889)(182.24139282,103.89385061)
\curveto(182.24140148,104.05384857)(182.22640149,104.18884844)(182.19639282,104.29885061)
\curveto(182.18640153,104.34884828)(182.18140154,104.40384822)(182.18139282,104.46385061)
\curveto(182.17140155,104.5238481)(182.15640156,104.58384804)(182.13639282,104.64385061)
\curveto(182.08640163,104.79384783)(182.03640168,104.93884769)(181.98639282,105.07885061)
\curveto(181.92640179,105.21884741)(181.85640186,105.35384727)(181.77639282,105.48385061)
\curveto(181.68640203,105.623847)(181.58140214,105.74384688)(181.46139282,105.84385061)
\curveto(181.34140238,105.94384668)(181.21140251,106.03884659)(181.07139282,106.12885061)
\curveto(180.97140275,106.18884644)(180.86140286,106.23384639)(180.74139282,106.26385061)
\curveto(180.6214031,106.30384632)(180.5164032,106.35384627)(180.42639282,106.41385061)
\curveto(180.36640335,106.46384616)(180.32640339,106.53384609)(180.30639282,106.62385061)
\curveto(180.29640342,106.64384598)(180.29140343,106.66884596)(180.29139282,106.69885061)
\curveto(180.29140343,106.7288459)(180.28640343,106.75384587)(180.27639282,106.77385061)
}
}
{
\newrgbcolor{curcolor}{0 0 0}
\pscustom[linestyle=none,fillstyle=solid,fillcolor=curcolor]
{
\newpath
\moveto(180.27639282,115.12345998)
\curveto(180.27640344,115.22345513)(180.28640343,115.31845503)(180.30639282,115.40845998)
\curveto(180.3164034,115.49845485)(180.34640337,115.56345479)(180.39639282,115.60345998)
\curveto(180.47640324,115.66345469)(180.58140314,115.69345466)(180.71139282,115.69345998)
\lineto(181.10139282,115.69345998)
\lineto(182.60139282,115.69345998)
\lineto(188.99139282,115.69345998)
\lineto(190.16139282,115.69345998)
\lineto(190.47639282,115.69345998)
\curveto(190.57639314,115.70345465)(190.65639306,115.68845466)(190.71639282,115.64845998)
\curveto(190.79639292,115.59845475)(190.84639287,115.52345483)(190.86639282,115.42345998)
\curveto(190.87639284,115.33345502)(190.88139284,115.22345513)(190.88139282,115.09345998)
\lineto(190.88139282,114.86845998)
\curveto(190.86139286,114.78845556)(190.84639287,114.71845563)(190.83639282,114.65845998)
\curveto(190.8163929,114.59845575)(190.77639294,114.5484558)(190.71639282,114.50845998)
\curveto(190.65639306,114.46845588)(190.58139314,114.4484559)(190.49139282,114.44845998)
\lineto(190.19139282,114.44845998)
\lineto(189.09639282,114.44845998)
\lineto(183.75639282,114.44845998)
\curveto(183.66640005,114.42845592)(183.59140013,114.41345594)(183.53139282,114.40345998)
\curveto(183.46140026,114.40345595)(183.40140032,114.37345598)(183.35139282,114.31345998)
\curveto(183.30140042,114.24345611)(183.27640044,114.1534562)(183.27639282,114.04345998)
\curveto(183.26640045,113.94345641)(183.26140046,113.83345652)(183.26139282,113.71345998)
\lineto(183.26139282,112.57345998)
\lineto(183.26139282,112.07845998)
\curveto(183.25140047,111.91845843)(183.19140053,111.80845854)(183.08139282,111.74845998)
\curveto(183.05140067,111.72845862)(183.0214007,111.71845863)(182.99139282,111.71845998)
\curveto(182.95140077,111.71845863)(182.90640081,111.71345864)(182.85639282,111.70345998)
\curveto(182.73640098,111.68345867)(182.62640109,111.68845866)(182.52639282,111.71845998)
\curveto(182.42640129,111.75845859)(182.35640136,111.81345854)(182.31639282,111.88345998)
\curveto(182.26640145,111.96345839)(182.24140148,112.08345827)(182.24139282,112.24345998)
\curveto(182.24140148,112.40345795)(182.22640149,112.53845781)(182.19639282,112.64845998)
\curveto(182.18640153,112.69845765)(182.18140154,112.7534576)(182.18139282,112.81345998)
\curveto(182.17140155,112.87345748)(182.15640156,112.93345742)(182.13639282,112.99345998)
\curveto(182.08640163,113.14345721)(182.03640168,113.28845706)(181.98639282,113.42845998)
\curveto(181.92640179,113.56845678)(181.85640186,113.70345665)(181.77639282,113.83345998)
\curveto(181.68640203,113.97345638)(181.58140214,114.09345626)(181.46139282,114.19345998)
\curveto(181.34140238,114.29345606)(181.21140251,114.38845596)(181.07139282,114.47845998)
\curveto(180.97140275,114.53845581)(180.86140286,114.58345577)(180.74139282,114.61345998)
\curveto(180.6214031,114.6534557)(180.5164032,114.70345565)(180.42639282,114.76345998)
\curveto(180.36640335,114.81345554)(180.32640339,114.88345547)(180.30639282,114.97345998)
\curveto(180.29640342,114.99345536)(180.29140343,115.01845533)(180.29139282,115.04845998)
\curveto(180.29140343,115.07845527)(180.28640343,115.10345525)(180.27639282,115.12345998)
}
}
{
\newrgbcolor{curcolor}{0 0 0}
\pscustom[linestyle=none,fillstyle=solid,fillcolor=curcolor]
{
\newpath
\moveto(138.37873047,31.67142873)
\lineto(138.37873047,32.58642873)
\curveto(138.37874116,32.68642608)(138.37874116,32.78142599)(138.37873047,32.87142873)
\curveto(138.37874116,32.96142581)(138.39874114,33.03642573)(138.43873047,33.09642873)
\curveto(138.49874104,33.18642558)(138.57874096,33.24642552)(138.67873047,33.27642873)
\curveto(138.77874076,33.31642545)(138.88374066,33.36142541)(138.99373047,33.41142873)
\curveto(139.18374036,33.49142528)(139.37374017,33.56142521)(139.56373047,33.62142873)
\curveto(139.75373979,33.69142508)(139.9437396,33.766425)(140.13373047,33.84642873)
\curveto(140.31373923,33.91642485)(140.49873904,33.98142479)(140.68873047,34.04142873)
\curveto(140.86873867,34.10142467)(141.04873849,34.1714246)(141.22873047,34.25142873)
\curveto(141.36873817,34.31142446)(141.51373803,34.3664244)(141.66373047,34.41642873)
\curveto(141.81373773,34.4664243)(141.95873758,34.52142425)(142.09873047,34.58142873)
\curveto(142.54873699,34.76142401)(143.00373654,34.93142384)(143.46373047,35.09142873)
\curveto(143.91373563,35.25142352)(144.36373518,35.42142335)(144.81373047,35.60142873)
\curveto(144.86373468,35.62142315)(144.91373463,35.63642313)(144.96373047,35.64642873)
\lineto(145.11373047,35.70642873)
\curveto(145.33373421,35.79642297)(145.55873398,35.88142289)(145.78873047,35.96142873)
\curveto(146.00873353,36.04142273)(146.22873331,36.12642264)(146.44873047,36.21642873)
\curveto(146.538733,36.25642251)(146.64873289,36.29642247)(146.77873047,36.33642873)
\curveto(146.89873264,36.37642239)(146.96873257,36.44142233)(146.98873047,36.53142873)
\curveto(146.99873254,36.5714222)(146.99873254,36.60142217)(146.98873047,36.62142873)
\lineto(146.92873047,36.68142873)
\curveto(146.87873266,36.73142204)(146.82373272,36.766422)(146.76373047,36.78642873)
\curveto(146.70373284,36.81642195)(146.6387329,36.84642192)(146.56873047,36.87642873)
\lineto(145.93873047,37.11642873)
\curveto(145.71873382,37.19642157)(145.50373404,37.27642149)(145.29373047,37.35642873)
\lineto(145.14373047,37.41642873)
\lineto(144.96373047,37.47642873)
\curveto(144.77373477,37.55642121)(144.58373496,37.62642114)(144.39373047,37.68642873)
\curveto(144.19373535,37.75642101)(143.99373555,37.83142094)(143.79373047,37.91142873)
\curveto(143.21373633,38.15142062)(142.62873691,38.3714204)(142.03873047,38.57142873)
\curveto(141.44873809,38.78141999)(140.86373868,39.00641976)(140.28373047,39.24642873)
\curveto(140.08373946,39.32641944)(139.87873966,39.40141937)(139.66873047,39.47142873)
\curveto(139.45874008,39.55141922)(139.25374029,39.63141914)(139.05373047,39.71142873)
\curveto(138.97374057,39.75141902)(138.87374067,39.78641898)(138.75373047,39.81642873)
\curveto(138.63374091,39.85641891)(138.54874099,39.91141886)(138.49873047,39.98142873)
\curveto(138.45874108,40.04141873)(138.42874111,40.11641865)(138.40873047,40.20642873)
\curveto(138.38874115,40.30641846)(138.37874116,40.41641835)(138.37873047,40.53642873)
\curveto(138.36874117,40.65641811)(138.36874117,40.77641799)(138.37873047,40.89642873)
\curveto(138.37874116,41.01641775)(138.37874116,41.12641764)(138.37873047,41.22642873)
\curveto(138.37874116,41.31641745)(138.37874116,41.40641736)(138.37873047,41.49642873)
\curveto(138.37874116,41.59641717)(138.39874114,41.6714171)(138.43873047,41.72142873)
\curveto(138.48874105,41.81141696)(138.57874096,41.86141691)(138.70873047,41.87142873)
\curveto(138.8387407,41.88141689)(138.97874056,41.88641688)(139.12873047,41.88642873)
\lineto(140.77873047,41.88642873)
\lineto(147.04873047,41.88642873)
\lineto(148.30873047,41.88642873)
\curveto(148.41873112,41.88641688)(148.52873101,41.88641688)(148.63873047,41.88642873)
\curveto(148.74873079,41.89641687)(148.83373071,41.87641689)(148.89373047,41.82642873)
\curveto(148.95373059,41.79641697)(148.99373055,41.75141702)(149.01373047,41.69142873)
\curveto(149.02373052,41.63141714)(149.0387305,41.56141721)(149.05873047,41.48142873)
\lineto(149.05873047,41.24142873)
\lineto(149.05873047,40.88142873)
\curveto(149.04873049,40.771418)(149.00373054,40.69141808)(148.92373047,40.64142873)
\curveto(148.89373065,40.62141815)(148.86373068,40.60641816)(148.83373047,40.59642873)
\curveto(148.79373075,40.59641817)(148.74873079,40.58641818)(148.69873047,40.56642873)
\lineto(148.53373047,40.56642873)
\curveto(148.47373107,40.55641821)(148.40373114,40.55141822)(148.32373047,40.55142873)
\curveto(148.2437313,40.56141821)(148.16873137,40.5664182)(148.09873047,40.56642873)
\lineto(147.25873047,40.56642873)
\lineto(142.83373047,40.56642873)
\curveto(142.58373696,40.5664182)(142.33373721,40.5664182)(142.08373047,40.56642873)
\curveto(141.82373772,40.5664182)(141.57373797,40.56141821)(141.33373047,40.55142873)
\curveto(141.23373831,40.55141822)(141.12373842,40.54641822)(141.00373047,40.53642873)
\curveto(140.88373866,40.52641824)(140.82373872,40.4714183)(140.82373047,40.37142873)
\lineto(140.83873047,40.37142873)
\curveto(140.85873868,40.30141847)(140.92373862,40.24141853)(141.03373047,40.19142873)
\curveto(141.1437384,40.15141862)(141.2387383,40.11641865)(141.31873047,40.08642873)
\curveto(141.48873805,40.01641875)(141.66373788,39.95141882)(141.84373047,39.89142873)
\curveto(142.01373753,39.83141894)(142.18373736,39.76141901)(142.35373047,39.68142873)
\curveto(142.40373714,39.66141911)(142.44873709,39.64641912)(142.48873047,39.63642873)
\curveto(142.52873701,39.62641914)(142.57373697,39.61141916)(142.62373047,39.59142873)
\curveto(142.80373674,39.51141926)(142.98873655,39.44141933)(143.17873047,39.38142873)
\curveto(143.35873618,39.33141944)(143.538736,39.2664195)(143.71873047,39.18642873)
\curveto(143.86873567,39.11641965)(144.02373552,39.05641971)(144.18373047,39.00642873)
\curveto(144.33373521,38.95641981)(144.48373506,38.90141987)(144.63373047,38.84142873)
\curveto(145.10373444,38.64142013)(145.57873396,38.46142031)(146.05873047,38.30142873)
\curveto(146.52873301,38.14142063)(146.99373255,37.9664208)(147.45373047,37.77642873)
\curveto(147.63373191,37.69642107)(147.81373173,37.62642114)(147.99373047,37.56642873)
\curveto(148.17373137,37.50642126)(148.35373119,37.44142133)(148.53373047,37.37142873)
\curveto(148.6437309,37.32142145)(148.74873079,37.2714215)(148.84873047,37.22142873)
\curveto(148.9387306,37.18142159)(149.00373054,37.09642167)(149.04373047,36.96642873)
\curveto(149.05373049,36.94642182)(149.05873048,36.92142185)(149.05873047,36.89142873)
\curveto(149.04873049,36.8714219)(149.04873049,36.84642192)(149.05873047,36.81642873)
\curveto(149.06873047,36.78642198)(149.07373047,36.75142202)(149.07373047,36.71142873)
\curveto(149.06373048,36.6714221)(149.05873048,36.63142214)(149.05873047,36.59142873)
\lineto(149.05873047,36.29142873)
\curveto(149.05873048,36.19142258)(149.03373051,36.11142266)(148.98373047,36.05142873)
\curveto(148.93373061,35.9714228)(148.86373068,35.91142286)(148.77373047,35.87142873)
\curveto(148.67373087,35.84142293)(148.57373097,35.80142297)(148.47373047,35.75142873)
\curveto(148.27373127,35.6714231)(148.06873147,35.59142318)(147.85873047,35.51142873)
\curveto(147.6387319,35.44142333)(147.42873211,35.3664234)(147.22873047,35.28642873)
\curveto(147.04873249,35.20642356)(146.86873267,35.13642363)(146.68873047,35.07642873)
\curveto(146.49873304,35.02642374)(146.31373323,34.96142381)(146.13373047,34.88142873)
\curveto(145.57373397,34.65142412)(145.00873453,34.43642433)(144.43873047,34.23642873)
\curveto(143.86873567,34.03642473)(143.30373624,33.82142495)(142.74373047,33.59142873)
\lineto(142.11373047,33.35142873)
\curveto(141.89373765,33.28142549)(141.68373786,33.20642556)(141.48373047,33.12642873)
\curveto(141.37373817,33.07642569)(141.26873827,33.03142574)(141.16873047,32.99142873)
\curveto(141.05873848,32.96142581)(140.96373858,32.91142586)(140.88373047,32.84142873)
\curveto(140.86373868,32.83142594)(140.85373869,32.82142595)(140.85373047,32.81142873)
\lineto(140.82373047,32.78142873)
\lineto(140.82373047,32.70642873)
\lineto(140.85373047,32.67642873)
\curveto(140.85373869,32.6664261)(140.85873868,32.65642611)(140.86873047,32.64642873)
\curveto(140.91873862,32.62642614)(140.97373857,32.61642615)(141.03373047,32.61642873)
\curveto(141.09373845,32.61642615)(141.15373839,32.60642616)(141.21373047,32.58642873)
\lineto(141.37873047,32.58642873)
\curveto(141.4387381,32.5664262)(141.50373804,32.56142621)(141.57373047,32.57142873)
\curveto(141.6437379,32.58142619)(141.71373783,32.58642618)(141.78373047,32.58642873)
\lineto(142.59373047,32.58642873)
\lineto(147.15373047,32.58642873)
\lineto(148.33873047,32.58642873)
\curveto(148.44873109,32.58642618)(148.55873098,32.58142619)(148.66873047,32.57142873)
\curveto(148.77873076,32.5714262)(148.86373068,32.54642622)(148.92373047,32.49642873)
\curveto(149.00373054,32.44642632)(149.04873049,32.35642641)(149.05873047,32.22642873)
\lineto(149.05873047,31.83642873)
\lineto(149.05873047,31.64142873)
\curveto(149.05873048,31.59142718)(149.04873049,31.54142723)(149.02873047,31.49142873)
\curveto(148.98873055,31.36142741)(148.90373064,31.28642748)(148.77373047,31.26642873)
\curveto(148.6437309,31.25642751)(148.49373105,31.25142752)(148.32373047,31.25142873)
\lineto(146.58373047,31.25142873)
\lineto(140.58373047,31.25142873)
\lineto(139.17373047,31.25142873)
\curveto(139.06374048,31.25142752)(138.94874059,31.24642752)(138.82873047,31.23642873)
\curveto(138.70874083,31.23642753)(138.61374093,31.26142751)(138.54373047,31.31142873)
\curveto(138.48374106,31.35142742)(138.43374111,31.42642734)(138.39373047,31.53642873)
\curveto(138.38374116,31.55642721)(138.38374116,31.57642719)(138.39373047,31.59642873)
\curveto(138.39374115,31.62642714)(138.38874115,31.65142712)(138.37873047,31.67142873)
}
}
{
\newrgbcolor{curcolor}{0 0 0}
\pscustom[linestyle=none,fillstyle=solid,fillcolor=curcolor]
{
\newpath
\moveto(148.50373047,50.87353811)
\curveto(148.66373088,50.90353028)(148.79873074,50.88853029)(148.90873047,50.82853811)
\curveto(149.00873053,50.76853041)(149.08373046,50.68853049)(149.13373047,50.58853811)
\curveto(149.15373039,50.53853064)(149.16373038,50.4835307)(149.16373047,50.42353811)
\curveto(149.16373038,50.37353081)(149.17373037,50.31853086)(149.19373047,50.25853811)
\curveto(149.2437303,50.03853114)(149.22873031,49.81853136)(149.14873047,49.59853811)
\curveto(149.07873046,49.38853179)(148.98873055,49.24353194)(148.87873047,49.16353811)
\curveto(148.80873073,49.11353207)(148.72873081,49.06853211)(148.63873047,49.02853811)
\curveto(148.538731,48.98853219)(148.45873108,48.93853224)(148.39873047,48.87853811)
\curveto(148.37873116,48.85853232)(148.35873118,48.83353235)(148.33873047,48.80353811)
\curveto(148.31873122,48.7835324)(148.31373123,48.75353243)(148.32373047,48.71353811)
\curveto(148.35373119,48.60353258)(148.40873113,48.49853268)(148.48873047,48.39853811)
\curveto(148.56873097,48.30853287)(148.6387309,48.21853296)(148.69873047,48.12853811)
\curveto(148.77873076,47.99853318)(148.85373069,47.85853332)(148.92373047,47.70853811)
\curveto(148.98373056,47.55853362)(149.0387305,47.39853378)(149.08873047,47.22853811)
\curveto(149.11873042,47.12853405)(149.1387304,47.01853416)(149.14873047,46.89853811)
\curveto(149.15873038,46.78853439)(149.17373037,46.6785345)(149.19373047,46.56853811)
\curveto(149.20373034,46.51853466)(149.20873033,46.47353471)(149.20873047,46.43353811)
\lineto(149.20873047,46.32853811)
\curveto(149.22873031,46.21853496)(149.22873031,46.11353507)(149.20873047,46.01353811)
\lineto(149.20873047,45.87853811)
\curveto(149.19873034,45.82853535)(149.19373035,45.7785354)(149.19373047,45.72853811)
\curveto(149.19373035,45.6785355)(149.18373036,45.63353555)(149.16373047,45.59353811)
\curveto(149.15373039,45.55353563)(149.14873039,45.51853566)(149.14873047,45.48853811)
\curveto(149.15873038,45.46853571)(149.15873038,45.44353574)(149.14873047,45.41353811)
\lineto(149.08873047,45.17353811)
\curveto(149.07873046,45.09353609)(149.05873048,45.01853616)(149.02873047,44.94853811)
\curveto(148.89873064,44.64853653)(148.75373079,44.40353678)(148.59373047,44.21353811)
\curveto(148.42373112,44.03353715)(148.18873135,43.8835373)(147.88873047,43.76353811)
\curveto(147.66873187,43.67353751)(147.40373214,43.62853755)(147.09373047,43.62853811)
\lineto(146.77873047,43.62853811)
\curveto(146.72873281,43.63853754)(146.67873286,43.64353754)(146.62873047,43.64353811)
\lineto(146.44873047,43.67353811)
\lineto(146.11873047,43.79353811)
\curveto(146.00873353,43.83353735)(145.90873363,43.8835373)(145.81873047,43.94353811)
\curveto(145.52873401,44.12353706)(145.31373423,44.36853681)(145.17373047,44.67853811)
\curveto(145.03373451,44.98853619)(144.90873463,45.32853585)(144.79873047,45.69853811)
\curveto(144.75873478,45.83853534)(144.72873481,45.9835352)(144.70873047,46.13353811)
\curveto(144.68873485,46.2835349)(144.66373488,46.43353475)(144.63373047,46.58353811)
\curveto(144.61373493,46.65353453)(144.60373494,46.71853446)(144.60373047,46.77853811)
\curveto(144.60373494,46.84853433)(144.59373495,46.92353426)(144.57373047,47.00353811)
\curveto(144.55373499,47.07353411)(144.543735,47.14353404)(144.54373047,47.21353811)
\curveto(144.53373501,47.2835339)(144.51873502,47.35853382)(144.49873047,47.43853811)
\curveto(144.4387351,47.68853349)(144.38873515,47.92353326)(144.34873047,48.14353811)
\curveto(144.29873524,48.36353282)(144.18373536,48.53853264)(144.00373047,48.66853811)
\curveto(143.92373562,48.72853245)(143.82373572,48.7785324)(143.70373047,48.81853811)
\curveto(143.57373597,48.85853232)(143.43373611,48.85853232)(143.28373047,48.81853811)
\curveto(143.0437365,48.75853242)(142.85373669,48.66853251)(142.71373047,48.54853811)
\curveto(142.57373697,48.43853274)(142.46373708,48.2785329)(142.38373047,48.06853811)
\curveto(142.33373721,47.94853323)(142.29873724,47.80353338)(142.27873047,47.63353811)
\curveto(142.25873728,47.47353371)(142.24873729,47.30353388)(142.24873047,47.12353811)
\curveto(142.24873729,46.94353424)(142.25873728,46.76853441)(142.27873047,46.59853811)
\curveto(142.29873724,46.42853475)(142.32873721,46.2835349)(142.36873047,46.16353811)
\curveto(142.42873711,45.99353519)(142.51373703,45.82853535)(142.62373047,45.66853811)
\curveto(142.68373686,45.58853559)(142.76373678,45.51353567)(142.86373047,45.44353811)
\curveto(142.95373659,45.3835358)(143.05373649,45.32853585)(143.16373047,45.27853811)
\curveto(143.2437363,45.24853593)(143.32873621,45.21853596)(143.41873047,45.18853811)
\curveto(143.50873603,45.16853601)(143.57873596,45.12353606)(143.62873047,45.05353811)
\curveto(143.65873588,45.01353617)(143.68373586,44.94353624)(143.70373047,44.84353811)
\curveto(143.71373583,44.75353643)(143.71873582,44.65853652)(143.71873047,44.55853811)
\curveto(143.71873582,44.45853672)(143.71373583,44.35853682)(143.70373047,44.25853811)
\curveto(143.68373586,44.16853701)(143.65873588,44.10353708)(143.62873047,44.06353811)
\curveto(143.59873594,44.02353716)(143.54873599,43.99353719)(143.47873047,43.97353811)
\curveto(143.40873613,43.95353723)(143.33373621,43.95353723)(143.25373047,43.97353811)
\curveto(143.12373642,44.00353718)(143.00373654,44.03353715)(142.89373047,44.06353811)
\curveto(142.77373677,44.10353708)(142.65873688,44.14853703)(142.54873047,44.19853811)
\curveto(142.19873734,44.38853679)(141.92873761,44.62853655)(141.73873047,44.91853811)
\curveto(141.538738,45.20853597)(141.37873816,45.56853561)(141.25873047,45.99853811)
\curveto(141.2387383,46.09853508)(141.22373832,46.19853498)(141.21373047,46.29853811)
\curveto(141.20373834,46.40853477)(141.18873835,46.51853466)(141.16873047,46.62853811)
\curveto(141.15873838,46.66853451)(141.15873838,46.73353445)(141.16873047,46.82353811)
\curveto(141.16873837,46.91353427)(141.15873838,46.96853421)(141.13873047,46.98853811)
\curveto(141.12873841,47.68853349)(141.20873833,48.29853288)(141.37873047,48.81853811)
\curveto(141.54873799,49.33853184)(141.87373767,49.70353148)(142.35373047,49.91353811)
\curveto(142.55373699,50.00353118)(142.78873675,50.05353113)(143.05873047,50.06353811)
\curveto(143.31873622,50.0835311)(143.59373595,50.09353109)(143.88373047,50.09353811)
\lineto(147.19873047,50.09353811)
\curveto(147.3387322,50.09353109)(147.47373207,50.09853108)(147.60373047,50.10853811)
\curveto(147.73373181,50.11853106)(147.8387317,50.14853103)(147.91873047,50.19853811)
\curveto(147.98873155,50.24853093)(148.0387315,50.31353087)(148.06873047,50.39353811)
\curveto(148.10873143,50.4835307)(148.1387314,50.56853061)(148.15873047,50.64853811)
\curveto(148.16873137,50.72853045)(148.21373133,50.78853039)(148.29373047,50.82853811)
\curveto(148.32373122,50.84853033)(148.35373119,50.85853032)(148.38373047,50.85853811)
\curveto(148.41373113,50.85853032)(148.45373109,50.86353032)(148.50373047,50.87353811)
\moveto(146.83873047,48.72853811)
\curveto(146.69873284,48.78853239)(146.538733,48.81853236)(146.35873047,48.81853811)
\curveto(146.16873337,48.82853235)(145.97373357,48.83353235)(145.77373047,48.83353811)
\curveto(145.66373388,48.83353235)(145.56373398,48.82853235)(145.47373047,48.81853811)
\curveto(145.38373416,48.80853237)(145.31373423,48.76853241)(145.26373047,48.69853811)
\curveto(145.2437343,48.66853251)(145.23373431,48.59853258)(145.23373047,48.48853811)
\curveto(145.25373429,48.46853271)(145.26373428,48.43353275)(145.26373047,48.38353811)
\curveto(145.26373428,48.33353285)(145.27373427,48.28853289)(145.29373047,48.24853811)
\curveto(145.31373423,48.16853301)(145.33373421,48.0785331)(145.35373047,47.97853811)
\lineto(145.41373047,47.67853811)
\curveto(145.41373413,47.64853353)(145.41873412,47.61353357)(145.42873047,47.57353811)
\lineto(145.42873047,47.46853811)
\curveto(145.46873407,47.31853386)(145.49373405,47.15353403)(145.50373047,46.97353811)
\curveto(145.50373404,46.80353438)(145.52373402,46.64353454)(145.56373047,46.49353811)
\curveto(145.58373396,46.41353477)(145.60373394,46.33853484)(145.62373047,46.26853811)
\curveto(145.63373391,46.20853497)(145.64873389,46.13853504)(145.66873047,46.05853811)
\curveto(145.71873382,45.89853528)(145.78373376,45.74853543)(145.86373047,45.60853811)
\curveto(145.93373361,45.46853571)(146.02373352,45.34853583)(146.13373047,45.24853811)
\curveto(146.2437333,45.14853603)(146.37873316,45.07353611)(146.53873047,45.02353811)
\curveto(146.68873285,44.97353621)(146.87373267,44.95353623)(147.09373047,44.96353811)
\curveto(147.19373235,44.96353622)(147.28873225,44.9785362)(147.37873047,45.00853811)
\curveto(147.45873208,45.04853613)(147.53373201,45.09353609)(147.60373047,45.14353811)
\curveto(147.71373183,45.22353596)(147.80873173,45.32853585)(147.88873047,45.45853811)
\curveto(147.95873158,45.58853559)(148.01873152,45.72853545)(148.06873047,45.87853811)
\curveto(148.07873146,45.92853525)(148.08373146,45.9785352)(148.08373047,46.02853811)
\curveto(148.08373146,46.0785351)(148.08873145,46.12853505)(148.09873047,46.17853811)
\curveto(148.11873142,46.24853493)(148.13373141,46.33353485)(148.14373047,46.43353811)
\curveto(148.1437314,46.54353464)(148.13373141,46.63353455)(148.11373047,46.70353811)
\curveto(148.09373145,46.76353442)(148.08873145,46.82353436)(148.09873047,46.88353811)
\curveto(148.09873144,46.94353424)(148.08873145,47.00353418)(148.06873047,47.06353811)
\curveto(148.04873149,47.14353404)(148.03373151,47.21853396)(148.02373047,47.28853811)
\curveto(148.01373153,47.36853381)(147.99373155,47.44353374)(147.96373047,47.51353811)
\curveto(147.8437317,47.80353338)(147.69873184,48.04853313)(147.52873047,48.24853811)
\curveto(147.35873218,48.45853272)(147.12873241,48.61853256)(146.83873047,48.72853811)
}
}
{
\newrgbcolor{curcolor}{0 0 0}
\pscustom[linestyle=none,fillstyle=solid,fillcolor=curcolor]
{
\newpath
\moveto(141.15373047,55.69017873)
\curveto(141.15373839,55.92017394)(141.21373833,56.05017381)(141.33373047,56.08017873)
\curveto(141.4437381,56.11017375)(141.60873793,56.12517374)(141.82873047,56.12517873)
\lineto(142.11373047,56.12517873)
\curveto(142.20373734,56.12517374)(142.27873726,56.10017376)(142.33873047,56.05017873)
\curveto(142.41873712,55.99017387)(142.46373708,55.90517396)(142.47373047,55.79517873)
\curveto(142.47373707,55.68517418)(142.48873705,55.57517429)(142.51873047,55.46517873)
\curveto(142.54873699,55.32517454)(142.57873696,55.19017467)(142.60873047,55.06017873)
\curveto(142.6387369,54.94017492)(142.67873686,54.82517504)(142.72873047,54.71517873)
\curveto(142.85873668,54.42517544)(143.0387365,54.19017567)(143.26873047,54.01017873)
\curveto(143.48873605,53.83017603)(143.7437358,53.67517619)(144.03373047,53.54517873)
\curveto(144.1437354,53.50517636)(144.25873528,53.47517639)(144.37873047,53.45517873)
\curveto(144.48873505,53.43517643)(144.60373494,53.41017645)(144.72373047,53.38017873)
\curveto(144.77373477,53.37017649)(144.82373472,53.3651765)(144.87373047,53.36517873)
\curveto(144.92373462,53.37517649)(144.97373457,53.37517649)(145.02373047,53.36517873)
\curveto(145.1437344,53.33517653)(145.28373426,53.32017654)(145.44373047,53.32017873)
\curveto(145.59373395,53.33017653)(145.7387338,53.33517653)(145.87873047,53.33517873)
\lineto(147.72373047,53.33517873)
\lineto(148.06873047,53.33517873)
\curveto(148.18873135,53.33517653)(148.30373124,53.33017653)(148.41373047,53.32017873)
\curveto(148.52373102,53.31017655)(148.61873092,53.30517656)(148.69873047,53.30517873)
\curveto(148.77873076,53.31517655)(148.84873069,53.29517657)(148.90873047,53.24517873)
\curveto(148.97873056,53.19517667)(149.01873052,53.11517675)(149.02873047,53.00517873)
\curveto(149.0387305,52.90517696)(149.0437305,52.79517707)(149.04373047,52.67517873)
\lineto(149.04373047,52.40517873)
\curveto(149.02373052,52.35517751)(149.00873053,52.30517756)(148.99873047,52.25517873)
\curveto(148.97873056,52.21517765)(148.95373059,52.18517768)(148.92373047,52.16517873)
\curveto(148.85373069,52.11517775)(148.76873077,52.08517778)(148.66873047,52.07517873)
\lineto(148.33873047,52.07517873)
\lineto(147.18373047,52.07517873)
\lineto(143.02873047,52.07517873)
\lineto(141.99373047,52.07517873)
\lineto(141.69373047,52.07517873)
\curveto(141.59373795,52.08517778)(141.50873803,52.11517775)(141.43873047,52.16517873)
\curveto(141.39873814,52.19517767)(141.36873817,52.24517762)(141.34873047,52.31517873)
\curveto(141.32873821,52.39517747)(141.31873822,52.48017738)(141.31873047,52.57017873)
\curveto(141.30873823,52.6601772)(141.30873823,52.75017711)(141.31873047,52.84017873)
\curveto(141.32873821,52.93017693)(141.3437382,53.00017686)(141.36373047,53.05017873)
\curveto(141.39373815,53.13017673)(141.45373809,53.18017668)(141.54373047,53.20017873)
\curveto(141.62373792,53.23017663)(141.71373783,53.24517662)(141.81373047,53.24517873)
\lineto(142.11373047,53.24517873)
\curveto(142.21373733,53.24517662)(142.30373724,53.2651766)(142.38373047,53.30517873)
\curveto(142.40373714,53.31517655)(142.41873712,53.32517654)(142.42873047,53.33517873)
\lineto(142.47373047,53.38017873)
\curveto(142.47373707,53.49017637)(142.42873711,53.58017628)(142.33873047,53.65017873)
\curveto(142.2387373,53.72017614)(142.15873738,53.78017608)(142.09873047,53.83017873)
\lineto(142.00873047,53.92017873)
\curveto(141.89873764,54.01017585)(141.78373776,54.13517573)(141.66373047,54.29517873)
\curveto(141.543738,54.45517541)(141.45373809,54.60517526)(141.39373047,54.74517873)
\curveto(141.3437382,54.83517503)(141.30873823,54.93017493)(141.28873047,55.03017873)
\curveto(141.25873828,55.13017473)(141.22873831,55.23517463)(141.19873047,55.34517873)
\curveto(141.18873835,55.40517446)(141.18373836,55.4651744)(141.18373047,55.52517873)
\curveto(141.17373837,55.58517428)(141.16373838,55.64017422)(141.15373047,55.69017873)
}
}
{
\newrgbcolor{curcolor}{0 0 0}
\pscustom[linestyle=none,fillstyle=solid,fillcolor=curcolor]
{
}
}
{
\newrgbcolor{curcolor}{0 0 0}
\pscustom[linestyle=none,fillstyle=solid,fillcolor=curcolor]
{
\newpath
\moveto(143.97373047,67.99510061)
\lineto(144.22873047,67.99510061)
\curveto(144.30873523,68.0050929)(144.38373516,68.00009291)(144.45373047,67.98010061)
\lineto(144.69373047,67.98010061)
\lineto(144.85873047,67.98010061)
\curveto(144.95873458,67.96009295)(145.06373448,67.95009296)(145.17373047,67.95010061)
\curveto(145.27373427,67.95009296)(145.37373417,67.94009297)(145.47373047,67.92010061)
\lineto(145.62373047,67.92010061)
\curveto(145.76373378,67.89009302)(145.90373364,67.87009304)(146.04373047,67.86010061)
\curveto(146.17373337,67.85009306)(146.30373324,67.82509308)(146.43373047,67.78510061)
\curveto(146.51373303,67.76509314)(146.59873294,67.74509316)(146.68873047,67.72510061)
\lineto(146.92873047,67.66510061)
\lineto(147.22873047,67.54510061)
\curveto(147.31873222,67.51509339)(147.40873213,67.48009343)(147.49873047,67.44010061)
\curveto(147.71873182,67.34009357)(147.93373161,67.2050937)(148.14373047,67.03510061)
\curveto(148.35373119,66.87509403)(148.52373102,66.70009421)(148.65373047,66.51010061)
\curveto(148.69373085,66.46009445)(148.73373081,66.40009451)(148.77373047,66.33010061)
\curveto(148.80373074,66.27009464)(148.8387307,66.2100947)(148.87873047,66.15010061)
\curveto(148.92873061,66.07009484)(148.96873057,65.97509493)(148.99873047,65.86510061)
\curveto(149.02873051,65.75509515)(149.05873048,65.65009526)(149.08873047,65.55010061)
\curveto(149.12873041,65.44009547)(149.15373039,65.33009558)(149.16373047,65.22010061)
\curveto(149.17373037,65.1100958)(149.18873035,64.99509591)(149.20873047,64.87510061)
\curveto(149.21873032,64.83509607)(149.21873032,64.79009612)(149.20873047,64.74010061)
\curveto(149.20873033,64.70009621)(149.21373033,64.66009625)(149.22373047,64.62010061)
\curveto(149.23373031,64.58009633)(149.2387303,64.52509638)(149.23873047,64.45510061)
\curveto(149.2387303,64.38509652)(149.23373031,64.33509657)(149.22373047,64.30510061)
\curveto(149.20373034,64.25509665)(149.19873034,64.2100967)(149.20873047,64.17010061)
\curveto(149.21873032,64.13009678)(149.21873032,64.09509681)(149.20873047,64.06510061)
\lineto(149.20873047,63.97510061)
\curveto(149.18873035,63.91509699)(149.17373037,63.85009706)(149.16373047,63.78010061)
\curveto(149.16373038,63.72009719)(149.15873038,63.65509725)(149.14873047,63.58510061)
\curveto(149.09873044,63.41509749)(149.04873049,63.25509765)(148.99873047,63.10510061)
\curveto(148.94873059,62.95509795)(148.88373066,62.8100981)(148.80373047,62.67010061)
\curveto(148.76373078,62.62009829)(148.73373081,62.56509834)(148.71373047,62.50510061)
\curveto(148.68373086,62.45509845)(148.64873089,62.4050985)(148.60873047,62.35510061)
\curveto(148.42873111,62.11509879)(148.20873133,61.91509899)(147.94873047,61.75510061)
\curveto(147.68873185,61.59509931)(147.40373214,61.45509945)(147.09373047,61.33510061)
\curveto(146.95373259,61.27509963)(146.81373273,61.23009968)(146.67373047,61.20010061)
\curveto(146.52373302,61.17009974)(146.36873317,61.13509977)(146.20873047,61.09510061)
\curveto(146.09873344,61.07509983)(145.98873355,61.06009985)(145.87873047,61.05010061)
\curveto(145.76873377,61.04009987)(145.65873388,61.02509988)(145.54873047,61.00510061)
\curveto(145.50873403,60.99509991)(145.46873407,60.99009992)(145.42873047,60.99010061)
\curveto(145.38873415,61.00009991)(145.34873419,61.00009991)(145.30873047,60.99010061)
\curveto(145.25873428,60.98009993)(145.20873433,60.97509993)(145.15873047,60.97510061)
\lineto(144.99373047,60.97510061)
\curveto(144.9437346,60.95509995)(144.89373465,60.95009996)(144.84373047,60.96010061)
\curveto(144.78373476,60.97009994)(144.72873481,60.97009994)(144.67873047,60.96010061)
\curveto(144.6387349,60.95009996)(144.59373495,60.95009996)(144.54373047,60.96010061)
\curveto(144.49373505,60.97009994)(144.4437351,60.96509994)(144.39373047,60.94510061)
\curveto(144.32373522,60.92509998)(144.24873529,60.92009999)(144.16873047,60.93010061)
\curveto(144.07873546,60.94009997)(143.99373555,60.94509996)(143.91373047,60.94510061)
\curveto(143.82373572,60.94509996)(143.72373582,60.94009997)(143.61373047,60.93010061)
\curveto(143.49373605,60.92009999)(143.39373615,60.92509998)(143.31373047,60.94510061)
\lineto(143.02873047,60.94510061)
\lineto(142.39873047,60.99010061)
\curveto(142.29873724,61.00009991)(142.20373734,61.0100999)(142.11373047,61.02010061)
\lineto(141.81373047,61.05010061)
\curveto(141.76373778,61.07009984)(141.71373783,61.07509983)(141.66373047,61.06510061)
\curveto(141.60373794,61.06509984)(141.54873799,61.07509983)(141.49873047,61.09510061)
\curveto(141.32873821,61.14509976)(141.16373838,61.18509972)(141.00373047,61.21510061)
\curveto(140.83373871,61.24509966)(140.67373887,61.29509961)(140.52373047,61.36510061)
\curveto(140.06373948,61.55509935)(139.68873985,61.77509913)(139.39873047,62.02510061)
\curveto(139.10874043,62.28509862)(138.86374068,62.64509826)(138.66373047,63.10510061)
\curveto(138.61374093,63.23509767)(138.57874096,63.36509754)(138.55873047,63.49510061)
\curveto(138.538741,63.63509727)(138.51374103,63.77509713)(138.48373047,63.91510061)
\curveto(138.47374107,63.98509692)(138.46874107,64.05009686)(138.46873047,64.11010061)
\curveto(138.46874107,64.17009674)(138.46374108,64.23509667)(138.45373047,64.30510061)
\curveto(138.43374111,65.13509577)(138.58374096,65.8050951)(138.90373047,66.31510061)
\curveto(139.21374033,66.82509408)(139.65373989,67.2050937)(140.22373047,67.45510061)
\curveto(140.3437392,67.5050934)(140.46873907,67.55009336)(140.59873047,67.59010061)
\curveto(140.72873881,67.63009328)(140.86373868,67.67509323)(141.00373047,67.72510061)
\curveto(141.08373846,67.74509316)(141.16873837,67.76009315)(141.25873047,67.77010061)
\lineto(141.49873047,67.83010061)
\curveto(141.60873793,67.86009305)(141.71873782,67.87509303)(141.82873047,67.87510061)
\curveto(141.9387376,67.88509302)(142.04873749,67.90009301)(142.15873047,67.92010061)
\curveto(142.20873733,67.94009297)(142.25373729,67.94509296)(142.29373047,67.93510061)
\curveto(142.33373721,67.93509297)(142.37373717,67.94009297)(142.41373047,67.95010061)
\curveto(142.46373708,67.96009295)(142.51873702,67.96009295)(142.57873047,67.95010061)
\curveto(142.62873691,67.95009296)(142.67873686,67.95509295)(142.72873047,67.96510061)
\lineto(142.86373047,67.96510061)
\curveto(142.92373662,67.98509292)(142.99373655,67.98509292)(143.07373047,67.96510061)
\curveto(143.1437364,67.95509295)(143.20873633,67.96009295)(143.26873047,67.98010061)
\curveto(143.29873624,67.99009292)(143.3387362,67.99509291)(143.38873047,67.99510061)
\lineto(143.50873047,67.99510061)
\lineto(143.97373047,67.99510061)
\moveto(146.29873047,66.45010061)
\curveto(145.97873356,66.55009436)(145.61373393,66.6100943)(145.20373047,66.63010061)
\curveto(144.79373475,66.65009426)(144.38373516,66.66009425)(143.97373047,66.66010061)
\curveto(143.543736,66.66009425)(143.12373642,66.65009426)(142.71373047,66.63010061)
\curveto(142.30373724,66.6100943)(141.91873762,66.56509434)(141.55873047,66.49510061)
\curveto(141.19873834,66.42509448)(140.87873866,66.31509459)(140.59873047,66.16510061)
\curveto(140.30873923,66.02509488)(140.07373947,65.83009508)(139.89373047,65.58010061)
\curveto(139.78373976,65.42009549)(139.70373984,65.24009567)(139.65373047,65.04010061)
\curveto(139.59373995,64.84009607)(139.56373998,64.59509631)(139.56373047,64.30510061)
\curveto(139.58373996,64.28509662)(139.59373995,64.25009666)(139.59373047,64.20010061)
\curveto(139.58373996,64.15009676)(139.58373996,64.1100968)(139.59373047,64.08010061)
\curveto(139.61373993,64.00009691)(139.63373991,63.92509698)(139.65373047,63.85510061)
\curveto(139.66373988,63.79509711)(139.68373986,63.73009718)(139.71373047,63.66010061)
\curveto(139.83373971,63.39009752)(140.00373954,63.17009774)(140.22373047,63.00010061)
\curveto(140.43373911,62.84009807)(140.67873886,62.7050982)(140.95873047,62.59510061)
\curveto(141.06873847,62.54509836)(141.18873835,62.5050984)(141.31873047,62.47510061)
\curveto(141.4387381,62.45509845)(141.56373798,62.43009848)(141.69373047,62.40010061)
\curveto(141.7437378,62.38009853)(141.79873774,62.37009854)(141.85873047,62.37010061)
\curveto(141.90873763,62.37009854)(141.95873758,62.36509854)(142.00873047,62.35510061)
\curveto(142.09873744,62.34509856)(142.19373735,62.33509857)(142.29373047,62.32510061)
\curveto(142.38373716,62.31509859)(142.47873706,62.3050986)(142.57873047,62.29510061)
\curveto(142.65873688,62.29509861)(142.7437368,62.29009862)(142.83373047,62.28010061)
\lineto(143.07373047,62.28010061)
\lineto(143.25373047,62.28010061)
\curveto(143.28373626,62.27009864)(143.31873622,62.26509864)(143.35873047,62.26510061)
\lineto(143.49373047,62.26510061)
\lineto(143.94373047,62.26510061)
\curveto(144.02373552,62.26509864)(144.10873543,62.26009865)(144.19873047,62.25010061)
\curveto(144.27873526,62.25009866)(144.35373519,62.26009865)(144.42373047,62.28010061)
\lineto(144.69373047,62.28010061)
\curveto(144.71373483,62.28009863)(144.7437348,62.27509863)(144.78373047,62.26510061)
\curveto(144.81373473,62.26509864)(144.8387347,62.27009864)(144.85873047,62.28010061)
\curveto(144.95873458,62.29009862)(145.05873448,62.29509861)(145.15873047,62.29510061)
\curveto(145.24873429,62.3050986)(145.34873419,62.31509859)(145.45873047,62.32510061)
\curveto(145.57873396,62.35509855)(145.70373384,62.37009854)(145.83373047,62.37010061)
\curveto(145.95373359,62.38009853)(146.06873347,62.4050985)(146.17873047,62.44510061)
\curveto(146.47873306,62.52509838)(146.7437328,62.6100983)(146.97373047,62.70010061)
\curveto(147.20373234,62.80009811)(147.41873212,62.94509796)(147.61873047,63.13510061)
\curveto(147.81873172,63.34509756)(147.96873157,63.6100973)(148.06873047,63.93010061)
\curveto(148.08873145,63.97009694)(148.09873144,64.0050969)(148.09873047,64.03510061)
\curveto(148.08873145,64.07509683)(148.09373145,64.12009679)(148.11373047,64.17010061)
\curveto(148.12373142,64.2100967)(148.13373141,64.28009663)(148.14373047,64.38010061)
\curveto(148.15373139,64.49009642)(148.14873139,64.57509633)(148.12873047,64.63510061)
\curveto(148.10873143,64.7050962)(148.09873144,64.77509613)(148.09873047,64.84510061)
\curveto(148.08873145,64.91509599)(148.07373147,64.98009593)(148.05373047,65.04010061)
\curveto(147.99373155,65.24009567)(147.90873163,65.42009549)(147.79873047,65.58010061)
\curveto(147.77873176,65.6100953)(147.75873178,65.63509527)(147.73873047,65.65510061)
\lineto(147.67873047,65.71510061)
\curveto(147.65873188,65.75509515)(147.61873192,65.8050951)(147.55873047,65.86510061)
\curveto(147.41873212,65.96509494)(147.28873225,66.05009486)(147.16873047,66.12010061)
\curveto(147.04873249,66.19009472)(146.90373264,66.26009465)(146.73373047,66.33010061)
\curveto(146.66373288,66.36009455)(146.59373295,66.38009453)(146.52373047,66.39010061)
\curveto(146.45373309,66.4100945)(146.37873316,66.43009448)(146.29873047,66.45010061)
}
}
{
\newrgbcolor{curcolor}{0 0 0}
\pscustom[linestyle=none,fillstyle=solid,fillcolor=curcolor]
{
\newpath
\moveto(144.73873047,76.34470998)
\curveto(144.85873468,76.37470226)(144.99873454,76.39970223)(145.15873047,76.41970998)
\curveto(145.31873422,76.43970219)(145.48373406,76.44970218)(145.65373047,76.44970998)
\curveto(145.82373372,76.44970218)(145.98873355,76.43970219)(146.14873047,76.41970998)
\curveto(146.30873323,76.39970223)(146.44873309,76.37470226)(146.56873047,76.34470998)
\curveto(146.70873283,76.30470233)(146.83373271,76.26970236)(146.94373047,76.23970998)
\curveto(147.05373249,76.20970242)(147.16373238,76.16970246)(147.27373047,76.11970998)
\curveto(147.91373163,75.84970278)(148.39873114,75.4347032)(148.72873047,74.87470998)
\curveto(148.78873075,74.79470384)(148.8387307,74.70970392)(148.87873047,74.61970998)
\curveto(148.90873063,74.5297041)(148.9437306,74.4297042)(148.98373047,74.31970998)
\curveto(149.03373051,74.20970442)(149.06873047,74.08970454)(149.08873047,73.95970998)
\curveto(149.11873042,73.83970479)(149.14873039,73.70970492)(149.17873047,73.56970998)
\curveto(149.19873034,73.50970512)(149.20373034,73.44970518)(149.19373047,73.38970998)
\curveto(149.18373036,73.33970529)(149.18873035,73.27970535)(149.20873047,73.20970998)
\curveto(149.21873032,73.18970544)(149.21873032,73.16470547)(149.20873047,73.13470998)
\curveto(149.20873033,73.10470553)(149.21373033,73.07970555)(149.22373047,73.05970998)
\lineto(149.22373047,72.90970998)
\curveto(149.23373031,72.83970579)(149.23373031,72.78970584)(149.22373047,72.75970998)
\curveto(149.21373033,72.71970591)(149.20873033,72.67470596)(149.20873047,72.62470998)
\curveto(149.21873032,72.58470605)(149.21873032,72.54470609)(149.20873047,72.50470998)
\curveto(149.18873035,72.41470622)(149.17373037,72.32470631)(149.16373047,72.23470998)
\curveto(149.16373038,72.14470649)(149.15373039,72.05470658)(149.13373047,71.96470998)
\curveto(149.10373044,71.87470676)(149.07873046,71.78470685)(149.05873047,71.69470998)
\curveto(149.0387305,71.60470703)(149.00873053,71.51970711)(148.96873047,71.43970998)
\curveto(148.85873068,71.19970743)(148.72873081,70.97470766)(148.57873047,70.76470998)
\curveto(148.41873112,70.55470808)(148.2387313,70.37470826)(148.03873047,70.22470998)
\curveto(147.86873167,70.10470853)(147.69373185,69.99970863)(147.51373047,69.90970998)
\curveto(147.33373221,69.81970881)(147.1437324,69.7297089)(146.94373047,69.63970998)
\curveto(146.8437327,69.59970903)(146.7437328,69.56470907)(146.64373047,69.53470998)
\curveto(146.53373301,69.51470912)(146.42373312,69.48970914)(146.31373047,69.45970998)
\curveto(146.17373337,69.41970921)(146.03373351,69.39470924)(145.89373047,69.38470998)
\curveto(145.75373379,69.37470926)(145.61373393,69.35470928)(145.47373047,69.32470998)
\curveto(145.36373418,69.31470932)(145.26373428,69.30470933)(145.17373047,69.29470998)
\curveto(145.07373447,69.29470934)(144.97373457,69.28470935)(144.87373047,69.26470998)
\lineto(144.78373047,69.26470998)
\curveto(144.75373479,69.27470936)(144.72873481,69.27470936)(144.70873047,69.26470998)
\lineto(144.49873047,69.26470998)
\curveto(144.4387351,69.24470939)(144.37373517,69.2347094)(144.30373047,69.23470998)
\curveto(144.22373532,69.24470939)(144.14873539,69.24970938)(144.07873047,69.24970998)
\lineto(143.92873047,69.24970998)
\curveto(143.87873566,69.24970938)(143.82873571,69.25470938)(143.77873047,69.26470998)
\lineto(143.40373047,69.26470998)
\curveto(143.37373617,69.27470936)(143.3387362,69.27470936)(143.29873047,69.26470998)
\curveto(143.25873628,69.26470937)(143.21873632,69.26970936)(143.17873047,69.27970998)
\curveto(143.06873647,69.29970933)(142.95873658,69.31470932)(142.84873047,69.32470998)
\curveto(142.72873681,69.3347093)(142.61373693,69.34470929)(142.50373047,69.35470998)
\curveto(142.35373719,69.39470924)(142.20873733,69.41970921)(142.06873047,69.42970998)
\curveto(141.91873762,69.44970918)(141.77373777,69.47970915)(141.63373047,69.51970998)
\curveto(141.33373821,69.60970902)(141.04873849,69.70470893)(140.77873047,69.80470998)
\curveto(140.50873903,69.90470873)(140.25873928,70.0297086)(140.02873047,70.17970998)
\curveto(139.70873983,70.37970825)(139.42874011,70.62470801)(139.18873047,70.91470998)
\curveto(138.94874059,71.20470743)(138.76374078,71.54470709)(138.63373047,71.93470998)
\curveto(138.59374095,72.04470659)(138.56874097,72.15470648)(138.55873047,72.26470998)
\curveto(138.538741,72.38470625)(138.51374103,72.50470613)(138.48373047,72.62470998)
\curveto(138.47374107,72.69470594)(138.46874107,72.75970587)(138.46873047,72.81970998)
\curveto(138.46874107,72.87970575)(138.46374108,72.94470569)(138.45373047,73.01470998)
\curveto(138.43374111,73.71470492)(138.54874099,74.28970434)(138.79873047,74.73970998)
\curveto(139.04874049,75.18970344)(139.39874014,75.5347031)(139.84873047,75.77470998)
\curveto(140.07873946,75.88470275)(140.35373919,75.98470265)(140.67373047,76.07470998)
\curveto(140.7437388,76.09470254)(140.81873872,76.09470254)(140.89873047,76.07470998)
\curveto(140.96873857,76.06470257)(141.01873852,76.03970259)(141.04873047,75.99970998)
\curveto(141.07873846,75.96970266)(141.10373844,75.90970272)(141.12373047,75.81970998)
\curveto(141.13373841,75.7297029)(141.1437384,75.629703)(141.15373047,75.51970998)
\curveto(141.15373839,75.41970321)(141.14873839,75.31970331)(141.13873047,75.21970998)
\curveto(141.12873841,75.1297035)(141.10873843,75.06470357)(141.07873047,75.02470998)
\curveto(141.00873853,74.91470372)(140.89873864,74.8347038)(140.74873047,74.78470998)
\curveto(140.59873894,74.74470389)(140.46873907,74.68970394)(140.35873047,74.61970998)
\curveto(140.04873949,74.4297042)(139.81873972,74.14970448)(139.66873047,73.77970998)
\curveto(139.6387399,73.70970492)(139.61873992,73.634705)(139.60873047,73.55470998)
\curveto(139.59873994,73.48470515)(139.58373996,73.40970522)(139.56373047,73.32970998)
\curveto(139.55373999,73.27970535)(139.54873999,73.20970542)(139.54873047,73.11970998)
\curveto(139.54873999,73.03970559)(139.55373999,72.97470566)(139.56373047,72.92470998)
\curveto(139.58373996,72.88470575)(139.58873995,72.84970578)(139.57873047,72.81970998)
\curveto(139.56873997,72.78970584)(139.56873997,72.75470588)(139.57873047,72.71470998)
\lineto(139.63873047,72.47470998)
\curveto(139.65873988,72.40470623)(139.68373986,72.3347063)(139.71373047,72.26470998)
\curveto(139.87373967,71.88470675)(140.08373946,71.59470704)(140.34373047,71.39470998)
\curveto(140.60373894,71.20470743)(140.91873862,71.0297076)(141.28873047,70.86970998)
\curveto(141.36873817,70.83970779)(141.44873809,70.81470782)(141.52873047,70.79470998)
\curveto(141.60873793,70.78470785)(141.68873785,70.76470787)(141.76873047,70.73470998)
\curveto(141.87873766,70.70470793)(141.99373755,70.67970795)(142.11373047,70.65970998)
\curveto(142.23373731,70.64970798)(142.35373719,70.629708)(142.47373047,70.59970998)
\curveto(142.52373702,70.57970805)(142.57373697,70.56970806)(142.62373047,70.56970998)
\curveto(142.67373687,70.57970805)(142.72373682,70.57470806)(142.77373047,70.55470998)
\curveto(142.83373671,70.54470809)(142.91373663,70.54470809)(143.01373047,70.55470998)
\curveto(143.10373644,70.56470807)(143.15873638,70.57970805)(143.17873047,70.59970998)
\curveto(143.21873632,70.61970801)(143.2387363,70.64970798)(143.23873047,70.68970998)
\curveto(143.2387363,70.73970789)(143.22873631,70.78470785)(143.20873047,70.82470998)
\curveto(143.16873637,70.89470774)(143.12373642,70.95470768)(143.07373047,71.00470998)
\curveto(143.02373652,71.05470758)(142.97373657,71.11470752)(142.92373047,71.18470998)
\lineto(142.86373047,71.24470998)
\curveto(142.83373671,71.27470736)(142.80873673,71.30470733)(142.78873047,71.33470998)
\curveto(142.62873691,71.56470707)(142.49373705,71.83970679)(142.38373047,72.15970998)
\curveto(142.36373718,72.2297064)(142.34873719,72.29970633)(142.33873047,72.36970998)
\curveto(142.32873721,72.43970619)(142.31373723,72.51470612)(142.29373047,72.59470998)
\curveto(142.29373725,72.634706)(142.28873725,72.66970596)(142.27873047,72.69970998)
\curveto(142.26873727,72.7297059)(142.26873727,72.76470587)(142.27873047,72.80470998)
\curveto(142.27873726,72.85470578)(142.26873727,72.89470574)(142.24873047,72.92470998)
\lineto(142.24873047,73.08970998)
\lineto(142.24873047,73.17970998)
\curveto(142.2387373,73.2297054)(142.2387373,73.26970536)(142.24873047,73.29970998)
\curveto(142.25873728,73.34970528)(142.26373728,73.39970523)(142.26373047,73.44970998)
\curveto(142.25373729,73.50970512)(142.25373729,73.56470507)(142.26373047,73.61470998)
\curveto(142.29373725,73.72470491)(142.31373723,73.8297048)(142.32373047,73.92970998)
\curveto(142.33373721,74.03970459)(142.35873718,74.14470449)(142.39873047,74.24470998)
\curveto(142.538737,74.66470397)(142.72373682,75.00970362)(142.95373047,75.27970998)
\curveto(143.17373637,75.54970308)(143.45873608,75.78970284)(143.80873047,75.99970998)
\curveto(143.94873559,76.07970255)(144.09873544,76.14470249)(144.25873047,76.19470998)
\curveto(144.40873513,76.24470239)(144.56873497,76.29470234)(144.73873047,76.34470998)
\moveto(146.04373047,75.09970998)
\curveto(145.99373355,75.10970352)(145.94873359,75.11470352)(145.90873047,75.11470998)
\lineto(145.75873047,75.11470998)
\curveto(145.44873409,75.11470352)(145.16373438,75.07470356)(144.90373047,74.99470998)
\curveto(144.8437347,74.97470366)(144.78873475,74.95470368)(144.73873047,74.93470998)
\curveto(144.67873486,74.92470371)(144.62373492,74.90970372)(144.57373047,74.88970998)
\curveto(144.08373546,74.66970396)(143.73373581,74.32470431)(143.52373047,73.85470998)
\curveto(143.49373605,73.77470486)(143.46873607,73.69470494)(143.44873047,73.61470998)
\lineto(143.38873047,73.37470998)
\curveto(143.36873617,73.29470534)(143.35873618,73.20470543)(143.35873047,73.10470998)
\lineto(143.35873047,72.78970998)
\curveto(143.37873616,72.76970586)(143.38873615,72.7297059)(143.38873047,72.66970998)
\curveto(143.37873616,72.61970601)(143.37873616,72.57470606)(143.38873047,72.53470998)
\lineto(143.44873047,72.29470998)
\curveto(143.45873608,72.22470641)(143.47873606,72.15470648)(143.50873047,72.08470998)
\curveto(143.76873577,71.48470715)(144.23373531,71.07970755)(144.90373047,70.86970998)
\curveto(144.98373456,70.83970779)(145.06373448,70.81970781)(145.14373047,70.80970998)
\curveto(145.22373432,70.79970783)(145.30873423,70.78470785)(145.39873047,70.76470998)
\lineto(145.54873047,70.76470998)
\curveto(145.58873395,70.75470788)(145.65873388,70.74970788)(145.75873047,70.74970998)
\curveto(145.98873355,70.74970788)(146.18373336,70.76970786)(146.34373047,70.80970998)
\curveto(146.41373313,70.8297078)(146.47873306,70.84470779)(146.53873047,70.85470998)
\curveto(146.59873294,70.86470777)(146.66373288,70.88470775)(146.73373047,70.91470998)
\curveto(147.01373253,71.02470761)(147.25873228,71.16970746)(147.46873047,71.34970998)
\curveto(147.66873187,71.5297071)(147.82873171,71.76470687)(147.94873047,72.05470998)
\lineto(148.03873047,72.29470998)
\lineto(148.09873047,72.53470998)
\curveto(148.11873142,72.58470605)(148.12373142,72.62470601)(148.11373047,72.65470998)
\curveto(148.10373144,72.69470594)(148.10873143,72.73970589)(148.12873047,72.78970998)
\curveto(148.1387314,72.81970581)(148.1437314,72.87470576)(148.14373047,72.95470998)
\curveto(148.1437314,73.0347056)(148.1387314,73.09470554)(148.12873047,73.13470998)
\curveto(148.10873143,73.24470539)(148.09373145,73.34970528)(148.08373047,73.44970998)
\curveto(148.07373147,73.54970508)(148.0437315,73.64470499)(147.99373047,73.73470998)
\curveto(147.79373175,74.26470437)(147.41873212,74.65470398)(146.86873047,74.90470998)
\curveto(146.76873277,74.94470369)(146.66373288,74.97470366)(146.55373047,74.99470998)
\lineto(146.22373047,75.08470998)
\curveto(146.1437334,75.08470355)(146.08373346,75.08970354)(146.04373047,75.09970998)
}
}
{
\newrgbcolor{curcolor}{0 0 0}
\pscustom[linestyle=none,fillstyle=solid,fillcolor=curcolor]
{
\newpath
\moveto(147.42373047,78.63431936)
\lineto(147.42373047,79.26431936)
\lineto(147.42373047,79.45931936)
\curveto(147.42373212,79.52931683)(147.43373211,79.58931677)(147.45373047,79.63931936)
\curveto(147.49373205,79.70931665)(147.53373201,79.7593166)(147.57373047,79.78931936)
\curveto(147.62373192,79.82931653)(147.68873185,79.84931651)(147.76873047,79.84931936)
\curveto(147.84873169,79.8593165)(147.93373161,79.86431649)(148.02373047,79.86431936)
\lineto(148.74373047,79.86431936)
\curveto(149.22373032,79.86431649)(149.63372991,79.80431655)(149.97373047,79.68431936)
\curveto(150.31372923,79.56431679)(150.58872895,79.36931699)(150.79873047,79.09931936)
\curveto(150.84872869,79.02931733)(150.89372865,78.9593174)(150.93373047,78.88931936)
\curveto(150.98372856,78.82931753)(151.02872851,78.7543176)(151.06873047,78.66431936)
\curveto(151.07872846,78.64431771)(151.08872845,78.61431774)(151.09873047,78.57431936)
\curveto(151.11872842,78.53431782)(151.12372842,78.48931787)(151.11373047,78.43931936)
\curveto(151.08372846,78.34931801)(151.00872853,78.29431806)(150.88873047,78.27431936)
\curveto(150.77872876,78.2543181)(150.68372886,78.26931809)(150.60373047,78.31931936)
\curveto(150.53372901,78.34931801)(150.46872907,78.39431796)(150.40873047,78.45431936)
\curveto(150.35872918,78.52431783)(150.30872923,78.58931777)(150.25873047,78.64931936)
\curveto(150.20872933,78.71931764)(150.13372941,78.77931758)(150.03373047,78.82931936)
\curveto(149.9437296,78.88931747)(149.85372969,78.93931742)(149.76373047,78.97931936)
\curveto(149.73372981,78.99931736)(149.67372987,79.02431733)(149.58373047,79.05431936)
\curveto(149.50373004,79.08431727)(149.43373011,79.08931727)(149.37373047,79.06931936)
\curveto(149.23373031,79.03931732)(149.1437304,78.97931738)(149.10373047,78.88931936)
\curveto(149.07373047,78.80931755)(149.05873048,78.71931764)(149.05873047,78.61931936)
\curveto(149.05873048,78.51931784)(149.03373051,78.43431792)(148.98373047,78.36431936)
\curveto(148.91373063,78.27431808)(148.77373077,78.22931813)(148.56373047,78.22931936)
\lineto(148.00873047,78.22931936)
\lineto(147.78373047,78.22931936)
\curveto(147.70373184,78.23931812)(147.6387319,78.2593181)(147.58873047,78.28931936)
\curveto(147.50873203,78.34931801)(147.46373208,78.41931794)(147.45373047,78.49931936)
\curveto(147.4437321,78.51931784)(147.4387321,78.53931782)(147.43873047,78.55931936)
\curveto(147.4387321,78.58931777)(147.43373211,78.61431774)(147.42373047,78.63431936)
}
}
{
\newrgbcolor{curcolor}{0 0 0}
\pscustom[linestyle=none,fillstyle=solid,fillcolor=curcolor]
{
}
}
{
\newrgbcolor{curcolor}{0 0 0}
\pscustom[linestyle=none,fillstyle=solid,fillcolor=curcolor]
{
\newpath
\moveto(138.45373047,89.26463186)
\curveto(138.4437411,89.95462722)(138.56374098,90.55462662)(138.81373047,91.06463186)
\curveto(139.06374048,91.58462559)(139.39874014,91.9796252)(139.81873047,92.24963186)
\curveto(139.89873964,92.29962488)(139.98873955,92.34462483)(140.08873047,92.38463186)
\curveto(140.17873936,92.42462475)(140.27373927,92.46962471)(140.37373047,92.51963186)
\curveto(140.47373907,92.55962462)(140.57373897,92.58962459)(140.67373047,92.60963186)
\curveto(140.77373877,92.62962455)(140.87873866,92.64962453)(140.98873047,92.66963186)
\curveto(141.0387385,92.68962449)(141.08373846,92.69462448)(141.12373047,92.68463186)
\curveto(141.16373838,92.6746245)(141.20873833,92.6796245)(141.25873047,92.69963186)
\curveto(141.30873823,92.70962447)(141.39373815,92.71462446)(141.51373047,92.71463186)
\curveto(141.62373792,92.71462446)(141.70873783,92.70962447)(141.76873047,92.69963186)
\curveto(141.82873771,92.6796245)(141.88873765,92.66962451)(141.94873047,92.66963186)
\curveto(142.00873753,92.6796245)(142.06873747,92.6746245)(142.12873047,92.65463186)
\curveto(142.26873727,92.61462456)(142.40373714,92.5796246)(142.53373047,92.54963186)
\curveto(142.66373688,92.51962466)(142.78873675,92.4796247)(142.90873047,92.42963186)
\curveto(143.04873649,92.36962481)(143.17373637,92.29962488)(143.28373047,92.21963186)
\curveto(143.39373615,92.14962503)(143.50373604,92.0746251)(143.61373047,91.99463186)
\lineto(143.67373047,91.93463186)
\curveto(143.69373585,91.92462525)(143.71373583,91.90962527)(143.73373047,91.88963186)
\curveto(143.89373565,91.76962541)(144.0387355,91.63462554)(144.16873047,91.48463186)
\curveto(144.29873524,91.33462584)(144.42373512,91.174626)(144.54373047,91.00463186)
\curveto(144.76373478,90.69462648)(144.96873457,90.39962678)(145.15873047,90.11963186)
\curveto(145.29873424,89.88962729)(145.43373411,89.65962752)(145.56373047,89.42963186)
\curveto(145.69373385,89.20962797)(145.82873371,88.98962819)(145.96873047,88.76963186)
\curveto(146.1387334,88.51962866)(146.31873322,88.2796289)(146.50873047,88.04963186)
\curveto(146.69873284,87.82962935)(146.92373262,87.63962954)(147.18373047,87.47963186)
\curveto(147.2437323,87.43962974)(147.30373224,87.40462977)(147.36373047,87.37463186)
\curveto(147.41373213,87.34462983)(147.47873206,87.31462986)(147.55873047,87.28463186)
\curveto(147.62873191,87.26462991)(147.68873185,87.25962992)(147.73873047,87.26963186)
\curveto(147.80873173,87.28962989)(147.86373168,87.32462985)(147.90373047,87.37463186)
\curveto(147.93373161,87.42462975)(147.95373159,87.48462969)(147.96373047,87.55463186)
\lineto(147.96373047,87.79463186)
\lineto(147.96373047,88.54463186)
\lineto(147.96373047,91.34963186)
\lineto(147.96373047,92.00963186)
\curveto(147.96373158,92.09962508)(147.96873157,92.18462499)(147.97873047,92.26463186)
\curveto(147.97873156,92.34462483)(147.99873154,92.40962477)(148.03873047,92.45963186)
\curveto(148.07873146,92.50962467)(148.15373139,92.54962463)(148.26373047,92.57963186)
\curveto(148.36373118,92.61962456)(148.46373108,92.62962455)(148.56373047,92.60963186)
\lineto(148.69873047,92.60963186)
\curveto(148.76873077,92.58962459)(148.82873071,92.56962461)(148.87873047,92.54963186)
\curveto(148.92873061,92.52962465)(148.96873057,92.49462468)(148.99873047,92.44463186)
\curveto(149.0387305,92.39462478)(149.05873048,92.32462485)(149.05873047,92.23463186)
\lineto(149.05873047,91.96463186)
\lineto(149.05873047,91.06463186)
\lineto(149.05873047,87.55463186)
\lineto(149.05873047,86.48963186)
\curveto(149.05873048,86.40963077)(149.06373048,86.31963086)(149.07373047,86.21963186)
\curveto(149.07373047,86.11963106)(149.06373048,86.03463114)(149.04373047,85.96463186)
\curveto(148.97373057,85.75463142)(148.79373075,85.68963149)(148.50373047,85.76963186)
\curveto(148.46373108,85.7796314)(148.42873111,85.7796314)(148.39873047,85.76963186)
\curveto(148.35873118,85.76963141)(148.31373123,85.7796314)(148.26373047,85.79963186)
\curveto(148.18373136,85.81963136)(148.09873144,85.83963134)(148.00873047,85.85963186)
\curveto(147.91873162,85.8796313)(147.83373171,85.90463127)(147.75373047,85.93463186)
\curveto(147.26373228,86.09463108)(146.84873269,86.29463088)(146.50873047,86.53463186)
\curveto(146.25873328,86.71463046)(146.03373351,86.91963026)(145.83373047,87.14963186)
\curveto(145.62373392,87.3796298)(145.42873411,87.61962956)(145.24873047,87.86963186)
\curveto(145.06873447,88.12962905)(144.89873464,88.39462878)(144.73873047,88.66463186)
\curveto(144.56873497,88.94462823)(144.39373515,89.21462796)(144.21373047,89.47463186)
\curveto(144.13373541,89.58462759)(144.05873548,89.68962749)(143.98873047,89.78963186)
\curveto(143.91873562,89.89962728)(143.8437357,90.00962717)(143.76373047,90.11963186)
\curveto(143.73373581,90.15962702)(143.70373584,90.19462698)(143.67373047,90.22463186)
\curveto(143.63373591,90.26462691)(143.60373594,90.30462687)(143.58373047,90.34463186)
\curveto(143.47373607,90.48462669)(143.34873619,90.60962657)(143.20873047,90.71963186)
\curveto(143.17873636,90.73962644)(143.15373639,90.76462641)(143.13373047,90.79463186)
\curveto(143.10373644,90.82462635)(143.07373647,90.84962633)(143.04373047,90.86963186)
\curveto(142.9437366,90.94962623)(142.8437367,91.01462616)(142.74373047,91.06463186)
\curveto(142.6437369,91.12462605)(142.53373701,91.179626)(142.41373047,91.22963186)
\curveto(142.3437372,91.25962592)(142.26873727,91.2796259)(142.18873047,91.28963186)
\lineto(141.94873047,91.34963186)
\lineto(141.85873047,91.34963186)
\curveto(141.82873771,91.35962582)(141.79873774,91.36462581)(141.76873047,91.36463186)
\curveto(141.69873784,91.38462579)(141.60373794,91.38962579)(141.48373047,91.37963186)
\curveto(141.35373819,91.3796258)(141.25373829,91.36962581)(141.18373047,91.34963186)
\curveto(141.10373844,91.32962585)(141.02873851,91.30962587)(140.95873047,91.28963186)
\curveto(140.87873866,91.2796259)(140.79873874,91.25962592)(140.71873047,91.22963186)
\curveto(140.47873906,91.11962606)(140.27873926,90.96962621)(140.11873047,90.77963186)
\curveto(139.94873959,90.59962658)(139.80873973,90.3796268)(139.69873047,90.11963186)
\curveto(139.67873986,90.04962713)(139.66373988,89.9796272)(139.65373047,89.90963186)
\curveto(139.63373991,89.83962734)(139.61373993,89.76462741)(139.59373047,89.68463186)
\curveto(139.57373997,89.60462757)(139.56373998,89.49462768)(139.56373047,89.35463186)
\curveto(139.56373998,89.22462795)(139.57373997,89.11962806)(139.59373047,89.03963186)
\curveto(139.60373994,88.9796282)(139.60873993,88.92462825)(139.60873047,88.87463186)
\curveto(139.60873993,88.82462835)(139.61873992,88.7746284)(139.63873047,88.72463186)
\curveto(139.67873986,88.62462855)(139.71873982,88.52962865)(139.75873047,88.43963186)
\curveto(139.79873974,88.35962882)(139.8437397,88.2796289)(139.89373047,88.19963186)
\curveto(139.91373963,88.16962901)(139.9387396,88.13962904)(139.96873047,88.10963186)
\curveto(139.99873954,88.08962909)(140.02373952,88.06462911)(140.04373047,88.03463186)
\lineto(140.11873047,87.95963186)
\curveto(140.1387394,87.92962925)(140.15873938,87.90462927)(140.17873047,87.88463186)
\lineto(140.38873047,87.73463186)
\curveto(140.44873909,87.69462948)(140.51373903,87.64962953)(140.58373047,87.59963186)
\curveto(140.67373887,87.53962964)(140.77873876,87.48962969)(140.89873047,87.44963186)
\curveto(141.00873853,87.41962976)(141.11873842,87.38462979)(141.22873047,87.34463186)
\curveto(141.3387382,87.30462987)(141.48373806,87.2796299)(141.66373047,87.26963186)
\curveto(141.83373771,87.25962992)(141.95873758,87.22962995)(142.03873047,87.17963186)
\curveto(142.11873742,87.12963005)(142.16373738,87.05463012)(142.17373047,86.95463186)
\curveto(142.18373736,86.85463032)(142.18873735,86.74463043)(142.18873047,86.62463186)
\curveto(142.18873735,86.58463059)(142.19373735,86.54463063)(142.20373047,86.50463186)
\curveto(142.20373734,86.46463071)(142.19873734,86.42963075)(142.18873047,86.39963186)
\curveto(142.16873737,86.34963083)(142.15873738,86.29963088)(142.15873047,86.24963186)
\curveto(142.15873738,86.20963097)(142.14873739,86.16963101)(142.12873047,86.12963186)
\curveto(142.06873747,86.03963114)(141.93373761,85.99463118)(141.72373047,85.99463186)
\lineto(141.60373047,85.99463186)
\curveto(141.543738,86.00463117)(141.48373806,86.00963117)(141.42373047,86.00963186)
\curveto(141.35373819,86.01963116)(141.28873825,86.02963115)(141.22873047,86.03963186)
\curveto(141.11873842,86.05963112)(141.01873852,86.0796311)(140.92873047,86.09963186)
\curveto(140.82873871,86.11963106)(140.73373881,86.14963103)(140.64373047,86.18963186)
\curveto(140.57373897,86.20963097)(140.51373903,86.22963095)(140.46373047,86.24963186)
\lineto(140.28373047,86.30963186)
\curveto(140.02373952,86.42963075)(139.77873976,86.58463059)(139.54873047,86.77463186)
\curveto(139.31874022,86.9746302)(139.13374041,87.18962999)(138.99373047,87.41963186)
\curveto(138.91374063,87.52962965)(138.84874069,87.64462953)(138.79873047,87.76463186)
\lineto(138.64873047,88.15463186)
\curveto(138.59874094,88.26462891)(138.56874097,88.3796288)(138.55873047,88.49963186)
\curveto(138.538741,88.61962856)(138.51374103,88.74462843)(138.48373047,88.87463186)
\curveto(138.48374106,88.94462823)(138.48374106,89.00962817)(138.48373047,89.06963186)
\curveto(138.47374107,89.12962805)(138.46374108,89.19462798)(138.45373047,89.26463186)
}
}
{
\newrgbcolor{curcolor}{0 0 0}
\pscustom[linestyle=none,fillstyle=solid,fillcolor=curcolor]
{
\newpath
\moveto(143.97373047,101.36424123)
\lineto(144.22873047,101.36424123)
\curveto(144.30873523,101.37423353)(144.38373516,101.36923353)(144.45373047,101.34924123)
\lineto(144.69373047,101.34924123)
\lineto(144.85873047,101.34924123)
\curveto(144.95873458,101.32923357)(145.06373448,101.31923358)(145.17373047,101.31924123)
\curveto(145.27373427,101.31923358)(145.37373417,101.30923359)(145.47373047,101.28924123)
\lineto(145.62373047,101.28924123)
\curveto(145.76373378,101.25923364)(145.90373364,101.23923366)(146.04373047,101.22924123)
\curveto(146.17373337,101.21923368)(146.30373324,101.19423371)(146.43373047,101.15424123)
\curveto(146.51373303,101.13423377)(146.59873294,101.11423379)(146.68873047,101.09424123)
\lineto(146.92873047,101.03424123)
\lineto(147.22873047,100.91424123)
\curveto(147.31873222,100.88423402)(147.40873213,100.84923405)(147.49873047,100.80924123)
\curveto(147.71873182,100.70923419)(147.93373161,100.57423433)(148.14373047,100.40424123)
\curveto(148.35373119,100.24423466)(148.52373102,100.06923483)(148.65373047,99.87924123)
\curveto(148.69373085,99.82923507)(148.73373081,99.76923513)(148.77373047,99.69924123)
\curveto(148.80373074,99.63923526)(148.8387307,99.57923532)(148.87873047,99.51924123)
\curveto(148.92873061,99.43923546)(148.96873057,99.34423556)(148.99873047,99.23424123)
\curveto(149.02873051,99.12423578)(149.05873048,99.01923588)(149.08873047,98.91924123)
\curveto(149.12873041,98.80923609)(149.15373039,98.6992362)(149.16373047,98.58924123)
\curveto(149.17373037,98.47923642)(149.18873035,98.36423654)(149.20873047,98.24424123)
\curveto(149.21873032,98.2042367)(149.21873032,98.15923674)(149.20873047,98.10924123)
\curveto(149.20873033,98.06923683)(149.21373033,98.02923687)(149.22373047,97.98924123)
\curveto(149.23373031,97.94923695)(149.2387303,97.89423701)(149.23873047,97.82424123)
\curveto(149.2387303,97.75423715)(149.23373031,97.7042372)(149.22373047,97.67424123)
\curveto(149.20373034,97.62423728)(149.19873034,97.57923732)(149.20873047,97.53924123)
\curveto(149.21873032,97.4992374)(149.21873032,97.46423744)(149.20873047,97.43424123)
\lineto(149.20873047,97.34424123)
\curveto(149.18873035,97.28423762)(149.17373037,97.21923768)(149.16373047,97.14924123)
\curveto(149.16373038,97.08923781)(149.15873038,97.02423788)(149.14873047,96.95424123)
\curveto(149.09873044,96.78423812)(149.04873049,96.62423828)(148.99873047,96.47424123)
\curveto(148.94873059,96.32423858)(148.88373066,96.17923872)(148.80373047,96.03924123)
\curveto(148.76373078,95.98923891)(148.73373081,95.93423897)(148.71373047,95.87424123)
\curveto(148.68373086,95.82423908)(148.64873089,95.77423913)(148.60873047,95.72424123)
\curveto(148.42873111,95.48423942)(148.20873133,95.28423962)(147.94873047,95.12424123)
\curveto(147.68873185,94.96423994)(147.40373214,94.82424008)(147.09373047,94.70424123)
\curveto(146.95373259,94.64424026)(146.81373273,94.5992403)(146.67373047,94.56924123)
\curveto(146.52373302,94.53924036)(146.36873317,94.5042404)(146.20873047,94.46424123)
\curveto(146.09873344,94.44424046)(145.98873355,94.42924047)(145.87873047,94.41924123)
\curveto(145.76873377,94.40924049)(145.65873388,94.39424051)(145.54873047,94.37424123)
\curveto(145.50873403,94.36424054)(145.46873407,94.35924054)(145.42873047,94.35924123)
\curveto(145.38873415,94.36924053)(145.34873419,94.36924053)(145.30873047,94.35924123)
\curveto(145.25873428,94.34924055)(145.20873433,94.34424056)(145.15873047,94.34424123)
\lineto(144.99373047,94.34424123)
\curveto(144.9437346,94.32424058)(144.89373465,94.31924058)(144.84373047,94.32924123)
\curveto(144.78373476,94.33924056)(144.72873481,94.33924056)(144.67873047,94.32924123)
\curveto(144.6387349,94.31924058)(144.59373495,94.31924058)(144.54373047,94.32924123)
\curveto(144.49373505,94.33924056)(144.4437351,94.33424057)(144.39373047,94.31424123)
\curveto(144.32373522,94.29424061)(144.24873529,94.28924061)(144.16873047,94.29924123)
\curveto(144.07873546,94.30924059)(143.99373555,94.31424059)(143.91373047,94.31424123)
\curveto(143.82373572,94.31424059)(143.72373582,94.30924059)(143.61373047,94.29924123)
\curveto(143.49373605,94.28924061)(143.39373615,94.29424061)(143.31373047,94.31424123)
\lineto(143.02873047,94.31424123)
\lineto(142.39873047,94.35924123)
\curveto(142.29873724,94.36924053)(142.20373734,94.37924052)(142.11373047,94.38924123)
\lineto(141.81373047,94.41924123)
\curveto(141.76373778,94.43924046)(141.71373783,94.44424046)(141.66373047,94.43424123)
\curveto(141.60373794,94.43424047)(141.54873799,94.44424046)(141.49873047,94.46424123)
\curveto(141.32873821,94.51424039)(141.16373838,94.55424035)(141.00373047,94.58424123)
\curveto(140.83373871,94.61424029)(140.67373887,94.66424024)(140.52373047,94.73424123)
\curveto(140.06373948,94.92423998)(139.68873985,95.14423976)(139.39873047,95.39424123)
\curveto(139.10874043,95.65423925)(138.86374068,96.01423889)(138.66373047,96.47424123)
\curveto(138.61374093,96.6042383)(138.57874096,96.73423817)(138.55873047,96.86424123)
\curveto(138.538741,97.0042379)(138.51374103,97.14423776)(138.48373047,97.28424123)
\curveto(138.47374107,97.35423755)(138.46874107,97.41923748)(138.46873047,97.47924123)
\curveto(138.46874107,97.53923736)(138.46374108,97.6042373)(138.45373047,97.67424123)
\curveto(138.43374111,98.5042364)(138.58374096,99.17423573)(138.90373047,99.68424123)
\curveto(139.21374033,100.19423471)(139.65373989,100.57423433)(140.22373047,100.82424123)
\curveto(140.3437392,100.87423403)(140.46873907,100.91923398)(140.59873047,100.95924123)
\curveto(140.72873881,100.9992339)(140.86373868,101.04423386)(141.00373047,101.09424123)
\curveto(141.08373846,101.11423379)(141.16873837,101.12923377)(141.25873047,101.13924123)
\lineto(141.49873047,101.19924123)
\curveto(141.60873793,101.22923367)(141.71873782,101.24423366)(141.82873047,101.24424123)
\curveto(141.9387376,101.25423365)(142.04873749,101.26923363)(142.15873047,101.28924123)
\curveto(142.20873733,101.30923359)(142.25373729,101.31423359)(142.29373047,101.30424123)
\curveto(142.33373721,101.3042336)(142.37373717,101.30923359)(142.41373047,101.31924123)
\curveto(142.46373708,101.32923357)(142.51873702,101.32923357)(142.57873047,101.31924123)
\curveto(142.62873691,101.31923358)(142.67873686,101.32423358)(142.72873047,101.33424123)
\lineto(142.86373047,101.33424123)
\curveto(142.92373662,101.35423355)(142.99373655,101.35423355)(143.07373047,101.33424123)
\curveto(143.1437364,101.32423358)(143.20873633,101.32923357)(143.26873047,101.34924123)
\curveto(143.29873624,101.35923354)(143.3387362,101.36423354)(143.38873047,101.36424123)
\lineto(143.50873047,101.36424123)
\lineto(143.97373047,101.36424123)
\moveto(146.29873047,99.81924123)
\curveto(145.97873356,99.91923498)(145.61373393,99.97923492)(145.20373047,99.99924123)
\curveto(144.79373475,100.01923488)(144.38373516,100.02923487)(143.97373047,100.02924123)
\curveto(143.543736,100.02923487)(143.12373642,100.01923488)(142.71373047,99.99924123)
\curveto(142.30373724,99.97923492)(141.91873762,99.93423497)(141.55873047,99.86424123)
\curveto(141.19873834,99.79423511)(140.87873866,99.68423522)(140.59873047,99.53424123)
\curveto(140.30873923,99.39423551)(140.07373947,99.1992357)(139.89373047,98.94924123)
\curveto(139.78373976,98.78923611)(139.70373984,98.60923629)(139.65373047,98.40924123)
\curveto(139.59373995,98.20923669)(139.56373998,97.96423694)(139.56373047,97.67424123)
\curveto(139.58373996,97.65423725)(139.59373995,97.61923728)(139.59373047,97.56924123)
\curveto(139.58373996,97.51923738)(139.58373996,97.47923742)(139.59373047,97.44924123)
\curveto(139.61373993,97.36923753)(139.63373991,97.29423761)(139.65373047,97.22424123)
\curveto(139.66373988,97.16423774)(139.68373986,97.0992378)(139.71373047,97.02924123)
\curveto(139.83373971,96.75923814)(140.00373954,96.53923836)(140.22373047,96.36924123)
\curveto(140.43373911,96.20923869)(140.67873886,96.07423883)(140.95873047,95.96424123)
\curveto(141.06873847,95.91423899)(141.18873835,95.87423903)(141.31873047,95.84424123)
\curveto(141.4387381,95.82423908)(141.56373798,95.7992391)(141.69373047,95.76924123)
\curveto(141.7437378,95.74923915)(141.79873774,95.73923916)(141.85873047,95.73924123)
\curveto(141.90873763,95.73923916)(141.95873758,95.73423917)(142.00873047,95.72424123)
\curveto(142.09873744,95.71423919)(142.19373735,95.7042392)(142.29373047,95.69424123)
\curveto(142.38373716,95.68423922)(142.47873706,95.67423923)(142.57873047,95.66424123)
\curveto(142.65873688,95.66423924)(142.7437368,95.65923924)(142.83373047,95.64924123)
\lineto(143.07373047,95.64924123)
\lineto(143.25373047,95.64924123)
\curveto(143.28373626,95.63923926)(143.31873622,95.63423927)(143.35873047,95.63424123)
\lineto(143.49373047,95.63424123)
\lineto(143.94373047,95.63424123)
\curveto(144.02373552,95.63423927)(144.10873543,95.62923927)(144.19873047,95.61924123)
\curveto(144.27873526,95.61923928)(144.35373519,95.62923927)(144.42373047,95.64924123)
\lineto(144.69373047,95.64924123)
\curveto(144.71373483,95.64923925)(144.7437348,95.64423926)(144.78373047,95.63424123)
\curveto(144.81373473,95.63423927)(144.8387347,95.63923926)(144.85873047,95.64924123)
\curveto(144.95873458,95.65923924)(145.05873448,95.66423924)(145.15873047,95.66424123)
\curveto(145.24873429,95.67423923)(145.34873419,95.68423922)(145.45873047,95.69424123)
\curveto(145.57873396,95.72423918)(145.70373384,95.73923916)(145.83373047,95.73924123)
\curveto(145.95373359,95.74923915)(146.06873347,95.77423913)(146.17873047,95.81424123)
\curveto(146.47873306,95.89423901)(146.7437328,95.97923892)(146.97373047,96.06924123)
\curveto(147.20373234,96.16923873)(147.41873212,96.31423859)(147.61873047,96.50424123)
\curveto(147.81873172,96.71423819)(147.96873157,96.97923792)(148.06873047,97.29924123)
\curveto(148.08873145,97.33923756)(148.09873144,97.37423753)(148.09873047,97.40424123)
\curveto(148.08873145,97.44423746)(148.09373145,97.48923741)(148.11373047,97.53924123)
\curveto(148.12373142,97.57923732)(148.13373141,97.64923725)(148.14373047,97.74924123)
\curveto(148.15373139,97.85923704)(148.14873139,97.94423696)(148.12873047,98.00424123)
\curveto(148.10873143,98.07423683)(148.09873144,98.14423676)(148.09873047,98.21424123)
\curveto(148.08873145,98.28423662)(148.07373147,98.34923655)(148.05373047,98.40924123)
\curveto(147.99373155,98.60923629)(147.90873163,98.78923611)(147.79873047,98.94924123)
\curveto(147.77873176,98.97923592)(147.75873178,99.0042359)(147.73873047,99.02424123)
\lineto(147.67873047,99.08424123)
\curveto(147.65873188,99.12423578)(147.61873192,99.17423573)(147.55873047,99.23424123)
\curveto(147.41873212,99.33423557)(147.28873225,99.41923548)(147.16873047,99.48924123)
\curveto(147.04873249,99.55923534)(146.90373264,99.62923527)(146.73373047,99.69924123)
\curveto(146.66373288,99.72923517)(146.59373295,99.74923515)(146.52373047,99.75924123)
\curveto(146.45373309,99.77923512)(146.37873316,99.7992351)(146.29873047,99.81924123)
}
}
{
\newrgbcolor{curcolor}{0 0 0}
\pscustom[linestyle=none,fillstyle=solid,fillcolor=curcolor]
{
\newpath
\moveto(138.45373047,106.77385061)
\curveto(138.45374109,106.87384575)(138.46374108,106.96884566)(138.48373047,107.05885061)
\curveto(138.49374105,107.14884548)(138.52374102,107.21384541)(138.57373047,107.25385061)
\curveto(138.65374089,107.31384531)(138.75874078,107.34384528)(138.88873047,107.34385061)
\lineto(139.27873047,107.34385061)
\lineto(140.77873047,107.34385061)
\lineto(147.16873047,107.34385061)
\lineto(148.33873047,107.34385061)
\lineto(148.65373047,107.34385061)
\curveto(148.75373079,107.35384527)(148.83373071,107.33884529)(148.89373047,107.29885061)
\curveto(148.97373057,107.24884538)(149.02373052,107.17384545)(149.04373047,107.07385061)
\curveto(149.05373049,106.98384564)(149.05873048,106.87384575)(149.05873047,106.74385061)
\lineto(149.05873047,106.51885061)
\curveto(149.0387305,106.43884619)(149.02373052,106.36884626)(149.01373047,106.30885061)
\curveto(148.99373055,106.24884638)(148.95373059,106.19884643)(148.89373047,106.15885061)
\curveto(148.83373071,106.11884651)(148.75873078,106.09884653)(148.66873047,106.09885061)
\lineto(148.36873047,106.09885061)
\lineto(147.27373047,106.09885061)
\lineto(141.93373047,106.09885061)
\curveto(141.8437377,106.07884655)(141.76873777,106.06384656)(141.70873047,106.05385061)
\curveto(141.6387379,106.05384657)(141.57873796,106.0238466)(141.52873047,105.96385061)
\curveto(141.47873806,105.89384673)(141.45373809,105.80384682)(141.45373047,105.69385061)
\curveto(141.4437381,105.59384703)(141.4387381,105.48384714)(141.43873047,105.36385061)
\lineto(141.43873047,104.22385061)
\lineto(141.43873047,103.72885061)
\curveto(141.42873811,103.56884906)(141.36873817,103.45884917)(141.25873047,103.39885061)
\curveto(141.22873831,103.37884925)(141.19873834,103.36884926)(141.16873047,103.36885061)
\curveto(141.12873841,103.36884926)(141.08373846,103.36384926)(141.03373047,103.35385061)
\curveto(140.91373863,103.33384929)(140.80373874,103.33884929)(140.70373047,103.36885061)
\curveto(140.60373894,103.40884922)(140.53373901,103.46384916)(140.49373047,103.53385061)
\curveto(140.4437391,103.61384901)(140.41873912,103.73384889)(140.41873047,103.89385061)
\curveto(140.41873912,104.05384857)(140.40373914,104.18884844)(140.37373047,104.29885061)
\curveto(140.36373918,104.34884828)(140.35873918,104.40384822)(140.35873047,104.46385061)
\curveto(140.34873919,104.5238481)(140.33373921,104.58384804)(140.31373047,104.64385061)
\curveto(140.26373928,104.79384783)(140.21373933,104.93884769)(140.16373047,105.07885061)
\curveto(140.10373944,105.21884741)(140.03373951,105.35384727)(139.95373047,105.48385061)
\curveto(139.86373968,105.623847)(139.75873978,105.74384688)(139.63873047,105.84385061)
\curveto(139.51874002,105.94384668)(139.38874015,106.03884659)(139.24873047,106.12885061)
\curveto(139.14874039,106.18884644)(139.0387405,106.23384639)(138.91873047,106.26385061)
\curveto(138.79874074,106.30384632)(138.69374085,106.35384627)(138.60373047,106.41385061)
\curveto(138.543741,106.46384616)(138.50374104,106.53384609)(138.48373047,106.62385061)
\curveto(138.47374107,106.64384598)(138.46874107,106.66884596)(138.46873047,106.69885061)
\curveto(138.46874107,106.7288459)(138.46374108,106.75384587)(138.45373047,106.77385061)
}
}
{
\newrgbcolor{curcolor}{0 0 0}
\pscustom[linestyle=none,fillstyle=solid,fillcolor=curcolor]
{
\newpath
\moveto(138.45373047,115.12345998)
\curveto(138.45374109,115.22345513)(138.46374108,115.31845503)(138.48373047,115.40845998)
\curveto(138.49374105,115.49845485)(138.52374102,115.56345479)(138.57373047,115.60345998)
\curveto(138.65374089,115.66345469)(138.75874078,115.69345466)(138.88873047,115.69345998)
\lineto(139.27873047,115.69345998)
\lineto(140.77873047,115.69345998)
\lineto(147.16873047,115.69345998)
\lineto(148.33873047,115.69345998)
\lineto(148.65373047,115.69345998)
\curveto(148.75373079,115.70345465)(148.83373071,115.68845466)(148.89373047,115.64845998)
\curveto(148.97373057,115.59845475)(149.02373052,115.52345483)(149.04373047,115.42345998)
\curveto(149.05373049,115.33345502)(149.05873048,115.22345513)(149.05873047,115.09345998)
\lineto(149.05873047,114.86845998)
\curveto(149.0387305,114.78845556)(149.02373052,114.71845563)(149.01373047,114.65845998)
\curveto(148.99373055,114.59845575)(148.95373059,114.5484558)(148.89373047,114.50845998)
\curveto(148.83373071,114.46845588)(148.75873078,114.4484559)(148.66873047,114.44845998)
\lineto(148.36873047,114.44845998)
\lineto(147.27373047,114.44845998)
\lineto(141.93373047,114.44845998)
\curveto(141.8437377,114.42845592)(141.76873777,114.41345594)(141.70873047,114.40345998)
\curveto(141.6387379,114.40345595)(141.57873796,114.37345598)(141.52873047,114.31345998)
\curveto(141.47873806,114.24345611)(141.45373809,114.1534562)(141.45373047,114.04345998)
\curveto(141.4437381,113.94345641)(141.4387381,113.83345652)(141.43873047,113.71345998)
\lineto(141.43873047,112.57345998)
\lineto(141.43873047,112.07845998)
\curveto(141.42873811,111.91845843)(141.36873817,111.80845854)(141.25873047,111.74845998)
\curveto(141.22873831,111.72845862)(141.19873834,111.71845863)(141.16873047,111.71845998)
\curveto(141.12873841,111.71845863)(141.08373846,111.71345864)(141.03373047,111.70345998)
\curveto(140.91373863,111.68345867)(140.80373874,111.68845866)(140.70373047,111.71845998)
\curveto(140.60373894,111.75845859)(140.53373901,111.81345854)(140.49373047,111.88345998)
\curveto(140.4437391,111.96345839)(140.41873912,112.08345827)(140.41873047,112.24345998)
\curveto(140.41873912,112.40345795)(140.40373914,112.53845781)(140.37373047,112.64845998)
\curveto(140.36373918,112.69845765)(140.35873918,112.7534576)(140.35873047,112.81345998)
\curveto(140.34873919,112.87345748)(140.33373921,112.93345742)(140.31373047,112.99345998)
\curveto(140.26373928,113.14345721)(140.21373933,113.28845706)(140.16373047,113.42845998)
\curveto(140.10373944,113.56845678)(140.03373951,113.70345665)(139.95373047,113.83345998)
\curveto(139.86373968,113.97345638)(139.75873978,114.09345626)(139.63873047,114.19345998)
\curveto(139.51874002,114.29345606)(139.38874015,114.38845596)(139.24873047,114.47845998)
\curveto(139.14874039,114.53845581)(139.0387405,114.58345577)(138.91873047,114.61345998)
\curveto(138.79874074,114.6534557)(138.69374085,114.70345565)(138.60373047,114.76345998)
\curveto(138.543741,114.81345554)(138.50374104,114.88345547)(138.48373047,114.97345998)
\curveto(138.47374107,114.99345536)(138.46874107,115.01845533)(138.46873047,115.04845998)
\curveto(138.46874107,115.07845527)(138.46374108,115.10345525)(138.45373047,115.12345998)
}
}
{
\newrgbcolor{curcolor}{0 0 0}
\pscustom[linestyle=none,fillstyle=solid,fillcolor=curcolor]
{
\newpath
\moveto(159.29006165,31.67142873)
\lineto(159.29006165,32.58642873)
\curveto(159.29007234,32.68642608)(159.29007234,32.78142599)(159.29006165,32.87142873)
\curveto(159.29007234,32.96142581)(159.31007232,33.03642573)(159.35006165,33.09642873)
\curveto(159.41007222,33.18642558)(159.49007214,33.24642552)(159.59006165,33.27642873)
\curveto(159.69007194,33.31642545)(159.79507184,33.36142541)(159.90506165,33.41142873)
\curveto(160.09507154,33.49142528)(160.28507135,33.56142521)(160.47506165,33.62142873)
\curveto(160.66507097,33.69142508)(160.85507078,33.766425)(161.04506165,33.84642873)
\curveto(161.22507041,33.91642485)(161.41007022,33.98142479)(161.60006165,34.04142873)
\curveto(161.78006985,34.10142467)(161.96006967,34.1714246)(162.14006165,34.25142873)
\curveto(162.28006935,34.31142446)(162.42506921,34.3664244)(162.57506165,34.41642873)
\curveto(162.72506891,34.4664243)(162.87006876,34.52142425)(163.01006165,34.58142873)
\curveto(163.46006817,34.76142401)(163.91506772,34.93142384)(164.37506165,35.09142873)
\curveto(164.82506681,35.25142352)(165.27506636,35.42142335)(165.72506165,35.60142873)
\curveto(165.77506586,35.62142315)(165.82506581,35.63642313)(165.87506165,35.64642873)
\lineto(166.02506165,35.70642873)
\curveto(166.24506539,35.79642297)(166.47006516,35.88142289)(166.70006165,35.96142873)
\curveto(166.92006471,36.04142273)(167.14006449,36.12642264)(167.36006165,36.21642873)
\curveto(167.45006418,36.25642251)(167.56006407,36.29642247)(167.69006165,36.33642873)
\curveto(167.81006382,36.37642239)(167.88006375,36.44142233)(167.90006165,36.53142873)
\curveto(167.91006372,36.5714222)(167.91006372,36.60142217)(167.90006165,36.62142873)
\lineto(167.84006165,36.68142873)
\curveto(167.79006384,36.73142204)(167.7350639,36.766422)(167.67506165,36.78642873)
\curveto(167.61506402,36.81642195)(167.55006408,36.84642192)(167.48006165,36.87642873)
\lineto(166.85006165,37.11642873)
\curveto(166.630065,37.19642157)(166.41506522,37.27642149)(166.20506165,37.35642873)
\lineto(166.05506165,37.41642873)
\lineto(165.87506165,37.47642873)
\curveto(165.68506595,37.55642121)(165.49506614,37.62642114)(165.30506165,37.68642873)
\curveto(165.10506653,37.75642101)(164.90506673,37.83142094)(164.70506165,37.91142873)
\curveto(164.12506751,38.15142062)(163.54006809,38.3714204)(162.95006165,38.57142873)
\curveto(162.36006927,38.78141999)(161.77506986,39.00641976)(161.19506165,39.24642873)
\curveto(160.99507064,39.32641944)(160.79007084,39.40141937)(160.58006165,39.47142873)
\curveto(160.37007126,39.55141922)(160.16507147,39.63141914)(159.96506165,39.71142873)
\curveto(159.88507175,39.75141902)(159.78507185,39.78641898)(159.66506165,39.81642873)
\curveto(159.54507209,39.85641891)(159.46007217,39.91141886)(159.41006165,39.98142873)
\curveto(159.37007226,40.04141873)(159.34007229,40.11641865)(159.32006165,40.20642873)
\curveto(159.30007233,40.30641846)(159.29007234,40.41641835)(159.29006165,40.53642873)
\curveto(159.28007235,40.65641811)(159.28007235,40.77641799)(159.29006165,40.89642873)
\curveto(159.29007234,41.01641775)(159.29007234,41.12641764)(159.29006165,41.22642873)
\curveto(159.29007234,41.31641745)(159.29007234,41.40641736)(159.29006165,41.49642873)
\curveto(159.29007234,41.59641717)(159.31007232,41.6714171)(159.35006165,41.72142873)
\curveto(159.40007223,41.81141696)(159.49007214,41.86141691)(159.62006165,41.87142873)
\curveto(159.75007188,41.88141689)(159.89007174,41.88641688)(160.04006165,41.88642873)
\lineto(161.69006165,41.88642873)
\lineto(167.96006165,41.88642873)
\lineto(169.22006165,41.88642873)
\curveto(169.3300623,41.88641688)(169.44006219,41.88641688)(169.55006165,41.88642873)
\curveto(169.66006197,41.89641687)(169.74506189,41.87641689)(169.80506165,41.82642873)
\curveto(169.86506177,41.79641697)(169.90506173,41.75141702)(169.92506165,41.69142873)
\curveto(169.9350617,41.63141714)(169.95006168,41.56141721)(169.97006165,41.48142873)
\lineto(169.97006165,41.24142873)
\lineto(169.97006165,40.88142873)
\curveto(169.96006167,40.771418)(169.91506172,40.69141808)(169.83506165,40.64142873)
\curveto(169.80506183,40.62141815)(169.77506186,40.60641816)(169.74506165,40.59642873)
\curveto(169.70506193,40.59641817)(169.66006197,40.58641818)(169.61006165,40.56642873)
\lineto(169.44506165,40.56642873)
\curveto(169.38506225,40.55641821)(169.31506232,40.55141822)(169.23506165,40.55142873)
\curveto(169.15506248,40.56141821)(169.08006255,40.5664182)(169.01006165,40.56642873)
\lineto(168.17006165,40.56642873)
\lineto(163.74506165,40.56642873)
\curveto(163.49506814,40.5664182)(163.24506839,40.5664182)(162.99506165,40.56642873)
\curveto(162.7350689,40.5664182)(162.48506915,40.56141821)(162.24506165,40.55142873)
\curveto(162.14506949,40.55141822)(162.0350696,40.54641822)(161.91506165,40.53642873)
\curveto(161.79506984,40.52641824)(161.7350699,40.4714183)(161.73506165,40.37142873)
\lineto(161.75006165,40.37142873)
\curveto(161.77006986,40.30141847)(161.8350698,40.24141853)(161.94506165,40.19142873)
\curveto(162.05506958,40.15141862)(162.15006948,40.11641865)(162.23006165,40.08642873)
\curveto(162.40006923,40.01641875)(162.57506906,39.95141882)(162.75506165,39.89142873)
\curveto(162.92506871,39.83141894)(163.09506854,39.76141901)(163.26506165,39.68142873)
\curveto(163.31506832,39.66141911)(163.36006827,39.64641912)(163.40006165,39.63642873)
\curveto(163.44006819,39.62641914)(163.48506815,39.61141916)(163.53506165,39.59142873)
\curveto(163.71506792,39.51141926)(163.90006773,39.44141933)(164.09006165,39.38142873)
\curveto(164.27006736,39.33141944)(164.45006718,39.2664195)(164.63006165,39.18642873)
\curveto(164.78006685,39.11641965)(164.9350667,39.05641971)(165.09506165,39.00642873)
\curveto(165.24506639,38.95641981)(165.39506624,38.90141987)(165.54506165,38.84142873)
\curveto(166.01506562,38.64142013)(166.49006514,38.46142031)(166.97006165,38.30142873)
\curveto(167.44006419,38.14142063)(167.90506373,37.9664208)(168.36506165,37.77642873)
\curveto(168.54506309,37.69642107)(168.72506291,37.62642114)(168.90506165,37.56642873)
\curveto(169.08506255,37.50642126)(169.26506237,37.44142133)(169.44506165,37.37142873)
\curveto(169.55506208,37.32142145)(169.66006197,37.2714215)(169.76006165,37.22142873)
\curveto(169.85006178,37.18142159)(169.91506172,37.09642167)(169.95506165,36.96642873)
\curveto(169.96506167,36.94642182)(169.97006166,36.92142185)(169.97006165,36.89142873)
\curveto(169.96006167,36.8714219)(169.96006167,36.84642192)(169.97006165,36.81642873)
\curveto(169.98006165,36.78642198)(169.98506165,36.75142202)(169.98506165,36.71142873)
\curveto(169.97506166,36.6714221)(169.97006166,36.63142214)(169.97006165,36.59142873)
\lineto(169.97006165,36.29142873)
\curveto(169.97006166,36.19142258)(169.94506169,36.11142266)(169.89506165,36.05142873)
\curveto(169.84506179,35.9714228)(169.77506186,35.91142286)(169.68506165,35.87142873)
\curveto(169.58506205,35.84142293)(169.48506215,35.80142297)(169.38506165,35.75142873)
\curveto(169.18506245,35.6714231)(168.98006265,35.59142318)(168.77006165,35.51142873)
\curveto(168.55006308,35.44142333)(168.34006329,35.3664234)(168.14006165,35.28642873)
\curveto(167.96006367,35.20642356)(167.78006385,35.13642363)(167.60006165,35.07642873)
\curveto(167.41006422,35.02642374)(167.22506441,34.96142381)(167.04506165,34.88142873)
\curveto(166.48506515,34.65142412)(165.92006571,34.43642433)(165.35006165,34.23642873)
\curveto(164.78006685,34.03642473)(164.21506742,33.82142495)(163.65506165,33.59142873)
\lineto(163.02506165,33.35142873)
\curveto(162.80506883,33.28142549)(162.59506904,33.20642556)(162.39506165,33.12642873)
\curveto(162.28506935,33.07642569)(162.18006945,33.03142574)(162.08006165,32.99142873)
\curveto(161.97006966,32.96142581)(161.87506976,32.91142586)(161.79506165,32.84142873)
\curveto(161.77506986,32.83142594)(161.76506987,32.82142595)(161.76506165,32.81142873)
\lineto(161.73506165,32.78142873)
\lineto(161.73506165,32.70642873)
\lineto(161.76506165,32.67642873)
\curveto(161.76506987,32.6664261)(161.77006986,32.65642611)(161.78006165,32.64642873)
\curveto(161.8300698,32.62642614)(161.88506975,32.61642615)(161.94506165,32.61642873)
\curveto(162.00506963,32.61642615)(162.06506957,32.60642616)(162.12506165,32.58642873)
\lineto(162.29006165,32.58642873)
\curveto(162.35006928,32.5664262)(162.41506922,32.56142621)(162.48506165,32.57142873)
\curveto(162.55506908,32.58142619)(162.62506901,32.58642618)(162.69506165,32.58642873)
\lineto(163.50506165,32.58642873)
\lineto(168.06506165,32.58642873)
\lineto(169.25006165,32.58642873)
\curveto(169.36006227,32.58642618)(169.47006216,32.58142619)(169.58006165,32.57142873)
\curveto(169.69006194,32.5714262)(169.77506186,32.54642622)(169.83506165,32.49642873)
\curveto(169.91506172,32.44642632)(169.96006167,32.35642641)(169.97006165,32.22642873)
\lineto(169.97006165,31.83642873)
\lineto(169.97006165,31.64142873)
\curveto(169.97006166,31.59142718)(169.96006167,31.54142723)(169.94006165,31.49142873)
\curveto(169.90006173,31.36142741)(169.81506182,31.28642748)(169.68506165,31.26642873)
\curveto(169.55506208,31.25642751)(169.40506223,31.25142752)(169.23506165,31.25142873)
\lineto(167.49506165,31.25142873)
\lineto(161.49506165,31.25142873)
\lineto(160.08506165,31.25142873)
\curveto(159.97507166,31.25142752)(159.86007177,31.24642752)(159.74006165,31.23642873)
\curveto(159.62007201,31.23642753)(159.52507211,31.26142751)(159.45506165,31.31142873)
\curveto(159.39507224,31.35142742)(159.34507229,31.42642734)(159.30506165,31.53642873)
\curveto(159.29507234,31.55642721)(159.29507234,31.57642719)(159.30506165,31.59642873)
\curveto(159.30507233,31.62642714)(159.30007233,31.65142712)(159.29006165,31.67142873)
}
}
{
\newrgbcolor{curcolor}{0 0 0}
\pscustom[linestyle=none,fillstyle=solid,fillcolor=curcolor]
{
\newpath
\moveto(169.41506165,50.87353811)
\curveto(169.57506206,50.90353028)(169.71006192,50.88853029)(169.82006165,50.82853811)
\curveto(169.92006171,50.76853041)(169.99506164,50.68853049)(170.04506165,50.58853811)
\curveto(170.06506157,50.53853064)(170.07506156,50.4835307)(170.07506165,50.42353811)
\curveto(170.07506156,50.37353081)(170.08506155,50.31853086)(170.10506165,50.25853811)
\curveto(170.15506148,50.03853114)(170.14006149,49.81853136)(170.06006165,49.59853811)
\curveto(169.99006164,49.38853179)(169.90006173,49.24353194)(169.79006165,49.16353811)
\curveto(169.72006191,49.11353207)(169.64006199,49.06853211)(169.55006165,49.02853811)
\curveto(169.45006218,48.98853219)(169.37006226,48.93853224)(169.31006165,48.87853811)
\curveto(169.29006234,48.85853232)(169.27006236,48.83353235)(169.25006165,48.80353811)
\curveto(169.2300624,48.7835324)(169.22506241,48.75353243)(169.23506165,48.71353811)
\curveto(169.26506237,48.60353258)(169.32006231,48.49853268)(169.40006165,48.39853811)
\curveto(169.48006215,48.30853287)(169.55006208,48.21853296)(169.61006165,48.12853811)
\curveto(169.69006194,47.99853318)(169.76506187,47.85853332)(169.83506165,47.70853811)
\curveto(169.89506174,47.55853362)(169.95006168,47.39853378)(170.00006165,47.22853811)
\curveto(170.0300616,47.12853405)(170.05006158,47.01853416)(170.06006165,46.89853811)
\curveto(170.07006156,46.78853439)(170.08506155,46.6785345)(170.10506165,46.56853811)
\curveto(170.11506152,46.51853466)(170.12006151,46.47353471)(170.12006165,46.43353811)
\lineto(170.12006165,46.32853811)
\curveto(170.14006149,46.21853496)(170.14006149,46.11353507)(170.12006165,46.01353811)
\lineto(170.12006165,45.87853811)
\curveto(170.11006152,45.82853535)(170.10506153,45.7785354)(170.10506165,45.72853811)
\curveto(170.10506153,45.6785355)(170.09506154,45.63353555)(170.07506165,45.59353811)
\curveto(170.06506157,45.55353563)(170.06006157,45.51853566)(170.06006165,45.48853811)
\curveto(170.07006156,45.46853571)(170.07006156,45.44353574)(170.06006165,45.41353811)
\lineto(170.00006165,45.17353811)
\curveto(169.99006164,45.09353609)(169.97006166,45.01853616)(169.94006165,44.94853811)
\curveto(169.81006182,44.64853653)(169.66506197,44.40353678)(169.50506165,44.21353811)
\curveto(169.3350623,44.03353715)(169.10006253,43.8835373)(168.80006165,43.76353811)
\curveto(168.58006305,43.67353751)(168.31506332,43.62853755)(168.00506165,43.62853811)
\lineto(167.69006165,43.62853811)
\curveto(167.64006399,43.63853754)(167.59006404,43.64353754)(167.54006165,43.64353811)
\lineto(167.36006165,43.67353811)
\lineto(167.03006165,43.79353811)
\curveto(166.92006471,43.83353735)(166.82006481,43.8835373)(166.73006165,43.94353811)
\curveto(166.44006519,44.12353706)(166.22506541,44.36853681)(166.08506165,44.67853811)
\curveto(165.94506569,44.98853619)(165.82006581,45.32853585)(165.71006165,45.69853811)
\curveto(165.67006596,45.83853534)(165.64006599,45.9835352)(165.62006165,46.13353811)
\curveto(165.60006603,46.2835349)(165.57506606,46.43353475)(165.54506165,46.58353811)
\curveto(165.52506611,46.65353453)(165.51506612,46.71853446)(165.51506165,46.77853811)
\curveto(165.51506612,46.84853433)(165.50506613,46.92353426)(165.48506165,47.00353811)
\curveto(165.46506617,47.07353411)(165.45506618,47.14353404)(165.45506165,47.21353811)
\curveto(165.44506619,47.2835339)(165.4300662,47.35853382)(165.41006165,47.43853811)
\curveto(165.35006628,47.68853349)(165.30006633,47.92353326)(165.26006165,48.14353811)
\curveto(165.21006642,48.36353282)(165.09506654,48.53853264)(164.91506165,48.66853811)
\curveto(164.8350668,48.72853245)(164.7350669,48.7785324)(164.61506165,48.81853811)
\curveto(164.48506715,48.85853232)(164.34506729,48.85853232)(164.19506165,48.81853811)
\curveto(163.95506768,48.75853242)(163.76506787,48.66853251)(163.62506165,48.54853811)
\curveto(163.48506815,48.43853274)(163.37506826,48.2785329)(163.29506165,48.06853811)
\curveto(163.24506839,47.94853323)(163.21006842,47.80353338)(163.19006165,47.63353811)
\curveto(163.17006846,47.47353371)(163.16006847,47.30353388)(163.16006165,47.12353811)
\curveto(163.16006847,46.94353424)(163.17006846,46.76853441)(163.19006165,46.59853811)
\curveto(163.21006842,46.42853475)(163.24006839,46.2835349)(163.28006165,46.16353811)
\curveto(163.34006829,45.99353519)(163.42506821,45.82853535)(163.53506165,45.66853811)
\curveto(163.59506804,45.58853559)(163.67506796,45.51353567)(163.77506165,45.44353811)
\curveto(163.86506777,45.3835358)(163.96506767,45.32853585)(164.07506165,45.27853811)
\curveto(164.15506748,45.24853593)(164.24006739,45.21853596)(164.33006165,45.18853811)
\curveto(164.42006721,45.16853601)(164.49006714,45.12353606)(164.54006165,45.05353811)
\curveto(164.57006706,45.01353617)(164.59506704,44.94353624)(164.61506165,44.84353811)
\curveto(164.62506701,44.75353643)(164.630067,44.65853652)(164.63006165,44.55853811)
\curveto(164.630067,44.45853672)(164.62506701,44.35853682)(164.61506165,44.25853811)
\curveto(164.59506704,44.16853701)(164.57006706,44.10353708)(164.54006165,44.06353811)
\curveto(164.51006712,44.02353716)(164.46006717,43.99353719)(164.39006165,43.97353811)
\curveto(164.32006731,43.95353723)(164.24506739,43.95353723)(164.16506165,43.97353811)
\curveto(164.0350676,44.00353718)(163.91506772,44.03353715)(163.80506165,44.06353811)
\curveto(163.68506795,44.10353708)(163.57006806,44.14853703)(163.46006165,44.19853811)
\curveto(163.11006852,44.38853679)(162.84006879,44.62853655)(162.65006165,44.91853811)
\curveto(162.45006918,45.20853597)(162.29006934,45.56853561)(162.17006165,45.99853811)
\curveto(162.15006948,46.09853508)(162.1350695,46.19853498)(162.12506165,46.29853811)
\curveto(162.11506952,46.40853477)(162.10006953,46.51853466)(162.08006165,46.62853811)
\curveto(162.07006956,46.66853451)(162.07006956,46.73353445)(162.08006165,46.82353811)
\curveto(162.08006955,46.91353427)(162.07006956,46.96853421)(162.05006165,46.98853811)
\curveto(162.04006959,47.68853349)(162.12006951,48.29853288)(162.29006165,48.81853811)
\curveto(162.46006917,49.33853184)(162.78506885,49.70353148)(163.26506165,49.91353811)
\curveto(163.46506817,50.00353118)(163.70006793,50.05353113)(163.97006165,50.06353811)
\curveto(164.2300674,50.0835311)(164.50506713,50.09353109)(164.79506165,50.09353811)
\lineto(168.11006165,50.09353811)
\curveto(168.25006338,50.09353109)(168.38506325,50.09853108)(168.51506165,50.10853811)
\curveto(168.64506299,50.11853106)(168.75006288,50.14853103)(168.83006165,50.19853811)
\curveto(168.90006273,50.24853093)(168.95006268,50.31353087)(168.98006165,50.39353811)
\curveto(169.02006261,50.4835307)(169.05006258,50.56853061)(169.07006165,50.64853811)
\curveto(169.08006255,50.72853045)(169.12506251,50.78853039)(169.20506165,50.82853811)
\curveto(169.2350624,50.84853033)(169.26506237,50.85853032)(169.29506165,50.85853811)
\curveto(169.32506231,50.85853032)(169.36506227,50.86353032)(169.41506165,50.87353811)
\moveto(167.75006165,48.72853811)
\curveto(167.61006402,48.78853239)(167.45006418,48.81853236)(167.27006165,48.81853811)
\curveto(167.08006455,48.82853235)(166.88506475,48.83353235)(166.68506165,48.83353811)
\curveto(166.57506506,48.83353235)(166.47506516,48.82853235)(166.38506165,48.81853811)
\curveto(166.29506534,48.80853237)(166.22506541,48.76853241)(166.17506165,48.69853811)
\curveto(166.15506548,48.66853251)(166.14506549,48.59853258)(166.14506165,48.48853811)
\curveto(166.16506547,48.46853271)(166.17506546,48.43353275)(166.17506165,48.38353811)
\curveto(166.17506546,48.33353285)(166.18506545,48.28853289)(166.20506165,48.24853811)
\curveto(166.22506541,48.16853301)(166.24506539,48.0785331)(166.26506165,47.97853811)
\lineto(166.32506165,47.67853811)
\curveto(166.32506531,47.64853353)(166.3300653,47.61353357)(166.34006165,47.57353811)
\lineto(166.34006165,47.46853811)
\curveto(166.38006525,47.31853386)(166.40506523,47.15353403)(166.41506165,46.97353811)
\curveto(166.41506522,46.80353438)(166.4350652,46.64353454)(166.47506165,46.49353811)
\curveto(166.49506514,46.41353477)(166.51506512,46.33853484)(166.53506165,46.26853811)
\curveto(166.54506509,46.20853497)(166.56006507,46.13853504)(166.58006165,46.05853811)
\curveto(166.630065,45.89853528)(166.69506494,45.74853543)(166.77506165,45.60853811)
\curveto(166.84506479,45.46853571)(166.9350647,45.34853583)(167.04506165,45.24853811)
\curveto(167.15506448,45.14853603)(167.29006434,45.07353611)(167.45006165,45.02353811)
\curveto(167.60006403,44.97353621)(167.78506385,44.95353623)(168.00506165,44.96353811)
\curveto(168.10506353,44.96353622)(168.20006343,44.9785362)(168.29006165,45.00853811)
\curveto(168.37006326,45.04853613)(168.44506319,45.09353609)(168.51506165,45.14353811)
\curveto(168.62506301,45.22353596)(168.72006291,45.32853585)(168.80006165,45.45853811)
\curveto(168.87006276,45.58853559)(168.9300627,45.72853545)(168.98006165,45.87853811)
\curveto(168.99006264,45.92853525)(168.99506264,45.9785352)(168.99506165,46.02853811)
\curveto(168.99506264,46.0785351)(169.00006263,46.12853505)(169.01006165,46.17853811)
\curveto(169.0300626,46.24853493)(169.04506259,46.33353485)(169.05506165,46.43353811)
\curveto(169.05506258,46.54353464)(169.04506259,46.63353455)(169.02506165,46.70353811)
\curveto(169.00506263,46.76353442)(169.00006263,46.82353436)(169.01006165,46.88353811)
\curveto(169.01006262,46.94353424)(169.00006263,47.00353418)(168.98006165,47.06353811)
\curveto(168.96006267,47.14353404)(168.94506269,47.21853396)(168.93506165,47.28853811)
\curveto(168.92506271,47.36853381)(168.90506273,47.44353374)(168.87506165,47.51353811)
\curveto(168.75506288,47.80353338)(168.61006302,48.04853313)(168.44006165,48.24853811)
\curveto(168.27006336,48.45853272)(168.04006359,48.61853256)(167.75006165,48.72853811)
}
}
{
\newrgbcolor{curcolor}{0 0 0}
\pscustom[linestyle=none,fillstyle=solid,fillcolor=curcolor]
{
\newpath
\moveto(162.06506165,55.69017873)
\curveto(162.06506957,55.92017394)(162.12506951,56.05017381)(162.24506165,56.08017873)
\curveto(162.35506928,56.11017375)(162.52006911,56.12517374)(162.74006165,56.12517873)
\lineto(163.02506165,56.12517873)
\curveto(163.11506852,56.12517374)(163.19006844,56.10017376)(163.25006165,56.05017873)
\curveto(163.3300683,55.99017387)(163.37506826,55.90517396)(163.38506165,55.79517873)
\curveto(163.38506825,55.68517418)(163.40006823,55.57517429)(163.43006165,55.46517873)
\curveto(163.46006817,55.32517454)(163.49006814,55.19017467)(163.52006165,55.06017873)
\curveto(163.55006808,54.94017492)(163.59006804,54.82517504)(163.64006165,54.71517873)
\curveto(163.77006786,54.42517544)(163.95006768,54.19017567)(164.18006165,54.01017873)
\curveto(164.40006723,53.83017603)(164.65506698,53.67517619)(164.94506165,53.54517873)
\curveto(165.05506658,53.50517636)(165.17006646,53.47517639)(165.29006165,53.45517873)
\curveto(165.40006623,53.43517643)(165.51506612,53.41017645)(165.63506165,53.38017873)
\curveto(165.68506595,53.37017649)(165.7350659,53.3651765)(165.78506165,53.36517873)
\curveto(165.8350658,53.37517649)(165.88506575,53.37517649)(165.93506165,53.36517873)
\curveto(166.05506558,53.33517653)(166.19506544,53.32017654)(166.35506165,53.32017873)
\curveto(166.50506513,53.33017653)(166.65006498,53.33517653)(166.79006165,53.33517873)
\lineto(168.63506165,53.33517873)
\lineto(168.98006165,53.33517873)
\curveto(169.10006253,53.33517653)(169.21506242,53.33017653)(169.32506165,53.32017873)
\curveto(169.4350622,53.31017655)(169.5300621,53.30517656)(169.61006165,53.30517873)
\curveto(169.69006194,53.31517655)(169.76006187,53.29517657)(169.82006165,53.24517873)
\curveto(169.89006174,53.19517667)(169.9300617,53.11517675)(169.94006165,53.00517873)
\curveto(169.95006168,52.90517696)(169.95506168,52.79517707)(169.95506165,52.67517873)
\lineto(169.95506165,52.40517873)
\curveto(169.9350617,52.35517751)(169.92006171,52.30517756)(169.91006165,52.25517873)
\curveto(169.89006174,52.21517765)(169.86506177,52.18517768)(169.83506165,52.16517873)
\curveto(169.76506187,52.11517775)(169.68006195,52.08517778)(169.58006165,52.07517873)
\lineto(169.25006165,52.07517873)
\lineto(168.09506165,52.07517873)
\lineto(163.94006165,52.07517873)
\lineto(162.90506165,52.07517873)
\lineto(162.60506165,52.07517873)
\curveto(162.50506913,52.08517778)(162.42006921,52.11517775)(162.35006165,52.16517873)
\curveto(162.31006932,52.19517767)(162.28006935,52.24517762)(162.26006165,52.31517873)
\curveto(162.24006939,52.39517747)(162.2300694,52.48017738)(162.23006165,52.57017873)
\curveto(162.22006941,52.6601772)(162.22006941,52.75017711)(162.23006165,52.84017873)
\curveto(162.24006939,52.93017693)(162.25506938,53.00017686)(162.27506165,53.05017873)
\curveto(162.30506933,53.13017673)(162.36506927,53.18017668)(162.45506165,53.20017873)
\curveto(162.5350691,53.23017663)(162.62506901,53.24517662)(162.72506165,53.24517873)
\lineto(163.02506165,53.24517873)
\curveto(163.12506851,53.24517662)(163.21506842,53.2651766)(163.29506165,53.30517873)
\curveto(163.31506832,53.31517655)(163.3300683,53.32517654)(163.34006165,53.33517873)
\lineto(163.38506165,53.38017873)
\curveto(163.38506825,53.49017637)(163.34006829,53.58017628)(163.25006165,53.65017873)
\curveto(163.15006848,53.72017614)(163.07006856,53.78017608)(163.01006165,53.83017873)
\lineto(162.92006165,53.92017873)
\curveto(162.81006882,54.01017585)(162.69506894,54.13517573)(162.57506165,54.29517873)
\curveto(162.45506918,54.45517541)(162.36506927,54.60517526)(162.30506165,54.74517873)
\curveto(162.25506938,54.83517503)(162.22006941,54.93017493)(162.20006165,55.03017873)
\curveto(162.17006946,55.13017473)(162.14006949,55.23517463)(162.11006165,55.34517873)
\curveto(162.10006953,55.40517446)(162.09506954,55.4651744)(162.09506165,55.52517873)
\curveto(162.08506955,55.58517428)(162.07506956,55.64017422)(162.06506165,55.69017873)
}
}
{
\newrgbcolor{curcolor}{0 0 0}
\pscustom[linestyle=none,fillstyle=solid,fillcolor=curcolor]
{
}
}
{
\newrgbcolor{curcolor}{0 0 0}
\pscustom[linestyle=none,fillstyle=solid,fillcolor=curcolor]
{
\newpath
\moveto(159.36506165,65.05510061)
\curveto(159.36507227,65.15509575)(159.37507226,65.25009566)(159.39506165,65.34010061)
\curveto(159.40507223,65.43009548)(159.4350722,65.49509541)(159.48506165,65.53510061)
\curveto(159.56507207,65.59509531)(159.67007196,65.62509528)(159.80006165,65.62510061)
\lineto(160.19006165,65.62510061)
\lineto(161.69006165,65.62510061)
\lineto(168.08006165,65.62510061)
\lineto(169.25006165,65.62510061)
\lineto(169.56506165,65.62510061)
\curveto(169.66506197,65.63509527)(169.74506189,65.62009529)(169.80506165,65.58010061)
\curveto(169.88506175,65.53009538)(169.9350617,65.45509545)(169.95506165,65.35510061)
\curveto(169.96506167,65.26509564)(169.97006166,65.15509575)(169.97006165,65.02510061)
\lineto(169.97006165,64.80010061)
\curveto(169.95006168,64.72009619)(169.9350617,64.65009626)(169.92506165,64.59010061)
\curveto(169.90506173,64.53009638)(169.86506177,64.48009643)(169.80506165,64.44010061)
\curveto(169.74506189,64.40009651)(169.67006196,64.38009653)(169.58006165,64.38010061)
\lineto(169.28006165,64.38010061)
\lineto(168.18506165,64.38010061)
\lineto(162.84506165,64.38010061)
\curveto(162.75506888,64.36009655)(162.68006895,64.34509656)(162.62006165,64.33510061)
\curveto(162.55006908,64.33509657)(162.49006914,64.3050966)(162.44006165,64.24510061)
\curveto(162.39006924,64.17509673)(162.36506927,64.08509682)(162.36506165,63.97510061)
\curveto(162.35506928,63.87509703)(162.35006928,63.76509714)(162.35006165,63.64510061)
\lineto(162.35006165,62.50510061)
\lineto(162.35006165,62.01010061)
\curveto(162.34006929,61.85009906)(162.28006935,61.74009917)(162.17006165,61.68010061)
\curveto(162.14006949,61.66009925)(162.11006952,61.65009926)(162.08006165,61.65010061)
\curveto(162.04006959,61.65009926)(161.99506964,61.64509926)(161.94506165,61.63510061)
\curveto(161.82506981,61.61509929)(161.71506992,61.62009929)(161.61506165,61.65010061)
\curveto(161.51507012,61.69009922)(161.44507019,61.74509916)(161.40506165,61.81510061)
\curveto(161.35507028,61.89509901)(161.3300703,62.01509889)(161.33006165,62.17510061)
\curveto(161.3300703,62.33509857)(161.31507032,62.47009844)(161.28506165,62.58010061)
\curveto(161.27507036,62.63009828)(161.27007036,62.68509822)(161.27006165,62.74510061)
\curveto(161.26007037,62.8050981)(161.24507039,62.86509804)(161.22506165,62.92510061)
\curveto(161.17507046,63.07509783)(161.12507051,63.22009769)(161.07506165,63.36010061)
\curveto(161.01507062,63.50009741)(160.94507069,63.63509727)(160.86506165,63.76510061)
\curveto(160.77507086,63.905097)(160.67007096,64.02509688)(160.55006165,64.12510061)
\curveto(160.4300712,64.22509668)(160.30007133,64.32009659)(160.16006165,64.41010061)
\curveto(160.06007157,64.47009644)(159.95007168,64.51509639)(159.83006165,64.54510061)
\curveto(159.71007192,64.58509632)(159.60507203,64.63509627)(159.51506165,64.69510061)
\curveto(159.45507218,64.74509616)(159.41507222,64.81509609)(159.39506165,64.90510061)
\curveto(159.38507225,64.92509598)(159.38007225,64.95009596)(159.38006165,64.98010061)
\curveto(159.38007225,65.0100959)(159.37507226,65.03509587)(159.36506165,65.05510061)
}
}
{
\newrgbcolor{curcolor}{0 0 0}
\pscustom[linestyle=none,fillstyle=solid,fillcolor=curcolor]
{
\newpath
\moveto(164.88506165,76.34470998)
\lineto(165.14006165,76.34470998)
\curveto(165.22006641,76.35470228)(165.29506634,76.34970228)(165.36506165,76.32970998)
\lineto(165.60506165,76.32970998)
\lineto(165.77006165,76.32970998)
\curveto(165.87006576,76.30970232)(165.97506566,76.29970233)(166.08506165,76.29970998)
\curveto(166.18506545,76.29970233)(166.28506535,76.28970234)(166.38506165,76.26970998)
\lineto(166.53506165,76.26970998)
\curveto(166.67506496,76.23970239)(166.81506482,76.21970241)(166.95506165,76.20970998)
\curveto(167.08506455,76.19970243)(167.21506442,76.17470246)(167.34506165,76.13470998)
\curveto(167.42506421,76.11470252)(167.51006412,76.09470254)(167.60006165,76.07470998)
\lineto(167.84006165,76.01470998)
\lineto(168.14006165,75.89470998)
\curveto(168.2300634,75.86470277)(168.32006331,75.8297028)(168.41006165,75.78970998)
\curveto(168.630063,75.68970294)(168.84506279,75.55470308)(169.05506165,75.38470998)
\curveto(169.26506237,75.22470341)(169.4350622,75.04970358)(169.56506165,74.85970998)
\curveto(169.60506203,74.80970382)(169.64506199,74.74970388)(169.68506165,74.67970998)
\curveto(169.71506192,74.61970401)(169.75006188,74.55970407)(169.79006165,74.49970998)
\curveto(169.84006179,74.41970421)(169.88006175,74.32470431)(169.91006165,74.21470998)
\curveto(169.94006169,74.10470453)(169.97006166,73.99970463)(170.00006165,73.89970998)
\curveto(170.04006159,73.78970484)(170.06506157,73.67970495)(170.07506165,73.56970998)
\curveto(170.08506155,73.45970517)(170.10006153,73.34470529)(170.12006165,73.22470998)
\curveto(170.1300615,73.18470545)(170.1300615,73.13970549)(170.12006165,73.08970998)
\curveto(170.12006151,73.04970558)(170.12506151,73.00970562)(170.13506165,72.96970998)
\curveto(170.14506149,72.9297057)(170.15006148,72.87470576)(170.15006165,72.80470998)
\curveto(170.15006148,72.7347059)(170.14506149,72.68470595)(170.13506165,72.65470998)
\curveto(170.11506152,72.60470603)(170.11006152,72.55970607)(170.12006165,72.51970998)
\curveto(170.1300615,72.47970615)(170.1300615,72.44470619)(170.12006165,72.41470998)
\lineto(170.12006165,72.32470998)
\curveto(170.10006153,72.26470637)(170.08506155,72.19970643)(170.07506165,72.12970998)
\curveto(170.07506156,72.06970656)(170.07006156,72.00470663)(170.06006165,71.93470998)
\curveto(170.01006162,71.76470687)(169.96006167,71.60470703)(169.91006165,71.45470998)
\curveto(169.86006177,71.30470733)(169.79506184,71.15970747)(169.71506165,71.01970998)
\curveto(169.67506196,70.96970766)(169.64506199,70.91470772)(169.62506165,70.85470998)
\curveto(169.59506204,70.80470783)(169.56006207,70.75470788)(169.52006165,70.70470998)
\curveto(169.34006229,70.46470817)(169.12006251,70.26470837)(168.86006165,70.10470998)
\curveto(168.60006303,69.94470869)(168.31506332,69.80470883)(168.00506165,69.68470998)
\curveto(167.86506377,69.62470901)(167.72506391,69.57970905)(167.58506165,69.54970998)
\curveto(167.4350642,69.51970911)(167.28006435,69.48470915)(167.12006165,69.44470998)
\curveto(167.01006462,69.42470921)(166.90006473,69.40970922)(166.79006165,69.39970998)
\curveto(166.68006495,69.38970924)(166.57006506,69.37470926)(166.46006165,69.35470998)
\curveto(166.42006521,69.34470929)(166.38006525,69.33970929)(166.34006165,69.33970998)
\curveto(166.30006533,69.34970928)(166.26006537,69.34970928)(166.22006165,69.33970998)
\curveto(166.17006546,69.3297093)(166.12006551,69.32470931)(166.07006165,69.32470998)
\lineto(165.90506165,69.32470998)
\curveto(165.85506578,69.30470933)(165.80506583,69.29970933)(165.75506165,69.30970998)
\curveto(165.69506594,69.31970931)(165.64006599,69.31970931)(165.59006165,69.30970998)
\curveto(165.55006608,69.29970933)(165.50506613,69.29970933)(165.45506165,69.30970998)
\curveto(165.40506623,69.31970931)(165.35506628,69.31470932)(165.30506165,69.29470998)
\curveto(165.2350664,69.27470936)(165.16006647,69.26970936)(165.08006165,69.27970998)
\curveto(164.99006664,69.28970934)(164.90506673,69.29470934)(164.82506165,69.29470998)
\curveto(164.7350669,69.29470934)(164.635067,69.28970934)(164.52506165,69.27970998)
\curveto(164.40506723,69.26970936)(164.30506733,69.27470936)(164.22506165,69.29470998)
\lineto(163.94006165,69.29470998)
\lineto(163.31006165,69.33970998)
\curveto(163.21006842,69.34970928)(163.11506852,69.35970927)(163.02506165,69.36970998)
\lineto(162.72506165,69.39970998)
\curveto(162.67506896,69.41970921)(162.62506901,69.42470921)(162.57506165,69.41470998)
\curveto(162.51506912,69.41470922)(162.46006917,69.42470921)(162.41006165,69.44470998)
\curveto(162.24006939,69.49470914)(162.07506956,69.5347091)(161.91506165,69.56470998)
\curveto(161.74506989,69.59470904)(161.58507005,69.64470899)(161.43506165,69.71470998)
\curveto(160.97507066,69.90470873)(160.60007103,70.12470851)(160.31006165,70.37470998)
\curveto(160.02007161,70.634708)(159.77507186,70.99470764)(159.57506165,71.45470998)
\curveto(159.52507211,71.58470705)(159.49007214,71.71470692)(159.47006165,71.84470998)
\curveto(159.45007218,71.98470665)(159.42507221,72.12470651)(159.39506165,72.26470998)
\curveto(159.38507225,72.3347063)(159.38007225,72.39970623)(159.38006165,72.45970998)
\curveto(159.38007225,72.51970611)(159.37507226,72.58470605)(159.36506165,72.65470998)
\curveto(159.34507229,73.48470515)(159.49507214,74.15470448)(159.81506165,74.66470998)
\curveto(160.12507151,75.17470346)(160.56507107,75.55470308)(161.13506165,75.80470998)
\curveto(161.25507038,75.85470278)(161.38007025,75.89970273)(161.51006165,75.93970998)
\curveto(161.64006999,75.97970265)(161.77506986,76.02470261)(161.91506165,76.07470998)
\curveto(161.99506964,76.09470254)(162.08006955,76.10970252)(162.17006165,76.11970998)
\lineto(162.41006165,76.17970998)
\curveto(162.52006911,76.20970242)(162.630069,76.22470241)(162.74006165,76.22470998)
\curveto(162.85006878,76.2347024)(162.96006867,76.24970238)(163.07006165,76.26970998)
\curveto(163.12006851,76.28970234)(163.16506847,76.29470234)(163.20506165,76.28470998)
\curveto(163.24506839,76.28470235)(163.28506835,76.28970234)(163.32506165,76.29970998)
\curveto(163.37506826,76.30970232)(163.4300682,76.30970232)(163.49006165,76.29970998)
\curveto(163.54006809,76.29970233)(163.59006804,76.30470233)(163.64006165,76.31470998)
\lineto(163.77506165,76.31470998)
\curveto(163.8350678,76.3347023)(163.90506773,76.3347023)(163.98506165,76.31470998)
\curveto(164.05506758,76.30470233)(164.12006751,76.30970232)(164.18006165,76.32970998)
\curveto(164.21006742,76.33970229)(164.25006738,76.34470229)(164.30006165,76.34470998)
\lineto(164.42006165,76.34470998)
\lineto(164.88506165,76.34470998)
\moveto(167.21006165,74.79970998)
\curveto(166.89006474,74.89970373)(166.52506511,74.95970367)(166.11506165,74.97970998)
\curveto(165.70506593,74.99970363)(165.29506634,75.00970362)(164.88506165,75.00970998)
\curveto(164.45506718,75.00970362)(164.0350676,74.99970363)(163.62506165,74.97970998)
\curveto(163.21506842,74.95970367)(162.8300688,74.91470372)(162.47006165,74.84470998)
\curveto(162.11006952,74.77470386)(161.79006984,74.66470397)(161.51006165,74.51470998)
\curveto(161.22007041,74.37470426)(160.98507065,74.17970445)(160.80506165,73.92970998)
\curveto(160.69507094,73.76970486)(160.61507102,73.58970504)(160.56506165,73.38970998)
\curveto(160.50507113,73.18970544)(160.47507116,72.94470569)(160.47506165,72.65470998)
\curveto(160.49507114,72.634706)(160.50507113,72.59970603)(160.50506165,72.54970998)
\curveto(160.49507114,72.49970613)(160.49507114,72.45970617)(160.50506165,72.42970998)
\curveto(160.52507111,72.34970628)(160.54507109,72.27470636)(160.56506165,72.20470998)
\curveto(160.57507106,72.14470649)(160.59507104,72.07970655)(160.62506165,72.00970998)
\curveto(160.74507089,71.73970689)(160.91507072,71.51970711)(161.13506165,71.34970998)
\curveto(161.34507029,71.18970744)(161.59007004,71.05470758)(161.87006165,70.94470998)
\curveto(161.98006965,70.89470774)(162.10006953,70.85470778)(162.23006165,70.82470998)
\curveto(162.35006928,70.80470783)(162.47506916,70.77970785)(162.60506165,70.74970998)
\curveto(162.65506898,70.7297079)(162.71006892,70.71970791)(162.77006165,70.71970998)
\curveto(162.82006881,70.71970791)(162.87006876,70.71470792)(162.92006165,70.70470998)
\curveto(163.01006862,70.69470794)(163.10506853,70.68470795)(163.20506165,70.67470998)
\curveto(163.29506834,70.66470797)(163.39006824,70.65470798)(163.49006165,70.64470998)
\curveto(163.57006806,70.64470799)(163.65506798,70.63970799)(163.74506165,70.62970998)
\lineto(163.98506165,70.62970998)
\lineto(164.16506165,70.62970998)
\curveto(164.19506744,70.61970801)(164.2300674,70.61470802)(164.27006165,70.61470998)
\lineto(164.40506165,70.61470998)
\lineto(164.85506165,70.61470998)
\curveto(164.9350667,70.61470802)(165.02006661,70.60970802)(165.11006165,70.59970998)
\curveto(165.19006644,70.59970803)(165.26506637,70.60970802)(165.33506165,70.62970998)
\lineto(165.60506165,70.62970998)
\curveto(165.62506601,70.629708)(165.65506598,70.62470801)(165.69506165,70.61470998)
\curveto(165.72506591,70.61470802)(165.75006588,70.61970801)(165.77006165,70.62970998)
\curveto(165.87006576,70.63970799)(165.97006566,70.64470799)(166.07006165,70.64470998)
\curveto(166.16006547,70.65470798)(166.26006537,70.66470797)(166.37006165,70.67470998)
\curveto(166.49006514,70.70470793)(166.61506502,70.71970791)(166.74506165,70.71970998)
\curveto(166.86506477,70.7297079)(166.98006465,70.75470788)(167.09006165,70.79470998)
\curveto(167.39006424,70.87470776)(167.65506398,70.95970767)(167.88506165,71.04970998)
\curveto(168.11506352,71.14970748)(168.3300633,71.29470734)(168.53006165,71.48470998)
\curveto(168.7300629,71.69470694)(168.88006275,71.95970667)(168.98006165,72.27970998)
\curveto(169.00006263,72.31970631)(169.01006262,72.35470628)(169.01006165,72.38470998)
\curveto(169.00006263,72.42470621)(169.00506263,72.46970616)(169.02506165,72.51970998)
\curveto(169.0350626,72.55970607)(169.04506259,72.629706)(169.05506165,72.72970998)
\curveto(169.06506257,72.83970579)(169.06006257,72.92470571)(169.04006165,72.98470998)
\curveto(169.02006261,73.05470558)(169.01006262,73.12470551)(169.01006165,73.19470998)
\curveto(169.00006263,73.26470537)(168.98506265,73.3297053)(168.96506165,73.38970998)
\curveto(168.90506273,73.58970504)(168.82006281,73.76970486)(168.71006165,73.92970998)
\curveto(168.69006294,73.95970467)(168.67006296,73.98470465)(168.65006165,74.00470998)
\lineto(168.59006165,74.06470998)
\curveto(168.57006306,74.10470453)(168.5300631,74.15470448)(168.47006165,74.21470998)
\curveto(168.3300633,74.31470432)(168.20006343,74.39970423)(168.08006165,74.46970998)
\curveto(167.96006367,74.53970409)(167.81506382,74.60970402)(167.64506165,74.67970998)
\curveto(167.57506406,74.70970392)(167.50506413,74.7297039)(167.43506165,74.73970998)
\curveto(167.36506427,74.75970387)(167.29006434,74.77970385)(167.21006165,74.79970998)
}
}
{
\newrgbcolor{curcolor}{0 0 0}
\pscustom[linestyle=none,fillstyle=solid,fillcolor=curcolor]
{
\newpath
\moveto(168.33506165,78.63431936)
\lineto(168.33506165,79.26431936)
\lineto(168.33506165,79.45931936)
\curveto(168.3350633,79.52931683)(168.34506329,79.58931677)(168.36506165,79.63931936)
\curveto(168.40506323,79.70931665)(168.44506319,79.7593166)(168.48506165,79.78931936)
\curveto(168.5350631,79.82931653)(168.60006303,79.84931651)(168.68006165,79.84931936)
\curveto(168.76006287,79.8593165)(168.84506279,79.86431649)(168.93506165,79.86431936)
\lineto(169.65506165,79.86431936)
\curveto(170.1350615,79.86431649)(170.54506109,79.80431655)(170.88506165,79.68431936)
\curveto(171.22506041,79.56431679)(171.50006013,79.36931699)(171.71006165,79.09931936)
\curveto(171.76005987,79.02931733)(171.80505983,78.9593174)(171.84506165,78.88931936)
\curveto(171.89505974,78.82931753)(171.94005969,78.7543176)(171.98006165,78.66431936)
\curveto(171.99005964,78.64431771)(172.00005963,78.61431774)(172.01006165,78.57431936)
\curveto(172.0300596,78.53431782)(172.0350596,78.48931787)(172.02506165,78.43931936)
\curveto(171.99505964,78.34931801)(171.92005971,78.29431806)(171.80006165,78.27431936)
\curveto(171.69005994,78.2543181)(171.59506004,78.26931809)(171.51506165,78.31931936)
\curveto(171.44506019,78.34931801)(171.38006025,78.39431796)(171.32006165,78.45431936)
\curveto(171.27006036,78.52431783)(171.22006041,78.58931777)(171.17006165,78.64931936)
\curveto(171.12006051,78.71931764)(171.04506059,78.77931758)(170.94506165,78.82931936)
\curveto(170.85506078,78.88931747)(170.76506087,78.93931742)(170.67506165,78.97931936)
\curveto(170.64506099,78.99931736)(170.58506105,79.02431733)(170.49506165,79.05431936)
\curveto(170.41506122,79.08431727)(170.34506129,79.08931727)(170.28506165,79.06931936)
\curveto(170.14506149,79.03931732)(170.05506158,78.97931738)(170.01506165,78.88931936)
\curveto(169.98506165,78.80931755)(169.97006166,78.71931764)(169.97006165,78.61931936)
\curveto(169.97006166,78.51931784)(169.94506169,78.43431792)(169.89506165,78.36431936)
\curveto(169.82506181,78.27431808)(169.68506195,78.22931813)(169.47506165,78.22931936)
\lineto(168.92006165,78.22931936)
\lineto(168.69506165,78.22931936)
\curveto(168.61506302,78.23931812)(168.55006308,78.2593181)(168.50006165,78.28931936)
\curveto(168.42006321,78.34931801)(168.37506326,78.41931794)(168.36506165,78.49931936)
\curveto(168.35506328,78.51931784)(168.35006328,78.53931782)(168.35006165,78.55931936)
\curveto(168.35006328,78.58931777)(168.34506329,78.61431774)(168.33506165,78.63431936)
}
}
{
\newrgbcolor{curcolor}{0 0 0}
\pscustom[linestyle=none,fillstyle=solid,fillcolor=curcolor]
{
}
}
{
\newrgbcolor{curcolor}{0 0 0}
\pscustom[linestyle=none,fillstyle=solid,fillcolor=curcolor]
{
\newpath
\moveto(159.36506165,89.26463186)
\curveto(159.35507228,89.95462722)(159.47507216,90.55462662)(159.72506165,91.06463186)
\curveto(159.97507166,91.58462559)(160.31007132,91.9796252)(160.73006165,92.24963186)
\curveto(160.81007082,92.29962488)(160.90007073,92.34462483)(161.00006165,92.38463186)
\curveto(161.09007054,92.42462475)(161.18507045,92.46962471)(161.28506165,92.51963186)
\curveto(161.38507025,92.55962462)(161.48507015,92.58962459)(161.58506165,92.60963186)
\curveto(161.68506995,92.62962455)(161.79006984,92.64962453)(161.90006165,92.66963186)
\curveto(161.95006968,92.68962449)(161.99506964,92.69462448)(162.03506165,92.68463186)
\curveto(162.07506956,92.6746245)(162.12006951,92.6796245)(162.17006165,92.69963186)
\curveto(162.22006941,92.70962447)(162.30506933,92.71462446)(162.42506165,92.71463186)
\curveto(162.5350691,92.71462446)(162.62006901,92.70962447)(162.68006165,92.69963186)
\curveto(162.74006889,92.6796245)(162.80006883,92.66962451)(162.86006165,92.66963186)
\curveto(162.92006871,92.6796245)(162.98006865,92.6746245)(163.04006165,92.65463186)
\curveto(163.18006845,92.61462456)(163.31506832,92.5796246)(163.44506165,92.54963186)
\curveto(163.57506806,92.51962466)(163.70006793,92.4796247)(163.82006165,92.42963186)
\curveto(163.96006767,92.36962481)(164.08506755,92.29962488)(164.19506165,92.21963186)
\curveto(164.30506733,92.14962503)(164.41506722,92.0746251)(164.52506165,91.99463186)
\lineto(164.58506165,91.93463186)
\curveto(164.60506703,91.92462525)(164.62506701,91.90962527)(164.64506165,91.88963186)
\curveto(164.80506683,91.76962541)(164.95006668,91.63462554)(165.08006165,91.48463186)
\curveto(165.21006642,91.33462584)(165.3350663,91.174626)(165.45506165,91.00463186)
\curveto(165.67506596,90.69462648)(165.88006575,90.39962678)(166.07006165,90.11963186)
\curveto(166.21006542,89.88962729)(166.34506529,89.65962752)(166.47506165,89.42963186)
\curveto(166.60506503,89.20962797)(166.74006489,88.98962819)(166.88006165,88.76963186)
\curveto(167.05006458,88.51962866)(167.2300644,88.2796289)(167.42006165,88.04963186)
\curveto(167.61006402,87.82962935)(167.8350638,87.63962954)(168.09506165,87.47963186)
\curveto(168.15506348,87.43962974)(168.21506342,87.40462977)(168.27506165,87.37463186)
\curveto(168.32506331,87.34462983)(168.39006324,87.31462986)(168.47006165,87.28463186)
\curveto(168.54006309,87.26462991)(168.60006303,87.25962992)(168.65006165,87.26963186)
\curveto(168.72006291,87.28962989)(168.77506286,87.32462985)(168.81506165,87.37463186)
\curveto(168.84506279,87.42462975)(168.86506277,87.48462969)(168.87506165,87.55463186)
\lineto(168.87506165,87.79463186)
\lineto(168.87506165,88.54463186)
\lineto(168.87506165,91.34963186)
\lineto(168.87506165,92.00963186)
\curveto(168.87506276,92.09962508)(168.88006275,92.18462499)(168.89006165,92.26463186)
\curveto(168.89006274,92.34462483)(168.91006272,92.40962477)(168.95006165,92.45963186)
\curveto(168.99006264,92.50962467)(169.06506257,92.54962463)(169.17506165,92.57963186)
\curveto(169.27506236,92.61962456)(169.37506226,92.62962455)(169.47506165,92.60963186)
\lineto(169.61006165,92.60963186)
\curveto(169.68006195,92.58962459)(169.74006189,92.56962461)(169.79006165,92.54963186)
\curveto(169.84006179,92.52962465)(169.88006175,92.49462468)(169.91006165,92.44463186)
\curveto(169.95006168,92.39462478)(169.97006166,92.32462485)(169.97006165,92.23463186)
\lineto(169.97006165,91.96463186)
\lineto(169.97006165,91.06463186)
\lineto(169.97006165,87.55463186)
\lineto(169.97006165,86.48963186)
\curveto(169.97006166,86.40963077)(169.97506166,86.31963086)(169.98506165,86.21963186)
\curveto(169.98506165,86.11963106)(169.97506166,86.03463114)(169.95506165,85.96463186)
\curveto(169.88506175,85.75463142)(169.70506193,85.68963149)(169.41506165,85.76963186)
\curveto(169.37506226,85.7796314)(169.34006229,85.7796314)(169.31006165,85.76963186)
\curveto(169.27006236,85.76963141)(169.22506241,85.7796314)(169.17506165,85.79963186)
\curveto(169.09506254,85.81963136)(169.01006262,85.83963134)(168.92006165,85.85963186)
\curveto(168.8300628,85.8796313)(168.74506289,85.90463127)(168.66506165,85.93463186)
\curveto(168.17506346,86.09463108)(167.76006387,86.29463088)(167.42006165,86.53463186)
\curveto(167.17006446,86.71463046)(166.94506469,86.91963026)(166.74506165,87.14963186)
\curveto(166.5350651,87.3796298)(166.34006529,87.61962956)(166.16006165,87.86963186)
\curveto(165.98006565,88.12962905)(165.81006582,88.39462878)(165.65006165,88.66463186)
\curveto(165.48006615,88.94462823)(165.30506633,89.21462796)(165.12506165,89.47463186)
\curveto(165.04506659,89.58462759)(164.97006666,89.68962749)(164.90006165,89.78963186)
\curveto(164.8300668,89.89962728)(164.75506688,90.00962717)(164.67506165,90.11963186)
\curveto(164.64506699,90.15962702)(164.61506702,90.19462698)(164.58506165,90.22463186)
\curveto(164.54506709,90.26462691)(164.51506712,90.30462687)(164.49506165,90.34463186)
\curveto(164.38506725,90.48462669)(164.26006737,90.60962657)(164.12006165,90.71963186)
\curveto(164.09006754,90.73962644)(164.06506757,90.76462641)(164.04506165,90.79463186)
\curveto(164.01506762,90.82462635)(163.98506765,90.84962633)(163.95506165,90.86963186)
\curveto(163.85506778,90.94962623)(163.75506788,91.01462616)(163.65506165,91.06463186)
\curveto(163.55506808,91.12462605)(163.44506819,91.179626)(163.32506165,91.22963186)
\curveto(163.25506838,91.25962592)(163.18006845,91.2796259)(163.10006165,91.28963186)
\lineto(162.86006165,91.34963186)
\lineto(162.77006165,91.34963186)
\curveto(162.74006889,91.35962582)(162.71006892,91.36462581)(162.68006165,91.36463186)
\curveto(162.61006902,91.38462579)(162.51506912,91.38962579)(162.39506165,91.37963186)
\curveto(162.26506937,91.3796258)(162.16506947,91.36962581)(162.09506165,91.34963186)
\curveto(162.01506962,91.32962585)(161.94006969,91.30962587)(161.87006165,91.28963186)
\curveto(161.79006984,91.2796259)(161.71006992,91.25962592)(161.63006165,91.22963186)
\curveto(161.39007024,91.11962606)(161.19007044,90.96962621)(161.03006165,90.77963186)
\curveto(160.86007077,90.59962658)(160.72007091,90.3796268)(160.61006165,90.11963186)
\curveto(160.59007104,90.04962713)(160.57507106,89.9796272)(160.56506165,89.90963186)
\curveto(160.54507109,89.83962734)(160.52507111,89.76462741)(160.50506165,89.68463186)
\curveto(160.48507115,89.60462757)(160.47507116,89.49462768)(160.47506165,89.35463186)
\curveto(160.47507116,89.22462795)(160.48507115,89.11962806)(160.50506165,89.03963186)
\curveto(160.51507112,88.9796282)(160.52007111,88.92462825)(160.52006165,88.87463186)
\curveto(160.52007111,88.82462835)(160.5300711,88.7746284)(160.55006165,88.72463186)
\curveto(160.59007104,88.62462855)(160.630071,88.52962865)(160.67006165,88.43963186)
\curveto(160.71007092,88.35962882)(160.75507088,88.2796289)(160.80506165,88.19963186)
\curveto(160.82507081,88.16962901)(160.85007078,88.13962904)(160.88006165,88.10963186)
\curveto(160.91007072,88.08962909)(160.9350707,88.06462911)(160.95506165,88.03463186)
\lineto(161.03006165,87.95963186)
\curveto(161.05007058,87.92962925)(161.07007056,87.90462927)(161.09006165,87.88463186)
\lineto(161.30006165,87.73463186)
\curveto(161.36007027,87.69462948)(161.42507021,87.64962953)(161.49506165,87.59963186)
\curveto(161.58507005,87.53962964)(161.69006994,87.48962969)(161.81006165,87.44963186)
\curveto(161.92006971,87.41962976)(162.0300696,87.38462979)(162.14006165,87.34463186)
\curveto(162.25006938,87.30462987)(162.39506924,87.2796299)(162.57506165,87.26963186)
\curveto(162.74506889,87.25962992)(162.87006876,87.22962995)(162.95006165,87.17963186)
\curveto(163.0300686,87.12963005)(163.07506856,87.05463012)(163.08506165,86.95463186)
\curveto(163.09506854,86.85463032)(163.10006853,86.74463043)(163.10006165,86.62463186)
\curveto(163.10006853,86.58463059)(163.10506853,86.54463063)(163.11506165,86.50463186)
\curveto(163.11506852,86.46463071)(163.11006852,86.42963075)(163.10006165,86.39963186)
\curveto(163.08006855,86.34963083)(163.07006856,86.29963088)(163.07006165,86.24963186)
\curveto(163.07006856,86.20963097)(163.06006857,86.16963101)(163.04006165,86.12963186)
\curveto(162.98006865,86.03963114)(162.84506879,85.99463118)(162.63506165,85.99463186)
\lineto(162.51506165,85.99463186)
\curveto(162.45506918,86.00463117)(162.39506924,86.00963117)(162.33506165,86.00963186)
\curveto(162.26506937,86.01963116)(162.20006943,86.02963115)(162.14006165,86.03963186)
\curveto(162.0300696,86.05963112)(161.9300697,86.0796311)(161.84006165,86.09963186)
\curveto(161.74006989,86.11963106)(161.64506999,86.14963103)(161.55506165,86.18963186)
\curveto(161.48507015,86.20963097)(161.42507021,86.22963095)(161.37506165,86.24963186)
\lineto(161.19506165,86.30963186)
\curveto(160.9350707,86.42963075)(160.69007094,86.58463059)(160.46006165,86.77463186)
\curveto(160.2300714,86.9746302)(160.04507159,87.18962999)(159.90506165,87.41963186)
\curveto(159.82507181,87.52962965)(159.76007187,87.64462953)(159.71006165,87.76463186)
\lineto(159.56006165,88.15463186)
\curveto(159.51007212,88.26462891)(159.48007215,88.3796288)(159.47006165,88.49963186)
\curveto(159.45007218,88.61962856)(159.42507221,88.74462843)(159.39506165,88.87463186)
\curveto(159.39507224,88.94462823)(159.39507224,89.00962817)(159.39506165,89.06963186)
\curveto(159.38507225,89.12962805)(159.37507226,89.19462798)(159.36506165,89.26463186)
}
}
{
\newrgbcolor{curcolor}{0 0 0}
\pscustom[linestyle=none,fillstyle=solid,fillcolor=curcolor]
{
\newpath
\moveto(164.88506165,101.36424123)
\lineto(165.14006165,101.36424123)
\curveto(165.22006641,101.37423353)(165.29506634,101.36923353)(165.36506165,101.34924123)
\lineto(165.60506165,101.34924123)
\lineto(165.77006165,101.34924123)
\curveto(165.87006576,101.32923357)(165.97506566,101.31923358)(166.08506165,101.31924123)
\curveto(166.18506545,101.31923358)(166.28506535,101.30923359)(166.38506165,101.28924123)
\lineto(166.53506165,101.28924123)
\curveto(166.67506496,101.25923364)(166.81506482,101.23923366)(166.95506165,101.22924123)
\curveto(167.08506455,101.21923368)(167.21506442,101.19423371)(167.34506165,101.15424123)
\curveto(167.42506421,101.13423377)(167.51006412,101.11423379)(167.60006165,101.09424123)
\lineto(167.84006165,101.03424123)
\lineto(168.14006165,100.91424123)
\curveto(168.2300634,100.88423402)(168.32006331,100.84923405)(168.41006165,100.80924123)
\curveto(168.630063,100.70923419)(168.84506279,100.57423433)(169.05506165,100.40424123)
\curveto(169.26506237,100.24423466)(169.4350622,100.06923483)(169.56506165,99.87924123)
\curveto(169.60506203,99.82923507)(169.64506199,99.76923513)(169.68506165,99.69924123)
\curveto(169.71506192,99.63923526)(169.75006188,99.57923532)(169.79006165,99.51924123)
\curveto(169.84006179,99.43923546)(169.88006175,99.34423556)(169.91006165,99.23424123)
\curveto(169.94006169,99.12423578)(169.97006166,99.01923588)(170.00006165,98.91924123)
\curveto(170.04006159,98.80923609)(170.06506157,98.6992362)(170.07506165,98.58924123)
\curveto(170.08506155,98.47923642)(170.10006153,98.36423654)(170.12006165,98.24424123)
\curveto(170.1300615,98.2042367)(170.1300615,98.15923674)(170.12006165,98.10924123)
\curveto(170.12006151,98.06923683)(170.12506151,98.02923687)(170.13506165,97.98924123)
\curveto(170.14506149,97.94923695)(170.15006148,97.89423701)(170.15006165,97.82424123)
\curveto(170.15006148,97.75423715)(170.14506149,97.7042372)(170.13506165,97.67424123)
\curveto(170.11506152,97.62423728)(170.11006152,97.57923732)(170.12006165,97.53924123)
\curveto(170.1300615,97.4992374)(170.1300615,97.46423744)(170.12006165,97.43424123)
\lineto(170.12006165,97.34424123)
\curveto(170.10006153,97.28423762)(170.08506155,97.21923768)(170.07506165,97.14924123)
\curveto(170.07506156,97.08923781)(170.07006156,97.02423788)(170.06006165,96.95424123)
\curveto(170.01006162,96.78423812)(169.96006167,96.62423828)(169.91006165,96.47424123)
\curveto(169.86006177,96.32423858)(169.79506184,96.17923872)(169.71506165,96.03924123)
\curveto(169.67506196,95.98923891)(169.64506199,95.93423897)(169.62506165,95.87424123)
\curveto(169.59506204,95.82423908)(169.56006207,95.77423913)(169.52006165,95.72424123)
\curveto(169.34006229,95.48423942)(169.12006251,95.28423962)(168.86006165,95.12424123)
\curveto(168.60006303,94.96423994)(168.31506332,94.82424008)(168.00506165,94.70424123)
\curveto(167.86506377,94.64424026)(167.72506391,94.5992403)(167.58506165,94.56924123)
\curveto(167.4350642,94.53924036)(167.28006435,94.5042404)(167.12006165,94.46424123)
\curveto(167.01006462,94.44424046)(166.90006473,94.42924047)(166.79006165,94.41924123)
\curveto(166.68006495,94.40924049)(166.57006506,94.39424051)(166.46006165,94.37424123)
\curveto(166.42006521,94.36424054)(166.38006525,94.35924054)(166.34006165,94.35924123)
\curveto(166.30006533,94.36924053)(166.26006537,94.36924053)(166.22006165,94.35924123)
\curveto(166.17006546,94.34924055)(166.12006551,94.34424056)(166.07006165,94.34424123)
\lineto(165.90506165,94.34424123)
\curveto(165.85506578,94.32424058)(165.80506583,94.31924058)(165.75506165,94.32924123)
\curveto(165.69506594,94.33924056)(165.64006599,94.33924056)(165.59006165,94.32924123)
\curveto(165.55006608,94.31924058)(165.50506613,94.31924058)(165.45506165,94.32924123)
\curveto(165.40506623,94.33924056)(165.35506628,94.33424057)(165.30506165,94.31424123)
\curveto(165.2350664,94.29424061)(165.16006647,94.28924061)(165.08006165,94.29924123)
\curveto(164.99006664,94.30924059)(164.90506673,94.31424059)(164.82506165,94.31424123)
\curveto(164.7350669,94.31424059)(164.635067,94.30924059)(164.52506165,94.29924123)
\curveto(164.40506723,94.28924061)(164.30506733,94.29424061)(164.22506165,94.31424123)
\lineto(163.94006165,94.31424123)
\lineto(163.31006165,94.35924123)
\curveto(163.21006842,94.36924053)(163.11506852,94.37924052)(163.02506165,94.38924123)
\lineto(162.72506165,94.41924123)
\curveto(162.67506896,94.43924046)(162.62506901,94.44424046)(162.57506165,94.43424123)
\curveto(162.51506912,94.43424047)(162.46006917,94.44424046)(162.41006165,94.46424123)
\curveto(162.24006939,94.51424039)(162.07506956,94.55424035)(161.91506165,94.58424123)
\curveto(161.74506989,94.61424029)(161.58507005,94.66424024)(161.43506165,94.73424123)
\curveto(160.97507066,94.92423998)(160.60007103,95.14423976)(160.31006165,95.39424123)
\curveto(160.02007161,95.65423925)(159.77507186,96.01423889)(159.57506165,96.47424123)
\curveto(159.52507211,96.6042383)(159.49007214,96.73423817)(159.47006165,96.86424123)
\curveto(159.45007218,97.0042379)(159.42507221,97.14423776)(159.39506165,97.28424123)
\curveto(159.38507225,97.35423755)(159.38007225,97.41923748)(159.38006165,97.47924123)
\curveto(159.38007225,97.53923736)(159.37507226,97.6042373)(159.36506165,97.67424123)
\curveto(159.34507229,98.5042364)(159.49507214,99.17423573)(159.81506165,99.68424123)
\curveto(160.12507151,100.19423471)(160.56507107,100.57423433)(161.13506165,100.82424123)
\curveto(161.25507038,100.87423403)(161.38007025,100.91923398)(161.51006165,100.95924123)
\curveto(161.64006999,100.9992339)(161.77506986,101.04423386)(161.91506165,101.09424123)
\curveto(161.99506964,101.11423379)(162.08006955,101.12923377)(162.17006165,101.13924123)
\lineto(162.41006165,101.19924123)
\curveto(162.52006911,101.22923367)(162.630069,101.24423366)(162.74006165,101.24424123)
\curveto(162.85006878,101.25423365)(162.96006867,101.26923363)(163.07006165,101.28924123)
\curveto(163.12006851,101.30923359)(163.16506847,101.31423359)(163.20506165,101.30424123)
\curveto(163.24506839,101.3042336)(163.28506835,101.30923359)(163.32506165,101.31924123)
\curveto(163.37506826,101.32923357)(163.4300682,101.32923357)(163.49006165,101.31924123)
\curveto(163.54006809,101.31923358)(163.59006804,101.32423358)(163.64006165,101.33424123)
\lineto(163.77506165,101.33424123)
\curveto(163.8350678,101.35423355)(163.90506773,101.35423355)(163.98506165,101.33424123)
\curveto(164.05506758,101.32423358)(164.12006751,101.32923357)(164.18006165,101.34924123)
\curveto(164.21006742,101.35923354)(164.25006738,101.36423354)(164.30006165,101.36424123)
\lineto(164.42006165,101.36424123)
\lineto(164.88506165,101.36424123)
\moveto(167.21006165,99.81924123)
\curveto(166.89006474,99.91923498)(166.52506511,99.97923492)(166.11506165,99.99924123)
\curveto(165.70506593,100.01923488)(165.29506634,100.02923487)(164.88506165,100.02924123)
\curveto(164.45506718,100.02923487)(164.0350676,100.01923488)(163.62506165,99.99924123)
\curveto(163.21506842,99.97923492)(162.8300688,99.93423497)(162.47006165,99.86424123)
\curveto(162.11006952,99.79423511)(161.79006984,99.68423522)(161.51006165,99.53424123)
\curveto(161.22007041,99.39423551)(160.98507065,99.1992357)(160.80506165,98.94924123)
\curveto(160.69507094,98.78923611)(160.61507102,98.60923629)(160.56506165,98.40924123)
\curveto(160.50507113,98.20923669)(160.47507116,97.96423694)(160.47506165,97.67424123)
\curveto(160.49507114,97.65423725)(160.50507113,97.61923728)(160.50506165,97.56924123)
\curveto(160.49507114,97.51923738)(160.49507114,97.47923742)(160.50506165,97.44924123)
\curveto(160.52507111,97.36923753)(160.54507109,97.29423761)(160.56506165,97.22424123)
\curveto(160.57507106,97.16423774)(160.59507104,97.0992378)(160.62506165,97.02924123)
\curveto(160.74507089,96.75923814)(160.91507072,96.53923836)(161.13506165,96.36924123)
\curveto(161.34507029,96.20923869)(161.59007004,96.07423883)(161.87006165,95.96424123)
\curveto(161.98006965,95.91423899)(162.10006953,95.87423903)(162.23006165,95.84424123)
\curveto(162.35006928,95.82423908)(162.47506916,95.7992391)(162.60506165,95.76924123)
\curveto(162.65506898,95.74923915)(162.71006892,95.73923916)(162.77006165,95.73924123)
\curveto(162.82006881,95.73923916)(162.87006876,95.73423917)(162.92006165,95.72424123)
\curveto(163.01006862,95.71423919)(163.10506853,95.7042392)(163.20506165,95.69424123)
\curveto(163.29506834,95.68423922)(163.39006824,95.67423923)(163.49006165,95.66424123)
\curveto(163.57006806,95.66423924)(163.65506798,95.65923924)(163.74506165,95.64924123)
\lineto(163.98506165,95.64924123)
\lineto(164.16506165,95.64924123)
\curveto(164.19506744,95.63923926)(164.2300674,95.63423927)(164.27006165,95.63424123)
\lineto(164.40506165,95.63424123)
\lineto(164.85506165,95.63424123)
\curveto(164.9350667,95.63423927)(165.02006661,95.62923927)(165.11006165,95.61924123)
\curveto(165.19006644,95.61923928)(165.26506637,95.62923927)(165.33506165,95.64924123)
\lineto(165.60506165,95.64924123)
\curveto(165.62506601,95.64923925)(165.65506598,95.64423926)(165.69506165,95.63424123)
\curveto(165.72506591,95.63423927)(165.75006588,95.63923926)(165.77006165,95.64924123)
\curveto(165.87006576,95.65923924)(165.97006566,95.66423924)(166.07006165,95.66424123)
\curveto(166.16006547,95.67423923)(166.26006537,95.68423922)(166.37006165,95.69424123)
\curveto(166.49006514,95.72423918)(166.61506502,95.73923916)(166.74506165,95.73924123)
\curveto(166.86506477,95.74923915)(166.98006465,95.77423913)(167.09006165,95.81424123)
\curveto(167.39006424,95.89423901)(167.65506398,95.97923892)(167.88506165,96.06924123)
\curveto(168.11506352,96.16923873)(168.3300633,96.31423859)(168.53006165,96.50424123)
\curveto(168.7300629,96.71423819)(168.88006275,96.97923792)(168.98006165,97.29924123)
\curveto(169.00006263,97.33923756)(169.01006262,97.37423753)(169.01006165,97.40424123)
\curveto(169.00006263,97.44423746)(169.00506263,97.48923741)(169.02506165,97.53924123)
\curveto(169.0350626,97.57923732)(169.04506259,97.64923725)(169.05506165,97.74924123)
\curveto(169.06506257,97.85923704)(169.06006257,97.94423696)(169.04006165,98.00424123)
\curveto(169.02006261,98.07423683)(169.01006262,98.14423676)(169.01006165,98.21424123)
\curveto(169.00006263,98.28423662)(168.98506265,98.34923655)(168.96506165,98.40924123)
\curveto(168.90506273,98.60923629)(168.82006281,98.78923611)(168.71006165,98.94924123)
\curveto(168.69006294,98.97923592)(168.67006296,99.0042359)(168.65006165,99.02424123)
\lineto(168.59006165,99.08424123)
\curveto(168.57006306,99.12423578)(168.5300631,99.17423573)(168.47006165,99.23424123)
\curveto(168.3300633,99.33423557)(168.20006343,99.41923548)(168.08006165,99.48924123)
\curveto(167.96006367,99.55923534)(167.81506382,99.62923527)(167.64506165,99.69924123)
\curveto(167.57506406,99.72923517)(167.50506413,99.74923515)(167.43506165,99.75924123)
\curveto(167.36506427,99.77923512)(167.29006434,99.7992351)(167.21006165,99.81924123)
}
}
{
\newrgbcolor{curcolor}{0 0 0}
\pscustom[linestyle=none,fillstyle=solid,fillcolor=curcolor]
{
\newpath
\moveto(159.36506165,106.77385061)
\curveto(159.36507227,106.87384575)(159.37507226,106.96884566)(159.39506165,107.05885061)
\curveto(159.40507223,107.14884548)(159.4350722,107.21384541)(159.48506165,107.25385061)
\curveto(159.56507207,107.31384531)(159.67007196,107.34384528)(159.80006165,107.34385061)
\lineto(160.19006165,107.34385061)
\lineto(161.69006165,107.34385061)
\lineto(168.08006165,107.34385061)
\lineto(169.25006165,107.34385061)
\lineto(169.56506165,107.34385061)
\curveto(169.66506197,107.35384527)(169.74506189,107.33884529)(169.80506165,107.29885061)
\curveto(169.88506175,107.24884538)(169.9350617,107.17384545)(169.95506165,107.07385061)
\curveto(169.96506167,106.98384564)(169.97006166,106.87384575)(169.97006165,106.74385061)
\lineto(169.97006165,106.51885061)
\curveto(169.95006168,106.43884619)(169.9350617,106.36884626)(169.92506165,106.30885061)
\curveto(169.90506173,106.24884638)(169.86506177,106.19884643)(169.80506165,106.15885061)
\curveto(169.74506189,106.11884651)(169.67006196,106.09884653)(169.58006165,106.09885061)
\lineto(169.28006165,106.09885061)
\lineto(168.18506165,106.09885061)
\lineto(162.84506165,106.09885061)
\curveto(162.75506888,106.07884655)(162.68006895,106.06384656)(162.62006165,106.05385061)
\curveto(162.55006908,106.05384657)(162.49006914,106.0238466)(162.44006165,105.96385061)
\curveto(162.39006924,105.89384673)(162.36506927,105.80384682)(162.36506165,105.69385061)
\curveto(162.35506928,105.59384703)(162.35006928,105.48384714)(162.35006165,105.36385061)
\lineto(162.35006165,104.22385061)
\lineto(162.35006165,103.72885061)
\curveto(162.34006929,103.56884906)(162.28006935,103.45884917)(162.17006165,103.39885061)
\curveto(162.14006949,103.37884925)(162.11006952,103.36884926)(162.08006165,103.36885061)
\curveto(162.04006959,103.36884926)(161.99506964,103.36384926)(161.94506165,103.35385061)
\curveto(161.82506981,103.33384929)(161.71506992,103.33884929)(161.61506165,103.36885061)
\curveto(161.51507012,103.40884922)(161.44507019,103.46384916)(161.40506165,103.53385061)
\curveto(161.35507028,103.61384901)(161.3300703,103.73384889)(161.33006165,103.89385061)
\curveto(161.3300703,104.05384857)(161.31507032,104.18884844)(161.28506165,104.29885061)
\curveto(161.27507036,104.34884828)(161.27007036,104.40384822)(161.27006165,104.46385061)
\curveto(161.26007037,104.5238481)(161.24507039,104.58384804)(161.22506165,104.64385061)
\curveto(161.17507046,104.79384783)(161.12507051,104.93884769)(161.07506165,105.07885061)
\curveto(161.01507062,105.21884741)(160.94507069,105.35384727)(160.86506165,105.48385061)
\curveto(160.77507086,105.623847)(160.67007096,105.74384688)(160.55006165,105.84385061)
\curveto(160.4300712,105.94384668)(160.30007133,106.03884659)(160.16006165,106.12885061)
\curveto(160.06007157,106.18884644)(159.95007168,106.23384639)(159.83006165,106.26385061)
\curveto(159.71007192,106.30384632)(159.60507203,106.35384627)(159.51506165,106.41385061)
\curveto(159.45507218,106.46384616)(159.41507222,106.53384609)(159.39506165,106.62385061)
\curveto(159.38507225,106.64384598)(159.38007225,106.66884596)(159.38006165,106.69885061)
\curveto(159.38007225,106.7288459)(159.37507226,106.75384587)(159.36506165,106.77385061)
}
}
{
\newrgbcolor{curcolor}{0 0 0}
\pscustom[linestyle=none,fillstyle=solid,fillcolor=curcolor]
{
\newpath
\moveto(159.36506165,115.12345998)
\curveto(159.36507227,115.22345513)(159.37507226,115.31845503)(159.39506165,115.40845998)
\curveto(159.40507223,115.49845485)(159.4350722,115.56345479)(159.48506165,115.60345998)
\curveto(159.56507207,115.66345469)(159.67007196,115.69345466)(159.80006165,115.69345998)
\lineto(160.19006165,115.69345998)
\lineto(161.69006165,115.69345998)
\lineto(168.08006165,115.69345998)
\lineto(169.25006165,115.69345998)
\lineto(169.56506165,115.69345998)
\curveto(169.66506197,115.70345465)(169.74506189,115.68845466)(169.80506165,115.64845998)
\curveto(169.88506175,115.59845475)(169.9350617,115.52345483)(169.95506165,115.42345998)
\curveto(169.96506167,115.33345502)(169.97006166,115.22345513)(169.97006165,115.09345998)
\lineto(169.97006165,114.86845998)
\curveto(169.95006168,114.78845556)(169.9350617,114.71845563)(169.92506165,114.65845998)
\curveto(169.90506173,114.59845575)(169.86506177,114.5484558)(169.80506165,114.50845998)
\curveto(169.74506189,114.46845588)(169.67006196,114.4484559)(169.58006165,114.44845998)
\lineto(169.28006165,114.44845998)
\lineto(168.18506165,114.44845998)
\lineto(162.84506165,114.44845998)
\curveto(162.75506888,114.42845592)(162.68006895,114.41345594)(162.62006165,114.40345998)
\curveto(162.55006908,114.40345595)(162.49006914,114.37345598)(162.44006165,114.31345998)
\curveto(162.39006924,114.24345611)(162.36506927,114.1534562)(162.36506165,114.04345998)
\curveto(162.35506928,113.94345641)(162.35006928,113.83345652)(162.35006165,113.71345998)
\lineto(162.35006165,112.57345998)
\lineto(162.35006165,112.07845998)
\curveto(162.34006929,111.91845843)(162.28006935,111.80845854)(162.17006165,111.74845998)
\curveto(162.14006949,111.72845862)(162.11006952,111.71845863)(162.08006165,111.71845998)
\curveto(162.04006959,111.71845863)(161.99506964,111.71345864)(161.94506165,111.70345998)
\curveto(161.82506981,111.68345867)(161.71506992,111.68845866)(161.61506165,111.71845998)
\curveto(161.51507012,111.75845859)(161.44507019,111.81345854)(161.40506165,111.88345998)
\curveto(161.35507028,111.96345839)(161.3300703,112.08345827)(161.33006165,112.24345998)
\curveto(161.3300703,112.40345795)(161.31507032,112.53845781)(161.28506165,112.64845998)
\curveto(161.27507036,112.69845765)(161.27007036,112.7534576)(161.27006165,112.81345998)
\curveto(161.26007037,112.87345748)(161.24507039,112.93345742)(161.22506165,112.99345998)
\curveto(161.17507046,113.14345721)(161.12507051,113.28845706)(161.07506165,113.42845998)
\curveto(161.01507062,113.56845678)(160.94507069,113.70345665)(160.86506165,113.83345998)
\curveto(160.77507086,113.97345638)(160.67007096,114.09345626)(160.55006165,114.19345998)
\curveto(160.4300712,114.29345606)(160.30007133,114.38845596)(160.16006165,114.47845998)
\curveto(160.06007157,114.53845581)(159.95007168,114.58345577)(159.83006165,114.61345998)
\curveto(159.71007192,114.6534557)(159.60507203,114.70345565)(159.51506165,114.76345998)
\curveto(159.45507218,114.81345554)(159.41507222,114.88345547)(159.39506165,114.97345998)
\curveto(159.38507225,114.99345536)(159.38007225,115.01845533)(159.38006165,115.04845998)
\curveto(159.38007225,115.07845527)(159.37507226,115.10345525)(159.36506165,115.12345998)
}
}
{
\newrgbcolor{curcolor}{0 0 0}
\pscustom[linestyle=none,fillstyle=solid,fillcolor=curcolor]
{
\newpath
\moveto(201.112724,31.67142873)
\lineto(201.112724,32.58642873)
\curveto(201.11273469,32.68642608)(201.11273469,32.78142599)(201.112724,32.87142873)
\curveto(201.11273469,32.96142581)(201.13273467,33.03642573)(201.172724,33.09642873)
\curveto(201.23273457,33.18642558)(201.31273449,33.24642552)(201.412724,33.27642873)
\curveto(201.51273429,33.31642545)(201.61773419,33.36142541)(201.727724,33.41142873)
\curveto(201.91773389,33.49142528)(202.1077337,33.56142521)(202.297724,33.62142873)
\curveto(202.48773332,33.69142508)(202.67773313,33.766425)(202.867724,33.84642873)
\curveto(203.04773276,33.91642485)(203.23273257,33.98142479)(203.422724,34.04142873)
\curveto(203.6027322,34.10142467)(203.78273202,34.1714246)(203.962724,34.25142873)
\curveto(204.1027317,34.31142446)(204.24773156,34.3664244)(204.397724,34.41642873)
\curveto(204.54773126,34.4664243)(204.69273111,34.52142425)(204.832724,34.58142873)
\curveto(205.28273052,34.76142401)(205.73773007,34.93142384)(206.197724,35.09142873)
\curveto(206.64772916,35.25142352)(207.09772871,35.42142335)(207.547724,35.60142873)
\curveto(207.59772821,35.62142315)(207.64772816,35.63642313)(207.697724,35.64642873)
\lineto(207.847724,35.70642873)
\curveto(208.06772774,35.79642297)(208.29272751,35.88142289)(208.522724,35.96142873)
\curveto(208.74272706,36.04142273)(208.96272684,36.12642264)(209.182724,36.21642873)
\curveto(209.27272653,36.25642251)(209.38272642,36.29642247)(209.512724,36.33642873)
\curveto(209.63272617,36.37642239)(209.7027261,36.44142233)(209.722724,36.53142873)
\curveto(209.73272607,36.5714222)(209.73272607,36.60142217)(209.722724,36.62142873)
\lineto(209.662724,36.68142873)
\curveto(209.61272619,36.73142204)(209.55772625,36.766422)(209.497724,36.78642873)
\curveto(209.43772637,36.81642195)(209.37272643,36.84642192)(209.302724,36.87642873)
\lineto(208.672724,37.11642873)
\curveto(208.45272735,37.19642157)(208.23772757,37.27642149)(208.027724,37.35642873)
\lineto(207.877724,37.41642873)
\lineto(207.697724,37.47642873)
\curveto(207.5077283,37.55642121)(207.31772849,37.62642114)(207.127724,37.68642873)
\curveto(206.92772888,37.75642101)(206.72772908,37.83142094)(206.527724,37.91142873)
\curveto(205.94772986,38.15142062)(205.36273044,38.3714204)(204.772724,38.57142873)
\curveto(204.18273162,38.78141999)(203.59773221,39.00641976)(203.017724,39.24642873)
\curveto(202.81773299,39.32641944)(202.61273319,39.40141937)(202.402724,39.47142873)
\curveto(202.19273361,39.55141922)(201.98773382,39.63141914)(201.787724,39.71142873)
\curveto(201.7077341,39.75141902)(201.6077342,39.78641898)(201.487724,39.81642873)
\curveto(201.36773444,39.85641891)(201.28273452,39.91141886)(201.232724,39.98142873)
\curveto(201.19273461,40.04141873)(201.16273464,40.11641865)(201.142724,40.20642873)
\curveto(201.12273468,40.30641846)(201.11273469,40.41641835)(201.112724,40.53642873)
\curveto(201.1027347,40.65641811)(201.1027347,40.77641799)(201.112724,40.89642873)
\curveto(201.11273469,41.01641775)(201.11273469,41.12641764)(201.112724,41.22642873)
\curveto(201.11273469,41.31641745)(201.11273469,41.40641736)(201.112724,41.49642873)
\curveto(201.11273469,41.59641717)(201.13273467,41.6714171)(201.172724,41.72142873)
\curveto(201.22273458,41.81141696)(201.31273449,41.86141691)(201.442724,41.87142873)
\curveto(201.57273423,41.88141689)(201.71273409,41.88641688)(201.862724,41.88642873)
\lineto(203.512724,41.88642873)
\lineto(209.782724,41.88642873)
\lineto(211.042724,41.88642873)
\curveto(211.15272465,41.88641688)(211.26272454,41.88641688)(211.372724,41.88642873)
\curveto(211.48272432,41.89641687)(211.56772424,41.87641689)(211.627724,41.82642873)
\curveto(211.68772412,41.79641697)(211.72772408,41.75141702)(211.747724,41.69142873)
\curveto(211.75772405,41.63141714)(211.77272403,41.56141721)(211.792724,41.48142873)
\lineto(211.792724,41.24142873)
\lineto(211.792724,40.88142873)
\curveto(211.78272402,40.771418)(211.73772407,40.69141808)(211.657724,40.64142873)
\curveto(211.62772418,40.62141815)(211.59772421,40.60641816)(211.567724,40.59642873)
\curveto(211.52772428,40.59641817)(211.48272432,40.58641818)(211.432724,40.56642873)
\lineto(211.267724,40.56642873)
\curveto(211.2077246,40.55641821)(211.13772467,40.55141822)(211.057724,40.55142873)
\curveto(210.97772483,40.56141821)(210.9027249,40.5664182)(210.832724,40.56642873)
\lineto(209.992724,40.56642873)
\lineto(205.567724,40.56642873)
\curveto(205.31773049,40.5664182)(205.06773074,40.5664182)(204.817724,40.56642873)
\curveto(204.55773125,40.5664182)(204.3077315,40.56141821)(204.067724,40.55142873)
\curveto(203.96773184,40.55141822)(203.85773195,40.54641822)(203.737724,40.53642873)
\curveto(203.61773219,40.52641824)(203.55773225,40.4714183)(203.557724,40.37142873)
\lineto(203.572724,40.37142873)
\curveto(203.59273221,40.30141847)(203.65773215,40.24141853)(203.767724,40.19142873)
\curveto(203.87773193,40.15141862)(203.97273183,40.11641865)(204.052724,40.08642873)
\curveto(204.22273158,40.01641875)(204.39773141,39.95141882)(204.577724,39.89142873)
\curveto(204.74773106,39.83141894)(204.91773089,39.76141901)(205.087724,39.68142873)
\curveto(205.13773067,39.66141911)(205.18273062,39.64641912)(205.222724,39.63642873)
\curveto(205.26273054,39.62641914)(205.3077305,39.61141916)(205.357724,39.59142873)
\curveto(205.53773027,39.51141926)(205.72273008,39.44141933)(205.912724,39.38142873)
\curveto(206.09272971,39.33141944)(206.27272953,39.2664195)(206.452724,39.18642873)
\curveto(206.6027292,39.11641965)(206.75772905,39.05641971)(206.917724,39.00642873)
\curveto(207.06772874,38.95641981)(207.21772859,38.90141987)(207.367724,38.84142873)
\curveto(207.83772797,38.64142013)(208.31272749,38.46142031)(208.792724,38.30142873)
\curveto(209.26272654,38.14142063)(209.72772608,37.9664208)(210.187724,37.77642873)
\curveto(210.36772544,37.69642107)(210.54772526,37.62642114)(210.727724,37.56642873)
\curveto(210.9077249,37.50642126)(211.08772472,37.44142133)(211.267724,37.37142873)
\curveto(211.37772443,37.32142145)(211.48272432,37.2714215)(211.582724,37.22142873)
\curveto(211.67272413,37.18142159)(211.73772407,37.09642167)(211.777724,36.96642873)
\curveto(211.78772402,36.94642182)(211.79272401,36.92142185)(211.792724,36.89142873)
\curveto(211.78272402,36.8714219)(211.78272402,36.84642192)(211.792724,36.81642873)
\curveto(211.802724,36.78642198)(211.807724,36.75142202)(211.807724,36.71142873)
\curveto(211.79772401,36.6714221)(211.79272401,36.63142214)(211.792724,36.59142873)
\lineto(211.792724,36.29142873)
\curveto(211.79272401,36.19142258)(211.76772404,36.11142266)(211.717724,36.05142873)
\curveto(211.66772414,35.9714228)(211.59772421,35.91142286)(211.507724,35.87142873)
\curveto(211.4077244,35.84142293)(211.3077245,35.80142297)(211.207724,35.75142873)
\curveto(211.0077248,35.6714231)(210.802725,35.59142318)(210.592724,35.51142873)
\curveto(210.37272543,35.44142333)(210.16272564,35.3664234)(209.962724,35.28642873)
\curveto(209.78272602,35.20642356)(209.6027262,35.13642363)(209.422724,35.07642873)
\curveto(209.23272657,35.02642374)(209.04772676,34.96142381)(208.867724,34.88142873)
\curveto(208.3077275,34.65142412)(207.74272806,34.43642433)(207.172724,34.23642873)
\curveto(206.6027292,34.03642473)(206.03772977,33.82142495)(205.477724,33.59142873)
\lineto(204.847724,33.35142873)
\curveto(204.62773118,33.28142549)(204.41773139,33.20642556)(204.217724,33.12642873)
\curveto(204.1077317,33.07642569)(204.0027318,33.03142574)(203.902724,32.99142873)
\curveto(203.79273201,32.96142581)(203.69773211,32.91142586)(203.617724,32.84142873)
\curveto(203.59773221,32.83142594)(203.58773222,32.82142595)(203.587724,32.81142873)
\lineto(203.557724,32.78142873)
\lineto(203.557724,32.70642873)
\lineto(203.587724,32.67642873)
\curveto(203.58773222,32.6664261)(203.59273221,32.65642611)(203.602724,32.64642873)
\curveto(203.65273215,32.62642614)(203.7077321,32.61642615)(203.767724,32.61642873)
\curveto(203.82773198,32.61642615)(203.88773192,32.60642616)(203.947724,32.58642873)
\lineto(204.112724,32.58642873)
\curveto(204.17273163,32.5664262)(204.23773157,32.56142621)(204.307724,32.57142873)
\curveto(204.37773143,32.58142619)(204.44773136,32.58642618)(204.517724,32.58642873)
\lineto(205.327724,32.58642873)
\lineto(209.887724,32.58642873)
\lineto(211.072724,32.58642873)
\curveto(211.18272462,32.58642618)(211.29272451,32.58142619)(211.402724,32.57142873)
\curveto(211.51272429,32.5714262)(211.59772421,32.54642622)(211.657724,32.49642873)
\curveto(211.73772407,32.44642632)(211.78272402,32.35642641)(211.792724,32.22642873)
\lineto(211.792724,31.83642873)
\lineto(211.792724,31.64142873)
\curveto(211.79272401,31.59142718)(211.78272402,31.54142723)(211.762724,31.49142873)
\curveto(211.72272408,31.36142741)(211.63772417,31.28642748)(211.507724,31.26642873)
\curveto(211.37772443,31.25642751)(211.22772458,31.25142752)(211.057724,31.25142873)
\lineto(209.317724,31.25142873)
\lineto(203.317724,31.25142873)
\lineto(201.907724,31.25142873)
\curveto(201.79773401,31.25142752)(201.68273412,31.24642752)(201.562724,31.23642873)
\curveto(201.44273436,31.23642753)(201.34773446,31.26142751)(201.277724,31.31142873)
\curveto(201.21773459,31.35142742)(201.16773464,31.42642734)(201.127724,31.53642873)
\curveto(201.11773469,31.55642721)(201.11773469,31.57642719)(201.127724,31.59642873)
\curveto(201.12773468,31.62642714)(201.12273468,31.65142712)(201.112724,31.67142873)
}
}
{
\newrgbcolor{curcolor}{0 0 0}
\pscustom[linestyle=none,fillstyle=solid,fillcolor=curcolor]
{
\newpath
\moveto(211.237724,50.87353811)
\curveto(211.39772441,50.90353028)(211.53272427,50.88853029)(211.642724,50.82853811)
\curveto(211.74272406,50.76853041)(211.81772399,50.68853049)(211.867724,50.58853811)
\curveto(211.88772392,50.53853064)(211.89772391,50.4835307)(211.897724,50.42353811)
\curveto(211.89772391,50.37353081)(211.9077239,50.31853086)(211.927724,50.25853811)
\curveto(211.97772383,50.03853114)(211.96272384,49.81853136)(211.882724,49.59853811)
\curveto(211.81272399,49.38853179)(211.72272408,49.24353194)(211.612724,49.16353811)
\curveto(211.54272426,49.11353207)(211.46272434,49.06853211)(211.372724,49.02853811)
\curveto(211.27272453,48.98853219)(211.19272461,48.93853224)(211.132724,48.87853811)
\curveto(211.11272469,48.85853232)(211.09272471,48.83353235)(211.072724,48.80353811)
\curveto(211.05272475,48.7835324)(211.04772476,48.75353243)(211.057724,48.71353811)
\curveto(211.08772472,48.60353258)(211.14272466,48.49853268)(211.222724,48.39853811)
\curveto(211.3027245,48.30853287)(211.37272443,48.21853296)(211.432724,48.12853811)
\curveto(211.51272429,47.99853318)(211.58772422,47.85853332)(211.657724,47.70853811)
\curveto(211.71772409,47.55853362)(211.77272403,47.39853378)(211.822724,47.22853811)
\curveto(211.85272395,47.12853405)(211.87272393,47.01853416)(211.882724,46.89853811)
\curveto(211.89272391,46.78853439)(211.9077239,46.6785345)(211.927724,46.56853811)
\curveto(211.93772387,46.51853466)(211.94272386,46.47353471)(211.942724,46.43353811)
\lineto(211.942724,46.32853811)
\curveto(211.96272384,46.21853496)(211.96272384,46.11353507)(211.942724,46.01353811)
\lineto(211.942724,45.87853811)
\curveto(211.93272387,45.82853535)(211.92772388,45.7785354)(211.927724,45.72853811)
\curveto(211.92772388,45.6785355)(211.91772389,45.63353555)(211.897724,45.59353811)
\curveto(211.88772392,45.55353563)(211.88272392,45.51853566)(211.882724,45.48853811)
\curveto(211.89272391,45.46853571)(211.89272391,45.44353574)(211.882724,45.41353811)
\lineto(211.822724,45.17353811)
\curveto(211.81272399,45.09353609)(211.79272401,45.01853616)(211.762724,44.94853811)
\curveto(211.63272417,44.64853653)(211.48772432,44.40353678)(211.327724,44.21353811)
\curveto(211.15772465,44.03353715)(210.92272488,43.8835373)(210.622724,43.76353811)
\curveto(210.4027254,43.67353751)(210.13772567,43.62853755)(209.827724,43.62853811)
\lineto(209.512724,43.62853811)
\curveto(209.46272634,43.63853754)(209.41272639,43.64353754)(209.362724,43.64353811)
\lineto(209.182724,43.67353811)
\lineto(208.852724,43.79353811)
\curveto(208.74272706,43.83353735)(208.64272716,43.8835373)(208.552724,43.94353811)
\curveto(208.26272754,44.12353706)(208.04772776,44.36853681)(207.907724,44.67853811)
\curveto(207.76772804,44.98853619)(207.64272816,45.32853585)(207.532724,45.69853811)
\curveto(207.49272831,45.83853534)(207.46272834,45.9835352)(207.442724,46.13353811)
\curveto(207.42272838,46.2835349)(207.39772841,46.43353475)(207.367724,46.58353811)
\curveto(207.34772846,46.65353453)(207.33772847,46.71853446)(207.337724,46.77853811)
\curveto(207.33772847,46.84853433)(207.32772848,46.92353426)(207.307724,47.00353811)
\curveto(207.28772852,47.07353411)(207.27772853,47.14353404)(207.277724,47.21353811)
\curveto(207.26772854,47.2835339)(207.25272855,47.35853382)(207.232724,47.43853811)
\curveto(207.17272863,47.68853349)(207.12272868,47.92353326)(207.082724,48.14353811)
\curveto(207.03272877,48.36353282)(206.91772889,48.53853264)(206.737724,48.66853811)
\curveto(206.65772915,48.72853245)(206.55772925,48.7785324)(206.437724,48.81853811)
\curveto(206.3077295,48.85853232)(206.16772964,48.85853232)(206.017724,48.81853811)
\curveto(205.77773003,48.75853242)(205.58773022,48.66853251)(205.447724,48.54853811)
\curveto(205.3077305,48.43853274)(205.19773061,48.2785329)(205.117724,48.06853811)
\curveto(205.06773074,47.94853323)(205.03273077,47.80353338)(205.012724,47.63353811)
\curveto(204.99273081,47.47353371)(204.98273082,47.30353388)(204.982724,47.12353811)
\curveto(204.98273082,46.94353424)(204.99273081,46.76853441)(205.012724,46.59853811)
\curveto(205.03273077,46.42853475)(205.06273074,46.2835349)(205.102724,46.16353811)
\curveto(205.16273064,45.99353519)(205.24773056,45.82853535)(205.357724,45.66853811)
\curveto(205.41773039,45.58853559)(205.49773031,45.51353567)(205.597724,45.44353811)
\curveto(205.68773012,45.3835358)(205.78773002,45.32853585)(205.897724,45.27853811)
\curveto(205.97772983,45.24853593)(206.06272974,45.21853596)(206.152724,45.18853811)
\curveto(206.24272956,45.16853601)(206.31272949,45.12353606)(206.362724,45.05353811)
\curveto(206.39272941,45.01353617)(206.41772939,44.94353624)(206.437724,44.84353811)
\curveto(206.44772936,44.75353643)(206.45272935,44.65853652)(206.452724,44.55853811)
\curveto(206.45272935,44.45853672)(206.44772936,44.35853682)(206.437724,44.25853811)
\curveto(206.41772939,44.16853701)(206.39272941,44.10353708)(206.362724,44.06353811)
\curveto(206.33272947,44.02353716)(206.28272952,43.99353719)(206.212724,43.97353811)
\curveto(206.14272966,43.95353723)(206.06772974,43.95353723)(205.987724,43.97353811)
\curveto(205.85772995,44.00353718)(205.73773007,44.03353715)(205.627724,44.06353811)
\curveto(205.5077303,44.10353708)(205.39273041,44.14853703)(205.282724,44.19853811)
\curveto(204.93273087,44.38853679)(204.66273114,44.62853655)(204.472724,44.91853811)
\curveto(204.27273153,45.20853597)(204.11273169,45.56853561)(203.992724,45.99853811)
\curveto(203.97273183,46.09853508)(203.95773185,46.19853498)(203.947724,46.29853811)
\curveto(203.93773187,46.40853477)(203.92273188,46.51853466)(203.902724,46.62853811)
\curveto(203.89273191,46.66853451)(203.89273191,46.73353445)(203.902724,46.82353811)
\curveto(203.9027319,46.91353427)(203.89273191,46.96853421)(203.872724,46.98853811)
\curveto(203.86273194,47.68853349)(203.94273186,48.29853288)(204.112724,48.81853811)
\curveto(204.28273152,49.33853184)(204.6077312,49.70353148)(205.087724,49.91353811)
\curveto(205.28773052,50.00353118)(205.52273028,50.05353113)(205.792724,50.06353811)
\curveto(206.05272975,50.0835311)(206.32772948,50.09353109)(206.617724,50.09353811)
\lineto(209.932724,50.09353811)
\curveto(210.07272573,50.09353109)(210.2077256,50.09853108)(210.337724,50.10853811)
\curveto(210.46772534,50.11853106)(210.57272523,50.14853103)(210.652724,50.19853811)
\curveto(210.72272508,50.24853093)(210.77272503,50.31353087)(210.802724,50.39353811)
\curveto(210.84272496,50.4835307)(210.87272493,50.56853061)(210.892724,50.64853811)
\curveto(210.9027249,50.72853045)(210.94772486,50.78853039)(211.027724,50.82853811)
\curveto(211.05772475,50.84853033)(211.08772472,50.85853032)(211.117724,50.85853811)
\curveto(211.14772466,50.85853032)(211.18772462,50.86353032)(211.237724,50.87353811)
\moveto(209.572724,48.72853811)
\curveto(209.43272637,48.78853239)(209.27272653,48.81853236)(209.092724,48.81853811)
\curveto(208.9027269,48.82853235)(208.7077271,48.83353235)(208.507724,48.83353811)
\curveto(208.39772741,48.83353235)(208.29772751,48.82853235)(208.207724,48.81853811)
\curveto(208.11772769,48.80853237)(208.04772776,48.76853241)(207.997724,48.69853811)
\curveto(207.97772783,48.66853251)(207.96772784,48.59853258)(207.967724,48.48853811)
\curveto(207.98772782,48.46853271)(207.99772781,48.43353275)(207.997724,48.38353811)
\curveto(207.99772781,48.33353285)(208.0077278,48.28853289)(208.027724,48.24853811)
\curveto(208.04772776,48.16853301)(208.06772774,48.0785331)(208.087724,47.97853811)
\lineto(208.147724,47.67853811)
\curveto(208.14772766,47.64853353)(208.15272765,47.61353357)(208.162724,47.57353811)
\lineto(208.162724,47.46853811)
\curveto(208.2027276,47.31853386)(208.22772758,47.15353403)(208.237724,46.97353811)
\curveto(208.23772757,46.80353438)(208.25772755,46.64353454)(208.297724,46.49353811)
\curveto(208.31772749,46.41353477)(208.33772747,46.33853484)(208.357724,46.26853811)
\curveto(208.36772744,46.20853497)(208.38272742,46.13853504)(208.402724,46.05853811)
\curveto(208.45272735,45.89853528)(208.51772729,45.74853543)(208.597724,45.60853811)
\curveto(208.66772714,45.46853571)(208.75772705,45.34853583)(208.867724,45.24853811)
\curveto(208.97772683,45.14853603)(209.11272669,45.07353611)(209.272724,45.02353811)
\curveto(209.42272638,44.97353621)(209.6077262,44.95353623)(209.827724,44.96353811)
\curveto(209.92772588,44.96353622)(210.02272578,44.9785362)(210.112724,45.00853811)
\curveto(210.19272561,45.04853613)(210.26772554,45.09353609)(210.337724,45.14353811)
\curveto(210.44772536,45.22353596)(210.54272526,45.32853585)(210.622724,45.45853811)
\curveto(210.69272511,45.58853559)(210.75272505,45.72853545)(210.802724,45.87853811)
\curveto(210.81272499,45.92853525)(210.81772499,45.9785352)(210.817724,46.02853811)
\curveto(210.81772499,46.0785351)(210.82272498,46.12853505)(210.832724,46.17853811)
\curveto(210.85272495,46.24853493)(210.86772494,46.33353485)(210.877724,46.43353811)
\curveto(210.87772493,46.54353464)(210.86772494,46.63353455)(210.847724,46.70353811)
\curveto(210.82772498,46.76353442)(210.82272498,46.82353436)(210.832724,46.88353811)
\curveto(210.83272497,46.94353424)(210.82272498,47.00353418)(210.802724,47.06353811)
\curveto(210.78272502,47.14353404)(210.76772504,47.21853396)(210.757724,47.28853811)
\curveto(210.74772506,47.36853381)(210.72772508,47.44353374)(210.697724,47.51353811)
\curveto(210.57772523,47.80353338)(210.43272537,48.04853313)(210.262724,48.24853811)
\curveto(210.09272571,48.45853272)(209.86272594,48.61853256)(209.572724,48.72853811)
}
}
{
\newrgbcolor{curcolor}{0 0 0}
\pscustom[linestyle=none,fillstyle=solid,fillcolor=curcolor]
{
\newpath
\moveto(203.887724,55.69017873)
\curveto(203.88773192,55.92017394)(203.94773186,56.05017381)(204.067724,56.08017873)
\curveto(204.17773163,56.11017375)(204.34273146,56.12517374)(204.562724,56.12517873)
\lineto(204.847724,56.12517873)
\curveto(204.93773087,56.12517374)(205.01273079,56.10017376)(205.072724,56.05017873)
\curveto(205.15273065,55.99017387)(205.19773061,55.90517396)(205.207724,55.79517873)
\curveto(205.2077306,55.68517418)(205.22273058,55.57517429)(205.252724,55.46517873)
\curveto(205.28273052,55.32517454)(205.31273049,55.19017467)(205.342724,55.06017873)
\curveto(205.37273043,54.94017492)(205.41273039,54.82517504)(205.462724,54.71517873)
\curveto(205.59273021,54.42517544)(205.77273003,54.19017567)(206.002724,54.01017873)
\curveto(206.22272958,53.83017603)(206.47772933,53.67517619)(206.767724,53.54517873)
\curveto(206.87772893,53.50517636)(206.99272881,53.47517639)(207.112724,53.45517873)
\curveto(207.22272858,53.43517643)(207.33772847,53.41017645)(207.457724,53.38017873)
\curveto(207.5077283,53.37017649)(207.55772825,53.3651765)(207.607724,53.36517873)
\curveto(207.65772815,53.37517649)(207.7077281,53.37517649)(207.757724,53.36517873)
\curveto(207.87772793,53.33517653)(208.01772779,53.32017654)(208.177724,53.32017873)
\curveto(208.32772748,53.33017653)(208.47272733,53.33517653)(208.612724,53.33517873)
\lineto(210.457724,53.33517873)
\lineto(210.802724,53.33517873)
\curveto(210.92272488,53.33517653)(211.03772477,53.33017653)(211.147724,53.32017873)
\curveto(211.25772455,53.31017655)(211.35272445,53.30517656)(211.432724,53.30517873)
\curveto(211.51272429,53.31517655)(211.58272422,53.29517657)(211.642724,53.24517873)
\curveto(211.71272409,53.19517667)(211.75272405,53.11517675)(211.762724,53.00517873)
\curveto(211.77272403,52.90517696)(211.77772403,52.79517707)(211.777724,52.67517873)
\lineto(211.777724,52.40517873)
\curveto(211.75772405,52.35517751)(211.74272406,52.30517756)(211.732724,52.25517873)
\curveto(211.71272409,52.21517765)(211.68772412,52.18517768)(211.657724,52.16517873)
\curveto(211.58772422,52.11517775)(211.5027243,52.08517778)(211.402724,52.07517873)
\lineto(211.072724,52.07517873)
\lineto(209.917724,52.07517873)
\lineto(205.762724,52.07517873)
\lineto(204.727724,52.07517873)
\lineto(204.427724,52.07517873)
\curveto(204.32773148,52.08517778)(204.24273156,52.11517775)(204.172724,52.16517873)
\curveto(204.13273167,52.19517767)(204.1027317,52.24517762)(204.082724,52.31517873)
\curveto(204.06273174,52.39517747)(204.05273175,52.48017738)(204.052724,52.57017873)
\curveto(204.04273176,52.6601772)(204.04273176,52.75017711)(204.052724,52.84017873)
\curveto(204.06273174,52.93017693)(204.07773173,53.00017686)(204.097724,53.05017873)
\curveto(204.12773168,53.13017673)(204.18773162,53.18017668)(204.277724,53.20017873)
\curveto(204.35773145,53.23017663)(204.44773136,53.24517662)(204.547724,53.24517873)
\lineto(204.847724,53.24517873)
\curveto(204.94773086,53.24517662)(205.03773077,53.2651766)(205.117724,53.30517873)
\curveto(205.13773067,53.31517655)(205.15273065,53.32517654)(205.162724,53.33517873)
\lineto(205.207724,53.38017873)
\curveto(205.2077306,53.49017637)(205.16273064,53.58017628)(205.072724,53.65017873)
\curveto(204.97273083,53.72017614)(204.89273091,53.78017608)(204.832724,53.83017873)
\lineto(204.742724,53.92017873)
\curveto(204.63273117,54.01017585)(204.51773129,54.13517573)(204.397724,54.29517873)
\curveto(204.27773153,54.45517541)(204.18773162,54.60517526)(204.127724,54.74517873)
\curveto(204.07773173,54.83517503)(204.04273176,54.93017493)(204.022724,55.03017873)
\curveto(203.99273181,55.13017473)(203.96273184,55.23517463)(203.932724,55.34517873)
\curveto(203.92273188,55.40517446)(203.91773189,55.4651744)(203.917724,55.52517873)
\curveto(203.9077319,55.58517428)(203.89773191,55.64017422)(203.887724,55.69017873)
}
}
{
\newrgbcolor{curcolor}{0 0 0}
\pscustom[linestyle=none,fillstyle=solid,fillcolor=curcolor]
{
}
}
{
\newrgbcolor{curcolor}{0 0 0}
\pscustom[linestyle=none,fillstyle=solid,fillcolor=curcolor]
{
\newpath
\moveto(201.187724,65.05510061)
\curveto(201.18773462,65.15509575)(201.19773461,65.25009566)(201.217724,65.34010061)
\curveto(201.22773458,65.43009548)(201.25773455,65.49509541)(201.307724,65.53510061)
\curveto(201.38773442,65.59509531)(201.49273431,65.62509528)(201.622724,65.62510061)
\lineto(202.012724,65.62510061)
\lineto(203.512724,65.62510061)
\lineto(209.902724,65.62510061)
\lineto(211.072724,65.62510061)
\lineto(211.387724,65.62510061)
\curveto(211.48772432,65.63509527)(211.56772424,65.62009529)(211.627724,65.58010061)
\curveto(211.7077241,65.53009538)(211.75772405,65.45509545)(211.777724,65.35510061)
\curveto(211.78772402,65.26509564)(211.79272401,65.15509575)(211.792724,65.02510061)
\lineto(211.792724,64.80010061)
\curveto(211.77272403,64.72009619)(211.75772405,64.65009626)(211.747724,64.59010061)
\curveto(211.72772408,64.53009638)(211.68772412,64.48009643)(211.627724,64.44010061)
\curveto(211.56772424,64.40009651)(211.49272431,64.38009653)(211.402724,64.38010061)
\lineto(211.102724,64.38010061)
\lineto(210.007724,64.38010061)
\lineto(204.667724,64.38010061)
\curveto(204.57773123,64.36009655)(204.5027313,64.34509656)(204.442724,64.33510061)
\curveto(204.37273143,64.33509657)(204.31273149,64.3050966)(204.262724,64.24510061)
\curveto(204.21273159,64.17509673)(204.18773162,64.08509682)(204.187724,63.97510061)
\curveto(204.17773163,63.87509703)(204.17273163,63.76509714)(204.172724,63.64510061)
\lineto(204.172724,62.50510061)
\lineto(204.172724,62.01010061)
\curveto(204.16273164,61.85009906)(204.1027317,61.74009917)(203.992724,61.68010061)
\curveto(203.96273184,61.66009925)(203.93273187,61.65009926)(203.902724,61.65010061)
\curveto(203.86273194,61.65009926)(203.81773199,61.64509926)(203.767724,61.63510061)
\curveto(203.64773216,61.61509929)(203.53773227,61.62009929)(203.437724,61.65010061)
\curveto(203.33773247,61.69009922)(203.26773254,61.74509916)(203.227724,61.81510061)
\curveto(203.17773263,61.89509901)(203.15273265,62.01509889)(203.152724,62.17510061)
\curveto(203.15273265,62.33509857)(203.13773267,62.47009844)(203.107724,62.58010061)
\curveto(203.09773271,62.63009828)(203.09273271,62.68509822)(203.092724,62.74510061)
\curveto(203.08273272,62.8050981)(203.06773274,62.86509804)(203.047724,62.92510061)
\curveto(202.99773281,63.07509783)(202.94773286,63.22009769)(202.897724,63.36010061)
\curveto(202.83773297,63.50009741)(202.76773304,63.63509727)(202.687724,63.76510061)
\curveto(202.59773321,63.905097)(202.49273331,64.02509688)(202.372724,64.12510061)
\curveto(202.25273355,64.22509668)(202.12273368,64.32009659)(201.982724,64.41010061)
\curveto(201.88273392,64.47009644)(201.77273403,64.51509639)(201.652724,64.54510061)
\curveto(201.53273427,64.58509632)(201.42773438,64.63509627)(201.337724,64.69510061)
\curveto(201.27773453,64.74509616)(201.23773457,64.81509609)(201.217724,64.90510061)
\curveto(201.2077346,64.92509598)(201.2027346,64.95009596)(201.202724,64.98010061)
\curveto(201.2027346,65.0100959)(201.19773461,65.03509587)(201.187724,65.05510061)
}
}
{
\newrgbcolor{curcolor}{0 0 0}
\pscustom[linestyle=none,fillstyle=solid,fillcolor=curcolor]
{
\newpath
\moveto(208.717724,76.35970998)
\curveto(208.75772705,76.36970226)(208.807727,76.36970226)(208.867724,76.35970998)
\curveto(208.92772688,76.35970227)(208.97772683,76.35470228)(209.017724,76.34470998)
\curveto(209.05772675,76.34470229)(209.09772671,76.33970229)(209.137724,76.32970998)
\lineto(209.242724,76.32970998)
\curveto(209.32272648,76.30970232)(209.4027264,76.29470234)(209.482724,76.28470998)
\curveto(209.56272624,76.27470236)(209.63772617,76.25470238)(209.707724,76.22470998)
\curveto(209.78772602,76.20470243)(209.86272594,76.18470245)(209.932724,76.16470998)
\curveto(210.0027258,76.14470249)(210.07772573,76.11470252)(210.157724,76.07470998)
\curveto(210.57772523,75.89470274)(210.91772489,75.63970299)(211.177724,75.30970998)
\curveto(211.43772437,74.97970365)(211.64272416,74.58970404)(211.792724,74.13970998)
\curveto(211.83272397,74.01970461)(211.85772395,73.89470474)(211.867724,73.76470998)
\curveto(211.88772392,73.64470499)(211.91272389,73.51970511)(211.942724,73.38970998)
\curveto(211.95272385,73.3297053)(211.95772385,73.26470537)(211.957724,73.19470998)
\curveto(211.95772385,73.1347055)(211.96272384,73.06970556)(211.972724,72.99970998)
\lineto(211.972724,72.87970998)
\lineto(211.972724,72.68470998)
\curveto(211.98272382,72.62470601)(211.97772383,72.56970606)(211.957724,72.51970998)
\curveto(211.93772387,72.44970618)(211.93272387,72.38470625)(211.942724,72.32470998)
\curveto(211.95272385,72.26470637)(211.94772386,72.20470643)(211.927724,72.14470998)
\curveto(211.91772389,72.09470654)(211.91272389,72.04970658)(211.912724,72.00970998)
\curveto(211.91272389,71.96970666)(211.9027239,71.92470671)(211.882724,71.87470998)
\curveto(211.86272394,71.79470684)(211.84272396,71.71970691)(211.822724,71.64970998)
\curveto(211.81272399,71.57970705)(211.79772401,71.50970712)(211.777724,71.43970998)
\curveto(211.6077242,70.95970767)(211.39772441,70.55970807)(211.147724,70.23970998)
\curveto(210.88772492,69.9297087)(210.53272527,69.67970895)(210.082724,69.48970998)
\curveto(210.02272578,69.45970917)(209.96272584,69.4347092)(209.902724,69.41470998)
\curveto(209.83272597,69.40470923)(209.75772605,69.38970924)(209.677724,69.36970998)
\curveto(209.61772619,69.34970928)(209.55272625,69.3347093)(209.482724,69.32470998)
\curveto(209.41272639,69.31470932)(209.34272646,69.29970933)(209.272724,69.27970998)
\curveto(209.22272658,69.26970936)(209.18272662,69.26470937)(209.152724,69.26470998)
\lineto(209.032724,69.26470998)
\curveto(208.99272681,69.25470938)(208.94272686,69.24470939)(208.882724,69.23470998)
\curveto(208.82272698,69.2347094)(208.77272703,69.23970939)(208.732724,69.24970998)
\lineto(208.597724,69.24970998)
\curveto(208.54772726,69.25970937)(208.49772731,69.26470937)(208.447724,69.26470998)
\curveto(208.34772746,69.28470935)(208.25272755,69.29970933)(208.162724,69.30970998)
\curveto(208.06272774,69.31970931)(207.96772784,69.33970929)(207.877724,69.36970998)
\curveto(207.72772808,69.41970921)(207.58772822,69.47470916)(207.457724,69.53470998)
\curveto(207.32772848,69.59470904)(207.2077286,69.66470897)(207.097724,69.74470998)
\curveto(207.04772876,69.77470886)(207.0077288,69.80470883)(206.977724,69.83470998)
\curveto(206.94772886,69.87470876)(206.91272889,69.90970872)(206.872724,69.93970998)
\curveto(206.79272901,69.99970863)(206.72272908,70.06970856)(206.662724,70.14970998)
\curveto(206.61272919,70.20970842)(206.56772924,70.26970836)(206.527724,70.32970998)
\lineto(206.377724,70.53970998)
\curveto(206.33772947,70.58970804)(206.3027295,70.63970799)(206.272724,70.68970998)
\curveto(206.23272957,70.73970789)(206.17772963,70.77470786)(206.107724,70.79470998)
\curveto(206.07772973,70.79470784)(206.05272975,70.78470785)(206.032724,70.76470998)
\curveto(206.0027298,70.75470788)(205.97772983,70.74470789)(205.957724,70.73470998)
\curveto(205.9077299,70.69470794)(205.86272994,70.64470799)(205.822724,70.58470998)
\curveto(205.77273003,70.5347081)(205.72773008,70.48470815)(205.687724,70.43470998)
\curveto(205.65773015,70.39470824)(205.6027302,70.34470829)(205.522724,70.28470998)
\curveto(205.49273031,70.26470837)(205.46773034,70.2347084)(205.447724,70.19470998)
\curveto(205.41773039,70.16470847)(205.38273042,70.13970849)(205.342724,70.11970998)
\curveto(205.13273067,69.94970868)(204.88773092,69.81970881)(204.607724,69.72970998)
\curveto(204.52773128,69.70970892)(204.44773136,69.69470894)(204.367724,69.68470998)
\curveto(204.28773152,69.67470896)(204.2077316,69.65970897)(204.127724,69.63970998)
\curveto(204.07773173,69.61970901)(204.01273179,69.60970902)(203.932724,69.60970998)
\curveto(203.84273196,69.60970902)(203.77273203,69.61970901)(203.722724,69.63970998)
\curveto(203.62273218,69.63970899)(203.55273225,69.64470899)(203.512724,69.65470998)
\curveto(203.43273237,69.67470896)(203.36273244,69.68970894)(203.302724,69.69970998)
\curveto(203.23273257,69.70970892)(203.16273264,69.72470891)(203.092724,69.74470998)
\curveto(202.66273314,69.89470874)(202.31773349,70.10970852)(202.057724,70.38970998)
\curveto(201.79773401,70.67970795)(201.58273422,71.0297076)(201.412724,71.43970998)
\curveto(201.36273444,71.54970708)(201.33273447,71.66470697)(201.322724,71.78470998)
\curveto(201.3027345,71.91470672)(201.27273453,72.04470659)(201.232724,72.17470998)
\curveto(201.23273457,72.25470638)(201.23273457,72.32470631)(201.232724,72.38470998)
\curveto(201.22273458,72.45470618)(201.21273459,72.5297061)(201.202724,72.60970998)
\curveto(201.18273462,73.39970523)(201.31273449,74.05470458)(201.592724,74.57470998)
\curveto(201.87273393,75.10470353)(202.28273352,75.48470315)(202.822724,75.71470998)
\curveto(203.05273275,75.82470281)(203.33773247,75.89470274)(203.677724,75.92470998)
\curveto(204.0077318,75.96470267)(204.31273149,75.9347027)(204.592724,75.83470998)
\curveto(204.72273108,75.79470284)(204.84273096,75.74470289)(204.952724,75.68470998)
\curveto(205.06273074,75.634703)(205.16773064,75.57470306)(205.267724,75.50470998)
\curveto(205.3077305,75.48470315)(205.34273046,75.45470318)(205.372724,75.41470998)
\lineto(205.462724,75.32470998)
\curveto(205.55273025,75.27470336)(205.61773019,75.21470342)(205.657724,75.14470998)
\curveto(205.7077301,75.09470354)(205.75773005,75.03970359)(205.807724,74.97970998)
\curveto(205.84772996,74.9297037)(205.89272991,74.88470375)(205.942724,74.84470998)
\curveto(205.96272984,74.82470381)(205.98772982,74.80470383)(206.017724,74.78470998)
\curveto(206.03772977,74.77470386)(206.06272974,74.77470386)(206.092724,74.78470998)
\curveto(206.14272966,74.79470384)(206.19272961,74.82470381)(206.242724,74.87470998)
\curveto(206.28272952,74.92470371)(206.32272948,74.97970365)(206.362724,75.03970998)
\lineto(206.482724,75.21970998)
\curveto(206.51272929,75.27970335)(206.54272926,75.3297033)(206.572724,75.36970998)
\curveto(206.81272899,75.69970293)(207.12272868,75.94970268)(207.502724,76.11970998)
\curveto(207.58272822,76.15970247)(207.66772814,76.18970244)(207.757724,76.20970998)
\curveto(207.84772796,76.23970239)(207.93772787,76.26470237)(208.027724,76.28470998)
\curveto(208.07772773,76.29470234)(208.13272767,76.30470233)(208.192724,76.31470998)
\lineto(208.342724,76.34470998)
\curveto(208.4027274,76.35470228)(208.46772734,76.35470228)(208.537724,76.34470998)
\curveto(208.59772721,76.3347023)(208.65772715,76.33970229)(208.717724,76.35970998)
\moveto(203.677724,70.97470998)
\curveto(203.78773202,70.94470769)(203.92773188,70.93970769)(204.097724,70.95970998)
\curveto(204.25773155,70.97970765)(204.38273142,71.00470763)(204.472724,71.03470998)
\curveto(204.79273101,71.14470749)(205.03773077,71.29470734)(205.207724,71.48470998)
\curveto(205.36773044,71.67470696)(205.49773031,71.93970669)(205.597724,72.27970998)
\curveto(205.62773018,72.40970622)(205.65273015,72.57470606)(205.672724,72.77470998)
\curveto(205.68273012,72.97470566)(205.66773014,73.14470549)(205.627724,73.28470998)
\curveto(205.54773026,73.57470506)(205.43773037,73.81470482)(205.297724,74.00470998)
\curveto(205.14773066,74.20470443)(204.94773086,74.35970427)(204.697724,74.46970998)
\curveto(204.64773116,74.48970414)(204.6027312,74.49970413)(204.562724,74.49970998)
\curveto(204.52273128,74.50970412)(204.47773133,74.52470411)(204.427724,74.54470998)
\curveto(204.31773149,74.57470406)(204.17773163,74.59470404)(204.007724,74.60470998)
\curveto(203.83773197,74.61470402)(203.69273211,74.60470403)(203.572724,74.57470998)
\curveto(203.48273232,74.55470408)(203.39773241,74.5297041)(203.317724,74.49970998)
\curveto(203.23773257,74.47970415)(203.15773265,74.44470419)(203.077724,74.39470998)
\curveto(202.807733,74.22470441)(202.61273319,73.99970463)(202.492724,73.71970998)
\curveto(202.37273343,73.44970518)(202.31273349,73.08970554)(202.312724,72.63970998)
\curveto(202.33273347,72.61970601)(202.33773347,72.58970604)(202.327724,72.54970998)
\curveto(202.31773349,72.50970612)(202.31773349,72.47470616)(202.327724,72.44470998)
\curveto(202.34773346,72.39470624)(202.36273344,72.33970629)(202.372724,72.27970998)
\curveto(202.37273343,72.2297064)(202.38273342,72.17970645)(202.402724,72.12970998)
\curveto(202.49273331,71.88970674)(202.6077332,71.67970695)(202.747724,71.49970998)
\curveto(202.87773293,71.31970731)(203.05773275,71.17970745)(203.287724,71.07970998)
\curveto(203.34773246,71.05970757)(203.41273239,71.03970759)(203.482724,71.01970998)
\curveto(203.54273226,71.00970762)(203.6077322,70.99470764)(203.677724,70.97470998)
\moveto(209.212724,74.99470998)
\curveto(209.02272678,75.04470359)(208.81772699,75.04970358)(208.597724,75.00970998)
\curveto(208.37772743,74.97970365)(208.19772761,74.9347037)(208.057724,74.87470998)
\curveto(207.68772812,74.70470393)(207.38272842,74.44470419)(207.142724,74.09470998)
\curveto(206.9027289,73.75470488)(206.78272902,73.31970531)(206.782724,72.78970998)
\curveto(206.802729,72.75970587)(206.807729,72.71970591)(206.797724,72.66970998)
\curveto(206.77772903,72.61970601)(206.77272903,72.57970605)(206.782724,72.54970998)
\lineto(206.842724,72.27970998)
\curveto(206.85272895,72.19970643)(206.86772894,72.11970651)(206.887724,72.03970998)
\curveto(206.99772881,71.73970689)(207.14272866,71.47470716)(207.322724,71.24470998)
\curveto(207.5027283,71.02470761)(207.73272807,70.85470778)(208.012724,70.73470998)
\curveto(208.09272771,70.70470793)(208.17272763,70.67970795)(208.252724,70.65970998)
\curveto(208.33272747,70.63970799)(208.41772739,70.61970801)(208.507724,70.59970998)
\curveto(208.62772718,70.56970806)(208.77772703,70.55970807)(208.957724,70.56970998)
\curveto(209.13772667,70.58970804)(209.27772653,70.61470802)(209.377724,70.64470998)
\curveto(209.42772638,70.66470797)(209.47272633,70.67470796)(209.512724,70.67470998)
\curveto(209.54272626,70.68470795)(209.58272622,70.69970793)(209.632724,70.71970998)
\curveto(209.85272595,70.81970781)(210.05272575,70.94970768)(210.232724,71.10970998)
\curveto(210.41272539,71.27970735)(210.54772526,71.47470716)(210.637724,71.69470998)
\curveto(210.67772513,71.76470687)(210.71272509,71.85970677)(210.742724,71.97970998)
\curveto(210.83272497,72.19970643)(210.87772493,72.45470618)(210.877724,72.74470998)
\lineto(210.877724,73.02970998)
\curveto(210.85772495,73.1297055)(210.84272496,73.22470541)(210.832724,73.31470998)
\curveto(210.82272498,73.40470523)(210.802725,73.49470514)(210.772724,73.58470998)
\curveto(210.69272511,73.84470479)(210.56272524,74.08470455)(210.382724,74.30470998)
\curveto(210.19272561,74.5347041)(209.97772583,74.70470393)(209.737724,74.81470998)
\curveto(209.65772615,74.85470378)(209.57772623,74.88470375)(209.497724,74.90470998)
\curveto(209.4077264,74.9347037)(209.31272649,74.96470367)(209.212724,74.99470998)
}
}
{
\newrgbcolor{curcolor}{0 0 0}
\pscustom[linestyle=none,fillstyle=solid,fillcolor=curcolor]
{
\newpath
\moveto(210.157724,78.63431936)
\lineto(210.157724,79.26431936)
\lineto(210.157724,79.45931936)
\curveto(210.15772565,79.52931683)(210.16772564,79.58931677)(210.187724,79.63931936)
\curveto(210.22772558,79.70931665)(210.26772554,79.7593166)(210.307724,79.78931936)
\curveto(210.35772545,79.82931653)(210.42272538,79.84931651)(210.502724,79.84931936)
\curveto(210.58272522,79.8593165)(210.66772514,79.86431649)(210.757724,79.86431936)
\lineto(211.477724,79.86431936)
\curveto(211.95772385,79.86431649)(212.36772344,79.80431655)(212.707724,79.68431936)
\curveto(213.04772276,79.56431679)(213.32272248,79.36931699)(213.532724,79.09931936)
\curveto(213.58272222,79.02931733)(213.62772218,78.9593174)(213.667724,78.88931936)
\curveto(213.71772209,78.82931753)(213.76272204,78.7543176)(213.802724,78.66431936)
\curveto(213.81272199,78.64431771)(213.82272198,78.61431774)(213.832724,78.57431936)
\curveto(213.85272195,78.53431782)(213.85772195,78.48931787)(213.847724,78.43931936)
\curveto(213.81772199,78.34931801)(213.74272206,78.29431806)(213.622724,78.27431936)
\curveto(213.51272229,78.2543181)(213.41772239,78.26931809)(213.337724,78.31931936)
\curveto(213.26772254,78.34931801)(213.2027226,78.39431796)(213.142724,78.45431936)
\curveto(213.09272271,78.52431783)(213.04272276,78.58931777)(212.992724,78.64931936)
\curveto(212.94272286,78.71931764)(212.86772294,78.77931758)(212.767724,78.82931936)
\curveto(212.67772313,78.88931747)(212.58772322,78.93931742)(212.497724,78.97931936)
\curveto(212.46772334,78.99931736)(212.4077234,79.02431733)(212.317724,79.05431936)
\curveto(212.23772357,79.08431727)(212.16772364,79.08931727)(212.107724,79.06931936)
\curveto(211.96772384,79.03931732)(211.87772393,78.97931738)(211.837724,78.88931936)
\curveto(211.807724,78.80931755)(211.79272401,78.71931764)(211.792724,78.61931936)
\curveto(211.79272401,78.51931784)(211.76772404,78.43431792)(211.717724,78.36431936)
\curveto(211.64772416,78.27431808)(211.5077243,78.22931813)(211.297724,78.22931936)
\lineto(210.742724,78.22931936)
\lineto(210.517724,78.22931936)
\curveto(210.43772537,78.23931812)(210.37272543,78.2593181)(210.322724,78.28931936)
\curveto(210.24272556,78.34931801)(210.19772561,78.41931794)(210.187724,78.49931936)
\curveto(210.17772563,78.51931784)(210.17272563,78.53931782)(210.172724,78.55931936)
\curveto(210.17272563,78.58931777)(210.16772564,78.61431774)(210.157724,78.63431936)
}
}
{
\newrgbcolor{curcolor}{0 0 0}
\pscustom[linestyle=none,fillstyle=solid,fillcolor=curcolor]
{
}
}
{
\newrgbcolor{curcolor}{0 0 0}
\pscustom[linestyle=none,fillstyle=solid,fillcolor=curcolor]
{
\newpath
\moveto(201.187724,89.26463186)
\curveto(201.17773463,89.95462722)(201.29773451,90.55462662)(201.547724,91.06463186)
\curveto(201.79773401,91.58462559)(202.13273367,91.9796252)(202.552724,92.24963186)
\curveto(202.63273317,92.29962488)(202.72273308,92.34462483)(202.822724,92.38463186)
\curveto(202.91273289,92.42462475)(203.0077328,92.46962471)(203.107724,92.51963186)
\curveto(203.2077326,92.55962462)(203.3077325,92.58962459)(203.407724,92.60963186)
\curveto(203.5077323,92.62962455)(203.61273219,92.64962453)(203.722724,92.66963186)
\curveto(203.77273203,92.68962449)(203.81773199,92.69462448)(203.857724,92.68463186)
\curveto(203.89773191,92.6746245)(203.94273186,92.6796245)(203.992724,92.69963186)
\curveto(204.04273176,92.70962447)(204.12773168,92.71462446)(204.247724,92.71463186)
\curveto(204.35773145,92.71462446)(204.44273136,92.70962447)(204.502724,92.69963186)
\curveto(204.56273124,92.6796245)(204.62273118,92.66962451)(204.682724,92.66963186)
\curveto(204.74273106,92.6796245)(204.802731,92.6746245)(204.862724,92.65463186)
\curveto(205.0027308,92.61462456)(205.13773067,92.5796246)(205.267724,92.54963186)
\curveto(205.39773041,92.51962466)(205.52273028,92.4796247)(205.642724,92.42963186)
\curveto(205.78273002,92.36962481)(205.9077299,92.29962488)(206.017724,92.21963186)
\curveto(206.12772968,92.14962503)(206.23772957,92.0746251)(206.347724,91.99463186)
\lineto(206.407724,91.93463186)
\curveto(206.42772938,91.92462525)(206.44772936,91.90962527)(206.467724,91.88963186)
\curveto(206.62772918,91.76962541)(206.77272903,91.63462554)(206.902724,91.48463186)
\curveto(207.03272877,91.33462584)(207.15772865,91.174626)(207.277724,91.00463186)
\curveto(207.49772831,90.69462648)(207.7027281,90.39962678)(207.892724,90.11963186)
\curveto(208.03272777,89.88962729)(208.16772764,89.65962752)(208.297724,89.42963186)
\curveto(208.42772738,89.20962797)(208.56272724,88.98962819)(208.702724,88.76963186)
\curveto(208.87272693,88.51962866)(209.05272675,88.2796289)(209.242724,88.04963186)
\curveto(209.43272637,87.82962935)(209.65772615,87.63962954)(209.917724,87.47963186)
\curveto(209.97772583,87.43962974)(210.03772577,87.40462977)(210.097724,87.37463186)
\curveto(210.14772566,87.34462983)(210.21272559,87.31462986)(210.292724,87.28463186)
\curveto(210.36272544,87.26462991)(210.42272538,87.25962992)(210.472724,87.26963186)
\curveto(210.54272526,87.28962989)(210.59772521,87.32462985)(210.637724,87.37463186)
\curveto(210.66772514,87.42462975)(210.68772512,87.48462969)(210.697724,87.55463186)
\lineto(210.697724,87.79463186)
\lineto(210.697724,88.54463186)
\lineto(210.697724,91.34963186)
\lineto(210.697724,92.00963186)
\curveto(210.69772511,92.09962508)(210.7027251,92.18462499)(210.712724,92.26463186)
\curveto(210.71272509,92.34462483)(210.73272507,92.40962477)(210.772724,92.45963186)
\curveto(210.81272499,92.50962467)(210.88772492,92.54962463)(210.997724,92.57963186)
\curveto(211.09772471,92.61962456)(211.19772461,92.62962455)(211.297724,92.60963186)
\lineto(211.432724,92.60963186)
\curveto(211.5027243,92.58962459)(211.56272424,92.56962461)(211.612724,92.54963186)
\curveto(211.66272414,92.52962465)(211.7027241,92.49462468)(211.732724,92.44463186)
\curveto(211.77272403,92.39462478)(211.79272401,92.32462485)(211.792724,92.23463186)
\lineto(211.792724,91.96463186)
\lineto(211.792724,91.06463186)
\lineto(211.792724,87.55463186)
\lineto(211.792724,86.48963186)
\curveto(211.79272401,86.40963077)(211.79772401,86.31963086)(211.807724,86.21963186)
\curveto(211.807724,86.11963106)(211.79772401,86.03463114)(211.777724,85.96463186)
\curveto(211.7077241,85.75463142)(211.52772428,85.68963149)(211.237724,85.76963186)
\curveto(211.19772461,85.7796314)(211.16272464,85.7796314)(211.132724,85.76963186)
\curveto(211.09272471,85.76963141)(211.04772476,85.7796314)(210.997724,85.79963186)
\curveto(210.91772489,85.81963136)(210.83272497,85.83963134)(210.742724,85.85963186)
\curveto(210.65272515,85.8796313)(210.56772524,85.90463127)(210.487724,85.93463186)
\curveto(209.99772581,86.09463108)(209.58272622,86.29463088)(209.242724,86.53463186)
\curveto(208.99272681,86.71463046)(208.76772704,86.91963026)(208.567724,87.14963186)
\curveto(208.35772745,87.3796298)(208.16272764,87.61962956)(207.982724,87.86963186)
\curveto(207.802728,88.12962905)(207.63272817,88.39462878)(207.472724,88.66463186)
\curveto(207.3027285,88.94462823)(207.12772868,89.21462796)(206.947724,89.47463186)
\curveto(206.86772894,89.58462759)(206.79272901,89.68962749)(206.722724,89.78963186)
\curveto(206.65272915,89.89962728)(206.57772923,90.00962717)(206.497724,90.11963186)
\curveto(206.46772934,90.15962702)(206.43772937,90.19462698)(206.407724,90.22463186)
\curveto(206.36772944,90.26462691)(206.33772947,90.30462687)(206.317724,90.34463186)
\curveto(206.2077296,90.48462669)(206.08272972,90.60962657)(205.942724,90.71963186)
\curveto(205.91272989,90.73962644)(205.88772992,90.76462641)(205.867724,90.79463186)
\curveto(205.83772997,90.82462635)(205.80773,90.84962633)(205.777724,90.86963186)
\curveto(205.67773013,90.94962623)(205.57773023,91.01462616)(205.477724,91.06463186)
\curveto(205.37773043,91.12462605)(205.26773054,91.179626)(205.147724,91.22963186)
\curveto(205.07773073,91.25962592)(205.0027308,91.2796259)(204.922724,91.28963186)
\lineto(204.682724,91.34963186)
\lineto(204.592724,91.34963186)
\curveto(204.56273124,91.35962582)(204.53273127,91.36462581)(204.502724,91.36463186)
\curveto(204.43273137,91.38462579)(204.33773147,91.38962579)(204.217724,91.37963186)
\curveto(204.08773172,91.3796258)(203.98773182,91.36962581)(203.917724,91.34963186)
\curveto(203.83773197,91.32962585)(203.76273204,91.30962587)(203.692724,91.28963186)
\curveto(203.61273219,91.2796259)(203.53273227,91.25962592)(203.452724,91.22963186)
\curveto(203.21273259,91.11962606)(203.01273279,90.96962621)(202.852724,90.77963186)
\curveto(202.68273312,90.59962658)(202.54273326,90.3796268)(202.432724,90.11963186)
\curveto(202.41273339,90.04962713)(202.39773341,89.9796272)(202.387724,89.90963186)
\curveto(202.36773344,89.83962734)(202.34773346,89.76462741)(202.327724,89.68463186)
\curveto(202.3077335,89.60462757)(202.29773351,89.49462768)(202.297724,89.35463186)
\curveto(202.29773351,89.22462795)(202.3077335,89.11962806)(202.327724,89.03963186)
\curveto(202.33773347,88.9796282)(202.34273346,88.92462825)(202.342724,88.87463186)
\curveto(202.34273346,88.82462835)(202.35273345,88.7746284)(202.372724,88.72463186)
\curveto(202.41273339,88.62462855)(202.45273335,88.52962865)(202.492724,88.43963186)
\curveto(202.53273327,88.35962882)(202.57773323,88.2796289)(202.627724,88.19963186)
\curveto(202.64773316,88.16962901)(202.67273313,88.13962904)(202.702724,88.10963186)
\curveto(202.73273307,88.08962909)(202.75773305,88.06462911)(202.777724,88.03463186)
\lineto(202.852724,87.95963186)
\curveto(202.87273293,87.92962925)(202.89273291,87.90462927)(202.912724,87.88463186)
\lineto(203.122724,87.73463186)
\curveto(203.18273262,87.69462948)(203.24773256,87.64962953)(203.317724,87.59963186)
\curveto(203.4077324,87.53962964)(203.51273229,87.48962969)(203.632724,87.44963186)
\curveto(203.74273206,87.41962976)(203.85273195,87.38462979)(203.962724,87.34463186)
\curveto(204.07273173,87.30462987)(204.21773159,87.2796299)(204.397724,87.26963186)
\curveto(204.56773124,87.25962992)(204.69273111,87.22962995)(204.772724,87.17963186)
\curveto(204.85273095,87.12963005)(204.89773091,87.05463012)(204.907724,86.95463186)
\curveto(204.91773089,86.85463032)(204.92273088,86.74463043)(204.922724,86.62463186)
\curveto(204.92273088,86.58463059)(204.92773088,86.54463063)(204.937724,86.50463186)
\curveto(204.93773087,86.46463071)(204.93273087,86.42963075)(204.922724,86.39963186)
\curveto(204.9027309,86.34963083)(204.89273091,86.29963088)(204.892724,86.24963186)
\curveto(204.89273091,86.20963097)(204.88273092,86.16963101)(204.862724,86.12963186)
\curveto(204.802731,86.03963114)(204.66773114,85.99463118)(204.457724,85.99463186)
\lineto(204.337724,85.99463186)
\curveto(204.27773153,86.00463117)(204.21773159,86.00963117)(204.157724,86.00963186)
\curveto(204.08773172,86.01963116)(204.02273178,86.02963115)(203.962724,86.03963186)
\curveto(203.85273195,86.05963112)(203.75273205,86.0796311)(203.662724,86.09963186)
\curveto(203.56273224,86.11963106)(203.46773234,86.14963103)(203.377724,86.18963186)
\curveto(203.3077325,86.20963097)(203.24773256,86.22963095)(203.197724,86.24963186)
\lineto(203.017724,86.30963186)
\curveto(202.75773305,86.42963075)(202.51273329,86.58463059)(202.282724,86.77463186)
\curveto(202.05273375,86.9746302)(201.86773394,87.18962999)(201.727724,87.41963186)
\curveto(201.64773416,87.52962965)(201.58273422,87.64462953)(201.532724,87.76463186)
\lineto(201.382724,88.15463186)
\curveto(201.33273447,88.26462891)(201.3027345,88.3796288)(201.292724,88.49963186)
\curveto(201.27273453,88.61962856)(201.24773456,88.74462843)(201.217724,88.87463186)
\curveto(201.21773459,88.94462823)(201.21773459,89.00962817)(201.217724,89.06963186)
\curveto(201.2077346,89.12962805)(201.19773461,89.19462798)(201.187724,89.26463186)
}
}
{
\newrgbcolor{curcolor}{0 0 0}
\pscustom[linestyle=none,fillstyle=solid,fillcolor=curcolor]
{
\newpath
\moveto(206.707724,101.36424123)
\lineto(206.962724,101.36424123)
\curveto(207.04272876,101.37423353)(207.11772869,101.36923353)(207.187724,101.34924123)
\lineto(207.427724,101.34924123)
\lineto(207.592724,101.34924123)
\curveto(207.69272811,101.32923357)(207.79772801,101.31923358)(207.907724,101.31924123)
\curveto(208.0077278,101.31923358)(208.1077277,101.30923359)(208.207724,101.28924123)
\lineto(208.357724,101.28924123)
\curveto(208.49772731,101.25923364)(208.63772717,101.23923366)(208.777724,101.22924123)
\curveto(208.9077269,101.21923368)(209.03772677,101.19423371)(209.167724,101.15424123)
\curveto(209.24772656,101.13423377)(209.33272647,101.11423379)(209.422724,101.09424123)
\lineto(209.662724,101.03424123)
\lineto(209.962724,100.91424123)
\curveto(210.05272575,100.88423402)(210.14272566,100.84923405)(210.232724,100.80924123)
\curveto(210.45272535,100.70923419)(210.66772514,100.57423433)(210.877724,100.40424123)
\curveto(211.08772472,100.24423466)(211.25772455,100.06923483)(211.387724,99.87924123)
\curveto(211.42772438,99.82923507)(211.46772434,99.76923513)(211.507724,99.69924123)
\curveto(211.53772427,99.63923526)(211.57272423,99.57923532)(211.612724,99.51924123)
\curveto(211.66272414,99.43923546)(211.7027241,99.34423556)(211.732724,99.23424123)
\curveto(211.76272404,99.12423578)(211.79272401,99.01923588)(211.822724,98.91924123)
\curveto(211.86272394,98.80923609)(211.88772392,98.6992362)(211.897724,98.58924123)
\curveto(211.9077239,98.47923642)(211.92272388,98.36423654)(211.942724,98.24424123)
\curveto(211.95272385,98.2042367)(211.95272385,98.15923674)(211.942724,98.10924123)
\curveto(211.94272386,98.06923683)(211.94772386,98.02923687)(211.957724,97.98924123)
\curveto(211.96772384,97.94923695)(211.97272383,97.89423701)(211.972724,97.82424123)
\curveto(211.97272383,97.75423715)(211.96772384,97.7042372)(211.957724,97.67424123)
\curveto(211.93772387,97.62423728)(211.93272387,97.57923732)(211.942724,97.53924123)
\curveto(211.95272385,97.4992374)(211.95272385,97.46423744)(211.942724,97.43424123)
\lineto(211.942724,97.34424123)
\curveto(211.92272388,97.28423762)(211.9077239,97.21923768)(211.897724,97.14924123)
\curveto(211.89772391,97.08923781)(211.89272391,97.02423788)(211.882724,96.95424123)
\curveto(211.83272397,96.78423812)(211.78272402,96.62423828)(211.732724,96.47424123)
\curveto(211.68272412,96.32423858)(211.61772419,96.17923872)(211.537724,96.03924123)
\curveto(211.49772431,95.98923891)(211.46772434,95.93423897)(211.447724,95.87424123)
\curveto(211.41772439,95.82423908)(211.38272442,95.77423913)(211.342724,95.72424123)
\curveto(211.16272464,95.48423942)(210.94272486,95.28423962)(210.682724,95.12424123)
\curveto(210.42272538,94.96423994)(210.13772567,94.82424008)(209.827724,94.70424123)
\curveto(209.68772612,94.64424026)(209.54772626,94.5992403)(209.407724,94.56924123)
\curveto(209.25772655,94.53924036)(209.1027267,94.5042404)(208.942724,94.46424123)
\curveto(208.83272697,94.44424046)(208.72272708,94.42924047)(208.612724,94.41924123)
\curveto(208.5027273,94.40924049)(208.39272741,94.39424051)(208.282724,94.37424123)
\curveto(208.24272756,94.36424054)(208.2027276,94.35924054)(208.162724,94.35924123)
\curveto(208.12272768,94.36924053)(208.08272772,94.36924053)(208.042724,94.35924123)
\curveto(207.99272781,94.34924055)(207.94272786,94.34424056)(207.892724,94.34424123)
\lineto(207.727724,94.34424123)
\curveto(207.67772813,94.32424058)(207.62772818,94.31924058)(207.577724,94.32924123)
\curveto(207.51772829,94.33924056)(207.46272834,94.33924056)(207.412724,94.32924123)
\curveto(207.37272843,94.31924058)(207.32772848,94.31924058)(207.277724,94.32924123)
\curveto(207.22772858,94.33924056)(207.17772863,94.33424057)(207.127724,94.31424123)
\curveto(207.05772875,94.29424061)(206.98272882,94.28924061)(206.902724,94.29924123)
\curveto(206.81272899,94.30924059)(206.72772908,94.31424059)(206.647724,94.31424123)
\curveto(206.55772925,94.31424059)(206.45772935,94.30924059)(206.347724,94.29924123)
\curveto(206.22772958,94.28924061)(206.12772968,94.29424061)(206.047724,94.31424123)
\lineto(205.762724,94.31424123)
\lineto(205.132724,94.35924123)
\curveto(205.03273077,94.36924053)(204.93773087,94.37924052)(204.847724,94.38924123)
\lineto(204.547724,94.41924123)
\curveto(204.49773131,94.43924046)(204.44773136,94.44424046)(204.397724,94.43424123)
\curveto(204.33773147,94.43424047)(204.28273152,94.44424046)(204.232724,94.46424123)
\curveto(204.06273174,94.51424039)(203.89773191,94.55424035)(203.737724,94.58424123)
\curveto(203.56773224,94.61424029)(203.4077324,94.66424024)(203.257724,94.73424123)
\curveto(202.79773301,94.92423998)(202.42273338,95.14423976)(202.132724,95.39424123)
\curveto(201.84273396,95.65423925)(201.59773421,96.01423889)(201.397724,96.47424123)
\curveto(201.34773446,96.6042383)(201.31273449,96.73423817)(201.292724,96.86424123)
\curveto(201.27273453,97.0042379)(201.24773456,97.14423776)(201.217724,97.28424123)
\curveto(201.2077346,97.35423755)(201.2027346,97.41923748)(201.202724,97.47924123)
\curveto(201.2027346,97.53923736)(201.19773461,97.6042373)(201.187724,97.67424123)
\curveto(201.16773464,98.5042364)(201.31773449,99.17423573)(201.637724,99.68424123)
\curveto(201.94773386,100.19423471)(202.38773342,100.57423433)(202.957724,100.82424123)
\curveto(203.07773273,100.87423403)(203.2027326,100.91923398)(203.332724,100.95924123)
\curveto(203.46273234,100.9992339)(203.59773221,101.04423386)(203.737724,101.09424123)
\curveto(203.81773199,101.11423379)(203.9027319,101.12923377)(203.992724,101.13924123)
\lineto(204.232724,101.19924123)
\curveto(204.34273146,101.22923367)(204.45273135,101.24423366)(204.562724,101.24424123)
\curveto(204.67273113,101.25423365)(204.78273102,101.26923363)(204.892724,101.28924123)
\curveto(204.94273086,101.30923359)(204.98773082,101.31423359)(205.027724,101.30424123)
\curveto(205.06773074,101.3042336)(205.1077307,101.30923359)(205.147724,101.31924123)
\curveto(205.19773061,101.32923357)(205.25273055,101.32923357)(205.312724,101.31924123)
\curveto(205.36273044,101.31923358)(205.41273039,101.32423358)(205.462724,101.33424123)
\lineto(205.597724,101.33424123)
\curveto(205.65773015,101.35423355)(205.72773008,101.35423355)(205.807724,101.33424123)
\curveto(205.87772993,101.32423358)(205.94272986,101.32923357)(206.002724,101.34924123)
\curveto(206.03272977,101.35923354)(206.07272973,101.36423354)(206.122724,101.36424123)
\lineto(206.242724,101.36424123)
\lineto(206.707724,101.36424123)
\moveto(209.032724,99.81924123)
\curveto(208.71272709,99.91923498)(208.34772746,99.97923492)(207.937724,99.99924123)
\curveto(207.52772828,100.01923488)(207.11772869,100.02923487)(206.707724,100.02924123)
\curveto(206.27772953,100.02923487)(205.85772995,100.01923488)(205.447724,99.99924123)
\curveto(205.03773077,99.97923492)(204.65273115,99.93423497)(204.292724,99.86424123)
\curveto(203.93273187,99.79423511)(203.61273219,99.68423522)(203.332724,99.53424123)
\curveto(203.04273276,99.39423551)(202.807733,99.1992357)(202.627724,98.94924123)
\curveto(202.51773329,98.78923611)(202.43773337,98.60923629)(202.387724,98.40924123)
\curveto(202.32773348,98.20923669)(202.29773351,97.96423694)(202.297724,97.67424123)
\curveto(202.31773349,97.65423725)(202.32773348,97.61923728)(202.327724,97.56924123)
\curveto(202.31773349,97.51923738)(202.31773349,97.47923742)(202.327724,97.44924123)
\curveto(202.34773346,97.36923753)(202.36773344,97.29423761)(202.387724,97.22424123)
\curveto(202.39773341,97.16423774)(202.41773339,97.0992378)(202.447724,97.02924123)
\curveto(202.56773324,96.75923814)(202.73773307,96.53923836)(202.957724,96.36924123)
\curveto(203.16773264,96.20923869)(203.41273239,96.07423883)(203.692724,95.96424123)
\curveto(203.802732,95.91423899)(203.92273188,95.87423903)(204.052724,95.84424123)
\curveto(204.17273163,95.82423908)(204.29773151,95.7992391)(204.427724,95.76924123)
\curveto(204.47773133,95.74923915)(204.53273127,95.73923916)(204.592724,95.73924123)
\curveto(204.64273116,95.73923916)(204.69273111,95.73423917)(204.742724,95.72424123)
\curveto(204.83273097,95.71423919)(204.92773088,95.7042392)(205.027724,95.69424123)
\curveto(205.11773069,95.68423922)(205.21273059,95.67423923)(205.312724,95.66424123)
\curveto(205.39273041,95.66423924)(205.47773033,95.65923924)(205.567724,95.64924123)
\lineto(205.807724,95.64924123)
\lineto(205.987724,95.64924123)
\curveto(206.01772979,95.63923926)(206.05272975,95.63423927)(206.092724,95.63424123)
\lineto(206.227724,95.63424123)
\lineto(206.677724,95.63424123)
\curveto(206.75772905,95.63423927)(206.84272896,95.62923927)(206.932724,95.61924123)
\curveto(207.01272879,95.61923928)(207.08772872,95.62923927)(207.157724,95.64924123)
\lineto(207.427724,95.64924123)
\curveto(207.44772836,95.64923925)(207.47772833,95.64423926)(207.517724,95.63424123)
\curveto(207.54772826,95.63423927)(207.57272823,95.63923926)(207.592724,95.64924123)
\curveto(207.69272811,95.65923924)(207.79272801,95.66423924)(207.892724,95.66424123)
\curveto(207.98272782,95.67423923)(208.08272772,95.68423922)(208.192724,95.69424123)
\curveto(208.31272749,95.72423918)(208.43772737,95.73923916)(208.567724,95.73924123)
\curveto(208.68772712,95.74923915)(208.802727,95.77423913)(208.912724,95.81424123)
\curveto(209.21272659,95.89423901)(209.47772633,95.97923892)(209.707724,96.06924123)
\curveto(209.93772587,96.16923873)(210.15272565,96.31423859)(210.352724,96.50424123)
\curveto(210.55272525,96.71423819)(210.7027251,96.97923792)(210.802724,97.29924123)
\curveto(210.82272498,97.33923756)(210.83272497,97.37423753)(210.832724,97.40424123)
\curveto(210.82272498,97.44423746)(210.82772498,97.48923741)(210.847724,97.53924123)
\curveto(210.85772495,97.57923732)(210.86772494,97.64923725)(210.877724,97.74924123)
\curveto(210.88772492,97.85923704)(210.88272492,97.94423696)(210.862724,98.00424123)
\curveto(210.84272496,98.07423683)(210.83272497,98.14423676)(210.832724,98.21424123)
\curveto(210.82272498,98.28423662)(210.807725,98.34923655)(210.787724,98.40924123)
\curveto(210.72772508,98.60923629)(210.64272516,98.78923611)(210.532724,98.94924123)
\curveto(210.51272529,98.97923592)(210.49272531,99.0042359)(210.472724,99.02424123)
\lineto(210.412724,99.08424123)
\curveto(210.39272541,99.12423578)(210.35272545,99.17423573)(210.292724,99.23424123)
\curveto(210.15272565,99.33423557)(210.02272578,99.41923548)(209.902724,99.48924123)
\curveto(209.78272602,99.55923534)(209.63772617,99.62923527)(209.467724,99.69924123)
\curveto(209.39772641,99.72923517)(209.32772648,99.74923515)(209.257724,99.75924123)
\curveto(209.18772662,99.77923512)(209.11272669,99.7992351)(209.032724,99.81924123)
}
}
{
\newrgbcolor{curcolor}{0 0 0}
\pscustom[linestyle=none,fillstyle=solid,fillcolor=curcolor]
{
\newpath
\moveto(201.187724,106.77385061)
\curveto(201.18773462,106.87384575)(201.19773461,106.96884566)(201.217724,107.05885061)
\curveto(201.22773458,107.14884548)(201.25773455,107.21384541)(201.307724,107.25385061)
\curveto(201.38773442,107.31384531)(201.49273431,107.34384528)(201.622724,107.34385061)
\lineto(202.012724,107.34385061)
\lineto(203.512724,107.34385061)
\lineto(209.902724,107.34385061)
\lineto(211.072724,107.34385061)
\lineto(211.387724,107.34385061)
\curveto(211.48772432,107.35384527)(211.56772424,107.33884529)(211.627724,107.29885061)
\curveto(211.7077241,107.24884538)(211.75772405,107.17384545)(211.777724,107.07385061)
\curveto(211.78772402,106.98384564)(211.79272401,106.87384575)(211.792724,106.74385061)
\lineto(211.792724,106.51885061)
\curveto(211.77272403,106.43884619)(211.75772405,106.36884626)(211.747724,106.30885061)
\curveto(211.72772408,106.24884638)(211.68772412,106.19884643)(211.627724,106.15885061)
\curveto(211.56772424,106.11884651)(211.49272431,106.09884653)(211.402724,106.09885061)
\lineto(211.102724,106.09885061)
\lineto(210.007724,106.09885061)
\lineto(204.667724,106.09885061)
\curveto(204.57773123,106.07884655)(204.5027313,106.06384656)(204.442724,106.05385061)
\curveto(204.37273143,106.05384657)(204.31273149,106.0238466)(204.262724,105.96385061)
\curveto(204.21273159,105.89384673)(204.18773162,105.80384682)(204.187724,105.69385061)
\curveto(204.17773163,105.59384703)(204.17273163,105.48384714)(204.172724,105.36385061)
\lineto(204.172724,104.22385061)
\lineto(204.172724,103.72885061)
\curveto(204.16273164,103.56884906)(204.1027317,103.45884917)(203.992724,103.39885061)
\curveto(203.96273184,103.37884925)(203.93273187,103.36884926)(203.902724,103.36885061)
\curveto(203.86273194,103.36884926)(203.81773199,103.36384926)(203.767724,103.35385061)
\curveto(203.64773216,103.33384929)(203.53773227,103.33884929)(203.437724,103.36885061)
\curveto(203.33773247,103.40884922)(203.26773254,103.46384916)(203.227724,103.53385061)
\curveto(203.17773263,103.61384901)(203.15273265,103.73384889)(203.152724,103.89385061)
\curveto(203.15273265,104.05384857)(203.13773267,104.18884844)(203.107724,104.29885061)
\curveto(203.09773271,104.34884828)(203.09273271,104.40384822)(203.092724,104.46385061)
\curveto(203.08273272,104.5238481)(203.06773274,104.58384804)(203.047724,104.64385061)
\curveto(202.99773281,104.79384783)(202.94773286,104.93884769)(202.897724,105.07885061)
\curveto(202.83773297,105.21884741)(202.76773304,105.35384727)(202.687724,105.48385061)
\curveto(202.59773321,105.623847)(202.49273331,105.74384688)(202.372724,105.84385061)
\curveto(202.25273355,105.94384668)(202.12273368,106.03884659)(201.982724,106.12885061)
\curveto(201.88273392,106.18884644)(201.77273403,106.23384639)(201.652724,106.26385061)
\curveto(201.53273427,106.30384632)(201.42773438,106.35384627)(201.337724,106.41385061)
\curveto(201.27773453,106.46384616)(201.23773457,106.53384609)(201.217724,106.62385061)
\curveto(201.2077346,106.64384598)(201.2027346,106.66884596)(201.202724,106.69885061)
\curveto(201.2027346,106.7288459)(201.19773461,106.75384587)(201.187724,106.77385061)
}
}
{
\newrgbcolor{curcolor}{0 0 0}
\pscustom[linestyle=none,fillstyle=solid,fillcolor=curcolor]
{
\newpath
\moveto(201.187724,115.12345998)
\curveto(201.18773462,115.22345513)(201.19773461,115.31845503)(201.217724,115.40845998)
\curveto(201.22773458,115.49845485)(201.25773455,115.56345479)(201.307724,115.60345998)
\curveto(201.38773442,115.66345469)(201.49273431,115.69345466)(201.622724,115.69345998)
\lineto(202.012724,115.69345998)
\lineto(203.512724,115.69345998)
\lineto(209.902724,115.69345998)
\lineto(211.072724,115.69345998)
\lineto(211.387724,115.69345998)
\curveto(211.48772432,115.70345465)(211.56772424,115.68845466)(211.627724,115.64845998)
\curveto(211.7077241,115.59845475)(211.75772405,115.52345483)(211.777724,115.42345998)
\curveto(211.78772402,115.33345502)(211.79272401,115.22345513)(211.792724,115.09345998)
\lineto(211.792724,114.86845998)
\curveto(211.77272403,114.78845556)(211.75772405,114.71845563)(211.747724,114.65845998)
\curveto(211.72772408,114.59845575)(211.68772412,114.5484558)(211.627724,114.50845998)
\curveto(211.56772424,114.46845588)(211.49272431,114.4484559)(211.402724,114.44845998)
\lineto(211.102724,114.44845998)
\lineto(210.007724,114.44845998)
\lineto(204.667724,114.44845998)
\curveto(204.57773123,114.42845592)(204.5027313,114.41345594)(204.442724,114.40345998)
\curveto(204.37273143,114.40345595)(204.31273149,114.37345598)(204.262724,114.31345998)
\curveto(204.21273159,114.24345611)(204.18773162,114.1534562)(204.187724,114.04345998)
\curveto(204.17773163,113.94345641)(204.17273163,113.83345652)(204.172724,113.71345998)
\lineto(204.172724,112.57345998)
\lineto(204.172724,112.07845998)
\curveto(204.16273164,111.91845843)(204.1027317,111.80845854)(203.992724,111.74845998)
\curveto(203.96273184,111.72845862)(203.93273187,111.71845863)(203.902724,111.71845998)
\curveto(203.86273194,111.71845863)(203.81773199,111.71345864)(203.767724,111.70345998)
\curveto(203.64773216,111.68345867)(203.53773227,111.68845866)(203.437724,111.71845998)
\curveto(203.33773247,111.75845859)(203.26773254,111.81345854)(203.227724,111.88345998)
\curveto(203.17773263,111.96345839)(203.15273265,112.08345827)(203.152724,112.24345998)
\curveto(203.15273265,112.40345795)(203.13773267,112.53845781)(203.107724,112.64845998)
\curveto(203.09773271,112.69845765)(203.09273271,112.7534576)(203.092724,112.81345998)
\curveto(203.08273272,112.87345748)(203.06773274,112.93345742)(203.047724,112.99345998)
\curveto(202.99773281,113.14345721)(202.94773286,113.28845706)(202.897724,113.42845998)
\curveto(202.83773297,113.56845678)(202.76773304,113.70345665)(202.687724,113.83345998)
\curveto(202.59773321,113.97345638)(202.49273331,114.09345626)(202.372724,114.19345998)
\curveto(202.25273355,114.29345606)(202.12273368,114.38845596)(201.982724,114.47845998)
\curveto(201.88273392,114.53845581)(201.77273403,114.58345577)(201.652724,114.61345998)
\curveto(201.53273427,114.6534557)(201.42773438,114.70345565)(201.337724,114.76345998)
\curveto(201.27773453,114.81345554)(201.23773457,114.88345547)(201.217724,114.97345998)
\curveto(201.2077346,114.99345536)(201.2027346,115.01845533)(201.202724,115.04845998)
\curveto(201.2027346,115.07845527)(201.19773461,115.10345525)(201.187724,115.12345998)
}
}
{
\newrgbcolor{curcolor}{0 0 0}
\pscustom[linestyle=none,fillstyle=solid,fillcolor=curcolor]
{
\newpath
\moveto(117.46738403,31.67142873)
\lineto(117.46738403,32.58642873)
\curveto(117.46739473,32.68642608)(117.46739473,32.78142599)(117.46738403,32.87142873)
\curveto(117.46739473,32.96142581)(117.48739471,33.03642573)(117.52738403,33.09642873)
\curveto(117.58739461,33.18642558)(117.66739453,33.24642552)(117.76738403,33.27642873)
\curveto(117.86739433,33.31642545)(117.97239422,33.36142541)(118.08238403,33.41142873)
\curveto(118.27239392,33.49142528)(118.46239373,33.56142521)(118.65238403,33.62142873)
\curveto(118.84239335,33.69142508)(119.03239316,33.766425)(119.22238403,33.84642873)
\curveto(119.40239279,33.91642485)(119.58739261,33.98142479)(119.77738403,34.04142873)
\curveto(119.95739224,34.10142467)(120.13739206,34.1714246)(120.31738403,34.25142873)
\curveto(120.45739174,34.31142446)(120.60239159,34.3664244)(120.75238403,34.41642873)
\curveto(120.90239129,34.4664243)(121.04739115,34.52142425)(121.18738403,34.58142873)
\curveto(121.63739056,34.76142401)(122.0923901,34.93142384)(122.55238403,35.09142873)
\curveto(123.00238919,35.25142352)(123.45238874,35.42142335)(123.90238403,35.60142873)
\curveto(123.95238824,35.62142315)(124.00238819,35.63642313)(124.05238403,35.64642873)
\lineto(124.20238403,35.70642873)
\curveto(124.42238777,35.79642297)(124.64738755,35.88142289)(124.87738403,35.96142873)
\curveto(125.0973871,36.04142273)(125.31738688,36.12642264)(125.53738403,36.21642873)
\curveto(125.62738657,36.25642251)(125.73738646,36.29642247)(125.86738403,36.33642873)
\curveto(125.98738621,36.37642239)(126.05738614,36.44142233)(126.07738403,36.53142873)
\curveto(126.08738611,36.5714222)(126.08738611,36.60142217)(126.07738403,36.62142873)
\lineto(126.01738403,36.68142873)
\curveto(125.96738623,36.73142204)(125.91238628,36.766422)(125.85238403,36.78642873)
\curveto(125.7923864,36.81642195)(125.72738647,36.84642192)(125.65738403,36.87642873)
\lineto(125.02738403,37.11642873)
\curveto(124.80738739,37.19642157)(124.5923876,37.27642149)(124.38238403,37.35642873)
\lineto(124.23238403,37.41642873)
\lineto(124.05238403,37.47642873)
\curveto(123.86238833,37.55642121)(123.67238852,37.62642114)(123.48238403,37.68642873)
\curveto(123.28238891,37.75642101)(123.08238911,37.83142094)(122.88238403,37.91142873)
\curveto(122.30238989,38.15142062)(121.71739048,38.3714204)(121.12738403,38.57142873)
\curveto(120.53739166,38.78141999)(119.95239224,39.00641976)(119.37238403,39.24642873)
\curveto(119.17239302,39.32641944)(118.96739323,39.40141937)(118.75738403,39.47142873)
\curveto(118.54739365,39.55141922)(118.34239385,39.63141914)(118.14238403,39.71142873)
\curveto(118.06239413,39.75141902)(117.96239423,39.78641898)(117.84238403,39.81642873)
\curveto(117.72239447,39.85641891)(117.63739456,39.91141886)(117.58738403,39.98142873)
\curveto(117.54739465,40.04141873)(117.51739468,40.11641865)(117.49738403,40.20642873)
\curveto(117.47739472,40.30641846)(117.46739473,40.41641835)(117.46738403,40.53642873)
\curveto(117.45739474,40.65641811)(117.45739474,40.77641799)(117.46738403,40.89642873)
\curveto(117.46739473,41.01641775)(117.46739473,41.12641764)(117.46738403,41.22642873)
\curveto(117.46739473,41.31641745)(117.46739473,41.40641736)(117.46738403,41.49642873)
\curveto(117.46739473,41.59641717)(117.48739471,41.6714171)(117.52738403,41.72142873)
\curveto(117.57739462,41.81141696)(117.66739453,41.86141691)(117.79738403,41.87142873)
\curveto(117.92739427,41.88141689)(118.06739413,41.88641688)(118.21738403,41.88642873)
\lineto(119.86738403,41.88642873)
\lineto(126.13738403,41.88642873)
\lineto(127.39738403,41.88642873)
\curveto(127.50738469,41.88641688)(127.61738458,41.88641688)(127.72738403,41.88642873)
\curveto(127.83738436,41.89641687)(127.92238427,41.87641689)(127.98238403,41.82642873)
\curveto(128.04238415,41.79641697)(128.08238411,41.75141702)(128.10238403,41.69142873)
\curveto(128.11238408,41.63141714)(128.12738407,41.56141721)(128.14738403,41.48142873)
\lineto(128.14738403,41.24142873)
\lineto(128.14738403,40.88142873)
\curveto(128.13738406,40.771418)(128.0923841,40.69141808)(128.01238403,40.64142873)
\curveto(127.98238421,40.62141815)(127.95238424,40.60641816)(127.92238403,40.59642873)
\curveto(127.88238431,40.59641817)(127.83738436,40.58641818)(127.78738403,40.56642873)
\lineto(127.62238403,40.56642873)
\curveto(127.56238463,40.55641821)(127.4923847,40.55141822)(127.41238403,40.55142873)
\curveto(127.33238486,40.56141821)(127.25738494,40.5664182)(127.18738403,40.56642873)
\lineto(126.34738403,40.56642873)
\lineto(121.92238403,40.56642873)
\curveto(121.67239052,40.5664182)(121.42239077,40.5664182)(121.17238403,40.56642873)
\curveto(120.91239128,40.5664182)(120.66239153,40.56141821)(120.42238403,40.55142873)
\curveto(120.32239187,40.55141822)(120.21239198,40.54641822)(120.09238403,40.53642873)
\curveto(119.97239222,40.52641824)(119.91239228,40.4714183)(119.91238403,40.37142873)
\lineto(119.92738403,40.37142873)
\curveto(119.94739225,40.30141847)(120.01239218,40.24141853)(120.12238403,40.19142873)
\curveto(120.23239196,40.15141862)(120.32739187,40.11641865)(120.40738403,40.08642873)
\curveto(120.57739162,40.01641875)(120.75239144,39.95141882)(120.93238403,39.89142873)
\curveto(121.10239109,39.83141894)(121.27239092,39.76141901)(121.44238403,39.68142873)
\curveto(121.4923907,39.66141911)(121.53739066,39.64641912)(121.57738403,39.63642873)
\curveto(121.61739058,39.62641914)(121.66239053,39.61141916)(121.71238403,39.59142873)
\curveto(121.8923903,39.51141926)(122.07739012,39.44141933)(122.26738403,39.38142873)
\curveto(122.44738975,39.33141944)(122.62738957,39.2664195)(122.80738403,39.18642873)
\curveto(122.95738924,39.11641965)(123.11238908,39.05641971)(123.27238403,39.00642873)
\curveto(123.42238877,38.95641981)(123.57238862,38.90141987)(123.72238403,38.84142873)
\curveto(124.192388,38.64142013)(124.66738753,38.46142031)(125.14738403,38.30142873)
\curveto(125.61738658,38.14142063)(126.08238611,37.9664208)(126.54238403,37.77642873)
\curveto(126.72238547,37.69642107)(126.90238529,37.62642114)(127.08238403,37.56642873)
\curveto(127.26238493,37.50642126)(127.44238475,37.44142133)(127.62238403,37.37142873)
\curveto(127.73238446,37.32142145)(127.83738436,37.2714215)(127.93738403,37.22142873)
\curveto(128.02738417,37.18142159)(128.0923841,37.09642167)(128.13238403,36.96642873)
\curveto(128.14238405,36.94642182)(128.14738405,36.92142185)(128.14738403,36.89142873)
\curveto(128.13738406,36.8714219)(128.13738406,36.84642192)(128.14738403,36.81642873)
\curveto(128.15738404,36.78642198)(128.16238403,36.75142202)(128.16238403,36.71142873)
\curveto(128.15238404,36.6714221)(128.14738405,36.63142214)(128.14738403,36.59142873)
\lineto(128.14738403,36.29142873)
\curveto(128.14738405,36.19142258)(128.12238407,36.11142266)(128.07238403,36.05142873)
\curveto(128.02238417,35.9714228)(127.95238424,35.91142286)(127.86238403,35.87142873)
\curveto(127.76238443,35.84142293)(127.66238453,35.80142297)(127.56238403,35.75142873)
\curveto(127.36238483,35.6714231)(127.15738504,35.59142318)(126.94738403,35.51142873)
\curveto(126.72738547,35.44142333)(126.51738568,35.3664234)(126.31738403,35.28642873)
\curveto(126.13738606,35.20642356)(125.95738624,35.13642363)(125.77738403,35.07642873)
\curveto(125.58738661,35.02642374)(125.40238679,34.96142381)(125.22238403,34.88142873)
\curveto(124.66238753,34.65142412)(124.0973881,34.43642433)(123.52738403,34.23642873)
\curveto(122.95738924,34.03642473)(122.3923898,33.82142495)(121.83238403,33.59142873)
\lineto(121.20238403,33.35142873)
\curveto(120.98239121,33.28142549)(120.77239142,33.20642556)(120.57238403,33.12642873)
\curveto(120.46239173,33.07642569)(120.35739184,33.03142574)(120.25738403,32.99142873)
\curveto(120.14739205,32.96142581)(120.05239214,32.91142586)(119.97238403,32.84142873)
\curveto(119.95239224,32.83142594)(119.94239225,32.82142595)(119.94238403,32.81142873)
\lineto(119.91238403,32.78142873)
\lineto(119.91238403,32.70642873)
\lineto(119.94238403,32.67642873)
\curveto(119.94239225,32.6664261)(119.94739225,32.65642611)(119.95738403,32.64642873)
\curveto(120.00739219,32.62642614)(120.06239213,32.61642615)(120.12238403,32.61642873)
\curveto(120.18239201,32.61642615)(120.24239195,32.60642616)(120.30238403,32.58642873)
\lineto(120.46738403,32.58642873)
\curveto(120.52739167,32.5664262)(120.5923916,32.56142621)(120.66238403,32.57142873)
\curveto(120.73239146,32.58142619)(120.80239139,32.58642618)(120.87238403,32.58642873)
\lineto(121.68238403,32.58642873)
\lineto(126.24238403,32.58642873)
\lineto(127.42738403,32.58642873)
\curveto(127.53738466,32.58642618)(127.64738455,32.58142619)(127.75738403,32.57142873)
\curveto(127.86738433,32.5714262)(127.95238424,32.54642622)(128.01238403,32.49642873)
\curveto(128.0923841,32.44642632)(128.13738406,32.35642641)(128.14738403,32.22642873)
\lineto(128.14738403,31.83642873)
\lineto(128.14738403,31.64142873)
\curveto(128.14738405,31.59142718)(128.13738406,31.54142723)(128.11738403,31.49142873)
\curveto(128.07738412,31.36142741)(127.9923842,31.28642748)(127.86238403,31.26642873)
\curveto(127.73238446,31.25642751)(127.58238461,31.25142752)(127.41238403,31.25142873)
\lineto(125.67238403,31.25142873)
\lineto(119.67238403,31.25142873)
\lineto(118.26238403,31.25142873)
\curveto(118.15239404,31.25142752)(118.03739416,31.24642752)(117.91738403,31.23642873)
\curveto(117.7973944,31.23642753)(117.70239449,31.26142751)(117.63238403,31.31142873)
\curveto(117.57239462,31.35142742)(117.52239467,31.42642734)(117.48238403,31.53642873)
\curveto(117.47239472,31.55642721)(117.47239472,31.57642719)(117.48238403,31.59642873)
\curveto(117.48239471,31.62642714)(117.47739472,31.65142712)(117.46738403,31.67142873)
}
}
{
\newrgbcolor{curcolor}{0 0 0}
\pscustom[linestyle=none,fillstyle=solid,fillcolor=curcolor]
{
\newpath
\moveto(127.59238403,50.87353811)
\curveto(127.75238444,50.90353028)(127.88738431,50.88853029)(127.99738403,50.82853811)
\curveto(128.0973841,50.76853041)(128.17238402,50.68853049)(128.22238403,50.58853811)
\curveto(128.24238395,50.53853064)(128.25238394,50.4835307)(128.25238403,50.42353811)
\curveto(128.25238394,50.37353081)(128.26238393,50.31853086)(128.28238403,50.25853811)
\curveto(128.33238386,50.03853114)(128.31738388,49.81853136)(128.23738403,49.59853811)
\curveto(128.16738403,49.38853179)(128.07738412,49.24353194)(127.96738403,49.16353811)
\curveto(127.8973843,49.11353207)(127.81738438,49.06853211)(127.72738403,49.02853811)
\curveto(127.62738457,48.98853219)(127.54738465,48.93853224)(127.48738403,48.87853811)
\curveto(127.46738473,48.85853232)(127.44738475,48.83353235)(127.42738403,48.80353811)
\curveto(127.40738479,48.7835324)(127.40238479,48.75353243)(127.41238403,48.71353811)
\curveto(127.44238475,48.60353258)(127.4973847,48.49853268)(127.57738403,48.39853811)
\curveto(127.65738454,48.30853287)(127.72738447,48.21853296)(127.78738403,48.12853811)
\curveto(127.86738433,47.99853318)(127.94238425,47.85853332)(128.01238403,47.70853811)
\curveto(128.07238412,47.55853362)(128.12738407,47.39853378)(128.17738403,47.22853811)
\curveto(128.20738399,47.12853405)(128.22738397,47.01853416)(128.23738403,46.89853811)
\curveto(128.24738395,46.78853439)(128.26238393,46.6785345)(128.28238403,46.56853811)
\curveto(128.2923839,46.51853466)(128.2973839,46.47353471)(128.29738403,46.43353811)
\lineto(128.29738403,46.32853811)
\curveto(128.31738388,46.21853496)(128.31738388,46.11353507)(128.29738403,46.01353811)
\lineto(128.29738403,45.87853811)
\curveto(128.28738391,45.82853535)(128.28238391,45.7785354)(128.28238403,45.72853811)
\curveto(128.28238391,45.6785355)(128.27238392,45.63353555)(128.25238403,45.59353811)
\curveto(128.24238395,45.55353563)(128.23738396,45.51853566)(128.23738403,45.48853811)
\curveto(128.24738395,45.46853571)(128.24738395,45.44353574)(128.23738403,45.41353811)
\lineto(128.17738403,45.17353811)
\curveto(128.16738403,45.09353609)(128.14738405,45.01853616)(128.11738403,44.94853811)
\curveto(127.98738421,44.64853653)(127.84238435,44.40353678)(127.68238403,44.21353811)
\curveto(127.51238468,44.03353715)(127.27738492,43.8835373)(126.97738403,43.76353811)
\curveto(126.75738544,43.67353751)(126.4923857,43.62853755)(126.18238403,43.62853811)
\lineto(125.86738403,43.62853811)
\curveto(125.81738638,43.63853754)(125.76738643,43.64353754)(125.71738403,43.64353811)
\lineto(125.53738403,43.67353811)
\lineto(125.20738403,43.79353811)
\curveto(125.0973871,43.83353735)(124.9973872,43.8835373)(124.90738403,43.94353811)
\curveto(124.61738758,44.12353706)(124.40238779,44.36853681)(124.26238403,44.67853811)
\curveto(124.12238807,44.98853619)(123.9973882,45.32853585)(123.88738403,45.69853811)
\curveto(123.84738835,45.83853534)(123.81738838,45.9835352)(123.79738403,46.13353811)
\curveto(123.77738842,46.2835349)(123.75238844,46.43353475)(123.72238403,46.58353811)
\curveto(123.70238849,46.65353453)(123.6923885,46.71853446)(123.69238403,46.77853811)
\curveto(123.6923885,46.84853433)(123.68238851,46.92353426)(123.66238403,47.00353811)
\curveto(123.64238855,47.07353411)(123.63238856,47.14353404)(123.63238403,47.21353811)
\curveto(123.62238857,47.2835339)(123.60738859,47.35853382)(123.58738403,47.43853811)
\curveto(123.52738867,47.68853349)(123.47738872,47.92353326)(123.43738403,48.14353811)
\curveto(123.38738881,48.36353282)(123.27238892,48.53853264)(123.09238403,48.66853811)
\curveto(123.01238918,48.72853245)(122.91238928,48.7785324)(122.79238403,48.81853811)
\curveto(122.66238953,48.85853232)(122.52238967,48.85853232)(122.37238403,48.81853811)
\curveto(122.13239006,48.75853242)(121.94239025,48.66853251)(121.80238403,48.54853811)
\curveto(121.66239053,48.43853274)(121.55239064,48.2785329)(121.47238403,48.06853811)
\curveto(121.42239077,47.94853323)(121.38739081,47.80353338)(121.36738403,47.63353811)
\curveto(121.34739085,47.47353371)(121.33739086,47.30353388)(121.33738403,47.12353811)
\curveto(121.33739086,46.94353424)(121.34739085,46.76853441)(121.36738403,46.59853811)
\curveto(121.38739081,46.42853475)(121.41739078,46.2835349)(121.45738403,46.16353811)
\curveto(121.51739068,45.99353519)(121.60239059,45.82853535)(121.71238403,45.66853811)
\curveto(121.77239042,45.58853559)(121.85239034,45.51353567)(121.95238403,45.44353811)
\curveto(122.04239015,45.3835358)(122.14239005,45.32853585)(122.25238403,45.27853811)
\curveto(122.33238986,45.24853593)(122.41738978,45.21853596)(122.50738403,45.18853811)
\curveto(122.5973896,45.16853601)(122.66738953,45.12353606)(122.71738403,45.05353811)
\curveto(122.74738945,45.01353617)(122.77238942,44.94353624)(122.79238403,44.84353811)
\curveto(122.80238939,44.75353643)(122.80738939,44.65853652)(122.80738403,44.55853811)
\curveto(122.80738939,44.45853672)(122.80238939,44.35853682)(122.79238403,44.25853811)
\curveto(122.77238942,44.16853701)(122.74738945,44.10353708)(122.71738403,44.06353811)
\curveto(122.68738951,44.02353716)(122.63738956,43.99353719)(122.56738403,43.97353811)
\curveto(122.4973897,43.95353723)(122.42238977,43.95353723)(122.34238403,43.97353811)
\curveto(122.21238998,44.00353718)(122.0923901,44.03353715)(121.98238403,44.06353811)
\curveto(121.86239033,44.10353708)(121.74739045,44.14853703)(121.63738403,44.19853811)
\curveto(121.28739091,44.38853679)(121.01739118,44.62853655)(120.82738403,44.91853811)
\curveto(120.62739157,45.20853597)(120.46739173,45.56853561)(120.34738403,45.99853811)
\curveto(120.32739187,46.09853508)(120.31239188,46.19853498)(120.30238403,46.29853811)
\curveto(120.2923919,46.40853477)(120.27739192,46.51853466)(120.25738403,46.62853811)
\curveto(120.24739195,46.66853451)(120.24739195,46.73353445)(120.25738403,46.82353811)
\curveto(120.25739194,46.91353427)(120.24739195,46.96853421)(120.22738403,46.98853811)
\curveto(120.21739198,47.68853349)(120.2973919,48.29853288)(120.46738403,48.81853811)
\curveto(120.63739156,49.33853184)(120.96239123,49.70353148)(121.44238403,49.91353811)
\curveto(121.64239055,50.00353118)(121.87739032,50.05353113)(122.14738403,50.06353811)
\curveto(122.40738979,50.0835311)(122.68238951,50.09353109)(122.97238403,50.09353811)
\lineto(126.28738403,50.09353811)
\curveto(126.42738577,50.09353109)(126.56238563,50.09853108)(126.69238403,50.10853811)
\curveto(126.82238537,50.11853106)(126.92738527,50.14853103)(127.00738403,50.19853811)
\curveto(127.07738512,50.24853093)(127.12738507,50.31353087)(127.15738403,50.39353811)
\curveto(127.197385,50.4835307)(127.22738497,50.56853061)(127.24738403,50.64853811)
\curveto(127.25738494,50.72853045)(127.30238489,50.78853039)(127.38238403,50.82853811)
\curveto(127.41238478,50.84853033)(127.44238475,50.85853032)(127.47238403,50.85853811)
\curveto(127.50238469,50.85853032)(127.54238465,50.86353032)(127.59238403,50.87353811)
\moveto(125.92738403,48.72853811)
\curveto(125.78738641,48.78853239)(125.62738657,48.81853236)(125.44738403,48.81853811)
\curveto(125.25738694,48.82853235)(125.06238713,48.83353235)(124.86238403,48.83353811)
\curveto(124.75238744,48.83353235)(124.65238754,48.82853235)(124.56238403,48.81853811)
\curveto(124.47238772,48.80853237)(124.40238779,48.76853241)(124.35238403,48.69853811)
\curveto(124.33238786,48.66853251)(124.32238787,48.59853258)(124.32238403,48.48853811)
\curveto(124.34238785,48.46853271)(124.35238784,48.43353275)(124.35238403,48.38353811)
\curveto(124.35238784,48.33353285)(124.36238783,48.28853289)(124.38238403,48.24853811)
\curveto(124.40238779,48.16853301)(124.42238777,48.0785331)(124.44238403,47.97853811)
\lineto(124.50238403,47.67853811)
\curveto(124.50238769,47.64853353)(124.50738769,47.61353357)(124.51738403,47.57353811)
\lineto(124.51738403,47.46853811)
\curveto(124.55738764,47.31853386)(124.58238761,47.15353403)(124.59238403,46.97353811)
\curveto(124.5923876,46.80353438)(124.61238758,46.64353454)(124.65238403,46.49353811)
\curveto(124.67238752,46.41353477)(124.6923875,46.33853484)(124.71238403,46.26853811)
\curveto(124.72238747,46.20853497)(124.73738746,46.13853504)(124.75738403,46.05853811)
\curveto(124.80738739,45.89853528)(124.87238732,45.74853543)(124.95238403,45.60853811)
\curveto(125.02238717,45.46853571)(125.11238708,45.34853583)(125.22238403,45.24853811)
\curveto(125.33238686,45.14853603)(125.46738673,45.07353611)(125.62738403,45.02353811)
\curveto(125.77738642,44.97353621)(125.96238623,44.95353623)(126.18238403,44.96353811)
\curveto(126.28238591,44.96353622)(126.37738582,44.9785362)(126.46738403,45.00853811)
\curveto(126.54738565,45.04853613)(126.62238557,45.09353609)(126.69238403,45.14353811)
\curveto(126.80238539,45.22353596)(126.8973853,45.32853585)(126.97738403,45.45853811)
\curveto(127.04738515,45.58853559)(127.10738509,45.72853545)(127.15738403,45.87853811)
\curveto(127.16738503,45.92853525)(127.17238502,45.9785352)(127.17238403,46.02853811)
\curveto(127.17238502,46.0785351)(127.17738502,46.12853505)(127.18738403,46.17853811)
\curveto(127.20738499,46.24853493)(127.22238497,46.33353485)(127.23238403,46.43353811)
\curveto(127.23238496,46.54353464)(127.22238497,46.63353455)(127.20238403,46.70353811)
\curveto(127.18238501,46.76353442)(127.17738502,46.82353436)(127.18738403,46.88353811)
\curveto(127.18738501,46.94353424)(127.17738502,47.00353418)(127.15738403,47.06353811)
\curveto(127.13738506,47.14353404)(127.12238507,47.21853396)(127.11238403,47.28853811)
\curveto(127.10238509,47.36853381)(127.08238511,47.44353374)(127.05238403,47.51353811)
\curveto(126.93238526,47.80353338)(126.78738541,48.04853313)(126.61738403,48.24853811)
\curveto(126.44738575,48.45853272)(126.21738598,48.61853256)(125.92738403,48.72853811)
}
}
{
\newrgbcolor{curcolor}{0 0 0}
\pscustom[linestyle=none,fillstyle=solid,fillcolor=curcolor]
{
\newpath
\moveto(120.24238403,55.69017873)
\curveto(120.24239195,55.92017394)(120.30239189,56.05017381)(120.42238403,56.08017873)
\curveto(120.53239166,56.11017375)(120.6973915,56.12517374)(120.91738403,56.12517873)
\lineto(121.20238403,56.12517873)
\curveto(121.2923909,56.12517374)(121.36739083,56.10017376)(121.42738403,56.05017873)
\curveto(121.50739069,55.99017387)(121.55239064,55.90517396)(121.56238403,55.79517873)
\curveto(121.56239063,55.68517418)(121.57739062,55.57517429)(121.60738403,55.46517873)
\curveto(121.63739056,55.32517454)(121.66739053,55.19017467)(121.69738403,55.06017873)
\curveto(121.72739047,54.94017492)(121.76739043,54.82517504)(121.81738403,54.71517873)
\curveto(121.94739025,54.42517544)(122.12739007,54.19017567)(122.35738403,54.01017873)
\curveto(122.57738962,53.83017603)(122.83238936,53.67517619)(123.12238403,53.54517873)
\curveto(123.23238896,53.50517636)(123.34738885,53.47517639)(123.46738403,53.45517873)
\curveto(123.57738862,53.43517643)(123.6923885,53.41017645)(123.81238403,53.38017873)
\curveto(123.86238833,53.37017649)(123.91238828,53.3651765)(123.96238403,53.36517873)
\curveto(124.01238818,53.37517649)(124.06238813,53.37517649)(124.11238403,53.36517873)
\curveto(124.23238796,53.33517653)(124.37238782,53.32017654)(124.53238403,53.32017873)
\curveto(124.68238751,53.33017653)(124.82738737,53.33517653)(124.96738403,53.33517873)
\lineto(126.81238403,53.33517873)
\lineto(127.15738403,53.33517873)
\curveto(127.27738492,53.33517653)(127.3923848,53.33017653)(127.50238403,53.32017873)
\curveto(127.61238458,53.31017655)(127.70738449,53.30517656)(127.78738403,53.30517873)
\curveto(127.86738433,53.31517655)(127.93738426,53.29517657)(127.99738403,53.24517873)
\curveto(128.06738413,53.19517667)(128.10738409,53.11517675)(128.11738403,53.00517873)
\curveto(128.12738407,52.90517696)(128.13238406,52.79517707)(128.13238403,52.67517873)
\lineto(128.13238403,52.40517873)
\curveto(128.11238408,52.35517751)(128.0973841,52.30517756)(128.08738403,52.25517873)
\curveto(128.06738413,52.21517765)(128.04238415,52.18517768)(128.01238403,52.16517873)
\curveto(127.94238425,52.11517775)(127.85738434,52.08517778)(127.75738403,52.07517873)
\lineto(127.42738403,52.07517873)
\lineto(126.27238403,52.07517873)
\lineto(122.11738403,52.07517873)
\lineto(121.08238403,52.07517873)
\lineto(120.78238403,52.07517873)
\curveto(120.68239151,52.08517778)(120.5973916,52.11517775)(120.52738403,52.16517873)
\curveto(120.48739171,52.19517767)(120.45739174,52.24517762)(120.43738403,52.31517873)
\curveto(120.41739178,52.39517747)(120.40739179,52.48017738)(120.40738403,52.57017873)
\curveto(120.3973918,52.6601772)(120.3973918,52.75017711)(120.40738403,52.84017873)
\curveto(120.41739178,52.93017693)(120.43239176,53.00017686)(120.45238403,53.05017873)
\curveto(120.48239171,53.13017673)(120.54239165,53.18017668)(120.63238403,53.20017873)
\curveto(120.71239148,53.23017663)(120.80239139,53.24517662)(120.90238403,53.24517873)
\lineto(121.20238403,53.24517873)
\curveto(121.30239089,53.24517662)(121.3923908,53.2651766)(121.47238403,53.30517873)
\curveto(121.4923907,53.31517655)(121.50739069,53.32517654)(121.51738403,53.33517873)
\lineto(121.56238403,53.38017873)
\curveto(121.56239063,53.49017637)(121.51739068,53.58017628)(121.42738403,53.65017873)
\curveto(121.32739087,53.72017614)(121.24739095,53.78017608)(121.18738403,53.83017873)
\lineto(121.09738403,53.92017873)
\curveto(120.98739121,54.01017585)(120.87239132,54.13517573)(120.75238403,54.29517873)
\curveto(120.63239156,54.45517541)(120.54239165,54.60517526)(120.48238403,54.74517873)
\curveto(120.43239176,54.83517503)(120.3973918,54.93017493)(120.37738403,55.03017873)
\curveto(120.34739185,55.13017473)(120.31739188,55.23517463)(120.28738403,55.34517873)
\curveto(120.27739192,55.40517446)(120.27239192,55.4651744)(120.27238403,55.52517873)
\curveto(120.26239193,55.58517428)(120.25239194,55.64017422)(120.24238403,55.69017873)
}
}
{
\newrgbcolor{curcolor}{0 0 0}
\pscustom[linestyle=none,fillstyle=solid,fillcolor=curcolor]
{
}
}
{
\newrgbcolor{curcolor}{0 0 0}
\pscustom[linestyle=none,fillstyle=solid,fillcolor=curcolor]
{
\newpath
\moveto(123.06238403,67.99510061)
\lineto(123.31738403,67.99510061)
\curveto(123.3973888,68.0050929)(123.47238872,68.00009291)(123.54238403,67.98010061)
\lineto(123.78238403,67.98010061)
\lineto(123.94738403,67.98010061)
\curveto(124.04738815,67.96009295)(124.15238804,67.95009296)(124.26238403,67.95010061)
\curveto(124.36238783,67.95009296)(124.46238773,67.94009297)(124.56238403,67.92010061)
\lineto(124.71238403,67.92010061)
\curveto(124.85238734,67.89009302)(124.9923872,67.87009304)(125.13238403,67.86010061)
\curveto(125.26238693,67.85009306)(125.3923868,67.82509308)(125.52238403,67.78510061)
\curveto(125.60238659,67.76509314)(125.68738651,67.74509316)(125.77738403,67.72510061)
\lineto(126.01738403,67.66510061)
\lineto(126.31738403,67.54510061)
\curveto(126.40738579,67.51509339)(126.4973857,67.48009343)(126.58738403,67.44010061)
\curveto(126.80738539,67.34009357)(127.02238517,67.2050937)(127.23238403,67.03510061)
\curveto(127.44238475,66.87509403)(127.61238458,66.70009421)(127.74238403,66.51010061)
\curveto(127.78238441,66.46009445)(127.82238437,66.40009451)(127.86238403,66.33010061)
\curveto(127.8923843,66.27009464)(127.92738427,66.2100947)(127.96738403,66.15010061)
\curveto(128.01738418,66.07009484)(128.05738414,65.97509493)(128.08738403,65.86510061)
\curveto(128.11738408,65.75509515)(128.14738405,65.65009526)(128.17738403,65.55010061)
\curveto(128.21738398,65.44009547)(128.24238395,65.33009558)(128.25238403,65.22010061)
\curveto(128.26238393,65.1100958)(128.27738392,64.99509591)(128.29738403,64.87510061)
\curveto(128.30738389,64.83509607)(128.30738389,64.79009612)(128.29738403,64.74010061)
\curveto(128.2973839,64.70009621)(128.30238389,64.66009625)(128.31238403,64.62010061)
\curveto(128.32238387,64.58009633)(128.32738387,64.52509638)(128.32738403,64.45510061)
\curveto(128.32738387,64.38509652)(128.32238387,64.33509657)(128.31238403,64.30510061)
\curveto(128.2923839,64.25509665)(128.28738391,64.2100967)(128.29738403,64.17010061)
\curveto(128.30738389,64.13009678)(128.30738389,64.09509681)(128.29738403,64.06510061)
\lineto(128.29738403,63.97510061)
\curveto(128.27738392,63.91509699)(128.26238393,63.85009706)(128.25238403,63.78010061)
\curveto(128.25238394,63.72009719)(128.24738395,63.65509725)(128.23738403,63.58510061)
\curveto(128.18738401,63.41509749)(128.13738406,63.25509765)(128.08738403,63.10510061)
\curveto(128.03738416,62.95509795)(127.97238422,62.8100981)(127.89238403,62.67010061)
\curveto(127.85238434,62.62009829)(127.82238437,62.56509834)(127.80238403,62.50510061)
\curveto(127.77238442,62.45509845)(127.73738446,62.4050985)(127.69738403,62.35510061)
\curveto(127.51738468,62.11509879)(127.2973849,61.91509899)(127.03738403,61.75510061)
\curveto(126.77738542,61.59509931)(126.4923857,61.45509945)(126.18238403,61.33510061)
\curveto(126.04238615,61.27509963)(125.90238629,61.23009968)(125.76238403,61.20010061)
\curveto(125.61238658,61.17009974)(125.45738674,61.13509977)(125.29738403,61.09510061)
\curveto(125.18738701,61.07509983)(125.07738712,61.06009985)(124.96738403,61.05010061)
\curveto(124.85738734,61.04009987)(124.74738745,61.02509988)(124.63738403,61.00510061)
\curveto(124.5973876,60.99509991)(124.55738764,60.99009992)(124.51738403,60.99010061)
\curveto(124.47738772,61.00009991)(124.43738776,61.00009991)(124.39738403,60.99010061)
\curveto(124.34738785,60.98009993)(124.2973879,60.97509993)(124.24738403,60.97510061)
\lineto(124.08238403,60.97510061)
\curveto(124.03238816,60.95509995)(123.98238821,60.95009996)(123.93238403,60.96010061)
\curveto(123.87238832,60.97009994)(123.81738838,60.97009994)(123.76738403,60.96010061)
\curveto(123.72738847,60.95009996)(123.68238851,60.95009996)(123.63238403,60.96010061)
\curveto(123.58238861,60.97009994)(123.53238866,60.96509994)(123.48238403,60.94510061)
\curveto(123.41238878,60.92509998)(123.33738886,60.92009999)(123.25738403,60.93010061)
\curveto(123.16738903,60.94009997)(123.08238911,60.94509996)(123.00238403,60.94510061)
\curveto(122.91238928,60.94509996)(122.81238938,60.94009997)(122.70238403,60.93010061)
\curveto(122.58238961,60.92009999)(122.48238971,60.92509998)(122.40238403,60.94510061)
\lineto(122.11738403,60.94510061)
\lineto(121.48738403,60.99010061)
\curveto(121.38739081,61.00009991)(121.2923909,61.0100999)(121.20238403,61.02010061)
\lineto(120.90238403,61.05010061)
\curveto(120.85239134,61.07009984)(120.80239139,61.07509983)(120.75238403,61.06510061)
\curveto(120.6923915,61.06509984)(120.63739156,61.07509983)(120.58738403,61.09510061)
\curveto(120.41739178,61.14509976)(120.25239194,61.18509972)(120.09238403,61.21510061)
\curveto(119.92239227,61.24509966)(119.76239243,61.29509961)(119.61238403,61.36510061)
\curveto(119.15239304,61.55509935)(118.77739342,61.77509913)(118.48738403,62.02510061)
\curveto(118.197394,62.28509862)(117.95239424,62.64509826)(117.75238403,63.10510061)
\curveto(117.70239449,63.23509767)(117.66739453,63.36509754)(117.64738403,63.49510061)
\curveto(117.62739457,63.63509727)(117.60239459,63.77509713)(117.57238403,63.91510061)
\curveto(117.56239463,63.98509692)(117.55739464,64.05009686)(117.55738403,64.11010061)
\curveto(117.55739464,64.17009674)(117.55239464,64.23509667)(117.54238403,64.30510061)
\curveto(117.52239467,65.13509577)(117.67239452,65.8050951)(117.99238403,66.31510061)
\curveto(118.30239389,66.82509408)(118.74239345,67.2050937)(119.31238403,67.45510061)
\curveto(119.43239276,67.5050934)(119.55739264,67.55009336)(119.68738403,67.59010061)
\curveto(119.81739238,67.63009328)(119.95239224,67.67509323)(120.09238403,67.72510061)
\curveto(120.17239202,67.74509316)(120.25739194,67.76009315)(120.34738403,67.77010061)
\lineto(120.58738403,67.83010061)
\curveto(120.6973915,67.86009305)(120.80739139,67.87509303)(120.91738403,67.87510061)
\curveto(121.02739117,67.88509302)(121.13739106,67.90009301)(121.24738403,67.92010061)
\curveto(121.2973909,67.94009297)(121.34239085,67.94509296)(121.38238403,67.93510061)
\curveto(121.42239077,67.93509297)(121.46239073,67.94009297)(121.50238403,67.95010061)
\curveto(121.55239064,67.96009295)(121.60739059,67.96009295)(121.66738403,67.95010061)
\curveto(121.71739048,67.95009296)(121.76739043,67.95509295)(121.81738403,67.96510061)
\lineto(121.95238403,67.96510061)
\curveto(122.01239018,67.98509292)(122.08239011,67.98509292)(122.16238403,67.96510061)
\curveto(122.23238996,67.95509295)(122.2973899,67.96009295)(122.35738403,67.98010061)
\curveto(122.38738981,67.99009292)(122.42738977,67.99509291)(122.47738403,67.99510061)
\lineto(122.59738403,67.99510061)
\lineto(123.06238403,67.99510061)
\moveto(125.38738403,66.45010061)
\curveto(125.06738713,66.55009436)(124.70238749,66.6100943)(124.29238403,66.63010061)
\curveto(123.88238831,66.65009426)(123.47238872,66.66009425)(123.06238403,66.66010061)
\curveto(122.63238956,66.66009425)(122.21238998,66.65009426)(121.80238403,66.63010061)
\curveto(121.3923908,66.6100943)(121.00739119,66.56509434)(120.64738403,66.49510061)
\curveto(120.28739191,66.42509448)(119.96739223,66.31509459)(119.68738403,66.16510061)
\curveto(119.3973928,66.02509488)(119.16239303,65.83009508)(118.98238403,65.58010061)
\curveto(118.87239332,65.42009549)(118.7923934,65.24009567)(118.74238403,65.04010061)
\curveto(118.68239351,64.84009607)(118.65239354,64.59509631)(118.65238403,64.30510061)
\curveto(118.67239352,64.28509662)(118.68239351,64.25009666)(118.68238403,64.20010061)
\curveto(118.67239352,64.15009676)(118.67239352,64.1100968)(118.68238403,64.08010061)
\curveto(118.70239349,64.00009691)(118.72239347,63.92509698)(118.74238403,63.85510061)
\curveto(118.75239344,63.79509711)(118.77239342,63.73009718)(118.80238403,63.66010061)
\curveto(118.92239327,63.39009752)(119.0923931,63.17009774)(119.31238403,63.00010061)
\curveto(119.52239267,62.84009807)(119.76739243,62.7050982)(120.04738403,62.59510061)
\curveto(120.15739204,62.54509836)(120.27739192,62.5050984)(120.40738403,62.47510061)
\curveto(120.52739167,62.45509845)(120.65239154,62.43009848)(120.78238403,62.40010061)
\curveto(120.83239136,62.38009853)(120.88739131,62.37009854)(120.94738403,62.37010061)
\curveto(120.9973912,62.37009854)(121.04739115,62.36509854)(121.09738403,62.35510061)
\curveto(121.18739101,62.34509856)(121.28239091,62.33509857)(121.38238403,62.32510061)
\curveto(121.47239072,62.31509859)(121.56739063,62.3050986)(121.66738403,62.29510061)
\curveto(121.74739045,62.29509861)(121.83239036,62.29009862)(121.92238403,62.28010061)
\lineto(122.16238403,62.28010061)
\lineto(122.34238403,62.28010061)
\curveto(122.37238982,62.27009864)(122.40738979,62.26509864)(122.44738403,62.26510061)
\lineto(122.58238403,62.26510061)
\lineto(123.03238403,62.26510061)
\curveto(123.11238908,62.26509864)(123.197389,62.26009865)(123.28738403,62.25010061)
\curveto(123.36738883,62.25009866)(123.44238875,62.26009865)(123.51238403,62.28010061)
\lineto(123.78238403,62.28010061)
\curveto(123.80238839,62.28009863)(123.83238836,62.27509863)(123.87238403,62.26510061)
\curveto(123.90238829,62.26509864)(123.92738827,62.27009864)(123.94738403,62.28010061)
\curveto(124.04738815,62.29009862)(124.14738805,62.29509861)(124.24738403,62.29510061)
\curveto(124.33738786,62.3050986)(124.43738776,62.31509859)(124.54738403,62.32510061)
\curveto(124.66738753,62.35509855)(124.7923874,62.37009854)(124.92238403,62.37010061)
\curveto(125.04238715,62.38009853)(125.15738704,62.4050985)(125.26738403,62.44510061)
\curveto(125.56738663,62.52509838)(125.83238636,62.6100983)(126.06238403,62.70010061)
\curveto(126.2923859,62.80009811)(126.50738569,62.94509796)(126.70738403,63.13510061)
\curveto(126.90738529,63.34509756)(127.05738514,63.6100973)(127.15738403,63.93010061)
\curveto(127.17738502,63.97009694)(127.18738501,64.0050969)(127.18738403,64.03510061)
\curveto(127.17738502,64.07509683)(127.18238501,64.12009679)(127.20238403,64.17010061)
\curveto(127.21238498,64.2100967)(127.22238497,64.28009663)(127.23238403,64.38010061)
\curveto(127.24238495,64.49009642)(127.23738496,64.57509633)(127.21738403,64.63510061)
\curveto(127.197385,64.7050962)(127.18738501,64.77509613)(127.18738403,64.84510061)
\curveto(127.17738502,64.91509599)(127.16238503,64.98009593)(127.14238403,65.04010061)
\curveto(127.08238511,65.24009567)(126.9973852,65.42009549)(126.88738403,65.58010061)
\curveto(126.86738533,65.6100953)(126.84738535,65.63509527)(126.82738403,65.65510061)
\lineto(126.76738403,65.71510061)
\curveto(126.74738545,65.75509515)(126.70738549,65.8050951)(126.64738403,65.86510061)
\curveto(126.50738569,65.96509494)(126.37738582,66.05009486)(126.25738403,66.12010061)
\curveto(126.13738606,66.19009472)(125.9923862,66.26009465)(125.82238403,66.33010061)
\curveto(125.75238644,66.36009455)(125.68238651,66.38009453)(125.61238403,66.39010061)
\curveto(125.54238665,66.4100945)(125.46738673,66.43009448)(125.38738403,66.45010061)
}
}
{
\newrgbcolor{curcolor}{0 0 0}
\pscustom[linestyle=none,fillstyle=solid,fillcolor=curcolor]
{
\newpath
\moveto(117.54238403,72.59470998)
\curveto(117.53239466,73.28470535)(117.65239454,73.88470475)(117.90238403,74.39470998)
\curveto(118.15239404,74.91470372)(118.48739371,75.30970332)(118.90738403,75.57970998)
\curveto(118.98739321,75.629703)(119.07739312,75.67470296)(119.17738403,75.71470998)
\curveto(119.26739293,75.75470288)(119.36239283,75.79970283)(119.46238403,75.84970998)
\curveto(119.56239263,75.88970274)(119.66239253,75.91970271)(119.76238403,75.93970998)
\curveto(119.86239233,75.95970267)(119.96739223,75.97970265)(120.07738403,75.99970998)
\curveto(120.12739207,76.01970261)(120.17239202,76.02470261)(120.21238403,76.01470998)
\curveto(120.25239194,76.00470263)(120.2973919,76.00970262)(120.34738403,76.02970998)
\curveto(120.3973918,76.03970259)(120.48239171,76.04470259)(120.60238403,76.04470998)
\curveto(120.71239148,76.04470259)(120.7973914,76.03970259)(120.85738403,76.02970998)
\curveto(120.91739128,76.00970262)(120.97739122,75.99970263)(121.03738403,75.99970998)
\curveto(121.0973911,76.00970262)(121.15739104,76.00470263)(121.21738403,75.98470998)
\curveto(121.35739084,75.94470269)(121.4923907,75.90970272)(121.62238403,75.87970998)
\curveto(121.75239044,75.84970278)(121.87739032,75.80970282)(121.99738403,75.75970998)
\curveto(122.13739006,75.69970293)(122.26238993,75.629703)(122.37238403,75.54970998)
\curveto(122.48238971,75.47970315)(122.5923896,75.40470323)(122.70238403,75.32470998)
\lineto(122.76238403,75.26470998)
\curveto(122.78238941,75.25470338)(122.80238939,75.23970339)(122.82238403,75.21970998)
\curveto(122.98238921,75.09970353)(123.12738907,74.96470367)(123.25738403,74.81470998)
\curveto(123.38738881,74.66470397)(123.51238868,74.50470413)(123.63238403,74.33470998)
\curveto(123.85238834,74.02470461)(124.05738814,73.7297049)(124.24738403,73.44970998)
\curveto(124.38738781,73.21970541)(124.52238767,72.98970564)(124.65238403,72.75970998)
\curveto(124.78238741,72.53970609)(124.91738728,72.31970631)(125.05738403,72.09970998)
\curveto(125.22738697,71.84970678)(125.40738679,71.60970702)(125.59738403,71.37970998)
\curveto(125.78738641,71.15970747)(126.01238618,70.96970766)(126.27238403,70.80970998)
\curveto(126.33238586,70.76970786)(126.3923858,70.7347079)(126.45238403,70.70470998)
\curveto(126.50238569,70.67470796)(126.56738563,70.64470799)(126.64738403,70.61470998)
\curveto(126.71738548,70.59470804)(126.77738542,70.58970804)(126.82738403,70.59970998)
\curveto(126.8973853,70.61970801)(126.95238524,70.65470798)(126.99238403,70.70470998)
\curveto(127.02238517,70.75470788)(127.04238515,70.81470782)(127.05238403,70.88470998)
\lineto(127.05238403,71.12470998)
\lineto(127.05238403,71.87470998)
\lineto(127.05238403,74.67970998)
\lineto(127.05238403,75.33970998)
\curveto(127.05238514,75.4297032)(127.05738514,75.51470312)(127.06738403,75.59470998)
\curveto(127.06738513,75.67470296)(127.08738511,75.73970289)(127.12738403,75.78970998)
\curveto(127.16738503,75.83970279)(127.24238495,75.87970275)(127.35238403,75.90970998)
\curveto(127.45238474,75.94970268)(127.55238464,75.95970267)(127.65238403,75.93970998)
\lineto(127.78738403,75.93970998)
\curveto(127.85738434,75.91970271)(127.91738428,75.89970273)(127.96738403,75.87970998)
\curveto(128.01738418,75.85970277)(128.05738414,75.82470281)(128.08738403,75.77470998)
\curveto(128.12738407,75.72470291)(128.14738405,75.65470298)(128.14738403,75.56470998)
\lineto(128.14738403,75.29470998)
\lineto(128.14738403,74.39470998)
\lineto(128.14738403,70.88470998)
\lineto(128.14738403,69.81970998)
\curveto(128.14738405,69.73970889)(128.15238404,69.64970898)(128.16238403,69.54970998)
\curveto(128.16238403,69.44970918)(128.15238404,69.36470927)(128.13238403,69.29470998)
\curveto(128.06238413,69.08470955)(127.88238431,69.01970961)(127.59238403,69.09970998)
\curveto(127.55238464,69.10970952)(127.51738468,69.10970952)(127.48738403,69.09970998)
\curveto(127.44738475,69.09970953)(127.40238479,69.10970952)(127.35238403,69.12970998)
\curveto(127.27238492,69.14970948)(127.18738501,69.16970946)(127.09738403,69.18970998)
\curveto(127.00738519,69.20970942)(126.92238527,69.2347094)(126.84238403,69.26470998)
\curveto(126.35238584,69.42470921)(125.93738626,69.62470901)(125.59738403,69.86470998)
\curveto(125.34738685,70.04470859)(125.12238707,70.24970838)(124.92238403,70.47970998)
\curveto(124.71238748,70.70970792)(124.51738768,70.94970768)(124.33738403,71.19970998)
\curveto(124.15738804,71.45970717)(123.98738821,71.72470691)(123.82738403,71.99470998)
\curveto(123.65738854,72.27470636)(123.48238871,72.54470609)(123.30238403,72.80470998)
\curveto(123.22238897,72.91470572)(123.14738905,73.01970561)(123.07738403,73.11970998)
\curveto(123.00738919,73.2297054)(122.93238926,73.33970529)(122.85238403,73.44970998)
\curveto(122.82238937,73.48970514)(122.7923894,73.52470511)(122.76238403,73.55470998)
\curveto(122.72238947,73.59470504)(122.6923895,73.634705)(122.67238403,73.67470998)
\curveto(122.56238963,73.81470482)(122.43738976,73.93970469)(122.29738403,74.04970998)
\curveto(122.26738993,74.06970456)(122.24238995,74.09470454)(122.22238403,74.12470998)
\curveto(122.19239,74.15470448)(122.16239003,74.17970445)(122.13238403,74.19970998)
\curveto(122.03239016,74.27970435)(121.93239026,74.34470429)(121.83238403,74.39470998)
\curveto(121.73239046,74.45470418)(121.62239057,74.50970412)(121.50238403,74.55970998)
\curveto(121.43239076,74.58970404)(121.35739084,74.60970402)(121.27738403,74.61970998)
\lineto(121.03738403,74.67970998)
\lineto(120.94738403,74.67970998)
\curveto(120.91739128,74.68970394)(120.88739131,74.69470394)(120.85738403,74.69470998)
\curveto(120.78739141,74.71470392)(120.6923915,74.71970391)(120.57238403,74.70970998)
\curveto(120.44239175,74.70970392)(120.34239185,74.69970393)(120.27238403,74.67970998)
\curveto(120.192392,74.65970397)(120.11739208,74.63970399)(120.04738403,74.61970998)
\curveto(119.96739223,74.60970402)(119.88739231,74.58970404)(119.80738403,74.55970998)
\curveto(119.56739263,74.44970418)(119.36739283,74.29970433)(119.20738403,74.10970998)
\curveto(119.03739316,73.9297047)(118.8973933,73.70970492)(118.78738403,73.44970998)
\curveto(118.76739343,73.37970525)(118.75239344,73.30970532)(118.74238403,73.23970998)
\curveto(118.72239347,73.16970546)(118.70239349,73.09470554)(118.68238403,73.01470998)
\curveto(118.66239353,72.9347057)(118.65239354,72.82470581)(118.65238403,72.68470998)
\curveto(118.65239354,72.55470608)(118.66239353,72.44970618)(118.68238403,72.36970998)
\curveto(118.6923935,72.30970632)(118.6973935,72.25470638)(118.69738403,72.20470998)
\curveto(118.6973935,72.15470648)(118.70739349,72.10470653)(118.72738403,72.05470998)
\curveto(118.76739343,71.95470668)(118.80739339,71.85970677)(118.84738403,71.76970998)
\curveto(118.88739331,71.68970694)(118.93239326,71.60970702)(118.98238403,71.52970998)
\curveto(119.00239319,71.49970713)(119.02739317,71.46970716)(119.05738403,71.43970998)
\curveto(119.08739311,71.41970721)(119.11239308,71.39470724)(119.13238403,71.36470998)
\lineto(119.20738403,71.28970998)
\curveto(119.22739297,71.25970737)(119.24739295,71.2347074)(119.26738403,71.21470998)
\lineto(119.47738403,71.06470998)
\curveto(119.53739266,71.02470761)(119.60239259,70.97970765)(119.67238403,70.92970998)
\curveto(119.76239243,70.86970776)(119.86739233,70.81970781)(119.98738403,70.77970998)
\curveto(120.0973921,70.74970788)(120.20739199,70.71470792)(120.31738403,70.67470998)
\curveto(120.42739177,70.634708)(120.57239162,70.60970802)(120.75238403,70.59970998)
\curveto(120.92239127,70.58970804)(121.04739115,70.55970807)(121.12738403,70.50970998)
\curveto(121.20739099,70.45970817)(121.25239094,70.38470825)(121.26238403,70.28470998)
\curveto(121.27239092,70.18470845)(121.27739092,70.07470856)(121.27738403,69.95470998)
\curveto(121.27739092,69.91470872)(121.28239091,69.87470876)(121.29238403,69.83470998)
\curveto(121.2923909,69.79470884)(121.28739091,69.75970887)(121.27738403,69.72970998)
\curveto(121.25739094,69.67970895)(121.24739095,69.629709)(121.24738403,69.57970998)
\curveto(121.24739095,69.53970909)(121.23739096,69.49970913)(121.21738403,69.45970998)
\curveto(121.15739104,69.36970926)(121.02239117,69.32470931)(120.81238403,69.32470998)
\lineto(120.69238403,69.32470998)
\curveto(120.63239156,69.3347093)(120.57239162,69.33970929)(120.51238403,69.33970998)
\curveto(120.44239175,69.34970928)(120.37739182,69.35970927)(120.31738403,69.36970998)
\curveto(120.20739199,69.38970924)(120.10739209,69.40970922)(120.01738403,69.42970998)
\curveto(119.91739228,69.44970918)(119.82239237,69.47970915)(119.73238403,69.51970998)
\curveto(119.66239253,69.53970909)(119.60239259,69.55970907)(119.55238403,69.57970998)
\lineto(119.37238403,69.63970998)
\curveto(119.11239308,69.75970887)(118.86739333,69.91470872)(118.63738403,70.10470998)
\curveto(118.40739379,70.30470833)(118.22239397,70.51970811)(118.08238403,70.74970998)
\curveto(118.00239419,70.85970777)(117.93739426,70.97470766)(117.88738403,71.09470998)
\lineto(117.73738403,71.48470998)
\curveto(117.68739451,71.59470704)(117.65739454,71.70970692)(117.64738403,71.82970998)
\curveto(117.62739457,71.94970668)(117.60239459,72.07470656)(117.57238403,72.20470998)
\curveto(117.57239462,72.27470636)(117.57239462,72.33970629)(117.57238403,72.39970998)
\curveto(117.56239463,72.45970617)(117.55239464,72.52470611)(117.54238403,72.59470998)
}
}
{
\newrgbcolor{curcolor}{0 0 0}
\pscustom[linestyle=none,fillstyle=solid,fillcolor=curcolor]
{
\newpath
\moveto(126.51238403,78.63431936)
\lineto(126.51238403,79.26431936)
\lineto(126.51238403,79.45931936)
\curveto(126.51238568,79.52931683)(126.52238567,79.58931677)(126.54238403,79.63931936)
\curveto(126.58238561,79.70931665)(126.62238557,79.7593166)(126.66238403,79.78931936)
\curveto(126.71238548,79.82931653)(126.77738542,79.84931651)(126.85738403,79.84931936)
\curveto(126.93738526,79.8593165)(127.02238517,79.86431649)(127.11238403,79.86431936)
\lineto(127.83238403,79.86431936)
\curveto(128.31238388,79.86431649)(128.72238347,79.80431655)(129.06238403,79.68431936)
\curveto(129.40238279,79.56431679)(129.67738252,79.36931699)(129.88738403,79.09931936)
\curveto(129.93738226,79.02931733)(129.98238221,78.9593174)(130.02238403,78.88931936)
\curveto(130.07238212,78.82931753)(130.11738208,78.7543176)(130.15738403,78.66431936)
\curveto(130.16738203,78.64431771)(130.17738202,78.61431774)(130.18738403,78.57431936)
\curveto(130.20738199,78.53431782)(130.21238198,78.48931787)(130.20238403,78.43931936)
\curveto(130.17238202,78.34931801)(130.0973821,78.29431806)(129.97738403,78.27431936)
\curveto(129.86738233,78.2543181)(129.77238242,78.26931809)(129.69238403,78.31931936)
\curveto(129.62238257,78.34931801)(129.55738264,78.39431796)(129.49738403,78.45431936)
\curveto(129.44738275,78.52431783)(129.3973828,78.58931777)(129.34738403,78.64931936)
\curveto(129.2973829,78.71931764)(129.22238297,78.77931758)(129.12238403,78.82931936)
\curveto(129.03238316,78.88931747)(128.94238325,78.93931742)(128.85238403,78.97931936)
\curveto(128.82238337,78.99931736)(128.76238343,79.02431733)(128.67238403,79.05431936)
\curveto(128.5923836,79.08431727)(128.52238367,79.08931727)(128.46238403,79.06931936)
\curveto(128.32238387,79.03931732)(128.23238396,78.97931738)(128.19238403,78.88931936)
\curveto(128.16238403,78.80931755)(128.14738405,78.71931764)(128.14738403,78.61931936)
\curveto(128.14738405,78.51931784)(128.12238407,78.43431792)(128.07238403,78.36431936)
\curveto(128.00238419,78.27431808)(127.86238433,78.22931813)(127.65238403,78.22931936)
\lineto(127.09738403,78.22931936)
\lineto(126.87238403,78.22931936)
\curveto(126.7923854,78.23931812)(126.72738547,78.2593181)(126.67738403,78.28931936)
\curveto(126.5973856,78.34931801)(126.55238564,78.41931794)(126.54238403,78.49931936)
\curveto(126.53238566,78.51931784)(126.52738567,78.53931782)(126.52738403,78.55931936)
\curveto(126.52738567,78.58931777)(126.52238567,78.61431774)(126.51238403,78.63431936)
}
}
{
\newrgbcolor{curcolor}{0 0 0}
\pscustom[linestyle=none,fillstyle=solid,fillcolor=curcolor]
{
}
}
{
\newrgbcolor{curcolor}{0 0 0}
\pscustom[linestyle=none,fillstyle=solid,fillcolor=curcolor]
{
\newpath
\moveto(117.54238403,89.26463186)
\curveto(117.53239466,89.95462722)(117.65239454,90.55462662)(117.90238403,91.06463186)
\curveto(118.15239404,91.58462559)(118.48739371,91.9796252)(118.90738403,92.24963186)
\curveto(118.98739321,92.29962488)(119.07739312,92.34462483)(119.17738403,92.38463186)
\curveto(119.26739293,92.42462475)(119.36239283,92.46962471)(119.46238403,92.51963186)
\curveto(119.56239263,92.55962462)(119.66239253,92.58962459)(119.76238403,92.60963186)
\curveto(119.86239233,92.62962455)(119.96739223,92.64962453)(120.07738403,92.66963186)
\curveto(120.12739207,92.68962449)(120.17239202,92.69462448)(120.21238403,92.68463186)
\curveto(120.25239194,92.6746245)(120.2973919,92.6796245)(120.34738403,92.69963186)
\curveto(120.3973918,92.70962447)(120.48239171,92.71462446)(120.60238403,92.71463186)
\curveto(120.71239148,92.71462446)(120.7973914,92.70962447)(120.85738403,92.69963186)
\curveto(120.91739128,92.6796245)(120.97739122,92.66962451)(121.03738403,92.66963186)
\curveto(121.0973911,92.6796245)(121.15739104,92.6746245)(121.21738403,92.65463186)
\curveto(121.35739084,92.61462456)(121.4923907,92.5796246)(121.62238403,92.54963186)
\curveto(121.75239044,92.51962466)(121.87739032,92.4796247)(121.99738403,92.42963186)
\curveto(122.13739006,92.36962481)(122.26238993,92.29962488)(122.37238403,92.21963186)
\curveto(122.48238971,92.14962503)(122.5923896,92.0746251)(122.70238403,91.99463186)
\lineto(122.76238403,91.93463186)
\curveto(122.78238941,91.92462525)(122.80238939,91.90962527)(122.82238403,91.88963186)
\curveto(122.98238921,91.76962541)(123.12738907,91.63462554)(123.25738403,91.48463186)
\curveto(123.38738881,91.33462584)(123.51238868,91.174626)(123.63238403,91.00463186)
\curveto(123.85238834,90.69462648)(124.05738814,90.39962678)(124.24738403,90.11963186)
\curveto(124.38738781,89.88962729)(124.52238767,89.65962752)(124.65238403,89.42963186)
\curveto(124.78238741,89.20962797)(124.91738728,88.98962819)(125.05738403,88.76963186)
\curveto(125.22738697,88.51962866)(125.40738679,88.2796289)(125.59738403,88.04963186)
\curveto(125.78738641,87.82962935)(126.01238618,87.63962954)(126.27238403,87.47963186)
\curveto(126.33238586,87.43962974)(126.3923858,87.40462977)(126.45238403,87.37463186)
\curveto(126.50238569,87.34462983)(126.56738563,87.31462986)(126.64738403,87.28463186)
\curveto(126.71738548,87.26462991)(126.77738542,87.25962992)(126.82738403,87.26963186)
\curveto(126.8973853,87.28962989)(126.95238524,87.32462985)(126.99238403,87.37463186)
\curveto(127.02238517,87.42462975)(127.04238515,87.48462969)(127.05238403,87.55463186)
\lineto(127.05238403,87.79463186)
\lineto(127.05238403,88.54463186)
\lineto(127.05238403,91.34963186)
\lineto(127.05238403,92.00963186)
\curveto(127.05238514,92.09962508)(127.05738514,92.18462499)(127.06738403,92.26463186)
\curveto(127.06738513,92.34462483)(127.08738511,92.40962477)(127.12738403,92.45963186)
\curveto(127.16738503,92.50962467)(127.24238495,92.54962463)(127.35238403,92.57963186)
\curveto(127.45238474,92.61962456)(127.55238464,92.62962455)(127.65238403,92.60963186)
\lineto(127.78738403,92.60963186)
\curveto(127.85738434,92.58962459)(127.91738428,92.56962461)(127.96738403,92.54963186)
\curveto(128.01738418,92.52962465)(128.05738414,92.49462468)(128.08738403,92.44463186)
\curveto(128.12738407,92.39462478)(128.14738405,92.32462485)(128.14738403,92.23463186)
\lineto(128.14738403,91.96463186)
\lineto(128.14738403,91.06463186)
\lineto(128.14738403,87.55463186)
\lineto(128.14738403,86.48963186)
\curveto(128.14738405,86.40963077)(128.15238404,86.31963086)(128.16238403,86.21963186)
\curveto(128.16238403,86.11963106)(128.15238404,86.03463114)(128.13238403,85.96463186)
\curveto(128.06238413,85.75463142)(127.88238431,85.68963149)(127.59238403,85.76963186)
\curveto(127.55238464,85.7796314)(127.51738468,85.7796314)(127.48738403,85.76963186)
\curveto(127.44738475,85.76963141)(127.40238479,85.7796314)(127.35238403,85.79963186)
\curveto(127.27238492,85.81963136)(127.18738501,85.83963134)(127.09738403,85.85963186)
\curveto(127.00738519,85.8796313)(126.92238527,85.90463127)(126.84238403,85.93463186)
\curveto(126.35238584,86.09463108)(125.93738626,86.29463088)(125.59738403,86.53463186)
\curveto(125.34738685,86.71463046)(125.12238707,86.91963026)(124.92238403,87.14963186)
\curveto(124.71238748,87.3796298)(124.51738768,87.61962956)(124.33738403,87.86963186)
\curveto(124.15738804,88.12962905)(123.98738821,88.39462878)(123.82738403,88.66463186)
\curveto(123.65738854,88.94462823)(123.48238871,89.21462796)(123.30238403,89.47463186)
\curveto(123.22238897,89.58462759)(123.14738905,89.68962749)(123.07738403,89.78963186)
\curveto(123.00738919,89.89962728)(122.93238926,90.00962717)(122.85238403,90.11963186)
\curveto(122.82238937,90.15962702)(122.7923894,90.19462698)(122.76238403,90.22463186)
\curveto(122.72238947,90.26462691)(122.6923895,90.30462687)(122.67238403,90.34463186)
\curveto(122.56238963,90.48462669)(122.43738976,90.60962657)(122.29738403,90.71963186)
\curveto(122.26738993,90.73962644)(122.24238995,90.76462641)(122.22238403,90.79463186)
\curveto(122.19239,90.82462635)(122.16239003,90.84962633)(122.13238403,90.86963186)
\curveto(122.03239016,90.94962623)(121.93239026,91.01462616)(121.83238403,91.06463186)
\curveto(121.73239046,91.12462605)(121.62239057,91.179626)(121.50238403,91.22963186)
\curveto(121.43239076,91.25962592)(121.35739084,91.2796259)(121.27738403,91.28963186)
\lineto(121.03738403,91.34963186)
\lineto(120.94738403,91.34963186)
\curveto(120.91739128,91.35962582)(120.88739131,91.36462581)(120.85738403,91.36463186)
\curveto(120.78739141,91.38462579)(120.6923915,91.38962579)(120.57238403,91.37963186)
\curveto(120.44239175,91.3796258)(120.34239185,91.36962581)(120.27238403,91.34963186)
\curveto(120.192392,91.32962585)(120.11739208,91.30962587)(120.04738403,91.28963186)
\curveto(119.96739223,91.2796259)(119.88739231,91.25962592)(119.80738403,91.22963186)
\curveto(119.56739263,91.11962606)(119.36739283,90.96962621)(119.20738403,90.77963186)
\curveto(119.03739316,90.59962658)(118.8973933,90.3796268)(118.78738403,90.11963186)
\curveto(118.76739343,90.04962713)(118.75239344,89.9796272)(118.74238403,89.90963186)
\curveto(118.72239347,89.83962734)(118.70239349,89.76462741)(118.68238403,89.68463186)
\curveto(118.66239353,89.60462757)(118.65239354,89.49462768)(118.65238403,89.35463186)
\curveto(118.65239354,89.22462795)(118.66239353,89.11962806)(118.68238403,89.03963186)
\curveto(118.6923935,88.9796282)(118.6973935,88.92462825)(118.69738403,88.87463186)
\curveto(118.6973935,88.82462835)(118.70739349,88.7746284)(118.72738403,88.72463186)
\curveto(118.76739343,88.62462855)(118.80739339,88.52962865)(118.84738403,88.43963186)
\curveto(118.88739331,88.35962882)(118.93239326,88.2796289)(118.98238403,88.19963186)
\curveto(119.00239319,88.16962901)(119.02739317,88.13962904)(119.05738403,88.10963186)
\curveto(119.08739311,88.08962909)(119.11239308,88.06462911)(119.13238403,88.03463186)
\lineto(119.20738403,87.95963186)
\curveto(119.22739297,87.92962925)(119.24739295,87.90462927)(119.26738403,87.88463186)
\lineto(119.47738403,87.73463186)
\curveto(119.53739266,87.69462948)(119.60239259,87.64962953)(119.67238403,87.59963186)
\curveto(119.76239243,87.53962964)(119.86739233,87.48962969)(119.98738403,87.44963186)
\curveto(120.0973921,87.41962976)(120.20739199,87.38462979)(120.31738403,87.34463186)
\curveto(120.42739177,87.30462987)(120.57239162,87.2796299)(120.75238403,87.26963186)
\curveto(120.92239127,87.25962992)(121.04739115,87.22962995)(121.12738403,87.17963186)
\curveto(121.20739099,87.12963005)(121.25239094,87.05463012)(121.26238403,86.95463186)
\curveto(121.27239092,86.85463032)(121.27739092,86.74463043)(121.27738403,86.62463186)
\curveto(121.27739092,86.58463059)(121.28239091,86.54463063)(121.29238403,86.50463186)
\curveto(121.2923909,86.46463071)(121.28739091,86.42963075)(121.27738403,86.39963186)
\curveto(121.25739094,86.34963083)(121.24739095,86.29963088)(121.24738403,86.24963186)
\curveto(121.24739095,86.20963097)(121.23739096,86.16963101)(121.21738403,86.12963186)
\curveto(121.15739104,86.03963114)(121.02239117,85.99463118)(120.81238403,85.99463186)
\lineto(120.69238403,85.99463186)
\curveto(120.63239156,86.00463117)(120.57239162,86.00963117)(120.51238403,86.00963186)
\curveto(120.44239175,86.01963116)(120.37739182,86.02963115)(120.31738403,86.03963186)
\curveto(120.20739199,86.05963112)(120.10739209,86.0796311)(120.01738403,86.09963186)
\curveto(119.91739228,86.11963106)(119.82239237,86.14963103)(119.73238403,86.18963186)
\curveto(119.66239253,86.20963097)(119.60239259,86.22963095)(119.55238403,86.24963186)
\lineto(119.37238403,86.30963186)
\curveto(119.11239308,86.42963075)(118.86739333,86.58463059)(118.63738403,86.77463186)
\curveto(118.40739379,86.9746302)(118.22239397,87.18962999)(118.08238403,87.41963186)
\curveto(118.00239419,87.52962965)(117.93739426,87.64462953)(117.88738403,87.76463186)
\lineto(117.73738403,88.15463186)
\curveto(117.68739451,88.26462891)(117.65739454,88.3796288)(117.64738403,88.49963186)
\curveto(117.62739457,88.61962856)(117.60239459,88.74462843)(117.57238403,88.87463186)
\curveto(117.57239462,88.94462823)(117.57239462,89.00962817)(117.57238403,89.06963186)
\curveto(117.56239463,89.12962805)(117.55239464,89.19462798)(117.54238403,89.26463186)
}
}
{
\newrgbcolor{curcolor}{0 0 0}
\pscustom[linestyle=none,fillstyle=solid,fillcolor=curcolor]
{
\newpath
\moveto(123.06238403,101.36424123)
\lineto(123.31738403,101.36424123)
\curveto(123.3973888,101.37423353)(123.47238872,101.36923353)(123.54238403,101.34924123)
\lineto(123.78238403,101.34924123)
\lineto(123.94738403,101.34924123)
\curveto(124.04738815,101.32923357)(124.15238804,101.31923358)(124.26238403,101.31924123)
\curveto(124.36238783,101.31923358)(124.46238773,101.30923359)(124.56238403,101.28924123)
\lineto(124.71238403,101.28924123)
\curveto(124.85238734,101.25923364)(124.9923872,101.23923366)(125.13238403,101.22924123)
\curveto(125.26238693,101.21923368)(125.3923868,101.19423371)(125.52238403,101.15424123)
\curveto(125.60238659,101.13423377)(125.68738651,101.11423379)(125.77738403,101.09424123)
\lineto(126.01738403,101.03424123)
\lineto(126.31738403,100.91424123)
\curveto(126.40738579,100.88423402)(126.4973857,100.84923405)(126.58738403,100.80924123)
\curveto(126.80738539,100.70923419)(127.02238517,100.57423433)(127.23238403,100.40424123)
\curveto(127.44238475,100.24423466)(127.61238458,100.06923483)(127.74238403,99.87924123)
\curveto(127.78238441,99.82923507)(127.82238437,99.76923513)(127.86238403,99.69924123)
\curveto(127.8923843,99.63923526)(127.92738427,99.57923532)(127.96738403,99.51924123)
\curveto(128.01738418,99.43923546)(128.05738414,99.34423556)(128.08738403,99.23424123)
\curveto(128.11738408,99.12423578)(128.14738405,99.01923588)(128.17738403,98.91924123)
\curveto(128.21738398,98.80923609)(128.24238395,98.6992362)(128.25238403,98.58924123)
\curveto(128.26238393,98.47923642)(128.27738392,98.36423654)(128.29738403,98.24424123)
\curveto(128.30738389,98.2042367)(128.30738389,98.15923674)(128.29738403,98.10924123)
\curveto(128.2973839,98.06923683)(128.30238389,98.02923687)(128.31238403,97.98924123)
\curveto(128.32238387,97.94923695)(128.32738387,97.89423701)(128.32738403,97.82424123)
\curveto(128.32738387,97.75423715)(128.32238387,97.7042372)(128.31238403,97.67424123)
\curveto(128.2923839,97.62423728)(128.28738391,97.57923732)(128.29738403,97.53924123)
\curveto(128.30738389,97.4992374)(128.30738389,97.46423744)(128.29738403,97.43424123)
\lineto(128.29738403,97.34424123)
\curveto(128.27738392,97.28423762)(128.26238393,97.21923768)(128.25238403,97.14924123)
\curveto(128.25238394,97.08923781)(128.24738395,97.02423788)(128.23738403,96.95424123)
\curveto(128.18738401,96.78423812)(128.13738406,96.62423828)(128.08738403,96.47424123)
\curveto(128.03738416,96.32423858)(127.97238422,96.17923872)(127.89238403,96.03924123)
\curveto(127.85238434,95.98923891)(127.82238437,95.93423897)(127.80238403,95.87424123)
\curveto(127.77238442,95.82423908)(127.73738446,95.77423913)(127.69738403,95.72424123)
\curveto(127.51738468,95.48423942)(127.2973849,95.28423962)(127.03738403,95.12424123)
\curveto(126.77738542,94.96423994)(126.4923857,94.82424008)(126.18238403,94.70424123)
\curveto(126.04238615,94.64424026)(125.90238629,94.5992403)(125.76238403,94.56924123)
\curveto(125.61238658,94.53924036)(125.45738674,94.5042404)(125.29738403,94.46424123)
\curveto(125.18738701,94.44424046)(125.07738712,94.42924047)(124.96738403,94.41924123)
\curveto(124.85738734,94.40924049)(124.74738745,94.39424051)(124.63738403,94.37424123)
\curveto(124.5973876,94.36424054)(124.55738764,94.35924054)(124.51738403,94.35924123)
\curveto(124.47738772,94.36924053)(124.43738776,94.36924053)(124.39738403,94.35924123)
\curveto(124.34738785,94.34924055)(124.2973879,94.34424056)(124.24738403,94.34424123)
\lineto(124.08238403,94.34424123)
\curveto(124.03238816,94.32424058)(123.98238821,94.31924058)(123.93238403,94.32924123)
\curveto(123.87238832,94.33924056)(123.81738838,94.33924056)(123.76738403,94.32924123)
\curveto(123.72738847,94.31924058)(123.68238851,94.31924058)(123.63238403,94.32924123)
\curveto(123.58238861,94.33924056)(123.53238866,94.33424057)(123.48238403,94.31424123)
\curveto(123.41238878,94.29424061)(123.33738886,94.28924061)(123.25738403,94.29924123)
\curveto(123.16738903,94.30924059)(123.08238911,94.31424059)(123.00238403,94.31424123)
\curveto(122.91238928,94.31424059)(122.81238938,94.30924059)(122.70238403,94.29924123)
\curveto(122.58238961,94.28924061)(122.48238971,94.29424061)(122.40238403,94.31424123)
\lineto(122.11738403,94.31424123)
\lineto(121.48738403,94.35924123)
\curveto(121.38739081,94.36924053)(121.2923909,94.37924052)(121.20238403,94.38924123)
\lineto(120.90238403,94.41924123)
\curveto(120.85239134,94.43924046)(120.80239139,94.44424046)(120.75238403,94.43424123)
\curveto(120.6923915,94.43424047)(120.63739156,94.44424046)(120.58738403,94.46424123)
\curveto(120.41739178,94.51424039)(120.25239194,94.55424035)(120.09238403,94.58424123)
\curveto(119.92239227,94.61424029)(119.76239243,94.66424024)(119.61238403,94.73424123)
\curveto(119.15239304,94.92423998)(118.77739342,95.14423976)(118.48738403,95.39424123)
\curveto(118.197394,95.65423925)(117.95239424,96.01423889)(117.75238403,96.47424123)
\curveto(117.70239449,96.6042383)(117.66739453,96.73423817)(117.64738403,96.86424123)
\curveto(117.62739457,97.0042379)(117.60239459,97.14423776)(117.57238403,97.28424123)
\curveto(117.56239463,97.35423755)(117.55739464,97.41923748)(117.55738403,97.47924123)
\curveto(117.55739464,97.53923736)(117.55239464,97.6042373)(117.54238403,97.67424123)
\curveto(117.52239467,98.5042364)(117.67239452,99.17423573)(117.99238403,99.68424123)
\curveto(118.30239389,100.19423471)(118.74239345,100.57423433)(119.31238403,100.82424123)
\curveto(119.43239276,100.87423403)(119.55739264,100.91923398)(119.68738403,100.95924123)
\curveto(119.81739238,100.9992339)(119.95239224,101.04423386)(120.09238403,101.09424123)
\curveto(120.17239202,101.11423379)(120.25739194,101.12923377)(120.34738403,101.13924123)
\lineto(120.58738403,101.19924123)
\curveto(120.6973915,101.22923367)(120.80739139,101.24423366)(120.91738403,101.24424123)
\curveto(121.02739117,101.25423365)(121.13739106,101.26923363)(121.24738403,101.28924123)
\curveto(121.2973909,101.30923359)(121.34239085,101.31423359)(121.38238403,101.30424123)
\curveto(121.42239077,101.3042336)(121.46239073,101.30923359)(121.50238403,101.31924123)
\curveto(121.55239064,101.32923357)(121.60739059,101.32923357)(121.66738403,101.31924123)
\curveto(121.71739048,101.31923358)(121.76739043,101.32423358)(121.81738403,101.33424123)
\lineto(121.95238403,101.33424123)
\curveto(122.01239018,101.35423355)(122.08239011,101.35423355)(122.16238403,101.33424123)
\curveto(122.23238996,101.32423358)(122.2973899,101.32923357)(122.35738403,101.34924123)
\curveto(122.38738981,101.35923354)(122.42738977,101.36423354)(122.47738403,101.36424123)
\lineto(122.59738403,101.36424123)
\lineto(123.06238403,101.36424123)
\moveto(125.38738403,99.81924123)
\curveto(125.06738713,99.91923498)(124.70238749,99.97923492)(124.29238403,99.99924123)
\curveto(123.88238831,100.01923488)(123.47238872,100.02923487)(123.06238403,100.02924123)
\curveto(122.63238956,100.02923487)(122.21238998,100.01923488)(121.80238403,99.99924123)
\curveto(121.3923908,99.97923492)(121.00739119,99.93423497)(120.64738403,99.86424123)
\curveto(120.28739191,99.79423511)(119.96739223,99.68423522)(119.68738403,99.53424123)
\curveto(119.3973928,99.39423551)(119.16239303,99.1992357)(118.98238403,98.94924123)
\curveto(118.87239332,98.78923611)(118.7923934,98.60923629)(118.74238403,98.40924123)
\curveto(118.68239351,98.20923669)(118.65239354,97.96423694)(118.65238403,97.67424123)
\curveto(118.67239352,97.65423725)(118.68239351,97.61923728)(118.68238403,97.56924123)
\curveto(118.67239352,97.51923738)(118.67239352,97.47923742)(118.68238403,97.44924123)
\curveto(118.70239349,97.36923753)(118.72239347,97.29423761)(118.74238403,97.22424123)
\curveto(118.75239344,97.16423774)(118.77239342,97.0992378)(118.80238403,97.02924123)
\curveto(118.92239327,96.75923814)(119.0923931,96.53923836)(119.31238403,96.36924123)
\curveto(119.52239267,96.20923869)(119.76739243,96.07423883)(120.04738403,95.96424123)
\curveto(120.15739204,95.91423899)(120.27739192,95.87423903)(120.40738403,95.84424123)
\curveto(120.52739167,95.82423908)(120.65239154,95.7992391)(120.78238403,95.76924123)
\curveto(120.83239136,95.74923915)(120.88739131,95.73923916)(120.94738403,95.73924123)
\curveto(120.9973912,95.73923916)(121.04739115,95.73423917)(121.09738403,95.72424123)
\curveto(121.18739101,95.71423919)(121.28239091,95.7042392)(121.38238403,95.69424123)
\curveto(121.47239072,95.68423922)(121.56739063,95.67423923)(121.66738403,95.66424123)
\curveto(121.74739045,95.66423924)(121.83239036,95.65923924)(121.92238403,95.64924123)
\lineto(122.16238403,95.64924123)
\lineto(122.34238403,95.64924123)
\curveto(122.37238982,95.63923926)(122.40738979,95.63423927)(122.44738403,95.63424123)
\lineto(122.58238403,95.63424123)
\lineto(123.03238403,95.63424123)
\curveto(123.11238908,95.63423927)(123.197389,95.62923927)(123.28738403,95.61924123)
\curveto(123.36738883,95.61923928)(123.44238875,95.62923927)(123.51238403,95.64924123)
\lineto(123.78238403,95.64924123)
\curveto(123.80238839,95.64923925)(123.83238836,95.64423926)(123.87238403,95.63424123)
\curveto(123.90238829,95.63423927)(123.92738827,95.63923926)(123.94738403,95.64924123)
\curveto(124.04738815,95.65923924)(124.14738805,95.66423924)(124.24738403,95.66424123)
\curveto(124.33738786,95.67423923)(124.43738776,95.68423922)(124.54738403,95.69424123)
\curveto(124.66738753,95.72423918)(124.7923874,95.73923916)(124.92238403,95.73924123)
\curveto(125.04238715,95.74923915)(125.15738704,95.77423913)(125.26738403,95.81424123)
\curveto(125.56738663,95.89423901)(125.83238636,95.97923892)(126.06238403,96.06924123)
\curveto(126.2923859,96.16923873)(126.50738569,96.31423859)(126.70738403,96.50424123)
\curveto(126.90738529,96.71423819)(127.05738514,96.97923792)(127.15738403,97.29924123)
\curveto(127.17738502,97.33923756)(127.18738501,97.37423753)(127.18738403,97.40424123)
\curveto(127.17738502,97.44423746)(127.18238501,97.48923741)(127.20238403,97.53924123)
\curveto(127.21238498,97.57923732)(127.22238497,97.64923725)(127.23238403,97.74924123)
\curveto(127.24238495,97.85923704)(127.23738496,97.94423696)(127.21738403,98.00424123)
\curveto(127.197385,98.07423683)(127.18738501,98.14423676)(127.18738403,98.21424123)
\curveto(127.17738502,98.28423662)(127.16238503,98.34923655)(127.14238403,98.40924123)
\curveto(127.08238511,98.60923629)(126.9973852,98.78923611)(126.88738403,98.94924123)
\curveto(126.86738533,98.97923592)(126.84738535,99.0042359)(126.82738403,99.02424123)
\lineto(126.76738403,99.08424123)
\curveto(126.74738545,99.12423578)(126.70738549,99.17423573)(126.64738403,99.23424123)
\curveto(126.50738569,99.33423557)(126.37738582,99.41923548)(126.25738403,99.48924123)
\curveto(126.13738606,99.55923534)(125.9923862,99.62923527)(125.82238403,99.69924123)
\curveto(125.75238644,99.72923517)(125.68238651,99.74923515)(125.61238403,99.75924123)
\curveto(125.54238665,99.77923512)(125.46738673,99.7992351)(125.38738403,99.81924123)
}
}
{
\newrgbcolor{curcolor}{0 0 0}
\pscustom[linestyle=none,fillstyle=solid,fillcolor=curcolor]
{
\newpath
\moveto(117.54238403,106.77385061)
\curveto(117.54239465,106.87384575)(117.55239464,106.96884566)(117.57238403,107.05885061)
\curveto(117.58239461,107.14884548)(117.61239458,107.21384541)(117.66238403,107.25385061)
\curveto(117.74239445,107.31384531)(117.84739435,107.34384528)(117.97738403,107.34385061)
\lineto(118.36738403,107.34385061)
\lineto(119.86738403,107.34385061)
\lineto(126.25738403,107.34385061)
\lineto(127.42738403,107.34385061)
\lineto(127.74238403,107.34385061)
\curveto(127.84238435,107.35384527)(127.92238427,107.33884529)(127.98238403,107.29885061)
\curveto(128.06238413,107.24884538)(128.11238408,107.17384545)(128.13238403,107.07385061)
\curveto(128.14238405,106.98384564)(128.14738405,106.87384575)(128.14738403,106.74385061)
\lineto(128.14738403,106.51885061)
\curveto(128.12738407,106.43884619)(128.11238408,106.36884626)(128.10238403,106.30885061)
\curveto(128.08238411,106.24884638)(128.04238415,106.19884643)(127.98238403,106.15885061)
\curveto(127.92238427,106.11884651)(127.84738435,106.09884653)(127.75738403,106.09885061)
\lineto(127.45738403,106.09885061)
\lineto(126.36238403,106.09885061)
\lineto(121.02238403,106.09885061)
\curveto(120.93239126,106.07884655)(120.85739134,106.06384656)(120.79738403,106.05385061)
\curveto(120.72739147,106.05384657)(120.66739153,106.0238466)(120.61738403,105.96385061)
\curveto(120.56739163,105.89384673)(120.54239165,105.80384682)(120.54238403,105.69385061)
\curveto(120.53239166,105.59384703)(120.52739167,105.48384714)(120.52738403,105.36385061)
\lineto(120.52738403,104.22385061)
\lineto(120.52738403,103.72885061)
\curveto(120.51739168,103.56884906)(120.45739174,103.45884917)(120.34738403,103.39885061)
\curveto(120.31739188,103.37884925)(120.28739191,103.36884926)(120.25738403,103.36885061)
\curveto(120.21739198,103.36884926)(120.17239202,103.36384926)(120.12238403,103.35385061)
\curveto(120.00239219,103.33384929)(119.8923923,103.33884929)(119.79238403,103.36885061)
\curveto(119.6923925,103.40884922)(119.62239257,103.46384916)(119.58238403,103.53385061)
\curveto(119.53239266,103.61384901)(119.50739269,103.73384889)(119.50738403,103.89385061)
\curveto(119.50739269,104.05384857)(119.4923927,104.18884844)(119.46238403,104.29885061)
\curveto(119.45239274,104.34884828)(119.44739275,104.40384822)(119.44738403,104.46385061)
\curveto(119.43739276,104.5238481)(119.42239277,104.58384804)(119.40238403,104.64385061)
\curveto(119.35239284,104.79384783)(119.30239289,104.93884769)(119.25238403,105.07885061)
\curveto(119.192393,105.21884741)(119.12239307,105.35384727)(119.04238403,105.48385061)
\curveto(118.95239324,105.623847)(118.84739335,105.74384688)(118.72738403,105.84385061)
\curveto(118.60739359,105.94384668)(118.47739372,106.03884659)(118.33738403,106.12885061)
\curveto(118.23739396,106.18884644)(118.12739407,106.23384639)(118.00738403,106.26385061)
\curveto(117.88739431,106.30384632)(117.78239441,106.35384627)(117.69238403,106.41385061)
\curveto(117.63239456,106.46384616)(117.5923946,106.53384609)(117.57238403,106.62385061)
\curveto(117.56239463,106.64384598)(117.55739464,106.66884596)(117.55738403,106.69885061)
\curveto(117.55739464,106.7288459)(117.55239464,106.75384587)(117.54238403,106.77385061)
}
}
{
\newrgbcolor{curcolor}{0 0 0}
\pscustom[linestyle=none,fillstyle=solid,fillcolor=curcolor]
{
\newpath
\moveto(117.54238403,115.12345998)
\curveto(117.54239465,115.22345513)(117.55239464,115.31845503)(117.57238403,115.40845998)
\curveto(117.58239461,115.49845485)(117.61239458,115.56345479)(117.66238403,115.60345998)
\curveto(117.74239445,115.66345469)(117.84739435,115.69345466)(117.97738403,115.69345998)
\lineto(118.36738403,115.69345998)
\lineto(119.86738403,115.69345998)
\lineto(126.25738403,115.69345998)
\lineto(127.42738403,115.69345998)
\lineto(127.74238403,115.69345998)
\curveto(127.84238435,115.70345465)(127.92238427,115.68845466)(127.98238403,115.64845998)
\curveto(128.06238413,115.59845475)(128.11238408,115.52345483)(128.13238403,115.42345998)
\curveto(128.14238405,115.33345502)(128.14738405,115.22345513)(128.14738403,115.09345998)
\lineto(128.14738403,114.86845998)
\curveto(128.12738407,114.78845556)(128.11238408,114.71845563)(128.10238403,114.65845998)
\curveto(128.08238411,114.59845575)(128.04238415,114.5484558)(127.98238403,114.50845998)
\curveto(127.92238427,114.46845588)(127.84738435,114.4484559)(127.75738403,114.44845998)
\lineto(127.45738403,114.44845998)
\lineto(126.36238403,114.44845998)
\lineto(121.02238403,114.44845998)
\curveto(120.93239126,114.42845592)(120.85739134,114.41345594)(120.79738403,114.40345998)
\curveto(120.72739147,114.40345595)(120.66739153,114.37345598)(120.61738403,114.31345998)
\curveto(120.56739163,114.24345611)(120.54239165,114.1534562)(120.54238403,114.04345998)
\curveto(120.53239166,113.94345641)(120.52739167,113.83345652)(120.52738403,113.71345998)
\lineto(120.52738403,112.57345998)
\lineto(120.52738403,112.07845998)
\curveto(120.51739168,111.91845843)(120.45739174,111.80845854)(120.34738403,111.74845998)
\curveto(120.31739188,111.72845862)(120.28739191,111.71845863)(120.25738403,111.71845998)
\curveto(120.21739198,111.71845863)(120.17239202,111.71345864)(120.12238403,111.70345998)
\curveto(120.00239219,111.68345867)(119.8923923,111.68845866)(119.79238403,111.71845998)
\curveto(119.6923925,111.75845859)(119.62239257,111.81345854)(119.58238403,111.88345998)
\curveto(119.53239266,111.96345839)(119.50739269,112.08345827)(119.50738403,112.24345998)
\curveto(119.50739269,112.40345795)(119.4923927,112.53845781)(119.46238403,112.64845998)
\curveto(119.45239274,112.69845765)(119.44739275,112.7534576)(119.44738403,112.81345998)
\curveto(119.43739276,112.87345748)(119.42239277,112.93345742)(119.40238403,112.99345998)
\curveto(119.35239284,113.14345721)(119.30239289,113.28845706)(119.25238403,113.42845998)
\curveto(119.192393,113.56845678)(119.12239307,113.70345665)(119.04238403,113.83345998)
\curveto(118.95239324,113.97345638)(118.84739335,114.09345626)(118.72738403,114.19345998)
\curveto(118.60739359,114.29345606)(118.47739372,114.38845596)(118.33738403,114.47845998)
\curveto(118.23739396,114.53845581)(118.12739407,114.58345577)(118.00738403,114.61345998)
\curveto(117.88739431,114.6534557)(117.78239441,114.70345565)(117.69238403,114.76345998)
\curveto(117.63239456,114.81345554)(117.5923946,114.88345547)(117.57238403,114.97345998)
\curveto(117.56239463,114.99345536)(117.55739464,115.01845533)(117.55738403,115.04845998)
\curveto(117.55739464,115.07845527)(117.55239464,115.10345525)(117.54238403,115.12345998)
}
}
{
\newrgbcolor{curcolor}{0.7019608 0.7019608 0.7019608}
\pscustom[linestyle=none,fillstyle=solid,fillcolor=curcolor]
{
\newpath
\moveto(1024.2857,127.5)
\lineto(1024.2857,400.62018)
\lineto(769.8846,400.5097)
\lineto(763.83,395.53785)
\lineto(758.5488,384.99754)
\lineto(741.3573,384.97544)
\lineto(736.2749,379.49537)
\lineto(730.9716,364.38096)
\lineto(725.8009,360.40349)
\lineto(692.2133,360.40349)
\lineto(686.2913,348.55945)
\lineto(625.3917,348.42687)
\lineto(620.1768,344.31681)
\lineto(613.9897,344.27261)
\lineto(609.6144,340.69289)
\lineto(581.1534,340.20675)
\lineto(576.513,336.49444)
\lineto(548.1845,336.49444)
\lineto(542.3067,328.4511)
\lineto(537.0918,328.18593)
\lineto(531.2581,320.14259)
\lineto(514.4202,320.18679)
\lineto(509.6914,316.51867)
\lineto(470.84472,316.51867)
\lineto(465.27625,304.80722)
\lineto(459.48681,300.56458)
\lineto(431.37932,300.56458)
\lineto(426.91571,296.5871)
\lineto(414.76231,296.6313)
\lineto(409.90095,292.56544)
\lineto(370.8333,292.52124)
\lineto(366.01614,288.45537)
\lineto(353.86274,288.63215)
\lineto(338.04122,276.47875)
\lineto(321.38002,276.69972)
\lineto(314.7509,266.35829)
\lineto(309.53598,251.55324)
\lineto(303.56977,242.7144)
\lineto(298.97358,228.66065)
\lineto(265.12084,228.52807)
\lineto(260.03851,224.46221)
\lineto(254.42585,224.24124)
\lineto(249.21094,212.48559)
\lineto(242.93536,212.39719)
\lineto(238.19554,208.5302)
\lineto(231.93749,179.74999)
\lineto(227.24999,172.49999)
\lineto(221.24999,168.31249)
\lineto(199.06249,168.62499)
\lineto(193.74999,156.49999)
\lineto(187.87499,156.31249)
\lineto(182.87499,152.43749)
\lineto(171.18749,152.49999)
\lineto(163.81249,142.49999)
\lineto(155.18749,136.37499)
\lineto(121.49999,136.31249)
\lineto(115.04686,127.49999)
\lineto(1024.2857,127.5)
\closepath
}
}
{
\newrgbcolor{curcolor}{0.60000002 0.60000002 0.60000002}
\pscustom[linestyle=none,fillstyle=solid,fillcolor=curcolor]
{
\newpath
\moveto(1024.2857,127.5)
\lineto(1024.2857,372.68316)
\lineto(886.03005,372.68316)
\lineto(879.7166,364.15999)
\lineto(697.4472,364.41253)
\lineto(692.1439,360.43506)
\lineto(686.58806,328.36271)
\lineto(553.24792,328.42581)
\lineto(542.38878,320.40772)
\lineto(498.00521,320.47082)
\lineto(486.76726,292.12341)
\lineto(459.2406,292.06031)
\lineto(454.46429,288.32584)
\lineto(442.4442,288.39284)
\lineto(439.03755,282.40075)
\lineto(431.99805,276.46611)
\lineto(426.00027,276.46611)
\lineto(420.3813,272.26766)
\lineto(359.17237,272.14139)
\lineto(352.32227,265.41756)
\lineto(348.84987,260.39837)
\lineto(321.26008,260.39837)
\lineto(315.42013,248.43437)
\lineto(310.0537,220.1501)
\lineto(304.18219,219.89756)
\lineto(298.75262,212.51082)
\lineto(265.10191,212.51082)
\lineto(260.55622,208.50178)
\lineto(254.3059,208.50978)
\lineto(249.19201,196.41163)
\lineto(243.28893,196.38003)
\lineto(237.38585,191.23456)
\lineto(232.24038,172.48361)
\lineto(226.09375,172.56261)
\lineto(221.3125,168.46886)
\lineto(199.75,168.37516)
\lineto(193.625,156.56266)
\lineto(188.25,152.62516)
\lineto(166,152.43766)
\lineto(159.9375,144.78141)
\lineto(121.1875,144.78141)
\lineto(117.70312,133.42203)
\lineto(115.90625,130.42203)
\lineto(112.71875,127.50016)
\curveto(141.03945,127.50003)(813.25079,127.5)(1024.2857,127.5)
\closepath
}
}
{
\newrgbcolor{curcolor}{0.50196081 0.50196081 0.50196081}
\pscustom[linestyle=none,fillstyle=solid,fillcolor=curcolor]
{
\newpath
\moveto(67.142854,127.5)
\lineto(1024.2857,127.5)
\lineto(1024.2857,180.625)
\lineto(763.92857,180.625)
\lineto(757.94643,176.16071)
\lineto(586.58296,176.16071)
\lineto(581.34279,172.23101)
\lineto(530.96143,172.23101)
\lineto(525.595,168.1904)
\lineto(237.89092,168.31667)
\lineto(231.76687,156.19484)
\lineto(199.06319,156.00543)
\lineto(193.19167,139.90613)
\lineto(121.53398,139.96923)
\lineto(117.24083,133.52951)
\lineto(110.67484,128.98382)
\lineto(67.142854,128.98382)
\closepath
}
}
\end{pspicture}

\caption{Linea de tiempo de la creación de elementos en el sistema}
\label{tiempos_area_2}
\end{figure}

\section{Conclusiones}
Habiendose presentado los resultados finales del tiempo de evaluación del
proyecto, ahora se pasa a detallar las conclusiones a las que a llegado este
trabajo.

Primeramente, se ha visto que si bien la aplicación esta terminada, para la
mejora real en los metodos de adquisición de conocimiento, son necesarias dos
caracteristicas:

\begin{itemize}
\item Mantener y fidelizar a los estudiantes en el sistema.
\item Siempre fomentar la producción de contenido por parte de todos los
usuarios 
\end{itemize}

%\section{Recomendaciones}

